\documentclass[12pt]{llncs}
%\documentclass{jktr}

\usepackage[pdftex]{hyperref}                   
\usepackage {listings}
\usepackage {mathpartir}
\usepackage{bcprules}
%\usepackage{listings}
                       
\usepackage{graphicx} 
%\usepackage[margins=2.5cm,nohead,nofoot]{geometry}
%\usepackage{geometry}
\usepackage{amsfonts}
\usepackage{amstext}
\usepackage{latexsym}
\usepackage{amssymb}
\usepackage{color}


%\include{myPreamble}
\documentclass[12pt]{llncs}
%\documentclass{jktr}

\usepackage[pdftex]{hyperref}                   
\usepackage {listings}
\usepackage {mathpartir}
\usepackage{bcprules}
%\usepackage{listings}
                       
\usepackage{graphicx} 
%\usepackage[margins=2.5cm,nohead,nofoot]{geometry}
%\usepackage{geometry}
\usepackage{amsfonts}
\usepackage{amstext}
\usepackage{latexsym}
\usepackage{amssymb}
\usepackage{color}


%\include{myPreamble}
\documentclass[12pt]{llncs}
%\documentclass{jktr}

\usepackage[pdftex]{hyperref}                   
\usepackage {listings}
\usepackage {mathpartir}
\usepackage{bcprules}
%\usepackage{listings}
                       
\usepackage{graphicx} 
%\usepackage[margins=2.5cm,nohead,nofoot]{geometry}
%\usepackage{geometry}
\usepackage{amsfonts}
\usepackage{amstext}
\usepackage{latexsym}
\usepackage{amssymb}
\usepackage{color}


%\include{myPreamble}
\documentclass[12pt]{llncs}
%\documentclass{jktr}

\usepackage[pdftex]{hyperref}                   
\usepackage {listings}
\usepackage {mathpartir}
\usepackage{bcprules}
%\usepackage{listings}
                       
\usepackage{graphicx} 
%\usepackage[margins=2.5cm,nohead,nofoot]{geometry}
%\usepackage{geometry}
\usepackage{amsfonts}
\usepackage{amstext}
\usepackage{latexsym}
\usepackage{amssymb}
\usepackage{color}


%\include{myPreamble}
\include{qm2pi.local} 

%\ifpdf
%\usepackage[pdftex]{graphicx}
%\else
%\usepackage{graphicx}
%\fi

 % \ifpdf
%  \usepackage{pdfsync}
%  \if


%\title{Brief Article}
%\author{David F. Snyder}
%\author{L.G. Meredith}

%\address{Dept. of Math., Texas State University--San Marcos, San Marcos, TX 78666}
       
\pagestyle{empty}


\begin{document}

\lstset{language=[Objective]Caml,frame=shadowbox}

\input{qm2pi.front}

% section front matter (end)

\input{qm2pi.intro} 
 
% section introduction (end)

% \input{qm2pi.knotations} 

% section notation (end)

\input{qm2pi.process.calculi} 

% section concurrent_process_calculi_and_spatial_logics_ (end)
    
%\input{qm2pi.knots2pi} 

%\input{qm2pi.trefoil} 

%\input{qm2pi.mainthm} 

% subsection basic_interpretation (end)

%\input{qm2pi.rho.presentation} 
\subsection{The syntax and semantics of the notation system}\label{sub:the_syntax_and_semantics_of_the_notation_system} % (fold)

We now summarize a technical presentation of the calculus that
embodies our theory of dynamics. The typical presentation of such a
calculus follows the style of giving generators and relations on
them. The grammar, below, describing term constructors, freely
generates the set of processes, $\Proc$. This set is then quotiented
by a relation known as structural congruence and it is over this set
that the notion of dynamics is expressed. This presentation is
essentially that of \cite{MeredithR05} with the addition of
polyadicity and summation. For readability we have relegated some of
the technical subtleties to an appendix.

\subsubsection{Process grammar}\label{subsub:process_grammar}

\begin{mathpar}
  \inferrule* [lab=synchronization] {} {{M} \bc \pzero \;|\; x?F \;|\; x!C }
  \and
  \inferrule* [lab=abstraction] {} {{F} \bc (x)P}
  \and
  \inferrule* [lab=concretion] {} {{C} \bc \langle Q \rangle}
  \and
  \inferrule* [lab=process] {} {{P,Q} \bc M \;| \;P|Q \;|\; @{x}}
  \and
  \inferrule* [lab=name] {} {{x} \bc \quotep{P}}
\end{mathpar} 

Note that $\vec{x}$ (resp. $\vec{P}$) denotes a vector of names
(resp. processes) of length $|\vec{x}|$ (resp. $|\vec{P}|$). We adopt
the following useful abbreviations.

\begin{mathpar}
   x?(\vec{y}).P := x.(\vec{y})P \and  x\clift{\vec{P}} := x.\clift{\vec{P}}
   \and x!(y) := \lift{x}{\dropn{y}}
   \and \Pi_{i=0}^{n-1}P_i := P_0 | \ldots | P_{n-1}
\end{mathpar}

\subsubsection{Structural congruence}

\paragraph{Free and bound names and alpha-equivalence.} At the
core of structural equivalence is alpha-equivalence which identifies
process that are the same up to a change of variable. Formally, we
recognize the distinction between free and bound names. The free names
of a process, $\freenames{P}$, may be calculated recursively as
follows:

\begin{mathpar}
\freenames{\pzero} := \emptyset
  \and \\
  \freenames{x?(y).P} := \{ x \} \cup (\freenames{P} \setminus \{ y \})
  \and 
  \freenames{x!\langle P \rangle} := \{ x \} \cup \{ P \} 
  \and \\
  \freenames{P|Q} := \freenames{P} \cup \freenames{Q}
  \and \\
  \freenames{@{x}} := \{ x \}
\end{mathpar}

$\pi$
$\quotep{\pi}$

$\freenames{-} : \pi \to \mathcal{P}(\quotep{\pi})$

\begin{eqnarray*}
  \freenames{\pzero} & := & \emptyset \\
  \freenames{x?(y).P} & := & \{ x \} \cup (\freenames{P} \setminus \{ y \}) \\
  \freenames{x!\langle P \rangle} & := & \{ x \} \cup \{ P \} \\
  \freenames{P|Q} & := & \freenames{P} \cup \freenames{Q} \\
  \freenames{\dropn{x}} & := & \{ x \}
\end{eqnarray*}

The bound names of a process, $\boundnames{P}$, are those names occurring in $P$
that are not free. For example, in $x?(y).0$, the name $x$ is free, while $y$ is bound.

\begin{mathpar}
  \inferrule* [lab=monoidal-laws] {} { P|Q \equiv Q|P \and P|0 \equiv P \and P|(Q|R) \equiv (P|Q)|R }
\end{mathpar}

\begin{mathpar}
  \inferrule* [lab=alpha-equivalence] {} { (x)P \equiv (y)P\{y/x\} \and y \not\in \freenames{P} }
\end{mathpar}

\begin{definition}
Then two processes, $P,Q$, are alpha-equivalent if $P = Q\{\vec{y}/\vec{x}\}$ for
some $\vec{x} \in \boundnames{Q},\vec{y} \in \boundnames{P}$, where $Q\{\vec{y}/\vec{x}\}$
denotes the capture-avoiding substitution of $\vec{y}$ for $\vec{x}$ in $Q$.
\end{definition}

\begin{definition}
  The {\em structural congruence} \cite{SangiorgiWalker} , $\equiv$,
  between processes is the least congruence containing
  alpha-equivalence, satisfying the abelian monoid laws
  (associativity, commutativity and $\pzero$ as identity) for parallel
  composition $|$ and for summation $+$.
\end{definition}

\subsection{Name equivalence}

We take name equivalence, written $\nameeq$, to be the smallest
equivalence relation generated by the following rules.

\begin{mathpar}
\inferrule*[lab=Quote-drop]
{ }
{ \quotep{@{x}} \nameeq x }

\inferrule*[lab=Struct-equiv]
{ P \scong Q }
{ \quotep{P} \nameeq \quotep{Q} }
\end{mathpar}

The astute reader will have noticed that the mutual recursion of names
and processes imposes a mutual recursion on alpha-equivalence and
structural equivalence via name-equivalence. Fortunately, all of this
works out pleasantly and we may calculate in the natural way, free of
concern. The reader interested in the details is referred to the
appendix \ref{appendix:rho_details}.

\subsection{Substitution}

We use $\Proc$ for the set of processes, $\QProc$ for the set of
names, and $\id{\{}\vec{y} / \vec{x} \id{\}}$ to denote partial maps,
$s : \QProc \rightarrow \QProc$. A map, $s$ lifts, uniquely, to a map
on process terms, $\widehat{s} : \Proc \rightarrow \Proc$ by the
following equations.

\begin{mathpar}
  (0) \psubstp{Q}{P} := 0 \\
  (R \juxtap S) \psubstp{Q}{P}
  :=    
  (R)\psubstp{Q}{P} \juxtap (S) \psubstp{Q}{P} \\
  (x?(y).R) \psubstp{Q}{P}    
  :=    
  (x)\substp{Q}{P} (z)\concat( (R \psubstn{z}{y}) \psubstp{Q}{P} ) \\
  (\lift{x}{R}) \psubstp{Q}{P}  
  :=
  \lift{(x)\substp{Q}{P}}{ R \psubstp{Q}{P} } \\
%   (\dropn{x})  \psubstp{Q}{P}       
%   := 
%   \left\{ 
%     \begin{array}{ccc} 
%       \dropn{\quotep{Q}} & & x \nameeq \quotep{P} \\
%       \dropn{x} & & otherwise \\
%     \end{array}
%   \right. 
  (\dropn{x})  \psubstp{Q}{P}       
  := 
  \left\{ 
    \begin{array}{ccc} 
      Q & & x \nameeq \quotep{P} \\
      \dropn{x} & & otherwise \\
    \end{array}
  \right.
\end{mathpar}
 

where

\begin{eqnarray}
  (x)\id{\{} \lpquote Q \rpquote / \lpquote P \rpquote \id{\}}            = 
  \left\{ 
    \begin{array}{ccc}
      \lpquote Q \rpquote & & x \nameeq \lpquote P \rpquote \\
      x & & otherwise \\
    \end{array}
  \right. \nonumber
\end{eqnarray}

and $z$ is chosen distinct from $\quotep{P}$, $\quotep{Q}$, the free
names in $Q$, and all the names in $R$. Our $\alpha$-equivalence will
be built in the standard way from this substitution.

\begin{remark}\label{rem:no_self_referential_names}
  One consequence of these definitions is that $\forall P. \quotep{P}
  \not\in \freenames{P}$.
\end{remark}

\subsection{ Dynamic quote: an example }

Anticipating something of what's to come, consider applying the
substitution, $\widehat{\id{\{}u / z \id{\}}}$, to the following pair
of processes, $\lift{w}{y!(z)}$ and $w[ \lpquote y!(z) \rpquote ]$.

\begin{eqnarray}
	\lift{w}{y!(z)}\widehat{\id{\{}u / z \id{\}}}
		& = &
		\lift{w}{y!(u)} \nonumber\\
	w[ \lpquote y!(z) \rpquote ] \widehat{ \id{\{}u / z \id{\}} }
		& = &
		w[ \lpquote y!(z) \rpquote ] \nonumber
\end{eqnarray}

Because the body of the process between quotes is impervious to
substitution, we get radically different answers. In fact, by
examining the first process in an input context,
e.g. $x?(z).\lift{w}{y!(z)}$, we see that the process under the lift
operator may be shaped by prefixed inputs binding a name inside it. In
this sense, the lift operator will be seen as a way to dynamically
construct processes before reifying them as names.

Finally equipped with these standard features we can present the
dynamics of the calculus.

\subsubsection{Operational semantics} 

Finally, we introduce the computational dynamics. What marks these
algebras as distinct from other more traditionally studied algebraic
structures, e.g. vector spaces or polynomial rings, is the manner in
which dynamics is captured. In traditional structures, dynamics is typically
expressed through morphisms between such structures, as in linear maps
between vector spaces or morphisms between rings. In algebras
associated with the semantics of computation, the dynamics is
expressed as part of the algebraic structure itself, through a
reduction reduction relation typically denoted by $\red$. Below, we
give a recursive presentation of this relation for the calculus used
in the encoding.

$\red \subseteq \pi \times \pi$
$\red : \pi \to \mathcal{P}(\pi)$

\begin{mathpar}
  \inferrule* [lab=Comm] { \textsf{match}( x_{src}, x_{trgt} ) } { x_{trgt}?(y)P \; | \; x_{src}!\langle {Q} \rangle \red P\{\quotep{Q}/y}\} }
  \and \\
  \inferrule* [lab=Par] {{P} \red {P}'} {{{P} | {Q}} \red {{P}' | {Q}}}
  \and
  \inferrule* [lab=Equiv]{{{P} \scong {P}'} \andalso {{P}' \red {Q}'} \andalso {{Q}' \scong {Q}}}{{P} \red {Q}}
\end{mathpar}

\begin{eqnarray*}
  match_{\equiv} (\quotep{P},\quotep{Q}) & := & P \equiv Q \\
  match_{\dagger}(\quotep{P},\quotep{Q}) & := & \forall R. P|Q \red^{*} R => R \red^{*} 0 \\
  match_{K}(\quotep{P},\quotep{Q}) & := & K \mbox{ for some context } K
\end{eqnarray*}

$u?(x)P | u!\langle Q \rangle \red P\{\quotep{Q}/x\}$

%We write $\wred$ for $\red^*$, and $P\red$ if $\exists Q $ such that $ P \red Q$.
We write $P\red$ if $\exists Q $ such that $ P \red Q$ and $P\not\red$, otherwise.

\section{Replication}

As mentioned before, it is known that replication (and hence
recursion) can be implemented in a higher-order process algebra
\cite{SangiorgiWalker}. As our first example of calculation with the
machinery thus far presented we give the construction explicitly in
the {\rhoc}.

\begin{eqnarray}
	D_{x} & := & \prefix{x}{y}{(\binpar{\outputp{x}{y}}{@{y}})} \nonumber\\
	\bangp_{x}{P} & := & \binpar{{x}!\langle{\binpar{D_{x}}{P}}\rangle}{D_{x}} \nonumber
\end{eqnarray}

\begin{eqnarray}
	\bangp_{x}{P} & & \nonumber\\
	=
	& {x}!\langle{(\prefix{x}{y}{(\outputp{x}{y} | @{y})) | P}}\rangle 
	      | \prefix{x}{y}{(\outputp{x}{y} | @{y})} & \nonumber\\
	\red
	& (\outputp{x}{y} | @{y})\substn{\quotep{(\prefix{x}{y}{(@{y} | \outputp{x}{y})) | P}}}{y} & \nonumber\\
	=
	& \outputp{x}{\quotep{(\prefix{x}{y}{(\outputp{x}{y} | @{y})) | P}}}
	  | {(\prefix{x}{y}{(\outputp{x}{y} | @{y})) | P}} & \nonumber\\
	\red
	& \ldots & \nonumber\\
	\red^*
	& P | P | \ldots & \nonumber
\end{eqnarray}

Of course, this encoding, as an implementation, runs away, unfolding
$\bangp{P}$ eagerly. A lazier and more implementable replication
operator, restricted to input-guarded processes, may be obtained as follows.

\begin{eqnarray}
\bangp{\prefix{u}{v}{P}} 
	:= 
	\binpar{\lift{x}{\prefix{u}{v}{(\binpar{D(x)}{P})}}}{D(x)} \nonumber
\end{eqnarray}

\begin{remark}
  Note that the lazier definition still does not deal with summation
  or mixed summation (i.e. sums over input and output). The reader is
  invited to construct definitions of replication that deal with these
  features. 

  Further, the definitions are parameterized in a name, $x$. Can you,
  gentle reader, make a definition that eliminates this parameter and
  guarantees no accidental interaction between the replication
  machinery and the process being replicated -- i.e. no accidental
  sharing of names used by the process to get its work done and the
  name(s) used by the replication to effect copying. This latter
  revision of the definition of replication is crucial to obtaining
  the expected identity $!!P \sim !P$.
\end{remark}

\begin{remark}\label{rem:paradoxical_combinator}
  The reader familiar with the lambda calculus will have noticed the
  similarity between $D$ and the paradoxical combinator.

  [Ed. note: the existence of this seems to suggest we have to be more
  restrictive on the set of processes and names we admit if we are to
  support no-cloning.]
\end{remark}

\subsubsection{Bisimulation}

The computational dynamics gives rise to another kind of equivalence,
the equivalence of computational behavior. As previously mentioned
this is typically captured \emph{via} some form of bisimulation.

% The notion we use in this paper is weak barbed bisimulation
% \cite{milner91polyadicpi}.

The notion we use in this paper is derived from weak barbed
bisimulation \cite{milner91polyadicpi}. 

\begin{definition}
An \emph{observation relation}, $\downarrow_{\mathcal N}$, over a set
of names, $\mathcal N$, is the smallest relation satisfying the rules
below.

\infrule[Out-barb]{y \in {\mathcal N}, \; x \nameeq y}
		  {\outputp{x}{v} \downarrow_{\mathcal N} x}
\infrule[Par-barb]{\mbox{$P\downarrow_{\mathcal N} x$ or $Q\downarrow_{\mathcal N} x$}}
		  {\binpar{P}{Q} \downarrow_{\mathcal N} x}

We write $P \Downarrow_{\mathcal N} x$ if there is $Q$ such that 
$P \wred Q$ and $Q \downarrow_{\mathcal N} x$.
\end{definition}

\begin{definition}
%\label{def.bbisim}
An  ${\mathcal N}$-\emph{barbed bisimulation} over a set of names, ${\mathcal N}$, is a symmetric binary relation 
${\mathcal S}_{\mathcal N}$ between agents such that $P\rel{S}_{\mathcal N}Q$ implies:
\begin{enumerate}
\item If $P \red P'$ then $Q \wred Q'$ and $P'\rel{S}_{\mathcal N} Q'$.
\item If $P\downarrow_{\mathcal N} x$, then $Q\Downarrow_{\mathcal N} x$.
\end{enumerate}
$P$ is ${\mathcal N}$-barbed bisimilar to $Q$, written
$P \wbbisim_{\mathcal N} Q$, if $P \rel{S}_{\mathcal N} Q$ for some ${\mathcal N}$-barbed bisimulation ${\mathcal S}_{\mathcal N}$.
\end{definition}

$\mathcal{R} \subseteq \pi \times \pi$

$P \mathcal{R} Q => \forall P'. P \red P' \Rightarrow \exists Q'. Q \red Q', P' \mathcal{R} Q'$

$P \vdash x \Rightarrow Q \vdash x$

\begin{mathpar}
  \inferrule*[lab=Out-barb]{x \nameeq y}{{y}!\langle{Q}\rangle \vdash x}
  \and
  \inferrule*[lab=Par-barb]{\mbox{$P\vdash x$ or $Q\vdash x$}}{\binpar{P}{Q} \vdash x}
\end{mathpar}

\subsubsection{Contexts}

One of the principle advantages of computational calculi like the
$\pi$-calculus is a well-defined notion of context,
contextual-equivalence and a correlation between
contextual-equivalence and notions of bisimulation. The notion of
context allows the decomposition of a process into (sub-)process and
its syntactic environment, its context. Thus, a context may be
thought of as a process with a ``hole'' (written $\Box$) in it. The
application of a context $M$ to a process $P$, written $M[P]$, is
tantamount to filling the hole in $M$ with $P$. In this paper we do
not need the full weight of this theory, but do make use of the notion
of context in the proof the main theorem. 

\begin{mathpar}
  \inferrule* [lab=summation] {} {{M_{M},M_{N}} \bc \Box \;|\; x.M_{A} \;|\; M_{M}+M_{N}}
  \and
  \inferrule* [lab=agent] {} {{M_{A}} \bc (\vec{x})M_{P} \;| \; \clift{P_0,\ldots,M_{P},\ldots,P_N}}
  \and \\
  \inferrule* [lab=process] {} {{M_{P}} \bc M_{N} \;| \;P|M_{P} }
\end{mathpar} 

\begin{mathpar}
  \inferrule* [lab=sychronization] {} {M_{N} \bc \Box \;|\; x?M_{F} \;|\; x!M_{C}}
  \and
  \inferrule* [lab=abstraction] {} {{M_{F}} \bc (x)M_{P} }
  \and
  \inferrule* [lab=concretion] {} {{M_{C}} \bc \langle M_{P} \rangle }
  \and \\
  \inferrule* [lab=process] {} {{M_{P}} \bc M_{N} \;| \;P|M_{P} }
\end{mathpar}

\begin{definition}[contextual application] Given a context $M$, and
  process $P$, we define the \emph{contextual application}, $M[P] :=
  M\{P/\Box\}$. That is, the contextual application of M to P is the
  substitution of $P$ for $\Box$ in $M$.
\end{definition}

$\meaningof{-} : L \to \mathcal{P}(\pi)$

\begin{mathpar}
  \inferrule* [lab=collection] {} {\meaningof{true} = \pi, \and \meaningof{~E} = \pi \setminus \meaningof{E}, \and \meaningof{E_{1} \& E_{2}} = \meaningof{E_{1}} \cap \meaningof{E_{2}}}
\end{mathpar}

\begin{mathpar}
  \inferrule* [lab=structure] {} {\meaningof{0} = \{ P \in \pi | P \equiv 0 \}, \and \\ \meaningof{E_1 | E_2} = \{ P \in \pi | P \equiv P_{1} | P_{2}, P_{1} \in \meaningof{E_{1}}, P_{2} \in \meaningof{E_2}\} }
\end{mathpar}

\begin{mathpar}
 \inferrule* [lab=behavior] {} {\meaningof{\langle a?b \rangle E} = \{ P \in \pi | P \equiv Q | u?(y)P', \\ \and \\\\ \and \\ \;\;\; u \in \meaningof{a}, \forall z.P'\{z/y\} \in \meaningof{E\{z/b\}}\}, \and \\ \meaningof{a!E} = \{ P \in \pi | P \equiv Q | x!\langle P' \rangle, x \in \meaningof{a} P' \in \meaningof{E}\} }
\end{mathpar}

\begin{mathpar}
 \inferrule* [lab=nominal] {} {\meaningof{\quotep{E}} = \{ \quotep{P} \in \quotep{\pi} | P \in \meaningof{E} \}, \and \meaningof{\quotep{P}} = \{ \quotep{Q} \in \quotep{\pi} | P \equiv Q \} \and \\ \meaningof{@\quotep{E}} = \{ P \in \pi | P \equiv @x, x \in \meaningof{E} \}}
\end{mathpar}

\begin{eqnarray*}
  \\
  \meaningof{-} : TS \to ST
\end{eqnarray*}

\begin{eqnarray*}
  \\
  L : TS \to ST
\end{eqnarray*}

\begin{eqnarray*}
  \\
  P \models E \iff P \in \meaningof{E}
\end{eqnarray*}

\begin{eqnarray*}
  P \approx_{L} Q \iff \forall E \in L. P \models E \iff Q \models E
\end{eqnarray*}

\begin{eqnarray*}
  P \approx_{K} Q
\end{eqnarray*}

\begin{eqnarray*}
  P \approx Q
\end{eqnarray*}

$\approx_{K} = \approx = \approx_{L}$

\subsubsection{Contextual duality}

Note that contexts extend the quotation operation to a family of
operations from processes to names. Given a context, $M$, we can
define a \emph{nominal context}, $\quotep{M}$ by $\quotep{M}[P] :=
\quotep{M[P]}$. To foreshadow what is to come we observe that these
operations enjoy a duality with processes very much like the duality
between vectors and maps from vectors to scalars.

Further, because the calculus is essentially higher-order, we have a
correspondence between contexts and processes. More specifically,
given a name $x$ and a context $M$ we can construct $M^{*}_{x}$ such
that 

\begin{mathpar}
  M^{*}_{x} | \lift{x}{P} \red M[P]
\end{mathpar}

namely,

\begin{mathpar}
  M^{*}_{x} := x?(u).M[\dropn{u}]
\end{mathpar}

The dependence of $M^{*}_{x}$ on a name makes it an abstraction, 

\begin{mathpar}
  M^{*} := (x)x?(u).M[\dropn{u}]
\end{mathpar}

\subsection{Additional notation}

It will sometimes be convenient to denote the process a name
quotes. We already have the notation $x = \quotep{P}$, but it will be
convenient to introduce an alternate notation, $\procn{x}$, when we
want to emphasize the connection to the use of the name. Note that, by
virtue of name equivalence, $\quotep{\procn{x}} \nameeq x$; so, the
notation is consistent with previous definitions.

Further, because names have structure it is possible to effect
substitutions on the basis of that structure. This means we need to
upgrade our notation for substitutions, which we accomplish by
adapting comprehension notation. Thus,

\begin{mathpar}
  P\{ y / x : x \in S \}
\end{mathpar}

is interpreted to mean the process derived from P by replacing (in a
capture-avoiding manner) each occurrence of $x$ in $S$ by $y$. For example,

\begin{mathpar}
  P\{ \quotep{\procn{x}|\procn{x}} / x : x \in \freenames{P} \}
\end{mathpar}

will replace each (occurrence) of a free name $x$ in $P$ by
$\quotep{\procn{x}|\procn{x}}$.

Also, we will avail ourselves of the notation $x^{L}$ and $x^{R}$ to
denote injections of a name into disjoint copies of the name
space. There are numerous ways to accomplish this. One example can be
found in \cite{MeredithR05}. This notation overloads to vectors of
names: $\vec{x}^{\pi} := (x_{i}^{\pi} \; : \; 0 \leq i < |\vec{x}| )$ where $\pi \in \{L,R\}$.

We also use $P^{\Box} := P|\Box$.

In \cite{MeredithR05} an interpretation of the new operator is
given. It turns out that there are several possible interpretations
all enjoying the requisite algebraic properties of the operator (see
\cite{milner91polyadicpi}). We will therefore make liberal use of
$(\nu\; \vec{x})P$.

% subsection the_syntax_and_semantics_of_the_notation_system (end)   

\input{qm2pi.qmops} 

\input{qm2pi.sterngerlach} 

\input{qm2pi.metric} 

% section concurrent_process_calculi (end)

%\input{qm2pi.proofsketch}

% section proof sketch (end)

%\input{qm2pi.slviaknots} 

% section spatial logic via knots (end)

\input{qm2pi.conclusion}

% section conclusion (end)

%\input{qm2pi.dtcodes} 

% section wiring algorithm (end)

\input{qm2pi.ack} 

% section acknowledgments (end)

\newpage


\bibliographystyle{plain}   
\bibliography{../../biblios/main.bib}

\input{qm2pi.rhodetails}

\end{document}

 

%\ifpdf
%\usepackage[pdftex]{graphicx}
%\else
%\usepackage{graphicx}
%\fi

 % \ifpdf
%  \usepackage{pdfsync}
%  \if


%\title{Brief Article}
%\author{David F. Snyder}
%\author{L.G. Meredith}

%\address{Dept. of Math., Texas State University--San Marcos, San Marcos, TX 78666}
       
\pagestyle{empty}


\begin{document}

\lstset{language=[Objective]Caml,frame=shadowbox}

\documentclass[12pt]{llncs}
%\documentclass{jktr}

\usepackage[pdftex]{hyperref}                   
\usepackage {listings}
\usepackage {mathpartir}
\usepackage{bcprules}
%\usepackage{listings}
                       
\usepackage{graphicx} 
%\usepackage[margins=2.5cm,nohead,nofoot]{geometry}
%\usepackage{geometry}
\usepackage{amsfonts}
\usepackage{amstext}
\usepackage{latexsym}
\usepackage{amssymb}
\usepackage{color}


%\include{myPreamble}
\include{qm2pi.local} 

%\ifpdf
%\usepackage[pdftex]{graphicx}
%\else
%\usepackage{graphicx}
%\fi

 % \ifpdf
%  \usepackage{pdfsync}
%  \if


%\title{Brief Article}
%\author{David F. Snyder}
%\author{L.G. Meredith}

%\address{Dept. of Math., Texas State University--San Marcos, San Marcos, TX 78666}
       
\pagestyle{empty}


\begin{document}

\lstset{language=[Objective]Caml,frame=shadowbox}

\input{qm2pi.front}

% section front matter (end)

\input{qm2pi.intro} 
 
% section introduction (end)

% \input{qm2pi.knotations} 

% section notation (end)

\input{qm2pi.process.calculi} 

% section concurrent_process_calculi_and_spatial_logics_ (end)
    
%\input{qm2pi.knots2pi} 

%\input{qm2pi.trefoil} 

%\input{qm2pi.mainthm} 

% subsection basic_interpretation (end)

%\input{qm2pi.rho.presentation} 
\subsection{The syntax and semantics of the notation system}\label{sub:the_syntax_and_semantics_of_the_notation_system} % (fold)

We now summarize a technical presentation of the calculus that
embodies our theory of dynamics. The typical presentation of such a
calculus follows the style of giving generators and relations on
them. The grammar, below, describing term constructors, freely
generates the set of processes, $\Proc$. This set is then quotiented
by a relation known as structural congruence and it is over this set
that the notion of dynamics is expressed. This presentation is
essentially that of \cite{MeredithR05} with the addition of
polyadicity and summation. For readability we have relegated some of
the technical subtleties to an appendix.

\subsubsection{Process grammar}\label{subsub:process_grammar}

\begin{mathpar}
  \inferrule* [lab=synchronization] {} {{M} \bc \pzero \;|\; x?F \;|\; x!C }
  \and
  \inferrule* [lab=abstraction] {} {{F} \bc (x)P}
  \and
  \inferrule* [lab=concretion] {} {{C} \bc \langle Q \rangle}
  \and
  \inferrule* [lab=process] {} {{P,Q} \bc M \;| \;P|Q \;|\; @{x}}
  \and
  \inferrule* [lab=name] {} {{x} \bc \quotep{P}}
\end{mathpar} 

Note that $\vec{x}$ (resp. $\vec{P}$) denotes a vector of names
(resp. processes) of length $|\vec{x}|$ (resp. $|\vec{P}|$). We adopt
the following useful abbreviations.

\begin{mathpar}
   x?(\vec{y}).P := x.(\vec{y})P \and  x\clift{\vec{P}} := x.\clift{\vec{P}}
   \and x!(y) := \lift{x}{\dropn{y}}
   \and \Pi_{i=0}^{n-1}P_i := P_0 | \ldots | P_{n-1}
\end{mathpar}

\subsubsection{Structural congruence}

\paragraph{Free and bound names and alpha-equivalence.} At the
core of structural equivalence is alpha-equivalence which identifies
process that are the same up to a change of variable. Formally, we
recognize the distinction between free and bound names. The free names
of a process, $\freenames{P}$, may be calculated recursively as
follows:

\begin{mathpar}
\freenames{\pzero} := \emptyset
  \and \\
  \freenames{x?(y).P} := \{ x \} \cup (\freenames{P} \setminus \{ y \})
  \and 
  \freenames{x!\langle P \rangle} := \{ x \} \cup \{ P \} 
  \and \\
  \freenames{P|Q} := \freenames{P} \cup \freenames{Q}
  \and \\
  \freenames{@{x}} := \{ x \}
\end{mathpar}

$\pi$
$\quotep{\pi}$

$\freenames{-} : \pi \to \mathcal{P}(\quotep{\pi})$

\begin{eqnarray*}
  \freenames{\pzero} & := & \emptyset \\
  \freenames{x?(y).P} & := & \{ x \} \cup (\freenames{P} \setminus \{ y \}) \\
  \freenames{x!\langle P \rangle} & := & \{ x \} \cup \{ P \} \\
  \freenames{P|Q} & := & \freenames{P} \cup \freenames{Q} \\
  \freenames{\dropn{x}} & := & \{ x \}
\end{eqnarray*}

The bound names of a process, $\boundnames{P}$, are those names occurring in $P$
that are not free. For example, in $x?(y).0$, the name $x$ is free, while $y$ is bound.

\begin{mathpar}
  \inferrule* [lab=monoidal-laws] {} { P|Q \equiv Q|P \and P|0 \equiv P \and P|(Q|R) \equiv (P|Q)|R }
\end{mathpar}

\begin{mathpar}
  \inferrule* [lab=alpha-equivalence] {} { (x)P \equiv (y)P\{y/x\} \and y \not\in \freenames{P} }
\end{mathpar}

\begin{definition}
Then two processes, $P,Q$, are alpha-equivalent if $P = Q\{\vec{y}/\vec{x}\}$ for
some $\vec{x} \in \boundnames{Q},\vec{y} \in \boundnames{P}$, where $Q\{\vec{y}/\vec{x}\}$
denotes the capture-avoiding substitution of $\vec{y}$ for $\vec{x}$ in $Q$.
\end{definition}

\begin{definition}
  The {\em structural congruence} \cite{SangiorgiWalker} , $\equiv$,
  between processes is the least congruence containing
  alpha-equivalence, satisfying the abelian monoid laws
  (associativity, commutativity and $\pzero$ as identity) for parallel
  composition $|$ and for summation $+$.
\end{definition}

\subsection{Name equivalence}

We take name equivalence, written $\nameeq$, to be the smallest
equivalence relation generated by the following rules.

\begin{mathpar}
\inferrule*[lab=Quote-drop]
{ }
{ \quotep{@{x}} \nameeq x }

\inferrule*[lab=Struct-equiv]
{ P \scong Q }
{ \quotep{P} \nameeq \quotep{Q} }
\end{mathpar}

The astute reader will have noticed that the mutual recursion of names
and processes imposes a mutual recursion on alpha-equivalence and
structural equivalence via name-equivalence. Fortunately, all of this
works out pleasantly and we may calculate in the natural way, free of
concern. The reader interested in the details is referred to the
appendix \ref{appendix:rho_details}.

\subsection{Substitution}

We use $\Proc$ for the set of processes, $\QProc$ for the set of
names, and $\id{\{}\vec{y} / \vec{x} \id{\}}$ to denote partial maps,
$s : \QProc \rightarrow \QProc$. A map, $s$ lifts, uniquely, to a map
on process terms, $\widehat{s} : \Proc \rightarrow \Proc$ by the
following equations.

\begin{mathpar}
  (0) \psubstp{Q}{P} := 0 \\
  (R \juxtap S) \psubstp{Q}{P}
  :=    
  (R)\psubstp{Q}{P} \juxtap (S) \psubstp{Q}{P} \\
  (x?(y).R) \psubstp{Q}{P}    
  :=    
  (x)\substp{Q}{P} (z)\concat( (R \psubstn{z}{y}) \psubstp{Q}{P} ) \\
  (\lift{x}{R}) \psubstp{Q}{P}  
  :=
  \lift{(x)\substp{Q}{P}}{ R \psubstp{Q}{P} } \\
%   (\dropn{x})  \psubstp{Q}{P}       
%   := 
%   \left\{ 
%     \begin{array}{ccc} 
%       \dropn{\quotep{Q}} & & x \nameeq \quotep{P} \\
%       \dropn{x} & & otherwise \\
%     \end{array}
%   \right. 
  (\dropn{x})  \psubstp{Q}{P}       
  := 
  \left\{ 
    \begin{array}{ccc} 
      Q & & x \nameeq \quotep{P} \\
      \dropn{x} & & otherwise \\
    \end{array}
  \right.
\end{mathpar}
 

where

\begin{eqnarray}
  (x)\id{\{} \lpquote Q \rpquote / \lpquote P \rpquote \id{\}}            = 
  \left\{ 
    \begin{array}{ccc}
      \lpquote Q \rpquote & & x \nameeq \lpquote P \rpquote \\
      x & & otherwise \\
    \end{array}
  \right. \nonumber
\end{eqnarray}

and $z$ is chosen distinct from $\quotep{P}$, $\quotep{Q}$, the free
names in $Q$, and all the names in $R$. Our $\alpha$-equivalence will
be built in the standard way from this substitution.

\begin{remark}\label{rem:no_self_referential_names}
  One consequence of these definitions is that $\forall P. \quotep{P}
  \not\in \freenames{P}$.
\end{remark}

\subsection{ Dynamic quote: an example }

Anticipating something of what's to come, consider applying the
substitution, $\widehat{\id{\{}u / z \id{\}}}$, to the following pair
of processes, $\lift{w}{y!(z)}$ and $w[ \lpquote y!(z) \rpquote ]$.

\begin{eqnarray}
	\lift{w}{y!(z)}\widehat{\id{\{}u / z \id{\}}}
		& = &
		\lift{w}{y!(u)} \nonumber\\
	w[ \lpquote y!(z) \rpquote ] \widehat{ \id{\{}u / z \id{\}} }
		& = &
		w[ \lpquote y!(z) \rpquote ] \nonumber
\end{eqnarray}

Because the body of the process between quotes is impervious to
substitution, we get radically different answers. In fact, by
examining the first process in an input context,
e.g. $x?(z).\lift{w}{y!(z)}$, we see that the process under the lift
operator may be shaped by prefixed inputs binding a name inside it. In
this sense, the lift operator will be seen as a way to dynamically
construct processes before reifying them as names.

Finally equipped with these standard features we can present the
dynamics of the calculus.

\subsubsection{Operational semantics} 

Finally, we introduce the computational dynamics. What marks these
algebras as distinct from other more traditionally studied algebraic
structures, e.g. vector spaces or polynomial rings, is the manner in
which dynamics is captured. In traditional structures, dynamics is typically
expressed through morphisms between such structures, as in linear maps
between vector spaces or morphisms between rings. In algebras
associated with the semantics of computation, the dynamics is
expressed as part of the algebraic structure itself, through a
reduction reduction relation typically denoted by $\red$. Below, we
give a recursive presentation of this relation for the calculus used
in the encoding.

$\red \subseteq \pi \times \pi$
$\red : \pi \to \mathcal{P}(\pi)$

\begin{mathpar}
  \inferrule* [lab=Comm] { \textsf{match}( x_{src}, x_{trgt} ) } { x_{trgt}?(y)P \; | \; x_{src}!\langle {Q} \rangle \red P\{\quotep{Q}/y}\} }
  \and \\
  \inferrule* [lab=Par] {{P} \red {P}'} {{{P} | {Q}} \red {{P}' | {Q}}}
  \and
  \inferrule* [lab=Equiv]{{{P} \scong {P}'} \andalso {{P}' \red {Q}'} \andalso {{Q}' \scong {Q}}}{{P} \red {Q}}
\end{mathpar}

\begin{eqnarray*}
  match_{\equiv} (\quotep{P},\quotep{Q}) & := & P \equiv Q \\
  match_{\dagger}(\quotep{P},\quotep{Q}) & := & \forall R. P|Q \red^{*} R => R \red^{*} 0 \\
  match_{K}(\quotep{P},\quotep{Q}) & := & K \mbox{ for some context } K
\end{eqnarray*}

$u?(x)P | u!\langle Q \rangle \red P\{\quotep{Q}/x\}$

%We write $\wred$ for $\red^*$, and $P\red$ if $\exists Q $ such that $ P \red Q$.
We write $P\red$ if $\exists Q $ such that $ P \red Q$ and $P\not\red$, otherwise.

\section{Replication}

As mentioned before, it is known that replication (and hence
recursion) can be implemented in a higher-order process algebra
\cite{SangiorgiWalker}. As our first example of calculation with the
machinery thus far presented we give the construction explicitly in
the {\rhoc}.

\begin{eqnarray}
	D_{x} & := & \prefix{x}{y}{(\binpar{\outputp{x}{y}}{@{y}})} \nonumber\\
	\bangp_{x}{P} & := & \binpar{{x}!\langle{\binpar{D_{x}}{P}}\rangle}{D_{x}} \nonumber
\end{eqnarray}

\begin{eqnarray}
	\bangp_{x}{P} & & \nonumber\\
	=
	& {x}!\langle{(\prefix{x}{y}{(\outputp{x}{y} | @{y})) | P}}\rangle 
	      | \prefix{x}{y}{(\outputp{x}{y} | @{y})} & \nonumber\\
	\red
	& (\outputp{x}{y} | @{y})\substn{\quotep{(\prefix{x}{y}{(@{y} | \outputp{x}{y})) | P}}}{y} & \nonumber\\
	=
	& \outputp{x}{\quotep{(\prefix{x}{y}{(\outputp{x}{y} | @{y})) | P}}}
	  | {(\prefix{x}{y}{(\outputp{x}{y} | @{y})) | P}} & \nonumber\\
	\red
	& \ldots & \nonumber\\
	\red^*
	& P | P | \ldots & \nonumber
\end{eqnarray}

Of course, this encoding, as an implementation, runs away, unfolding
$\bangp{P}$ eagerly. A lazier and more implementable replication
operator, restricted to input-guarded processes, may be obtained as follows.

\begin{eqnarray}
\bangp{\prefix{u}{v}{P}} 
	:= 
	\binpar{\lift{x}{\prefix{u}{v}{(\binpar{D(x)}{P})}}}{D(x)} \nonumber
\end{eqnarray}

\begin{remark}
  Note that the lazier definition still does not deal with summation
  or mixed summation (i.e. sums over input and output). The reader is
  invited to construct definitions of replication that deal with these
  features. 

  Further, the definitions are parameterized in a name, $x$. Can you,
  gentle reader, make a definition that eliminates this parameter and
  guarantees no accidental interaction between the replication
  machinery and the process being replicated -- i.e. no accidental
  sharing of names used by the process to get its work done and the
  name(s) used by the replication to effect copying. This latter
  revision of the definition of replication is crucial to obtaining
  the expected identity $!!P \sim !P$.
\end{remark}

\begin{remark}\label{rem:paradoxical_combinator}
  The reader familiar with the lambda calculus will have noticed the
  similarity between $D$ and the paradoxical combinator.

  [Ed. note: the existence of this seems to suggest we have to be more
  restrictive on the set of processes and names we admit if we are to
  support no-cloning.]
\end{remark}

\subsubsection{Bisimulation}

The computational dynamics gives rise to another kind of equivalence,
the equivalence of computational behavior. As previously mentioned
this is typically captured \emph{via} some form of bisimulation.

% The notion we use in this paper is weak barbed bisimulation
% \cite{milner91polyadicpi}.

The notion we use in this paper is derived from weak barbed
bisimulation \cite{milner91polyadicpi}. 

\begin{definition}
An \emph{observation relation}, $\downarrow_{\mathcal N}$, over a set
of names, $\mathcal N$, is the smallest relation satisfying the rules
below.

\infrule[Out-barb]{y \in {\mathcal N}, \; x \nameeq y}
		  {\outputp{x}{v} \downarrow_{\mathcal N} x}
\infrule[Par-barb]{\mbox{$P\downarrow_{\mathcal N} x$ or $Q\downarrow_{\mathcal N} x$}}
		  {\binpar{P}{Q} \downarrow_{\mathcal N} x}

We write $P \Downarrow_{\mathcal N} x$ if there is $Q$ such that 
$P \wred Q$ and $Q \downarrow_{\mathcal N} x$.
\end{definition}

\begin{definition}
%\label{def.bbisim}
An  ${\mathcal N}$-\emph{barbed bisimulation} over a set of names, ${\mathcal N}$, is a symmetric binary relation 
${\mathcal S}_{\mathcal N}$ between agents such that $P\rel{S}_{\mathcal N}Q$ implies:
\begin{enumerate}
\item If $P \red P'$ then $Q \wred Q'$ and $P'\rel{S}_{\mathcal N} Q'$.
\item If $P\downarrow_{\mathcal N} x$, then $Q\Downarrow_{\mathcal N} x$.
\end{enumerate}
$P$ is ${\mathcal N}$-barbed bisimilar to $Q$, written
$P \wbbisim_{\mathcal N} Q$, if $P \rel{S}_{\mathcal N} Q$ for some ${\mathcal N}$-barbed bisimulation ${\mathcal S}_{\mathcal N}$.
\end{definition}

$\mathcal{R} \subseteq \pi \times \pi$

$P \mathcal{R} Q => \forall P'. P \red P' \Rightarrow \exists Q'. Q \red Q', P' \mathcal{R} Q'$

$P \vdash x \Rightarrow Q \vdash x$

\begin{mathpar}
  \inferrule*[lab=Out-barb]{x \nameeq y}{{y}!\langle{Q}\rangle \vdash x}
  \and
  \inferrule*[lab=Par-barb]{\mbox{$P\vdash x$ or $Q\vdash x$}}{\binpar{P}{Q} \vdash x}
\end{mathpar}

\subsubsection{Contexts}

One of the principle advantages of computational calculi like the
$\pi$-calculus is a well-defined notion of context,
contextual-equivalence and a correlation between
contextual-equivalence and notions of bisimulation. The notion of
context allows the decomposition of a process into (sub-)process and
its syntactic environment, its context. Thus, a context may be
thought of as a process with a ``hole'' (written $\Box$) in it. The
application of a context $M$ to a process $P$, written $M[P]$, is
tantamount to filling the hole in $M$ with $P$. In this paper we do
not need the full weight of this theory, but do make use of the notion
of context in the proof the main theorem. 

\begin{mathpar}
  \inferrule* [lab=summation] {} {{M_{M},M_{N}} \bc \Box \;|\; x.M_{A} \;|\; M_{M}+M_{N}}
  \and
  \inferrule* [lab=agent] {} {{M_{A}} \bc (\vec{x})M_{P} \;| \; \clift{P_0,\ldots,M_{P},\ldots,P_N}}
  \and \\
  \inferrule* [lab=process] {} {{M_{P}} \bc M_{N} \;| \;P|M_{P} }
\end{mathpar} 

\begin{mathpar}
  \inferrule* [lab=sychronization] {} {M_{N} \bc \Box \;|\; x?M_{F} \;|\; x!M_{C}}
  \and
  \inferrule* [lab=abstraction] {} {{M_{F}} \bc (x)M_{P} }
  \and
  \inferrule* [lab=concretion] {} {{M_{C}} \bc \langle M_{P} \rangle }
  \and \\
  \inferrule* [lab=process] {} {{M_{P}} \bc M_{N} \;| \;P|M_{P} }
\end{mathpar}

\begin{definition}[contextual application] Given a context $M$, and
  process $P$, we define the \emph{contextual application}, $M[P] :=
  M\{P/\Box\}$. That is, the contextual application of M to P is the
  substitution of $P$ for $\Box$ in $M$.
\end{definition}

$\meaningof{-} : L \to \mathcal{P}(\pi)$

\begin{mathpar}
  \inferrule* [lab=collection] {} {\meaningof{true} = \pi, \and \meaningof{~E} = \pi \setminus \meaningof{E}, \and \meaningof{E_{1} \& E_{2}} = \meaningof{E_{1}} \cap \meaningof{E_{2}}}
\end{mathpar}

\begin{mathpar}
  \inferrule* [lab=structure] {} {\meaningof{0} = \{ P \in \pi | P \equiv 0 \}, \and \\ \meaningof{E_1 | E_2} = \{ P \in \pi | P \equiv P_{1} | P_{2}, P_{1} \in \meaningof{E_{1}}, P_{2} \in \meaningof{E_2}\} }
\end{mathpar}

\begin{mathpar}
 \inferrule* [lab=behavior] {} {\meaningof{\langle a?b \rangle E} = \{ P \in \pi | P \equiv Q | u?(y)P', \\ \and \\\\ \and \\ \;\;\; u \in \meaningof{a}, \forall z.P'\{z/y\} \in \meaningof{E\{z/b\}}\}, \and \\ \meaningof{a!E} = \{ P \in \pi | P \equiv Q | x!\langle P' \rangle, x \in \meaningof{a} P' \in \meaningof{E}\} }
\end{mathpar}

\begin{mathpar}
 \inferrule* [lab=nominal] {} {\meaningof{\quotep{E}} = \{ \quotep{P} \in \quotep{\pi} | P \in \meaningof{E} \}, \and \meaningof{\quotep{P}} = \{ \quotep{Q} \in \quotep{\pi} | P \equiv Q \} \and \\ \meaningof{@\quotep{E}} = \{ P \in \pi | P \equiv @x, x \in \meaningof{E} \}}
\end{mathpar}

\begin{eqnarray*}
  \\
  \meaningof{-} : TS \to ST
\end{eqnarray*}

\begin{eqnarray*}
  \\
  L : TS \to ST
\end{eqnarray*}

\begin{eqnarray*}
  \\
  P \models E \iff P \in \meaningof{E}
\end{eqnarray*}

\begin{eqnarray*}
  P \approx_{L} Q \iff \forall E \in L. P \models E \iff Q \models E
\end{eqnarray*}

\begin{eqnarray*}
  P \approx_{K} Q
\end{eqnarray*}

\begin{eqnarray*}
  P \approx Q
\end{eqnarray*}

$\approx_{K} = \approx = \approx_{L}$

\subsubsection{Contextual duality}

Note that contexts extend the quotation operation to a family of
operations from processes to names. Given a context, $M$, we can
define a \emph{nominal context}, $\quotep{M}$ by $\quotep{M}[P] :=
\quotep{M[P]}$. To foreshadow what is to come we observe that these
operations enjoy a duality with processes very much like the duality
between vectors and maps from vectors to scalars.

Further, because the calculus is essentially higher-order, we have a
correspondence between contexts and processes. More specifically,
given a name $x$ and a context $M$ we can construct $M^{*}_{x}$ such
that 

\begin{mathpar}
  M^{*}_{x} | \lift{x}{P} \red M[P]
\end{mathpar}

namely,

\begin{mathpar}
  M^{*}_{x} := x?(u).M[\dropn{u}]
\end{mathpar}

The dependence of $M^{*}_{x}$ on a name makes it an abstraction, 

\begin{mathpar}
  M^{*} := (x)x?(u).M[\dropn{u}]
\end{mathpar}

\subsection{Additional notation}

It will sometimes be convenient to denote the process a name
quotes. We already have the notation $x = \quotep{P}$, but it will be
convenient to introduce an alternate notation, $\procn{x}$, when we
want to emphasize the connection to the use of the name. Note that, by
virtue of name equivalence, $\quotep{\procn{x}} \nameeq x$; so, the
notation is consistent with previous definitions.

Further, because names have structure it is possible to effect
substitutions on the basis of that structure. This means we need to
upgrade our notation for substitutions, which we accomplish by
adapting comprehension notation. Thus,

\begin{mathpar}
  P\{ y / x : x \in S \}
\end{mathpar}

is interpreted to mean the process derived from P by replacing (in a
capture-avoiding manner) each occurrence of $x$ in $S$ by $y$. For example,

\begin{mathpar}
  P\{ \quotep{\procn{x}|\procn{x}} / x : x \in \freenames{P} \}
\end{mathpar}

will replace each (occurrence) of a free name $x$ in $P$ by
$\quotep{\procn{x}|\procn{x}}$.

Also, we will avail ourselves of the notation $x^{L}$ and $x^{R}$ to
denote injections of a name into disjoint copies of the name
space. There are numerous ways to accomplish this. One example can be
found in \cite{MeredithR05}. This notation overloads to vectors of
names: $\vec{x}^{\pi} := (x_{i}^{\pi} \; : \; 0 \leq i < |\vec{x}| )$ where $\pi \in \{L,R\}$.

We also use $P^{\Box} := P|\Box$.

In \cite{MeredithR05} an interpretation of the new operator is
given. It turns out that there are several possible interpretations
all enjoying the requisite algebraic properties of the operator (see
\cite{milner91polyadicpi}). We will therefore make liberal use of
$(\nu\; \vec{x})P$.

% subsection the_syntax_and_semantics_of_the_notation_system (end)   

\input{qm2pi.qmops} 

\input{qm2pi.sterngerlach} 

\input{qm2pi.metric} 

% section concurrent_process_calculi (end)

%\input{qm2pi.proofsketch}

% section proof sketch (end)

%\input{qm2pi.slviaknots} 

% section spatial logic via knots (end)

\input{qm2pi.conclusion}

% section conclusion (end)

%\input{qm2pi.dtcodes} 

% section wiring algorithm (end)

\input{qm2pi.ack} 

% section acknowledgments (end)

\newpage


\bibliographystyle{plain}   
\bibliography{../../biblios/main.bib}

\input{qm2pi.rhodetails}

\end{document}



% section front matter (end)

\section{Introduction}\label{sec:introduction} % (fold)
In this draft of the material i am going to have to dispense with the
usual writing conventions adopted in papers on these topics. i'm going
to have adopt whatever tone i need at the time i'm writing up the
calculations. Sometimes this may be very conversational; others it may
be the barest mathematical grunts; others still it may be that i have
lifted text from one of my other papers because the exposition of some
point was better said there. i hope that my readers are not unduly put
out by this decision. i'm not doing this to flout convention or be
rebellious. i find these calculations very technically challenging. To
keep everything going technically, something has to give; i have to
let go of some cognitive burden. So, the academic writing style --
with all of its trade-offs in terms of facilitating technical
communication -- is what i'm letting go of. Perhaps subsequent drafts
can be tightened and polished, but for now, i'm going to speak as if
we were sitting together in a coffee shop with a laptop, wifi and a
pad of paper and a pencil.

So, here's what i have to say. We -- you and i, comfortably ensconced
in our coffee shop and well-equipped with our tools -- can realize and
carry out the calculations of quantum mechanics over a very different
formal theory of dynamics, a formal theory of dynamics that
corresponds to a theory of concurrent computation with
\emph{reflection}. It has the advantage that the underlying theory is
already `quantized', but supports analogues all of the continuuous
operations. Strikingly, this underlying theory has recently been
connected with a notion of metric that we can show, by calculating
together, coincides with the metric induced by the inner product.

There are a lot of reasons why you might be interested in seeing
calculations of this form. Here's why i'm interested. For the past
several centuries there has been no competitor to the ``Newtonian''
account of dynamics. As a result the predominant share of accounts of
dynamical systems and situations have had to be formulated in terms of
the Newtonian machinery. i view this as an intellectually dangerous
position to occupy. Everything, despite it's intrinsic shape, turns
into a nail to be hit with this hammer. Recently, however, the theory
of computation has matured to the point where we have candidates for
theories of dynamics that offer very different perspective on
reasoning about dynamical systems and situations. Testing these
candidates against very successful accounts of dynamical situations,
like quantum mechanics, is going to give us some sense of how mature
they are and some measure of the quality of these accounts of
dynamics.

\subsection{Summary of contributions and outline of paper}

So, we're going to develop an interpretation of the operations of
quantum mechanics normally interpreted by Hilbert spaces and
operators. We're going to do this over a theory of computation. Note
that this is very different than the usual quantum computation program
which develops notions of computation over quantum mechanics. Rather,
we are developing a story that aligns with Wheeler's slogan: It from
Bit. To do this we will first provide an account of the theory of
computation at play here. Then we will dive into a calculation-driven
interpretation of the operations of quantum mechanics.

The reason we take this approach is that -- until very recently --
there hasn't been an axiomatic account of quantum mechanics. As a
result there has been no sharp delineation of the mathematical theory
supporting interpretation of the physical theory and the physical
theory, itself. So, ambient features of the maths are free to be
exploited (or supressed) without a real accounting of their physical
relevance. There is no sharp statement ``here's the physical theory''
qua \emph{theory} and ``here's the mathematical interpretation''
enabling a judgment of how faithful the interpretation is -- apart
from experimental observation. When there is an axiomatic account we
can judge how well a given mathematical formalism supports an
interpretation of the axioms, independent of
experimentation. Likewise, we can judge how well we have captured our
physical evidence and experience with our axiomatics, independent of
any specific mathematical implementation, with accidental detail that
may or may not have physical significance. 

In lieu of a fully fleshed out and vetted axiomatic account of quantum
mechanics, interpreting the operational notions in service of modeling
physical systems will have to suffice. In other words, we are not in
the business of providing a model of Hilbert spaces and operators. We
are in the business of providing a model of quantum mechanics because
we are motivated by testing our notions of dynamics against physical
theory; and, the predictive calculations of the physical theory must
serve as the best formulation -- shy of a fully fleshed out axiomatic
account -- of the physical theory itself (as they have for scientific
theories since time immemorial). Put another way, despite a
whole-hearted commitment to an It-from-Bit ontology, we are firmly
aligned with the shut-up-and-calculate camp as the best way to obtain
results either from the physical perspective or as a quality assurance
measure of our fledgling theory of dynamics.

In detail, we present a reflective process calculus. Then we develop
intuitive correspondences between the notions available in this
calculus and the usual physical notions supporting quantum mechanical
calculations. Thus, 

\begin{table}[htp]
  \center{
    \fbox{
      \begin{tabular}{c|c}
        quantum mechanics & process calculus \\
        \hline
        scalar & name \\
        state vector & process \\
        dual & contextual duals \\
        matrix & formal sums of process-context-dual pairs \\
        orthogonality & process annihilation \\
        inner product & execution-formula + quoting
      \end{tabular}
    }
  }
  \caption{QM - process calculi correspondences}
\end{table}

Then we tighten up these intuitions to operational definitions. We
employ the Dirac notation as the best proxy we can find for an
abstract syntax of the quantum mechanical notions. The definitions we
develop put us in contact with equational constraints coming from the
theory that we demonstrate the definitions and calculations satisfy.

This puts us in a position to shut up and calculate for the
Stern-Gerlach experimental set up, showing how these predictive
calculations become calculations on processes in our theory of a
reflective process calculus.

Penultimately, we demonstrate that the notion of metric coming from
the inner product coincides with the notion of metric available from
the theory of bisimulation. This demonstration gives us the right to
think of space as arising from behavior. Finally, we consider where we
might go from the new vantage point we have obtained.

% section introduction (end) 
 
% section introduction (end)

% \documentclass[12pt]{llncs}
%\documentclass{jktr}

\usepackage[pdftex]{hyperref}                   
\usepackage {listings}
\usepackage {mathpartir}
\usepackage{bcprules}
%\usepackage{listings}
                       
\usepackage{graphicx} 
%\usepackage[margins=2.5cm,nohead,nofoot]{geometry}
%\usepackage{geometry}
\usepackage{amsfonts}
\usepackage{amstext}
\usepackage{latexsym}
\usepackage{amssymb}
\usepackage{color}


%\include{myPreamble}
\include{qm2pi.local} 

%\ifpdf
%\usepackage[pdftex]{graphicx}
%\else
%\usepackage{graphicx}
%\fi

 % \ifpdf
%  \usepackage{pdfsync}
%  \if


%\title{Brief Article}
%\author{David F. Snyder}
%\author{L.G. Meredith}

%\address{Dept. of Math., Texas State University--San Marcos, San Marcos, TX 78666}
       
\pagestyle{empty}


\begin{document}

\lstset{language=[Objective]Caml,frame=shadowbox}

\input{qm2pi.front}

% section front matter (end)

\input{qm2pi.intro} 
 
% section introduction (end)

% \input{qm2pi.knotations} 

% section notation (end)

\input{qm2pi.process.calculi} 

% section concurrent_process_calculi_and_spatial_logics_ (end)
    
%\input{qm2pi.knots2pi} 

%\input{qm2pi.trefoil} 

%\input{qm2pi.mainthm} 

% subsection basic_interpretation (end)

%\input{qm2pi.rho.presentation} 
\subsection{The syntax and semantics of the notation system}\label{sub:the_syntax_and_semantics_of_the_notation_system} % (fold)

We now summarize a technical presentation of the calculus that
embodies our theory of dynamics. The typical presentation of such a
calculus follows the style of giving generators and relations on
them. The grammar, below, describing term constructors, freely
generates the set of processes, $\Proc$. This set is then quotiented
by a relation known as structural congruence and it is over this set
that the notion of dynamics is expressed. This presentation is
essentially that of \cite{MeredithR05} with the addition of
polyadicity and summation. For readability we have relegated some of
the technical subtleties to an appendix.

\subsubsection{Process grammar}\label{subsub:process_grammar}

\begin{mathpar}
  \inferrule* [lab=synchronization] {} {{M} \bc \pzero \;|\; x?F \;|\; x!C }
  \and
  \inferrule* [lab=abstraction] {} {{F} \bc (x)P}
  \and
  \inferrule* [lab=concretion] {} {{C} \bc \langle Q \rangle}
  \and
  \inferrule* [lab=process] {} {{P,Q} \bc M \;| \;P|Q \;|\; @{x}}
  \and
  \inferrule* [lab=name] {} {{x} \bc \quotep{P}}
\end{mathpar} 

Note that $\vec{x}$ (resp. $\vec{P}$) denotes a vector of names
(resp. processes) of length $|\vec{x}|$ (resp. $|\vec{P}|$). We adopt
the following useful abbreviations.

\begin{mathpar}
   x?(\vec{y}).P := x.(\vec{y})P \and  x\clift{\vec{P}} := x.\clift{\vec{P}}
   \and x!(y) := \lift{x}{\dropn{y}}
   \and \Pi_{i=0}^{n-1}P_i := P_0 | \ldots | P_{n-1}
\end{mathpar}

\subsubsection{Structural congruence}

\paragraph{Free and bound names and alpha-equivalence.} At the
core of structural equivalence is alpha-equivalence which identifies
process that are the same up to a change of variable. Formally, we
recognize the distinction between free and bound names. The free names
of a process, $\freenames{P}$, may be calculated recursively as
follows:

\begin{mathpar}
\freenames{\pzero} := \emptyset
  \and \\
  \freenames{x?(y).P} := \{ x \} \cup (\freenames{P} \setminus \{ y \})
  \and 
  \freenames{x!\langle P \rangle} := \{ x \} \cup \{ P \} 
  \and \\
  \freenames{P|Q} := \freenames{P} \cup \freenames{Q}
  \and \\
  \freenames{@{x}} := \{ x \}
\end{mathpar}

$\pi$
$\quotep{\pi}$

$\freenames{-} : \pi \to \mathcal{P}(\quotep{\pi})$

\begin{eqnarray*}
  \freenames{\pzero} & := & \emptyset \\
  \freenames{x?(y).P} & := & \{ x \} \cup (\freenames{P} \setminus \{ y \}) \\
  \freenames{x!\langle P \rangle} & := & \{ x \} \cup \{ P \} \\
  \freenames{P|Q} & := & \freenames{P} \cup \freenames{Q} \\
  \freenames{\dropn{x}} & := & \{ x \}
\end{eqnarray*}

The bound names of a process, $\boundnames{P}$, are those names occurring in $P$
that are not free. For example, in $x?(y).0$, the name $x$ is free, while $y$ is bound.

\begin{mathpar}
  \inferrule* [lab=monoidal-laws] {} { P|Q \equiv Q|P \and P|0 \equiv P \and P|(Q|R) \equiv (P|Q)|R }
\end{mathpar}

\begin{mathpar}
  \inferrule* [lab=alpha-equivalence] {} { (x)P \equiv (y)P\{y/x\} \and y \not\in \freenames{P} }
\end{mathpar}

\begin{definition}
Then two processes, $P,Q$, are alpha-equivalent if $P = Q\{\vec{y}/\vec{x}\}$ for
some $\vec{x} \in \boundnames{Q},\vec{y} \in \boundnames{P}$, where $Q\{\vec{y}/\vec{x}\}$
denotes the capture-avoiding substitution of $\vec{y}$ for $\vec{x}$ in $Q$.
\end{definition}

\begin{definition}
  The {\em structural congruence} \cite{SangiorgiWalker} , $\equiv$,
  between processes is the least congruence containing
  alpha-equivalence, satisfying the abelian monoid laws
  (associativity, commutativity and $\pzero$ as identity) for parallel
  composition $|$ and for summation $+$.
\end{definition}

\subsection{Name equivalence}

We take name equivalence, written $\nameeq$, to be the smallest
equivalence relation generated by the following rules.

\begin{mathpar}
\inferrule*[lab=Quote-drop]
{ }
{ \quotep{@{x}} \nameeq x }

\inferrule*[lab=Struct-equiv]
{ P \scong Q }
{ \quotep{P} \nameeq \quotep{Q} }
\end{mathpar}

The astute reader will have noticed that the mutual recursion of names
and processes imposes a mutual recursion on alpha-equivalence and
structural equivalence via name-equivalence. Fortunately, all of this
works out pleasantly and we may calculate in the natural way, free of
concern. The reader interested in the details is referred to the
appendix \ref{appendix:rho_details}.

\subsection{Substitution}

We use $\Proc$ for the set of processes, $\QProc$ for the set of
names, and $\id{\{}\vec{y} / \vec{x} \id{\}}$ to denote partial maps,
$s : \QProc \rightarrow \QProc$. A map, $s$ lifts, uniquely, to a map
on process terms, $\widehat{s} : \Proc \rightarrow \Proc$ by the
following equations.

\begin{mathpar}
  (0) \psubstp{Q}{P} := 0 \\
  (R \juxtap S) \psubstp{Q}{P}
  :=    
  (R)\psubstp{Q}{P} \juxtap (S) \psubstp{Q}{P} \\
  (x?(y).R) \psubstp{Q}{P}    
  :=    
  (x)\substp{Q}{P} (z)\concat( (R \psubstn{z}{y}) \psubstp{Q}{P} ) \\
  (\lift{x}{R}) \psubstp{Q}{P}  
  :=
  \lift{(x)\substp{Q}{P}}{ R \psubstp{Q}{P} } \\
%   (\dropn{x})  \psubstp{Q}{P}       
%   := 
%   \left\{ 
%     \begin{array}{ccc} 
%       \dropn{\quotep{Q}} & & x \nameeq \quotep{P} \\
%       \dropn{x} & & otherwise \\
%     \end{array}
%   \right. 
  (\dropn{x})  \psubstp{Q}{P}       
  := 
  \left\{ 
    \begin{array}{ccc} 
      Q & & x \nameeq \quotep{P} \\
      \dropn{x} & & otherwise \\
    \end{array}
  \right.
\end{mathpar}
 

where

\begin{eqnarray}
  (x)\id{\{} \lpquote Q \rpquote / \lpquote P \rpquote \id{\}}            = 
  \left\{ 
    \begin{array}{ccc}
      \lpquote Q \rpquote & & x \nameeq \lpquote P \rpquote \\
      x & & otherwise \\
    \end{array}
  \right. \nonumber
\end{eqnarray}

and $z$ is chosen distinct from $\quotep{P}$, $\quotep{Q}$, the free
names in $Q$, and all the names in $R$. Our $\alpha$-equivalence will
be built in the standard way from this substitution.

\begin{remark}\label{rem:no_self_referential_names}
  One consequence of these definitions is that $\forall P. \quotep{P}
  \not\in \freenames{P}$.
\end{remark}

\subsection{ Dynamic quote: an example }

Anticipating something of what's to come, consider applying the
substitution, $\widehat{\id{\{}u / z \id{\}}}$, to the following pair
of processes, $\lift{w}{y!(z)}$ and $w[ \lpquote y!(z) \rpquote ]$.

\begin{eqnarray}
	\lift{w}{y!(z)}\widehat{\id{\{}u / z \id{\}}}
		& = &
		\lift{w}{y!(u)} \nonumber\\
	w[ \lpquote y!(z) \rpquote ] \widehat{ \id{\{}u / z \id{\}} }
		& = &
		w[ \lpquote y!(z) \rpquote ] \nonumber
\end{eqnarray}

Because the body of the process between quotes is impervious to
substitution, we get radically different answers. In fact, by
examining the first process in an input context,
e.g. $x?(z).\lift{w}{y!(z)}$, we see that the process under the lift
operator may be shaped by prefixed inputs binding a name inside it. In
this sense, the lift operator will be seen as a way to dynamically
construct processes before reifying them as names.

Finally equipped with these standard features we can present the
dynamics of the calculus.

\subsubsection{Operational semantics} 

Finally, we introduce the computational dynamics. What marks these
algebras as distinct from other more traditionally studied algebraic
structures, e.g. vector spaces or polynomial rings, is the manner in
which dynamics is captured. In traditional structures, dynamics is typically
expressed through morphisms between such structures, as in linear maps
between vector spaces or morphisms between rings. In algebras
associated with the semantics of computation, the dynamics is
expressed as part of the algebraic structure itself, through a
reduction reduction relation typically denoted by $\red$. Below, we
give a recursive presentation of this relation for the calculus used
in the encoding.

$\red \subseteq \pi \times \pi$
$\red : \pi \to \mathcal{P}(\pi)$

\begin{mathpar}
  \inferrule* [lab=Comm] { \textsf{match}( x_{src}, x_{trgt} ) } { x_{trgt}?(y)P \; | \; x_{src}!\langle {Q} \rangle \red P\{\quotep{Q}/y}\} }
  \and \\
  \inferrule* [lab=Par] {{P} \red {P}'} {{{P} | {Q}} \red {{P}' | {Q}}}
  \and
  \inferrule* [lab=Equiv]{{{P} \scong {P}'} \andalso {{P}' \red {Q}'} \andalso {{Q}' \scong {Q}}}{{P} \red {Q}}
\end{mathpar}

\begin{eqnarray*}
  match_{\equiv} (\quotep{P},\quotep{Q}) & := & P \equiv Q \\
  match_{\dagger}(\quotep{P},\quotep{Q}) & := & \forall R. P|Q \red^{*} R => R \red^{*} 0 \\
  match_{K}(\quotep{P},\quotep{Q}) & := & K \mbox{ for some context } K
\end{eqnarray*}

$u?(x)P | u!\langle Q \rangle \red P\{\quotep{Q}/x\}$

%We write $\wred$ for $\red^*$, and $P\red$ if $\exists Q $ such that $ P \red Q$.
We write $P\red$ if $\exists Q $ such that $ P \red Q$ and $P\not\red$, otherwise.

\section{Replication}

As mentioned before, it is known that replication (and hence
recursion) can be implemented in a higher-order process algebra
\cite{SangiorgiWalker}. As our first example of calculation with the
machinery thus far presented we give the construction explicitly in
the {\rhoc}.

\begin{eqnarray}
	D_{x} & := & \prefix{x}{y}{(\binpar{\outputp{x}{y}}{@{y}})} \nonumber\\
	\bangp_{x}{P} & := & \binpar{{x}!\langle{\binpar{D_{x}}{P}}\rangle}{D_{x}} \nonumber
\end{eqnarray}

\begin{eqnarray}
	\bangp_{x}{P} & & \nonumber\\
	=
	& {x}!\langle{(\prefix{x}{y}{(\outputp{x}{y} | @{y})) | P}}\rangle 
	      | \prefix{x}{y}{(\outputp{x}{y} | @{y})} & \nonumber\\
	\red
	& (\outputp{x}{y} | @{y})\substn{\quotep{(\prefix{x}{y}{(@{y} | \outputp{x}{y})) | P}}}{y} & \nonumber\\
	=
	& \outputp{x}{\quotep{(\prefix{x}{y}{(\outputp{x}{y} | @{y})) | P}}}
	  | {(\prefix{x}{y}{(\outputp{x}{y} | @{y})) | P}} & \nonumber\\
	\red
	& \ldots & \nonumber\\
	\red^*
	& P | P | \ldots & \nonumber
\end{eqnarray}

Of course, this encoding, as an implementation, runs away, unfolding
$\bangp{P}$ eagerly. A lazier and more implementable replication
operator, restricted to input-guarded processes, may be obtained as follows.

\begin{eqnarray}
\bangp{\prefix{u}{v}{P}} 
	:= 
	\binpar{\lift{x}{\prefix{u}{v}{(\binpar{D(x)}{P})}}}{D(x)} \nonumber
\end{eqnarray}

\begin{remark}
  Note that the lazier definition still does not deal with summation
  or mixed summation (i.e. sums over input and output). The reader is
  invited to construct definitions of replication that deal with these
  features. 

  Further, the definitions are parameterized in a name, $x$. Can you,
  gentle reader, make a definition that eliminates this parameter and
  guarantees no accidental interaction between the replication
  machinery and the process being replicated -- i.e. no accidental
  sharing of names used by the process to get its work done and the
  name(s) used by the replication to effect copying. This latter
  revision of the definition of replication is crucial to obtaining
  the expected identity $!!P \sim !P$.
\end{remark}

\begin{remark}\label{rem:paradoxical_combinator}
  The reader familiar with the lambda calculus will have noticed the
  similarity between $D$ and the paradoxical combinator.

  [Ed. note: the existence of this seems to suggest we have to be more
  restrictive on the set of processes and names we admit if we are to
  support no-cloning.]
\end{remark}

\subsubsection{Bisimulation}

The computational dynamics gives rise to another kind of equivalence,
the equivalence of computational behavior. As previously mentioned
this is typically captured \emph{via} some form of bisimulation.

% The notion we use in this paper is weak barbed bisimulation
% \cite{milner91polyadicpi}.

The notion we use in this paper is derived from weak barbed
bisimulation \cite{milner91polyadicpi}. 

\begin{definition}
An \emph{observation relation}, $\downarrow_{\mathcal N}$, over a set
of names, $\mathcal N$, is the smallest relation satisfying the rules
below.

\infrule[Out-barb]{y \in {\mathcal N}, \; x \nameeq y}
		  {\outputp{x}{v} \downarrow_{\mathcal N} x}
\infrule[Par-barb]{\mbox{$P\downarrow_{\mathcal N} x$ or $Q\downarrow_{\mathcal N} x$}}
		  {\binpar{P}{Q} \downarrow_{\mathcal N} x}

We write $P \Downarrow_{\mathcal N} x$ if there is $Q$ such that 
$P \wred Q$ and $Q \downarrow_{\mathcal N} x$.
\end{definition}

\begin{definition}
%\label{def.bbisim}
An  ${\mathcal N}$-\emph{barbed bisimulation} over a set of names, ${\mathcal N}$, is a symmetric binary relation 
${\mathcal S}_{\mathcal N}$ between agents such that $P\rel{S}_{\mathcal N}Q$ implies:
\begin{enumerate}
\item If $P \red P'$ then $Q \wred Q'$ and $P'\rel{S}_{\mathcal N} Q'$.
\item If $P\downarrow_{\mathcal N} x$, then $Q\Downarrow_{\mathcal N} x$.
\end{enumerate}
$P$ is ${\mathcal N}$-barbed bisimilar to $Q$, written
$P \wbbisim_{\mathcal N} Q$, if $P \rel{S}_{\mathcal N} Q$ for some ${\mathcal N}$-barbed bisimulation ${\mathcal S}_{\mathcal N}$.
\end{definition}

$\mathcal{R} \subseteq \pi \times \pi$

$P \mathcal{R} Q => \forall P'. P \red P' \Rightarrow \exists Q'. Q \red Q', P' \mathcal{R} Q'$

$P \vdash x \Rightarrow Q \vdash x$

\begin{mathpar}
  \inferrule*[lab=Out-barb]{x \nameeq y}{{y}!\langle{Q}\rangle \vdash x}
  \and
  \inferrule*[lab=Par-barb]{\mbox{$P\vdash x$ or $Q\vdash x$}}{\binpar{P}{Q} \vdash x}
\end{mathpar}

\subsubsection{Contexts}

One of the principle advantages of computational calculi like the
$\pi$-calculus is a well-defined notion of context,
contextual-equivalence and a correlation between
contextual-equivalence and notions of bisimulation. The notion of
context allows the decomposition of a process into (sub-)process and
its syntactic environment, its context. Thus, a context may be
thought of as a process with a ``hole'' (written $\Box$) in it. The
application of a context $M$ to a process $P$, written $M[P]$, is
tantamount to filling the hole in $M$ with $P$. In this paper we do
not need the full weight of this theory, but do make use of the notion
of context in the proof the main theorem. 

\begin{mathpar}
  \inferrule* [lab=summation] {} {{M_{M},M_{N}} \bc \Box \;|\; x.M_{A} \;|\; M_{M}+M_{N}}
  \and
  \inferrule* [lab=agent] {} {{M_{A}} \bc (\vec{x})M_{P} \;| \; \clift{P_0,\ldots,M_{P},\ldots,P_N}}
  \and \\
  \inferrule* [lab=process] {} {{M_{P}} \bc M_{N} \;| \;P|M_{P} }
\end{mathpar} 

\begin{mathpar}
  \inferrule* [lab=sychronization] {} {M_{N} \bc \Box \;|\; x?M_{F} \;|\; x!M_{C}}
  \and
  \inferrule* [lab=abstraction] {} {{M_{F}} \bc (x)M_{P} }
  \and
  \inferrule* [lab=concretion] {} {{M_{C}} \bc \langle M_{P} \rangle }
  \and \\
  \inferrule* [lab=process] {} {{M_{P}} \bc M_{N} \;| \;P|M_{P} }
\end{mathpar}

\begin{definition}[contextual application] Given a context $M$, and
  process $P$, we define the \emph{contextual application}, $M[P] :=
  M\{P/\Box\}$. That is, the contextual application of M to P is the
  substitution of $P$ for $\Box$ in $M$.
\end{definition}

$\meaningof{-} : L \to \mathcal{P}(\pi)$

\begin{mathpar}
  \inferrule* [lab=collection] {} {\meaningof{true} = \pi, \and \meaningof{~E} = \pi \setminus \meaningof{E}, \and \meaningof{E_{1} \& E_{2}} = \meaningof{E_{1}} \cap \meaningof{E_{2}}}
\end{mathpar}

\begin{mathpar}
  \inferrule* [lab=structure] {} {\meaningof{0} = \{ P \in \pi | P \equiv 0 \}, \and \\ \meaningof{E_1 | E_2} = \{ P \in \pi | P \equiv P_{1} | P_{2}, P_{1} \in \meaningof{E_{1}}, P_{2} \in \meaningof{E_2}\} }
\end{mathpar}

\begin{mathpar}
 \inferrule* [lab=behavior] {} {\meaningof{\langle a?b \rangle E} = \{ P \in \pi | P \equiv Q | u?(y)P', \\ \and \\\\ \and \\ \;\;\; u \in \meaningof{a}, \forall z.P'\{z/y\} \in \meaningof{E\{z/b\}}\}, \and \\ \meaningof{a!E} = \{ P \in \pi | P \equiv Q | x!\langle P' \rangle, x \in \meaningof{a} P' \in \meaningof{E}\} }
\end{mathpar}

\begin{mathpar}
 \inferrule* [lab=nominal] {} {\meaningof{\quotep{E}} = \{ \quotep{P} \in \quotep{\pi} | P \in \meaningof{E} \}, \and \meaningof{\quotep{P}} = \{ \quotep{Q} \in \quotep{\pi} | P \equiv Q \} \and \\ \meaningof{@\quotep{E}} = \{ P \in \pi | P \equiv @x, x \in \meaningof{E} \}}
\end{mathpar}

\begin{eqnarray*}
  \\
  \meaningof{-} : TS \to ST
\end{eqnarray*}

\begin{eqnarray*}
  \\
  L : TS \to ST
\end{eqnarray*}

\begin{eqnarray*}
  \\
  P \models E \iff P \in \meaningof{E}
\end{eqnarray*}

\begin{eqnarray*}
  P \approx_{L} Q \iff \forall E \in L. P \models E \iff Q \models E
\end{eqnarray*}

\begin{eqnarray*}
  P \approx_{K} Q
\end{eqnarray*}

\begin{eqnarray*}
  P \approx Q
\end{eqnarray*}

$\approx_{K} = \approx = \approx_{L}$

\subsubsection{Contextual duality}

Note that contexts extend the quotation operation to a family of
operations from processes to names. Given a context, $M$, we can
define a \emph{nominal context}, $\quotep{M}$ by $\quotep{M}[P] :=
\quotep{M[P]}$. To foreshadow what is to come we observe that these
operations enjoy a duality with processes very much like the duality
between vectors and maps from vectors to scalars.

Further, because the calculus is essentially higher-order, we have a
correspondence between contexts and processes. More specifically,
given a name $x$ and a context $M$ we can construct $M^{*}_{x}$ such
that 

\begin{mathpar}
  M^{*}_{x} | \lift{x}{P} \red M[P]
\end{mathpar}

namely,

\begin{mathpar}
  M^{*}_{x} := x?(u).M[\dropn{u}]
\end{mathpar}

The dependence of $M^{*}_{x}$ on a name makes it an abstraction, 

\begin{mathpar}
  M^{*} := (x)x?(u).M[\dropn{u}]
\end{mathpar}

\subsection{Additional notation}

It will sometimes be convenient to denote the process a name
quotes. We already have the notation $x = \quotep{P}$, but it will be
convenient to introduce an alternate notation, $\procn{x}$, when we
want to emphasize the connection to the use of the name. Note that, by
virtue of name equivalence, $\quotep{\procn{x}} \nameeq x$; so, the
notation is consistent with previous definitions.

Further, because names have structure it is possible to effect
substitutions on the basis of that structure. This means we need to
upgrade our notation for substitutions, which we accomplish by
adapting comprehension notation. Thus,

\begin{mathpar}
  P\{ y / x : x \in S \}
\end{mathpar}

is interpreted to mean the process derived from P by replacing (in a
capture-avoiding manner) each occurrence of $x$ in $S$ by $y$. For example,

\begin{mathpar}
  P\{ \quotep{\procn{x}|\procn{x}} / x : x \in \freenames{P} \}
\end{mathpar}

will replace each (occurrence) of a free name $x$ in $P$ by
$\quotep{\procn{x}|\procn{x}}$.

Also, we will avail ourselves of the notation $x^{L}$ and $x^{R}$ to
denote injections of a name into disjoint copies of the name
space. There are numerous ways to accomplish this. One example can be
found in \cite{MeredithR05}. This notation overloads to vectors of
names: $\vec{x}^{\pi} := (x_{i}^{\pi} \; : \; 0 \leq i < |\vec{x}| )$ where $\pi \in \{L,R\}$.

We also use $P^{\Box} := P|\Box$.

In \cite{MeredithR05} an interpretation of the new operator is
given. It turns out that there are several possible interpretations
all enjoying the requisite algebraic properties of the operator (see
\cite{milner91polyadicpi}). We will therefore make liberal use of
$(\nu\; \vec{x})P$.

% subsection the_syntax_and_semantics_of_the_notation_system (end)   

\input{qm2pi.qmops} 

\input{qm2pi.sterngerlach} 

\input{qm2pi.metric} 

% section concurrent_process_calculi (end)

%\input{qm2pi.proofsketch}

% section proof sketch (end)

%\input{qm2pi.slviaknots} 

% section spatial logic via knots (end)

\input{qm2pi.conclusion}

% section conclusion (end)

%\input{qm2pi.dtcodes} 

% section wiring algorithm (end)

\input{qm2pi.ack} 

% section acknowledgments (end)

\newpage


\bibliographystyle{plain}   
\bibliography{../../biblios/main.bib}

\input{qm2pi.rhodetails}

\end{document}

 

% section notation (end)

\input{qm2pi.process.calculi} 

% section concurrent_process_calculi_and_spatial_logics_ (end)
    
%\documentclass[12pt]{llncs}
%\documentclass{jktr}

\usepackage[pdftex]{hyperref}                   
\usepackage {listings}
\usepackage {mathpartir}
\usepackage{bcprules}
%\usepackage{listings}
                       
\usepackage{graphicx} 
%\usepackage[margins=2.5cm,nohead,nofoot]{geometry}
%\usepackage{geometry}
\usepackage{amsfonts}
\usepackage{amstext}
\usepackage{latexsym}
\usepackage{amssymb}
\usepackage{color}


%\include{myPreamble}
\include{qm2pi.local} 

%\ifpdf
%\usepackage[pdftex]{graphicx}
%\else
%\usepackage{graphicx}
%\fi

 % \ifpdf
%  \usepackage{pdfsync}
%  \if


%\title{Brief Article}
%\author{David F. Snyder}
%\author{L.G. Meredith}

%\address{Dept. of Math., Texas State University--San Marcos, San Marcos, TX 78666}
       
\pagestyle{empty}


\begin{document}

\lstset{language=[Objective]Caml,frame=shadowbox}

\input{qm2pi.front}

% section front matter (end)

\input{qm2pi.intro} 
 
% section introduction (end)

% \input{qm2pi.knotations} 

% section notation (end)

\input{qm2pi.process.calculi} 

% section concurrent_process_calculi_and_spatial_logics_ (end)
    
%\input{qm2pi.knots2pi} 

%\input{qm2pi.trefoil} 

%\input{qm2pi.mainthm} 

% subsection basic_interpretation (end)

%\input{qm2pi.rho.presentation} 
\subsection{The syntax and semantics of the notation system}\label{sub:the_syntax_and_semantics_of_the_notation_system} % (fold)

We now summarize a technical presentation of the calculus that
embodies our theory of dynamics. The typical presentation of such a
calculus follows the style of giving generators and relations on
them. The grammar, below, describing term constructors, freely
generates the set of processes, $\Proc$. This set is then quotiented
by a relation known as structural congruence and it is over this set
that the notion of dynamics is expressed. This presentation is
essentially that of \cite{MeredithR05} with the addition of
polyadicity and summation. For readability we have relegated some of
the technical subtleties to an appendix.

\subsubsection{Process grammar}\label{subsub:process_grammar}

\begin{mathpar}
  \inferrule* [lab=synchronization] {} {{M} \bc \pzero \;|\; x?F \;|\; x!C }
  \and
  \inferrule* [lab=abstraction] {} {{F} \bc (x)P}
  \and
  \inferrule* [lab=concretion] {} {{C} \bc \langle Q \rangle}
  \and
  \inferrule* [lab=process] {} {{P,Q} \bc M \;| \;P|Q \;|\; @{x}}
  \and
  \inferrule* [lab=name] {} {{x} \bc \quotep{P}}
\end{mathpar} 

Note that $\vec{x}$ (resp. $\vec{P}$) denotes a vector of names
(resp. processes) of length $|\vec{x}|$ (resp. $|\vec{P}|$). We adopt
the following useful abbreviations.

\begin{mathpar}
   x?(\vec{y}).P := x.(\vec{y})P \and  x\clift{\vec{P}} := x.\clift{\vec{P}}
   \and x!(y) := \lift{x}{\dropn{y}}
   \and \Pi_{i=0}^{n-1}P_i := P_0 | \ldots | P_{n-1}
\end{mathpar}

\subsubsection{Structural congruence}

\paragraph{Free and bound names and alpha-equivalence.} At the
core of structural equivalence is alpha-equivalence which identifies
process that are the same up to a change of variable. Formally, we
recognize the distinction between free and bound names. The free names
of a process, $\freenames{P}$, may be calculated recursively as
follows:

\begin{mathpar}
\freenames{\pzero} := \emptyset
  \and \\
  \freenames{x?(y).P} := \{ x \} \cup (\freenames{P} \setminus \{ y \})
  \and 
  \freenames{x!\langle P \rangle} := \{ x \} \cup \{ P \} 
  \and \\
  \freenames{P|Q} := \freenames{P} \cup \freenames{Q}
  \and \\
  \freenames{@{x}} := \{ x \}
\end{mathpar}

$\pi$
$\quotep{\pi}$

$\freenames{-} : \pi \to \mathcal{P}(\quotep{\pi})$

\begin{eqnarray*}
  \freenames{\pzero} & := & \emptyset \\
  \freenames{x?(y).P} & := & \{ x \} \cup (\freenames{P} \setminus \{ y \}) \\
  \freenames{x!\langle P \rangle} & := & \{ x \} \cup \{ P \} \\
  \freenames{P|Q} & := & \freenames{P} \cup \freenames{Q} \\
  \freenames{\dropn{x}} & := & \{ x \}
\end{eqnarray*}

The bound names of a process, $\boundnames{P}$, are those names occurring in $P$
that are not free. For example, in $x?(y).0$, the name $x$ is free, while $y$ is bound.

\begin{mathpar}
  \inferrule* [lab=monoidal-laws] {} { P|Q \equiv Q|P \and P|0 \equiv P \and P|(Q|R) \equiv (P|Q)|R }
\end{mathpar}

\begin{mathpar}
  \inferrule* [lab=alpha-equivalence] {} { (x)P \equiv (y)P\{y/x\} \and y \not\in \freenames{P} }
\end{mathpar}

\begin{definition}
Then two processes, $P,Q$, are alpha-equivalent if $P = Q\{\vec{y}/\vec{x}\}$ for
some $\vec{x} \in \boundnames{Q},\vec{y} \in \boundnames{P}$, where $Q\{\vec{y}/\vec{x}\}$
denotes the capture-avoiding substitution of $\vec{y}$ for $\vec{x}$ in $Q$.
\end{definition}

\begin{definition}
  The {\em structural congruence} \cite{SangiorgiWalker} , $\equiv$,
  between processes is the least congruence containing
  alpha-equivalence, satisfying the abelian monoid laws
  (associativity, commutativity and $\pzero$ as identity) for parallel
  composition $|$ and for summation $+$.
\end{definition}

\subsection{Name equivalence}

We take name equivalence, written $\nameeq$, to be the smallest
equivalence relation generated by the following rules.

\begin{mathpar}
\inferrule*[lab=Quote-drop]
{ }
{ \quotep{@{x}} \nameeq x }

\inferrule*[lab=Struct-equiv]
{ P \scong Q }
{ \quotep{P} \nameeq \quotep{Q} }
\end{mathpar}

The astute reader will have noticed that the mutual recursion of names
and processes imposes a mutual recursion on alpha-equivalence and
structural equivalence via name-equivalence. Fortunately, all of this
works out pleasantly and we may calculate in the natural way, free of
concern. The reader interested in the details is referred to the
appendix \ref{appendix:rho_details}.

\subsection{Substitution}

We use $\Proc$ for the set of processes, $\QProc$ for the set of
names, and $\id{\{}\vec{y} / \vec{x} \id{\}}$ to denote partial maps,
$s : \QProc \rightarrow \QProc$. A map, $s$ lifts, uniquely, to a map
on process terms, $\widehat{s} : \Proc \rightarrow \Proc$ by the
following equations.

\begin{mathpar}
  (0) \psubstp{Q}{P} := 0 \\
  (R \juxtap S) \psubstp{Q}{P}
  :=    
  (R)\psubstp{Q}{P} \juxtap (S) \psubstp{Q}{P} \\
  (x?(y).R) \psubstp{Q}{P}    
  :=    
  (x)\substp{Q}{P} (z)\concat( (R \psubstn{z}{y}) \psubstp{Q}{P} ) \\
  (\lift{x}{R}) \psubstp{Q}{P}  
  :=
  \lift{(x)\substp{Q}{P}}{ R \psubstp{Q}{P} } \\
%   (\dropn{x})  \psubstp{Q}{P}       
%   := 
%   \left\{ 
%     \begin{array}{ccc} 
%       \dropn{\quotep{Q}} & & x \nameeq \quotep{P} \\
%       \dropn{x} & & otherwise \\
%     \end{array}
%   \right. 
  (\dropn{x})  \psubstp{Q}{P}       
  := 
  \left\{ 
    \begin{array}{ccc} 
      Q & & x \nameeq \quotep{P} \\
      \dropn{x} & & otherwise \\
    \end{array}
  \right.
\end{mathpar}
 

where

\begin{eqnarray}
  (x)\id{\{} \lpquote Q \rpquote / \lpquote P \rpquote \id{\}}            = 
  \left\{ 
    \begin{array}{ccc}
      \lpquote Q \rpquote & & x \nameeq \lpquote P \rpquote \\
      x & & otherwise \\
    \end{array}
  \right. \nonumber
\end{eqnarray}

and $z$ is chosen distinct from $\quotep{P}$, $\quotep{Q}$, the free
names in $Q$, and all the names in $R$. Our $\alpha$-equivalence will
be built in the standard way from this substitution.

\begin{remark}\label{rem:no_self_referential_names}
  One consequence of these definitions is that $\forall P. \quotep{P}
  \not\in \freenames{P}$.
\end{remark}

\subsection{ Dynamic quote: an example }

Anticipating something of what's to come, consider applying the
substitution, $\widehat{\id{\{}u / z \id{\}}}$, to the following pair
of processes, $\lift{w}{y!(z)}$ and $w[ \lpquote y!(z) \rpquote ]$.

\begin{eqnarray}
	\lift{w}{y!(z)}\widehat{\id{\{}u / z \id{\}}}
		& = &
		\lift{w}{y!(u)} \nonumber\\
	w[ \lpquote y!(z) \rpquote ] \widehat{ \id{\{}u / z \id{\}} }
		& = &
		w[ \lpquote y!(z) \rpquote ] \nonumber
\end{eqnarray}

Because the body of the process between quotes is impervious to
substitution, we get radically different answers. In fact, by
examining the first process in an input context,
e.g. $x?(z).\lift{w}{y!(z)}$, we see that the process under the lift
operator may be shaped by prefixed inputs binding a name inside it. In
this sense, the lift operator will be seen as a way to dynamically
construct processes before reifying them as names.

Finally equipped with these standard features we can present the
dynamics of the calculus.

\subsubsection{Operational semantics} 

Finally, we introduce the computational dynamics. What marks these
algebras as distinct from other more traditionally studied algebraic
structures, e.g. vector spaces or polynomial rings, is the manner in
which dynamics is captured. In traditional structures, dynamics is typically
expressed through morphisms between such structures, as in linear maps
between vector spaces or morphisms between rings. In algebras
associated with the semantics of computation, the dynamics is
expressed as part of the algebraic structure itself, through a
reduction reduction relation typically denoted by $\red$. Below, we
give a recursive presentation of this relation for the calculus used
in the encoding.

$\red \subseteq \pi \times \pi$
$\red : \pi \to \mathcal{P}(\pi)$

\begin{mathpar}
  \inferrule* [lab=Comm] { \textsf{match}( x_{src}, x_{trgt} ) } { x_{trgt}?(y)P \; | \; x_{src}!\langle {Q} \rangle \red P\{\quotep{Q}/y}\} }
  \and \\
  \inferrule* [lab=Par] {{P} \red {P}'} {{{P} | {Q}} \red {{P}' | {Q}}}
  \and
  \inferrule* [lab=Equiv]{{{P} \scong {P}'} \andalso {{P}' \red {Q}'} \andalso {{Q}' \scong {Q}}}{{P} \red {Q}}
\end{mathpar}

\begin{eqnarray*}
  match_{\equiv} (\quotep{P},\quotep{Q}) & := & P \equiv Q \\
  match_{\dagger}(\quotep{P},\quotep{Q}) & := & \forall R. P|Q \red^{*} R => R \red^{*} 0 \\
  match_{K}(\quotep{P},\quotep{Q}) & := & K \mbox{ for some context } K
\end{eqnarray*}

$u?(x)P | u!\langle Q \rangle \red P\{\quotep{Q}/x\}$

%We write $\wred$ for $\red^*$, and $P\red$ if $\exists Q $ such that $ P \red Q$.
We write $P\red$ if $\exists Q $ such that $ P \red Q$ and $P\not\red$, otherwise.

\section{Replication}

As mentioned before, it is known that replication (and hence
recursion) can be implemented in a higher-order process algebra
\cite{SangiorgiWalker}. As our first example of calculation with the
machinery thus far presented we give the construction explicitly in
the {\rhoc}.

\begin{eqnarray}
	D_{x} & := & \prefix{x}{y}{(\binpar{\outputp{x}{y}}{@{y}})} \nonumber\\
	\bangp_{x}{P} & := & \binpar{{x}!\langle{\binpar{D_{x}}{P}}\rangle}{D_{x}} \nonumber
\end{eqnarray}

\begin{eqnarray}
	\bangp_{x}{P} & & \nonumber\\
	=
	& {x}!\langle{(\prefix{x}{y}{(\outputp{x}{y} | @{y})) | P}}\rangle 
	      | \prefix{x}{y}{(\outputp{x}{y} | @{y})} & \nonumber\\
	\red
	& (\outputp{x}{y} | @{y})\substn{\quotep{(\prefix{x}{y}{(@{y} | \outputp{x}{y})) | P}}}{y} & \nonumber\\
	=
	& \outputp{x}{\quotep{(\prefix{x}{y}{(\outputp{x}{y} | @{y})) | P}}}
	  | {(\prefix{x}{y}{(\outputp{x}{y} | @{y})) | P}} & \nonumber\\
	\red
	& \ldots & \nonumber\\
	\red^*
	& P | P | \ldots & \nonumber
\end{eqnarray}

Of course, this encoding, as an implementation, runs away, unfolding
$\bangp{P}$ eagerly. A lazier and more implementable replication
operator, restricted to input-guarded processes, may be obtained as follows.

\begin{eqnarray}
\bangp{\prefix{u}{v}{P}} 
	:= 
	\binpar{\lift{x}{\prefix{u}{v}{(\binpar{D(x)}{P})}}}{D(x)} \nonumber
\end{eqnarray}

\begin{remark}
  Note that the lazier definition still does not deal with summation
  or mixed summation (i.e. sums over input and output). The reader is
  invited to construct definitions of replication that deal with these
  features. 

  Further, the definitions are parameterized in a name, $x$. Can you,
  gentle reader, make a definition that eliminates this parameter and
  guarantees no accidental interaction between the replication
  machinery and the process being replicated -- i.e. no accidental
  sharing of names used by the process to get its work done and the
  name(s) used by the replication to effect copying. This latter
  revision of the definition of replication is crucial to obtaining
  the expected identity $!!P \sim !P$.
\end{remark}

\begin{remark}\label{rem:paradoxical_combinator}
  The reader familiar with the lambda calculus will have noticed the
  similarity between $D$ and the paradoxical combinator.

  [Ed. note: the existence of this seems to suggest we have to be more
  restrictive on the set of processes and names we admit if we are to
  support no-cloning.]
\end{remark}

\subsubsection{Bisimulation}

The computational dynamics gives rise to another kind of equivalence,
the equivalence of computational behavior. As previously mentioned
this is typically captured \emph{via} some form of bisimulation.

% The notion we use in this paper is weak barbed bisimulation
% \cite{milner91polyadicpi}.

The notion we use in this paper is derived from weak barbed
bisimulation \cite{milner91polyadicpi}. 

\begin{definition}
An \emph{observation relation}, $\downarrow_{\mathcal N}$, over a set
of names, $\mathcal N$, is the smallest relation satisfying the rules
below.

\infrule[Out-barb]{y \in {\mathcal N}, \; x \nameeq y}
		  {\outputp{x}{v} \downarrow_{\mathcal N} x}
\infrule[Par-barb]{\mbox{$P\downarrow_{\mathcal N} x$ or $Q\downarrow_{\mathcal N} x$}}
		  {\binpar{P}{Q} \downarrow_{\mathcal N} x}

We write $P \Downarrow_{\mathcal N} x$ if there is $Q$ such that 
$P \wred Q$ and $Q \downarrow_{\mathcal N} x$.
\end{definition}

\begin{definition}
%\label{def.bbisim}
An  ${\mathcal N}$-\emph{barbed bisimulation} over a set of names, ${\mathcal N}$, is a symmetric binary relation 
${\mathcal S}_{\mathcal N}$ between agents such that $P\rel{S}_{\mathcal N}Q$ implies:
\begin{enumerate}
\item If $P \red P'$ then $Q \wred Q'$ and $P'\rel{S}_{\mathcal N} Q'$.
\item If $P\downarrow_{\mathcal N} x$, then $Q\Downarrow_{\mathcal N} x$.
\end{enumerate}
$P$ is ${\mathcal N}$-barbed bisimilar to $Q$, written
$P \wbbisim_{\mathcal N} Q$, if $P \rel{S}_{\mathcal N} Q$ for some ${\mathcal N}$-barbed bisimulation ${\mathcal S}_{\mathcal N}$.
\end{definition}

$\mathcal{R} \subseteq \pi \times \pi$

$P \mathcal{R} Q => \forall P'. P \red P' \Rightarrow \exists Q'. Q \red Q', P' \mathcal{R} Q'$

$P \vdash x \Rightarrow Q \vdash x$

\begin{mathpar}
  \inferrule*[lab=Out-barb]{x \nameeq y}{{y}!\langle{Q}\rangle \vdash x}
  \and
  \inferrule*[lab=Par-barb]{\mbox{$P\vdash x$ or $Q\vdash x$}}{\binpar{P}{Q} \vdash x}
\end{mathpar}

\subsubsection{Contexts}

One of the principle advantages of computational calculi like the
$\pi$-calculus is a well-defined notion of context,
contextual-equivalence and a correlation between
contextual-equivalence and notions of bisimulation. The notion of
context allows the decomposition of a process into (sub-)process and
its syntactic environment, its context. Thus, a context may be
thought of as a process with a ``hole'' (written $\Box$) in it. The
application of a context $M$ to a process $P$, written $M[P]$, is
tantamount to filling the hole in $M$ with $P$. In this paper we do
not need the full weight of this theory, but do make use of the notion
of context in the proof the main theorem. 

\begin{mathpar}
  \inferrule* [lab=summation] {} {{M_{M},M_{N}} \bc \Box \;|\; x.M_{A} \;|\; M_{M}+M_{N}}
  \and
  \inferrule* [lab=agent] {} {{M_{A}} \bc (\vec{x})M_{P} \;| \; \clift{P_0,\ldots,M_{P},\ldots,P_N}}
  \and \\
  \inferrule* [lab=process] {} {{M_{P}} \bc M_{N} \;| \;P|M_{P} }
\end{mathpar} 

\begin{mathpar}
  \inferrule* [lab=sychronization] {} {M_{N} \bc \Box \;|\; x?M_{F} \;|\; x!M_{C}}
  \and
  \inferrule* [lab=abstraction] {} {{M_{F}} \bc (x)M_{P} }
  \and
  \inferrule* [lab=concretion] {} {{M_{C}} \bc \langle M_{P} \rangle }
  \and \\
  \inferrule* [lab=process] {} {{M_{P}} \bc M_{N} \;| \;P|M_{P} }
\end{mathpar}

\begin{definition}[contextual application] Given a context $M$, and
  process $P$, we define the \emph{contextual application}, $M[P] :=
  M\{P/\Box\}$. That is, the contextual application of M to P is the
  substitution of $P$ for $\Box$ in $M$.
\end{definition}

$\meaningof{-} : L \to \mathcal{P}(\pi)$

\begin{mathpar}
  \inferrule* [lab=collection] {} {\meaningof{true} = \pi, \and \meaningof{~E} = \pi \setminus \meaningof{E}, \and \meaningof{E_{1} \& E_{2}} = \meaningof{E_{1}} \cap \meaningof{E_{2}}}
\end{mathpar}

\begin{mathpar}
  \inferrule* [lab=structure] {} {\meaningof{0} = \{ P \in \pi | P \equiv 0 \}, \and \\ \meaningof{E_1 | E_2} = \{ P \in \pi | P \equiv P_{1} | P_{2}, P_{1} \in \meaningof{E_{1}}, P_{2} \in \meaningof{E_2}\} }
\end{mathpar}

\begin{mathpar}
 \inferrule* [lab=behavior] {} {\meaningof{\langle a?b \rangle E} = \{ P \in \pi | P \equiv Q | u?(y)P', \\ \and \\\\ \and \\ \;\;\; u \in \meaningof{a}, \forall z.P'\{z/y\} \in \meaningof{E\{z/b\}}\}, \and \\ \meaningof{a!E} = \{ P \in \pi | P \equiv Q | x!\langle P' \rangle, x \in \meaningof{a} P' \in \meaningof{E}\} }
\end{mathpar}

\begin{mathpar}
 \inferrule* [lab=nominal] {} {\meaningof{\quotep{E}} = \{ \quotep{P} \in \quotep{\pi} | P \in \meaningof{E} \}, \and \meaningof{\quotep{P}} = \{ \quotep{Q} \in \quotep{\pi} | P \equiv Q \} \and \\ \meaningof{@\quotep{E}} = \{ P \in \pi | P \equiv @x, x \in \meaningof{E} \}}
\end{mathpar}

\begin{eqnarray*}
  \\
  \meaningof{-} : TS \to ST
\end{eqnarray*}

\begin{eqnarray*}
  \\
  L : TS \to ST
\end{eqnarray*}

\begin{eqnarray*}
  \\
  P \models E \iff P \in \meaningof{E}
\end{eqnarray*}

\begin{eqnarray*}
  P \approx_{L} Q \iff \forall E \in L. P \models E \iff Q \models E
\end{eqnarray*}

\begin{eqnarray*}
  P \approx_{K} Q
\end{eqnarray*}

\begin{eqnarray*}
  P \approx Q
\end{eqnarray*}

$\approx_{K} = \approx = \approx_{L}$

\subsubsection{Contextual duality}

Note that contexts extend the quotation operation to a family of
operations from processes to names. Given a context, $M$, we can
define a \emph{nominal context}, $\quotep{M}$ by $\quotep{M}[P] :=
\quotep{M[P]}$. To foreshadow what is to come we observe that these
operations enjoy a duality with processes very much like the duality
between vectors and maps from vectors to scalars.

Further, because the calculus is essentially higher-order, we have a
correspondence between contexts and processes. More specifically,
given a name $x$ and a context $M$ we can construct $M^{*}_{x}$ such
that 

\begin{mathpar}
  M^{*}_{x} | \lift{x}{P} \red M[P]
\end{mathpar}

namely,

\begin{mathpar}
  M^{*}_{x} := x?(u).M[\dropn{u}]
\end{mathpar}

The dependence of $M^{*}_{x}$ on a name makes it an abstraction, 

\begin{mathpar}
  M^{*} := (x)x?(u).M[\dropn{u}]
\end{mathpar}

\subsection{Additional notation}

It will sometimes be convenient to denote the process a name
quotes. We already have the notation $x = \quotep{P}$, but it will be
convenient to introduce an alternate notation, $\procn{x}$, when we
want to emphasize the connection to the use of the name. Note that, by
virtue of name equivalence, $\quotep{\procn{x}} \nameeq x$; so, the
notation is consistent with previous definitions.

Further, because names have structure it is possible to effect
substitutions on the basis of that structure. This means we need to
upgrade our notation for substitutions, which we accomplish by
adapting comprehension notation. Thus,

\begin{mathpar}
  P\{ y / x : x \in S \}
\end{mathpar}

is interpreted to mean the process derived from P by replacing (in a
capture-avoiding manner) each occurrence of $x$ in $S$ by $y$. For example,

\begin{mathpar}
  P\{ \quotep{\procn{x}|\procn{x}} / x : x \in \freenames{P} \}
\end{mathpar}

will replace each (occurrence) of a free name $x$ in $P$ by
$\quotep{\procn{x}|\procn{x}}$.

Also, we will avail ourselves of the notation $x^{L}$ and $x^{R}$ to
denote injections of a name into disjoint copies of the name
space. There are numerous ways to accomplish this. One example can be
found in \cite{MeredithR05}. This notation overloads to vectors of
names: $\vec{x}^{\pi} := (x_{i}^{\pi} \; : \; 0 \leq i < |\vec{x}| )$ where $\pi \in \{L,R\}$.

We also use $P^{\Box} := P|\Box$.

In \cite{MeredithR05} an interpretation of the new operator is
given. It turns out that there are several possible interpretations
all enjoying the requisite algebraic properties of the operator (see
\cite{milner91polyadicpi}). We will therefore make liberal use of
$(\nu\; \vec{x})P$.

% subsection the_syntax_and_semantics_of_the_notation_system (end)   

\input{qm2pi.qmops} 

\input{qm2pi.sterngerlach} 

\input{qm2pi.metric} 

% section concurrent_process_calculi (end)

%\input{qm2pi.proofsketch}

% section proof sketch (end)

%\input{qm2pi.slviaknots} 

% section spatial logic via knots (end)

\input{qm2pi.conclusion}

% section conclusion (end)

%\input{qm2pi.dtcodes} 

% section wiring algorithm (end)

\input{qm2pi.ack} 

% section acknowledgments (end)

\newpage


\bibliographystyle{plain}   
\bibliography{../../biblios/main.bib}

\input{qm2pi.rhodetails}

\end{document}

 

%\documentclass[12pt]{llncs}
%\documentclass{jktr}

\usepackage[pdftex]{hyperref}                   
\usepackage {listings}
\usepackage {mathpartir}
\usepackage{bcprules}
%\usepackage{listings}
                       
\usepackage{graphicx} 
%\usepackage[margins=2.5cm,nohead,nofoot]{geometry}
%\usepackage{geometry}
\usepackage{amsfonts}
\usepackage{amstext}
\usepackage{latexsym}
\usepackage{amssymb}
\usepackage{color}


%\include{myPreamble}
\include{qm2pi.local} 

%\ifpdf
%\usepackage[pdftex]{graphicx}
%\else
%\usepackage{graphicx}
%\fi

 % \ifpdf
%  \usepackage{pdfsync}
%  \if


%\title{Brief Article}
%\author{David F. Snyder}
%\author{L.G. Meredith}

%\address{Dept. of Math., Texas State University--San Marcos, San Marcos, TX 78666}
       
\pagestyle{empty}


\begin{document}

\lstset{language=[Objective]Caml,frame=shadowbox}

\input{qm2pi.front}

% section front matter (end)

\input{qm2pi.intro} 
 
% section introduction (end)

% \input{qm2pi.knotations} 

% section notation (end)

\input{qm2pi.process.calculi} 

% section concurrent_process_calculi_and_spatial_logics_ (end)
    
%\input{qm2pi.knots2pi} 

%\input{qm2pi.trefoil} 

%\input{qm2pi.mainthm} 

% subsection basic_interpretation (end)

%\input{qm2pi.rho.presentation} 
\subsection{The syntax and semantics of the notation system}\label{sub:the_syntax_and_semantics_of_the_notation_system} % (fold)

We now summarize a technical presentation of the calculus that
embodies our theory of dynamics. The typical presentation of such a
calculus follows the style of giving generators and relations on
them. The grammar, below, describing term constructors, freely
generates the set of processes, $\Proc$. This set is then quotiented
by a relation known as structural congruence and it is over this set
that the notion of dynamics is expressed. This presentation is
essentially that of \cite{MeredithR05} with the addition of
polyadicity and summation. For readability we have relegated some of
the technical subtleties to an appendix.

\subsubsection{Process grammar}\label{subsub:process_grammar}

\begin{mathpar}
  \inferrule* [lab=synchronization] {} {{M} \bc \pzero \;|\; x?F \;|\; x!C }
  \and
  \inferrule* [lab=abstraction] {} {{F} \bc (x)P}
  \and
  \inferrule* [lab=concretion] {} {{C} \bc \langle Q \rangle}
  \and
  \inferrule* [lab=process] {} {{P,Q} \bc M \;| \;P|Q \;|\; @{x}}
  \and
  \inferrule* [lab=name] {} {{x} \bc \quotep{P}}
\end{mathpar} 

Note that $\vec{x}$ (resp. $\vec{P}$) denotes a vector of names
(resp. processes) of length $|\vec{x}|$ (resp. $|\vec{P}|$). We adopt
the following useful abbreviations.

\begin{mathpar}
   x?(\vec{y}).P := x.(\vec{y})P \and  x\clift{\vec{P}} := x.\clift{\vec{P}}
   \and x!(y) := \lift{x}{\dropn{y}}
   \and \Pi_{i=0}^{n-1}P_i := P_0 | \ldots | P_{n-1}
\end{mathpar}

\subsubsection{Structural congruence}

\paragraph{Free and bound names and alpha-equivalence.} At the
core of structural equivalence is alpha-equivalence which identifies
process that are the same up to a change of variable. Formally, we
recognize the distinction between free and bound names. The free names
of a process, $\freenames{P}$, may be calculated recursively as
follows:

\begin{mathpar}
\freenames{\pzero} := \emptyset
  \and \\
  \freenames{x?(y).P} := \{ x \} \cup (\freenames{P} \setminus \{ y \})
  \and 
  \freenames{x!\langle P \rangle} := \{ x \} \cup \{ P \} 
  \and \\
  \freenames{P|Q} := \freenames{P} \cup \freenames{Q}
  \and \\
  \freenames{@{x}} := \{ x \}
\end{mathpar}

$\pi$
$\quotep{\pi}$

$\freenames{-} : \pi \to \mathcal{P}(\quotep{\pi})$

\begin{eqnarray*}
  \freenames{\pzero} & := & \emptyset \\
  \freenames{x?(y).P} & := & \{ x \} \cup (\freenames{P} \setminus \{ y \}) \\
  \freenames{x!\langle P \rangle} & := & \{ x \} \cup \{ P \} \\
  \freenames{P|Q} & := & \freenames{P} \cup \freenames{Q} \\
  \freenames{\dropn{x}} & := & \{ x \}
\end{eqnarray*}

The bound names of a process, $\boundnames{P}$, are those names occurring in $P$
that are not free. For example, in $x?(y).0$, the name $x$ is free, while $y$ is bound.

\begin{mathpar}
  \inferrule* [lab=monoidal-laws] {} { P|Q \equiv Q|P \and P|0 \equiv P \and P|(Q|R) \equiv (P|Q)|R }
\end{mathpar}

\begin{mathpar}
  \inferrule* [lab=alpha-equivalence] {} { (x)P \equiv (y)P\{y/x\} \and y \not\in \freenames{P} }
\end{mathpar}

\begin{definition}
Then two processes, $P,Q$, are alpha-equivalent if $P = Q\{\vec{y}/\vec{x}\}$ for
some $\vec{x} \in \boundnames{Q},\vec{y} \in \boundnames{P}$, where $Q\{\vec{y}/\vec{x}\}$
denotes the capture-avoiding substitution of $\vec{y}$ for $\vec{x}$ in $Q$.
\end{definition}

\begin{definition}
  The {\em structural congruence} \cite{SangiorgiWalker} , $\equiv$,
  between processes is the least congruence containing
  alpha-equivalence, satisfying the abelian monoid laws
  (associativity, commutativity and $\pzero$ as identity) for parallel
  composition $|$ and for summation $+$.
\end{definition}

\subsection{Name equivalence}

We take name equivalence, written $\nameeq$, to be the smallest
equivalence relation generated by the following rules.

\begin{mathpar}
\inferrule*[lab=Quote-drop]
{ }
{ \quotep{@{x}} \nameeq x }

\inferrule*[lab=Struct-equiv]
{ P \scong Q }
{ \quotep{P} \nameeq \quotep{Q} }
\end{mathpar}

The astute reader will have noticed that the mutual recursion of names
and processes imposes a mutual recursion on alpha-equivalence and
structural equivalence via name-equivalence. Fortunately, all of this
works out pleasantly and we may calculate in the natural way, free of
concern. The reader interested in the details is referred to the
appendix \ref{appendix:rho_details}.

\subsection{Substitution}

We use $\Proc$ for the set of processes, $\QProc$ for the set of
names, and $\id{\{}\vec{y} / \vec{x} \id{\}}$ to denote partial maps,
$s : \QProc \rightarrow \QProc$. A map, $s$ lifts, uniquely, to a map
on process terms, $\widehat{s} : \Proc \rightarrow \Proc$ by the
following equations.

\begin{mathpar}
  (0) \psubstp{Q}{P} := 0 \\
  (R \juxtap S) \psubstp{Q}{P}
  :=    
  (R)\psubstp{Q}{P} \juxtap (S) \psubstp{Q}{P} \\
  (x?(y).R) \psubstp{Q}{P}    
  :=    
  (x)\substp{Q}{P} (z)\concat( (R \psubstn{z}{y}) \psubstp{Q}{P} ) \\
  (\lift{x}{R}) \psubstp{Q}{P}  
  :=
  \lift{(x)\substp{Q}{P}}{ R \psubstp{Q}{P} } \\
%   (\dropn{x})  \psubstp{Q}{P}       
%   := 
%   \left\{ 
%     \begin{array}{ccc} 
%       \dropn{\quotep{Q}} & & x \nameeq \quotep{P} \\
%       \dropn{x} & & otherwise \\
%     \end{array}
%   \right. 
  (\dropn{x})  \psubstp{Q}{P}       
  := 
  \left\{ 
    \begin{array}{ccc} 
      Q & & x \nameeq \quotep{P} \\
      \dropn{x} & & otherwise \\
    \end{array}
  \right.
\end{mathpar}
 

where

\begin{eqnarray}
  (x)\id{\{} \lpquote Q \rpquote / \lpquote P \rpquote \id{\}}            = 
  \left\{ 
    \begin{array}{ccc}
      \lpquote Q \rpquote & & x \nameeq \lpquote P \rpquote \\
      x & & otherwise \\
    \end{array}
  \right. \nonumber
\end{eqnarray}

and $z$ is chosen distinct from $\quotep{P}$, $\quotep{Q}$, the free
names in $Q$, and all the names in $R$. Our $\alpha$-equivalence will
be built in the standard way from this substitution.

\begin{remark}\label{rem:no_self_referential_names}
  One consequence of these definitions is that $\forall P. \quotep{P}
  \not\in \freenames{P}$.
\end{remark}

\subsection{ Dynamic quote: an example }

Anticipating something of what's to come, consider applying the
substitution, $\widehat{\id{\{}u / z \id{\}}}$, to the following pair
of processes, $\lift{w}{y!(z)}$ and $w[ \lpquote y!(z) \rpquote ]$.

\begin{eqnarray}
	\lift{w}{y!(z)}\widehat{\id{\{}u / z \id{\}}}
		& = &
		\lift{w}{y!(u)} \nonumber\\
	w[ \lpquote y!(z) \rpquote ] \widehat{ \id{\{}u / z \id{\}} }
		& = &
		w[ \lpquote y!(z) \rpquote ] \nonumber
\end{eqnarray}

Because the body of the process between quotes is impervious to
substitution, we get radically different answers. In fact, by
examining the first process in an input context,
e.g. $x?(z).\lift{w}{y!(z)}$, we see that the process under the lift
operator may be shaped by prefixed inputs binding a name inside it. In
this sense, the lift operator will be seen as a way to dynamically
construct processes before reifying them as names.

Finally equipped with these standard features we can present the
dynamics of the calculus.

\subsubsection{Operational semantics} 

Finally, we introduce the computational dynamics. What marks these
algebras as distinct from other more traditionally studied algebraic
structures, e.g. vector spaces or polynomial rings, is the manner in
which dynamics is captured. In traditional structures, dynamics is typically
expressed through morphisms between such structures, as in linear maps
between vector spaces or morphisms between rings. In algebras
associated with the semantics of computation, the dynamics is
expressed as part of the algebraic structure itself, through a
reduction reduction relation typically denoted by $\red$. Below, we
give a recursive presentation of this relation for the calculus used
in the encoding.

$\red \subseteq \pi \times \pi$
$\red : \pi \to \mathcal{P}(\pi)$

\begin{mathpar}
  \inferrule* [lab=Comm] { \textsf{match}( x_{src}, x_{trgt} ) } { x_{trgt}?(y)P \; | \; x_{src}!\langle {Q} \rangle \red P\{\quotep{Q}/y}\} }
  \and \\
  \inferrule* [lab=Par] {{P} \red {P}'} {{{P} | {Q}} \red {{P}' | {Q}}}
  \and
  \inferrule* [lab=Equiv]{{{P} \scong {P}'} \andalso {{P}' \red {Q}'} \andalso {{Q}' \scong {Q}}}{{P} \red {Q}}
\end{mathpar}

\begin{eqnarray*}
  match_{\equiv} (\quotep{P},\quotep{Q}) & := & P \equiv Q \\
  match_{\dagger}(\quotep{P},\quotep{Q}) & := & \forall R. P|Q \red^{*} R => R \red^{*} 0 \\
  match_{K}(\quotep{P},\quotep{Q}) & := & K \mbox{ for some context } K
\end{eqnarray*}

$u?(x)P | u!\langle Q \rangle \red P\{\quotep{Q}/x\}$

%We write $\wred$ for $\red^*$, and $P\red$ if $\exists Q $ such that $ P \red Q$.
We write $P\red$ if $\exists Q $ such that $ P \red Q$ and $P\not\red$, otherwise.

\section{Replication}

As mentioned before, it is known that replication (and hence
recursion) can be implemented in a higher-order process algebra
\cite{SangiorgiWalker}. As our first example of calculation with the
machinery thus far presented we give the construction explicitly in
the {\rhoc}.

\begin{eqnarray}
	D_{x} & := & \prefix{x}{y}{(\binpar{\outputp{x}{y}}{@{y}})} \nonumber\\
	\bangp_{x}{P} & := & \binpar{{x}!\langle{\binpar{D_{x}}{P}}\rangle}{D_{x}} \nonumber
\end{eqnarray}

\begin{eqnarray}
	\bangp_{x}{P} & & \nonumber\\
	=
	& {x}!\langle{(\prefix{x}{y}{(\outputp{x}{y} | @{y})) | P}}\rangle 
	      | \prefix{x}{y}{(\outputp{x}{y} | @{y})} & \nonumber\\
	\red
	& (\outputp{x}{y} | @{y})\substn{\quotep{(\prefix{x}{y}{(@{y} | \outputp{x}{y})) | P}}}{y} & \nonumber\\
	=
	& \outputp{x}{\quotep{(\prefix{x}{y}{(\outputp{x}{y} | @{y})) | P}}}
	  | {(\prefix{x}{y}{(\outputp{x}{y} | @{y})) | P}} & \nonumber\\
	\red
	& \ldots & \nonumber\\
	\red^*
	& P | P | \ldots & \nonumber
\end{eqnarray}

Of course, this encoding, as an implementation, runs away, unfolding
$\bangp{P}$ eagerly. A lazier and more implementable replication
operator, restricted to input-guarded processes, may be obtained as follows.

\begin{eqnarray}
\bangp{\prefix{u}{v}{P}} 
	:= 
	\binpar{\lift{x}{\prefix{u}{v}{(\binpar{D(x)}{P})}}}{D(x)} \nonumber
\end{eqnarray}

\begin{remark}
  Note that the lazier definition still does not deal with summation
  or mixed summation (i.e. sums over input and output). The reader is
  invited to construct definitions of replication that deal with these
  features. 

  Further, the definitions are parameterized in a name, $x$. Can you,
  gentle reader, make a definition that eliminates this parameter and
  guarantees no accidental interaction between the replication
  machinery and the process being replicated -- i.e. no accidental
  sharing of names used by the process to get its work done and the
  name(s) used by the replication to effect copying. This latter
  revision of the definition of replication is crucial to obtaining
  the expected identity $!!P \sim !P$.
\end{remark}

\begin{remark}\label{rem:paradoxical_combinator}
  The reader familiar with the lambda calculus will have noticed the
  similarity between $D$ and the paradoxical combinator.

  [Ed. note: the existence of this seems to suggest we have to be more
  restrictive on the set of processes and names we admit if we are to
  support no-cloning.]
\end{remark}

\subsubsection{Bisimulation}

The computational dynamics gives rise to another kind of equivalence,
the equivalence of computational behavior. As previously mentioned
this is typically captured \emph{via} some form of bisimulation.

% The notion we use in this paper is weak barbed bisimulation
% \cite{milner91polyadicpi}.

The notion we use in this paper is derived from weak barbed
bisimulation \cite{milner91polyadicpi}. 

\begin{definition}
An \emph{observation relation}, $\downarrow_{\mathcal N}$, over a set
of names, $\mathcal N$, is the smallest relation satisfying the rules
below.

\infrule[Out-barb]{y \in {\mathcal N}, \; x \nameeq y}
		  {\outputp{x}{v} \downarrow_{\mathcal N} x}
\infrule[Par-barb]{\mbox{$P\downarrow_{\mathcal N} x$ or $Q\downarrow_{\mathcal N} x$}}
		  {\binpar{P}{Q} \downarrow_{\mathcal N} x}

We write $P \Downarrow_{\mathcal N} x$ if there is $Q$ such that 
$P \wred Q$ and $Q \downarrow_{\mathcal N} x$.
\end{definition}

\begin{definition}
%\label{def.bbisim}
An  ${\mathcal N}$-\emph{barbed bisimulation} over a set of names, ${\mathcal N}$, is a symmetric binary relation 
${\mathcal S}_{\mathcal N}$ between agents such that $P\rel{S}_{\mathcal N}Q$ implies:
\begin{enumerate}
\item If $P \red P'$ then $Q \wred Q'$ and $P'\rel{S}_{\mathcal N} Q'$.
\item If $P\downarrow_{\mathcal N} x$, then $Q\Downarrow_{\mathcal N} x$.
\end{enumerate}
$P$ is ${\mathcal N}$-barbed bisimilar to $Q$, written
$P \wbbisim_{\mathcal N} Q$, if $P \rel{S}_{\mathcal N} Q$ for some ${\mathcal N}$-barbed bisimulation ${\mathcal S}_{\mathcal N}$.
\end{definition}

$\mathcal{R} \subseteq \pi \times \pi$

$P \mathcal{R} Q => \forall P'. P \red P' \Rightarrow \exists Q'. Q \red Q', P' \mathcal{R} Q'$

$P \vdash x \Rightarrow Q \vdash x$

\begin{mathpar}
  \inferrule*[lab=Out-barb]{x \nameeq y}{{y}!\langle{Q}\rangle \vdash x}
  \and
  \inferrule*[lab=Par-barb]{\mbox{$P\vdash x$ or $Q\vdash x$}}{\binpar{P}{Q} \vdash x}
\end{mathpar}

\subsubsection{Contexts}

One of the principle advantages of computational calculi like the
$\pi$-calculus is a well-defined notion of context,
contextual-equivalence and a correlation between
contextual-equivalence and notions of bisimulation. The notion of
context allows the decomposition of a process into (sub-)process and
its syntactic environment, its context. Thus, a context may be
thought of as a process with a ``hole'' (written $\Box$) in it. The
application of a context $M$ to a process $P$, written $M[P]$, is
tantamount to filling the hole in $M$ with $P$. In this paper we do
not need the full weight of this theory, but do make use of the notion
of context in the proof the main theorem. 

\begin{mathpar}
  \inferrule* [lab=summation] {} {{M_{M},M_{N}} \bc \Box \;|\; x.M_{A} \;|\; M_{M}+M_{N}}
  \and
  \inferrule* [lab=agent] {} {{M_{A}} \bc (\vec{x})M_{P} \;| \; \clift{P_0,\ldots,M_{P},\ldots,P_N}}
  \and \\
  \inferrule* [lab=process] {} {{M_{P}} \bc M_{N} \;| \;P|M_{P} }
\end{mathpar} 

\begin{mathpar}
  \inferrule* [lab=sychronization] {} {M_{N} \bc \Box \;|\; x?M_{F} \;|\; x!M_{C}}
  \and
  \inferrule* [lab=abstraction] {} {{M_{F}} \bc (x)M_{P} }
  \and
  \inferrule* [lab=concretion] {} {{M_{C}} \bc \langle M_{P} \rangle }
  \and \\
  \inferrule* [lab=process] {} {{M_{P}} \bc M_{N} \;| \;P|M_{P} }
\end{mathpar}

\begin{definition}[contextual application] Given a context $M$, and
  process $P$, we define the \emph{contextual application}, $M[P] :=
  M\{P/\Box\}$. That is, the contextual application of M to P is the
  substitution of $P$ for $\Box$ in $M$.
\end{definition}

$\meaningof{-} : L \to \mathcal{P}(\pi)$

\begin{mathpar}
  \inferrule* [lab=collection] {} {\meaningof{true} = \pi, \and \meaningof{~E} = \pi \setminus \meaningof{E}, \and \meaningof{E_{1} \& E_{2}} = \meaningof{E_{1}} \cap \meaningof{E_{2}}}
\end{mathpar}

\begin{mathpar}
  \inferrule* [lab=structure] {} {\meaningof{0} = \{ P \in \pi | P \equiv 0 \}, \and \\ \meaningof{E_1 | E_2} = \{ P \in \pi | P \equiv P_{1} | P_{2}, P_{1} \in \meaningof{E_{1}}, P_{2} \in \meaningof{E_2}\} }
\end{mathpar}

\begin{mathpar}
 \inferrule* [lab=behavior] {} {\meaningof{\langle a?b \rangle E} = \{ P \in \pi | P \equiv Q | u?(y)P', \\ \and \\\\ \and \\ \;\;\; u \in \meaningof{a}, \forall z.P'\{z/y\} \in \meaningof{E\{z/b\}}\}, \and \\ \meaningof{a!E} = \{ P \in \pi | P \equiv Q | x!\langle P' \rangle, x \in \meaningof{a} P' \in \meaningof{E}\} }
\end{mathpar}

\begin{mathpar}
 \inferrule* [lab=nominal] {} {\meaningof{\quotep{E}} = \{ \quotep{P} \in \quotep{\pi} | P \in \meaningof{E} \}, \and \meaningof{\quotep{P}} = \{ \quotep{Q} \in \quotep{\pi} | P \equiv Q \} \and \\ \meaningof{@\quotep{E}} = \{ P \in \pi | P \equiv @x, x \in \meaningof{E} \}}
\end{mathpar}

\begin{eqnarray*}
  \\
  \meaningof{-} : TS \to ST
\end{eqnarray*}

\begin{eqnarray*}
  \\
  L : TS \to ST
\end{eqnarray*}

\begin{eqnarray*}
  \\
  P \models E \iff P \in \meaningof{E}
\end{eqnarray*}

\begin{eqnarray*}
  P \approx_{L} Q \iff \forall E \in L. P \models E \iff Q \models E
\end{eqnarray*}

\begin{eqnarray*}
  P \approx_{K} Q
\end{eqnarray*}

\begin{eqnarray*}
  P \approx Q
\end{eqnarray*}

$\approx_{K} = \approx = \approx_{L}$

\subsubsection{Contextual duality}

Note that contexts extend the quotation operation to a family of
operations from processes to names. Given a context, $M$, we can
define a \emph{nominal context}, $\quotep{M}$ by $\quotep{M}[P] :=
\quotep{M[P]}$. To foreshadow what is to come we observe that these
operations enjoy a duality with processes very much like the duality
between vectors and maps from vectors to scalars.

Further, because the calculus is essentially higher-order, we have a
correspondence between contexts and processes. More specifically,
given a name $x$ and a context $M$ we can construct $M^{*}_{x}$ such
that 

\begin{mathpar}
  M^{*}_{x} | \lift{x}{P} \red M[P]
\end{mathpar}

namely,

\begin{mathpar}
  M^{*}_{x} := x?(u).M[\dropn{u}]
\end{mathpar}

The dependence of $M^{*}_{x}$ on a name makes it an abstraction, 

\begin{mathpar}
  M^{*} := (x)x?(u).M[\dropn{u}]
\end{mathpar}

\subsection{Additional notation}

It will sometimes be convenient to denote the process a name
quotes. We already have the notation $x = \quotep{P}$, but it will be
convenient to introduce an alternate notation, $\procn{x}$, when we
want to emphasize the connection to the use of the name. Note that, by
virtue of name equivalence, $\quotep{\procn{x}} \nameeq x$; so, the
notation is consistent with previous definitions.

Further, because names have structure it is possible to effect
substitutions on the basis of that structure. This means we need to
upgrade our notation for substitutions, which we accomplish by
adapting comprehension notation. Thus,

\begin{mathpar}
  P\{ y / x : x \in S \}
\end{mathpar}

is interpreted to mean the process derived from P by replacing (in a
capture-avoiding manner) each occurrence of $x$ in $S$ by $y$. For example,

\begin{mathpar}
  P\{ \quotep{\procn{x}|\procn{x}} / x : x \in \freenames{P} \}
\end{mathpar}

will replace each (occurrence) of a free name $x$ in $P$ by
$\quotep{\procn{x}|\procn{x}}$.

Also, we will avail ourselves of the notation $x^{L}$ and $x^{R}$ to
denote injections of a name into disjoint copies of the name
space. There are numerous ways to accomplish this. One example can be
found in \cite{MeredithR05}. This notation overloads to vectors of
names: $\vec{x}^{\pi} := (x_{i}^{\pi} \; : \; 0 \leq i < |\vec{x}| )$ where $\pi \in \{L,R\}$.

We also use $P^{\Box} := P|\Box$.

In \cite{MeredithR05} an interpretation of the new operator is
given. It turns out that there are several possible interpretations
all enjoying the requisite algebraic properties of the operator (see
\cite{milner91polyadicpi}). We will therefore make liberal use of
$(\nu\; \vec{x})P$.

% subsection the_syntax_and_semantics_of_the_notation_system (end)   

\input{qm2pi.qmops} 

\input{qm2pi.sterngerlach} 

\input{qm2pi.metric} 

% section concurrent_process_calculi (end)

%\input{qm2pi.proofsketch}

% section proof sketch (end)

%\input{qm2pi.slviaknots} 

% section spatial logic via knots (end)

\input{qm2pi.conclusion}

% section conclusion (end)

%\input{qm2pi.dtcodes} 

% section wiring algorithm (end)

\input{qm2pi.ack} 

% section acknowledgments (end)

\newpage


\bibliographystyle{plain}   
\bibliography{../../biblios/main.bib}

\input{qm2pi.rhodetails}

\end{document}

 

%\documentclass[12pt]{llncs}
%\documentclass{jktr}

\usepackage[pdftex]{hyperref}                   
\usepackage {listings}
\usepackage {mathpartir}
\usepackage{bcprules}
%\usepackage{listings}
                       
\usepackage{graphicx} 
%\usepackage[margins=2.5cm,nohead,nofoot]{geometry}
%\usepackage{geometry}
\usepackage{amsfonts}
\usepackage{amstext}
\usepackage{latexsym}
\usepackage{amssymb}
\usepackage{color}


%\include{myPreamble}
\include{qm2pi.local} 

%\ifpdf
%\usepackage[pdftex]{graphicx}
%\else
%\usepackage{graphicx}
%\fi

 % \ifpdf
%  \usepackage{pdfsync}
%  \if


%\title{Brief Article}
%\author{David F. Snyder}
%\author{L.G. Meredith}

%\address{Dept. of Math., Texas State University--San Marcos, San Marcos, TX 78666}
       
\pagestyle{empty}


\begin{document}

\lstset{language=[Objective]Caml,frame=shadowbox}

\input{qm2pi.front}

% section front matter (end)

\input{qm2pi.intro} 
 
% section introduction (end)

% \input{qm2pi.knotations} 

% section notation (end)

\input{qm2pi.process.calculi} 

% section concurrent_process_calculi_and_spatial_logics_ (end)
    
%\input{qm2pi.knots2pi} 

%\input{qm2pi.trefoil} 

%\input{qm2pi.mainthm} 

% subsection basic_interpretation (end)

%\input{qm2pi.rho.presentation} 
\subsection{The syntax and semantics of the notation system}\label{sub:the_syntax_and_semantics_of_the_notation_system} % (fold)

We now summarize a technical presentation of the calculus that
embodies our theory of dynamics. The typical presentation of such a
calculus follows the style of giving generators and relations on
them. The grammar, below, describing term constructors, freely
generates the set of processes, $\Proc$. This set is then quotiented
by a relation known as structural congruence and it is over this set
that the notion of dynamics is expressed. This presentation is
essentially that of \cite{MeredithR05} with the addition of
polyadicity and summation. For readability we have relegated some of
the technical subtleties to an appendix.

\subsubsection{Process grammar}\label{subsub:process_grammar}

\begin{mathpar}
  \inferrule* [lab=synchronization] {} {{M} \bc \pzero \;|\; x?F \;|\; x!C }
  \and
  \inferrule* [lab=abstraction] {} {{F} \bc (x)P}
  \and
  \inferrule* [lab=concretion] {} {{C} \bc \langle Q \rangle}
  \and
  \inferrule* [lab=process] {} {{P,Q} \bc M \;| \;P|Q \;|\; @{x}}
  \and
  \inferrule* [lab=name] {} {{x} \bc \quotep{P}}
\end{mathpar} 

Note that $\vec{x}$ (resp. $\vec{P}$) denotes a vector of names
(resp. processes) of length $|\vec{x}|$ (resp. $|\vec{P}|$). We adopt
the following useful abbreviations.

\begin{mathpar}
   x?(\vec{y}).P := x.(\vec{y})P \and  x\clift{\vec{P}} := x.\clift{\vec{P}}
   \and x!(y) := \lift{x}{\dropn{y}}
   \and \Pi_{i=0}^{n-1}P_i := P_0 | \ldots | P_{n-1}
\end{mathpar}

\subsubsection{Structural congruence}

\paragraph{Free and bound names and alpha-equivalence.} At the
core of structural equivalence is alpha-equivalence which identifies
process that are the same up to a change of variable. Formally, we
recognize the distinction between free and bound names. The free names
of a process, $\freenames{P}$, may be calculated recursively as
follows:

\begin{mathpar}
\freenames{\pzero} := \emptyset
  \and \\
  \freenames{x?(y).P} := \{ x \} \cup (\freenames{P} \setminus \{ y \})
  \and 
  \freenames{x!\langle P \rangle} := \{ x \} \cup \{ P \} 
  \and \\
  \freenames{P|Q} := \freenames{P} \cup \freenames{Q}
  \and \\
  \freenames{@{x}} := \{ x \}
\end{mathpar}

$\pi$
$\quotep{\pi}$

$\freenames{-} : \pi \to \mathcal{P}(\quotep{\pi})$

\begin{eqnarray*}
  \freenames{\pzero} & := & \emptyset \\
  \freenames{x?(y).P} & := & \{ x \} \cup (\freenames{P} \setminus \{ y \}) \\
  \freenames{x!\langle P \rangle} & := & \{ x \} \cup \{ P \} \\
  \freenames{P|Q} & := & \freenames{P} \cup \freenames{Q} \\
  \freenames{\dropn{x}} & := & \{ x \}
\end{eqnarray*}

The bound names of a process, $\boundnames{P}$, are those names occurring in $P$
that are not free. For example, in $x?(y).0$, the name $x$ is free, while $y$ is bound.

\begin{mathpar}
  \inferrule* [lab=monoidal-laws] {} { P|Q \equiv Q|P \and P|0 \equiv P \and P|(Q|R) \equiv (P|Q)|R }
\end{mathpar}

\begin{mathpar}
  \inferrule* [lab=alpha-equivalence] {} { (x)P \equiv (y)P\{y/x\} \and y \not\in \freenames{P} }
\end{mathpar}

\begin{definition}
Then two processes, $P,Q$, are alpha-equivalent if $P = Q\{\vec{y}/\vec{x}\}$ for
some $\vec{x} \in \boundnames{Q},\vec{y} \in \boundnames{P}$, where $Q\{\vec{y}/\vec{x}\}$
denotes the capture-avoiding substitution of $\vec{y}$ for $\vec{x}$ in $Q$.
\end{definition}

\begin{definition}
  The {\em structural congruence} \cite{SangiorgiWalker} , $\equiv$,
  between processes is the least congruence containing
  alpha-equivalence, satisfying the abelian monoid laws
  (associativity, commutativity and $\pzero$ as identity) for parallel
  composition $|$ and for summation $+$.
\end{definition}

\subsection{Name equivalence}

We take name equivalence, written $\nameeq$, to be the smallest
equivalence relation generated by the following rules.

\begin{mathpar}
\inferrule*[lab=Quote-drop]
{ }
{ \quotep{@{x}} \nameeq x }

\inferrule*[lab=Struct-equiv]
{ P \scong Q }
{ \quotep{P} \nameeq \quotep{Q} }
\end{mathpar}

The astute reader will have noticed that the mutual recursion of names
and processes imposes a mutual recursion on alpha-equivalence and
structural equivalence via name-equivalence. Fortunately, all of this
works out pleasantly and we may calculate in the natural way, free of
concern. The reader interested in the details is referred to the
appendix \ref{appendix:rho_details}.

\subsection{Substitution}

We use $\Proc$ for the set of processes, $\QProc$ for the set of
names, and $\id{\{}\vec{y} / \vec{x} \id{\}}$ to denote partial maps,
$s : \QProc \rightarrow \QProc$. A map, $s$ lifts, uniquely, to a map
on process terms, $\widehat{s} : \Proc \rightarrow \Proc$ by the
following equations.

\begin{mathpar}
  (0) \psubstp{Q}{P} := 0 \\
  (R \juxtap S) \psubstp{Q}{P}
  :=    
  (R)\psubstp{Q}{P} \juxtap (S) \psubstp{Q}{P} \\
  (x?(y).R) \psubstp{Q}{P}    
  :=    
  (x)\substp{Q}{P} (z)\concat( (R \psubstn{z}{y}) \psubstp{Q}{P} ) \\
  (\lift{x}{R}) \psubstp{Q}{P}  
  :=
  \lift{(x)\substp{Q}{P}}{ R \psubstp{Q}{P} } \\
%   (\dropn{x})  \psubstp{Q}{P}       
%   := 
%   \left\{ 
%     \begin{array}{ccc} 
%       \dropn{\quotep{Q}} & & x \nameeq \quotep{P} \\
%       \dropn{x} & & otherwise \\
%     \end{array}
%   \right. 
  (\dropn{x})  \psubstp{Q}{P}       
  := 
  \left\{ 
    \begin{array}{ccc} 
      Q & & x \nameeq \quotep{P} \\
      \dropn{x} & & otherwise \\
    \end{array}
  \right.
\end{mathpar}
 

where

\begin{eqnarray}
  (x)\id{\{} \lpquote Q \rpquote / \lpquote P \rpquote \id{\}}            = 
  \left\{ 
    \begin{array}{ccc}
      \lpquote Q \rpquote & & x \nameeq \lpquote P \rpquote \\
      x & & otherwise \\
    \end{array}
  \right. \nonumber
\end{eqnarray}

and $z$ is chosen distinct from $\quotep{P}$, $\quotep{Q}$, the free
names in $Q$, and all the names in $R$. Our $\alpha$-equivalence will
be built in the standard way from this substitution.

\begin{remark}\label{rem:no_self_referential_names}
  One consequence of these definitions is that $\forall P. \quotep{P}
  \not\in \freenames{P}$.
\end{remark}

\subsection{ Dynamic quote: an example }

Anticipating something of what's to come, consider applying the
substitution, $\widehat{\id{\{}u / z \id{\}}}$, to the following pair
of processes, $\lift{w}{y!(z)}$ and $w[ \lpquote y!(z) \rpquote ]$.

\begin{eqnarray}
	\lift{w}{y!(z)}\widehat{\id{\{}u / z \id{\}}}
		& = &
		\lift{w}{y!(u)} \nonumber\\
	w[ \lpquote y!(z) \rpquote ] \widehat{ \id{\{}u / z \id{\}} }
		& = &
		w[ \lpquote y!(z) \rpquote ] \nonumber
\end{eqnarray}

Because the body of the process between quotes is impervious to
substitution, we get radically different answers. In fact, by
examining the first process in an input context,
e.g. $x?(z).\lift{w}{y!(z)}$, we see that the process under the lift
operator may be shaped by prefixed inputs binding a name inside it. In
this sense, the lift operator will be seen as a way to dynamically
construct processes before reifying them as names.

Finally equipped with these standard features we can present the
dynamics of the calculus.

\subsubsection{Operational semantics} 

Finally, we introduce the computational dynamics. What marks these
algebras as distinct from other more traditionally studied algebraic
structures, e.g. vector spaces or polynomial rings, is the manner in
which dynamics is captured. In traditional structures, dynamics is typically
expressed through morphisms between such structures, as in linear maps
between vector spaces or morphisms between rings. In algebras
associated with the semantics of computation, the dynamics is
expressed as part of the algebraic structure itself, through a
reduction reduction relation typically denoted by $\red$. Below, we
give a recursive presentation of this relation for the calculus used
in the encoding.

$\red \subseteq \pi \times \pi$
$\red : \pi \to \mathcal{P}(\pi)$

\begin{mathpar}
  \inferrule* [lab=Comm] { \textsf{match}( x_{src}, x_{trgt} ) } { x_{trgt}?(y)P \; | \; x_{src}!\langle {Q} \rangle \red P\{\quotep{Q}/y}\} }
  \and \\
  \inferrule* [lab=Par] {{P} \red {P}'} {{{P} | {Q}} \red {{P}' | {Q}}}
  \and
  \inferrule* [lab=Equiv]{{{P} \scong {P}'} \andalso {{P}' \red {Q}'} \andalso {{Q}' \scong {Q}}}{{P} \red {Q}}
\end{mathpar}

\begin{eqnarray*}
  match_{\equiv} (\quotep{P},\quotep{Q}) & := & P \equiv Q \\
  match_{\dagger}(\quotep{P},\quotep{Q}) & := & \forall R. P|Q \red^{*} R => R \red^{*} 0 \\
  match_{K}(\quotep{P},\quotep{Q}) & := & K \mbox{ for some context } K
\end{eqnarray*}

$u?(x)P | u!\langle Q \rangle \red P\{\quotep{Q}/x\}$

%We write $\wred$ for $\red^*$, and $P\red$ if $\exists Q $ such that $ P \red Q$.
We write $P\red$ if $\exists Q $ such that $ P \red Q$ and $P\not\red$, otherwise.

\section{Replication}

As mentioned before, it is known that replication (and hence
recursion) can be implemented in a higher-order process algebra
\cite{SangiorgiWalker}. As our first example of calculation with the
machinery thus far presented we give the construction explicitly in
the {\rhoc}.

\begin{eqnarray}
	D_{x} & := & \prefix{x}{y}{(\binpar{\outputp{x}{y}}{@{y}})} \nonumber\\
	\bangp_{x}{P} & := & \binpar{{x}!\langle{\binpar{D_{x}}{P}}\rangle}{D_{x}} \nonumber
\end{eqnarray}

\begin{eqnarray}
	\bangp_{x}{P} & & \nonumber\\
	=
	& {x}!\langle{(\prefix{x}{y}{(\outputp{x}{y} | @{y})) | P}}\rangle 
	      | \prefix{x}{y}{(\outputp{x}{y} | @{y})} & \nonumber\\
	\red
	& (\outputp{x}{y} | @{y})\substn{\quotep{(\prefix{x}{y}{(@{y} | \outputp{x}{y})) | P}}}{y} & \nonumber\\
	=
	& \outputp{x}{\quotep{(\prefix{x}{y}{(\outputp{x}{y} | @{y})) | P}}}
	  | {(\prefix{x}{y}{(\outputp{x}{y} | @{y})) | P}} & \nonumber\\
	\red
	& \ldots & \nonumber\\
	\red^*
	& P | P | \ldots & \nonumber
\end{eqnarray}

Of course, this encoding, as an implementation, runs away, unfolding
$\bangp{P}$ eagerly. A lazier and more implementable replication
operator, restricted to input-guarded processes, may be obtained as follows.

\begin{eqnarray}
\bangp{\prefix{u}{v}{P}} 
	:= 
	\binpar{\lift{x}{\prefix{u}{v}{(\binpar{D(x)}{P})}}}{D(x)} \nonumber
\end{eqnarray}

\begin{remark}
  Note that the lazier definition still does not deal with summation
  or mixed summation (i.e. sums over input and output). The reader is
  invited to construct definitions of replication that deal with these
  features. 

  Further, the definitions are parameterized in a name, $x$. Can you,
  gentle reader, make a definition that eliminates this parameter and
  guarantees no accidental interaction between the replication
  machinery and the process being replicated -- i.e. no accidental
  sharing of names used by the process to get its work done and the
  name(s) used by the replication to effect copying. This latter
  revision of the definition of replication is crucial to obtaining
  the expected identity $!!P \sim !P$.
\end{remark}

\begin{remark}\label{rem:paradoxical_combinator}
  The reader familiar with the lambda calculus will have noticed the
  similarity between $D$ and the paradoxical combinator.

  [Ed. note: the existence of this seems to suggest we have to be more
  restrictive on the set of processes and names we admit if we are to
  support no-cloning.]
\end{remark}

\subsubsection{Bisimulation}

The computational dynamics gives rise to another kind of equivalence,
the equivalence of computational behavior. As previously mentioned
this is typically captured \emph{via} some form of bisimulation.

% The notion we use in this paper is weak barbed bisimulation
% \cite{milner91polyadicpi}.

The notion we use in this paper is derived from weak barbed
bisimulation \cite{milner91polyadicpi}. 

\begin{definition}
An \emph{observation relation}, $\downarrow_{\mathcal N}$, over a set
of names, $\mathcal N$, is the smallest relation satisfying the rules
below.

\infrule[Out-barb]{y \in {\mathcal N}, \; x \nameeq y}
		  {\outputp{x}{v} \downarrow_{\mathcal N} x}
\infrule[Par-barb]{\mbox{$P\downarrow_{\mathcal N} x$ or $Q\downarrow_{\mathcal N} x$}}
		  {\binpar{P}{Q} \downarrow_{\mathcal N} x}

We write $P \Downarrow_{\mathcal N} x$ if there is $Q$ such that 
$P \wred Q$ and $Q \downarrow_{\mathcal N} x$.
\end{definition}

\begin{definition}
%\label{def.bbisim}
An  ${\mathcal N}$-\emph{barbed bisimulation} over a set of names, ${\mathcal N}$, is a symmetric binary relation 
${\mathcal S}_{\mathcal N}$ between agents such that $P\rel{S}_{\mathcal N}Q$ implies:
\begin{enumerate}
\item If $P \red P'$ then $Q \wred Q'$ and $P'\rel{S}_{\mathcal N} Q'$.
\item If $P\downarrow_{\mathcal N} x$, then $Q\Downarrow_{\mathcal N} x$.
\end{enumerate}
$P$ is ${\mathcal N}$-barbed bisimilar to $Q$, written
$P \wbbisim_{\mathcal N} Q$, if $P \rel{S}_{\mathcal N} Q$ for some ${\mathcal N}$-barbed bisimulation ${\mathcal S}_{\mathcal N}$.
\end{definition}

$\mathcal{R} \subseteq \pi \times \pi$

$P \mathcal{R} Q => \forall P'. P \red P' \Rightarrow \exists Q'. Q \red Q', P' \mathcal{R} Q'$

$P \vdash x \Rightarrow Q \vdash x$

\begin{mathpar}
  \inferrule*[lab=Out-barb]{x \nameeq y}{{y}!\langle{Q}\rangle \vdash x}
  \and
  \inferrule*[lab=Par-barb]{\mbox{$P\vdash x$ or $Q\vdash x$}}{\binpar{P}{Q} \vdash x}
\end{mathpar}

\subsubsection{Contexts}

One of the principle advantages of computational calculi like the
$\pi$-calculus is a well-defined notion of context,
contextual-equivalence and a correlation between
contextual-equivalence and notions of bisimulation. The notion of
context allows the decomposition of a process into (sub-)process and
its syntactic environment, its context. Thus, a context may be
thought of as a process with a ``hole'' (written $\Box$) in it. The
application of a context $M$ to a process $P$, written $M[P]$, is
tantamount to filling the hole in $M$ with $P$. In this paper we do
not need the full weight of this theory, but do make use of the notion
of context in the proof the main theorem. 

\begin{mathpar}
  \inferrule* [lab=summation] {} {{M_{M},M_{N}} \bc \Box \;|\; x.M_{A} \;|\; M_{M}+M_{N}}
  \and
  \inferrule* [lab=agent] {} {{M_{A}} \bc (\vec{x})M_{P} \;| \; \clift{P_0,\ldots,M_{P},\ldots,P_N}}
  \and \\
  \inferrule* [lab=process] {} {{M_{P}} \bc M_{N} \;| \;P|M_{P} }
\end{mathpar} 

\begin{mathpar}
  \inferrule* [lab=sychronization] {} {M_{N} \bc \Box \;|\; x?M_{F} \;|\; x!M_{C}}
  \and
  \inferrule* [lab=abstraction] {} {{M_{F}} \bc (x)M_{P} }
  \and
  \inferrule* [lab=concretion] {} {{M_{C}} \bc \langle M_{P} \rangle }
  \and \\
  \inferrule* [lab=process] {} {{M_{P}} \bc M_{N} \;| \;P|M_{P} }
\end{mathpar}

\begin{definition}[contextual application] Given a context $M$, and
  process $P$, we define the \emph{contextual application}, $M[P] :=
  M\{P/\Box\}$. That is, the contextual application of M to P is the
  substitution of $P$ for $\Box$ in $M$.
\end{definition}

$\meaningof{-} : L \to \mathcal{P}(\pi)$

\begin{mathpar}
  \inferrule* [lab=collection] {} {\meaningof{true} = \pi, \and \meaningof{~E} = \pi \setminus \meaningof{E}, \and \meaningof{E_{1} \& E_{2}} = \meaningof{E_{1}} \cap \meaningof{E_{2}}}
\end{mathpar}

\begin{mathpar}
  \inferrule* [lab=structure] {} {\meaningof{0} = \{ P \in \pi | P \equiv 0 \}, \and \\ \meaningof{E_1 | E_2} = \{ P \in \pi | P \equiv P_{1} | P_{2}, P_{1} \in \meaningof{E_{1}}, P_{2} \in \meaningof{E_2}\} }
\end{mathpar}

\begin{mathpar}
 \inferrule* [lab=behavior] {} {\meaningof{\langle a?b \rangle E} = \{ P \in \pi | P \equiv Q | u?(y)P', \\ \and \\\\ \and \\ \;\;\; u \in \meaningof{a}, \forall z.P'\{z/y\} \in \meaningof{E\{z/b\}}\}, \and \\ \meaningof{a!E} = \{ P \in \pi | P \equiv Q | x!\langle P' \rangle, x \in \meaningof{a} P' \in \meaningof{E}\} }
\end{mathpar}

\begin{mathpar}
 \inferrule* [lab=nominal] {} {\meaningof{\quotep{E}} = \{ \quotep{P} \in \quotep{\pi} | P \in \meaningof{E} \}, \and \meaningof{\quotep{P}} = \{ \quotep{Q} \in \quotep{\pi} | P \equiv Q \} \and \\ \meaningof{@\quotep{E}} = \{ P \in \pi | P \equiv @x, x \in \meaningof{E} \}}
\end{mathpar}

\begin{eqnarray*}
  \\
  \meaningof{-} : TS \to ST
\end{eqnarray*}

\begin{eqnarray*}
  \\
  L : TS \to ST
\end{eqnarray*}

\begin{eqnarray*}
  \\
  P \models E \iff P \in \meaningof{E}
\end{eqnarray*}

\begin{eqnarray*}
  P \approx_{L} Q \iff \forall E \in L. P \models E \iff Q \models E
\end{eqnarray*}

\begin{eqnarray*}
  P \approx_{K} Q
\end{eqnarray*}

\begin{eqnarray*}
  P \approx Q
\end{eqnarray*}

$\approx_{K} = \approx = \approx_{L}$

\subsubsection{Contextual duality}

Note that contexts extend the quotation operation to a family of
operations from processes to names. Given a context, $M$, we can
define a \emph{nominal context}, $\quotep{M}$ by $\quotep{M}[P] :=
\quotep{M[P]}$. To foreshadow what is to come we observe that these
operations enjoy a duality with processes very much like the duality
between vectors and maps from vectors to scalars.

Further, because the calculus is essentially higher-order, we have a
correspondence between contexts and processes. More specifically,
given a name $x$ and a context $M$ we can construct $M^{*}_{x}$ such
that 

\begin{mathpar}
  M^{*}_{x} | \lift{x}{P} \red M[P]
\end{mathpar}

namely,

\begin{mathpar}
  M^{*}_{x} := x?(u).M[\dropn{u}]
\end{mathpar}

The dependence of $M^{*}_{x}$ on a name makes it an abstraction, 

\begin{mathpar}
  M^{*} := (x)x?(u).M[\dropn{u}]
\end{mathpar}

\subsection{Additional notation}

It will sometimes be convenient to denote the process a name
quotes. We already have the notation $x = \quotep{P}$, but it will be
convenient to introduce an alternate notation, $\procn{x}$, when we
want to emphasize the connection to the use of the name. Note that, by
virtue of name equivalence, $\quotep{\procn{x}} \nameeq x$; so, the
notation is consistent with previous definitions.

Further, because names have structure it is possible to effect
substitutions on the basis of that structure. This means we need to
upgrade our notation for substitutions, which we accomplish by
adapting comprehension notation. Thus,

\begin{mathpar}
  P\{ y / x : x \in S \}
\end{mathpar}

is interpreted to mean the process derived from P by replacing (in a
capture-avoiding manner) each occurrence of $x$ in $S$ by $y$. For example,

\begin{mathpar}
  P\{ \quotep{\procn{x}|\procn{x}} / x : x \in \freenames{P} \}
\end{mathpar}

will replace each (occurrence) of a free name $x$ in $P$ by
$\quotep{\procn{x}|\procn{x}}$.

Also, we will avail ourselves of the notation $x^{L}$ and $x^{R}$ to
denote injections of a name into disjoint copies of the name
space. There are numerous ways to accomplish this. One example can be
found in \cite{MeredithR05}. This notation overloads to vectors of
names: $\vec{x}^{\pi} := (x_{i}^{\pi} \; : \; 0 \leq i < |\vec{x}| )$ where $\pi \in \{L,R\}$.

We also use $P^{\Box} := P|\Box$.

In \cite{MeredithR05} an interpretation of the new operator is
given. It turns out that there are several possible interpretations
all enjoying the requisite algebraic properties of the operator (see
\cite{milner91polyadicpi}). We will therefore make liberal use of
$(\nu\; \vec{x})P$.

% subsection the_syntax_and_semantics_of_the_notation_system (end)   

\input{qm2pi.qmops} 

\input{qm2pi.sterngerlach} 

\input{qm2pi.metric} 

% section concurrent_process_calculi (end)

%\input{qm2pi.proofsketch}

% section proof sketch (end)

%\input{qm2pi.slviaknots} 

% section spatial logic via knots (end)

\input{qm2pi.conclusion}

% section conclusion (end)

%\input{qm2pi.dtcodes} 

% section wiring algorithm (end)

\input{qm2pi.ack} 

% section acknowledgments (end)

\newpage


\bibliographystyle{plain}   
\bibliography{../../biblios/main.bib}

\input{qm2pi.rhodetails}

\end{document}

 

% subsection basic_interpretation (end)

%\input{qm2pi.rho.presentation} 
\subsection{The syntax and semantics of the notation system}\label{sub:the_syntax_and_semantics_of_the_notation_system} % (fold)

We now summarize a technical presentation of the calculus that
embodies our theory of dynamics. The typical presentation of such a
calculus follows the style of giving generators and relations on
them. The grammar, below, describing term constructors, freely
generates the set of processes, $\Proc$. This set is then quotiented
by a relation known as structural congruence and it is over this set
that the notion of dynamics is expressed. This presentation is
essentially that of \cite{MeredithR05} with the addition of
polyadicity and summation. For readability we have relegated some of
the technical subtleties to an appendix.

\subsubsection{Process grammar}\label{subsub:process_grammar}

\begin{mathpar}
  \inferrule* [lab=synchronization] {} {{M} \bc \pzero \;|\; x?F \;|\; x!C }
  \and
  \inferrule* [lab=abstraction] {} {{F} \bc (x)P}
  \and
  \inferrule* [lab=concretion] {} {{C} \bc \langle Q \rangle}
  \and
  \inferrule* [lab=process] {} {{P,Q} \bc M \;| \;P|Q \;|\; @{x}}
  \and
  \inferrule* [lab=name] {} {{x} \bc \quotep{P}}
\end{mathpar} 

Note that $\vec{x}$ (resp. $\vec{P}$) denotes a vector of names
(resp. processes) of length $|\vec{x}|$ (resp. $|\vec{P}|$). We adopt
the following useful abbreviations.

\begin{mathpar}
   x?(\vec{y}).P := x.(\vec{y})P \and  x\clift{\vec{P}} := x.\clift{\vec{P}}
   \and x!(y) := \lift{x}{\dropn{y}}
   \and \Pi_{i=0}^{n-1}P_i := P_0 | \ldots | P_{n-1}
\end{mathpar}

\subsubsection{Structural congruence}

\paragraph{Free and bound names and alpha-equivalence.} At the
core of structural equivalence is alpha-equivalence which identifies
process that are the same up to a change of variable. Formally, we
recognize the distinction between free and bound names. The free names
of a process, $\freenames{P}$, may be calculated recursively as
follows:

\begin{mathpar}
\freenames{\pzero} := \emptyset
  \and \\
  \freenames{x?(y).P} := \{ x \} \cup (\freenames{P} \setminus \{ y \})
  \and 
  \freenames{x!\langle P \rangle} := \{ x \} \cup \{ P \} 
  \and \\
  \freenames{P|Q} := \freenames{P} \cup \freenames{Q}
  \and \\
  \freenames{@{x}} := \{ x \}
\end{mathpar}

$\pi$
$\quotep{\pi}$

$\freenames{-} : \pi \to \mathcal{P}(\quotep{\pi})$

\begin{eqnarray*}
  \freenames{\pzero} & := & \emptyset \\
  \freenames{x?(y).P} & := & \{ x \} \cup (\freenames{P} \setminus \{ y \}) \\
  \freenames{x!\langle P \rangle} & := & \{ x \} \cup \{ P \} \\
  \freenames{P|Q} & := & \freenames{P} \cup \freenames{Q} \\
  \freenames{\dropn{x}} & := & \{ x \}
\end{eqnarray*}

The bound names of a process, $\boundnames{P}$, are those names occurring in $P$
that are not free. For example, in $x?(y).0$, the name $x$ is free, while $y$ is bound.

\begin{mathpar}
  \inferrule* [lab=monoidal-laws] {} { P|Q \equiv Q|P \and P|0 \equiv P \and P|(Q|R) \equiv (P|Q)|R }
\end{mathpar}

\begin{mathpar}
  \inferrule* [lab=alpha-equivalence] {} { (x)P \equiv (y)P\{y/x\} \and y \not\in \freenames{P} }
\end{mathpar}

\begin{definition}
Then two processes, $P,Q$, are alpha-equivalent if $P = Q\{\vec{y}/\vec{x}\}$ for
some $\vec{x} \in \boundnames{Q},\vec{y} \in \boundnames{P}$, where $Q\{\vec{y}/\vec{x}\}$
denotes the capture-avoiding substitution of $\vec{y}$ for $\vec{x}$ in $Q$.
\end{definition}

\begin{definition}
  The {\em structural congruence} \cite{SangiorgiWalker} , $\equiv$,
  between processes is the least congruence containing
  alpha-equivalence, satisfying the abelian monoid laws
  (associativity, commutativity and $\pzero$ as identity) for parallel
  composition $|$ and for summation $+$.
\end{definition}

\subsection{Name equivalence}

We take name equivalence, written $\nameeq$, to be the smallest
equivalence relation generated by the following rules.

\begin{mathpar}
\inferrule*[lab=Quote-drop]
{ }
{ \quotep{@{x}} \nameeq x }

\inferrule*[lab=Struct-equiv]
{ P \scong Q }
{ \quotep{P} \nameeq \quotep{Q} }
\end{mathpar}

The astute reader will have noticed that the mutual recursion of names
and processes imposes a mutual recursion on alpha-equivalence and
structural equivalence via name-equivalence. Fortunately, all of this
works out pleasantly and we may calculate in the natural way, free of
concern. The reader interested in the details is referred to the
appendix \ref{appendix:rho_details}.

\subsection{Substitution}

We use $\Proc$ for the set of processes, $\QProc$ for the set of
names, and $\id{\{}\vec{y} / \vec{x} \id{\}}$ to denote partial maps,
$s : \QProc \rightarrow \QProc$. A map, $s$ lifts, uniquely, to a map
on process terms, $\widehat{s} : \Proc \rightarrow \Proc$ by the
following equations.

\begin{mathpar}
  (0) \psubstp{Q}{P} := 0 \\
  (R \juxtap S) \psubstp{Q}{P}
  :=    
  (R)\psubstp{Q}{P} \juxtap (S) \psubstp{Q}{P} \\
  (x?(y).R) \psubstp{Q}{P}    
  :=    
  (x)\substp{Q}{P} (z)\concat( (R \psubstn{z}{y}) \psubstp{Q}{P} ) \\
  (\lift{x}{R}) \psubstp{Q}{P}  
  :=
  \lift{(x)\substp{Q}{P}}{ R \psubstp{Q}{P} } \\
%   (\dropn{x})  \psubstp{Q}{P}       
%   := 
%   \left\{ 
%     \begin{array}{ccc} 
%       \dropn{\quotep{Q}} & & x \nameeq \quotep{P} \\
%       \dropn{x} & & otherwise \\
%     \end{array}
%   \right. 
  (\dropn{x})  \psubstp{Q}{P}       
  := 
  \left\{ 
    \begin{array}{ccc} 
      Q & & x \nameeq \quotep{P} \\
      \dropn{x} & & otherwise \\
    \end{array}
  \right.
\end{mathpar}
 

where

\begin{eqnarray}
  (x)\id{\{} \lpquote Q \rpquote / \lpquote P \rpquote \id{\}}            = 
  \left\{ 
    \begin{array}{ccc}
      \lpquote Q \rpquote & & x \nameeq \lpquote P \rpquote \\
      x & & otherwise \\
    \end{array}
  \right. \nonumber
\end{eqnarray}

and $z$ is chosen distinct from $\quotep{P}$, $\quotep{Q}$, the free
names in $Q$, and all the names in $R$. Our $\alpha$-equivalence will
be built in the standard way from this substitution.

\begin{remark}\label{rem:no_self_referential_names}
  One consequence of these definitions is that $\forall P. \quotep{P}
  \not\in \freenames{P}$.
\end{remark}

\subsection{ Dynamic quote: an example }

Anticipating something of what's to come, consider applying the
substitution, $\widehat{\id{\{}u / z \id{\}}}$, to the following pair
of processes, $\lift{w}{y!(z)}$ and $w[ \lpquote y!(z) \rpquote ]$.

\begin{eqnarray}
	\lift{w}{y!(z)}\widehat{\id{\{}u / z \id{\}}}
		& = &
		\lift{w}{y!(u)} \nonumber\\
	w[ \lpquote y!(z) \rpquote ] \widehat{ \id{\{}u / z \id{\}} }
		& = &
		w[ \lpquote y!(z) \rpquote ] \nonumber
\end{eqnarray}

Because the body of the process between quotes is impervious to
substitution, we get radically different answers. In fact, by
examining the first process in an input context,
e.g. $x?(z).\lift{w}{y!(z)}$, we see that the process under the lift
operator may be shaped by prefixed inputs binding a name inside it. In
this sense, the lift operator will be seen as a way to dynamically
construct processes before reifying them as names.

Finally equipped with these standard features we can present the
dynamics of the calculus.

\subsubsection{Operational semantics} 

Finally, we introduce the computational dynamics. What marks these
algebras as distinct from other more traditionally studied algebraic
structures, e.g. vector spaces or polynomial rings, is the manner in
which dynamics is captured. In traditional structures, dynamics is typically
expressed through morphisms between such structures, as in linear maps
between vector spaces or morphisms between rings. In algebras
associated with the semantics of computation, the dynamics is
expressed as part of the algebraic structure itself, through a
reduction reduction relation typically denoted by $\red$. Below, we
give a recursive presentation of this relation for the calculus used
in the encoding.

$\red \subseteq \pi \times \pi$
$\red : \pi \to \mathcal{P}(\pi)$

\begin{mathpar}
  \inferrule* [lab=Comm] { \textsf{match}( x_{src}, x_{trgt} ) } { x_{trgt}?(y)P \; | \; x_{src}!\langle {Q} \rangle \red P\{\quotep{Q}/y}\} }
  \and \\
  \inferrule* [lab=Par] {{P} \red {P}'} {{{P} | {Q}} \red {{P}' | {Q}}}
  \and
  \inferrule* [lab=Equiv]{{{P} \scong {P}'} \andalso {{P}' \red {Q}'} \andalso {{Q}' \scong {Q}}}{{P} \red {Q}}
\end{mathpar}

\begin{eqnarray*}
  match_{\equiv} (\quotep{P},\quotep{Q}) & := & P \equiv Q \\
  match_{\dagger}(\quotep{P},\quotep{Q}) & := & \forall R. P|Q \red^{*} R => R \red^{*} 0 \\
  match_{K}(\quotep{P},\quotep{Q}) & := & K \mbox{ for some context } K
\end{eqnarray*}

$u?(x)P | u!\langle Q \rangle \red P\{\quotep{Q}/x\}$

%We write $\wred$ for $\red^*$, and $P\red$ if $\exists Q $ such that $ P \red Q$.
We write $P\red$ if $\exists Q $ such that $ P \red Q$ and $P\not\red$, otherwise.

\section{Replication}

As mentioned before, it is known that replication (and hence
recursion) can be implemented in a higher-order process algebra
\cite{SangiorgiWalker}. As our first example of calculation with the
machinery thus far presented we give the construction explicitly in
the {\rhoc}.

\begin{eqnarray}
	D_{x} & := & \prefix{x}{y}{(\binpar{\outputp{x}{y}}{@{y}})} \nonumber\\
	\bangp_{x}{P} & := & \binpar{{x}!\langle{\binpar{D_{x}}{P}}\rangle}{D_{x}} \nonumber
\end{eqnarray}

\begin{eqnarray}
	\bangp_{x}{P} & & \nonumber\\
	=
	& {x}!\langle{(\prefix{x}{y}{(\outputp{x}{y} | @{y})) | P}}\rangle 
	      | \prefix{x}{y}{(\outputp{x}{y} | @{y})} & \nonumber\\
	\red
	& (\outputp{x}{y} | @{y})\substn{\quotep{(\prefix{x}{y}{(@{y} | \outputp{x}{y})) | P}}}{y} & \nonumber\\
	=
	& \outputp{x}{\quotep{(\prefix{x}{y}{(\outputp{x}{y} | @{y})) | P}}}
	  | {(\prefix{x}{y}{(\outputp{x}{y} | @{y})) | P}} & \nonumber\\
	\red
	& \ldots & \nonumber\\
	\red^*
	& P | P | \ldots & \nonumber
\end{eqnarray}

Of course, this encoding, as an implementation, runs away, unfolding
$\bangp{P}$ eagerly. A lazier and more implementable replication
operator, restricted to input-guarded processes, may be obtained as follows.

\begin{eqnarray}
\bangp{\prefix{u}{v}{P}} 
	:= 
	\binpar{\lift{x}{\prefix{u}{v}{(\binpar{D(x)}{P})}}}{D(x)} \nonumber
\end{eqnarray}

\begin{remark}
  Note that the lazier definition still does not deal with summation
  or mixed summation (i.e. sums over input and output). The reader is
  invited to construct definitions of replication that deal with these
  features. 

  Further, the definitions are parameterized in a name, $x$. Can you,
  gentle reader, make a definition that eliminates this parameter and
  guarantees no accidental interaction between the replication
  machinery and the process being replicated -- i.e. no accidental
  sharing of names used by the process to get its work done and the
  name(s) used by the replication to effect copying. This latter
  revision of the definition of replication is crucial to obtaining
  the expected identity $!!P \sim !P$.
\end{remark}

\begin{remark}\label{rem:paradoxical_combinator}
  The reader familiar with the lambda calculus will have noticed the
  similarity between $D$ and the paradoxical combinator.

  [Ed. note: the existence of this seems to suggest we have to be more
  restrictive on the set of processes and names we admit if we are to
  support no-cloning.]
\end{remark}

\subsubsection{Bisimulation}

The computational dynamics gives rise to another kind of equivalence,
the equivalence of computational behavior. As previously mentioned
this is typically captured \emph{via} some form of bisimulation.

% The notion we use in this paper is weak barbed bisimulation
% \cite{milner91polyadicpi}.

The notion we use in this paper is derived from weak barbed
bisimulation \cite{milner91polyadicpi}. 

\begin{definition}
An \emph{observation relation}, $\downarrow_{\mathcal N}$, over a set
of names, $\mathcal N$, is the smallest relation satisfying the rules
below.

\infrule[Out-barb]{y \in {\mathcal N}, \; x \nameeq y}
		  {\outputp{x}{v} \downarrow_{\mathcal N} x}
\infrule[Par-barb]{\mbox{$P\downarrow_{\mathcal N} x$ or $Q\downarrow_{\mathcal N} x$}}
		  {\binpar{P}{Q} \downarrow_{\mathcal N} x}

We write $P \Downarrow_{\mathcal N} x$ if there is $Q$ such that 
$P \wred Q$ and $Q \downarrow_{\mathcal N} x$.
\end{definition}

\begin{definition}
%\label{def.bbisim}
An  ${\mathcal N}$-\emph{barbed bisimulation} over a set of names, ${\mathcal N}$, is a symmetric binary relation 
${\mathcal S}_{\mathcal N}$ between agents such that $P\rel{S}_{\mathcal N}Q$ implies:
\begin{enumerate}
\item If $P \red P'$ then $Q \wred Q'$ and $P'\rel{S}_{\mathcal N} Q'$.
\item If $P\downarrow_{\mathcal N} x$, then $Q\Downarrow_{\mathcal N} x$.
\end{enumerate}
$P$ is ${\mathcal N}$-barbed bisimilar to $Q$, written
$P \wbbisim_{\mathcal N} Q$, if $P \rel{S}_{\mathcal N} Q$ for some ${\mathcal N}$-barbed bisimulation ${\mathcal S}_{\mathcal N}$.
\end{definition}

$\mathcal{R} \subseteq \pi \times \pi$

$P \mathcal{R} Q => \forall P'. P \red P' \Rightarrow \exists Q'. Q \red Q', P' \mathcal{R} Q'$

$P \vdash x \Rightarrow Q \vdash x$

\begin{mathpar}
  \inferrule*[lab=Out-barb]{x \nameeq y}{{y}!\langle{Q}\rangle \vdash x}
  \and
  \inferrule*[lab=Par-barb]{\mbox{$P\vdash x$ or $Q\vdash x$}}{\binpar{P}{Q} \vdash x}
\end{mathpar}

\subsubsection{Contexts}

One of the principle advantages of computational calculi like the
$\pi$-calculus is a well-defined notion of context,
contextual-equivalence and a correlation between
contextual-equivalence and notions of bisimulation. The notion of
context allows the decomposition of a process into (sub-)process and
its syntactic environment, its context. Thus, a context may be
thought of as a process with a ``hole'' (written $\Box$) in it. The
application of a context $M$ to a process $P$, written $M[P]$, is
tantamount to filling the hole in $M$ with $P$. In this paper we do
not need the full weight of this theory, but do make use of the notion
of context in the proof the main theorem. 

\begin{mathpar}
  \inferrule* [lab=summation] {} {{M_{M},M_{N}} \bc \Box \;|\; x.M_{A} \;|\; M_{M}+M_{N}}
  \and
  \inferrule* [lab=agent] {} {{M_{A}} \bc (\vec{x})M_{P} \;| \; \clift{P_0,\ldots,M_{P},\ldots,P_N}}
  \and \\
  \inferrule* [lab=process] {} {{M_{P}} \bc M_{N} \;| \;P|M_{P} }
\end{mathpar} 

\begin{mathpar}
  \inferrule* [lab=sychronization] {} {M_{N} \bc \Box \;|\; x?M_{F} \;|\; x!M_{C}}
  \and
  \inferrule* [lab=abstraction] {} {{M_{F}} \bc (x)M_{P} }
  \and
  \inferrule* [lab=concretion] {} {{M_{C}} \bc \langle M_{P} \rangle }
  \and \\
  \inferrule* [lab=process] {} {{M_{P}} \bc M_{N} \;| \;P|M_{P} }
\end{mathpar}

\begin{definition}[contextual application] Given a context $M$, and
  process $P$, we define the \emph{contextual application}, $M[P] :=
  M\{P/\Box\}$. That is, the contextual application of M to P is the
  substitution of $P$ for $\Box$ in $M$.
\end{definition}

$\meaningof{-} : L \to \mathcal{P}(\pi)$

\begin{mathpar}
  \inferrule* [lab=collection] {} {\meaningof{true} = \pi, \and \meaningof{~E} = \pi \setminus \meaningof{E}, \and \meaningof{E_{1} \& E_{2}} = \meaningof{E_{1}} \cap \meaningof{E_{2}}}
\end{mathpar}

\begin{mathpar}
  \inferrule* [lab=structure] {} {\meaningof{0} = \{ P \in \pi | P \equiv 0 \}, \and \\ \meaningof{E_1 | E_2} = \{ P \in \pi | P \equiv P_{1} | P_{2}, P_{1} \in \meaningof{E_{1}}, P_{2} \in \meaningof{E_2}\} }
\end{mathpar}

\begin{mathpar}
 \inferrule* [lab=behavior] {} {\meaningof{\langle a?b \rangle E} = \{ P \in \pi | P \equiv Q | u?(y)P', \\ \and \\\\ \and \\ \;\;\; u \in \meaningof{a}, \forall z.P'\{z/y\} \in \meaningof{E\{z/b\}}\}, \and \\ \meaningof{a!E} = \{ P \in \pi | P \equiv Q | x!\langle P' \rangle, x \in \meaningof{a} P' \in \meaningof{E}\} }
\end{mathpar}

\begin{mathpar}
 \inferrule* [lab=nominal] {} {\meaningof{\quotep{E}} = \{ \quotep{P} \in \quotep{\pi} | P \in \meaningof{E} \}, \and \meaningof{\quotep{P}} = \{ \quotep{Q} \in \quotep{\pi} | P \equiv Q \} \and \\ \meaningof{@\quotep{E}} = \{ P \in \pi | P \equiv @x, x \in \meaningof{E} \}}
\end{mathpar}

\begin{eqnarray*}
  \\
  \meaningof{-} : TS \to ST
\end{eqnarray*}

\begin{eqnarray*}
  \\
  L : TS \to ST
\end{eqnarray*}

\begin{eqnarray*}
  \\
  P \models E \iff P \in \meaningof{E}
\end{eqnarray*}

\begin{eqnarray*}
  P \approx_{L} Q \iff \forall E \in L. P \models E \iff Q \models E
\end{eqnarray*}

\begin{eqnarray*}
  P \approx_{K} Q
\end{eqnarray*}

\begin{eqnarray*}
  P \approx Q
\end{eqnarray*}

$\approx_{K} = \approx = \approx_{L}$

\subsubsection{Contextual duality}

Note that contexts extend the quotation operation to a family of
operations from processes to names. Given a context, $M$, we can
define a \emph{nominal context}, $\quotep{M}$ by $\quotep{M}[P] :=
\quotep{M[P]}$. To foreshadow what is to come we observe that these
operations enjoy a duality with processes very much like the duality
between vectors and maps from vectors to scalars.

Further, because the calculus is essentially higher-order, we have a
correspondence between contexts and processes. More specifically,
given a name $x$ and a context $M$ we can construct $M^{*}_{x}$ such
that 

\begin{mathpar}
  M^{*}_{x} | \lift{x}{P} \red M[P]
\end{mathpar}

namely,

\begin{mathpar}
  M^{*}_{x} := x?(u).M[\dropn{u}]
\end{mathpar}

The dependence of $M^{*}_{x}$ on a name makes it an abstraction, 

\begin{mathpar}
  M^{*} := (x)x?(u).M[\dropn{u}]
\end{mathpar}

\subsection{Additional notation}

It will sometimes be convenient to denote the process a name
quotes. We already have the notation $x = \quotep{P}$, but it will be
convenient to introduce an alternate notation, $\procn{x}$, when we
want to emphasize the connection to the use of the name. Note that, by
virtue of name equivalence, $\quotep{\procn{x}} \nameeq x$; so, the
notation is consistent with previous definitions.

Further, because names have structure it is possible to effect
substitutions on the basis of that structure. This means we need to
upgrade our notation for substitutions, which we accomplish by
adapting comprehension notation. Thus,

\begin{mathpar}
  P\{ y / x : x \in S \}
\end{mathpar}

is interpreted to mean the process derived from P by replacing (in a
capture-avoiding manner) each occurrence of $x$ in $S$ by $y$. For example,

\begin{mathpar}
  P\{ \quotep{\procn{x}|\procn{x}} / x : x \in \freenames{P} \}
\end{mathpar}

will replace each (occurrence) of a free name $x$ in $P$ by
$\quotep{\procn{x}|\procn{x}}$.

Also, we will avail ourselves of the notation $x^{L}$ and $x^{R}$ to
denote injections of a name into disjoint copies of the name
space. There are numerous ways to accomplish this. One example can be
found in \cite{MeredithR05}. This notation overloads to vectors of
names: $\vec{x}^{\pi} := (x_{i}^{\pi} \; : \; 0 \leq i < |\vec{x}| )$ where $\pi \in \{L,R\}$.

We also use $P^{\Box} := P|\Box$.

In \cite{MeredithR05} an interpretation of the new operator is
given. It turns out that there are several possible interpretations
all enjoying the requisite algebraic properties of the operator (see
\cite{milner91polyadicpi}). We will therefore make liberal use of
$(\nu\; \vec{x})P$.

% subsection the_syntax_and_semantics_of_the_notation_system (end)   

\section{Interpretation of QM}
\subsection{Supporting definitions}
\subsubsection{Multiplication}
\begin{mathpar}
  \quotep{Q} \cdot \quotep{R} := \quotep{Q|R}
  \and \\
  \quotep{Q} \cdot P := P\{ \quotep{Q|R} / \quotep{R} : \quotep{R} \in \freenames{P} \}
\end{mathpar}

\paragraph{Discussion}
The first line needs little explanation. The second line says that
each free name of the process is replaced with the multiplication of
that name by the scalar. Multiplication of a scalar (name) by a state
(process) results in a process all the names of which have been `moved
over' by parallel composition with the process the scalar
quotes. There is a subtlety that the bound names have to be
manipulated so that multiplied names aren't accidentally
captured. There are many ways to achieve this.

\begin{remark}\label{rem:multiplication_identities}
  The reader is invited to verify that for all $x,y,z \in \QProc$ and $P \in \Proc$
  \begin{mathpar}
    x \cdot \quotep{0} \equiv x 
    \and
    x \cdot y \equiv y \cdot x
    \and
    x \cdot (y \cdot z) \equiv (x \cdot y) \cdot z
    \and \\
    \quotep{0} \cdot P \equiv P
    \and \\
    x \cdot (y \cdot P) \equiv (x \cdot y) \cdot P
    \and \\
    x \cdot (P|Q) \equiv (x \cdot P) | (x \cdot Q)
    \and \\    
  \end{mathpar}
\end{remark}

\subsubsection{Tensor product}

We define a tensor product on processes by structural induction.

\paragraph{Tensor of sums} First note that all summations, including
$\pzero$ and sequence, can be written $\Sigma_{i} x_{i}.A_{i} +
\Sigma_{j} x_{j}.C_{j}$, where we have grouped input-guarded processes
together and output-guarded processes together.

Thus, we can define the tensor product of two summations, $N_{1}\otimes N_{2}$, where

\begin{mathpar}
  N_{1} := \Sigma_{i} x_{i}.A_{i} + \Sigma_{j} x_{j}.C_{j}
  \and
  N_{2} := \Sigma_{i'} y_{i'}.B_{i'} + \Sigma_{j'} y_{j'}.D_{j'} 
\end{mathpar}

as follows.

\begin{mathpar}
  \Sigma_{i} x_{i}.A_{i} + \Sigma_{j} x_{j}.C_{j} \otimes \Sigma_{i'}
  y_{i'}.B_{i'} + \Sigma_{j'} y_{j'}.D_{j'} 
  \and \\
  := \; \Sigma_{i} \Sigma_{i'} \quotep{\stackrel{\vee}{x_{i}}| \stackrel{\vee}{y_{i'}}}.(A_{i}\otimes B_{i'}) \; | \; \Sigma_{i'} \Sigma_{i} \quotep{\stackrel{\vee}{y_{i'}}|\stackrel{\vee}{x_{i}}}.(B_{i'}\otimes A_{i})
  \and
  \;\; | \;\; \Sigma_{j} \Sigma_{j'} \quotep{\stackrel{\vee}{x_{j}}|\stackrel{\vee}{y_{j'}}}.(A_{j}\otimes B_{j'}) \; | \; \Sigma_{j'} \Sigma_{j} \quotep{\stackrel{\vee}{y_{j'}}|\stackrel{\vee}{x_{j}}}.(B_{j'}\otimes A_{j})
\end{mathpar}

\begin{remark}
  Do we need to $x^{L}$ and $y^{R}$ for this construction as well?
\end{remark}

\paragraph{Tensor of parallel compositions} Next, we distribute tensor
over par.

\begin{mathpar}
  P_{1}|P_{2} \otimes Q_{1}|Q_{2} := (P_{1} \otimes Q_{1}) | (P_{1}
  \otimes Q_{2}) | (P_{2} \otimes Q_{1}) | (P_{2} \otimes Q_{2})
\end{mathpar}

\paragraph{Tensor with dropped names} We treat tensor of a
process with a dropped name as parallel composition.

\begin{mathpar}
  P \otimes \dropn{x} := P | \dropn{x}
\end{mathpar}

\paragraph{Tensor of agents}

Finally, we need to define tensor on agents. Note that the definition
of tensor on normal products only tensors inputs with inputs and
outputs with outputs. Thus, we only have to define the operation on
``homogeneous'' pairings.

\begin{mathpar}
  (\vec{x})P \otimes (\vec{y})Q
  \and \\
  := (x_{0}^{L}|y_{0}^{R},\ldots,x_{0}^{L}|y_{n}^{R},\ldots,x_{m}^{L}|y_{0}^{R},\ldots,x_{m}^{L}|y_{n}^R)(P\{ \vec{x}^{L}/\vec{x}\} \otimes Q \{ \vec{y}^{R}/\vec{y}\})
  \and \\
  \clift{\vec{P}} \otimes \clift{\vec{Q}}
  \and \\
  := \clift{P_{0}\otimes Q_{0},\ldots,P_{0}\otimes Q_{n},\ldots,P_{m}\otimes Q_{0},\ldots,P_{m}\otimes Q_{n}}
\end{mathpar}

\begin{remark}
  Observe that arities of tensored abstractions matches arities of
  tensored concretions if the original arities matched. Note also that
  the length of the arities corresponds to the increase in dimension
  we see in ordinary vector space tensor product.
\end{remark}

\begin{remark}
  Operationally, this definition distributes the tensor down to
  components ``linked'' by summation. Tensor over summation is
  intriguing in that it mixes names. Moreover, as a consequence of the
  way it mixes names we have the identities for all $x \in \QProc$ and
  $P,Q \in \Proc$

  \begin{mathpar}
    (x \cdot P) \otimes Q \equiv x \cdot (P \otimes Q) \equiv P \otimes (x \cdot Q)
    \and
    P \otimes \pzero \equiv P
  \end{mathpar}

  that the reader is invited to verify.
\end{remark}

\subsubsection{Annihilation}
\begin{mathpar}
  P^{\perp} := \{ Q | \forall R. P|Q \red^{*} R \Rightarrow R \red^{*} \pzero \}
  \and \\
  P^{\underline{\perp}} := \Sigma_{Q \in P^{\perp}} \quotep{Q}?(y).(\dropn{y}|Q) | \Sigma_{Q \in P^{\perp}} \quotep{Q}\clift{\Box}
\end{mathpar}

\paragraph{Discussion} The reader will note that $P^{\perp}$ is a
\emph{set} of processes, while $P^{\underline{\perp}}$ is a
\emph{context}. We call the set $P^{\perp}$ the \emph{annihilators} of
$P$. The parallel composition of a process in the annihilators of $P$
with $P$ will result in a process, the state space of which has all
paths eventually leading to $\pzero$. Execution may endure loops; but
under reasonable conditions of fairness (naturally guaranteed under
most notions of bisimulation) such a composite process cannot get
stuck in such a loop and will, eventually pop out and terminate.

The context $P^{\underline{\perp}}$ is ready and willing to ``take the
$P$ out of'' the process to which it is applied. It will effectively
transmit the code of the process to which it is applied to one of the
annihilators and run the process against it.

\subsubsection{Evaluation}
We fix $M$ a domain of fully abstract interpretation with an equality
coincident with bisimulation. We take $\meaningof{\cdot} : \Proc \to
M$ to be the map interpreting processes and $\nmeaningof{\cdot} : \M
\to Proc$ to be the map running the other way. Then we define

\begin{mathpar}
  \int P := \nmeaningof{\meaningof{P}}
\end{mathpar}

\paragraph{Discussion}
There are many fully abstract interpretations of Milner's
$\pi$-calculus. Any of them can be used as a basis for interpreting
the reflective calculus here. Equipped with such a domain it is
largely a matter of grinding through to check that the Yoneda
construction for the normalization-by-evaluation program can be
extended to this setting.

\begin{remark}
  The reader is invited to verify that $\int (P^{\underline{\perp}}[P]) = 0$.
\end{remark}

\subsection{Quantum mechanics}

Table \ref{tbl:core_qm_op_defns} gives the core operational definitions

\begin{table}[htp]\label{tbl:core_qm_op_defns}
  \center{
    \fbox{
      \begin{tabular}{c|c}
        quantum mechanics & process calculus \\
        \hline
        scalar & $x := \quotep{P}$ \\
        state vector & $\state{P} := P$ \\
        dual & $\state{P}^{*} := \event{P^{\underline{\perp}}} := \quotep{P^{\underline{\perp}}}[-]$ \\
        matrix & $ \Sigma_{\alpha} \state{P_{\alpha}}x_{\alpha}\event{Q_{\alpha}}$ \\
        vector addition & $\state{P} + \state{Q} := \state{P | Q}$ \\
        tensor product & $\state{P} \otimes \state{Q} := \state{P \otimes Q}$ \\
        inner product & $\innerprod{P}{Q} := \quotep{\int P^{\underline{\perp}}[Q]}$ \\
      \end{tabular}
    }
  }
  \caption{QM - operational definitions}
\end{table}

where

\begin{mathpar}
  \prmatrix{P}{Q} := \fprmatrix{P}{\quotep{\pzero}}{Q}
  \and
  \fprmatrix{P}{x}{Q} := (\state{P},x,\event{Q})
  \and
  (\fprmatrix{P}{x}{Q})(\state{R}) := x \cdot \innerprod{Q}{R} \cdot \state{P}
  \and
  (\fprmatrix{P}{x}{Q})(\event{R}) := x \cdot \innerprod{R}{P} \cdot \event{Q}
\end{mathpar}

\paragraph{Discussion}
As promised: vectors (aka states) are represented as processes; duals
as contextual duals; inner product definition should be compared with
standard inner product definition for ....

\begin{remark}
  Assuming $\int (P^{\underline{\perp}}[P]) = 0$, the reader is
  invited to verify that $(\fprmatrix{P}{x}{P})(\state{P}) = x \cdot \state{P}$.
\end{remark}

\begin{remark}
  The reader is invited to verify that $\innerprod{P}{Q}$ could
  equally well have been written $\quotep{\int \stackrel{\vee}{x}}$
  where $x = \event{P^{\underline{\perp}}}(Q)$.

  One of the motivations for this remark is that there is another way
  to factor these operations. We could package up evaluation in the dual:

  \begin{mathpar}
    \state{P}^{*} := \event{\int P^{\underline{\perp}}} := \quotep{\int P^{\underline{\perp}}}[-]
  \end{mathpar}

  and then have inner product defined by
  
  \begin{mathpar}
    \innerprod{P}{Q} := \event{P}(Q)
  \end{mathpar}

  Hopefully, experience with the calculations will provide guidance on
  the best factoring.
\end{remark}

\begin{remark}
  Assuming $\int (P^{\underline{\perp}}[P]) = 0$, the reader is
  invited to verify that $\forall P,Q. (\prmatrix{0}{Q})(\state{0}) =
  \state{0}$ and dually $(\prmatrix{P}{0})(\event{0}) = \event{0}$.
\end{remark}

\begin{remark}
  i'm a little worried that i don't (yet) have proper support for
  complex conjugacy. But, the observation above may give us a
  clue. According to Abramsky, it must be the case that the scalars
  are iso to the homset of the identity for the tensor -- which the
  observation above characterizes. 

  For now, we will simply bookmark the notion with $\overline{x}$.
\end{remark}

\subsubsection{Adjointness}

We need to give a definition of $(\cdot)^{\dagger}$ for matrices. The
obvious candidate definition is
\begin{mathpar}
(\Sigma_{\alpha}\fprmatrix{P_{\alpha}}{x_{\alpha}}{Q_{\alpha}})^{\dagger}
= \Sigma_{\alpha}\fprmatrix{(Q_{\alpha}^{\underline{\perp}})^{*}}{\overline{x}_{\alpha}}{P_{\alpha}^{\underline{\perp}}} 
\end{mathpar}

But, $(Q_{\alpha}^{\underline{\perp}})^{*}$ requires a name along
which to communicate the process to achieve the context application.

\subsubsection{Basis for a basis}
If processes label states and ``addition'' of states (a.k.a. vector
addition) is interpreted as parallel composition, what corresponds to
notions of linear independence and basis? Here, we recall that Yoshida
has developed a set of \emph{combinators} for an asynchronous verison
of Milner's $\pi$-calculus. These are a finite set of processes such
any process can be expressed as parallel composition of these
combinators together with liberal uses of the new operator and
replication. We can simply give a translation of these into the
present calculus and have reasonable expectation that the property
carries over. That is, that the resultant set allows to express all
processes via parallel composition. Note, however, that there is no
new operator or replication in this calculus. As a result, we expect
that the corresponding set is actually infinite. That is, we expect
that the space is actually infinite dimensional.

\begin{remark}
  The attentive reader may be a bit concerned. Certainly, the
  collection $S$, $K$ and $I$ is a finite set of
  combinators. Shouldn't we expect to see a finite set of combinators
  for an effectively equivalent system? i am very sympathetic to this
  critique and feel it warrants full attention. On the other hand, i
  also have in mind the following analogy. The natural numbers, as a
  monoid under addition, has exactly $1$ generator, while the natural
  numbers, as a monoid under multiplication, has countably many
  generators (the primes). We observe that the application of the
  lambda calculus is much less resource sensitive than the parallel
  composition of the $\pi$-calculus. Could it be the case that we have
  an analogy of the form
  
  \begin{mathpar}
    m + n : MN :: m*n : M|N
  \end{mathpar}

  giving a similar blow up in the set of ``primes''?  This is such a
  wonderful thought that, even if it's not true, i think it's worth
  writing down.
\end{remark}
 

\documentclass[12pt]{llncs}
%\documentclass{jktr}

\usepackage[pdftex]{hyperref}                   
\usepackage {listings}
\usepackage {mathpartir}
\usepackage{bcprules}
%\usepackage{listings}
                       
\usepackage{graphicx} 
%\usepackage[margins=2.5cm,nohead,nofoot]{geometry}
%\usepackage{geometry}
\usepackage{amsfonts}
\usepackage{amstext}
\usepackage{latexsym}
\usepackage{amssymb}
\usepackage{color}


%\include{myPreamble}
\include{qm2pi.local} 

%\ifpdf
%\usepackage[pdftex]{graphicx}
%\else
%\usepackage{graphicx}
%\fi

 % \ifpdf
%  \usepackage{pdfsync}
%  \if


%\title{Brief Article}
%\author{David F. Snyder}
%\author{L.G. Meredith}

%\address{Dept. of Math., Texas State University--San Marcos, San Marcos, TX 78666}
       
\pagestyle{empty}


\begin{document}

\lstset{language=[Objective]Caml,frame=shadowbox}

\input{qm2pi.front}

% section front matter (end)

\input{qm2pi.intro} 
 
% section introduction (end)

% \input{qm2pi.knotations} 

% section notation (end)

\input{qm2pi.process.calculi} 

% section concurrent_process_calculi_and_spatial_logics_ (end)
    
%\input{qm2pi.knots2pi} 

%\input{qm2pi.trefoil} 

%\input{qm2pi.mainthm} 

% subsection basic_interpretation (end)

%\input{qm2pi.rho.presentation} 
\subsection{The syntax and semantics of the notation system}\label{sub:the_syntax_and_semantics_of_the_notation_system} % (fold)

We now summarize a technical presentation of the calculus that
embodies our theory of dynamics. The typical presentation of such a
calculus follows the style of giving generators and relations on
them. The grammar, below, describing term constructors, freely
generates the set of processes, $\Proc$. This set is then quotiented
by a relation known as structural congruence and it is over this set
that the notion of dynamics is expressed. This presentation is
essentially that of \cite{MeredithR05} with the addition of
polyadicity and summation. For readability we have relegated some of
the technical subtleties to an appendix.

\subsubsection{Process grammar}\label{subsub:process_grammar}

\begin{mathpar}
  \inferrule* [lab=synchronization] {} {{M} \bc \pzero \;|\; x?F \;|\; x!C }
  \and
  \inferrule* [lab=abstraction] {} {{F} \bc (x)P}
  \and
  \inferrule* [lab=concretion] {} {{C} \bc \langle Q \rangle}
  \and
  \inferrule* [lab=process] {} {{P,Q} \bc M \;| \;P|Q \;|\; @{x}}
  \and
  \inferrule* [lab=name] {} {{x} \bc \quotep{P}}
\end{mathpar} 

Note that $\vec{x}$ (resp. $\vec{P}$) denotes a vector of names
(resp. processes) of length $|\vec{x}|$ (resp. $|\vec{P}|$). We adopt
the following useful abbreviations.

\begin{mathpar}
   x?(\vec{y}).P := x.(\vec{y})P \and  x\clift{\vec{P}} := x.\clift{\vec{P}}
   \and x!(y) := \lift{x}{\dropn{y}}
   \and \Pi_{i=0}^{n-1}P_i := P_0 | \ldots | P_{n-1}
\end{mathpar}

\subsubsection{Structural congruence}

\paragraph{Free and bound names and alpha-equivalence.} At the
core of structural equivalence is alpha-equivalence which identifies
process that are the same up to a change of variable. Formally, we
recognize the distinction between free and bound names. The free names
of a process, $\freenames{P}$, may be calculated recursively as
follows:

\begin{mathpar}
\freenames{\pzero} := \emptyset
  \and \\
  \freenames{x?(y).P} := \{ x \} \cup (\freenames{P} \setminus \{ y \})
  \and 
  \freenames{x!\langle P \rangle} := \{ x \} \cup \{ P \} 
  \and \\
  \freenames{P|Q} := \freenames{P} \cup \freenames{Q}
  \and \\
  \freenames{@{x}} := \{ x \}
\end{mathpar}

$\pi$
$\quotep{\pi}$

$\freenames{-} : \pi \to \mathcal{P}(\quotep{\pi})$

\begin{eqnarray*}
  \freenames{\pzero} & := & \emptyset \\
  \freenames{x?(y).P} & := & \{ x \} \cup (\freenames{P} \setminus \{ y \}) \\
  \freenames{x!\langle P \rangle} & := & \{ x \} \cup \{ P \} \\
  \freenames{P|Q} & := & \freenames{P} \cup \freenames{Q} \\
  \freenames{\dropn{x}} & := & \{ x \}
\end{eqnarray*}

The bound names of a process, $\boundnames{P}$, are those names occurring in $P$
that are not free. For example, in $x?(y).0$, the name $x$ is free, while $y$ is bound.

\begin{mathpar}
  \inferrule* [lab=monoidal-laws] {} { P|Q \equiv Q|P \and P|0 \equiv P \and P|(Q|R) \equiv (P|Q)|R }
\end{mathpar}

\begin{mathpar}
  \inferrule* [lab=alpha-equivalence] {} { (x)P \equiv (y)P\{y/x\} \and y \not\in \freenames{P} }
\end{mathpar}

\begin{definition}
Then two processes, $P,Q$, are alpha-equivalent if $P = Q\{\vec{y}/\vec{x}\}$ for
some $\vec{x} \in \boundnames{Q},\vec{y} \in \boundnames{P}$, where $Q\{\vec{y}/\vec{x}\}$
denotes the capture-avoiding substitution of $\vec{y}$ for $\vec{x}$ in $Q$.
\end{definition}

\begin{definition}
  The {\em structural congruence} \cite{SangiorgiWalker} , $\equiv$,
  between processes is the least congruence containing
  alpha-equivalence, satisfying the abelian monoid laws
  (associativity, commutativity and $\pzero$ as identity) for parallel
  composition $|$ and for summation $+$.
\end{definition}

\subsection{Name equivalence}

We take name equivalence, written $\nameeq$, to be the smallest
equivalence relation generated by the following rules.

\begin{mathpar}
\inferrule*[lab=Quote-drop]
{ }
{ \quotep{@{x}} \nameeq x }

\inferrule*[lab=Struct-equiv]
{ P \scong Q }
{ \quotep{P} \nameeq \quotep{Q} }
\end{mathpar}

The astute reader will have noticed that the mutual recursion of names
and processes imposes a mutual recursion on alpha-equivalence and
structural equivalence via name-equivalence. Fortunately, all of this
works out pleasantly and we may calculate in the natural way, free of
concern. The reader interested in the details is referred to the
appendix \ref{appendix:rho_details}.

\subsection{Substitution}

We use $\Proc$ for the set of processes, $\QProc$ for the set of
names, and $\id{\{}\vec{y} / \vec{x} \id{\}}$ to denote partial maps,
$s : \QProc \rightarrow \QProc$. A map, $s$ lifts, uniquely, to a map
on process terms, $\widehat{s} : \Proc \rightarrow \Proc$ by the
following equations.

\begin{mathpar}
  (0) \psubstp{Q}{P} := 0 \\
  (R \juxtap S) \psubstp{Q}{P}
  :=    
  (R)\psubstp{Q}{P} \juxtap (S) \psubstp{Q}{P} \\
  (x?(y).R) \psubstp{Q}{P}    
  :=    
  (x)\substp{Q}{P} (z)\concat( (R \psubstn{z}{y}) \psubstp{Q}{P} ) \\
  (\lift{x}{R}) \psubstp{Q}{P}  
  :=
  \lift{(x)\substp{Q}{P}}{ R \psubstp{Q}{P} } \\
%   (\dropn{x})  \psubstp{Q}{P}       
%   := 
%   \left\{ 
%     \begin{array}{ccc} 
%       \dropn{\quotep{Q}} & & x \nameeq \quotep{P} \\
%       \dropn{x} & & otherwise \\
%     \end{array}
%   \right. 
  (\dropn{x})  \psubstp{Q}{P}       
  := 
  \left\{ 
    \begin{array}{ccc} 
      Q & & x \nameeq \quotep{P} \\
      \dropn{x} & & otherwise \\
    \end{array}
  \right.
\end{mathpar}
 

where

\begin{eqnarray}
  (x)\id{\{} \lpquote Q \rpquote / \lpquote P \rpquote \id{\}}            = 
  \left\{ 
    \begin{array}{ccc}
      \lpquote Q \rpquote & & x \nameeq \lpquote P \rpquote \\
      x & & otherwise \\
    \end{array}
  \right. \nonumber
\end{eqnarray}

and $z$ is chosen distinct from $\quotep{P}$, $\quotep{Q}$, the free
names in $Q$, and all the names in $R$. Our $\alpha$-equivalence will
be built in the standard way from this substitution.

\begin{remark}\label{rem:no_self_referential_names}
  One consequence of these definitions is that $\forall P. \quotep{P}
  \not\in \freenames{P}$.
\end{remark}

\subsection{ Dynamic quote: an example }

Anticipating something of what's to come, consider applying the
substitution, $\widehat{\id{\{}u / z \id{\}}}$, to the following pair
of processes, $\lift{w}{y!(z)}$ and $w[ \lpquote y!(z) \rpquote ]$.

\begin{eqnarray}
	\lift{w}{y!(z)}\widehat{\id{\{}u / z \id{\}}}
		& = &
		\lift{w}{y!(u)} \nonumber\\
	w[ \lpquote y!(z) \rpquote ] \widehat{ \id{\{}u / z \id{\}} }
		& = &
		w[ \lpquote y!(z) \rpquote ] \nonumber
\end{eqnarray}

Because the body of the process between quotes is impervious to
substitution, we get radically different answers. In fact, by
examining the first process in an input context,
e.g. $x?(z).\lift{w}{y!(z)}$, we see that the process under the lift
operator may be shaped by prefixed inputs binding a name inside it. In
this sense, the lift operator will be seen as a way to dynamically
construct processes before reifying them as names.

Finally equipped with these standard features we can present the
dynamics of the calculus.

\subsubsection{Operational semantics} 

Finally, we introduce the computational dynamics. What marks these
algebras as distinct from other more traditionally studied algebraic
structures, e.g. vector spaces or polynomial rings, is the manner in
which dynamics is captured. In traditional structures, dynamics is typically
expressed through morphisms between such structures, as in linear maps
between vector spaces or morphisms between rings. In algebras
associated with the semantics of computation, the dynamics is
expressed as part of the algebraic structure itself, through a
reduction reduction relation typically denoted by $\red$. Below, we
give a recursive presentation of this relation for the calculus used
in the encoding.

$\red \subseteq \pi \times \pi$
$\red : \pi \to \mathcal{P}(\pi)$

\begin{mathpar}
  \inferrule* [lab=Comm] { \textsf{match}( x_{src}, x_{trgt} ) } { x_{trgt}?(y)P \; | \; x_{src}!\langle {Q} \rangle \red P\{\quotep{Q}/y}\} }
  \and \\
  \inferrule* [lab=Par] {{P} \red {P}'} {{{P} | {Q}} \red {{P}' | {Q}}}
  \and
  \inferrule* [lab=Equiv]{{{P} \scong {P}'} \andalso {{P}' \red {Q}'} \andalso {{Q}' \scong {Q}}}{{P} \red {Q}}
\end{mathpar}

\begin{eqnarray*}
  match_{\equiv} (\quotep{P},\quotep{Q}) & := & P \equiv Q \\
  match_{\dagger}(\quotep{P},\quotep{Q}) & := & \forall R. P|Q \red^{*} R => R \red^{*} 0 \\
  match_{K}(\quotep{P},\quotep{Q}) & := & K \mbox{ for some context } K
\end{eqnarray*}

$u?(x)P | u!\langle Q \rangle \red P\{\quotep{Q}/x\}$

%We write $\wred$ for $\red^*$, and $P\red$ if $\exists Q $ such that $ P \red Q$.
We write $P\red$ if $\exists Q $ such that $ P \red Q$ and $P\not\red$, otherwise.

\section{Replication}

As mentioned before, it is known that replication (and hence
recursion) can be implemented in a higher-order process algebra
\cite{SangiorgiWalker}. As our first example of calculation with the
machinery thus far presented we give the construction explicitly in
the {\rhoc}.

\begin{eqnarray}
	D_{x} & := & \prefix{x}{y}{(\binpar{\outputp{x}{y}}{@{y}})} \nonumber\\
	\bangp_{x}{P} & := & \binpar{{x}!\langle{\binpar{D_{x}}{P}}\rangle}{D_{x}} \nonumber
\end{eqnarray}

\begin{eqnarray}
	\bangp_{x}{P} & & \nonumber\\
	=
	& {x}!\langle{(\prefix{x}{y}{(\outputp{x}{y} | @{y})) | P}}\rangle 
	      | \prefix{x}{y}{(\outputp{x}{y} | @{y})} & \nonumber\\
	\red
	& (\outputp{x}{y} | @{y})\substn{\quotep{(\prefix{x}{y}{(@{y} | \outputp{x}{y})) | P}}}{y} & \nonumber\\
	=
	& \outputp{x}{\quotep{(\prefix{x}{y}{(\outputp{x}{y} | @{y})) | P}}}
	  | {(\prefix{x}{y}{(\outputp{x}{y} | @{y})) | P}} & \nonumber\\
	\red
	& \ldots & \nonumber\\
	\red^*
	& P | P | \ldots & \nonumber
\end{eqnarray}

Of course, this encoding, as an implementation, runs away, unfolding
$\bangp{P}$ eagerly. A lazier and more implementable replication
operator, restricted to input-guarded processes, may be obtained as follows.

\begin{eqnarray}
\bangp{\prefix{u}{v}{P}} 
	:= 
	\binpar{\lift{x}{\prefix{u}{v}{(\binpar{D(x)}{P})}}}{D(x)} \nonumber
\end{eqnarray}

\begin{remark}
  Note that the lazier definition still does not deal with summation
  or mixed summation (i.e. sums over input and output). The reader is
  invited to construct definitions of replication that deal with these
  features. 

  Further, the definitions are parameterized in a name, $x$. Can you,
  gentle reader, make a definition that eliminates this parameter and
  guarantees no accidental interaction between the replication
  machinery and the process being replicated -- i.e. no accidental
  sharing of names used by the process to get its work done and the
  name(s) used by the replication to effect copying. This latter
  revision of the definition of replication is crucial to obtaining
  the expected identity $!!P \sim !P$.
\end{remark}

\begin{remark}\label{rem:paradoxical_combinator}
  The reader familiar with the lambda calculus will have noticed the
  similarity between $D$ and the paradoxical combinator.

  [Ed. note: the existence of this seems to suggest we have to be more
  restrictive on the set of processes and names we admit if we are to
  support no-cloning.]
\end{remark}

\subsubsection{Bisimulation}

The computational dynamics gives rise to another kind of equivalence,
the equivalence of computational behavior. As previously mentioned
this is typically captured \emph{via} some form of bisimulation.

% The notion we use in this paper is weak barbed bisimulation
% \cite{milner91polyadicpi}.

The notion we use in this paper is derived from weak barbed
bisimulation \cite{milner91polyadicpi}. 

\begin{definition}
An \emph{observation relation}, $\downarrow_{\mathcal N}$, over a set
of names, $\mathcal N$, is the smallest relation satisfying the rules
below.

\infrule[Out-barb]{y \in {\mathcal N}, \; x \nameeq y}
		  {\outputp{x}{v} \downarrow_{\mathcal N} x}
\infrule[Par-barb]{\mbox{$P\downarrow_{\mathcal N} x$ or $Q\downarrow_{\mathcal N} x$}}
		  {\binpar{P}{Q} \downarrow_{\mathcal N} x}

We write $P \Downarrow_{\mathcal N} x$ if there is $Q$ such that 
$P \wred Q$ and $Q \downarrow_{\mathcal N} x$.
\end{definition}

\begin{definition}
%\label{def.bbisim}
An  ${\mathcal N}$-\emph{barbed bisimulation} over a set of names, ${\mathcal N}$, is a symmetric binary relation 
${\mathcal S}_{\mathcal N}$ between agents such that $P\rel{S}_{\mathcal N}Q$ implies:
\begin{enumerate}
\item If $P \red P'$ then $Q \wred Q'$ and $P'\rel{S}_{\mathcal N} Q'$.
\item If $P\downarrow_{\mathcal N} x$, then $Q\Downarrow_{\mathcal N} x$.
\end{enumerate}
$P$ is ${\mathcal N}$-barbed bisimilar to $Q$, written
$P \wbbisim_{\mathcal N} Q$, if $P \rel{S}_{\mathcal N} Q$ for some ${\mathcal N}$-barbed bisimulation ${\mathcal S}_{\mathcal N}$.
\end{definition}

$\mathcal{R} \subseteq \pi \times \pi$

$P \mathcal{R} Q => \forall P'. P \red P' \Rightarrow \exists Q'. Q \red Q', P' \mathcal{R} Q'$

$P \vdash x \Rightarrow Q \vdash x$

\begin{mathpar}
  \inferrule*[lab=Out-barb]{x \nameeq y}{{y}!\langle{Q}\rangle \vdash x}
  \and
  \inferrule*[lab=Par-barb]{\mbox{$P\vdash x$ or $Q\vdash x$}}{\binpar{P}{Q} \vdash x}
\end{mathpar}

\subsubsection{Contexts}

One of the principle advantages of computational calculi like the
$\pi$-calculus is a well-defined notion of context,
contextual-equivalence and a correlation between
contextual-equivalence and notions of bisimulation. The notion of
context allows the decomposition of a process into (sub-)process and
its syntactic environment, its context. Thus, a context may be
thought of as a process with a ``hole'' (written $\Box$) in it. The
application of a context $M$ to a process $P$, written $M[P]$, is
tantamount to filling the hole in $M$ with $P$. In this paper we do
not need the full weight of this theory, but do make use of the notion
of context in the proof the main theorem. 

\begin{mathpar}
  \inferrule* [lab=summation] {} {{M_{M},M_{N}} \bc \Box \;|\; x.M_{A} \;|\; M_{M}+M_{N}}
  \and
  \inferrule* [lab=agent] {} {{M_{A}} \bc (\vec{x})M_{P} \;| \; \clift{P_0,\ldots,M_{P},\ldots,P_N}}
  \and \\
  \inferrule* [lab=process] {} {{M_{P}} \bc M_{N} \;| \;P|M_{P} }
\end{mathpar} 

\begin{mathpar}
  \inferrule* [lab=sychronization] {} {M_{N} \bc \Box \;|\; x?M_{F} \;|\; x!M_{C}}
  \and
  \inferrule* [lab=abstraction] {} {{M_{F}} \bc (x)M_{P} }
  \and
  \inferrule* [lab=concretion] {} {{M_{C}} \bc \langle M_{P} \rangle }
  \and \\
  \inferrule* [lab=process] {} {{M_{P}} \bc M_{N} \;| \;P|M_{P} }
\end{mathpar}

\begin{definition}[contextual application] Given a context $M$, and
  process $P$, we define the \emph{contextual application}, $M[P] :=
  M\{P/\Box\}$. That is, the contextual application of M to P is the
  substitution of $P$ for $\Box$ in $M$.
\end{definition}

$\meaningof{-} : L \to \mathcal{P}(\pi)$

\begin{mathpar}
  \inferrule* [lab=collection] {} {\meaningof{true} = \pi, \and \meaningof{~E} = \pi \setminus \meaningof{E}, \and \meaningof{E_{1} \& E_{2}} = \meaningof{E_{1}} \cap \meaningof{E_{2}}}
\end{mathpar}

\begin{mathpar}
  \inferrule* [lab=structure] {} {\meaningof{0} = \{ P \in \pi | P \equiv 0 \}, \and \\ \meaningof{E_1 | E_2} = \{ P \in \pi | P \equiv P_{1} | P_{2}, P_{1} \in \meaningof{E_{1}}, P_{2} \in \meaningof{E_2}\} }
\end{mathpar}

\begin{mathpar}
 \inferrule* [lab=behavior] {} {\meaningof{\langle a?b \rangle E} = \{ P \in \pi | P \equiv Q | u?(y)P', \\ \and \\\\ \and \\ \;\;\; u \in \meaningof{a}, \forall z.P'\{z/y\} \in \meaningof{E\{z/b\}}\}, \and \\ \meaningof{a!E} = \{ P \in \pi | P \equiv Q | x!\langle P' \rangle, x \in \meaningof{a} P' \in \meaningof{E}\} }
\end{mathpar}

\begin{mathpar}
 \inferrule* [lab=nominal] {} {\meaningof{\quotep{E}} = \{ \quotep{P} \in \quotep{\pi} | P \in \meaningof{E} \}, \and \meaningof{\quotep{P}} = \{ \quotep{Q} \in \quotep{\pi} | P \equiv Q \} \and \\ \meaningof{@\quotep{E}} = \{ P \in \pi | P \equiv @x, x \in \meaningof{E} \}}
\end{mathpar}

\begin{eqnarray*}
  \\
  \meaningof{-} : TS \to ST
\end{eqnarray*}

\begin{eqnarray*}
  \\
  L : TS \to ST
\end{eqnarray*}

\begin{eqnarray*}
  \\
  P \models E \iff P \in \meaningof{E}
\end{eqnarray*}

\begin{eqnarray*}
  P \approx_{L} Q \iff \forall E \in L. P \models E \iff Q \models E
\end{eqnarray*}

\begin{eqnarray*}
  P \approx_{K} Q
\end{eqnarray*}

\begin{eqnarray*}
  P \approx Q
\end{eqnarray*}

$\approx_{K} = \approx = \approx_{L}$

\subsubsection{Contextual duality}

Note that contexts extend the quotation operation to a family of
operations from processes to names. Given a context, $M$, we can
define a \emph{nominal context}, $\quotep{M}$ by $\quotep{M}[P] :=
\quotep{M[P]}$. To foreshadow what is to come we observe that these
operations enjoy a duality with processes very much like the duality
between vectors and maps from vectors to scalars.

Further, because the calculus is essentially higher-order, we have a
correspondence between contexts and processes. More specifically,
given a name $x$ and a context $M$ we can construct $M^{*}_{x}$ such
that 

\begin{mathpar}
  M^{*}_{x} | \lift{x}{P} \red M[P]
\end{mathpar}

namely,

\begin{mathpar}
  M^{*}_{x} := x?(u).M[\dropn{u}]
\end{mathpar}

The dependence of $M^{*}_{x}$ on a name makes it an abstraction, 

\begin{mathpar}
  M^{*} := (x)x?(u).M[\dropn{u}]
\end{mathpar}

\subsection{Additional notation}

It will sometimes be convenient to denote the process a name
quotes. We already have the notation $x = \quotep{P}$, but it will be
convenient to introduce an alternate notation, $\procn{x}$, when we
want to emphasize the connection to the use of the name. Note that, by
virtue of name equivalence, $\quotep{\procn{x}} \nameeq x$; so, the
notation is consistent with previous definitions.

Further, because names have structure it is possible to effect
substitutions on the basis of that structure. This means we need to
upgrade our notation for substitutions, which we accomplish by
adapting comprehension notation. Thus,

\begin{mathpar}
  P\{ y / x : x \in S \}
\end{mathpar}

is interpreted to mean the process derived from P by replacing (in a
capture-avoiding manner) each occurrence of $x$ in $S$ by $y$. For example,

\begin{mathpar}
  P\{ \quotep{\procn{x}|\procn{x}} / x : x \in \freenames{P} \}
\end{mathpar}

will replace each (occurrence) of a free name $x$ in $P$ by
$\quotep{\procn{x}|\procn{x}}$.

Also, we will avail ourselves of the notation $x^{L}$ and $x^{R}$ to
denote injections of a name into disjoint copies of the name
space. There are numerous ways to accomplish this. One example can be
found in \cite{MeredithR05}. This notation overloads to vectors of
names: $\vec{x}^{\pi} := (x_{i}^{\pi} \; : \; 0 \leq i < |\vec{x}| )$ where $\pi \in \{L,R\}$.

We also use $P^{\Box} := P|\Box$.

In \cite{MeredithR05} an interpretation of the new operator is
given. It turns out that there are several possible interpretations
all enjoying the requisite algebraic properties of the operator (see
\cite{milner91polyadicpi}). We will therefore make liberal use of
$(\nu\; \vec{x})P$.

% subsection the_syntax_and_semantics_of_the_notation_system (end)   

\input{qm2pi.qmops} 

\input{qm2pi.sterngerlach} 

\input{qm2pi.metric} 

% section concurrent_process_calculi (end)

%\input{qm2pi.proofsketch}

% section proof sketch (end)

%\input{qm2pi.slviaknots} 

% section spatial logic via knots (end)

\input{qm2pi.conclusion}

% section conclusion (end)

%\input{qm2pi.dtcodes} 

% section wiring algorithm (end)

\input{qm2pi.ack} 

% section acknowledgments (end)

\newpage


\bibliographystyle{plain}   
\bibliography{../../biblios/main.bib}

\input{qm2pi.rhodetails}

\end{document}

 

\documentclass[12pt]{llncs}
%\documentclass{jktr}

\usepackage[pdftex]{hyperref}                   
\usepackage {listings}
\usepackage {mathpartir}
\usepackage{bcprules}
%\usepackage{listings}
                       
\usepackage{graphicx} 
%\usepackage[margins=2.5cm,nohead,nofoot]{geometry}
%\usepackage{geometry}
\usepackage{amsfonts}
\usepackage{amstext}
\usepackage{latexsym}
\usepackage{amssymb}
\usepackage{color}


%\include{myPreamble}
\include{qm2pi.local} 

%\ifpdf
%\usepackage[pdftex]{graphicx}
%\else
%\usepackage{graphicx}
%\fi

 % \ifpdf
%  \usepackage{pdfsync}
%  \if


%\title{Brief Article}
%\author{David F. Snyder}
%\author{L.G. Meredith}

%\address{Dept. of Math., Texas State University--San Marcos, San Marcos, TX 78666}
       
\pagestyle{empty}


\begin{document}

\lstset{language=[Objective]Caml,frame=shadowbox}

\input{qm2pi.front}

% section front matter (end)

\input{qm2pi.intro} 
 
% section introduction (end)

% \input{qm2pi.knotations} 

% section notation (end)

\input{qm2pi.process.calculi} 

% section concurrent_process_calculi_and_spatial_logics_ (end)
    
%\input{qm2pi.knots2pi} 

%\input{qm2pi.trefoil} 

%\input{qm2pi.mainthm} 

% subsection basic_interpretation (end)

%\input{qm2pi.rho.presentation} 
\subsection{The syntax and semantics of the notation system}\label{sub:the_syntax_and_semantics_of_the_notation_system} % (fold)

We now summarize a technical presentation of the calculus that
embodies our theory of dynamics. The typical presentation of such a
calculus follows the style of giving generators and relations on
them. The grammar, below, describing term constructors, freely
generates the set of processes, $\Proc$. This set is then quotiented
by a relation known as structural congruence and it is over this set
that the notion of dynamics is expressed. This presentation is
essentially that of \cite{MeredithR05} with the addition of
polyadicity and summation. For readability we have relegated some of
the technical subtleties to an appendix.

\subsubsection{Process grammar}\label{subsub:process_grammar}

\begin{mathpar}
  \inferrule* [lab=synchronization] {} {{M} \bc \pzero \;|\; x?F \;|\; x!C }
  \and
  \inferrule* [lab=abstraction] {} {{F} \bc (x)P}
  \and
  \inferrule* [lab=concretion] {} {{C} \bc \langle Q \rangle}
  \and
  \inferrule* [lab=process] {} {{P,Q} \bc M \;| \;P|Q \;|\; @{x}}
  \and
  \inferrule* [lab=name] {} {{x} \bc \quotep{P}}
\end{mathpar} 

Note that $\vec{x}$ (resp. $\vec{P}$) denotes a vector of names
(resp. processes) of length $|\vec{x}|$ (resp. $|\vec{P}|$). We adopt
the following useful abbreviations.

\begin{mathpar}
   x?(\vec{y}).P := x.(\vec{y})P \and  x\clift{\vec{P}} := x.\clift{\vec{P}}
   \and x!(y) := \lift{x}{\dropn{y}}
   \and \Pi_{i=0}^{n-1}P_i := P_0 | \ldots | P_{n-1}
\end{mathpar}

\subsubsection{Structural congruence}

\paragraph{Free and bound names and alpha-equivalence.} At the
core of structural equivalence is alpha-equivalence which identifies
process that are the same up to a change of variable. Formally, we
recognize the distinction between free and bound names. The free names
of a process, $\freenames{P}$, may be calculated recursively as
follows:

\begin{mathpar}
\freenames{\pzero} := \emptyset
  \and \\
  \freenames{x?(y).P} := \{ x \} \cup (\freenames{P} \setminus \{ y \})
  \and 
  \freenames{x!\langle P \rangle} := \{ x \} \cup \{ P \} 
  \and \\
  \freenames{P|Q} := \freenames{P} \cup \freenames{Q}
  \and \\
  \freenames{@{x}} := \{ x \}
\end{mathpar}

$\pi$
$\quotep{\pi}$

$\freenames{-} : \pi \to \mathcal{P}(\quotep{\pi})$

\begin{eqnarray*}
  \freenames{\pzero} & := & \emptyset \\
  \freenames{x?(y).P} & := & \{ x \} \cup (\freenames{P} \setminus \{ y \}) \\
  \freenames{x!\langle P \rangle} & := & \{ x \} \cup \{ P \} \\
  \freenames{P|Q} & := & \freenames{P} \cup \freenames{Q} \\
  \freenames{\dropn{x}} & := & \{ x \}
\end{eqnarray*}

The bound names of a process, $\boundnames{P}$, are those names occurring in $P$
that are not free. For example, in $x?(y).0$, the name $x$ is free, while $y$ is bound.

\begin{mathpar}
  \inferrule* [lab=monoidal-laws] {} { P|Q \equiv Q|P \and P|0 \equiv P \and P|(Q|R) \equiv (P|Q)|R }
\end{mathpar}

\begin{mathpar}
  \inferrule* [lab=alpha-equivalence] {} { (x)P \equiv (y)P\{y/x\} \and y \not\in \freenames{P} }
\end{mathpar}

\begin{definition}
Then two processes, $P,Q$, are alpha-equivalent if $P = Q\{\vec{y}/\vec{x}\}$ for
some $\vec{x} \in \boundnames{Q},\vec{y} \in \boundnames{P}$, where $Q\{\vec{y}/\vec{x}\}$
denotes the capture-avoiding substitution of $\vec{y}$ for $\vec{x}$ in $Q$.
\end{definition}

\begin{definition}
  The {\em structural congruence} \cite{SangiorgiWalker} , $\equiv$,
  between processes is the least congruence containing
  alpha-equivalence, satisfying the abelian monoid laws
  (associativity, commutativity and $\pzero$ as identity) for parallel
  composition $|$ and for summation $+$.
\end{definition}

\subsection{Name equivalence}

We take name equivalence, written $\nameeq$, to be the smallest
equivalence relation generated by the following rules.

\begin{mathpar}
\inferrule*[lab=Quote-drop]
{ }
{ \quotep{@{x}} \nameeq x }

\inferrule*[lab=Struct-equiv]
{ P \scong Q }
{ \quotep{P} \nameeq \quotep{Q} }
\end{mathpar}

The astute reader will have noticed that the mutual recursion of names
and processes imposes a mutual recursion on alpha-equivalence and
structural equivalence via name-equivalence. Fortunately, all of this
works out pleasantly and we may calculate in the natural way, free of
concern. The reader interested in the details is referred to the
appendix \ref{appendix:rho_details}.

\subsection{Substitution}

We use $\Proc$ for the set of processes, $\QProc$ for the set of
names, and $\id{\{}\vec{y} / \vec{x} \id{\}}$ to denote partial maps,
$s : \QProc \rightarrow \QProc$. A map, $s$ lifts, uniquely, to a map
on process terms, $\widehat{s} : \Proc \rightarrow \Proc$ by the
following equations.

\begin{mathpar}
  (0) \psubstp{Q}{P} := 0 \\
  (R \juxtap S) \psubstp{Q}{P}
  :=    
  (R)\psubstp{Q}{P} \juxtap (S) \psubstp{Q}{P} \\
  (x?(y).R) \psubstp{Q}{P}    
  :=    
  (x)\substp{Q}{P} (z)\concat( (R \psubstn{z}{y}) \psubstp{Q}{P} ) \\
  (\lift{x}{R}) \psubstp{Q}{P}  
  :=
  \lift{(x)\substp{Q}{P}}{ R \psubstp{Q}{P} } \\
%   (\dropn{x})  \psubstp{Q}{P}       
%   := 
%   \left\{ 
%     \begin{array}{ccc} 
%       \dropn{\quotep{Q}} & & x \nameeq \quotep{P} \\
%       \dropn{x} & & otherwise \\
%     \end{array}
%   \right. 
  (\dropn{x})  \psubstp{Q}{P}       
  := 
  \left\{ 
    \begin{array}{ccc} 
      Q & & x \nameeq \quotep{P} \\
      \dropn{x} & & otherwise \\
    \end{array}
  \right.
\end{mathpar}
 

where

\begin{eqnarray}
  (x)\id{\{} \lpquote Q \rpquote / \lpquote P \rpquote \id{\}}            = 
  \left\{ 
    \begin{array}{ccc}
      \lpquote Q \rpquote & & x \nameeq \lpquote P \rpquote \\
      x & & otherwise \\
    \end{array}
  \right. \nonumber
\end{eqnarray}

and $z$ is chosen distinct from $\quotep{P}$, $\quotep{Q}$, the free
names in $Q$, and all the names in $R$. Our $\alpha$-equivalence will
be built in the standard way from this substitution.

\begin{remark}\label{rem:no_self_referential_names}
  One consequence of these definitions is that $\forall P. \quotep{P}
  \not\in \freenames{P}$.
\end{remark}

\subsection{ Dynamic quote: an example }

Anticipating something of what's to come, consider applying the
substitution, $\widehat{\id{\{}u / z \id{\}}}$, to the following pair
of processes, $\lift{w}{y!(z)}$ and $w[ \lpquote y!(z) \rpquote ]$.

\begin{eqnarray}
	\lift{w}{y!(z)}\widehat{\id{\{}u / z \id{\}}}
		& = &
		\lift{w}{y!(u)} \nonumber\\
	w[ \lpquote y!(z) \rpquote ] \widehat{ \id{\{}u / z \id{\}} }
		& = &
		w[ \lpquote y!(z) \rpquote ] \nonumber
\end{eqnarray}

Because the body of the process between quotes is impervious to
substitution, we get radically different answers. In fact, by
examining the first process in an input context,
e.g. $x?(z).\lift{w}{y!(z)}$, we see that the process under the lift
operator may be shaped by prefixed inputs binding a name inside it. In
this sense, the lift operator will be seen as a way to dynamically
construct processes before reifying them as names.

Finally equipped with these standard features we can present the
dynamics of the calculus.

\subsubsection{Operational semantics} 

Finally, we introduce the computational dynamics. What marks these
algebras as distinct from other more traditionally studied algebraic
structures, e.g. vector spaces or polynomial rings, is the manner in
which dynamics is captured. In traditional structures, dynamics is typically
expressed through morphisms between such structures, as in linear maps
between vector spaces or morphisms between rings. In algebras
associated with the semantics of computation, the dynamics is
expressed as part of the algebraic structure itself, through a
reduction reduction relation typically denoted by $\red$. Below, we
give a recursive presentation of this relation for the calculus used
in the encoding.

$\red \subseteq \pi \times \pi$
$\red : \pi \to \mathcal{P}(\pi)$

\begin{mathpar}
  \inferrule* [lab=Comm] { \textsf{match}( x_{src}, x_{trgt} ) } { x_{trgt}?(y)P \; | \; x_{src}!\langle {Q} \rangle \red P\{\quotep{Q}/y}\} }
  \and \\
  \inferrule* [lab=Par] {{P} \red {P}'} {{{P} | {Q}} \red {{P}' | {Q}}}
  \and
  \inferrule* [lab=Equiv]{{{P} \scong {P}'} \andalso {{P}' \red {Q}'} \andalso {{Q}' \scong {Q}}}{{P} \red {Q}}
\end{mathpar}

\begin{eqnarray*}
  match_{\equiv} (\quotep{P},\quotep{Q}) & := & P \equiv Q \\
  match_{\dagger}(\quotep{P},\quotep{Q}) & := & \forall R. P|Q \red^{*} R => R \red^{*} 0 \\
  match_{K}(\quotep{P},\quotep{Q}) & := & K \mbox{ for some context } K
\end{eqnarray*}

$u?(x)P | u!\langle Q \rangle \red P\{\quotep{Q}/x\}$

%We write $\wred$ for $\red^*$, and $P\red$ if $\exists Q $ such that $ P \red Q$.
We write $P\red$ if $\exists Q $ such that $ P \red Q$ and $P\not\red$, otherwise.

\section{Replication}

As mentioned before, it is known that replication (and hence
recursion) can be implemented in a higher-order process algebra
\cite{SangiorgiWalker}. As our first example of calculation with the
machinery thus far presented we give the construction explicitly in
the {\rhoc}.

\begin{eqnarray}
	D_{x} & := & \prefix{x}{y}{(\binpar{\outputp{x}{y}}{@{y}})} \nonumber\\
	\bangp_{x}{P} & := & \binpar{{x}!\langle{\binpar{D_{x}}{P}}\rangle}{D_{x}} \nonumber
\end{eqnarray}

\begin{eqnarray}
	\bangp_{x}{P} & & \nonumber\\
	=
	& {x}!\langle{(\prefix{x}{y}{(\outputp{x}{y} | @{y})) | P}}\rangle 
	      | \prefix{x}{y}{(\outputp{x}{y} | @{y})} & \nonumber\\
	\red
	& (\outputp{x}{y} | @{y})\substn{\quotep{(\prefix{x}{y}{(@{y} | \outputp{x}{y})) | P}}}{y} & \nonumber\\
	=
	& \outputp{x}{\quotep{(\prefix{x}{y}{(\outputp{x}{y} | @{y})) | P}}}
	  | {(\prefix{x}{y}{(\outputp{x}{y} | @{y})) | P}} & \nonumber\\
	\red
	& \ldots & \nonumber\\
	\red^*
	& P | P | \ldots & \nonumber
\end{eqnarray}

Of course, this encoding, as an implementation, runs away, unfolding
$\bangp{P}$ eagerly. A lazier and more implementable replication
operator, restricted to input-guarded processes, may be obtained as follows.

\begin{eqnarray}
\bangp{\prefix{u}{v}{P}} 
	:= 
	\binpar{\lift{x}{\prefix{u}{v}{(\binpar{D(x)}{P})}}}{D(x)} \nonumber
\end{eqnarray}

\begin{remark}
  Note that the lazier definition still does not deal with summation
  or mixed summation (i.e. sums over input and output). The reader is
  invited to construct definitions of replication that deal with these
  features. 

  Further, the definitions are parameterized in a name, $x$. Can you,
  gentle reader, make a definition that eliminates this parameter and
  guarantees no accidental interaction between the replication
  machinery and the process being replicated -- i.e. no accidental
  sharing of names used by the process to get its work done and the
  name(s) used by the replication to effect copying. This latter
  revision of the definition of replication is crucial to obtaining
  the expected identity $!!P \sim !P$.
\end{remark}

\begin{remark}\label{rem:paradoxical_combinator}
  The reader familiar with the lambda calculus will have noticed the
  similarity between $D$ and the paradoxical combinator.

  [Ed. note: the existence of this seems to suggest we have to be more
  restrictive on the set of processes and names we admit if we are to
  support no-cloning.]
\end{remark}

\subsubsection{Bisimulation}

The computational dynamics gives rise to another kind of equivalence,
the equivalence of computational behavior. As previously mentioned
this is typically captured \emph{via} some form of bisimulation.

% The notion we use in this paper is weak barbed bisimulation
% \cite{milner91polyadicpi}.

The notion we use in this paper is derived from weak barbed
bisimulation \cite{milner91polyadicpi}. 

\begin{definition}
An \emph{observation relation}, $\downarrow_{\mathcal N}$, over a set
of names, $\mathcal N$, is the smallest relation satisfying the rules
below.

\infrule[Out-barb]{y \in {\mathcal N}, \; x \nameeq y}
		  {\outputp{x}{v} \downarrow_{\mathcal N} x}
\infrule[Par-barb]{\mbox{$P\downarrow_{\mathcal N} x$ or $Q\downarrow_{\mathcal N} x$}}
		  {\binpar{P}{Q} \downarrow_{\mathcal N} x}

We write $P \Downarrow_{\mathcal N} x$ if there is $Q$ such that 
$P \wred Q$ and $Q \downarrow_{\mathcal N} x$.
\end{definition}

\begin{definition}
%\label{def.bbisim}
An  ${\mathcal N}$-\emph{barbed bisimulation} over a set of names, ${\mathcal N}$, is a symmetric binary relation 
${\mathcal S}_{\mathcal N}$ between agents such that $P\rel{S}_{\mathcal N}Q$ implies:
\begin{enumerate}
\item If $P \red P'$ then $Q \wred Q'$ and $P'\rel{S}_{\mathcal N} Q'$.
\item If $P\downarrow_{\mathcal N} x$, then $Q\Downarrow_{\mathcal N} x$.
\end{enumerate}
$P$ is ${\mathcal N}$-barbed bisimilar to $Q$, written
$P \wbbisim_{\mathcal N} Q$, if $P \rel{S}_{\mathcal N} Q$ for some ${\mathcal N}$-barbed bisimulation ${\mathcal S}_{\mathcal N}$.
\end{definition}

$\mathcal{R} \subseteq \pi \times \pi$

$P \mathcal{R} Q => \forall P'. P \red P' \Rightarrow \exists Q'. Q \red Q', P' \mathcal{R} Q'$

$P \vdash x \Rightarrow Q \vdash x$

\begin{mathpar}
  \inferrule*[lab=Out-barb]{x \nameeq y}{{y}!\langle{Q}\rangle \vdash x}
  \and
  \inferrule*[lab=Par-barb]{\mbox{$P\vdash x$ or $Q\vdash x$}}{\binpar{P}{Q} \vdash x}
\end{mathpar}

\subsubsection{Contexts}

One of the principle advantages of computational calculi like the
$\pi$-calculus is a well-defined notion of context,
contextual-equivalence and a correlation between
contextual-equivalence and notions of bisimulation. The notion of
context allows the decomposition of a process into (sub-)process and
its syntactic environment, its context. Thus, a context may be
thought of as a process with a ``hole'' (written $\Box$) in it. The
application of a context $M$ to a process $P$, written $M[P]$, is
tantamount to filling the hole in $M$ with $P$. In this paper we do
not need the full weight of this theory, but do make use of the notion
of context in the proof the main theorem. 

\begin{mathpar}
  \inferrule* [lab=summation] {} {{M_{M},M_{N}} \bc \Box \;|\; x.M_{A} \;|\; M_{M}+M_{N}}
  \and
  \inferrule* [lab=agent] {} {{M_{A}} \bc (\vec{x})M_{P} \;| \; \clift{P_0,\ldots,M_{P},\ldots,P_N}}
  \and \\
  \inferrule* [lab=process] {} {{M_{P}} \bc M_{N} \;| \;P|M_{P} }
\end{mathpar} 

\begin{mathpar}
  \inferrule* [lab=sychronization] {} {M_{N} \bc \Box \;|\; x?M_{F} \;|\; x!M_{C}}
  \and
  \inferrule* [lab=abstraction] {} {{M_{F}} \bc (x)M_{P} }
  \and
  \inferrule* [lab=concretion] {} {{M_{C}} \bc \langle M_{P} \rangle }
  \and \\
  \inferrule* [lab=process] {} {{M_{P}} \bc M_{N} \;| \;P|M_{P} }
\end{mathpar}

\begin{definition}[contextual application] Given a context $M$, and
  process $P$, we define the \emph{contextual application}, $M[P] :=
  M\{P/\Box\}$. That is, the contextual application of M to P is the
  substitution of $P$ for $\Box$ in $M$.
\end{definition}

$\meaningof{-} : L \to \mathcal{P}(\pi)$

\begin{mathpar}
  \inferrule* [lab=collection] {} {\meaningof{true} = \pi, \and \meaningof{~E} = \pi \setminus \meaningof{E}, \and \meaningof{E_{1} \& E_{2}} = \meaningof{E_{1}} \cap \meaningof{E_{2}}}
\end{mathpar}

\begin{mathpar}
  \inferrule* [lab=structure] {} {\meaningof{0} = \{ P \in \pi | P \equiv 0 \}, \and \\ \meaningof{E_1 | E_2} = \{ P \in \pi | P \equiv P_{1} | P_{2}, P_{1} \in \meaningof{E_{1}}, P_{2} \in \meaningof{E_2}\} }
\end{mathpar}

\begin{mathpar}
 \inferrule* [lab=behavior] {} {\meaningof{\langle a?b \rangle E} = \{ P \in \pi | P \equiv Q | u?(y)P', \\ \and \\\\ \and \\ \;\;\; u \in \meaningof{a}, \forall z.P'\{z/y\} \in \meaningof{E\{z/b\}}\}, \and \\ \meaningof{a!E} = \{ P \in \pi | P \equiv Q | x!\langle P' \rangle, x \in \meaningof{a} P' \in \meaningof{E}\} }
\end{mathpar}

\begin{mathpar}
 \inferrule* [lab=nominal] {} {\meaningof{\quotep{E}} = \{ \quotep{P} \in \quotep{\pi} | P \in \meaningof{E} \}, \and \meaningof{\quotep{P}} = \{ \quotep{Q} \in \quotep{\pi} | P \equiv Q \} \and \\ \meaningof{@\quotep{E}} = \{ P \in \pi | P \equiv @x, x \in \meaningof{E} \}}
\end{mathpar}

\begin{eqnarray*}
  \\
  \meaningof{-} : TS \to ST
\end{eqnarray*}

\begin{eqnarray*}
  \\
  L : TS \to ST
\end{eqnarray*}

\begin{eqnarray*}
  \\
  P \models E \iff P \in \meaningof{E}
\end{eqnarray*}

\begin{eqnarray*}
  P \approx_{L} Q \iff \forall E \in L. P \models E \iff Q \models E
\end{eqnarray*}

\begin{eqnarray*}
  P \approx_{K} Q
\end{eqnarray*}

\begin{eqnarray*}
  P \approx Q
\end{eqnarray*}

$\approx_{K} = \approx = \approx_{L}$

\subsubsection{Contextual duality}

Note that contexts extend the quotation operation to a family of
operations from processes to names. Given a context, $M$, we can
define a \emph{nominal context}, $\quotep{M}$ by $\quotep{M}[P] :=
\quotep{M[P]}$. To foreshadow what is to come we observe that these
operations enjoy a duality with processes very much like the duality
between vectors and maps from vectors to scalars.

Further, because the calculus is essentially higher-order, we have a
correspondence between contexts and processes. More specifically,
given a name $x$ and a context $M$ we can construct $M^{*}_{x}$ such
that 

\begin{mathpar}
  M^{*}_{x} | \lift{x}{P} \red M[P]
\end{mathpar}

namely,

\begin{mathpar}
  M^{*}_{x} := x?(u).M[\dropn{u}]
\end{mathpar}

The dependence of $M^{*}_{x}$ on a name makes it an abstraction, 

\begin{mathpar}
  M^{*} := (x)x?(u).M[\dropn{u}]
\end{mathpar}

\subsection{Additional notation}

It will sometimes be convenient to denote the process a name
quotes. We already have the notation $x = \quotep{P}$, but it will be
convenient to introduce an alternate notation, $\procn{x}$, when we
want to emphasize the connection to the use of the name. Note that, by
virtue of name equivalence, $\quotep{\procn{x}} \nameeq x$; so, the
notation is consistent with previous definitions.

Further, because names have structure it is possible to effect
substitutions on the basis of that structure. This means we need to
upgrade our notation for substitutions, which we accomplish by
adapting comprehension notation. Thus,

\begin{mathpar}
  P\{ y / x : x \in S \}
\end{mathpar}

is interpreted to mean the process derived from P by replacing (in a
capture-avoiding manner) each occurrence of $x$ in $S$ by $y$. For example,

\begin{mathpar}
  P\{ \quotep{\procn{x}|\procn{x}} / x : x \in \freenames{P} \}
\end{mathpar}

will replace each (occurrence) of a free name $x$ in $P$ by
$\quotep{\procn{x}|\procn{x}}$.

Also, we will avail ourselves of the notation $x^{L}$ and $x^{R}$ to
denote injections of a name into disjoint copies of the name
space. There are numerous ways to accomplish this. One example can be
found in \cite{MeredithR05}. This notation overloads to vectors of
names: $\vec{x}^{\pi} := (x_{i}^{\pi} \; : \; 0 \leq i < |\vec{x}| )$ where $\pi \in \{L,R\}$.

We also use $P^{\Box} := P|\Box$.

In \cite{MeredithR05} an interpretation of the new operator is
given. It turns out that there are several possible interpretations
all enjoying the requisite algebraic properties of the operator (see
\cite{milner91polyadicpi}). We will therefore make liberal use of
$(\nu\; \vec{x})P$.

% subsection the_syntax_and_semantics_of_the_notation_system (end)   

\input{qm2pi.qmops} 

\input{qm2pi.sterngerlach} 

\input{qm2pi.metric} 

% section concurrent_process_calculi (end)

%\input{qm2pi.proofsketch}

% section proof sketch (end)

%\input{qm2pi.slviaknots} 

% section spatial logic via knots (end)

\input{qm2pi.conclusion}

% section conclusion (end)

%\input{qm2pi.dtcodes} 

% section wiring algorithm (end)

\input{qm2pi.ack} 

% section acknowledgments (end)

\newpage


\bibliographystyle{plain}   
\bibliography{../../biblios/main.bib}

\input{qm2pi.rhodetails}

\end{document}

 

% section concurrent_process_calculi (end)

%\documentclass[12pt]{llncs}
%\documentclass{jktr}

\usepackage[pdftex]{hyperref}                   
\usepackage {listings}
\usepackage {mathpartir}
\usepackage{bcprules}
%\usepackage{listings}
                       
\usepackage{graphicx} 
%\usepackage[margins=2.5cm,nohead,nofoot]{geometry}
%\usepackage{geometry}
\usepackage{amsfonts}
\usepackage{amstext}
\usepackage{latexsym}
\usepackage{amssymb}
\usepackage{color}


%\include{myPreamble}
\include{qm2pi.local} 

%\ifpdf
%\usepackage[pdftex]{graphicx}
%\else
%\usepackage{graphicx}
%\fi

 % \ifpdf
%  \usepackage{pdfsync}
%  \if


%\title{Brief Article}
%\author{David F. Snyder}
%\author{L.G. Meredith}

%\address{Dept. of Math., Texas State University--San Marcos, San Marcos, TX 78666}
       
\pagestyle{empty}


\begin{document}

\lstset{language=[Objective]Caml,frame=shadowbox}

\input{qm2pi.front}

% section front matter (end)

\input{qm2pi.intro} 
 
% section introduction (end)

% \input{qm2pi.knotations} 

% section notation (end)

\input{qm2pi.process.calculi} 

% section concurrent_process_calculi_and_spatial_logics_ (end)
    
%\input{qm2pi.knots2pi} 

%\input{qm2pi.trefoil} 

%\input{qm2pi.mainthm} 

% subsection basic_interpretation (end)

%\input{qm2pi.rho.presentation} 
\subsection{The syntax and semantics of the notation system}\label{sub:the_syntax_and_semantics_of_the_notation_system} % (fold)

We now summarize a technical presentation of the calculus that
embodies our theory of dynamics. The typical presentation of such a
calculus follows the style of giving generators and relations on
them. The grammar, below, describing term constructors, freely
generates the set of processes, $\Proc$. This set is then quotiented
by a relation known as structural congruence and it is over this set
that the notion of dynamics is expressed. This presentation is
essentially that of \cite{MeredithR05} with the addition of
polyadicity and summation. For readability we have relegated some of
the technical subtleties to an appendix.

\subsubsection{Process grammar}\label{subsub:process_grammar}

\begin{mathpar}
  \inferrule* [lab=synchronization] {} {{M} \bc \pzero \;|\; x?F \;|\; x!C }
  \and
  \inferrule* [lab=abstraction] {} {{F} \bc (x)P}
  \and
  \inferrule* [lab=concretion] {} {{C} \bc \langle Q \rangle}
  \and
  \inferrule* [lab=process] {} {{P,Q} \bc M \;| \;P|Q \;|\; @{x}}
  \and
  \inferrule* [lab=name] {} {{x} \bc \quotep{P}}
\end{mathpar} 

Note that $\vec{x}$ (resp. $\vec{P}$) denotes a vector of names
(resp. processes) of length $|\vec{x}|$ (resp. $|\vec{P}|$). We adopt
the following useful abbreviations.

\begin{mathpar}
   x?(\vec{y}).P := x.(\vec{y})P \and  x\clift{\vec{P}} := x.\clift{\vec{P}}
   \and x!(y) := \lift{x}{\dropn{y}}
   \and \Pi_{i=0}^{n-1}P_i := P_0 | \ldots | P_{n-1}
\end{mathpar}

\subsubsection{Structural congruence}

\paragraph{Free and bound names and alpha-equivalence.} At the
core of structural equivalence is alpha-equivalence which identifies
process that are the same up to a change of variable. Formally, we
recognize the distinction between free and bound names. The free names
of a process, $\freenames{P}$, may be calculated recursively as
follows:

\begin{mathpar}
\freenames{\pzero} := \emptyset
  \and \\
  \freenames{x?(y).P} := \{ x \} \cup (\freenames{P} \setminus \{ y \})
  \and 
  \freenames{x!\langle P \rangle} := \{ x \} \cup \{ P \} 
  \and \\
  \freenames{P|Q} := \freenames{P} \cup \freenames{Q}
  \and \\
  \freenames{@{x}} := \{ x \}
\end{mathpar}

$\pi$
$\quotep{\pi}$

$\freenames{-} : \pi \to \mathcal{P}(\quotep{\pi})$

\begin{eqnarray*}
  \freenames{\pzero} & := & \emptyset \\
  \freenames{x?(y).P} & := & \{ x \} \cup (\freenames{P} \setminus \{ y \}) \\
  \freenames{x!\langle P \rangle} & := & \{ x \} \cup \{ P \} \\
  \freenames{P|Q} & := & \freenames{P} \cup \freenames{Q} \\
  \freenames{\dropn{x}} & := & \{ x \}
\end{eqnarray*}

The bound names of a process, $\boundnames{P}$, are those names occurring in $P$
that are not free. For example, in $x?(y).0$, the name $x$ is free, while $y$ is bound.

\begin{mathpar}
  \inferrule* [lab=monoidal-laws] {} { P|Q \equiv Q|P \and P|0 \equiv P \and P|(Q|R) \equiv (P|Q)|R }
\end{mathpar}

\begin{mathpar}
  \inferrule* [lab=alpha-equivalence] {} { (x)P \equiv (y)P\{y/x\} \and y \not\in \freenames{P} }
\end{mathpar}

\begin{definition}
Then two processes, $P,Q$, are alpha-equivalent if $P = Q\{\vec{y}/\vec{x}\}$ for
some $\vec{x} \in \boundnames{Q},\vec{y} \in \boundnames{P}$, where $Q\{\vec{y}/\vec{x}\}$
denotes the capture-avoiding substitution of $\vec{y}$ for $\vec{x}$ in $Q$.
\end{definition}

\begin{definition}
  The {\em structural congruence} \cite{SangiorgiWalker} , $\equiv$,
  between processes is the least congruence containing
  alpha-equivalence, satisfying the abelian monoid laws
  (associativity, commutativity and $\pzero$ as identity) for parallel
  composition $|$ and for summation $+$.
\end{definition}

\subsection{Name equivalence}

We take name equivalence, written $\nameeq$, to be the smallest
equivalence relation generated by the following rules.

\begin{mathpar}
\inferrule*[lab=Quote-drop]
{ }
{ \quotep{@{x}} \nameeq x }

\inferrule*[lab=Struct-equiv]
{ P \scong Q }
{ \quotep{P} \nameeq \quotep{Q} }
\end{mathpar}

The astute reader will have noticed that the mutual recursion of names
and processes imposes a mutual recursion on alpha-equivalence and
structural equivalence via name-equivalence. Fortunately, all of this
works out pleasantly and we may calculate in the natural way, free of
concern. The reader interested in the details is referred to the
appendix \ref{appendix:rho_details}.

\subsection{Substitution}

We use $\Proc$ for the set of processes, $\QProc$ for the set of
names, and $\id{\{}\vec{y} / \vec{x} \id{\}}$ to denote partial maps,
$s : \QProc \rightarrow \QProc$. A map, $s$ lifts, uniquely, to a map
on process terms, $\widehat{s} : \Proc \rightarrow \Proc$ by the
following equations.

\begin{mathpar}
  (0) \psubstp{Q}{P} := 0 \\
  (R \juxtap S) \psubstp{Q}{P}
  :=    
  (R)\psubstp{Q}{P} \juxtap (S) \psubstp{Q}{P} \\
  (x?(y).R) \psubstp{Q}{P}    
  :=    
  (x)\substp{Q}{P} (z)\concat( (R \psubstn{z}{y}) \psubstp{Q}{P} ) \\
  (\lift{x}{R}) \psubstp{Q}{P}  
  :=
  \lift{(x)\substp{Q}{P}}{ R \psubstp{Q}{P} } \\
%   (\dropn{x})  \psubstp{Q}{P}       
%   := 
%   \left\{ 
%     \begin{array}{ccc} 
%       \dropn{\quotep{Q}} & & x \nameeq \quotep{P} \\
%       \dropn{x} & & otherwise \\
%     \end{array}
%   \right. 
  (\dropn{x})  \psubstp{Q}{P}       
  := 
  \left\{ 
    \begin{array}{ccc} 
      Q & & x \nameeq \quotep{P} \\
      \dropn{x} & & otherwise \\
    \end{array}
  \right.
\end{mathpar}
 

where

\begin{eqnarray}
  (x)\id{\{} \lpquote Q \rpquote / \lpquote P \rpquote \id{\}}            = 
  \left\{ 
    \begin{array}{ccc}
      \lpquote Q \rpquote & & x \nameeq \lpquote P \rpquote \\
      x & & otherwise \\
    \end{array}
  \right. \nonumber
\end{eqnarray}

and $z$ is chosen distinct from $\quotep{P}$, $\quotep{Q}$, the free
names in $Q$, and all the names in $R$. Our $\alpha$-equivalence will
be built in the standard way from this substitution.

\begin{remark}\label{rem:no_self_referential_names}
  One consequence of these definitions is that $\forall P. \quotep{P}
  \not\in \freenames{P}$.
\end{remark}

\subsection{ Dynamic quote: an example }

Anticipating something of what's to come, consider applying the
substitution, $\widehat{\id{\{}u / z \id{\}}}$, to the following pair
of processes, $\lift{w}{y!(z)}$ and $w[ \lpquote y!(z) \rpquote ]$.

\begin{eqnarray}
	\lift{w}{y!(z)}\widehat{\id{\{}u / z \id{\}}}
		& = &
		\lift{w}{y!(u)} \nonumber\\
	w[ \lpquote y!(z) \rpquote ] \widehat{ \id{\{}u / z \id{\}} }
		& = &
		w[ \lpquote y!(z) \rpquote ] \nonumber
\end{eqnarray}

Because the body of the process between quotes is impervious to
substitution, we get radically different answers. In fact, by
examining the first process in an input context,
e.g. $x?(z).\lift{w}{y!(z)}$, we see that the process under the lift
operator may be shaped by prefixed inputs binding a name inside it. In
this sense, the lift operator will be seen as a way to dynamically
construct processes before reifying them as names.

Finally equipped with these standard features we can present the
dynamics of the calculus.

\subsubsection{Operational semantics} 

Finally, we introduce the computational dynamics. What marks these
algebras as distinct from other more traditionally studied algebraic
structures, e.g. vector spaces or polynomial rings, is the manner in
which dynamics is captured. In traditional structures, dynamics is typically
expressed through morphisms between such structures, as in linear maps
between vector spaces or morphisms between rings. In algebras
associated with the semantics of computation, the dynamics is
expressed as part of the algebraic structure itself, through a
reduction reduction relation typically denoted by $\red$. Below, we
give a recursive presentation of this relation for the calculus used
in the encoding.

$\red \subseteq \pi \times \pi$
$\red : \pi \to \mathcal{P}(\pi)$

\begin{mathpar}
  \inferrule* [lab=Comm] { \textsf{match}( x_{src}, x_{trgt} ) } { x_{trgt}?(y)P \; | \; x_{src}!\langle {Q} \rangle \red P\{\quotep{Q}/y}\} }
  \and \\
  \inferrule* [lab=Par] {{P} \red {P}'} {{{P} | {Q}} \red {{P}' | {Q}}}
  \and
  \inferrule* [lab=Equiv]{{{P} \scong {P}'} \andalso {{P}' \red {Q}'} \andalso {{Q}' \scong {Q}}}{{P} \red {Q}}
\end{mathpar}

\begin{eqnarray*}
  match_{\equiv} (\quotep{P},\quotep{Q}) & := & P \equiv Q \\
  match_{\dagger}(\quotep{P},\quotep{Q}) & := & \forall R. P|Q \red^{*} R => R \red^{*} 0 \\
  match_{K}(\quotep{P},\quotep{Q}) & := & K \mbox{ for some context } K
\end{eqnarray*}

$u?(x)P | u!\langle Q \rangle \red P\{\quotep{Q}/x\}$

%We write $\wred$ for $\red^*$, and $P\red$ if $\exists Q $ such that $ P \red Q$.
We write $P\red$ if $\exists Q $ such that $ P \red Q$ and $P\not\red$, otherwise.

\section{Replication}

As mentioned before, it is known that replication (and hence
recursion) can be implemented in a higher-order process algebra
\cite{SangiorgiWalker}. As our first example of calculation with the
machinery thus far presented we give the construction explicitly in
the {\rhoc}.

\begin{eqnarray}
	D_{x} & := & \prefix{x}{y}{(\binpar{\outputp{x}{y}}{@{y}})} \nonumber\\
	\bangp_{x}{P} & := & \binpar{{x}!\langle{\binpar{D_{x}}{P}}\rangle}{D_{x}} \nonumber
\end{eqnarray}

\begin{eqnarray}
	\bangp_{x}{P} & & \nonumber\\
	=
	& {x}!\langle{(\prefix{x}{y}{(\outputp{x}{y} | @{y})) | P}}\rangle 
	      | \prefix{x}{y}{(\outputp{x}{y} | @{y})} & \nonumber\\
	\red
	& (\outputp{x}{y} | @{y})\substn{\quotep{(\prefix{x}{y}{(@{y} | \outputp{x}{y})) | P}}}{y} & \nonumber\\
	=
	& \outputp{x}{\quotep{(\prefix{x}{y}{(\outputp{x}{y} | @{y})) | P}}}
	  | {(\prefix{x}{y}{(\outputp{x}{y} | @{y})) | P}} & \nonumber\\
	\red
	& \ldots & \nonumber\\
	\red^*
	& P | P | \ldots & \nonumber
\end{eqnarray}

Of course, this encoding, as an implementation, runs away, unfolding
$\bangp{P}$ eagerly. A lazier and more implementable replication
operator, restricted to input-guarded processes, may be obtained as follows.

\begin{eqnarray}
\bangp{\prefix{u}{v}{P}} 
	:= 
	\binpar{\lift{x}{\prefix{u}{v}{(\binpar{D(x)}{P})}}}{D(x)} \nonumber
\end{eqnarray}

\begin{remark}
  Note that the lazier definition still does not deal with summation
  or mixed summation (i.e. sums over input and output). The reader is
  invited to construct definitions of replication that deal with these
  features. 

  Further, the definitions are parameterized in a name, $x$. Can you,
  gentle reader, make a definition that eliminates this parameter and
  guarantees no accidental interaction between the replication
  machinery and the process being replicated -- i.e. no accidental
  sharing of names used by the process to get its work done and the
  name(s) used by the replication to effect copying. This latter
  revision of the definition of replication is crucial to obtaining
  the expected identity $!!P \sim !P$.
\end{remark}

\begin{remark}\label{rem:paradoxical_combinator}
  The reader familiar with the lambda calculus will have noticed the
  similarity between $D$ and the paradoxical combinator.

  [Ed. note: the existence of this seems to suggest we have to be more
  restrictive on the set of processes and names we admit if we are to
  support no-cloning.]
\end{remark}

\subsubsection{Bisimulation}

The computational dynamics gives rise to another kind of equivalence,
the equivalence of computational behavior. As previously mentioned
this is typically captured \emph{via} some form of bisimulation.

% The notion we use in this paper is weak barbed bisimulation
% \cite{milner91polyadicpi}.

The notion we use in this paper is derived from weak barbed
bisimulation \cite{milner91polyadicpi}. 

\begin{definition}
An \emph{observation relation}, $\downarrow_{\mathcal N}$, over a set
of names, $\mathcal N$, is the smallest relation satisfying the rules
below.

\infrule[Out-barb]{y \in {\mathcal N}, \; x \nameeq y}
		  {\outputp{x}{v} \downarrow_{\mathcal N} x}
\infrule[Par-barb]{\mbox{$P\downarrow_{\mathcal N} x$ or $Q\downarrow_{\mathcal N} x$}}
		  {\binpar{P}{Q} \downarrow_{\mathcal N} x}

We write $P \Downarrow_{\mathcal N} x$ if there is $Q$ such that 
$P \wred Q$ and $Q \downarrow_{\mathcal N} x$.
\end{definition}

\begin{definition}
%\label{def.bbisim}
An  ${\mathcal N}$-\emph{barbed bisimulation} over a set of names, ${\mathcal N}$, is a symmetric binary relation 
${\mathcal S}_{\mathcal N}$ between agents such that $P\rel{S}_{\mathcal N}Q$ implies:
\begin{enumerate}
\item If $P \red P'$ then $Q \wred Q'$ and $P'\rel{S}_{\mathcal N} Q'$.
\item If $P\downarrow_{\mathcal N} x$, then $Q\Downarrow_{\mathcal N} x$.
\end{enumerate}
$P$ is ${\mathcal N}$-barbed bisimilar to $Q$, written
$P \wbbisim_{\mathcal N} Q$, if $P \rel{S}_{\mathcal N} Q$ for some ${\mathcal N}$-barbed bisimulation ${\mathcal S}_{\mathcal N}$.
\end{definition}

$\mathcal{R} \subseteq \pi \times \pi$

$P \mathcal{R} Q => \forall P'. P \red P' \Rightarrow \exists Q'. Q \red Q', P' \mathcal{R} Q'$

$P \vdash x \Rightarrow Q \vdash x$

\begin{mathpar}
  \inferrule*[lab=Out-barb]{x \nameeq y}{{y}!\langle{Q}\rangle \vdash x}
  \and
  \inferrule*[lab=Par-barb]{\mbox{$P\vdash x$ or $Q\vdash x$}}{\binpar{P}{Q} \vdash x}
\end{mathpar}

\subsubsection{Contexts}

One of the principle advantages of computational calculi like the
$\pi$-calculus is a well-defined notion of context,
contextual-equivalence and a correlation between
contextual-equivalence and notions of bisimulation. The notion of
context allows the decomposition of a process into (sub-)process and
its syntactic environment, its context. Thus, a context may be
thought of as a process with a ``hole'' (written $\Box$) in it. The
application of a context $M$ to a process $P$, written $M[P]$, is
tantamount to filling the hole in $M$ with $P$. In this paper we do
not need the full weight of this theory, but do make use of the notion
of context in the proof the main theorem. 

\begin{mathpar}
  \inferrule* [lab=summation] {} {{M_{M},M_{N}} \bc \Box \;|\; x.M_{A} \;|\; M_{M}+M_{N}}
  \and
  \inferrule* [lab=agent] {} {{M_{A}} \bc (\vec{x})M_{P} \;| \; \clift{P_0,\ldots,M_{P},\ldots,P_N}}
  \and \\
  \inferrule* [lab=process] {} {{M_{P}} \bc M_{N} \;| \;P|M_{P} }
\end{mathpar} 

\begin{mathpar}
  \inferrule* [lab=sychronization] {} {M_{N} \bc \Box \;|\; x?M_{F} \;|\; x!M_{C}}
  \and
  \inferrule* [lab=abstraction] {} {{M_{F}} \bc (x)M_{P} }
  \and
  \inferrule* [lab=concretion] {} {{M_{C}} \bc \langle M_{P} \rangle }
  \and \\
  \inferrule* [lab=process] {} {{M_{P}} \bc M_{N} \;| \;P|M_{P} }
\end{mathpar}

\begin{definition}[contextual application] Given a context $M$, and
  process $P$, we define the \emph{contextual application}, $M[P] :=
  M\{P/\Box\}$. That is, the contextual application of M to P is the
  substitution of $P$ for $\Box$ in $M$.
\end{definition}

$\meaningof{-} : L \to \mathcal{P}(\pi)$

\begin{mathpar}
  \inferrule* [lab=collection] {} {\meaningof{true} = \pi, \and \meaningof{~E} = \pi \setminus \meaningof{E}, \and \meaningof{E_{1} \& E_{2}} = \meaningof{E_{1}} \cap \meaningof{E_{2}}}
\end{mathpar}

\begin{mathpar}
  \inferrule* [lab=structure] {} {\meaningof{0} = \{ P \in \pi | P \equiv 0 \}, \and \\ \meaningof{E_1 | E_2} = \{ P \in \pi | P \equiv P_{1} | P_{2}, P_{1} \in \meaningof{E_{1}}, P_{2} \in \meaningof{E_2}\} }
\end{mathpar}

\begin{mathpar}
 \inferrule* [lab=behavior] {} {\meaningof{\langle a?b \rangle E} = \{ P \in \pi | P \equiv Q | u?(y)P', \\ \and \\\\ \and \\ \;\;\; u \in \meaningof{a}, \forall z.P'\{z/y\} \in \meaningof{E\{z/b\}}\}, \and \\ \meaningof{a!E} = \{ P \in \pi | P \equiv Q | x!\langle P' \rangle, x \in \meaningof{a} P' \in \meaningof{E}\} }
\end{mathpar}

\begin{mathpar}
 \inferrule* [lab=nominal] {} {\meaningof{\quotep{E}} = \{ \quotep{P} \in \quotep{\pi} | P \in \meaningof{E} \}, \and \meaningof{\quotep{P}} = \{ \quotep{Q} \in \quotep{\pi} | P \equiv Q \} \and \\ \meaningof{@\quotep{E}} = \{ P \in \pi | P \equiv @x, x \in \meaningof{E} \}}
\end{mathpar}

\begin{eqnarray*}
  \\
  \meaningof{-} : TS \to ST
\end{eqnarray*}

\begin{eqnarray*}
  \\
  L : TS \to ST
\end{eqnarray*}

\begin{eqnarray*}
  \\
  P \models E \iff P \in \meaningof{E}
\end{eqnarray*}

\begin{eqnarray*}
  P \approx_{L} Q \iff \forall E \in L. P \models E \iff Q \models E
\end{eqnarray*}

\begin{eqnarray*}
  P \approx_{K} Q
\end{eqnarray*}

\begin{eqnarray*}
  P \approx Q
\end{eqnarray*}

$\approx_{K} = \approx = \approx_{L}$

\subsubsection{Contextual duality}

Note that contexts extend the quotation operation to a family of
operations from processes to names. Given a context, $M$, we can
define a \emph{nominal context}, $\quotep{M}$ by $\quotep{M}[P] :=
\quotep{M[P]}$. To foreshadow what is to come we observe that these
operations enjoy a duality with processes very much like the duality
between vectors and maps from vectors to scalars.

Further, because the calculus is essentially higher-order, we have a
correspondence between contexts and processes. More specifically,
given a name $x$ and a context $M$ we can construct $M^{*}_{x}$ such
that 

\begin{mathpar}
  M^{*}_{x} | \lift{x}{P} \red M[P]
\end{mathpar}

namely,

\begin{mathpar}
  M^{*}_{x} := x?(u).M[\dropn{u}]
\end{mathpar}

The dependence of $M^{*}_{x}$ on a name makes it an abstraction, 

\begin{mathpar}
  M^{*} := (x)x?(u).M[\dropn{u}]
\end{mathpar}

\subsection{Additional notation}

It will sometimes be convenient to denote the process a name
quotes. We already have the notation $x = \quotep{P}$, but it will be
convenient to introduce an alternate notation, $\procn{x}$, when we
want to emphasize the connection to the use of the name. Note that, by
virtue of name equivalence, $\quotep{\procn{x}} \nameeq x$; so, the
notation is consistent with previous definitions.

Further, because names have structure it is possible to effect
substitutions on the basis of that structure. This means we need to
upgrade our notation for substitutions, which we accomplish by
adapting comprehension notation. Thus,

\begin{mathpar}
  P\{ y / x : x \in S \}
\end{mathpar}

is interpreted to mean the process derived from P by replacing (in a
capture-avoiding manner) each occurrence of $x$ in $S$ by $y$. For example,

\begin{mathpar}
  P\{ \quotep{\procn{x}|\procn{x}} / x : x \in \freenames{P} \}
\end{mathpar}

will replace each (occurrence) of a free name $x$ in $P$ by
$\quotep{\procn{x}|\procn{x}}$.

Also, we will avail ourselves of the notation $x^{L}$ and $x^{R}$ to
denote injections of a name into disjoint copies of the name
space. There are numerous ways to accomplish this. One example can be
found in \cite{MeredithR05}. This notation overloads to vectors of
names: $\vec{x}^{\pi} := (x_{i}^{\pi} \; : \; 0 \leq i < |\vec{x}| )$ where $\pi \in \{L,R\}$.

We also use $P^{\Box} := P|\Box$.

In \cite{MeredithR05} an interpretation of the new operator is
given. It turns out that there are several possible interpretations
all enjoying the requisite algebraic properties of the operator (see
\cite{milner91polyadicpi}). We will therefore make liberal use of
$(\nu\; \vec{x})P$.

% subsection the_syntax_and_semantics_of_the_notation_system (end)   

\input{qm2pi.qmops} 

\input{qm2pi.sterngerlach} 

\input{qm2pi.metric} 

% section concurrent_process_calculi (end)

%\input{qm2pi.proofsketch}

% section proof sketch (end)

%\input{qm2pi.slviaknots} 

% section spatial logic via knots (end)

\input{qm2pi.conclusion}

% section conclusion (end)

%\input{qm2pi.dtcodes} 

% section wiring algorithm (end)

\input{qm2pi.ack} 

% section acknowledgments (end)

\newpage


\bibliographystyle{plain}   
\bibliography{../../biblios/main.bib}

\input{qm2pi.rhodetails}

\end{document}



% section proof sketch (end)

%\section{Unlikely characters: spatial logic for
  knots}\label{sub:characteristic_formulae} % (fold)

Associated to the mobile process calculi are a family of logics known
as the Hennessy-Milner logics. These logics typically enjoy a
semantics interpreting formulae as sets of processes that when
factored through the encoding outlined above allows an identification
of classes of knots with logical formulae. In the context of this
encoding the sub-family known as the spatial logics \cite{CairesC03}
\cite{CairesC04} \cite{Caires04} are of particular interest providing
several important features for expressing and reasoning about
properties (i.e. classes) of knots. We hint here at how this may be done.

%\begin{description}
%\item [structural connectives] 
\subsubsection{Structural connectives} The spatial logics enjoy
structural connectives corresponding, at the logical level, to the
parallel composition ($P | Q$) and new name ($(\nu \; x)P$)
connectives for processes. As illustrated in the examples below, these
connectives are extremely expressive given the shape of our encoding.
%\item [decideable satisfaction]

\subsubsection{Decideable satisfaction}
In \cite{Caires04} the satisfaction relation is shown to be decideable
for a rich class of processes. It further turns out that the image of
the our encoding is a proper subset of that class. This result
provides the basis for an algorithm by which to search for knots
enjoying a given property.
%\item [characteristic formulae]

\subsubsection{Characteristic formulae}
In the same paper \cite{Caires04} , Caires presents a means of calculating
characteristic formulae, selecting equivalence classes of processes
up to a pre--specified depth limit on the support set of names. Composed with our
encoding, this characteristic formula can be used to select
characteristic formulae for knots.
%\end{description}

\subsubsection{Spatial logic formulae}

The grammar below (segmented for comprehension) summarizes the syntax
of spatial logic formulae. We employ illustrative examples in the
sequel to provide an intuitive understanding of their meaning
referring the reader to \cite{Caires04} for a more detailed explication
of the semantics.

\begin{mathpar}
  \inferrule* [lab=boolean] {} {{A,B} \bc T \;|\; \neg A \;|\; A \wedge B \;|\; \eta = \eta'}
  \and
  \inferrule* [lab=spatial] {} {|\; \pzero \;|\; A | B \;|\; x \text{\textregistered} A \;|\; \forall x . A \;|\;  H x . A}
  \and
  \inferrule* [lab=behavioral] {} {|\; \alpha . A}
  \and 
  \inferrule* [lab=recursion] {} {|\; X(\vec{u}) \;|\; \mu X(\vec{u}) . A}
  \and
  \inferrule* [lab=action] {} {\alpha \bc \langle x?(\vec{y}) \rangle \;|\; \langle x!(\vec{y}) \rangle \;|\; \langle \tau \rangle}
  \and 
  \inferrule* [lab=name] {} {\eta \bc x \;|\; \tau}
\end{mathpar} 

% subsection characteristic_formulae (end)   	 

\subsection{Example formulae}\label{sub:example_formulae_} % (fold)

\subsubsection{Crossing as formula.}
% 
% \begin{align*}
%   \frac{d}{dx} \sin x &= \cos x 
%   & \frac{d}{dx} e^x &= e^x \\
%   \frac{d}{dx} \cos x &= - \sin x 
%   & \frac{d}{dx} \log x &= \frac{1}{x} \\
% \end{align*} 

\begin{align*}
 \mu C(x_{0},x_{1},y_{0},y_{1},u).&(\langle x_{0}?(z) \rangle(\langle u! \rangle\langle y_{1}!z \rangle C(x_{0},x_{1},y_{0},y_{1},u)) & \\
  & \wedge \langle y_{1}?(z) \rangle (\langle u! \rangle \langle x_{0}!z \rangle C(x_{0},x_{1},y_{0},y_{1},u)) & \\
  & \wedge \langle x_{1}?(z) \rangle (\langle u? \rangle \langle y_{0}!z \rangle C(x_{0},x_{1},y_{0},y_{1},u)) & \\
  & \wedge \langle y_{0}?(z) \rangle (\langle u? \rangle \langle x_{1}!z \rangle C(x_{0},x_{1},y_{0},y_{1},u))) &
\end{align*}

The lexicographical similarity between the shape of this formulae and
the shape of definition of the process representing a crossing reveals
the intuitive meaning of this formulae. It describes the capabilities
of a process that has the right to represent a crossing. For example
it picks out processes that may perform an input on the port $x_0$ in
its initial menu of capabilities. What differentiates the formula
from the process, however, is that the crossing process is the
smallest candidate to satisfy the formula. Infinitely many other
processes -- with internal behavior hidden behind this interface, so
to speak -- also satisfy this formula. Even this simple formula,
then, can be seen to open a new view onto knots, providing a
computational interpretation of \emph{virtual} knots.

Note that this formula is derived by hand. A similar formula can be
derived by employing Caires' calculation of characteristic formula
\cite{Caires04} to the process representing a crossing. In light of
this discussion, we let
$\meaningof{C}_{\phi}(x0,x1,y0,y1,u)$ denote a formula specifying the
dynamics we wish to capture of a crossing. To guarantee we preserve
the shape of the interface and minimal semantics we demand that
$\meaningof{C}_{\phi}(x0,x1,y0,y1,u) \Rightarrow
\textbf{C}(x0,x1,y0,y1,u)$ where $\textbf{C}(x0,x1,y0,y1,u)$ denotes
the formula above.
                            
\subsubsection{Crossing number constraints.}
The moral content of the context lemma (Lemma \ref{context}) is that the notion of
``locality'' in the Reidemeister moves is effectively captured by the
parallel composition operator of the process calculus. This intuition
extends through the logic. Given a formula,
$\meaningof{C}_{\phi}(x0,x1,y0,y1,u)$, we can use the structural
connectives to specify constraints on crossing numbers, such as at
least $n$ crossings, or exactly $n$ crossings.
\begin{mathpar}
  \inferrule* [lab=at-least-n] {} { K^{\geq n}_{\phi}(\vec{xs},\vec{ys}) := \Pi_{i=0}^{n-1} Hu . \meaningof{C}_{\phi}(xs_i,ys_i,u) | T }
  \and 
  \inferrule* [lab=exactly-n] {} { K^{= n}_{\phi}(\vec{xs},\vec{ys}) := \Pi_{i=0}^{n-1} Hu . \meaningof{C}_{\phi}(xs_i,ys_i,u) | \neg (\forall x_0,y_0,x_1,y_1,u . \meaningof{C}_{\phi}(x_0,y_0,x_1,y_1,u) | T) }
\end{mathpar}

To round out this section, recall that the encoding of an $n$-crossing
knot decomposes into a parallel composition of $n$ \emph{copies} of a
crossing process together with a wiring harness. To specify different
knot classes with the same crossing number amounts to specifying
logical constraints on the wiring harness. In the interest of space,
we defer examples to a forthcoming paper. Suffice it to say that both
the conditions ``alternating knot'' and ``contains the tangle
corresponding to 5/3'' are expressible. For example, it is possible to
calculate the characteristic formula of a process corresponding to the
tangle 5/3 and conjoin it into the classifying formula via the
composition connective of the logic.

Finally, we wish to observe that it is entirely within reason to
contemplate a more domain-specific version of spatial logic tailored
to the shape of processes in the image of the encoding. Such a
domain-specific logic would have a better claim to the title formal
language of knot properties.

% subsection example_formulae_ (end)

% section knots_as_processes (end) 

% section spatial logic via knots (end)

\section{Conclusions and future work}

\paragraph{Testing physical space}
You, gentle reader, may wonder why of all the theorems to be proved
given this set up we pick the one above. In some sense it's hardly
central to quantum mechanics. We see it as central in the sense that
it firmly establishes a notion of physical space arising from a notion
of the equivalence of behavior. Relating bisimulation to a metric is a
big step forward, but one is faced with interpreting the relationship
of that metric space to something more physical. Quantum mechanical
notions of ``physical'' space are still far from intuitive, but by
relating this idea of distance as testing to calculations that predict
physical circumstances we are making a not insignificant step forward
toward an understanding of the physical space we inhabit as
essentially dynamic.

\paragraph{Effectivity and simulation}
One of the observations we have yet to make is that the entire program
spelled out here is effective. We have built various interpreters for
the reflective calculus at work in this interpretation. In principle,
then, we can simulate quantum mechanics on a computer. The place where
the simulation may lose fidelity is the infinitely branching summation
for the annihilator.

In this connection i also want to point out that the evaluation style
calculation of the inner product puts the non-determinism of the
summation right at the heart of measurement. This suggests that
Milner's original reduction-based formulation of the dynamics of his
calculi in terms of sums was not just notationally suggestive of a
notion of measure-and-continue but captured some significant part of
the physics.

\paragraph{Quantum continuations}
In light of this last observation i want to point out that the
predominant account of quantum mechanics is missing a key aspect of a
truly compositional story of the physical situation. In a real lab,
when a measurement is made the observation can be made to feed into
another device that then makes another measurement conditioned on the
results of the first. This means that after the superposition was
collapsed the entire experimental set up remained in
superposition. While QM offers a means of writing this down it doesn't
quite line up well with the well-trodden formulation of computation
and continuation that we see so succinctly expressed in Milner's
calculi. This suggests that there might be advantages to this account
of dynamics waiting to be explored.

\paragraph{Quantum logic}
In this connection, we also note that by virtue of having the
Hennessy-Milner construction, we can pull the construction through the
interpretation of QM. This gives us a natural candidate for a quantum
logic that enjoys an extremely tight connection with it's domain of
interpretation, making the construction much less ad hoc (rather it is
the image of functor!).

\paragraph{Quantum probabiity}
i have questions about the basis of the interpretation of inner
product as probability amplitude. In particular, using which
axiomatization of probability theory does the notion of probability
amplitude earn the right to be so dubbed? In other words, where is the
proof that the operation for calculating a probability amplitude (and
then squaring) satisfies the axioms of what it means to calculate a
probability? Even if such a proof exists (i have yet to find it in the
literature), i wonder if it might not be possible to turn things on
their heads. Can we view the calculation of the probability amplitude
as an axiomatization of probability? If so, then the definition we
give for calculating probability amplitude may provide the basis for
an \emph{effective} theory of probability.

\paragraph{Quantum vs ``biological'' information}
Finally, i want to conclude with a more philosophical observation. At
a recent workshop in which QM was a predominant topic i noticed
something about quantum information. The speaker was giving a riveting
discussion of axiomatic QM and showing how properties of ``no
cloning'' and ``no deleting'' emerged as consequences of the
axiomatization. Theorems of this form are necessary to give us a sense
of confidence that our axioms characterize the physical theory. What
struck me, though, was that if quantum information is neither erasable
nor replicable it is markedly different from \emph{life}. Two of the
things we know about life is that

\begin{itemize}
  \item it ends;
  \item to gain some measure of persistence, to transcend it's
    finitude it is imminently copyable.
\end{itemize}

Both of these qualities are summarized succinctly in the aphorism: all
flesh is grass. For me these two kinds of ``information'' -- call them
quantum and biological -- are end points on a spectrum of strategies
for persistence. At one end, we have those curious entities that enjoy
uniqueness and permanence; at the other, we have those who in the face
of a certain end and an uncertain present make a go of passing
something on. To me one of the more remarkable aspects of the latter
strategy is that in the presence of noise (and certain features of
copying) we get a kind of dynamism, a chance for improvement against a
given persistent condition.

% subsection other_calculi_other_bisimulations_and_geometry_as_behavior (end)




% section conclusion (end)

%\documentclass[12pt]{llncs}
%\documentclass{jktr}

\usepackage[pdftex]{hyperref}                   
\usepackage {listings}
\usepackage {mathpartir}
\usepackage{bcprules}
%\usepackage{listings}
                       
\usepackage{graphicx} 
%\usepackage[margins=2.5cm,nohead,nofoot]{geometry}
%\usepackage{geometry}
\usepackage{amsfonts}
\usepackage{amstext}
\usepackage{latexsym}
\usepackage{amssymb}
\usepackage{color}


%\include{myPreamble}
\include{qm2pi.local} 

%\ifpdf
%\usepackage[pdftex]{graphicx}
%\else
%\usepackage{graphicx}
%\fi

 % \ifpdf
%  \usepackage{pdfsync}
%  \if


%\title{Brief Article}
%\author{David F. Snyder}
%\author{L.G. Meredith}

%\address{Dept. of Math., Texas State University--San Marcos, San Marcos, TX 78666}
       
\pagestyle{empty}


\begin{document}

\lstset{language=[Objective]Caml,frame=shadowbox}

\input{qm2pi.front}

% section front matter (end)

\input{qm2pi.intro} 
 
% section introduction (end)

% \input{qm2pi.knotations} 

% section notation (end)

\input{qm2pi.process.calculi} 

% section concurrent_process_calculi_and_spatial_logics_ (end)
    
%\input{qm2pi.knots2pi} 

%\input{qm2pi.trefoil} 

%\input{qm2pi.mainthm} 

% subsection basic_interpretation (end)

%\input{qm2pi.rho.presentation} 
\subsection{The syntax and semantics of the notation system}\label{sub:the_syntax_and_semantics_of_the_notation_system} % (fold)

We now summarize a technical presentation of the calculus that
embodies our theory of dynamics. The typical presentation of such a
calculus follows the style of giving generators and relations on
them. The grammar, below, describing term constructors, freely
generates the set of processes, $\Proc$. This set is then quotiented
by a relation known as structural congruence and it is over this set
that the notion of dynamics is expressed. This presentation is
essentially that of \cite{MeredithR05} with the addition of
polyadicity and summation. For readability we have relegated some of
the technical subtleties to an appendix.

\subsubsection{Process grammar}\label{subsub:process_grammar}

\begin{mathpar}
  \inferrule* [lab=synchronization] {} {{M} \bc \pzero \;|\; x?F \;|\; x!C }
  \and
  \inferrule* [lab=abstraction] {} {{F} \bc (x)P}
  \and
  \inferrule* [lab=concretion] {} {{C} \bc \langle Q \rangle}
  \and
  \inferrule* [lab=process] {} {{P,Q} \bc M \;| \;P|Q \;|\; @{x}}
  \and
  \inferrule* [lab=name] {} {{x} \bc \quotep{P}}
\end{mathpar} 

Note that $\vec{x}$ (resp. $\vec{P}$) denotes a vector of names
(resp. processes) of length $|\vec{x}|$ (resp. $|\vec{P}|$). We adopt
the following useful abbreviations.

\begin{mathpar}
   x?(\vec{y}).P := x.(\vec{y})P \and  x\clift{\vec{P}} := x.\clift{\vec{P}}
   \and x!(y) := \lift{x}{\dropn{y}}
   \and \Pi_{i=0}^{n-1}P_i := P_0 | \ldots | P_{n-1}
\end{mathpar}

\subsubsection{Structural congruence}

\paragraph{Free and bound names and alpha-equivalence.} At the
core of structural equivalence is alpha-equivalence which identifies
process that are the same up to a change of variable. Formally, we
recognize the distinction between free and bound names. The free names
of a process, $\freenames{P}$, may be calculated recursively as
follows:

\begin{mathpar}
\freenames{\pzero} := \emptyset
  \and \\
  \freenames{x?(y).P} := \{ x \} \cup (\freenames{P} \setminus \{ y \})
  \and 
  \freenames{x!\langle P \rangle} := \{ x \} \cup \{ P \} 
  \and \\
  \freenames{P|Q} := \freenames{P} \cup \freenames{Q}
  \and \\
  \freenames{@{x}} := \{ x \}
\end{mathpar}

$\pi$
$\quotep{\pi}$

$\freenames{-} : \pi \to \mathcal{P}(\quotep{\pi})$

\begin{eqnarray*}
  \freenames{\pzero} & := & \emptyset \\
  \freenames{x?(y).P} & := & \{ x \} \cup (\freenames{P} \setminus \{ y \}) \\
  \freenames{x!\langle P \rangle} & := & \{ x \} \cup \{ P \} \\
  \freenames{P|Q} & := & \freenames{P} \cup \freenames{Q} \\
  \freenames{\dropn{x}} & := & \{ x \}
\end{eqnarray*}

The bound names of a process, $\boundnames{P}$, are those names occurring in $P$
that are not free. For example, in $x?(y).0$, the name $x$ is free, while $y$ is bound.

\begin{mathpar}
  \inferrule* [lab=monoidal-laws] {} { P|Q \equiv Q|P \and P|0 \equiv P \and P|(Q|R) \equiv (P|Q)|R }
\end{mathpar}

\begin{mathpar}
  \inferrule* [lab=alpha-equivalence] {} { (x)P \equiv (y)P\{y/x\} \and y \not\in \freenames{P} }
\end{mathpar}

\begin{definition}
Then two processes, $P,Q$, are alpha-equivalent if $P = Q\{\vec{y}/\vec{x}\}$ for
some $\vec{x} \in \boundnames{Q},\vec{y} \in \boundnames{P}$, where $Q\{\vec{y}/\vec{x}\}$
denotes the capture-avoiding substitution of $\vec{y}$ for $\vec{x}$ in $Q$.
\end{definition}

\begin{definition}
  The {\em structural congruence} \cite{SangiorgiWalker} , $\equiv$,
  between processes is the least congruence containing
  alpha-equivalence, satisfying the abelian monoid laws
  (associativity, commutativity and $\pzero$ as identity) for parallel
  composition $|$ and for summation $+$.
\end{definition}

\subsection{Name equivalence}

We take name equivalence, written $\nameeq$, to be the smallest
equivalence relation generated by the following rules.

\begin{mathpar}
\inferrule*[lab=Quote-drop]
{ }
{ \quotep{@{x}} \nameeq x }

\inferrule*[lab=Struct-equiv]
{ P \scong Q }
{ \quotep{P} \nameeq \quotep{Q} }
\end{mathpar}

The astute reader will have noticed that the mutual recursion of names
and processes imposes a mutual recursion on alpha-equivalence and
structural equivalence via name-equivalence. Fortunately, all of this
works out pleasantly and we may calculate in the natural way, free of
concern. The reader interested in the details is referred to the
appendix \ref{appendix:rho_details}.

\subsection{Substitution}

We use $\Proc$ for the set of processes, $\QProc$ for the set of
names, and $\id{\{}\vec{y} / \vec{x} \id{\}}$ to denote partial maps,
$s : \QProc \rightarrow \QProc$. A map, $s$ lifts, uniquely, to a map
on process terms, $\widehat{s} : \Proc \rightarrow \Proc$ by the
following equations.

\begin{mathpar}
  (0) \psubstp{Q}{P} := 0 \\
  (R \juxtap S) \psubstp{Q}{P}
  :=    
  (R)\psubstp{Q}{P} \juxtap (S) \psubstp{Q}{P} \\
  (x?(y).R) \psubstp{Q}{P}    
  :=    
  (x)\substp{Q}{P} (z)\concat( (R \psubstn{z}{y}) \psubstp{Q}{P} ) \\
  (\lift{x}{R}) \psubstp{Q}{P}  
  :=
  \lift{(x)\substp{Q}{P}}{ R \psubstp{Q}{P} } \\
%   (\dropn{x})  \psubstp{Q}{P}       
%   := 
%   \left\{ 
%     \begin{array}{ccc} 
%       \dropn{\quotep{Q}} & & x \nameeq \quotep{P} \\
%       \dropn{x} & & otherwise \\
%     \end{array}
%   \right. 
  (\dropn{x})  \psubstp{Q}{P}       
  := 
  \left\{ 
    \begin{array}{ccc} 
      Q & & x \nameeq \quotep{P} \\
      \dropn{x} & & otherwise \\
    \end{array}
  \right.
\end{mathpar}
 

where

\begin{eqnarray}
  (x)\id{\{} \lpquote Q \rpquote / \lpquote P \rpquote \id{\}}            = 
  \left\{ 
    \begin{array}{ccc}
      \lpquote Q \rpquote & & x \nameeq \lpquote P \rpquote \\
      x & & otherwise \\
    \end{array}
  \right. \nonumber
\end{eqnarray}

and $z$ is chosen distinct from $\quotep{P}$, $\quotep{Q}$, the free
names in $Q$, and all the names in $R$. Our $\alpha$-equivalence will
be built in the standard way from this substitution.

\begin{remark}\label{rem:no_self_referential_names}
  One consequence of these definitions is that $\forall P. \quotep{P}
  \not\in \freenames{P}$.
\end{remark}

\subsection{ Dynamic quote: an example }

Anticipating something of what's to come, consider applying the
substitution, $\widehat{\id{\{}u / z \id{\}}}$, to the following pair
of processes, $\lift{w}{y!(z)}$ and $w[ \lpquote y!(z) \rpquote ]$.

\begin{eqnarray}
	\lift{w}{y!(z)}\widehat{\id{\{}u / z \id{\}}}
		& = &
		\lift{w}{y!(u)} \nonumber\\
	w[ \lpquote y!(z) \rpquote ] \widehat{ \id{\{}u / z \id{\}} }
		& = &
		w[ \lpquote y!(z) \rpquote ] \nonumber
\end{eqnarray}

Because the body of the process between quotes is impervious to
substitution, we get radically different answers. In fact, by
examining the first process in an input context,
e.g. $x?(z).\lift{w}{y!(z)}$, we see that the process under the lift
operator may be shaped by prefixed inputs binding a name inside it. In
this sense, the lift operator will be seen as a way to dynamically
construct processes before reifying them as names.

Finally equipped with these standard features we can present the
dynamics of the calculus.

\subsubsection{Operational semantics} 

Finally, we introduce the computational dynamics. What marks these
algebras as distinct from other more traditionally studied algebraic
structures, e.g. vector spaces or polynomial rings, is the manner in
which dynamics is captured. In traditional structures, dynamics is typically
expressed through morphisms between such structures, as in linear maps
between vector spaces or morphisms between rings. In algebras
associated with the semantics of computation, the dynamics is
expressed as part of the algebraic structure itself, through a
reduction reduction relation typically denoted by $\red$. Below, we
give a recursive presentation of this relation for the calculus used
in the encoding.

$\red \subseteq \pi \times \pi$
$\red : \pi \to \mathcal{P}(\pi)$

\begin{mathpar}
  \inferrule* [lab=Comm] { \textsf{match}( x_{src}, x_{trgt} ) } { x_{trgt}?(y)P \; | \; x_{src}!\langle {Q} \rangle \red P\{\quotep{Q}/y}\} }
  \and \\
  \inferrule* [lab=Par] {{P} \red {P}'} {{{P} | {Q}} \red {{P}' | {Q}}}
  \and
  \inferrule* [lab=Equiv]{{{P} \scong {P}'} \andalso {{P}' \red {Q}'} \andalso {{Q}' \scong {Q}}}{{P} \red {Q}}
\end{mathpar}

\begin{eqnarray*}
  match_{\equiv} (\quotep{P},\quotep{Q}) & := & P \equiv Q \\
  match_{\dagger}(\quotep{P},\quotep{Q}) & := & \forall R. P|Q \red^{*} R => R \red^{*} 0 \\
  match_{K}(\quotep{P},\quotep{Q}) & := & K \mbox{ for some context } K
\end{eqnarray*}

$u?(x)P | u!\langle Q \rangle \red P\{\quotep{Q}/x\}$

%We write $\wred$ for $\red^*$, and $P\red$ if $\exists Q $ such that $ P \red Q$.
We write $P\red$ if $\exists Q $ such that $ P \red Q$ and $P\not\red$, otherwise.

\section{Replication}

As mentioned before, it is known that replication (and hence
recursion) can be implemented in a higher-order process algebra
\cite{SangiorgiWalker}. As our first example of calculation with the
machinery thus far presented we give the construction explicitly in
the {\rhoc}.

\begin{eqnarray}
	D_{x} & := & \prefix{x}{y}{(\binpar{\outputp{x}{y}}{@{y}})} \nonumber\\
	\bangp_{x}{P} & := & \binpar{{x}!\langle{\binpar{D_{x}}{P}}\rangle}{D_{x}} \nonumber
\end{eqnarray}

\begin{eqnarray}
	\bangp_{x}{P} & & \nonumber\\
	=
	& {x}!\langle{(\prefix{x}{y}{(\outputp{x}{y} | @{y})) | P}}\rangle 
	      | \prefix{x}{y}{(\outputp{x}{y} | @{y})} & \nonumber\\
	\red
	& (\outputp{x}{y} | @{y})\substn{\quotep{(\prefix{x}{y}{(@{y} | \outputp{x}{y})) | P}}}{y} & \nonumber\\
	=
	& \outputp{x}{\quotep{(\prefix{x}{y}{(\outputp{x}{y} | @{y})) | P}}}
	  | {(\prefix{x}{y}{(\outputp{x}{y} | @{y})) | P}} & \nonumber\\
	\red
	& \ldots & \nonumber\\
	\red^*
	& P | P | \ldots & \nonumber
\end{eqnarray}

Of course, this encoding, as an implementation, runs away, unfolding
$\bangp{P}$ eagerly. A lazier and more implementable replication
operator, restricted to input-guarded processes, may be obtained as follows.

\begin{eqnarray}
\bangp{\prefix{u}{v}{P}} 
	:= 
	\binpar{\lift{x}{\prefix{u}{v}{(\binpar{D(x)}{P})}}}{D(x)} \nonumber
\end{eqnarray}

\begin{remark}
  Note that the lazier definition still does not deal with summation
  or mixed summation (i.e. sums over input and output). The reader is
  invited to construct definitions of replication that deal with these
  features. 

  Further, the definitions are parameterized in a name, $x$. Can you,
  gentle reader, make a definition that eliminates this parameter and
  guarantees no accidental interaction between the replication
  machinery and the process being replicated -- i.e. no accidental
  sharing of names used by the process to get its work done and the
  name(s) used by the replication to effect copying. This latter
  revision of the definition of replication is crucial to obtaining
  the expected identity $!!P \sim !P$.
\end{remark}

\begin{remark}\label{rem:paradoxical_combinator}
  The reader familiar with the lambda calculus will have noticed the
  similarity between $D$ and the paradoxical combinator.

  [Ed. note: the existence of this seems to suggest we have to be more
  restrictive on the set of processes and names we admit if we are to
  support no-cloning.]
\end{remark}

\subsubsection{Bisimulation}

The computational dynamics gives rise to another kind of equivalence,
the equivalence of computational behavior. As previously mentioned
this is typically captured \emph{via} some form of bisimulation.

% The notion we use in this paper is weak barbed bisimulation
% \cite{milner91polyadicpi}.

The notion we use in this paper is derived from weak barbed
bisimulation \cite{milner91polyadicpi}. 

\begin{definition}
An \emph{observation relation}, $\downarrow_{\mathcal N}$, over a set
of names, $\mathcal N$, is the smallest relation satisfying the rules
below.

\infrule[Out-barb]{y \in {\mathcal N}, \; x \nameeq y}
		  {\outputp{x}{v} \downarrow_{\mathcal N} x}
\infrule[Par-barb]{\mbox{$P\downarrow_{\mathcal N} x$ or $Q\downarrow_{\mathcal N} x$}}
		  {\binpar{P}{Q} \downarrow_{\mathcal N} x}

We write $P \Downarrow_{\mathcal N} x$ if there is $Q$ such that 
$P \wred Q$ and $Q \downarrow_{\mathcal N} x$.
\end{definition}

\begin{definition}
%\label{def.bbisim}
An  ${\mathcal N}$-\emph{barbed bisimulation} over a set of names, ${\mathcal N}$, is a symmetric binary relation 
${\mathcal S}_{\mathcal N}$ between agents such that $P\rel{S}_{\mathcal N}Q$ implies:
\begin{enumerate}
\item If $P \red P'$ then $Q \wred Q'$ and $P'\rel{S}_{\mathcal N} Q'$.
\item If $P\downarrow_{\mathcal N} x$, then $Q\Downarrow_{\mathcal N} x$.
\end{enumerate}
$P$ is ${\mathcal N}$-barbed bisimilar to $Q$, written
$P \wbbisim_{\mathcal N} Q$, if $P \rel{S}_{\mathcal N} Q$ for some ${\mathcal N}$-barbed bisimulation ${\mathcal S}_{\mathcal N}$.
\end{definition}

$\mathcal{R} \subseteq \pi \times \pi$

$P \mathcal{R} Q => \forall P'. P \red P' \Rightarrow \exists Q'. Q \red Q', P' \mathcal{R} Q'$

$P \vdash x \Rightarrow Q \vdash x$

\begin{mathpar}
  \inferrule*[lab=Out-barb]{x \nameeq y}{{y}!\langle{Q}\rangle \vdash x}
  \and
  \inferrule*[lab=Par-barb]{\mbox{$P\vdash x$ or $Q\vdash x$}}{\binpar{P}{Q} \vdash x}
\end{mathpar}

\subsubsection{Contexts}

One of the principle advantages of computational calculi like the
$\pi$-calculus is a well-defined notion of context,
contextual-equivalence and a correlation between
contextual-equivalence and notions of bisimulation. The notion of
context allows the decomposition of a process into (sub-)process and
its syntactic environment, its context. Thus, a context may be
thought of as a process with a ``hole'' (written $\Box$) in it. The
application of a context $M$ to a process $P$, written $M[P]$, is
tantamount to filling the hole in $M$ with $P$. In this paper we do
not need the full weight of this theory, but do make use of the notion
of context in the proof the main theorem. 

\begin{mathpar}
  \inferrule* [lab=summation] {} {{M_{M},M_{N}} \bc \Box \;|\; x.M_{A} \;|\; M_{M}+M_{N}}
  \and
  \inferrule* [lab=agent] {} {{M_{A}} \bc (\vec{x})M_{P} \;| \; \clift{P_0,\ldots,M_{P},\ldots,P_N}}
  \and \\
  \inferrule* [lab=process] {} {{M_{P}} \bc M_{N} \;| \;P|M_{P} }
\end{mathpar} 

\begin{mathpar}
  \inferrule* [lab=sychronization] {} {M_{N} \bc \Box \;|\; x?M_{F} \;|\; x!M_{C}}
  \and
  \inferrule* [lab=abstraction] {} {{M_{F}} \bc (x)M_{P} }
  \and
  \inferrule* [lab=concretion] {} {{M_{C}} \bc \langle M_{P} \rangle }
  \and \\
  \inferrule* [lab=process] {} {{M_{P}} \bc M_{N} \;| \;P|M_{P} }
\end{mathpar}

\begin{definition}[contextual application] Given a context $M$, and
  process $P$, we define the \emph{contextual application}, $M[P] :=
  M\{P/\Box\}$. That is, the contextual application of M to P is the
  substitution of $P$ for $\Box$ in $M$.
\end{definition}

$\meaningof{-} : L \to \mathcal{P}(\pi)$

\begin{mathpar}
  \inferrule* [lab=collection] {} {\meaningof{true} = \pi, \and \meaningof{~E} = \pi \setminus \meaningof{E}, \and \meaningof{E_{1} \& E_{2}} = \meaningof{E_{1}} \cap \meaningof{E_{2}}}
\end{mathpar}

\begin{mathpar}
  \inferrule* [lab=structure] {} {\meaningof{0} = \{ P \in \pi | P \equiv 0 \}, \and \\ \meaningof{E_1 | E_2} = \{ P \in \pi | P \equiv P_{1} | P_{2}, P_{1} \in \meaningof{E_{1}}, P_{2} \in \meaningof{E_2}\} }
\end{mathpar}

\begin{mathpar}
 \inferrule* [lab=behavior] {} {\meaningof{\langle a?b \rangle E} = \{ P \in \pi | P \equiv Q | u?(y)P', \\ \and \\\\ \and \\ \;\;\; u \in \meaningof{a}, \forall z.P'\{z/y\} \in \meaningof{E\{z/b\}}\}, \and \\ \meaningof{a!E} = \{ P \in \pi | P \equiv Q | x!\langle P' \rangle, x \in \meaningof{a} P' \in \meaningof{E}\} }
\end{mathpar}

\begin{mathpar}
 \inferrule* [lab=nominal] {} {\meaningof{\quotep{E}} = \{ \quotep{P} \in \quotep{\pi} | P \in \meaningof{E} \}, \and \meaningof{\quotep{P}} = \{ \quotep{Q} \in \quotep{\pi} | P \equiv Q \} \and \\ \meaningof{@\quotep{E}} = \{ P \in \pi | P \equiv @x, x \in \meaningof{E} \}}
\end{mathpar}

\begin{eqnarray*}
  \\
  \meaningof{-} : TS \to ST
\end{eqnarray*}

\begin{eqnarray*}
  \\
  L : TS \to ST
\end{eqnarray*}

\begin{eqnarray*}
  \\
  P \models E \iff P \in \meaningof{E}
\end{eqnarray*}

\begin{eqnarray*}
  P \approx_{L} Q \iff \forall E \in L. P \models E \iff Q \models E
\end{eqnarray*}

\begin{eqnarray*}
  P \approx_{K} Q
\end{eqnarray*}

\begin{eqnarray*}
  P \approx Q
\end{eqnarray*}

$\approx_{K} = \approx = \approx_{L}$

\subsubsection{Contextual duality}

Note that contexts extend the quotation operation to a family of
operations from processes to names. Given a context, $M$, we can
define a \emph{nominal context}, $\quotep{M}$ by $\quotep{M}[P] :=
\quotep{M[P]}$. To foreshadow what is to come we observe that these
operations enjoy a duality with processes very much like the duality
between vectors and maps from vectors to scalars.

Further, because the calculus is essentially higher-order, we have a
correspondence between contexts and processes. More specifically,
given a name $x$ and a context $M$ we can construct $M^{*}_{x}$ such
that 

\begin{mathpar}
  M^{*}_{x} | \lift{x}{P} \red M[P]
\end{mathpar}

namely,

\begin{mathpar}
  M^{*}_{x} := x?(u).M[\dropn{u}]
\end{mathpar}

The dependence of $M^{*}_{x}$ on a name makes it an abstraction, 

\begin{mathpar}
  M^{*} := (x)x?(u).M[\dropn{u}]
\end{mathpar}

\subsection{Additional notation}

It will sometimes be convenient to denote the process a name
quotes. We already have the notation $x = \quotep{P}$, but it will be
convenient to introduce an alternate notation, $\procn{x}$, when we
want to emphasize the connection to the use of the name. Note that, by
virtue of name equivalence, $\quotep{\procn{x}} \nameeq x$; so, the
notation is consistent with previous definitions.

Further, because names have structure it is possible to effect
substitutions on the basis of that structure. This means we need to
upgrade our notation for substitutions, which we accomplish by
adapting comprehension notation. Thus,

\begin{mathpar}
  P\{ y / x : x \in S \}
\end{mathpar}

is interpreted to mean the process derived from P by replacing (in a
capture-avoiding manner) each occurrence of $x$ in $S$ by $y$. For example,

\begin{mathpar}
  P\{ \quotep{\procn{x}|\procn{x}} / x : x \in \freenames{P} \}
\end{mathpar}

will replace each (occurrence) of a free name $x$ in $P$ by
$\quotep{\procn{x}|\procn{x}}$.

Also, we will avail ourselves of the notation $x^{L}$ and $x^{R}$ to
denote injections of a name into disjoint copies of the name
space. There are numerous ways to accomplish this. One example can be
found in \cite{MeredithR05}. This notation overloads to vectors of
names: $\vec{x}^{\pi} := (x_{i}^{\pi} \; : \; 0 \leq i < |\vec{x}| )$ where $\pi \in \{L,R\}$.

We also use $P^{\Box} := P|\Box$.

In \cite{MeredithR05} an interpretation of the new operator is
given. It turns out that there are several possible interpretations
all enjoying the requisite algebraic properties of the operator (see
\cite{milner91polyadicpi}). We will therefore make liberal use of
$(\nu\; \vec{x})P$.

% subsection the_syntax_and_semantics_of_the_notation_system (end)   

\input{qm2pi.qmops} 

\input{qm2pi.sterngerlach} 

\input{qm2pi.metric} 

% section concurrent_process_calculi (end)

%\input{qm2pi.proofsketch}

% section proof sketch (end)

%\input{qm2pi.slviaknots} 

% section spatial logic via knots (end)

\input{qm2pi.conclusion}

% section conclusion (end)

%\input{qm2pi.dtcodes} 

% section wiring algorithm (end)

\input{qm2pi.ack} 

% section acknowledgments (end)

\newpage


\bibliographystyle{plain}   
\bibliography{../../biblios/main.bib}

\input{qm2pi.rhodetails}

\end{document}

 

% section wiring algorithm (end)

\documentclass[12pt]{llncs}
%\documentclass{jktr}

\usepackage[pdftex]{hyperref}                   
\usepackage {listings}
\usepackage {mathpartir}
\usepackage{bcprules}
%\usepackage{listings}
                       
\usepackage{graphicx} 
%\usepackage[margins=2.5cm,nohead,nofoot]{geometry}
%\usepackage{geometry}
\usepackage{amsfonts}
\usepackage{amstext}
\usepackage{latexsym}
\usepackage{amssymb}
\usepackage{color}


%\include{myPreamble}
\include{qm2pi.local} 

%\ifpdf
%\usepackage[pdftex]{graphicx}
%\else
%\usepackage{graphicx}
%\fi

 % \ifpdf
%  \usepackage{pdfsync}
%  \if


%\title{Brief Article}
%\author{David F. Snyder}
%\author{L.G. Meredith}

%\address{Dept. of Math., Texas State University--San Marcos, San Marcos, TX 78666}
       
\pagestyle{empty}


\begin{document}

\lstset{language=[Objective]Caml,frame=shadowbox}

\input{qm2pi.front}

% section front matter (end)

\input{qm2pi.intro} 
 
% section introduction (end)

% \input{qm2pi.knotations} 

% section notation (end)

\input{qm2pi.process.calculi} 

% section concurrent_process_calculi_and_spatial_logics_ (end)
    
%\input{qm2pi.knots2pi} 

%\input{qm2pi.trefoil} 

%\input{qm2pi.mainthm} 

% subsection basic_interpretation (end)

%\input{qm2pi.rho.presentation} 
\subsection{The syntax and semantics of the notation system}\label{sub:the_syntax_and_semantics_of_the_notation_system} % (fold)

We now summarize a technical presentation of the calculus that
embodies our theory of dynamics. The typical presentation of such a
calculus follows the style of giving generators and relations on
them. The grammar, below, describing term constructors, freely
generates the set of processes, $\Proc$. This set is then quotiented
by a relation known as structural congruence and it is over this set
that the notion of dynamics is expressed. This presentation is
essentially that of \cite{MeredithR05} with the addition of
polyadicity and summation. For readability we have relegated some of
the technical subtleties to an appendix.

\subsubsection{Process grammar}\label{subsub:process_grammar}

\begin{mathpar}
  \inferrule* [lab=synchronization] {} {{M} \bc \pzero \;|\; x?F \;|\; x!C }
  \and
  \inferrule* [lab=abstraction] {} {{F} \bc (x)P}
  \and
  \inferrule* [lab=concretion] {} {{C} \bc \langle Q \rangle}
  \and
  \inferrule* [lab=process] {} {{P,Q} \bc M \;| \;P|Q \;|\; @{x}}
  \and
  \inferrule* [lab=name] {} {{x} \bc \quotep{P}}
\end{mathpar} 

Note that $\vec{x}$ (resp. $\vec{P}$) denotes a vector of names
(resp. processes) of length $|\vec{x}|$ (resp. $|\vec{P}|$). We adopt
the following useful abbreviations.

\begin{mathpar}
   x?(\vec{y}).P := x.(\vec{y})P \and  x\clift{\vec{P}} := x.\clift{\vec{P}}
   \and x!(y) := \lift{x}{\dropn{y}}
   \and \Pi_{i=0}^{n-1}P_i := P_0 | \ldots | P_{n-1}
\end{mathpar}

\subsubsection{Structural congruence}

\paragraph{Free and bound names and alpha-equivalence.} At the
core of structural equivalence is alpha-equivalence which identifies
process that are the same up to a change of variable. Formally, we
recognize the distinction between free and bound names. The free names
of a process, $\freenames{P}$, may be calculated recursively as
follows:

\begin{mathpar}
\freenames{\pzero} := \emptyset
  \and \\
  \freenames{x?(y).P} := \{ x \} \cup (\freenames{P} \setminus \{ y \})
  \and 
  \freenames{x!\langle P \rangle} := \{ x \} \cup \{ P \} 
  \and \\
  \freenames{P|Q} := \freenames{P} \cup \freenames{Q}
  \and \\
  \freenames{@{x}} := \{ x \}
\end{mathpar}

$\pi$
$\quotep{\pi}$

$\freenames{-} : \pi \to \mathcal{P}(\quotep{\pi})$

\begin{eqnarray*}
  \freenames{\pzero} & := & \emptyset \\
  \freenames{x?(y).P} & := & \{ x \} \cup (\freenames{P} \setminus \{ y \}) \\
  \freenames{x!\langle P \rangle} & := & \{ x \} \cup \{ P \} \\
  \freenames{P|Q} & := & \freenames{P} \cup \freenames{Q} \\
  \freenames{\dropn{x}} & := & \{ x \}
\end{eqnarray*}

The bound names of a process, $\boundnames{P}$, are those names occurring in $P$
that are not free. For example, in $x?(y).0$, the name $x$ is free, while $y$ is bound.

\begin{mathpar}
  \inferrule* [lab=monoidal-laws] {} { P|Q \equiv Q|P \and P|0 \equiv P \and P|(Q|R) \equiv (P|Q)|R }
\end{mathpar}

\begin{mathpar}
  \inferrule* [lab=alpha-equivalence] {} { (x)P \equiv (y)P\{y/x\} \and y \not\in \freenames{P} }
\end{mathpar}

\begin{definition}
Then two processes, $P,Q$, are alpha-equivalent if $P = Q\{\vec{y}/\vec{x}\}$ for
some $\vec{x} \in \boundnames{Q},\vec{y} \in \boundnames{P}$, where $Q\{\vec{y}/\vec{x}\}$
denotes the capture-avoiding substitution of $\vec{y}$ for $\vec{x}$ in $Q$.
\end{definition}

\begin{definition}
  The {\em structural congruence} \cite{SangiorgiWalker} , $\equiv$,
  between processes is the least congruence containing
  alpha-equivalence, satisfying the abelian monoid laws
  (associativity, commutativity and $\pzero$ as identity) for parallel
  composition $|$ and for summation $+$.
\end{definition}

\subsection{Name equivalence}

We take name equivalence, written $\nameeq$, to be the smallest
equivalence relation generated by the following rules.

\begin{mathpar}
\inferrule*[lab=Quote-drop]
{ }
{ \quotep{@{x}} \nameeq x }

\inferrule*[lab=Struct-equiv]
{ P \scong Q }
{ \quotep{P} \nameeq \quotep{Q} }
\end{mathpar}

The astute reader will have noticed that the mutual recursion of names
and processes imposes a mutual recursion on alpha-equivalence and
structural equivalence via name-equivalence. Fortunately, all of this
works out pleasantly and we may calculate in the natural way, free of
concern. The reader interested in the details is referred to the
appendix \ref{appendix:rho_details}.

\subsection{Substitution}

We use $\Proc$ for the set of processes, $\QProc$ for the set of
names, and $\id{\{}\vec{y} / \vec{x} \id{\}}$ to denote partial maps,
$s : \QProc \rightarrow \QProc$. A map, $s$ lifts, uniquely, to a map
on process terms, $\widehat{s} : \Proc \rightarrow \Proc$ by the
following equations.

\begin{mathpar}
  (0) \psubstp{Q}{P} := 0 \\
  (R \juxtap S) \psubstp{Q}{P}
  :=    
  (R)\psubstp{Q}{P} \juxtap (S) \psubstp{Q}{P} \\
  (x?(y).R) \psubstp{Q}{P}    
  :=    
  (x)\substp{Q}{P} (z)\concat( (R \psubstn{z}{y}) \psubstp{Q}{P} ) \\
  (\lift{x}{R}) \psubstp{Q}{P}  
  :=
  \lift{(x)\substp{Q}{P}}{ R \psubstp{Q}{P} } \\
%   (\dropn{x})  \psubstp{Q}{P}       
%   := 
%   \left\{ 
%     \begin{array}{ccc} 
%       \dropn{\quotep{Q}} & & x \nameeq \quotep{P} \\
%       \dropn{x} & & otherwise \\
%     \end{array}
%   \right. 
  (\dropn{x})  \psubstp{Q}{P}       
  := 
  \left\{ 
    \begin{array}{ccc} 
      Q & & x \nameeq \quotep{P} \\
      \dropn{x} & & otherwise \\
    \end{array}
  \right.
\end{mathpar}
 

where

\begin{eqnarray}
  (x)\id{\{} \lpquote Q \rpquote / \lpquote P \rpquote \id{\}}            = 
  \left\{ 
    \begin{array}{ccc}
      \lpquote Q \rpquote & & x \nameeq \lpquote P \rpquote \\
      x & & otherwise \\
    \end{array}
  \right. \nonumber
\end{eqnarray}

and $z$ is chosen distinct from $\quotep{P}$, $\quotep{Q}$, the free
names in $Q$, and all the names in $R$. Our $\alpha$-equivalence will
be built in the standard way from this substitution.

\begin{remark}\label{rem:no_self_referential_names}
  One consequence of these definitions is that $\forall P. \quotep{P}
  \not\in \freenames{P}$.
\end{remark}

\subsection{ Dynamic quote: an example }

Anticipating something of what's to come, consider applying the
substitution, $\widehat{\id{\{}u / z \id{\}}}$, to the following pair
of processes, $\lift{w}{y!(z)}$ and $w[ \lpquote y!(z) \rpquote ]$.

\begin{eqnarray}
	\lift{w}{y!(z)}\widehat{\id{\{}u / z \id{\}}}
		& = &
		\lift{w}{y!(u)} \nonumber\\
	w[ \lpquote y!(z) \rpquote ] \widehat{ \id{\{}u / z \id{\}} }
		& = &
		w[ \lpquote y!(z) \rpquote ] \nonumber
\end{eqnarray}

Because the body of the process between quotes is impervious to
substitution, we get radically different answers. In fact, by
examining the first process in an input context,
e.g. $x?(z).\lift{w}{y!(z)}$, we see that the process under the lift
operator may be shaped by prefixed inputs binding a name inside it. In
this sense, the lift operator will be seen as a way to dynamically
construct processes before reifying them as names.

Finally equipped with these standard features we can present the
dynamics of the calculus.

\subsubsection{Operational semantics} 

Finally, we introduce the computational dynamics. What marks these
algebras as distinct from other more traditionally studied algebraic
structures, e.g. vector spaces or polynomial rings, is the manner in
which dynamics is captured. In traditional structures, dynamics is typically
expressed through morphisms between such structures, as in linear maps
between vector spaces or morphisms between rings. In algebras
associated with the semantics of computation, the dynamics is
expressed as part of the algebraic structure itself, through a
reduction reduction relation typically denoted by $\red$. Below, we
give a recursive presentation of this relation for the calculus used
in the encoding.

$\red \subseteq \pi \times \pi$
$\red : \pi \to \mathcal{P}(\pi)$

\begin{mathpar}
  \inferrule* [lab=Comm] { \textsf{match}( x_{src}, x_{trgt} ) } { x_{trgt}?(y)P \; | \; x_{src}!\langle {Q} \rangle \red P\{\quotep{Q}/y}\} }
  \and \\
  \inferrule* [lab=Par] {{P} \red {P}'} {{{P} | {Q}} \red {{P}' | {Q}}}
  \and
  \inferrule* [lab=Equiv]{{{P} \scong {P}'} \andalso {{P}' \red {Q}'} \andalso {{Q}' \scong {Q}}}{{P} \red {Q}}
\end{mathpar}

\begin{eqnarray*}
  match_{\equiv} (\quotep{P},\quotep{Q}) & := & P \equiv Q \\
  match_{\dagger}(\quotep{P},\quotep{Q}) & := & \forall R. P|Q \red^{*} R => R \red^{*} 0 \\
  match_{K}(\quotep{P},\quotep{Q}) & := & K \mbox{ for some context } K
\end{eqnarray*}

$u?(x)P | u!\langle Q \rangle \red P\{\quotep{Q}/x\}$

%We write $\wred$ for $\red^*$, and $P\red$ if $\exists Q $ such that $ P \red Q$.
We write $P\red$ if $\exists Q $ such that $ P \red Q$ and $P\not\red$, otherwise.

\section{Replication}

As mentioned before, it is known that replication (and hence
recursion) can be implemented in a higher-order process algebra
\cite{SangiorgiWalker}. As our first example of calculation with the
machinery thus far presented we give the construction explicitly in
the {\rhoc}.

\begin{eqnarray}
	D_{x} & := & \prefix{x}{y}{(\binpar{\outputp{x}{y}}{@{y}})} \nonumber\\
	\bangp_{x}{P} & := & \binpar{{x}!\langle{\binpar{D_{x}}{P}}\rangle}{D_{x}} \nonumber
\end{eqnarray}

\begin{eqnarray}
	\bangp_{x}{P} & & \nonumber\\
	=
	& {x}!\langle{(\prefix{x}{y}{(\outputp{x}{y} | @{y})) | P}}\rangle 
	      | \prefix{x}{y}{(\outputp{x}{y} | @{y})} & \nonumber\\
	\red
	& (\outputp{x}{y} | @{y})\substn{\quotep{(\prefix{x}{y}{(@{y} | \outputp{x}{y})) | P}}}{y} & \nonumber\\
	=
	& \outputp{x}{\quotep{(\prefix{x}{y}{(\outputp{x}{y} | @{y})) | P}}}
	  | {(\prefix{x}{y}{(\outputp{x}{y} | @{y})) | P}} & \nonumber\\
	\red
	& \ldots & \nonumber\\
	\red^*
	& P | P | \ldots & \nonumber
\end{eqnarray}

Of course, this encoding, as an implementation, runs away, unfolding
$\bangp{P}$ eagerly. A lazier and more implementable replication
operator, restricted to input-guarded processes, may be obtained as follows.

\begin{eqnarray}
\bangp{\prefix{u}{v}{P}} 
	:= 
	\binpar{\lift{x}{\prefix{u}{v}{(\binpar{D(x)}{P})}}}{D(x)} \nonumber
\end{eqnarray}

\begin{remark}
  Note that the lazier definition still does not deal with summation
  or mixed summation (i.e. sums over input and output). The reader is
  invited to construct definitions of replication that deal with these
  features. 

  Further, the definitions are parameterized in a name, $x$. Can you,
  gentle reader, make a definition that eliminates this parameter and
  guarantees no accidental interaction between the replication
  machinery and the process being replicated -- i.e. no accidental
  sharing of names used by the process to get its work done and the
  name(s) used by the replication to effect copying. This latter
  revision of the definition of replication is crucial to obtaining
  the expected identity $!!P \sim !P$.
\end{remark}

\begin{remark}\label{rem:paradoxical_combinator}
  The reader familiar with the lambda calculus will have noticed the
  similarity between $D$ and the paradoxical combinator.

  [Ed. note: the existence of this seems to suggest we have to be more
  restrictive on the set of processes and names we admit if we are to
  support no-cloning.]
\end{remark}

\subsubsection{Bisimulation}

The computational dynamics gives rise to another kind of equivalence,
the equivalence of computational behavior. As previously mentioned
this is typically captured \emph{via} some form of bisimulation.

% The notion we use in this paper is weak barbed bisimulation
% \cite{milner91polyadicpi}.

The notion we use in this paper is derived from weak barbed
bisimulation \cite{milner91polyadicpi}. 

\begin{definition}
An \emph{observation relation}, $\downarrow_{\mathcal N}$, over a set
of names, $\mathcal N$, is the smallest relation satisfying the rules
below.

\infrule[Out-barb]{y \in {\mathcal N}, \; x \nameeq y}
		  {\outputp{x}{v} \downarrow_{\mathcal N} x}
\infrule[Par-barb]{\mbox{$P\downarrow_{\mathcal N} x$ or $Q\downarrow_{\mathcal N} x$}}
		  {\binpar{P}{Q} \downarrow_{\mathcal N} x}

We write $P \Downarrow_{\mathcal N} x$ if there is $Q$ such that 
$P \wred Q$ and $Q \downarrow_{\mathcal N} x$.
\end{definition}

\begin{definition}
%\label{def.bbisim}
An  ${\mathcal N}$-\emph{barbed bisimulation} over a set of names, ${\mathcal N}$, is a symmetric binary relation 
${\mathcal S}_{\mathcal N}$ between agents such that $P\rel{S}_{\mathcal N}Q$ implies:
\begin{enumerate}
\item If $P \red P'$ then $Q \wred Q'$ and $P'\rel{S}_{\mathcal N} Q'$.
\item If $P\downarrow_{\mathcal N} x$, then $Q\Downarrow_{\mathcal N} x$.
\end{enumerate}
$P$ is ${\mathcal N}$-barbed bisimilar to $Q$, written
$P \wbbisim_{\mathcal N} Q$, if $P \rel{S}_{\mathcal N} Q$ for some ${\mathcal N}$-barbed bisimulation ${\mathcal S}_{\mathcal N}$.
\end{definition}

$\mathcal{R} \subseteq \pi \times \pi$

$P \mathcal{R} Q => \forall P'. P \red P' \Rightarrow \exists Q'. Q \red Q', P' \mathcal{R} Q'$

$P \vdash x \Rightarrow Q \vdash x$

\begin{mathpar}
  \inferrule*[lab=Out-barb]{x \nameeq y}{{y}!\langle{Q}\rangle \vdash x}
  \and
  \inferrule*[lab=Par-barb]{\mbox{$P\vdash x$ or $Q\vdash x$}}{\binpar{P}{Q} \vdash x}
\end{mathpar}

\subsubsection{Contexts}

One of the principle advantages of computational calculi like the
$\pi$-calculus is a well-defined notion of context,
contextual-equivalence and a correlation between
contextual-equivalence and notions of bisimulation. The notion of
context allows the decomposition of a process into (sub-)process and
its syntactic environment, its context. Thus, a context may be
thought of as a process with a ``hole'' (written $\Box$) in it. The
application of a context $M$ to a process $P$, written $M[P]$, is
tantamount to filling the hole in $M$ with $P$. In this paper we do
not need the full weight of this theory, but do make use of the notion
of context in the proof the main theorem. 

\begin{mathpar}
  \inferrule* [lab=summation] {} {{M_{M},M_{N}} \bc \Box \;|\; x.M_{A} \;|\; M_{M}+M_{N}}
  \and
  \inferrule* [lab=agent] {} {{M_{A}} \bc (\vec{x})M_{P} \;| \; \clift{P_0,\ldots,M_{P},\ldots,P_N}}
  \and \\
  \inferrule* [lab=process] {} {{M_{P}} \bc M_{N} \;| \;P|M_{P} }
\end{mathpar} 

\begin{mathpar}
  \inferrule* [lab=sychronization] {} {M_{N} \bc \Box \;|\; x?M_{F} \;|\; x!M_{C}}
  \and
  \inferrule* [lab=abstraction] {} {{M_{F}} \bc (x)M_{P} }
  \and
  \inferrule* [lab=concretion] {} {{M_{C}} \bc \langle M_{P} \rangle }
  \and \\
  \inferrule* [lab=process] {} {{M_{P}} \bc M_{N} \;| \;P|M_{P} }
\end{mathpar}

\begin{definition}[contextual application] Given a context $M$, and
  process $P$, we define the \emph{contextual application}, $M[P] :=
  M\{P/\Box\}$. That is, the contextual application of M to P is the
  substitution of $P$ for $\Box$ in $M$.
\end{definition}

$\meaningof{-} : L \to \mathcal{P}(\pi)$

\begin{mathpar}
  \inferrule* [lab=collection] {} {\meaningof{true} = \pi, \and \meaningof{~E} = \pi \setminus \meaningof{E}, \and \meaningof{E_{1} \& E_{2}} = \meaningof{E_{1}} \cap \meaningof{E_{2}}}
\end{mathpar}

\begin{mathpar}
  \inferrule* [lab=structure] {} {\meaningof{0} = \{ P \in \pi | P \equiv 0 \}, \and \\ \meaningof{E_1 | E_2} = \{ P \in \pi | P \equiv P_{1} | P_{2}, P_{1} \in \meaningof{E_{1}}, P_{2} \in \meaningof{E_2}\} }
\end{mathpar}

\begin{mathpar}
 \inferrule* [lab=behavior] {} {\meaningof{\langle a?b \rangle E} = \{ P \in \pi | P \equiv Q | u?(y)P', \\ \and \\\\ \and \\ \;\;\; u \in \meaningof{a}, \forall z.P'\{z/y\} \in \meaningof{E\{z/b\}}\}, \and \\ \meaningof{a!E} = \{ P \in \pi | P \equiv Q | x!\langle P' \rangle, x \in \meaningof{a} P' \in \meaningof{E}\} }
\end{mathpar}

\begin{mathpar}
 \inferrule* [lab=nominal] {} {\meaningof{\quotep{E}} = \{ \quotep{P} \in \quotep{\pi} | P \in \meaningof{E} \}, \and \meaningof{\quotep{P}} = \{ \quotep{Q} \in \quotep{\pi} | P \equiv Q \} \and \\ \meaningof{@\quotep{E}} = \{ P \in \pi | P \equiv @x, x \in \meaningof{E} \}}
\end{mathpar}

\begin{eqnarray*}
  \\
  \meaningof{-} : TS \to ST
\end{eqnarray*}

\begin{eqnarray*}
  \\
  L : TS \to ST
\end{eqnarray*}

\begin{eqnarray*}
  \\
  P \models E \iff P \in \meaningof{E}
\end{eqnarray*}

\begin{eqnarray*}
  P \approx_{L} Q \iff \forall E \in L. P \models E \iff Q \models E
\end{eqnarray*}

\begin{eqnarray*}
  P \approx_{K} Q
\end{eqnarray*}

\begin{eqnarray*}
  P \approx Q
\end{eqnarray*}

$\approx_{K} = \approx = \approx_{L}$

\subsubsection{Contextual duality}

Note that contexts extend the quotation operation to a family of
operations from processes to names. Given a context, $M$, we can
define a \emph{nominal context}, $\quotep{M}$ by $\quotep{M}[P] :=
\quotep{M[P]}$. To foreshadow what is to come we observe that these
operations enjoy a duality with processes very much like the duality
between vectors and maps from vectors to scalars.

Further, because the calculus is essentially higher-order, we have a
correspondence between contexts and processes. More specifically,
given a name $x$ and a context $M$ we can construct $M^{*}_{x}$ such
that 

\begin{mathpar}
  M^{*}_{x} | \lift{x}{P} \red M[P]
\end{mathpar}

namely,

\begin{mathpar}
  M^{*}_{x} := x?(u).M[\dropn{u}]
\end{mathpar}

The dependence of $M^{*}_{x}$ on a name makes it an abstraction, 

\begin{mathpar}
  M^{*} := (x)x?(u).M[\dropn{u}]
\end{mathpar}

\subsection{Additional notation}

It will sometimes be convenient to denote the process a name
quotes. We already have the notation $x = \quotep{P}$, but it will be
convenient to introduce an alternate notation, $\procn{x}$, when we
want to emphasize the connection to the use of the name. Note that, by
virtue of name equivalence, $\quotep{\procn{x}} \nameeq x$; so, the
notation is consistent with previous definitions.

Further, because names have structure it is possible to effect
substitutions on the basis of that structure. This means we need to
upgrade our notation for substitutions, which we accomplish by
adapting comprehension notation. Thus,

\begin{mathpar}
  P\{ y / x : x \in S \}
\end{mathpar}

is interpreted to mean the process derived from P by replacing (in a
capture-avoiding manner) each occurrence of $x$ in $S$ by $y$. For example,

\begin{mathpar}
  P\{ \quotep{\procn{x}|\procn{x}} / x : x \in \freenames{P} \}
\end{mathpar}

will replace each (occurrence) of a free name $x$ in $P$ by
$\quotep{\procn{x}|\procn{x}}$.

Also, we will avail ourselves of the notation $x^{L}$ and $x^{R}$ to
denote injections of a name into disjoint copies of the name
space. There are numerous ways to accomplish this. One example can be
found in \cite{MeredithR05}. This notation overloads to vectors of
names: $\vec{x}^{\pi} := (x_{i}^{\pi} \; : \; 0 \leq i < |\vec{x}| )$ where $\pi \in \{L,R\}$.

We also use $P^{\Box} := P|\Box$.

In \cite{MeredithR05} an interpretation of the new operator is
given. It turns out that there are several possible interpretations
all enjoying the requisite algebraic properties of the operator (see
\cite{milner91polyadicpi}). We will therefore make liberal use of
$(\nu\; \vec{x})P$.

% subsection the_syntax_and_semantics_of_the_notation_system (end)   

\input{qm2pi.qmops} 

\input{qm2pi.sterngerlach} 

\input{qm2pi.metric} 

% section concurrent_process_calculi (end)

%\input{qm2pi.proofsketch}

% section proof sketch (end)

%\input{qm2pi.slviaknots} 

% section spatial logic via knots (end)

\input{qm2pi.conclusion}

% section conclusion (end)

%\input{qm2pi.dtcodes} 

% section wiring algorithm (end)

\input{qm2pi.ack} 

% section acknowledgments (end)

\newpage


\bibliographystyle{plain}   
\bibliography{../../biblios/main.bib}

\input{qm2pi.rhodetails}

\end{document}

 

% section acknowledgments (end)

\newpage


\bibliographystyle{plain}   
\bibliography{../../biblios/main.bib}

\documentclass[12pt]{llncs}
%\documentclass{jktr}

\usepackage[pdftex]{hyperref}                   
\usepackage {listings}
\usepackage {mathpartir}
\usepackage{bcprules}
%\usepackage{listings}
                       
\usepackage{graphicx} 
%\usepackage[margins=2.5cm,nohead,nofoot]{geometry}
%\usepackage{geometry}
\usepackage{amsfonts}
\usepackage{amstext}
\usepackage{latexsym}
\usepackage{amssymb}
\usepackage{color}


%\include{myPreamble}
\include{qm2pi.local} 

%\ifpdf
%\usepackage[pdftex]{graphicx}
%\else
%\usepackage{graphicx}
%\fi

 % \ifpdf
%  \usepackage{pdfsync}
%  \if


%\title{Brief Article}
%\author{David F. Snyder}
%\author{L.G. Meredith}

%\address{Dept. of Math., Texas State University--San Marcos, San Marcos, TX 78666}
       
\pagestyle{empty}


\begin{document}

\lstset{language=[Objective]Caml,frame=shadowbox}

\input{qm2pi.front}

% section front matter (end)

\input{qm2pi.intro} 
 
% section introduction (end)

% \input{qm2pi.knotations} 

% section notation (end)

\input{qm2pi.process.calculi} 

% section concurrent_process_calculi_and_spatial_logics_ (end)
    
%\input{qm2pi.knots2pi} 

%\input{qm2pi.trefoil} 

%\input{qm2pi.mainthm} 

% subsection basic_interpretation (end)

%\input{qm2pi.rho.presentation} 
\subsection{The syntax and semantics of the notation system}\label{sub:the_syntax_and_semantics_of_the_notation_system} % (fold)

We now summarize a technical presentation of the calculus that
embodies our theory of dynamics. The typical presentation of such a
calculus follows the style of giving generators and relations on
them. The grammar, below, describing term constructors, freely
generates the set of processes, $\Proc$. This set is then quotiented
by a relation known as structural congruence and it is over this set
that the notion of dynamics is expressed. This presentation is
essentially that of \cite{MeredithR05} with the addition of
polyadicity and summation. For readability we have relegated some of
the technical subtleties to an appendix.

\subsubsection{Process grammar}\label{subsub:process_grammar}

\begin{mathpar}
  \inferrule* [lab=synchronization] {} {{M} \bc \pzero \;|\; x?F \;|\; x!C }
  \and
  \inferrule* [lab=abstraction] {} {{F} \bc (x)P}
  \and
  \inferrule* [lab=concretion] {} {{C} \bc \langle Q \rangle}
  \and
  \inferrule* [lab=process] {} {{P,Q} \bc M \;| \;P|Q \;|\; @{x}}
  \and
  \inferrule* [lab=name] {} {{x} \bc \quotep{P}}
\end{mathpar} 

Note that $\vec{x}$ (resp. $\vec{P}$) denotes a vector of names
(resp. processes) of length $|\vec{x}|$ (resp. $|\vec{P}|$). We adopt
the following useful abbreviations.

\begin{mathpar}
   x?(\vec{y}).P := x.(\vec{y})P \and  x\clift{\vec{P}} := x.\clift{\vec{P}}
   \and x!(y) := \lift{x}{\dropn{y}}
   \and \Pi_{i=0}^{n-1}P_i := P_0 | \ldots | P_{n-1}
\end{mathpar}

\subsubsection{Structural congruence}

\paragraph{Free and bound names and alpha-equivalence.} At the
core of structural equivalence is alpha-equivalence which identifies
process that are the same up to a change of variable. Formally, we
recognize the distinction between free and bound names. The free names
of a process, $\freenames{P}$, may be calculated recursively as
follows:

\begin{mathpar}
\freenames{\pzero} := \emptyset
  \and \\
  \freenames{x?(y).P} := \{ x \} \cup (\freenames{P} \setminus \{ y \})
  \and 
  \freenames{x!\langle P \rangle} := \{ x \} \cup \{ P \} 
  \and \\
  \freenames{P|Q} := \freenames{P} \cup \freenames{Q}
  \and \\
  \freenames{@{x}} := \{ x \}
\end{mathpar}

$\pi$
$\quotep{\pi}$

$\freenames{-} : \pi \to \mathcal{P}(\quotep{\pi})$

\begin{eqnarray*}
  \freenames{\pzero} & := & \emptyset \\
  \freenames{x?(y).P} & := & \{ x \} \cup (\freenames{P} \setminus \{ y \}) \\
  \freenames{x!\langle P \rangle} & := & \{ x \} \cup \{ P \} \\
  \freenames{P|Q} & := & \freenames{P} \cup \freenames{Q} \\
  \freenames{\dropn{x}} & := & \{ x \}
\end{eqnarray*}

The bound names of a process, $\boundnames{P}$, are those names occurring in $P$
that are not free. For example, in $x?(y).0$, the name $x$ is free, while $y$ is bound.

\begin{mathpar}
  \inferrule* [lab=monoidal-laws] {} { P|Q \equiv Q|P \and P|0 \equiv P \and P|(Q|R) \equiv (P|Q)|R }
\end{mathpar}

\begin{mathpar}
  \inferrule* [lab=alpha-equivalence] {} { (x)P \equiv (y)P\{y/x\} \and y \not\in \freenames{P} }
\end{mathpar}

\begin{definition}
Then two processes, $P,Q$, are alpha-equivalent if $P = Q\{\vec{y}/\vec{x}\}$ for
some $\vec{x} \in \boundnames{Q},\vec{y} \in \boundnames{P}$, where $Q\{\vec{y}/\vec{x}\}$
denotes the capture-avoiding substitution of $\vec{y}$ for $\vec{x}$ in $Q$.
\end{definition}

\begin{definition}
  The {\em structural congruence} \cite{SangiorgiWalker} , $\equiv$,
  between processes is the least congruence containing
  alpha-equivalence, satisfying the abelian monoid laws
  (associativity, commutativity and $\pzero$ as identity) for parallel
  composition $|$ and for summation $+$.
\end{definition}

\subsection{Name equivalence}

We take name equivalence, written $\nameeq$, to be the smallest
equivalence relation generated by the following rules.

\begin{mathpar}
\inferrule*[lab=Quote-drop]
{ }
{ \quotep{@{x}} \nameeq x }

\inferrule*[lab=Struct-equiv]
{ P \scong Q }
{ \quotep{P} \nameeq \quotep{Q} }
\end{mathpar}

The astute reader will have noticed that the mutual recursion of names
and processes imposes a mutual recursion on alpha-equivalence and
structural equivalence via name-equivalence. Fortunately, all of this
works out pleasantly and we may calculate in the natural way, free of
concern. The reader interested in the details is referred to the
appendix \ref{appendix:rho_details}.

\subsection{Substitution}

We use $\Proc$ for the set of processes, $\QProc$ for the set of
names, and $\id{\{}\vec{y} / \vec{x} \id{\}}$ to denote partial maps,
$s : \QProc \rightarrow \QProc$. A map, $s$ lifts, uniquely, to a map
on process terms, $\widehat{s} : \Proc \rightarrow \Proc$ by the
following equations.

\begin{mathpar}
  (0) \psubstp{Q}{P} := 0 \\
  (R \juxtap S) \psubstp{Q}{P}
  :=    
  (R)\psubstp{Q}{P} \juxtap (S) \psubstp{Q}{P} \\
  (x?(y).R) \psubstp{Q}{P}    
  :=    
  (x)\substp{Q}{P} (z)\concat( (R \psubstn{z}{y}) \psubstp{Q}{P} ) \\
  (\lift{x}{R}) \psubstp{Q}{P}  
  :=
  \lift{(x)\substp{Q}{P}}{ R \psubstp{Q}{P} } \\
%   (\dropn{x})  \psubstp{Q}{P}       
%   := 
%   \left\{ 
%     \begin{array}{ccc} 
%       \dropn{\quotep{Q}} & & x \nameeq \quotep{P} \\
%       \dropn{x} & & otherwise \\
%     \end{array}
%   \right. 
  (\dropn{x})  \psubstp{Q}{P}       
  := 
  \left\{ 
    \begin{array}{ccc} 
      Q & & x \nameeq \quotep{P} \\
      \dropn{x} & & otherwise \\
    \end{array}
  \right.
\end{mathpar}
 

where

\begin{eqnarray}
  (x)\id{\{} \lpquote Q \rpquote / \lpquote P \rpquote \id{\}}            = 
  \left\{ 
    \begin{array}{ccc}
      \lpquote Q \rpquote & & x \nameeq \lpquote P \rpquote \\
      x & & otherwise \\
    \end{array}
  \right. \nonumber
\end{eqnarray}

and $z$ is chosen distinct from $\quotep{P}$, $\quotep{Q}$, the free
names in $Q$, and all the names in $R$. Our $\alpha$-equivalence will
be built in the standard way from this substitution.

\begin{remark}\label{rem:no_self_referential_names}
  One consequence of these definitions is that $\forall P. \quotep{P}
  \not\in \freenames{P}$.
\end{remark}

\subsection{ Dynamic quote: an example }

Anticipating something of what's to come, consider applying the
substitution, $\widehat{\id{\{}u / z \id{\}}}$, to the following pair
of processes, $\lift{w}{y!(z)}$ and $w[ \lpquote y!(z) \rpquote ]$.

\begin{eqnarray}
	\lift{w}{y!(z)}\widehat{\id{\{}u / z \id{\}}}
		& = &
		\lift{w}{y!(u)} \nonumber\\
	w[ \lpquote y!(z) \rpquote ] \widehat{ \id{\{}u / z \id{\}} }
		& = &
		w[ \lpquote y!(z) \rpquote ] \nonumber
\end{eqnarray}

Because the body of the process between quotes is impervious to
substitution, we get radically different answers. In fact, by
examining the first process in an input context,
e.g. $x?(z).\lift{w}{y!(z)}$, we see that the process under the lift
operator may be shaped by prefixed inputs binding a name inside it. In
this sense, the lift operator will be seen as a way to dynamically
construct processes before reifying them as names.

Finally equipped with these standard features we can present the
dynamics of the calculus.

\subsubsection{Operational semantics} 

Finally, we introduce the computational dynamics. What marks these
algebras as distinct from other more traditionally studied algebraic
structures, e.g. vector spaces or polynomial rings, is the manner in
which dynamics is captured. In traditional structures, dynamics is typically
expressed through morphisms between such structures, as in linear maps
between vector spaces or morphisms between rings. In algebras
associated with the semantics of computation, the dynamics is
expressed as part of the algebraic structure itself, through a
reduction reduction relation typically denoted by $\red$. Below, we
give a recursive presentation of this relation for the calculus used
in the encoding.

$\red \subseteq \pi \times \pi$
$\red : \pi \to \mathcal{P}(\pi)$

\begin{mathpar}
  \inferrule* [lab=Comm] { \textsf{match}( x_{src}, x_{trgt} ) } { x_{trgt}?(y)P \; | \; x_{src}!\langle {Q} \rangle \red P\{\quotep{Q}/y}\} }
  \and \\
  \inferrule* [lab=Par] {{P} \red {P}'} {{{P} | {Q}} \red {{P}' | {Q}}}
  \and
  \inferrule* [lab=Equiv]{{{P} \scong {P}'} \andalso {{P}' \red {Q}'} \andalso {{Q}' \scong {Q}}}{{P} \red {Q}}
\end{mathpar}

\begin{eqnarray*}
  match_{\equiv} (\quotep{P},\quotep{Q}) & := & P \equiv Q \\
  match_{\dagger}(\quotep{P},\quotep{Q}) & := & \forall R. P|Q \red^{*} R => R \red^{*} 0 \\
  match_{K}(\quotep{P},\quotep{Q}) & := & K \mbox{ for some context } K
\end{eqnarray*}

$u?(x)P | u!\langle Q \rangle \red P\{\quotep{Q}/x\}$

%We write $\wred$ for $\red^*$, and $P\red$ if $\exists Q $ such that $ P \red Q$.
We write $P\red$ if $\exists Q $ such that $ P \red Q$ and $P\not\red$, otherwise.

\section{Replication}

As mentioned before, it is known that replication (and hence
recursion) can be implemented in a higher-order process algebra
\cite{SangiorgiWalker}. As our first example of calculation with the
machinery thus far presented we give the construction explicitly in
the {\rhoc}.

\begin{eqnarray}
	D_{x} & := & \prefix{x}{y}{(\binpar{\outputp{x}{y}}{@{y}})} \nonumber\\
	\bangp_{x}{P} & := & \binpar{{x}!\langle{\binpar{D_{x}}{P}}\rangle}{D_{x}} \nonumber
\end{eqnarray}

\begin{eqnarray}
	\bangp_{x}{P} & & \nonumber\\
	=
	& {x}!\langle{(\prefix{x}{y}{(\outputp{x}{y} | @{y})) | P}}\rangle 
	      | \prefix{x}{y}{(\outputp{x}{y} | @{y})} & \nonumber\\
	\red
	& (\outputp{x}{y} | @{y})\substn{\quotep{(\prefix{x}{y}{(@{y} | \outputp{x}{y})) | P}}}{y} & \nonumber\\
	=
	& \outputp{x}{\quotep{(\prefix{x}{y}{(\outputp{x}{y} | @{y})) | P}}}
	  | {(\prefix{x}{y}{(\outputp{x}{y} | @{y})) | P}} & \nonumber\\
	\red
	& \ldots & \nonumber\\
	\red^*
	& P | P | \ldots & \nonumber
\end{eqnarray}

Of course, this encoding, as an implementation, runs away, unfolding
$\bangp{P}$ eagerly. A lazier and more implementable replication
operator, restricted to input-guarded processes, may be obtained as follows.

\begin{eqnarray}
\bangp{\prefix{u}{v}{P}} 
	:= 
	\binpar{\lift{x}{\prefix{u}{v}{(\binpar{D(x)}{P})}}}{D(x)} \nonumber
\end{eqnarray}

\begin{remark}
  Note that the lazier definition still does not deal with summation
  or mixed summation (i.e. sums over input and output). The reader is
  invited to construct definitions of replication that deal with these
  features. 

  Further, the definitions are parameterized in a name, $x$. Can you,
  gentle reader, make a definition that eliminates this parameter and
  guarantees no accidental interaction between the replication
  machinery and the process being replicated -- i.e. no accidental
  sharing of names used by the process to get its work done and the
  name(s) used by the replication to effect copying. This latter
  revision of the definition of replication is crucial to obtaining
  the expected identity $!!P \sim !P$.
\end{remark}

\begin{remark}\label{rem:paradoxical_combinator}
  The reader familiar with the lambda calculus will have noticed the
  similarity between $D$ and the paradoxical combinator.

  [Ed. note: the existence of this seems to suggest we have to be more
  restrictive on the set of processes and names we admit if we are to
  support no-cloning.]
\end{remark}

\subsubsection{Bisimulation}

The computational dynamics gives rise to another kind of equivalence,
the equivalence of computational behavior. As previously mentioned
this is typically captured \emph{via} some form of bisimulation.

% The notion we use in this paper is weak barbed bisimulation
% \cite{milner91polyadicpi}.

The notion we use in this paper is derived from weak barbed
bisimulation \cite{milner91polyadicpi}. 

\begin{definition}
An \emph{observation relation}, $\downarrow_{\mathcal N}$, over a set
of names, $\mathcal N$, is the smallest relation satisfying the rules
below.

\infrule[Out-barb]{y \in {\mathcal N}, \; x \nameeq y}
		  {\outputp{x}{v} \downarrow_{\mathcal N} x}
\infrule[Par-barb]{\mbox{$P\downarrow_{\mathcal N} x$ or $Q\downarrow_{\mathcal N} x$}}
		  {\binpar{P}{Q} \downarrow_{\mathcal N} x}

We write $P \Downarrow_{\mathcal N} x$ if there is $Q$ such that 
$P \wred Q$ and $Q \downarrow_{\mathcal N} x$.
\end{definition}

\begin{definition}
%\label{def.bbisim}
An  ${\mathcal N}$-\emph{barbed bisimulation} over a set of names, ${\mathcal N}$, is a symmetric binary relation 
${\mathcal S}_{\mathcal N}$ between agents such that $P\rel{S}_{\mathcal N}Q$ implies:
\begin{enumerate}
\item If $P \red P'$ then $Q \wred Q'$ and $P'\rel{S}_{\mathcal N} Q'$.
\item If $P\downarrow_{\mathcal N} x$, then $Q\Downarrow_{\mathcal N} x$.
\end{enumerate}
$P$ is ${\mathcal N}$-barbed bisimilar to $Q$, written
$P \wbbisim_{\mathcal N} Q$, if $P \rel{S}_{\mathcal N} Q$ for some ${\mathcal N}$-barbed bisimulation ${\mathcal S}_{\mathcal N}$.
\end{definition}

$\mathcal{R} \subseteq \pi \times \pi$

$P \mathcal{R} Q => \forall P'. P \red P' \Rightarrow \exists Q'. Q \red Q', P' \mathcal{R} Q'$

$P \vdash x \Rightarrow Q \vdash x$

\begin{mathpar}
  \inferrule*[lab=Out-barb]{x \nameeq y}{{y}!\langle{Q}\rangle \vdash x}
  \and
  \inferrule*[lab=Par-barb]{\mbox{$P\vdash x$ or $Q\vdash x$}}{\binpar{P}{Q} \vdash x}
\end{mathpar}

\subsubsection{Contexts}

One of the principle advantages of computational calculi like the
$\pi$-calculus is a well-defined notion of context,
contextual-equivalence and a correlation between
contextual-equivalence and notions of bisimulation. The notion of
context allows the decomposition of a process into (sub-)process and
its syntactic environment, its context. Thus, a context may be
thought of as a process with a ``hole'' (written $\Box$) in it. The
application of a context $M$ to a process $P$, written $M[P]$, is
tantamount to filling the hole in $M$ with $P$. In this paper we do
not need the full weight of this theory, but do make use of the notion
of context in the proof the main theorem. 

\begin{mathpar}
  \inferrule* [lab=summation] {} {{M_{M},M_{N}} \bc \Box \;|\; x.M_{A} \;|\; M_{M}+M_{N}}
  \and
  \inferrule* [lab=agent] {} {{M_{A}} \bc (\vec{x})M_{P} \;| \; \clift{P_0,\ldots,M_{P},\ldots,P_N}}
  \and \\
  \inferrule* [lab=process] {} {{M_{P}} \bc M_{N} \;| \;P|M_{P} }
\end{mathpar} 

\begin{mathpar}
  \inferrule* [lab=sychronization] {} {M_{N} \bc \Box \;|\; x?M_{F} \;|\; x!M_{C}}
  \and
  \inferrule* [lab=abstraction] {} {{M_{F}} \bc (x)M_{P} }
  \and
  \inferrule* [lab=concretion] {} {{M_{C}} \bc \langle M_{P} \rangle }
  \and \\
  \inferrule* [lab=process] {} {{M_{P}} \bc M_{N} \;| \;P|M_{P} }
\end{mathpar}

\begin{definition}[contextual application] Given a context $M$, and
  process $P$, we define the \emph{contextual application}, $M[P] :=
  M\{P/\Box\}$. That is, the contextual application of M to P is the
  substitution of $P$ for $\Box$ in $M$.
\end{definition}

$\meaningof{-} : L \to \mathcal{P}(\pi)$

\begin{mathpar}
  \inferrule* [lab=collection] {} {\meaningof{true} = \pi, \and \meaningof{~E} = \pi \setminus \meaningof{E}, \and \meaningof{E_{1} \& E_{2}} = \meaningof{E_{1}} \cap \meaningof{E_{2}}}
\end{mathpar}

\begin{mathpar}
  \inferrule* [lab=structure] {} {\meaningof{0} = \{ P \in \pi | P \equiv 0 \}, \and \\ \meaningof{E_1 | E_2} = \{ P \in \pi | P \equiv P_{1} | P_{2}, P_{1} \in \meaningof{E_{1}}, P_{2} \in \meaningof{E_2}\} }
\end{mathpar}

\begin{mathpar}
 \inferrule* [lab=behavior] {} {\meaningof{\langle a?b \rangle E} = \{ P \in \pi | P \equiv Q | u?(y)P', \\ \and \\\\ \and \\ \;\;\; u \in \meaningof{a}, \forall z.P'\{z/y\} \in \meaningof{E\{z/b\}}\}, \and \\ \meaningof{a!E} = \{ P \in \pi | P \equiv Q | x!\langle P' \rangle, x \in \meaningof{a} P' \in \meaningof{E}\} }
\end{mathpar}

\begin{mathpar}
 \inferrule* [lab=nominal] {} {\meaningof{\quotep{E}} = \{ \quotep{P} \in \quotep{\pi} | P \in \meaningof{E} \}, \and \meaningof{\quotep{P}} = \{ \quotep{Q} \in \quotep{\pi} | P \equiv Q \} \and \\ \meaningof{@\quotep{E}} = \{ P \in \pi | P \equiv @x, x \in \meaningof{E} \}}
\end{mathpar}

\begin{eqnarray*}
  \\
  \meaningof{-} : TS \to ST
\end{eqnarray*}

\begin{eqnarray*}
  \\
  L : TS \to ST
\end{eqnarray*}

\begin{eqnarray*}
  \\
  P \models E \iff P \in \meaningof{E}
\end{eqnarray*}

\begin{eqnarray*}
  P \approx_{L} Q \iff \forall E \in L. P \models E \iff Q \models E
\end{eqnarray*}

\begin{eqnarray*}
  P \approx_{K} Q
\end{eqnarray*}

\begin{eqnarray*}
  P \approx Q
\end{eqnarray*}

$\approx_{K} = \approx = \approx_{L}$

\subsubsection{Contextual duality}

Note that contexts extend the quotation operation to a family of
operations from processes to names. Given a context, $M$, we can
define a \emph{nominal context}, $\quotep{M}$ by $\quotep{M}[P] :=
\quotep{M[P]}$. To foreshadow what is to come we observe that these
operations enjoy a duality with processes very much like the duality
between vectors and maps from vectors to scalars.

Further, because the calculus is essentially higher-order, we have a
correspondence between contexts and processes. More specifically,
given a name $x$ and a context $M$ we can construct $M^{*}_{x}$ such
that 

\begin{mathpar}
  M^{*}_{x} | \lift{x}{P} \red M[P]
\end{mathpar}

namely,

\begin{mathpar}
  M^{*}_{x} := x?(u).M[\dropn{u}]
\end{mathpar}

The dependence of $M^{*}_{x}$ on a name makes it an abstraction, 

\begin{mathpar}
  M^{*} := (x)x?(u).M[\dropn{u}]
\end{mathpar}

\subsection{Additional notation}

It will sometimes be convenient to denote the process a name
quotes. We already have the notation $x = \quotep{P}$, but it will be
convenient to introduce an alternate notation, $\procn{x}$, when we
want to emphasize the connection to the use of the name. Note that, by
virtue of name equivalence, $\quotep{\procn{x}} \nameeq x$; so, the
notation is consistent with previous definitions.

Further, because names have structure it is possible to effect
substitutions on the basis of that structure. This means we need to
upgrade our notation for substitutions, which we accomplish by
adapting comprehension notation. Thus,

\begin{mathpar}
  P\{ y / x : x \in S \}
\end{mathpar}

is interpreted to mean the process derived from P by replacing (in a
capture-avoiding manner) each occurrence of $x$ in $S$ by $y$. For example,

\begin{mathpar}
  P\{ \quotep{\procn{x}|\procn{x}} / x : x \in \freenames{P} \}
\end{mathpar}

will replace each (occurrence) of a free name $x$ in $P$ by
$\quotep{\procn{x}|\procn{x}}$.

Also, we will avail ourselves of the notation $x^{L}$ and $x^{R}$ to
denote injections of a name into disjoint copies of the name
space. There are numerous ways to accomplish this. One example can be
found in \cite{MeredithR05}. This notation overloads to vectors of
names: $\vec{x}^{\pi} := (x_{i}^{\pi} \; : \; 0 \leq i < |\vec{x}| )$ where $\pi \in \{L,R\}$.

We also use $P^{\Box} := P|\Box$.

In \cite{MeredithR05} an interpretation of the new operator is
given. It turns out that there are several possible interpretations
all enjoying the requisite algebraic properties of the operator (see
\cite{milner91polyadicpi}). We will therefore make liberal use of
$(\nu\; \vec{x})P$.

% subsection the_syntax_and_semantics_of_the_notation_system (end)   

\input{qm2pi.qmops} 

\input{qm2pi.sterngerlach} 

\input{qm2pi.metric} 

% section concurrent_process_calculi (end)

%\input{qm2pi.proofsketch}

% section proof sketch (end)

%\input{qm2pi.slviaknots} 

% section spatial logic via knots (end)

\input{qm2pi.conclusion}

% section conclusion (end)

%\input{qm2pi.dtcodes} 

% section wiring algorithm (end)

\input{qm2pi.ack} 

% section acknowledgments (end)

\newpage


\bibliographystyle{plain}   
\bibliography{../../biblios/main.bib}

\input{qm2pi.rhodetails}

\end{document}



\end{document}

 

%\ifpdf
%\usepackage[pdftex]{graphicx}
%\else
%\usepackage{graphicx}
%\fi

 % \ifpdf
%  \usepackage{pdfsync}
%  \if


%\title{Brief Article}
%\author{David F. Snyder}
%\author{L.G. Meredith}

%\address{Dept. of Math., Texas State University--San Marcos, San Marcos, TX 78666}
       
\pagestyle{empty}


\begin{document}

\lstset{language=[Objective]Caml,frame=shadowbox}

\documentclass[12pt]{llncs}
%\documentclass{jktr}

\usepackage[pdftex]{hyperref}                   
\usepackage {listings}
\usepackage {mathpartir}
\usepackage{bcprules}
%\usepackage{listings}
                       
\usepackage{graphicx} 
%\usepackage[margins=2.5cm,nohead,nofoot]{geometry}
%\usepackage{geometry}
\usepackage{amsfonts}
\usepackage{amstext}
\usepackage{latexsym}
\usepackage{amssymb}
\usepackage{color}


%\include{myPreamble}
\documentclass[12pt]{llncs}
%\documentclass{jktr}

\usepackage[pdftex]{hyperref}                   
\usepackage {listings}
\usepackage {mathpartir}
\usepackage{bcprules}
%\usepackage{listings}
                       
\usepackage{graphicx} 
%\usepackage[margins=2.5cm,nohead,nofoot]{geometry}
%\usepackage{geometry}
\usepackage{amsfonts}
\usepackage{amstext}
\usepackage{latexsym}
\usepackage{amssymb}
\usepackage{color}


%\include{myPreamble}
\include{qm2pi.local} 

%\ifpdf
%\usepackage[pdftex]{graphicx}
%\else
%\usepackage{graphicx}
%\fi

 % \ifpdf
%  \usepackage{pdfsync}
%  \if


%\title{Brief Article}
%\author{David F. Snyder}
%\author{L.G. Meredith}

%\address{Dept. of Math., Texas State University--San Marcos, San Marcos, TX 78666}
       
\pagestyle{empty}


\begin{document}

\lstset{language=[Objective]Caml,frame=shadowbox}

\input{qm2pi.front}

% section front matter (end)

\input{qm2pi.intro} 
 
% section introduction (end)

% \input{qm2pi.knotations} 

% section notation (end)

\input{qm2pi.process.calculi} 

% section concurrent_process_calculi_and_spatial_logics_ (end)
    
%\input{qm2pi.knots2pi} 

%\input{qm2pi.trefoil} 

%\input{qm2pi.mainthm} 

% subsection basic_interpretation (end)

%\input{qm2pi.rho.presentation} 
\subsection{The syntax and semantics of the notation system}\label{sub:the_syntax_and_semantics_of_the_notation_system} % (fold)

We now summarize a technical presentation of the calculus that
embodies our theory of dynamics. The typical presentation of such a
calculus follows the style of giving generators and relations on
them. The grammar, below, describing term constructors, freely
generates the set of processes, $\Proc$. This set is then quotiented
by a relation known as structural congruence and it is over this set
that the notion of dynamics is expressed. This presentation is
essentially that of \cite{MeredithR05} with the addition of
polyadicity and summation. For readability we have relegated some of
the technical subtleties to an appendix.

\subsubsection{Process grammar}\label{subsub:process_grammar}

\begin{mathpar}
  \inferrule* [lab=synchronization] {} {{M} \bc \pzero \;|\; x?F \;|\; x!C }
  \and
  \inferrule* [lab=abstraction] {} {{F} \bc (x)P}
  \and
  \inferrule* [lab=concretion] {} {{C} \bc \langle Q \rangle}
  \and
  \inferrule* [lab=process] {} {{P,Q} \bc M \;| \;P|Q \;|\; @{x}}
  \and
  \inferrule* [lab=name] {} {{x} \bc \quotep{P}}
\end{mathpar} 

Note that $\vec{x}$ (resp. $\vec{P}$) denotes a vector of names
(resp. processes) of length $|\vec{x}|$ (resp. $|\vec{P}|$). We adopt
the following useful abbreviations.

\begin{mathpar}
   x?(\vec{y}).P := x.(\vec{y})P \and  x\clift{\vec{P}} := x.\clift{\vec{P}}
   \and x!(y) := \lift{x}{\dropn{y}}
   \and \Pi_{i=0}^{n-1}P_i := P_0 | \ldots | P_{n-1}
\end{mathpar}

\subsubsection{Structural congruence}

\paragraph{Free and bound names and alpha-equivalence.} At the
core of structural equivalence is alpha-equivalence which identifies
process that are the same up to a change of variable. Formally, we
recognize the distinction between free and bound names. The free names
of a process, $\freenames{P}$, may be calculated recursively as
follows:

\begin{mathpar}
\freenames{\pzero} := \emptyset
  \and \\
  \freenames{x?(y).P} := \{ x \} \cup (\freenames{P} \setminus \{ y \})
  \and 
  \freenames{x!\langle P \rangle} := \{ x \} \cup \{ P \} 
  \and \\
  \freenames{P|Q} := \freenames{P} \cup \freenames{Q}
  \and \\
  \freenames{@{x}} := \{ x \}
\end{mathpar}

$\pi$
$\quotep{\pi}$

$\freenames{-} : \pi \to \mathcal{P}(\quotep{\pi})$

\begin{eqnarray*}
  \freenames{\pzero} & := & \emptyset \\
  \freenames{x?(y).P} & := & \{ x \} \cup (\freenames{P} \setminus \{ y \}) \\
  \freenames{x!\langle P \rangle} & := & \{ x \} \cup \{ P \} \\
  \freenames{P|Q} & := & \freenames{P} \cup \freenames{Q} \\
  \freenames{\dropn{x}} & := & \{ x \}
\end{eqnarray*}

The bound names of a process, $\boundnames{P}$, are those names occurring in $P$
that are not free. For example, in $x?(y).0$, the name $x$ is free, while $y$ is bound.

\begin{mathpar}
  \inferrule* [lab=monoidal-laws] {} { P|Q \equiv Q|P \and P|0 \equiv P \and P|(Q|R) \equiv (P|Q)|R }
\end{mathpar}

\begin{mathpar}
  \inferrule* [lab=alpha-equivalence] {} { (x)P \equiv (y)P\{y/x\} \and y \not\in \freenames{P} }
\end{mathpar}

\begin{definition}
Then two processes, $P,Q$, are alpha-equivalent if $P = Q\{\vec{y}/\vec{x}\}$ for
some $\vec{x} \in \boundnames{Q},\vec{y} \in \boundnames{P}$, where $Q\{\vec{y}/\vec{x}\}$
denotes the capture-avoiding substitution of $\vec{y}$ for $\vec{x}$ in $Q$.
\end{definition}

\begin{definition}
  The {\em structural congruence} \cite{SangiorgiWalker} , $\equiv$,
  between processes is the least congruence containing
  alpha-equivalence, satisfying the abelian monoid laws
  (associativity, commutativity and $\pzero$ as identity) for parallel
  composition $|$ and for summation $+$.
\end{definition}

\subsection{Name equivalence}

We take name equivalence, written $\nameeq$, to be the smallest
equivalence relation generated by the following rules.

\begin{mathpar}
\inferrule*[lab=Quote-drop]
{ }
{ \quotep{@{x}} \nameeq x }

\inferrule*[lab=Struct-equiv]
{ P \scong Q }
{ \quotep{P} \nameeq \quotep{Q} }
\end{mathpar}

The astute reader will have noticed that the mutual recursion of names
and processes imposes a mutual recursion on alpha-equivalence and
structural equivalence via name-equivalence. Fortunately, all of this
works out pleasantly and we may calculate in the natural way, free of
concern. The reader interested in the details is referred to the
appendix \ref{appendix:rho_details}.

\subsection{Substitution}

We use $\Proc$ for the set of processes, $\QProc$ for the set of
names, and $\id{\{}\vec{y} / \vec{x} \id{\}}$ to denote partial maps,
$s : \QProc \rightarrow \QProc$. A map, $s$ lifts, uniquely, to a map
on process terms, $\widehat{s} : \Proc \rightarrow \Proc$ by the
following equations.

\begin{mathpar}
  (0) \psubstp{Q}{P} := 0 \\
  (R \juxtap S) \psubstp{Q}{P}
  :=    
  (R)\psubstp{Q}{P} \juxtap (S) \psubstp{Q}{P} \\
  (x?(y).R) \psubstp{Q}{P}    
  :=    
  (x)\substp{Q}{P} (z)\concat( (R \psubstn{z}{y}) \psubstp{Q}{P} ) \\
  (\lift{x}{R}) \psubstp{Q}{P}  
  :=
  \lift{(x)\substp{Q}{P}}{ R \psubstp{Q}{P} } \\
%   (\dropn{x})  \psubstp{Q}{P}       
%   := 
%   \left\{ 
%     \begin{array}{ccc} 
%       \dropn{\quotep{Q}} & & x \nameeq \quotep{P} \\
%       \dropn{x} & & otherwise \\
%     \end{array}
%   \right. 
  (\dropn{x})  \psubstp{Q}{P}       
  := 
  \left\{ 
    \begin{array}{ccc} 
      Q & & x \nameeq \quotep{P} \\
      \dropn{x} & & otherwise \\
    \end{array}
  \right.
\end{mathpar}
 

where

\begin{eqnarray}
  (x)\id{\{} \lpquote Q \rpquote / \lpquote P \rpquote \id{\}}            = 
  \left\{ 
    \begin{array}{ccc}
      \lpquote Q \rpquote & & x \nameeq \lpquote P \rpquote \\
      x & & otherwise \\
    \end{array}
  \right. \nonumber
\end{eqnarray}

and $z$ is chosen distinct from $\quotep{P}$, $\quotep{Q}$, the free
names in $Q$, and all the names in $R$. Our $\alpha$-equivalence will
be built in the standard way from this substitution.

\begin{remark}\label{rem:no_self_referential_names}
  One consequence of these definitions is that $\forall P. \quotep{P}
  \not\in \freenames{P}$.
\end{remark}

\subsection{ Dynamic quote: an example }

Anticipating something of what's to come, consider applying the
substitution, $\widehat{\id{\{}u / z \id{\}}}$, to the following pair
of processes, $\lift{w}{y!(z)}$ and $w[ \lpquote y!(z) \rpquote ]$.

\begin{eqnarray}
	\lift{w}{y!(z)}\widehat{\id{\{}u / z \id{\}}}
		& = &
		\lift{w}{y!(u)} \nonumber\\
	w[ \lpquote y!(z) \rpquote ] \widehat{ \id{\{}u / z \id{\}} }
		& = &
		w[ \lpquote y!(z) \rpquote ] \nonumber
\end{eqnarray}

Because the body of the process between quotes is impervious to
substitution, we get radically different answers. In fact, by
examining the first process in an input context,
e.g. $x?(z).\lift{w}{y!(z)}$, we see that the process under the lift
operator may be shaped by prefixed inputs binding a name inside it. In
this sense, the lift operator will be seen as a way to dynamically
construct processes before reifying them as names.

Finally equipped with these standard features we can present the
dynamics of the calculus.

\subsubsection{Operational semantics} 

Finally, we introduce the computational dynamics. What marks these
algebras as distinct from other more traditionally studied algebraic
structures, e.g. vector spaces or polynomial rings, is the manner in
which dynamics is captured. In traditional structures, dynamics is typically
expressed through morphisms between such structures, as in linear maps
between vector spaces or morphisms between rings. In algebras
associated with the semantics of computation, the dynamics is
expressed as part of the algebraic structure itself, through a
reduction reduction relation typically denoted by $\red$. Below, we
give a recursive presentation of this relation for the calculus used
in the encoding.

$\red \subseteq \pi \times \pi$
$\red : \pi \to \mathcal{P}(\pi)$

\begin{mathpar}
  \inferrule* [lab=Comm] { \textsf{match}( x_{src}, x_{trgt} ) } { x_{trgt}?(y)P \; | \; x_{src}!\langle {Q} \rangle \red P\{\quotep{Q}/y}\} }
  \and \\
  \inferrule* [lab=Par] {{P} \red {P}'} {{{P} | {Q}} \red {{P}' | {Q}}}
  \and
  \inferrule* [lab=Equiv]{{{P} \scong {P}'} \andalso {{P}' \red {Q}'} \andalso {{Q}' \scong {Q}}}{{P} \red {Q}}
\end{mathpar}

\begin{eqnarray*}
  match_{\equiv} (\quotep{P},\quotep{Q}) & := & P \equiv Q \\
  match_{\dagger}(\quotep{P},\quotep{Q}) & := & \forall R. P|Q \red^{*} R => R \red^{*} 0 \\
  match_{K}(\quotep{P},\quotep{Q}) & := & K \mbox{ for some context } K
\end{eqnarray*}

$u?(x)P | u!\langle Q \rangle \red P\{\quotep{Q}/x\}$

%We write $\wred$ for $\red^*$, and $P\red$ if $\exists Q $ such that $ P \red Q$.
We write $P\red$ if $\exists Q $ such that $ P \red Q$ and $P\not\red$, otherwise.

\section{Replication}

As mentioned before, it is known that replication (and hence
recursion) can be implemented in a higher-order process algebra
\cite{SangiorgiWalker}. As our first example of calculation with the
machinery thus far presented we give the construction explicitly in
the {\rhoc}.

\begin{eqnarray}
	D_{x} & := & \prefix{x}{y}{(\binpar{\outputp{x}{y}}{@{y}})} \nonumber\\
	\bangp_{x}{P} & := & \binpar{{x}!\langle{\binpar{D_{x}}{P}}\rangle}{D_{x}} \nonumber
\end{eqnarray}

\begin{eqnarray}
	\bangp_{x}{P} & & \nonumber\\
	=
	& {x}!\langle{(\prefix{x}{y}{(\outputp{x}{y} | @{y})) | P}}\rangle 
	      | \prefix{x}{y}{(\outputp{x}{y} | @{y})} & \nonumber\\
	\red
	& (\outputp{x}{y} | @{y})\substn{\quotep{(\prefix{x}{y}{(@{y} | \outputp{x}{y})) | P}}}{y} & \nonumber\\
	=
	& \outputp{x}{\quotep{(\prefix{x}{y}{(\outputp{x}{y} | @{y})) | P}}}
	  | {(\prefix{x}{y}{(\outputp{x}{y} | @{y})) | P}} & \nonumber\\
	\red
	& \ldots & \nonumber\\
	\red^*
	& P | P | \ldots & \nonumber
\end{eqnarray}

Of course, this encoding, as an implementation, runs away, unfolding
$\bangp{P}$ eagerly. A lazier and more implementable replication
operator, restricted to input-guarded processes, may be obtained as follows.

\begin{eqnarray}
\bangp{\prefix{u}{v}{P}} 
	:= 
	\binpar{\lift{x}{\prefix{u}{v}{(\binpar{D(x)}{P})}}}{D(x)} \nonumber
\end{eqnarray}

\begin{remark}
  Note that the lazier definition still does not deal with summation
  or mixed summation (i.e. sums over input and output). The reader is
  invited to construct definitions of replication that deal with these
  features. 

  Further, the definitions are parameterized in a name, $x$. Can you,
  gentle reader, make a definition that eliminates this parameter and
  guarantees no accidental interaction between the replication
  machinery and the process being replicated -- i.e. no accidental
  sharing of names used by the process to get its work done and the
  name(s) used by the replication to effect copying. This latter
  revision of the definition of replication is crucial to obtaining
  the expected identity $!!P \sim !P$.
\end{remark}

\begin{remark}\label{rem:paradoxical_combinator}
  The reader familiar with the lambda calculus will have noticed the
  similarity between $D$ and the paradoxical combinator.

  [Ed. note: the existence of this seems to suggest we have to be more
  restrictive on the set of processes and names we admit if we are to
  support no-cloning.]
\end{remark}

\subsubsection{Bisimulation}

The computational dynamics gives rise to another kind of equivalence,
the equivalence of computational behavior. As previously mentioned
this is typically captured \emph{via} some form of bisimulation.

% The notion we use in this paper is weak barbed bisimulation
% \cite{milner91polyadicpi}.

The notion we use in this paper is derived from weak barbed
bisimulation \cite{milner91polyadicpi}. 

\begin{definition}
An \emph{observation relation}, $\downarrow_{\mathcal N}$, over a set
of names, $\mathcal N$, is the smallest relation satisfying the rules
below.

\infrule[Out-barb]{y \in {\mathcal N}, \; x \nameeq y}
		  {\outputp{x}{v} \downarrow_{\mathcal N} x}
\infrule[Par-barb]{\mbox{$P\downarrow_{\mathcal N} x$ or $Q\downarrow_{\mathcal N} x$}}
		  {\binpar{P}{Q} \downarrow_{\mathcal N} x}

We write $P \Downarrow_{\mathcal N} x$ if there is $Q$ such that 
$P \wred Q$ and $Q \downarrow_{\mathcal N} x$.
\end{definition}

\begin{definition}
%\label{def.bbisim}
An  ${\mathcal N}$-\emph{barbed bisimulation} over a set of names, ${\mathcal N}$, is a symmetric binary relation 
${\mathcal S}_{\mathcal N}$ between agents such that $P\rel{S}_{\mathcal N}Q$ implies:
\begin{enumerate}
\item If $P \red P'$ then $Q \wred Q'$ and $P'\rel{S}_{\mathcal N} Q'$.
\item If $P\downarrow_{\mathcal N} x$, then $Q\Downarrow_{\mathcal N} x$.
\end{enumerate}
$P$ is ${\mathcal N}$-barbed bisimilar to $Q$, written
$P \wbbisim_{\mathcal N} Q$, if $P \rel{S}_{\mathcal N} Q$ for some ${\mathcal N}$-barbed bisimulation ${\mathcal S}_{\mathcal N}$.
\end{definition}

$\mathcal{R} \subseteq \pi \times \pi$

$P \mathcal{R} Q => \forall P'. P \red P' \Rightarrow \exists Q'. Q \red Q', P' \mathcal{R} Q'$

$P \vdash x \Rightarrow Q \vdash x$

\begin{mathpar}
  \inferrule*[lab=Out-barb]{x \nameeq y}{{y}!\langle{Q}\rangle \vdash x}
  \and
  \inferrule*[lab=Par-barb]{\mbox{$P\vdash x$ or $Q\vdash x$}}{\binpar{P}{Q} \vdash x}
\end{mathpar}

\subsubsection{Contexts}

One of the principle advantages of computational calculi like the
$\pi$-calculus is a well-defined notion of context,
contextual-equivalence and a correlation between
contextual-equivalence and notions of bisimulation. The notion of
context allows the decomposition of a process into (sub-)process and
its syntactic environment, its context. Thus, a context may be
thought of as a process with a ``hole'' (written $\Box$) in it. The
application of a context $M$ to a process $P$, written $M[P]$, is
tantamount to filling the hole in $M$ with $P$. In this paper we do
not need the full weight of this theory, but do make use of the notion
of context in the proof the main theorem. 

\begin{mathpar}
  \inferrule* [lab=summation] {} {{M_{M},M_{N}} \bc \Box \;|\; x.M_{A} \;|\; M_{M}+M_{N}}
  \and
  \inferrule* [lab=agent] {} {{M_{A}} \bc (\vec{x})M_{P} \;| \; \clift{P_0,\ldots,M_{P},\ldots,P_N}}
  \and \\
  \inferrule* [lab=process] {} {{M_{P}} \bc M_{N} \;| \;P|M_{P} }
\end{mathpar} 

\begin{mathpar}
  \inferrule* [lab=sychronization] {} {M_{N} \bc \Box \;|\; x?M_{F} \;|\; x!M_{C}}
  \and
  \inferrule* [lab=abstraction] {} {{M_{F}} \bc (x)M_{P} }
  \and
  \inferrule* [lab=concretion] {} {{M_{C}} \bc \langle M_{P} \rangle }
  \and \\
  \inferrule* [lab=process] {} {{M_{P}} \bc M_{N} \;| \;P|M_{P} }
\end{mathpar}

\begin{definition}[contextual application] Given a context $M$, and
  process $P$, we define the \emph{contextual application}, $M[P] :=
  M\{P/\Box\}$. That is, the contextual application of M to P is the
  substitution of $P$ for $\Box$ in $M$.
\end{definition}

$\meaningof{-} : L \to \mathcal{P}(\pi)$

\begin{mathpar}
  \inferrule* [lab=collection] {} {\meaningof{true} = \pi, \and \meaningof{~E} = \pi \setminus \meaningof{E}, \and \meaningof{E_{1} \& E_{2}} = \meaningof{E_{1}} \cap \meaningof{E_{2}}}
\end{mathpar}

\begin{mathpar}
  \inferrule* [lab=structure] {} {\meaningof{0} = \{ P \in \pi | P \equiv 0 \}, \and \\ \meaningof{E_1 | E_2} = \{ P \in \pi | P \equiv P_{1} | P_{2}, P_{1} \in \meaningof{E_{1}}, P_{2} \in \meaningof{E_2}\} }
\end{mathpar}

\begin{mathpar}
 \inferrule* [lab=behavior] {} {\meaningof{\langle a?b \rangle E} = \{ P \in \pi | P \equiv Q | u?(y)P', \\ \and \\\\ \and \\ \;\;\; u \in \meaningof{a}, \forall z.P'\{z/y\} \in \meaningof{E\{z/b\}}\}, \and \\ \meaningof{a!E} = \{ P \in \pi | P \equiv Q | x!\langle P' \rangle, x \in \meaningof{a} P' \in \meaningof{E}\} }
\end{mathpar}

\begin{mathpar}
 \inferrule* [lab=nominal] {} {\meaningof{\quotep{E}} = \{ \quotep{P} \in \quotep{\pi} | P \in \meaningof{E} \}, \and \meaningof{\quotep{P}} = \{ \quotep{Q} \in \quotep{\pi} | P \equiv Q \} \and \\ \meaningof{@\quotep{E}} = \{ P \in \pi | P \equiv @x, x \in \meaningof{E} \}}
\end{mathpar}

\begin{eqnarray*}
  \\
  \meaningof{-} : TS \to ST
\end{eqnarray*}

\begin{eqnarray*}
  \\
  L : TS \to ST
\end{eqnarray*}

\begin{eqnarray*}
  \\
  P \models E \iff P \in \meaningof{E}
\end{eqnarray*}

\begin{eqnarray*}
  P \approx_{L} Q \iff \forall E \in L. P \models E \iff Q \models E
\end{eqnarray*}

\begin{eqnarray*}
  P \approx_{K} Q
\end{eqnarray*}

\begin{eqnarray*}
  P \approx Q
\end{eqnarray*}

$\approx_{K} = \approx = \approx_{L}$

\subsubsection{Contextual duality}

Note that contexts extend the quotation operation to a family of
operations from processes to names. Given a context, $M$, we can
define a \emph{nominal context}, $\quotep{M}$ by $\quotep{M}[P] :=
\quotep{M[P]}$. To foreshadow what is to come we observe that these
operations enjoy a duality with processes very much like the duality
between vectors and maps from vectors to scalars.

Further, because the calculus is essentially higher-order, we have a
correspondence between contexts and processes. More specifically,
given a name $x$ and a context $M$ we can construct $M^{*}_{x}$ such
that 

\begin{mathpar}
  M^{*}_{x} | \lift{x}{P} \red M[P]
\end{mathpar}

namely,

\begin{mathpar}
  M^{*}_{x} := x?(u).M[\dropn{u}]
\end{mathpar}

The dependence of $M^{*}_{x}$ on a name makes it an abstraction, 

\begin{mathpar}
  M^{*} := (x)x?(u).M[\dropn{u}]
\end{mathpar}

\subsection{Additional notation}

It will sometimes be convenient to denote the process a name
quotes. We already have the notation $x = \quotep{P}$, but it will be
convenient to introduce an alternate notation, $\procn{x}$, when we
want to emphasize the connection to the use of the name. Note that, by
virtue of name equivalence, $\quotep{\procn{x}} \nameeq x$; so, the
notation is consistent with previous definitions.

Further, because names have structure it is possible to effect
substitutions on the basis of that structure. This means we need to
upgrade our notation for substitutions, which we accomplish by
adapting comprehension notation. Thus,

\begin{mathpar}
  P\{ y / x : x \in S \}
\end{mathpar}

is interpreted to mean the process derived from P by replacing (in a
capture-avoiding manner) each occurrence of $x$ in $S$ by $y$. For example,

\begin{mathpar}
  P\{ \quotep{\procn{x}|\procn{x}} / x : x \in \freenames{P} \}
\end{mathpar}

will replace each (occurrence) of a free name $x$ in $P$ by
$\quotep{\procn{x}|\procn{x}}$.

Also, we will avail ourselves of the notation $x^{L}$ and $x^{R}$ to
denote injections of a name into disjoint copies of the name
space. There are numerous ways to accomplish this. One example can be
found in \cite{MeredithR05}. This notation overloads to vectors of
names: $\vec{x}^{\pi} := (x_{i}^{\pi} \; : \; 0 \leq i < |\vec{x}| )$ where $\pi \in \{L,R\}$.

We also use $P^{\Box} := P|\Box$.

In \cite{MeredithR05} an interpretation of the new operator is
given. It turns out that there are several possible interpretations
all enjoying the requisite algebraic properties of the operator (see
\cite{milner91polyadicpi}). We will therefore make liberal use of
$(\nu\; \vec{x})P$.

% subsection the_syntax_and_semantics_of_the_notation_system (end)   

\input{qm2pi.qmops} 

\input{qm2pi.sterngerlach} 

\input{qm2pi.metric} 

% section concurrent_process_calculi (end)

%\input{qm2pi.proofsketch}

% section proof sketch (end)

%\input{qm2pi.slviaknots} 

% section spatial logic via knots (end)

\input{qm2pi.conclusion}

% section conclusion (end)

%\input{qm2pi.dtcodes} 

% section wiring algorithm (end)

\input{qm2pi.ack} 

% section acknowledgments (end)

\newpage


\bibliographystyle{plain}   
\bibliography{../../biblios/main.bib}

\input{qm2pi.rhodetails}

\end{document}

 

%\ifpdf
%\usepackage[pdftex]{graphicx}
%\else
%\usepackage{graphicx}
%\fi

 % \ifpdf
%  \usepackage{pdfsync}
%  \if


%\title{Brief Article}
%\author{David F. Snyder}
%\author{L.G. Meredith}

%\address{Dept. of Math., Texas State University--San Marcos, San Marcos, TX 78666}
       
\pagestyle{empty}


\begin{document}

\lstset{language=[Objective]Caml,frame=shadowbox}

\documentclass[12pt]{llncs}
%\documentclass{jktr}

\usepackage[pdftex]{hyperref}                   
\usepackage {listings}
\usepackage {mathpartir}
\usepackage{bcprules}
%\usepackage{listings}
                       
\usepackage{graphicx} 
%\usepackage[margins=2.5cm,nohead,nofoot]{geometry}
%\usepackage{geometry}
\usepackage{amsfonts}
\usepackage{amstext}
\usepackage{latexsym}
\usepackage{amssymb}
\usepackage{color}


%\include{myPreamble}
\include{qm2pi.local} 

%\ifpdf
%\usepackage[pdftex]{graphicx}
%\else
%\usepackage{graphicx}
%\fi

 % \ifpdf
%  \usepackage{pdfsync}
%  \if


%\title{Brief Article}
%\author{David F. Snyder}
%\author{L.G. Meredith}

%\address{Dept. of Math., Texas State University--San Marcos, San Marcos, TX 78666}
       
\pagestyle{empty}


\begin{document}

\lstset{language=[Objective]Caml,frame=shadowbox}

\input{qm2pi.front}

% section front matter (end)

\input{qm2pi.intro} 
 
% section introduction (end)

% \input{qm2pi.knotations} 

% section notation (end)

\input{qm2pi.process.calculi} 

% section concurrent_process_calculi_and_spatial_logics_ (end)
    
%\input{qm2pi.knots2pi} 

%\input{qm2pi.trefoil} 

%\input{qm2pi.mainthm} 

% subsection basic_interpretation (end)

%\input{qm2pi.rho.presentation} 
\subsection{The syntax and semantics of the notation system}\label{sub:the_syntax_and_semantics_of_the_notation_system} % (fold)

We now summarize a technical presentation of the calculus that
embodies our theory of dynamics. The typical presentation of such a
calculus follows the style of giving generators and relations on
them. The grammar, below, describing term constructors, freely
generates the set of processes, $\Proc$. This set is then quotiented
by a relation known as structural congruence and it is over this set
that the notion of dynamics is expressed. This presentation is
essentially that of \cite{MeredithR05} with the addition of
polyadicity and summation. For readability we have relegated some of
the technical subtleties to an appendix.

\subsubsection{Process grammar}\label{subsub:process_grammar}

\begin{mathpar}
  \inferrule* [lab=synchronization] {} {{M} \bc \pzero \;|\; x?F \;|\; x!C }
  \and
  \inferrule* [lab=abstraction] {} {{F} \bc (x)P}
  \and
  \inferrule* [lab=concretion] {} {{C} \bc \langle Q \rangle}
  \and
  \inferrule* [lab=process] {} {{P,Q} \bc M \;| \;P|Q \;|\; @{x}}
  \and
  \inferrule* [lab=name] {} {{x} \bc \quotep{P}}
\end{mathpar} 

Note that $\vec{x}$ (resp. $\vec{P}$) denotes a vector of names
(resp. processes) of length $|\vec{x}|$ (resp. $|\vec{P}|$). We adopt
the following useful abbreviations.

\begin{mathpar}
   x?(\vec{y}).P := x.(\vec{y})P \and  x\clift{\vec{P}} := x.\clift{\vec{P}}
   \and x!(y) := \lift{x}{\dropn{y}}
   \and \Pi_{i=0}^{n-1}P_i := P_0 | \ldots | P_{n-1}
\end{mathpar}

\subsubsection{Structural congruence}

\paragraph{Free and bound names and alpha-equivalence.} At the
core of structural equivalence is alpha-equivalence which identifies
process that are the same up to a change of variable. Formally, we
recognize the distinction between free and bound names. The free names
of a process, $\freenames{P}$, may be calculated recursively as
follows:

\begin{mathpar}
\freenames{\pzero} := \emptyset
  \and \\
  \freenames{x?(y).P} := \{ x \} \cup (\freenames{P} \setminus \{ y \})
  \and 
  \freenames{x!\langle P \rangle} := \{ x \} \cup \{ P \} 
  \and \\
  \freenames{P|Q} := \freenames{P} \cup \freenames{Q}
  \and \\
  \freenames{@{x}} := \{ x \}
\end{mathpar}

$\pi$
$\quotep{\pi}$

$\freenames{-} : \pi \to \mathcal{P}(\quotep{\pi})$

\begin{eqnarray*}
  \freenames{\pzero} & := & \emptyset \\
  \freenames{x?(y).P} & := & \{ x \} \cup (\freenames{P} \setminus \{ y \}) \\
  \freenames{x!\langle P \rangle} & := & \{ x \} \cup \{ P \} \\
  \freenames{P|Q} & := & \freenames{P} \cup \freenames{Q} \\
  \freenames{\dropn{x}} & := & \{ x \}
\end{eqnarray*}

The bound names of a process, $\boundnames{P}$, are those names occurring in $P$
that are not free. For example, in $x?(y).0$, the name $x$ is free, while $y$ is bound.

\begin{mathpar}
  \inferrule* [lab=monoidal-laws] {} { P|Q \equiv Q|P \and P|0 \equiv P \and P|(Q|R) \equiv (P|Q)|R }
\end{mathpar}

\begin{mathpar}
  \inferrule* [lab=alpha-equivalence] {} { (x)P \equiv (y)P\{y/x\} \and y \not\in \freenames{P} }
\end{mathpar}

\begin{definition}
Then two processes, $P,Q$, are alpha-equivalent if $P = Q\{\vec{y}/\vec{x}\}$ for
some $\vec{x} \in \boundnames{Q},\vec{y} \in \boundnames{P}$, where $Q\{\vec{y}/\vec{x}\}$
denotes the capture-avoiding substitution of $\vec{y}$ for $\vec{x}$ in $Q$.
\end{definition}

\begin{definition}
  The {\em structural congruence} \cite{SangiorgiWalker} , $\equiv$,
  between processes is the least congruence containing
  alpha-equivalence, satisfying the abelian monoid laws
  (associativity, commutativity and $\pzero$ as identity) for parallel
  composition $|$ and for summation $+$.
\end{definition}

\subsection{Name equivalence}

We take name equivalence, written $\nameeq$, to be the smallest
equivalence relation generated by the following rules.

\begin{mathpar}
\inferrule*[lab=Quote-drop]
{ }
{ \quotep{@{x}} \nameeq x }

\inferrule*[lab=Struct-equiv]
{ P \scong Q }
{ \quotep{P} \nameeq \quotep{Q} }
\end{mathpar}

The astute reader will have noticed that the mutual recursion of names
and processes imposes a mutual recursion on alpha-equivalence and
structural equivalence via name-equivalence. Fortunately, all of this
works out pleasantly and we may calculate in the natural way, free of
concern. The reader interested in the details is referred to the
appendix \ref{appendix:rho_details}.

\subsection{Substitution}

We use $\Proc$ for the set of processes, $\QProc$ for the set of
names, and $\id{\{}\vec{y} / \vec{x} \id{\}}$ to denote partial maps,
$s : \QProc \rightarrow \QProc$. A map, $s$ lifts, uniquely, to a map
on process terms, $\widehat{s} : \Proc \rightarrow \Proc$ by the
following equations.

\begin{mathpar}
  (0) \psubstp{Q}{P} := 0 \\
  (R \juxtap S) \psubstp{Q}{P}
  :=    
  (R)\psubstp{Q}{P} \juxtap (S) \psubstp{Q}{P} \\
  (x?(y).R) \psubstp{Q}{P}    
  :=    
  (x)\substp{Q}{P} (z)\concat( (R \psubstn{z}{y}) \psubstp{Q}{P} ) \\
  (\lift{x}{R}) \psubstp{Q}{P}  
  :=
  \lift{(x)\substp{Q}{P}}{ R \psubstp{Q}{P} } \\
%   (\dropn{x})  \psubstp{Q}{P}       
%   := 
%   \left\{ 
%     \begin{array}{ccc} 
%       \dropn{\quotep{Q}} & & x \nameeq \quotep{P} \\
%       \dropn{x} & & otherwise \\
%     \end{array}
%   \right. 
  (\dropn{x})  \psubstp{Q}{P}       
  := 
  \left\{ 
    \begin{array}{ccc} 
      Q & & x \nameeq \quotep{P} \\
      \dropn{x} & & otherwise \\
    \end{array}
  \right.
\end{mathpar}
 

where

\begin{eqnarray}
  (x)\id{\{} \lpquote Q \rpquote / \lpquote P \rpquote \id{\}}            = 
  \left\{ 
    \begin{array}{ccc}
      \lpquote Q \rpquote & & x \nameeq \lpquote P \rpquote \\
      x & & otherwise \\
    \end{array}
  \right. \nonumber
\end{eqnarray}

and $z$ is chosen distinct from $\quotep{P}$, $\quotep{Q}$, the free
names in $Q$, and all the names in $R$. Our $\alpha$-equivalence will
be built in the standard way from this substitution.

\begin{remark}\label{rem:no_self_referential_names}
  One consequence of these definitions is that $\forall P. \quotep{P}
  \not\in \freenames{P}$.
\end{remark}

\subsection{ Dynamic quote: an example }

Anticipating something of what's to come, consider applying the
substitution, $\widehat{\id{\{}u / z \id{\}}}$, to the following pair
of processes, $\lift{w}{y!(z)}$ and $w[ \lpquote y!(z) \rpquote ]$.

\begin{eqnarray}
	\lift{w}{y!(z)}\widehat{\id{\{}u / z \id{\}}}
		& = &
		\lift{w}{y!(u)} \nonumber\\
	w[ \lpquote y!(z) \rpquote ] \widehat{ \id{\{}u / z \id{\}} }
		& = &
		w[ \lpquote y!(z) \rpquote ] \nonumber
\end{eqnarray}

Because the body of the process between quotes is impervious to
substitution, we get radically different answers. In fact, by
examining the first process in an input context,
e.g. $x?(z).\lift{w}{y!(z)}$, we see that the process under the lift
operator may be shaped by prefixed inputs binding a name inside it. In
this sense, the lift operator will be seen as a way to dynamically
construct processes before reifying them as names.

Finally equipped with these standard features we can present the
dynamics of the calculus.

\subsubsection{Operational semantics} 

Finally, we introduce the computational dynamics. What marks these
algebras as distinct from other more traditionally studied algebraic
structures, e.g. vector spaces or polynomial rings, is the manner in
which dynamics is captured. In traditional structures, dynamics is typically
expressed through morphisms between such structures, as in linear maps
between vector spaces or morphisms between rings. In algebras
associated with the semantics of computation, the dynamics is
expressed as part of the algebraic structure itself, through a
reduction reduction relation typically denoted by $\red$. Below, we
give a recursive presentation of this relation for the calculus used
in the encoding.

$\red \subseteq \pi \times \pi$
$\red : \pi \to \mathcal{P}(\pi)$

\begin{mathpar}
  \inferrule* [lab=Comm] { \textsf{match}( x_{src}, x_{trgt} ) } { x_{trgt}?(y)P \; | \; x_{src}!\langle {Q} \rangle \red P\{\quotep{Q}/y}\} }
  \and \\
  \inferrule* [lab=Par] {{P} \red {P}'} {{{P} | {Q}} \red {{P}' | {Q}}}
  \and
  \inferrule* [lab=Equiv]{{{P} \scong {P}'} \andalso {{P}' \red {Q}'} \andalso {{Q}' \scong {Q}}}{{P} \red {Q}}
\end{mathpar}

\begin{eqnarray*}
  match_{\equiv} (\quotep{P},\quotep{Q}) & := & P \equiv Q \\
  match_{\dagger}(\quotep{P},\quotep{Q}) & := & \forall R. P|Q \red^{*} R => R \red^{*} 0 \\
  match_{K}(\quotep{P},\quotep{Q}) & := & K \mbox{ for some context } K
\end{eqnarray*}

$u?(x)P | u!\langle Q \rangle \red P\{\quotep{Q}/x\}$

%We write $\wred$ for $\red^*$, and $P\red$ if $\exists Q $ such that $ P \red Q$.
We write $P\red$ if $\exists Q $ such that $ P \red Q$ and $P\not\red$, otherwise.

\section{Replication}

As mentioned before, it is known that replication (and hence
recursion) can be implemented in a higher-order process algebra
\cite{SangiorgiWalker}. As our first example of calculation with the
machinery thus far presented we give the construction explicitly in
the {\rhoc}.

\begin{eqnarray}
	D_{x} & := & \prefix{x}{y}{(\binpar{\outputp{x}{y}}{@{y}})} \nonumber\\
	\bangp_{x}{P} & := & \binpar{{x}!\langle{\binpar{D_{x}}{P}}\rangle}{D_{x}} \nonumber
\end{eqnarray}

\begin{eqnarray}
	\bangp_{x}{P} & & \nonumber\\
	=
	& {x}!\langle{(\prefix{x}{y}{(\outputp{x}{y} | @{y})) | P}}\rangle 
	      | \prefix{x}{y}{(\outputp{x}{y} | @{y})} & \nonumber\\
	\red
	& (\outputp{x}{y} | @{y})\substn{\quotep{(\prefix{x}{y}{(@{y} | \outputp{x}{y})) | P}}}{y} & \nonumber\\
	=
	& \outputp{x}{\quotep{(\prefix{x}{y}{(\outputp{x}{y} | @{y})) | P}}}
	  | {(\prefix{x}{y}{(\outputp{x}{y} | @{y})) | P}} & \nonumber\\
	\red
	& \ldots & \nonumber\\
	\red^*
	& P | P | \ldots & \nonumber
\end{eqnarray}

Of course, this encoding, as an implementation, runs away, unfolding
$\bangp{P}$ eagerly. A lazier and more implementable replication
operator, restricted to input-guarded processes, may be obtained as follows.

\begin{eqnarray}
\bangp{\prefix{u}{v}{P}} 
	:= 
	\binpar{\lift{x}{\prefix{u}{v}{(\binpar{D(x)}{P})}}}{D(x)} \nonumber
\end{eqnarray}

\begin{remark}
  Note that the lazier definition still does not deal with summation
  or mixed summation (i.e. sums over input and output). The reader is
  invited to construct definitions of replication that deal with these
  features. 

  Further, the definitions are parameterized in a name, $x$. Can you,
  gentle reader, make a definition that eliminates this parameter and
  guarantees no accidental interaction between the replication
  machinery and the process being replicated -- i.e. no accidental
  sharing of names used by the process to get its work done and the
  name(s) used by the replication to effect copying. This latter
  revision of the definition of replication is crucial to obtaining
  the expected identity $!!P \sim !P$.
\end{remark}

\begin{remark}\label{rem:paradoxical_combinator}
  The reader familiar with the lambda calculus will have noticed the
  similarity between $D$ and the paradoxical combinator.

  [Ed. note: the existence of this seems to suggest we have to be more
  restrictive on the set of processes and names we admit if we are to
  support no-cloning.]
\end{remark}

\subsubsection{Bisimulation}

The computational dynamics gives rise to another kind of equivalence,
the equivalence of computational behavior. As previously mentioned
this is typically captured \emph{via} some form of bisimulation.

% The notion we use in this paper is weak barbed bisimulation
% \cite{milner91polyadicpi}.

The notion we use in this paper is derived from weak barbed
bisimulation \cite{milner91polyadicpi}. 

\begin{definition}
An \emph{observation relation}, $\downarrow_{\mathcal N}$, over a set
of names, $\mathcal N$, is the smallest relation satisfying the rules
below.

\infrule[Out-barb]{y \in {\mathcal N}, \; x \nameeq y}
		  {\outputp{x}{v} \downarrow_{\mathcal N} x}
\infrule[Par-barb]{\mbox{$P\downarrow_{\mathcal N} x$ or $Q\downarrow_{\mathcal N} x$}}
		  {\binpar{P}{Q} \downarrow_{\mathcal N} x}

We write $P \Downarrow_{\mathcal N} x$ if there is $Q$ such that 
$P \wred Q$ and $Q \downarrow_{\mathcal N} x$.
\end{definition}

\begin{definition}
%\label{def.bbisim}
An  ${\mathcal N}$-\emph{barbed bisimulation} over a set of names, ${\mathcal N}$, is a symmetric binary relation 
${\mathcal S}_{\mathcal N}$ between agents such that $P\rel{S}_{\mathcal N}Q$ implies:
\begin{enumerate}
\item If $P \red P'$ then $Q \wred Q'$ and $P'\rel{S}_{\mathcal N} Q'$.
\item If $P\downarrow_{\mathcal N} x$, then $Q\Downarrow_{\mathcal N} x$.
\end{enumerate}
$P$ is ${\mathcal N}$-barbed bisimilar to $Q$, written
$P \wbbisim_{\mathcal N} Q$, if $P \rel{S}_{\mathcal N} Q$ for some ${\mathcal N}$-barbed bisimulation ${\mathcal S}_{\mathcal N}$.
\end{definition}

$\mathcal{R} \subseteq \pi \times \pi$

$P \mathcal{R} Q => \forall P'. P \red P' \Rightarrow \exists Q'. Q \red Q', P' \mathcal{R} Q'$

$P \vdash x \Rightarrow Q \vdash x$

\begin{mathpar}
  \inferrule*[lab=Out-barb]{x \nameeq y}{{y}!\langle{Q}\rangle \vdash x}
  \and
  \inferrule*[lab=Par-barb]{\mbox{$P\vdash x$ or $Q\vdash x$}}{\binpar{P}{Q} \vdash x}
\end{mathpar}

\subsubsection{Contexts}

One of the principle advantages of computational calculi like the
$\pi$-calculus is a well-defined notion of context,
contextual-equivalence and a correlation between
contextual-equivalence and notions of bisimulation. The notion of
context allows the decomposition of a process into (sub-)process and
its syntactic environment, its context. Thus, a context may be
thought of as a process with a ``hole'' (written $\Box$) in it. The
application of a context $M$ to a process $P$, written $M[P]$, is
tantamount to filling the hole in $M$ with $P$. In this paper we do
not need the full weight of this theory, but do make use of the notion
of context in the proof the main theorem. 

\begin{mathpar}
  \inferrule* [lab=summation] {} {{M_{M},M_{N}} \bc \Box \;|\; x.M_{A} \;|\; M_{M}+M_{N}}
  \and
  \inferrule* [lab=agent] {} {{M_{A}} \bc (\vec{x})M_{P} \;| \; \clift{P_0,\ldots,M_{P},\ldots,P_N}}
  \and \\
  \inferrule* [lab=process] {} {{M_{P}} \bc M_{N} \;| \;P|M_{P} }
\end{mathpar} 

\begin{mathpar}
  \inferrule* [lab=sychronization] {} {M_{N} \bc \Box \;|\; x?M_{F} \;|\; x!M_{C}}
  \and
  \inferrule* [lab=abstraction] {} {{M_{F}} \bc (x)M_{P} }
  \and
  \inferrule* [lab=concretion] {} {{M_{C}} \bc \langle M_{P} \rangle }
  \and \\
  \inferrule* [lab=process] {} {{M_{P}} \bc M_{N} \;| \;P|M_{P} }
\end{mathpar}

\begin{definition}[contextual application] Given a context $M$, and
  process $P$, we define the \emph{contextual application}, $M[P] :=
  M\{P/\Box\}$. That is, the contextual application of M to P is the
  substitution of $P$ for $\Box$ in $M$.
\end{definition}

$\meaningof{-} : L \to \mathcal{P}(\pi)$

\begin{mathpar}
  \inferrule* [lab=collection] {} {\meaningof{true} = \pi, \and \meaningof{~E} = \pi \setminus \meaningof{E}, \and \meaningof{E_{1} \& E_{2}} = \meaningof{E_{1}} \cap \meaningof{E_{2}}}
\end{mathpar}

\begin{mathpar}
  \inferrule* [lab=structure] {} {\meaningof{0} = \{ P \in \pi | P \equiv 0 \}, \and \\ \meaningof{E_1 | E_2} = \{ P \in \pi | P \equiv P_{1} | P_{2}, P_{1} \in \meaningof{E_{1}}, P_{2} \in \meaningof{E_2}\} }
\end{mathpar}

\begin{mathpar}
 \inferrule* [lab=behavior] {} {\meaningof{\langle a?b \rangle E} = \{ P \in \pi | P \equiv Q | u?(y)P', \\ \and \\\\ \and \\ \;\;\; u \in \meaningof{a}, \forall z.P'\{z/y\} \in \meaningof{E\{z/b\}}\}, \and \\ \meaningof{a!E} = \{ P \in \pi | P \equiv Q | x!\langle P' \rangle, x \in \meaningof{a} P' \in \meaningof{E}\} }
\end{mathpar}

\begin{mathpar}
 \inferrule* [lab=nominal] {} {\meaningof{\quotep{E}} = \{ \quotep{P} \in \quotep{\pi} | P \in \meaningof{E} \}, \and \meaningof{\quotep{P}} = \{ \quotep{Q} \in \quotep{\pi} | P \equiv Q \} \and \\ \meaningof{@\quotep{E}} = \{ P \in \pi | P \equiv @x, x \in \meaningof{E} \}}
\end{mathpar}

\begin{eqnarray*}
  \\
  \meaningof{-} : TS \to ST
\end{eqnarray*}

\begin{eqnarray*}
  \\
  L : TS \to ST
\end{eqnarray*}

\begin{eqnarray*}
  \\
  P \models E \iff P \in \meaningof{E}
\end{eqnarray*}

\begin{eqnarray*}
  P \approx_{L} Q \iff \forall E \in L. P \models E \iff Q \models E
\end{eqnarray*}

\begin{eqnarray*}
  P \approx_{K} Q
\end{eqnarray*}

\begin{eqnarray*}
  P \approx Q
\end{eqnarray*}

$\approx_{K} = \approx = \approx_{L}$

\subsubsection{Contextual duality}

Note that contexts extend the quotation operation to a family of
operations from processes to names. Given a context, $M$, we can
define a \emph{nominal context}, $\quotep{M}$ by $\quotep{M}[P] :=
\quotep{M[P]}$. To foreshadow what is to come we observe that these
operations enjoy a duality with processes very much like the duality
between vectors and maps from vectors to scalars.

Further, because the calculus is essentially higher-order, we have a
correspondence between contexts and processes. More specifically,
given a name $x$ and a context $M$ we can construct $M^{*}_{x}$ such
that 

\begin{mathpar}
  M^{*}_{x} | \lift{x}{P} \red M[P]
\end{mathpar}

namely,

\begin{mathpar}
  M^{*}_{x} := x?(u).M[\dropn{u}]
\end{mathpar}

The dependence of $M^{*}_{x}$ on a name makes it an abstraction, 

\begin{mathpar}
  M^{*} := (x)x?(u).M[\dropn{u}]
\end{mathpar}

\subsection{Additional notation}

It will sometimes be convenient to denote the process a name
quotes. We already have the notation $x = \quotep{P}$, but it will be
convenient to introduce an alternate notation, $\procn{x}$, when we
want to emphasize the connection to the use of the name. Note that, by
virtue of name equivalence, $\quotep{\procn{x}} \nameeq x$; so, the
notation is consistent with previous definitions.

Further, because names have structure it is possible to effect
substitutions on the basis of that structure. This means we need to
upgrade our notation for substitutions, which we accomplish by
adapting comprehension notation. Thus,

\begin{mathpar}
  P\{ y / x : x \in S \}
\end{mathpar}

is interpreted to mean the process derived from P by replacing (in a
capture-avoiding manner) each occurrence of $x$ in $S$ by $y$. For example,

\begin{mathpar}
  P\{ \quotep{\procn{x}|\procn{x}} / x : x \in \freenames{P} \}
\end{mathpar}

will replace each (occurrence) of a free name $x$ in $P$ by
$\quotep{\procn{x}|\procn{x}}$.

Also, we will avail ourselves of the notation $x^{L}$ and $x^{R}$ to
denote injections of a name into disjoint copies of the name
space. There are numerous ways to accomplish this. One example can be
found in \cite{MeredithR05}. This notation overloads to vectors of
names: $\vec{x}^{\pi} := (x_{i}^{\pi} \; : \; 0 \leq i < |\vec{x}| )$ where $\pi \in \{L,R\}$.

We also use $P^{\Box} := P|\Box$.

In \cite{MeredithR05} an interpretation of the new operator is
given. It turns out that there are several possible interpretations
all enjoying the requisite algebraic properties of the operator (see
\cite{milner91polyadicpi}). We will therefore make liberal use of
$(\nu\; \vec{x})P$.

% subsection the_syntax_and_semantics_of_the_notation_system (end)   

\input{qm2pi.qmops} 

\input{qm2pi.sterngerlach} 

\input{qm2pi.metric} 

% section concurrent_process_calculi (end)

%\input{qm2pi.proofsketch}

% section proof sketch (end)

%\input{qm2pi.slviaknots} 

% section spatial logic via knots (end)

\input{qm2pi.conclusion}

% section conclusion (end)

%\input{qm2pi.dtcodes} 

% section wiring algorithm (end)

\input{qm2pi.ack} 

% section acknowledgments (end)

\newpage


\bibliographystyle{plain}   
\bibliography{../../biblios/main.bib}

\input{qm2pi.rhodetails}

\end{document}



% section front matter (end)

\section{Introduction}\label{sec:introduction} % (fold)
In this draft of the material i am going to have to dispense with the
usual writing conventions adopted in papers on these topics. i'm going
to have adopt whatever tone i need at the time i'm writing up the
calculations. Sometimes this may be very conversational; others it may
be the barest mathematical grunts; others still it may be that i have
lifted text from one of my other papers because the exposition of some
point was better said there. i hope that my readers are not unduly put
out by this decision. i'm not doing this to flout convention or be
rebellious. i find these calculations very technically challenging. To
keep everything going technically, something has to give; i have to
let go of some cognitive burden. So, the academic writing style --
with all of its trade-offs in terms of facilitating technical
communication -- is what i'm letting go of. Perhaps subsequent drafts
can be tightened and polished, but for now, i'm going to speak as if
we were sitting together in a coffee shop with a laptop, wifi and a
pad of paper and a pencil.

So, here's what i have to say. We -- you and i, comfortably ensconced
in our coffee shop and well-equipped with our tools -- can realize and
carry out the calculations of quantum mechanics over a very different
formal theory of dynamics, a formal theory of dynamics that
corresponds to a theory of concurrent computation with
\emph{reflection}. It has the advantage that the underlying theory is
already `quantized', but supports analogues all of the continuuous
operations. Strikingly, this underlying theory has recently been
connected with a notion of metric that we can show, by calculating
together, coincides with the metric induced by the inner product.

There are a lot of reasons why you might be interested in seeing
calculations of this form. Here's why i'm interested. For the past
several centuries there has been no competitor to the ``Newtonian''
account of dynamics. As a result the predominant share of accounts of
dynamical systems and situations have had to be formulated in terms of
the Newtonian machinery. i view this as an intellectually dangerous
position to occupy. Everything, despite it's intrinsic shape, turns
into a nail to be hit with this hammer. Recently, however, the theory
of computation has matured to the point where we have candidates for
theories of dynamics that offer very different perspective on
reasoning about dynamical systems and situations. Testing these
candidates against very successful accounts of dynamical situations,
like quantum mechanics, is going to give us some sense of how mature
they are and some measure of the quality of these accounts of
dynamics.

\subsection{Summary of contributions and outline of paper}

So, we're going to develop an interpretation of the operations of
quantum mechanics normally interpreted by Hilbert spaces and
operators. We're going to do this over a theory of computation. Note
that this is very different than the usual quantum computation program
which develops notions of computation over quantum mechanics. Rather,
we are developing a story that aligns with Wheeler's slogan: It from
Bit. To do this we will first provide an account of the theory of
computation at play here. Then we will dive into a calculation-driven
interpretation of the operations of quantum mechanics.

The reason we take this approach is that -- until very recently --
there hasn't been an axiomatic account of quantum mechanics. As a
result there has been no sharp delineation of the mathematical theory
supporting interpretation of the physical theory and the physical
theory, itself. So, ambient features of the maths are free to be
exploited (or supressed) without a real accounting of their physical
relevance. There is no sharp statement ``here's the physical theory''
qua \emph{theory} and ``here's the mathematical interpretation''
enabling a judgment of how faithful the interpretation is -- apart
from experimental observation. When there is an axiomatic account we
can judge how well a given mathematical formalism supports an
interpretation of the axioms, independent of
experimentation. Likewise, we can judge how well we have captured our
physical evidence and experience with our axiomatics, independent of
any specific mathematical implementation, with accidental detail that
may or may not have physical significance. 

In lieu of a fully fleshed out and vetted axiomatic account of quantum
mechanics, interpreting the operational notions in service of modeling
physical systems will have to suffice. In other words, we are not in
the business of providing a model of Hilbert spaces and operators. We
are in the business of providing a model of quantum mechanics because
we are motivated by testing our notions of dynamics against physical
theory; and, the predictive calculations of the physical theory must
serve as the best formulation -- shy of a fully fleshed out axiomatic
account -- of the physical theory itself (as they have for scientific
theories since time immemorial). Put another way, despite a
whole-hearted commitment to an It-from-Bit ontology, we are firmly
aligned with the shut-up-and-calculate camp as the best way to obtain
results either from the physical perspective or as a quality assurance
measure of our fledgling theory of dynamics.

In detail, we present a reflective process calculus. Then we develop
intuitive correspondences between the notions available in this
calculus and the usual physical notions supporting quantum mechanical
calculations. Thus, 

\begin{table}[htp]
  \center{
    \fbox{
      \begin{tabular}{c|c}
        quantum mechanics & process calculus \\
        \hline
        scalar & name \\
        state vector & process \\
        dual & contextual duals \\
        matrix & formal sums of process-context-dual pairs \\
        orthogonality & process annihilation \\
        inner product & execution-formula + quoting
      \end{tabular}
    }
  }
  \caption{QM - process calculi correspondences}
\end{table}

Then we tighten up these intuitions to operational definitions. We
employ the Dirac notation as the best proxy we can find for an
abstract syntax of the quantum mechanical notions. The definitions we
develop put us in contact with equational constraints coming from the
theory that we demonstrate the definitions and calculations satisfy.

This puts us in a position to shut up and calculate for the
Stern-Gerlach experimental set up, showing how these predictive
calculations become calculations on processes in our theory of a
reflective process calculus.

Penultimately, we demonstrate that the notion of metric coming from
the inner product coincides with the notion of metric available from
the theory of bisimulation. This demonstration gives us the right to
think of space as arising from behavior. Finally, we consider where we
might go from the new vantage point we have obtained.

% section introduction (end) 
 
% section introduction (end)

% \documentclass[12pt]{llncs}
%\documentclass{jktr}

\usepackage[pdftex]{hyperref}                   
\usepackage {listings}
\usepackage {mathpartir}
\usepackage{bcprules}
%\usepackage{listings}
                       
\usepackage{graphicx} 
%\usepackage[margins=2.5cm,nohead,nofoot]{geometry}
%\usepackage{geometry}
\usepackage{amsfonts}
\usepackage{amstext}
\usepackage{latexsym}
\usepackage{amssymb}
\usepackage{color}


%\include{myPreamble}
\include{qm2pi.local} 

%\ifpdf
%\usepackage[pdftex]{graphicx}
%\else
%\usepackage{graphicx}
%\fi

 % \ifpdf
%  \usepackage{pdfsync}
%  \if


%\title{Brief Article}
%\author{David F. Snyder}
%\author{L.G. Meredith}

%\address{Dept. of Math., Texas State University--San Marcos, San Marcos, TX 78666}
       
\pagestyle{empty}


\begin{document}

\lstset{language=[Objective]Caml,frame=shadowbox}

\input{qm2pi.front}

% section front matter (end)

\input{qm2pi.intro} 
 
% section introduction (end)

% \input{qm2pi.knotations} 

% section notation (end)

\input{qm2pi.process.calculi} 

% section concurrent_process_calculi_and_spatial_logics_ (end)
    
%\input{qm2pi.knots2pi} 

%\input{qm2pi.trefoil} 

%\input{qm2pi.mainthm} 

% subsection basic_interpretation (end)

%\input{qm2pi.rho.presentation} 
\subsection{The syntax and semantics of the notation system}\label{sub:the_syntax_and_semantics_of_the_notation_system} % (fold)

We now summarize a technical presentation of the calculus that
embodies our theory of dynamics. The typical presentation of such a
calculus follows the style of giving generators and relations on
them. The grammar, below, describing term constructors, freely
generates the set of processes, $\Proc$. This set is then quotiented
by a relation known as structural congruence and it is over this set
that the notion of dynamics is expressed. This presentation is
essentially that of \cite{MeredithR05} with the addition of
polyadicity and summation. For readability we have relegated some of
the technical subtleties to an appendix.

\subsubsection{Process grammar}\label{subsub:process_grammar}

\begin{mathpar}
  \inferrule* [lab=synchronization] {} {{M} \bc \pzero \;|\; x?F \;|\; x!C }
  \and
  \inferrule* [lab=abstraction] {} {{F} \bc (x)P}
  \and
  \inferrule* [lab=concretion] {} {{C} \bc \langle Q \rangle}
  \and
  \inferrule* [lab=process] {} {{P,Q} \bc M \;| \;P|Q \;|\; @{x}}
  \and
  \inferrule* [lab=name] {} {{x} \bc \quotep{P}}
\end{mathpar} 

Note that $\vec{x}$ (resp. $\vec{P}$) denotes a vector of names
(resp. processes) of length $|\vec{x}|$ (resp. $|\vec{P}|$). We adopt
the following useful abbreviations.

\begin{mathpar}
   x?(\vec{y}).P := x.(\vec{y})P \and  x\clift{\vec{P}} := x.\clift{\vec{P}}
   \and x!(y) := \lift{x}{\dropn{y}}
   \and \Pi_{i=0}^{n-1}P_i := P_0 | \ldots | P_{n-1}
\end{mathpar}

\subsubsection{Structural congruence}

\paragraph{Free and bound names and alpha-equivalence.} At the
core of structural equivalence is alpha-equivalence which identifies
process that are the same up to a change of variable. Formally, we
recognize the distinction between free and bound names. The free names
of a process, $\freenames{P}$, may be calculated recursively as
follows:

\begin{mathpar}
\freenames{\pzero} := \emptyset
  \and \\
  \freenames{x?(y).P} := \{ x \} \cup (\freenames{P} \setminus \{ y \})
  \and 
  \freenames{x!\langle P \rangle} := \{ x \} \cup \{ P \} 
  \and \\
  \freenames{P|Q} := \freenames{P} \cup \freenames{Q}
  \and \\
  \freenames{@{x}} := \{ x \}
\end{mathpar}

$\pi$
$\quotep{\pi}$

$\freenames{-} : \pi \to \mathcal{P}(\quotep{\pi})$

\begin{eqnarray*}
  \freenames{\pzero} & := & \emptyset \\
  \freenames{x?(y).P} & := & \{ x \} \cup (\freenames{P} \setminus \{ y \}) \\
  \freenames{x!\langle P \rangle} & := & \{ x \} \cup \{ P \} \\
  \freenames{P|Q} & := & \freenames{P} \cup \freenames{Q} \\
  \freenames{\dropn{x}} & := & \{ x \}
\end{eqnarray*}

The bound names of a process, $\boundnames{P}$, are those names occurring in $P$
that are not free. For example, in $x?(y).0$, the name $x$ is free, while $y$ is bound.

\begin{mathpar}
  \inferrule* [lab=monoidal-laws] {} { P|Q \equiv Q|P \and P|0 \equiv P \and P|(Q|R) \equiv (P|Q)|R }
\end{mathpar}

\begin{mathpar}
  \inferrule* [lab=alpha-equivalence] {} { (x)P \equiv (y)P\{y/x\} \and y \not\in \freenames{P} }
\end{mathpar}

\begin{definition}
Then two processes, $P,Q$, are alpha-equivalent if $P = Q\{\vec{y}/\vec{x}\}$ for
some $\vec{x} \in \boundnames{Q},\vec{y} \in \boundnames{P}$, where $Q\{\vec{y}/\vec{x}\}$
denotes the capture-avoiding substitution of $\vec{y}$ for $\vec{x}$ in $Q$.
\end{definition}

\begin{definition}
  The {\em structural congruence} \cite{SangiorgiWalker} , $\equiv$,
  between processes is the least congruence containing
  alpha-equivalence, satisfying the abelian monoid laws
  (associativity, commutativity and $\pzero$ as identity) for parallel
  composition $|$ and for summation $+$.
\end{definition}

\subsection{Name equivalence}

We take name equivalence, written $\nameeq$, to be the smallest
equivalence relation generated by the following rules.

\begin{mathpar}
\inferrule*[lab=Quote-drop]
{ }
{ \quotep{@{x}} \nameeq x }

\inferrule*[lab=Struct-equiv]
{ P \scong Q }
{ \quotep{P} \nameeq \quotep{Q} }
\end{mathpar}

The astute reader will have noticed that the mutual recursion of names
and processes imposes a mutual recursion on alpha-equivalence and
structural equivalence via name-equivalence. Fortunately, all of this
works out pleasantly and we may calculate in the natural way, free of
concern. The reader interested in the details is referred to the
appendix \ref{appendix:rho_details}.

\subsection{Substitution}

We use $\Proc$ for the set of processes, $\QProc$ for the set of
names, and $\id{\{}\vec{y} / \vec{x} \id{\}}$ to denote partial maps,
$s : \QProc \rightarrow \QProc$. A map, $s$ lifts, uniquely, to a map
on process terms, $\widehat{s} : \Proc \rightarrow \Proc$ by the
following equations.

\begin{mathpar}
  (0) \psubstp{Q}{P} := 0 \\
  (R \juxtap S) \psubstp{Q}{P}
  :=    
  (R)\psubstp{Q}{P} \juxtap (S) \psubstp{Q}{P} \\
  (x?(y).R) \psubstp{Q}{P}    
  :=    
  (x)\substp{Q}{P} (z)\concat( (R \psubstn{z}{y}) \psubstp{Q}{P} ) \\
  (\lift{x}{R}) \psubstp{Q}{P}  
  :=
  \lift{(x)\substp{Q}{P}}{ R \psubstp{Q}{P} } \\
%   (\dropn{x})  \psubstp{Q}{P}       
%   := 
%   \left\{ 
%     \begin{array}{ccc} 
%       \dropn{\quotep{Q}} & & x \nameeq \quotep{P} \\
%       \dropn{x} & & otherwise \\
%     \end{array}
%   \right. 
  (\dropn{x})  \psubstp{Q}{P}       
  := 
  \left\{ 
    \begin{array}{ccc} 
      Q & & x \nameeq \quotep{P} \\
      \dropn{x} & & otherwise \\
    \end{array}
  \right.
\end{mathpar}
 

where

\begin{eqnarray}
  (x)\id{\{} \lpquote Q \rpquote / \lpquote P \rpquote \id{\}}            = 
  \left\{ 
    \begin{array}{ccc}
      \lpquote Q \rpquote & & x \nameeq \lpquote P \rpquote \\
      x & & otherwise \\
    \end{array}
  \right. \nonumber
\end{eqnarray}

and $z$ is chosen distinct from $\quotep{P}$, $\quotep{Q}$, the free
names in $Q$, and all the names in $R$. Our $\alpha$-equivalence will
be built in the standard way from this substitution.

\begin{remark}\label{rem:no_self_referential_names}
  One consequence of these definitions is that $\forall P. \quotep{P}
  \not\in \freenames{P}$.
\end{remark}

\subsection{ Dynamic quote: an example }

Anticipating something of what's to come, consider applying the
substitution, $\widehat{\id{\{}u / z \id{\}}}$, to the following pair
of processes, $\lift{w}{y!(z)}$ and $w[ \lpquote y!(z) \rpquote ]$.

\begin{eqnarray}
	\lift{w}{y!(z)}\widehat{\id{\{}u / z \id{\}}}
		& = &
		\lift{w}{y!(u)} \nonumber\\
	w[ \lpquote y!(z) \rpquote ] \widehat{ \id{\{}u / z \id{\}} }
		& = &
		w[ \lpquote y!(z) \rpquote ] \nonumber
\end{eqnarray}

Because the body of the process between quotes is impervious to
substitution, we get radically different answers. In fact, by
examining the first process in an input context,
e.g. $x?(z).\lift{w}{y!(z)}$, we see that the process under the lift
operator may be shaped by prefixed inputs binding a name inside it. In
this sense, the lift operator will be seen as a way to dynamically
construct processes before reifying them as names.

Finally equipped with these standard features we can present the
dynamics of the calculus.

\subsubsection{Operational semantics} 

Finally, we introduce the computational dynamics. What marks these
algebras as distinct from other more traditionally studied algebraic
structures, e.g. vector spaces or polynomial rings, is the manner in
which dynamics is captured. In traditional structures, dynamics is typically
expressed through morphisms between such structures, as in linear maps
between vector spaces or morphisms between rings. In algebras
associated with the semantics of computation, the dynamics is
expressed as part of the algebraic structure itself, through a
reduction reduction relation typically denoted by $\red$. Below, we
give a recursive presentation of this relation for the calculus used
in the encoding.

$\red \subseteq \pi \times \pi$
$\red : \pi \to \mathcal{P}(\pi)$

\begin{mathpar}
  \inferrule* [lab=Comm] { \textsf{match}( x_{src}, x_{trgt} ) } { x_{trgt}?(y)P \; | \; x_{src}!\langle {Q} \rangle \red P\{\quotep{Q}/y}\} }
  \and \\
  \inferrule* [lab=Par] {{P} \red {P}'} {{{P} | {Q}} \red {{P}' | {Q}}}
  \and
  \inferrule* [lab=Equiv]{{{P} \scong {P}'} \andalso {{P}' \red {Q}'} \andalso {{Q}' \scong {Q}}}{{P} \red {Q}}
\end{mathpar}

\begin{eqnarray*}
  match_{\equiv} (\quotep{P},\quotep{Q}) & := & P \equiv Q \\
  match_{\dagger}(\quotep{P},\quotep{Q}) & := & \forall R. P|Q \red^{*} R => R \red^{*} 0 \\
  match_{K}(\quotep{P},\quotep{Q}) & := & K \mbox{ for some context } K
\end{eqnarray*}

$u?(x)P | u!\langle Q \rangle \red P\{\quotep{Q}/x\}$

%We write $\wred$ for $\red^*$, and $P\red$ if $\exists Q $ such that $ P \red Q$.
We write $P\red$ if $\exists Q $ such that $ P \red Q$ and $P\not\red$, otherwise.

\section{Replication}

As mentioned before, it is known that replication (and hence
recursion) can be implemented in a higher-order process algebra
\cite{SangiorgiWalker}. As our first example of calculation with the
machinery thus far presented we give the construction explicitly in
the {\rhoc}.

\begin{eqnarray}
	D_{x} & := & \prefix{x}{y}{(\binpar{\outputp{x}{y}}{@{y}})} \nonumber\\
	\bangp_{x}{P} & := & \binpar{{x}!\langle{\binpar{D_{x}}{P}}\rangle}{D_{x}} \nonumber
\end{eqnarray}

\begin{eqnarray}
	\bangp_{x}{P} & & \nonumber\\
	=
	& {x}!\langle{(\prefix{x}{y}{(\outputp{x}{y} | @{y})) | P}}\rangle 
	      | \prefix{x}{y}{(\outputp{x}{y} | @{y})} & \nonumber\\
	\red
	& (\outputp{x}{y} | @{y})\substn{\quotep{(\prefix{x}{y}{(@{y} | \outputp{x}{y})) | P}}}{y} & \nonumber\\
	=
	& \outputp{x}{\quotep{(\prefix{x}{y}{(\outputp{x}{y} | @{y})) | P}}}
	  | {(\prefix{x}{y}{(\outputp{x}{y} | @{y})) | P}} & \nonumber\\
	\red
	& \ldots & \nonumber\\
	\red^*
	& P | P | \ldots & \nonumber
\end{eqnarray}

Of course, this encoding, as an implementation, runs away, unfolding
$\bangp{P}$ eagerly. A lazier and more implementable replication
operator, restricted to input-guarded processes, may be obtained as follows.

\begin{eqnarray}
\bangp{\prefix{u}{v}{P}} 
	:= 
	\binpar{\lift{x}{\prefix{u}{v}{(\binpar{D(x)}{P})}}}{D(x)} \nonumber
\end{eqnarray}

\begin{remark}
  Note that the lazier definition still does not deal with summation
  or mixed summation (i.e. sums over input and output). The reader is
  invited to construct definitions of replication that deal with these
  features. 

  Further, the definitions are parameterized in a name, $x$. Can you,
  gentle reader, make a definition that eliminates this parameter and
  guarantees no accidental interaction between the replication
  machinery and the process being replicated -- i.e. no accidental
  sharing of names used by the process to get its work done and the
  name(s) used by the replication to effect copying. This latter
  revision of the definition of replication is crucial to obtaining
  the expected identity $!!P \sim !P$.
\end{remark}

\begin{remark}\label{rem:paradoxical_combinator}
  The reader familiar with the lambda calculus will have noticed the
  similarity between $D$ and the paradoxical combinator.

  [Ed. note: the existence of this seems to suggest we have to be more
  restrictive on the set of processes and names we admit if we are to
  support no-cloning.]
\end{remark}

\subsubsection{Bisimulation}

The computational dynamics gives rise to another kind of equivalence,
the equivalence of computational behavior. As previously mentioned
this is typically captured \emph{via} some form of bisimulation.

% The notion we use in this paper is weak barbed bisimulation
% \cite{milner91polyadicpi}.

The notion we use in this paper is derived from weak barbed
bisimulation \cite{milner91polyadicpi}. 

\begin{definition}
An \emph{observation relation}, $\downarrow_{\mathcal N}$, over a set
of names, $\mathcal N$, is the smallest relation satisfying the rules
below.

\infrule[Out-barb]{y \in {\mathcal N}, \; x \nameeq y}
		  {\outputp{x}{v} \downarrow_{\mathcal N} x}
\infrule[Par-barb]{\mbox{$P\downarrow_{\mathcal N} x$ or $Q\downarrow_{\mathcal N} x$}}
		  {\binpar{P}{Q} \downarrow_{\mathcal N} x}

We write $P \Downarrow_{\mathcal N} x$ if there is $Q$ such that 
$P \wred Q$ and $Q \downarrow_{\mathcal N} x$.
\end{definition}

\begin{definition}
%\label{def.bbisim}
An  ${\mathcal N}$-\emph{barbed bisimulation} over a set of names, ${\mathcal N}$, is a symmetric binary relation 
${\mathcal S}_{\mathcal N}$ between agents such that $P\rel{S}_{\mathcal N}Q$ implies:
\begin{enumerate}
\item If $P \red P'$ then $Q \wred Q'$ and $P'\rel{S}_{\mathcal N} Q'$.
\item If $P\downarrow_{\mathcal N} x$, then $Q\Downarrow_{\mathcal N} x$.
\end{enumerate}
$P$ is ${\mathcal N}$-barbed bisimilar to $Q$, written
$P \wbbisim_{\mathcal N} Q$, if $P \rel{S}_{\mathcal N} Q$ for some ${\mathcal N}$-barbed bisimulation ${\mathcal S}_{\mathcal N}$.
\end{definition}

$\mathcal{R} \subseteq \pi \times \pi$

$P \mathcal{R} Q => \forall P'. P \red P' \Rightarrow \exists Q'. Q \red Q', P' \mathcal{R} Q'$

$P \vdash x \Rightarrow Q \vdash x$

\begin{mathpar}
  \inferrule*[lab=Out-barb]{x \nameeq y}{{y}!\langle{Q}\rangle \vdash x}
  \and
  \inferrule*[lab=Par-barb]{\mbox{$P\vdash x$ or $Q\vdash x$}}{\binpar{P}{Q} \vdash x}
\end{mathpar}

\subsubsection{Contexts}

One of the principle advantages of computational calculi like the
$\pi$-calculus is a well-defined notion of context,
contextual-equivalence and a correlation between
contextual-equivalence and notions of bisimulation. The notion of
context allows the decomposition of a process into (sub-)process and
its syntactic environment, its context. Thus, a context may be
thought of as a process with a ``hole'' (written $\Box$) in it. The
application of a context $M$ to a process $P$, written $M[P]$, is
tantamount to filling the hole in $M$ with $P$. In this paper we do
not need the full weight of this theory, but do make use of the notion
of context in the proof the main theorem. 

\begin{mathpar}
  \inferrule* [lab=summation] {} {{M_{M},M_{N}} \bc \Box \;|\; x.M_{A} \;|\; M_{M}+M_{N}}
  \and
  \inferrule* [lab=agent] {} {{M_{A}} \bc (\vec{x})M_{P} \;| \; \clift{P_0,\ldots,M_{P},\ldots,P_N}}
  \and \\
  \inferrule* [lab=process] {} {{M_{P}} \bc M_{N} \;| \;P|M_{P} }
\end{mathpar} 

\begin{mathpar}
  \inferrule* [lab=sychronization] {} {M_{N} \bc \Box \;|\; x?M_{F} \;|\; x!M_{C}}
  \and
  \inferrule* [lab=abstraction] {} {{M_{F}} \bc (x)M_{P} }
  \and
  \inferrule* [lab=concretion] {} {{M_{C}} \bc \langle M_{P} \rangle }
  \and \\
  \inferrule* [lab=process] {} {{M_{P}} \bc M_{N} \;| \;P|M_{P} }
\end{mathpar}

\begin{definition}[contextual application] Given a context $M$, and
  process $P$, we define the \emph{contextual application}, $M[P] :=
  M\{P/\Box\}$. That is, the contextual application of M to P is the
  substitution of $P$ for $\Box$ in $M$.
\end{definition}

$\meaningof{-} : L \to \mathcal{P}(\pi)$

\begin{mathpar}
  \inferrule* [lab=collection] {} {\meaningof{true} = \pi, \and \meaningof{~E} = \pi \setminus \meaningof{E}, \and \meaningof{E_{1} \& E_{2}} = \meaningof{E_{1}} \cap \meaningof{E_{2}}}
\end{mathpar}

\begin{mathpar}
  \inferrule* [lab=structure] {} {\meaningof{0} = \{ P \in \pi | P \equiv 0 \}, \and \\ \meaningof{E_1 | E_2} = \{ P \in \pi | P \equiv P_{1} | P_{2}, P_{1} \in \meaningof{E_{1}}, P_{2} \in \meaningof{E_2}\} }
\end{mathpar}

\begin{mathpar}
 \inferrule* [lab=behavior] {} {\meaningof{\langle a?b \rangle E} = \{ P \in \pi | P \equiv Q | u?(y)P', \\ \and \\\\ \and \\ \;\;\; u \in \meaningof{a}, \forall z.P'\{z/y\} \in \meaningof{E\{z/b\}}\}, \and \\ \meaningof{a!E} = \{ P \in \pi | P \equiv Q | x!\langle P' \rangle, x \in \meaningof{a} P' \in \meaningof{E}\} }
\end{mathpar}

\begin{mathpar}
 \inferrule* [lab=nominal] {} {\meaningof{\quotep{E}} = \{ \quotep{P} \in \quotep{\pi} | P \in \meaningof{E} \}, \and \meaningof{\quotep{P}} = \{ \quotep{Q} \in \quotep{\pi} | P \equiv Q \} \and \\ \meaningof{@\quotep{E}} = \{ P \in \pi | P \equiv @x, x \in \meaningof{E} \}}
\end{mathpar}

\begin{eqnarray*}
  \\
  \meaningof{-} : TS \to ST
\end{eqnarray*}

\begin{eqnarray*}
  \\
  L : TS \to ST
\end{eqnarray*}

\begin{eqnarray*}
  \\
  P \models E \iff P \in \meaningof{E}
\end{eqnarray*}

\begin{eqnarray*}
  P \approx_{L} Q \iff \forall E \in L. P \models E \iff Q \models E
\end{eqnarray*}

\begin{eqnarray*}
  P \approx_{K} Q
\end{eqnarray*}

\begin{eqnarray*}
  P \approx Q
\end{eqnarray*}

$\approx_{K} = \approx = \approx_{L}$

\subsubsection{Contextual duality}

Note that contexts extend the quotation operation to a family of
operations from processes to names. Given a context, $M$, we can
define a \emph{nominal context}, $\quotep{M}$ by $\quotep{M}[P] :=
\quotep{M[P]}$. To foreshadow what is to come we observe that these
operations enjoy a duality with processes very much like the duality
between vectors and maps from vectors to scalars.

Further, because the calculus is essentially higher-order, we have a
correspondence between contexts and processes. More specifically,
given a name $x$ and a context $M$ we can construct $M^{*}_{x}$ such
that 

\begin{mathpar}
  M^{*}_{x} | \lift{x}{P} \red M[P]
\end{mathpar}

namely,

\begin{mathpar}
  M^{*}_{x} := x?(u).M[\dropn{u}]
\end{mathpar}

The dependence of $M^{*}_{x}$ on a name makes it an abstraction, 

\begin{mathpar}
  M^{*} := (x)x?(u).M[\dropn{u}]
\end{mathpar}

\subsection{Additional notation}

It will sometimes be convenient to denote the process a name
quotes. We already have the notation $x = \quotep{P}$, but it will be
convenient to introduce an alternate notation, $\procn{x}$, when we
want to emphasize the connection to the use of the name. Note that, by
virtue of name equivalence, $\quotep{\procn{x}} \nameeq x$; so, the
notation is consistent with previous definitions.

Further, because names have structure it is possible to effect
substitutions on the basis of that structure. This means we need to
upgrade our notation for substitutions, which we accomplish by
adapting comprehension notation. Thus,

\begin{mathpar}
  P\{ y / x : x \in S \}
\end{mathpar}

is interpreted to mean the process derived from P by replacing (in a
capture-avoiding manner) each occurrence of $x$ in $S$ by $y$. For example,

\begin{mathpar}
  P\{ \quotep{\procn{x}|\procn{x}} / x : x \in \freenames{P} \}
\end{mathpar}

will replace each (occurrence) of a free name $x$ in $P$ by
$\quotep{\procn{x}|\procn{x}}$.

Also, we will avail ourselves of the notation $x^{L}$ and $x^{R}$ to
denote injections of a name into disjoint copies of the name
space. There are numerous ways to accomplish this. One example can be
found in \cite{MeredithR05}. This notation overloads to vectors of
names: $\vec{x}^{\pi} := (x_{i}^{\pi} \; : \; 0 \leq i < |\vec{x}| )$ where $\pi \in \{L,R\}$.

We also use $P^{\Box} := P|\Box$.

In \cite{MeredithR05} an interpretation of the new operator is
given. It turns out that there are several possible interpretations
all enjoying the requisite algebraic properties of the operator (see
\cite{milner91polyadicpi}). We will therefore make liberal use of
$(\nu\; \vec{x})P$.

% subsection the_syntax_and_semantics_of_the_notation_system (end)   

\input{qm2pi.qmops} 

\input{qm2pi.sterngerlach} 

\input{qm2pi.metric} 

% section concurrent_process_calculi (end)

%\input{qm2pi.proofsketch}

% section proof sketch (end)

%\input{qm2pi.slviaknots} 

% section spatial logic via knots (end)

\input{qm2pi.conclusion}

% section conclusion (end)

%\input{qm2pi.dtcodes} 

% section wiring algorithm (end)

\input{qm2pi.ack} 

% section acknowledgments (end)

\newpage


\bibliographystyle{plain}   
\bibliography{../../biblios/main.bib}

\input{qm2pi.rhodetails}

\end{document}

 

% section notation (end)

\input{qm2pi.process.calculi} 

% section concurrent_process_calculi_and_spatial_logics_ (end)
    
%\documentclass[12pt]{llncs}
%\documentclass{jktr}

\usepackage[pdftex]{hyperref}                   
\usepackage {listings}
\usepackage {mathpartir}
\usepackage{bcprules}
%\usepackage{listings}
                       
\usepackage{graphicx} 
%\usepackage[margins=2.5cm,nohead,nofoot]{geometry}
%\usepackage{geometry}
\usepackage{amsfonts}
\usepackage{amstext}
\usepackage{latexsym}
\usepackage{amssymb}
\usepackage{color}


%\include{myPreamble}
\include{qm2pi.local} 

%\ifpdf
%\usepackage[pdftex]{graphicx}
%\else
%\usepackage{graphicx}
%\fi

 % \ifpdf
%  \usepackage{pdfsync}
%  \if


%\title{Brief Article}
%\author{David F. Snyder}
%\author{L.G. Meredith}

%\address{Dept. of Math., Texas State University--San Marcos, San Marcos, TX 78666}
       
\pagestyle{empty}


\begin{document}

\lstset{language=[Objective]Caml,frame=shadowbox}

\input{qm2pi.front}

% section front matter (end)

\input{qm2pi.intro} 
 
% section introduction (end)

% \input{qm2pi.knotations} 

% section notation (end)

\input{qm2pi.process.calculi} 

% section concurrent_process_calculi_and_spatial_logics_ (end)
    
%\input{qm2pi.knots2pi} 

%\input{qm2pi.trefoil} 

%\input{qm2pi.mainthm} 

% subsection basic_interpretation (end)

%\input{qm2pi.rho.presentation} 
\subsection{The syntax and semantics of the notation system}\label{sub:the_syntax_and_semantics_of_the_notation_system} % (fold)

We now summarize a technical presentation of the calculus that
embodies our theory of dynamics. The typical presentation of such a
calculus follows the style of giving generators and relations on
them. The grammar, below, describing term constructors, freely
generates the set of processes, $\Proc$. This set is then quotiented
by a relation known as structural congruence and it is over this set
that the notion of dynamics is expressed. This presentation is
essentially that of \cite{MeredithR05} with the addition of
polyadicity and summation. For readability we have relegated some of
the technical subtleties to an appendix.

\subsubsection{Process grammar}\label{subsub:process_grammar}

\begin{mathpar}
  \inferrule* [lab=synchronization] {} {{M} \bc \pzero \;|\; x?F \;|\; x!C }
  \and
  \inferrule* [lab=abstraction] {} {{F} \bc (x)P}
  \and
  \inferrule* [lab=concretion] {} {{C} \bc \langle Q \rangle}
  \and
  \inferrule* [lab=process] {} {{P,Q} \bc M \;| \;P|Q \;|\; @{x}}
  \and
  \inferrule* [lab=name] {} {{x} \bc \quotep{P}}
\end{mathpar} 

Note that $\vec{x}$ (resp. $\vec{P}$) denotes a vector of names
(resp. processes) of length $|\vec{x}|$ (resp. $|\vec{P}|$). We adopt
the following useful abbreviations.

\begin{mathpar}
   x?(\vec{y}).P := x.(\vec{y})P \and  x\clift{\vec{P}} := x.\clift{\vec{P}}
   \and x!(y) := \lift{x}{\dropn{y}}
   \and \Pi_{i=0}^{n-1}P_i := P_0 | \ldots | P_{n-1}
\end{mathpar}

\subsubsection{Structural congruence}

\paragraph{Free and bound names and alpha-equivalence.} At the
core of structural equivalence is alpha-equivalence which identifies
process that are the same up to a change of variable. Formally, we
recognize the distinction between free and bound names. The free names
of a process, $\freenames{P}$, may be calculated recursively as
follows:

\begin{mathpar}
\freenames{\pzero} := \emptyset
  \and \\
  \freenames{x?(y).P} := \{ x \} \cup (\freenames{P} \setminus \{ y \})
  \and 
  \freenames{x!\langle P \rangle} := \{ x \} \cup \{ P \} 
  \and \\
  \freenames{P|Q} := \freenames{P} \cup \freenames{Q}
  \and \\
  \freenames{@{x}} := \{ x \}
\end{mathpar}

$\pi$
$\quotep{\pi}$

$\freenames{-} : \pi \to \mathcal{P}(\quotep{\pi})$

\begin{eqnarray*}
  \freenames{\pzero} & := & \emptyset \\
  \freenames{x?(y).P} & := & \{ x \} \cup (\freenames{P} \setminus \{ y \}) \\
  \freenames{x!\langle P \rangle} & := & \{ x \} \cup \{ P \} \\
  \freenames{P|Q} & := & \freenames{P} \cup \freenames{Q} \\
  \freenames{\dropn{x}} & := & \{ x \}
\end{eqnarray*}

The bound names of a process, $\boundnames{P}$, are those names occurring in $P$
that are not free. For example, in $x?(y).0$, the name $x$ is free, while $y$ is bound.

\begin{mathpar}
  \inferrule* [lab=monoidal-laws] {} { P|Q \equiv Q|P \and P|0 \equiv P \and P|(Q|R) \equiv (P|Q)|R }
\end{mathpar}

\begin{mathpar}
  \inferrule* [lab=alpha-equivalence] {} { (x)P \equiv (y)P\{y/x\} \and y \not\in \freenames{P} }
\end{mathpar}

\begin{definition}
Then two processes, $P,Q$, are alpha-equivalent if $P = Q\{\vec{y}/\vec{x}\}$ for
some $\vec{x} \in \boundnames{Q},\vec{y} \in \boundnames{P}$, where $Q\{\vec{y}/\vec{x}\}$
denotes the capture-avoiding substitution of $\vec{y}$ for $\vec{x}$ in $Q$.
\end{definition}

\begin{definition}
  The {\em structural congruence} \cite{SangiorgiWalker} , $\equiv$,
  between processes is the least congruence containing
  alpha-equivalence, satisfying the abelian monoid laws
  (associativity, commutativity and $\pzero$ as identity) for parallel
  composition $|$ and for summation $+$.
\end{definition}

\subsection{Name equivalence}

We take name equivalence, written $\nameeq$, to be the smallest
equivalence relation generated by the following rules.

\begin{mathpar}
\inferrule*[lab=Quote-drop]
{ }
{ \quotep{@{x}} \nameeq x }

\inferrule*[lab=Struct-equiv]
{ P \scong Q }
{ \quotep{P} \nameeq \quotep{Q} }
\end{mathpar}

The astute reader will have noticed that the mutual recursion of names
and processes imposes a mutual recursion on alpha-equivalence and
structural equivalence via name-equivalence. Fortunately, all of this
works out pleasantly and we may calculate in the natural way, free of
concern. The reader interested in the details is referred to the
appendix \ref{appendix:rho_details}.

\subsection{Substitution}

We use $\Proc$ for the set of processes, $\QProc$ for the set of
names, and $\id{\{}\vec{y} / \vec{x} \id{\}}$ to denote partial maps,
$s : \QProc \rightarrow \QProc$. A map, $s$ lifts, uniquely, to a map
on process terms, $\widehat{s} : \Proc \rightarrow \Proc$ by the
following equations.

\begin{mathpar}
  (0) \psubstp{Q}{P} := 0 \\
  (R \juxtap S) \psubstp{Q}{P}
  :=    
  (R)\psubstp{Q}{P} \juxtap (S) \psubstp{Q}{P} \\
  (x?(y).R) \psubstp{Q}{P}    
  :=    
  (x)\substp{Q}{P} (z)\concat( (R \psubstn{z}{y}) \psubstp{Q}{P} ) \\
  (\lift{x}{R}) \psubstp{Q}{P}  
  :=
  \lift{(x)\substp{Q}{P}}{ R \psubstp{Q}{P} } \\
%   (\dropn{x})  \psubstp{Q}{P}       
%   := 
%   \left\{ 
%     \begin{array}{ccc} 
%       \dropn{\quotep{Q}} & & x \nameeq \quotep{P} \\
%       \dropn{x} & & otherwise \\
%     \end{array}
%   \right. 
  (\dropn{x})  \psubstp{Q}{P}       
  := 
  \left\{ 
    \begin{array}{ccc} 
      Q & & x \nameeq \quotep{P} \\
      \dropn{x} & & otherwise \\
    \end{array}
  \right.
\end{mathpar}
 

where

\begin{eqnarray}
  (x)\id{\{} \lpquote Q \rpquote / \lpquote P \rpquote \id{\}}            = 
  \left\{ 
    \begin{array}{ccc}
      \lpquote Q \rpquote & & x \nameeq \lpquote P \rpquote \\
      x & & otherwise \\
    \end{array}
  \right. \nonumber
\end{eqnarray}

and $z$ is chosen distinct from $\quotep{P}$, $\quotep{Q}$, the free
names in $Q$, and all the names in $R$. Our $\alpha$-equivalence will
be built in the standard way from this substitution.

\begin{remark}\label{rem:no_self_referential_names}
  One consequence of these definitions is that $\forall P. \quotep{P}
  \not\in \freenames{P}$.
\end{remark}

\subsection{ Dynamic quote: an example }

Anticipating something of what's to come, consider applying the
substitution, $\widehat{\id{\{}u / z \id{\}}}$, to the following pair
of processes, $\lift{w}{y!(z)}$ and $w[ \lpquote y!(z) \rpquote ]$.

\begin{eqnarray}
	\lift{w}{y!(z)}\widehat{\id{\{}u / z \id{\}}}
		& = &
		\lift{w}{y!(u)} \nonumber\\
	w[ \lpquote y!(z) \rpquote ] \widehat{ \id{\{}u / z \id{\}} }
		& = &
		w[ \lpquote y!(z) \rpquote ] \nonumber
\end{eqnarray}

Because the body of the process between quotes is impervious to
substitution, we get radically different answers. In fact, by
examining the first process in an input context,
e.g. $x?(z).\lift{w}{y!(z)}$, we see that the process under the lift
operator may be shaped by prefixed inputs binding a name inside it. In
this sense, the lift operator will be seen as a way to dynamically
construct processes before reifying them as names.

Finally equipped with these standard features we can present the
dynamics of the calculus.

\subsubsection{Operational semantics} 

Finally, we introduce the computational dynamics. What marks these
algebras as distinct from other more traditionally studied algebraic
structures, e.g. vector spaces or polynomial rings, is the manner in
which dynamics is captured. In traditional structures, dynamics is typically
expressed through morphisms between such structures, as in linear maps
between vector spaces or morphisms between rings. In algebras
associated with the semantics of computation, the dynamics is
expressed as part of the algebraic structure itself, through a
reduction reduction relation typically denoted by $\red$. Below, we
give a recursive presentation of this relation for the calculus used
in the encoding.

$\red \subseteq \pi \times \pi$
$\red : \pi \to \mathcal{P}(\pi)$

\begin{mathpar}
  \inferrule* [lab=Comm] { \textsf{match}( x_{src}, x_{trgt} ) } { x_{trgt}?(y)P \; | \; x_{src}!\langle {Q} \rangle \red P\{\quotep{Q}/y}\} }
  \and \\
  \inferrule* [lab=Par] {{P} \red {P}'} {{{P} | {Q}} \red {{P}' | {Q}}}
  \and
  \inferrule* [lab=Equiv]{{{P} \scong {P}'} \andalso {{P}' \red {Q}'} \andalso {{Q}' \scong {Q}}}{{P} \red {Q}}
\end{mathpar}

\begin{eqnarray*}
  match_{\equiv} (\quotep{P},\quotep{Q}) & := & P \equiv Q \\
  match_{\dagger}(\quotep{P},\quotep{Q}) & := & \forall R. P|Q \red^{*} R => R \red^{*} 0 \\
  match_{K}(\quotep{P},\quotep{Q}) & := & K \mbox{ for some context } K
\end{eqnarray*}

$u?(x)P | u!\langle Q \rangle \red P\{\quotep{Q}/x\}$

%We write $\wred$ for $\red^*$, and $P\red$ if $\exists Q $ such that $ P \red Q$.
We write $P\red$ if $\exists Q $ such that $ P \red Q$ and $P\not\red$, otherwise.

\section{Replication}

As mentioned before, it is known that replication (and hence
recursion) can be implemented in a higher-order process algebra
\cite{SangiorgiWalker}. As our first example of calculation with the
machinery thus far presented we give the construction explicitly in
the {\rhoc}.

\begin{eqnarray}
	D_{x} & := & \prefix{x}{y}{(\binpar{\outputp{x}{y}}{@{y}})} \nonumber\\
	\bangp_{x}{P} & := & \binpar{{x}!\langle{\binpar{D_{x}}{P}}\rangle}{D_{x}} \nonumber
\end{eqnarray}

\begin{eqnarray}
	\bangp_{x}{P} & & \nonumber\\
	=
	& {x}!\langle{(\prefix{x}{y}{(\outputp{x}{y} | @{y})) | P}}\rangle 
	      | \prefix{x}{y}{(\outputp{x}{y} | @{y})} & \nonumber\\
	\red
	& (\outputp{x}{y} | @{y})\substn{\quotep{(\prefix{x}{y}{(@{y} | \outputp{x}{y})) | P}}}{y} & \nonumber\\
	=
	& \outputp{x}{\quotep{(\prefix{x}{y}{(\outputp{x}{y} | @{y})) | P}}}
	  | {(\prefix{x}{y}{(\outputp{x}{y} | @{y})) | P}} & \nonumber\\
	\red
	& \ldots & \nonumber\\
	\red^*
	& P | P | \ldots & \nonumber
\end{eqnarray}

Of course, this encoding, as an implementation, runs away, unfolding
$\bangp{P}$ eagerly. A lazier and more implementable replication
operator, restricted to input-guarded processes, may be obtained as follows.

\begin{eqnarray}
\bangp{\prefix{u}{v}{P}} 
	:= 
	\binpar{\lift{x}{\prefix{u}{v}{(\binpar{D(x)}{P})}}}{D(x)} \nonumber
\end{eqnarray}

\begin{remark}
  Note that the lazier definition still does not deal with summation
  or mixed summation (i.e. sums over input and output). The reader is
  invited to construct definitions of replication that deal with these
  features. 

  Further, the definitions are parameterized in a name, $x$. Can you,
  gentle reader, make a definition that eliminates this parameter and
  guarantees no accidental interaction between the replication
  machinery and the process being replicated -- i.e. no accidental
  sharing of names used by the process to get its work done and the
  name(s) used by the replication to effect copying. This latter
  revision of the definition of replication is crucial to obtaining
  the expected identity $!!P \sim !P$.
\end{remark}

\begin{remark}\label{rem:paradoxical_combinator}
  The reader familiar with the lambda calculus will have noticed the
  similarity between $D$ and the paradoxical combinator.

  [Ed. note: the existence of this seems to suggest we have to be more
  restrictive on the set of processes and names we admit if we are to
  support no-cloning.]
\end{remark}

\subsubsection{Bisimulation}

The computational dynamics gives rise to another kind of equivalence,
the equivalence of computational behavior. As previously mentioned
this is typically captured \emph{via} some form of bisimulation.

% The notion we use in this paper is weak barbed bisimulation
% \cite{milner91polyadicpi}.

The notion we use in this paper is derived from weak barbed
bisimulation \cite{milner91polyadicpi}. 

\begin{definition}
An \emph{observation relation}, $\downarrow_{\mathcal N}$, over a set
of names, $\mathcal N$, is the smallest relation satisfying the rules
below.

\infrule[Out-barb]{y \in {\mathcal N}, \; x \nameeq y}
		  {\outputp{x}{v} \downarrow_{\mathcal N} x}
\infrule[Par-barb]{\mbox{$P\downarrow_{\mathcal N} x$ or $Q\downarrow_{\mathcal N} x$}}
		  {\binpar{P}{Q} \downarrow_{\mathcal N} x}

We write $P \Downarrow_{\mathcal N} x$ if there is $Q$ such that 
$P \wred Q$ and $Q \downarrow_{\mathcal N} x$.
\end{definition}

\begin{definition}
%\label{def.bbisim}
An  ${\mathcal N}$-\emph{barbed bisimulation} over a set of names, ${\mathcal N}$, is a symmetric binary relation 
${\mathcal S}_{\mathcal N}$ between agents such that $P\rel{S}_{\mathcal N}Q$ implies:
\begin{enumerate}
\item If $P \red P'$ then $Q \wred Q'$ and $P'\rel{S}_{\mathcal N} Q'$.
\item If $P\downarrow_{\mathcal N} x$, then $Q\Downarrow_{\mathcal N} x$.
\end{enumerate}
$P$ is ${\mathcal N}$-barbed bisimilar to $Q$, written
$P \wbbisim_{\mathcal N} Q$, if $P \rel{S}_{\mathcal N} Q$ for some ${\mathcal N}$-barbed bisimulation ${\mathcal S}_{\mathcal N}$.
\end{definition}

$\mathcal{R} \subseteq \pi \times \pi$

$P \mathcal{R} Q => \forall P'. P \red P' \Rightarrow \exists Q'. Q \red Q', P' \mathcal{R} Q'$

$P \vdash x \Rightarrow Q \vdash x$

\begin{mathpar}
  \inferrule*[lab=Out-barb]{x \nameeq y}{{y}!\langle{Q}\rangle \vdash x}
  \and
  \inferrule*[lab=Par-barb]{\mbox{$P\vdash x$ or $Q\vdash x$}}{\binpar{P}{Q} \vdash x}
\end{mathpar}

\subsubsection{Contexts}

One of the principle advantages of computational calculi like the
$\pi$-calculus is a well-defined notion of context,
contextual-equivalence and a correlation between
contextual-equivalence and notions of bisimulation. The notion of
context allows the decomposition of a process into (sub-)process and
its syntactic environment, its context. Thus, a context may be
thought of as a process with a ``hole'' (written $\Box$) in it. The
application of a context $M$ to a process $P$, written $M[P]$, is
tantamount to filling the hole in $M$ with $P$. In this paper we do
not need the full weight of this theory, but do make use of the notion
of context in the proof the main theorem. 

\begin{mathpar}
  \inferrule* [lab=summation] {} {{M_{M},M_{N}} \bc \Box \;|\; x.M_{A} \;|\; M_{M}+M_{N}}
  \and
  \inferrule* [lab=agent] {} {{M_{A}} \bc (\vec{x})M_{P} \;| \; \clift{P_0,\ldots,M_{P},\ldots,P_N}}
  \and \\
  \inferrule* [lab=process] {} {{M_{P}} \bc M_{N} \;| \;P|M_{P} }
\end{mathpar} 

\begin{mathpar}
  \inferrule* [lab=sychronization] {} {M_{N} \bc \Box \;|\; x?M_{F} \;|\; x!M_{C}}
  \and
  \inferrule* [lab=abstraction] {} {{M_{F}} \bc (x)M_{P} }
  \and
  \inferrule* [lab=concretion] {} {{M_{C}} \bc \langle M_{P} \rangle }
  \and \\
  \inferrule* [lab=process] {} {{M_{P}} \bc M_{N} \;| \;P|M_{P} }
\end{mathpar}

\begin{definition}[contextual application] Given a context $M$, and
  process $P$, we define the \emph{contextual application}, $M[P] :=
  M\{P/\Box\}$. That is, the contextual application of M to P is the
  substitution of $P$ for $\Box$ in $M$.
\end{definition}

$\meaningof{-} : L \to \mathcal{P}(\pi)$

\begin{mathpar}
  \inferrule* [lab=collection] {} {\meaningof{true} = \pi, \and \meaningof{~E} = \pi \setminus \meaningof{E}, \and \meaningof{E_{1} \& E_{2}} = \meaningof{E_{1}} \cap \meaningof{E_{2}}}
\end{mathpar}

\begin{mathpar}
  \inferrule* [lab=structure] {} {\meaningof{0} = \{ P \in \pi | P \equiv 0 \}, \and \\ \meaningof{E_1 | E_2} = \{ P \in \pi | P \equiv P_{1} | P_{2}, P_{1} \in \meaningof{E_{1}}, P_{2} \in \meaningof{E_2}\} }
\end{mathpar}

\begin{mathpar}
 \inferrule* [lab=behavior] {} {\meaningof{\langle a?b \rangle E} = \{ P \in \pi | P \equiv Q | u?(y)P', \\ \and \\\\ \and \\ \;\;\; u \in \meaningof{a}, \forall z.P'\{z/y\} \in \meaningof{E\{z/b\}}\}, \and \\ \meaningof{a!E} = \{ P \in \pi | P \equiv Q | x!\langle P' \rangle, x \in \meaningof{a} P' \in \meaningof{E}\} }
\end{mathpar}

\begin{mathpar}
 \inferrule* [lab=nominal] {} {\meaningof{\quotep{E}} = \{ \quotep{P} \in \quotep{\pi} | P \in \meaningof{E} \}, \and \meaningof{\quotep{P}} = \{ \quotep{Q} \in \quotep{\pi} | P \equiv Q \} \and \\ \meaningof{@\quotep{E}} = \{ P \in \pi | P \equiv @x, x \in \meaningof{E} \}}
\end{mathpar}

\begin{eqnarray*}
  \\
  \meaningof{-} : TS \to ST
\end{eqnarray*}

\begin{eqnarray*}
  \\
  L : TS \to ST
\end{eqnarray*}

\begin{eqnarray*}
  \\
  P \models E \iff P \in \meaningof{E}
\end{eqnarray*}

\begin{eqnarray*}
  P \approx_{L} Q \iff \forall E \in L. P \models E \iff Q \models E
\end{eqnarray*}

\begin{eqnarray*}
  P \approx_{K} Q
\end{eqnarray*}

\begin{eqnarray*}
  P \approx Q
\end{eqnarray*}

$\approx_{K} = \approx = \approx_{L}$

\subsubsection{Contextual duality}

Note that contexts extend the quotation operation to a family of
operations from processes to names. Given a context, $M$, we can
define a \emph{nominal context}, $\quotep{M}$ by $\quotep{M}[P] :=
\quotep{M[P]}$. To foreshadow what is to come we observe that these
operations enjoy a duality with processes very much like the duality
between vectors and maps from vectors to scalars.

Further, because the calculus is essentially higher-order, we have a
correspondence between contexts and processes. More specifically,
given a name $x$ and a context $M$ we can construct $M^{*}_{x}$ such
that 

\begin{mathpar}
  M^{*}_{x} | \lift{x}{P} \red M[P]
\end{mathpar}

namely,

\begin{mathpar}
  M^{*}_{x} := x?(u).M[\dropn{u}]
\end{mathpar}

The dependence of $M^{*}_{x}$ on a name makes it an abstraction, 

\begin{mathpar}
  M^{*} := (x)x?(u).M[\dropn{u}]
\end{mathpar}

\subsection{Additional notation}

It will sometimes be convenient to denote the process a name
quotes. We already have the notation $x = \quotep{P}$, but it will be
convenient to introduce an alternate notation, $\procn{x}$, when we
want to emphasize the connection to the use of the name. Note that, by
virtue of name equivalence, $\quotep{\procn{x}} \nameeq x$; so, the
notation is consistent with previous definitions.

Further, because names have structure it is possible to effect
substitutions on the basis of that structure. This means we need to
upgrade our notation for substitutions, which we accomplish by
adapting comprehension notation. Thus,

\begin{mathpar}
  P\{ y / x : x \in S \}
\end{mathpar}

is interpreted to mean the process derived from P by replacing (in a
capture-avoiding manner) each occurrence of $x$ in $S$ by $y$. For example,

\begin{mathpar}
  P\{ \quotep{\procn{x}|\procn{x}} / x : x \in \freenames{P} \}
\end{mathpar}

will replace each (occurrence) of a free name $x$ in $P$ by
$\quotep{\procn{x}|\procn{x}}$.

Also, we will avail ourselves of the notation $x^{L}$ and $x^{R}$ to
denote injections of a name into disjoint copies of the name
space. There are numerous ways to accomplish this. One example can be
found in \cite{MeredithR05}. This notation overloads to vectors of
names: $\vec{x}^{\pi} := (x_{i}^{\pi} \; : \; 0 \leq i < |\vec{x}| )$ where $\pi \in \{L,R\}$.

We also use $P^{\Box} := P|\Box$.

In \cite{MeredithR05} an interpretation of the new operator is
given. It turns out that there are several possible interpretations
all enjoying the requisite algebraic properties of the operator (see
\cite{milner91polyadicpi}). We will therefore make liberal use of
$(\nu\; \vec{x})P$.

% subsection the_syntax_and_semantics_of_the_notation_system (end)   

\input{qm2pi.qmops} 

\input{qm2pi.sterngerlach} 

\input{qm2pi.metric} 

% section concurrent_process_calculi (end)

%\input{qm2pi.proofsketch}

% section proof sketch (end)

%\input{qm2pi.slviaknots} 

% section spatial logic via knots (end)

\input{qm2pi.conclusion}

% section conclusion (end)

%\input{qm2pi.dtcodes} 

% section wiring algorithm (end)

\input{qm2pi.ack} 

% section acknowledgments (end)

\newpage


\bibliographystyle{plain}   
\bibliography{../../biblios/main.bib}

\input{qm2pi.rhodetails}

\end{document}

 

%\documentclass[12pt]{llncs}
%\documentclass{jktr}

\usepackage[pdftex]{hyperref}                   
\usepackage {listings}
\usepackage {mathpartir}
\usepackage{bcprules}
%\usepackage{listings}
                       
\usepackage{graphicx} 
%\usepackage[margins=2.5cm,nohead,nofoot]{geometry}
%\usepackage{geometry}
\usepackage{amsfonts}
\usepackage{amstext}
\usepackage{latexsym}
\usepackage{amssymb}
\usepackage{color}


%\include{myPreamble}
\include{qm2pi.local} 

%\ifpdf
%\usepackage[pdftex]{graphicx}
%\else
%\usepackage{graphicx}
%\fi

 % \ifpdf
%  \usepackage{pdfsync}
%  \if


%\title{Brief Article}
%\author{David F. Snyder}
%\author{L.G. Meredith}

%\address{Dept. of Math., Texas State University--San Marcos, San Marcos, TX 78666}
       
\pagestyle{empty}


\begin{document}

\lstset{language=[Objective]Caml,frame=shadowbox}

\input{qm2pi.front}

% section front matter (end)

\input{qm2pi.intro} 
 
% section introduction (end)

% \input{qm2pi.knotations} 

% section notation (end)

\input{qm2pi.process.calculi} 

% section concurrent_process_calculi_and_spatial_logics_ (end)
    
%\input{qm2pi.knots2pi} 

%\input{qm2pi.trefoil} 

%\input{qm2pi.mainthm} 

% subsection basic_interpretation (end)

%\input{qm2pi.rho.presentation} 
\subsection{The syntax and semantics of the notation system}\label{sub:the_syntax_and_semantics_of_the_notation_system} % (fold)

We now summarize a technical presentation of the calculus that
embodies our theory of dynamics. The typical presentation of such a
calculus follows the style of giving generators and relations on
them. The grammar, below, describing term constructors, freely
generates the set of processes, $\Proc$. This set is then quotiented
by a relation known as structural congruence and it is over this set
that the notion of dynamics is expressed. This presentation is
essentially that of \cite{MeredithR05} with the addition of
polyadicity and summation. For readability we have relegated some of
the technical subtleties to an appendix.

\subsubsection{Process grammar}\label{subsub:process_grammar}

\begin{mathpar}
  \inferrule* [lab=synchronization] {} {{M} \bc \pzero \;|\; x?F \;|\; x!C }
  \and
  \inferrule* [lab=abstraction] {} {{F} \bc (x)P}
  \and
  \inferrule* [lab=concretion] {} {{C} \bc \langle Q \rangle}
  \and
  \inferrule* [lab=process] {} {{P,Q} \bc M \;| \;P|Q \;|\; @{x}}
  \and
  \inferrule* [lab=name] {} {{x} \bc \quotep{P}}
\end{mathpar} 

Note that $\vec{x}$ (resp. $\vec{P}$) denotes a vector of names
(resp. processes) of length $|\vec{x}|$ (resp. $|\vec{P}|$). We adopt
the following useful abbreviations.

\begin{mathpar}
   x?(\vec{y}).P := x.(\vec{y})P \and  x\clift{\vec{P}} := x.\clift{\vec{P}}
   \and x!(y) := \lift{x}{\dropn{y}}
   \and \Pi_{i=0}^{n-1}P_i := P_0 | \ldots | P_{n-1}
\end{mathpar}

\subsubsection{Structural congruence}

\paragraph{Free and bound names and alpha-equivalence.} At the
core of structural equivalence is alpha-equivalence which identifies
process that are the same up to a change of variable. Formally, we
recognize the distinction between free and bound names. The free names
of a process, $\freenames{P}$, may be calculated recursively as
follows:

\begin{mathpar}
\freenames{\pzero} := \emptyset
  \and \\
  \freenames{x?(y).P} := \{ x \} \cup (\freenames{P} \setminus \{ y \})
  \and 
  \freenames{x!\langle P \rangle} := \{ x \} \cup \{ P \} 
  \and \\
  \freenames{P|Q} := \freenames{P} \cup \freenames{Q}
  \and \\
  \freenames{@{x}} := \{ x \}
\end{mathpar}

$\pi$
$\quotep{\pi}$

$\freenames{-} : \pi \to \mathcal{P}(\quotep{\pi})$

\begin{eqnarray*}
  \freenames{\pzero} & := & \emptyset \\
  \freenames{x?(y).P} & := & \{ x \} \cup (\freenames{P} \setminus \{ y \}) \\
  \freenames{x!\langle P \rangle} & := & \{ x \} \cup \{ P \} \\
  \freenames{P|Q} & := & \freenames{P} \cup \freenames{Q} \\
  \freenames{\dropn{x}} & := & \{ x \}
\end{eqnarray*}

The bound names of a process, $\boundnames{P}$, are those names occurring in $P$
that are not free. For example, in $x?(y).0$, the name $x$ is free, while $y$ is bound.

\begin{mathpar}
  \inferrule* [lab=monoidal-laws] {} { P|Q \equiv Q|P \and P|0 \equiv P \and P|(Q|R) \equiv (P|Q)|R }
\end{mathpar}

\begin{mathpar}
  \inferrule* [lab=alpha-equivalence] {} { (x)P \equiv (y)P\{y/x\} \and y \not\in \freenames{P} }
\end{mathpar}

\begin{definition}
Then two processes, $P,Q$, are alpha-equivalent if $P = Q\{\vec{y}/\vec{x}\}$ for
some $\vec{x} \in \boundnames{Q},\vec{y} \in \boundnames{P}$, where $Q\{\vec{y}/\vec{x}\}$
denotes the capture-avoiding substitution of $\vec{y}$ for $\vec{x}$ in $Q$.
\end{definition}

\begin{definition}
  The {\em structural congruence} \cite{SangiorgiWalker} , $\equiv$,
  between processes is the least congruence containing
  alpha-equivalence, satisfying the abelian monoid laws
  (associativity, commutativity and $\pzero$ as identity) for parallel
  composition $|$ and for summation $+$.
\end{definition}

\subsection{Name equivalence}

We take name equivalence, written $\nameeq$, to be the smallest
equivalence relation generated by the following rules.

\begin{mathpar}
\inferrule*[lab=Quote-drop]
{ }
{ \quotep{@{x}} \nameeq x }

\inferrule*[lab=Struct-equiv]
{ P \scong Q }
{ \quotep{P} \nameeq \quotep{Q} }
\end{mathpar}

The astute reader will have noticed that the mutual recursion of names
and processes imposes a mutual recursion on alpha-equivalence and
structural equivalence via name-equivalence. Fortunately, all of this
works out pleasantly and we may calculate in the natural way, free of
concern. The reader interested in the details is referred to the
appendix \ref{appendix:rho_details}.

\subsection{Substitution}

We use $\Proc$ for the set of processes, $\QProc$ for the set of
names, and $\id{\{}\vec{y} / \vec{x} \id{\}}$ to denote partial maps,
$s : \QProc \rightarrow \QProc$. A map, $s$ lifts, uniquely, to a map
on process terms, $\widehat{s} : \Proc \rightarrow \Proc$ by the
following equations.

\begin{mathpar}
  (0) \psubstp{Q}{P} := 0 \\
  (R \juxtap S) \psubstp{Q}{P}
  :=    
  (R)\psubstp{Q}{P} \juxtap (S) \psubstp{Q}{P} \\
  (x?(y).R) \psubstp{Q}{P}    
  :=    
  (x)\substp{Q}{P} (z)\concat( (R \psubstn{z}{y}) \psubstp{Q}{P} ) \\
  (\lift{x}{R}) \psubstp{Q}{P}  
  :=
  \lift{(x)\substp{Q}{P}}{ R \psubstp{Q}{P} } \\
%   (\dropn{x})  \psubstp{Q}{P}       
%   := 
%   \left\{ 
%     \begin{array}{ccc} 
%       \dropn{\quotep{Q}} & & x \nameeq \quotep{P} \\
%       \dropn{x} & & otherwise \\
%     \end{array}
%   \right. 
  (\dropn{x})  \psubstp{Q}{P}       
  := 
  \left\{ 
    \begin{array}{ccc} 
      Q & & x \nameeq \quotep{P} \\
      \dropn{x} & & otherwise \\
    \end{array}
  \right.
\end{mathpar}
 

where

\begin{eqnarray}
  (x)\id{\{} \lpquote Q \rpquote / \lpquote P \rpquote \id{\}}            = 
  \left\{ 
    \begin{array}{ccc}
      \lpquote Q \rpquote & & x \nameeq \lpquote P \rpquote \\
      x & & otherwise \\
    \end{array}
  \right. \nonumber
\end{eqnarray}

and $z$ is chosen distinct from $\quotep{P}$, $\quotep{Q}$, the free
names in $Q$, and all the names in $R$. Our $\alpha$-equivalence will
be built in the standard way from this substitution.

\begin{remark}\label{rem:no_self_referential_names}
  One consequence of these definitions is that $\forall P. \quotep{P}
  \not\in \freenames{P}$.
\end{remark}

\subsection{ Dynamic quote: an example }

Anticipating something of what's to come, consider applying the
substitution, $\widehat{\id{\{}u / z \id{\}}}$, to the following pair
of processes, $\lift{w}{y!(z)}$ and $w[ \lpquote y!(z) \rpquote ]$.

\begin{eqnarray}
	\lift{w}{y!(z)}\widehat{\id{\{}u / z \id{\}}}
		& = &
		\lift{w}{y!(u)} \nonumber\\
	w[ \lpquote y!(z) \rpquote ] \widehat{ \id{\{}u / z \id{\}} }
		& = &
		w[ \lpquote y!(z) \rpquote ] \nonumber
\end{eqnarray}

Because the body of the process between quotes is impervious to
substitution, we get radically different answers. In fact, by
examining the first process in an input context,
e.g. $x?(z).\lift{w}{y!(z)}$, we see that the process under the lift
operator may be shaped by prefixed inputs binding a name inside it. In
this sense, the lift operator will be seen as a way to dynamically
construct processes before reifying them as names.

Finally equipped with these standard features we can present the
dynamics of the calculus.

\subsubsection{Operational semantics} 

Finally, we introduce the computational dynamics. What marks these
algebras as distinct from other more traditionally studied algebraic
structures, e.g. vector spaces or polynomial rings, is the manner in
which dynamics is captured. In traditional structures, dynamics is typically
expressed through morphisms between such structures, as in linear maps
between vector spaces or morphisms between rings. In algebras
associated with the semantics of computation, the dynamics is
expressed as part of the algebraic structure itself, through a
reduction reduction relation typically denoted by $\red$. Below, we
give a recursive presentation of this relation for the calculus used
in the encoding.

$\red \subseteq \pi \times \pi$
$\red : \pi \to \mathcal{P}(\pi)$

\begin{mathpar}
  \inferrule* [lab=Comm] { \textsf{match}( x_{src}, x_{trgt} ) } { x_{trgt}?(y)P \; | \; x_{src}!\langle {Q} \rangle \red P\{\quotep{Q}/y}\} }
  \and \\
  \inferrule* [lab=Par] {{P} \red {P}'} {{{P} | {Q}} \red {{P}' | {Q}}}
  \and
  \inferrule* [lab=Equiv]{{{P} \scong {P}'} \andalso {{P}' \red {Q}'} \andalso {{Q}' \scong {Q}}}{{P} \red {Q}}
\end{mathpar}

\begin{eqnarray*}
  match_{\equiv} (\quotep{P},\quotep{Q}) & := & P \equiv Q \\
  match_{\dagger}(\quotep{P},\quotep{Q}) & := & \forall R. P|Q \red^{*} R => R \red^{*} 0 \\
  match_{K}(\quotep{P},\quotep{Q}) & := & K \mbox{ for some context } K
\end{eqnarray*}

$u?(x)P | u!\langle Q \rangle \red P\{\quotep{Q}/x\}$

%We write $\wred$ for $\red^*$, and $P\red$ if $\exists Q $ such that $ P \red Q$.
We write $P\red$ if $\exists Q $ such that $ P \red Q$ and $P\not\red$, otherwise.

\section{Replication}

As mentioned before, it is known that replication (and hence
recursion) can be implemented in a higher-order process algebra
\cite{SangiorgiWalker}. As our first example of calculation with the
machinery thus far presented we give the construction explicitly in
the {\rhoc}.

\begin{eqnarray}
	D_{x} & := & \prefix{x}{y}{(\binpar{\outputp{x}{y}}{@{y}})} \nonumber\\
	\bangp_{x}{P} & := & \binpar{{x}!\langle{\binpar{D_{x}}{P}}\rangle}{D_{x}} \nonumber
\end{eqnarray}

\begin{eqnarray}
	\bangp_{x}{P} & & \nonumber\\
	=
	& {x}!\langle{(\prefix{x}{y}{(\outputp{x}{y} | @{y})) | P}}\rangle 
	      | \prefix{x}{y}{(\outputp{x}{y} | @{y})} & \nonumber\\
	\red
	& (\outputp{x}{y} | @{y})\substn{\quotep{(\prefix{x}{y}{(@{y} | \outputp{x}{y})) | P}}}{y} & \nonumber\\
	=
	& \outputp{x}{\quotep{(\prefix{x}{y}{(\outputp{x}{y} | @{y})) | P}}}
	  | {(\prefix{x}{y}{(\outputp{x}{y} | @{y})) | P}} & \nonumber\\
	\red
	& \ldots & \nonumber\\
	\red^*
	& P | P | \ldots & \nonumber
\end{eqnarray}

Of course, this encoding, as an implementation, runs away, unfolding
$\bangp{P}$ eagerly. A lazier and more implementable replication
operator, restricted to input-guarded processes, may be obtained as follows.

\begin{eqnarray}
\bangp{\prefix{u}{v}{P}} 
	:= 
	\binpar{\lift{x}{\prefix{u}{v}{(\binpar{D(x)}{P})}}}{D(x)} \nonumber
\end{eqnarray}

\begin{remark}
  Note that the lazier definition still does not deal with summation
  or mixed summation (i.e. sums over input and output). The reader is
  invited to construct definitions of replication that deal with these
  features. 

  Further, the definitions are parameterized in a name, $x$. Can you,
  gentle reader, make a definition that eliminates this parameter and
  guarantees no accidental interaction between the replication
  machinery and the process being replicated -- i.e. no accidental
  sharing of names used by the process to get its work done and the
  name(s) used by the replication to effect copying. This latter
  revision of the definition of replication is crucial to obtaining
  the expected identity $!!P \sim !P$.
\end{remark}

\begin{remark}\label{rem:paradoxical_combinator}
  The reader familiar with the lambda calculus will have noticed the
  similarity between $D$ and the paradoxical combinator.

  [Ed. note: the existence of this seems to suggest we have to be more
  restrictive on the set of processes and names we admit if we are to
  support no-cloning.]
\end{remark}

\subsubsection{Bisimulation}

The computational dynamics gives rise to another kind of equivalence,
the equivalence of computational behavior. As previously mentioned
this is typically captured \emph{via} some form of bisimulation.

% The notion we use in this paper is weak barbed bisimulation
% \cite{milner91polyadicpi}.

The notion we use in this paper is derived from weak barbed
bisimulation \cite{milner91polyadicpi}. 

\begin{definition}
An \emph{observation relation}, $\downarrow_{\mathcal N}$, over a set
of names, $\mathcal N$, is the smallest relation satisfying the rules
below.

\infrule[Out-barb]{y \in {\mathcal N}, \; x \nameeq y}
		  {\outputp{x}{v} \downarrow_{\mathcal N} x}
\infrule[Par-barb]{\mbox{$P\downarrow_{\mathcal N} x$ or $Q\downarrow_{\mathcal N} x$}}
		  {\binpar{P}{Q} \downarrow_{\mathcal N} x}

We write $P \Downarrow_{\mathcal N} x$ if there is $Q$ such that 
$P \wred Q$ and $Q \downarrow_{\mathcal N} x$.
\end{definition}

\begin{definition}
%\label{def.bbisim}
An  ${\mathcal N}$-\emph{barbed bisimulation} over a set of names, ${\mathcal N}$, is a symmetric binary relation 
${\mathcal S}_{\mathcal N}$ between agents such that $P\rel{S}_{\mathcal N}Q$ implies:
\begin{enumerate}
\item If $P \red P'$ then $Q \wred Q'$ and $P'\rel{S}_{\mathcal N} Q'$.
\item If $P\downarrow_{\mathcal N} x$, then $Q\Downarrow_{\mathcal N} x$.
\end{enumerate}
$P$ is ${\mathcal N}$-barbed bisimilar to $Q$, written
$P \wbbisim_{\mathcal N} Q$, if $P \rel{S}_{\mathcal N} Q$ for some ${\mathcal N}$-barbed bisimulation ${\mathcal S}_{\mathcal N}$.
\end{definition}

$\mathcal{R} \subseteq \pi \times \pi$

$P \mathcal{R} Q => \forall P'. P \red P' \Rightarrow \exists Q'. Q \red Q', P' \mathcal{R} Q'$

$P \vdash x \Rightarrow Q \vdash x$

\begin{mathpar}
  \inferrule*[lab=Out-barb]{x \nameeq y}{{y}!\langle{Q}\rangle \vdash x}
  \and
  \inferrule*[lab=Par-barb]{\mbox{$P\vdash x$ or $Q\vdash x$}}{\binpar{P}{Q} \vdash x}
\end{mathpar}

\subsubsection{Contexts}

One of the principle advantages of computational calculi like the
$\pi$-calculus is a well-defined notion of context,
contextual-equivalence and a correlation between
contextual-equivalence and notions of bisimulation. The notion of
context allows the decomposition of a process into (sub-)process and
its syntactic environment, its context. Thus, a context may be
thought of as a process with a ``hole'' (written $\Box$) in it. The
application of a context $M$ to a process $P$, written $M[P]$, is
tantamount to filling the hole in $M$ with $P$. In this paper we do
not need the full weight of this theory, but do make use of the notion
of context in the proof the main theorem. 

\begin{mathpar}
  \inferrule* [lab=summation] {} {{M_{M},M_{N}} \bc \Box \;|\; x.M_{A} \;|\; M_{M}+M_{N}}
  \and
  \inferrule* [lab=agent] {} {{M_{A}} \bc (\vec{x})M_{P} \;| \; \clift{P_0,\ldots,M_{P},\ldots,P_N}}
  \and \\
  \inferrule* [lab=process] {} {{M_{P}} \bc M_{N} \;| \;P|M_{P} }
\end{mathpar} 

\begin{mathpar}
  \inferrule* [lab=sychronization] {} {M_{N} \bc \Box \;|\; x?M_{F} \;|\; x!M_{C}}
  \and
  \inferrule* [lab=abstraction] {} {{M_{F}} \bc (x)M_{P} }
  \and
  \inferrule* [lab=concretion] {} {{M_{C}} \bc \langle M_{P} \rangle }
  \and \\
  \inferrule* [lab=process] {} {{M_{P}} \bc M_{N} \;| \;P|M_{P} }
\end{mathpar}

\begin{definition}[contextual application] Given a context $M$, and
  process $P$, we define the \emph{contextual application}, $M[P] :=
  M\{P/\Box\}$. That is, the contextual application of M to P is the
  substitution of $P$ for $\Box$ in $M$.
\end{definition}

$\meaningof{-} : L \to \mathcal{P}(\pi)$

\begin{mathpar}
  \inferrule* [lab=collection] {} {\meaningof{true} = \pi, \and \meaningof{~E} = \pi \setminus \meaningof{E}, \and \meaningof{E_{1} \& E_{2}} = \meaningof{E_{1}} \cap \meaningof{E_{2}}}
\end{mathpar}

\begin{mathpar}
  \inferrule* [lab=structure] {} {\meaningof{0} = \{ P \in \pi | P \equiv 0 \}, \and \\ \meaningof{E_1 | E_2} = \{ P \in \pi | P \equiv P_{1} | P_{2}, P_{1} \in \meaningof{E_{1}}, P_{2} \in \meaningof{E_2}\} }
\end{mathpar}

\begin{mathpar}
 \inferrule* [lab=behavior] {} {\meaningof{\langle a?b \rangle E} = \{ P \in \pi | P \equiv Q | u?(y)P', \\ \and \\\\ \and \\ \;\;\; u \in \meaningof{a}, \forall z.P'\{z/y\} \in \meaningof{E\{z/b\}}\}, \and \\ \meaningof{a!E} = \{ P \in \pi | P \equiv Q | x!\langle P' \rangle, x \in \meaningof{a} P' \in \meaningof{E}\} }
\end{mathpar}

\begin{mathpar}
 \inferrule* [lab=nominal] {} {\meaningof{\quotep{E}} = \{ \quotep{P} \in \quotep{\pi} | P \in \meaningof{E} \}, \and \meaningof{\quotep{P}} = \{ \quotep{Q} \in \quotep{\pi} | P \equiv Q \} \and \\ \meaningof{@\quotep{E}} = \{ P \in \pi | P \equiv @x, x \in \meaningof{E} \}}
\end{mathpar}

\begin{eqnarray*}
  \\
  \meaningof{-} : TS \to ST
\end{eqnarray*}

\begin{eqnarray*}
  \\
  L : TS \to ST
\end{eqnarray*}

\begin{eqnarray*}
  \\
  P \models E \iff P \in \meaningof{E}
\end{eqnarray*}

\begin{eqnarray*}
  P \approx_{L} Q \iff \forall E \in L. P \models E \iff Q \models E
\end{eqnarray*}

\begin{eqnarray*}
  P \approx_{K} Q
\end{eqnarray*}

\begin{eqnarray*}
  P \approx Q
\end{eqnarray*}

$\approx_{K} = \approx = \approx_{L}$

\subsubsection{Contextual duality}

Note that contexts extend the quotation operation to a family of
operations from processes to names. Given a context, $M$, we can
define a \emph{nominal context}, $\quotep{M}$ by $\quotep{M}[P] :=
\quotep{M[P]}$. To foreshadow what is to come we observe that these
operations enjoy a duality with processes very much like the duality
between vectors and maps from vectors to scalars.

Further, because the calculus is essentially higher-order, we have a
correspondence between contexts and processes. More specifically,
given a name $x$ and a context $M$ we can construct $M^{*}_{x}$ such
that 

\begin{mathpar}
  M^{*}_{x} | \lift{x}{P} \red M[P]
\end{mathpar}

namely,

\begin{mathpar}
  M^{*}_{x} := x?(u).M[\dropn{u}]
\end{mathpar}

The dependence of $M^{*}_{x}$ on a name makes it an abstraction, 

\begin{mathpar}
  M^{*} := (x)x?(u).M[\dropn{u}]
\end{mathpar}

\subsection{Additional notation}

It will sometimes be convenient to denote the process a name
quotes. We already have the notation $x = \quotep{P}$, but it will be
convenient to introduce an alternate notation, $\procn{x}$, when we
want to emphasize the connection to the use of the name. Note that, by
virtue of name equivalence, $\quotep{\procn{x}} \nameeq x$; so, the
notation is consistent with previous definitions.

Further, because names have structure it is possible to effect
substitutions on the basis of that structure. This means we need to
upgrade our notation for substitutions, which we accomplish by
adapting comprehension notation. Thus,

\begin{mathpar}
  P\{ y / x : x \in S \}
\end{mathpar}

is interpreted to mean the process derived from P by replacing (in a
capture-avoiding manner) each occurrence of $x$ in $S$ by $y$. For example,

\begin{mathpar}
  P\{ \quotep{\procn{x}|\procn{x}} / x : x \in \freenames{P} \}
\end{mathpar}

will replace each (occurrence) of a free name $x$ in $P$ by
$\quotep{\procn{x}|\procn{x}}$.

Also, we will avail ourselves of the notation $x^{L}$ and $x^{R}$ to
denote injections of a name into disjoint copies of the name
space. There are numerous ways to accomplish this. One example can be
found in \cite{MeredithR05}. This notation overloads to vectors of
names: $\vec{x}^{\pi} := (x_{i}^{\pi} \; : \; 0 \leq i < |\vec{x}| )$ where $\pi \in \{L,R\}$.

We also use $P^{\Box} := P|\Box$.

In \cite{MeredithR05} an interpretation of the new operator is
given. It turns out that there are several possible interpretations
all enjoying the requisite algebraic properties of the operator (see
\cite{milner91polyadicpi}). We will therefore make liberal use of
$(\nu\; \vec{x})P$.

% subsection the_syntax_and_semantics_of_the_notation_system (end)   

\input{qm2pi.qmops} 

\input{qm2pi.sterngerlach} 

\input{qm2pi.metric} 

% section concurrent_process_calculi (end)

%\input{qm2pi.proofsketch}

% section proof sketch (end)

%\input{qm2pi.slviaknots} 

% section spatial logic via knots (end)

\input{qm2pi.conclusion}

% section conclusion (end)

%\input{qm2pi.dtcodes} 

% section wiring algorithm (end)

\input{qm2pi.ack} 

% section acknowledgments (end)

\newpage


\bibliographystyle{plain}   
\bibliography{../../biblios/main.bib}

\input{qm2pi.rhodetails}

\end{document}

 

%\documentclass[12pt]{llncs}
%\documentclass{jktr}

\usepackage[pdftex]{hyperref}                   
\usepackage {listings}
\usepackage {mathpartir}
\usepackage{bcprules}
%\usepackage{listings}
                       
\usepackage{graphicx} 
%\usepackage[margins=2.5cm,nohead,nofoot]{geometry}
%\usepackage{geometry}
\usepackage{amsfonts}
\usepackage{amstext}
\usepackage{latexsym}
\usepackage{amssymb}
\usepackage{color}


%\include{myPreamble}
\include{qm2pi.local} 

%\ifpdf
%\usepackage[pdftex]{graphicx}
%\else
%\usepackage{graphicx}
%\fi

 % \ifpdf
%  \usepackage{pdfsync}
%  \if


%\title{Brief Article}
%\author{David F. Snyder}
%\author{L.G. Meredith}

%\address{Dept. of Math., Texas State University--San Marcos, San Marcos, TX 78666}
       
\pagestyle{empty}


\begin{document}

\lstset{language=[Objective]Caml,frame=shadowbox}

\input{qm2pi.front}

% section front matter (end)

\input{qm2pi.intro} 
 
% section introduction (end)

% \input{qm2pi.knotations} 

% section notation (end)

\input{qm2pi.process.calculi} 

% section concurrent_process_calculi_and_spatial_logics_ (end)
    
%\input{qm2pi.knots2pi} 

%\input{qm2pi.trefoil} 

%\input{qm2pi.mainthm} 

% subsection basic_interpretation (end)

%\input{qm2pi.rho.presentation} 
\subsection{The syntax and semantics of the notation system}\label{sub:the_syntax_and_semantics_of_the_notation_system} % (fold)

We now summarize a technical presentation of the calculus that
embodies our theory of dynamics. The typical presentation of such a
calculus follows the style of giving generators and relations on
them. The grammar, below, describing term constructors, freely
generates the set of processes, $\Proc$. This set is then quotiented
by a relation known as structural congruence and it is over this set
that the notion of dynamics is expressed. This presentation is
essentially that of \cite{MeredithR05} with the addition of
polyadicity and summation. For readability we have relegated some of
the technical subtleties to an appendix.

\subsubsection{Process grammar}\label{subsub:process_grammar}

\begin{mathpar}
  \inferrule* [lab=synchronization] {} {{M} \bc \pzero \;|\; x?F \;|\; x!C }
  \and
  \inferrule* [lab=abstraction] {} {{F} \bc (x)P}
  \and
  \inferrule* [lab=concretion] {} {{C} \bc \langle Q \rangle}
  \and
  \inferrule* [lab=process] {} {{P,Q} \bc M \;| \;P|Q \;|\; @{x}}
  \and
  \inferrule* [lab=name] {} {{x} \bc \quotep{P}}
\end{mathpar} 

Note that $\vec{x}$ (resp. $\vec{P}$) denotes a vector of names
(resp. processes) of length $|\vec{x}|$ (resp. $|\vec{P}|$). We adopt
the following useful abbreviations.

\begin{mathpar}
   x?(\vec{y}).P := x.(\vec{y})P \and  x\clift{\vec{P}} := x.\clift{\vec{P}}
   \and x!(y) := \lift{x}{\dropn{y}}
   \and \Pi_{i=0}^{n-1}P_i := P_0 | \ldots | P_{n-1}
\end{mathpar}

\subsubsection{Structural congruence}

\paragraph{Free and bound names and alpha-equivalence.} At the
core of structural equivalence is alpha-equivalence which identifies
process that are the same up to a change of variable. Formally, we
recognize the distinction between free and bound names. The free names
of a process, $\freenames{P}$, may be calculated recursively as
follows:

\begin{mathpar}
\freenames{\pzero} := \emptyset
  \and \\
  \freenames{x?(y).P} := \{ x \} \cup (\freenames{P} \setminus \{ y \})
  \and 
  \freenames{x!\langle P \rangle} := \{ x \} \cup \{ P \} 
  \and \\
  \freenames{P|Q} := \freenames{P} \cup \freenames{Q}
  \and \\
  \freenames{@{x}} := \{ x \}
\end{mathpar}

$\pi$
$\quotep{\pi}$

$\freenames{-} : \pi \to \mathcal{P}(\quotep{\pi})$

\begin{eqnarray*}
  \freenames{\pzero} & := & \emptyset \\
  \freenames{x?(y).P} & := & \{ x \} \cup (\freenames{P} \setminus \{ y \}) \\
  \freenames{x!\langle P \rangle} & := & \{ x \} \cup \{ P \} \\
  \freenames{P|Q} & := & \freenames{P} \cup \freenames{Q} \\
  \freenames{\dropn{x}} & := & \{ x \}
\end{eqnarray*}

The bound names of a process, $\boundnames{P}$, are those names occurring in $P$
that are not free. For example, in $x?(y).0$, the name $x$ is free, while $y$ is bound.

\begin{mathpar}
  \inferrule* [lab=monoidal-laws] {} { P|Q \equiv Q|P \and P|0 \equiv P \and P|(Q|R) \equiv (P|Q)|R }
\end{mathpar}

\begin{mathpar}
  \inferrule* [lab=alpha-equivalence] {} { (x)P \equiv (y)P\{y/x\} \and y \not\in \freenames{P} }
\end{mathpar}

\begin{definition}
Then two processes, $P,Q$, are alpha-equivalent if $P = Q\{\vec{y}/\vec{x}\}$ for
some $\vec{x} \in \boundnames{Q},\vec{y} \in \boundnames{P}$, where $Q\{\vec{y}/\vec{x}\}$
denotes the capture-avoiding substitution of $\vec{y}$ for $\vec{x}$ in $Q$.
\end{definition}

\begin{definition}
  The {\em structural congruence} \cite{SangiorgiWalker} , $\equiv$,
  between processes is the least congruence containing
  alpha-equivalence, satisfying the abelian monoid laws
  (associativity, commutativity and $\pzero$ as identity) for parallel
  composition $|$ and for summation $+$.
\end{definition}

\subsection{Name equivalence}

We take name equivalence, written $\nameeq$, to be the smallest
equivalence relation generated by the following rules.

\begin{mathpar}
\inferrule*[lab=Quote-drop]
{ }
{ \quotep{@{x}} \nameeq x }

\inferrule*[lab=Struct-equiv]
{ P \scong Q }
{ \quotep{P} \nameeq \quotep{Q} }
\end{mathpar}

The astute reader will have noticed that the mutual recursion of names
and processes imposes a mutual recursion on alpha-equivalence and
structural equivalence via name-equivalence. Fortunately, all of this
works out pleasantly and we may calculate in the natural way, free of
concern. The reader interested in the details is referred to the
appendix \ref{appendix:rho_details}.

\subsection{Substitution}

We use $\Proc$ for the set of processes, $\QProc$ for the set of
names, and $\id{\{}\vec{y} / \vec{x} \id{\}}$ to denote partial maps,
$s : \QProc \rightarrow \QProc$. A map, $s$ lifts, uniquely, to a map
on process terms, $\widehat{s} : \Proc \rightarrow \Proc$ by the
following equations.

\begin{mathpar}
  (0) \psubstp{Q}{P} := 0 \\
  (R \juxtap S) \psubstp{Q}{P}
  :=    
  (R)\psubstp{Q}{P} \juxtap (S) \psubstp{Q}{P} \\
  (x?(y).R) \psubstp{Q}{P}    
  :=    
  (x)\substp{Q}{P} (z)\concat( (R \psubstn{z}{y}) \psubstp{Q}{P} ) \\
  (\lift{x}{R}) \psubstp{Q}{P}  
  :=
  \lift{(x)\substp{Q}{P}}{ R \psubstp{Q}{P} } \\
%   (\dropn{x})  \psubstp{Q}{P}       
%   := 
%   \left\{ 
%     \begin{array}{ccc} 
%       \dropn{\quotep{Q}} & & x \nameeq \quotep{P} \\
%       \dropn{x} & & otherwise \\
%     \end{array}
%   \right. 
  (\dropn{x})  \psubstp{Q}{P}       
  := 
  \left\{ 
    \begin{array}{ccc} 
      Q & & x \nameeq \quotep{P} \\
      \dropn{x} & & otherwise \\
    \end{array}
  \right.
\end{mathpar}
 

where

\begin{eqnarray}
  (x)\id{\{} \lpquote Q \rpquote / \lpquote P \rpquote \id{\}}            = 
  \left\{ 
    \begin{array}{ccc}
      \lpquote Q \rpquote & & x \nameeq \lpquote P \rpquote \\
      x & & otherwise \\
    \end{array}
  \right. \nonumber
\end{eqnarray}

and $z$ is chosen distinct from $\quotep{P}$, $\quotep{Q}$, the free
names in $Q$, and all the names in $R$. Our $\alpha$-equivalence will
be built in the standard way from this substitution.

\begin{remark}\label{rem:no_self_referential_names}
  One consequence of these definitions is that $\forall P. \quotep{P}
  \not\in \freenames{P}$.
\end{remark}

\subsection{ Dynamic quote: an example }

Anticipating something of what's to come, consider applying the
substitution, $\widehat{\id{\{}u / z \id{\}}}$, to the following pair
of processes, $\lift{w}{y!(z)}$ and $w[ \lpquote y!(z) \rpquote ]$.

\begin{eqnarray}
	\lift{w}{y!(z)}\widehat{\id{\{}u / z \id{\}}}
		& = &
		\lift{w}{y!(u)} \nonumber\\
	w[ \lpquote y!(z) \rpquote ] \widehat{ \id{\{}u / z \id{\}} }
		& = &
		w[ \lpquote y!(z) \rpquote ] \nonumber
\end{eqnarray}

Because the body of the process between quotes is impervious to
substitution, we get radically different answers. In fact, by
examining the first process in an input context,
e.g. $x?(z).\lift{w}{y!(z)}$, we see that the process under the lift
operator may be shaped by prefixed inputs binding a name inside it. In
this sense, the lift operator will be seen as a way to dynamically
construct processes before reifying them as names.

Finally equipped with these standard features we can present the
dynamics of the calculus.

\subsubsection{Operational semantics} 

Finally, we introduce the computational dynamics. What marks these
algebras as distinct from other more traditionally studied algebraic
structures, e.g. vector spaces or polynomial rings, is the manner in
which dynamics is captured. In traditional structures, dynamics is typically
expressed through morphisms between such structures, as in linear maps
between vector spaces or morphisms between rings. In algebras
associated with the semantics of computation, the dynamics is
expressed as part of the algebraic structure itself, through a
reduction reduction relation typically denoted by $\red$. Below, we
give a recursive presentation of this relation for the calculus used
in the encoding.

$\red \subseteq \pi \times \pi$
$\red : \pi \to \mathcal{P}(\pi)$

\begin{mathpar}
  \inferrule* [lab=Comm] { \textsf{match}( x_{src}, x_{trgt} ) } { x_{trgt}?(y)P \; | \; x_{src}!\langle {Q} \rangle \red P\{\quotep{Q}/y}\} }
  \and \\
  \inferrule* [lab=Par] {{P} \red {P}'} {{{P} | {Q}} \red {{P}' | {Q}}}
  \and
  \inferrule* [lab=Equiv]{{{P} \scong {P}'} \andalso {{P}' \red {Q}'} \andalso {{Q}' \scong {Q}}}{{P} \red {Q}}
\end{mathpar}

\begin{eqnarray*}
  match_{\equiv} (\quotep{P},\quotep{Q}) & := & P \equiv Q \\
  match_{\dagger}(\quotep{P},\quotep{Q}) & := & \forall R. P|Q \red^{*} R => R \red^{*} 0 \\
  match_{K}(\quotep{P},\quotep{Q}) & := & K \mbox{ for some context } K
\end{eqnarray*}

$u?(x)P | u!\langle Q \rangle \red P\{\quotep{Q}/x\}$

%We write $\wred$ for $\red^*$, and $P\red$ if $\exists Q $ such that $ P \red Q$.
We write $P\red$ if $\exists Q $ such that $ P \red Q$ and $P\not\red$, otherwise.

\section{Replication}

As mentioned before, it is known that replication (and hence
recursion) can be implemented in a higher-order process algebra
\cite{SangiorgiWalker}. As our first example of calculation with the
machinery thus far presented we give the construction explicitly in
the {\rhoc}.

\begin{eqnarray}
	D_{x} & := & \prefix{x}{y}{(\binpar{\outputp{x}{y}}{@{y}})} \nonumber\\
	\bangp_{x}{P} & := & \binpar{{x}!\langle{\binpar{D_{x}}{P}}\rangle}{D_{x}} \nonumber
\end{eqnarray}

\begin{eqnarray}
	\bangp_{x}{P} & & \nonumber\\
	=
	& {x}!\langle{(\prefix{x}{y}{(\outputp{x}{y} | @{y})) | P}}\rangle 
	      | \prefix{x}{y}{(\outputp{x}{y} | @{y})} & \nonumber\\
	\red
	& (\outputp{x}{y} | @{y})\substn{\quotep{(\prefix{x}{y}{(@{y} | \outputp{x}{y})) | P}}}{y} & \nonumber\\
	=
	& \outputp{x}{\quotep{(\prefix{x}{y}{(\outputp{x}{y} | @{y})) | P}}}
	  | {(\prefix{x}{y}{(\outputp{x}{y} | @{y})) | P}} & \nonumber\\
	\red
	& \ldots & \nonumber\\
	\red^*
	& P | P | \ldots & \nonumber
\end{eqnarray}

Of course, this encoding, as an implementation, runs away, unfolding
$\bangp{P}$ eagerly. A lazier and more implementable replication
operator, restricted to input-guarded processes, may be obtained as follows.

\begin{eqnarray}
\bangp{\prefix{u}{v}{P}} 
	:= 
	\binpar{\lift{x}{\prefix{u}{v}{(\binpar{D(x)}{P})}}}{D(x)} \nonumber
\end{eqnarray}

\begin{remark}
  Note that the lazier definition still does not deal with summation
  or mixed summation (i.e. sums over input and output). The reader is
  invited to construct definitions of replication that deal with these
  features. 

  Further, the definitions are parameterized in a name, $x$. Can you,
  gentle reader, make a definition that eliminates this parameter and
  guarantees no accidental interaction between the replication
  machinery and the process being replicated -- i.e. no accidental
  sharing of names used by the process to get its work done and the
  name(s) used by the replication to effect copying. This latter
  revision of the definition of replication is crucial to obtaining
  the expected identity $!!P \sim !P$.
\end{remark}

\begin{remark}\label{rem:paradoxical_combinator}
  The reader familiar with the lambda calculus will have noticed the
  similarity between $D$ and the paradoxical combinator.

  [Ed. note: the existence of this seems to suggest we have to be more
  restrictive on the set of processes and names we admit if we are to
  support no-cloning.]
\end{remark}

\subsubsection{Bisimulation}

The computational dynamics gives rise to another kind of equivalence,
the equivalence of computational behavior. As previously mentioned
this is typically captured \emph{via} some form of bisimulation.

% The notion we use in this paper is weak barbed bisimulation
% \cite{milner91polyadicpi}.

The notion we use in this paper is derived from weak barbed
bisimulation \cite{milner91polyadicpi}. 

\begin{definition}
An \emph{observation relation}, $\downarrow_{\mathcal N}$, over a set
of names, $\mathcal N$, is the smallest relation satisfying the rules
below.

\infrule[Out-barb]{y \in {\mathcal N}, \; x \nameeq y}
		  {\outputp{x}{v} \downarrow_{\mathcal N} x}
\infrule[Par-barb]{\mbox{$P\downarrow_{\mathcal N} x$ or $Q\downarrow_{\mathcal N} x$}}
		  {\binpar{P}{Q} \downarrow_{\mathcal N} x}

We write $P \Downarrow_{\mathcal N} x$ if there is $Q$ such that 
$P \wred Q$ and $Q \downarrow_{\mathcal N} x$.
\end{definition}

\begin{definition}
%\label{def.bbisim}
An  ${\mathcal N}$-\emph{barbed bisimulation} over a set of names, ${\mathcal N}$, is a symmetric binary relation 
${\mathcal S}_{\mathcal N}$ between agents such that $P\rel{S}_{\mathcal N}Q$ implies:
\begin{enumerate}
\item If $P \red P'$ then $Q \wred Q'$ and $P'\rel{S}_{\mathcal N} Q'$.
\item If $P\downarrow_{\mathcal N} x$, then $Q\Downarrow_{\mathcal N} x$.
\end{enumerate}
$P$ is ${\mathcal N}$-barbed bisimilar to $Q$, written
$P \wbbisim_{\mathcal N} Q$, if $P \rel{S}_{\mathcal N} Q$ for some ${\mathcal N}$-barbed bisimulation ${\mathcal S}_{\mathcal N}$.
\end{definition}

$\mathcal{R} \subseteq \pi \times \pi$

$P \mathcal{R} Q => \forall P'. P \red P' \Rightarrow \exists Q'. Q \red Q', P' \mathcal{R} Q'$

$P \vdash x \Rightarrow Q \vdash x$

\begin{mathpar}
  \inferrule*[lab=Out-barb]{x \nameeq y}{{y}!\langle{Q}\rangle \vdash x}
  \and
  \inferrule*[lab=Par-barb]{\mbox{$P\vdash x$ or $Q\vdash x$}}{\binpar{P}{Q} \vdash x}
\end{mathpar}

\subsubsection{Contexts}

One of the principle advantages of computational calculi like the
$\pi$-calculus is a well-defined notion of context,
contextual-equivalence and a correlation between
contextual-equivalence and notions of bisimulation. The notion of
context allows the decomposition of a process into (sub-)process and
its syntactic environment, its context. Thus, a context may be
thought of as a process with a ``hole'' (written $\Box$) in it. The
application of a context $M$ to a process $P$, written $M[P]$, is
tantamount to filling the hole in $M$ with $P$. In this paper we do
not need the full weight of this theory, but do make use of the notion
of context in the proof the main theorem. 

\begin{mathpar}
  \inferrule* [lab=summation] {} {{M_{M},M_{N}} \bc \Box \;|\; x.M_{A} \;|\; M_{M}+M_{N}}
  \and
  \inferrule* [lab=agent] {} {{M_{A}} \bc (\vec{x})M_{P} \;| \; \clift{P_0,\ldots,M_{P},\ldots,P_N}}
  \and \\
  \inferrule* [lab=process] {} {{M_{P}} \bc M_{N} \;| \;P|M_{P} }
\end{mathpar} 

\begin{mathpar}
  \inferrule* [lab=sychronization] {} {M_{N} \bc \Box \;|\; x?M_{F} \;|\; x!M_{C}}
  \and
  \inferrule* [lab=abstraction] {} {{M_{F}} \bc (x)M_{P} }
  \and
  \inferrule* [lab=concretion] {} {{M_{C}} \bc \langle M_{P} \rangle }
  \and \\
  \inferrule* [lab=process] {} {{M_{P}} \bc M_{N} \;| \;P|M_{P} }
\end{mathpar}

\begin{definition}[contextual application] Given a context $M$, and
  process $P$, we define the \emph{contextual application}, $M[P] :=
  M\{P/\Box\}$. That is, the contextual application of M to P is the
  substitution of $P$ for $\Box$ in $M$.
\end{definition}

$\meaningof{-} : L \to \mathcal{P}(\pi)$

\begin{mathpar}
  \inferrule* [lab=collection] {} {\meaningof{true} = \pi, \and \meaningof{~E} = \pi \setminus \meaningof{E}, \and \meaningof{E_{1} \& E_{2}} = \meaningof{E_{1}} \cap \meaningof{E_{2}}}
\end{mathpar}

\begin{mathpar}
  \inferrule* [lab=structure] {} {\meaningof{0} = \{ P \in \pi | P \equiv 0 \}, \and \\ \meaningof{E_1 | E_2} = \{ P \in \pi | P \equiv P_{1} | P_{2}, P_{1} \in \meaningof{E_{1}}, P_{2} \in \meaningof{E_2}\} }
\end{mathpar}

\begin{mathpar}
 \inferrule* [lab=behavior] {} {\meaningof{\langle a?b \rangle E} = \{ P \in \pi | P \equiv Q | u?(y)P', \\ \and \\\\ \and \\ \;\;\; u \in \meaningof{a}, \forall z.P'\{z/y\} \in \meaningof{E\{z/b\}}\}, \and \\ \meaningof{a!E} = \{ P \in \pi | P \equiv Q | x!\langle P' \rangle, x \in \meaningof{a} P' \in \meaningof{E}\} }
\end{mathpar}

\begin{mathpar}
 \inferrule* [lab=nominal] {} {\meaningof{\quotep{E}} = \{ \quotep{P} \in \quotep{\pi} | P \in \meaningof{E} \}, \and \meaningof{\quotep{P}} = \{ \quotep{Q} \in \quotep{\pi} | P \equiv Q \} \and \\ \meaningof{@\quotep{E}} = \{ P \in \pi | P \equiv @x, x \in \meaningof{E} \}}
\end{mathpar}

\begin{eqnarray*}
  \\
  \meaningof{-} : TS \to ST
\end{eqnarray*}

\begin{eqnarray*}
  \\
  L : TS \to ST
\end{eqnarray*}

\begin{eqnarray*}
  \\
  P \models E \iff P \in \meaningof{E}
\end{eqnarray*}

\begin{eqnarray*}
  P \approx_{L} Q \iff \forall E \in L. P \models E \iff Q \models E
\end{eqnarray*}

\begin{eqnarray*}
  P \approx_{K} Q
\end{eqnarray*}

\begin{eqnarray*}
  P \approx Q
\end{eqnarray*}

$\approx_{K} = \approx = \approx_{L}$

\subsubsection{Contextual duality}

Note that contexts extend the quotation operation to a family of
operations from processes to names. Given a context, $M$, we can
define a \emph{nominal context}, $\quotep{M}$ by $\quotep{M}[P] :=
\quotep{M[P]}$. To foreshadow what is to come we observe that these
operations enjoy a duality with processes very much like the duality
between vectors and maps from vectors to scalars.

Further, because the calculus is essentially higher-order, we have a
correspondence between contexts and processes. More specifically,
given a name $x$ and a context $M$ we can construct $M^{*}_{x}$ such
that 

\begin{mathpar}
  M^{*}_{x} | \lift{x}{P} \red M[P]
\end{mathpar}

namely,

\begin{mathpar}
  M^{*}_{x} := x?(u).M[\dropn{u}]
\end{mathpar}

The dependence of $M^{*}_{x}$ on a name makes it an abstraction, 

\begin{mathpar}
  M^{*} := (x)x?(u).M[\dropn{u}]
\end{mathpar}

\subsection{Additional notation}

It will sometimes be convenient to denote the process a name
quotes. We already have the notation $x = \quotep{P}$, but it will be
convenient to introduce an alternate notation, $\procn{x}$, when we
want to emphasize the connection to the use of the name. Note that, by
virtue of name equivalence, $\quotep{\procn{x}} \nameeq x$; so, the
notation is consistent with previous definitions.

Further, because names have structure it is possible to effect
substitutions on the basis of that structure. This means we need to
upgrade our notation for substitutions, which we accomplish by
adapting comprehension notation. Thus,

\begin{mathpar}
  P\{ y / x : x \in S \}
\end{mathpar}

is interpreted to mean the process derived from P by replacing (in a
capture-avoiding manner) each occurrence of $x$ in $S$ by $y$. For example,

\begin{mathpar}
  P\{ \quotep{\procn{x}|\procn{x}} / x : x \in \freenames{P} \}
\end{mathpar}

will replace each (occurrence) of a free name $x$ in $P$ by
$\quotep{\procn{x}|\procn{x}}$.

Also, we will avail ourselves of the notation $x^{L}$ and $x^{R}$ to
denote injections of a name into disjoint copies of the name
space. There are numerous ways to accomplish this. One example can be
found in \cite{MeredithR05}. This notation overloads to vectors of
names: $\vec{x}^{\pi} := (x_{i}^{\pi} \; : \; 0 \leq i < |\vec{x}| )$ where $\pi \in \{L,R\}$.

We also use $P^{\Box} := P|\Box$.

In \cite{MeredithR05} an interpretation of the new operator is
given. It turns out that there are several possible interpretations
all enjoying the requisite algebraic properties of the operator (see
\cite{milner91polyadicpi}). We will therefore make liberal use of
$(\nu\; \vec{x})P$.

% subsection the_syntax_and_semantics_of_the_notation_system (end)   

\input{qm2pi.qmops} 

\input{qm2pi.sterngerlach} 

\input{qm2pi.metric} 

% section concurrent_process_calculi (end)

%\input{qm2pi.proofsketch}

% section proof sketch (end)

%\input{qm2pi.slviaknots} 

% section spatial logic via knots (end)

\input{qm2pi.conclusion}

% section conclusion (end)

%\input{qm2pi.dtcodes} 

% section wiring algorithm (end)

\input{qm2pi.ack} 

% section acknowledgments (end)

\newpage


\bibliographystyle{plain}   
\bibliography{../../biblios/main.bib}

\input{qm2pi.rhodetails}

\end{document}

 

% subsection basic_interpretation (end)

%\input{qm2pi.rho.presentation} 
\subsection{The syntax and semantics of the notation system}\label{sub:the_syntax_and_semantics_of_the_notation_system} % (fold)

We now summarize a technical presentation of the calculus that
embodies our theory of dynamics. The typical presentation of such a
calculus follows the style of giving generators and relations on
them. The grammar, below, describing term constructors, freely
generates the set of processes, $\Proc$. This set is then quotiented
by a relation known as structural congruence and it is over this set
that the notion of dynamics is expressed. This presentation is
essentially that of \cite{MeredithR05} with the addition of
polyadicity and summation. For readability we have relegated some of
the technical subtleties to an appendix.

\subsubsection{Process grammar}\label{subsub:process_grammar}

\begin{mathpar}
  \inferrule* [lab=synchronization] {} {{M} \bc \pzero \;|\; x?F \;|\; x!C }
  \and
  \inferrule* [lab=abstraction] {} {{F} \bc (x)P}
  \and
  \inferrule* [lab=concretion] {} {{C} \bc \langle Q \rangle}
  \and
  \inferrule* [lab=process] {} {{P,Q} \bc M \;| \;P|Q \;|\; @{x}}
  \and
  \inferrule* [lab=name] {} {{x} \bc \quotep{P}}
\end{mathpar} 

Note that $\vec{x}$ (resp. $\vec{P}$) denotes a vector of names
(resp. processes) of length $|\vec{x}|$ (resp. $|\vec{P}|$). We adopt
the following useful abbreviations.

\begin{mathpar}
   x?(\vec{y}).P := x.(\vec{y})P \and  x\clift{\vec{P}} := x.\clift{\vec{P}}
   \and x!(y) := \lift{x}{\dropn{y}}
   \and \Pi_{i=0}^{n-1}P_i := P_0 | \ldots | P_{n-1}
\end{mathpar}

\subsubsection{Structural congruence}

\paragraph{Free and bound names and alpha-equivalence.} At the
core of structural equivalence is alpha-equivalence which identifies
process that are the same up to a change of variable. Formally, we
recognize the distinction between free and bound names. The free names
of a process, $\freenames{P}$, may be calculated recursively as
follows:

\begin{mathpar}
\freenames{\pzero} := \emptyset
  \and \\
  \freenames{x?(y).P} := \{ x \} \cup (\freenames{P} \setminus \{ y \})
  \and 
  \freenames{x!\langle P \rangle} := \{ x \} \cup \{ P \} 
  \and \\
  \freenames{P|Q} := \freenames{P} \cup \freenames{Q}
  \and \\
  \freenames{@{x}} := \{ x \}
\end{mathpar}

$\pi$
$\quotep{\pi}$

$\freenames{-} : \pi \to \mathcal{P}(\quotep{\pi})$

\begin{eqnarray*}
  \freenames{\pzero} & := & \emptyset \\
  \freenames{x?(y).P} & := & \{ x \} \cup (\freenames{P} \setminus \{ y \}) \\
  \freenames{x!\langle P \rangle} & := & \{ x \} \cup \{ P \} \\
  \freenames{P|Q} & := & \freenames{P} \cup \freenames{Q} \\
  \freenames{\dropn{x}} & := & \{ x \}
\end{eqnarray*}

The bound names of a process, $\boundnames{P}$, are those names occurring in $P$
that are not free. For example, in $x?(y).0$, the name $x$ is free, while $y$ is bound.

\begin{mathpar}
  \inferrule* [lab=monoidal-laws] {} { P|Q \equiv Q|P \and P|0 \equiv P \and P|(Q|R) \equiv (P|Q)|R }
\end{mathpar}

\begin{mathpar}
  \inferrule* [lab=alpha-equivalence] {} { (x)P \equiv (y)P\{y/x\} \and y \not\in \freenames{P} }
\end{mathpar}

\begin{definition}
Then two processes, $P,Q$, are alpha-equivalent if $P = Q\{\vec{y}/\vec{x}\}$ for
some $\vec{x} \in \boundnames{Q},\vec{y} \in \boundnames{P}$, where $Q\{\vec{y}/\vec{x}\}$
denotes the capture-avoiding substitution of $\vec{y}$ for $\vec{x}$ in $Q$.
\end{definition}

\begin{definition}
  The {\em structural congruence} \cite{SangiorgiWalker} , $\equiv$,
  between processes is the least congruence containing
  alpha-equivalence, satisfying the abelian monoid laws
  (associativity, commutativity and $\pzero$ as identity) for parallel
  composition $|$ and for summation $+$.
\end{definition}

\subsection{Name equivalence}

We take name equivalence, written $\nameeq$, to be the smallest
equivalence relation generated by the following rules.

\begin{mathpar}
\inferrule*[lab=Quote-drop]
{ }
{ \quotep{@{x}} \nameeq x }

\inferrule*[lab=Struct-equiv]
{ P \scong Q }
{ \quotep{P} \nameeq \quotep{Q} }
\end{mathpar}

The astute reader will have noticed that the mutual recursion of names
and processes imposes a mutual recursion on alpha-equivalence and
structural equivalence via name-equivalence. Fortunately, all of this
works out pleasantly and we may calculate in the natural way, free of
concern. The reader interested in the details is referred to the
appendix \ref{appendix:rho_details}.

\subsection{Substitution}

We use $\Proc$ for the set of processes, $\QProc$ for the set of
names, and $\id{\{}\vec{y} / \vec{x} \id{\}}$ to denote partial maps,
$s : \QProc \rightarrow \QProc$. A map, $s$ lifts, uniquely, to a map
on process terms, $\widehat{s} : \Proc \rightarrow \Proc$ by the
following equations.

\begin{mathpar}
  (0) \psubstp{Q}{P} := 0 \\
  (R \juxtap S) \psubstp{Q}{P}
  :=    
  (R)\psubstp{Q}{P} \juxtap (S) \psubstp{Q}{P} \\
  (x?(y).R) \psubstp{Q}{P}    
  :=    
  (x)\substp{Q}{P} (z)\concat( (R \psubstn{z}{y}) \psubstp{Q}{P} ) \\
  (\lift{x}{R}) \psubstp{Q}{P}  
  :=
  \lift{(x)\substp{Q}{P}}{ R \psubstp{Q}{P} } \\
%   (\dropn{x})  \psubstp{Q}{P}       
%   := 
%   \left\{ 
%     \begin{array}{ccc} 
%       \dropn{\quotep{Q}} & & x \nameeq \quotep{P} \\
%       \dropn{x} & & otherwise \\
%     \end{array}
%   \right. 
  (\dropn{x})  \psubstp{Q}{P}       
  := 
  \left\{ 
    \begin{array}{ccc} 
      Q & & x \nameeq \quotep{P} \\
      \dropn{x} & & otherwise \\
    \end{array}
  \right.
\end{mathpar}
 

where

\begin{eqnarray}
  (x)\id{\{} \lpquote Q \rpquote / \lpquote P \rpquote \id{\}}            = 
  \left\{ 
    \begin{array}{ccc}
      \lpquote Q \rpquote & & x \nameeq \lpquote P \rpquote \\
      x & & otherwise \\
    \end{array}
  \right. \nonumber
\end{eqnarray}

and $z$ is chosen distinct from $\quotep{P}$, $\quotep{Q}$, the free
names in $Q$, and all the names in $R$. Our $\alpha$-equivalence will
be built in the standard way from this substitution.

\begin{remark}\label{rem:no_self_referential_names}
  One consequence of these definitions is that $\forall P. \quotep{P}
  \not\in \freenames{P}$.
\end{remark}

\subsection{ Dynamic quote: an example }

Anticipating something of what's to come, consider applying the
substitution, $\widehat{\id{\{}u / z \id{\}}}$, to the following pair
of processes, $\lift{w}{y!(z)}$ and $w[ \lpquote y!(z) \rpquote ]$.

\begin{eqnarray}
	\lift{w}{y!(z)}\widehat{\id{\{}u / z \id{\}}}
		& = &
		\lift{w}{y!(u)} \nonumber\\
	w[ \lpquote y!(z) \rpquote ] \widehat{ \id{\{}u / z \id{\}} }
		& = &
		w[ \lpquote y!(z) \rpquote ] \nonumber
\end{eqnarray}

Because the body of the process between quotes is impervious to
substitution, we get radically different answers. In fact, by
examining the first process in an input context,
e.g. $x?(z).\lift{w}{y!(z)}$, we see that the process under the lift
operator may be shaped by prefixed inputs binding a name inside it. In
this sense, the lift operator will be seen as a way to dynamically
construct processes before reifying them as names.

Finally equipped with these standard features we can present the
dynamics of the calculus.

\subsubsection{Operational semantics} 

Finally, we introduce the computational dynamics. What marks these
algebras as distinct from other more traditionally studied algebraic
structures, e.g. vector spaces or polynomial rings, is the manner in
which dynamics is captured. In traditional structures, dynamics is typically
expressed through morphisms between such structures, as in linear maps
between vector spaces or morphisms between rings. In algebras
associated with the semantics of computation, the dynamics is
expressed as part of the algebraic structure itself, through a
reduction reduction relation typically denoted by $\red$. Below, we
give a recursive presentation of this relation for the calculus used
in the encoding.

$\red \subseteq \pi \times \pi$
$\red : \pi \to \mathcal{P}(\pi)$

\begin{mathpar}
  \inferrule* [lab=Comm] { \textsf{match}( x_{src}, x_{trgt} ) } { x_{trgt}?(y)P \; | \; x_{src}!\langle {Q} \rangle \red P\{\quotep{Q}/y}\} }
  \and \\
  \inferrule* [lab=Par] {{P} \red {P}'} {{{P} | {Q}} \red {{P}' | {Q}}}
  \and
  \inferrule* [lab=Equiv]{{{P} \scong {P}'} \andalso {{P}' \red {Q}'} \andalso {{Q}' \scong {Q}}}{{P} \red {Q}}
\end{mathpar}

\begin{eqnarray*}
  match_{\equiv} (\quotep{P},\quotep{Q}) & := & P \equiv Q \\
  match_{\dagger}(\quotep{P},\quotep{Q}) & := & \forall R. P|Q \red^{*} R => R \red^{*} 0 \\
  match_{K}(\quotep{P},\quotep{Q}) & := & K \mbox{ for some context } K
\end{eqnarray*}

$u?(x)P | u!\langle Q \rangle \red P\{\quotep{Q}/x\}$

%We write $\wred$ for $\red^*$, and $P\red$ if $\exists Q $ such that $ P \red Q$.
We write $P\red$ if $\exists Q $ such that $ P \red Q$ and $P\not\red$, otherwise.

\section{Replication}

As mentioned before, it is known that replication (and hence
recursion) can be implemented in a higher-order process algebra
\cite{SangiorgiWalker}. As our first example of calculation with the
machinery thus far presented we give the construction explicitly in
the {\rhoc}.

\begin{eqnarray}
	D_{x} & := & \prefix{x}{y}{(\binpar{\outputp{x}{y}}{@{y}})} \nonumber\\
	\bangp_{x}{P} & := & \binpar{{x}!\langle{\binpar{D_{x}}{P}}\rangle}{D_{x}} \nonumber
\end{eqnarray}

\begin{eqnarray}
	\bangp_{x}{P} & & \nonumber\\
	=
	& {x}!\langle{(\prefix{x}{y}{(\outputp{x}{y} | @{y})) | P}}\rangle 
	      | \prefix{x}{y}{(\outputp{x}{y} | @{y})} & \nonumber\\
	\red
	& (\outputp{x}{y} | @{y})\substn{\quotep{(\prefix{x}{y}{(@{y} | \outputp{x}{y})) | P}}}{y} & \nonumber\\
	=
	& \outputp{x}{\quotep{(\prefix{x}{y}{(\outputp{x}{y} | @{y})) | P}}}
	  | {(\prefix{x}{y}{(\outputp{x}{y} | @{y})) | P}} & \nonumber\\
	\red
	& \ldots & \nonumber\\
	\red^*
	& P | P | \ldots & \nonumber
\end{eqnarray}

Of course, this encoding, as an implementation, runs away, unfolding
$\bangp{P}$ eagerly. A lazier and more implementable replication
operator, restricted to input-guarded processes, may be obtained as follows.

\begin{eqnarray}
\bangp{\prefix{u}{v}{P}} 
	:= 
	\binpar{\lift{x}{\prefix{u}{v}{(\binpar{D(x)}{P})}}}{D(x)} \nonumber
\end{eqnarray}

\begin{remark}
  Note that the lazier definition still does not deal with summation
  or mixed summation (i.e. sums over input and output). The reader is
  invited to construct definitions of replication that deal with these
  features. 

  Further, the definitions are parameterized in a name, $x$. Can you,
  gentle reader, make a definition that eliminates this parameter and
  guarantees no accidental interaction between the replication
  machinery and the process being replicated -- i.e. no accidental
  sharing of names used by the process to get its work done and the
  name(s) used by the replication to effect copying. This latter
  revision of the definition of replication is crucial to obtaining
  the expected identity $!!P \sim !P$.
\end{remark}

\begin{remark}\label{rem:paradoxical_combinator}
  The reader familiar with the lambda calculus will have noticed the
  similarity between $D$ and the paradoxical combinator.

  [Ed. note: the existence of this seems to suggest we have to be more
  restrictive on the set of processes and names we admit if we are to
  support no-cloning.]
\end{remark}

\subsubsection{Bisimulation}

The computational dynamics gives rise to another kind of equivalence,
the equivalence of computational behavior. As previously mentioned
this is typically captured \emph{via} some form of bisimulation.

% The notion we use in this paper is weak barbed bisimulation
% \cite{milner91polyadicpi}.

The notion we use in this paper is derived from weak barbed
bisimulation \cite{milner91polyadicpi}. 

\begin{definition}
An \emph{observation relation}, $\downarrow_{\mathcal N}$, over a set
of names, $\mathcal N$, is the smallest relation satisfying the rules
below.

\infrule[Out-barb]{y \in {\mathcal N}, \; x \nameeq y}
		  {\outputp{x}{v} \downarrow_{\mathcal N} x}
\infrule[Par-barb]{\mbox{$P\downarrow_{\mathcal N} x$ or $Q\downarrow_{\mathcal N} x$}}
		  {\binpar{P}{Q} \downarrow_{\mathcal N} x}

We write $P \Downarrow_{\mathcal N} x$ if there is $Q$ such that 
$P \wred Q$ and $Q \downarrow_{\mathcal N} x$.
\end{definition}

\begin{definition}
%\label{def.bbisim}
An  ${\mathcal N}$-\emph{barbed bisimulation} over a set of names, ${\mathcal N}$, is a symmetric binary relation 
${\mathcal S}_{\mathcal N}$ between agents such that $P\rel{S}_{\mathcal N}Q$ implies:
\begin{enumerate}
\item If $P \red P'$ then $Q \wred Q'$ and $P'\rel{S}_{\mathcal N} Q'$.
\item If $P\downarrow_{\mathcal N} x$, then $Q\Downarrow_{\mathcal N} x$.
\end{enumerate}
$P$ is ${\mathcal N}$-barbed bisimilar to $Q$, written
$P \wbbisim_{\mathcal N} Q$, if $P \rel{S}_{\mathcal N} Q$ for some ${\mathcal N}$-barbed bisimulation ${\mathcal S}_{\mathcal N}$.
\end{definition}

$\mathcal{R} \subseteq \pi \times \pi$

$P \mathcal{R} Q => \forall P'. P \red P' \Rightarrow \exists Q'. Q \red Q', P' \mathcal{R} Q'$

$P \vdash x \Rightarrow Q \vdash x$

\begin{mathpar}
  \inferrule*[lab=Out-barb]{x \nameeq y}{{y}!\langle{Q}\rangle \vdash x}
  \and
  \inferrule*[lab=Par-barb]{\mbox{$P\vdash x$ or $Q\vdash x$}}{\binpar{P}{Q} \vdash x}
\end{mathpar}

\subsubsection{Contexts}

One of the principle advantages of computational calculi like the
$\pi$-calculus is a well-defined notion of context,
contextual-equivalence and a correlation between
contextual-equivalence and notions of bisimulation. The notion of
context allows the decomposition of a process into (sub-)process and
its syntactic environment, its context. Thus, a context may be
thought of as a process with a ``hole'' (written $\Box$) in it. The
application of a context $M$ to a process $P$, written $M[P]$, is
tantamount to filling the hole in $M$ with $P$. In this paper we do
not need the full weight of this theory, but do make use of the notion
of context in the proof the main theorem. 

\begin{mathpar}
  \inferrule* [lab=summation] {} {{M_{M},M_{N}} \bc \Box \;|\; x.M_{A} \;|\; M_{M}+M_{N}}
  \and
  \inferrule* [lab=agent] {} {{M_{A}} \bc (\vec{x})M_{P} \;| \; \clift{P_0,\ldots,M_{P},\ldots,P_N}}
  \and \\
  \inferrule* [lab=process] {} {{M_{P}} \bc M_{N} \;| \;P|M_{P} }
\end{mathpar} 

\begin{mathpar}
  \inferrule* [lab=sychronization] {} {M_{N} \bc \Box \;|\; x?M_{F} \;|\; x!M_{C}}
  \and
  \inferrule* [lab=abstraction] {} {{M_{F}} \bc (x)M_{P} }
  \and
  \inferrule* [lab=concretion] {} {{M_{C}} \bc \langle M_{P} \rangle }
  \and \\
  \inferrule* [lab=process] {} {{M_{P}} \bc M_{N} \;| \;P|M_{P} }
\end{mathpar}

\begin{definition}[contextual application] Given a context $M$, and
  process $P$, we define the \emph{contextual application}, $M[P] :=
  M\{P/\Box\}$. That is, the contextual application of M to P is the
  substitution of $P$ for $\Box$ in $M$.
\end{definition}

$\meaningof{-} : L \to \mathcal{P}(\pi)$

\begin{mathpar}
  \inferrule* [lab=collection] {} {\meaningof{true} = \pi, \and \meaningof{~E} = \pi \setminus \meaningof{E}, \and \meaningof{E_{1} \& E_{2}} = \meaningof{E_{1}} \cap \meaningof{E_{2}}}
\end{mathpar}

\begin{mathpar}
  \inferrule* [lab=structure] {} {\meaningof{0} = \{ P \in \pi | P \equiv 0 \}, \and \\ \meaningof{E_1 | E_2} = \{ P \in \pi | P \equiv P_{1} | P_{2}, P_{1} \in \meaningof{E_{1}}, P_{2} \in \meaningof{E_2}\} }
\end{mathpar}

\begin{mathpar}
 \inferrule* [lab=behavior] {} {\meaningof{\langle a?b \rangle E} = \{ P \in \pi | P \equiv Q | u?(y)P', \\ \and \\\\ \and \\ \;\;\; u \in \meaningof{a}, \forall z.P'\{z/y\} \in \meaningof{E\{z/b\}}\}, \and \\ \meaningof{a!E} = \{ P \in \pi | P \equiv Q | x!\langle P' \rangle, x \in \meaningof{a} P' \in \meaningof{E}\} }
\end{mathpar}

\begin{mathpar}
 \inferrule* [lab=nominal] {} {\meaningof{\quotep{E}} = \{ \quotep{P} \in \quotep{\pi} | P \in \meaningof{E} \}, \and \meaningof{\quotep{P}} = \{ \quotep{Q} \in \quotep{\pi} | P \equiv Q \} \and \\ \meaningof{@\quotep{E}} = \{ P \in \pi | P \equiv @x, x \in \meaningof{E} \}}
\end{mathpar}

\begin{eqnarray*}
  \\
  \meaningof{-} : TS \to ST
\end{eqnarray*}

\begin{eqnarray*}
  \\
  L : TS \to ST
\end{eqnarray*}

\begin{eqnarray*}
  \\
  P \models E \iff P \in \meaningof{E}
\end{eqnarray*}

\begin{eqnarray*}
  P \approx_{L} Q \iff \forall E \in L. P \models E \iff Q \models E
\end{eqnarray*}

\begin{eqnarray*}
  P \approx_{K} Q
\end{eqnarray*}

\begin{eqnarray*}
  P \approx Q
\end{eqnarray*}

$\approx_{K} = \approx = \approx_{L}$

\subsubsection{Contextual duality}

Note that contexts extend the quotation operation to a family of
operations from processes to names. Given a context, $M$, we can
define a \emph{nominal context}, $\quotep{M}$ by $\quotep{M}[P] :=
\quotep{M[P]}$. To foreshadow what is to come we observe that these
operations enjoy a duality with processes very much like the duality
between vectors and maps from vectors to scalars.

Further, because the calculus is essentially higher-order, we have a
correspondence between contexts and processes. More specifically,
given a name $x$ and a context $M$ we can construct $M^{*}_{x}$ such
that 

\begin{mathpar}
  M^{*}_{x} | \lift{x}{P} \red M[P]
\end{mathpar}

namely,

\begin{mathpar}
  M^{*}_{x} := x?(u).M[\dropn{u}]
\end{mathpar}

The dependence of $M^{*}_{x}$ on a name makes it an abstraction, 

\begin{mathpar}
  M^{*} := (x)x?(u).M[\dropn{u}]
\end{mathpar}

\subsection{Additional notation}

It will sometimes be convenient to denote the process a name
quotes. We already have the notation $x = \quotep{P}$, but it will be
convenient to introduce an alternate notation, $\procn{x}$, when we
want to emphasize the connection to the use of the name. Note that, by
virtue of name equivalence, $\quotep{\procn{x}} \nameeq x$; so, the
notation is consistent with previous definitions.

Further, because names have structure it is possible to effect
substitutions on the basis of that structure. This means we need to
upgrade our notation for substitutions, which we accomplish by
adapting comprehension notation. Thus,

\begin{mathpar}
  P\{ y / x : x \in S \}
\end{mathpar}

is interpreted to mean the process derived from P by replacing (in a
capture-avoiding manner) each occurrence of $x$ in $S$ by $y$. For example,

\begin{mathpar}
  P\{ \quotep{\procn{x}|\procn{x}} / x : x \in \freenames{P} \}
\end{mathpar}

will replace each (occurrence) of a free name $x$ in $P$ by
$\quotep{\procn{x}|\procn{x}}$.

Also, we will avail ourselves of the notation $x^{L}$ and $x^{R}$ to
denote injections of a name into disjoint copies of the name
space. There are numerous ways to accomplish this. One example can be
found in \cite{MeredithR05}. This notation overloads to vectors of
names: $\vec{x}^{\pi} := (x_{i}^{\pi} \; : \; 0 \leq i < |\vec{x}| )$ where $\pi \in \{L,R\}$.

We also use $P^{\Box} := P|\Box$.

In \cite{MeredithR05} an interpretation of the new operator is
given. It turns out that there are several possible interpretations
all enjoying the requisite algebraic properties of the operator (see
\cite{milner91polyadicpi}). We will therefore make liberal use of
$(\nu\; \vec{x})P$.

% subsection the_syntax_and_semantics_of_the_notation_system (end)   

\section{Interpretation of QM}
\subsection{Supporting definitions}
\subsubsection{Multiplication}
\begin{mathpar}
  \quotep{Q} \cdot \quotep{R} := \quotep{Q|R}
  \and \\
  \quotep{Q} \cdot P := P\{ \quotep{Q|R} / \quotep{R} : \quotep{R} \in \freenames{P} \}
\end{mathpar}

\paragraph{Discussion}
The first line needs little explanation. The second line says that
each free name of the process is replaced with the multiplication of
that name by the scalar. Multiplication of a scalar (name) by a state
(process) results in a process all the names of which have been `moved
over' by parallel composition with the process the scalar
quotes. There is a subtlety that the bound names have to be
manipulated so that multiplied names aren't accidentally
captured. There are many ways to achieve this.

\begin{remark}\label{rem:multiplication_identities}
  The reader is invited to verify that for all $x,y,z \in \QProc$ and $P \in \Proc$
  \begin{mathpar}
    x \cdot \quotep{0} \equiv x 
    \and
    x \cdot y \equiv y \cdot x
    \and
    x \cdot (y \cdot z) \equiv (x \cdot y) \cdot z
    \and \\
    \quotep{0} \cdot P \equiv P
    \and \\
    x \cdot (y \cdot P) \equiv (x \cdot y) \cdot P
    \and \\
    x \cdot (P|Q) \equiv (x \cdot P) | (x \cdot Q)
    \and \\    
  \end{mathpar}
\end{remark}

\subsubsection{Tensor product}

We define a tensor product on processes by structural induction.

\paragraph{Tensor of sums} First note that all summations, including
$\pzero$ and sequence, can be written $\Sigma_{i} x_{i}.A_{i} +
\Sigma_{j} x_{j}.C_{j}$, where we have grouped input-guarded processes
together and output-guarded processes together.

Thus, we can define the tensor product of two summations, $N_{1}\otimes N_{2}$, where

\begin{mathpar}
  N_{1} := \Sigma_{i} x_{i}.A_{i} + \Sigma_{j} x_{j}.C_{j}
  \and
  N_{2} := \Sigma_{i'} y_{i'}.B_{i'} + \Sigma_{j'} y_{j'}.D_{j'} 
\end{mathpar}

as follows.

\begin{mathpar}
  \Sigma_{i} x_{i}.A_{i} + \Sigma_{j} x_{j}.C_{j} \otimes \Sigma_{i'}
  y_{i'}.B_{i'} + \Sigma_{j'} y_{j'}.D_{j'} 
  \and \\
  := \; \Sigma_{i} \Sigma_{i'} \quotep{\stackrel{\vee}{x_{i}}| \stackrel{\vee}{y_{i'}}}.(A_{i}\otimes B_{i'}) \; | \; \Sigma_{i'} \Sigma_{i} \quotep{\stackrel{\vee}{y_{i'}}|\stackrel{\vee}{x_{i}}}.(B_{i'}\otimes A_{i})
  \and
  \;\; | \;\; \Sigma_{j} \Sigma_{j'} \quotep{\stackrel{\vee}{x_{j}}|\stackrel{\vee}{y_{j'}}}.(A_{j}\otimes B_{j'}) \; | \; \Sigma_{j'} \Sigma_{j} \quotep{\stackrel{\vee}{y_{j'}}|\stackrel{\vee}{x_{j}}}.(B_{j'}\otimes A_{j})
\end{mathpar}

\begin{remark}
  Do we need to $x^{L}$ and $y^{R}$ for this construction as well?
\end{remark}

\paragraph{Tensor of parallel compositions} Next, we distribute tensor
over par.

\begin{mathpar}
  P_{1}|P_{2} \otimes Q_{1}|Q_{2} := (P_{1} \otimes Q_{1}) | (P_{1}
  \otimes Q_{2}) | (P_{2} \otimes Q_{1}) | (P_{2} \otimes Q_{2})
\end{mathpar}

\paragraph{Tensor with dropped names} We treat tensor of a
process with a dropped name as parallel composition.

\begin{mathpar}
  P \otimes \dropn{x} := P | \dropn{x}
\end{mathpar}

\paragraph{Tensor of agents}

Finally, we need to define tensor on agents. Note that the definition
of tensor on normal products only tensors inputs with inputs and
outputs with outputs. Thus, we only have to define the operation on
``homogeneous'' pairings.

\begin{mathpar}
  (\vec{x})P \otimes (\vec{y})Q
  \and \\
  := (x_{0}^{L}|y_{0}^{R},\ldots,x_{0}^{L}|y_{n}^{R},\ldots,x_{m}^{L}|y_{0}^{R},\ldots,x_{m}^{L}|y_{n}^R)(P\{ \vec{x}^{L}/\vec{x}\} \otimes Q \{ \vec{y}^{R}/\vec{y}\})
  \and \\
  \clift{\vec{P}} \otimes \clift{\vec{Q}}
  \and \\
  := \clift{P_{0}\otimes Q_{0},\ldots,P_{0}\otimes Q_{n},\ldots,P_{m}\otimes Q_{0},\ldots,P_{m}\otimes Q_{n}}
\end{mathpar}

\begin{remark}
  Observe that arities of tensored abstractions matches arities of
  tensored concretions if the original arities matched. Note also that
  the length of the arities corresponds to the increase in dimension
  we see in ordinary vector space tensor product.
\end{remark}

\begin{remark}
  Operationally, this definition distributes the tensor down to
  components ``linked'' by summation. Tensor over summation is
  intriguing in that it mixes names. Moreover, as a consequence of the
  way it mixes names we have the identities for all $x \in \QProc$ and
  $P,Q \in \Proc$

  \begin{mathpar}
    (x \cdot P) \otimes Q \equiv x \cdot (P \otimes Q) \equiv P \otimes (x \cdot Q)
    \and
    P \otimes \pzero \equiv P
  \end{mathpar}

  that the reader is invited to verify.
\end{remark}

\subsubsection{Annihilation}
\begin{mathpar}
  P^{\perp} := \{ Q | \forall R. P|Q \red^{*} R \Rightarrow R \red^{*} \pzero \}
  \and \\
  P^{\underline{\perp}} := \Sigma_{Q \in P^{\perp}} \quotep{Q}?(y).(\dropn{y}|Q) | \Sigma_{Q \in P^{\perp}} \quotep{Q}\clift{\Box}
\end{mathpar}

\paragraph{Discussion} The reader will note that $P^{\perp}$ is a
\emph{set} of processes, while $P^{\underline{\perp}}$ is a
\emph{context}. We call the set $P^{\perp}$ the \emph{annihilators} of
$P$. The parallel composition of a process in the annihilators of $P$
with $P$ will result in a process, the state space of which has all
paths eventually leading to $\pzero$. Execution may endure loops; but
under reasonable conditions of fairness (naturally guaranteed under
most notions of bisimulation) such a composite process cannot get
stuck in such a loop and will, eventually pop out and terminate.

The context $P^{\underline{\perp}}$ is ready and willing to ``take the
$P$ out of'' the process to which it is applied. It will effectively
transmit the code of the process to which it is applied to one of the
annihilators and run the process against it.

\subsubsection{Evaluation}
We fix $M$ a domain of fully abstract interpretation with an equality
coincident with bisimulation. We take $\meaningof{\cdot} : \Proc \to
M$ to be the map interpreting processes and $\nmeaningof{\cdot} : \M
\to Proc$ to be the map running the other way. Then we define

\begin{mathpar}
  \int P := \nmeaningof{\meaningof{P}}
\end{mathpar}

\paragraph{Discussion}
There are many fully abstract interpretations of Milner's
$\pi$-calculus. Any of them can be used as a basis for interpreting
the reflective calculus here. Equipped with such a domain it is
largely a matter of grinding through to check that the Yoneda
construction for the normalization-by-evaluation program can be
extended to this setting.

\begin{remark}
  The reader is invited to verify that $\int (P^{\underline{\perp}}[P]) = 0$.
\end{remark}

\subsection{Quantum mechanics}

Table \ref{tbl:core_qm_op_defns} gives the core operational definitions

\begin{table}[htp]\label{tbl:core_qm_op_defns}
  \center{
    \fbox{
      \begin{tabular}{c|c}
        quantum mechanics & process calculus \\
        \hline
        scalar & $x := \quotep{P}$ \\
        state vector & $\state{P} := P$ \\
        dual & $\state{P}^{*} := \event{P^{\underline{\perp}}} := \quotep{P^{\underline{\perp}}}[-]$ \\
        matrix & $ \Sigma_{\alpha} \state{P_{\alpha}}x_{\alpha}\event{Q_{\alpha}}$ \\
        vector addition & $\state{P} + \state{Q} := \state{P | Q}$ \\
        tensor product & $\state{P} \otimes \state{Q} := \state{P \otimes Q}$ \\
        inner product & $\innerprod{P}{Q} := \quotep{\int P^{\underline{\perp}}[Q]}$ \\
      \end{tabular}
    }
  }
  \caption{QM - operational definitions}
\end{table}

where

\begin{mathpar}
  \prmatrix{P}{Q} := \fprmatrix{P}{\quotep{\pzero}}{Q}
  \and
  \fprmatrix{P}{x}{Q} := (\state{P},x,\event{Q})
  \and
  (\fprmatrix{P}{x}{Q})(\state{R}) := x \cdot \innerprod{Q}{R} \cdot \state{P}
  \and
  (\fprmatrix{P}{x}{Q})(\event{R}) := x \cdot \innerprod{R}{P} \cdot \event{Q}
\end{mathpar}

\paragraph{Discussion}
As promised: vectors (aka states) are represented as processes; duals
as contextual duals; inner product definition should be compared with
standard inner product definition for ....

\begin{remark}
  Assuming $\int (P^{\underline{\perp}}[P]) = 0$, the reader is
  invited to verify that $(\fprmatrix{P}{x}{P})(\state{P}) = x \cdot \state{P}$.
\end{remark}

\begin{remark}
  The reader is invited to verify that $\innerprod{P}{Q}$ could
  equally well have been written $\quotep{\int \stackrel{\vee}{x}}$
  where $x = \event{P^{\underline{\perp}}}(Q)$.

  One of the motivations for this remark is that there is another way
  to factor these operations. We could package up evaluation in the dual:

  \begin{mathpar}
    \state{P}^{*} := \event{\int P^{\underline{\perp}}} := \quotep{\int P^{\underline{\perp}}}[-]
  \end{mathpar}

  and then have inner product defined by
  
  \begin{mathpar}
    \innerprod{P}{Q} := \event{P}(Q)
  \end{mathpar}

  Hopefully, experience with the calculations will provide guidance on
  the best factoring.
\end{remark}

\begin{remark}
  Assuming $\int (P^{\underline{\perp}}[P]) = 0$, the reader is
  invited to verify that $\forall P,Q. (\prmatrix{0}{Q})(\state{0}) =
  \state{0}$ and dually $(\prmatrix{P}{0})(\event{0}) = \event{0}$.
\end{remark}

\begin{remark}
  i'm a little worried that i don't (yet) have proper support for
  complex conjugacy. But, the observation above may give us a
  clue. According to Abramsky, it must be the case that the scalars
  are iso to the homset of the identity for the tensor -- which the
  observation above characterizes. 

  For now, we will simply bookmark the notion with $\overline{x}$.
\end{remark}

\subsubsection{Adjointness}

We need to give a definition of $(\cdot)^{\dagger}$ for matrices. The
obvious candidate definition is
\begin{mathpar}
(\Sigma_{\alpha}\fprmatrix{P_{\alpha}}{x_{\alpha}}{Q_{\alpha}})^{\dagger}
= \Sigma_{\alpha}\fprmatrix{(Q_{\alpha}^{\underline{\perp}})^{*}}{\overline{x}_{\alpha}}{P_{\alpha}^{\underline{\perp}}} 
\end{mathpar}

But, $(Q_{\alpha}^{\underline{\perp}})^{*}$ requires a name along
which to communicate the process to achieve the context application.

\subsubsection{Basis for a basis}
If processes label states and ``addition'' of states (a.k.a. vector
addition) is interpreted as parallel composition, what corresponds to
notions of linear independence and basis? Here, we recall that Yoshida
has developed a set of \emph{combinators} for an asynchronous verison
of Milner's $\pi$-calculus. These are a finite set of processes such
any process can be expressed as parallel composition of these
combinators together with liberal uses of the new operator and
replication. We can simply give a translation of these into the
present calculus and have reasonable expectation that the property
carries over. That is, that the resultant set allows to express all
processes via parallel composition. Note, however, that there is no
new operator or replication in this calculus. As a result, we expect
that the corresponding set is actually infinite. That is, we expect
that the space is actually infinite dimensional.

\begin{remark}
  The attentive reader may be a bit concerned. Certainly, the
  collection $S$, $K$ and $I$ is a finite set of
  combinators. Shouldn't we expect to see a finite set of combinators
  for an effectively equivalent system? i am very sympathetic to this
  critique and feel it warrants full attention. On the other hand, i
  also have in mind the following analogy. The natural numbers, as a
  monoid under addition, has exactly $1$ generator, while the natural
  numbers, as a monoid under multiplication, has countably many
  generators (the primes). We observe that the application of the
  lambda calculus is much less resource sensitive than the parallel
  composition of the $\pi$-calculus. Could it be the case that we have
  an analogy of the form
  
  \begin{mathpar}
    m + n : MN :: m*n : M|N
  \end{mathpar}

  giving a similar blow up in the set of ``primes''?  This is such a
  wonderful thought that, even if it's not true, i think it's worth
  writing down.
\end{remark}
 

\documentclass[12pt]{llncs}
%\documentclass{jktr}

\usepackage[pdftex]{hyperref}                   
\usepackage {listings}
\usepackage {mathpartir}
\usepackage{bcprules}
%\usepackage{listings}
                       
\usepackage{graphicx} 
%\usepackage[margins=2.5cm,nohead,nofoot]{geometry}
%\usepackage{geometry}
\usepackage{amsfonts}
\usepackage{amstext}
\usepackage{latexsym}
\usepackage{amssymb}
\usepackage{color}


%\include{myPreamble}
\include{qm2pi.local} 

%\ifpdf
%\usepackage[pdftex]{graphicx}
%\else
%\usepackage{graphicx}
%\fi

 % \ifpdf
%  \usepackage{pdfsync}
%  \if


%\title{Brief Article}
%\author{David F. Snyder}
%\author{L.G. Meredith}

%\address{Dept. of Math., Texas State University--San Marcos, San Marcos, TX 78666}
       
\pagestyle{empty}


\begin{document}

\lstset{language=[Objective]Caml,frame=shadowbox}

\input{qm2pi.front}

% section front matter (end)

\input{qm2pi.intro} 
 
% section introduction (end)

% \input{qm2pi.knotations} 

% section notation (end)

\input{qm2pi.process.calculi} 

% section concurrent_process_calculi_and_spatial_logics_ (end)
    
%\input{qm2pi.knots2pi} 

%\input{qm2pi.trefoil} 

%\input{qm2pi.mainthm} 

% subsection basic_interpretation (end)

%\input{qm2pi.rho.presentation} 
\subsection{The syntax and semantics of the notation system}\label{sub:the_syntax_and_semantics_of_the_notation_system} % (fold)

We now summarize a technical presentation of the calculus that
embodies our theory of dynamics. The typical presentation of such a
calculus follows the style of giving generators and relations on
them. The grammar, below, describing term constructors, freely
generates the set of processes, $\Proc$. This set is then quotiented
by a relation known as structural congruence and it is over this set
that the notion of dynamics is expressed. This presentation is
essentially that of \cite{MeredithR05} with the addition of
polyadicity and summation. For readability we have relegated some of
the technical subtleties to an appendix.

\subsubsection{Process grammar}\label{subsub:process_grammar}

\begin{mathpar}
  \inferrule* [lab=synchronization] {} {{M} \bc \pzero \;|\; x?F \;|\; x!C }
  \and
  \inferrule* [lab=abstraction] {} {{F} \bc (x)P}
  \and
  \inferrule* [lab=concretion] {} {{C} \bc \langle Q \rangle}
  \and
  \inferrule* [lab=process] {} {{P,Q} \bc M \;| \;P|Q \;|\; @{x}}
  \and
  \inferrule* [lab=name] {} {{x} \bc \quotep{P}}
\end{mathpar} 

Note that $\vec{x}$ (resp. $\vec{P}$) denotes a vector of names
(resp. processes) of length $|\vec{x}|$ (resp. $|\vec{P}|$). We adopt
the following useful abbreviations.

\begin{mathpar}
   x?(\vec{y}).P := x.(\vec{y})P \and  x\clift{\vec{P}} := x.\clift{\vec{P}}
   \and x!(y) := \lift{x}{\dropn{y}}
   \and \Pi_{i=0}^{n-1}P_i := P_0 | \ldots | P_{n-1}
\end{mathpar}

\subsubsection{Structural congruence}

\paragraph{Free and bound names and alpha-equivalence.} At the
core of structural equivalence is alpha-equivalence which identifies
process that are the same up to a change of variable. Formally, we
recognize the distinction between free and bound names. The free names
of a process, $\freenames{P}$, may be calculated recursively as
follows:

\begin{mathpar}
\freenames{\pzero} := \emptyset
  \and \\
  \freenames{x?(y).P} := \{ x \} \cup (\freenames{P} \setminus \{ y \})
  \and 
  \freenames{x!\langle P \rangle} := \{ x \} \cup \{ P \} 
  \and \\
  \freenames{P|Q} := \freenames{P} \cup \freenames{Q}
  \and \\
  \freenames{@{x}} := \{ x \}
\end{mathpar}

$\pi$
$\quotep{\pi}$

$\freenames{-} : \pi \to \mathcal{P}(\quotep{\pi})$

\begin{eqnarray*}
  \freenames{\pzero} & := & \emptyset \\
  \freenames{x?(y).P} & := & \{ x \} \cup (\freenames{P} \setminus \{ y \}) \\
  \freenames{x!\langle P \rangle} & := & \{ x \} \cup \{ P \} \\
  \freenames{P|Q} & := & \freenames{P} \cup \freenames{Q} \\
  \freenames{\dropn{x}} & := & \{ x \}
\end{eqnarray*}

The bound names of a process, $\boundnames{P}$, are those names occurring in $P$
that are not free. For example, in $x?(y).0$, the name $x$ is free, while $y$ is bound.

\begin{mathpar}
  \inferrule* [lab=monoidal-laws] {} { P|Q \equiv Q|P \and P|0 \equiv P \and P|(Q|R) \equiv (P|Q)|R }
\end{mathpar}

\begin{mathpar}
  \inferrule* [lab=alpha-equivalence] {} { (x)P \equiv (y)P\{y/x\} \and y \not\in \freenames{P} }
\end{mathpar}

\begin{definition}
Then two processes, $P,Q$, are alpha-equivalent if $P = Q\{\vec{y}/\vec{x}\}$ for
some $\vec{x} \in \boundnames{Q},\vec{y} \in \boundnames{P}$, where $Q\{\vec{y}/\vec{x}\}$
denotes the capture-avoiding substitution of $\vec{y}$ for $\vec{x}$ in $Q$.
\end{definition}

\begin{definition}
  The {\em structural congruence} \cite{SangiorgiWalker} , $\equiv$,
  between processes is the least congruence containing
  alpha-equivalence, satisfying the abelian monoid laws
  (associativity, commutativity and $\pzero$ as identity) for parallel
  composition $|$ and for summation $+$.
\end{definition}

\subsection{Name equivalence}

We take name equivalence, written $\nameeq$, to be the smallest
equivalence relation generated by the following rules.

\begin{mathpar}
\inferrule*[lab=Quote-drop]
{ }
{ \quotep{@{x}} \nameeq x }

\inferrule*[lab=Struct-equiv]
{ P \scong Q }
{ \quotep{P} \nameeq \quotep{Q} }
\end{mathpar}

The astute reader will have noticed that the mutual recursion of names
and processes imposes a mutual recursion on alpha-equivalence and
structural equivalence via name-equivalence. Fortunately, all of this
works out pleasantly and we may calculate in the natural way, free of
concern. The reader interested in the details is referred to the
appendix \ref{appendix:rho_details}.

\subsection{Substitution}

We use $\Proc$ for the set of processes, $\QProc$ for the set of
names, and $\id{\{}\vec{y} / \vec{x} \id{\}}$ to denote partial maps,
$s : \QProc \rightarrow \QProc$. A map, $s$ lifts, uniquely, to a map
on process terms, $\widehat{s} : \Proc \rightarrow \Proc$ by the
following equations.

\begin{mathpar}
  (0) \psubstp{Q}{P} := 0 \\
  (R \juxtap S) \psubstp{Q}{P}
  :=    
  (R)\psubstp{Q}{P} \juxtap (S) \psubstp{Q}{P} \\
  (x?(y).R) \psubstp{Q}{P}    
  :=    
  (x)\substp{Q}{P} (z)\concat( (R \psubstn{z}{y}) \psubstp{Q}{P} ) \\
  (\lift{x}{R}) \psubstp{Q}{P}  
  :=
  \lift{(x)\substp{Q}{P}}{ R \psubstp{Q}{P} } \\
%   (\dropn{x})  \psubstp{Q}{P}       
%   := 
%   \left\{ 
%     \begin{array}{ccc} 
%       \dropn{\quotep{Q}} & & x \nameeq \quotep{P} \\
%       \dropn{x} & & otherwise \\
%     \end{array}
%   \right. 
  (\dropn{x})  \psubstp{Q}{P}       
  := 
  \left\{ 
    \begin{array}{ccc} 
      Q & & x \nameeq \quotep{P} \\
      \dropn{x} & & otherwise \\
    \end{array}
  \right.
\end{mathpar}
 

where

\begin{eqnarray}
  (x)\id{\{} \lpquote Q \rpquote / \lpquote P \rpquote \id{\}}            = 
  \left\{ 
    \begin{array}{ccc}
      \lpquote Q \rpquote & & x \nameeq \lpquote P \rpquote \\
      x & & otherwise \\
    \end{array}
  \right. \nonumber
\end{eqnarray}

and $z$ is chosen distinct from $\quotep{P}$, $\quotep{Q}$, the free
names in $Q$, and all the names in $R$. Our $\alpha$-equivalence will
be built in the standard way from this substitution.

\begin{remark}\label{rem:no_self_referential_names}
  One consequence of these definitions is that $\forall P. \quotep{P}
  \not\in \freenames{P}$.
\end{remark}

\subsection{ Dynamic quote: an example }

Anticipating something of what's to come, consider applying the
substitution, $\widehat{\id{\{}u / z \id{\}}}$, to the following pair
of processes, $\lift{w}{y!(z)}$ and $w[ \lpquote y!(z) \rpquote ]$.

\begin{eqnarray}
	\lift{w}{y!(z)}\widehat{\id{\{}u / z \id{\}}}
		& = &
		\lift{w}{y!(u)} \nonumber\\
	w[ \lpquote y!(z) \rpquote ] \widehat{ \id{\{}u / z \id{\}} }
		& = &
		w[ \lpquote y!(z) \rpquote ] \nonumber
\end{eqnarray}

Because the body of the process between quotes is impervious to
substitution, we get radically different answers. In fact, by
examining the first process in an input context,
e.g. $x?(z).\lift{w}{y!(z)}$, we see that the process under the lift
operator may be shaped by prefixed inputs binding a name inside it. In
this sense, the lift operator will be seen as a way to dynamically
construct processes before reifying them as names.

Finally equipped with these standard features we can present the
dynamics of the calculus.

\subsubsection{Operational semantics} 

Finally, we introduce the computational dynamics. What marks these
algebras as distinct from other more traditionally studied algebraic
structures, e.g. vector spaces or polynomial rings, is the manner in
which dynamics is captured. In traditional structures, dynamics is typically
expressed through morphisms between such structures, as in linear maps
between vector spaces or morphisms between rings. In algebras
associated with the semantics of computation, the dynamics is
expressed as part of the algebraic structure itself, through a
reduction reduction relation typically denoted by $\red$. Below, we
give a recursive presentation of this relation for the calculus used
in the encoding.

$\red \subseteq \pi \times \pi$
$\red : \pi \to \mathcal{P}(\pi)$

\begin{mathpar}
  \inferrule* [lab=Comm] { \textsf{match}( x_{src}, x_{trgt} ) } { x_{trgt}?(y)P \; | \; x_{src}!\langle {Q} \rangle \red P\{\quotep{Q}/y}\} }
  \and \\
  \inferrule* [lab=Par] {{P} \red {P}'} {{{P} | {Q}} \red {{P}' | {Q}}}
  \and
  \inferrule* [lab=Equiv]{{{P} \scong {P}'} \andalso {{P}' \red {Q}'} \andalso {{Q}' \scong {Q}}}{{P} \red {Q}}
\end{mathpar}

\begin{eqnarray*}
  match_{\equiv} (\quotep{P},\quotep{Q}) & := & P \equiv Q \\
  match_{\dagger}(\quotep{P},\quotep{Q}) & := & \forall R. P|Q \red^{*} R => R \red^{*} 0 \\
  match_{K}(\quotep{P},\quotep{Q}) & := & K \mbox{ for some context } K
\end{eqnarray*}

$u?(x)P | u!\langle Q \rangle \red P\{\quotep{Q}/x\}$

%We write $\wred$ for $\red^*$, and $P\red$ if $\exists Q $ such that $ P \red Q$.
We write $P\red$ if $\exists Q $ such that $ P \red Q$ and $P\not\red$, otherwise.

\section{Replication}

As mentioned before, it is known that replication (and hence
recursion) can be implemented in a higher-order process algebra
\cite{SangiorgiWalker}. As our first example of calculation with the
machinery thus far presented we give the construction explicitly in
the {\rhoc}.

\begin{eqnarray}
	D_{x} & := & \prefix{x}{y}{(\binpar{\outputp{x}{y}}{@{y}})} \nonumber\\
	\bangp_{x}{P} & := & \binpar{{x}!\langle{\binpar{D_{x}}{P}}\rangle}{D_{x}} \nonumber
\end{eqnarray}

\begin{eqnarray}
	\bangp_{x}{P} & & \nonumber\\
	=
	& {x}!\langle{(\prefix{x}{y}{(\outputp{x}{y} | @{y})) | P}}\rangle 
	      | \prefix{x}{y}{(\outputp{x}{y} | @{y})} & \nonumber\\
	\red
	& (\outputp{x}{y} | @{y})\substn{\quotep{(\prefix{x}{y}{(@{y} | \outputp{x}{y})) | P}}}{y} & \nonumber\\
	=
	& \outputp{x}{\quotep{(\prefix{x}{y}{(\outputp{x}{y} | @{y})) | P}}}
	  | {(\prefix{x}{y}{(\outputp{x}{y} | @{y})) | P}} & \nonumber\\
	\red
	& \ldots & \nonumber\\
	\red^*
	& P | P | \ldots & \nonumber
\end{eqnarray}

Of course, this encoding, as an implementation, runs away, unfolding
$\bangp{P}$ eagerly. A lazier and more implementable replication
operator, restricted to input-guarded processes, may be obtained as follows.

\begin{eqnarray}
\bangp{\prefix{u}{v}{P}} 
	:= 
	\binpar{\lift{x}{\prefix{u}{v}{(\binpar{D(x)}{P})}}}{D(x)} \nonumber
\end{eqnarray}

\begin{remark}
  Note that the lazier definition still does not deal with summation
  or mixed summation (i.e. sums over input and output). The reader is
  invited to construct definitions of replication that deal with these
  features. 

  Further, the definitions are parameterized in a name, $x$. Can you,
  gentle reader, make a definition that eliminates this parameter and
  guarantees no accidental interaction between the replication
  machinery and the process being replicated -- i.e. no accidental
  sharing of names used by the process to get its work done and the
  name(s) used by the replication to effect copying. This latter
  revision of the definition of replication is crucial to obtaining
  the expected identity $!!P \sim !P$.
\end{remark}

\begin{remark}\label{rem:paradoxical_combinator}
  The reader familiar with the lambda calculus will have noticed the
  similarity between $D$ and the paradoxical combinator.

  [Ed. note: the existence of this seems to suggest we have to be more
  restrictive on the set of processes and names we admit if we are to
  support no-cloning.]
\end{remark}

\subsubsection{Bisimulation}

The computational dynamics gives rise to another kind of equivalence,
the equivalence of computational behavior. As previously mentioned
this is typically captured \emph{via} some form of bisimulation.

% The notion we use in this paper is weak barbed bisimulation
% \cite{milner91polyadicpi}.

The notion we use in this paper is derived from weak barbed
bisimulation \cite{milner91polyadicpi}. 

\begin{definition}
An \emph{observation relation}, $\downarrow_{\mathcal N}$, over a set
of names, $\mathcal N$, is the smallest relation satisfying the rules
below.

\infrule[Out-barb]{y \in {\mathcal N}, \; x \nameeq y}
		  {\outputp{x}{v} \downarrow_{\mathcal N} x}
\infrule[Par-barb]{\mbox{$P\downarrow_{\mathcal N} x$ or $Q\downarrow_{\mathcal N} x$}}
		  {\binpar{P}{Q} \downarrow_{\mathcal N} x}

We write $P \Downarrow_{\mathcal N} x$ if there is $Q$ such that 
$P \wred Q$ and $Q \downarrow_{\mathcal N} x$.
\end{definition}

\begin{definition}
%\label{def.bbisim}
An  ${\mathcal N}$-\emph{barbed bisimulation} over a set of names, ${\mathcal N}$, is a symmetric binary relation 
${\mathcal S}_{\mathcal N}$ between agents such that $P\rel{S}_{\mathcal N}Q$ implies:
\begin{enumerate}
\item If $P \red P'$ then $Q \wred Q'$ and $P'\rel{S}_{\mathcal N} Q'$.
\item If $P\downarrow_{\mathcal N} x$, then $Q\Downarrow_{\mathcal N} x$.
\end{enumerate}
$P$ is ${\mathcal N}$-barbed bisimilar to $Q$, written
$P \wbbisim_{\mathcal N} Q$, if $P \rel{S}_{\mathcal N} Q$ for some ${\mathcal N}$-barbed bisimulation ${\mathcal S}_{\mathcal N}$.
\end{definition}

$\mathcal{R} \subseteq \pi \times \pi$

$P \mathcal{R} Q => \forall P'. P \red P' \Rightarrow \exists Q'. Q \red Q', P' \mathcal{R} Q'$

$P \vdash x \Rightarrow Q \vdash x$

\begin{mathpar}
  \inferrule*[lab=Out-barb]{x \nameeq y}{{y}!\langle{Q}\rangle \vdash x}
  \and
  \inferrule*[lab=Par-barb]{\mbox{$P\vdash x$ or $Q\vdash x$}}{\binpar{P}{Q} \vdash x}
\end{mathpar}

\subsubsection{Contexts}

One of the principle advantages of computational calculi like the
$\pi$-calculus is a well-defined notion of context,
contextual-equivalence and a correlation between
contextual-equivalence and notions of bisimulation. The notion of
context allows the decomposition of a process into (sub-)process and
its syntactic environment, its context. Thus, a context may be
thought of as a process with a ``hole'' (written $\Box$) in it. The
application of a context $M$ to a process $P$, written $M[P]$, is
tantamount to filling the hole in $M$ with $P$. In this paper we do
not need the full weight of this theory, but do make use of the notion
of context in the proof the main theorem. 

\begin{mathpar}
  \inferrule* [lab=summation] {} {{M_{M},M_{N}} \bc \Box \;|\; x.M_{A} \;|\; M_{M}+M_{N}}
  \and
  \inferrule* [lab=agent] {} {{M_{A}} \bc (\vec{x})M_{P} \;| \; \clift{P_0,\ldots,M_{P},\ldots,P_N}}
  \and \\
  \inferrule* [lab=process] {} {{M_{P}} \bc M_{N} \;| \;P|M_{P} }
\end{mathpar} 

\begin{mathpar}
  \inferrule* [lab=sychronization] {} {M_{N} \bc \Box \;|\; x?M_{F} \;|\; x!M_{C}}
  \and
  \inferrule* [lab=abstraction] {} {{M_{F}} \bc (x)M_{P} }
  \and
  \inferrule* [lab=concretion] {} {{M_{C}} \bc \langle M_{P} \rangle }
  \and \\
  \inferrule* [lab=process] {} {{M_{P}} \bc M_{N} \;| \;P|M_{P} }
\end{mathpar}

\begin{definition}[contextual application] Given a context $M$, and
  process $P$, we define the \emph{contextual application}, $M[P] :=
  M\{P/\Box\}$. That is, the contextual application of M to P is the
  substitution of $P$ for $\Box$ in $M$.
\end{definition}

$\meaningof{-} : L \to \mathcal{P}(\pi)$

\begin{mathpar}
  \inferrule* [lab=collection] {} {\meaningof{true} = \pi, \and \meaningof{~E} = \pi \setminus \meaningof{E}, \and \meaningof{E_{1} \& E_{2}} = \meaningof{E_{1}} \cap \meaningof{E_{2}}}
\end{mathpar}

\begin{mathpar}
  \inferrule* [lab=structure] {} {\meaningof{0} = \{ P \in \pi | P \equiv 0 \}, \and \\ \meaningof{E_1 | E_2} = \{ P \in \pi | P \equiv P_{1} | P_{2}, P_{1} \in \meaningof{E_{1}}, P_{2} \in \meaningof{E_2}\} }
\end{mathpar}

\begin{mathpar}
 \inferrule* [lab=behavior] {} {\meaningof{\langle a?b \rangle E} = \{ P \in \pi | P \equiv Q | u?(y)P', \\ \and \\\\ \and \\ \;\;\; u \in \meaningof{a}, \forall z.P'\{z/y\} \in \meaningof{E\{z/b\}}\}, \and \\ \meaningof{a!E} = \{ P \in \pi | P \equiv Q | x!\langle P' \rangle, x \in \meaningof{a} P' \in \meaningof{E}\} }
\end{mathpar}

\begin{mathpar}
 \inferrule* [lab=nominal] {} {\meaningof{\quotep{E}} = \{ \quotep{P} \in \quotep{\pi} | P \in \meaningof{E} \}, \and \meaningof{\quotep{P}} = \{ \quotep{Q} \in \quotep{\pi} | P \equiv Q \} \and \\ \meaningof{@\quotep{E}} = \{ P \in \pi | P \equiv @x, x \in \meaningof{E} \}}
\end{mathpar}

\begin{eqnarray*}
  \\
  \meaningof{-} : TS \to ST
\end{eqnarray*}

\begin{eqnarray*}
  \\
  L : TS \to ST
\end{eqnarray*}

\begin{eqnarray*}
  \\
  P \models E \iff P \in \meaningof{E}
\end{eqnarray*}

\begin{eqnarray*}
  P \approx_{L} Q \iff \forall E \in L. P \models E \iff Q \models E
\end{eqnarray*}

\begin{eqnarray*}
  P \approx_{K} Q
\end{eqnarray*}

\begin{eqnarray*}
  P \approx Q
\end{eqnarray*}

$\approx_{K} = \approx = \approx_{L}$

\subsubsection{Contextual duality}

Note that contexts extend the quotation operation to a family of
operations from processes to names. Given a context, $M$, we can
define a \emph{nominal context}, $\quotep{M}$ by $\quotep{M}[P] :=
\quotep{M[P]}$. To foreshadow what is to come we observe that these
operations enjoy a duality with processes very much like the duality
between vectors and maps from vectors to scalars.

Further, because the calculus is essentially higher-order, we have a
correspondence between contexts and processes. More specifically,
given a name $x$ and a context $M$ we can construct $M^{*}_{x}$ such
that 

\begin{mathpar}
  M^{*}_{x} | \lift{x}{P} \red M[P]
\end{mathpar}

namely,

\begin{mathpar}
  M^{*}_{x} := x?(u).M[\dropn{u}]
\end{mathpar}

The dependence of $M^{*}_{x}$ on a name makes it an abstraction, 

\begin{mathpar}
  M^{*} := (x)x?(u).M[\dropn{u}]
\end{mathpar}

\subsection{Additional notation}

It will sometimes be convenient to denote the process a name
quotes. We already have the notation $x = \quotep{P}$, but it will be
convenient to introduce an alternate notation, $\procn{x}$, when we
want to emphasize the connection to the use of the name. Note that, by
virtue of name equivalence, $\quotep{\procn{x}} \nameeq x$; so, the
notation is consistent with previous definitions.

Further, because names have structure it is possible to effect
substitutions on the basis of that structure. This means we need to
upgrade our notation for substitutions, which we accomplish by
adapting comprehension notation. Thus,

\begin{mathpar}
  P\{ y / x : x \in S \}
\end{mathpar}

is interpreted to mean the process derived from P by replacing (in a
capture-avoiding manner) each occurrence of $x$ in $S$ by $y$. For example,

\begin{mathpar}
  P\{ \quotep{\procn{x}|\procn{x}} / x : x \in \freenames{P} \}
\end{mathpar}

will replace each (occurrence) of a free name $x$ in $P$ by
$\quotep{\procn{x}|\procn{x}}$.

Also, we will avail ourselves of the notation $x^{L}$ and $x^{R}$ to
denote injections of a name into disjoint copies of the name
space. There are numerous ways to accomplish this. One example can be
found in \cite{MeredithR05}. This notation overloads to vectors of
names: $\vec{x}^{\pi} := (x_{i}^{\pi} \; : \; 0 \leq i < |\vec{x}| )$ where $\pi \in \{L,R\}$.

We also use $P^{\Box} := P|\Box$.

In \cite{MeredithR05} an interpretation of the new operator is
given. It turns out that there are several possible interpretations
all enjoying the requisite algebraic properties of the operator (see
\cite{milner91polyadicpi}). We will therefore make liberal use of
$(\nu\; \vec{x})P$.

% subsection the_syntax_and_semantics_of_the_notation_system (end)   

\input{qm2pi.qmops} 

\input{qm2pi.sterngerlach} 

\input{qm2pi.metric} 

% section concurrent_process_calculi (end)

%\input{qm2pi.proofsketch}

% section proof sketch (end)

%\input{qm2pi.slviaknots} 

% section spatial logic via knots (end)

\input{qm2pi.conclusion}

% section conclusion (end)

%\input{qm2pi.dtcodes} 

% section wiring algorithm (end)

\input{qm2pi.ack} 

% section acknowledgments (end)

\newpage


\bibliographystyle{plain}   
\bibliography{../../biblios/main.bib}

\input{qm2pi.rhodetails}

\end{document}

 

\documentclass[12pt]{llncs}
%\documentclass{jktr}

\usepackage[pdftex]{hyperref}                   
\usepackage {listings}
\usepackage {mathpartir}
\usepackage{bcprules}
%\usepackage{listings}
                       
\usepackage{graphicx} 
%\usepackage[margins=2.5cm,nohead,nofoot]{geometry}
%\usepackage{geometry}
\usepackage{amsfonts}
\usepackage{amstext}
\usepackage{latexsym}
\usepackage{amssymb}
\usepackage{color}


%\include{myPreamble}
\include{qm2pi.local} 

%\ifpdf
%\usepackage[pdftex]{graphicx}
%\else
%\usepackage{graphicx}
%\fi

 % \ifpdf
%  \usepackage{pdfsync}
%  \if


%\title{Brief Article}
%\author{David F. Snyder}
%\author{L.G. Meredith}

%\address{Dept. of Math., Texas State University--San Marcos, San Marcos, TX 78666}
       
\pagestyle{empty}


\begin{document}

\lstset{language=[Objective]Caml,frame=shadowbox}

\input{qm2pi.front}

% section front matter (end)

\input{qm2pi.intro} 
 
% section introduction (end)

% \input{qm2pi.knotations} 

% section notation (end)

\input{qm2pi.process.calculi} 

% section concurrent_process_calculi_and_spatial_logics_ (end)
    
%\input{qm2pi.knots2pi} 

%\input{qm2pi.trefoil} 

%\input{qm2pi.mainthm} 

% subsection basic_interpretation (end)

%\input{qm2pi.rho.presentation} 
\subsection{The syntax and semantics of the notation system}\label{sub:the_syntax_and_semantics_of_the_notation_system} % (fold)

We now summarize a technical presentation of the calculus that
embodies our theory of dynamics. The typical presentation of such a
calculus follows the style of giving generators and relations on
them. The grammar, below, describing term constructors, freely
generates the set of processes, $\Proc$. This set is then quotiented
by a relation known as structural congruence and it is over this set
that the notion of dynamics is expressed. This presentation is
essentially that of \cite{MeredithR05} with the addition of
polyadicity and summation. For readability we have relegated some of
the technical subtleties to an appendix.

\subsubsection{Process grammar}\label{subsub:process_grammar}

\begin{mathpar}
  \inferrule* [lab=synchronization] {} {{M} \bc \pzero \;|\; x?F \;|\; x!C }
  \and
  \inferrule* [lab=abstraction] {} {{F} \bc (x)P}
  \and
  \inferrule* [lab=concretion] {} {{C} \bc \langle Q \rangle}
  \and
  \inferrule* [lab=process] {} {{P,Q} \bc M \;| \;P|Q \;|\; @{x}}
  \and
  \inferrule* [lab=name] {} {{x} \bc \quotep{P}}
\end{mathpar} 

Note that $\vec{x}$ (resp. $\vec{P}$) denotes a vector of names
(resp. processes) of length $|\vec{x}|$ (resp. $|\vec{P}|$). We adopt
the following useful abbreviations.

\begin{mathpar}
   x?(\vec{y}).P := x.(\vec{y})P \and  x\clift{\vec{P}} := x.\clift{\vec{P}}
   \and x!(y) := \lift{x}{\dropn{y}}
   \and \Pi_{i=0}^{n-1}P_i := P_0 | \ldots | P_{n-1}
\end{mathpar}

\subsubsection{Structural congruence}

\paragraph{Free and bound names and alpha-equivalence.} At the
core of structural equivalence is alpha-equivalence which identifies
process that are the same up to a change of variable. Formally, we
recognize the distinction between free and bound names. The free names
of a process, $\freenames{P}$, may be calculated recursively as
follows:

\begin{mathpar}
\freenames{\pzero} := \emptyset
  \and \\
  \freenames{x?(y).P} := \{ x \} \cup (\freenames{P} \setminus \{ y \})
  \and 
  \freenames{x!\langle P \rangle} := \{ x \} \cup \{ P \} 
  \and \\
  \freenames{P|Q} := \freenames{P} \cup \freenames{Q}
  \and \\
  \freenames{@{x}} := \{ x \}
\end{mathpar}

$\pi$
$\quotep{\pi}$

$\freenames{-} : \pi \to \mathcal{P}(\quotep{\pi})$

\begin{eqnarray*}
  \freenames{\pzero} & := & \emptyset \\
  \freenames{x?(y).P} & := & \{ x \} \cup (\freenames{P} \setminus \{ y \}) \\
  \freenames{x!\langle P \rangle} & := & \{ x \} \cup \{ P \} \\
  \freenames{P|Q} & := & \freenames{P} \cup \freenames{Q} \\
  \freenames{\dropn{x}} & := & \{ x \}
\end{eqnarray*}

The bound names of a process, $\boundnames{P}$, are those names occurring in $P$
that are not free. For example, in $x?(y).0$, the name $x$ is free, while $y$ is bound.

\begin{mathpar}
  \inferrule* [lab=monoidal-laws] {} { P|Q \equiv Q|P \and P|0 \equiv P \and P|(Q|R) \equiv (P|Q)|R }
\end{mathpar}

\begin{mathpar}
  \inferrule* [lab=alpha-equivalence] {} { (x)P \equiv (y)P\{y/x\} \and y \not\in \freenames{P} }
\end{mathpar}

\begin{definition}
Then two processes, $P,Q$, are alpha-equivalent if $P = Q\{\vec{y}/\vec{x}\}$ for
some $\vec{x} \in \boundnames{Q},\vec{y} \in \boundnames{P}$, where $Q\{\vec{y}/\vec{x}\}$
denotes the capture-avoiding substitution of $\vec{y}$ for $\vec{x}$ in $Q$.
\end{definition}

\begin{definition}
  The {\em structural congruence} \cite{SangiorgiWalker} , $\equiv$,
  between processes is the least congruence containing
  alpha-equivalence, satisfying the abelian monoid laws
  (associativity, commutativity and $\pzero$ as identity) for parallel
  composition $|$ and for summation $+$.
\end{definition}

\subsection{Name equivalence}

We take name equivalence, written $\nameeq$, to be the smallest
equivalence relation generated by the following rules.

\begin{mathpar}
\inferrule*[lab=Quote-drop]
{ }
{ \quotep{@{x}} \nameeq x }

\inferrule*[lab=Struct-equiv]
{ P \scong Q }
{ \quotep{P} \nameeq \quotep{Q} }
\end{mathpar}

The astute reader will have noticed that the mutual recursion of names
and processes imposes a mutual recursion on alpha-equivalence and
structural equivalence via name-equivalence. Fortunately, all of this
works out pleasantly and we may calculate in the natural way, free of
concern. The reader interested in the details is referred to the
appendix \ref{appendix:rho_details}.

\subsection{Substitution}

We use $\Proc$ for the set of processes, $\QProc$ for the set of
names, and $\id{\{}\vec{y} / \vec{x} \id{\}}$ to denote partial maps,
$s : \QProc \rightarrow \QProc$. A map, $s$ lifts, uniquely, to a map
on process terms, $\widehat{s} : \Proc \rightarrow \Proc$ by the
following equations.

\begin{mathpar}
  (0) \psubstp{Q}{P} := 0 \\
  (R \juxtap S) \psubstp{Q}{P}
  :=    
  (R)\psubstp{Q}{P} \juxtap (S) \psubstp{Q}{P} \\
  (x?(y).R) \psubstp{Q}{P}    
  :=    
  (x)\substp{Q}{P} (z)\concat( (R \psubstn{z}{y}) \psubstp{Q}{P} ) \\
  (\lift{x}{R}) \psubstp{Q}{P}  
  :=
  \lift{(x)\substp{Q}{P}}{ R \psubstp{Q}{P} } \\
%   (\dropn{x})  \psubstp{Q}{P}       
%   := 
%   \left\{ 
%     \begin{array}{ccc} 
%       \dropn{\quotep{Q}} & & x \nameeq \quotep{P} \\
%       \dropn{x} & & otherwise \\
%     \end{array}
%   \right. 
  (\dropn{x})  \psubstp{Q}{P}       
  := 
  \left\{ 
    \begin{array}{ccc} 
      Q & & x \nameeq \quotep{P} \\
      \dropn{x} & & otherwise \\
    \end{array}
  \right.
\end{mathpar}
 

where

\begin{eqnarray}
  (x)\id{\{} \lpquote Q \rpquote / \lpquote P \rpquote \id{\}}            = 
  \left\{ 
    \begin{array}{ccc}
      \lpquote Q \rpquote & & x \nameeq \lpquote P \rpquote \\
      x & & otherwise \\
    \end{array}
  \right. \nonumber
\end{eqnarray}

and $z$ is chosen distinct from $\quotep{P}$, $\quotep{Q}$, the free
names in $Q$, and all the names in $R$. Our $\alpha$-equivalence will
be built in the standard way from this substitution.

\begin{remark}\label{rem:no_self_referential_names}
  One consequence of these definitions is that $\forall P. \quotep{P}
  \not\in \freenames{P}$.
\end{remark}

\subsection{ Dynamic quote: an example }

Anticipating something of what's to come, consider applying the
substitution, $\widehat{\id{\{}u / z \id{\}}}$, to the following pair
of processes, $\lift{w}{y!(z)}$ and $w[ \lpquote y!(z) \rpquote ]$.

\begin{eqnarray}
	\lift{w}{y!(z)}\widehat{\id{\{}u / z \id{\}}}
		& = &
		\lift{w}{y!(u)} \nonumber\\
	w[ \lpquote y!(z) \rpquote ] \widehat{ \id{\{}u / z \id{\}} }
		& = &
		w[ \lpquote y!(z) \rpquote ] \nonumber
\end{eqnarray}

Because the body of the process between quotes is impervious to
substitution, we get radically different answers. In fact, by
examining the first process in an input context,
e.g. $x?(z).\lift{w}{y!(z)}$, we see that the process under the lift
operator may be shaped by prefixed inputs binding a name inside it. In
this sense, the lift operator will be seen as a way to dynamically
construct processes before reifying them as names.

Finally equipped with these standard features we can present the
dynamics of the calculus.

\subsubsection{Operational semantics} 

Finally, we introduce the computational dynamics. What marks these
algebras as distinct from other more traditionally studied algebraic
structures, e.g. vector spaces or polynomial rings, is the manner in
which dynamics is captured. In traditional structures, dynamics is typically
expressed through morphisms between such structures, as in linear maps
between vector spaces or morphisms between rings. In algebras
associated with the semantics of computation, the dynamics is
expressed as part of the algebraic structure itself, through a
reduction reduction relation typically denoted by $\red$. Below, we
give a recursive presentation of this relation for the calculus used
in the encoding.

$\red \subseteq \pi \times \pi$
$\red : \pi \to \mathcal{P}(\pi)$

\begin{mathpar}
  \inferrule* [lab=Comm] { \textsf{match}( x_{src}, x_{trgt} ) } { x_{trgt}?(y)P \; | \; x_{src}!\langle {Q} \rangle \red P\{\quotep{Q}/y}\} }
  \and \\
  \inferrule* [lab=Par] {{P} \red {P}'} {{{P} | {Q}} \red {{P}' | {Q}}}
  \and
  \inferrule* [lab=Equiv]{{{P} \scong {P}'} \andalso {{P}' \red {Q}'} \andalso {{Q}' \scong {Q}}}{{P} \red {Q}}
\end{mathpar}

\begin{eqnarray*}
  match_{\equiv} (\quotep{P},\quotep{Q}) & := & P \equiv Q \\
  match_{\dagger}(\quotep{P},\quotep{Q}) & := & \forall R. P|Q \red^{*} R => R \red^{*} 0 \\
  match_{K}(\quotep{P},\quotep{Q}) & := & K \mbox{ for some context } K
\end{eqnarray*}

$u?(x)P | u!\langle Q \rangle \red P\{\quotep{Q}/x\}$

%We write $\wred$ for $\red^*$, and $P\red$ if $\exists Q $ such that $ P \red Q$.
We write $P\red$ if $\exists Q $ such that $ P \red Q$ and $P\not\red$, otherwise.

\section{Replication}

As mentioned before, it is known that replication (and hence
recursion) can be implemented in a higher-order process algebra
\cite{SangiorgiWalker}. As our first example of calculation with the
machinery thus far presented we give the construction explicitly in
the {\rhoc}.

\begin{eqnarray}
	D_{x} & := & \prefix{x}{y}{(\binpar{\outputp{x}{y}}{@{y}})} \nonumber\\
	\bangp_{x}{P} & := & \binpar{{x}!\langle{\binpar{D_{x}}{P}}\rangle}{D_{x}} \nonumber
\end{eqnarray}

\begin{eqnarray}
	\bangp_{x}{P} & & \nonumber\\
	=
	& {x}!\langle{(\prefix{x}{y}{(\outputp{x}{y} | @{y})) | P}}\rangle 
	      | \prefix{x}{y}{(\outputp{x}{y} | @{y})} & \nonumber\\
	\red
	& (\outputp{x}{y} | @{y})\substn{\quotep{(\prefix{x}{y}{(@{y} | \outputp{x}{y})) | P}}}{y} & \nonumber\\
	=
	& \outputp{x}{\quotep{(\prefix{x}{y}{(\outputp{x}{y} | @{y})) | P}}}
	  | {(\prefix{x}{y}{(\outputp{x}{y} | @{y})) | P}} & \nonumber\\
	\red
	& \ldots & \nonumber\\
	\red^*
	& P | P | \ldots & \nonumber
\end{eqnarray}

Of course, this encoding, as an implementation, runs away, unfolding
$\bangp{P}$ eagerly. A lazier and more implementable replication
operator, restricted to input-guarded processes, may be obtained as follows.

\begin{eqnarray}
\bangp{\prefix{u}{v}{P}} 
	:= 
	\binpar{\lift{x}{\prefix{u}{v}{(\binpar{D(x)}{P})}}}{D(x)} \nonumber
\end{eqnarray}

\begin{remark}
  Note that the lazier definition still does not deal with summation
  or mixed summation (i.e. sums over input and output). The reader is
  invited to construct definitions of replication that deal with these
  features. 

  Further, the definitions are parameterized in a name, $x$. Can you,
  gentle reader, make a definition that eliminates this parameter and
  guarantees no accidental interaction between the replication
  machinery and the process being replicated -- i.e. no accidental
  sharing of names used by the process to get its work done and the
  name(s) used by the replication to effect copying. This latter
  revision of the definition of replication is crucial to obtaining
  the expected identity $!!P \sim !P$.
\end{remark}

\begin{remark}\label{rem:paradoxical_combinator}
  The reader familiar with the lambda calculus will have noticed the
  similarity between $D$ and the paradoxical combinator.

  [Ed. note: the existence of this seems to suggest we have to be more
  restrictive on the set of processes and names we admit if we are to
  support no-cloning.]
\end{remark}

\subsubsection{Bisimulation}

The computational dynamics gives rise to another kind of equivalence,
the equivalence of computational behavior. As previously mentioned
this is typically captured \emph{via} some form of bisimulation.

% The notion we use in this paper is weak barbed bisimulation
% \cite{milner91polyadicpi}.

The notion we use in this paper is derived from weak barbed
bisimulation \cite{milner91polyadicpi}. 

\begin{definition}
An \emph{observation relation}, $\downarrow_{\mathcal N}$, over a set
of names, $\mathcal N$, is the smallest relation satisfying the rules
below.

\infrule[Out-barb]{y \in {\mathcal N}, \; x \nameeq y}
		  {\outputp{x}{v} \downarrow_{\mathcal N} x}
\infrule[Par-barb]{\mbox{$P\downarrow_{\mathcal N} x$ or $Q\downarrow_{\mathcal N} x$}}
		  {\binpar{P}{Q} \downarrow_{\mathcal N} x}

We write $P \Downarrow_{\mathcal N} x$ if there is $Q$ such that 
$P \wred Q$ and $Q \downarrow_{\mathcal N} x$.
\end{definition}

\begin{definition}
%\label{def.bbisim}
An  ${\mathcal N}$-\emph{barbed bisimulation} over a set of names, ${\mathcal N}$, is a symmetric binary relation 
${\mathcal S}_{\mathcal N}$ between agents such that $P\rel{S}_{\mathcal N}Q$ implies:
\begin{enumerate}
\item If $P \red P'$ then $Q \wred Q'$ and $P'\rel{S}_{\mathcal N} Q'$.
\item If $P\downarrow_{\mathcal N} x$, then $Q\Downarrow_{\mathcal N} x$.
\end{enumerate}
$P$ is ${\mathcal N}$-barbed bisimilar to $Q$, written
$P \wbbisim_{\mathcal N} Q$, if $P \rel{S}_{\mathcal N} Q$ for some ${\mathcal N}$-barbed bisimulation ${\mathcal S}_{\mathcal N}$.
\end{definition}

$\mathcal{R} \subseteq \pi \times \pi$

$P \mathcal{R} Q => \forall P'. P \red P' \Rightarrow \exists Q'. Q \red Q', P' \mathcal{R} Q'$

$P \vdash x \Rightarrow Q \vdash x$

\begin{mathpar}
  \inferrule*[lab=Out-barb]{x \nameeq y}{{y}!\langle{Q}\rangle \vdash x}
  \and
  \inferrule*[lab=Par-barb]{\mbox{$P\vdash x$ or $Q\vdash x$}}{\binpar{P}{Q} \vdash x}
\end{mathpar}

\subsubsection{Contexts}

One of the principle advantages of computational calculi like the
$\pi$-calculus is a well-defined notion of context,
contextual-equivalence and a correlation between
contextual-equivalence and notions of bisimulation. The notion of
context allows the decomposition of a process into (sub-)process and
its syntactic environment, its context. Thus, a context may be
thought of as a process with a ``hole'' (written $\Box$) in it. The
application of a context $M$ to a process $P$, written $M[P]$, is
tantamount to filling the hole in $M$ with $P$. In this paper we do
not need the full weight of this theory, but do make use of the notion
of context in the proof the main theorem. 

\begin{mathpar}
  \inferrule* [lab=summation] {} {{M_{M},M_{N}} \bc \Box \;|\; x.M_{A} \;|\; M_{M}+M_{N}}
  \and
  \inferrule* [lab=agent] {} {{M_{A}} \bc (\vec{x})M_{P} \;| \; \clift{P_0,\ldots,M_{P},\ldots,P_N}}
  \and \\
  \inferrule* [lab=process] {} {{M_{P}} \bc M_{N} \;| \;P|M_{P} }
\end{mathpar} 

\begin{mathpar}
  \inferrule* [lab=sychronization] {} {M_{N} \bc \Box \;|\; x?M_{F} \;|\; x!M_{C}}
  \and
  \inferrule* [lab=abstraction] {} {{M_{F}} \bc (x)M_{P} }
  \and
  \inferrule* [lab=concretion] {} {{M_{C}} \bc \langle M_{P} \rangle }
  \and \\
  \inferrule* [lab=process] {} {{M_{P}} \bc M_{N} \;| \;P|M_{P} }
\end{mathpar}

\begin{definition}[contextual application] Given a context $M$, and
  process $P$, we define the \emph{contextual application}, $M[P] :=
  M\{P/\Box\}$. That is, the contextual application of M to P is the
  substitution of $P$ for $\Box$ in $M$.
\end{definition}

$\meaningof{-} : L \to \mathcal{P}(\pi)$

\begin{mathpar}
  \inferrule* [lab=collection] {} {\meaningof{true} = \pi, \and \meaningof{~E} = \pi \setminus \meaningof{E}, \and \meaningof{E_{1} \& E_{2}} = \meaningof{E_{1}} \cap \meaningof{E_{2}}}
\end{mathpar}

\begin{mathpar}
  \inferrule* [lab=structure] {} {\meaningof{0} = \{ P \in \pi | P \equiv 0 \}, \and \\ \meaningof{E_1 | E_2} = \{ P \in \pi | P \equiv P_{1} | P_{2}, P_{1} \in \meaningof{E_{1}}, P_{2} \in \meaningof{E_2}\} }
\end{mathpar}

\begin{mathpar}
 \inferrule* [lab=behavior] {} {\meaningof{\langle a?b \rangle E} = \{ P \in \pi | P \equiv Q | u?(y)P', \\ \and \\\\ \and \\ \;\;\; u \in \meaningof{a}, \forall z.P'\{z/y\} \in \meaningof{E\{z/b\}}\}, \and \\ \meaningof{a!E} = \{ P \in \pi | P \equiv Q | x!\langle P' \rangle, x \in \meaningof{a} P' \in \meaningof{E}\} }
\end{mathpar}

\begin{mathpar}
 \inferrule* [lab=nominal] {} {\meaningof{\quotep{E}} = \{ \quotep{P} \in \quotep{\pi} | P \in \meaningof{E} \}, \and \meaningof{\quotep{P}} = \{ \quotep{Q} \in \quotep{\pi} | P \equiv Q \} \and \\ \meaningof{@\quotep{E}} = \{ P \in \pi | P \equiv @x, x \in \meaningof{E} \}}
\end{mathpar}

\begin{eqnarray*}
  \\
  \meaningof{-} : TS \to ST
\end{eqnarray*}

\begin{eqnarray*}
  \\
  L : TS \to ST
\end{eqnarray*}

\begin{eqnarray*}
  \\
  P \models E \iff P \in \meaningof{E}
\end{eqnarray*}

\begin{eqnarray*}
  P \approx_{L} Q \iff \forall E \in L. P \models E \iff Q \models E
\end{eqnarray*}

\begin{eqnarray*}
  P \approx_{K} Q
\end{eqnarray*}

\begin{eqnarray*}
  P \approx Q
\end{eqnarray*}

$\approx_{K} = \approx = \approx_{L}$

\subsubsection{Contextual duality}

Note that contexts extend the quotation operation to a family of
operations from processes to names. Given a context, $M$, we can
define a \emph{nominal context}, $\quotep{M}$ by $\quotep{M}[P] :=
\quotep{M[P]}$. To foreshadow what is to come we observe that these
operations enjoy a duality with processes very much like the duality
between vectors and maps from vectors to scalars.

Further, because the calculus is essentially higher-order, we have a
correspondence between contexts and processes. More specifically,
given a name $x$ and a context $M$ we can construct $M^{*}_{x}$ such
that 

\begin{mathpar}
  M^{*}_{x} | \lift{x}{P} \red M[P]
\end{mathpar}

namely,

\begin{mathpar}
  M^{*}_{x} := x?(u).M[\dropn{u}]
\end{mathpar}

The dependence of $M^{*}_{x}$ on a name makes it an abstraction, 

\begin{mathpar}
  M^{*} := (x)x?(u).M[\dropn{u}]
\end{mathpar}

\subsection{Additional notation}

It will sometimes be convenient to denote the process a name
quotes. We already have the notation $x = \quotep{P}$, but it will be
convenient to introduce an alternate notation, $\procn{x}$, when we
want to emphasize the connection to the use of the name. Note that, by
virtue of name equivalence, $\quotep{\procn{x}} \nameeq x$; so, the
notation is consistent with previous definitions.

Further, because names have structure it is possible to effect
substitutions on the basis of that structure. This means we need to
upgrade our notation for substitutions, which we accomplish by
adapting comprehension notation. Thus,

\begin{mathpar}
  P\{ y / x : x \in S \}
\end{mathpar}

is interpreted to mean the process derived from P by replacing (in a
capture-avoiding manner) each occurrence of $x$ in $S$ by $y$. For example,

\begin{mathpar}
  P\{ \quotep{\procn{x}|\procn{x}} / x : x \in \freenames{P} \}
\end{mathpar}

will replace each (occurrence) of a free name $x$ in $P$ by
$\quotep{\procn{x}|\procn{x}}$.

Also, we will avail ourselves of the notation $x^{L}$ and $x^{R}$ to
denote injections of a name into disjoint copies of the name
space. There are numerous ways to accomplish this. One example can be
found in \cite{MeredithR05}. This notation overloads to vectors of
names: $\vec{x}^{\pi} := (x_{i}^{\pi} \; : \; 0 \leq i < |\vec{x}| )$ where $\pi \in \{L,R\}$.

We also use $P^{\Box} := P|\Box$.

In \cite{MeredithR05} an interpretation of the new operator is
given. It turns out that there are several possible interpretations
all enjoying the requisite algebraic properties of the operator (see
\cite{milner91polyadicpi}). We will therefore make liberal use of
$(\nu\; \vec{x})P$.

% subsection the_syntax_and_semantics_of_the_notation_system (end)   

\input{qm2pi.qmops} 

\input{qm2pi.sterngerlach} 

\input{qm2pi.metric} 

% section concurrent_process_calculi (end)

%\input{qm2pi.proofsketch}

% section proof sketch (end)

%\input{qm2pi.slviaknots} 

% section spatial logic via knots (end)

\input{qm2pi.conclusion}

% section conclusion (end)

%\input{qm2pi.dtcodes} 

% section wiring algorithm (end)

\input{qm2pi.ack} 

% section acknowledgments (end)

\newpage


\bibliographystyle{plain}   
\bibliography{../../biblios/main.bib}

\input{qm2pi.rhodetails}

\end{document}

 

% section concurrent_process_calculi (end)

%\documentclass[12pt]{llncs}
%\documentclass{jktr}

\usepackage[pdftex]{hyperref}                   
\usepackage {listings}
\usepackage {mathpartir}
\usepackage{bcprules}
%\usepackage{listings}
                       
\usepackage{graphicx} 
%\usepackage[margins=2.5cm,nohead,nofoot]{geometry}
%\usepackage{geometry}
\usepackage{amsfonts}
\usepackage{amstext}
\usepackage{latexsym}
\usepackage{amssymb}
\usepackage{color}


%\include{myPreamble}
\include{qm2pi.local} 

%\ifpdf
%\usepackage[pdftex]{graphicx}
%\else
%\usepackage{graphicx}
%\fi

 % \ifpdf
%  \usepackage{pdfsync}
%  \if


%\title{Brief Article}
%\author{David F. Snyder}
%\author{L.G. Meredith}

%\address{Dept. of Math., Texas State University--San Marcos, San Marcos, TX 78666}
       
\pagestyle{empty}


\begin{document}

\lstset{language=[Objective]Caml,frame=shadowbox}

\input{qm2pi.front}

% section front matter (end)

\input{qm2pi.intro} 
 
% section introduction (end)

% \input{qm2pi.knotations} 

% section notation (end)

\input{qm2pi.process.calculi} 

% section concurrent_process_calculi_and_spatial_logics_ (end)
    
%\input{qm2pi.knots2pi} 

%\input{qm2pi.trefoil} 

%\input{qm2pi.mainthm} 

% subsection basic_interpretation (end)

%\input{qm2pi.rho.presentation} 
\subsection{The syntax and semantics of the notation system}\label{sub:the_syntax_and_semantics_of_the_notation_system} % (fold)

We now summarize a technical presentation of the calculus that
embodies our theory of dynamics. The typical presentation of such a
calculus follows the style of giving generators and relations on
them. The grammar, below, describing term constructors, freely
generates the set of processes, $\Proc$. This set is then quotiented
by a relation known as structural congruence and it is over this set
that the notion of dynamics is expressed. This presentation is
essentially that of \cite{MeredithR05} with the addition of
polyadicity and summation. For readability we have relegated some of
the technical subtleties to an appendix.

\subsubsection{Process grammar}\label{subsub:process_grammar}

\begin{mathpar}
  \inferrule* [lab=synchronization] {} {{M} \bc \pzero \;|\; x?F \;|\; x!C }
  \and
  \inferrule* [lab=abstraction] {} {{F} \bc (x)P}
  \and
  \inferrule* [lab=concretion] {} {{C} \bc \langle Q \rangle}
  \and
  \inferrule* [lab=process] {} {{P,Q} \bc M \;| \;P|Q \;|\; @{x}}
  \and
  \inferrule* [lab=name] {} {{x} \bc \quotep{P}}
\end{mathpar} 

Note that $\vec{x}$ (resp. $\vec{P}$) denotes a vector of names
(resp. processes) of length $|\vec{x}|$ (resp. $|\vec{P}|$). We adopt
the following useful abbreviations.

\begin{mathpar}
   x?(\vec{y}).P := x.(\vec{y})P \and  x\clift{\vec{P}} := x.\clift{\vec{P}}
   \and x!(y) := \lift{x}{\dropn{y}}
   \and \Pi_{i=0}^{n-1}P_i := P_0 | \ldots | P_{n-1}
\end{mathpar}

\subsubsection{Structural congruence}

\paragraph{Free and bound names and alpha-equivalence.} At the
core of structural equivalence is alpha-equivalence which identifies
process that are the same up to a change of variable. Formally, we
recognize the distinction between free and bound names. The free names
of a process, $\freenames{P}$, may be calculated recursively as
follows:

\begin{mathpar}
\freenames{\pzero} := \emptyset
  \and \\
  \freenames{x?(y).P} := \{ x \} \cup (\freenames{P} \setminus \{ y \})
  \and 
  \freenames{x!\langle P \rangle} := \{ x \} \cup \{ P \} 
  \and \\
  \freenames{P|Q} := \freenames{P} \cup \freenames{Q}
  \and \\
  \freenames{@{x}} := \{ x \}
\end{mathpar}

$\pi$
$\quotep{\pi}$

$\freenames{-} : \pi \to \mathcal{P}(\quotep{\pi})$

\begin{eqnarray*}
  \freenames{\pzero} & := & \emptyset \\
  \freenames{x?(y).P} & := & \{ x \} \cup (\freenames{P} \setminus \{ y \}) \\
  \freenames{x!\langle P \rangle} & := & \{ x \} \cup \{ P \} \\
  \freenames{P|Q} & := & \freenames{P} \cup \freenames{Q} \\
  \freenames{\dropn{x}} & := & \{ x \}
\end{eqnarray*}

The bound names of a process, $\boundnames{P}$, are those names occurring in $P$
that are not free. For example, in $x?(y).0$, the name $x$ is free, while $y$ is bound.

\begin{mathpar}
  \inferrule* [lab=monoidal-laws] {} { P|Q \equiv Q|P \and P|0 \equiv P \and P|(Q|R) \equiv (P|Q)|R }
\end{mathpar}

\begin{mathpar}
  \inferrule* [lab=alpha-equivalence] {} { (x)P \equiv (y)P\{y/x\} \and y \not\in \freenames{P} }
\end{mathpar}

\begin{definition}
Then two processes, $P,Q$, are alpha-equivalent if $P = Q\{\vec{y}/\vec{x}\}$ for
some $\vec{x} \in \boundnames{Q},\vec{y} \in \boundnames{P}$, where $Q\{\vec{y}/\vec{x}\}$
denotes the capture-avoiding substitution of $\vec{y}$ for $\vec{x}$ in $Q$.
\end{definition}

\begin{definition}
  The {\em structural congruence} \cite{SangiorgiWalker} , $\equiv$,
  between processes is the least congruence containing
  alpha-equivalence, satisfying the abelian monoid laws
  (associativity, commutativity and $\pzero$ as identity) for parallel
  composition $|$ and for summation $+$.
\end{definition}

\subsection{Name equivalence}

We take name equivalence, written $\nameeq$, to be the smallest
equivalence relation generated by the following rules.

\begin{mathpar}
\inferrule*[lab=Quote-drop]
{ }
{ \quotep{@{x}} \nameeq x }

\inferrule*[lab=Struct-equiv]
{ P \scong Q }
{ \quotep{P} \nameeq \quotep{Q} }
\end{mathpar}

The astute reader will have noticed that the mutual recursion of names
and processes imposes a mutual recursion on alpha-equivalence and
structural equivalence via name-equivalence. Fortunately, all of this
works out pleasantly and we may calculate in the natural way, free of
concern. The reader interested in the details is referred to the
appendix \ref{appendix:rho_details}.

\subsection{Substitution}

We use $\Proc$ for the set of processes, $\QProc$ for the set of
names, and $\id{\{}\vec{y} / \vec{x} \id{\}}$ to denote partial maps,
$s : \QProc \rightarrow \QProc$. A map, $s$ lifts, uniquely, to a map
on process terms, $\widehat{s} : \Proc \rightarrow \Proc$ by the
following equations.

\begin{mathpar}
  (0) \psubstp{Q}{P} := 0 \\
  (R \juxtap S) \psubstp{Q}{P}
  :=    
  (R)\psubstp{Q}{P} \juxtap (S) \psubstp{Q}{P} \\
  (x?(y).R) \psubstp{Q}{P}    
  :=    
  (x)\substp{Q}{P} (z)\concat( (R \psubstn{z}{y}) \psubstp{Q}{P} ) \\
  (\lift{x}{R}) \psubstp{Q}{P}  
  :=
  \lift{(x)\substp{Q}{P}}{ R \psubstp{Q}{P} } \\
%   (\dropn{x})  \psubstp{Q}{P}       
%   := 
%   \left\{ 
%     \begin{array}{ccc} 
%       \dropn{\quotep{Q}} & & x \nameeq \quotep{P} \\
%       \dropn{x} & & otherwise \\
%     \end{array}
%   \right. 
  (\dropn{x})  \psubstp{Q}{P}       
  := 
  \left\{ 
    \begin{array}{ccc} 
      Q & & x \nameeq \quotep{P} \\
      \dropn{x} & & otherwise \\
    \end{array}
  \right.
\end{mathpar}
 

where

\begin{eqnarray}
  (x)\id{\{} \lpquote Q \rpquote / \lpquote P \rpquote \id{\}}            = 
  \left\{ 
    \begin{array}{ccc}
      \lpquote Q \rpquote & & x \nameeq \lpquote P \rpquote \\
      x & & otherwise \\
    \end{array}
  \right. \nonumber
\end{eqnarray}

and $z$ is chosen distinct from $\quotep{P}$, $\quotep{Q}$, the free
names in $Q$, and all the names in $R$. Our $\alpha$-equivalence will
be built in the standard way from this substitution.

\begin{remark}\label{rem:no_self_referential_names}
  One consequence of these definitions is that $\forall P. \quotep{P}
  \not\in \freenames{P}$.
\end{remark}

\subsection{ Dynamic quote: an example }

Anticipating something of what's to come, consider applying the
substitution, $\widehat{\id{\{}u / z \id{\}}}$, to the following pair
of processes, $\lift{w}{y!(z)}$ and $w[ \lpquote y!(z) \rpquote ]$.

\begin{eqnarray}
	\lift{w}{y!(z)}\widehat{\id{\{}u / z \id{\}}}
		& = &
		\lift{w}{y!(u)} \nonumber\\
	w[ \lpquote y!(z) \rpquote ] \widehat{ \id{\{}u / z \id{\}} }
		& = &
		w[ \lpquote y!(z) \rpquote ] \nonumber
\end{eqnarray}

Because the body of the process between quotes is impervious to
substitution, we get radically different answers. In fact, by
examining the first process in an input context,
e.g. $x?(z).\lift{w}{y!(z)}$, we see that the process under the lift
operator may be shaped by prefixed inputs binding a name inside it. In
this sense, the lift operator will be seen as a way to dynamically
construct processes before reifying them as names.

Finally equipped with these standard features we can present the
dynamics of the calculus.

\subsubsection{Operational semantics} 

Finally, we introduce the computational dynamics. What marks these
algebras as distinct from other more traditionally studied algebraic
structures, e.g. vector spaces or polynomial rings, is the manner in
which dynamics is captured. In traditional structures, dynamics is typically
expressed through morphisms between such structures, as in linear maps
between vector spaces or morphisms between rings. In algebras
associated with the semantics of computation, the dynamics is
expressed as part of the algebraic structure itself, through a
reduction reduction relation typically denoted by $\red$. Below, we
give a recursive presentation of this relation for the calculus used
in the encoding.

$\red \subseteq \pi \times \pi$
$\red : \pi \to \mathcal{P}(\pi)$

\begin{mathpar}
  \inferrule* [lab=Comm] { \textsf{match}( x_{src}, x_{trgt} ) } { x_{trgt}?(y)P \; | \; x_{src}!\langle {Q} \rangle \red P\{\quotep{Q}/y}\} }
  \and \\
  \inferrule* [lab=Par] {{P} \red {P}'} {{{P} | {Q}} \red {{P}' | {Q}}}
  \and
  \inferrule* [lab=Equiv]{{{P} \scong {P}'} \andalso {{P}' \red {Q}'} \andalso {{Q}' \scong {Q}}}{{P} \red {Q}}
\end{mathpar}

\begin{eqnarray*}
  match_{\equiv} (\quotep{P},\quotep{Q}) & := & P \equiv Q \\
  match_{\dagger}(\quotep{P},\quotep{Q}) & := & \forall R. P|Q \red^{*} R => R \red^{*} 0 \\
  match_{K}(\quotep{P},\quotep{Q}) & := & K \mbox{ for some context } K
\end{eqnarray*}

$u?(x)P | u!\langle Q \rangle \red P\{\quotep{Q}/x\}$

%We write $\wred$ for $\red^*$, and $P\red$ if $\exists Q $ such that $ P \red Q$.
We write $P\red$ if $\exists Q $ such that $ P \red Q$ and $P\not\red$, otherwise.

\section{Replication}

As mentioned before, it is known that replication (and hence
recursion) can be implemented in a higher-order process algebra
\cite{SangiorgiWalker}. As our first example of calculation with the
machinery thus far presented we give the construction explicitly in
the {\rhoc}.

\begin{eqnarray}
	D_{x} & := & \prefix{x}{y}{(\binpar{\outputp{x}{y}}{@{y}})} \nonumber\\
	\bangp_{x}{P} & := & \binpar{{x}!\langle{\binpar{D_{x}}{P}}\rangle}{D_{x}} \nonumber
\end{eqnarray}

\begin{eqnarray}
	\bangp_{x}{P} & & \nonumber\\
	=
	& {x}!\langle{(\prefix{x}{y}{(\outputp{x}{y} | @{y})) | P}}\rangle 
	      | \prefix{x}{y}{(\outputp{x}{y} | @{y})} & \nonumber\\
	\red
	& (\outputp{x}{y} | @{y})\substn{\quotep{(\prefix{x}{y}{(@{y} | \outputp{x}{y})) | P}}}{y} & \nonumber\\
	=
	& \outputp{x}{\quotep{(\prefix{x}{y}{(\outputp{x}{y} | @{y})) | P}}}
	  | {(\prefix{x}{y}{(\outputp{x}{y} | @{y})) | P}} & \nonumber\\
	\red
	& \ldots & \nonumber\\
	\red^*
	& P | P | \ldots & \nonumber
\end{eqnarray}

Of course, this encoding, as an implementation, runs away, unfolding
$\bangp{P}$ eagerly. A lazier and more implementable replication
operator, restricted to input-guarded processes, may be obtained as follows.

\begin{eqnarray}
\bangp{\prefix{u}{v}{P}} 
	:= 
	\binpar{\lift{x}{\prefix{u}{v}{(\binpar{D(x)}{P})}}}{D(x)} \nonumber
\end{eqnarray}

\begin{remark}
  Note that the lazier definition still does not deal with summation
  or mixed summation (i.e. sums over input and output). The reader is
  invited to construct definitions of replication that deal with these
  features. 

  Further, the definitions are parameterized in a name, $x$. Can you,
  gentle reader, make a definition that eliminates this parameter and
  guarantees no accidental interaction between the replication
  machinery and the process being replicated -- i.e. no accidental
  sharing of names used by the process to get its work done and the
  name(s) used by the replication to effect copying. This latter
  revision of the definition of replication is crucial to obtaining
  the expected identity $!!P \sim !P$.
\end{remark}

\begin{remark}\label{rem:paradoxical_combinator}
  The reader familiar with the lambda calculus will have noticed the
  similarity between $D$ and the paradoxical combinator.

  [Ed. note: the existence of this seems to suggest we have to be more
  restrictive on the set of processes and names we admit if we are to
  support no-cloning.]
\end{remark}

\subsubsection{Bisimulation}

The computational dynamics gives rise to another kind of equivalence,
the equivalence of computational behavior. As previously mentioned
this is typically captured \emph{via} some form of bisimulation.

% The notion we use in this paper is weak barbed bisimulation
% \cite{milner91polyadicpi}.

The notion we use in this paper is derived from weak barbed
bisimulation \cite{milner91polyadicpi}. 

\begin{definition}
An \emph{observation relation}, $\downarrow_{\mathcal N}$, over a set
of names, $\mathcal N$, is the smallest relation satisfying the rules
below.

\infrule[Out-barb]{y \in {\mathcal N}, \; x \nameeq y}
		  {\outputp{x}{v} \downarrow_{\mathcal N} x}
\infrule[Par-barb]{\mbox{$P\downarrow_{\mathcal N} x$ or $Q\downarrow_{\mathcal N} x$}}
		  {\binpar{P}{Q} \downarrow_{\mathcal N} x}

We write $P \Downarrow_{\mathcal N} x$ if there is $Q$ such that 
$P \wred Q$ and $Q \downarrow_{\mathcal N} x$.
\end{definition}

\begin{definition}
%\label{def.bbisim}
An  ${\mathcal N}$-\emph{barbed bisimulation} over a set of names, ${\mathcal N}$, is a symmetric binary relation 
${\mathcal S}_{\mathcal N}$ between agents such that $P\rel{S}_{\mathcal N}Q$ implies:
\begin{enumerate}
\item If $P \red P'$ then $Q \wred Q'$ and $P'\rel{S}_{\mathcal N} Q'$.
\item If $P\downarrow_{\mathcal N} x$, then $Q\Downarrow_{\mathcal N} x$.
\end{enumerate}
$P$ is ${\mathcal N}$-barbed bisimilar to $Q$, written
$P \wbbisim_{\mathcal N} Q$, if $P \rel{S}_{\mathcal N} Q$ for some ${\mathcal N}$-barbed bisimulation ${\mathcal S}_{\mathcal N}$.
\end{definition}

$\mathcal{R} \subseteq \pi \times \pi$

$P \mathcal{R} Q => \forall P'. P \red P' \Rightarrow \exists Q'. Q \red Q', P' \mathcal{R} Q'$

$P \vdash x \Rightarrow Q \vdash x$

\begin{mathpar}
  \inferrule*[lab=Out-barb]{x \nameeq y}{{y}!\langle{Q}\rangle \vdash x}
  \and
  \inferrule*[lab=Par-barb]{\mbox{$P\vdash x$ or $Q\vdash x$}}{\binpar{P}{Q} \vdash x}
\end{mathpar}

\subsubsection{Contexts}

One of the principle advantages of computational calculi like the
$\pi$-calculus is a well-defined notion of context,
contextual-equivalence and a correlation between
contextual-equivalence and notions of bisimulation. The notion of
context allows the decomposition of a process into (sub-)process and
its syntactic environment, its context. Thus, a context may be
thought of as a process with a ``hole'' (written $\Box$) in it. The
application of a context $M$ to a process $P$, written $M[P]$, is
tantamount to filling the hole in $M$ with $P$. In this paper we do
not need the full weight of this theory, but do make use of the notion
of context in the proof the main theorem. 

\begin{mathpar}
  \inferrule* [lab=summation] {} {{M_{M},M_{N}} \bc \Box \;|\; x.M_{A} \;|\; M_{M}+M_{N}}
  \and
  \inferrule* [lab=agent] {} {{M_{A}} \bc (\vec{x})M_{P} \;| \; \clift{P_0,\ldots,M_{P},\ldots,P_N}}
  \and \\
  \inferrule* [lab=process] {} {{M_{P}} \bc M_{N} \;| \;P|M_{P} }
\end{mathpar} 

\begin{mathpar}
  \inferrule* [lab=sychronization] {} {M_{N} \bc \Box \;|\; x?M_{F} \;|\; x!M_{C}}
  \and
  \inferrule* [lab=abstraction] {} {{M_{F}} \bc (x)M_{P} }
  \and
  \inferrule* [lab=concretion] {} {{M_{C}} \bc \langle M_{P} \rangle }
  \and \\
  \inferrule* [lab=process] {} {{M_{P}} \bc M_{N} \;| \;P|M_{P} }
\end{mathpar}

\begin{definition}[contextual application] Given a context $M$, and
  process $P$, we define the \emph{contextual application}, $M[P] :=
  M\{P/\Box\}$. That is, the contextual application of M to P is the
  substitution of $P$ for $\Box$ in $M$.
\end{definition}

$\meaningof{-} : L \to \mathcal{P}(\pi)$

\begin{mathpar}
  \inferrule* [lab=collection] {} {\meaningof{true} = \pi, \and \meaningof{~E} = \pi \setminus \meaningof{E}, \and \meaningof{E_{1} \& E_{2}} = \meaningof{E_{1}} \cap \meaningof{E_{2}}}
\end{mathpar}

\begin{mathpar}
  \inferrule* [lab=structure] {} {\meaningof{0} = \{ P \in \pi | P \equiv 0 \}, \and \\ \meaningof{E_1 | E_2} = \{ P \in \pi | P \equiv P_{1} | P_{2}, P_{1} \in \meaningof{E_{1}}, P_{2} \in \meaningof{E_2}\} }
\end{mathpar}

\begin{mathpar}
 \inferrule* [lab=behavior] {} {\meaningof{\langle a?b \rangle E} = \{ P \in \pi | P \equiv Q | u?(y)P', \\ \and \\\\ \and \\ \;\;\; u \in \meaningof{a}, \forall z.P'\{z/y\} \in \meaningof{E\{z/b\}}\}, \and \\ \meaningof{a!E} = \{ P \in \pi | P \equiv Q | x!\langle P' \rangle, x \in \meaningof{a} P' \in \meaningof{E}\} }
\end{mathpar}

\begin{mathpar}
 \inferrule* [lab=nominal] {} {\meaningof{\quotep{E}} = \{ \quotep{P} \in \quotep{\pi} | P \in \meaningof{E} \}, \and \meaningof{\quotep{P}} = \{ \quotep{Q} \in \quotep{\pi} | P \equiv Q \} \and \\ \meaningof{@\quotep{E}} = \{ P \in \pi | P \equiv @x, x \in \meaningof{E} \}}
\end{mathpar}

\begin{eqnarray*}
  \\
  \meaningof{-} : TS \to ST
\end{eqnarray*}

\begin{eqnarray*}
  \\
  L : TS \to ST
\end{eqnarray*}

\begin{eqnarray*}
  \\
  P \models E \iff P \in \meaningof{E}
\end{eqnarray*}

\begin{eqnarray*}
  P \approx_{L} Q \iff \forall E \in L. P \models E \iff Q \models E
\end{eqnarray*}

\begin{eqnarray*}
  P \approx_{K} Q
\end{eqnarray*}

\begin{eqnarray*}
  P \approx Q
\end{eqnarray*}

$\approx_{K} = \approx = \approx_{L}$

\subsubsection{Contextual duality}

Note that contexts extend the quotation operation to a family of
operations from processes to names. Given a context, $M$, we can
define a \emph{nominal context}, $\quotep{M}$ by $\quotep{M}[P] :=
\quotep{M[P]}$. To foreshadow what is to come we observe that these
operations enjoy a duality with processes very much like the duality
between vectors and maps from vectors to scalars.

Further, because the calculus is essentially higher-order, we have a
correspondence between contexts and processes. More specifically,
given a name $x$ and a context $M$ we can construct $M^{*}_{x}$ such
that 

\begin{mathpar}
  M^{*}_{x} | \lift{x}{P} \red M[P]
\end{mathpar}

namely,

\begin{mathpar}
  M^{*}_{x} := x?(u).M[\dropn{u}]
\end{mathpar}

The dependence of $M^{*}_{x}$ on a name makes it an abstraction, 

\begin{mathpar}
  M^{*} := (x)x?(u).M[\dropn{u}]
\end{mathpar}

\subsection{Additional notation}

It will sometimes be convenient to denote the process a name
quotes. We already have the notation $x = \quotep{P}$, but it will be
convenient to introduce an alternate notation, $\procn{x}$, when we
want to emphasize the connection to the use of the name. Note that, by
virtue of name equivalence, $\quotep{\procn{x}} \nameeq x$; so, the
notation is consistent with previous definitions.

Further, because names have structure it is possible to effect
substitutions on the basis of that structure. This means we need to
upgrade our notation for substitutions, which we accomplish by
adapting comprehension notation. Thus,

\begin{mathpar}
  P\{ y / x : x \in S \}
\end{mathpar}

is interpreted to mean the process derived from P by replacing (in a
capture-avoiding manner) each occurrence of $x$ in $S$ by $y$. For example,

\begin{mathpar}
  P\{ \quotep{\procn{x}|\procn{x}} / x : x \in \freenames{P} \}
\end{mathpar}

will replace each (occurrence) of a free name $x$ in $P$ by
$\quotep{\procn{x}|\procn{x}}$.

Also, we will avail ourselves of the notation $x^{L}$ and $x^{R}$ to
denote injections of a name into disjoint copies of the name
space. There are numerous ways to accomplish this. One example can be
found in \cite{MeredithR05}. This notation overloads to vectors of
names: $\vec{x}^{\pi} := (x_{i}^{\pi} \; : \; 0 \leq i < |\vec{x}| )$ where $\pi \in \{L,R\}$.

We also use $P^{\Box} := P|\Box$.

In \cite{MeredithR05} an interpretation of the new operator is
given. It turns out that there are several possible interpretations
all enjoying the requisite algebraic properties of the operator (see
\cite{milner91polyadicpi}). We will therefore make liberal use of
$(\nu\; \vec{x})P$.

% subsection the_syntax_and_semantics_of_the_notation_system (end)   

\input{qm2pi.qmops} 

\input{qm2pi.sterngerlach} 

\input{qm2pi.metric} 

% section concurrent_process_calculi (end)

%\input{qm2pi.proofsketch}

% section proof sketch (end)

%\input{qm2pi.slviaknots} 

% section spatial logic via knots (end)

\input{qm2pi.conclusion}

% section conclusion (end)

%\input{qm2pi.dtcodes} 

% section wiring algorithm (end)

\input{qm2pi.ack} 

% section acknowledgments (end)

\newpage


\bibliographystyle{plain}   
\bibliography{../../biblios/main.bib}

\input{qm2pi.rhodetails}

\end{document}



% section proof sketch (end)

%\section{Unlikely characters: spatial logic for
  knots}\label{sub:characteristic_formulae} % (fold)

Associated to the mobile process calculi are a family of logics known
as the Hennessy-Milner logics. These logics typically enjoy a
semantics interpreting formulae as sets of processes that when
factored through the encoding outlined above allows an identification
of classes of knots with logical formulae. In the context of this
encoding the sub-family known as the spatial logics \cite{CairesC03}
\cite{CairesC04} \cite{Caires04} are of particular interest providing
several important features for expressing and reasoning about
properties (i.e. classes) of knots. We hint here at how this may be done.

%\begin{description}
%\item [structural connectives] 
\subsubsection{Structural connectives} The spatial logics enjoy
structural connectives corresponding, at the logical level, to the
parallel composition ($P | Q$) and new name ($(\nu \; x)P$)
connectives for processes. As illustrated in the examples below, these
connectives are extremely expressive given the shape of our encoding.
%\item [decideable satisfaction]

\subsubsection{Decideable satisfaction}
In \cite{Caires04} the satisfaction relation is shown to be decideable
for a rich class of processes. It further turns out that the image of
the our encoding is a proper subset of that class. This result
provides the basis for an algorithm by which to search for knots
enjoying a given property.
%\item [characteristic formulae]

\subsubsection{Characteristic formulae}
In the same paper \cite{Caires04} , Caires presents a means of calculating
characteristic formulae, selecting equivalence classes of processes
up to a pre--specified depth limit on the support set of names. Composed with our
encoding, this characteristic formula can be used to select
characteristic formulae for knots.
%\end{description}

\subsubsection{Spatial logic formulae}

The grammar below (segmented for comprehension) summarizes the syntax
of spatial logic formulae. We employ illustrative examples in the
sequel to provide an intuitive understanding of their meaning
referring the reader to \cite{Caires04} for a more detailed explication
of the semantics.

\begin{mathpar}
  \inferrule* [lab=boolean] {} {{A,B} \bc T \;|\; \neg A \;|\; A \wedge B \;|\; \eta = \eta'}
  \and
  \inferrule* [lab=spatial] {} {|\; \pzero \;|\; A | B \;|\; x \text{\textregistered} A \;|\; \forall x . A \;|\;  H x . A}
  \and
  \inferrule* [lab=behavioral] {} {|\; \alpha . A}
  \and 
  \inferrule* [lab=recursion] {} {|\; X(\vec{u}) \;|\; \mu X(\vec{u}) . A}
  \and
  \inferrule* [lab=action] {} {\alpha \bc \langle x?(\vec{y}) \rangle \;|\; \langle x!(\vec{y}) \rangle \;|\; \langle \tau \rangle}
  \and 
  \inferrule* [lab=name] {} {\eta \bc x \;|\; \tau}
\end{mathpar} 

% subsection characteristic_formulae (end)   	 

\subsection{Example formulae}\label{sub:example_formulae_} % (fold)

\subsubsection{Crossing as formula.}
% 
% \begin{align*}
%   \frac{d}{dx} \sin x &= \cos x 
%   & \frac{d}{dx} e^x &= e^x \\
%   \frac{d}{dx} \cos x &= - \sin x 
%   & \frac{d}{dx} \log x &= \frac{1}{x} \\
% \end{align*} 

\begin{align*}
 \mu C(x_{0},x_{1},y_{0},y_{1},u).&(\langle x_{0}?(z) \rangle(\langle u! \rangle\langle y_{1}!z \rangle C(x_{0},x_{1},y_{0},y_{1},u)) & \\
  & \wedge \langle y_{1}?(z) \rangle (\langle u! \rangle \langle x_{0}!z \rangle C(x_{0},x_{1},y_{0},y_{1},u)) & \\
  & \wedge \langle x_{1}?(z) \rangle (\langle u? \rangle \langle y_{0}!z \rangle C(x_{0},x_{1},y_{0},y_{1},u)) & \\
  & \wedge \langle y_{0}?(z) \rangle (\langle u? \rangle \langle x_{1}!z \rangle C(x_{0},x_{1},y_{0},y_{1},u))) &
\end{align*}

The lexicographical similarity between the shape of this formulae and
the shape of definition of the process representing a crossing reveals
the intuitive meaning of this formulae. It describes the capabilities
of a process that has the right to represent a crossing. For example
it picks out processes that may perform an input on the port $x_0$ in
its initial menu of capabilities. What differentiates the formula
from the process, however, is that the crossing process is the
smallest candidate to satisfy the formula. Infinitely many other
processes -- with internal behavior hidden behind this interface, so
to speak -- also satisfy this formula. Even this simple formula,
then, can be seen to open a new view onto knots, providing a
computational interpretation of \emph{virtual} knots.

Note that this formula is derived by hand. A similar formula can be
derived by employing Caires' calculation of characteristic formula
\cite{Caires04} to the process representing a crossing. In light of
this discussion, we let
$\meaningof{C}_{\phi}(x0,x1,y0,y1,u)$ denote a formula specifying the
dynamics we wish to capture of a crossing. To guarantee we preserve
the shape of the interface and minimal semantics we demand that
$\meaningof{C}_{\phi}(x0,x1,y0,y1,u) \Rightarrow
\textbf{C}(x0,x1,y0,y1,u)$ where $\textbf{C}(x0,x1,y0,y1,u)$ denotes
the formula above.
                            
\subsubsection{Crossing number constraints.}
The moral content of the context lemma (Lemma \ref{context}) is that the notion of
``locality'' in the Reidemeister moves is effectively captured by the
parallel composition operator of the process calculus. This intuition
extends through the logic. Given a formula,
$\meaningof{C}_{\phi}(x0,x1,y0,y1,u)$, we can use the structural
connectives to specify constraints on crossing numbers, such as at
least $n$ crossings, or exactly $n$ crossings.
\begin{mathpar}
  \inferrule* [lab=at-least-n] {} { K^{\geq n}_{\phi}(\vec{xs},\vec{ys}) := \Pi_{i=0}^{n-1} Hu . \meaningof{C}_{\phi}(xs_i,ys_i,u) | T }
  \and 
  \inferrule* [lab=exactly-n] {} { K^{= n}_{\phi}(\vec{xs},\vec{ys}) := \Pi_{i=0}^{n-1} Hu . \meaningof{C}_{\phi}(xs_i,ys_i,u) | \neg (\forall x_0,y_0,x_1,y_1,u . \meaningof{C}_{\phi}(x_0,y_0,x_1,y_1,u) | T) }
\end{mathpar}

To round out this section, recall that the encoding of an $n$-crossing
knot decomposes into a parallel composition of $n$ \emph{copies} of a
crossing process together with a wiring harness. To specify different
knot classes with the same crossing number amounts to specifying
logical constraints on the wiring harness. In the interest of space,
we defer examples to a forthcoming paper. Suffice it to say that both
the conditions ``alternating knot'' and ``contains the tangle
corresponding to 5/3'' are expressible. For example, it is possible to
calculate the characteristic formula of a process corresponding to the
tangle 5/3 and conjoin it into the classifying formula via the
composition connective of the logic.

Finally, we wish to observe that it is entirely within reason to
contemplate a more domain-specific version of spatial logic tailored
to the shape of processes in the image of the encoding. Such a
domain-specific logic would have a better claim to the title formal
language of knot properties.

% subsection example_formulae_ (end)

% section knots_as_processes (end) 

% section spatial logic via knots (end)

\section{Conclusions and future work}

\paragraph{Testing physical space}
You, gentle reader, may wonder why of all the theorems to be proved
given this set up we pick the one above. In some sense it's hardly
central to quantum mechanics. We see it as central in the sense that
it firmly establishes a notion of physical space arising from a notion
of the equivalence of behavior. Relating bisimulation to a metric is a
big step forward, but one is faced with interpreting the relationship
of that metric space to something more physical. Quantum mechanical
notions of ``physical'' space are still far from intuitive, but by
relating this idea of distance as testing to calculations that predict
physical circumstances we are making a not insignificant step forward
toward an understanding of the physical space we inhabit as
essentially dynamic.

\paragraph{Effectivity and simulation}
One of the observations we have yet to make is that the entire program
spelled out here is effective. We have built various interpreters for
the reflective calculus at work in this interpretation. In principle,
then, we can simulate quantum mechanics on a computer. The place where
the simulation may lose fidelity is the infinitely branching summation
for the annihilator.

In this connection i also want to point out that the evaluation style
calculation of the inner product puts the non-determinism of the
summation right at the heart of measurement. This suggests that
Milner's original reduction-based formulation of the dynamics of his
calculi in terms of sums was not just notationally suggestive of a
notion of measure-and-continue but captured some significant part of
the physics.

\paragraph{Quantum continuations}
In light of this last observation i want to point out that the
predominant account of quantum mechanics is missing a key aspect of a
truly compositional story of the physical situation. In a real lab,
when a measurement is made the observation can be made to feed into
another device that then makes another measurement conditioned on the
results of the first. This means that after the superposition was
collapsed the entire experimental set up remained in
superposition. While QM offers a means of writing this down it doesn't
quite line up well with the well-trodden formulation of computation
and continuation that we see so succinctly expressed in Milner's
calculi. This suggests that there might be advantages to this account
of dynamics waiting to be explored.

\paragraph{Quantum logic}
In this connection, we also note that by virtue of having the
Hennessy-Milner construction, we can pull the construction through the
interpretation of QM. This gives us a natural candidate for a quantum
logic that enjoys an extremely tight connection with it's domain of
interpretation, making the construction much less ad hoc (rather it is
the image of functor!).

\paragraph{Quantum probabiity}
i have questions about the basis of the interpretation of inner
product as probability amplitude. In particular, using which
axiomatization of probability theory does the notion of probability
amplitude earn the right to be so dubbed? In other words, where is the
proof that the operation for calculating a probability amplitude (and
then squaring) satisfies the axioms of what it means to calculate a
probability? Even if such a proof exists (i have yet to find it in the
literature), i wonder if it might not be possible to turn things on
their heads. Can we view the calculation of the probability amplitude
as an axiomatization of probability? If so, then the definition we
give for calculating probability amplitude may provide the basis for
an \emph{effective} theory of probability.

\paragraph{Quantum vs ``biological'' information}
Finally, i want to conclude with a more philosophical observation. At
a recent workshop in which QM was a predominant topic i noticed
something about quantum information. The speaker was giving a riveting
discussion of axiomatic QM and showing how properties of ``no
cloning'' and ``no deleting'' emerged as consequences of the
axiomatization. Theorems of this form are necessary to give us a sense
of confidence that our axioms characterize the physical theory. What
struck me, though, was that if quantum information is neither erasable
nor replicable it is markedly different from \emph{life}. Two of the
things we know about life is that

\begin{itemize}
  \item it ends;
  \item to gain some measure of persistence, to transcend it's
    finitude it is imminently copyable.
\end{itemize}

Both of these qualities are summarized succinctly in the aphorism: all
flesh is grass. For me these two kinds of ``information'' -- call them
quantum and biological -- are end points on a spectrum of strategies
for persistence. At one end, we have those curious entities that enjoy
uniqueness and permanence; at the other, we have those who in the face
of a certain end and an uncertain present make a go of passing
something on. To me one of the more remarkable aspects of the latter
strategy is that in the presence of noise (and certain features of
copying) we get a kind of dynamism, a chance for improvement against a
given persistent condition.

% subsection other_calculi_other_bisimulations_and_geometry_as_behavior (end)




% section conclusion (end)

%\documentclass[12pt]{llncs}
%\documentclass{jktr}

\usepackage[pdftex]{hyperref}                   
\usepackage {listings}
\usepackage {mathpartir}
\usepackage{bcprules}
%\usepackage{listings}
                       
\usepackage{graphicx} 
%\usepackage[margins=2.5cm,nohead,nofoot]{geometry}
%\usepackage{geometry}
\usepackage{amsfonts}
\usepackage{amstext}
\usepackage{latexsym}
\usepackage{amssymb}
\usepackage{color}


%\include{myPreamble}
\include{qm2pi.local} 

%\ifpdf
%\usepackage[pdftex]{graphicx}
%\else
%\usepackage{graphicx}
%\fi

 % \ifpdf
%  \usepackage{pdfsync}
%  \if


%\title{Brief Article}
%\author{David F. Snyder}
%\author{L.G. Meredith}

%\address{Dept. of Math., Texas State University--San Marcos, San Marcos, TX 78666}
       
\pagestyle{empty}


\begin{document}

\lstset{language=[Objective]Caml,frame=shadowbox}

\input{qm2pi.front}

% section front matter (end)

\input{qm2pi.intro} 
 
% section introduction (end)

% \input{qm2pi.knotations} 

% section notation (end)

\input{qm2pi.process.calculi} 

% section concurrent_process_calculi_and_spatial_logics_ (end)
    
%\input{qm2pi.knots2pi} 

%\input{qm2pi.trefoil} 

%\input{qm2pi.mainthm} 

% subsection basic_interpretation (end)

%\input{qm2pi.rho.presentation} 
\subsection{The syntax and semantics of the notation system}\label{sub:the_syntax_and_semantics_of_the_notation_system} % (fold)

We now summarize a technical presentation of the calculus that
embodies our theory of dynamics. The typical presentation of such a
calculus follows the style of giving generators and relations on
them. The grammar, below, describing term constructors, freely
generates the set of processes, $\Proc$. This set is then quotiented
by a relation known as structural congruence and it is over this set
that the notion of dynamics is expressed. This presentation is
essentially that of \cite{MeredithR05} with the addition of
polyadicity and summation. For readability we have relegated some of
the technical subtleties to an appendix.

\subsubsection{Process grammar}\label{subsub:process_grammar}

\begin{mathpar}
  \inferrule* [lab=synchronization] {} {{M} \bc \pzero \;|\; x?F \;|\; x!C }
  \and
  \inferrule* [lab=abstraction] {} {{F} \bc (x)P}
  \and
  \inferrule* [lab=concretion] {} {{C} \bc \langle Q \rangle}
  \and
  \inferrule* [lab=process] {} {{P,Q} \bc M \;| \;P|Q \;|\; @{x}}
  \and
  \inferrule* [lab=name] {} {{x} \bc \quotep{P}}
\end{mathpar} 

Note that $\vec{x}$ (resp. $\vec{P}$) denotes a vector of names
(resp. processes) of length $|\vec{x}|$ (resp. $|\vec{P}|$). We adopt
the following useful abbreviations.

\begin{mathpar}
   x?(\vec{y}).P := x.(\vec{y})P \and  x\clift{\vec{P}} := x.\clift{\vec{P}}
   \and x!(y) := \lift{x}{\dropn{y}}
   \and \Pi_{i=0}^{n-1}P_i := P_0 | \ldots | P_{n-1}
\end{mathpar}

\subsubsection{Structural congruence}

\paragraph{Free and bound names and alpha-equivalence.} At the
core of structural equivalence is alpha-equivalence which identifies
process that are the same up to a change of variable. Formally, we
recognize the distinction between free and bound names. The free names
of a process, $\freenames{P}$, may be calculated recursively as
follows:

\begin{mathpar}
\freenames{\pzero} := \emptyset
  \and \\
  \freenames{x?(y).P} := \{ x \} \cup (\freenames{P} \setminus \{ y \})
  \and 
  \freenames{x!\langle P \rangle} := \{ x \} \cup \{ P \} 
  \and \\
  \freenames{P|Q} := \freenames{P} \cup \freenames{Q}
  \and \\
  \freenames{@{x}} := \{ x \}
\end{mathpar}

$\pi$
$\quotep{\pi}$

$\freenames{-} : \pi \to \mathcal{P}(\quotep{\pi})$

\begin{eqnarray*}
  \freenames{\pzero} & := & \emptyset \\
  \freenames{x?(y).P} & := & \{ x \} \cup (\freenames{P} \setminus \{ y \}) \\
  \freenames{x!\langle P \rangle} & := & \{ x \} \cup \{ P \} \\
  \freenames{P|Q} & := & \freenames{P} \cup \freenames{Q} \\
  \freenames{\dropn{x}} & := & \{ x \}
\end{eqnarray*}

The bound names of a process, $\boundnames{P}$, are those names occurring in $P$
that are not free. For example, in $x?(y).0$, the name $x$ is free, while $y$ is bound.

\begin{mathpar}
  \inferrule* [lab=monoidal-laws] {} { P|Q \equiv Q|P \and P|0 \equiv P \and P|(Q|R) \equiv (P|Q)|R }
\end{mathpar}

\begin{mathpar}
  \inferrule* [lab=alpha-equivalence] {} { (x)P \equiv (y)P\{y/x\} \and y \not\in \freenames{P} }
\end{mathpar}

\begin{definition}
Then two processes, $P,Q$, are alpha-equivalent if $P = Q\{\vec{y}/\vec{x}\}$ for
some $\vec{x} \in \boundnames{Q},\vec{y} \in \boundnames{P}$, where $Q\{\vec{y}/\vec{x}\}$
denotes the capture-avoiding substitution of $\vec{y}$ for $\vec{x}$ in $Q$.
\end{definition}

\begin{definition}
  The {\em structural congruence} \cite{SangiorgiWalker} , $\equiv$,
  between processes is the least congruence containing
  alpha-equivalence, satisfying the abelian monoid laws
  (associativity, commutativity and $\pzero$ as identity) for parallel
  composition $|$ and for summation $+$.
\end{definition}

\subsection{Name equivalence}

We take name equivalence, written $\nameeq$, to be the smallest
equivalence relation generated by the following rules.

\begin{mathpar}
\inferrule*[lab=Quote-drop]
{ }
{ \quotep{@{x}} \nameeq x }

\inferrule*[lab=Struct-equiv]
{ P \scong Q }
{ \quotep{P} \nameeq \quotep{Q} }
\end{mathpar}

The astute reader will have noticed that the mutual recursion of names
and processes imposes a mutual recursion on alpha-equivalence and
structural equivalence via name-equivalence. Fortunately, all of this
works out pleasantly and we may calculate in the natural way, free of
concern. The reader interested in the details is referred to the
appendix \ref{appendix:rho_details}.

\subsection{Substitution}

We use $\Proc$ for the set of processes, $\QProc$ for the set of
names, and $\id{\{}\vec{y} / \vec{x} \id{\}}$ to denote partial maps,
$s : \QProc \rightarrow \QProc$. A map, $s$ lifts, uniquely, to a map
on process terms, $\widehat{s} : \Proc \rightarrow \Proc$ by the
following equations.

\begin{mathpar}
  (0) \psubstp{Q}{P} := 0 \\
  (R \juxtap S) \psubstp{Q}{P}
  :=    
  (R)\psubstp{Q}{P} \juxtap (S) \psubstp{Q}{P} \\
  (x?(y).R) \psubstp{Q}{P}    
  :=    
  (x)\substp{Q}{P} (z)\concat( (R \psubstn{z}{y}) \psubstp{Q}{P} ) \\
  (\lift{x}{R}) \psubstp{Q}{P}  
  :=
  \lift{(x)\substp{Q}{P}}{ R \psubstp{Q}{P} } \\
%   (\dropn{x})  \psubstp{Q}{P}       
%   := 
%   \left\{ 
%     \begin{array}{ccc} 
%       \dropn{\quotep{Q}} & & x \nameeq \quotep{P} \\
%       \dropn{x} & & otherwise \\
%     \end{array}
%   \right. 
  (\dropn{x})  \psubstp{Q}{P}       
  := 
  \left\{ 
    \begin{array}{ccc} 
      Q & & x \nameeq \quotep{P} \\
      \dropn{x} & & otherwise \\
    \end{array}
  \right.
\end{mathpar}
 

where

\begin{eqnarray}
  (x)\id{\{} \lpquote Q \rpquote / \lpquote P \rpquote \id{\}}            = 
  \left\{ 
    \begin{array}{ccc}
      \lpquote Q \rpquote & & x \nameeq \lpquote P \rpquote \\
      x & & otherwise \\
    \end{array}
  \right. \nonumber
\end{eqnarray}

and $z$ is chosen distinct from $\quotep{P}$, $\quotep{Q}$, the free
names in $Q$, and all the names in $R$. Our $\alpha$-equivalence will
be built in the standard way from this substitution.

\begin{remark}\label{rem:no_self_referential_names}
  One consequence of these definitions is that $\forall P. \quotep{P}
  \not\in \freenames{P}$.
\end{remark}

\subsection{ Dynamic quote: an example }

Anticipating something of what's to come, consider applying the
substitution, $\widehat{\id{\{}u / z \id{\}}}$, to the following pair
of processes, $\lift{w}{y!(z)}$ and $w[ \lpquote y!(z) \rpquote ]$.

\begin{eqnarray}
	\lift{w}{y!(z)}\widehat{\id{\{}u / z \id{\}}}
		& = &
		\lift{w}{y!(u)} \nonumber\\
	w[ \lpquote y!(z) \rpquote ] \widehat{ \id{\{}u / z \id{\}} }
		& = &
		w[ \lpquote y!(z) \rpquote ] \nonumber
\end{eqnarray}

Because the body of the process between quotes is impervious to
substitution, we get radically different answers. In fact, by
examining the first process in an input context,
e.g. $x?(z).\lift{w}{y!(z)}$, we see that the process under the lift
operator may be shaped by prefixed inputs binding a name inside it. In
this sense, the lift operator will be seen as a way to dynamically
construct processes before reifying them as names.

Finally equipped with these standard features we can present the
dynamics of the calculus.

\subsubsection{Operational semantics} 

Finally, we introduce the computational dynamics. What marks these
algebras as distinct from other more traditionally studied algebraic
structures, e.g. vector spaces or polynomial rings, is the manner in
which dynamics is captured. In traditional structures, dynamics is typically
expressed through morphisms between such structures, as in linear maps
between vector spaces or morphisms between rings. In algebras
associated with the semantics of computation, the dynamics is
expressed as part of the algebraic structure itself, through a
reduction reduction relation typically denoted by $\red$. Below, we
give a recursive presentation of this relation for the calculus used
in the encoding.

$\red \subseteq \pi \times \pi$
$\red : \pi \to \mathcal{P}(\pi)$

\begin{mathpar}
  \inferrule* [lab=Comm] { \textsf{match}( x_{src}, x_{trgt} ) } { x_{trgt}?(y)P \; | \; x_{src}!\langle {Q} \rangle \red P\{\quotep{Q}/y}\} }
  \and \\
  \inferrule* [lab=Par] {{P} \red {P}'} {{{P} | {Q}} \red {{P}' | {Q}}}
  \and
  \inferrule* [lab=Equiv]{{{P} \scong {P}'} \andalso {{P}' \red {Q}'} \andalso {{Q}' \scong {Q}}}{{P} \red {Q}}
\end{mathpar}

\begin{eqnarray*}
  match_{\equiv} (\quotep{P},\quotep{Q}) & := & P \equiv Q \\
  match_{\dagger}(\quotep{P},\quotep{Q}) & := & \forall R. P|Q \red^{*} R => R \red^{*} 0 \\
  match_{K}(\quotep{P},\quotep{Q}) & := & K \mbox{ for some context } K
\end{eqnarray*}

$u?(x)P | u!\langle Q \rangle \red P\{\quotep{Q}/x\}$

%We write $\wred$ for $\red^*$, and $P\red$ if $\exists Q $ such that $ P \red Q$.
We write $P\red$ if $\exists Q $ such that $ P \red Q$ and $P\not\red$, otherwise.

\section{Replication}

As mentioned before, it is known that replication (and hence
recursion) can be implemented in a higher-order process algebra
\cite{SangiorgiWalker}. As our first example of calculation with the
machinery thus far presented we give the construction explicitly in
the {\rhoc}.

\begin{eqnarray}
	D_{x} & := & \prefix{x}{y}{(\binpar{\outputp{x}{y}}{@{y}})} \nonumber\\
	\bangp_{x}{P} & := & \binpar{{x}!\langle{\binpar{D_{x}}{P}}\rangle}{D_{x}} \nonumber
\end{eqnarray}

\begin{eqnarray}
	\bangp_{x}{P} & & \nonumber\\
	=
	& {x}!\langle{(\prefix{x}{y}{(\outputp{x}{y} | @{y})) | P}}\rangle 
	      | \prefix{x}{y}{(\outputp{x}{y} | @{y})} & \nonumber\\
	\red
	& (\outputp{x}{y} | @{y})\substn{\quotep{(\prefix{x}{y}{(@{y} | \outputp{x}{y})) | P}}}{y} & \nonumber\\
	=
	& \outputp{x}{\quotep{(\prefix{x}{y}{(\outputp{x}{y} | @{y})) | P}}}
	  | {(\prefix{x}{y}{(\outputp{x}{y} | @{y})) | P}} & \nonumber\\
	\red
	& \ldots & \nonumber\\
	\red^*
	& P | P | \ldots & \nonumber
\end{eqnarray}

Of course, this encoding, as an implementation, runs away, unfolding
$\bangp{P}$ eagerly. A lazier and more implementable replication
operator, restricted to input-guarded processes, may be obtained as follows.

\begin{eqnarray}
\bangp{\prefix{u}{v}{P}} 
	:= 
	\binpar{\lift{x}{\prefix{u}{v}{(\binpar{D(x)}{P})}}}{D(x)} \nonumber
\end{eqnarray}

\begin{remark}
  Note that the lazier definition still does not deal with summation
  or mixed summation (i.e. sums over input and output). The reader is
  invited to construct definitions of replication that deal with these
  features. 

  Further, the definitions are parameterized in a name, $x$. Can you,
  gentle reader, make a definition that eliminates this parameter and
  guarantees no accidental interaction between the replication
  machinery and the process being replicated -- i.e. no accidental
  sharing of names used by the process to get its work done and the
  name(s) used by the replication to effect copying. This latter
  revision of the definition of replication is crucial to obtaining
  the expected identity $!!P \sim !P$.
\end{remark}

\begin{remark}\label{rem:paradoxical_combinator}
  The reader familiar with the lambda calculus will have noticed the
  similarity between $D$ and the paradoxical combinator.

  [Ed. note: the existence of this seems to suggest we have to be more
  restrictive on the set of processes and names we admit if we are to
  support no-cloning.]
\end{remark}

\subsubsection{Bisimulation}

The computational dynamics gives rise to another kind of equivalence,
the equivalence of computational behavior. As previously mentioned
this is typically captured \emph{via} some form of bisimulation.

% The notion we use in this paper is weak barbed bisimulation
% \cite{milner91polyadicpi}.

The notion we use in this paper is derived from weak barbed
bisimulation \cite{milner91polyadicpi}. 

\begin{definition}
An \emph{observation relation}, $\downarrow_{\mathcal N}$, over a set
of names, $\mathcal N$, is the smallest relation satisfying the rules
below.

\infrule[Out-barb]{y \in {\mathcal N}, \; x \nameeq y}
		  {\outputp{x}{v} \downarrow_{\mathcal N} x}
\infrule[Par-barb]{\mbox{$P\downarrow_{\mathcal N} x$ or $Q\downarrow_{\mathcal N} x$}}
		  {\binpar{P}{Q} \downarrow_{\mathcal N} x}

We write $P \Downarrow_{\mathcal N} x$ if there is $Q$ such that 
$P \wred Q$ and $Q \downarrow_{\mathcal N} x$.
\end{definition}

\begin{definition}
%\label{def.bbisim}
An  ${\mathcal N}$-\emph{barbed bisimulation} over a set of names, ${\mathcal N}$, is a symmetric binary relation 
${\mathcal S}_{\mathcal N}$ between agents such that $P\rel{S}_{\mathcal N}Q$ implies:
\begin{enumerate}
\item If $P \red P'$ then $Q \wred Q'$ and $P'\rel{S}_{\mathcal N} Q'$.
\item If $P\downarrow_{\mathcal N} x$, then $Q\Downarrow_{\mathcal N} x$.
\end{enumerate}
$P$ is ${\mathcal N}$-barbed bisimilar to $Q$, written
$P \wbbisim_{\mathcal N} Q$, if $P \rel{S}_{\mathcal N} Q$ for some ${\mathcal N}$-barbed bisimulation ${\mathcal S}_{\mathcal N}$.
\end{definition}

$\mathcal{R} \subseteq \pi \times \pi$

$P \mathcal{R} Q => \forall P'. P \red P' \Rightarrow \exists Q'. Q \red Q', P' \mathcal{R} Q'$

$P \vdash x \Rightarrow Q \vdash x$

\begin{mathpar}
  \inferrule*[lab=Out-barb]{x \nameeq y}{{y}!\langle{Q}\rangle \vdash x}
  \and
  \inferrule*[lab=Par-barb]{\mbox{$P\vdash x$ or $Q\vdash x$}}{\binpar{P}{Q} \vdash x}
\end{mathpar}

\subsubsection{Contexts}

One of the principle advantages of computational calculi like the
$\pi$-calculus is a well-defined notion of context,
contextual-equivalence and a correlation between
contextual-equivalence and notions of bisimulation. The notion of
context allows the decomposition of a process into (sub-)process and
its syntactic environment, its context. Thus, a context may be
thought of as a process with a ``hole'' (written $\Box$) in it. The
application of a context $M$ to a process $P$, written $M[P]$, is
tantamount to filling the hole in $M$ with $P$. In this paper we do
not need the full weight of this theory, but do make use of the notion
of context in the proof the main theorem. 

\begin{mathpar}
  \inferrule* [lab=summation] {} {{M_{M},M_{N}} \bc \Box \;|\; x.M_{A} \;|\; M_{M}+M_{N}}
  \and
  \inferrule* [lab=agent] {} {{M_{A}} \bc (\vec{x})M_{P} \;| \; \clift{P_0,\ldots,M_{P},\ldots,P_N}}
  \and \\
  \inferrule* [lab=process] {} {{M_{P}} \bc M_{N} \;| \;P|M_{P} }
\end{mathpar} 

\begin{mathpar}
  \inferrule* [lab=sychronization] {} {M_{N} \bc \Box \;|\; x?M_{F} \;|\; x!M_{C}}
  \and
  \inferrule* [lab=abstraction] {} {{M_{F}} \bc (x)M_{P} }
  \and
  \inferrule* [lab=concretion] {} {{M_{C}} \bc \langle M_{P} \rangle }
  \and \\
  \inferrule* [lab=process] {} {{M_{P}} \bc M_{N} \;| \;P|M_{P} }
\end{mathpar}

\begin{definition}[contextual application] Given a context $M$, and
  process $P$, we define the \emph{contextual application}, $M[P] :=
  M\{P/\Box\}$. That is, the contextual application of M to P is the
  substitution of $P$ for $\Box$ in $M$.
\end{definition}

$\meaningof{-} : L \to \mathcal{P}(\pi)$

\begin{mathpar}
  \inferrule* [lab=collection] {} {\meaningof{true} = \pi, \and \meaningof{~E} = \pi \setminus \meaningof{E}, \and \meaningof{E_{1} \& E_{2}} = \meaningof{E_{1}} \cap \meaningof{E_{2}}}
\end{mathpar}

\begin{mathpar}
  \inferrule* [lab=structure] {} {\meaningof{0} = \{ P \in \pi | P \equiv 0 \}, \and \\ \meaningof{E_1 | E_2} = \{ P \in \pi | P \equiv P_{1} | P_{2}, P_{1} \in \meaningof{E_{1}}, P_{2} \in \meaningof{E_2}\} }
\end{mathpar}

\begin{mathpar}
 \inferrule* [lab=behavior] {} {\meaningof{\langle a?b \rangle E} = \{ P \in \pi | P \equiv Q | u?(y)P', \\ \and \\\\ \and \\ \;\;\; u \in \meaningof{a}, \forall z.P'\{z/y\} \in \meaningof{E\{z/b\}}\}, \and \\ \meaningof{a!E} = \{ P \in \pi | P \equiv Q | x!\langle P' \rangle, x \in \meaningof{a} P' \in \meaningof{E}\} }
\end{mathpar}

\begin{mathpar}
 \inferrule* [lab=nominal] {} {\meaningof{\quotep{E}} = \{ \quotep{P} \in \quotep{\pi} | P \in \meaningof{E} \}, \and \meaningof{\quotep{P}} = \{ \quotep{Q} \in \quotep{\pi} | P \equiv Q \} \and \\ \meaningof{@\quotep{E}} = \{ P \in \pi | P \equiv @x, x \in \meaningof{E} \}}
\end{mathpar}

\begin{eqnarray*}
  \\
  \meaningof{-} : TS \to ST
\end{eqnarray*}

\begin{eqnarray*}
  \\
  L : TS \to ST
\end{eqnarray*}

\begin{eqnarray*}
  \\
  P \models E \iff P \in \meaningof{E}
\end{eqnarray*}

\begin{eqnarray*}
  P \approx_{L} Q \iff \forall E \in L. P \models E \iff Q \models E
\end{eqnarray*}

\begin{eqnarray*}
  P \approx_{K} Q
\end{eqnarray*}

\begin{eqnarray*}
  P \approx Q
\end{eqnarray*}

$\approx_{K} = \approx = \approx_{L}$

\subsubsection{Contextual duality}

Note that contexts extend the quotation operation to a family of
operations from processes to names. Given a context, $M$, we can
define a \emph{nominal context}, $\quotep{M}$ by $\quotep{M}[P] :=
\quotep{M[P]}$. To foreshadow what is to come we observe that these
operations enjoy a duality with processes very much like the duality
between vectors and maps from vectors to scalars.

Further, because the calculus is essentially higher-order, we have a
correspondence between contexts and processes. More specifically,
given a name $x$ and a context $M$ we can construct $M^{*}_{x}$ such
that 

\begin{mathpar}
  M^{*}_{x} | \lift{x}{P} \red M[P]
\end{mathpar}

namely,

\begin{mathpar}
  M^{*}_{x} := x?(u).M[\dropn{u}]
\end{mathpar}

The dependence of $M^{*}_{x}$ on a name makes it an abstraction, 

\begin{mathpar}
  M^{*} := (x)x?(u).M[\dropn{u}]
\end{mathpar}

\subsection{Additional notation}

It will sometimes be convenient to denote the process a name
quotes. We already have the notation $x = \quotep{P}$, but it will be
convenient to introduce an alternate notation, $\procn{x}$, when we
want to emphasize the connection to the use of the name. Note that, by
virtue of name equivalence, $\quotep{\procn{x}} \nameeq x$; so, the
notation is consistent with previous definitions.

Further, because names have structure it is possible to effect
substitutions on the basis of that structure. This means we need to
upgrade our notation for substitutions, which we accomplish by
adapting comprehension notation. Thus,

\begin{mathpar}
  P\{ y / x : x \in S \}
\end{mathpar}

is interpreted to mean the process derived from P by replacing (in a
capture-avoiding manner) each occurrence of $x$ in $S$ by $y$. For example,

\begin{mathpar}
  P\{ \quotep{\procn{x}|\procn{x}} / x : x \in \freenames{P} \}
\end{mathpar}

will replace each (occurrence) of a free name $x$ in $P$ by
$\quotep{\procn{x}|\procn{x}}$.

Also, we will avail ourselves of the notation $x^{L}$ and $x^{R}$ to
denote injections of a name into disjoint copies of the name
space. There are numerous ways to accomplish this. One example can be
found in \cite{MeredithR05}. This notation overloads to vectors of
names: $\vec{x}^{\pi} := (x_{i}^{\pi} \; : \; 0 \leq i < |\vec{x}| )$ where $\pi \in \{L,R\}$.

We also use $P^{\Box} := P|\Box$.

In \cite{MeredithR05} an interpretation of the new operator is
given. It turns out that there are several possible interpretations
all enjoying the requisite algebraic properties of the operator (see
\cite{milner91polyadicpi}). We will therefore make liberal use of
$(\nu\; \vec{x})P$.

% subsection the_syntax_and_semantics_of_the_notation_system (end)   

\input{qm2pi.qmops} 

\input{qm2pi.sterngerlach} 

\input{qm2pi.metric} 

% section concurrent_process_calculi (end)

%\input{qm2pi.proofsketch}

% section proof sketch (end)

%\input{qm2pi.slviaknots} 

% section spatial logic via knots (end)

\input{qm2pi.conclusion}

% section conclusion (end)

%\input{qm2pi.dtcodes} 

% section wiring algorithm (end)

\input{qm2pi.ack} 

% section acknowledgments (end)

\newpage


\bibliographystyle{plain}   
\bibliography{../../biblios/main.bib}

\input{qm2pi.rhodetails}

\end{document}

 

% section wiring algorithm (end)

\documentclass[12pt]{llncs}
%\documentclass{jktr}

\usepackage[pdftex]{hyperref}                   
\usepackage {listings}
\usepackage {mathpartir}
\usepackage{bcprules}
%\usepackage{listings}
                       
\usepackage{graphicx} 
%\usepackage[margins=2.5cm,nohead,nofoot]{geometry}
%\usepackage{geometry}
\usepackage{amsfonts}
\usepackage{amstext}
\usepackage{latexsym}
\usepackage{amssymb}
\usepackage{color}


%\include{myPreamble}
\include{qm2pi.local} 

%\ifpdf
%\usepackage[pdftex]{graphicx}
%\else
%\usepackage{graphicx}
%\fi

 % \ifpdf
%  \usepackage{pdfsync}
%  \if


%\title{Brief Article}
%\author{David F. Snyder}
%\author{L.G. Meredith}

%\address{Dept. of Math., Texas State University--San Marcos, San Marcos, TX 78666}
       
\pagestyle{empty}


\begin{document}

\lstset{language=[Objective]Caml,frame=shadowbox}

\input{qm2pi.front}

% section front matter (end)

\input{qm2pi.intro} 
 
% section introduction (end)

% \input{qm2pi.knotations} 

% section notation (end)

\input{qm2pi.process.calculi} 

% section concurrent_process_calculi_and_spatial_logics_ (end)
    
%\input{qm2pi.knots2pi} 

%\input{qm2pi.trefoil} 

%\input{qm2pi.mainthm} 

% subsection basic_interpretation (end)

%\input{qm2pi.rho.presentation} 
\subsection{The syntax and semantics of the notation system}\label{sub:the_syntax_and_semantics_of_the_notation_system} % (fold)

We now summarize a technical presentation of the calculus that
embodies our theory of dynamics. The typical presentation of such a
calculus follows the style of giving generators and relations on
them. The grammar, below, describing term constructors, freely
generates the set of processes, $\Proc$. This set is then quotiented
by a relation known as structural congruence and it is over this set
that the notion of dynamics is expressed. This presentation is
essentially that of \cite{MeredithR05} with the addition of
polyadicity and summation. For readability we have relegated some of
the technical subtleties to an appendix.

\subsubsection{Process grammar}\label{subsub:process_grammar}

\begin{mathpar}
  \inferrule* [lab=synchronization] {} {{M} \bc \pzero \;|\; x?F \;|\; x!C }
  \and
  \inferrule* [lab=abstraction] {} {{F} \bc (x)P}
  \and
  \inferrule* [lab=concretion] {} {{C} \bc \langle Q \rangle}
  \and
  \inferrule* [lab=process] {} {{P,Q} \bc M \;| \;P|Q \;|\; @{x}}
  \and
  \inferrule* [lab=name] {} {{x} \bc \quotep{P}}
\end{mathpar} 

Note that $\vec{x}$ (resp. $\vec{P}$) denotes a vector of names
(resp. processes) of length $|\vec{x}|$ (resp. $|\vec{P}|$). We adopt
the following useful abbreviations.

\begin{mathpar}
   x?(\vec{y}).P := x.(\vec{y})P \and  x\clift{\vec{P}} := x.\clift{\vec{P}}
   \and x!(y) := \lift{x}{\dropn{y}}
   \and \Pi_{i=0}^{n-1}P_i := P_0 | \ldots | P_{n-1}
\end{mathpar}

\subsubsection{Structural congruence}

\paragraph{Free and bound names and alpha-equivalence.} At the
core of structural equivalence is alpha-equivalence which identifies
process that are the same up to a change of variable. Formally, we
recognize the distinction between free and bound names. The free names
of a process, $\freenames{P}$, may be calculated recursively as
follows:

\begin{mathpar}
\freenames{\pzero} := \emptyset
  \and \\
  \freenames{x?(y).P} := \{ x \} \cup (\freenames{P} \setminus \{ y \})
  \and 
  \freenames{x!\langle P \rangle} := \{ x \} \cup \{ P \} 
  \and \\
  \freenames{P|Q} := \freenames{P} \cup \freenames{Q}
  \and \\
  \freenames{@{x}} := \{ x \}
\end{mathpar}

$\pi$
$\quotep{\pi}$

$\freenames{-} : \pi \to \mathcal{P}(\quotep{\pi})$

\begin{eqnarray*}
  \freenames{\pzero} & := & \emptyset \\
  \freenames{x?(y).P} & := & \{ x \} \cup (\freenames{P} \setminus \{ y \}) \\
  \freenames{x!\langle P \rangle} & := & \{ x \} \cup \{ P \} \\
  \freenames{P|Q} & := & \freenames{P} \cup \freenames{Q} \\
  \freenames{\dropn{x}} & := & \{ x \}
\end{eqnarray*}

The bound names of a process, $\boundnames{P}$, are those names occurring in $P$
that are not free. For example, in $x?(y).0$, the name $x$ is free, while $y$ is bound.

\begin{mathpar}
  \inferrule* [lab=monoidal-laws] {} { P|Q \equiv Q|P \and P|0 \equiv P \and P|(Q|R) \equiv (P|Q)|R }
\end{mathpar}

\begin{mathpar}
  \inferrule* [lab=alpha-equivalence] {} { (x)P \equiv (y)P\{y/x\} \and y \not\in \freenames{P} }
\end{mathpar}

\begin{definition}
Then two processes, $P,Q$, are alpha-equivalent if $P = Q\{\vec{y}/\vec{x}\}$ for
some $\vec{x} \in \boundnames{Q},\vec{y} \in \boundnames{P}$, where $Q\{\vec{y}/\vec{x}\}$
denotes the capture-avoiding substitution of $\vec{y}$ for $\vec{x}$ in $Q$.
\end{definition}

\begin{definition}
  The {\em structural congruence} \cite{SangiorgiWalker} , $\equiv$,
  between processes is the least congruence containing
  alpha-equivalence, satisfying the abelian monoid laws
  (associativity, commutativity and $\pzero$ as identity) for parallel
  composition $|$ and for summation $+$.
\end{definition}

\subsection{Name equivalence}

We take name equivalence, written $\nameeq$, to be the smallest
equivalence relation generated by the following rules.

\begin{mathpar}
\inferrule*[lab=Quote-drop]
{ }
{ \quotep{@{x}} \nameeq x }

\inferrule*[lab=Struct-equiv]
{ P \scong Q }
{ \quotep{P} \nameeq \quotep{Q} }
\end{mathpar}

The astute reader will have noticed that the mutual recursion of names
and processes imposes a mutual recursion on alpha-equivalence and
structural equivalence via name-equivalence. Fortunately, all of this
works out pleasantly and we may calculate in the natural way, free of
concern. The reader interested in the details is referred to the
appendix \ref{appendix:rho_details}.

\subsection{Substitution}

We use $\Proc$ for the set of processes, $\QProc$ for the set of
names, and $\id{\{}\vec{y} / \vec{x} \id{\}}$ to denote partial maps,
$s : \QProc \rightarrow \QProc$. A map, $s$ lifts, uniquely, to a map
on process terms, $\widehat{s} : \Proc \rightarrow \Proc$ by the
following equations.

\begin{mathpar}
  (0) \psubstp{Q}{P} := 0 \\
  (R \juxtap S) \psubstp{Q}{P}
  :=    
  (R)\psubstp{Q}{P} \juxtap (S) \psubstp{Q}{P} \\
  (x?(y).R) \psubstp{Q}{P}    
  :=    
  (x)\substp{Q}{P} (z)\concat( (R \psubstn{z}{y}) \psubstp{Q}{P} ) \\
  (\lift{x}{R}) \psubstp{Q}{P}  
  :=
  \lift{(x)\substp{Q}{P}}{ R \psubstp{Q}{P} } \\
%   (\dropn{x})  \psubstp{Q}{P}       
%   := 
%   \left\{ 
%     \begin{array}{ccc} 
%       \dropn{\quotep{Q}} & & x \nameeq \quotep{P} \\
%       \dropn{x} & & otherwise \\
%     \end{array}
%   \right. 
  (\dropn{x})  \psubstp{Q}{P}       
  := 
  \left\{ 
    \begin{array}{ccc} 
      Q & & x \nameeq \quotep{P} \\
      \dropn{x} & & otherwise \\
    \end{array}
  \right.
\end{mathpar}
 

where

\begin{eqnarray}
  (x)\id{\{} \lpquote Q \rpquote / \lpquote P \rpquote \id{\}}            = 
  \left\{ 
    \begin{array}{ccc}
      \lpquote Q \rpquote & & x \nameeq \lpquote P \rpquote \\
      x & & otherwise \\
    \end{array}
  \right. \nonumber
\end{eqnarray}

and $z$ is chosen distinct from $\quotep{P}$, $\quotep{Q}$, the free
names in $Q$, and all the names in $R$. Our $\alpha$-equivalence will
be built in the standard way from this substitution.

\begin{remark}\label{rem:no_self_referential_names}
  One consequence of these definitions is that $\forall P. \quotep{P}
  \not\in \freenames{P}$.
\end{remark}

\subsection{ Dynamic quote: an example }

Anticipating something of what's to come, consider applying the
substitution, $\widehat{\id{\{}u / z \id{\}}}$, to the following pair
of processes, $\lift{w}{y!(z)}$ and $w[ \lpquote y!(z) \rpquote ]$.

\begin{eqnarray}
	\lift{w}{y!(z)}\widehat{\id{\{}u / z \id{\}}}
		& = &
		\lift{w}{y!(u)} \nonumber\\
	w[ \lpquote y!(z) \rpquote ] \widehat{ \id{\{}u / z \id{\}} }
		& = &
		w[ \lpquote y!(z) \rpquote ] \nonumber
\end{eqnarray}

Because the body of the process between quotes is impervious to
substitution, we get radically different answers. In fact, by
examining the first process in an input context,
e.g. $x?(z).\lift{w}{y!(z)}$, we see that the process under the lift
operator may be shaped by prefixed inputs binding a name inside it. In
this sense, the lift operator will be seen as a way to dynamically
construct processes before reifying them as names.

Finally equipped with these standard features we can present the
dynamics of the calculus.

\subsubsection{Operational semantics} 

Finally, we introduce the computational dynamics. What marks these
algebras as distinct from other more traditionally studied algebraic
structures, e.g. vector spaces or polynomial rings, is the manner in
which dynamics is captured. In traditional structures, dynamics is typically
expressed through morphisms between such structures, as in linear maps
between vector spaces or morphisms between rings. In algebras
associated with the semantics of computation, the dynamics is
expressed as part of the algebraic structure itself, through a
reduction reduction relation typically denoted by $\red$. Below, we
give a recursive presentation of this relation for the calculus used
in the encoding.

$\red \subseteq \pi \times \pi$
$\red : \pi \to \mathcal{P}(\pi)$

\begin{mathpar}
  \inferrule* [lab=Comm] { \textsf{match}( x_{src}, x_{trgt} ) } { x_{trgt}?(y)P \; | \; x_{src}!\langle {Q} \rangle \red P\{\quotep{Q}/y}\} }
  \and \\
  \inferrule* [lab=Par] {{P} \red {P}'} {{{P} | {Q}} \red {{P}' | {Q}}}
  \and
  \inferrule* [lab=Equiv]{{{P} \scong {P}'} \andalso {{P}' \red {Q}'} \andalso {{Q}' \scong {Q}}}{{P} \red {Q}}
\end{mathpar}

\begin{eqnarray*}
  match_{\equiv} (\quotep{P},\quotep{Q}) & := & P \equiv Q \\
  match_{\dagger}(\quotep{P},\quotep{Q}) & := & \forall R. P|Q \red^{*} R => R \red^{*} 0 \\
  match_{K}(\quotep{P},\quotep{Q}) & := & K \mbox{ for some context } K
\end{eqnarray*}

$u?(x)P | u!\langle Q \rangle \red P\{\quotep{Q}/x\}$

%We write $\wred$ for $\red^*$, and $P\red$ if $\exists Q $ such that $ P \red Q$.
We write $P\red$ if $\exists Q $ such that $ P \red Q$ and $P\not\red$, otherwise.

\section{Replication}

As mentioned before, it is known that replication (and hence
recursion) can be implemented in a higher-order process algebra
\cite{SangiorgiWalker}. As our first example of calculation with the
machinery thus far presented we give the construction explicitly in
the {\rhoc}.

\begin{eqnarray}
	D_{x} & := & \prefix{x}{y}{(\binpar{\outputp{x}{y}}{@{y}})} \nonumber\\
	\bangp_{x}{P} & := & \binpar{{x}!\langle{\binpar{D_{x}}{P}}\rangle}{D_{x}} \nonumber
\end{eqnarray}

\begin{eqnarray}
	\bangp_{x}{P} & & \nonumber\\
	=
	& {x}!\langle{(\prefix{x}{y}{(\outputp{x}{y} | @{y})) | P}}\rangle 
	      | \prefix{x}{y}{(\outputp{x}{y} | @{y})} & \nonumber\\
	\red
	& (\outputp{x}{y} | @{y})\substn{\quotep{(\prefix{x}{y}{(@{y} | \outputp{x}{y})) | P}}}{y} & \nonumber\\
	=
	& \outputp{x}{\quotep{(\prefix{x}{y}{(\outputp{x}{y} | @{y})) | P}}}
	  | {(\prefix{x}{y}{(\outputp{x}{y} | @{y})) | P}} & \nonumber\\
	\red
	& \ldots & \nonumber\\
	\red^*
	& P | P | \ldots & \nonumber
\end{eqnarray}

Of course, this encoding, as an implementation, runs away, unfolding
$\bangp{P}$ eagerly. A lazier and more implementable replication
operator, restricted to input-guarded processes, may be obtained as follows.

\begin{eqnarray}
\bangp{\prefix{u}{v}{P}} 
	:= 
	\binpar{\lift{x}{\prefix{u}{v}{(\binpar{D(x)}{P})}}}{D(x)} \nonumber
\end{eqnarray}

\begin{remark}
  Note that the lazier definition still does not deal with summation
  or mixed summation (i.e. sums over input and output). The reader is
  invited to construct definitions of replication that deal with these
  features. 

  Further, the definitions are parameterized in a name, $x$. Can you,
  gentle reader, make a definition that eliminates this parameter and
  guarantees no accidental interaction between the replication
  machinery and the process being replicated -- i.e. no accidental
  sharing of names used by the process to get its work done and the
  name(s) used by the replication to effect copying. This latter
  revision of the definition of replication is crucial to obtaining
  the expected identity $!!P \sim !P$.
\end{remark}

\begin{remark}\label{rem:paradoxical_combinator}
  The reader familiar with the lambda calculus will have noticed the
  similarity between $D$ and the paradoxical combinator.

  [Ed. note: the existence of this seems to suggest we have to be more
  restrictive on the set of processes and names we admit if we are to
  support no-cloning.]
\end{remark}

\subsubsection{Bisimulation}

The computational dynamics gives rise to another kind of equivalence,
the equivalence of computational behavior. As previously mentioned
this is typically captured \emph{via} some form of bisimulation.

% The notion we use in this paper is weak barbed bisimulation
% \cite{milner91polyadicpi}.

The notion we use in this paper is derived from weak barbed
bisimulation \cite{milner91polyadicpi}. 

\begin{definition}
An \emph{observation relation}, $\downarrow_{\mathcal N}$, over a set
of names, $\mathcal N$, is the smallest relation satisfying the rules
below.

\infrule[Out-barb]{y \in {\mathcal N}, \; x \nameeq y}
		  {\outputp{x}{v} \downarrow_{\mathcal N} x}
\infrule[Par-barb]{\mbox{$P\downarrow_{\mathcal N} x$ or $Q\downarrow_{\mathcal N} x$}}
		  {\binpar{P}{Q} \downarrow_{\mathcal N} x}

We write $P \Downarrow_{\mathcal N} x$ if there is $Q$ such that 
$P \wred Q$ and $Q \downarrow_{\mathcal N} x$.
\end{definition}

\begin{definition}
%\label{def.bbisim}
An  ${\mathcal N}$-\emph{barbed bisimulation} over a set of names, ${\mathcal N}$, is a symmetric binary relation 
${\mathcal S}_{\mathcal N}$ between agents such that $P\rel{S}_{\mathcal N}Q$ implies:
\begin{enumerate}
\item If $P \red P'$ then $Q \wred Q'$ and $P'\rel{S}_{\mathcal N} Q'$.
\item If $P\downarrow_{\mathcal N} x$, then $Q\Downarrow_{\mathcal N} x$.
\end{enumerate}
$P$ is ${\mathcal N}$-barbed bisimilar to $Q$, written
$P \wbbisim_{\mathcal N} Q$, if $P \rel{S}_{\mathcal N} Q$ for some ${\mathcal N}$-barbed bisimulation ${\mathcal S}_{\mathcal N}$.
\end{definition}

$\mathcal{R} \subseteq \pi \times \pi$

$P \mathcal{R} Q => \forall P'. P \red P' \Rightarrow \exists Q'. Q \red Q', P' \mathcal{R} Q'$

$P \vdash x \Rightarrow Q \vdash x$

\begin{mathpar}
  \inferrule*[lab=Out-barb]{x \nameeq y}{{y}!\langle{Q}\rangle \vdash x}
  \and
  \inferrule*[lab=Par-barb]{\mbox{$P\vdash x$ or $Q\vdash x$}}{\binpar{P}{Q} \vdash x}
\end{mathpar}

\subsubsection{Contexts}

One of the principle advantages of computational calculi like the
$\pi$-calculus is a well-defined notion of context,
contextual-equivalence and a correlation between
contextual-equivalence and notions of bisimulation. The notion of
context allows the decomposition of a process into (sub-)process and
its syntactic environment, its context. Thus, a context may be
thought of as a process with a ``hole'' (written $\Box$) in it. The
application of a context $M$ to a process $P$, written $M[P]$, is
tantamount to filling the hole in $M$ with $P$. In this paper we do
not need the full weight of this theory, but do make use of the notion
of context in the proof the main theorem. 

\begin{mathpar}
  \inferrule* [lab=summation] {} {{M_{M},M_{N}} \bc \Box \;|\; x.M_{A} \;|\; M_{M}+M_{N}}
  \and
  \inferrule* [lab=agent] {} {{M_{A}} \bc (\vec{x})M_{P} \;| \; \clift{P_0,\ldots,M_{P},\ldots,P_N}}
  \and \\
  \inferrule* [lab=process] {} {{M_{P}} \bc M_{N} \;| \;P|M_{P} }
\end{mathpar} 

\begin{mathpar}
  \inferrule* [lab=sychronization] {} {M_{N} \bc \Box \;|\; x?M_{F} \;|\; x!M_{C}}
  \and
  \inferrule* [lab=abstraction] {} {{M_{F}} \bc (x)M_{P} }
  \and
  \inferrule* [lab=concretion] {} {{M_{C}} \bc \langle M_{P} \rangle }
  \and \\
  \inferrule* [lab=process] {} {{M_{P}} \bc M_{N} \;| \;P|M_{P} }
\end{mathpar}

\begin{definition}[contextual application] Given a context $M$, and
  process $P$, we define the \emph{contextual application}, $M[P] :=
  M\{P/\Box\}$. That is, the contextual application of M to P is the
  substitution of $P$ for $\Box$ in $M$.
\end{definition}

$\meaningof{-} : L \to \mathcal{P}(\pi)$

\begin{mathpar}
  \inferrule* [lab=collection] {} {\meaningof{true} = \pi, \and \meaningof{~E} = \pi \setminus \meaningof{E}, \and \meaningof{E_{1} \& E_{2}} = \meaningof{E_{1}} \cap \meaningof{E_{2}}}
\end{mathpar}

\begin{mathpar}
  \inferrule* [lab=structure] {} {\meaningof{0} = \{ P \in \pi | P \equiv 0 \}, \and \\ \meaningof{E_1 | E_2} = \{ P \in \pi | P \equiv P_{1} | P_{2}, P_{1} \in \meaningof{E_{1}}, P_{2} \in \meaningof{E_2}\} }
\end{mathpar}

\begin{mathpar}
 \inferrule* [lab=behavior] {} {\meaningof{\langle a?b \rangle E} = \{ P \in \pi | P \equiv Q | u?(y)P', \\ \and \\\\ \and \\ \;\;\; u \in \meaningof{a}, \forall z.P'\{z/y\} \in \meaningof{E\{z/b\}}\}, \and \\ \meaningof{a!E} = \{ P \in \pi | P \equiv Q | x!\langle P' \rangle, x \in \meaningof{a} P' \in \meaningof{E}\} }
\end{mathpar}

\begin{mathpar}
 \inferrule* [lab=nominal] {} {\meaningof{\quotep{E}} = \{ \quotep{P} \in \quotep{\pi} | P \in \meaningof{E} \}, \and \meaningof{\quotep{P}} = \{ \quotep{Q} \in \quotep{\pi} | P \equiv Q \} \and \\ \meaningof{@\quotep{E}} = \{ P \in \pi | P \equiv @x, x \in \meaningof{E} \}}
\end{mathpar}

\begin{eqnarray*}
  \\
  \meaningof{-} : TS \to ST
\end{eqnarray*}

\begin{eqnarray*}
  \\
  L : TS \to ST
\end{eqnarray*}

\begin{eqnarray*}
  \\
  P \models E \iff P \in \meaningof{E}
\end{eqnarray*}

\begin{eqnarray*}
  P \approx_{L} Q \iff \forall E \in L. P \models E \iff Q \models E
\end{eqnarray*}

\begin{eqnarray*}
  P \approx_{K} Q
\end{eqnarray*}

\begin{eqnarray*}
  P \approx Q
\end{eqnarray*}

$\approx_{K} = \approx = \approx_{L}$

\subsubsection{Contextual duality}

Note that contexts extend the quotation operation to a family of
operations from processes to names. Given a context, $M$, we can
define a \emph{nominal context}, $\quotep{M}$ by $\quotep{M}[P] :=
\quotep{M[P]}$. To foreshadow what is to come we observe that these
operations enjoy a duality with processes very much like the duality
between vectors and maps from vectors to scalars.

Further, because the calculus is essentially higher-order, we have a
correspondence between contexts and processes. More specifically,
given a name $x$ and a context $M$ we can construct $M^{*}_{x}$ such
that 

\begin{mathpar}
  M^{*}_{x} | \lift{x}{P} \red M[P]
\end{mathpar}

namely,

\begin{mathpar}
  M^{*}_{x} := x?(u).M[\dropn{u}]
\end{mathpar}

The dependence of $M^{*}_{x}$ on a name makes it an abstraction, 

\begin{mathpar}
  M^{*} := (x)x?(u).M[\dropn{u}]
\end{mathpar}

\subsection{Additional notation}

It will sometimes be convenient to denote the process a name
quotes. We already have the notation $x = \quotep{P}$, but it will be
convenient to introduce an alternate notation, $\procn{x}$, when we
want to emphasize the connection to the use of the name. Note that, by
virtue of name equivalence, $\quotep{\procn{x}} \nameeq x$; so, the
notation is consistent with previous definitions.

Further, because names have structure it is possible to effect
substitutions on the basis of that structure. This means we need to
upgrade our notation for substitutions, which we accomplish by
adapting comprehension notation. Thus,

\begin{mathpar}
  P\{ y / x : x \in S \}
\end{mathpar}

is interpreted to mean the process derived from P by replacing (in a
capture-avoiding manner) each occurrence of $x$ in $S$ by $y$. For example,

\begin{mathpar}
  P\{ \quotep{\procn{x}|\procn{x}} / x : x \in \freenames{P} \}
\end{mathpar}

will replace each (occurrence) of a free name $x$ in $P$ by
$\quotep{\procn{x}|\procn{x}}$.

Also, we will avail ourselves of the notation $x^{L}$ and $x^{R}$ to
denote injections of a name into disjoint copies of the name
space. There are numerous ways to accomplish this. One example can be
found in \cite{MeredithR05}. This notation overloads to vectors of
names: $\vec{x}^{\pi} := (x_{i}^{\pi} \; : \; 0 \leq i < |\vec{x}| )$ where $\pi \in \{L,R\}$.

We also use $P^{\Box} := P|\Box$.

In \cite{MeredithR05} an interpretation of the new operator is
given. It turns out that there are several possible interpretations
all enjoying the requisite algebraic properties of the operator (see
\cite{milner91polyadicpi}). We will therefore make liberal use of
$(\nu\; \vec{x})P$.

% subsection the_syntax_and_semantics_of_the_notation_system (end)   

\input{qm2pi.qmops} 

\input{qm2pi.sterngerlach} 

\input{qm2pi.metric} 

% section concurrent_process_calculi (end)

%\input{qm2pi.proofsketch}

% section proof sketch (end)

%\input{qm2pi.slviaknots} 

% section spatial logic via knots (end)

\input{qm2pi.conclusion}

% section conclusion (end)

%\input{qm2pi.dtcodes} 

% section wiring algorithm (end)

\input{qm2pi.ack} 

% section acknowledgments (end)

\newpage


\bibliographystyle{plain}   
\bibliography{../../biblios/main.bib}

\input{qm2pi.rhodetails}

\end{document}

 

% section acknowledgments (end)

\newpage


\bibliographystyle{plain}   
\bibliography{../../biblios/main.bib}

\documentclass[12pt]{llncs}
%\documentclass{jktr}

\usepackage[pdftex]{hyperref}                   
\usepackage {listings}
\usepackage {mathpartir}
\usepackage{bcprules}
%\usepackage{listings}
                       
\usepackage{graphicx} 
%\usepackage[margins=2.5cm,nohead,nofoot]{geometry}
%\usepackage{geometry}
\usepackage{amsfonts}
\usepackage{amstext}
\usepackage{latexsym}
\usepackage{amssymb}
\usepackage{color}


%\include{myPreamble}
\include{qm2pi.local} 

%\ifpdf
%\usepackage[pdftex]{graphicx}
%\else
%\usepackage{graphicx}
%\fi

 % \ifpdf
%  \usepackage{pdfsync}
%  \if


%\title{Brief Article}
%\author{David F. Snyder}
%\author{L.G. Meredith}

%\address{Dept. of Math., Texas State University--San Marcos, San Marcos, TX 78666}
       
\pagestyle{empty}


\begin{document}

\lstset{language=[Objective]Caml,frame=shadowbox}

\input{qm2pi.front}

% section front matter (end)

\input{qm2pi.intro} 
 
% section introduction (end)

% \input{qm2pi.knotations} 

% section notation (end)

\input{qm2pi.process.calculi} 

% section concurrent_process_calculi_and_spatial_logics_ (end)
    
%\input{qm2pi.knots2pi} 

%\input{qm2pi.trefoil} 

%\input{qm2pi.mainthm} 

% subsection basic_interpretation (end)

%\input{qm2pi.rho.presentation} 
\subsection{The syntax and semantics of the notation system}\label{sub:the_syntax_and_semantics_of_the_notation_system} % (fold)

We now summarize a technical presentation of the calculus that
embodies our theory of dynamics. The typical presentation of such a
calculus follows the style of giving generators and relations on
them. The grammar, below, describing term constructors, freely
generates the set of processes, $\Proc$. This set is then quotiented
by a relation known as structural congruence and it is over this set
that the notion of dynamics is expressed. This presentation is
essentially that of \cite{MeredithR05} with the addition of
polyadicity and summation. For readability we have relegated some of
the technical subtleties to an appendix.

\subsubsection{Process grammar}\label{subsub:process_grammar}

\begin{mathpar}
  \inferrule* [lab=synchronization] {} {{M} \bc \pzero \;|\; x?F \;|\; x!C }
  \and
  \inferrule* [lab=abstraction] {} {{F} \bc (x)P}
  \and
  \inferrule* [lab=concretion] {} {{C} \bc \langle Q \rangle}
  \and
  \inferrule* [lab=process] {} {{P,Q} \bc M \;| \;P|Q \;|\; @{x}}
  \and
  \inferrule* [lab=name] {} {{x} \bc \quotep{P}}
\end{mathpar} 

Note that $\vec{x}$ (resp. $\vec{P}$) denotes a vector of names
(resp. processes) of length $|\vec{x}|$ (resp. $|\vec{P}|$). We adopt
the following useful abbreviations.

\begin{mathpar}
   x?(\vec{y}).P := x.(\vec{y})P \and  x\clift{\vec{P}} := x.\clift{\vec{P}}
   \and x!(y) := \lift{x}{\dropn{y}}
   \and \Pi_{i=0}^{n-1}P_i := P_0 | \ldots | P_{n-1}
\end{mathpar}

\subsubsection{Structural congruence}

\paragraph{Free and bound names and alpha-equivalence.} At the
core of structural equivalence is alpha-equivalence which identifies
process that are the same up to a change of variable. Formally, we
recognize the distinction between free and bound names. The free names
of a process, $\freenames{P}$, may be calculated recursively as
follows:

\begin{mathpar}
\freenames{\pzero} := \emptyset
  \and \\
  \freenames{x?(y).P} := \{ x \} \cup (\freenames{P} \setminus \{ y \})
  \and 
  \freenames{x!\langle P \rangle} := \{ x \} \cup \{ P \} 
  \and \\
  \freenames{P|Q} := \freenames{P} \cup \freenames{Q}
  \and \\
  \freenames{@{x}} := \{ x \}
\end{mathpar}

$\pi$
$\quotep{\pi}$

$\freenames{-} : \pi \to \mathcal{P}(\quotep{\pi})$

\begin{eqnarray*}
  \freenames{\pzero} & := & \emptyset \\
  \freenames{x?(y).P} & := & \{ x \} \cup (\freenames{P} \setminus \{ y \}) \\
  \freenames{x!\langle P \rangle} & := & \{ x \} \cup \{ P \} \\
  \freenames{P|Q} & := & \freenames{P} \cup \freenames{Q} \\
  \freenames{\dropn{x}} & := & \{ x \}
\end{eqnarray*}

The bound names of a process, $\boundnames{P}$, are those names occurring in $P$
that are not free. For example, in $x?(y).0$, the name $x$ is free, while $y$ is bound.

\begin{mathpar}
  \inferrule* [lab=monoidal-laws] {} { P|Q \equiv Q|P \and P|0 \equiv P \and P|(Q|R) \equiv (P|Q)|R }
\end{mathpar}

\begin{mathpar}
  \inferrule* [lab=alpha-equivalence] {} { (x)P \equiv (y)P\{y/x\} \and y \not\in \freenames{P} }
\end{mathpar}

\begin{definition}
Then two processes, $P,Q$, are alpha-equivalent if $P = Q\{\vec{y}/\vec{x}\}$ for
some $\vec{x} \in \boundnames{Q},\vec{y} \in \boundnames{P}$, where $Q\{\vec{y}/\vec{x}\}$
denotes the capture-avoiding substitution of $\vec{y}$ for $\vec{x}$ in $Q$.
\end{definition}

\begin{definition}
  The {\em structural congruence} \cite{SangiorgiWalker} , $\equiv$,
  between processes is the least congruence containing
  alpha-equivalence, satisfying the abelian monoid laws
  (associativity, commutativity and $\pzero$ as identity) for parallel
  composition $|$ and for summation $+$.
\end{definition}

\subsection{Name equivalence}

We take name equivalence, written $\nameeq$, to be the smallest
equivalence relation generated by the following rules.

\begin{mathpar}
\inferrule*[lab=Quote-drop]
{ }
{ \quotep{@{x}} \nameeq x }

\inferrule*[lab=Struct-equiv]
{ P \scong Q }
{ \quotep{P} \nameeq \quotep{Q} }
\end{mathpar}

The astute reader will have noticed that the mutual recursion of names
and processes imposes a mutual recursion on alpha-equivalence and
structural equivalence via name-equivalence. Fortunately, all of this
works out pleasantly and we may calculate in the natural way, free of
concern. The reader interested in the details is referred to the
appendix \ref{appendix:rho_details}.

\subsection{Substitution}

We use $\Proc$ for the set of processes, $\QProc$ for the set of
names, and $\id{\{}\vec{y} / \vec{x} \id{\}}$ to denote partial maps,
$s : \QProc \rightarrow \QProc$. A map, $s$ lifts, uniquely, to a map
on process terms, $\widehat{s} : \Proc \rightarrow \Proc$ by the
following equations.

\begin{mathpar}
  (0) \psubstp{Q}{P} := 0 \\
  (R \juxtap S) \psubstp{Q}{P}
  :=    
  (R)\psubstp{Q}{P} \juxtap (S) \psubstp{Q}{P} \\
  (x?(y).R) \psubstp{Q}{P}    
  :=    
  (x)\substp{Q}{P} (z)\concat( (R \psubstn{z}{y}) \psubstp{Q}{P} ) \\
  (\lift{x}{R}) \psubstp{Q}{P}  
  :=
  \lift{(x)\substp{Q}{P}}{ R \psubstp{Q}{P} } \\
%   (\dropn{x})  \psubstp{Q}{P}       
%   := 
%   \left\{ 
%     \begin{array}{ccc} 
%       \dropn{\quotep{Q}} & & x \nameeq \quotep{P} \\
%       \dropn{x} & & otherwise \\
%     \end{array}
%   \right. 
  (\dropn{x})  \psubstp{Q}{P}       
  := 
  \left\{ 
    \begin{array}{ccc} 
      Q & & x \nameeq \quotep{P} \\
      \dropn{x} & & otherwise \\
    \end{array}
  \right.
\end{mathpar}
 

where

\begin{eqnarray}
  (x)\id{\{} \lpquote Q \rpquote / \lpquote P \rpquote \id{\}}            = 
  \left\{ 
    \begin{array}{ccc}
      \lpquote Q \rpquote & & x \nameeq \lpquote P \rpquote \\
      x & & otherwise \\
    \end{array}
  \right. \nonumber
\end{eqnarray}

and $z$ is chosen distinct from $\quotep{P}$, $\quotep{Q}$, the free
names in $Q$, and all the names in $R$. Our $\alpha$-equivalence will
be built in the standard way from this substitution.

\begin{remark}\label{rem:no_self_referential_names}
  One consequence of these definitions is that $\forall P. \quotep{P}
  \not\in \freenames{P}$.
\end{remark}

\subsection{ Dynamic quote: an example }

Anticipating something of what's to come, consider applying the
substitution, $\widehat{\id{\{}u / z \id{\}}}$, to the following pair
of processes, $\lift{w}{y!(z)}$ and $w[ \lpquote y!(z) \rpquote ]$.

\begin{eqnarray}
	\lift{w}{y!(z)}\widehat{\id{\{}u / z \id{\}}}
		& = &
		\lift{w}{y!(u)} \nonumber\\
	w[ \lpquote y!(z) \rpquote ] \widehat{ \id{\{}u / z \id{\}} }
		& = &
		w[ \lpquote y!(z) \rpquote ] \nonumber
\end{eqnarray}

Because the body of the process between quotes is impervious to
substitution, we get radically different answers. In fact, by
examining the first process in an input context,
e.g. $x?(z).\lift{w}{y!(z)}$, we see that the process under the lift
operator may be shaped by prefixed inputs binding a name inside it. In
this sense, the lift operator will be seen as a way to dynamically
construct processes before reifying them as names.

Finally equipped with these standard features we can present the
dynamics of the calculus.

\subsubsection{Operational semantics} 

Finally, we introduce the computational dynamics. What marks these
algebras as distinct from other more traditionally studied algebraic
structures, e.g. vector spaces or polynomial rings, is the manner in
which dynamics is captured. In traditional structures, dynamics is typically
expressed through morphisms between such structures, as in linear maps
between vector spaces or morphisms between rings. In algebras
associated with the semantics of computation, the dynamics is
expressed as part of the algebraic structure itself, through a
reduction reduction relation typically denoted by $\red$. Below, we
give a recursive presentation of this relation for the calculus used
in the encoding.

$\red \subseteq \pi \times \pi$
$\red : \pi \to \mathcal{P}(\pi)$

\begin{mathpar}
  \inferrule* [lab=Comm] { \textsf{match}( x_{src}, x_{trgt} ) } { x_{trgt}?(y)P \; | \; x_{src}!\langle {Q} \rangle \red P\{\quotep{Q}/y}\} }
  \and \\
  \inferrule* [lab=Par] {{P} \red {P}'} {{{P} | {Q}} \red {{P}' | {Q}}}
  \and
  \inferrule* [lab=Equiv]{{{P} \scong {P}'} \andalso {{P}' \red {Q}'} \andalso {{Q}' \scong {Q}}}{{P} \red {Q}}
\end{mathpar}

\begin{eqnarray*}
  match_{\equiv} (\quotep{P},\quotep{Q}) & := & P \equiv Q \\
  match_{\dagger}(\quotep{P},\quotep{Q}) & := & \forall R. P|Q \red^{*} R => R \red^{*} 0 \\
  match_{K}(\quotep{P},\quotep{Q}) & := & K \mbox{ for some context } K
\end{eqnarray*}

$u?(x)P | u!\langle Q \rangle \red P\{\quotep{Q}/x\}$

%We write $\wred$ for $\red^*$, and $P\red$ if $\exists Q $ such that $ P \red Q$.
We write $P\red$ if $\exists Q $ such that $ P \red Q$ and $P\not\red$, otherwise.

\section{Replication}

As mentioned before, it is known that replication (and hence
recursion) can be implemented in a higher-order process algebra
\cite{SangiorgiWalker}. As our first example of calculation with the
machinery thus far presented we give the construction explicitly in
the {\rhoc}.

\begin{eqnarray}
	D_{x} & := & \prefix{x}{y}{(\binpar{\outputp{x}{y}}{@{y}})} \nonumber\\
	\bangp_{x}{P} & := & \binpar{{x}!\langle{\binpar{D_{x}}{P}}\rangle}{D_{x}} \nonumber
\end{eqnarray}

\begin{eqnarray}
	\bangp_{x}{P} & & \nonumber\\
	=
	& {x}!\langle{(\prefix{x}{y}{(\outputp{x}{y} | @{y})) | P}}\rangle 
	      | \prefix{x}{y}{(\outputp{x}{y} | @{y})} & \nonumber\\
	\red
	& (\outputp{x}{y} | @{y})\substn{\quotep{(\prefix{x}{y}{(@{y} | \outputp{x}{y})) | P}}}{y} & \nonumber\\
	=
	& \outputp{x}{\quotep{(\prefix{x}{y}{(\outputp{x}{y} | @{y})) | P}}}
	  | {(\prefix{x}{y}{(\outputp{x}{y} | @{y})) | P}} & \nonumber\\
	\red
	& \ldots & \nonumber\\
	\red^*
	& P | P | \ldots & \nonumber
\end{eqnarray}

Of course, this encoding, as an implementation, runs away, unfolding
$\bangp{P}$ eagerly. A lazier and more implementable replication
operator, restricted to input-guarded processes, may be obtained as follows.

\begin{eqnarray}
\bangp{\prefix{u}{v}{P}} 
	:= 
	\binpar{\lift{x}{\prefix{u}{v}{(\binpar{D(x)}{P})}}}{D(x)} \nonumber
\end{eqnarray}

\begin{remark}
  Note that the lazier definition still does not deal with summation
  or mixed summation (i.e. sums over input and output). The reader is
  invited to construct definitions of replication that deal with these
  features. 

  Further, the definitions are parameterized in a name, $x$. Can you,
  gentle reader, make a definition that eliminates this parameter and
  guarantees no accidental interaction between the replication
  machinery and the process being replicated -- i.e. no accidental
  sharing of names used by the process to get its work done and the
  name(s) used by the replication to effect copying. This latter
  revision of the definition of replication is crucial to obtaining
  the expected identity $!!P \sim !P$.
\end{remark}

\begin{remark}\label{rem:paradoxical_combinator}
  The reader familiar with the lambda calculus will have noticed the
  similarity between $D$ and the paradoxical combinator.

  [Ed. note: the existence of this seems to suggest we have to be more
  restrictive on the set of processes and names we admit if we are to
  support no-cloning.]
\end{remark}

\subsubsection{Bisimulation}

The computational dynamics gives rise to another kind of equivalence,
the equivalence of computational behavior. As previously mentioned
this is typically captured \emph{via} some form of bisimulation.

% The notion we use in this paper is weak barbed bisimulation
% \cite{milner91polyadicpi}.

The notion we use in this paper is derived from weak barbed
bisimulation \cite{milner91polyadicpi}. 

\begin{definition}
An \emph{observation relation}, $\downarrow_{\mathcal N}$, over a set
of names, $\mathcal N$, is the smallest relation satisfying the rules
below.

\infrule[Out-barb]{y \in {\mathcal N}, \; x \nameeq y}
		  {\outputp{x}{v} \downarrow_{\mathcal N} x}
\infrule[Par-barb]{\mbox{$P\downarrow_{\mathcal N} x$ or $Q\downarrow_{\mathcal N} x$}}
		  {\binpar{P}{Q} \downarrow_{\mathcal N} x}

We write $P \Downarrow_{\mathcal N} x$ if there is $Q$ such that 
$P \wred Q$ and $Q \downarrow_{\mathcal N} x$.
\end{definition}

\begin{definition}
%\label{def.bbisim}
An  ${\mathcal N}$-\emph{barbed bisimulation} over a set of names, ${\mathcal N}$, is a symmetric binary relation 
${\mathcal S}_{\mathcal N}$ between agents such that $P\rel{S}_{\mathcal N}Q$ implies:
\begin{enumerate}
\item If $P \red P'$ then $Q \wred Q'$ and $P'\rel{S}_{\mathcal N} Q'$.
\item If $P\downarrow_{\mathcal N} x$, then $Q\Downarrow_{\mathcal N} x$.
\end{enumerate}
$P$ is ${\mathcal N}$-barbed bisimilar to $Q$, written
$P \wbbisim_{\mathcal N} Q$, if $P \rel{S}_{\mathcal N} Q$ for some ${\mathcal N}$-barbed bisimulation ${\mathcal S}_{\mathcal N}$.
\end{definition}

$\mathcal{R} \subseteq \pi \times \pi$

$P \mathcal{R} Q => \forall P'. P \red P' \Rightarrow \exists Q'. Q \red Q', P' \mathcal{R} Q'$

$P \vdash x \Rightarrow Q \vdash x$

\begin{mathpar}
  \inferrule*[lab=Out-barb]{x \nameeq y}{{y}!\langle{Q}\rangle \vdash x}
  \and
  \inferrule*[lab=Par-barb]{\mbox{$P\vdash x$ or $Q\vdash x$}}{\binpar{P}{Q} \vdash x}
\end{mathpar}

\subsubsection{Contexts}

One of the principle advantages of computational calculi like the
$\pi$-calculus is a well-defined notion of context,
contextual-equivalence and a correlation between
contextual-equivalence and notions of bisimulation. The notion of
context allows the decomposition of a process into (sub-)process and
its syntactic environment, its context. Thus, a context may be
thought of as a process with a ``hole'' (written $\Box$) in it. The
application of a context $M$ to a process $P$, written $M[P]$, is
tantamount to filling the hole in $M$ with $P$. In this paper we do
not need the full weight of this theory, but do make use of the notion
of context in the proof the main theorem. 

\begin{mathpar}
  \inferrule* [lab=summation] {} {{M_{M},M_{N}} \bc \Box \;|\; x.M_{A} \;|\; M_{M}+M_{N}}
  \and
  \inferrule* [lab=agent] {} {{M_{A}} \bc (\vec{x})M_{P} \;| \; \clift{P_0,\ldots,M_{P},\ldots,P_N}}
  \and \\
  \inferrule* [lab=process] {} {{M_{P}} \bc M_{N} \;| \;P|M_{P} }
\end{mathpar} 

\begin{mathpar}
  \inferrule* [lab=sychronization] {} {M_{N} \bc \Box \;|\; x?M_{F} \;|\; x!M_{C}}
  \and
  \inferrule* [lab=abstraction] {} {{M_{F}} \bc (x)M_{P} }
  \and
  \inferrule* [lab=concretion] {} {{M_{C}} \bc \langle M_{P} \rangle }
  \and \\
  \inferrule* [lab=process] {} {{M_{P}} \bc M_{N} \;| \;P|M_{P} }
\end{mathpar}

\begin{definition}[contextual application] Given a context $M$, and
  process $P$, we define the \emph{contextual application}, $M[P] :=
  M\{P/\Box\}$. That is, the contextual application of M to P is the
  substitution of $P$ for $\Box$ in $M$.
\end{definition}

$\meaningof{-} : L \to \mathcal{P}(\pi)$

\begin{mathpar}
  \inferrule* [lab=collection] {} {\meaningof{true} = \pi, \and \meaningof{~E} = \pi \setminus \meaningof{E}, \and \meaningof{E_{1} \& E_{2}} = \meaningof{E_{1}} \cap \meaningof{E_{2}}}
\end{mathpar}

\begin{mathpar}
  \inferrule* [lab=structure] {} {\meaningof{0} = \{ P \in \pi | P \equiv 0 \}, \and \\ \meaningof{E_1 | E_2} = \{ P \in \pi | P \equiv P_{1} | P_{2}, P_{1} \in \meaningof{E_{1}}, P_{2} \in \meaningof{E_2}\} }
\end{mathpar}

\begin{mathpar}
 \inferrule* [lab=behavior] {} {\meaningof{\langle a?b \rangle E} = \{ P \in \pi | P \equiv Q | u?(y)P', \\ \and \\\\ \and \\ \;\;\; u \in \meaningof{a}, \forall z.P'\{z/y\} \in \meaningof{E\{z/b\}}\}, \and \\ \meaningof{a!E} = \{ P \in \pi | P \equiv Q | x!\langle P' \rangle, x \in \meaningof{a} P' \in \meaningof{E}\} }
\end{mathpar}

\begin{mathpar}
 \inferrule* [lab=nominal] {} {\meaningof{\quotep{E}} = \{ \quotep{P} \in \quotep{\pi} | P \in \meaningof{E} \}, \and \meaningof{\quotep{P}} = \{ \quotep{Q} \in \quotep{\pi} | P \equiv Q \} \and \\ \meaningof{@\quotep{E}} = \{ P \in \pi | P \equiv @x, x \in \meaningof{E} \}}
\end{mathpar}

\begin{eqnarray*}
  \\
  \meaningof{-} : TS \to ST
\end{eqnarray*}

\begin{eqnarray*}
  \\
  L : TS \to ST
\end{eqnarray*}

\begin{eqnarray*}
  \\
  P \models E \iff P \in \meaningof{E}
\end{eqnarray*}

\begin{eqnarray*}
  P \approx_{L} Q \iff \forall E \in L. P \models E \iff Q \models E
\end{eqnarray*}

\begin{eqnarray*}
  P \approx_{K} Q
\end{eqnarray*}

\begin{eqnarray*}
  P \approx Q
\end{eqnarray*}

$\approx_{K} = \approx = \approx_{L}$

\subsubsection{Contextual duality}

Note that contexts extend the quotation operation to a family of
operations from processes to names. Given a context, $M$, we can
define a \emph{nominal context}, $\quotep{M}$ by $\quotep{M}[P] :=
\quotep{M[P]}$. To foreshadow what is to come we observe that these
operations enjoy a duality with processes very much like the duality
between vectors and maps from vectors to scalars.

Further, because the calculus is essentially higher-order, we have a
correspondence between contexts and processes. More specifically,
given a name $x$ and a context $M$ we can construct $M^{*}_{x}$ such
that 

\begin{mathpar}
  M^{*}_{x} | \lift{x}{P} \red M[P]
\end{mathpar}

namely,

\begin{mathpar}
  M^{*}_{x} := x?(u).M[\dropn{u}]
\end{mathpar}

The dependence of $M^{*}_{x}$ on a name makes it an abstraction, 

\begin{mathpar}
  M^{*} := (x)x?(u).M[\dropn{u}]
\end{mathpar}

\subsection{Additional notation}

It will sometimes be convenient to denote the process a name
quotes. We already have the notation $x = \quotep{P}$, but it will be
convenient to introduce an alternate notation, $\procn{x}$, when we
want to emphasize the connection to the use of the name. Note that, by
virtue of name equivalence, $\quotep{\procn{x}} \nameeq x$; so, the
notation is consistent with previous definitions.

Further, because names have structure it is possible to effect
substitutions on the basis of that structure. This means we need to
upgrade our notation for substitutions, which we accomplish by
adapting comprehension notation. Thus,

\begin{mathpar}
  P\{ y / x : x \in S \}
\end{mathpar}

is interpreted to mean the process derived from P by replacing (in a
capture-avoiding manner) each occurrence of $x$ in $S$ by $y$. For example,

\begin{mathpar}
  P\{ \quotep{\procn{x}|\procn{x}} / x : x \in \freenames{P} \}
\end{mathpar}

will replace each (occurrence) of a free name $x$ in $P$ by
$\quotep{\procn{x}|\procn{x}}$.

Also, we will avail ourselves of the notation $x^{L}$ and $x^{R}$ to
denote injections of a name into disjoint copies of the name
space. There are numerous ways to accomplish this. One example can be
found in \cite{MeredithR05}. This notation overloads to vectors of
names: $\vec{x}^{\pi} := (x_{i}^{\pi} \; : \; 0 \leq i < |\vec{x}| )$ where $\pi \in \{L,R\}$.

We also use $P^{\Box} := P|\Box$.

In \cite{MeredithR05} an interpretation of the new operator is
given. It turns out that there are several possible interpretations
all enjoying the requisite algebraic properties of the operator (see
\cite{milner91polyadicpi}). We will therefore make liberal use of
$(\nu\; \vec{x})P$.

% subsection the_syntax_and_semantics_of_the_notation_system (end)   

\input{qm2pi.qmops} 

\input{qm2pi.sterngerlach} 

\input{qm2pi.metric} 

% section concurrent_process_calculi (end)

%\input{qm2pi.proofsketch}

% section proof sketch (end)

%\input{qm2pi.slviaknots} 

% section spatial logic via knots (end)

\input{qm2pi.conclusion}

% section conclusion (end)

%\input{qm2pi.dtcodes} 

% section wiring algorithm (end)

\input{qm2pi.ack} 

% section acknowledgments (end)

\newpage


\bibliographystyle{plain}   
\bibliography{../../biblios/main.bib}

\input{qm2pi.rhodetails}

\end{document}



\end{document}



% section front matter (end)

\section{Introduction}\label{sec:introduction} % (fold)
In this draft of the material i am going to have to dispense with the
usual writing conventions adopted in papers on these topics. i'm going
to have adopt whatever tone i need at the time i'm writing up the
calculations. Sometimes this may be very conversational; others it may
be the barest mathematical grunts; others still it may be that i have
lifted text from one of my other papers because the exposition of some
point was better said there. i hope that my readers are not unduly put
out by this decision. i'm not doing this to flout convention or be
rebellious. i find these calculations very technically challenging. To
keep everything going technically, something has to give; i have to
let go of some cognitive burden. So, the academic writing style --
with all of its trade-offs in terms of facilitating technical
communication -- is what i'm letting go of. Perhaps subsequent drafts
can be tightened and polished, but for now, i'm going to speak as if
we were sitting together in a coffee shop with a laptop, wifi and a
pad of paper and a pencil.

So, here's what i have to say. We -- you and i, comfortably ensconced
in our coffee shop and well-equipped with our tools -- can realize and
carry out the calculations of quantum mechanics over a very different
formal theory of dynamics, a formal theory of dynamics that
corresponds to a theory of concurrent computation with
\emph{reflection}. It has the advantage that the underlying theory is
already `quantized', but supports analogues all of the continuuous
operations. Strikingly, this underlying theory has recently been
connected with a notion of metric that we can show, by calculating
together, coincides with the metric induced by the inner product.

There are a lot of reasons why you might be interested in seeing
calculations of this form. Here's why i'm interested. For the past
several centuries there has been no competitor to the ``Newtonian''
account of dynamics. As a result the predominant share of accounts of
dynamical systems and situations have had to be formulated in terms of
the Newtonian machinery. i view this as an intellectually dangerous
position to occupy. Everything, despite it's intrinsic shape, turns
into a nail to be hit with this hammer. Recently, however, the theory
of computation has matured to the point where we have candidates for
theories of dynamics that offer very different perspective on
reasoning about dynamical systems and situations. Testing these
candidates against very successful accounts of dynamical situations,
like quantum mechanics, is going to give us some sense of how mature
they are and some measure of the quality of these accounts of
dynamics.

\subsection{Summary of contributions and outline of paper}

So, we're going to develop an interpretation of the operations of
quantum mechanics normally interpreted by Hilbert spaces and
operators. We're going to do this over a theory of computation. Note
that this is very different than the usual quantum computation program
which develops notions of computation over quantum mechanics. Rather,
we are developing a story that aligns with Wheeler's slogan: It from
Bit. To do this we will first provide an account of the theory of
computation at play here. Then we will dive into a calculation-driven
interpretation of the operations of quantum mechanics.

The reason we take this approach is that -- until very recently --
there hasn't been an axiomatic account of quantum mechanics. As a
result there has been no sharp delineation of the mathematical theory
supporting interpretation of the physical theory and the physical
theory, itself. So, ambient features of the maths are free to be
exploited (or supressed) without a real accounting of their physical
relevance. There is no sharp statement ``here's the physical theory''
qua \emph{theory} and ``here's the mathematical interpretation''
enabling a judgment of how faithful the interpretation is -- apart
from experimental observation. When there is an axiomatic account we
can judge how well a given mathematical formalism supports an
interpretation of the axioms, independent of
experimentation. Likewise, we can judge how well we have captured our
physical evidence and experience with our axiomatics, independent of
any specific mathematical implementation, with accidental detail that
may or may not have physical significance. 

In lieu of a fully fleshed out and vetted axiomatic account of quantum
mechanics, interpreting the operational notions in service of modeling
physical systems will have to suffice. In other words, we are not in
the business of providing a model of Hilbert spaces and operators. We
are in the business of providing a model of quantum mechanics because
we are motivated by testing our notions of dynamics against physical
theory; and, the predictive calculations of the physical theory must
serve as the best formulation -- shy of a fully fleshed out axiomatic
account -- of the physical theory itself (as they have for scientific
theories since time immemorial). Put another way, despite a
whole-hearted commitment to an It-from-Bit ontology, we are firmly
aligned with the shut-up-and-calculate camp as the best way to obtain
results either from the physical perspective or as a quality assurance
measure of our fledgling theory of dynamics.

In detail, we present a reflective process calculus. Then we develop
intuitive correspondences between the notions available in this
calculus and the usual physical notions supporting quantum mechanical
calculations. Thus, 

\begin{table}[htp]
  \center{
    \fbox{
      \begin{tabular}{c|c}
        quantum mechanics & process calculus \\
        \hline
        scalar & name \\
        state vector & process \\
        dual & contextual duals \\
        matrix & formal sums of process-context-dual pairs \\
        orthogonality & process annihilation \\
        inner product & execution-formula + quoting
      \end{tabular}
    }
  }
  \caption{QM - process calculi correspondences}
\end{table}

Then we tighten up these intuitions to operational definitions. We
employ the Dirac notation as the best proxy we can find for an
abstract syntax of the quantum mechanical notions. The definitions we
develop put us in contact with equational constraints coming from the
theory that we demonstrate the definitions and calculations satisfy.

This puts us in a position to shut up and calculate for the
Stern-Gerlach experimental set up, showing how these predictive
calculations become calculations on processes in our theory of a
reflective process calculus.

Penultimately, we demonstrate that the notion of metric coming from
the inner product coincides with the notion of metric available from
the theory of bisimulation. This demonstration gives us the right to
think of space as arising from behavior. Finally, we consider where we
might go from the new vantage point we have obtained.

% section introduction (end) 
 
% section introduction (end)

% \documentclass[12pt]{llncs}
%\documentclass{jktr}

\usepackage[pdftex]{hyperref}                   
\usepackage {listings}
\usepackage {mathpartir}
\usepackage{bcprules}
%\usepackage{listings}
                       
\usepackage{graphicx} 
%\usepackage[margins=2.5cm,nohead,nofoot]{geometry}
%\usepackage{geometry}
\usepackage{amsfonts}
\usepackage{amstext}
\usepackage{latexsym}
\usepackage{amssymb}
\usepackage{color}


%\include{myPreamble}
\documentclass[12pt]{llncs}
%\documentclass{jktr}

\usepackage[pdftex]{hyperref}                   
\usepackage {listings}
\usepackage {mathpartir}
\usepackage{bcprules}
%\usepackage{listings}
                       
\usepackage{graphicx} 
%\usepackage[margins=2.5cm,nohead,nofoot]{geometry}
%\usepackage{geometry}
\usepackage{amsfonts}
\usepackage{amstext}
\usepackage{latexsym}
\usepackage{amssymb}
\usepackage{color}


%\include{myPreamble}
\include{qm2pi.local} 

%\ifpdf
%\usepackage[pdftex]{graphicx}
%\else
%\usepackage{graphicx}
%\fi

 % \ifpdf
%  \usepackage{pdfsync}
%  \if


%\title{Brief Article}
%\author{David F. Snyder}
%\author{L.G. Meredith}

%\address{Dept. of Math., Texas State University--San Marcos, San Marcos, TX 78666}
       
\pagestyle{empty}


\begin{document}

\lstset{language=[Objective]Caml,frame=shadowbox}

\input{qm2pi.front}

% section front matter (end)

\input{qm2pi.intro} 
 
% section introduction (end)

% \input{qm2pi.knotations} 

% section notation (end)

\input{qm2pi.process.calculi} 

% section concurrent_process_calculi_and_spatial_logics_ (end)
    
%\input{qm2pi.knots2pi} 

%\input{qm2pi.trefoil} 

%\input{qm2pi.mainthm} 

% subsection basic_interpretation (end)

%\input{qm2pi.rho.presentation} 
\subsection{The syntax and semantics of the notation system}\label{sub:the_syntax_and_semantics_of_the_notation_system} % (fold)

We now summarize a technical presentation of the calculus that
embodies our theory of dynamics. The typical presentation of such a
calculus follows the style of giving generators and relations on
them. The grammar, below, describing term constructors, freely
generates the set of processes, $\Proc$. This set is then quotiented
by a relation known as structural congruence and it is over this set
that the notion of dynamics is expressed. This presentation is
essentially that of \cite{MeredithR05} with the addition of
polyadicity and summation. For readability we have relegated some of
the technical subtleties to an appendix.

\subsubsection{Process grammar}\label{subsub:process_grammar}

\begin{mathpar}
  \inferrule* [lab=synchronization] {} {{M} \bc \pzero \;|\; x?F \;|\; x!C }
  \and
  \inferrule* [lab=abstraction] {} {{F} \bc (x)P}
  \and
  \inferrule* [lab=concretion] {} {{C} \bc \langle Q \rangle}
  \and
  \inferrule* [lab=process] {} {{P,Q} \bc M \;| \;P|Q \;|\; @{x}}
  \and
  \inferrule* [lab=name] {} {{x} \bc \quotep{P}}
\end{mathpar} 

Note that $\vec{x}$ (resp. $\vec{P}$) denotes a vector of names
(resp. processes) of length $|\vec{x}|$ (resp. $|\vec{P}|$). We adopt
the following useful abbreviations.

\begin{mathpar}
   x?(\vec{y}).P := x.(\vec{y})P \and  x\clift{\vec{P}} := x.\clift{\vec{P}}
   \and x!(y) := \lift{x}{\dropn{y}}
   \and \Pi_{i=0}^{n-1}P_i := P_0 | \ldots | P_{n-1}
\end{mathpar}

\subsubsection{Structural congruence}

\paragraph{Free and bound names and alpha-equivalence.} At the
core of structural equivalence is alpha-equivalence which identifies
process that are the same up to a change of variable. Formally, we
recognize the distinction between free and bound names. The free names
of a process, $\freenames{P}$, may be calculated recursively as
follows:

\begin{mathpar}
\freenames{\pzero} := \emptyset
  \and \\
  \freenames{x?(y).P} := \{ x \} \cup (\freenames{P} \setminus \{ y \})
  \and 
  \freenames{x!\langle P \rangle} := \{ x \} \cup \{ P \} 
  \and \\
  \freenames{P|Q} := \freenames{P} \cup \freenames{Q}
  \and \\
  \freenames{@{x}} := \{ x \}
\end{mathpar}

$\pi$
$\quotep{\pi}$

$\freenames{-} : \pi \to \mathcal{P}(\quotep{\pi})$

\begin{eqnarray*}
  \freenames{\pzero} & := & \emptyset \\
  \freenames{x?(y).P} & := & \{ x \} \cup (\freenames{P} \setminus \{ y \}) \\
  \freenames{x!\langle P \rangle} & := & \{ x \} \cup \{ P \} \\
  \freenames{P|Q} & := & \freenames{P} \cup \freenames{Q} \\
  \freenames{\dropn{x}} & := & \{ x \}
\end{eqnarray*}

The bound names of a process, $\boundnames{P}$, are those names occurring in $P$
that are not free. For example, in $x?(y).0$, the name $x$ is free, while $y$ is bound.

\begin{mathpar}
  \inferrule* [lab=monoidal-laws] {} { P|Q \equiv Q|P \and P|0 \equiv P \and P|(Q|R) \equiv (P|Q)|R }
\end{mathpar}

\begin{mathpar}
  \inferrule* [lab=alpha-equivalence] {} { (x)P \equiv (y)P\{y/x\} \and y \not\in \freenames{P} }
\end{mathpar}

\begin{definition}
Then two processes, $P,Q$, are alpha-equivalent if $P = Q\{\vec{y}/\vec{x}\}$ for
some $\vec{x} \in \boundnames{Q},\vec{y} \in \boundnames{P}$, where $Q\{\vec{y}/\vec{x}\}$
denotes the capture-avoiding substitution of $\vec{y}$ for $\vec{x}$ in $Q$.
\end{definition}

\begin{definition}
  The {\em structural congruence} \cite{SangiorgiWalker} , $\equiv$,
  between processes is the least congruence containing
  alpha-equivalence, satisfying the abelian monoid laws
  (associativity, commutativity and $\pzero$ as identity) for parallel
  composition $|$ and for summation $+$.
\end{definition}

\subsection{Name equivalence}

We take name equivalence, written $\nameeq$, to be the smallest
equivalence relation generated by the following rules.

\begin{mathpar}
\inferrule*[lab=Quote-drop]
{ }
{ \quotep{@{x}} \nameeq x }

\inferrule*[lab=Struct-equiv]
{ P \scong Q }
{ \quotep{P} \nameeq \quotep{Q} }
\end{mathpar}

The astute reader will have noticed that the mutual recursion of names
and processes imposes a mutual recursion on alpha-equivalence and
structural equivalence via name-equivalence. Fortunately, all of this
works out pleasantly and we may calculate in the natural way, free of
concern. The reader interested in the details is referred to the
appendix \ref{appendix:rho_details}.

\subsection{Substitution}

We use $\Proc$ for the set of processes, $\QProc$ for the set of
names, and $\id{\{}\vec{y} / \vec{x} \id{\}}$ to denote partial maps,
$s : \QProc \rightarrow \QProc$. A map, $s$ lifts, uniquely, to a map
on process terms, $\widehat{s} : \Proc \rightarrow \Proc$ by the
following equations.

\begin{mathpar}
  (0) \psubstp{Q}{P} := 0 \\
  (R \juxtap S) \psubstp{Q}{P}
  :=    
  (R)\psubstp{Q}{P} \juxtap (S) \psubstp{Q}{P} \\
  (x?(y).R) \psubstp{Q}{P}    
  :=    
  (x)\substp{Q}{P} (z)\concat( (R \psubstn{z}{y}) \psubstp{Q}{P} ) \\
  (\lift{x}{R}) \psubstp{Q}{P}  
  :=
  \lift{(x)\substp{Q}{P}}{ R \psubstp{Q}{P} } \\
%   (\dropn{x})  \psubstp{Q}{P}       
%   := 
%   \left\{ 
%     \begin{array}{ccc} 
%       \dropn{\quotep{Q}} & & x \nameeq \quotep{P} \\
%       \dropn{x} & & otherwise \\
%     \end{array}
%   \right. 
  (\dropn{x})  \psubstp{Q}{P}       
  := 
  \left\{ 
    \begin{array}{ccc} 
      Q & & x \nameeq \quotep{P} \\
      \dropn{x} & & otherwise \\
    \end{array}
  \right.
\end{mathpar}
 

where

\begin{eqnarray}
  (x)\id{\{} \lpquote Q \rpquote / \lpquote P \rpquote \id{\}}            = 
  \left\{ 
    \begin{array}{ccc}
      \lpquote Q \rpquote & & x \nameeq \lpquote P \rpquote \\
      x & & otherwise \\
    \end{array}
  \right. \nonumber
\end{eqnarray}

and $z$ is chosen distinct from $\quotep{P}$, $\quotep{Q}$, the free
names in $Q$, and all the names in $R$. Our $\alpha$-equivalence will
be built in the standard way from this substitution.

\begin{remark}\label{rem:no_self_referential_names}
  One consequence of these definitions is that $\forall P. \quotep{P}
  \not\in \freenames{P}$.
\end{remark}

\subsection{ Dynamic quote: an example }

Anticipating something of what's to come, consider applying the
substitution, $\widehat{\id{\{}u / z \id{\}}}$, to the following pair
of processes, $\lift{w}{y!(z)}$ and $w[ \lpquote y!(z) \rpquote ]$.

\begin{eqnarray}
	\lift{w}{y!(z)}\widehat{\id{\{}u / z \id{\}}}
		& = &
		\lift{w}{y!(u)} \nonumber\\
	w[ \lpquote y!(z) \rpquote ] \widehat{ \id{\{}u / z \id{\}} }
		& = &
		w[ \lpquote y!(z) \rpquote ] \nonumber
\end{eqnarray}

Because the body of the process between quotes is impervious to
substitution, we get radically different answers. In fact, by
examining the first process in an input context,
e.g. $x?(z).\lift{w}{y!(z)}$, we see that the process under the lift
operator may be shaped by prefixed inputs binding a name inside it. In
this sense, the lift operator will be seen as a way to dynamically
construct processes before reifying them as names.

Finally equipped with these standard features we can present the
dynamics of the calculus.

\subsubsection{Operational semantics} 

Finally, we introduce the computational dynamics. What marks these
algebras as distinct from other more traditionally studied algebraic
structures, e.g. vector spaces or polynomial rings, is the manner in
which dynamics is captured. In traditional structures, dynamics is typically
expressed through morphisms between such structures, as in linear maps
between vector spaces or morphisms between rings. In algebras
associated with the semantics of computation, the dynamics is
expressed as part of the algebraic structure itself, through a
reduction reduction relation typically denoted by $\red$. Below, we
give a recursive presentation of this relation for the calculus used
in the encoding.

$\red \subseteq \pi \times \pi$
$\red : \pi \to \mathcal{P}(\pi)$

\begin{mathpar}
  \inferrule* [lab=Comm] { \textsf{match}( x_{src}, x_{trgt} ) } { x_{trgt}?(y)P \; | \; x_{src}!\langle {Q} \rangle \red P\{\quotep{Q}/y}\} }
  \and \\
  \inferrule* [lab=Par] {{P} \red {P}'} {{{P} | {Q}} \red {{P}' | {Q}}}
  \and
  \inferrule* [lab=Equiv]{{{P} \scong {P}'} \andalso {{P}' \red {Q}'} \andalso {{Q}' \scong {Q}}}{{P} \red {Q}}
\end{mathpar}

\begin{eqnarray*}
  match_{\equiv} (\quotep{P},\quotep{Q}) & := & P \equiv Q \\
  match_{\dagger}(\quotep{P},\quotep{Q}) & := & \forall R. P|Q \red^{*} R => R \red^{*} 0 \\
  match_{K}(\quotep{P},\quotep{Q}) & := & K \mbox{ for some context } K
\end{eqnarray*}

$u?(x)P | u!\langle Q \rangle \red P\{\quotep{Q}/x\}$

%We write $\wred$ for $\red^*$, and $P\red$ if $\exists Q $ such that $ P \red Q$.
We write $P\red$ if $\exists Q $ such that $ P \red Q$ and $P\not\red$, otherwise.

\section{Replication}

As mentioned before, it is known that replication (and hence
recursion) can be implemented in a higher-order process algebra
\cite{SangiorgiWalker}. As our first example of calculation with the
machinery thus far presented we give the construction explicitly in
the {\rhoc}.

\begin{eqnarray}
	D_{x} & := & \prefix{x}{y}{(\binpar{\outputp{x}{y}}{@{y}})} \nonumber\\
	\bangp_{x}{P} & := & \binpar{{x}!\langle{\binpar{D_{x}}{P}}\rangle}{D_{x}} \nonumber
\end{eqnarray}

\begin{eqnarray}
	\bangp_{x}{P} & & \nonumber\\
	=
	& {x}!\langle{(\prefix{x}{y}{(\outputp{x}{y} | @{y})) | P}}\rangle 
	      | \prefix{x}{y}{(\outputp{x}{y} | @{y})} & \nonumber\\
	\red
	& (\outputp{x}{y} | @{y})\substn{\quotep{(\prefix{x}{y}{(@{y} | \outputp{x}{y})) | P}}}{y} & \nonumber\\
	=
	& \outputp{x}{\quotep{(\prefix{x}{y}{(\outputp{x}{y} | @{y})) | P}}}
	  | {(\prefix{x}{y}{(\outputp{x}{y} | @{y})) | P}} & \nonumber\\
	\red
	& \ldots & \nonumber\\
	\red^*
	& P | P | \ldots & \nonumber
\end{eqnarray}

Of course, this encoding, as an implementation, runs away, unfolding
$\bangp{P}$ eagerly. A lazier and more implementable replication
operator, restricted to input-guarded processes, may be obtained as follows.

\begin{eqnarray}
\bangp{\prefix{u}{v}{P}} 
	:= 
	\binpar{\lift{x}{\prefix{u}{v}{(\binpar{D(x)}{P})}}}{D(x)} \nonumber
\end{eqnarray}

\begin{remark}
  Note that the lazier definition still does not deal with summation
  or mixed summation (i.e. sums over input and output). The reader is
  invited to construct definitions of replication that deal with these
  features. 

  Further, the definitions are parameterized in a name, $x$. Can you,
  gentle reader, make a definition that eliminates this parameter and
  guarantees no accidental interaction between the replication
  machinery and the process being replicated -- i.e. no accidental
  sharing of names used by the process to get its work done and the
  name(s) used by the replication to effect copying. This latter
  revision of the definition of replication is crucial to obtaining
  the expected identity $!!P \sim !P$.
\end{remark}

\begin{remark}\label{rem:paradoxical_combinator}
  The reader familiar with the lambda calculus will have noticed the
  similarity between $D$ and the paradoxical combinator.

  [Ed. note: the existence of this seems to suggest we have to be more
  restrictive on the set of processes and names we admit if we are to
  support no-cloning.]
\end{remark}

\subsubsection{Bisimulation}

The computational dynamics gives rise to another kind of equivalence,
the equivalence of computational behavior. As previously mentioned
this is typically captured \emph{via} some form of bisimulation.

% The notion we use in this paper is weak barbed bisimulation
% \cite{milner91polyadicpi}.

The notion we use in this paper is derived from weak barbed
bisimulation \cite{milner91polyadicpi}. 

\begin{definition}
An \emph{observation relation}, $\downarrow_{\mathcal N}$, over a set
of names, $\mathcal N$, is the smallest relation satisfying the rules
below.

\infrule[Out-barb]{y \in {\mathcal N}, \; x \nameeq y}
		  {\outputp{x}{v} \downarrow_{\mathcal N} x}
\infrule[Par-barb]{\mbox{$P\downarrow_{\mathcal N} x$ or $Q\downarrow_{\mathcal N} x$}}
		  {\binpar{P}{Q} \downarrow_{\mathcal N} x}

We write $P \Downarrow_{\mathcal N} x$ if there is $Q$ such that 
$P \wred Q$ and $Q \downarrow_{\mathcal N} x$.
\end{definition}

\begin{definition}
%\label{def.bbisim}
An  ${\mathcal N}$-\emph{barbed bisimulation} over a set of names, ${\mathcal N}$, is a symmetric binary relation 
${\mathcal S}_{\mathcal N}$ between agents such that $P\rel{S}_{\mathcal N}Q$ implies:
\begin{enumerate}
\item If $P \red P'$ then $Q \wred Q'$ and $P'\rel{S}_{\mathcal N} Q'$.
\item If $P\downarrow_{\mathcal N} x$, then $Q\Downarrow_{\mathcal N} x$.
\end{enumerate}
$P$ is ${\mathcal N}$-barbed bisimilar to $Q$, written
$P \wbbisim_{\mathcal N} Q$, if $P \rel{S}_{\mathcal N} Q$ for some ${\mathcal N}$-barbed bisimulation ${\mathcal S}_{\mathcal N}$.
\end{definition}

$\mathcal{R} \subseteq \pi \times \pi$

$P \mathcal{R} Q => \forall P'. P \red P' \Rightarrow \exists Q'. Q \red Q', P' \mathcal{R} Q'$

$P \vdash x \Rightarrow Q \vdash x$

\begin{mathpar}
  \inferrule*[lab=Out-barb]{x \nameeq y}{{y}!\langle{Q}\rangle \vdash x}
  \and
  \inferrule*[lab=Par-barb]{\mbox{$P\vdash x$ or $Q\vdash x$}}{\binpar{P}{Q} \vdash x}
\end{mathpar}

\subsubsection{Contexts}

One of the principle advantages of computational calculi like the
$\pi$-calculus is a well-defined notion of context,
contextual-equivalence and a correlation between
contextual-equivalence and notions of bisimulation. The notion of
context allows the decomposition of a process into (sub-)process and
its syntactic environment, its context. Thus, a context may be
thought of as a process with a ``hole'' (written $\Box$) in it. The
application of a context $M$ to a process $P$, written $M[P]$, is
tantamount to filling the hole in $M$ with $P$. In this paper we do
not need the full weight of this theory, but do make use of the notion
of context in the proof the main theorem. 

\begin{mathpar}
  \inferrule* [lab=summation] {} {{M_{M},M_{N}} \bc \Box \;|\; x.M_{A} \;|\; M_{M}+M_{N}}
  \and
  \inferrule* [lab=agent] {} {{M_{A}} \bc (\vec{x})M_{P} \;| \; \clift{P_0,\ldots,M_{P},\ldots,P_N}}
  \and \\
  \inferrule* [lab=process] {} {{M_{P}} \bc M_{N} \;| \;P|M_{P} }
\end{mathpar} 

\begin{mathpar}
  \inferrule* [lab=sychronization] {} {M_{N} \bc \Box \;|\; x?M_{F} \;|\; x!M_{C}}
  \and
  \inferrule* [lab=abstraction] {} {{M_{F}} \bc (x)M_{P} }
  \and
  \inferrule* [lab=concretion] {} {{M_{C}} \bc \langle M_{P} \rangle }
  \and \\
  \inferrule* [lab=process] {} {{M_{P}} \bc M_{N} \;| \;P|M_{P} }
\end{mathpar}

\begin{definition}[contextual application] Given a context $M$, and
  process $P$, we define the \emph{contextual application}, $M[P] :=
  M\{P/\Box\}$. That is, the contextual application of M to P is the
  substitution of $P$ for $\Box$ in $M$.
\end{definition}

$\meaningof{-} : L \to \mathcal{P}(\pi)$

\begin{mathpar}
  \inferrule* [lab=collection] {} {\meaningof{true} = \pi, \and \meaningof{~E} = \pi \setminus \meaningof{E}, \and \meaningof{E_{1} \& E_{2}} = \meaningof{E_{1}} \cap \meaningof{E_{2}}}
\end{mathpar}

\begin{mathpar}
  \inferrule* [lab=structure] {} {\meaningof{0} = \{ P \in \pi | P \equiv 0 \}, \and \\ \meaningof{E_1 | E_2} = \{ P \in \pi | P \equiv P_{1} | P_{2}, P_{1} \in \meaningof{E_{1}}, P_{2} \in \meaningof{E_2}\} }
\end{mathpar}

\begin{mathpar}
 \inferrule* [lab=behavior] {} {\meaningof{\langle a?b \rangle E} = \{ P \in \pi | P \equiv Q | u?(y)P', \\ \and \\\\ \and \\ \;\;\; u \in \meaningof{a}, \forall z.P'\{z/y\} \in \meaningof{E\{z/b\}}\}, \and \\ \meaningof{a!E} = \{ P \in \pi | P \equiv Q | x!\langle P' \rangle, x \in \meaningof{a} P' \in \meaningof{E}\} }
\end{mathpar}

\begin{mathpar}
 \inferrule* [lab=nominal] {} {\meaningof{\quotep{E}} = \{ \quotep{P} \in \quotep{\pi} | P \in \meaningof{E} \}, \and \meaningof{\quotep{P}} = \{ \quotep{Q} \in \quotep{\pi} | P \equiv Q \} \and \\ \meaningof{@\quotep{E}} = \{ P \in \pi | P \equiv @x, x \in \meaningof{E} \}}
\end{mathpar}

\begin{eqnarray*}
  \\
  \meaningof{-} : TS \to ST
\end{eqnarray*}

\begin{eqnarray*}
  \\
  L : TS \to ST
\end{eqnarray*}

\begin{eqnarray*}
  \\
  P \models E \iff P \in \meaningof{E}
\end{eqnarray*}

\begin{eqnarray*}
  P \approx_{L} Q \iff \forall E \in L. P \models E \iff Q \models E
\end{eqnarray*}

\begin{eqnarray*}
  P \approx_{K} Q
\end{eqnarray*}

\begin{eqnarray*}
  P \approx Q
\end{eqnarray*}

$\approx_{K} = \approx = \approx_{L}$

\subsubsection{Contextual duality}

Note that contexts extend the quotation operation to a family of
operations from processes to names. Given a context, $M$, we can
define a \emph{nominal context}, $\quotep{M}$ by $\quotep{M}[P] :=
\quotep{M[P]}$. To foreshadow what is to come we observe that these
operations enjoy a duality with processes very much like the duality
between vectors and maps from vectors to scalars.

Further, because the calculus is essentially higher-order, we have a
correspondence between contexts and processes. More specifically,
given a name $x$ and a context $M$ we can construct $M^{*}_{x}$ such
that 

\begin{mathpar}
  M^{*}_{x} | \lift{x}{P} \red M[P]
\end{mathpar}

namely,

\begin{mathpar}
  M^{*}_{x} := x?(u).M[\dropn{u}]
\end{mathpar}

The dependence of $M^{*}_{x}$ on a name makes it an abstraction, 

\begin{mathpar}
  M^{*} := (x)x?(u).M[\dropn{u}]
\end{mathpar}

\subsection{Additional notation}

It will sometimes be convenient to denote the process a name
quotes. We already have the notation $x = \quotep{P}$, but it will be
convenient to introduce an alternate notation, $\procn{x}$, when we
want to emphasize the connection to the use of the name. Note that, by
virtue of name equivalence, $\quotep{\procn{x}} \nameeq x$; so, the
notation is consistent with previous definitions.

Further, because names have structure it is possible to effect
substitutions on the basis of that structure. This means we need to
upgrade our notation for substitutions, which we accomplish by
adapting comprehension notation. Thus,

\begin{mathpar}
  P\{ y / x : x \in S \}
\end{mathpar}

is interpreted to mean the process derived from P by replacing (in a
capture-avoiding manner) each occurrence of $x$ in $S$ by $y$. For example,

\begin{mathpar}
  P\{ \quotep{\procn{x}|\procn{x}} / x : x \in \freenames{P} \}
\end{mathpar}

will replace each (occurrence) of a free name $x$ in $P$ by
$\quotep{\procn{x}|\procn{x}}$.

Also, we will avail ourselves of the notation $x^{L}$ and $x^{R}$ to
denote injections of a name into disjoint copies of the name
space. There are numerous ways to accomplish this. One example can be
found in \cite{MeredithR05}. This notation overloads to vectors of
names: $\vec{x}^{\pi} := (x_{i}^{\pi} \; : \; 0 \leq i < |\vec{x}| )$ where $\pi \in \{L,R\}$.

We also use $P^{\Box} := P|\Box$.

In \cite{MeredithR05} an interpretation of the new operator is
given. It turns out that there are several possible interpretations
all enjoying the requisite algebraic properties of the operator (see
\cite{milner91polyadicpi}). We will therefore make liberal use of
$(\nu\; \vec{x})P$.

% subsection the_syntax_and_semantics_of_the_notation_system (end)   

\input{qm2pi.qmops} 

\input{qm2pi.sterngerlach} 

\input{qm2pi.metric} 

% section concurrent_process_calculi (end)

%\input{qm2pi.proofsketch}

% section proof sketch (end)

%\input{qm2pi.slviaknots} 

% section spatial logic via knots (end)

\input{qm2pi.conclusion}

% section conclusion (end)

%\input{qm2pi.dtcodes} 

% section wiring algorithm (end)

\input{qm2pi.ack} 

% section acknowledgments (end)

\newpage


\bibliographystyle{plain}   
\bibliography{../../biblios/main.bib}

\input{qm2pi.rhodetails}

\end{document}

 

%\ifpdf
%\usepackage[pdftex]{graphicx}
%\else
%\usepackage{graphicx}
%\fi

 % \ifpdf
%  \usepackage{pdfsync}
%  \if


%\title{Brief Article}
%\author{David F. Snyder}
%\author{L.G. Meredith}

%\address{Dept. of Math., Texas State University--San Marcos, San Marcos, TX 78666}
       
\pagestyle{empty}


\begin{document}

\lstset{language=[Objective]Caml,frame=shadowbox}

\documentclass[12pt]{llncs}
%\documentclass{jktr}

\usepackage[pdftex]{hyperref}                   
\usepackage {listings}
\usepackage {mathpartir}
\usepackage{bcprules}
%\usepackage{listings}
                       
\usepackage{graphicx} 
%\usepackage[margins=2.5cm,nohead,nofoot]{geometry}
%\usepackage{geometry}
\usepackage{amsfonts}
\usepackage{amstext}
\usepackage{latexsym}
\usepackage{amssymb}
\usepackage{color}


%\include{myPreamble}
\include{qm2pi.local} 

%\ifpdf
%\usepackage[pdftex]{graphicx}
%\else
%\usepackage{graphicx}
%\fi

 % \ifpdf
%  \usepackage{pdfsync}
%  \if


%\title{Brief Article}
%\author{David F. Snyder}
%\author{L.G. Meredith}

%\address{Dept. of Math., Texas State University--San Marcos, San Marcos, TX 78666}
       
\pagestyle{empty}


\begin{document}

\lstset{language=[Objective]Caml,frame=shadowbox}

\input{qm2pi.front}

% section front matter (end)

\input{qm2pi.intro} 
 
% section introduction (end)

% \input{qm2pi.knotations} 

% section notation (end)

\input{qm2pi.process.calculi} 

% section concurrent_process_calculi_and_spatial_logics_ (end)
    
%\input{qm2pi.knots2pi} 

%\input{qm2pi.trefoil} 

%\input{qm2pi.mainthm} 

% subsection basic_interpretation (end)

%\input{qm2pi.rho.presentation} 
\subsection{The syntax and semantics of the notation system}\label{sub:the_syntax_and_semantics_of_the_notation_system} % (fold)

We now summarize a technical presentation of the calculus that
embodies our theory of dynamics. The typical presentation of such a
calculus follows the style of giving generators and relations on
them. The grammar, below, describing term constructors, freely
generates the set of processes, $\Proc$. This set is then quotiented
by a relation known as structural congruence and it is over this set
that the notion of dynamics is expressed. This presentation is
essentially that of \cite{MeredithR05} with the addition of
polyadicity and summation. For readability we have relegated some of
the technical subtleties to an appendix.

\subsubsection{Process grammar}\label{subsub:process_grammar}

\begin{mathpar}
  \inferrule* [lab=synchronization] {} {{M} \bc \pzero \;|\; x?F \;|\; x!C }
  \and
  \inferrule* [lab=abstraction] {} {{F} \bc (x)P}
  \and
  \inferrule* [lab=concretion] {} {{C} \bc \langle Q \rangle}
  \and
  \inferrule* [lab=process] {} {{P,Q} \bc M \;| \;P|Q \;|\; @{x}}
  \and
  \inferrule* [lab=name] {} {{x} \bc \quotep{P}}
\end{mathpar} 

Note that $\vec{x}$ (resp. $\vec{P}$) denotes a vector of names
(resp. processes) of length $|\vec{x}|$ (resp. $|\vec{P}|$). We adopt
the following useful abbreviations.

\begin{mathpar}
   x?(\vec{y}).P := x.(\vec{y})P \and  x\clift{\vec{P}} := x.\clift{\vec{P}}
   \and x!(y) := \lift{x}{\dropn{y}}
   \and \Pi_{i=0}^{n-1}P_i := P_0 | \ldots | P_{n-1}
\end{mathpar}

\subsubsection{Structural congruence}

\paragraph{Free and bound names and alpha-equivalence.} At the
core of structural equivalence is alpha-equivalence which identifies
process that are the same up to a change of variable. Formally, we
recognize the distinction between free and bound names. The free names
of a process, $\freenames{P}$, may be calculated recursively as
follows:

\begin{mathpar}
\freenames{\pzero} := \emptyset
  \and \\
  \freenames{x?(y).P} := \{ x \} \cup (\freenames{P} \setminus \{ y \})
  \and 
  \freenames{x!\langle P \rangle} := \{ x \} \cup \{ P \} 
  \and \\
  \freenames{P|Q} := \freenames{P} \cup \freenames{Q}
  \and \\
  \freenames{@{x}} := \{ x \}
\end{mathpar}

$\pi$
$\quotep{\pi}$

$\freenames{-} : \pi \to \mathcal{P}(\quotep{\pi})$

\begin{eqnarray*}
  \freenames{\pzero} & := & \emptyset \\
  \freenames{x?(y).P} & := & \{ x \} \cup (\freenames{P} \setminus \{ y \}) \\
  \freenames{x!\langle P \rangle} & := & \{ x \} \cup \{ P \} \\
  \freenames{P|Q} & := & \freenames{P} \cup \freenames{Q} \\
  \freenames{\dropn{x}} & := & \{ x \}
\end{eqnarray*}

The bound names of a process, $\boundnames{P}$, are those names occurring in $P$
that are not free. For example, in $x?(y).0$, the name $x$ is free, while $y$ is bound.

\begin{mathpar}
  \inferrule* [lab=monoidal-laws] {} { P|Q \equiv Q|P \and P|0 \equiv P \and P|(Q|R) \equiv (P|Q)|R }
\end{mathpar}

\begin{mathpar}
  \inferrule* [lab=alpha-equivalence] {} { (x)P \equiv (y)P\{y/x\} \and y \not\in \freenames{P} }
\end{mathpar}

\begin{definition}
Then two processes, $P,Q$, are alpha-equivalent if $P = Q\{\vec{y}/\vec{x}\}$ for
some $\vec{x} \in \boundnames{Q},\vec{y} \in \boundnames{P}$, where $Q\{\vec{y}/\vec{x}\}$
denotes the capture-avoiding substitution of $\vec{y}$ for $\vec{x}$ in $Q$.
\end{definition}

\begin{definition}
  The {\em structural congruence} \cite{SangiorgiWalker} , $\equiv$,
  between processes is the least congruence containing
  alpha-equivalence, satisfying the abelian monoid laws
  (associativity, commutativity and $\pzero$ as identity) for parallel
  composition $|$ and for summation $+$.
\end{definition}

\subsection{Name equivalence}

We take name equivalence, written $\nameeq$, to be the smallest
equivalence relation generated by the following rules.

\begin{mathpar}
\inferrule*[lab=Quote-drop]
{ }
{ \quotep{@{x}} \nameeq x }

\inferrule*[lab=Struct-equiv]
{ P \scong Q }
{ \quotep{P} \nameeq \quotep{Q} }
\end{mathpar}

The astute reader will have noticed that the mutual recursion of names
and processes imposes a mutual recursion on alpha-equivalence and
structural equivalence via name-equivalence. Fortunately, all of this
works out pleasantly and we may calculate in the natural way, free of
concern. The reader interested in the details is referred to the
appendix \ref{appendix:rho_details}.

\subsection{Substitution}

We use $\Proc$ for the set of processes, $\QProc$ for the set of
names, and $\id{\{}\vec{y} / \vec{x} \id{\}}$ to denote partial maps,
$s : \QProc \rightarrow \QProc$. A map, $s$ lifts, uniquely, to a map
on process terms, $\widehat{s} : \Proc \rightarrow \Proc$ by the
following equations.

\begin{mathpar}
  (0) \psubstp{Q}{P} := 0 \\
  (R \juxtap S) \psubstp{Q}{P}
  :=    
  (R)\psubstp{Q}{P} \juxtap (S) \psubstp{Q}{P} \\
  (x?(y).R) \psubstp{Q}{P}    
  :=    
  (x)\substp{Q}{P} (z)\concat( (R \psubstn{z}{y}) \psubstp{Q}{P} ) \\
  (\lift{x}{R}) \psubstp{Q}{P}  
  :=
  \lift{(x)\substp{Q}{P}}{ R \psubstp{Q}{P} } \\
%   (\dropn{x})  \psubstp{Q}{P}       
%   := 
%   \left\{ 
%     \begin{array}{ccc} 
%       \dropn{\quotep{Q}} & & x \nameeq \quotep{P} \\
%       \dropn{x} & & otherwise \\
%     \end{array}
%   \right. 
  (\dropn{x})  \psubstp{Q}{P}       
  := 
  \left\{ 
    \begin{array}{ccc} 
      Q & & x \nameeq \quotep{P} \\
      \dropn{x} & & otherwise \\
    \end{array}
  \right.
\end{mathpar}
 

where

\begin{eqnarray}
  (x)\id{\{} \lpquote Q \rpquote / \lpquote P \rpquote \id{\}}            = 
  \left\{ 
    \begin{array}{ccc}
      \lpquote Q \rpquote & & x \nameeq \lpquote P \rpquote \\
      x & & otherwise \\
    \end{array}
  \right. \nonumber
\end{eqnarray}

and $z$ is chosen distinct from $\quotep{P}$, $\quotep{Q}$, the free
names in $Q$, and all the names in $R$. Our $\alpha$-equivalence will
be built in the standard way from this substitution.

\begin{remark}\label{rem:no_self_referential_names}
  One consequence of these definitions is that $\forall P. \quotep{P}
  \not\in \freenames{P}$.
\end{remark}

\subsection{ Dynamic quote: an example }

Anticipating something of what's to come, consider applying the
substitution, $\widehat{\id{\{}u / z \id{\}}}$, to the following pair
of processes, $\lift{w}{y!(z)}$ and $w[ \lpquote y!(z) \rpquote ]$.

\begin{eqnarray}
	\lift{w}{y!(z)}\widehat{\id{\{}u / z \id{\}}}
		& = &
		\lift{w}{y!(u)} \nonumber\\
	w[ \lpquote y!(z) \rpquote ] \widehat{ \id{\{}u / z \id{\}} }
		& = &
		w[ \lpquote y!(z) \rpquote ] \nonumber
\end{eqnarray}

Because the body of the process between quotes is impervious to
substitution, we get radically different answers. In fact, by
examining the first process in an input context,
e.g. $x?(z).\lift{w}{y!(z)}$, we see that the process under the lift
operator may be shaped by prefixed inputs binding a name inside it. In
this sense, the lift operator will be seen as a way to dynamically
construct processes before reifying them as names.

Finally equipped with these standard features we can present the
dynamics of the calculus.

\subsubsection{Operational semantics} 

Finally, we introduce the computational dynamics. What marks these
algebras as distinct from other more traditionally studied algebraic
structures, e.g. vector spaces or polynomial rings, is the manner in
which dynamics is captured. In traditional structures, dynamics is typically
expressed through morphisms between such structures, as in linear maps
between vector spaces or morphisms between rings. In algebras
associated with the semantics of computation, the dynamics is
expressed as part of the algebraic structure itself, through a
reduction reduction relation typically denoted by $\red$. Below, we
give a recursive presentation of this relation for the calculus used
in the encoding.

$\red \subseteq \pi \times \pi$
$\red : \pi \to \mathcal{P}(\pi)$

\begin{mathpar}
  \inferrule* [lab=Comm] { \textsf{match}( x_{src}, x_{trgt} ) } { x_{trgt}?(y)P \; | \; x_{src}!\langle {Q} \rangle \red P\{\quotep{Q}/y}\} }
  \and \\
  \inferrule* [lab=Par] {{P} \red {P}'} {{{P} | {Q}} \red {{P}' | {Q}}}
  \and
  \inferrule* [lab=Equiv]{{{P} \scong {P}'} \andalso {{P}' \red {Q}'} \andalso {{Q}' \scong {Q}}}{{P} \red {Q}}
\end{mathpar}

\begin{eqnarray*}
  match_{\equiv} (\quotep{P},\quotep{Q}) & := & P \equiv Q \\
  match_{\dagger}(\quotep{P},\quotep{Q}) & := & \forall R. P|Q \red^{*} R => R \red^{*} 0 \\
  match_{K}(\quotep{P},\quotep{Q}) & := & K \mbox{ for some context } K
\end{eqnarray*}

$u?(x)P | u!\langle Q \rangle \red P\{\quotep{Q}/x\}$

%We write $\wred$ for $\red^*$, and $P\red$ if $\exists Q $ such that $ P \red Q$.
We write $P\red$ if $\exists Q $ such that $ P \red Q$ and $P\not\red$, otherwise.

\section{Replication}

As mentioned before, it is known that replication (and hence
recursion) can be implemented in a higher-order process algebra
\cite{SangiorgiWalker}. As our first example of calculation with the
machinery thus far presented we give the construction explicitly in
the {\rhoc}.

\begin{eqnarray}
	D_{x} & := & \prefix{x}{y}{(\binpar{\outputp{x}{y}}{@{y}})} \nonumber\\
	\bangp_{x}{P} & := & \binpar{{x}!\langle{\binpar{D_{x}}{P}}\rangle}{D_{x}} \nonumber
\end{eqnarray}

\begin{eqnarray}
	\bangp_{x}{P} & & \nonumber\\
	=
	& {x}!\langle{(\prefix{x}{y}{(\outputp{x}{y} | @{y})) | P}}\rangle 
	      | \prefix{x}{y}{(\outputp{x}{y} | @{y})} & \nonumber\\
	\red
	& (\outputp{x}{y} | @{y})\substn{\quotep{(\prefix{x}{y}{(@{y} | \outputp{x}{y})) | P}}}{y} & \nonumber\\
	=
	& \outputp{x}{\quotep{(\prefix{x}{y}{(\outputp{x}{y} | @{y})) | P}}}
	  | {(\prefix{x}{y}{(\outputp{x}{y} | @{y})) | P}} & \nonumber\\
	\red
	& \ldots & \nonumber\\
	\red^*
	& P | P | \ldots & \nonumber
\end{eqnarray}

Of course, this encoding, as an implementation, runs away, unfolding
$\bangp{P}$ eagerly. A lazier and more implementable replication
operator, restricted to input-guarded processes, may be obtained as follows.

\begin{eqnarray}
\bangp{\prefix{u}{v}{P}} 
	:= 
	\binpar{\lift{x}{\prefix{u}{v}{(\binpar{D(x)}{P})}}}{D(x)} \nonumber
\end{eqnarray}

\begin{remark}
  Note that the lazier definition still does not deal with summation
  or mixed summation (i.e. sums over input and output). The reader is
  invited to construct definitions of replication that deal with these
  features. 

  Further, the definitions are parameterized in a name, $x$. Can you,
  gentle reader, make a definition that eliminates this parameter and
  guarantees no accidental interaction between the replication
  machinery and the process being replicated -- i.e. no accidental
  sharing of names used by the process to get its work done and the
  name(s) used by the replication to effect copying. This latter
  revision of the definition of replication is crucial to obtaining
  the expected identity $!!P \sim !P$.
\end{remark}

\begin{remark}\label{rem:paradoxical_combinator}
  The reader familiar with the lambda calculus will have noticed the
  similarity between $D$ and the paradoxical combinator.

  [Ed. note: the existence of this seems to suggest we have to be more
  restrictive on the set of processes and names we admit if we are to
  support no-cloning.]
\end{remark}

\subsubsection{Bisimulation}

The computational dynamics gives rise to another kind of equivalence,
the equivalence of computational behavior. As previously mentioned
this is typically captured \emph{via} some form of bisimulation.

% The notion we use in this paper is weak barbed bisimulation
% \cite{milner91polyadicpi}.

The notion we use in this paper is derived from weak barbed
bisimulation \cite{milner91polyadicpi}. 

\begin{definition}
An \emph{observation relation}, $\downarrow_{\mathcal N}$, over a set
of names, $\mathcal N$, is the smallest relation satisfying the rules
below.

\infrule[Out-barb]{y \in {\mathcal N}, \; x \nameeq y}
		  {\outputp{x}{v} \downarrow_{\mathcal N} x}
\infrule[Par-barb]{\mbox{$P\downarrow_{\mathcal N} x$ or $Q\downarrow_{\mathcal N} x$}}
		  {\binpar{P}{Q} \downarrow_{\mathcal N} x}

We write $P \Downarrow_{\mathcal N} x$ if there is $Q$ such that 
$P \wred Q$ and $Q \downarrow_{\mathcal N} x$.
\end{definition}

\begin{definition}
%\label{def.bbisim}
An  ${\mathcal N}$-\emph{barbed bisimulation} over a set of names, ${\mathcal N}$, is a symmetric binary relation 
${\mathcal S}_{\mathcal N}$ between agents such that $P\rel{S}_{\mathcal N}Q$ implies:
\begin{enumerate}
\item If $P \red P'$ then $Q \wred Q'$ and $P'\rel{S}_{\mathcal N} Q'$.
\item If $P\downarrow_{\mathcal N} x$, then $Q\Downarrow_{\mathcal N} x$.
\end{enumerate}
$P$ is ${\mathcal N}$-barbed bisimilar to $Q$, written
$P \wbbisim_{\mathcal N} Q$, if $P \rel{S}_{\mathcal N} Q$ for some ${\mathcal N}$-barbed bisimulation ${\mathcal S}_{\mathcal N}$.
\end{definition}

$\mathcal{R} \subseteq \pi \times \pi$

$P \mathcal{R} Q => \forall P'. P \red P' \Rightarrow \exists Q'. Q \red Q', P' \mathcal{R} Q'$

$P \vdash x \Rightarrow Q \vdash x$

\begin{mathpar}
  \inferrule*[lab=Out-barb]{x \nameeq y}{{y}!\langle{Q}\rangle \vdash x}
  \and
  \inferrule*[lab=Par-barb]{\mbox{$P\vdash x$ or $Q\vdash x$}}{\binpar{P}{Q} \vdash x}
\end{mathpar}

\subsubsection{Contexts}

One of the principle advantages of computational calculi like the
$\pi$-calculus is a well-defined notion of context,
contextual-equivalence and a correlation between
contextual-equivalence and notions of bisimulation. The notion of
context allows the decomposition of a process into (sub-)process and
its syntactic environment, its context. Thus, a context may be
thought of as a process with a ``hole'' (written $\Box$) in it. The
application of a context $M$ to a process $P$, written $M[P]$, is
tantamount to filling the hole in $M$ with $P$. In this paper we do
not need the full weight of this theory, but do make use of the notion
of context in the proof the main theorem. 

\begin{mathpar}
  \inferrule* [lab=summation] {} {{M_{M},M_{N}} \bc \Box \;|\; x.M_{A} \;|\; M_{M}+M_{N}}
  \and
  \inferrule* [lab=agent] {} {{M_{A}} \bc (\vec{x})M_{P} \;| \; \clift{P_0,\ldots,M_{P},\ldots,P_N}}
  \and \\
  \inferrule* [lab=process] {} {{M_{P}} \bc M_{N} \;| \;P|M_{P} }
\end{mathpar} 

\begin{mathpar}
  \inferrule* [lab=sychronization] {} {M_{N} \bc \Box \;|\; x?M_{F} \;|\; x!M_{C}}
  \and
  \inferrule* [lab=abstraction] {} {{M_{F}} \bc (x)M_{P} }
  \and
  \inferrule* [lab=concretion] {} {{M_{C}} \bc \langle M_{P} \rangle }
  \and \\
  \inferrule* [lab=process] {} {{M_{P}} \bc M_{N} \;| \;P|M_{P} }
\end{mathpar}

\begin{definition}[contextual application] Given a context $M$, and
  process $P$, we define the \emph{contextual application}, $M[P] :=
  M\{P/\Box\}$. That is, the contextual application of M to P is the
  substitution of $P$ for $\Box$ in $M$.
\end{definition}

$\meaningof{-} : L \to \mathcal{P}(\pi)$

\begin{mathpar}
  \inferrule* [lab=collection] {} {\meaningof{true} = \pi, \and \meaningof{~E} = \pi \setminus \meaningof{E}, \and \meaningof{E_{1} \& E_{2}} = \meaningof{E_{1}} \cap \meaningof{E_{2}}}
\end{mathpar}

\begin{mathpar}
  \inferrule* [lab=structure] {} {\meaningof{0} = \{ P \in \pi | P \equiv 0 \}, \and \\ \meaningof{E_1 | E_2} = \{ P \in \pi | P \equiv P_{1} | P_{2}, P_{1} \in \meaningof{E_{1}}, P_{2} \in \meaningof{E_2}\} }
\end{mathpar}

\begin{mathpar}
 \inferrule* [lab=behavior] {} {\meaningof{\langle a?b \rangle E} = \{ P \in \pi | P \equiv Q | u?(y)P', \\ \and \\\\ \and \\ \;\;\; u \in \meaningof{a}, \forall z.P'\{z/y\} \in \meaningof{E\{z/b\}}\}, \and \\ \meaningof{a!E} = \{ P \in \pi | P \equiv Q | x!\langle P' \rangle, x \in \meaningof{a} P' \in \meaningof{E}\} }
\end{mathpar}

\begin{mathpar}
 \inferrule* [lab=nominal] {} {\meaningof{\quotep{E}} = \{ \quotep{P} \in \quotep{\pi} | P \in \meaningof{E} \}, \and \meaningof{\quotep{P}} = \{ \quotep{Q} \in \quotep{\pi} | P \equiv Q \} \and \\ \meaningof{@\quotep{E}} = \{ P \in \pi | P \equiv @x, x \in \meaningof{E} \}}
\end{mathpar}

\begin{eqnarray*}
  \\
  \meaningof{-} : TS \to ST
\end{eqnarray*}

\begin{eqnarray*}
  \\
  L : TS \to ST
\end{eqnarray*}

\begin{eqnarray*}
  \\
  P \models E \iff P \in \meaningof{E}
\end{eqnarray*}

\begin{eqnarray*}
  P \approx_{L} Q \iff \forall E \in L. P \models E \iff Q \models E
\end{eqnarray*}

\begin{eqnarray*}
  P \approx_{K} Q
\end{eqnarray*}

\begin{eqnarray*}
  P \approx Q
\end{eqnarray*}

$\approx_{K} = \approx = \approx_{L}$

\subsubsection{Contextual duality}

Note that contexts extend the quotation operation to a family of
operations from processes to names. Given a context, $M$, we can
define a \emph{nominal context}, $\quotep{M}$ by $\quotep{M}[P] :=
\quotep{M[P]}$. To foreshadow what is to come we observe that these
operations enjoy a duality with processes very much like the duality
between vectors and maps from vectors to scalars.

Further, because the calculus is essentially higher-order, we have a
correspondence between contexts and processes. More specifically,
given a name $x$ and a context $M$ we can construct $M^{*}_{x}$ such
that 

\begin{mathpar}
  M^{*}_{x} | \lift{x}{P} \red M[P]
\end{mathpar}

namely,

\begin{mathpar}
  M^{*}_{x} := x?(u).M[\dropn{u}]
\end{mathpar}

The dependence of $M^{*}_{x}$ on a name makes it an abstraction, 

\begin{mathpar}
  M^{*} := (x)x?(u).M[\dropn{u}]
\end{mathpar}

\subsection{Additional notation}

It will sometimes be convenient to denote the process a name
quotes. We already have the notation $x = \quotep{P}$, but it will be
convenient to introduce an alternate notation, $\procn{x}$, when we
want to emphasize the connection to the use of the name. Note that, by
virtue of name equivalence, $\quotep{\procn{x}} \nameeq x$; so, the
notation is consistent with previous definitions.

Further, because names have structure it is possible to effect
substitutions on the basis of that structure. This means we need to
upgrade our notation for substitutions, which we accomplish by
adapting comprehension notation. Thus,

\begin{mathpar}
  P\{ y / x : x \in S \}
\end{mathpar}

is interpreted to mean the process derived from P by replacing (in a
capture-avoiding manner) each occurrence of $x$ in $S$ by $y$. For example,

\begin{mathpar}
  P\{ \quotep{\procn{x}|\procn{x}} / x : x \in \freenames{P} \}
\end{mathpar}

will replace each (occurrence) of a free name $x$ in $P$ by
$\quotep{\procn{x}|\procn{x}}$.

Also, we will avail ourselves of the notation $x^{L}$ and $x^{R}$ to
denote injections of a name into disjoint copies of the name
space. There are numerous ways to accomplish this. One example can be
found in \cite{MeredithR05}. This notation overloads to vectors of
names: $\vec{x}^{\pi} := (x_{i}^{\pi} \; : \; 0 \leq i < |\vec{x}| )$ where $\pi \in \{L,R\}$.

We also use $P^{\Box} := P|\Box$.

In \cite{MeredithR05} an interpretation of the new operator is
given. It turns out that there are several possible interpretations
all enjoying the requisite algebraic properties of the operator (see
\cite{milner91polyadicpi}). We will therefore make liberal use of
$(\nu\; \vec{x})P$.

% subsection the_syntax_and_semantics_of_the_notation_system (end)   

\input{qm2pi.qmops} 

\input{qm2pi.sterngerlach} 

\input{qm2pi.metric} 

% section concurrent_process_calculi (end)

%\input{qm2pi.proofsketch}

% section proof sketch (end)

%\input{qm2pi.slviaknots} 

% section spatial logic via knots (end)

\input{qm2pi.conclusion}

% section conclusion (end)

%\input{qm2pi.dtcodes} 

% section wiring algorithm (end)

\input{qm2pi.ack} 

% section acknowledgments (end)

\newpage


\bibliographystyle{plain}   
\bibliography{../../biblios/main.bib}

\input{qm2pi.rhodetails}

\end{document}



% section front matter (end)

\section{Introduction}\label{sec:introduction} % (fold)
In this draft of the material i am going to have to dispense with the
usual writing conventions adopted in papers on these topics. i'm going
to have adopt whatever tone i need at the time i'm writing up the
calculations. Sometimes this may be very conversational; others it may
be the barest mathematical grunts; others still it may be that i have
lifted text from one of my other papers because the exposition of some
point was better said there. i hope that my readers are not unduly put
out by this decision. i'm not doing this to flout convention or be
rebellious. i find these calculations very technically challenging. To
keep everything going technically, something has to give; i have to
let go of some cognitive burden. So, the academic writing style --
with all of its trade-offs in terms of facilitating technical
communication -- is what i'm letting go of. Perhaps subsequent drafts
can be tightened and polished, but for now, i'm going to speak as if
we were sitting together in a coffee shop with a laptop, wifi and a
pad of paper and a pencil.

So, here's what i have to say. We -- you and i, comfortably ensconced
in our coffee shop and well-equipped with our tools -- can realize and
carry out the calculations of quantum mechanics over a very different
formal theory of dynamics, a formal theory of dynamics that
corresponds to a theory of concurrent computation with
\emph{reflection}. It has the advantage that the underlying theory is
already `quantized', but supports analogues all of the continuuous
operations. Strikingly, this underlying theory has recently been
connected with a notion of metric that we can show, by calculating
together, coincides with the metric induced by the inner product.

There are a lot of reasons why you might be interested in seeing
calculations of this form. Here's why i'm interested. For the past
several centuries there has been no competitor to the ``Newtonian''
account of dynamics. As a result the predominant share of accounts of
dynamical systems and situations have had to be formulated in terms of
the Newtonian machinery. i view this as an intellectually dangerous
position to occupy. Everything, despite it's intrinsic shape, turns
into a nail to be hit with this hammer. Recently, however, the theory
of computation has matured to the point where we have candidates for
theories of dynamics that offer very different perspective on
reasoning about dynamical systems and situations. Testing these
candidates against very successful accounts of dynamical situations,
like quantum mechanics, is going to give us some sense of how mature
they are and some measure of the quality of these accounts of
dynamics.

\subsection{Summary of contributions and outline of paper}

So, we're going to develop an interpretation of the operations of
quantum mechanics normally interpreted by Hilbert spaces and
operators. We're going to do this over a theory of computation. Note
that this is very different than the usual quantum computation program
which develops notions of computation over quantum mechanics. Rather,
we are developing a story that aligns with Wheeler's slogan: It from
Bit. To do this we will first provide an account of the theory of
computation at play here. Then we will dive into a calculation-driven
interpretation of the operations of quantum mechanics.

The reason we take this approach is that -- until very recently --
there hasn't been an axiomatic account of quantum mechanics. As a
result there has been no sharp delineation of the mathematical theory
supporting interpretation of the physical theory and the physical
theory, itself. So, ambient features of the maths are free to be
exploited (or supressed) without a real accounting of their physical
relevance. There is no sharp statement ``here's the physical theory''
qua \emph{theory} and ``here's the mathematical interpretation''
enabling a judgment of how faithful the interpretation is -- apart
from experimental observation. When there is an axiomatic account we
can judge how well a given mathematical formalism supports an
interpretation of the axioms, independent of
experimentation. Likewise, we can judge how well we have captured our
physical evidence and experience with our axiomatics, independent of
any specific mathematical implementation, with accidental detail that
may or may not have physical significance. 

In lieu of a fully fleshed out and vetted axiomatic account of quantum
mechanics, interpreting the operational notions in service of modeling
physical systems will have to suffice. In other words, we are not in
the business of providing a model of Hilbert spaces and operators. We
are in the business of providing a model of quantum mechanics because
we are motivated by testing our notions of dynamics against physical
theory; and, the predictive calculations of the physical theory must
serve as the best formulation -- shy of a fully fleshed out axiomatic
account -- of the physical theory itself (as they have for scientific
theories since time immemorial). Put another way, despite a
whole-hearted commitment to an It-from-Bit ontology, we are firmly
aligned with the shut-up-and-calculate camp as the best way to obtain
results either from the physical perspective or as a quality assurance
measure of our fledgling theory of dynamics.

In detail, we present a reflective process calculus. Then we develop
intuitive correspondences between the notions available in this
calculus and the usual physical notions supporting quantum mechanical
calculations. Thus, 

\begin{table}[htp]
  \center{
    \fbox{
      \begin{tabular}{c|c}
        quantum mechanics & process calculus \\
        \hline
        scalar & name \\
        state vector & process \\
        dual & contextual duals \\
        matrix & formal sums of process-context-dual pairs \\
        orthogonality & process annihilation \\
        inner product & execution-formula + quoting
      \end{tabular}
    }
  }
  \caption{QM - process calculi correspondences}
\end{table}

Then we tighten up these intuitions to operational definitions. We
employ the Dirac notation as the best proxy we can find for an
abstract syntax of the quantum mechanical notions. The definitions we
develop put us in contact with equational constraints coming from the
theory that we demonstrate the definitions and calculations satisfy.

This puts us in a position to shut up and calculate for the
Stern-Gerlach experimental set up, showing how these predictive
calculations become calculations on processes in our theory of a
reflective process calculus.

Penultimately, we demonstrate that the notion of metric coming from
the inner product coincides with the notion of metric available from
the theory of bisimulation. This demonstration gives us the right to
think of space as arising from behavior. Finally, we consider where we
might go from the new vantage point we have obtained.

% section introduction (end) 
 
% section introduction (end)

% \documentclass[12pt]{llncs}
%\documentclass{jktr}

\usepackage[pdftex]{hyperref}                   
\usepackage {listings}
\usepackage {mathpartir}
\usepackage{bcprules}
%\usepackage{listings}
                       
\usepackage{graphicx} 
%\usepackage[margins=2.5cm,nohead,nofoot]{geometry}
%\usepackage{geometry}
\usepackage{amsfonts}
\usepackage{amstext}
\usepackage{latexsym}
\usepackage{amssymb}
\usepackage{color}


%\include{myPreamble}
\include{qm2pi.local} 

%\ifpdf
%\usepackage[pdftex]{graphicx}
%\else
%\usepackage{graphicx}
%\fi

 % \ifpdf
%  \usepackage{pdfsync}
%  \if


%\title{Brief Article}
%\author{David F. Snyder}
%\author{L.G. Meredith}

%\address{Dept. of Math., Texas State University--San Marcos, San Marcos, TX 78666}
       
\pagestyle{empty}


\begin{document}

\lstset{language=[Objective]Caml,frame=shadowbox}

\input{qm2pi.front}

% section front matter (end)

\input{qm2pi.intro} 
 
% section introduction (end)

% \input{qm2pi.knotations} 

% section notation (end)

\input{qm2pi.process.calculi} 

% section concurrent_process_calculi_and_spatial_logics_ (end)
    
%\input{qm2pi.knots2pi} 

%\input{qm2pi.trefoil} 

%\input{qm2pi.mainthm} 

% subsection basic_interpretation (end)

%\input{qm2pi.rho.presentation} 
\subsection{The syntax and semantics of the notation system}\label{sub:the_syntax_and_semantics_of_the_notation_system} % (fold)

We now summarize a technical presentation of the calculus that
embodies our theory of dynamics. The typical presentation of such a
calculus follows the style of giving generators and relations on
them. The grammar, below, describing term constructors, freely
generates the set of processes, $\Proc$. This set is then quotiented
by a relation known as structural congruence and it is over this set
that the notion of dynamics is expressed. This presentation is
essentially that of \cite{MeredithR05} with the addition of
polyadicity and summation. For readability we have relegated some of
the technical subtleties to an appendix.

\subsubsection{Process grammar}\label{subsub:process_grammar}

\begin{mathpar}
  \inferrule* [lab=synchronization] {} {{M} \bc \pzero \;|\; x?F \;|\; x!C }
  \and
  \inferrule* [lab=abstraction] {} {{F} \bc (x)P}
  \and
  \inferrule* [lab=concretion] {} {{C} \bc \langle Q \rangle}
  \and
  \inferrule* [lab=process] {} {{P,Q} \bc M \;| \;P|Q \;|\; @{x}}
  \and
  \inferrule* [lab=name] {} {{x} \bc \quotep{P}}
\end{mathpar} 

Note that $\vec{x}$ (resp. $\vec{P}$) denotes a vector of names
(resp. processes) of length $|\vec{x}|$ (resp. $|\vec{P}|$). We adopt
the following useful abbreviations.

\begin{mathpar}
   x?(\vec{y}).P := x.(\vec{y})P \and  x\clift{\vec{P}} := x.\clift{\vec{P}}
   \and x!(y) := \lift{x}{\dropn{y}}
   \and \Pi_{i=0}^{n-1}P_i := P_0 | \ldots | P_{n-1}
\end{mathpar}

\subsubsection{Structural congruence}

\paragraph{Free and bound names and alpha-equivalence.} At the
core of structural equivalence is alpha-equivalence which identifies
process that are the same up to a change of variable. Formally, we
recognize the distinction between free and bound names. The free names
of a process, $\freenames{P}$, may be calculated recursively as
follows:

\begin{mathpar}
\freenames{\pzero} := \emptyset
  \and \\
  \freenames{x?(y).P} := \{ x \} \cup (\freenames{P} \setminus \{ y \})
  \and 
  \freenames{x!\langle P \rangle} := \{ x \} \cup \{ P \} 
  \and \\
  \freenames{P|Q} := \freenames{P} \cup \freenames{Q}
  \and \\
  \freenames{@{x}} := \{ x \}
\end{mathpar}

$\pi$
$\quotep{\pi}$

$\freenames{-} : \pi \to \mathcal{P}(\quotep{\pi})$

\begin{eqnarray*}
  \freenames{\pzero} & := & \emptyset \\
  \freenames{x?(y).P} & := & \{ x \} \cup (\freenames{P} \setminus \{ y \}) \\
  \freenames{x!\langle P \rangle} & := & \{ x \} \cup \{ P \} \\
  \freenames{P|Q} & := & \freenames{P} \cup \freenames{Q} \\
  \freenames{\dropn{x}} & := & \{ x \}
\end{eqnarray*}

The bound names of a process, $\boundnames{P}$, are those names occurring in $P$
that are not free. For example, in $x?(y).0$, the name $x$ is free, while $y$ is bound.

\begin{mathpar}
  \inferrule* [lab=monoidal-laws] {} { P|Q \equiv Q|P \and P|0 \equiv P \and P|(Q|R) \equiv (P|Q)|R }
\end{mathpar}

\begin{mathpar}
  \inferrule* [lab=alpha-equivalence] {} { (x)P \equiv (y)P\{y/x\} \and y \not\in \freenames{P} }
\end{mathpar}

\begin{definition}
Then two processes, $P,Q$, are alpha-equivalent if $P = Q\{\vec{y}/\vec{x}\}$ for
some $\vec{x} \in \boundnames{Q},\vec{y} \in \boundnames{P}$, where $Q\{\vec{y}/\vec{x}\}$
denotes the capture-avoiding substitution of $\vec{y}$ for $\vec{x}$ in $Q$.
\end{definition}

\begin{definition}
  The {\em structural congruence} \cite{SangiorgiWalker} , $\equiv$,
  between processes is the least congruence containing
  alpha-equivalence, satisfying the abelian monoid laws
  (associativity, commutativity and $\pzero$ as identity) for parallel
  composition $|$ and for summation $+$.
\end{definition}

\subsection{Name equivalence}

We take name equivalence, written $\nameeq$, to be the smallest
equivalence relation generated by the following rules.

\begin{mathpar}
\inferrule*[lab=Quote-drop]
{ }
{ \quotep{@{x}} \nameeq x }

\inferrule*[lab=Struct-equiv]
{ P \scong Q }
{ \quotep{P} \nameeq \quotep{Q} }
\end{mathpar}

The astute reader will have noticed that the mutual recursion of names
and processes imposes a mutual recursion on alpha-equivalence and
structural equivalence via name-equivalence. Fortunately, all of this
works out pleasantly and we may calculate in the natural way, free of
concern. The reader interested in the details is referred to the
appendix \ref{appendix:rho_details}.

\subsection{Substitution}

We use $\Proc$ for the set of processes, $\QProc$ for the set of
names, and $\id{\{}\vec{y} / \vec{x} \id{\}}$ to denote partial maps,
$s : \QProc \rightarrow \QProc$. A map, $s$ lifts, uniquely, to a map
on process terms, $\widehat{s} : \Proc \rightarrow \Proc$ by the
following equations.

\begin{mathpar}
  (0) \psubstp{Q}{P} := 0 \\
  (R \juxtap S) \psubstp{Q}{P}
  :=    
  (R)\psubstp{Q}{P} \juxtap (S) \psubstp{Q}{P} \\
  (x?(y).R) \psubstp{Q}{P}    
  :=    
  (x)\substp{Q}{P} (z)\concat( (R \psubstn{z}{y}) \psubstp{Q}{P} ) \\
  (\lift{x}{R}) \psubstp{Q}{P}  
  :=
  \lift{(x)\substp{Q}{P}}{ R \psubstp{Q}{P} } \\
%   (\dropn{x})  \psubstp{Q}{P}       
%   := 
%   \left\{ 
%     \begin{array}{ccc} 
%       \dropn{\quotep{Q}} & & x \nameeq \quotep{P} \\
%       \dropn{x} & & otherwise \\
%     \end{array}
%   \right. 
  (\dropn{x})  \psubstp{Q}{P}       
  := 
  \left\{ 
    \begin{array}{ccc} 
      Q & & x \nameeq \quotep{P} \\
      \dropn{x} & & otherwise \\
    \end{array}
  \right.
\end{mathpar}
 

where

\begin{eqnarray}
  (x)\id{\{} \lpquote Q \rpquote / \lpquote P \rpquote \id{\}}            = 
  \left\{ 
    \begin{array}{ccc}
      \lpquote Q \rpquote & & x \nameeq \lpquote P \rpquote \\
      x & & otherwise \\
    \end{array}
  \right. \nonumber
\end{eqnarray}

and $z$ is chosen distinct from $\quotep{P}$, $\quotep{Q}$, the free
names in $Q$, and all the names in $R$. Our $\alpha$-equivalence will
be built in the standard way from this substitution.

\begin{remark}\label{rem:no_self_referential_names}
  One consequence of these definitions is that $\forall P. \quotep{P}
  \not\in \freenames{P}$.
\end{remark}

\subsection{ Dynamic quote: an example }

Anticipating something of what's to come, consider applying the
substitution, $\widehat{\id{\{}u / z \id{\}}}$, to the following pair
of processes, $\lift{w}{y!(z)}$ and $w[ \lpquote y!(z) \rpquote ]$.

\begin{eqnarray}
	\lift{w}{y!(z)}\widehat{\id{\{}u / z \id{\}}}
		& = &
		\lift{w}{y!(u)} \nonumber\\
	w[ \lpquote y!(z) \rpquote ] \widehat{ \id{\{}u / z \id{\}} }
		& = &
		w[ \lpquote y!(z) \rpquote ] \nonumber
\end{eqnarray}

Because the body of the process between quotes is impervious to
substitution, we get radically different answers. In fact, by
examining the first process in an input context,
e.g. $x?(z).\lift{w}{y!(z)}$, we see that the process under the lift
operator may be shaped by prefixed inputs binding a name inside it. In
this sense, the lift operator will be seen as a way to dynamically
construct processes before reifying them as names.

Finally equipped with these standard features we can present the
dynamics of the calculus.

\subsubsection{Operational semantics} 

Finally, we introduce the computational dynamics. What marks these
algebras as distinct from other more traditionally studied algebraic
structures, e.g. vector spaces or polynomial rings, is the manner in
which dynamics is captured. In traditional structures, dynamics is typically
expressed through morphisms between such structures, as in linear maps
between vector spaces or morphisms between rings. In algebras
associated with the semantics of computation, the dynamics is
expressed as part of the algebraic structure itself, through a
reduction reduction relation typically denoted by $\red$. Below, we
give a recursive presentation of this relation for the calculus used
in the encoding.

$\red \subseteq \pi \times \pi$
$\red : \pi \to \mathcal{P}(\pi)$

\begin{mathpar}
  \inferrule* [lab=Comm] { \textsf{match}( x_{src}, x_{trgt} ) } { x_{trgt}?(y)P \; | \; x_{src}!\langle {Q} \rangle \red P\{\quotep{Q}/y}\} }
  \and \\
  \inferrule* [lab=Par] {{P} \red {P}'} {{{P} | {Q}} \red {{P}' | {Q}}}
  \and
  \inferrule* [lab=Equiv]{{{P} \scong {P}'} \andalso {{P}' \red {Q}'} \andalso {{Q}' \scong {Q}}}{{P} \red {Q}}
\end{mathpar}

\begin{eqnarray*}
  match_{\equiv} (\quotep{P},\quotep{Q}) & := & P \equiv Q \\
  match_{\dagger}(\quotep{P},\quotep{Q}) & := & \forall R. P|Q \red^{*} R => R \red^{*} 0 \\
  match_{K}(\quotep{P},\quotep{Q}) & := & K \mbox{ for some context } K
\end{eqnarray*}

$u?(x)P | u!\langle Q \rangle \red P\{\quotep{Q}/x\}$

%We write $\wred$ for $\red^*$, and $P\red$ if $\exists Q $ such that $ P \red Q$.
We write $P\red$ if $\exists Q $ such that $ P \red Q$ and $P\not\red$, otherwise.

\section{Replication}

As mentioned before, it is known that replication (and hence
recursion) can be implemented in a higher-order process algebra
\cite{SangiorgiWalker}. As our first example of calculation with the
machinery thus far presented we give the construction explicitly in
the {\rhoc}.

\begin{eqnarray}
	D_{x} & := & \prefix{x}{y}{(\binpar{\outputp{x}{y}}{@{y}})} \nonumber\\
	\bangp_{x}{P} & := & \binpar{{x}!\langle{\binpar{D_{x}}{P}}\rangle}{D_{x}} \nonumber
\end{eqnarray}

\begin{eqnarray}
	\bangp_{x}{P} & & \nonumber\\
	=
	& {x}!\langle{(\prefix{x}{y}{(\outputp{x}{y} | @{y})) | P}}\rangle 
	      | \prefix{x}{y}{(\outputp{x}{y} | @{y})} & \nonumber\\
	\red
	& (\outputp{x}{y} | @{y})\substn{\quotep{(\prefix{x}{y}{(@{y} | \outputp{x}{y})) | P}}}{y} & \nonumber\\
	=
	& \outputp{x}{\quotep{(\prefix{x}{y}{(\outputp{x}{y} | @{y})) | P}}}
	  | {(\prefix{x}{y}{(\outputp{x}{y} | @{y})) | P}} & \nonumber\\
	\red
	& \ldots & \nonumber\\
	\red^*
	& P | P | \ldots & \nonumber
\end{eqnarray}

Of course, this encoding, as an implementation, runs away, unfolding
$\bangp{P}$ eagerly. A lazier and more implementable replication
operator, restricted to input-guarded processes, may be obtained as follows.

\begin{eqnarray}
\bangp{\prefix{u}{v}{P}} 
	:= 
	\binpar{\lift{x}{\prefix{u}{v}{(\binpar{D(x)}{P})}}}{D(x)} \nonumber
\end{eqnarray}

\begin{remark}
  Note that the lazier definition still does not deal with summation
  or mixed summation (i.e. sums over input and output). The reader is
  invited to construct definitions of replication that deal with these
  features. 

  Further, the definitions are parameterized in a name, $x$. Can you,
  gentle reader, make a definition that eliminates this parameter and
  guarantees no accidental interaction between the replication
  machinery and the process being replicated -- i.e. no accidental
  sharing of names used by the process to get its work done and the
  name(s) used by the replication to effect copying. This latter
  revision of the definition of replication is crucial to obtaining
  the expected identity $!!P \sim !P$.
\end{remark}

\begin{remark}\label{rem:paradoxical_combinator}
  The reader familiar with the lambda calculus will have noticed the
  similarity between $D$ and the paradoxical combinator.

  [Ed. note: the existence of this seems to suggest we have to be more
  restrictive on the set of processes and names we admit if we are to
  support no-cloning.]
\end{remark}

\subsubsection{Bisimulation}

The computational dynamics gives rise to another kind of equivalence,
the equivalence of computational behavior. As previously mentioned
this is typically captured \emph{via} some form of bisimulation.

% The notion we use in this paper is weak barbed bisimulation
% \cite{milner91polyadicpi}.

The notion we use in this paper is derived from weak barbed
bisimulation \cite{milner91polyadicpi}. 

\begin{definition}
An \emph{observation relation}, $\downarrow_{\mathcal N}$, over a set
of names, $\mathcal N$, is the smallest relation satisfying the rules
below.

\infrule[Out-barb]{y \in {\mathcal N}, \; x \nameeq y}
		  {\outputp{x}{v} \downarrow_{\mathcal N} x}
\infrule[Par-barb]{\mbox{$P\downarrow_{\mathcal N} x$ or $Q\downarrow_{\mathcal N} x$}}
		  {\binpar{P}{Q} \downarrow_{\mathcal N} x}

We write $P \Downarrow_{\mathcal N} x$ if there is $Q$ such that 
$P \wred Q$ and $Q \downarrow_{\mathcal N} x$.
\end{definition}

\begin{definition}
%\label{def.bbisim}
An  ${\mathcal N}$-\emph{barbed bisimulation} over a set of names, ${\mathcal N}$, is a symmetric binary relation 
${\mathcal S}_{\mathcal N}$ between agents such that $P\rel{S}_{\mathcal N}Q$ implies:
\begin{enumerate}
\item If $P \red P'$ then $Q \wred Q'$ and $P'\rel{S}_{\mathcal N} Q'$.
\item If $P\downarrow_{\mathcal N} x$, then $Q\Downarrow_{\mathcal N} x$.
\end{enumerate}
$P$ is ${\mathcal N}$-barbed bisimilar to $Q$, written
$P \wbbisim_{\mathcal N} Q$, if $P \rel{S}_{\mathcal N} Q$ for some ${\mathcal N}$-barbed bisimulation ${\mathcal S}_{\mathcal N}$.
\end{definition}

$\mathcal{R} \subseteq \pi \times \pi$

$P \mathcal{R} Q => \forall P'. P \red P' \Rightarrow \exists Q'. Q \red Q', P' \mathcal{R} Q'$

$P \vdash x \Rightarrow Q \vdash x$

\begin{mathpar}
  \inferrule*[lab=Out-barb]{x \nameeq y}{{y}!\langle{Q}\rangle \vdash x}
  \and
  \inferrule*[lab=Par-barb]{\mbox{$P\vdash x$ or $Q\vdash x$}}{\binpar{P}{Q} \vdash x}
\end{mathpar}

\subsubsection{Contexts}

One of the principle advantages of computational calculi like the
$\pi$-calculus is a well-defined notion of context,
contextual-equivalence and a correlation between
contextual-equivalence and notions of bisimulation. The notion of
context allows the decomposition of a process into (sub-)process and
its syntactic environment, its context. Thus, a context may be
thought of as a process with a ``hole'' (written $\Box$) in it. The
application of a context $M$ to a process $P$, written $M[P]$, is
tantamount to filling the hole in $M$ with $P$. In this paper we do
not need the full weight of this theory, but do make use of the notion
of context in the proof the main theorem. 

\begin{mathpar}
  \inferrule* [lab=summation] {} {{M_{M},M_{N}} \bc \Box \;|\; x.M_{A} \;|\; M_{M}+M_{N}}
  \and
  \inferrule* [lab=agent] {} {{M_{A}} \bc (\vec{x})M_{P} \;| \; \clift{P_0,\ldots,M_{P},\ldots,P_N}}
  \and \\
  \inferrule* [lab=process] {} {{M_{P}} \bc M_{N} \;| \;P|M_{P} }
\end{mathpar} 

\begin{mathpar}
  \inferrule* [lab=sychronization] {} {M_{N} \bc \Box \;|\; x?M_{F} \;|\; x!M_{C}}
  \and
  \inferrule* [lab=abstraction] {} {{M_{F}} \bc (x)M_{P} }
  \and
  \inferrule* [lab=concretion] {} {{M_{C}} \bc \langle M_{P} \rangle }
  \and \\
  \inferrule* [lab=process] {} {{M_{P}} \bc M_{N} \;| \;P|M_{P} }
\end{mathpar}

\begin{definition}[contextual application] Given a context $M$, and
  process $P$, we define the \emph{contextual application}, $M[P] :=
  M\{P/\Box\}$. That is, the contextual application of M to P is the
  substitution of $P$ for $\Box$ in $M$.
\end{definition}

$\meaningof{-} : L \to \mathcal{P}(\pi)$

\begin{mathpar}
  \inferrule* [lab=collection] {} {\meaningof{true} = \pi, \and \meaningof{~E} = \pi \setminus \meaningof{E}, \and \meaningof{E_{1} \& E_{2}} = \meaningof{E_{1}} \cap \meaningof{E_{2}}}
\end{mathpar}

\begin{mathpar}
  \inferrule* [lab=structure] {} {\meaningof{0} = \{ P \in \pi | P \equiv 0 \}, \and \\ \meaningof{E_1 | E_2} = \{ P \in \pi | P \equiv P_{1} | P_{2}, P_{1} \in \meaningof{E_{1}}, P_{2} \in \meaningof{E_2}\} }
\end{mathpar}

\begin{mathpar}
 \inferrule* [lab=behavior] {} {\meaningof{\langle a?b \rangle E} = \{ P \in \pi | P \equiv Q | u?(y)P', \\ \and \\\\ \and \\ \;\;\; u \in \meaningof{a}, \forall z.P'\{z/y\} \in \meaningof{E\{z/b\}}\}, \and \\ \meaningof{a!E} = \{ P \in \pi | P \equiv Q | x!\langle P' \rangle, x \in \meaningof{a} P' \in \meaningof{E}\} }
\end{mathpar}

\begin{mathpar}
 \inferrule* [lab=nominal] {} {\meaningof{\quotep{E}} = \{ \quotep{P} \in \quotep{\pi} | P \in \meaningof{E} \}, \and \meaningof{\quotep{P}} = \{ \quotep{Q} \in \quotep{\pi} | P \equiv Q \} \and \\ \meaningof{@\quotep{E}} = \{ P \in \pi | P \equiv @x, x \in \meaningof{E} \}}
\end{mathpar}

\begin{eqnarray*}
  \\
  \meaningof{-} : TS \to ST
\end{eqnarray*}

\begin{eqnarray*}
  \\
  L : TS \to ST
\end{eqnarray*}

\begin{eqnarray*}
  \\
  P \models E \iff P \in \meaningof{E}
\end{eqnarray*}

\begin{eqnarray*}
  P \approx_{L} Q \iff \forall E \in L. P \models E \iff Q \models E
\end{eqnarray*}

\begin{eqnarray*}
  P \approx_{K} Q
\end{eqnarray*}

\begin{eqnarray*}
  P \approx Q
\end{eqnarray*}

$\approx_{K} = \approx = \approx_{L}$

\subsubsection{Contextual duality}

Note that contexts extend the quotation operation to a family of
operations from processes to names. Given a context, $M$, we can
define a \emph{nominal context}, $\quotep{M}$ by $\quotep{M}[P] :=
\quotep{M[P]}$. To foreshadow what is to come we observe that these
operations enjoy a duality with processes very much like the duality
between vectors and maps from vectors to scalars.

Further, because the calculus is essentially higher-order, we have a
correspondence between contexts and processes. More specifically,
given a name $x$ and a context $M$ we can construct $M^{*}_{x}$ such
that 

\begin{mathpar}
  M^{*}_{x} | \lift{x}{P} \red M[P]
\end{mathpar}

namely,

\begin{mathpar}
  M^{*}_{x} := x?(u).M[\dropn{u}]
\end{mathpar}

The dependence of $M^{*}_{x}$ on a name makes it an abstraction, 

\begin{mathpar}
  M^{*} := (x)x?(u).M[\dropn{u}]
\end{mathpar}

\subsection{Additional notation}

It will sometimes be convenient to denote the process a name
quotes. We already have the notation $x = \quotep{P}$, but it will be
convenient to introduce an alternate notation, $\procn{x}$, when we
want to emphasize the connection to the use of the name. Note that, by
virtue of name equivalence, $\quotep{\procn{x}} \nameeq x$; so, the
notation is consistent with previous definitions.

Further, because names have structure it is possible to effect
substitutions on the basis of that structure. This means we need to
upgrade our notation for substitutions, which we accomplish by
adapting comprehension notation. Thus,

\begin{mathpar}
  P\{ y / x : x \in S \}
\end{mathpar}

is interpreted to mean the process derived from P by replacing (in a
capture-avoiding manner) each occurrence of $x$ in $S$ by $y$. For example,

\begin{mathpar}
  P\{ \quotep{\procn{x}|\procn{x}} / x : x \in \freenames{P} \}
\end{mathpar}

will replace each (occurrence) of a free name $x$ in $P$ by
$\quotep{\procn{x}|\procn{x}}$.

Also, we will avail ourselves of the notation $x^{L}$ and $x^{R}$ to
denote injections of a name into disjoint copies of the name
space. There are numerous ways to accomplish this. One example can be
found in \cite{MeredithR05}. This notation overloads to vectors of
names: $\vec{x}^{\pi} := (x_{i}^{\pi} \; : \; 0 \leq i < |\vec{x}| )$ where $\pi \in \{L,R\}$.

We also use $P^{\Box} := P|\Box$.

In \cite{MeredithR05} an interpretation of the new operator is
given. It turns out that there are several possible interpretations
all enjoying the requisite algebraic properties of the operator (see
\cite{milner91polyadicpi}). We will therefore make liberal use of
$(\nu\; \vec{x})P$.

% subsection the_syntax_and_semantics_of_the_notation_system (end)   

\input{qm2pi.qmops} 

\input{qm2pi.sterngerlach} 

\input{qm2pi.metric} 

% section concurrent_process_calculi (end)

%\input{qm2pi.proofsketch}

% section proof sketch (end)

%\input{qm2pi.slviaknots} 

% section spatial logic via knots (end)

\input{qm2pi.conclusion}

% section conclusion (end)

%\input{qm2pi.dtcodes} 

% section wiring algorithm (end)

\input{qm2pi.ack} 

% section acknowledgments (end)

\newpage


\bibliographystyle{plain}   
\bibliography{../../biblios/main.bib}

\input{qm2pi.rhodetails}

\end{document}

 

% section notation (end)

\input{qm2pi.process.calculi} 

% section concurrent_process_calculi_and_spatial_logics_ (end)
    
%\documentclass[12pt]{llncs}
%\documentclass{jktr}

\usepackage[pdftex]{hyperref}                   
\usepackage {listings}
\usepackage {mathpartir}
\usepackage{bcprules}
%\usepackage{listings}
                       
\usepackage{graphicx} 
%\usepackage[margins=2.5cm,nohead,nofoot]{geometry}
%\usepackage{geometry}
\usepackage{amsfonts}
\usepackage{amstext}
\usepackage{latexsym}
\usepackage{amssymb}
\usepackage{color}


%\include{myPreamble}
\include{qm2pi.local} 

%\ifpdf
%\usepackage[pdftex]{graphicx}
%\else
%\usepackage{graphicx}
%\fi

 % \ifpdf
%  \usepackage{pdfsync}
%  \if


%\title{Brief Article}
%\author{David F. Snyder}
%\author{L.G. Meredith}

%\address{Dept. of Math., Texas State University--San Marcos, San Marcos, TX 78666}
       
\pagestyle{empty}


\begin{document}

\lstset{language=[Objective]Caml,frame=shadowbox}

\input{qm2pi.front}

% section front matter (end)

\input{qm2pi.intro} 
 
% section introduction (end)

% \input{qm2pi.knotations} 

% section notation (end)

\input{qm2pi.process.calculi} 

% section concurrent_process_calculi_and_spatial_logics_ (end)
    
%\input{qm2pi.knots2pi} 

%\input{qm2pi.trefoil} 

%\input{qm2pi.mainthm} 

% subsection basic_interpretation (end)

%\input{qm2pi.rho.presentation} 
\subsection{The syntax and semantics of the notation system}\label{sub:the_syntax_and_semantics_of_the_notation_system} % (fold)

We now summarize a technical presentation of the calculus that
embodies our theory of dynamics. The typical presentation of such a
calculus follows the style of giving generators and relations on
them. The grammar, below, describing term constructors, freely
generates the set of processes, $\Proc$. This set is then quotiented
by a relation known as structural congruence and it is over this set
that the notion of dynamics is expressed. This presentation is
essentially that of \cite{MeredithR05} with the addition of
polyadicity and summation. For readability we have relegated some of
the technical subtleties to an appendix.

\subsubsection{Process grammar}\label{subsub:process_grammar}

\begin{mathpar}
  \inferrule* [lab=synchronization] {} {{M} \bc \pzero \;|\; x?F \;|\; x!C }
  \and
  \inferrule* [lab=abstraction] {} {{F} \bc (x)P}
  \and
  \inferrule* [lab=concretion] {} {{C} \bc \langle Q \rangle}
  \and
  \inferrule* [lab=process] {} {{P,Q} \bc M \;| \;P|Q \;|\; @{x}}
  \and
  \inferrule* [lab=name] {} {{x} \bc \quotep{P}}
\end{mathpar} 

Note that $\vec{x}$ (resp. $\vec{P}$) denotes a vector of names
(resp. processes) of length $|\vec{x}|$ (resp. $|\vec{P}|$). We adopt
the following useful abbreviations.

\begin{mathpar}
   x?(\vec{y}).P := x.(\vec{y})P \and  x\clift{\vec{P}} := x.\clift{\vec{P}}
   \and x!(y) := \lift{x}{\dropn{y}}
   \and \Pi_{i=0}^{n-1}P_i := P_0 | \ldots | P_{n-1}
\end{mathpar}

\subsubsection{Structural congruence}

\paragraph{Free and bound names and alpha-equivalence.} At the
core of structural equivalence is alpha-equivalence which identifies
process that are the same up to a change of variable. Formally, we
recognize the distinction between free and bound names. The free names
of a process, $\freenames{P}$, may be calculated recursively as
follows:

\begin{mathpar}
\freenames{\pzero} := \emptyset
  \and \\
  \freenames{x?(y).P} := \{ x \} \cup (\freenames{P} \setminus \{ y \})
  \and 
  \freenames{x!\langle P \rangle} := \{ x \} \cup \{ P \} 
  \and \\
  \freenames{P|Q} := \freenames{P} \cup \freenames{Q}
  \and \\
  \freenames{@{x}} := \{ x \}
\end{mathpar}

$\pi$
$\quotep{\pi}$

$\freenames{-} : \pi \to \mathcal{P}(\quotep{\pi})$

\begin{eqnarray*}
  \freenames{\pzero} & := & \emptyset \\
  \freenames{x?(y).P} & := & \{ x \} \cup (\freenames{P} \setminus \{ y \}) \\
  \freenames{x!\langle P \rangle} & := & \{ x \} \cup \{ P \} \\
  \freenames{P|Q} & := & \freenames{P} \cup \freenames{Q} \\
  \freenames{\dropn{x}} & := & \{ x \}
\end{eqnarray*}

The bound names of a process, $\boundnames{P}$, are those names occurring in $P$
that are not free. For example, in $x?(y).0$, the name $x$ is free, while $y$ is bound.

\begin{mathpar}
  \inferrule* [lab=monoidal-laws] {} { P|Q \equiv Q|P \and P|0 \equiv P \and P|(Q|R) \equiv (P|Q)|R }
\end{mathpar}

\begin{mathpar}
  \inferrule* [lab=alpha-equivalence] {} { (x)P \equiv (y)P\{y/x\} \and y \not\in \freenames{P} }
\end{mathpar}

\begin{definition}
Then two processes, $P,Q$, are alpha-equivalent if $P = Q\{\vec{y}/\vec{x}\}$ for
some $\vec{x} \in \boundnames{Q},\vec{y} \in \boundnames{P}$, where $Q\{\vec{y}/\vec{x}\}$
denotes the capture-avoiding substitution of $\vec{y}$ for $\vec{x}$ in $Q$.
\end{definition}

\begin{definition}
  The {\em structural congruence} \cite{SangiorgiWalker} , $\equiv$,
  between processes is the least congruence containing
  alpha-equivalence, satisfying the abelian monoid laws
  (associativity, commutativity and $\pzero$ as identity) for parallel
  composition $|$ and for summation $+$.
\end{definition}

\subsection{Name equivalence}

We take name equivalence, written $\nameeq$, to be the smallest
equivalence relation generated by the following rules.

\begin{mathpar}
\inferrule*[lab=Quote-drop]
{ }
{ \quotep{@{x}} \nameeq x }

\inferrule*[lab=Struct-equiv]
{ P \scong Q }
{ \quotep{P} \nameeq \quotep{Q} }
\end{mathpar}

The astute reader will have noticed that the mutual recursion of names
and processes imposes a mutual recursion on alpha-equivalence and
structural equivalence via name-equivalence. Fortunately, all of this
works out pleasantly and we may calculate in the natural way, free of
concern. The reader interested in the details is referred to the
appendix \ref{appendix:rho_details}.

\subsection{Substitution}

We use $\Proc$ for the set of processes, $\QProc$ for the set of
names, and $\id{\{}\vec{y} / \vec{x} \id{\}}$ to denote partial maps,
$s : \QProc \rightarrow \QProc$. A map, $s$ lifts, uniquely, to a map
on process terms, $\widehat{s} : \Proc \rightarrow \Proc$ by the
following equations.

\begin{mathpar}
  (0) \psubstp{Q}{P} := 0 \\
  (R \juxtap S) \psubstp{Q}{P}
  :=    
  (R)\psubstp{Q}{P} \juxtap (S) \psubstp{Q}{P} \\
  (x?(y).R) \psubstp{Q}{P}    
  :=    
  (x)\substp{Q}{P} (z)\concat( (R \psubstn{z}{y}) \psubstp{Q}{P} ) \\
  (\lift{x}{R}) \psubstp{Q}{P}  
  :=
  \lift{(x)\substp{Q}{P}}{ R \psubstp{Q}{P} } \\
%   (\dropn{x})  \psubstp{Q}{P}       
%   := 
%   \left\{ 
%     \begin{array}{ccc} 
%       \dropn{\quotep{Q}} & & x \nameeq \quotep{P} \\
%       \dropn{x} & & otherwise \\
%     \end{array}
%   \right. 
  (\dropn{x})  \psubstp{Q}{P}       
  := 
  \left\{ 
    \begin{array}{ccc} 
      Q & & x \nameeq \quotep{P} \\
      \dropn{x} & & otherwise \\
    \end{array}
  \right.
\end{mathpar}
 

where

\begin{eqnarray}
  (x)\id{\{} \lpquote Q \rpquote / \lpquote P \rpquote \id{\}}            = 
  \left\{ 
    \begin{array}{ccc}
      \lpquote Q \rpquote & & x \nameeq \lpquote P \rpquote \\
      x & & otherwise \\
    \end{array}
  \right. \nonumber
\end{eqnarray}

and $z$ is chosen distinct from $\quotep{P}$, $\quotep{Q}$, the free
names in $Q$, and all the names in $R$. Our $\alpha$-equivalence will
be built in the standard way from this substitution.

\begin{remark}\label{rem:no_self_referential_names}
  One consequence of these definitions is that $\forall P. \quotep{P}
  \not\in \freenames{P}$.
\end{remark}

\subsection{ Dynamic quote: an example }

Anticipating something of what's to come, consider applying the
substitution, $\widehat{\id{\{}u / z \id{\}}}$, to the following pair
of processes, $\lift{w}{y!(z)}$ and $w[ \lpquote y!(z) \rpquote ]$.

\begin{eqnarray}
	\lift{w}{y!(z)}\widehat{\id{\{}u / z \id{\}}}
		& = &
		\lift{w}{y!(u)} \nonumber\\
	w[ \lpquote y!(z) \rpquote ] \widehat{ \id{\{}u / z \id{\}} }
		& = &
		w[ \lpquote y!(z) \rpquote ] \nonumber
\end{eqnarray}

Because the body of the process between quotes is impervious to
substitution, we get radically different answers. In fact, by
examining the first process in an input context,
e.g. $x?(z).\lift{w}{y!(z)}$, we see that the process under the lift
operator may be shaped by prefixed inputs binding a name inside it. In
this sense, the lift operator will be seen as a way to dynamically
construct processes before reifying them as names.

Finally equipped with these standard features we can present the
dynamics of the calculus.

\subsubsection{Operational semantics} 

Finally, we introduce the computational dynamics. What marks these
algebras as distinct from other more traditionally studied algebraic
structures, e.g. vector spaces or polynomial rings, is the manner in
which dynamics is captured. In traditional structures, dynamics is typically
expressed through morphisms between such structures, as in linear maps
between vector spaces or morphisms between rings. In algebras
associated with the semantics of computation, the dynamics is
expressed as part of the algebraic structure itself, through a
reduction reduction relation typically denoted by $\red$. Below, we
give a recursive presentation of this relation for the calculus used
in the encoding.

$\red \subseteq \pi \times \pi$
$\red : \pi \to \mathcal{P}(\pi)$

\begin{mathpar}
  \inferrule* [lab=Comm] { \textsf{match}( x_{src}, x_{trgt} ) } { x_{trgt}?(y)P \; | \; x_{src}!\langle {Q} \rangle \red P\{\quotep{Q}/y}\} }
  \and \\
  \inferrule* [lab=Par] {{P} \red {P}'} {{{P} | {Q}} \red {{P}' | {Q}}}
  \and
  \inferrule* [lab=Equiv]{{{P} \scong {P}'} \andalso {{P}' \red {Q}'} \andalso {{Q}' \scong {Q}}}{{P} \red {Q}}
\end{mathpar}

\begin{eqnarray*}
  match_{\equiv} (\quotep{P},\quotep{Q}) & := & P \equiv Q \\
  match_{\dagger}(\quotep{P},\quotep{Q}) & := & \forall R. P|Q \red^{*} R => R \red^{*} 0 \\
  match_{K}(\quotep{P},\quotep{Q}) & := & K \mbox{ for some context } K
\end{eqnarray*}

$u?(x)P | u!\langle Q \rangle \red P\{\quotep{Q}/x\}$

%We write $\wred$ for $\red^*$, and $P\red$ if $\exists Q $ such that $ P \red Q$.
We write $P\red$ if $\exists Q $ such that $ P \red Q$ and $P\not\red$, otherwise.

\section{Replication}

As mentioned before, it is known that replication (and hence
recursion) can be implemented in a higher-order process algebra
\cite{SangiorgiWalker}. As our first example of calculation with the
machinery thus far presented we give the construction explicitly in
the {\rhoc}.

\begin{eqnarray}
	D_{x} & := & \prefix{x}{y}{(\binpar{\outputp{x}{y}}{@{y}})} \nonumber\\
	\bangp_{x}{P} & := & \binpar{{x}!\langle{\binpar{D_{x}}{P}}\rangle}{D_{x}} \nonumber
\end{eqnarray}

\begin{eqnarray}
	\bangp_{x}{P} & & \nonumber\\
	=
	& {x}!\langle{(\prefix{x}{y}{(\outputp{x}{y} | @{y})) | P}}\rangle 
	      | \prefix{x}{y}{(\outputp{x}{y} | @{y})} & \nonumber\\
	\red
	& (\outputp{x}{y} | @{y})\substn{\quotep{(\prefix{x}{y}{(@{y} | \outputp{x}{y})) | P}}}{y} & \nonumber\\
	=
	& \outputp{x}{\quotep{(\prefix{x}{y}{(\outputp{x}{y} | @{y})) | P}}}
	  | {(\prefix{x}{y}{(\outputp{x}{y} | @{y})) | P}} & \nonumber\\
	\red
	& \ldots & \nonumber\\
	\red^*
	& P | P | \ldots & \nonumber
\end{eqnarray}

Of course, this encoding, as an implementation, runs away, unfolding
$\bangp{P}$ eagerly. A lazier and more implementable replication
operator, restricted to input-guarded processes, may be obtained as follows.

\begin{eqnarray}
\bangp{\prefix{u}{v}{P}} 
	:= 
	\binpar{\lift{x}{\prefix{u}{v}{(\binpar{D(x)}{P})}}}{D(x)} \nonumber
\end{eqnarray}

\begin{remark}
  Note that the lazier definition still does not deal with summation
  or mixed summation (i.e. sums over input and output). The reader is
  invited to construct definitions of replication that deal with these
  features. 

  Further, the definitions are parameterized in a name, $x$. Can you,
  gentle reader, make a definition that eliminates this parameter and
  guarantees no accidental interaction between the replication
  machinery and the process being replicated -- i.e. no accidental
  sharing of names used by the process to get its work done and the
  name(s) used by the replication to effect copying. This latter
  revision of the definition of replication is crucial to obtaining
  the expected identity $!!P \sim !P$.
\end{remark}

\begin{remark}\label{rem:paradoxical_combinator}
  The reader familiar with the lambda calculus will have noticed the
  similarity between $D$ and the paradoxical combinator.

  [Ed. note: the existence of this seems to suggest we have to be more
  restrictive on the set of processes and names we admit if we are to
  support no-cloning.]
\end{remark}

\subsubsection{Bisimulation}

The computational dynamics gives rise to another kind of equivalence,
the equivalence of computational behavior. As previously mentioned
this is typically captured \emph{via} some form of bisimulation.

% The notion we use in this paper is weak barbed bisimulation
% \cite{milner91polyadicpi}.

The notion we use in this paper is derived from weak barbed
bisimulation \cite{milner91polyadicpi}. 

\begin{definition}
An \emph{observation relation}, $\downarrow_{\mathcal N}$, over a set
of names, $\mathcal N$, is the smallest relation satisfying the rules
below.

\infrule[Out-barb]{y \in {\mathcal N}, \; x \nameeq y}
		  {\outputp{x}{v} \downarrow_{\mathcal N} x}
\infrule[Par-barb]{\mbox{$P\downarrow_{\mathcal N} x$ or $Q\downarrow_{\mathcal N} x$}}
		  {\binpar{P}{Q} \downarrow_{\mathcal N} x}

We write $P \Downarrow_{\mathcal N} x$ if there is $Q$ such that 
$P \wred Q$ and $Q \downarrow_{\mathcal N} x$.
\end{definition}

\begin{definition}
%\label{def.bbisim}
An  ${\mathcal N}$-\emph{barbed bisimulation} over a set of names, ${\mathcal N}$, is a symmetric binary relation 
${\mathcal S}_{\mathcal N}$ between agents such that $P\rel{S}_{\mathcal N}Q$ implies:
\begin{enumerate}
\item If $P \red P'$ then $Q \wred Q'$ and $P'\rel{S}_{\mathcal N} Q'$.
\item If $P\downarrow_{\mathcal N} x$, then $Q\Downarrow_{\mathcal N} x$.
\end{enumerate}
$P$ is ${\mathcal N}$-barbed bisimilar to $Q$, written
$P \wbbisim_{\mathcal N} Q$, if $P \rel{S}_{\mathcal N} Q$ for some ${\mathcal N}$-barbed bisimulation ${\mathcal S}_{\mathcal N}$.
\end{definition}

$\mathcal{R} \subseteq \pi \times \pi$

$P \mathcal{R} Q => \forall P'. P \red P' \Rightarrow \exists Q'. Q \red Q', P' \mathcal{R} Q'$

$P \vdash x \Rightarrow Q \vdash x$

\begin{mathpar}
  \inferrule*[lab=Out-barb]{x \nameeq y}{{y}!\langle{Q}\rangle \vdash x}
  \and
  \inferrule*[lab=Par-barb]{\mbox{$P\vdash x$ or $Q\vdash x$}}{\binpar{P}{Q} \vdash x}
\end{mathpar}

\subsubsection{Contexts}

One of the principle advantages of computational calculi like the
$\pi$-calculus is a well-defined notion of context,
contextual-equivalence and a correlation between
contextual-equivalence and notions of bisimulation. The notion of
context allows the decomposition of a process into (sub-)process and
its syntactic environment, its context. Thus, a context may be
thought of as a process with a ``hole'' (written $\Box$) in it. The
application of a context $M$ to a process $P$, written $M[P]$, is
tantamount to filling the hole in $M$ with $P$. In this paper we do
not need the full weight of this theory, but do make use of the notion
of context in the proof the main theorem. 

\begin{mathpar}
  \inferrule* [lab=summation] {} {{M_{M},M_{N}} \bc \Box \;|\; x.M_{A} \;|\; M_{M}+M_{N}}
  \and
  \inferrule* [lab=agent] {} {{M_{A}} \bc (\vec{x})M_{P} \;| \; \clift{P_0,\ldots,M_{P},\ldots,P_N}}
  \and \\
  \inferrule* [lab=process] {} {{M_{P}} \bc M_{N} \;| \;P|M_{P} }
\end{mathpar} 

\begin{mathpar}
  \inferrule* [lab=sychronization] {} {M_{N} \bc \Box \;|\; x?M_{F} \;|\; x!M_{C}}
  \and
  \inferrule* [lab=abstraction] {} {{M_{F}} \bc (x)M_{P} }
  \and
  \inferrule* [lab=concretion] {} {{M_{C}} \bc \langle M_{P} \rangle }
  \and \\
  \inferrule* [lab=process] {} {{M_{P}} \bc M_{N} \;| \;P|M_{P} }
\end{mathpar}

\begin{definition}[contextual application] Given a context $M$, and
  process $P$, we define the \emph{contextual application}, $M[P] :=
  M\{P/\Box\}$. That is, the contextual application of M to P is the
  substitution of $P$ for $\Box$ in $M$.
\end{definition}

$\meaningof{-} : L \to \mathcal{P}(\pi)$

\begin{mathpar}
  \inferrule* [lab=collection] {} {\meaningof{true} = \pi, \and \meaningof{~E} = \pi \setminus \meaningof{E}, \and \meaningof{E_{1} \& E_{2}} = \meaningof{E_{1}} \cap \meaningof{E_{2}}}
\end{mathpar}

\begin{mathpar}
  \inferrule* [lab=structure] {} {\meaningof{0} = \{ P \in \pi | P \equiv 0 \}, \and \\ \meaningof{E_1 | E_2} = \{ P \in \pi | P \equiv P_{1} | P_{2}, P_{1} \in \meaningof{E_{1}}, P_{2} \in \meaningof{E_2}\} }
\end{mathpar}

\begin{mathpar}
 \inferrule* [lab=behavior] {} {\meaningof{\langle a?b \rangle E} = \{ P \in \pi | P \equiv Q | u?(y)P', \\ \and \\\\ \and \\ \;\;\; u \in \meaningof{a}, \forall z.P'\{z/y\} \in \meaningof{E\{z/b\}}\}, \and \\ \meaningof{a!E} = \{ P \in \pi | P \equiv Q | x!\langle P' \rangle, x \in \meaningof{a} P' \in \meaningof{E}\} }
\end{mathpar}

\begin{mathpar}
 \inferrule* [lab=nominal] {} {\meaningof{\quotep{E}} = \{ \quotep{P} \in \quotep{\pi} | P \in \meaningof{E} \}, \and \meaningof{\quotep{P}} = \{ \quotep{Q} \in \quotep{\pi} | P \equiv Q \} \and \\ \meaningof{@\quotep{E}} = \{ P \in \pi | P \equiv @x, x \in \meaningof{E} \}}
\end{mathpar}

\begin{eqnarray*}
  \\
  \meaningof{-} : TS \to ST
\end{eqnarray*}

\begin{eqnarray*}
  \\
  L : TS \to ST
\end{eqnarray*}

\begin{eqnarray*}
  \\
  P \models E \iff P \in \meaningof{E}
\end{eqnarray*}

\begin{eqnarray*}
  P \approx_{L} Q \iff \forall E \in L. P \models E \iff Q \models E
\end{eqnarray*}

\begin{eqnarray*}
  P \approx_{K} Q
\end{eqnarray*}

\begin{eqnarray*}
  P \approx Q
\end{eqnarray*}

$\approx_{K} = \approx = \approx_{L}$

\subsubsection{Contextual duality}

Note that contexts extend the quotation operation to a family of
operations from processes to names. Given a context, $M$, we can
define a \emph{nominal context}, $\quotep{M}$ by $\quotep{M}[P] :=
\quotep{M[P]}$. To foreshadow what is to come we observe that these
operations enjoy a duality with processes very much like the duality
between vectors and maps from vectors to scalars.

Further, because the calculus is essentially higher-order, we have a
correspondence between contexts and processes. More specifically,
given a name $x$ and a context $M$ we can construct $M^{*}_{x}$ such
that 

\begin{mathpar}
  M^{*}_{x} | \lift{x}{P} \red M[P]
\end{mathpar}

namely,

\begin{mathpar}
  M^{*}_{x} := x?(u).M[\dropn{u}]
\end{mathpar}

The dependence of $M^{*}_{x}$ on a name makes it an abstraction, 

\begin{mathpar}
  M^{*} := (x)x?(u).M[\dropn{u}]
\end{mathpar}

\subsection{Additional notation}

It will sometimes be convenient to denote the process a name
quotes. We already have the notation $x = \quotep{P}$, but it will be
convenient to introduce an alternate notation, $\procn{x}$, when we
want to emphasize the connection to the use of the name. Note that, by
virtue of name equivalence, $\quotep{\procn{x}} \nameeq x$; so, the
notation is consistent with previous definitions.

Further, because names have structure it is possible to effect
substitutions on the basis of that structure. This means we need to
upgrade our notation for substitutions, which we accomplish by
adapting comprehension notation. Thus,

\begin{mathpar}
  P\{ y / x : x \in S \}
\end{mathpar}

is interpreted to mean the process derived from P by replacing (in a
capture-avoiding manner) each occurrence of $x$ in $S$ by $y$. For example,

\begin{mathpar}
  P\{ \quotep{\procn{x}|\procn{x}} / x : x \in \freenames{P} \}
\end{mathpar}

will replace each (occurrence) of a free name $x$ in $P$ by
$\quotep{\procn{x}|\procn{x}}$.

Also, we will avail ourselves of the notation $x^{L}$ and $x^{R}$ to
denote injections of a name into disjoint copies of the name
space. There are numerous ways to accomplish this. One example can be
found in \cite{MeredithR05}. This notation overloads to vectors of
names: $\vec{x}^{\pi} := (x_{i}^{\pi} \; : \; 0 \leq i < |\vec{x}| )$ where $\pi \in \{L,R\}$.

We also use $P^{\Box} := P|\Box$.

In \cite{MeredithR05} an interpretation of the new operator is
given. It turns out that there are several possible interpretations
all enjoying the requisite algebraic properties of the operator (see
\cite{milner91polyadicpi}). We will therefore make liberal use of
$(\nu\; \vec{x})P$.

% subsection the_syntax_and_semantics_of_the_notation_system (end)   

\input{qm2pi.qmops} 

\input{qm2pi.sterngerlach} 

\input{qm2pi.metric} 

% section concurrent_process_calculi (end)

%\input{qm2pi.proofsketch}

% section proof sketch (end)

%\input{qm2pi.slviaknots} 

% section spatial logic via knots (end)

\input{qm2pi.conclusion}

% section conclusion (end)

%\input{qm2pi.dtcodes} 

% section wiring algorithm (end)

\input{qm2pi.ack} 

% section acknowledgments (end)

\newpage


\bibliographystyle{plain}   
\bibliography{../../biblios/main.bib}

\input{qm2pi.rhodetails}

\end{document}

 

%\documentclass[12pt]{llncs}
%\documentclass{jktr}

\usepackage[pdftex]{hyperref}                   
\usepackage {listings}
\usepackage {mathpartir}
\usepackage{bcprules}
%\usepackage{listings}
                       
\usepackage{graphicx} 
%\usepackage[margins=2.5cm,nohead,nofoot]{geometry}
%\usepackage{geometry}
\usepackage{amsfonts}
\usepackage{amstext}
\usepackage{latexsym}
\usepackage{amssymb}
\usepackage{color}


%\include{myPreamble}
\include{qm2pi.local} 

%\ifpdf
%\usepackage[pdftex]{graphicx}
%\else
%\usepackage{graphicx}
%\fi

 % \ifpdf
%  \usepackage{pdfsync}
%  \if


%\title{Brief Article}
%\author{David F. Snyder}
%\author{L.G. Meredith}

%\address{Dept. of Math., Texas State University--San Marcos, San Marcos, TX 78666}
       
\pagestyle{empty}


\begin{document}

\lstset{language=[Objective]Caml,frame=shadowbox}

\input{qm2pi.front}

% section front matter (end)

\input{qm2pi.intro} 
 
% section introduction (end)

% \input{qm2pi.knotations} 

% section notation (end)

\input{qm2pi.process.calculi} 

% section concurrent_process_calculi_and_spatial_logics_ (end)
    
%\input{qm2pi.knots2pi} 

%\input{qm2pi.trefoil} 

%\input{qm2pi.mainthm} 

% subsection basic_interpretation (end)

%\input{qm2pi.rho.presentation} 
\subsection{The syntax and semantics of the notation system}\label{sub:the_syntax_and_semantics_of_the_notation_system} % (fold)

We now summarize a technical presentation of the calculus that
embodies our theory of dynamics. The typical presentation of such a
calculus follows the style of giving generators and relations on
them. The grammar, below, describing term constructors, freely
generates the set of processes, $\Proc$. This set is then quotiented
by a relation known as structural congruence and it is over this set
that the notion of dynamics is expressed. This presentation is
essentially that of \cite{MeredithR05} with the addition of
polyadicity and summation. For readability we have relegated some of
the technical subtleties to an appendix.

\subsubsection{Process grammar}\label{subsub:process_grammar}

\begin{mathpar}
  \inferrule* [lab=synchronization] {} {{M} \bc \pzero \;|\; x?F \;|\; x!C }
  \and
  \inferrule* [lab=abstraction] {} {{F} \bc (x)P}
  \and
  \inferrule* [lab=concretion] {} {{C} \bc \langle Q \rangle}
  \and
  \inferrule* [lab=process] {} {{P,Q} \bc M \;| \;P|Q \;|\; @{x}}
  \and
  \inferrule* [lab=name] {} {{x} \bc \quotep{P}}
\end{mathpar} 

Note that $\vec{x}$ (resp. $\vec{P}$) denotes a vector of names
(resp. processes) of length $|\vec{x}|$ (resp. $|\vec{P}|$). We adopt
the following useful abbreviations.

\begin{mathpar}
   x?(\vec{y}).P := x.(\vec{y})P \and  x\clift{\vec{P}} := x.\clift{\vec{P}}
   \and x!(y) := \lift{x}{\dropn{y}}
   \and \Pi_{i=0}^{n-1}P_i := P_0 | \ldots | P_{n-1}
\end{mathpar}

\subsubsection{Structural congruence}

\paragraph{Free and bound names and alpha-equivalence.} At the
core of structural equivalence is alpha-equivalence which identifies
process that are the same up to a change of variable. Formally, we
recognize the distinction between free and bound names. The free names
of a process, $\freenames{P}$, may be calculated recursively as
follows:

\begin{mathpar}
\freenames{\pzero} := \emptyset
  \and \\
  \freenames{x?(y).P} := \{ x \} \cup (\freenames{P} \setminus \{ y \})
  \and 
  \freenames{x!\langle P \rangle} := \{ x \} \cup \{ P \} 
  \and \\
  \freenames{P|Q} := \freenames{P} \cup \freenames{Q}
  \and \\
  \freenames{@{x}} := \{ x \}
\end{mathpar}

$\pi$
$\quotep{\pi}$

$\freenames{-} : \pi \to \mathcal{P}(\quotep{\pi})$

\begin{eqnarray*}
  \freenames{\pzero} & := & \emptyset \\
  \freenames{x?(y).P} & := & \{ x \} \cup (\freenames{P} \setminus \{ y \}) \\
  \freenames{x!\langle P \rangle} & := & \{ x \} \cup \{ P \} \\
  \freenames{P|Q} & := & \freenames{P} \cup \freenames{Q} \\
  \freenames{\dropn{x}} & := & \{ x \}
\end{eqnarray*}

The bound names of a process, $\boundnames{P}$, are those names occurring in $P$
that are not free. For example, in $x?(y).0$, the name $x$ is free, while $y$ is bound.

\begin{mathpar}
  \inferrule* [lab=monoidal-laws] {} { P|Q \equiv Q|P \and P|0 \equiv P \and P|(Q|R) \equiv (P|Q)|R }
\end{mathpar}

\begin{mathpar}
  \inferrule* [lab=alpha-equivalence] {} { (x)P \equiv (y)P\{y/x\} \and y \not\in \freenames{P} }
\end{mathpar}

\begin{definition}
Then two processes, $P,Q$, are alpha-equivalent if $P = Q\{\vec{y}/\vec{x}\}$ for
some $\vec{x} \in \boundnames{Q},\vec{y} \in \boundnames{P}$, where $Q\{\vec{y}/\vec{x}\}$
denotes the capture-avoiding substitution of $\vec{y}$ for $\vec{x}$ in $Q$.
\end{definition}

\begin{definition}
  The {\em structural congruence} \cite{SangiorgiWalker} , $\equiv$,
  between processes is the least congruence containing
  alpha-equivalence, satisfying the abelian monoid laws
  (associativity, commutativity and $\pzero$ as identity) for parallel
  composition $|$ and for summation $+$.
\end{definition}

\subsection{Name equivalence}

We take name equivalence, written $\nameeq$, to be the smallest
equivalence relation generated by the following rules.

\begin{mathpar}
\inferrule*[lab=Quote-drop]
{ }
{ \quotep{@{x}} \nameeq x }

\inferrule*[lab=Struct-equiv]
{ P \scong Q }
{ \quotep{P} \nameeq \quotep{Q} }
\end{mathpar}

The astute reader will have noticed that the mutual recursion of names
and processes imposes a mutual recursion on alpha-equivalence and
structural equivalence via name-equivalence. Fortunately, all of this
works out pleasantly and we may calculate in the natural way, free of
concern. The reader interested in the details is referred to the
appendix \ref{appendix:rho_details}.

\subsection{Substitution}

We use $\Proc$ for the set of processes, $\QProc$ for the set of
names, and $\id{\{}\vec{y} / \vec{x} \id{\}}$ to denote partial maps,
$s : \QProc \rightarrow \QProc$. A map, $s$ lifts, uniquely, to a map
on process terms, $\widehat{s} : \Proc \rightarrow \Proc$ by the
following equations.

\begin{mathpar}
  (0) \psubstp{Q}{P} := 0 \\
  (R \juxtap S) \psubstp{Q}{P}
  :=    
  (R)\psubstp{Q}{P} \juxtap (S) \psubstp{Q}{P} \\
  (x?(y).R) \psubstp{Q}{P}    
  :=    
  (x)\substp{Q}{P} (z)\concat( (R \psubstn{z}{y}) \psubstp{Q}{P} ) \\
  (\lift{x}{R}) \psubstp{Q}{P}  
  :=
  \lift{(x)\substp{Q}{P}}{ R \psubstp{Q}{P} } \\
%   (\dropn{x})  \psubstp{Q}{P}       
%   := 
%   \left\{ 
%     \begin{array}{ccc} 
%       \dropn{\quotep{Q}} & & x \nameeq \quotep{P} \\
%       \dropn{x} & & otherwise \\
%     \end{array}
%   \right. 
  (\dropn{x})  \psubstp{Q}{P}       
  := 
  \left\{ 
    \begin{array}{ccc} 
      Q & & x \nameeq \quotep{P} \\
      \dropn{x} & & otherwise \\
    \end{array}
  \right.
\end{mathpar}
 

where

\begin{eqnarray}
  (x)\id{\{} \lpquote Q \rpquote / \lpquote P \rpquote \id{\}}            = 
  \left\{ 
    \begin{array}{ccc}
      \lpquote Q \rpquote & & x \nameeq \lpquote P \rpquote \\
      x & & otherwise \\
    \end{array}
  \right. \nonumber
\end{eqnarray}

and $z$ is chosen distinct from $\quotep{P}$, $\quotep{Q}$, the free
names in $Q$, and all the names in $R$. Our $\alpha$-equivalence will
be built in the standard way from this substitution.

\begin{remark}\label{rem:no_self_referential_names}
  One consequence of these definitions is that $\forall P. \quotep{P}
  \not\in \freenames{P}$.
\end{remark}

\subsection{ Dynamic quote: an example }

Anticipating something of what's to come, consider applying the
substitution, $\widehat{\id{\{}u / z \id{\}}}$, to the following pair
of processes, $\lift{w}{y!(z)}$ and $w[ \lpquote y!(z) \rpquote ]$.

\begin{eqnarray}
	\lift{w}{y!(z)}\widehat{\id{\{}u / z \id{\}}}
		& = &
		\lift{w}{y!(u)} \nonumber\\
	w[ \lpquote y!(z) \rpquote ] \widehat{ \id{\{}u / z \id{\}} }
		& = &
		w[ \lpquote y!(z) \rpquote ] \nonumber
\end{eqnarray}

Because the body of the process between quotes is impervious to
substitution, we get radically different answers. In fact, by
examining the first process in an input context,
e.g. $x?(z).\lift{w}{y!(z)}$, we see that the process under the lift
operator may be shaped by prefixed inputs binding a name inside it. In
this sense, the lift operator will be seen as a way to dynamically
construct processes before reifying them as names.

Finally equipped with these standard features we can present the
dynamics of the calculus.

\subsubsection{Operational semantics} 

Finally, we introduce the computational dynamics. What marks these
algebras as distinct from other more traditionally studied algebraic
structures, e.g. vector spaces or polynomial rings, is the manner in
which dynamics is captured. In traditional structures, dynamics is typically
expressed through morphisms between such structures, as in linear maps
between vector spaces or morphisms between rings. In algebras
associated with the semantics of computation, the dynamics is
expressed as part of the algebraic structure itself, through a
reduction reduction relation typically denoted by $\red$. Below, we
give a recursive presentation of this relation for the calculus used
in the encoding.

$\red \subseteq \pi \times \pi$
$\red : \pi \to \mathcal{P}(\pi)$

\begin{mathpar}
  \inferrule* [lab=Comm] { \textsf{match}( x_{src}, x_{trgt} ) } { x_{trgt}?(y)P \; | \; x_{src}!\langle {Q} \rangle \red P\{\quotep{Q}/y}\} }
  \and \\
  \inferrule* [lab=Par] {{P} \red {P}'} {{{P} | {Q}} \red {{P}' | {Q}}}
  \and
  \inferrule* [lab=Equiv]{{{P} \scong {P}'} \andalso {{P}' \red {Q}'} \andalso {{Q}' \scong {Q}}}{{P} \red {Q}}
\end{mathpar}

\begin{eqnarray*}
  match_{\equiv} (\quotep{P},\quotep{Q}) & := & P \equiv Q \\
  match_{\dagger}(\quotep{P},\quotep{Q}) & := & \forall R. P|Q \red^{*} R => R \red^{*} 0 \\
  match_{K}(\quotep{P},\quotep{Q}) & := & K \mbox{ for some context } K
\end{eqnarray*}

$u?(x)P | u!\langle Q \rangle \red P\{\quotep{Q}/x\}$

%We write $\wred$ for $\red^*$, and $P\red$ if $\exists Q $ such that $ P \red Q$.
We write $P\red$ if $\exists Q $ such that $ P \red Q$ and $P\not\red$, otherwise.

\section{Replication}

As mentioned before, it is known that replication (and hence
recursion) can be implemented in a higher-order process algebra
\cite{SangiorgiWalker}. As our first example of calculation with the
machinery thus far presented we give the construction explicitly in
the {\rhoc}.

\begin{eqnarray}
	D_{x} & := & \prefix{x}{y}{(\binpar{\outputp{x}{y}}{@{y}})} \nonumber\\
	\bangp_{x}{P} & := & \binpar{{x}!\langle{\binpar{D_{x}}{P}}\rangle}{D_{x}} \nonumber
\end{eqnarray}

\begin{eqnarray}
	\bangp_{x}{P} & & \nonumber\\
	=
	& {x}!\langle{(\prefix{x}{y}{(\outputp{x}{y} | @{y})) | P}}\rangle 
	      | \prefix{x}{y}{(\outputp{x}{y} | @{y})} & \nonumber\\
	\red
	& (\outputp{x}{y} | @{y})\substn{\quotep{(\prefix{x}{y}{(@{y} | \outputp{x}{y})) | P}}}{y} & \nonumber\\
	=
	& \outputp{x}{\quotep{(\prefix{x}{y}{(\outputp{x}{y} | @{y})) | P}}}
	  | {(\prefix{x}{y}{(\outputp{x}{y} | @{y})) | P}} & \nonumber\\
	\red
	& \ldots & \nonumber\\
	\red^*
	& P | P | \ldots & \nonumber
\end{eqnarray}

Of course, this encoding, as an implementation, runs away, unfolding
$\bangp{P}$ eagerly. A lazier and more implementable replication
operator, restricted to input-guarded processes, may be obtained as follows.

\begin{eqnarray}
\bangp{\prefix{u}{v}{P}} 
	:= 
	\binpar{\lift{x}{\prefix{u}{v}{(\binpar{D(x)}{P})}}}{D(x)} \nonumber
\end{eqnarray}

\begin{remark}
  Note that the lazier definition still does not deal with summation
  or mixed summation (i.e. sums over input and output). The reader is
  invited to construct definitions of replication that deal with these
  features. 

  Further, the definitions are parameterized in a name, $x$. Can you,
  gentle reader, make a definition that eliminates this parameter and
  guarantees no accidental interaction between the replication
  machinery and the process being replicated -- i.e. no accidental
  sharing of names used by the process to get its work done and the
  name(s) used by the replication to effect copying. This latter
  revision of the definition of replication is crucial to obtaining
  the expected identity $!!P \sim !P$.
\end{remark}

\begin{remark}\label{rem:paradoxical_combinator}
  The reader familiar with the lambda calculus will have noticed the
  similarity between $D$ and the paradoxical combinator.

  [Ed. note: the existence of this seems to suggest we have to be more
  restrictive on the set of processes and names we admit if we are to
  support no-cloning.]
\end{remark}

\subsubsection{Bisimulation}

The computational dynamics gives rise to another kind of equivalence,
the equivalence of computational behavior. As previously mentioned
this is typically captured \emph{via} some form of bisimulation.

% The notion we use in this paper is weak barbed bisimulation
% \cite{milner91polyadicpi}.

The notion we use in this paper is derived from weak barbed
bisimulation \cite{milner91polyadicpi}. 

\begin{definition}
An \emph{observation relation}, $\downarrow_{\mathcal N}$, over a set
of names, $\mathcal N$, is the smallest relation satisfying the rules
below.

\infrule[Out-barb]{y \in {\mathcal N}, \; x \nameeq y}
		  {\outputp{x}{v} \downarrow_{\mathcal N} x}
\infrule[Par-barb]{\mbox{$P\downarrow_{\mathcal N} x$ or $Q\downarrow_{\mathcal N} x$}}
		  {\binpar{P}{Q} \downarrow_{\mathcal N} x}

We write $P \Downarrow_{\mathcal N} x$ if there is $Q$ such that 
$P \wred Q$ and $Q \downarrow_{\mathcal N} x$.
\end{definition}

\begin{definition}
%\label{def.bbisim}
An  ${\mathcal N}$-\emph{barbed bisimulation} over a set of names, ${\mathcal N}$, is a symmetric binary relation 
${\mathcal S}_{\mathcal N}$ between agents such that $P\rel{S}_{\mathcal N}Q$ implies:
\begin{enumerate}
\item If $P \red P'$ then $Q \wred Q'$ and $P'\rel{S}_{\mathcal N} Q'$.
\item If $P\downarrow_{\mathcal N} x$, then $Q\Downarrow_{\mathcal N} x$.
\end{enumerate}
$P$ is ${\mathcal N}$-barbed bisimilar to $Q$, written
$P \wbbisim_{\mathcal N} Q$, if $P \rel{S}_{\mathcal N} Q$ for some ${\mathcal N}$-barbed bisimulation ${\mathcal S}_{\mathcal N}$.
\end{definition}

$\mathcal{R} \subseteq \pi \times \pi$

$P \mathcal{R} Q => \forall P'. P \red P' \Rightarrow \exists Q'. Q \red Q', P' \mathcal{R} Q'$

$P \vdash x \Rightarrow Q \vdash x$

\begin{mathpar}
  \inferrule*[lab=Out-barb]{x \nameeq y}{{y}!\langle{Q}\rangle \vdash x}
  \and
  \inferrule*[lab=Par-barb]{\mbox{$P\vdash x$ or $Q\vdash x$}}{\binpar{P}{Q} \vdash x}
\end{mathpar}

\subsubsection{Contexts}

One of the principle advantages of computational calculi like the
$\pi$-calculus is a well-defined notion of context,
contextual-equivalence and a correlation between
contextual-equivalence and notions of bisimulation. The notion of
context allows the decomposition of a process into (sub-)process and
its syntactic environment, its context. Thus, a context may be
thought of as a process with a ``hole'' (written $\Box$) in it. The
application of a context $M$ to a process $P$, written $M[P]$, is
tantamount to filling the hole in $M$ with $P$. In this paper we do
not need the full weight of this theory, but do make use of the notion
of context in the proof the main theorem. 

\begin{mathpar}
  \inferrule* [lab=summation] {} {{M_{M},M_{N}} \bc \Box \;|\; x.M_{A} \;|\; M_{M}+M_{N}}
  \and
  \inferrule* [lab=agent] {} {{M_{A}} \bc (\vec{x})M_{P} \;| \; \clift{P_0,\ldots,M_{P},\ldots,P_N}}
  \and \\
  \inferrule* [lab=process] {} {{M_{P}} \bc M_{N} \;| \;P|M_{P} }
\end{mathpar} 

\begin{mathpar}
  \inferrule* [lab=sychronization] {} {M_{N} \bc \Box \;|\; x?M_{F} \;|\; x!M_{C}}
  \and
  \inferrule* [lab=abstraction] {} {{M_{F}} \bc (x)M_{P} }
  \and
  \inferrule* [lab=concretion] {} {{M_{C}} \bc \langle M_{P} \rangle }
  \and \\
  \inferrule* [lab=process] {} {{M_{P}} \bc M_{N} \;| \;P|M_{P} }
\end{mathpar}

\begin{definition}[contextual application] Given a context $M$, and
  process $P$, we define the \emph{contextual application}, $M[P] :=
  M\{P/\Box\}$. That is, the contextual application of M to P is the
  substitution of $P$ for $\Box$ in $M$.
\end{definition}

$\meaningof{-} : L \to \mathcal{P}(\pi)$

\begin{mathpar}
  \inferrule* [lab=collection] {} {\meaningof{true} = \pi, \and \meaningof{~E} = \pi \setminus \meaningof{E}, \and \meaningof{E_{1} \& E_{2}} = \meaningof{E_{1}} \cap \meaningof{E_{2}}}
\end{mathpar}

\begin{mathpar}
  \inferrule* [lab=structure] {} {\meaningof{0} = \{ P \in \pi | P \equiv 0 \}, \and \\ \meaningof{E_1 | E_2} = \{ P \in \pi | P \equiv P_{1} | P_{2}, P_{1} \in \meaningof{E_{1}}, P_{2} \in \meaningof{E_2}\} }
\end{mathpar}

\begin{mathpar}
 \inferrule* [lab=behavior] {} {\meaningof{\langle a?b \rangle E} = \{ P \in \pi | P \equiv Q | u?(y)P', \\ \and \\\\ \and \\ \;\;\; u \in \meaningof{a}, \forall z.P'\{z/y\} \in \meaningof{E\{z/b\}}\}, \and \\ \meaningof{a!E} = \{ P \in \pi | P \equiv Q | x!\langle P' \rangle, x \in \meaningof{a} P' \in \meaningof{E}\} }
\end{mathpar}

\begin{mathpar}
 \inferrule* [lab=nominal] {} {\meaningof{\quotep{E}} = \{ \quotep{P} \in \quotep{\pi} | P \in \meaningof{E} \}, \and \meaningof{\quotep{P}} = \{ \quotep{Q} \in \quotep{\pi} | P \equiv Q \} \and \\ \meaningof{@\quotep{E}} = \{ P \in \pi | P \equiv @x, x \in \meaningof{E} \}}
\end{mathpar}

\begin{eqnarray*}
  \\
  \meaningof{-} : TS \to ST
\end{eqnarray*}

\begin{eqnarray*}
  \\
  L : TS \to ST
\end{eqnarray*}

\begin{eqnarray*}
  \\
  P \models E \iff P \in \meaningof{E}
\end{eqnarray*}

\begin{eqnarray*}
  P \approx_{L} Q \iff \forall E \in L. P \models E \iff Q \models E
\end{eqnarray*}

\begin{eqnarray*}
  P \approx_{K} Q
\end{eqnarray*}

\begin{eqnarray*}
  P \approx Q
\end{eqnarray*}

$\approx_{K} = \approx = \approx_{L}$

\subsubsection{Contextual duality}

Note that contexts extend the quotation operation to a family of
operations from processes to names. Given a context, $M$, we can
define a \emph{nominal context}, $\quotep{M}$ by $\quotep{M}[P] :=
\quotep{M[P]}$. To foreshadow what is to come we observe that these
operations enjoy a duality with processes very much like the duality
between vectors and maps from vectors to scalars.

Further, because the calculus is essentially higher-order, we have a
correspondence between contexts and processes. More specifically,
given a name $x$ and a context $M$ we can construct $M^{*}_{x}$ such
that 

\begin{mathpar}
  M^{*}_{x} | \lift{x}{P} \red M[P]
\end{mathpar}

namely,

\begin{mathpar}
  M^{*}_{x} := x?(u).M[\dropn{u}]
\end{mathpar}

The dependence of $M^{*}_{x}$ on a name makes it an abstraction, 

\begin{mathpar}
  M^{*} := (x)x?(u).M[\dropn{u}]
\end{mathpar}

\subsection{Additional notation}

It will sometimes be convenient to denote the process a name
quotes. We already have the notation $x = \quotep{P}$, but it will be
convenient to introduce an alternate notation, $\procn{x}$, when we
want to emphasize the connection to the use of the name. Note that, by
virtue of name equivalence, $\quotep{\procn{x}} \nameeq x$; so, the
notation is consistent with previous definitions.

Further, because names have structure it is possible to effect
substitutions on the basis of that structure. This means we need to
upgrade our notation for substitutions, which we accomplish by
adapting comprehension notation. Thus,

\begin{mathpar}
  P\{ y / x : x \in S \}
\end{mathpar}

is interpreted to mean the process derived from P by replacing (in a
capture-avoiding manner) each occurrence of $x$ in $S$ by $y$. For example,

\begin{mathpar}
  P\{ \quotep{\procn{x}|\procn{x}} / x : x \in \freenames{P} \}
\end{mathpar}

will replace each (occurrence) of a free name $x$ in $P$ by
$\quotep{\procn{x}|\procn{x}}$.

Also, we will avail ourselves of the notation $x^{L}$ and $x^{R}$ to
denote injections of a name into disjoint copies of the name
space. There are numerous ways to accomplish this. One example can be
found in \cite{MeredithR05}. This notation overloads to vectors of
names: $\vec{x}^{\pi} := (x_{i}^{\pi} \; : \; 0 \leq i < |\vec{x}| )$ where $\pi \in \{L,R\}$.

We also use $P^{\Box} := P|\Box$.

In \cite{MeredithR05} an interpretation of the new operator is
given. It turns out that there are several possible interpretations
all enjoying the requisite algebraic properties of the operator (see
\cite{milner91polyadicpi}). We will therefore make liberal use of
$(\nu\; \vec{x})P$.

% subsection the_syntax_and_semantics_of_the_notation_system (end)   

\input{qm2pi.qmops} 

\input{qm2pi.sterngerlach} 

\input{qm2pi.metric} 

% section concurrent_process_calculi (end)

%\input{qm2pi.proofsketch}

% section proof sketch (end)

%\input{qm2pi.slviaknots} 

% section spatial logic via knots (end)

\input{qm2pi.conclusion}

% section conclusion (end)

%\input{qm2pi.dtcodes} 

% section wiring algorithm (end)

\input{qm2pi.ack} 

% section acknowledgments (end)

\newpage


\bibliographystyle{plain}   
\bibliography{../../biblios/main.bib}

\input{qm2pi.rhodetails}

\end{document}

 

%\documentclass[12pt]{llncs}
%\documentclass{jktr}

\usepackage[pdftex]{hyperref}                   
\usepackage {listings}
\usepackage {mathpartir}
\usepackage{bcprules}
%\usepackage{listings}
                       
\usepackage{graphicx} 
%\usepackage[margins=2.5cm,nohead,nofoot]{geometry}
%\usepackage{geometry}
\usepackage{amsfonts}
\usepackage{amstext}
\usepackage{latexsym}
\usepackage{amssymb}
\usepackage{color}


%\include{myPreamble}
\include{qm2pi.local} 

%\ifpdf
%\usepackage[pdftex]{graphicx}
%\else
%\usepackage{graphicx}
%\fi

 % \ifpdf
%  \usepackage{pdfsync}
%  \if


%\title{Brief Article}
%\author{David F. Snyder}
%\author{L.G. Meredith}

%\address{Dept. of Math., Texas State University--San Marcos, San Marcos, TX 78666}
       
\pagestyle{empty}


\begin{document}

\lstset{language=[Objective]Caml,frame=shadowbox}

\input{qm2pi.front}

% section front matter (end)

\input{qm2pi.intro} 
 
% section introduction (end)

% \input{qm2pi.knotations} 

% section notation (end)

\input{qm2pi.process.calculi} 

% section concurrent_process_calculi_and_spatial_logics_ (end)
    
%\input{qm2pi.knots2pi} 

%\input{qm2pi.trefoil} 

%\input{qm2pi.mainthm} 

% subsection basic_interpretation (end)

%\input{qm2pi.rho.presentation} 
\subsection{The syntax and semantics of the notation system}\label{sub:the_syntax_and_semantics_of_the_notation_system} % (fold)

We now summarize a technical presentation of the calculus that
embodies our theory of dynamics. The typical presentation of such a
calculus follows the style of giving generators and relations on
them. The grammar, below, describing term constructors, freely
generates the set of processes, $\Proc$. This set is then quotiented
by a relation known as structural congruence and it is over this set
that the notion of dynamics is expressed. This presentation is
essentially that of \cite{MeredithR05} with the addition of
polyadicity and summation. For readability we have relegated some of
the technical subtleties to an appendix.

\subsubsection{Process grammar}\label{subsub:process_grammar}

\begin{mathpar}
  \inferrule* [lab=synchronization] {} {{M} \bc \pzero \;|\; x?F \;|\; x!C }
  \and
  \inferrule* [lab=abstraction] {} {{F} \bc (x)P}
  \and
  \inferrule* [lab=concretion] {} {{C} \bc \langle Q \rangle}
  \and
  \inferrule* [lab=process] {} {{P,Q} \bc M \;| \;P|Q \;|\; @{x}}
  \and
  \inferrule* [lab=name] {} {{x} \bc \quotep{P}}
\end{mathpar} 

Note that $\vec{x}$ (resp. $\vec{P}$) denotes a vector of names
(resp. processes) of length $|\vec{x}|$ (resp. $|\vec{P}|$). We adopt
the following useful abbreviations.

\begin{mathpar}
   x?(\vec{y}).P := x.(\vec{y})P \and  x\clift{\vec{P}} := x.\clift{\vec{P}}
   \and x!(y) := \lift{x}{\dropn{y}}
   \and \Pi_{i=0}^{n-1}P_i := P_0 | \ldots | P_{n-1}
\end{mathpar}

\subsubsection{Structural congruence}

\paragraph{Free and bound names and alpha-equivalence.} At the
core of structural equivalence is alpha-equivalence which identifies
process that are the same up to a change of variable. Formally, we
recognize the distinction between free and bound names. The free names
of a process, $\freenames{P}$, may be calculated recursively as
follows:

\begin{mathpar}
\freenames{\pzero} := \emptyset
  \and \\
  \freenames{x?(y).P} := \{ x \} \cup (\freenames{P} \setminus \{ y \})
  \and 
  \freenames{x!\langle P \rangle} := \{ x \} \cup \{ P \} 
  \and \\
  \freenames{P|Q} := \freenames{P} \cup \freenames{Q}
  \and \\
  \freenames{@{x}} := \{ x \}
\end{mathpar}

$\pi$
$\quotep{\pi}$

$\freenames{-} : \pi \to \mathcal{P}(\quotep{\pi})$

\begin{eqnarray*}
  \freenames{\pzero} & := & \emptyset \\
  \freenames{x?(y).P} & := & \{ x \} \cup (\freenames{P} \setminus \{ y \}) \\
  \freenames{x!\langle P \rangle} & := & \{ x \} \cup \{ P \} \\
  \freenames{P|Q} & := & \freenames{P} \cup \freenames{Q} \\
  \freenames{\dropn{x}} & := & \{ x \}
\end{eqnarray*}

The bound names of a process, $\boundnames{P}$, are those names occurring in $P$
that are not free. For example, in $x?(y).0$, the name $x$ is free, while $y$ is bound.

\begin{mathpar}
  \inferrule* [lab=monoidal-laws] {} { P|Q \equiv Q|P \and P|0 \equiv P \and P|(Q|R) \equiv (P|Q)|R }
\end{mathpar}

\begin{mathpar}
  \inferrule* [lab=alpha-equivalence] {} { (x)P \equiv (y)P\{y/x\} \and y \not\in \freenames{P} }
\end{mathpar}

\begin{definition}
Then two processes, $P,Q$, are alpha-equivalent if $P = Q\{\vec{y}/\vec{x}\}$ for
some $\vec{x} \in \boundnames{Q},\vec{y} \in \boundnames{P}$, where $Q\{\vec{y}/\vec{x}\}$
denotes the capture-avoiding substitution of $\vec{y}$ for $\vec{x}$ in $Q$.
\end{definition}

\begin{definition}
  The {\em structural congruence} \cite{SangiorgiWalker} , $\equiv$,
  between processes is the least congruence containing
  alpha-equivalence, satisfying the abelian monoid laws
  (associativity, commutativity and $\pzero$ as identity) for parallel
  composition $|$ and for summation $+$.
\end{definition}

\subsection{Name equivalence}

We take name equivalence, written $\nameeq$, to be the smallest
equivalence relation generated by the following rules.

\begin{mathpar}
\inferrule*[lab=Quote-drop]
{ }
{ \quotep{@{x}} \nameeq x }

\inferrule*[lab=Struct-equiv]
{ P \scong Q }
{ \quotep{P} \nameeq \quotep{Q} }
\end{mathpar}

The astute reader will have noticed that the mutual recursion of names
and processes imposes a mutual recursion on alpha-equivalence and
structural equivalence via name-equivalence. Fortunately, all of this
works out pleasantly and we may calculate in the natural way, free of
concern. The reader interested in the details is referred to the
appendix \ref{appendix:rho_details}.

\subsection{Substitution}

We use $\Proc$ for the set of processes, $\QProc$ for the set of
names, and $\id{\{}\vec{y} / \vec{x} \id{\}}$ to denote partial maps,
$s : \QProc \rightarrow \QProc$. A map, $s$ lifts, uniquely, to a map
on process terms, $\widehat{s} : \Proc \rightarrow \Proc$ by the
following equations.

\begin{mathpar}
  (0) \psubstp{Q}{P} := 0 \\
  (R \juxtap S) \psubstp{Q}{P}
  :=    
  (R)\psubstp{Q}{P} \juxtap (S) \psubstp{Q}{P} \\
  (x?(y).R) \psubstp{Q}{P}    
  :=    
  (x)\substp{Q}{P} (z)\concat( (R \psubstn{z}{y}) \psubstp{Q}{P} ) \\
  (\lift{x}{R}) \psubstp{Q}{P}  
  :=
  \lift{(x)\substp{Q}{P}}{ R \psubstp{Q}{P} } \\
%   (\dropn{x})  \psubstp{Q}{P}       
%   := 
%   \left\{ 
%     \begin{array}{ccc} 
%       \dropn{\quotep{Q}} & & x \nameeq \quotep{P} \\
%       \dropn{x} & & otherwise \\
%     \end{array}
%   \right. 
  (\dropn{x})  \psubstp{Q}{P}       
  := 
  \left\{ 
    \begin{array}{ccc} 
      Q & & x \nameeq \quotep{P} \\
      \dropn{x} & & otherwise \\
    \end{array}
  \right.
\end{mathpar}
 

where

\begin{eqnarray}
  (x)\id{\{} \lpquote Q \rpquote / \lpquote P \rpquote \id{\}}            = 
  \left\{ 
    \begin{array}{ccc}
      \lpquote Q \rpquote & & x \nameeq \lpquote P \rpquote \\
      x & & otherwise \\
    \end{array}
  \right. \nonumber
\end{eqnarray}

and $z$ is chosen distinct from $\quotep{P}$, $\quotep{Q}$, the free
names in $Q$, and all the names in $R$. Our $\alpha$-equivalence will
be built in the standard way from this substitution.

\begin{remark}\label{rem:no_self_referential_names}
  One consequence of these definitions is that $\forall P. \quotep{P}
  \not\in \freenames{P}$.
\end{remark}

\subsection{ Dynamic quote: an example }

Anticipating something of what's to come, consider applying the
substitution, $\widehat{\id{\{}u / z \id{\}}}$, to the following pair
of processes, $\lift{w}{y!(z)}$ and $w[ \lpquote y!(z) \rpquote ]$.

\begin{eqnarray}
	\lift{w}{y!(z)}\widehat{\id{\{}u / z \id{\}}}
		& = &
		\lift{w}{y!(u)} \nonumber\\
	w[ \lpquote y!(z) \rpquote ] \widehat{ \id{\{}u / z \id{\}} }
		& = &
		w[ \lpquote y!(z) \rpquote ] \nonumber
\end{eqnarray}

Because the body of the process between quotes is impervious to
substitution, we get radically different answers. In fact, by
examining the first process in an input context,
e.g. $x?(z).\lift{w}{y!(z)}$, we see that the process under the lift
operator may be shaped by prefixed inputs binding a name inside it. In
this sense, the lift operator will be seen as a way to dynamically
construct processes before reifying them as names.

Finally equipped with these standard features we can present the
dynamics of the calculus.

\subsubsection{Operational semantics} 

Finally, we introduce the computational dynamics. What marks these
algebras as distinct from other more traditionally studied algebraic
structures, e.g. vector spaces or polynomial rings, is the manner in
which dynamics is captured. In traditional structures, dynamics is typically
expressed through morphisms between such structures, as in linear maps
between vector spaces or morphisms between rings. In algebras
associated with the semantics of computation, the dynamics is
expressed as part of the algebraic structure itself, through a
reduction reduction relation typically denoted by $\red$. Below, we
give a recursive presentation of this relation for the calculus used
in the encoding.

$\red \subseteq \pi \times \pi$
$\red : \pi \to \mathcal{P}(\pi)$

\begin{mathpar}
  \inferrule* [lab=Comm] { \textsf{match}( x_{src}, x_{trgt} ) } { x_{trgt}?(y)P \; | \; x_{src}!\langle {Q} \rangle \red P\{\quotep{Q}/y}\} }
  \and \\
  \inferrule* [lab=Par] {{P} \red {P}'} {{{P} | {Q}} \red {{P}' | {Q}}}
  \and
  \inferrule* [lab=Equiv]{{{P} \scong {P}'} \andalso {{P}' \red {Q}'} \andalso {{Q}' \scong {Q}}}{{P} \red {Q}}
\end{mathpar}

\begin{eqnarray*}
  match_{\equiv} (\quotep{P},\quotep{Q}) & := & P \equiv Q \\
  match_{\dagger}(\quotep{P},\quotep{Q}) & := & \forall R. P|Q \red^{*} R => R \red^{*} 0 \\
  match_{K}(\quotep{P},\quotep{Q}) & := & K \mbox{ for some context } K
\end{eqnarray*}

$u?(x)P | u!\langle Q \rangle \red P\{\quotep{Q}/x\}$

%We write $\wred$ for $\red^*$, and $P\red$ if $\exists Q $ such that $ P \red Q$.
We write $P\red$ if $\exists Q $ such that $ P \red Q$ and $P\not\red$, otherwise.

\section{Replication}

As mentioned before, it is known that replication (and hence
recursion) can be implemented in a higher-order process algebra
\cite{SangiorgiWalker}. As our first example of calculation with the
machinery thus far presented we give the construction explicitly in
the {\rhoc}.

\begin{eqnarray}
	D_{x} & := & \prefix{x}{y}{(\binpar{\outputp{x}{y}}{@{y}})} \nonumber\\
	\bangp_{x}{P} & := & \binpar{{x}!\langle{\binpar{D_{x}}{P}}\rangle}{D_{x}} \nonumber
\end{eqnarray}

\begin{eqnarray}
	\bangp_{x}{P} & & \nonumber\\
	=
	& {x}!\langle{(\prefix{x}{y}{(\outputp{x}{y} | @{y})) | P}}\rangle 
	      | \prefix{x}{y}{(\outputp{x}{y} | @{y})} & \nonumber\\
	\red
	& (\outputp{x}{y} | @{y})\substn{\quotep{(\prefix{x}{y}{(@{y} | \outputp{x}{y})) | P}}}{y} & \nonumber\\
	=
	& \outputp{x}{\quotep{(\prefix{x}{y}{(\outputp{x}{y} | @{y})) | P}}}
	  | {(\prefix{x}{y}{(\outputp{x}{y} | @{y})) | P}} & \nonumber\\
	\red
	& \ldots & \nonumber\\
	\red^*
	& P | P | \ldots & \nonumber
\end{eqnarray}

Of course, this encoding, as an implementation, runs away, unfolding
$\bangp{P}$ eagerly. A lazier and more implementable replication
operator, restricted to input-guarded processes, may be obtained as follows.

\begin{eqnarray}
\bangp{\prefix{u}{v}{P}} 
	:= 
	\binpar{\lift{x}{\prefix{u}{v}{(\binpar{D(x)}{P})}}}{D(x)} \nonumber
\end{eqnarray}

\begin{remark}
  Note that the lazier definition still does not deal with summation
  or mixed summation (i.e. sums over input and output). The reader is
  invited to construct definitions of replication that deal with these
  features. 

  Further, the definitions are parameterized in a name, $x$. Can you,
  gentle reader, make a definition that eliminates this parameter and
  guarantees no accidental interaction between the replication
  machinery and the process being replicated -- i.e. no accidental
  sharing of names used by the process to get its work done and the
  name(s) used by the replication to effect copying. This latter
  revision of the definition of replication is crucial to obtaining
  the expected identity $!!P \sim !P$.
\end{remark}

\begin{remark}\label{rem:paradoxical_combinator}
  The reader familiar with the lambda calculus will have noticed the
  similarity between $D$ and the paradoxical combinator.

  [Ed. note: the existence of this seems to suggest we have to be more
  restrictive on the set of processes and names we admit if we are to
  support no-cloning.]
\end{remark}

\subsubsection{Bisimulation}

The computational dynamics gives rise to another kind of equivalence,
the equivalence of computational behavior. As previously mentioned
this is typically captured \emph{via} some form of bisimulation.

% The notion we use in this paper is weak barbed bisimulation
% \cite{milner91polyadicpi}.

The notion we use in this paper is derived from weak barbed
bisimulation \cite{milner91polyadicpi}. 

\begin{definition}
An \emph{observation relation}, $\downarrow_{\mathcal N}$, over a set
of names, $\mathcal N$, is the smallest relation satisfying the rules
below.

\infrule[Out-barb]{y \in {\mathcal N}, \; x \nameeq y}
		  {\outputp{x}{v} \downarrow_{\mathcal N} x}
\infrule[Par-barb]{\mbox{$P\downarrow_{\mathcal N} x$ or $Q\downarrow_{\mathcal N} x$}}
		  {\binpar{P}{Q} \downarrow_{\mathcal N} x}

We write $P \Downarrow_{\mathcal N} x$ if there is $Q$ such that 
$P \wred Q$ and $Q \downarrow_{\mathcal N} x$.
\end{definition}

\begin{definition}
%\label{def.bbisim}
An  ${\mathcal N}$-\emph{barbed bisimulation} over a set of names, ${\mathcal N}$, is a symmetric binary relation 
${\mathcal S}_{\mathcal N}$ between agents such that $P\rel{S}_{\mathcal N}Q$ implies:
\begin{enumerate}
\item If $P \red P'$ then $Q \wred Q'$ and $P'\rel{S}_{\mathcal N} Q'$.
\item If $P\downarrow_{\mathcal N} x$, then $Q\Downarrow_{\mathcal N} x$.
\end{enumerate}
$P$ is ${\mathcal N}$-barbed bisimilar to $Q$, written
$P \wbbisim_{\mathcal N} Q$, if $P \rel{S}_{\mathcal N} Q$ for some ${\mathcal N}$-barbed bisimulation ${\mathcal S}_{\mathcal N}$.
\end{definition}

$\mathcal{R} \subseteq \pi \times \pi$

$P \mathcal{R} Q => \forall P'. P \red P' \Rightarrow \exists Q'. Q \red Q', P' \mathcal{R} Q'$

$P \vdash x \Rightarrow Q \vdash x$

\begin{mathpar}
  \inferrule*[lab=Out-barb]{x \nameeq y}{{y}!\langle{Q}\rangle \vdash x}
  \and
  \inferrule*[lab=Par-barb]{\mbox{$P\vdash x$ or $Q\vdash x$}}{\binpar{P}{Q} \vdash x}
\end{mathpar}

\subsubsection{Contexts}

One of the principle advantages of computational calculi like the
$\pi$-calculus is a well-defined notion of context,
contextual-equivalence and a correlation between
contextual-equivalence and notions of bisimulation. The notion of
context allows the decomposition of a process into (sub-)process and
its syntactic environment, its context. Thus, a context may be
thought of as a process with a ``hole'' (written $\Box$) in it. The
application of a context $M$ to a process $P$, written $M[P]$, is
tantamount to filling the hole in $M$ with $P$. In this paper we do
not need the full weight of this theory, but do make use of the notion
of context in the proof the main theorem. 

\begin{mathpar}
  \inferrule* [lab=summation] {} {{M_{M},M_{N}} \bc \Box \;|\; x.M_{A} \;|\; M_{M}+M_{N}}
  \and
  \inferrule* [lab=agent] {} {{M_{A}} \bc (\vec{x})M_{P} \;| \; \clift{P_0,\ldots,M_{P},\ldots,P_N}}
  \and \\
  \inferrule* [lab=process] {} {{M_{P}} \bc M_{N} \;| \;P|M_{P} }
\end{mathpar} 

\begin{mathpar}
  \inferrule* [lab=sychronization] {} {M_{N} \bc \Box \;|\; x?M_{F} \;|\; x!M_{C}}
  \and
  \inferrule* [lab=abstraction] {} {{M_{F}} \bc (x)M_{P} }
  \and
  \inferrule* [lab=concretion] {} {{M_{C}} \bc \langle M_{P} \rangle }
  \and \\
  \inferrule* [lab=process] {} {{M_{P}} \bc M_{N} \;| \;P|M_{P} }
\end{mathpar}

\begin{definition}[contextual application] Given a context $M$, and
  process $P$, we define the \emph{contextual application}, $M[P] :=
  M\{P/\Box\}$. That is, the contextual application of M to P is the
  substitution of $P$ for $\Box$ in $M$.
\end{definition}

$\meaningof{-} : L \to \mathcal{P}(\pi)$

\begin{mathpar}
  \inferrule* [lab=collection] {} {\meaningof{true} = \pi, \and \meaningof{~E} = \pi \setminus \meaningof{E}, \and \meaningof{E_{1} \& E_{2}} = \meaningof{E_{1}} \cap \meaningof{E_{2}}}
\end{mathpar}

\begin{mathpar}
  \inferrule* [lab=structure] {} {\meaningof{0} = \{ P \in \pi | P \equiv 0 \}, \and \\ \meaningof{E_1 | E_2} = \{ P \in \pi | P \equiv P_{1} | P_{2}, P_{1} \in \meaningof{E_{1}}, P_{2} \in \meaningof{E_2}\} }
\end{mathpar}

\begin{mathpar}
 \inferrule* [lab=behavior] {} {\meaningof{\langle a?b \rangle E} = \{ P \in \pi | P \equiv Q | u?(y)P', \\ \and \\\\ \and \\ \;\;\; u \in \meaningof{a}, \forall z.P'\{z/y\} \in \meaningof{E\{z/b\}}\}, \and \\ \meaningof{a!E} = \{ P \in \pi | P \equiv Q | x!\langle P' \rangle, x \in \meaningof{a} P' \in \meaningof{E}\} }
\end{mathpar}

\begin{mathpar}
 \inferrule* [lab=nominal] {} {\meaningof{\quotep{E}} = \{ \quotep{P} \in \quotep{\pi} | P \in \meaningof{E} \}, \and \meaningof{\quotep{P}} = \{ \quotep{Q} \in \quotep{\pi} | P \equiv Q \} \and \\ \meaningof{@\quotep{E}} = \{ P \in \pi | P \equiv @x, x \in \meaningof{E} \}}
\end{mathpar}

\begin{eqnarray*}
  \\
  \meaningof{-} : TS \to ST
\end{eqnarray*}

\begin{eqnarray*}
  \\
  L : TS \to ST
\end{eqnarray*}

\begin{eqnarray*}
  \\
  P \models E \iff P \in \meaningof{E}
\end{eqnarray*}

\begin{eqnarray*}
  P \approx_{L} Q \iff \forall E \in L. P \models E \iff Q \models E
\end{eqnarray*}

\begin{eqnarray*}
  P \approx_{K} Q
\end{eqnarray*}

\begin{eqnarray*}
  P \approx Q
\end{eqnarray*}

$\approx_{K} = \approx = \approx_{L}$

\subsubsection{Contextual duality}

Note that contexts extend the quotation operation to a family of
operations from processes to names. Given a context, $M$, we can
define a \emph{nominal context}, $\quotep{M}$ by $\quotep{M}[P] :=
\quotep{M[P]}$. To foreshadow what is to come we observe that these
operations enjoy a duality with processes very much like the duality
between vectors and maps from vectors to scalars.

Further, because the calculus is essentially higher-order, we have a
correspondence between contexts and processes. More specifically,
given a name $x$ and a context $M$ we can construct $M^{*}_{x}$ such
that 

\begin{mathpar}
  M^{*}_{x} | \lift{x}{P} \red M[P]
\end{mathpar}

namely,

\begin{mathpar}
  M^{*}_{x} := x?(u).M[\dropn{u}]
\end{mathpar}

The dependence of $M^{*}_{x}$ on a name makes it an abstraction, 

\begin{mathpar}
  M^{*} := (x)x?(u).M[\dropn{u}]
\end{mathpar}

\subsection{Additional notation}

It will sometimes be convenient to denote the process a name
quotes. We already have the notation $x = \quotep{P}$, but it will be
convenient to introduce an alternate notation, $\procn{x}$, when we
want to emphasize the connection to the use of the name. Note that, by
virtue of name equivalence, $\quotep{\procn{x}} \nameeq x$; so, the
notation is consistent with previous definitions.

Further, because names have structure it is possible to effect
substitutions on the basis of that structure. This means we need to
upgrade our notation for substitutions, which we accomplish by
adapting comprehension notation. Thus,

\begin{mathpar}
  P\{ y / x : x \in S \}
\end{mathpar}

is interpreted to mean the process derived from P by replacing (in a
capture-avoiding manner) each occurrence of $x$ in $S$ by $y$. For example,

\begin{mathpar}
  P\{ \quotep{\procn{x}|\procn{x}} / x : x \in \freenames{P} \}
\end{mathpar}

will replace each (occurrence) of a free name $x$ in $P$ by
$\quotep{\procn{x}|\procn{x}}$.

Also, we will avail ourselves of the notation $x^{L}$ and $x^{R}$ to
denote injections of a name into disjoint copies of the name
space. There are numerous ways to accomplish this. One example can be
found in \cite{MeredithR05}. This notation overloads to vectors of
names: $\vec{x}^{\pi} := (x_{i}^{\pi} \; : \; 0 \leq i < |\vec{x}| )$ where $\pi \in \{L,R\}$.

We also use $P^{\Box} := P|\Box$.

In \cite{MeredithR05} an interpretation of the new operator is
given. It turns out that there are several possible interpretations
all enjoying the requisite algebraic properties of the operator (see
\cite{milner91polyadicpi}). We will therefore make liberal use of
$(\nu\; \vec{x})P$.

% subsection the_syntax_and_semantics_of_the_notation_system (end)   

\input{qm2pi.qmops} 

\input{qm2pi.sterngerlach} 

\input{qm2pi.metric} 

% section concurrent_process_calculi (end)

%\input{qm2pi.proofsketch}

% section proof sketch (end)

%\input{qm2pi.slviaknots} 

% section spatial logic via knots (end)

\input{qm2pi.conclusion}

% section conclusion (end)

%\input{qm2pi.dtcodes} 

% section wiring algorithm (end)

\input{qm2pi.ack} 

% section acknowledgments (end)

\newpage


\bibliographystyle{plain}   
\bibliography{../../biblios/main.bib}

\input{qm2pi.rhodetails}

\end{document}

 

% subsection basic_interpretation (end)

%\input{qm2pi.rho.presentation} 
\subsection{The syntax and semantics of the notation system}\label{sub:the_syntax_and_semantics_of_the_notation_system} % (fold)

We now summarize a technical presentation of the calculus that
embodies our theory of dynamics. The typical presentation of such a
calculus follows the style of giving generators and relations on
them. The grammar, below, describing term constructors, freely
generates the set of processes, $\Proc$. This set is then quotiented
by a relation known as structural congruence and it is over this set
that the notion of dynamics is expressed. This presentation is
essentially that of \cite{MeredithR05} with the addition of
polyadicity and summation. For readability we have relegated some of
the technical subtleties to an appendix.

\subsubsection{Process grammar}\label{subsub:process_grammar}

\begin{mathpar}
  \inferrule* [lab=synchronization] {} {{M} \bc \pzero \;|\; x?F \;|\; x!C }
  \and
  \inferrule* [lab=abstraction] {} {{F} \bc (x)P}
  \and
  \inferrule* [lab=concretion] {} {{C} \bc \langle Q \rangle}
  \and
  \inferrule* [lab=process] {} {{P,Q} \bc M \;| \;P|Q \;|\; @{x}}
  \and
  \inferrule* [lab=name] {} {{x} \bc \quotep{P}}
\end{mathpar} 

Note that $\vec{x}$ (resp. $\vec{P}$) denotes a vector of names
(resp. processes) of length $|\vec{x}|$ (resp. $|\vec{P}|$). We adopt
the following useful abbreviations.

\begin{mathpar}
   x?(\vec{y}).P := x.(\vec{y})P \and  x\clift{\vec{P}} := x.\clift{\vec{P}}
   \and x!(y) := \lift{x}{\dropn{y}}
   \and \Pi_{i=0}^{n-1}P_i := P_0 | \ldots | P_{n-1}
\end{mathpar}

\subsubsection{Structural congruence}

\paragraph{Free and bound names and alpha-equivalence.} At the
core of structural equivalence is alpha-equivalence which identifies
process that are the same up to a change of variable. Formally, we
recognize the distinction between free and bound names. The free names
of a process, $\freenames{P}$, may be calculated recursively as
follows:

\begin{mathpar}
\freenames{\pzero} := \emptyset
  \and \\
  \freenames{x?(y).P} := \{ x \} \cup (\freenames{P} \setminus \{ y \})
  \and 
  \freenames{x!\langle P \rangle} := \{ x \} \cup \{ P \} 
  \and \\
  \freenames{P|Q} := \freenames{P} \cup \freenames{Q}
  \and \\
  \freenames{@{x}} := \{ x \}
\end{mathpar}

$\pi$
$\quotep{\pi}$

$\freenames{-} : \pi \to \mathcal{P}(\quotep{\pi})$

\begin{eqnarray*}
  \freenames{\pzero} & := & \emptyset \\
  \freenames{x?(y).P} & := & \{ x \} \cup (\freenames{P} \setminus \{ y \}) \\
  \freenames{x!\langle P \rangle} & := & \{ x \} \cup \{ P \} \\
  \freenames{P|Q} & := & \freenames{P} \cup \freenames{Q} \\
  \freenames{\dropn{x}} & := & \{ x \}
\end{eqnarray*}

The bound names of a process, $\boundnames{P}$, are those names occurring in $P$
that are not free. For example, in $x?(y).0$, the name $x$ is free, while $y$ is bound.

\begin{mathpar}
  \inferrule* [lab=monoidal-laws] {} { P|Q \equiv Q|P \and P|0 \equiv P \and P|(Q|R) \equiv (P|Q)|R }
\end{mathpar}

\begin{mathpar}
  \inferrule* [lab=alpha-equivalence] {} { (x)P \equiv (y)P\{y/x\} \and y \not\in \freenames{P} }
\end{mathpar}

\begin{definition}
Then two processes, $P,Q$, are alpha-equivalent if $P = Q\{\vec{y}/\vec{x}\}$ for
some $\vec{x} \in \boundnames{Q},\vec{y} \in \boundnames{P}$, where $Q\{\vec{y}/\vec{x}\}$
denotes the capture-avoiding substitution of $\vec{y}$ for $\vec{x}$ in $Q$.
\end{definition}

\begin{definition}
  The {\em structural congruence} \cite{SangiorgiWalker} , $\equiv$,
  between processes is the least congruence containing
  alpha-equivalence, satisfying the abelian monoid laws
  (associativity, commutativity and $\pzero$ as identity) for parallel
  composition $|$ and for summation $+$.
\end{definition}

\subsection{Name equivalence}

We take name equivalence, written $\nameeq$, to be the smallest
equivalence relation generated by the following rules.

\begin{mathpar}
\inferrule*[lab=Quote-drop]
{ }
{ \quotep{@{x}} \nameeq x }

\inferrule*[lab=Struct-equiv]
{ P \scong Q }
{ \quotep{P} \nameeq \quotep{Q} }
\end{mathpar}

The astute reader will have noticed that the mutual recursion of names
and processes imposes a mutual recursion on alpha-equivalence and
structural equivalence via name-equivalence. Fortunately, all of this
works out pleasantly and we may calculate in the natural way, free of
concern. The reader interested in the details is referred to the
appendix \ref{appendix:rho_details}.

\subsection{Substitution}

We use $\Proc$ for the set of processes, $\QProc$ for the set of
names, and $\id{\{}\vec{y} / \vec{x} \id{\}}$ to denote partial maps,
$s : \QProc \rightarrow \QProc$. A map, $s$ lifts, uniquely, to a map
on process terms, $\widehat{s} : \Proc \rightarrow \Proc$ by the
following equations.

\begin{mathpar}
  (0) \psubstp{Q}{P} := 0 \\
  (R \juxtap S) \psubstp{Q}{P}
  :=    
  (R)\psubstp{Q}{P} \juxtap (S) \psubstp{Q}{P} \\
  (x?(y).R) \psubstp{Q}{P}    
  :=    
  (x)\substp{Q}{P} (z)\concat( (R \psubstn{z}{y}) \psubstp{Q}{P} ) \\
  (\lift{x}{R}) \psubstp{Q}{P}  
  :=
  \lift{(x)\substp{Q}{P}}{ R \psubstp{Q}{P} } \\
%   (\dropn{x})  \psubstp{Q}{P}       
%   := 
%   \left\{ 
%     \begin{array}{ccc} 
%       \dropn{\quotep{Q}} & & x \nameeq \quotep{P} \\
%       \dropn{x} & & otherwise \\
%     \end{array}
%   \right. 
  (\dropn{x})  \psubstp{Q}{P}       
  := 
  \left\{ 
    \begin{array}{ccc} 
      Q & & x \nameeq \quotep{P} \\
      \dropn{x} & & otherwise \\
    \end{array}
  \right.
\end{mathpar}
 

where

\begin{eqnarray}
  (x)\id{\{} \lpquote Q \rpquote / \lpquote P \rpquote \id{\}}            = 
  \left\{ 
    \begin{array}{ccc}
      \lpquote Q \rpquote & & x \nameeq \lpquote P \rpquote \\
      x & & otherwise \\
    \end{array}
  \right. \nonumber
\end{eqnarray}

and $z$ is chosen distinct from $\quotep{P}$, $\quotep{Q}$, the free
names in $Q$, and all the names in $R$. Our $\alpha$-equivalence will
be built in the standard way from this substitution.

\begin{remark}\label{rem:no_self_referential_names}
  One consequence of these definitions is that $\forall P. \quotep{P}
  \not\in \freenames{P}$.
\end{remark}

\subsection{ Dynamic quote: an example }

Anticipating something of what's to come, consider applying the
substitution, $\widehat{\id{\{}u / z \id{\}}}$, to the following pair
of processes, $\lift{w}{y!(z)}$ and $w[ \lpquote y!(z) \rpquote ]$.

\begin{eqnarray}
	\lift{w}{y!(z)}\widehat{\id{\{}u / z \id{\}}}
		& = &
		\lift{w}{y!(u)} \nonumber\\
	w[ \lpquote y!(z) \rpquote ] \widehat{ \id{\{}u / z \id{\}} }
		& = &
		w[ \lpquote y!(z) \rpquote ] \nonumber
\end{eqnarray}

Because the body of the process between quotes is impervious to
substitution, we get radically different answers. In fact, by
examining the first process in an input context,
e.g. $x?(z).\lift{w}{y!(z)}$, we see that the process under the lift
operator may be shaped by prefixed inputs binding a name inside it. In
this sense, the lift operator will be seen as a way to dynamically
construct processes before reifying them as names.

Finally equipped with these standard features we can present the
dynamics of the calculus.

\subsubsection{Operational semantics} 

Finally, we introduce the computational dynamics. What marks these
algebras as distinct from other more traditionally studied algebraic
structures, e.g. vector spaces or polynomial rings, is the manner in
which dynamics is captured. In traditional structures, dynamics is typically
expressed through morphisms between such structures, as in linear maps
between vector spaces or morphisms between rings. In algebras
associated with the semantics of computation, the dynamics is
expressed as part of the algebraic structure itself, through a
reduction reduction relation typically denoted by $\red$. Below, we
give a recursive presentation of this relation for the calculus used
in the encoding.

$\red \subseteq \pi \times \pi$
$\red : \pi \to \mathcal{P}(\pi)$

\begin{mathpar}
  \inferrule* [lab=Comm] { \textsf{match}( x_{src}, x_{trgt} ) } { x_{trgt}?(y)P \; | \; x_{src}!\langle {Q} \rangle \red P\{\quotep{Q}/y}\} }
  \and \\
  \inferrule* [lab=Par] {{P} \red {P}'} {{{P} | {Q}} \red {{P}' | {Q}}}
  \and
  \inferrule* [lab=Equiv]{{{P} \scong {P}'} \andalso {{P}' \red {Q}'} \andalso {{Q}' \scong {Q}}}{{P} \red {Q}}
\end{mathpar}

\begin{eqnarray*}
  match_{\equiv} (\quotep{P},\quotep{Q}) & := & P \equiv Q \\
  match_{\dagger}(\quotep{P},\quotep{Q}) & := & \forall R. P|Q \red^{*} R => R \red^{*} 0 \\
  match_{K}(\quotep{P},\quotep{Q}) & := & K \mbox{ for some context } K
\end{eqnarray*}

$u?(x)P | u!\langle Q \rangle \red P\{\quotep{Q}/x\}$

%We write $\wred$ for $\red^*$, and $P\red$ if $\exists Q $ such that $ P \red Q$.
We write $P\red$ if $\exists Q $ such that $ P \red Q$ and $P\not\red$, otherwise.

\section{Replication}

As mentioned before, it is known that replication (and hence
recursion) can be implemented in a higher-order process algebra
\cite{SangiorgiWalker}. As our first example of calculation with the
machinery thus far presented we give the construction explicitly in
the {\rhoc}.

\begin{eqnarray}
	D_{x} & := & \prefix{x}{y}{(\binpar{\outputp{x}{y}}{@{y}})} \nonumber\\
	\bangp_{x}{P} & := & \binpar{{x}!\langle{\binpar{D_{x}}{P}}\rangle}{D_{x}} \nonumber
\end{eqnarray}

\begin{eqnarray}
	\bangp_{x}{P} & & \nonumber\\
	=
	& {x}!\langle{(\prefix{x}{y}{(\outputp{x}{y} | @{y})) | P}}\rangle 
	      | \prefix{x}{y}{(\outputp{x}{y} | @{y})} & \nonumber\\
	\red
	& (\outputp{x}{y} | @{y})\substn{\quotep{(\prefix{x}{y}{(@{y} | \outputp{x}{y})) | P}}}{y} & \nonumber\\
	=
	& \outputp{x}{\quotep{(\prefix{x}{y}{(\outputp{x}{y} | @{y})) | P}}}
	  | {(\prefix{x}{y}{(\outputp{x}{y} | @{y})) | P}} & \nonumber\\
	\red
	& \ldots & \nonumber\\
	\red^*
	& P | P | \ldots & \nonumber
\end{eqnarray}

Of course, this encoding, as an implementation, runs away, unfolding
$\bangp{P}$ eagerly. A lazier and more implementable replication
operator, restricted to input-guarded processes, may be obtained as follows.

\begin{eqnarray}
\bangp{\prefix{u}{v}{P}} 
	:= 
	\binpar{\lift{x}{\prefix{u}{v}{(\binpar{D(x)}{P})}}}{D(x)} \nonumber
\end{eqnarray}

\begin{remark}
  Note that the lazier definition still does not deal with summation
  or mixed summation (i.e. sums over input and output). The reader is
  invited to construct definitions of replication that deal with these
  features. 

  Further, the definitions are parameterized in a name, $x$. Can you,
  gentle reader, make a definition that eliminates this parameter and
  guarantees no accidental interaction between the replication
  machinery and the process being replicated -- i.e. no accidental
  sharing of names used by the process to get its work done and the
  name(s) used by the replication to effect copying. This latter
  revision of the definition of replication is crucial to obtaining
  the expected identity $!!P \sim !P$.
\end{remark}

\begin{remark}\label{rem:paradoxical_combinator}
  The reader familiar with the lambda calculus will have noticed the
  similarity between $D$ and the paradoxical combinator.

  [Ed. note: the existence of this seems to suggest we have to be more
  restrictive on the set of processes and names we admit if we are to
  support no-cloning.]
\end{remark}

\subsubsection{Bisimulation}

The computational dynamics gives rise to another kind of equivalence,
the equivalence of computational behavior. As previously mentioned
this is typically captured \emph{via} some form of bisimulation.

% The notion we use in this paper is weak barbed bisimulation
% \cite{milner91polyadicpi}.

The notion we use in this paper is derived from weak barbed
bisimulation \cite{milner91polyadicpi}. 

\begin{definition}
An \emph{observation relation}, $\downarrow_{\mathcal N}$, over a set
of names, $\mathcal N$, is the smallest relation satisfying the rules
below.

\infrule[Out-barb]{y \in {\mathcal N}, \; x \nameeq y}
		  {\outputp{x}{v} \downarrow_{\mathcal N} x}
\infrule[Par-barb]{\mbox{$P\downarrow_{\mathcal N} x$ or $Q\downarrow_{\mathcal N} x$}}
		  {\binpar{P}{Q} \downarrow_{\mathcal N} x}

We write $P \Downarrow_{\mathcal N} x$ if there is $Q$ such that 
$P \wred Q$ and $Q \downarrow_{\mathcal N} x$.
\end{definition}

\begin{definition}
%\label{def.bbisim}
An  ${\mathcal N}$-\emph{barbed bisimulation} over a set of names, ${\mathcal N}$, is a symmetric binary relation 
${\mathcal S}_{\mathcal N}$ between agents such that $P\rel{S}_{\mathcal N}Q$ implies:
\begin{enumerate}
\item If $P \red P'$ then $Q \wred Q'$ and $P'\rel{S}_{\mathcal N} Q'$.
\item If $P\downarrow_{\mathcal N} x$, then $Q\Downarrow_{\mathcal N} x$.
\end{enumerate}
$P$ is ${\mathcal N}$-barbed bisimilar to $Q$, written
$P \wbbisim_{\mathcal N} Q$, if $P \rel{S}_{\mathcal N} Q$ for some ${\mathcal N}$-barbed bisimulation ${\mathcal S}_{\mathcal N}$.
\end{definition}

$\mathcal{R} \subseteq \pi \times \pi$

$P \mathcal{R} Q => \forall P'. P \red P' \Rightarrow \exists Q'. Q \red Q', P' \mathcal{R} Q'$

$P \vdash x \Rightarrow Q \vdash x$

\begin{mathpar}
  \inferrule*[lab=Out-barb]{x \nameeq y}{{y}!\langle{Q}\rangle \vdash x}
  \and
  \inferrule*[lab=Par-barb]{\mbox{$P\vdash x$ or $Q\vdash x$}}{\binpar{P}{Q} \vdash x}
\end{mathpar}

\subsubsection{Contexts}

One of the principle advantages of computational calculi like the
$\pi$-calculus is a well-defined notion of context,
contextual-equivalence and a correlation between
contextual-equivalence and notions of bisimulation. The notion of
context allows the decomposition of a process into (sub-)process and
its syntactic environment, its context. Thus, a context may be
thought of as a process with a ``hole'' (written $\Box$) in it. The
application of a context $M$ to a process $P$, written $M[P]$, is
tantamount to filling the hole in $M$ with $P$. In this paper we do
not need the full weight of this theory, but do make use of the notion
of context in the proof the main theorem. 

\begin{mathpar}
  \inferrule* [lab=summation] {} {{M_{M},M_{N}} \bc \Box \;|\; x.M_{A} \;|\; M_{M}+M_{N}}
  \and
  \inferrule* [lab=agent] {} {{M_{A}} \bc (\vec{x})M_{P} \;| \; \clift{P_0,\ldots,M_{P},\ldots,P_N}}
  \and \\
  \inferrule* [lab=process] {} {{M_{P}} \bc M_{N} \;| \;P|M_{P} }
\end{mathpar} 

\begin{mathpar}
  \inferrule* [lab=sychronization] {} {M_{N} \bc \Box \;|\; x?M_{F} \;|\; x!M_{C}}
  \and
  \inferrule* [lab=abstraction] {} {{M_{F}} \bc (x)M_{P} }
  \and
  \inferrule* [lab=concretion] {} {{M_{C}} \bc \langle M_{P} \rangle }
  \and \\
  \inferrule* [lab=process] {} {{M_{P}} \bc M_{N} \;| \;P|M_{P} }
\end{mathpar}

\begin{definition}[contextual application] Given a context $M$, and
  process $P$, we define the \emph{contextual application}, $M[P] :=
  M\{P/\Box\}$. That is, the contextual application of M to P is the
  substitution of $P$ for $\Box$ in $M$.
\end{definition}

$\meaningof{-} : L \to \mathcal{P}(\pi)$

\begin{mathpar}
  \inferrule* [lab=collection] {} {\meaningof{true} = \pi, \and \meaningof{~E} = \pi \setminus \meaningof{E}, \and \meaningof{E_{1} \& E_{2}} = \meaningof{E_{1}} \cap \meaningof{E_{2}}}
\end{mathpar}

\begin{mathpar}
  \inferrule* [lab=structure] {} {\meaningof{0} = \{ P \in \pi | P \equiv 0 \}, \and \\ \meaningof{E_1 | E_2} = \{ P \in \pi | P \equiv P_{1} | P_{2}, P_{1} \in \meaningof{E_{1}}, P_{2} \in \meaningof{E_2}\} }
\end{mathpar}

\begin{mathpar}
 \inferrule* [lab=behavior] {} {\meaningof{\langle a?b \rangle E} = \{ P \in \pi | P \equiv Q | u?(y)P', \\ \and \\\\ \and \\ \;\;\; u \in \meaningof{a}, \forall z.P'\{z/y\} \in \meaningof{E\{z/b\}}\}, \and \\ \meaningof{a!E} = \{ P \in \pi | P \equiv Q | x!\langle P' \rangle, x \in \meaningof{a} P' \in \meaningof{E}\} }
\end{mathpar}

\begin{mathpar}
 \inferrule* [lab=nominal] {} {\meaningof{\quotep{E}} = \{ \quotep{P} \in \quotep{\pi} | P \in \meaningof{E} \}, \and \meaningof{\quotep{P}} = \{ \quotep{Q} \in \quotep{\pi} | P \equiv Q \} \and \\ \meaningof{@\quotep{E}} = \{ P \in \pi | P \equiv @x, x \in \meaningof{E} \}}
\end{mathpar}

\begin{eqnarray*}
  \\
  \meaningof{-} : TS \to ST
\end{eqnarray*}

\begin{eqnarray*}
  \\
  L : TS \to ST
\end{eqnarray*}

\begin{eqnarray*}
  \\
  P \models E \iff P \in \meaningof{E}
\end{eqnarray*}

\begin{eqnarray*}
  P \approx_{L} Q \iff \forall E \in L. P \models E \iff Q \models E
\end{eqnarray*}

\begin{eqnarray*}
  P \approx_{K} Q
\end{eqnarray*}

\begin{eqnarray*}
  P \approx Q
\end{eqnarray*}

$\approx_{K} = \approx = \approx_{L}$

\subsubsection{Contextual duality}

Note that contexts extend the quotation operation to a family of
operations from processes to names. Given a context, $M$, we can
define a \emph{nominal context}, $\quotep{M}$ by $\quotep{M}[P] :=
\quotep{M[P]}$. To foreshadow what is to come we observe that these
operations enjoy a duality with processes very much like the duality
between vectors and maps from vectors to scalars.

Further, because the calculus is essentially higher-order, we have a
correspondence between contexts and processes. More specifically,
given a name $x$ and a context $M$ we can construct $M^{*}_{x}$ such
that 

\begin{mathpar}
  M^{*}_{x} | \lift{x}{P} \red M[P]
\end{mathpar}

namely,

\begin{mathpar}
  M^{*}_{x} := x?(u).M[\dropn{u}]
\end{mathpar}

The dependence of $M^{*}_{x}$ on a name makes it an abstraction, 

\begin{mathpar}
  M^{*} := (x)x?(u).M[\dropn{u}]
\end{mathpar}

\subsection{Additional notation}

It will sometimes be convenient to denote the process a name
quotes. We already have the notation $x = \quotep{P}$, but it will be
convenient to introduce an alternate notation, $\procn{x}$, when we
want to emphasize the connection to the use of the name. Note that, by
virtue of name equivalence, $\quotep{\procn{x}} \nameeq x$; so, the
notation is consistent with previous definitions.

Further, because names have structure it is possible to effect
substitutions on the basis of that structure. This means we need to
upgrade our notation for substitutions, which we accomplish by
adapting comprehension notation. Thus,

\begin{mathpar}
  P\{ y / x : x \in S \}
\end{mathpar}

is interpreted to mean the process derived from P by replacing (in a
capture-avoiding manner) each occurrence of $x$ in $S$ by $y$. For example,

\begin{mathpar}
  P\{ \quotep{\procn{x}|\procn{x}} / x : x \in \freenames{P} \}
\end{mathpar}

will replace each (occurrence) of a free name $x$ in $P$ by
$\quotep{\procn{x}|\procn{x}}$.

Also, we will avail ourselves of the notation $x^{L}$ and $x^{R}$ to
denote injections of a name into disjoint copies of the name
space. There are numerous ways to accomplish this. One example can be
found in \cite{MeredithR05}. This notation overloads to vectors of
names: $\vec{x}^{\pi} := (x_{i}^{\pi} \; : \; 0 \leq i < |\vec{x}| )$ where $\pi \in \{L,R\}$.

We also use $P^{\Box} := P|\Box$.

In \cite{MeredithR05} an interpretation of the new operator is
given. It turns out that there are several possible interpretations
all enjoying the requisite algebraic properties of the operator (see
\cite{milner91polyadicpi}). We will therefore make liberal use of
$(\nu\; \vec{x})P$.

% subsection the_syntax_and_semantics_of_the_notation_system (end)   

\section{Interpretation of QM}
\subsection{Supporting definitions}
\subsubsection{Multiplication}
\begin{mathpar}
  \quotep{Q} \cdot \quotep{R} := \quotep{Q|R}
  \and \\
  \quotep{Q} \cdot P := P\{ \quotep{Q|R} / \quotep{R} : \quotep{R} \in \freenames{P} \}
\end{mathpar}

\paragraph{Discussion}
The first line needs little explanation. The second line says that
each free name of the process is replaced with the multiplication of
that name by the scalar. Multiplication of a scalar (name) by a state
(process) results in a process all the names of which have been `moved
over' by parallel composition with the process the scalar
quotes. There is a subtlety that the bound names have to be
manipulated so that multiplied names aren't accidentally
captured. There are many ways to achieve this.

\begin{remark}\label{rem:multiplication_identities}
  The reader is invited to verify that for all $x,y,z \in \QProc$ and $P \in \Proc$
  \begin{mathpar}
    x \cdot \quotep{0} \equiv x 
    \and
    x \cdot y \equiv y \cdot x
    \and
    x \cdot (y \cdot z) \equiv (x \cdot y) \cdot z
    \and \\
    \quotep{0} \cdot P \equiv P
    \and \\
    x \cdot (y \cdot P) \equiv (x \cdot y) \cdot P
    \and \\
    x \cdot (P|Q) \equiv (x \cdot P) | (x \cdot Q)
    \and \\    
  \end{mathpar}
\end{remark}

\subsubsection{Tensor product}

We define a tensor product on processes by structural induction.

\paragraph{Tensor of sums} First note that all summations, including
$\pzero$ and sequence, can be written $\Sigma_{i} x_{i}.A_{i} +
\Sigma_{j} x_{j}.C_{j}$, where we have grouped input-guarded processes
together and output-guarded processes together.

Thus, we can define the tensor product of two summations, $N_{1}\otimes N_{2}$, where

\begin{mathpar}
  N_{1} := \Sigma_{i} x_{i}.A_{i} + \Sigma_{j} x_{j}.C_{j}
  \and
  N_{2} := \Sigma_{i'} y_{i'}.B_{i'} + \Sigma_{j'} y_{j'}.D_{j'} 
\end{mathpar}

as follows.

\begin{mathpar}
  \Sigma_{i} x_{i}.A_{i} + \Sigma_{j} x_{j}.C_{j} \otimes \Sigma_{i'}
  y_{i'}.B_{i'} + \Sigma_{j'} y_{j'}.D_{j'} 
  \and \\
  := \; \Sigma_{i} \Sigma_{i'} \quotep{\stackrel{\vee}{x_{i}}| \stackrel{\vee}{y_{i'}}}.(A_{i}\otimes B_{i'}) \; | \; \Sigma_{i'} \Sigma_{i} \quotep{\stackrel{\vee}{y_{i'}}|\stackrel{\vee}{x_{i}}}.(B_{i'}\otimes A_{i})
  \and
  \;\; | \;\; \Sigma_{j} \Sigma_{j'} \quotep{\stackrel{\vee}{x_{j}}|\stackrel{\vee}{y_{j'}}}.(A_{j}\otimes B_{j'}) \; | \; \Sigma_{j'} \Sigma_{j} \quotep{\stackrel{\vee}{y_{j'}}|\stackrel{\vee}{x_{j}}}.(B_{j'}\otimes A_{j})
\end{mathpar}

\begin{remark}
  Do we need to $x^{L}$ and $y^{R}$ for this construction as well?
\end{remark}

\paragraph{Tensor of parallel compositions} Next, we distribute tensor
over par.

\begin{mathpar}
  P_{1}|P_{2} \otimes Q_{1}|Q_{2} := (P_{1} \otimes Q_{1}) | (P_{1}
  \otimes Q_{2}) | (P_{2} \otimes Q_{1}) | (P_{2} \otimes Q_{2})
\end{mathpar}

\paragraph{Tensor with dropped names} We treat tensor of a
process with a dropped name as parallel composition.

\begin{mathpar}
  P \otimes \dropn{x} := P | \dropn{x}
\end{mathpar}

\paragraph{Tensor of agents}

Finally, we need to define tensor on agents. Note that the definition
of tensor on normal products only tensors inputs with inputs and
outputs with outputs. Thus, we only have to define the operation on
``homogeneous'' pairings.

\begin{mathpar}
  (\vec{x})P \otimes (\vec{y})Q
  \and \\
  := (x_{0}^{L}|y_{0}^{R},\ldots,x_{0}^{L}|y_{n}^{R},\ldots,x_{m}^{L}|y_{0}^{R},\ldots,x_{m}^{L}|y_{n}^R)(P\{ \vec{x}^{L}/\vec{x}\} \otimes Q \{ \vec{y}^{R}/\vec{y}\})
  \and \\
  \clift{\vec{P}} \otimes \clift{\vec{Q}}
  \and \\
  := \clift{P_{0}\otimes Q_{0},\ldots,P_{0}\otimes Q_{n},\ldots,P_{m}\otimes Q_{0},\ldots,P_{m}\otimes Q_{n}}
\end{mathpar}

\begin{remark}
  Observe that arities of tensored abstractions matches arities of
  tensored concretions if the original arities matched. Note also that
  the length of the arities corresponds to the increase in dimension
  we see in ordinary vector space tensor product.
\end{remark}

\begin{remark}
  Operationally, this definition distributes the tensor down to
  components ``linked'' by summation. Tensor over summation is
  intriguing in that it mixes names. Moreover, as a consequence of the
  way it mixes names we have the identities for all $x \in \QProc$ and
  $P,Q \in \Proc$

  \begin{mathpar}
    (x \cdot P) \otimes Q \equiv x \cdot (P \otimes Q) \equiv P \otimes (x \cdot Q)
    \and
    P \otimes \pzero \equiv P
  \end{mathpar}

  that the reader is invited to verify.
\end{remark}

\subsubsection{Annihilation}
\begin{mathpar}
  P^{\perp} := \{ Q | \forall R. P|Q \red^{*} R \Rightarrow R \red^{*} \pzero \}
  \and \\
  P^{\underline{\perp}} := \Sigma_{Q \in P^{\perp}} \quotep{Q}?(y).(\dropn{y}|Q) | \Sigma_{Q \in P^{\perp}} \quotep{Q}\clift{\Box}
\end{mathpar}

\paragraph{Discussion} The reader will note that $P^{\perp}$ is a
\emph{set} of processes, while $P^{\underline{\perp}}$ is a
\emph{context}. We call the set $P^{\perp}$ the \emph{annihilators} of
$P$. The parallel composition of a process in the annihilators of $P$
with $P$ will result in a process, the state space of which has all
paths eventually leading to $\pzero$. Execution may endure loops; but
under reasonable conditions of fairness (naturally guaranteed under
most notions of bisimulation) such a composite process cannot get
stuck in such a loop and will, eventually pop out and terminate.

The context $P^{\underline{\perp}}$ is ready and willing to ``take the
$P$ out of'' the process to which it is applied. It will effectively
transmit the code of the process to which it is applied to one of the
annihilators and run the process against it.

\subsubsection{Evaluation}
We fix $M$ a domain of fully abstract interpretation with an equality
coincident with bisimulation. We take $\meaningof{\cdot} : \Proc \to
M$ to be the map interpreting processes and $\nmeaningof{\cdot} : \M
\to Proc$ to be the map running the other way. Then we define

\begin{mathpar}
  \int P := \nmeaningof{\meaningof{P}}
\end{mathpar}

\paragraph{Discussion}
There are many fully abstract interpretations of Milner's
$\pi$-calculus. Any of them can be used as a basis for interpreting
the reflective calculus here. Equipped with such a domain it is
largely a matter of grinding through to check that the Yoneda
construction for the normalization-by-evaluation program can be
extended to this setting.

\begin{remark}
  The reader is invited to verify that $\int (P^{\underline{\perp}}[P]) = 0$.
\end{remark}

\subsection{Quantum mechanics}

Table \ref{tbl:core_qm_op_defns} gives the core operational definitions

\begin{table}[htp]\label{tbl:core_qm_op_defns}
  \center{
    \fbox{
      \begin{tabular}{c|c}
        quantum mechanics & process calculus \\
        \hline
        scalar & $x := \quotep{P}$ \\
        state vector & $\state{P} := P$ \\
        dual & $\state{P}^{*} := \event{P^{\underline{\perp}}} := \quotep{P^{\underline{\perp}}}[-]$ \\
        matrix & $ \Sigma_{\alpha} \state{P_{\alpha}}x_{\alpha}\event{Q_{\alpha}}$ \\
        vector addition & $\state{P} + \state{Q} := \state{P | Q}$ \\
        tensor product & $\state{P} \otimes \state{Q} := \state{P \otimes Q}$ \\
        inner product & $\innerprod{P}{Q} := \quotep{\int P^{\underline{\perp}}[Q]}$ \\
      \end{tabular}
    }
  }
  \caption{QM - operational definitions}
\end{table}

where

\begin{mathpar}
  \prmatrix{P}{Q} := \fprmatrix{P}{\quotep{\pzero}}{Q}
  \and
  \fprmatrix{P}{x}{Q} := (\state{P},x,\event{Q})
  \and
  (\fprmatrix{P}{x}{Q})(\state{R}) := x \cdot \innerprod{Q}{R} \cdot \state{P}
  \and
  (\fprmatrix{P}{x}{Q})(\event{R}) := x \cdot \innerprod{R}{P} \cdot \event{Q}
\end{mathpar}

\paragraph{Discussion}
As promised: vectors (aka states) are represented as processes; duals
as contextual duals; inner product definition should be compared with
standard inner product definition for ....

\begin{remark}
  Assuming $\int (P^{\underline{\perp}}[P]) = 0$, the reader is
  invited to verify that $(\fprmatrix{P}{x}{P})(\state{P}) = x \cdot \state{P}$.
\end{remark}

\begin{remark}
  The reader is invited to verify that $\innerprod{P}{Q}$ could
  equally well have been written $\quotep{\int \stackrel{\vee}{x}}$
  where $x = \event{P^{\underline{\perp}}}(Q)$.

  One of the motivations for this remark is that there is another way
  to factor these operations. We could package up evaluation in the dual:

  \begin{mathpar}
    \state{P}^{*} := \event{\int P^{\underline{\perp}}} := \quotep{\int P^{\underline{\perp}}}[-]
  \end{mathpar}

  and then have inner product defined by
  
  \begin{mathpar}
    \innerprod{P}{Q} := \event{P}(Q)
  \end{mathpar}

  Hopefully, experience with the calculations will provide guidance on
  the best factoring.
\end{remark}

\begin{remark}
  Assuming $\int (P^{\underline{\perp}}[P]) = 0$, the reader is
  invited to verify that $\forall P,Q. (\prmatrix{0}{Q})(\state{0}) =
  \state{0}$ and dually $(\prmatrix{P}{0})(\event{0}) = \event{0}$.
\end{remark}

\begin{remark}
  i'm a little worried that i don't (yet) have proper support for
  complex conjugacy. But, the observation above may give us a
  clue. According to Abramsky, it must be the case that the scalars
  are iso to the homset of the identity for the tensor -- which the
  observation above characterizes. 

  For now, we will simply bookmark the notion with $\overline{x}$.
\end{remark}

\subsubsection{Adjointness}

We need to give a definition of $(\cdot)^{\dagger}$ for matrices. The
obvious candidate definition is
\begin{mathpar}
(\Sigma_{\alpha}\fprmatrix{P_{\alpha}}{x_{\alpha}}{Q_{\alpha}})^{\dagger}
= \Sigma_{\alpha}\fprmatrix{(Q_{\alpha}^{\underline{\perp}})^{*}}{\overline{x}_{\alpha}}{P_{\alpha}^{\underline{\perp}}} 
\end{mathpar}

But, $(Q_{\alpha}^{\underline{\perp}})^{*}$ requires a name along
which to communicate the process to achieve the context application.

\subsubsection{Basis for a basis}
If processes label states and ``addition'' of states (a.k.a. vector
addition) is interpreted as parallel composition, what corresponds to
notions of linear independence and basis? Here, we recall that Yoshida
has developed a set of \emph{combinators} for an asynchronous verison
of Milner's $\pi$-calculus. These are a finite set of processes such
any process can be expressed as parallel composition of these
combinators together with liberal uses of the new operator and
replication. We can simply give a translation of these into the
present calculus and have reasonable expectation that the property
carries over. That is, that the resultant set allows to express all
processes via parallel composition. Note, however, that there is no
new operator or replication in this calculus. As a result, we expect
that the corresponding set is actually infinite. That is, we expect
that the space is actually infinite dimensional.

\begin{remark}
  The attentive reader may be a bit concerned. Certainly, the
  collection $S$, $K$ and $I$ is a finite set of
  combinators. Shouldn't we expect to see a finite set of combinators
  for an effectively equivalent system? i am very sympathetic to this
  critique and feel it warrants full attention. On the other hand, i
  also have in mind the following analogy. The natural numbers, as a
  monoid under addition, has exactly $1$ generator, while the natural
  numbers, as a monoid under multiplication, has countably many
  generators (the primes). We observe that the application of the
  lambda calculus is much less resource sensitive than the parallel
  composition of the $\pi$-calculus. Could it be the case that we have
  an analogy of the form
  
  \begin{mathpar}
    m + n : MN :: m*n : M|N
  \end{mathpar}

  giving a similar blow up in the set of ``primes''?  This is such a
  wonderful thought that, even if it's not true, i think it's worth
  writing down.
\end{remark}
 

\documentclass[12pt]{llncs}
%\documentclass{jktr}

\usepackage[pdftex]{hyperref}                   
\usepackage {listings}
\usepackage {mathpartir}
\usepackage{bcprules}
%\usepackage{listings}
                       
\usepackage{graphicx} 
%\usepackage[margins=2.5cm,nohead,nofoot]{geometry}
%\usepackage{geometry}
\usepackage{amsfonts}
\usepackage{amstext}
\usepackage{latexsym}
\usepackage{amssymb}
\usepackage{color}


%\include{myPreamble}
\include{qm2pi.local} 

%\ifpdf
%\usepackage[pdftex]{graphicx}
%\else
%\usepackage{graphicx}
%\fi

 % \ifpdf
%  \usepackage{pdfsync}
%  \if


%\title{Brief Article}
%\author{David F. Snyder}
%\author{L.G. Meredith}

%\address{Dept. of Math., Texas State University--San Marcos, San Marcos, TX 78666}
       
\pagestyle{empty}


\begin{document}

\lstset{language=[Objective]Caml,frame=shadowbox}

\input{qm2pi.front}

% section front matter (end)

\input{qm2pi.intro} 
 
% section introduction (end)

% \input{qm2pi.knotations} 

% section notation (end)

\input{qm2pi.process.calculi} 

% section concurrent_process_calculi_and_spatial_logics_ (end)
    
%\input{qm2pi.knots2pi} 

%\input{qm2pi.trefoil} 

%\input{qm2pi.mainthm} 

% subsection basic_interpretation (end)

%\input{qm2pi.rho.presentation} 
\subsection{The syntax and semantics of the notation system}\label{sub:the_syntax_and_semantics_of_the_notation_system} % (fold)

We now summarize a technical presentation of the calculus that
embodies our theory of dynamics. The typical presentation of such a
calculus follows the style of giving generators and relations on
them. The grammar, below, describing term constructors, freely
generates the set of processes, $\Proc$. This set is then quotiented
by a relation known as structural congruence and it is over this set
that the notion of dynamics is expressed. This presentation is
essentially that of \cite{MeredithR05} with the addition of
polyadicity and summation. For readability we have relegated some of
the technical subtleties to an appendix.

\subsubsection{Process grammar}\label{subsub:process_grammar}

\begin{mathpar}
  \inferrule* [lab=synchronization] {} {{M} \bc \pzero \;|\; x?F \;|\; x!C }
  \and
  \inferrule* [lab=abstraction] {} {{F} \bc (x)P}
  \and
  \inferrule* [lab=concretion] {} {{C} \bc \langle Q \rangle}
  \and
  \inferrule* [lab=process] {} {{P,Q} \bc M \;| \;P|Q \;|\; @{x}}
  \and
  \inferrule* [lab=name] {} {{x} \bc \quotep{P}}
\end{mathpar} 

Note that $\vec{x}$ (resp. $\vec{P}$) denotes a vector of names
(resp. processes) of length $|\vec{x}|$ (resp. $|\vec{P}|$). We adopt
the following useful abbreviations.

\begin{mathpar}
   x?(\vec{y}).P := x.(\vec{y})P \and  x\clift{\vec{P}} := x.\clift{\vec{P}}
   \and x!(y) := \lift{x}{\dropn{y}}
   \and \Pi_{i=0}^{n-1}P_i := P_0 | \ldots | P_{n-1}
\end{mathpar}

\subsubsection{Structural congruence}

\paragraph{Free and bound names and alpha-equivalence.} At the
core of structural equivalence is alpha-equivalence which identifies
process that are the same up to a change of variable. Formally, we
recognize the distinction between free and bound names. The free names
of a process, $\freenames{P}$, may be calculated recursively as
follows:

\begin{mathpar}
\freenames{\pzero} := \emptyset
  \and \\
  \freenames{x?(y).P} := \{ x \} \cup (\freenames{P} \setminus \{ y \})
  \and 
  \freenames{x!\langle P \rangle} := \{ x \} \cup \{ P \} 
  \and \\
  \freenames{P|Q} := \freenames{P} \cup \freenames{Q}
  \and \\
  \freenames{@{x}} := \{ x \}
\end{mathpar}

$\pi$
$\quotep{\pi}$

$\freenames{-} : \pi \to \mathcal{P}(\quotep{\pi})$

\begin{eqnarray*}
  \freenames{\pzero} & := & \emptyset \\
  \freenames{x?(y).P} & := & \{ x \} \cup (\freenames{P} \setminus \{ y \}) \\
  \freenames{x!\langle P \rangle} & := & \{ x \} \cup \{ P \} \\
  \freenames{P|Q} & := & \freenames{P} \cup \freenames{Q} \\
  \freenames{\dropn{x}} & := & \{ x \}
\end{eqnarray*}

The bound names of a process, $\boundnames{P}$, are those names occurring in $P$
that are not free. For example, in $x?(y).0$, the name $x$ is free, while $y$ is bound.

\begin{mathpar}
  \inferrule* [lab=monoidal-laws] {} { P|Q \equiv Q|P \and P|0 \equiv P \and P|(Q|R) \equiv (P|Q)|R }
\end{mathpar}

\begin{mathpar}
  \inferrule* [lab=alpha-equivalence] {} { (x)P \equiv (y)P\{y/x\} \and y \not\in \freenames{P} }
\end{mathpar}

\begin{definition}
Then two processes, $P,Q$, are alpha-equivalent if $P = Q\{\vec{y}/\vec{x}\}$ for
some $\vec{x} \in \boundnames{Q},\vec{y} \in \boundnames{P}$, where $Q\{\vec{y}/\vec{x}\}$
denotes the capture-avoiding substitution of $\vec{y}$ for $\vec{x}$ in $Q$.
\end{definition}

\begin{definition}
  The {\em structural congruence} \cite{SangiorgiWalker} , $\equiv$,
  between processes is the least congruence containing
  alpha-equivalence, satisfying the abelian monoid laws
  (associativity, commutativity and $\pzero$ as identity) for parallel
  composition $|$ and for summation $+$.
\end{definition}

\subsection{Name equivalence}

We take name equivalence, written $\nameeq$, to be the smallest
equivalence relation generated by the following rules.

\begin{mathpar}
\inferrule*[lab=Quote-drop]
{ }
{ \quotep{@{x}} \nameeq x }

\inferrule*[lab=Struct-equiv]
{ P \scong Q }
{ \quotep{P} \nameeq \quotep{Q} }
\end{mathpar}

The astute reader will have noticed that the mutual recursion of names
and processes imposes a mutual recursion on alpha-equivalence and
structural equivalence via name-equivalence. Fortunately, all of this
works out pleasantly and we may calculate in the natural way, free of
concern. The reader interested in the details is referred to the
appendix \ref{appendix:rho_details}.

\subsection{Substitution}

We use $\Proc$ for the set of processes, $\QProc$ for the set of
names, and $\id{\{}\vec{y} / \vec{x} \id{\}}$ to denote partial maps,
$s : \QProc \rightarrow \QProc$. A map, $s$ lifts, uniquely, to a map
on process terms, $\widehat{s} : \Proc \rightarrow \Proc$ by the
following equations.

\begin{mathpar}
  (0) \psubstp{Q}{P} := 0 \\
  (R \juxtap S) \psubstp{Q}{P}
  :=    
  (R)\psubstp{Q}{P} \juxtap (S) \psubstp{Q}{P} \\
  (x?(y).R) \psubstp{Q}{P}    
  :=    
  (x)\substp{Q}{P} (z)\concat( (R \psubstn{z}{y}) \psubstp{Q}{P} ) \\
  (\lift{x}{R}) \psubstp{Q}{P}  
  :=
  \lift{(x)\substp{Q}{P}}{ R \psubstp{Q}{P} } \\
%   (\dropn{x})  \psubstp{Q}{P}       
%   := 
%   \left\{ 
%     \begin{array}{ccc} 
%       \dropn{\quotep{Q}} & & x \nameeq \quotep{P} \\
%       \dropn{x} & & otherwise \\
%     \end{array}
%   \right. 
  (\dropn{x})  \psubstp{Q}{P}       
  := 
  \left\{ 
    \begin{array}{ccc} 
      Q & & x \nameeq \quotep{P} \\
      \dropn{x} & & otherwise \\
    \end{array}
  \right.
\end{mathpar}
 

where

\begin{eqnarray}
  (x)\id{\{} \lpquote Q \rpquote / \lpquote P \rpquote \id{\}}            = 
  \left\{ 
    \begin{array}{ccc}
      \lpquote Q \rpquote & & x \nameeq \lpquote P \rpquote \\
      x & & otherwise \\
    \end{array}
  \right. \nonumber
\end{eqnarray}

and $z$ is chosen distinct from $\quotep{P}$, $\quotep{Q}$, the free
names in $Q$, and all the names in $R$. Our $\alpha$-equivalence will
be built in the standard way from this substitution.

\begin{remark}\label{rem:no_self_referential_names}
  One consequence of these definitions is that $\forall P. \quotep{P}
  \not\in \freenames{P}$.
\end{remark}

\subsection{ Dynamic quote: an example }

Anticipating something of what's to come, consider applying the
substitution, $\widehat{\id{\{}u / z \id{\}}}$, to the following pair
of processes, $\lift{w}{y!(z)}$ and $w[ \lpquote y!(z) \rpquote ]$.

\begin{eqnarray}
	\lift{w}{y!(z)}\widehat{\id{\{}u / z \id{\}}}
		& = &
		\lift{w}{y!(u)} \nonumber\\
	w[ \lpquote y!(z) \rpquote ] \widehat{ \id{\{}u / z \id{\}} }
		& = &
		w[ \lpquote y!(z) \rpquote ] \nonumber
\end{eqnarray}

Because the body of the process between quotes is impervious to
substitution, we get radically different answers. In fact, by
examining the first process in an input context,
e.g. $x?(z).\lift{w}{y!(z)}$, we see that the process under the lift
operator may be shaped by prefixed inputs binding a name inside it. In
this sense, the lift operator will be seen as a way to dynamically
construct processes before reifying them as names.

Finally equipped with these standard features we can present the
dynamics of the calculus.

\subsubsection{Operational semantics} 

Finally, we introduce the computational dynamics. What marks these
algebras as distinct from other more traditionally studied algebraic
structures, e.g. vector spaces or polynomial rings, is the manner in
which dynamics is captured. In traditional structures, dynamics is typically
expressed through morphisms between such structures, as in linear maps
between vector spaces or morphisms between rings. In algebras
associated with the semantics of computation, the dynamics is
expressed as part of the algebraic structure itself, through a
reduction reduction relation typically denoted by $\red$. Below, we
give a recursive presentation of this relation for the calculus used
in the encoding.

$\red \subseteq \pi \times \pi$
$\red : \pi \to \mathcal{P}(\pi)$

\begin{mathpar}
  \inferrule* [lab=Comm] { \textsf{match}( x_{src}, x_{trgt} ) } { x_{trgt}?(y)P \; | \; x_{src}!\langle {Q} \rangle \red P\{\quotep{Q}/y}\} }
  \and \\
  \inferrule* [lab=Par] {{P} \red {P}'} {{{P} | {Q}} \red {{P}' | {Q}}}
  \and
  \inferrule* [lab=Equiv]{{{P} \scong {P}'} \andalso {{P}' \red {Q}'} \andalso {{Q}' \scong {Q}}}{{P} \red {Q}}
\end{mathpar}

\begin{eqnarray*}
  match_{\equiv} (\quotep{P},\quotep{Q}) & := & P \equiv Q \\
  match_{\dagger}(\quotep{P},\quotep{Q}) & := & \forall R. P|Q \red^{*} R => R \red^{*} 0 \\
  match_{K}(\quotep{P},\quotep{Q}) & := & K \mbox{ for some context } K
\end{eqnarray*}

$u?(x)P | u!\langle Q \rangle \red P\{\quotep{Q}/x\}$

%We write $\wred$ for $\red^*$, and $P\red$ if $\exists Q $ such that $ P \red Q$.
We write $P\red$ if $\exists Q $ such that $ P \red Q$ and $P\not\red$, otherwise.

\section{Replication}

As mentioned before, it is known that replication (and hence
recursion) can be implemented in a higher-order process algebra
\cite{SangiorgiWalker}. As our first example of calculation with the
machinery thus far presented we give the construction explicitly in
the {\rhoc}.

\begin{eqnarray}
	D_{x} & := & \prefix{x}{y}{(\binpar{\outputp{x}{y}}{@{y}})} \nonumber\\
	\bangp_{x}{P} & := & \binpar{{x}!\langle{\binpar{D_{x}}{P}}\rangle}{D_{x}} \nonumber
\end{eqnarray}

\begin{eqnarray}
	\bangp_{x}{P} & & \nonumber\\
	=
	& {x}!\langle{(\prefix{x}{y}{(\outputp{x}{y} | @{y})) | P}}\rangle 
	      | \prefix{x}{y}{(\outputp{x}{y} | @{y})} & \nonumber\\
	\red
	& (\outputp{x}{y} | @{y})\substn{\quotep{(\prefix{x}{y}{(@{y} | \outputp{x}{y})) | P}}}{y} & \nonumber\\
	=
	& \outputp{x}{\quotep{(\prefix{x}{y}{(\outputp{x}{y} | @{y})) | P}}}
	  | {(\prefix{x}{y}{(\outputp{x}{y} | @{y})) | P}} & \nonumber\\
	\red
	& \ldots & \nonumber\\
	\red^*
	& P | P | \ldots & \nonumber
\end{eqnarray}

Of course, this encoding, as an implementation, runs away, unfolding
$\bangp{P}$ eagerly. A lazier and more implementable replication
operator, restricted to input-guarded processes, may be obtained as follows.

\begin{eqnarray}
\bangp{\prefix{u}{v}{P}} 
	:= 
	\binpar{\lift{x}{\prefix{u}{v}{(\binpar{D(x)}{P})}}}{D(x)} \nonumber
\end{eqnarray}

\begin{remark}
  Note that the lazier definition still does not deal with summation
  or mixed summation (i.e. sums over input and output). The reader is
  invited to construct definitions of replication that deal with these
  features. 

  Further, the definitions are parameterized in a name, $x$. Can you,
  gentle reader, make a definition that eliminates this parameter and
  guarantees no accidental interaction between the replication
  machinery and the process being replicated -- i.e. no accidental
  sharing of names used by the process to get its work done and the
  name(s) used by the replication to effect copying. This latter
  revision of the definition of replication is crucial to obtaining
  the expected identity $!!P \sim !P$.
\end{remark}

\begin{remark}\label{rem:paradoxical_combinator}
  The reader familiar with the lambda calculus will have noticed the
  similarity between $D$ and the paradoxical combinator.

  [Ed. note: the existence of this seems to suggest we have to be more
  restrictive on the set of processes and names we admit if we are to
  support no-cloning.]
\end{remark}

\subsubsection{Bisimulation}

The computational dynamics gives rise to another kind of equivalence,
the equivalence of computational behavior. As previously mentioned
this is typically captured \emph{via} some form of bisimulation.

% The notion we use in this paper is weak barbed bisimulation
% \cite{milner91polyadicpi}.

The notion we use in this paper is derived from weak barbed
bisimulation \cite{milner91polyadicpi}. 

\begin{definition}
An \emph{observation relation}, $\downarrow_{\mathcal N}$, over a set
of names, $\mathcal N$, is the smallest relation satisfying the rules
below.

\infrule[Out-barb]{y \in {\mathcal N}, \; x \nameeq y}
		  {\outputp{x}{v} \downarrow_{\mathcal N} x}
\infrule[Par-barb]{\mbox{$P\downarrow_{\mathcal N} x$ or $Q\downarrow_{\mathcal N} x$}}
		  {\binpar{P}{Q} \downarrow_{\mathcal N} x}

We write $P \Downarrow_{\mathcal N} x$ if there is $Q$ such that 
$P \wred Q$ and $Q \downarrow_{\mathcal N} x$.
\end{definition}

\begin{definition}
%\label{def.bbisim}
An  ${\mathcal N}$-\emph{barbed bisimulation} over a set of names, ${\mathcal N}$, is a symmetric binary relation 
${\mathcal S}_{\mathcal N}$ between agents such that $P\rel{S}_{\mathcal N}Q$ implies:
\begin{enumerate}
\item If $P \red P'$ then $Q \wred Q'$ and $P'\rel{S}_{\mathcal N} Q'$.
\item If $P\downarrow_{\mathcal N} x$, then $Q\Downarrow_{\mathcal N} x$.
\end{enumerate}
$P$ is ${\mathcal N}$-barbed bisimilar to $Q$, written
$P \wbbisim_{\mathcal N} Q$, if $P \rel{S}_{\mathcal N} Q$ for some ${\mathcal N}$-barbed bisimulation ${\mathcal S}_{\mathcal N}$.
\end{definition}

$\mathcal{R} \subseteq \pi \times \pi$

$P \mathcal{R} Q => \forall P'. P \red P' \Rightarrow \exists Q'. Q \red Q', P' \mathcal{R} Q'$

$P \vdash x \Rightarrow Q \vdash x$

\begin{mathpar}
  \inferrule*[lab=Out-barb]{x \nameeq y}{{y}!\langle{Q}\rangle \vdash x}
  \and
  \inferrule*[lab=Par-barb]{\mbox{$P\vdash x$ or $Q\vdash x$}}{\binpar{P}{Q} \vdash x}
\end{mathpar}

\subsubsection{Contexts}

One of the principle advantages of computational calculi like the
$\pi$-calculus is a well-defined notion of context,
contextual-equivalence and a correlation between
contextual-equivalence and notions of bisimulation. The notion of
context allows the decomposition of a process into (sub-)process and
its syntactic environment, its context. Thus, a context may be
thought of as a process with a ``hole'' (written $\Box$) in it. The
application of a context $M$ to a process $P$, written $M[P]$, is
tantamount to filling the hole in $M$ with $P$. In this paper we do
not need the full weight of this theory, but do make use of the notion
of context in the proof the main theorem. 

\begin{mathpar}
  \inferrule* [lab=summation] {} {{M_{M},M_{N}} \bc \Box \;|\; x.M_{A} \;|\; M_{M}+M_{N}}
  \and
  \inferrule* [lab=agent] {} {{M_{A}} \bc (\vec{x})M_{P} \;| \; \clift{P_0,\ldots,M_{P},\ldots,P_N}}
  \and \\
  \inferrule* [lab=process] {} {{M_{P}} \bc M_{N} \;| \;P|M_{P} }
\end{mathpar} 

\begin{mathpar}
  \inferrule* [lab=sychronization] {} {M_{N} \bc \Box \;|\; x?M_{F} \;|\; x!M_{C}}
  \and
  \inferrule* [lab=abstraction] {} {{M_{F}} \bc (x)M_{P} }
  \and
  \inferrule* [lab=concretion] {} {{M_{C}} \bc \langle M_{P} \rangle }
  \and \\
  \inferrule* [lab=process] {} {{M_{P}} \bc M_{N} \;| \;P|M_{P} }
\end{mathpar}

\begin{definition}[contextual application] Given a context $M$, and
  process $P$, we define the \emph{contextual application}, $M[P] :=
  M\{P/\Box\}$. That is, the contextual application of M to P is the
  substitution of $P$ for $\Box$ in $M$.
\end{definition}

$\meaningof{-} : L \to \mathcal{P}(\pi)$

\begin{mathpar}
  \inferrule* [lab=collection] {} {\meaningof{true} = \pi, \and \meaningof{~E} = \pi \setminus \meaningof{E}, \and \meaningof{E_{1} \& E_{2}} = \meaningof{E_{1}} \cap \meaningof{E_{2}}}
\end{mathpar}

\begin{mathpar}
  \inferrule* [lab=structure] {} {\meaningof{0} = \{ P \in \pi | P \equiv 0 \}, \and \\ \meaningof{E_1 | E_2} = \{ P \in \pi | P \equiv P_{1} | P_{2}, P_{1} \in \meaningof{E_{1}}, P_{2} \in \meaningof{E_2}\} }
\end{mathpar}

\begin{mathpar}
 \inferrule* [lab=behavior] {} {\meaningof{\langle a?b \rangle E} = \{ P \in \pi | P \equiv Q | u?(y)P', \\ \and \\\\ \and \\ \;\;\; u \in \meaningof{a}, \forall z.P'\{z/y\} \in \meaningof{E\{z/b\}}\}, \and \\ \meaningof{a!E} = \{ P \in \pi | P \equiv Q | x!\langle P' \rangle, x \in \meaningof{a} P' \in \meaningof{E}\} }
\end{mathpar}

\begin{mathpar}
 \inferrule* [lab=nominal] {} {\meaningof{\quotep{E}} = \{ \quotep{P} \in \quotep{\pi} | P \in \meaningof{E} \}, \and \meaningof{\quotep{P}} = \{ \quotep{Q} \in \quotep{\pi} | P \equiv Q \} \and \\ \meaningof{@\quotep{E}} = \{ P \in \pi | P \equiv @x, x \in \meaningof{E} \}}
\end{mathpar}

\begin{eqnarray*}
  \\
  \meaningof{-} : TS \to ST
\end{eqnarray*}

\begin{eqnarray*}
  \\
  L : TS \to ST
\end{eqnarray*}

\begin{eqnarray*}
  \\
  P \models E \iff P \in \meaningof{E}
\end{eqnarray*}

\begin{eqnarray*}
  P \approx_{L} Q \iff \forall E \in L. P \models E \iff Q \models E
\end{eqnarray*}

\begin{eqnarray*}
  P \approx_{K} Q
\end{eqnarray*}

\begin{eqnarray*}
  P \approx Q
\end{eqnarray*}

$\approx_{K} = \approx = \approx_{L}$

\subsubsection{Contextual duality}

Note that contexts extend the quotation operation to a family of
operations from processes to names. Given a context, $M$, we can
define a \emph{nominal context}, $\quotep{M}$ by $\quotep{M}[P] :=
\quotep{M[P]}$. To foreshadow what is to come we observe that these
operations enjoy a duality with processes very much like the duality
between vectors and maps from vectors to scalars.

Further, because the calculus is essentially higher-order, we have a
correspondence between contexts and processes. More specifically,
given a name $x$ and a context $M$ we can construct $M^{*}_{x}$ such
that 

\begin{mathpar}
  M^{*}_{x} | \lift{x}{P} \red M[P]
\end{mathpar}

namely,

\begin{mathpar}
  M^{*}_{x} := x?(u).M[\dropn{u}]
\end{mathpar}

The dependence of $M^{*}_{x}$ on a name makes it an abstraction, 

\begin{mathpar}
  M^{*} := (x)x?(u).M[\dropn{u}]
\end{mathpar}

\subsection{Additional notation}

It will sometimes be convenient to denote the process a name
quotes. We already have the notation $x = \quotep{P}$, but it will be
convenient to introduce an alternate notation, $\procn{x}$, when we
want to emphasize the connection to the use of the name. Note that, by
virtue of name equivalence, $\quotep{\procn{x}} \nameeq x$; so, the
notation is consistent with previous definitions.

Further, because names have structure it is possible to effect
substitutions on the basis of that structure. This means we need to
upgrade our notation for substitutions, which we accomplish by
adapting comprehension notation. Thus,

\begin{mathpar}
  P\{ y / x : x \in S \}
\end{mathpar}

is interpreted to mean the process derived from P by replacing (in a
capture-avoiding manner) each occurrence of $x$ in $S$ by $y$. For example,

\begin{mathpar}
  P\{ \quotep{\procn{x}|\procn{x}} / x : x \in \freenames{P} \}
\end{mathpar}

will replace each (occurrence) of a free name $x$ in $P$ by
$\quotep{\procn{x}|\procn{x}}$.

Also, we will avail ourselves of the notation $x^{L}$ and $x^{R}$ to
denote injections of a name into disjoint copies of the name
space. There are numerous ways to accomplish this. One example can be
found in \cite{MeredithR05}. This notation overloads to vectors of
names: $\vec{x}^{\pi} := (x_{i}^{\pi} \; : \; 0 \leq i < |\vec{x}| )$ where $\pi \in \{L,R\}$.

We also use $P^{\Box} := P|\Box$.

In \cite{MeredithR05} an interpretation of the new operator is
given. It turns out that there are several possible interpretations
all enjoying the requisite algebraic properties of the operator (see
\cite{milner91polyadicpi}). We will therefore make liberal use of
$(\nu\; \vec{x})P$.

% subsection the_syntax_and_semantics_of_the_notation_system (end)   

\input{qm2pi.qmops} 

\input{qm2pi.sterngerlach} 

\input{qm2pi.metric} 

% section concurrent_process_calculi (end)

%\input{qm2pi.proofsketch}

% section proof sketch (end)

%\input{qm2pi.slviaknots} 

% section spatial logic via knots (end)

\input{qm2pi.conclusion}

% section conclusion (end)

%\input{qm2pi.dtcodes} 

% section wiring algorithm (end)

\input{qm2pi.ack} 

% section acknowledgments (end)

\newpage


\bibliographystyle{plain}   
\bibliography{../../biblios/main.bib}

\input{qm2pi.rhodetails}

\end{document}

 

\documentclass[12pt]{llncs}
%\documentclass{jktr}

\usepackage[pdftex]{hyperref}                   
\usepackage {listings}
\usepackage {mathpartir}
\usepackage{bcprules}
%\usepackage{listings}
                       
\usepackage{graphicx} 
%\usepackage[margins=2.5cm,nohead,nofoot]{geometry}
%\usepackage{geometry}
\usepackage{amsfonts}
\usepackage{amstext}
\usepackage{latexsym}
\usepackage{amssymb}
\usepackage{color}


%\include{myPreamble}
\include{qm2pi.local} 

%\ifpdf
%\usepackage[pdftex]{graphicx}
%\else
%\usepackage{graphicx}
%\fi

 % \ifpdf
%  \usepackage{pdfsync}
%  \if


%\title{Brief Article}
%\author{David F. Snyder}
%\author{L.G. Meredith}

%\address{Dept. of Math., Texas State University--San Marcos, San Marcos, TX 78666}
       
\pagestyle{empty}


\begin{document}

\lstset{language=[Objective]Caml,frame=shadowbox}

\input{qm2pi.front}

% section front matter (end)

\input{qm2pi.intro} 
 
% section introduction (end)

% \input{qm2pi.knotations} 

% section notation (end)

\input{qm2pi.process.calculi} 

% section concurrent_process_calculi_and_spatial_logics_ (end)
    
%\input{qm2pi.knots2pi} 

%\input{qm2pi.trefoil} 

%\input{qm2pi.mainthm} 

% subsection basic_interpretation (end)

%\input{qm2pi.rho.presentation} 
\subsection{The syntax and semantics of the notation system}\label{sub:the_syntax_and_semantics_of_the_notation_system} % (fold)

We now summarize a technical presentation of the calculus that
embodies our theory of dynamics. The typical presentation of such a
calculus follows the style of giving generators and relations on
them. The grammar, below, describing term constructors, freely
generates the set of processes, $\Proc$. This set is then quotiented
by a relation known as structural congruence and it is over this set
that the notion of dynamics is expressed. This presentation is
essentially that of \cite{MeredithR05} with the addition of
polyadicity and summation. For readability we have relegated some of
the technical subtleties to an appendix.

\subsubsection{Process grammar}\label{subsub:process_grammar}

\begin{mathpar}
  \inferrule* [lab=synchronization] {} {{M} \bc \pzero \;|\; x?F \;|\; x!C }
  \and
  \inferrule* [lab=abstraction] {} {{F} \bc (x)P}
  \and
  \inferrule* [lab=concretion] {} {{C} \bc \langle Q \rangle}
  \and
  \inferrule* [lab=process] {} {{P,Q} \bc M \;| \;P|Q \;|\; @{x}}
  \and
  \inferrule* [lab=name] {} {{x} \bc \quotep{P}}
\end{mathpar} 

Note that $\vec{x}$ (resp. $\vec{P}$) denotes a vector of names
(resp. processes) of length $|\vec{x}|$ (resp. $|\vec{P}|$). We adopt
the following useful abbreviations.

\begin{mathpar}
   x?(\vec{y}).P := x.(\vec{y})P \and  x\clift{\vec{P}} := x.\clift{\vec{P}}
   \and x!(y) := \lift{x}{\dropn{y}}
   \and \Pi_{i=0}^{n-1}P_i := P_0 | \ldots | P_{n-1}
\end{mathpar}

\subsubsection{Structural congruence}

\paragraph{Free and bound names and alpha-equivalence.} At the
core of structural equivalence is alpha-equivalence which identifies
process that are the same up to a change of variable. Formally, we
recognize the distinction between free and bound names. The free names
of a process, $\freenames{P}$, may be calculated recursively as
follows:

\begin{mathpar}
\freenames{\pzero} := \emptyset
  \and \\
  \freenames{x?(y).P} := \{ x \} \cup (\freenames{P} \setminus \{ y \})
  \and 
  \freenames{x!\langle P \rangle} := \{ x \} \cup \{ P \} 
  \and \\
  \freenames{P|Q} := \freenames{P} \cup \freenames{Q}
  \and \\
  \freenames{@{x}} := \{ x \}
\end{mathpar}

$\pi$
$\quotep{\pi}$

$\freenames{-} : \pi \to \mathcal{P}(\quotep{\pi})$

\begin{eqnarray*}
  \freenames{\pzero} & := & \emptyset \\
  \freenames{x?(y).P} & := & \{ x \} \cup (\freenames{P} \setminus \{ y \}) \\
  \freenames{x!\langle P \rangle} & := & \{ x \} \cup \{ P \} \\
  \freenames{P|Q} & := & \freenames{P} \cup \freenames{Q} \\
  \freenames{\dropn{x}} & := & \{ x \}
\end{eqnarray*}

The bound names of a process, $\boundnames{P}$, are those names occurring in $P$
that are not free. For example, in $x?(y).0$, the name $x$ is free, while $y$ is bound.

\begin{mathpar}
  \inferrule* [lab=monoidal-laws] {} { P|Q \equiv Q|P \and P|0 \equiv P \and P|(Q|R) \equiv (P|Q)|R }
\end{mathpar}

\begin{mathpar}
  \inferrule* [lab=alpha-equivalence] {} { (x)P \equiv (y)P\{y/x\} \and y \not\in \freenames{P} }
\end{mathpar}

\begin{definition}
Then two processes, $P,Q$, are alpha-equivalent if $P = Q\{\vec{y}/\vec{x}\}$ for
some $\vec{x} \in \boundnames{Q},\vec{y} \in \boundnames{P}$, where $Q\{\vec{y}/\vec{x}\}$
denotes the capture-avoiding substitution of $\vec{y}$ for $\vec{x}$ in $Q$.
\end{definition}

\begin{definition}
  The {\em structural congruence} \cite{SangiorgiWalker} , $\equiv$,
  between processes is the least congruence containing
  alpha-equivalence, satisfying the abelian monoid laws
  (associativity, commutativity and $\pzero$ as identity) for parallel
  composition $|$ and for summation $+$.
\end{definition}

\subsection{Name equivalence}

We take name equivalence, written $\nameeq$, to be the smallest
equivalence relation generated by the following rules.

\begin{mathpar}
\inferrule*[lab=Quote-drop]
{ }
{ \quotep{@{x}} \nameeq x }

\inferrule*[lab=Struct-equiv]
{ P \scong Q }
{ \quotep{P} \nameeq \quotep{Q} }
\end{mathpar}

The astute reader will have noticed that the mutual recursion of names
and processes imposes a mutual recursion on alpha-equivalence and
structural equivalence via name-equivalence. Fortunately, all of this
works out pleasantly and we may calculate in the natural way, free of
concern. The reader interested in the details is referred to the
appendix \ref{appendix:rho_details}.

\subsection{Substitution}

We use $\Proc$ for the set of processes, $\QProc$ for the set of
names, and $\id{\{}\vec{y} / \vec{x} \id{\}}$ to denote partial maps,
$s : \QProc \rightarrow \QProc$. A map, $s$ lifts, uniquely, to a map
on process terms, $\widehat{s} : \Proc \rightarrow \Proc$ by the
following equations.

\begin{mathpar}
  (0) \psubstp{Q}{P} := 0 \\
  (R \juxtap S) \psubstp{Q}{P}
  :=    
  (R)\psubstp{Q}{P} \juxtap (S) \psubstp{Q}{P} \\
  (x?(y).R) \psubstp{Q}{P}    
  :=    
  (x)\substp{Q}{P} (z)\concat( (R \psubstn{z}{y}) \psubstp{Q}{P} ) \\
  (\lift{x}{R}) \psubstp{Q}{P}  
  :=
  \lift{(x)\substp{Q}{P}}{ R \psubstp{Q}{P} } \\
%   (\dropn{x})  \psubstp{Q}{P}       
%   := 
%   \left\{ 
%     \begin{array}{ccc} 
%       \dropn{\quotep{Q}} & & x \nameeq \quotep{P} \\
%       \dropn{x} & & otherwise \\
%     \end{array}
%   \right. 
  (\dropn{x})  \psubstp{Q}{P}       
  := 
  \left\{ 
    \begin{array}{ccc} 
      Q & & x \nameeq \quotep{P} \\
      \dropn{x} & & otherwise \\
    \end{array}
  \right.
\end{mathpar}
 

where

\begin{eqnarray}
  (x)\id{\{} \lpquote Q \rpquote / \lpquote P \rpquote \id{\}}            = 
  \left\{ 
    \begin{array}{ccc}
      \lpquote Q \rpquote & & x \nameeq \lpquote P \rpquote \\
      x & & otherwise \\
    \end{array}
  \right. \nonumber
\end{eqnarray}

and $z$ is chosen distinct from $\quotep{P}$, $\quotep{Q}$, the free
names in $Q$, and all the names in $R$. Our $\alpha$-equivalence will
be built in the standard way from this substitution.

\begin{remark}\label{rem:no_self_referential_names}
  One consequence of these definitions is that $\forall P. \quotep{P}
  \not\in \freenames{P}$.
\end{remark}

\subsection{ Dynamic quote: an example }

Anticipating something of what's to come, consider applying the
substitution, $\widehat{\id{\{}u / z \id{\}}}$, to the following pair
of processes, $\lift{w}{y!(z)}$ and $w[ \lpquote y!(z) \rpquote ]$.

\begin{eqnarray}
	\lift{w}{y!(z)}\widehat{\id{\{}u / z \id{\}}}
		& = &
		\lift{w}{y!(u)} \nonumber\\
	w[ \lpquote y!(z) \rpquote ] \widehat{ \id{\{}u / z \id{\}} }
		& = &
		w[ \lpquote y!(z) \rpquote ] \nonumber
\end{eqnarray}

Because the body of the process between quotes is impervious to
substitution, we get radically different answers. In fact, by
examining the first process in an input context,
e.g. $x?(z).\lift{w}{y!(z)}$, we see that the process under the lift
operator may be shaped by prefixed inputs binding a name inside it. In
this sense, the lift operator will be seen as a way to dynamically
construct processes before reifying them as names.

Finally equipped with these standard features we can present the
dynamics of the calculus.

\subsubsection{Operational semantics} 

Finally, we introduce the computational dynamics. What marks these
algebras as distinct from other more traditionally studied algebraic
structures, e.g. vector spaces or polynomial rings, is the manner in
which dynamics is captured. In traditional structures, dynamics is typically
expressed through morphisms between such structures, as in linear maps
between vector spaces or morphisms between rings. In algebras
associated with the semantics of computation, the dynamics is
expressed as part of the algebraic structure itself, through a
reduction reduction relation typically denoted by $\red$. Below, we
give a recursive presentation of this relation for the calculus used
in the encoding.

$\red \subseteq \pi \times \pi$
$\red : \pi \to \mathcal{P}(\pi)$

\begin{mathpar}
  \inferrule* [lab=Comm] { \textsf{match}( x_{src}, x_{trgt} ) } { x_{trgt}?(y)P \; | \; x_{src}!\langle {Q} \rangle \red P\{\quotep{Q}/y}\} }
  \and \\
  \inferrule* [lab=Par] {{P} \red {P}'} {{{P} | {Q}} \red {{P}' | {Q}}}
  \and
  \inferrule* [lab=Equiv]{{{P} \scong {P}'} \andalso {{P}' \red {Q}'} \andalso {{Q}' \scong {Q}}}{{P} \red {Q}}
\end{mathpar}

\begin{eqnarray*}
  match_{\equiv} (\quotep{P},\quotep{Q}) & := & P \equiv Q \\
  match_{\dagger}(\quotep{P},\quotep{Q}) & := & \forall R. P|Q \red^{*} R => R \red^{*} 0 \\
  match_{K}(\quotep{P},\quotep{Q}) & := & K \mbox{ for some context } K
\end{eqnarray*}

$u?(x)P | u!\langle Q \rangle \red P\{\quotep{Q}/x\}$

%We write $\wred$ for $\red^*$, and $P\red$ if $\exists Q $ such that $ P \red Q$.
We write $P\red$ if $\exists Q $ such that $ P \red Q$ and $P\not\red$, otherwise.

\section{Replication}

As mentioned before, it is known that replication (and hence
recursion) can be implemented in a higher-order process algebra
\cite{SangiorgiWalker}. As our first example of calculation with the
machinery thus far presented we give the construction explicitly in
the {\rhoc}.

\begin{eqnarray}
	D_{x} & := & \prefix{x}{y}{(\binpar{\outputp{x}{y}}{@{y}})} \nonumber\\
	\bangp_{x}{P} & := & \binpar{{x}!\langle{\binpar{D_{x}}{P}}\rangle}{D_{x}} \nonumber
\end{eqnarray}

\begin{eqnarray}
	\bangp_{x}{P} & & \nonumber\\
	=
	& {x}!\langle{(\prefix{x}{y}{(\outputp{x}{y} | @{y})) | P}}\rangle 
	      | \prefix{x}{y}{(\outputp{x}{y} | @{y})} & \nonumber\\
	\red
	& (\outputp{x}{y} | @{y})\substn{\quotep{(\prefix{x}{y}{(@{y} | \outputp{x}{y})) | P}}}{y} & \nonumber\\
	=
	& \outputp{x}{\quotep{(\prefix{x}{y}{(\outputp{x}{y} | @{y})) | P}}}
	  | {(\prefix{x}{y}{(\outputp{x}{y} | @{y})) | P}} & \nonumber\\
	\red
	& \ldots & \nonumber\\
	\red^*
	& P | P | \ldots & \nonumber
\end{eqnarray}

Of course, this encoding, as an implementation, runs away, unfolding
$\bangp{P}$ eagerly. A lazier and more implementable replication
operator, restricted to input-guarded processes, may be obtained as follows.

\begin{eqnarray}
\bangp{\prefix{u}{v}{P}} 
	:= 
	\binpar{\lift{x}{\prefix{u}{v}{(\binpar{D(x)}{P})}}}{D(x)} \nonumber
\end{eqnarray}

\begin{remark}
  Note that the lazier definition still does not deal with summation
  or mixed summation (i.e. sums over input and output). The reader is
  invited to construct definitions of replication that deal with these
  features. 

  Further, the definitions are parameterized in a name, $x$. Can you,
  gentle reader, make a definition that eliminates this parameter and
  guarantees no accidental interaction between the replication
  machinery and the process being replicated -- i.e. no accidental
  sharing of names used by the process to get its work done and the
  name(s) used by the replication to effect copying. This latter
  revision of the definition of replication is crucial to obtaining
  the expected identity $!!P \sim !P$.
\end{remark}

\begin{remark}\label{rem:paradoxical_combinator}
  The reader familiar with the lambda calculus will have noticed the
  similarity between $D$ and the paradoxical combinator.

  [Ed. note: the existence of this seems to suggest we have to be more
  restrictive on the set of processes and names we admit if we are to
  support no-cloning.]
\end{remark}

\subsubsection{Bisimulation}

The computational dynamics gives rise to another kind of equivalence,
the equivalence of computational behavior. As previously mentioned
this is typically captured \emph{via} some form of bisimulation.

% The notion we use in this paper is weak barbed bisimulation
% \cite{milner91polyadicpi}.

The notion we use in this paper is derived from weak barbed
bisimulation \cite{milner91polyadicpi}. 

\begin{definition}
An \emph{observation relation}, $\downarrow_{\mathcal N}$, over a set
of names, $\mathcal N$, is the smallest relation satisfying the rules
below.

\infrule[Out-barb]{y \in {\mathcal N}, \; x \nameeq y}
		  {\outputp{x}{v} \downarrow_{\mathcal N} x}
\infrule[Par-barb]{\mbox{$P\downarrow_{\mathcal N} x$ or $Q\downarrow_{\mathcal N} x$}}
		  {\binpar{P}{Q} \downarrow_{\mathcal N} x}

We write $P \Downarrow_{\mathcal N} x$ if there is $Q$ such that 
$P \wred Q$ and $Q \downarrow_{\mathcal N} x$.
\end{definition}

\begin{definition}
%\label{def.bbisim}
An  ${\mathcal N}$-\emph{barbed bisimulation} over a set of names, ${\mathcal N}$, is a symmetric binary relation 
${\mathcal S}_{\mathcal N}$ between agents such that $P\rel{S}_{\mathcal N}Q$ implies:
\begin{enumerate}
\item If $P \red P'$ then $Q \wred Q'$ and $P'\rel{S}_{\mathcal N} Q'$.
\item If $P\downarrow_{\mathcal N} x$, then $Q\Downarrow_{\mathcal N} x$.
\end{enumerate}
$P$ is ${\mathcal N}$-barbed bisimilar to $Q$, written
$P \wbbisim_{\mathcal N} Q$, if $P \rel{S}_{\mathcal N} Q$ for some ${\mathcal N}$-barbed bisimulation ${\mathcal S}_{\mathcal N}$.
\end{definition}

$\mathcal{R} \subseteq \pi \times \pi$

$P \mathcal{R} Q => \forall P'. P \red P' \Rightarrow \exists Q'. Q \red Q', P' \mathcal{R} Q'$

$P \vdash x \Rightarrow Q \vdash x$

\begin{mathpar}
  \inferrule*[lab=Out-barb]{x \nameeq y}{{y}!\langle{Q}\rangle \vdash x}
  \and
  \inferrule*[lab=Par-barb]{\mbox{$P\vdash x$ or $Q\vdash x$}}{\binpar{P}{Q} \vdash x}
\end{mathpar}

\subsubsection{Contexts}

One of the principle advantages of computational calculi like the
$\pi$-calculus is a well-defined notion of context,
contextual-equivalence and a correlation between
contextual-equivalence and notions of bisimulation. The notion of
context allows the decomposition of a process into (sub-)process and
its syntactic environment, its context. Thus, a context may be
thought of as a process with a ``hole'' (written $\Box$) in it. The
application of a context $M$ to a process $P$, written $M[P]$, is
tantamount to filling the hole in $M$ with $P$. In this paper we do
not need the full weight of this theory, but do make use of the notion
of context in the proof the main theorem. 

\begin{mathpar}
  \inferrule* [lab=summation] {} {{M_{M},M_{N}} \bc \Box \;|\; x.M_{A} \;|\; M_{M}+M_{N}}
  \and
  \inferrule* [lab=agent] {} {{M_{A}} \bc (\vec{x})M_{P} \;| \; \clift{P_0,\ldots,M_{P},\ldots,P_N}}
  \and \\
  \inferrule* [lab=process] {} {{M_{P}} \bc M_{N} \;| \;P|M_{P} }
\end{mathpar} 

\begin{mathpar}
  \inferrule* [lab=sychronization] {} {M_{N} \bc \Box \;|\; x?M_{F} \;|\; x!M_{C}}
  \and
  \inferrule* [lab=abstraction] {} {{M_{F}} \bc (x)M_{P} }
  \and
  \inferrule* [lab=concretion] {} {{M_{C}} \bc \langle M_{P} \rangle }
  \and \\
  \inferrule* [lab=process] {} {{M_{P}} \bc M_{N} \;| \;P|M_{P} }
\end{mathpar}

\begin{definition}[contextual application] Given a context $M$, and
  process $P$, we define the \emph{contextual application}, $M[P] :=
  M\{P/\Box\}$. That is, the contextual application of M to P is the
  substitution of $P$ for $\Box$ in $M$.
\end{definition}

$\meaningof{-} : L \to \mathcal{P}(\pi)$

\begin{mathpar}
  \inferrule* [lab=collection] {} {\meaningof{true} = \pi, \and \meaningof{~E} = \pi \setminus \meaningof{E}, \and \meaningof{E_{1} \& E_{2}} = \meaningof{E_{1}} \cap \meaningof{E_{2}}}
\end{mathpar}

\begin{mathpar}
  \inferrule* [lab=structure] {} {\meaningof{0} = \{ P \in \pi | P \equiv 0 \}, \and \\ \meaningof{E_1 | E_2} = \{ P \in \pi | P \equiv P_{1} | P_{2}, P_{1} \in \meaningof{E_{1}}, P_{2} \in \meaningof{E_2}\} }
\end{mathpar}

\begin{mathpar}
 \inferrule* [lab=behavior] {} {\meaningof{\langle a?b \rangle E} = \{ P \in \pi | P \equiv Q | u?(y)P', \\ \and \\\\ \and \\ \;\;\; u \in \meaningof{a}, \forall z.P'\{z/y\} \in \meaningof{E\{z/b\}}\}, \and \\ \meaningof{a!E} = \{ P \in \pi | P \equiv Q | x!\langle P' \rangle, x \in \meaningof{a} P' \in \meaningof{E}\} }
\end{mathpar}

\begin{mathpar}
 \inferrule* [lab=nominal] {} {\meaningof{\quotep{E}} = \{ \quotep{P} \in \quotep{\pi} | P \in \meaningof{E} \}, \and \meaningof{\quotep{P}} = \{ \quotep{Q} \in \quotep{\pi} | P \equiv Q \} \and \\ \meaningof{@\quotep{E}} = \{ P \in \pi | P \equiv @x, x \in \meaningof{E} \}}
\end{mathpar}

\begin{eqnarray*}
  \\
  \meaningof{-} : TS \to ST
\end{eqnarray*}

\begin{eqnarray*}
  \\
  L : TS \to ST
\end{eqnarray*}

\begin{eqnarray*}
  \\
  P \models E \iff P \in \meaningof{E}
\end{eqnarray*}

\begin{eqnarray*}
  P \approx_{L} Q \iff \forall E \in L. P \models E \iff Q \models E
\end{eqnarray*}

\begin{eqnarray*}
  P \approx_{K} Q
\end{eqnarray*}

\begin{eqnarray*}
  P \approx Q
\end{eqnarray*}

$\approx_{K} = \approx = \approx_{L}$

\subsubsection{Contextual duality}

Note that contexts extend the quotation operation to a family of
operations from processes to names. Given a context, $M$, we can
define a \emph{nominal context}, $\quotep{M}$ by $\quotep{M}[P] :=
\quotep{M[P]}$. To foreshadow what is to come we observe that these
operations enjoy a duality with processes very much like the duality
between vectors and maps from vectors to scalars.

Further, because the calculus is essentially higher-order, we have a
correspondence between contexts and processes. More specifically,
given a name $x$ and a context $M$ we can construct $M^{*}_{x}$ such
that 

\begin{mathpar}
  M^{*}_{x} | \lift{x}{P} \red M[P]
\end{mathpar}

namely,

\begin{mathpar}
  M^{*}_{x} := x?(u).M[\dropn{u}]
\end{mathpar}

The dependence of $M^{*}_{x}$ on a name makes it an abstraction, 

\begin{mathpar}
  M^{*} := (x)x?(u).M[\dropn{u}]
\end{mathpar}

\subsection{Additional notation}

It will sometimes be convenient to denote the process a name
quotes. We already have the notation $x = \quotep{P}$, but it will be
convenient to introduce an alternate notation, $\procn{x}$, when we
want to emphasize the connection to the use of the name. Note that, by
virtue of name equivalence, $\quotep{\procn{x}} \nameeq x$; so, the
notation is consistent with previous definitions.

Further, because names have structure it is possible to effect
substitutions on the basis of that structure. This means we need to
upgrade our notation for substitutions, which we accomplish by
adapting comprehension notation. Thus,

\begin{mathpar}
  P\{ y / x : x \in S \}
\end{mathpar}

is interpreted to mean the process derived from P by replacing (in a
capture-avoiding manner) each occurrence of $x$ in $S$ by $y$. For example,

\begin{mathpar}
  P\{ \quotep{\procn{x}|\procn{x}} / x : x \in \freenames{P} \}
\end{mathpar}

will replace each (occurrence) of a free name $x$ in $P$ by
$\quotep{\procn{x}|\procn{x}}$.

Also, we will avail ourselves of the notation $x^{L}$ and $x^{R}$ to
denote injections of a name into disjoint copies of the name
space. There are numerous ways to accomplish this. One example can be
found in \cite{MeredithR05}. This notation overloads to vectors of
names: $\vec{x}^{\pi} := (x_{i}^{\pi} \; : \; 0 \leq i < |\vec{x}| )$ where $\pi \in \{L,R\}$.

We also use $P^{\Box} := P|\Box$.

In \cite{MeredithR05} an interpretation of the new operator is
given. It turns out that there are several possible interpretations
all enjoying the requisite algebraic properties of the operator (see
\cite{milner91polyadicpi}). We will therefore make liberal use of
$(\nu\; \vec{x})P$.

% subsection the_syntax_and_semantics_of_the_notation_system (end)   

\input{qm2pi.qmops} 

\input{qm2pi.sterngerlach} 

\input{qm2pi.metric} 

% section concurrent_process_calculi (end)

%\input{qm2pi.proofsketch}

% section proof sketch (end)

%\input{qm2pi.slviaknots} 

% section spatial logic via knots (end)

\input{qm2pi.conclusion}

% section conclusion (end)

%\input{qm2pi.dtcodes} 

% section wiring algorithm (end)

\input{qm2pi.ack} 

% section acknowledgments (end)

\newpage


\bibliographystyle{plain}   
\bibliography{../../biblios/main.bib}

\input{qm2pi.rhodetails}

\end{document}

 

% section concurrent_process_calculi (end)

%\documentclass[12pt]{llncs}
%\documentclass{jktr}

\usepackage[pdftex]{hyperref}                   
\usepackage {listings}
\usepackage {mathpartir}
\usepackage{bcprules}
%\usepackage{listings}
                       
\usepackage{graphicx} 
%\usepackage[margins=2.5cm,nohead,nofoot]{geometry}
%\usepackage{geometry}
\usepackage{amsfonts}
\usepackage{amstext}
\usepackage{latexsym}
\usepackage{amssymb}
\usepackage{color}


%\include{myPreamble}
\include{qm2pi.local} 

%\ifpdf
%\usepackage[pdftex]{graphicx}
%\else
%\usepackage{graphicx}
%\fi

 % \ifpdf
%  \usepackage{pdfsync}
%  \if


%\title{Brief Article}
%\author{David F. Snyder}
%\author{L.G. Meredith}

%\address{Dept. of Math., Texas State University--San Marcos, San Marcos, TX 78666}
       
\pagestyle{empty}


\begin{document}

\lstset{language=[Objective]Caml,frame=shadowbox}

\input{qm2pi.front}

% section front matter (end)

\input{qm2pi.intro} 
 
% section introduction (end)

% \input{qm2pi.knotations} 

% section notation (end)

\input{qm2pi.process.calculi} 

% section concurrent_process_calculi_and_spatial_logics_ (end)
    
%\input{qm2pi.knots2pi} 

%\input{qm2pi.trefoil} 

%\input{qm2pi.mainthm} 

% subsection basic_interpretation (end)

%\input{qm2pi.rho.presentation} 
\subsection{The syntax and semantics of the notation system}\label{sub:the_syntax_and_semantics_of_the_notation_system} % (fold)

We now summarize a technical presentation of the calculus that
embodies our theory of dynamics. The typical presentation of such a
calculus follows the style of giving generators and relations on
them. The grammar, below, describing term constructors, freely
generates the set of processes, $\Proc$. This set is then quotiented
by a relation known as structural congruence and it is over this set
that the notion of dynamics is expressed. This presentation is
essentially that of \cite{MeredithR05} with the addition of
polyadicity and summation. For readability we have relegated some of
the technical subtleties to an appendix.

\subsubsection{Process grammar}\label{subsub:process_grammar}

\begin{mathpar}
  \inferrule* [lab=synchronization] {} {{M} \bc \pzero \;|\; x?F \;|\; x!C }
  \and
  \inferrule* [lab=abstraction] {} {{F} \bc (x)P}
  \and
  \inferrule* [lab=concretion] {} {{C} \bc \langle Q \rangle}
  \and
  \inferrule* [lab=process] {} {{P,Q} \bc M \;| \;P|Q \;|\; @{x}}
  \and
  \inferrule* [lab=name] {} {{x} \bc \quotep{P}}
\end{mathpar} 

Note that $\vec{x}$ (resp. $\vec{P}$) denotes a vector of names
(resp. processes) of length $|\vec{x}|$ (resp. $|\vec{P}|$). We adopt
the following useful abbreviations.

\begin{mathpar}
   x?(\vec{y}).P := x.(\vec{y})P \and  x\clift{\vec{P}} := x.\clift{\vec{P}}
   \and x!(y) := \lift{x}{\dropn{y}}
   \and \Pi_{i=0}^{n-1}P_i := P_0 | \ldots | P_{n-1}
\end{mathpar}

\subsubsection{Structural congruence}

\paragraph{Free and bound names and alpha-equivalence.} At the
core of structural equivalence is alpha-equivalence which identifies
process that are the same up to a change of variable. Formally, we
recognize the distinction between free and bound names. The free names
of a process, $\freenames{P}$, may be calculated recursively as
follows:

\begin{mathpar}
\freenames{\pzero} := \emptyset
  \and \\
  \freenames{x?(y).P} := \{ x \} \cup (\freenames{P} \setminus \{ y \})
  \and 
  \freenames{x!\langle P \rangle} := \{ x \} \cup \{ P \} 
  \and \\
  \freenames{P|Q} := \freenames{P} \cup \freenames{Q}
  \and \\
  \freenames{@{x}} := \{ x \}
\end{mathpar}

$\pi$
$\quotep{\pi}$

$\freenames{-} : \pi \to \mathcal{P}(\quotep{\pi})$

\begin{eqnarray*}
  \freenames{\pzero} & := & \emptyset \\
  \freenames{x?(y).P} & := & \{ x \} \cup (\freenames{P} \setminus \{ y \}) \\
  \freenames{x!\langle P \rangle} & := & \{ x \} \cup \{ P \} \\
  \freenames{P|Q} & := & \freenames{P} \cup \freenames{Q} \\
  \freenames{\dropn{x}} & := & \{ x \}
\end{eqnarray*}

The bound names of a process, $\boundnames{P}$, are those names occurring in $P$
that are not free. For example, in $x?(y).0$, the name $x$ is free, while $y$ is bound.

\begin{mathpar}
  \inferrule* [lab=monoidal-laws] {} { P|Q \equiv Q|P \and P|0 \equiv P \and P|(Q|R) \equiv (P|Q)|R }
\end{mathpar}

\begin{mathpar}
  \inferrule* [lab=alpha-equivalence] {} { (x)P \equiv (y)P\{y/x\} \and y \not\in \freenames{P} }
\end{mathpar}

\begin{definition}
Then two processes, $P,Q$, are alpha-equivalent if $P = Q\{\vec{y}/\vec{x}\}$ for
some $\vec{x} \in \boundnames{Q},\vec{y} \in \boundnames{P}$, where $Q\{\vec{y}/\vec{x}\}$
denotes the capture-avoiding substitution of $\vec{y}$ for $\vec{x}$ in $Q$.
\end{definition}

\begin{definition}
  The {\em structural congruence} \cite{SangiorgiWalker} , $\equiv$,
  between processes is the least congruence containing
  alpha-equivalence, satisfying the abelian monoid laws
  (associativity, commutativity and $\pzero$ as identity) for parallel
  composition $|$ and for summation $+$.
\end{definition}

\subsection{Name equivalence}

We take name equivalence, written $\nameeq$, to be the smallest
equivalence relation generated by the following rules.

\begin{mathpar}
\inferrule*[lab=Quote-drop]
{ }
{ \quotep{@{x}} \nameeq x }

\inferrule*[lab=Struct-equiv]
{ P \scong Q }
{ \quotep{P} \nameeq \quotep{Q} }
\end{mathpar}

The astute reader will have noticed that the mutual recursion of names
and processes imposes a mutual recursion on alpha-equivalence and
structural equivalence via name-equivalence. Fortunately, all of this
works out pleasantly and we may calculate in the natural way, free of
concern. The reader interested in the details is referred to the
appendix \ref{appendix:rho_details}.

\subsection{Substitution}

We use $\Proc$ for the set of processes, $\QProc$ for the set of
names, and $\id{\{}\vec{y} / \vec{x} \id{\}}$ to denote partial maps,
$s : \QProc \rightarrow \QProc$. A map, $s$ lifts, uniquely, to a map
on process terms, $\widehat{s} : \Proc \rightarrow \Proc$ by the
following equations.

\begin{mathpar}
  (0) \psubstp{Q}{P} := 0 \\
  (R \juxtap S) \psubstp{Q}{P}
  :=    
  (R)\psubstp{Q}{P} \juxtap (S) \psubstp{Q}{P} \\
  (x?(y).R) \psubstp{Q}{P}    
  :=    
  (x)\substp{Q}{P} (z)\concat( (R \psubstn{z}{y}) \psubstp{Q}{P} ) \\
  (\lift{x}{R}) \psubstp{Q}{P}  
  :=
  \lift{(x)\substp{Q}{P}}{ R \psubstp{Q}{P} } \\
%   (\dropn{x})  \psubstp{Q}{P}       
%   := 
%   \left\{ 
%     \begin{array}{ccc} 
%       \dropn{\quotep{Q}} & & x \nameeq \quotep{P} \\
%       \dropn{x} & & otherwise \\
%     \end{array}
%   \right. 
  (\dropn{x})  \psubstp{Q}{P}       
  := 
  \left\{ 
    \begin{array}{ccc} 
      Q & & x \nameeq \quotep{P} \\
      \dropn{x} & & otherwise \\
    \end{array}
  \right.
\end{mathpar}
 

where

\begin{eqnarray}
  (x)\id{\{} \lpquote Q \rpquote / \lpquote P \rpquote \id{\}}            = 
  \left\{ 
    \begin{array}{ccc}
      \lpquote Q \rpquote & & x \nameeq \lpquote P \rpquote \\
      x & & otherwise \\
    \end{array}
  \right. \nonumber
\end{eqnarray}

and $z$ is chosen distinct from $\quotep{P}$, $\quotep{Q}$, the free
names in $Q$, and all the names in $R$. Our $\alpha$-equivalence will
be built in the standard way from this substitution.

\begin{remark}\label{rem:no_self_referential_names}
  One consequence of these definitions is that $\forall P. \quotep{P}
  \not\in \freenames{P}$.
\end{remark}

\subsection{ Dynamic quote: an example }

Anticipating something of what's to come, consider applying the
substitution, $\widehat{\id{\{}u / z \id{\}}}$, to the following pair
of processes, $\lift{w}{y!(z)}$ and $w[ \lpquote y!(z) \rpquote ]$.

\begin{eqnarray}
	\lift{w}{y!(z)}\widehat{\id{\{}u / z \id{\}}}
		& = &
		\lift{w}{y!(u)} \nonumber\\
	w[ \lpquote y!(z) \rpquote ] \widehat{ \id{\{}u / z \id{\}} }
		& = &
		w[ \lpquote y!(z) \rpquote ] \nonumber
\end{eqnarray}

Because the body of the process between quotes is impervious to
substitution, we get radically different answers. In fact, by
examining the first process in an input context,
e.g. $x?(z).\lift{w}{y!(z)}$, we see that the process under the lift
operator may be shaped by prefixed inputs binding a name inside it. In
this sense, the lift operator will be seen as a way to dynamically
construct processes before reifying them as names.

Finally equipped with these standard features we can present the
dynamics of the calculus.

\subsubsection{Operational semantics} 

Finally, we introduce the computational dynamics. What marks these
algebras as distinct from other more traditionally studied algebraic
structures, e.g. vector spaces or polynomial rings, is the manner in
which dynamics is captured. In traditional structures, dynamics is typically
expressed through morphisms between such structures, as in linear maps
between vector spaces or morphisms between rings. In algebras
associated with the semantics of computation, the dynamics is
expressed as part of the algebraic structure itself, through a
reduction reduction relation typically denoted by $\red$. Below, we
give a recursive presentation of this relation for the calculus used
in the encoding.

$\red \subseteq \pi \times \pi$
$\red : \pi \to \mathcal{P}(\pi)$

\begin{mathpar}
  \inferrule* [lab=Comm] { \textsf{match}( x_{src}, x_{trgt} ) } { x_{trgt}?(y)P \; | \; x_{src}!\langle {Q} \rangle \red P\{\quotep{Q}/y}\} }
  \and \\
  \inferrule* [lab=Par] {{P} \red {P}'} {{{P} | {Q}} \red {{P}' | {Q}}}
  \and
  \inferrule* [lab=Equiv]{{{P} \scong {P}'} \andalso {{P}' \red {Q}'} \andalso {{Q}' \scong {Q}}}{{P} \red {Q}}
\end{mathpar}

\begin{eqnarray*}
  match_{\equiv} (\quotep{P},\quotep{Q}) & := & P \equiv Q \\
  match_{\dagger}(\quotep{P},\quotep{Q}) & := & \forall R. P|Q \red^{*} R => R \red^{*} 0 \\
  match_{K}(\quotep{P},\quotep{Q}) & := & K \mbox{ for some context } K
\end{eqnarray*}

$u?(x)P | u!\langle Q \rangle \red P\{\quotep{Q}/x\}$

%We write $\wred$ for $\red^*$, and $P\red$ if $\exists Q $ such that $ P \red Q$.
We write $P\red$ if $\exists Q $ such that $ P \red Q$ and $P\not\red$, otherwise.

\section{Replication}

As mentioned before, it is known that replication (and hence
recursion) can be implemented in a higher-order process algebra
\cite{SangiorgiWalker}. As our first example of calculation with the
machinery thus far presented we give the construction explicitly in
the {\rhoc}.

\begin{eqnarray}
	D_{x} & := & \prefix{x}{y}{(\binpar{\outputp{x}{y}}{@{y}})} \nonumber\\
	\bangp_{x}{P} & := & \binpar{{x}!\langle{\binpar{D_{x}}{P}}\rangle}{D_{x}} \nonumber
\end{eqnarray}

\begin{eqnarray}
	\bangp_{x}{P} & & \nonumber\\
	=
	& {x}!\langle{(\prefix{x}{y}{(\outputp{x}{y} | @{y})) | P}}\rangle 
	      | \prefix{x}{y}{(\outputp{x}{y} | @{y})} & \nonumber\\
	\red
	& (\outputp{x}{y} | @{y})\substn{\quotep{(\prefix{x}{y}{(@{y} | \outputp{x}{y})) | P}}}{y} & \nonumber\\
	=
	& \outputp{x}{\quotep{(\prefix{x}{y}{(\outputp{x}{y} | @{y})) | P}}}
	  | {(\prefix{x}{y}{(\outputp{x}{y} | @{y})) | P}} & \nonumber\\
	\red
	& \ldots & \nonumber\\
	\red^*
	& P | P | \ldots & \nonumber
\end{eqnarray}

Of course, this encoding, as an implementation, runs away, unfolding
$\bangp{P}$ eagerly. A lazier and more implementable replication
operator, restricted to input-guarded processes, may be obtained as follows.

\begin{eqnarray}
\bangp{\prefix{u}{v}{P}} 
	:= 
	\binpar{\lift{x}{\prefix{u}{v}{(\binpar{D(x)}{P})}}}{D(x)} \nonumber
\end{eqnarray}

\begin{remark}
  Note that the lazier definition still does not deal with summation
  or mixed summation (i.e. sums over input and output). The reader is
  invited to construct definitions of replication that deal with these
  features. 

  Further, the definitions are parameterized in a name, $x$. Can you,
  gentle reader, make a definition that eliminates this parameter and
  guarantees no accidental interaction between the replication
  machinery and the process being replicated -- i.e. no accidental
  sharing of names used by the process to get its work done and the
  name(s) used by the replication to effect copying. This latter
  revision of the definition of replication is crucial to obtaining
  the expected identity $!!P \sim !P$.
\end{remark}

\begin{remark}\label{rem:paradoxical_combinator}
  The reader familiar with the lambda calculus will have noticed the
  similarity between $D$ and the paradoxical combinator.

  [Ed. note: the existence of this seems to suggest we have to be more
  restrictive on the set of processes and names we admit if we are to
  support no-cloning.]
\end{remark}

\subsubsection{Bisimulation}

The computational dynamics gives rise to another kind of equivalence,
the equivalence of computational behavior. As previously mentioned
this is typically captured \emph{via} some form of bisimulation.

% The notion we use in this paper is weak barbed bisimulation
% \cite{milner91polyadicpi}.

The notion we use in this paper is derived from weak barbed
bisimulation \cite{milner91polyadicpi}. 

\begin{definition}
An \emph{observation relation}, $\downarrow_{\mathcal N}$, over a set
of names, $\mathcal N$, is the smallest relation satisfying the rules
below.

\infrule[Out-barb]{y \in {\mathcal N}, \; x \nameeq y}
		  {\outputp{x}{v} \downarrow_{\mathcal N} x}
\infrule[Par-barb]{\mbox{$P\downarrow_{\mathcal N} x$ or $Q\downarrow_{\mathcal N} x$}}
		  {\binpar{P}{Q} \downarrow_{\mathcal N} x}

We write $P \Downarrow_{\mathcal N} x$ if there is $Q$ such that 
$P \wred Q$ and $Q \downarrow_{\mathcal N} x$.
\end{definition}

\begin{definition}
%\label{def.bbisim}
An  ${\mathcal N}$-\emph{barbed bisimulation} over a set of names, ${\mathcal N}$, is a symmetric binary relation 
${\mathcal S}_{\mathcal N}$ between agents such that $P\rel{S}_{\mathcal N}Q$ implies:
\begin{enumerate}
\item If $P \red P'$ then $Q \wred Q'$ and $P'\rel{S}_{\mathcal N} Q'$.
\item If $P\downarrow_{\mathcal N} x$, then $Q\Downarrow_{\mathcal N} x$.
\end{enumerate}
$P$ is ${\mathcal N}$-barbed bisimilar to $Q$, written
$P \wbbisim_{\mathcal N} Q$, if $P \rel{S}_{\mathcal N} Q$ for some ${\mathcal N}$-barbed bisimulation ${\mathcal S}_{\mathcal N}$.
\end{definition}

$\mathcal{R} \subseteq \pi \times \pi$

$P \mathcal{R} Q => \forall P'. P \red P' \Rightarrow \exists Q'. Q \red Q', P' \mathcal{R} Q'$

$P \vdash x \Rightarrow Q \vdash x$

\begin{mathpar}
  \inferrule*[lab=Out-barb]{x \nameeq y}{{y}!\langle{Q}\rangle \vdash x}
  \and
  \inferrule*[lab=Par-barb]{\mbox{$P\vdash x$ or $Q\vdash x$}}{\binpar{P}{Q} \vdash x}
\end{mathpar}

\subsubsection{Contexts}

One of the principle advantages of computational calculi like the
$\pi$-calculus is a well-defined notion of context,
contextual-equivalence and a correlation between
contextual-equivalence and notions of bisimulation. The notion of
context allows the decomposition of a process into (sub-)process and
its syntactic environment, its context. Thus, a context may be
thought of as a process with a ``hole'' (written $\Box$) in it. The
application of a context $M$ to a process $P$, written $M[P]$, is
tantamount to filling the hole in $M$ with $P$. In this paper we do
not need the full weight of this theory, but do make use of the notion
of context in the proof the main theorem. 

\begin{mathpar}
  \inferrule* [lab=summation] {} {{M_{M},M_{N}} \bc \Box \;|\; x.M_{A} \;|\; M_{M}+M_{N}}
  \and
  \inferrule* [lab=agent] {} {{M_{A}} \bc (\vec{x})M_{P} \;| \; \clift{P_0,\ldots,M_{P},\ldots,P_N}}
  \and \\
  \inferrule* [lab=process] {} {{M_{P}} \bc M_{N} \;| \;P|M_{P} }
\end{mathpar} 

\begin{mathpar}
  \inferrule* [lab=sychronization] {} {M_{N} \bc \Box \;|\; x?M_{F} \;|\; x!M_{C}}
  \and
  \inferrule* [lab=abstraction] {} {{M_{F}} \bc (x)M_{P} }
  \and
  \inferrule* [lab=concretion] {} {{M_{C}} \bc \langle M_{P} \rangle }
  \and \\
  \inferrule* [lab=process] {} {{M_{P}} \bc M_{N} \;| \;P|M_{P} }
\end{mathpar}

\begin{definition}[contextual application] Given a context $M$, and
  process $P$, we define the \emph{contextual application}, $M[P] :=
  M\{P/\Box\}$. That is, the contextual application of M to P is the
  substitution of $P$ for $\Box$ in $M$.
\end{definition}

$\meaningof{-} : L \to \mathcal{P}(\pi)$

\begin{mathpar}
  \inferrule* [lab=collection] {} {\meaningof{true} = \pi, \and \meaningof{~E} = \pi \setminus \meaningof{E}, \and \meaningof{E_{1} \& E_{2}} = \meaningof{E_{1}} \cap \meaningof{E_{2}}}
\end{mathpar}

\begin{mathpar}
  \inferrule* [lab=structure] {} {\meaningof{0} = \{ P \in \pi | P \equiv 0 \}, \and \\ \meaningof{E_1 | E_2} = \{ P \in \pi | P \equiv P_{1} | P_{2}, P_{1} \in \meaningof{E_{1}}, P_{2} \in \meaningof{E_2}\} }
\end{mathpar}

\begin{mathpar}
 \inferrule* [lab=behavior] {} {\meaningof{\langle a?b \rangle E} = \{ P \in \pi | P \equiv Q | u?(y)P', \\ \and \\\\ \and \\ \;\;\; u \in \meaningof{a}, \forall z.P'\{z/y\} \in \meaningof{E\{z/b\}}\}, \and \\ \meaningof{a!E} = \{ P \in \pi | P \equiv Q | x!\langle P' \rangle, x \in \meaningof{a} P' \in \meaningof{E}\} }
\end{mathpar}

\begin{mathpar}
 \inferrule* [lab=nominal] {} {\meaningof{\quotep{E}} = \{ \quotep{P} \in \quotep{\pi} | P \in \meaningof{E} \}, \and \meaningof{\quotep{P}} = \{ \quotep{Q} \in \quotep{\pi} | P \equiv Q \} \and \\ \meaningof{@\quotep{E}} = \{ P \in \pi | P \equiv @x, x \in \meaningof{E} \}}
\end{mathpar}

\begin{eqnarray*}
  \\
  \meaningof{-} : TS \to ST
\end{eqnarray*}

\begin{eqnarray*}
  \\
  L : TS \to ST
\end{eqnarray*}

\begin{eqnarray*}
  \\
  P \models E \iff P \in \meaningof{E}
\end{eqnarray*}

\begin{eqnarray*}
  P \approx_{L} Q \iff \forall E \in L. P \models E \iff Q \models E
\end{eqnarray*}

\begin{eqnarray*}
  P \approx_{K} Q
\end{eqnarray*}

\begin{eqnarray*}
  P \approx Q
\end{eqnarray*}

$\approx_{K} = \approx = \approx_{L}$

\subsubsection{Contextual duality}

Note that contexts extend the quotation operation to a family of
operations from processes to names. Given a context, $M$, we can
define a \emph{nominal context}, $\quotep{M}$ by $\quotep{M}[P] :=
\quotep{M[P]}$. To foreshadow what is to come we observe that these
operations enjoy a duality with processes very much like the duality
between vectors and maps from vectors to scalars.

Further, because the calculus is essentially higher-order, we have a
correspondence between contexts and processes. More specifically,
given a name $x$ and a context $M$ we can construct $M^{*}_{x}$ such
that 

\begin{mathpar}
  M^{*}_{x} | \lift{x}{P} \red M[P]
\end{mathpar}

namely,

\begin{mathpar}
  M^{*}_{x} := x?(u).M[\dropn{u}]
\end{mathpar}

The dependence of $M^{*}_{x}$ on a name makes it an abstraction, 

\begin{mathpar}
  M^{*} := (x)x?(u).M[\dropn{u}]
\end{mathpar}

\subsection{Additional notation}

It will sometimes be convenient to denote the process a name
quotes. We already have the notation $x = \quotep{P}$, but it will be
convenient to introduce an alternate notation, $\procn{x}$, when we
want to emphasize the connection to the use of the name. Note that, by
virtue of name equivalence, $\quotep{\procn{x}} \nameeq x$; so, the
notation is consistent with previous definitions.

Further, because names have structure it is possible to effect
substitutions on the basis of that structure. This means we need to
upgrade our notation for substitutions, which we accomplish by
adapting comprehension notation. Thus,

\begin{mathpar}
  P\{ y / x : x \in S \}
\end{mathpar}

is interpreted to mean the process derived from P by replacing (in a
capture-avoiding manner) each occurrence of $x$ in $S$ by $y$. For example,

\begin{mathpar}
  P\{ \quotep{\procn{x}|\procn{x}} / x : x \in \freenames{P} \}
\end{mathpar}

will replace each (occurrence) of a free name $x$ in $P$ by
$\quotep{\procn{x}|\procn{x}}$.

Also, we will avail ourselves of the notation $x^{L}$ and $x^{R}$ to
denote injections of a name into disjoint copies of the name
space. There are numerous ways to accomplish this. One example can be
found in \cite{MeredithR05}. This notation overloads to vectors of
names: $\vec{x}^{\pi} := (x_{i}^{\pi} \; : \; 0 \leq i < |\vec{x}| )$ where $\pi \in \{L,R\}$.

We also use $P^{\Box} := P|\Box$.

In \cite{MeredithR05} an interpretation of the new operator is
given. It turns out that there are several possible interpretations
all enjoying the requisite algebraic properties of the operator (see
\cite{milner91polyadicpi}). We will therefore make liberal use of
$(\nu\; \vec{x})P$.

% subsection the_syntax_and_semantics_of_the_notation_system (end)   

\input{qm2pi.qmops} 

\input{qm2pi.sterngerlach} 

\input{qm2pi.metric} 

% section concurrent_process_calculi (end)

%\input{qm2pi.proofsketch}

% section proof sketch (end)

%\input{qm2pi.slviaknots} 

% section spatial logic via knots (end)

\input{qm2pi.conclusion}

% section conclusion (end)

%\input{qm2pi.dtcodes} 

% section wiring algorithm (end)

\input{qm2pi.ack} 

% section acknowledgments (end)

\newpage


\bibliographystyle{plain}   
\bibliography{../../biblios/main.bib}

\input{qm2pi.rhodetails}

\end{document}



% section proof sketch (end)

%\section{Unlikely characters: spatial logic for
  knots}\label{sub:characteristic_formulae} % (fold)

Associated to the mobile process calculi are a family of logics known
as the Hennessy-Milner logics. These logics typically enjoy a
semantics interpreting formulae as sets of processes that when
factored through the encoding outlined above allows an identification
of classes of knots with logical formulae. In the context of this
encoding the sub-family known as the spatial logics \cite{CairesC03}
\cite{CairesC04} \cite{Caires04} are of particular interest providing
several important features for expressing and reasoning about
properties (i.e. classes) of knots. We hint here at how this may be done.

%\begin{description}
%\item [structural connectives] 
\subsubsection{Structural connectives} The spatial logics enjoy
structural connectives corresponding, at the logical level, to the
parallel composition ($P | Q$) and new name ($(\nu \; x)P$)
connectives for processes. As illustrated in the examples below, these
connectives are extremely expressive given the shape of our encoding.
%\item [decideable satisfaction]

\subsubsection{Decideable satisfaction}
In \cite{Caires04} the satisfaction relation is shown to be decideable
for a rich class of processes. It further turns out that the image of
the our encoding is a proper subset of that class. This result
provides the basis for an algorithm by which to search for knots
enjoying a given property.
%\item [characteristic formulae]

\subsubsection{Characteristic formulae}
In the same paper \cite{Caires04} , Caires presents a means of calculating
characteristic formulae, selecting equivalence classes of processes
up to a pre--specified depth limit on the support set of names. Composed with our
encoding, this characteristic formula can be used to select
characteristic formulae for knots.
%\end{description}

\subsubsection{Spatial logic formulae}

The grammar below (segmented for comprehension) summarizes the syntax
of spatial logic formulae. We employ illustrative examples in the
sequel to provide an intuitive understanding of their meaning
referring the reader to \cite{Caires04} for a more detailed explication
of the semantics.

\begin{mathpar}
  \inferrule* [lab=boolean] {} {{A,B} \bc T \;|\; \neg A \;|\; A \wedge B \;|\; \eta = \eta'}
  \and
  \inferrule* [lab=spatial] {} {|\; \pzero \;|\; A | B \;|\; x \text{\textregistered} A \;|\; \forall x . A \;|\;  H x . A}
  \and
  \inferrule* [lab=behavioral] {} {|\; \alpha . A}
  \and 
  \inferrule* [lab=recursion] {} {|\; X(\vec{u}) \;|\; \mu X(\vec{u}) . A}
  \and
  \inferrule* [lab=action] {} {\alpha \bc \langle x?(\vec{y}) \rangle \;|\; \langle x!(\vec{y}) \rangle \;|\; \langle \tau \rangle}
  \and 
  \inferrule* [lab=name] {} {\eta \bc x \;|\; \tau}
\end{mathpar} 

% subsection characteristic_formulae (end)   	 

\subsection{Example formulae}\label{sub:example_formulae_} % (fold)

\subsubsection{Crossing as formula.}
% 
% \begin{align*}
%   \frac{d}{dx} \sin x &= \cos x 
%   & \frac{d}{dx} e^x &= e^x \\
%   \frac{d}{dx} \cos x &= - \sin x 
%   & \frac{d}{dx} \log x &= \frac{1}{x} \\
% \end{align*} 

\begin{align*}
 \mu C(x_{0},x_{1},y_{0},y_{1},u).&(\langle x_{0}?(z) \rangle(\langle u! \rangle\langle y_{1}!z \rangle C(x_{0},x_{1},y_{0},y_{1},u)) & \\
  & \wedge \langle y_{1}?(z) \rangle (\langle u! \rangle \langle x_{0}!z \rangle C(x_{0},x_{1},y_{0},y_{1},u)) & \\
  & \wedge \langle x_{1}?(z) \rangle (\langle u? \rangle \langle y_{0}!z \rangle C(x_{0},x_{1},y_{0},y_{1},u)) & \\
  & \wedge \langle y_{0}?(z) \rangle (\langle u? \rangle \langle x_{1}!z \rangle C(x_{0},x_{1},y_{0},y_{1},u))) &
\end{align*}

The lexicographical similarity between the shape of this formulae and
the shape of definition of the process representing a crossing reveals
the intuitive meaning of this formulae. It describes the capabilities
of a process that has the right to represent a crossing. For example
it picks out processes that may perform an input on the port $x_0$ in
its initial menu of capabilities. What differentiates the formula
from the process, however, is that the crossing process is the
smallest candidate to satisfy the formula. Infinitely many other
processes -- with internal behavior hidden behind this interface, so
to speak -- also satisfy this formula. Even this simple formula,
then, can be seen to open a new view onto knots, providing a
computational interpretation of \emph{virtual} knots.

Note that this formula is derived by hand. A similar formula can be
derived by employing Caires' calculation of characteristic formula
\cite{Caires04} to the process representing a crossing. In light of
this discussion, we let
$\meaningof{C}_{\phi}(x0,x1,y0,y1,u)$ denote a formula specifying the
dynamics we wish to capture of a crossing. To guarantee we preserve
the shape of the interface and minimal semantics we demand that
$\meaningof{C}_{\phi}(x0,x1,y0,y1,u) \Rightarrow
\textbf{C}(x0,x1,y0,y1,u)$ where $\textbf{C}(x0,x1,y0,y1,u)$ denotes
the formula above.
                            
\subsubsection{Crossing number constraints.}
The moral content of the context lemma (Lemma \ref{context}) is that the notion of
``locality'' in the Reidemeister moves is effectively captured by the
parallel composition operator of the process calculus. This intuition
extends through the logic. Given a formula,
$\meaningof{C}_{\phi}(x0,x1,y0,y1,u)$, we can use the structural
connectives to specify constraints on crossing numbers, such as at
least $n$ crossings, or exactly $n$ crossings.
\begin{mathpar}
  \inferrule* [lab=at-least-n] {} { K^{\geq n}_{\phi}(\vec{xs},\vec{ys}) := \Pi_{i=0}^{n-1} Hu . \meaningof{C}_{\phi}(xs_i,ys_i,u) | T }
  \and 
  \inferrule* [lab=exactly-n] {} { K^{= n}_{\phi}(\vec{xs},\vec{ys}) := \Pi_{i=0}^{n-1} Hu . \meaningof{C}_{\phi}(xs_i,ys_i,u) | \neg (\forall x_0,y_0,x_1,y_1,u . \meaningof{C}_{\phi}(x_0,y_0,x_1,y_1,u) | T) }
\end{mathpar}

To round out this section, recall that the encoding of an $n$-crossing
knot decomposes into a parallel composition of $n$ \emph{copies} of a
crossing process together with a wiring harness. To specify different
knot classes with the same crossing number amounts to specifying
logical constraints on the wiring harness. In the interest of space,
we defer examples to a forthcoming paper. Suffice it to say that both
the conditions ``alternating knot'' and ``contains the tangle
corresponding to 5/3'' are expressible. For example, it is possible to
calculate the characteristic formula of a process corresponding to the
tangle 5/3 and conjoin it into the classifying formula via the
composition connective of the logic.

Finally, we wish to observe that it is entirely within reason to
contemplate a more domain-specific version of spatial logic tailored
to the shape of processes in the image of the encoding. Such a
domain-specific logic would have a better claim to the title formal
language of knot properties.

% subsection example_formulae_ (end)

% section knots_as_processes (end) 

% section spatial logic via knots (end)

\section{Conclusions and future work}

\paragraph{Testing physical space}
You, gentle reader, may wonder why of all the theorems to be proved
given this set up we pick the one above. In some sense it's hardly
central to quantum mechanics. We see it as central in the sense that
it firmly establishes a notion of physical space arising from a notion
of the equivalence of behavior. Relating bisimulation to a metric is a
big step forward, but one is faced with interpreting the relationship
of that metric space to something more physical. Quantum mechanical
notions of ``physical'' space are still far from intuitive, but by
relating this idea of distance as testing to calculations that predict
physical circumstances we are making a not insignificant step forward
toward an understanding of the physical space we inhabit as
essentially dynamic.

\paragraph{Effectivity and simulation}
One of the observations we have yet to make is that the entire program
spelled out here is effective. We have built various interpreters for
the reflective calculus at work in this interpretation. In principle,
then, we can simulate quantum mechanics on a computer. The place where
the simulation may lose fidelity is the infinitely branching summation
for the annihilator.

In this connection i also want to point out that the evaluation style
calculation of the inner product puts the non-determinism of the
summation right at the heart of measurement. This suggests that
Milner's original reduction-based formulation of the dynamics of his
calculi in terms of sums was not just notationally suggestive of a
notion of measure-and-continue but captured some significant part of
the physics.

\paragraph{Quantum continuations}
In light of this last observation i want to point out that the
predominant account of quantum mechanics is missing a key aspect of a
truly compositional story of the physical situation. In a real lab,
when a measurement is made the observation can be made to feed into
another device that then makes another measurement conditioned on the
results of the first. This means that after the superposition was
collapsed the entire experimental set up remained in
superposition. While QM offers a means of writing this down it doesn't
quite line up well with the well-trodden formulation of computation
and continuation that we see so succinctly expressed in Milner's
calculi. This suggests that there might be advantages to this account
of dynamics waiting to be explored.

\paragraph{Quantum logic}
In this connection, we also note that by virtue of having the
Hennessy-Milner construction, we can pull the construction through the
interpretation of QM. This gives us a natural candidate for a quantum
logic that enjoys an extremely tight connection with it's domain of
interpretation, making the construction much less ad hoc (rather it is
the image of functor!).

\paragraph{Quantum probabiity}
i have questions about the basis of the interpretation of inner
product as probability amplitude. In particular, using which
axiomatization of probability theory does the notion of probability
amplitude earn the right to be so dubbed? In other words, where is the
proof that the operation for calculating a probability amplitude (and
then squaring) satisfies the axioms of what it means to calculate a
probability? Even if such a proof exists (i have yet to find it in the
literature), i wonder if it might not be possible to turn things on
their heads. Can we view the calculation of the probability amplitude
as an axiomatization of probability? If so, then the definition we
give for calculating probability amplitude may provide the basis for
an \emph{effective} theory of probability.

\paragraph{Quantum vs ``biological'' information}
Finally, i want to conclude with a more philosophical observation. At
a recent workshop in which QM was a predominant topic i noticed
something about quantum information. The speaker was giving a riveting
discussion of axiomatic QM and showing how properties of ``no
cloning'' and ``no deleting'' emerged as consequences of the
axiomatization. Theorems of this form are necessary to give us a sense
of confidence that our axioms characterize the physical theory. What
struck me, though, was that if quantum information is neither erasable
nor replicable it is markedly different from \emph{life}. Two of the
things we know about life is that

\begin{itemize}
  \item it ends;
  \item to gain some measure of persistence, to transcend it's
    finitude it is imminently copyable.
\end{itemize}

Both of these qualities are summarized succinctly in the aphorism: all
flesh is grass. For me these two kinds of ``information'' -- call them
quantum and biological -- are end points on a spectrum of strategies
for persistence. At one end, we have those curious entities that enjoy
uniqueness and permanence; at the other, we have those who in the face
of a certain end and an uncertain present make a go of passing
something on. To me one of the more remarkable aspects of the latter
strategy is that in the presence of noise (and certain features of
copying) we get a kind of dynamism, a chance for improvement against a
given persistent condition.

% subsection other_calculi_other_bisimulations_and_geometry_as_behavior (end)




% section conclusion (end)

%\documentclass[12pt]{llncs}
%\documentclass{jktr}

\usepackage[pdftex]{hyperref}                   
\usepackage {listings}
\usepackage {mathpartir}
\usepackage{bcprules}
%\usepackage{listings}
                       
\usepackage{graphicx} 
%\usepackage[margins=2.5cm,nohead,nofoot]{geometry}
%\usepackage{geometry}
\usepackage{amsfonts}
\usepackage{amstext}
\usepackage{latexsym}
\usepackage{amssymb}
\usepackage{color}


%\include{myPreamble}
\include{qm2pi.local} 

%\ifpdf
%\usepackage[pdftex]{graphicx}
%\else
%\usepackage{graphicx}
%\fi

 % \ifpdf
%  \usepackage{pdfsync}
%  \if


%\title{Brief Article}
%\author{David F. Snyder}
%\author{L.G. Meredith}

%\address{Dept. of Math., Texas State University--San Marcos, San Marcos, TX 78666}
       
\pagestyle{empty}


\begin{document}

\lstset{language=[Objective]Caml,frame=shadowbox}

\input{qm2pi.front}

% section front matter (end)

\input{qm2pi.intro} 
 
% section introduction (end)

% \input{qm2pi.knotations} 

% section notation (end)

\input{qm2pi.process.calculi} 

% section concurrent_process_calculi_and_spatial_logics_ (end)
    
%\input{qm2pi.knots2pi} 

%\input{qm2pi.trefoil} 

%\input{qm2pi.mainthm} 

% subsection basic_interpretation (end)

%\input{qm2pi.rho.presentation} 
\subsection{The syntax and semantics of the notation system}\label{sub:the_syntax_and_semantics_of_the_notation_system} % (fold)

We now summarize a technical presentation of the calculus that
embodies our theory of dynamics. The typical presentation of such a
calculus follows the style of giving generators and relations on
them. The grammar, below, describing term constructors, freely
generates the set of processes, $\Proc$. This set is then quotiented
by a relation known as structural congruence and it is over this set
that the notion of dynamics is expressed. This presentation is
essentially that of \cite{MeredithR05} with the addition of
polyadicity and summation. For readability we have relegated some of
the technical subtleties to an appendix.

\subsubsection{Process grammar}\label{subsub:process_grammar}

\begin{mathpar}
  \inferrule* [lab=synchronization] {} {{M} \bc \pzero \;|\; x?F \;|\; x!C }
  \and
  \inferrule* [lab=abstraction] {} {{F} \bc (x)P}
  \and
  \inferrule* [lab=concretion] {} {{C} \bc \langle Q \rangle}
  \and
  \inferrule* [lab=process] {} {{P,Q} \bc M \;| \;P|Q \;|\; @{x}}
  \and
  \inferrule* [lab=name] {} {{x} \bc \quotep{P}}
\end{mathpar} 

Note that $\vec{x}$ (resp. $\vec{P}$) denotes a vector of names
(resp. processes) of length $|\vec{x}|$ (resp. $|\vec{P}|$). We adopt
the following useful abbreviations.

\begin{mathpar}
   x?(\vec{y}).P := x.(\vec{y})P \and  x\clift{\vec{P}} := x.\clift{\vec{P}}
   \and x!(y) := \lift{x}{\dropn{y}}
   \and \Pi_{i=0}^{n-1}P_i := P_0 | \ldots | P_{n-1}
\end{mathpar}

\subsubsection{Structural congruence}

\paragraph{Free and bound names and alpha-equivalence.} At the
core of structural equivalence is alpha-equivalence which identifies
process that are the same up to a change of variable. Formally, we
recognize the distinction between free and bound names. The free names
of a process, $\freenames{P}$, may be calculated recursively as
follows:

\begin{mathpar}
\freenames{\pzero} := \emptyset
  \and \\
  \freenames{x?(y).P} := \{ x \} \cup (\freenames{P} \setminus \{ y \})
  \and 
  \freenames{x!\langle P \rangle} := \{ x \} \cup \{ P \} 
  \and \\
  \freenames{P|Q} := \freenames{P} \cup \freenames{Q}
  \and \\
  \freenames{@{x}} := \{ x \}
\end{mathpar}

$\pi$
$\quotep{\pi}$

$\freenames{-} : \pi \to \mathcal{P}(\quotep{\pi})$

\begin{eqnarray*}
  \freenames{\pzero} & := & \emptyset \\
  \freenames{x?(y).P} & := & \{ x \} \cup (\freenames{P} \setminus \{ y \}) \\
  \freenames{x!\langle P \rangle} & := & \{ x \} \cup \{ P \} \\
  \freenames{P|Q} & := & \freenames{P} \cup \freenames{Q} \\
  \freenames{\dropn{x}} & := & \{ x \}
\end{eqnarray*}

The bound names of a process, $\boundnames{P}$, are those names occurring in $P$
that are not free. For example, in $x?(y).0$, the name $x$ is free, while $y$ is bound.

\begin{mathpar}
  \inferrule* [lab=monoidal-laws] {} { P|Q \equiv Q|P \and P|0 \equiv P \and P|(Q|R) \equiv (P|Q)|R }
\end{mathpar}

\begin{mathpar}
  \inferrule* [lab=alpha-equivalence] {} { (x)P \equiv (y)P\{y/x\} \and y \not\in \freenames{P} }
\end{mathpar}

\begin{definition}
Then two processes, $P,Q$, are alpha-equivalent if $P = Q\{\vec{y}/\vec{x}\}$ for
some $\vec{x} \in \boundnames{Q},\vec{y} \in \boundnames{P}$, where $Q\{\vec{y}/\vec{x}\}$
denotes the capture-avoiding substitution of $\vec{y}$ for $\vec{x}$ in $Q$.
\end{definition}

\begin{definition}
  The {\em structural congruence} \cite{SangiorgiWalker} , $\equiv$,
  between processes is the least congruence containing
  alpha-equivalence, satisfying the abelian monoid laws
  (associativity, commutativity and $\pzero$ as identity) for parallel
  composition $|$ and for summation $+$.
\end{definition}

\subsection{Name equivalence}

We take name equivalence, written $\nameeq$, to be the smallest
equivalence relation generated by the following rules.

\begin{mathpar}
\inferrule*[lab=Quote-drop]
{ }
{ \quotep{@{x}} \nameeq x }

\inferrule*[lab=Struct-equiv]
{ P \scong Q }
{ \quotep{P} \nameeq \quotep{Q} }
\end{mathpar}

The astute reader will have noticed that the mutual recursion of names
and processes imposes a mutual recursion on alpha-equivalence and
structural equivalence via name-equivalence. Fortunately, all of this
works out pleasantly and we may calculate in the natural way, free of
concern. The reader interested in the details is referred to the
appendix \ref{appendix:rho_details}.

\subsection{Substitution}

We use $\Proc$ for the set of processes, $\QProc$ for the set of
names, and $\id{\{}\vec{y} / \vec{x} \id{\}}$ to denote partial maps,
$s : \QProc \rightarrow \QProc$. A map, $s$ lifts, uniquely, to a map
on process terms, $\widehat{s} : \Proc \rightarrow \Proc$ by the
following equations.

\begin{mathpar}
  (0) \psubstp{Q}{P} := 0 \\
  (R \juxtap S) \psubstp{Q}{P}
  :=    
  (R)\psubstp{Q}{P} \juxtap (S) \psubstp{Q}{P} \\
  (x?(y).R) \psubstp{Q}{P}    
  :=    
  (x)\substp{Q}{P} (z)\concat( (R \psubstn{z}{y}) \psubstp{Q}{P} ) \\
  (\lift{x}{R}) \psubstp{Q}{P}  
  :=
  \lift{(x)\substp{Q}{P}}{ R \psubstp{Q}{P} } \\
%   (\dropn{x})  \psubstp{Q}{P}       
%   := 
%   \left\{ 
%     \begin{array}{ccc} 
%       \dropn{\quotep{Q}} & & x \nameeq \quotep{P} \\
%       \dropn{x} & & otherwise \\
%     \end{array}
%   \right. 
  (\dropn{x})  \psubstp{Q}{P}       
  := 
  \left\{ 
    \begin{array}{ccc} 
      Q & & x \nameeq \quotep{P} \\
      \dropn{x} & & otherwise \\
    \end{array}
  \right.
\end{mathpar}
 

where

\begin{eqnarray}
  (x)\id{\{} \lpquote Q \rpquote / \lpquote P \rpquote \id{\}}            = 
  \left\{ 
    \begin{array}{ccc}
      \lpquote Q \rpquote & & x \nameeq \lpquote P \rpquote \\
      x & & otherwise \\
    \end{array}
  \right. \nonumber
\end{eqnarray}

and $z$ is chosen distinct from $\quotep{P}$, $\quotep{Q}$, the free
names in $Q$, and all the names in $R$. Our $\alpha$-equivalence will
be built in the standard way from this substitution.

\begin{remark}\label{rem:no_self_referential_names}
  One consequence of these definitions is that $\forall P. \quotep{P}
  \not\in \freenames{P}$.
\end{remark}

\subsection{ Dynamic quote: an example }

Anticipating something of what's to come, consider applying the
substitution, $\widehat{\id{\{}u / z \id{\}}}$, to the following pair
of processes, $\lift{w}{y!(z)}$ and $w[ \lpquote y!(z) \rpquote ]$.

\begin{eqnarray}
	\lift{w}{y!(z)}\widehat{\id{\{}u / z \id{\}}}
		& = &
		\lift{w}{y!(u)} \nonumber\\
	w[ \lpquote y!(z) \rpquote ] \widehat{ \id{\{}u / z \id{\}} }
		& = &
		w[ \lpquote y!(z) \rpquote ] \nonumber
\end{eqnarray}

Because the body of the process between quotes is impervious to
substitution, we get radically different answers. In fact, by
examining the first process in an input context,
e.g. $x?(z).\lift{w}{y!(z)}$, we see that the process under the lift
operator may be shaped by prefixed inputs binding a name inside it. In
this sense, the lift operator will be seen as a way to dynamically
construct processes before reifying them as names.

Finally equipped with these standard features we can present the
dynamics of the calculus.

\subsubsection{Operational semantics} 

Finally, we introduce the computational dynamics. What marks these
algebras as distinct from other more traditionally studied algebraic
structures, e.g. vector spaces or polynomial rings, is the manner in
which dynamics is captured. In traditional structures, dynamics is typically
expressed through morphisms between such structures, as in linear maps
between vector spaces or morphisms between rings. In algebras
associated with the semantics of computation, the dynamics is
expressed as part of the algebraic structure itself, through a
reduction reduction relation typically denoted by $\red$. Below, we
give a recursive presentation of this relation for the calculus used
in the encoding.

$\red \subseteq \pi \times \pi$
$\red : \pi \to \mathcal{P}(\pi)$

\begin{mathpar}
  \inferrule* [lab=Comm] { \textsf{match}( x_{src}, x_{trgt} ) } { x_{trgt}?(y)P \; | \; x_{src}!\langle {Q} \rangle \red P\{\quotep{Q}/y}\} }
  \and \\
  \inferrule* [lab=Par] {{P} \red {P}'} {{{P} | {Q}} \red {{P}' | {Q}}}
  \and
  \inferrule* [lab=Equiv]{{{P} \scong {P}'} \andalso {{P}' \red {Q}'} \andalso {{Q}' \scong {Q}}}{{P} \red {Q}}
\end{mathpar}

\begin{eqnarray*}
  match_{\equiv} (\quotep{P},\quotep{Q}) & := & P \equiv Q \\
  match_{\dagger}(\quotep{P},\quotep{Q}) & := & \forall R. P|Q \red^{*} R => R \red^{*} 0 \\
  match_{K}(\quotep{P},\quotep{Q}) & := & K \mbox{ for some context } K
\end{eqnarray*}

$u?(x)P | u!\langle Q \rangle \red P\{\quotep{Q}/x\}$

%We write $\wred$ for $\red^*$, and $P\red$ if $\exists Q $ such that $ P \red Q$.
We write $P\red$ if $\exists Q $ such that $ P \red Q$ and $P\not\red$, otherwise.

\section{Replication}

As mentioned before, it is known that replication (and hence
recursion) can be implemented in a higher-order process algebra
\cite{SangiorgiWalker}. As our first example of calculation with the
machinery thus far presented we give the construction explicitly in
the {\rhoc}.

\begin{eqnarray}
	D_{x} & := & \prefix{x}{y}{(\binpar{\outputp{x}{y}}{@{y}})} \nonumber\\
	\bangp_{x}{P} & := & \binpar{{x}!\langle{\binpar{D_{x}}{P}}\rangle}{D_{x}} \nonumber
\end{eqnarray}

\begin{eqnarray}
	\bangp_{x}{P} & & \nonumber\\
	=
	& {x}!\langle{(\prefix{x}{y}{(\outputp{x}{y} | @{y})) | P}}\rangle 
	      | \prefix{x}{y}{(\outputp{x}{y} | @{y})} & \nonumber\\
	\red
	& (\outputp{x}{y} | @{y})\substn{\quotep{(\prefix{x}{y}{(@{y} | \outputp{x}{y})) | P}}}{y} & \nonumber\\
	=
	& \outputp{x}{\quotep{(\prefix{x}{y}{(\outputp{x}{y} | @{y})) | P}}}
	  | {(\prefix{x}{y}{(\outputp{x}{y} | @{y})) | P}} & \nonumber\\
	\red
	& \ldots & \nonumber\\
	\red^*
	& P | P | \ldots & \nonumber
\end{eqnarray}

Of course, this encoding, as an implementation, runs away, unfolding
$\bangp{P}$ eagerly. A lazier and more implementable replication
operator, restricted to input-guarded processes, may be obtained as follows.

\begin{eqnarray}
\bangp{\prefix{u}{v}{P}} 
	:= 
	\binpar{\lift{x}{\prefix{u}{v}{(\binpar{D(x)}{P})}}}{D(x)} \nonumber
\end{eqnarray}

\begin{remark}
  Note that the lazier definition still does not deal with summation
  or mixed summation (i.e. sums over input and output). The reader is
  invited to construct definitions of replication that deal with these
  features. 

  Further, the definitions are parameterized in a name, $x$. Can you,
  gentle reader, make a definition that eliminates this parameter and
  guarantees no accidental interaction between the replication
  machinery and the process being replicated -- i.e. no accidental
  sharing of names used by the process to get its work done and the
  name(s) used by the replication to effect copying. This latter
  revision of the definition of replication is crucial to obtaining
  the expected identity $!!P \sim !P$.
\end{remark}

\begin{remark}\label{rem:paradoxical_combinator}
  The reader familiar with the lambda calculus will have noticed the
  similarity between $D$ and the paradoxical combinator.

  [Ed. note: the existence of this seems to suggest we have to be more
  restrictive on the set of processes and names we admit if we are to
  support no-cloning.]
\end{remark}

\subsubsection{Bisimulation}

The computational dynamics gives rise to another kind of equivalence,
the equivalence of computational behavior. As previously mentioned
this is typically captured \emph{via} some form of bisimulation.

% The notion we use in this paper is weak barbed bisimulation
% \cite{milner91polyadicpi}.

The notion we use in this paper is derived from weak barbed
bisimulation \cite{milner91polyadicpi}. 

\begin{definition}
An \emph{observation relation}, $\downarrow_{\mathcal N}$, over a set
of names, $\mathcal N$, is the smallest relation satisfying the rules
below.

\infrule[Out-barb]{y \in {\mathcal N}, \; x \nameeq y}
		  {\outputp{x}{v} \downarrow_{\mathcal N} x}
\infrule[Par-barb]{\mbox{$P\downarrow_{\mathcal N} x$ or $Q\downarrow_{\mathcal N} x$}}
		  {\binpar{P}{Q} \downarrow_{\mathcal N} x}

We write $P \Downarrow_{\mathcal N} x$ if there is $Q$ such that 
$P \wred Q$ and $Q \downarrow_{\mathcal N} x$.
\end{definition}

\begin{definition}
%\label{def.bbisim}
An  ${\mathcal N}$-\emph{barbed bisimulation} over a set of names, ${\mathcal N}$, is a symmetric binary relation 
${\mathcal S}_{\mathcal N}$ between agents such that $P\rel{S}_{\mathcal N}Q$ implies:
\begin{enumerate}
\item If $P \red P'$ then $Q \wred Q'$ and $P'\rel{S}_{\mathcal N} Q'$.
\item If $P\downarrow_{\mathcal N} x$, then $Q\Downarrow_{\mathcal N} x$.
\end{enumerate}
$P$ is ${\mathcal N}$-barbed bisimilar to $Q$, written
$P \wbbisim_{\mathcal N} Q$, if $P \rel{S}_{\mathcal N} Q$ for some ${\mathcal N}$-barbed bisimulation ${\mathcal S}_{\mathcal N}$.
\end{definition}

$\mathcal{R} \subseteq \pi \times \pi$

$P \mathcal{R} Q => \forall P'. P \red P' \Rightarrow \exists Q'. Q \red Q', P' \mathcal{R} Q'$

$P \vdash x \Rightarrow Q \vdash x$

\begin{mathpar}
  \inferrule*[lab=Out-barb]{x \nameeq y}{{y}!\langle{Q}\rangle \vdash x}
  \and
  \inferrule*[lab=Par-barb]{\mbox{$P\vdash x$ or $Q\vdash x$}}{\binpar{P}{Q} \vdash x}
\end{mathpar}

\subsubsection{Contexts}

One of the principle advantages of computational calculi like the
$\pi$-calculus is a well-defined notion of context,
contextual-equivalence and a correlation between
contextual-equivalence and notions of bisimulation. The notion of
context allows the decomposition of a process into (sub-)process and
its syntactic environment, its context. Thus, a context may be
thought of as a process with a ``hole'' (written $\Box$) in it. The
application of a context $M$ to a process $P$, written $M[P]$, is
tantamount to filling the hole in $M$ with $P$. In this paper we do
not need the full weight of this theory, but do make use of the notion
of context in the proof the main theorem. 

\begin{mathpar}
  \inferrule* [lab=summation] {} {{M_{M},M_{N}} \bc \Box \;|\; x.M_{A} \;|\; M_{M}+M_{N}}
  \and
  \inferrule* [lab=agent] {} {{M_{A}} \bc (\vec{x})M_{P} \;| \; \clift{P_0,\ldots,M_{P},\ldots,P_N}}
  \and \\
  \inferrule* [lab=process] {} {{M_{P}} \bc M_{N} \;| \;P|M_{P} }
\end{mathpar} 

\begin{mathpar}
  \inferrule* [lab=sychronization] {} {M_{N} \bc \Box \;|\; x?M_{F} \;|\; x!M_{C}}
  \and
  \inferrule* [lab=abstraction] {} {{M_{F}} \bc (x)M_{P} }
  \and
  \inferrule* [lab=concretion] {} {{M_{C}} \bc \langle M_{P} \rangle }
  \and \\
  \inferrule* [lab=process] {} {{M_{P}} \bc M_{N} \;| \;P|M_{P} }
\end{mathpar}

\begin{definition}[contextual application] Given a context $M$, and
  process $P$, we define the \emph{contextual application}, $M[P] :=
  M\{P/\Box\}$. That is, the contextual application of M to P is the
  substitution of $P$ for $\Box$ in $M$.
\end{definition}

$\meaningof{-} : L \to \mathcal{P}(\pi)$

\begin{mathpar}
  \inferrule* [lab=collection] {} {\meaningof{true} = \pi, \and \meaningof{~E} = \pi \setminus \meaningof{E}, \and \meaningof{E_{1} \& E_{2}} = \meaningof{E_{1}} \cap \meaningof{E_{2}}}
\end{mathpar}

\begin{mathpar}
  \inferrule* [lab=structure] {} {\meaningof{0} = \{ P \in \pi | P \equiv 0 \}, \and \\ \meaningof{E_1 | E_2} = \{ P \in \pi | P \equiv P_{1} | P_{2}, P_{1} \in \meaningof{E_{1}}, P_{2} \in \meaningof{E_2}\} }
\end{mathpar}

\begin{mathpar}
 \inferrule* [lab=behavior] {} {\meaningof{\langle a?b \rangle E} = \{ P \in \pi | P \equiv Q | u?(y)P', \\ \and \\\\ \and \\ \;\;\; u \in \meaningof{a}, \forall z.P'\{z/y\} \in \meaningof{E\{z/b\}}\}, \and \\ \meaningof{a!E} = \{ P \in \pi | P \equiv Q | x!\langle P' \rangle, x \in \meaningof{a} P' \in \meaningof{E}\} }
\end{mathpar}

\begin{mathpar}
 \inferrule* [lab=nominal] {} {\meaningof{\quotep{E}} = \{ \quotep{P} \in \quotep{\pi} | P \in \meaningof{E} \}, \and \meaningof{\quotep{P}} = \{ \quotep{Q} \in \quotep{\pi} | P \equiv Q \} \and \\ \meaningof{@\quotep{E}} = \{ P \in \pi | P \equiv @x, x \in \meaningof{E} \}}
\end{mathpar}

\begin{eqnarray*}
  \\
  \meaningof{-} : TS \to ST
\end{eqnarray*}

\begin{eqnarray*}
  \\
  L : TS \to ST
\end{eqnarray*}

\begin{eqnarray*}
  \\
  P \models E \iff P \in \meaningof{E}
\end{eqnarray*}

\begin{eqnarray*}
  P \approx_{L} Q \iff \forall E \in L. P \models E \iff Q \models E
\end{eqnarray*}

\begin{eqnarray*}
  P \approx_{K} Q
\end{eqnarray*}

\begin{eqnarray*}
  P \approx Q
\end{eqnarray*}

$\approx_{K} = \approx = \approx_{L}$

\subsubsection{Contextual duality}

Note that contexts extend the quotation operation to a family of
operations from processes to names. Given a context, $M$, we can
define a \emph{nominal context}, $\quotep{M}$ by $\quotep{M}[P] :=
\quotep{M[P]}$. To foreshadow what is to come we observe that these
operations enjoy a duality with processes very much like the duality
between vectors and maps from vectors to scalars.

Further, because the calculus is essentially higher-order, we have a
correspondence between contexts and processes. More specifically,
given a name $x$ and a context $M$ we can construct $M^{*}_{x}$ such
that 

\begin{mathpar}
  M^{*}_{x} | \lift{x}{P} \red M[P]
\end{mathpar}

namely,

\begin{mathpar}
  M^{*}_{x} := x?(u).M[\dropn{u}]
\end{mathpar}

The dependence of $M^{*}_{x}$ on a name makes it an abstraction, 

\begin{mathpar}
  M^{*} := (x)x?(u).M[\dropn{u}]
\end{mathpar}

\subsection{Additional notation}

It will sometimes be convenient to denote the process a name
quotes. We already have the notation $x = \quotep{P}$, but it will be
convenient to introduce an alternate notation, $\procn{x}$, when we
want to emphasize the connection to the use of the name. Note that, by
virtue of name equivalence, $\quotep{\procn{x}} \nameeq x$; so, the
notation is consistent with previous definitions.

Further, because names have structure it is possible to effect
substitutions on the basis of that structure. This means we need to
upgrade our notation for substitutions, which we accomplish by
adapting comprehension notation. Thus,

\begin{mathpar}
  P\{ y / x : x \in S \}
\end{mathpar}

is interpreted to mean the process derived from P by replacing (in a
capture-avoiding manner) each occurrence of $x$ in $S$ by $y$. For example,

\begin{mathpar}
  P\{ \quotep{\procn{x}|\procn{x}} / x : x \in \freenames{P} \}
\end{mathpar}

will replace each (occurrence) of a free name $x$ in $P$ by
$\quotep{\procn{x}|\procn{x}}$.

Also, we will avail ourselves of the notation $x^{L}$ and $x^{R}$ to
denote injections of a name into disjoint copies of the name
space. There are numerous ways to accomplish this. One example can be
found in \cite{MeredithR05}. This notation overloads to vectors of
names: $\vec{x}^{\pi} := (x_{i}^{\pi} \; : \; 0 \leq i < |\vec{x}| )$ where $\pi \in \{L,R\}$.

We also use $P^{\Box} := P|\Box$.

In \cite{MeredithR05} an interpretation of the new operator is
given. It turns out that there are several possible interpretations
all enjoying the requisite algebraic properties of the operator (see
\cite{milner91polyadicpi}). We will therefore make liberal use of
$(\nu\; \vec{x})P$.

% subsection the_syntax_and_semantics_of_the_notation_system (end)   

\input{qm2pi.qmops} 

\input{qm2pi.sterngerlach} 

\input{qm2pi.metric} 

% section concurrent_process_calculi (end)

%\input{qm2pi.proofsketch}

% section proof sketch (end)

%\input{qm2pi.slviaknots} 

% section spatial logic via knots (end)

\input{qm2pi.conclusion}

% section conclusion (end)

%\input{qm2pi.dtcodes} 

% section wiring algorithm (end)

\input{qm2pi.ack} 

% section acknowledgments (end)

\newpage


\bibliographystyle{plain}   
\bibliography{../../biblios/main.bib}

\input{qm2pi.rhodetails}

\end{document}

 

% section wiring algorithm (end)

\documentclass[12pt]{llncs}
%\documentclass{jktr}

\usepackage[pdftex]{hyperref}                   
\usepackage {listings}
\usepackage {mathpartir}
\usepackage{bcprules}
%\usepackage{listings}
                       
\usepackage{graphicx} 
%\usepackage[margins=2.5cm,nohead,nofoot]{geometry}
%\usepackage{geometry}
\usepackage{amsfonts}
\usepackage{amstext}
\usepackage{latexsym}
\usepackage{amssymb}
\usepackage{color}


%\include{myPreamble}
\include{qm2pi.local} 

%\ifpdf
%\usepackage[pdftex]{graphicx}
%\else
%\usepackage{graphicx}
%\fi

 % \ifpdf
%  \usepackage{pdfsync}
%  \if


%\title{Brief Article}
%\author{David F. Snyder}
%\author{L.G. Meredith}

%\address{Dept. of Math., Texas State University--San Marcos, San Marcos, TX 78666}
       
\pagestyle{empty}


\begin{document}

\lstset{language=[Objective]Caml,frame=shadowbox}

\input{qm2pi.front}

% section front matter (end)

\input{qm2pi.intro} 
 
% section introduction (end)

% \input{qm2pi.knotations} 

% section notation (end)

\input{qm2pi.process.calculi} 

% section concurrent_process_calculi_and_spatial_logics_ (end)
    
%\input{qm2pi.knots2pi} 

%\input{qm2pi.trefoil} 

%\input{qm2pi.mainthm} 

% subsection basic_interpretation (end)

%\input{qm2pi.rho.presentation} 
\subsection{The syntax and semantics of the notation system}\label{sub:the_syntax_and_semantics_of_the_notation_system} % (fold)

We now summarize a technical presentation of the calculus that
embodies our theory of dynamics. The typical presentation of such a
calculus follows the style of giving generators and relations on
them. The grammar, below, describing term constructors, freely
generates the set of processes, $\Proc$. This set is then quotiented
by a relation known as structural congruence and it is over this set
that the notion of dynamics is expressed. This presentation is
essentially that of \cite{MeredithR05} with the addition of
polyadicity and summation. For readability we have relegated some of
the technical subtleties to an appendix.

\subsubsection{Process grammar}\label{subsub:process_grammar}

\begin{mathpar}
  \inferrule* [lab=synchronization] {} {{M} \bc \pzero \;|\; x?F \;|\; x!C }
  \and
  \inferrule* [lab=abstraction] {} {{F} \bc (x)P}
  \and
  \inferrule* [lab=concretion] {} {{C} \bc \langle Q \rangle}
  \and
  \inferrule* [lab=process] {} {{P,Q} \bc M \;| \;P|Q \;|\; @{x}}
  \and
  \inferrule* [lab=name] {} {{x} \bc \quotep{P}}
\end{mathpar} 

Note that $\vec{x}$ (resp. $\vec{P}$) denotes a vector of names
(resp. processes) of length $|\vec{x}|$ (resp. $|\vec{P}|$). We adopt
the following useful abbreviations.

\begin{mathpar}
   x?(\vec{y}).P := x.(\vec{y})P \and  x\clift{\vec{P}} := x.\clift{\vec{P}}
   \and x!(y) := \lift{x}{\dropn{y}}
   \and \Pi_{i=0}^{n-1}P_i := P_0 | \ldots | P_{n-1}
\end{mathpar}

\subsubsection{Structural congruence}

\paragraph{Free and bound names and alpha-equivalence.} At the
core of structural equivalence is alpha-equivalence which identifies
process that are the same up to a change of variable. Formally, we
recognize the distinction between free and bound names. The free names
of a process, $\freenames{P}$, may be calculated recursively as
follows:

\begin{mathpar}
\freenames{\pzero} := \emptyset
  \and \\
  \freenames{x?(y).P} := \{ x \} \cup (\freenames{P} \setminus \{ y \})
  \and 
  \freenames{x!\langle P \rangle} := \{ x \} \cup \{ P \} 
  \and \\
  \freenames{P|Q} := \freenames{P} \cup \freenames{Q}
  \and \\
  \freenames{@{x}} := \{ x \}
\end{mathpar}

$\pi$
$\quotep{\pi}$

$\freenames{-} : \pi \to \mathcal{P}(\quotep{\pi})$

\begin{eqnarray*}
  \freenames{\pzero} & := & \emptyset \\
  \freenames{x?(y).P} & := & \{ x \} \cup (\freenames{P} \setminus \{ y \}) \\
  \freenames{x!\langle P \rangle} & := & \{ x \} \cup \{ P \} \\
  \freenames{P|Q} & := & \freenames{P} \cup \freenames{Q} \\
  \freenames{\dropn{x}} & := & \{ x \}
\end{eqnarray*}

The bound names of a process, $\boundnames{P}$, are those names occurring in $P$
that are not free. For example, in $x?(y).0$, the name $x$ is free, while $y$ is bound.

\begin{mathpar}
  \inferrule* [lab=monoidal-laws] {} { P|Q \equiv Q|P \and P|0 \equiv P \and P|(Q|R) \equiv (P|Q)|R }
\end{mathpar}

\begin{mathpar}
  \inferrule* [lab=alpha-equivalence] {} { (x)P \equiv (y)P\{y/x\} \and y \not\in \freenames{P} }
\end{mathpar}

\begin{definition}
Then two processes, $P,Q$, are alpha-equivalent if $P = Q\{\vec{y}/\vec{x}\}$ for
some $\vec{x} \in \boundnames{Q},\vec{y} \in \boundnames{P}$, where $Q\{\vec{y}/\vec{x}\}$
denotes the capture-avoiding substitution of $\vec{y}$ for $\vec{x}$ in $Q$.
\end{definition}

\begin{definition}
  The {\em structural congruence} \cite{SangiorgiWalker} , $\equiv$,
  between processes is the least congruence containing
  alpha-equivalence, satisfying the abelian monoid laws
  (associativity, commutativity and $\pzero$ as identity) for parallel
  composition $|$ and for summation $+$.
\end{definition}

\subsection{Name equivalence}

We take name equivalence, written $\nameeq$, to be the smallest
equivalence relation generated by the following rules.

\begin{mathpar}
\inferrule*[lab=Quote-drop]
{ }
{ \quotep{@{x}} \nameeq x }

\inferrule*[lab=Struct-equiv]
{ P \scong Q }
{ \quotep{P} \nameeq \quotep{Q} }
\end{mathpar}

The astute reader will have noticed that the mutual recursion of names
and processes imposes a mutual recursion on alpha-equivalence and
structural equivalence via name-equivalence. Fortunately, all of this
works out pleasantly and we may calculate in the natural way, free of
concern. The reader interested in the details is referred to the
appendix \ref{appendix:rho_details}.

\subsection{Substitution}

We use $\Proc$ for the set of processes, $\QProc$ for the set of
names, and $\id{\{}\vec{y} / \vec{x} \id{\}}$ to denote partial maps,
$s : \QProc \rightarrow \QProc$. A map, $s$ lifts, uniquely, to a map
on process terms, $\widehat{s} : \Proc \rightarrow \Proc$ by the
following equations.

\begin{mathpar}
  (0) \psubstp{Q}{P} := 0 \\
  (R \juxtap S) \psubstp{Q}{P}
  :=    
  (R)\psubstp{Q}{P} \juxtap (S) \psubstp{Q}{P} \\
  (x?(y).R) \psubstp{Q}{P}    
  :=    
  (x)\substp{Q}{P} (z)\concat( (R \psubstn{z}{y}) \psubstp{Q}{P} ) \\
  (\lift{x}{R}) \psubstp{Q}{P}  
  :=
  \lift{(x)\substp{Q}{P}}{ R \psubstp{Q}{P} } \\
%   (\dropn{x})  \psubstp{Q}{P}       
%   := 
%   \left\{ 
%     \begin{array}{ccc} 
%       \dropn{\quotep{Q}} & & x \nameeq \quotep{P} \\
%       \dropn{x} & & otherwise \\
%     \end{array}
%   \right. 
  (\dropn{x})  \psubstp{Q}{P}       
  := 
  \left\{ 
    \begin{array}{ccc} 
      Q & & x \nameeq \quotep{P} \\
      \dropn{x} & & otherwise \\
    \end{array}
  \right.
\end{mathpar}
 

where

\begin{eqnarray}
  (x)\id{\{} \lpquote Q \rpquote / \lpquote P \rpquote \id{\}}            = 
  \left\{ 
    \begin{array}{ccc}
      \lpquote Q \rpquote & & x \nameeq \lpquote P \rpquote \\
      x & & otherwise \\
    \end{array}
  \right. \nonumber
\end{eqnarray}

and $z$ is chosen distinct from $\quotep{P}$, $\quotep{Q}$, the free
names in $Q$, and all the names in $R$. Our $\alpha$-equivalence will
be built in the standard way from this substitution.

\begin{remark}\label{rem:no_self_referential_names}
  One consequence of these definitions is that $\forall P. \quotep{P}
  \not\in \freenames{P}$.
\end{remark}

\subsection{ Dynamic quote: an example }

Anticipating something of what's to come, consider applying the
substitution, $\widehat{\id{\{}u / z \id{\}}}$, to the following pair
of processes, $\lift{w}{y!(z)}$ and $w[ \lpquote y!(z) \rpquote ]$.

\begin{eqnarray}
	\lift{w}{y!(z)}\widehat{\id{\{}u / z \id{\}}}
		& = &
		\lift{w}{y!(u)} \nonumber\\
	w[ \lpquote y!(z) \rpquote ] \widehat{ \id{\{}u / z \id{\}} }
		& = &
		w[ \lpquote y!(z) \rpquote ] \nonumber
\end{eqnarray}

Because the body of the process between quotes is impervious to
substitution, we get radically different answers. In fact, by
examining the first process in an input context,
e.g. $x?(z).\lift{w}{y!(z)}$, we see that the process under the lift
operator may be shaped by prefixed inputs binding a name inside it. In
this sense, the lift operator will be seen as a way to dynamically
construct processes before reifying them as names.

Finally equipped with these standard features we can present the
dynamics of the calculus.

\subsubsection{Operational semantics} 

Finally, we introduce the computational dynamics. What marks these
algebras as distinct from other more traditionally studied algebraic
structures, e.g. vector spaces or polynomial rings, is the manner in
which dynamics is captured. In traditional structures, dynamics is typically
expressed through morphisms between such structures, as in linear maps
between vector spaces or morphisms between rings. In algebras
associated with the semantics of computation, the dynamics is
expressed as part of the algebraic structure itself, through a
reduction reduction relation typically denoted by $\red$. Below, we
give a recursive presentation of this relation for the calculus used
in the encoding.

$\red \subseteq \pi \times \pi$
$\red : \pi \to \mathcal{P}(\pi)$

\begin{mathpar}
  \inferrule* [lab=Comm] { \textsf{match}( x_{src}, x_{trgt} ) } { x_{trgt}?(y)P \; | \; x_{src}!\langle {Q} \rangle \red P\{\quotep{Q}/y}\} }
  \and \\
  \inferrule* [lab=Par] {{P} \red {P}'} {{{P} | {Q}} \red {{P}' | {Q}}}
  \and
  \inferrule* [lab=Equiv]{{{P} \scong {P}'} \andalso {{P}' \red {Q}'} \andalso {{Q}' \scong {Q}}}{{P} \red {Q}}
\end{mathpar}

\begin{eqnarray*}
  match_{\equiv} (\quotep{P},\quotep{Q}) & := & P \equiv Q \\
  match_{\dagger}(\quotep{P},\quotep{Q}) & := & \forall R. P|Q \red^{*} R => R \red^{*} 0 \\
  match_{K}(\quotep{P},\quotep{Q}) & := & K \mbox{ for some context } K
\end{eqnarray*}

$u?(x)P | u!\langle Q \rangle \red P\{\quotep{Q}/x\}$

%We write $\wred$ for $\red^*$, and $P\red$ if $\exists Q $ such that $ P \red Q$.
We write $P\red$ if $\exists Q $ such that $ P \red Q$ and $P\not\red$, otherwise.

\section{Replication}

As mentioned before, it is known that replication (and hence
recursion) can be implemented in a higher-order process algebra
\cite{SangiorgiWalker}. As our first example of calculation with the
machinery thus far presented we give the construction explicitly in
the {\rhoc}.

\begin{eqnarray}
	D_{x} & := & \prefix{x}{y}{(\binpar{\outputp{x}{y}}{@{y}})} \nonumber\\
	\bangp_{x}{P} & := & \binpar{{x}!\langle{\binpar{D_{x}}{P}}\rangle}{D_{x}} \nonumber
\end{eqnarray}

\begin{eqnarray}
	\bangp_{x}{P} & & \nonumber\\
	=
	& {x}!\langle{(\prefix{x}{y}{(\outputp{x}{y} | @{y})) | P}}\rangle 
	      | \prefix{x}{y}{(\outputp{x}{y} | @{y})} & \nonumber\\
	\red
	& (\outputp{x}{y} | @{y})\substn{\quotep{(\prefix{x}{y}{(@{y} | \outputp{x}{y})) | P}}}{y} & \nonumber\\
	=
	& \outputp{x}{\quotep{(\prefix{x}{y}{(\outputp{x}{y} | @{y})) | P}}}
	  | {(\prefix{x}{y}{(\outputp{x}{y} | @{y})) | P}} & \nonumber\\
	\red
	& \ldots & \nonumber\\
	\red^*
	& P | P | \ldots & \nonumber
\end{eqnarray}

Of course, this encoding, as an implementation, runs away, unfolding
$\bangp{P}$ eagerly. A lazier and more implementable replication
operator, restricted to input-guarded processes, may be obtained as follows.

\begin{eqnarray}
\bangp{\prefix{u}{v}{P}} 
	:= 
	\binpar{\lift{x}{\prefix{u}{v}{(\binpar{D(x)}{P})}}}{D(x)} \nonumber
\end{eqnarray}

\begin{remark}
  Note that the lazier definition still does not deal with summation
  or mixed summation (i.e. sums over input and output). The reader is
  invited to construct definitions of replication that deal with these
  features. 

  Further, the definitions are parameterized in a name, $x$. Can you,
  gentle reader, make a definition that eliminates this parameter and
  guarantees no accidental interaction between the replication
  machinery and the process being replicated -- i.e. no accidental
  sharing of names used by the process to get its work done and the
  name(s) used by the replication to effect copying. This latter
  revision of the definition of replication is crucial to obtaining
  the expected identity $!!P \sim !P$.
\end{remark}

\begin{remark}\label{rem:paradoxical_combinator}
  The reader familiar with the lambda calculus will have noticed the
  similarity between $D$ and the paradoxical combinator.

  [Ed. note: the existence of this seems to suggest we have to be more
  restrictive on the set of processes and names we admit if we are to
  support no-cloning.]
\end{remark}

\subsubsection{Bisimulation}

The computational dynamics gives rise to another kind of equivalence,
the equivalence of computational behavior. As previously mentioned
this is typically captured \emph{via} some form of bisimulation.

% The notion we use in this paper is weak barbed bisimulation
% \cite{milner91polyadicpi}.

The notion we use in this paper is derived from weak barbed
bisimulation \cite{milner91polyadicpi}. 

\begin{definition}
An \emph{observation relation}, $\downarrow_{\mathcal N}$, over a set
of names, $\mathcal N$, is the smallest relation satisfying the rules
below.

\infrule[Out-barb]{y \in {\mathcal N}, \; x \nameeq y}
		  {\outputp{x}{v} \downarrow_{\mathcal N} x}
\infrule[Par-barb]{\mbox{$P\downarrow_{\mathcal N} x$ or $Q\downarrow_{\mathcal N} x$}}
		  {\binpar{P}{Q} \downarrow_{\mathcal N} x}

We write $P \Downarrow_{\mathcal N} x$ if there is $Q$ such that 
$P \wred Q$ and $Q \downarrow_{\mathcal N} x$.
\end{definition}

\begin{definition}
%\label{def.bbisim}
An  ${\mathcal N}$-\emph{barbed bisimulation} over a set of names, ${\mathcal N}$, is a symmetric binary relation 
${\mathcal S}_{\mathcal N}$ between agents such that $P\rel{S}_{\mathcal N}Q$ implies:
\begin{enumerate}
\item If $P \red P'$ then $Q \wred Q'$ and $P'\rel{S}_{\mathcal N} Q'$.
\item If $P\downarrow_{\mathcal N} x$, then $Q\Downarrow_{\mathcal N} x$.
\end{enumerate}
$P$ is ${\mathcal N}$-barbed bisimilar to $Q$, written
$P \wbbisim_{\mathcal N} Q$, if $P \rel{S}_{\mathcal N} Q$ for some ${\mathcal N}$-barbed bisimulation ${\mathcal S}_{\mathcal N}$.
\end{definition}

$\mathcal{R} \subseteq \pi \times \pi$

$P \mathcal{R} Q => \forall P'. P \red P' \Rightarrow \exists Q'. Q \red Q', P' \mathcal{R} Q'$

$P \vdash x \Rightarrow Q \vdash x$

\begin{mathpar}
  \inferrule*[lab=Out-barb]{x \nameeq y}{{y}!\langle{Q}\rangle \vdash x}
  \and
  \inferrule*[lab=Par-barb]{\mbox{$P\vdash x$ or $Q\vdash x$}}{\binpar{P}{Q} \vdash x}
\end{mathpar}

\subsubsection{Contexts}

One of the principle advantages of computational calculi like the
$\pi$-calculus is a well-defined notion of context,
contextual-equivalence and a correlation between
contextual-equivalence and notions of bisimulation. The notion of
context allows the decomposition of a process into (sub-)process and
its syntactic environment, its context. Thus, a context may be
thought of as a process with a ``hole'' (written $\Box$) in it. The
application of a context $M$ to a process $P$, written $M[P]$, is
tantamount to filling the hole in $M$ with $P$. In this paper we do
not need the full weight of this theory, but do make use of the notion
of context in the proof the main theorem. 

\begin{mathpar}
  \inferrule* [lab=summation] {} {{M_{M},M_{N}} \bc \Box \;|\; x.M_{A} \;|\; M_{M}+M_{N}}
  \and
  \inferrule* [lab=agent] {} {{M_{A}} \bc (\vec{x})M_{P} \;| \; \clift{P_0,\ldots,M_{P},\ldots,P_N}}
  \and \\
  \inferrule* [lab=process] {} {{M_{P}} \bc M_{N} \;| \;P|M_{P} }
\end{mathpar} 

\begin{mathpar}
  \inferrule* [lab=sychronization] {} {M_{N} \bc \Box \;|\; x?M_{F} \;|\; x!M_{C}}
  \and
  \inferrule* [lab=abstraction] {} {{M_{F}} \bc (x)M_{P} }
  \and
  \inferrule* [lab=concretion] {} {{M_{C}} \bc \langle M_{P} \rangle }
  \and \\
  \inferrule* [lab=process] {} {{M_{P}} \bc M_{N} \;| \;P|M_{P} }
\end{mathpar}

\begin{definition}[contextual application] Given a context $M$, and
  process $P$, we define the \emph{contextual application}, $M[P] :=
  M\{P/\Box\}$. That is, the contextual application of M to P is the
  substitution of $P$ for $\Box$ in $M$.
\end{definition}

$\meaningof{-} : L \to \mathcal{P}(\pi)$

\begin{mathpar}
  \inferrule* [lab=collection] {} {\meaningof{true} = \pi, \and \meaningof{~E} = \pi \setminus \meaningof{E}, \and \meaningof{E_{1} \& E_{2}} = \meaningof{E_{1}} \cap \meaningof{E_{2}}}
\end{mathpar}

\begin{mathpar}
  \inferrule* [lab=structure] {} {\meaningof{0} = \{ P \in \pi | P \equiv 0 \}, \and \\ \meaningof{E_1 | E_2} = \{ P \in \pi | P \equiv P_{1} | P_{2}, P_{1} \in \meaningof{E_{1}}, P_{2} \in \meaningof{E_2}\} }
\end{mathpar}

\begin{mathpar}
 \inferrule* [lab=behavior] {} {\meaningof{\langle a?b \rangle E} = \{ P \in \pi | P \equiv Q | u?(y)P', \\ \and \\\\ \and \\ \;\;\; u \in \meaningof{a}, \forall z.P'\{z/y\} \in \meaningof{E\{z/b\}}\}, \and \\ \meaningof{a!E} = \{ P \in \pi | P \equiv Q | x!\langle P' \rangle, x \in \meaningof{a} P' \in \meaningof{E}\} }
\end{mathpar}

\begin{mathpar}
 \inferrule* [lab=nominal] {} {\meaningof{\quotep{E}} = \{ \quotep{P} \in \quotep{\pi} | P \in \meaningof{E} \}, \and \meaningof{\quotep{P}} = \{ \quotep{Q} \in \quotep{\pi} | P \equiv Q \} \and \\ \meaningof{@\quotep{E}} = \{ P \in \pi | P \equiv @x, x \in \meaningof{E} \}}
\end{mathpar}

\begin{eqnarray*}
  \\
  \meaningof{-} : TS \to ST
\end{eqnarray*}

\begin{eqnarray*}
  \\
  L : TS \to ST
\end{eqnarray*}

\begin{eqnarray*}
  \\
  P \models E \iff P \in \meaningof{E}
\end{eqnarray*}

\begin{eqnarray*}
  P \approx_{L} Q \iff \forall E \in L. P \models E \iff Q \models E
\end{eqnarray*}

\begin{eqnarray*}
  P \approx_{K} Q
\end{eqnarray*}

\begin{eqnarray*}
  P \approx Q
\end{eqnarray*}

$\approx_{K} = \approx = \approx_{L}$

\subsubsection{Contextual duality}

Note that contexts extend the quotation operation to a family of
operations from processes to names. Given a context, $M$, we can
define a \emph{nominal context}, $\quotep{M}$ by $\quotep{M}[P] :=
\quotep{M[P]}$. To foreshadow what is to come we observe that these
operations enjoy a duality with processes very much like the duality
between vectors and maps from vectors to scalars.

Further, because the calculus is essentially higher-order, we have a
correspondence between contexts and processes. More specifically,
given a name $x$ and a context $M$ we can construct $M^{*}_{x}$ such
that 

\begin{mathpar}
  M^{*}_{x} | \lift{x}{P} \red M[P]
\end{mathpar}

namely,

\begin{mathpar}
  M^{*}_{x} := x?(u).M[\dropn{u}]
\end{mathpar}

The dependence of $M^{*}_{x}$ on a name makes it an abstraction, 

\begin{mathpar}
  M^{*} := (x)x?(u).M[\dropn{u}]
\end{mathpar}

\subsection{Additional notation}

It will sometimes be convenient to denote the process a name
quotes. We already have the notation $x = \quotep{P}$, but it will be
convenient to introduce an alternate notation, $\procn{x}$, when we
want to emphasize the connection to the use of the name. Note that, by
virtue of name equivalence, $\quotep{\procn{x}} \nameeq x$; so, the
notation is consistent with previous definitions.

Further, because names have structure it is possible to effect
substitutions on the basis of that structure. This means we need to
upgrade our notation for substitutions, which we accomplish by
adapting comprehension notation. Thus,

\begin{mathpar}
  P\{ y / x : x \in S \}
\end{mathpar}

is interpreted to mean the process derived from P by replacing (in a
capture-avoiding manner) each occurrence of $x$ in $S$ by $y$. For example,

\begin{mathpar}
  P\{ \quotep{\procn{x}|\procn{x}} / x : x \in \freenames{P} \}
\end{mathpar}

will replace each (occurrence) of a free name $x$ in $P$ by
$\quotep{\procn{x}|\procn{x}}$.

Also, we will avail ourselves of the notation $x^{L}$ and $x^{R}$ to
denote injections of a name into disjoint copies of the name
space. There are numerous ways to accomplish this. One example can be
found in \cite{MeredithR05}. This notation overloads to vectors of
names: $\vec{x}^{\pi} := (x_{i}^{\pi} \; : \; 0 \leq i < |\vec{x}| )$ where $\pi \in \{L,R\}$.

We also use $P^{\Box} := P|\Box$.

In \cite{MeredithR05} an interpretation of the new operator is
given. It turns out that there are several possible interpretations
all enjoying the requisite algebraic properties of the operator (see
\cite{milner91polyadicpi}). We will therefore make liberal use of
$(\nu\; \vec{x})P$.

% subsection the_syntax_and_semantics_of_the_notation_system (end)   

\input{qm2pi.qmops} 

\input{qm2pi.sterngerlach} 

\input{qm2pi.metric} 

% section concurrent_process_calculi (end)

%\input{qm2pi.proofsketch}

% section proof sketch (end)

%\input{qm2pi.slviaknots} 

% section spatial logic via knots (end)

\input{qm2pi.conclusion}

% section conclusion (end)

%\input{qm2pi.dtcodes} 

% section wiring algorithm (end)

\input{qm2pi.ack} 

% section acknowledgments (end)

\newpage


\bibliographystyle{plain}   
\bibliography{../../biblios/main.bib}

\input{qm2pi.rhodetails}

\end{document}

 

% section acknowledgments (end)

\newpage


\bibliographystyle{plain}   
\bibliography{../../biblios/main.bib}

\documentclass[12pt]{llncs}
%\documentclass{jktr}

\usepackage[pdftex]{hyperref}                   
\usepackage {listings}
\usepackage {mathpartir}
\usepackage{bcprules}
%\usepackage{listings}
                       
\usepackage{graphicx} 
%\usepackage[margins=2.5cm,nohead,nofoot]{geometry}
%\usepackage{geometry}
\usepackage{amsfonts}
\usepackage{amstext}
\usepackage{latexsym}
\usepackage{amssymb}
\usepackage{color}


%\include{myPreamble}
\include{qm2pi.local} 

%\ifpdf
%\usepackage[pdftex]{graphicx}
%\else
%\usepackage{graphicx}
%\fi

 % \ifpdf
%  \usepackage{pdfsync}
%  \if


%\title{Brief Article}
%\author{David F. Snyder}
%\author{L.G. Meredith}

%\address{Dept. of Math., Texas State University--San Marcos, San Marcos, TX 78666}
       
\pagestyle{empty}


\begin{document}

\lstset{language=[Objective]Caml,frame=shadowbox}

\input{qm2pi.front}

% section front matter (end)

\input{qm2pi.intro} 
 
% section introduction (end)

% \input{qm2pi.knotations} 

% section notation (end)

\input{qm2pi.process.calculi} 

% section concurrent_process_calculi_and_spatial_logics_ (end)
    
%\input{qm2pi.knots2pi} 

%\input{qm2pi.trefoil} 

%\input{qm2pi.mainthm} 

% subsection basic_interpretation (end)

%\input{qm2pi.rho.presentation} 
\subsection{The syntax and semantics of the notation system}\label{sub:the_syntax_and_semantics_of_the_notation_system} % (fold)

We now summarize a technical presentation of the calculus that
embodies our theory of dynamics. The typical presentation of such a
calculus follows the style of giving generators and relations on
them. The grammar, below, describing term constructors, freely
generates the set of processes, $\Proc$. This set is then quotiented
by a relation known as structural congruence and it is over this set
that the notion of dynamics is expressed. This presentation is
essentially that of \cite{MeredithR05} with the addition of
polyadicity and summation. For readability we have relegated some of
the technical subtleties to an appendix.

\subsubsection{Process grammar}\label{subsub:process_grammar}

\begin{mathpar}
  \inferrule* [lab=synchronization] {} {{M} \bc \pzero \;|\; x?F \;|\; x!C }
  \and
  \inferrule* [lab=abstraction] {} {{F} \bc (x)P}
  \and
  \inferrule* [lab=concretion] {} {{C} \bc \langle Q \rangle}
  \and
  \inferrule* [lab=process] {} {{P,Q} \bc M \;| \;P|Q \;|\; @{x}}
  \and
  \inferrule* [lab=name] {} {{x} \bc \quotep{P}}
\end{mathpar} 

Note that $\vec{x}$ (resp. $\vec{P}$) denotes a vector of names
(resp. processes) of length $|\vec{x}|$ (resp. $|\vec{P}|$). We adopt
the following useful abbreviations.

\begin{mathpar}
   x?(\vec{y}).P := x.(\vec{y})P \and  x\clift{\vec{P}} := x.\clift{\vec{P}}
   \and x!(y) := \lift{x}{\dropn{y}}
   \and \Pi_{i=0}^{n-1}P_i := P_0 | \ldots | P_{n-1}
\end{mathpar}

\subsubsection{Structural congruence}

\paragraph{Free and bound names and alpha-equivalence.} At the
core of structural equivalence is alpha-equivalence which identifies
process that are the same up to a change of variable. Formally, we
recognize the distinction between free and bound names. The free names
of a process, $\freenames{P}$, may be calculated recursively as
follows:

\begin{mathpar}
\freenames{\pzero} := \emptyset
  \and \\
  \freenames{x?(y).P} := \{ x \} \cup (\freenames{P} \setminus \{ y \})
  \and 
  \freenames{x!\langle P \rangle} := \{ x \} \cup \{ P \} 
  \and \\
  \freenames{P|Q} := \freenames{P} \cup \freenames{Q}
  \and \\
  \freenames{@{x}} := \{ x \}
\end{mathpar}

$\pi$
$\quotep{\pi}$

$\freenames{-} : \pi \to \mathcal{P}(\quotep{\pi})$

\begin{eqnarray*}
  \freenames{\pzero} & := & \emptyset \\
  \freenames{x?(y).P} & := & \{ x \} \cup (\freenames{P} \setminus \{ y \}) \\
  \freenames{x!\langle P \rangle} & := & \{ x \} \cup \{ P \} \\
  \freenames{P|Q} & := & \freenames{P} \cup \freenames{Q} \\
  \freenames{\dropn{x}} & := & \{ x \}
\end{eqnarray*}

The bound names of a process, $\boundnames{P}$, are those names occurring in $P$
that are not free. For example, in $x?(y).0$, the name $x$ is free, while $y$ is bound.

\begin{mathpar}
  \inferrule* [lab=monoidal-laws] {} { P|Q \equiv Q|P \and P|0 \equiv P \and P|(Q|R) \equiv (P|Q)|R }
\end{mathpar}

\begin{mathpar}
  \inferrule* [lab=alpha-equivalence] {} { (x)P \equiv (y)P\{y/x\} \and y \not\in \freenames{P} }
\end{mathpar}

\begin{definition}
Then two processes, $P,Q$, are alpha-equivalent if $P = Q\{\vec{y}/\vec{x}\}$ for
some $\vec{x} \in \boundnames{Q},\vec{y} \in \boundnames{P}$, where $Q\{\vec{y}/\vec{x}\}$
denotes the capture-avoiding substitution of $\vec{y}$ for $\vec{x}$ in $Q$.
\end{definition}

\begin{definition}
  The {\em structural congruence} \cite{SangiorgiWalker} , $\equiv$,
  between processes is the least congruence containing
  alpha-equivalence, satisfying the abelian monoid laws
  (associativity, commutativity and $\pzero$ as identity) for parallel
  composition $|$ and for summation $+$.
\end{definition}

\subsection{Name equivalence}

We take name equivalence, written $\nameeq$, to be the smallest
equivalence relation generated by the following rules.

\begin{mathpar}
\inferrule*[lab=Quote-drop]
{ }
{ \quotep{@{x}} \nameeq x }

\inferrule*[lab=Struct-equiv]
{ P \scong Q }
{ \quotep{P} \nameeq \quotep{Q} }
\end{mathpar}

The astute reader will have noticed that the mutual recursion of names
and processes imposes a mutual recursion on alpha-equivalence and
structural equivalence via name-equivalence. Fortunately, all of this
works out pleasantly and we may calculate in the natural way, free of
concern. The reader interested in the details is referred to the
appendix \ref{appendix:rho_details}.

\subsection{Substitution}

We use $\Proc$ for the set of processes, $\QProc$ for the set of
names, and $\id{\{}\vec{y} / \vec{x} \id{\}}$ to denote partial maps,
$s : \QProc \rightarrow \QProc$. A map, $s$ lifts, uniquely, to a map
on process terms, $\widehat{s} : \Proc \rightarrow \Proc$ by the
following equations.

\begin{mathpar}
  (0) \psubstp{Q}{P} := 0 \\
  (R \juxtap S) \psubstp{Q}{P}
  :=    
  (R)\psubstp{Q}{P} \juxtap (S) \psubstp{Q}{P} \\
  (x?(y).R) \psubstp{Q}{P}    
  :=    
  (x)\substp{Q}{P} (z)\concat( (R \psubstn{z}{y}) \psubstp{Q}{P} ) \\
  (\lift{x}{R}) \psubstp{Q}{P}  
  :=
  \lift{(x)\substp{Q}{P}}{ R \psubstp{Q}{P} } \\
%   (\dropn{x})  \psubstp{Q}{P}       
%   := 
%   \left\{ 
%     \begin{array}{ccc} 
%       \dropn{\quotep{Q}} & & x \nameeq \quotep{P} \\
%       \dropn{x} & & otherwise \\
%     \end{array}
%   \right. 
  (\dropn{x})  \psubstp{Q}{P}       
  := 
  \left\{ 
    \begin{array}{ccc} 
      Q & & x \nameeq \quotep{P} \\
      \dropn{x} & & otherwise \\
    \end{array}
  \right.
\end{mathpar}
 

where

\begin{eqnarray}
  (x)\id{\{} \lpquote Q \rpquote / \lpquote P \rpquote \id{\}}            = 
  \left\{ 
    \begin{array}{ccc}
      \lpquote Q \rpquote & & x \nameeq \lpquote P \rpquote \\
      x & & otherwise \\
    \end{array}
  \right. \nonumber
\end{eqnarray}

and $z$ is chosen distinct from $\quotep{P}$, $\quotep{Q}$, the free
names in $Q$, and all the names in $R$. Our $\alpha$-equivalence will
be built in the standard way from this substitution.

\begin{remark}\label{rem:no_self_referential_names}
  One consequence of these definitions is that $\forall P. \quotep{P}
  \not\in \freenames{P}$.
\end{remark}

\subsection{ Dynamic quote: an example }

Anticipating something of what's to come, consider applying the
substitution, $\widehat{\id{\{}u / z \id{\}}}$, to the following pair
of processes, $\lift{w}{y!(z)}$ and $w[ \lpquote y!(z) \rpquote ]$.

\begin{eqnarray}
	\lift{w}{y!(z)}\widehat{\id{\{}u / z \id{\}}}
		& = &
		\lift{w}{y!(u)} \nonumber\\
	w[ \lpquote y!(z) \rpquote ] \widehat{ \id{\{}u / z \id{\}} }
		& = &
		w[ \lpquote y!(z) \rpquote ] \nonumber
\end{eqnarray}

Because the body of the process between quotes is impervious to
substitution, we get radically different answers. In fact, by
examining the first process in an input context,
e.g. $x?(z).\lift{w}{y!(z)}$, we see that the process under the lift
operator may be shaped by prefixed inputs binding a name inside it. In
this sense, the lift operator will be seen as a way to dynamically
construct processes before reifying them as names.

Finally equipped with these standard features we can present the
dynamics of the calculus.

\subsubsection{Operational semantics} 

Finally, we introduce the computational dynamics. What marks these
algebras as distinct from other more traditionally studied algebraic
structures, e.g. vector spaces or polynomial rings, is the manner in
which dynamics is captured. In traditional structures, dynamics is typically
expressed through morphisms between such structures, as in linear maps
between vector spaces or morphisms between rings. In algebras
associated with the semantics of computation, the dynamics is
expressed as part of the algebraic structure itself, through a
reduction reduction relation typically denoted by $\red$. Below, we
give a recursive presentation of this relation for the calculus used
in the encoding.

$\red \subseteq \pi \times \pi$
$\red : \pi \to \mathcal{P}(\pi)$

\begin{mathpar}
  \inferrule* [lab=Comm] { \textsf{match}( x_{src}, x_{trgt} ) } { x_{trgt}?(y)P \; | \; x_{src}!\langle {Q} \rangle \red P\{\quotep{Q}/y}\} }
  \and \\
  \inferrule* [lab=Par] {{P} \red {P}'} {{{P} | {Q}} \red {{P}' | {Q}}}
  \and
  \inferrule* [lab=Equiv]{{{P} \scong {P}'} \andalso {{P}' \red {Q}'} \andalso {{Q}' \scong {Q}}}{{P} \red {Q}}
\end{mathpar}

\begin{eqnarray*}
  match_{\equiv} (\quotep{P},\quotep{Q}) & := & P \equiv Q \\
  match_{\dagger}(\quotep{P},\quotep{Q}) & := & \forall R. P|Q \red^{*} R => R \red^{*} 0 \\
  match_{K}(\quotep{P},\quotep{Q}) & := & K \mbox{ for some context } K
\end{eqnarray*}

$u?(x)P | u!\langle Q \rangle \red P\{\quotep{Q}/x\}$

%We write $\wred$ for $\red^*$, and $P\red$ if $\exists Q $ such that $ P \red Q$.
We write $P\red$ if $\exists Q $ such that $ P \red Q$ and $P\not\red$, otherwise.

\section{Replication}

As mentioned before, it is known that replication (and hence
recursion) can be implemented in a higher-order process algebra
\cite{SangiorgiWalker}. As our first example of calculation with the
machinery thus far presented we give the construction explicitly in
the {\rhoc}.

\begin{eqnarray}
	D_{x} & := & \prefix{x}{y}{(\binpar{\outputp{x}{y}}{@{y}})} \nonumber\\
	\bangp_{x}{P} & := & \binpar{{x}!\langle{\binpar{D_{x}}{P}}\rangle}{D_{x}} \nonumber
\end{eqnarray}

\begin{eqnarray}
	\bangp_{x}{P} & & \nonumber\\
	=
	& {x}!\langle{(\prefix{x}{y}{(\outputp{x}{y} | @{y})) | P}}\rangle 
	      | \prefix{x}{y}{(\outputp{x}{y} | @{y})} & \nonumber\\
	\red
	& (\outputp{x}{y} | @{y})\substn{\quotep{(\prefix{x}{y}{(@{y} | \outputp{x}{y})) | P}}}{y} & \nonumber\\
	=
	& \outputp{x}{\quotep{(\prefix{x}{y}{(\outputp{x}{y} | @{y})) | P}}}
	  | {(\prefix{x}{y}{(\outputp{x}{y} | @{y})) | P}} & \nonumber\\
	\red
	& \ldots & \nonumber\\
	\red^*
	& P | P | \ldots & \nonumber
\end{eqnarray}

Of course, this encoding, as an implementation, runs away, unfolding
$\bangp{P}$ eagerly. A lazier and more implementable replication
operator, restricted to input-guarded processes, may be obtained as follows.

\begin{eqnarray}
\bangp{\prefix{u}{v}{P}} 
	:= 
	\binpar{\lift{x}{\prefix{u}{v}{(\binpar{D(x)}{P})}}}{D(x)} \nonumber
\end{eqnarray}

\begin{remark}
  Note that the lazier definition still does not deal with summation
  or mixed summation (i.e. sums over input and output). The reader is
  invited to construct definitions of replication that deal with these
  features. 

  Further, the definitions are parameterized in a name, $x$. Can you,
  gentle reader, make a definition that eliminates this parameter and
  guarantees no accidental interaction between the replication
  machinery and the process being replicated -- i.e. no accidental
  sharing of names used by the process to get its work done and the
  name(s) used by the replication to effect copying. This latter
  revision of the definition of replication is crucial to obtaining
  the expected identity $!!P \sim !P$.
\end{remark}

\begin{remark}\label{rem:paradoxical_combinator}
  The reader familiar with the lambda calculus will have noticed the
  similarity between $D$ and the paradoxical combinator.

  [Ed. note: the existence of this seems to suggest we have to be more
  restrictive on the set of processes and names we admit if we are to
  support no-cloning.]
\end{remark}

\subsubsection{Bisimulation}

The computational dynamics gives rise to another kind of equivalence,
the equivalence of computational behavior. As previously mentioned
this is typically captured \emph{via} some form of bisimulation.

% The notion we use in this paper is weak barbed bisimulation
% \cite{milner91polyadicpi}.

The notion we use in this paper is derived from weak barbed
bisimulation \cite{milner91polyadicpi}. 

\begin{definition}
An \emph{observation relation}, $\downarrow_{\mathcal N}$, over a set
of names, $\mathcal N$, is the smallest relation satisfying the rules
below.

\infrule[Out-barb]{y \in {\mathcal N}, \; x \nameeq y}
		  {\outputp{x}{v} \downarrow_{\mathcal N} x}
\infrule[Par-barb]{\mbox{$P\downarrow_{\mathcal N} x$ or $Q\downarrow_{\mathcal N} x$}}
		  {\binpar{P}{Q} \downarrow_{\mathcal N} x}

We write $P \Downarrow_{\mathcal N} x$ if there is $Q$ such that 
$P \wred Q$ and $Q \downarrow_{\mathcal N} x$.
\end{definition}

\begin{definition}
%\label{def.bbisim}
An  ${\mathcal N}$-\emph{barbed bisimulation} over a set of names, ${\mathcal N}$, is a symmetric binary relation 
${\mathcal S}_{\mathcal N}$ between agents such that $P\rel{S}_{\mathcal N}Q$ implies:
\begin{enumerate}
\item If $P \red P'$ then $Q \wred Q'$ and $P'\rel{S}_{\mathcal N} Q'$.
\item If $P\downarrow_{\mathcal N} x$, then $Q\Downarrow_{\mathcal N} x$.
\end{enumerate}
$P$ is ${\mathcal N}$-barbed bisimilar to $Q$, written
$P \wbbisim_{\mathcal N} Q$, if $P \rel{S}_{\mathcal N} Q$ for some ${\mathcal N}$-barbed bisimulation ${\mathcal S}_{\mathcal N}$.
\end{definition}

$\mathcal{R} \subseteq \pi \times \pi$

$P \mathcal{R} Q => \forall P'. P \red P' \Rightarrow \exists Q'. Q \red Q', P' \mathcal{R} Q'$

$P \vdash x \Rightarrow Q \vdash x$

\begin{mathpar}
  \inferrule*[lab=Out-barb]{x \nameeq y}{{y}!\langle{Q}\rangle \vdash x}
  \and
  \inferrule*[lab=Par-barb]{\mbox{$P\vdash x$ or $Q\vdash x$}}{\binpar{P}{Q} \vdash x}
\end{mathpar}

\subsubsection{Contexts}

One of the principle advantages of computational calculi like the
$\pi$-calculus is a well-defined notion of context,
contextual-equivalence and a correlation between
contextual-equivalence and notions of bisimulation. The notion of
context allows the decomposition of a process into (sub-)process and
its syntactic environment, its context. Thus, a context may be
thought of as a process with a ``hole'' (written $\Box$) in it. The
application of a context $M$ to a process $P$, written $M[P]$, is
tantamount to filling the hole in $M$ with $P$. In this paper we do
not need the full weight of this theory, but do make use of the notion
of context in the proof the main theorem. 

\begin{mathpar}
  \inferrule* [lab=summation] {} {{M_{M},M_{N}} \bc \Box \;|\; x.M_{A} \;|\; M_{M}+M_{N}}
  \and
  \inferrule* [lab=agent] {} {{M_{A}} \bc (\vec{x})M_{P} \;| \; \clift{P_0,\ldots,M_{P},\ldots,P_N}}
  \and \\
  \inferrule* [lab=process] {} {{M_{P}} \bc M_{N} \;| \;P|M_{P} }
\end{mathpar} 

\begin{mathpar}
  \inferrule* [lab=sychronization] {} {M_{N} \bc \Box \;|\; x?M_{F} \;|\; x!M_{C}}
  \and
  \inferrule* [lab=abstraction] {} {{M_{F}} \bc (x)M_{P} }
  \and
  \inferrule* [lab=concretion] {} {{M_{C}} \bc \langle M_{P} \rangle }
  \and \\
  \inferrule* [lab=process] {} {{M_{P}} \bc M_{N} \;| \;P|M_{P} }
\end{mathpar}

\begin{definition}[contextual application] Given a context $M$, and
  process $P$, we define the \emph{contextual application}, $M[P] :=
  M\{P/\Box\}$. That is, the contextual application of M to P is the
  substitution of $P$ for $\Box$ in $M$.
\end{definition}

$\meaningof{-} : L \to \mathcal{P}(\pi)$

\begin{mathpar}
  \inferrule* [lab=collection] {} {\meaningof{true} = \pi, \and \meaningof{~E} = \pi \setminus \meaningof{E}, \and \meaningof{E_{1} \& E_{2}} = \meaningof{E_{1}} \cap \meaningof{E_{2}}}
\end{mathpar}

\begin{mathpar}
  \inferrule* [lab=structure] {} {\meaningof{0} = \{ P \in \pi | P \equiv 0 \}, \and \\ \meaningof{E_1 | E_2} = \{ P \in \pi | P \equiv P_{1} | P_{2}, P_{1} \in \meaningof{E_{1}}, P_{2} \in \meaningof{E_2}\} }
\end{mathpar}

\begin{mathpar}
 \inferrule* [lab=behavior] {} {\meaningof{\langle a?b \rangle E} = \{ P \in \pi | P \equiv Q | u?(y)P', \\ \and \\\\ \and \\ \;\;\; u \in \meaningof{a}, \forall z.P'\{z/y\} \in \meaningof{E\{z/b\}}\}, \and \\ \meaningof{a!E} = \{ P \in \pi | P \equiv Q | x!\langle P' \rangle, x \in \meaningof{a} P' \in \meaningof{E}\} }
\end{mathpar}

\begin{mathpar}
 \inferrule* [lab=nominal] {} {\meaningof{\quotep{E}} = \{ \quotep{P} \in \quotep{\pi} | P \in \meaningof{E} \}, \and \meaningof{\quotep{P}} = \{ \quotep{Q} \in \quotep{\pi} | P \equiv Q \} \and \\ \meaningof{@\quotep{E}} = \{ P \in \pi | P \equiv @x, x \in \meaningof{E} \}}
\end{mathpar}

\begin{eqnarray*}
  \\
  \meaningof{-} : TS \to ST
\end{eqnarray*}

\begin{eqnarray*}
  \\
  L : TS \to ST
\end{eqnarray*}

\begin{eqnarray*}
  \\
  P \models E \iff P \in \meaningof{E}
\end{eqnarray*}

\begin{eqnarray*}
  P \approx_{L} Q \iff \forall E \in L. P \models E \iff Q \models E
\end{eqnarray*}

\begin{eqnarray*}
  P \approx_{K} Q
\end{eqnarray*}

\begin{eqnarray*}
  P \approx Q
\end{eqnarray*}

$\approx_{K} = \approx = \approx_{L}$

\subsubsection{Contextual duality}

Note that contexts extend the quotation operation to a family of
operations from processes to names. Given a context, $M$, we can
define a \emph{nominal context}, $\quotep{M}$ by $\quotep{M}[P] :=
\quotep{M[P]}$. To foreshadow what is to come we observe that these
operations enjoy a duality with processes very much like the duality
between vectors and maps from vectors to scalars.

Further, because the calculus is essentially higher-order, we have a
correspondence between contexts and processes. More specifically,
given a name $x$ and a context $M$ we can construct $M^{*}_{x}$ such
that 

\begin{mathpar}
  M^{*}_{x} | \lift{x}{P} \red M[P]
\end{mathpar}

namely,

\begin{mathpar}
  M^{*}_{x} := x?(u).M[\dropn{u}]
\end{mathpar}

The dependence of $M^{*}_{x}$ on a name makes it an abstraction, 

\begin{mathpar}
  M^{*} := (x)x?(u).M[\dropn{u}]
\end{mathpar}

\subsection{Additional notation}

It will sometimes be convenient to denote the process a name
quotes. We already have the notation $x = \quotep{P}$, but it will be
convenient to introduce an alternate notation, $\procn{x}$, when we
want to emphasize the connection to the use of the name. Note that, by
virtue of name equivalence, $\quotep{\procn{x}} \nameeq x$; so, the
notation is consistent with previous definitions.

Further, because names have structure it is possible to effect
substitutions on the basis of that structure. This means we need to
upgrade our notation for substitutions, which we accomplish by
adapting comprehension notation. Thus,

\begin{mathpar}
  P\{ y / x : x \in S \}
\end{mathpar}

is interpreted to mean the process derived from P by replacing (in a
capture-avoiding manner) each occurrence of $x$ in $S$ by $y$. For example,

\begin{mathpar}
  P\{ \quotep{\procn{x}|\procn{x}} / x : x \in \freenames{P} \}
\end{mathpar}

will replace each (occurrence) of a free name $x$ in $P$ by
$\quotep{\procn{x}|\procn{x}}$.

Also, we will avail ourselves of the notation $x^{L}$ and $x^{R}$ to
denote injections of a name into disjoint copies of the name
space. There are numerous ways to accomplish this. One example can be
found in \cite{MeredithR05}. This notation overloads to vectors of
names: $\vec{x}^{\pi} := (x_{i}^{\pi} \; : \; 0 \leq i < |\vec{x}| )$ where $\pi \in \{L,R\}$.

We also use $P^{\Box} := P|\Box$.

In \cite{MeredithR05} an interpretation of the new operator is
given. It turns out that there are several possible interpretations
all enjoying the requisite algebraic properties of the operator (see
\cite{milner91polyadicpi}). We will therefore make liberal use of
$(\nu\; \vec{x})P$.

% subsection the_syntax_and_semantics_of_the_notation_system (end)   

\input{qm2pi.qmops} 

\input{qm2pi.sterngerlach} 

\input{qm2pi.metric} 

% section concurrent_process_calculi (end)

%\input{qm2pi.proofsketch}

% section proof sketch (end)

%\input{qm2pi.slviaknots} 

% section spatial logic via knots (end)

\input{qm2pi.conclusion}

% section conclusion (end)

%\input{qm2pi.dtcodes} 

% section wiring algorithm (end)

\input{qm2pi.ack} 

% section acknowledgments (end)

\newpage


\bibliographystyle{plain}   
\bibliography{../../biblios/main.bib}

\input{qm2pi.rhodetails}

\end{document}



\end{document}

 

% section notation (end)

\input{qm2pi.process.calculi} 

% section concurrent_process_calculi_and_spatial_logics_ (end)
    
%\documentclass[12pt]{llncs}
%\documentclass{jktr}

\usepackage[pdftex]{hyperref}                   
\usepackage {listings}
\usepackage {mathpartir}
\usepackage{bcprules}
%\usepackage{listings}
                       
\usepackage{graphicx} 
%\usepackage[margins=2.5cm,nohead,nofoot]{geometry}
%\usepackage{geometry}
\usepackage{amsfonts}
\usepackage{amstext}
\usepackage{latexsym}
\usepackage{amssymb}
\usepackage{color}


%\include{myPreamble}
\documentclass[12pt]{llncs}
%\documentclass{jktr}

\usepackage[pdftex]{hyperref}                   
\usepackage {listings}
\usepackage {mathpartir}
\usepackage{bcprules}
%\usepackage{listings}
                       
\usepackage{graphicx} 
%\usepackage[margins=2.5cm,nohead,nofoot]{geometry}
%\usepackage{geometry}
\usepackage{amsfonts}
\usepackage{amstext}
\usepackage{latexsym}
\usepackage{amssymb}
\usepackage{color}


%\include{myPreamble}
\include{qm2pi.local} 

%\ifpdf
%\usepackage[pdftex]{graphicx}
%\else
%\usepackage{graphicx}
%\fi

 % \ifpdf
%  \usepackage{pdfsync}
%  \if


%\title{Brief Article}
%\author{David F. Snyder}
%\author{L.G. Meredith}

%\address{Dept. of Math., Texas State University--San Marcos, San Marcos, TX 78666}
       
\pagestyle{empty}


\begin{document}

\lstset{language=[Objective]Caml,frame=shadowbox}

\input{qm2pi.front}

% section front matter (end)

\input{qm2pi.intro} 
 
% section introduction (end)

% \input{qm2pi.knotations} 

% section notation (end)

\input{qm2pi.process.calculi} 

% section concurrent_process_calculi_and_spatial_logics_ (end)
    
%\input{qm2pi.knots2pi} 

%\input{qm2pi.trefoil} 

%\input{qm2pi.mainthm} 

% subsection basic_interpretation (end)

%\input{qm2pi.rho.presentation} 
\subsection{The syntax and semantics of the notation system}\label{sub:the_syntax_and_semantics_of_the_notation_system} % (fold)

We now summarize a technical presentation of the calculus that
embodies our theory of dynamics. The typical presentation of such a
calculus follows the style of giving generators and relations on
them. The grammar, below, describing term constructors, freely
generates the set of processes, $\Proc$. This set is then quotiented
by a relation known as structural congruence and it is over this set
that the notion of dynamics is expressed. This presentation is
essentially that of \cite{MeredithR05} with the addition of
polyadicity and summation. For readability we have relegated some of
the technical subtleties to an appendix.

\subsubsection{Process grammar}\label{subsub:process_grammar}

\begin{mathpar}
  \inferrule* [lab=synchronization] {} {{M} \bc \pzero \;|\; x?F \;|\; x!C }
  \and
  \inferrule* [lab=abstraction] {} {{F} \bc (x)P}
  \and
  \inferrule* [lab=concretion] {} {{C} \bc \langle Q \rangle}
  \and
  \inferrule* [lab=process] {} {{P,Q} \bc M \;| \;P|Q \;|\; @{x}}
  \and
  \inferrule* [lab=name] {} {{x} \bc \quotep{P}}
\end{mathpar} 

Note that $\vec{x}$ (resp. $\vec{P}$) denotes a vector of names
(resp. processes) of length $|\vec{x}|$ (resp. $|\vec{P}|$). We adopt
the following useful abbreviations.

\begin{mathpar}
   x?(\vec{y}).P := x.(\vec{y})P \and  x\clift{\vec{P}} := x.\clift{\vec{P}}
   \and x!(y) := \lift{x}{\dropn{y}}
   \and \Pi_{i=0}^{n-1}P_i := P_0 | \ldots | P_{n-1}
\end{mathpar}

\subsubsection{Structural congruence}

\paragraph{Free and bound names and alpha-equivalence.} At the
core of structural equivalence is alpha-equivalence which identifies
process that are the same up to a change of variable. Formally, we
recognize the distinction between free and bound names. The free names
of a process, $\freenames{P}$, may be calculated recursively as
follows:

\begin{mathpar}
\freenames{\pzero} := \emptyset
  \and \\
  \freenames{x?(y).P} := \{ x \} \cup (\freenames{P} \setminus \{ y \})
  \and 
  \freenames{x!\langle P \rangle} := \{ x \} \cup \{ P \} 
  \and \\
  \freenames{P|Q} := \freenames{P} \cup \freenames{Q}
  \and \\
  \freenames{@{x}} := \{ x \}
\end{mathpar}

$\pi$
$\quotep{\pi}$

$\freenames{-} : \pi \to \mathcal{P}(\quotep{\pi})$

\begin{eqnarray*}
  \freenames{\pzero} & := & \emptyset \\
  \freenames{x?(y).P} & := & \{ x \} \cup (\freenames{P} \setminus \{ y \}) \\
  \freenames{x!\langle P \rangle} & := & \{ x \} \cup \{ P \} \\
  \freenames{P|Q} & := & \freenames{P} \cup \freenames{Q} \\
  \freenames{\dropn{x}} & := & \{ x \}
\end{eqnarray*}

The bound names of a process, $\boundnames{P}$, are those names occurring in $P$
that are not free. For example, in $x?(y).0$, the name $x$ is free, while $y$ is bound.

\begin{mathpar}
  \inferrule* [lab=monoidal-laws] {} { P|Q \equiv Q|P \and P|0 \equiv P \and P|(Q|R) \equiv (P|Q)|R }
\end{mathpar}

\begin{mathpar}
  \inferrule* [lab=alpha-equivalence] {} { (x)P \equiv (y)P\{y/x\} \and y \not\in \freenames{P} }
\end{mathpar}

\begin{definition}
Then two processes, $P,Q$, are alpha-equivalent if $P = Q\{\vec{y}/\vec{x}\}$ for
some $\vec{x} \in \boundnames{Q},\vec{y} \in \boundnames{P}$, where $Q\{\vec{y}/\vec{x}\}$
denotes the capture-avoiding substitution of $\vec{y}$ for $\vec{x}$ in $Q$.
\end{definition}

\begin{definition}
  The {\em structural congruence} \cite{SangiorgiWalker} , $\equiv$,
  between processes is the least congruence containing
  alpha-equivalence, satisfying the abelian monoid laws
  (associativity, commutativity and $\pzero$ as identity) for parallel
  composition $|$ and for summation $+$.
\end{definition}

\subsection{Name equivalence}

We take name equivalence, written $\nameeq$, to be the smallest
equivalence relation generated by the following rules.

\begin{mathpar}
\inferrule*[lab=Quote-drop]
{ }
{ \quotep{@{x}} \nameeq x }

\inferrule*[lab=Struct-equiv]
{ P \scong Q }
{ \quotep{P} \nameeq \quotep{Q} }
\end{mathpar}

The astute reader will have noticed that the mutual recursion of names
and processes imposes a mutual recursion on alpha-equivalence and
structural equivalence via name-equivalence. Fortunately, all of this
works out pleasantly and we may calculate in the natural way, free of
concern. The reader interested in the details is referred to the
appendix \ref{appendix:rho_details}.

\subsection{Substitution}

We use $\Proc$ for the set of processes, $\QProc$ for the set of
names, and $\id{\{}\vec{y} / \vec{x} \id{\}}$ to denote partial maps,
$s : \QProc \rightarrow \QProc$. A map, $s$ lifts, uniquely, to a map
on process terms, $\widehat{s} : \Proc \rightarrow \Proc$ by the
following equations.

\begin{mathpar}
  (0) \psubstp{Q}{P} := 0 \\
  (R \juxtap S) \psubstp{Q}{P}
  :=    
  (R)\psubstp{Q}{P} \juxtap (S) \psubstp{Q}{P} \\
  (x?(y).R) \psubstp{Q}{P}    
  :=    
  (x)\substp{Q}{P} (z)\concat( (R \psubstn{z}{y}) \psubstp{Q}{P} ) \\
  (\lift{x}{R}) \psubstp{Q}{P}  
  :=
  \lift{(x)\substp{Q}{P}}{ R \psubstp{Q}{P} } \\
%   (\dropn{x})  \psubstp{Q}{P}       
%   := 
%   \left\{ 
%     \begin{array}{ccc} 
%       \dropn{\quotep{Q}} & & x \nameeq \quotep{P} \\
%       \dropn{x} & & otherwise \\
%     \end{array}
%   \right. 
  (\dropn{x})  \psubstp{Q}{P}       
  := 
  \left\{ 
    \begin{array}{ccc} 
      Q & & x \nameeq \quotep{P} \\
      \dropn{x} & & otherwise \\
    \end{array}
  \right.
\end{mathpar}
 

where

\begin{eqnarray}
  (x)\id{\{} \lpquote Q \rpquote / \lpquote P \rpquote \id{\}}            = 
  \left\{ 
    \begin{array}{ccc}
      \lpquote Q \rpquote & & x \nameeq \lpquote P \rpquote \\
      x & & otherwise \\
    \end{array}
  \right. \nonumber
\end{eqnarray}

and $z$ is chosen distinct from $\quotep{P}$, $\quotep{Q}$, the free
names in $Q$, and all the names in $R$. Our $\alpha$-equivalence will
be built in the standard way from this substitution.

\begin{remark}\label{rem:no_self_referential_names}
  One consequence of these definitions is that $\forall P. \quotep{P}
  \not\in \freenames{P}$.
\end{remark}

\subsection{ Dynamic quote: an example }

Anticipating something of what's to come, consider applying the
substitution, $\widehat{\id{\{}u / z \id{\}}}$, to the following pair
of processes, $\lift{w}{y!(z)}$ and $w[ \lpquote y!(z) \rpquote ]$.

\begin{eqnarray}
	\lift{w}{y!(z)}\widehat{\id{\{}u / z \id{\}}}
		& = &
		\lift{w}{y!(u)} \nonumber\\
	w[ \lpquote y!(z) \rpquote ] \widehat{ \id{\{}u / z \id{\}} }
		& = &
		w[ \lpquote y!(z) \rpquote ] \nonumber
\end{eqnarray}

Because the body of the process between quotes is impervious to
substitution, we get radically different answers. In fact, by
examining the first process in an input context,
e.g. $x?(z).\lift{w}{y!(z)}$, we see that the process under the lift
operator may be shaped by prefixed inputs binding a name inside it. In
this sense, the lift operator will be seen as a way to dynamically
construct processes before reifying them as names.

Finally equipped with these standard features we can present the
dynamics of the calculus.

\subsubsection{Operational semantics} 

Finally, we introduce the computational dynamics. What marks these
algebras as distinct from other more traditionally studied algebraic
structures, e.g. vector spaces or polynomial rings, is the manner in
which dynamics is captured. In traditional structures, dynamics is typically
expressed through morphisms between such structures, as in linear maps
between vector spaces or morphisms between rings. In algebras
associated with the semantics of computation, the dynamics is
expressed as part of the algebraic structure itself, through a
reduction reduction relation typically denoted by $\red$. Below, we
give a recursive presentation of this relation for the calculus used
in the encoding.

$\red \subseteq \pi \times \pi$
$\red : \pi \to \mathcal{P}(\pi)$

\begin{mathpar}
  \inferrule* [lab=Comm] { \textsf{match}( x_{src}, x_{trgt} ) } { x_{trgt}?(y)P \; | \; x_{src}!\langle {Q} \rangle \red P\{\quotep{Q}/y}\} }
  \and \\
  \inferrule* [lab=Par] {{P} \red {P}'} {{{P} | {Q}} \red {{P}' | {Q}}}
  \and
  \inferrule* [lab=Equiv]{{{P} \scong {P}'} \andalso {{P}' \red {Q}'} \andalso {{Q}' \scong {Q}}}{{P} \red {Q}}
\end{mathpar}

\begin{eqnarray*}
  match_{\equiv} (\quotep{P},\quotep{Q}) & := & P \equiv Q \\
  match_{\dagger}(\quotep{P},\quotep{Q}) & := & \forall R. P|Q \red^{*} R => R \red^{*} 0 \\
  match_{K}(\quotep{P},\quotep{Q}) & := & K \mbox{ for some context } K
\end{eqnarray*}

$u?(x)P | u!\langle Q \rangle \red P\{\quotep{Q}/x\}$

%We write $\wred$ for $\red^*$, and $P\red$ if $\exists Q $ such that $ P \red Q$.
We write $P\red$ if $\exists Q $ such that $ P \red Q$ and $P\not\red$, otherwise.

\section{Replication}

As mentioned before, it is known that replication (and hence
recursion) can be implemented in a higher-order process algebra
\cite{SangiorgiWalker}. As our first example of calculation with the
machinery thus far presented we give the construction explicitly in
the {\rhoc}.

\begin{eqnarray}
	D_{x} & := & \prefix{x}{y}{(\binpar{\outputp{x}{y}}{@{y}})} \nonumber\\
	\bangp_{x}{P} & := & \binpar{{x}!\langle{\binpar{D_{x}}{P}}\rangle}{D_{x}} \nonumber
\end{eqnarray}

\begin{eqnarray}
	\bangp_{x}{P} & & \nonumber\\
	=
	& {x}!\langle{(\prefix{x}{y}{(\outputp{x}{y} | @{y})) | P}}\rangle 
	      | \prefix{x}{y}{(\outputp{x}{y} | @{y})} & \nonumber\\
	\red
	& (\outputp{x}{y} | @{y})\substn{\quotep{(\prefix{x}{y}{(@{y} | \outputp{x}{y})) | P}}}{y} & \nonumber\\
	=
	& \outputp{x}{\quotep{(\prefix{x}{y}{(\outputp{x}{y} | @{y})) | P}}}
	  | {(\prefix{x}{y}{(\outputp{x}{y} | @{y})) | P}} & \nonumber\\
	\red
	& \ldots & \nonumber\\
	\red^*
	& P | P | \ldots & \nonumber
\end{eqnarray}

Of course, this encoding, as an implementation, runs away, unfolding
$\bangp{P}$ eagerly. A lazier and more implementable replication
operator, restricted to input-guarded processes, may be obtained as follows.

\begin{eqnarray}
\bangp{\prefix{u}{v}{P}} 
	:= 
	\binpar{\lift{x}{\prefix{u}{v}{(\binpar{D(x)}{P})}}}{D(x)} \nonumber
\end{eqnarray}

\begin{remark}
  Note that the lazier definition still does not deal with summation
  or mixed summation (i.e. sums over input and output). The reader is
  invited to construct definitions of replication that deal with these
  features. 

  Further, the definitions are parameterized in a name, $x$. Can you,
  gentle reader, make a definition that eliminates this parameter and
  guarantees no accidental interaction between the replication
  machinery and the process being replicated -- i.e. no accidental
  sharing of names used by the process to get its work done and the
  name(s) used by the replication to effect copying. This latter
  revision of the definition of replication is crucial to obtaining
  the expected identity $!!P \sim !P$.
\end{remark}

\begin{remark}\label{rem:paradoxical_combinator}
  The reader familiar with the lambda calculus will have noticed the
  similarity between $D$ and the paradoxical combinator.

  [Ed. note: the existence of this seems to suggest we have to be more
  restrictive on the set of processes and names we admit if we are to
  support no-cloning.]
\end{remark}

\subsubsection{Bisimulation}

The computational dynamics gives rise to another kind of equivalence,
the equivalence of computational behavior. As previously mentioned
this is typically captured \emph{via} some form of bisimulation.

% The notion we use in this paper is weak barbed bisimulation
% \cite{milner91polyadicpi}.

The notion we use in this paper is derived from weak barbed
bisimulation \cite{milner91polyadicpi}. 

\begin{definition}
An \emph{observation relation}, $\downarrow_{\mathcal N}$, over a set
of names, $\mathcal N$, is the smallest relation satisfying the rules
below.

\infrule[Out-barb]{y \in {\mathcal N}, \; x \nameeq y}
		  {\outputp{x}{v} \downarrow_{\mathcal N} x}
\infrule[Par-barb]{\mbox{$P\downarrow_{\mathcal N} x$ or $Q\downarrow_{\mathcal N} x$}}
		  {\binpar{P}{Q} \downarrow_{\mathcal N} x}

We write $P \Downarrow_{\mathcal N} x$ if there is $Q$ such that 
$P \wred Q$ and $Q \downarrow_{\mathcal N} x$.
\end{definition}

\begin{definition}
%\label{def.bbisim}
An  ${\mathcal N}$-\emph{barbed bisimulation} over a set of names, ${\mathcal N}$, is a symmetric binary relation 
${\mathcal S}_{\mathcal N}$ between agents such that $P\rel{S}_{\mathcal N}Q$ implies:
\begin{enumerate}
\item If $P \red P'$ then $Q \wred Q'$ and $P'\rel{S}_{\mathcal N} Q'$.
\item If $P\downarrow_{\mathcal N} x$, then $Q\Downarrow_{\mathcal N} x$.
\end{enumerate}
$P$ is ${\mathcal N}$-barbed bisimilar to $Q$, written
$P \wbbisim_{\mathcal N} Q$, if $P \rel{S}_{\mathcal N} Q$ for some ${\mathcal N}$-barbed bisimulation ${\mathcal S}_{\mathcal N}$.
\end{definition}

$\mathcal{R} \subseteq \pi \times \pi$

$P \mathcal{R} Q => \forall P'. P \red P' \Rightarrow \exists Q'. Q \red Q', P' \mathcal{R} Q'$

$P \vdash x \Rightarrow Q \vdash x$

\begin{mathpar}
  \inferrule*[lab=Out-barb]{x \nameeq y}{{y}!\langle{Q}\rangle \vdash x}
  \and
  \inferrule*[lab=Par-barb]{\mbox{$P\vdash x$ or $Q\vdash x$}}{\binpar{P}{Q} \vdash x}
\end{mathpar}

\subsubsection{Contexts}

One of the principle advantages of computational calculi like the
$\pi$-calculus is a well-defined notion of context,
contextual-equivalence and a correlation between
contextual-equivalence and notions of bisimulation. The notion of
context allows the decomposition of a process into (sub-)process and
its syntactic environment, its context. Thus, a context may be
thought of as a process with a ``hole'' (written $\Box$) in it. The
application of a context $M$ to a process $P$, written $M[P]$, is
tantamount to filling the hole in $M$ with $P$. In this paper we do
not need the full weight of this theory, but do make use of the notion
of context in the proof the main theorem. 

\begin{mathpar}
  \inferrule* [lab=summation] {} {{M_{M},M_{N}} \bc \Box \;|\; x.M_{A} \;|\; M_{M}+M_{N}}
  \and
  \inferrule* [lab=agent] {} {{M_{A}} \bc (\vec{x})M_{P} \;| \; \clift{P_0,\ldots,M_{P},\ldots,P_N}}
  \and \\
  \inferrule* [lab=process] {} {{M_{P}} \bc M_{N} \;| \;P|M_{P} }
\end{mathpar} 

\begin{mathpar}
  \inferrule* [lab=sychronization] {} {M_{N} \bc \Box \;|\; x?M_{F} \;|\; x!M_{C}}
  \and
  \inferrule* [lab=abstraction] {} {{M_{F}} \bc (x)M_{P} }
  \and
  \inferrule* [lab=concretion] {} {{M_{C}} \bc \langle M_{P} \rangle }
  \and \\
  \inferrule* [lab=process] {} {{M_{P}} \bc M_{N} \;| \;P|M_{P} }
\end{mathpar}

\begin{definition}[contextual application] Given a context $M$, and
  process $P$, we define the \emph{contextual application}, $M[P] :=
  M\{P/\Box\}$. That is, the contextual application of M to P is the
  substitution of $P$ for $\Box$ in $M$.
\end{definition}

$\meaningof{-} : L \to \mathcal{P}(\pi)$

\begin{mathpar}
  \inferrule* [lab=collection] {} {\meaningof{true} = \pi, \and \meaningof{~E} = \pi \setminus \meaningof{E}, \and \meaningof{E_{1} \& E_{2}} = \meaningof{E_{1}} \cap \meaningof{E_{2}}}
\end{mathpar}

\begin{mathpar}
  \inferrule* [lab=structure] {} {\meaningof{0} = \{ P \in \pi | P \equiv 0 \}, \and \\ \meaningof{E_1 | E_2} = \{ P \in \pi | P \equiv P_{1} | P_{2}, P_{1} \in \meaningof{E_{1}}, P_{2} \in \meaningof{E_2}\} }
\end{mathpar}

\begin{mathpar}
 \inferrule* [lab=behavior] {} {\meaningof{\langle a?b \rangle E} = \{ P \in \pi | P \equiv Q | u?(y)P', \\ \and \\\\ \and \\ \;\;\; u \in \meaningof{a}, \forall z.P'\{z/y\} \in \meaningof{E\{z/b\}}\}, \and \\ \meaningof{a!E} = \{ P \in \pi | P \equiv Q | x!\langle P' \rangle, x \in \meaningof{a} P' \in \meaningof{E}\} }
\end{mathpar}

\begin{mathpar}
 \inferrule* [lab=nominal] {} {\meaningof{\quotep{E}} = \{ \quotep{P} \in \quotep{\pi} | P \in \meaningof{E} \}, \and \meaningof{\quotep{P}} = \{ \quotep{Q} \in \quotep{\pi} | P \equiv Q \} \and \\ \meaningof{@\quotep{E}} = \{ P \in \pi | P \equiv @x, x \in \meaningof{E} \}}
\end{mathpar}

\begin{eqnarray*}
  \\
  \meaningof{-} : TS \to ST
\end{eqnarray*}

\begin{eqnarray*}
  \\
  L : TS \to ST
\end{eqnarray*}

\begin{eqnarray*}
  \\
  P \models E \iff P \in \meaningof{E}
\end{eqnarray*}

\begin{eqnarray*}
  P \approx_{L} Q \iff \forall E \in L. P \models E \iff Q \models E
\end{eqnarray*}

\begin{eqnarray*}
  P \approx_{K} Q
\end{eqnarray*}

\begin{eqnarray*}
  P \approx Q
\end{eqnarray*}

$\approx_{K} = \approx = \approx_{L}$

\subsubsection{Contextual duality}

Note that contexts extend the quotation operation to a family of
operations from processes to names. Given a context, $M$, we can
define a \emph{nominal context}, $\quotep{M}$ by $\quotep{M}[P] :=
\quotep{M[P]}$. To foreshadow what is to come we observe that these
operations enjoy a duality with processes very much like the duality
between vectors and maps from vectors to scalars.

Further, because the calculus is essentially higher-order, we have a
correspondence between contexts and processes. More specifically,
given a name $x$ and a context $M$ we can construct $M^{*}_{x}$ such
that 

\begin{mathpar}
  M^{*}_{x} | \lift{x}{P} \red M[P]
\end{mathpar}

namely,

\begin{mathpar}
  M^{*}_{x} := x?(u).M[\dropn{u}]
\end{mathpar}

The dependence of $M^{*}_{x}$ on a name makes it an abstraction, 

\begin{mathpar}
  M^{*} := (x)x?(u).M[\dropn{u}]
\end{mathpar}

\subsection{Additional notation}

It will sometimes be convenient to denote the process a name
quotes. We already have the notation $x = \quotep{P}$, but it will be
convenient to introduce an alternate notation, $\procn{x}$, when we
want to emphasize the connection to the use of the name. Note that, by
virtue of name equivalence, $\quotep{\procn{x}} \nameeq x$; so, the
notation is consistent with previous definitions.

Further, because names have structure it is possible to effect
substitutions on the basis of that structure. This means we need to
upgrade our notation for substitutions, which we accomplish by
adapting comprehension notation. Thus,

\begin{mathpar}
  P\{ y / x : x \in S \}
\end{mathpar}

is interpreted to mean the process derived from P by replacing (in a
capture-avoiding manner) each occurrence of $x$ in $S$ by $y$. For example,

\begin{mathpar}
  P\{ \quotep{\procn{x}|\procn{x}} / x : x \in \freenames{P} \}
\end{mathpar}

will replace each (occurrence) of a free name $x$ in $P$ by
$\quotep{\procn{x}|\procn{x}}$.

Also, we will avail ourselves of the notation $x^{L}$ and $x^{R}$ to
denote injections of a name into disjoint copies of the name
space. There are numerous ways to accomplish this. One example can be
found in \cite{MeredithR05}. This notation overloads to vectors of
names: $\vec{x}^{\pi} := (x_{i}^{\pi} \; : \; 0 \leq i < |\vec{x}| )$ where $\pi \in \{L,R\}$.

We also use $P^{\Box} := P|\Box$.

In \cite{MeredithR05} an interpretation of the new operator is
given. It turns out that there are several possible interpretations
all enjoying the requisite algebraic properties of the operator (see
\cite{milner91polyadicpi}). We will therefore make liberal use of
$(\nu\; \vec{x})P$.

% subsection the_syntax_and_semantics_of_the_notation_system (end)   

\input{qm2pi.qmops} 

\input{qm2pi.sterngerlach} 

\input{qm2pi.metric} 

% section concurrent_process_calculi (end)

%\input{qm2pi.proofsketch}

% section proof sketch (end)

%\input{qm2pi.slviaknots} 

% section spatial logic via knots (end)

\input{qm2pi.conclusion}

% section conclusion (end)

%\input{qm2pi.dtcodes} 

% section wiring algorithm (end)

\input{qm2pi.ack} 

% section acknowledgments (end)

\newpage


\bibliographystyle{plain}   
\bibliography{../../biblios/main.bib}

\input{qm2pi.rhodetails}

\end{document}

 

%\ifpdf
%\usepackage[pdftex]{graphicx}
%\else
%\usepackage{graphicx}
%\fi

 % \ifpdf
%  \usepackage{pdfsync}
%  \if


%\title{Brief Article}
%\author{David F. Snyder}
%\author{L.G. Meredith}

%\address{Dept. of Math., Texas State University--San Marcos, San Marcos, TX 78666}
       
\pagestyle{empty}


\begin{document}

\lstset{language=[Objective]Caml,frame=shadowbox}

\documentclass[12pt]{llncs}
%\documentclass{jktr}

\usepackage[pdftex]{hyperref}                   
\usepackage {listings}
\usepackage {mathpartir}
\usepackage{bcprules}
%\usepackage{listings}
                       
\usepackage{graphicx} 
%\usepackage[margins=2.5cm,nohead,nofoot]{geometry}
%\usepackage{geometry}
\usepackage{amsfonts}
\usepackage{amstext}
\usepackage{latexsym}
\usepackage{amssymb}
\usepackage{color}


%\include{myPreamble}
\include{qm2pi.local} 

%\ifpdf
%\usepackage[pdftex]{graphicx}
%\else
%\usepackage{graphicx}
%\fi

 % \ifpdf
%  \usepackage{pdfsync}
%  \if


%\title{Brief Article}
%\author{David F. Snyder}
%\author{L.G. Meredith}

%\address{Dept. of Math., Texas State University--San Marcos, San Marcos, TX 78666}
       
\pagestyle{empty}


\begin{document}

\lstset{language=[Objective]Caml,frame=shadowbox}

\input{qm2pi.front}

% section front matter (end)

\input{qm2pi.intro} 
 
% section introduction (end)

% \input{qm2pi.knotations} 

% section notation (end)

\input{qm2pi.process.calculi} 

% section concurrent_process_calculi_and_spatial_logics_ (end)
    
%\input{qm2pi.knots2pi} 

%\input{qm2pi.trefoil} 

%\input{qm2pi.mainthm} 

% subsection basic_interpretation (end)

%\input{qm2pi.rho.presentation} 
\subsection{The syntax and semantics of the notation system}\label{sub:the_syntax_and_semantics_of_the_notation_system} % (fold)

We now summarize a technical presentation of the calculus that
embodies our theory of dynamics. The typical presentation of such a
calculus follows the style of giving generators and relations on
them. The grammar, below, describing term constructors, freely
generates the set of processes, $\Proc$. This set is then quotiented
by a relation known as structural congruence and it is over this set
that the notion of dynamics is expressed. This presentation is
essentially that of \cite{MeredithR05} with the addition of
polyadicity and summation. For readability we have relegated some of
the technical subtleties to an appendix.

\subsubsection{Process grammar}\label{subsub:process_grammar}

\begin{mathpar}
  \inferrule* [lab=synchronization] {} {{M} \bc \pzero \;|\; x?F \;|\; x!C }
  \and
  \inferrule* [lab=abstraction] {} {{F} \bc (x)P}
  \and
  \inferrule* [lab=concretion] {} {{C} \bc \langle Q \rangle}
  \and
  \inferrule* [lab=process] {} {{P,Q} \bc M \;| \;P|Q \;|\; @{x}}
  \and
  \inferrule* [lab=name] {} {{x} \bc \quotep{P}}
\end{mathpar} 

Note that $\vec{x}$ (resp. $\vec{P}$) denotes a vector of names
(resp. processes) of length $|\vec{x}|$ (resp. $|\vec{P}|$). We adopt
the following useful abbreviations.

\begin{mathpar}
   x?(\vec{y}).P := x.(\vec{y})P \and  x\clift{\vec{P}} := x.\clift{\vec{P}}
   \and x!(y) := \lift{x}{\dropn{y}}
   \and \Pi_{i=0}^{n-1}P_i := P_0 | \ldots | P_{n-1}
\end{mathpar}

\subsubsection{Structural congruence}

\paragraph{Free and bound names and alpha-equivalence.} At the
core of structural equivalence is alpha-equivalence which identifies
process that are the same up to a change of variable. Formally, we
recognize the distinction between free and bound names. The free names
of a process, $\freenames{P}$, may be calculated recursively as
follows:

\begin{mathpar}
\freenames{\pzero} := \emptyset
  \and \\
  \freenames{x?(y).P} := \{ x \} \cup (\freenames{P} \setminus \{ y \})
  \and 
  \freenames{x!\langle P \rangle} := \{ x \} \cup \{ P \} 
  \and \\
  \freenames{P|Q} := \freenames{P} \cup \freenames{Q}
  \and \\
  \freenames{@{x}} := \{ x \}
\end{mathpar}

$\pi$
$\quotep{\pi}$

$\freenames{-} : \pi \to \mathcal{P}(\quotep{\pi})$

\begin{eqnarray*}
  \freenames{\pzero} & := & \emptyset \\
  \freenames{x?(y).P} & := & \{ x \} \cup (\freenames{P} \setminus \{ y \}) \\
  \freenames{x!\langle P \rangle} & := & \{ x \} \cup \{ P \} \\
  \freenames{P|Q} & := & \freenames{P} \cup \freenames{Q} \\
  \freenames{\dropn{x}} & := & \{ x \}
\end{eqnarray*}

The bound names of a process, $\boundnames{P}$, are those names occurring in $P$
that are not free. For example, in $x?(y).0$, the name $x$ is free, while $y$ is bound.

\begin{mathpar}
  \inferrule* [lab=monoidal-laws] {} { P|Q \equiv Q|P \and P|0 \equiv P \and P|(Q|R) \equiv (P|Q)|R }
\end{mathpar}

\begin{mathpar}
  \inferrule* [lab=alpha-equivalence] {} { (x)P \equiv (y)P\{y/x\} \and y \not\in \freenames{P} }
\end{mathpar}

\begin{definition}
Then two processes, $P,Q$, are alpha-equivalent if $P = Q\{\vec{y}/\vec{x}\}$ for
some $\vec{x} \in \boundnames{Q},\vec{y} \in \boundnames{P}$, where $Q\{\vec{y}/\vec{x}\}$
denotes the capture-avoiding substitution of $\vec{y}$ for $\vec{x}$ in $Q$.
\end{definition}

\begin{definition}
  The {\em structural congruence} \cite{SangiorgiWalker} , $\equiv$,
  between processes is the least congruence containing
  alpha-equivalence, satisfying the abelian monoid laws
  (associativity, commutativity and $\pzero$ as identity) for parallel
  composition $|$ and for summation $+$.
\end{definition}

\subsection{Name equivalence}

We take name equivalence, written $\nameeq$, to be the smallest
equivalence relation generated by the following rules.

\begin{mathpar}
\inferrule*[lab=Quote-drop]
{ }
{ \quotep{@{x}} \nameeq x }

\inferrule*[lab=Struct-equiv]
{ P \scong Q }
{ \quotep{P} \nameeq \quotep{Q} }
\end{mathpar}

The astute reader will have noticed that the mutual recursion of names
and processes imposes a mutual recursion on alpha-equivalence and
structural equivalence via name-equivalence. Fortunately, all of this
works out pleasantly and we may calculate in the natural way, free of
concern. The reader interested in the details is referred to the
appendix \ref{appendix:rho_details}.

\subsection{Substitution}

We use $\Proc$ for the set of processes, $\QProc$ for the set of
names, and $\id{\{}\vec{y} / \vec{x} \id{\}}$ to denote partial maps,
$s : \QProc \rightarrow \QProc$. A map, $s$ lifts, uniquely, to a map
on process terms, $\widehat{s} : \Proc \rightarrow \Proc$ by the
following equations.

\begin{mathpar}
  (0) \psubstp{Q}{P} := 0 \\
  (R \juxtap S) \psubstp{Q}{P}
  :=    
  (R)\psubstp{Q}{P} \juxtap (S) \psubstp{Q}{P} \\
  (x?(y).R) \psubstp{Q}{P}    
  :=    
  (x)\substp{Q}{P} (z)\concat( (R \psubstn{z}{y}) \psubstp{Q}{P} ) \\
  (\lift{x}{R}) \psubstp{Q}{P}  
  :=
  \lift{(x)\substp{Q}{P}}{ R \psubstp{Q}{P} } \\
%   (\dropn{x})  \psubstp{Q}{P}       
%   := 
%   \left\{ 
%     \begin{array}{ccc} 
%       \dropn{\quotep{Q}} & & x \nameeq \quotep{P} \\
%       \dropn{x} & & otherwise \\
%     \end{array}
%   \right. 
  (\dropn{x})  \psubstp{Q}{P}       
  := 
  \left\{ 
    \begin{array}{ccc} 
      Q & & x \nameeq \quotep{P} \\
      \dropn{x} & & otherwise \\
    \end{array}
  \right.
\end{mathpar}
 

where

\begin{eqnarray}
  (x)\id{\{} \lpquote Q \rpquote / \lpquote P \rpquote \id{\}}            = 
  \left\{ 
    \begin{array}{ccc}
      \lpquote Q \rpquote & & x \nameeq \lpquote P \rpquote \\
      x & & otherwise \\
    \end{array}
  \right. \nonumber
\end{eqnarray}

and $z$ is chosen distinct from $\quotep{P}$, $\quotep{Q}$, the free
names in $Q$, and all the names in $R$. Our $\alpha$-equivalence will
be built in the standard way from this substitution.

\begin{remark}\label{rem:no_self_referential_names}
  One consequence of these definitions is that $\forall P. \quotep{P}
  \not\in \freenames{P}$.
\end{remark}

\subsection{ Dynamic quote: an example }

Anticipating something of what's to come, consider applying the
substitution, $\widehat{\id{\{}u / z \id{\}}}$, to the following pair
of processes, $\lift{w}{y!(z)}$ and $w[ \lpquote y!(z) \rpquote ]$.

\begin{eqnarray}
	\lift{w}{y!(z)}\widehat{\id{\{}u / z \id{\}}}
		& = &
		\lift{w}{y!(u)} \nonumber\\
	w[ \lpquote y!(z) \rpquote ] \widehat{ \id{\{}u / z \id{\}} }
		& = &
		w[ \lpquote y!(z) \rpquote ] \nonumber
\end{eqnarray}

Because the body of the process between quotes is impervious to
substitution, we get radically different answers. In fact, by
examining the first process in an input context,
e.g. $x?(z).\lift{w}{y!(z)}$, we see that the process under the lift
operator may be shaped by prefixed inputs binding a name inside it. In
this sense, the lift operator will be seen as a way to dynamically
construct processes before reifying them as names.

Finally equipped with these standard features we can present the
dynamics of the calculus.

\subsubsection{Operational semantics} 

Finally, we introduce the computational dynamics. What marks these
algebras as distinct from other more traditionally studied algebraic
structures, e.g. vector spaces or polynomial rings, is the manner in
which dynamics is captured. In traditional structures, dynamics is typically
expressed through morphisms between such structures, as in linear maps
between vector spaces or morphisms between rings. In algebras
associated with the semantics of computation, the dynamics is
expressed as part of the algebraic structure itself, through a
reduction reduction relation typically denoted by $\red$. Below, we
give a recursive presentation of this relation for the calculus used
in the encoding.

$\red \subseteq \pi \times \pi$
$\red : \pi \to \mathcal{P}(\pi)$

\begin{mathpar}
  \inferrule* [lab=Comm] { \textsf{match}( x_{src}, x_{trgt} ) } { x_{trgt}?(y)P \; | \; x_{src}!\langle {Q} \rangle \red P\{\quotep{Q}/y}\} }
  \and \\
  \inferrule* [lab=Par] {{P} \red {P}'} {{{P} | {Q}} \red {{P}' | {Q}}}
  \and
  \inferrule* [lab=Equiv]{{{P} \scong {P}'} \andalso {{P}' \red {Q}'} \andalso {{Q}' \scong {Q}}}{{P} \red {Q}}
\end{mathpar}

\begin{eqnarray*}
  match_{\equiv} (\quotep{P},\quotep{Q}) & := & P \equiv Q \\
  match_{\dagger}(\quotep{P},\quotep{Q}) & := & \forall R. P|Q \red^{*} R => R \red^{*} 0 \\
  match_{K}(\quotep{P},\quotep{Q}) & := & K \mbox{ for some context } K
\end{eqnarray*}

$u?(x)P | u!\langle Q \rangle \red P\{\quotep{Q}/x\}$

%We write $\wred$ for $\red^*$, and $P\red$ if $\exists Q $ such that $ P \red Q$.
We write $P\red$ if $\exists Q $ such that $ P \red Q$ and $P\not\red$, otherwise.

\section{Replication}

As mentioned before, it is known that replication (and hence
recursion) can be implemented in a higher-order process algebra
\cite{SangiorgiWalker}. As our first example of calculation with the
machinery thus far presented we give the construction explicitly in
the {\rhoc}.

\begin{eqnarray}
	D_{x} & := & \prefix{x}{y}{(\binpar{\outputp{x}{y}}{@{y}})} \nonumber\\
	\bangp_{x}{P} & := & \binpar{{x}!\langle{\binpar{D_{x}}{P}}\rangle}{D_{x}} \nonumber
\end{eqnarray}

\begin{eqnarray}
	\bangp_{x}{P} & & \nonumber\\
	=
	& {x}!\langle{(\prefix{x}{y}{(\outputp{x}{y} | @{y})) | P}}\rangle 
	      | \prefix{x}{y}{(\outputp{x}{y} | @{y})} & \nonumber\\
	\red
	& (\outputp{x}{y} | @{y})\substn{\quotep{(\prefix{x}{y}{(@{y} | \outputp{x}{y})) | P}}}{y} & \nonumber\\
	=
	& \outputp{x}{\quotep{(\prefix{x}{y}{(\outputp{x}{y} | @{y})) | P}}}
	  | {(\prefix{x}{y}{(\outputp{x}{y} | @{y})) | P}} & \nonumber\\
	\red
	& \ldots & \nonumber\\
	\red^*
	& P | P | \ldots & \nonumber
\end{eqnarray}

Of course, this encoding, as an implementation, runs away, unfolding
$\bangp{P}$ eagerly. A lazier and more implementable replication
operator, restricted to input-guarded processes, may be obtained as follows.

\begin{eqnarray}
\bangp{\prefix{u}{v}{P}} 
	:= 
	\binpar{\lift{x}{\prefix{u}{v}{(\binpar{D(x)}{P})}}}{D(x)} \nonumber
\end{eqnarray}

\begin{remark}
  Note that the lazier definition still does not deal with summation
  or mixed summation (i.e. sums over input and output). The reader is
  invited to construct definitions of replication that deal with these
  features. 

  Further, the definitions are parameterized in a name, $x$. Can you,
  gentle reader, make a definition that eliminates this parameter and
  guarantees no accidental interaction between the replication
  machinery and the process being replicated -- i.e. no accidental
  sharing of names used by the process to get its work done and the
  name(s) used by the replication to effect copying. This latter
  revision of the definition of replication is crucial to obtaining
  the expected identity $!!P \sim !P$.
\end{remark}

\begin{remark}\label{rem:paradoxical_combinator}
  The reader familiar with the lambda calculus will have noticed the
  similarity between $D$ and the paradoxical combinator.

  [Ed. note: the existence of this seems to suggest we have to be more
  restrictive on the set of processes and names we admit if we are to
  support no-cloning.]
\end{remark}

\subsubsection{Bisimulation}

The computational dynamics gives rise to another kind of equivalence,
the equivalence of computational behavior. As previously mentioned
this is typically captured \emph{via} some form of bisimulation.

% The notion we use in this paper is weak barbed bisimulation
% \cite{milner91polyadicpi}.

The notion we use in this paper is derived from weak barbed
bisimulation \cite{milner91polyadicpi}. 

\begin{definition}
An \emph{observation relation}, $\downarrow_{\mathcal N}$, over a set
of names, $\mathcal N$, is the smallest relation satisfying the rules
below.

\infrule[Out-barb]{y \in {\mathcal N}, \; x \nameeq y}
		  {\outputp{x}{v} \downarrow_{\mathcal N} x}
\infrule[Par-barb]{\mbox{$P\downarrow_{\mathcal N} x$ or $Q\downarrow_{\mathcal N} x$}}
		  {\binpar{P}{Q} \downarrow_{\mathcal N} x}

We write $P \Downarrow_{\mathcal N} x$ if there is $Q$ such that 
$P \wred Q$ and $Q \downarrow_{\mathcal N} x$.
\end{definition}

\begin{definition}
%\label{def.bbisim}
An  ${\mathcal N}$-\emph{barbed bisimulation} over a set of names, ${\mathcal N}$, is a symmetric binary relation 
${\mathcal S}_{\mathcal N}$ between agents such that $P\rel{S}_{\mathcal N}Q$ implies:
\begin{enumerate}
\item If $P \red P'$ then $Q \wred Q'$ and $P'\rel{S}_{\mathcal N} Q'$.
\item If $P\downarrow_{\mathcal N} x$, then $Q\Downarrow_{\mathcal N} x$.
\end{enumerate}
$P$ is ${\mathcal N}$-barbed bisimilar to $Q$, written
$P \wbbisim_{\mathcal N} Q$, if $P \rel{S}_{\mathcal N} Q$ for some ${\mathcal N}$-barbed bisimulation ${\mathcal S}_{\mathcal N}$.
\end{definition}

$\mathcal{R} \subseteq \pi \times \pi$

$P \mathcal{R} Q => \forall P'. P \red P' \Rightarrow \exists Q'. Q \red Q', P' \mathcal{R} Q'$

$P \vdash x \Rightarrow Q \vdash x$

\begin{mathpar}
  \inferrule*[lab=Out-barb]{x \nameeq y}{{y}!\langle{Q}\rangle \vdash x}
  \and
  \inferrule*[lab=Par-barb]{\mbox{$P\vdash x$ or $Q\vdash x$}}{\binpar{P}{Q} \vdash x}
\end{mathpar}

\subsubsection{Contexts}

One of the principle advantages of computational calculi like the
$\pi$-calculus is a well-defined notion of context,
contextual-equivalence and a correlation between
contextual-equivalence and notions of bisimulation. The notion of
context allows the decomposition of a process into (sub-)process and
its syntactic environment, its context. Thus, a context may be
thought of as a process with a ``hole'' (written $\Box$) in it. The
application of a context $M$ to a process $P$, written $M[P]$, is
tantamount to filling the hole in $M$ with $P$. In this paper we do
not need the full weight of this theory, but do make use of the notion
of context in the proof the main theorem. 

\begin{mathpar}
  \inferrule* [lab=summation] {} {{M_{M},M_{N}} \bc \Box \;|\; x.M_{A} \;|\; M_{M}+M_{N}}
  \and
  \inferrule* [lab=agent] {} {{M_{A}} \bc (\vec{x})M_{P} \;| \; \clift{P_0,\ldots,M_{P},\ldots,P_N}}
  \and \\
  \inferrule* [lab=process] {} {{M_{P}} \bc M_{N} \;| \;P|M_{P} }
\end{mathpar} 

\begin{mathpar}
  \inferrule* [lab=sychronization] {} {M_{N} \bc \Box \;|\; x?M_{F} \;|\; x!M_{C}}
  \and
  \inferrule* [lab=abstraction] {} {{M_{F}} \bc (x)M_{P} }
  \and
  \inferrule* [lab=concretion] {} {{M_{C}} \bc \langle M_{P} \rangle }
  \and \\
  \inferrule* [lab=process] {} {{M_{P}} \bc M_{N} \;| \;P|M_{P} }
\end{mathpar}

\begin{definition}[contextual application] Given a context $M$, and
  process $P$, we define the \emph{contextual application}, $M[P] :=
  M\{P/\Box\}$. That is, the contextual application of M to P is the
  substitution of $P$ for $\Box$ in $M$.
\end{definition}

$\meaningof{-} : L \to \mathcal{P}(\pi)$

\begin{mathpar}
  \inferrule* [lab=collection] {} {\meaningof{true} = \pi, \and \meaningof{~E} = \pi \setminus \meaningof{E}, \and \meaningof{E_{1} \& E_{2}} = \meaningof{E_{1}} \cap \meaningof{E_{2}}}
\end{mathpar}

\begin{mathpar}
  \inferrule* [lab=structure] {} {\meaningof{0} = \{ P \in \pi | P \equiv 0 \}, \and \\ \meaningof{E_1 | E_2} = \{ P \in \pi | P \equiv P_{1} | P_{2}, P_{1} \in \meaningof{E_{1}}, P_{2} \in \meaningof{E_2}\} }
\end{mathpar}

\begin{mathpar}
 \inferrule* [lab=behavior] {} {\meaningof{\langle a?b \rangle E} = \{ P \in \pi | P \equiv Q | u?(y)P', \\ \and \\\\ \and \\ \;\;\; u \in \meaningof{a}, \forall z.P'\{z/y\} \in \meaningof{E\{z/b\}}\}, \and \\ \meaningof{a!E} = \{ P \in \pi | P \equiv Q | x!\langle P' \rangle, x \in \meaningof{a} P' \in \meaningof{E}\} }
\end{mathpar}

\begin{mathpar}
 \inferrule* [lab=nominal] {} {\meaningof{\quotep{E}} = \{ \quotep{P} \in \quotep{\pi} | P \in \meaningof{E} \}, \and \meaningof{\quotep{P}} = \{ \quotep{Q} \in \quotep{\pi} | P \equiv Q \} \and \\ \meaningof{@\quotep{E}} = \{ P \in \pi | P \equiv @x, x \in \meaningof{E} \}}
\end{mathpar}

\begin{eqnarray*}
  \\
  \meaningof{-} : TS \to ST
\end{eqnarray*}

\begin{eqnarray*}
  \\
  L : TS \to ST
\end{eqnarray*}

\begin{eqnarray*}
  \\
  P \models E \iff P \in \meaningof{E}
\end{eqnarray*}

\begin{eqnarray*}
  P \approx_{L} Q \iff \forall E \in L. P \models E \iff Q \models E
\end{eqnarray*}

\begin{eqnarray*}
  P \approx_{K} Q
\end{eqnarray*}

\begin{eqnarray*}
  P \approx Q
\end{eqnarray*}

$\approx_{K} = \approx = \approx_{L}$

\subsubsection{Contextual duality}

Note that contexts extend the quotation operation to a family of
operations from processes to names. Given a context, $M$, we can
define a \emph{nominal context}, $\quotep{M}$ by $\quotep{M}[P] :=
\quotep{M[P]}$. To foreshadow what is to come we observe that these
operations enjoy a duality with processes very much like the duality
between vectors and maps from vectors to scalars.

Further, because the calculus is essentially higher-order, we have a
correspondence between contexts and processes. More specifically,
given a name $x$ and a context $M$ we can construct $M^{*}_{x}$ such
that 

\begin{mathpar}
  M^{*}_{x} | \lift{x}{P} \red M[P]
\end{mathpar}

namely,

\begin{mathpar}
  M^{*}_{x} := x?(u).M[\dropn{u}]
\end{mathpar}

The dependence of $M^{*}_{x}$ on a name makes it an abstraction, 

\begin{mathpar}
  M^{*} := (x)x?(u).M[\dropn{u}]
\end{mathpar}

\subsection{Additional notation}

It will sometimes be convenient to denote the process a name
quotes. We already have the notation $x = \quotep{P}$, but it will be
convenient to introduce an alternate notation, $\procn{x}$, when we
want to emphasize the connection to the use of the name. Note that, by
virtue of name equivalence, $\quotep{\procn{x}} \nameeq x$; so, the
notation is consistent with previous definitions.

Further, because names have structure it is possible to effect
substitutions on the basis of that structure. This means we need to
upgrade our notation for substitutions, which we accomplish by
adapting comprehension notation. Thus,

\begin{mathpar}
  P\{ y / x : x \in S \}
\end{mathpar}

is interpreted to mean the process derived from P by replacing (in a
capture-avoiding manner) each occurrence of $x$ in $S$ by $y$. For example,

\begin{mathpar}
  P\{ \quotep{\procn{x}|\procn{x}} / x : x \in \freenames{P} \}
\end{mathpar}

will replace each (occurrence) of a free name $x$ in $P$ by
$\quotep{\procn{x}|\procn{x}}$.

Also, we will avail ourselves of the notation $x^{L}$ and $x^{R}$ to
denote injections of a name into disjoint copies of the name
space. There are numerous ways to accomplish this. One example can be
found in \cite{MeredithR05}. This notation overloads to vectors of
names: $\vec{x}^{\pi} := (x_{i}^{\pi} \; : \; 0 \leq i < |\vec{x}| )$ where $\pi \in \{L,R\}$.

We also use $P^{\Box} := P|\Box$.

In \cite{MeredithR05} an interpretation of the new operator is
given. It turns out that there are several possible interpretations
all enjoying the requisite algebraic properties of the operator (see
\cite{milner91polyadicpi}). We will therefore make liberal use of
$(\nu\; \vec{x})P$.

% subsection the_syntax_and_semantics_of_the_notation_system (end)   

\input{qm2pi.qmops} 

\input{qm2pi.sterngerlach} 

\input{qm2pi.metric} 

% section concurrent_process_calculi (end)

%\input{qm2pi.proofsketch}

% section proof sketch (end)

%\input{qm2pi.slviaknots} 

% section spatial logic via knots (end)

\input{qm2pi.conclusion}

% section conclusion (end)

%\input{qm2pi.dtcodes} 

% section wiring algorithm (end)

\input{qm2pi.ack} 

% section acknowledgments (end)

\newpage


\bibliographystyle{plain}   
\bibliography{../../biblios/main.bib}

\input{qm2pi.rhodetails}

\end{document}



% section front matter (end)

\section{Introduction}\label{sec:introduction} % (fold)
In this draft of the material i am going to have to dispense with the
usual writing conventions adopted in papers on these topics. i'm going
to have adopt whatever tone i need at the time i'm writing up the
calculations. Sometimes this may be very conversational; others it may
be the barest mathematical grunts; others still it may be that i have
lifted text from one of my other papers because the exposition of some
point was better said there. i hope that my readers are not unduly put
out by this decision. i'm not doing this to flout convention or be
rebellious. i find these calculations very technically challenging. To
keep everything going technically, something has to give; i have to
let go of some cognitive burden. So, the academic writing style --
with all of its trade-offs in terms of facilitating technical
communication -- is what i'm letting go of. Perhaps subsequent drafts
can be tightened and polished, but for now, i'm going to speak as if
we were sitting together in a coffee shop with a laptop, wifi and a
pad of paper and a pencil.

So, here's what i have to say. We -- you and i, comfortably ensconced
in our coffee shop and well-equipped with our tools -- can realize and
carry out the calculations of quantum mechanics over a very different
formal theory of dynamics, a formal theory of dynamics that
corresponds to a theory of concurrent computation with
\emph{reflection}. It has the advantage that the underlying theory is
already `quantized', but supports analogues all of the continuuous
operations. Strikingly, this underlying theory has recently been
connected with a notion of metric that we can show, by calculating
together, coincides with the metric induced by the inner product.

There are a lot of reasons why you might be interested in seeing
calculations of this form. Here's why i'm interested. For the past
several centuries there has been no competitor to the ``Newtonian''
account of dynamics. As a result the predominant share of accounts of
dynamical systems and situations have had to be formulated in terms of
the Newtonian machinery. i view this as an intellectually dangerous
position to occupy. Everything, despite it's intrinsic shape, turns
into a nail to be hit with this hammer. Recently, however, the theory
of computation has matured to the point where we have candidates for
theories of dynamics that offer very different perspective on
reasoning about dynamical systems and situations. Testing these
candidates against very successful accounts of dynamical situations,
like quantum mechanics, is going to give us some sense of how mature
they are and some measure of the quality of these accounts of
dynamics.

\subsection{Summary of contributions and outline of paper}

So, we're going to develop an interpretation of the operations of
quantum mechanics normally interpreted by Hilbert spaces and
operators. We're going to do this over a theory of computation. Note
that this is very different than the usual quantum computation program
which develops notions of computation over quantum mechanics. Rather,
we are developing a story that aligns with Wheeler's slogan: It from
Bit. To do this we will first provide an account of the theory of
computation at play here. Then we will dive into a calculation-driven
interpretation of the operations of quantum mechanics.

The reason we take this approach is that -- until very recently --
there hasn't been an axiomatic account of quantum mechanics. As a
result there has been no sharp delineation of the mathematical theory
supporting interpretation of the physical theory and the physical
theory, itself. So, ambient features of the maths are free to be
exploited (or supressed) without a real accounting of their physical
relevance. There is no sharp statement ``here's the physical theory''
qua \emph{theory} and ``here's the mathematical interpretation''
enabling a judgment of how faithful the interpretation is -- apart
from experimental observation. When there is an axiomatic account we
can judge how well a given mathematical formalism supports an
interpretation of the axioms, independent of
experimentation. Likewise, we can judge how well we have captured our
physical evidence and experience with our axiomatics, independent of
any specific mathematical implementation, with accidental detail that
may or may not have physical significance. 

In lieu of a fully fleshed out and vetted axiomatic account of quantum
mechanics, interpreting the operational notions in service of modeling
physical systems will have to suffice. In other words, we are not in
the business of providing a model of Hilbert spaces and operators. We
are in the business of providing a model of quantum mechanics because
we are motivated by testing our notions of dynamics against physical
theory; and, the predictive calculations of the physical theory must
serve as the best formulation -- shy of a fully fleshed out axiomatic
account -- of the physical theory itself (as they have for scientific
theories since time immemorial). Put another way, despite a
whole-hearted commitment to an It-from-Bit ontology, we are firmly
aligned with the shut-up-and-calculate camp as the best way to obtain
results either from the physical perspective or as a quality assurance
measure of our fledgling theory of dynamics.

In detail, we present a reflective process calculus. Then we develop
intuitive correspondences between the notions available in this
calculus and the usual physical notions supporting quantum mechanical
calculations. Thus, 

\begin{table}[htp]
  \center{
    \fbox{
      \begin{tabular}{c|c}
        quantum mechanics & process calculus \\
        \hline
        scalar & name \\
        state vector & process \\
        dual & contextual duals \\
        matrix & formal sums of process-context-dual pairs \\
        orthogonality & process annihilation \\
        inner product & execution-formula + quoting
      \end{tabular}
    }
  }
  \caption{QM - process calculi correspondences}
\end{table}

Then we tighten up these intuitions to operational definitions. We
employ the Dirac notation as the best proxy we can find for an
abstract syntax of the quantum mechanical notions. The definitions we
develop put us in contact with equational constraints coming from the
theory that we demonstrate the definitions and calculations satisfy.

This puts us in a position to shut up and calculate for the
Stern-Gerlach experimental set up, showing how these predictive
calculations become calculations on processes in our theory of a
reflective process calculus.

Penultimately, we demonstrate that the notion of metric coming from
the inner product coincides with the notion of metric available from
the theory of bisimulation. This demonstration gives us the right to
think of space as arising from behavior. Finally, we consider where we
might go from the new vantage point we have obtained.

% section introduction (end) 
 
% section introduction (end)

% \documentclass[12pt]{llncs}
%\documentclass{jktr}

\usepackage[pdftex]{hyperref}                   
\usepackage {listings}
\usepackage {mathpartir}
\usepackage{bcprules}
%\usepackage{listings}
                       
\usepackage{graphicx} 
%\usepackage[margins=2.5cm,nohead,nofoot]{geometry}
%\usepackage{geometry}
\usepackage{amsfonts}
\usepackage{amstext}
\usepackage{latexsym}
\usepackage{amssymb}
\usepackage{color}


%\include{myPreamble}
\include{qm2pi.local} 

%\ifpdf
%\usepackage[pdftex]{graphicx}
%\else
%\usepackage{graphicx}
%\fi

 % \ifpdf
%  \usepackage{pdfsync}
%  \if


%\title{Brief Article}
%\author{David F. Snyder}
%\author{L.G. Meredith}

%\address{Dept. of Math., Texas State University--San Marcos, San Marcos, TX 78666}
       
\pagestyle{empty}


\begin{document}

\lstset{language=[Objective]Caml,frame=shadowbox}

\input{qm2pi.front}

% section front matter (end)

\input{qm2pi.intro} 
 
% section introduction (end)

% \input{qm2pi.knotations} 

% section notation (end)

\input{qm2pi.process.calculi} 

% section concurrent_process_calculi_and_spatial_logics_ (end)
    
%\input{qm2pi.knots2pi} 

%\input{qm2pi.trefoil} 

%\input{qm2pi.mainthm} 

% subsection basic_interpretation (end)

%\input{qm2pi.rho.presentation} 
\subsection{The syntax and semantics of the notation system}\label{sub:the_syntax_and_semantics_of_the_notation_system} % (fold)

We now summarize a technical presentation of the calculus that
embodies our theory of dynamics. The typical presentation of such a
calculus follows the style of giving generators and relations on
them. The grammar, below, describing term constructors, freely
generates the set of processes, $\Proc$. This set is then quotiented
by a relation known as structural congruence and it is over this set
that the notion of dynamics is expressed. This presentation is
essentially that of \cite{MeredithR05} with the addition of
polyadicity and summation. For readability we have relegated some of
the technical subtleties to an appendix.

\subsubsection{Process grammar}\label{subsub:process_grammar}

\begin{mathpar}
  \inferrule* [lab=synchronization] {} {{M} \bc \pzero \;|\; x?F \;|\; x!C }
  \and
  \inferrule* [lab=abstraction] {} {{F} \bc (x)P}
  \and
  \inferrule* [lab=concretion] {} {{C} \bc \langle Q \rangle}
  \and
  \inferrule* [lab=process] {} {{P,Q} \bc M \;| \;P|Q \;|\; @{x}}
  \and
  \inferrule* [lab=name] {} {{x} \bc \quotep{P}}
\end{mathpar} 

Note that $\vec{x}$ (resp. $\vec{P}$) denotes a vector of names
(resp. processes) of length $|\vec{x}|$ (resp. $|\vec{P}|$). We adopt
the following useful abbreviations.

\begin{mathpar}
   x?(\vec{y}).P := x.(\vec{y})P \and  x\clift{\vec{P}} := x.\clift{\vec{P}}
   \and x!(y) := \lift{x}{\dropn{y}}
   \and \Pi_{i=0}^{n-1}P_i := P_0 | \ldots | P_{n-1}
\end{mathpar}

\subsubsection{Structural congruence}

\paragraph{Free and bound names and alpha-equivalence.} At the
core of structural equivalence is alpha-equivalence which identifies
process that are the same up to a change of variable. Formally, we
recognize the distinction between free and bound names. The free names
of a process, $\freenames{P}$, may be calculated recursively as
follows:

\begin{mathpar}
\freenames{\pzero} := \emptyset
  \and \\
  \freenames{x?(y).P} := \{ x \} \cup (\freenames{P} \setminus \{ y \})
  \and 
  \freenames{x!\langle P \rangle} := \{ x \} \cup \{ P \} 
  \and \\
  \freenames{P|Q} := \freenames{P} \cup \freenames{Q}
  \and \\
  \freenames{@{x}} := \{ x \}
\end{mathpar}

$\pi$
$\quotep{\pi}$

$\freenames{-} : \pi \to \mathcal{P}(\quotep{\pi})$

\begin{eqnarray*}
  \freenames{\pzero} & := & \emptyset \\
  \freenames{x?(y).P} & := & \{ x \} \cup (\freenames{P} \setminus \{ y \}) \\
  \freenames{x!\langle P \rangle} & := & \{ x \} \cup \{ P \} \\
  \freenames{P|Q} & := & \freenames{P} \cup \freenames{Q} \\
  \freenames{\dropn{x}} & := & \{ x \}
\end{eqnarray*}

The bound names of a process, $\boundnames{P}$, are those names occurring in $P$
that are not free. For example, in $x?(y).0$, the name $x$ is free, while $y$ is bound.

\begin{mathpar}
  \inferrule* [lab=monoidal-laws] {} { P|Q \equiv Q|P \and P|0 \equiv P \and P|(Q|R) \equiv (P|Q)|R }
\end{mathpar}

\begin{mathpar}
  \inferrule* [lab=alpha-equivalence] {} { (x)P \equiv (y)P\{y/x\} \and y \not\in \freenames{P} }
\end{mathpar}

\begin{definition}
Then two processes, $P,Q$, are alpha-equivalent if $P = Q\{\vec{y}/\vec{x}\}$ for
some $\vec{x} \in \boundnames{Q},\vec{y} \in \boundnames{P}$, where $Q\{\vec{y}/\vec{x}\}$
denotes the capture-avoiding substitution of $\vec{y}$ for $\vec{x}$ in $Q$.
\end{definition}

\begin{definition}
  The {\em structural congruence} \cite{SangiorgiWalker} , $\equiv$,
  between processes is the least congruence containing
  alpha-equivalence, satisfying the abelian monoid laws
  (associativity, commutativity and $\pzero$ as identity) for parallel
  composition $|$ and for summation $+$.
\end{definition}

\subsection{Name equivalence}

We take name equivalence, written $\nameeq$, to be the smallest
equivalence relation generated by the following rules.

\begin{mathpar}
\inferrule*[lab=Quote-drop]
{ }
{ \quotep{@{x}} \nameeq x }

\inferrule*[lab=Struct-equiv]
{ P \scong Q }
{ \quotep{P} \nameeq \quotep{Q} }
\end{mathpar}

The astute reader will have noticed that the mutual recursion of names
and processes imposes a mutual recursion on alpha-equivalence and
structural equivalence via name-equivalence. Fortunately, all of this
works out pleasantly and we may calculate in the natural way, free of
concern. The reader interested in the details is referred to the
appendix \ref{appendix:rho_details}.

\subsection{Substitution}

We use $\Proc$ for the set of processes, $\QProc$ for the set of
names, and $\id{\{}\vec{y} / \vec{x} \id{\}}$ to denote partial maps,
$s : \QProc \rightarrow \QProc$. A map, $s$ lifts, uniquely, to a map
on process terms, $\widehat{s} : \Proc \rightarrow \Proc$ by the
following equations.

\begin{mathpar}
  (0) \psubstp{Q}{P} := 0 \\
  (R \juxtap S) \psubstp{Q}{P}
  :=    
  (R)\psubstp{Q}{P} \juxtap (S) \psubstp{Q}{P} \\
  (x?(y).R) \psubstp{Q}{P}    
  :=    
  (x)\substp{Q}{P} (z)\concat( (R \psubstn{z}{y}) \psubstp{Q}{P} ) \\
  (\lift{x}{R}) \psubstp{Q}{P}  
  :=
  \lift{(x)\substp{Q}{P}}{ R \psubstp{Q}{P} } \\
%   (\dropn{x})  \psubstp{Q}{P}       
%   := 
%   \left\{ 
%     \begin{array}{ccc} 
%       \dropn{\quotep{Q}} & & x \nameeq \quotep{P} \\
%       \dropn{x} & & otherwise \\
%     \end{array}
%   \right. 
  (\dropn{x})  \psubstp{Q}{P}       
  := 
  \left\{ 
    \begin{array}{ccc} 
      Q & & x \nameeq \quotep{P} \\
      \dropn{x} & & otherwise \\
    \end{array}
  \right.
\end{mathpar}
 

where

\begin{eqnarray}
  (x)\id{\{} \lpquote Q \rpquote / \lpquote P \rpquote \id{\}}            = 
  \left\{ 
    \begin{array}{ccc}
      \lpquote Q \rpquote & & x \nameeq \lpquote P \rpquote \\
      x & & otherwise \\
    \end{array}
  \right. \nonumber
\end{eqnarray}

and $z$ is chosen distinct from $\quotep{P}$, $\quotep{Q}$, the free
names in $Q$, and all the names in $R$. Our $\alpha$-equivalence will
be built in the standard way from this substitution.

\begin{remark}\label{rem:no_self_referential_names}
  One consequence of these definitions is that $\forall P. \quotep{P}
  \not\in \freenames{P}$.
\end{remark}

\subsection{ Dynamic quote: an example }

Anticipating something of what's to come, consider applying the
substitution, $\widehat{\id{\{}u / z \id{\}}}$, to the following pair
of processes, $\lift{w}{y!(z)}$ and $w[ \lpquote y!(z) \rpquote ]$.

\begin{eqnarray}
	\lift{w}{y!(z)}\widehat{\id{\{}u / z \id{\}}}
		& = &
		\lift{w}{y!(u)} \nonumber\\
	w[ \lpquote y!(z) \rpquote ] \widehat{ \id{\{}u / z \id{\}} }
		& = &
		w[ \lpquote y!(z) \rpquote ] \nonumber
\end{eqnarray}

Because the body of the process between quotes is impervious to
substitution, we get radically different answers. In fact, by
examining the first process in an input context,
e.g. $x?(z).\lift{w}{y!(z)}$, we see that the process under the lift
operator may be shaped by prefixed inputs binding a name inside it. In
this sense, the lift operator will be seen as a way to dynamically
construct processes before reifying them as names.

Finally equipped with these standard features we can present the
dynamics of the calculus.

\subsubsection{Operational semantics} 

Finally, we introduce the computational dynamics. What marks these
algebras as distinct from other more traditionally studied algebraic
structures, e.g. vector spaces or polynomial rings, is the manner in
which dynamics is captured. In traditional structures, dynamics is typically
expressed through morphisms between such structures, as in linear maps
between vector spaces or morphisms between rings. In algebras
associated with the semantics of computation, the dynamics is
expressed as part of the algebraic structure itself, through a
reduction reduction relation typically denoted by $\red$. Below, we
give a recursive presentation of this relation for the calculus used
in the encoding.

$\red \subseteq \pi \times \pi$
$\red : \pi \to \mathcal{P}(\pi)$

\begin{mathpar}
  \inferrule* [lab=Comm] { \textsf{match}( x_{src}, x_{trgt} ) } { x_{trgt}?(y)P \; | \; x_{src}!\langle {Q} \rangle \red P\{\quotep{Q}/y}\} }
  \and \\
  \inferrule* [lab=Par] {{P} \red {P}'} {{{P} | {Q}} \red {{P}' | {Q}}}
  \and
  \inferrule* [lab=Equiv]{{{P} \scong {P}'} \andalso {{P}' \red {Q}'} \andalso {{Q}' \scong {Q}}}{{P} \red {Q}}
\end{mathpar}

\begin{eqnarray*}
  match_{\equiv} (\quotep{P},\quotep{Q}) & := & P \equiv Q \\
  match_{\dagger}(\quotep{P},\quotep{Q}) & := & \forall R. P|Q \red^{*} R => R \red^{*} 0 \\
  match_{K}(\quotep{P},\quotep{Q}) & := & K \mbox{ for some context } K
\end{eqnarray*}

$u?(x)P | u!\langle Q \rangle \red P\{\quotep{Q}/x\}$

%We write $\wred$ for $\red^*$, and $P\red$ if $\exists Q $ such that $ P \red Q$.
We write $P\red$ if $\exists Q $ such that $ P \red Q$ and $P\not\red$, otherwise.

\section{Replication}

As mentioned before, it is known that replication (and hence
recursion) can be implemented in a higher-order process algebra
\cite{SangiorgiWalker}. As our first example of calculation with the
machinery thus far presented we give the construction explicitly in
the {\rhoc}.

\begin{eqnarray}
	D_{x} & := & \prefix{x}{y}{(\binpar{\outputp{x}{y}}{@{y}})} \nonumber\\
	\bangp_{x}{P} & := & \binpar{{x}!\langle{\binpar{D_{x}}{P}}\rangle}{D_{x}} \nonumber
\end{eqnarray}

\begin{eqnarray}
	\bangp_{x}{P} & & \nonumber\\
	=
	& {x}!\langle{(\prefix{x}{y}{(\outputp{x}{y} | @{y})) | P}}\rangle 
	      | \prefix{x}{y}{(\outputp{x}{y} | @{y})} & \nonumber\\
	\red
	& (\outputp{x}{y} | @{y})\substn{\quotep{(\prefix{x}{y}{(@{y} | \outputp{x}{y})) | P}}}{y} & \nonumber\\
	=
	& \outputp{x}{\quotep{(\prefix{x}{y}{(\outputp{x}{y} | @{y})) | P}}}
	  | {(\prefix{x}{y}{(\outputp{x}{y} | @{y})) | P}} & \nonumber\\
	\red
	& \ldots & \nonumber\\
	\red^*
	& P | P | \ldots & \nonumber
\end{eqnarray}

Of course, this encoding, as an implementation, runs away, unfolding
$\bangp{P}$ eagerly. A lazier and more implementable replication
operator, restricted to input-guarded processes, may be obtained as follows.

\begin{eqnarray}
\bangp{\prefix{u}{v}{P}} 
	:= 
	\binpar{\lift{x}{\prefix{u}{v}{(\binpar{D(x)}{P})}}}{D(x)} \nonumber
\end{eqnarray}

\begin{remark}
  Note that the lazier definition still does not deal with summation
  or mixed summation (i.e. sums over input and output). The reader is
  invited to construct definitions of replication that deal with these
  features. 

  Further, the definitions are parameterized in a name, $x$. Can you,
  gentle reader, make a definition that eliminates this parameter and
  guarantees no accidental interaction between the replication
  machinery and the process being replicated -- i.e. no accidental
  sharing of names used by the process to get its work done and the
  name(s) used by the replication to effect copying. This latter
  revision of the definition of replication is crucial to obtaining
  the expected identity $!!P \sim !P$.
\end{remark}

\begin{remark}\label{rem:paradoxical_combinator}
  The reader familiar with the lambda calculus will have noticed the
  similarity between $D$ and the paradoxical combinator.

  [Ed. note: the existence of this seems to suggest we have to be more
  restrictive on the set of processes and names we admit if we are to
  support no-cloning.]
\end{remark}

\subsubsection{Bisimulation}

The computational dynamics gives rise to another kind of equivalence,
the equivalence of computational behavior. As previously mentioned
this is typically captured \emph{via} some form of bisimulation.

% The notion we use in this paper is weak barbed bisimulation
% \cite{milner91polyadicpi}.

The notion we use in this paper is derived from weak barbed
bisimulation \cite{milner91polyadicpi}. 

\begin{definition}
An \emph{observation relation}, $\downarrow_{\mathcal N}$, over a set
of names, $\mathcal N$, is the smallest relation satisfying the rules
below.

\infrule[Out-barb]{y \in {\mathcal N}, \; x \nameeq y}
		  {\outputp{x}{v} \downarrow_{\mathcal N} x}
\infrule[Par-barb]{\mbox{$P\downarrow_{\mathcal N} x$ or $Q\downarrow_{\mathcal N} x$}}
		  {\binpar{P}{Q} \downarrow_{\mathcal N} x}

We write $P \Downarrow_{\mathcal N} x$ if there is $Q$ such that 
$P \wred Q$ and $Q \downarrow_{\mathcal N} x$.
\end{definition}

\begin{definition}
%\label{def.bbisim}
An  ${\mathcal N}$-\emph{barbed bisimulation} over a set of names, ${\mathcal N}$, is a symmetric binary relation 
${\mathcal S}_{\mathcal N}$ between agents such that $P\rel{S}_{\mathcal N}Q$ implies:
\begin{enumerate}
\item If $P \red P'$ then $Q \wred Q'$ and $P'\rel{S}_{\mathcal N} Q'$.
\item If $P\downarrow_{\mathcal N} x$, then $Q\Downarrow_{\mathcal N} x$.
\end{enumerate}
$P$ is ${\mathcal N}$-barbed bisimilar to $Q$, written
$P \wbbisim_{\mathcal N} Q$, if $P \rel{S}_{\mathcal N} Q$ for some ${\mathcal N}$-barbed bisimulation ${\mathcal S}_{\mathcal N}$.
\end{definition}

$\mathcal{R} \subseteq \pi \times \pi$

$P \mathcal{R} Q => \forall P'. P \red P' \Rightarrow \exists Q'. Q \red Q', P' \mathcal{R} Q'$

$P \vdash x \Rightarrow Q \vdash x$

\begin{mathpar}
  \inferrule*[lab=Out-barb]{x \nameeq y}{{y}!\langle{Q}\rangle \vdash x}
  \and
  \inferrule*[lab=Par-barb]{\mbox{$P\vdash x$ or $Q\vdash x$}}{\binpar{P}{Q} \vdash x}
\end{mathpar}

\subsubsection{Contexts}

One of the principle advantages of computational calculi like the
$\pi$-calculus is a well-defined notion of context,
contextual-equivalence and a correlation between
contextual-equivalence and notions of bisimulation. The notion of
context allows the decomposition of a process into (sub-)process and
its syntactic environment, its context. Thus, a context may be
thought of as a process with a ``hole'' (written $\Box$) in it. The
application of a context $M$ to a process $P$, written $M[P]$, is
tantamount to filling the hole in $M$ with $P$. In this paper we do
not need the full weight of this theory, but do make use of the notion
of context in the proof the main theorem. 

\begin{mathpar}
  \inferrule* [lab=summation] {} {{M_{M},M_{N}} \bc \Box \;|\; x.M_{A} \;|\; M_{M}+M_{N}}
  \and
  \inferrule* [lab=agent] {} {{M_{A}} \bc (\vec{x})M_{P} \;| \; \clift{P_0,\ldots,M_{P},\ldots,P_N}}
  \and \\
  \inferrule* [lab=process] {} {{M_{P}} \bc M_{N} \;| \;P|M_{P} }
\end{mathpar} 

\begin{mathpar}
  \inferrule* [lab=sychronization] {} {M_{N} \bc \Box \;|\; x?M_{F} \;|\; x!M_{C}}
  \and
  \inferrule* [lab=abstraction] {} {{M_{F}} \bc (x)M_{P} }
  \and
  \inferrule* [lab=concretion] {} {{M_{C}} \bc \langle M_{P} \rangle }
  \and \\
  \inferrule* [lab=process] {} {{M_{P}} \bc M_{N} \;| \;P|M_{P} }
\end{mathpar}

\begin{definition}[contextual application] Given a context $M$, and
  process $P$, we define the \emph{contextual application}, $M[P] :=
  M\{P/\Box\}$. That is, the contextual application of M to P is the
  substitution of $P$ for $\Box$ in $M$.
\end{definition}

$\meaningof{-} : L \to \mathcal{P}(\pi)$

\begin{mathpar}
  \inferrule* [lab=collection] {} {\meaningof{true} = \pi, \and \meaningof{~E} = \pi \setminus \meaningof{E}, \and \meaningof{E_{1} \& E_{2}} = \meaningof{E_{1}} \cap \meaningof{E_{2}}}
\end{mathpar}

\begin{mathpar}
  \inferrule* [lab=structure] {} {\meaningof{0} = \{ P \in \pi | P \equiv 0 \}, \and \\ \meaningof{E_1 | E_2} = \{ P \in \pi | P \equiv P_{1} | P_{2}, P_{1} \in \meaningof{E_{1}}, P_{2} \in \meaningof{E_2}\} }
\end{mathpar}

\begin{mathpar}
 \inferrule* [lab=behavior] {} {\meaningof{\langle a?b \rangle E} = \{ P \in \pi | P \equiv Q | u?(y)P', \\ \and \\\\ \and \\ \;\;\; u \in \meaningof{a}, \forall z.P'\{z/y\} \in \meaningof{E\{z/b\}}\}, \and \\ \meaningof{a!E} = \{ P \in \pi | P \equiv Q | x!\langle P' \rangle, x \in \meaningof{a} P' \in \meaningof{E}\} }
\end{mathpar}

\begin{mathpar}
 \inferrule* [lab=nominal] {} {\meaningof{\quotep{E}} = \{ \quotep{P} \in \quotep{\pi} | P \in \meaningof{E} \}, \and \meaningof{\quotep{P}} = \{ \quotep{Q} \in \quotep{\pi} | P \equiv Q \} \and \\ \meaningof{@\quotep{E}} = \{ P \in \pi | P \equiv @x, x \in \meaningof{E} \}}
\end{mathpar}

\begin{eqnarray*}
  \\
  \meaningof{-} : TS \to ST
\end{eqnarray*}

\begin{eqnarray*}
  \\
  L : TS \to ST
\end{eqnarray*}

\begin{eqnarray*}
  \\
  P \models E \iff P \in \meaningof{E}
\end{eqnarray*}

\begin{eqnarray*}
  P \approx_{L} Q \iff \forall E \in L. P \models E \iff Q \models E
\end{eqnarray*}

\begin{eqnarray*}
  P \approx_{K} Q
\end{eqnarray*}

\begin{eqnarray*}
  P \approx Q
\end{eqnarray*}

$\approx_{K} = \approx = \approx_{L}$

\subsubsection{Contextual duality}

Note that contexts extend the quotation operation to a family of
operations from processes to names. Given a context, $M$, we can
define a \emph{nominal context}, $\quotep{M}$ by $\quotep{M}[P] :=
\quotep{M[P]}$. To foreshadow what is to come we observe that these
operations enjoy a duality with processes very much like the duality
between vectors and maps from vectors to scalars.

Further, because the calculus is essentially higher-order, we have a
correspondence between contexts and processes. More specifically,
given a name $x$ and a context $M$ we can construct $M^{*}_{x}$ such
that 

\begin{mathpar}
  M^{*}_{x} | \lift{x}{P} \red M[P]
\end{mathpar}

namely,

\begin{mathpar}
  M^{*}_{x} := x?(u).M[\dropn{u}]
\end{mathpar}

The dependence of $M^{*}_{x}$ on a name makes it an abstraction, 

\begin{mathpar}
  M^{*} := (x)x?(u).M[\dropn{u}]
\end{mathpar}

\subsection{Additional notation}

It will sometimes be convenient to denote the process a name
quotes. We already have the notation $x = \quotep{P}$, but it will be
convenient to introduce an alternate notation, $\procn{x}$, when we
want to emphasize the connection to the use of the name. Note that, by
virtue of name equivalence, $\quotep{\procn{x}} \nameeq x$; so, the
notation is consistent with previous definitions.

Further, because names have structure it is possible to effect
substitutions on the basis of that structure. This means we need to
upgrade our notation for substitutions, which we accomplish by
adapting comprehension notation. Thus,

\begin{mathpar}
  P\{ y / x : x \in S \}
\end{mathpar}

is interpreted to mean the process derived from P by replacing (in a
capture-avoiding manner) each occurrence of $x$ in $S$ by $y$. For example,

\begin{mathpar}
  P\{ \quotep{\procn{x}|\procn{x}} / x : x \in \freenames{P} \}
\end{mathpar}

will replace each (occurrence) of a free name $x$ in $P$ by
$\quotep{\procn{x}|\procn{x}}$.

Also, we will avail ourselves of the notation $x^{L}$ and $x^{R}$ to
denote injections of a name into disjoint copies of the name
space. There are numerous ways to accomplish this. One example can be
found in \cite{MeredithR05}. This notation overloads to vectors of
names: $\vec{x}^{\pi} := (x_{i}^{\pi} \; : \; 0 \leq i < |\vec{x}| )$ where $\pi \in \{L,R\}$.

We also use $P^{\Box} := P|\Box$.

In \cite{MeredithR05} an interpretation of the new operator is
given. It turns out that there are several possible interpretations
all enjoying the requisite algebraic properties of the operator (see
\cite{milner91polyadicpi}). We will therefore make liberal use of
$(\nu\; \vec{x})P$.

% subsection the_syntax_and_semantics_of_the_notation_system (end)   

\input{qm2pi.qmops} 

\input{qm2pi.sterngerlach} 

\input{qm2pi.metric} 

% section concurrent_process_calculi (end)

%\input{qm2pi.proofsketch}

% section proof sketch (end)

%\input{qm2pi.slviaknots} 

% section spatial logic via knots (end)

\input{qm2pi.conclusion}

% section conclusion (end)

%\input{qm2pi.dtcodes} 

% section wiring algorithm (end)

\input{qm2pi.ack} 

% section acknowledgments (end)

\newpage


\bibliographystyle{plain}   
\bibliography{../../biblios/main.bib}

\input{qm2pi.rhodetails}

\end{document}

 

% section notation (end)

\input{qm2pi.process.calculi} 

% section concurrent_process_calculi_and_spatial_logics_ (end)
    
%\documentclass[12pt]{llncs}
%\documentclass{jktr}

\usepackage[pdftex]{hyperref}                   
\usepackage {listings}
\usepackage {mathpartir}
\usepackage{bcprules}
%\usepackage{listings}
                       
\usepackage{graphicx} 
%\usepackage[margins=2.5cm,nohead,nofoot]{geometry}
%\usepackage{geometry}
\usepackage{amsfonts}
\usepackage{amstext}
\usepackage{latexsym}
\usepackage{amssymb}
\usepackage{color}


%\include{myPreamble}
\include{qm2pi.local} 

%\ifpdf
%\usepackage[pdftex]{graphicx}
%\else
%\usepackage{graphicx}
%\fi

 % \ifpdf
%  \usepackage{pdfsync}
%  \if


%\title{Brief Article}
%\author{David F. Snyder}
%\author{L.G. Meredith}

%\address{Dept. of Math., Texas State University--San Marcos, San Marcos, TX 78666}
       
\pagestyle{empty}


\begin{document}

\lstset{language=[Objective]Caml,frame=shadowbox}

\input{qm2pi.front}

% section front matter (end)

\input{qm2pi.intro} 
 
% section introduction (end)

% \input{qm2pi.knotations} 

% section notation (end)

\input{qm2pi.process.calculi} 

% section concurrent_process_calculi_and_spatial_logics_ (end)
    
%\input{qm2pi.knots2pi} 

%\input{qm2pi.trefoil} 

%\input{qm2pi.mainthm} 

% subsection basic_interpretation (end)

%\input{qm2pi.rho.presentation} 
\subsection{The syntax and semantics of the notation system}\label{sub:the_syntax_and_semantics_of_the_notation_system} % (fold)

We now summarize a technical presentation of the calculus that
embodies our theory of dynamics. The typical presentation of such a
calculus follows the style of giving generators and relations on
them. The grammar, below, describing term constructors, freely
generates the set of processes, $\Proc$. This set is then quotiented
by a relation known as structural congruence and it is over this set
that the notion of dynamics is expressed. This presentation is
essentially that of \cite{MeredithR05} with the addition of
polyadicity and summation. For readability we have relegated some of
the technical subtleties to an appendix.

\subsubsection{Process grammar}\label{subsub:process_grammar}

\begin{mathpar}
  \inferrule* [lab=synchronization] {} {{M} \bc \pzero \;|\; x?F \;|\; x!C }
  \and
  \inferrule* [lab=abstraction] {} {{F} \bc (x)P}
  \and
  \inferrule* [lab=concretion] {} {{C} \bc \langle Q \rangle}
  \and
  \inferrule* [lab=process] {} {{P,Q} \bc M \;| \;P|Q \;|\; @{x}}
  \and
  \inferrule* [lab=name] {} {{x} \bc \quotep{P}}
\end{mathpar} 

Note that $\vec{x}$ (resp. $\vec{P}$) denotes a vector of names
(resp. processes) of length $|\vec{x}|$ (resp. $|\vec{P}|$). We adopt
the following useful abbreviations.

\begin{mathpar}
   x?(\vec{y}).P := x.(\vec{y})P \and  x\clift{\vec{P}} := x.\clift{\vec{P}}
   \and x!(y) := \lift{x}{\dropn{y}}
   \and \Pi_{i=0}^{n-1}P_i := P_0 | \ldots | P_{n-1}
\end{mathpar}

\subsubsection{Structural congruence}

\paragraph{Free and bound names and alpha-equivalence.} At the
core of structural equivalence is alpha-equivalence which identifies
process that are the same up to a change of variable. Formally, we
recognize the distinction between free and bound names. The free names
of a process, $\freenames{P}$, may be calculated recursively as
follows:

\begin{mathpar}
\freenames{\pzero} := \emptyset
  \and \\
  \freenames{x?(y).P} := \{ x \} \cup (\freenames{P} \setminus \{ y \})
  \and 
  \freenames{x!\langle P \rangle} := \{ x \} \cup \{ P \} 
  \and \\
  \freenames{P|Q} := \freenames{P} \cup \freenames{Q}
  \and \\
  \freenames{@{x}} := \{ x \}
\end{mathpar}

$\pi$
$\quotep{\pi}$

$\freenames{-} : \pi \to \mathcal{P}(\quotep{\pi})$

\begin{eqnarray*}
  \freenames{\pzero} & := & \emptyset \\
  \freenames{x?(y).P} & := & \{ x \} \cup (\freenames{P} \setminus \{ y \}) \\
  \freenames{x!\langle P \rangle} & := & \{ x \} \cup \{ P \} \\
  \freenames{P|Q} & := & \freenames{P} \cup \freenames{Q} \\
  \freenames{\dropn{x}} & := & \{ x \}
\end{eqnarray*}

The bound names of a process, $\boundnames{P}$, are those names occurring in $P$
that are not free. For example, in $x?(y).0$, the name $x$ is free, while $y$ is bound.

\begin{mathpar}
  \inferrule* [lab=monoidal-laws] {} { P|Q \equiv Q|P \and P|0 \equiv P \and P|(Q|R) \equiv (P|Q)|R }
\end{mathpar}

\begin{mathpar}
  \inferrule* [lab=alpha-equivalence] {} { (x)P \equiv (y)P\{y/x\} \and y \not\in \freenames{P} }
\end{mathpar}

\begin{definition}
Then two processes, $P,Q$, are alpha-equivalent if $P = Q\{\vec{y}/\vec{x}\}$ for
some $\vec{x} \in \boundnames{Q},\vec{y} \in \boundnames{P}$, where $Q\{\vec{y}/\vec{x}\}$
denotes the capture-avoiding substitution of $\vec{y}$ for $\vec{x}$ in $Q$.
\end{definition}

\begin{definition}
  The {\em structural congruence} \cite{SangiorgiWalker} , $\equiv$,
  between processes is the least congruence containing
  alpha-equivalence, satisfying the abelian monoid laws
  (associativity, commutativity and $\pzero$ as identity) for parallel
  composition $|$ and for summation $+$.
\end{definition}

\subsection{Name equivalence}

We take name equivalence, written $\nameeq$, to be the smallest
equivalence relation generated by the following rules.

\begin{mathpar}
\inferrule*[lab=Quote-drop]
{ }
{ \quotep{@{x}} \nameeq x }

\inferrule*[lab=Struct-equiv]
{ P \scong Q }
{ \quotep{P} \nameeq \quotep{Q} }
\end{mathpar}

The astute reader will have noticed that the mutual recursion of names
and processes imposes a mutual recursion on alpha-equivalence and
structural equivalence via name-equivalence. Fortunately, all of this
works out pleasantly and we may calculate in the natural way, free of
concern. The reader interested in the details is referred to the
appendix \ref{appendix:rho_details}.

\subsection{Substitution}

We use $\Proc$ for the set of processes, $\QProc$ for the set of
names, and $\id{\{}\vec{y} / \vec{x} \id{\}}$ to denote partial maps,
$s : \QProc \rightarrow \QProc$. A map, $s$ lifts, uniquely, to a map
on process terms, $\widehat{s} : \Proc \rightarrow \Proc$ by the
following equations.

\begin{mathpar}
  (0) \psubstp{Q}{P} := 0 \\
  (R \juxtap S) \psubstp{Q}{P}
  :=    
  (R)\psubstp{Q}{P} \juxtap (S) \psubstp{Q}{P} \\
  (x?(y).R) \psubstp{Q}{P}    
  :=    
  (x)\substp{Q}{P} (z)\concat( (R \psubstn{z}{y}) \psubstp{Q}{P} ) \\
  (\lift{x}{R}) \psubstp{Q}{P}  
  :=
  \lift{(x)\substp{Q}{P}}{ R \psubstp{Q}{P} } \\
%   (\dropn{x})  \psubstp{Q}{P}       
%   := 
%   \left\{ 
%     \begin{array}{ccc} 
%       \dropn{\quotep{Q}} & & x \nameeq \quotep{P} \\
%       \dropn{x} & & otherwise \\
%     \end{array}
%   \right. 
  (\dropn{x})  \psubstp{Q}{P}       
  := 
  \left\{ 
    \begin{array}{ccc} 
      Q & & x \nameeq \quotep{P} \\
      \dropn{x} & & otherwise \\
    \end{array}
  \right.
\end{mathpar}
 

where

\begin{eqnarray}
  (x)\id{\{} \lpquote Q \rpquote / \lpquote P \rpquote \id{\}}            = 
  \left\{ 
    \begin{array}{ccc}
      \lpquote Q \rpquote & & x \nameeq \lpquote P \rpquote \\
      x & & otherwise \\
    \end{array}
  \right. \nonumber
\end{eqnarray}

and $z$ is chosen distinct from $\quotep{P}$, $\quotep{Q}$, the free
names in $Q$, and all the names in $R$. Our $\alpha$-equivalence will
be built in the standard way from this substitution.

\begin{remark}\label{rem:no_self_referential_names}
  One consequence of these definitions is that $\forall P. \quotep{P}
  \not\in \freenames{P}$.
\end{remark}

\subsection{ Dynamic quote: an example }

Anticipating something of what's to come, consider applying the
substitution, $\widehat{\id{\{}u / z \id{\}}}$, to the following pair
of processes, $\lift{w}{y!(z)}$ and $w[ \lpquote y!(z) \rpquote ]$.

\begin{eqnarray}
	\lift{w}{y!(z)}\widehat{\id{\{}u / z \id{\}}}
		& = &
		\lift{w}{y!(u)} \nonumber\\
	w[ \lpquote y!(z) \rpquote ] \widehat{ \id{\{}u / z \id{\}} }
		& = &
		w[ \lpquote y!(z) \rpquote ] \nonumber
\end{eqnarray}

Because the body of the process between quotes is impervious to
substitution, we get radically different answers. In fact, by
examining the first process in an input context,
e.g. $x?(z).\lift{w}{y!(z)}$, we see that the process under the lift
operator may be shaped by prefixed inputs binding a name inside it. In
this sense, the lift operator will be seen as a way to dynamically
construct processes before reifying them as names.

Finally equipped with these standard features we can present the
dynamics of the calculus.

\subsubsection{Operational semantics} 

Finally, we introduce the computational dynamics. What marks these
algebras as distinct from other more traditionally studied algebraic
structures, e.g. vector spaces or polynomial rings, is the manner in
which dynamics is captured. In traditional structures, dynamics is typically
expressed through morphisms between such structures, as in linear maps
between vector spaces or morphisms between rings. In algebras
associated with the semantics of computation, the dynamics is
expressed as part of the algebraic structure itself, through a
reduction reduction relation typically denoted by $\red$. Below, we
give a recursive presentation of this relation for the calculus used
in the encoding.

$\red \subseteq \pi \times \pi$
$\red : \pi \to \mathcal{P}(\pi)$

\begin{mathpar}
  \inferrule* [lab=Comm] { \textsf{match}( x_{src}, x_{trgt} ) } { x_{trgt}?(y)P \; | \; x_{src}!\langle {Q} \rangle \red P\{\quotep{Q}/y}\} }
  \and \\
  \inferrule* [lab=Par] {{P} \red {P}'} {{{P} | {Q}} \red {{P}' | {Q}}}
  \and
  \inferrule* [lab=Equiv]{{{P} \scong {P}'} \andalso {{P}' \red {Q}'} \andalso {{Q}' \scong {Q}}}{{P} \red {Q}}
\end{mathpar}

\begin{eqnarray*}
  match_{\equiv} (\quotep{P},\quotep{Q}) & := & P \equiv Q \\
  match_{\dagger}(\quotep{P},\quotep{Q}) & := & \forall R. P|Q \red^{*} R => R \red^{*} 0 \\
  match_{K}(\quotep{P},\quotep{Q}) & := & K \mbox{ for some context } K
\end{eqnarray*}

$u?(x)P | u!\langle Q \rangle \red P\{\quotep{Q}/x\}$

%We write $\wred$ for $\red^*$, and $P\red$ if $\exists Q $ such that $ P \red Q$.
We write $P\red$ if $\exists Q $ such that $ P \red Q$ and $P\not\red$, otherwise.

\section{Replication}

As mentioned before, it is known that replication (and hence
recursion) can be implemented in a higher-order process algebra
\cite{SangiorgiWalker}. As our first example of calculation with the
machinery thus far presented we give the construction explicitly in
the {\rhoc}.

\begin{eqnarray}
	D_{x} & := & \prefix{x}{y}{(\binpar{\outputp{x}{y}}{@{y}})} \nonumber\\
	\bangp_{x}{P} & := & \binpar{{x}!\langle{\binpar{D_{x}}{P}}\rangle}{D_{x}} \nonumber
\end{eqnarray}

\begin{eqnarray}
	\bangp_{x}{P} & & \nonumber\\
	=
	& {x}!\langle{(\prefix{x}{y}{(\outputp{x}{y} | @{y})) | P}}\rangle 
	      | \prefix{x}{y}{(\outputp{x}{y} | @{y})} & \nonumber\\
	\red
	& (\outputp{x}{y} | @{y})\substn{\quotep{(\prefix{x}{y}{(@{y} | \outputp{x}{y})) | P}}}{y} & \nonumber\\
	=
	& \outputp{x}{\quotep{(\prefix{x}{y}{(\outputp{x}{y} | @{y})) | P}}}
	  | {(\prefix{x}{y}{(\outputp{x}{y} | @{y})) | P}} & \nonumber\\
	\red
	& \ldots & \nonumber\\
	\red^*
	& P | P | \ldots & \nonumber
\end{eqnarray}

Of course, this encoding, as an implementation, runs away, unfolding
$\bangp{P}$ eagerly. A lazier and more implementable replication
operator, restricted to input-guarded processes, may be obtained as follows.

\begin{eqnarray}
\bangp{\prefix{u}{v}{P}} 
	:= 
	\binpar{\lift{x}{\prefix{u}{v}{(\binpar{D(x)}{P})}}}{D(x)} \nonumber
\end{eqnarray}

\begin{remark}
  Note that the lazier definition still does not deal with summation
  or mixed summation (i.e. sums over input and output). The reader is
  invited to construct definitions of replication that deal with these
  features. 

  Further, the definitions are parameterized in a name, $x$. Can you,
  gentle reader, make a definition that eliminates this parameter and
  guarantees no accidental interaction between the replication
  machinery and the process being replicated -- i.e. no accidental
  sharing of names used by the process to get its work done and the
  name(s) used by the replication to effect copying. This latter
  revision of the definition of replication is crucial to obtaining
  the expected identity $!!P \sim !P$.
\end{remark}

\begin{remark}\label{rem:paradoxical_combinator}
  The reader familiar with the lambda calculus will have noticed the
  similarity between $D$ and the paradoxical combinator.

  [Ed. note: the existence of this seems to suggest we have to be more
  restrictive on the set of processes and names we admit if we are to
  support no-cloning.]
\end{remark}

\subsubsection{Bisimulation}

The computational dynamics gives rise to another kind of equivalence,
the equivalence of computational behavior. As previously mentioned
this is typically captured \emph{via} some form of bisimulation.

% The notion we use in this paper is weak barbed bisimulation
% \cite{milner91polyadicpi}.

The notion we use in this paper is derived from weak barbed
bisimulation \cite{milner91polyadicpi}. 

\begin{definition}
An \emph{observation relation}, $\downarrow_{\mathcal N}$, over a set
of names, $\mathcal N$, is the smallest relation satisfying the rules
below.

\infrule[Out-barb]{y \in {\mathcal N}, \; x \nameeq y}
		  {\outputp{x}{v} \downarrow_{\mathcal N} x}
\infrule[Par-barb]{\mbox{$P\downarrow_{\mathcal N} x$ or $Q\downarrow_{\mathcal N} x$}}
		  {\binpar{P}{Q} \downarrow_{\mathcal N} x}

We write $P \Downarrow_{\mathcal N} x$ if there is $Q$ such that 
$P \wred Q$ and $Q \downarrow_{\mathcal N} x$.
\end{definition}

\begin{definition}
%\label{def.bbisim}
An  ${\mathcal N}$-\emph{barbed bisimulation} over a set of names, ${\mathcal N}$, is a symmetric binary relation 
${\mathcal S}_{\mathcal N}$ between agents such that $P\rel{S}_{\mathcal N}Q$ implies:
\begin{enumerate}
\item If $P \red P'$ then $Q \wred Q'$ and $P'\rel{S}_{\mathcal N} Q'$.
\item If $P\downarrow_{\mathcal N} x$, then $Q\Downarrow_{\mathcal N} x$.
\end{enumerate}
$P$ is ${\mathcal N}$-barbed bisimilar to $Q$, written
$P \wbbisim_{\mathcal N} Q$, if $P \rel{S}_{\mathcal N} Q$ for some ${\mathcal N}$-barbed bisimulation ${\mathcal S}_{\mathcal N}$.
\end{definition}

$\mathcal{R} \subseteq \pi \times \pi$

$P \mathcal{R} Q => \forall P'. P \red P' \Rightarrow \exists Q'. Q \red Q', P' \mathcal{R} Q'$

$P \vdash x \Rightarrow Q \vdash x$

\begin{mathpar}
  \inferrule*[lab=Out-barb]{x \nameeq y}{{y}!\langle{Q}\rangle \vdash x}
  \and
  \inferrule*[lab=Par-barb]{\mbox{$P\vdash x$ or $Q\vdash x$}}{\binpar{P}{Q} \vdash x}
\end{mathpar}

\subsubsection{Contexts}

One of the principle advantages of computational calculi like the
$\pi$-calculus is a well-defined notion of context,
contextual-equivalence and a correlation between
contextual-equivalence and notions of bisimulation. The notion of
context allows the decomposition of a process into (sub-)process and
its syntactic environment, its context. Thus, a context may be
thought of as a process with a ``hole'' (written $\Box$) in it. The
application of a context $M$ to a process $P$, written $M[P]$, is
tantamount to filling the hole in $M$ with $P$. In this paper we do
not need the full weight of this theory, but do make use of the notion
of context in the proof the main theorem. 

\begin{mathpar}
  \inferrule* [lab=summation] {} {{M_{M},M_{N}} \bc \Box \;|\; x.M_{A} \;|\; M_{M}+M_{N}}
  \and
  \inferrule* [lab=agent] {} {{M_{A}} \bc (\vec{x})M_{P} \;| \; \clift{P_0,\ldots,M_{P},\ldots,P_N}}
  \and \\
  \inferrule* [lab=process] {} {{M_{P}} \bc M_{N} \;| \;P|M_{P} }
\end{mathpar} 

\begin{mathpar}
  \inferrule* [lab=sychronization] {} {M_{N} \bc \Box \;|\; x?M_{F} \;|\; x!M_{C}}
  \and
  \inferrule* [lab=abstraction] {} {{M_{F}} \bc (x)M_{P} }
  \and
  \inferrule* [lab=concretion] {} {{M_{C}} \bc \langle M_{P} \rangle }
  \and \\
  \inferrule* [lab=process] {} {{M_{P}} \bc M_{N} \;| \;P|M_{P} }
\end{mathpar}

\begin{definition}[contextual application] Given a context $M$, and
  process $P$, we define the \emph{contextual application}, $M[P] :=
  M\{P/\Box\}$. That is, the contextual application of M to P is the
  substitution of $P$ for $\Box$ in $M$.
\end{definition}

$\meaningof{-} : L \to \mathcal{P}(\pi)$

\begin{mathpar}
  \inferrule* [lab=collection] {} {\meaningof{true} = \pi, \and \meaningof{~E} = \pi \setminus \meaningof{E}, \and \meaningof{E_{1} \& E_{2}} = \meaningof{E_{1}} \cap \meaningof{E_{2}}}
\end{mathpar}

\begin{mathpar}
  \inferrule* [lab=structure] {} {\meaningof{0} = \{ P \in \pi | P \equiv 0 \}, \and \\ \meaningof{E_1 | E_2} = \{ P \in \pi | P \equiv P_{1} | P_{2}, P_{1} \in \meaningof{E_{1}}, P_{2} \in \meaningof{E_2}\} }
\end{mathpar}

\begin{mathpar}
 \inferrule* [lab=behavior] {} {\meaningof{\langle a?b \rangle E} = \{ P \in \pi | P \equiv Q | u?(y)P', \\ \and \\\\ \and \\ \;\;\; u \in \meaningof{a}, \forall z.P'\{z/y\} \in \meaningof{E\{z/b\}}\}, \and \\ \meaningof{a!E} = \{ P \in \pi | P \equiv Q | x!\langle P' \rangle, x \in \meaningof{a} P' \in \meaningof{E}\} }
\end{mathpar}

\begin{mathpar}
 \inferrule* [lab=nominal] {} {\meaningof{\quotep{E}} = \{ \quotep{P} \in \quotep{\pi} | P \in \meaningof{E} \}, \and \meaningof{\quotep{P}} = \{ \quotep{Q} \in \quotep{\pi} | P \equiv Q \} \and \\ \meaningof{@\quotep{E}} = \{ P \in \pi | P \equiv @x, x \in \meaningof{E} \}}
\end{mathpar}

\begin{eqnarray*}
  \\
  \meaningof{-} : TS \to ST
\end{eqnarray*}

\begin{eqnarray*}
  \\
  L : TS \to ST
\end{eqnarray*}

\begin{eqnarray*}
  \\
  P \models E \iff P \in \meaningof{E}
\end{eqnarray*}

\begin{eqnarray*}
  P \approx_{L} Q \iff \forall E \in L. P \models E \iff Q \models E
\end{eqnarray*}

\begin{eqnarray*}
  P \approx_{K} Q
\end{eqnarray*}

\begin{eqnarray*}
  P \approx Q
\end{eqnarray*}

$\approx_{K} = \approx = \approx_{L}$

\subsubsection{Contextual duality}

Note that contexts extend the quotation operation to a family of
operations from processes to names. Given a context, $M$, we can
define a \emph{nominal context}, $\quotep{M}$ by $\quotep{M}[P] :=
\quotep{M[P]}$. To foreshadow what is to come we observe that these
operations enjoy a duality with processes very much like the duality
between vectors and maps from vectors to scalars.

Further, because the calculus is essentially higher-order, we have a
correspondence between contexts and processes. More specifically,
given a name $x$ and a context $M$ we can construct $M^{*}_{x}$ such
that 

\begin{mathpar}
  M^{*}_{x} | \lift{x}{P} \red M[P]
\end{mathpar}

namely,

\begin{mathpar}
  M^{*}_{x} := x?(u).M[\dropn{u}]
\end{mathpar}

The dependence of $M^{*}_{x}$ on a name makes it an abstraction, 

\begin{mathpar}
  M^{*} := (x)x?(u).M[\dropn{u}]
\end{mathpar}

\subsection{Additional notation}

It will sometimes be convenient to denote the process a name
quotes. We already have the notation $x = \quotep{P}$, but it will be
convenient to introduce an alternate notation, $\procn{x}$, when we
want to emphasize the connection to the use of the name. Note that, by
virtue of name equivalence, $\quotep{\procn{x}} \nameeq x$; so, the
notation is consistent with previous definitions.

Further, because names have structure it is possible to effect
substitutions on the basis of that structure. This means we need to
upgrade our notation for substitutions, which we accomplish by
adapting comprehension notation. Thus,

\begin{mathpar}
  P\{ y / x : x \in S \}
\end{mathpar}

is interpreted to mean the process derived from P by replacing (in a
capture-avoiding manner) each occurrence of $x$ in $S$ by $y$. For example,

\begin{mathpar}
  P\{ \quotep{\procn{x}|\procn{x}} / x : x \in \freenames{P} \}
\end{mathpar}

will replace each (occurrence) of a free name $x$ in $P$ by
$\quotep{\procn{x}|\procn{x}}$.

Also, we will avail ourselves of the notation $x^{L}$ and $x^{R}$ to
denote injections of a name into disjoint copies of the name
space. There are numerous ways to accomplish this. One example can be
found in \cite{MeredithR05}. This notation overloads to vectors of
names: $\vec{x}^{\pi} := (x_{i}^{\pi} \; : \; 0 \leq i < |\vec{x}| )$ where $\pi \in \{L,R\}$.

We also use $P^{\Box} := P|\Box$.

In \cite{MeredithR05} an interpretation of the new operator is
given. It turns out that there are several possible interpretations
all enjoying the requisite algebraic properties of the operator (see
\cite{milner91polyadicpi}). We will therefore make liberal use of
$(\nu\; \vec{x})P$.

% subsection the_syntax_and_semantics_of_the_notation_system (end)   

\input{qm2pi.qmops} 

\input{qm2pi.sterngerlach} 

\input{qm2pi.metric} 

% section concurrent_process_calculi (end)

%\input{qm2pi.proofsketch}

% section proof sketch (end)

%\input{qm2pi.slviaknots} 

% section spatial logic via knots (end)

\input{qm2pi.conclusion}

% section conclusion (end)

%\input{qm2pi.dtcodes} 

% section wiring algorithm (end)

\input{qm2pi.ack} 

% section acknowledgments (end)

\newpage


\bibliographystyle{plain}   
\bibliography{../../biblios/main.bib}

\input{qm2pi.rhodetails}

\end{document}

 

%\documentclass[12pt]{llncs}
%\documentclass{jktr}

\usepackage[pdftex]{hyperref}                   
\usepackage {listings}
\usepackage {mathpartir}
\usepackage{bcprules}
%\usepackage{listings}
                       
\usepackage{graphicx} 
%\usepackage[margins=2.5cm,nohead,nofoot]{geometry}
%\usepackage{geometry}
\usepackage{amsfonts}
\usepackage{amstext}
\usepackage{latexsym}
\usepackage{amssymb}
\usepackage{color}


%\include{myPreamble}
\include{qm2pi.local} 

%\ifpdf
%\usepackage[pdftex]{graphicx}
%\else
%\usepackage{graphicx}
%\fi

 % \ifpdf
%  \usepackage{pdfsync}
%  \if


%\title{Brief Article}
%\author{David F. Snyder}
%\author{L.G. Meredith}

%\address{Dept. of Math., Texas State University--San Marcos, San Marcos, TX 78666}
       
\pagestyle{empty}


\begin{document}

\lstset{language=[Objective]Caml,frame=shadowbox}

\input{qm2pi.front}

% section front matter (end)

\input{qm2pi.intro} 
 
% section introduction (end)

% \input{qm2pi.knotations} 

% section notation (end)

\input{qm2pi.process.calculi} 

% section concurrent_process_calculi_and_spatial_logics_ (end)
    
%\input{qm2pi.knots2pi} 

%\input{qm2pi.trefoil} 

%\input{qm2pi.mainthm} 

% subsection basic_interpretation (end)

%\input{qm2pi.rho.presentation} 
\subsection{The syntax and semantics of the notation system}\label{sub:the_syntax_and_semantics_of_the_notation_system} % (fold)

We now summarize a technical presentation of the calculus that
embodies our theory of dynamics. The typical presentation of such a
calculus follows the style of giving generators and relations on
them. The grammar, below, describing term constructors, freely
generates the set of processes, $\Proc$. This set is then quotiented
by a relation known as structural congruence and it is over this set
that the notion of dynamics is expressed. This presentation is
essentially that of \cite{MeredithR05} with the addition of
polyadicity and summation. For readability we have relegated some of
the technical subtleties to an appendix.

\subsubsection{Process grammar}\label{subsub:process_grammar}

\begin{mathpar}
  \inferrule* [lab=synchronization] {} {{M} \bc \pzero \;|\; x?F \;|\; x!C }
  \and
  \inferrule* [lab=abstraction] {} {{F} \bc (x)P}
  \and
  \inferrule* [lab=concretion] {} {{C} \bc \langle Q \rangle}
  \and
  \inferrule* [lab=process] {} {{P,Q} \bc M \;| \;P|Q \;|\; @{x}}
  \and
  \inferrule* [lab=name] {} {{x} \bc \quotep{P}}
\end{mathpar} 

Note that $\vec{x}$ (resp. $\vec{P}$) denotes a vector of names
(resp. processes) of length $|\vec{x}|$ (resp. $|\vec{P}|$). We adopt
the following useful abbreviations.

\begin{mathpar}
   x?(\vec{y}).P := x.(\vec{y})P \and  x\clift{\vec{P}} := x.\clift{\vec{P}}
   \and x!(y) := \lift{x}{\dropn{y}}
   \and \Pi_{i=0}^{n-1}P_i := P_0 | \ldots | P_{n-1}
\end{mathpar}

\subsubsection{Structural congruence}

\paragraph{Free and bound names and alpha-equivalence.} At the
core of structural equivalence is alpha-equivalence which identifies
process that are the same up to a change of variable. Formally, we
recognize the distinction between free and bound names. The free names
of a process, $\freenames{P}$, may be calculated recursively as
follows:

\begin{mathpar}
\freenames{\pzero} := \emptyset
  \and \\
  \freenames{x?(y).P} := \{ x \} \cup (\freenames{P} \setminus \{ y \})
  \and 
  \freenames{x!\langle P \rangle} := \{ x \} \cup \{ P \} 
  \and \\
  \freenames{P|Q} := \freenames{P} \cup \freenames{Q}
  \and \\
  \freenames{@{x}} := \{ x \}
\end{mathpar}

$\pi$
$\quotep{\pi}$

$\freenames{-} : \pi \to \mathcal{P}(\quotep{\pi})$

\begin{eqnarray*}
  \freenames{\pzero} & := & \emptyset \\
  \freenames{x?(y).P} & := & \{ x \} \cup (\freenames{P} \setminus \{ y \}) \\
  \freenames{x!\langle P \rangle} & := & \{ x \} \cup \{ P \} \\
  \freenames{P|Q} & := & \freenames{P} \cup \freenames{Q} \\
  \freenames{\dropn{x}} & := & \{ x \}
\end{eqnarray*}

The bound names of a process, $\boundnames{P}$, are those names occurring in $P$
that are not free. For example, in $x?(y).0$, the name $x$ is free, while $y$ is bound.

\begin{mathpar}
  \inferrule* [lab=monoidal-laws] {} { P|Q \equiv Q|P \and P|0 \equiv P \and P|(Q|R) \equiv (P|Q)|R }
\end{mathpar}

\begin{mathpar}
  \inferrule* [lab=alpha-equivalence] {} { (x)P \equiv (y)P\{y/x\} \and y \not\in \freenames{P} }
\end{mathpar}

\begin{definition}
Then two processes, $P,Q$, are alpha-equivalent if $P = Q\{\vec{y}/\vec{x}\}$ for
some $\vec{x} \in \boundnames{Q},\vec{y} \in \boundnames{P}$, where $Q\{\vec{y}/\vec{x}\}$
denotes the capture-avoiding substitution of $\vec{y}$ for $\vec{x}$ in $Q$.
\end{definition}

\begin{definition}
  The {\em structural congruence} \cite{SangiorgiWalker} , $\equiv$,
  between processes is the least congruence containing
  alpha-equivalence, satisfying the abelian monoid laws
  (associativity, commutativity and $\pzero$ as identity) for parallel
  composition $|$ and for summation $+$.
\end{definition}

\subsection{Name equivalence}

We take name equivalence, written $\nameeq$, to be the smallest
equivalence relation generated by the following rules.

\begin{mathpar}
\inferrule*[lab=Quote-drop]
{ }
{ \quotep{@{x}} \nameeq x }

\inferrule*[lab=Struct-equiv]
{ P \scong Q }
{ \quotep{P} \nameeq \quotep{Q} }
\end{mathpar}

The astute reader will have noticed that the mutual recursion of names
and processes imposes a mutual recursion on alpha-equivalence and
structural equivalence via name-equivalence. Fortunately, all of this
works out pleasantly and we may calculate in the natural way, free of
concern. The reader interested in the details is referred to the
appendix \ref{appendix:rho_details}.

\subsection{Substitution}

We use $\Proc$ for the set of processes, $\QProc$ for the set of
names, and $\id{\{}\vec{y} / \vec{x} \id{\}}$ to denote partial maps,
$s : \QProc \rightarrow \QProc$. A map, $s$ lifts, uniquely, to a map
on process terms, $\widehat{s} : \Proc \rightarrow \Proc$ by the
following equations.

\begin{mathpar}
  (0) \psubstp{Q}{P} := 0 \\
  (R \juxtap S) \psubstp{Q}{P}
  :=    
  (R)\psubstp{Q}{P} \juxtap (S) \psubstp{Q}{P} \\
  (x?(y).R) \psubstp{Q}{P}    
  :=    
  (x)\substp{Q}{P} (z)\concat( (R \psubstn{z}{y}) \psubstp{Q}{P} ) \\
  (\lift{x}{R}) \psubstp{Q}{P}  
  :=
  \lift{(x)\substp{Q}{P}}{ R \psubstp{Q}{P} } \\
%   (\dropn{x})  \psubstp{Q}{P}       
%   := 
%   \left\{ 
%     \begin{array}{ccc} 
%       \dropn{\quotep{Q}} & & x \nameeq \quotep{P} \\
%       \dropn{x} & & otherwise \\
%     \end{array}
%   \right. 
  (\dropn{x})  \psubstp{Q}{P}       
  := 
  \left\{ 
    \begin{array}{ccc} 
      Q & & x \nameeq \quotep{P} \\
      \dropn{x} & & otherwise \\
    \end{array}
  \right.
\end{mathpar}
 

where

\begin{eqnarray}
  (x)\id{\{} \lpquote Q \rpquote / \lpquote P \rpquote \id{\}}            = 
  \left\{ 
    \begin{array}{ccc}
      \lpquote Q \rpquote & & x \nameeq \lpquote P \rpquote \\
      x & & otherwise \\
    \end{array}
  \right. \nonumber
\end{eqnarray}

and $z$ is chosen distinct from $\quotep{P}$, $\quotep{Q}$, the free
names in $Q$, and all the names in $R$. Our $\alpha$-equivalence will
be built in the standard way from this substitution.

\begin{remark}\label{rem:no_self_referential_names}
  One consequence of these definitions is that $\forall P. \quotep{P}
  \not\in \freenames{P}$.
\end{remark}

\subsection{ Dynamic quote: an example }

Anticipating something of what's to come, consider applying the
substitution, $\widehat{\id{\{}u / z \id{\}}}$, to the following pair
of processes, $\lift{w}{y!(z)}$ and $w[ \lpquote y!(z) \rpquote ]$.

\begin{eqnarray}
	\lift{w}{y!(z)}\widehat{\id{\{}u / z \id{\}}}
		& = &
		\lift{w}{y!(u)} \nonumber\\
	w[ \lpquote y!(z) \rpquote ] \widehat{ \id{\{}u / z \id{\}} }
		& = &
		w[ \lpquote y!(z) \rpquote ] \nonumber
\end{eqnarray}

Because the body of the process between quotes is impervious to
substitution, we get radically different answers. In fact, by
examining the first process in an input context,
e.g. $x?(z).\lift{w}{y!(z)}$, we see that the process under the lift
operator may be shaped by prefixed inputs binding a name inside it. In
this sense, the lift operator will be seen as a way to dynamically
construct processes before reifying them as names.

Finally equipped with these standard features we can present the
dynamics of the calculus.

\subsubsection{Operational semantics} 

Finally, we introduce the computational dynamics. What marks these
algebras as distinct from other more traditionally studied algebraic
structures, e.g. vector spaces or polynomial rings, is the manner in
which dynamics is captured. In traditional structures, dynamics is typically
expressed through morphisms between such structures, as in linear maps
between vector spaces or morphisms between rings. In algebras
associated with the semantics of computation, the dynamics is
expressed as part of the algebraic structure itself, through a
reduction reduction relation typically denoted by $\red$. Below, we
give a recursive presentation of this relation for the calculus used
in the encoding.

$\red \subseteq \pi \times \pi$
$\red : \pi \to \mathcal{P}(\pi)$

\begin{mathpar}
  \inferrule* [lab=Comm] { \textsf{match}( x_{src}, x_{trgt} ) } { x_{trgt}?(y)P \; | \; x_{src}!\langle {Q} \rangle \red P\{\quotep{Q}/y}\} }
  \and \\
  \inferrule* [lab=Par] {{P} \red {P}'} {{{P} | {Q}} \red {{P}' | {Q}}}
  \and
  \inferrule* [lab=Equiv]{{{P} \scong {P}'} \andalso {{P}' \red {Q}'} \andalso {{Q}' \scong {Q}}}{{P} \red {Q}}
\end{mathpar}

\begin{eqnarray*}
  match_{\equiv} (\quotep{P},\quotep{Q}) & := & P \equiv Q \\
  match_{\dagger}(\quotep{P},\quotep{Q}) & := & \forall R. P|Q \red^{*} R => R \red^{*} 0 \\
  match_{K}(\quotep{P},\quotep{Q}) & := & K \mbox{ for some context } K
\end{eqnarray*}

$u?(x)P | u!\langle Q \rangle \red P\{\quotep{Q}/x\}$

%We write $\wred$ for $\red^*$, and $P\red$ if $\exists Q $ such that $ P \red Q$.
We write $P\red$ if $\exists Q $ such that $ P \red Q$ and $P\not\red$, otherwise.

\section{Replication}

As mentioned before, it is known that replication (and hence
recursion) can be implemented in a higher-order process algebra
\cite{SangiorgiWalker}. As our first example of calculation with the
machinery thus far presented we give the construction explicitly in
the {\rhoc}.

\begin{eqnarray}
	D_{x} & := & \prefix{x}{y}{(\binpar{\outputp{x}{y}}{@{y}})} \nonumber\\
	\bangp_{x}{P} & := & \binpar{{x}!\langle{\binpar{D_{x}}{P}}\rangle}{D_{x}} \nonumber
\end{eqnarray}

\begin{eqnarray}
	\bangp_{x}{P} & & \nonumber\\
	=
	& {x}!\langle{(\prefix{x}{y}{(\outputp{x}{y} | @{y})) | P}}\rangle 
	      | \prefix{x}{y}{(\outputp{x}{y} | @{y})} & \nonumber\\
	\red
	& (\outputp{x}{y} | @{y})\substn{\quotep{(\prefix{x}{y}{(@{y} | \outputp{x}{y})) | P}}}{y} & \nonumber\\
	=
	& \outputp{x}{\quotep{(\prefix{x}{y}{(\outputp{x}{y} | @{y})) | P}}}
	  | {(\prefix{x}{y}{(\outputp{x}{y} | @{y})) | P}} & \nonumber\\
	\red
	& \ldots & \nonumber\\
	\red^*
	& P | P | \ldots & \nonumber
\end{eqnarray}

Of course, this encoding, as an implementation, runs away, unfolding
$\bangp{P}$ eagerly. A lazier and more implementable replication
operator, restricted to input-guarded processes, may be obtained as follows.

\begin{eqnarray}
\bangp{\prefix{u}{v}{P}} 
	:= 
	\binpar{\lift{x}{\prefix{u}{v}{(\binpar{D(x)}{P})}}}{D(x)} \nonumber
\end{eqnarray}

\begin{remark}
  Note that the lazier definition still does not deal with summation
  or mixed summation (i.e. sums over input and output). The reader is
  invited to construct definitions of replication that deal with these
  features. 

  Further, the definitions are parameterized in a name, $x$. Can you,
  gentle reader, make a definition that eliminates this parameter and
  guarantees no accidental interaction between the replication
  machinery and the process being replicated -- i.e. no accidental
  sharing of names used by the process to get its work done and the
  name(s) used by the replication to effect copying. This latter
  revision of the definition of replication is crucial to obtaining
  the expected identity $!!P \sim !P$.
\end{remark}

\begin{remark}\label{rem:paradoxical_combinator}
  The reader familiar with the lambda calculus will have noticed the
  similarity between $D$ and the paradoxical combinator.

  [Ed. note: the existence of this seems to suggest we have to be more
  restrictive on the set of processes and names we admit if we are to
  support no-cloning.]
\end{remark}

\subsubsection{Bisimulation}

The computational dynamics gives rise to another kind of equivalence,
the equivalence of computational behavior. As previously mentioned
this is typically captured \emph{via} some form of bisimulation.

% The notion we use in this paper is weak barbed bisimulation
% \cite{milner91polyadicpi}.

The notion we use in this paper is derived from weak barbed
bisimulation \cite{milner91polyadicpi}. 

\begin{definition}
An \emph{observation relation}, $\downarrow_{\mathcal N}$, over a set
of names, $\mathcal N$, is the smallest relation satisfying the rules
below.

\infrule[Out-barb]{y \in {\mathcal N}, \; x \nameeq y}
		  {\outputp{x}{v} \downarrow_{\mathcal N} x}
\infrule[Par-barb]{\mbox{$P\downarrow_{\mathcal N} x$ or $Q\downarrow_{\mathcal N} x$}}
		  {\binpar{P}{Q} \downarrow_{\mathcal N} x}

We write $P \Downarrow_{\mathcal N} x$ if there is $Q$ such that 
$P \wred Q$ and $Q \downarrow_{\mathcal N} x$.
\end{definition}

\begin{definition}
%\label{def.bbisim}
An  ${\mathcal N}$-\emph{barbed bisimulation} over a set of names, ${\mathcal N}$, is a symmetric binary relation 
${\mathcal S}_{\mathcal N}$ between agents such that $P\rel{S}_{\mathcal N}Q$ implies:
\begin{enumerate}
\item If $P \red P'$ then $Q \wred Q'$ and $P'\rel{S}_{\mathcal N} Q'$.
\item If $P\downarrow_{\mathcal N} x$, then $Q\Downarrow_{\mathcal N} x$.
\end{enumerate}
$P$ is ${\mathcal N}$-barbed bisimilar to $Q$, written
$P \wbbisim_{\mathcal N} Q$, if $P \rel{S}_{\mathcal N} Q$ for some ${\mathcal N}$-barbed bisimulation ${\mathcal S}_{\mathcal N}$.
\end{definition}

$\mathcal{R} \subseteq \pi \times \pi$

$P \mathcal{R} Q => \forall P'. P \red P' \Rightarrow \exists Q'. Q \red Q', P' \mathcal{R} Q'$

$P \vdash x \Rightarrow Q \vdash x$

\begin{mathpar}
  \inferrule*[lab=Out-barb]{x \nameeq y}{{y}!\langle{Q}\rangle \vdash x}
  \and
  \inferrule*[lab=Par-barb]{\mbox{$P\vdash x$ or $Q\vdash x$}}{\binpar{P}{Q} \vdash x}
\end{mathpar}

\subsubsection{Contexts}

One of the principle advantages of computational calculi like the
$\pi$-calculus is a well-defined notion of context,
contextual-equivalence and a correlation between
contextual-equivalence and notions of bisimulation. The notion of
context allows the decomposition of a process into (sub-)process and
its syntactic environment, its context. Thus, a context may be
thought of as a process with a ``hole'' (written $\Box$) in it. The
application of a context $M$ to a process $P$, written $M[P]$, is
tantamount to filling the hole in $M$ with $P$. In this paper we do
not need the full weight of this theory, but do make use of the notion
of context in the proof the main theorem. 

\begin{mathpar}
  \inferrule* [lab=summation] {} {{M_{M},M_{N}} \bc \Box \;|\; x.M_{A} \;|\; M_{M}+M_{N}}
  \and
  \inferrule* [lab=agent] {} {{M_{A}} \bc (\vec{x})M_{P} \;| \; \clift{P_0,\ldots,M_{P},\ldots,P_N}}
  \and \\
  \inferrule* [lab=process] {} {{M_{P}} \bc M_{N} \;| \;P|M_{P} }
\end{mathpar} 

\begin{mathpar}
  \inferrule* [lab=sychronization] {} {M_{N} \bc \Box \;|\; x?M_{F} \;|\; x!M_{C}}
  \and
  \inferrule* [lab=abstraction] {} {{M_{F}} \bc (x)M_{P} }
  \and
  \inferrule* [lab=concretion] {} {{M_{C}} \bc \langle M_{P} \rangle }
  \and \\
  \inferrule* [lab=process] {} {{M_{P}} \bc M_{N} \;| \;P|M_{P} }
\end{mathpar}

\begin{definition}[contextual application] Given a context $M$, and
  process $P$, we define the \emph{contextual application}, $M[P] :=
  M\{P/\Box\}$. That is, the contextual application of M to P is the
  substitution of $P$ for $\Box$ in $M$.
\end{definition}

$\meaningof{-} : L \to \mathcal{P}(\pi)$

\begin{mathpar}
  \inferrule* [lab=collection] {} {\meaningof{true} = \pi, \and \meaningof{~E} = \pi \setminus \meaningof{E}, \and \meaningof{E_{1} \& E_{2}} = \meaningof{E_{1}} \cap \meaningof{E_{2}}}
\end{mathpar}

\begin{mathpar}
  \inferrule* [lab=structure] {} {\meaningof{0} = \{ P \in \pi | P \equiv 0 \}, \and \\ \meaningof{E_1 | E_2} = \{ P \in \pi | P \equiv P_{1} | P_{2}, P_{1} \in \meaningof{E_{1}}, P_{2} \in \meaningof{E_2}\} }
\end{mathpar}

\begin{mathpar}
 \inferrule* [lab=behavior] {} {\meaningof{\langle a?b \rangle E} = \{ P \in \pi | P \equiv Q | u?(y)P', \\ \and \\\\ \and \\ \;\;\; u \in \meaningof{a}, \forall z.P'\{z/y\} \in \meaningof{E\{z/b\}}\}, \and \\ \meaningof{a!E} = \{ P \in \pi | P \equiv Q | x!\langle P' \rangle, x \in \meaningof{a} P' \in \meaningof{E}\} }
\end{mathpar}

\begin{mathpar}
 \inferrule* [lab=nominal] {} {\meaningof{\quotep{E}} = \{ \quotep{P} \in \quotep{\pi} | P \in \meaningof{E} \}, \and \meaningof{\quotep{P}} = \{ \quotep{Q} \in \quotep{\pi} | P \equiv Q \} \and \\ \meaningof{@\quotep{E}} = \{ P \in \pi | P \equiv @x, x \in \meaningof{E} \}}
\end{mathpar}

\begin{eqnarray*}
  \\
  \meaningof{-} : TS \to ST
\end{eqnarray*}

\begin{eqnarray*}
  \\
  L : TS \to ST
\end{eqnarray*}

\begin{eqnarray*}
  \\
  P \models E \iff P \in \meaningof{E}
\end{eqnarray*}

\begin{eqnarray*}
  P \approx_{L} Q \iff \forall E \in L. P \models E \iff Q \models E
\end{eqnarray*}

\begin{eqnarray*}
  P \approx_{K} Q
\end{eqnarray*}

\begin{eqnarray*}
  P \approx Q
\end{eqnarray*}

$\approx_{K} = \approx = \approx_{L}$

\subsubsection{Contextual duality}

Note that contexts extend the quotation operation to a family of
operations from processes to names. Given a context, $M$, we can
define a \emph{nominal context}, $\quotep{M}$ by $\quotep{M}[P] :=
\quotep{M[P]}$. To foreshadow what is to come we observe that these
operations enjoy a duality with processes very much like the duality
between vectors and maps from vectors to scalars.

Further, because the calculus is essentially higher-order, we have a
correspondence between contexts and processes. More specifically,
given a name $x$ and a context $M$ we can construct $M^{*}_{x}$ such
that 

\begin{mathpar}
  M^{*}_{x} | \lift{x}{P} \red M[P]
\end{mathpar}

namely,

\begin{mathpar}
  M^{*}_{x} := x?(u).M[\dropn{u}]
\end{mathpar}

The dependence of $M^{*}_{x}$ on a name makes it an abstraction, 

\begin{mathpar}
  M^{*} := (x)x?(u).M[\dropn{u}]
\end{mathpar}

\subsection{Additional notation}

It will sometimes be convenient to denote the process a name
quotes. We already have the notation $x = \quotep{P}$, but it will be
convenient to introduce an alternate notation, $\procn{x}$, when we
want to emphasize the connection to the use of the name. Note that, by
virtue of name equivalence, $\quotep{\procn{x}} \nameeq x$; so, the
notation is consistent with previous definitions.

Further, because names have structure it is possible to effect
substitutions on the basis of that structure. This means we need to
upgrade our notation for substitutions, which we accomplish by
adapting comprehension notation. Thus,

\begin{mathpar}
  P\{ y / x : x \in S \}
\end{mathpar}

is interpreted to mean the process derived from P by replacing (in a
capture-avoiding manner) each occurrence of $x$ in $S$ by $y$. For example,

\begin{mathpar}
  P\{ \quotep{\procn{x}|\procn{x}} / x : x \in \freenames{P} \}
\end{mathpar}

will replace each (occurrence) of a free name $x$ in $P$ by
$\quotep{\procn{x}|\procn{x}}$.

Also, we will avail ourselves of the notation $x^{L}$ and $x^{R}$ to
denote injections of a name into disjoint copies of the name
space. There are numerous ways to accomplish this. One example can be
found in \cite{MeredithR05}. This notation overloads to vectors of
names: $\vec{x}^{\pi} := (x_{i}^{\pi} \; : \; 0 \leq i < |\vec{x}| )$ where $\pi \in \{L,R\}$.

We also use $P^{\Box} := P|\Box$.

In \cite{MeredithR05} an interpretation of the new operator is
given. It turns out that there are several possible interpretations
all enjoying the requisite algebraic properties of the operator (see
\cite{milner91polyadicpi}). We will therefore make liberal use of
$(\nu\; \vec{x})P$.

% subsection the_syntax_and_semantics_of_the_notation_system (end)   

\input{qm2pi.qmops} 

\input{qm2pi.sterngerlach} 

\input{qm2pi.metric} 

% section concurrent_process_calculi (end)

%\input{qm2pi.proofsketch}

% section proof sketch (end)

%\input{qm2pi.slviaknots} 

% section spatial logic via knots (end)

\input{qm2pi.conclusion}

% section conclusion (end)

%\input{qm2pi.dtcodes} 

% section wiring algorithm (end)

\input{qm2pi.ack} 

% section acknowledgments (end)

\newpage


\bibliographystyle{plain}   
\bibliography{../../biblios/main.bib}

\input{qm2pi.rhodetails}

\end{document}

 

%\documentclass[12pt]{llncs}
%\documentclass{jktr}

\usepackage[pdftex]{hyperref}                   
\usepackage {listings}
\usepackage {mathpartir}
\usepackage{bcprules}
%\usepackage{listings}
                       
\usepackage{graphicx} 
%\usepackage[margins=2.5cm,nohead,nofoot]{geometry}
%\usepackage{geometry}
\usepackage{amsfonts}
\usepackage{amstext}
\usepackage{latexsym}
\usepackage{amssymb}
\usepackage{color}


%\include{myPreamble}
\include{qm2pi.local} 

%\ifpdf
%\usepackage[pdftex]{graphicx}
%\else
%\usepackage{graphicx}
%\fi

 % \ifpdf
%  \usepackage{pdfsync}
%  \if


%\title{Brief Article}
%\author{David F. Snyder}
%\author{L.G. Meredith}

%\address{Dept. of Math., Texas State University--San Marcos, San Marcos, TX 78666}
       
\pagestyle{empty}


\begin{document}

\lstset{language=[Objective]Caml,frame=shadowbox}

\input{qm2pi.front}

% section front matter (end)

\input{qm2pi.intro} 
 
% section introduction (end)

% \input{qm2pi.knotations} 

% section notation (end)

\input{qm2pi.process.calculi} 

% section concurrent_process_calculi_and_spatial_logics_ (end)
    
%\input{qm2pi.knots2pi} 

%\input{qm2pi.trefoil} 

%\input{qm2pi.mainthm} 

% subsection basic_interpretation (end)

%\input{qm2pi.rho.presentation} 
\subsection{The syntax and semantics of the notation system}\label{sub:the_syntax_and_semantics_of_the_notation_system} % (fold)

We now summarize a technical presentation of the calculus that
embodies our theory of dynamics. The typical presentation of such a
calculus follows the style of giving generators and relations on
them. The grammar, below, describing term constructors, freely
generates the set of processes, $\Proc$. This set is then quotiented
by a relation known as structural congruence and it is over this set
that the notion of dynamics is expressed. This presentation is
essentially that of \cite{MeredithR05} with the addition of
polyadicity and summation. For readability we have relegated some of
the technical subtleties to an appendix.

\subsubsection{Process grammar}\label{subsub:process_grammar}

\begin{mathpar}
  \inferrule* [lab=synchronization] {} {{M} \bc \pzero \;|\; x?F \;|\; x!C }
  \and
  \inferrule* [lab=abstraction] {} {{F} \bc (x)P}
  \and
  \inferrule* [lab=concretion] {} {{C} \bc \langle Q \rangle}
  \and
  \inferrule* [lab=process] {} {{P,Q} \bc M \;| \;P|Q \;|\; @{x}}
  \and
  \inferrule* [lab=name] {} {{x} \bc \quotep{P}}
\end{mathpar} 

Note that $\vec{x}$ (resp. $\vec{P}$) denotes a vector of names
(resp. processes) of length $|\vec{x}|$ (resp. $|\vec{P}|$). We adopt
the following useful abbreviations.

\begin{mathpar}
   x?(\vec{y}).P := x.(\vec{y})P \and  x\clift{\vec{P}} := x.\clift{\vec{P}}
   \and x!(y) := \lift{x}{\dropn{y}}
   \and \Pi_{i=0}^{n-1}P_i := P_0 | \ldots | P_{n-1}
\end{mathpar}

\subsubsection{Structural congruence}

\paragraph{Free and bound names and alpha-equivalence.} At the
core of structural equivalence is alpha-equivalence which identifies
process that are the same up to a change of variable. Formally, we
recognize the distinction between free and bound names. The free names
of a process, $\freenames{P}$, may be calculated recursively as
follows:

\begin{mathpar}
\freenames{\pzero} := \emptyset
  \and \\
  \freenames{x?(y).P} := \{ x \} \cup (\freenames{P} \setminus \{ y \})
  \and 
  \freenames{x!\langle P \rangle} := \{ x \} \cup \{ P \} 
  \and \\
  \freenames{P|Q} := \freenames{P} \cup \freenames{Q}
  \and \\
  \freenames{@{x}} := \{ x \}
\end{mathpar}

$\pi$
$\quotep{\pi}$

$\freenames{-} : \pi \to \mathcal{P}(\quotep{\pi})$

\begin{eqnarray*}
  \freenames{\pzero} & := & \emptyset \\
  \freenames{x?(y).P} & := & \{ x \} \cup (\freenames{P} \setminus \{ y \}) \\
  \freenames{x!\langle P \rangle} & := & \{ x \} \cup \{ P \} \\
  \freenames{P|Q} & := & \freenames{P} \cup \freenames{Q} \\
  \freenames{\dropn{x}} & := & \{ x \}
\end{eqnarray*}

The bound names of a process, $\boundnames{P}$, are those names occurring in $P$
that are not free. For example, in $x?(y).0$, the name $x$ is free, while $y$ is bound.

\begin{mathpar}
  \inferrule* [lab=monoidal-laws] {} { P|Q \equiv Q|P \and P|0 \equiv P \and P|(Q|R) \equiv (P|Q)|R }
\end{mathpar}

\begin{mathpar}
  \inferrule* [lab=alpha-equivalence] {} { (x)P \equiv (y)P\{y/x\} \and y \not\in \freenames{P} }
\end{mathpar}

\begin{definition}
Then two processes, $P,Q$, are alpha-equivalent if $P = Q\{\vec{y}/\vec{x}\}$ for
some $\vec{x} \in \boundnames{Q},\vec{y} \in \boundnames{P}$, where $Q\{\vec{y}/\vec{x}\}$
denotes the capture-avoiding substitution of $\vec{y}$ for $\vec{x}$ in $Q$.
\end{definition}

\begin{definition}
  The {\em structural congruence} \cite{SangiorgiWalker} , $\equiv$,
  between processes is the least congruence containing
  alpha-equivalence, satisfying the abelian monoid laws
  (associativity, commutativity and $\pzero$ as identity) for parallel
  composition $|$ and for summation $+$.
\end{definition}

\subsection{Name equivalence}

We take name equivalence, written $\nameeq$, to be the smallest
equivalence relation generated by the following rules.

\begin{mathpar}
\inferrule*[lab=Quote-drop]
{ }
{ \quotep{@{x}} \nameeq x }

\inferrule*[lab=Struct-equiv]
{ P \scong Q }
{ \quotep{P} \nameeq \quotep{Q} }
\end{mathpar}

The astute reader will have noticed that the mutual recursion of names
and processes imposes a mutual recursion on alpha-equivalence and
structural equivalence via name-equivalence. Fortunately, all of this
works out pleasantly and we may calculate in the natural way, free of
concern. The reader interested in the details is referred to the
appendix \ref{appendix:rho_details}.

\subsection{Substitution}

We use $\Proc$ for the set of processes, $\QProc$ for the set of
names, and $\id{\{}\vec{y} / \vec{x} \id{\}}$ to denote partial maps,
$s : \QProc \rightarrow \QProc$. A map, $s$ lifts, uniquely, to a map
on process terms, $\widehat{s} : \Proc \rightarrow \Proc$ by the
following equations.

\begin{mathpar}
  (0) \psubstp{Q}{P} := 0 \\
  (R \juxtap S) \psubstp{Q}{P}
  :=    
  (R)\psubstp{Q}{P} \juxtap (S) \psubstp{Q}{P} \\
  (x?(y).R) \psubstp{Q}{P}    
  :=    
  (x)\substp{Q}{P} (z)\concat( (R \psubstn{z}{y}) \psubstp{Q}{P} ) \\
  (\lift{x}{R}) \psubstp{Q}{P}  
  :=
  \lift{(x)\substp{Q}{P}}{ R \psubstp{Q}{P} } \\
%   (\dropn{x})  \psubstp{Q}{P}       
%   := 
%   \left\{ 
%     \begin{array}{ccc} 
%       \dropn{\quotep{Q}} & & x \nameeq \quotep{P} \\
%       \dropn{x} & & otherwise \\
%     \end{array}
%   \right. 
  (\dropn{x})  \psubstp{Q}{P}       
  := 
  \left\{ 
    \begin{array}{ccc} 
      Q & & x \nameeq \quotep{P} \\
      \dropn{x} & & otherwise \\
    \end{array}
  \right.
\end{mathpar}
 

where

\begin{eqnarray}
  (x)\id{\{} \lpquote Q \rpquote / \lpquote P \rpquote \id{\}}            = 
  \left\{ 
    \begin{array}{ccc}
      \lpquote Q \rpquote & & x \nameeq \lpquote P \rpquote \\
      x & & otherwise \\
    \end{array}
  \right. \nonumber
\end{eqnarray}

and $z$ is chosen distinct from $\quotep{P}$, $\quotep{Q}$, the free
names in $Q$, and all the names in $R$. Our $\alpha$-equivalence will
be built in the standard way from this substitution.

\begin{remark}\label{rem:no_self_referential_names}
  One consequence of these definitions is that $\forall P. \quotep{P}
  \not\in \freenames{P}$.
\end{remark}

\subsection{ Dynamic quote: an example }

Anticipating something of what's to come, consider applying the
substitution, $\widehat{\id{\{}u / z \id{\}}}$, to the following pair
of processes, $\lift{w}{y!(z)}$ and $w[ \lpquote y!(z) \rpquote ]$.

\begin{eqnarray}
	\lift{w}{y!(z)}\widehat{\id{\{}u / z \id{\}}}
		& = &
		\lift{w}{y!(u)} \nonumber\\
	w[ \lpquote y!(z) \rpquote ] \widehat{ \id{\{}u / z \id{\}} }
		& = &
		w[ \lpquote y!(z) \rpquote ] \nonumber
\end{eqnarray}

Because the body of the process between quotes is impervious to
substitution, we get radically different answers. In fact, by
examining the first process in an input context,
e.g. $x?(z).\lift{w}{y!(z)}$, we see that the process under the lift
operator may be shaped by prefixed inputs binding a name inside it. In
this sense, the lift operator will be seen as a way to dynamically
construct processes before reifying them as names.

Finally equipped with these standard features we can present the
dynamics of the calculus.

\subsubsection{Operational semantics} 

Finally, we introduce the computational dynamics. What marks these
algebras as distinct from other more traditionally studied algebraic
structures, e.g. vector spaces or polynomial rings, is the manner in
which dynamics is captured. In traditional structures, dynamics is typically
expressed through morphisms between such structures, as in linear maps
between vector spaces or morphisms between rings. In algebras
associated with the semantics of computation, the dynamics is
expressed as part of the algebraic structure itself, through a
reduction reduction relation typically denoted by $\red$. Below, we
give a recursive presentation of this relation for the calculus used
in the encoding.

$\red \subseteq \pi \times \pi$
$\red : \pi \to \mathcal{P}(\pi)$

\begin{mathpar}
  \inferrule* [lab=Comm] { \textsf{match}( x_{src}, x_{trgt} ) } { x_{trgt}?(y)P \; | \; x_{src}!\langle {Q} \rangle \red P\{\quotep{Q}/y}\} }
  \and \\
  \inferrule* [lab=Par] {{P} \red {P}'} {{{P} | {Q}} \red {{P}' | {Q}}}
  \and
  \inferrule* [lab=Equiv]{{{P} \scong {P}'} \andalso {{P}' \red {Q}'} \andalso {{Q}' \scong {Q}}}{{P} \red {Q}}
\end{mathpar}

\begin{eqnarray*}
  match_{\equiv} (\quotep{P},\quotep{Q}) & := & P \equiv Q \\
  match_{\dagger}(\quotep{P},\quotep{Q}) & := & \forall R. P|Q \red^{*} R => R \red^{*} 0 \\
  match_{K}(\quotep{P},\quotep{Q}) & := & K \mbox{ for some context } K
\end{eqnarray*}

$u?(x)P | u!\langle Q \rangle \red P\{\quotep{Q}/x\}$

%We write $\wred$ for $\red^*$, and $P\red$ if $\exists Q $ such that $ P \red Q$.
We write $P\red$ if $\exists Q $ such that $ P \red Q$ and $P\not\red$, otherwise.

\section{Replication}

As mentioned before, it is known that replication (and hence
recursion) can be implemented in a higher-order process algebra
\cite{SangiorgiWalker}. As our first example of calculation with the
machinery thus far presented we give the construction explicitly in
the {\rhoc}.

\begin{eqnarray}
	D_{x} & := & \prefix{x}{y}{(\binpar{\outputp{x}{y}}{@{y}})} \nonumber\\
	\bangp_{x}{P} & := & \binpar{{x}!\langle{\binpar{D_{x}}{P}}\rangle}{D_{x}} \nonumber
\end{eqnarray}

\begin{eqnarray}
	\bangp_{x}{P} & & \nonumber\\
	=
	& {x}!\langle{(\prefix{x}{y}{(\outputp{x}{y} | @{y})) | P}}\rangle 
	      | \prefix{x}{y}{(\outputp{x}{y} | @{y})} & \nonumber\\
	\red
	& (\outputp{x}{y} | @{y})\substn{\quotep{(\prefix{x}{y}{(@{y} | \outputp{x}{y})) | P}}}{y} & \nonumber\\
	=
	& \outputp{x}{\quotep{(\prefix{x}{y}{(\outputp{x}{y} | @{y})) | P}}}
	  | {(\prefix{x}{y}{(\outputp{x}{y} | @{y})) | P}} & \nonumber\\
	\red
	& \ldots & \nonumber\\
	\red^*
	& P | P | \ldots & \nonumber
\end{eqnarray}

Of course, this encoding, as an implementation, runs away, unfolding
$\bangp{P}$ eagerly. A lazier and more implementable replication
operator, restricted to input-guarded processes, may be obtained as follows.

\begin{eqnarray}
\bangp{\prefix{u}{v}{P}} 
	:= 
	\binpar{\lift{x}{\prefix{u}{v}{(\binpar{D(x)}{P})}}}{D(x)} \nonumber
\end{eqnarray}

\begin{remark}
  Note that the lazier definition still does not deal with summation
  or mixed summation (i.e. sums over input and output). The reader is
  invited to construct definitions of replication that deal with these
  features. 

  Further, the definitions are parameterized in a name, $x$. Can you,
  gentle reader, make a definition that eliminates this parameter and
  guarantees no accidental interaction between the replication
  machinery and the process being replicated -- i.e. no accidental
  sharing of names used by the process to get its work done and the
  name(s) used by the replication to effect copying. This latter
  revision of the definition of replication is crucial to obtaining
  the expected identity $!!P \sim !P$.
\end{remark}

\begin{remark}\label{rem:paradoxical_combinator}
  The reader familiar with the lambda calculus will have noticed the
  similarity between $D$ and the paradoxical combinator.

  [Ed. note: the existence of this seems to suggest we have to be more
  restrictive on the set of processes and names we admit if we are to
  support no-cloning.]
\end{remark}

\subsubsection{Bisimulation}

The computational dynamics gives rise to another kind of equivalence,
the equivalence of computational behavior. As previously mentioned
this is typically captured \emph{via} some form of bisimulation.

% The notion we use in this paper is weak barbed bisimulation
% \cite{milner91polyadicpi}.

The notion we use in this paper is derived from weak barbed
bisimulation \cite{milner91polyadicpi}. 

\begin{definition}
An \emph{observation relation}, $\downarrow_{\mathcal N}$, over a set
of names, $\mathcal N$, is the smallest relation satisfying the rules
below.

\infrule[Out-barb]{y \in {\mathcal N}, \; x \nameeq y}
		  {\outputp{x}{v} \downarrow_{\mathcal N} x}
\infrule[Par-barb]{\mbox{$P\downarrow_{\mathcal N} x$ or $Q\downarrow_{\mathcal N} x$}}
		  {\binpar{P}{Q} \downarrow_{\mathcal N} x}

We write $P \Downarrow_{\mathcal N} x$ if there is $Q$ such that 
$P \wred Q$ and $Q \downarrow_{\mathcal N} x$.
\end{definition}

\begin{definition}
%\label{def.bbisim}
An  ${\mathcal N}$-\emph{barbed bisimulation} over a set of names, ${\mathcal N}$, is a symmetric binary relation 
${\mathcal S}_{\mathcal N}$ between agents such that $P\rel{S}_{\mathcal N}Q$ implies:
\begin{enumerate}
\item If $P \red P'$ then $Q \wred Q'$ and $P'\rel{S}_{\mathcal N} Q'$.
\item If $P\downarrow_{\mathcal N} x$, then $Q\Downarrow_{\mathcal N} x$.
\end{enumerate}
$P$ is ${\mathcal N}$-barbed bisimilar to $Q$, written
$P \wbbisim_{\mathcal N} Q$, if $P \rel{S}_{\mathcal N} Q$ for some ${\mathcal N}$-barbed bisimulation ${\mathcal S}_{\mathcal N}$.
\end{definition}

$\mathcal{R} \subseteq \pi \times \pi$

$P \mathcal{R} Q => \forall P'. P \red P' \Rightarrow \exists Q'. Q \red Q', P' \mathcal{R} Q'$

$P \vdash x \Rightarrow Q \vdash x$

\begin{mathpar}
  \inferrule*[lab=Out-barb]{x \nameeq y}{{y}!\langle{Q}\rangle \vdash x}
  \and
  \inferrule*[lab=Par-barb]{\mbox{$P\vdash x$ or $Q\vdash x$}}{\binpar{P}{Q} \vdash x}
\end{mathpar}

\subsubsection{Contexts}

One of the principle advantages of computational calculi like the
$\pi$-calculus is a well-defined notion of context,
contextual-equivalence and a correlation between
contextual-equivalence and notions of bisimulation. The notion of
context allows the decomposition of a process into (sub-)process and
its syntactic environment, its context. Thus, a context may be
thought of as a process with a ``hole'' (written $\Box$) in it. The
application of a context $M$ to a process $P$, written $M[P]$, is
tantamount to filling the hole in $M$ with $P$. In this paper we do
not need the full weight of this theory, but do make use of the notion
of context in the proof the main theorem. 

\begin{mathpar}
  \inferrule* [lab=summation] {} {{M_{M},M_{N}} \bc \Box \;|\; x.M_{A} \;|\; M_{M}+M_{N}}
  \and
  \inferrule* [lab=agent] {} {{M_{A}} \bc (\vec{x})M_{P} \;| \; \clift{P_0,\ldots,M_{P},\ldots,P_N}}
  \and \\
  \inferrule* [lab=process] {} {{M_{P}} \bc M_{N} \;| \;P|M_{P} }
\end{mathpar} 

\begin{mathpar}
  \inferrule* [lab=sychronization] {} {M_{N} \bc \Box \;|\; x?M_{F} \;|\; x!M_{C}}
  \and
  \inferrule* [lab=abstraction] {} {{M_{F}} \bc (x)M_{P} }
  \and
  \inferrule* [lab=concretion] {} {{M_{C}} \bc \langle M_{P} \rangle }
  \and \\
  \inferrule* [lab=process] {} {{M_{P}} \bc M_{N} \;| \;P|M_{P} }
\end{mathpar}

\begin{definition}[contextual application] Given a context $M$, and
  process $P$, we define the \emph{contextual application}, $M[P] :=
  M\{P/\Box\}$. That is, the contextual application of M to P is the
  substitution of $P$ for $\Box$ in $M$.
\end{definition}

$\meaningof{-} : L \to \mathcal{P}(\pi)$

\begin{mathpar}
  \inferrule* [lab=collection] {} {\meaningof{true} = \pi, \and \meaningof{~E} = \pi \setminus \meaningof{E}, \and \meaningof{E_{1} \& E_{2}} = \meaningof{E_{1}} \cap \meaningof{E_{2}}}
\end{mathpar}

\begin{mathpar}
  \inferrule* [lab=structure] {} {\meaningof{0} = \{ P \in \pi | P \equiv 0 \}, \and \\ \meaningof{E_1 | E_2} = \{ P \in \pi | P \equiv P_{1} | P_{2}, P_{1} \in \meaningof{E_{1}}, P_{2} \in \meaningof{E_2}\} }
\end{mathpar}

\begin{mathpar}
 \inferrule* [lab=behavior] {} {\meaningof{\langle a?b \rangle E} = \{ P \in \pi | P \equiv Q | u?(y)P', \\ \and \\\\ \and \\ \;\;\; u \in \meaningof{a}, \forall z.P'\{z/y\} \in \meaningof{E\{z/b\}}\}, \and \\ \meaningof{a!E} = \{ P \in \pi | P \equiv Q | x!\langle P' \rangle, x \in \meaningof{a} P' \in \meaningof{E}\} }
\end{mathpar}

\begin{mathpar}
 \inferrule* [lab=nominal] {} {\meaningof{\quotep{E}} = \{ \quotep{P} \in \quotep{\pi} | P \in \meaningof{E} \}, \and \meaningof{\quotep{P}} = \{ \quotep{Q} \in \quotep{\pi} | P \equiv Q \} \and \\ \meaningof{@\quotep{E}} = \{ P \in \pi | P \equiv @x, x \in \meaningof{E} \}}
\end{mathpar}

\begin{eqnarray*}
  \\
  \meaningof{-} : TS \to ST
\end{eqnarray*}

\begin{eqnarray*}
  \\
  L : TS \to ST
\end{eqnarray*}

\begin{eqnarray*}
  \\
  P \models E \iff P \in \meaningof{E}
\end{eqnarray*}

\begin{eqnarray*}
  P \approx_{L} Q \iff \forall E \in L. P \models E \iff Q \models E
\end{eqnarray*}

\begin{eqnarray*}
  P \approx_{K} Q
\end{eqnarray*}

\begin{eqnarray*}
  P \approx Q
\end{eqnarray*}

$\approx_{K} = \approx = \approx_{L}$

\subsubsection{Contextual duality}

Note that contexts extend the quotation operation to a family of
operations from processes to names. Given a context, $M$, we can
define a \emph{nominal context}, $\quotep{M}$ by $\quotep{M}[P] :=
\quotep{M[P]}$. To foreshadow what is to come we observe that these
operations enjoy a duality with processes very much like the duality
between vectors and maps from vectors to scalars.

Further, because the calculus is essentially higher-order, we have a
correspondence between contexts and processes. More specifically,
given a name $x$ and a context $M$ we can construct $M^{*}_{x}$ such
that 

\begin{mathpar}
  M^{*}_{x} | \lift{x}{P} \red M[P]
\end{mathpar}

namely,

\begin{mathpar}
  M^{*}_{x} := x?(u).M[\dropn{u}]
\end{mathpar}

The dependence of $M^{*}_{x}$ on a name makes it an abstraction, 

\begin{mathpar}
  M^{*} := (x)x?(u).M[\dropn{u}]
\end{mathpar}

\subsection{Additional notation}

It will sometimes be convenient to denote the process a name
quotes. We already have the notation $x = \quotep{P}$, but it will be
convenient to introduce an alternate notation, $\procn{x}$, when we
want to emphasize the connection to the use of the name. Note that, by
virtue of name equivalence, $\quotep{\procn{x}} \nameeq x$; so, the
notation is consistent with previous definitions.

Further, because names have structure it is possible to effect
substitutions on the basis of that structure. This means we need to
upgrade our notation for substitutions, which we accomplish by
adapting comprehension notation. Thus,

\begin{mathpar}
  P\{ y / x : x \in S \}
\end{mathpar}

is interpreted to mean the process derived from P by replacing (in a
capture-avoiding manner) each occurrence of $x$ in $S$ by $y$. For example,

\begin{mathpar}
  P\{ \quotep{\procn{x}|\procn{x}} / x : x \in \freenames{P} \}
\end{mathpar}

will replace each (occurrence) of a free name $x$ in $P$ by
$\quotep{\procn{x}|\procn{x}}$.

Also, we will avail ourselves of the notation $x^{L}$ and $x^{R}$ to
denote injections of a name into disjoint copies of the name
space. There are numerous ways to accomplish this. One example can be
found in \cite{MeredithR05}. This notation overloads to vectors of
names: $\vec{x}^{\pi} := (x_{i}^{\pi} \; : \; 0 \leq i < |\vec{x}| )$ where $\pi \in \{L,R\}$.

We also use $P^{\Box} := P|\Box$.

In \cite{MeredithR05} an interpretation of the new operator is
given. It turns out that there are several possible interpretations
all enjoying the requisite algebraic properties of the operator (see
\cite{milner91polyadicpi}). We will therefore make liberal use of
$(\nu\; \vec{x})P$.

% subsection the_syntax_and_semantics_of_the_notation_system (end)   

\input{qm2pi.qmops} 

\input{qm2pi.sterngerlach} 

\input{qm2pi.metric} 

% section concurrent_process_calculi (end)

%\input{qm2pi.proofsketch}

% section proof sketch (end)

%\input{qm2pi.slviaknots} 

% section spatial logic via knots (end)

\input{qm2pi.conclusion}

% section conclusion (end)

%\input{qm2pi.dtcodes} 

% section wiring algorithm (end)

\input{qm2pi.ack} 

% section acknowledgments (end)

\newpage


\bibliographystyle{plain}   
\bibliography{../../biblios/main.bib}

\input{qm2pi.rhodetails}

\end{document}

 

% subsection basic_interpretation (end)

%\input{qm2pi.rho.presentation} 
\subsection{The syntax and semantics of the notation system}\label{sub:the_syntax_and_semantics_of_the_notation_system} % (fold)

We now summarize a technical presentation of the calculus that
embodies our theory of dynamics. The typical presentation of such a
calculus follows the style of giving generators and relations on
them. The grammar, below, describing term constructors, freely
generates the set of processes, $\Proc$. This set is then quotiented
by a relation known as structural congruence and it is over this set
that the notion of dynamics is expressed. This presentation is
essentially that of \cite{MeredithR05} with the addition of
polyadicity and summation. For readability we have relegated some of
the technical subtleties to an appendix.

\subsubsection{Process grammar}\label{subsub:process_grammar}

\begin{mathpar}
  \inferrule* [lab=synchronization] {} {{M} \bc \pzero \;|\; x?F \;|\; x!C }
  \and
  \inferrule* [lab=abstraction] {} {{F} \bc (x)P}
  \and
  \inferrule* [lab=concretion] {} {{C} \bc \langle Q \rangle}
  \and
  \inferrule* [lab=process] {} {{P,Q} \bc M \;| \;P|Q \;|\; @{x}}
  \and
  \inferrule* [lab=name] {} {{x} \bc \quotep{P}}
\end{mathpar} 

Note that $\vec{x}$ (resp. $\vec{P}$) denotes a vector of names
(resp. processes) of length $|\vec{x}|$ (resp. $|\vec{P}|$). We adopt
the following useful abbreviations.

\begin{mathpar}
   x?(\vec{y}).P := x.(\vec{y})P \and  x\clift{\vec{P}} := x.\clift{\vec{P}}
   \and x!(y) := \lift{x}{\dropn{y}}
   \and \Pi_{i=0}^{n-1}P_i := P_0 | \ldots | P_{n-1}
\end{mathpar}

\subsubsection{Structural congruence}

\paragraph{Free and bound names and alpha-equivalence.} At the
core of structural equivalence is alpha-equivalence which identifies
process that are the same up to a change of variable. Formally, we
recognize the distinction between free and bound names. The free names
of a process, $\freenames{P}$, may be calculated recursively as
follows:

\begin{mathpar}
\freenames{\pzero} := \emptyset
  \and \\
  \freenames{x?(y).P} := \{ x \} \cup (\freenames{P} \setminus \{ y \})
  \and 
  \freenames{x!\langle P \rangle} := \{ x \} \cup \{ P \} 
  \and \\
  \freenames{P|Q} := \freenames{P} \cup \freenames{Q}
  \and \\
  \freenames{@{x}} := \{ x \}
\end{mathpar}

$\pi$
$\quotep{\pi}$

$\freenames{-} : \pi \to \mathcal{P}(\quotep{\pi})$

\begin{eqnarray*}
  \freenames{\pzero} & := & \emptyset \\
  \freenames{x?(y).P} & := & \{ x \} \cup (\freenames{P} \setminus \{ y \}) \\
  \freenames{x!\langle P \rangle} & := & \{ x \} \cup \{ P \} \\
  \freenames{P|Q} & := & \freenames{P} \cup \freenames{Q} \\
  \freenames{\dropn{x}} & := & \{ x \}
\end{eqnarray*}

The bound names of a process, $\boundnames{P}$, are those names occurring in $P$
that are not free. For example, in $x?(y).0$, the name $x$ is free, while $y$ is bound.

\begin{mathpar}
  \inferrule* [lab=monoidal-laws] {} { P|Q \equiv Q|P \and P|0 \equiv P \and P|(Q|R) \equiv (P|Q)|R }
\end{mathpar}

\begin{mathpar}
  \inferrule* [lab=alpha-equivalence] {} { (x)P \equiv (y)P\{y/x\} \and y \not\in \freenames{P} }
\end{mathpar}

\begin{definition}
Then two processes, $P,Q$, are alpha-equivalent if $P = Q\{\vec{y}/\vec{x}\}$ for
some $\vec{x} \in \boundnames{Q},\vec{y} \in \boundnames{P}$, where $Q\{\vec{y}/\vec{x}\}$
denotes the capture-avoiding substitution of $\vec{y}$ for $\vec{x}$ in $Q$.
\end{definition}

\begin{definition}
  The {\em structural congruence} \cite{SangiorgiWalker} , $\equiv$,
  between processes is the least congruence containing
  alpha-equivalence, satisfying the abelian monoid laws
  (associativity, commutativity and $\pzero$ as identity) for parallel
  composition $|$ and for summation $+$.
\end{definition}

\subsection{Name equivalence}

We take name equivalence, written $\nameeq$, to be the smallest
equivalence relation generated by the following rules.

\begin{mathpar}
\inferrule*[lab=Quote-drop]
{ }
{ \quotep{@{x}} \nameeq x }

\inferrule*[lab=Struct-equiv]
{ P \scong Q }
{ \quotep{P} \nameeq \quotep{Q} }
\end{mathpar}

The astute reader will have noticed that the mutual recursion of names
and processes imposes a mutual recursion on alpha-equivalence and
structural equivalence via name-equivalence. Fortunately, all of this
works out pleasantly and we may calculate in the natural way, free of
concern. The reader interested in the details is referred to the
appendix \ref{appendix:rho_details}.

\subsection{Substitution}

We use $\Proc$ for the set of processes, $\QProc$ for the set of
names, and $\id{\{}\vec{y} / \vec{x} \id{\}}$ to denote partial maps,
$s : \QProc \rightarrow \QProc$. A map, $s$ lifts, uniquely, to a map
on process terms, $\widehat{s} : \Proc \rightarrow \Proc$ by the
following equations.

\begin{mathpar}
  (0) \psubstp{Q}{P} := 0 \\
  (R \juxtap S) \psubstp{Q}{P}
  :=    
  (R)\psubstp{Q}{P} \juxtap (S) \psubstp{Q}{P} \\
  (x?(y).R) \psubstp{Q}{P}    
  :=    
  (x)\substp{Q}{P} (z)\concat( (R \psubstn{z}{y}) \psubstp{Q}{P} ) \\
  (\lift{x}{R}) \psubstp{Q}{P}  
  :=
  \lift{(x)\substp{Q}{P}}{ R \psubstp{Q}{P} } \\
%   (\dropn{x})  \psubstp{Q}{P}       
%   := 
%   \left\{ 
%     \begin{array}{ccc} 
%       \dropn{\quotep{Q}} & & x \nameeq \quotep{P} \\
%       \dropn{x} & & otherwise \\
%     \end{array}
%   \right. 
  (\dropn{x})  \psubstp{Q}{P}       
  := 
  \left\{ 
    \begin{array}{ccc} 
      Q & & x \nameeq \quotep{P} \\
      \dropn{x} & & otherwise \\
    \end{array}
  \right.
\end{mathpar}
 

where

\begin{eqnarray}
  (x)\id{\{} \lpquote Q \rpquote / \lpquote P \rpquote \id{\}}            = 
  \left\{ 
    \begin{array}{ccc}
      \lpquote Q \rpquote & & x \nameeq \lpquote P \rpquote \\
      x & & otherwise \\
    \end{array}
  \right. \nonumber
\end{eqnarray}

and $z$ is chosen distinct from $\quotep{P}$, $\quotep{Q}$, the free
names in $Q$, and all the names in $R$. Our $\alpha$-equivalence will
be built in the standard way from this substitution.

\begin{remark}\label{rem:no_self_referential_names}
  One consequence of these definitions is that $\forall P. \quotep{P}
  \not\in \freenames{P}$.
\end{remark}

\subsection{ Dynamic quote: an example }

Anticipating something of what's to come, consider applying the
substitution, $\widehat{\id{\{}u / z \id{\}}}$, to the following pair
of processes, $\lift{w}{y!(z)}$ and $w[ \lpquote y!(z) \rpquote ]$.

\begin{eqnarray}
	\lift{w}{y!(z)}\widehat{\id{\{}u / z \id{\}}}
		& = &
		\lift{w}{y!(u)} \nonumber\\
	w[ \lpquote y!(z) \rpquote ] \widehat{ \id{\{}u / z \id{\}} }
		& = &
		w[ \lpquote y!(z) \rpquote ] \nonumber
\end{eqnarray}

Because the body of the process between quotes is impervious to
substitution, we get radically different answers. In fact, by
examining the first process in an input context,
e.g. $x?(z).\lift{w}{y!(z)}$, we see that the process under the lift
operator may be shaped by prefixed inputs binding a name inside it. In
this sense, the lift operator will be seen as a way to dynamically
construct processes before reifying them as names.

Finally equipped with these standard features we can present the
dynamics of the calculus.

\subsubsection{Operational semantics} 

Finally, we introduce the computational dynamics. What marks these
algebras as distinct from other more traditionally studied algebraic
structures, e.g. vector spaces or polynomial rings, is the manner in
which dynamics is captured. In traditional structures, dynamics is typically
expressed through morphisms between such structures, as in linear maps
between vector spaces or morphisms between rings. In algebras
associated with the semantics of computation, the dynamics is
expressed as part of the algebraic structure itself, through a
reduction reduction relation typically denoted by $\red$. Below, we
give a recursive presentation of this relation for the calculus used
in the encoding.

$\red \subseteq \pi \times \pi$
$\red : \pi \to \mathcal{P}(\pi)$

\begin{mathpar}
  \inferrule* [lab=Comm] { \textsf{match}( x_{src}, x_{trgt} ) } { x_{trgt}?(y)P \; | \; x_{src}!\langle {Q} \rangle \red P\{\quotep{Q}/y}\} }
  \and \\
  \inferrule* [lab=Par] {{P} \red {P}'} {{{P} | {Q}} \red {{P}' | {Q}}}
  \and
  \inferrule* [lab=Equiv]{{{P} \scong {P}'} \andalso {{P}' \red {Q}'} \andalso {{Q}' \scong {Q}}}{{P} \red {Q}}
\end{mathpar}

\begin{eqnarray*}
  match_{\equiv} (\quotep{P},\quotep{Q}) & := & P \equiv Q \\
  match_{\dagger}(\quotep{P},\quotep{Q}) & := & \forall R. P|Q \red^{*} R => R \red^{*} 0 \\
  match_{K}(\quotep{P},\quotep{Q}) & := & K \mbox{ for some context } K
\end{eqnarray*}

$u?(x)P | u!\langle Q \rangle \red P\{\quotep{Q}/x\}$

%We write $\wred$ for $\red^*$, and $P\red$ if $\exists Q $ such that $ P \red Q$.
We write $P\red$ if $\exists Q $ such that $ P \red Q$ and $P\not\red$, otherwise.

\section{Replication}

As mentioned before, it is known that replication (and hence
recursion) can be implemented in a higher-order process algebra
\cite{SangiorgiWalker}. As our first example of calculation with the
machinery thus far presented we give the construction explicitly in
the {\rhoc}.

\begin{eqnarray}
	D_{x} & := & \prefix{x}{y}{(\binpar{\outputp{x}{y}}{@{y}})} \nonumber\\
	\bangp_{x}{P} & := & \binpar{{x}!\langle{\binpar{D_{x}}{P}}\rangle}{D_{x}} \nonumber
\end{eqnarray}

\begin{eqnarray}
	\bangp_{x}{P} & & \nonumber\\
	=
	& {x}!\langle{(\prefix{x}{y}{(\outputp{x}{y} | @{y})) | P}}\rangle 
	      | \prefix{x}{y}{(\outputp{x}{y} | @{y})} & \nonumber\\
	\red
	& (\outputp{x}{y} | @{y})\substn{\quotep{(\prefix{x}{y}{(@{y} | \outputp{x}{y})) | P}}}{y} & \nonumber\\
	=
	& \outputp{x}{\quotep{(\prefix{x}{y}{(\outputp{x}{y} | @{y})) | P}}}
	  | {(\prefix{x}{y}{(\outputp{x}{y} | @{y})) | P}} & \nonumber\\
	\red
	& \ldots & \nonumber\\
	\red^*
	& P | P | \ldots & \nonumber
\end{eqnarray}

Of course, this encoding, as an implementation, runs away, unfolding
$\bangp{P}$ eagerly. A lazier and more implementable replication
operator, restricted to input-guarded processes, may be obtained as follows.

\begin{eqnarray}
\bangp{\prefix{u}{v}{P}} 
	:= 
	\binpar{\lift{x}{\prefix{u}{v}{(\binpar{D(x)}{P})}}}{D(x)} \nonumber
\end{eqnarray}

\begin{remark}
  Note that the lazier definition still does not deal with summation
  or mixed summation (i.e. sums over input and output). The reader is
  invited to construct definitions of replication that deal with these
  features. 

  Further, the definitions are parameterized in a name, $x$. Can you,
  gentle reader, make a definition that eliminates this parameter and
  guarantees no accidental interaction between the replication
  machinery and the process being replicated -- i.e. no accidental
  sharing of names used by the process to get its work done and the
  name(s) used by the replication to effect copying. This latter
  revision of the definition of replication is crucial to obtaining
  the expected identity $!!P \sim !P$.
\end{remark}

\begin{remark}\label{rem:paradoxical_combinator}
  The reader familiar with the lambda calculus will have noticed the
  similarity between $D$ and the paradoxical combinator.

  [Ed. note: the existence of this seems to suggest we have to be more
  restrictive on the set of processes and names we admit if we are to
  support no-cloning.]
\end{remark}

\subsubsection{Bisimulation}

The computational dynamics gives rise to another kind of equivalence,
the equivalence of computational behavior. As previously mentioned
this is typically captured \emph{via} some form of bisimulation.

% The notion we use in this paper is weak barbed bisimulation
% \cite{milner91polyadicpi}.

The notion we use in this paper is derived from weak barbed
bisimulation \cite{milner91polyadicpi}. 

\begin{definition}
An \emph{observation relation}, $\downarrow_{\mathcal N}$, over a set
of names, $\mathcal N$, is the smallest relation satisfying the rules
below.

\infrule[Out-barb]{y \in {\mathcal N}, \; x \nameeq y}
		  {\outputp{x}{v} \downarrow_{\mathcal N} x}
\infrule[Par-barb]{\mbox{$P\downarrow_{\mathcal N} x$ or $Q\downarrow_{\mathcal N} x$}}
		  {\binpar{P}{Q} \downarrow_{\mathcal N} x}

We write $P \Downarrow_{\mathcal N} x$ if there is $Q$ such that 
$P \wred Q$ and $Q \downarrow_{\mathcal N} x$.
\end{definition}

\begin{definition}
%\label{def.bbisim}
An  ${\mathcal N}$-\emph{barbed bisimulation} over a set of names, ${\mathcal N}$, is a symmetric binary relation 
${\mathcal S}_{\mathcal N}$ between agents such that $P\rel{S}_{\mathcal N}Q$ implies:
\begin{enumerate}
\item If $P \red P'$ then $Q \wred Q'$ and $P'\rel{S}_{\mathcal N} Q'$.
\item If $P\downarrow_{\mathcal N} x$, then $Q\Downarrow_{\mathcal N} x$.
\end{enumerate}
$P$ is ${\mathcal N}$-barbed bisimilar to $Q$, written
$P \wbbisim_{\mathcal N} Q$, if $P \rel{S}_{\mathcal N} Q$ for some ${\mathcal N}$-barbed bisimulation ${\mathcal S}_{\mathcal N}$.
\end{definition}

$\mathcal{R} \subseteq \pi \times \pi$

$P \mathcal{R} Q => \forall P'. P \red P' \Rightarrow \exists Q'. Q \red Q', P' \mathcal{R} Q'$

$P \vdash x \Rightarrow Q \vdash x$

\begin{mathpar}
  \inferrule*[lab=Out-barb]{x \nameeq y}{{y}!\langle{Q}\rangle \vdash x}
  \and
  \inferrule*[lab=Par-barb]{\mbox{$P\vdash x$ or $Q\vdash x$}}{\binpar{P}{Q} \vdash x}
\end{mathpar}

\subsubsection{Contexts}

One of the principle advantages of computational calculi like the
$\pi$-calculus is a well-defined notion of context,
contextual-equivalence and a correlation between
contextual-equivalence and notions of bisimulation. The notion of
context allows the decomposition of a process into (sub-)process and
its syntactic environment, its context. Thus, a context may be
thought of as a process with a ``hole'' (written $\Box$) in it. The
application of a context $M$ to a process $P$, written $M[P]$, is
tantamount to filling the hole in $M$ with $P$. In this paper we do
not need the full weight of this theory, but do make use of the notion
of context in the proof the main theorem. 

\begin{mathpar}
  \inferrule* [lab=summation] {} {{M_{M},M_{N}} \bc \Box \;|\; x.M_{A} \;|\; M_{M}+M_{N}}
  \and
  \inferrule* [lab=agent] {} {{M_{A}} \bc (\vec{x})M_{P} \;| \; \clift{P_0,\ldots,M_{P},\ldots,P_N}}
  \and \\
  \inferrule* [lab=process] {} {{M_{P}} \bc M_{N} \;| \;P|M_{P} }
\end{mathpar} 

\begin{mathpar}
  \inferrule* [lab=sychronization] {} {M_{N} \bc \Box \;|\; x?M_{F} \;|\; x!M_{C}}
  \and
  \inferrule* [lab=abstraction] {} {{M_{F}} \bc (x)M_{P} }
  \and
  \inferrule* [lab=concretion] {} {{M_{C}} \bc \langle M_{P} \rangle }
  \and \\
  \inferrule* [lab=process] {} {{M_{P}} \bc M_{N} \;| \;P|M_{P} }
\end{mathpar}

\begin{definition}[contextual application] Given a context $M$, and
  process $P$, we define the \emph{contextual application}, $M[P] :=
  M\{P/\Box\}$. That is, the contextual application of M to P is the
  substitution of $P$ for $\Box$ in $M$.
\end{definition}

$\meaningof{-} : L \to \mathcal{P}(\pi)$

\begin{mathpar}
  \inferrule* [lab=collection] {} {\meaningof{true} = \pi, \and \meaningof{~E} = \pi \setminus \meaningof{E}, \and \meaningof{E_{1} \& E_{2}} = \meaningof{E_{1}} \cap \meaningof{E_{2}}}
\end{mathpar}

\begin{mathpar}
  \inferrule* [lab=structure] {} {\meaningof{0} = \{ P \in \pi | P \equiv 0 \}, \and \\ \meaningof{E_1 | E_2} = \{ P \in \pi | P \equiv P_{1} | P_{2}, P_{1} \in \meaningof{E_{1}}, P_{2} \in \meaningof{E_2}\} }
\end{mathpar}

\begin{mathpar}
 \inferrule* [lab=behavior] {} {\meaningof{\langle a?b \rangle E} = \{ P \in \pi | P \equiv Q | u?(y)P', \\ \and \\\\ \and \\ \;\;\; u \in \meaningof{a}, \forall z.P'\{z/y\} \in \meaningof{E\{z/b\}}\}, \and \\ \meaningof{a!E} = \{ P \in \pi | P \equiv Q | x!\langle P' \rangle, x \in \meaningof{a} P' \in \meaningof{E}\} }
\end{mathpar}

\begin{mathpar}
 \inferrule* [lab=nominal] {} {\meaningof{\quotep{E}} = \{ \quotep{P} \in \quotep{\pi} | P \in \meaningof{E} \}, \and \meaningof{\quotep{P}} = \{ \quotep{Q} \in \quotep{\pi} | P \equiv Q \} \and \\ \meaningof{@\quotep{E}} = \{ P \in \pi | P \equiv @x, x \in \meaningof{E} \}}
\end{mathpar}

\begin{eqnarray*}
  \\
  \meaningof{-} : TS \to ST
\end{eqnarray*}

\begin{eqnarray*}
  \\
  L : TS \to ST
\end{eqnarray*}

\begin{eqnarray*}
  \\
  P \models E \iff P \in \meaningof{E}
\end{eqnarray*}

\begin{eqnarray*}
  P \approx_{L} Q \iff \forall E \in L. P \models E \iff Q \models E
\end{eqnarray*}

\begin{eqnarray*}
  P \approx_{K} Q
\end{eqnarray*}

\begin{eqnarray*}
  P \approx Q
\end{eqnarray*}

$\approx_{K} = \approx = \approx_{L}$

\subsubsection{Contextual duality}

Note that contexts extend the quotation operation to a family of
operations from processes to names. Given a context, $M$, we can
define a \emph{nominal context}, $\quotep{M}$ by $\quotep{M}[P] :=
\quotep{M[P]}$. To foreshadow what is to come we observe that these
operations enjoy a duality with processes very much like the duality
between vectors and maps from vectors to scalars.

Further, because the calculus is essentially higher-order, we have a
correspondence between contexts and processes. More specifically,
given a name $x$ and a context $M$ we can construct $M^{*}_{x}$ such
that 

\begin{mathpar}
  M^{*}_{x} | \lift{x}{P} \red M[P]
\end{mathpar}

namely,

\begin{mathpar}
  M^{*}_{x} := x?(u).M[\dropn{u}]
\end{mathpar}

The dependence of $M^{*}_{x}$ on a name makes it an abstraction, 

\begin{mathpar}
  M^{*} := (x)x?(u).M[\dropn{u}]
\end{mathpar}

\subsection{Additional notation}

It will sometimes be convenient to denote the process a name
quotes. We already have the notation $x = \quotep{P}$, but it will be
convenient to introduce an alternate notation, $\procn{x}$, when we
want to emphasize the connection to the use of the name. Note that, by
virtue of name equivalence, $\quotep{\procn{x}} \nameeq x$; so, the
notation is consistent with previous definitions.

Further, because names have structure it is possible to effect
substitutions on the basis of that structure. This means we need to
upgrade our notation for substitutions, which we accomplish by
adapting comprehension notation. Thus,

\begin{mathpar}
  P\{ y / x : x \in S \}
\end{mathpar}

is interpreted to mean the process derived from P by replacing (in a
capture-avoiding manner) each occurrence of $x$ in $S$ by $y$. For example,

\begin{mathpar}
  P\{ \quotep{\procn{x}|\procn{x}} / x : x \in \freenames{P} \}
\end{mathpar}

will replace each (occurrence) of a free name $x$ in $P$ by
$\quotep{\procn{x}|\procn{x}}$.

Also, we will avail ourselves of the notation $x^{L}$ and $x^{R}$ to
denote injections of a name into disjoint copies of the name
space. There are numerous ways to accomplish this. One example can be
found in \cite{MeredithR05}. This notation overloads to vectors of
names: $\vec{x}^{\pi} := (x_{i}^{\pi} \; : \; 0 \leq i < |\vec{x}| )$ where $\pi \in \{L,R\}$.

We also use $P^{\Box} := P|\Box$.

In \cite{MeredithR05} an interpretation of the new operator is
given. It turns out that there are several possible interpretations
all enjoying the requisite algebraic properties of the operator (see
\cite{milner91polyadicpi}). We will therefore make liberal use of
$(\nu\; \vec{x})P$.

% subsection the_syntax_and_semantics_of_the_notation_system (end)   

\section{Interpretation of QM}
\subsection{Supporting definitions}
\subsubsection{Multiplication}
\begin{mathpar}
  \quotep{Q} \cdot \quotep{R} := \quotep{Q|R}
  \and \\
  \quotep{Q} \cdot P := P\{ \quotep{Q|R} / \quotep{R} : \quotep{R} \in \freenames{P} \}
\end{mathpar}

\paragraph{Discussion}
The first line needs little explanation. The second line says that
each free name of the process is replaced with the multiplication of
that name by the scalar. Multiplication of a scalar (name) by a state
(process) results in a process all the names of which have been `moved
over' by parallel composition with the process the scalar
quotes. There is a subtlety that the bound names have to be
manipulated so that multiplied names aren't accidentally
captured. There are many ways to achieve this.

\begin{remark}\label{rem:multiplication_identities}
  The reader is invited to verify that for all $x,y,z \in \QProc$ and $P \in \Proc$
  \begin{mathpar}
    x \cdot \quotep{0} \equiv x 
    \and
    x \cdot y \equiv y \cdot x
    \and
    x \cdot (y \cdot z) \equiv (x \cdot y) \cdot z
    \and \\
    \quotep{0} \cdot P \equiv P
    \and \\
    x \cdot (y \cdot P) \equiv (x \cdot y) \cdot P
    \and \\
    x \cdot (P|Q) \equiv (x \cdot P) | (x \cdot Q)
    \and \\    
  \end{mathpar}
\end{remark}

\subsubsection{Tensor product}

We define a tensor product on processes by structural induction.

\paragraph{Tensor of sums} First note that all summations, including
$\pzero$ and sequence, can be written $\Sigma_{i} x_{i}.A_{i} +
\Sigma_{j} x_{j}.C_{j}$, where we have grouped input-guarded processes
together and output-guarded processes together.

Thus, we can define the tensor product of two summations, $N_{1}\otimes N_{2}$, where

\begin{mathpar}
  N_{1} := \Sigma_{i} x_{i}.A_{i} + \Sigma_{j} x_{j}.C_{j}
  \and
  N_{2} := \Sigma_{i'} y_{i'}.B_{i'} + \Sigma_{j'} y_{j'}.D_{j'} 
\end{mathpar}

as follows.

\begin{mathpar}
  \Sigma_{i} x_{i}.A_{i} + \Sigma_{j} x_{j}.C_{j} \otimes \Sigma_{i'}
  y_{i'}.B_{i'} + \Sigma_{j'} y_{j'}.D_{j'} 
  \and \\
  := \; \Sigma_{i} \Sigma_{i'} \quotep{\stackrel{\vee}{x_{i}}| \stackrel{\vee}{y_{i'}}}.(A_{i}\otimes B_{i'}) \; | \; \Sigma_{i'} \Sigma_{i} \quotep{\stackrel{\vee}{y_{i'}}|\stackrel{\vee}{x_{i}}}.(B_{i'}\otimes A_{i})
  \and
  \;\; | \;\; \Sigma_{j} \Sigma_{j'} \quotep{\stackrel{\vee}{x_{j}}|\stackrel{\vee}{y_{j'}}}.(A_{j}\otimes B_{j'}) \; | \; \Sigma_{j'} \Sigma_{j} \quotep{\stackrel{\vee}{y_{j'}}|\stackrel{\vee}{x_{j}}}.(B_{j'}\otimes A_{j})
\end{mathpar}

\begin{remark}
  Do we need to $x^{L}$ and $y^{R}$ for this construction as well?
\end{remark}

\paragraph{Tensor of parallel compositions} Next, we distribute tensor
over par.

\begin{mathpar}
  P_{1}|P_{2} \otimes Q_{1}|Q_{2} := (P_{1} \otimes Q_{1}) | (P_{1}
  \otimes Q_{2}) | (P_{2} \otimes Q_{1}) | (P_{2} \otimes Q_{2})
\end{mathpar}

\paragraph{Tensor with dropped names} We treat tensor of a
process with a dropped name as parallel composition.

\begin{mathpar}
  P \otimes \dropn{x} := P | \dropn{x}
\end{mathpar}

\paragraph{Tensor of agents}

Finally, we need to define tensor on agents. Note that the definition
of tensor on normal products only tensors inputs with inputs and
outputs with outputs. Thus, we only have to define the operation on
``homogeneous'' pairings.

\begin{mathpar}
  (\vec{x})P \otimes (\vec{y})Q
  \and \\
  := (x_{0}^{L}|y_{0}^{R},\ldots,x_{0}^{L}|y_{n}^{R},\ldots,x_{m}^{L}|y_{0}^{R},\ldots,x_{m}^{L}|y_{n}^R)(P\{ \vec{x}^{L}/\vec{x}\} \otimes Q \{ \vec{y}^{R}/\vec{y}\})
  \and \\
  \clift{\vec{P}} \otimes \clift{\vec{Q}}
  \and \\
  := \clift{P_{0}\otimes Q_{0},\ldots,P_{0}\otimes Q_{n},\ldots,P_{m}\otimes Q_{0},\ldots,P_{m}\otimes Q_{n}}
\end{mathpar}

\begin{remark}
  Observe that arities of tensored abstractions matches arities of
  tensored concretions if the original arities matched. Note also that
  the length of the arities corresponds to the increase in dimension
  we see in ordinary vector space tensor product.
\end{remark}

\begin{remark}
  Operationally, this definition distributes the tensor down to
  components ``linked'' by summation. Tensor over summation is
  intriguing in that it mixes names. Moreover, as a consequence of the
  way it mixes names we have the identities for all $x \in \QProc$ and
  $P,Q \in \Proc$

  \begin{mathpar}
    (x \cdot P) \otimes Q \equiv x \cdot (P \otimes Q) \equiv P \otimes (x \cdot Q)
    \and
    P \otimes \pzero \equiv P
  \end{mathpar}

  that the reader is invited to verify.
\end{remark}

\subsubsection{Annihilation}
\begin{mathpar}
  P^{\perp} := \{ Q | \forall R. P|Q \red^{*} R \Rightarrow R \red^{*} \pzero \}
  \and \\
  P^{\underline{\perp}} := \Sigma_{Q \in P^{\perp}} \quotep{Q}?(y).(\dropn{y}|Q) | \Sigma_{Q \in P^{\perp}} \quotep{Q}\clift{\Box}
\end{mathpar}

\paragraph{Discussion} The reader will note that $P^{\perp}$ is a
\emph{set} of processes, while $P^{\underline{\perp}}$ is a
\emph{context}. We call the set $P^{\perp}$ the \emph{annihilators} of
$P$. The parallel composition of a process in the annihilators of $P$
with $P$ will result in a process, the state space of which has all
paths eventually leading to $\pzero$. Execution may endure loops; but
under reasonable conditions of fairness (naturally guaranteed under
most notions of bisimulation) such a composite process cannot get
stuck in such a loop and will, eventually pop out and terminate.

The context $P^{\underline{\perp}}$ is ready and willing to ``take the
$P$ out of'' the process to which it is applied. It will effectively
transmit the code of the process to which it is applied to one of the
annihilators and run the process against it.

\subsubsection{Evaluation}
We fix $M$ a domain of fully abstract interpretation with an equality
coincident with bisimulation. We take $\meaningof{\cdot} : \Proc \to
M$ to be the map interpreting processes and $\nmeaningof{\cdot} : \M
\to Proc$ to be the map running the other way. Then we define

\begin{mathpar}
  \int P := \nmeaningof{\meaningof{P}}
\end{mathpar}

\paragraph{Discussion}
There are many fully abstract interpretations of Milner's
$\pi$-calculus. Any of them can be used as a basis for interpreting
the reflective calculus here. Equipped with such a domain it is
largely a matter of grinding through to check that the Yoneda
construction for the normalization-by-evaluation program can be
extended to this setting.

\begin{remark}
  The reader is invited to verify that $\int (P^{\underline{\perp}}[P]) = 0$.
\end{remark}

\subsection{Quantum mechanics}

Table \ref{tbl:core_qm_op_defns} gives the core operational definitions

\begin{table}[htp]\label{tbl:core_qm_op_defns}
  \center{
    \fbox{
      \begin{tabular}{c|c}
        quantum mechanics & process calculus \\
        \hline
        scalar & $x := \quotep{P}$ \\
        state vector & $\state{P} := P$ \\
        dual & $\state{P}^{*} := \event{P^{\underline{\perp}}} := \quotep{P^{\underline{\perp}}}[-]$ \\
        matrix & $ \Sigma_{\alpha} \state{P_{\alpha}}x_{\alpha}\event{Q_{\alpha}}$ \\
        vector addition & $\state{P} + \state{Q} := \state{P | Q}$ \\
        tensor product & $\state{P} \otimes \state{Q} := \state{P \otimes Q}$ \\
        inner product & $\innerprod{P}{Q} := \quotep{\int P^{\underline{\perp}}[Q]}$ \\
      \end{tabular}
    }
  }
  \caption{QM - operational definitions}
\end{table}

where

\begin{mathpar}
  \prmatrix{P}{Q} := \fprmatrix{P}{\quotep{\pzero}}{Q}
  \and
  \fprmatrix{P}{x}{Q} := (\state{P},x,\event{Q})
  \and
  (\fprmatrix{P}{x}{Q})(\state{R}) := x \cdot \innerprod{Q}{R} \cdot \state{P}
  \and
  (\fprmatrix{P}{x}{Q})(\event{R}) := x \cdot \innerprod{R}{P} \cdot \event{Q}
\end{mathpar}

\paragraph{Discussion}
As promised: vectors (aka states) are represented as processes; duals
as contextual duals; inner product definition should be compared with
standard inner product definition for ....

\begin{remark}
  Assuming $\int (P^{\underline{\perp}}[P]) = 0$, the reader is
  invited to verify that $(\fprmatrix{P}{x}{P})(\state{P}) = x \cdot \state{P}$.
\end{remark}

\begin{remark}
  The reader is invited to verify that $\innerprod{P}{Q}$ could
  equally well have been written $\quotep{\int \stackrel{\vee}{x}}$
  where $x = \event{P^{\underline{\perp}}}(Q)$.

  One of the motivations for this remark is that there is another way
  to factor these operations. We could package up evaluation in the dual:

  \begin{mathpar}
    \state{P}^{*} := \event{\int P^{\underline{\perp}}} := \quotep{\int P^{\underline{\perp}}}[-]
  \end{mathpar}

  and then have inner product defined by
  
  \begin{mathpar}
    \innerprod{P}{Q} := \event{P}(Q)
  \end{mathpar}

  Hopefully, experience with the calculations will provide guidance on
  the best factoring.
\end{remark}

\begin{remark}
  Assuming $\int (P^{\underline{\perp}}[P]) = 0$, the reader is
  invited to verify that $\forall P,Q. (\prmatrix{0}{Q})(\state{0}) =
  \state{0}$ and dually $(\prmatrix{P}{0})(\event{0}) = \event{0}$.
\end{remark}

\begin{remark}
  i'm a little worried that i don't (yet) have proper support for
  complex conjugacy. But, the observation above may give us a
  clue. According to Abramsky, it must be the case that the scalars
  are iso to the homset of the identity for the tensor -- which the
  observation above characterizes. 

  For now, we will simply bookmark the notion with $\overline{x}$.
\end{remark}

\subsubsection{Adjointness}

We need to give a definition of $(\cdot)^{\dagger}$ for matrices. The
obvious candidate definition is
\begin{mathpar}
(\Sigma_{\alpha}\fprmatrix{P_{\alpha}}{x_{\alpha}}{Q_{\alpha}})^{\dagger}
= \Sigma_{\alpha}\fprmatrix{(Q_{\alpha}^{\underline{\perp}})^{*}}{\overline{x}_{\alpha}}{P_{\alpha}^{\underline{\perp}}} 
\end{mathpar}

But, $(Q_{\alpha}^{\underline{\perp}})^{*}$ requires a name along
which to communicate the process to achieve the context application.

\subsubsection{Basis for a basis}
If processes label states and ``addition'' of states (a.k.a. vector
addition) is interpreted as parallel composition, what corresponds to
notions of linear independence and basis? Here, we recall that Yoshida
has developed a set of \emph{combinators} for an asynchronous verison
of Milner's $\pi$-calculus. These are a finite set of processes such
any process can be expressed as parallel composition of these
combinators together with liberal uses of the new operator and
replication. We can simply give a translation of these into the
present calculus and have reasonable expectation that the property
carries over. That is, that the resultant set allows to express all
processes via parallel composition. Note, however, that there is no
new operator or replication in this calculus. As a result, we expect
that the corresponding set is actually infinite. That is, we expect
that the space is actually infinite dimensional.

\begin{remark}
  The attentive reader may be a bit concerned. Certainly, the
  collection $S$, $K$ and $I$ is a finite set of
  combinators. Shouldn't we expect to see a finite set of combinators
  for an effectively equivalent system? i am very sympathetic to this
  critique and feel it warrants full attention. On the other hand, i
  also have in mind the following analogy. The natural numbers, as a
  monoid under addition, has exactly $1$ generator, while the natural
  numbers, as a monoid under multiplication, has countably many
  generators (the primes). We observe that the application of the
  lambda calculus is much less resource sensitive than the parallel
  composition of the $\pi$-calculus. Could it be the case that we have
  an analogy of the form
  
  \begin{mathpar}
    m + n : MN :: m*n : M|N
  \end{mathpar}

  giving a similar blow up in the set of ``primes''?  This is such a
  wonderful thought that, even if it's not true, i think it's worth
  writing down.
\end{remark}
 

\documentclass[12pt]{llncs}
%\documentclass{jktr}

\usepackage[pdftex]{hyperref}                   
\usepackage {listings}
\usepackage {mathpartir}
\usepackage{bcprules}
%\usepackage{listings}
                       
\usepackage{graphicx} 
%\usepackage[margins=2.5cm,nohead,nofoot]{geometry}
%\usepackage{geometry}
\usepackage{amsfonts}
\usepackage{amstext}
\usepackage{latexsym}
\usepackage{amssymb}
\usepackage{color}


%\include{myPreamble}
\include{qm2pi.local} 

%\ifpdf
%\usepackage[pdftex]{graphicx}
%\else
%\usepackage{graphicx}
%\fi

 % \ifpdf
%  \usepackage{pdfsync}
%  \if


%\title{Brief Article}
%\author{David F. Snyder}
%\author{L.G. Meredith}

%\address{Dept. of Math., Texas State University--San Marcos, San Marcos, TX 78666}
       
\pagestyle{empty}


\begin{document}

\lstset{language=[Objective]Caml,frame=shadowbox}

\input{qm2pi.front}

% section front matter (end)

\input{qm2pi.intro} 
 
% section introduction (end)

% \input{qm2pi.knotations} 

% section notation (end)

\input{qm2pi.process.calculi} 

% section concurrent_process_calculi_and_spatial_logics_ (end)
    
%\input{qm2pi.knots2pi} 

%\input{qm2pi.trefoil} 

%\input{qm2pi.mainthm} 

% subsection basic_interpretation (end)

%\input{qm2pi.rho.presentation} 
\subsection{The syntax and semantics of the notation system}\label{sub:the_syntax_and_semantics_of_the_notation_system} % (fold)

We now summarize a technical presentation of the calculus that
embodies our theory of dynamics. The typical presentation of such a
calculus follows the style of giving generators and relations on
them. The grammar, below, describing term constructors, freely
generates the set of processes, $\Proc$. This set is then quotiented
by a relation known as structural congruence and it is over this set
that the notion of dynamics is expressed. This presentation is
essentially that of \cite{MeredithR05} with the addition of
polyadicity and summation. For readability we have relegated some of
the technical subtleties to an appendix.

\subsubsection{Process grammar}\label{subsub:process_grammar}

\begin{mathpar}
  \inferrule* [lab=synchronization] {} {{M} \bc \pzero \;|\; x?F \;|\; x!C }
  \and
  \inferrule* [lab=abstraction] {} {{F} \bc (x)P}
  \and
  \inferrule* [lab=concretion] {} {{C} \bc \langle Q \rangle}
  \and
  \inferrule* [lab=process] {} {{P,Q} \bc M \;| \;P|Q \;|\; @{x}}
  \and
  \inferrule* [lab=name] {} {{x} \bc \quotep{P}}
\end{mathpar} 

Note that $\vec{x}$ (resp. $\vec{P}$) denotes a vector of names
(resp. processes) of length $|\vec{x}|$ (resp. $|\vec{P}|$). We adopt
the following useful abbreviations.

\begin{mathpar}
   x?(\vec{y}).P := x.(\vec{y})P \and  x\clift{\vec{P}} := x.\clift{\vec{P}}
   \and x!(y) := \lift{x}{\dropn{y}}
   \and \Pi_{i=0}^{n-1}P_i := P_0 | \ldots | P_{n-1}
\end{mathpar}

\subsubsection{Structural congruence}

\paragraph{Free and bound names and alpha-equivalence.} At the
core of structural equivalence is alpha-equivalence which identifies
process that are the same up to a change of variable. Formally, we
recognize the distinction between free and bound names. The free names
of a process, $\freenames{P}$, may be calculated recursively as
follows:

\begin{mathpar}
\freenames{\pzero} := \emptyset
  \and \\
  \freenames{x?(y).P} := \{ x \} \cup (\freenames{P} \setminus \{ y \})
  \and 
  \freenames{x!\langle P \rangle} := \{ x \} \cup \{ P \} 
  \and \\
  \freenames{P|Q} := \freenames{P} \cup \freenames{Q}
  \and \\
  \freenames{@{x}} := \{ x \}
\end{mathpar}

$\pi$
$\quotep{\pi}$

$\freenames{-} : \pi \to \mathcal{P}(\quotep{\pi})$

\begin{eqnarray*}
  \freenames{\pzero} & := & \emptyset \\
  \freenames{x?(y).P} & := & \{ x \} \cup (\freenames{P} \setminus \{ y \}) \\
  \freenames{x!\langle P \rangle} & := & \{ x \} \cup \{ P \} \\
  \freenames{P|Q} & := & \freenames{P} \cup \freenames{Q} \\
  \freenames{\dropn{x}} & := & \{ x \}
\end{eqnarray*}

The bound names of a process, $\boundnames{P}$, are those names occurring in $P$
that are not free. For example, in $x?(y).0$, the name $x$ is free, while $y$ is bound.

\begin{mathpar}
  \inferrule* [lab=monoidal-laws] {} { P|Q \equiv Q|P \and P|0 \equiv P \and P|(Q|R) \equiv (P|Q)|R }
\end{mathpar}

\begin{mathpar}
  \inferrule* [lab=alpha-equivalence] {} { (x)P \equiv (y)P\{y/x\} \and y \not\in \freenames{P} }
\end{mathpar}

\begin{definition}
Then two processes, $P,Q$, are alpha-equivalent if $P = Q\{\vec{y}/\vec{x}\}$ for
some $\vec{x} \in \boundnames{Q},\vec{y} \in \boundnames{P}$, where $Q\{\vec{y}/\vec{x}\}$
denotes the capture-avoiding substitution of $\vec{y}$ for $\vec{x}$ in $Q$.
\end{definition}

\begin{definition}
  The {\em structural congruence} \cite{SangiorgiWalker} , $\equiv$,
  between processes is the least congruence containing
  alpha-equivalence, satisfying the abelian monoid laws
  (associativity, commutativity and $\pzero$ as identity) for parallel
  composition $|$ and for summation $+$.
\end{definition}

\subsection{Name equivalence}

We take name equivalence, written $\nameeq$, to be the smallest
equivalence relation generated by the following rules.

\begin{mathpar}
\inferrule*[lab=Quote-drop]
{ }
{ \quotep{@{x}} \nameeq x }

\inferrule*[lab=Struct-equiv]
{ P \scong Q }
{ \quotep{P} \nameeq \quotep{Q} }
\end{mathpar}

The astute reader will have noticed that the mutual recursion of names
and processes imposes a mutual recursion on alpha-equivalence and
structural equivalence via name-equivalence. Fortunately, all of this
works out pleasantly and we may calculate in the natural way, free of
concern. The reader interested in the details is referred to the
appendix \ref{appendix:rho_details}.

\subsection{Substitution}

We use $\Proc$ for the set of processes, $\QProc$ for the set of
names, and $\id{\{}\vec{y} / \vec{x} \id{\}}$ to denote partial maps,
$s : \QProc \rightarrow \QProc$. A map, $s$ lifts, uniquely, to a map
on process terms, $\widehat{s} : \Proc \rightarrow \Proc$ by the
following equations.

\begin{mathpar}
  (0) \psubstp{Q}{P} := 0 \\
  (R \juxtap S) \psubstp{Q}{P}
  :=    
  (R)\psubstp{Q}{P} \juxtap (S) \psubstp{Q}{P} \\
  (x?(y).R) \psubstp{Q}{P}    
  :=    
  (x)\substp{Q}{P} (z)\concat( (R \psubstn{z}{y}) \psubstp{Q}{P} ) \\
  (\lift{x}{R}) \psubstp{Q}{P}  
  :=
  \lift{(x)\substp{Q}{P}}{ R \psubstp{Q}{P} } \\
%   (\dropn{x})  \psubstp{Q}{P}       
%   := 
%   \left\{ 
%     \begin{array}{ccc} 
%       \dropn{\quotep{Q}} & & x \nameeq \quotep{P} \\
%       \dropn{x} & & otherwise \\
%     \end{array}
%   \right. 
  (\dropn{x})  \psubstp{Q}{P}       
  := 
  \left\{ 
    \begin{array}{ccc} 
      Q & & x \nameeq \quotep{P} \\
      \dropn{x} & & otherwise \\
    \end{array}
  \right.
\end{mathpar}
 

where

\begin{eqnarray}
  (x)\id{\{} \lpquote Q \rpquote / \lpquote P \rpquote \id{\}}            = 
  \left\{ 
    \begin{array}{ccc}
      \lpquote Q \rpquote & & x \nameeq \lpquote P \rpquote \\
      x & & otherwise \\
    \end{array}
  \right. \nonumber
\end{eqnarray}

and $z$ is chosen distinct from $\quotep{P}$, $\quotep{Q}$, the free
names in $Q$, and all the names in $R$. Our $\alpha$-equivalence will
be built in the standard way from this substitution.

\begin{remark}\label{rem:no_self_referential_names}
  One consequence of these definitions is that $\forall P. \quotep{P}
  \not\in \freenames{P}$.
\end{remark}

\subsection{ Dynamic quote: an example }

Anticipating something of what's to come, consider applying the
substitution, $\widehat{\id{\{}u / z \id{\}}}$, to the following pair
of processes, $\lift{w}{y!(z)}$ and $w[ \lpquote y!(z) \rpquote ]$.

\begin{eqnarray}
	\lift{w}{y!(z)}\widehat{\id{\{}u / z \id{\}}}
		& = &
		\lift{w}{y!(u)} \nonumber\\
	w[ \lpquote y!(z) \rpquote ] \widehat{ \id{\{}u / z \id{\}} }
		& = &
		w[ \lpquote y!(z) \rpquote ] \nonumber
\end{eqnarray}

Because the body of the process between quotes is impervious to
substitution, we get radically different answers. In fact, by
examining the first process in an input context,
e.g. $x?(z).\lift{w}{y!(z)}$, we see that the process under the lift
operator may be shaped by prefixed inputs binding a name inside it. In
this sense, the lift operator will be seen as a way to dynamically
construct processes before reifying them as names.

Finally equipped with these standard features we can present the
dynamics of the calculus.

\subsubsection{Operational semantics} 

Finally, we introduce the computational dynamics. What marks these
algebras as distinct from other more traditionally studied algebraic
structures, e.g. vector spaces or polynomial rings, is the manner in
which dynamics is captured. In traditional structures, dynamics is typically
expressed through morphisms between such structures, as in linear maps
between vector spaces or morphisms between rings. In algebras
associated with the semantics of computation, the dynamics is
expressed as part of the algebraic structure itself, through a
reduction reduction relation typically denoted by $\red$. Below, we
give a recursive presentation of this relation for the calculus used
in the encoding.

$\red \subseteq \pi \times \pi$
$\red : \pi \to \mathcal{P}(\pi)$

\begin{mathpar}
  \inferrule* [lab=Comm] { \textsf{match}( x_{src}, x_{trgt} ) } { x_{trgt}?(y)P \; | \; x_{src}!\langle {Q} \rangle \red P\{\quotep{Q}/y}\} }
  \and \\
  \inferrule* [lab=Par] {{P} \red {P}'} {{{P} | {Q}} \red {{P}' | {Q}}}
  \and
  \inferrule* [lab=Equiv]{{{P} \scong {P}'} \andalso {{P}' \red {Q}'} \andalso {{Q}' \scong {Q}}}{{P} \red {Q}}
\end{mathpar}

\begin{eqnarray*}
  match_{\equiv} (\quotep{P},\quotep{Q}) & := & P \equiv Q \\
  match_{\dagger}(\quotep{P},\quotep{Q}) & := & \forall R. P|Q \red^{*} R => R \red^{*} 0 \\
  match_{K}(\quotep{P},\quotep{Q}) & := & K \mbox{ for some context } K
\end{eqnarray*}

$u?(x)P | u!\langle Q \rangle \red P\{\quotep{Q}/x\}$

%We write $\wred$ for $\red^*$, and $P\red$ if $\exists Q $ such that $ P \red Q$.
We write $P\red$ if $\exists Q $ such that $ P \red Q$ and $P\not\red$, otherwise.

\section{Replication}

As mentioned before, it is known that replication (and hence
recursion) can be implemented in a higher-order process algebra
\cite{SangiorgiWalker}. As our first example of calculation with the
machinery thus far presented we give the construction explicitly in
the {\rhoc}.

\begin{eqnarray}
	D_{x} & := & \prefix{x}{y}{(\binpar{\outputp{x}{y}}{@{y}})} \nonumber\\
	\bangp_{x}{P} & := & \binpar{{x}!\langle{\binpar{D_{x}}{P}}\rangle}{D_{x}} \nonumber
\end{eqnarray}

\begin{eqnarray}
	\bangp_{x}{P} & & \nonumber\\
	=
	& {x}!\langle{(\prefix{x}{y}{(\outputp{x}{y} | @{y})) | P}}\rangle 
	      | \prefix{x}{y}{(\outputp{x}{y} | @{y})} & \nonumber\\
	\red
	& (\outputp{x}{y} | @{y})\substn{\quotep{(\prefix{x}{y}{(@{y} | \outputp{x}{y})) | P}}}{y} & \nonumber\\
	=
	& \outputp{x}{\quotep{(\prefix{x}{y}{(\outputp{x}{y} | @{y})) | P}}}
	  | {(\prefix{x}{y}{(\outputp{x}{y} | @{y})) | P}} & \nonumber\\
	\red
	& \ldots & \nonumber\\
	\red^*
	& P | P | \ldots & \nonumber
\end{eqnarray}

Of course, this encoding, as an implementation, runs away, unfolding
$\bangp{P}$ eagerly. A lazier and more implementable replication
operator, restricted to input-guarded processes, may be obtained as follows.

\begin{eqnarray}
\bangp{\prefix{u}{v}{P}} 
	:= 
	\binpar{\lift{x}{\prefix{u}{v}{(\binpar{D(x)}{P})}}}{D(x)} \nonumber
\end{eqnarray}

\begin{remark}
  Note that the lazier definition still does not deal with summation
  or mixed summation (i.e. sums over input and output). The reader is
  invited to construct definitions of replication that deal with these
  features. 

  Further, the definitions are parameterized in a name, $x$. Can you,
  gentle reader, make a definition that eliminates this parameter and
  guarantees no accidental interaction between the replication
  machinery and the process being replicated -- i.e. no accidental
  sharing of names used by the process to get its work done and the
  name(s) used by the replication to effect copying. This latter
  revision of the definition of replication is crucial to obtaining
  the expected identity $!!P \sim !P$.
\end{remark}

\begin{remark}\label{rem:paradoxical_combinator}
  The reader familiar with the lambda calculus will have noticed the
  similarity between $D$ and the paradoxical combinator.

  [Ed. note: the existence of this seems to suggest we have to be more
  restrictive on the set of processes and names we admit if we are to
  support no-cloning.]
\end{remark}

\subsubsection{Bisimulation}

The computational dynamics gives rise to another kind of equivalence,
the equivalence of computational behavior. As previously mentioned
this is typically captured \emph{via} some form of bisimulation.

% The notion we use in this paper is weak barbed bisimulation
% \cite{milner91polyadicpi}.

The notion we use in this paper is derived from weak barbed
bisimulation \cite{milner91polyadicpi}. 

\begin{definition}
An \emph{observation relation}, $\downarrow_{\mathcal N}$, over a set
of names, $\mathcal N$, is the smallest relation satisfying the rules
below.

\infrule[Out-barb]{y \in {\mathcal N}, \; x \nameeq y}
		  {\outputp{x}{v} \downarrow_{\mathcal N} x}
\infrule[Par-barb]{\mbox{$P\downarrow_{\mathcal N} x$ or $Q\downarrow_{\mathcal N} x$}}
		  {\binpar{P}{Q} \downarrow_{\mathcal N} x}

We write $P \Downarrow_{\mathcal N} x$ if there is $Q$ such that 
$P \wred Q$ and $Q \downarrow_{\mathcal N} x$.
\end{definition}

\begin{definition}
%\label{def.bbisim}
An  ${\mathcal N}$-\emph{barbed bisimulation} over a set of names, ${\mathcal N}$, is a symmetric binary relation 
${\mathcal S}_{\mathcal N}$ between agents such that $P\rel{S}_{\mathcal N}Q$ implies:
\begin{enumerate}
\item If $P \red P'$ then $Q \wred Q'$ and $P'\rel{S}_{\mathcal N} Q'$.
\item If $P\downarrow_{\mathcal N} x$, then $Q\Downarrow_{\mathcal N} x$.
\end{enumerate}
$P$ is ${\mathcal N}$-barbed bisimilar to $Q$, written
$P \wbbisim_{\mathcal N} Q$, if $P \rel{S}_{\mathcal N} Q$ for some ${\mathcal N}$-barbed bisimulation ${\mathcal S}_{\mathcal N}$.
\end{definition}

$\mathcal{R} \subseteq \pi \times \pi$

$P \mathcal{R} Q => \forall P'. P \red P' \Rightarrow \exists Q'. Q \red Q', P' \mathcal{R} Q'$

$P \vdash x \Rightarrow Q \vdash x$

\begin{mathpar}
  \inferrule*[lab=Out-barb]{x \nameeq y}{{y}!\langle{Q}\rangle \vdash x}
  \and
  \inferrule*[lab=Par-barb]{\mbox{$P\vdash x$ or $Q\vdash x$}}{\binpar{P}{Q} \vdash x}
\end{mathpar}

\subsubsection{Contexts}

One of the principle advantages of computational calculi like the
$\pi$-calculus is a well-defined notion of context,
contextual-equivalence and a correlation between
contextual-equivalence and notions of bisimulation. The notion of
context allows the decomposition of a process into (sub-)process and
its syntactic environment, its context. Thus, a context may be
thought of as a process with a ``hole'' (written $\Box$) in it. The
application of a context $M$ to a process $P$, written $M[P]$, is
tantamount to filling the hole in $M$ with $P$. In this paper we do
not need the full weight of this theory, but do make use of the notion
of context in the proof the main theorem. 

\begin{mathpar}
  \inferrule* [lab=summation] {} {{M_{M},M_{N}} \bc \Box \;|\; x.M_{A} \;|\; M_{M}+M_{N}}
  \and
  \inferrule* [lab=agent] {} {{M_{A}} \bc (\vec{x})M_{P} \;| \; \clift{P_0,\ldots,M_{P},\ldots,P_N}}
  \and \\
  \inferrule* [lab=process] {} {{M_{P}} \bc M_{N} \;| \;P|M_{P} }
\end{mathpar} 

\begin{mathpar}
  \inferrule* [lab=sychronization] {} {M_{N} \bc \Box \;|\; x?M_{F} \;|\; x!M_{C}}
  \and
  \inferrule* [lab=abstraction] {} {{M_{F}} \bc (x)M_{P} }
  \and
  \inferrule* [lab=concretion] {} {{M_{C}} \bc \langle M_{P} \rangle }
  \and \\
  \inferrule* [lab=process] {} {{M_{P}} \bc M_{N} \;| \;P|M_{P} }
\end{mathpar}

\begin{definition}[contextual application] Given a context $M$, and
  process $P$, we define the \emph{contextual application}, $M[P] :=
  M\{P/\Box\}$. That is, the contextual application of M to P is the
  substitution of $P$ for $\Box$ in $M$.
\end{definition}

$\meaningof{-} : L \to \mathcal{P}(\pi)$

\begin{mathpar}
  \inferrule* [lab=collection] {} {\meaningof{true} = \pi, \and \meaningof{~E} = \pi \setminus \meaningof{E}, \and \meaningof{E_{1} \& E_{2}} = \meaningof{E_{1}} \cap \meaningof{E_{2}}}
\end{mathpar}

\begin{mathpar}
  \inferrule* [lab=structure] {} {\meaningof{0} = \{ P \in \pi | P \equiv 0 \}, \and \\ \meaningof{E_1 | E_2} = \{ P \in \pi | P \equiv P_{1} | P_{2}, P_{1} \in \meaningof{E_{1}}, P_{2} \in \meaningof{E_2}\} }
\end{mathpar}

\begin{mathpar}
 \inferrule* [lab=behavior] {} {\meaningof{\langle a?b \rangle E} = \{ P \in \pi | P \equiv Q | u?(y)P', \\ \and \\\\ \and \\ \;\;\; u \in \meaningof{a}, \forall z.P'\{z/y\} \in \meaningof{E\{z/b\}}\}, \and \\ \meaningof{a!E} = \{ P \in \pi | P \equiv Q | x!\langle P' \rangle, x \in \meaningof{a} P' \in \meaningof{E}\} }
\end{mathpar}

\begin{mathpar}
 \inferrule* [lab=nominal] {} {\meaningof{\quotep{E}} = \{ \quotep{P} \in \quotep{\pi} | P \in \meaningof{E} \}, \and \meaningof{\quotep{P}} = \{ \quotep{Q} \in \quotep{\pi} | P \equiv Q \} \and \\ \meaningof{@\quotep{E}} = \{ P \in \pi | P \equiv @x, x \in \meaningof{E} \}}
\end{mathpar}

\begin{eqnarray*}
  \\
  \meaningof{-} : TS \to ST
\end{eqnarray*}

\begin{eqnarray*}
  \\
  L : TS \to ST
\end{eqnarray*}

\begin{eqnarray*}
  \\
  P \models E \iff P \in \meaningof{E}
\end{eqnarray*}

\begin{eqnarray*}
  P \approx_{L} Q \iff \forall E \in L. P \models E \iff Q \models E
\end{eqnarray*}

\begin{eqnarray*}
  P \approx_{K} Q
\end{eqnarray*}

\begin{eqnarray*}
  P \approx Q
\end{eqnarray*}

$\approx_{K} = \approx = \approx_{L}$

\subsubsection{Contextual duality}

Note that contexts extend the quotation operation to a family of
operations from processes to names. Given a context, $M$, we can
define a \emph{nominal context}, $\quotep{M}$ by $\quotep{M}[P] :=
\quotep{M[P]}$. To foreshadow what is to come we observe that these
operations enjoy a duality with processes very much like the duality
between vectors and maps from vectors to scalars.

Further, because the calculus is essentially higher-order, we have a
correspondence between contexts and processes. More specifically,
given a name $x$ and a context $M$ we can construct $M^{*}_{x}$ such
that 

\begin{mathpar}
  M^{*}_{x} | \lift{x}{P} \red M[P]
\end{mathpar}

namely,

\begin{mathpar}
  M^{*}_{x} := x?(u).M[\dropn{u}]
\end{mathpar}

The dependence of $M^{*}_{x}$ on a name makes it an abstraction, 

\begin{mathpar}
  M^{*} := (x)x?(u).M[\dropn{u}]
\end{mathpar}

\subsection{Additional notation}

It will sometimes be convenient to denote the process a name
quotes. We already have the notation $x = \quotep{P}$, but it will be
convenient to introduce an alternate notation, $\procn{x}$, when we
want to emphasize the connection to the use of the name. Note that, by
virtue of name equivalence, $\quotep{\procn{x}} \nameeq x$; so, the
notation is consistent with previous definitions.

Further, because names have structure it is possible to effect
substitutions on the basis of that structure. This means we need to
upgrade our notation for substitutions, which we accomplish by
adapting comprehension notation. Thus,

\begin{mathpar}
  P\{ y / x : x \in S \}
\end{mathpar}

is interpreted to mean the process derived from P by replacing (in a
capture-avoiding manner) each occurrence of $x$ in $S$ by $y$. For example,

\begin{mathpar}
  P\{ \quotep{\procn{x}|\procn{x}} / x : x \in \freenames{P} \}
\end{mathpar}

will replace each (occurrence) of a free name $x$ in $P$ by
$\quotep{\procn{x}|\procn{x}}$.

Also, we will avail ourselves of the notation $x^{L}$ and $x^{R}$ to
denote injections of a name into disjoint copies of the name
space. There are numerous ways to accomplish this. One example can be
found in \cite{MeredithR05}. This notation overloads to vectors of
names: $\vec{x}^{\pi} := (x_{i}^{\pi} \; : \; 0 \leq i < |\vec{x}| )$ where $\pi \in \{L,R\}$.

We also use $P^{\Box} := P|\Box$.

In \cite{MeredithR05} an interpretation of the new operator is
given. It turns out that there are several possible interpretations
all enjoying the requisite algebraic properties of the operator (see
\cite{milner91polyadicpi}). We will therefore make liberal use of
$(\nu\; \vec{x})P$.

% subsection the_syntax_and_semantics_of_the_notation_system (end)   

\input{qm2pi.qmops} 

\input{qm2pi.sterngerlach} 

\input{qm2pi.metric} 

% section concurrent_process_calculi (end)

%\input{qm2pi.proofsketch}

% section proof sketch (end)

%\input{qm2pi.slviaknots} 

% section spatial logic via knots (end)

\input{qm2pi.conclusion}

% section conclusion (end)

%\input{qm2pi.dtcodes} 

% section wiring algorithm (end)

\input{qm2pi.ack} 

% section acknowledgments (end)

\newpage


\bibliographystyle{plain}   
\bibliography{../../biblios/main.bib}

\input{qm2pi.rhodetails}

\end{document}

 

\documentclass[12pt]{llncs}
%\documentclass{jktr}

\usepackage[pdftex]{hyperref}                   
\usepackage {listings}
\usepackage {mathpartir}
\usepackage{bcprules}
%\usepackage{listings}
                       
\usepackage{graphicx} 
%\usepackage[margins=2.5cm,nohead,nofoot]{geometry}
%\usepackage{geometry}
\usepackage{amsfonts}
\usepackage{amstext}
\usepackage{latexsym}
\usepackage{amssymb}
\usepackage{color}


%\include{myPreamble}
\include{qm2pi.local} 

%\ifpdf
%\usepackage[pdftex]{graphicx}
%\else
%\usepackage{graphicx}
%\fi

 % \ifpdf
%  \usepackage{pdfsync}
%  \if


%\title{Brief Article}
%\author{David F. Snyder}
%\author{L.G. Meredith}

%\address{Dept. of Math., Texas State University--San Marcos, San Marcos, TX 78666}
       
\pagestyle{empty}


\begin{document}

\lstset{language=[Objective]Caml,frame=shadowbox}

\input{qm2pi.front}

% section front matter (end)

\input{qm2pi.intro} 
 
% section introduction (end)

% \input{qm2pi.knotations} 

% section notation (end)

\input{qm2pi.process.calculi} 

% section concurrent_process_calculi_and_spatial_logics_ (end)
    
%\input{qm2pi.knots2pi} 

%\input{qm2pi.trefoil} 

%\input{qm2pi.mainthm} 

% subsection basic_interpretation (end)

%\input{qm2pi.rho.presentation} 
\subsection{The syntax and semantics of the notation system}\label{sub:the_syntax_and_semantics_of_the_notation_system} % (fold)

We now summarize a technical presentation of the calculus that
embodies our theory of dynamics. The typical presentation of such a
calculus follows the style of giving generators and relations on
them. The grammar, below, describing term constructors, freely
generates the set of processes, $\Proc$. This set is then quotiented
by a relation known as structural congruence and it is over this set
that the notion of dynamics is expressed. This presentation is
essentially that of \cite{MeredithR05} with the addition of
polyadicity and summation. For readability we have relegated some of
the technical subtleties to an appendix.

\subsubsection{Process grammar}\label{subsub:process_grammar}

\begin{mathpar}
  \inferrule* [lab=synchronization] {} {{M} \bc \pzero \;|\; x?F \;|\; x!C }
  \and
  \inferrule* [lab=abstraction] {} {{F} \bc (x)P}
  \and
  \inferrule* [lab=concretion] {} {{C} \bc \langle Q \rangle}
  \and
  \inferrule* [lab=process] {} {{P,Q} \bc M \;| \;P|Q \;|\; @{x}}
  \and
  \inferrule* [lab=name] {} {{x} \bc \quotep{P}}
\end{mathpar} 

Note that $\vec{x}$ (resp. $\vec{P}$) denotes a vector of names
(resp. processes) of length $|\vec{x}|$ (resp. $|\vec{P}|$). We adopt
the following useful abbreviations.

\begin{mathpar}
   x?(\vec{y}).P := x.(\vec{y})P \and  x\clift{\vec{P}} := x.\clift{\vec{P}}
   \and x!(y) := \lift{x}{\dropn{y}}
   \and \Pi_{i=0}^{n-1}P_i := P_0 | \ldots | P_{n-1}
\end{mathpar}

\subsubsection{Structural congruence}

\paragraph{Free and bound names and alpha-equivalence.} At the
core of structural equivalence is alpha-equivalence which identifies
process that are the same up to a change of variable. Formally, we
recognize the distinction between free and bound names. The free names
of a process, $\freenames{P}$, may be calculated recursively as
follows:

\begin{mathpar}
\freenames{\pzero} := \emptyset
  \and \\
  \freenames{x?(y).P} := \{ x \} \cup (\freenames{P} \setminus \{ y \})
  \and 
  \freenames{x!\langle P \rangle} := \{ x \} \cup \{ P \} 
  \and \\
  \freenames{P|Q} := \freenames{P} \cup \freenames{Q}
  \and \\
  \freenames{@{x}} := \{ x \}
\end{mathpar}

$\pi$
$\quotep{\pi}$

$\freenames{-} : \pi \to \mathcal{P}(\quotep{\pi})$

\begin{eqnarray*}
  \freenames{\pzero} & := & \emptyset \\
  \freenames{x?(y).P} & := & \{ x \} \cup (\freenames{P} \setminus \{ y \}) \\
  \freenames{x!\langle P \rangle} & := & \{ x \} \cup \{ P \} \\
  \freenames{P|Q} & := & \freenames{P} \cup \freenames{Q} \\
  \freenames{\dropn{x}} & := & \{ x \}
\end{eqnarray*}

The bound names of a process, $\boundnames{P}$, are those names occurring in $P$
that are not free. For example, in $x?(y).0$, the name $x$ is free, while $y$ is bound.

\begin{mathpar}
  \inferrule* [lab=monoidal-laws] {} { P|Q \equiv Q|P \and P|0 \equiv P \and P|(Q|R) \equiv (P|Q)|R }
\end{mathpar}

\begin{mathpar}
  \inferrule* [lab=alpha-equivalence] {} { (x)P \equiv (y)P\{y/x\} \and y \not\in \freenames{P} }
\end{mathpar}

\begin{definition}
Then two processes, $P,Q$, are alpha-equivalent if $P = Q\{\vec{y}/\vec{x}\}$ for
some $\vec{x} \in \boundnames{Q},\vec{y} \in \boundnames{P}$, where $Q\{\vec{y}/\vec{x}\}$
denotes the capture-avoiding substitution of $\vec{y}$ for $\vec{x}$ in $Q$.
\end{definition}

\begin{definition}
  The {\em structural congruence} \cite{SangiorgiWalker} , $\equiv$,
  between processes is the least congruence containing
  alpha-equivalence, satisfying the abelian monoid laws
  (associativity, commutativity and $\pzero$ as identity) for parallel
  composition $|$ and for summation $+$.
\end{definition}

\subsection{Name equivalence}

We take name equivalence, written $\nameeq$, to be the smallest
equivalence relation generated by the following rules.

\begin{mathpar}
\inferrule*[lab=Quote-drop]
{ }
{ \quotep{@{x}} \nameeq x }

\inferrule*[lab=Struct-equiv]
{ P \scong Q }
{ \quotep{P} \nameeq \quotep{Q} }
\end{mathpar}

The astute reader will have noticed that the mutual recursion of names
and processes imposes a mutual recursion on alpha-equivalence and
structural equivalence via name-equivalence. Fortunately, all of this
works out pleasantly and we may calculate in the natural way, free of
concern. The reader interested in the details is referred to the
appendix \ref{appendix:rho_details}.

\subsection{Substitution}

We use $\Proc$ for the set of processes, $\QProc$ for the set of
names, and $\id{\{}\vec{y} / \vec{x} \id{\}}$ to denote partial maps,
$s : \QProc \rightarrow \QProc$. A map, $s$ lifts, uniquely, to a map
on process terms, $\widehat{s} : \Proc \rightarrow \Proc$ by the
following equations.

\begin{mathpar}
  (0) \psubstp{Q}{P} := 0 \\
  (R \juxtap S) \psubstp{Q}{P}
  :=    
  (R)\psubstp{Q}{P} \juxtap (S) \psubstp{Q}{P} \\
  (x?(y).R) \psubstp{Q}{P}    
  :=    
  (x)\substp{Q}{P} (z)\concat( (R \psubstn{z}{y}) \psubstp{Q}{P} ) \\
  (\lift{x}{R}) \psubstp{Q}{P}  
  :=
  \lift{(x)\substp{Q}{P}}{ R \psubstp{Q}{P} } \\
%   (\dropn{x})  \psubstp{Q}{P}       
%   := 
%   \left\{ 
%     \begin{array}{ccc} 
%       \dropn{\quotep{Q}} & & x \nameeq \quotep{P} \\
%       \dropn{x} & & otherwise \\
%     \end{array}
%   \right. 
  (\dropn{x})  \psubstp{Q}{P}       
  := 
  \left\{ 
    \begin{array}{ccc} 
      Q & & x \nameeq \quotep{P} \\
      \dropn{x} & & otherwise \\
    \end{array}
  \right.
\end{mathpar}
 

where

\begin{eqnarray}
  (x)\id{\{} \lpquote Q \rpquote / \lpquote P \rpquote \id{\}}            = 
  \left\{ 
    \begin{array}{ccc}
      \lpquote Q \rpquote & & x \nameeq \lpquote P \rpquote \\
      x & & otherwise \\
    \end{array}
  \right. \nonumber
\end{eqnarray}

and $z$ is chosen distinct from $\quotep{P}$, $\quotep{Q}$, the free
names in $Q$, and all the names in $R$. Our $\alpha$-equivalence will
be built in the standard way from this substitution.

\begin{remark}\label{rem:no_self_referential_names}
  One consequence of these definitions is that $\forall P. \quotep{P}
  \not\in \freenames{P}$.
\end{remark}

\subsection{ Dynamic quote: an example }

Anticipating something of what's to come, consider applying the
substitution, $\widehat{\id{\{}u / z \id{\}}}$, to the following pair
of processes, $\lift{w}{y!(z)}$ and $w[ \lpquote y!(z) \rpquote ]$.

\begin{eqnarray}
	\lift{w}{y!(z)}\widehat{\id{\{}u / z \id{\}}}
		& = &
		\lift{w}{y!(u)} \nonumber\\
	w[ \lpquote y!(z) \rpquote ] \widehat{ \id{\{}u / z \id{\}} }
		& = &
		w[ \lpquote y!(z) \rpquote ] \nonumber
\end{eqnarray}

Because the body of the process between quotes is impervious to
substitution, we get radically different answers. In fact, by
examining the first process in an input context,
e.g. $x?(z).\lift{w}{y!(z)}$, we see that the process under the lift
operator may be shaped by prefixed inputs binding a name inside it. In
this sense, the lift operator will be seen as a way to dynamically
construct processes before reifying them as names.

Finally equipped with these standard features we can present the
dynamics of the calculus.

\subsubsection{Operational semantics} 

Finally, we introduce the computational dynamics. What marks these
algebras as distinct from other more traditionally studied algebraic
structures, e.g. vector spaces or polynomial rings, is the manner in
which dynamics is captured. In traditional structures, dynamics is typically
expressed through morphisms between such structures, as in linear maps
between vector spaces or morphisms between rings. In algebras
associated with the semantics of computation, the dynamics is
expressed as part of the algebraic structure itself, through a
reduction reduction relation typically denoted by $\red$. Below, we
give a recursive presentation of this relation for the calculus used
in the encoding.

$\red \subseteq \pi \times \pi$
$\red : \pi \to \mathcal{P}(\pi)$

\begin{mathpar}
  \inferrule* [lab=Comm] { \textsf{match}( x_{src}, x_{trgt} ) } { x_{trgt}?(y)P \; | \; x_{src}!\langle {Q} \rangle \red P\{\quotep{Q}/y}\} }
  \and \\
  \inferrule* [lab=Par] {{P} \red {P}'} {{{P} | {Q}} \red {{P}' | {Q}}}
  \and
  \inferrule* [lab=Equiv]{{{P} \scong {P}'} \andalso {{P}' \red {Q}'} \andalso {{Q}' \scong {Q}}}{{P} \red {Q}}
\end{mathpar}

\begin{eqnarray*}
  match_{\equiv} (\quotep{P},\quotep{Q}) & := & P \equiv Q \\
  match_{\dagger}(\quotep{P},\quotep{Q}) & := & \forall R. P|Q \red^{*} R => R \red^{*} 0 \\
  match_{K}(\quotep{P},\quotep{Q}) & := & K \mbox{ for some context } K
\end{eqnarray*}

$u?(x)P | u!\langle Q \rangle \red P\{\quotep{Q}/x\}$

%We write $\wred$ for $\red^*$, and $P\red$ if $\exists Q $ such that $ P \red Q$.
We write $P\red$ if $\exists Q $ such that $ P \red Q$ and $P\not\red$, otherwise.

\section{Replication}

As mentioned before, it is known that replication (and hence
recursion) can be implemented in a higher-order process algebra
\cite{SangiorgiWalker}. As our first example of calculation with the
machinery thus far presented we give the construction explicitly in
the {\rhoc}.

\begin{eqnarray}
	D_{x} & := & \prefix{x}{y}{(\binpar{\outputp{x}{y}}{@{y}})} \nonumber\\
	\bangp_{x}{P} & := & \binpar{{x}!\langle{\binpar{D_{x}}{P}}\rangle}{D_{x}} \nonumber
\end{eqnarray}

\begin{eqnarray}
	\bangp_{x}{P} & & \nonumber\\
	=
	& {x}!\langle{(\prefix{x}{y}{(\outputp{x}{y} | @{y})) | P}}\rangle 
	      | \prefix{x}{y}{(\outputp{x}{y} | @{y})} & \nonumber\\
	\red
	& (\outputp{x}{y} | @{y})\substn{\quotep{(\prefix{x}{y}{(@{y} | \outputp{x}{y})) | P}}}{y} & \nonumber\\
	=
	& \outputp{x}{\quotep{(\prefix{x}{y}{(\outputp{x}{y} | @{y})) | P}}}
	  | {(\prefix{x}{y}{(\outputp{x}{y} | @{y})) | P}} & \nonumber\\
	\red
	& \ldots & \nonumber\\
	\red^*
	& P | P | \ldots & \nonumber
\end{eqnarray}

Of course, this encoding, as an implementation, runs away, unfolding
$\bangp{P}$ eagerly. A lazier and more implementable replication
operator, restricted to input-guarded processes, may be obtained as follows.

\begin{eqnarray}
\bangp{\prefix{u}{v}{P}} 
	:= 
	\binpar{\lift{x}{\prefix{u}{v}{(\binpar{D(x)}{P})}}}{D(x)} \nonumber
\end{eqnarray}

\begin{remark}
  Note that the lazier definition still does not deal with summation
  or mixed summation (i.e. sums over input and output). The reader is
  invited to construct definitions of replication that deal with these
  features. 

  Further, the definitions are parameterized in a name, $x$. Can you,
  gentle reader, make a definition that eliminates this parameter and
  guarantees no accidental interaction between the replication
  machinery and the process being replicated -- i.e. no accidental
  sharing of names used by the process to get its work done and the
  name(s) used by the replication to effect copying. This latter
  revision of the definition of replication is crucial to obtaining
  the expected identity $!!P \sim !P$.
\end{remark}

\begin{remark}\label{rem:paradoxical_combinator}
  The reader familiar with the lambda calculus will have noticed the
  similarity between $D$ and the paradoxical combinator.

  [Ed. note: the existence of this seems to suggest we have to be more
  restrictive on the set of processes and names we admit if we are to
  support no-cloning.]
\end{remark}

\subsubsection{Bisimulation}

The computational dynamics gives rise to another kind of equivalence,
the equivalence of computational behavior. As previously mentioned
this is typically captured \emph{via} some form of bisimulation.

% The notion we use in this paper is weak barbed bisimulation
% \cite{milner91polyadicpi}.

The notion we use in this paper is derived from weak barbed
bisimulation \cite{milner91polyadicpi}. 

\begin{definition}
An \emph{observation relation}, $\downarrow_{\mathcal N}$, over a set
of names, $\mathcal N$, is the smallest relation satisfying the rules
below.

\infrule[Out-barb]{y \in {\mathcal N}, \; x \nameeq y}
		  {\outputp{x}{v} \downarrow_{\mathcal N} x}
\infrule[Par-barb]{\mbox{$P\downarrow_{\mathcal N} x$ or $Q\downarrow_{\mathcal N} x$}}
		  {\binpar{P}{Q} \downarrow_{\mathcal N} x}

We write $P \Downarrow_{\mathcal N} x$ if there is $Q$ such that 
$P \wred Q$ and $Q \downarrow_{\mathcal N} x$.
\end{definition}

\begin{definition}
%\label{def.bbisim}
An  ${\mathcal N}$-\emph{barbed bisimulation} over a set of names, ${\mathcal N}$, is a symmetric binary relation 
${\mathcal S}_{\mathcal N}$ between agents such that $P\rel{S}_{\mathcal N}Q$ implies:
\begin{enumerate}
\item If $P \red P'$ then $Q \wred Q'$ and $P'\rel{S}_{\mathcal N} Q'$.
\item If $P\downarrow_{\mathcal N} x$, then $Q\Downarrow_{\mathcal N} x$.
\end{enumerate}
$P$ is ${\mathcal N}$-barbed bisimilar to $Q$, written
$P \wbbisim_{\mathcal N} Q$, if $P \rel{S}_{\mathcal N} Q$ for some ${\mathcal N}$-barbed bisimulation ${\mathcal S}_{\mathcal N}$.
\end{definition}

$\mathcal{R} \subseteq \pi \times \pi$

$P \mathcal{R} Q => \forall P'. P \red P' \Rightarrow \exists Q'. Q \red Q', P' \mathcal{R} Q'$

$P \vdash x \Rightarrow Q \vdash x$

\begin{mathpar}
  \inferrule*[lab=Out-barb]{x \nameeq y}{{y}!\langle{Q}\rangle \vdash x}
  \and
  \inferrule*[lab=Par-barb]{\mbox{$P\vdash x$ or $Q\vdash x$}}{\binpar{P}{Q} \vdash x}
\end{mathpar}

\subsubsection{Contexts}

One of the principle advantages of computational calculi like the
$\pi$-calculus is a well-defined notion of context,
contextual-equivalence and a correlation between
contextual-equivalence and notions of bisimulation. The notion of
context allows the decomposition of a process into (sub-)process and
its syntactic environment, its context. Thus, a context may be
thought of as a process with a ``hole'' (written $\Box$) in it. The
application of a context $M$ to a process $P$, written $M[P]$, is
tantamount to filling the hole in $M$ with $P$. In this paper we do
not need the full weight of this theory, but do make use of the notion
of context in the proof the main theorem. 

\begin{mathpar}
  \inferrule* [lab=summation] {} {{M_{M},M_{N}} \bc \Box \;|\; x.M_{A} \;|\; M_{M}+M_{N}}
  \and
  \inferrule* [lab=agent] {} {{M_{A}} \bc (\vec{x})M_{P} \;| \; \clift{P_0,\ldots,M_{P},\ldots,P_N}}
  \and \\
  \inferrule* [lab=process] {} {{M_{P}} \bc M_{N} \;| \;P|M_{P} }
\end{mathpar} 

\begin{mathpar}
  \inferrule* [lab=sychronization] {} {M_{N} \bc \Box \;|\; x?M_{F} \;|\; x!M_{C}}
  \and
  \inferrule* [lab=abstraction] {} {{M_{F}} \bc (x)M_{P} }
  \and
  \inferrule* [lab=concretion] {} {{M_{C}} \bc \langle M_{P} \rangle }
  \and \\
  \inferrule* [lab=process] {} {{M_{P}} \bc M_{N} \;| \;P|M_{P} }
\end{mathpar}

\begin{definition}[contextual application] Given a context $M$, and
  process $P$, we define the \emph{contextual application}, $M[P] :=
  M\{P/\Box\}$. That is, the contextual application of M to P is the
  substitution of $P$ for $\Box$ in $M$.
\end{definition}

$\meaningof{-} : L \to \mathcal{P}(\pi)$

\begin{mathpar}
  \inferrule* [lab=collection] {} {\meaningof{true} = \pi, \and \meaningof{~E} = \pi \setminus \meaningof{E}, \and \meaningof{E_{1} \& E_{2}} = \meaningof{E_{1}} \cap \meaningof{E_{2}}}
\end{mathpar}

\begin{mathpar}
  \inferrule* [lab=structure] {} {\meaningof{0} = \{ P \in \pi | P \equiv 0 \}, \and \\ \meaningof{E_1 | E_2} = \{ P \in \pi | P \equiv P_{1} | P_{2}, P_{1} \in \meaningof{E_{1}}, P_{2} \in \meaningof{E_2}\} }
\end{mathpar}

\begin{mathpar}
 \inferrule* [lab=behavior] {} {\meaningof{\langle a?b \rangle E} = \{ P \in \pi | P \equiv Q | u?(y)P', \\ \and \\\\ \and \\ \;\;\; u \in \meaningof{a}, \forall z.P'\{z/y\} \in \meaningof{E\{z/b\}}\}, \and \\ \meaningof{a!E} = \{ P \in \pi | P \equiv Q | x!\langle P' \rangle, x \in \meaningof{a} P' \in \meaningof{E}\} }
\end{mathpar}

\begin{mathpar}
 \inferrule* [lab=nominal] {} {\meaningof{\quotep{E}} = \{ \quotep{P} \in \quotep{\pi} | P \in \meaningof{E} \}, \and \meaningof{\quotep{P}} = \{ \quotep{Q} \in \quotep{\pi} | P \equiv Q \} \and \\ \meaningof{@\quotep{E}} = \{ P \in \pi | P \equiv @x, x \in \meaningof{E} \}}
\end{mathpar}

\begin{eqnarray*}
  \\
  \meaningof{-} : TS \to ST
\end{eqnarray*}

\begin{eqnarray*}
  \\
  L : TS \to ST
\end{eqnarray*}

\begin{eqnarray*}
  \\
  P \models E \iff P \in \meaningof{E}
\end{eqnarray*}

\begin{eqnarray*}
  P \approx_{L} Q \iff \forall E \in L. P \models E \iff Q \models E
\end{eqnarray*}

\begin{eqnarray*}
  P \approx_{K} Q
\end{eqnarray*}

\begin{eqnarray*}
  P \approx Q
\end{eqnarray*}

$\approx_{K} = \approx = \approx_{L}$

\subsubsection{Contextual duality}

Note that contexts extend the quotation operation to a family of
operations from processes to names. Given a context, $M$, we can
define a \emph{nominal context}, $\quotep{M}$ by $\quotep{M}[P] :=
\quotep{M[P]}$. To foreshadow what is to come we observe that these
operations enjoy a duality with processes very much like the duality
between vectors and maps from vectors to scalars.

Further, because the calculus is essentially higher-order, we have a
correspondence between contexts and processes. More specifically,
given a name $x$ and a context $M$ we can construct $M^{*}_{x}$ such
that 

\begin{mathpar}
  M^{*}_{x} | \lift{x}{P} \red M[P]
\end{mathpar}

namely,

\begin{mathpar}
  M^{*}_{x} := x?(u).M[\dropn{u}]
\end{mathpar}

The dependence of $M^{*}_{x}$ on a name makes it an abstraction, 

\begin{mathpar}
  M^{*} := (x)x?(u).M[\dropn{u}]
\end{mathpar}

\subsection{Additional notation}

It will sometimes be convenient to denote the process a name
quotes. We already have the notation $x = \quotep{P}$, but it will be
convenient to introduce an alternate notation, $\procn{x}$, when we
want to emphasize the connection to the use of the name. Note that, by
virtue of name equivalence, $\quotep{\procn{x}} \nameeq x$; so, the
notation is consistent with previous definitions.

Further, because names have structure it is possible to effect
substitutions on the basis of that structure. This means we need to
upgrade our notation for substitutions, which we accomplish by
adapting comprehension notation. Thus,

\begin{mathpar}
  P\{ y / x : x \in S \}
\end{mathpar}

is interpreted to mean the process derived from P by replacing (in a
capture-avoiding manner) each occurrence of $x$ in $S$ by $y$. For example,

\begin{mathpar}
  P\{ \quotep{\procn{x}|\procn{x}} / x : x \in \freenames{P} \}
\end{mathpar}

will replace each (occurrence) of a free name $x$ in $P$ by
$\quotep{\procn{x}|\procn{x}}$.

Also, we will avail ourselves of the notation $x^{L}$ and $x^{R}$ to
denote injections of a name into disjoint copies of the name
space. There are numerous ways to accomplish this. One example can be
found in \cite{MeredithR05}. This notation overloads to vectors of
names: $\vec{x}^{\pi} := (x_{i}^{\pi} \; : \; 0 \leq i < |\vec{x}| )$ where $\pi \in \{L,R\}$.

We also use $P^{\Box} := P|\Box$.

In \cite{MeredithR05} an interpretation of the new operator is
given. It turns out that there are several possible interpretations
all enjoying the requisite algebraic properties of the operator (see
\cite{milner91polyadicpi}). We will therefore make liberal use of
$(\nu\; \vec{x})P$.

% subsection the_syntax_and_semantics_of_the_notation_system (end)   

\input{qm2pi.qmops} 

\input{qm2pi.sterngerlach} 

\input{qm2pi.metric} 

% section concurrent_process_calculi (end)

%\input{qm2pi.proofsketch}

% section proof sketch (end)

%\input{qm2pi.slviaknots} 

% section spatial logic via knots (end)

\input{qm2pi.conclusion}

% section conclusion (end)

%\input{qm2pi.dtcodes} 

% section wiring algorithm (end)

\input{qm2pi.ack} 

% section acknowledgments (end)

\newpage


\bibliographystyle{plain}   
\bibliography{../../biblios/main.bib}

\input{qm2pi.rhodetails}

\end{document}

 

% section concurrent_process_calculi (end)

%\documentclass[12pt]{llncs}
%\documentclass{jktr}

\usepackage[pdftex]{hyperref}                   
\usepackage {listings}
\usepackage {mathpartir}
\usepackage{bcprules}
%\usepackage{listings}
                       
\usepackage{graphicx} 
%\usepackage[margins=2.5cm,nohead,nofoot]{geometry}
%\usepackage{geometry}
\usepackage{amsfonts}
\usepackage{amstext}
\usepackage{latexsym}
\usepackage{amssymb}
\usepackage{color}


%\include{myPreamble}
\include{qm2pi.local} 

%\ifpdf
%\usepackage[pdftex]{graphicx}
%\else
%\usepackage{graphicx}
%\fi

 % \ifpdf
%  \usepackage{pdfsync}
%  \if


%\title{Brief Article}
%\author{David F. Snyder}
%\author{L.G. Meredith}

%\address{Dept. of Math., Texas State University--San Marcos, San Marcos, TX 78666}
       
\pagestyle{empty}


\begin{document}

\lstset{language=[Objective]Caml,frame=shadowbox}

\input{qm2pi.front}

% section front matter (end)

\input{qm2pi.intro} 
 
% section introduction (end)

% \input{qm2pi.knotations} 

% section notation (end)

\input{qm2pi.process.calculi} 

% section concurrent_process_calculi_and_spatial_logics_ (end)
    
%\input{qm2pi.knots2pi} 

%\input{qm2pi.trefoil} 

%\input{qm2pi.mainthm} 

% subsection basic_interpretation (end)

%\input{qm2pi.rho.presentation} 
\subsection{The syntax and semantics of the notation system}\label{sub:the_syntax_and_semantics_of_the_notation_system} % (fold)

We now summarize a technical presentation of the calculus that
embodies our theory of dynamics. The typical presentation of such a
calculus follows the style of giving generators and relations on
them. The grammar, below, describing term constructors, freely
generates the set of processes, $\Proc$. This set is then quotiented
by a relation known as structural congruence and it is over this set
that the notion of dynamics is expressed. This presentation is
essentially that of \cite{MeredithR05} with the addition of
polyadicity and summation. For readability we have relegated some of
the technical subtleties to an appendix.

\subsubsection{Process grammar}\label{subsub:process_grammar}

\begin{mathpar}
  \inferrule* [lab=synchronization] {} {{M} \bc \pzero \;|\; x?F \;|\; x!C }
  \and
  \inferrule* [lab=abstraction] {} {{F} \bc (x)P}
  \and
  \inferrule* [lab=concretion] {} {{C} \bc \langle Q \rangle}
  \and
  \inferrule* [lab=process] {} {{P,Q} \bc M \;| \;P|Q \;|\; @{x}}
  \and
  \inferrule* [lab=name] {} {{x} \bc \quotep{P}}
\end{mathpar} 

Note that $\vec{x}$ (resp. $\vec{P}$) denotes a vector of names
(resp. processes) of length $|\vec{x}|$ (resp. $|\vec{P}|$). We adopt
the following useful abbreviations.

\begin{mathpar}
   x?(\vec{y}).P := x.(\vec{y})P \and  x\clift{\vec{P}} := x.\clift{\vec{P}}
   \and x!(y) := \lift{x}{\dropn{y}}
   \and \Pi_{i=0}^{n-1}P_i := P_0 | \ldots | P_{n-1}
\end{mathpar}

\subsubsection{Structural congruence}

\paragraph{Free and bound names and alpha-equivalence.} At the
core of structural equivalence is alpha-equivalence which identifies
process that are the same up to a change of variable. Formally, we
recognize the distinction between free and bound names. The free names
of a process, $\freenames{P}$, may be calculated recursively as
follows:

\begin{mathpar}
\freenames{\pzero} := \emptyset
  \and \\
  \freenames{x?(y).P} := \{ x \} \cup (\freenames{P} \setminus \{ y \})
  \and 
  \freenames{x!\langle P \rangle} := \{ x \} \cup \{ P \} 
  \and \\
  \freenames{P|Q} := \freenames{P} \cup \freenames{Q}
  \and \\
  \freenames{@{x}} := \{ x \}
\end{mathpar}

$\pi$
$\quotep{\pi}$

$\freenames{-} : \pi \to \mathcal{P}(\quotep{\pi})$

\begin{eqnarray*}
  \freenames{\pzero} & := & \emptyset \\
  \freenames{x?(y).P} & := & \{ x \} \cup (\freenames{P} \setminus \{ y \}) \\
  \freenames{x!\langle P \rangle} & := & \{ x \} \cup \{ P \} \\
  \freenames{P|Q} & := & \freenames{P} \cup \freenames{Q} \\
  \freenames{\dropn{x}} & := & \{ x \}
\end{eqnarray*}

The bound names of a process, $\boundnames{P}$, are those names occurring in $P$
that are not free. For example, in $x?(y).0$, the name $x$ is free, while $y$ is bound.

\begin{mathpar}
  \inferrule* [lab=monoidal-laws] {} { P|Q \equiv Q|P \and P|0 \equiv P \and P|(Q|R) \equiv (P|Q)|R }
\end{mathpar}

\begin{mathpar}
  \inferrule* [lab=alpha-equivalence] {} { (x)P \equiv (y)P\{y/x\} \and y \not\in \freenames{P} }
\end{mathpar}

\begin{definition}
Then two processes, $P,Q$, are alpha-equivalent if $P = Q\{\vec{y}/\vec{x}\}$ for
some $\vec{x} \in \boundnames{Q},\vec{y} \in \boundnames{P}$, where $Q\{\vec{y}/\vec{x}\}$
denotes the capture-avoiding substitution of $\vec{y}$ for $\vec{x}$ in $Q$.
\end{definition}

\begin{definition}
  The {\em structural congruence} \cite{SangiorgiWalker} , $\equiv$,
  between processes is the least congruence containing
  alpha-equivalence, satisfying the abelian monoid laws
  (associativity, commutativity and $\pzero$ as identity) for parallel
  composition $|$ and for summation $+$.
\end{definition}

\subsection{Name equivalence}

We take name equivalence, written $\nameeq$, to be the smallest
equivalence relation generated by the following rules.

\begin{mathpar}
\inferrule*[lab=Quote-drop]
{ }
{ \quotep{@{x}} \nameeq x }

\inferrule*[lab=Struct-equiv]
{ P \scong Q }
{ \quotep{P} \nameeq \quotep{Q} }
\end{mathpar}

The astute reader will have noticed that the mutual recursion of names
and processes imposes a mutual recursion on alpha-equivalence and
structural equivalence via name-equivalence. Fortunately, all of this
works out pleasantly and we may calculate in the natural way, free of
concern. The reader interested in the details is referred to the
appendix \ref{appendix:rho_details}.

\subsection{Substitution}

We use $\Proc$ for the set of processes, $\QProc$ for the set of
names, and $\id{\{}\vec{y} / \vec{x} \id{\}}$ to denote partial maps,
$s : \QProc \rightarrow \QProc$. A map, $s$ lifts, uniquely, to a map
on process terms, $\widehat{s} : \Proc \rightarrow \Proc$ by the
following equations.

\begin{mathpar}
  (0) \psubstp{Q}{P} := 0 \\
  (R \juxtap S) \psubstp{Q}{P}
  :=    
  (R)\psubstp{Q}{P} \juxtap (S) \psubstp{Q}{P} \\
  (x?(y).R) \psubstp{Q}{P}    
  :=    
  (x)\substp{Q}{P} (z)\concat( (R \psubstn{z}{y}) \psubstp{Q}{P} ) \\
  (\lift{x}{R}) \psubstp{Q}{P}  
  :=
  \lift{(x)\substp{Q}{P}}{ R \psubstp{Q}{P} } \\
%   (\dropn{x})  \psubstp{Q}{P}       
%   := 
%   \left\{ 
%     \begin{array}{ccc} 
%       \dropn{\quotep{Q}} & & x \nameeq \quotep{P} \\
%       \dropn{x} & & otherwise \\
%     \end{array}
%   \right. 
  (\dropn{x})  \psubstp{Q}{P}       
  := 
  \left\{ 
    \begin{array}{ccc} 
      Q & & x \nameeq \quotep{P} \\
      \dropn{x} & & otherwise \\
    \end{array}
  \right.
\end{mathpar}
 

where

\begin{eqnarray}
  (x)\id{\{} \lpquote Q \rpquote / \lpquote P \rpquote \id{\}}            = 
  \left\{ 
    \begin{array}{ccc}
      \lpquote Q \rpquote & & x \nameeq \lpquote P \rpquote \\
      x & & otherwise \\
    \end{array}
  \right. \nonumber
\end{eqnarray}

and $z$ is chosen distinct from $\quotep{P}$, $\quotep{Q}$, the free
names in $Q$, and all the names in $R$. Our $\alpha$-equivalence will
be built in the standard way from this substitution.

\begin{remark}\label{rem:no_self_referential_names}
  One consequence of these definitions is that $\forall P. \quotep{P}
  \not\in \freenames{P}$.
\end{remark}

\subsection{ Dynamic quote: an example }

Anticipating something of what's to come, consider applying the
substitution, $\widehat{\id{\{}u / z \id{\}}}$, to the following pair
of processes, $\lift{w}{y!(z)}$ and $w[ \lpquote y!(z) \rpquote ]$.

\begin{eqnarray}
	\lift{w}{y!(z)}\widehat{\id{\{}u / z \id{\}}}
		& = &
		\lift{w}{y!(u)} \nonumber\\
	w[ \lpquote y!(z) \rpquote ] \widehat{ \id{\{}u / z \id{\}} }
		& = &
		w[ \lpquote y!(z) \rpquote ] \nonumber
\end{eqnarray}

Because the body of the process between quotes is impervious to
substitution, we get radically different answers. In fact, by
examining the first process in an input context,
e.g. $x?(z).\lift{w}{y!(z)}$, we see that the process under the lift
operator may be shaped by prefixed inputs binding a name inside it. In
this sense, the lift operator will be seen as a way to dynamically
construct processes before reifying them as names.

Finally equipped with these standard features we can present the
dynamics of the calculus.

\subsubsection{Operational semantics} 

Finally, we introduce the computational dynamics. What marks these
algebras as distinct from other more traditionally studied algebraic
structures, e.g. vector spaces or polynomial rings, is the manner in
which dynamics is captured. In traditional structures, dynamics is typically
expressed through morphisms between such structures, as in linear maps
between vector spaces or morphisms between rings. In algebras
associated with the semantics of computation, the dynamics is
expressed as part of the algebraic structure itself, through a
reduction reduction relation typically denoted by $\red$. Below, we
give a recursive presentation of this relation for the calculus used
in the encoding.

$\red \subseteq \pi \times \pi$
$\red : \pi \to \mathcal{P}(\pi)$

\begin{mathpar}
  \inferrule* [lab=Comm] { \textsf{match}( x_{src}, x_{trgt} ) } { x_{trgt}?(y)P \; | \; x_{src}!\langle {Q} \rangle \red P\{\quotep{Q}/y}\} }
  \and \\
  \inferrule* [lab=Par] {{P} \red {P}'} {{{P} | {Q}} \red {{P}' | {Q}}}
  \and
  \inferrule* [lab=Equiv]{{{P} \scong {P}'} \andalso {{P}' \red {Q}'} \andalso {{Q}' \scong {Q}}}{{P} \red {Q}}
\end{mathpar}

\begin{eqnarray*}
  match_{\equiv} (\quotep{P},\quotep{Q}) & := & P \equiv Q \\
  match_{\dagger}(\quotep{P},\quotep{Q}) & := & \forall R. P|Q \red^{*} R => R \red^{*} 0 \\
  match_{K}(\quotep{P},\quotep{Q}) & := & K \mbox{ for some context } K
\end{eqnarray*}

$u?(x)P | u!\langle Q \rangle \red P\{\quotep{Q}/x\}$

%We write $\wred$ for $\red^*$, and $P\red$ if $\exists Q $ such that $ P \red Q$.
We write $P\red$ if $\exists Q $ such that $ P \red Q$ and $P\not\red$, otherwise.

\section{Replication}

As mentioned before, it is known that replication (and hence
recursion) can be implemented in a higher-order process algebra
\cite{SangiorgiWalker}. As our first example of calculation with the
machinery thus far presented we give the construction explicitly in
the {\rhoc}.

\begin{eqnarray}
	D_{x} & := & \prefix{x}{y}{(\binpar{\outputp{x}{y}}{@{y}})} \nonumber\\
	\bangp_{x}{P} & := & \binpar{{x}!\langle{\binpar{D_{x}}{P}}\rangle}{D_{x}} \nonumber
\end{eqnarray}

\begin{eqnarray}
	\bangp_{x}{P} & & \nonumber\\
	=
	& {x}!\langle{(\prefix{x}{y}{(\outputp{x}{y} | @{y})) | P}}\rangle 
	      | \prefix{x}{y}{(\outputp{x}{y} | @{y})} & \nonumber\\
	\red
	& (\outputp{x}{y} | @{y})\substn{\quotep{(\prefix{x}{y}{(@{y} | \outputp{x}{y})) | P}}}{y} & \nonumber\\
	=
	& \outputp{x}{\quotep{(\prefix{x}{y}{(\outputp{x}{y} | @{y})) | P}}}
	  | {(\prefix{x}{y}{(\outputp{x}{y} | @{y})) | P}} & \nonumber\\
	\red
	& \ldots & \nonumber\\
	\red^*
	& P | P | \ldots & \nonumber
\end{eqnarray}

Of course, this encoding, as an implementation, runs away, unfolding
$\bangp{P}$ eagerly. A lazier and more implementable replication
operator, restricted to input-guarded processes, may be obtained as follows.

\begin{eqnarray}
\bangp{\prefix{u}{v}{P}} 
	:= 
	\binpar{\lift{x}{\prefix{u}{v}{(\binpar{D(x)}{P})}}}{D(x)} \nonumber
\end{eqnarray}

\begin{remark}
  Note that the lazier definition still does not deal with summation
  or mixed summation (i.e. sums over input and output). The reader is
  invited to construct definitions of replication that deal with these
  features. 

  Further, the definitions are parameterized in a name, $x$. Can you,
  gentle reader, make a definition that eliminates this parameter and
  guarantees no accidental interaction between the replication
  machinery and the process being replicated -- i.e. no accidental
  sharing of names used by the process to get its work done and the
  name(s) used by the replication to effect copying. This latter
  revision of the definition of replication is crucial to obtaining
  the expected identity $!!P \sim !P$.
\end{remark}

\begin{remark}\label{rem:paradoxical_combinator}
  The reader familiar with the lambda calculus will have noticed the
  similarity between $D$ and the paradoxical combinator.

  [Ed. note: the existence of this seems to suggest we have to be more
  restrictive on the set of processes and names we admit if we are to
  support no-cloning.]
\end{remark}

\subsubsection{Bisimulation}

The computational dynamics gives rise to another kind of equivalence,
the equivalence of computational behavior. As previously mentioned
this is typically captured \emph{via} some form of bisimulation.

% The notion we use in this paper is weak barbed bisimulation
% \cite{milner91polyadicpi}.

The notion we use in this paper is derived from weak barbed
bisimulation \cite{milner91polyadicpi}. 

\begin{definition}
An \emph{observation relation}, $\downarrow_{\mathcal N}$, over a set
of names, $\mathcal N$, is the smallest relation satisfying the rules
below.

\infrule[Out-barb]{y \in {\mathcal N}, \; x \nameeq y}
		  {\outputp{x}{v} \downarrow_{\mathcal N} x}
\infrule[Par-barb]{\mbox{$P\downarrow_{\mathcal N} x$ or $Q\downarrow_{\mathcal N} x$}}
		  {\binpar{P}{Q} \downarrow_{\mathcal N} x}

We write $P \Downarrow_{\mathcal N} x$ if there is $Q$ such that 
$P \wred Q$ and $Q \downarrow_{\mathcal N} x$.
\end{definition}

\begin{definition}
%\label{def.bbisim}
An  ${\mathcal N}$-\emph{barbed bisimulation} over a set of names, ${\mathcal N}$, is a symmetric binary relation 
${\mathcal S}_{\mathcal N}$ between agents such that $P\rel{S}_{\mathcal N}Q$ implies:
\begin{enumerate}
\item If $P \red P'$ then $Q \wred Q'$ and $P'\rel{S}_{\mathcal N} Q'$.
\item If $P\downarrow_{\mathcal N} x$, then $Q\Downarrow_{\mathcal N} x$.
\end{enumerate}
$P$ is ${\mathcal N}$-barbed bisimilar to $Q$, written
$P \wbbisim_{\mathcal N} Q$, if $P \rel{S}_{\mathcal N} Q$ for some ${\mathcal N}$-barbed bisimulation ${\mathcal S}_{\mathcal N}$.
\end{definition}

$\mathcal{R} \subseteq \pi \times \pi$

$P \mathcal{R} Q => \forall P'. P \red P' \Rightarrow \exists Q'. Q \red Q', P' \mathcal{R} Q'$

$P \vdash x \Rightarrow Q \vdash x$

\begin{mathpar}
  \inferrule*[lab=Out-barb]{x \nameeq y}{{y}!\langle{Q}\rangle \vdash x}
  \and
  \inferrule*[lab=Par-barb]{\mbox{$P\vdash x$ or $Q\vdash x$}}{\binpar{P}{Q} \vdash x}
\end{mathpar}

\subsubsection{Contexts}

One of the principle advantages of computational calculi like the
$\pi$-calculus is a well-defined notion of context,
contextual-equivalence and a correlation between
contextual-equivalence and notions of bisimulation. The notion of
context allows the decomposition of a process into (sub-)process and
its syntactic environment, its context. Thus, a context may be
thought of as a process with a ``hole'' (written $\Box$) in it. The
application of a context $M$ to a process $P$, written $M[P]$, is
tantamount to filling the hole in $M$ with $P$. In this paper we do
not need the full weight of this theory, but do make use of the notion
of context in the proof the main theorem. 

\begin{mathpar}
  \inferrule* [lab=summation] {} {{M_{M},M_{N}} \bc \Box \;|\; x.M_{A} \;|\; M_{M}+M_{N}}
  \and
  \inferrule* [lab=agent] {} {{M_{A}} \bc (\vec{x})M_{P} \;| \; \clift{P_0,\ldots,M_{P},\ldots,P_N}}
  \and \\
  \inferrule* [lab=process] {} {{M_{P}} \bc M_{N} \;| \;P|M_{P} }
\end{mathpar} 

\begin{mathpar}
  \inferrule* [lab=sychronization] {} {M_{N} \bc \Box \;|\; x?M_{F} \;|\; x!M_{C}}
  \and
  \inferrule* [lab=abstraction] {} {{M_{F}} \bc (x)M_{P} }
  \and
  \inferrule* [lab=concretion] {} {{M_{C}} \bc \langle M_{P} \rangle }
  \and \\
  \inferrule* [lab=process] {} {{M_{P}} \bc M_{N} \;| \;P|M_{P} }
\end{mathpar}

\begin{definition}[contextual application] Given a context $M$, and
  process $P$, we define the \emph{contextual application}, $M[P] :=
  M\{P/\Box\}$. That is, the contextual application of M to P is the
  substitution of $P$ for $\Box$ in $M$.
\end{definition}

$\meaningof{-} : L \to \mathcal{P}(\pi)$

\begin{mathpar}
  \inferrule* [lab=collection] {} {\meaningof{true} = \pi, \and \meaningof{~E} = \pi \setminus \meaningof{E}, \and \meaningof{E_{1} \& E_{2}} = \meaningof{E_{1}} \cap \meaningof{E_{2}}}
\end{mathpar}

\begin{mathpar}
  \inferrule* [lab=structure] {} {\meaningof{0} = \{ P \in \pi | P \equiv 0 \}, \and \\ \meaningof{E_1 | E_2} = \{ P \in \pi | P \equiv P_{1} | P_{2}, P_{1} \in \meaningof{E_{1}}, P_{2} \in \meaningof{E_2}\} }
\end{mathpar}

\begin{mathpar}
 \inferrule* [lab=behavior] {} {\meaningof{\langle a?b \rangle E} = \{ P \in \pi | P \equiv Q | u?(y)P', \\ \and \\\\ \and \\ \;\;\; u \in \meaningof{a}, \forall z.P'\{z/y\} \in \meaningof{E\{z/b\}}\}, \and \\ \meaningof{a!E} = \{ P \in \pi | P \equiv Q | x!\langle P' \rangle, x \in \meaningof{a} P' \in \meaningof{E}\} }
\end{mathpar}

\begin{mathpar}
 \inferrule* [lab=nominal] {} {\meaningof{\quotep{E}} = \{ \quotep{P} \in \quotep{\pi} | P \in \meaningof{E} \}, \and \meaningof{\quotep{P}} = \{ \quotep{Q} \in \quotep{\pi} | P \equiv Q \} \and \\ \meaningof{@\quotep{E}} = \{ P \in \pi | P \equiv @x, x \in \meaningof{E} \}}
\end{mathpar}

\begin{eqnarray*}
  \\
  \meaningof{-} : TS \to ST
\end{eqnarray*}

\begin{eqnarray*}
  \\
  L : TS \to ST
\end{eqnarray*}

\begin{eqnarray*}
  \\
  P \models E \iff P \in \meaningof{E}
\end{eqnarray*}

\begin{eqnarray*}
  P \approx_{L} Q \iff \forall E \in L. P \models E \iff Q \models E
\end{eqnarray*}

\begin{eqnarray*}
  P \approx_{K} Q
\end{eqnarray*}

\begin{eqnarray*}
  P \approx Q
\end{eqnarray*}

$\approx_{K} = \approx = \approx_{L}$

\subsubsection{Contextual duality}

Note that contexts extend the quotation operation to a family of
operations from processes to names. Given a context, $M$, we can
define a \emph{nominal context}, $\quotep{M}$ by $\quotep{M}[P] :=
\quotep{M[P]}$. To foreshadow what is to come we observe that these
operations enjoy a duality with processes very much like the duality
between vectors and maps from vectors to scalars.

Further, because the calculus is essentially higher-order, we have a
correspondence between contexts and processes. More specifically,
given a name $x$ and a context $M$ we can construct $M^{*}_{x}$ such
that 

\begin{mathpar}
  M^{*}_{x} | \lift{x}{P} \red M[P]
\end{mathpar}

namely,

\begin{mathpar}
  M^{*}_{x} := x?(u).M[\dropn{u}]
\end{mathpar}

The dependence of $M^{*}_{x}$ on a name makes it an abstraction, 

\begin{mathpar}
  M^{*} := (x)x?(u).M[\dropn{u}]
\end{mathpar}

\subsection{Additional notation}

It will sometimes be convenient to denote the process a name
quotes. We already have the notation $x = \quotep{P}$, but it will be
convenient to introduce an alternate notation, $\procn{x}$, when we
want to emphasize the connection to the use of the name. Note that, by
virtue of name equivalence, $\quotep{\procn{x}} \nameeq x$; so, the
notation is consistent with previous definitions.

Further, because names have structure it is possible to effect
substitutions on the basis of that structure. This means we need to
upgrade our notation for substitutions, which we accomplish by
adapting comprehension notation. Thus,

\begin{mathpar}
  P\{ y / x : x \in S \}
\end{mathpar}

is interpreted to mean the process derived from P by replacing (in a
capture-avoiding manner) each occurrence of $x$ in $S$ by $y$. For example,

\begin{mathpar}
  P\{ \quotep{\procn{x}|\procn{x}} / x : x \in \freenames{P} \}
\end{mathpar}

will replace each (occurrence) of a free name $x$ in $P$ by
$\quotep{\procn{x}|\procn{x}}$.

Also, we will avail ourselves of the notation $x^{L}$ and $x^{R}$ to
denote injections of a name into disjoint copies of the name
space. There are numerous ways to accomplish this. One example can be
found in \cite{MeredithR05}. This notation overloads to vectors of
names: $\vec{x}^{\pi} := (x_{i}^{\pi} \; : \; 0 \leq i < |\vec{x}| )$ where $\pi \in \{L,R\}$.

We also use $P^{\Box} := P|\Box$.

In \cite{MeredithR05} an interpretation of the new operator is
given. It turns out that there are several possible interpretations
all enjoying the requisite algebraic properties of the operator (see
\cite{milner91polyadicpi}). We will therefore make liberal use of
$(\nu\; \vec{x})P$.

% subsection the_syntax_and_semantics_of_the_notation_system (end)   

\input{qm2pi.qmops} 

\input{qm2pi.sterngerlach} 

\input{qm2pi.metric} 

% section concurrent_process_calculi (end)

%\input{qm2pi.proofsketch}

% section proof sketch (end)

%\input{qm2pi.slviaknots} 

% section spatial logic via knots (end)

\input{qm2pi.conclusion}

% section conclusion (end)

%\input{qm2pi.dtcodes} 

% section wiring algorithm (end)

\input{qm2pi.ack} 

% section acknowledgments (end)

\newpage


\bibliographystyle{plain}   
\bibliography{../../biblios/main.bib}

\input{qm2pi.rhodetails}

\end{document}



% section proof sketch (end)

%\section{Unlikely characters: spatial logic for
  knots}\label{sub:characteristic_formulae} % (fold)

Associated to the mobile process calculi are a family of logics known
as the Hennessy-Milner logics. These logics typically enjoy a
semantics interpreting formulae as sets of processes that when
factored through the encoding outlined above allows an identification
of classes of knots with logical formulae. In the context of this
encoding the sub-family known as the spatial logics \cite{CairesC03}
\cite{CairesC04} \cite{Caires04} are of particular interest providing
several important features for expressing and reasoning about
properties (i.e. classes) of knots. We hint here at how this may be done.

%\begin{description}
%\item [structural connectives] 
\subsubsection{Structural connectives} The spatial logics enjoy
structural connectives corresponding, at the logical level, to the
parallel composition ($P | Q$) and new name ($(\nu \; x)P$)
connectives for processes. As illustrated in the examples below, these
connectives are extremely expressive given the shape of our encoding.
%\item [decideable satisfaction]

\subsubsection{Decideable satisfaction}
In \cite{Caires04} the satisfaction relation is shown to be decideable
for a rich class of processes. It further turns out that the image of
the our encoding is a proper subset of that class. This result
provides the basis for an algorithm by which to search for knots
enjoying a given property.
%\item [characteristic formulae]

\subsubsection{Characteristic formulae}
In the same paper \cite{Caires04} , Caires presents a means of calculating
characteristic formulae, selecting equivalence classes of processes
up to a pre--specified depth limit on the support set of names. Composed with our
encoding, this characteristic formula can be used to select
characteristic formulae for knots.
%\end{description}

\subsubsection{Spatial logic formulae}

The grammar below (segmented for comprehension) summarizes the syntax
of spatial logic formulae. We employ illustrative examples in the
sequel to provide an intuitive understanding of their meaning
referring the reader to \cite{Caires04} for a more detailed explication
of the semantics.

\begin{mathpar}
  \inferrule* [lab=boolean] {} {{A,B} \bc T \;|\; \neg A \;|\; A \wedge B \;|\; \eta = \eta'}
  \and
  \inferrule* [lab=spatial] {} {|\; \pzero \;|\; A | B \;|\; x \text{\textregistered} A \;|\; \forall x . A \;|\;  H x . A}
  \and
  \inferrule* [lab=behavioral] {} {|\; \alpha . A}
  \and 
  \inferrule* [lab=recursion] {} {|\; X(\vec{u}) \;|\; \mu X(\vec{u}) . A}
  \and
  \inferrule* [lab=action] {} {\alpha \bc \langle x?(\vec{y}) \rangle \;|\; \langle x!(\vec{y}) \rangle \;|\; \langle \tau \rangle}
  \and 
  \inferrule* [lab=name] {} {\eta \bc x \;|\; \tau}
\end{mathpar} 

% subsection characteristic_formulae (end)   	 

\subsection{Example formulae}\label{sub:example_formulae_} % (fold)

\subsubsection{Crossing as formula.}
% 
% \begin{align*}
%   \frac{d}{dx} \sin x &= \cos x 
%   & \frac{d}{dx} e^x &= e^x \\
%   \frac{d}{dx} \cos x &= - \sin x 
%   & \frac{d}{dx} \log x &= \frac{1}{x} \\
% \end{align*} 

\begin{align*}
 \mu C(x_{0},x_{1},y_{0},y_{1},u).&(\langle x_{0}?(z) \rangle(\langle u! \rangle\langle y_{1}!z \rangle C(x_{0},x_{1},y_{0},y_{1},u)) & \\
  & \wedge \langle y_{1}?(z) \rangle (\langle u! \rangle \langle x_{0}!z \rangle C(x_{0},x_{1},y_{0},y_{1},u)) & \\
  & \wedge \langle x_{1}?(z) \rangle (\langle u? \rangle \langle y_{0}!z \rangle C(x_{0},x_{1},y_{0},y_{1},u)) & \\
  & \wedge \langle y_{0}?(z) \rangle (\langle u? \rangle \langle x_{1}!z \rangle C(x_{0},x_{1},y_{0},y_{1},u))) &
\end{align*}

The lexicographical similarity between the shape of this formulae and
the shape of definition of the process representing a crossing reveals
the intuitive meaning of this formulae. It describes the capabilities
of a process that has the right to represent a crossing. For example
it picks out processes that may perform an input on the port $x_0$ in
its initial menu of capabilities. What differentiates the formula
from the process, however, is that the crossing process is the
smallest candidate to satisfy the formula. Infinitely many other
processes -- with internal behavior hidden behind this interface, so
to speak -- also satisfy this formula. Even this simple formula,
then, can be seen to open a new view onto knots, providing a
computational interpretation of \emph{virtual} knots.

Note that this formula is derived by hand. A similar formula can be
derived by employing Caires' calculation of characteristic formula
\cite{Caires04} to the process representing a crossing. In light of
this discussion, we let
$\meaningof{C}_{\phi}(x0,x1,y0,y1,u)$ denote a formula specifying the
dynamics we wish to capture of a crossing. To guarantee we preserve
the shape of the interface and minimal semantics we demand that
$\meaningof{C}_{\phi}(x0,x1,y0,y1,u) \Rightarrow
\textbf{C}(x0,x1,y0,y1,u)$ where $\textbf{C}(x0,x1,y0,y1,u)$ denotes
the formula above.
                            
\subsubsection{Crossing number constraints.}
The moral content of the context lemma (Lemma \ref{context}) is that the notion of
``locality'' in the Reidemeister moves is effectively captured by the
parallel composition operator of the process calculus. This intuition
extends through the logic. Given a formula,
$\meaningof{C}_{\phi}(x0,x1,y0,y1,u)$, we can use the structural
connectives to specify constraints on crossing numbers, such as at
least $n$ crossings, or exactly $n$ crossings.
\begin{mathpar}
  \inferrule* [lab=at-least-n] {} { K^{\geq n}_{\phi}(\vec{xs},\vec{ys}) := \Pi_{i=0}^{n-1} Hu . \meaningof{C}_{\phi}(xs_i,ys_i,u) | T }
  \and 
  \inferrule* [lab=exactly-n] {} { K^{= n}_{\phi}(\vec{xs},\vec{ys}) := \Pi_{i=0}^{n-1} Hu . \meaningof{C}_{\phi}(xs_i,ys_i,u) | \neg (\forall x_0,y_0,x_1,y_1,u . \meaningof{C}_{\phi}(x_0,y_0,x_1,y_1,u) | T) }
\end{mathpar}

To round out this section, recall that the encoding of an $n$-crossing
knot decomposes into a parallel composition of $n$ \emph{copies} of a
crossing process together with a wiring harness. To specify different
knot classes with the same crossing number amounts to specifying
logical constraints on the wiring harness. In the interest of space,
we defer examples to a forthcoming paper. Suffice it to say that both
the conditions ``alternating knot'' and ``contains the tangle
corresponding to 5/3'' are expressible. For example, it is possible to
calculate the characteristic formula of a process corresponding to the
tangle 5/3 and conjoin it into the classifying formula via the
composition connective of the logic.

Finally, we wish to observe that it is entirely within reason to
contemplate a more domain-specific version of spatial logic tailored
to the shape of processes in the image of the encoding. Such a
domain-specific logic would have a better claim to the title formal
language of knot properties.

% subsection example_formulae_ (end)

% section knots_as_processes (end) 

% section spatial logic via knots (end)

\section{Conclusions and future work}

\paragraph{Testing physical space}
You, gentle reader, may wonder why of all the theorems to be proved
given this set up we pick the one above. In some sense it's hardly
central to quantum mechanics. We see it as central in the sense that
it firmly establishes a notion of physical space arising from a notion
of the equivalence of behavior. Relating bisimulation to a metric is a
big step forward, but one is faced with interpreting the relationship
of that metric space to something more physical. Quantum mechanical
notions of ``physical'' space are still far from intuitive, but by
relating this idea of distance as testing to calculations that predict
physical circumstances we are making a not insignificant step forward
toward an understanding of the physical space we inhabit as
essentially dynamic.

\paragraph{Effectivity and simulation}
One of the observations we have yet to make is that the entire program
spelled out here is effective. We have built various interpreters for
the reflective calculus at work in this interpretation. In principle,
then, we can simulate quantum mechanics on a computer. The place where
the simulation may lose fidelity is the infinitely branching summation
for the annihilator.

In this connection i also want to point out that the evaluation style
calculation of the inner product puts the non-determinism of the
summation right at the heart of measurement. This suggests that
Milner's original reduction-based formulation of the dynamics of his
calculi in terms of sums was not just notationally suggestive of a
notion of measure-and-continue but captured some significant part of
the physics.

\paragraph{Quantum continuations}
In light of this last observation i want to point out that the
predominant account of quantum mechanics is missing a key aspect of a
truly compositional story of the physical situation. In a real lab,
when a measurement is made the observation can be made to feed into
another device that then makes another measurement conditioned on the
results of the first. This means that after the superposition was
collapsed the entire experimental set up remained in
superposition. While QM offers a means of writing this down it doesn't
quite line up well with the well-trodden formulation of computation
and continuation that we see so succinctly expressed in Milner's
calculi. This suggests that there might be advantages to this account
of dynamics waiting to be explored.

\paragraph{Quantum logic}
In this connection, we also note that by virtue of having the
Hennessy-Milner construction, we can pull the construction through the
interpretation of QM. This gives us a natural candidate for a quantum
logic that enjoys an extremely tight connection with it's domain of
interpretation, making the construction much less ad hoc (rather it is
the image of functor!).

\paragraph{Quantum probabiity}
i have questions about the basis of the interpretation of inner
product as probability amplitude. In particular, using which
axiomatization of probability theory does the notion of probability
amplitude earn the right to be so dubbed? In other words, where is the
proof that the operation for calculating a probability amplitude (and
then squaring) satisfies the axioms of what it means to calculate a
probability? Even if such a proof exists (i have yet to find it in the
literature), i wonder if it might not be possible to turn things on
their heads. Can we view the calculation of the probability amplitude
as an axiomatization of probability? If so, then the definition we
give for calculating probability amplitude may provide the basis for
an \emph{effective} theory of probability.

\paragraph{Quantum vs ``biological'' information}
Finally, i want to conclude with a more philosophical observation. At
a recent workshop in which QM was a predominant topic i noticed
something about quantum information. The speaker was giving a riveting
discussion of axiomatic QM and showing how properties of ``no
cloning'' and ``no deleting'' emerged as consequences of the
axiomatization. Theorems of this form are necessary to give us a sense
of confidence that our axioms characterize the physical theory. What
struck me, though, was that if quantum information is neither erasable
nor replicable it is markedly different from \emph{life}. Two of the
things we know about life is that

\begin{itemize}
  \item it ends;
  \item to gain some measure of persistence, to transcend it's
    finitude it is imminently copyable.
\end{itemize}

Both of these qualities are summarized succinctly in the aphorism: all
flesh is grass. For me these two kinds of ``information'' -- call them
quantum and biological -- are end points on a spectrum of strategies
for persistence. At one end, we have those curious entities that enjoy
uniqueness and permanence; at the other, we have those who in the face
of a certain end and an uncertain present make a go of passing
something on. To me one of the more remarkable aspects of the latter
strategy is that in the presence of noise (and certain features of
copying) we get a kind of dynamism, a chance for improvement against a
given persistent condition.

% subsection other_calculi_other_bisimulations_and_geometry_as_behavior (end)




% section conclusion (end)

%\documentclass[12pt]{llncs}
%\documentclass{jktr}

\usepackage[pdftex]{hyperref}                   
\usepackage {listings}
\usepackage {mathpartir}
\usepackage{bcprules}
%\usepackage{listings}
                       
\usepackage{graphicx} 
%\usepackage[margins=2.5cm,nohead,nofoot]{geometry}
%\usepackage{geometry}
\usepackage{amsfonts}
\usepackage{amstext}
\usepackage{latexsym}
\usepackage{amssymb}
\usepackage{color}


%\include{myPreamble}
\include{qm2pi.local} 

%\ifpdf
%\usepackage[pdftex]{graphicx}
%\else
%\usepackage{graphicx}
%\fi

 % \ifpdf
%  \usepackage{pdfsync}
%  \if


%\title{Brief Article}
%\author{David F. Snyder}
%\author{L.G. Meredith}

%\address{Dept. of Math., Texas State University--San Marcos, San Marcos, TX 78666}
       
\pagestyle{empty}


\begin{document}

\lstset{language=[Objective]Caml,frame=shadowbox}

\input{qm2pi.front}

% section front matter (end)

\input{qm2pi.intro} 
 
% section introduction (end)

% \input{qm2pi.knotations} 

% section notation (end)

\input{qm2pi.process.calculi} 

% section concurrent_process_calculi_and_spatial_logics_ (end)
    
%\input{qm2pi.knots2pi} 

%\input{qm2pi.trefoil} 

%\input{qm2pi.mainthm} 

% subsection basic_interpretation (end)

%\input{qm2pi.rho.presentation} 
\subsection{The syntax and semantics of the notation system}\label{sub:the_syntax_and_semantics_of_the_notation_system} % (fold)

We now summarize a technical presentation of the calculus that
embodies our theory of dynamics. The typical presentation of such a
calculus follows the style of giving generators and relations on
them. The grammar, below, describing term constructors, freely
generates the set of processes, $\Proc$. This set is then quotiented
by a relation known as structural congruence and it is over this set
that the notion of dynamics is expressed. This presentation is
essentially that of \cite{MeredithR05} with the addition of
polyadicity and summation. For readability we have relegated some of
the technical subtleties to an appendix.

\subsubsection{Process grammar}\label{subsub:process_grammar}

\begin{mathpar}
  \inferrule* [lab=synchronization] {} {{M} \bc \pzero \;|\; x?F \;|\; x!C }
  \and
  \inferrule* [lab=abstraction] {} {{F} \bc (x)P}
  \and
  \inferrule* [lab=concretion] {} {{C} \bc \langle Q \rangle}
  \and
  \inferrule* [lab=process] {} {{P,Q} \bc M \;| \;P|Q \;|\; @{x}}
  \and
  \inferrule* [lab=name] {} {{x} \bc \quotep{P}}
\end{mathpar} 

Note that $\vec{x}$ (resp. $\vec{P}$) denotes a vector of names
(resp. processes) of length $|\vec{x}|$ (resp. $|\vec{P}|$). We adopt
the following useful abbreviations.

\begin{mathpar}
   x?(\vec{y}).P := x.(\vec{y})P \and  x\clift{\vec{P}} := x.\clift{\vec{P}}
   \and x!(y) := \lift{x}{\dropn{y}}
   \and \Pi_{i=0}^{n-1}P_i := P_0 | \ldots | P_{n-1}
\end{mathpar}

\subsubsection{Structural congruence}

\paragraph{Free and bound names and alpha-equivalence.} At the
core of structural equivalence is alpha-equivalence which identifies
process that are the same up to a change of variable. Formally, we
recognize the distinction between free and bound names. The free names
of a process, $\freenames{P}$, may be calculated recursively as
follows:

\begin{mathpar}
\freenames{\pzero} := \emptyset
  \and \\
  \freenames{x?(y).P} := \{ x \} \cup (\freenames{P} \setminus \{ y \})
  \and 
  \freenames{x!\langle P \rangle} := \{ x \} \cup \{ P \} 
  \and \\
  \freenames{P|Q} := \freenames{P} \cup \freenames{Q}
  \and \\
  \freenames{@{x}} := \{ x \}
\end{mathpar}

$\pi$
$\quotep{\pi}$

$\freenames{-} : \pi \to \mathcal{P}(\quotep{\pi})$

\begin{eqnarray*}
  \freenames{\pzero} & := & \emptyset \\
  \freenames{x?(y).P} & := & \{ x \} \cup (\freenames{P} \setminus \{ y \}) \\
  \freenames{x!\langle P \rangle} & := & \{ x \} \cup \{ P \} \\
  \freenames{P|Q} & := & \freenames{P} \cup \freenames{Q} \\
  \freenames{\dropn{x}} & := & \{ x \}
\end{eqnarray*}

The bound names of a process, $\boundnames{P}$, are those names occurring in $P$
that are not free. For example, in $x?(y).0$, the name $x$ is free, while $y$ is bound.

\begin{mathpar}
  \inferrule* [lab=monoidal-laws] {} { P|Q \equiv Q|P \and P|0 \equiv P \and P|(Q|R) \equiv (P|Q)|R }
\end{mathpar}

\begin{mathpar}
  \inferrule* [lab=alpha-equivalence] {} { (x)P \equiv (y)P\{y/x\} \and y \not\in \freenames{P} }
\end{mathpar}

\begin{definition}
Then two processes, $P,Q$, are alpha-equivalent if $P = Q\{\vec{y}/\vec{x}\}$ for
some $\vec{x} \in \boundnames{Q},\vec{y} \in \boundnames{P}$, where $Q\{\vec{y}/\vec{x}\}$
denotes the capture-avoiding substitution of $\vec{y}$ for $\vec{x}$ in $Q$.
\end{definition}

\begin{definition}
  The {\em structural congruence} \cite{SangiorgiWalker} , $\equiv$,
  between processes is the least congruence containing
  alpha-equivalence, satisfying the abelian monoid laws
  (associativity, commutativity and $\pzero$ as identity) for parallel
  composition $|$ and for summation $+$.
\end{definition}

\subsection{Name equivalence}

We take name equivalence, written $\nameeq$, to be the smallest
equivalence relation generated by the following rules.

\begin{mathpar}
\inferrule*[lab=Quote-drop]
{ }
{ \quotep{@{x}} \nameeq x }

\inferrule*[lab=Struct-equiv]
{ P \scong Q }
{ \quotep{P} \nameeq \quotep{Q} }
\end{mathpar}

The astute reader will have noticed that the mutual recursion of names
and processes imposes a mutual recursion on alpha-equivalence and
structural equivalence via name-equivalence. Fortunately, all of this
works out pleasantly and we may calculate in the natural way, free of
concern. The reader interested in the details is referred to the
appendix \ref{appendix:rho_details}.

\subsection{Substitution}

We use $\Proc$ for the set of processes, $\QProc$ for the set of
names, and $\id{\{}\vec{y} / \vec{x} \id{\}}$ to denote partial maps,
$s : \QProc \rightarrow \QProc$. A map, $s$ lifts, uniquely, to a map
on process terms, $\widehat{s} : \Proc \rightarrow \Proc$ by the
following equations.

\begin{mathpar}
  (0) \psubstp{Q}{P} := 0 \\
  (R \juxtap S) \psubstp{Q}{P}
  :=    
  (R)\psubstp{Q}{P} \juxtap (S) \psubstp{Q}{P} \\
  (x?(y).R) \psubstp{Q}{P}    
  :=    
  (x)\substp{Q}{P} (z)\concat( (R \psubstn{z}{y}) \psubstp{Q}{P} ) \\
  (\lift{x}{R}) \psubstp{Q}{P}  
  :=
  \lift{(x)\substp{Q}{P}}{ R \psubstp{Q}{P} } \\
%   (\dropn{x})  \psubstp{Q}{P}       
%   := 
%   \left\{ 
%     \begin{array}{ccc} 
%       \dropn{\quotep{Q}} & & x \nameeq \quotep{P} \\
%       \dropn{x} & & otherwise \\
%     \end{array}
%   \right. 
  (\dropn{x})  \psubstp{Q}{P}       
  := 
  \left\{ 
    \begin{array}{ccc} 
      Q & & x \nameeq \quotep{P} \\
      \dropn{x} & & otherwise \\
    \end{array}
  \right.
\end{mathpar}
 

where

\begin{eqnarray}
  (x)\id{\{} \lpquote Q \rpquote / \lpquote P \rpquote \id{\}}            = 
  \left\{ 
    \begin{array}{ccc}
      \lpquote Q \rpquote & & x \nameeq \lpquote P \rpquote \\
      x & & otherwise \\
    \end{array}
  \right. \nonumber
\end{eqnarray}

and $z$ is chosen distinct from $\quotep{P}$, $\quotep{Q}$, the free
names in $Q$, and all the names in $R$. Our $\alpha$-equivalence will
be built in the standard way from this substitution.

\begin{remark}\label{rem:no_self_referential_names}
  One consequence of these definitions is that $\forall P. \quotep{P}
  \not\in \freenames{P}$.
\end{remark}

\subsection{ Dynamic quote: an example }

Anticipating something of what's to come, consider applying the
substitution, $\widehat{\id{\{}u / z \id{\}}}$, to the following pair
of processes, $\lift{w}{y!(z)}$ and $w[ \lpquote y!(z) \rpquote ]$.

\begin{eqnarray}
	\lift{w}{y!(z)}\widehat{\id{\{}u / z \id{\}}}
		& = &
		\lift{w}{y!(u)} \nonumber\\
	w[ \lpquote y!(z) \rpquote ] \widehat{ \id{\{}u / z \id{\}} }
		& = &
		w[ \lpquote y!(z) \rpquote ] \nonumber
\end{eqnarray}

Because the body of the process between quotes is impervious to
substitution, we get radically different answers. In fact, by
examining the first process in an input context,
e.g. $x?(z).\lift{w}{y!(z)}$, we see that the process under the lift
operator may be shaped by prefixed inputs binding a name inside it. In
this sense, the lift operator will be seen as a way to dynamically
construct processes before reifying them as names.

Finally equipped with these standard features we can present the
dynamics of the calculus.

\subsubsection{Operational semantics} 

Finally, we introduce the computational dynamics. What marks these
algebras as distinct from other more traditionally studied algebraic
structures, e.g. vector spaces or polynomial rings, is the manner in
which dynamics is captured. In traditional structures, dynamics is typically
expressed through morphisms between such structures, as in linear maps
between vector spaces or morphisms between rings. In algebras
associated with the semantics of computation, the dynamics is
expressed as part of the algebraic structure itself, through a
reduction reduction relation typically denoted by $\red$. Below, we
give a recursive presentation of this relation for the calculus used
in the encoding.

$\red \subseteq \pi \times \pi$
$\red : \pi \to \mathcal{P}(\pi)$

\begin{mathpar}
  \inferrule* [lab=Comm] { \textsf{match}( x_{src}, x_{trgt} ) } { x_{trgt}?(y)P \; | \; x_{src}!\langle {Q} \rangle \red P\{\quotep{Q}/y}\} }
  \and \\
  \inferrule* [lab=Par] {{P} \red {P}'} {{{P} | {Q}} \red {{P}' | {Q}}}
  \and
  \inferrule* [lab=Equiv]{{{P} \scong {P}'} \andalso {{P}' \red {Q}'} \andalso {{Q}' \scong {Q}}}{{P} \red {Q}}
\end{mathpar}

\begin{eqnarray*}
  match_{\equiv} (\quotep{P},\quotep{Q}) & := & P \equiv Q \\
  match_{\dagger}(\quotep{P},\quotep{Q}) & := & \forall R. P|Q \red^{*} R => R \red^{*} 0 \\
  match_{K}(\quotep{P},\quotep{Q}) & := & K \mbox{ for some context } K
\end{eqnarray*}

$u?(x)P | u!\langle Q \rangle \red P\{\quotep{Q}/x\}$

%We write $\wred$ for $\red^*$, and $P\red$ if $\exists Q $ such that $ P \red Q$.
We write $P\red$ if $\exists Q $ such that $ P \red Q$ and $P\not\red$, otherwise.

\section{Replication}

As mentioned before, it is known that replication (and hence
recursion) can be implemented in a higher-order process algebra
\cite{SangiorgiWalker}. As our first example of calculation with the
machinery thus far presented we give the construction explicitly in
the {\rhoc}.

\begin{eqnarray}
	D_{x} & := & \prefix{x}{y}{(\binpar{\outputp{x}{y}}{@{y}})} \nonumber\\
	\bangp_{x}{P} & := & \binpar{{x}!\langle{\binpar{D_{x}}{P}}\rangle}{D_{x}} \nonumber
\end{eqnarray}

\begin{eqnarray}
	\bangp_{x}{P} & & \nonumber\\
	=
	& {x}!\langle{(\prefix{x}{y}{(\outputp{x}{y} | @{y})) | P}}\rangle 
	      | \prefix{x}{y}{(\outputp{x}{y} | @{y})} & \nonumber\\
	\red
	& (\outputp{x}{y} | @{y})\substn{\quotep{(\prefix{x}{y}{(@{y} | \outputp{x}{y})) | P}}}{y} & \nonumber\\
	=
	& \outputp{x}{\quotep{(\prefix{x}{y}{(\outputp{x}{y} | @{y})) | P}}}
	  | {(\prefix{x}{y}{(\outputp{x}{y} | @{y})) | P}} & \nonumber\\
	\red
	& \ldots & \nonumber\\
	\red^*
	& P | P | \ldots & \nonumber
\end{eqnarray}

Of course, this encoding, as an implementation, runs away, unfolding
$\bangp{P}$ eagerly. A lazier and more implementable replication
operator, restricted to input-guarded processes, may be obtained as follows.

\begin{eqnarray}
\bangp{\prefix{u}{v}{P}} 
	:= 
	\binpar{\lift{x}{\prefix{u}{v}{(\binpar{D(x)}{P})}}}{D(x)} \nonumber
\end{eqnarray}

\begin{remark}
  Note that the lazier definition still does not deal with summation
  or mixed summation (i.e. sums over input and output). The reader is
  invited to construct definitions of replication that deal with these
  features. 

  Further, the definitions are parameterized in a name, $x$. Can you,
  gentle reader, make a definition that eliminates this parameter and
  guarantees no accidental interaction between the replication
  machinery and the process being replicated -- i.e. no accidental
  sharing of names used by the process to get its work done and the
  name(s) used by the replication to effect copying. This latter
  revision of the definition of replication is crucial to obtaining
  the expected identity $!!P \sim !P$.
\end{remark}

\begin{remark}\label{rem:paradoxical_combinator}
  The reader familiar with the lambda calculus will have noticed the
  similarity between $D$ and the paradoxical combinator.

  [Ed. note: the existence of this seems to suggest we have to be more
  restrictive on the set of processes and names we admit if we are to
  support no-cloning.]
\end{remark}

\subsubsection{Bisimulation}

The computational dynamics gives rise to another kind of equivalence,
the equivalence of computational behavior. As previously mentioned
this is typically captured \emph{via} some form of bisimulation.

% The notion we use in this paper is weak barbed bisimulation
% \cite{milner91polyadicpi}.

The notion we use in this paper is derived from weak barbed
bisimulation \cite{milner91polyadicpi}. 

\begin{definition}
An \emph{observation relation}, $\downarrow_{\mathcal N}$, over a set
of names, $\mathcal N$, is the smallest relation satisfying the rules
below.

\infrule[Out-barb]{y \in {\mathcal N}, \; x \nameeq y}
		  {\outputp{x}{v} \downarrow_{\mathcal N} x}
\infrule[Par-barb]{\mbox{$P\downarrow_{\mathcal N} x$ or $Q\downarrow_{\mathcal N} x$}}
		  {\binpar{P}{Q} \downarrow_{\mathcal N} x}

We write $P \Downarrow_{\mathcal N} x$ if there is $Q$ such that 
$P \wred Q$ and $Q \downarrow_{\mathcal N} x$.
\end{definition}

\begin{definition}
%\label{def.bbisim}
An  ${\mathcal N}$-\emph{barbed bisimulation} over a set of names, ${\mathcal N}$, is a symmetric binary relation 
${\mathcal S}_{\mathcal N}$ between agents such that $P\rel{S}_{\mathcal N}Q$ implies:
\begin{enumerate}
\item If $P \red P'$ then $Q \wred Q'$ and $P'\rel{S}_{\mathcal N} Q'$.
\item If $P\downarrow_{\mathcal N} x$, then $Q\Downarrow_{\mathcal N} x$.
\end{enumerate}
$P$ is ${\mathcal N}$-barbed bisimilar to $Q$, written
$P \wbbisim_{\mathcal N} Q$, if $P \rel{S}_{\mathcal N} Q$ for some ${\mathcal N}$-barbed bisimulation ${\mathcal S}_{\mathcal N}$.
\end{definition}

$\mathcal{R} \subseteq \pi \times \pi$

$P \mathcal{R} Q => \forall P'. P \red P' \Rightarrow \exists Q'. Q \red Q', P' \mathcal{R} Q'$

$P \vdash x \Rightarrow Q \vdash x$

\begin{mathpar}
  \inferrule*[lab=Out-barb]{x \nameeq y}{{y}!\langle{Q}\rangle \vdash x}
  \and
  \inferrule*[lab=Par-barb]{\mbox{$P\vdash x$ or $Q\vdash x$}}{\binpar{P}{Q} \vdash x}
\end{mathpar}

\subsubsection{Contexts}

One of the principle advantages of computational calculi like the
$\pi$-calculus is a well-defined notion of context,
contextual-equivalence and a correlation between
contextual-equivalence and notions of bisimulation. The notion of
context allows the decomposition of a process into (sub-)process and
its syntactic environment, its context. Thus, a context may be
thought of as a process with a ``hole'' (written $\Box$) in it. The
application of a context $M$ to a process $P$, written $M[P]$, is
tantamount to filling the hole in $M$ with $P$. In this paper we do
not need the full weight of this theory, but do make use of the notion
of context in the proof the main theorem. 

\begin{mathpar}
  \inferrule* [lab=summation] {} {{M_{M},M_{N}} \bc \Box \;|\; x.M_{A} \;|\; M_{M}+M_{N}}
  \and
  \inferrule* [lab=agent] {} {{M_{A}} \bc (\vec{x})M_{P} \;| \; \clift{P_0,\ldots,M_{P},\ldots,P_N}}
  \and \\
  \inferrule* [lab=process] {} {{M_{P}} \bc M_{N} \;| \;P|M_{P} }
\end{mathpar} 

\begin{mathpar}
  \inferrule* [lab=sychronization] {} {M_{N} \bc \Box \;|\; x?M_{F} \;|\; x!M_{C}}
  \and
  \inferrule* [lab=abstraction] {} {{M_{F}} \bc (x)M_{P} }
  \and
  \inferrule* [lab=concretion] {} {{M_{C}} \bc \langle M_{P} \rangle }
  \and \\
  \inferrule* [lab=process] {} {{M_{P}} \bc M_{N} \;| \;P|M_{P} }
\end{mathpar}

\begin{definition}[contextual application] Given a context $M$, and
  process $P$, we define the \emph{contextual application}, $M[P] :=
  M\{P/\Box\}$. That is, the contextual application of M to P is the
  substitution of $P$ for $\Box$ in $M$.
\end{definition}

$\meaningof{-} : L \to \mathcal{P}(\pi)$

\begin{mathpar}
  \inferrule* [lab=collection] {} {\meaningof{true} = \pi, \and \meaningof{~E} = \pi \setminus \meaningof{E}, \and \meaningof{E_{1} \& E_{2}} = \meaningof{E_{1}} \cap \meaningof{E_{2}}}
\end{mathpar}

\begin{mathpar}
  \inferrule* [lab=structure] {} {\meaningof{0} = \{ P \in \pi | P \equiv 0 \}, \and \\ \meaningof{E_1 | E_2} = \{ P \in \pi | P \equiv P_{1} | P_{2}, P_{1} \in \meaningof{E_{1}}, P_{2} \in \meaningof{E_2}\} }
\end{mathpar}

\begin{mathpar}
 \inferrule* [lab=behavior] {} {\meaningof{\langle a?b \rangle E} = \{ P \in \pi | P \equiv Q | u?(y)P', \\ \and \\\\ \and \\ \;\;\; u \in \meaningof{a}, \forall z.P'\{z/y\} \in \meaningof{E\{z/b\}}\}, \and \\ \meaningof{a!E} = \{ P \in \pi | P \equiv Q | x!\langle P' \rangle, x \in \meaningof{a} P' \in \meaningof{E}\} }
\end{mathpar}

\begin{mathpar}
 \inferrule* [lab=nominal] {} {\meaningof{\quotep{E}} = \{ \quotep{P} \in \quotep{\pi} | P \in \meaningof{E} \}, \and \meaningof{\quotep{P}} = \{ \quotep{Q} \in \quotep{\pi} | P \equiv Q \} \and \\ \meaningof{@\quotep{E}} = \{ P \in \pi | P \equiv @x, x \in \meaningof{E} \}}
\end{mathpar}

\begin{eqnarray*}
  \\
  \meaningof{-} : TS \to ST
\end{eqnarray*}

\begin{eqnarray*}
  \\
  L : TS \to ST
\end{eqnarray*}

\begin{eqnarray*}
  \\
  P \models E \iff P \in \meaningof{E}
\end{eqnarray*}

\begin{eqnarray*}
  P \approx_{L} Q \iff \forall E \in L. P \models E \iff Q \models E
\end{eqnarray*}

\begin{eqnarray*}
  P \approx_{K} Q
\end{eqnarray*}

\begin{eqnarray*}
  P \approx Q
\end{eqnarray*}

$\approx_{K} = \approx = \approx_{L}$

\subsubsection{Contextual duality}

Note that contexts extend the quotation operation to a family of
operations from processes to names. Given a context, $M$, we can
define a \emph{nominal context}, $\quotep{M}$ by $\quotep{M}[P] :=
\quotep{M[P]}$. To foreshadow what is to come we observe that these
operations enjoy a duality with processes very much like the duality
between vectors and maps from vectors to scalars.

Further, because the calculus is essentially higher-order, we have a
correspondence between contexts and processes. More specifically,
given a name $x$ and a context $M$ we can construct $M^{*}_{x}$ such
that 

\begin{mathpar}
  M^{*}_{x} | \lift{x}{P} \red M[P]
\end{mathpar}

namely,

\begin{mathpar}
  M^{*}_{x} := x?(u).M[\dropn{u}]
\end{mathpar}

The dependence of $M^{*}_{x}$ on a name makes it an abstraction, 

\begin{mathpar}
  M^{*} := (x)x?(u).M[\dropn{u}]
\end{mathpar}

\subsection{Additional notation}

It will sometimes be convenient to denote the process a name
quotes. We already have the notation $x = \quotep{P}$, but it will be
convenient to introduce an alternate notation, $\procn{x}$, when we
want to emphasize the connection to the use of the name. Note that, by
virtue of name equivalence, $\quotep{\procn{x}} \nameeq x$; so, the
notation is consistent with previous definitions.

Further, because names have structure it is possible to effect
substitutions on the basis of that structure. This means we need to
upgrade our notation for substitutions, which we accomplish by
adapting comprehension notation. Thus,

\begin{mathpar}
  P\{ y / x : x \in S \}
\end{mathpar}

is interpreted to mean the process derived from P by replacing (in a
capture-avoiding manner) each occurrence of $x$ in $S$ by $y$. For example,

\begin{mathpar}
  P\{ \quotep{\procn{x}|\procn{x}} / x : x \in \freenames{P} \}
\end{mathpar}

will replace each (occurrence) of a free name $x$ in $P$ by
$\quotep{\procn{x}|\procn{x}}$.

Also, we will avail ourselves of the notation $x^{L}$ and $x^{R}$ to
denote injections of a name into disjoint copies of the name
space. There are numerous ways to accomplish this. One example can be
found in \cite{MeredithR05}. This notation overloads to vectors of
names: $\vec{x}^{\pi} := (x_{i}^{\pi} \; : \; 0 \leq i < |\vec{x}| )$ where $\pi \in \{L,R\}$.

We also use $P^{\Box} := P|\Box$.

In \cite{MeredithR05} an interpretation of the new operator is
given. It turns out that there are several possible interpretations
all enjoying the requisite algebraic properties of the operator (see
\cite{milner91polyadicpi}). We will therefore make liberal use of
$(\nu\; \vec{x})P$.

% subsection the_syntax_and_semantics_of_the_notation_system (end)   

\input{qm2pi.qmops} 

\input{qm2pi.sterngerlach} 

\input{qm2pi.metric} 

% section concurrent_process_calculi (end)

%\input{qm2pi.proofsketch}

% section proof sketch (end)

%\input{qm2pi.slviaknots} 

% section spatial logic via knots (end)

\input{qm2pi.conclusion}

% section conclusion (end)

%\input{qm2pi.dtcodes} 

% section wiring algorithm (end)

\input{qm2pi.ack} 

% section acknowledgments (end)

\newpage


\bibliographystyle{plain}   
\bibliography{../../biblios/main.bib}

\input{qm2pi.rhodetails}

\end{document}

 

% section wiring algorithm (end)

\documentclass[12pt]{llncs}
%\documentclass{jktr}

\usepackage[pdftex]{hyperref}                   
\usepackage {listings}
\usepackage {mathpartir}
\usepackage{bcprules}
%\usepackage{listings}
                       
\usepackage{graphicx} 
%\usepackage[margins=2.5cm,nohead,nofoot]{geometry}
%\usepackage{geometry}
\usepackage{amsfonts}
\usepackage{amstext}
\usepackage{latexsym}
\usepackage{amssymb}
\usepackage{color}


%\include{myPreamble}
\include{qm2pi.local} 

%\ifpdf
%\usepackage[pdftex]{graphicx}
%\else
%\usepackage{graphicx}
%\fi

 % \ifpdf
%  \usepackage{pdfsync}
%  \if


%\title{Brief Article}
%\author{David F. Snyder}
%\author{L.G. Meredith}

%\address{Dept. of Math., Texas State University--San Marcos, San Marcos, TX 78666}
       
\pagestyle{empty}


\begin{document}

\lstset{language=[Objective]Caml,frame=shadowbox}

\input{qm2pi.front}

% section front matter (end)

\input{qm2pi.intro} 
 
% section introduction (end)

% \input{qm2pi.knotations} 

% section notation (end)

\input{qm2pi.process.calculi} 

% section concurrent_process_calculi_and_spatial_logics_ (end)
    
%\input{qm2pi.knots2pi} 

%\input{qm2pi.trefoil} 

%\input{qm2pi.mainthm} 

% subsection basic_interpretation (end)

%\input{qm2pi.rho.presentation} 
\subsection{The syntax and semantics of the notation system}\label{sub:the_syntax_and_semantics_of_the_notation_system} % (fold)

We now summarize a technical presentation of the calculus that
embodies our theory of dynamics. The typical presentation of such a
calculus follows the style of giving generators and relations on
them. The grammar, below, describing term constructors, freely
generates the set of processes, $\Proc$. This set is then quotiented
by a relation known as structural congruence and it is over this set
that the notion of dynamics is expressed. This presentation is
essentially that of \cite{MeredithR05} with the addition of
polyadicity and summation. For readability we have relegated some of
the technical subtleties to an appendix.

\subsubsection{Process grammar}\label{subsub:process_grammar}

\begin{mathpar}
  \inferrule* [lab=synchronization] {} {{M} \bc \pzero \;|\; x?F \;|\; x!C }
  \and
  \inferrule* [lab=abstraction] {} {{F} \bc (x)P}
  \and
  \inferrule* [lab=concretion] {} {{C} \bc \langle Q \rangle}
  \and
  \inferrule* [lab=process] {} {{P,Q} \bc M \;| \;P|Q \;|\; @{x}}
  \and
  \inferrule* [lab=name] {} {{x} \bc \quotep{P}}
\end{mathpar} 

Note that $\vec{x}$ (resp. $\vec{P}$) denotes a vector of names
(resp. processes) of length $|\vec{x}|$ (resp. $|\vec{P}|$). We adopt
the following useful abbreviations.

\begin{mathpar}
   x?(\vec{y}).P := x.(\vec{y})P \and  x\clift{\vec{P}} := x.\clift{\vec{P}}
   \and x!(y) := \lift{x}{\dropn{y}}
   \and \Pi_{i=0}^{n-1}P_i := P_0 | \ldots | P_{n-1}
\end{mathpar}

\subsubsection{Structural congruence}

\paragraph{Free and bound names and alpha-equivalence.} At the
core of structural equivalence is alpha-equivalence which identifies
process that are the same up to a change of variable. Formally, we
recognize the distinction between free and bound names. The free names
of a process, $\freenames{P}$, may be calculated recursively as
follows:

\begin{mathpar}
\freenames{\pzero} := \emptyset
  \and \\
  \freenames{x?(y).P} := \{ x \} \cup (\freenames{P} \setminus \{ y \})
  \and 
  \freenames{x!\langle P \rangle} := \{ x \} \cup \{ P \} 
  \and \\
  \freenames{P|Q} := \freenames{P} \cup \freenames{Q}
  \and \\
  \freenames{@{x}} := \{ x \}
\end{mathpar}

$\pi$
$\quotep{\pi}$

$\freenames{-} : \pi \to \mathcal{P}(\quotep{\pi})$

\begin{eqnarray*}
  \freenames{\pzero} & := & \emptyset \\
  \freenames{x?(y).P} & := & \{ x \} \cup (\freenames{P} \setminus \{ y \}) \\
  \freenames{x!\langle P \rangle} & := & \{ x \} \cup \{ P \} \\
  \freenames{P|Q} & := & \freenames{P} \cup \freenames{Q} \\
  \freenames{\dropn{x}} & := & \{ x \}
\end{eqnarray*}

The bound names of a process, $\boundnames{P}$, are those names occurring in $P$
that are not free. For example, in $x?(y).0$, the name $x$ is free, while $y$ is bound.

\begin{mathpar}
  \inferrule* [lab=monoidal-laws] {} { P|Q \equiv Q|P \and P|0 \equiv P \and P|(Q|R) \equiv (P|Q)|R }
\end{mathpar}

\begin{mathpar}
  \inferrule* [lab=alpha-equivalence] {} { (x)P \equiv (y)P\{y/x\} \and y \not\in \freenames{P} }
\end{mathpar}

\begin{definition}
Then two processes, $P,Q$, are alpha-equivalent if $P = Q\{\vec{y}/\vec{x}\}$ for
some $\vec{x} \in \boundnames{Q},\vec{y} \in \boundnames{P}$, where $Q\{\vec{y}/\vec{x}\}$
denotes the capture-avoiding substitution of $\vec{y}$ for $\vec{x}$ in $Q$.
\end{definition}

\begin{definition}
  The {\em structural congruence} \cite{SangiorgiWalker} , $\equiv$,
  between processes is the least congruence containing
  alpha-equivalence, satisfying the abelian monoid laws
  (associativity, commutativity and $\pzero$ as identity) for parallel
  composition $|$ and for summation $+$.
\end{definition}

\subsection{Name equivalence}

We take name equivalence, written $\nameeq$, to be the smallest
equivalence relation generated by the following rules.

\begin{mathpar}
\inferrule*[lab=Quote-drop]
{ }
{ \quotep{@{x}} \nameeq x }

\inferrule*[lab=Struct-equiv]
{ P \scong Q }
{ \quotep{P} \nameeq \quotep{Q} }
\end{mathpar}

The astute reader will have noticed that the mutual recursion of names
and processes imposes a mutual recursion on alpha-equivalence and
structural equivalence via name-equivalence. Fortunately, all of this
works out pleasantly and we may calculate in the natural way, free of
concern. The reader interested in the details is referred to the
appendix \ref{appendix:rho_details}.

\subsection{Substitution}

We use $\Proc$ for the set of processes, $\QProc$ for the set of
names, and $\id{\{}\vec{y} / \vec{x} \id{\}}$ to denote partial maps,
$s : \QProc \rightarrow \QProc$. A map, $s$ lifts, uniquely, to a map
on process terms, $\widehat{s} : \Proc \rightarrow \Proc$ by the
following equations.

\begin{mathpar}
  (0) \psubstp{Q}{P} := 0 \\
  (R \juxtap S) \psubstp{Q}{P}
  :=    
  (R)\psubstp{Q}{P} \juxtap (S) \psubstp{Q}{P} \\
  (x?(y).R) \psubstp{Q}{P}    
  :=    
  (x)\substp{Q}{P} (z)\concat( (R \psubstn{z}{y}) \psubstp{Q}{P} ) \\
  (\lift{x}{R}) \psubstp{Q}{P}  
  :=
  \lift{(x)\substp{Q}{P}}{ R \psubstp{Q}{P} } \\
%   (\dropn{x})  \psubstp{Q}{P}       
%   := 
%   \left\{ 
%     \begin{array}{ccc} 
%       \dropn{\quotep{Q}} & & x \nameeq \quotep{P} \\
%       \dropn{x} & & otherwise \\
%     \end{array}
%   \right. 
  (\dropn{x})  \psubstp{Q}{P}       
  := 
  \left\{ 
    \begin{array}{ccc} 
      Q & & x \nameeq \quotep{P} \\
      \dropn{x} & & otherwise \\
    \end{array}
  \right.
\end{mathpar}
 

where

\begin{eqnarray}
  (x)\id{\{} \lpquote Q \rpquote / \lpquote P \rpquote \id{\}}            = 
  \left\{ 
    \begin{array}{ccc}
      \lpquote Q \rpquote & & x \nameeq \lpquote P \rpquote \\
      x & & otherwise \\
    \end{array}
  \right. \nonumber
\end{eqnarray}

and $z$ is chosen distinct from $\quotep{P}$, $\quotep{Q}$, the free
names in $Q$, and all the names in $R$. Our $\alpha$-equivalence will
be built in the standard way from this substitution.

\begin{remark}\label{rem:no_self_referential_names}
  One consequence of these definitions is that $\forall P. \quotep{P}
  \not\in \freenames{P}$.
\end{remark}

\subsection{ Dynamic quote: an example }

Anticipating something of what's to come, consider applying the
substitution, $\widehat{\id{\{}u / z \id{\}}}$, to the following pair
of processes, $\lift{w}{y!(z)}$ and $w[ \lpquote y!(z) \rpquote ]$.

\begin{eqnarray}
	\lift{w}{y!(z)}\widehat{\id{\{}u / z \id{\}}}
		& = &
		\lift{w}{y!(u)} \nonumber\\
	w[ \lpquote y!(z) \rpquote ] \widehat{ \id{\{}u / z \id{\}} }
		& = &
		w[ \lpquote y!(z) \rpquote ] \nonumber
\end{eqnarray}

Because the body of the process between quotes is impervious to
substitution, we get radically different answers. In fact, by
examining the first process in an input context,
e.g. $x?(z).\lift{w}{y!(z)}$, we see that the process under the lift
operator may be shaped by prefixed inputs binding a name inside it. In
this sense, the lift operator will be seen as a way to dynamically
construct processes before reifying them as names.

Finally equipped with these standard features we can present the
dynamics of the calculus.

\subsubsection{Operational semantics} 

Finally, we introduce the computational dynamics. What marks these
algebras as distinct from other more traditionally studied algebraic
structures, e.g. vector spaces or polynomial rings, is the manner in
which dynamics is captured. In traditional structures, dynamics is typically
expressed through morphisms between such structures, as in linear maps
between vector spaces or morphisms between rings. In algebras
associated with the semantics of computation, the dynamics is
expressed as part of the algebraic structure itself, through a
reduction reduction relation typically denoted by $\red$. Below, we
give a recursive presentation of this relation for the calculus used
in the encoding.

$\red \subseteq \pi \times \pi$
$\red : \pi \to \mathcal{P}(\pi)$

\begin{mathpar}
  \inferrule* [lab=Comm] { \textsf{match}( x_{src}, x_{trgt} ) } { x_{trgt}?(y)P \; | \; x_{src}!\langle {Q} \rangle \red P\{\quotep{Q}/y}\} }
  \and \\
  \inferrule* [lab=Par] {{P} \red {P}'} {{{P} | {Q}} \red {{P}' | {Q}}}
  \and
  \inferrule* [lab=Equiv]{{{P} \scong {P}'} \andalso {{P}' \red {Q}'} \andalso {{Q}' \scong {Q}}}{{P} \red {Q}}
\end{mathpar}

\begin{eqnarray*}
  match_{\equiv} (\quotep{P},\quotep{Q}) & := & P \equiv Q \\
  match_{\dagger}(\quotep{P},\quotep{Q}) & := & \forall R. P|Q \red^{*} R => R \red^{*} 0 \\
  match_{K}(\quotep{P},\quotep{Q}) & := & K \mbox{ for some context } K
\end{eqnarray*}

$u?(x)P | u!\langle Q \rangle \red P\{\quotep{Q}/x\}$

%We write $\wred$ for $\red^*$, and $P\red$ if $\exists Q $ such that $ P \red Q$.
We write $P\red$ if $\exists Q $ such that $ P \red Q$ and $P\not\red$, otherwise.

\section{Replication}

As mentioned before, it is known that replication (and hence
recursion) can be implemented in a higher-order process algebra
\cite{SangiorgiWalker}. As our first example of calculation with the
machinery thus far presented we give the construction explicitly in
the {\rhoc}.

\begin{eqnarray}
	D_{x} & := & \prefix{x}{y}{(\binpar{\outputp{x}{y}}{@{y}})} \nonumber\\
	\bangp_{x}{P} & := & \binpar{{x}!\langle{\binpar{D_{x}}{P}}\rangle}{D_{x}} \nonumber
\end{eqnarray}

\begin{eqnarray}
	\bangp_{x}{P} & & \nonumber\\
	=
	& {x}!\langle{(\prefix{x}{y}{(\outputp{x}{y} | @{y})) | P}}\rangle 
	      | \prefix{x}{y}{(\outputp{x}{y} | @{y})} & \nonumber\\
	\red
	& (\outputp{x}{y} | @{y})\substn{\quotep{(\prefix{x}{y}{(@{y} | \outputp{x}{y})) | P}}}{y} & \nonumber\\
	=
	& \outputp{x}{\quotep{(\prefix{x}{y}{(\outputp{x}{y} | @{y})) | P}}}
	  | {(\prefix{x}{y}{(\outputp{x}{y} | @{y})) | P}} & \nonumber\\
	\red
	& \ldots & \nonumber\\
	\red^*
	& P | P | \ldots & \nonumber
\end{eqnarray}

Of course, this encoding, as an implementation, runs away, unfolding
$\bangp{P}$ eagerly. A lazier and more implementable replication
operator, restricted to input-guarded processes, may be obtained as follows.

\begin{eqnarray}
\bangp{\prefix{u}{v}{P}} 
	:= 
	\binpar{\lift{x}{\prefix{u}{v}{(\binpar{D(x)}{P})}}}{D(x)} \nonumber
\end{eqnarray}

\begin{remark}
  Note that the lazier definition still does not deal with summation
  or mixed summation (i.e. sums over input and output). The reader is
  invited to construct definitions of replication that deal with these
  features. 

  Further, the definitions are parameterized in a name, $x$. Can you,
  gentle reader, make a definition that eliminates this parameter and
  guarantees no accidental interaction between the replication
  machinery and the process being replicated -- i.e. no accidental
  sharing of names used by the process to get its work done and the
  name(s) used by the replication to effect copying. This latter
  revision of the definition of replication is crucial to obtaining
  the expected identity $!!P \sim !P$.
\end{remark}

\begin{remark}\label{rem:paradoxical_combinator}
  The reader familiar with the lambda calculus will have noticed the
  similarity between $D$ and the paradoxical combinator.

  [Ed. note: the existence of this seems to suggest we have to be more
  restrictive on the set of processes and names we admit if we are to
  support no-cloning.]
\end{remark}

\subsubsection{Bisimulation}

The computational dynamics gives rise to another kind of equivalence,
the equivalence of computational behavior. As previously mentioned
this is typically captured \emph{via} some form of bisimulation.

% The notion we use in this paper is weak barbed bisimulation
% \cite{milner91polyadicpi}.

The notion we use in this paper is derived from weak barbed
bisimulation \cite{milner91polyadicpi}. 

\begin{definition}
An \emph{observation relation}, $\downarrow_{\mathcal N}$, over a set
of names, $\mathcal N$, is the smallest relation satisfying the rules
below.

\infrule[Out-barb]{y \in {\mathcal N}, \; x \nameeq y}
		  {\outputp{x}{v} \downarrow_{\mathcal N} x}
\infrule[Par-barb]{\mbox{$P\downarrow_{\mathcal N} x$ or $Q\downarrow_{\mathcal N} x$}}
		  {\binpar{P}{Q} \downarrow_{\mathcal N} x}

We write $P \Downarrow_{\mathcal N} x$ if there is $Q$ such that 
$P \wred Q$ and $Q \downarrow_{\mathcal N} x$.
\end{definition}

\begin{definition}
%\label{def.bbisim}
An  ${\mathcal N}$-\emph{barbed bisimulation} over a set of names, ${\mathcal N}$, is a symmetric binary relation 
${\mathcal S}_{\mathcal N}$ between agents such that $P\rel{S}_{\mathcal N}Q$ implies:
\begin{enumerate}
\item If $P \red P'$ then $Q \wred Q'$ and $P'\rel{S}_{\mathcal N} Q'$.
\item If $P\downarrow_{\mathcal N} x$, then $Q\Downarrow_{\mathcal N} x$.
\end{enumerate}
$P$ is ${\mathcal N}$-barbed bisimilar to $Q$, written
$P \wbbisim_{\mathcal N} Q$, if $P \rel{S}_{\mathcal N} Q$ for some ${\mathcal N}$-barbed bisimulation ${\mathcal S}_{\mathcal N}$.
\end{definition}

$\mathcal{R} \subseteq \pi \times \pi$

$P \mathcal{R} Q => \forall P'. P \red P' \Rightarrow \exists Q'. Q \red Q', P' \mathcal{R} Q'$

$P \vdash x \Rightarrow Q \vdash x$

\begin{mathpar}
  \inferrule*[lab=Out-barb]{x \nameeq y}{{y}!\langle{Q}\rangle \vdash x}
  \and
  \inferrule*[lab=Par-barb]{\mbox{$P\vdash x$ or $Q\vdash x$}}{\binpar{P}{Q} \vdash x}
\end{mathpar}

\subsubsection{Contexts}

One of the principle advantages of computational calculi like the
$\pi$-calculus is a well-defined notion of context,
contextual-equivalence and a correlation between
contextual-equivalence and notions of bisimulation. The notion of
context allows the decomposition of a process into (sub-)process and
its syntactic environment, its context. Thus, a context may be
thought of as a process with a ``hole'' (written $\Box$) in it. The
application of a context $M$ to a process $P$, written $M[P]$, is
tantamount to filling the hole in $M$ with $P$. In this paper we do
not need the full weight of this theory, but do make use of the notion
of context in the proof the main theorem. 

\begin{mathpar}
  \inferrule* [lab=summation] {} {{M_{M},M_{N}} \bc \Box \;|\; x.M_{A} \;|\; M_{M}+M_{N}}
  \and
  \inferrule* [lab=agent] {} {{M_{A}} \bc (\vec{x})M_{P} \;| \; \clift{P_0,\ldots,M_{P},\ldots,P_N}}
  \and \\
  \inferrule* [lab=process] {} {{M_{P}} \bc M_{N} \;| \;P|M_{P} }
\end{mathpar} 

\begin{mathpar}
  \inferrule* [lab=sychronization] {} {M_{N} \bc \Box \;|\; x?M_{F} \;|\; x!M_{C}}
  \and
  \inferrule* [lab=abstraction] {} {{M_{F}} \bc (x)M_{P} }
  \and
  \inferrule* [lab=concretion] {} {{M_{C}} \bc \langle M_{P} \rangle }
  \and \\
  \inferrule* [lab=process] {} {{M_{P}} \bc M_{N} \;| \;P|M_{P} }
\end{mathpar}

\begin{definition}[contextual application] Given a context $M$, and
  process $P$, we define the \emph{contextual application}, $M[P] :=
  M\{P/\Box\}$. That is, the contextual application of M to P is the
  substitution of $P$ for $\Box$ in $M$.
\end{definition}

$\meaningof{-} : L \to \mathcal{P}(\pi)$

\begin{mathpar}
  \inferrule* [lab=collection] {} {\meaningof{true} = \pi, \and \meaningof{~E} = \pi \setminus \meaningof{E}, \and \meaningof{E_{1} \& E_{2}} = \meaningof{E_{1}} \cap \meaningof{E_{2}}}
\end{mathpar}

\begin{mathpar}
  \inferrule* [lab=structure] {} {\meaningof{0} = \{ P \in \pi | P \equiv 0 \}, \and \\ \meaningof{E_1 | E_2} = \{ P \in \pi | P \equiv P_{1} | P_{2}, P_{1} \in \meaningof{E_{1}}, P_{2} \in \meaningof{E_2}\} }
\end{mathpar}

\begin{mathpar}
 \inferrule* [lab=behavior] {} {\meaningof{\langle a?b \rangle E} = \{ P \in \pi | P \equiv Q | u?(y)P', \\ \and \\\\ \and \\ \;\;\; u \in \meaningof{a}, \forall z.P'\{z/y\} \in \meaningof{E\{z/b\}}\}, \and \\ \meaningof{a!E} = \{ P \in \pi | P \equiv Q | x!\langle P' \rangle, x \in \meaningof{a} P' \in \meaningof{E}\} }
\end{mathpar}

\begin{mathpar}
 \inferrule* [lab=nominal] {} {\meaningof{\quotep{E}} = \{ \quotep{P} \in \quotep{\pi} | P \in \meaningof{E} \}, \and \meaningof{\quotep{P}} = \{ \quotep{Q} \in \quotep{\pi} | P \equiv Q \} \and \\ \meaningof{@\quotep{E}} = \{ P \in \pi | P \equiv @x, x \in \meaningof{E} \}}
\end{mathpar}

\begin{eqnarray*}
  \\
  \meaningof{-} : TS \to ST
\end{eqnarray*}

\begin{eqnarray*}
  \\
  L : TS \to ST
\end{eqnarray*}

\begin{eqnarray*}
  \\
  P \models E \iff P \in \meaningof{E}
\end{eqnarray*}

\begin{eqnarray*}
  P \approx_{L} Q \iff \forall E \in L. P \models E \iff Q \models E
\end{eqnarray*}

\begin{eqnarray*}
  P \approx_{K} Q
\end{eqnarray*}

\begin{eqnarray*}
  P \approx Q
\end{eqnarray*}

$\approx_{K} = \approx = \approx_{L}$

\subsubsection{Contextual duality}

Note that contexts extend the quotation operation to a family of
operations from processes to names. Given a context, $M$, we can
define a \emph{nominal context}, $\quotep{M}$ by $\quotep{M}[P] :=
\quotep{M[P]}$. To foreshadow what is to come we observe that these
operations enjoy a duality with processes very much like the duality
between vectors and maps from vectors to scalars.

Further, because the calculus is essentially higher-order, we have a
correspondence between contexts and processes. More specifically,
given a name $x$ and a context $M$ we can construct $M^{*}_{x}$ such
that 

\begin{mathpar}
  M^{*}_{x} | \lift{x}{P} \red M[P]
\end{mathpar}

namely,

\begin{mathpar}
  M^{*}_{x} := x?(u).M[\dropn{u}]
\end{mathpar}

The dependence of $M^{*}_{x}$ on a name makes it an abstraction, 

\begin{mathpar}
  M^{*} := (x)x?(u).M[\dropn{u}]
\end{mathpar}

\subsection{Additional notation}

It will sometimes be convenient to denote the process a name
quotes. We already have the notation $x = \quotep{P}$, but it will be
convenient to introduce an alternate notation, $\procn{x}$, when we
want to emphasize the connection to the use of the name. Note that, by
virtue of name equivalence, $\quotep{\procn{x}} \nameeq x$; so, the
notation is consistent with previous definitions.

Further, because names have structure it is possible to effect
substitutions on the basis of that structure. This means we need to
upgrade our notation for substitutions, which we accomplish by
adapting comprehension notation. Thus,

\begin{mathpar}
  P\{ y / x : x \in S \}
\end{mathpar}

is interpreted to mean the process derived from P by replacing (in a
capture-avoiding manner) each occurrence of $x$ in $S$ by $y$. For example,

\begin{mathpar}
  P\{ \quotep{\procn{x}|\procn{x}} / x : x \in \freenames{P} \}
\end{mathpar}

will replace each (occurrence) of a free name $x$ in $P$ by
$\quotep{\procn{x}|\procn{x}}$.

Also, we will avail ourselves of the notation $x^{L}$ and $x^{R}$ to
denote injections of a name into disjoint copies of the name
space. There are numerous ways to accomplish this. One example can be
found in \cite{MeredithR05}. This notation overloads to vectors of
names: $\vec{x}^{\pi} := (x_{i}^{\pi} \; : \; 0 \leq i < |\vec{x}| )$ where $\pi \in \{L,R\}$.

We also use $P^{\Box} := P|\Box$.

In \cite{MeredithR05} an interpretation of the new operator is
given. It turns out that there are several possible interpretations
all enjoying the requisite algebraic properties of the operator (see
\cite{milner91polyadicpi}). We will therefore make liberal use of
$(\nu\; \vec{x})P$.

% subsection the_syntax_and_semantics_of_the_notation_system (end)   

\input{qm2pi.qmops} 

\input{qm2pi.sterngerlach} 

\input{qm2pi.metric} 

% section concurrent_process_calculi (end)

%\input{qm2pi.proofsketch}

% section proof sketch (end)

%\input{qm2pi.slviaknots} 

% section spatial logic via knots (end)

\input{qm2pi.conclusion}

% section conclusion (end)

%\input{qm2pi.dtcodes} 

% section wiring algorithm (end)

\input{qm2pi.ack} 

% section acknowledgments (end)

\newpage


\bibliographystyle{plain}   
\bibliography{../../biblios/main.bib}

\input{qm2pi.rhodetails}

\end{document}

 

% section acknowledgments (end)

\newpage


\bibliographystyle{plain}   
\bibliography{../../biblios/main.bib}

\documentclass[12pt]{llncs}
%\documentclass{jktr}

\usepackage[pdftex]{hyperref}                   
\usepackage {listings}
\usepackage {mathpartir}
\usepackage{bcprules}
%\usepackage{listings}
                       
\usepackage{graphicx} 
%\usepackage[margins=2.5cm,nohead,nofoot]{geometry}
%\usepackage{geometry}
\usepackage{amsfonts}
\usepackage{amstext}
\usepackage{latexsym}
\usepackage{amssymb}
\usepackage{color}


%\include{myPreamble}
\include{qm2pi.local} 

%\ifpdf
%\usepackage[pdftex]{graphicx}
%\else
%\usepackage{graphicx}
%\fi

 % \ifpdf
%  \usepackage{pdfsync}
%  \if


%\title{Brief Article}
%\author{David F. Snyder}
%\author{L.G. Meredith}

%\address{Dept. of Math., Texas State University--San Marcos, San Marcos, TX 78666}
       
\pagestyle{empty}


\begin{document}

\lstset{language=[Objective]Caml,frame=shadowbox}

\input{qm2pi.front}

% section front matter (end)

\input{qm2pi.intro} 
 
% section introduction (end)

% \input{qm2pi.knotations} 

% section notation (end)

\input{qm2pi.process.calculi} 

% section concurrent_process_calculi_and_spatial_logics_ (end)
    
%\input{qm2pi.knots2pi} 

%\input{qm2pi.trefoil} 

%\input{qm2pi.mainthm} 

% subsection basic_interpretation (end)

%\input{qm2pi.rho.presentation} 
\subsection{The syntax and semantics of the notation system}\label{sub:the_syntax_and_semantics_of_the_notation_system} % (fold)

We now summarize a technical presentation of the calculus that
embodies our theory of dynamics. The typical presentation of such a
calculus follows the style of giving generators and relations on
them. The grammar, below, describing term constructors, freely
generates the set of processes, $\Proc$. This set is then quotiented
by a relation known as structural congruence and it is over this set
that the notion of dynamics is expressed. This presentation is
essentially that of \cite{MeredithR05} with the addition of
polyadicity and summation. For readability we have relegated some of
the technical subtleties to an appendix.

\subsubsection{Process grammar}\label{subsub:process_grammar}

\begin{mathpar}
  \inferrule* [lab=synchronization] {} {{M} \bc \pzero \;|\; x?F \;|\; x!C }
  \and
  \inferrule* [lab=abstraction] {} {{F} \bc (x)P}
  \and
  \inferrule* [lab=concretion] {} {{C} \bc \langle Q \rangle}
  \and
  \inferrule* [lab=process] {} {{P,Q} \bc M \;| \;P|Q \;|\; @{x}}
  \and
  \inferrule* [lab=name] {} {{x} \bc \quotep{P}}
\end{mathpar} 

Note that $\vec{x}$ (resp. $\vec{P}$) denotes a vector of names
(resp. processes) of length $|\vec{x}|$ (resp. $|\vec{P}|$). We adopt
the following useful abbreviations.

\begin{mathpar}
   x?(\vec{y}).P := x.(\vec{y})P \and  x\clift{\vec{P}} := x.\clift{\vec{P}}
   \and x!(y) := \lift{x}{\dropn{y}}
   \and \Pi_{i=0}^{n-1}P_i := P_0 | \ldots | P_{n-1}
\end{mathpar}

\subsubsection{Structural congruence}

\paragraph{Free and bound names and alpha-equivalence.} At the
core of structural equivalence is alpha-equivalence which identifies
process that are the same up to a change of variable. Formally, we
recognize the distinction between free and bound names. The free names
of a process, $\freenames{P}$, may be calculated recursively as
follows:

\begin{mathpar}
\freenames{\pzero} := \emptyset
  \and \\
  \freenames{x?(y).P} := \{ x \} \cup (\freenames{P} \setminus \{ y \})
  \and 
  \freenames{x!\langle P \rangle} := \{ x \} \cup \{ P \} 
  \and \\
  \freenames{P|Q} := \freenames{P} \cup \freenames{Q}
  \and \\
  \freenames{@{x}} := \{ x \}
\end{mathpar}

$\pi$
$\quotep{\pi}$

$\freenames{-} : \pi \to \mathcal{P}(\quotep{\pi})$

\begin{eqnarray*}
  \freenames{\pzero} & := & \emptyset \\
  \freenames{x?(y).P} & := & \{ x \} \cup (\freenames{P} \setminus \{ y \}) \\
  \freenames{x!\langle P \rangle} & := & \{ x \} \cup \{ P \} \\
  \freenames{P|Q} & := & \freenames{P} \cup \freenames{Q} \\
  \freenames{\dropn{x}} & := & \{ x \}
\end{eqnarray*}

The bound names of a process, $\boundnames{P}$, are those names occurring in $P$
that are not free. For example, in $x?(y).0$, the name $x$ is free, while $y$ is bound.

\begin{mathpar}
  \inferrule* [lab=monoidal-laws] {} { P|Q \equiv Q|P \and P|0 \equiv P \and P|(Q|R) \equiv (P|Q)|R }
\end{mathpar}

\begin{mathpar}
  \inferrule* [lab=alpha-equivalence] {} { (x)P \equiv (y)P\{y/x\} \and y \not\in \freenames{P} }
\end{mathpar}

\begin{definition}
Then two processes, $P,Q$, are alpha-equivalent if $P = Q\{\vec{y}/\vec{x}\}$ for
some $\vec{x} \in \boundnames{Q},\vec{y} \in \boundnames{P}$, where $Q\{\vec{y}/\vec{x}\}$
denotes the capture-avoiding substitution of $\vec{y}$ for $\vec{x}$ in $Q$.
\end{definition}

\begin{definition}
  The {\em structural congruence} \cite{SangiorgiWalker} , $\equiv$,
  between processes is the least congruence containing
  alpha-equivalence, satisfying the abelian monoid laws
  (associativity, commutativity and $\pzero$ as identity) for parallel
  composition $|$ and for summation $+$.
\end{definition}

\subsection{Name equivalence}

We take name equivalence, written $\nameeq$, to be the smallest
equivalence relation generated by the following rules.

\begin{mathpar}
\inferrule*[lab=Quote-drop]
{ }
{ \quotep{@{x}} \nameeq x }

\inferrule*[lab=Struct-equiv]
{ P \scong Q }
{ \quotep{P} \nameeq \quotep{Q} }
\end{mathpar}

The astute reader will have noticed that the mutual recursion of names
and processes imposes a mutual recursion on alpha-equivalence and
structural equivalence via name-equivalence. Fortunately, all of this
works out pleasantly and we may calculate in the natural way, free of
concern. The reader interested in the details is referred to the
appendix \ref{appendix:rho_details}.

\subsection{Substitution}

We use $\Proc$ for the set of processes, $\QProc$ for the set of
names, and $\id{\{}\vec{y} / \vec{x} \id{\}}$ to denote partial maps,
$s : \QProc \rightarrow \QProc$. A map, $s$ lifts, uniquely, to a map
on process terms, $\widehat{s} : \Proc \rightarrow \Proc$ by the
following equations.

\begin{mathpar}
  (0) \psubstp{Q}{P} := 0 \\
  (R \juxtap S) \psubstp{Q}{P}
  :=    
  (R)\psubstp{Q}{P} \juxtap (S) \psubstp{Q}{P} \\
  (x?(y).R) \psubstp{Q}{P}    
  :=    
  (x)\substp{Q}{P} (z)\concat( (R \psubstn{z}{y}) \psubstp{Q}{P} ) \\
  (\lift{x}{R}) \psubstp{Q}{P}  
  :=
  \lift{(x)\substp{Q}{P}}{ R \psubstp{Q}{P} } \\
%   (\dropn{x})  \psubstp{Q}{P}       
%   := 
%   \left\{ 
%     \begin{array}{ccc} 
%       \dropn{\quotep{Q}} & & x \nameeq \quotep{P} \\
%       \dropn{x} & & otherwise \\
%     \end{array}
%   \right. 
  (\dropn{x})  \psubstp{Q}{P}       
  := 
  \left\{ 
    \begin{array}{ccc} 
      Q & & x \nameeq \quotep{P} \\
      \dropn{x} & & otherwise \\
    \end{array}
  \right.
\end{mathpar}
 

where

\begin{eqnarray}
  (x)\id{\{} \lpquote Q \rpquote / \lpquote P \rpquote \id{\}}            = 
  \left\{ 
    \begin{array}{ccc}
      \lpquote Q \rpquote & & x \nameeq \lpquote P \rpquote \\
      x & & otherwise \\
    \end{array}
  \right. \nonumber
\end{eqnarray}

and $z$ is chosen distinct from $\quotep{P}$, $\quotep{Q}$, the free
names in $Q$, and all the names in $R$. Our $\alpha$-equivalence will
be built in the standard way from this substitution.

\begin{remark}\label{rem:no_self_referential_names}
  One consequence of these definitions is that $\forall P. \quotep{P}
  \not\in \freenames{P}$.
\end{remark}

\subsection{ Dynamic quote: an example }

Anticipating something of what's to come, consider applying the
substitution, $\widehat{\id{\{}u / z \id{\}}}$, to the following pair
of processes, $\lift{w}{y!(z)}$ and $w[ \lpquote y!(z) \rpquote ]$.

\begin{eqnarray}
	\lift{w}{y!(z)}\widehat{\id{\{}u / z \id{\}}}
		& = &
		\lift{w}{y!(u)} \nonumber\\
	w[ \lpquote y!(z) \rpquote ] \widehat{ \id{\{}u / z \id{\}} }
		& = &
		w[ \lpquote y!(z) \rpquote ] \nonumber
\end{eqnarray}

Because the body of the process between quotes is impervious to
substitution, we get radically different answers. In fact, by
examining the first process in an input context,
e.g. $x?(z).\lift{w}{y!(z)}$, we see that the process under the lift
operator may be shaped by prefixed inputs binding a name inside it. In
this sense, the lift operator will be seen as a way to dynamically
construct processes before reifying them as names.

Finally equipped with these standard features we can present the
dynamics of the calculus.

\subsubsection{Operational semantics} 

Finally, we introduce the computational dynamics. What marks these
algebras as distinct from other more traditionally studied algebraic
structures, e.g. vector spaces or polynomial rings, is the manner in
which dynamics is captured. In traditional structures, dynamics is typically
expressed through morphisms between such structures, as in linear maps
between vector spaces or morphisms between rings. In algebras
associated with the semantics of computation, the dynamics is
expressed as part of the algebraic structure itself, through a
reduction reduction relation typically denoted by $\red$. Below, we
give a recursive presentation of this relation for the calculus used
in the encoding.

$\red \subseteq \pi \times \pi$
$\red : \pi \to \mathcal{P}(\pi)$

\begin{mathpar}
  \inferrule* [lab=Comm] { \textsf{match}( x_{src}, x_{trgt} ) } { x_{trgt}?(y)P \; | \; x_{src}!\langle {Q} \rangle \red P\{\quotep{Q}/y}\} }
  \and \\
  \inferrule* [lab=Par] {{P} \red {P}'} {{{P} | {Q}} \red {{P}' | {Q}}}
  \and
  \inferrule* [lab=Equiv]{{{P} \scong {P}'} \andalso {{P}' \red {Q}'} \andalso {{Q}' \scong {Q}}}{{P} \red {Q}}
\end{mathpar}

\begin{eqnarray*}
  match_{\equiv} (\quotep{P},\quotep{Q}) & := & P \equiv Q \\
  match_{\dagger}(\quotep{P},\quotep{Q}) & := & \forall R. P|Q \red^{*} R => R \red^{*} 0 \\
  match_{K}(\quotep{P},\quotep{Q}) & := & K \mbox{ for some context } K
\end{eqnarray*}

$u?(x)P | u!\langle Q \rangle \red P\{\quotep{Q}/x\}$

%We write $\wred$ for $\red^*$, and $P\red$ if $\exists Q $ such that $ P \red Q$.
We write $P\red$ if $\exists Q $ such that $ P \red Q$ and $P\not\red$, otherwise.

\section{Replication}

As mentioned before, it is known that replication (and hence
recursion) can be implemented in a higher-order process algebra
\cite{SangiorgiWalker}. As our first example of calculation with the
machinery thus far presented we give the construction explicitly in
the {\rhoc}.

\begin{eqnarray}
	D_{x} & := & \prefix{x}{y}{(\binpar{\outputp{x}{y}}{@{y}})} \nonumber\\
	\bangp_{x}{P} & := & \binpar{{x}!\langle{\binpar{D_{x}}{P}}\rangle}{D_{x}} \nonumber
\end{eqnarray}

\begin{eqnarray}
	\bangp_{x}{P} & & \nonumber\\
	=
	& {x}!\langle{(\prefix{x}{y}{(\outputp{x}{y} | @{y})) | P}}\rangle 
	      | \prefix{x}{y}{(\outputp{x}{y} | @{y})} & \nonumber\\
	\red
	& (\outputp{x}{y} | @{y})\substn{\quotep{(\prefix{x}{y}{(@{y} | \outputp{x}{y})) | P}}}{y} & \nonumber\\
	=
	& \outputp{x}{\quotep{(\prefix{x}{y}{(\outputp{x}{y} | @{y})) | P}}}
	  | {(\prefix{x}{y}{(\outputp{x}{y} | @{y})) | P}} & \nonumber\\
	\red
	& \ldots & \nonumber\\
	\red^*
	& P | P | \ldots & \nonumber
\end{eqnarray}

Of course, this encoding, as an implementation, runs away, unfolding
$\bangp{P}$ eagerly. A lazier and more implementable replication
operator, restricted to input-guarded processes, may be obtained as follows.

\begin{eqnarray}
\bangp{\prefix{u}{v}{P}} 
	:= 
	\binpar{\lift{x}{\prefix{u}{v}{(\binpar{D(x)}{P})}}}{D(x)} \nonumber
\end{eqnarray}

\begin{remark}
  Note that the lazier definition still does not deal with summation
  or mixed summation (i.e. sums over input and output). The reader is
  invited to construct definitions of replication that deal with these
  features. 

  Further, the definitions are parameterized in a name, $x$. Can you,
  gentle reader, make a definition that eliminates this parameter and
  guarantees no accidental interaction between the replication
  machinery and the process being replicated -- i.e. no accidental
  sharing of names used by the process to get its work done and the
  name(s) used by the replication to effect copying. This latter
  revision of the definition of replication is crucial to obtaining
  the expected identity $!!P \sim !P$.
\end{remark}

\begin{remark}\label{rem:paradoxical_combinator}
  The reader familiar with the lambda calculus will have noticed the
  similarity between $D$ and the paradoxical combinator.

  [Ed. note: the existence of this seems to suggest we have to be more
  restrictive on the set of processes and names we admit if we are to
  support no-cloning.]
\end{remark}

\subsubsection{Bisimulation}

The computational dynamics gives rise to another kind of equivalence,
the equivalence of computational behavior. As previously mentioned
this is typically captured \emph{via} some form of bisimulation.

% The notion we use in this paper is weak barbed bisimulation
% \cite{milner91polyadicpi}.

The notion we use in this paper is derived from weak barbed
bisimulation \cite{milner91polyadicpi}. 

\begin{definition}
An \emph{observation relation}, $\downarrow_{\mathcal N}$, over a set
of names, $\mathcal N$, is the smallest relation satisfying the rules
below.

\infrule[Out-barb]{y \in {\mathcal N}, \; x \nameeq y}
		  {\outputp{x}{v} \downarrow_{\mathcal N} x}
\infrule[Par-barb]{\mbox{$P\downarrow_{\mathcal N} x$ or $Q\downarrow_{\mathcal N} x$}}
		  {\binpar{P}{Q} \downarrow_{\mathcal N} x}

We write $P \Downarrow_{\mathcal N} x$ if there is $Q$ such that 
$P \wred Q$ and $Q \downarrow_{\mathcal N} x$.
\end{definition}

\begin{definition}
%\label{def.bbisim}
An  ${\mathcal N}$-\emph{barbed bisimulation} over a set of names, ${\mathcal N}$, is a symmetric binary relation 
${\mathcal S}_{\mathcal N}$ between agents such that $P\rel{S}_{\mathcal N}Q$ implies:
\begin{enumerate}
\item If $P \red P'$ then $Q \wred Q'$ and $P'\rel{S}_{\mathcal N} Q'$.
\item If $P\downarrow_{\mathcal N} x$, then $Q\Downarrow_{\mathcal N} x$.
\end{enumerate}
$P$ is ${\mathcal N}$-barbed bisimilar to $Q$, written
$P \wbbisim_{\mathcal N} Q$, if $P \rel{S}_{\mathcal N} Q$ for some ${\mathcal N}$-barbed bisimulation ${\mathcal S}_{\mathcal N}$.
\end{definition}

$\mathcal{R} \subseteq \pi \times \pi$

$P \mathcal{R} Q => \forall P'. P \red P' \Rightarrow \exists Q'. Q \red Q', P' \mathcal{R} Q'$

$P \vdash x \Rightarrow Q \vdash x$

\begin{mathpar}
  \inferrule*[lab=Out-barb]{x \nameeq y}{{y}!\langle{Q}\rangle \vdash x}
  \and
  \inferrule*[lab=Par-barb]{\mbox{$P\vdash x$ or $Q\vdash x$}}{\binpar{P}{Q} \vdash x}
\end{mathpar}

\subsubsection{Contexts}

One of the principle advantages of computational calculi like the
$\pi$-calculus is a well-defined notion of context,
contextual-equivalence and a correlation between
contextual-equivalence and notions of bisimulation. The notion of
context allows the decomposition of a process into (sub-)process and
its syntactic environment, its context. Thus, a context may be
thought of as a process with a ``hole'' (written $\Box$) in it. The
application of a context $M$ to a process $P$, written $M[P]$, is
tantamount to filling the hole in $M$ with $P$. In this paper we do
not need the full weight of this theory, but do make use of the notion
of context in the proof the main theorem. 

\begin{mathpar}
  \inferrule* [lab=summation] {} {{M_{M},M_{N}} \bc \Box \;|\; x.M_{A} \;|\; M_{M}+M_{N}}
  \and
  \inferrule* [lab=agent] {} {{M_{A}} \bc (\vec{x})M_{P} \;| \; \clift{P_0,\ldots,M_{P},\ldots,P_N}}
  \and \\
  \inferrule* [lab=process] {} {{M_{P}} \bc M_{N} \;| \;P|M_{P} }
\end{mathpar} 

\begin{mathpar}
  \inferrule* [lab=sychronization] {} {M_{N} \bc \Box \;|\; x?M_{F} \;|\; x!M_{C}}
  \and
  \inferrule* [lab=abstraction] {} {{M_{F}} \bc (x)M_{P} }
  \and
  \inferrule* [lab=concretion] {} {{M_{C}} \bc \langle M_{P} \rangle }
  \and \\
  \inferrule* [lab=process] {} {{M_{P}} \bc M_{N} \;| \;P|M_{P} }
\end{mathpar}

\begin{definition}[contextual application] Given a context $M$, and
  process $P$, we define the \emph{contextual application}, $M[P] :=
  M\{P/\Box\}$. That is, the contextual application of M to P is the
  substitution of $P$ for $\Box$ in $M$.
\end{definition}

$\meaningof{-} : L \to \mathcal{P}(\pi)$

\begin{mathpar}
  \inferrule* [lab=collection] {} {\meaningof{true} = \pi, \and \meaningof{~E} = \pi \setminus \meaningof{E}, \and \meaningof{E_{1} \& E_{2}} = \meaningof{E_{1}} \cap \meaningof{E_{2}}}
\end{mathpar}

\begin{mathpar}
  \inferrule* [lab=structure] {} {\meaningof{0} = \{ P \in \pi | P \equiv 0 \}, \and \\ \meaningof{E_1 | E_2} = \{ P \in \pi | P \equiv P_{1} | P_{2}, P_{1} \in \meaningof{E_{1}}, P_{2} \in \meaningof{E_2}\} }
\end{mathpar}

\begin{mathpar}
 \inferrule* [lab=behavior] {} {\meaningof{\langle a?b \rangle E} = \{ P \in \pi | P \equiv Q | u?(y)P', \\ \and \\\\ \and \\ \;\;\; u \in \meaningof{a}, \forall z.P'\{z/y\} \in \meaningof{E\{z/b\}}\}, \and \\ \meaningof{a!E} = \{ P \in \pi | P \equiv Q | x!\langle P' \rangle, x \in \meaningof{a} P' \in \meaningof{E}\} }
\end{mathpar}

\begin{mathpar}
 \inferrule* [lab=nominal] {} {\meaningof{\quotep{E}} = \{ \quotep{P} \in \quotep{\pi} | P \in \meaningof{E} \}, \and \meaningof{\quotep{P}} = \{ \quotep{Q} \in \quotep{\pi} | P \equiv Q \} \and \\ \meaningof{@\quotep{E}} = \{ P \in \pi | P \equiv @x, x \in \meaningof{E} \}}
\end{mathpar}

\begin{eqnarray*}
  \\
  \meaningof{-} : TS \to ST
\end{eqnarray*}

\begin{eqnarray*}
  \\
  L : TS \to ST
\end{eqnarray*}

\begin{eqnarray*}
  \\
  P \models E \iff P \in \meaningof{E}
\end{eqnarray*}

\begin{eqnarray*}
  P \approx_{L} Q \iff \forall E \in L. P \models E \iff Q \models E
\end{eqnarray*}

\begin{eqnarray*}
  P \approx_{K} Q
\end{eqnarray*}

\begin{eqnarray*}
  P \approx Q
\end{eqnarray*}

$\approx_{K} = \approx = \approx_{L}$

\subsubsection{Contextual duality}

Note that contexts extend the quotation operation to a family of
operations from processes to names. Given a context, $M$, we can
define a \emph{nominal context}, $\quotep{M}$ by $\quotep{M}[P] :=
\quotep{M[P]}$. To foreshadow what is to come we observe that these
operations enjoy a duality with processes very much like the duality
between vectors and maps from vectors to scalars.

Further, because the calculus is essentially higher-order, we have a
correspondence between contexts and processes. More specifically,
given a name $x$ and a context $M$ we can construct $M^{*}_{x}$ such
that 

\begin{mathpar}
  M^{*}_{x} | \lift{x}{P} \red M[P]
\end{mathpar}

namely,

\begin{mathpar}
  M^{*}_{x} := x?(u).M[\dropn{u}]
\end{mathpar}

The dependence of $M^{*}_{x}$ on a name makes it an abstraction, 

\begin{mathpar}
  M^{*} := (x)x?(u).M[\dropn{u}]
\end{mathpar}

\subsection{Additional notation}

It will sometimes be convenient to denote the process a name
quotes. We already have the notation $x = \quotep{P}$, but it will be
convenient to introduce an alternate notation, $\procn{x}$, when we
want to emphasize the connection to the use of the name. Note that, by
virtue of name equivalence, $\quotep{\procn{x}} \nameeq x$; so, the
notation is consistent with previous definitions.

Further, because names have structure it is possible to effect
substitutions on the basis of that structure. This means we need to
upgrade our notation for substitutions, which we accomplish by
adapting comprehension notation. Thus,

\begin{mathpar}
  P\{ y / x : x \in S \}
\end{mathpar}

is interpreted to mean the process derived from P by replacing (in a
capture-avoiding manner) each occurrence of $x$ in $S$ by $y$. For example,

\begin{mathpar}
  P\{ \quotep{\procn{x}|\procn{x}} / x : x \in \freenames{P} \}
\end{mathpar}

will replace each (occurrence) of a free name $x$ in $P$ by
$\quotep{\procn{x}|\procn{x}}$.

Also, we will avail ourselves of the notation $x^{L}$ and $x^{R}$ to
denote injections of a name into disjoint copies of the name
space. There are numerous ways to accomplish this. One example can be
found in \cite{MeredithR05}. This notation overloads to vectors of
names: $\vec{x}^{\pi} := (x_{i}^{\pi} \; : \; 0 \leq i < |\vec{x}| )$ where $\pi \in \{L,R\}$.

We also use $P^{\Box} := P|\Box$.

In \cite{MeredithR05} an interpretation of the new operator is
given. It turns out that there are several possible interpretations
all enjoying the requisite algebraic properties of the operator (see
\cite{milner91polyadicpi}). We will therefore make liberal use of
$(\nu\; \vec{x})P$.

% subsection the_syntax_and_semantics_of_the_notation_system (end)   

\input{qm2pi.qmops} 

\input{qm2pi.sterngerlach} 

\input{qm2pi.metric} 

% section concurrent_process_calculi (end)

%\input{qm2pi.proofsketch}

% section proof sketch (end)

%\input{qm2pi.slviaknots} 

% section spatial logic via knots (end)

\input{qm2pi.conclusion}

% section conclusion (end)

%\input{qm2pi.dtcodes} 

% section wiring algorithm (end)

\input{qm2pi.ack} 

% section acknowledgments (end)

\newpage


\bibliographystyle{plain}   
\bibliography{../../biblios/main.bib}

\input{qm2pi.rhodetails}

\end{document}



\end{document}

 

%\documentclass[12pt]{llncs}
%\documentclass{jktr}

\usepackage[pdftex]{hyperref}                   
\usepackage {listings}
\usepackage {mathpartir}
\usepackage{bcprules}
%\usepackage{listings}
                       
\usepackage{graphicx} 
%\usepackage[margins=2.5cm,nohead,nofoot]{geometry}
%\usepackage{geometry}
\usepackage{amsfonts}
\usepackage{amstext}
\usepackage{latexsym}
\usepackage{amssymb}
\usepackage{color}


%\include{myPreamble}
\documentclass[12pt]{llncs}
%\documentclass{jktr}

\usepackage[pdftex]{hyperref}                   
\usepackage {listings}
\usepackage {mathpartir}
\usepackage{bcprules}
%\usepackage{listings}
                       
\usepackage{graphicx} 
%\usepackage[margins=2.5cm,nohead,nofoot]{geometry}
%\usepackage{geometry}
\usepackage{amsfonts}
\usepackage{amstext}
\usepackage{latexsym}
\usepackage{amssymb}
\usepackage{color}


%\include{myPreamble}
\include{qm2pi.local} 

%\ifpdf
%\usepackage[pdftex]{graphicx}
%\else
%\usepackage{graphicx}
%\fi

 % \ifpdf
%  \usepackage{pdfsync}
%  \if


%\title{Brief Article}
%\author{David F. Snyder}
%\author{L.G. Meredith}

%\address{Dept. of Math., Texas State University--San Marcos, San Marcos, TX 78666}
       
\pagestyle{empty}


\begin{document}

\lstset{language=[Objective]Caml,frame=shadowbox}

\input{qm2pi.front}

% section front matter (end)

\input{qm2pi.intro} 
 
% section introduction (end)

% \input{qm2pi.knotations} 

% section notation (end)

\input{qm2pi.process.calculi} 

% section concurrent_process_calculi_and_spatial_logics_ (end)
    
%\input{qm2pi.knots2pi} 

%\input{qm2pi.trefoil} 

%\input{qm2pi.mainthm} 

% subsection basic_interpretation (end)

%\input{qm2pi.rho.presentation} 
\subsection{The syntax and semantics of the notation system}\label{sub:the_syntax_and_semantics_of_the_notation_system} % (fold)

We now summarize a technical presentation of the calculus that
embodies our theory of dynamics. The typical presentation of such a
calculus follows the style of giving generators and relations on
them. The grammar, below, describing term constructors, freely
generates the set of processes, $\Proc$. This set is then quotiented
by a relation known as structural congruence and it is over this set
that the notion of dynamics is expressed. This presentation is
essentially that of \cite{MeredithR05} with the addition of
polyadicity and summation. For readability we have relegated some of
the technical subtleties to an appendix.

\subsubsection{Process grammar}\label{subsub:process_grammar}

\begin{mathpar}
  \inferrule* [lab=synchronization] {} {{M} \bc \pzero \;|\; x?F \;|\; x!C }
  \and
  \inferrule* [lab=abstraction] {} {{F} \bc (x)P}
  \and
  \inferrule* [lab=concretion] {} {{C} \bc \langle Q \rangle}
  \and
  \inferrule* [lab=process] {} {{P,Q} \bc M \;| \;P|Q \;|\; @{x}}
  \and
  \inferrule* [lab=name] {} {{x} \bc \quotep{P}}
\end{mathpar} 

Note that $\vec{x}$ (resp. $\vec{P}$) denotes a vector of names
(resp. processes) of length $|\vec{x}|$ (resp. $|\vec{P}|$). We adopt
the following useful abbreviations.

\begin{mathpar}
   x?(\vec{y}).P := x.(\vec{y})P \and  x\clift{\vec{P}} := x.\clift{\vec{P}}
   \and x!(y) := \lift{x}{\dropn{y}}
   \and \Pi_{i=0}^{n-1}P_i := P_0 | \ldots | P_{n-1}
\end{mathpar}

\subsubsection{Structural congruence}

\paragraph{Free and bound names and alpha-equivalence.} At the
core of structural equivalence is alpha-equivalence which identifies
process that are the same up to a change of variable. Formally, we
recognize the distinction between free and bound names. The free names
of a process, $\freenames{P}$, may be calculated recursively as
follows:

\begin{mathpar}
\freenames{\pzero} := \emptyset
  \and \\
  \freenames{x?(y).P} := \{ x \} \cup (\freenames{P} \setminus \{ y \})
  \and 
  \freenames{x!\langle P \rangle} := \{ x \} \cup \{ P \} 
  \and \\
  \freenames{P|Q} := \freenames{P} \cup \freenames{Q}
  \and \\
  \freenames{@{x}} := \{ x \}
\end{mathpar}

$\pi$
$\quotep{\pi}$

$\freenames{-} : \pi \to \mathcal{P}(\quotep{\pi})$

\begin{eqnarray*}
  \freenames{\pzero} & := & \emptyset \\
  \freenames{x?(y).P} & := & \{ x \} \cup (\freenames{P} \setminus \{ y \}) \\
  \freenames{x!\langle P \rangle} & := & \{ x \} \cup \{ P \} \\
  \freenames{P|Q} & := & \freenames{P} \cup \freenames{Q} \\
  \freenames{\dropn{x}} & := & \{ x \}
\end{eqnarray*}

The bound names of a process, $\boundnames{P}$, are those names occurring in $P$
that are not free. For example, in $x?(y).0$, the name $x$ is free, while $y$ is bound.

\begin{mathpar}
  \inferrule* [lab=monoidal-laws] {} { P|Q \equiv Q|P \and P|0 \equiv P \and P|(Q|R) \equiv (P|Q)|R }
\end{mathpar}

\begin{mathpar}
  \inferrule* [lab=alpha-equivalence] {} { (x)P \equiv (y)P\{y/x\} \and y \not\in \freenames{P} }
\end{mathpar}

\begin{definition}
Then two processes, $P,Q$, are alpha-equivalent if $P = Q\{\vec{y}/\vec{x}\}$ for
some $\vec{x} \in \boundnames{Q},\vec{y} \in \boundnames{P}$, where $Q\{\vec{y}/\vec{x}\}$
denotes the capture-avoiding substitution of $\vec{y}$ for $\vec{x}$ in $Q$.
\end{definition}

\begin{definition}
  The {\em structural congruence} \cite{SangiorgiWalker} , $\equiv$,
  between processes is the least congruence containing
  alpha-equivalence, satisfying the abelian monoid laws
  (associativity, commutativity and $\pzero$ as identity) for parallel
  composition $|$ and for summation $+$.
\end{definition}

\subsection{Name equivalence}

We take name equivalence, written $\nameeq$, to be the smallest
equivalence relation generated by the following rules.

\begin{mathpar}
\inferrule*[lab=Quote-drop]
{ }
{ \quotep{@{x}} \nameeq x }

\inferrule*[lab=Struct-equiv]
{ P \scong Q }
{ \quotep{P} \nameeq \quotep{Q} }
\end{mathpar}

The astute reader will have noticed that the mutual recursion of names
and processes imposes a mutual recursion on alpha-equivalence and
structural equivalence via name-equivalence. Fortunately, all of this
works out pleasantly and we may calculate in the natural way, free of
concern. The reader interested in the details is referred to the
appendix \ref{appendix:rho_details}.

\subsection{Substitution}

We use $\Proc$ for the set of processes, $\QProc$ for the set of
names, and $\id{\{}\vec{y} / \vec{x} \id{\}}$ to denote partial maps,
$s : \QProc \rightarrow \QProc$. A map, $s$ lifts, uniquely, to a map
on process terms, $\widehat{s} : \Proc \rightarrow \Proc$ by the
following equations.

\begin{mathpar}
  (0) \psubstp{Q}{P} := 0 \\
  (R \juxtap S) \psubstp{Q}{P}
  :=    
  (R)\psubstp{Q}{P} \juxtap (S) \psubstp{Q}{P} \\
  (x?(y).R) \psubstp{Q}{P}    
  :=    
  (x)\substp{Q}{P} (z)\concat( (R \psubstn{z}{y}) \psubstp{Q}{P} ) \\
  (\lift{x}{R}) \psubstp{Q}{P}  
  :=
  \lift{(x)\substp{Q}{P}}{ R \psubstp{Q}{P} } \\
%   (\dropn{x})  \psubstp{Q}{P}       
%   := 
%   \left\{ 
%     \begin{array}{ccc} 
%       \dropn{\quotep{Q}} & & x \nameeq \quotep{P} \\
%       \dropn{x} & & otherwise \\
%     \end{array}
%   \right. 
  (\dropn{x})  \psubstp{Q}{P}       
  := 
  \left\{ 
    \begin{array}{ccc} 
      Q & & x \nameeq \quotep{P} \\
      \dropn{x} & & otherwise \\
    \end{array}
  \right.
\end{mathpar}
 

where

\begin{eqnarray}
  (x)\id{\{} \lpquote Q \rpquote / \lpquote P \rpquote \id{\}}            = 
  \left\{ 
    \begin{array}{ccc}
      \lpquote Q \rpquote & & x \nameeq \lpquote P \rpquote \\
      x & & otherwise \\
    \end{array}
  \right. \nonumber
\end{eqnarray}

and $z$ is chosen distinct from $\quotep{P}$, $\quotep{Q}$, the free
names in $Q$, and all the names in $R$. Our $\alpha$-equivalence will
be built in the standard way from this substitution.

\begin{remark}\label{rem:no_self_referential_names}
  One consequence of these definitions is that $\forall P. \quotep{P}
  \not\in \freenames{P}$.
\end{remark}

\subsection{ Dynamic quote: an example }

Anticipating something of what's to come, consider applying the
substitution, $\widehat{\id{\{}u / z \id{\}}}$, to the following pair
of processes, $\lift{w}{y!(z)}$ and $w[ \lpquote y!(z) \rpquote ]$.

\begin{eqnarray}
	\lift{w}{y!(z)}\widehat{\id{\{}u / z \id{\}}}
		& = &
		\lift{w}{y!(u)} \nonumber\\
	w[ \lpquote y!(z) \rpquote ] \widehat{ \id{\{}u / z \id{\}} }
		& = &
		w[ \lpquote y!(z) \rpquote ] \nonumber
\end{eqnarray}

Because the body of the process between quotes is impervious to
substitution, we get radically different answers. In fact, by
examining the first process in an input context,
e.g. $x?(z).\lift{w}{y!(z)}$, we see that the process under the lift
operator may be shaped by prefixed inputs binding a name inside it. In
this sense, the lift operator will be seen as a way to dynamically
construct processes before reifying them as names.

Finally equipped with these standard features we can present the
dynamics of the calculus.

\subsubsection{Operational semantics} 

Finally, we introduce the computational dynamics. What marks these
algebras as distinct from other more traditionally studied algebraic
structures, e.g. vector spaces or polynomial rings, is the manner in
which dynamics is captured. In traditional structures, dynamics is typically
expressed through morphisms between such structures, as in linear maps
between vector spaces or morphisms between rings. In algebras
associated with the semantics of computation, the dynamics is
expressed as part of the algebraic structure itself, through a
reduction reduction relation typically denoted by $\red$. Below, we
give a recursive presentation of this relation for the calculus used
in the encoding.

$\red \subseteq \pi \times \pi$
$\red : \pi \to \mathcal{P}(\pi)$

\begin{mathpar}
  \inferrule* [lab=Comm] { \textsf{match}( x_{src}, x_{trgt} ) } { x_{trgt}?(y)P \; | \; x_{src}!\langle {Q} \rangle \red P\{\quotep{Q}/y}\} }
  \and \\
  \inferrule* [lab=Par] {{P} \red {P}'} {{{P} | {Q}} \red {{P}' | {Q}}}
  \and
  \inferrule* [lab=Equiv]{{{P} \scong {P}'} \andalso {{P}' \red {Q}'} \andalso {{Q}' \scong {Q}}}{{P} \red {Q}}
\end{mathpar}

\begin{eqnarray*}
  match_{\equiv} (\quotep{P},\quotep{Q}) & := & P \equiv Q \\
  match_{\dagger}(\quotep{P},\quotep{Q}) & := & \forall R. P|Q \red^{*} R => R \red^{*} 0 \\
  match_{K}(\quotep{P},\quotep{Q}) & := & K \mbox{ for some context } K
\end{eqnarray*}

$u?(x)P | u!\langle Q \rangle \red P\{\quotep{Q}/x\}$

%We write $\wred$ for $\red^*$, and $P\red$ if $\exists Q $ such that $ P \red Q$.
We write $P\red$ if $\exists Q $ such that $ P \red Q$ and $P\not\red$, otherwise.

\section{Replication}

As mentioned before, it is known that replication (and hence
recursion) can be implemented in a higher-order process algebra
\cite{SangiorgiWalker}. As our first example of calculation with the
machinery thus far presented we give the construction explicitly in
the {\rhoc}.

\begin{eqnarray}
	D_{x} & := & \prefix{x}{y}{(\binpar{\outputp{x}{y}}{@{y}})} \nonumber\\
	\bangp_{x}{P} & := & \binpar{{x}!\langle{\binpar{D_{x}}{P}}\rangle}{D_{x}} \nonumber
\end{eqnarray}

\begin{eqnarray}
	\bangp_{x}{P} & & \nonumber\\
	=
	& {x}!\langle{(\prefix{x}{y}{(\outputp{x}{y} | @{y})) | P}}\rangle 
	      | \prefix{x}{y}{(\outputp{x}{y} | @{y})} & \nonumber\\
	\red
	& (\outputp{x}{y} | @{y})\substn{\quotep{(\prefix{x}{y}{(@{y} | \outputp{x}{y})) | P}}}{y} & \nonumber\\
	=
	& \outputp{x}{\quotep{(\prefix{x}{y}{(\outputp{x}{y} | @{y})) | P}}}
	  | {(\prefix{x}{y}{(\outputp{x}{y} | @{y})) | P}} & \nonumber\\
	\red
	& \ldots & \nonumber\\
	\red^*
	& P | P | \ldots & \nonumber
\end{eqnarray}

Of course, this encoding, as an implementation, runs away, unfolding
$\bangp{P}$ eagerly. A lazier and more implementable replication
operator, restricted to input-guarded processes, may be obtained as follows.

\begin{eqnarray}
\bangp{\prefix{u}{v}{P}} 
	:= 
	\binpar{\lift{x}{\prefix{u}{v}{(\binpar{D(x)}{P})}}}{D(x)} \nonumber
\end{eqnarray}

\begin{remark}
  Note that the lazier definition still does not deal with summation
  or mixed summation (i.e. sums over input and output). The reader is
  invited to construct definitions of replication that deal with these
  features. 

  Further, the definitions are parameterized in a name, $x$. Can you,
  gentle reader, make a definition that eliminates this parameter and
  guarantees no accidental interaction between the replication
  machinery and the process being replicated -- i.e. no accidental
  sharing of names used by the process to get its work done and the
  name(s) used by the replication to effect copying. This latter
  revision of the definition of replication is crucial to obtaining
  the expected identity $!!P \sim !P$.
\end{remark}

\begin{remark}\label{rem:paradoxical_combinator}
  The reader familiar with the lambda calculus will have noticed the
  similarity between $D$ and the paradoxical combinator.

  [Ed. note: the existence of this seems to suggest we have to be more
  restrictive on the set of processes and names we admit if we are to
  support no-cloning.]
\end{remark}

\subsubsection{Bisimulation}

The computational dynamics gives rise to another kind of equivalence,
the equivalence of computational behavior. As previously mentioned
this is typically captured \emph{via} some form of bisimulation.

% The notion we use in this paper is weak barbed bisimulation
% \cite{milner91polyadicpi}.

The notion we use in this paper is derived from weak barbed
bisimulation \cite{milner91polyadicpi}. 

\begin{definition}
An \emph{observation relation}, $\downarrow_{\mathcal N}$, over a set
of names, $\mathcal N$, is the smallest relation satisfying the rules
below.

\infrule[Out-barb]{y \in {\mathcal N}, \; x \nameeq y}
		  {\outputp{x}{v} \downarrow_{\mathcal N} x}
\infrule[Par-barb]{\mbox{$P\downarrow_{\mathcal N} x$ or $Q\downarrow_{\mathcal N} x$}}
		  {\binpar{P}{Q} \downarrow_{\mathcal N} x}

We write $P \Downarrow_{\mathcal N} x$ if there is $Q$ such that 
$P \wred Q$ and $Q \downarrow_{\mathcal N} x$.
\end{definition}

\begin{definition}
%\label{def.bbisim}
An  ${\mathcal N}$-\emph{barbed bisimulation} over a set of names, ${\mathcal N}$, is a symmetric binary relation 
${\mathcal S}_{\mathcal N}$ between agents such that $P\rel{S}_{\mathcal N}Q$ implies:
\begin{enumerate}
\item If $P \red P'$ then $Q \wred Q'$ and $P'\rel{S}_{\mathcal N} Q'$.
\item If $P\downarrow_{\mathcal N} x$, then $Q\Downarrow_{\mathcal N} x$.
\end{enumerate}
$P$ is ${\mathcal N}$-barbed bisimilar to $Q$, written
$P \wbbisim_{\mathcal N} Q$, if $P \rel{S}_{\mathcal N} Q$ for some ${\mathcal N}$-barbed bisimulation ${\mathcal S}_{\mathcal N}$.
\end{definition}

$\mathcal{R} \subseteq \pi \times \pi$

$P \mathcal{R} Q => \forall P'. P \red P' \Rightarrow \exists Q'. Q \red Q', P' \mathcal{R} Q'$

$P \vdash x \Rightarrow Q \vdash x$

\begin{mathpar}
  \inferrule*[lab=Out-barb]{x \nameeq y}{{y}!\langle{Q}\rangle \vdash x}
  \and
  \inferrule*[lab=Par-barb]{\mbox{$P\vdash x$ or $Q\vdash x$}}{\binpar{P}{Q} \vdash x}
\end{mathpar}

\subsubsection{Contexts}

One of the principle advantages of computational calculi like the
$\pi$-calculus is a well-defined notion of context,
contextual-equivalence and a correlation between
contextual-equivalence and notions of bisimulation. The notion of
context allows the decomposition of a process into (sub-)process and
its syntactic environment, its context. Thus, a context may be
thought of as a process with a ``hole'' (written $\Box$) in it. The
application of a context $M$ to a process $P$, written $M[P]$, is
tantamount to filling the hole in $M$ with $P$. In this paper we do
not need the full weight of this theory, but do make use of the notion
of context in the proof the main theorem. 

\begin{mathpar}
  \inferrule* [lab=summation] {} {{M_{M},M_{N}} \bc \Box \;|\; x.M_{A} \;|\; M_{M}+M_{N}}
  \and
  \inferrule* [lab=agent] {} {{M_{A}} \bc (\vec{x})M_{P} \;| \; \clift{P_0,\ldots,M_{P},\ldots,P_N}}
  \and \\
  \inferrule* [lab=process] {} {{M_{P}} \bc M_{N} \;| \;P|M_{P} }
\end{mathpar} 

\begin{mathpar}
  \inferrule* [lab=sychronization] {} {M_{N} \bc \Box \;|\; x?M_{F} \;|\; x!M_{C}}
  \and
  \inferrule* [lab=abstraction] {} {{M_{F}} \bc (x)M_{P} }
  \and
  \inferrule* [lab=concretion] {} {{M_{C}} \bc \langle M_{P} \rangle }
  \and \\
  \inferrule* [lab=process] {} {{M_{P}} \bc M_{N} \;| \;P|M_{P} }
\end{mathpar}

\begin{definition}[contextual application] Given a context $M$, and
  process $P$, we define the \emph{contextual application}, $M[P] :=
  M\{P/\Box\}$. That is, the contextual application of M to P is the
  substitution of $P$ for $\Box$ in $M$.
\end{definition}

$\meaningof{-} : L \to \mathcal{P}(\pi)$

\begin{mathpar}
  \inferrule* [lab=collection] {} {\meaningof{true} = \pi, \and \meaningof{~E} = \pi \setminus \meaningof{E}, \and \meaningof{E_{1} \& E_{2}} = \meaningof{E_{1}} \cap \meaningof{E_{2}}}
\end{mathpar}

\begin{mathpar}
  \inferrule* [lab=structure] {} {\meaningof{0} = \{ P \in \pi | P \equiv 0 \}, \and \\ \meaningof{E_1 | E_2} = \{ P \in \pi | P \equiv P_{1} | P_{2}, P_{1} \in \meaningof{E_{1}}, P_{2} \in \meaningof{E_2}\} }
\end{mathpar}

\begin{mathpar}
 \inferrule* [lab=behavior] {} {\meaningof{\langle a?b \rangle E} = \{ P \in \pi | P \equiv Q | u?(y)P', \\ \and \\\\ \and \\ \;\;\; u \in \meaningof{a}, \forall z.P'\{z/y\} \in \meaningof{E\{z/b\}}\}, \and \\ \meaningof{a!E} = \{ P \in \pi | P \equiv Q | x!\langle P' \rangle, x \in \meaningof{a} P' \in \meaningof{E}\} }
\end{mathpar}

\begin{mathpar}
 \inferrule* [lab=nominal] {} {\meaningof{\quotep{E}} = \{ \quotep{P} \in \quotep{\pi} | P \in \meaningof{E} \}, \and \meaningof{\quotep{P}} = \{ \quotep{Q} \in \quotep{\pi} | P \equiv Q \} \and \\ \meaningof{@\quotep{E}} = \{ P \in \pi | P \equiv @x, x \in \meaningof{E} \}}
\end{mathpar}

\begin{eqnarray*}
  \\
  \meaningof{-} : TS \to ST
\end{eqnarray*}

\begin{eqnarray*}
  \\
  L : TS \to ST
\end{eqnarray*}

\begin{eqnarray*}
  \\
  P \models E \iff P \in \meaningof{E}
\end{eqnarray*}

\begin{eqnarray*}
  P \approx_{L} Q \iff \forall E \in L. P \models E \iff Q \models E
\end{eqnarray*}

\begin{eqnarray*}
  P \approx_{K} Q
\end{eqnarray*}

\begin{eqnarray*}
  P \approx Q
\end{eqnarray*}

$\approx_{K} = \approx = \approx_{L}$

\subsubsection{Contextual duality}

Note that contexts extend the quotation operation to a family of
operations from processes to names. Given a context, $M$, we can
define a \emph{nominal context}, $\quotep{M}$ by $\quotep{M}[P] :=
\quotep{M[P]}$. To foreshadow what is to come we observe that these
operations enjoy a duality with processes very much like the duality
between vectors and maps from vectors to scalars.

Further, because the calculus is essentially higher-order, we have a
correspondence between contexts and processes. More specifically,
given a name $x$ and a context $M$ we can construct $M^{*}_{x}$ such
that 

\begin{mathpar}
  M^{*}_{x} | \lift{x}{P} \red M[P]
\end{mathpar}

namely,

\begin{mathpar}
  M^{*}_{x} := x?(u).M[\dropn{u}]
\end{mathpar}

The dependence of $M^{*}_{x}$ on a name makes it an abstraction, 

\begin{mathpar}
  M^{*} := (x)x?(u).M[\dropn{u}]
\end{mathpar}

\subsection{Additional notation}

It will sometimes be convenient to denote the process a name
quotes. We already have the notation $x = \quotep{P}$, but it will be
convenient to introduce an alternate notation, $\procn{x}$, when we
want to emphasize the connection to the use of the name. Note that, by
virtue of name equivalence, $\quotep{\procn{x}} \nameeq x$; so, the
notation is consistent with previous definitions.

Further, because names have structure it is possible to effect
substitutions on the basis of that structure. This means we need to
upgrade our notation for substitutions, which we accomplish by
adapting comprehension notation. Thus,

\begin{mathpar}
  P\{ y / x : x \in S \}
\end{mathpar}

is interpreted to mean the process derived from P by replacing (in a
capture-avoiding manner) each occurrence of $x$ in $S$ by $y$. For example,

\begin{mathpar}
  P\{ \quotep{\procn{x}|\procn{x}} / x : x \in \freenames{P} \}
\end{mathpar}

will replace each (occurrence) of a free name $x$ in $P$ by
$\quotep{\procn{x}|\procn{x}}$.

Also, we will avail ourselves of the notation $x^{L}$ and $x^{R}$ to
denote injections of a name into disjoint copies of the name
space. There are numerous ways to accomplish this. One example can be
found in \cite{MeredithR05}. This notation overloads to vectors of
names: $\vec{x}^{\pi} := (x_{i}^{\pi} \; : \; 0 \leq i < |\vec{x}| )$ where $\pi \in \{L,R\}$.

We also use $P^{\Box} := P|\Box$.

In \cite{MeredithR05} an interpretation of the new operator is
given. It turns out that there are several possible interpretations
all enjoying the requisite algebraic properties of the operator (see
\cite{milner91polyadicpi}). We will therefore make liberal use of
$(\nu\; \vec{x})P$.

% subsection the_syntax_and_semantics_of_the_notation_system (end)   

\input{qm2pi.qmops} 

\input{qm2pi.sterngerlach} 

\input{qm2pi.metric} 

% section concurrent_process_calculi (end)

%\input{qm2pi.proofsketch}

% section proof sketch (end)

%\input{qm2pi.slviaknots} 

% section spatial logic via knots (end)

\input{qm2pi.conclusion}

% section conclusion (end)

%\input{qm2pi.dtcodes} 

% section wiring algorithm (end)

\input{qm2pi.ack} 

% section acknowledgments (end)

\newpage


\bibliographystyle{plain}   
\bibliography{../../biblios/main.bib}

\input{qm2pi.rhodetails}

\end{document}

 

%\ifpdf
%\usepackage[pdftex]{graphicx}
%\else
%\usepackage{graphicx}
%\fi

 % \ifpdf
%  \usepackage{pdfsync}
%  \if


%\title{Brief Article}
%\author{David F. Snyder}
%\author{L.G. Meredith}

%\address{Dept. of Math., Texas State University--San Marcos, San Marcos, TX 78666}
       
\pagestyle{empty}


\begin{document}

\lstset{language=[Objective]Caml,frame=shadowbox}

\documentclass[12pt]{llncs}
%\documentclass{jktr}

\usepackage[pdftex]{hyperref}                   
\usepackage {listings}
\usepackage {mathpartir}
\usepackage{bcprules}
%\usepackage{listings}
                       
\usepackage{graphicx} 
%\usepackage[margins=2.5cm,nohead,nofoot]{geometry}
%\usepackage{geometry}
\usepackage{amsfonts}
\usepackage{amstext}
\usepackage{latexsym}
\usepackage{amssymb}
\usepackage{color}


%\include{myPreamble}
\include{qm2pi.local} 

%\ifpdf
%\usepackage[pdftex]{graphicx}
%\else
%\usepackage{graphicx}
%\fi

 % \ifpdf
%  \usepackage{pdfsync}
%  \if


%\title{Brief Article}
%\author{David F. Snyder}
%\author{L.G. Meredith}

%\address{Dept. of Math., Texas State University--San Marcos, San Marcos, TX 78666}
       
\pagestyle{empty}


\begin{document}

\lstset{language=[Objective]Caml,frame=shadowbox}

\input{qm2pi.front}

% section front matter (end)

\input{qm2pi.intro} 
 
% section introduction (end)

% \input{qm2pi.knotations} 

% section notation (end)

\input{qm2pi.process.calculi} 

% section concurrent_process_calculi_and_spatial_logics_ (end)
    
%\input{qm2pi.knots2pi} 

%\input{qm2pi.trefoil} 

%\input{qm2pi.mainthm} 

% subsection basic_interpretation (end)

%\input{qm2pi.rho.presentation} 
\subsection{The syntax and semantics of the notation system}\label{sub:the_syntax_and_semantics_of_the_notation_system} % (fold)

We now summarize a technical presentation of the calculus that
embodies our theory of dynamics. The typical presentation of such a
calculus follows the style of giving generators and relations on
them. The grammar, below, describing term constructors, freely
generates the set of processes, $\Proc$. This set is then quotiented
by a relation known as structural congruence and it is over this set
that the notion of dynamics is expressed. This presentation is
essentially that of \cite{MeredithR05} with the addition of
polyadicity and summation. For readability we have relegated some of
the technical subtleties to an appendix.

\subsubsection{Process grammar}\label{subsub:process_grammar}

\begin{mathpar}
  \inferrule* [lab=synchronization] {} {{M} \bc \pzero \;|\; x?F \;|\; x!C }
  \and
  \inferrule* [lab=abstraction] {} {{F} \bc (x)P}
  \and
  \inferrule* [lab=concretion] {} {{C} \bc \langle Q \rangle}
  \and
  \inferrule* [lab=process] {} {{P,Q} \bc M \;| \;P|Q \;|\; @{x}}
  \and
  \inferrule* [lab=name] {} {{x} \bc \quotep{P}}
\end{mathpar} 

Note that $\vec{x}$ (resp. $\vec{P}$) denotes a vector of names
(resp. processes) of length $|\vec{x}|$ (resp. $|\vec{P}|$). We adopt
the following useful abbreviations.

\begin{mathpar}
   x?(\vec{y}).P := x.(\vec{y})P \and  x\clift{\vec{P}} := x.\clift{\vec{P}}
   \and x!(y) := \lift{x}{\dropn{y}}
   \and \Pi_{i=0}^{n-1}P_i := P_0 | \ldots | P_{n-1}
\end{mathpar}

\subsubsection{Structural congruence}

\paragraph{Free and bound names and alpha-equivalence.} At the
core of structural equivalence is alpha-equivalence which identifies
process that are the same up to a change of variable. Formally, we
recognize the distinction between free and bound names. The free names
of a process, $\freenames{P}$, may be calculated recursively as
follows:

\begin{mathpar}
\freenames{\pzero} := \emptyset
  \and \\
  \freenames{x?(y).P} := \{ x \} \cup (\freenames{P} \setminus \{ y \})
  \and 
  \freenames{x!\langle P \rangle} := \{ x \} \cup \{ P \} 
  \and \\
  \freenames{P|Q} := \freenames{P} \cup \freenames{Q}
  \and \\
  \freenames{@{x}} := \{ x \}
\end{mathpar}

$\pi$
$\quotep{\pi}$

$\freenames{-} : \pi \to \mathcal{P}(\quotep{\pi})$

\begin{eqnarray*}
  \freenames{\pzero} & := & \emptyset \\
  \freenames{x?(y).P} & := & \{ x \} \cup (\freenames{P} \setminus \{ y \}) \\
  \freenames{x!\langle P \rangle} & := & \{ x \} \cup \{ P \} \\
  \freenames{P|Q} & := & \freenames{P} \cup \freenames{Q} \\
  \freenames{\dropn{x}} & := & \{ x \}
\end{eqnarray*}

The bound names of a process, $\boundnames{P}$, are those names occurring in $P$
that are not free. For example, in $x?(y).0$, the name $x$ is free, while $y$ is bound.

\begin{mathpar}
  \inferrule* [lab=monoidal-laws] {} { P|Q \equiv Q|P \and P|0 \equiv P \and P|(Q|R) \equiv (P|Q)|R }
\end{mathpar}

\begin{mathpar}
  \inferrule* [lab=alpha-equivalence] {} { (x)P \equiv (y)P\{y/x\} \and y \not\in \freenames{P} }
\end{mathpar}

\begin{definition}
Then two processes, $P,Q$, are alpha-equivalent if $P = Q\{\vec{y}/\vec{x}\}$ for
some $\vec{x} \in \boundnames{Q},\vec{y} \in \boundnames{P}$, where $Q\{\vec{y}/\vec{x}\}$
denotes the capture-avoiding substitution of $\vec{y}$ for $\vec{x}$ in $Q$.
\end{definition}

\begin{definition}
  The {\em structural congruence} \cite{SangiorgiWalker} , $\equiv$,
  between processes is the least congruence containing
  alpha-equivalence, satisfying the abelian monoid laws
  (associativity, commutativity and $\pzero$ as identity) for parallel
  composition $|$ and for summation $+$.
\end{definition}

\subsection{Name equivalence}

We take name equivalence, written $\nameeq$, to be the smallest
equivalence relation generated by the following rules.

\begin{mathpar}
\inferrule*[lab=Quote-drop]
{ }
{ \quotep{@{x}} \nameeq x }

\inferrule*[lab=Struct-equiv]
{ P \scong Q }
{ \quotep{P} \nameeq \quotep{Q} }
\end{mathpar}

The astute reader will have noticed that the mutual recursion of names
and processes imposes a mutual recursion on alpha-equivalence and
structural equivalence via name-equivalence. Fortunately, all of this
works out pleasantly and we may calculate in the natural way, free of
concern. The reader interested in the details is referred to the
appendix \ref{appendix:rho_details}.

\subsection{Substitution}

We use $\Proc$ for the set of processes, $\QProc$ for the set of
names, and $\id{\{}\vec{y} / \vec{x} \id{\}}$ to denote partial maps,
$s : \QProc \rightarrow \QProc$. A map, $s$ lifts, uniquely, to a map
on process terms, $\widehat{s} : \Proc \rightarrow \Proc$ by the
following equations.

\begin{mathpar}
  (0) \psubstp{Q}{P} := 0 \\
  (R \juxtap S) \psubstp{Q}{P}
  :=    
  (R)\psubstp{Q}{P} \juxtap (S) \psubstp{Q}{P} \\
  (x?(y).R) \psubstp{Q}{P}    
  :=    
  (x)\substp{Q}{P} (z)\concat( (R \psubstn{z}{y}) \psubstp{Q}{P} ) \\
  (\lift{x}{R}) \psubstp{Q}{P}  
  :=
  \lift{(x)\substp{Q}{P}}{ R \psubstp{Q}{P} } \\
%   (\dropn{x})  \psubstp{Q}{P}       
%   := 
%   \left\{ 
%     \begin{array}{ccc} 
%       \dropn{\quotep{Q}} & & x \nameeq \quotep{P} \\
%       \dropn{x} & & otherwise \\
%     \end{array}
%   \right. 
  (\dropn{x})  \psubstp{Q}{P}       
  := 
  \left\{ 
    \begin{array}{ccc} 
      Q & & x \nameeq \quotep{P} \\
      \dropn{x} & & otherwise \\
    \end{array}
  \right.
\end{mathpar}
 

where

\begin{eqnarray}
  (x)\id{\{} \lpquote Q \rpquote / \lpquote P \rpquote \id{\}}            = 
  \left\{ 
    \begin{array}{ccc}
      \lpquote Q \rpquote & & x \nameeq \lpquote P \rpquote \\
      x & & otherwise \\
    \end{array}
  \right. \nonumber
\end{eqnarray}

and $z$ is chosen distinct from $\quotep{P}$, $\quotep{Q}$, the free
names in $Q$, and all the names in $R$. Our $\alpha$-equivalence will
be built in the standard way from this substitution.

\begin{remark}\label{rem:no_self_referential_names}
  One consequence of these definitions is that $\forall P. \quotep{P}
  \not\in \freenames{P}$.
\end{remark}

\subsection{ Dynamic quote: an example }

Anticipating something of what's to come, consider applying the
substitution, $\widehat{\id{\{}u / z \id{\}}}$, to the following pair
of processes, $\lift{w}{y!(z)}$ and $w[ \lpquote y!(z) \rpquote ]$.

\begin{eqnarray}
	\lift{w}{y!(z)}\widehat{\id{\{}u / z \id{\}}}
		& = &
		\lift{w}{y!(u)} \nonumber\\
	w[ \lpquote y!(z) \rpquote ] \widehat{ \id{\{}u / z \id{\}} }
		& = &
		w[ \lpquote y!(z) \rpquote ] \nonumber
\end{eqnarray}

Because the body of the process between quotes is impervious to
substitution, we get radically different answers. In fact, by
examining the first process in an input context,
e.g. $x?(z).\lift{w}{y!(z)}$, we see that the process under the lift
operator may be shaped by prefixed inputs binding a name inside it. In
this sense, the lift operator will be seen as a way to dynamically
construct processes before reifying them as names.

Finally equipped with these standard features we can present the
dynamics of the calculus.

\subsubsection{Operational semantics} 

Finally, we introduce the computational dynamics. What marks these
algebras as distinct from other more traditionally studied algebraic
structures, e.g. vector spaces or polynomial rings, is the manner in
which dynamics is captured. In traditional structures, dynamics is typically
expressed through morphisms between such structures, as in linear maps
between vector spaces or morphisms between rings. In algebras
associated with the semantics of computation, the dynamics is
expressed as part of the algebraic structure itself, through a
reduction reduction relation typically denoted by $\red$. Below, we
give a recursive presentation of this relation for the calculus used
in the encoding.

$\red \subseteq \pi \times \pi$
$\red : \pi \to \mathcal{P}(\pi)$

\begin{mathpar}
  \inferrule* [lab=Comm] { \textsf{match}( x_{src}, x_{trgt} ) } { x_{trgt}?(y)P \; | \; x_{src}!\langle {Q} \rangle \red P\{\quotep{Q}/y}\} }
  \and \\
  \inferrule* [lab=Par] {{P} \red {P}'} {{{P} | {Q}} \red {{P}' | {Q}}}
  \and
  \inferrule* [lab=Equiv]{{{P} \scong {P}'} \andalso {{P}' \red {Q}'} \andalso {{Q}' \scong {Q}}}{{P} \red {Q}}
\end{mathpar}

\begin{eqnarray*}
  match_{\equiv} (\quotep{P},\quotep{Q}) & := & P \equiv Q \\
  match_{\dagger}(\quotep{P},\quotep{Q}) & := & \forall R. P|Q \red^{*} R => R \red^{*} 0 \\
  match_{K}(\quotep{P},\quotep{Q}) & := & K \mbox{ for some context } K
\end{eqnarray*}

$u?(x)P | u!\langle Q \rangle \red P\{\quotep{Q}/x\}$

%We write $\wred$ for $\red^*$, and $P\red$ if $\exists Q $ such that $ P \red Q$.
We write $P\red$ if $\exists Q $ such that $ P \red Q$ and $P\not\red$, otherwise.

\section{Replication}

As mentioned before, it is known that replication (and hence
recursion) can be implemented in a higher-order process algebra
\cite{SangiorgiWalker}. As our first example of calculation with the
machinery thus far presented we give the construction explicitly in
the {\rhoc}.

\begin{eqnarray}
	D_{x} & := & \prefix{x}{y}{(\binpar{\outputp{x}{y}}{@{y}})} \nonumber\\
	\bangp_{x}{P} & := & \binpar{{x}!\langle{\binpar{D_{x}}{P}}\rangle}{D_{x}} \nonumber
\end{eqnarray}

\begin{eqnarray}
	\bangp_{x}{P} & & \nonumber\\
	=
	& {x}!\langle{(\prefix{x}{y}{(\outputp{x}{y} | @{y})) | P}}\rangle 
	      | \prefix{x}{y}{(\outputp{x}{y} | @{y})} & \nonumber\\
	\red
	& (\outputp{x}{y} | @{y})\substn{\quotep{(\prefix{x}{y}{(@{y} | \outputp{x}{y})) | P}}}{y} & \nonumber\\
	=
	& \outputp{x}{\quotep{(\prefix{x}{y}{(\outputp{x}{y} | @{y})) | P}}}
	  | {(\prefix{x}{y}{(\outputp{x}{y} | @{y})) | P}} & \nonumber\\
	\red
	& \ldots & \nonumber\\
	\red^*
	& P | P | \ldots & \nonumber
\end{eqnarray}

Of course, this encoding, as an implementation, runs away, unfolding
$\bangp{P}$ eagerly. A lazier and more implementable replication
operator, restricted to input-guarded processes, may be obtained as follows.

\begin{eqnarray}
\bangp{\prefix{u}{v}{P}} 
	:= 
	\binpar{\lift{x}{\prefix{u}{v}{(\binpar{D(x)}{P})}}}{D(x)} \nonumber
\end{eqnarray}

\begin{remark}
  Note that the lazier definition still does not deal with summation
  or mixed summation (i.e. sums over input and output). The reader is
  invited to construct definitions of replication that deal with these
  features. 

  Further, the definitions are parameterized in a name, $x$. Can you,
  gentle reader, make a definition that eliminates this parameter and
  guarantees no accidental interaction between the replication
  machinery and the process being replicated -- i.e. no accidental
  sharing of names used by the process to get its work done and the
  name(s) used by the replication to effect copying. This latter
  revision of the definition of replication is crucial to obtaining
  the expected identity $!!P \sim !P$.
\end{remark}

\begin{remark}\label{rem:paradoxical_combinator}
  The reader familiar with the lambda calculus will have noticed the
  similarity between $D$ and the paradoxical combinator.

  [Ed. note: the existence of this seems to suggest we have to be more
  restrictive on the set of processes and names we admit if we are to
  support no-cloning.]
\end{remark}

\subsubsection{Bisimulation}

The computational dynamics gives rise to another kind of equivalence,
the equivalence of computational behavior. As previously mentioned
this is typically captured \emph{via} some form of bisimulation.

% The notion we use in this paper is weak barbed bisimulation
% \cite{milner91polyadicpi}.

The notion we use in this paper is derived from weak barbed
bisimulation \cite{milner91polyadicpi}. 

\begin{definition}
An \emph{observation relation}, $\downarrow_{\mathcal N}$, over a set
of names, $\mathcal N$, is the smallest relation satisfying the rules
below.

\infrule[Out-barb]{y \in {\mathcal N}, \; x \nameeq y}
		  {\outputp{x}{v} \downarrow_{\mathcal N} x}
\infrule[Par-barb]{\mbox{$P\downarrow_{\mathcal N} x$ or $Q\downarrow_{\mathcal N} x$}}
		  {\binpar{P}{Q} \downarrow_{\mathcal N} x}

We write $P \Downarrow_{\mathcal N} x$ if there is $Q$ such that 
$P \wred Q$ and $Q \downarrow_{\mathcal N} x$.
\end{definition}

\begin{definition}
%\label{def.bbisim}
An  ${\mathcal N}$-\emph{barbed bisimulation} over a set of names, ${\mathcal N}$, is a symmetric binary relation 
${\mathcal S}_{\mathcal N}$ between agents such that $P\rel{S}_{\mathcal N}Q$ implies:
\begin{enumerate}
\item If $P \red P'$ then $Q \wred Q'$ and $P'\rel{S}_{\mathcal N} Q'$.
\item If $P\downarrow_{\mathcal N} x$, then $Q\Downarrow_{\mathcal N} x$.
\end{enumerate}
$P$ is ${\mathcal N}$-barbed bisimilar to $Q$, written
$P \wbbisim_{\mathcal N} Q$, if $P \rel{S}_{\mathcal N} Q$ for some ${\mathcal N}$-barbed bisimulation ${\mathcal S}_{\mathcal N}$.
\end{definition}

$\mathcal{R} \subseteq \pi \times \pi$

$P \mathcal{R} Q => \forall P'. P \red P' \Rightarrow \exists Q'. Q \red Q', P' \mathcal{R} Q'$

$P \vdash x \Rightarrow Q \vdash x$

\begin{mathpar}
  \inferrule*[lab=Out-barb]{x \nameeq y}{{y}!\langle{Q}\rangle \vdash x}
  \and
  \inferrule*[lab=Par-barb]{\mbox{$P\vdash x$ or $Q\vdash x$}}{\binpar{P}{Q} \vdash x}
\end{mathpar}

\subsubsection{Contexts}

One of the principle advantages of computational calculi like the
$\pi$-calculus is a well-defined notion of context,
contextual-equivalence and a correlation between
contextual-equivalence and notions of bisimulation. The notion of
context allows the decomposition of a process into (sub-)process and
its syntactic environment, its context. Thus, a context may be
thought of as a process with a ``hole'' (written $\Box$) in it. The
application of a context $M$ to a process $P$, written $M[P]$, is
tantamount to filling the hole in $M$ with $P$. In this paper we do
not need the full weight of this theory, but do make use of the notion
of context in the proof the main theorem. 

\begin{mathpar}
  \inferrule* [lab=summation] {} {{M_{M},M_{N}} \bc \Box \;|\; x.M_{A} \;|\; M_{M}+M_{N}}
  \and
  \inferrule* [lab=agent] {} {{M_{A}} \bc (\vec{x})M_{P} \;| \; \clift{P_0,\ldots,M_{P},\ldots,P_N}}
  \and \\
  \inferrule* [lab=process] {} {{M_{P}} \bc M_{N} \;| \;P|M_{P} }
\end{mathpar} 

\begin{mathpar}
  \inferrule* [lab=sychronization] {} {M_{N} \bc \Box \;|\; x?M_{F} \;|\; x!M_{C}}
  \and
  \inferrule* [lab=abstraction] {} {{M_{F}} \bc (x)M_{P} }
  \and
  \inferrule* [lab=concretion] {} {{M_{C}} \bc \langle M_{P} \rangle }
  \and \\
  \inferrule* [lab=process] {} {{M_{P}} \bc M_{N} \;| \;P|M_{P} }
\end{mathpar}

\begin{definition}[contextual application] Given a context $M$, and
  process $P$, we define the \emph{contextual application}, $M[P] :=
  M\{P/\Box\}$. That is, the contextual application of M to P is the
  substitution of $P$ for $\Box$ in $M$.
\end{definition}

$\meaningof{-} : L \to \mathcal{P}(\pi)$

\begin{mathpar}
  \inferrule* [lab=collection] {} {\meaningof{true} = \pi, \and \meaningof{~E} = \pi \setminus \meaningof{E}, \and \meaningof{E_{1} \& E_{2}} = \meaningof{E_{1}} \cap \meaningof{E_{2}}}
\end{mathpar}

\begin{mathpar}
  \inferrule* [lab=structure] {} {\meaningof{0} = \{ P \in \pi | P \equiv 0 \}, \and \\ \meaningof{E_1 | E_2} = \{ P \in \pi | P \equiv P_{1} | P_{2}, P_{1} \in \meaningof{E_{1}}, P_{2} \in \meaningof{E_2}\} }
\end{mathpar}

\begin{mathpar}
 \inferrule* [lab=behavior] {} {\meaningof{\langle a?b \rangle E} = \{ P \in \pi | P \equiv Q | u?(y)P', \\ \and \\\\ \and \\ \;\;\; u \in \meaningof{a}, \forall z.P'\{z/y\} \in \meaningof{E\{z/b\}}\}, \and \\ \meaningof{a!E} = \{ P \in \pi | P \equiv Q | x!\langle P' \rangle, x \in \meaningof{a} P' \in \meaningof{E}\} }
\end{mathpar}

\begin{mathpar}
 \inferrule* [lab=nominal] {} {\meaningof{\quotep{E}} = \{ \quotep{P} \in \quotep{\pi} | P \in \meaningof{E} \}, \and \meaningof{\quotep{P}} = \{ \quotep{Q} \in \quotep{\pi} | P \equiv Q \} \and \\ \meaningof{@\quotep{E}} = \{ P \in \pi | P \equiv @x, x \in \meaningof{E} \}}
\end{mathpar}

\begin{eqnarray*}
  \\
  \meaningof{-} : TS \to ST
\end{eqnarray*}

\begin{eqnarray*}
  \\
  L : TS \to ST
\end{eqnarray*}

\begin{eqnarray*}
  \\
  P \models E \iff P \in \meaningof{E}
\end{eqnarray*}

\begin{eqnarray*}
  P \approx_{L} Q \iff \forall E \in L. P \models E \iff Q \models E
\end{eqnarray*}

\begin{eqnarray*}
  P \approx_{K} Q
\end{eqnarray*}

\begin{eqnarray*}
  P \approx Q
\end{eqnarray*}

$\approx_{K} = \approx = \approx_{L}$

\subsubsection{Contextual duality}

Note that contexts extend the quotation operation to a family of
operations from processes to names. Given a context, $M$, we can
define a \emph{nominal context}, $\quotep{M}$ by $\quotep{M}[P] :=
\quotep{M[P]}$. To foreshadow what is to come we observe that these
operations enjoy a duality with processes very much like the duality
between vectors and maps from vectors to scalars.

Further, because the calculus is essentially higher-order, we have a
correspondence between contexts and processes. More specifically,
given a name $x$ and a context $M$ we can construct $M^{*}_{x}$ such
that 

\begin{mathpar}
  M^{*}_{x} | \lift{x}{P} \red M[P]
\end{mathpar}

namely,

\begin{mathpar}
  M^{*}_{x} := x?(u).M[\dropn{u}]
\end{mathpar}

The dependence of $M^{*}_{x}$ on a name makes it an abstraction, 

\begin{mathpar}
  M^{*} := (x)x?(u).M[\dropn{u}]
\end{mathpar}

\subsection{Additional notation}

It will sometimes be convenient to denote the process a name
quotes. We already have the notation $x = \quotep{P}$, but it will be
convenient to introduce an alternate notation, $\procn{x}$, when we
want to emphasize the connection to the use of the name. Note that, by
virtue of name equivalence, $\quotep{\procn{x}} \nameeq x$; so, the
notation is consistent with previous definitions.

Further, because names have structure it is possible to effect
substitutions on the basis of that structure. This means we need to
upgrade our notation for substitutions, which we accomplish by
adapting comprehension notation. Thus,

\begin{mathpar}
  P\{ y / x : x \in S \}
\end{mathpar}

is interpreted to mean the process derived from P by replacing (in a
capture-avoiding manner) each occurrence of $x$ in $S$ by $y$. For example,

\begin{mathpar}
  P\{ \quotep{\procn{x}|\procn{x}} / x : x \in \freenames{P} \}
\end{mathpar}

will replace each (occurrence) of a free name $x$ in $P$ by
$\quotep{\procn{x}|\procn{x}}$.

Also, we will avail ourselves of the notation $x^{L}$ and $x^{R}$ to
denote injections of a name into disjoint copies of the name
space. There are numerous ways to accomplish this. One example can be
found in \cite{MeredithR05}. This notation overloads to vectors of
names: $\vec{x}^{\pi} := (x_{i}^{\pi} \; : \; 0 \leq i < |\vec{x}| )$ where $\pi \in \{L,R\}$.

We also use $P^{\Box} := P|\Box$.

In \cite{MeredithR05} an interpretation of the new operator is
given. It turns out that there are several possible interpretations
all enjoying the requisite algebraic properties of the operator (see
\cite{milner91polyadicpi}). We will therefore make liberal use of
$(\nu\; \vec{x})P$.

% subsection the_syntax_and_semantics_of_the_notation_system (end)   

\input{qm2pi.qmops} 

\input{qm2pi.sterngerlach} 

\input{qm2pi.metric} 

% section concurrent_process_calculi (end)

%\input{qm2pi.proofsketch}

% section proof sketch (end)

%\input{qm2pi.slviaknots} 

% section spatial logic via knots (end)

\input{qm2pi.conclusion}

% section conclusion (end)

%\input{qm2pi.dtcodes} 

% section wiring algorithm (end)

\input{qm2pi.ack} 

% section acknowledgments (end)

\newpage


\bibliographystyle{plain}   
\bibliography{../../biblios/main.bib}

\input{qm2pi.rhodetails}

\end{document}



% section front matter (end)

\section{Introduction}\label{sec:introduction} % (fold)
In this draft of the material i am going to have to dispense with the
usual writing conventions adopted in papers on these topics. i'm going
to have adopt whatever tone i need at the time i'm writing up the
calculations. Sometimes this may be very conversational; others it may
be the barest mathematical grunts; others still it may be that i have
lifted text from one of my other papers because the exposition of some
point was better said there. i hope that my readers are not unduly put
out by this decision. i'm not doing this to flout convention or be
rebellious. i find these calculations very technically challenging. To
keep everything going technically, something has to give; i have to
let go of some cognitive burden. So, the academic writing style --
with all of its trade-offs in terms of facilitating technical
communication -- is what i'm letting go of. Perhaps subsequent drafts
can be tightened and polished, but for now, i'm going to speak as if
we were sitting together in a coffee shop with a laptop, wifi and a
pad of paper and a pencil.

So, here's what i have to say. We -- you and i, comfortably ensconced
in our coffee shop and well-equipped with our tools -- can realize and
carry out the calculations of quantum mechanics over a very different
formal theory of dynamics, a formal theory of dynamics that
corresponds to a theory of concurrent computation with
\emph{reflection}. It has the advantage that the underlying theory is
already `quantized', but supports analogues all of the continuuous
operations. Strikingly, this underlying theory has recently been
connected with a notion of metric that we can show, by calculating
together, coincides with the metric induced by the inner product.

There are a lot of reasons why you might be interested in seeing
calculations of this form. Here's why i'm interested. For the past
several centuries there has been no competitor to the ``Newtonian''
account of dynamics. As a result the predominant share of accounts of
dynamical systems and situations have had to be formulated in terms of
the Newtonian machinery. i view this as an intellectually dangerous
position to occupy. Everything, despite it's intrinsic shape, turns
into a nail to be hit with this hammer. Recently, however, the theory
of computation has matured to the point where we have candidates for
theories of dynamics that offer very different perspective on
reasoning about dynamical systems and situations. Testing these
candidates against very successful accounts of dynamical situations,
like quantum mechanics, is going to give us some sense of how mature
they are and some measure of the quality of these accounts of
dynamics.

\subsection{Summary of contributions and outline of paper}

So, we're going to develop an interpretation of the operations of
quantum mechanics normally interpreted by Hilbert spaces and
operators. We're going to do this over a theory of computation. Note
that this is very different than the usual quantum computation program
which develops notions of computation over quantum mechanics. Rather,
we are developing a story that aligns with Wheeler's slogan: It from
Bit. To do this we will first provide an account of the theory of
computation at play here. Then we will dive into a calculation-driven
interpretation of the operations of quantum mechanics.

The reason we take this approach is that -- until very recently --
there hasn't been an axiomatic account of quantum mechanics. As a
result there has been no sharp delineation of the mathematical theory
supporting interpretation of the physical theory and the physical
theory, itself. So, ambient features of the maths are free to be
exploited (or supressed) without a real accounting of their physical
relevance. There is no sharp statement ``here's the physical theory''
qua \emph{theory} and ``here's the mathematical interpretation''
enabling a judgment of how faithful the interpretation is -- apart
from experimental observation. When there is an axiomatic account we
can judge how well a given mathematical formalism supports an
interpretation of the axioms, independent of
experimentation. Likewise, we can judge how well we have captured our
physical evidence and experience with our axiomatics, independent of
any specific mathematical implementation, with accidental detail that
may or may not have physical significance. 

In lieu of a fully fleshed out and vetted axiomatic account of quantum
mechanics, interpreting the operational notions in service of modeling
physical systems will have to suffice. In other words, we are not in
the business of providing a model of Hilbert spaces and operators. We
are in the business of providing a model of quantum mechanics because
we are motivated by testing our notions of dynamics against physical
theory; and, the predictive calculations of the physical theory must
serve as the best formulation -- shy of a fully fleshed out axiomatic
account -- of the physical theory itself (as they have for scientific
theories since time immemorial). Put another way, despite a
whole-hearted commitment to an It-from-Bit ontology, we are firmly
aligned with the shut-up-and-calculate camp as the best way to obtain
results either from the physical perspective or as a quality assurance
measure of our fledgling theory of dynamics.

In detail, we present a reflective process calculus. Then we develop
intuitive correspondences between the notions available in this
calculus and the usual physical notions supporting quantum mechanical
calculations. Thus, 

\begin{table}[htp]
  \center{
    \fbox{
      \begin{tabular}{c|c}
        quantum mechanics & process calculus \\
        \hline
        scalar & name \\
        state vector & process \\
        dual & contextual duals \\
        matrix & formal sums of process-context-dual pairs \\
        orthogonality & process annihilation \\
        inner product & execution-formula + quoting
      \end{tabular}
    }
  }
  \caption{QM - process calculi correspondences}
\end{table}

Then we tighten up these intuitions to operational definitions. We
employ the Dirac notation as the best proxy we can find for an
abstract syntax of the quantum mechanical notions. The definitions we
develop put us in contact with equational constraints coming from the
theory that we demonstrate the definitions and calculations satisfy.

This puts us in a position to shut up and calculate for the
Stern-Gerlach experimental set up, showing how these predictive
calculations become calculations on processes in our theory of a
reflective process calculus.

Penultimately, we demonstrate that the notion of metric coming from
the inner product coincides with the notion of metric available from
the theory of bisimulation. This demonstration gives us the right to
think of space as arising from behavior. Finally, we consider where we
might go from the new vantage point we have obtained.

% section introduction (end) 
 
% section introduction (end)

% \documentclass[12pt]{llncs}
%\documentclass{jktr}

\usepackage[pdftex]{hyperref}                   
\usepackage {listings}
\usepackage {mathpartir}
\usepackage{bcprules}
%\usepackage{listings}
                       
\usepackage{graphicx} 
%\usepackage[margins=2.5cm,nohead,nofoot]{geometry}
%\usepackage{geometry}
\usepackage{amsfonts}
\usepackage{amstext}
\usepackage{latexsym}
\usepackage{amssymb}
\usepackage{color}


%\include{myPreamble}
\include{qm2pi.local} 

%\ifpdf
%\usepackage[pdftex]{graphicx}
%\else
%\usepackage{graphicx}
%\fi

 % \ifpdf
%  \usepackage{pdfsync}
%  \if


%\title{Brief Article}
%\author{David F. Snyder}
%\author{L.G. Meredith}

%\address{Dept. of Math., Texas State University--San Marcos, San Marcos, TX 78666}
       
\pagestyle{empty}


\begin{document}

\lstset{language=[Objective]Caml,frame=shadowbox}

\input{qm2pi.front}

% section front matter (end)

\input{qm2pi.intro} 
 
% section introduction (end)

% \input{qm2pi.knotations} 

% section notation (end)

\input{qm2pi.process.calculi} 

% section concurrent_process_calculi_and_spatial_logics_ (end)
    
%\input{qm2pi.knots2pi} 

%\input{qm2pi.trefoil} 

%\input{qm2pi.mainthm} 

% subsection basic_interpretation (end)

%\input{qm2pi.rho.presentation} 
\subsection{The syntax and semantics of the notation system}\label{sub:the_syntax_and_semantics_of_the_notation_system} % (fold)

We now summarize a technical presentation of the calculus that
embodies our theory of dynamics. The typical presentation of such a
calculus follows the style of giving generators and relations on
them. The grammar, below, describing term constructors, freely
generates the set of processes, $\Proc$. This set is then quotiented
by a relation known as structural congruence and it is over this set
that the notion of dynamics is expressed. This presentation is
essentially that of \cite{MeredithR05} with the addition of
polyadicity and summation. For readability we have relegated some of
the technical subtleties to an appendix.

\subsubsection{Process grammar}\label{subsub:process_grammar}

\begin{mathpar}
  \inferrule* [lab=synchronization] {} {{M} \bc \pzero \;|\; x?F \;|\; x!C }
  \and
  \inferrule* [lab=abstraction] {} {{F} \bc (x)P}
  \and
  \inferrule* [lab=concretion] {} {{C} \bc \langle Q \rangle}
  \and
  \inferrule* [lab=process] {} {{P,Q} \bc M \;| \;P|Q \;|\; @{x}}
  \and
  \inferrule* [lab=name] {} {{x} \bc \quotep{P}}
\end{mathpar} 

Note that $\vec{x}$ (resp. $\vec{P}$) denotes a vector of names
(resp. processes) of length $|\vec{x}|$ (resp. $|\vec{P}|$). We adopt
the following useful abbreviations.

\begin{mathpar}
   x?(\vec{y}).P := x.(\vec{y})P \and  x\clift{\vec{P}} := x.\clift{\vec{P}}
   \and x!(y) := \lift{x}{\dropn{y}}
   \and \Pi_{i=0}^{n-1}P_i := P_0 | \ldots | P_{n-1}
\end{mathpar}

\subsubsection{Structural congruence}

\paragraph{Free and bound names and alpha-equivalence.} At the
core of structural equivalence is alpha-equivalence which identifies
process that are the same up to a change of variable. Formally, we
recognize the distinction between free and bound names. The free names
of a process, $\freenames{P}$, may be calculated recursively as
follows:

\begin{mathpar}
\freenames{\pzero} := \emptyset
  \and \\
  \freenames{x?(y).P} := \{ x \} \cup (\freenames{P} \setminus \{ y \})
  \and 
  \freenames{x!\langle P \rangle} := \{ x \} \cup \{ P \} 
  \and \\
  \freenames{P|Q} := \freenames{P} \cup \freenames{Q}
  \and \\
  \freenames{@{x}} := \{ x \}
\end{mathpar}

$\pi$
$\quotep{\pi}$

$\freenames{-} : \pi \to \mathcal{P}(\quotep{\pi})$

\begin{eqnarray*}
  \freenames{\pzero} & := & \emptyset \\
  \freenames{x?(y).P} & := & \{ x \} \cup (\freenames{P} \setminus \{ y \}) \\
  \freenames{x!\langle P \rangle} & := & \{ x \} \cup \{ P \} \\
  \freenames{P|Q} & := & \freenames{P} \cup \freenames{Q} \\
  \freenames{\dropn{x}} & := & \{ x \}
\end{eqnarray*}

The bound names of a process, $\boundnames{P}$, are those names occurring in $P$
that are not free. For example, in $x?(y).0$, the name $x$ is free, while $y$ is bound.

\begin{mathpar}
  \inferrule* [lab=monoidal-laws] {} { P|Q \equiv Q|P \and P|0 \equiv P \and P|(Q|R) \equiv (P|Q)|R }
\end{mathpar}

\begin{mathpar}
  \inferrule* [lab=alpha-equivalence] {} { (x)P \equiv (y)P\{y/x\} \and y \not\in \freenames{P} }
\end{mathpar}

\begin{definition}
Then two processes, $P,Q$, are alpha-equivalent if $P = Q\{\vec{y}/\vec{x}\}$ for
some $\vec{x} \in \boundnames{Q},\vec{y} \in \boundnames{P}$, where $Q\{\vec{y}/\vec{x}\}$
denotes the capture-avoiding substitution of $\vec{y}$ for $\vec{x}$ in $Q$.
\end{definition}

\begin{definition}
  The {\em structural congruence} \cite{SangiorgiWalker} , $\equiv$,
  between processes is the least congruence containing
  alpha-equivalence, satisfying the abelian monoid laws
  (associativity, commutativity and $\pzero$ as identity) for parallel
  composition $|$ and for summation $+$.
\end{definition}

\subsection{Name equivalence}

We take name equivalence, written $\nameeq$, to be the smallest
equivalence relation generated by the following rules.

\begin{mathpar}
\inferrule*[lab=Quote-drop]
{ }
{ \quotep{@{x}} \nameeq x }

\inferrule*[lab=Struct-equiv]
{ P \scong Q }
{ \quotep{P} \nameeq \quotep{Q} }
\end{mathpar}

The astute reader will have noticed that the mutual recursion of names
and processes imposes a mutual recursion on alpha-equivalence and
structural equivalence via name-equivalence. Fortunately, all of this
works out pleasantly and we may calculate in the natural way, free of
concern. The reader interested in the details is referred to the
appendix \ref{appendix:rho_details}.

\subsection{Substitution}

We use $\Proc$ for the set of processes, $\QProc$ for the set of
names, and $\id{\{}\vec{y} / \vec{x} \id{\}}$ to denote partial maps,
$s : \QProc \rightarrow \QProc$. A map, $s$ lifts, uniquely, to a map
on process terms, $\widehat{s} : \Proc \rightarrow \Proc$ by the
following equations.

\begin{mathpar}
  (0) \psubstp{Q}{P} := 0 \\
  (R \juxtap S) \psubstp{Q}{P}
  :=    
  (R)\psubstp{Q}{P} \juxtap (S) \psubstp{Q}{P} \\
  (x?(y).R) \psubstp{Q}{P}    
  :=    
  (x)\substp{Q}{P} (z)\concat( (R \psubstn{z}{y}) \psubstp{Q}{P} ) \\
  (\lift{x}{R}) \psubstp{Q}{P}  
  :=
  \lift{(x)\substp{Q}{P}}{ R \psubstp{Q}{P} } \\
%   (\dropn{x})  \psubstp{Q}{P}       
%   := 
%   \left\{ 
%     \begin{array}{ccc} 
%       \dropn{\quotep{Q}} & & x \nameeq \quotep{P} \\
%       \dropn{x} & & otherwise \\
%     \end{array}
%   \right. 
  (\dropn{x})  \psubstp{Q}{P}       
  := 
  \left\{ 
    \begin{array}{ccc} 
      Q & & x \nameeq \quotep{P} \\
      \dropn{x} & & otherwise \\
    \end{array}
  \right.
\end{mathpar}
 

where

\begin{eqnarray}
  (x)\id{\{} \lpquote Q \rpquote / \lpquote P \rpquote \id{\}}            = 
  \left\{ 
    \begin{array}{ccc}
      \lpquote Q \rpquote & & x \nameeq \lpquote P \rpquote \\
      x & & otherwise \\
    \end{array}
  \right. \nonumber
\end{eqnarray}

and $z$ is chosen distinct from $\quotep{P}$, $\quotep{Q}$, the free
names in $Q$, and all the names in $R$. Our $\alpha$-equivalence will
be built in the standard way from this substitution.

\begin{remark}\label{rem:no_self_referential_names}
  One consequence of these definitions is that $\forall P. \quotep{P}
  \not\in \freenames{P}$.
\end{remark}

\subsection{ Dynamic quote: an example }

Anticipating something of what's to come, consider applying the
substitution, $\widehat{\id{\{}u / z \id{\}}}$, to the following pair
of processes, $\lift{w}{y!(z)}$ and $w[ \lpquote y!(z) \rpquote ]$.

\begin{eqnarray}
	\lift{w}{y!(z)}\widehat{\id{\{}u / z \id{\}}}
		& = &
		\lift{w}{y!(u)} \nonumber\\
	w[ \lpquote y!(z) \rpquote ] \widehat{ \id{\{}u / z \id{\}} }
		& = &
		w[ \lpquote y!(z) \rpquote ] \nonumber
\end{eqnarray}

Because the body of the process between quotes is impervious to
substitution, we get radically different answers. In fact, by
examining the first process in an input context,
e.g. $x?(z).\lift{w}{y!(z)}$, we see that the process under the lift
operator may be shaped by prefixed inputs binding a name inside it. In
this sense, the lift operator will be seen as a way to dynamically
construct processes before reifying them as names.

Finally equipped with these standard features we can present the
dynamics of the calculus.

\subsubsection{Operational semantics} 

Finally, we introduce the computational dynamics. What marks these
algebras as distinct from other more traditionally studied algebraic
structures, e.g. vector spaces or polynomial rings, is the manner in
which dynamics is captured. In traditional structures, dynamics is typically
expressed through morphisms between such structures, as in linear maps
between vector spaces or morphisms between rings. In algebras
associated with the semantics of computation, the dynamics is
expressed as part of the algebraic structure itself, through a
reduction reduction relation typically denoted by $\red$. Below, we
give a recursive presentation of this relation for the calculus used
in the encoding.

$\red \subseteq \pi \times \pi$
$\red : \pi \to \mathcal{P}(\pi)$

\begin{mathpar}
  \inferrule* [lab=Comm] { \textsf{match}( x_{src}, x_{trgt} ) } { x_{trgt}?(y)P \; | \; x_{src}!\langle {Q} \rangle \red P\{\quotep{Q}/y}\} }
  \and \\
  \inferrule* [lab=Par] {{P} \red {P}'} {{{P} | {Q}} \red {{P}' | {Q}}}
  \and
  \inferrule* [lab=Equiv]{{{P} \scong {P}'} \andalso {{P}' \red {Q}'} \andalso {{Q}' \scong {Q}}}{{P} \red {Q}}
\end{mathpar}

\begin{eqnarray*}
  match_{\equiv} (\quotep{P},\quotep{Q}) & := & P \equiv Q \\
  match_{\dagger}(\quotep{P},\quotep{Q}) & := & \forall R. P|Q \red^{*} R => R \red^{*} 0 \\
  match_{K}(\quotep{P},\quotep{Q}) & := & K \mbox{ for some context } K
\end{eqnarray*}

$u?(x)P | u!\langle Q \rangle \red P\{\quotep{Q}/x\}$

%We write $\wred$ for $\red^*$, and $P\red$ if $\exists Q $ such that $ P \red Q$.
We write $P\red$ if $\exists Q $ such that $ P \red Q$ and $P\not\red$, otherwise.

\section{Replication}

As mentioned before, it is known that replication (and hence
recursion) can be implemented in a higher-order process algebra
\cite{SangiorgiWalker}. As our first example of calculation with the
machinery thus far presented we give the construction explicitly in
the {\rhoc}.

\begin{eqnarray}
	D_{x} & := & \prefix{x}{y}{(\binpar{\outputp{x}{y}}{@{y}})} \nonumber\\
	\bangp_{x}{P} & := & \binpar{{x}!\langle{\binpar{D_{x}}{P}}\rangle}{D_{x}} \nonumber
\end{eqnarray}

\begin{eqnarray}
	\bangp_{x}{P} & & \nonumber\\
	=
	& {x}!\langle{(\prefix{x}{y}{(\outputp{x}{y} | @{y})) | P}}\rangle 
	      | \prefix{x}{y}{(\outputp{x}{y} | @{y})} & \nonumber\\
	\red
	& (\outputp{x}{y} | @{y})\substn{\quotep{(\prefix{x}{y}{(@{y} | \outputp{x}{y})) | P}}}{y} & \nonumber\\
	=
	& \outputp{x}{\quotep{(\prefix{x}{y}{(\outputp{x}{y} | @{y})) | P}}}
	  | {(\prefix{x}{y}{(\outputp{x}{y} | @{y})) | P}} & \nonumber\\
	\red
	& \ldots & \nonumber\\
	\red^*
	& P | P | \ldots & \nonumber
\end{eqnarray}

Of course, this encoding, as an implementation, runs away, unfolding
$\bangp{P}$ eagerly. A lazier and more implementable replication
operator, restricted to input-guarded processes, may be obtained as follows.

\begin{eqnarray}
\bangp{\prefix{u}{v}{P}} 
	:= 
	\binpar{\lift{x}{\prefix{u}{v}{(\binpar{D(x)}{P})}}}{D(x)} \nonumber
\end{eqnarray}

\begin{remark}
  Note that the lazier definition still does not deal with summation
  or mixed summation (i.e. sums over input and output). The reader is
  invited to construct definitions of replication that deal with these
  features. 

  Further, the definitions are parameterized in a name, $x$. Can you,
  gentle reader, make a definition that eliminates this parameter and
  guarantees no accidental interaction between the replication
  machinery and the process being replicated -- i.e. no accidental
  sharing of names used by the process to get its work done and the
  name(s) used by the replication to effect copying. This latter
  revision of the definition of replication is crucial to obtaining
  the expected identity $!!P \sim !P$.
\end{remark}

\begin{remark}\label{rem:paradoxical_combinator}
  The reader familiar with the lambda calculus will have noticed the
  similarity between $D$ and the paradoxical combinator.

  [Ed. note: the existence of this seems to suggest we have to be more
  restrictive on the set of processes and names we admit if we are to
  support no-cloning.]
\end{remark}

\subsubsection{Bisimulation}

The computational dynamics gives rise to another kind of equivalence,
the equivalence of computational behavior. As previously mentioned
this is typically captured \emph{via} some form of bisimulation.

% The notion we use in this paper is weak barbed bisimulation
% \cite{milner91polyadicpi}.

The notion we use in this paper is derived from weak barbed
bisimulation \cite{milner91polyadicpi}. 

\begin{definition}
An \emph{observation relation}, $\downarrow_{\mathcal N}$, over a set
of names, $\mathcal N$, is the smallest relation satisfying the rules
below.

\infrule[Out-barb]{y \in {\mathcal N}, \; x \nameeq y}
		  {\outputp{x}{v} \downarrow_{\mathcal N} x}
\infrule[Par-barb]{\mbox{$P\downarrow_{\mathcal N} x$ or $Q\downarrow_{\mathcal N} x$}}
		  {\binpar{P}{Q} \downarrow_{\mathcal N} x}

We write $P \Downarrow_{\mathcal N} x$ if there is $Q$ such that 
$P \wred Q$ and $Q \downarrow_{\mathcal N} x$.
\end{definition}

\begin{definition}
%\label{def.bbisim}
An  ${\mathcal N}$-\emph{barbed bisimulation} over a set of names, ${\mathcal N}$, is a symmetric binary relation 
${\mathcal S}_{\mathcal N}$ between agents such that $P\rel{S}_{\mathcal N}Q$ implies:
\begin{enumerate}
\item If $P \red P'$ then $Q \wred Q'$ and $P'\rel{S}_{\mathcal N} Q'$.
\item If $P\downarrow_{\mathcal N} x$, then $Q\Downarrow_{\mathcal N} x$.
\end{enumerate}
$P$ is ${\mathcal N}$-barbed bisimilar to $Q$, written
$P \wbbisim_{\mathcal N} Q$, if $P \rel{S}_{\mathcal N} Q$ for some ${\mathcal N}$-barbed bisimulation ${\mathcal S}_{\mathcal N}$.
\end{definition}

$\mathcal{R} \subseteq \pi \times \pi$

$P \mathcal{R} Q => \forall P'. P \red P' \Rightarrow \exists Q'. Q \red Q', P' \mathcal{R} Q'$

$P \vdash x \Rightarrow Q \vdash x$

\begin{mathpar}
  \inferrule*[lab=Out-barb]{x \nameeq y}{{y}!\langle{Q}\rangle \vdash x}
  \and
  \inferrule*[lab=Par-barb]{\mbox{$P\vdash x$ or $Q\vdash x$}}{\binpar{P}{Q} \vdash x}
\end{mathpar}

\subsubsection{Contexts}

One of the principle advantages of computational calculi like the
$\pi$-calculus is a well-defined notion of context,
contextual-equivalence and a correlation between
contextual-equivalence and notions of bisimulation. The notion of
context allows the decomposition of a process into (sub-)process and
its syntactic environment, its context. Thus, a context may be
thought of as a process with a ``hole'' (written $\Box$) in it. The
application of a context $M$ to a process $P$, written $M[P]$, is
tantamount to filling the hole in $M$ with $P$. In this paper we do
not need the full weight of this theory, but do make use of the notion
of context in the proof the main theorem. 

\begin{mathpar}
  \inferrule* [lab=summation] {} {{M_{M},M_{N}} \bc \Box \;|\; x.M_{A} \;|\; M_{M}+M_{N}}
  \and
  \inferrule* [lab=agent] {} {{M_{A}} \bc (\vec{x})M_{P} \;| \; \clift{P_0,\ldots,M_{P},\ldots,P_N}}
  \and \\
  \inferrule* [lab=process] {} {{M_{P}} \bc M_{N} \;| \;P|M_{P} }
\end{mathpar} 

\begin{mathpar}
  \inferrule* [lab=sychronization] {} {M_{N} \bc \Box \;|\; x?M_{F} \;|\; x!M_{C}}
  \and
  \inferrule* [lab=abstraction] {} {{M_{F}} \bc (x)M_{P} }
  \and
  \inferrule* [lab=concretion] {} {{M_{C}} \bc \langle M_{P} \rangle }
  \and \\
  \inferrule* [lab=process] {} {{M_{P}} \bc M_{N} \;| \;P|M_{P} }
\end{mathpar}

\begin{definition}[contextual application] Given a context $M$, and
  process $P$, we define the \emph{contextual application}, $M[P] :=
  M\{P/\Box\}$. That is, the contextual application of M to P is the
  substitution of $P$ for $\Box$ in $M$.
\end{definition}

$\meaningof{-} : L \to \mathcal{P}(\pi)$

\begin{mathpar}
  \inferrule* [lab=collection] {} {\meaningof{true} = \pi, \and \meaningof{~E} = \pi \setminus \meaningof{E}, \and \meaningof{E_{1} \& E_{2}} = \meaningof{E_{1}} \cap \meaningof{E_{2}}}
\end{mathpar}

\begin{mathpar}
  \inferrule* [lab=structure] {} {\meaningof{0} = \{ P \in \pi | P \equiv 0 \}, \and \\ \meaningof{E_1 | E_2} = \{ P \in \pi | P \equiv P_{1} | P_{2}, P_{1} \in \meaningof{E_{1}}, P_{2} \in \meaningof{E_2}\} }
\end{mathpar}

\begin{mathpar}
 \inferrule* [lab=behavior] {} {\meaningof{\langle a?b \rangle E} = \{ P \in \pi | P \equiv Q | u?(y)P', \\ \and \\\\ \and \\ \;\;\; u \in \meaningof{a}, \forall z.P'\{z/y\} \in \meaningof{E\{z/b\}}\}, \and \\ \meaningof{a!E} = \{ P \in \pi | P \equiv Q | x!\langle P' \rangle, x \in \meaningof{a} P' \in \meaningof{E}\} }
\end{mathpar}

\begin{mathpar}
 \inferrule* [lab=nominal] {} {\meaningof{\quotep{E}} = \{ \quotep{P} \in \quotep{\pi} | P \in \meaningof{E} \}, \and \meaningof{\quotep{P}} = \{ \quotep{Q} \in \quotep{\pi} | P \equiv Q \} \and \\ \meaningof{@\quotep{E}} = \{ P \in \pi | P \equiv @x, x \in \meaningof{E} \}}
\end{mathpar}

\begin{eqnarray*}
  \\
  \meaningof{-} : TS \to ST
\end{eqnarray*}

\begin{eqnarray*}
  \\
  L : TS \to ST
\end{eqnarray*}

\begin{eqnarray*}
  \\
  P \models E \iff P \in \meaningof{E}
\end{eqnarray*}

\begin{eqnarray*}
  P \approx_{L} Q \iff \forall E \in L. P \models E \iff Q \models E
\end{eqnarray*}

\begin{eqnarray*}
  P \approx_{K} Q
\end{eqnarray*}

\begin{eqnarray*}
  P \approx Q
\end{eqnarray*}

$\approx_{K} = \approx = \approx_{L}$

\subsubsection{Contextual duality}

Note that contexts extend the quotation operation to a family of
operations from processes to names. Given a context, $M$, we can
define a \emph{nominal context}, $\quotep{M}$ by $\quotep{M}[P] :=
\quotep{M[P]}$. To foreshadow what is to come we observe that these
operations enjoy a duality with processes very much like the duality
between vectors and maps from vectors to scalars.

Further, because the calculus is essentially higher-order, we have a
correspondence between contexts and processes. More specifically,
given a name $x$ and a context $M$ we can construct $M^{*}_{x}$ such
that 

\begin{mathpar}
  M^{*}_{x} | \lift{x}{P} \red M[P]
\end{mathpar}

namely,

\begin{mathpar}
  M^{*}_{x} := x?(u).M[\dropn{u}]
\end{mathpar}

The dependence of $M^{*}_{x}$ on a name makes it an abstraction, 

\begin{mathpar}
  M^{*} := (x)x?(u).M[\dropn{u}]
\end{mathpar}

\subsection{Additional notation}

It will sometimes be convenient to denote the process a name
quotes. We already have the notation $x = \quotep{P}$, but it will be
convenient to introduce an alternate notation, $\procn{x}$, when we
want to emphasize the connection to the use of the name. Note that, by
virtue of name equivalence, $\quotep{\procn{x}} \nameeq x$; so, the
notation is consistent with previous definitions.

Further, because names have structure it is possible to effect
substitutions on the basis of that structure. This means we need to
upgrade our notation for substitutions, which we accomplish by
adapting comprehension notation. Thus,

\begin{mathpar}
  P\{ y / x : x \in S \}
\end{mathpar}

is interpreted to mean the process derived from P by replacing (in a
capture-avoiding manner) each occurrence of $x$ in $S$ by $y$. For example,

\begin{mathpar}
  P\{ \quotep{\procn{x}|\procn{x}} / x : x \in \freenames{P} \}
\end{mathpar}

will replace each (occurrence) of a free name $x$ in $P$ by
$\quotep{\procn{x}|\procn{x}}$.

Also, we will avail ourselves of the notation $x^{L}$ and $x^{R}$ to
denote injections of a name into disjoint copies of the name
space. There are numerous ways to accomplish this. One example can be
found in \cite{MeredithR05}. This notation overloads to vectors of
names: $\vec{x}^{\pi} := (x_{i}^{\pi} \; : \; 0 \leq i < |\vec{x}| )$ where $\pi \in \{L,R\}$.

We also use $P^{\Box} := P|\Box$.

In \cite{MeredithR05} an interpretation of the new operator is
given. It turns out that there are several possible interpretations
all enjoying the requisite algebraic properties of the operator (see
\cite{milner91polyadicpi}). We will therefore make liberal use of
$(\nu\; \vec{x})P$.

% subsection the_syntax_and_semantics_of_the_notation_system (end)   

\input{qm2pi.qmops} 

\input{qm2pi.sterngerlach} 

\input{qm2pi.metric} 

% section concurrent_process_calculi (end)

%\input{qm2pi.proofsketch}

% section proof sketch (end)

%\input{qm2pi.slviaknots} 

% section spatial logic via knots (end)

\input{qm2pi.conclusion}

% section conclusion (end)

%\input{qm2pi.dtcodes} 

% section wiring algorithm (end)

\input{qm2pi.ack} 

% section acknowledgments (end)

\newpage


\bibliographystyle{plain}   
\bibliography{../../biblios/main.bib}

\input{qm2pi.rhodetails}

\end{document}

 

% section notation (end)

\input{qm2pi.process.calculi} 

% section concurrent_process_calculi_and_spatial_logics_ (end)
    
%\documentclass[12pt]{llncs}
%\documentclass{jktr}

\usepackage[pdftex]{hyperref}                   
\usepackage {listings}
\usepackage {mathpartir}
\usepackage{bcprules}
%\usepackage{listings}
                       
\usepackage{graphicx} 
%\usepackage[margins=2.5cm,nohead,nofoot]{geometry}
%\usepackage{geometry}
\usepackage{amsfonts}
\usepackage{amstext}
\usepackage{latexsym}
\usepackage{amssymb}
\usepackage{color}


%\include{myPreamble}
\include{qm2pi.local} 

%\ifpdf
%\usepackage[pdftex]{graphicx}
%\else
%\usepackage{graphicx}
%\fi

 % \ifpdf
%  \usepackage{pdfsync}
%  \if


%\title{Brief Article}
%\author{David F. Snyder}
%\author{L.G. Meredith}

%\address{Dept. of Math., Texas State University--San Marcos, San Marcos, TX 78666}
       
\pagestyle{empty}


\begin{document}

\lstset{language=[Objective]Caml,frame=shadowbox}

\input{qm2pi.front}

% section front matter (end)

\input{qm2pi.intro} 
 
% section introduction (end)

% \input{qm2pi.knotations} 

% section notation (end)

\input{qm2pi.process.calculi} 

% section concurrent_process_calculi_and_spatial_logics_ (end)
    
%\input{qm2pi.knots2pi} 

%\input{qm2pi.trefoil} 

%\input{qm2pi.mainthm} 

% subsection basic_interpretation (end)

%\input{qm2pi.rho.presentation} 
\subsection{The syntax and semantics of the notation system}\label{sub:the_syntax_and_semantics_of_the_notation_system} % (fold)

We now summarize a technical presentation of the calculus that
embodies our theory of dynamics. The typical presentation of such a
calculus follows the style of giving generators and relations on
them. The grammar, below, describing term constructors, freely
generates the set of processes, $\Proc$. This set is then quotiented
by a relation known as structural congruence and it is over this set
that the notion of dynamics is expressed. This presentation is
essentially that of \cite{MeredithR05} with the addition of
polyadicity and summation. For readability we have relegated some of
the technical subtleties to an appendix.

\subsubsection{Process grammar}\label{subsub:process_grammar}

\begin{mathpar}
  \inferrule* [lab=synchronization] {} {{M} \bc \pzero \;|\; x?F \;|\; x!C }
  \and
  \inferrule* [lab=abstraction] {} {{F} \bc (x)P}
  \and
  \inferrule* [lab=concretion] {} {{C} \bc \langle Q \rangle}
  \and
  \inferrule* [lab=process] {} {{P,Q} \bc M \;| \;P|Q \;|\; @{x}}
  \and
  \inferrule* [lab=name] {} {{x} \bc \quotep{P}}
\end{mathpar} 

Note that $\vec{x}$ (resp. $\vec{P}$) denotes a vector of names
(resp. processes) of length $|\vec{x}|$ (resp. $|\vec{P}|$). We adopt
the following useful abbreviations.

\begin{mathpar}
   x?(\vec{y}).P := x.(\vec{y})P \and  x\clift{\vec{P}} := x.\clift{\vec{P}}
   \and x!(y) := \lift{x}{\dropn{y}}
   \and \Pi_{i=0}^{n-1}P_i := P_0 | \ldots | P_{n-1}
\end{mathpar}

\subsubsection{Structural congruence}

\paragraph{Free and bound names and alpha-equivalence.} At the
core of structural equivalence is alpha-equivalence which identifies
process that are the same up to a change of variable. Formally, we
recognize the distinction between free and bound names. The free names
of a process, $\freenames{P}$, may be calculated recursively as
follows:

\begin{mathpar}
\freenames{\pzero} := \emptyset
  \and \\
  \freenames{x?(y).P} := \{ x \} \cup (\freenames{P} \setminus \{ y \})
  \and 
  \freenames{x!\langle P \rangle} := \{ x \} \cup \{ P \} 
  \and \\
  \freenames{P|Q} := \freenames{P} \cup \freenames{Q}
  \and \\
  \freenames{@{x}} := \{ x \}
\end{mathpar}

$\pi$
$\quotep{\pi}$

$\freenames{-} : \pi \to \mathcal{P}(\quotep{\pi})$

\begin{eqnarray*}
  \freenames{\pzero} & := & \emptyset \\
  \freenames{x?(y).P} & := & \{ x \} \cup (\freenames{P} \setminus \{ y \}) \\
  \freenames{x!\langle P \rangle} & := & \{ x \} \cup \{ P \} \\
  \freenames{P|Q} & := & \freenames{P} \cup \freenames{Q} \\
  \freenames{\dropn{x}} & := & \{ x \}
\end{eqnarray*}

The bound names of a process, $\boundnames{P}$, are those names occurring in $P$
that are not free. For example, in $x?(y).0$, the name $x$ is free, while $y$ is bound.

\begin{mathpar}
  \inferrule* [lab=monoidal-laws] {} { P|Q \equiv Q|P \and P|0 \equiv P \and P|(Q|R) \equiv (P|Q)|R }
\end{mathpar}

\begin{mathpar}
  \inferrule* [lab=alpha-equivalence] {} { (x)P \equiv (y)P\{y/x\} \and y \not\in \freenames{P} }
\end{mathpar}

\begin{definition}
Then two processes, $P,Q$, are alpha-equivalent if $P = Q\{\vec{y}/\vec{x}\}$ for
some $\vec{x} \in \boundnames{Q},\vec{y} \in \boundnames{P}$, where $Q\{\vec{y}/\vec{x}\}$
denotes the capture-avoiding substitution of $\vec{y}$ for $\vec{x}$ in $Q$.
\end{definition}

\begin{definition}
  The {\em structural congruence} \cite{SangiorgiWalker} , $\equiv$,
  between processes is the least congruence containing
  alpha-equivalence, satisfying the abelian monoid laws
  (associativity, commutativity and $\pzero$ as identity) for parallel
  composition $|$ and for summation $+$.
\end{definition}

\subsection{Name equivalence}

We take name equivalence, written $\nameeq$, to be the smallest
equivalence relation generated by the following rules.

\begin{mathpar}
\inferrule*[lab=Quote-drop]
{ }
{ \quotep{@{x}} \nameeq x }

\inferrule*[lab=Struct-equiv]
{ P \scong Q }
{ \quotep{P} \nameeq \quotep{Q} }
\end{mathpar}

The astute reader will have noticed that the mutual recursion of names
and processes imposes a mutual recursion on alpha-equivalence and
structural equivalence via name-equivalence. Fortunately, all of this
works out pleasantly and we may calculate in the natural way, free of
concern. The reader interested in the details is referred to the
appendix \ref{appendix:rho_details}.

\subsection{Substitution}

We use $\Proc$ for the set of processes, $\QProc$ for the set of
names, and $\id{\{}\vec{y} / \vec{x} \id{\}}$ to denote partial maps,
$s : \QProc \rightarrow \QProc$. A map, $s$ lifts, uniquely, to a map
on process terms, $\widehat{s} : \Proc \rightarrow \Proc$ by the
following equations.

\begin{mathpar}
  (0) \psubstp{Q}{P} := 0 \\
  (R \juxtap S) \psubstp{Q}{P}
  :=    
  (R)\psubstp{Q}{P} \juxtap (S) \psubstp{Q}{P} \\
  (x?(y).R) \psubstp{Q}{P}    
  :=    
  (x)\substp{Q}{P} (z)\concat( (R \psubstn{z}{y}) \psubstp{Q}{P} ) \\
  (\lift{x}{R}) \psubstp{Q}{P}  
  :=
  \lift{(x)\substp{Q}{P}}{ R \psubstp{Q}{P} } \\
%   (\dropn{x})  \psubstp{Q}{P}       
%   := 
%   \left\{ 
%     \begin{array}{ccc} 
%       \dropn{\quotep{Q}} & & x \nameeq \quotep{P} \\
%       \dropn{x} & & otherwise \\
%     \end{array}
%   \right. 
  (\dropn{x})  \psubstp{Q}{P}       
  := 
  \left\{ 
    \begin{array}{ccc} 
      Q & & x \nameeq \quotep{P} \\
      \dropn{x} & & otherwise \\
    \end{array}
  \right.
\end{mathpar}
 

where

\begin{eqnarray}
  (x)\id{\{} \lpquote Q \rpquote / \lpquote P \rpquote \id{\}}            = 
  \left\{ 
    \begin{array}{ccc}
      \lpquote Q \rpquote & & x \nameeq \lpquote P \rpquote \\
      x & & otherwise \\
    \end{array}
  \right. \nonumber
\end{eqnarray}

and $z$ is chosen distinct from $\quotep{P}$, $\quotep{Q}$, the free
names in $Q$, and all the names in $R$. Our $\alpha$-equivalence will
be built in the standard way from this substitution.

\begin{remark}\label{rem:no_self_referential_names}
  One consequence of these definitions is that $\forall P. \quotep{P}
  \not\in \freenames{P}$.
\end{remark}

\subsection{ Dynamic quote: an example }

Anticipating something of what's to come, consider applying the
substitution, $\widehat{\id{\{}u / z \id{\}}}$, to the following pair
of processes, $\lift{w}{y!(z)}$ and $w[ \lpquote y!(z) \rpquote ]$.

\begin{eqnarray}
	\lift{w}{y!(z)}\widehat{\id{\{}u / z \id{\}}}
		& = &
		\lift{w}{y!(u)} \nonumber\\
	w[ \lpquote y!(z) \rpquote ] \widehat{ \id{\{}u / z \id{\}} }
		& = &
		w[ \lpquote y!(z) \rpquote ] \nonumber
\end{eqnarray}

Because the body of the process between quotes is impervious to
substitution, we get radically different answers. In fact, by
examining the first process in an input context,
e.g. $x?(z).\lift{w}{y!(z)}$, we see that the process under the lift
operator may be shaped by prefixed inputs binding a name inside it. In
this sense, the lift operator will be seen as a way to dynamically
construct processes before reifying them as names.

Finally equipped with these standard features we can present the
dynamics of the calculus.

\subsubsection{Operational semantics} 

Finally, we introduce the computational dynamics. What marks these
algebras as distinct from other more traditionally studied algebraic
structures, e.g. vector spaces or polynomial rings, is the manner in
which dynamics is captured. In traditional structures, dynamics is typically
expressed through morphisms between such structures, as in linear maps
between vector spaces or morphisms between rings. In algebras
associated with the semantics of computation, the dynamics is
expressed as part of the algebraic structure itself, through a
reduction reduction relation typically denoted by $\red$. Below, we
give a recursive presentation of this relation for the calculus used
in the encoding.

$\red \subseteq \pi \times \pi$
$\red : \pi \to \mathcal{P}(\pi)$

\begin{mathpar}
  \inferrule* [lab=Comm] { \textsf{match}( x_{src}, x_{trgt} ) } { x_{trgt}?(y)P \; | \; x_{src}!\langle {Q} \rangle \red P\{\quotep{Q}/y}\} }
  \and \\
  \inferrule* [lab=Par] {{P} \red {P}'} {{{P} | {Q}} \red {{P}' | {Q}}}
  \and
  \inferrule* [lab=Equiv]{{{P} \scong {P}'} \andalso {{P}' \red {Q}'} \andalso {{Q}' \scong {Q}}}{{P} \red {Q}}
\end{mathpar}

\begin{eqnarray*}
  match_{\equiv} (\quotep{P},\quotep{Q}) & := & P \equiv Q \\
  match_{\dagger}(\quotep{P},\quotep{Q}) & := & \forall R. P|Q \red^{*} R => R \red^{*} 0 \\
  match_{K}(\quotep{P},\quotep{Q}) & := & K \mbox{ for some context } K
\end{eqnarray*}

$u?(x)P | u!\langle Q \rangle \red P\{\quotep{Q}/x\}$

%We write $\wred$ for $\red^*$, and $P\red$ if $\exists Q $ such that $ P \red Q$.
We write $P\red$ if $\exists Q $ such that $ P \red Q$ and $P\not\red$, otherwise.

\section{Replication}

As mentioned before, it is known that replication (and hence
recursion) can be implemented in a higher-order process algebra
\cite{SangiorgiWalker}. As our first example of calculation with the
machinery thus far presented we give the construction explicitly in
the {\rhoc}.

\begin{eqnarray}
	D_{x} & := & \prefix{x}{y}{(\binpar{\outputp{x}{y}}{@{y}})} \nonumber\\
	\bangp_{x}{P} & := & \binpar{{x}!\langle{\binpar{D_{x}}{P}}\rangle}{D_{x}} \nonumber
\end{eqnarray}

\begin{eqnarray}
	\bangp_{x}{P} & & \nonumber\\
	=
	& {x}!\langle{(\prefix{x}{y}{(\outputp{x}{y} | @{y})) | P}}\rangle 
	      | \prefix{x}{y}{(\outputp{x}{y} | @{y})} & \nonumber\\
	\red
	& (\outputp{x}{y} | @{y})\substn{\quotep{(\prefix{x}{y}{(@{y} | \outputp{x}{y})) | P}}}{y} & \nonumber\\
	=
	& \outputp{x}{\quotep{(\prefix{x}{y}{(\outputp{x}{y} | @{y})) | P}}}
	  | {(\prefix{x}{y}{(\outputp{x}{y} | @{y})) | P}} & \nonumber\\
	\red
	& \ldots & \nonumber\\
	\red^*
	& P | P | \ldots & \nonumber
\end{eqnarray}

Of course, this encoding, as an implementation, runs away, unfolding
$\bangp{P}$ eagerly. A lazier and more implementable replication
operator, restricted to input-guarded processes, may be obtained as follows.

\begin{eqnarray}
\bangp{\prefix{u}{v}{P}} 
	:= 
	\binpar{\lift{x}{\prefix{u}{v}{(\binpar{D(x)}{P})}}}{D(x)} \nonumber
\end{eqnarray}

\begin{remark}
  Note that the lazier definition still does not deal with summation
  or mixed summation (i.e. sums over input and output). The reader is
  invited to construct definitions of replication that deal with these
  features. 

  Further, the definitions are parameterized in a name, $x$. Can you,
  gentle reader, make a definition that eliminates this parameter and
  guarantees no accidental interaction between the replication
  machinery and the process being replicated -- i.e. no accidental
  sharing of names used by the process to get its work done and the
  name(s) used by the replication to effect copying. This latter
  revision of the definition of replication is crucial to obtaining
  the expected identity $!!P \sim !P$.
\end{remark}

\begin{remark}\label{rem:paradoxical_combinator}
  The reader familiar with the lambda calculus will have noticed the
  similarity between $D$ and the paradoxical combinator.

  [Ed. note: the existence of this seems to suggest we have to be more
  restrictive on the set of processes and names we admit if we are to
  support no-cloning.]
\end{remark}

\subsubsection{Bisimulation}

The computational dynamics gives rise to another kind of equivalence,
the equivalence of computational behavior. As previously mentioned
this is typically captured \emph{via} some form of bisimulation.

% The notion we use in this paper is weak barbed bisimulation
% \cite{milner91polyadicpi}.

The notion we use in this paper is derived from weak barbed
bisimulation \cite{milner91polyadicpi}. 

\begin{definition}
An \emph{observation relation}, $\downarrow_{\mathcal N}$, over a set
of names, $\mathcal N$, is the smallest relation satisfying the rules
below.

\infrule[Out-barb]{y \in {\mathcal N}, \; x \nameeq y}
		  {\outputp{x}{v} \downarrow_{\mathcal N} x}
\infrule[Par-barb]{\mbox{$P\downarrow_{\mathcal N} x$ or $Q\downarrow_{\mathcal N} x$}}
		  {\binpar{P}{Q} \downarrow_{\mathcal N} x}

We write $P \Downarrow_{\mathcal N} x$ if there is $Q$ such that 
$P \wred Q$ and $Q \downarrow_{\mathcal N} x$.
\end{definition}

\begin{definition}
%\label{def.bbisim}
An  ${\mathcal N}$-\emph{barbed bisimulation} over a set of names, ${\mathcal N}$, is a symmetric binary relation 
${\mathcal S}_{\mathcal N}$ between agents such that $P\rel{S}_{\mathcal N}Q$ implies:
\begin{enumerate}
\item If $P \red P'$ then $Q \wred Q'$ and $P'\rel{S}_{\mathcal N} Q'$.
\item If $P\downarrow_{\mathcal N} x$, then $Q\Downarrow_{\mathcal N} x$.
\end{enumerate}
$P$ is ${\mathcal N}$-barbed bisimilar to $Q$, written
$P \wbbisim_{\mathcal N} Q$, if $P \rel{S}_{\mathcal N} Q$ for some ${\mathcal N}$-barbed bisimulation ${\mathcal S}_{\mathcal N}$.
\end{definition}

$\mathcal{R} \subseteq \pi \times \pi$

$P \mathcal{R} Q => \forall P'. P \red P' \Rightarrow \exists Q'. Q \red Q', P' \mathcal{R} Q'$

$P \vdash x \Rightarrow Q \vdash x$

\begin{mathpar}
  \inferrule*[lab=Out-barb]{x \nameeq y}{{y}!\langle{Q}\rangle \vdash x}
  \and
  \inferrule*[lab=Par-barb]{\mbox{$P\vdash x$ or $Q\vdash x$}}{\binpar{P}{Q} \vdash x}
\end{mathpar}

\subsubsection{Contexts}

One of the principle advantages of computational calculi like the
$\pi$-calculus is a well-defined notion of context,
contextual-equivalence and a correlation between
contextual-equivalence and notions of bisimulation. The notion of
context allows the decomposition of a process into (sub-)process and
its syntactic environment, its context. Thus, a context may be
thought of as a process with a ``hole'' (written $\Box$) in it. The
application of a context $M$ to a process $P$, written $M[P]$, is
tantamount to filling the hole in $M$ with $P$. In this paper we do
not need the full weight of this theory, but do make use of the notion
of context in the proof the main theorem. 

\begin{mathpar}
  \inferrule* [lab=summation] {} {{M_{M},M_{N}} \bc \Box \;|\; x.M_{A} \;|\; M_{M}+M_{N}}
  \and
  \inferrule* [lab=agent] {} {{M_{A}} \bc (\vec{x})M_{P} \;| \; \clift{P_0,\ldots,M_{P},\ldots,P_N}}
  \and \\
  \inferrule* [lab=process] {} {{M_{P}} \bc M_{N} \;| \;P|M_{P} }
\end{mathpar} 

\begin{mathpar}
  \inferrule* [lab=sychronization] {} {M_{N} \bc \Box \;|\; x?M_{F} \;|\; x!M_{C}}
  \and
  \inferrule* [lab=abstraction] {} {{M_{F}} \bc (x)M_{P} }
  \and
  \inferrule* [lab=concretion] {} {{M_{C}} \bc \langle M_{P} \rangle }
  \and \\
  \inferrule* [lab=process] {} {{M_{P}} \bc M_{N} \;| \;P|M_{P} }
\end{mathpar}

\begin{definition}[contextual application] Given a context $M$, and
  process $P$, we define the \emph{contextual application}, $M[P] :=
  M\{P/\Box\}$. That is, the contextual application of M to P is the
  substitution of $P$ for $\Box$ in $M$.
\end{definition}

$\meaningof{-} : L \to \mathcal{P}(\pi)$

\begin{mathpar}
  \inferrule* [lab=collection] {} {\meaningof{true} = \pi, \and \meaningof{~E} = \pi \setminus \meaningof{E}, \and \meaningof{E_{1} \& E_{2}} = \meaningof{E_{1}} \cap \meaningof{E_{2}}}
\end{mathpar}

\begin{mathpar}
  \inferrule* [lab=structure] {} {\meaningof{0} = \{ P \in \pi | P \equiv 0 \}, \and \\ \meaningof{E_1 | E_2} = \{ P \in \pi | P \equiv P_{1} | P_{2}, P_{1} \in \meaningof{E_{1}}, P_{2} \in \meaningof{E_2}\} }
\end{mathpar}

\begin{mathpar}
 \inferrule* [lab=behavior] {} {\meaningof{\langle a?b \rangle E} = \{ P \in \pi | P \equiv Q | u?(y)P', \\ \and \\\\ \and \\ \;\;\; u \in \meaningof{a}, \forall z.P'\{z/y\} \in \meaningof{E\{z/b\}}\}, \and \\ \meaningof{a!E} = \{ P \in \pi | P \equiv Q | x!\langle P' \rangle, x \in \meaningof{a} P' \in \meaningof{E}\} }
\end{mathpar}

\begin{mathpar}
 \inferrule* [lab=nominal] {} {\meaningof{\quotep{E}} = \{ \quotep{P} \in \quotep{\pi} | P \in \meaningof{E} \}, \and \meaningof{\quotep{P}} = \{ \quotep{Q} \in \quotep{\pi} | P \equiv Q \} \and \\ \meaningof{@\quotep{E}} = \{ P \in \pi | P \equiv @x, x \in \meaningof{E} \}}
\end{mathpar}

\begin{eqnarray*}
  \\
  \meaningof{-} : TS \to ST
\end{eqnarray*}

\begin{eqnarray*}
  \\
  L : TS \to ST
\end{eqnarray*}

\begin{eqnarray*}
  \\
  P \models E \iff P \in \meaningof{E}
\end{eqnarray*}

\begin{eqnarray*}
  P \approx_{L} Q \iff \forall E \in L. P \models E \iff Q \models E
\end{eqnarray*}

\begin{eqnarray*}
  P \approx_{K} Q
\end{eqnarray*}

\begin{eqnarray*}
  P \approx Q
\end{eqnarray*}

$\approx_{K} = \approx = \approx_{L}$

\subsubsection{Contextual duality}

Note that contexts extend the quotation operation to a family of
operations from processes to names. Given a context, $M$, we can
define a \emph{nominal context}, $\quotep{M}$ by $\quotep{M}[P] :=
\quotep{M[P]}$. To foreshadow what is to come we observe that these
operations enjoy a duality with processes very much like the duality
between vectors and maps from vectors to scalars.

Further, because the calculus is essentially higher-order, we have a
correspondence between contexts and processes. More specifically,
given a name $x$ and a context $M$ we can construct $M^{*}_{x}$ such
that 

\begin{mathpar}
  M^{*}_{x} | \lift{x}{P} \red M[P]
\end{mathpar}

namely,

\begin{mathpar}
  M^{*}_{x} := x?(u).M[\dropn{u}]
\end{mathpar}

The dependence of $M^{*}_{x}$ on a name makes it an abstraction, 

\begin{mathpar}
  M^{*} := (x)x?(u).M[\dropn{u}]
\end{mathpar}

\subsection{Additional notation}

It will sometimes be convenient to denote the process a name
quotes. We already have the notation $x = \quotep{P}$, but it will be
convenient to introduce an alternate notation, $\procn{x}$, when we
want to emphasize the connection to the use of the name. Note that, by
virtue of name equivalence, $\quotep{\procn{x}} \nameeq x$; so, the
notation is consistent with previous definitions.

Further, because names have structure it is possible to effect
substitutions on the basis of that structure. This means we need to
upgrade our notation for substitutions, which we accomplish by
adapting comprehension notation. Thus,

\begin{mathpar}
  P\{ y / x : x \in S \}
\end{mathpar}

is interpreted to mean the process derived from P by replacing (in a
capture-avoiding manner) each occurrence of $x$ in $S$ by $y$. For example,

\begin{mathpar}
  P\{ \quotep{\procn{x}|\procn{x}} / x : x \in \freenames{P} \}
\end{mathpar}

will replace each (occurrence) of a free name $x$ in $P$ by
$\quotep{\procn{x}|\procn{x}}$.

Also, we will avail ourselves of the notation $x^{L}$ and $x^{R}$ to
denote injections of a name into disjoint copies of the name
space. There are numerous ways to accomplish this. One example can be
found in \cite{MeredithR05}. This notation overloads to vectors of
names: $\vec{x}^{\pi} := (x_{i}^{\pi} \; : \; 0 \leq i < |\vec{x}| )$ where $\pi \in \{L,R\}$.

We also use $P^{\Box} := P|\Box$.

In \cite{MeredithR05} an interpretation of the new operator is
given. It turns out that there are several possible interpretations
all enjoying the requisite algebraic properties of the operator (see
\cite{milner91polyadicpi}). We will therefore make liberal use of
$(\nu\; \vec{x})P$.

% subsection the_syntax_and_semantics_of_the_notation_system (end)   

\input{qm2pi.qmops} 

\input{qm2pi.sterngerlach} 

\input{qm2pi.metric} 

% section concurrent_process_calculi (end)

%\input{qm2pi.proofsketch}

% section proof sketch (end)

%\input{qm2pi.slviaknots} 

% section spatial logic via knots (end)

\input{qm2pi.conclusion}

% section conclusion (end)

%\input{qm2pi.dtcodes} 

% section wiring algorithm (end)

\input{qm2pi.ack} 

% section acknowledgments (end)

\newpage


\bibliographystyle{plain}   
\bibliography{../../biblios/main.bib}

\input{qm2pi.rhodetails}

\end{document}

 

%\documentclass[12pt]{llncs}
%\documentclass{jktr}

\usepackage[pdftex]{hyperref}                   
\usepackage {listings}
\usepackage {mathpartir}
\usepackage{bcprules}
%\usepackage{listings}
                       
\usepackage{graphicx} 
%\usepackage[margins=2.5cm,nohead,nofoot]{geometry}
%\usepackage{geometry}
\usepackage{amsfonts}
\usepackage{amstext}
\usepackage{latexsym}
\usepackage{amssymb}
\usepackage{color}


%\include{myPreamble}
\include{qm2pi.local} 

%\ifpdf
%\usepackage[pdftex]{graphicx}
%\else
%\usepackage{graphicx}
%\fi

 % \ifpdf
%  \usepackage{pdfsync}
%  \if


%\title{Brief Article}
%\author{David F. Snyder}
%\author{L.G. Meredith}

%\address{Dept. of Math., Texas State University--San Marcos, San Marcos, TX 78666}
       
\pagestyle{empty}


\begin{document}

\lstset{language=[Objective]Caml,frame=shadowbox}

\input{qm2pi.front}

% section front matter (end)

\input{qm2pi.intro} 
 
% section introduction (end)

% \input{qm2pi.knotations} 

% section notation (end)

\input{qm2pi.process.calculi} 

% section concurrent_process_calculi_and_spatial_logics_ (end)
    
%\input{qm2pi.knots2pi} 

%\input{qm2pi.trefoil} 

%\input{qm2pi.mainthm} 

% subsection basic_interpretation (end)

%\input{qm2pi.rho.presentation} 
\subsection{The syntax and semantics of the notation system}\label{sub:the_syntax_and_semantics_of_the_notation_system} % (fold)

We now summarize a technical presentation of the calculus that
embodies our theory of dynamics. The typical presentation of such a
calculus follows the style of giving generators and relations on
them. The grammar, below, describing term constructors, freely
generates the set of processes, $\Proc$. This set is then quotiented
by a relation known as structural congruence and it is over this set
that the notion of dynamics is expressed. This presentation is
essentially that of \cite{MeredithR05} with the addition of
polyadicity and summation. For readability we have relegated some of
the technical subtleties to an appendix.

\subsubsection{Process grammar}\label{subsub:process_grammar}

\begin{mathpar}
  \inferrule* [lab=synchronization] {} {{M} \bc \pzero \;|\; x?F \;|\; x!C }
  \and
  \inferrule* [lab=abstraction] {} {{F} \bc (x)P}
  \and
  \inferrule* [lab=concretion] {} {{C} \bc \langle Q \rangle}
  \and
  \inferrule* [lab=process] {} {{P,Q} \bc M \;| \;P|Q \;|\; @{x}}
  \and
  \inferrule* [lab=name] {} {{x} \bc \quotep{P}}
\end{mathpar} 

Note that $\vec{x}$ (resp. $\vec{P}$) denotes a vector of names
(resp. processes) of length $|\vec{x}|$ (resp. $|\vec{P}|$). We adopt
the following useful abbreviations.

\begin{mathpar}
   x?(\vec{y}).P := x.(\vec{y})P \and  x\clift{\vec{P}} := x.\clift{\vec{P}}
   \and x!(y) := \lift{x}{\dropn{y}}
   \and \Pi_{i=0}^{n-1}P_i := P_0 | \ldots | P_{n-1}
\end{mathpar}

\subsubsection{Structural congruence}

\paragraph{Free and bound names and alpha-equivalence.} At the
core of structural equivalence is alpha-equivalence which identifies
process that are the same up to a change of variable. Formally, we
recognize the distinction between free and bound names. The free names
of a process, $\freenames{P}$, may be calculated recursively as
follows:

\begin{mathpar}
\freenames{\pzero} := \emptyset
  \and \\
  \freenames{x?(y).P} := \{ x \} \cup (\freenames{P} \setminus \{ y \})
  \and 
  \freenames{x!\langle P \rangle} := \{ x \} \cup \{ P \} 
  \and \\
  \freenames{P|Q} := \freenames{P} \cup \freenames{Q}
  \and \\
  \freenames{@{x}} := \{ x \}
\end{mathpar}

$\pi$
$\quotep{\pi}$

$\freenames{-} : \pi \to \mathcal{P}(\quotep{\pi})$

\begin{eqnarray*}
  \freenames{\pzero} & := & \emptyset \\
  \freenames{x?(y).P} & := & \{ x \} \cup (\freenames{P} \setminus \{ y \}) \\
  \freenames{x!\langle P \rangle} & := & \{ x \} \cup \{ P \} \\
  \freenames{P|Q} & := & \freenames{P} \cup \freenames{Q} \\
  \freenames{\dropn{x}} & := & \{ x \}
\end{eqnarray*}

The bound names of a process, $\boundnames{P}$, are those names occurring in $P$
that are not free. For example, in $x?(y).0$, the name $x$ is free, while $y$ is bound.

\begin{mathpar}
  \inferrule* [lab=monoidal-laws] {} { P|Q \equiv Q|P \and P|0 \equiv P \and P|(Q|R) \equiv (P|Q)|R }
\end{mathpar}

\begin{mathpar}
  \inferrule* [lab=alpha-equivalence] {} { (x)P \equiv (y)P\{y/x\} \and y \not\in \freenames{P} }
\end{mathpar}

\begin{definition}
Then two processes, $P,Q$, are alpha-equivalent if $P = Q\{\vec{y}/\vec{x}\}$ for
some $\vec{x} \in \boundnames{Q},\vec{y} \in \boundnames{P}$, where $Q\{\vec{y}/\vec{x}\}$
denotes the capture-avoiding substitution of $\vec{y}$ for $\vec{x}$ in $Q$.
\end{definition}

\begin{definition}
  The {\em structural congruence} \cite{SangiorgiWalker} , $\equiv$,
  between processes is the least congruence containing
  alpha-equivalence, satisfying the abelian monoid laws
  (associativity, commutativity and $\pzero$ as identity) for parallel
  composition $|$ and for summation $+$.
\end{definition}

\subsection{Name equivalence}

We take name equivalence, written $\nameeq$, to be the smallest
equivalence relation generated by the following rules.

\begin{mathpar}
\inferrule*[lab=Quote-drop]
{ }
{ \quotep{@{x}} \nameeq x }

\inferrule*[lab=Struct-equiv]
{ P \scong Q }
{ \quotep{P} \nameeq \quotep{Q} }
\end{mathpar}

The astute reader will have noticed that the mutual recursion of names
and processes imposes a mutual recursion on alpha-equivalence and
structural equivalence via name-equivalence. Fortunately, all of this
works out pleasantly and we may calculate in the natural way, free of
concern. The reader interested in the details is referred to the
appendix \ref{appendix:rho_details}.

\subsection{Substitution}

We use $\Proc$ for the set of processes, $\QProc$ for the set of
names, and $\id{\{}\vec{y} / \vec{x} \id{\}}$ to denote partial maps,
$s : \QProc \rightarrow \QProc$. A map, $s$ lifts, uniquely, to a map
on process terms, $\widehat{s} : \Proc \rightarrow \Proc$ by the
following equations.

\begin{mathpar}
  (0) \psubstp{Q}{P} := 0 \\
  (R \juxtap S) \psubstp{Q}{P}
  :=    
  (R)\psubstp{Q}{P} \juxtap (S) \psubstp{Q}{P} \\
  (x?(y).R) \psubstp{Q}{P}    
  :=    
  (x)\substp{Q}{P} (z)\concat( (R \psubstn{z}{y}) \psubstp{Q}{P} ) \\
  (\lift{x}{R}) \psubstp{Q}{P}  
  :=
  \lift{(x)\substp{Q}{P}}{ R \psubstp{Q}{P} } \\
%   (\dropn{x})  \psubstp{Q}{P}       
%   := 
%   \left\{ 
%     \begin{array}{ccc} 
%       \dropn{\quotep{Q}} & & x \nameeq \quotep{P} \\
%       \dropn{x} & & otherwise \\
%     \end{array}
%   \right. 
  (\dropn{x})  \psubstp{Q}{P}       
  := 
  \left\{ 
    \begin{array}{ccc} 
      Q & & x \nameeq \quotep{P} \\
      \dropn{x} & & otherwise \\
    \end{array}
  \right.
\end{mathpar}
 

where

\begin{eqnarray}
  (x)\id{\{} \lpquote Q \rpquote / \lpquote P \rpquote \id{\}}            = 
  \left\{ 
    \begin{array}{ccc}
      \lpquote Q \rpquote & & x \nameeq \lpquote P \rpquote \\
      x & & otherwise \\
    \end{array}
  \right. \nonumber
\end{eqnarray}

and $z$ is chosen distinct from $\quotep{P}$, $\quotep{Q}$, the free
names in $Q$, and all the names in $R$. Our $\alpha$-equivalence will
be built in the standard way from this substitution.

\begin{remark}\label{rem:no_self_referential_names}
  One consequence of these definitions is that $\forall P. \quotep{P}
  \not\in \freenames{P}$.
\end{remark}

\subsection{ Dynamic quote: an example }

Anticipating something of what's to come, consider applying the
substitution, $\widehat{\id{\{}u / z \id{\}}}$, to the following pair
of processes, $\lift{w}{y!(z)}$ and $w[ \lpquote y!(z) \rpquote ]$.

\begin{eqnarray}
	\lift{w}{y!(z)}\widehat{\id{\{}u / z \id{\}}}
		& = &
		\lift{w}{y!(u)} \nonumber\\
	w[ \lpquote y!(z) \rpquote ] \widehat{ \id{\{}u / z \id{\}} }
		& = &
		w[ \lpquote y!(z) \rpquote ] \nonumber
\end{eqnarray}

Because the body of the process between quotes is impervious to
substitution, we get radically different answers. In fact, by
examining the first process in an input context,
e.g. $x?(z).\lift{w}{y!(z)}$, we see that the process under the lift
operator may be shaped by prefixed inputs binding a name inside it. In
this sense, the lift operator will be seen as a way to dynamically
construct processes before reifying them as names.

Finally equipped with these standard features we can present the
dynamics of the calculus.

\subsubsection{Operational semantics} 

Finally, we introduce the computational dynamics. What marks these
algebras as distinct from other more traditionally studied algebraic
structures, e.g. vector spaces or polynomial rings, is the manner in
which dynamics is captured. In traditional structures, dynamics is typically
expressed through morphisms between such structures, as in linear maps
between vector spaces or morphisms between rings. In algebras
associated with the semantics of computation, the dynamics is
expressed as part of the algebraic structure itself, through a
reduction reduction relation typically denoted by $\red$. Below, we
give a recursive presentation of this relation for the calculus used
in the encoding.

$\red \subseteq \pi \times \pi$
$\red : \pi \to \mathcal{P}(\pi)$

\begin{mathpar}
  \inferrule* [lab=Comm] { \textsf{match}( x_{src}, x_{trgt} ) } { x_{trgt}?(y)P \; | \; x_{src}!\langle {Q} \rangle \red P\{\quotep{Q}/y}\} }
  \and \\
  \inferrule* [lab=Par] {{P} \red {P}'} {{{P} | {Q}} \red {{P}' | {Q}}}
  \and
  \inferrule* [lab=Equiv]{{{P} \scong {P}'} \andalso {{P}' \red {Q}'} \andalso {{Q}' \scong {Q}}}{{P} \red {Q}}
\end{mathpar}

\begin{eqnarray*}
  match_{\equiv} (\quotep{P},\quotep{Q}) & := & P \equiv Q \\
  match_{\dagger}(\quotep{P},\quotep{Q}) & := & \forall R. P|Q \red^{*} R => R \red^{*} 0 \\
  match_{K}(\quotep{P},\quotep{Q}) & := & K \mbox{ for some context } K
\end{eqnarray*}

$u?(x)P | u!\langle Q \rangle \red P\{\quotep{Q}/x\}$

%We write $\wred$ for $\red^*$, and $P\red$ if $\exists Q $ such that $ P \red Q$.
We write $P\red$ if $\exists Q $ such that $ P \red Q$ and $P\not\red$, otherwise.

\section{Replication}

As mentioned before, it is known that replication (and hence
recursion) can be implemented in a higher-order process algebra
\cite{SangiorgiWalker}. As our first example of calculation with the
machinery thus far presented we give the construction explicitly in
the {\rhoc}.

\begin{eqnarray}
	D_{x} & := & \prefix{x}{y}{(\binpar{\outputp{x}{y}}{@{y}})} \nonumber\\
	\bangp_{x}{P} & := & \binpar{{x}!\langle{\binpar{D_{x}}{P}}\rangle}{D_{x}} \nonumber
\end{eqnarray}

\begin{eqnarray}
	\bangp_{x}{P} & & \nonumber\\
	=
	& {x}!\langle{(\prefix{x}{y}{(\outputp{x}{y} | @{y})) | P}}\rangle 
	      | \prefix{x}{y}{(\outputp{x}{y} | @{y})} & \nonumber\\
	\red
	& (\outputp{x}{y} | @{y})\substn{\quotep{(\prefix{x}{y}{(@{y} | \outputp{x}{y})) | P}}}{y} & \nonumber\\
	=
	& \outputp{x}{\quotep{(\prefix{x}{y}{(\outputp{x}{y} | @{y})) | P}}}
	  | {(\prefix{x}{y}{(\outputp{x}{y} | @{y})) | P}} & \nonumber\\
	\red
	& \ldots & \nonumber\\
	\red^*
	& P | P | \ldots & \nonumber
\end{eqnarray}

Of course, this encoding, as an implementation, runs away, unfolding
$\bangp{P}$ eagerly. A lazier and more implementable replication
operator, restricted to input-guarded processes, may be obtained as follows.

\begin{eqnarray}
\bangp{\prefix{u}{v}{P}} 
	:= 
	\binpar{\lift{x}{\prefix{u}{v}{(\binpar{D(x)}{P})}}}{D(x)} \nonumber
\end{eqnarray}

\begin{remark}
  Note that the lazier definition still does not deal with summation
  or mixed summation (i.e. sums over input and output). The reader is
  invited to construct definitions of replication that deal with these
  features. 

  Further, the definitions are parameterized in a name, $x$. Can you,
  gentle reader, make a definition that eliminates this parameter and
  guarantees no accidental interaction between the replication
  machinery and the process being replicated -- i.e. no accidental
  sharing of names used by the process to get its work done and the
  name(s) used by the replication to effect copying. This latter
  revision of the definition of replication is crucial to obtaining
  the expected identity $!!P \sim !P$.
\end{remark}

\begin{remark}\label{rem:paradoxical_combinator}
  The reader familiar with the lambda calculus will have noticed the
  similarity between $D$ and the paradoxical combinator.

  [Ed. note: the existence of this seems to suggest we have to be more
  restrictive on the set of processes and names we admit if we are to
  support no-cloning.]
\end{remark}

\subsubsection{Bisimulation}

The computational dynamics gives rise to another kind of equivalence,
the equivalence of computational behavior. As previously mentioned
this is typically captured \emph{via} some form of bisimulation.

% The notion we use in this paper is weak barbed bisimulation
% \cite{milner91polyadicpi}.

The notion we use in this paper is derived from weak barbed
bisimulation \cite{milner91polyadicpi}. 

\begin{definition}
An \emph{observation relation}, $\downarrow_{\mathcal N}$, over a set
of names, $\mathcal N$, is the smallest relation satisfying the rules
below.

\infrule[Out-barb]{y \in {\mathcal N}, \; x \nameeq y}
		  {\outputp{x}{v} \downarrow_{\mathcal N} x}
\infrule[Par-barb]{\mbox{$P\downarrow_{\mathcal N} x$ or $Q\downarrow_{\mathcal N} x$}}
		  {\binpar{P}{Q} \downarrow_{\mathcal N} x}

We write $P \Downarrow_{\mathcal N} x$ if there is $Q$ such that 
$P \wred Q$ and $Q \downarrow_{\mathcal N} x$.
\end{definition}

\begin{definition}
%\label{def.bbisim}
An  ${\mathcal N}$-\emph{barbed bisimulation} over a set of names, ${\mathcal N}$, is a symmetric binary relation 
${\mathcal S}_{\mathcal N}$ between agents such that $P\rel{S}_{\mathcal N}Q$ implies:
\begin{enumerate}
\item If $P \red P'$ then $Q \wred Q'$ and $P'\rel{S}_{\mathcal N} Q'$.
\item If $P\downarrow_{\mathcal N} x$, then $Q\Downarrow_{\mathcal N} x$.
\end{enumerate}
$P$ is ${\mathcal N}$-barbed bisimilar to $Q$, written
$P \wbbisim_{\mathcal N} Q$, if $P \rel{S}_{\mathcal N} Q$ for some ${\mathcal N}$-barbed bisimulation ${\mathcal S}_{\mathcal N}$.
\end{definition}

$\mathcal{R} \subseteq \pi \times \pi$

$P \mathcal{R} Q => \forall P'. P \red P' \Rightarrow \exists Q'. Q \red Q', P' \mathcal{R} Q'$

$P \vdash x \Rightarrow Q \vdash x$

\begin{mathpar}
  \inferrule*[lab=Out-barb]{x \nameeq y}{{y}!\langle{Q}\rangle \vdash x}
  \and
  \inferrule*[lab=Par-barb]{\mbox{$P\vdash x$ or $Q\vdash x$}}{\binpar{P}{Q} \vdash x}
\end{mathpar}

\subsubsection{Contexts}

One of the principle advantages of computational calculi like the
$\pi$-calculus is a well-defined notion of context,
contextual-equivalence and a correlation between
contextual-equivalence and notions of bisimulation. The notion of
context allows the decomposition of a process into (sub-)process and
its syntactic environment, its context. Thus, a context may be
thought of as a process with a ``hole'' (written $\Box$) in it. The
application of a context $M$ to a process $P$, written $M[P]$, is
tantamount to filling the hole in $M$ with $P$. In this paper we do
not need the full weight of this theory, but do make use of the notion
of context in the proof the main theorem. 

\begin{mathpar}
  \inferrule* [lab=summation] {} {{M_{M},M_{N}} \bc \Box \;|\; x.M_{A} \;|\; M_{M}+M_{N}}
  \and
  \inferrule* [lab=agent] {} {{M_{A}} \bc (\vec{x})M_{P} \;| \; \clift{P_0,\ldots,M_{P},\ldots,P_N}}
  \and \\
  \inferrule* [lab=process] {} {{M_{P}} \bc M_{N} \;| \;P|M_{P} }
\end{mathpar} 

\begin{mathpar}
  \inferrule* [lab=sychronization] {} {M_{N} \bc \Box \;|\; x?M_{F} \;|\; x!M_{C}}
  \and
  \inferrule* [lab=abstraction] {} {{M_{F}} \bc (x)M_{P} }
  \and
  \inferrule* [lab=concretion] {} {{M_{C}} \bc \langle M_{P} \rangle }
  \and \\
  \inferrule* [lab=process] {} {{M_{P}} \bc M_{N} \;| \;P|M_{P} }
\end{mathpar}

\begin{definition}[contextual application] Given a context $M$, and
  process $P$, we define the \emph{contextual application}, $M[P] :=
  M\{P/\Box\}$. That is, the contextual application of M to P is the
  substitution of $P$ for $\Box$ in $M$.
\end{definition}

$\meaningof{-} : L \to \mathcal{P}(\pi)$

\begin{mathpar}
  \inferrule* [lab=collection] {} {\meaningof{true} = \pi, \and \meaningof{~E} = \pi \setminus \meaningof{E}, \and \meaningof{E_{1} \& E_{2}} = \meaningof{E_{1}} \cap \meaningof{E_{2}}}
\end{mathpar}

\begin{mathpar}
  \inferrule* [lab=structure] {} {\meaningof{0} = \{ P \in \pi | P \equiv 0 \}, \and \\ \meaningof{E_1 | E_2} = \{ P \in \pi | P \equiv P_{1} | P_{2}, P_{1} \in \meaningof{E_{1}}, P_{2} \in \meaningof{E_2}\} }
\end{mathpar}

\begin{mathpar}
 \inferrule* [lab=behavior] {} {\meaningof{\langle a?b \rangle E} = \{ P \in \pi | P \equiv Q | u?(y)P', \\ \and \\\\ \and \\ \;\;\; u \in \meaningof{a}, \forall z.P'\{z/y\} \in \meaningof{E\{z/b\}}\}, \and \\ \meaningof{a!E} = \{ P \in \pi | P \equiv Q | x!\langle P' \rangle, x \in \meaningof{a} P' \in \meaningof{E}\} }
\end{mathpar}

\begin{mathpar}
 \inferrule* [lab=nominal] {} {\meaningof{\quotep{E}} = \{ \quotep{P} \in \quotep{\pi} | P \in \meaningof{E} \}, \and \meaningof{\quotep{P}} = \{ \quotep{Q} \in \quotep{\pi} | P \equiv Q \} \and \\ \meaningof{@\quotep{E}} = \{ P \in \pi | P \equiv @x, x \in \meaningof{E} \}}
\end{mathpar}

\begin{eqnarray*}
  \\
  \meaningof{-} : TS \to ST
\end{eqnarray*}

\begin{eqnarray*}
  \\
  L : TS \to ST
\end{eqnarray*}

\begin{eqnarray*}
  \\
  P \models E \iff P \in \meaningof{E}
\end{eqnarray*}

\begin{eqnarray*}
  P \approx_{L} Q \iff \forall E \in L. P \models E \iff Q \models E
\end{eqnarray*}

\begin{eqnarray*}
  P \approx_{K} Q
\end{eqnarray*}

\begin{eqnarray*}
  P \approx Q
\end{eqnarray*}

$\approx_{K} = \approx = \approx_{L}$

\subsubsection{Contextual duality}

Note that contexts extend the quotation operation to a family of
operations from processes to names. Given a context, $M$, we can
define a \emph{nominal context}, $\quotep{M}$ by $\quotep{M}[P] :=
\quotep{M[P]}$. To foreshadow what is to come we observe that these
operations enjoy a duality with processes very much like the duality
between vectors and maps from vectors to scalars.

Further, because the calculus is essentially higher-order, we have a
correspondence between contexts and processes. More specifically,
given a name $x$ and a context $M$ we can construct $M^{*}_{x}$ such
that 

\begin{mathpar}
  M^{*}_{x} | \lift{x}{P} \red M[P]
\end{mathpar}

namely,

\begin{mathpar}
  M^{*}_{x} := x?(u).M[\dropn{u}]
\end{mathpar}

The dependence of $M^{*}_{x}$ on a name makes it an abstraction, 

\begin{mathpar}
  M^{*} := (x)x?(u).M[\dropn{u}]
\end{mathpar}

\subsection{Additional notation}

It will sometimes be convenient to denote the process a name
quotes. We already have the notation $x = \quotep{P}$, but it will be
convenient to introduce an alternate notation, $\procn{x}$, when we
want to emphasize the connection to the use of the name. Note that, by
virtue of name equivalence, $\quotep{\procn{x}} \nameeq x$; so, the
notation is consistent with previous definitions.

Further, because names have structure it is possible to effect
substitutions on the basis of that structure. This means we need to
upgrade our notation for substitutions, which we accomplish by
adapting comprehension notation. Thus,

\begin{mathpar}
  P\{ y / x : x \in S \}
\end{mathpar}

is interpreted to mean the process derived from P by replacing (in a
capture-avoiding manner) each occurrence of $x$ in $S$ by $y$. For example,

\begin{mathpar}
  P\{ \quotep{\procn{x}|\procn{x}} / x : x \in \freenames{P} \}
\end{mathpar}

will replace each (occurrence) of a free name $x$ in $P$ by
$\quotep{\procn{x}|\procn{x}}$.

Also, we will avail ourselves of the notation $x^{L}$ and $x^{R}$ to
denote injections of a name into disjoint copies of the name
space. There are numerous ways to accomplish this. One example can be
found in \cite{MeredithR05}. This notation overloads to vectors of
names: $\vec{x}^{\pi} := (x_{i}^{\pi} \; : \; 0 \leq i < |\vec{x}| )$ where $\pi \in \{L,R\}$.

We also use $P^{\Box} := P|\Box$.

In \cite{MeredithR05} an interpretation of the new operator is
given. It turns out that there are several possible interpretations
all enjoying the requisite algebraic properties of the operator (see
\cite{milner91polyadicpi}). We will therefore make liberal use of
$(\nu\; \vec{x})P$.

% subsection the_syntax_and_semantics_of_the_notation_system (end)   

\input{qm2pi.qmops} 

\input{qm2pi.sterngerlach} 

\input{qm2pi.metric} 

% section concurrent_process_calculi (end)

%\input{qm2pi.proofsketch}

% section proof sketch (end)

%\input{qm2pi.slviaknots} 

% section spatial logic via knots (end)

\input{qm2pi.conclusion}

% section conclusion (end)

%\input{qm2pi.dtcodes} 

% section wiring algorithm (end)

\input{qm2pi.ack} 

% section acknowledgments (end)

\newpage


\bibliographystyle{plain}   
\bibliography{../../biblios/main.bib}

\input{qm2pi.rhodetails}

\end{document}

 

%\documentclass[12pt]{llncs}
%\documentclass{jktr}

\usepackage[pdftex]{hyperref}                   
\usepackage {listings}
\usepackage {mathpartir}
\usepackage{bcprules}
%\usepackage{listings}
                       
\usepackage{graphicx} 
%\usepackage[margins=2.5cm,nohead,nofoot]{geometry}
%\usepackage{geometry}
\usepackage{amsfonts}
\usepackage{amstext}
\usepackage{latexsym}
\usepackage{amssymb}
\usepackage{color}


%\include{myPreamble}
\include{qm2pi.local} 

%\ifpdf
%\usepackage[pdftex]{graphicx}
%\else
%\usepackage{graphicx}
%\fi

 % \ifpdf
%  \usepackage{pdfsync}
%  \if


%\title{Brief Article}
%\author{David F. Snyder}
%\author{L.G. Meredith}

%\address{Dept. of Math., Texas State University--San Marcos, San Marcos, TX 78666}
       
\pagestyle{empty}


\begin{document}

\lstset{language=[Objective]Caml,frame=shadowbox}

\input{qm2pi.front}

% section front matter (end)

\input{qm2pi.intro} 
 
% section introduction (end)

% \input{qm2pi.knotations} 

% section notation (end)

\input{qm2pi.process.calculi} 

% section concurrent_process_calculi_and_spatial_logics_ (end)
    
%\input{qm2pi.knots2pi} 

%\input{qm2pi.trefoil} 

%\input{qm2pi.mainthm} 

% subsection basic_interpretation (end)

%\input{qm2pi.rho.presentation} 
\subsection{The syntax and semantics of the notation system}\label{sub:the_syntax_and_semantics_of_the_notation_system} % (fold)

We now summarize a technical presentation of the calculus that
embodies our theory of dynamics. The typical presentation of such a
calculus follows the style of giving generators and relations on
them. The grammar, below, describing term constructors, freely
generates the set of processes, $\Proc$. This set is then quotiented
by a relation known as structural congruence and it is over this set
that the notion of dynamics is expressed. This presentation is
essentially that of \cite{MeredithR05} with the addition of
polyadicity and summation. For readability we have relegated some of
the technical subtleties to an appendix.

\subsubsection{Process grammar}\label{subsub:process_grammar}

\begin{mathpar}
  \inferrule* [lab=synchronization] {} {{M} \bc \pzero \;|\; x?F \;|\; x!C }
  \and
  \inferrule* [lab=abstraction] {} {{F} \bc (x)P}
  \and
  \inferrule* [lab=concretion] {} {{C} \bc \langle Q \rangle}
  \and
  \inferrule* [lab=process] {} {{P,Q} \bc M \;| \;P|Q \;|\; @{x}}
  \and
  \inferrule* [lab=name] {} {{x} \bc \quotep{P}}
\end{mathpar} 

Note that $\vec{x}$ (resp. $\vec{P}$) denotes a vector of names
(resp. processes) of length $|\vec{x}|$ (resp. $|\vec{P}|$). We adopt
the following useful abbreviations.

\begin{mathpar}
   x?(\vec{y}).P := x.(\vec{y})P \and  x\clift{\vec{P}} := x.\clift{\vec{P}}
   \and x!(y) := \lift{x}{\dropn{y}}
   \and \Pi_{i=0}^{n-1}P_i := P_0 | \ldots | P_{n-1}
\end{mathpar}

\subsubsection{Structural congruence}

\paragraph{Free and bound names and alpha-equivalence.} At the
core of structural equivalence is alpha-equivalence which identifies
process that are the same up to a change of variable. Formally, we
recognize the distinction between free and bound names. The free names
of a process, $\freenames{P}$, may be calculated recursively as
follows:

\begin{mathpar}
\freenames{\pzero} := \emptyset
  \and \\
  \freenames{x?(y).P} := \{ x \} \cup (\freenames{P} \setminus \{ y \})
  \and 
  \freenames{x!\langle P \rangle} := \{ x \} \cup \{ P \} 
  \and \\
  \freenames{P|Q} := \freenames{P} \cup \freenames{Q}
  \and \\
  \freenames{@{x}} := \{ x \}
\end{mathpar}

$\pi$
$\quotep{\pi}$

$\freenames{-} : \pi \to \mathcal{P}(\quotep{\pi})$

\begin{eqnarray*}
  \freenames{\pzero} & := & \emptyset \\
  \freenames{x?(y).P} & := & \{ x \} \cup (\freenames{P} \setminus \{ y \}) \\
  \freenames{x!\langle P \rangle} & := & \{ x \} \cup \{ P \} \\
  \freenames{P|Q} & := & \freenames{P} \cup \freenames{Q} \\
  \freenames{\dropn{x}} & := & \{ x \}
\end{eqnarray*}

The bound names of a process, $\boundnames{P}$, are those names occurring in $P$
that are not free. For example, in $x?(y).0$, the name $x$ is free, while $y$ is bound.

\begin{mathpar}
  \inferrule* [lab=monoidal-laws] {} { P|Q \equiv Q|P \and P|0 \equiv P \and P|(Q|R) \equiv (P|Q)|R }
\end{mathpar}

\begin{mathpar}
  \inferrule* [lab=alpha-equivalence] {} { (x)P \equiv (y)P\{y/x\} \and y \not\in \freenames{P} }
\end{mathpar}

\begin{definition}
Then two processes, $P,Q$, are alpha-equivalent if $P = Q\{\vec{y}/\vec{x}\}$ for
some $\vec{x} \in \boundnames{Q},\vec{y} \in \boundnames{P}$, where $Q\{\vec{y}/\vec{x}\}$
denotes the capture-avoiding substitution of $\vec{y}$ for $\vec{x}$ in $Q$.
\end{definition}

\begin{definition}
  The {\em structural congruence} \cite{SangiorgiWalker} , $\equiv$,
  between processes is the least congruence containing
  alpha-equivalence, satisfying the abelian monoid laws
  (associativity, commutativity and $\pzero$ as identity) for parallel
  composition $|$ and for summation $+$.
\end{definition}

\subsection{Name equivalence}

We take name equivalence, written $\nameeq$, to be the smallest
equivalence relation generated by the following rules.

\begin{mathpar}
\inferrule*[lab=Quote-drop]
{ }
{ \quotep{@{x}} \nameeq x }

\inferrule*[lab=Struct-equiv]
{ P \scong Q }
{ \quotep{P} \nameeq \quotep{Q} }
\end{mathpar}

The astute reader will have noticed that the mutual recursion of names
and processes imposes a mutual recursion on alpha-equivalence and
structural equivalence via name-equivalence. Fortunately, all of this
works out pleasantly and we may calculate in the natural way, free of
concern. The reader interested in the details is referred to the
appendix \ref{appendix:rho_details}.

\subsection{Substitution}

We use $\Proc$ for the set of processes, $\QProc$ for the set of
names, and $\id{\{}\vec{y} / \vec{x} \id{\}}$ to denote partial maps,
$s : \QProc \rightarrow \QProc$. A map, $s$ lifts, uniquely, to a map
on process terms, $\widehat{s} : \Proc \rightarrow \Proc$ by the
following equations.

\begin{mathpar}
  (0) \psubstp{Q}{P} := 0 \\
  (R \juxtap S) \psubstp{Q}{P}
  :=    
  (R)\psubstp{Q}{P} \juxtap (S) \psubstp{Q}{P} \\
  (x?(y).R) \psubstp{Q}{P}    
  :=    
  (x)\substp{Q}{P} (z)\concat( (R \psubstn{z}{y}) \psubstp{Q}{P} ) \\
  (\lift{x}{R}) \psubstp{Q}{P}  
  :=
  \lift{(x)\substp{Q}{P}}{ R \psubstp{Q}{P} } \\
%   (\dropn{x})  \psubstp{Q}{P}       
%   := 
%   \left\{ 
%     \begin{array}{ccc} 
%       \dropn{\quotep{Q}} & & x \nameeq \quotep{P} \\
%       \dropn{x} & & otherwise \\
%     \end{array}
%   \right. 
  (\dropn{x})  \psubstp{Q}{P}       
  := 
  \left\{ 
    \begin{array}{ccc} 
      Q & & x \nameeq \quotep{P} \\
      \dropn{x} & & otherwise \\
    \end{array}
  \right.
\end{mathpar}
 

where

\begin{eqnarray}
  (x)\id{\{} \lpquote Q \rpquote / \lpquote P \rpquote \id{\}}            = 
  \left\{ 
    \begin{array}{ccc}
      \lpquote Q \rpquote & & x \nameeq \lpquote P \rpquote \\
      x & & otherwise \\
    \end{array}
  \right. \nonumber
\end{eqnarray}

and $z$ is chosen distinct from $\quotep{P}$, $\quotep{Q}$, the free
names in $Q$, and all the names in $R$. Our $\alpha$-equivalence will
be built in the standard way from this substitution.

\begin{remark}\label{rem:no_self_referential_names}
  One consequence of these definitions is that $\forall P. \quotep{P}
  \not\in \freenames{P}$.
\end{remark}

\subsection{ Dynamic quote: an example }

Anticipating something of what's to come, consider applying the
substitution, $\widehat{\id{\{}u / z \id{\}}}$, to the following pair
of processes, $\lift{w}{y!(z)}$ and $w[ \lpquote y!(z) \rpquote ]$.

\begin{eqnarray}
	\lift{w}{y!(z)}\widehat{\id{\{}u / z \id{\}}}
		& = &
		\lift{w}{y!(u)} \nonumber\\
	w[ \lpquote y!(z) \rpquote ] \widehat{ \id{\{}u / z \id{\}} }
		& = &
		w[ \lpquote y!(z) \rpquote ] \nonumber
\end{eqnarray}

Because the body of the process between quotes is impervious to
substitution, we get radically different answers. In fact, by
examining the first process in an input context,
e.g. $x?(z).\lift{w}{y!(z)}$, we see that the process under the lift
operator may be shaped by prefixed inputs binding a name inside it. In
this sense, the lift operator will be seen as a way to dynamically
construct processes before reifying them as names.

Finally equipped with these standard features we can present the
dynamics of the calculus.

\subsubsection{Operational semantics} 

Finally, we introduce the computational dynamics. What marks these
algebras as distinct from other more traditionally studied algebraic
structures, e.g. vector spaces or polynomial rings, is the manner in
which dynamics is captured. In traditional structures, dynamics is typically
expressed through morphisms between such structures, as in linear maps
between vector spaces or morphisms between rings. In algebras
associated with the semantics of computation, the dynamics is
expressed as part of the algebraic structure itself, through a
reduction reduction relation typically denoted by $\red$. Below, we
give a recursive presentation of this relation for the calculus used
in the encoding.

$\red \subseteq \pi \times \pi$
$\red : \pi \to \mathcal{P}(\pi)$

\begin{mathpar}
  \inferrule* [lab=Comm] { \textsf{match}( x_{src}, x_{trgt} ) } { x_{trgt}?(y)P \; | \; x_{src}!\langle {Q} \rangle \red P\{\quotep{Q}/y}\} }
  \and \\
  \inferrule* [lab=Par] {{P} \red {P}'} {{{P} | {Q}} \red {{P}' | {Q}}}
  \and
  \inferrule* [lab=Equiv]{{{P} \scong {P}'} \andalso {{P}' \red {Q}'} \andalso {{Q}' \scong {Q}}}{{P} \red {Q}}
\end{mathpar}

\begin{eqnarray*}
  match_{\equiv} (\quotep{P},\quotep{Q}) & := & P \equiv Q \\
  match_{\dagger}(\quotep{P},\quotep{Q}) & := & \forall R. P|Q \red^{*} R => R \red^{*} 0 \\
  match_{K}(\quotep{P},\quotep{Q}) & := & K \mbox{ for some context } K
\end{eqnarray*}

$u?(x)P | u!\langle Q \rangle \red P\{\quotep{Q}/x\}$

%We write $\wred$ for $\red^*$, and $P\red$ if $\exists Q $ such that $ P \red Q$.
We write $P\red$ if $\exists Q $ such that $ P \red Q$ and $P\not\red$, otherwise.

\section{Replication}

As mentioned before, it is known that replication (and hence
recursion) can be implemented in a higher-order process algebra
\cite{SangiorgiWalker}. As our first example of calculation with the
machinery thus far presented we give the construction explicitly in
the {\rhoc}.

\begin{eqnarray}
	D_{x} & := & \prefix{x}{y}{(\binpar{\outputp{x}{y}}{@{y}})} \nonumber\\
	\bangp_{x}{P} & := & \binpar{{x}!\langle{\binpar{D_{x}}{P}}\rangle}{D_{x}} \nonumber
\end{eqnarray}

\begin{eqnarray}
	\bangp_{x}{P} & & \nonumber\\
	=
	& {x}!\langle{(\prefix{x}{y}{(\outputp{x}{y} | @{y})) | P}}\rangle 
	      | \prefix{x}{y}{(\outputp{x}{y} | @{y})} & \nonumber\\
	\red
	& (\outputp{x}{y} | @{y})\substn{\quotep{(\prefix{x}{y}{(@{y} | \outputp{x}{y})) | P}}}{y} & \nonumber\\
	=
	& \outputp{x}{\quotep{(\prefix{x}{y}{(\outputp{x}{y} | @{y})) | P}}}
	  | {(\prefix{x}{y}{(\outputp{x}{y} | @{y})) | P}} & \nonumber\\
	\red
	& \ldots & \nonumber\\
	\red^*
	& P | P | \ldots & \nonumber
\end{eqnarray}

Of course, this encoding, as an implementation, runs away, unfolding
$\bangp{P}$ eagerly. A lazier and more implementable replication
operator, restricted to input-guarded processes, may be obtained as follows.

\begin{eqnarray}
\bangp{\prefix{u}{v}{P}} 
	:= 
	\binpar{\lift{x}{\prefix{u}{v}{(\binpar{D(x)}{P})}}}{D(x)} \nonumber
\end{eqnarray}

\begin{remark}
  Note that the lazier definition still does not deal with summation
  or mixed summation (i.e. sums over input and output). The reader is
  invited to construct definitions of replication that deal with these
  features. 

  Further, the definitions are parameterized in a name, $x$. Can you,
  gentle reader, make a definition that eliminates this parameter and
  guarantees no accidental interaction between the replication
  machinery and the process being replicated -- i.e. no accidental
  sharing of names used by the process to get its work done and the
  name(s) used by the replication to effect copying. This latter
  revision of the definition of replication is crucial to obtaining
  the expected identity $!!P \sim !P$.
\end{remark}

\begin{remark}\label{rem:paradoxical_combinator}
  The reader familiar with the lambda calculus will have noticed the
  similarity between $D$ and the paradoxical combinator.

  [Ed. note: the existence of this seems to suggest we have to be more
  restrictive on the set of processes and names we admit if we are to
  support no-cloning.]
\end{remark}

\subsubsection{Bisimulation}

The computational dynamics gives rise to another kind of equivalence,
the equivalence of computational behavior. As previously mentioned
this is typically captured \emph{via} some form of bisimulation.

% The notion we use in this paper is weak barbed bisimulation
% \cite{milner91polyadicpi}.

The notion we use in this paper is derived from weak barbed
bisimulation \cite{milner91polyadicpi}. 

\begin{definition}
An \emph{observation relation}, $\downarrow_{\mathcal N}$, over a set
of names, $\mathcal N$, is the smallest relation satisfying the rules
below.

\infrule[Out-barb]{y \in {\mathcal N}, \; x \nameeq y}
		  {\outputp{x}{v} \downarrow_{\mathcal N} x}
\infrule[Par-barb]{\mbox{$P\downarrow_{\mathcal N} x$ or $Q\downarrow_{\mathcal N} x$}}
		  {\binpar{P}{Q} \downarrow_{\mathcal N} x}

We write $P \Downarrow_{\mathcal N} x$ if there is $Q$ such that 
$P \wred Q$ and $Q \downarrow_{\mathcal N} x$.
\end{definition}

\begin{definition}
%\label{def.bbisim}
An  ${\mathcal N}$-\emph{barbed bisimulation} over a set of names, ${\mathcal N}$, is a symmetric binary relation 
${\mathcal S}_{\mathcal N}$ between agents such that $P\rel{S}_{\mathcal N}Q$ implies:
\begin{enumerate}
\item If $P \red P'$ then $Q \wred Q'$ and $P'\rel{S}_{\mathcal N} Q'$.
\item If $P\downarrow_{\mathcal N} x$, then $Q\Downarrow_{\mathcal N} x$.
\end{enumerate}
$P$ is ${\mathcal N}$-barbed bisimilar to $Q$, written
$P \wbbisim_{\mathcal N} Q$, if $P \rel{S}_{\mathcal N} Q$ for some ${\mathcal N}$-barbed bisimulation ${\mathcal S}_{\mathcal N}$.
\end{definition}

$\mathcal{R} \subseteq \pi \times \pi$

$P \mathcal{R} Q => \forall P'. P \red P' \Rightarrow \exists Q'. Q \red Q', P' \mathcal{R} Q'$

$P \vdash x \Rightarrow Q \vdash x$

\begin{mathpar}
  \inferrule*[lab=Out-barb]{x \nameeq y}{{y}!\langle{Q}\rangle \vdash x}
  \and
  \inferrule*[lab=Par-barb]{\mbox{$P\vdash x$ or $Q\vdash x$}}{\binpar{P}{Q} \vdash x}
\end{mathpar}

\subsubsection{Contexts}

One of the principle advantages of computational calculi like the
$\pi$-calculus is a well-defined notion of context,
contextual-equivalence and a correlation between
contextual-equivalence and notions of bisimulation. The notion of
context allows the decomposition of a process into (sub-)process and
its syntactic environment, its context. Thus, a context may be
thought of as a process with a ``hole'' (written $\Box$) in it. The
application of a context $M$ to a process $P$, written $M[P]$, is
tantamount to filling the hole in $M$ with $P$. In this paper we do
not need the full weight of this theory, but do make use of the notion
of context in the proof the main theorem. 

\begin{mathpar}
  \inferrule* [lab=summation] {} {{M_{M},M_{N}} \bc \Box \;|\; x.M_{A} \;|\; M_{M}+M_{N}}
  \and
  \inferrule* [lab=agent] {} {{M_{A}} \bc (\vec{x})M_{P} \;| \; \clift{P_0,\ldots,M_{P},\ldots,P_N}}
  \and \\
  \inferrule* [lab=process] {} {{M_{P}} \bc M_{N} \;| \;P|M_{P} }
\end{mathpar} 

\begin{mathpar}
  \inferrule* [lab=sychronization] {} {M_{N} \bc \Box \;|\; x?M_{F} \;|\; x!M_{C}}
  \and
  \inferrule* [lab=abstraction] {} {{M_{F}} \bc (x)M_{P} }
  \and
  \inferrule* [lab=concretion] {} {{M_{C}} \bc \langle M_{P} \rangle }
  \and \\
  \inferrule* [lab=process] {} {{M_{P}} \bc M_{N} \;| \;P|M_{P} }
\end{mathpar}

\begin{definition}[contextual application] Given a context $M$, and
  process $P$, we define the \emph{contextual application}, $M[P] :=
  M\{P/\Box\}$. That is, the contextual application of M to P is the
  substitution of $P$ for $\Box$ in $M$.
\end{definition}

$\meaningof{-} : L \to \mathcal{P}(\pi)$

\begin{mathpar}
  \inferrule* [lab=collection] {} {\meaningof{true} = \pi, \and \meaningof{~E} = \pi \setminus \meaningof{E}, \and \meaningof{E_{1} \& E_{2}} = \meaningof{E_{1}} \cap \meaningof{E_{2}}}
\end{mathpar}

\begin{mathpar}
  \inferrule* [lab=structure] {} {\meaningof{0} = \{ P \in \pi | P \equiv 0 \}, \and \\ \meaningof{E_1 | E_2} = \{ P \in \pi | P \equiv P_{1} | P_{2}, P_{1} \in \meaningof{E_{1}}, P_{2} \in \meaningof{E_2}\} }
\end{mathpar}

\begin{mathpar}
 \inferrule* [lab=behavior] {} {\meaningof{\langle a?b \rangle E} = \{ P \in \pi | P \equiv Q | u?(y)P', \\ \and \\\\ \and \\ \;\;\; u \in \meaningof{a}, \forall z.P'\{z/y\} \in \meaningof{E\{z/b\}}\}, \and \\ \meaningof{a!E} = \{ P \in \pi | P \equiv Q | x!\langle P' \rangle, x \in \meaningof{a} P' \in \meaningof{E}\} }
\end{mathpar}

\begin{mathpar}
 \inferrule* [lab=nominal] {} {\meaningof{\quotep{E}} = \{ \quotep{P} \in \quotep{\pi} | P \in \meaningof{E} \}, \and \meaningof{\quotep{P}} = \{ \quotep{Q} \in \quotep{\pi} | P \equiv Q \} \and \\ \meaningof{@\quotep{E}} = \{ P \in \pi | P \equiv @x, x \in \meaningof{E} \}}
\end{mathpar}

\begin{eqnarray*}
  \\
  \meaningof{-} : TS \to ST
\end{eqnarray*}

\begin{eqnarray*}
  \\
  L : TS \to ST
\end{eqnarray*}

\begin{eqnarray*}
  \\
  P \models E \iff P \in \meaningof{E}
\end{eqnarray*}

\begin{eqnarray*}
  P \approx_{L} Q \iff \forall E \in L. P \models E \iff Q \models E
\end{eqnarray*}

\begin{eqnarray*}
  P \approx_{K} Q
\end{eqnarray*}

\begin{eqnarray*}
  P \approx Q
\end{eqnarray*}

$\approx_{K} = \approx = \approx_{L}$

\subsubsection{Contextual duality}

Note that contexts extend the quotation operation to a family of
operations from processes to names. Given a context, $M$, we can
define a \emph{nominal context}, $\quotep{M}$ by $\quotep{M}[P] :=
\quotep{M[P]}$. To foreshadow what is to come we observe that these
operations enjoy a duality with processes very much like the duality
between vectors and maps from vectors to scalars.

Further, because the calculus is essentially higher-order, we have a
correspondence between contexts and processes. More specifically,
given a name $x$ and a context $M$ we can construct $M^{*}_{x}$ such
that 

\begin{mathpar}
  M^{*}_{x} | \lift{x}{P} \red M[P]
\end{mathpar}

namely,

\begin{mathpar}
  M^{*}_{x} := x?(u).M[\dropn{u}]
\end{mathpar}

The dependence of $M^{*}_{x}$ on a name makes it an abstraction, 

\begin{mathpar}
  M^{*} := (x)x?(u).M[\dropn{u}]
\end{mathpar}

\subsection{Additional notation}

It will sometimes be convenient to denote the process a name
quotes. We already have the notation $x = \quotep{P}$, but it will be
convenient to introduce an alternate notation, $\procn{x}$, when we
want to emphasize the connection to the use of the name. Note that, by
virtue of name equivalence, $\quotep{\procn{x}} \nameeq x$; so, the
notation is consistent with previous definitions.

Further, because names have structure it is possible to effect
substitutions on the basis of that structure. This means we need to
upgrade our notation for substitutions, which we accomplish by
adapting comprehension notation. Thus,

\begin{mathpar}
  P\{ y / x : x \in S \}
\end{mathpar}

is interpreted to mean the process derived from P by replacing (in a
capture-avoiding manner) each occurrence of $x$ in $S$ by $y$. For example,

\begin{mathpar}
  P\{ \quotep{\procn{x}|\procn{x}} / x : x \in \freenames{P} \}
\end{mathpar}

will replace each (occurrence) of a free name $x$ in $P$ by
$\quotep{\procn{x}|\procn{x}}$.

Also, we will avail ourselves of the notation $x^{L}$ and $x^{R}$ to
denote injections of a name into disjoint copies of the name
space. There are numerous ways to accomplish this. One example can be
found in \cite{MeredithR05}. This notation overloads to vectors of
names: $\vec{x}^{\pi} := (x_{i}^{\pi} \; : \; 0 \leq i < |\vec{x}| )$ where $\pi \in \{L,R\}$.

We also use $P^{\Box} := P|\Box$.

In \cite{MeredithR05} an interpretation of the new operator is
given. It turns out that there are several possible interpretations
all enjoying the requisite algebraic properties of the operator (see
\cite{milner91polyadicpi}). We will therefore make liberal use of
$(\nu\; \vec{x})P$.

% subsection the_syntax_and_semantics_of_the_notation_system (end)   

\input{qm2pi.qmops} 

\input{qm2pi.sterngerlach} 

\input{qm2pi.metric} 

% section concurrent_process_calculi (end)

%\input{qm2pi.proofsketch}

% section proof sketch (end)

%\input{qm2pi.slviaknots} 

% section spatial logic via knots (end)

\input{qm2pi.conclusion}

% section conclusion (end)

%\input{qm2pi.dtcodes} 

% section wiring algorithm (end)

\input{qm2pi.ack} 

% section acknowledgments (end)

\newpage


\bibliographystyle{plain}   
\bibliography{../../biblios/main.bib}

\input{qm2pi.rhodetails}

\end{document}

 

% subsection basic_interpretation (end)

%\input{qm2pi.rho.presentation} 
\subsection{The syntax and semantics of the notation system}\label{sub:the_syntax_and_semantics_of_the_notation_system} % (fold)

We now summarize a technical presentation of the calculus that
embodies our theory of dynamics. The typical presentation of such a
calculus follows the style of giving generators and relations on
them. The grammar, below, describing term constructors, freely
generates the set of processes, $\Proc$. This set is then quotiented
by a relation known as structural congruence and it is over this set
that the notion of dynamics is expressed. This presentation is
essentially that of \cite{MeredithR05} with the addition of
polyadicity and summation. For readability we have relegated some of
the technical subtleties to an appendix.

\subsubsection{Process grammar}\label{subsub:process_grammar}

\begin{mathpar}
  \inferrule* [lab=synchronization] {} {{M} \bc \pzero \;|\; x?F \;|\; x!C }
  \and
  \inferrule* [lab=abstraction] {} {{F} \bc (x)P}
  \and
  \inferrule* [lab=concretion] {} {{C} \bc \langle Q \rangle}
  \and
  \inferrule* [lab=process] {} {{P,Q} \bc M \;| \;P|Q \;|\; @{x}}
  \and
  \inferrule* [lab=name] {} {{x} \bc \quotep{P}}
\end{mathpar} 

Note that $\vec{x}$ (resp. $\vec{P}$) denotes a vector of names
(resp. processes) of length $|\vec{x}|$ (resp. $|\vec{P}|$). We adopt
the following useful abbreviations.

\begin{mathpar}
   x?(\vec{y}).P := x.(\vec{y})P \and  x\clift{\vec{P}} := x.\clift{\vec{P}}
   \and x!(y) := \lift{x}{\dropn{y}}
   \and \Pi_{i=0}^{n-1}P_i := P_0 | \ldots | P_{n-1}
\end{mathpar}

\subsubsection{Structural congruence}

\paragraph{Free and bound names and alpha-equivalence.} At the
core of structural equivalence is alpha-equivalence which identifies
process that are the same up to a change of variable. Formally, we
recognize the distinction between free and bound names. The free names
of a process, $\freenames{P}$, may be calculated recursively as
follows:

\begin{mathpar}
\freenames{\pzero} := \emptyset
  \and \\
  \freenames{x?(y).P} := \{ x \} \cup (\freenames{P} \setminus \{ y \})
  \and 
  \freenames{x!\langle P \rangle} := \{ x \} \cup \{ P \} 
  \and \\
  \freenames{P|Q} := \freenames{P} \cup \freenames{Q}
  \and \\
  \freenames{@{x}} := \{ x \}
\end{mathpar}

$\pi$
$\quotep{\pi}$

$\freenames{-} : \pi \to \mathcal{P}(\quotep{\pi})$

\begin{eqnarray*}
  \freenames{\pzero} & := & \emptyset \\
  \freenames{x?(y).P} & := & \{ x \} \cup (\freenames{P} \setminus \{ y \}) \\
  \freenames{x!\langle P \rangle} & := & \{ x \} \cup \{ P \} \\
  \freenames{P|Q} & := & \freenames{P} \cup \freenames{Q} \\
  \freenames{\dropn{x}} & := & \{ x \}
\end{eqnarray*}

The bound names of a process, $\boundnames{P}$, are those names occurring in $P$
that are not free. For example, in $x?(y).0$, the name $x$ is free, while $y$ is bound.

\begin{mathpar}
  \inferrule* [lab=monoidal-laws] {} { P|Q \equiv Q|P \and P|0 \equiv P \and P|(Q|R) \equiv (P|Q)|R }
\end{mathpar}

\begin{mathpar}
  \inferrule* [lab=alpha-equivalence] {} { (x)P \equiv (y)P\{y/x\} \and y \not\in \freenames{P} }
\end{mathpar}

\begin{definition}
Then two processes, $P,Q$, are alpha-equivalent if $P = Q\{\vec{y}/\vec{x}\}$ for
some $\vec{x} \in \boundnames{Q},\vec{y} \in \boundnames{P}$, where $Q\{\vec{y}/\vec{x}\}$
denotes the capture-avoiding substitution of $\vec{y}$ for $\vec{x}$ in $Q$.
\end{definition}

\begin{definition}
  The {\em structural congruence} \cite{SangiorgiWalker} , $\equiv$,
  between processes is the least congruence containing
  alpha-equivalence, satisfying the abelian monoid laws
  (associativity, commutativity and $\pzero$ as identity) for parallel
  composition $|$ and for summation $+$.
\end{definition}

\subsection{Name equivalence}

We take name equivalence, written $\nameeq$, to be the smallest
equivalence relation generated by the following rules.

\begin{mathpar}
\inferrule*[lab=Quote-drop]
{ }
{ \quotep{@{x}} \nameeq x }

\inferrule*[lab=Struct-equiv]
{ P \scong Q }
{ \quotep{P} \nameeq \quotep{Q} }
\end{mathpar}

The astute reader will have noticed that the mutual recursion of names
and processes imposes a mutual recursion on alpha-equivalence and
structural equivalence via name-equivalence. Fortunately, all of this
works out pleasantly and we may calculate in the natural way, free of
concern. The reader interested in the details is referred to the
appendix \ref{appendix:rho_details}.

\subsection{Substitution}

We use $\Proc$ for the set of processes, $\QProc$ for the set of
names, and $\id{\{}\vec{y} / \vec{x} \id{\}}$ to denote partial maps,
$s : \QProc \rightarrow \QProc$. A map, $s$ lifts, uniquely, to a map
on process terms, $\widehat{s} : \Proc \rightarrow \Proc$ by the
following equations.

\begin{mathpar}
  (0) \psubstp{Q}{P} := 0 \\
  (R \juxtap S) \psubstp{Q}{P}
  :=    
  (R)\psubstp{Q}{P} \juxtap (S) \psubstp{Q}{P} \\
  (x?(y).R) \psubstp{Q}{P}    
  :=    
  (x)\substp{Q}{P} (z)\concat( (R \psubstn{z}{y}) \psubstp{Q}{P} ) \\
  (\lift{x}{R}) \psubstp{Q}{P}  
  :=
  \lift{(x)\substp{Q}{P}}{ R \psubstp{Q}{P} } \\
%   (\dropn{x})  \psubstp{Q}{P}       
%   := 
%   \left\{ 
%     \begin{array}{ccc} 
%       \dropn{\quotep{Q}} & & x \nameeq \quotep{P} \\
%       \dropn{x} & & otherwise \\
%     \end{array}
%   \right. 
  (\dropn{x})  \psubstp{Q}{P}       
  := 
  \left\{ 
    \begin{array}{ccc} 
      Q & & x \nameeq \quotep{P} \\
      \dropn{x} & & otherwise \\
    \end{array}
  \right.
\end{mathpar}
 

where

\begin{eqnarray}
  (x)\id{\{} \lpquote Q \rpquote / \lpquote P \rpquote \id{\}}            = 
  \left\{ 
    \begin{array}{ccc}
      \lpquote Q \rpquote & & x \nameeq \lpquote P \rpquote \\
      x & & otherwise \\
    \end{array}
  \right. \nonumber
\end{eqnarray}

and $z$ is chosen distinct from $\quotep{P}$, $\quotep{Q}$, the free
names in $Q$, and all the names in $R$. Our $\alpha$-equivalence will
be built in the standard way from this substitution.

\begin{remark}\label{rem:no_self_referential_names}
  One consequence of these definitions is that $\forall P. \quotep{P}
  \not\in \freenames{P}$.
\end{remark}

\subsection{ Dynamic quote: an example }

Anticipating something of what's to come, consider applying the
substitution, $\widehat{\id{\{}u / z \id{\}}}$, to the following pair
of processes, $\lift{w}{y!(z)}$ and $w[ \lpquote y!(z) \rpquote ]$.

\begin{eqnarray}
	\lift{w}{y!(z)}\widehat{\id{\{}u / z \id{\}}}
		& = &
		\lift{w}{y!(u)} \nonumber\\
	w[ \lpquote y!(z) \rpquote ] \widehat{ \id{\{}u / z \id{\}} }
		& = &
		w[ \lpquote y!(z) \rpquote ] \nonumber
\end{eqnarray}

Because the body of the process between quotes is impervious to
substitution, we get radically different answers. In fact, by
examining the first process in an input context,
e.g. $x?(z).\lift{w}{y!(z)}$, we see that the process under the lift
operator may be shaped by prefixed inputs binding a name inside it. In
this sense, the lift operator will be seen as a way to dynamically
construct processes before reifying them as names.

Finally equipped with these standard features we can present the
dynamics of the calculus.

\subsubsection{Operational semantics} 

Finally, we introduce the computational dynamics. What marks these
algebras as distinct from other more traditionally studied algebraic
structures, e.g. vector spaces or polynomial rings, is the manner in
which dynamics is captured. In traditional structures, dynamics is typically
expressed through morphisms between such structures, as in linear maps
between vector spaces or morphisms between rings. In algebras
associated with the semantics of computation, the dynamics is
expressed as part of the algebraic structure itself, through a
reduction reduction relation typically denoted by $\red$. Below, we
give a recursive presentation of this relation for the calculus used
in the encoding.

$\red \subseteq \pi \times \pi$
$\red : \pi \to \mathcal{P}(\pi)$

\begin{mathpar}
  \inferrule* [lab=Comm] { \textsf{match}( x_{src}, x_{trgt} ) } { x_{trgt}?(y)P \; | \; x_{src}!\langle {Q} \rangle \red P\{\quotep{Q}/y}\} }
  \and \\
  \inferrule* [lab=Par] {{P} \red {P}'} {{{P} | {Q}} \red {{P}' | {Q}}}
  \and
  \inferrule* [lab=Equiv]{{{P} \scong {P}'} \andalso {{P}' \red {Q}'} \andalso {{Q}' \scong {Q}}}{{P} \red {Q}}
\end{mathpar}

\begin{eqnarray*}
  match_{\equiv} (\quotep{P},\quotep{Q}) & := & P \equiv Q \\
  match_{\dagger}(\quotep{P},\quotep{Q}) & := & \forall R. P|Q \red^{*} R => R \red^{*} 0 \\
  match_{K}(\quotep{P},\quotep{Q}) & := & K \mbox{ for some context } K
\end{eqnarray*}

$u?(x)P | u!\langle Q \rangle \red P\{\quotep{Q}/x\}$

%We write $\wred$ for $\red^*$, and $P\red$ if $\exists Q $ such that $ P \red Q$.
We write $P\red$ if $\exists Q $ such that $ P \red Q$ and $P\not\red$, otherwise.

\section{Replication}

As mentioned before, it is known that replication (and hence
recursion) can be implemented in a higher-order process algebra
\cite{SangiorgiWalker}. As our first example of calculation with the
machinery thus far presented we give the construction explicitly in
the {\rhoc}.

\begin{eqnarray}
	D_{x} & := & \prefix{x}{y}{(\binpar{\outputp{x}{y}}{@{y}})} \nonumber\\
	\bangp_{x}{P} & := & \binpar{{x}!\langle{\binpar{D_{x}}{P}}\rangle}{D_{x}} \nonumber
\end{eqnarray}

\begin{eqnarray}
	\bangp_{x}{P} & & \nonumber\\
	=
	& {x}!\langle{(\prefix{x}{y}{(\outputp{x}{y} | @{y})) | P}}\rangle 
	      | \prefix{x}{y}{(\outputp{x}{y} | @{y})} & \nonumber\\
	\red
	& (\outputp{x}{y} | @{y})\substn{\quotep{(\prefix{x}{y}{(@{y} | \outputp{x}{y})) | P}}}{y} & \nonumber\\
	=
	& \outputp{x}{\quotep{(\prefix{x}{y}{(\outputp{x}{y} | @{y})) | P}}}
	  | {(\prefix{x}{y}{(\outputp{x}{y} | @{y})) | P}} & \nonumber\\
	\red
	& \ldots & \nonumber\\
	\red^*
	& P | P | \ldots & \nonumber
\end{eqnarray}

Of course, this encoding, as an implementation, runs away, unfolding
$\bangp{P}$ eagerly. A lazier and more implementable replication
operator, restricted to input-guarded processes, may be obtained as follows.

\begin{eqnarray}
\bangp{\prefix{u}{v}{P}} 
	:= 
	\binpar{\lift{x}{\prefix{u}{v}{(\binpar{D(x)}{P})}}}{D(x)} \nonumber
\end{eqnarray}

\begin{remark}
  Note that the lazier definition still does not deal with summation
  or mixed summation (i.e. sums over input and output). The reader is
  invited to construct definitions of replication that deal with these
  features. 

  Further, the definitions are parameterized in a name, $x$. Can you,
  gentle reader, make a definition that eliminates this parameter and
  guarantees no accidental interaction between the replication
  machinery and the process being replicated -- i.e. no accidental
  sharing of names used by the process to get its work done and the
  name(s) used by the replication to effect copying. This latter
  revision of the definition of replication is crucial to obtaining
  the expected identity $!!P \sim !P$.
\end{remark}

\begin{remark}\label{rem:paradoxical_combinator}
  The reader familiar with the lambda calculus will have noticed the
  similarity between $D$ and the paradoxical combinator.

  [Ed. note: the existence of this seems to suggest we have to be more
  restrictive on the set of processes and names we admit if we are to
  support no-cloning.]
\end{remark}

\subsubsection{Bisimulation}

The computational dynamics gives rise to another kind of equivalence,
the equivalence of computational behavior. As previously mentioned
this is typically captured \emph{via} some form of bisimulation.

% The notion we use in this paper is weak barbed bisimulation
% \cite{milner91polyadicpi}.

The notion we use in this paper is derived from weak barbed
bisimulation \cite{milner91polyadicpi}. 

\begin{definition}
An \emph{observation relation}, $\downarrow_{\mathcal N}$, over a set
of names, $\mathcal N$, is the smallest relation satisfying the rules
below.

\infrule[Out-barb]{y \in {\mathcal N}, \; x \nameeq y}
		  {\outputp{x}{v} \downarrow_{\mathcal N} x}
\infrule[Par-barb]{\mbox{$P\downarrow_{\mathcal N} x$ or $Q\downarrow_{\mathcal N} x$}}
		  {\binpar{P}{Q} \downarrow_{\mathcal N} x}

We write $P \Downarrow_{\mathcal N} x$ if there is $Q$ such that 
$P \wred Q$ and $Q \downarrow_{\mathcal N} x$.
\end{definition}

\begin{definition}
%\label{def.bbisim}
An  ${\mathcal N}$-\emph{barbed bisimulation} over a set of names, ${\mathcal N}$, is a symmetric binary relation 
${\mathcal S}_{\mathcal N}$ between agents such that $P\rel{S}_{\mathcal N}Q$ implies:
\begin{enumerate}
\item If $P \red P'$ then $Q \wred Q'$ and $P'\rel{S}_{\mathcal N} Q'$.
\item If $P\downarrow_{\mathcal N} x$, then $Q\Downarrow_{\mathcal N} x$.
\end{enumerate}
$P$ is ${\mathcal N}$-barbed bisimilar to $Q$, written
$P \wbbisim_{\mathcal N} Q$, if $P \rel{S}_{\mathcal N} Q$ for some ${\mathcal N}$-barbed bisimulation ${\mathcal S}_{\mathcal N}$.
\end{definition}

$\mathcal{R} \subseteq \pi \times \pi$

$P \mathcal{R} Q => \forall P'. P \red P' \Rightarrow \exists Q'. Q \red Q', P' \mathcal{R} Q'$

$P \vdash x \Rightarrow Q \vdash x$

\begin{mathpar}
  \inferrule*[lab=Out-barb]{x \nameeq y}{{y}!\langle{Q}\rangle \vdash x}
  \and
  \inferrule*[lab=Par-barb]{\mbox{$P\vdash x$ or $Q\vdash x$}}{\binpar{P}{Q} \vdash x}
\end{mathpar}

\subsubsection{Contexts}

One of the principle advantages of computational calculi like the
$\pi$-calculus is a well-defined notion of context,
contextual-equivalence and a correlation between
contextual-equivalence and notions of bisimulation. The notion of
context allows the decomposition of a process into (sub-)process and
its syntactic environment, its context. Thus, a context may be
thought of as a process with a ``hole'' (written $\Box$) in it. The
application of a context $M$ to a process $P$, written $M[P]$, is
tantamount to filling the hole in $M$ with $P$. In this paper we do
not need the full weight of this theory, but do make use of the notion
of context in the proof the main theorem. 

\begin{mathpar}
  \inferrule* [lab=summation] {} {{M_{M},M_{N}} \bc \Box \;|\; x.M_{A} \;|\; M_{M}+M_{N}}
  \and
  \inferrule* [lab=agent] {} {{M_{A}} \bc (\vec{x})M_{P} \;| \; \clift{P_0,\ldots,M_{P},\ldots,P_N}}
  \and \\
  \inferrule* [lab=process] {} {{M_{P}} \bc M_{N} \;| \;P|M_{P} }
\end{mathpar} 

\begin{mathpar}
  \inferrule* [lab=sychronization] {} {M_{N} \bc \Box \;|\; x?M_{F} \;|\; x!M_{C}}
  \and
  \inferrule* [lab=abstraction] {} {{M_{F}} \bc (x)M_{P} }
  \and
  \inferrule* [lab=concretion] {} {{M_{C}} \bc \langle M_{P} \rangle }
  \and \\
  \inferrule* [lab=process] {} {{M_{P}} \bc M_{N} \;| \;P|M_{P} }
\end{mathpar}

\begin{definition}[contextual application] Given a context $M$, and
  process $P$, we define the \emph{contextual application}, $M[P] :=
  M\{P/\Box\}$. That is, the contextual application of M to P is the
  substitution of $P$ for $\Box$ in $M$.
\end{definition}

$\meaningof{-} : L \to \mathcal{P}(\pi)$

\begin{mathpar}
  \inferrule* [lab=collection] {} {\meaningof{true} = \pi, \and \meaningof{~E} = \pi \setminus \meaningof{E}, \and \meaningof{E_{1} \& E_{2}} = \meaningof{E_{1}} \cap \meaningof{E_{2}}}
\end{mathpar}

\begin{mathpar}
  \inferrule* [lab=structure] {} {\meaningof{0} = \{ P \in \pi | P \equiv 0 \}, \and \\ \meaningof{E_1 | E_2} = \{ P \in \pi | P \equiv P_{1} | P_{2}, P_{1} \in \meaningof{E_{1}}, P_{2} \in \meaningof{E_2}\} }
\end{mathpar}

\begin{mathpar}
 \inferrule* [lab=behavior] {} {\meaningof{\langle a?b \rangle E} = \{ P \in \pi | P \equiv Q | u?(y)P', \\ \and \\\\ \and \\ \;\;\; u \in \meaningof{a}, \forall z.P'\{z/y\} \in \meaningof{E\{z/b\}}\}, \and \\ \meaningof{a!E} = \{ P \in \pi | P \equiv Q | x!\langle P' \rangle, x \in \meaningof{a} P' \in \meaningof{E}\} }
\end{mathpar}

\begin{mathpar}
 \inferrule* [lab=nominal] {} {\meaningof{\quotep{E}} = \{ \quotep{P} \in \quotep{\pi} | P \in \meaningof{E} \}, \and \meaningof{\quotep{P}} = \{ \quotep{Q} \in \quotep{\pi} | P \equiv Q \} \and \\ \meaningof{@\quotep{E}} = \{ P \in \pi | P \equiv @x, x \in \meaningof{E} \}}
\end{mathpar}

\begin{eqnarray*}
  \\
  \meaningof{-} : TS \to ST
\end{eqnarray*}

\begin{eqnarray*}
  \\
  L : TS \to ST
\end{eqnarray*}

\begin{eqnarray*}
  \\
  P \models E \iff P \in \meaningof{E}
\end{eqnarray*}

\begin{eqnarray*}
  P \approx_{L} Q \iff \forall E \in L. P \models E \iff Q \models E
\end{eqnarray*}

\begin{eqnarray*}
  P \approx_{K} Q
\end{eqnarray*}

\begin{eqnarray*}
  P \approx Q
\end{eqnarray*}

$\approx_{K} = \approx = \approx_{L}$

\subsubsection{Contextual duality}

Note that contexts extend the quotation operation to a family of
operations from processes to names. Given a context, $M$, we can
define a \emph{nominal context}, $\quotep{M}$ by $\quotep{M}[P] :=
\quotep{M[P]}$. To foreshadow what is to come we observe that these
operations enjoy a duality with processes very much like the duality
between vectors and maps from vectors to scalars.

Further, because the calculus is essentially higher-order, we have a
correspondence between contexts and processes. More specifically,
given a name $x$ and a context $M$ we can construct $M^{*}_{x}$ such
that 

\begin{mathpar}
  M^{*}_{x} | \lift{x}{P} \red M[P]
\end{mathpar}

namely,

\begin{mathpar}
  M^{*}_{x} := x?(u).M[\dropn{u}]
\end{mathpar}

The dependence of $M^{*}_{x}$ on a name makes it an abstraction, 

\begin{mathpar}
  M^{*} := (x)x?(u).M[\dropn{u}]
\end{mathpar}

\subsection{Additional notation}

It will sometimes be convenient to denote the process a name
quotes. We already have the notation $x = \quotep{P}$, but it will be
convenient to introduce an alternate notation, $\procn{x}$, when we
want to emphasize the connection to the use of the name. Note that, by
virtue of name equivalence, $\quotep{\procn{x}} \nameeq x$; so, the
notation is consistent with previous definitions.

Further, because names have structure it is possible to effect
substitutions on the basis of that structure. This means we need to
upgrade our notation for substitutions, which we accomplish by
adapting comprehension notation. Thus,

\begin{mathpar}
  P\{ y / x : x \in S \}
\end{mathpar}

is interpreted to mean the process derived from P by replacing (in a
capture-avoiding manner) each occurrence of $x$ in $S$ by $y$. For example,

\begin{mathpar}
  P\{ \quotep{\procn{x}|\procn{x}} / x : x \in \freenames{P} \}
\end{mathpar}

will replace each (occurrence) of a free name $x$ in $P$ by
$\quotep{\procn{x}|\procn{x}}$.

Also, we will avail ourselves of the notation $x^{L}$ and $x^{R}$ to
denote injections of a name into disjoint copies of the name
space. There are numerous ways to accomplish this. One example can be
found in \cite{MeredithR05}. This notation overloads to vectors of
names: $\vec{x}^{\pi} := (x_{i}^{\pi} \; : \; 0 \leq i < |\vec{x}| )$ where $\pi \in \{L,R\}$.

We also use $P^{\Box} := P|\Box$.

In \cite{MeredithR05} an interpretation of the new operator is
given. It turns out that there are several possible interpretations
all enjoying the requisite algebraic properties of the operator (see
\cite{milner91polyadicpi}). We will therefore make liberal use of
$(\nu\; \vec{x})P$.

% subsection the_syntax_and_semantics_of_the_notation_system (end)   

\section{Interpretation of QM}
\subsection{Supporting definitions}
\subsubsection{Multiplication}
\begin{mathpar}
  \quotep{Q} \cdot \quotep{R} := \quotep{Q|R}
  \and \\
  \quotep{Q} \cdot P := P\{ \quotep{Q|R} / \quotep{R} : \quotep{R} \in \freenames{P} \}
\end{mathpar}

\paragraph{Discussion}
The first line needs little explanation. The second line says that
each free name of the process is replaced with the multiplication of
that name by the scalar. Multiplication of a scalar (name) by a state
(process) results in a process all the names of which have been `moved
over' by parallel composition with the process the scalar
quotes. There is a subtlety that the bound names have to be
manipulated so that multiplied names aren't accidentally
captured. There are many ways to achieve this.

\begin{remark}\label{rem:multiplication_identities}
  The reader is invited to verify that for all $x,y,z \in \QProc$ and $P \in \Proc$
  \begin{mathpar}
    x \cdot \quotep{0} \equiv x 
    \and
    x \cdot y \equiv y \cdot x
    \and
    x \cdot (y \cdot z) \equiv (x \cdot y) \cdot z
    \and \\
    \quotep{0} \cdot P \equiv P
    \and \\
    x \cdot (y \cdot P) \equiv (x \cdot y) \cdot P
    \and \\
    x \cdot (P|Q) \equiv (x \cdot P) | (x \cdot Q)
    \and \\    
  \end{mathpar}
\end{remark}

\subsubsection{Tensor product}

We define a tensor product on processes by structural induction.

\paragraph{Tensor of sums} First note that all summations, including
$\pzero$ and sequence, can be written $\Sigma_{i} x_{i}.A_{i} +
\Sigma_{j} x_{j}.C_{j}$, where we have grouped input-guarded processes
together and output-guarded processes together.

Thus, we can define the tensor product of two summations, $N_{1}\otimes N_{2}$, where

\begin{mathpar}
  N_{1} := \Sigma_{i} x_{i}.A_{i} + \Sigma_{j} x_{j}.C_{j}
  \and
  N_{2} := \Sigma_{i'} y_{i'}.B_{i'} + \Sigma_{j'} y_{j'}.D_{j'} 
\end{mathpar}

as follows.

\begin{mathpar}
  \Sigma_{i} x_{i}.A_{i} + \Sigma_{j} x_{j}.C_{j} \otimes \Sigma_{i'}
  y_{i'}.B_{i'} + \Sigma_{j'} y_{j'}.D_{j'} 
  \and \\
  := \; \Sigma_{i} \Sigma_{i'} \quotep{\stackrel{\vee}{x_{i}}| \stackrel{\vee}{y_{i'}}}.(A_{i}\otimes B_{i'}) \; | \; \Sigma_{i'} \Sigma_{i} \quotep{\stackrel{\vee}{y_{i'}}|\stackrel{\vee}{x_{i}}}.(B_{i'}\otimes A_{i})
  \and
  \;\; | \;\; \Sigma_{j} \Sigma_{j'} \quotep{\stackrel{\vee}{x_{j}}|\stackrel{\vee}{y_{j'}}}.(A_{j}\otimes B_{j'}) \; | \; \Sigma_{j'} \Sigma_{j} \quotep{\stackrel{\vee}{y_{j'}}|\stackrel{\vee}{x_{j}}}.(B_{j'}\otimes A_{j})
\end{mathpar}

\begin{remark}
  Do we need to $x^{L}$ and $y^{R}$ for this construction as well?
\end{remark}

\paragraph{Tensor of parallel compositions} Next, we distribute tensor
over par.

\begin{mathpar}
  P_{1}|P_{2} \otimes Q_{1}|Q_{2} := (P_{1} \otimes Q_{1}) | (P_{1}
  \otimes Q_{2}) | (P_{2} \otimes Q_{1}) | (P_{2} \otimes Q_{2})
\end{mathpar}

\paragraph{Tensor with dropped names} We treat tensor of a
process with a dropped name as parallel composition.

\begin{mathpar}
  P \otimes \dropn{x} := P | \dropn{x}
\end{mathpar}

\paragraph{Tensor of agents}

Finally, we need to define tensor on agents. Note that the definition
of tensor on normal products only tensors inputs with inputs and
outputs with outputs. Thus, we only have to define the operation on
``homogeneous'' pairings.

\begin{mathpar}
  (\vec{x})P \otimes (\vec{y})Q
  \and \\
  := (x_{0}^{L}|y_{0}^{R},\ldots,x_{0}^{L}|y_{n}^{R},\ldots,x_{m}^{L}|y_{0}^{R},\ldots,x_{m}^{L}|y_{n}^R)(P\{ \vec{x}^{L}/\vec{x}\} \otimes Q \{ \vec{y}^{R}/\vec{y}\})
  \and \\
  \clift{\vec{P}} \otimes \clift{\vec{Q}}
  \and \\
  := \clift{P_{0}\otimes Q_{0},\ldots,P_{0}\otimes Q_{n},\ldots,P_{m}\otimes Q_{0},\ldots,P_{m}\otimes Q_{n}}
\end{mathpar}

\begin{remark}
  Observe that arities of tensored abstractions matches arities of
  tensored concretions if the original arities matched. Note also that
  the length of the arities corresponds to the increase in dimension
  we see in ordinary vector space tensor product.
\end{remark}

\begin{remark}
  Operationally, this definition distributes the tensor down to
  components ``linked'' by summation. Tensor over summation is
  intriguing in that it mixes names. Moreover, as a consequence of the
  way it mixes names we have the identities for all $x \in \QProc$ and
  $P,Q \in \Proc$

  \begin{mathpar}
    (x \cdot P) \otimes Q \equiv x \cdot (P \otimes Q) \equiv P \otimes (x \cdot Q)
    \and
    P \otimes \pzero \equiv P
  \end{mathpar}

  that the reader is invited to verify.
\end{remark}

\subsubsection{Annihilation}
\begin{mathpar}
  P^{\perp} := \{ Q | \forall R. P|Q \red^{*} R \Rightarrow R \red^{*} \pzero \}
  \and \\
  P^{\underline{\perp}} := \Sigma_{Q \in P^{\perp}} \quotep{Q}?(y).(\dropn{y}|Q) | \Sigma_{Q \in P^{\perp}} \quotep{Q}\clift{\Box}
\end{mathpar}

\paragraph{Discussion} The reader will note that $P^{\perp}$ is a
\emph{set} of processes, while $P^{\underline{\perp}}$ is a
\emph{context}. We call the set $P^{\perp}$ the \emph{annihilators} of
$P$. The parallel composition of a process in the annihilators of $P$
with $P$ will result in a process, the state space of which has all
paths eventually leading to $\pzero$. Execution may endure loops; but
under reasonable conditions of fairness (naturally guaranteed under
most notions of bisimulation) such a composite process cannot get
stuck in such a loop and will, eventually pop out and terminate.

The context $P^{\underline{\perp}}$ is ready and willing to ``take the
$P$ out of'' the process to which it is applied. It will effectively
transmit the code of the process to which it is applied to one of the
annihilators and run the process against it.

\subsubsection{Evaluation}
We fix $M$ a domain of fully abstract interpretation with an equality
coincident with bisimulation. We take $\meaningof{\cdot} : \Proc \to
M$ to be the map interpreting processes and $\nmeaningof{\cdot} : \M
\to Proc$ to be the map running the other way. Then we define

\begin{mathpar}
  \int P := \nmeaningof{\meaningof{P}}
\end{mathpar}

\paragraph{Discussion}
There are many fully abstract interpretations of Milner's
$\pi$-calculus. Any of them can be used as a basis for interpreting
the reflective calculus here. Equipped with such a domain it is
largely a matter of grinding through to check that the Yoneda
construction for the normalization-by-evaluation program can be
extended to this setting.

\begin{remark}
  The reader is invited to verify that $\int (P^{\underline{\perp}}[P]) = 0$.
\end{remark}

\subsection{Quantum mechanics}

Table \ref{tbl:core_qm_op_defns} gives the core operational definitions

\begin{table}[htp]\label{tbl:core_qm_op_defns}
  \center{
    \fbox{
      \begin{tabular}{c|c}
        quantum mechanics & process calculus \\
        \hline
        scalar & $x := \quotep{P}$ \\
        state vector & $\state{P} := P$ \\
        dual & $\state{P}^{*} := \event{P^{\underline{\perp}}} := \quotep{P^{\underline{\perp}}}[-]$ \\
        matrix & $ \Sigma_{\alpha} \state{P_{\alpha}}x_{\alpha}\event{Q_{\alpha}}$ \\
        vector addition & $\state{P} + \state{Q} := \state{P | Q}$ \\
        tensor product & $\state{P} \otimes \state{Q} := \state{P \otimes Q}$ \\
        inner product & $\innerprod{P}{Q} := \quotep{\int P^{\underline{\perp}}[Q]}$ \\
      \end{tabular}
    }
  }
  \caption{QM - operational definitions}
\end{table}

where

\begin{mathpar}
  \prmatrix{P}{Q} := \fprmatrix{P}{\quotep{\pzero}}{Q}
  \and
  \fprmatrix{P}{x}{Q} := (\state{P},x,\event{Q})
  \and
  (\fprmatrix{P}{x}{Q})(\state{R}) := x \cdot \innerprod{Q}{R} \cdot \state{P}
  \and
  (\fprmatrix{P}{x}{Q})(\event{R}) := x \cdot \innerprod{R}{P} \cdot \event{Q}
\end{mathpar}

\paragraph{Discussion}
As promised: vectors (aka states) are represented as processes; duals
as contextual duals; inner product definition should be compared with
standard inner product definition for ....

\begin{remark}
  Assuming $\int (P^{\underline{\perp}}[P]) = 0$, the reader is
  invited to verify that $(\fprmatrix{P}{x}{P})(\state{P}) = x \cdot \state{P}$.
\end{remark}

\begin{remark}
  The reader is invited to verify that $\innerprod{P}{Q}$ could
  equally well have been written $\quotep{\int \stackrel{\vee}{x}}$
  where $x = \event{P^{\underline{\perp}}}(Q)$.

  One of the motivations for this remark is that there is another way
  to factor these operations. We could package up evaluation in the dual:

  \begin{mathpar}
    \state{P}^{*} := \event{\int P^{\underline{\perp}}} := \quotep{\int P^{\underline{\perp}}}[-]
  \end{mathpar}

  and then have inner product defined by
  
  \begin{mathpar}
    \innerprod{P}{Q} := \event{P}(Q)
  \end{mathpar}

  Hopefully, experience with the calculations will provide guidance on
  the best factoring.
\end{remark}

\begin{remark}
  Assuming $\int (P^{\underline{\perp}}[P]) = 0$, the reader is
  invited to verify that $\forall P,Q. (\prmatrix{0}{Q})(\state{0}) =
  \state{0}$ and dually $(\prmatrix{P}{0})(\event{0}) = \event{0}$.
\end{remark}

\begin{remark}
  i'm a little worried that i don't (yet) have proper support for
  complex conjugacy. But, the observation above may give us a
  clue. According to Abramsky, it must be the case that the scalars
  are iso to the homset of the identity for the tensor -- which the
  observation above characterizes. 

  For now, we will simply bookmark the notion with $\overline{x}$.
\end{remark}

\subsubsection{Adjointness}

We need to give a definition of $(\cdot)^{\dagger}$ for matrices. The
obvious candidate definition is
\begin{mathpar}
(\Sigma_{\alpha}\fprmatrix{P_{\alpha}}{x_{\alpha}}{Q_{\alpha}})^{\dagger}
= \Sigma_{\alpha}\fprmatrix{(Q_{\alpha}^{\underline{\perp}})^{*}}{\overline{x}_{\alpha}}{P_{\alpha}^{\underline{\perp}}} 
\end{mathpar}

But, $(Q_{\alpha}^{\underline{\perp}})^{*}$ requires a name along
which to communicate the process to achieve the context application.

\subsubsection{Basis for a basis}
If processes label states and ``addition'' of states (a.k.a. vector
addition) is interpreted as parallel composition, what corresponds to
notions of linear independence and basis? Here, we recall that Yoshida
has developed a set of \emph{combinators} for an asynchronous verison
of Milner's $\pi$-calculus. These are a finite set of processes such
any process can be expressed as parallel composition of these
combinators together with liberal uses of the new operator and
replication. We can simply give a translation of these into the
present calculus and have reasonable expectation that the property
carries over. That is, that the resultant set allows to express all
processes via parallel composition. Note, however, that there is no
new operator or replication in this calculus. As a result, we expect
that the corresponding set is actually infinite. That is, we expect
that the space is actually infinite dimensional.

\begin{remark}
  The attentive reader may be a bit concerned. Certainly, the
  collection $S$, $K$ and $I$ is a finite set of
  combinators. Shouldn't we expect to see a finite set of combinators
  for an effectively equivalent system? i am very sympathetic to this
  critique and feel it warrants full attention. On the other hand, i
  also have in mind the following analogy. The natural numbers, as a
  monoid under addition, has exactly $1$ generator, while the natural
  numbers, as a monoid under multiplication, has countably many
  generators (the primes). We observe that the application of the
  lambda calculus is much less resource sensitive than the parallel
  composition of the $\pi$-calculus. Could it be the case that we have
  an analogy of the form
  
  \begin{mathpar}
    m + n : MN :: m*n : M|N
  \end{mathpar}

  giving a similar blow up in the set of ``primes''?  This is such a
  wonderful thought that, even if it's not true, i think it's worth
  writing down.
\end{remark}
 

\documentclass[12pt]{llncs}
%\documentclass{jktr}

\usepackage[pdftex]{hyperref}                   
\usepackage {listings}
\usepackage {mathpartir}
\usepackage{bcprules}
%\usepackage{listings}
                       
\usepackage{graphicx} 
%\usepackage[margins=2.5cm,nohead,nofoot]{geometry}
%\usepackage{geometry}
\usepackage{amsfonts}
\usepackage{amstext}
\usepackage{latexsym}
\usepackage{amssymb}
\usepackage{color}


%\include{myPreamble}
\include{qm2pi.local} 

%\ifpdf
%\usepackage[pdftex]{graphicx}
%\else
%\usepackage{graphicx}
%\fi

 % \ifpdf
%  \usepackage{pdfsync}
%  \if


%\title{Brief Article}
%\author{David F. Snyder}
%\author{L.G. Meredith}

%\address{Dept. of Math., Texas State University--San Marcos, San Marcos, TX 78666}
       
\pagestyle{empty}


\begin{document}

\lstset{language=[Objective]Caml,frame=shadowbox}

\input{qm2pi.front}

% section front matter (end)

\input{qm2pi.intro} 
 
% section introduction (end)

% \input{qm2pi.knotations} 

% section notation (end)

\input{qm2pi.process.calculi} 

% section concurrent_process_calculi_and_spatial_logics_ (end)
    
%\input{qm2pi.knots2pi} 

%\input{qm2pi.trefoil} 

%\input{qm2pi.mainthm} 

% subsection basic_interpretation (end)

%\input{qm2pi.rho.presentation} 
\subsection{The syntax and semantics of the notation system}\label{sub:the_syntax_and_semantics_of_the_notation_system} % (fold)

We now summarize a technical presentation of the calculus that
embodies our theory of dynamics. The typical presentation of such a
calculus follows the style of giving generators and relations on
them. The grammar, below, describing term constructors, freely
generates the set of processes, $\Proc$. This set is then quotiented
by a relation known as structural congruence and it is over this set
that the notion of dynamics is expressed. This presentation is
essentially that of \cite{MeredithR05} with the addition of
polyadicity and summation. For readability we have relegated some of
the technical subtleties to an appendix.

\subsubsection{Process grammar}\label{subsub:process_grammar}

\begin{mathpar}
  \inferrule* [lab=synchronization] {} {{M} \bc \pzero \;|\; x?F \;|\; x!C }
  \and
  \inferrule* [lab=abstraction] {} {{F} \bc (x)P}
  \and
  \inferrule* [lab=concretion] {} {{C} \bc \langle Q \rangle}
  \and
  \inferrule* [lab=process] {} {{P,Q} \bc M \;| \;P|Q \;|\; @{x}}
  \and
  \inferrule* [lab=name] {} {{x} \bc \quotep{P}}
\end{mathpar} 

Note that $\vec{x}$ (resp. $\vec{P}$) denotes a vector of names
(resp. processes) of length $|\vec{x}|$ (resp. $|\vec{P}|$). We adopt
the following useful abbreviations.

\begin{mathpar}
   x?(\vec{y}).P := x.(\vec{y})P \and  x\clift{\vec{P}} := x.\clift{\vec{P}}
   \and x!(y) := \lift{x}{\dropn{y}}
   \and \Pi_{i=0}^{n-1}P_i := P_0 | \ldots | P_{n-1}
\end{mathpar}

\subsubsection{Structural congruence}

\paragraph{Free and bound names and alpha-equivalence.} At the
core of structural equivalence is alpha-equivalence which identifies
process that are the same up to a change of variable. Formally, we
recognize the distinction between free and bound names. The free names
of a process, $\freenames{P}$, may be calculated recursively as
follows:

\begin{mathpar}
\freenames{\pzero} := \emptyset
  \and \\
  \freenames{x?(y).P} := \{ x \} \cup (\freenames{P} \setminus \{ y \})
  \and 
  \freenames{x!\langle P \rangle} := \{ x \} \cup \{ P \} 
  \and \\
  \freenames{P|Q} := \freenames{P} \cup \freenames{Q}
  \and \\
  \freenames{@{x}} := \{ x \}
\end{mathpar}

$\pi$
$\quotep{\pi}$

$\freenames{-} : \pi \to \mathcal{P}(\quotep{\pi})$

\begin{eqnarray*}
  \freenames{\pzero} & := & \emptyset \\
  \freenames{x?(y).P} & := & \{ x \} \cup (\freenames{P} \setminus \{ y \}) \\
  \freenames{x!\langle P \rangle} & := & \{ x \} \cup \{ P \} \\
  \freenames{P|Q} & := & \freenames{P} \cup \freenames{Q} \\
  \freenames{\dropn{x}} & := & \{ x \}
\end{eqnarray*}

The bound names of a process, $\boundnames{P}$, are those names occurring in $P$
that are not free. For example, in $x?(y).0$, the name $x$ is free, while $y$ is bound.

\begin{mathpar}
  \inferrule* [lab=monoidal-laws] {} { P|Q \equiv Q|P \and P|0 \equiv P \and P|(Q|R) \equiv (P|Q)|R }
\end{mathpar}

\begin{mathpar}
  \inferrule* [lab=alpha-equivalence] {} { (x)P \equiv (y)P\{y/x\} \and y \not\in \freenames{P} }
\end{mathpar}

\begin{definition}
Then two processes, $P,Q$, are alpha-equivalent if $P = Q\{\vec{y}/\vec{x}\}$ for
some $\vec{x} \in \boundnames{Q},\vec{y} \in \boundnames{P}$, where $Q\{\vec{y}/\vec{x}\}$
denotes the capture-avoiding substitution of $\vec{y}$ for $\vec{x}$ in $Q$.
\end{definition}

\begin{definition}
  The {\em structural congruence} \cite{SangiorgiWalker} , $\equiv$,
  between processes is the least congruence containing
  alpha-equivalence, satisfying the abelian monoid laws
  (associativity, commutativity and $\pzero$ as identity) for parallel
  composition $|$ and for summation $+$.
\end{definition}

\subsection{Name equivalence}

We take name equivalence, written $\nameeq$, to be the smallest
equivalence relation generated by the following rules.

\begin{mathpar}
\inferrule*[lab=Quote-drop]
{ }
{ \quotep{@{x}} \nameeq x }

\inferrule*[lab=Struct-equiv]
{ P \scong Q }
{ \quotep{P} \nameeq \quotep{Q} }
\end{mathpar}

The astute reader will have noticed that the mutual recursion of names
and processes imposes a mutual recursion on alpha-equivalence and
structural equivalence via name-equivalence. Fortunately, all of this
works out pleasantly and we may calculate in the natural way, free of
concern. The reader interested in the details is referred to the
appendix \ref{appendix:rho_details}.

\subsection{Substitution}

We use $\Proc$ for the set of processes, $\QProc$ for the set of
names, and $\id{\{}\vec{y} / \vec{x} \id{\}}$ to denote partial maps,
$s : \QProc \rightarrow \QProc$. A map, $s$ lifts, uniquely, to a map
on process terms, $\widehat{s} : \Proc \rightarrow \Proc$ by the
following equations.

\begin{mathpar}
  (0) \psubstp{Q}{P} := 0 \\
  (R \juxtap S) \psubstp{Q}{P}
  :=    
  (R)\psubstp{Q}{P} \juxtap (S) \psubstp{Q}{P} \\
  (x?(y).R) \psubstp{Q}{P}    
  :=    
  (x)\substp{Q}{P} (z)\concat( (R \psubstn{z}{y}) \psubstp{Q}{P} ) \\
  (\lift{x}{R}) \psubstp{Q}{P}  
  :=
  \lift{(x)\substp{Q}{P}}{ R \psubstp{Q}{P} } \\
%   (\dropn{x})  \psubstp{Q}{P}       
%   := 
%   \left\{ 
%     \begin{array}{ccc} 
%       \dropn{\quotep{Q}} & & x \nameeq \quotep{P} \\
%       \dropn{x} & & otherwise \\
%     \end{array}
%   \right. 
  (\dropn{x})  \psubstp{Q}{P}       
  := 
  \left\{ 
    \begin{array}{ccc} 
      Q & & x \nameeq \quotep{P} \\
      \dropn{x} & & otherwise \\
    \end{array}
  \right.
\end{mathpar}
 

where

\begin{eqnarray}
  (x)\id{\{} \lpquote Q \rpquote / \lpquote P \rpquote \id{\}}            = 
  \left\{ 
    \begin{array}{ccc}
      \lpquote Q \rpquote & & x \nameeq \lpquote P \rpquote \\
      x & & otherwise \\
    \end{array}
  \right. \nonumber
\end{eqnarray}

and $z$ is chosen distinct from $\quotep{P}$, $\quotep{Q}$, the free
names in $Q$, and all the names in $R$. Our $\alpha$-equivalence will
be built in the standard way from this substitution.

\begin{remark}\label{rem:no_self_referential_names}
  One consequence of these definitions is that $\forall P. \quotep{P}
  \not\in \freenames{P}$.
\end{remark}

\subsection{ Dynamic quote: an example }

Anticipating something of what's to come, consider applying the
substitution, $\widehat{\id{\{}u / z \id{\}}}$, to the following pair
of processes, $\lift{w}{y!(z)}$ and $w[ \lpquote y!(z) \rpquote ]$.

\begin{eqnarray}
	\lift{w}{y!(z)}\widehat{\id{\{}u / z \id{\}}}
		& = &
		\lift{w}{y!(u)} \nonumber\\
	w[ \lpquote y!(z) \rpquote ] \widehat{ \id{\{}u / z \id{\}} }
		& = &
		w[ \lpquote y!(z) \rpquote ] \nonumber
\end{eqnarray}

Because the body of the process between quotes is impervious to
substitution, we get radically different answers. In fact, by
examining the first process in an input context,
e.g. $x?(z).\lift{w}{y!(z)}$, we see that the process under the lift
operator may be shaped by prefixed inputs binding a name inside it. In
this sense, the lift operator will be seen as a way to dynamically
construct processes before reifying them as names.

Finally equipped with these standard features we can present the
dynamics of the calculus.

\subsubsection{Operational semantics} 

Finally, we introduce the computational dynamics. What marks these
algebras as distinct from other more traditionally studied algebraic
structures, e.g. vector spaces or polynomial rings, is the manner in
which dynamics is captured. In traditional structures, dynamics is typically
expressed through morphisms between such structures, as in linear maps
between vector spaces or morphisms between rings. In algebras
associated with the semantics of computation, the dynamics is
expressed as part of the algebraic structure itself, through a
reduction reduction relation typically denoted by $\red$. Below, we
give a recursive presentation of this relation for the calculus used
in the encoding.

$\red \subseteq \pi \times \pi$
$\red : \pi \to \mathcal{P}(\pi)$

\begin{mathpar}
  \inferrule* [lab=Comm] { \textsf{match}( x_{src}, x_{trgt} ) } { x_{trgt}?(y)P \; | \; x_{src}!\langle {Q} \rangle \red P\{\quotep{Q}/y}\} }
  \and \\
  \inferrule* [lab=Par] {{P} \red {P}'} {{{P} | {Q}} \red {{P}' | {Q}}}
  \and
  \inferrule* [lab=Equiv]{{{P} \scong {P}'} \andalso {{P}' \red {Q}'} \andalso {{Q}' \scong {Q}}}{{P} \red {Q}}
\end{mathpar}

\begin{eqnarray*}
  match_{\equiv} (\quotep{P},\quotep{Q}) & := & P \equiv Q \\
  match_{\dagger}(\quotep{P},\quotep{Q}) & := & \forall R. P|Q \red^{*} R => R \red^{*} 0 \\
  match_{K}(\quotep{P},\quotep{Q}) & := & K \mbox{ for some context } K
\end{eqnarray*}

$u?(x)P | u!\langle Q \rangle \red P\{\quotep{Q}/x\}$

%We write $\wred$ for $\red^*$, and $P\red$ if $\exists Q $ such that $ P \red Q$.
We write $P\red$ if $\exists Q $ such that $ P \red Q$ and $P\not\red$, otherwise.

\section{Replication}

As mentioned before, it is known that replication (and hence
recursion) can be implemented in a higher-order process algebra
\cite{SangiorgiWalker}. As our first example of calculation with the
machinery thus far presented we give the construction explicitly in
the {\rhoc}.

\begin{eqnarray}
	D_{x} & := & \prefix{x}{y}{(\binpar{\outputp{x}{y}}{@{y}})} \nonumber\\
	\bangp_{x}{P} & := & \binpar{{x}!\langle{\binpar{D_{x}}{P}}\rangle}{D_{x}} \nonumber
\end{eqnarray}

\begin{eqnarray}
	\bangp_{x}{P} & & \nonumber\\
	=
	& {x}!\langle{(\prefix{x}{y}{(\outputp{x}{y} | @{y})) | P}}\rangle 
	      | \prefix{x}{y}{(\outputp{x}{y} | @{y})} & \nonumber\\
	\red
	& (\outputp{x}{y} | @{y})\substn{\quotep{(\prefix{x}{y}{(@{y} | \outputp{x}{y})) | P}}}{y} & \nonumber\\
	=
	& \outputp{x}{\quotep{(\prefix{x}{y}{(\outputp{x}{y} | @{y})) | P}}}
	  | {(\prefix{x}{y}{(\outputp{x}{y} | @{y})) | P}} & \nonumber\\
	\red
	& \ldots & \nonumber\\
	\red^*
	& P | P | \ldots & \nonumber
\end{eqnarray}

Of course, this encoding, as an implementation, runs away, unfolding
$\bangp{P}$ eagerly. A lazier and more implementable replication
operator, restricted to input-guarded processes, may be obtained as follows.

\begin{eqnarray}
\bangp{\prefix{u}{v}{P}} 
	:= 
	\binpar{\lift{x}{\prefix{u}{v}{(\binpar{D(x)}{P})}}}{D(x)} \nonumber
\end{eqnarray}

\begin{remark}
  Note that the lazier definition still does not deal with summation
  or mixed summation (i.e. sums over input and output). The reader is
  invited to construct definitions of replication that deal with these
  features. 

  Further, the definitions are parameterized in a name, $x$. Can you,
  gentle reader, make a definition that eliminates this parameter and
  guarantees no accidental interaction between the replication
  machinery and the process being replicated -- i.e. no accidental
  sharing of names used by the process to get its work done and the
  name(s) used by the replication to effect copying. This latter
  revision of the definition of replication is crucial to obtaining
  the expected identity $!!P \sim !P$.
\end{remark}

\begin{remark}\label{rem:paradoxical_combinator}
  The reader familiar with the lambda calculus will have noticed the
  similarity between $D$ and the paradoxical combinator.

  [Ed. note: the existence of this seems to suggest we have to be more
  restrictive on the set of processes and names we admit if we are to
  support no-cloning.]
\end{remark}

\subsubsection{Bisimulation}

The computational dynamics gives rise to another kind of equivalence,
the equivalence of computational behavior. As previously mentioned
this is typically captured \emph{via} some form of bisimulation.

% The notion we use in this paper is weak barbed bisimulation
% \cite{milner91polyadicpi}.

The notion we use in this paper is derived from weak barbed
bisimulation \cite{milner91polyadicpi}. 

\begin{definition}
An \emph{observation relation}, $\downarrow_{\mathcal N}$, over a set
of names, $\mathcal N$, is the smallest relation satisfying the rules
below.

\infrule[Out-barb]{y \in {\mathcal N}, \; x \nameeq y}
		  {\outputp{x}{v} \downarrow_{\mathcal N} x}
\infrule[Par-barb]{\mbox{$P\downarrow_{\mathcal N} x$ or $Q\downarrow_{\mathcal N} x$}}
		  {\binpar{P}{Q} \downarrow_{\mathcal N} x}

We write $P \Downarrow_{\mathcal N} x$ if there is $Q$ such that 
$P \wred Q$ and $Q \downarrow_{\mathcal N} x$.
\end{definition}

\begin{definition}
%\label{def.bbisim}
An  ${\mathcal N}$-\emph{barbed bisimulation} over a set of names, ${\mathcal N}$, is a symmetric binary relation 
${\mathcal S}_{\mathcal N}$ between agents such that $P\rel{S}_{\mathcal N}Q$ implies:
\begin{enumerate}
\item If $P \red P'$ then $Q \wred Q'$ and $P'\rel{S}_{\mathcal N} Q'$.
\item If $P\downarrow_{\mathcal N} x$, then $Q\Downarrow_{\mathcal N} x$.
\end{enumerate}
$P$ is ${\mathcal N}$-barbed bisimilar to $Q$, written
$P \wbbisim_{\mathcal N} Q$, if $P \rel{S}_{\mathcal N} Q$ for some ${\mathcal N}$-barbed bisimulation ${\mathcal S}_{\mathcal N}$.
\end{definition}

$\mathcal{R} \subseteq \pi \times \pi$

$P \mathcal{R} Q => \forall P'. P \red P' \Rightarrow \exists Q'. Q \red Q', P' \mathcal{R} Q'$

$P \vdash x \Rightarrow Q \vdash x$

\begin{mathpar}
  \inferrule*[lab=Out-barb]{x \nameeq y}{{y}!\langle{Q}\rangle \vdash x}
  \and
  \inferrule*[lab=Par-barb]{\mbox{$P\vdash x$ or $Q\vdash x$}}{\binpar{P}{Q} \vdash x}
\end{mathpar}

\subsubsection{Contexts}

One of the principle advantages of computational calculi like the
$\pi$-calculus is a well-defined notion of context,
contextual-equivalence and a correlation between
contextual-equivalence and notions of bisimulation. The notion of
context allows the decomposition of a process into (sub-)process and
its syntactic environment, its context. Thus, a context may be
thought of as a process with a ``hole'' (written $\Box$) in it. The
application of a context $M$ to a process $P$, written $M[P]$, is
tantamount to filling the hole in $M$ with $P$. In this paper we do
not need the full weight of this theory, but do make use of the notion
of context in the proof the main theorem. 

\begin{mathpar}
  \inferrule* [lab=summation] {} {{M_{M},M_{N}} \bc \Box \;|\; x.M_{A} \;|\; M_{M}+M_{N}}
  \and
  \inferrule* [lab=agent] {} {{M_{A}} \bc (\vec{x})M_{P} \;| \; \clift{P_0,\ldots,M_{P},\ldots,P_N}}
  \and \\
  \inferrule* [lab=process] {} {{M_{P}} \bc M_{N} \;| \;P|M_{P} }
\end{mathpar} 

\begin{mathpar}
  \inferrule* [lab=sychronization] {} {M_{N} \bc \Box \;|\; x?M_{F} \;|\; x!M_{C}}
  \and
  \inferrule* [lab=abstraction] {} {{M_{F}} \bc (x)M_{P} }
  \and
  \inferrule* [lab=concretion] {} {{M_{C}} \bc \langle M_{P} \rangle }
  \and \\
  \inferrule* [lab=process] {} {{M_{P}} \bc M_{N} \;| \;P|M_{P} }
\end{mathpar}

\begin{definition}[contextual application] Given a context $M$, and
  process $P$, we define the \emph{contextual application}, $M[P] :=
  M\{P/\Box\}$. That is, the contextual application of M to P is the
  substitution of $P$ for $\Box$ in $M$.
\end{definition}

$\meaningof{-} : L \to \mathcal{P}(\pi)$

\begin{mathpar}
  \inferrule* [lab=collection] {} {\meaningof{true} = \pi, \and \meaningof{~E} = \pi \setminus \meaningof{E}, \and \meaningof{E_{1} \& E_{2}} = \meaningof{E_{1}} \cap \meaningof{E_{2}}}
\end{mathpar}

\begin{mathpar}
  \inferrule* [lab=structure] {} {\meaningof{0} = \{ P \in \pi | P \equiv 0 \}, \and \\ \meaningof{E_1 | E_2} = \{ P \in \pi | P \equiv P_{1} | P_{2}, P_{1} \in \meaningof{E_{1}}, P_{2} \in \meaningof{E_2}\} }
\end{mathpar}

\begin{mathpar}
 \inferrule* [lab=behavior] {} {\meaningof{\langle a?b \rangle E} = \{ P \in \pi | P \equiv Q | u?(y)P', \\ \and \\\\ \and \\ \;\;\; u \in \meaningof{a}, \forall z.P'\{z/y\} \in \meaningof{E\{z/b\}}\}, \and \\ \meaningof{a!E} = \{ P \in \pi | P \equiv Q | x!\langle P' \rangle, x \in \meaningof{a} P' \in \meaningof{E}\} }
\end{mathpar}

\begin{mathpar}
 \inferrule* [lab=nominal] {} {\meaningof{\quotep{E}} = \{ \quotep{P} \in \quotep{\pi} | P \in \meaningof{E} \}, \and \meaningof{\quotep{P}} = \{ \quotep{Q} \in \quotep{\pi} | P \equiv Q \} \and \\ \meaningof{@\quotep{E}} = \{ P \in \pi | P \equiv @x, x \in \meaningof{E} \}}
\end{mathpar}

\begin{eqnarray*}
  \\
  \meaningof{-} : TS \to ST
\end{eqnarray*}

\begin{eqnarray*}
  \\
  L : TS \to ST
\end{eqnarray*}

\begin{eqnarray*}
  \\
  P \models E \iff P \in \meaningof{E}
\end{eqnarray*}

\begin{eqnarray*}
  P \approx_{L} Q \iff \forall E \in L. P \models E \iff Q \models E
\end{eqnarray*}

\begin{eqnarray*}
  P \approx_{K} Q
\end{eqnarray*}

\begin{eqnarray*}
  P \approx Q
\end{eqnarray*}

$\approx_{K} = \approx = \approx_{L}$

\subsubsection{Contextual duality}

Note that contexts extend the quotation operation to a family of
operations from processes to names. Given a context, $M$, we can
define a \emph{nominal context}, $\quotep{M}$ by $\quotep{M}[P] :=
\quotep{M[P]}$. To foreshadow what is to come we observe that these
operations enjoy a duality with processes very much like the duality
between vectors and maps from vectors to scalars.

Further, because the calculus is essentially higher-order, we have a
correspondence between contexts and processes. More specifically,
given a name $x$ and a context $M$ we can construct $M^{*}_{x}$ such
that 

\begin{mathpar}
  M^{*}_{x} | \lift{x}{P} \red M[P]
\end{mathpar}

namely,

\begin{mathpar}
  M^{*}_{x} := x?(u).M[\dropn{u}]
\end{mathpar}

The dependence of $M^{*}_{x}$ on a name makes it an abstraction, 

\begin{mathpar}
  M^{*} := (x)x?(u).M[\dropn{u}]
\end{mathpar}

\subsection{Additional notation}

It will sometimes be convenient to denote the process a name
quotes. We already have the notation $x = \quotep{P}$, but it will be
convenient to introduce an alternate notation, $\procn{x}$, when we
want to emphasize the connection to the use of the name. Note that, by
virtue of name equivalence, $\quotep{\procn{x}} \nameeq x$; so, the
notation is consistent with previous definitions.

Further, because names have structure it is possible to effect
substitutions on the basis of that structure. This means we need to
upgrade our notation for substitutions, which we accomplish by
adapting comprehension notation. Thus,

\begin{mathpar}
  P\{ y / x : x \in S \}
\end{mathpar}

is interpreted to mean the process derived from P by replacing (in a
capture-avoiding manner) each occurrence of $x$ in $S$ by $y$. For example,

\begin{mathpar}
  P\{ \quotep{\procn{x}|\procn{x}} / x : x \in \freenames{P} \}
\end{mathpar}

will replace each (occurrence) of a free name $x$ in $P$ by
$\quotep{\procn{x}|\procn{x}}$.

Also, we will avail ourselves of the notation $x^{L}$ and $x^{R}$ to
denote injections of a name into disjoint copies of the name
space. There are numerous ways to accomplish this. One example can be
found in \cite{MeredithR05}. This notation overloads to vectors of
names: $\vec{x}^{\pi} := (x_{i}^{\pi} \; : \; 0 \leq i < |\vec{x}| )$ where $\pi \in \{L,R\}$.

We also use $P^{\Box} := P|\Box$.

In \cite{MeredithR05} an interpretation of the new operator is
given. It turns out that there are several possible interpretations
all enjoying the requisite algebraic properties of the operator (see
\cite{milner91polyadicpi}). We will therefore make liberal use of
$(\nu\; \vec{x})P$.

% subsection the_syntax_and_semantics_of_the_notation_system (end)   

\input{qm2pi.qmops} 

\input{qm2pi.sterngerlach} 

\input{qm2pi.metric} 

% section concurrent_process_calculi (end)

%\input{qm2pi.proofsketch}

% section proof sketch (end)

%\input{qm2pi.slviaknots} 

% section spatial logic via knots (end)

\input{qm2pi.conclusion}

% section conclusion (end)

%\input{qm2pi.dtcodes} 

% section wiring algorithm (end)

\input{qm2pi.ack} 

% section acknowledgments (end)

\newpage


\bibliographystyle{plain}   
\bibliography{../../biblios/main.bib}

\input{qm2pi.rhodetails}

\end{document}

 

\documentclass[12pt]{llncs}
%\documentclass{jktr}

\usepackage[pdftex]{hyperref}                   
\usepackage {listings}
\usepackage {mathpartir}
\usepackage{bcprules}
%\usepackage{listings}
                       
\usepackage{graphicx} 
%\usepackage[margins=2.5cm,nohead,nofoot]{geometry}
%\usepackage{geometry}
\usepackage{amsfonts}
\usepackage{amstext}
\usepackage{latexsym}
\usepackage{amssymb}
\usepackage{color}


%\include{myPreamble}
\include{qm2pi.local} 

%\ifpdf
%\usepackage[pdftex]{graphicx}
%\else
%\usepackage{graphicx}
%\fi

 % \ifpdf
%  \usepackage{pdfsync}
%  \if


%\title{Brief Article}
%\author{David F. Snyder}
%\author{L.G. Meredith}

%\address{Dept. of Math., Texas State University--San Marcos, San Marcos, TX 78666}
       
\pagestyle{empty}


\begin{document}

\lstset{language=[Objective]Caml,frame=shadowbox}

\input{qm2pi.front}

% section front matter (end)

\input{qm2pi.intro} 
 
% section introduction (end)

% \input{qm2pi.knotations} 

% section notation (end)

\input{qm2pi.process.calculi} 

% section concurrent_process_calculi_and_spatial_logics_ (end)
    
%\input{qm2pi.knots2pi} 

%\input{qm2pi.trefoil} 

%\input{qm2pi.mainthm} 

% subsection basic_interpretation (end)

%\input{qm2pi.rho.presentation} 
\subsection{The syntax and semantics of the notation system}\label{sub:the_syntax_and_semantics_of_the_notation_system} % (fold)

We now summarize a technical presentation of the calculus that
embodies our theory of dynamics. The typical presentation of such a
calculus follows the style of giving generators and relations on
them. The grammar, below, describing term constructors, freely
generates the set of processes, $\Proc$. This set is then quotiented
by a relation known as structural congruence and it is over this set
that the notion of dynamics is expressed. This presentation is
essentially that of \cite{MeredithR05} with the addition of
polyadicity and summation. For readability we have relegated some of
the technical subtleties to an appendix.

\subsubsection{Process grammar}\label{subsub:process_grammar}

\begin{mathpar}
  \inferrule* [lab=synchronization] {} {{M} \bc \pzero \;|\; x?F \;|\; x!C }
  \and
  \inferrule* [lab=abstraction] {} {{F} \bc (x)P}
  \and
  \inferrule* [lab=concretion] {} {{C} \bc \langle Q \rangle}
  \and
  \inferrule* [lab=process] {} {{P,Q} \bc M \;| \;P|Q \;|\; @{x}}
  \and
  \inferrule* [lab=name] {} {{x} \bc \quotep{P}}
\end{mathpar} 

Note that $\vec{x}$ (resp. $\vec{P}$) denotes a vector of names
(resp. processes) of length $|\vec{x}|$ (resp. $|\vec{P}|$). We adopt
the following useful abbreviations.

\begin{mathpar}
   x?(\vec{y}).P := x.(\vec{y})P \and  x\clift{\vec{P}} := x.\clift{\vec{P}}
   \and x!(y) := \lift{x}{\dropn{y}}
   \and \Pi_{i=0}^{n-1}P_i := P_0 | \ldots | P_{n-1}
\end{mathpar}

\subsubsection{Structural congruence}

\paragraph{Free and bound names and alpha-equivalence.} At the
core of structural equivalence is alpha-equivalence which identifies
process that are the same up to a change of variable. Formally, we
recognize the distinction between free and bound names. The free names
of a process, $\freenames{P}$, may be calculated recursively as
follows:

\begin{mathpar}
\freenames{\pzero} := \emptyset
  \and \\
  \freenames{x?(y).P} := \{ x \} \cup (\freenames{P} \setminus \{ y \})
  \and 
  \freenames{x!\langle P \rangle} := \{ x \} \cup \{ P \} 
  \and \\
  \freenames{P|Q} := \freenames{P} \cup \freenames{Q}
  \and \\
  \freenames{@{x}} := \{ x \}
\end{mathpar}

$\pi$
$\quotep{\pi}$

$\freenames{-} : \pi \to \mathcal{P}(\quotep{\pi})$

\begin{eqnarray*}
  \freenames{\pzero} & := & \emptyset \\
  \freenames{x?(y).P} & := & \{ x \} \cup (\freenames{P} \setminus \{ y \}) \\
  \freenames{x!\langle P \rangle} & := & \{ x \} \cup \{ P \} \\
  \freenames{P|Q} & := & \freenames{P} \cup \freenames{Q} \\
  \freenames{\dropn{x}} & := & \{ x \}
\end{eqnarray*}

The bound names of a process, $\boundnames{P}$, are those names occurring in $P$
that are not free. For example, in $x?(y).0$, the name $x$ is free, while $y$ is bound.

\begin{mathpar}
  \inferrule* [lab=monoidal-laws] {} { P|Q \equiv Q|P \and P|0 \equiv P \and P|(Q|R) \equiv (P|Q)|R }
\end{mathpar}

\begin{mathpar}
  \inferrule* [lab=alpha-equivalence] {} { (x)P \equiv (y)P\{y/x\} \and y \not\in \freenames{P} }
\end{mathpar}

\begin{definition}
Then two processes, $P,Q$, are alpha-equivalent if $P = Q\{\vec{y}/\vec{x}\}$ for
some $\vec{x} \in \boundnames{Q},\vec{y} \in \boundnames{P}$, where $Q\{\vec{y}/\vec{x}\}$
denotes the capture-avoiding substitution of $\vec{y}$ for $\vec{x}$ in $Q$.
\end{definition}

\begin{definition}
  The {\em structural congruence} \cite{SangiorgiWalker} , $\equiv$,
  between processes is the least congruence containing
  alpha-equivalence, satisfying the abelian monoid laws
  (associativity, commutativity and $\pzero$ as identity) for parallel
  composition $|$ and for summation $+$.
\end{definition}

\subsection{Name equivalence}

We take name equivalence, written $\nameeq$, to be the smallest
equivalence relation generated by the following rules.

\begin{mathpar}
\inferrule*[lab=Quote-drop]
{ }
{ \quotep{@{x}} \nameeq x }

\inferrule*[lab=Struct-equiv]
{ P \scong Q }
{ \quotep{P} \nameeq \quotep{Q} }
\end{mathpar}

The astute reader will have noticed that the mutual recursion of names
and processes imposes a mutual recursion on alpha-equivalence and
structural equivalence via name-equivalence. Fortunately, all of this
works out pleasantly and we may calculate in the natural way, free of
concern. The reader interested in the details is referred to the
appendix \ref{appendix:rho_details}.

\subsection{Substitution}

We use $\Proc$ for the set of processes, $\QProc$ for the set of
names, and $\id{\{}\vec{y} / \vec{x} \id{\}}$ to denote partial maps,
$s : \QProc \rightarrow \QProc$. A map, $s$ lifts, uniquely, to a map
on process terms, $\widehat{s} : \Proc \rightarrow \Proc$ by the
following equations.

\begin{mathpar}
  (0) \psubstp{Q}{P} := 0 \\
  (R \juxtap S) \psubstp{Q}{P}
  :=    
  (R)\psubstp{Q}{P} \juxtap (S) \psubstp{Q}{P} \\
  (x?(y).R) \psubstp{Q}{P}    
  :=    
  (x)\substp{Q}{P} (z)\concat( (R \psubstn{z}{y}) \psubstp{Q}{P} ) \\
  (\lift{x}{R}) \psubstp{Q}{P}  
  :=
  \lift{(x)\substp{Q}{P}}{ R \psubstp{Q}{P} } \\
%   (\dropn{x})  \psubstp{Q}{P}       
%   := 
%   \left\{ 
%     \begin{array}{ccc} 
%       \dropn{\quotep{Q}} & & x \nameeq \quotep{P} \\
%       \dropn{x} & & otherwise \\
%     \end{array}
%   \right. 
  (\dropn{x})  \psubstp{Q}{P}       
  := 
  \left\{ 
    \begin{array}{ccc} 
      Q & & x \nameeq \quotep{P} \\
      \dropn{x} & & otherwise \\
    \end{array}
  \right.
\end{mathpar}
 

where

\begin{eqnarray}
  (x)\id{\{} \lpquote Q \rpquote / \lpquote P \rpquote \id{\}}            = 
  \left\{ 
    \begin{array}{ccc}
      \lpquote Q \rpquote & & x \nameeq \lpquote P \rpquote \\
      x & & otherwise \\
    \end{array}
  \right. \nonumber
\end{eqnarray}

and $z$ is chosen distinct from $\quotep{P}$, $\quotep{Q}$, the free
names in $Q$, and all the names in $R$. Our $\alpha$-equivalence will
be built in the standard way from this substitution.

\begin{remark}\label{rem:no_self_referential_names}
  One consequence of these definitions is that $\forall P. \quotep{P}
  \not\in \freenames{P}$.
\end{remark}

\subsection{ Dynamic quote: an example }

Anticipating something of what's to come, consider applying the
substitution, $\widehat{\id{\{}u / z \id{\}}}$, to the following pair
of processes, $\lift{w}{y!(z)}$ and $w[ \lpquote y!(z) \rpquote ]$.

\begin{eqnarray}
	\lift{w}{y!(z)}\widehat{\id{\{}u / z \id{\}}}
		& = &
		\lift{w}{y!(u)} \nonumber\\
	w[ \lpquote y!(z) \rpquote ] \widehat{ \id{\{}u / z \id{\}} }
		& = &
		w[ \lpquote y!(z) \rpquote ] \nonumber
\end{eqnarray}

Because the body of the process between quotes is impervious to
substitution, we get radically different answers. In fact, by
examining the first process in an input context,
e.g. $x?(z).\lift{w}{y!(z)}$, we see that the process under the lift
operator may be shaped by prefixed inputs binding a name inside it. In
this sense, the lift operator will be seen as a way to dynamically
construct processes before reifying them as names.

Finally equipped with these standard features we can present the
dynamics of the calculus.

\subsubsection{Operational semantics} 

Finally, we introduce the computational dynamics. What marks these
algebras as distinct from other more traditionally studied algebraic
structures, e.g. vector spaces or polynomial rings, is the manner in
which dynamics is captured. In traditional structures, dynamics is typically
expressed through morphisms between such structures, as in linear maps
between vector spaces or morphisms between rings. In algebras
associated with the semantics of computation, the dynamics is
expressed as part of the algebraic structure itself, through a
reduction reduction relation typically denoted by $\red$. Below, we
give a recursive presentation of this relation for the calculus used
in the encoding.

$\red \subseteq \pi \times \pi$
$\red : \pi \to \mathcal{P}(\pi)$

\begin{mathpar}
  \inferrule* [lab=Comm] { \textsf{match}( x_{src}, x_{trgt} ) } { x_{trgt}?(y)P \; | \; x_{src}!\langle {Q} \rangle \red P\{\quotep{Q}/y}\} }
  \and \\
  \inferrule* [lab=Par] {{P} \red {P}'} {{{P} | {Q}} \red {{P}' | {Q}}}
  \and
  \inferrule* [lab=Equiv]{{{P} \scong {P}'} \andalso {{P}' \red {Q}'} \andalso {{Q}' \scong {Q}}}{{P} \red {Q}}
\end{mathpar}

\begin{eqnarray*}
  match_{\equiv} (\quotep{P},\quotep{Q}) & := & P \equiv Q \\
  match_{\dagger}(\quotep{P},\quotep{Q}) & := & \forall R. P|Q \red^{*} R => R \red^{*} 0 \\
  match_{K}(\quotep{P},\quotep{Q}) & := & K \mbox{ for some context } K
\end{eqnarray*}

$u?(x)P | u!\langle Q \rangle \red P\{\quotep{Q}/x\}$

%We write $\wred$ for $\red^*$, and $P\red$ if $\exists Q $ such that $ P \red Q$.
We write $P\red$ if $\exists Q $ such that $ P \red Q$ and $P\not\red$, otherwise.

\section{Replication}

As mentioned before, it is known that replication (and hence
recursion) can be implemented in a higher-order process algebra
\cite{SangiorgiWalker}. As our first example of calculation with the
machinery thus far presented we give the construction explicitly in
the {\rhoc}.

\begin{eqnarray}
	D_{x} & := & \prefix{x}{y}{(\binpar{\outputp{x}{y}}{@{y}})} \nonumber\\
	\bangp_{x}{P} & := & \binpar{{x}!\langle{\binpar{D_{x}}{P}}\rangle}{D_{x}} \nonumber
\end{eqnarray}

\begin{eqnarray}
	\bangp_{x}{P} & & \nonumber\\
	=
	& {x}!\langle{(\prefix{x}{y}{(\outputp{x}{y} | @{y})) | P}}\rangle 
	      | \prefix{x}{y}{(\outputp{x}{y} | @{y})} & \nonumber\\
	\red
	& (\outputp{x}{y} | @{y})\substn{\quotep{(\prefix{x}{y}{(@{y} | \outputp{x}{y})) | P}}}{y} & \nonumber\\
	=
	& \outputp{x}{\quotep{(\prefix{x}{y}{(\outputp{x}{y} | @{y})) | P}}}
	  | {(\prefix{x}{y}{(\outputp{x}{y} | @{y})) | P}} & \nonumber\\
	\red
	& \ldots & \nonumber\\
	\red^*
	& P | P | \ldots & \nonumber
\end{eqnarray}

Of course, this encoding, as an implementation, runs away, unfolding
$\bangp{P}$ eagerly. A lazier and more implementable replication
operator, restricted to input-guarded processes, may be obtained as follows.

\begin{eqnarray}
\bangp{\prefix{u}{v}{P}} 
	:= 
	\binpar{\lift{x}{\prefix{u}{v}{(\binpar{D(x)}{P})}}}{D(x)} \nonumber
\end{eqnarray}

\begin{remark}
  Note that the lazier definition still does not deal with summation
  or mixed summation (i.e. sums over input and output). The reader is
  invited to construct definitions of replication that deal with these
  features. 

  Further, the definitions are parameterized in a name, $x$. Can you,
  gentle reader, make a definition that eliminates this parameter and
  guarantees no accidental interaction between the replication
  machinery and the process being replicated -- i.e. no accidental
  sharing of names used by the process to get its work done and the
  name(s) used by the replication to effect copying. This latter
  revision of the definition of replication is crucial to obtaining
  the expected identity $!!P \sim !P$.
\end{remark}

\begin{remark}\label{rem:paradoxical_combinator}
  The reader familiar with the lambda calculus will have noticed the
  similarity between $D$ and the paradoxical combinator.

  [Ed. note: the existence of this seems to suggest we have to be more
  restrictive on the set of processes and names we admit if we are to
  support no-cloning.]
\end{remark}

\subsubsection{Bisimulation}

The computational dynamics gives rise to another kind of equivalence,
the equivalence of computational behavior. As previously mentioned
this is typically captured \emph{via} some form of bisimulation.

% The notion we use in this paper is weak barbed bisimulation
% \cite{milner91polyadicpi}.

The notion we use in this paper is derived from weak barbed
bisimulation \cite{milner91polyadicpi}. 

\begin{definition}
An \emph{observation relation}, $\downarrow_{\mathcal N}$, over a set
of names, $\mathcal N$, is the smallest relation satisfying the rules
below.

\infrule[Out-barb]{y \in {\mathcal N}, \; x \nameeq y}
		  {\outputp{x}{v} \downarrow_{\mathcal N} x}
\infrule[Par-barb]{\mbox{$P\downarrow_{\mathcal N} x$ or $Q\downarrow_{\mathcal N} x$}}
		  {\binpar{P}{Q} \downarrow_{\mathcal N} x}

We write $P \Downarrow_{\mathcal N} x$ if there is $Q$ such that 
$P \wred Q$ and $Q \downarrow_{\mathcal N} x$.
\end{definition}

\begin{definition}
%\label{def.bbisim}
An  ${\mathcal N}$-\emph{barbed bisimulation} over a set of names, ${\mathcal N}$, is a symmetric binary relation 
${\mathcal S}_{\mathcal N}$ between agents such that $P\rel{S}_{\mathcal N}Q$ implies:
\begin{enumerate}
\item If $P \red P'$ then $Q \wred Q'$ and $P'\rel{S}_{\mathcal N} Q'$.
\item If $P\downarrow_{\mathcal N} x$, then $Q\Downarrow_{\mathcal N} x$.
\end{enumerate}
$P$ is ${\mathcal N}$-barbed bisimilar to $Q$, written
$P \wbbisim_{\mathcal N} Q$, if $P \rel{S}_{\mathcal N} Q$ for some ${\mathcal N}$-barbed bisimulation ${\mathcal S}_{\mathcal N}$.
\end{definition}

$\mathcal{R} \subseteq \pi \times \pi$

$P \mathcal{R} Q => \forall P'. P \red P' \Rightarrow \exists Q'. Q \red Q', P' \mathcal{R} Q'$

$P \vdash x \Rightarrow Q \vdash x$

\begin{mathpar}
  \inferrule*[lab=Out-barb]{x \nameeq y}{{y}!\langle{Q}\rangle \vdash x}
  \and
  \inferrule*[lab=Par-barb]{\mbox{$P\vdash x$ or $Q\vdash x$}}{\binpar{P}{Q} \vdash x}
\end{mathpar}

\subsubsection{Contexts}

One of the principle advantages of computational calculi like the
$\pi$-calculus is a well-defined notion of context,
contextual-equivalence and a correlation between
contextual-equivalence and notions of bisimulation. The notion of
context allows the decomposition of a process into (sub-)process and
its syntactic environment, its context. Thus, a context may be
thought of as a process with a ``hole'' (written $\Box$) in it. The
application of a context $M$ to a process $P$, written $M[P]$, is
tantamount to filling the hole in $M$ with $P$. In this paper we do
not need the full weight of this theory, but do make use of the notion
of context in the proof the main theorem. 

\begin{mathpar}
  \inferrule* [lab=summation] {} {{M_{M},M_{N}} \bc \Box \;|\; x.M_{A} \;|\; M_{M}+M_{N}}
  \and
  \inferrule* [lab=agent] {} {{M_{A}} \bc (\vec{x})M_{P} \;| \; \clift{P_0,\ldots,M_{P},\ldots,P_N}}
  \and \\
  \inferrule* [lab=process] {} {{M_{P}} \bc M_{N} \;| \;P|M_{P} }
\end{mathpar} 

\begin{mathpar}
  \inferrule* [lab=sychronization] {} {M_{N} \bc \Box \;|\; x?M_{F} \;|\; x!M_{C}}
  \and
  \inferrule* [lab=abstraction] {} {{M_{F}} \bc (x)M_{P} }
  \and
  \inferrule* [lab=concretion] {} {{M_{C}} \bc \langle M_{P} \rangle }
  \and \\
  \inferrule* [lab=process] {} {{M_{P}} \bc M_{N} \;| \;P|M_{P} }
\end{mathpar}

\begin{definition}[contextual application] Given a context $M$, and
  process $P$, we define the \emph{contextual application}, $M[P] :=
  M\{P/\Box\}$. That is, the contextual application of M to P is the
  substitution of $P$ for $\Box$ in $M$.
\end{definition}

$\meaningof{-} : L \to \mathcal{P}(\pi)$

\begin{mathpar}
  \inferrule* [lab=collection] {} {\meaningof{true} = \pi, \and \meaningof{~E} = \pi \setminus \meaningof{E}, \and \meaningof{E_{1} \& E_{2}} = \meaningof{E_{1}} \cap \meaningof{E_{2}}}
\end{mathpar}

\begin{mathpar}
  \inferrule* [lab=structure] {} {\meaningof{0} = \{ P \in \pi | P \equiv 0 \}, \and \\ \meaningof{E_1 | E_2} = \{ P \in \pi | P \equiv P_{1} | P_{2}, P_{1} \in \meaningof{E_{1}}, P_{2} \in \meaningof{E_2}\} }
\end{mathpar}

\begin{mathpar}
 \inferrule* [lab=behavior] {} {\meaningof{\langle a?b \rangle E} = \{ P \in \pi | P \equiv Q | u?(y)P', \\ \and \\\\ \and \\ \;\;\; u \in \meaningof{a}, \forall z.P'\{z/y\} \in \meaningof{E\{z/b\}}\}, \and \\ \meaningof{a!E} = \{ P \in \pi | P \equiv Q | x!\langle P' \rangle, x \in \meaningof{a} P' \in \meaningof{E}\} }
\end{mathpar}

\begin{mathpar}
 \inferrule* [lab=nominal] {} {\meaningof{\quotep{E}} = \{ \quotep{P} \in \quotep{\pi} | P \in \meaningof{E} \}, \and \meaningof{\quotep{P}} = \{ \quotep{Q} \in \quotep{\pi} | P \equiv Q \} \and \\ \meaningof{@\quotep{E}} = \{ P \in \pi | P \equiv @x, x \in \meaningof{E} \}}
\end{mathpar}

\begin{eqnarray*}
  \\
  \meaningof{-} : TS \to ST
\end{eqnarray*}

\begin{eqnarray*}
  \\
  L : TS \to ST
\end{eqnarray*}

\begin{eqnarray*}
  \\
  P \models E \iff P \in \meaningof{E}
\end{eqnarray*}

\begin{eqnarray*}
  P \approx_{L} Q \iff \forall E \in L. P \models E \iff Q \models E
\end{eqnarray*}

\begin{eqnarray*}
  P \approx_{K} Q
\end{eqnarray*}

\begin{eqnarray*}
  P \approx Q
\end{eqnarray*}

$\approx_{K} = \approx = \approx_{L}$

\subsubsection{Contextual duality}

Note that contexts extend the quotation operation to a family of
operations from processes to names. Given a context, $M$, we can
define a \emph{nominal context}, $\quotep{M}$ by $\quotep{M}[P] :=
\quotep{M[P]}$. To foreshadow what is to come we observe that these
operations enjoy a duality with processes very much like the duality
between vectors and maps from vectors to scalars.

Further, because the calculus is essentially higher-order, we have a
correspondence between contexts and processes. More specifically,
given a name $x$ and a context $M$ we can construct $M^{*}_{x}$ such
that 

\begin{mathpar}
  M^{*}_{x} | \lift{x}{P} \red M[P]
\end{mathpar}

namely,

\begin{mathpar}
  M^{*}_{x} := x?(u).M[\dropn{u}]
\end{mathpar}

The dependence of $M^{*}_{x}$ on a name makes it an abstraction, 

\begin{mathpar}
  M^{*} := (x)x?(u).M[\dropn{u}]
\end{mathpar}

\subsection{Additional notation}

It will sometimes be convenient to denote the process a name
quotes. We already have the notation $x = \quotep{P}$, but it will be
convenient to introduce an alternate notation, $\procn{x}$, when we
want to emphasize the connection to the use of the name. Note that, by
virtue of name equivalence, $\quotep{\procn{x}} \nameeq x$; so, the
notation is consistent with previous definitions.

Further, because names have structure it is possible to effect
substitutions on the basis of that structure. This means we need to
upgrade our notation for substitutions, which we accomplish by
adapting comprehension notation. Thus,

\begin{mathpar}
  P\{ y / x : x \in S \}
\end{mathpar}

is interpreted to mean the process derived from P by replacing (in a
capture-avoiding manner) each occurrence of $x$ in $S$ by $y$. For example,

\begin{mathpar}
  P\{ \quotep{\procn{x}|\procn{x}} / x : x \in \freenames{P} \}
\end{mathpar}

will replace each (occurrence) of a free name $x$ in $P$ by
$\quotep{\procn{x}|\procn{x}}$.

Also, we will avail ourselves of the notation $x^{L}$ and $x^{R}$ to
denote injections of a name into disjoint copies of the name
space. There are numerous ways to accomplish this. One example can be
found in \cite{MeredithR05}. This notation overloads to vectors of
names: $\vec{x}^{\pi} := (x_{i}^{\pi} \; : \; 0 \leq i < |\vec{x}| )$ where $\pi \in \{L,R\}$.

We also use $P^{\Box} := P|\Box$.

In \cite{MeredithR05} an interpretation of the new operator is
given. It turns out that there are several possible interpretations
all enjoying the requisite algebraic properties of the operator (see
\cite{milner91polyadicpi}). We will therefore make liberal use of
$(\nu\; \vec{x})P$.

% subsection the_syntax_and_semantics_of_the_notation_system (end)   

\input{qm2pi.qmops} 

\input{qm2pi.sterngerlach} 

\input{qm2pi.metric} 

% section concurrent_process_calculi (end)

%\input{qm2pi.proofsketch}

% section proof sketch (end)

%\input{qm2pi.slviaknots} 

% section spatial logic via knots (end)

\input{qm2pi.conclusion}

% section conclusion (end)

%\input{qm2pi.dtcodes} 

% section wiring algorithm (end)

\input{qm2pi.ack} 

% section acknowledgments (end)

\newpage


\bibliographystyle{plain}   
\bibliography{../../biblios/main.bib}

\input{qm2pi.rhodetails}

\end{document}

 

% section concurrent_process_calculi (end)

%\documentclass[12pt]{llncs}
%\documentclass{jktr}

\usepackage[pdftex]{hyperref}                   
\usepackage {listings}
\usepackage {mathpartir}
\usepackage{bcprules}
%\usepackage{listings}
                       
\usepackage{graphicx} 
%\usepackage[margins=2.5cm,nohead,nofoot]{geometry}
%\usepackage{geometry}
\usepackage{amsfonts}
\usepackage{amstext}
\usepackage{latexsym}
\usepackage{amssymb}
\usepackage{color}


%\include{myPreamble}
\include{qm2pi.local} 

%\ifpdf
%\usepackage[pdftex]{graphicx}
%\else
%\usepackage{graphicx}
%\fi

 % \ifpdf
%  \usepackage{pdfsync}
%  \if


%\title{Brief Article}
%\author{David F. Snyder}
%\author{L.G. Meredith}

%\address{Dept. of Math., Texas State University--San Marcos, San Marcos, TX 78666}
       
\pagestyle{empty}


\begin{document}

\lstset{language=[Objective]Caml,frame=shadowbox}

\input{qm2pi.front}

% section front matter (end)

\input{qm2pi.intro} 
 
% section introduction (end)

% \input{qm2pi.knotations} 

% section notation (end)

\input{qm2pi.process.calculi} 

% section concurrent_process_calculi_and_spatial_logics_ (end)
    
%\input{qm2pi.knots2pi} 

%\input{qm2pi.trefoil} 

%\input{qm2pi.mainthm} 

% subsection basic_interpretation (end)

%\input{qm2pi.rho.presentation} 
\subsection{The syntax and semantics of the notation system}\label{sub:the_syntax_and_semantics_of_the_notation_system} % (fold)

We now summarize a technical presentation of the calculus that
embodies our theory of dynamics. The typical presentation of such a
calculus follows the style of giving generators and relations on
them. The grammar, below, describing term constructors, freely
generates the set of processes, $\Proc$. This set is then quotiented
by a relation known as structural congruence and it is over this set
that the notion of dynamics is expressed. This presentation is
essentially that of \cite{MeredithR05} with the addition of
polyadicity and summation. For readability we have relegated some of
the technical subtleties to an appendix.

\subsubsection{Process grammar}\label{subsub:process_grammar}

\begin{mathpar}
  \inferrule* [lab=synchronization] {} {{M} \bc \pzero \;|\; x?F \;|\; x!C }
  \and
  \inferrule* [lab=abstraction] {} {{F} \bc (x)P}
  \and
  \inferrule* [lab=concretion] {} {{C} \bc \langle Q \rangle}
  \and
  \inferrule* [lab=process] {} {{P,Q} \bc M \;| \;P|Q \;|\; @{x}}
  \and
  \inferrule* [lab=name] {} {{x} \bc \quotep{P}}
\end{mathpar} 

Note that $\vec{x}$ (resp. $\vec{P}$) denotes a vector of names
(resp. processes) of length $|\vec{x}|$ (resp. $|\vec{P}|$). We adopt
the following useful abbreviations.

\begin{mathpar}
   x?(\vec{y}).P := x.(\vec{y})P \and  x\clift{\vec{P}} := x.\clift{\vec{P}}
   \and x!(y) := \lift{x}{\dropn{y}}
   \and \Pi_{i=0}^{n-1}P_i := P_0 | \ldots | P_{n-1}
\end{mathpar}

\subsubsection{Structural congruence}

\paragraph{Free and bound names and alpha-equivalence.} At the
core of structural equivalence is alpha-equivalence which identifies
process that are the same up to a change of variable. Formally, we
recognize the distinction between free and bound names. The free names
of a process, $\freenames{P}$, may be calculated recursively as
follows:

\begin{mathpar}
\freenames{\pzero} := \emptyset
  \and \\
  \freenames{x?(y).P} := \{ x \} \cup (\freenames{P} \setminus \{ y \})
  \and 
  \freenames{x!\langle P \rangle} := \{ x \} \cup \{ P \} 
  \and \\
  \freenames{P|Q} := \freenames{P} \cup \freenames{Q}
  \and \\
  \freenames{@{x}} := \{ x \}
\end{mathpar}

$\pi$
$\quotep{\pi}$

$\freenames{-} : \pi \to \mathcal{P}(\quotep{\pi})$

\begin{eqnarray*}
  \freenames{\pzero} & := & \emptyset \\
  \freenames{x?(y).P} & := & \{ x \} \cup (\freenames{P} \setminus \{ y \}) \\
  \freenames{x!\langle P \rangle} & := & \{ x \} \cup \{ P \} \\
  \freenames{P|Q} & := & \freenames{P} \cup \freenames{Q} \\
  \freenames{\dropn{x}} & := & \{ x \}
\end{eqnarray*}

The bound names of a process, $\boundnames{P}$, are those names occurring in $P$
that are not free. For example, in $x?(y).0$, the name $x$ is free, while $y$ is bound.

\begin{mathpar}
  \inferrule* [lab=monoidal-laws] {} { P|Q \equiv Q|P \and P|0 \equiv P \and P|(Q|R) \equiv (P|Q)|R }
\end{mathpar}

\begin{mathpar}
  \inferrule* [lab=alpha-equivalence] {} { (x)P \equiv (y)P\{y/x\} \and y \not\in \freenames{P} }
\end{mathpar}

\begin{definition}
Then two processes, $P,Q$, are alpha-equivalent if $P = Q\{\vec{y}/\vec{x}\}$ for
some $\vec{x} \in \boundnames{Q},\vec{y} \in \boundnames{P}$, where $Q\{\vec{y}/\vec{x}\}$
denotes the capture-avoiding substitution of $\vec{y}$ for $\vec{x}$ in $Q$.
\end{definition}

\begin{definition}
  The {\em structural congruence} \cite{SangiorgiWalker} , $\equiv$,
  between processes is the least congruence containing
  alpha-equivalence, satisfying the abelian monoid laws
  (associativity, commutativity and $\pzero$ as identity) for parallel
  composition $|$ and for summation $+$.
\end{definition}

\subsection{Name equivalence}

We take name equivalence, written $\nameeq$, to be the smallest
equivalence relation generated by the following rules.

\begin{mathpar}
\inferrule*[lab=Quote-drop]
{ }
{ \quotep{@{x}} \nameeq x }

\inferrule*[lab=Struct-equiv]
{ P \scong Q }
{ \quotep{P} \nameeq \quotep{Q} }
\end{mathpar}

The astute reader will have noticed that the mutual recursion of names
and processes imposes a mutual recursion on alpha-equivalence and
structural equivalence via name-equivalence. Fortunately, all of this
works out pleasantly and we may calculate in the natural way, free of
concern. The reader interested in the details is referred to the
appendix \ref{appendix:rho_details}.

\subsection{Substitution}

We use $\Proc$ for the set of processes, $\QProc$ for the set of
names, and $\id{\{}\vec{y} / \vec{x} \id{\}}$ to denote partial maps,
$s : \QProc \rightarrow \QProc$. A map, $s$ lifts, uniquely, to a map
on process terms, $\widehat{s} : \Proc \rightarrow \Proc$ by the
following equations.

\begin{mathpar}
  (0) \psubstp{Q}{P} := 0 \\
  (R \juxtap S) \psubstp{Q}{P}
  :=    
  (R)\psubstp{Q}{P} \juxtap (S) \psubstp{Q}{P} \\
  (x?(y).R) \psubstp{Q}{P}    
  :=    
  (x)\substp{Q}{P} (z)\concat( (R \psubstn{z}{y}) \psubstp{Q}{P} ) \\
  (\lift{x}{R}) \psubstp{Q}{P}  
  :=
  \lift{(x)\substp{Q}{P}}{ R \psubstp{Q}{P} } \\
%   (\dropn{x})  \psubstp{Q}{P}       
%   := 
%   \left\{ 
%     \begin{array}{ccc} 
%       \dropn{\quotep{Q}} & & x \nameeq \quotep{P} \\
%       \dropn{x} & & otherwise \\
%     \end{array}
%   \right. 
  (\dropn{x})  \psubstp{Q}{P}       
  := 
  \left\{ 
    \begin{array}{ccc} 
      Q & & x \nameeq \quotep{P} \\
      \dropn{x} & & otherwise \\
    \end{array}
  \right.
\end{mathpar}
 

where

\begin{eqnarray}
  (x)\id{\{} \lpquote Q \rpquote / \lpquote P \rpquote \id{\}}            = 
  \left\{ 
    \begin{array}{ccc}
      \lpquote Q \rpquote & & x \nameeq \lpquote P \rpquote \\
      x & & otherwise \\
    \end{array}
  \right. \nonumber
\end{eqnarray}

and $z$ is chosen distinct from $\quotep{P}$, $\quotep{Q}$, the free
names in $Q$, and all the names in $R$. Our $\alpha$-equivalence will
be built in the standard way from this substitution.

\begin{remark}\label{rem:no_self_referential_names}
  One consequence of these definitions is that $\forall P. \quotep{P}
  \not\in \freenames{P}$.
\end{remark}

\subsection{ Dynamic quote: an example }

Anticipating something of what's to come, consider applying the
substitution, $\widehat{\id{\{}u / z \id{\}}}$, to the following pair
of processes, $\lift{w}{y!(z)}$ and $w[ \lpquote y!(z) \rpquote ]$.

\begin{eqnarray}
	\lift{w}{y!(z)}\widehat{\id{\{}u / z \id{\}}}
		& = &
		\lift{w}{y!(u)} \nonumber\\
	w[ \lpquote y!(z) \rpquote ] \widehat{ \id{\{}u / z \id{\}} }
		& = &
		w[ \lpquote y!(z) \rpquote ] \nonumber
\end{eqnarray}

Because the body of the process between quotes is impervious to
substitution, we get radically different answers. In fact, by
examining the first process in an input context,
e.g. $x?(z).\lift{w}{y!(z)}$, we see that the process under the lift
operator may be shaped by prefixed inputs binding a name inside it. In
this sense, the lift operator will be seen as a way to dynamically
construct processes before reifying them as names.

Finally equipped with these standard features we can present the
dynamics of the calculus.

\subsubsection{Operational semantics} 

Finally, we introduce the computational dynamics. What marks these
algebras as distinct from other more traditionally studied algebraic
structures, e.g. vector spaces or polynomial rings, is the manner in
which dynamics is captured. In traditional structures, dynamics is typically
expressed through morphisms between such structures, as in linear maps
between vector spaces or morphisms between rings. In algebras
associated with the semantics of computation, the dynamics is
expressed as part of the algebraic structure itself, through a
reduction reduction relation typically denoted by $\red$. Below, we
give a recursive presentation of this relation for the calculus used
in the encoding.

$\red \subseteq \pi \times \pi$
$\red : \pi \to \mathcal{P}(\pi)$

\begin{mathpar}
  \inferrule* [lab=Comm] { \textsf{match}( x_{src}, x_{trgt} ) } { x_{trgt}?(y)P \; | \; x_{src}!\langle {Q} \rangle \red P\{\quotep{Q}/y}\} }
  \and \\
  \inferrule* [lab=Par] {{P} \red {P}'} {{{P} | {Q}} \red {{P}' | {Q}}}
  \and
  \inferrule* [lab=Equiv]{{{P} \scong {P}'} \andalso {{P}' \red {Q}'} \andalso {{Q}' \scong {Q}}}{{P} \red {Q}}
\end{mathpar}

\begin{eqnarray*}
  match_{\equiv} (\quotep{P},\quotep{Q}) & := & P \equiv Q \\
  match_{\dagger}(\quotep{P},\quotep{Q}) & := & \forall R. P|Q \red^{*} R => R \red^{*} 0 \\
  match_{K}(\quotep{P},\quotep{Q}) & := & K \mbox{ for some context } K
\end{eqnarray*}

$u?(x)P | u!\langle Q \rangle \red P\{\quotep{Q}/x\}$

%We write $\wred$ for $\red^*$, and $P\red$ if $\exists Q $ such that $ P \red Q$.
We write $P\red$ if $\exists Q $ such that $ P \red Q$ and $P\not\red$, otherwise.

\section{Replication}

As mentioned before, it is known that replication (and hence
recursion) can be implemented in a higher-order process algebra
\cite{SangiorgiWalker}. As our first example of calculation with the
machinery thus far presented we give the construction explicitly in
the {\rhoc}.

\begin{eqnarray}
	D_{x} & := & \prefix{x}{y}{(\binpar{\outputp{x}{y}}{@{y}})} \nonumber\\
	\bangp_{x}{P} & := & \binpar{{x}!\langle{\binpar{D_{x}}{P}}\rangle}{D_{x}} \nonumber
\end{eqnarray}

\begin{eqnarray}
	\bangp_{x}{P} & & \nonumber\\
	=
	& {x}!\langle{(\prefix{x}{y}{(\outputp{x}{y} | @{y})) | P}}\rangle 
	      | \prefix{x}{y}{(\outputp{x}{y} | @{y})} & \nonumber\\
	\red
	& (\outputp{x}{y} | @{y})\substn{\quotep{(\prefix{x}{y}{(@{y} | \outputp{x}{y})) | P}}}{y} & \nonumber\\
	=
	& \outputp{x}{\quotep{(\prefix{x}{y}{(\outputp{x}{y} | @{y})) | P}}}
	  | {(\prefix{x}{y}{(\outputp{x}{y} | @{y})) | P}} & \nonumber\\
	\red
	& \ldots & \nonumber\\
	\red^*
	& P | P | \ldots & \nonumber
\end{eqnarray}

Of course, this encoding, as an implementation, runs away, unfolding
$\bangp{P}$ eagerly. A lazier and more implementable replication
operator, restricted to input-guarded processes, may be obtained as follows.

\begin{eqnarray}
\bangp{\prefix{u}{v}{P}} 
	:= 
	\binpar{\lift{x}{\prefix{u}{v}{(\binpar{D(x)}{P})}}}{D(x)} \nonumber
\end{eqnarray}

\begin{remark}
  Note that the lazier definition still does not deal with summation
  or mixed summation (i.e. sums over input and output). The reader is
  invited to construct definitions of replication that deal with these
  features. 

  Further, the definitions are parameterized in a name, $x$. Can you,
  gentle reader, make a definition that eliminates this parameter and
  guarantees no accidental interaction between the replication
  machinery and the process being replicated -- i.e. no accidental
  sharing of names used by the process to get its work done and the
  name(s) used by the replication to effect copying. This latter
  revision of the definition of replication is crucial to obtaining
  the expected identity $!!P \sim !P$.
\end{remark}

\begin{remark}\label{rem:paradoxical_combinator}
  The reader familiar with the lambda calculus will have noticed the
  similarity between $D$ and the paradoxical combinator.

  [Ed. note: the existence of this seems to suggest we have to be more
  restrictive on the set of processes and names we admit if we are to
  support no-cloning.]
\end{remark}

\subsubsection{Bisimulation}

The computational dynamics gives rise to another kind of equivalence,
the equivalence of computational behavior. As previously mentioned
this is typically captured \emph{via} some form of bisimulation.

% The notion we use in this paper is weak barbed bisimulation
% \cite{milner91polyadicpi}.

The notion we use in this paper is derived from weak barbed
bisimulation \cite{milner91polyadicpi}. 

\begin{definition}
An \emph{observation relation}, $\downarrow_{\mathcal N}$, over a set
of names, $\mathcal N$, is the smallest relation satisfying the rules
below.

\infrule[Out-barb]{y \in {\mathcal N}, \; x \nameeq y}
		  {\outputp{x}{v} \downarrow_{\mathcal N} x}
\infrule[Par-barb]{\mbox{$P\downarrow_{\mathcal N} x$ or $Q\downarrow_{\mathcal N} x$}}
		  {\binpar{P}{Q} \downarrow_{\mathcal N} x}

We write $P \Downarrow_{\mathcal N} x$ if there is $Q$ such that 
$P \wred Q$ and $Q \downarrow_{\mathcal N} x$.
\end{definition}

\begin{definition}
%\label{def.bbisim}
An  ${\mathcal N}$-\emph{barbed bisimulation} over a set of names, ${\mathcal N}$, is a symmetric binary relation 
${\mathcal S}_{\mathcal N}$ between agents such that $P\rel{S}_{\mathcal N}Q$ implies:
\begin{enumerate}
\item If $P \red P'$ then $Q \wred Q'$ and $P'\rel{S}_{\mathcal N} Q'$.
\item If $P\downarrow_{\mathcal N} x$, then $Q\Downarrow_{\mathcal N} x$.
\end{enumerate}
$P$ is ${\mathcal N}$-barbed bisimilar to $Q$, written
$P \wbbisim_{\mathcal N} Q$, if $P \rel{S}_{\mathcal N} Q$ for some ${\mathcal N}$-barbed bisimulation ${\mathcal S}_{\mathcal N}$.
\end{definition}

$\mathcal{R} \subseteq \pi \times \pi$

$P \mathcal{R} Q => \forall P'. P \red P' \Rightarrow \exists Q'. Q \red Q', P' \mathcal{R} Q'$

$P \vdash x \Rightarrow Q \vdash x$

\begin{mathpar}
  \inferrule*[lab=Out-barb]{x \nameeq y}{{y}!\langle{Q}\rangle \vdash x}
  \and
  \inferrule*[lab=Par-barb]{\mbox{$P\vdash x$ or $Q\vdash x$}}{\binpar{P}{Q} \vdash x}
\end{mathpar}

\subsubsection{Contexts}

One of the principle advantages of computational calculi like the
$\pi$-calculus is a well-defined notion of context,
contextual-equivalence and a correlation between
contextual-equivalence and notions of bisimulation. The notion of
context allows the decomposition of a process into (sub-)process and
its syntactic environment, its context. Thus, a context may be
thought of as a process with a ``hole'' (written $\Box$) in it. The
application of a context $M$ to a process $P$, written $M[P]$, is
tantamount to filling the hole in $M$ with $P$. In this paper we do
not need the full weight of this theory, but do make use of the notion
of context in the proof the main theorem. 

\begin{mathpar}
  \inferrule* [lab=summation] {} {{M_{M},M_{N}} \bc \Box \;|\; x.M_{A} \;|\; M_{M}+M_{N}}
  \and
  \inferrule* [lab=agent] {} {{M_{A}} \bc (\vec{x})M_{P} \;| \; \clift{P_0,\ldots,M_{P},\ldots,P_N}}
  \and \\
  \inferrule* [lab=process] {} {{M_{P}} \bc M_{N} \;| \;P|M_{P} }
\end{mathpar} 

\begin{mathpar}
  \inferrule* [lab=sychronization] {} {M_{N} \bc \Box \;|\; x?M_{F} \;|\; x!M_{C}}
  \and
  \inferrule* [lab=abstraction] {} {{M_{F}} \bc (x)M_{P} }
  \and
  \inferrule* [lab=concretion] {} {{M_{C}} \bc \langle M_{P} \rangle }
  \and \\
  \inferrule* [lab=process] {} {{M_{P}} \bc M_{N} \;| \;P|M_{P} }
\end{mathpar}

\begin{definition}[contextual application] Given a context $M$, and
  process $P$, we define the \emph{contextual application}, $M[P] :=
  M\{P/\Box\}$. That is, the contextual application of M to P is the
  substitution of $P$ for $\Box$ in $M$.
\end{definition}

$\meaningof{-} : L \to \mathcal{P}(\pi)$

\begin{mathpar}
  \inferrule* [lab=collection] {} {\meaningof{true} = \pi, \and \meaningof{~E} = \pi \setminus \meaningof{E}, \and \meaningof{E_{1} \& E_{2}} = \meaningof{E_{1}} \cap \meaningof{E_{2}}}
\end{mathpar}

\begin{mathpar}
  \inferrule* [lab=structure] {} {\meaningof{0} = \{ P \in \pi | P \equiv 0 \}, \and \\ \meaningof{E_1 | E_2} = \{ P \in \pi | P \equiv P_{1} | P_{2}, P_{1} \in \meaningof{E_{1}}, P_{2} \in \meaningof{E_2}\} }
\end{mathpar}

\begin{mathpar}
 \inferrule* [lab=behavior] {} {\meaningof{\langle a?b \rangle E} = \{ P \in \pi | P \equiv Q | u?(y)P', \\ \and \\\\ \and \\ \;\;\; u \in \meaningof{a}, \forall z.P'\{z/y\} \in \meaningof{E\{z/b\}}\}, \and \\ \meaningof{a!E} = \{ P \in \pi | P \equiv Q | x!\langle P' \rangle, x \in \meaningof{a} P' \in \meaningof{E}\} }
\end{mathpar}

\begin{mathpar}
 \inferrule* [lab=nominal] {} {\meaningof{\quotep{E}} = \{ \quotep{P} \in \quotep{\pi} | P \in \meaningof{E} \}, \and \meaningof{\quotep{P}} = \{ \quotep{Q} \in \quotep{\pi} | P \equiv Q \} \and \\ \meaningof{@\quotep{E}} = \{ P \in \pi | P \equiv @x, x \in \meaningof{E} \}}
\end{mathpar}

\begin{eqnarray*}
  \\
  \meaningof{-} : TS \to ST
\end{eqnarray*}

\begin{eqnarray*}
  \\
  L : TS \to ST
\end{eqnarray*}

\begin{eqnarray*}
  \\
  P \models E \iff P \in \meaningof{E}
\end{eqnarray*}

\begin{eqnarray*}
  P \approx_{L} Q \iff \forall E \in L. P \models E \iff Q \models E
\end{eqnarray*}

\begin{eqnarray*}
  P \approx_{K} Q
\end{eqnarray*}

\begin{eqnarray*}
  P \approx Q
\end{eqnarray*}

$\approx_{K} = \approx = \approx_{L}$

\subsubsection{Contextual duality}

Note that contexts extend the quotation operation to a family of
operations from processes to names. Given a context, $M$, we can
define a \emph{nominal context}, $\quotep{M}$ by $\quotep{M}[P] :=
\quotep{M[P]}$. To foreshadow what is to come we observe that these
operations enjoy a duality with processes very much like the duality
between vectors and maps from vectors to scalars.

Further, because the calculus is essentially higher-order, we have a
correspondence between contexts and processes. More specifically,
given a name $x$ and a context $M$ we can construct $M^{*}_{x}$ such
that 

\begin{mathpar}
  M^{*}_{x} | \lift{x}{P} \red M[P]
\end{mathpar}

namely,

\begin{mathpar}
  M^{*}_{x} := x?(u).M[\dropn{u}]
\end{mathpar}

The dependence of $M^{*}_{x}$ on a name makes it an abstraction, 

\begin{mathpar}
  M^{*} := (x)x?(u).M[\dropn{u}]
\end{mathpar}

\subsection{Additional notation}

It will sometimes be convenient to denote the process a name
quotes. We already have the notation $x = \quotep{P}$, but it will be
convenient to introduce an alternate notation, $\procn{x}$, when we
want to emphasize the connection to the use of the name. Note that, by
virtue of name equivalence, $\quotep{\procn{x}} \nameeq x$; so, the
notation is consistent with previous definitions.

Further, because names have structure it is possible to effect
substitutions on the basis of that structure. This means we need to
upgrade our notation for substitutions, which we accomplish by
adapting comprehension notation. Thus,

\begin{mathpar}
  P\{ y / x : x \in S \}
\end{mathpar}

is interpreted to mean the process derived from P by replacing (in a
capture-avoiding manner) each occurrence of $x$ in $S$ by $y$. For example,

\begin{mathpar}
  P\{ \quotep{\procn{x}|\procn{x}} / x : x \in \freenames{P} \}
\end{mathpar}

will replace each (occurrence) of a free name $x$ in $P$ by
$\quotep{\procn{x}|\procn{x}}$.

Also, we will avail ourselves of the notation $x^{L}$ and $x^{R}$ to
denote injections of a name into disjoint copies of the name
space. There are numerous ways to accomplish this. One example can be
found in \cite{MeredithR05}. This notation overloads to vectors of
names: $\vec{x}^{\pi} := (x_{i}^{\pi} \; : \; 0 \leq i < |\vec{x}| )$ where $\pi \in \{L,R\}$.

We also use $P^{\Box} := P|\Box$.

In \cite{MeredithR05} an interpretation of the new operator is
given. It turns out that there are several possible interpretations
all enjoying the requisite algebraic properties of the operator (see
\cite{milner91polyadicpi}). We will therefore make liberal use of
$(\nu\; \vec{x})P$.

% subsection the_syntax_and_semantics_of_the_notation_system (end)   

\input{qm2pi.qmops} 

\input{qm2pi.sterngerlach} 

\input{qm2pi.metric} 

% section concurrent_process_calculi (end)

%\input{qm2pi.proofsketch}

% section proof sketch (end)

%\input{qm2pi.slviaknots} 

% section spatial logic via knots (end)

\input{qm2pi.conclusion}

% section conclusion (end)

%\input{qm2pi.dtcodes} 

% section wiring algorithm (end)

\input{qm2pi.ack} 

% section acknowledgments (end)

\newpage


\bibliographystyle{plain}   
\bibliography{../../biblios/main.bib}

\input{qm2pi.rhodetails}

\end{document}



% section proof sketch (end)

%\section{Unlikely characters: spatial logic for
  knots}\label{sub:characteristic_formulae} % (fold)

Associated to the mobile process calculi are a family of logics known
as the Hennessy-Milner logics. These logics typically enjoy a
semantics interpreting formulae as sets of processes that when
factored through the encoding outlined above allows an identification
of classes of knots with logical formulae. In the context of this
encoding the sub-family known as the spatial logics \cite{CairesC03}
\cite{CairesC04} \cite{Caires04} are of particular interest providing
several important features for expressing and reasoning about
properties (i.e. classes) of knots. We hint here at how this may be done.

%\begin{description}
%\item [structural connectives] 
\subsubsection{Structural connectives} The spatial logics enjoy
structural connectives corresponding, at the logical level, to the
parallel composition ($P | Q$) and new name ($(\nu \; x)P$)
connectives for processes. As illustrated in the examples below, these
connectives are extremely expressive given the shape of our encoding.
%\item [decideable satisfaction]

\subsubsection{Decideable satisfaction}
In \cite{Caires04} the satisfaction relation is shown to be decideable
for a rich class of processes. It further turns out that the image of
the our encoding is a proper subset of that class. This result
provides the basis for an algorithm by which to search for knots
enjoying a given property.
%\item [characteristic formulae]

\subsubsection{Characteristic formulae}
In the same paper \cite{Caires04} , Caires presents a means of calculating
characteristic formulae, selecting equivalence classes of processes
up to a pre--specified depth limit on the support set of names. Composed with our
encoding, this characteristic formula can be used to select
characteristic formulae for knots.
%\end{description}

\subsubsection{Spatial logic formulae}

The grammar below (segmented for comprehension) summarizes the syntax
of spatial logic formulae. We employ illustrative examples in the
sequel to provide an intuitive understanding of their meaning
referring the reader to \cite{Caires04} for a more detailed explication
of the semantics.

\begin{mathpar}
  \inferrule* [lab=boolean] {} {{A,B} \bc T \;|\; \neg A \;|\; A \wedge B \;|\; \eta = \eta'}
  \and
  \inferrule* [lab=spatial] {} {|\; \pzero \;|\; A | B \;|\; x \text{\textregistered} A \;|\; \forall x . A \;|\;  H x . A}
  \and
  \inferrule* [lab=behavioral] {} {|\; \alpha . A}
  \and 
  \inferrule* [lab=recursion] {} {|\; X(\vec{u}) \;|\; \mu X(\vec{u}) . A}
  \and
  \inferrule* [lab=action] {} {\alpha \bc \langle x?(\vec{y}) \rangle \;|\; \langle x!(\vec{y}) \rangle \;|\; \langle \tau \rangle}
  \and 
  \inferrule* [lab=name] {} {\eta \bc x \;|\; \tau}
\end{mathpar} 

% subsection characteristic_formulae (end)   	 

\subsection{Example formulae}\label{sub:example_formulae_} % (fold)

\subsubsection{Crossing as formula.}
% 
% \begin{align*}
%   \frac{d}{dx} \sin x &= \cos x 
%   & \frac{d}{dx} e^x &= e^x \\
%   \frac{d}{dx} \cos x &= - \sin x 
%   & \frac{d}{dx} \log x &= \frac{1}{x} \\
% \end{align*} 

\begin{align*}
 \mu C(x_{0},x_{1},y_{0},y_{1},u).&(\langle x_{0}?(z) \rangle(\langle u! \rangle\langle y_{1}!z \rangle C(x_{0},x_{1},y_{0},y_{1},u)) & \\
  & \wedge \langle y_{1}?(z) \rangle (\langle u! \rangle \langle x_{0}!z \rangle C(x_{0},x_{1},y_{0},y_{1},u)) & \\
  & \wedge \langle x_{1}?(z) \rangle (\langle u? \rangle \langle y_{0}!z \rangle C(x_{0},x_{1},y_{0},y_{1},u)) & \\
  & \wedge \langle y_{0}?(z) \rangle (\langle u? \rangle \langle x_{1}!z \rangle C(x_{0},x_{1},y_{0},y_{1},u))) &
\end{align*}

The lexicographical similarity between the shape of this formulae and
the shape of definition of the process representing a crossing reveals
the intuitive meaning of this formulae. It describes the capabilities
of a process that has the right to represent a crossing. For example
it picks out processes that may perform an input on the port $x_0$ in
its initial menu of capabilities. What differentiates the formula
from the process, however, is that the crossing process is the
smallest candidate to satisfy the formula. Infinitely many other
processes -- with internal behavior hidden behind this interface, so
to speak -- also satisfy this formula. Even this simple formula,
then, can be seen to open a new view onto knots, providing a
computational interpretation of \emph{virtual} knots.

Note that this formula is derived by hand. A similar formula can be
derived by employing Caires' calculation of characteristic formula
\cite{Caires04} to the process representing a crossing. In light of
this discussion, we let
$\meaningof{C}_{\phi}(x0,x1,y0,y1,u)$ denote a formula specifying the
dynamics we wish to capture of a crossing. To guarantee we preserve
the shape of the interface and minimal semantics we demand that
$\meaningof{C}_{\phi}(x0,x1,y0,y1,u) \Rightarrow
\textbf{C}(x0,x1,y0,y1,u)$ where $\textbf{C}(x0,x1,y0,y1,u)$ denotes
the formula above.
                            
\subsubsection{Crossing number constraints.}
The moral content of the context lemma (Lemma \ref{context}) is that the notion of
``locality'' in the Reidemeister moves is effectively captured by the
parallel composition operator of the process calculus. This intuition
extends through the logic. Given a formula,
$\meaningof{C}_{\phi}(x0,x1,y0,y1,u)$, we can use the structural
connectives to specify constraints on crossing numbers, such as at
least $n$ crossings, or exactly $n$ crossings.
\begin{mathpar}
  \inferrule* [lab=at-least-n] {} { K^{\geq n}_{\phi}(\vec{xs},\vec{ys}) := \Pi_{i=0}^{n-1} Hu . \meaningof{C}_{\phi}(xs_i,ys_i,u) | T }
  \and 
  \inferrule* [lab=exactly-n] {} { K^{= n}_{\phi}(\vec{xs},\vec{ys}) := \Pi_{i=0}^{n-1} Hu . \meaningof{C}_{\phi}(xs_i,ys_i,u) | \neg (\forall x_0,y_0,x_1,y_1,u . \meaningof{C}_{\phi}(x_0,y_0,x_1,y_1,u) | T) }
\end{mathpar}

To round out this section, recall that the encoding of an $n$-crossing
knot decomposes into a parallel composition of $n$ \emph{copies} of a
crossing process together with a wiring harness. To specify different
knot classes with the same crossing number amounts to specifying
logical constraints on the wiring harness. In the interest of space,
we defer examples to a forthcoming paper. Suffice it to say that both
the conditions ``alternating knot'' and ``contains the tangle
corresponding to 5/3'' are expressible. For example, it is possible to
calculate the characteristic formula of a process corresponding to the
tangle 5/3 and conjoin it into the classifying formula via the
composition connective of the logic.

Finally, we wish to observe that it is entirely within reason to
contemplate a more domain-specific version of spatial logic tailored
to the shape of processes in the image of the encoding. Such a
domain-specific logic would have a better claim to the title formal
language of knot properties.

% subsection example_formulae_ (end)

% section knots_as_processes (end) 

% section spatial logic via knots (end)

\section{Conclusions and future work}

\paragraph{Testing physical space}
You, gentle reader, may wonder why of all the theorems to be proved
given this set up we pick the one above. In some sense it's hardly
central to quantum mechanics. We see it as central in the sense that
it firmly establishes a notion of physical space arising from a notion
of the equivalence of behavior. Relating bisimulation to a metric is a
big step forward, but one is faced with interpreting the relationship
of that metric space to something more physical. Quantum mechanical
notions of ``physical'' space are still far from intuitive, but by
relating this idea of distance as testing to calculations that predict
physical circumstances we are making a not insignificant step forward
toward an understanding of the physical space we inhabit as
essentially dynamic.

\paragraph{Effectivity and simulation}
One of the observations we have yet to make is that the entire program
spelled out here is effective. We have built various interpreters for
the reflective calculus at work in this interpretation. In principle,
then, we can simulate quantum mechanics on a computer. The place where
the simulation may lose fidelity is the infinitely branching summation
for the annihilator.

In this connection i also want to point out that the evaluation style
calculation of the inner product puts the non-determinism of the
summation right at the heart of measurement. This suggests that
Milner's original reduction-based formulation of the dynamics of his
calculi in terms of sums was not just notationally suggestive of a
notion of measure-and-continue but captured some significant part of
the physics.

\paragraph{Quantum continuations}
In light of this last observation i want to point out that the
predominant account of quantum mechanics is missing a key aspect of a
truly compositional story of the physical situation. In a real lab,
when a measurement is made the observation can be made to feed into
another device that then makes another measurement conditioned on the
results of the first. This means that after the superposition was
collapsed the entire experimental set up remained in
superposition. While QM offers a means of writing this down it doesn't
quite line up well with the well-trodden formulation of computation
and continuation that we see so succinctly expressed in Milner's
calculi. This suggests that there might be advantages to this account
of dynamics waiting to be explored.

\paragraph{Quantum logic}
In this connection, we also note that by virtue of having the
Hennessy-Milner construction, we can pull the construction through the
interpretation of QM. This gives us a natural candidate for a quantum
logic that enjoys an extremely tight connection with it's domain of
interpretation, making the construction much less ad hoc (rather it is
the image of functor!).

\paragraph{Quantum probabiity}
i have questions about the basis of the interpretation of inner
product as probability amplitude. In particular, using which
axiomatization of probability theory does the notion of probability
amplitude earn the right to be so dubbed? In other words, where is the
proof that the operation for calculating a probability amplitude (and
then squaring) satisfies the axioms of what it means to calculate a
probability? Even if such a proof exists (i have yet to find it in the
literature), i wonder if it might not be possible to turn things on
their heads. Can we view the calculation of the probability amplitude
as an axiomatization of probability? If so, then the definition we
give for calculating probability amplitude may provide the basis for
an \emph{effective} theory of probability.

\paragraph{Quantum vs ``biological'' information}
Finally, i want to conclude with a more philosophical observation. At
a recent workshop in which QM was a predominant topic i noticed
something about quantum information. The speaker was giving a riveting
discussion of axiomatic QM and showing how properties of ``no
cloning'' and ``no deleting'' emerged as consequences of the
axiomatization. Theorems of this form are necessary to give us a sense
of confidence that our axioms characterize the physical theory. What
struck me, though, was that if quantum information is neither erasable
nor replicable it is markedly different from \emph{life}. Two of the
things we know about life is that

\begin{itemize}
  \item it ends;
  \item to gain some measure of persistence, to transcend it's
    finitude it is imminently copyable.
\end{itemize}

Both of these qualities are summarized succinctly in the aphorism: all
flesh is grass. For me these two kinds of ``information'' -- call them
quantum and biological -- are end points on a spectrum of strategies
for persistence. At one end, we have those curious entities that enjoy
uniqueness and permanence; at the other, we have those who in the face
of a certain end and an uncertain present make a go of passing
something on. To me one of the more remarkable aspects of the latter
strategy is that in the presence of noise (and certain features of
copying) we get a kind of dynamism, a chance for improvement against a
given persistent condition.

% subsection other_calculi_other_bisimulations_and_geometry_as_behavior (end)




% section conclusion (end)

%\documentclass[12pt]{llncs}
%\documentclass{jktr}

\usepackage[pdftex]{hyperref}                   
\usepackage {listings}
\usepackage {mathpartir}
\usepackage{bcprules}
%\usepackage{listings}
                       
\usepackage{graphicx} 
%\usepackage[margins=2.5cm,nohead,nofoot]{geometry}
%\usepackage{geometry}
\usepackage{amsfonts}
\usepackage{amstext}
\usepackage{latexsym}
\usepackage{amssymb}
\usepackage{color}


%\include{myPreamble}
\include{qm2pi.local} 

%\ifpdf
%\usepackage[pdftex]{graphicx}
%\else
%\usepackage{graphicx}
%\fi

 % \ifpdf
%  \usepackage{pdfsync}
%  \if


%\title{Brief Article}
%\author{David F. Snyder}
%\author{L.G. Meredith}

%\address{Dept. of Math., Texas State University--San Marcos, San Marcos, TX 78666}
       
\pagestyle{empty}


\begin{document}

\lstset{language=[Objective]Caml,frame=shadowbox}

\input{qm2pi.front}

% section front matter (end)

\input{qm2pi.intro} 
 
% section introduction (end)

% \input{qm2pi.knotations} 

% section notation (end)

\input{qm2pi.process.calculi} 

% section concurrent_process_calculi_and_spatial_logics_ (end)
    
%\input{qm2pi.knots2pi} 

%\input{qm2pi.trefoil} 

%\input{qm2pi.mainthm} 

% subsection basic_interpretation (end)

%\input{qm2pi.rho.presentation} 
\subsection{The syntax and semantics of the notation system}\label{sub:the_syntax_and_semantics_of_the_notation_system} % (fold)

We now summarize a technical presentation of the calculus that
embodies our theory of dynamics. The typical presentation of such a
calculus follows the style of giving generators and relations on
them. The grammar, below, describing term constructors, freely
generates the set of processes, $\Proc$. This set is then quotiented
by a relation known as structural congruence and it is over this set
that the notion of dynamics is expressed. This presentation is
essentially that of \cite{MeredithR05} with the addition of
polyadicity and summation. For readability we have relegated some of
the technical subtleties to an appendix.

\subsubsection{Process grammar}\label{subsub:process_grammar}

\begin{mathpar}
  \inferrule* [lab=synchronization] {} {{M} \bc \pzero \;|\; x?F \;|\; x!C }
  \and
  \inferrule* [lab=abstraction] {} {{F} \bc (x)P}
  \and
  \inferrule* [lab=concretion] {} {{C} \bc \langle Q \rangle}
  \and
  \inferrule* [lab=process] {} {{P,Q} \bc M \;| \;P|Q \;|\; @{x}}
  \and
  \inferrule* [lab=name] {} {{x} \bc \quotep{P}}
\end{mathpar} 

Note that $\vec{x}$ (resp. $\vec{P}$) denotes a vector of names
(resp. processes) of length $|\vec{x}|$ (resp. $|\vec{P}|$). We adopt
the following useful abbreviations.

\begin{mathpar}
   x?(\vec{y}).P := x.(\vec{y})P \and  x\clift{\vec{P}} := x.\clift{\vec{P}}
   \and x!(y) := \lift{x}{\dropn{y}}
   \and \Pi_{i=0}^{n-1}P_i := P_0 | \ldots | P_{n-1}
\end{mathpar}

\subsubsection{Structural congruence}

\paragraph{Free and bound names and alpha-equivalence.} At the
core of structural equivalence is alpha-equivalence which identifies
process that are the same up to a change of variable. Formally, we
recognize the distinction between free and bound names. The free names
of a process, $\freenames{P}$, may be calculated recursively as
follows:

\begin{mathpar}
\freenames{\pzero} := \emptyset
  \and \\
  \freenames{x?(y).P} := \{ x \} \cup (\freenames{P} \setminus \{ y \})
  \and 
  \freenames{x!\langle P \rangle} := \{ x \} \cup \{ P \} 
  \and \\
  \freenames{P|Q} := \freenames{P} \cup \freenames{Q}
  \and \\
  \freenames{@{x}} := \{ x \}
\end{mathpar}

$\pi$
$\quotep{\pi}$

$\freenames{-} : \pi \to \mathcal{P}(\quotep{\pi})$

\begin{eqnarray*}
  \freenames{\pzero} & := & \emptyset \\
  \freenames{x?(y).P} & := & \{ x \} \cup (\freenames{P} \setminus \{ y \}) \\
  \freenames{x!\langle P \rangle} & := & \{ x \} \cup \{ P \} \\
  \freenames{P|Q} & := & \freenames{P} \cup \freenames{Q} \\
  \freenames{\dropn{x}} & := & \{ x \}
\end{eqnarray*}

The bound names of a process, $\boundnames{P}$, are those names occurring in $P$
that are not free. For example, in $x?(y).0$, the name $x$ is free, while $y$ is bound.

\begin{mathpar}
  \inferrule* [lab=monoidal-laws] {} { P|Q \equiv Q|P \and P|0 \equiv P \and P|(Q|R) \equiv (P|Q)|R }
\end{mathpar}

\begin{mathpar}
  \inferrule* [lab=alpha-equivalence] {} { (x)P \equiv (y)P\{y/x\} \and y \not\in \freenames{P} }
\end{mathpar}

\begin{definition}
Then two processes, $P,Q$, are alpha-equivalent if $P = Q\{\vec{y}/\vec{x}\}$ for
some $\vec{x} \in \boundnames{Q},\vec{y} \in \boundnames{P}$, where $Q\{\vec{y}/\vec{x}\}$
denotes the capture-avoiding substitution of $\vec{y}$ for $\vec{x}$ in $Q$.
\end{definition}

\begin{definition}
  The {\em structural congruence} \cite{SangiorgiWalker} , $\equiv$,
  between processes is the least congruence containing
  alpha-equivalence, satisfying the abelian monoid laws
  (associativity, commutativity and $\pzero$ as identity) for parallel
  composition $|$ and for summation $+$.
\end{definition}

\subsection{Name equivalence}

We take name equivalence, written $\nameeq$, to be the smallest
equivalence relation generated by the following rules.

\begin{mathpar}
\inferrule*[lab=Quote-drop]
{ }
{ \quotep{@{x}} \nameeq x }

\inferrule*[lab=Struct-equiv]
{ P \scong Q }
{ \quotep{P} \nameeq \quotep{Q} }
\end{mathpar}

The astute reader will have noticed that the mutual recursion of names
and processes imposes a mutual recursion on alpha-equivalence and
structural equivalence via name-equivalence. Fortunately, all of this
works out pleasantly and we may calculate in the natural way, free of
concern. The reader interested in the details is referred to the
appendix \ref{appendix:rho_details}.

\subsection{Substitution}

We use $\Proc$ for the set of processes, $\QProc$ for the set of
names, and $\id{\{}\vec{y} / \vec{x} \id{\}}$ to denote partial maps,
$s : \QProc \rightarrow \QProc$. A map, $s$ lifts, uniquely, to a map
on process terms, $\widehat{s} : \Proc \rightarrow \Proc$ by the
following equations.

\begin{mathpar}
  (0) \psubstp{Q}{P} := 0 \\
  (R \juxtap S) \psubstp{Q}{P}
  :=    
  (R)\psubstp{Q}{P} \juxtap (S) \psubstp{Q}{P} \\
  (x?(y).R) \psubstp{Q}{P}    
  :=    
  (x)\substp{Q}{P} (z)\concat( (R \psubstn{z}{y}) \psubstp{Q}{P} ) \\
  (\lift{x}{R}) \psubstp{Q}{P}  
  :=
  \lift{(x)\substp{Q}{P}}{ R \psubstp{Q}{P} } \\
%   (\dropn{x})  \psubstp{Q}{P}       
%   := 
%   \left\{ 
%     \begin{array}{ccc} 
%       \dropn{\quotep{Q}} & & x \nameeq \quotep{P} \\
%       \dropn{x} & & otherwise \\
%     \end{array}
%   \right. 
  (\dropn{x})  \psubstp{Q}{P}       
  := 
  \left\{ 
    \begin{array}{ccc} 
      Q & & x \nameeq \quotep{P} \\
      \dropn{x} & & otherwise \\
    \end{array}
  \right.
\end{mathpar}
 

where

\begin{eqnarray}
  (x)\id{\{} \lpquote Q \rpquote / \lpquote P \rpquote \id{\}}            = 
  \left\{ 
    \begin{array}{ccc}
      \lpquote Q \rpquote & & x \nameeq \lpquote P \rpquote \\
      x & & otherwise \\
    \end{array}
  \right. \nonumber
\end{eqnarray}

and $z$ is chosen distinct from $\quotep{P}$, $\quotep{Q}$, the free
names in $Q$, and all the names in $R$. Our $\alpha$-equivalence will
be built in the standard way from this substitution.

\begin{remark}\label{rem:no_self_referential_names}
  One consequence of these definitions is that $\forall P. \quotep{P}
  \not\in \freenames{P}$.
\end{remark}

\subsection{ Dynamic quote: an example }

Anticipating something of what's to come, consider applying the
substitution, $\widehat{\id{\{}u / z \id{\}}}$, to the following pair
of processes, $\lift{w}{y!(z)}$ and $w[ \lpquote y!(z) \rpquote ]$.

\begin{eqnarray}
	\lift{w}{y!(z)}\widehat{\id{\{}u / z \id{\}}}
		& = &
		\lift{w}{y!(u)} \nonumber\\
	w[ \lpquote y!(z) \rpquote ] \widehat{ \id{\{}u / z \id{\}} }
		& = &
		w[ \lpquote y!(z) \rpquote ] \nonumber
\end{eqnarray}

Because the body of the process between quotes is impervious to
substitution, we get radically different answers. In fact, by
examining the first process in an input context,
e.g. $x?(z).\lift{w}{y!(z)}$, we see that the process under the lift
operator may be shaped by prefixed inputs binding a name inside it. In
this sense, the lift operator will be seen as a way to dynamically
construct processes before reifying them as names.

Finally equipped with these standard features we can present the
dynamics of the calculus.

\subsubsection{Operational semantics} 

Finally, we introduce the computational dynamics. What marks these
algebras as distinct from other more traditionally studied algebraic
structures, e.g. vector spaces or polynomial rings, is the manner in
which dynamics is captured. In traditional structures, dynamics is typically
expressed through morphisms between such structures, as in linear maps
between vector spaces or morphisms between rings. In algebras
associated with the semantics of computation, the dynamics is
expressed as part of the algebraic structure itself, through a
reduction reduction relation typically denoted by $\red$. Below, we
give a recursive presentation of this relation for the calculus used
in the encoding.

$\red \subseteq \pi \times \pi$
$\red : \pi \to \mathcal{P}(\pi)$

\begin{mathpar}
  \inferrule* [lab=Comm] { \textsf{match}( x_{src}, x_{trgt} ) } { x_{trgt}?(y)P \; | \; x_{src}!\langle {Q} \rangle \red P\{\quotep{Q}/y}\} }
  \and \\
  \inferrule* [lab=Par] {{P} \red {P}'} {{{P} | {Q}} \red {{P}' | {Q}}}
  \and
  \inferrule* [lab=Equiv]{{{P} \scong {P}'} \andalso {{P}' \red {Q}'} \andalso {{Q}' \scong {Q}}}{{P} \red {Q}}
\end{mathpar}

\begin{eqnarray*}
  match_{\equiv} (\quotep{P},\quotep{Q}) & := & P \equiv Q \\
  match_{\dagger}(\quotep{P},\quotep{Q}) & := & \forall R. P|Q \red^{*} R => R \red^{*} 0 \\
  match_{K}(\quotep{P},\quotep{Q}) & := & K \mbox{ for some context } K
\end{eqnarray*}

$u?(x)P | u!\langle Q \rangle \red P\{\quotep{Q}/x\}$

%We write $\wred$ for $\red^*$, and $P\red$ if $\exists Q $ such that $ P \red Q$.
We write $P\red$ if $\exists Q $ such that $ P \red Q$ and $P\not\red$, otherwise.

\section{Replication}

As mentioned before, it is known that replication (and hence
recursion) can be implemented in a higher-order process algebra
\cite{SangiorgiWalker}. As our first example of calculation with the
machinery thus far presented we give the construction explicitly in
the {\rhoc}.

\begin{eqnarray}
	D_{x} & := & \prefix{x}{y}{(\binpar{\outputp{x}{y}}{@{y}})} \nonumber\\
	\bangp_{x}{P} & := & \binpar{{x}!\langle{\binpar{D_{x}}{P}}\rangle}{D_{x}} \nonumber
\end{eqnarray}

\begin{eqnarray}
	\bangp_{x}{P} & & \nonumber\\
	=
	& {x}!\langle{(\prefix{x}{y}{(\outputp{x}{y} | @{y})) | P}}\rangle 
	      | \prefix{x}{y}{(\outputp{x}{y} | @{y})} & \nonumber\\
	\red
	& (\outputp{x}{y} | @{y})\substn{\quotep{(\prefix{x}{y}{(@{y} | \outputp{x}{y})) | P}}}{y} & \nonumber\\
	=
	& \outputp{x}{\quotep{(\prefix{x}{y}{(\outputp{x}{y} | @{y})) | P}}}
	  | {(\prefix{x}{y}{(\outputp{x}{y} | @{y})) | P}} & \nonumber\\
	\red
	& \ldots & \nonumber\\
	\red^*
	& P | P | \ldots & \nonumber
\end{eqnarray}

Of course, this encoding, as an implementation, runs away, unfolding
$\bangp{P}$ eagerly. A lazier and more implementable replication
operator, restricted to input-guarded processes, may be obtained as follows.

\begin{eqnarray}
\bangp{\prefix{u}{v}{P}} 
	:= 
	\binpar{\lift{x}{\prefix{u}{v}{(\binpar{D(x)}{P})}}}{D(x)} \nonumber
\end{eqnarray}

\begin{remark}
  Note that the lazier definition still does not deal with summation
  or mixed summation (i.e. sums over input and output). The reader is
  invited to construct definitions of replication that deal with these
  features. 

  Further, the definitions are parameterized in a name, $x$. Can you,
  gentle reader, make a definition that eliminates this parameter and
  guarantees no accidental interaction between the replication
  machinery and the process being replicated -- i.e. no accidental
  sharing of names used by the process to get its work done and the
  name(s) used by the replication to effect copying. This latter
  revision of the definition of replication is crucial to obtaining
  the expected identity $!!P \sim !P$.
\end{remark}

\begin{remark}\label{rem:paradoxical_combinator}
  The reader familiar with the lambda calculus will have noticed the
  similarity between $D$ and the paradoxical combinator.

  [Ed. note: the existence of this seems to suggest we have to be more
  restrictive on the set of processes and names we admit if we are to
  support no-cloning.]
\end{remark}

\subsubsection{Bisimulation}

The computational dynamics gives rise to another kind of equivalence,
the equivalence of computational behavior. As previously mentioned
this is typically captured \emph{via} some form of bisimulation.

% The notion we use in this paper is weak barbed bisimulation
% \cite{milner91polyadicpi}.

The notion we use in this paper is derived from weak barbed
bisimulation \cite{milner91polyadicpi}. 

\begin{definition}
An \emph{observation relation}, $\downarrow_{\mathcal N}$, over a set
of names, $\mathcal N$, is the smallest relation satisfying the rules
below.

\infrule[Out-barb]{y \in {\mathcal N}, \; x \nameeq y}
		  {\outputp{x}{v} \downarrow_{\mathcal N} x}
\infrule[Par-barb]{\mbox{$P\downarrow_{\mathcal N} x$ or $Q\downarrow_{\mathcal N} x$}}
		  {\binpar{P}{Q} \downarrow_{\mathcal N} x}

We write $P \Downarrow_{\mathcal N} x$ if there is $Q$ such that 
$P \wred Q$ and $Q \downarrow_{\mathcal N} x$.
\end{definition}

\begin{definition}
%\label{def.bbisim}
An  ${\mathcal N}$-\emph{barbed bisimulation} over a set of names, ${\mathcal N}$, is a symmetric binary relation 
${\mathcal S}_{\mathcal N}$ between agents such that $P\rel{S}_{\mathcal N}Q$ implies:
\begin{enumerate}
\item If $P \red P'$ then $Q \wred Q'$ and $P'\rel{S}_{\mathcal N} Q'$.
\item If $P\downarrow_{\mathcal N} x$, then $Q\Downarrow_{\mathcal N} x$.
\end{enumerate}
$P$ is ${\mathcal N}$-barbed bisimilar to $Q$, written
$P \wbbisim_{\mathcal N} Q$, if $P \rel{S}_{\mathcal N} Q$ for some ${\mathcal N}$-barbed bisimulation ${\mathcal S}_{\mathcal N}$.
\end{definition}

$\mathcal{R} \subseteq \pi \times \pi$

$P \mathcal{R} Q => \forall P'. P \red P' \Rightarrow \exists Q'. Q \red Q', P' \mathcal{R} Q'$

$P \vdash x \Rightarrow Q \vdash x$

\begin{mathpar}
  \inferrule*[lab=Out-barb]{x \nameeq y}{{y}!\langle{Q}\rangle \vdash x}
  \and
  \inferrule*[lab=Par-barb]{\mbox{$P\vdash x$ or $Q\vdash x$}}{\binpar{P}{Q} \vdash x}
\end{mathpar}

\subsubsection{Contexts}

One of the principle advantages of computational calculi like the
$\pi$-calculus is a well-defined notion of context,
contextual-equivalence and a correlation between
contextual-equivalence and notions of bisimulation. The notion of
context allows the decomposition of a process into (sub-)process and
its syntactic environment, its context. Thus, a context may be
thought of as a process with a ``hole'' (written $\Box$) in it. The
application of a context $M$ to a process $P$, written $M[P]$, is
tantamount to filling the hole in $M$ with $P$. In this paper we do
not need the full weight of this theory, but do make use of the notion
of context in the proof the main theorem. 

\begin{mathpar}
  \inferrule* [lab=summation] {} {{M_{M},M_{N}} \bc \Box \;|\; x.M_{A} \;|\; M_{M}+M_{N}}
  \and
  \inferrule* [lab=agent] {} {{M_{A}} \bc (\vec{x})M_{P} \;| \; \clift{P_0,\ldots,M_{P},\ldots,P_N}}
  \and \\
  \inferrule* [lab=process] {} {{M_{P}} \bc M_{N} \;| \;P|M_{P} }
\end{mathpar} 

\begin{mathpar}
  \inferrule* [lab=sychronization] {} {M_{N} \bc \Box \;|\; x?M_{F} \;|\; x!M_{C}}
  \and
  \inferrule* [lab=abstraction] {} {{M_{F}} \bc (x)M_{P} }
  \and
  \inferrule* [lab=concretion] {} {{M_{C}} \bc \langle M_{P} \rangle }
  \and \\
  \inferrule* [lab=process] {} {{M_{P}} \bc M_{N} \;| \;P|M_{P} }
\end{mathpar}

\begin{definition}[contextual application] Given a context $M$, and
  process $P$, we define the \emph{contextual application}, $M[P] :=
  M\{P/\Box\}$. That is, the contextual application of M to P is the
  substitution of $P$ for $\Box$ in $M$.
\end{definition}

$\meaningof{-} : L \to \mathcal{P}(\pi)$

\begin{mathpar}
  \inferrule* [lab=collection] {} {\meaningof{true} = \pi, \and \meaningof{~E} = \pi \setminus \meaningof{E}, \and \meaningof{E_{1} \& E_{2}} = \meaningof{E_{1}} \cap \meaningof{E_{2}}}
\end{mathpar}

\begin{mathpar}
  \inferrule* [lab=structure] {} {\meaningof{0} = \{ P \in \pi | P \equiv 0 \}, \and \\ \meaningof{E_1 | E_2} = \{ P \in \pi | P \equiv P_{1} | P_{2}, P_{1} \in \meaningof{E_{1}}, P_{2} \in \meaningof{E_2}\} }
\end{mathpar}

\begin{mathpar}
 \inferrule* [lab=behavior] {} {\meaningof{\langle a?b \rangle E} = \{ P \in \pi | P \equiv Q | u?(y)P', \\ \and \\\\ \and \\ \;\;\; u \in \meaningof{a}, \forall z.P'\{z/y\} \in \meaningof{E\{z/b\}}\}, \and \\ \meaningof{a!E} = \{ P \in \pi | P \equiv Q | x!\langle P' \rangle, x \in \meaningof{a} P' \in \meaningof{E}\} }
\end{mathpar}

\begin{mathpar}
 \inferrule* [lab=nominal] {} {\meaningof{\quotep{E}} = \{ \quotep{P} \in \quotep{\pi} | P \in \meaningof{E} \}, \and \meaningof{\quotep{P}} = \{ \quotep{Q} \in \quotep{\pi} | P \equiv Q \} \and \\ \meaningof{@\quotep{E}} = \{ P \in \pi | P \equiv @x, x \in \meaningof{E} \}}
\end{mathpar}

\begin{eqnarray*}
  \\
  \meaningof{-} : TS \to ST
\end{eqnarray*}

\begin{eqnarray*}
  \\
  L : TS \to ST
\end{eqnarray*}

\begin{eqnarray*}
  \\
  P \models E \iff P \in \meaningof{E}
\end{eqnarray*}

\begin{eqnarray*}
  P \approx_{L} Q \iff \forall E \in L. P \models E \iff Q \models E
\end{eqnarray*}

\begin{eqnarray*}
  P \approx_{K} Q
\end{eqnarray*}

\begin{eqnarray*}
  P \approx Q
\end{eqnarray*}

$\approx_{K} = \approx = \approx_{L}$

\subsubsection{Contextual duality}

Note that contexts extend the quotation operation to a family of
operations from processes to names. Given a context, $M$, we can
define a \emph{nominal context}, $\quotep{M}$ by $\quotep{M}[P] :=
\quotep{M[P]}$. To foreshadow what is to come we observe that these
operations enjoy a duality with processes very much like the duality
between vectors and maps from vectors to scalars.

Further, because the calculus is essentially higher-order, we have a
correspondence between contexts and processes. More specifically,
given a name $x$ and a context $M$ we can construct $M^{*}_{x}$ such
that 

\begin{mathpar}
  M^{*}_{x} | \lift{x}{P} \red M[P]
\end{mathpar}

namely,

\begin{mathpar}
  M^{*}_{x} := x?(u).M[\dropn{u}]
\end{mathpar}

The dependence of $M^{*}_{x}$ on a name makes it an abstraction, 

\begin{mathpar}
  M^{*} := (x)x?(u).M[\dropn{u}]
\end{mathpar}

\subsection{Additional notation}

It will sometimes be convenient to denote the process a name
quotes. We already have the notation $x = \quotep{P}$, but it will be
convenient to introduce an alternate notation, $\procn{x}$, when we
want to emphasize the connection to the use of the name. Note that, by
virtue of name equivalence, $\quotep{\procn{x}} \nameeq x$; so, the
notation is consistent with previous definitions.

Further, because names have structure it is possible to effect
substitutions on the basis of that structure. This means we need to
upgrade our notation for substitutions, which we accomplish by
adapting comprehension notation. Thus,

\begin{mathpar}
  P\{ y / x : x \in S \}
\end{mathpar}

is interpreted to mean the process derived from P by replacing (in a
capture-avoiding manner) each occurrence of $x$ in $S$ by $y$. For example,

\begin{mathpar}
  P\{ \quotep{\procn{x}|\procn{x}} / x : x \in \freenames{P} \}
\end{mathpar}

will replace each (occurrence) of a free name $x$ in $P$ by
$\quotep{\procn{x}|\procn{x}}$.

Also, we will avail ourselves of the notation $x^{L}$ and $x^{R}$ to
denote injections of a name into disjoint copies of the name
space. There are numerous ways to accomplish this. One example can be
found in \cite{MeredithR05}. This notation overloads to vectors of
names: $\vec{x}^{\pi} := (x_{i}^{\pi} \; : \; 0 \leq i < |\vec{x}| )$ where $\pi \in \{L,R\}$.

We also use $P^{\Box} := P|\Box$.

In \cite{MeredithR05} an interpretation of the new operator is
given. It turns out that there are several possible interpretations
all enjoying the requisite algebraic properties of the operator (see
\cite{milner91polyadicpi}). We will therefore make liberal use of
$(\nu\; \vec{x})P$.

% subsection the_syntax_and_semantics_of_the_notation_system (end)   

\input{qm2pi.qmops} 

\input{qm2pi.sterngerlach} 

\input{qm2pi.metric} 

% section concurrent_process_calculi (end)

%\input{qm2pi.proofsketch}

% section proof sketch (end)

%\input{qm2pi.slviaknots} 

% section spatial logic via knots (end)

\input{qm2pi.conclusion}

% section conclusion (end)

%\input{qm2pi.dtcodes} 

% section wiring algorithm (end)

\input{qm2pi.ack} 

% section acknowledgments (end)

\newpage


\bibliographystyle{plain}   
\bibliography{../../biblios/main.bib}

\input{qm2pi.rhodetails}

\end{document}

 

% section wiring algorithm (end)

\documentclass[12pt]{llncs}
%\documentclass{jktr}

\usepackage[pdftex]{hyperref}                   
\usepackage {listings}
\usepackage {mathpartir}
\usepackage{bcprules}
%\usepackage{listings}
                       
\usepackage{graphicx} 
%\usepackage[margins=2.5cm,nohead,nofoot]{geometry}
%\usepackage{geometry}
\usepackage{amsfonts}
\usepackage{amstext}
\usepackage{latexsym}
\usepackage{amssymb}
\usepackage{color}


%\include{myPreamble}
\include{qm2pi.local} 

%\ifpdf
%\usepackage[pdftex]{graphicx}
%\else
%\usepackage{graphicx}
%\fi

 % \ifpdf
%  \usepackage{pdfsync}
%  \if


%\title{Brief Article}
%\author{David F. Snyder}
%\author{L.G. Meredith}

%\address{Dept. of Math., Texas State University--San Marcos, San Marcos, TX 78666}
       
\pagestyle{empty}


\begin{document}

\lstset{language=[Objective]Caml,frame=shadowbox}

\input{qm2pi.front}

% section front matter (end)

\input{qm2pi.intro} 
 
% section introduction (end)

% \input{qm2pi.knotations} 

% section notation (end)

\input{qm2pi.process.calculi} 

% section concurrent_process_calculi_and_spatial_logics_ (end)
    
%\input{qm2pi.knots2pi} 

%\input{qm2pi.trefoil} 

%\input{qm2pi.mainthm} 

% subsection basic_interpretation (end)

%\input{qm2pi.rho.presentation} 
\subsection{The syntax and semantics of the notation system}\label{sub:the_syntax_and_semantics_of_the_notation_system} % (fold)

We now summarize a technical presentation of the calculus that
embodies our theory of dynamics. The typical presentation of such a
calculus follows the style of giving generators and relations on
them. The grammar, below, describing term constructors, freely
generates the set of processes, $\Proc$. This set is then quotiented
by a relation known as structural congruence and it is over this set
that the notion of dynamics is expressed. This presentation is
essentially that of \cite{MeredithR05} with the addition of
polyadicity and summation. For readability we have relegated some of
the technical subtleties to an appendix.

\subsubsection{Process grammar}\label{subsub:process_grammar}

\begin{mathpar}
  \inferrule* [lab=synchronization] {} {{M} \bc \pzero \;|\; x?F \;|\; x!C }
  \and
  \inferrule* [lab=abstraction] {} {{F} \bc (x)P}
  \and
  \inferrule* [lab=concretion] {} {{C} \bc \langle Q \rangle}
  \and
  \inferrule* [lab=process] {} {{P,Q} \bc M \;| \;P|Q \;|\; @{x}}
  \and
  \inferrule* [lab=name] {} {{x} \bc \quotep{P}}
\end{mathpar} 

Note that $\vec{x}$ (resp. $\vec{P}$) denotes a vector of names
(resp. processes) of length $|\vec{x}|$ (resp. $|\vec{P}|$). We adopt
the following useful abbreviations.

\begin{mathpar}
   x?(\vec{y}).P := x.(\vec{y})P \and  x\clift{\vec{P}} := x.\clift{\vec{P}}
   \and x!(y) := \lift{x}{\dropn{y}}
   \and \Pi_{i=0}^{n-1}P_i := P_0 | \ldots | P_{n-1}
\end{mathpar}

\subsubsection{Structural congruence}

\paragraph{Free and bound names and alpha-equivalence.} At the
core of structural equivalence is alpha-equivalence which identifies
process that are the same up to a change of variable. Formally, we
recognize the distinction between free and bound names. The free names
of a process, $\freenames{P}$, may be calculated recursively as
follows:

\begin{mathpar}
\freenames{\pzero} := \emptyset
  \and \\
  \freenames{x?(y).P} := \{ x \} \cup (\freenames{P} \setminus \{ y \})
  \and 
  \freenames{x!\langle P \rangle} := \{ x \} \cup \{ P \} 
  \and \\
  \freenames{P|Q} := \freenames{P} \cup \freenames{Q}
  \and \\
  \freenames{@{x}} := \{ x \}
\end{mathpar}

$\pi$
$\quotep{\pi}$

$\freenames{-} : \pi \to \mathcal{P}(\quotep{\pi})$

\begin{eqnarray*}
  \freenames{\pzero} & := & \emptyset \\
  \freenames{x?(y).P} & := & \{ x \} \cup (\freenames{P} \setminus \{ y \}) \\
  \freenames{x!\langle P \rangle} & := & \{ x \} \cup \{ P \} \\
  \freenames{P|Q} & := & \freenames{P} \cup \freenames{Q} \\
  \freenames{\dropn{x}} & := & \{ x \}
\end{eqnarray*}

The bound names of a process, $\boundnames{P}$, are those names occurring in $P$
that are not free. For example, in $x?(y).0$, the name $x$ is free, while $y$ is bound.

\begin{mathpar}
  \inferrule* [lab=monoidal-laws] {} { P|Q \equiv Q|P \and P|0 \equiv P \and P|(Q|R) \equiv (P|Q)|R }
\end{mathpar}

\begin{mathpar}
  \inferrule* [lab=alpha-equivalence] {} { (x)P \equiv (y)P\{y/x\} \and y \not\in \freenames{P} }
\end{mathpar}

\begin{definition}
Then two processes, $P,Q$, are alpha-equivalent if $P = Q\{\vec{y}/\vec{x}\}$ for
some $\vec{x} \in \boundnames{Q},\vec{y} \in \boundnames{P}$, where $Q\{\vec{y}/\vec{x}\}$
denotes the capture-avoiding substitution of $\vec{y}$ for $\vec{x}$ in $Q$.
\end{definition}

\begin{definition}
  The {\em structural congruence} \cite{SangiorgiWalker} , $\equiv$,
  between processes is the least congruence containing
  alpha-equivalence, satisfying the abelian monoid laws
  (associativity, commutativity and $\pzero$ as identity) for parallel
  composition $|$ and for summation $+$.
\end{definition}

\subsection{Name equivalence}

We take name equivalence, written $\nameeq$, to be the smallest
equivalence relation generated by the following rules.

\begin{mathpar}
\inferrule*[lab=Quote-drop]
{ }
{ \quotep{@{x}} \nameeq x }

\inferrule*[lab=Struct-equiv]
{ P \scong Q }
{ \quotep{P} \nameeq \quotep{Q} }
\end{mathpar}

The astute reader will have noticed that the mutual recursion of names
and processes imposes a mutual recursion on alpha-equivalence and
structural equivalence via name-equivalence. Fortunately, all of this
works out pleasantly and we may calculate in the natural way, free of
concern. The reader interested in the details is referred to the
appendix \ref{appendix:rho_details}.

\subsection{Substitution}

We use $\Proc$ for the set of processes, $\QProc$ for the set of
names, and $\id{\{}\vec{y} / \vec{x} \id{\}}$ to denote partial maps,
$s : \QProc \rightarrow \QProc$. A map, $s$ lifts, uniquely, to a map
on process terms, $\widehat{s} : \Proc \rightarrow \Proc$ by the
following equations.

\begin{mathpar}
  (0) \psubstp{Q}{P} := 0 \\
  (R \juxtap S) \psubstp{Q}{P}
  :=    
  (R)\psubstp{Q}{P} \juxtap (S) \psubstp{Q}{P} \\
  (x?(y).R) \psubstp{Q}{P}    
  :=    
  (x)\substp{Q}{P} (z)\concat( (R \psubstn{z}{y}) \psubstp{Q}{P} ) \\
  (\lift{x}{R}) \psubstp{Q}{P}  
  :=
  \lift{(x)\substp{Q}{P}}{ R \psubstp{Q}{P} } \\
%   (\dropn{x})  \psubstp{Q}{P}       
%   := 
%   \left\{ 
%     \begin{array}{ccc} 
%       \dropn{\quotep{Q}} & & x \nameeq \quotep{P} \\
%       \dropn{x} & & otherwise \\
%     \end{array}
%   \right. 
  (\dropn{x})  \psubstp{Q}{P}       
  := 
  \left\{ 
    \begin{array}{ccc} 
      Q & & x \nameeq \quotep{P} \\
      \dropn{x} & & otherwise \\
    \end{array}
  \right.
\end{mathpar}
 

where

\begin{eqnarray}
  (x)\id{\{} \lpquote Q \rpquote / \lpquote P \rpquote \id{\}}            = 
  \left\{ 
    \begin{array}{ccc}
      \lpquote Q \rpquote & & x \nameeq \lpquote P \rpquote \\
      x & & otherwise \\
    \end{array}
  \right. \nonumber
\end{eqnarray}

and $z$ is chosen distinct from $\quotep{P}$, $\quotep{Q}$, the free
names in $Q$, and all the names in $R$. Our $\alpha$-equivalence will
be built in the standard way from this substitution.

\begin{remark}\label{rem:no_self_referential_names}
  One consequence of these definitions is that $\forall P. \quotep{P}
  \not\in \freenames{P}$.
\end{remark}

\subsection{ Dynamic quote: an example }

Anticipating something of what's to come, consider applying the
substitution, $\widehat{\id{\{}u / z \id{\}}}$, to the following pair
of processes, $\lift{w}{y!(z)}$ and $w[ \lpquote y!(z) \rpquote ]$.

\begin{eqnarray}
	\lift{w}{y!(z)}\widehat{\id{\{}u / z \id{\}}}
		& = &
		\lift{w}{y!(u)} \nonumber\\
	w[ \lpquote y!(z) \rpquote ] \widehat{ \id{\{}u / z \id{\}} }
		& = &
		w[ \lpquote y!(z) \rpquote ] \nonumber
\end{eqnarray}

Because the body of the process between quotes is impervious to
substitution, we get radically different answers. In fact, by
examining the first process in an input context,
e.g. $x?(z).\lift{w}{y!(z)}$, we see that the process under the lift
operator may be shaped by prefixed inputs binding a name inside it. In
this sense, the lift operator will be seen as a way to dynamically
construct processes before reifying them as names.

Finally equipped with these standard features we can present the
dynamics of the calculus.

\subsubsection{Operational semantics} 

Finally, we introduce the computational dynamics. What marks these
algebras as distinct from other more traditionally studied algebraic
structures, e.g. vector spaces or polynomial rings, is the manner in
which dynamics is captured. In traditional structures, dynamics is typically
expressed through morphisms between such structures, as in linear maps
between vector spaces or morphisms between rings. In algebras
associated with the semantics of computation, the dynamics is
expressed as part of the algebraic structure itself, through a
reduction reduction relation typically denoted by $\red$. Below, we
give a recursive presentation of this relation for the calculus used
in the encoding.

$\red \subseteq \pi \times \pi$
$\red : \pi \to \mathcal{P}(\pi)$

\begin{mathpar}
  \inferrule* [lab=Comm] { \textsf{match}( x_{src}, x_{trgt} ) } { x_{trgt}?(y)P \; | \; x_{src}!\langle {Q} \rangle \red P\{\quotep{Q}/y}\} }
  \and \\
  \inferrule* [lab=Par] {{P} \red {P}'} {{{P} | {Q}} \red {{P}' | {Q}}}
  \and
  \inferrule* [lab=Equiv]{{{P} \scong {P}'} \andalso {{P}' \red {Q}'} \andalso {{Q}' \scong {Q}}}{{P} \red {Q}}
\end{mathpar}

\begin{eqnarray*}
  match_{\equiv} (\quotep{P},\quotep{Q}) & := & P \equiv Q \\
  match_{\dagger}(\quotep{P},\quotep{Q}) & := & \forall R. P|Q \red^{*} R => R \red^{*} 0 \\
  match_{K}(\quotep{P},\quotep{Q}) & := & K \mbox{ for some context } K
\end{eqnarray*}

$u?(x)P | u!\langle Q \rangle \red P\{\quotep{Q}/x\}$

%We write $\wred$ for $\red^*$, and $P\red$ if $\exists Q $ such that $ P \red Q$.
We write $P\red$ if $\exists Q $ such that $ P \red Q$ and $P\not\red$, otherwise.

\section{Replication}

As mentioned before, it is known that replication (and hence
recursion) can be implemented in a higher-order process algebra
\cite{SangiorgiWalker}. As our first example of calculation with the
machinery thus far presented we give the construction explicitly in
the {\rhoc}.

\begin{eqnarray}
	D_{x} & := & \prefix{x}{y}{(\binpar{\outputp{x}{y}}{@{y}})} \nonumber\\
	\bangp_{x}{P} & := & \binpar{{x}!\langle{\binpar{D_{x}}{P}}\rangle}{D_{x}} \nonumber
\end{eqnarray}

\begin{eqnarray}
	\bangp_{x}{P} & & \nonumber\\
	=
	& {x}!\langle{(\prefix{x}{y}{(\outputp{x}{y} | @{y})) | P}}\rangle 
	      | \prefix{x}{y}{(\outputp{x}{y} | @{y})} & \nonumber\\
	\red
	& (\outputp{x}{y} | @{y})\substn{\quotep{(\prefix{x}{y}{(@{y} | \outputp{x}{y})) | P}}}{y} & \nonumber\\
	=
	& \outputp{x}{\quotep{(\prefix{x}{y}{(\outputp{x}{y} | @{y})) | P}}}
	  | {(\prefix{x}{y}{(\outputp{x}{y} | @{y})) | P}} & \nonumber\\
	\red
	& \ldots & \nonumber\\
	\red^*
	& P | P | \ldots & \nonumber
\end{eqnarray}

Of course, this encoding, as an implementation, runs away, unfolding
$\bangp{P}$ eagerly. A lazier and more implementable replication
operator, restricted to input-guarded processes, may be obtained as follows.

\begin{eqnarray}
\bangp{\prefix{u}{v}{P}} 
	:= 
	\binpar{\lift{x}{\prefix{u}{v}{(\binpar{D(x)}{P})}}}{D(x)} \nonumber
\end{eqnarray}

\begin{remark}
  Note that the lazier definition still does not deal with summation
  or mixed summation (i.e. sums over input and output). The reader is
  invited to construct definitions of replication that deal with these
  features. 

  Further, the definitions are parameterized in a name, $x$. Can you,
  gentle reader, make a definition that eliminates this parameter and
  guarantees no accidental interaction between the replication
  machinery and the process being replicated -- i.e. no accidental
  sharing of names used by the process to get its work done and the
  name(s) used by the replication to effect copying. This latter
  revision of the definition of replication is crucial to obtaining
  the expected identity $!!P \sim !P$.
\end{remark}

\begin{remark}\label{rem:paradoxical_combinator}
  The reader familiar with the lambda calculus will have noticed the
  similarity between $D$ and the paradoxical combinator.

  [Ed. note: the existence of this seems to suggest we have to be more
  restrictive on the set of processes and names we admit if we are to
  support no-cloning.]
\end{remark}

\subsubsection{Bisimulation}

The computational dynamics gives rise to another kind of equivalence,
the equivalence of computational behavior. As previously mentioned
this is typically captured \emph{via} some form of bisimulation.

% The notion we use in this paper is weak barbed bisimulation
% \cite{milner91polyadicpi}.

The notion we use in this paper is derived from weak barbed
bisimulation \cite{milner91polyadicpi}. 

\begin{definition}
An \emph{observation relation}, $\downarrow_{\mathcal N}$, over a set
of names, $\mathcal N$, is the smallest relation satisfying the rules
below.

\infrule[Out-barb]{y \in {\mathcal N}, \; x \nameeq y}
		  {\outputp{x}{v} \downarrow_{\mathcal N} x}
\infrule[Par-barb]{\mbox{$P\downarrow_{\mathcal N} x$ or $Q\downarrow_{\mathcal N} x$}}
		  {\binpar{P}{Q} \downarrow_{\mathcal N} x}

We write $P \Downarrow_{\mathcal N} x$ if there is $Q$ such that 
$P \wred Q$ and $Q \downarrow_{\mathcal N} x$.
\end{definition}

\begin{definition}
%\label{def.bbisim}
An  ${\mathcal N}$-\emph{barbed bisimulation} over a set of names, ${\mathcal N}$, is a symmetric binary relation 
${\mathcal S}_{\mathcal N}$ between agents such that $P\rel{S}_{\mathcal N}Q$ implies:
\begin{enumerate}
\item If $P \red P'$ then $Q \wred Q'$ and $P'\rel{S}_{\mathcal N} Q'$.
\item If $P\downarrow_{\mathcal N} x$, then $Q\Downarrow_{\mathcal N} x$.
\end{enumerate}
$P$ is ${\mathcal N}$-barbed bisimilar to $Q$, written
$P \wbbisim_{\mathcal N} Q$, if $P \rel{S}_{\mathcal N} Q$ for some ${\mathcal N}$-barbed bisimulation ${\mathcal S}_{\mathcal N}$.
\end{definition}

$\mathcal{R} \subseteq \pi \times \pi$

$P \mathcal{R} Q => \forall P'. P \red P' \Rightarrow \exists Q'. Q \red Q', P' \mathcal{R} Q'$

$P \vdash x \Rightarrow Q \vdash x$

\begin{mathpar}
  \inferrule*[lab=Out-barb]{x \nameeq y}{{y}!\langle{Q}\rangle \vdash x}
  \and
  \inferrule*[lab=Par-barb]{\mbox{$P\vdash x$ or $Q\vdash x$}}{\binpar{P}{Q} \vdash x}
\end{mathpar}

\subsubsection{Contexts}

One of the principle advantages of computational calculi like the
$\pi$-calculus is a well-defined notion of context,
contextual-equivalence and a correlation between
contextual-equivalence and notions of bisimulation. The notion of
context allows the decomposition of a process into (sub-)process and
its syntactic environment, its context. Thus, a context may be
thought of as a process with a ``hole'' (written $\Box$) in it. The
application of a context $M$ to a process $P$, written $M[P]$, is
tantamount to filling the hole in $M$ with $P$. In this paper we do
not need the full weight of this theory, but do make use of the notion
of context in the proof the main theorem. 

\begin{mathpar}
  \inferrule* [lab=summation] {} {{M_{M},M_{N}} \bc \Box \;|\; x.M_{A} \;|\; M_{M}+M_{N}}
  \and
  \inferrule* [lab=agent] {} {{M_{A}} \bc (\vec{x})M_{P} \;| \; \clift{P_0,\ldots,M_{P},\ldots,P_N}}
  \and \\
  \inferrule* [lab=process] {} {{M_{P}} \bc M_{N} \;| \;P|M_{P} }
\end{mathpar} 

\begin{mathpar}
  \inferrule* [lab=sychronization] {} {M_{N} \bc \Box \;|\; x?M_{F} \;|\; x!M_{C}}
  \and
  \inferrule* [lab=abstraction] {} {{M_{F}} \bc (x)M_{P} }
  \and
  \inferrule* [lab=concretion] {} {{M_{C}} \bc \langle M_{P} \rangle }
  \and \\
  \inferrule* [lab=process] {} {{M_{P}} \bc M_{N} \;| \;P|M_{P} }
\end{mathpar}

\begin{definition}[contextual application] Given a context $M$, and
  process $P$, we define the \emph{contextual application}, $M[P] :=
  M\{P/\Box\}$. That is, the contextual application of M to P is the
  substitution of $P$ for $\Box$ in $M$.
\end{definition}

$\meaningof{-} : L \to \mathcal{P}(\pi)$

\begin{mathpar}
  \inferrule* [lab=collection] {} {\meaningof{true} = \pi, \and \meaningof{~E} = \pi \setminus \meaningof{E}, \and \meaningof{E_{1} \& E_{2}} = \meaningof{E_{1}} \cap \meaningof{E_{2}}}
\end{mathpar}

\begin{mathpar}
  \inferrule* [lab=structure] {} {\meaningof{0} = \{ P \in \pi | P \equiv 0 \}, \and \\ \meaningof{E_1 | E_2} = \{ P \in \pi | P \equiv P_{1} | P_{2}, P_{1} \in \meaningof{E_{1}}, P_{2} \in \meaningof{E_2}\} }
\end{mathpar}

\begin{mathpar}
 \inferrule* [lab=behavior] {} {\meaningof{\langle a?b \rangle E} = \{ P \in \pi | P \equiv Q | u?(y)P', \\ \and \\\\ \and \\ \;\;\; u \in \meaningof{a}, \forall z.P'\{z/y\} \in \meaningof{E\{z/b\}}\}, \and \\ \meaningof{a!E} = \{ P \in \pi | P \equiv Q | x!\langle P' \rangle, x \in \meaningof{a} P' \in \meaningof{E}\} }
\end{mathpar}

\begin{mathpar}
 \inferrule* [lab=nominal] {} {\meaningof{\quotep{E}} = \{ \quotep{P} \in \quotep{\pi} | P \in \meaningof{E} \}, \and \meaningof{\quotep{P}} = \{ \quotep{Q} \in \quotep{\pi} | P \equiv Q \} \and \\ \meaningof{@\quotep{E}} = \{ P \in \pi | P \equiv @x, x \in \meaningof{E} \}}
\end{mathpar}

\begin{eqnarray*}
  \\
  \meaningof{-} : TS \to ST
\end{eqnarray*}

\begin{eqnarray*}
  \\
  L : TS \to ST
\end{eqnarray*}

\begin{eqnarray*}
  \\
  P \models E \iff P \in \meaningof{E}
\end{eqnarray*}

\begin{eqnarray*}
  P \approx_{L} Q \iff \forall E \in L. P \models E \iff Q \models E
\end{eqnarray*}

\begin{eqnarray*}
  P \approx_{K} Q
\end{eqnarray*}

\begin{eqnarray*}
  P \approx Q
\end{eqnarray*}

$\approx_{K} = \approx = \approx_{L}$

\subsubsection{Contextual duality}

Note that contexts extend the quotation operation to a family of
operations from processes to names. Given a context, $M$, we can
define a \emph{nominal context}, $\quotep{M}$ by $\quotep{M}[P] :=
\quotep{M[P]}$. To foreshadow what is to come we observe that these
operations enjoy a duality with processes very much like the duality
between vectors and maps from vectors to scalars.

Further, because the calculus is essentially higher-order, we have a
correspondence between contexts and processes. More specifically,
given a name $x$ and a context $M$ we can construct $M^{*}_{x}$ such
that 

\begin{mathpar}
  M^{*}_{x} | \lift{x}{P} \red M[P]
\end{mathpar}

namely,

\begin{mathpar}
  M^{*}_{x} := x?(u).M[\dropn{u}]
\end{mathpar}

The dependence of $M^{*}_{x}$ on a name makes it an abstraction, 

\begin{mathpar}
  M^{*} := (x)x?(u).M[\dropn{u}]
\end{mathpar}

\subsection{Additional notation}

It will sometimes be convenient to denote the process a name
quotes. We already have the notation $x = \quotep{P}$, but it will be
convenient to introduce an alternate notation, $\procn{x}$, when we
want to emphasize the connection to the use of the name. Note that, by
virtue of name equivalence, $\quotep{\procn{x}} \nameeq x$; so, the
notation is consistent with previous definitions.

Further, because names have structure it is possible to effect
substitutions on the basis of that structure. This means we need to
upgrade our notation for substitutions, which we accomplish by
adapting comprehension notation. Thus,

\begin{mathpar}
  P\{ y / x : x \in S \}
\end{mathpar}

is interpreted to mean the process derived from P by replacing (in a
capture-avoiding manner) each occurrence of $x$ in $S$ by $y$. For example,

\begin{mathpar}
  P\{ \quotep{\procn{x}|\procn{x}} / x : x \in \freenames{P} \}
\end{mathpar}

will replace each (occurrence) of a free name $x$ in $P$ by
$\quotep{\procn{x}|\procn{x}}$.

Also, we will avail ourselves of the notation $x^{L}$ and $x^{R}$ to
denote injections of a name into disjoint copies of the name
space. There are numerous ways to accomplish this. One example can be
found in \cite{MeredithR05}. This notation overloads to vectors of
names: $\vec{x}^{\pi} := (x_{i}^{\pi} \; : \; 0 \leq i < |\vec{x}| )$ where $\pi \in \{L,R\}$.

We also use $P^{\Box} := P|\Box$.

In \cite{MeredithR05} an interpretation of the new operator is
given. It turns out that there are several possible interpretations
all enjoying the requisite algebraic properties of the operator (see
\cite{milner91polyadicpi}). We will therefore make liberal use of
$(\nu\; \vec{x})P$.

% subsection the_syntax_and_semantics_of_the_notation_system (end)   

\input{qm2pi.qmops} 

\input{qm2pi.sterngerlach} 

\input{qm2pi.metric} 

% section concurrent_process_calculi (end)

%\input{qm2pi.proofsketch}

% section proof sketch (end)

%\input{qm2pi.slviaknots} 

% section spatial logic via knots (end)

\input{qm2pi.conclusion}

% section conclusion (end)

%\input{qm2pi.dtcodes} 

% section wiring algorithm (end)

\input{qm2pi.ack} 

% section acknowledgments (end)

\newpage


\bibliographystyle{plain}   
\bibliography{../../biblios/main.bib}

\input{qm2pi.rhodetails}

\end{document}

 

% section acknowledgments (end)

\newpage


\bibliographystyle{plain}   
\bibliography{../../biblios/main.bib}

\documentclass[12pt]{llncs}
%\documentclass{jktr}

\usepackage[pdftex]{hyperref}                   
\usepackage {listings}
\usepackage {mathpartir}
\usepackage{bcprules}
%\usepackage{listings}
                       
\usepackage{graphicx} 
%\usepackage[margins=2.5cm,nohead,nofoot]{geometry}
%\usepackage{geometry}
\usepackage{amsfonts}
\usepackage{amstext}
\usepackage{latexsym}
\usepackage{amssymb}
\usepackage{color}


%\include{myPreamble}
\include{qm2pi.local} 

%\ifpdf
%\usepackage[pdftex]{graphicx}
%\else
%\usepackage{graphicx}
%\fi

 % \ifpdf
%  \usepackage{pdfsync}
%  \if


%\title{Brief Article}
%\author{David F. Snyder}
%\author{L.G. Meredith}

%\address{Dept. of Math., Texas State University--San Marcos, San Marcos, TX 78666}
       
\pagestyle{empty}


\begin{document}

\lstset{language=[Objective]Caml,frame=shadowbox}

\input{qm2pi.front}

% section front matter (end)

\input{qm2pi.intro} 
 
% section introduction (end)

% \input{qm2pi.knotations} 

% section notation (end)

\input{qm2pi.process.calculi} 

% section concurrent_process_calculi_and_spatial_logics_ (end)
    
%\input{qm2pi.knots2pi} 

%\input{qm2pi.trefoil} 

%\input{qm2pi.mainthm} 

% subsection basic_interpretation (end)

%\input{qm2pi.rho.presentation} 
\subsection{The syntax and semantics of the notation system}\label{sub:the_syntax_and_semantics_of_the_notation_system} % (fold)

We now summarize a technical presentation of the calculus that
embodies our theory of dynamics. The typical presentation of such a
calculus follows the style of giving generators and relations on
them. The grammar, below, describing term constructors, freely
generates the set of processes, $\Proc$. This set is then quotiented
by a relation known as structural congruence and it is over this set
that the notion of dynamics is expressed. This presentation is
essentially that of \cite{MeredithR05} with the addition of
polyadicity and summation. For readability we have relegated some of
the technical subtleties to an appendix.

\subsubsection{Process grammar}\label{subsub:process_grammar}

\begin{mathpar}
  \inferrule* [lab=synchronization] {} {{M} \bc \pzero \;|\; x?F \;|\; x!C }
  \and
  \inferrule* [lab=abstraction] {} {{F} \bc (x)P}
  \and
  \inferrule* [lab=concretion] {} {{C} \bc \langle Q \rangle}
  \and
  \inferrule* [lab=process] {} {{P,Q} \bc M \;| \;P|Q \;|\; @{x}}
  \and
  \inferrule* [lab=name] {} {{x} \bc \quotep{P}}
\end{mathpar} 

Note that $\vec{x}$ (resp. $\vec{P}$) denotes a vector of names
(resp. processes) of length $|\vec{x}|$ (resp. $|\vec{P}|$). We adopt
the following useful abbreviations.

\begin{mathpar}
   x?(\vec{y}).P := x.(\vec{y})P \and  x\clift{\vec{P}} := x.\clift{\vec{P}}
   \and x!(y) := \lift{x}{\dropn{y}}
   \and \Pi_{i=0}^{n-1}P_i := P_0 | \ldots | P_{n-1}
\end{mathpar}

\subsubsection{Structural congruence}

\paragraph{Free and bound names and alpha-equivalence.} At the
core of structural equivalence is alpha-equivalence which identifies
process that are the same up to a change of variable. Formally, we
recognize the distinction between free and bound names. The free names
of a process, $\freenames{P}$, may be calculated recursively as
follows:

\begin{mathpar}
\freenames{\pzero} := \emptyset
  \and \\
  \freenames{x?(y).P} := \{ x \} \cup (\freenames{P} \setminus \{ y \})
  \and 
  \freenames{x!\langle P \rangle} := \{ x \} \cup \{ P \} 
  \and \\
  \freenames{P|Q} := \freenames{P} \cup \freenames{Q}
  \and \\
  \freenames{@{x}} := \{ x \}
\end{mathpar}

$\pi$
$\quotep{\pi}$

$\freenames{-} : \pi \to \mathcal{P}(\quotep{\pi})$

\begin{eqnarray*}
  \freenames{\pzero} & := & \emptyset \\
  \freenames{x?(y).P} & := & \{ x \} \cup (\freenames{P} \setminus \{ y \}) \\
  \freenames{x!\langle P \rangle} & := & \{ x \} \cup \{ P \} \\
  \freenames{P|Q} & := & \freenames{P} \cup \freenames{Q} \\
  \freenames{\dropn{x}} & := & \{ x \}
\end{eqnarray*}

The bound names of a process, $\boundnames{P}$, are those names occurring in $P$
that are not free. For example, in $x?(y).0$, the name $x$ is free, while $y$ is bound.

\begin{mathpar}
  \inferrule* [lab=monoidal-laws] {} { P|Q \equiv Q|P \and P|0 \equiv P \and P|(Q|R) \equiv (P|Q)|R }
\end{mathpar}

\begin{mathpar}
  \inferrule* [lab=alpha-equivalence] {} { (x)P \equiv (y)P\{y/x\} \and y \not\in \freenames{P} }
\end{mathpar}

\begin{definition}
Then two processes, $P,Q$, are alpha-equivalent if $P = Q\{\vec{y}/\vec{x}\}$ for
some $\vec{x} \in \boundnames{Q},\vec{y} \in \boundnames{P}$, where $Q\{\vec{y}/\vec{x}\}$
denotes the capture-avoiding substitution of $\vec{y}$ for $\vec{x}$ in $Q$.
\end{definition}

\begin{definition}
  The {\em structural congruence} \cite{SangiorgiWalker} , $\equiv$,
  between processes is the least congruence containing
  alpha-equivalence, satisfying the abelian monoid laws
  (associativity, commutativity and $\pzero$ as identity) for parallel
  composition $|$ and for summation $+$.
\end{definition}

\subsection{Name equivalence}

We take name equivalence, written $\nameeq$, to be the smallest
equivalence relation generated by the following rules.

\begin{mathpar}
\inferrule*[lab=Quote-drop]
{ }
{ \quotep{@{x}} \nameeq x }

\inferrule*[lab=Struct-equiv]
{ P \scong Q }
{ \quotep{P} \nameeq \quotep{Q} }
\end{mathpar}

The astute reader will have noticed that the mutual recursion of names
and processes imposes a mutual recursion on alpha-equivalence and
structural equivalence via name-equivalence. Fortunately, all of this
works out pleasantly and we may calculate in the natural way, free of
concern. The reader interested in the details is referred to the
appendix \ref{appendix:rho_details}.

\subsection{Substitution}

We use $\Proc$ for the set of processes, $\QProc$ for the set of
names, and $\id{\{}\vec{y} / \vec{x} \id{\}}$ to denote partial maps,
$s : \QProc \rightarrow \QProc$. A map, $s$ lifts, uniquely, to a map
on process terms, $\widehat{s} : \Proc \rightarrow \Proc$ by the
following equations.

\begin{mathpar}
  (0) \psubstp{Q}{P} := 0 \\
  (R \juxtap S) \psubstp{Q}{P}
  :=    
  (R)\psubstp{Q}{P} \juxtap (S) \psubstp{Q}{P} \\
  (x?(y).R) \psubstp{Q}{P}    
  :=    
  (x)\substp{Q}{P} (z)\concat( (R \psubstn{z}{y}) \psubstp{Q}{P} ) \\
  (\lift{x}{R}) \psubstp{Q}{P}  
  :=
  \lift{(x)\substp{Q}{P}}{ R \psubstp{Q}{P} } \\
%   (\dropn{x})  \psubstp{Q}{P}       
%   := 
%   \left\{ 
%     \begin{array}{ccc} 
%       \dropn{\quotep{Q}} & & x \nameeq \quotep{P} \\
%       \dropn{x} & & otherwise \\
%     \end{array}
%   \right. 
  (\dropn{x})  \psubstp{Q}{P}       
  := 
  \left\{ 
    \begin{array}{ccc} 
      Q & & x \nameeq \quotep{P} \\
      \dropn{x} & & otherwise \\
    \end{array}
  \right.
\end{mathpar}
 

where

\begin{eqnarray}
  (x)\id{\{} \lpquote Q \rpquote / \lpquote P \rpquote \id{\}}            = 
  \left\{ 
    \begin{array}{ccc}
      \lpquote Q \rpquote & & x \nameeq \lpquote P \rpquote \\
      x & & otherwise \\
    \end{array}
  \right. \nonumber
\end{eqnarray}

and $z$ is chosen distinct from $\quotep{P}$, $\quotep{Q}$, the free
names in $Q$, and all the names in $R$. Our $\alpha$-equivalence will
be built in the standard way from this substitution.

\begin{remark}\label{rem:no_self_referential_names}
  One consequence of these definitions is that $\forall P. \quotep{P}
  \not\in \freenames{P}$.
\end{remark}

\subsection{ Dynamic quote: an example }

Anticipating something of what's to come, consider applying the
substitution, $\widehat{\id{\{}u / z \id{\}}}$, to the following pair
of processes, $\lift{w}{y!(z)}$ and $w[ \lpquote y!(z) \rpquote ]$.

\begin{eqnarray}
	\lift{w}{y!(z)}\widehat{\id{\{}u / z \id{\}}}
		& = &
		\lift{w}{y!(u)} \nonumber\\
	w[ \lpquote y!(z) \rpquote ] \widehat{ \id{\{}u / z \id{\}} }
		& = &
		w[ \lpquote y!(z) \rpquote ] \nonumber
\end{eqnarray}

Because the body of the process between quotes is impervious to
substitution, we get radically different answers. In fact, by
examining the first process in an input context,
e.g. $x?(z).\lift{w}{y!(z)}$, we see that the process under the lift
operator may be shaped by prefixed inputs binding a name inside it. In
this sense, the lift operator will be seen as a way to dynamically
construct processes before reifying them as names.

Finally equipped with these standard features we can present the
dynamics of the calculus.

\subsubsection{Operational semantics} 

Finally, we introduce the computational dynamics. What marks these
algebras as distinct from other more traditionally studied algebraic
structures, e.g. vector spaces or polynomial rings, is the manner in
which dynamics is captured. In traditional structures, dynamics is typically
expressed through morphisms between such structures, as in linear maps
between vector spaces or morphisms between rings. In algebras
associated with the semantics of computation, the dynamics is
expressed as part of the algebraic structure itself, through a
reduction reduction relation typically denoted by $\red$. Below, we
give a recursive presentation of this relation for the calculus used
in the encoding.

$\red \subseteq \pi \times \pi$
$\red : \pi \to \mathcal{P}(\pi)$

\begin{mathpar}
  \inferrule* [lab=Comm] { \textsf{match}( x_{src}, x_{trgt} ) } { x_{trgt}?(y)P \; | \; x_{src}!\langle {Q} \rangle \red P\{\quotep{Q}/y}\} }
  \and \\
  \inferrule* [lab=Par] {{P} \red {P}'} {{{P} | {Q}} \red {{P}' | {Q}}}
  \and
  \inferrule* [lab=Equiv]{{{P} \scong {P}'} \andalso {{P}' \red {Q}'} \andalso {{Q}' \scong {Q}}}{{P} \red {Q}}
\end{mathpar}

\begin{eqnarray*}
  match_{\equiv} (\quotep{P},\quotep{Q}) & := & P \equiv Q \\
  match_{\dagger}(\quotep{P},\quotep{Q}) & := & \forall R. P|Q \red^{*} R => R \red^{*} 0 \\
  match_{K}(\quotep{P},\quotep{Q}) & := & K \mbox{ for some context } K
\end{eqnarray*}

$u?(x)P | u!\langle Q \rangle \red P\{\quotep{Q}/x\}$

%We write $\wred$ for $\red^*$, and $P\red$ if $\exists Q $ such that $ P \red Q$.
We write $P\red$ if $\exists Q $ such that $ P \red Q$ and $P\not\red$, otherwise.

\section{Replication}

As mentioned before, it is known that replication (and hence
recursion) can be implemented in a higher-order process algebra
\cite{SangiorgiWalker}. As our first example of calculation with the
machinery thus far presented we give the construction explicitly in
the {\rhoc}.

\begin{eqnarray}
	D_{x} & := & \prefix{x}{y}{(\binpar{\outputp{x}{y}}{@{y}})} \nonumber\\
	\bangp_{x}{P} & := & \binpar{{x}!\langle{\binpar{D_{x}}{P}}\rangle}{D_{x}} \nonumber
\end{eqnarray}

\begin{eqnarray}
	\bangp_{x}{P} & & \nonumber\\
	=
	& {x}!\langle{(\prefix{x}{y}{(\outputp{x}{y} | @{y})) | P}}\rangle 
	      | \prefix{x}{y}{(\outputp{x}{y} | @{y})} & \nonumber\\
	\red
	& (\outputp{x}{y} | @{y})\substn{\quotep{(\prefix{x}{y}{(@{y} | \outputp{x}{y})) | P}}}{y} & \nonumber\\
	=
	& \outputp{x}{\quotep{(\prefix{x}{y}{(\outputp{x}{y} | @{y})) | P}}}
	  | {(\prefix{x}{y}{(\outputp{x}{y} | @{y})) | P}} & \nonumber\\
	\red
	& \ldots & \nonumber\\
	\red^*
	& P | P | \ldots & \nonumber
\end{eqnarray}

Of course, this encoding, as an implementation, runs away, unfolding
$\bangp{P}$ eagerly. A lazier and more implementable replication
operator, restricted to input-guarded processes, may be obtained as follows.

\begin{eqnarray}
\bangp{\prefix{u}{v}{P}} 
	:= 
	\binpar{\lift{x}{\prefix{u}{v}{(\binpar{D(x)}{P})}}}{D(x)} \nonumber
\end{eqnarray}

\begin{remark}
  Note that the lazier definition still does not deal with summation
  or mixed summation (i.e. sums over input and output). The reader is
  invited to construct definitions of replication that deal with these
  features. 

  Further, the definitions are parameterized in a name, $x$. Can you,
  gentle reader, make a definition that eliminates this parameter and
  guarantees no accidental interaction between the replication
  machinery and the process being replicated -- i.e. no accidental
  sharing of names used by the process to get its work done and the
  name(s) used by the replication to effect copying. This latter
  revision of the definition of replication is crucial to obtaining
  the expected identity $!!P \sim !P$.
\end{remark}

\begin{remark}\label{rem:paradoxical_combinator}
  The reader familiar with the lambda calculus will have noticed the
  similarity between $D$ and the paradoxical combinator.

  [Ed. note: the existence of this seems to suggest we have to be more
  restrictive on the set of processes and names we admit if we are to
  support no-cloning.]
\end{remark}

\subsubsection{Bisimulation}

The computational dynamics gives rise to another kind of equivalence,
the equivalence of computational behavior. As previously mentioned
this is typically captured \emph{via} some form of bisimulation.

% The notion we use in this paper is weak barbed bisimulation
% \cite{milner91polyadicpi}.

The notion we use in this paper is derived from weak barbed
bisimulation \cite{milner91polyadicpi}. 

\begin{definition}
An \emph{observation relation}, $\downarrow_{\mathcal N}$, over a set
of names, $\mathcal N$, is the smallest relation satisfying the rules
below.

\infrule[Out-barb]{y \in {\mathcal N}, \; x \nameeq y}
		  {\outputp{x}{v} \downarrow_{\mathcal N} x}
\infrule[Par-barb]{\mbox{$P\downarrow_{\mathcal N} x$ or $Q\downarrow_{\mathcal N} x$}}
		  {\binpar{P}{Q} \downarrow_{\mathcal N} x}

We write $P \Downarrow_{\mathcal N} x$ if there is $Q$ such that 
$P \wred Q$ and $Q \downarrow_{\mathcal N} x$.
\end{definition}

\begin{definition}
%\label{def.bbisim}
An  ${\mathcal N}$-\emph{barbed bisimulation} over a set of names, ${\mathcal N}$, is a symmetric binary relation 
${\mathcal S}_{\mathcal N}$ between agents such that $P\rel{S}_{\mathcal N}Q$ implies:
\begin{enumerate}
\item If $P \red P'$ then $Q \wred Q'$ and $P'\rel{S}_{\mathcal N} Q'$.
\item If $P\downarrow_{\mathcal N} x$, then $Q\Downarrow_{\mathcal N} x$.
\end{enumerate}
$P$ is ${\mathcal N}$-barbed bisimilar to $Q$, written
$P \wbbisim_{\mathcal N} Q$, if $P \rel{S}_{\mathcal N} Q$ for some ${\mathcal N}$-barbed bisimulation ${\mathcal S}_{\mathcal N}$.
\end{definition}

$\mathcal{R} \subseteq \pi \times \pi$

$P \mathcal{R} Q => \forall P'. P \red P' \Rightarrow \exists Q'. Q \red Q', P' \mathcal{R} Q'$

$P \vdash x \Rightarrow Q \vdash x$

\begin{mathpar}
  \inferrule*[lab=Out-barb]{x \nameeq y}{{y}!\langle{Q}\rangle \vdash x}
  \and
  \inferrule*[lab=Par-barb]{\mbox{$P\vdash x$ or $Q\vdash x$}}{\binpar{P}{Q} \vdash x}
\end{mathpar}

\subsubsection{Contexts}

One of the principle advantages of computational calculi like the
$\pi$-calculus is a well-defined notion of context,
contextual-equivalence and a correlation between
contextual-equivalence and notions of bisimulation. The notion of
context allows the decomposition of a process into (sub-)process and
its syntactic environment, its context. Thus, a context may be
thought of as a process with a ``hole'' (written $\Box$) in it. The
application of a context $M$ to a process $P$, written $M[P]$, is
tantamount to filling the hole in $M$ with $P$. In this paper we do
not need the full weight of this theory, but do make use of the notion
of context in the proof the main theorem. 

\begin{mathpar}
  \inferrule* [lab=summation] {} {{M_{M},M_{N}} \bc \Box \;|\; x.M_{A} \;|\; M_{M}+M_{N}}
  \and
  \inferrule* [lab=agent] {} {{M_{A}} \bc (\vec{x})M_{P} \;| \; \clift{P_0,\ldots,M_{P},\ldots,P_N}}
  \and \\
  \inferrule* [lab=process] {} {{M_{P}} \bc M_{N} \;| \;P|M_{P} }
\end{mathpar} 

\begin{mathpar}
  \inferrule* [lab=sychronization] {} {M_{N} \bc \Box \;|\; x?M_{F} \;|\; x!M_{C}}
  \and
  \inferrule* [lab=abstraction] {} {{M_{F}} \bc (x)M_{P} }
  \and
  \inferrule* [lab=concretion] {} {{M_{C}} \bc \langle M_{P} \rangle }
  \and \\
  \inferrule* [lab=process] {} {{M_{P}} \bc M_{N} \;| \;P|M_{P} }
\end{mathpar}

\begin{definition}[contextual application] Given a context $M$, and
  process $P$, we define the \emph{contextual application}, $M[P] :=
  M\{P/\Box\}$. That is, the contextual application of M to P is the
  substitution of $P$ for $\Box$ in $M$.
\end{definition}

$\meaningof{-} : L \to \mathcal{P}(\pi)$

\begin{mathpar}
  \inferrule* [lab=collection] {} {\meaningof{true} = \pi, \and \meaningof{~E} = \pi \setminus \meaningof{E}, \and \meaningof{E_{1} \& E_{2}} = \meaningof{E_{1}} \cap \meaningof{E_{2}}}
\end{mathpar}

\begin{mathpar}
  \inferrule* [lab=structure] {} {\meaningof{0} = \{ P \in \pi | P \equiv 0 \}, \and \\ \meaningof{E_1 | E_2} = \{ P \in \pi | P \equiv P_{1} | P_{2}, P_{1} \in \meaningof{E_{1}}, P_{2} \in \meaningof{E_2}\} }
\end{mathpar}

\begin{mathpar}
 \inferrule* [lab=behavior] {} {\meaningof{\langle a?b \rangle E} = \{ P \in \pi | P \equiv Q | u?(y)P', \\ \and \\\\ \and \\ \;\;\; u \in \meaningof{a}, \forall z.P'\{z/y\} \in \meaningof{E\{z/b\}}\}, \and \\ \meaningof{a!E} = \{ P \in \pi | P \equiv Q | x!\langle P' \rangle, x \in \meaningof{a} P' \in \meaningof{E}\} }
\end{mathpar}

\begin{mathpar}
 \inferrule* [lab=nominal] {} {\meaningof{\quotep{E}} = \{ \quotep{P} \in \quotep{\pi} | P \in \meaningof{E} \}, \and \meaningof{\quotep{P}} = \{ \quotep{Q} \in \quotep{\pi} | P \equiv Q \} \and \\ \meaningof{@\quotep{E}} = \{ P \in \pi | P \equiv @x, x \in \meaningof{E} \}}
\end{mathpar}

\begin{eqnarray*}
  \\
  \meaningof{-} : TS \to ST
\end{eqnarray*}

\begin{eqnarray*}
  \\
  L : TS \to ST
\end{eqnarray*}

\begin{eqnarray*}
  \\
  P \models E \iff P \in \meaningof{E}
\end{eqnarray*}

\begin{eqnarray*}
  P \approx_{L} Q \iff \forall E \in L. P \models E \iff Q \models E
\end{eqnarray*}

\begin{eqnarray*}
  P \approx_{K} Q
\end{eqnarray*}

\begin{eqnarray*}
  P \approx Q
\end{eqnarray*}

$\approx_{K} = \approx = \approx_{L}$

\subsubsection{Contextual duality}

Note that contexts extend the quotation operation to a family of
operations from processes to names. Given a context, $M$, we can
define a \emph{nominal context}, $\quotep{M}$ by $\quotep{M}[P] :=
\quotep{M[P]}$. To foreshadow what is to come we observe that these
operations enjoy a duality with processes very much like the duality
between vectors and maps from vectors to scalars.

Further, because the calculus is essentially higher-order, we have a
correspondence between contexts and processes. More specifically,
given a name $x$ and a context $M$ we can construct $M^{*}_{x}$ such
that 

\begin{mathpar}
  M^{*}_{x} | \lift{x}{P} \red M[P]
\end{mathpar}

namely,

\begin{mathpar}
  M^{*}_{x} := x?(u).M[\dropn{u}]
\end{mathpar}

The dependence of $M^{*}_{x}$ on a name makes it an abstraction, 

\begin{mathpar}
  M^{*} := (x)x?(u).M[\dropn{u}]
\end{mathpar}

\subsection{Additional notation}

It will sometimes be convenient to denote the process a name
quotes. We already have the notation $x = \quotep{P}$, but it will be
convenient to introduce an alternate notation, $\procn{x}$, when we
want to emphasize the connection to the use of the name. Note that, by
virtue of name equivalence, $\quotep{\procn{x}} \nameeq x$; so, the
notation is consistent with previous definitions.

Further, because names have structure it is possible to effect
substitutions on the basis of that structure. This means we need to
upgrade our notation for substitutions, which we accomplish by
adapting comprehension notation. Thus,

\begin{mathpar}
  P\{ y / x : x \in S \}
\end{mathpar}

is interpreted to mean the process derived from P by replacing (in a
capture-avoiding manner) each occurrence of $x$ in $S$ by $y$. For example,

\begin{mathpar}
  P\{ \quotep{\procn{x}|\procn{x}} / x : x \in \freenames{P} \}
\end{mathpar}

will replace each (occurrence) of a free name $x$ in $P$ by
$\quotep{\procn{x}|\procn{x}}$.

Also, we will avail ourselves of the notation $x^{L}$ and $x^{R}$ to
denote injections of a name into disjoint copies of the name
space. There are numerous ways to accomplish this. One example can be
found in \cite{MeredithR05}. This notation overloads to vectors of
names: $\vec{x}^{\pi} := (x_{i}^{\pi} \; : \; 0 \leq i < |\vec{x}| )$ where $\pi \in \{L,R\}$.

We also use $P^{\Box} := P|\Box$.

In \cite{MeredithR05} an interpretation of the new operator is
given. It turns out that there are several possible interpretations
all enjoying the requisite algebraic properties of the operator (see
\cite{milner91polyadicpi}). We will therefore make liberal use of
$(\nu\; \vec{x})P$.

% subsection the_syntax_and_semantics_of_the_notation_system (end)   

\input{qm2pi.qmops} 

\input{qm2pi.sterngerlach} 

\input{qm2pi.metric} 

% section concurrent_process_calculi (end)

%\input{qm2pi.proofsketch}

% section proof sketch (end)

%\input{qm2pi.slviaknots} 

% section spatial logic via knots (end)

\input{qm2pi.conclusion}

% section conclusion (end)

%\input{qm2pi.dtcodes} 

% section wiring algorithm (end)

\input{qm2pi.ack} 

% section acknowledgments (end)

\newpage


\bibliographystyle{plain}   
\bibliography{../../biblios/main.bib}

\input{qm2pi.rhodetails}

\end{document}



\end{document}

 

%\documentclass[12pt]{llncs}
%\documentclass{jktr}

\usepackage[pdftex]{hyperref}                   
\usepackage {listings}
\usepackage {mathpartir}
\usepackage{bcprules}
%\usepackage{listings}
                       
\usepackage{graphicx} 
%\usepackage[margins=2.5cm,nohead,nofoot]{geometry}
%\usepackage{geometry}
\usepackage{amsfonts}
\usepackage{amstext}
\usepackage{latexsym}
\usepackage{amssymb}
\usepackage{color}


%\include{myPreamble}
\documentclass[12pt]{llncs}
%\documentclass{jktr}

\usepackage[pdftex]{hyperref}                   
\usepackage {listings}
\usepackage {mathpartir}
\usepackage{bcprules}
%\usepackage{listings}
                       
\usepackage{graphicx} 
%\usepackage[margins=2.5cm,nohead,nofoot]{geometry}
%\usepackage{geometry}
\usepackage{amsfonts}
\usepackage{amstext}
\usepackage{latexsym}
\usepackage{amssymb}
\usepackage{color}


%\include{myPreamble}
\include{qm2pi.local} 

%\ifpdf
%\usepackage[pdftex]{graphicx}
%\else
%\usepackage{graphicx}
%\fi

 % \ifpdf
%  \usepackage{pdfsync}
%  \if


%\title{Brief Article}
%\author{David F. Snyder}
%\author{L.G. Meredith}

%\address{Dept. of Math., Texas State University--San Marcos, San Marcos, TX 78666}
       
\pagestyle{empty}


\begin{document}

\lstset{language=[Objective]Caml,frame=shadowbox}

\input{qm2pi.front}

% section front matter (end)

\input{qm2pi.intro} 
 
% section introduction (end)

% \input{qm2pi.knotations} 

% section notation (end)

\input{qm2pi.process.calculi} 

% section concurrent_process_calculi_and_spatial_logics_ (end)
    
%\input{qm2pi.knots2pi} 

%\input{qm2pi.trefoil} 

%\input{qm2pi.mainthm} 

% subsection basic_interpretation (end)

%\input{qm2pi.rho.presentation} 
\subsection{The syntax and semantics of the notation system}\label{sub:the_syntax_and_semantics_of_the_notation_system} % (fold)

We now summarize a technical presentation of the calculus that
embodies our theory of dynamics. The typical presentation of such a
calculus follows the style of giving generators and relations on
them. The grammar, below, describing term constructors, freely
generates the set of processes, $\Proc$. This set is then quotiented
by a relation known as structural congruence and it is over this set
that the notion of dynamics is expressed. This presentation is
essentially that of \cite{MeredithR05} with the addition of
polyadicity and summation. For readability we have relegated some of
the technical subtleties to an appendix.

\subsubsection{Process grammar}\label{subsub:process_grammar}

\begin{mathpar}
  \inferrule* [lab=synchronization] {} {{M} \bc \pzero \;|\; x?F \;|\; x!C }
  \and
  \inferrule* [lab=abstraction] {} {{F} \bc (x)P}
  \and
  \inferrule* [lab=concretion] {} {{C} \bc \langle Q \rangle}
  \and
  \inferrule* [lab=process] {} {{P,Q} \bc M \;| \;P|Q \;|\; @{x}}
  \and
  \inferrule* [lab=name] {} {{x} \bc \quotep{P}}
\end{mathpar} 

Note that $\vec{x}$ (resp. $\vec{P}$) denotes a vector of names
(resp. processes) of length $|\vec{x}|$ (resp. $|\vec{P}|$). We adopt
the following useful abbreviations.

\begin{mathpar}
   x?(\vec{y}).P := x.(\vec{y})P \and  x\clift{\vec{P}} := x.\clift{\vec{P}}
   \and x!(y) := \lift{x}{\dropn{y}}
   \and \Pi_{i=0}^{n-1}P_i := P_0 | \ldots | P_{n-1}
\end{mathpar}

\subsubsection{Structural congruence}

\paragraph{Free and bound names and alpha-equivalence.} At the
core of structural equivalence is alpha-equivalence which identifies
process that are the same up to a change of variable. Formally, we
recognize the distinction between free and bound names. The free names
of a process, $\freenames{P}$, may be calculated recursively as
follows:

\begin{mathpar}
\freenames{\pzero} := \emptyset
  \and \\
  \freenames{x?(y).P} := \{ x \} \cup (\freenames{P} \setminus \{ y \})
  \and 
  \freenames{x!\langle P \rangle} := \{ x \} \cup \{ P \} 
  \and \\
  \freenames{P|Q} := \freenames{P} \cup \freenames{Q}
  \and \\
  \freenames{@{x}} := \{ x \}
\end{mathpar}

$\pi$
$\quotep{\pi}$

$\freenames{-} : \pi \to \mathcal{P}(\quotep{\pi})$

\begin{eqnarray*}
  \freenames{\pzero} & := & \emptyset \\
  \freenames{x?(y).P} & := & \{ x \} \cup (\freenames{P} \setminus \{ y \}) \\
  \freenames{x!\langle P \rangle} & := & \{ x \} \cup \{ P \} \\
  \freenames{P|Q} & := & \freenames{P} \cup \freenames{Q} \\
  \freenames{\dropn{x}} & := & \{ x \}
\end{eqnarray*}

The bound names of a process, $\boundnames{P}$, are those names occurring in $P$
that are not free. For example, in $x?(y).0$, the name $x$ is free, while $y$ is bound.

\begin{mathpar}
  \inferrule* [lab=monoidal-laws] {} { P|Q \equiv Q|P \and P|0 \equiv P \and P|(Q|R) \equiv (P|Q)|R }
\end{mathpar}

\begin{mathpar}
  \inferrule* [lab=alpha-equivalence] {} { (x)P \equiv (y)P\{y/x\} \and y \not\in \freenames{P} }
\end{mathpar}

\begin{definition}
Then two processes, $P,Q$, are alpha-equivalent if $P = Q\{\vec{y}/\vec{x}\}$ for
some $\vec{x} \in \boundnames{Q},\vec{y} \in \boundnames{P}$, where $Q\{\vec{y}/\vec{x}\}$
denotes the capture-avoiding substitution of $\vec{y}$ for $\vec{x}$ in $Q$.
\end{definition}

\begin{definition}
  The {\em structural congruence} \cite{SangiorgiWalker} , $\equiv$,
  between processes is the least congruence containing
  alpha-equivalence, satisfying the abelian monoid laws
  (associativity, commutativity and $\pzero$ as identity) for parallel
  composition $|$ and for summation $+$.
\end{definition}

\subsection{Name equivalence}

We take name equivalence, written $\nameeq$, to be the smallest
equivalence relation generated by the following rules.

\begin{mathpar}
\inferrule*[lab=Quote-drop]
{ }
{ \quotep{@{x}} \nameeq x }

\inferrule*[lab=Struct-equiv]
{ P \scong Q }
{ \quotep{P} \nameeq \quotep{Q} }
\end{mathpar}

The astute reader will have noticed that the mutual recursion of names
and processes imposes a mutual recursion on alpha-equivalence and
structural equivalence via name-equivalence. Fortunately, all of this
works out pleasantly and we may calculate in the natural way, free of
concern. The reader interested in the details is referred to the
appendix \ref{appendix:rho_details}.

\subsection{Substitution}

We use $\Proc$ for the set of processes, $\QProc$ for the set of
names, and $\id{\{}\vec{y} / \vec{x} \id{\}}$ to denote partial maps,
$s : \QProc \rightarrow \QProc$. A map, $s$ lifts, uniquely, to a map
on process terms, $\widehat{s} : \Proc \rightarrow \Proc$ by the
following equations.

\begin{mathpar}
  (0) \psubstp{Q}{P} := 0 \\
  (R \juxtap S) \psubstp{Q}{P}
  :=    
  (R)\psubstp{Q}{P} \juxtap (S) \psubstp{Q}{P} \\
  (x?(y).R) \psubstp{Q}{P}    
  :=    
  (x)\substp{Q}{P} (z)\concat( (R \psubstn{z}{y}) \psubstp{Q}{P} ) \\
  (\lift{x}{R}) \psubstp{Q}{P}  
  :=
  \lift{(x)\substp{Q}{P}}{ R \psubstp{Q}{P} } \\
%   (\dropn{x})  \psubstp{Q}{P}       
%   := 
%   \left\{ 
%     \begin{array}{ccc} 
%       \dropn{\quotep{Q}} & & x \nameeq \quotep{P} \\
%       \dropn{x} & & otherwise \\
%     \end{array}
%   \right. 
  (\dropn{x})  \psubstp{Q}{P}       
  := 
  \left\{ 
    \begin{array}{ccc} 
      Q & & x \nameeq \quotep{P} \\
      \dropn{x} & & otherwise \\
    \end{array}
  \right.
\end{mathpar}
 

where

\begin{eqnarray}
  (x)\id{\{} \lpquote Q \rpquote / \lpquote P \rpquote \id{\}}            = 
  \left\{ 
    \begin{array}{ccc}
      \lpquote Q \rpquote & & x \nameeq \lpquote P \rpquote \\
      x & & otherwise \\
    \end{array}
  \right. \nonumber
\end{eqnarray}

and $z$ is chosen distinct from $\quotep{P}$, $\quotep{Q}$, the free
names in $Q$, and all the names in $R$. Our $\alpha$-equivalence will
be built in the standard way from this substitution.

\begin{remark}\label{rem:no_self_referential_names}
  One consequence of these definitions is that $\forall P. \quotep{P}
  \not\in \freenames{P}$.
\end{remark}

\subsection{ Dynamic quote: an example }

Anticipating something of what's to come, consider applying the
substitution, $\widehat{\id{\{}u / z \id{\}}}$, to the following pair
of processes, $\lift{w}{y!(z)}$ and $w[ \lpquote y!(z) \rpquote ]$.

\begin{eqnarray}
	\lift{w}{y!(z)}\widehat{\id{\{}u / z \id{\}}}
		& = &
		\lift{w}{y!(u)} \nonumber\\
	w[ \lpquote y!(z) \rpquote ] \widehat{ \id{\{}u / z \id{\}} }
		& = &
		w[ \lpquote y!(z) \rpquote ] \nonumber
\end{eqnarray}

Because the body of the process between quotes is impervious to
substitution, we get radically different answers. In fact, by
examining the first process in an input context,
e.g. $x?(z).\lift{w}{y!(z)}$, we see that the process under the lift
operator may be shaped by prefixed inputs binding a name inside it. In
this sense, the lift operator will be seen as a way to dynamically
construct processes before reifying them as names.

Finally equipped with these standard features we can present the
dynamics of the calculus.

\subsubsection{Operational semantics} 

Finally, we introduce the computational dynamics. What marks these
algebras as distinct from other more traditionally studied algebraic
structures, e.g. vector spaces or polynomial rings, is the manner in
which dynamics is captured. In traditional structures, dynamics is typically
expressed through morphisms between such structures, as in linear maps
between vector spaces or morphisms between rings. In algebras
associated with the semantics of computation, the dynamics is
expressed as part of the algebraic structure itself, through a
reduction reduction relation typically denoted by $\red$. Below, we
give a recursive presentation of this relation for the calculus used
in the encoding.

$\red \subseteq \pi \times \pi$
$\red : \pi \to \mathcal{P}(\pi)$

\begin{mathpar}
  \inferrule* [lab=Comm] { \textsf{match}( x_{src}, x_{trgt} ) } { x_{trgt}?(y)P \; | \; x_{src}!\langle {Q} \rangle \red P\{\quotep{Q}/y}\} }
  \and \\
  \inferrule* [lab=Par] {{P} \red {P}'} {{{P} | {Q}} \red {{P}' | {Q}}}
  \and
  \inferrule* [lab=Equiv]{{{P} \scong {P}'} \andalso {{P}' \red {Q}'} \andalso {{Q}' \scong {Q}}}{{P} \red {Q}}
\end{mathpar}

\begin{eqnarray*}
  match_{\equiv} (\quotep{P},\quotep{Q}) & := & P \equiv Q \\
  match_{\dagger}(\quotep{P},\quotep{Q}) & := & \forall R. P|Q \red^{*} R => R \red^{*} 0 \\
  match_{K}(\quotep{P},\quotep{Q}) & := & K \mbox{ for some context } K
\end{eqnarray*}

$u?(x)P | u!\langle Q \rangle \red P\{\quotep{Q}/x\}$

%We write $\wred$ for $\red^*$, and $P\red$ if $\exists Q $ such that $ P \red Q$.
We write $P\red$ if $\exists Q $ such that $ P \red Q$ and $P\not\red$, otherwise.

\section{Replication}

As mentioned before, it is known that replication (and hence
recursion) can be implemented in a higher-order process algebra
\cite{SangiorgiWalker}. As our first example of calculation with the
machinery thus far presented we give the construction explicitly in
the {\rhoc}.

\begin{eqnarray}
	D_{x} & := & \prefix{x}{y}{(\binpar{\outputp{x}{y}}{@{y}})} \nonumber\\
	\bangp_{x}{P} & := & \binpar{{x}!\langle{\binpar{D_{x}}{P}}\rangle}{D_{x}} \nonumber
\end{eqnarray}

\begin{eqnarray}
	\bangp_{x}{P} & & \nonumber\\
	=
	& {x}!\langle{(\prefix{x}{y}{(\outputp{x}{y} | @{y})) | P}}\rangle 
	      | \prefix{x}{y}{(\outputp{x}{y} | @{y})} & \nonumber\\
	\red
	& (\outputp{x}{y} | @{y})\substn{\quotep{(\prefix{x}{y}{(@{y} | \outputp{x}{y})) | P}}}{y} & \nonumber\\
	=
	& \outputp{x}{\quotep{(\prefix{x}{y}{(\outputp{x}{y} | @{y})) | P}}}
	  | {(\prefix{x}{y}{(\outputp{x}{y} | @{y})) | P}} & \nonumber\\
	\red
	& \ldots & \nonumber\\
	\red^*
	& P | P | \ldots & \nonumber
\end{eqnarray}

Of course, this encoding, as an implementation, runs away, unfolding
$\bangp{P}$ eagerly. A lazier and more implementable replication
operator, restricted to input-guarded processes, may be obtained as follows.

\begin{eqnarray}
\bangp{\prefix{u}{v}{P}} 
	:= 
	\binpar{\lift{x}{\prefix{u}{v}{(\binpar{D(x)}{P})}}}{D(x)} \nonumber
\end{eqnarray}

\begin{remark}
  Note that the lazier definition still does not deal with summation
  or mixed summation (i.e. sums over input and output). The reader is
  invited to construct definitions of replication that deal with these
  features. 

  Further, the definitions are parameterized in a name, $x$. Can you,
  gentle reader, make a definition that eliminates this parameter and
  guarantees no accidental interaction between the replication
  machinery and the process being replicated -- i.e. no accidental
  sharing of names used by the process to get its work done and the
  name(s) used by the replication to effect copying. This latter
  revision of the definition of replication is crucial to obtaining
  the expected identity $!!P \sim !P$.
\end{remark}

\begin{remark}\label{rem:paradoxical_combinator}
  The reader familiar with the lambda calculus will have noticed the
  similarity between $D$ and the paradoxical combinator.

  [Ed. note: the existence of this seems to suggest we have to be more
  restrictive on the set of processes and names we admit if we are to
  support no-cloning.]
\end{remark}

\subsubsection{Bisimulation}

The computational dynamics gives rise to another kind of equivalence,
the equivalence of computational behavior. As previously mentioned
this is typically captured \emph{via} some form of bisimulation.

% The notion we use in this paper is weak barbed bisimulation
% \cite{milner91polyadicpi}.

The notion we use in this paper is derived from weak barbed
bisimulation \cite{milner91polyadicpi}. 

\begin{definition}
An \emph{observation relation}, $\downarrow_{\mathcal N}$, over a set
of names, $\mathcal N$, is the smallest relation satisfying the rules
below.

\infrule[Out-barb]{y \in {\mathcal N}, \; x \nameeq y}
		  {\outputp{x}{v} \downarrow_{\mathcal N} x}
\infrule[Par-barb]{\mbox{$P\downarrow_{\mathcal N} x$ or $Q\downarrow_{\mathcal N} x$}}
		  {\binpar{P}{Q} \downarrow_{\mathcal N} x}

We write $P \Downarrow_{\mathcal N} x$ if there is $Q$ such that 
$P \wred Q$ and $Q \downarrow_{\mathcal N} x$.
\end{definition}

\begin{definition}
%\label{def.bbisim}
An  ${\mathcal N}$-\emph{barbed bisimulation} over a set of names, ${\mathcal N}$, is a symmetric binary relation 
${\mathcal S}_{\mathcal N}$ between agents such that $P\rel{S}_{\mathcal N}Q$ implies:
\begin{enumerate}
\item If $P \red P'$ then $Q \wred Q'$ and $P'\rel{S}_{\mathcal N} Q'$.
\item If $P\downarrow_{\mathcal N} x$, then $Q\Downarrow_{\mathcal N} x$.
\end{enumerate}
$P$ is ${\mathcal N}$-barbed bisimilar to $Q$, written
$P \wbbisim_{\mathcal N} Q$, if $P \rel{S}_{\mathcal N} Q$ for some ${\mathcal N}$-barbed bisimulation ${\mathcal S}_{\mathcal N}$.
\end{definition}

$\mathcal{R} \subseteq \pi \times \pi$

$P \mathcal{R} Q => \forall P'. P \red P' \Rightarrow \exists Q'. Q \red Q', P' \mathcal{R} Q'$

$P \vdash x \Rightarrow Q \vdash x$

\begin{mathpar}
  \inferrule*[lab=Out-barb]{x \nameeq y}{{y}!\langle{Q}\rangle \vdash x}
  \and
  \inferrule*[lab=Par-barb]{\mbox{$P\vdash x$ or $Q\vdash x$}}{\binpar{P}{Q} \vdash x}
\end{mathpar}

\subsubsection{Contexts}

One of the principle advantages of computational calculi like the
$\pi$-calculus is a well-defined notion of context,
contextual-equivalence and a correlation between
contextual-equivalence and notions of bisimulation. The notion of
context allows the decomposition of a process into (sub-)process and
its syntactic environment, its context. Thus, a context may be
thought of as a process with a ``hole'' (written $\Box$) in it. The
application of a context $M$ to a process $P$, written $M[P]$, is
tantamount to filling the hole in $M$ with $P$. In this paper we do
not need the full weight of this theory, but do make use of the notion
of context in the proof the main theorem. 

\begin{mathpar}
  \inferrule* [lab=summation] {} {{M_{M},M_{N}} \bc \Box \;|\; x.M_{A} \;|\; M_{M}+M_{N}}
  \and
  \inferrule* [lab=agent] {} {{M_{A}} \bc (\vec{x})M_{P} \;| \; \clift{P_0,\ldots,M_{P},\ldots,P_N}}
  \and \\
  \inferrule* [lab=process] {} {{M_{P}} \bc M_{N} \;| \;P|M_{P} }
\end{mathpar} 

\begin{mathpar}
  \inferrule* [lab=sychronization] {} {M_{N} \bc \Box \;|\; x?M_{F} \;|\; x!M_{C}}
  \and
  \inferrule* [lab=abstraction] {} {{M_{F}} \bc (x)M_{P} }
  \and
  \inferrule* [lab=concretion] {} {{M_{C}} \bc \langle M_{P} \rangle }
  \and \\
  \inferrule* [lab=process] {} {{M_{P}} \bc M_{N} \;| \;P|M_{P} }
\end{mathpar}

\begin{definition}[contextual application] Given a context $M$, and
  process $P$, we define the \emph{contextual application}, $M[P] :=
  M\{P/\Box\}$. That is, the contextual application of M to P is the
  substitution of $P$ for $\Box$ in $M$.
\end{definition}

$\meaningof{-} : L \to \mathcal{P}(\pi)$

\begin{mathpar}
  \inferrule* [lab=collection] {} {\meaningof{true} = \pi, \and \meaningof{~E} = \pi \setminus \meaningof{E}, \and \meaningof{E_{1} \& E_{2}} = \meaningof{E_{1}} \cap \meaningof{E_{2}}}
\end{mathpar}

\begin{mathpar}
  \inferrule* [lab=structure] {} {\meaningof{0} = \{ P \in \pi | P \equiv 0 \}, \and \\ \meaningof{E_1 | E_2} = \{ P \in \pi | P \equiv P_{1} | P_{2}, P_{1} \in \meaningof{E_{1}}, P_{2} \in \meaningof{E_2}\} }
\end{mathpar}

\begin{mathpar}
 \inferrule* [lab=behavior] {} {\meaningof{\langle a?b \rangle E} = \{ P \in \pi | P \equiv Q | u?(y)P', \\ \and \\\\ \and \\ \;\;\; u \in \meaningof{a}, \forall z.P'\{z/y\} \in \meaningof{E\{z/b\}}\}, \and \\ \meaningof{a!E} = \{ P \in \pi | P \equiv Q | x!\langle P' \rangle, x \in \meaningof{a} P' \in \meaningof{E}\} }
\end{mathpar}

\begin{mathpar}
 \inferrule* [lab=nominal] {} {\meaningof{\quotep{E}} = \{ \quotep{P} \in \quotep{\pi} | P \in \meaningof{E} \}, \and \meaningof{\quotep{P}} = \{ \quotep{Q} \in \quotep{\pi} | P \equiv Q \} \and \\ \meaningof{@\quotep{E}} = \{ P \in \pi | P \equiv @x, x \in \meaningof{E} \}}
\end{mathpar}

\begin{eqnarray*}
  \\
  \meaningof{-} : TS \to ST
\end{eqnarray*}

\begin{eqnarray*}
  \\
  L : TS \to ST
\end{eqnarray*}

\begin{eqnarray*}
  \\
  P \models E \iff P \in \meaningof{E}
\end{eqnarray*}

\begin{eqnarray*}
  P \approx_{L} Q \iff \forall E \in L. P \models E \iff Q \models E
\end{eqnarray*}

\begin{eqnarray*}
  P \approx_{K} Q
\end{eqnarray*}

\begin{eqnarray*}
  P \approx Q
\end{eqnarray*}

$\approx_{K} = \approx = \approx_{L}$

\subsubsection{Contextual duality}

Note that contexts extend the quotation operation to a family of
operations from processes to names. Given a context, $M$, we can
define a \emph{nominal context}, $\quotep{M}$ by $\quotep{M}[P] :=
\quotep{M[P]}$. To foreshadow what is to come we observe that these
operations enjoy a duality with processes very much like the duality
between vectors and maps from vectors to scalars.

Further, because the calculus is essentially higher-order, we have a
correspondence between contexts and processes. More specifically,
given a name $x$ and a context $M$ we can construct $M^{*}_{x}$ such
that 

\begin{mathpar}
  M^{*}_{x} | \lift{x}{P} \red M[P]
\end{mathpar}

namely,

\begin{mathpar}
  M^{*}_{x} := x?(u).M[\dropn{u}]
\end{mathpar}

The dependence of $M^{*}_{x}$ on a name makes it an abstraction, 

\begin{mathpar}
  M^{*} := (x)x?(u).M[\dropn{u}]
\end{mathpar}

\subsection{Additional notation}

It will sometimes be convenient to denote the process a name
quotes. We already have the notation $x = \quotep{P}$, but it will be
convenient to introduce an alternate notation, $\procn{x}$, when we
want to emphasize the connection to the use of the name. Note that, by
virtue of name equivalence, $\quotep{\procn{x}} \nameeq x$; so, the
notation is consistent with previous definitions.

Further, because names have structure it is possible to effect
substitutions on the basis of that structure. This means we need to
upgrade our notation for substitutions, which we accomplish by
adapting comprehension notation. Thus,

\begin{mathpar}
  P\{ y / x : x \in S \}
\end{mathpar}

is interpreted to mean the process derived from P by replacing (in a
capture-avoiding manner) each occurrence of $x$ in $S$ by $y$. For example,

\begin{mathpar}
  P\{ \quotep{\procn{x}|\procn{x}} / x : x \in \freenames{P} \}
\end{mathpar}

will replace each (occurrence) of a free name $x$ in $P$ by
$\quotep{\procn{x}|\procn{x}}$.

Also, we will avail ourselves of the notation $x^{L}$ and $x^{R}$ to
denote injections of a name into disjoint copies of the name
space. There are numerous ways to accomplish this. One example can be
found in \cite{MeredithR05}. This notation overloads to vectors of
names: $\vec{x}^{\pi} := (x_{i}^{\pi} \; : \; 0 \leq i < |\vec{x}| )$ where $\pi \in \{L,R\}$.

We also use $P^{\Box} := P|\Box$.

In \cite{MeredithR05} an interpretation of the new operator is
given. It turns out that there are several possible interpretations
all enjoying the requisite algebraic properties of the operator (see
\cite{milner91polyadicpi}). We will therefore make liberal use of
$(\nu\; \vec{x})P$.

% subsection the_syntax_and_semantics_of_the_notation_system (end)   

\input{qm2pi.qmops} 

\input{qm2pi.sterngerlach} 

\input{qm2pi.metric} 

% section concurrent_process_calculi (end)

%\input{qm2pi.proofsketch}

% section proof sketch (end)

%\input{qm2pi.slviaknots} 

% section spatial logic via knots (end)

\input{qm2pi.conclusion}

% section conclusion (end)

%\input{qm2pi.dtcodes} 

% section wiring algorithm (end)

\input{qm2pi.ack} 

% section acknowledgments (end)

\newpage


\bibliographystyle{plain}   
\bibliography{../../biblios/main.bib}

\input{qm2pi.rhodetails}

\end{document}

 

%\ifpdf
%\usepackage[pdftex]{graphicx}
%\else
%\usepackage{graphicx}
%\fi

 % \ifpdf
%  \usepackage{pdfsync}
%  \if


%\title{Brief Article}
%\author{David F. Snyder}
%\author{L.G. Meredith}

%\address{Dept. of Math., Texas State University--San Marcos, San Marcos, TX 78666}
       
\pagestyle{empty}


\begin{document}

\lstset{language=[Objective]Caml,frame=shadowbox}

\documentclass[12pt]{llncs}
%\documentclass{jktr}

\usepackage[pdftex]{hyperref}                   
\usepackage {listings}
\usepackage {mathpartir}
\usepackage{bcprules}
%\usepackage{listings}
                       
\usepackage{graphicx} 
%\usepackage[margins=2.5cm,nohead,nofoot]{geometry}
%\usepackage{geometry}
\usepackage{amsfonts}
\usepackage{amstext}
\usepackage{latexsym}
\usepackage{amssymb}
\usepackage{color}


%\include{myPreamble}
\include{qm2pi.local} 

%\ifpdf
%\usepackage[pdftex]{graphicx}
%\else
%\usepackage{graphicx}
%\fi

 % \ifpdf
%  \usepackage{pdfsync}
%  \if


%\title{Brief Article}
%\author{David F. Snyder}
%\author{L.G. Meredith}

%\address{Dept. of Math., Texas State University--San Marcos, San Marcos, TX 78666}
       
\pagestyle{empty}


\begin{document}

\lstset{language=[Objective]Caml,frame=shadowbox}

\input{qm2pi.front}

% section front matter (end)

\input{qm2pi.intro} 
 
% section introduction (end)

% \input{qm2pi.knotations} 

% section notation (end)

\input{qm2pi.process.calculi} 

% section concurrent_process_calculi_and_spatial_logics_ (end)
    
%\input{qm2pi.knots2pi} 

%\input{qm2pi.trefoil} 

%\input{qm2pi.mainthm} 

% subsection basic_interpretation (end)

%\input{qm2pi.rho.presentation} 
\subsection{The syntax and semantics of the notation system}\label{sub:the_syntax_and_semantics_of_the_notation_system} % (fold)

We now summarize a technical presentation of the calculus that
embodies our theory of dynamics. The typical presentation of such a
calculus follows the style of giving generators and relations on
them. The grammar, below, describing term constructors, freely
generates the set of processes, $\Proc$. This set is then quotiented
by a relation known as structural congruence and it is over this set
that the notion of dynamics is expressed. This presentation is
essentially that of \cite{MeredithR05} with the addition of
polyadicity and summation. For readability we have relegated some of
the technical subtleties to an appendix.

\subsubsection{Process grammar}\label{subsub:process_grammar}

\begin{mathpar}
  \inferrule* [lab=synchronization] {} {{M} \bc \pzero \;|\; x?F \;|\; x!C }
  \and
  \inferrule* [lab=abstraction] {} {{F} \bc (x)P}
  \and
  \inferrule* [lab=concretion] {} {{C} \bc \langle Q \rangle}
  \and
  \inferrule* [lab=process] {} {{P,Q} \bc M \;| \;P|Q \;|\; @{x}}
  \and
  \inferrule* [lab=name] {} {{x} \bc \quotep{P}}
\end{mathpar} 

Note that $\vec{x}$ (resp. $\vec{P}$) denotes a vector of names
(resp. processes) of length $|\vec{x}|$ (resp. $|\vec{P}|$). We adopt
the following useful abbreviations.

\begin{mathpar}
   x?(\vec{y}).P := x.(\vec{y})P \and  x\clift{\vec{P}} := x.\clift{\vec{P}}
   \and x!(y) := \lift{x}{\dropn{y}}
   \and \Pi_{i=0}^{n-1}P_i := P_0 | \ldots | P_{n-1}
\end{mathpar}

\subsubsection{Structural congruence}

\paragraph{Free and bound names and alpha-equivalence.} At the
core of structural equivalence is alpha-equivalence which identifies
process that are the same up to a change of variable. Formally, we
recognize the distinction between free and bound names. The free names
of a process, $\freenames{P}$, may be calculated recursively as
follows:

\begin{mathpar}
\freenames{\pzero} := \emptyset
  \and \\
  \freenames{x?(y).P} := \{ x \} \cup (\freenames{P} \setminus \{ y \})
  \and 
  \freenames{x!\langle P \rangle} := \{ x \} \cup \{ P \} 
  \and \\
  \freenames{P|Q} := \freenames{P} \cup \freenames{Q}
  \and \\
  \freenames{@{x}} := \{ x \}
\end{mathpar}

$\pi$
$\quotep{\pi}$

$\freenames{-} : \pi \to \mathcal{P}(\quotep{\pi})$

\begin{eqnarray*}
  \freenames{\pzero} & := & \emptyset \\
  \freenames{x?(y).P} & := & \{ x \} \cup (\freenames{P} \setminus \{ y \}) \\
  \freenames{x!\langle P \rangle} & := & \{ x \} \cup \{ P \} \\
  \freenames{P|Q} & := & \freenames{P} \cup \freenames{Q} \\
  \freenames{\dropn{x}} & := & \{ x \}
\end{eqnarray*}

The bound names of a process, $\boundnames{P}$, are those names occurring in $P$
that are not free. For example, in $x?(y).0$, the name $x$ is free, while $y$ is bound.

\begin{mathpar}
  \inferrule* [lab=monoidal-laws] {} { P|Q \equiv Q|P \and P|0 \equiv P \and P|(Q|R) \equiv (P|Q)|R }
\end{mathpar}

\begin{mathpar}
  \inferrule* [lab=alpha-equivalence] {} { (x)P \equiv (y)P\{y/x\} \and y \not\in \freenames{P} }
\end{mathpar}

\begin{definition}
Then two processes, $P,Q$, are alpha-equivalent if $P = Q\{\vec{y}/\vec{x}\}$ for
some $\vec{x} \in \boundnames{Q},\vec{y} \in \boundnames{P}$, where $Q\{\vec{y}/\vec{x}\}$
denotes the capture-avoiding substitution of $\vec{y}$ for $\vec{x}$ in $Q$.
\end{definition}

\begin{definition}
  The {\em structural congruence} \cite{SangiorgiWalker} , $\equiv$,
  between processes is the least congruence containing
  alpha-equivalence, satisfying the abelian monoid laws
  (associativity, commutativity and $\pzero$ as identity) for parallel
  composition $|$ and for summation $+$.
\end{definition}

\subsection{Name equivalence}

We take name equivalence, written $\nameeq$, to be the smallest
equivalence relation generated by the following rules.

\begin{mathpar}
\inferrule*[lab=Quote-drop]
{ }
{ \quotep{@{x}} \nameeq x }

\inferrule*[lab=Struct-equiv]
{ P \scong Q }
{ \quotep{P} \nameeq \quotep{Q} }
\end{mathpar}

The astute reader will have noticed that the mutual recursion of names
and processes imposes a mutual recursion on alpha-equivalence and
structural equivalence via name-equivalence. Fortunately, all of this
works out pleasantly and we may calculate in the natural way, free of
concern. The reader interested in the details is referred to the
appendix \ref{appendix:rho_details}.

\subsection{Substitution}

We use $\Proc$ for the set of processes, $\QProc$ for the set of
names, and $\id{\{}\vec{y} / \vec{x} \id{\}}$ to denote partial maps,
$s : \QProc \rightarrow \QProc$. A map, $s$ lifts, uniquely, to a map
on process terms, $\widehat{s} : \Proc \rightarrow \Proc$ by the
following equations.

\begin{mathpar}
  (0) \psubstp{Q}{P} := 0 \\
  (R \juxtap S) \psubstp{Q}{P}
  :=    
  (R)\psubstp{Q}{P} \juxtap (S) \psubstp{Q}{P} \\
  (x?(y).R) \psubstp{Q}{P}    
  :=    
  (x)\substp{Q}{P} (z)\concat( (R \psubstn{z}{y}) \psubstp{Q}{P} ) \\
  (\lift{x}{R}) \psubstp{Q}{P}  
  :=
  \lift{(x)\substp{Q}{P}}{ R \psubstp{Q}{P} } \\
%   (\dropn{x})  \psubstp{Q}{P}       
%   := 
%   \left\{ 
%     \begin{array}{ccc} 
%       \dropn{\quotep{Q}} & & x \nameeq \quotep{P} \\
%       \dropn{x} & & otherwise \\
%     \end{array}
%   \right. 
  (\dropn{x})  \psubstp{Q}{P}       
  := 
  \left\{ 
    \begin{array}{ccc} 
      Q & & x \nameeq \quotep{P} \\
      \dropn{x} & & otherwise \\
    \end{array}
  \right.
\end{mathpar}
 

where

\begin{eqnarray}
  (x)\id{\{} \lpquote Q \rpquote / \lpquote P \rpquote \id{\}}            = 
  \left\{ 
    \begin{array}{ccc}
      \lpquote Q \rpquote & & x \nameeq \lpquote P \rpquote \\
      x & & otherwise \\
    \end{array}
  \right. \nonumber
\end{eqnarray}

and $z$ is chosen distinct from $\quotep{P}$, $\quotep{Q}$, the free
names in $Q$, and all the names in $R$. Our $\alpha$-equivalence will
be built in the standard way from this substitution.

\begin{remark}\label{rem:no_self_referential_names}
  One consequence of these definitions is that $\forall P. \quotep{P}
  \not\in \freenames{P}$.
\end{remark}

\subsection{ Dynamic quote: an example }

Anticipating something of what's to come, consider applying the
substitution, $\widehat{\id{\{}u / z \id{\}}}$, to the following pair
of processes, $\lift{w}{y!(z)}$ and $w[ \lpquote y!(z) \rpquote ]$.

\begin{eqnarray}
	\lift{w}{y!(z)}\widehat{\id{\{}u / z \id{\}}}
		& = &
		\lift{w}{y!(u)} \nonumber\\
	w[ \lpquote y!(z) \rpquote ] \widehat{ \id{\{}u / z \id{\}} }
		& = &
		w[ \lpquote y!(z) \rpquote ] \nonumber
\end{eqnarray}

Because the body of the process between quotes is impervious to
substitution, we get radically different answers. In fact, by
examining the first process in an input context,
e.g. $x?(z).\lift{w}{y!(z)}$, we see that the process under the lift
operator may be shaped by prefixed inputs binding a name inside it. In
this sense, the lift operator will be seen as a way to dynamically
construct processes before reifying them as names.

Finally equipped with these standard features we can present the
dynamics of the calculus.

\subsubsection{Operational semantics} 

Finally, we introduce the computational dynamics. What marks these
algebras as distinct from other more traditionally studied algebraic
structures, e.g. vector spaces or polynomial rings, is the manner in
which dynamics is captured. In traditional structures, dynamics is typically
expressed through morphisms between such structures, as in linear maps
between vector spaces or morphisms between rings. In algebras
associated with the semantics of computation, the dynamics is
expressed as part of the algebraic structure itself, through a
reduction reduction relation typically denoted by $\red$. Below, we
give a recursive presentation of this relation for the calculus used
in the encoding.

$\red \subseteq \pi \times \pi$
$\red : \pi \to \mathcal{P}(\pi)$

\begin{mathpar}
  \inferrule* [lab=Comm] { \textsf{match}( x_{src}, x_{trgt} ) } { x_{trgt}?(y)P \; | \; x_{src}!\langle {Q} \rangle \red P\{\quotep{Q}/y}\} }
  \and \\
  \inferrule* [lab=Par] {{P} \red {P}'} {{{P} | {Q}} \red {{P}' | {Q}}}
  \and
  \inferrule* [lab=Equiv]{{{P} \scong {P}'} \andalso {{P}' \red {Q}'} \andalso {{Q}' \scong {Q}}}{{P} \red {Q}}
\end{mathpar}

\begin{eqnarray*}
  match_{\equiv} (\quotep{P},\quotep{Q}) & := & P \equiv Q \\
  match_{\dagger}(\quotep{P},\quotep{Q}) & := & \forall R. P|Q \red^{*} R => R \red^{*} 0 \\
  match_{K}(\quotep{P},\quotep{Q}) & := & K \mbox{ for some context } K
\end{eqnarray*}

$u?(x)P | u!\langle Q \rangle \red P\{\quotep{Q}/x\}$

%We write $\wred$ for $\red^*$, and $P\red$ if $\exists Q $ such that $ P \red Q$.
We write $P\red$ if $\exists Q $ such that $ P \red Q$ and $P\not\red$, otherwise.

\section{Replication}

As mentioned before, it is known that replication (and hence
recursion) can be implemented in a higher-order process algebra
\cite{SangiorgiWalker}. As our first example of calculation with the
machinery thus far presented we give the construction explicitly in
the {\rhoc}.

\begin{eqnarray}
	D_{x} & := & \prefix{x}{y}{(\binpar{\outputp{x}{y}}{@{y}})} \nonumber\\
	\bangp_{x}{P} & := & \binpar{{x}!\langle{\binpar{D_{x}}{P}}\rangle}{D_{x}} \nonumber
\end{eqnarray}

\begin{eqnarray}
	\bangp_{x}{P} & & \nonumber\\
	=
	& {x}!\langle{(\prefix{x}{y}{(\outputp{x}{y} | @{y})) | P}}\rangle 
	      | \prefix{x}{y}{(\outputp{x}{y} | @{y})} & \nonumber\\
	\red
	& (\outputp{x}{y} | @{y})\substn{\quotep{(\prefix{x}{y}{(@{y} | \outputp{x}{y})) | P}}}{y} & \nonumber\\
	=
	& \outputp{x}{\quotep{(\prefix{x}{y}{(\outputp{x}{y} | @{y})) | P}}}
	  | {(\prefix{x}{y}{(\outputp{x}{y} | @{y})) | P}} & \nonumber\\
	\red
	& \ldots & \nonumber\\
	\red^*
	& P | P | \ldots & \nonumber
\end{eqnarray}

Of course, this encoding, as an implementation, runs away, unfolding
$\bangp{P}$ eagerly. A lazier and more implementable replication
operator, restricted to input-guarded processes, may be obtained as follows.

\begin{eqnarray}
\bangp{\prefix{u}{v}{P}} 
	:= 
	\binpar{\lift{x}{\prefix{u}{v}{(\binpar{D(x)}{P})}}}{D(x)} \nonumber
\end{eqnarray}

\begin{remark}
  Note that the lazier definition still does not deal with summation
  or mixed summation (i.e. sums over input and output). The reader is
  invited to construct definitions of replication that deal with these
  features. 

  Further, the definitions are parameterized in a name, $x$. Can you,
  gentle reader, make a definition that eliminates this parameter and
  guarantees no accidental interaction between the replication
  machinery and the process being replicated -- i.e. no accidental
  sharing of names used by the process to get its work done and the
  name(s) used by the replication to effect copying. This latter
  revision of the definition of replication is crucial to obtaining
  the expected identity $!!P \sim !P$.
\end{remark}

\begin{remark}\label{rem:paradoxical_combinator}
  The reader familiar with the lambda calculus will have noticed the
  similarity between $D$ and the paradoxical combinator.

  [Ed. note: the existence of this seems to suggest we have to be more
  restrictive on the set of processes and names we admit if we are to
  support no-cloning.]
\end{remark}

\subsubsection{Bisimulation}

The computational dynamics gives rise to another kind of equivalence,
the equivalence of computational behavior. As previously mentioned
this is typically captured \emph{via} some form of bisimulation.

% The notion we use in this paper is weak barbed bisimulation
% \cite{milner91polyadicpi}.

The notion we use in this paper is derived from weak barbed
bisimulation \cite{milner91polyadicpi}. 

\begin{definition}
An \emph{observation relation}, $\downarrow_{\mathcal N}$, over a set
of names, $\mathcal N$, is the smallest relation satisfying the rules
below.

\infrule[Out-barb]{y \in {\mathcal N}, \; x \nameeq y}
		  {\outputp{x}{v} \downarrow_{\mathcal N} x}
\infrule[Par-barb]{\mbox{$P\downarrow_{\mathcal N} x$ or $Q\downarrow_{\mathcal N} x$}}
		  {\binpar{P}{Q} \downarrow_{\mathcal N} x}

We write $P \Downarrow_{\mathcal N} x$ if there is $Q$ such that 
$P \wred Q$ and $Q \downarrow_{\mathcal N} x$.
\end{definition}

\begin{definition}
%\label{def.bbisim}
An  ${\mathcal N}$-\emph{barbed bisimulation} over a set of names, ${\mathcal N}$, is a symmetric binary relation 
${\mathcal S}_{\mathcal N}$ between agents such that $P\rel{S}_{\mathcal N}Q$ implies:
\begin{enumerate}
\item If $P \red P'$ then $Q \wred Q'$ and $P'\rel{S}_{\mathcal N} Q'$.
\item If $P\downarrow_{\mathcal N} x$, then $Q\Downarrow_{\mathcal N} x$.
\end{enumerate}
$P$ is ${\mathcal N}$-barbed bisimilar to $Q$, written
$P \wbbisim_{\mathcal N} Q$, if $P \rel{S}_{\mathcal N} Q$ for some ${\mathcal N}$-barbed bisimulation ${\mathcal S}_{\mathcal N}$.
\end{definition}

$\mathcal{R} \subseteq \pi \times \pi$

$P \mathcal{R} Q => \forall P'. P \red P' \Rightarrow \exists Q'. Q \red Q', P' \mathcal{R} Q'$

$P \vdash x \Rightarrow Q \vdash x$

\begin{mathpar}
  \inferrule*[lab=Out-barb]{x \nameeq y}{{y}!\langle{Q}\rangle \vdash x}
  \and
  \inferrule*[lab=Par-barb]{\mbox{$P\vdash x$ or $Q\vdash x$}}{\binpar{P}{Q} \vdash x}
\end{mathpar}

\subsubsection{Contexts}

One of the principle advantages of computational calculi like the
$\pi$-calculus is a well-defined notion of context,
contextual-equivalence and a correlation between
contextual-equivalence and notions of bisimulation. The notion of
context allows the decomposition of a process into (sub-)process and
its syntactic environment, its context. Thus, a context may be
thought of as a process with a ``hole'' (written $\Box$) in it. The
application of a context $M$ to a process $P$, written $M[P]$, is
tantamount to filling the hole in $M$ with $P$. In this paper we do
not need the full weight of this theory, but do make use of the notion
of context in the proof the main theorem. 

\begin{mathpar}
  \inferrule* [lab=summation] {} {{M_{M},M_{N}} \bc \Box \;|\; x.M_{A} \;|\; M_{M}+M_{N}}
  \and
  \inferrule* [lab=agent] {} {{M_{A}} \bc (\vec{x})M_{P} \;| \; \clift{P_0,\ldots,M_{P},\ldots,P_N}}
  \and \\
  \inferrule* [lab=process] {} {{M_{P}} \bc M_{N} \;| \;P|M_{P} }
\end{mathpar} 

\begin{mathpar}
  \inferrule* [lab=sychronization] {} {M_{N} \bc \Box \;|\; x?M_{F} \;|\; x!M_{C}}
  \and
  \inferrule* [lab=abstraction] {} {{M_{F}} \bc (x)M_{P} }
  \and
  \inferrule* [lab=concretion] {} {{M_{C}} \bc \langle M_{P} \rangle }
  \and \\
  \inferrule* [lab=process] {} {{M_{P}} \bc M_{N} \;| \;P|M_{P} }
\end{mathpar}

\begin{definition}[contextual application] Given a context $M$, and
  process $P$, we define the \emph{contextual application}, $M[P] :=
  M\{P/\Box\}$. That is, the contextual application of M to P is the
  substitution of $P$ for $\Box$ in $M$.
\end{definition}

$\meaningof{-} : L \to \mathcal{P}(\pi)$

\begin{mathpar}
  \inferrule* [lab=collection] {} {\meaningof{true} = \pi, \and \meaningof{~E} = \pi \setminus \meaningof{E}, \and \meaningof{E_{1} \& E_{2}} = \meaningof{E_{1}} \cap \meaningof{E_{2}}}
\end{mathpar}

\begin{mathpar}
  \inferrule* [lab=structure] {} {\meaningof{0} = \{ P \in \pi | P \equiv 0 \}, \and \\ \meaningof{E_1 | E_2} = \{ P \in \pi | P \equiv P_{1} | P_{2}, P_{1} \in \meaningof{E_{1}}, P_{2} \in \meaningof{E_2}\} }
\end{mathpar}

\begin{mathpar}
 \inferrule* [lab=behavior] {} {\meaningof{\langle a?b \rangle E} = \{ P \in \pi | P \equiv Q | u?(y)P', \\ \and \\\\ \and \\ \;\;\; u \in \meaningof{a}, \forall z.P'\{z/y\} \in \meaningof{E\{z/b\}}\}, \and \\ \meaningof{a!E} = \{ P \in \pi | P \equiv Q | x!\langle P' \rangle, x \in \meaningof{a} P' \in \meaningof{E}\} }
\end{mathpar}

\begin{mathpar}
 \inferrule* [lab=nominal] {} {\meaningof{\quotep{E}} = \{ \quotep{P} \in \quotep{\pi} | P \in \meaningof{E} \}, \and \meaningof{\quotep{P}} = \{ \quotep{Q} \in \quotep{\pi} | P \equiv Q \} \and \\ \meaningof{@\quotep{E}} = \{ P \in \pi | P \equiv @x, x \in \meaningof{E} \}}
\end{mathpar}

\begin{eqnarray*}
  \\
  \meaningof{-} : TS \to ST
\end{eqnarray*}

\begin{eqnarray*}
  \\
  L : TS \to ST
\end{eqnarray*}

\begin{eqnarray*}
  \\
  P \models E \iff P \in \meaningof{E}
\end{eqnarray*}

\begin{eqnarray*}
  P \approx_{L} Q \iff \forall E \in L. P \models E \iff Q \models E
\end{eqnarray*}

\begin{eqnarray*}
  P \approx_{K} Q
\end{eqnarray*}

\begin{eqnarray*}
  P \approx Q
\end{eqnarray*}

$\approx_{K} = \approx = \approx_{L}$

\subsubsection{Contextual duality}

Note that contexts extend the quotation operation to a family of
operations from processes to names. Given a context, $M$, we can
define a \emph{nominal context}, $\quotep{M}$ by $\quotep{M}[P] :=
\quotep{M[P]}$. To foreshadow what is to come we observe that these
operations enjoy a duality with processes very much like the duality
between vectors and maps from vectors to scalars.

Further, because the calculus is essentially higher-order, we have a
correspondence between contexts and processes. More specifically,
given a name $x$ and a context $M$ we can construct $M^{*}_{x}$ such
that 

\begin{mathpar}
  M^{*}_{x} | \lift{x}{P} \red M[P]
\end{mathpar}

namely,

\begin{mathpar}
  M^{*}_{x} := x?(u).M[\dropn{u}]
\end{mathpar}

The dependence of $M^{*}_{x}$ on a name makes it an abstraction, 

\begin{mathpar}
  M^{*} := (x)x?(u).M[\dropn{u}]
\end{mathpar}

\subsection{Additional notation}

It will sometimes be convenient to denote the process a name
quotes. We already have the notation $x = \quotep{P}$, but it will be
convenient to introduce an alternate notation, $\procn{x}$, when we
want to emphasize the connection to the use of the name. Note that, by
virtue of name equivalence, $\quotep{\procn{x}} \nameeq x$; so, the
notation is consistent with previous definitions.

Further, because names have structure it is possible to effect
substitutions on the basis of that structure. This means we need to
upgrade our notation for substitutions, which we accomplish by
adapting comprehension notation. Thus,

\begin{mathpar}
  P\{ y / x : x \in S \}
\end{mathpar}

is interpreted to mean the process derived from P by replacing (in a
capture-avoiding manner) each occurrence of $x$ in $S$ by $y$. For example,

\begin{mathpar}
  P\{ \quotep{\procn{x}|\procn{x}} / x : x \in \freenames{P} \}
\end{mathpar}

will replace each (occurrence) of a free name $x$ in $P$ by
$\quotep{\procn{x}|\procn{x}}$.

Also, we will avail ourselves of the notation $x^{L}$ and $x^{R}$ to
denote injections of a name into disjoint copies of the name
space. There are numerous ways to accomplish this. One example can be
found in \cite{MeredithR05}. This notation overloads to vectors of
names: $\vec{x}^{\pi} := (x_{i}^{\pi} \; : \; 0 \leq i < |\vec{x}| )$ where $\pi \in \{L,R\}$.

We also use $P^{\Box} := P|\Box$.

In \cite{MeredithR05} an interpretation of the new operator is
given. It turns out that there are several possible interpretations
all enjoying the requisite algebraic properties of the operator (see
\cite{milner91polyadicpi}). We will therefore make liberal use of
$(\nu\; \vec{x})P$.

% subsection the_syntax_and_semantics_of_the_notation_system (end)   

\input{qm2pi.qmops} 

\input{qm2pi.sterngerlach} 

\input{qm2pi.metric} 

% section concurrent_process_calculi (end)

%\input{qm2pi.proofsketch}

% section proof sketch (end)

%\input{qm2pi.slviaknots} 

% section spatial logic via knots (end)

\input{qm2pi.conclusion}

% section conclusion (end)

%\input{qm2pi.dtcodes} 

% section wiring algorithm (end)

\input{qm2pi.ack} 

% section acknowledgments (end)

\newpage


\bibliographystyle{plain}   
\bibliography{../../biblios/main.bib}

\input{qm2pi.rhodetails}

\end{document}



% section front matter (end)

\section{Introduction}\label{sec:introduction} % (fold)
In this draft of the material i am going to have to dispense with the
usual writing conventions adopted in papers on these topics. i'm going
to have adopt whatever tone i need at the time i'm writing up the
calculations. Sometimes this may be very conversational; others it may
be the barest mathematical grunts; others still it may be that i have
lifted text from one of my other papers because the exposition of some
point was better said there. i hope that my readers are not unduly put
out by this decision. i'm not doing this to flout convention or be
rebellious. i find these calculations very technically challenging. To
keep everything going technically, something has to give; i have to
let go of some cognitive burden. So, the academic writing style --
with all of its trade-offs in terms of facilitating technical
communication -- is what i'm letting go of. Perhaps subsequent drafts
can be tightened and polished, but for now, i'm going to speak as if
we were sitting together in a coffee shop with a laptop, wifi and a
pad of paper and a pencil.

So, here's what i have to say. We -- you and i, comfortably ensconced
in our coffee shop and well-equipped with our tools -- can realize and
carry out the calculations of quantum mechanics over a very different
formal theory of dynamics, a formal theory of dynamics that
corresponds to a theory of concurrent computation with
\emph{reflection}. It has the advantage that the underlying theory is
already `quantized', but supports analogues all of the continuuous
operations. Strikingly, this underlying theory has recently been
connected with a notion of metric that we can show, by calculating
together, coincides with the metric induced by the inner product.

There are a lot of reasons why you might be interested in seeing
calculations of this form. Here's why i'm interested. For the past
several centuries there has been no competitor to the ``Newtonian''
account of dynamics. As a result the predominant share of accounts of
dynamical systems and situations have had to be formulated in terms of
the Newtonian machinery. i view this as an intellectually dangerous
position to occupy. Everything, despite it's intrinsic shape, turns
into a nail to be hit with this hammer. Recently, however, the theory
of computation has matured to the point where we have candidates for
theories of dynamics that offer very different perspective on
reasoning about dynamical systems and situations. Testing these
candidates against very successful accounts of dynamical situations,
like quantum mechanics, is going to give us some sense of how mature
they are and some measure of the quality of these accounts of
dynamics.

\subsection{Summary of contributions and outline of paper}

So, we're going to develop an interpretation of the operations of
quantum mechanics normally interpreted by Hilbert spaces and
operators. We're going to do this over a theory of computation. Note
that this is very different than the usual quantum computation program
which develops notions of computation over quantum mechanics. Rather,
we are developing a story that aligns with Wheeler's slogan: It from
Bit. To do this we will first provide an account of the theory of
computation at play here. Then we will dive into a calculation-driven
interpretation of the operations of quantum mechanics.

The reason we take this approach is that -- until very recently --
there hasn't been an axiomatic account of quantum mechanics. As a
result there has been no sharp delineation of the mathematical theory
supporting interpretation of the physical theory and the physical
theory, itself. So, ambient features of the maths are free to be
exploited (or supressed) without a real accounting of their physical
relevance. There is no sharp statement ``here's the physical theory''
qua \emph{theory} and ``here's the mathematical interpretation''
enabling a judgment of how faithful the interpretation is -- apart
from experimental observation. When there is an axiomatic account we
can judge how well a given mathematical formalism supports an
interpretation of the axioms, independent of
experimentation. Likewise, we can judge how well we have captured our
physical evidence and experience with our axiomatics, independent of
any specific mathematical implementation, with accidental detail that
may or may not have physical significance. 

In lieu of a fully fleshed out and vetted axiomatic account of quantum
mechanics, interpreting the operational notions in service of modeling
physical systems will have to suffice. In other words, we are not in
the business of providing a model of Hilbert spaces and operators. We
are in the business of providing a model of quantum mechanics because
we are motivated by testing our notions of dynamics against physical
theory; and, the predictive calculations of the physical theory must
serve as the best formulation -- shy of a fully fleshed out axiomatic
account -- of the physical theory itself (as they have for scientific
theories since time immemorial). Put another way, despite a
whole-hearted commitment to an It-from-Bit ontology, we are firmly
aligned with the shut-up-and-calculate camp as the best way to obtain
results either from the physical perspective or as a quality assurance
measure of our fledgling theory of dynamics.

In detail, we present a reflective process calculus. Then we develop
intuitive correspondences between the notions available in this
calculus and the usual physical notions supporting quantum mechanical
calculations. Thus, 

\begin{table}[htp]
  \center{
    \fbox{
      \begin{tabular}{c|c}
        quantum mechanics & process calculus \\
        \hline
        scalar & name \\
        state vector & process \\
        dual & contextual duals \\
        matrix & formal sums of process-context-dual pairs \\
        orthogonality & process annihilation \\
        inner product & execution-formula + quoting
      \end{tabular}
    }
  }
  \caption{QM - process calculi correspondences}
\end{table}

Then we tighten up these intuitions to operational definitions. We
employ the Dirac notation as the best proxy we can find for an
abstract syntax of the quantum mechanical notions. The definitions we
develop put us in contact with equational constraints coming from the
theory that we demonstrate the definitions and calculations satisfy.

This puts us in a position to shut up and calculate for the
Stern-Gerlach experimental set up, showing how these predictive
calculations become calculations on processes in our theory of a
reflective process calculus.

Penultimately, we demonstrate that the notion of metric coming from
the inner product coincides with the notion of metric available from
the theory of bisimulation. This demonstration gives us the right to
think of space as arising from behavior. Finally, we consider where we
might go from the new vantage point we have obtained.

% section introduction (end) 
 
% section introduction (end)

% \documentclass[12pt]{llncs}
%\documentclass{jktr}

\usepackage[pdftex]{hyperref}                   
\usepackage {listings}
\usepackage {mathpartir}
\usepackage{bcprules}
%\usepackage{listings}
                       
\usepackage{graphicx} 
%\usepackage[margins=2.5cm,nohead,nofoot]{geometry}
%\usepackage{geometry}
\usepackage{amsfonts}
\usepackage{amstext}
\usepackage{latexsym}
\usepackage{amssymb}
\usepackage{color}


%\include{myPreamble}
\include{qm2pi.local} 

%\ifpdf
%\usepackage[pdftex]{graphicx}
%\else
%\usepackage{graphicx}
%\fi

 % \ifpdf
%  \usepackage{pdfsync}
%  \if


%\title{Brief Article}
%\author{David F. Snyder}
%\author{L.G. Meredith}

%\address{Dept. of Math., Texas State University--San Marcos, San Marcos, TX 78666}
       
\pagestyle{empty}


\begin{document}

\lstset{language=[Objective]Caml,frame=shadowbox}

\input{qm2pi.front}

% section front matter (end)

\input{qm2pi.intro} 
 
% section introduction (end)

% \input{qm2pi.knotations} 

% section notation (end)

\input{qm2pi.process.calculi} 

% section concurrent_process_calculi_and_spatial_logics_ (end)
    
%\input{qm2pi.knots2pi} 

%\input{qm2pi.trefoil} 

%\input{qm2pi.mainthm} 

% subsection basic_interpretation (end)

%\input{qm2pi.rho.presentation} 
\subsection{The syntax and semantics of the notation system}\label{sub:the_syntax_and_semantics_of_the_notation_system} % (fold)

We now summarize a technical presentation of the calculus that
embodies our theory of dynamics. The typical presentation of such a
calculus follows the style of giving generators and relations on
them. The grammar, below, describing term constructors, freely
generates the set of processes, $\Proc$. This set is then quotiented
by a relation known as structural congruence and it is over this set
that the notion of dynamics is expressed. This presentation is
essentially that of \cite{MeredithR05} with the addition of
polyadicity and summation. For readability we have relegated some of
the technical subtleties to an appendix.

\subsubsection{Process grammar}\label{subsub:process_grammar}

\begin{mathpar}
  \inferrule* [lab=synchronization] {} {{M} \bc \pzero \;|\; x?F \;|\; x!C }
  \and
  \inferrule* [lab=abstraction] {} {{F} \bc (x)P}
  \and
  \inferrule* [lab=concretion] {} {{C} \bc \langle Q \rangle}
  \and
  \inferrule* [lab=process] {} {{P,Q} \bc M \;| \;P|Q \;|\; @{x}}
  \and
  \inferrule* [lab=name] {} {{x} \bc \quotep{P}}
\end{mathpar} 

Note that $\vec{x}$ (resp. $\vec{P}$) denotes a vector of names
(resp. processes) of length $|\vec{x}|$ (resp. $|\vec{P}|$). We adopt
the following useful abbreviations.

\begin{mathpar}
   x?(\vec{y}).P := x.(\vec{y})P \and  x\clift{\vec{P}} := x.\clift{\vec{P}}
   \and x!(y) := \lift{x}{\dropn{y}}
   \and \Pi_{i=0}^{n-1}P_i := P_0 | \ldots | P_{n-1}
\end{mathpar}

\subsubsection{Structural congruence}

\paragraph{Free and bound names and alpha-equivalence.} At the
core of structural equivalence is alpha-equivalence which identifies
process that are the same up to a change of variable. Formally, we
recognize the distinction between free and bound names. The free names
of a process, $\freenames{P}$, may be calculated recursively as
follows:

\begin{mathpar}
\freenames{\pzero} := \emptyset
  \and \\
  \freenames{x?(y).P} := \{ x \} \cup (\freenames{P} \setminus \{ y \})
  \and 
  \freenames{x!\langle P \rangle} := \{ x \} \cup \{ P \} 
  \and \\
  \freenames{P|Q} := \freenames{P} \cup \freenames{Q}
  \and \\
  \freenames{@{x}} := \{ x \}
\end{mathpar}

$\pi$
$\quotep{\pi}$

$\freenames{-} : \pi \to \mathcal{P}(\quotep{\pi})$

\begin{eqnarray*}
  \freenames{\pzero} & := & \emptyset \\
  \freenames{x?(y).P} & := & \{ x \} \cup (\freenames{P} \setminus \{ y \}) \\
  \freenames{x!\langle P \rangle} & := & \{ x \} \cup \{ P \} \\
  \freenames{P|Q} & := & \freenames{P} \cup \freenames{Q} \\
  \freenames{\dropn{x}} & := & \{ x \}
\end{eqnarray*}

The bound names of a process, $\boundnames{P}$, are those names occurring in $P$
that are not free. For example, in $x?(y).0$, the name $x$ is free, while $y$ is bound.

\begin{mathpar}
  \inferrule* [lab=monoidal-laws] {} { P|Q \equiv Q|P \and P|0 \equiv P \and P|(Q|R) \equiv (P|Q)|R }
\end{mathpar}

\begin{mathpar}
  \inferrule* [lab=alpha-equivalence] {} { (x)P \equiv (y)P\{y/x\} \and y \not\in \freenames{P} }
\end{mathpar}

\begin{definition}
Then two processes, $P,Q$, are alpha-equivalent if $P = Q\{\vec{y}/\vec{x}\}$ for
some $\vec{x} \in \boundnames{Q},\vec{y} \in \boundnames{P}$, where $Q\{\vec{y}/\vec{x}\}$
denotes the capture-avoiding substitution of $\vec{y}$ for $\vec{x}$ in $Q$.
\end{definition}

\begin{definition}
  The {\em structural congruence} \cite{SangiorgiWalker} , $\equiv$,
  between processes is the least congruence containing
  alpha-equivalence, satisfying the abelian monoid laws
  (associativity, commutativity and $\pzero$ as identity) for parallel
  composition $|$ and for summation $+$.
\end{definition}

\subsection{Name equivalence}

We take name equivalence, written $\nameeq$, to be the smallest
equivalence relation generated by the following rules.

\begin{mathpar}
\inferrule*[lab=Quote-drop]
{ }
{ \quotep{@{x}} \nameeq x }

\inferrule*[lab=Struct-equiv]
{ P \scong Q }
{ \quotep{P} \nameeq \quotep{Q} }
\end{mathpar}

The astute reader will have noticed that the mutual recursion of names
and processes imposes a mutual recursion on alpha-equivalence and
structural equivalence via name-equivalence. Fortunately, all of this
works out pleasantly and we may calculate in the natural way, free of
concern. The reader interested in the details is referred to the
appendix \ref{appendix:rho_details}.

\subsection{Substitution}

We use $\Proc$ for the set of processes, $\QProc$ for the set of
names, and $\id{\{}\vec{y} / \vec{x} \id{\}}$ to denote partial maps,
$s : \QProc \rightarrow \QProc$. A map, $s$ lifts, uniquely, to a map
on process terms, $\widehat{s} : \Proc \rightarrow \Proc$ by the
following equations.

\begin{mathpar}
  (0) \psubstp{Q}{P} := 0 \\
  (R \juxtap S) \psubstp{Q}{P}
  :=    
  (R)\psubstp{Q}{P} \juxtap (S) \psubstp{Q}{P} \\
  (x?(y).R) \psubstp{Q}{P}    
  :=    
  (x)\substp{Q}{P} (z)\concat( (R \psubstn{z}{y}) \psubstp{Q}{P} ) \\
  (\lift{x}{R}) \psubstp{Q}{P}  
  :=
  \lift{(x)\substp{Q}{P}}{ R \psubstp{Q}{P} } \\
%   (\dropn{x})  \psubstp{Q}{P}       
%   := 
%   \left\{ 
%     \begin{array}{ccc} 
%       \dropn{\quotep{Q}} & & x \nameeq \quotep{P} \\
%       \dropn{x} & & otherwise \\
%     \end{array}
%   \right. 
  (\dropn{x})  \psubstp{Q}{P}       
  := 
  \left\{ 
    \begin{array}{ccc} 
      Q & & x \nameeq \quotep{P} \\
      \dropn{x} & & otherwise \\
    \end{array}
  \right.
\end{mathpar}
 

where

\begin{eqnarray}
  (x)\id{\{} \lpquote Q \rpquote / \lpquote P \rpquote \id{\}}            = 
  \left\{ 
    \begin{array}{ccc}
      \lpquote Q \rpquote & & x \nameeq \lpquote P \rpquote \\
      x & & otherwise \\
    \end{array}
  \right. \nonumber
\end{eqnarray}

and $z$ is chosen distinct from $\quotep{P}$, $\quotep{Q}$, the free
names in $Q$, and all the names in $R$. Our $\alpha$-equivalence will
be built in the standard way from this substitution.

\begin{remark}\label{rem:no_self_referential_names}
  One consequence of these definitions is that $\forall P. \quotep{P}
  \not\in \freenames{P}$.
\end{remark}

\subsection{ Dynamic quote: an example }

Anticipating something of what's to come, consider applying the
substitution, $\widehat{\id{\{}u / z \id{\}}}$, to the following pair
of processes, $\lift{w}{y!(z)}$ and $w[ \lpquote y!(z) \rpquote ]$.

\begin{eqnarray}
	\lift{w}{y!(z)}\widehat{\id{\{}u / z \id{\}}}
		& = &
		\lift{w}{y!(u)} \nonumber\\
	w[ \lpquote y!(z) \rpquote ] \widehat{ \id{\{}u / z \id{\}} }
		& = &
		w[ \lpquote y!(z) \rpquote ] \nonumber
\end{eqnarray}

Because the body of the process between quotes is impervious to
substitution, we get radically different answers. In fact, by
examining the first process in an input context,
e.g. $x?(z).\lift{w}{y!(z)}$, we see that the process under the lift
operator may be shaped by prefixed inputs binding a name inside it. In
this sense, the lift operator will be seen as a way to dynamically
construct processes before reifying them as names.

Finally equipped with these standard features we can present the
dynamics of the calculus.

\subsubsection{Operational semantics} 

Finally, we introduce the computational dynamics. What marks these
algebras as distinct from other more traditionally studied algebraic
structures, e.g. vector spaces or polynomial rings, is the manner in
which dynamics is captured. In traditional structures, dynamics is typically
expressed through morphisms between such structures, as in linear maps
between vector spaces or morphisms between rings. In algebras
associated with the semantics of computation, the dynamics is
expressed as part of the algebraic structure itself, through a
reduction reduction relation typically denoted by $\red$. Below, we
give a recursive presentation of this relation for the calculus used
in the encoding.

$\red \subseteq \pi \times \pi$
$\red : \pi \to \mathcal{P}(\pi)$

\begin{mathpar}
  \inferrule* [lab=Comm] { \textsf{match}( x_{src}, x_{trgt} ) } { x_{trgt}?(y)P \; | \; x_{src}!\langle {Q} \rangle \red P\{\quotep{Q}/y}\} }
  \and \\
  \inferrule* [lab=Par] {{P} \red {P}'} {{{P} | {Q}} \red {{P}' | {Q}}}
  \and
  \inferrule* [lab=Equiv]{{{P} \scong {P}'} \andalso {{P}' \red {Q}'} \andalso {{Q}' \scong {Q}}}{{P} \red {Q}}
\end{mathpar}

\begin{eqnarray*}
  match_{\equiv} (\quotep{P},\quotep{Q}) & := & P \equiv Q \\
  match_{\dagger}(\quotep{P},\quotep{Q}) & := & \forall R. P|Q \red^{*} R => R \red^{*} 0 \\
  match_{K}(\quotep{P},\quotep{Q}) & := & K \mbox{ for some context } K
\end{eqnarray*}

$u?(x)P | u!\langle Q \rangle \red P\{\quotep{Q}/x\}$

%We write $\wred$ for $\red^*$, and $P\red$ if $\exists Q $ such that $ P \red Q$.
We write $P\red$ if $\exists Q $ such that $ P \red Q$ and $P\not\red$, otherwise.

\section{Replication}

As mentioned before, it is known that replication (and hence
recursion) can be implemented in a higher-order process algebra
\cite{SangiorgiWalker}. As our first example of calculation with the
machinery thus far presented we give the construction explicitly in
the {\rhoc}.

\begin{eqnarray}
	D_{x} & := & \prefix{x}{y}{(\binpar{\outputp{x}{y}}{@{y}})} \nonumber\\
	\bangp_{x}{P} & := & \binpar{{x}!\langle{\binpar{D_{x}}{P}}\rangle}{D_{x}} \nonumber
\end{eqnarray}

\begin{eqnarray}
	\bangp_{x}{P} & & \nonumber\\
	=
	& {x}!\langle{(\prefix{x}{y}{(\outputp{x}{y} | @{y})) | P}}\rangle 
	      | \prefix{x}{y}{(\outputp{x}{y} | @{y})} & \nonumber\\
	\red
	& (\outputp{x}{y} | @{y})\substn{\quotep{(\prefix{x}{y}{(@{y} | \outputp{x}{y})) | P}}}{y} & \nonumber\\
	=
	& \outputp{x}{\quotep{(\prefix{x}{y}{(\outputp{x}{y} | @{y})) | P}}}
	  | {(\prefix{x}{y}{(\outputp{x}{y} | @{y})) | P}} & \nonumber\\
	\red
	& \ldots & \nonumber\\
	\red^*
	& P | P | \ldots & \nonumber
\end{eqnarray}

Of course, this encoding, as an implementation, runs away, unfolding
$\bangp{P}$ eagerly. A lazier and more implementable replication
operator, restricted to input-guarded processes, may be obtained as follows.

\begin{eqnarray}
\bangp{\prefix{u}{v}{P}} 
	:= 
	\binpar{\lift{x}{\prefix{u}{v}{(\binpar{D(x)}{P})}}}{D(x)} \nonumber
\end{eqnarray}

\begin{remark}
  Note that the lazier definition still does not deal with summation
  or mixed summation (i.e. sums over input and output). The reader is
  invited to construct definitions of replication that deal with these
  features. 

  Further, the definitions are parameterized in a name, $x$. Can you,
  gentle reader, make a definition that eliminates this parameter and
  guarantees no accidental interaction between the replication
  machinery and the process being replicated -- i.e. no accidental
  sharing of names used by the process to get its work done and the
  name(s) used by the replication to effect copying. This latter
  revision of the definition of replication is crucial to obtaining
  the expected identity $!!P \sim !P$.
\end{remark}

\begin{remark}\label{rem:paradoxical_combinator}
  The reader familiar with the lambda calculus will have noticed the
  similarity between $D$ and the paradoxical combinator.

  [Ed. note: the existence of this seems to suggest we have to be more
  restrictive on the set of processes and names we admit if we are to
  support no-cloning.]
\end{remark}

\subsubsection{Bisimulation}

The computational dynamics gives rise to another kind of equivalence,
the equivalence of computational behavior. As previously mentioned
this is typically captured \emph{via} some form of bisimulation.

% The notion we use in this paper is weak barbed bisimulation
% \cite{milner91polyadicpi}.

The notion we use in this paper is derived from weak barbed
bisimulation \cite{milner91polyadicpi}. 

\begin{definition}
An \emph{observation relation}, $\downarrow_{\mathcal N}$, over a set
of names, $\mathcal N$, is the smallest relation satisfying the rules
below.

\infrule[Out-barb]{y \in {\mathcal N}, \; x \nameeq y}
		  {\outputp{x}{v} \downarrow_{\mathcal N} x}
\infrule[Par-barb]{\mbox{$P\downarrow_{\mathcal N} x$ or $Q\downarrow_{\mathcal N} x$}}
		  {\binpar{P}{Q} \downarrow_{\mathcal N} x}

We write $P \Downarrow_{\mathcal N} x$ if there is $Q$ such that 
$P \wred Q$ and $Q \downarrow_{\mathcal N} x$.
\end{definition}

\begin{definition}
%\label{def.bbisim}
An  ${\mathcal N}$-\emph{barbed bisimulation} over a set of names, ${\mathcal N}$, is a symmetric binary relation 
${\mathcal S}_{\mathcal N}$ between agents such that $P\rel{S}_{\mathcal N}Q$ implies:
\begin{enumerate}
\item If $P \red P'$ then $Q \wred Q'$ and $P'\rel{S}_{\mathcal N} Q'$.
\item If $P\downarrow_{\mathcal N} x$, then $Q\Downarrow_{\mathcal N} x$.
\end{enumerate}
$P$ is ${\mathcal N}$-barbed bisimilar to $Q$, written
$P \wbbisim_{\mathcal N} Q$, if $P \rel{S}_{\mathcal N} Q$ for some ${\mathcal N}$-barbed bisimulation ${\mathcal S}_{\mathcal N}$.
\end{definition}

$\mathcal{R} \subseteq \pi \times \pi$

$P \mathcal{R} Q => \forall P'. P \red P' \Rightarrow \exists Q'. Q \red Q', P' \mathcal{R} Q'$

$P \vdash x \Rightarrow Q \vdash x$

\begin{mathpar}
  \inferrule*[lab=Out-barb]{x \nameeq y}{{y}!\langle{Q}\rangle \vdash x}
  \and
  \inferrule*[lab=Par-barb]{\mbox{$P\vdash x$ or $Q\vdash x$}}{\binpar{P}{Q} \vdash x}
\end{mathpar}

\subsubsection{Contexts}

One of the principle advantages of computational calculi like the
$\pi$-calculus is a well-defined notion of context,
contextual-equivalence and a correlation between
contextual-equivalence and notions of bisimulation. The notion of
context allows the decomposition of a process into (sub-)process and
its syntactic environment, its context. Thus, a context may be
thought of as a process with a ``hole'' (written $\Box$) in it. The
application of a context $M$ to a process $P$, written $M[P]$, is
tantamount to filling the hole in $M$ with $P$. In this paper we do
not need the full weight of this theory, but do make use of the notion
of context in the proof the main theorem. 

\begin{mathpar}
  \inferrule* [lab=summation] {} {{M_{M},M_{N}} \bc \Box \;|\; x.M_{A} \;|\; M_{M}+M_{N}}
  \and
  \inferrule* [lab=agent] {} {{M_{A}} \bc (\vec{x})M_{P} \;| \; \clift{P_0,\ldots,M_{P},\ldots,P_N}}
  \and \\
  \inferrule* [lab=process] {} {{M_{P}} \bc M_{N} \;| \;P|M_{P} }
\end{mathpar} 

\begin{mathpar}
  \inferrule* [lab=sychronization] {} {M_{N} \bc \Box \;|\; x?M_{F} \;|\; x!M_{C}}
  \and
  \inferrule* [lab=abstraction] {} {{M_{F}} \bc (x)M_{P} }
  \and
  \inferrule* [lab=concretion] {} {{M_{C}} \bc \langle M_{P} \rangle }
  \and \\
  \inferrule* [lab=process] {} {{M_{P}} \bc M_{N} \;| \;P|M_{P} }
\end{mathpar}

\begin{definition}[contextual application] Given a context $M$, and
  process $P$, we define the \emph{contextual application}, $M[P] :=
  M\{P/\Box\}$. That is, the contextual application of M to P is the
  substitution of $P$ for $\Box$ in $M$.
\end{definition}

$\meaningof{-} : L \to \mathcal{P}(\pi)$

\begin{mathpar}
  \inferrule* [lab=collection] {} {\meaningof{true} = \pi, \and \meaningof{~E} = \pi \setminus \meaningof{E}, \and \meaningof{E_{1} \& E_{2}} = \meaningof{E_{1}} \cap \meaningof{E_{2}}}
\end{mathpar}

\begin{mathpar}
  \inferrule* [lab=structure] {} {\meaningof{0} = \{ P \in \pi | P \equiv 0 \}, \and \\ \meaningof{E_1 | E_2} = \{ P \in \pi | P \equiv P_{1} | P_{2}, P_{1} \in \meaningof{E_{1}}, P_{2} \in \meaningof{E_2}\} }
\end{mathpar}

\begin{mathpar}
 \inferrule* [lab=behavior] {} {\meaningof{\langle a?b \rangle E} = \{ P \in \pi | P \equiv Q | u?(y)P', \\ \and \\\\ \and \\ \;\;\; u \in \meaningof{a}, \forall z.P'\{z/y\} \in \meaningof{E\{z/b\}}\}, \and \\ \meaningof{a!E} = \{ P \in \pi | P \equiv Q | x!\langle P' \rangle, x \in \meaningof{a} P' \in \meaningof{E}\} }
\end{mathpar}

\begin{mathpar}
 \inferrule* [lab=nominal] {} {\meaningof{\quotep{E}} = \{ \quotep{P} \in \quotep{\pi} | P \in \meaningof{E} \}, \and \meaningof{\quotep{P}} = \{ \quotep{Q} \in \quotep{\pi} | P \equiv Q \} \and \\ \meaningof{@\quotep{E}} = \{ P \in \pi | P \equiv @x, x \in \meaningof{E} \}}
\end{mathpar}

\begin{eqnarray*}
  \\
  \meaningof{-} : TS \to ST
\end{eqnarray*}

\begin{eqnarray*}
  \\
  L : TS \to ST
\end{eqnarray*}

\begin{eqnarray*}
  \\
  P \models E \iff P \in \meaningof{E}
\end{eqnarray*}

\begin{eqnarray*}
  P \approx_{L} Q \iff \forall E \in L. P \models E \iff Q \models E
\end{eqnarray*}

\begin{eqnarray*}
  P \approx_{K} Q
\end{eqnarray*}

\begin{eqnarray*}
  P \approx Q
\end{eqnarray*}

$\approx_{K} = \approx = \approx_{L}$

\subsubsection{Contextual duality}

Note that contexts extend the quotation operation to a family of
operations from processes to names. Given a context, $M$, we can
define a \emph{nominal context}, $\quotep{M}$ by $\quotep{M}[P] :=
\quotep{M[P]}$. To foreshadow what is to come we observe that these
operations enjoy a duality with processes very much like the duality
between vectors and maps from vectors to scalars.

Further, because the calculus is essentially higher-order, we have a
correspondence between contexts and processes. More specifically,
given a name $x$ and a context $M$ we can construct $M^{*}_{x}$ such
that 

\begin{mathpar}
  M^{*}_{x} | \lift{x}{P} \red M[P]
\end{mathpar}

namely,

\begin{mathpar}
  M^{*}_{x} := x?(u).M[\dropn{u}]
\end{mathpar}

The dependence of $M^{*}_{x}$ on a name makes it an abstraction, 

\begin{mathpar}
  M^{*} := (x)x?(u).M[\dropn{u}]
\end{mathpar}

\subsection{Additional notation}

It will sometimes be convenient to denote the process a name
quotes. We already have the notation $x = \quotep{P}$, but it will be
convenient to introduce an alternate notation, $\procn{x}$, when we
want to emphasize the connection to the use of the name. Note that, by
virtue of name equivalence, $\quotep{\procn{x}} \nameeq x$; so, the
notation is consistent with previous definitions.

Further, because names have structure it is possible to effect
substitutions on the basis of that structure. This means we need to
upgrade our notation for substitutions, which we accomplish by
adapting comprehension notation. Thus,

\begin{mathpar}
  P\{ y / x : x \in S \}
\end{mathpar}

is interpreted to mean the process derived from P by replacing (in a
capture-avoiding manner) each occurrence of $x$ in $S$ by $y$. For example,

\begin{mathpar}
  P\{ \quotep{\procn{x}|\procn{x}} / x : x \in \freenames{P} \}
\end{mathpar}

will replace each (occurrence) of a free name $x$ in $P$ by
$\quotep{\procn{x}|\procn{x}}$.

Also, we will avail ourselves of the notation $x^{L}$ and $x^{R}$ to
denote injections of a name into disjoint copies of the name
space. There are numerous ways to accomplish this. One example can be
found in \cite{MeredithR05}. This notation overloads to vectors of
names: $\vec{x}^{\pi} := (x_{i}^{\pi} \; : \; 0 \leq i < |\vec{x}| )$ where $\pi \in \{L,R\}$.

We also use $P^{\Box} := P|\Box$.

In \cite{MeredithR05} an interpretation of the new operator is
given. It turns out that there are several possible interpretations
all enjoying the requisite algebraic properties of the operator (see
\cite{milner91polyadicpi}). We will therefore make liberal use of
$(\nu\; \vec{x})P$.

% subsection the_syntax_and_semantics_of_the_notation_system (end)   

\input{qm2pi.qmops} 

\input{qm2pi.sterngerlach} 

\input{qm2pi.metric} 

% section concurrent_process_calculi (end)

%\input{qm2pi.proofsketch}

% section proof sketch (end)

%\input{qm2pi.slviaknots} 

% section spatial logic via knots (end)

\input{qm2pi.conclusion}

% section conclusion (end)

%\input{qm2pi.dtcodes} 

% section wiring algorithm (end)

\input{qm2pi.ack} 

% section acknowledgments (end)

\newpage


\bibliographystyle{plain}   
\bibliography{../../biblios/main.bib}

\input{qm2pi.rhodetails}

\end{document}

 

% section notation (end)

\input{qm2pi.process.calculi} 

% section concurrent_process_calculi_and_spatial_logics_ (end)
    
%\documentclass[12pt]{llncs}
%\documentclass{jktr}

\usepackage[pdftex]{hyperref}                   
\usepackage {listings}
\usepackage {mathpartir}
\usepackage{bcprules}
%\usepackage{listings}
                       
\usepackage{graphicx} 
%\usepackage[margins=2.5cm,nohead,nofoot]{geometry}
%\usepackage{geometry}
\usepackage{amsfonts}
\usepackage{amstext}
\usepackage{latexsym}
\usepackage{amssymb}
\usepackage{color}


%\include{myPreamble}
\include{qm2pi.local} 

%\ifpdf
%\usepackage[pdftex]{graphicx}
%\else
%\usepackage{graphicx}
%\fi

 % \ifpdf
%  \usepackage{pdfsync}
%  \if


%\title{Brief Article}
%\author{David F. Snyder}
%\author{L.G. Meredith}

%\address{Dept. of Math., Texas State University--San Marcos, San Marcos, TX 78666}
       
\pagestyle{empty}


\begin{document}

\lstset{language=[Objective]Caml,frame=shadowbox}

\input{qm2pi.front}

% section front matter (end)

\input{qm2pi.intro} 
 
% section introduction (end)

% \input{qm2pi.knotations} 

% section notation (end)

\input{qm2pi.process.calculi} 

% section concurrent_process_calculi_and_spatial_logics_ (end)
    
%\input{qm2pi.knots2pi} 

%\input{qm2pi.trefoil} 

%\input{qm2pi.mainthm} 

% subsection basic_interpretation (end)

%\input{qm2pi.rho.presentation} 
\subsection{The syntax and semantics of the notation system}\label{sub:the_syntax_and_semantics_of_the_notation_system} % (fold)

We now summarize a technical presentation of the calculus that
embodies our theory of dynamics. The typical presentation of such a
calculus follows the style of giving generators and relations on
them. The grammar, below, describing term constructors, freely
generates the set of processes, $\Proc$. This set is then quotiented
by a relation known as structural congruence and it is over this set
that the notion of dynamics is expressed. This presentation is
essentially that of \cite{MeredithR05} with the addition of
polyadicity and summation. For readability we have relegated some of
the technical subtleties to an appendix.

\subsubsection{Process grammar}\label{subsub:process_grammar}

\begin{mathpar}
  \inferrule* [lab=synchronization] {} {{M} \bc \pzero \;|\; x?F \;|\; x!C }
  \and
  \inferrule* [lab=abstraction] {} {{F} \bc (x)P}
  \and
  \inferrule* [lab=concretion] {} {{C} \bc \langle Q \rangle}
  \and
  \inferrule* [lab=process] {} {{P,Q} \bc M \;| \;P|Q \;|\; @{x}}
  \and
  \inferrule* [lab=name] {} {{x} \bc \quotep{P}}
\end{mathpar} 

Note that $\vec{x}$ (resp. $\vec{P}$) denotes a vector of names
(resp. processes) of length $|\vec{x}|$ (resp. $|\vec{P}|$). We adopt
the following useful abbreviations.

\begin{mathpar}
   x?(\vec{y}).P := x.(\vec{y})P \and  x\clift{\vec{P}} := x.\clift{\vec{P}}
   \and x!(y) := \lift{x}{\dropn{y}}
   \and \Pi_{i=0}^{n-1}P_i := P_0 | \ldots | P_{n-1}
\end{mathpar}

\subsubsection{Structural congruence}

\paragraph{Free and bound names and alpha-equivalence.} At the
core of structural equivalence is alpha-equivalence which identifies
process that are the same up to a change of variable. Formally, we
recognize the distinction between free and bound names. The free names
of a process, $\freenames{P}$, may be calculated recursively as
follows:

\begin{mathpar}
\freenames{\pzero} := \emptyset
  \and \\
  \freenames{x?(y).P} := \{ x \} \cup (\freenames{P} \setminus \{ y \})
  \and 
  \freenames{x!\langle P \rangle} := \{ x \} \cup \{ P \} 
  \and \\
  \freenames{P|Q} := \freenames{P} \cup \freenames{Q}
  \and \\
  \freenames{@{x}} := \{ x \}
\end{mathpar}

$\pi$
$\quotep{\pi}$

$\freenames{-} : \pi \to \mathcal{P}(\quotep{\pi})$

\begin{eqnarray*}
  \freenames{\pzero} & := & \emptyset \\
  \freenames{x?(y).P} & := & \{ x \} \cup (\freenames{P} \setminus \{ y \}) \\
  \freenames{x!\langle P \rangle} & := & \{ x \} \cup \{ P \} \\
  \freenames{P|Q} & := & \freenames{P} \cup \freenames{Q} \\
  \freenames{\dropn{x}} & := & \{ x \}
\end{eqnarray*}

The bound names of a process, $\boundnames{P}$, are those names occurring in $P$
that are not free. For example, in $x?(y).0$, the name $x$ is free, while $y$ is bound.

\begin{mathpar}
  \inferrule* [lab=monoidal-laws] {} { P|Q \equiv Q|P \and P|0 \equiv P \and P|(Q|R) \equiv (P|Q)|R }
\end{mathpar}

\begin{mathpar}
  \inferrule* [lab=alpha-equivalence] {} { (x)P \equiv (y)P\{y/x\} \and y \not\in \freenames{P} }
\end{mathpar}

\begin{definition}
Then two processes, $P,Q$, are alpha-equivalent if $P = Q\{\vec{y}/\vec{x}\}$ for
some $\vec{x} \in \boundnames{Q},\vec{y} \in \boundnames{P}$, where $Q\{\vec{y}/\vec{x}\}$
denotes the capture-avoiding substitution of $\vec{y}$ for $\vec{x}$ in $Q$.
\end{definition}

\begin{definition}
  The {\em structural congruence} \cite{SangiorgiWalker} , $\equiv$,
  between processes is the least congruence containing
  alpha-equivalence, satisfying the abelian monoid laws
  (associativity, commutativity and $\pzero$ as identity) for parallel
  composition $|$ and for summation $+$.
\end{definition}

\subsection{Name equivalence}

We take name equivalence, written $\nameeq$, to be the smallest
equivalence relation generated by the following rules.

\begin{mathpar}
\inferrule*[lab=Quote-drop]
{ }
{ \quotep{@{x}} \nameeq x }

\inferrule*[lab=Struct-equiv]
{ P \scong Q }
{ \quotep{P} \nameeq \quotep{Q} }
\end{mathpar}

The astute reader will have noticed that the mutual recursion of names
and processes imposes a mutual recursion on alpha-equivalence and
structural equivalence via name-equivalence. Fortunately, all of this
works out pleasantly and we may calculate in the natural way, free of
concern. The reader interested in the details is referred to the
appendix \ref{appendix:rho_details}.

\subsection{Substitution}

We use $\Proc$ for the set of processes, $\QProc$ for the set of
names, and $\id{\{}\vec{y} / \vec{x} \id{\}}$ to denote partial maps,
$s : \QProc \rightarrow \QProc$. A map, $s$ lifts, uniquely, to a map
on process terms, $\widehat{s} : \Proc \rightarrow \Proc$ by the
following equations.

\begin{mathpar}
  (0) \psubstp{Q}{P} := 0 \\
  (R \juxtap S) \psubstp{Q}{P}
  :=    
  (R)\psubstp{Q}{P} \juxtap (S) \psubstp{Q}{P} \\
  (x?(y).R) \psubstp{Q}{P}    
  :=    
  (x)\substp{Q}{P} (z)\concat( (R \psubstn{z}{y}) \psubstp{Q}{P} ) \\
  (\lift{x}{R}) \psubstp{Q}{P}  
  :=
  \lift{(x)\substp{Q}{P}}{ R \psubstp{Q}{P} } \\
%   (\dropn{x})  \psubstp{Q}{P}       
%   := 
%   \left\{ 
%     \begin{array}{ccc} 
%       \dropn{\quotep{Q}} & & x \nameeq \quotep{P} \\
%       \dropn{x} & & otherwise \\
%     \end{array}
%   \right. 
  (\dropn{x})  \psubstp{Q}{P}       
  := 
  \left\{ 
    \begin{array}{ccc} 
      Q & & x \nameeq \quotep{P} \\
      \dropn{x} & & otherwise \\
    \end{array}
  \right.
\end{mathpar}
 

where

\begin{eqnarray}
  (x)\id{\{} \lpquote Q \rpquote / \lpquote P \rpquote \id{\}}            = 
  \left\{ 
    \begin{array}{ccc}
      \lpquote Q \rpquote & & x \nameeq \lpquote P \rpquote \\
      x & & otherwise \\
    \end{array}
  \right. \nonumber
\end{eqnarray}

and $z$ is chosen distinct from $\quotep{P}$, $\quotep{Q}$, the free
names in $Q$, and all the names in $R$. Our $\alpha$-equivalence will
be built in the standard way from this substitution.

\begin{remark}\label{rem:no_self_referential_names}
  One consequence of these definitions is that $\forall P. \quotep{P}
  \not\in \freenames{P}$.
\end{remark}

\subsection{ Dynamic quote: an example }

Anticipating something of what's to come, consider applying the
substitution, $\widehat{\id{\{}u / z \id{\}}}$, to the following pair
of processes, $\lift{w}{y!(z)}$ and $w[ \lpquote y!(z) \rpquote ]$.

\begin{eqnarray}
	\lift{w}{y!(z)}\widehat{\id{\{}u / z \id{\}}}
		& = &
		\lift{w}{y!(u)} \nonumber\\
	w[ \lpquote y!(z) \rpquote ] \widehat{ \id{\{}u / z \id{\}} }
		& = &
		w[ \lpquote y!(z) \rpquote ] \nonumber
\end{eqnarray}

Because the body of the process between quotes is impervious to
substitution, we get radically different answers. In fact, by
examining the first process in an input context,
e.g. $x?(z).\lift{w}{y!(z)}$, we see that the process under the lift
operator may be shaped by prefixed inputs binding a name inside it. In
this sense, the lift operator will be seen as a way to dynamically
construct processes before reifying them as names.

Finally equipped with these standard features we can present the
dynamics of the calculus.

\subsubsection{Operational semantics} 

Finally, we introduce the computational dynamics. What marks these
algebras as distinct from other more traditionally studied algebraic
structures, e.g. vector spaces or polynomial rings, is the manner in
which dynamics is captured. In traditional structures, dynamics is typically
expressed through morphisms between such structures, as in linear maps
between vector spaces or morphisms between rings. In algebras
associated with the semantics of computation, the dynamics is
expressed as part of the algebraic structure itself, through a
reduction reduction relation typically denoted by $\red$. Below, we
give a recursive presentation of this relation for the calculus used
in the encoding.

$\red \subseteq \pi \times \pi$
$\red : \pi \to \mathcal{P}(\pi)$

\begin{mathpar}
  \inferrule* [lab=Comm] { \textsf{match}( x_{src}, x_{trgt} ) } { x_{trgt}?(y)P \; | \; x_{src}!\langle {Q} \rangle \red P\{\quotep{Q}/y}\} }
  \and \\
  \inferrule* [lab=Par] {{P} \red {P}'} {{{P} | {Q}} \red {{P}' | {Q}}}
  \and
  \inferrule* [lab=Equiv]{{{P} \scong {P}'} \andalso {{P}' \red {Q}'} \andalso {{Q}' \scong {Q}}}{{P} \red {Q}}
\end{mathpar}

\begin{eqnarray*}
  match_{\equiv} (\quotep{P},\quotep{Q}) & := & P \equiv Q \\
  match_{\dagger}(\quotep{P},\quotep{Q}) & := & \forall R. P|Q \red^{*} R => R \red^{*} 0 \\
  match_{K}(\quotep{P},\quotep{Q}) & := & K \mbox{ for some context } K
\end{eqnarray*}

$u?(x)P | u!\langle Q \rangle \red P\{\quotep{Q}/x\}$

%We write $\wred$ for $\red^*$, and $P\red$ if $\exists Q $ such that $ P \red Q$.
We write $P\red$ if $\exists Q $ such that $ P \red Q$ and $P\not\red$, otherwise.

\section{Replication}

As mentioned before, it is known that replication (and hence
recursion) can be implemented in a higher-order process algebra
\cite{SangiorgiWalker}. As our first example of calculation with the
machinery thus far presented we give the construction explicitly in
the {\rhoc}.

\begin{eqnarray}
	D_{x} & := & \prefix{x}{y}{(\binpar{\outputp{x}{y}}{@{y}})} \nonumber\\
	\bangp_{x}{P} & := & \binpar{{x}!\langle{\binpar{D_{x}}{P}}\rangle}{D_{x}} \nonumber
\end{eqnarray}

\begin{eqnarray}
	\bangp_{x}{P} & & \nonumber\\
	=
	& {x}!\langle{(\prefix{x}{y}{(\outputp{x}{y} | @{y})) | P}}\rangle 
	      | \prefix{x}{y}{(\outputp{x}{y} | @{y})} & \nonumber\\
	\red
	& (\outputp{x}{y} | @{y})\substn{\quotep{(\prefix{x}{y}{(@{y} | \outputp{x}{y})) | P}}}{y} & \nonumber\\
	=
	& \outputp{x}{\quotep{(\prefix{x}{y}{(\outputp{x}{y} | @{y})) | P}}}
	  | {(\prefix{x}{y}{(\outputp{x}{y} | @{y})) | P}} & \nonumber\\
	\red
	& \ldots & \nonumber\\
	\red^*
	& P | P | \ldots & \nonumber
\end{eqnarray}

Of course, this encoding, as an implementation, runs away, unfolding
$\bangp{P}$ eagerly. A lazier and more implementable replication
operator, restricted to input-guarded processes, may be obtained as follows.

\begin{eqnarray}
\bangp{\prefix{u}{v}{P}} 
	:= 
	\binpar{\lift{x}{\prefix{u}{v}{(\binpar{D(x)}{P})}}}{D(x)} \nonumber
\end{eqnarray}

\begin{remark}
  Note that the lazier definition still does not deal with summation
  or mixed summation (i.e. sums over input and output). The reader is
  invited to construct definitions of replication that deal with these
  features. 

  Further, the definitions are parameterized in a name, $x$. Can you,
  gentle reader, make a definition that eliminates this parameter and
  guarantees no accidental interaction between the replication
  machinery and the process being replicated -- i.e. no accidental
  sharing of names used by the process to get its work done and the
  name(s) used by the replication to effect copying. This latter
  revision of the definition of replication is crucial to obtaining
  the expected identity $!!P \sim !P$.
\end{remark}

\begin{remark}\label{rem:paradoxical_combinator}
  The reader familiar with the lambda calculus will have noticed the
  similarity between $D$ and the paradoxical combinator.

  [Ed. note: the existence of this seems to suggest we have to be more
  restrictive on the set of processes and names we admit if we are to
  support no-cloning.]
\end{remark}

\subsubsection{Bisimulation}

The computational dynamics gives rise to another kind of equivalence,
the equivalence of computational behavior. As previously mentioned
this is typically captured \emph{via} some form of bisimulation.

% The notion we use in this paper is weak barbed bisimulation
% \cite{milner91polyadicpi}.

The notion we use in this paper is derived from weak barbed
bisimulation \cite{milner91polyadicpi}. 

\begin{definition}
An \emph{observation relation}, $\downarrow_{\mathcal N}$, over a set
of names, $\mathcal N$, is the smallest relation satisfying the rules
below.

\infrule[Out-barb]{y \in {\mathcal N}, \; x \nameeq y}
		  {\outputp{x}{v} \downarrow_{\mathcal N} x}
\infrule[Par-barb]{\mbox{$P\downarrow_{\mathcal N} x$ or $Q\downarrow_{\mathcal N} x$}}
		  {\binpar{P}{Q} \downarrow_{\mathcal N} x}

We write $P \Downarrow_{\mathcal N} x$ if there is $Q$ such that 
$P \wred Q$ and $Q \downarrow_{\mathcal N} x$.
\end{definition}

\begin{definition}
%\label{def.bbisim}
An  ${\mathcal N}$-\emph{barbed bisimulation} over a set of names, ${\mathcal N}$, is a symmetric binary relation 
${\mathcal S}_{\mathcal N}$ between agents such that $P\rel{S}_{\mathcal N}Q$ implies:
\begin{enumerate}
\item If $P \red P'$ then $Q \wred Q'$ and $P'\rel{S}_{\mathcal N} Q'$.
\item If $P\downarrow_{\mathcal N} x$, then $Q\Downarrow_{\mathcal N} x$.
\end{enumerate}
$P$ is ${\mathcal N}$-barbed bisimilar to $Q$, written
$P \wbbisim_{\mathcal N} Q$, if $P \rel{S}_{\mathcal N} Q$ for some ${\mathcal N}$-barbed bisimulation ${\mathcal S}_{\mathcal N}$.
\end{definition}

$\mathcal{R} \subseteq \pi \times \pi$

$P \mathcal{R} Q => \forall P'. P \red P' \Rightarrow \exists Q'. Q \red Q', P' \mathcal{R} Q'$

$P \vdash x \Rightarrow Q \vdash x$

\begin{mathpar}
  \inferrule*[lab=Out-barb]{x \nameeq y}{{y}!\langle{Q}\rangle \vdash x}
  \and
  \inferrule*[lab=Par-barb]{\mbox{$P\vdash x$ or $Q\vdash x$}}{\binpar{P}{Q} \vdash x}
\end{mathpar}

\subsubsection{Contexts}

One of the principle advantages of computational calculi like the
$\pi$-calculus is a well-defined notion of context,
contextual-equivalence and a correlation between
contextual-equivalence and notions of bisimulation. The notion of
context allows the decomposition of a process into (sub-)process and
its syntactic environment, its context. Thus, a context may be
thought of as a process with a ``hole'' (written $\Box$) in it. The
application of a context $M$ to a process $P$, written $M[P]$, is
tantamount to filling the hole in $M$ with $P$. In this paper we do
not need the full weight of this theory, but do make use of the notion
of context in the proof the main theorem. 

\begin{mathpar}
  \inferrule* [lab=summation] {} {{M_{M},M_{N}} \bc \Box \;|\; x.M_{A} \;|\; M_{M}+M_{N}}
  \and
  \inferrule* [lab=agent] {} {{M_{A}} \bc (\vec{x})M_{P} \;| \; \clift{P_0,\ldots,M_{P},\ldots,P_N}}
  \and \\
  \inferrule* [lab=process] {} {{M_{P}} \bc M_{N} \;| \;P|M_{P} }
\end{mathpar} 

\begin{mathpar}
  \inferrule* [lab=sychronization] {} {M_{N} \bc \Box \;|\; x?M_{F} \;|\; x!M_{C}}
  \and
  \inferrule* [lab=abstraction] {} {{M_{F}} \bc (x)M_{P} }
  \and
  \inferrule* [lab=concretion] {} {{M_{C}} \bc \langle M_{P} \rangle }
  \and \\
  \inferrule* [lab=process] {} {{M_{P}} \bc M_{N} \;| \;P|M_{P} }
\end{mathpar}

\begin{definition}[contextual application] Given a context $M$, and
  process $P$, we define the \emph{contextual application}, $M[P] :=
  M\{P/\Box\}$. That is, the contextual application of M to P is the
  substitution of $P$ for $\Box$ in $M$.
\end{definition}

$\meaningof{-} : L \to \mathcal{P}(\pi)$

\begin{mathpar}
  \inferrule* [lab=collection] {} {\meaningof{true} = \pi, \and \meaningof{~E} = \pi \setminus \meaningof{E}, \and \meaningof{E_{1} \& E_{2}} = \meaningof{E_{1}} \cap \meaningof{E_{2}}}
\end{mathpar}

\begin{mathpar}
  \inferrule* [lab=structure] {} {\meaningof{0} = \{ P \in \pi | P \equiv 0 \}, \and \\ \meaningof{E_1 | E_2} = \{ P \in \pi | P \equiv P_{1} | P_{2}, P_{1} \in \meaningof{E_{1}}, P_{2} \in \meaningof{E_2}\} }
\end{mathpar}

\begin{mathpar}
 \inferrule* [lab=behavior] {} {\meaningof{\langle a?b \rangle E} = \{ P \in \pi | P \equiv Q | u?(y)P', \\ \and \\\\ \and \\ \;\;\; u \in \meaningof{a}, \forall z.P'\{z/y\} \in \meaningof{E\{z/b\}}\}, \and \\ \meaningof{a!E} = \{ P \in \pi | P \equiv Q | x!\langle P' \rangle, x \in \meaningof{a} P' \in \meaningof{E}\} }
\end{mathpar}

\begin{mathpar}
 \inferrule* [lab=nominal] {} {\meaningof{\quotep{E}} = \{ \quotep{P} \in \quotep{\pi} | P \in \meaningof{E} \}, \and \meaningof{\quotep{P}} = \{ \quotep{Q} \in \quotep{\pi} | P \equiv Q \} \and \\ \meaningof{@\quotep{E}} = \{ P \in \pi | P \equiv @x, x \in \meaningof{E} \}}
\end{mathpar}

\begin{eqnarray*}
  \\
  \meaningof{-} : TS \to ST
\end{eqnarray*}

\begin{eqnarray*}
  \\
  L : TS \to ST
\end{eqnarray*}

\begin{eqnarray*}
  \\
  P \models E \iff P \in \meaningof{E}
\end{eqnarray*}

\begin{eqnarray*}
  P \approx_{L} Q \iff \forall E \in L. P \models E \iff Q \models E
\end{eqnarray*}

\begin{eqnarray*}
  P \approx_{K} Q
\end{eqnarray*}

\begin{eqnarray*}
  P \approx Q
\end{eqnarray*}

$\approx_{K} = \approx = \approx_{L}$

\subsubsection{Contextual duality}

Note that contexts extend the quotation operation to a family of
operations from processes to names. Given a context, $M$, we can
define a \emph{nominal context}, $\quotep{M}$ by $\quotep{M}[P] :=
\quotep{M[P]}$. To foreshadow what is to come we observe that these
operations enjoy a duality with processes very much like the duality
between vectors and maps from vectors to scalars.

Further, because the calculus is essentially higher-order, we have a
correspondence between contexts and processes. More specifically,
given a name $x$ and a context $M$ we can construct $M^{*}_{x}$ such
that 

\begin{mathpar}
  M^{*}_{x} | \lift{x}{P} \red M[P]
\end{mathpar}

namely,

\begin{mathpar}
  M^{*}_{x} := x?(u).M[\dropn{u}]
\end{mathpar}

The dependence of $M^{*}_{x}$ on a name makes it an abstraction, 

\begin{mathpar}
  M^{*} := (x)x?(u).M[\dropn{u}]
\end{mathpar}

\subsection{Additional notation}

It will sometimes be convenient to denote the process a name
quotes. We already have the notation $x = \quotep{P}$, but it will be
convenient to introduce an alternate notation, $\procn{x}$, when we
want to emphasize the connection to the use of the name. Note that, by
virtue of name equivalence, $\quotep{\procn{x}} \nameeq x$; so, the
notation is consistent with previous definitions.

Further, because names have structure it is possible to effect
substitutions on the basis of that structure. This means we need to
upgrade our notation for substitutions, which we accomplish by
adapting comprehension notation. Thus,

\begin{mathpar}
  P\{ y / x : x \in S \}
\end{mathpar}

is interpreted to mean the process derived from P by replacing (in a
capture-avoiding manner) each occurrence of $x$ in $S$ by $y$. For example,

\begin{mathpar}
  P\{ \quotep{\procn{x}|\procn{x}} / x : x \in \freenames{P} \}
\end{mathpar}

will replace each (occurrence) of a free name $x$ in $P$ by
$\quotep{\procn{x}|\procn{x}}$.

Also, we will avail ourselves of the notation $x^{L}$ and $x^{R}$ to
denote injections of a name into disjoint copies of the name
space. There are numerous ways to accomplish this. One example can be
found in \cite{MeredithR05}. This notation overloads to vectors of
names: $\vec{x}^{\pi} := (x_{i}^{\pi} \; : \; 0 \leq i < |\vec{x}| )$ where $\pi \in \{L,R\}$.

We also use $P^{\Box} := P|\Box$.

In \cite{MeredithR05} an interpretation of the new operator is
given. It turns out that there are several possible interpretations
all enjoying the requisite algebraic properties of the operator (see
\cite{milner91polyadicpi}). We will therefore make liberal use of
$(\nu\; \vec{x})P$.

% subsection the_syntax_and_semantics_of_the_notation_system (end)   

\input{qm2pi.qmops} 

\input{qm2pi.sterngerlach} 

\input{qm2pi.metric} 

% section concurrent_process_calculi (end)

%\input{qm2pi.proofsketch}

% section proof sketch (end)

%\input{qm2pi.slviaknots} 

% section spatial logic via knots (end)

\input{qm2pi.conclusion}

% section conclusion (end)

%\input{qm2pi.dtcodes} 

% section wiring algorithm (end)

\input{qm2pi.ack} 

% section acknowledgments (end)

\newpage


\bibliographystyle{plain}   
\bibliography{../../biblios/main.bib}

\input{qm2pi.rhodetails}

\end{document}

 

%\documentclass[12pt]{llncs}
%\documentclass{jktr}

\usepackage[pdftex]{hyperref}                   
\usepackage {listings}
\usepackage {mathpartir}
\usepackage{bcprules}
%\usepackage{listings}
                       
\usepackage{graphicx} 
%\usepackage[margins=2.5cm,nohead,nofoot]{geometry}
%\usepackage{geometry}
\usepackage{amsfonts}
\usepackage{amstext}
\usepackage{latexsym}
\usepackage{amssymb}
\usepackage{color}


%\include{myPreamble}
\include{qm2pi.local} 

%\ifpdf
%\usepackage[pdftex]{graphicx}
%\else
%\usepackage{graphicx}
%\fi

 % \ifpdf
%  \usepackage{pdfsync}
%  \if


%\title{Brief Article}
%\author{David F. Snyder}
%\author{L.G. Meredith}

%\address{Dept. of Math., Texas State University--San Marcos, San Marcos, TX 78666}
       
\pagestyle{empty}


\begin{document}

\lstset{language=[Objective]Caml,frame=shadowbox}

\input{qm2pi.front}

% section front matter (end)

\input{qm2pi.intro} 
 
% section introduction (end)

% \input{qm2pi.knotations} 

% section notation (end)

\input{qm2pi.process.calculi} 

% section concurrent_process_calculi_and_spatial_logics_ (end)
    
%\input{qm2pi.knots2pi} 

%\input{qm2pi.trefoil} 

%\input{qm2pi.mainthm} 

% subsection basic_interpretation (end)

%\input{qm2pi.rho.presentation} 
\subsection{The syntax and semantics of the notation system}\label{sub:the_syntax_and_semantics_of_the_notation_system} % (fold)

We now summarize a technical presentation of the calculus that
embodies our theory of dynamics. The typical presentation of such a
calculus follows the style of giving generators and relations on
them. The grammar, below, describing term constructors, freely
generates the set of processes, $\Proc$. This set is then quotiented
by a relation known as structural congruence and it is over this set
that the notion of dynamics is expressed. This presentation is
essentially that of \cite{MeredithR05} with the addition of
polyadicity and summation. For readability we have relegated some of
the technical subtleties to an appendix.

\subsubsection{Process grammar}\label{subsub:process_grammar}

\begin{mathpar}
  \inferrule* [lab=synchronization] {} {{M} \bc \pzero \;|\; x?F \;|\; x!C }
  \and
  \inferrule* [lab=abstraction] {} {{F} \bc (x)P}
  \and
  \inferrule* [lab=concretion] {} {{C} \bc \langle Q \rangle}
  \and
  \inferrule* [lab=process] {} {{P,Q} \bc M \;| \;P|Q \;|\; @{x}}
  \and
  \inferrule* [lab=name] {} {{x} \bc \quotep{P}}
\end{mathpar} 

Note that $\vec{x}$ (resp. $\vec{P}$) denotes a vector of names
(resp. processes) of length $|\vec{x}|$ (resp. $|\vec{P}|$). We adopt
the following useful abbreviations.

\begin{mathpar}
   x?(\vec{y}).P := x.(\vec{y})P \and  x\clift{\vec{P}} := x.\clift{\vec{P}}
   \and x!(y) := \lift{x}{\dropn{y}}
   \and \Pi_{i=0}^{n-1}P_i := P_0 | \ldots | P_{n-1}
\end{mathpar}

\subsubsection{Structural congruence}

\paragraph{Free and bound names and alpha-equivalence.} At the
core of structural equivalence is alpha-equivalence which identifies
process that are the same up to a change of variable. Formally, we
recognize the distinction between free and bound names. The free names
of a process, $\freenames{P}$, may be calculated recursively as
follows:

\begin{mathpar}
\freenames{\pzero} := \emptyset
  \and \\
  \freenames{x?(y).P} := \{ x \} \cup (\freenames{P} \setminus \{ y \})
  \and 
  \freenames{x!\langle P \rangle} := \{ x \} \cup \{ P \} 
  \and \\
  \freenames{P|Q} := \freenames{P} \cup \freenames{Q}
  \and \\
  \freenames{@{x}} := \{ x \}
\end{mathpar}

$\pi$
$\quotep{\pi}$

$\freenames{-} : \pi \to \mathcal{P}(\quotep{\pi})$

\begin{eqnarray*}
  \freenames{\pzero} & := & \emptyset \\
  \freenames{x?(y).P} & := & \{ x \} \cup (\freenames{P} \setminus \{ y \}) \\
  \freenames{x!\langle P \rangle} & := & \{ x \} \cup \{ P \} \\
  \freenames{P|Q} & := & \freenames{P} \cup \freenames{Q} \\
  \freenames{\dropn{x}} & := & \{ x \}
\end{eqnarray*}

The bound names of a process, $\boundnames{P}$, are those names occurring in $P$
that are not free. For example, in $x?(y).0$, the name $x$ is free, while $y$ is bound.

\begin{mathpar}
  \inferrule* [lab=monoidal-laws] {} { P|Q \equiv Q|P \and P|0 \equiv P \and P|(Q|R) \equiv (P|Q)|R }
\end{mathpar}

\begin{mathpar}
  \inferrule* [lab=alpha-equivalence] {} { (x)P \equiv (y)P\{y/x\} \and y \not\in \freenames{P} }
\end{mathpar}

\begin{definition}
Then two processes, $P,Q$, are alpha-equivalent if $P = Q\{\vec{y}/\vec{x}\}$ for
some $\vec{x} \in \boundnames{Q},\vec{y} \in \boundnames{P}$, where $Q\{\vec{y}/\vec{x}\}$
denotes the capture-avoiding substitution of $\vec{y}$ for $\vec{x}$ in $Q$.
\end{definition}

\begin{definition}
  The {\em structural congruence} \cite{SangiorgiWalker} , $\equiv$,
  between processes is the least congruence containing
  alpha-equivalence, satisfying the abelian monoid laws
  (associativity, commutativity and $\pzero$ as identity) for parallel
  composition $|$ and for summation $+$.
\end{definition}

\subsection{Name equivalence}

We take name equivalence, written $\nameeq$, to be the smallest
equivalence relation generated by the following rules.

\begin{mathpar}
\inferrule*[lab=Quote-drop]
{ }
{ \quotep{@{x}} \nameeq x }

\inferrule*[lab=Struct-equiv]
{ P \scong Q }
{ \quotep{P} \nameeq \quotep{Q} }
\end{mathpar}

The astute reader will have noticed that the mutual recursion of names
and processes imposes a mutual recursion on alpha-equivalence and
structural equivalence via name-equivalence. Fortunately, all of this
works out pleasantly and we may calculate in the natural way, free of
concern. The reader interested in the details is referred to the
appendix \ref{appendix:rho_details}.

\subsection{Substitution}

We use $\Proc$ for the set of processes, $\QProc$ for the set of
names, and $\id{\{}\vec{y} / \vec{x} \id{\}}$ to denote partial maps,
$s : \QProc \rightarrow \QProc$. A map, $s$ lifts, uniquely, to a map
on process terms, $\widehat{s} : \Proc \rightarrow \Proc$ by the
following equations.

\begin{mathpar}
  (0) \psubstp{Q}{P} := 0 \\
  (R \juxtap S) \psubstp{Q}{P}
  :=    
  (R)\psubstp{Q}{P} \juxtap (S) \psubstp{Q}{P} \\
  (x?(y).R) \psubstp{Q}{P}    
  :=    
  (x)\substp{Q}{P} (z)\concat( (R \psubstn{z}{y}) \psubstp{Q}{P} ) \\
  (\lift{x}{R}) \psubstp{Q}{P}  
  :=
  \lift{(x)\substp{Q}{P}}{ R \psubstp{Q}{P} } \\
%   (\dropn{x})  \psubstp{Q}{P}       
%   := 
%   \left\{ 
%     \begin{array}{ccc} 
%       \dropn{\quotep{Q}} & & x \nameeq \quotep{P} \\
%       \dropn{x} & & otherwise \\
%     \end{array}
%   \right. 
  (\dropn{x})  \psubstp{Q}{P}       
  := 
  \left\{ 
    \begin{array}{ccc} 
      Q & & x \nameeq \quotep{P} \\
      \dropn{x} & & otherwise \\
    \end{array}
  \right.
\end{mathpar}
 

where

\begin{eqnarray}
  (x)\id{\{} \lpquote Q \rpquote / \lpquote P \rpquote \id{\}}            = 
  \left\{ 
    \begin{array}{ccc}
      \lpquote Q \rpquote & & x \nameeq \lpquote P \rpquote \\
      x & & otherwise \\
    \end{array}
  \right. \nonumber
\end{eqnarray}

and $z$ is chosen distinct from $\quotep{P}$, $\quotep{Q}$, the free
names in $Q$, and all the names in $R$. Our $\alpha$-equivalence will
be built in the standard way from this substitution.

\begin{remark}\label{rem:no_self_referential_names}
  One consequence of these definitions is that $\forall P. \quotep{P}
  \not\in \freenames{P}$.
\end{remark}

\subsection{ Dynamic quote: an example }

Anticipating something of what's to come, consider applying the
substitution, $\widehat{\id{\{}u / z \id{\}}}$, to the following pair
of processes, $\lift{w}{y!(z)}$ and $w[ \lpquote y!(z) \rpquote ]$.

\begin{eqnarray}
	\lift{w}{y!(z)}\widehat{\id{\{}u / z \id{\}}}
		& = &
		\lift{w}{y!(u)} \nonumber\\
	w[ \lpquote y!(z) \rpquote ] \widehat{ \id{\{}u / z \id{\}} }
		& = &
		w[ \lpquote y!(z) \rpquote ] \nonumber
\end{eqnarray}

Because the body of the process between quotes is impervious to
substitution, we get radically different answers. In fact, by
examining the first process in an input context,
e.g. $x?(z).\lift{w}{y!(z)}$, we see that the process under the lift
operator may be shaped by prefixed inputs binding a name inside it. In
this sense, the lift operator will be seen as a way to dynamically
construct processes before reifying them as names.

Finally equipped with these standard features we can present the
dynamics of the calculus.

\subsubsection{Operational semantics} 

Finally, we introduce the computational dynamics. What marks these
algebras as distinct from other more traditionally studied algebraic
structures, e.g. vector spaces or polynomial rings, is the manner in
which dynamics is captured. In traditional structures, dynamics is typically
expressed through morphisms between such structures, as in linear maps
between vector spaces or morphisms between rings. In algebras
associated with the semantics of computation, the dynamics is
expressed as part of the algebraic structure itself, through a
reduction reduction relation typically denoted by $\red$. Below, we
give a recursive presentation of this relation for the calculus used
in the encoding.

$\red \subseteq \pi \times \pi$
$\red : \pi \to \mathcal{P}(\pi)$

\begin{mathpar}
  \inferrule* [lab=Comm] { \textsf{match}( x_{src}, x_{trgt} ) } { x_{trgt}?(y)P \; | \; x_{src}!\langle {Q} \rangle \red P\{\quotep{Q}/y}\} }
  \and \\
  \inferrule* [lab=Par] {{P} \red {P}'} {{{P} | {Q}} \red {{P}' | {Q}}}
  \and
  \inferrule* [lab=Equiv]{{{P} \scong {P}'} \andalso {{P}' \red {Q}'} \andalso {{Q}' \scong {Q}}}{{P} \red {Q}}
\end{mathpar}

\begin{eqnarray*}
  match_{\equiv} (\quotep{P},\quotep{Q}) & := & P \equiv Q \\
  match_{\dagger}(\quotep{P},\quotep{Q}) & := & \forall R. P|Q \red^{*} R => R \red^{*} 0 \\
  match_{K}(\quotep{P},\quotep{Q}) & := & K \mbox{ for some context } K
\end{eqnarray*}

$u?(x)P | u!\langle Q \rangle \red P\{\quotep{Q}/x\}$

%We write $\wred$ for $\red^*$, and $P\red$ if $\exists Q $ such that $ P \red Q$.
We write $P\red$ if $\exists Q $ such that $ P \red Q$ and $P\not\red$, otherwise.

\section{Replication}

As mentioned before, it is known that replication (and hence
recursion) can be implemented in a higher-order process algebra
\cite{SangiorgiWalker}. As our first example of calculation with the
machinery thus far presented we give the construction explicitly in
the {\rhoc}.

\begin{eqnarray}
	D_{x} & := & \prefix{x}{y}{(\binpar{\outputp{x}{y}}{@{y}})} \nonumber\\
	\bangp_{x}{P} & := & \binpar{{x}!\langle{\binpar{D_{x}}{P}}\rangle}{D_{x}} \nonumber
\end{eqnarray}

\begin{eqnarray}
	\bangp_{x}{P} & & \nonumber\\
	=
	& {x}!\langle{(\prefix{x}{y}{(\outputp{x}{y} | @{y})) | P}}\rangle 
	      | \prefix{x}{y}{(\outputp{x}{y} | @{y})} & \nonumber\\
	\red
	& (\outputp{x}{y} | @{y})\substn{\quotep{(\prefix{x}{y}{(@{y} | \outputp{x}{y})) | P}}}{y} & \nonumber\\
	=
	& \outputp{x}{\quotep{(\prefix{x}{y}{(\outputp{x}{y} | @{y})) | P}}}
	  | {(\prefix{x}{y}{(\outputp{x}{y} | @{y})) | P}} & \nonumber\\
	\red
	& \ldots & \nonumber\\
	\red^*
	& P | P | \ldots & \nonumber
\end{eqnarray}

Of course, this encoding, as an implementation, runs away, unfolding
$\bangp{P}$ eagerly. A lazier and more implementable replication
operator, restricted to input-guarded processes, may be obtained as follows.

\begin{eqnarray}
\bangp{\prefix{u}{v}{P}} 
	:= 
	\binpar{\lift{x}{\prefix{u}{v}{(\binpar{D(x)}{P})}}}{D(x)} \nonumber
\end{eqnarray}

\begin{remark}
  Note that the lazier definition still does not deal with summation
  or mixed summation (i.e. sums over input and output). The reader is
  invited to construct definitions of replication that deal with these
  features. 

  Further, the definitions are parameterized in a name, $x$. Can you,
  gentle reader, make a definition that eliminates this parameter and
  guarantees no accidental interaction between the replication
  machinery and the process being replicated -- i.e. no accidental
  sharing of names used by the process to get its work done and the
  name(s) used by the replication to effect copying. This latter
  revision of the definition of replication is crucial to obtaining
  the expected identity $!!P \sim !P$.
\end{remark}

\begin{remark}\label{rem:paradoxical_combinator}
  The reader familiar with the lambda calculus will have noticed the
  similarity between $D$ and the paradoxical combinator.

  [Ed. note: the existence of this seems to suggest we have to be more
  restrictive on the set of processes and names we admit if we are to
  support no-cloning.]
\end{remark}

\subsubsection{Bisimulation}

The computational dynamics gives rise to another kind of equivalence,
the equivalence of computational behavior. As previously mentioned
this is typically captured \emph{via} some form of bisimulation.

% The notion we use in this paper is weak barbed bisimulation
% \cite{milner91polyadicpi}.

The notion we use in this paper is derived from weak barbed
bisimulation \cite{milner91polyadicpi}. 

\begin{definition}
An \emph{observation relation}, $\downarrow_{\mathcal N}$, over a set
of names, $\mathcal N$, is the smallest relation satisfying the rules
below.

\infrule[Out-barb]{y \in {\mathcal N}, \; x \nameeq y}
		  {\outputp{x}{v} \downarrow_{\mathcal N} x}
\infrule[Par-barb]{\mbox{$P\downarrow_{\mathcal N} x$ or $Q\downarrow_{\mathcal N} x$}}
		  {\binpar{P}{Q} \downarrow_{\mathcal N} x}

We write $P \Downarrow_{\mathcal N} x$ if there is $Q$ such that 
$P \wred Q$ and $Q \downarrow_{\mathcal N} x$.
\end{definition}

\begin{definition}
%\label{def.bbisim}
An  ${\mathcal N}$-\emph{barbed bisimulation} over a set of names, ${\mathcal N}$, is a symmetric binary relation 
${\mathcal S}_{\mathcal N}$ between agents such that $P\rel{S}_{\mathcal N}Q$ implies:
\begin{enumerate}
\item If $P \red P'$ then $Q \wred Q'$ and $P'\rel{S}_{\mathcal N} Q'$.
\item If $P\downarrow_{\mathcal N} x$, then $Q\Downarrow_{\mathcal N} x$.
\end{enumerate}
$P$ is ${\mathcal N}$-barbed bisimilar to $Q$, written
$P \wbbisim_{\mathcal N} Q$, if $P \rel{S}_{\mathcal N} Q$ for some ${\mathcal N}$-barbed bisimulation ${\mathcal S}_{\mathcal N}$.
\end{definition}

$\mathcal{R} \subseteq \pi \times \pi$

$P \mathcal{R} Q => \forall P'. P \red P' \Rightarrow \exists Q'. Q \red Q', P' \mathcal{R} Q'$

$P \vdash x \Rightarrow Q \vdash x$

\begin{mathpar}
  \inferrule*[lab=Out-barb]{x \nameeq y}{{y}!\langle{Q}\rangle \vdash x}
  \and
  \inferrule*[lab=Par-barb]{\mbox{$P\vdash x$ or $Q\vdash x$}}{\binpar{P}{Q} \vdash x}
\end{mathpar}

\subsubsection{Contexts}

One of the principle advantages of computational calculi like the
$\pi$-calculus is a well-defined notion of context,
contextual-equivalence and a correlation between
contextual-equivalence and notions of bisimulation. The notion of
context allows the decomposition of a process into (sub-)process and
its syntactic environment, its context. Thus, a context may be
thought of as a process with a ``hole'' (written $\Box$) in it. The
application of a context $M$ to a process $P$, written $M[P]$, is
tantamount to filling the hole in $M$ with $P$. In this paper we do
not need the full weight of this theory, but do make use of the notion
of context in the proof the main theorem. 

\begin{mathpar}
  \inferrule* [lab=summation] {} {{M_{M},M_{N}} \bc \Box \;|\; x.M_{A} \;|\; M_{M}+M_{N}}
  \and
  \inferrule* [lab=agent] {} {{M_{A}} \bc (\vec{x})M_{P} \;| \; \clift{P_0,\ldots,M_{P},\ldots,P_N}}
  \and \\
  \inferrule* [lab=process] {} {{M_{P}} \bc M_{N} \;| \;P|M_{P} }
\end{mathpar} 

\begin{mathpar}
  \inferrule* [lab=sychronization] {} {M_{N} \bc \Box \;|\; x?M_{F} \;|\; x!M_{C}}
  \and
  \inferrule* [lab=abstraction] {} {{M_{F}} \bc (x)M_{P} }
  \and
  \inferrule* [lab=concretion] {} {{M_{C}} \bc \langle M_{P} \rangle }
  \and \\
  \inferrule* [lab=process] {} {{M_{P}} \bc M_{N} \;| \;P|M_{P} }
\end{mathpar}

\begin{definition}[contextual application] Given a context $M$, and
  process $P$, we define the \emph{contextual application}, $M[P] :=
  M\{P/\Box\}$. That is, the contextual application of M to P is the
  substitution of $P$ for $\Box$ in $M$.
\end{definition}

$\meaningof{-} : L \to \mathcal{P}(\pi)$

\begin{mathpar}
  \inferrule* [lab=collection] {} {\meaningof{true} = \pi, \and \meaningof{~E} = \pi \setminus \meaningof{E}, \and \meaningof{E_{1} \& E_{2}} = \meaningof{E_{1}} \cap \meaningof{E_{2}}}
\end{mathpar}

\begin{mathpar}
  \inferrule* [lab=structure] {} {\meaningof{0} = \{ P \in \pi | P \equiv 0 \}, \and \\ \meaningof{E_1 | E_2} = \{ P \in \pi | P \equiv P_{1} | P_{2}, P_{1} \in \meaningof{E_{1}}, P_{2} \in \meaningof{E_2}\} }
\end{mathpar}

\begin{mathpar}
 \inferrule* [lab=behavior] {} {\meaningof{\langle a?b \rangle E} = \{ P \in \pi | P \equiv Q | u?(y)P', \\ \and \\\\ \and \\ \;\;\; u \in \meaningof{a}, \forall z.P'\{z/y\} \in \meaningof{E\{z/b\}}\}, \and \\ \meaningof{a!E} = \{ P \in \pi | P \equiv Q | x!\langle P' \rangle, x \in \meaningof{a} P' \in \meaningof{E}\} }
\end{mathpar}

\begin{mathpar}
 \inferrule* [lab=nominal] {} {\meaningof{\quotep{E}} = \{ \quotep{P} \in \quotep{\pi} | P \in \meaningof{E} \}, \and \meaningof{\quotep{P}} = \{ \quotep{Q} \in \quotep{\pi} | P \equiv Q \} \and \\ \meaningof{@\quotep{E}} = \{ P \in \pi | P \equiv @x, x \in \meaningof{E} \}}
\end{mathpar}

\begin{eqnarray*}
  \\
  \meaningof{-} : TS \to ST
\end{eqnarray*}

\begin{eqnarray*}
  \\
  L : TS \to ST
\end{eqnarray*}

\begin{eqnarray*}
  \\
  P \models E \iff P \in \meaningof{E}
\end{eqnarray*}

\begin{eqnarray*}
  P \approx_{L} Q \iff \forall E \in L. P \models E \iff Q \models E
\end{eqnarray*}

\begin{eqnarray*}
  P \approx_{K} Q
\end{eqnarray*}

\begin{eqnarray*}
  P \approx Q
\end{eqnarray*}

$\approx_{K} = \approx = \approx_{L}$

\subsubsection{Contextual duality}

Note that contexts extend the quotation operation to a family of
operations from processes to names. Given a context, $M$, we can
define a \emph{nominal context}, $\quotep{M}$ by $\quotep{M}[P] :=
\quotep{M[P]}$. To foreshadow what is to come we observe that these
operations enjoy a duality with processes very much like the duality
between vectors and maps from vectors to scalars.

Further, because the calculus is essentially higher-order, we have a
correspondence between contexts and processes. More specifically,
given a name $x$ and a context $M$ we can construct $M^{*}_{x}$ such
that 

\begin{mathpar}
  M^{*}_{x} | \lift{x}{P} \red M[P]
\end{mathpar}

namely,

\begin{mathpar}
  M^{*}_{x} := x?(u).M[\dropn{u}]
\end{mathpar}

The dependence of $M^{*}_{x}$ on a name makes it an abstraction, 

\begin{mathpar}
  M^{*} := (x)x?(u).M[\dropn{u}]
\end{mathpar}

\subsection{Additional notation}

It will sometimes be convenient to denote the process a name
quotes. We already have the notation $x = \quotep{P}$, but it will be
convenient to introduce an alternate notation, $\procn{x}$, when we
want to emphasize the connection to the use of the name. Note that, by
virtue of name equivalence, $\quotep{\procn{x}} \nameeq x$; so, the
notation is consistent with previous definitions.

Further, because names have structure it is possible to effect
substitutions on the basis of that structure. This means we need to
upgrade our notation for substitutions, which we accomplish by
adapting comprehension notation. Thus,

\begin{mathpar}
  P\{ y / x : x \in S \}
\end{mathpar}

is interpreted to mean the process derived from P by replacing (in a
capture-avoiding manner) each occurrence of $x$ in $S$ by $y$. For example,

\begin{mathpar}
  P\{ \quotep{\procn{x}|\procn{x}} / x : x \in \freenames{P} \}
\end{mathpar}

will replace each (occurrence) of a free name $x$ in $P$ by
$\quotep{\procn{x}|\procn{x}}$.

Also, we will avail ourselves of the notation $x^{L}$ and $x^{R}$ to
denote injections of a name into disjoint copies of the name
space. There are numerous ways to accomplish this. One example can be
found in \cite{MeredithR05}. This notation overloads to vectors of
names: $\vec{x}^{\pi} := (x_{i}^{\pi} \; : \; 0 \leq i < |\vec{x}| )$ where $\pi \in \{L,R\}$.

We also use $P^{\Box} := P|\Box$.

In \cite{MeredithR05} an interpretation of the new operator is
given. It turns out that there are several possible interpretations
all enjoying the requisite algebraic properties of the operator (see
\cite{milner91polyadicpi}). We will therefore make liberal use of
$(\nu\; \vec{x})P$.

% subsection the_syntax_and_semantics_of_the_notation_system (end)   

\input{qm2pi.qmops} 

\input{qm2pi.sterngerlach} 

\input{qm2pi.metric} 

% section concurrent_process_calculi (end)

%\input{qm2pi.proofsketch}

% section proof sketch (end)

%\input{qm2pi.slviaknots} 

% section spatial logic via knots (end)

\input{qm2pi.conclusion}

% section conclusion (end)

%\input{qm2pi.dtcodes} 

% section wiring algorithm (end)

\input{qm2pi.ack} 

% section acknowledgments (end)

\newpage


\bibliographystyle{plain}   
\bibliography{../../biblios/main.bib}

\input{qm2pi.rhodetails}

\end{document}

 

%\documentclass[12pt]{llncs}
%\documentclass{jktr}

\usepackage[pdftex]{hyperref}                   
\usepackage {listings}
\usepackage {mathpartir}
\usepackage{bcprules}
%\usepackage{listings}
                       
\usepackage{graphicx} 
%\usepackage[margins=2.5cm,nohead,nofoot]{geometry}
%\usepackage{geometry}
\usepackage{amsfonts}
\usepackage{amstext}
\usepackage{latexsym}
\usepackage{amssymb}
\usepackage{color}


%\include{myPreamble}
\include{qm2pi.local} 

%\ifpdf
%\usepackage[pdftex]{graphicx}
%\else
%\usepackage{graphicx}
%\fi

 % \ifpdf
%  \usepackage{pdfsync}
%  \if


%\title{Brief Article}
%\author{David F. Snyder}
%\author{L.G. Meredith}

%\address{Dept. of Math., Texas State University--San Marcos, San Marcos, TX 78666}
       
\pagestyle{empty}


\begin{document}

\lstset{language=[Objective]Caml,frame=shadowbox}

\input{qm2pi.front}

% section front matter (end)

\input{qm2pi.intro} 
 
% section introduction (end)

% \input{qm2pi.knotations} 

% section notation (end)

\input{qm2pi.process.calculi} 

% section concurrent_process_calculi_and_spatial_logics_ (end)
    
%\input{qm2pi.knots2pi} 

%\input{qm2pi.trefoil} 

%\input{qm2pi.mainthm} 

% subsection basic_interpretation (end)

%\input{qm2pi.rho.presentation} 
\subsection{The syntax and semantics of the notation system}\label{sub:the_syntax_and_semantics_of_the_notation_system} % (fold)

We now summarize a technical presentation of the calculus that
embodies our theory of dynamics. The typical presentation of such a
calculus follows the style of giving generators and relations on
them. The grammar, below, describing term constructors, freely
generates the set of processes, $\Proc$. This set is then quotiented
by a relation known as structural congruence and it is over this set
that the notion of dynamics is expressed. This presentation is
essentially that of \cite{MeredithR05} with the addition of
polyadicity and summation. For readability we have relegated some of
the technical subtleties to an appendix.

\subsubsection{Process grammar}\label{subsub:process_grammar}

\begin{mathpar}
  \inferrule* [lab=synchronization] {} {{M} \bc \pzero \;|\; x?F \;|\; x!C }
  \and
  \inferrule* [lab=abstraction] {} {{F} \bc (x)P}
  \and
  \inferrule* [lab=concretion] {} {{C} \bc \langle Q \rangle}
  \and
  \inferrule* [lab=process] {} {{P,Q} \bc M \;| \;P|Q \;|\; @{x}}
  \and
  \inferrule* [lab=name] {} {{x} \bc \quotep{P}}
\end{mathpar} 

Note that $\vec{x}$ (resp. $\vec{P}$) denotes a vector of names
(resp. processes) of length $|\vec{x}|$ (resp. $|\vec{P}|$). We adopt
the following useful abbreviations.

\begin{mathpar}
   x?(\vec{y}).P := x.(\vec{y})P \and  x\clift{\vec{P}} := x.\clift{\vec{P}}
   \and x!(y) := \lift{x}{\dropn{y}}
   \and \Pi_{i=0}^{n-1}P_i := P_0 | \ldots | P_{n-1}
\end{mathpar}

\subsubsection{Structural congruence}

\paragraph{Free and bound names and alpha-equivalence.} At the
core of structural equivalence is alpha-equivalence which identifies
process that are the same up to a change of variable. Formally, we
recognize the distinction between free and bound names. The free names
of a process, $\freenames{P}$, may be calculated recursively as
follows:

\begin{mathpar}
\freenames{\pzero} := \emptyset
  \and \\
  \freenames{x?(y).P} := \{ x \} \cup (\freenames{P} \setminus \{ y \})
  \and 
  \freenames{x!\langle P \rangle} := \{ x \} \cup \{ P \} 
  \and \\
  \freenames{P|Q} := \freenames{P} \cup \freenames{Q}
  \and \\
  \freenames{@{x}} := \{ x \}
\end{mathpar}

$\pi$
$\quotep{\pi}$

$\freenames{-} : \pi \to \mathcal{P}(\quotep{\pi})$

\begin{eqnarray*}
  \freenames{\pzero} & := & \emptyset \\
  \freenames{x?(y).P} & := & \{ x \} \cup (\freenames{P} \setminus \{ y \}) \\
  \freenames{x!\langle P \rangle} & := & \{ x \} \cup \{ P \} \\
  \freenames{P|Q} & := & \freenames{P} \cup \freenames{Q} \\
  \freenames{\dropn{x}} & := & \{ x \}
\end{eqnarray*}

The bound names of a process, $\boundnames{P}$, are those names occurring in $P$
that are not free. For example, in $x?(y).0$, the name $x$ is free, while $y$ is bound.

\begin{mathpar}
  \inferrule* [lab=monoidal-laws] {} { P|Q \equiv Q|P \and P|0 \equiv P \and P|(Q|R) \equiv (P|Q)|R }
\end{mathpar}

\begin{mathpar}
  \inferrule* [lab=alpha-equivalence] {} { (x)P \equiv (y)P\{y/x\} \and y \not\in \freenames{P} }
\end{mathpar}

\begin{definition}
Then two processes, $P,Q$, are alpha-equivalent if $P = Q\{\vec{y}/\vec{x}\}$ for
some $\vec{x} \in \boundnames{Q},\vec{y} \in \boundnames{P}$, where $Q\{\vec{y}/\vec{x}\}$
denotes the capture-avoiding substitution of $\vec{y}$ for $\vec{x}$ in $Q$.
\end{definition}

\begin{definition}
  The {\em structural congruence} \cite{SangiorgiWalker} , $\equiv$,
  between processes is the least congruence containing
  alpha-equivalence, satisfying the abelian monoid laws
  (associativity, commutativity and $\pzero$ as identity) for parallel
  composition $|$ and for summation $+$.
\end{definition}

\subsection{Name equivalence}

We take name equivalence, written $\nameeq$, to be the smallest
equivalence relation generated by the following rules.

\begin{mathpar}
\inferrule*[lab=Quote-drop]
{ }
{ \quotep{@{x}} \nameeq x }

\inferrule*[lab=Struct-equiv]
{ P \scong Q }
{ \quotep{P} \nameeq \quotep{Q} }
\end{mathpar}

The astute reader will have noticed that the mutual recursion of names
and processes imposes a mutual recursion on alpha-equivalence and
structural equivalence via name-equivalence. Fortunately, all of this
works out pleasantly and we may calculate in the natural way, free of
concern. The reader interested in the details is referred to the
appendix \ref{appendix:rho_details}.

\subsection{Substitution}

We use $\Proc$ for the set of processes, $\QProc$ for the set of
names, and $\id{\{}\vec{y} / \vec{x} \id{\}}$ to denote partial maps,
$s : \QProc \rightarrow \QProc$. A map, $s$ lifts, uniquely, to a map
on process terms, $\widehat{s} : \Proc \rightarrow \Proc$ by the
following equations.

\begin{mathpar}
  (0) \psubstp{Q}{P} := 0 \\
  (R \juxtap S) \psubstp{Q}{P}
  :=    
  (R)\psubstp{Q}{P} \juxtap (S) \psubstp{Q}{P} \\
  (x?(y).R) \psubstp{Q}{P}    
  :=    
  (x)\substp{Q}{P} (z)\concat( (R \psubstn{z}{y}) \psubstp{Q}{P} ) \\
  (\lift{x}{R}) \psubstp{Q}{P}  
  :=
  \lift{(x)\substp{Q}{P}}{ R \psubstp{Q}{P} } \\
%   (\dropn{x})  \psubstp{Q}{P}       
%   := 
%   \left\{ 
%     \begin{array}{ccc} 
%       \dropn{\quotep{Q}} & & x \nameeq \quotep{P} \\
%       \dropn{x} & & otherwise \\
%     \end{array}
%   \right. 
  (\dropn{x})  \psubstp{Q}{P}       
  := 
  \left\{ 
    \begin{array}{ccc} 
      Q & & x \nameeq \quotep{P} \\
      \dropn{x} & & otherwise \\
    \end{array}
  \right.
\end{mathpar}
 

where

\begin{eqnarray}
  (x)\id{\{} \lpquote Q \rpquote / \lpquote P \rpquote \id{\}}            = 
  \left\{ 
    \begin{array}{ccc}
      \lpquote Q \rpquote & & x \nameeq \lpquote P \rpquote \\
      x & & otherwise \\
    \end{array}
  \right. \nonumber
\end{eqnarray}

and $z$ is chosen distinct from $\quotep{P}$, $\quotep{Q}$, the free
names in $Q$, and all the names in $R$. Our $\alpha$-equivalence will
be built in the standard way from this substitution.

\begin{remark}\label{rem:no_self_referential_names}
  One consequence of these definitions is that $\forall P. \quotep{P}
  \not\in \freenames{P}$.
\end{remark}

\subsection{ Dynamic quote: an example }

Anticipating something of what's to come, consider applying the
substitution, $\widehat{\id{\{}u / z \id{\}}}$, to the following pair
of processes, $\lift{w}{y!(z)}$ and $w[ \lpquote y!(z) \rpquote ]$.

\begin{eqnarray}
	\lift{w}{y!(z)}\widehat{\id{\{}u / z \id{\}}}
		& = &
		\lift{w}{y!(u)} \nonumber\\
	w[ \lpquote y!(z) \rpquote ] \widehat{ \id{\{}u / z \id{\}} }
		& = &
		w[ \lpquote y!(z) \rpquote ] \nonumber
\end{eqnarray}

Because the body of the process between quotes is impervious to
substitution, we get radically different answers. In fact, by
examining the first process in an input context,
e.g. $x?(z).\lift{w}{y!(z)}$, we see that the process under the lift
operator may be shaped by prefixed inputs binding a name inside it. In
this sense, the lift operator will be seen as a way to dynamically
construct processes before reifying them as names.

Finally equipped with these standard features we can present the
dynamics of the calculus.

\subsubsection{Operational semantics} 

Finally, we introduce the computational dynamics. What marks these
algebras as distinct from other more traditionally studied algebraic
structures, e.g. vector spaces or polynomial rings, is the manner in
which dynamics is captured. In traditional structures, dynamics is typically
expressed through morphisms between such structures, as in linear maps
between vector spaces or morphisms between rings. In algebras
associated with the semantics of computation, the dynamics is
expressed as part of the algebraic structure itself, through a
reduction reduction relation typically denoted by $\red$. Below, we
give a recursive presentation of this relation for the calculus used
in the encoding.

$\red \subseteq \pi \times \pi$
$\red : \pi \to \mathcal{P}(\pi)$

\begin{mathpar}
  \inferrule* [lab=Comm] { \textsf{match}( x_{src}, x_{trgt} ) } { x_{trgt}?(y)P \; | \; x_{src}!\langle {Q} \rangle \red P\{\quotep{Q}/y}\} }
  \and \\
  \inferrule* [lab=Par] {{P} \red {P}'} {{{P} | {Q}} \red {{P}' | {Q}}}
  \and
  \inferrule* [lab=Equiv]{{{P} \scong {P}'} \andalso {{P}' \red {Q}'} \andalso {{Q}' \scong {Q}}}{{P} \red {Q}}
\end{mathpar}

\begin{eqnarray*}
  match_{\equiv} (\quotep{P},\quotep{Q}) & := & P \equiv Q \\
  match_{\dagger}(\quotep{P},\quotep{Q}) & := & \forall R. P|Q \red^{*} R => R \red^{*} 0 \\
  match_{K}(\quotep{P},\quotep{Q}) & := & K \mbox{ for some context } K
\end{eqnarray*}

$u?(x)P | u!\langle Q \rangle \red P\{\quotep{Q}/x\}$

%We write $\wred$ for $\red^*$, and $P\red$ if $\exists Q $ such that $ P \red Q$.
We write $P\red$ if $\exists Q $ such that $ P \red Q$ and $P\not\red$, otherwise.

\section{Replication}

As mentioned before, it is known that replication (and hence
recursion) can be implemented in a higher-order process algebra
\cite{SangiorgiWalker}. As our first example of calculation with the
machinery thus far presented we give the construction explicitly in
the {\rhoc}.

\begin{eqnarray}
	D_{x} & := & \prefix{x}{y}{(\binpar{\outputp{x}{y}}{@{y}})} \nonumber\\
	\bangp_{x}{P} & := & \binpar{{x}!\langle{\binpar{D_{x}}{P}}\rangle}{D_{x}} \nonumber
\end{eqnarray}

\begin{eqnarray}
	\bangp_{x}{P} & & \nonumber\\
	=
	& {x}!\langle{(\prefix{x}{y}{(\outputp{x}{y} | @{y})) | P}}\rangle 
	      | \prefix{x}{y}{(\outputp{x}{y} | @{y})} & \nonumber\\
	\red
	& (\outputp{x}{y} | @{y})\substn{\quotep{(\prefix{x}{y}{(@{y} | \outputp{x}{y})) | P}}}{y} & \nonumber\\
	=
	& \outputp{x}{\quotep{(\prefix{x}{y}{(\outputp{x}{y} | @{y})) | P}}}
	  | {(\prefix{x}{y}{(\outputp{x}{y} | @{y})) | P}} & \nonumber\\
	\red
	& \ldots & \nonumber\\
	\red^*
	& P | P | \ldots & \nonumber
\end{eqnarray}

Of course, this encoding, as an implementation, runs away, unfolding
$\bangp{P}$ eagerly. A lazier and more implementable replication
operator, restricted to input-guarded processes, may be obtained as follows.

\begin{eqnarray}
\bangp{\prefix{u}{v}{P}} 
	:= 
	\binpar{\lift{x}{\prefix{u}{v}{(\binpar{D(x)}{P})}}}{D(x)} \nonumber
\end{eqnarray}

\begin{remark}
  Note that the lazier definition still does not deal with summation
  or mixed summation (i.e. sums over input and output). The reader is
  invited to construct definitions of replication that deal with these
  features. 

  Further, the definitions are parameterized in a name, $x$. Can you,
  gentle reader, make a definition that eliminates this parameter and
  guarantees no accidental interaction between the replication
  machinery and the process being replicated -- i.e. no accidental
  sharing of names used by the process to get its work done and the
  name(s) used by the replication to effect copying. This latter
  revision of the definition of replication is crucial to obtaining
  the expected identity $!!P \sim !P$.
\end{remark}

\begin{remark}\label{rem:paradoxical_combinator}
  The reader familiar with the lambda calculus will have noticed the
  similarity between $D$ and the paradoxical combinator.

  [Ed. note: the existence of this seems to suggest we have to be more
  restrictive on the set of processes and names we admit if we are to
  support no-cloning.]
\end{remark}

\subsubsection{Bisimulation}

The computational dynamics gives rise to another kind of equivalence,
the equivalence of computational behavior. As previously mentioned
this is typically captured \emph{via} some form of bisimulation.

% The notion we use in this paper is weak barbed bisimulation
% \cite{milner91polyadicpi}.

The notion we use in this paper is derived from weak barbed
bisimulation \cite{milner91polyadicpi}. 

\begin{definition}
An \emph{observation relation}, $\downarrow_{\mathcal N}$, over a set
of names, $\mathcal N$, is the smallest relation satisfying the rules
below.

\infrule[Out-barb]{y \in {\mathcal N}, \; x \nameeq y}
		  {\outputp{x}{v} \downarrow_{\mathcal N} x}
\infrule[Par-barb]{\mbox{$P\downarrow_{\mathcal N} x$ or $Q\downarrow_{\mathcal N} x$}}
		  {\binpar{P}{Q} \downarrow_{\mathcal N} x}

We write $P \Downarrow_{\mathcal N} x$ if there is $Q$ such that 
$P \wred Q$ and $Q \downarrow_{\mathcal N} x$.
\end{definition}

\begin{definition}
%\label{def.bbisim}
An  ${\mathcal N}$-\emph{barbed bisimulation} over a set of names, ${\mathcal N}$, is a symmetric binary relation 
${\mathcal S}_{\mathcal N}$ between agents such that $P\rel{S}_{\mathcal N}Q$ implies:
\begin{enumerate}
\item If $P \red P'$ then $Q \wred Q'$ and $P'\rel{S}_{\mathcal N} Q'$.
\item If $P\downarrow_{\mathcal N} x$, then $Q\Downarrow_{\mathcal N} x$.
\end{enumerate}
$P$ is ${\mathcal N}$-barbed bisimilar to $Q$, written
$P \wbbisim_{\mathcal N} Q$, if $P \rel{S}_{\mathcal N} Q$ for some ${\mathcal N}$-barbed bisimulation ${\mathcal S}_{\mathcal N}$.
\end{definition}

$\mathcal{R} \subseteq \pi \times \pi$

$P \mathcal{R} Q => \forall P'. P \red P' \Rightarrow \exists Q'. Q \red Q', P' \mathcal{R} Q'$

$P \vdash x \Rightarrow Q \vdash x$

\begin{mathpar}
  \inferrule*[lab=Out-barb]{x \nameeq y}{{y}!\langle{Q}\rangle \vdash x}
  \and
  \inferrule*[lab=Par-barb]{\mbox{$P\vdash x$ or $Q\vdash x$}}{\binpar{P}{Q} \vdash x}
\end{mathpar}

\subsubsection{Contexts}

One of the principle advantages of computational calculi like the
$\pi$-calculus is a well-defined notion of context,
contextual-equivalence and a correlation between
contextual-equivalence and notions of bisimulation. The notion of
context allows the decomposition of a process into (sub-)process and
its syntactic environment, its context. Thus, a context may be
thought of as a process with a ``hole'' (written $\Box$) in it. The
application of a context $M$ to a process $P$, written $M[P]$, is
tantamount to filling the hole in $M$ with $P$. In this paper we do
not need the full weight of this theory, but do make use of the notion
of context in the proof the main theorem. 

\begin{mathpar}
  \inferrule* [lab=summation] {} {{M_{M},M_{N}} \bc \Box \;|\; x.M_{A} \;|\; M_{M}+M_{N}}
  \and
  \inferrule* [lab=agent] {} {{M_{A}} \bc (\vec{x})M_{P} \;| \; \clift{P_0,\ldots,M_{P},\ldots,P_N}}
  \and \\
  \inferrule* [lab=process] {} {{M_{P}} \bc M_{N} \;| \;P|M_{P} }
\end{mathpar} 

\begin{mathpar}
  \inferrule* [lab=sychronization] {} {M_{N} \bc \Box \;|\; x?M_{F} \;|\; x!M_{C}}
  \and
  \inferrule* [lab=abstraction] {} {{M_{F}} \bc (x)M_{P} }
  \and
  \inferrule* [lab=concretion] {} {{M_{C}} \bc \langle M_{P} \rangle }
  \and \\
  \inferrule* [lab=process] {} {{M_{P}} \bc M_{N} \;| \;P|M_{P} }
\end{mathpar}

\begin{definition}[contextual application] Given a context $M$, and
  process $P$, we define the \emph{contextual application}, $M[P] :=
  M\{P/\Box\}$. That is, the contextual application of M to P is the
  substitution of $P$ for $\Box$ in $M$.
\end{definition}

$\meaningof{-} : L \to \mathcal{P}(\pi)$

\begin{mathpar}
  \inferrule* [lab=collection] {} {\meaningof{true} = \pi, \and \meaningof{~E} = \pi \setminus \meaningof{E}, \and \meaningof{E_{1} \& E_{2}} = \meaningof{E_{1}} \cap \meaningof{E_{2}}}
\end{mathpar}

\begin{mathpar}
  \inferrule* [lab=structure] {} {\meaningof{0} = \{ P \in \pi | P \equiv 0 \}, \and \\ \meaningof{E_1 | E_2} = \{ P \in \pi | P \equiv P_{1} | P_{2}, P_{1} \in \meaningof{E_{1}}, P_{2} \in \meaningof{E_2}\} }
\end{mathpar}

\begin{mathpar}
 \inferrule* [lab=behavior] {} {\meaningof{\langle a?b \rangle E} = \{ P \in \pi | P \equiv Q | u?(y)P', \\ \and \\\\ \and \\ \;\;\; u \in \meaningof{a}, \forall z.P'\{z/y\} \in \meaningof{E\{z/b\}}\}, \and \\ \meaningof{a!E} = \{ P \in \pi | P \equiv Q | x!\langle P' \rangle, x \in \meaningof{a} P' \in \meaningof{E}\} }
\end{mathpar}

\begin{mathpar}
 \inferrule* [lab=nominal] {} {\meaningof{\quotep{E}} = \{ \quotep{P} \in \quotep{\pi} | P \in \meaningof{E} \}, \and \meaningof{\quotep{P}} = \{ \quotep{Q} \in \quotep{\pi} | P \equiv Q \} \and \\ \meaningof{@\quotep{E}} = \{ P \in \pi | P \equiv @x, x \in \meaningof{E} \}}
\end{mathpar}

\begin{eqnarray*}
  \\
  \meaningof{-} : TS \to ST
\end{eqnarray*}

\begin{eqnarray*}
  \\
  L : TS \to ST
\end{eqnarray*}

\begin{eqnarray*}
  \\
  P \models E \iff P \in \meaningof{E}
\end{eqnarray*}

\begin{eqnarray*}
  P \approx_{L} Q \iff \forall E \in L. P \models E \iff Q \models E
\end{eqnarray*}

\begin{eqnarray*}
  P \approx_{K} Q
\end{eqnarray*}

\begin{eqnarray*}
  P \approx Q
\end{eqnarray*}

$\approx_{K} = \approx = \approx_{L}$

\subsubsection{Contextual duality}

Note that contexts extend the quotation operation to a family of
operations from processes to names. Given a context, $M$, we can
define a \emph{nominal context}, $\quotep{M}$ by $\quotep{M}[P] :=
\quotep{M[P]}$. To foreshadow what is to come we observe that these
operations enjoy a duality with processes very much like the duality
between vectors and maps from vectors to scalars.

Further, because the calculus is essentially higher-order, we have a
correspondence between contexts and processes. More specifically,
given a name $x$ and a context $M$ we can construct $M^{*}_{x}$ such
that 

\begin{mathpar}
  M^{*}_{x} | \lift{x}{P} \red M[P]
\end{mathpar}

namely,

\begin{mathpar}
  M^{*}_{x} := x?(u).M[\dropn{u}]
\end{mathpar}

The dependence of $M^{*}_{x}$ on a name makes it an abstraction, 

\begin{mathpar}
  M^{*} := (x)x?(u).M[\dropn{u}]
\end{mathpar}

\subsection{Additional notation}

It will sometimes be convenient to denote the process a name
quotes. We already have the notation $x = \quotep{P}$, but it will be
convenient to introduce an alternate notation, $\procn{x}$, when we
want to emphasize the connection to the use of the name. Note that, by
virtue of name equivalence, $\quotep{\procn{x}} \nameeq x$; so, the
notation is consistent with previous definitions.

Further, because names have structure it is possible to effect
substitutions on the basis of that structure. This means we need to
upgrade our notation for substitutions, which we accomplish by
adapting comprehension notation. Thus,

\begin{mathpar}
  P\{ y / x : x \in S \}
\end{mathpar}

is interpreted to mean the process derived from P by replacing (in a
capture-avoiding manner) each occurrence of $x$ in $S$ by $y$. For example,

\begin{mathpar}
  P\{ \quotep{\procn{x}|\procn{x}} / x : x \in \freenames{P} \}
\end{mathpar}

will replace each (occurrence) of a free name $x$ in $P$ by
$\quotep{\procn{x}|\procn{x}}$.

Also, we will avail ourselves of the notation $x^{L}$ and $x^{R}$ to
denote injections of a name into disjoint copies of the name
space. There are numerous ways to accomplish this. One example can be
found in \cite{MeredithR05}. This notation overloads to vectors of
names: $\vec{x}^{\pi} := (x_{i}^{\pi} \; : \; 0 \leq i < |\vec{x}| )$ where $\pi \in \{L,R\}$.

We also use $P^{\Box} := P|\Box$.

In \cite{MeredithR05} an interpretation of the new operator is
given. It turns out that there are several possible interpretations
all enjoying the requisite algebraic properties of the operator (see
\cite{milner91polyadicpi}). We will therefore make liberal use of
$(\nu\; \vec{x})P$.

% subsection the_syntax_and_semantics_of_the_notation_system (end)   

\input{qm2pi.qmops} 

\input{qm2pi.sterngerlach} 

\input{qm2pi.metric} 

% section concurrent_process_calculi (end)

%\input{qm2pi.proofsketch}

% section proof sketch (end)

%\input{qm2pi.slviaknots} 

% section spatial logic via knots (end)

\input{qm2pi.conclusion}

% section conclusion (end)

%\input{qm2pi.dtcodes} 

% section wiring algorithm (end)

\input{qm2pi.ack} 

% section acknowledgments (end)

\newpage


\bibliographystyle{plain}   
\bibliography{../../biblios/main.bib}

\input{qm2pi.rhodetails}

\end{document}

 

% subsection basic_interpretation (end)

%\input{qm2pi.rho.presentation} 
\subsection{The syntax and semantics of the notation system}\label{sub:the_syntax_and_semantics_of_the_notation_system} % (fold)

We now summarize a technical presentation of the calculus that
embodies our theory of dynamics. The typical presentation of such a
calculus follows the style of giving generators and relations on
them. The grammar, below, describing term constructors, freely
generates the set of processes, $\Proc$. This set is then quotiented
by a relation known as structural congruence and it is over this set
that the notion of dynamics is expressed. This presentation is
essentially that of \cite{MeredithR05} with the addition of
polyadicity and summation. For readability we have relegated some of
the technical subtleties to an appendix.

\subsubsection{Process grammar}\label{subsub:process_grammar}

\begin{mathpar}
  \inferrule* [lab=synchronization] {} {{M} \bc \pzero \;|\; x?F \;|\; x!C }
  \and
  \inferrule* [lab=abstraction] {} {{F} \bc (x)P}
  \and
  \inferrule* [lab=concretion] {} {{C} \bc \langle Q \rangle}
  \and
  \inferrule* [lab=process] {} {{P,Q} \bc M \;| \;P|Q \;|\; @{x}}
  \and
  \inferrule* [lab=name] {} {{x} \bc \quotep{P}}
\end{mathpar} 

Note that $\vec{x}$ (resp. $\vec{P}$) denotes a vector of names
(resp. processes) of length $|\vec{x}|$ (resp. $|\vec{P}|$). We adopt
the following useful abbreviations.

\begin{mathpar}
   x?(\vec{y}).P := x.(\vec{y})P \and  x\clift{\vec{P}} := x.\clift{\vec{P}}
   \and x!(y) := \lift{x}{\dropn{y}}
   \and \Pi_{i=0}^{n-1}P_i := P_0 | \ldots | P_{n-1}
\end{mathpar}

\subsubsection{Structural congruence}

\paragraph{Free and bound names and alpha-equivalence.} At the
core of structural equivalence is alpha-equivalence which identifies
process that are the same up to a change of variable. Formally, we
recognize the distinction between free and bound names. The free names
of a process, $\freenames{P}$, may be calculated recursively as
follows:

\begin{mathpar}
\freenames{\pzero} := \emptyset
  \and \\
  \freenames{x?(y).P} := \{ x \} \cup (\freenames{P} \setminus \{ y \})
  \and 
  \freenames{x!\langle P \rangle} := \{ x \} \cup \{ P \} 
  \and \\
  \freenames{P|Q} := \freenames{P} \cup \freenames{Q}
  \and \\
  \freenames{@{x}} := \{ x \}
\end{mathpar}

$\pi$
$\quotep{\pi}$

$\freenames{-} : \pi \to \mathcal{P}(\quotep{\pi})$

\begin{eqnarray*}
  \freenames{\pzero} & := & \emptyset \\
  \freenames{x?(y).P} & := & \{ x \} \cup (\freenames{P} \setminus \{ y \}) \\
  \freenames{x!\langle P \rangle} & := & \{ x \} \cup \{ P \} \\
  \freenames{P|Q} & := & \freenames{P} \cup \freenames{Q} \\
  \freenames{\dropn{x}} & := & \{ x \}
\end{eqnarray*}

The bound names of a process, $\boundnames{P}$, are those names occurring in $P$
that are not free. For example, in $x?(y).0$, the name $x$ is free, while $y$ is bound.

\begin{mathpar}
  \inferrule* [lab=monoidal-laws] {} { P|Q \equiv Q|P \and P|0 \equiv P \and P|(Q|R) \equiv (P|Q)|R }
\end{mathpar}

\begin{mathpar}
  \inferrule* [lab=alpha-equivalence] {} { (x)P \equiv (y)P\{y/x\} \and y \not\in \freenames{P} }
\end{mathpar}

\begin{definition}
Then two processes, $P,Q$, are alpha-equivalent if $P = Q\{\vec{y}/\vec{x}\}$ for
some $\vec{x} \in \boundnames{Q},\vec{y} \in \boundnames{P}$, where $Q\{\vec{y}/\vec{x}\}$
denotes the capture-avoiding substitution of $\vec{y}$ for $\vec{x}$ in $Q$.
\end{definition}

\begin{definition}
  The {\em structural congruence} \cite{SangiorgiWalker} , $\equiv$,
  between processes is the least congruence containing
  alpha-equivalence, satisfying the abelian monoid laws
  (associativity, commutativity and $\pzero$ as identity) for parallel
  composition $|$ and for summation $+$.
\end{definition}

\subsection{Name equivalence}

We take name equivalence, written $\nameeq$, to be the smallest
equivalence relation generated by the following rules.

\begin{mathpar}
\inferrule*[lab=Quote-drop]
{ }
{ \quotep{@{x}} \nameeq x }

\inferrule*[lab=Struct-equiv]
{ P \scong Q }
{ \quotep{P} \nameeq \quotep{Q} }
\end{mathpar}

The astute reader will have noticed that the mutual recursion of names
and processes imposes a mutual recursion on alpha-equivalence and
structural equivalence via name-equivalence. Fortunately, all of this
works out pleasantly and we may calculate in the natural way, free of
concern. The reader interested in the details is referred to the
appendix \ref{appendix:rho_details}.

\subsection{Substitution}

We use $\Proc$ for the set of processes, $\QProc$ for the set of
names, and $\id{\{}\vec{y} / \vec{x} \id{\}}$ to denote partial maps,
$s : \QProc \rightarrow \QProc$. A map, $s$ lifts, uniquely, to a map
on process terms, $\widehat{s} : \Proc \rightarrow \Proc$ by the
following equations.

\begin{mathpar}
  (0) \psubstp{Q}{P} := 0 \\
  (R \juxtap S) \psubstp{Q}{P}
  :=    
  (R)\psubstp{Q}{P} \juxtap (S) \psubstp{Q}{P} \\
  (x?(y).R) \psubstp{Q}{P}    
  :=    
  (x)\substp{Q}{P} (z)\concat( (R \psubstn{z}{y}) \psubstp{Q}{P} ) \\
  (\lift{x}{R}) \psubstp{Q}{P}  
  :=
  \lift{(x)\substp{Q}{P}}{ R \psubstp{Q}{P} } \\
%   (\dropn{x})  \psubstp{Q}{P}       
%   := 
%   \left\{ 
%     \begin{array}{ccc} 
%       \dropn{\quotep{Q}} & & x \nameeq \quotep{P} \\
%       \dropn{x} & & otherwise \\
%     \end{array}
%   \right. 
  (\dropn{x})  \psubstp{Q}{P}       
  := 
  \left\{ 
    \begin{array}{ccc} 
      Q & & x \nameeq \quotep{P} \\
      \dropn{x} & & otherwise \\
    \end{array}
  \right.
\end{mathpar}
 

where

\begin{eqnarray}
  (x)\id{\{} \lpquote Q \rpquote / \lpquote P \rpquote \id{\}}            = 
  \left\{ 
    \begin{array}{ccc}
      \lpquote Q \rpquote & & x \nameeq \lpquote P \rpquote \\
      x & & otherwise \\
    \end{array}
  \right. \nonumber
\end{eqnarray}

and $z$ is chosen distinct from $\quotep{P}$, $\quotep{Q}$, the free
names in $Q$, and all the names in $R$. Our $\alpha$-equivalence will
be built in the standard way from this substitution.

\begin{remark}\label{rem:no_self_referential_names}
  One consequence of these definitions is that $\forall P. \quotep{P}
  \not\in \freenames{P}$.
\end{remark}

\subsection{ Dynamic quote: an example }

Anticipating something of what's to come, consider applying the
substitution, $\widehat{\id{\{}u / z \id{\}}}$, to the following pair
of processes, $\lift{w}{y!(z)}$ and $w[ \lpquote y!(z) \rpquote ]$.

\begin{eqnarray}
	\lift{w}{y!(z)}\widehat{\id{\{}u / z \id{\}}}
		& = &
		\lift{w}{y!(u)} \nonumber\\
	w[ \lpquote y!(z) \rpquote ] \widehat{ \id{\{}u / z \id{\}} }
		& = &
		w[ \lpquote y!(z) \rpquote ] \nonumber
\end{eqnarray}

Because the body of the process between quotes is impervious to
substitution, we get radically different answers. In fact, by
examining the first process in an input context,
e.g. $x?(z).\lift{w}{y!(z)}$, we see that the process under the lift
operator may be shaped by prefixed inputs binding a name inside it. In
this sense, the lift operator will be seen as a way to dynamically
construct processes before reifying them as names.

Finally equipped with these standard features we can present the
dynamics of the calculus.

\subsubsection{Operational semantics} 

Finally, we introduce the computational dynamics. What marks these
algebras as distinct from other more traditionally studied algebraic
structures, e.g. vector spaces or polynomial rings, is the manner in
which dynamics is captured. In traditional structures, dynamics is typically
expressed through morphisms between such structures, as in linear maps
between vector spaces or morphisms between rings. In algebras
associated with the semantics of computation, the dynamics is
expressed as part of the algebraic structure itself, through a
reduction reduction relation typically denoted by $\red$. Below, we
give a recursive presentation of this relation for the calculus used
in the encoding.

$\red \subseteq \pi \times \pi$
$\red : \pi \to \mathcal{P}(\pi)$

\begin{mathpar}
  \inferrule* [lab=Comm] { \textsf{match}( x_{src}, x_{trgt} ) } { x_{trgt}?(y)P \; | \; x_{src}!\langle {Q} \rangle \red P\{\quotep{Q}/y}\} }
  \and \\
  \inferrule* [lab=Par] {{P} \red {P}'} {{{P} | {Q}} \red {{P}' | {Q}}}
  \and
  \inferrule* [lab=Equiv]{{{P} \scong {P}'} \andalso {{P}' \red {Q}'} \andalso {{Q}' \scong {Q}}}{{P} \red {Q}}
\end{mathpar}

\begin{eqnarray*}
  match_{\equiv} (\quotep{P},\quotep{Q}) & := & P \equiv Q \\
  match_{\dagger}(\quotep{P},\quotep{Q}) & := & \forall R. P|Q \red^{*} R => R \red^{*} 0 \\
  match_{K}(\quotep{P},\quotep{Q}) & := & K \mbox{ for some context } K
\end{eqnarray*}

$u?(x)P | u!\langle Q \rangle \red P\{\quotep{Q}/x\}$

%We write $\wred$ for $\red^*$, and $P\red$ if $\exists Q $ such that $ P \red Q$.
We write $P\red$ if $\exists Q $ such that $ P \red Q$ and $P\not\red$, otherwise.

\section{Replication}

As mentioned before, it is known that replication (and hence
recursion) can be implemented in a higher-order process algebra
\cite{SangiorgiWalker}. As our first example of calculation with the
machinery thus far presented we give the construction explicitly in
the {\rhoc}.

\begin{eqnarray}
	D_{x} & := & \prefix{x}{y}{(\binpar{\outputp{x}{y}}{@{y}})} \nonumber\\
	\bangp_{x}{P} & := & \binpar{{x}!\langle{\binpar{D_{x}}{P}}\rangle}{D_{x}} \nonumber
\end{eqnarray}

\begin{eqnarray}
	\bangp_{x}{P} & & \nonumber\\
	=
	& {x}!\langle{(\prefix{x}{y}{(\outputp{x}{y} | @{y})) | P}}\rangle 
	      | \prefix{x}{y}{(\outputp{x}{y} | @{y})} & \nonumber\\
	\red
	& (\outputp{x}{y} | @{y})\substn{\quotep{(\prefix{x}{y}{(@{y} | \outputp{x}{y})) | P}}}{y} & \nonumber\\
	=
	& \outputp{x}{\quotep{(\prefix{x}{y}{(\outputp{x}{y} | @{y})) | P}}}
	  | {(\prefix{x}{y}{(\outputp{x}{y} | @{y})) | P}} & \nonumber\\
	\red
	& \ldots & \nonumber\\
	\red^*
	& P | P | \ldots & \nonumber
\end{eqnarray}

Of course, this encoding, as an implementation, runs away, unfolding
$\bangp{P}$ eagerly. A lazier and more implementable replication
operator, restricted to input-guarded processes, may be obtained as follows.

\begin{eqnarray}
\bangp{\prefix{u}{v}{P}} 
	:= 
	\binpar{\lift{x}{\prefix{u}{v}{(\binpar{D(x)}{P})}}}{D(x)} \nonumber
\end{eqnarray}

\begin{remark}
  Note that the lazier definition still does not deal with summation
  or mixed summation (i.e. sums over input and output). The reader is
  invited to construct definitions of replication that deal with these
  features. 

  Further, the definitions are parameterized in a name, $x$. Can you,
  gentle reader, make a definition that eliminates this parameter and
  guarantees no accidental interaction between the replication
  machinery and the process being replicated -- i.e. no accidental
  sharing of names used by the process to get its work done and the
  name(s) used by the replication to effect copying. This latter
  revision of the definition of replication is crucial to obtaining
  the expected identity $!!P \sim !P$.
\end{remark}

\begin{remark}\label{rem:paradoxical_combinator}
  The reader familiar with the lambda calculus will have noticed the
  similarity between $D$ and the paradoxical combinator.

  [Ed. note: the existence of this seems to suggest we have to be more
  restrictive on the set of processes and names we admit if we are to
  support no-cloning.]
\end{remark}

\subsubsection{Bisimulation}

The computational dynamics gives rise to another kind of equivalence,
the equivalence of computational behavior. As previously mentioned
this is typically captured \emph{via} some form of bisimulation.

% The notion we use in this paper is weak barbed bisimulation
% \cite{milner91polyadicpi}.

The notion we use in this paper is derived from weak barbed
bisimulation \cite{milner91polyadicpi}. 

\begin{definition}
An \emph{observation relation}, $\downarrow_{\mathcal N}$, over a set
of names, $\mathcal N$, is the smallest relation satisfying the rules
below.

\infrule[Out-barb]{y \in {\mathcal N}, \; x \nameeq y}
		  {\outputp{x}{v} \downarrow_{\mathcal N} x}
\infrule[Par-barb]{\mbox{$P\downarrow_{\mathcal N} x$ or $Q\downarrow_{\mathcal N} x$}}
		  {\binpar{P}{Q} \downarrow_{\mathcal N} x}

We write $P \Downarrow_{\mathcal N} x$ if there is $Q$ such that 
$P \wred Q$ and $Q \downarrow_{\mathcal N} x$.
\end{definition}

\begin{definition}
%\label{def.bbisim}
An  ${\mathcal N}$-\emph{barbed bisimulation} over a set of names, ${\mathcal N}$, is a symmetric binary relation 
${\mathcal S}_{\mathcal N}$ between agents such that $P\rel{S}_{\mathcal N}Q$ implies:
\begin{enumerate}
\item If $P \red P'$ then $Q \wred Q'$ and $P'\rel{S}_{\mathcal N} Q'$.
\item If $P\downarrow_{\mathcal N} x$, then $Q\Downarrow_{\mathcal N} x$.
\end{enumerate}
$P$ is ${\mathcal N}$-barbed bisimilar to $Q$, written
$P \wbbisim_{\mathcal N} Q$, if $P \rel{S}_{\mathcal N} Q$ for some ${\mathcal N}$-barbed bisimulation ${\mathcal S}_{\mathcal N}$.
\end{definition}

$\mathcal{R} \subseteq \pi \times \pi$

$P \mathcal{R} Q => \forall P'. P \red P' \Rightarrow \exists Q'. Q \red Q', P' \mathcal{R} Q'$

$P \vdash x \Rightarrow Q \vdash x$

\begin{mathpar}
  \inferrule*[lab=Out-barb]{x \nameeq y}{{y}!\langle{Q}\rangle \vdash x}
  \and
  \inferrule*[lab=Par-barb]{\mbox{$P\vdash x$ or $Q\vdash x$}}{\binpar{P}{Q} \vdash x}
\end{mathpar}

\subsubsection{Contexts}

One of the principle advantages of computational calculi like the
$\pi$-calculus is a well-defined notion of context,
contextual-equivalence and a correlation between
contextual-equivalence and notions of bisimulation. The notion of
context allows the decomposition of a process into (sub-)process and
its syntactic environment, its context. Thus, a context may be
thought of as a process with a ``hole'' (written $\Box$) in it. The
application of a context $M$ to a process $P$, written $M[P]$, is
tantamount to filling the hole in $M$ with $P$. In this paper we do
not need the full weight of this theory, but do make use of the notion
of context in the proof the main theorem. 

\begin{mathpar}
  \inferrule* [lab=summation] {} {{M_{M},M_{N}} \bc \Box \;|\; x.M_{A} \;|\; M_{M}+M_{N}}
  \and
  \inferrule* [lab=agent] {} {{M_{A}} \bc (\vec{x})M_{P} \;| \; \clift{P_0,\ldots,M_{P},\ldots,P_N}}
  \and \\
  \inferrule* [lab=process] {} {{M_{P}} \bc M_{N} \;| \;P|M_{P} }
\end{mathpar} 

\begin{mathpar}
  \inferrule* [lab=sychronization] {} {M_{N} \bc \Box \;|\; x?M_{F} \;|\; x!M_{C}}
  \and
  \inferrule* [lab=abstraction] {} {{M_{F}} \bc (x)M_{P} }
  \and
  \inferrule* [lab=concretion] {} {{M_{C}} \bc \langle M_{P} \rangle }
  \and \\
  \inferrule* [lab=process] {} {{M_{P}} \bc M_{N} \;| \;P|M_{P} }
\end{mathpar}

\begin{definition}[contextual application] Given a context $M$, and
  process $P$, we define the \emph{contextual application}, $M[P] :=
  M\{P/\Box\}$. That is, the contextual application of M to P is the
  substitution of $P$ for $\Box$ in $M$.
\end{definition}

$\meaningof{-} : L \to \mathcal{P}(\pi)$

\begin{mathpar}
  \inferrule* [lab=collection] {} {\meaningof{true} = \pi, \and \meaningof{~E} = \pi \setminus \meaningof{E}, \and \meaningof{E_{1} \& E_{2}} = \meaningof{E_{1}} \cap \meaningof{E_{2}}}
\end{mathpar}

\begin{mathpar}
  \inferrule* [lab=structure] {} {\meaningof{0} = \{ P \in \pi | P \equiv 0 \}, \and \\ \meaningof{E_1 | E_2} = \{ P \in \pi | P \equiv P_{1} | P_{2}, P_{1} \in \meaningof{E_{1}}, P_{2} \in \meaningof{E_2}\} }
\end{mathpar}

\begin{mathpar}
 \inferrule* [lab=behavior] {} {\meaningof{\langle a?b \rangle E} = \{ P \in \pi | P \equiv Q | u?(y)P', \\ \and \\\\ \and \\ \;\;\; u \in \meaningof{a}, \forall z.P'\{z/y\} \in \meaningof{E\{z/b\}}\}, \and \\ \meaningof{a!E} = \{ P \in \pi | P \equiv Q | x!\langle P' \rangle, x \in \meaningof{a} P' \in \meaningof{E}\} }
\end{mathpar}

\begin{mathpar}
 \inferrule* [lab=nominal] {} {\meaningof{\quotep{E}} = \{ \quotep{P} \in \quotep{\pi} | P \in \meaningof{E} \}, \and \meaningof{\quotep{P}} = \{ \quotep{Q} \in \quotep{\pi} | P \equiv Q \} \and \\ \meaningof{@\quotep{E}} = \{ P \in \pi | P \equiv @x, x \in \meaningof{E} \}}
\end{mathpar}

\begin{eqnarray*}
  \\
  \meaningof{-} : TS \to ST
\end{eqnarray*}

\begin{eqnarray*}
  \\
  L : TS \to ST
\end{eqnarray*}

\begin{eqnarray*}
  \\
  P \models E \iff P \in \meaningof{E}
\end{eqnarray*}

\begin{eqnarray*}
  P \approx_{L} Q \iff \forall E \in L. P \models E \iff Q \models E
\end{eqnarray*}

\begin{eqnarray*}
  P \approx_{K} Q
\end{eqnarray*}

\begin{eqnarray*}
  P \approx Q
\end{eqnarray*}

$\approx_{K} = \approx = \approx_{L}$

\subsubsection{Contextual duality}

Note that contexts extend the quotation operation to a family of
operations from processes to names. Given a context, $M$, we can
define a \emph{nominal context}, $\quotep{M}$ by $\quotep{M}[P] :=
\quotep{M[P]}$. To foreshadow what is to come we observe that these
operations enjoy a duality with processes very much like the duality
between vectors and maps from vectors to scalars.

Further, because the calculus is essentially higher-order, we have a
correspondence between contexts and processes. More specifically,
given a name $x$ and a context $M$ we can construct $M^{*}_{x}$ such
that 

\begin{mathpar}
  M^{*}_{x} | \lift{x}{P} \red M[P]
\end{mathpar}

namely,

\begin{mathpar}
  M^{*}_{x} := x?(u).M[\dropn{u}]
\end{mathpar}

The dependence of $M^{*}_{x}$ on a name makes it an abstraction, 

\begin{mathpar}
  M^{*} := (x)x?(u).M[\dropn{u}]
\end{mathpar}

\subsection{Additional notation}

It will sometimes be convenient to denote the process a name
quotes. We already have the notation $x = \quotep{P}$, but it will be
convenient to introduce an alternate notation, $\procn{x}$, when we
want to emphasize the connection to the use of the name. Note that, by
virtue of name equivalence, $\quotep{\procn{x}} \nameeq x$; so, the
notation is consistent with previous definitions.

Further, because names have structure it is possible to effect
substitutions on the basis of that structure. This means we need to
upgrade our notation for substitutions, which we accomplish by
adapting comprehension notation. Thus,

\begin{mathpar}
  P\{ y / x : x \in S \}
\end{mathpar}

is interpreted to mean the process derived from P by replacing (in a
capture-avoiding manner) each occurrence of $x$ in $S$ by $y$. For example,

\begin{mathpar}
  P\{ \quotep{\procn{x}|\procn{x}} / x : x \in \freenames{P} \}
\end{mathpar}

will replace each (occurrence) of a free name $x$ in $P$ by
$\quotep{\procn{x}|\procn{x}}$.

Also, we will avail ourselves of the notation $x^{L}$ and $x^{R}$ to
denote injections of a name into disjoint copies of the name
space. There are numerous ways to accomplish this. One example can be
found in \cite{MeredithR05}. This notation overloads to vectors of
names: $\vec{x}^{\pi} := (x_{i}^{\pi} \; : \; 0 \leq i < |\vec{x}| )$ where $\pi \in \{L,R\}$.

We also use $P^{\Box} := P|\Box$.

In \cite{MeredithR05} an interpretation of the new operator is
given. It turns out that there are several possible interpretations
all enjoying the requisite algebraic properties of the operator (see
\cite{milner91polyadicpi}). We will therefore make liberal use of
$(\nu\; \vec{x})P$.

% subsection the_syntax_and_semantics_of_the_notation_system (end)   

\section{Interpretation of QM}
\subsection{Supporting definitions}
\subsubsection{Multiplication}
\begin{mathpar}
  \quotep{Q} \cdot \quotep{R} := \quotep{Q|R}
  \and \\
  \quotep{Q} \cdot P := P\{ \quotep{Q|R} / \quotep{R} : \quotep{R} \in \freenames{P} \}
\end{mathpar}

\paragraph{Discussion}
The first line needs little explanation. The second line says that
each free name of the process is replaced with the multiplication of
that name by the scalar. Multiplication of a scalar (name) by a state
(process) results in a process all the names of which have been `moved
over' by parallel composition with the process the scalar
quotes. There is a subtlety that the bound names have to be
manipulated so that multiplied names aren't accidentally
captured. There are many ways to achieve this.

\begin{remark}\label{rem:multiplication_identities}
  The reader is invited to verify that for all $x,y,z \in \QProc$ and $P \in \Proc$
  \begin{mathpar}
    x \cdot \quotep{0} \equiv x 
    \and
    x \cdot y \equiv y \cdot x
    \and
    x \cdot (y \cdot z) \equiv (x \cdot y) \cdot z
    \and \\
    \quotep{0} \cdot P \equiv P
    \and \\
    x \cdot (y \cdot P) \equiv (x \cdot y) \cdot P
    \and \\
    x \cdot (P|Q) \equiv (x \cdot P) | (x \cdot Q)
    \and \\    
  \end{mathpar}
\end{remark}

\subsubsection{Tensor product}

We define a tensor product on processes by structural induction.

\paragraph{Tensor of sums} First note that all summations, including
$\pzero$ and sequence, can be written $\Sigma_{i} x_{i}.A_{i} +
\Sigma_{j} x_{j}.C_{j}$, where we have grouped input-guarded processes
together and output-guarded processes together.

Thus, we can define the tensor product of two summations, $N_{1}\otimes N_{2}$, where

\begin{mathpar}
  N_{1} := \Sigma_{i} x_{i}.A_{i} + \Sigma_{j} x_{j}.C_{j}
  \and
  N_{2} := \Sigma_{i'} y_{i'}.B_{i'} + \Sigma_{j'} y_{j'}.D_{j'} 
\end{mathpar}

as follows.

\begin{mathpar}
  \Sigma_{i} x_{i}.A_{i} + \Sigma_{j} x_{j}.C_{j} \otimes \Sigma_{i'}
  y_{i'}.B_{i'} + \Sigma_{j'} y_{j'}.D_{j'} 
  \and \\
  := \; \Sigma_{i} \Sigma_{i'} \quotep{\stackrel{\vee}{x_{i}}| \stackrel{\vee}{y_{i'}}}.(A_{i}\otimes B_{i'}) \; | \; \Sigma_{i'} \Sigma_{i} \quotep{\stackrel{\vee}{y_{i'}}|\stackrel{\vee}{x_{i}}}.(B_{i'}\otimes A_{i})
  \and
  \;\; | \;\; \Sigma_{j} \Sigma_{j'} \quotep{\stackrel{\vee}{x_{j}}|\stackrel{\vee}{y_{j'}}}.(A_{j}\otimes B_{j'}) \; | \; \Sigma_{j'} \Sigma_{j} \quotep{\stackrel{\vee}{y_{j'}}|\stackrel{\vee}{x_{j}}}.(B_{j'}\otimes A_{j})
\end{mathpar}

\begin{remark}
  Do we need to $x^{L}$ and $y^{R}$ for this construction as well?
\end{remark}

\paragraph{Tensor of parallel compositions} Next, we distribute tensor
over par.

\begin{mathpar}
  P_{1}|P_{2} \otimes Q_{1}|Q_{2} := (P_{1} \otimes Q_{1}) | (P_{1}
  \otimes Q_{2}) | (P_{2} \otimes Q_{1}) | (P_{2} \otimes Q_{2})
\end{mathpar}

\paragraph{Tensor with dropped names} We treat tensor of a
process with a dropped name as parallel composition.

\begin{mathpar}
  P \otimes \dropn{x} := P | \dropn{x}
\end{mathpar}

\paragraph{Tensor of agents}

Finally, we need to define tensor on agents. Note that the definition
of tensor on normal products only tensors inputs with inputs and
outputs with outputs. Thus, we only have to define the operation on
``homogeneous'' pairings.

\begin{mathpar}
  (\vec{x})P \otimes (\vec{y})Q
  \and \\
  := (x_{0}^{L}|y_{0}^{R},\ldots,x_{0}^{L}|y_{n}^{R},\ldots,x_{m}^{L}|y_{0}^{R},\ldots,x_{m}^{L}|y_{n}^R)(P\{ \vec{x}^{L}/\vec{x}\} \otimes Q \{ \vec{y}^{R}/\vec{y}\})
  \and \\
  \clift{\vec{P}} \otimes \clift{\vec{Q}}
  \and \\
  := \clift{P_{0}\otimes Q_{0},\ldots,P_{0}\otimes Q_{n},\ldots,P_{m}\otimes Q_{0},\ldots,P_{m}\otimes Q_{n}}
\end{mathpar}

\begin{remark}
  Observe that arities of tensored abstractions matches arities of
  tensored concretions if the original arities matched. Note also that
  the length of the arities corresponds to the increase in dimension
  we see in ordinary vector space tensor product.
\end{remark}

\begin{remark}
  Operationally, this definition distributes the tensor down to
  components ``linked'' by summation. Tensor over summation is
  intriguing in that it mixes names. Moreover, as a consequence of the
  way it mixes names we have the identities for all $x \in \QProc$ and
  $P,Q \in \Proc$

  \begin{mathpar}
    (x \cdot P) \otimes Q \equiv x \cdot (P \otimes Q) \equiv P \otimes (x \cdot Q)
    \and
    P \otimes \pzero \equiv P
  \end{mathpar}

  that the reader is invited to verify.
\end{remark}

\subsubsection{Annihilation}
\begin{mathpar}
  P^{\perp} := \{ Q | \forall R. P|Q \red^{*} R \Rightarrow R \red^{*} \pzero \}
  \and \\
  P^{\underline{\perp}} := \Sigma_{Q \in P^{\perp}} \quotep{Q}?(y).(\dropn{y}|Q) | \Sigma_{Q \in P^{\perp}} \quotep{Q}\clift{\Box}
\end{mathpar}

\paragraph{Discussion} The reader will note that $P^{\perp}$ is a
\emph{set} of processes, while $P^{\underline{\perp}}$ is a
\emph{context}. We call the set $P^{\perp}$ the \emph{annihilators} of
$P$. The parallel composition of a process in the annihilators of $P$
with $P$ will result in a process, the state space of which has all
paths eventually leading to $\pzero$. Execution may endure loops; but
under reasonable conditions of fairness (naturally guaranteed under
most notions of bisimulation) such a composite process cannot get
stuck in such a loop and will, eventually pop out and terminate.

The context $P^{\underline{\perp}}$ is ready and willing to ``take the
$P$ out of'' the process to which it is applied. It will effectively
transmit the code of the process to which it is applied to one of the
annihilators and run the process against it.

\subsubsection{Evaluation}
We fix $M$ a domain of fully abstract interpretation with an equality
coincident with bisimulation. We take $\meaningof{\cdot} : \Proc \to
M$ to be the map interpreting processes and $\nmeaningof{\cdot} : \M
\to Proc$ to be the map running the other way. Then we define

\begin{mathpar}
  \int P := \nmeaningof{\meaningof{P}}
\end{mathpar}

\paragraph{Discussion}
There are many fully abstract interpretations of Milner's
$\pi$-calculus. Any of them can be used as a basis for interpreting
the reflective calculus here. Equipped with such a domain it is
largely a matter of grinding through to check that the Yoneda
construction for the normalization-by-evaluation program can be
extended to this setting.

\begin{remark}
  The reader is invited to verify that $\int (P^{\underline{\perp}}[P]) = 0$.
\end{remark}

\subsection{Quantum mechanics}

Table \ref{tbl:core_qm_op_defns} gives the core operational definitions

\begin{table}[htp]\label{tbl:core_qm_op_defns}
  \center{
    \fbox{
      \begin{tabular}{c|c}
        quantum mechanics & process calculus \\
        \hline
        scalar & $x := \quotep{P}$ \\
        state vector & $\state{P} := P$ \\
        dual & $\state{P}^{*} := \event{P^{\underline{\perp}}} := \quotep{P^{\underline{\perp}}}[-]$ \\
        matrix & $ \Sigma_{\alpha} \state{P_{\alpha}}x_{\alpha}\event{Q_{\alpha}}$ \\
        vector addition & $\state{P} + \state{Q} := \state{P | Q}$ \\
        tensor product & $\state{P} \otimes \state{Q} := \state{P \otimes Q}$ \\
        inner product & $\innerprod{P}{Q} := \quotep{\int P^{\underline{\perp}}[Q]}$ \\
      \end{tabular}
    }
  }
  \caption{QM - operational definitions}
\end{table}

where

\begin{mathpar}
  \prmatrix{P}{Q} := \fprmatrix{P}{\quotep{\pzero}}{Q}
  \and
  \fprmatrix{P}{x}{Q} := (\state{P},x,\event{Q})
  \and
  (\fprmatrix{P}{x}{Q})(\state{R}) := x \cdot \innerprod{Q}{R} \cdot \state{P}
  \and
  (\fprmatrix{P}{x}{Q})(\event{R}) := x \cdot \innerprod{R}{P} \cdot \event{Q}
\end{mathpar}

\paragraph{Discussion}
As promised: vectors (aka states) are represented as processes; duals
as contextual duals; inner product definition should be compared with
standard inner product definition for ....

\begin{remark}
  Assuming $\int (P^{\underline{\perp}}[P]) = 0$, the reader is
  invited to verify that $(\fprmatrix{P}{x}{P})(\state{P}) = x \cdot \state{P}$.
\end{remark}

\begin{remark}
  The reader is invited to verify that $\innerprod{P}{Q}$ could
  equally well have been written $\quotep{\int \stackrel{\vee}{x}}$
  where $x = \event{P^{\underline{\perp}}}(Q)$.

  One of the motivations for this remark is that there is another way
  to factor these operations. We could package up evaluation in the dual:

  \begin{mathpar}
    \state{P}^{*} := \event{\int P^{\underline{\perp}}} := \quotep{\int P^{\underline{\perp}}}[-]
  \end{mathpar}

  and then have inner product defined by
  
  \begin{mathpar}
    \innerprod{P}{Q} := \event{P}(Q)
  \end{mathpar}

  Hopefully, experience with the calculations will provide guidance on
  the best factoring.
\end{remark}

\begin{remark}
  Assuming $\int (P^{\underline{\perp}}[P]) = 0$, the reader is
  invited to verify that $\forall P,Q. (\prmatrix{0}{Q})(\state{0}) =
  \state{0}$ and dually $(\prmatrix{P}{0})(\event{0}) = \event{0}$.
\end{remark}

\begin{remark}
  i'm a little worried that i don't (yet) have proper support for
  complex conjugacy. But, the observation above may give us a
  clue. According to Abramsky, it must be the case that the scalars
  are iso to the homset of the identity for the tensor -- which the
  observation above characterizes. 

  For now, we will simply bookmark the notion with $\overline{x}$.
\end{remark}

\subsubsection{Adjointness}

We need to give a definition of $(\cdot)^{\dagger}$ for matrices. The
obvious candidate definition is
\begin{mathpar}
(\Sigma_{\alpha}\fprmatrix{P_{\alpha}}{x_{\alpha}}{Q_{\alpha}})^{\dagger}
= \Sigma_{\alpha}\fprmatrix{(Q_{\alpha}^{\underline{\perp}})^{*}}{\overline{x}_{\alpha}}{P_{\alpha}^{\underline{\perp}}} 
\end{mathpar}

But, $(Q_{\alpha}^{\underline{\perp}})^{*}$ requires a name along
which to communicate the process to achieve the context application.

\subsubsection{Basis for a basis}
If processes label states and ``addition'' of states (a.k.a. vector
addition) is interpreted as parallel composition, what corresponds to
notions of linear independence and basis? Here, we recall that Yoshida
has developed a set of \emph{combinators} for an asynchronous verison
of Milner's $\pi$-calculus. These are a finite set of processes such
any process can be expressed as parallel composition of these
combinators together with liberal uses of the new operator and
replication. We can simply give a translation of these into the
present calculus and have reasonable expectation that the property
carries over. That is, that the resultant set allows to express all
processes via parallel composition. Note, however, that there is no
new operator or replication in this calculus. As a result, we expect
that the corresponding set is actually infinite. That is, we expect
that the space is actually infinite dimensional.

\begin{remark}
  The attentive reader may be a bit concerned. Certainly, the
  collection $S$, $K$ and $I$ is a finite set of
  combinators. Shouldn't we expect to see a finite set of combinators
  for an effectively equivalent system? i am very sympathetic to this
  critique and feel it warrants full attention. On the other hand, i
  also have in mind the following analogy. The natural numbers, as a
  monoid under addition, has exactly $1$ generator, while the natural
  numbers, as a monoid under multiplication, has countably many
  generators (the primes). We observe that the application of the
  lambda calculus is much less resource sensitive than the parallel
  composition of the $\pi$-calculus. Could it be the case that we have
  an analogy of the form
  
  \begin{mathpar}
    m + n : MN :: m*n : M|N
  \end{mathpar}

  giving a similar blow up in the set of ``primes''?  This is such a
  wonderful thought that, even if it's not true, i think it's worth
  writing down.
\end{remark}
 

\documentclass[12pt]{llncs}
%\documentclass{jktr}

\usepackage[pdftex]{hyperref}                   
\usepackage {listings}
\usepackage {mathpartir}
\usepackage{bcprules}
%\usepackage{listings}
                       
\usepackage{graphicx} 
%\usepackage[margins=2.5cm,nohead,nofoot]{geometry}
%\usepackage{geometry}
\usepackage{amsfonts}
\usepackage{amstext}
\usepackage{latexsym}
\usepackage{amssymb}
\usepackage{color}


%\include{myPreamble}
\include{qm2pi.local} 

%\ifpdf
%\usepackage[pdftex]{graphicx}
%\else
%\usepackage{graphicx}
%\fi

 % \ifpdf
%  \usepackage{pdfsync}
%  \if


%\title{Brief Article}
%\author{David F. Snyder}
%\author{L.G. Meredith}

%\address{Dept. of Math., Texas State University--San Marcos, San Marcos, TX 78666}
       
\pagestyle{empty}


\begin{document}

\lstset{language=[Objective]Caml,frame=shadowbox}

\input{qm2pi.front}

% section front matter (end)

\input{qm2pi.intro} 
 
% section introduction (end)

% \input{qm2pi.knotations} 

% section notation (end)

\input{qm2pi.process.calculi} 

% section concurrent_process_calculi_and_spatial_logics_ (end)
    
%\input{qm2pi.knots2pi} 

%\input{qm2pi.trefoil} 

%\input{qm2pi.mainthm} 

% subsection basic_interpretation (end)

%\input{qm2pi.rho.presentation} 
\subsection{The syntax and semantics of the notation system}\label{sub:the_syntax_and_semantics_of_the_notation_system} % (fold)

We now summarize a technical presentation of the calculus that
embodies our theory of dynamics. The typical presentation of such a
calculus follows the style of giving generators and relations on
them. The grammar, below, describing term constructors, freely
generates the set of processes, $\Proc$. This set is then quotiented
by a relation known as structural congruence and it is over this set
that the notion of dynamics is expressed. This presentation is
essentially that of \cite{MeredithR05} with the addition of
polyadicity and summation. For readability we have relegated some of
the technical subtleties to an appendix.

\subsubsection{Process grammar}\label{subsub:process_grammar}

\begin{mathpar}
  \inferrule* [lab=synchronization] {} {{M} \bc \pzero \;|\; x?F \;|\; x!C }
  \and
  \inferrule* [lab=abstraction] {} {{F} \bc (x)P}
  \and
  \inferrule* [lab=concretion] {} {{C} \bc \langle Q \rangle}
  \and
  \inferrule* [lab=process] {} {{P,Q} \bc M \;| \;P|Q \;|\; @{x}}
  \and
  \inferrule* [lab=name] {} {{x} \bc \quotep{P}}
\end{mathpar} 

Note that $\vec{x}$ (resp. $\vec{P}$) denotes a vector of names
(resp. processes) of length $|\vec{x}|$ (resp. $|\vec{P}|$). We adopt
the following useful abbreviations.

\begin{mathpar}
   x?(\vec{y}).P := x.(\vec{y})P \and  x\clift{\vec{P}} := x.\clift{\vec{P}}
   \and x!(y) := \lift{x}{\dropn{y}}
   \and \Pi_{i=0}^{n-1}P_i := P_0 | \ldots | P_{n-1}
\end{mathpar}

\subsubsection{Structural congruence}

\paragraph{Free and bound names and alpha-equivalence.} At the
core of structural equivalence is alpha-equivalence which identifies
process that are the same up to a change of variable. Formally, we
recognize the distinction between free and bound names. The free names
of a process, $\freenames{P}$, may be calculated recursively as
follows:

\begin{mathpar}
\freenames{\pzero} := \emptyset
  \and \\
  \freenames{x?(y).P} := \{ x \} \cup (\freenames{P} \setminus \{ y \})
  \and 
  \freenames{x!\langle P \rangle} := \{ x \} \cup \{ P \} 
  \and \\
  \freenames{P|Q} := \freenames{P} \cup \freenames{Q}
  \and \\
  \freenames{@{x}} := \{ x \}
\end{mathpar}

$\pi$
$\quotep{\pi}$

$\freenames{-} : \pi \to \mathcal{P}(\quotep{\pi})$

\begin{eqnarray*}
  \freenames{\pzero} & := & \emptyset \\
  \freenames{x?(y).P} & := & \{ x \} \cup (\freenames{P} \setminus \{ y \}) \\
  \freenames{x!\langle P \rangle} & := & \{ x \} \cup \{ P \} \\
  \freenames{P|Q} & := & \freenames{P} \cup \freenames{Q} \\
  \freenames{\dropn{x}} & := & \{ x \}
\end{eqnarray*}

The bound names of a process, $\boundnames{P}$, are those names occurring in $P$
that are not free. For example, in $x?(y).0$, the name $x$ is free, while $y$ is bound.

\begin{mathpar}
  \inferrule* [lab=monoidal-laws] {} { P|Q \equiv Q|P \and P|0 \equiv P \and P|(Q|R) \equiv (P|Q)|R }
\end{mathpar}

\begin{mathpar}
  \inferrule* [lab=alpha-equivalence] {} { (x)P \equiv (y)P\{y/x\} \and y \not\in \freenames{P} }
\end{mathpar}

\begin{definition}
Then two processes, $P,Q$, are alpha-equivalent if $P = Q\{\vec{y}/\vec{x}\}$ for
some $\vec{x} \in \boundnames{Q},\vec{y} \in \boundnames{P}$, where $Q\{\vec{y}/\vec{x}\}$
denotes the capture-avoiding substitution of $\vec{y}$ for $\vec{x}$ in $Q$.
\end{definition}

\begin{definition}
  The {\em structural congruence} \cite{SangiorgiWalker} , $\equiv$,
  between processes is the least congruence containing
  alpha-equivalence, satisfying the abelian monoid laws
  (associativity, commutativity and $\pzero$ as identity) for parallel
  composition $|$ and for summation $+$.
\end{definition}

\subsection{Name equivalence}

We take name equivalence, written $\nameeq$, to be the smallest
equivalence relation generated by the following rules.

\begin{mathpar}
\inferrule*[lab=Quote-drop]
{ }
{ \quotep{@{x}} \nameeq x }

\inferrule*[lab=Struct-equiv]
{ P \scong Q }
{ \quotep{P} \nameeq \quotep{Q} }
\end{mathpar}

The astute reader will have noticed that the mutual recursion of names
and processes imposes a mutual recursion on alpha-equivalence and
structural equivalence via name-equivalence. Fortunately, all of this
works out pleasantly and we may calculate in the natural way, free of
concern. The reader interested in the details is referred to the
appendix \ref{appendix:rho_details}.

\subsection{Substitution}

We use $\Proc$ for the set of processes, $\QProc$ for the set of
names, and $\id{\{}\vec{y} / \vec{x} \id{\}}$ to denote partial maps,
$s : \QProc \rightarrow \QProc$. A map, $s$ lifts, uniquely, to a map
on process terms, $\widehat{s} : \Proc \rightarrow \Proc$ by the
following equations.

\begin{mathpar}
  (0) \psubstp{Q}{P} := 0 \\
  (R \juxtap S) \psubstp{Q}{P}
  :=    
  (R)\psubstp{Q}{P} \juxtap (S) \psubstp{Q}{P} \\
  (x?(y).R) \psubstp{Q}{P}    
  :=    
  (x)\substp{Q}{P} (z)\concat( (R \psubstn{z}{y}) \psubstp{Q}{P} ) \\
  (\lift{x}{R}) \psubstp{Q}{P}  
  :=
  \lift{(x)\substp{Q}{P}}{ R \psubstp{Q}{P} } \\
%   (\dropn{x})  \psubstp{Q}{P}       
%   := 
%   \left\{ 
%     \begin{array}{ccc} 
%       \dropn{\quotep{Q}} & & x \nameeq \quotep{P} \\
%       \dropn{x} & & otherwise \\
%     \end{array}
%   \right. 
  (\dropn{x})  \psubstp{Q}{P}       
  := 
  \left\{ 
    \begin{array}{ccc} 
      Q & & x \nameeq \quotep{P} \\
      \dropn{x} & & otherwise \\
    \end{array}
  \right.
\end{mathpar}
 

where

\begin{eqnarray}
  (x)\id{\{} \lpquote Q \rpquote / \lpquote P \rpquote \id{\}}            = 
  \left\{ 
    \begin{array}{ccc}
      \lpquote Q \rpquote & & x \nameeq \lpquote P \rpquote \\
      x & & otherwise \\
    \end{array}
  \right. \nonumber
\end{eqnarray}

and $z$ is chosen distinct from $\quotep{P}$, $\quotep{Q}$, the free
names in $Q$, and all the names in $R$. Our $\alpha$-equivalence will
be built in the standard way from this substitution.

\begin{remark}\label{rem:no_self_referential_names}
  One consequence of these definitions is that $\forall P. \quotep{P}
  \not\in \freenames{P}$.
\end{remark}

\subsection{ Dynamic quote: an example }

Anticipating something of what's to come, consider applying the
substitution, $\widehat{\id{\{}u / z \id{\}}}$, to the following pair
of processes, $\lift{w}{y!(z)}$ and $w[ \lpquote y!(z) \rpquote ]$.

\begin{eqnarray}
	\lift{w}{y!(z)}\widehat{\id{\{}u / z \id{\}}}
		& = &
		\lift{w}{y!(u)} \nonumber\\
	w[ \lpquote y!(z) \rpquote ] \widehat{ \id{\{}u / z \id{\}} }
		& = &
		w[ \lpquote y!(z) \rpquote ] \nonumber
\end{eqnarray}

Because the body of the process between quotes is impervious to
substitution, we get radically different answers. In fact, by
examining the first process in an input context,
e.g. $x?(z).\lift{w}{y!(z)}$, we see that the process under the lift
operator may be shaped by prefixed inputs binding a name inside it. In
this sense, the lift operator will be seen as a way to dynamically
construct processes before reifying them as names.

Finally equipped with these standard features we can present the
dynamics of the calculus.

\subsubsection{Operational semantics} 

Finally, we introduce the computational dynamics. What marks these
algebras as distinct from other more traditionally studied algebraic
structures, e.g. vector spaces or polynomial rings, is the manner in
which dynamics is captured. In traditional structures, dynamics is typically
expressed through morphisms between such structures, as in linear maps
between vector spaces or morphisms between rings. In algebras
associated with the semantics of computation, the dynamics is
expressed as part of the algebraic structure itself, through a
reduction reduction relation typically denoted by $\red$. Below, we
give a recursive presentation of this relation for the calculus used
in the encoding.

$\red \subseteq \pi \times \pi$
$\red : \pi \to \mathcal{P}(\pi)$

\begin{mathpar}
  \inferrule* [lab=Comm] { \textsf{match}( x_{src}, x_{trgt} ) } { x_{trgt}?(y)P \; | \; x_{src}!\langle {Q} \rangle \red P\{\quotep{Q}/y}\} }
  \and \\
  \inferrule* [lab=Par] {{P} \red {P}'} {{{P} | {Q}} \red {{P}' | {Q}}}
  \and
  \inferrule* [lab=Equiv]{{{P} \scong {P}'} \andalso {{P}' \red {Q}'} \andalso {{Q}' \scong {Q}}}{{P} \red {Q}}
\end{mathpar}

\begin{eqnarray*}
  match_{\equiv} (\quotep{P},\quotep{Q}) & := & P \equiv Q \\
  match_{\dagger}(\quotep{P},\quotep{Q}) & := & \forall R. P|Q \red^{*} R => R \red^{*} 0 \\
  match_{K}(\quotep{P},\quotep{Q}) & := & K \mbox{ for some context } K
\end{eqnarray*}

$u?(x)P | u!\langle Q \rangle \red P\{\quotep{Q}/x\}$

%We write $\wred$ for $\red^*$, and $P\red$ if $\exists Q $ such that $ P \red Q$.
We write $P\red$ if $\exists Q $ such that $ P \red Q$ and $P\not\red$, otherwise.

\section{Replication}

As mentioned before, it is known that replication (and hence
recursion) can be implemented in a higher-order process algebra
\cite{SangiorgiWalker}. As our first example of calculation with the
machinery thus far presented we give the construction explicitly in
the {\rhoc}.

\begin{eqnarray}
	D_{x} & := & \prefix{x}{y}{(\binpar{\outputp{x}{y}}{@{y}})} \nonumber\\
	\bangp_{x}{P} & := & \binpar{{x}!\langle{\binpar{D_{x}}{P}}\rangle}{D_{x}} \nonumber
\end{eqnarray}

\begin{eqnarray}
	\bangp_{x}{P} & & \nonumber\\
	=
	& {x}!\langle{(\prefix{x}{y}{(\outputp{x}{y} | @{y})) | P}}\rangle 
	      | \prefix{x}{y}{(\outputp{x}{y} | @{y})} & \nonumber\\
	\red
	& (\outputp{x}{y} | @{y})\substn{\quotep{(\prefix{x}{y}{(@{y} | \outputp{x}{y})) | P}}}{y} & \nonumber\\
	=
	& \outputp{x}{\quotep{(\prefix{x}{y}{(\outputp{x}{y} | @{y})) | P}}}
	  | {(\prefix{x}{y}{(\outputp{x}{y} | @{y})) | P}} & \nonumber\\
	\red
	& \ldots & \nonumber\\
	\red^*
	& P | P | \ldots & \nonumber
\end{eqnarray}

Of course, this encoding, as an implementation, runs away, unfolding
$\bangp{P}$ eagerly. A lazier and more implementable replication
operator, restricted to input-guarded processes, may be obtained as follows.

\begin{eqnarray}
\bangp{\prefix{u}{v}{P}} 
	:= 
	\binpar{\lift{x}{\prefix{u}{v}{(\binpar{D(x)}{P})}}}{D(x)} \nonumber
\end{eqnarray}

\begin{remark}
  Note that the lazier definition still does not deal with summation
  or mixed summation (i.e. sums over input and output). The reader is
  invited to construct definitions of replication that deal with these
  features. 

  Further, the definitions are parameterized in a name, $x$. Can you,
  gentle reader, make a definition that eliminates this parameter and
  guarantees no accidental interaction between the replication
  machinery and the process being replicated -- i.e. no accidental
  sharing of names used by the process to get its work done and the
  name(s) used by the replication to effect copying. This latter
  revision of the definition of replication is crucial to obtaining
  the expected identity $!!P \sim !P$.
\end{remark}

\begin{remark}\label{rem:paradoxical_combinator}
  The reader familiar with the lambda calculus will have noticed the
  similarity between $D$ and the paradoxical combinator.

  [Ed. note: the existence of this seems to suggest we have to be more
  restrictive on the set of processes and names we admit if we are to
  support no-cloning.]
\end{remark}

\subsubsection{Bisimulation}

The computational dynamics gives rise to another kind of equivalence,
the equivalence of computational behavior. As previously mentioned
this is typically captured \emph{via} some form of bisimulation.

% The notion we use in this paper is weak barbed bisimulation
% \cite{milner91polyadicpi}.

The notion we use in this paper is derived from weak barbed
bisimulation \cite{milner91polyadicpi}. 

\begin{definition}
An \emph{observation relation}, $\downarrow_{\mathcal N}$, over a set
of names, $\mathcal N$, is the smallest relation satisfying the rules
below.

\infrule[Out-barb]{y \in {\mathcal N}, \; x \nameeq y}
		  {\outputp{x}{v} \downarrow_{\mathcal N} x}
\infrule[Par-barb]{\mbox{$P\downarrow_{\mathcal N} x$ or $Q\downarrow_{\mathcal N} x$}}
		  {\binpar{P}{Q} \downarrow_{\mathcal N} x}

We write $P \Downarrow_{\mathcal N} x$ if there is $Q$ such that 
$P \wred Q$ and $Q \downarrow_{\mathcal N} x$.
\end{definition}

\begin{definition}
%\label{def.bbisim}
An  ${\mathcal N}$-\emph{barbed bisimulation} over a set of names, ${\mathcal N}$, is a symmetric binary relation 
${\mathcal S}_{\mathcal N}$ between agents such that $P\rel{S}_{\mathcal N}Q$ implies:
\begin{enumerate}
\item If $P \red P'$ then $Q \wred Q'$ and $P'\rel{S}_{\mathcal N} Q'$.
\item If $P\downarrow_{\mathcal N} x$, then $Q\Downarrow_{\mathcal N} x$.
\end{enumerate}
$P$ is ${\mathcal N}$-barbed bisimilar to $Q$, written
$P \wbbisim_{\mathcal N} Q$, if $P \rel{S}_{\mathcal N} Q$ for some ${\mathcal N}$-barbed bisimulation ${\mathcal S}_{\mathcal N}$.
\end{definition}

$\mathcal{R} \subseteq \pi \times \pi$

$P \mathcal{R} Q => \forall P'. P \red P' \Rightarrow \exists Q'. Q \red Q', P' \mathcal{R} Q'$

$P \vdash x \Rightarrow Q \vdash x$

\begin{mathpar}
  \inferrule*[lab=Out-barb]{x \nameeq y}{{y}!\langle{Q}\rangle \vdash x}
  \and
  \inferrule*[lab=Par-barb]{\mbox{$P\vdash x$ or $Q\vdash x$}}{\binpar{P}{Q} \vdash x}
\end{mathpar}

\subsubsection{Contexts}

One of the principle advantages of computational calculi like the
$\pi$-calculus is a well-defined notion of context,
contextual-equivalence and a correlation between
contextual-equivalence and notions of bisimulation. The notion of
context allows the decomposition of a process into (sub-)process and
its syntactic environment, its context. Thus, a context may be
thought of as a process with a ``hole'' (written $\Box$) in it. The
application of a context $M$ to a process $P$, written $M[P]$, is
tantamount to filling the hole in $M$ with $P$. In this paper we do
not need the full weight of this theory, but do make use of the notion
of context in the proof the main theorem. 

\begin{mathpar}
  \inferrule* [lab=summation] {} {{M_{M},M_{N}} \bc \Box \;|\; x.M_{A} \;|\; M_{M}+M_{N}}
  \and
  \inferrule* [lab=agent] {} {{M_{A}} \bc (\vec{x})M_{P} \;| \; \clift{P_0,\ldots,M_{P},\ldots,P_N}}
  \and \\
  \inferrule* [lab=process] {} {{M_{P}} \bc M_{N} \;| \;P|M_{P} }
\end{mathpar} 

\begin{mathpar}
  \inferrule* [lab=sychronization] {} {M_{N} \bc \Box \;|\; x?M_{F} \;|\; x!M_{C}}
  \and
  \inferrule* [lab=abstraction] {} {{M_{F}} \bc (x)M_{P} }
  \and
  \inferrule* [lab=concretion] {} {{M_{C}} \bc \langle M_{P} \rangle }
  \and \\
  \inferrule* [lab=process] {} {{M_{P}} \bc M_{N} \;| \;P|M_{P} }
\end{mathpar}

\begin{definition}[contextual application] Given a context $M$, and
  process $P$, we define the \emph{contextual application}, $M[P] :=
  M\{P/\Box\}$. That is, the contextual application of M to P is the
  substitution of $P$ for $\Box$ in $M$.
\end{definition}

$\meaningof{-} : L \to \mathcal{P}(\pi)$

\begin{mathpar}
  \inferrule* [lab=collection] {} {\meaningof{true} = \pi, \and \meaningof{~E} = \pi \setminus \meaningof{E}, \and \meaningof{E_{1} \& E_{2}} = \meaningof{E_{1}} \cap \meaningof{E_{2}}}
\end{mathpar}

\begin{mathpar}
  \inferrule* [lab=structure] {} {\meaningof{0} = \{ P \in \pi | P \equiv 0 \}, \and \\ \meaningof{E_1 | E_2} = \{ P \in \pi | P \equiv P_{1} | P_{2}, P_{1} \in \meaningof{E_{1}}, P_{2} \in \meaningof{E_2}\} }
\end{mathpar}

\begin{mathpar}
 \inferrule* [lab=behavior] {} {\meaningof{\langle a?b \rangle E} = \{ P \in \pi | P \equiv Q | u?(y)P', \\ \and \\\\ \and \\ \;\;\; u \in \meaningof{a}, \forall z.P'\{z/y\} \in \meaningof{E\{z/b\}}\}, \and \\ \meaningof{a!E} = \{ P \in \pi | P \equiv Q | x!\langle P' \rangle, x \in \meaningof{a} P' \in \meaningof{E}\} }
\end{mathpar}

\begin{mathpar}
 \inferrule* [lab=nominal] {} {\meaningof{\quotep{E}} = \{ \quotep{P} \in \quotep{\pi} | P \in \meaningof{E} \}, \and \meaningof{\quotep{P}} = \{ \quotep{Q} \in \quotep{\pi} | P \equiv Q \} \and \\ \meaningof{@\quotep{E}} = \{ P \in \pi | P \equiv @x, x \in \meaningof{E} \}}
\end{mathpar}

\begin{eqnarray*}
  \\
  \meaningof{-} : TS \to ST
\end{eqnarray*}

\begin{eqnarray*}
  \\
  L : TS \to ST
\end{eqnarray*}

\begin{eqnarray*}
  \\
  P \models E \iff P \in \meaningof{E}
\end{eqnarray*}

\begin{eqnarray*}
  P \approx_{L} Q \iff \forall E \in L. P \models E \iff Q \models E
\end{eqnarray*}

\begin{eqnarray*}
  P \approx_{K} Q
\end{eqnarray*}

\begin{eqnarray*}
  P \approx Q
\end{eqnarray*}

$\approx_{K} = \approx = \approx_{L}$

\subsubsection{Contextual duality}

Note that contexts extend the quotation operation to a family of
operations from processes to names. Given a context, $M$, we can
define a \emph{nominal context}, $\quotep{M}$ by $\quotep{M}[P] :=
\quotep{M[P]}$. To foreshadow what is to come we observe that these
operations enjoy a duality with processes very much like the duality
between vectors and maps from vectors to scalars.

Further, because the calculus is essentially higher-order, we have a
correspondence between contexts and processes. More specifically,
given a name $x$ and a context $M$ we can construct $M^{*}_{x}$ such
that 

\begin{mathpar}
  M^{*}_{x} | \lift{x}{P} \red M[P]
\end{mathpar}

namely,

\begin{mathpar}
  M^{*}_{x} := x?(u).M[\dropn{u}]
\end{mathpar}

The dependence of $M^{*}_{x}$ on a name makes it an abstraction, 

\begin{mathpar}
  M^{*} := (x)x?(u).M[\dropn{u}]
\end{mathpar}

\subsection{Additional notation}

It will sometimes be convenient to denote the process a name
quotes. We already have the notation $x = \quotep{P}$, but it will be
convenient to introduce an alternate notation, $\procn{x}$, when we
want to emphasize the connection to the use of the name. Note that, by
virtue of name equivalence, $\quotep{\procn{x}} \nameeq x$; so, the
notation is consistent with previous definitions.

Further, because names have structure it is possible to effect
substitutions on the basis of that structure. This means we need to
upgrade our notation for substitutions, which we accomplish by
adapting comprehension notation. Thus,

\begin{mathpar}
  P\{ y / x : x \in S \}
\end{mathpar}

is interpreted to mean the process derived from P by replacing (in a
capture-avoiding manner) each occurrence of $x$ in $S$ by $y$. For example,

\begin{mathpar}
  P\{ \quotep{\procn{x}|\procn{x}} / x : x \in \freenames{P} \}
\end{mathpar}

will replace each (occurrence) of a free name $x$ in $P$ by
$\quotep{\procn{x}|\procn{x}}$.

Also, we will avail ourselves of the notation $x^{L}$ and $x^{R}$ to
denote injections of a name into disjoint copies of the name
space. There are numerous ways to accomplish this. One example can be
found in \cite{MeredithR05}. This notation overloads to vectors of
names: $\vec{x}^{\pi} := (x_{i}^{\pi} \; : \; 0 \leq i < |\vec{x}| )$ where $\pi \in \{L,R\}$.

We also use $P^{\Box} := P|\Box$.

In \cite{MeredithR05} an interpretation of the new operator is
given. It turns out that there are several possible interpretations
all enjoying the requisite algebraic properties of the operator (see
\cite{milner91polyadicpi}). We will therefore make liberal use of
$(\nu\; \vec{x})P$.

% subsection the_syntax_and_semantics_of_the_notation_system (end)   

\input{qm2pi.qmops} 

\input{qm2pi.sterngerlach} 

\input{qm2pi.metric} 

% section concurrent_process_calculi (end)

%\input{qm2pi.proofsketch}

% section proof sketch (end)

%\input{qm2pi.slviaknots} 

% section spatial logic via knots (end)

\input{qm2pi.conclusion}

% section conclusion (end)

%\input{qm2pi.dtcodes} 

% section wiring algorithm (end)

\input{qm2pi.ack} 

% section acknowledgments (end)

\newpage


\bibliographystyle{plain}   
\bibliography{../../biblios/main.bib}

\input{qm2pi.rhodetails}

\end{document}

 

\documentclass[12pt]{llncs}
%\documentclass{jktr}

\usepackage[pdftex]{hyperref}                   
\usepackage {listings}
\usepackage {mathpartir}
\usepackage{bcprules}
%\usepackage{listings}
                       
\usepackage{graphicx} 
%\usepackage[margins=2.5cm,nohead,nofoot]{geometry}
%\usepackage{geometry}
\usepackage{amsfonts}
\usepackage{amstext}
\usepackage{latexsym}
\usepackage{amssymb}
\usepackage{color}


%\include{myPreamble}
\include{qm2pi.local} 

%\ifpdf
%\usepackage[pdftex]{graphicx}
%\else
%\usepackage{graphicx}
%\fi

 % \ifpdf
%  \usepackage{pdfsync}
%  \if


%\title{Brief Article}
%\author{David F. Snyder}
%\author{L.G. Meredith}

%\address{Dept. of Math., Texas State University--San Marcos, San Marcos, TX 78666}
       
\pagestyle{empty}


\begin{document}

\lstset{language=[Objective]Caml,frame=shadowbox}

\input{qm2pi.front}

% section front matter (end)

\input{qm2pi.intro} 
 
% section introduction (end)

% \input{qm2pi.knotations} 

% section notation (end)

\input{qm2pi.process.calculi} 

% section concurrent_process_calculi_and_spatial_logics_ (end)
    
%\input{qm2pi.knots2pi} 

%\input{qm2pi.trefoil} 

%\input{qm2pi.mainthm} 

% subsection basic_interpretation (end)

%\input{qm2pi.rho.presentation} 
\subsection{The syntax and semantics of the notation system}\label{sub:the_syntax_and_semantics_of_the_notation_system} % (fold)

We now summarize a technical presentation of the calculus that
embodies our theory of dynamics. The typical presentation of such a
calculus follows the style of giving generators and relations on
them. The grammar, below, describing term constructors, freely
generates the set of processes, $\Proc$. This set is then quotiented
by a relation known as structural congruence and it is over this set
that the notion of dynamics is expressed. This presentation is
essentially that of \cite{MeredithR05} with the addition of
polyadicity and summation. For readability we have relegated some of
the technical subtleties to an appendix.

\subsubsection{Process grammar}\label{subsub:process_grammar}

\begin{mathpar}
  \inferrule* [lab=synchronization] {} {{M} \bc \pzero \;|\; x?F \;|\; x!C }
  \and
  \inferrule* [lab=abstraction] {} {{F} \bc (x)P}
  \and
  \inferrule* [lab=concretion] {} {{C} \bc \langle Q \rangle}
  \and
  \inferrule* [lab=process] {} {{P,Q} \bc M \;| \;P|Q \;|\; @{x}}
  \and
  \inferrule* [lab=name] {} {{x} \bc \quotep{P}}
\end{mathpar} 

Note that $\vec{x}$ (resp. $\vec{P}$) denotes a vector of names
(resp. processes) of length $|\vec{x}|$ (resp. $|\vec{P}|$). We adopt
the following useful abbreviations.

\begin{mathpar}
   x?(\vec{y}).P := x.(\vec{y})P \and  x\clift{\vec{P}} := x.\clift{\vec{P}}
   \and x!(y) := \lift{x}{\dropn{y}}
   \and \Pi_{i=0}^{n-1}P_i := P_0 | \ldots | P_{n-1}
\end{mathpar}

\subsubsection{Structural congruence}

\paragraph{Free and bound names and alpha-equivalence.} At the
core of structural equivalence is alpha-equivalence which identifies
process that are the same up to a change of variable. Formally, we
recognize the distinction between free and bound names. The free names
of a process, $\freenames{P}$, may be calculated recursively as
follows:

\begin{mathpar}
\freenames{\pzero} := \emptyset
  \and \\
  \freenames{x?(y).P} := \{ x \} \cup (\freenames{P} \setminus \{ y \})
  \and 
  \freenames{x!\langle P \rangle} := \{ x \} \cup \{ P \} 
  \and \\
  \freenames{P|Q} := \freenames{P} \cup \freenames{Q}
  \and \\
  \freenames{@{x}} := \{ x \}
\end{mathpar}

$\pi$
$\quotep{\pi}$

$\freenames{-} : \pi \to \mathcal{P}(\quotep{\pi})$

\begin{eqnarray*}
  \freenames{\pzero} & := & \emptyset \\
  \freenames{x?(y).P} & := & \{ x \} \cup (\freenames{P} \setminus \{ y \}) \\
  \freenames{x!\langle P \rangle} & := & \{ x \} \cup \{ P \} \\
  \freenames{P|Q} & := & \freenames{P} \cup \freenames{Q} \\
  \freenames{\dropn{x}} & := & \{ x \}
\end{eqnarray*}

The bound names of a process, $\boundnames{P}$, are those names occurring in $P$
that are not free. For example, in $x?(y).0$, the name $x$ is free, while $y$ is bound.

\begin{mathpar}
  \inferrule* [lab=monoidal-laws] {} { P|Q \equiv Q|P \and P|0 \equiv P \and P|(Q|R) \equiv (P|Q)|R }
\end{mathpar}

\begin{mathpar}
  \inferrule* [lab=alpha-equivalence] {} { (x)P \equiv (y)P\{y/x\} \and y \not\in \freenames{P} }
\end{mathpar}

\begin{definition}
Then two processes, $P,Q$, are alpha-equivalent if $P = Q\{\vec{y}/\vec{x}\}$ for
some $\vec{x} \in \boundnames{Q},\vec{y} \in \boundnames{P}$, where $Q\{\vec{y}/\vec{x}\}$
denotes the capture-avoiding substitution of $\vec{y}$ for $\vec{x}$ in $Q$.
\end{definition}

\begin{definition}
  The {\em structural congruence} \cite{SangiorgiWalker} , $\equiv$,
  between processes is the least congruence containing
  alpha-equivalence, satisfying the abelian monoid laws
  (associativity, commutativity and $\pzero$ as identity) for parallel
  composition $|$ and for summation $+$.
\end{definition}

\subsection{Name equivalence}

We take name equivalence, written $\nameeq$, to be the smallest
equivalence relation generated by the following rules.

\begin{mathpar}
\inferrule*[lab=Quote-drop]
{ }
{ \quotep{@{x}} \nameeq x }

\inferrule*[lab=Struct-equiv]
{ P \scong Q }
{ \quotep{P} \nameeq \quotep{Q} }
\end{mathpar}

The astute reader will have noticed that the mutual recursion of names
and processes imposes a mutual recursion on alpha-equivalence and
structural equivalence via name-equivalence. Fortunately, all of this
works out pleasantly and we may calculate in the natural way, free of
concern. The reader interested in the details is referred to the
appendix \ref{appendix:rho_details}.

\subsection{Substitution}

We use $\Proc$ for the set of processes, $\QProc$ for the set of
names, and $\id{\{}\vec{y} / \vec{x} \id{\}}$ to denote partial maps,
$s : \QProc \rightarrow \QProc$. A map, $s$ lifts, uniquely, to a map
on process terms, $\widehat{s} : \Proc \rightarrow \Proc$ by the
following equations.

\begin{mathpar}
  (0) \psubstp{Q}{P} := 0 \\
  (R \juxtap S) \psubstp{Q}{P}
  :=    
  (R)\psubstp{Q}{P} \juxtap (S) \psubstp{Q}{P} \\
  (x?(y).R) \psubstp{Q}{P}    
  :=    
  (x)\substp{Q}{P} (z)\concat( (R \psubstn{z}{y}) \psubstp{Q}{P} ) \\
  (\lift{x}{R}) \psubstp{Q}{P}  
  :=
  \lift{(x)\substp{Q}{P}}{ R \psubstp{Q}{P} } \\
%   (\dropn{x})  \psubstp{Q}{P}       
%   := 
%   \left\{ 
%     \begin{array}{ccc} 
%       \dropn{\quotep{Q}} & & x \nameeq \quotep{P} \\
%       \dropn{x} & & otherwise \\
%     \end{array}
%   \right. 
  (\dropn{x})  \psubstp{Q}{P}       
  := 
  \left\{ 
    \begin{array}{ccc} 
      Q & & x \nameeq \quotep{P} \\
      \dropn{x} & & otherwise \\
    \end{array}
  \right.
\end{mathpar}
 

where

\begin{eqnarray}
  (x)\id{\{} \lpquote Q \rpquote / \lpquote P \rpquote \id{\}}            = 
  \left\{ 
    \begin{array}{ccc}
      \lpquote Q \rpquote & & x \nameeq \lpquote P \rpquote \\
      x & & otherwise \\
    \end{array}
  \right. \nonumber
\end{eqnarray}

and $z$ is chosen distinct from $\quotep{P}$, $\quotep{Q}$, the free
names in $Q$, and all the names in $R$. Our $\alpha$-equivalence will
be built in the standard way from this substitution.

\begin{remark}\label{rem:no_self_referential_names}
  One consequence of these definitions is that $\forall P. \quotep{P}
  \not\in \freenames{P}$.
\end{remark}

\subsection{ Dynamic quote: an example }

Anticipating something of what's to come, consider applying the
substitution, $\widehat{\id{\{}u / z \id{\}}}$, to the following pair
of processes, $\lift{w}{y!(z)}$ and $w[ \lpquote y!(z) \rpquote ]$.

\begin{eqnarray}
	\lift{w}{y!(z)}\widehat{\id{\{}u / z \id{\}}}
		& = &
		\lift{w}{y!(u)} \nonumber\\
	w[ \lpquote y!(z) \rpquote ] \widehat{ \id{\{}u / z \id{\}} }
		& = &
		w[ \lpquote y!(z) \rpquote ] \nonumber
\end{eqnarray}

Because the body of the process between quotes is impervious to
substitution, we get radically different answers. In fact, by
examining the first process in an input context,
e.g. $x?(z).\lift{w}{y!(z)}$, we see that the process under the lift
operator may be shaped by prefixed inputs binding a name inside it. In
this sense, the lift operator will be seen as a way to dynamically
construct processes before reifying them as names.

Finally equipped with these standard features we can present the
dynamics of the calculus.

\subsubsection{Operational semantics} 

Finally, we introduce the computational dynamics. What marks these
algebras as distinct from other more traditionally studied algebraic
structures, e.g. vector spaces or polynomial rings, is the manner in
which dynamics is captured. In traditional structures, dynamics is typically
expressed through morphisms between such structures, as in linear maps
between vector spaces or morphisms between rings. In algebras
associated with the semantics of computation, the dynamics is
expressed as part of the algebraic structure itself, through a
reduction reduction relation typically denoted by $\red$. Below, we
give a recursive presentation of this relation for the calculus used
in the encoding.

$\red \subseteq \pi \times \pi$
$\red : \pi \to \mathcal{P}(\pi)$

\begin{mathpar}
  \inferrule* [lab=Comm] { \textsf{match}( x_{src}, x_{trgt} ) } { x_{trgt}?(y)P \; | \; x_{src}!\langle {Q} \rangle \red P\{\quotep{Q}/y}\} }
  \and \\
  \inferrule* [lab=Par] {{P} \red {P}'} {{{P} | {Q}} \red {{P}' | {Q}}}
  \and
  \inferrule* [lab=Equiv]{{{P} \scong {P}'} \andalso {{P}' \red {Q}'} \andalso {{Q}' \scong {Q}}}{{P} \red {Q}}
\end{mathpar}

\begin{eqnarray*}
  match_{\equiv} (\quotep{P},\quotep{Q}) & := & P \equiv Q \\
  match_{\dagger}(\quotep{P},\quotep{Q}) & := & \forall R. P|Q \red^{*} R => R \red^{*} 0 \\
  match_{K}(\quotep{P},\quotep{Q}) & := & K \mbox{ for some context } K
\end{eqnarray*}

$u?(x)P | u!\langle Q \rangle \red P\{\quotep{Q}/x\}$

%We write $\wred$ for $\red^*$, and $P\red$ if $\exists Q $ such that $ P \red Q$.
We write $P\red$ if $\exists Q $ such that $ P \red Q$ and $P\not\red$, otherwise.

\section{Replication}

As mentioned before, it is known that replication (and hence
recursion) can be implemented in a higher-order process algebra
\cite{SangiorgiWalker}. As our first example of calculation with the
machinery thus far presented we give the construction explicitly in
the {\rhoc}.

\begin{eqnarray}
	D_{x} & := & \prefix{x}{y}{(\binpar{\outputp{x}{y}}{@{y}})} \nonumber\\
	\bangp_{x}{P} & := & \binpar{{x}!\langle{\binpar{D_{x}}{P}}\rangle}{D_{x}} \nonumber
\end{eqnarray}

\begin{eqnarray}
	\bangp_{x}{P} & & \nonumber\\
	=
	& {x}!\langle{(\prefix{x}{y}{(\outputp{x}{y} | @{y})) | P}}\rangle 
	      | \prefix{x}{y}{(\outputp{x}{y} | @{y})} & \nonumber\\
	\red
	& (\outputp{x}{y} | @{y})\substn{\quotep{(\prefix{x}{y}{(@{y} | \outputp{x}{y})) | P}}}{y} & \nonumber\\
	=
	& \outputp{x}{\quotep{(\prefix{x}{y}{(\outputp{x}{y} | @{y})) | P}}}
	  | {(\prefix{x}{y}{(\outputp{x}{y} | @{y})) | P}} & \nonumber\\
	\red
	& \ldots & \nonumber\\
	\red^*
	& P | P | \ldots & \nonumber
\end{eqnarray}

Of course, this encoding, as an implementation, runs away, unfolding
$\bangp{P}$ eagerly. A lazier and more implementable replication
operator, restricted to input-guarded processes, may be obtained as follows.

\begin{eqnarray}
\bangp{\prefix{u}{v}{P}} 
	:= 
	\binpar{\lift{x}{\prefix{u}{v}{(\binpar{D(x)}{P})}}}{D(x)} \nonumber
\end{eqnarray}

\begin{remark}
  Note that the lazier definition still does not deal with summation
  or mixed summation (i.e. sums over input and output). The reader is
  invited to construct definitions of replication that deal with these
  features. 

  Further, the definitions are parameterized in a name, $x$. Can you,
  gentle reader, make a definition that eliminates this parameter and
  guarantees no accidental interaction between the replication
  machinery and the process being replicated -- i.e. no accidental
  sharing of names used by the process to get its work done and the
  name(s) used by the replication to effect copying. This latter
  revision of the definition of replication is crucial to obtaining
  the expected identity $!!P \sim !P$.
\end{remark}

\begin{remark}\label{rem:paradoxical_combinator}
  The reader familiar with the lambda calculus will have noticed the
  similarity between $D$ and the paradoxical combinator.

  [Ed. note: the existence of this seems to suggest we have to be more
  restrictive on the set of processes and names we admit if we are to
  support no-cloning.]
\end{remark}

\subsubsection{Bisimulation}

The computational dynamics gives rise to another kind of equivalence,
the equivalence of computational behavior. As previously mentioned
this is typically captured \emph{via} some form of bisimulation.

% The notion we use in this paper is weak barbed bisimulation
% \cite{milner91polyadicpi}.

The notion we use in this paper is derived from weak barbed
bisimulation \cite{milner91polyadicpi}. 

\begin{definition}
An \emph{observation relation}, $\downarrow_{\mathcal N}$, over a set
of names, $\mathcal N$, is the smallest relation satisfying the rules
below.

\infrule[Out-barb]{y \in {\mathcal N}, \; x \nameeq y}
		  {\outputp{x}{v} \downarrow_{\mathcal N} x}
\infrule[Par-barb]{\mbox{$P\downarrow_{\mathcal N} x$ or $Q\downarrow_{\mathcal N} x$}}
		  {\binpar{P}{Q} \downarrow_{\mathcal N} x}

We write $P \Downarrow_{\mathcal N} x$ if there is $Q$ such that 
$P \wred Q$ and $Q \downarrow_{\mathcal N} x$.
\end{definition}

\begin{definition}
%\label{def.bbisim}
An  ${\mathcal N}$-\emph{barbed bisimulation} over a set of names, ${\mathcal N}$, is a symmetric binary relation 
${\mathcal S}_{\mathcal N}$ between agents such that $P\rel{S}_{\mathcal N}Q$ implies:
\begin{enumerate}
\item If $P \red P'$ then $Q \wred Q'$ and $P'\rel{S}_{\mathcal N} Q'$.
\item If $P\downarrow_{\mathcal N} x$, then $Q\Downarrow_{\mathcal N} x$.
\end{enumerate}
$P$ is ${\mathcal N}$-barbed bisimilar to $Q$, written
$P \wbbisim_{\mathcal N} Q$, if $P \rel{S}_{\mathcal N} Q$ for some ${\mathcal N}$-barbed bisimulation ${\mathcal S}_{\mathcal N}$.
\end{definition}

$\mathcal{R} \subseteq \pi \times \pi$

$P \mathcal{R} Q => \forall P'. P \red P' \Rightarrow \exists Q'. Q \red Q', P' \mathcal{R} Q'$

$P \vdash x \Rightarrow Q \vdash x$

\begin{mathpar}
  \inferrule*[lab=Out-barb]{x \nameeq y}{{y}!\langle{Q}\rangle \vdash x}
  \and
  \inferrule*[lab=Par-barb]{\mbox{$P\vdash x$ or $Q\vdash x$}}{\binpar{P}{Q} \vdash x}
\end{mathpar}

\subsubsection{Contexts}

One of the principle advantages of computational calculi like the
$\pi$-calculus is a well-defined notion of context,
contextual-equivalence and a correlation between
contextual-equivalence and notions of bisimulation. The notion of
context allows the decomposition of a process into (sub-)process and
its syntactic environment, its context. Thus, a context may be
thought of as a process with a ``hole'' (written $\Box$) in it. The
application of a context $M$ to a process $P$, written $M[P]$, is
tantamount to filling the hole in $M$ with $P$. In this paper we do
not need the full weight of this theory, but do make use of the notion
of context in the proof the main theorem. 

\begin{mathpar}
  \inferrule* [lab=summation] {} {{M_{M},M_{N}} \bc \Box \;|\; x.M_{A} \;|\; M_{M}+M_{N}}
  \and
  \inferrule* [lab=agent] {} {{M_{A}} \bc (\vec{x})M_{P} \;| \; \clift{P_0,\ldots,M_{P},\ldots,P_N}}
  \and \\
  \inferrule* [lab=process] {} {{M_{P}} \bc M_{N} \;| \;P|M_{P} }
\end{mathpar} 

\begin{mathpar}
  \inferrule* [lab=sychronization] {} {M_{N} \bc \Box \;|\; x?M_{F} \;|\; x!M_{C}}
  \and
  \inferrule* [lab=abstraction] {} {{M_{F}} \bc (x)M_{P} }
  \and
  \inferrule* [lab=concretion] {} {{M_{C}} \bc \langle M_{P} \rangle }
  \and \\
  \inferrule* [lab=process] {} {{M_{P}} \bc M_{N} \;| \;P|M_{P} }
\end{mathpar}

\begin{definition}[contextual application] Given a context $M$, and
  process $P$, we define the \emph{contextual application}, $M[P] :=
  M\{P/\Box\}$. That is, the contextual application of M to P is the
  substitution of $P$ for $\Box$ in $M$.
\end{definition}

$\meaningof{-} : L \to \mathcal{P}(\pi)$

\begin{mathpar}
  \inferrule* [lab=collection] {} {\meaningof{true} = \pi, \and \meaningof{~E} = \pi \setminus \meaningof{E}, \and \meaningof{E_{1} \& E_{2}} = \meaningof{E_{1}} \cap \meaningof{E_{2}}}
\end{mathpar}

\begin{mathpar}
  \inferrule* [lab=structure] {} {\meaningof{0} = \{ P \in \pi | P \equiv 0 \}, \and \\ \meaningof{E_1 | E_2} = \{ P \in \pi | P \equiv P_{1} | P_{2}, P_{1} \in \meaningof{E_{1}}, P_{2} \in \meaningof{E_2}\} }
\end{mathpar}

\begin{mathpar}
 \inferrule* [lab=behavior] {} {\meaningof{\langle a?b \rangle E} = \{ P \in \pi | P \equiv Q | u?(y)P', \\ \and \\\\ \and \\ \;\;\; u \in \meaningof{a}, \forall z.P'\{z/y\} \in \meaningof{E\{z/b\}}\}, \and \\ \meaningof{a!E} = \{ P \in \pi | P \equiv Q | x!\langle P' \rangle, x \in \meaningof{a} P' \in \meaningof{E}\} }
\end{mathpar}

\begin{mathpar}
 \inferrule* [lab=nominal] {} {\meaningof{\quotep{E}} = \{ \quotep{P} \in \quotep{\pi} | P \in \meaningof{E} \}, \and \meaningof{\quotep{P}} = \{ \quotep{Q} \in \quotep{\pi} | P \equiv Q \} \and \\ \meaningof{@\quotep{E}} = \{ P \in \pi | P \equiv @x, x \in \meaningof{E} \}}
\end{mathpar}

\begin{eqnarray*}
  \\
  \meaningof{-} : TS \to ST
\end{eqnarray*}

\begin{eqnarray*}
  \\
  L : TS \to ST
\end{eqnarray*}

\begin{eqnarray*}
  \\
  P \models E \iff P \in \meaningof{E}
\end{eqnarray*}

\begin{eqnarray*}
  P \approx_{L} Q \iff \forall E \in L. P \models E \iff Q \models E
\end{eqnarray*}

\begin{eqnarray*}
  P \approx_{K} Q
\end{eqnarray*}

\begin{eqnarray*}
  P \approx Q
\end{eqnarray*}

$\approx_{K} = \approx = \approx_{L}$

\subsubsection{Contextual duality}

Note that contexts extend the quotation operation to a family of
operations from processes to names. Given a context, $M$, we can
define a \emph{nominal context}, $\quotep{M}$ by $\quotep{M}[P] :=
\quotep{M[P]}$. To foreshadow what is to come we observe that these
operations enjoy a duality with processes very much like the duality
between vectors and maps from vectors to scalars.

Further, because the calculus is essentially higher-order, we have a
correspondence between contexts and processes. More specifically,
given a name $x$ and a context $M$ we can construct $M^{*}_{x}$ such
that 

\begin{mathpar}
  M^{*}_{x} | \lift{x}{P} \red M[P]
\end{mathpar}

namely,

\begin{mathpar}
  M^{*}_{x} := x?(u).M[\dropn{u}]
\end{mathpar}

The dependence of $M^{*}_{x}$ on a name makes it an abstraction, 

\begin{mathpar}
  M^{*} := (x)x?(u).M[\dropn{u}]
\end{mathpar}

\subsection{Additional notation}

It will sometimes be convenient to denote the process a name
quotes. We already have the notation $x = \quotep{P}$, but it will be
convenient to introduce an alternate notation, $\procn{x}$, when we
want to emphasize the connection to the use of the name. Note that, by
virtue of name equivalence, $\quotep{\procn{x}} \nameeq x$; so, the
notation is consistent with previous definitions.

Further, because names have structure it is possible to effect
substitutions on the basis of that structure. This means we need to
upgrade our notation for substitutions, which we accomplish by
adapting comprehension notation. Thus,

\begin{mathpar}
  P\{ y / x : x \in S \}
\end{mathpar}

is interpreted to mean the process derived from P by replacing (in a
capture-avoiding manner) each occurrence of $x$ in $S$ by $y$. For example,

\begin{mathpar}
  P\{ \quotep{\procn{x}|\procn{x}} / x : x \in \freenames{P} \}
\end{mathpar}

will replace each (occurrence) of a free name $x$ in $P$ by
$\quotep{\procn{x}|\procn{x}}$.

Also, we will avail ourselves of the notation $x^{L}$ and $x^{R}$ to
denote injections of a name into disjoint copies of the name
space. There are numerous ways to accomplish this. One example can be
found in \cite{MeredithR05}. This notation overloads to vectors of
names: $\vec{x}^{\pi} := (x_{i}^{\pi} \; : \; 0 \leq i < |\vec{x}| )$ where $\pi \in \{L,R\}$.

We also use $P^{\Box} := P|\Box$.

In \cite{MeredithR05} an interpretation of the new operator is
given. It turns out that there are several possible interpretations
all enjoying the requisite algebraic properties of the operator (see
\cite{milner91polyadicpi}). We will therefore make liberal use of
$(\nu\; \vec{x})P$.

% subsection the_syntax_and_semantics_of_the_notation_system (end)   

\input{qm2pi.qmops} 

\input{qm2pi.sterngerlach} 

\input{qm2pi.metric} 

% section concurrent_process_calculi (end)

%\input{qm2pi.proofsketch}

% section proof sketch (end)

%\input{qm2pi.slviaknots} 

% section spatial logic via knots (end)

\input{qm2pi.conclusion}

% section conclusion (end)

%\input{qm2pi.dtcodes} 

% section wiring algorithm (end)

\input{qm2pi.ack} 

% section acknowledgments (end)

\newpage


\bibliographystyle{plain}   
\bibliography{../../biblios/main.bib}

\input{qm2pi.rhodetails}

\end{document}

 

% section concurrent_process_calculi (end)

%\documentclass[12pt]{llncs}
%\documentclass{jktr}

\usepackage[pdftex]{hyperref}                   
\usepackage {listings}
\usepackage {mathpartir}
\usepackage{bcprules}
%\usepackage{listings}
                       
\usepackage{graphicx} 
%\usepackage[margins=2.5cm,nohead,nofoot]{geometry}
%\usepackage{geometry}
\usepackage{amsfonts}
\usepackage{amstext}
\usepackage{latexsym}
\usepackage{amssymb}
\usepackage{color}


%\include{myPreamble}
\include{qm2pi.local} 

%\ifpdf
%\usepackage[pdftex]{graphicx}
%\else
%\usepackage{graphicx}
%\fi

 % \ifpdf
%  \usepackage{pdfsync}
%  \if


%\title{Brief Article}
%\author{David F. Snyder}
%\author{L.G. Meredith}

%\address{Dept. of Math., Texas State University--San Marcos, San Marcos, TX 78666}
       
\pagestyle{empty}


\begin{document}

\lstset{language=[Objective]Caml,frame=shadowbox}

\input{qm2pi.front}

% section front matter (end)

\input{qm2pi.intro} 
 
% section introduction (end)

% \input{qm2pi.knotations} 

% section notation (end)

\input{qm2pi.process.calculi} 

% section concurrent_process_calculi_and_spatial_logics_ (end)
    
%\input{qm2pi.knots2pi} 

%\input{qm2pi.trefoil} 

%\input{qm2pi.mainthm} 

% subsection basic_interpretation (end)

%\input{qm2pi.rho.presentation} 
\subsection{The syntax and semantics of the notation system}\label{sub:the_syntax_and_semantics_of_the_notation_system} % (fold)

We now summarize a technical presentation of the calculus that
embodies our theory of dynamics. The typical presentation of such a
calculus follows the style of giving generators and relations on
them. The grammar, below, describing term constructors, freely
generates the set of processes, $\Proc$. This set is then quotiented
by a relation known as structural congruence and it is over this set
that the notion of dynamics is expressed. This presentation is
essentially that of \cite{MeredithR05} with the addition of
polyadicity and summation. For readability we have relegated some of
the technical subtleties to an appendix.

\subsubsection{Process grammar}\label{subsub:process_grammar}

\begin{mathpar}
  \inferrule* [lab=synchronization] {} {{M} \bc \pzero \;|\; x?F \;|\; x!C }
  \and
  \inferrule* [lab=abstraction] {} {{F} \bc (x)P}
  \and
  \inferrule* [lab=concretion] {} {{C} \bc \langle Q \rangle}
  \and
  \inferrule* [lab=process] {} {{P,Q} \bc M \;| \;P|Q \;|\; @{x}}
  \and
  \inferrule* [lab=name] {} {{x} \bc \quotep{P}}
\end{mathpar} 

Note that $\vec{x}$ (resp. $\vec{P}$) denotes a vector of names
(resp. processes) of length $|\vec{x}|$ (resp. $|\vec{P}|$). We adopt
the following useful abbreviations.

\begin{mathpar}
   x?(\vec{y}).P := x.(\vec{y})P \and  x\clift{\vec{P}} := x.\clift{\vec{P}}
   \and x!(y) := \lift{x}{\dropn{y}}
   \and \Pi_{i=0}^{n-1}P_i := P_0 | \ldots | P_{n-1}
\end{mathpar}

\subsubsection{Structural congruence}

\paragraph{Free and bound names and alpha-equivalence.} At the
core of structural equivalence is alpha-equivalence which identifies
process that are the same up to a change of variable. Formally, we
recognize the distinction between free and bound names. The free names
of a process, $\freenames{P}$, may be calculated recursively as
follows:

\begin{mathpar}
\freenames{\pzero} := \emptyset
  \and \\
  \freenames{x?(y).P} := \{ x \} \cup (\freenames{P} \setminus \{ y \})
  \and 
  \freenames{x!\langle P \rangle} := \{ x \} \cup \{ P \} 
  \and \\
  \freenames{P|Q} := \freenames{P} \cup \freenames{Q}
  \and \\
  \freenames{@{x}} := \{ x \}
\end{mathpar}

$\pi$
$\quotep{\pi}$

$\freenames{-} : \pi \to \mathcal{P}(\quotep{\pi})$

\begin{eqnarray*}
  \freenames{\pzero} & := & \emptyset \\
  \freenames{x?(y).P} & := & \{ x \} \cup (\freenames{P} \setminus \{ y \}) \\
  \freenames{x!\langle P \rangle} & := & \{ x \} \cup \{ P \} \\
  \freenames{P|Q} & := & \freenames{P} \cup \freenames{Q} \\
  \freenames{\dropn{x}} & := & \{ x \}
\end{eqnarray*}

The bound names of a process, $\boundnames{P}$, are those names occurring in $P$
that are not free. For example, in $x?(y).0$, the name $x$ is free, while $y$ is bound.

\begin{mathpar}
  \inferrule* [lab=monoidal-laws] {} { P|Q \equiv Q|P \and P|0 \equiv P \and P|(Q|R) \equiv (P|Q)|R }
\end{mathpar}

\begin{mathpar}
  \inferrule* [lab=alpha-equivalence] {} { (x)P \equiv (y)P\{y/x\} \and y \not\in \freenames{P} }
\end{mathpar}

\begin{definition}
Then two processes, $P,Q$, are alpha-equivalent if $P = Q\{\vec{y}/\vec{x}\}$ for
some $\vec{x} \in \boundnames{Q},\vec{y} \in \boundnames{P}$, where $Q\{\vec{y}/\vec{x}\}$
denotes the capture-avoiding substitution of $\vec{y}$ for $\vec{x}$ in $Q$.
\end{definition}

\begin{definition}
  The {\em structural congruence} \cite{SangiorgiWalker} , $\equiv$,
  between processes is the least congruence containing
  alpha-equivalence, satisfying the abelian monoid laws
  (associativity, commutativity and $\pzero$ as identity) for parallel
  composition $|$ and for summation $+$.
\end{definition}

\subsection{Name equivalence}

We take name equivalence, written $\nameeq$, to be the smallest
equivalence relation generated by the following rules.

\begin{mathpar}
\inferrule*[lab=Quote-drop]
{ }
{ \quotep{@{x}} \nameeq x }

\inferrule*[lab=Struct-equiv]
{ P \scong Q }
{ \quotep{P} \nameeq \quotep{Q} }
\end{mathpar}

The astute reader will have noticed that the mutual recursion of names
and processes imposes a mutual recursion on alpha-equivalence and
structural equivalence via name-equivalence. Fortunately, all of this
works out pleasantly and we may calculate in the natural way, free of
concern. The reader interested in the details is referred to the
appendix \ref{appendix:rho_details}.

\subsection{Substitution}

We use $\Proc$ for the set of processes, $\QProc$ for the set of
names, and $\id{\{}\vec{y} / \vec{x} \id{\}}$ to denote partial maps,
$s : \QProc \rightarrow \QProc$. A map, $s$ lifts, uniquely, to a map
on process terms, $\widehat{s} : \Proc \rightarrow \Proc$ by the
following equations.

\begin{mathpar}
  (0) \psubstp{Q}{P} := 0 \\
  (R \juxtap S) \psubstp{Q}{P}
  :=    
  (R)\psubstp{Q}{P} \juxtap (S) \psubstp{Q}{P} \\
  (x?(y).R) \psubstp{Q}{P}    
  :=    
  (x)\substp{Q}{P} (z)\concat( (R \psubstn{z}{y}) \psubstp{Q}{P} ) \\
  (\lift{x}{R}) \psubstp{Q}{P}  
  :=
  \lift{(x)\substp{Q}{P}}{ R \psubstp{Q}{P} } \\
%   (\dropn{x})  \psubstp{Q}{P}       
%   := 
%   \left\{ 
%     \begin{array}{ccc} 
%       \dropn{\quotep{Q}} & & x \nameeq \quotep{P} \\
%       \dropn{x} & & otherwise \\
%     \end{array}
%   \right. 
  (\dropn{x})  \psubstp{Q}{P}       
  := 
  \left\{ 
    \begin{array}{ccc} 
      Q & & x \nameeq \quotep{P} \\
      \dropn{x} & & otherwise \\
    \end{array}
  \right.
\end{mathpar}
 

where

\begin{eqnarray}
  (x)\id{\{} \lpquote Q \rpquote / \lpquote P \rpquote \id{\}}            = 
  \left\{ 
    \begin{array}{ccc}
      \lpquote Q \rpquote & & x \nameeq \lpquote P \rpquote \\
      x & & otherwise \\
    \end{array}
  \right. \nonumber
\end{eqnarray}

and $z$ is chosen distinct from $\quotep{P}$, $\quotep{Q}$, the free
names in $Q$, and all the names in $R$. Our $\alpha$-equivalence will
be built in the standard way from this substitution.

\begin{remark}\label{rem:no_self_referential_names}
  One consequence of these definitions is that $\forall P. \quotep{P}
  \not\in \freenames{P}$.
\end{remark}

\subsection{ Dynamic quote: an example }

Anticipating something of what's to come, consider applying the
substitution, $\widehat{\id{\{}u / z \id{\}}}$, to the following pair
of processes, $\lift{w}{y!(z)}$ and $w[ \lpquote y!(z) \rpquote ]$.

\begin{eqnarray}
	\lift{w}{y!(z)}\widehat{\id{\{}u / z \id{\}}}
		& = &
		\lift{w}{y!(u)} \nonumber\\
	w[ \lpquote y!(z) \rpquote ] \widehat{ \id{\{}u / z \id{\}} }
		& = &
		w[ \lpquote y!(z) \rpquote ] \nonumber
\end{eqnarray}

Because the body of the process between quotes is impervious to
substitution, we get radically different answers. In fact, by
examining the first process in an input context,
e.g. $x?(z).\lift{w}{y!(z)}$, we see that the process under the lift
operator may be shaped by prefixed inputs binding a name inside it. In
this sense, the lift operator will be seen as a way to dynamically
construct processes before reifying them as names.

Finally equipped with these standard features we can present the
dynamics of the calculus.

\subsubsection{Operational semantics} 

Finally, we introduce the computational dynamics. What marks these
algebras as distinct from other more traditionally studied algebraic
structures, e.g. vector spaces or polynomial rings, is the manner in
which dynamics is captured. In traditional structures, dynamics is typically
expressed through morphisms between such structures, as in linear maps
between vector spaces or morphisms between rings. In algebras
associated with the semantics of computation, the dynamics is
expressed as part of the algebraic structure itself, through a
reduction reduction relation typically denoted by $\red$. Below, we
give a recursive presentation of this relation for the calculus used
in the encoding.

$\red \subseteq \pi \times \pi$
$\red : \pi \to \mathcal{P}(\pi)$

\begin{mathpar}
  \inferrule* [lab=Comm] { \textsf{match}( x_{src}, x_{trgt} ) } { x_{trgt}?(y)P \; | \; x_{src}!\langle {Q} \rangle \red P\{\quotep{Q}/y}\} }
  \and \\
  \inferrule* [lab=Par] {{P} \red {P}'} {{{P} | {Q}} \red {{P}' | {Q}}}
  \and
  \inferrule* [lab=Equiv]{{{P} \scong {P}'} \andalso {{P}' \red {Q}'} \andalso {{Q}' \scong {Q}}}{{P} \red {Q}}
\end{mathpar}

\begin{eqnarray*}
  match_{\equiv} (\quotep{P},\quotep{Q}) & := & P \equiv Q \\
  match_{\dagger}(\quotep{P},\quotep{Q}) & := & \forall R. P|Q \red^{*} R => R \red^{*} 0 \\
  match_{K}(\quotep{P},\quotep{Q}) & := & K \mbox{ for some context } K
\end{eqnarray*}

$u?(x)P | u!\langle Q \rangle \red P\{\quotep{Q}/x\}$

%We write $\wred$ for $\red^*$, and $P\red$ if $\exists Q $ such that $ P \red Q$.
We write $P\red$ if $\exists Q $ such that $ P \red Q$ and $P\not\red$, otherwise.

\section{Replication}

As mentioned before, it is known that replication (and hence
recursion) can be implemented in a higher-order process algebra
\cite{SangiorgiWalker}. As our first example of calculation with the
machinery thus far presented we give the construction explicitly in
the {\rhoc}.

\begin{eqnarray}
	D_{x} & := & \prefix{x}{y}{(\binpar{\outputp{x}{y}}{@{y}})} \nonumber\\
	\bangp_{x}{P} & := & \binpar{{x}!\langle{\binpar{D_{x}}{P}}\rangle}{D_{x}} \nonumber
\end{eqnarray}

\begin{eqnarray}
	\bangp_{x}{P} & & \nonumber\\
	=
	& {x}!\langle{(\prefix{x}{y}{(\outputp{x}{y} | @{y})) | P}}\rangle 
	      | \prefix{x}{y}{(\outputp{x}{y} | @{y})} & \nonumber\\
	\red
	& (\outputp{x}{y} | @{y})\substn{\quotep{(\prefix{x}{y}{(@{y} | \outputp{x}{y})) | P}}}{y} & \nonumber\\
	=
	& \outputp{x}{\quotep{(\prefix{x}{y}{(\outputp{x}{y} | @{y})) | P}}}
	  | {(\prefix{x}{y}{(\outputp{x}{y} | @{y})) | P}} & \nonumber\\
	\red
	& \ldots & \nonumber\\
	\red^*
	& P | P | \ldots & \nonumber
\end{eqnarray}

Of course, this encoding, as an implementation, runs away, unfolding
$\bangp{P}$ eagerly. A lazier and more implementable replication
operator, restricted to input-guarded processes, may be obtained as follows.

\begin{eqnarray}
\bangp{\prefix{u}{v}{P}} 
	:= 
	\binpar{\lift{x}{\prefix{u}{v}{(\binpar{D(x)}{P})}}}{D(x)} \nonumber
\end{eqnarray}

\begin{remark}
  Note that the lazier definition still does not deal with summation
  or mixed summation (i.e. sums over input and output). The reader is
  invited to construct definitions of replication that deal with these
  features. 

  Further, the definitions are parameterized in a name, $x$. Can you,
  gentle reader, make a definition that eliminates this parameter and
  guarantees no accidental interaction between the replication
  machinery and the process being replicated -- i.e. no accidental
  sharing of names used by the process to get its work done and the
  name(s) used by the replication to effect copying. This latter
  revision of the definition of replication is crucial to obtaining
  the expected identity $!!P \sim !P$.
\end{remark}

\begin{remark}\label{rem:paradoxical_combinator}
  The reader familiar with the lambda calculus will have noticed the
  similarity between $D$ and the paradoxical combinator.

  [Ed. note: the existence of this seems to suggest we have to be more
  restrictive on the set of processes and names we admit if we are to
  support no-cloning.]
\end{remark}

\subsubsection{Bisimulation}

The computational dynamics gives rise to another kind of equivalence,
the equivalence of computational behavior. As previously mentioned
this is typically captured \emph{via} some form of bisimulation.

% The notion we use in this paper is weak barbed bisimulation
% \cite{milner91polyadicpi}.

The notion we use in this paper is derived from weak barbed
bisimulation \cite{milner91polyadicpi}. 

\begin{definition}
An \emph{observation relation}, $\downarrow_{\mathcal N}$, over a set
of names, $\mathcal N$, is the smallest relation satisfying the rules
below.

\infrule[Out-barb]{y \in {\mathcal N}, \; x \nameeq y}
		  {\outputp{x}{v} \downarrow_{\mathcal N} x}
\infrule[Par-barb]{\mbox{$P\downarrow_{\mathcal N} x$ or $Q\downarrow_{\mathcal N} x$}}
		  {\binpar{P}{Q} \downarrow_{\mathcal N} x}

We write $P \Downarrow_{\mathcal N} x$ if there is $Q$ such that 
$P \wred Q$ and $Q \downarrow_{\mathcal N} x$.
\end{definition}

\begin{definition}
%\label{def.bbisim}
An  ${\mathcal N}$-\emph{barbed bisimulation} over a set of names, ${\mathcal N}$, is a symmetric binary relation 
${\mathcal S}_{\mathcal N}$ between agents such that $P\rel{S}_{\mathcal N}Q$ implies:
\begin{enumerate}
\item If $P \red P'$ then $Q \wred Q'$ and $P'\rel{S}_{\mathcal N} Q'$.
\item If $P\downarrow_{\mathcal N} x$, then $Q\Downarrow_{\mathcal N} x$.
\end{enumerate}
$P$ is ${\mathcal N}$-barbed bisimilar to $Q$, written
$P \wbbisim_{\mathcal N} Q$, if $P \rel{S}_{\mathcal N} Q$ for some ${\mathcal N}$-barbed bisimulation ${\mathcal S}_{\mathcal N}$.
\end{definition}

$\mathcal{R} \subseteq \pi \times \pi$

$P \mathcal{R} Q => \forall P'. P \red P' \Rightarrow \exists Q'. Q \red Q', P' \mathcal{R} Q'$

$P \vdash x \Rightarrow Q \vdash x$

\begin{mathpar}
  \inferrule*[lab=Out-barb]{x \nameeq y}{{y}!\langle{Q}\rangle \vdash x}
  \and
  \inferrule*[lab=Par-barb]{\mbox{$P\vdash x$ or $Q\vdash x$}}{\binpar{P}{Q} \vdash x}
\end{mathpar}

\subsubsection{Contexts}

One of the principle advantages of computational calculi like the
$\pi$-calculus is a well-defined notion of context,
contextual-equivalence and a correlation between
contextual-equivalence and notions of bisimulation. The notion of
context allows the decomposition of a process into (sub-)process and
its syntactic environment, its context. Thus, a context may be
thought of as a process with a ``hole'' (written $\Box$) in it. The
application of a context $M$ to a process $P$, written $M[P]$, is
tantamount to filling the hole in $M$ with $P$. In this paper we do
not need the full weight of this theory, but do make use of the notion
of context in the proof the main theorem. 

\begin{mathpar}
  \inferrule* [lab=summation] {} {{M_{M},M_{N}} \bc \Box \;|\; x.M_{A} \;|\; M_{M}+M_{N}}
  \and
  \inferrule* [lab=agent] {} {{M_{A}} \bc (\vec{x})M_{P} \;| \; \clift{P_0,\ldots,M_{P},\ldots,P_N}}
  \and \\
  \inferrule* [lab=process] {} {{M_{P}} \bc M_{N} \;| \;P|M_{P} }
\end{mathpar} 

\begin{mathpar}
  \inferrule* [lab=sychronization] {} {M_{N} \bc \Box \;|\; x?M_{F} \;|\; x!M_{C}}
  \and
  \inferrule* [lab=abstraction] {} {{M_{F}} \bc (x)M_{P} }
  \and
  \inferrule* [lab=concretion] {} {{M_{C}} \bc \langle M_{P} \rangle }
  \and \\
  \inferrule* [lab=process] {} {{M_{P}} \bc M_{N} \;| \;P|M_{P} }
\end{mathpar}

\begin{definition}[contextual application] Given a context $M$, and
  process $P$, we define the \emph{contextual application}, $M[P] :=
  M\{P/\Box\}$. That is, the contextual application of M to P is the
  substitution of $P$ for $\Box$ in $M$.
\end{definition}

$\meaningof{-} : L \to \mathcal{P}(\pi)$

\begin{mathpar}
  \inferrule* [lab=collection] {} {\meaningof{true} = \pi, \and \meaningof{~E} = \pi \setminus \meaningof{E}, \and \meaningof{E_{1} \& E_{2}} = \meaningof{E_{1}} \cap \meaningof{E_{2}}}
\end{mathpar}

\begin{mathpar}
  \inferrule* [lab=structure] {} {\meaningof{0} = \{ P \in \pi | P \equiv 0 \}, \and \\ \meaningof{E_1 | E_2} = \{ P \in \pi | P \equiv P_{1} | P_{2}, P_{1} \in \meaningof{E_{1}}, P_{2} \in \meaningof{E_2}\} }
\end{mathpar}

\begin{mathpar}
 \inferrule* [lab=behavior] {} {\meaningof{\langle a?b \rangle E} = \{ P \in \pi | P \equiv Q | u?(y)P', \\ \and \\\\ \and \\ \;\;\; u \in \meaningof{a}, \forall z.P'\{z/y\} \in \meaningof{E\{z/b\}}\}, \and \\ \meaningof{a!E} = \{ P \in \pi | P \equiv Q | x!\langle P' \rangle, x \in \meaningof{a} P' \in \meaningof{E}\} }
\end{mathpar}

\begin{mathpar}
 \inferrule* [lab=nominal] {} {\meaningof{\quotep{E}} = \{ \quotep{P} \in \quotep{\pi} | P \in \meaningof{E} \}, \and \meaningof{\quotep{P}} = \{ \quotep{Q} \in \quotep{\pi} | P \equiv Q \} \and \\ \meaningof{@\quotep{E}} = \{ P \in \pi | P \equiv @x, x \in \meaningof{E} \}}
\end{mathpar}

\begin{eqnarray*}
  \\
  \meaningof{-} : TS \to ST
\end{eqnarray*}

\begin{eqnarray*}
  \\
  L : TS \to ST
\end{eqnarray*}

\begin{eqnarray*}
  \\
  P \models E \iff P \in \meaningof{E}
\end{eqnarray*}

\begin{eqnarray*}
  P \approx_{L} Q \iff \forall E \in L. P \models E \iff Q \models E
\end{eqnarray*}

\begin{eqnarray*}
  P \approx_{K} Q
\end{eqnarray*}

\begin{eqnarray*}
  P \approx Q
\end{eqnarray*}

$\approx_{K} = \approx = \approx_{L}$

\subsubsection{Contextual duality}

Note that contexts extend the quotation operation to a family of
operations from processes to names. Given a context, $M$, we can
define a \emph{nominal context}, $\quotep{M}$ by $\quotep{M}[P] :=
\quotep{M[P]}$. To foreshadow what is to come we observe that these
operations enjoy a duality with processes very much like the duality
between vectors and maps from vectors to scalars.

Further, because the calculus is essentially higher-order, we have a
correspondence between contexts and processes. More specifically,
given a name $x$ and a context $M$ we can construct $M^{*}_{x}$ such
that 

\begin{mathpar}
  M^{*}_{x} | \lift{x}{P} \red M[P]
\end{mathpar}

namely,

\begin{mathpar}
  M^{*}_{x} := x?(u).M[\dropn{u}]
\end{mathpar}

The dependence of $M^{*}_{x}$ on a name makes it an abstraction, 

\begin{mathpar}
  M^{*} := (x)x?(u).M[\dropn{u}]
\end{mathpar}

\subsection{Additional notation}

It will sometimes be convenient to denote the process a name
quotes. We already have the notation $x = \quotep{P}$, but it will be
convenient to introduce an alternate notation, $\procn{x}$, when we
want to emphasize the connection to the use of the name. Note that, by
virtue of name equivalence, $\quotep{\procn{x}} \nameeq x$; so, the
notation is consistent with previous definitions.

Further, because names have structure it is possible to effect
substitutions on the basis of that structure. This means we need to
upgrade our notation for substitutions, which we accomplish by
adapting comprehension notation. Thus,

\begin{mathpar}
  P\{ y / x : x \in S \}
\end{mathpar}

is interpreted to mean the process derived from P by replacing (in a
capture-avoiding manner) each occurrence of $x$ in $S$ by $y$. For example,

\begin{mathpar}
  P\{ \quotep{\procn{x}|\procn{x}} / x : x \in \freenames{P} \}
\end{mathpar}

will replace each (occurrence) of a free name $x$ in $P$ by
$\quotep{\procn{x}|\procn{x}}$.

Also, we will avail ourselves of the notation $x^{L}$ and $x^{R}$ to
denote injections of a name into disjoint copies of the name
space. There are numerous ways to accomplish this. One example can be
found in \cite{MeredithR05}. This notation overloads to vectors of
names: $\vec{x}^{\pi} := (x_{i}^{\pi} \; : \; 0 \leq i < |\vec{x}| )$ where $\pi \in \{L,R\}$.

We also use $P^{\Box} := P|\Box$.

In \cite{MeredithR05} an interpretation of the new operator is
given. It turns out that there are several possible interpretations
all enjoying the requisite algebraic properties of the operator (see
\cite{milner91polyadicpi}). We will therefore make liberal use of
$(\nu\; \vec{x})P$.

% subsection the_syntax_and_semantics_of_the_notation_system (end)   

\input{qm2pi.qmops} 

\input{qm2pi.sterngerlach} 

\input{qm2pi.metric} 

% section concurrent_process_calculi (end)

%\input{qm2pi.proofsketch}

% section proof sketch (end)

%\input{qm2pi.slviaknots} 

% section spatial logic via knots (end)

\input{qm2pi.conclusion}

% section conclusion (end)

%\input{qm2pi.dtcodes} 

% section wiring algorithm (end)

\input{qm2pi.ack} 

% section acknowledgments (end)

\newpage


\bibliographystyle{plain}   
\bibliography{../../biblios/main.bib}

\input{qm2pi.rhodetails}

\end{document}



% section proof sketch (end)

%\section{Unlikely characters: spatial logic for
  knots}\label{sub:characteristic_formulae} % (fold)

Associated to the mobile process calculi are a family of logics known
as the Hennessy-Milner logics. These logics typically enjoy a
semantics interpreting formulae as sets of processes that when
factored through the encoding outlined above allows an identification
of classes of knots with logical formulae. In the context of this
encoding the sub-family known as the spatial logics \cite{CairesC03}
\cite{CairesC04} \cite{Caires04} are of particular interest providing
several important features for expressing and reasoning about
properties (i.e. classes) of knots. We hint here at how this may be done.

%\begin{description}
%\item [structural connectives] 
\subsubsection{Structural connectives} The spatial logics enjoy
structural connectives corresponding, at the logical level, to the
parallel composition ($P | Q$) and new name ($(\nu \; x)P$)
connectives for processes. As illustrated in the examples below, these
connectives are extremely expressive given the shape of our encoding.
%\item [decideable satisfaction]

\subsubsection{Decideable satisfaction}
In \cite{Caires04} the satisfaction relation is shown to be decideable
for a rich class of processes. It further turns out that the image of
the our encoding is a proper subset of that class. This result
provides the basis for an algorithm by which to search for knots
enjoying a given property.
%\item [characteristic formulae]

\subsubsection{Characteristic formulae}
In the same paper \cite{Caires04} , Caires presents a means of calculating
characteristic formulae, selecting equivalence classes of processes
up to a pre--specified depth limit on the support set of names. Composed with our
encoding, this characteristic formula can be used to select
characteristic formulae for knots.
%\end{description}

\subsubsection{Spatial logic formulae}

The grammar below (segmented for comprehension) summarizes the syntax
of spatial logic formulae. We employ illustrative examples in the
sequel to provide an intuitive understanding of their meaning
referring the reader to \cite{Caires04} for a more detailed explication
of the semantics.

\begin{mathpar}
  \inferrule* [lab=boolean] {} {{A,B} \bc T \;|\; \neg A \;|\; A \wedge B \;|\; \eta = \eta'}
  \and
  \inferrule* [lab=spatial] {} {|\; \pzero \;|\; A | B \;|\; x \text{\textregistered} A \;|\; \forall x . A \;|\;  H x . A}
  \and
  \inferrule* [lab=behavioral] {} {|\; \alpha . A}
  \and 
  \inferrule* [lab=recursion] {} {|\; X(\vec{u}) \;|\; \mu X(\vec{u}) . A}
  \and
  \inferrule* [lab=action] {} {\alpha \bc \langle x?(\vec{y}) \rangle \;|\; \langle x!(\vec{y}) \rangle \;|\; \langle \tau \rangle}
  \and 
  \inferrule* [lab=name] {} {\eta \bc x \;|\; \tau}
\end{mathpar} 

% subsection characteristic_formulae (end)   	 

\subsection{Example formulae}\label{sub:example_formulae_} % (fold)

\subsubsection{Crossing as formula.}
% 
% \begin{align*}
%   \frac{d}{dx} \sin x &= \cos x 
%   & \frac{d}{dx} e^x &= e^x \\
%   \frac{d}{dx} \cos x &= - \sin x 
%   & \frac{d}{dx} \log x &= \frac{1}{x} \\
% \end{align*} 

\begin{align*}
 \mu C(x_{0},x_{1},y_{0},y_{1},u).&(\langle x_{0}?(z) \rangle(\langle u! \rangle\langle y_{1}!z \rangle C(x_{0},x_{1},y_{0},y_{1},u)) & \\
  & \wedge \langle y_{1}?(z) \rangle (\langle u! \rangle \langle x_{0}!z \rangle C(x_{0},x_{1},y_{0},y_{1},u)) & \\
  & \wedge \langle x_{1}?(z) \rangle (\langle u? \rangle \langle y_{0}!z \rangle C(x_{0},x_{1},y_{0},y_{1},u)) & \\
  & \wedge \langle y_{0}?(z) \rangle (\langle u? \rangle \langle x_{1}!z \rangle C(x_{0},x_{1},y_{0},y_{1},u))) &
\end{align*}

The lexicographical similarity between the shape of this formulae and
the shape of definition of the process representing a crossing reveals
the intuitive meaning of this formulae. It describes the capabilities
of a process that has the right to represent a crossing. For example
it picks out processes that may perform an input on the port $x_0$ in
its initial menu of capabilities. What differentiates the formula
from the process, however, is that the crossing process is the
smallest candidate to satisfy the formula. Infinitely many other
processes -- with internal behavior hidden behind this interface, so
to speak -- also satisfy this formula. Even this simple formula,
then, can be seen to open a new view onto knots, providing a
computational interpretation of \emph{virtual} knots.

Note that this formula is derived by hand. A similar formula can be
derived by employing Caires' calculation of characteristic formula
\cite{Caires04} to the process representing a crossing. In light of
this discussion, we let
$\meaningof{C}_{\phi}(x0,x1,y0,y1,u)$ denote a formula specifying the
dynamics we wish to capture of a crossing. To guarantee we preserve
the shape of the interface and minimal semantics we demand that
$\meaningof{C}_{\phi}(x0,x1,y0,y1,u) \Rightarrow
\textbf{C}(x0,x1,y0,y1,u)$ where $\textbf{C}(x0,x1,y0,y1,u)$ denotes
the formula above.
                            
\subsubsection{Crossing number constraints.}
The moral content of the context lemma (Lemma \ref{context}) is that the notion of
``locality'' in the Reidemeister moves is effectively captured by the
parallel composition operator of the process calculus. This intuition
extends through the logic. Given a formula,
$\meaningof{C}_{\phi}(x0,x1,y0,y1,u)$, we can use the structural
connectives to specify constraints on crossing numbers, such as at
least $n$ crossings, or exactly $n$ crossings.
\begin{mathpar}
  \inferrule* [lab=at-least-n] {} { K^{\geq n}_{\phi}(\vec{xs},\vec{ys}) := \Pi_{i=0}^{n-1} Hu . \meaningof{C}_{\phi}(xs_i,ys_i,u) | T }
  \and 
  \inferrule* [lab=exactly-n] {} { K^{= n}_{\phi}(\vec{xs},\vec{ys}) := \Pi_{i=0}^{n-1} Hu . \meaningof{C}_{\phi}(xs_i,ys_i,u) | \neg (\forall x_0,y_0,x_1,y_1,u . \meaningof{C}_{\phi}(x_0,y_0,x_1,y_1,u) | T) }
\end{mathpar}

To round out this section, recall that the encoding of an $n$-crossing
knot decomposes into a parallel composition of $n$ \emph{copies} of a
crossing process together with a wiring harness. To specify different
knot classes with the same crossing number amounts to specifying
logical constraints on the wiring harness. In the interest of space,
we defer examples to a forthcoming paper. Suffice it to say that both
the conditions ``alternating knot'' and ``contains the tangle
corresponding to 5/3'' are expressible. For example, it is possible to
calculate the characteristic formula of a process corresponding to the
tangle 5/3 and conjoin it into the classifying formula via the
composition connective of the logic.

Finally, we wish to observe that it is entirely within reason to
contemplate a more domain-specific version of spatial logic tailored
to the shape of processes in the image of the encoding. Such a
domain-specific logic would have a better claim to the title formal
language of knot properties.

% subsection example_formulae_ (end)

% section knots_as_processes (end) 

% section spatial logic via knots (end)

\section{Conclusions and future work}

\paragraph{Testing physical space}
You, gentle reader, may wonder why of all the theorems to be proved
given this set up we pick the one above. In some sense it's hardly
central to quantum mechanics. We see it as central in the sense that
it firmly establishes a notion of physical space arising from a notion
of the equivalence of behavior. Relating bisimulation to a metric is a
big step forward, but one is faced with interpreting the relationship
of that metric space to something more physical. Quantum mechanical
notions of ``physical'' space are still far from intuitive, but by
relating this idea of distance as testing to calculations that predict
physical circumstances we are making a not insignificant step forward
toward an understanding of the physical space we inhabit as
essentially dynamic.

\paragraph{Effectivity and simulation}
One of the observations we have yet to make is that the entire program
spelled out here is effective. We have built various interpreters for
the reflective calculus at work in this interpretation. In principle,
then, we can simulate quantum mechanics on a computer. The place where
the simulation may lose fidelity is the infinitely branching summation
for the annihilator.

In this connection i also want to point out that the evaluation style
calculation of the inner product puts the non-determinism of the
summation right at the heart of measurement. This suggests that
Milner's original reduction-based formulation of the dynamics of his
calculi in terms of sums was not just notationally suggestive of a
notion of measure-and-continue but captured some significant part of
the physics.

\paragraph{Quantum continuations}
In light of this last observation i want to point out that the
predominant account of quantum mechanics is missing a key aspect of a
truly compositional story of the physical situation. In a real lab,
when a measurement is made the observation can be made to feed into
another device that then makes another measurement conditioned on the
results of the first. This means that after the superposition was
collapsed the entire experimental set up remained in
superposition. While QM offers a means of writing this down it doesn't
quite line up well with the well-trodden formulation of computation
and continuation that we see so succinctly expressed in Milner's
calculi. This suggests that there might be advantages to this account
of dynamics waiting to be explored.

\paragraph{Quantum logic}
In this connection, we also note that by virtue of having the
Hennessy-Milner construction, we can pull the construction through the
interpretation of QM. This gives us a natural candidate for a quantum
logic that enjoys an extremely tight connection with it's domain of
interpretation, making the construction much less ad hoc (rather it is
the image of functor!).

\paragraph{Quantum probabiity}
i have questions about the basis of the interpretation of inner
product as probability amplitude. In particular, using which
axiomatization of probability theory does the notion of probability
amplitude earn the right to be so dubbed? In other words, where is the
proof that the operation for calculating a probability amplitude (and
then squaring) satisfies the axioms of what it means to calculate a
probability? Even if such a proof exists (i have yet to find it in the
literature), i wonder if it might not be possible to turn things on
their heads. Can we view the calculation of the probability amplitude
as an axiomatization of probability? If so, then the definition we
give for calculating probability amplitude may provide the basis for
an \emph{effective} theory of probability.

\paragraph{Quantum vs ``biological'' information}
Finally, i want to conclude with a more philosophical observation. At
a recent workshop in which QM was a predominant topic i noticed
something about quantum information. The speaker was giving a riveting
discussion of axiomatic QM and showing how properties of ``no
cloning'' and ``no deleting'' emerged as consequences of the
axiomatization. Theorems of this form are necessary to give us a sense
of confidence that our axioms characterize the physical theory. What
struck me, though, was that if quantum information is neither erasable
nor replicable it is markedly different from \emph{life}. Two of the
things we know about life is that

\begin{itemize}
  \item it ends;
  \item to gain some measure of persistence, to transcend it's
    finitude it is imminently copyable.
\end{itemize}

Both of these qualities are summarized succinctly in the aphorism: all
flesh is grass. For me these two kinds of ``information'' -- call them
quantum and biological -- are end points on a spectrum of strategies
for persistence. At one end, we have those curious entities that enjoy
uniqueness and permanence; at the other, we have those who in the face
of a certain end and an uncertain present make a go of passing
something on. To me one of the more remarkable aspects of the latter
strategy is that in the presence of noise (and certain features of
copying) we get a kind of dynamism, a chance for improvement against a
given persistent condition.

% subsection other_calculi_other_bisimulations_and_geometry_as_behavior (end)




% section conclusion (end)

%\documentclass[12pt]{llncs}
%\documentclass{jktr}

\usepackage[pdftex]{hyperref}                   
\usepackage {listings}
\usepackage {mathpartir}
\usepackage{bcprules}
%\usepackage{listings}
                       
\usepackage{graphicx} 
%\usepackage[margins=2.5cm,nohead,nofoot]{geometry}
%\usepackage{geometry}
\usepackage{amsfonts}
\usepackage{amstext}
\usepackage{latexsym}
\usepackage{amssymb}
\usepackage{color}


%\include{myPreamble}
\include{qm2pi.local} 

%\ifpdf
%\usepackage[pdftex]{graphicx}
%\else
%\usepackage{graphicx}
%\fi

 % \ifpdf
%  \usepackage{pdfsync}
%  \if


%\title{Brief Article}
%\author{David F. Snyder}
%\author{L.G. Meredith}

%\address{Dept. of Math., Texas State University--San Marcos, San Marcos, TX 78666}
       
\pagestyle{empty}


\begin{document}

\lstset{language=[Objective]Caml,frame=shadowbox}

\input{qm2pi.front}

% section front matter (end)

\input{qm2pi.intro} 
 
% section introduction (end)

% \input{qm2pi.knotations} 

% section notation (end)

\input{qm2pi.process.calculi} 

% section concurrent_process_calculi_and_spatial_logics_ (end)
    
%\input{qm2pi.knots2pi} 

%\input{qm2pi.trefoil} 

%\input{qm2pi.mainthm} 

% subsection basic_interpretation (end)

%\input{qm2pi.rho.presentation} 
\subsection{The syntax and semantics of the notation system}\label{sub:the_syntax_and_semantics_of_the_notation_system} % (fold)

We now summarize a technical presentation of the calculus that
embodies our theory of dynamics. The typical presentation of such a
calculus follows the style of giving generators and relations on
them. The grammar, below, describing term constructors, freely
generates the set of processes, $\Proc$. This set is then quotiented
by a relation known as structural congruence and it is over this set
that the notion of dynamics is expressed. This presentation is
essentially that of \cite{MeredithR05} with the addition of
polyadicity and summation. For readability we have relegated some of
the technical subtleties to an appendix.

\subsubsection{Process grammar}\label{subsub:process_grammar}

\begin{mathpar}
  \inferrule* [lab=synchronization] {} {{M} \bc \pzero \;|\; x?F \;|\; x!C }
  \and
  \inferrule* [lab=abstraction] {} {{F} \bc (x)P}
  \and
  \inferrule* [lab=concretion] {} {{C} \bc \langle Q \rangle}
  \and
  \inferrule* [lab=process] {} {{P,Q} \bc M \;| \;P|Q \;|\; @{x}}
  \and
  \inferrule* [lab=name] {} {{x} \bc \quotep{P}}
\end{mathpar} 

Note that $\vec{x}$ (resp. $\vec{P}$) denotes a vector of names
(resp. processes) of length $|\vec{x}|$ (resp. $|\vec{P}|$). We adopt
the following useful abbreviations.

\begin{mathpar}
   x?(\vec{y}).P := x.(\vec{y})P \and  x\clift{\vec{P}} := x.\clift{\vec{P}}
   \and x!(y) := \lift{x}{\dropn{y}}
   \and \Pi_{i=0}^{n-1}P_i := P_0 | \ldots | P_{n-1}
\end{mathpar}

\subsubsection{Structural congruence}

\paragraph{Free and bound names and alpha-equivalence.} At the
core of structural equivalence is alpha-equivalence which identifies
process that are the same up to a change of variable. Formally, we
recognize the distinction between free and bound names. The free names
of a process, $\freenames{P}$, may be calculated recursively as
follows:

\begin{mathpar}
\freenames{\pzero} := \emptyset
  \and \\
  \freenames{x?(y).P} := \{ x \} \cup (\freenames{P} \setminus \{ y \})
  \and 
  \freenames{x!\langle P \rangle} := \{ x \} \cup \{ P \} 
  \and \\
  \freenames{P|Q} := \freenames{P} \cup \freenames{Q}
  \and \\
  \freenames{@{x}} := \{ x \}
\end{mathpar}

$\pi$
$\quotep{\pi}$

$\freenames{-} : \pi \to \mathcal{P}(\quotep{\pi})$

\begin{eqnarray*}
  \freenames{\pzero} & := & \emptyset \\
  \freenames{x?(y).P} & := & \{ x \} \cup (\freenames{P} \setminus \{ y \}) \\
  \freenames{x!\langle P \rangle} & := & \{ x \} \cup \{ P \} \\
  \freenames{P|Q} & := & \freenames{P} \cup \freenames{Q} \\
  \freenames{\dropn{x}} & := & \{ x \}
\end{eqnarray*}

The bound names of a process, $\boundnames{P}$, are those names occurring in $P$
that are not free. For example, in $x?(y).0$, the name $x$ is free, while $y$ is bound.

\begin{mathpar}
  \inferrule* [lab=monoidal-laws] {} { P|Q \equiv Q|P \and P|0 \equiv P \and P|(Q|R) \equiv (P|Q)|R }
\end{mathpar}

\begin{mathpar}
  \inferrule* [lab=alpha-equivalence] {} { (x)P \equiv (y)P\{y/x\} \and y \not\in \freenames{P} }
\end{mathpar}

\begin{definition}
Then two processes, $P,Q$, are alpha-equivalent if $P = Q\{\vec{y}/\vec{x}\}$ for
some $\vec{x} \in \boundnames{Q},\vec{y} \in \boundnames{P}$, where $Q\{\vec{y}/\vec{x}\}$
denotes the capture-avoiding substitution of $\vec{y}$ for $\vec{x}$ in $Q$.
\end{definition}

\begin{definition}
  The {\em structural congruence} \cite{SangiorgiWalker} , $\equiv$,
  between processes is the least congruence containing
  alpha-equivalence, satisfying the abelian monoid laws
  (associativity, commutativity and $\pzero$ as identity) for parallel
  composition $|$ and for summation $+$.
\end{definition}

\subsection{Name equivalence}

We take name equivalence, written $\nameeq$, to be the smallest
equivalence relation generated by the following rules.

\begin{mathpar}
\inferrule*[lab=Quote-drop]
{ }
{ \quotep{@{x}} \nameeq x }

\inferrule*[lab=Struct-equiv]
{ P \scong Q }
{ \quotep{P} \nameeq \quotep{Q} }
\end{mathpar}

The astute reader will have noticed that the mutual recursion of names
and processes imposes a mutual recursion on alpha-equivalence and
structural equivalence via name-equivalence. Fortunately, all of this
works out pleasantly and we may calculate in the natural way, free of
concern. The reader interested in the details is referred to the
appendix \ref{appendix:rho_details}.

\subsection{Substitution}

We use $\Proc$ for the set of processes, $\QProc$ for the set of
names, and $\id{\{}\vec{y} / \vec{x} \id{\}}$ to denote partial maps,
$s : \QProc \rightarrow \QProc$. A map, $s$ lifts, uniquely, to a map
on process terms, $\widehat{s} : \Proc \rightarrow \Proc$ by the
following equations.

\begin{mathpar}
  (0) \psubstp{Q}{P} := 0 \\
  (R \juxtap S) \psubstp{Q}{P}
  :=    
  (R)\psubstp{Q}{P} \juxtap (S) \psubstp{Q}{P} \\
  (x?(y).R) \psubstp{Q}{P}    
  :=    
  (x)\substp{Q}{P} (z)\concat( (R \psubstn{z}{y}) \psubstp{Q}{P} ) \\
  (\lift{x}{R}) \psubstp{Q}{P}  
  :=
  \lift{(x)\substp{Q}{P}}{ R \psubstp{Q}{P} } \\
%   (\dropn{x})  \psubstp{Q}{P}       
%   := 
%   \left\{ 
%     \begin{array}{ccc} 
%       \dropn{\quotep{Q}} & & x \nameeq \quotep{P} \\
%       \dropn{x} & & otherwise \\
%     \end{array}
%   \right. 
  (\dropn{x})  \psubstp{Q}{P}       
  := 
  \left\{ 
    \begin{array}{ccc} 
      Q & & x \nameeq \quotep{P} \\
      \dropn{x} & & otherwise \\
    \end{array}
  \right.
\end{mathpar}
 

where

\begin{eqnarray}
  (x)\id{\{} \lpquote Q \rpquote / \lpquote P \rpquote \id{\}}            = 
  \left\{ 
    \begin{array}{ccc}
      \lpquote Q \rpquote & & x \nameeq \lpquote P \rpquote \\
      x & & otherwise \\
    \end{array}
  \right. \nonumber
\end{eqnarray}

and $z$ is chosen distinct from $\quotep{P}$, $\quotep{Q}$, the free
names in $Q$, and all the names in $R$. Our $\alpha$-equivalence will
be built in the standard way from this substitution.

\begin{remark}\label{rem:no_self_referential_names}
  One consequence of these definitions is that $\forall P. \quotep{P}
  \not\in \freenames{P}$.
\end{remark}

\subsection{ Dynamic quote: an example }

Anticipating something of what's to come, consider applying the
substitution, $\widehat{\id{\{}u / z \id{\}}}$, to the following pair
of processes, $\lift{w}{y!(z)}$ and $w[ \lpquote y!(z) \rpquote ]$.

\begin{eqnarray}
	\lift{w}{y!(z)}\widehat{\id{\{}u / z \id{\}}}
		& = &
		\lift{w}{y!(u)} \nonumber\\
	w[ \lpquote y!(z) \rpquote ] \widehat{ \id{\{}u / z \id{\}} }
		& = &
		w[ \lpquote y!(z) \rpquote ] \nonumber
\end{eqnarray}

Because the body of the process between quotes is impervious to
substitution, we get radically different answers. In fact, by
examining the first process in an input context,
e.g. $x?(z).\lift{w}{y!(z)}$, we see that the process under the lift
operator may be shaped by prefixed inputs binding a name inside it. In
this sense, the lift operator will be seen as a way to dynamically
construct processes before reifying them as names.

Finally equipped with these standard features we can present the
dynamics of the calculus.

\subsubsection{Operational semantics} 

Finally, we introduce the computational dynamics. What marks these
algebras as distinct from other more traditionally studied algebraic
structures, e.g. vector spaces or polynomial rings, is the manner in
which dynamics is captured. In traditional structures, dynamics is typically
expressed through morphisms between such structures, as in linear maps
between vector spaces or morphisms between rings. In algebras
associated with the semantics of computation, the dynamics is
expressed as part of the algebraic structure itself, through a
reduction reduction relation typically denoted by $\red$. Below, we
give a recursive presentation of this relation for the calculus used
in the encoding.

$\red \subseteq \pi \times \pi$
$\red : \pi \to \mathcal{P}(\pi)$

\begin{mathpar}
  \inferrule* [lab=Comm] { \textsf{match}( x_{src}, x_{trgt} ) } { x_{trgt}?(y)P \; | \; x_{src}!\langle {Q} \rangle \red P\{\quotep{Q}/y}\} }
  \and \\
  \inferrule* [lab=Par] {{P} \red {P}'} {{{P} | {Q}} \red {{P}' | {Q}}}
  \and
  \inferrule* [lab=Equiv]{{{P} \scong {P}'} \andalso {{P}' \red {Q}'} \andalso {{Q}' \scong {Q}}}{{P} \red {Q}}
\end{mathpar}

\begin{eqnarray*}
  match_{\equiv} (\quotep{P},\quotep{Q}) & := & P \equiv Q \\
  match_{\dagger}(\quotep{P},\quotep{Q}) & := & \forall R. P|Q \red^{*} R => R \red^{*} 0 \\
  match_{K}(\quotep{P},\quotep{Q}) & := & K \mbox{ for some context } K
\end{eqnarray*}

$u?(x)P | u!\langle Q \rangle \red P\{\quotep{Q}/x\}$

%We write $\wred$ for $\red^*$, and $P\red$ if $\exists Q $ such that $ P \red Q$.
We write $P\red$ if $\exists Q $ such that $ P \red Q$ and $P\not\red$, otherwise.

\section{Replication}

As mentioned before, it is known that replication (and hence
recursion) can be implemented in a higher-order process algebra
\cite{SangiorgiWalker}. As our first example of calculation with the
machinery thus far presented we give the construction explicitly in
the {\rhoc}.

\begin{eqnarray}
	D_{x} & := & \prefix{x}{y}{(\binpar{\outputp{x}{y}}{@{y}})} \nonumber\\
	\bangp_{x}{P} & := & \binpar{{x}!\langle{\binpar{D_{x}}{P}}\rangle}{D_{x}} \nonumber
\end{eqnarray}

\begin{eqnarray}
	\bangp_{x}{P} & & \nonumber\\
	=
	& {x}!\langle{(\prefix{x}{y}{(\outputp{x}{y} | @{y})) | P}}\rangle 
	      | \prefix{x}{y}{(\outputp{x}{y} | @{y})} & \nonumber\\
	\red
	& (\outputp{x}{y} | @{y})\substn{\quotep{(\prefix{x}{y}{(@{y} | \outputp{x}{y})) | P}}}{y} & \nonumber\\
	=
	& \outputp{x}{\quotep{(\prefix{x}{y}{(\outputp{x}{y} | @{y})) | P}}}
	  | {(\prefix{x}{y}{(\outputp{x}{y} | @{y})) | P}} & \nonumber\\
	\red
	& \ldots & \nonumber\\
	\red^*
	& P | P | \ldots & \nonumber
\end{eqnarray}

Of course, this encoding, as an implementation, runs away, unfolding
$\bangp{P}$ eagerly. A lazier and more implementable replication
operator, restricted to input-guarded processes, may be obtained as follows.

\begin{eqnarray}
\bangp{\prefix{u}{v}{P}} 
	:= 
	\binpar{\lift{x}{\prefix{u}{v}{(\binpar{D(x)}{P})}}}{D(x)} \nonumber
\end{eqnarray}

\begin{remark}
  Note that the lazier definition still does not deal with summation
  or mixed summation (i.e. sums over input and output). The reader is
  invited to construct definitions of replication that deal with these
  features. 

  Further, the definitions are parameterized in a name, $x$. Can you,
  gentle reader, make a definition that eliminates this parameter and
  guarantees no accidental interaction between the replication
  machinery and the process being replicated -- i.e. no accidental
  sharing of names used by the process to get its work done and the
  name(s) used by the replication to effect copying. This latter
  revision of the definition of replication is crucial to obtaining
  the expected identity $!!P \sim !P$.
\end{remark}

\begin{remark}\label{rem:paradoxical_combinator}
  The reader familiar with the lambda calculus will have noticed the
  similarity between $D$ and the paradoxical combinator.

  [Ed. note: the existence of this seems to suggest we have to be more
  restrictive on the set of processes and names we admit if we are to
  support no-cloning.]
\end{remark}

\subsubsection{Bisimulation}

The computational dynamics gives rise to another kind of equivalence,
the equivalence of computational behavior. As previously mentioned
this is typically captured \emph{via} some form of bisimulation.

% The notion we use in this paper is weak barbed bisimulation
% \cite{milner91polyadicpi}.

The notion we use in this paper is derived from weak barbed
bisimulation \cite{milner91polyadicpi}. 

\begin{definition}
An \emph{observation relation}, $\downarrow_{\mathcal N}$, over a set
of names, $\mathcal N$, is the smallest relation satisfying the rules
below.

\infrule[Out-barb]{y \in {\mathcal N}, \; x \nameeq y}
		  {\outputp{x}{v} \downarrow_{\mathcal N} x}
\infrule[Par-barb]{\mbox{$P\downarrow_{\mathcal N} x$ or $Q\downarrow_{\mathcal N} x$}}
		  {\binpar{P}{Q} \downarrow_{\mathcal N} x}

We write $P \Downarrow_{\mathcal N} x$ if there is $Q$ such that 
$P \wred Q$ and $Q \downarrow_{\mathcal N} x$.
\end{definition}

\begin{definition}
%\label{def.bbisim}
An  ${\mathcal N}$-\emph{barbed bisimulation} over a set of names, ${\mathcal N}$, is a symmetric binary relation 
${\mathcal S}_{\mathcal N}$ between agents such that $P\rel{S}_{\mathcal N}Q$ implies:
\begin{enumerate}
\item If $P \red P'$ then $Q \wred Q'$ and $P'\rel{S}_{\mathcal N} Q'$.
\item If $P\downarrow_{\mathcal N} x$, then $Q\Downarrow_{\mathcal N} x$.
\end{enumerate}
$P$ is ${\mathcal N}$-barbed bisimilar to $Q$, written
$P \wbbisim_{\mathcal N} Q$, if $P \rel{S}_{\mathcal N} Q$ for some ${\mathcal N}$-barbed bisimulation ${\mathcal S}_{\mathcal N}$.
\end{definition}

$\mathcal{R} \subseteq \pi \times \pi$

$P \mathcal{R} Q => \forall P'. P \red P' \Rightarrow \exists Q'. Q \red Q', P' \mathcal{R} Q'$

$P \vdash x \Rightarrow Q \vdash x$

\begin{mathpar}
  \inferrule*[lab=Out-barb]{x \nameeq y}{{y}!\langle{Q}\rangle \vdash x}
  \and
  \inferrule*[lab=Par-barb]{\mbox{$P\vdash x$ or $Q\vdash x$}}{\binpar{P}{Q} \vdash x}
\end{mathpar}

\subsubsection{Contexts}

One of the principle advantages of computational calculi like the
$\pi$-calculus is a well-defined notion of context,
contextual-equivalence and a correlation between
contextual-equivalence and notions of bisimulation. The notion of
context allows the decomposition of a process into (sub-)process and
its syntactic environment, its context. Thus, a context may be
thought of as a process with a ``hole'' (written $\Box$) in it. The
application of a context $M$ to a process $P$, written $M[P]$, is
tantamount to filling the hole in $M$ with $P$. In this paper we do
not need the full weight of this theory, but do make use of the notion
of context in the proof the main theorem. 

\begin{mathpar}
  \inferrule* [lab=summation] {} {{M_{M},M_{N}} \bc \Box \;|\; x.M_{A} \;|\; M_{M}+M_{N}}
  \and
  \inferrule* [lab=agent] {} {{M_{A}} \bc (\vec{x})M_{P} \;| \; \clift{P_0,\ldots,M_{P},\ldots,P_N}}
  \and \\
  \inferrule* [lab=process] {} {{M_{P}} \bc M_{N} \;| \;P|M_{P} }
\end{mathpar} 

\begin{mathpar}
  \inferrule* [lab=sychronization] {} {M_{N} \bc \Box \;|\; x?M_{F} \;|\; x!M_{C}}
  \and
  \inferrule* [lab=abstraction] {} {{M_{F}} \bc (x)M_{P} }
  \and
  \inferrule* [lab=concretion] {} {{M_{C}} \bc \langle M_{P} \rangle }
  \and \\
  \inferrule* [lab=process] {} {{M_{P}} \bc M_{N} \;| \;P|M_{P} }
\end{mathpar}

\begin{definition}[contextual application] Given a context $M$, and
  process $P$, we define the \emph{contextual application}, $M[P] :=
  M\{P/\Box\}$. That is, the contextual application of M to P is the
  substitution of $P$ for $\Box$ in $M$.
\end{definition}

$\meaningof{-} : L \to \mathcal{P}(\pi)$

\begin{mathpar}
  \inferrule* [lab=collection] {} {\meaningof{true} = \pi, \and \meaningof{~E} = \pi \setminus \meaningof{E}, \and \meaningof{E_{1} \& E_{2}} = \meaningof{E_{1}} \cap \meaningof{E_{2}}}
\end{mathpar}

\begin{mathpar}
  \inferrule* [lab=structure] {} {\meaningof{0} = \{ P \in \pi | P \equiv 0 \}, \and \\ \meaningof{E_1 | E_2} = \{ P \in \pi | P \equiv P_{1} | P_{2}, P_{1} \in \meaningof{E_{1}}, P_{2} \in \meaningof{E_2}\} }
\end{mathpar}

\begin{mathpar}
 \inferrule* [lab=behavior] {} {\meaningof{\langle a?b \rangle E} = \{ P \in \pi | P \equiv Q | u?(y)P', \\ \and \\\\ \and \\ \;\;\; u \in \meaningof{a}, \forall z.P'\{z/y\} \in \meaningof{E\{z/b\}}\}, \and \\ \meaningof{a!E} = \{ P \in \pi | P \equiv Q | x!\langle P' \rangle, x \in \meaningof{a} P' \in \meaningof{E}\} }
\end{mathpar}

\begin{mathpar}
 \inferrule* [lab=nominal] {} {\meaningof{\quotep{E}} = \{ \quotep{P} \in \quotep{\pi} | P \in \meaningof{E} \}, \and \meaningof{\quotep{P}} = \{ \quotep{Q} \in \quotep{\pi} | P \equiv Q \} \and \\ \meaningof{@\quotep{E}} = \{ P \in \pi | P \equiv @x, x \in \meaningof{E} \}}
\end{mathpar}

\begin{eqnarray*}
  \\
  \meaningof{-} : TS \to ST
\end{eqnarray*}

\begin{eqnarray*}
  \\
  L : TS \to ST
\end{eqnarray*}

\begin{eqnarray*}
  \\
  P \models E \iff P \in \meaningof{E}
\end{eqnarray*}

\begin{eqnarray*}
  P \approx_{L} Q \iff \forall E \in L. P \models E \iff Q \models E
\end{eqnarray*}

\begin{eqnarray*}
  P \approx_{K} Q
\end{eqnarray*}

\begin{eqnarray*}
  P \approx Q
\end{eqnarray*}

$\approx_{K} = \approx = \approx_{L}$

\subsubsection{Contextual duality}

Note that contexts extend the quotation operation to a family of
operations from processes to names. Given a context, $M$, we can
define a \emph{nominal context}, $\quotep{M}$ by $\quotep{M}[P] :=
\quotep{M[P]}$. To foreshadow what is to come we observe that these
operations enjoy a duality with processes very much like the duality
between vectors and maps from vectors to scalars.

Further, because the calculus is essentially higher-order, we have a
correspondence between contexts and processes. More specifically,
given a name $x$ and a context $M$ we can construct $M^{*}_{x}$ such
that 

\begin{mathpar}
  M^{*}_{x} | \lift{x}{P} \red M[P]
\end{mathpar}

namely,

\begin{mathpar}
  M^{*}_{x} := x?(u).M[\dropn{u}]
\end{mathpar}

The dependence of $M^{*}_{x}$ on a name makes it an abstraction, 

\begin{mathpar}
  M^{*} := (x)x?(u).M[\dropn{u}]
\end{mathpar}

\subsection{Additional notation}

It will sometimes be convenient to denote the process a name
quotes. We already have the notation $x = \quotep{P}$, but it will be
convenient to introduce an alternate notation, $\procn{x}$, when we
want to emphasize the connection to the use of the name. Note that, by
virtue of name equivalence, $\quotep{\procn{x}} \nameeq x$; so, the
notation is consistent with previous definitions.

Further, because names have structure it is possible to effect
substitutions on the basis of that structure. This means we need to
upgrade our notation for substitutions, which we accomplish by
adapting comprehension notation. Thus,

\begin{mathpar}
  P\{ y / x : x \in S \}
\end{mathpar}

is interpreted to mean the process derived from P by replacing (in a
capture-avoiding manner) each occurrence of $x$ in $S$ by $y$. For example,

\begin{mathpar}
  P\{ \quotep{\procn{x}|\procn{x}} / x : x \in \freenames{P} \}
\end{mathpar}

will replace each (occurrence) of a free name $x$ in $P$ by
$\quotep{\procn{x}|\procn{x}}$.

Also, we will avail ourselves of the notation $x^{L}$ and $x^{R}$ to
denote injections of a name into disjoint copies of the name
space. There are numerous ways to accomplish this. One example can be
found in \cite{MeredithR05}. This notation overloads to vectors of
names: $\vec{x}^{\pi} := (x_{i}^{\pi} \; : \; 0 \leq i < |\vec{x}| )$ where $\pi \in \{L,R\}$.

We also use $P^{\Box} := P|\Box$.

In \cite{MeredithR05} an interpretation of the new operator is
given. It turns out that there are several possible interpretations
all enjoying the requisite algebraic properties of the operator (see
\cite{milner91polyadicpi}). We will therefore make liberal use of
$(\nu\; \vec{x})P$.

% subsection the_syntax_and_semantics_of_the_notation_system (end)   

\input{qm2pi.qmops} 

\input{qm2pi.sterngerlach} 

\input{qm2pi.metric} 

% section concurrent_process_calculi (end)

%\input{qm2pi.proofsketch}

% section proof sketch (end)

%\input{qm2pi.slviaknots} 

% section spatial logic via knots (end)

\input{qm2pi.conclusion}

% section conclusion (end)

%\input{qm2pi.dtcodes} 

% section wiring algorithm (end)

\input{qm2pi.ack} 

% section acknowledgments (end)

\newpage


\bibliographystyle{plain}   
\bibliography{../../biblios/main.bib}

\input{qm2pi.rhodetails}

\end{document}

 

% section wiring algorithm (end)

\documentclass[12pt]{llncs}
%\documentclass{jktr}

\usepackage[pdftex]{hyperref}                   
\usepackage {listings}
\usepackage {mathpartir}
\usepackage{bcprules}
%\usepackage{listings}
                       
\usepackage{graphicx} 
%\usepackage[margins=2.5cm,nohead,nofoot]{geometry}
%\usepackage{geometry}
\usepackage{amsfonts}
\usepackage{amstext}
\usepackage{latexsym}
\usepackage{amssymb}
\usepackage{color}


%\include{myPreamble}
\include{qm2pi.local} 

%\ifpdf
%\usepackage[pdftex]{graphicx}
%\else
%\usepackage{graphicx}
%\fi

 % \ifpdf
%  \usepackage{pdfsync}
%  \if


%\title{Brief Article}
%\author{David F. Snyder}
%\author{L.G. Meredith}

%\address{Dept. of Math., Texas State University--San Marcos, San Marcos, TX 78666}
       
\pagestyle{empty}


\begin{document}

\lstset{language=[Objective]Caml,frame=shadowbox}

\input{qm2pi.front}

% section front matter (end)

\input{qm2pi.intro} 
 
% section introduction (end)

% \input{qm2pi.knotations} 

% section notation (end)

\input{qm2pi.process.calculi} 

% section concurrent_process_calculi_and_spatial_logics_ (end)
    
%\input{qm2pi.knots2pi} 

%\input{qm2pi.trefoil} 

%\input{qm2pi.mainthm} 

% subsection basic_interpretation (end)

%\input{qm2pi.rho.presentation} 
\subsection{The syntax and semantics of the notation system}\label{sub:the_syntax_and_semantics_of_the_notation_system} % (fold)

We now summarize a technical presentation of the calculus that
embodies our theory of dynamics. The typical presentation of such a
calculus follows the style of giving generators and relations on
them. The grammar, below, describing term constructors, freely
generates the set of processes, $\Proc$. This set is then quotiented
by a relation known as structural congruence and it is over this set
that the notion of dynamics is expressed. This presentation is
essentially that of \cite{MeredithR05} with the addition of
polyadicity and summation. For readability we have relegated some of
the technical subtleties to an appendix.

\subsubsection{Process grammar}\label{subsub:process_grammar}

\begin{mathpar}
  \inferrule* [lab=synchronization] {} {{M} \bc \pzero \;|\; x?F \;|\; x!C }
  \and
  \inferrule* [lab=abstraction] {} {{F} \bc (x)P}
  \and
  \inferrule* [lab=concretion] {} {{C} \bc \langle Q \rangle}
  \and
  \inferrule* [lab=process] {} {{P,Q} \bc M \;| \;P|Q \;|\; @{x}}
  \and
  \inferrule* [lab=name] {} {{x} \bc \quotep{P}}
\end{mathpar} 

Note that $\vec{x}$ (resp. $\vec{P}$) denotes a vector of names
(resp. processes) of length $|\vec{x}|$ (resp. $|\vec{P}|$). We adopt
the following useful abbreviations.

\begin{mathpar}
   x?(\vec{y}).P := x.(\vec{y})P \and  x\clift{\vec{P}} := x.\clift{\vec{P}}
   \and x!(y) := \lift{x}{\dropn{y}}
   \and \Pi_{i=0}^{n-1}P_i := P_0 | \ldots | P_{n-1}
\end{mathpar}

\subsubsection{Structural congruence}

\paragraph{Free and bound names and alpha-equivalence.} At the
core of structural equivalence is alpha-equivalence which identifies
process that are the same up to a change of variable. Formally, we
recognize the distinction between free and bound names. The free names
of a process, $\freenames{P}$, may be calculated recursively as
follows:

\begin{mathpar}
\freenames{\pzero} := \emptyset
  \and \\
  \freenames{x?(y).P} := \{ x \} \cup (\freenames{P} \setminus \{ y \})
  \and 
  \freenames{x!\langle P \rangle} := \{ x \} \cup \{ P \} 
  \and \\
  \freenames{P|Q} := \freenames{P} \cup \freenames{Q}
  \and \\
  \freenames{@{x}} := \{ x \}
\end{mathpar}

$\pi$
$\quotep{\pi}$

$\freenames{-} : \pi \to \mathcal{P}(\quotep{\pi})$

\begin{eqnarray*}
  \freenames{\pzero} & := & \emptyset \\
  \freenames{x?(y).P} & := & \{ x \} \cup (\freenames{P} \setminus \{ y \}) \\
  \freenames{x!\langle P \rangle} & := & \{ x \} \cup \{ P \} \\
  \freenames{P|Q} & := & \freenames{P} \cup \freenames{Q} \\
  \freenames{\dropn{x}} & := & \{ x \}
\end{eqnarray*}

The bound names of a process, $\boundnames{P}$, are those names occurring in $P$
that are not free. For example, in $x?(y).0$, the name $x$ is free, while $y$ is bound.

\begin{mathpar}
  \inferrule* [lab=monoidal-laws] {} { P|Q \equiv Q|P \and P|0 \equiv P \and P|(Q|R) \equiv (P|Q)|R }
\end{mathpar}

\begin{mathpar}
  \inferrule* [lab=alpha-equivalence] {} { (x)P \equiv (y)P\{y/x\} \and y \not\in \freenames{P} }
\end{mathpar}

\begin{definition}
Then two processes, $P,Q$, are alpha-equivalent if $P = Q\{\vec{y}/\vec{x}\}$ for
some $\vec{x} \in \boundnames{Q},\vec{y} \in \boundnames{P}$, where $Q\{\vec{y}/\vec{x}\}$
denotes the capture-avoiding substitution of $\vec{y}$ for $\vec{x}$ in $Q$.
\end{definition}

\begin{definition}
  The {\em structural congruence} \cite{SangiorgiWalker} , $\equiv$,
  between processes is the least congruence containing
  alpha-equivalence, satisfying the abelian monoid laws
  (associativity, commutativity and $\pzero$ as identity) for parallel
  composition $|$ and for summation $+$.
\end{definition}

\subsection{Name equivalence}

We take name equivalence, written $\nameeq$, to be the smallest
equivalence relation generated by the following rules.

\begin{mathpar}
\inferrule*[lab=Quote-drop]
{ }
{ \quotep{@{x}} \nameeq x }

\inferrule*[lab=Struct-equiv]
{ P \scong Q }
{ \quotep{P} \nameeq \quotep{Q} }
\end{mathpar}

The astute reader will have noticed that the mutual recursion of names
and processes imposes a mutual recursion on alpha-equivalence and
structural equivalence via name-equivalence. Fortunately, all of this
works out pleasantly and we may calculate in the natural way, free of
concern. The reader interested in the details is referred to the
appendix \ref{appendix:rho_details}.

\subsection{Substitution}

We use $\Proc$ for the set of processes, $\QProc$ for the set of
names, and $\id{\{}\vec{y} / \vec{x} \id{\}}$ to denote partial maps,
$s : \QProc \rightarrow \QProc$. A map, $s$ lifts, uniquely, to a map
on process terms, $\widehat{s} : \Proc \rightarrow \Proc$ by the
following equations.

\begin{mathpar}
  (0) \psubstp{Q}{P} := 0 \\
  (R \juxtap S) \psubstp{Q}{P}
  :=    
  (R)\psubstp{Q}{P} \juxtap (S) \psubstp{Q}{P} \\
  (x?(y).R) \psubstp{Q}{P}    
  :=    
  (x)\substp{Q}{P} (z)\concat( (R \psubstn{z}{y}) \psubstp{Q}{P} ) \\
  (\lift{x}{R}) \psubstp{Q}{P}  
  :=
  \lift{(x)\substp{Q}{P}}{ R \psubstp{Q}{P} } \\
%   (\dropn{x})  \psubstp{Q}{P}       
%   := 
%   \left\{ 
%     \begin{array}{ccc} 
%       \dropn{\quotep{Q}} & & x \nameeq \quotep{P} \\
%       \dropn{x} & & otherwise \\
%     \end{array}
%   \right. 
  (\dropn{x})  \psubstp{Q}{P}       
  := 
  \left\{ 
    \begin{array}{ccc} 
      Q & & x \nameeq \quotep{P} \\
      \dropn{x} & & otherwise \\
    \end{array}
  \right.
\end{mathpar}
 

where

\begin{eqnarray}
  (x)\id{\{} \lpquote Q \rpquote / \lpquote P \rpquote \id{\}}            = 
  \left\{ 
    \begin{array}{ccc}
      \lpquote Q \rpquote & & x \nameeq \lpquote P \rpquote \\
      x & & otherwise \\
    \end{array}
  \right. \nonumber
\end{eqnarray}

and $z$ is chosen distinct from $\quotep{P}$, $\quotep{Q}$, the free
names in $Q$, and all the names in $R$. Our $\alpha$-equivalence will
be built in the standard way from this substitution.

\begin{remark}\label{rem:no_self_referential_names}
  One consequence of these definitions is that $\forall P. \quotep{P}
  \not\in \freenames{P}$.
\end{remark}

\subsection{ Dynamic quote: an example }

Anticipating something of what's to come, consider applying the
substitution, $\widehat{\id{\{}u / z \id{\}}}$, to the following pair
of processes, $\lift{w}{y!(z)}$ and $w[ \lpquote y!(z) \rpquote ]$.

\begin{eqnarray}
	\lift{w}{y!(z)}\widehat{\id{\{}u / z \id{\}}}
		& = &
		\lift{w}{y!(u)} \nonumber\\
	w[ \lpquote y!(z) \rpquote ] \widehat{ \id{\{}u / z \id{\}} }
		& = &
		w[ \lpquote y!(z) \rpquote ] \nonumber
\end{eqnarray}

Because the body of the process between quotes is impervious to
substitution, we get radically different answers. In fact, by
examining the first process in an input context,
e.g. $x?(z).\lift{w}{y!(z)}$, we see that the process under the lift
operator may be shaped by prefixed inputs binding a name inside it. In
this sense, the lift operator will be seen as a way to dynamically
construct processes before reifying them as names.

Finally equipped with these standard features we can present the
dynamics of the calculus.

\subsubsection{Operational semantics} 

Finally, we introduce the computational dynamics. What marks these
algebras as distinct from other more traditionally studied algebraic
structures, e.g. vector spaces or polynomial rings, is the manner in
which dynamics is captured. In traditional structures, dynamics is typically
expressed through morphisms between such structures, as in linear maps
between vector spaces or morphisms between rings. In algebras
associated with the semantics of computation, the dynamics is
expressed as part of the algebraic structure itself, through a
reduction reduction relation typically denoted by $\red$. Below, we
give a recursive presentation of this relation for the calculus used
in the encoding.

$\red \subseteq \pi \times \pi$
$\red : \pi \to \mathcal{P}(\pi)$

\begin{mathpar}
  \inferrule* [lab=Comm] { \textsf{match}( x_{src}, x_{trgt} ) } { x_{trgt}?(y)P \; | \; x_{src}!\langle {Q} \rangle \red P\{\quotep{Q}/y}\} }
  \and \\
  \inferrule* [lab=Par] {{P} \red {P}'} {{{P} | {Q}} \red {{P}' | {Q}}}
  \and
  \inferrule* [lab=Equiv]{{{P} \scong {P}'} \andalso {{P}' \red {Q}'} \andalso {{Q}' \scong {Q}}}{{P} \red {Q}}
\end{mathpar}

\begin{eqnarray*}
  match_{\equiv} (\quotep{P},\quotep{Q}) & := & P \equiv Q \\
  match_{\dagger}(\quotep{P},\quotep{Q}) & := & \forall R. P|Q \red^{*} R => R \red^{*} 0 \\
  match_{K}(\quotep{P},\quotep{Q}) & := & K \mbox{ for some context } K
\end{eqnarray*}

$u?(x)P | u!\langle Q \rangle \red P\{\quotep{Q}/x\}$

%We write $\wred$ for $\red^*$, and $P\red$ if $\exists Q $ such that $ P \red Q$.
We write $P\red$ if $\exists Q $ such that $ P \red Q$ and $P\not\red$, otherwise.

\section{Replication}

As mentioned before, it is known that replication (and hence
recursion) can be implemented in a higher-order process algebra
\cite{SangiorgiWalker}. As our first example of calculation with the
machinery thus far presented we give the construction explicitly in
the {\rhoc}.

\begin{eqnarray}
	D_{x} & := & \prefix{x}{y}{(\binpar{\outputp{x}{y}}{@{y}})} \nonumber\\
	\bangp_{x}{P} & := & \binpar{{x}!\langle{\binpar{D_{x}}{P}}\rangle}{D_{x}} \nonumber
\end{eqnarray}

\begin{eqnarray}
	\bangp_{x}{P} & & \nonumber\\
	=
	& {x}!\langle{(\prefix{x}{y}{(\outputp{x}{y} | @{y})) | P}}\rangle 
	      | \prefix{x}{y}{(\outputp{x}{y} | @{y})} & \nonumber\\
	\red
	& (\outputp{x}{y} | @{y})\substn{\quotep{(\prefix{x}{y}{(@{y} | \outputp{x}{y})) | P}}}{y} & \nonumber\\
	=
	& \outputp{x}{\quotep{(\prefix{x}{y}{(\outputp{x}{y} | @{y})) | P}}}
	  | {(\prefix{x}{y}{(\outputp{x}{y} | @{y})) | P}} & \nonumber\\
	\red
	& \ldots & \nonumber\\
	\red^*
	& P | P | \ldots & \nonumber
\end{eqnarray}

Of course, this encoding, as an implementation, runs away, unfolding
$\bangp{P}$ eagerly. A lazier and more implementable replication
operator, restricted to input-guarded processes, may be obtained as follows.

\begin{eqnarray}
\bangp{\prefix{u}{v}{P}} 
	:= 
	\binpar{\lift{x}{\prefix{u}{v}{(\binpar{D(x)}{P})}}}{D(x)} \nonumber
\end{eqnarray}

\begin{remark}
  Note that the lazier definition still does not deal with summation
  or mixed summation (i.e. sums over input and output). The reader is
  invited to construct definitions of replication that deal with these
  features. 

  Further, the definitions are parameterized in a name, $x$. Can you,
  gentle reader, make a definition that eliminates this parameter and
  guarantees no accidental interaction between the replication
  machinery and the process being replicated -- i.e. no accidental
  sharing of names used by the process to get its work done and the
  name(s) used by the replication to effect copying. This latter
  revision of the definition of replication is crucial to obtaining
  the expected identity $!!P \sim !P$.
\end{remark}

\begin{remark}\label{rem:paradoxical_combinator}
  The reader familiar with the lambda calculus will have noticed the
  similarity between $D$ and the paradoxical combinator.

  [Ed. note: the existence of this seems to suggest we have to be more
  restrictive on the set of processes and names we admit if we are to
  support no-cloning.]
\end{remark}

\subsubsection{Bisimulation}

The computational dynamics gives rise to another kind of equivalence,
the equivalence of computational behavior. As previously mentioned
this is typically captured \emph{via} some form of bisimulation.

% The notion we use in this paper is weak barbed bisimulation
% \cite{milner91polyadicpi}.

The notion we use in this paper is derived from weak barbed
bisimulation \cite{milner91polyadicpi}. 

\begin{definition}
An \emph{observation relation}, $\downarrow_{\mathcal N}$, over a set
of names, $\mathcal N$, is the smallest relation satisfying the rules
below.

\infrule[Out-barb]{y \in {\mathcal N}, \; x \nameeq y}
		  {\outputp{x}{v} \downarrow_{\mathcal N} x}
\infrule[Par-barb]{\mbox{$P\downarrow_{\mathcal N} x$ or $Q\downarrow_{\mathcal N} x$}}
		  {\binpar{P}{Q} \downarrow_{\mathcal N} x}

We write $P \Downarrow_{\mathcal N} x$ if there is $Q$ such that 
$P \wred Q$ and $Q \downarrow_{\mathcal N} x$.
\end{definition}

\begin{definition}
%\label{def.bbisim}
An  ${\mathcal N}$-\emph{barbed bisimulation} over a set of names, ${\mathcal N}$, is a symmetric binary relation 
${\mathcal S}_{\mathcal N}$ between agents such that $P\rel{S}_{\mathcal N}Q$ implies:
\begin{enumerate}
\item If $P \red P'$ then $Q \wred Q'$ and $P'\rel{S}_{\mathcal N} Q'$.
\item If $P\downarrow_{\mathcal N} x$, then $Q\Downarrow_{\mathcal N} x$.
\end{enumerate}
$P$ is ${\mathcal N}$-barbed bisimilar to $Q$, written
$P \wbbisim_{\mathcal N} Q$, if $P \rel{S}_{\mathcal N} Q$ for some ${\mathcal N}$-barbed bisimulation ${\mathcal S}_{\mathcal N}$.
\end{definition}

$\mathcal{R} \subseteq \pi \times \pi$

$P \mathcal{R} Q => \forall P'. P \red P' \Rightarrow \exists Q'. Q \red Q', P' \mathcal{R} Q'$

$P \vdash x \Rightarrow Q \vdash x$

\begin{mathpar}
  \inferrule*[lab=Out-barb]{x \nameeq y}{{y}!\langle{Q}\rangle \vdash x}
  \and
  \inferrule*[lab=Par-barb]{\mbox{$P\vdash x$ or $Q\vdash x$}}{\binpar{P}{Q} \vdash x}
\end{mathpar}

\subsubsection{Contexts}

One of the principle advantages of computational calculi like the
$\pi$-calculus is a well-defined notion of context,
contextual-equivalence and a correlation between
contextual-equivalence and notions of bisimulation. The notion of
context allows the decomposition of a process into (sub-)process and
its syntactic environment, its context. Thus, a context may be
thought of as a process with a ``hole'' (written $\Box$) in it. The
application of a context $M$ to a process $P$, written $M[P]$, is
tantamount to filling the hole in $M$ with $P$. In this paper we do
not need the full weight of this theory, but do make use of the notion
of context in the proof the main theorem. 

\begin{mathpar}
  \inferrule* [lab=summation] {} {{M_{M},M_{N}} \bc \Box \;|\; x.M_{A} \;|\; M_{M}+M_{N}}
  \and
  \inferrule* [lab=agent] {} {{M_{A}} \bc (\vec{x})M_{P} \;| \; \clift{P_0,\ldots,M_{P},\ldots,P_N}}
  \and \\
  \inferrule* [lab=process] {} {{M_{P}} \bc M_{N} \;| \;P|M_{P} }
\end{mathpar} 

\begin{mathpar}
  \inferrule* [lab=sychronization] {} {M_{N} \bc \Box \;|\; x?M_{F} \;|\; x!M_{C}}
  \and
  \inferrule* [lab=abstraction] {} {{M_{F}} \bc (x)M_{P} }
  \and
  \inferrule* [lab=concretion] {} {{M_{C}} \bc \langle M_{P} \rangle }
  \and \\
  \inferrule* [lab=process] {} {{M_{P}} \bc M_{N} \;| \;P|M_{P} }
\end{mathpar}

\begin{definition}[contextual application] Given a context $M$, and
  process $P$, we define the \emph{contextual application}, $M[P] :=
  M\{P/\Box\}$. That is, the contextual application of M to P is the
  substitution of $P$ for $\Box$ in $M$.
\end{definition}

$\meaningof{-} : L \to \mathcal{P}(\pi)$

\begin{mathpar}
  \inferrule* [lab=collection] {} {\meaningof{true} = \pi, \and \meaningof{~E} = \pi \setminus \meaningof{E}, \and \meaningof{E_{1} \& E_{2}} = \meaningof{E_{1}} \cap \meaningof{E_{2}}}
\end{mathpar}

\begin{mathpar}
  \inferrule* [lab=structure] {} {\meaningof{0} = \{ P \in \pi | P \equiv 0 \}, \and \\ \meaningof{E_1 | E_2} = \{ P \in \pi | P \equiv P_{1} | P_{2}, P_{1} \in \meaningof{E_{1}}, P_{2} \in \meaningof{E_2}\} }
\end{mathpar}

\begin{mathpar}
 \inferrule* [lab=behavior] {} {\meaningof{\langle a?b \rangle E} = \{ P \in \pi | P \equiv Q | u?(y)P', \\ \and \\\\ \and \\ \;\;\; u \in \meaningof{a}, \forall z.P'\{z/y\} \in \meaningof{E\{z/b\}}\}, \and \\ \meaningof{a!E} = \{ P \in \pi | P \equiv Q | x!\langle P' \rangle, x \in \meaningof{a} P' \in \meaningof{E}\} }
\end{mathpar}

\begin{mathpar}
 \inferrule* [lab=nominal] {} {\meaningof{\quotep{E}} = \{ \quotep{P} \in \quotep{\pi} | P \in \meaningof{E} \}, \and \meaningof{\quotep{P}} = \{ \quotep{Q} \in \quotep{\pi} | P \equiv Q \} \and \\ \meaningof{@\quotep{E}} = \{ P \in \pi | P \equiv @x, x \in \meaningof{E} \}}
\end{mathpar}

\begin{eqnarray*}
  \\
  \meaningof{-} : TS \to ST
\end{eqnarray*}

\begin{eqnarray*}
  \\
  L : TS \to ST
\end{eqnarray*}

\begin{eqnarray*}
  \\
  P \models E \iff P \in \meaningof{E}
\end{eqnarray*}

\begin{eqnarray*}
  P \approx_{L} Q \iff \forall E \in L. P \models E \iff Q \models E
\end{eqnarray*}

\begin{eqnarray*}
  P \approx_{K} Q
\end{eqnarray*}

\begin{eqnarray*}
  P \approx Q
\end{eqnarray*}

$\approx_{K} = \approx = \approx_{L}$

\subsubsection{Contextual duality}

Note that contexts extend the quotation operation to a family of
operations from processes to names. Given a context, $M$, we can
define a \emph{nominal context}, $\quotep{M}$ by $\quotep{M}[P] :=
\quotep{M[P]}$. To foreshadow what is to come we observe that these
operations enjoy a duality with processes very much like the duality
between vectors and maps from vectors to scalars.

Further, because the calculus is essentially higher-order, we have a
correspondence between contexts and processes. More specifically,
given a name $x$ and a context $M$ we can construct $M^{*}_{x}$ such
that 

\begin{mathpar}
  M^{*}_{x} | \lift{x}{P} \red M[P]
\end{mathpar}

namely,

\begin{mathpar}
  M^{*}_{x} := x?(u).M[\dropn{u}]
\end{mathpar}

The dependence of $M^{*}_{x}$ on a name makes it an abstraction, 

\begin{mathpar}
  M^{*} := (x)x?(u).M[\dropn{u}]
\end{mathpar}

\subsection{Additional notation}

It will sometimes be convenient to denote the process a name
quotes. We already have the notation $x = \quotep{P}$, but it will be
convenient to introduce an alternate notation, $\procn{x}$, when we
want to emphasize the connection to the use of the name. Note that, by
virtue of name equivalence, $\quotep{\procn{x}} \nameeq x$; so, the
notation is consistent with previous definitions.

Further, because names have structure it is possible to effect
substitutions on the basis of that structure. This means we need to
upgrade our notation for substitutions, which we accomplish by
adapting comprehension notation. Thus,

\begin{mathpar}
  P\{ y / x : x \in S \}
\end{mathpar}

is interpreted to mean the process derived from P by replacing (in a
capture-avoiding manner) each occurrence of $x$ in $S$ by $y$. For example,

\begin{mathpar}
  P\{ \quotep{\procn{x}|\procn{x}} / x : x \in \freenames{P} \}
\end{mathpar}

will replace each (occurrence) of a free name $x$ in $P$ by
$\quotep{\procn{x}|\procn{x}}$.

Also, we will avail ourselves of the notation $x^{L}$ and $x^{R}$ to
denote injections of a name into disjoint copies of the name
space. There are numerous ways to accomplish this. One example can be
found in \cite{MeredithR05}. This notation overloads to vectors of
names: $\vec{x}^{\pi} := (x_{i}^{\pi} \; : \; 0 \leq i < |\vec{x}| )$ where $\pi \in \{L,R\}$.

We also use $P^{\Box} := P|\Box$.

In \cite{MeredithR05} an interpretation of the new operator is
given. It turns out that there are several possible interpretations
all enjoying the requisite algebraic properties of the operator (see
\cite{milner91polyadicpi}). We will therefore make liberal use of
$(\nu\; \vec{x})P$.

% subsection the_syntax_and_semantics_of_the_notation_system (end)   

\input{qm2pi.qmops} 

\input{qm2pi.sterngerlach} 

\input{qm2pi.metric} 

% section concurrent_process_calculi (end)

%\input{qm2pi.proofsketch}

% section proof sketch (end)

%\input{qm2pi.slviaknots} 

% section spatial logic via knots (end)

\input{qm2pi.conclusion}

% section conclusion (end)

%\input{qm2pi.dtcodes} 

% section wiring algorithm (end)

\input{qm2pi.ack} 

% section acknowledgments (end)

\newpage


\bibliographystyle{plain}   
\bibliography{../../biblios/main.bib}

\input{qm2pi.rhodetails}

\end{document}

 

% section acknowledgments (end)

\newpage


\bibliographystyle{plain}   
\bibliography{../../biblios/main.bib}

\documentclass[12pt]{llncs}
%\documentclass{jktr}

\usepackage[pdftex]{hyperref}                   
\usepackage {listings}
\usepackage {mathpartir}
\usepackage{bcprules}
%\usepackage{listings}
                       
\usepackage{graphicx} 
%\usepackage[margins=2.5cm,nohead,nofoot]{geometry}
%\usepackage{geometry}
\usepackage{amsfonts}
\usepackage{amstext}
\usepackage{latexsym}
\usepackage{amssymb}
\usepackage{color}


%\include{myPreamble}
\include{qm2pi.local} 

%\ifpdf
%\usepackage[pdftex]{graphicx}
%\else
%\usepackage{graphicx}
%\fi

 % \ifpdf
%  \usepackage{pdfsync}
%  \if


%\title{Brief Article}
%\author{David F. Snyder}
%\author{L.G. Meredith}

%\address{Dept. of Math., Texas State University--San Marcos, San Marcos, TX 78666}
       
\pagestyle{empty}


\begin{document}

\lstset{language=[Objective]Caml,frame=shadowbox}

\input{qm2pi.front}

% section front matter (end)

\input{qm2pi.intro} 
 
% section introduction (end)

% \input{qm2pi.knotations} 

% section notation (end)

\input{qm2pi.process.calculi} 

% section concurrent_process_calculi_and_spatial_logics_ (end)
    
%\input{qm2pi.knots2pi} 

%\input{qm2pi.trefoil} 

%\input{qm2pi.mainthm} 

% subsection basic_interpretation (end)

%\input{qm2pi.rho.presentation} 
\subsection{The syntax and semantics of the notation system}\label{sub:the_syntax_and_semantics_of_the_notation_system} % (fold)

We now summarize a technical presentation of the calculus that
embodies our theory of dynamics. The typical presentation of such a
calculus follows the style of giving generators and relations on
them. The grammar, below, describing term constructors, freely
generates the set of processes, $\Proc$. This set is then quotiented
by a relation known as structural congruence and it is over this set
that the notion of dynamics is expressed. This presentation is
essentially that of \cite{MeredithR05} with the addition of
polyadicity and summation. For readability we have relegated some of
the technical subtleties to an appendix.

\subsubsection{Process grammar}\label{subsub:process_grammar}

\begin{mathpar}
  \inferrule* [lab=synchronization] {} {{M} \bc \pzero \;|\; x?F \;|\; x!C }
  \and
  \inferrule* [lab=abstraction] {} {{F} \bc (x)P}
  \and
  \inferrule* [lab=concretion] {} {{C} \bc \langle Q \rangle}
  \and
  \inferrule* [lab=process] {} {{P,Q} \bc M \;| \;P|Q \;|\; @{x}}
  \and
  \inferrule* [lab=name] {} {{x} \bc \quotep{P}}
\end{mathpar} 

Note that $\vec{x}$ (resp. $\vec{P}$) denotes a vector of names
(resp. processes) of length $|\vec{x}|$ (resp. $|\vec{P}|$). We adopt
the following useful abbreviations.

\begin{mathpar}
   x?(\vec{y}).P := x.(\vec{y})P \and  x\clift{\vec{P}} := x.\clift{\vec{P}}
   \and x!(y) := \lift{x}{\dropn{y}}
   \and \Pi_{i=0}^{n-1}P_i := P_0 | \ldots | P_{n-1}
\end{mathpar}

\subsubsection{Structural congruence}

\paragraph{Free and bound names and alpha-equivalence.} At the
core of structural equivalence is alpha-equivalence which identifies
process that are the same up to a change of variable. Formally, we
recognize the distinction between free and bound names. The free names
of a process, $\freenames{P}$, may be calculated recursively as
follows:

\begin{mathpar}
\freenames{\pzero} := \emptyset
  \and \\
  \freenames{x?(y).P} := \{ x \} \cup (\freenames{P} \setminus \{ y \})
  \and 
  \freenames{x!\langle P \rangle} := \{ x \} \cup \{ P \} 
  \and \\
  \freenames{P|Q} := \freenames{P} \cup \freenames{Q}
  \and \\
  \freenames{@{x}} := \{ x \}
\end{mathpar}

$\pi$
$\quotep{\pi}$

$\freenames{-} : \pi \to \mathcal{P}(\quotep{\pi})$

\begin{eqnarray*}
  \freenames{\pzero} & := & \emptyset \\
  \freenames{x?(y).P} & := & \{ x \} \cup (\freenames{P} \setminus \{ y \}) \\
  \freenames{x!\langle P \rangle} & := & \{ x \} \cup \{ P \} \\
  \freenames{P|Q} & := & \freenames{P} \cup \freenames{Q} \\
  \freenames{\dropn{x}} & := & \{ x \}
\end{eqnarray*}

The bound names of a process, $\boundnames{P}$, are those names occurring in $P$
that are not free. For example, in $x?(y).0$, the name $x$ is free, while $y$ is bound.

\begin{mathpar}
  \inferrule* [lab=monoidal-laws] {} { P|Q \equiv Q|P \and P|0 \equiv P \and P|(Q|R) \equiv (P|Q)|R }
\end{mathpar}

\begin{mathpar}
  \inferrule* [lab=alpha-equivalence] {} { (x)P \equiv (y)P\{y/x\} \and y \not\in \freenames{P} }
\end{mathpar}

\begin{definition}
Then two processes, $P,Q$, are alpha-equivalent if $P = Q\{\vec{y}/\vec{x}\}$ for
some $\vec{x} \in \boundnames{Q},\vec{y} \in \boundnames{P}$, where $Q\{\vec{y}/\vec{x}\}$
denotes the capture-avoiding substitution of $\vec{y}$ for $\vec{x}$ in $Q$.
\end{definition}

\begin{definition}
  The {\em structural congruence} \cite{SangiorgiWalker} , $\equiv$,
  between processes is the least congruence containing
  alpha-equivalence, satisfying the abelian monoid laws
  (associativity, commutativity and $\pzero$ as identity) for parallel
  composition $|$ and for summation $+$.
\end{definition}

\subsection{Name equivalence}

We take name equivalence, written $\nameeq$, to be the smallest
equivalence relation generated by the following rules.

\begin{mathpar}
\inferrule*[lab=Quote-drop]
{ }
{ \quotep{@{x}} \nameeq x }

\inferrule*[lab=Struct-equiv]
{ P \scong Q }
{ \quotep{P} \nameeq \quotep{Q} }
\end{mathpar}

The astute reader will have noticed that the mutual recursion of names
and processes imposes a mutual recursion on alpha-equivalence and
structural equivalence via name-equivalence. Fortunately, all of this
works out pleasantly and we may calculate in the natural way, free of
concern. The reader interested in the details is referred to the
appendix \ref{appendix:rho_details}.

\subsection{Substitution}

We use $\Proc$ for the set of processes, $\QProc$ for the set of
names, and $\id{\{}\vec{y} / \vec{x} \id{\}}$ to denote partial maps,
$s : \QProc \rightarrow \QProc$. A map, $s$ lifts, uniquely, to a map
on process terms, $\widehat{s} : \Proc \rightarrow \Proc$ by the
following equations.

\begin{mathpar}
  (0) \psubstp{Q}{P} := 0 \\
  (R \juxtap S) \psubstp{Q}{P}
  :=    
  (R)\psubstp{Q}{P} \juxtap (S) \psubstp{Q}{P} \\
  (x?(y).R) \psubstp{Q}{P}    
  :=    
  (x)\substp{Q}{P} (z)\concat( (R \psubstn{z}{y}) \psubstp{Q}{P} ) \\
  (\lift{x}{R}) \psubstp{Q}{P}  
  :=
  \lift{(x)\substp{Q}{P}}{ R \psubstp{Q}{P} } \\
%   (\dropn{x})  \psubstp{Q}{P}       
%   := 
%   \left\{ 
%     \begin{array}{ccc} 
%       \dropn{\quotep{Q}} & & x \nameeq \quotep{P} \\
%       \dropn{x} & & otherwise \\
%     \end{array}
%   \right. 
  (\dropn{x})  \psubstp{Q}{P}       
  := 
  \left\{ 
    \begin{array}{ccc} 
      Q & & x \nameeq \quotep{P} \\
      \dropn{x} & & otherwise \\
    \end{array}
  \right.
\end{mathpar}
 

where

\begin{eqnarray}
  (x)\id{\{} \lpquote Q \rpquote / \lpquote P \rpquote \id{\}}            = 
  \left\{ 
    \begin{array}{ccc}
      \lpquote Q \rpquote & & x \nameeq \lpquote P \rpquote \\
      x & & otherwise \\
    \end{array}
  \right. \nonumber
\end{eqnarray}

and $z$ is chosen distinct from $\quotep{P}$, $\quotep{Q}$, the free
names in $Q$, and all the names in $R$. Our $\alpha$-equivalence will
be built in the standard way from this substitution.

\begin{remark}\label{rem:no_self_referential_names}
  One consequence of these definitions is that $\forall P. \quotep{P}
  \not\in \freenames{P}$.
\end{remark}

\subsection{ Dynamic quote: an example }

Anticipating something of what's to come, consider applying the
substitution, $\widehat{\id{\{}u / z \id{\}}}$, to the following pair
of processes, $\lift{w}{y!(z)}$ and $w[ \lpquote y!(z) \rpquote ]$.

\begin{eqnarray}
	\lift{w}{y!(z)}\widehat{\id{\{}u / z \id{\}}}
		& = &
		\lift{w}{y!(u)} \nonumber\\
	w[ \lpquote y!(z) \rpquote ] \widehat{ \id{\{}u / z \id{\}} }
		& = &
		w[ \lpquote y!(z) \rpquote ] \nonumber
\end{eqnarray}

Because the body of the process between quotes is impervious to
substitution, we get radically different answers. In fact, by
examining the first process in an input context,
e.g. $x?(z).\lift{w}{y!(z)}$, we see that the process under the lift
operator may be shaped by prefixed inputs binding a name inside it. In
this sense, the lift operator will be seen as a way to dynamically
construct processes before reifying them as names.

Finally equipped with these standard features we can present the
dynamics of the calculus.

\subsubsection{Operational semantics} 

Finally, we introduce the computational dynamics. What marks these
algebras as distinct from other more traditionally studied algebraic
structures, e.g. vector spaces or polynomial rings, is the manner in
which dynamics is captured. In traditional structures, dynamics is typically
expressed through morphisms between such structures, as in linear maps
between vector spaces or morphisms between rings. In algebras
associated with the semantics of computation, the dynamics is
expressed as part of the algebraic structure itself, through a
reduction reduction relation typically denoted by $\red$. Below, we
give a recursive presentation of this relation for the calculus used
in the encoding.

$\red \subseteq \pi \times \pi$
$\red : \pi \to \mathcal{P}(\pi)$

\begin{mathpar}
  \inferrule* [lab=Comm] { \textsf{match}( x_{src}, x_{trgt} ) } { x_{trgt}?(y)P \; | \; x_{src}!\langle {Q} \rangle \red P\{\quotep{Q}/y}\} }
  \and \\
  \inferrule* [lab=Par] {{P} \red {P}'} {{{P} | {Q}} \red {{P}' | {Q}}}
  \and
  \inferrule* [lab=Equiv]{{{P} \scong {P}'} \andalso {{P}' \red {Q}'} \andalso {{Q}' \scong {Q}}}{{P} \red {Q}}
\end{mathpar}

\begin{eqnarray*}
  match_{\equiv} (\quotep{P},\quotep{Q}) & := & P \equiv Q \\
  match_{\dagger}(\quotep{P},\quotep{Q}) & := & \forall R. P|Q \red^{*} R => R \red^{*} 0 \\
  match_{K}(\quotep{P},\quotep{Q}) & := & K \mbox{ for some context } K
\end{eqnarray*}

$u?(x)P | u!\langle Q \rangle \red P\{\quotep{Q}/x\}$

%We write $\wred$ for $\red^*$, and $P\red$ if $\exists Q $ such that $ P \red Q$.
We write $P\red$ if $\exists Q $ such that $ P \red Q$ and $P\not\red$, otherwise.

\section{Replication}

As mentioned before, it is known that replication (and hence
recursion) can be implemented in a higher-order process algebra
\cite{SangiorgiWalker}. As our first example of calculation with the
machinery thus far presented we give the construction explicitly in
the {\rhoc}.

\begin{eqnarray}
	D_{x} & := & \prefix{x}{y}{(\binpar{\outputp{x}{y}}{@{y}})} \nonumber\\
	\bangp_{x}{P} & := & \binpar{{x}!\langle{\binpar{D_{x}}{P}}\rangle}{D_{x}} \nonumber
\end{eqnarray}

\begin{eqnarray}
	\bangp_{x}{P} & & \nonumber\\
	=
	& {x}!\langle{(\prefix{x}{y}{(\outputp{x}{y} | @{y})) | P}}\rangle 
	      | \prefix{x}{y}{(\outputp{x}{y} | @{y})} & \nonumber\\
	\red
	& (\outputp{x}{y} | @{y})\substn{\quotep{(\prefix{x}{y}{(@{y} | \outputp{x}{y})) | P}}}{y} & \nonumber\\
	=
	& \outputp{x}{\quotep{(\prefix{x}{y}{(\outputp{x}{y} | @{y})) | P}}}
	  | {(\prefix{x}{y}{(\outputp{x}{y} | @{y})) | P}} & \nonumber\\
	\red
	& \ldots & \nonumber\\
	\red^*
	& P | P | \ldots & \nonumber
\end{eqnarray}

Of course, this encoding, as an implementation, runs away, unfolding
$\bangp{P}$ eagerly. A lazier and more implementable replication
operator, restricted to input-guarded processes, may be obtained as follows.

\begin{eqnarray}
\bangp{\prefix{u}{v}{P}} 
	:= 
	\binpar{\lift{x}{\prefix{u}{v}{(\binpar{D(x)}{P})}}}{D(x)} \nonumber
\end{eqnarray}

\begin{remark}
  Note that the lazier definition still does not deal with summation
  or mixed summation (i.e. sums over input and output). The reader is
  invited to construct definitions of replication that deal with these
  features. 

  Further, the definitions are parameterized in a name, $x$. Can you,
  gentle reader, make a definition that eliminates this parameter and
  guarantees no accidental interaction between the replication
  machinery and the process being replicated -- i.e. no accidental
  sharing of names used by the process to get its work done and the
  name(s) used by the replication to effect copying. This latter
  revision of the definition of replication is crucial to obtaining
  the expected identity $!!P \sim !P$.
\end{remark}

\begin{remark}\label{rem:paradoxical_combinator}
  The reader familiar with the lambda calculus will have noticed the
  similarity between $D$ and the paradoxical combinator.

  [Ed. note: the existence of this seems to suggest we have to be more
  restrictive on the set of processes and names we admit if we are to
  support no-cloning.]
\end{remark}

\subsubsection{Bisimulation}

The computational dynamics gives rise to another kind of equivalence,
the equivalence of computational behavior. As previously mentioned
this is typically captured \emph{via} some form of bisimulation.

% The notion we use in this paper is weak barbed bisimulation
% \cite{milner91polyadicpi}.

The notion we use in this paper is derived from weak barbed
bisimulation \cite{milner91polyadicpi}. 

\begin{definition}
An \emph{observation relation}, $\downarrow_{\mathcal N}$, over a set
of names, $\mathcal N$, is the smallest relation satisfying the rules
below.

\infrule[Out-barb]{y \in {\mathcal N}, \; x \nameeq y}
		  {\outputp{x}{v} \downarrow_{\mathcal N} x}
\infrule[Par-barb]{\mbox{$P\downarrow_{\mathcal N} x$ or $Q\downarrow_{\mathcal N} x$}}
		  {\binpar{P}{Q} \downarrow_{\mathcal N} x}

We write $P \Downarrow_{\mathcal N} x$ if there is $Q$ such that 
$P \wred Q$ and $Q \downarrow_{\mathcal N} x$.
\end{definition}

\begin{definition}
%\label{def.bbisim}
An  ${\mathcal N}$-\emph{barbed bisimulation} over a set of names, ${\mathcal N}$, is a symmetric binary relation 
${\mathcal S}_{\mathcal N}$ between agents such that $P\rel{S}_{\mathcal N}Q$ implies:
\begin{enumerate}
\item If $P \red P'$ then $Q \wred Q'$ and $P'\rel{S}_{\mathcal N} Q'$.
\item If $P\downarrow_{\mathcal N} x$, then $Q\Downarrow_{\mathcal N} x$.
\end{enumerate}
$P$ is ${\mathcal N}$-barbed bisimilar to $Q$, written
$P \wbbisim_{\mathcal N} Q$, if $P \rel{S}_{\mathcal N} Q$ for some ${\mathcal N}$-barbed bisimulation ${\mathcal S}_{\mathcal N}$.
\end{definition}

$\mathcal{R} \subseteq \pi \times \pi$

$P \mathcal{R} Q => \forall P'. P \red P' \Rightarrow \exists Q'. Q \red Q', P' \mathcal{R} Q'$

$P \vdash x \Rightarrow Q \vdash x$

\begin{mathpar}
  \inferrule*[lab=Out-barb]{x \nameeq y}{{y}!\langle{Q}\rangle \vdash x}
  \and
  \inferrule*[lab=Par-barb]{\mbox{$P\vdash x$ or $Q\vdash x$}}{\binpar{P}{Q} \vdash x}
\end{mathpar}

\subsubsection{Contexts}

One of the principle advantages of computational calculi like the
$\pi$-calculus is a well-defined notion of context,
contextual-equivalence and a correlation between
contextual-equivalence and notions of bisimulation. The notion of
context allows the decomposition of a process into (sub-)process and
its syntactic environment, its context. Thus, a context may be
thought of as a process with a ``hole'' (written $\Box$) in it. The
application of a context $M$ to a process $P$, written $M[P]$, is
tantamount to filling the hole in $M$ with $P$. In this paper we do
not need the full weight of this theory, but do make use of the notion
of context in the proof the main theorem. 

\begin{mathpar}
  \inferrule* [lab=summation] {} {{M_{M},M_{N}} \bc \Box \;|\; x.M_{A} \;|\; M_{M}+M_{N}}
  \and
  \inferrule* [lab=agent] {} {{M_{A}} \bc (\vec{x})M_{P} \;| \; \clift{P_0,\ldots,M_{P},\ldots,P_N}}
  \and \\
  \inferrule* [lab=process] {} {{M_{P}} \bc M_{N} \;| \;P|M_{P} }
\end{mathpar} 

\begin{mathpar}
  \inferrule* [lab=sychronization] {} {M_{N} \bc \Box \;|\; x?M_{F} \;|\; x!M_{C}}
  \and
  \inferrule* [lab=abstraction] {} {{M_{F}} \bc (x)M_{P} }
  \and
  \inferrule* [lab=concretion] {} {{M_{C}} \bc \langle M_{P} \rangle }
  \and \\
  \inferrule* [lab=process] {} {{M_{P}} \bc M_{N} \;| \;P|M_{P} }
\end{mathpar}

\begin{definition}[contextual application] Given a context $M$, and
  process $P$, we define the \emph{contextual application}, $M[P] :=
  M\{P/\Box\}$. That is, the contextual application of M to P is the
  substitution of $P$ for $\Box$ in $M$.
\end{definition}

$\meaningof{-} : L \to \mathcal{P}(\pi)$

\begin{mathpar}
  \inferrule* [lab=collection] {} {\meaningof{true} = \pi, \and \meaningof{~E} = \pi \setminus \meaningof{E}, \and \meaningof{E_{1} \& E_{2}} = \meaningof{E_{1}} \cap \meaningof{E_{2}}}
\end{mathpar}

\begin{mathpar}
  \inferrule* [lab=structure] {} {\meaningof{0} = \{ P \in \pi | P \equiv 0 \}, \and \\ \meaningof{E_1 | E_2} = \{ P \in \pi | P \equiv P_{1} | P_{2}, P_{1} \in \meaningof{E_{1}}, P_{2} \in \meaningof{E_2}\} }
\end{mathpar}

\begin{mathpar}
 \inferrule* [lab=behavior] {} {\meaningof{\langle a?b \rangle E} = \{ P \in \pi | P \equiv Q | u?(y)P', \\ \and \\\\ \and \\ \;\;\; u \in \meaningof{a}, \forall z.P'\{z/y\} \in \meaningof{E\{z/b\}}\}, \and \\ \meaningof{a!E} = \{ P \in \pi | P \equiv Q | x!\langle P' \rangle, x \in \meaningof{a} P' \in \meaningof{E}\} }
\end{mathpar}

\begin{mathpar}
 \inferrule* [lab=nominal] {} {\meaningof{\quotep{E}} = \{ \quotep{P} \in \quotep{\pi} | P \in \meaningof{E} \}, \and \meaningof{\quotep{P}} = \{ \quotep{Q} \in \quotep{\pi} | P \equiv Q \} \and \\ \meaningof{@\quotep{E}} = \{ P \in \pi | P \equiv @x, x \in \meaningof{E} \}}
\end{mathpar}

\begin{eqnarray*}
  \\
  \meaningof{-} : TS \to ST
\end{eqnarray*}

\begin{eqnarray*}
  \\
  L : TS \to ST
\end{eqnarray*}

\begin{eqnarray*}
  \\
  P \models E \iff P \in \meaningof{E}
\end{eqnarray*}

\begin{eqnarray*}
  P \approx_{L} Q \iff \forall E \in L. P \models E \iff Q \models E
\end{eqnarray*}

\begin{eqnarray*}
  P \approx_{K} Q
\end{eqnarray*}

\begin{eqnarray*}
  P \approx Q
\end{eqnarray*}

$\approx_{K} = \approx = \approx_{L}$

\subsubsection{Contextual duality}

Note that contexts extend the quotation operation to a family of
operations from processes to names. Given a context, $M$, we can
define a \emph{nominal context}, $\quotep{M}$ by $\quotep{M}[P] :=
\quotep{M[P]}$. To foreshadow what is to come we observe that these
operations enjoy a duality with processes very much like the duality
between vectors and maps from vectors to scalars.

Further, because the calculus is essentially higher-order, we have a
correspondence between contexts and processes. More specifically,
given a name $x$ and a context $M$ we can construct $M^{*}_{x}$ such
that 

\begin{mathpar}
  M^{*}_{x} | \lift{x}{P} \red M[P]
\end{mathpar}

namely,

\begin{mathpar}
  M^{*}_{x} := x?(u).M[\dropn{u}]
\end{mathpar}

The dependence of $M^{*}_{x}$ on a name makes it an abstraction, 

\begin{mathpar}
  M^{*} := (x)x?(u).M[\dropn{u}]
\end{mathpar}

\subsection{Additional notation}

It will sometimes be convenient to denote the process a name
quotes. We already have the notation $x = \quotep{P}$, but it will be
convenient to introduce an alternate notation, $\procn{x}$, when we
want to emphasize the connection to the use of the name. Note that, by
virtue of name equivalence, $\quotep{\procn{x}} \nameeq x$; so, the
notation is consistent with previous definitions.

Further, because names have structure it is possible to effect
substitutions on the basis of that structure. This means we need to
upgrade our notation for substitutions, which we accomplish by
adapting comprehension notation. Thus,

\begin{mathpar}
  P\{ y / x : x \in S \}
\end{mathpar}

is interpreted to mean the process derived from P by replacing (in a
capture-avoiding manner) each occurrence of $x$ in $S$ by $y$. For example,

\begin{mathpar}
  P\{ \quotep{\procn{x}|\procn{x}} / x : x \in \freenames{P} \}
\end{mathpar}

will replace each (occurrence) of a free name $x$ in $P$ by
$\quotep{\procn{x}|\procn{x}}$.

Also, we will avail ourselves of the notation $x^{L}$ and $x^{R}$ to
denote injections of a name into disjoint copies of the name
space. There are numerous ways to accomplish this. One example can be
found in \cite{MeredithR05}. This notation overloads to vectors of
names: $\vec{x}^{\pi} := (x_{i}^{\pi} \; : \; 0 \leq i < |\vec{x}| )$ where $\pi \in \{L,R\}$.

We also use $P^{\Box} := P|\Box$.

In \cite{MeredithR05} an interpretation of the new operator is
given. It turns out that there are several possible interpretations
all enjoying the requisite algebraic properties of the operator (see
\cite{milner91polyadicpi}). We will therefore make liberal use of
$(\nu\; \vec{x})P$.

% subsection the_syntax_and_semantics_of_the_notation_system (end)   

\input{qm2pi.qmops} 

\input{qm2pi.sterngerlach} 

\input{qm2pi.metric} 

% section concurrent_process_calculi (end)

%\input{qm2pi.proofsketch}

% section proof sketch (end)

%\input{qm2pi.slviaknots} 

% section spatial logic via knots (end)

\input{qm2pi.conclusion}

% section conclusion (end)

%\input{qm2pi.dtcodes} 

% section wiring algorithm (end)

\input{qm2pi.ack} 

% section acknowledgments (end)

\newpage


\bibliographystyle{plain}   
\bibliography{../../biblios/main.bib}

\input{qm2pi.rhodetails}

\end{document}



\end{document}

 

% subsection basic_interpretation (end)

%\input{qm2pi.rho.presentation} 
\subsection{The syntax and semantics of the notation system}\label{sub:the_syntax_and_semantics_of_the_notation_system} % (fold)

We now summarize a technical presentation of the calculus that
embodies our theory of dynamics. The typical presentation of such a
calculus follows the style of giving generators and relations on
them. The grammar, below, describing term constructors, freely
generates the set of processes, $\Proc$. This set is then quotiented
by a relation known as structural congruence and it is over this set
that the notion of dynamics is expressed. This presentation is
essentially that of \cite{MeredithR05} with the addition of
polyadicity and summation. For readability we have relegated some of
the technical subtleties to an appendix.

\subsubsection{Process grammar}\label{subsub:process_grammar}

\begin{mathpar}
  \inferrule* [lab=synchronization] {} {{M} \bc \pzero \;|\; x?F \;|\; x!C }
  \and
  \inferrule* [lab=abstraction] {} {{F} \bc (x)P}
  \and
  \inferrule* [lab=concretion] {} {{C} \bc \langle Q \rangle}
  \and
  \inferrule* [lab=process] {} {{P,Q} \bc M \;| \;P|Q \;|\; @{x}}
  \and
  \inferrule* [lab=name] {} {{x} \bc \quotep{P}}
\end{mathpar} 

Note that $\vec{x}$ (resp. $\vec{P}$) denotes a vector of names
(resp. processes) of length $|\vec{x}|$ (resp. $|\vec{P}|$). We adopt
the following useful abbreviations.

\begin{mathpar}
   x?(\vec{y}).P := x.(\vec{y})P \and  x\clift{\vec{P}} := x.\clift{\vec{P}}
   \and x!(y) := \lift{x}{\dropn{y}}
   \and \Pi_{i=0}^{n-1}P_i := P_0 | \ldots | P_{n-1}
\end{mathpar}

\subsubsection{Structural congruence}

\paragraph{Free and bound names and alpha-equivalence.} At the
core of structural equivalence is alpha-equivalence which identifies
process that are the same up to a change of variable. Formally, we
recognize the distinction between free and bound names. The free names
of a process, $\freenames{P}$, may be calculated recursively as
follows:

\begin{mathpar}
\freenames{\pzero} := \emptyset
  \and \\
  \freenames{x?(y).P} := \{ x \} \cup (\freenames{P} \setminus \{ y \})
  \and 
  \freenames{x!\langle P \rangle} := \{ x \} \cup \{ P \} 
  \and \\
  \freenames{P|Q} := \freenames{P} \cup \freenames{Q}
  \and \\
  \freenames{@{x}} := \{ x \}
\end{mathpar}

$\pi$
$\quotep{\pi}$

$\freenames{-} : \pi \to \mathcal{P}(\quotep{\pi})$

\begin{eqnarray*}
  \freenames{\pzero} & := & \emptyset \\
  \freenames{x?(y).P} & := & \{ x \} \cup (\freenames{P} \setminus \{ y \}) \\
  \freenames{x!\langle P \rangle} & := & \{ x \} \cup \{ P \} \\
  \freenames{P|Q} & := & \freenames{P} \cup \freenames{Q} \\
  \freenames{\dropn{x}} & := & \{ x \}
\end{eqnarray*}

The bound names of a process, $\boundnames{P}$, are those names occurring in $P$
that are not free. For example, in $x?(y).0$, the name $x$ is free, while $y$ is bound.

\begin{mathpar}
  \inferrule* [lab=monoidal-laws] {} { P|Q \equiv Q|P \and P|0 \equiv P \and P|(Q|R) \equiv (P|Q)|R }
\end{mathpar}

\begin{mathpar}
  \inferrule* [lab=alpha-equivalence] {} { (x)P \equiv (y)P\{y/x\} \and y \not\in \freenames{P} }
\end{mathpar}

\begin{definition}
Then two processes, $P,Q$, are alpha-equivalent if $P = Q\{\vec{y}/\vec{x}\}$ for
some $\vec{x} \in \boundnames{Q},\vec{y} \in \boundnames{P}$, where $Q\{\vec{y}/\vec{x}\}$
denotes the capture-avoiding substitution of $\vec{y}$ for $\vec{x}$ in $Q$.
\end{definition}

\begin{definition}
  The {\em structural congruence} \cite{SangiorgiWalker} , $\equiv$,
  between processes is the least congruence containing
  alpha-equivalence, satisfying the abelian monoid laws
  (associativity, commutativity and $\pzero$ as identity) for parallel
  composition $|$ and for summation $+$.
\end{definition}

\subsection{Name equivalence}

We take name equivalence, written $\nameeq$, to be the smallest
equivalence relation generated by the following rules.

\begin{mathpar}
\inferrule*[lab=Quote-drop]
{ }
{ \quotep{@{x}} \nameeq x }

\inferrule*[lab=Struct-equiv]
{ P \scong Q }
{ \quotep{P} \nameeq \quotep{Q} }
\end{mathpar}

The astute reader will have noticed that the mutual recursion of names
and processes imposes a mutual recursion on alpha-equivalence and
structural equivalence via name-equivalence. Fortunately, all of this
works out pleasantly and we may calculate in the natural way, free of
concern. The reader interested in the details is referred to the
appendix \ref{appendix:rho_details}.

\subsection{Substitution}

We use $\Proc$ for the set of processes, $\QProc$ for the set of
names, and $\id{\{}\vec{y} / \vec{x} \id{\}}$ to denote partial maps,
$s : \QProc \rightarrow \QProc$. A map, $s$ lifts, uniquely, to a map
on process terms, $\widehat{s} : \Proc \rightarrow \Proc$ by the
following equations.

\begin{mathpar}
  (0) \psubstp{Q}{P} := 0 \\
  (R \juxtap S) \psubstp{Q}{P}
  :=    
  (R)\psubstp{Q}{P} \juxtap (S) \psubstp{Q}{P} \\
  (x?(y).R) \psubstp{Q}{P}    
  :=    
  (x)\substp{Q}{P} (z)\concat( (R \psubstn{z}{y}) \psubstp{Q}{P} ) \\
  (\lift{x}{R}) \psubstp{Q}{P}  
  :=
  \lift{(x)\substp{Q}{P}}{ R \psubstp{Q}{P} } \\
%   (\dropn{x})  \psubstp{Q}{P}       
%   := 
%   \left\{ 
%     \begin{array}{ccc} 
%       \dropn{\quotep{Q}} & & x \nameeq \quotep{P} \\
%       \dropn{x} & & otherwise \\
%     \end{array}
%   \right. 
  (\dropn{x})  \psubstp{Q}{P}       
  := 
  \left\{ 
    \begin{array}{ccc} 
      Q & & x \nameeq \quotep{P} \\
      \dropn{x} & & otherwise \\
    \end{array}
  \right.
\end{mathpar}
 

where

\begin{eqnarray}
  (x)\id{\{} \lpquote Q \rpquote / \lpquote P \rpquote \id{\}}            = 
  \left\{ 
    \begin{array}{ccc}
      \lpquote Q \rpquote & & x \nameeq \lpquote P \rpquote \\
      x & & otherwise \\
    \end{array}
  \right. \nonumber
\end{eqnarray}

and $z$ is chosen distinct from $\quotep{P}$, $\quotep{Q}$, the free
names in $Q$, and all the names in $R$. Our $\alpha$-equivalence will
be built in the standard way from this substitution.

\begin{remark}\label{rem:no_self_referential_names}
  One consequence of these definitions is that $\forall P. \quotep{P}
  \not\in \freenames{P}$.
\end{remark}

\subsection{ Dynamic quote: an example }

Anticipating something of what's to come, consider applying the
substitution, $\widehat{\id{\{}u / z \id{\}}}$, to the following pair
of processes, $\lift{w}{y!(z)}$ and $w[ \lpquote y!(z) \rpquote ]$.

\begin{eqnarray}
	\lift{w}{y!(z)}\widehat{\id{\{}u / z \id{\}}}
		& = &
		\lift{w}{y!(u)} \nonumber\\
	w[ \lpquote y!(z) \rpquote ] \widehat{ \id{\{}u / z \id{\}} }
		& = &
		w[ \lpquote y!(z) \rpquote ] \nonumber
\end{eqnarray}

Because the body of the process between quotes is impervious to
substitution, we get radically different answers. In fact, by
examining the first process in an input context,
e.g. $x?(z).\lift{w}{y!(z)}$, we see that the process under the lift
operator may be shaped by prefixed inputs binding a name inside it. In
this sense, the lift operator will be seen as a way to dynamically
construct processes before reifying them as names.

Finally equipped with these standard features we can present the
dynamics of the calculus.

\subsubsection{Operational semantics} 

Finally, we introduce the computational dynamics. What marks these
algebras as distinct from other more traditionally studied algebraic
structures, e.g. vector spaces or polynomial rings, is the manner in
which dynamics is captured. In traditional structures, dynamics is typically
expressed through morphisms between such structures, as in linear maps
between vector spaces or morphisms between rings. In algebras
associated with the semantics of computation, the dynamics is
expressed as part of the algebraic structure itself, through a
reduction reduction relation typically denoted by $\red$. Below, we
give a recursive presentation of this relation for the calculus used
in the encoding.

$\red \subseteq \pi \times \pi$
$\red : \pi \to \mathcal{P}(\pi)$

\begin{mathpar}
  \inferrule* [lab=Comm] { \textsf{match}( x_{src}, x_{trgt} ) } { x_{trgt}?(y)P \; | \; x_{src}!\langle {Q} \rangle \red P\{\quotep{Q}/y}\} }
  \and \\
  \inferrule* [lab=Par] {{P} \red {P}'} {{{P} | {Q}} \red {{P}' | {Q}}}
  \and
  \inferrule* [lab=Equiv]{{{P} \scong {P}'} \andalso {{P}' \red {Q}'} \andalso {{Q}' \scong {Q}}}{{P} \red {Q}}
\end{mathpar}

\begin{eqnarray*}
  match_{\equiv} (\quotep{P},\quotep{Q}) & := & P \equiv Q \\
  match_{\dagger}(\quotep{P},\quotep{Q}) & := & \forall R. P|Q \red^{*} R => R \red^{*} 0 \\
  match_{K}(\quotep{P},\quotep{Q}) & := & K \mbox{ for some context } K
\end{eqnarray*}

$u?(x)P | u!\langle Q \rangle \red P\{\quotep{Q}/x\}$

%We write $\wred$ for $\red^*$, and $P\red$ if $\exists Q $ such that $ P \red Q$.
We write $P\red$ if $\exists Q $ such that $ P \red Q$ and $P\not\red$, otherwise.

\section{Replication}

As mentioned before, it is known that replication (and hence
recursion) can be implemented in a higher-order process algebra
\cite{SangiorgiWalker}. As our first example of calculation with the
machinery thus far presented we give the construction explicitly in
the {\rhoc}.

\begin{eqnarray}
	D_{x} & := & \prefix{x}{y}{(\binpar{\outputp{x}{y}}{@{y}})} \nonumber\\
	\bangp_{x}{P} & := & \binpar{{x}!\langle{\binpar{D_{x}}{P}}\rangle}{D_{x}} \nonumber
\end{eqnarray}

\begin{eqnarray}
	\bangp_{x}{P} & & \nonumber\\
	=
	& {x}!\langle{(\prefix{x}{y}{(\outputp{x}{y} | @{y})) | P}}\rangle 
	      | \prefix{x}{y}{(\outputp{x}{y} | @{y})} & \nonumber\\
	\red
	& (\outputp{x}{y} | @{y})\substn{\quotep{(\prefix{x}{y}{(@{y} | \outputp{x}{y})) | P}}}{y} & \nonumber\\
	=
	& \outputp{x}{\quotep{(\prefix{x}{y}{(\outputp{x}{y} | @{y})) | P}}}
	  | {(\prefix{x}{y}{(\outputp{x}{y} | @{y})) | P}} & \nonumber\\
	\red
	& \ldots & \nonumber\\
	\red^*
	& P | P | \ldots & \nonumber
\end{eqnarray}

Of course, this encoding, as an implementation, runs away, unfolding
$\bangp{P}$ eagerly. A lazier and more implementable replication
operator, restricted to input-guarded processes, may be obtained as follows.

\begin{eqnarray}
\bangp{\prefix{u}{v}{P}} 
	:= 
	\binpar{\lift{x}{\prefix{u}{v}{(\binpar{D(x)}{P})}}}{D(x)} \nonumber
\end{eqnarray}

\begin{remark}
  Note that the lazier definition still does not deal with summation
  or mixed summation (i.e. sums over input and output). The reader is
  invited to construct definitions of replication that deal with these
  features. 

  Further, the definitions are parameterized in a name, $x$. Can you,
  gentle reader, make a definition that eliminates this parameter and
  guarantees no accidental interaction between the replication
  machinery and the process being replicated -- i.e. no accidental
  sharing of names used by the process to get its work done and the
  name(s) used by the replication to effect copying. This latter
  revision of the definition of replication is crucial to obtaining
  the expected identity $!!P \sim !P$.
\end{remark}

\begin{remark}\label{rem:paradoxical_combinator}
  The reader familiar with the lambda calculus will have noticed the
  similarity between $D$ and the paradoxical combinator.

  [Ed. note: the existence of this seems to suggest we have to be more
  restrictive on the set of processes and names we admit if we are to
  support no-cloning.]
\end{remark}

\subsubsection{Bisimulation}

The computational dynamics gives rise to another kind of equivalence,
the equivalence of computational behavior. As previously mentioned
this is typically captured \emph{via} some form of bisimulation.

% The notion we use in this paper is weak barbed bisimulation
% \cite{milner91polyadicpi}.

The notion we use in this paper is derived from weak barbed
bisimulation \cite{milner91polyadicpi}. 

\begin{definition}
An \emph{observation relation}, $\downarrow_{\mathcal N}$, over a set
of names, $\mathcal N$, is the smallest relation satisfying the rules
below.

\infrule[Out-barb]{y \in {\mathcal N}, \; x \nameeq y}
		  {\outputp{x}{v} \downarrow_{\mathcal N} x}
\infrule[Par-barb]{\mbox{$P\downarrow_{\mathcal N} x$ or $Q\downarrow_{\mathcal N} x$}}
		  {\binpar{P}{Q} \downarrow_{\mathcal N} x}

We write $P \Downarrow_{\mathcal N} x$ if there is $Q$ such that 
$P \wred Q$ and $Q \downarrow_{\mathcal N} x$.
\end{definition}

\begin{definition}
%\label{def.bbisim}
An  ${\mathcal N}$-\emph{barbed bisimulation} over a set of names, ${\mathcal N}$, is a symmetric binary relation 
${\mathcal S}_{\mathcal N}$ between agents such that $P\rel{S}_{\mathcal N}Q$ implies:
\begin{enumerate}
\item If $P \red P'$ then $Q \wred Q'$ and $P'\rel{S}_{\mathcal N} Q'$.
\item If $P\downarrow_{\mathcal N} x$, then $Q\Downarrow_{\mathcal N} x$.
\end{enumerate}
$P$ is ${\mathcal N}$-barbed bisimilar to $Q$, written
$P \wbbisim_{\mathcal N} Q$, if $P \rel{S}_{\mathcal N} Q$ for some ${\mathcal N}$-barbed bisimulation ${\mathcal S}_{\mathcal N}$.
\end{definition}

$\mathcal{R} \subseteq \pi \times \pi$

$P \mathcal{R} Q => \forall P'. P \red P' \Rightarrow \exists Q'. Q \red Q', P' \mathcal{R} Q'$

$P \vdash x \Rightarrow Q \vdash x$

\begin{mathpar}
  \inferrule*[lab=Out-barb]{x \nameeq y}{{y}!\langle{Q}\rangle \vdash x}
  \and
  \inferrule*[lab=Par-barb]{\mbox{$P\vdash x$ or $Q\vdash x$}}{\binpar{P}{Q} \vdash x}
\end{mathpar}

\subsubsection{Contexts}

One of the principle advantages of computational calculi like the
$\pi$-calculus is a well-defined notion of context,
contextual-equivalence and a correlation between
contextual-equivalence and notions of bisimulation. The notion of
context allows the decomposition of a process into (sub-)process and
its syntactic environment, its context. Thus, a context may be
thought of as a process with a ``hole'' (written $\Box$) in it. The
application of a context $M$ to a process $P$, written $M[P]$, is
tantamount to filling the hole in $M$ with $P$. In this paper we do
not need the full weight of this theory, but do make use of the notion
of context in the proof the main theorem. 

\begin{mathpar}
  \inferrule* [lab=summation] {} {{M_{M},M_{N}} \bc \Box \;|\; x.M_{A} \;|\; M_{M}+M_{N}}
  \and
  \inferrule* [lab=agent] {} {{M_{A}} \bc (\vec{x})M_{P} \;| \; \clift{P_0,\ldots,M_{P},\ldots,P_N}}
  \and \\
  \inferrule* [lab=process] {} {{M_{P}} \bc M_{N} \;| \;P|M_{P} }
\end{mathpar} 

\begin{mathpar}
  \inferrule* [lab=sychronization] {} {M_{N} \bc \Box \;|\; x?M_{F} \;|\; x!M_{C}}
  \and
  \inferrule* [lab=abstraction] {} {{M_{F}} \bc (x)M_{P} }
  \and
  \inferrule* [lab=concretion] {} {{M_{C}} \bc \langle M_{P} \rangle }
  \and \\
  \inferrule* [lab=process] {} {{M_{P}} \bc M_{N} \;| \;P|M_{P} }
\end{mathpar}

\begin{definition}[contextual application] Given a context $M$, and
  process $P$, we define the \emph{contextual application}, $M[P] :=
  M\{P/\Box\}$. That is, the contextual application of M to P is the
  substitution of $P$ for $\Box$ in $M$.
\end{definition}

$\meaningof{-} : L \to \mathcal{P}(\pi)$

\begin{mathpar}
  \inferrule* [lab=collection] {} {\meaningof{true} = \pi, \and \meaningof{~E} = \pi \setminus \meaningof{E}, \and \meaningof{E_{1} \& E_{2}} = \meaningof{E_{1}} \cap \meaningof{E_{2}}}
\end{mathpar}

\begin{mathpar}
  \inferrule* [lab=structure] {} {\meaningof{0} = \{ P \in \pi | P \equiv 0 \}, \and \\ \meaningof{E_1 | E_2} = \{ P \in \pi | P \equiv P_{1} | P_{2}, P_{1} \in \meaningof{E_{1}}, P_{2} \in \meaningof{E_2}\} }
\end{mathpar}

\begin{mathpar}
 \inferrule* [lab=behavior] {} {\meaningof{\langle a?b \rangle E} = \{ P \in \pi | P \equiv Q | u?(y)P', \\ \and \\\\ \and \\ \;\;\; u \in \meaningof{a}, \forall z.P'\{z/y\} \in \meaningof{E\{z/b\}}\}, \and \\ \meaningof{a!E} = \{ P \in \pi | P \equiv Q | x!\langle P' \rangle, x \in \meaningof{a} P' \in \meaningof{E}\} }
\end{mathpar}

\begin{mathpar}
 \inferrule* [lab=nominal] {} {\meaningof{\quotep{E}} = \{ \quotep{P} \in \quotep{\pi} | P \in \meaningof{E} \}, \and \meaningof{\quotep{P}} = \{ \quotep{Q} \in \quotep{\pi} | P \equiv Q \} \and \\ \meaningof{@\quotep{E}} = \{ P \in \pi | P \equiv @x, x \in \meaningof{E} \}}
\end{mathpar}

\begin{eqnarray*}
  \\
  \meaningof{-} : TS \to ST
\end{eqnarray*}

\begin{eqnarray*}
  \\
  L : TS \to ST
\end{eqnarray*}

\begin{eqnarray*}
  \\
  P \models E \iff P \in \meaningof{E}
\end{eqnarray*}

\begin{eqnarray*}
  P \approx_{L} Q \iff \forall E \in L. P \models E \iff Q \models E
\end{eqnarray*}

\begin{eqnarray*}
  P \approx_{K} Q
\end{eqnarray*}

\begin{eqnarray*}
  P \approx Q
\end{eqnarray*}

$\approx_{K} = \approx = \approx_{L}$

\subsubsection{Contextual duality}

Note that contexts extend the quotation operation to a family of
operations from processes to names. Given a context, $M$, we can
define a \emph{nominal context}, $\quotep{M}$ by $\quotep{M}[P] :=
\quotep{M[P]}$. To foreshadow what is to come we observe that these
operations enjoy a duality with processes very much like the duality
between vectors and maps from vectors to scalars.

Further, because the calculus is essentially higher-order, we have a
correspondence between contexts and processes. More specifically,
given a name $x$ and a context $M$ we can construct $M^{*}_{x}$ such
that 

\begin{mathpar}
  M^{*}_{x} | \lift{x}{P} \red M[P]
\end{mathpar}

namely,

\begin{mathpar}
  M^{*}_{x} := x?(u).M[\dropn{u}]
\end{mathpar}

The dependence of $M^{*}_{x}$ on a name makes it an abstraction, 

\begin{mathpar}
  M^{*} := (x)x?(u).M[\dropn{u}]
\end{mathpar}

\subsection{Additional notation}

It will sometimes be convenient to denote the process a name
quotes. We already have the notation $x = \quotep{P}$, but it will be
convenient to introduce an alternate notation, $\procn{x}$, when we
want to emphasize the connection to the use of the name. Note that, by
virtue of name equivalence, $\quotep{\procn{x}} \nameeq x$; so, the
notation is consistent with previous definitions.

Further, because names have structure it is possible to effect
substitutions on the basis of that structure. This means we need to
upgrade our notation for substitutions, which we accomplish by
adapting comprehension notation. Thus,

\begin{mathpar}
  P\{ y / x : x \in S \}
\end{mathpar}

is interpreted to mean the process derived from P by replacing (in a
capture-avoiding manner) each occurrence of $x$ in $S$ by $y$. For example,

\begin{mathpar}
  P\{ \quotep{\procn{x}|\procn{x}} / x : x \in \freenames{P} \}
\end{mathpar}

will replace each (occurrence) of a free name $x$ in $P$ by
$\quotep{\procn{x}|\procn{x}}$.

Also, we will avail ourselves of the notation $x^{L}$ and $x^{R}$ to
denote injections of a name into disjoint copies of the name
space. There are numerous ways to accomplish this. One example can be
found in \cite{MeredithR05}. This notation overloads to vectors of
names: $\vec{x}^{\pi} := (x_{i}^{\pi} \; : \; 0 \leq i < |\vec{x}| )$ where $\pi \in \{L,R\}$.

We also use $P^{\Box} := P|\Box$.

In \cite{MeredithR05} an interpretation of the new operator is
given. It turns out that there are several possible interpretations
all enjoying the requisite algebraic properties of the operator (see
\cite{milner91polyadicpi}). We will therefore make liberal use of
$(\nu\; \vec{x})P$.

% subsection the_syntax_and_semantics_of_the_notation_system (end)   

\section{Interpretation of QM}
\subsection{Supporting definitions}
\subsubsection{Multiplication}
\begin{mathpar}
  \quotep{Q} \cdot \quotep{R} := \quotep{Q|R}
  \and \\
  \quotep{Q} \cdot P := P\{ \quotep{Q|R} / \quotep{R} : \quotep{R} \in \freenames{P} \}
\end{mathpar}

\paragraph{Discussion}
The first line needs little explanation. The second line says that
each free name of the process is replaced with the multiplication of
that name by the scalar. Multiplication of a scalar (name) by a state
(process) results in a process all the names of which have been `moved
over' by parallel composition with the process the scalar
quotes. There is a subtlety that the bound names have to be
manipulated so that multiplied names aren't accidentally
captured. There are many ways to achieve this.

\begin{remark}\label{rem:multiplication_identities}
  The reader is invited to verify that for all $x,y,z \in \QProc$ and $P \in \Proc$
  \begin{mathpar}
    x \cdot \quotep{0} \equiv x 
    \and
    x \cdot y \equiv y \cdot x
    \and
    x \cdot (y \cdot z) \equiv (x \cdot y) \cdot z
    \and \\
    \quotep{0} \cdot P \equiv P
    \and \\
    x \cdot (y \cdot P) \equiv (x \cdot y) \cdot P
    \and \\
    x \cdot (P|Q) \equiv (x \cdot P) | (x \cdot Q)
    \and \\    
  \end{mathpar}
\end{remark}

\subsubsection{Tensor product}

We define a tensor product on processes by structural induction.

\paragraph{Tensor of sums} First note that all summations, including
$\pzero$ and sequence, can be written $\Sigma_{i} x_{i}.A_{i} +
\Sigma_{j} x_{j}.C_{j}$, where we have grouped input-guarded processes
together and output-guarded processes together.

Thus, we can define the tensor product of two summations, $N_{1}\otimes N_{2}$, where

\begin{mathpar}
  N_{1} := \Sigma_{i} x_{i}.A_{i} + \Sigma_{j} x_{j}.C_{j}
  \and
  N_{2} := \Sigma_{i'} y_{i'}.B_{i'} + \Sigma_{j'} y_{j'}.D_{j'} 
\end{mathpar}

as follows.

\begin{mathpar}
  \Sigma_{i} x_{i}.A_{i} + \Sigma_{j} x_{j}.C_{j} \otimes \Sigma_{i'}
  y_{i'}.B_{i'} + \Sigma_{j'} y_{j'}.D_{j'} 
  \and \\
  := \; \Sigma_{i} \Sigma_{i'} \quotep{\stackrel{\vee}{x_{i}}| \stackrel{\vee}{y_{i'}}}.(A_{i}\otimes B_{i'}) \; | \; \Sigma_{i'} \Sigma_{i} \quotep{\stackrel{\vee}{y_{i'}}|\stackrel{\vee}{x_{i}}}.(B_{i'}\otimes A_{i})
  \and
  \;\; | \;\; \Sigma_{j} \Sigma_{j'} \quotep{\stackrel{\vee}{x_{j}}|\stackrel{\vee}{y_{j'}}}.(A_{j}\otimes B_{j'}) \; | \; \Sigma_{j'} \Sigma_{j} \quotep{\stackrel{\vee}{y_{j'}}|\stackrel{\vee}{x_{j}}}.(B_{j'}\otimes A_{j})
\end{mathpar}

\begin{remark}
  Do we need to $x^{L}$ and $y^{R}$ for this construction as well?
\end{remark}

\paragraph{Tensor of parallel compositions} Next, we distribute tensor
over par.

\begin{mathpar}
  P_{1}|P_{2} \otimes Q_{1}|Q_{2} := (P_{1} \otimes Q_{1}) | (P_{1}
  \otimes Q_{2}) | (P_{2} \otimes Q_{1}) | (P_{2} \otimes Q_{2})
\end{mathpar}

\paragraph{Tensor with dropped names} We treat tensor of a
process with a dropped name as parallel composition.

\begin{mathpar}
  P \otimes \dropn{x} := P | \dropn{x}
\end{mathpar}

\paragraph{Tensor of agents}

Finally, we need to define tensor on agents. Note that the definition
of tensor on normal products only tensors inputs with inputs and
outputs with outputs. Thus, we only have to define the operation on
``homogeneous'' pairings.

\begin{mathpar}
  (\vec{x})P \otimes (\vec{y})Q
  \and \\
  := (x_{0}^{L}|y_{0}^{R},\ldots,x_{0}^{L}|y_{n}^{R},\ldots,x_{m}^{L}|y_{0}^{R},\ldots,x_{m}^{L}|y_{n}^R)(P\{ \vec{x}^{L}/\vec{x}\} \otimes Q \{ \vec{y}^{R}/\vec{y}\})
  \and \\
  \clift{\vec{P}} \otimes \clift{\vec{Q}}
  \and \\
  := \clift{P_{0}\otimes Q_{0},\ldots,P_{0}\otimes Q_{n},\ldots,P_{m}\otimes Q_{0},\ldots,P_{m}\otimes Q_{n}}
\end{mathpar}

\begin{remark}
  Observe that arities of tensored abstractions matches arities of
  tensored concretions if the original arities matched. Note also that
  the length of the arities corresponds to the increase in dimension
  we see in ordinary vector space tensor product.
\end{remark}

\begin{remark}
  Operationally, this definition distributes the tensor down to
  components ``linked'' by summation. Tensor over summation is
  intriguing in that it mixes names. Moreover, as a consequence of the
  way it mixes names we have the identities for all $x \in \QProc$ and
  $P,Q \in \Proc$

  \begin{mathpar}
    (x \cdot P) \otimes Q \equiv x \cdot (P \otimes Q) \equiv P \otimes (x \cdot Q)
    \and
    P \otimes \pzero \equiv P
  \end{mathpar}

  that the reader is invited to verify.
\end{remark}

\subsubsection{Annihilation}
\begin{mathpar}
  P^{\perp} := \{ Q | \forall R. P|Q \red^{*} R \Rightarrow R \red^{*} \pzero \}
  \and \\
  P^{\underline{\perp}} := \Sigma_{Q \in P^{\perp}} \quotep{Q}?(y).(\dropn{y}|Q) | \Sigma_{Q \in P^{\perp}} \quotep{Q}\clift{\Box}
\end{mathpar}

\paragraph{Discussion} The reader will note that $P^{\perp}$ is a
\emph{set} of processes, while $P^{\underline{\perp}}$ is a
\emph{context}. We call the set $P^{\perp}$ the \emph{annihilators} of
$P$. The parallel composition of a process in the annihilators of $P$
with $P$ will result in a process, the state space of which has all
paths eventually leading to $\pzero$. Execution may endure loops; but
under reasonable conditions of fairness (naturally guaranteed under
most notions of bisimulation) such a composite process cannot get
stuck in such a loop and will, eventually pop out and terminate.

The context $P^{\underline{\perp}}$ is ready and willing to ``take the
$P$ out of'' the process to which it is applied. It will effectively
transmit the code of the process to which it is applied to one of the
annihilators and run the process against it.

\subsubsection{Evaluation}
We fix $M$ a domain of fully abstract interpretation with an equality
coincident with bisimulation. We take $\meaningof{\cdot} : \Proc \to
M$ to be the map interpreting processes and $\nmeaningof{\cdot} : \M
\to Proc$ to be the map running the other way. Then we define

\begin{mathpar}
  \int P := \nmeaningof{\meaningof{P}}
\end{mathpar}

\paragraph{Discussion}
There are many fully abstract interpretations of Milner's
$\pi$-calculus. Any of them can be used as a basis for interpreting
the reflective calculus here. Equipped with such a domain it is
largely a matter of grinding through to check that the Yoneda
construction for the normalization-by-evaluation program can be
extended to this setting.

\begin{remark}
  The reader is invited to verify that $\int (P^{\underline{\perp}}[P]) = 0$.
\end{remark}

\subsection{Quantum mechanics}

Table \ref{tbl:core_qm_op_defns} gives the core operational definitions

\begin{table}[htp]\label{tbl:core_qm_op_defns}
  \center{
    \fbox{
      \begin{tabular}{c|c}
        quantum mechanics & process calculus \\
        \hline
        scalar & $x := \quotep{P}$ \\
        state vector & $\state{P} := P$ \\
        dual & $\state{P}^{*} := \event{P^{\underline{\perp}}} := \quotep{P^{\underline{\perp}}}[-]$ \\
        matrix & $ \Sigma_{\alpha} \state{P_{\alpha}}x_{\alpha}\event{Q_{\alpha}}$ \\
        vector addition & $\state{P} + \state{Q} := \state{P | Q}$ \\
        tensor product & $\state{P} \otimes \state{Q} := \state{P \otimes Q}$ \\
        inner product & $\innerprod{P}{Q} := \quotep{\int P^{\underline{\perp}}[Q]}$ \\
      \end{tabular}
    }
  }
  \caption{QM - operational definitions}
\end{table}

where

\begin{mathpar}
  \prmatrix{P}{Q} := \fprmatrix{P}{\quotep{\pzero}}{Q}
  \and
  \fprmatrix{P}{x}{Q} := (\state{P},x,\event{Q})
  \and
  (\fprmatrix{P}{x}{Q})(\state{R}) := x \cdot \innerprod{Q}{R} \cdot \state{P}
  \and
  (\fprmatrix{P}{x}{Q})(\event{R}) := x \cdot \innerprod{R}{P} \cdot \event{Q}
\end{mathpar}

\paragraph{Discussion}
As promised: vectors (aka states) are represented as processes; duals
as contextual duals; inner product definition should be compared with
standard inner product definition for ....

\begin{remark}
  Assuming $\int (P^{\underline{\perp}}[P]) = 0$, the reader is
  invited to verify that $(\fprmatrix{P}{x}{P})(\state{P}) = x \cdot \state{P}$.
\end{remark}

\begin{remark}
  The reader is invited to verify that $\innerprod{P}{Q}$ could
  equally well have been written $\quotep{\int \stackrel{\vee}{x}}$
  where $x = \event{P^{\underline{\perp}}}(Q)$.

  One of the motivations for this remark is that there is another way
  to factor these operations. We could package up evaluation in the dual:

  \begin{mathpar}
    \state{P}^{*} := \event{\int P^{\underline{\perp}}} := \quotep{\int P^{\underline{\perp}}}[-]
  \end{mathpar}

  and then have inner product defined by
  
  \begin{mathpar}
    \innerprod{P}{Q} := \event{P}(Q)
  \end{mathpar}

  Hopefully, experience with the calculations will provide guidance on
  the best factoring.
\end{remark}

\begin{remark}
  Assuming $\int (P^{\underline{\perp}}[P]) = 0$, the reader is
  invited to verify that $\forall P,Q. (\prmatrix{0}{Q})(\state{0}) =
  \state{0}$ and dually $(\prmatrix{P}{0})(\event{0}) = \event{0}$.
\end{remark}

\begin{remark}
  i'm a little worried that i don't (yet) have proper support for
  complex conjugacy. But, the observation above may give us a
  clue. According to Abramsky, it must be the case that the scalars
  are iso to the homset of the identity for the tensor -- which the
  observation above characterizes. 

  For now, we will simply bookmark the notion with $\overline{x}$.
\end{remark}

\subsubsection{Adjointness}

We need to give a definition of $(\cdot)^{\dagger}$ for matrices. The
obvious candidate definition is
\begin{mathpar}
(\Sigma_{\alpha}\fprmatrix{P_{\alpha}}{x_{\alpha}}{Q_{\alpha}})^{\dagger}
= \Sigma_{\alpha}\fprmatrix{(Q_{\alpha}^{\underline{\perp}})^{*}}{\overline{x}_{\alpha}}{P_{\alpha}^{\underline{\perp}}} 
\end{mathpar}

But, $(Q_{\alpha}^{\underline{\perp}})^{*}$ requires a name along
which to communicate the process to achieve the context application.

\subsubsection{Basis for a basis}
If processes label states and ``addition'' of states (a.k.a. vector
addition) is interpreted as parallel composition, what corresponds to
notions of linear independence and basis? Here, we recall that Yoshida
has developed a set of \emph{combinators} for an asynchronous verison
of Milner's $\pi$-calculus. These are a finite set of processes such
any process can be expressed as parallel composition of these
combinators together with liberal uses of the new operator and
replication. We can simply give a translation of these into the
present calculus and have reasonable expectation that the property
carries over. That is, that the resultant set allows to express all
processes via parallel composition. Note, however, that there is no
new operator or replication in this calculus. As a result, we expect
that the corresponding set is actually infinite. That is, we expect
that the space is actually infinite dimensional.

\begin{remark}
  The attentive reader may be a bit concerned. Certainly, the
  collection $S$, $K$ and $I$ is a finite set of
  combinators. Shouldn't we expect to see a finite set of combinators
  for an effectively equivalent system? i am very sympathetic to this
  critique and feel it warrants full attention. On the other hand, i
  also have in mind the following analogy. The natural numbers, as a
  monoid under addition, has exactly $1$ generator, while the natural
  numbers, as a monoid under multiplication, has countably many
  generators (the primes). We observe that the application of the
  lambda calculus is much less resource sensitive than the parallel
  composition of the $\pi$-calculus. Could it be the case that we have
  an analogy of the form
  
  \begin{mathpar}
    m + n : MN :: m*n : M|N
  \end{mathpar}

  giving a similar blow up in the set of ``primes''?  This is such a
  wonderful thought that, even if it's not true, i think it's worth
  writing down.
\end{remark}
 

\documentclass[12pt]{llncs}
%\documentclass{jktr}

\usepackage[pdftex]{hyperref}                   
\usepackage {listings}
\usepackage {mathpartir}
\usepackage{bcprules}
%\usepackage{listings}
                       
\usepackage{graphicx} 
%\usepackage[margins=2.5cm,nohead,nofoot]{geometry}
%\usepackage{geometry}
\usepackage{amsfonts}
\usepackage{amstext}
\usepackage{latexsym}
\usepackage{amssymb}
\usepackage{color}


%\include{myPreamble}
\documentclass[12pt]{llncs}
%\documentclass{jktr}

\usepackage[pdftex]{hyperref}                   
\usepackage {listings}
\usepackage {mathpartir}
\usepackage{bcprules}
%\usepackage{listings}
                       
\usepackage{graphicx} 
%\usepackage[margins=2.5cm,nohead,nofoot]{geometry}
%\usepackage{geometry}
\usepackage{amsfonts}
\usepackage{amstext}
\usepackage{latexsym}
\usepackage{amssymb}
\usepackage{color}


%\include{myPreamble}
\include{qm2pi.local} 

%\ifpdf
%\usepackage[pdftex]{graphicx}
%\else
%\usepackage{graphicx}
%\fi

 % \ifpdf
%  \usepackage{pdfsync}
%  \if


%\title{Brief Article}
%\author{David F. Snyder}
%\author{L.G. Meredith}

%\address{Dept. of Math., Texas State University--San Marcos, San Marcos, TX 78666}
       
\pagestyle{empty}


\begin{document}

\lstset{language=[Objective]Caml,frame=shadowbox}

\input{qm2pi.front}

% section front matter (end)

\input{qm2pi.intro} 
 
% section introduction (end)

% \input{qm2pi.knotations} 

% section notation (end)

\input{qm2pi.process.calculi} 

% section concurrent_process_calculi_and_spatial_logics_ (end)
    
%\input{qm2pi.knots2pi} 

%\input{qm2pi.trefoil} 

%\input{qm2pi.mainthm} 

% subsection basic_interpretation (end)

%\input{qm2pi.rho.presentation} 
\subsection{The syntax and semantics of the notation system}\label{sub:the_syntax_and_semantics_of_the_notation_system} % (fold)

We now summarize a technical presentation of the calculus that
embodies our theory of dynamics. The typical presentation of such a
calculus follows the style of giving generators and relations on
them. The grammar, below, describing term constructors, freely
generates the set of processes, $\Proc$. This set is then quotiented
by a relation known as structural congruence and it is over this set
that the notion of dynamics is expressed. This presentation is
essentially that of \cite{MeredithR05} with the addition of
polyadicity and summation. For readability we have relegated some of
the technical subtleties to an appendix.

\subsubsection{Process grammar}\label{subsub:process_grammar}

\begin{mathpar}
  \inferrule* [lab=synchronization] {} {{M} \bc \pzero \;|\; x?F \;|\; x!C }
  \and
  \inferrule* [lab=abstraction] {} {{F} \bc (x)P}
  \and
  \inferrule* [lab=concretion] {} {{C} \bc \langle Q \rangle}
  \and
  \inferrule* [lab=process] {} {{P,Q} \bc M \;| \;P|Q \;|\; @{x}}
  \and
  \inferrule* [lab=name] {} {{x} \bc \quotep{P}}
\end{mathpar} 

Note that $\vec{x}$ (resp. $\vec{P}$) denotes a vector of names
(resp. processes) of length $|\vec{x}|$ (resp. $|\vec{P}|$). We adopt
the following useful abbreviations.

\begin{mathpar}
   x?(\vec{y}).P := x.(\vec{y})P \and  x\clift{\vec{P}} := x.\clift{\vec{P}}
   \and x!(y) := \lift{x}{\dropn{y}}
   \and \Pi_{i=0}^{n-1}P_i := P_0 | \ldots | P_{n-1}
\end{mathpar}

\subsubsection{Structural congruence}

\paragraph{Free and bound names and alpha-equivalence.} At the
core of structural equivalence is alpha-equivalence which identifies
process that are the same up to a change of variable. Formally, we
recognize the distinction between free and bound names. The free names
of a process, $\freenames{P}$, may be calculated recursively as
follows:

\begin{mathpar}
\freenames{\pzero} := \emptyset
  \and \\
  \freenames{x?(y).P} := \{ x \} \cup (\freenames{P} \setminus \{ y \})
  \and 
  \freenames{x!\langle P \rangle} := \{ x \} \cup \{ P \} 
  \and \\
  \freenames{P|Q} := \freenames{P} \cup \freenames{Q}
  \and \\
  \freenames{@{x}} := \{ x \}
\end{mathpar}

$\pi$
$\quotep{\pi}$

$\freenames{-} : \pi \to \mathcal{P}(\quotep{\pi})$

\begin{eqnarray*}
  \freenames{\pzero} & := & \emptyset \\
  \freenames{x?(y).P} & := & \{ x \} \cup (\freenames{P} \setminus \{ y \}) \\
  \freenames{x!\langle P \rangle} & := & \{ x \} \cup \{ P \} \\
  \freenames{P|Q} & := & \freenames{P} \cup \freenames{Q} \\
  \freenames{\dropn{x}} & := & \{ x \}
\end{eqnarray*}

The bound names of a process, $\boundnames{P}$, are those names occurring in $P$
that are not free. For example, in $x?(y).0$, the name $x$ is free, while $y$ is bound.

\begin{mathpar}
  \inferrule* [lab=monoidal-laws] {} { P|Q \equiv Q|P \and P|0 \equiv P \and P|(Q|R) \equiv (P|Q)|R }
\end{mathpar}

\begin{mathpar}
  \inferrule* [lab=alpha-equivalence] {} { (x)P \equiv (y)P\{y/x\} \and y \not\in \freenames{P} }
\end{mathpar}

\begin{definition}
Then two processes, $P,Q$, are alpha-equivalent if $P = Q\{\vec{y}/\vec{x}\}$ for
some $\vec{x} \in \boundnames{Q},\vec{y} \in \boundnames{P}$, where $Q\{\vec{y}/\vec{x}\}$
denotes the capture-avoiding substitution of $\vec{y}$ for $\vec{x}$ in $Q$.
\end{definition}

\begin{definition}
  The {\em structural congruence} \cite{SangiorgiWalker} , $\equiv$,
  between processes is the least congruence containing
  alpha-equivalence, satisfying the abelian monoid laws
  (associativity, commutativity and $\pzero$ as identity) for parallel
  composition $|$ and for summation $+$.
\end{definition}

\subsection{Name equivalence}

We take name equivalence, written $\nameeq$, to be the smallest
equivalence relation generated by the following rules.

\begin{mathpar}
\inferrule*[lab=Quote-drop]
{ }
{ \quotep{@{x}} \nameeq x }

\inferrule*[lab=Struct-equiv]
{ P \scong Q }
{ \quotep{P} \nameeq \quotep{Q} }
\end{mathpar}

The astute reader will have noticed that the mutual recursion of names
and processes imposes a mutual recursion on alpha-equivalence and
structural equivalence via name-equivalence. Fortunately, all of this
works out pleasantly and we may calculate in the natural way, free of
concern. The reader interested in the details is referred to the
appendix \ref{appendix:rho_details}.

\subsection{Substitution}

We use $\Proc$ for the set of processes, $\QProc$ for the set of
names, and $\id{\{}\vec{y} / \vec{x} \id{\}}$ to denote partial maps,
$s : \QProc \rightarrow \QProc$. A map, $s$ lifts, uniquely, to a map
on process terms, $\widehat{s} : \Proc \rightarrow \Proc$ by the
following equations.

\begin{mathpar}
  (0) \psubstp{Q}{P} := 0 \\
  (R \juxtap S) \psubstp{Q}{P}
  :=    
  (R)\psubstp{Q}{P} \juxtap (S) \psubstp{Q}{P} \\
  (x?(y).R) \psubstp{Q}{P}    
  :=    
  (x)\substp{Q}{P} (z)\concat( (R \psubstn{z}{y}) \psubstp{Q}{P} ) \\
  (\lift{x}{R}) \psubstp{Q}{P}  
  :=
  \lift{(x)\substp{Q}{P}}{ R \psubstp{Q}{P} } \\
%   (\dropn{x})  \psubstp{Q}{P}       
%   := 
%   \left\{ 
%     \begin{array}{ccc} 
%       \dropn{\quotep{Q}} & & x \nameeq \quotep{P} \\
%       \dropn{x} & & otherwise \\
%     \end{array}
%   \right. 
  (\dropn{x})  \psubstp{Q}{P}       
  := 
  \left\{ 
    \begin{array}{ccc} 
      Q & & x \nameeq \quotep{P} \\
      \dropn{x} & & otherwise \\
    \end{array}
  \right.
\end{mathpar}
 

where

\begin{eqnarray}
  (x)\id{\{} \lpquote Q \rpquote / \lpquote P \rpquote \id{\}}            = 
  \left\{ 
    \begin{array}{ccc}
      \lpquote Q \rpquote & & x \nameeq \lpquote P \rpquote \\
      x & & otherwise \\
    \end{array}
  \right. \nonumber
\end{eqnarray}

and $z$ is chosen distinct from $\quotep{P}$, $\quotep{Q}$, the free
names in $Q$, and all the names in $R$. Our $\alpha$-equivalence will
be built in the standard way from this substitution.

\begin{remark}\label{rem:no_self_referential_names}
  One consequence of these definitions is that $\forall P. \quotep{P}
  \not\in \freenames{P}$.
\end{remark}

\subsection{ Dynamic quote: an example }

Anticipating something of what's to come, consider applying the
substitution, $\widehat{\id{\{}u / z \id{\}}}$, to the following pair
of processes, $\lift{w}{y!(z)}$ and $w[ \lpquote y!(z) \rpquote ]$.

\begin{eqnarray}
	\lift{w}{y!(z)}\widehat{\id{\{}u / z \id{\}}}
		& = &
		\lift{w}{y!(u)} \nonumber\\
	w[ \lpquote y!(z) \rpquote ] \widehat{ \id{\{}u / z \id{\}} }
		& = &
		w[ \lpquote y!(z) \rpquote ] \nonumber
\end{eqnarray}

Because the body of the process between quotes is impervious to
substitution, we get radically different answers. In fact, by
examining the first process in an input context,
e.g. $x?(z).\lift{w}{y!(z)}$, we see that the process under the lift
operator may be shaped by prefixed inputs binding a name inside it. In
this sense, the lift operator will be seen as a way to dynamically
construct processes before reifying them as names.

Finally equipped with these standard features we can present the
dynamics of the calculus.

\subsubsection{Operational semantics} 

Finally, we introduce the computational dynamics. What marks these
algebras as distinct from other more traditionally studied algebraic
structures, e.g. vector spaces or polynomial rings, is the manner in
which dynamics is captured. In traditional structures, dynamics is typically
expressed through morphisms between such structures, as in linear maps
between vector spaces or morphisms between rings. In algebras
associated with the semantics of computation, the dynamics is
expressed as part of the algebraic structure itself, through a
reduction reduction relation typically denoted by $\red$. Below, we
give a recursive presentation of this relation for the calculus used
in the encoding.

$\red \subseteq \pi \times \pi$
$\red : \pi \to \mathcal{P}(\pi)$

\begin{mathpar}
  \inferrule* [lab=Comm] { \textsf{match}( x_{src}, x_{trgt} ) } { x_{trgt}?(y)P \; | \; x_{src}!\langle {Q} \rangle \red P\{\quotep{Q}/y}\} }
  \and \\
  \inferrule* [lab=Par] {{P} \red {P}'} {{{P} | {Q}} \red {{P}' | {Q}}}
  \and
  \inferrule* [lab=Equiv]{{{P} \scong {P}'} \andalso {{P}' \red {Q}'} \andalso {{Q}' \scong {Q}}}{{P} \red {Q}}
\end{mathpar}

\begin{eqnarray*}
  match_{\equiv} (\quotep{P},\quotep{Q}) & := & P \equiv Q \\
  match_{\dagger}(\quotep{P},\quotep{Q}) & := & \forall R. P|Q \red^{*} R => R \red^{*} 0 \\
  match_{K}(\quotep{P},\quotep{Q}) & := & K \mbox{ for some context } K
\end{eqnarray*}

$u?(x)P | u!\langle Q \rangle \red P\{\quotep{Q}/x\}$

%We write $\wred$ for $\red^*$, and $P\red$ if $\exists Q $ such that $ P \red Q$.
We write $P\red$ if $\exists Q $ such that $ P \red Q$ and $P\not\red$, otherwise.

\section{Replication}

As mentioned before, it is known that replication (and hence
recursion) can be implemented in a higher-order process algebra
\cite{SangiorgiWalker}. As our first example of calculation with the
machinery thus far presented we give the construction explicitly in
the {\rhoc}.

\begin{eqnarray}
	D_{x} & := & \prefix{x}{y}{(\binpar{\outputp{x}{y}}{@{y}})} \nonumber\\
	\bangp_{x}{P} & := & \binpar{{x}!\langle{\binpar{D_{x}}{P}}\rangle}{D_{x}} \nonumber
\end{eqnarray}

\begin{eqnarray}
	\bangp_{x}{P} & & \nonumber\\
	=
	& {x}!\langle{(\prefix{x}{y}{(\outputp{x}{y} | @{y})) | P}}\rangle 
	      | \prefix{x}{y}{(\outputp{x}{y} | @{y})} & \nonumber\\
	\red
	& (\outputp{x}{y} | @{y})\substn{\quotep{(\prefix{x}{y}{(@{y} | \outputp{x}{y})) | P}}}{y} & \nonumber\\
	=
	& \outputp{x}{\quotep{(\prefix{x}{y}{(\outputp{x}{y} | @{y})) | P}}}
	  | {(\prefix{x}{y}{(\outputp{x}{y} | @{y})) | P}} & \nonumber\\
	\red
	& \ldots & \nonumber\\
	\red^*
	& P | P | \ldots & \nonumber
\end{eqnarray}

Of course, this encoding, as an implementation, runs away, unfolding
$\bangp{P}$ eagerly. A lazier and more implementable replication
operator, restricted to input-guarded processes, may be obtained as follows.

\begin{eqnarray}
\bangp{\prefix{u}{v}{P}} 
	:= 
	\binpar{\lift{x}{\prefix{u}{v}{(\binpar{D(x)}{P})}}}{D(x)} \nonumber
\end{eqnarray}

\begin{remark}
  Note that the lazier definition still does not deal with summation
  or mixed summation (i.e. sums over input and output). The reader is
  invited to construct definitions of replication that deal with these
  features. 

  Further, the definitions are parameterized in a name, $x$. Can you,
  gentle reader, make a definition that eliminates this parameter and
  guarantees no accidental interaction between the replication
  machinery and the process being replicated -- i.e. no accidental
  sharing of names used by the process to get its work done and the
  name(s) used by the replication to effect copying. This latter
  revision of the definition of replication is crucial to obtaining
  the expected identity $!!P \sim !P$.
\end{remark}

\begin{remark}\label{rem:paradoxical_combinator}
  The reader familiar with the lambda calculus will have noticed the
  similarity between $D$ and the paradoxical combinator.

  [Ed. note: the existence of this seems to suggest we have to be more
  restrictive on the set of processes and names we admit if we are to
  support no-cloning.]
\end{remark}

\subsubsection{Bisimulation}

The computational dynamics gives rise to another kind of equivalence,
the equivalence of computational behavior. As previously mentioned
this is typically captured \emph{via} some form of bisimulation.

% The notion we use in this paper is weak barbed bisimulation
% \cite{milner91polyadicpi}.

The notion we use in this paper is derived from weak barbed
bisimulation \cite{milner91polyadicpi}. 

\begin{definition}
An \emph{observation relation}, $\downarrow_{\mathcal N}$, over a set
of names, $\mathcal N$, is the smallest relation satisfying the rules
below.

\infrule[Out-barb]{y \in {\mathcal N}, \; x \nameeq y}
		  {\outputp{x}{v} \downarrow_{\mathcal N} x}
\infrule[Par-barb]{\mbox{$P\downarrow_{\mathcal N} x$ or $Q\downarrow_{\mathcal N} x$}}
		  {\binpar{P}{Q} \downarrow_{\mathcal N} x}

We write $P \Downarrow_{\mathcal N} x$ if there is $Q$ such that 
$P \wred Q$ and $Q \downarrow_{\mathcal N} x$.
\end{definition}

\begin{definition}
%\label{def.bbisim}
An  ${\mathcal N}$-\emph{barbed bisimulation} over a set of names, ${\mathcal N}$, is a symmetric binary relation 
${\mathcal S}_{\mathcal N}$ between agents such that $P\rel{S}_{\mathcal N}Q$ implies:
\begin{enumerate}
\item If $P \red P'$ then $Q \wred Q'$ and $P'\rel{S}_{\mathcal N} Q'$.
\item If $P\downarrow_{\mathcal N} x$, then $Q\Downarrow_{\mathcal N} x$.
\end{enumerate}
$P$ is ${\mathcal N}$-barbed bisimilar to $Q$, written
$P \wbbisim_{\mathcal N} Q$, if $P \rel{S}_{\mathcal N} Q$ for some ${\mathcal N}$-barbed bisimulation ${\mathcal S}_{\mathcal N}$.
\end{definition}

$\mathcal{R} \subseteq \pi \times \pi$

$P \mathcal{R} Q => \forall P'. P \red P' \Rightarrow \exists Q'. Q \red Q', P' \mathcal{R} Q'$

$P \vdash x \Rightarrow Q \vdash x$

\begin{mathpar}
  \inferrule*[lab=Out-barb]{x \nameeq y}{{y}!\langle{Q}\rangle \vdash x}
  \and
  \inferrule*[lab=Par-barb]{\mbox{$P\vdash x$ or $Q\vdash x$}}{\binpar{P}{Q} \vdash x}
\end{mathpar}

\subsubsection{Contexts}

One of the principle advantages of computational calculi like the
$\pi$-calculus is a well-defined notion of context,
contextual-equivalence and a correlation between
contextual-equivalence and notions of bisimulation. The notion of
context allows the decomposition of a process into (sub-)process and
its syntactic environment, its context. Thus, a context may be
thought of as a process with a ``hole'' (written $\Box$) in it. The
application of a context $M$ to a process $P$, written $M[P]$, is
tantamount to filling the hole in $M$ with $P$. In this paper we do
not need the full weight of this theory, but do make use of the notion
of context in the proof the main theorem. 

\begin{mathpar}
  \inferrule* [lab=summation] {} {{M_{M},M_{N}} \bc \Box \;|\; x.M_{A} \;|\; M_{M}+M_{N}}
  \and
  \inferrule* [lab=agent] {} {{M_{A}} \bc (\vec{x})M_{P} \;| \; \clift{P_0,\ldots,M_{P},\ldots,P_N}}
  \and \\
  \inferrule* [lab=process] {} {{M_{P}} \bc M_{N} \;| \;P|M_{P} }
\end{mathpar} 

\begin{mathpar}
  \inferrule* [lab=sychronization] {} {M_{N} \bc \Box \;|\; x?M_{F} \;|\; x!M_{C}}
  \and
  \inferrule* [lab=abstraction] {} {{M_{F}} \bc (x)M_{P} }
  \and
  \inferrule* [lab=concretion] {} {{M_{C}} \bc \langle M_{P} \rangle }
  \and \\
  \inferrule* [lab=process] {} {{M_{P}} \bc M_{N} \;| \;P|M_{P} }
\end{mathpar}

\begin{definition}[contextual application] Given a context $M$, and
  process $P$, we define the \emph{contextual application}, $M[P] :=
  M\{P/\Box\}$. That is, the contextual application of M to P is the
  substitution of $P$ for $\Box$ in $M$.
\end{definition}

$\meaningof{-} : L \to \mathcal{P}(\pi)$

\begin{mathpar}
  \inferrule* [lab=collection] {} {\meaningof{true} = \pi, \and \meaningof{~E} = \pi \setminus \meaningof{E}, \and \meaningof{E_{1} \& E_{2}} = \meaningof{E_{1}} \cap \meaningof{E_{2}}}
\end{mathpar}

\begin{mathpar}
  \inferrule* [lab=structure] {} {\meaningof{0} = \{ P \in \pi | P \equiv 0 \}, \and \\ \meaningof{E_1 | E_2} = \{ P \in \pi | P \equiv P_{1} | P_{2}, P_{1} \in \meaningof{E_{1}}, P_{2} \in \meaningof{E_2}\} }
\end{mathpar}

\begin{mathpar}
 \inferrule* [lab=behavior] {} {\meaningof{\langle a?b \rangle E} = \{ P \in \pi | P \equiv Q | u?(y)P', \\ \and \\\\ \and \\ \;\;\; u \in \meaningof{a}, \forall z.P'\{z/y\} \in \meaningof{E\{z/b\}}\}, \and \\ \meaningof{a!E} = \{ P \in \pi | P \equiv Q | x!\langle P' \rangle, x \in \meaningof{a} P' \in \meaningof{E}\} }
\end{mathpar}

\begin{mathpar}
 \inferrule* [lab=nominal] {} {\meaningof{\quotep{E}} = \{ \quotep{P} \in \quotep{\pi} | P \in \meaningof{E} \}, \and \meaningof{\quotep{P}} = \{ \quotep{Q} \in \quotep{\pi} | P \equiv Q \} \and \\ \meaningof{@\quotep{E}} = \{ P \in \pi | P \equiv @x, x \in \meaningof{E} \}}
\end{mathpar}

\begin{eqnarray*}
  \\
  \meaningof{-} : TS \to ST
\end{eqnarray*}

\begin{eqnarray*}
  \\
  L : TS \to ST
\end{eqnarray*}

\begin{eqnarray*}
  \\
  P \models E \iff P \in \meaningof{E}
\end{eqnarray*}

\begin{eqnarray*}
  P \approx_{L} Q \iff \forall E \in L. P \models E \iff Q \models E
\end{eqnarray*}

\begin{eqnarray*}
  P \approx_{K} Q
\end{eqnarray*}

\begin{eqnarray*}
  P \approx Q
\end{eqnarray*}

$\approx_{K} = \approx = \approx_{L}$

\subsubsection{Contextual duality}

Note that contexts extend the quotation operation to a family of
operations from processes to names. Given a context, $M$, we can
define a \emph{nominal context}, $\quotep{M}$ by $\quotep{M}[P] :=
\quotep{M[P]}$. To foreshadow what is to come we observe that these
operations enjoy a duality with processes very much like the duality
between vectors and maps from vectors to scalars.

Further, because the calculus is essentially higher-order, we have a
correspondence between contexts and processes. More specifically,
given a name $x$ and a context $M$ we can construct $M^{*}_{x}$ such
that 

\begin{mathpar}
  M^{*}_{x} | \lift{x}{P} \red M[P]
\end{mathpar}

namely,

\begin{mathpar}
  M^{*}_{x} := x?(u).M[\dropn{u}]
\end{mathpar}

The dependence of $M^{*}_{x}$ on a name makes it an abstraction, 

\begin{mathpar}
  M^{*} := (x)x?(u).M[\dropn{u}]
\end{mathpar}

\subsection{Additional notation}

It will sometimes be convenient to denote the process a name
quotes. We already have the notation $x = \quotep{P}$, but it will be
convenient to introduce an alternate notation, $\procn{x}$, when we
want to emphasize the connection to the use of the name. Note that, by
virtue of name equivalence, $\quotep{\procn{x}} \nameeq x$; so, the
notation is consistent with previous definitions.

Further, because names have structure it is possible to effect
substitutions on the basis of that structure. This means we need to
upgrade our notation for substitutions, which we accomplish by
adapting comprehension notation. Thus,

\begin{mathpar}
  P\{ y / x : x \in S \}
\end{mathpar}

is interpreted to mean the process derived from P by replacing (in a
capture-avoiding manner) each occurrence of $x$ in $S$ by $y$. For example,

\begin{mathpar}
  P\{ \quotep{\procn{x}|\procn{x}} / x : x \in \freenames{P} \}
\end{mathpar}

will replace each (occurrence) of a free name $x$ in $P$ by
$\quotep{\procn{x}|\procn{x}}$.

Also, we will avail ourselves of the notation $x^{L}$ and $x^{R}$ to
denote injections of a name into disjoint copies of the name
space. There are numerous ways to accomplish this. One example can be
found in \cite{MeredithR05}. This notation overloads to vectors of
names: $\vec{x}^{\pi} := (x_{i}^{\pi} \; : \; 0 \leq i < |\vec{x}| )$ where $\pi \in \{L,R\}$.

We also use $P^{\Box} := P|\Box$.

In \cite{MeredithR05} an interpretation of the new operator is
given. It turns out that there are several possible interpretations
all enjoying the requisite algebraic properties of the operator (see
\cite{milner91polyadicpi}). We will therefore make liberal use of
$(\nu\; \vec{x})P$.

% subsection the_syntax_and_semantics_of_the_notation_system (end)   

\input{qm2pi.qmops} 

\input{qm2pi.sterngerlach} 

\input{qm2pi.metric} 

% section concurrent_process_calculi (end)

%\input{qm2pi.proofsketch}

% section proof sketch (end)

%\input{qm2pi.slviaknots} 

% section spatial logic via knots (end)

\input{qm2pi.conclusion}

% section conclusion (end)

%\input{qm2pi.dtcodes} 

% section wiring algorithm (end)

\input{qm2pi.ack} 

% section acknowledgments (end)

\newpage


\bibliographystyle{plain}   
\bibliography{../../biblios/main.bib}

\input{qm2pi.rhodetails}

\end{document}

 

%\ifpdf
%\usepackage[pdftex]{graphicx}
%\else
%\usepackage{graphicx}
%\fi

 % \ifpdf
%  \usepackage{pdfsync}
%  \if


%\title{Brief Article}
%\author{David F. Snyder}
%\author{L.G. Meredith}

%\address{Dept. of Math., Texas State University--San Marcos, San Marcos, TX 78666}
       
\pagestyle{empty}


\begin{document}

\lstset{language=[Objective]Caml,frame=shadowbox}

\documentclass[12pt]{llncs}
%\documentclass{jktr}

\usepackage[pdftex]{hyperref}                   
\usepackage {listings}
\usepackage {mathpartir}
\usepackage{bcprules}
%\usepackage{listings}
                       
\usepackage{graphicx} 
%\usepackage[margins=2.5cm,nohead,nofoot]{geometry}
%\usepackage{geometry}
\usepackage{amsfonts}
\usepackage{amstext}
\usepackage{latexsym}
\usepackage{amssymb}
\usepackage{color}


%\include{myPreamble}
\include{qm2pi.local} 

%\ifpdf
%\usepackage[pdftex]{graphicx}
%\else
%\usepackage{graphicx}
%\fi

 % \ifpdf
%  \usepackage{pdfsync}
%  \if


%\title{Brief Article}
%\author{David F. Snyder}
%\author{L.G. Meredith}

%\address{Dept. of Math., Texas State University--San Marcos, San Marcos, TX 78666}
       
\pagestyle{empty}


\begin{document}

\lstset{language=[Objective]Caml,frame=shadowbox}

\input{qm2pi.front}

% section front matter (end)

\input{qm2pi.intro} 
 
% section introduction (end)

% \input{qm2pi.knotations} 

% section notation (end)

\input{qm2pi.process.calculi} 

% section concurrent_process_calculi_and_spatial_logics_ (end)
    
%\input{qm2pi.knots2pi} 

%\input{qm2pi.trefoil} 

%\input{qm2pi.mainthm} 

% subsection basic_interpretation (end)

%\input{qm2pi.rho.presentation} 
\subsection{The syntax and semantics of the notation system}\label{sub:the_syntax_and_semantics_of_the_notation_system} % (fold)

We now summarize a technical presentation of the calculus that
embodies our theory of dynamics. The typical presentation of such a
calculus follows the style of giving generators and relations on
them. The grammar, below, describing term constructors, freely
generates the set of processes, $\Proc$. This set is then quotiented
by a relation known as structural congruence and it is over this set
that the notion of dynamics is expressed. This presentation is
essentially that of \cite{MeredithR05} with the addition of
polyadicity and summation. For readability we have relegated some of
the technical subtleties to an appendix.

\subsubsection{Process grammar}\label{subsub:process_grammar}

\begin{mathpar}
  \inferrule* [lab=synchronization] {} {{M} \bc \pzero \;|\; x?F \;|\; x!C }
  \and
  \inferrule* [lab=abstraction] {} {{F} \bc (x)P}
  \and
  \inferrule* [lab=concretion] {} {{C} \bc \langle Q \rangle}
  \and
  \inferrule* [lab=process] {} {{P,Q} \bc M \;| \;P|Q \;|\; @{x}}
  \and
  \inferrule* [lab=name] {} {{x} \bc \quotep{P}}
\end{mathpar} 

Note that $\vec{x}$ (resp. $\vec{P}$) denotes a vector of names
(resp. processes) of length $|\vec{x}|$ (resp. $|\vec{P}|$). We adopt
the following useful abbreviations.

\begin{mathpar}
   x?(\vec{y}).P := x.(\vec{y})P \and  x\clift{\vec{P}} := x.\clift{\vec{P}}
   \and x!(y) := \lift{x}{\dropn{y}}
   \and \Pi_{i=0}^{n-1}P_i := P_0 | \ldots | P_{n-1}
\end{mathpar}

\subsubsection{Structural congruence}

\paragraph{Free and bound names and alpha-equivalence.} At the
core of structural equivalence is alpha-equivalence which identifies
process that are the same up to a change of variable. Formally, we
recognize the distinction between free and bound names. The free names
of a process, $\freenames{P}$, may be calculated recursively as
follows:

\begin{mathpar}
\freenames{\pzero} := \emptyset
  \and \\
  \freenames{x?(y).P} := \{ x \} \cup (\freenames{P} \setminus \{ y \})
  \and 
  \freenames{x!\langle P \rangle} := \{ x \} \cup \{ P \} 
  \and \\
  \freenames{P|Q} := \freenames{P} \cup \freenames{Q}
  \and \\
  \freenames{@{x}} := \{ x \}
\end{mathpar}

$\pi$
$\quotep{\pi}$

$\freenames{-} : \pi \to \mathcal{P}(\quotep{\pi})$

\begin{eqnarray*}
  \freenames{\pzero} & := & \emptyset \\
  \freenames{x?(y).P} & := & \{ x \} \cup (\freenames{P} \setminus \{ y \}) \\
  \freenames{x!\langle P \rangle} & := & \{ x \} \cup \{ P \} \\
  \freenames{P|Q} & := & \freenames{P} \cup \freenames{Q} \\
  \freenames{\dropn{x}} & := & \{ x \}
\end{eqnarray*}

The bound names of a process, $\boundnames{P}$, are those names occurring in $P$
that are not free. For example, in $x?(y).0$, the name $x$ is free, while $y$ is bound.

\begin{mathpar}
  \inferrule* [lab=monoidal-laws] {} { P|Q \equiv Q|P \and P|0 \equiv P \and P|(Q|R) \equiv (P|Q)|R }
\end{mathpar}

\begin{mathpar}
  \inferrule* [lab=alpha-equivalence] {} { (x)P \equiv (y)P\{y/x\} \and y \not\in \freenames{P} }
\end{mathpar}

\begin{definition}
Then two processes, $P,Q$, are alpha-equivalent if $P = Q\{\vec{y}/\vec{x}\}$ for
some $\vec{x} \in \boundnames{Q},\vec{y} \in \boundnames{P}$, where $Q\{\vec{y}/\vec{x}\}$
denotes the capture-avoiding substitution of $\vec{y}$ for $\vec{x}$ in $Q$.
\end{definition}

\begin{definition}
  The {\em structural congruence} \cite{SangiorgiWalker} , $\equiv$,
  between processes is the least congruence containing
  alpha-equivalence, satisfying the abelian monoid laws
  (associativity, commutativity and $\pzero$ as identity) for parallel
  composition $|$ and for summation $+$.
\end{definition}

\subsection{Name equivalence}

We take name equivalence, written $\nameeq$, to be the smallest
equivalence relation generated by the following rules.

\begin{mathpar}
\inferrule*[lab=Quote-drop]
{ }
{ \quotep{@{x}} \nameeq x }

\inferrule*[lab=Struct-equiv]
{ P \scong Q }
{ \quotep{P} \nameeq \quotep{Q} }
\end{mathpar}

The astute reader will have noticed that the mutual recursion of names
and processes imposes a mutual recursion on alpha-equivalence and
structural equivalence via name-equivalence. Fortunately, all of this
works out pleasantly and we may calculate in the natural way, free of
concern. The reader interested in the details is referred to the
appendix \ref{appendix:rho_details}.

\subsection{Substitution}

We use $\Proc$ for the set of processes, $\QProc$ for the set of
names, and $\id{\{}\vec{y} / \vec{x} \id{\}}$ to denote partial maps,
$s : \QProc \rightarrow \QProc$. A map, $s$ lifts, uniquely, to a map
on process terms, $\widehat{s} : \Proc \rightarrow \Proc$ by the
following equations.

\begin{mathpar}
  (0) \psubstp{Q}{P} := 0 \\
  (R \juxtap S) \psubstp{Q}{P}
  :=    
  (R)\psubstp{Q}{P} \juxtap (S) \psubstp{Q}{P} \\
  (x?(y).R) \psubstp{Q}{P}    
  :=    
  (x)\substp{Q}{P} (z)\concat( (R \psubstn{z}{y}) \psubstp{Q}{P} ) \\
  (\lift{x}{R}) \psubstp{Q}{P}  
  :=
  \lift{(x)\substp{Q}{P}}{ R \psubstp{Q}{P} } \\
%   (\dropn{x})  \psubstp{Q}{P}       
%   := 
%   \left\{ 
%     \begin{array}{ccc} 
%       \dropn{\quotep{Q}} & & x \nameeq \quotep{P} \\
%       \dropn{x} & & otherwise \\
%     \end{array}
%   \right. 
  (\dropn{x})  \psubstp{Q}{P}       
  := 
  \left\{ 
    \begin{array}{ccc} 
      Q & & x \nameeq \quotep{P} \\
      \dropn{x} & & otherwise \\
    \end{array}
  \right.
\end{mathpar}
 

where

\begin{eqnarray}
  (x)\id{\{} \lpquote Q \rpquote / \lpquote P \rpquote \id{\}}            = 
  \left\{ 
    \begin{array}{ccc}
      \lpquote Q \rpquote & & x \nameeq \lpquote P \rpquote \\
      x & & otherwise \\
    \end{array}
  \right. \nonumber
\end{eqnarray}

and $z$ is chosen distinct from $\quotep{P}$, $\quotep{Q}$, the free
names in $Q$, and all the names in $R$. Our $\alpha$-equivalence will
be built in the standard way from this substitution.

\begin{remark}\label{rem:no_self_referential_names}
  One consequence of these definitions is that $\forall P. \quotep{P}
  \not\in \freenames{P}$.
\end{remark}

\subsection{ Dynamic quote: an example }

Anticipating something of what's to come, consider applying the
substitution, $\widehat{\id{\{}u / z \id{\}}}$, to the following pair
of processes, $\lift{w}{y!(z)}$ and $w[ \lpquote y!(z) \rpquote ]$.

\begin{eqnarray}
	\lift{w}{y!(z)}\widehat{\id{\{}u / z \id{\}}}
		& = &
		\lift{w}{y!(u)} \nonumber\\
	w[ \lpquote y!(z) \rpquote ] \widehat{ \id{\{}u / z \id{\}} }
		& = &
		w[ \lpquote y!(z) \rpquote ] \nonumber
\end{eqnarray}

Because the body of the process between quotes is impervious to
substitution, we get radically different answers. In fact, by
examining the first process in an input context,
e.g. $x?(z).\lift{w}{y!(z)}$, we see that the process under the lift
operator may be shaped by prefixed inputs binding a name inside it. In
this sense, the lift operator will be seen as a way to dynamically
construct processes before reifying them as names.

Finally equipped with these standard features we can present the
dynamics of the calculus.

\subsubsection{Operational semantics} 

Finally, we introduce the computational dynamics. What marks these
algebras as distinct from other more traditionally studied algebraic
structures, e.g. vector spaces or polynomial rings, is the manner in
which dynamics is captured. In traditional structures, dynamics is typically
expressed through morphisms between such structures, as in linear maps
between vector spaces or morphisms between rings. In algebras
associated with the semantics of computation, the dynamics is
expressed as part of the algebraic structure itself, through a
reduction reduction relation typically denoted by $\red$. Below, we
give a recursive presentation of this relation for the calculus used
in the encoding.

$\red \subseteq \pi \times \pi$
$\red : \pi \to \mathcal{P}(\pi)$

\begin{mathpar}
  \inferrule* [lab=Comm] { \textsf{match}( x_{src}, x_{trgt} ) } { x_{trgt}?(y)P \; | \; x_{src}!\langle {Q} \rangle \red P\{\quotep{Q}/y}\} }
  \and \\
  \inferrule* [lab=Par] {{P} \red {P}'} {{{P} | {Q}} \red {{P}' | {Q}}}
  \and
  \inferrule* [lab=Equiv]{{{P} \scong {P}'} \andalso {{P}' \red {Q}'} \andalso {{Q}' \scong {Q}}}{{P} \red {Q}}
\end{mathpar}

\begin{eqnarray*}
  match_{\equiv} (\quotep{P},\quotep{Q}) & := & P \equiv Q \\
  match_{\dagger}(\quotep{P},\quotep{Q}) & := & \forall R. P|Q \red^{*} R => R \red^{*} 0 \\
  match_{K}(\quotep{P},\quotep{Q}) & := & K \mbox{ for some context } K
\end{eqnarray*}

$u?(x)P | u!\langle Q \rangle \red P\{\quotep{Q}/x\}$

%We write $\wred$ for $\red^*$, and $P\red$ if $\exists Q $ such that $ P \red Q$.
We write $P\red$ if $\exists Q $ such that $ P \red Q$ and $P\not\red$, otherwise.

\section{Replication}

As mentioned before, it is known that replication (and hence
recursion) can be implemented in a higher-order process algebra
\cite{SangiorgiWalker}. As our first example of calculation with the
machinery thus far presented we give the construction explicitly in
the {\rhoc}.

\begin{eqnarray}
	D_{x} & := & \prefix{x}{y}{(\binpar{\outputp{x}{y}}{@{y}})} \nonumber\\
	\bangp_{x}{P} & := & \binpar{{x}!\langle{\binpar{D_{x}}{P}}\rangle}{D_{x}} \nonumber
\end{eqnarray}

\begin{eqnarray}
	\bangp_{x}{P} & & \nonumber\\
	=
	& {x}!\langle{(\prefix{x}{y}{(\outputp{x}{y} | @{y})) | P}}\rangle 
	      | \prefix{x}{y}{(\outputp{x}{y} | @{y})} & \nonumber\\
	\red
	& (\outputp{x}{y} | @{y})\substn{\quotep{(\prefix{x}{y}{(@{y} | \outputp{x}{y})) | P}}}{y} & \nonumber\\
	=
	& \outputp{x}{\quotep{(\prefix{x}{y}{(\outputp{x}{y} | @{y})) | P}}}
	  | {(\prefix{x}{y}{(\outputp{x}{y} | @{y})) | P}} & \nonumber\\
	\red
	& \ldots & \nonumber\\
	\red^*
	& P | P | \ldots & \nonumber
\end{eqnarray}

Of course, this encoding, as an implementation, runs away, unfolding
$\bangp{P}$ eagerly. A lazier and more implementable replication
operator, restricted to input-guarded processes, may be obtained as follows.

\begin{eqnarray}
\bangp{\prefix{u}{v}{P}} 
	:= 
	\binpar{\lift{x}{\prefix{u}{v}{(\binpar{D(x)}{P})}}}{D(x)} \nonumber
\end{eqnarray}

\begin{remark}
  Note that the lazier definition still does not deal with summation
  or mixed summation (i.e. sums over input and output). The reader is
  invited to construct definitions of replication that deal with these
  features. 

  Further, the definitions are parameterized in a name, $x$. Can you,
  gentle reader, make a definition that eliminates this parameter and
  guarantees no accidental interaction between the replication
  machinery and the process being replicated -- i.e. no accidental
  sharing of names used by the process to get its work done and the
  name(s) used by the replication to effect copying. This latter
  revision of the definition of replication is crucial to obtaining
  the expected identity $!!P \sim !P$.
\end{remark}

\begin{remark}\label{rem:paradoxical_combinator}
  The reader familiar with the lambda calculus will have noticed the
  similarity between $D$ and the paradoxical combinator.

  [Ed. note: the existence of this seems to suggest we have to be more
  restrictive on the set of processes and names we admit if we are to
  support no-cloning.]
\end{remark}

\subsubsection{Bisimulation}

The computational dynamics gives rise to another kind of equivalence,
the equivalence of computational behavior. As previously mentioned
this is typically captured \emph{via} some form of bisimulation.

% The notion we use in this paper is weak barbed bisimulation
% \cite{milner91polyadicpi}.

The notion we use in this paper is derived from weak barbed
bisimulation \cite{milner91polyadicpi}. 

\begin{definition}
An \emph{observation relation}, $\downarrow_{\mathcal N}$, over a set
of names, $\mathcal N$, is the smallest relation satisfying the rules
below.

\infrule[Out-barb]{y \in {\mathcal N}, \; x \nameeq y}
		  {\outputp{x}{v} \downarrow_{\mathcal N} x}
\infrule[Par-barb]{\mbox{$P\downarrow_{\mathcal N} x$ or $Q\downarrow_{\mathcal N} x$}}
		  {\binpar{P}{Q} \downarrow_{\mathcal N} x}

We write $P \Downarrow_{\mathcal N} x$ if there is $Q$ such that 
$P \wred Q$ and $Q \downarrow_{\mathcal N} x$.
\end{definition}

\begin{definition}
%\label{def.bbisim}
An  ${\mathcal N}$-\emph{barbed bisimulation} over a set of names, ${\mathcal N}$, is a symmetric binary relation 
${\mathcal S}_{\mathcal N}$ between agents such that $P\rel{S}_{\mathcal N}Q$ implies:
\begin{enumerate}
\item If $P \red P'$ then $Q \wred Q'$ and $P'\rel{S}_{\mathcal N} Q'$.
\item If $P\downarrow_{\mathcal N} x$, then $Q\Downarrow_{\mathcal N} x$.
\end{enumerate}
$P$ is ${\mathcal N}$-barbed bisimilar to $Q$, written
$P \wbbisim_{\mathcal N} Q$, if $P \rel{S}_{\mathcal N} Q$ for some ${\mathcal N}$-barbed bisimulation ${\mathcal S}_{\mathcal N}$.
\end{definition}

$\mathcal{R} \subseteq \pi \times \pi$

$P \mathcal{R} Q => \forall P'. P \red P' \Rightarrow \exists Q'. Q \red Q', P' \mathcal{R} Q'$

$P \vdash x \Rightarrow Q \vdash x$

\begin{mathpar}
  \inferrule*[lab=Out-barb]{x \nameeq y}{{y}!\langle{Q}\rangle \vdash x}
  \and
  \inferrule*[lab=Par-barb]{\mbox{$P\vdash x$ or $Q\vdash x$}}{\binpar{P}{Q} \vdash x}
\end{mathpar}

\subsubsection{Contexts}

One of the principle advantages of computational calculi like the
$\pi$-calculus is a well-defined notion of context,
contextual-equivalence and a correlation between
contextual-equivalence and notions of bisimulation. The notion of
context allows the decomposition of a process into (sub-)process and
its syntactic environment, its context. Thus, a context may be
thought of as a process with a ``hole'' (written $\Box$) in it. The
application of a context $M$ to a process $P$, written $M[P]$, is
tantamount to filling the hole in $M$ with $P$. In this paper we do
not need the full weight of this theory, but do make use of the notion
of context in the proof the main theorem. 

\begin{mathpar}
  \inferrule* [lab=summation] {} {{M_{M},M_{N}} \bc \Box \;|\; x.M_{A} \;|\; M_{M}+M_{N}}
  \and
  \inferrule* [lab=agent] {} {{M_{A}} \bc (\vec{x})M_{P} \;| \; \clift{P_0,\ldots,M_{P},\ldots,P_N}}
  \and \\
  \inferrule* [lab=process] {} {{M_{P}} \bc M_{N} \;| \;P|M_{P} }
\end{mathpar} 

\begin{mathpar}
  \inferrule* [lab=sychronization] {} {M_{N} \bc \Box \;|\; x?M_{F} \;|\; x!M_{C}}
  \and
  \inferrule* [lab=abstraction] {} {{M_{F}} \bc (x)M_{P} }
  \and
  \inferrule* [lab=concretion] {} {{M_{C}} \bc \langle M_{P} \rangle }
  \and \\
  \inferrule* [lab=process] {} {{M_{P}} \bc M_{N} \;| \;P|M_{P} }
\end{mathpar}

\begin{definition}[contextual application] Given a context $M$, and
  process $P$, we define the \emph{contextual application}, $M[P] :=
  M\{P/\Box\}$. That is, the contextual application of M to P is the
  substitution of $P$ for $\Box$ in $M$.
\end{definition}

$\meaningof{-} : L \to \mathcal{P}(\pi)$

\begin{mathpar}
  \inferrule* [lab=collection] {} {\meaningof{true} = \pi, \and \meaningof{~E} = \pi \setminus \meaningof{E}, \and \meaningof{E_{1} \& E_{2}} = \meaningof{E_{1}} \cap \meaningof{E_{2}}}
\end{mathpar}

\begin{mathpar}
  \inferrule* [lab=structure] {} {\meaningof{0} = \{ P \in \pi | P \equiv 0 \}, \and \\ \meaningof{E_1 | E_2} = \{ P \in \pi | P \equiv P_{1} | P_{2}, P_{1} \in \meaningof{E_{1}}, P_{2} \in \meaningof{E_2}\} }
\end{mathpar}

\begin{mathpar}
 \inferrule* [lab=behavior] {} {\meaningof{\langle a?b \rangle E} = \{ P \in \pi | P \equiv Q | u?(y)P', \\ \and \\\\ \and \\ \;\;\; u \in \meaningof{a}, \forall z.P'\{z/y\} \in \meaningof{E\{z/b\}}\}, \and \\ \meaningof{a!E} = \{ P \in \pi | P \equiv Q | x!\langle P' \rangle, x \in \meaningof{a} P' \in \meaningof{E}\} }
\end{mathpar}

\begin{mathpar}
 \inferrule* [lab=nominal] {} {\meaningof{\quotep{E}} = \{ \quotep{P} \in \quotep{\pi} | P \in \meaningof{E} \}, \and \meaningof{\quotep{P}} = \{ \quotep{Q} \in \quotep{\pi} | P \equiv Q \} \and \\ \meaningof{@\quotep{E}} = \{ P \in \pi | P \equiv @x, x \in \meaningof{E} \}}
\end{mathpar}

\begin{eqnarray*}
  \\
  \meaningof{-} : TS \to ST
\end{eqnarray*}

\begin{eqnarray*}
  \\
  L : TS \to ST
\end{eqnarray*}

\begin{eqnarray*}
  \\
  P \models E \iff P \in \meaningof{E}
\end{eqnarray*}

\begin{eqnarray*}
  P \approx_{L} Q \iff \forall E \in L. P \models E \iff Q \models E
\end{eqnarray*}

\begin{eqnarray*}
  P \approx_{K} Q
\end{eqnarray*}

\begin{eqnarray*}
  P \approx Q
\end{eqnarray*}

$\approx_{K} = \approx = \approx_{L}$

\subsubsection{Contextual duality}

Note that contexts extend the quotation operation to a family of
operations from processes to names. Given a context, $M$, we can
define a \emph{nominal context}, $\quotep{M}$ by $\quotep{M}[P] :=
\quotep{M[P]}$. To foreshadow what is to come we observe that these
operations enjoy a duality with processes very much like the duality
between vectors and maps from vectors to scalars.

Further, because the calculus is essentially higher-order, we have a
correspondence between contexts and processes. More specifically,
given a name $x$ and a context $M$ we can construct $M^{*}_{x}$ such
that 

\begin{mathpar}
  M^{*}_{x} | \lift{x}{P} \red M[P]
\end{mathpar}

namely,

\begin{mathpar}
  M^{*}_{x} := x?(u).M[\dropn{u}]
\end{mathpar}

The dependence of $M^{*}_{x}$ on a name makes it an abstraction, 

\begin{mathpar}
  M^{*} := (x)x?(u).M[\dropn{u}]
\end{mathpar}

\subsection{Additional notation}

It will sometimes be convenient to denote the process a name
quotes. We already have the notation $x = \quotep{P}$, but it will be
convenient to introduce an alternate notation, $\procn{x}$, when we
want to emphasize the connection to the use of the name. Note that, by
virtue of name equivalence, $\quotep{\procn{x}} \nameeq x$; so, the
notation is consistent with previous definitions.

Further, because names have structure it is possible to effect
substitutions on the basis of that structure. This means we need to
upgrade our notation for substitutions, which we accomplish by
adapting comprehension notation. Thus,

\begin{mathpar}
  P\{ y / x : x \in S \}
\end{mathpar}

is interpreted to mean the process derived from P by replacing (in a
capture-avoiding manner) each occurrence of $x$ in $S$ by $y$. For example,

\begin{mathpar}
  P\{ \quotep{\procn{x}|\procn{x}} / x : x \in \freenames{P} \}
\end{mathpar}

will replace each (occurrence) of a free name $x$ in $P$ by
$\quotep{\procn{x}|\procn{x}}$.

Also, we will avail ourselves of the notation $x^{L}$ and $x^{R}$ to
denote injections of a name into disjoint copies of the name
space. There are numerous ways to accomplish this. One example can be
found in \cite{MeredithR05}. This notation overloads to vectors of
names: $\vec{x}^{\pi} := (x_{i}^{\pi} \; : \; 0 \leq i < |\vec{x}| )$ where $\pi \in \{L,R\}$.

We also use $P^{\Box} := P|\Box$.

In \cite{MeredithR05} an interpretation of the new operator is
given. It turns out that there are several possible interpretations
all enjoying the requisite algebraic properties of the operator (see
\cite{milner91polyadicpi}). We will therefore make liberal use of
$(\nu\; \vec{x})P$.

% subsection the_syntax_and_semantics_of_the_notation_system (end)   

\input{qm2pi.qmops} 

\input{qm2pi.sterngerlach} 

\input{qm2pi.metric} 

% section concurrent_process_calculi (end)

%\input{qm2pi.proofsketch}

% section proof sketch (end)

%\input{qm2pi.slviaknots} 

% section spatial logic via knots (end)

\input{qm2pi.conclusion}

% section conclusion (end)

%\input{qm2pi.dtcodes} 

% section wiring algorithm (end)

\input{qm2pi.ack} 

% section acknowledgments (end)

\newpage


\bibliographystyle{plain}   
\bibliography{../../biblios/main.bib}

\input{qm2pi.rhodetails}

\end{document}



% section front matter (end)

\section{Introduction}\label{sec:introduction} % (fold)
In this draft of the material i am going to have to dispense with the
usual writing conventions adopted in papers on these topics. i'm going
to have adopt whatever tone i need at the time i'm writing up the
calculations. Sometimes this may be very conversational; others it may
be the barest mathematical grunts; others still it may be that i have
lifted text from one of my other papers because the exposition of some
point was better said there. i hope that my readers are not unduly put
out by this decision. i'm not doing this to flout convention or be
rebellious. i find these calculations very technically challenging. To
keep everything going technically, something has to give; i have to
let go of some cognitive burden. So, the academic writing style --
with all of its trade-offs in terms of facilitating technical
communication -- is what i'm letting go of. Perhaps subsequent drafts
can be tightened and polished, but for now, i'm going to speak as if
we were sitting together in a coffee shop with a laptop, wifi and a
pad of paper and a pencil.

So, here's what i have to say. We -- you and i, comfortably ensconced
in our coffee shop and well-equipped with our tools -- can realize and
carry out the calculations of quantum mechanics over a very different
formal theory of dynamics, a formal theory of dynamics that
corresponds to a theory of concurrent computation with
\emph{reflection}. It has the advantage that the underlying theory is
already `quantized', but supports analogues all of the continuuous
operations. Strikingly, this underlying theory has recently been
connected with a notion of metric that we can show, by calculating
together, coincides with the metric induced by the inner product.

There are a lot of reasons why you might be interested in seeing
calculations of this form. Here's why i'm interested. For the past
several centuries there has been no competitor to the ``Newtonian''
account of dynamics. As a result the predominant share of accounts of
dynamical systems and situations have had to be formulated in terms of
the Newtonian machinery. i view this as an intellectually dangerous
position to occupy. Everything, despite it's intrinsic shape, turns
into a nail to be hit with this hammer. Recently, however, the theory
of computation has matured to the point where we have candidates for
theories of dynamics that offer very different perspective on
reasoning about dynamical systems and situations. Testing these
candidates against very successful accounts of dynamical situations,
like quantum mechanics, is going to give us some sense of how mature
they are and some measure of the quality of these accounts of
dynamics.

\subsection{Summary of contributions and outline of paper}

So, we're going to develop an interpretation of the operations of
quantum mechanics normally interpreted by Hilbert spaces and
operators. We're going to do this over a theory of computation. Note
that this is very different than the usual quantum computation program
which develops notions of computation over quantum mechanics. Rather,
we are developing a story that aligns with Wheeler's slogan: It from
Bit. To do this we will first provide an account of the theory of
computation at play here. Then we will dive into a calculation-driven
interpretation of the operations of quantum mechanics.

The reason we take this approach is that -- until very recently --
there hasn't been an axiomatic account of quantum mechanics. As a
result there has been no sharp delineation of the mathematical theory
supporting interpretation of the physical theory and the physical
theory, itself. So, ambient features of the maths are free to be
exploited (or supressed) without a real accounting of their physical
relevance. There is no sharp statement ``here's the physical theory''
qua \emph{theory} and ``here's the mathematical interpretation''
enabling a judgment of how faithful the interpretation is -- apart
from experimental observation. When there is an axiomatic account we
can judge how well a given mathematical formalism supports an
interpretation of the axioms, independent of
experimentation. Likewise, we can judge how well we have captured our
physical evidence and experience with our axiomatics, independent of
any specific mathematical implementation, with accidental detail that
may or may not have physical significance. 

In lieu of a fully fleshed out and vetted axiomatic account of quantum
mechanics, interpreting the operational notions in service of modeling
physical systems will have to suffice. In other words, we are not in
the business of providing a model of Hilbert spaces and operators. We
are in the business of providing a model of quantum mechanics because
we are motivated by testing our notions of dynamics against physical
theory; and, the predictive calculations of the physical theory must
serve as the best formulation -- shy of a fully fleshed out axiomatic
account -- of the physical theory itself (as they have for scientific
theories since time immemorial). Put another way, despite a
whole-hearted commitment to an It-from-Bit ontology, we are firmly
aligned with the shut-up-and-calculate camp as the best way to obtain
results either from the physical perspective or as a quality assurance
measure of our fledgling theory of dynamics.

In detail, we present a reflective process calculus. Then we develop
intuitive correspondences between the notions available in this
calculus and the usual physical notions supporting quantum mechanical
calculations. Thus, 

\begin{table}[htp]
  \center{
    \fbox{
      \begin{tabular}{c|c}
        quantum mechanics & process calculus \\
        \hline
        scalar & name \\
        state vector & process \\
        dual & contextual duals \\
        matrix & formal sums of process-context-dual pairs \\
        orthogonality & process annihilation \\
        inner product & execution-formula + quoting
      \end{tabular}
    }
  }
  \caption{QM - process calculi correspondences}
\end{table}

Then we tighten up these intuitions to operational definitions. We
employ the Dirac notation as the best proxy we can find for an
abstract syntax of the quantum mechanical notions. The definitions we
develop put us in contact with equational constraints coming from the
theory that we demonstrate the definitions and calculations satisfy.

This puts us in a position to shut up and calculate for the
Stern-Gerlach experimental set up, showing how these predictive
calculations become calculations on processes in our theory of a
reflective process calculus.

Penultimately, we demonstrate that the notion of metric coming from
the inner product coincides with the notion of metric available from
the theory of bisimulation. This demonstration gives us the right to
think of space as arising from behavior. Finally, we consider where we
might go from the new vantage point we have obtained.

% section introduction (end) 
 
% section introduction (end)

% \documentclass[12pt]{llncs}
%\documentclass{jktr}

\usepackage[pdftex]{hyperref}                   
\usepackage {listings}
\usepackage {mathpartir}
\usepackage{bcprules}
%\usepackage{listings}
                       
\usepackage{graphicx} 
%\usepackage[margins=2.5cm,nohead,nofoot]{geometry}
%\usepackage{geometry}
\usepackage{amsfonts}
\usepackage{amstext}
\usepackage{latexsym}
\usepackage{amssymb}
\usepackage{color}


%\include{myPreamble}
\include{qm2pi.local} 

%\ifpdf
%\usepackage[pdftex]{graphicx}
%\else
%\usepackage{graphicx}
%\fi

 % \ifpdf
%  \usepackage{pdfsync}
%  \if


%\title{Brief Article}
%\author{David F. Snyder}
%\author{L.G. Meredith}

%\address{Dept. of Math., Texas State University--San Marcos, San Marcos, TX 78666}
       
\pagestyle{empty}


\begin{document}

\lstset{language=[Objective]Caml,frame=shadowbox}

\input{qm2pi.front}

% section front matter (end)

\input{qm2pi.intro} 
 
% section introduction (end)

% \input{qm2pi.knotations} 

% section notation (end)

\input{qm2pi.process.calculi} 

% section concurrent_process_calculi_and_spatial_logics_ (end)
    
%\input{qm2pi.knots2pi} 

%\input{qm2pi.trefoil} 

%\input{qm2pi.mainthm} 

% subsection basic_interpretation (end)

%\input{qm2pi.rho.presentation} 
\subsection{The syntax and semantics of the notation system}\label{sub:the_syntax_and_semantics_of_the_notation_system} % (fold)

We now summarize a technical presentation of the calculus that
embodies our theory of dynamics. The typical presentation of such a
calculus follows the style of giving generators and relations on
them. The grammar, below, describing term constructors, freely
generates the set of processes, $\Proc$. This set is then quotiented
by a relation known as structural congruence and it is over this set
that the notion of dynamics is expressed. This presentation is
essentially that of \cite{MeredithR05} with the addition of
polyadicity and summation. For readability we have relegated some of
the technical subtleties to an appendix.

\subsubsection{Process grammar}\label{subsub:process_grammar}

\begin{mathpar}
  \inferrule* [lab=synchronization] {} {{M} \bc \pzero \;|\; x?F \;|\; x!C }
  \and
  \inferrule* [lab=abstraction] {} {{F} \bc (x)P}
  \and
  \inferrule* [lab=concretion] {} {{C} \bc \langle Q \rangle}
  \and
  \inferrule* [lab=process] {} {{P,Q} \bc M \;| \;P|Q \;|\; @{x}}
  \and
  \inferrule* [lab=name] {} {{x} \bc \quotep{P}}
\end{mathpar} 

Note that $\vec{x}$ (resp. $\vec{P}$) denotes a vector of names
(resp. processes) of length $|\vec{x}|$ (resp. $|\vec{P}|$). We adopt
the following useful abbreviations.

\begin{mathpar}
   x?(\vec{y}).P := x.(\vec{y})P \and  x\clift{\vec{P}} := x.\clift{\vec{P}}
   \and x!(y) := \lift{x}{\dropn{y}}
   \and \Pi_{i=0}^{n-1}P_i := P_0 | \ldots | P_{n-1}
\end{mathpar}

\subsubsection{Structural congruence}

\paragraph{Free and bound names and alpha-equivalence.} At the
core of structural equivalence is alpha-equivalence which identifies
process that are the same up to a change of variable. Formally, we
recognize the distinction between free and bound names. The free names
of a process, $\freenames{P}$, may be calculated recursively as
follows:

\begin{mathpar}
\freenames{\pzero} := \emptyset
  \and \\
  \freenames{x?(y).P} := \{ x \} \cup (\freenames{P} \setminus \{ y \})
  \and 
  \freenames{x!\langle P \rangle} := \{ x \} \cup \{ P \} 
  \and \\
  \freenames{P|Q} := \freenames{P} \cup \freenames{Q}
  \and \\
  \freenames{@{x}} := \{ x \}
\end{mathpar}

$\pi$
$\quotep{\pi}$

$\freenames{-} : \pi \to \mathcal{P}(\quotep{\pi})$

\begin{eqnarray*}
  \freenames{\pzero} & := & \emptyset \\
  \freenames{x?(y).P} & := & \{ x \} \cup (\freenames{P} \setminus \{ y \}) \\
  \freenames{x!\langle P \rangle} & := & \{ x \} \cup \{ P \} \\
  \freenames{P|Q} & := & \freenames{P} \cup \freenames{Q} \\
  \freenames{\dropn{x}} & := & \{ x \}
\end{eqnarray*}

The bound names of a process, $\boundnames{P}$, are those names occurring in $P$
that are not free. For example, in $x?(y).0$, the name $x$ is free, while $y$ is bound.

\begin{mathpar}
  \inferrule* [lab=monoidal-laws] {} { P|Q \equiv Q|P \and P|0 \equiv P \and P|(Q|R) \equiv (P|Q)|R }
\end{mathpar}

\begin{mathpar}
  \inferrule* [lab=alpha-equivalence] {} { (x)P \equiv (y)P\{y/x\} \and y \not\in \freenames{P} }
\end{mathpar}

\begin{definition}
Then two processes, $P,Q$, are alpha-equivalent if $P = Q\{\vec{y}/\vec{x}\}$ for
some $\vec{x} \in \boundnames{Q},\vec{y} \in \boundnames{P}$, where $Q\{\vec{y}/\vec{x}\}$
denotes the capture-avoiding substitution of $\vec{y}$ for $\vec{x}$ in $Q$.
\end{definition}

\begin{definition}
  The {\em structural congruence} \cite{SangiorgiWalker} , $\equiv$,
  between processes is the least congruence containing
  alpha-equivalence, satisfying the abelian monoid laws
  (associativity, commutativity and $\pzero$ as identity) for parallel
  composition $|$ and for summation $+$.
\end{definition}

\subsection{Name equivalence}

We take name equivalence, written $\nameeq$, to be the smallest
equivalence relation generated by the following rules.

\begin{mathpar}
\inferrule*[lab=Quote-drop]
{ }
{ \quotep{@{x}} \nameeq x }

\inferrule*[lab=Struct-equiv]
{ P \scong Q }
{ \quotep{P} \nameeq \quotep{Q} }
\end{mathpar}

The astute reader will have noticed that the mutual recursion of names
and processes imposes a mutual recursion on alpha-equivalence and
structural equivalence via name-equivalence. Fortunately, all of this
works out pleasantly and we may calculate in the natural way, free of
concern. The reader interested in the details is referred to the
appendix \ref{appendix:rho_details}.

\subsection{Substitution}

We use $\Proc$ for the set of processes, $\QProc$ for the set of
names, and $\id{\{}\vec{y} / \vec{x} \id{\}}$ to denote partial maps,
$s : \QProc \rightarrow \QProc$. A map, $s$ lifts, uniquely, to a map
on process terms, $\widehat{s} : \Proc \rightarrow \Proc$ by the
following equations.

\begin{mathpar}
  (0) \psubstp{Q}{P} := 0 \\
  (R \juxtap S) \psubstp{Q}{P}
  :=    
  (R)\psubstp{Q}{P} \juxtap (S) \psubstp{Q}{P} \\
  (x?(y).R) \psubstp{Q}{P}    
  :=    
  (x)\substp{Q}{P} (z)\concat( (R \psubstn{z}{y}) \psubstp{Q}{P} ) \\
  (\lift{x}{R}) \psubstp{Q}{P}  
  :=
  \lift{(x)\substp{Q}{P}}{ R \psubstp{Q}{P} } \\
%   (\dropn{x})  \psubstp{Q}{P}       
%   := 
%   \left\{ 
%     \begin{array}{ccc} 
%       \dropn{\quotep{Q}} & & x \nameeq \quotep{P} \\
%       \dropn{x} & & otherwise \\
%     \end{array}
%   \right. 
  (\dropn{x})  \psubstp{Q}{P}       
  := 
  \left\{ 
    \begin{array}{ccc} 
      Q & & x \nameeq \quotep{P} \\
      \dropn{x} & & otherwise \\
    \end{array}
  \right.
\end{mathpar}
 

where

\begin{eqnarray}
  (x)\id{\{} \lpquote Q \rpquote / \lpquote P \rpquote \id{\}}            = 
  \left\{ 
    \begin{array}{ccc}
      \lpquote Q \rpquote & & x \nameeq \lpquote P \rpquote \\
      x & & otherwise \\
    \end{array}
  \right. \nonumber
\end{eqnarray}

and $z$ is chosen distinct from $\quotep{P}$, $\quotep{Q}$, the free
names in $Q$, and all the names in $R$. Our $\alpha$-equivalence will
be built in the standard way from this substitution.

\begin{remark}\label{rem:no_self_referential_names}
  One consequence of these definitions is that $\forall P. \quotep{P}
  \not\in \freenames{P}$.
\end{remark}

\subsection{ Dynamic quote: an example }

Anticipating something of what's to come, consider applying the
substitution, $\widehat{\id{\{}u / z \id{\}}}$, to the following pair
of processes, $\lift{w}{y!(z)}$ and $w[ \lpquote y!(z) \rpquote ]$.

\begin{eqnarray}
	\lift{w}{y!(z)}\widehat{\id{\{}u / z \id{\}}}
		& = &
		\lift{w}{y!(u)} \nonumber\\
	w[ \lpquote y!(z) \rpquote ] \widehat{ \id{\{}u / z \id{\}} }
		& = &
		w[ \lpquote y!(z) \rpquote ] \nonumber
\end{eqnarray}

Because the body of the process between quotes is impervious to
substitution, we get radically different answers. In fact, by
examining the first process in an input context,
e.g. $x?(z).\lift{w}{y!(z)}$, we see that the process under the lift
operator may be shaped by prefixed inputs binding a name inside it. In
this sense, the lift operator will be seen as a way to dynamically
construct processes before reifying them as names.

Finally equipped with these standard features we can present the
dynamics of the calculus.

\subsubsection{Operational semantics} 

Finally, we introduce the computational dynamics. What marks these
algebras as distinct from other more traditionally studied algebraic
structures, e.g. vector spaces or polynomial rings, is the manner in
which dynamics is captured. In traditional structures, dynamics is typically
expressed through morphisms between such structures, as in linear maps
between vector spaces or morphisms between rings. In algebras
associated with the semantics of computation, the dynamics is
expressed as part of the algebraic structure itself, through a
reduction reduction relation typically denoted by $\red$. Below, we
give a recursive presentation of this relation for the calculus used
in the encoding.

$\red \subseteq \pi \times \pi$
$\red : \pi \to \mathcal{P}(\pi)$

\begin{mathpar}
  \inferrule* [lab=Comm] { \textsf{match}( x_{src}, x_{trgt} ) } { x_{trgt}?(y)P \; | \; x_{src}!\langle {Q} \rangle \red P\{\quotep{Q}/y}\} }
  \and \\
  \inferrule* [lab=Par] {{P} \red {P}'} {{{P} | {Q}} \red {{P}' | {Q}}}
  \and
  \inferrule* [lab=Equiv]{{{P} \scong {P}'} \andalso {{P}' \red {Q}'} \andalso {{Q}' \scong {Q}}}{{P} \red {Q}}
\end{mathpar}

\begin{eqnarray*}
  match_{\equiv} (\quotep{P},\quotep{Q}) & := & P \equiv Q \\
  match_{\dagger}(\quotep{P},\quotep{Q}) & := & \forall R. P|Q \red^{*} R => R \red^{*} 0 \\
  match_{K}(\quotep{P},\quotep{Q}) & := & K \mbox{ for some context } K
\end{eqnarray*}

$u?(x)P | u!\langle Q \rangle \red P\{\quotep{Q}/x\}$

%We write $\wred$ for $\red^*$, and $P\red$ if $\exists Q $ such that $ P \red Q$.
We write $P\red$ if $\exists Q $ such that $ P \red Q$ and $P\not\red$, otherwise.

\section{Replication}

As mentioned before, it is known that replication (and hence
recursion) can be implemented in a higher-order process algebra
\cite{SangiorgiWalker}. As our first example of calculation with the
machinery thus far presented we give the construction explicitly in
the {\rhoc}.

\begin{eqnarray}
	D_{x} & := & \prefix{x}{y}{(\binpar{\outputp{x}{y}}{@{y}})} \nonumber\\
	\bangp_{x}{P} & := & \binpar{{x}!\langle{\binpar{D_{x}}{P}}\rangle}{D_{x}} \nonumber
\end{eqnarray}

\begin{eqnarray}
	\bangp_{x}{P} & & \nonumber\\
	=
	& {x}!\langle{(\prefix{x}{y}{(\outputp{x}{y} | @{y})) | P}}\rangle 
	      | \prefix{x}{y}{(\outputp{x}{y} | @{y})} & \nonumber\\
	\red
	& (\outputp{x}{y} | @{y})\substn{\quotep{(\prefix{x}{y}{(@{y} | \outputp{x}{y})) | P}}}{y} & \nonumber\\
	=
	& \outputp{x}{\quotep{(\prefix{x}{y}{(\outputp{x}{y} | @{y})) | P}}}
	  | {(\prefix{x}{y}{(\outputp{x}{y} | @{y})) | P}} & \nonumber\\
	\red
	& \ldots & \nonumber\\
	\red^*
	& P | P | \ldots & \nonumber
\end{eqnarray}

Of course, this encoding, as an implementation, runs away, unfolding
$\bangp{P}$ eagerly. A lazier and more implementable replication
operator, restricted to input-guarded processes, may be obtained as follows.

\begin{eqnarray}
\bangp{\prefix{u}{v}{P}} 
	:= 
	\binpar{\lift{x}{\prefix{u}{v}{(\binpar{D(x)}{P})}}}{D(x)} \nonumber
\end{eqnarray}

\begin{remark}
  Note that the lazier definition still does not deal with summation
  or mixed summation (i.e. sums over input and output). The reader is
  invited to construct definitions of replication that deal with these
  features. 

  Further, the definitions are parameterized in a name, $x$. Can you,
  gentle reader, make a definition that eliminates this parameter and
  guarantees no accidental interaction between the replication
  machinery and the process being replicated -- i.e. no accidental
  sharing of names used by the process to get its work done and the
  name(s) used by the replication to effect copying. This latter
  revision of the definition of replication is crucial to obtaining
  the expected identity $!!P \sim !P$.
\end{remark}

\begin{remark}\label{rem:paradoxical_combinator}
  The reader familiar with the lambda calculus will have noticed the
  similarity between $D$ and the paradoxical combinator.

  [Ed. note: the existence of this seems to suggest we have to be more
  restrictive on the set of processes and names we admit if we are to
  support no-cloning.]
\end{remark}

\subsubsection{Bisimulation}

The computational dynamics gives rise to another kind of equivalence,
the equivalence of computational behavior. As previously mentioned
this is typically captured \emph{via} some form of bisimulation.

% The notion we use in this paper is weak barbed bisimulation
% \cite{milner91polyadicpi}.

The notion we use in this paper is derived from weak barbed
bisimulation \cite{milner91polyadicpi}. 

\begin{definition}
An \emph{observation relation}, $\downarrow_{\mathcal N}$, over a set
of names, $\mathcal N$, is the smallest relation satisfying the rules
below.

\infrule[Out-barb]{y \in {\mathcal N}, \; x \nameeq y}
		  {\outputp{x}{v} \downarrow_{\mathcal N} x}
\infrule[Par-barb]{\mbox{$P\downarrow_{\mathcal N} x$ or $Q\downarrow_{\mathcal N} x$}}
		  {\binpar{P}{Q} \downarrow_{\mathcal N} x}

We write $P \Downarrow_{\mathcal N} x$ if there is $Q$ such that 
$P \wred Q$ and $Q \downarrow_{\mathcal N} x$.
\end{definition}

\begin{definition}
%\label{def.bbisim}
An  ${\mathcal N}$-\emph{barbed bisimulation} over a set of names, ${\mathcal N}$, is a symmetric binary relation 
${\mathcal S}_{\mathcal N}$ between agents such that $P\rel{S}_{\mathcal N}Q$ implies:
\begin{enumerate}
\item If $P \red P'$ then $Q \wred Q'$ and $P'\rel{S}_{\mathcal N} Q'$.
\item If $P\downarrow_{\mathcal N} x$, then $Q\Downarrow_{\mathcal N} x$.
\end{enumerate}
$P$ is ${\mathcal N}$-barbed bisimilar to $Q$, written
$P \wbbisim_{\mathcal N} Q$, if $P \rel{S}_{\mathcal N} Q$ for some ${\mathcal N}$-barbed bisimulation ${\mathcal S}_{\mathcal N}$.
\end{definition}

$\mathcal{R} \subseteq \pi \times \pi$

$P \mathcal{R} Q => \forall P'. P \red P' \Rightarrow \exists Q'. Q \red Q', P' \mathcal{R} Q'$

$P \vdash x \Rightarrow Q \vdash x$

\begin{mathpar}
  \inferrule*[lab=Out-barb]{x \nameeq y}{{y}!\langle{Q}\rangle \vdash x}
  \and
  \inferrule*[lab=Par-barb]{\mbox{$P\vdash x$ or $Q\vdash x$}}{\binpar{P}{Q} \vdash x}
\end{mathpar}

\subsubsection{Contexts}

One of the principle advantages of computational calculi like the
$\pi$-calculus is a well-defined notion of context,
contextual-equivalence and a correlation between
contextual-equivalence and notions of bisimulation. The notion of
context allows the decomposition of a process into (sub-)process and
its syntactic environment, its context. Thus, a context may be
thought of as a process with a ``hole'' (written $\Box$) in it. The
application of a context $M$ to a process $P$, written $M[P]$, is
tantamount to filling the hole in $M$ with $P$. In this paper we do
not need the full weight of this theory, but do make use of the notion
of context in the proof the main theorem. 

\begin{mathpar}
  \inferrule* [lab=summation] {} {{M_{M},M_{N}} \bc \Box \;|\; x.M_{A} \;|\; M_{M}+M_{N}}
  \and
  \inferrule* [lab=agent] {} {{M_{A}} \bc (\vec{x})M_{P} \;| \; \clift{P_0,\ldots,M_{P},\ldots,P_N}}
  \and \\
  \inferrule* [lab=process] {} {{M_{P}} \bc M_{N} \;| \;P|M_{P} }
\end{mathpar} 

\begin{mathpar}
  \inferrule* [lab=sychronization] {} {M_{N} \bc \Box \;|\; x?M_{F} \;|\; x!M_{C}}
  \and
  \inferrule* [lab=abstraction] {} {{M_{F}} \bc (x)M_{P} }
  \and
  \inferrule* [lab=concretion] {} {{M_{C}} \bc \langle M_{P} \rangle }
  \and \\
  \inferrule* [lab=process] {} {{M_{P}} \bc M_{N} \;| \;P|M_{P} }
\end{mathpar}

\begin{definition}[contextual application] Given a context $M$, and
  process $P$, we define the \emph{contextual application}, $M[P] :=
  M\{P/\Box\}$. That is, the contextual application of M to P is the
  substitution of $P$ for $\Box$ in $M$.
\end{definition}

$\meaningof{-} : L \to \mathcal{P}(\pi)$

\begin{mathpar}
  \inferrule* [lab=collection] {} {\meaningof{true} = \pi, \and \meaningof{~E} = \pi \setminus \meaningof{E}, \and \meaningof{E_{1} \& E_{2}} = \meaningof{E_{1}} \cap \meaningof{E_{2}}}
\end{mathpar}

\begin{mathpar}
  \inferrule* [lab=structure] {} {\meaningof{0} = \{ P \in \pi | P \equiv 0 \}, \and \\ \meaningof{E_1 | E_2} = \{ P \in \pi | P \equiv P_{1} | P_{2}, P_{1} \in \meaningof{E_{1}}, P_{2} \in \meaningof{E_2}\} }
\end{mathpar}

\begin{mathpar}
 \inferrule* [lab=behavior] {} {\meaningof{\langle a?b \rangle E} = \{ P \in \pi | P \equiv Q | u?(y)P', \\ \and \\\\ \and \\ \;\;\; u \in \meaningof{a}, \forall z.P'\{z/y\} \in \meaningof{E\{z/b\}}\}, \and \\ \meaningof{a!E} = \{ P \in \pi | P \equiv Q | x!\langle P' \rangle, x \in \meaningof{a} P' \in \meaningof{E}\} }
\end{mathpar}

\begin{mathpar}
 \inferrule* [lab=nominal] {} {\meaningof{\quotep{E}} = \{ \quotep{P} \in \quotep{\pi} | P \in \meaningof{E} \}, \and \meaningof{\quotep{P}} = \{ \quotep{Q} \in \quotep{\pi} | P \equiv Q \} \and \\ \meaningof{@\quotep{E}} = \{ P \in \pi | P \equiv @x, x \in \meaningof{E} \}}
\end{mathpar}

\begin{eqnarray*}
  \\
  \meaningof{-} : TS \to ST
\end{eqnarray*}

\begin{eqnarray*}
  \\
  L : TS \to ST
\end{eqnarray*}

\begin{eqnarray*}
  \\
  P \models E \iff P \in \meaningof{E}
\end{eqnarray*}

\begin{eqnarray*}
  P \approx_{L} Q \iff \forall E \in L. P \models E \iff Q \models E
\end{eqnarray*}

\begin{eqnarray*}
  P \approx_{K} Q
\end{eqnarray*}

\begin{eqnarray*}
  P \approx Q
\end{eqnarray*}

$\approx_{K} = \approx = \approx_{L}$

\subsubsection{Contextual duality}

Note that contexts extend the quotation operation to a family of
operations from processes to names. Given a context, $M$, we can
define a \emph{nominal context}, $\quotep{M}$ by $\quotep{M}[P] :=
\quotep{M[P]}$. To foreshadow what is to come we observe that these
operations enjoy a duality with processes very much like the duality
between vectors and maps from vectors to scalars.

Further, because the calculus is essentially higher-order, we have a
correspondence between contexts and processes. More specifically,
given a name $x$ and a context $M$ we can construct $M^{*}_{x}$ such
that 

\begin{mathpar}
  M^{*}_{x} | \lift{x}{P} \red M[P]
\end{mathpar}

namely,

\begin{mathpar}
  M^{*}_{x} := x?(u).M[\dropn{u}]
\end{mathpar}

The dependence of $M^{*}_{x}$ on a name makes it an abstraction, 

\begin{mathpar}
  M^{*} := (x)x?(u).M[\dropn{u}]
\end{mathpar}

\subsection{Additional notation}

It will sometimes be convenient to denote the process a name
quotes. We already have the notation $x = \quotep{P}$, but it will be
convenient to introduce an alternate notation, $\procn{x}$, when we
want to emphasize the connection to the use of the name. Note that, by
virtue of name equivalence, $\quotep{\procn{x}} \nameeq x$; so, the
notation is consistent with previous definitions.

Further, because names have structure it is possible to effect
substitutions on the basis of that structure. This means we need to
upgrade our notation for substitutions, which we accomplish by
adapting comprehension notation. Thus,

\begin{mathpar}
  P\{ y / x : x \in S \}
\end{mathpar}

is interpreted to mean the process derived from P by replacing (in a
capture-avoiding manner) each occurrence of $x$ in $S$ by $y$. For example,

\begin{mathpar}
  P\{ \quotep{\procn{x}|\procn{x}} / x : x \in \freenames{P} \}
\end{mathpar}

will replace each (occurrence) of a free name $x$ in $P$ by
$\quotep{\procn{x}|\procn{x}}$.

Also, we will avail ourselves of the notation $x^{L}$ and $x^{R}$ to
denote injections of a name into disjoint copies of the name
space. There are numerous ways to accomplish this. One example can be
found in \cite{MeredithR05}. This notation overloads to vectors of
names: $\vec{x}^{\pi} := (x_{i}^{\pi} \; : \; 0 \leq i < |\vec{x}| )$ where $\pi \in \{L,R\}$.

We also use $P^{\Box} := P|\Box$.

In \cite{MeredithR05} an interpretation of the new operator is
given. It turns out that there are several possible interpretations
all enjoying the requisite algebraic properties of the operator (see
\cite{milner91polyadicpi}). We will therefore make liberal use of
$(\nu\; \vec{x})P$.

% subsection the_syntax_and_semantics_of_the_notation_system (end)   

\input{qm2pi.qmops} 

\input{qm2pi.sterngerlach} 

\input{qm2pi.metric} 

% section concurrent_process_calculi (end)

%\input{qm2pi.proofsketch}

% section proof sketch (end)

%\input{qm2pi.slviaknots} 

% section spatial logic via knots (end)

\input{qm2pi.conclusion}

% section conclusion (end)

%\input{qm2pi.dtcodes} 

% section wiring algorithm (end)

\input{qm2pi.ack} 

% section acknowledgments (end)

\newpage


\bibliographystyle{plain}   
\bibliography{../../biblios/main.bib}

\input{qm2pi.rhodetails}

\end{document}

 

% section notation (end)

\input{qm2pi.process.calculi} 

% section concurrent_process_calculi_and_spatial_logics_ (end)
    
%\documentclass[12pt]{llncs}
%\documentclass{jktr}

\usepackage[pdftex]{hyperref}                   
\usepackage {listings}
\usepackage {mathpartir}
\usepackage{bcprules}
%\usepackage{listings}
                       
\usepackage{graphicx} 
%\usepackage[margins=2.5cm,nohead,nofoot]{geometry}
%\usepackage{geometry}
\usepackage{amsfonts}
\usepackage{amstext}
\usepackage{latexsym}
\usepackage{amssymb}
\usepackage{color}


%\include{myPreamble}
\include{qm2pi.local} 

%\ifpdf
%\usepackage[pdftex]{graphicx}
%\else
%\usepackage{graphicx}
%\fi

 % \ifpdf
%  \usepackage{pdfsync}
%  \if


%\title{Brief Article}
%\author{David F. Snyder}
%\author{L.G. Meredith}

%\address{Dept. of Math., Texas State University--San Marcos, San Marcos, TX 78666}
       
\pagestyle{empty}


\begin{document}

\lstset{language=[Objective]Caml,frame=shadowbox}

\input{qm2pi.front}

% section front matter (end)

\input{qm2pi.intro} 
 
% section introduction (end)

% \input{qm2pi.knotations} 

% section notation (end)

\input{qm2pi.process.calculi} 

% section concurrent_process_calculi_and_spatial_logics_ (end)
    
%\input{qm2pi.knots2pi} 

%\input{qm2pi.trefoil} 

%\input{qm2pi.mainthm} 

% subsection basic_interpretation (end)

%\input{qm2pi.rho.presentation} 
\subsection{The syntax and semantics of the notation system}\label{sub:the_syntax_and_semantics_of_the_notation_system} % (fold)

We now summarize a technical presentation of the calculus that
embodies our theory of dynamics. The typical presentation of such a
calculus follows the style of giving generators and relations on
them. The grammar, below, describing term constructors, freely
generates the set of processes, $\Proc$. This set is then quotiented
by a relation known as structural congruence and it is over this set
that the notion of dynamics is expressed. This presentation is
essentially that of \cite{MeredithR05} with the addition of
polyadicity and summation. For readability we have relegated some of
the technical subtleties to an appendix.

\subsubsection{Process grammar}\label{subsub:process_grammar}

\begin{mathpar}
  \inferrule* [lab=synchronization] {} {{M} \bc \pzero \;|\; x?F \;|\; x!C }
  \and
  \inferrule* [lab=abstraction] {} {{F} \bc (x)P}
  \and
  \inferrule* [lab=concretion] {} {{C} \bc \langle Q \rangle}
  \and
  \inferrule* [lab=process] {} {{P,Q} \bc M \;| \;P|Q \;|\; @{x}}
  \and
  \inferrule* [lab=name] {} {{x} \bc \quotep{P}}
\end{mathpar} 

Note that $\vec{x}$ (resp. $\vec{P}$) denotes a vector of names
(resp. processes) of length $|\vec{x}|$ (resp. $|\vec{P}|$). We adopt
the following useful abbreviations.

\begin{mathpar}
   x?(\vec{y}).P := x.(\vec{y})P \and  x\clift{\vec{P}} := x.\clift{\vec{P}}
   \and x!(y) := \lift{x}{\dropn{y}}
   \and \Pi_{i=0}^{n-1}P_i := P_0 | \ldots | P_{n-1}
\end{mathpar}

\subsubsection{Structural congruence}

\paragraph{Free and bound names and alpha-equivalence.} At the
core of structural equivalence is alpha-equivalence which identifies
process that are the same up to a change of variable. Formally, we
recognize the distinction between free and bound names. The free names
of a process, $\freenames{P}$, may be calculated recursively as
follows:

\begin{mathpar}
\freenames{\pzero} := \emptyset
  \and \\
  \freenames{x?(y).P} := \{ x \} \cup (\freenames{P} \setminus \{ y \})
  \and 
  \freenames{x!\langle P \rangle} := \{ x \} \cup \{ P \} 
  \and \\
  \freenames{P|Q} := \freenames{P} \cup \freenames{Q}
  \and \\
  \freenames{@{x}} := \{ x \}
\end{mathpar}

$\pi$
$\quotep{\pi}$

$\freenames{-} : \pi \to \mathcal{P}(\quotep{\pi})$

\begin{eqnarray*}
  \freenames{\pzero} & := & \emptyset \\
  \freenames{x?(y).P} & := & \{ x \} \cup (\freenames{P} \setminus \{ y \}) \\
  \freenames{x!\langle P \rangle} & := & \{ x \} \cup \{ P \} \\
  \freenames{P|Q} & := & \freenames{P} \cup \freenames{Q} \\
  \freenames{\dropn{x}} & := & \{ x \}
\end{eqnarray*}

The bound names of a process, $\boundnames{P}$, are those names occurring in $P$
that are not free. For example, in $x?(y).0$, the name $x$ is free, while $y$ is bound.

\begin{mathpar}
  \inferrule* [lab=monoidal-laws] {} { P|Q \equiv Q|P \and P|0 \equiv P \and P|(Q|R) \equiv (P|Q)|R }
\end{mathpar}

\begin{mathpar}
  \inferrule* [lab=alpha-equivalence] {} { (x)P \equiv (y)P\{y/x\} \and y \not\in \freenames{P} }
\end{mathpar}

\begin{definition}
Then two processes, $P,Q$, are alpha-equivalent if $P = Q\{\vec{y}/\vec{x}\}$ for
some $\vec{x} \in \boundnames{Q},\vec{y} \in \boundnames{P}$, where $Q\{\vec{y}/\vec{x}\}$
denotes the capture-avoiding substitution of $\vec{y}$ for $\vec{x}$ in $Q$.
\end{definition}

\begin{definition}
  The {\em structural congruence} \cite{SangiorgiWalker} , $\equiv$,
  between processes is the least congruence containing
  alpha-equivalence, satisfying the abelian monoid laws
  (associativity, commutativity and $\pzero$ as identity) for parallel
  composition $|$ and for summation $+$.
\end{definition}

\subsection{Name equivalence}

We take name equivalence, written $\nameeq$, to be the smallest
equivalence relation generated by the following rules.

\begin{mathpar}
\inferrule*[lab=Quote-drop]
{ }
{ \quotep{@{x}} \nameeq x }

\inferrule*[lab=Struct-equiv]
{ P \scong Q }
{ \quotep{P} \nameeq \quotep{Q} }
\end{mathpar}

The astute reader will have noticed that the mutual recursion of names
and processes imposes a mutual recursion on alpha-equivalence and
structural equivalence via name-equivalence. Fortunately, all of this
works out pleasantly and we may calculate in the natural way, free of
concern. The reader interested in the details is referred to the
appendix \ref{appendix:rho_details}.

\subsection{Substitution}

We use $\Proc$ for the set of processes, $\QProc$ for the set of
names, and $\id{\{}\vec{y} / \vec{x} \id{\}}$ to denote partial maps,
$s : \QProc \rightarrow \QProc$. A map, $s$ lifts, uniquely, to a map
on process terms, $\widehat{s} : \Proc \rightarrow \Proc$ by the
following equations.

\begin{mathpar}
  (0) \psubstp{Q}{P} := 0 \\
  (R \juxtap S) \psubstp{Q}{P}
  :=    
  (R)\psubstp{Q}{P} \juxtap (S) \psubstp{Q}{P} \\
  (x?(y).R) \psubstp{Q}{P}    
  :=    
  (x)\substp{Q}{P} (z)\concat( (R \psubstn{z}{y}) \psubstp{Q}{P} ) \\
  (\lift{x}{R}) \psubstp{Q}{P}  
  :=
  \lift{(x)\substp{Q}{P}}{ R \psubstp{Q}{P} } \\
%   (\dropn{x})  \psubstp{Q}{P}       
%   := 
%   \left\{ 
%     \begin{array}{ccc} 
%       \dropn{\quotep{Q}} & & x \nameeq \quotep{P} \\
%       \dropn{x} & & otherwise \\
%     \end{array}
%   \right. 
  (\dropn{x})  \psubstp{Q}{P}       
  := 
  \left\{ 
    \begin{array}{ccc} 
      Q & & x \nameeq \quotep{P} \\
      \dropn{x} & & otherwise \\
    \end{array}
  \right.
\end{mathpar}
 

where

\begin{eqnarray}
  (x)\id{\{} \lpquote Q \rpquote / \lpquote P \rpquote \id{\}}            = 
  \left\{ 
    \begin{array}{ccc}
      \lpquote Q \rpquote & & x \nameeq \lpquote P \rpquote \\
      x & & otherwise \\
    \end{array}
  \right. \nonumber
\end{eqnarray}

and $z$ is chosen distinct from $\quotep{P}$, $\quotep{Q}$, the free
names in $Q$, and all the names in $R$. Our $\alpha$-equivalence will
be built in the standard way from this substitution.

\begin{remark}\label{rem:no_self_referential_names}
  One consequence of these definitions is that $\forall P. \quotep{P}
  \not\in \freenames{P}$.
\end{remark}

\subsection{ Dynamic quote: an example }

Anticipating something of what's to come, consider applying the
substitution, $\widehat{\id{\{}u / z \id{\}}}$, to the following pair
of processes, $\lift{w}{y!(z)}$ and $w[ \lpquote y!(z) \rpquote ]$.

\begin{eqnarray}
	\lift{w}{y!(z)}\widehat{\id{\{}u / z \id{\}}}
		& = &
		\lift{w}{y!(u)} \nonumber\\
	w[ \lpquote y!(z) \rpquote ] \widehat{ \id{\{}u / z \id{\}} }
		& = &
		w[ \lpquote y!(z) \rpquote ] \nonumber
\end{eqnarray}

Because the body of the process between quotes is impervious to
substitution, we get radically different answers. In fact, by
examining the first process in an input context,
e.g. $x?(z).\lift{w}{y!(z)}$, we see that the process under the lift
operator may be shaped by prefixed inputs binding a name inside it. In
this sense, the lift operator will be seen as a way to dynamically
construct processes before reifying them as names.

Finally equipped with these standard features we can present the
dynamics of the calculus.

\subsubsection{Operational semantics} 

Finally, we introduce the computational dynamics. What marks these
algebras as distinct from other more traditionally studied algebraic
structures, e.g. vector spaces or polynomial rings, is the manner in
which dynamics is captured. In traditional structures, dynamics is typically
expressed through morphisms between such structures, as in linear maps
between vector spaces or morphisms between rings. In algebras
associated with the semantics of computation, the dynamics is
expressed as part of the algebraic structure itself, through a
reduction reduction relation typically denoted by $\red$. Below, we
give a recursive presentation of this relation for the calculus used
in the encoding.

$\red \subseteq \pi \times \pi$
$\red : \pi \to \mathcal{P}(\pi)$

\begin{mathpar}
  \inferrule* [lab=Comm] { \textsf{match}( x_{src}, x_{trgt} ) } { x_{trgt}?(y)P \; | \; x_{src}!\langle {Q} \rangle \red P\{\quotep{Q}/y}\} }
  \and \\
  \inferrule* [lab=Par] {{P} \red {P}'} {{{P} | {Q}} \red {{P}' | {Q}}}
  \and
  \inferrule* [lab=Equiv]{{{P} \scong {P}'} \andalso {{P}' \red {Q}'} \andalso {{Q}' \scong {Q}}}{{P} \red {Q}}
\end{mathpar}

\begin{eqnarray*}
  match_{\equiv} (\quotep{P},\quotep{Q}) & := & P \equiv Q \\
  match_{\dagger}(\quotep{P},\quotep{Q}) & := & \forall R. P|Q \red^{*} R => R \red^{*} 0 \\
  match_{K}(\quotep{P},\quotep{Q}) & := & K \mbox{ for some context } K
\end{eqnarray*}

$u?(x)P | u!\langle Q \rangle \red P\{\quotep{Q}/x\}$

%We write $\wred$ for $\red^*$, and $P\red$ if $\exists Q $ such that $ P \red Q$.
We write $P\red$ if $\exists Q $ such that $ P \red Q$ and $P\not\red$, otherwise.

\section{Replication}

As mentioned before, it is known that replication (and hence
recursion) can be implemented in a higher-order process algebra
\cite{SangiorgiWalker}. As our first example of calculation with the
machinery thus far presented we give the construction explicitly in
the {\rhoc}.

\begin{eqnarray}
	D_{x} & := & \prefix{x}{y}{(\binpar{\outputp{x}{y}}{@{y}})} \nonumber\\
	\bangp_{x}{P} & := & \binpar{{x}!\langle{\binpar{D_{x}}{P}}\rangle}{D_{x}} \nonumber
\end{eqnarray}

\begin{eqnarray}
	\bangp_{x}{P} & & \nonumber\\
	=
	& {x}!\langle{(\prefix{x}{y}{(\outputp{x}{y} | @{y})) | P}}\rangle 
	      | \prefix{x}{y}{(\outputp{x}{y} | @{y})} & \nonumber\\
	\red
	& (\outputp{x}{y} | @{y})\substn{\quotep{(\prefix{x}{y}{(@{y} | \outputp{x}{y})) | P}}}{y} & \nonumber\\
	=
	& \outputp{x}{\quotep{(\prefix{x}{y}{(\outputp{x}{y} | @{y})) | P}}}
	  | {(\prefix{x}{y}{(\outputp{x}{y} | @{y})) | P}} & \nonumber\\
	\red
	& \ldots & \nonumber\\
	\red^*
	& P | P | \ldots & \nonumber
\end{eqnarray}

Of course, this encoding, as an implementation, runs away, unfolding
$\bangp{P}$ eagerly. A lazier and more implementable replication
operator, restricted to input-guarded processes, may be obtained as follows.

\begin{eqnarray}
\bangp{\prefix{u}{v}{P}} 
	:= 
	\binpar{\lift{x}{\prefix{u}{v}{(\binpar{D(x)}{P})}}}{D(x)} \nonumber
\end{eqnarray}

\begin{remark}
  Note that the lazier definition still does not deal with summation
  or mixed summation (i.e. sums over input and output). The reader is
  invited to construct definitions of replication that deal with these
  features. 

  Further, the definitions are parameterized in a name, $x$. Can you,
  gentle reader, make a definition that eliminates this parameter and
  guarantees no accidental interaction between the replication
  machinery and the process being replicated -- i.e. no accidental
  sharing of names used by the process to get its work done and the
  name(s) used by the replication to effect copying. This latter
  revision of the definition of replication is crucial to obtaining
  the expected identity $!!P \sim !P$.
\end{remark}

\begin{remark}\label{rem:paradoxical_combinator}
  The reader familiar with the lambda calculus will have noticed the
  similarity between $D$ and the paradoxical combinator.

  [Ed. note: the existence of this seems to suggest we have to be more
  restrictive on the set of processes and names we admit if we are to
  support no-cloning.]
\end{remark}

\subsubsection{Bisimulation}

The computational dynamics gives rise to another kind of equivalence,
the equivalence of computational behavior. As previously mentioned
this is typically captured \emph{via} some form of bisimulation.

% The notion we use in this paper is weak barbed bisimulation
% \cite{milner91polyadicpi}.

The notion we use in this paper is derived from weak barbed
bisimulation \cite{milner91polyadicpi}. 

\begin{definition}
An \emph{observation relation}, $\downarrow_{\mathcal N}$, over a set
of names, $\mathcal N$, is the smallest relation satisfying the rules
below.

\infrule[Out-barb]{y \in {\mathcal N}, \; x \nameeq y}
		  {\outputp{x}{v} \downarrow_{\mathcal N} x}
\infrule[Par-barb]{\mbox{$P\downarrow_{\mathcal N} x$ or $Q\downarrow_{\mathcal N} x$}}
		  {\binpar{P}{Q} \downarrow_{\mathcal N} x}

We write $P \Downarrow_{\mathcal N} x$ if there is $Q$ such that 
$P \wred Q$ and $Q \downarrow_{\mathcal N} x$.
\end{definition}

\begin{definition}
%\label{def.bbisim}
An  ${\mathcal N}$-\emph{barbed bisimulation} over a set of names, ${\mathcal N}$, is a symmetric binary relation 
${\mathcal S}_{\mathcal N}$ between agents such that $P\rel{S}_{\mathcal N}Q$ implies:
\begin{enumerate}
\item If $P \red P'$ then $Q \wred Q'$ and $P'\rel{S}_{\mathcal N} Q'$.
\item If $P\downarrow_{\mathcal N} x$, then $Q\Downarrow_{\mathcal N} x$.
\end{enumerate}
$P$ is ${\mathcal N}$-barbed bisimilar to $Q$, written
$P \wbbisim_{\mathcal N} Q$, if $P \rel{S}_{\mathcal N} Q$ for some ${\mathcal N}$-barbed bisimulation ${\mathcal S}_{\mathcal N}$.
\end{definition}

$\mathcal{R} \subseteq \pi \times \pi$

$P \mathcal{R} Q => \forall P'. P \red P' \Rightarrow \exists Q'. Q \red Q', P' \mathcal{R} Q'$

$P \vdash x \Rightarrow Q \vdash x$

\begin{mathpar}
  \inferrule*[lab=Out-barb]{x \nameeq y}{{y}!\langle{Q}\rangle \vdash x}
  \and
  \inferrule*[lab=Par-barb]{\mbox{$P\vdash x$ or $Q\vdash x$}}{\binpar{P}{Q} \vdash x}
\end{mathpar}

\subsubsection{Contexts}

One of the principle advantages of computational calculi like the
$\pi$-calculus is a well-defined notion of context,
contextual-equivalence and a correlation between
contextual-equivalence and notions of bisimulation. The notion of
context allows the decomposition of a process into (sub-)process and
its syntactic environment, its context. Thus, a context may be
thought of as a process with a ``hole'' (written $\Box$) in it. The
application of a context $M$ to a process $P$, written $M[P]$, is
tantamount to filling the hole in $M$ with $P$. In this paper we do
not need the full weight of this theory, but do make use of the notion
of context in the proof the main theorem. 

\begin{mathpar}
  \inferrule* [lab=summation] {} {{M_{M},M_{N}} \bc \Box \;|\; x.M_{A} \;|\; M_{M}+M_{N}}
  \and
  \inferrule* [lab=agent] {} {{M_{A}} \bc (\vec{x})M_{P} \;| \; \clift{P_0,\ldots,M_{P},\ldots,P_N}}
  \and \\
  \inferrule* [lab=process] {} {{M_{P}} \bc M_{N} \;| \;P|M_{P} }
\end{mathpar} 

\begin{mathpar}
  \inferrule* [lab=sychronization] {} {M_{N} \bc \Box \;|\; x?M_{F} \;|\; x!M_{C}}
  \and
  \inferrule* [lab=abstraction] {} {{M_{F}} \bc (x)M_{P} }
  \and
  \inferrule* [lab=concretion] {} {{M_{C}} \bc \langle M_{P} \rangle }
  \and \\
  \inferrule* [lab=process] {} {{M_{P}} \bc M_{N} \;| \;P|M_{P} }
\end{mathpar}

\begin{definition}[contextual application] Given a context $M$, and
  process $P$, we define the \emph{contextual application}, $M[P] :=
  M\{P/\Box\}$. That is, the contextual application of M to P is the
  substitution of $P$ for $\Box$ in $M$.
\end{definition}

$\meaningof{-} : L \to \mathcal{P}(\pi)$

\begin{mathpar}
  \inferrule* [lab=collection] {} {\meaningof{true} = \pi, \and \meaningof{~E} = \pi \setminus \meaningof{E}, \and \meaningof{E_{1} \& E_{2}} = \meaningof{E_{1}} \cap \meaningof{E_{2}}}
\end{mathpar}

\begin{mathpar}
  \inferrule* [lab=structure] {} {\meaningof{0} = \{ P \in \pi | P \equiv 0 \}, \and \\ \meaningof{E_1 | E_2} = \{ P \in \pi | P \equiv P_{1} | P_{2}, P_{1} \in \meaningof{E_{1}}, P_{2} \in \meaningof{E_2}\} }
\end{mathpar}

\begin{mathpar}
 \inferrule* [lab=behavior] {} {\meaningof{\langle a?b \rangle E} = \{ P \in \pi | P \equiv Q | u?(y)P', \\ \and \\\\ \and \\ \;\;\; u \in \meaningof{a}, \forall z.P'\{z/y\} \in \meaningof{E\{z/b\}}\}, \and \\ \meaningof{a!E} = \{ P \in \pi | P \equiv Q | x!\langle P' \rangle, x \in \meaningof{a} P' \in \meaningof{E}\} }
\end{mathpar}

\begin{mathpar}
 \inferrule* [lab=nominal] {} {\meaningof{\quotep{E}} = \{ \quotep{P} \in \quotep{\pi} | P \in \meaningof{E} \}, \and \meaningof{\quotep{P}} = \{ \quotep{Q} \in \quotep{\pi} | P \equiv Q \} \and \\ \meaningof{@\quotep{E}} = \{ P \in \pi | P \equiv @x, x \in \meaningof{E} \}}
\end{mathpar}

\begin{eqnarray*}
  \\
  \meaningof{-} : TS \to ST
\end{eqnarray*}

\begin{eqnarray*}
  \\
  L : TS \to ST
\end{eqnarray*}

\begin{eqnarray*}
  \\
  P \models E \iff P \in \meaningof{E}
\end{eqnarray*}

\begin{eqnarray*}
  P \approx_{L} Q \iff \forall E \in L. P \models E \iff Q \models E
\end{eqnarray*}

\begin{eqnarray*}
  P \approx_{K} Q
\end{eqnarray*}

\begin{eqnarray*}
  P \approx Q
\end{eqnarray*}

$\approx_{K} = \approx = \approx_{L}$

\subsubsection{Contextual duality}

Note that contexts extend the quotation operation to a family of
operations from processes to names. Given a context, $M$, we can
define a \emph{nominal context}, $\quotep{M}$ by $\quotep{M}[P] :=
\quotep{M[P]}$. To foreshadow what is to come we observe that these
operations enjoy a duality with processes very much like the duality
between vectors and maps from vectors to scalars.

Further, because the calculus is essentially higher-order, we have a
correspondence between contexts and processes. More specifically,
given a name $x$ and a context $M$ we can construct $M^{*}_{x}$ such
that 

\begin{mathpar}
  M^{*}_{x} | \lift{x}{P} \red M[P]
\end{mathpar}

namely,

\begin{mathpar}
  M^{*}_{x} := x?(u).M[\dropn{u}]
\end{mathpar}

The dependence of $M^{*}_{x}$ on a name makes it an abstraction, 

\begin{mathpar}
  M^{*} := (x)x?(u).M[\dropn{u}]
\end{mathpar}

\subsection{Additional notation}

It will sometimes be convenient to denote the process a name
quotes. We already have the notation $x = \quotep{P}$, but it will be
convenient to introduce an alternate notation, $\procn{x}$, when we
want to emphasize the connection to the use of the name. Note that, by
virtue of name equivalence, $\quotep{\procn{x}} \nameeq x$; so, the
notation is consistent with previous definitions.

Further, because names have structure it is possible to effect
substitutions on the basis of that structure. This means we need to
upgrade our notation for substitutions, which we accomplish by
adapting comprehension notation. Thus,

\begin{mathpar}
  P\{ y / x : x \in S \}
\end{mathpar}

is interpreted to mean the process derived from P by replacing (in a
capture-avoiding manner) each occurrence of $x$ in $S$ by $y$. For example,

\begin{mathpar}
  P\{ \quotep{\procn{x}|\procn{x}} / x : x \in \freenames{P} \}
\end{mathpar}

will replace each (occurrence) of a free name $x$ in $P$ by
$\quotep{\procn{x}|\procn{x}}$.

Also, we will avail ourselves of the notation $x^{L}$ and $x^{R}$ to
denote injections of a name into disjoint copies of the name
space. There are numerous ways to accomplish this. One example can be
found in \cite{MeredithR05}. This notation overloads to vectors of
names: $\vec{x}^{\pi} := (x_{i}^{\pi} \; : \; 0 \leq i < |\vec{x}| )$ where $\pi \in \{L,R\}$.

We also use $P^{\Box} := P|\Box$.

In \cite{MeredithR05} an interpretation of the new operator is
given. It turns out that there are several possible interpretations
all enjoying the requisite algebraic properties of the operator (see
\cite{milner91polyadicpi}). We will therefore make liberal use of
$(\nu\; \vec{x})P$.

% subsection the_syntax_and_semantics_of_the_notation_system (end)   

\input{qm2pi.qmops} 

\input{qm2pi.sterngerlach} 

\input{qm2pi.metric} 

% section concurrent_process_calculi (end)

%\input{qm2pi.proofsketch}

% section proof sketch (end)

%\input{qm2pi.slviaknots} 

% section spatial logic via knots (end)

\input{qm2pi.conclusion}

% section conclusion (end)

%\input{qm2pi.dtcodes} 

% section wiring algorithm (end)

\input{qm2pi.ack} 

% section acknowledgments (end)

\newpage


\bibliographystyle{plain}   
\bibliography{../../biblios/main.bib}

\input{qm2pi.rhodetails}

\end{document}

 

%\documentclass[12pt]{llncs}
%\documentclass{jktr}

\usepackage[pdftex]{hyperref}                   
\usepackage {listings}
\usepackage {mathpartir}
\usepackage{bcprules}
%\usepackage{listings}
                       
\usepackage{graphicx} 
%\usepackage[margins=2.5cm,nohead,nofoot]{geometry}
%\usepackage{geometry}
\usepackage{amsfonts}
\usepackage{amstext}
\usepackage{latexsym}
\usepackage{amssymb}
\usepackage{color}


%\include{myPreamble}
\include{qm2pi.local} 

%\ifpdf
%\usepackage[pdftex]{graphicx}
%\else
%\usepackage{graphicx}
%\fi

 % \ifpdf
%  \usepackage{pdfsync}
%  \if


%\title{Brief Article}
%\author{David F. Snyder}
%\author{L.G. Meredith}

%\address{Dept. of Math., Texas State University--San Marcos, San Marcos, TX 78666}
       
\pagestyle{empty}


\begin{document}

\lstset{language=[Objective]Caml,frame=shadowbox}

\input{qm2pi.front}

% section front matter (end)

\input{qm2pi.intro} 
 
% section introduction (end)

% \input{qm2pi.knotations} 

% section notation (end)

\input{qm2pi.process.calculi} 

% section concurrent_process_calculi_and_spatial_logics_ (end)
    
%\input{qm2pi.knots2pi} 

%\input{qm2pi.trefoil} 

%\input{qm2pi.mainthm} 

% subsection basic_interpretation (end)

%\input{qm2pi.rho.presentation} 
\subsection{The syntax and semantics of the notation system}\label{sub:the_syntax_and_semantics_of_the_notation_system} % (fold)

We now summarize a technical presentation of the calculus that
embodies our theory of dynamics. The typical presentation of such a
calculus follows the style of giving generators and relations on
them. The grammar, below, describing term constructors, freely
generates the set of processes, $\Proc$. This set is then quotiented
by a relation known as structural congruence and it is over this set
that the notion of dynamics is expressed. This presentation is
essentially that of \cite{MeredithR05} with the addition of
polyadicity and summation. For readability we have relegated some of
the technical subtleties to an appendix.

\subsubsection{Process grammar}\label{subsub:process_grammar}

\begin{mathpar}
  \inferrule* [lab=synchronization] {} {{M} \bc \pzero \;|\; x?F \;|\; x!C }
  \and
  \inferrule* [lab=abstraction] {} {{F} \bc (x)P}
  \and
  \inferrule* [lab=concretion] {} {{C} \bc \langle Q \rangle}
  \and
  \inferrule* [lab=process] {} {{P,Q} \bc M \;| \;P|Q \;|\; @{x}}
  \and
  \inferrule* [lab=name] {} {{x} \bc \quotep{P}}
\end{mathpar} 

Note that $\vec{x}$ (resp. $\vec{P}$) denotes a vector of names
(resp. processes) of length $|\vec{x}|$ (resp. $|\vec{P}|$). We adopt
the following useful abbreviations.

\begin{mathpar}
   x?(\vec{y}).P := x.(\vec{y})P \and  x\clift{\vec{P}} := x.\clift{\vec{P}}
   \and x!(y) := \lift{x}{\dropn{y}}
   \and \Pi_{i=0}^{n-1}P_i := P_0 | \ldots | P_{n-1}
\end{mathpar}

\subsubsection{Structural congruence}

\paragraph{Free and bound names and alpha-equivalence.} At the
core of structural equivalence is alpha-equivalence which identifies
process that are the same up to a change of variable. Formally, we
recognize the distinction between free and bound names. The free names
of a process, $\freenames{P}$, may be calculated recursively as
follows:

\begin{mathpar}
\freenames{\pzero} := \emptyset
  \and \\
  \freenames{x?(y).P} := \{ x \} \cup (\freenames{P} \setminus \{ y \})
  \and 
  \freenames{x!\langle P \rangle} := \{ x \} \cup \{ P \} 
  \and \\
  \freenames{P|Q} := \freenames{P} \cup \freenames{Q}
  \and \\
  \freenames{@{x}} := \{ x \}
\end{mathpar}

$\pi$
$\quotep{\pi}$

$\freenames{-} : \pi \to \mathcal{P}(\quotep{\pi})$

\begin{eqnarray*}
  \freenames{\pzero} & := & \emptyset \\
  \freenames{x?(y).P} & := & \{ x \} \cup (\freenames{P} \setminus \{ y \}) \\
  \freenames{x!\langle P \rangle} & := & \{ x \} \cup \{ P \} \\
  \freenames{P|Q} & := & \freenames{P} \cup \freenames{Q} \\
  \freenames{\dropn{x}} & := & \{ x \}
\end{eqnarray*}

The bound names of a process, $\boundnames{P}$, are those names occurring in $P$
that are not free. For example, in $x?(y).0$, the name $x$ is free, while $y$ is bound.

\begin{mathpar}
  \inferrule* [lab=monoidal-laws] {} { P|Q \equiv Q|P \and P|0 \equiv P \and P|(Q|R) \equiv (P|Q)|R }
\end{mathpar}

\begin{mathpar}
  \inferrule* [lab=alpha-equivalence] {} { (x)P \equiv (y)P\{y/x\} \and y \not\in \freenames{P} }
\end{mathpar}

\begin{definition}
Then two processes, $P,Q$, are alpha-equivalent if $P = Q\{\vec{y}/\vec{x}\}$ for
some $\vec{x} \in \boundnames{Q},\vec{y} \in \boundnames{P}$, where $Q\{\vec{y}/\vec{x}\}$
denotes the capture-avoiding substitution of $\vec{y}$ for $\vec{x}$ in $Q$.
\end{definition}

\begin{definition}
  The {\em structural congruence} \cite{SangiorgiWalker} , $\equiv$,
  between processes is the least congruence containing
  alpha-equivalence, satisfying the abelian monoid laws
  (associativity, commutativity and $\pzero$ as identity) for parallel
  composition $|$ and for summation $+$.
\end{definition}

\subsection{Name equivalence}

We take name equivalence, written $\nameeq$, to be the smallest
equivalence relation generated by the following rules.

\begin{mathpar}
\inferrule*[lab=Quote-drop]
{ }
{ \quotep{@{x}} \nameeq x }

\inferrule*[lab=Struct-equiv]
{ P \scong Q }
{ \quotep{P} \nameeq \quotep{Q} }
\end{mathpar}

The astute reader will have noticed that the mutual recursion of names
and processes imposes a mutual recursion on alpha-equivalence and
structural equivalence via name-equivalence. Fortunately, all of this
works out pleasantly and we may calculate in the natural way, free of
concern. The reader interested in the details is referred to the
appendix \ref{appendix:rho_details}.

\subsection{Substitution}

We use $\Proc$ for the set of processes, $\QProc$ for the set of
names, and $\id{\{}\vec{y} / \vec{x} \id{\}}$ to denote partial maps,
$s : \QProc \rightarrow \QProc$. A map, $s$ lifts, uniquely, to a map
on process terms, $\widehat{s} : \Proc \rightarrow \Proc$ by the
following equations.

\begin{mathpar}
  (0) \psubstp{Q}{P} := 0 \\
  (R \juxtap S) \psubstp{Q}{P}
  :=    
  (R)\psubstp{Q}{P} \juxtap (S) \psubstp{Q}{P} \\
  (x?(y).R) \psubstp{Q}{P}    
  :=    
  (x)\substp{Q}{P} (z)\concat( (R \psubstn{z}{y}) \psubstp{Q}{P} ) \\
  (\lift{x}{R}) \psubstp{Q}{P}  
  :=
  \lift{(x)\substp{Q}{P}}{ R \psubstp{Q}{P} } \\
%   (\dropn{x})  \psubstp{Q}{P}       
%   := 
%   \left\{ 
%     \begin{array}{ccc} 
%       \dropn{\quotep{Q}} & & x \nameeq \quotep{P} \\
%       \dropn{x} & & otherwise \\
%     \end{array}
%   \right. 
  (\dropn{x})  \psubstp{Q}{P}       
  := 
  \left\{ 
    \begin{array}{ccc} 
      Q & & x \nameeq \quotep{P} \\
      \dropn{x} & & otherwise \\
    \end{array}
  \right.
\end{mathpar}
 

where

\begin{eqnarray}
  (x)\id{\{} \lpquote Q \rpquote / \lpquote P \rpquote \id{\}}            = 
  \left\{ 
    \begin{array}{ccc}
      \lpquote Q \rpquote & & x \nameeq \lpquote P \rpquote \\
      x & & otherwise \\
    \end{array}
  \right. \nonumber
\end{eqnarray}

and $z$ is chosen distinct from $\quotep{P}$, $\quotep{Q}$, the free
names in $Q$, and all the names in $R$. Our $\alpha$-equivalence will
be built in the standard way from this substitution.

\begin{remark}\label{rem:no_self_referential_names}
  One consequence of these definitions is that $\forall P. \quotep{P}
  \not\in \freenames{P}$.
\end{remark}

\subsection{ Dynamic quote: an example }

Anticipating something of what's to come, consider applying the
substitution, $\widehat{\id{\{}u / z \id{\}}}$, to the following pair
of processes, $\lift{w}{y!(z)}$ and $w[ \lpquote y!(z) \rpquote ]$.

\begin{eqnarray}
	\lift{w}{y!(z)}\widehat{\id{\{}u / z \id{\}}}
		& = &
		\lift{w}{y!(u)} \nonumber\\
	w[ \lpquote y!(z) \rpquote ] \widehat{ \id{\{}u / z \id{\}} }
		& = &
		w[ \lpquote y!(z) \rpquote ] \nonumber
\end{eqnarray}

Because the body of the process between quotes is impervious to
substitution, we get radically different answers. In fact, by
examining the first process in an input context,
e.g. $x?(z).\lift{w}{y!(z)}$, we see that the process under the lift
operator may be shaped by prefixed inputs binding a name inside it. In
this sense, the lift operator will be seen as a way to dynamically
construct processes before reifying them as names.

Finally equipped with these standard features we can present the
dynamics of the calculus.

\subsubsection{Operational semantics} 

Finally, we introduce the computational dynamics. What marks these
algebras as distinct from other more traditionally studied algebraic
structures, e.g. vector spaces or polynomial rings, is the manner in
which dynamics is captured. In traditional structures, dynamics is typically
expressed through morphisms between such structures, as in linear maps
between vector spaces or morphisms between rings. In algebras
associated with the semantics of computation, the dynamics is
expressed as part of the algebraic structure itself, through a
reduction reduction relation typically denoted by $\red$. Below, we
give a recursive presentation of this relation for the calculus used
in the encoding.

$\red \subseteq \pi \times \pi$
$\red : \pi \to \mathcal{P}(\pi)$

\begin{mathpar}
  \inferrule* [lab=Comm] { \textsf{match}( x_{src}, x_{trgt} ) } { x_{trgt}?(y)P \; | \; x_{src}!\langle {Q} \rangle \red P\{\quotep{Q}/y}\} }
  \and \\
  \inferrule* [lab=Par] {{P} \red {P}'} {{{P} | {Q}} \red {{P}' | {Q}}}
  \and
  \inferrule* [lab=Equiv]{{{P} \scong {P}'} \andalso {{P}' \red {Q}'} \andalso {{Q}' \scong {Q}}}{{P} \red {Q}}
\end{mathpar}

\begin{eqnarray*}
  match_{\equiv} (\quotep{P},\quotep{Q}) & := & P \equiv Q \\
  match_{\dagger}(\quotep{P},\quotep{Q}) & := & \forall R. P|Q \red^{*} R => R \red^{*} 0 \\
  match_{K}(\quotep{P},\quotep{Q}) & := & K \mbox{ for some context } K
\end{eqnarray*}

$u?(x)P | u!\langle Q \rangle \red P\{\quotep{Q}/x\}$

%We write $\wred$ for $\red^*$, and $P\red$ if $\exists Q $ such that $ P \red Q$.
We write $P\red$ if $\exists Q $ such that $ P \red Q$ and $P\not\red$, otherwise.

\section{Replication}

As mentioned before, it is known that replication (and hence
recursion) can be implemented in a higher-order process algebra
\cite{SangiorgiWalker}. As our first example of calculation with the
machinery thus far presented we give the construction explicitly in
the {\rhoc}.

\begin{eqnarray}
	D_{x} & := & \prefix{x}{y}{(\binpar{\outputp{x}{y}}{@{y}})} \nonumber\\
	\bangp_{x}{P} & := & \binpar{{x}!\langle{\binpar{D_{x}}{P}}\rangle}{D_{x}} \nonumber
\end{eqnarray}

\begin{eqnarray}
	\bangp_{x}{P} & & \nonumber\\
	=
	& {x}!\langle{(\prefix{x}{y}{(\outputp{x}{y} | @{y})) | P}}\rangle 
	      | \prefix{x}{y}{(\outputp{x}{y} | @{y})} & \nonumber\\
	\red
	& (\outputp{x}{y} | @{y})\substn{\quotep{(\prefix{x}{y}{(@{y} | \outputp{x}{y})) | P}}}{y} & \nonumber\\
	=
	& \outputp{x}{\quotep{(\prefix{x}{y}{(\outputp{x}{y} | @{y})) | P}}}
	  | {(\prefix{x}{y}{(\outputp{x}{y} | @{y})) | P}} & \nonumber\\
	\red
	& \ldots & \nonumber\\
	\red^*
	& P | P | \ldots & \nonumber
\end{eqnarray}

Of course, this encoding, as an implementation, runs away, unfolding
$\bangp{P}$ eagerly. A lazier and more implementable replication
operator, restricted to input-guarded processes, may be obtained as follows.

\begin{eqnarray}
\bangp{\prefix{u}{v}{P}} 
	:= 
	\binpar{\lift{x}{\prefix{u}{v}{(\binpar{D(x)}{P})}}}{D(x)} \nonumber
\end{eqnarray}

\begin{remark}
  Note that the lazier definition still does not deal with summation
  or mixed summation (i.e. sums over input and output). The reader is
  invited to construct definitions of replication that deal with these
  features. 

  Further, the definitions are parameterized in a name, $x$. Can you,
  gentle reader, make a definition that eliminates this parameter and
  guarantees no accidental interaction between the replication
  machinery and the process being replicated -- i.e. no accidental
  sharing of names used by the process to get its work done and the
  name(s) used by the replication to effect copying. This latter
  revision of the definition of replication is crucial to obtaining
  the expected identity $!!P \sim !P$.
\end{remark}

\begin{remark}\label{rem:paradoxical_combinator}
  The reader familiar with the lambda calculus will have noticed the
  similarity between $D$ and the paradoxical combinator.

  [Ed. note: the existence of this seems to suggest we have to be more
  restrictive on the set of processes and names we admit if we are to
  support no-cloning.]
\end{remark}

\subsubsection{Bisimulation}

The computational dynamics gives rise to another kind of equivalence,
the equivalence of computational behavior. As previously mentioned
this is typically captured \emph{via} some form of bisimulation.

% The notion we use in this paper is weak barbed bisimulation
% \cite{milner91polyadicpi}.

The notion we use in this paper is derived from weak barbed
bisimulation \cite{milner91polyadicpi}. 

\begin{definition}
An \emph{observation relation}, $\downarrow_{\mathcal N}$, over a set
of names, $\mathcal N$, is the smallest relation satisfying the rules
below.

\infrule[Out-barb]{y \in {\mathcal N}, \; x \nameeq y}
		  {\outputp{x}{v} \downarrow_{\mathcal N} x}
\infrule[Par-barb]{\mbox{$P\downarrow_{\mathcal N} x$ or $Q\downarrow_{\mathcal N} x$}}
		  {\binpar{P}{Q} \downarrow_{\mathcal N} x}

We write $P \Downarrow_{\mathcal N} x$ if there is $Q$ such that 
$P \wred Q$ and $Q \downarrow_{\mathcal N} x$.
\end{definition}

\begin{definition}
%\label{def.bbisim}
An  ${\mathcal N}$-\emph{barbed bisimulation} over a set of names, ${\mathcal N}$, is a symmetric binary relation 
${\mathcal S}_{\mathcal N}$ between agents such that $P\rel{S}_{\mathcal N}Q$ implies:
\begin{enumerate}
\item If $P \red P'$ then $Q \wred Q'$ and $P'\rel{S}_{\mathcal N} Q'$.
\item If $P\downarrow_{\mathcal N} x$, then $Q\Downarrow_{\mathcal N} x$.
\end{enumerate}
$P$ is ${\mathcal N}$-barbed bisimilar to $Q$, written
$P \wbbisim_{\mathcal N} Q$, if $P \rel{S}_{\mathcal N} Q$ for some ${\mathcal N}$-barbed bisimulation ${\mathcal S}_{\mathcal N}$.
\end{definition}

$\mathcal{R} \subseteq \pi \times \pi$

$P \mathcal{R} Q => \forall P'. P \red P' \Rightarrow \exists Q'. Q \red Q', P' \mathcal{R} Q'$

$P \vdash x \Rightarrow Q \vdash x$

\begin{mathpar}
  \inferrule*[lab=Out-barb]{x \nameeq y}{{y}!\langle{Q}\rangle \vdash x}
  \and
  \inferrule*[lab=Par-barb]{\mbox{$P\vdash x$ or $Q\vdash x$}}{\binpar{P}{Q} \vdash x}
\end{mathpar}

\subsubsection{Contexts}

One of the principle advantages of computational calculi like the
$\pi$-calculus is a well-defined notion of context,
contextual-equivalence and a correlation between
contextual-equivalence and notions of bisimulation. The notion of
context allows the decomposition of a process into (sub-)process and
its syntactic environment, its context. Thus, a context may be
thought of as a process with a ``hole'' (written $\Box$) in it. The
application of a context $M$ to a process $P$, written $M[P]$, is
tantamount to filling the hole in $M$ with $P$. In this paper we do
not need the full weight of this theory, but do make use of the notion
of context in the proof the main theorem. 

\begin{mathpar}
  \inferrule* [lab=summation] {} {{M_{M},M_{N}} \bc \Box \;|\; x.M_{A} \;|\; M_{M}+M_{N}}
  \and
  \inferrule* [lab=agent] {} {{M_{A}} \bc (\vec{x})M_{P} \;| \; \clift{P_0,\ldots,M_{P},\ldots,P_N}}
  \and \\
  \inferrule* [lab=process] {} {{M_{P}} \bc M_{N} \;| \;P|M_{P} }
\end{mathpar} 

\begin{mathpar}
  \inferrule* [lab=sychronization] {} {M_{N} \bc \Box \;|\; x?M_{F} \;|\; x!M_{C}}
  \and
  \inferrule* [lab=abstraction] {} {{M_{F}} \bc (x)M_{P} }
  \and
  \inferrule* [lab=concretion] {} {{M_{C}} \bc \langle M_{P} \rangle }
  \and \\
  \inferrule* [lab=process] {} {{M_{P}} \bc M_{N} \;| \;P|M_{P} }
\end{mathpar}

\begin{definition}[contextual application] Given a context $M$, and
  process $P$, we define the \emph{contextual application}, $M[P] :=
  M\{P/\Box\}$. That is, the contextual application of M to P is the
  substitution of $P$ for $\Box$ in $M$.
\end{definition}

$\meaningof{-} : L \to \mathcal{P}(\pi)$

\begin{mathpar}
  \inferrule* [lab=collection] {} {\meaningof{true} = \pi, \and \meaningof{~E} = \pi \setminus \meaningof{E}, \and \meaningof{E_{1} \& E_{2}} = \meaningof{E_{1}} \cap \meaningof{E_{2}}}
\end{mathpar}

\begin{mathpar}
  \inferrule* [lab=structure] {} {\meaningof{0} = \{ P \in \pi | P \equiv 0 \}, \and \\ \meaningof{E_1 | E_2} = \{ P \in \pi | P \equiv P_{1} | P_{2}, P_{1} \in \meaningof{E_{1}}, P_{2} \in \meaningof{E_2}\} }
\end{mathpar}

\begin{mathpar}
 \inferrule* [lab=behavior] {} {\meaningof{\langle a?b \rangle E} = \{ P \in \pi | P \equiv Q | u?(y)P', \\ \and \\\\ \and \\ \;\;\; u \in \meaningof{a}, \forall z.P'\{z/y\} \in \meaningof{E\{z/b\}}\}, \and \\ \meaningof{a!E} = \{ P \in \pi | P \equiv Q | x!\langle P' \rangle, x \in \meaningof{a} P' \in \meaningof{E}\} }
\end{mathpar}

\begin{mathpar}
 \inferrule* [lab=nominal] {} {\meaningof{\quotep{E}} = \{ \quotep{P} \in \quotep{\pi} | P \in \meaningof{E} \}, \and \meaningof{\quotep{P}} = \{ \quotep{Q} \in \quotep{\pi} | P \equiv Q \} \and \\ \meaningof{@\quotep{E}} = \{ P \in \pi | P \equiv @x, x \in \meaningof{E} \}}
\end{mathpar}

\begin{eqnarray*}
  \\
  \meaningof{-} : TS \to ST
\end{eqnarray*}

\begin{eqnarray*}
  \\
  L : TS \to ST
\end{eqnarray*}

\begin{eqnarray*}
  \\
  P \models E \iff P \in \meaningof{E}
\end{eqnarray*}

\begin{eqnarray*}
  P \approx_{L} Q \iff \forall E \in L. P \models E \iff Q \models E
\end{eqnarray*}

\begin{eqnarray*}
  P \approx_{K} Q
\end{eqnarray*}

\begin{eqnarray*}
  P \approx Q
\end{eqnarray*}

$\approx_{K} = \approx = \approx_{L}$

\subsubsection{Contextual duality}

Note that contexts extend the quotation operation to a family of
operations from processes to names. Given a context, $M$, we can
define a \emph{nominal context}, $\quotep{M}$ by $\quotep{M}[P] :=
\quotep{M[P]}$. To foreshadow what is to come we observe that these
operations enjoy a duality with processes very much like the duality
between vectors and maps from vectors to scalars.

Further, because the calculus is essentially higher-order, we have a
correspondence between contexts and processes. More specifically,
given a name $x$ and a context $M$ we can construct $M^{*}_{x}$ such
that 

\begin{mathpar}
  M^{*}_{x} | \lift{x}{P} \red M[P]
\end{mathpar}

namely,

\begin{mathpar}
  M^{*}_{x} := x?(u).M[\dropn{u}]
\end{mathpar}

The dependence of $M^{*}_{x}$ on a name makes it an abstraction, 

\begin{mathpar}
  M^{*} := (x)x?(u).M[\dropn{u}]
\end{mathpar}

\subsection{Additional notation}

It will sometimes be convenient to denote the process a name
quotes. We already have the notation $x = \quotep{P}$, but it will be
convenient to introduce an alternate notation, $\procn{x}$, when we
want to emphasize the connection to the use of the name. Note that, by
virtue of name equivalence, $\quotep{\procn{x}} \nameeq x$; so, the
notation is consistent with previous definitions.

Further, because names have structure it is possible to effect
substitutions on the basis of that structure. This means we need to
upgrade our notation for substitutions, which we accomplish by
adapting comprehension notation. Thus,

\begin{mathpar}
  P\{ y / x : x \in S \}
\end{mathpar}

is interpreted to mean the process derived from P by replacing (in a
capture-avoiding manner) each occurrence of $x$ in $S$ by $y$. For example,

\begin{mathpar}
  P\{ \quotep{\procn{x}|\procn{x}} / x : x \in \freenames{P} \}
\end{mathpar}

will replace each (occurrence) of a free name $x$ in $P$ by
$\quotep{\procn{x}|\procn{x}}$.

Also, we will avail ourselves of the notation $x^{L}$ and $x^{R}$ to
denote injections of a name into disjoint copies of the name
space. There are numerous ways to accomplish this. One example can be
found in \cite{MeredithR05}. This notation overloads to vectors of
names: $\vec{x}^{\pi} := (x_{i}^{\pi} \; : \; 0 \leq i < |\vec{x}| )$ where $\pi \in \{L,R\}$.

We also use $P^{\Box} := P|\Box$.

In \cite{MeredithR05} an interpretation of the new operator is
given. It turns out that there are several possible interpretations
all enjoying the requisite algebraic properties of the operator (see
\cite{milner91polyadicpi}). We will therefore make liberal use of
$(\nu\; \vec{x})P$.

% subsection the_syntax_and_semantics_of_the_notation_system (end)   

\input{qm2pi.qmops} 

\input{qm2pi.sterngerlach} 

\input{qm2pi.metric} 

% section concurrent_process_calculi (end)

%\input{qm2pi.proofsketch}

% section proof sketch (end)

%\input{qm2pi.slviaknots} 

% section spatial logic via knots (end)

\input{qm2pi.conclusion}

% section conclusion (end)

%\input{qm2pi.dtcodes} 

% section wiring algorithm (end)

\input{qm2pi.ack} 

% section acknowledgments (end)

\newpage


\bibliographystyle{plain}   
\bibliography{../../biblios/main.bib}

\input{qm2pi.rhodetails}

\end{document}

 

%\documentclass[12pt]{llncs}
%\documentclass{jktr}

\usepackage[pdftex]{hyperref}                   
\usepackage {listings}
\usepackage {mathpartir}
\usepackage{bcprules}
%\usepackage{listings}
                       
\usepackage{graphicx} 
%\usepackage[margins=2.5cm,nohead,nofoot]{geometry}
%\usepackage{geometry}
\usepackage{amsfonts}
\usepackage{amstext}
\usepackage{latexsym}
\usepackage{amssymb}
\usepackage{color}


%\include{myPreamble}
\include{qm2pi.local} 

%\ifpdf
%\usepackage[pdftex]{graphicx}
%\else
%\usepackage{graphicx}
%\fi

 % \ifpdf
%  \usepackage{pdfsync}
%  \if


%\title{Brief Article}
%\author{David F. Snyder}
%\author{L.G. Meredith}

%\address{Dept. of Math., Texas State University--San Marcos, San Marcos, TX 78666}
       
\pagestyle{empty}


\begin{document}

\lstset{language=[Objective]Caml,frame=shadowbox}

\input{qm2pi.front}

% section front matter (end)

\input{qm2pi.intro} 
 
% section introduction (end)

% \input{qm2pi.knotations} 

% section notation (end)

\input{qm2pi.process.calculi} 

% section concurrent_process_calculi_and_spatial_logics_ (end)
    
%\input{qm2pi.knots2pi} 

%\input{qm2pi.trefoil} 

%\input{qm2pi.mainthm} 

% subsection basic_interpretation (end)

%\input{qm2pi.rho.presentation} 
\subsection{The syntax and semantics of the notation system}\label{sub:the_syntax_and_semantics_of_the_notation_system} % (fold)

We now summarize a technical presentation of the calculus that
embodies our theory of dynamics. The typical presentation of such a
calculus follows the style of giving generators and relations on
them. The grammar, below, describing term constructors, freely
generates the set of processes, $\Proc$. This set is then quotiented
by a relation known as structural congruence and it is over this set
that the notion of dynamics is expressed. This presentation is
essentially that of \cite{MeredithR05} with the addition of
polyadicity and summation. For readability we have relegated some of
the technical subtleties to an appendix.

\subsubsection{Process grammar}\label{subsub:process_grammar}

\begin{mathpar}
  \inferrule* [lab=synchronization] {} {{M} \bc \pzero \;|\; x?F \;|\; x!C }
  \and
  \inferrule* [lab=abstraction] {} {{F} \bc (x)P}
  \and
  \inferrule* [lab=concretion] {} {{C} \bc \langle Q \rangle}
  \and
  \inferrule* [lab=process] {} {{P,Q} \bc M \;| \;P|Q \;|\; @{x}}
  \and
  \inferrule* [lab=name] {} {{x} \bc \quotep{P}}
\end{mathpar} 

Note that $\vec{x}$ (resp. $\vec{P}$) denotes a vector of names
(resp. processes) of length $|\vec{x}|$ (resp. $|\vec{P}|$). We adopt
the following useful abbreviations.

\begin{mathpar}
   x?(\vec{y}).P := x.(\vec{y})P \and  x\clift{\vec{P}} := x.\clift{\vec{P}}
   \and x!(y) := \lift{x}{\dropn{y}}
   \and \Pi_{i=0}^{n-1}P_i := P_0 | \ldots | P_{n-1}
\end{mathpar}

\subsubsection{Structural congruence}

\paragraph{Free and bound names and alpha-equivalence.} At the
core of structural equivalence is alpha-equivalence which identifies
process that are the same up to a change of variable. Formally, we
recognize the distinction between free and bound names. The free names
of a process, $\freenames{P}$, may be calculated recursively as
follows:

\begin{mathpar}
\freenames{\pzero} := \emptyset
  \and \\
  \freenames{x?(y).P} := \{ x \} \cup (\freenames{P} \setminus \{ y \})
  \and 
  \freenames{x!\langle P \rangle} := \{ x \} \cup \{ P \} 
  \and \\
  \freenames{P|Q} := \freenames{P} \cup \freenames{Q}
  \and \\
  \freenames{@{x}} := \{ x \}
\end{mathpar}

$\pi$
$\quotep{\pi}$

$\freenames{-} : \pi \to \mathcal{P}(\quotep{\pi})$

\begin{eqnarray*}
  \freenames{\pzero} & := & \emptyset \\
  \freenames{x?(y).P} & := & \{ x \} \cup (\freenames{P} \setminus \{ y \}) \\
  \freenames{x!\langle P \rangle} & := & \{ x \} \cup \{ P \} \\
  \freenames{P|Q} & := & \freenames{P} \cup \freenames{Q} \\
  \freenames{\dropn{x}} & := & \{ x \}
\end{eqnarray*}

The bound names of a process, $\boundnames{P}$, are those names occurring in $P$
that are not free. For example, in $x?(y).0$, the name $x$ is free, while $y$ is bound.

\begin{mathpar}
  \inferrule* [lab=monoidal-laws] {} { P|Q \equiv Q|P \and P|0 \equiv P \and P|(Q|R) \equiv (P|Q)|R }
\end{mathpar}

\begin{mathpar}
  \inferrule* [lab=alpha-equivalence] {} { (x)P \equiv (y)P\{y/x\} \and y \not\in \freenames{P} }
\end{mathpar}

\begin{definition}
Then two processes, $P,Q$, are alpha-equivalent if $P = Q\{\vec{y}/\vec{x}\}$ for
some $\vec{x} \in \boundnames{Q},\vec{y} \in \boundnames{P}$, where $Q\{\vec{y}/\vec{x}\}$
denotes the capture-avoiding substitution of $\vec{y}$ for $\vec{x}$ in $Q$.
\end{definition}

\begin{definition}
  The {\em structural congruence} \cite{SangiorgiWalker} , $\equiv$,
  between processes is the least congruence containing
  alpha-equivalence, satisfying the abelian monoid laws
  (associativity, commutativity and $\pzero$ as identity) for parallel
  composition $|$ and for summation $+$.
\end{definition}

\subsection{Name equivalence}

We take name equivalence, written $\nameeq$, to be the smallest
equivalence relation generated by the following rules.

\begin{mathpar}
\inferrule*[lab=Quote-drop]
{ }
{ \quotep{@{x}} \nameeq x }

\inferrule*[lab=Struct-equiv]
{ P \scong Q }
{ \quotep{P} \nameeq \quotep{Q} }
\end{mathpar}

The astute reader will have noticed that the mutual recursion of names
and processes imposes a mutual recursion on alpha-equivalence and
structural equivalence via name-equivalence. Fortunately, all of this
works out pleasantly and we may calculate in the natural way, free of
concern. The reader interested in the details is referred to the
appendix \ref{appendix:rho_details}.

\subsection{Substitution}

We use $\Proc$ for the set of processes, $\QProc$ for the set of
names, and $\id{\{}\vec{y} / \vec{x} \id{\}}$ to denote partial maps,
$s : \QProc \rightarrow \QProc$. A map, $s$ lifts, uniquely, to a map
on process terms, $\widehat{s} : \Proc \rightarrow \Proc$ by the
following equations.

\begin{mathpar}
  (0) \psubstp{Q}{P} := 0 \\
  (R \juxtap S) \psubstp{Q}{P}
  :=    
  (R)\psubstp{Q}{P} \juxtap (S) \psubstp{Q}{P} \\
  (x?(y).R) \psubstp{Q}{P}    
  :=    
  (x)\substp{Q}{P} (z)\concat( (R \psubstn{z}{y}) \psubstp{Q}{P} ) \\
  (\lift{x}{R}) \psubstp{Q}{P}  
  :=
  \lift{(x)\substp{Q}{P}}{ R \psubstp{Q}{P} } \\
%   (\dropn{x})  \psubstp{Q}{P}       
%   := 
%   \left\{ 
%     \begin{array}{ccc} 
%       \dropn{\quotep{Q}} & & x \nameeq \quotep{P} \\
%       \dropn{x} & & otherwise \\
%     \end{array}
%   \right. 
  (\dropn{x})  \psubstp{Q}{P}       
  := 
  \left\{ 
    \begin{array}{ccc} 
      Q & & x \nameeq \quotep{P} \\
      \dropn{x} & & otherwise \\
    \end{array}
  \right.
\end{mathpar}
 

where

\begin{eqnarray}
  (x)\id{\{} \lpquote Q \rpquote / \lpquote P \rpquote \id{\}}            = 
  \left\{ 
    \begin{array}{ccc}
      \lpquote Q \rpquote & & x \nameeq \lpquote P \rpquote \\
      x & & otherwise \\
    \end{array}
  \right. \nonumber
\end{eqnarray}

and $z$ is chosen distinct from $\quotep{P}$, $\quotep{Q}$, the free
names in $Q$, and all the names in $R$. Our $\alpha$-equivalence will
be built in the standard way from this substitution.

\begin{remark}\label{rem:no_self_referential_names}
  One consequence of these definitions is that $\forall P. \quotep{P}
  \not\in \freenames{P}$.
\end{remark}

\subsection{ Dynamic quote: an example }

Anticipating something of what's to come, consider applying the
substitution, $\widehat{\id{\{}u / z \id{\}}}$, to the following pair
of processes, $\lift{w}{y!(z)}$ and $w[ \lpquote y!(z) \rpquote ]$.

\begin{eqnarray}
	\lift{w}{y!(z)}\widehat{\id{\{}u / z \id{\}}}
		& = &
		\lift{w}{y!(u)} \nonumber\\
	w[ \lpquote y!(z) \rpquote ] \widehat{ \id{\{}u / z \id{\}} }
		& = &
		w[ \lpquote y!(z) \rpquote ] \nonumber
\end{eqnarray}

Because the body of the process between quotes is impervious to
substitution, we get radically different answers. In fact, by
examining the first process in an input context,
e.g. $x?(z).\lift{w}{y!(z)}$, we see that the process under the lift
operator may be shaped by prefixed inputs binding a name inside it. In
this sense, the lift operator will be seen as a way to dynamically
construct processes before reifying them as names.

Finally equipped with these standard features we can present the
dynamics of the calculus.

\subsubsection{Operational semantics} 

Finally, we introduce the computational dynamics. What marks these
algebras as distinct from other more traditionally studied algebraic
structures, e.g. vector spaces or polynomial rings, is the manner in
which dynamics is captured. In traditional structures, dynamics is typically
expressed through morphisms between such structures, as in linear maps
between vector spaces or morphisms between rings. In algebras
associated with the semantics of computation, the dynamics is
expressed as part of the algebraic structure itself, through a
reduction reduction relation typically denoted by $\red$. Below, we
give a recursive presentation of this relation for the calculus used
in the encoding.

$\red \subseteq \pi \times \pi$
$\red : \pi \to \mathcal{P}(\pi)$

\begin{mathpar}
  \inferrule* [lab=Comm] { \textsf{match}( x_{src}, x_{trgt} ) } { x_{trgt}?(y)P \; | \; x_{src}!\langle {Q} \rangle \red P\{\quotep{Q}/y}\} }
  \and \\
  \inferrule* [lab=Par] {{P} \red {P}'} {{{P} | {Q}} \red {{P}' | {Q}}}
  \and
  \inferrule* [lab=Equiv]{{{P} \scong {P}'} \andalso {{P}' \red {Q}'} \andalso {{Q}' \scong {Q}}}{{P} \red {Q}}
\end{mathpar}

\begin{eqnarray*}
  match_{\equiv} (\quotep{P},\quotep{Q}) & := & P \equiv Q \\
  match_{\dagger}(\quotep{P},\quotep{Q}) & := & \forall R. P|Q \red^{*} R => R \red^{*} 0 \\
  match_{K}(\quotep{P},\quotep{Q}) & := & K \mbox{ for some context } K
\end{eqnarray*}

$u?(x)P | u!\langle Q \rangle \red P\{\quotep{Q}/x\}$

%We write $\wred$ for $\red^*$, and $P\red$ if $\exists Q $ such that $ P \red Q$.
We write $P\red$ if $\exists Q $ such that $ P \red Q$ and $P\not\red$, otherwise.

\section{Replication}

As mentioned before, it is known that replication (and hence
recursion) can be implemented in a higher-order process algebra
\cite{SangiorgiWalker}. As our first example of calculation with the
machinery thus far presented we give the construction explicitly in
the {\rhoc}.

\begin{eqnarray}
	D_{x} & := & \prefix{x}{y}{(\binpar{\outputp{x}{y}}{@{y}})} \nonumber\\
	\bangp_{x}{P} & := & \binpar{{x}!\langle{\binpar{D_{x}}{P}}\rangle}{D_{x}} \nonumber
\end{eqnarray}

\begin{eqnarray}
	\bangp_{x}{P} & & \nonumber\\
	=
	& {x}!\langle{(\prefix{x}{y}{(\outputp{x}{y} | @{y})) | P}}\rangle 
	      | \prefix{x}{y}{(\outputp{x}{y} | @{y})} & \nonumber\\
	\red
	& (\outputp{x}{y} | @{y})\substn{\quotep{(\prefix{x}{y}{(@{y} | \outputp{x}{y})) | P}}}{y} & \nonumber\\
	=
	& \outputp{x}{\quotep{(\prefix{x}{y}{(\outputp{x}{y} | @{y})) | P}}}
	  | {(\prefix{x}{y}{(\outputp{x}{y} | @{y})) | P}} & \nonumber\\
	\red
	& \ldots & \nonumber\\
	\red^*
	& P | P | \ldots & \nonumber
\end{eqnarray}

Of course, this encoding, as an implementation, runs away, unfolding
$\bangp{P}$ eagerly. A lazier and more implementable replication
operator, restricted to input-guarded processes, may be obtained as follows.

\begin{eqnarray}
\bangp{\prefix{u}{v}{P}} 
	:= 
	\binpar{\lift{x}{\prefix{u}{v}{(\binpar{D(x)}{P})}}}{D(x)} \nonumber
\end{eqnarray}

\begin{remark}
  Note that the lazier definition still does not deal with summation
  or mixed summation (i.e. sums over input and output). The reader is
  invited to construct definitions of replication that deal with these
  features. 

  Further, the definitions are parameterized in a name, $x$. Can you,
  gentle reader, make a definition that eliminates this parameter and
  guarantees no accidental interaction between the replication
  machinery and the process being replicated -- i.e. no accidental
  sharing of names used by the process to get its work done and the
  name(s) used by the replication to effect copying. This latter
  revision of the definition of replication is crucial to obtaining
  the expected identity $!!P \sim !P$.
\end{remark}

\begin{remark}\label{rem:paradoxical_combinator}
  The reader familiar with the lambda calculus will have noticed the
  similarity between $D$ and the paradoxical combinator.

  [Ed. note: the existence of this seems to suggest we have to be more
  restrictive on the set of processes and names we admit if we are to
  support no-cloning.]
\end{remark}

\subsubsection{Bisimulation}

The computational dynamics gives rise to another kind of equivalence,
the equivalence of computational behavior. As previously mentioned
this is typically captured \emph{via} some form of bisimulation.

% The notion we use in this paper is weak barbed bisimulation
% \cite{milner91polyadicpi}.

The notion we use in this paper is derived from weak barbed
bisimulation \cite{milner91polyadicpi}. 

\begin{definition}
An \emph{observation relation}, $\downarrow_{\mathcal N}$, over a set
of names, $\mathcal N$, is the smallest relation satisfying the rules
below.

\infrule[Out-barb]{y \in {\mathcal N}, \; x \nameeq y}
		  {\outputp{x}{v} \downarrow_{\mathcal N} x}
\infrule[Par-barb]{\mbox{$P\downarrow_{\mathcal N} x$ or $Q\downarrow_{\mathcal N} x$}}
		  {\binpar{P}{Q} \downarrow_{\mathcal N} x}

We write $P \Downarrow_{\mathcal N} x$ if there is $Q$ such that 
$P \wred Q$ and $Q \downarrow_{\mathcal N} x$.
\end{definition}

\begin{definition}
%\label{def.bbisim}
An  ${\mathcal N}$-\emph{barbed bisimulation} over a set of names, ${\mathcal N}$, is a symmetric binary relation 
${\mathcal S}_{\mathcal N}$ between agents such that $P\rel{S}_{\mathcal N}Q$ implies:
\begin{enumerate}
\item If $P \red P'$ then $Q \wred Q'$ and $P'\rel{S}_{\mathcal N} Q'$.
\item If $P\downarrow_{\mathcal N} x$, then $Q\Downarrow_{\mathcal N} x$.
\end{enumerate}
$P$ is ${\mathcal N}$-barbed bisimilar to $Q$, written
$P \wbbisim_{\mathcal N} Q$, if $P \rel{S}_{\mathcal N} Q$ for some ${\mathcal N}$-barbed bisimulation ${\mathcal S}_{\mathcal N}$.
\end{definition}

$\mathcal{R} \subseteq \pi \times \pi$

$P \mathcal{R} Q => \forall P'. P \red P' \Rightarrow \exists Q'. Q \red Q', P' \mathcal{R} Q'$

$P \vdash x \Rightarrow Q \vdash x$

\begin{mathpar}
  \inferrule*[lab=Out-barb]{x \nameeq y}{{y}!\langle{Q}\rangle \vdash x}
  \and
  \inferrule*[lab=Par-barb]{\mbox{$P\vdash x$ or $Q\vdash x$}}{\binpar{P}{Q} \vdash x}
\end{mathpar}

\subsubsection{Contexts}

One of the principle advantages of computational calculi like the
$\pi$-calculus is a well-defined notion of context,
contextual-equivalence and a correlation between
contextual-equivalence and notions of bisimulation. The notion of
context allows the decomposition of a process into (sub-)process and
its syntactic environment, its context. Thus, a context may be
thought of as a process with a ``hole'' (written $\Box$) in it. The
application of a context $M$ to a process $P$, written $M[P]$, is
tantamount to filling the hole in $M$ with $P$. In this paper we do
not need the full weight of this theory, but do make use of the notion
of context in the proof the main theorem. 

\begin{mathpar}
  \inferrule* [lab=summation] {} {{M_{M},M_{N}} \bc \Box \;|\; x.M_{A} \;|\; M_{M}+M_{N}}
  \and
  \inferrule* [lab=agent] {} {{M_{A}} \bc (\vec{x})M_{P} \;| \; \clift{P_0,\ldots,M_{P},\ldots,P_N}}
  \and \\
  \inferrule* [lab=process] {} {{M_{P}} \bc M_{N} \;| \;P|M_{P} }
\end{mathpar} 

\begin{mathpar}
  \inferrule* [lab=sychronization] {} {M_{N} \bc \Box \;|\; x?M_{F} \;|\; x!M_{C}}
  \and
  \inferrule* [lab=abstraction] {} {{M_{F}} \bc (x)M_{P} }
  \and
  \inferrule* [lab=concretion] {} {{M_{C}} \bc \langle M_{P} \rangle }
  \and \\
  \inferrule* [lab=process] {} {{M_{P}} \bc M_{N} \;| \;P|M_{P} }
\end{mathpar}

\begin{definition}[contextual application] Given a context $M$, and
  process $P$, we define the \emph{contextual application}, $M[P] :=
  M\{P/\Box\}$. That is, the contextual application of M to P is the
  substitution of $P$ for $\Box$ in $M$.
\end{definition}

$\meaningof{-} : L \to \mathcal{P}(\pi)$

\begin{mathpar}
  \inferrule* [lab=collection] {} {\meaningof{true} = \pi, \and \meaningof{~E} = \pi \setminus \meaningof{E}, \and \meaningof{E_{1} \& E_{2}} = \meaningof{E_{1}} \cap \meaningof{E_{2}}}
\end{mathpar}

\begin{mathpar}
  \inferrule* [lab=structure] {} {\meaningof{0} = \{ P \in \pi | P \equiv 0 \}, \and \\ \meaningof{E_1 | E_2} = \{ P \in \pi | P \equiv P_{1} | P_{2}, P_{1} \in \meaningof{E_{1}}, P_{2} \in \meaningof{E_2}\} }
\end{mathpar}

\begin{mathpar}
 \inferrule* [lab=behavior] {} {\meaningof{\langle a?b \rangle E} = \{ P \in \pi | P \equiv Q | u?(y)P', \\ \and \\\\ \and \\ \;\;\; u \in \meaningof{a}, \forall z.P'\{z/y\} \in \meaningof{E\{z/b\}}\}, \and \\ \meaningof{a!E} = \{ P \in \pi | P \equiv Q | x!\langle P' \rangle, x \in \meaningof{a} P' \in \meaningof{E}\} }
\end{mathpar}

\begin{mathpar}
 \inferrule* [lab=nominal] {} {\meaningof{\quotep{E}} = \{ \quotep{P} \in \quotep{\pi} | P \in \meaningof{E} \}, \and \meaningof{\quotep{P}} = \{ \quotep{Q} \in \quotep{\pi} | P \equiv Q \} \and \\ \meaningof{@\quotep{E}} = \{ P \in \pi | P \equiv @x, x \in \meaningof{E} \}}
\end{mathpar}

\begin{eqnarray*}
  \\
  \meaningof{-} : TS \to ST
\end{eqnarray*}

\begin{eqnarray*}
  \\
  L : TS \to ST
\end{eqnarray*}

\begin{eqnarray*}
  \\
  P \models E \iff P \in \meaningof{E}
\end{eqnarray*}

\begin{eqnarray*}
  P \approx_{L} Q \iff \forall E \in L. P \models E \iff Q \models E
\end{eqnarray*}

\begin{eqnarray*}
  P \approx_{K} Q
\end{eqnarray*}

\begin{eqnarray*}
  P \approx Q
\end{eqnarray*}

$\approx_{K} = \approx = \approx_{L}$

\subsubsection{Contextual duality}

Note that contexts extend the quotation operation to a family of
operations from processes to names. Given a context, $M$, we can
define a \emph{nominal context}, $\quotep{M}$ by $\quotep{M}[P] :=
\quotep{M[P]}$. To foreshadow what is to come we observe that these
operations enjoy a duality with processes very much like the duality
between vectors and maps from vectors to scalars.

Further, because the calculus is essentially higher-order, we have a
correspondence between contexts and processes. More specifically,
given a name $x$ and a context $M$ we can construct $M^{*}_{x}$ such
that 

\begin{mathpar}
  M^{*}_{x} | \lift{x}{P} \red M[P]
\end{mathpar}

namely,

\begin{mathpar}
  M^{*}_{x} := x?(u).M[\dropn{u}]
\end{mathpar}

The dependence of $M^{*}_{x}$ on a name makes it an abstraction, 

\begin{mathpar}
  M^{*} := (x)x?(u).M[\dropn{u}]
\end{mathpar}

\subsection{Additional notation}

It will sometimes be convenient to denote the process a name
quotes. We already have the notation $x = \quotep{P}$, but it will be
convenient to introduce an alternate notation, $\procn{x}$, when we
want to emphasize the connection to the use of the name. Note that, by
virtue of name equivalence, $\quotep{\procn{x}} \nameeq x$; so, the
notation is consistent with previous definitions.

Further, because names have structure it is possible to effect
substitutions on the basis of that structure. This means we need to
upgrade our notation for substitutions, which we accomplish by
adapting comprehension notation. Thus,

\begin{mathpar}
  P\{ y / x : x \in S \}
\end{mathpar}

is interpreted to mean the process derived from P by replacing (in a
capture-avoiding manner) each occurrence of $x$ in $S$ by $y$. For example,

\begin{mathpar}
  P\{ \quotep{\procn{x}|\procn{x}} / x : x \in \freenames{P} \}
\end{mathpar}

will replace each (occurrence) of a free name $x$ in $P$ by
$\quotep{\procn{x}|\procn{x}}$.

Also, we will avail ourselves of the notation $x^{L}$ and $x^{R}$ to
denote injections of a name into disjoint copies of the name
space. There are numerous ways to accomplish this. One example can be
found in \cite{MeredithR05}. This notation overloads to vectors of
names: $\vec{x}^{\pi} := (x_{i}^{\pi} \; : \; 0 \leq i < |\vec{x}| )$ where $\pi \in \{L,R\}$.

We also use $P^{\Box} := P|\Box$.

In \cite{MeredithR05} an interpretation of the new operator is
given. It turns out that there are several possible interpretations
all enjoying the requisite algebraic properties of the operator (see
\cite{milner91polyadicpi}). We will therefore make liberal use of
$(\nu\; \vec{x})P$.

% subsection the_syntax_and_semantics_of_the_notation_system (end)   

\input{qm2pi.qmops} 

\input{qm2pi.sterngerlach} 

\input{qm2pi.metric} 

% section concurrent_process_calculi (end)

%\input{qm2pi.proofsketch}

% section proof sketch (end)

%\input{qm2pi.slviaknots} 

% section spatial logic via knots (end)

\input{qm2pi.conclusion}

% section conclusion (end)

%\input{qm2pi.dtcodes} 

% section wiring algorithm (end)

\input{qm2pi.ack} 

% section acknowledgments (end)

\newpage


\bibliographystyle{plain}   
\bibliography{../../biblios/main.bib}

\input{qm2pi.rhodetails}

\end{document}

 

% subsection basic_interpretation (end)

%\input{qm2pi.rho.presentation} 
\subsection{The syntax and semantics of the notation system}\label{sub:the_syntax_and_semantics_of_the_notation_system} % (fold)

We now summarize a technical presentation of the calculus that
embodies our theory of dynamics. The typical presentation of such a
calculus follows the style of giving generators and relations on
them. The grammar, below, describing term constructors, freely
generates the set of processes, $\Proc$. This set is then quotiented
by a relation known as structural congruence and it is over this set
that the notion of dynamics is expressed. This presentation is
essentially that of \cite{MeredithR05} with the addition of
polyadicity and summation. For readability we have relegated some of
the technical subtleties to an appendix.

\subsubsection{Process grammar}\label{subsub:process_grammar}

\begin{mathpar}
  \inferrule* [lab=synchronization] {} {{M} \bc \pzero \;|\; x?F \;|\; x!C }
  \and
  \inferrule* [lab=abstraction] {} {{F} \bc (x)P}
  \and
  \inferrule* [lab=concretion] {} {{C} \bc \langle Q \rangle}
  \and
  \inferrule* [lab=process] {} {{P,Q} \bc M \;| \;P|Q \;|\; @{x}}
  \and
  \inferrule* [lab=name] {} {{x} \bc \quotep{P}}
\end{mathpar} 

Note that $\vec{x}$ (resp. $\vec{P}$) denotes a vector of names
(resp. processes) of length $|\vec{x}|$ (resp. $|\vec{P}|$). We adopt
the following useful abbreviations.

\begin{mathpar}
   x?(\vec{y}).P := x.(\vec{y})P \and  x\clift{\vec{P}} := x.\clift{\vec{P}}
   \and x!(y) := \lift{x}{\dropn{y}}
   \and \Pi_{i=0}^{n-1}P_i := P_0 | \ldots | P_{n-1}
\end{mathpar}

\subsubsection{Structural congruence}

\paragraph{Free and bound names and alpha-equivalence.} At the
core of structural equivalence is alpha-equivalence which identifies
process that are the same up to a change of variable. Formally, we
recognize the distinction between free and bound names. The free names
of a process, $\freenames{P}$, may be calculated recursively as
follows:

\begin{mathpar}
\freenames{\pzero} := \emptyset
  \and \\
  \freenames{x?(y).P} := \{ x \} \cup (\freenames{P} \setminus \{ y \})
  \and 
  \freenames{x!\langle P \rangle} := \{ x \} \cup \{ P \} 
  \and \\
  \freenames{P|Q} := \freenames{P} \cup \freenames{Q}
  \and \\
  \freenames{@{x}} := \{ x \}
\end{mathpar}

$\pi$
$\quotep{\pi}$

$\freenames{-} : \pi \to \mathcal{P}(\quotep{\pi})$

\begin{eqnarray*}
  \freenames{\pzero} & := & \emptyset \\
  \freenames{x?(y).P} & := & \{ x \} \cup (\freenames{P} \setminus \{ y \}) \\
  \freenames{x!\langle P \rangle} & := & \{ x \} \cup \{ P \} \\
  \freenames{P|Q} & := & \freenames{P} \cup \freenames{Q} \\
  \freenames{\dropn{x}} & := & \{ x \}
\end{eqnarray*}

The bound names of a process, $\boundnames{P}$, are those names occurring in $P$
that are not free. For example, in $x?(y).0$, the name $x$ is free, while $y$ is bound.

\begin{mathpar}
  \inferrule* [lab=monoidal-laws] {} { P|Q \equiv Q|P \and P|0 \equiv P \and P|(Q|R) \equiv (P|Q)|R }
\end{mathpar}

\begin{mathpar}
  \inferrule* [lab=alpha-equivalence] {} { (x)P \equiv (y)P\{y/x\} \and y \not\in \freenames{P} }
\end{mathpar}

\begin{definition}
Then two processes, $P,Q$, are alpha-equivalent if $P = Q\{\vec{y}/\vec{x}\}$ for
some $\vec{x} \in \boundnames{Q},\vec{y} \in \boundnames{P}$, where $Q\{\vec{y}/\vec{x}\}$
denotes the capture-avoiding substitution of $\vec{y}$ for $\vec{x}$ in $Q$.
\end{definition}

\begin{definition}
  The {\em structural congruence} \cite{SangiorgiWalker} , $\equiv$,
  between processes is the least congruence containing
  alpha-equivalence, satisfying the abelian monoid laws
  (associativity, commutativity and $\pzero$ as identity) for parallel
  composition $|$ and for summation $+$.
\end{definition}

\subsection{Name equivalence}

We take name equivalence, written $\nameeq$, to be the smallest
equivalence relation generated by the following rules.

\begin{mathpar}
\inferrule*[lab=Quote-drop]
{ }
{ \quotep{@{x}} \nameeq x }

\inferrule*[lab=Struct-equiv]
{ P \scong Q }
{ \quotep{P} \nameeq \quotep{Q} }
\end{mathpar}

The astute reader will have noticed that the mutual recursion of names
and processes imposes a mutual recursion on alpha-equivalence and
structural equivalence via name-equivalence. Fortunately, all of this
works out pleasantly and we may calculate in the natural way, free of
concern. The reader interested in the details is referred to the
appendix \ref{appendix:rho_details}.

\subsection{Substitution}

We use $\Proc$ for the set of processes, $\QProc$ for the set of
names, and $\id{\{}\vec{y} / \vec{x} \id{\}}$ to denote partial maps,
$s : \QProc \rightarrow \QProc$. A map, $s$ lifts, uniquely, to a map
on process terms, $\widehat{s} : \Proc \rightarrow \Proc$ by the
following equations.

\begin{mathpar}
  (0) \psubstp{Q}{P} := 0 \\
  (R \juxtap S) \psubstp{Q}{P}
  :=    
  (R)\psubstp{Q}{P} \juxtap (S) \psubstp{Q}{P} \\
  (x?(y).R) \psubstp{Q}{P}    
  :=    
  (x)\substp{Q}{P} (z)\concat( (R \psubstn{z}{y}) \psubstp{Q}{P} ) \\
  (\lift{x}{R}) \psubstp{Q}{P}  
  :=
  \lift{(x)\substp{Q}{P}}{ R \psubstp{Q}{P} } \\
%   (\dropn{x})  \psubstp{Q}{P}       
%   := 
%   \left\{ 
%     \begin{array}{ccc} 
%       \dropn{\quotep{Q}} & & x \nameeq \quotep{P} \\
%       \dropn{x} & & otherwise \\
%     \end{array}
%   \right. 
  (\dropn{x})  \psubstp{Q}{P}       
  := 
  \left\{ 
    \begin{array}{ccc} 
      Q & & x \nameeq \quotep{P} \\
      \dropn{x} & & otherwise \\
    \end{array}
  \right.
\end{mathpar}
 

where

\begin{eqnarray}
  (x)\id{\{} \lpquote Q \rpquote / \lpquote P \rpquote \id{\}}            = 
  \left\{ 
    \begin{array}{ccc}
      \lpquote Q \rpquote & & x \nameeq \lpquote P \rpquote \\
      x & & otherwise \\
    \end{array}
  \right. \nonumber
\end{eqnarray}

and $z$ is chosen distinct from $\quotep{P}$, $\quotep{Q}$, the free
names in $Q$, and all the names in $R$. Our $\alpha$-equivalence will
be built in the standard way from this substitution.

\begin{remark}\label{rem:no_self_referential_names}
  One consequence of these definitions is that $\forall P. \quotep{P}
  \not\in \freenames{P}$.
\end{remark}

\subsection{ Dynamic quote: an example }

Anticipating something of what's to come, consider applying the
substitution, $\widehat{\id{\{}u / z \id{\}}}$, to the following pair
of processes, $\lift{w}{y!(z)}$ and $w[ \lpquote y!(z) \rpquote ]$.

\begin{eqnarray}
	\lift{w}{y!(z)}\widehat{\id{\{}u / z \id{\}}}
		& = &
		\lift{w}{y!(u)} \nonumber\\
	w[ \lpquote y!(z) \rpquote ] \widehat{ \id{\{}u / z \id{\}} }
		& = &
		w[ \lpquote y!(z) \rpquote ] \nonumber
\end{eqnarray}

Because the body of the process between quotes is impervious to
substitution, we get radically different answers. In fact, by
examining the first process in an input context,
e.g. $x?(z).\lift{w}{y!(z)}$, we see that the process under the lift
operator may be shaped by prefixed inputs binding a name inside it. In
this sense, the lift operator will be seen as a way to dynamically
construct processes before reifying them as names.

Finally equipped with these standard features we can present the
dynamics of the calculus.

\subsubsection{Operational semantics} 

Finally, we introduce the computational dynamics. What marks these
algebras as distinct from other more traditionally studied algebraic
structures, e.g. vector spaces or polynomial rings, is the manner in
which dynamics is captured. In traditional structures, dynamics is typically
expressed through morphisms between such structures, as in linear maps
between vector spaces or morphisms between rings. In algebras
associated with the semantics of computation, the dynamics is
expressed as part of the algebraic structure itself, through a
reduction reduction relation typically denoted by $\red$. Below, we
give a recursive presentation of this relation for the calculus used
in the encoding.

$\red \subseteq \pi \times \pi$
$\red : \pi \to \mathcal{P}(\pi)$

\begin{mathpar}
  \inferrule* [lab=Comm] { \textsf{match}( x_{src}, x_{trgt} ) } { x_{trgt}?(y)P \; | \; x_{src}!\langle {Q} \rangle \red P\{\quotep{Q}/y}\} }
  \and \\
  \inferrule* [lab=Par] {{P} \red {P}'} {{{P} | {Q}} \red {{P}' | {Q}}}
  \and
  \inferrule* [lab=Equiv]{{{P} \scong {P}'} \andalso {{P}' \red {Q}'} \andalso {{Q}' \scong {Q}}}{{P} \red {Q}}
\end{mathpar}

\begin{eqnarray*}
  match_{\equiv} (\quotep{P},\quotep{Q}) & := & P \equiv Q \\
  match_{\dagger}(\quotep{P},\quotep{Q}) & := & \forall R. P|Q \red^{*} R => R \red^{*} 0 \\
  match_{K}(\quotep{P},\quotep{Q}) & := & K \mbox{ for some context } K
\end{eqnarray*}

$u?(x)P | u!\langle Q \rangle \red P\{\quotep{Q}/x\}$

%We write $\wred$ for $\red^*$, and $P\red$ if $\exists Q $ such that $ P \red Q$.
We write $P\red$ if $\exists Q $ such that $ P \red Q$ and $P\not\red$, otherwise.

\section{Replication}

As mentioned before, it is known that replication (and hence
recursion) can be implemented in a higher-order process algebra
\cite{SangiorgiWalker}. As our first example of calculation with the
machinery thus far presented we give the construction explicitly in
the {\rhoc}.

\begin{eqnarray}
	D_{x} & := & \prefix{x}{y}{(\binpar{\outputp{x}{y}}{@{y}})} \nonumber\\
	\bangp_{x}{P} & := & \binpar{{x}!\langle{\binpar{D_{x}}{P}}\rangle}{D_{x}} \nonumber
\end{eqnarray}

\begin{eqnarray}
	\bangp_{x}{P} & & \nonumber\\
	=
	& {x}!\langle{(\prefix{x}{y}{(\outputp{x}{y} | @{y})) | P}}\rangle 
	      | \prefix{x}{y}{(\outputp{x}{y} | @{y})} & \nonumber\\
	\red
	& (\outputp{x}{y} | @{y})\substn{\quotep{(\prefix{x}{y}{(@{y} | \outputp{x}{y})) | P}}}{y} & \nonumber\\
	=
	& \outputp{x}{\quotep{(\prefix{x}{y}{(\outputp{x}{y} | @{y})) | P}}}
	  | {(\prefix{x}{y}{(\outputp{x}{y} | @{y})) | P}} & \nonumber\\
	\red
	& \ldots & \nonumber\\
	\red^*
	& P | P | \ldots & \nonumber
\end{eqnarray}

Of course, this encoding, as an implementation, runs away, unfolding
$\bangp{P}$ eagerly. A lazier and more implementable replication
operator, restricted to input-guarded processes, may be obtained as follows.

\begin{eqnarray}
\bangp{\prefix{u}{v}{P}} 
	:= 
	\binpar{\lift{x}{\prefix{u}{v}{(\binpar{D(x)}{P})}}}{D(x)} \nonumber
\end{eqnarray}

\begin{remark}
  Note that the lazier definition still does not deal with summation
  or mixed summation (i.e. sums over input and output). The reader is
  invited to construct definitions of replication that deal with these
  features. 

  Further, the definitions are parameterized in a name, $x$. Can you,
  gentle reader, make a definition that eliminates this parameter and
  guarantees no accidental interaction between the replication
  machinery and the process being replicated -- i.e. no accidental
  sharing of names used by the process to get its work done and the
  name(s) used by the replication to effect copying. This latter
  revision of the definition of replication is crucial to obtaining
  the expected identity $!!P \sim !P$.
\end{remark}

\begin{remark}\label{rem:paradoxical_combinator}
  The reader familiar with the lambda calculus will have noticed the
  similarity between $D$ and the paradoxical combinator.

  [Ed. note: the existence of this seems to suggest we have to be more
  restrictive on the set of processes and names we admit if we are to
  support no-cloning.]
\end{remark}

\subsubsection{Bisimulation}

The computational dynamics gives rise to another kind of equivalence,
the equivalence of computational behavior. As previously mentioned
this is typically captured \emph{via} some form of bisimulation.

% The notion we use in this paper is weak barbed bisimulation
% \cite{milner91polyadicpi}.

The notion we use in this paper is derived from weak barbed
bisimulation \cite{milner91polyadicpi}. 

\begin{definition}
An \emph{observation relation}, $\downarrow_{\mathcal N}$, over a set
of names, $\mathcal N$, is the smallest relation satisfying the rules
below.

\infrule[Out-barb]{y \in {\mathcal N}, \; x \nameeq y}
		  {\outputp{x}{v} \downarrow_{\mathcal N} x}
\infrule[Par-barb]{\mbox{$P\downarrow_{\mathcal N} x$ or $Q\downarrow_{\mathcal N} x$}}
		  {\binpar{P}{Q} \downarrow_{\mathcal N} x}

We write $P \Downarrow_{\mathcal N} x$ if there is $Q$ such that 
$P \wred Q$ and $Q \downarrow_{\mathcal N} x$.
\end{definition}

\begin{definition}
%\label{def.bbisim}
An  ${\mathcal N}$-\emph{barbed bisimulation} over a set of names, ${\mathcal N}$, is a symmetric binary relation 
${\mathcal S}_{\mathcal N}$ between agents such that $P\rel{S}_{\mathcal N}Q$ implies:
\begin{enumerate}
\item If $P \red P'$ then $Q \wred Q'$ and $P'\rel{S}_{\mathcal N} Q'$.
\item If $P\downarrow_{\mathcal N} x$, then $Q\Downarrow_{\mathcal N} x$.
\end{enumerate}
$P$ is ${\mathcal N}$-barbed bisimilar to $Q$, written
$P \wbbisim_{\mathcal N} Q$, if $P \rel{S}_{\mathcal N} Q$ for some ${\mathcal N}$-barbed bisimulation ${\mathcal S}_{\mathcal N}$.
\end{definition}

$\mathcal{R} \subseteq \pi \times \pi$

$P \mathcal{R} Q => \forall P'. P \red P' \Rightarrow \exists Q'. Q \red Q', P' \mathcal{R} Q'$

$P \vdash x \Rightarrow Q \vdash x$

\begin{mathpar}
  \inferrule*[lab=Out-barb]{x \nameeq y}{{y}!\langle{Q}\rangle \vdash x}
  \and
  \inferrule*[lab=Par-barb]{\mbox{$P\vdash x$ or $Q\vdash x$}}{\binpar{P}{Q} \vdash x}
\end{mathpar}

\subsubsection{Contexts}

One of the principle advantages of computational calculi like the
$\pi$-calculus is a well-defined notion of context,
contextual-equivalence and a correlation between
contextual-equivalence and notions of bisimulation. The notion of
context allows the decomposition of a process into (sub-)process and
its syntactic environment, its context. Thus, a context may be
thought of as a process with a ``hole'' (written $\Box$) in it. The
application of a context $M$ to a process $P$, written $M[P]$, is
tantamount to filling the hole in $M$ with $P$. In this paper we do
not need the full weight of this theory, but do make use of the notion
of context in the proof the main theorem. 

\begin{mathpar}
  \inferrule* [lab=summation] {} {{M_{M},M_{N}} \bc \Box \;|\; x.M_{A} \;|\; M_{M}+M_{N}}
  \and
  \inferrule* [lab=agent] {} {{M_{A}} \bc (\vec{x})M_{P} \;| \; \clift{P_0,\ldots,M_{P},\ldots,P_N}}
  \and \\
  \inferrule* [lab=process] {} {{M_{P}} \bc M_{N} \;| \;P|M_{P} }
\end{mathpar} 

\begin{mathpar}
  \inferrule* [lab=sychronization] {} {M_{N} \bc \Box \;|\; x?M_{F} \;|\; x!M_{C}}
  \and
  \inferrule* [lab=abstraction] {} {{M_{F}} \bc (x)M_{P} }
  \and
  \inferrule* [lab=concretion] {} {{M_{C}} \bc \langle M_{P} \rangle }
  \and \\
  \inferrule* [lab=process] {} {{M_{P}} \bc M_{N} \;| \;P|M_{P} }
\end{mathpar}

\begin{definition}[contextual application] Given a context $M$, and
  process $P$, we define the \emph{contextual application}, $M[P] :=
  M\{P/\Box\}$. That is, the contextual application of M to P is the
  substitution of $P$ for $\Box$ in $M$.
\end{definition}

$\meaningof{-} : L \to \mathcal{P}(\pi)$

\begin{mathpar}
  \inferrule* [lab=collection] {} {\meaningof{true} = \pi, \and \meaningof{~E} = \pi \setminus \meaningof{E}, \and \meaningof{E_{1} \& E_{2}} = \meaningof{E_{1}} \cap \meaningof{E_{2}}}
\end{mathpar}

\begin{mathpar}
  \inferrule* [lab=structure] {} {\meaningof{0} = \{ P \in \pi | P \equiv 0 \}, \and \\ \meaningof{E_1 | E_2} = \{ P \in \pi | P \equiv P_{1} | P_{2}, P_{1} \in \meaningof{E_{1}}, P_{2} \in \meaningof{E_2}\} }
\end{mathpar}

\begin{mathpar}
 \inferrule* [lab=behavior] {} {\meaningof{\langle a?b \rangle E} = \{ P \in \pi | P \equiv Q | u?(y)P', \\ \and \\\\ \and \\ \;\;\; u \in \meaningof{a}, \forall z.P'\{z/y\} \in \meaningof{E\{z/b\}}\}, \and \\ \meaningof{a!E} = \{ P \in \pi | P \equiv Q | x!\langle P' \rangle, x \in \meaningof{a} P' \in \meaningof{E}\} }
\end{mathpar}

\begin{mathpar}
 \inferrule* [lab=nominal] {} {\meaningof{\quotep{E}} = \{ \quotep{P} \in \quotep{\pi} | P \in \meaningof{E} \}, \and \meaningof{\quotep{P}} = \{ \quotep{Q} \in \quotep{\pi} | P \equiv Q \} \and \\ \meaningof{@\quotep{E}} = \{ P \in \pi | P \equiv @x, x \in \meaningof{E} \}}
\end{mathpar}

\begin{eqnarray*}
  \\
  \meaningof{-} : TS \to ST
\end{eqnarray*}

\begin{eqnarray*}
  \\
  L : TS \to ST
\end{eqnarray*}

\begin{eqnarray*}
  \\
  P \models E \iff P \in \meaningof{E}
\end{eqnarray*}

\begin{eqnarray*}
  P \approx_{L} Q \iff \forall E \in L. P \models E \iff Q \models E
\end{eqnarray*}

\begin{eqnarray*}
  P \approx_{K} Q
\end{eqnarray*}

\begin{eqnarray*}
  P \approx Q
\end{eqnarray*}

$\approx_{K} = \approx = \approx_{L}$

\subsubsection{Contextual duality}

Note that contexts extend the quotation operation to a family of
operations from processes to names. Given a context, $M$, we can
define a \emph{nominal context}, $\quotep{M}$ by $\quotep{M}[P] :=
\quotep{M[P]}$. To foreshadow what is to come we observe that these
operations enjoy a duality with processes very much like the duality
between vectors and maps from vectors to scalars.

Further, because the calculus is essentially higher-order, we have a
correspondence between contexts and processes. More specifically,
given a name $x$ and a context $M$ we can construct $M^{*}_{x}$ such
that 

\begin{mathpar}
  M^{*}_{x} | \lift{x}{P} \red M[P]
\end{mathpar}

namely,

\begin{mathpar}
  M^{*}_{x} := x?(u).M[\dropn{u}]
\end{mathpar}

The dependence of $M^{*}_{x}$ on a name makes it an abstraction, 

\begin{mathpar}
  M^{*} := (x)x?(u).M[\dropn{u}]
\end{mathpar}

\subsection{Additional notation}

It will sometimes be convenient to denote the process a name
quotes. We already have the notation $x = \quotep{P}$, but it will be
convenient to introduce an alternate notation, $\procn{x}$, when we
want to emphasize the connection to the use of the name. Note that, by
virtue of name equivalence, $\quotep{\procn{x}} \nameeq x$; so, the
notation is consistent with previous definitions.

Further, because names have structure it is possible to effect
substitutions on the basis of that structure. This means we need to
upgrade our notation for substitutions, which we accomplish by
adapting comprehension notation. Thus,

\begin{mathpar}
  P\{ y / x : x \in S \}
\end{mathpar}

is interpreted to mean the process derived from P by replacing (in a
capture-avoiding manner) each occurrence of $x$ in $S$ by $y$. For example,

\begin{mathpar}
  P\{ \quotep{\procn{x}|\procn{x}} / x : x \in \freenames{P} \}
\end{mathpar}

will replace each (occurrence) of a free name $x$ in $P$ by
$\quotep{\procn{x}|\procn{x}}$.

Also, we will avail ourselves of the notation $x^{L}$ and $x^{R}$ to
denote injections of a name into disjoint copies of the name
space. There are numerous ways to accomplish this. One example can be
found in \cite{MeredithR05}. This notation overloads to vectors of
names: $\vec{x}^{\pi} := (x_{i}^{\pi} \; : \; 0 \leq i < |\vec{x}| )$ where $\pi \in \{L,R\}$.

We also use $P^{\Box} := P|\Box$.

In \cite{MeredithR05} an interpretation of the new operator is
given. It turns out that there are several possible interpretations
all enjoying the requisite algebraic properties of the operator (see
\cite{milner91polyadicpi}). We will therefore make liberal use of
$(\nu\; \vec{x})P$.

% subsection the_syntax_and_semantics_of_the_notation_system (end)   

\section{Interpretation of QM}
\subsection{Supporting definitions}
\subsubsection{Multiplication}
\begin{mathpar}
  \quotep{Q} \cdot \quotep{R} := \quotep{Q|R}
  \and \\
  \quotep{Q} \cdot P := P\{ \quotep{Q|R} / \quotep{R} : \quotep{R} \in \freenames{P} \}
\end{mathpar}

\paragraph{Discussion}
The first line needs little explanation. The second line says that
each free name of the process is replaced with the multiplication of
that name by the scalar. Multiplication of a scalar (name) by a state
(process) results in a process all the names of which have been `moved
over' by parallel composition with the process the scalar
quotes. There is a subtlety that the bound names have to be
manipulated so that multiplied names aren't accidentally
captured. There are many ways to achieve this.

\begin{remark}\label{rem:multiplication_identities}
  The reader is invited to verify that for all $x,y,z \in \QProc$ and $P \in \Proc$
  \begin{mathpar}
    x \cdot \quotep{0} \equiv x 
    \and
    x \cdot y \equiv y \cdot x
    \and
    x \cdot (y \cdot z) \equiv (x \cdot y) \cdot z
    \and \\
    \quotep{0} \cdot P \equiv P
    \and \\
    x \cdot (y \cdot P) \equiv (x \cdot y) \cdot P
    \and \\
    x \cdot (P|Q) \equiv (x \cdot P) | (x \cdot Q)
    \and \\    
  \end{mathpar}
\end{remark}

\subsubsection{Tensor product}

We define a tensor product on processes by structural induction.

\paragraph{Tensor of sums} First note that all summations, including
$\pzero$ and sequence, can be written $\Sigma_{i} x_{i}.A_{i} +
\Sigma_{j} x_{j}.C_{j}$, where we have grouped input-guarded processes
together and output-guarded processes together.

Thus, we can define the tensor product of two summations, $N_{1}\otimes N_{2}$, where

\begin{mathpar}
  N_{1} := \Sigma_{i} x_{i}.A_{i} + \Sigma_{j} x_{j}.C_{j}
  \and
  N_{2} := \Sigma_{i'} y_{i'}.B_{i'} + \Sigma_{j'} y_{j'}.D_{j'} 
\end{mathpar}

as follows.

\begin{mathpar}
  \Sigma_{i} x_{i}.A_{i} + \Sigma_{j} x_{j}.C_{j} \otimes \Sigma_{i'}
  y_{i'}.B_{i'} + \Sigma_{j'} y_{j'}.D_{j'} 
  \and \\
  := \; \Sigma_{i} \Sigma_{i'} \quotep{\stackrel{\vee}{x_{i}}| \stackrel{\vee}{y_{i'}}}.(A_{i}\otimes B_{i'}) \; | \; \Sigma_{i'} \Sigma_{i} \quotep{\stackrel{\vee}{y_{i'}}|\stackrel{\vee}{x_{i}}}.(B_{i'}\otimes A_{i})
  \and
  \;\; | \;\; \Sigma_{j} \Sigma_{j'} \quotep{\stackrel{\vee}{x_{j}}|\stackrel{\vee}{y_{j'}}}.(A_{j}\otimes B_{j'}) \; | \; \Sigma_{j'} \Sigma_{j} \quotep{\stackrel{\vee}{y_{j'}}|\stackrel{\vee}{x_{j}}}.(B_{j'}\otimes A_{j})
\end{mathpar}

\begin{remark}
  Do we need to $x^{L}$ and $y^{R}$ for this construction as well?
\end{remark}

\paragraph{Tensor of parallel compositions} Next, we distribute tensor
over par.

\begin{mathpar}
  P_{1}|P_{2} \otimes Q_{1}|Q_{2} := (P_{1} \otimes Q_{1}) | (P_{1}
  \otimes Q_{2}) | (P_{2} \otimes Q_{1}) | (P_{2} \otimes Q_{2})
\end{mathpar}

\paragraph{Tensor with dropped names} We treat tensor of a
process with a dropped name as parallel composition.

\begin{mathpar}
  P \otimes \dropn{x} := P | \dropn{x}
\end{mathpar}

\paragraph{Tensor of agents}

Finally, we need to define tensor on agents. Note that the definition
of tensor on normal products only tensors inputs with inputs and
outputs with outputs. Thus, we only have to define the operation on
``homogeneous'' pairings.

\begin{mathpar}
  (\vec{x})P \otimes (\vec{y})Q
  \and \\
  := (x_{0}^{L}|y_{0}^{R},\ldots,x_{0}^{L}|y_{n}^{R},\ldots,x_{m}^{L}|y_{0}^{R},\ldots,x_{m}^{L}|y_{n}^R)(P\{ \vec{x}^{L}/\vec{x}\} \otimes Q \{ \vec{y}^{R}/\vec{y}\})
  \and \\
  \clift{\vec{P}} \otimes \clift{\vec{Q}}
  \and \\
  := \clift{P_{0}\otimes Q_{0},\ldots,P_{0}\otimes Q_{n},\ldots,P_{m}\otimes Q_{0},\ldots,P_{m}\otimes Q_{n}}
\end{mathpar}

\begin{remark}
  Observe that arities of tensored abstractions matches arities of
  tensored concretions if the original arities matched. Note also that
  the length of the arities corresponds to the increase in dimension
  we see in ordinary vector space tensor product.
\end{remark}

\begin{remark}
  Operationally, this definition distributes the tensor down to
  components ``linked'' by summation. Tensor over summation is
  intriguing in that it mixes names. Moreover, as a consequence of the
  way it mixes names we have the identities for all $x \in \QProc$ and
  $P,Q \in \Proc$

  \begin{mathpar}
    (x \cdot P) \otimes Q \equiv x \cdot (P \otimes Q) \equiv P \otimes (x \cdot Q)
    \and
    P \otimes \pzero \equiv P
  \end{mathpar}

  that the reader is invited to verify.
\end{remark}

\subsubsection{Annihilation}
\begin{mathpar}
  P^{\perp} := \{ Q | \forall R. P|Q \red^{*} R \Rightarrow R \red^{*} \pzero \}
  \and \\
  P^{\underline{\perp}} := \Sigma_{Q \in P^{\perp}} \quotep{Q}?(y).(\dropn{y}|Q) | \Sigma_{Q \in P^{\perp}} \quotep{Q}\clift{\Box}
\end{mathpar}

\paragraph{Discussion} The reader will note that $P^{\perp}$ is a
\emph{set} of processes, while $P^{\underline{\perp}}$ is a
\emph{context}. We call the set $P^{\perp}$ the \emph{annihilators} of
$P$. The parallel composition of a process in the annihilators of $P$
with $P$ will result in a process, the state space of which has all
paths eventually leading to $\pzero$. Execution may endure loops; but
under reasonable conditions of fairness (naturally guaranteed under
most notions of bisimulation) such a composite process cannot get
stuck in such a loop and will, eventually pop out and terminate.

The context $P^{\underline{\perp}}$ is ready and willing to ``take the
$P$ out of'' the process to which it is applied. It will effectively
transmit the code of the process to which it is applied to one of the
annihilators and run the process against it.

\subsubsection{Evaluation}
We fix $M$ a domain of fully abstract interpretation with an equality
coincident with bisimulation. We take $\meaningof{\cdot} : \Proc \to
M$ to be the map interpreting processes and $\nmeaningof{\cdot} : \M
\to Proc$ to be the map running the other way. Then we define

\begin{mathpar}
  \int P := \nmeaningof{\meaningof{P}}
\end{mathpar}

\paragraph{Discussion}
There are many fully abstract interpretations of Milner's
$\pi$-calculus. Any of them can be used as a basis for interpreting
the reflective calculus here. Equipped with such a domain it is
largely a matter of grinding through to check that the Yoneda
construction for the normalization-by-evaluation program can be
extended to this setting.

\begin{remark}
  The reader is invited to verify that $\int (P^{\underline{\perp}}[P]) = 0$.
\end{remark}

\subsection{Quantum mechanics}

Table \ref{tbl:core_qm_op_defns} gives the core operational definitions

\begin{table}[htp]\label{tbl:core_qm_op_defns}
  \center{
    \fbox{
      \begin{tabular}{c|c}
        quantum mechanics & process calculus \\
        \hline
        scalar & $x := \quotep{P}$ \\
        state vector & $\state{P} := P$ \\
        dual & $\state{P}^{*} := \event{P^{\underline{\perp}}} := \quotep{P^{\underline{\perp}}}[-]$ \\
        matrix & $ \Sigma_{\alpha} \state{P_{\alpha}}x_{\alpha}\event{Q_{\alpha}}$ \\
        vector addition & $\state{P} + \state{Q} := \state{P | Q}$ \\
        tensor product & $\state{P} \otimes \state{Q} := \state{P \otimes Q}$ \\
        inner product & $\innerprod{P}{Q} := \quotep{\int P^{\underline{\perp}}[Q]}$ \\
      \end{tabular}
    }
  }
  \caption{QM - operational definitions}
\end{table}

where

\begin{mathpar}
  \prmatrix{P}{Q} := \fprmatrix{P}{\quotep{\pzero}}{Q}
  \and
  \fprmatrix{P}{x}{Q} := (\state{P},x,\event{Q})
  \and
  (\fprmatrix{P}{x}{Q})(\state{R}) := x \cdot \innerprod{Q}{R} \cdot \state{P}
  \and
  (\fprmatrix{P}{x}{Q})(\event{R}) := x \cdot \innerprod{R}{P} \cdot \event{Q}
\end{mathpar}

\paragraph{Discussion}
As promised: vectors (aka states) are represented as processes; duals
as contextual duals; inner product definition should be compared with
standard inner product definition for ....

\begin{remark}
  Assuming $\int (P^{\underline{\perp}}[P]) = 0$, the reader is
  invited to verify that $(\fprmatrix{P}{x}{P})(\state{P}) = x \cdot \state{P}$.
\end{remark}

\begin{remark}
  The reader is invited to verify that $\innerprod{P}{Q}$ could
  equally well have been written $\quotep{\int \stackrel{\vee}{x}}$
  where $x = \event{P^{\underline{\perp}}}(Q)$.

  One of the motivations for this remark is that there is another way
  to factor these operations. We could package up evaluation in the dual:

  \begin{mathpar}
    \state{P}^{*} := \event{\int P^{\underline{\perp}}} := \quotep{\int P^{\underline{\perp}}}[-]
  \end{mathpar}

  and then have inner product defined by
  
  \begin{mathpar}
    \innerprod{P}{Q} := \event{P}(Q)
  \end{mathpar}

  Hopefully, experience with the calculations will provide guidance on
  the best factoring.
\end{remark}

\begin{remark}
  Assuming $\int (P^{\underline{\perp}}[P]) = 0$, the reader is
  invited to verify that $\forall P,Q. (\prmatrix{0}{Q})(\state{0}) =
  \state{0}$ and dually $(\prmatrix{P}{0})(\event{0}) = \event{0}$.
\end{remark}

\begin{remark}
  i'm a little worried that i don't (yet) have proper support for
  complex conjugacy. But, the observation above may give us a
  clue. According to Abramsky, it must be the case that the scalars
  are iso to the homset of the identity for the tensor -- which the
  observation above characterizes. 

  For now, we will simply bookmark the notion with $\overline{x}$.
\end{remark}

\subsubsection{Adjointness}

We need to give a definition of $(\cdot)^{\dagger}$ for matrices. The
obvious candidate definition is
\begin{mathpar}
(\Sigma_{\alpha}\fprmatrix{P_{\alpha}}{x_{\alpha}}{Q_{\alpha}})^{\dagger}
= \Sigma_{\alpha}\fprmatrix{(Q_{\alpha}^{\underline{\perp}})^{*}}{\overline{x}_{\alpha}}{P_{\alpha}^{\underline{\perp}}} 
\end{mathpar}

But, $(Q_{\alpha}^{\underline{\perp}})^{*}$ requires a name along
which to communicate the process to achieve the context application.

\subsubsection{Basis for a basis}
If processes label states and ``addition'' of states (a.k.a. vector
addition) is interpreted as parallel composition, what corresponds to
notions of linear independence and basis? Here, we recall that Yoshida
has developed a set of \emph{combinators} for an asynchronous verison
of Milner's $\pi$-calculus. These are a finite set of processes such
any process can be expressed as parallel composition of these
combinators together with liberal uses of the new operator and
replication. We can simply give a translation of these into the
present calculus and have reasonable expectation that the property
carries over. That is, that the resultant set allows to express all
processes via parallel composition. Note, however, that there is no
new operator or replication in this calculus. As a result, we expect
that the corresponding set is actually infinite. That is, we expect
that the space is actually infinite dimensional.

\begin{remark}
  The attentive reader may be a bit concerned. Certainly, the
  collection $S$, $K$ and $I$ is a finite set of
  combinators. Shouldn't we expect to see a finite set of combinators
  for an effectively equivalent system? i am very sympathetic to this
  critique and feel it warrants full attention. On the other hand, i
  also have in mind the following analogy. The natural numbers, as a
  monoid under addition, has exactly $1$ generator, while the natural
  numbers, as a monoid under multiplication, has countably many
  generators (the primes). We observe that the application of the
  lambda calculus is much less resource sensitive than the parallel
  composition of the $\pi$-calculus. Could it be the case that we have
  an analogy of the form
  
  \begin{mathpar}
    m + n : MN :: m*n : M|N
  \end{mathpar}

  giving a similar blow up in the set of ``primes''?  This is such a
  wonderful thought that, even if it's not true, i think it's worth
  writing down.
\end{remark}
 

\documentclass[12pt]{llncs}
%\documentclass{jktr}

\usepackage[pdftex]{hyperref}                   
\usepackage {listings}
\usepackage {mathpartir}
\usepackage{bcprules}
%\usepackage{listings}
                       
\usepackage{graphicx} 
%\usepackage[margins=2.5cm,nohead,nofoot]{geometry}
%\usepackage{geometry}
\usepackage{amsfonts}
\usepackage{amstext}
\usepackage{latexsym}
\usepackage{amssymb}
\usepackage{color}


%\include{myPreamble}
\include{qm2pi.local} 

%\ifpdf
%\usepackage[pdftex]{graphicx}
%\else
%\usepackage{graphicx}
%\fi

 % \ifpdf
%  \usepackage{pdfsync}
%  \if


%\title{Brief Article}
%\author{David F. Snyder}
%\author{L.G. Meredith}

%\address{Dept. of Math., Texas State University--San Marcos, San Marcos, TX 78666}
       
\pagestyle{empty}


\begin{document}

\lstset{language=[Objective]Caml,frame=shadowbox}

\input{qm2pi.front}

% section front matter (end)

\input{qm2pi.intro} 
 
% section introduction (end)

% \input{qm2pi.knotations} 

% section notation (end)

\input{qm2pi.process.calculi} 

% section concurrent_process_calculi_and_spatial_logics_ (end)
    
%\input{qm2pi.knots2pi} 

%\input{qm2pi.trefoil} 

%\input{qm2pi.mainthm} 

% subsection basic_interpretation (end)

%\input{qm2pi.rho.presentation} 
\subsection{The syntax and semantics of the notation system}\label{sub:the_syntax_and_semantics_of_the_notation_system} % (fold)

We now summarize a technical presentation of the calculus that
embodies our theory of dynamics. The typical presentation of such a
calculus follows the style of giving generators and relations on
them. The grammar, below, describing term constructors, freely
generates the set of processes, $\Proc$. This set is then quotiented
by a relation known as structural congruence and it is over this set
that the notion of dynamics is expressed. This presentation is
essentially that of \cite{MeredithR05} with the addition of
polyadicity and summation. For readability we have relegated some of
the technical subtleties to an appendix.

\subsubsection{Process grammar}\label{subsub:process_grammar}

\begin{mathpar}
  \inferrule* [lab=synchronization] {} {{M} \bc \pzero \;|\; x?F \;|\; x!C }
  \and
  \inferrule* [lab=abstraction] {} {{F} \bc (x)P}
  \and
  \inferrule* [lab=concretion] {} {{C} \bc \langle Q \rangle}
  \and
  \inferrule* [lab=process] {} {{P,Q} \bc M \;| \;P|Q \;|\; @{x}}
  \and
  \inferrule* [lab=name] {} {{x} \bc \quotep{P}}
\end{mathpar} 

Note that $\vec{x}$ (resp. $\vec{P}$) denotes a vector of names
(resp. processes) of length $|\vec{x}|$ (resp. $|\vec{P}|$). We adopt
the following useful abbreviations.

\begin{mathpar}
   x?(\vec{y}).P := x.(\vec{y})P \and  x\clift{\vec{P}} := x.\clift{\vec{P}}
   \and x!(y) := \lift{x}{\dropn{y}}
   \and \Pi_{i=0}^{n-1}P_i := P_0 | \ldots | P_{n-1}
\end{mathpar}

\subsubsection{Structural congruence}

\paragraph{Free and bound names and alpha-equivalence.} At the
core of structural equivalence is alpha-equivalence which identifies
process that are the same up to a change of variable. Formally, we
recognize the distinction between free and bound names. The free names
of a process, $\freenames{P}$, may be calculated recursively as
follows:

\begin{mathpar}
\freenames{\pzero} := \emptyset
  \and \\
  \freenames{x?(y).P} := \{ x \} \cup (\freenames{P} \setminus \{ y \})
  \and 
  \freenames{x!\langle P \rangle} := \{ x \} \cup \{ P \} 
  \and \\
  \freenames{P|Q} := \freenames{P} \cup \freenames{Q}
  \and \\
  \freenames{@{x}} := \{ x \}
\end{mathpar}

$\pi$
$\quotep{\pi}$

$\freenames{-} : \pi \to \mathcal{P}(\quotep{\pi})$

\begin{eqnarray*}
  \freenames{\pzero} & := & \emptyset \\
  \freenames{x?(y).P} & := & \{ x \} \cup (\freenames{P} \setminus \{ y \}) \\
  \freenames{x!\langle P \rangle} & := & \{ x \} \cup \{ P \} \\
  \freenames{P|Q} & := & \freenames{P} \cup \freenames{Q} \\
  \freenames{\dropn{x}} & := & \{ x \}
\end{eqnarray*}

The bound names of a process, $\boundnames{P}$, are those names occurring in $P$
that are not free. For example, in $x?(y).0$, the name $x$ is free, while $y$ is bound.

\begin{mathpar}
  \inferrule* [lab=monoidal-laws] {} { P|Q \equiv Q|P \and P|0 \equiv P \and P|(Q|R) \equiv (P|Q)|R }
\end{mathpar}

\begin{mathpar}
  \inferrule* [lab=alpha-equivalence] {} { (x)P \equiv (y)P\{y/x\} \and y \not\in \freenames{P} }
\end{mathpar}

\begin{definition}
Then two processes, $P,Q$, are alpha-equivalent if $P = Q\{\vec{y}/\vec{x}\}$ for
some $\vec{x} \in \boundnames{Q},\vec{y} \in \boundnames{P}$, where $Q\{\vec{y}/\vec{x}\}$
denotes the capture-avoiding substitution of $\vec{y}$ for $\vec{x}$ in $Q$.
\end{definition}

\begin{definition}
  The {\em structural congruence} \cite{SangiorgiWalker} , $\equiv$,
  between processes is the least congruence containing
  alpha-equivalence, satisfying the abelian monoid laws
  (associativity, commutativity and $\pzero$ as identity) for parallel
  composition $|$ and for summation $+$.
\end{definition}

\subsection{Name equivalence}

We take name equivalence, written $\nameeq$, to be the smallest
equivalence relation generated by the following rules.

\begin{mathpar}
\inferrule*[lab=Quote-drop]
{ }
{ \quotep{@{x}} \nameeq x }

\inferrule*[lab=Struct-equiv]
{ P \scong Q }
{ \quotep{P} \nameeq \quotep{Q} }
\end{mathpar}

The astute reader will have noticed that the mutual recursion of names
and processes imposes a mutual recursion on alpha-equivalence and
structural equivalence via name-equivalence. Fortunately, all of this
works out pleasantly and we may calculate in the natural way, free of
concern. The reader interested in the details is referred to the
appendix \ref{appendix:rho_details}.

\subsection{Substitution}

We use $\Proc$ for the set of processes, $\QProc$ for the set of
names, and $\id{\{}\vec{y} / \vec{x} \id{\}}$ to denote partial maps,
$s : \QProc \rightarrow \QProc$. A map, $s$ lifts, uniquely, to a map
on process terms, $\widehat{s} : \Proc \rightarrow \Proc$ by the
following equations.

\begin{mathpar}
  (0) \psubstp{Q}{P} := 0 \\
  (R \juxtap S) \psubstp{Q}{P}
  :=    
  (R)\psubstp{Q}{P} \juxtap (S) \psubstp{Q}{P} \\
  (x?(y).R) \psubstp{Q}{P}    
  :=    
  (x)\substp{Q}{P} (z)\concat( (R \psubstn{z}{y}) \psubstp{Q}{P} ) \\
  (\lift{x}{R}) \psubstp{Q}{P}  
  :=
  \lift{(x)\substp{Q}{P}}{ R \psubstp{Q}{P} } \\
%   (\dropn{x})  \psubstp{Q}{P}       
%   := 
%   \left\{ 
%     \begin{array}{ccc} 
%       \dropn{\quotep{Q}} & & x \nameeq \quotep{P} \\
%       \dropn{x} & & otherwise \\
%     \end{array}
%   \right. 
  (\dropn{x})  \psubstp{Q}{P}       
  := 
  \left\{ 
    \begin{array}{ccc} 
      Q & & x \nameeq \quotep{P} \\
      \dropn{x} & & otherwise \\
    \end{array}
  \right.
\end{mathpar}
 

where

\begin{eqnarray}
  (x)\id{\{} \lpquote Q \rpquote / \lpquote P \rpquote \id{\}}            = 
  \left\{ 
    \begin{array}{ccc}
      \lpquote Q \rpquote & & x \nameeq \lpquote P \rpquote \\
      x & & otherwise \\
    \end{array}
  \right. \nonumber
\end{eqnarray}

and $z$ is chosen distinct from $\quotep{P}$, $\quotep{Q}$, the free
names in $Q$, and all the names in $R$. Our $\alpha$-equivalence will
be built in the standard way from this substitution.

\begin{remark}\label{rem:no_self_referential_names}
  One consequence of these definitions is that $\forall P. \quotep{P}
  \not\in \freenames{P}$.
\end{remark}

\subsection{ Dynamic quote: an example }

Anticipating something of what's to come, consider applying the
substitution, $\widehat{\id{\{}u / z \id{\}}}$, to the following pair
of processes, $\lift{w}{y!(z)}$ and $w[ \lpquote y!(z) \rpquote ]$.

\begin{eqnarray}
	\lift{w}{y!(z)}\widehat{\id{\{}u / z \id{\}}}
		& = &
		\lift{w}{y!(u)} \nonumber\\
	w[ \lpquote y!(z) \rpquote ] \widehat{ \id{\{}u / z \id{\}} }
		& = &
		w[ \lpquote y!(z) \rpquote ] \nonumber
\end{eqnarray}

Because the body of the process between quotes is impervious to
substitution, we get radically different answers. In fact, by
examining the first process in an input context,
e.g. $x?(z).\lift{w}{y!(z)}$, we see that the process under the lift
operator may be shaped by prefixed inputs binding a name inside it. In
this sense, the lift operator will be seen as a way to dynamically
construct processes before reifying them as names.

Finally equipped with these standard features we can present the
dynamics of the calculus.

\subsubsection{Operational semantics} 

Finally, we introduce the computational dynamics. What marks these
algebras as distinct from other more traditionally studied algebraic
structures, e.g. vector spaces or polynomial rings, is the manner in
which dynamics is captured. In traditional structures, dynamics is typically
expressed through morphisms between such structures, as in linear maps
between vector spaces or morphisms between rings. In algebras
associated with the semantics of computation, the dynamics is
expressed as part of the algebraic structure itself, through a
reduction reduction relation typically denoted by $\red$. Below, we
give a recursive presentation of this relation for the calculus used
in the encoding.

$\red \subseteq \pi \times \pi$
$\red : \pi \to \mathcal{P}(\pi)$

\begin{mathpar}
  \inferrule* [lab=Comm] { \textsf{match}( x_{src}, x_{trgt} ) } { x_{trgt}?(y)P \; | \; x_{src}!\langle {Q} \rangle \red P\{\quotep{Q}/y}\} }
  \and \\
  \inferrule* [lab=Par] {{P} \red {P}'} {{{P} | {Q}} \red {{P}' | {Q}}}
  \and
  \inferrule* [lab=Equiv]{{{P} \scong {P}'} \andalso {{P}' \red {Q}'} \andalso {{Q}' \scong {Q}}}{{P} \red {Q}}
\end{mathpar}

\begin{eqnarray*}
  match_{\equiv} (\quotep{P},\quotep{Q}) & := & P \equiv Q \\
  match_{\dagger}(\quotep{P},\quotep{Q}) & := & \forall R. P|Q \red^{*} R => R \red^{*} 0 \\
  match_{K}(\quotep{P},\quotep{Q}) & := & K \mbox{ for some context } K
\end{eqnarray*}

$u?(x)P | u!\langle Q \rangle \red P\{\quotep{Q}/x\}$

%We write $\wred$ for $\red^*$, and $P\red$ if $\exists Q $ such that $ P \red Q$.
We write $P\red$ if $\exists Q $ such that $ P \red Q$ and $P\not\red$, otherwise.

\section{Replication}

As mentioned before, it is known that replication (and hence
recursion) can be implemented in a higher-order process algebra
\cite{SangiorgiWalker}. As our first example of calculation with the
machinery thus far presented we give the construction explicitly in
the {\rhoc}.

\begin{eqnarray}
	D_{x} & := & \prefix{x}{y}{(\binpar{\outputp{x}{y}}{@{y}})} \nonumber\\
	\bangp_{x}{P} & := & \binpar{{x}!\langle{\binpar{D_{x}}{P}}\rangle}{D_{x}} \nonumber
\end{eqnarray}

\begin{eqnarray}
	\bangp_{x}{P} & & \nonumber\\
	=
	& {x}!\langle{(\prefix{x}{y}{(\outputp{x}{y} | @{y})) | P}}\rangle 
	      | \prefix{x}{y}{(\outputp{x}{y} | @{y})} & \nonumber\\
	\red
	& (\outputp{x}{y} | @{y})\substn{\quotep{(\prefix{x}{y}{(@{y} | \outputp{x}{y})) | P}}}{y} & \nonumber\\
	=
	& \outputp{x}{\quotep{(\prefix{x}{y}{(\outputp{x}{y} | @{y})) | P}}}
	  | {(\prefix{x}{y}{(\outputp{x}{y} | @{y})) | P}} & \nonumber\\
	\red
	& \ldots & \nonumber\\
	\red^*
	& P | P | \ldots & \nonumber
\end{eqnarray}

Of course, this encoding, as an implementation, runs away, unfolding
$\bangp{P}$ eagerly. A lazier and more implementable replication
operator, restricted to input-guarded processes, may be obtained as follows.

\begin{eqnarray}
\bangp{\prefix{u}{v}{P}} 
	:= 
	\binpar{\lift{x}{\prefix{u}{v}{(\binpar{D(x)}{P})}}}{D(x)} \nonumber
\end{eqnarray}

\begin{remark}
  Note that the lazier definition still does not deal with summation
  or mixed summation (i.e. sums over input and output). The reader is
  invited to construct definitions of replication that deal with these
  features. 

  Further, the definitions are parameterized in a name, $x$. Can you,
  gentle reader, make a definition that eliminates this parameter and
  guarantees no accidental interaction between the replication
  machinery and the process being replicated -- i.e. no accidental
  sharing of names used by the process to get its work done and the
  name(s) used by the replication to effect copying. This latter
  revision of the definition of replication is crucial to obtaining
  the expected identity $!!P \sim !P$.
\end{remark}

\begin{remark}\label{rem:paradoxical_combinator}
  The reader familiar with the lambda calculus will have noticed the
  similarity between $D$ and the paradoxical combinator.

  [Ed. note: the existence of this seems to suggest we have to be more
  restrictive on the set of processes and names we admit if we are to
  support no-cloning.]
\end{remark}

\subsubsection{Bisimulation}

The computational dynamics gives rise to another kind of equivalence,
the equivalence of computational behavior. As previously mentioned
this is typically captured \emph{via} some form of bisimulation.

% The notion we use in this paper is weak barbed bisimulation
% \cite{milner91polyadicpi}.

The notion we use in this paper is derived from weak barbed
bisimulation \cite{milner91polyadicpi}. 

\begin{definition}
An \emph{observation relation}, $\downarrow_{\mathcal N}$, over a set
of names, $\mathcal N$, is the smallest relation satisfying the rules
below.

\infrule[Out-barb]{y \in {\mathcal N}, \; x \nameeq y}
		  {\outputp{x}{v} \downarrow_{\mathcal N} x}
\infrule[Par-barb]{\mbox{$P\downarrow_{\mathcal N} x$ or $Q\downarrow_{\mathcal N} x$}}
		  {\binpar{P}{Q} \downarrow_{\mathcal N} x}

We write $P \Downarrow_{\mathcal N} x$ if there is $Q$ such that 
$P \wred Q$ and $Q \downarrow_{\mathcal N} x$.
\end{definition}

\begin{definition}
%\label{def.bbisim}
An  ${\mathcal N}$-\emph{barbed bisimulation} over a set of names, ${\mathcal N}$, is a symmetric binary relation 
${\mathcal S}_{\mathcal N}$ between agents such that $P\rel{S}_{\mathcal N}Q$ implies:
\begin{enumerate}
\item If $P \red P'$ then $Q \wred Q'$ and $P'\rel{S}_{\mathcal N} Q'$.
\item If $P\downarrow_{\mathcal N} x$, then $Q\Downarrow_{\mathcal N} x$.
\end{enumerate}
$P$ is ${\mathcal N}$-barbed bisimilar to $Q$, written
$P \wbbisim_{\mathcal N} Q$, if $P \rel{S}_{\mathcal N} Q$ for some ${\mathcal N}$-barbed bisimulation ${\mathcal S}_{\mathcal N}$.
\end{definition}

$\mathcal{R} \subseteq \pi \times \pi$

$P \mathcal{R} Q => \forall P'. P \red P' \Rightarrow \exists Q'. Q \red Q', P' \mathcal{R} Q'$

$P \vdash x \Rightarrow Q \vdash x$

\begin{mathpar}
  \inferrule*[lab=Out-barb]{x \nameeq y}{{y}!\langle{Q}\rangle \vdash x}
  \and
  \inferrule*[lab=Par-barb]{\mbox{$P\vdash x$ or $Q\vdash x$}}{\binpar{P}{Q} \vdash x}
\end{mathpar}

\subsubsection{Contexts}

One of the principle advantages of computational calculi like the
$\pi$-calculus is a well-defined notion of context,
contextual-equivalence and a correlation between
contextual-equivalence and notions of bisimulation. The notion of
context allows the decomposition of a process into (sub-)process and
its syntactic environment, its context. Thus, a context may be
thought of as a process with a ``hole'' (written $\Box$) in it. The
application of a context $M$ to a process $P$, written $M[P]$, is
tantamount to filling the hole in $M$ with $P$. In this paper we do
not need the full weight of this theory, but do make use of the notion
of context in the proof the main theorem. 

\begin{mathpar}
  \inferrule* [lab=summation] {} {{M_{M},M_{N}} \bc \Box \;|\; x.M_{A} \;|\; M_{M}+M_{N}}
  \and
  \inferrule* [lab=agent] {} {{M_{A}} \bc (\vec{x})M_{P} \;| \; \clift{P_0,\ldots,M_{P},\ldots,P_N}}
  \and \\
  \inferrule* [lab=process] {} {{M_{P}} \bc M_{N} \;| \;P|M_{P} }
\end{mathpar} 

\begin{mathpar}
  \inferrule* [lab=sychronization] {} {M_{N} \bc \Box \;|\; x?M_{F} \;|\; x!M_{C}}
  \and
  \inferrule* [lab=abstraction] {} {{M_{F}} \bc (x)M_{P} }
  \and
  \inferrule* [lab=concretion] {} {{M_{C}} \bc \langle M_{P} \rangle }
  \and \\
  \inferrule* [lab=process] {} {{M_{P}} \bc M_{N} \;| \;P|M_{P} }
\end{mathpar}

\begin{definition}[contextual application] Given a context $M$, and
  process $P$, we define the \emph{contextual application}, $M[P] :=
  M\{P/\Box\}$. That is, the contextual application of M to P is the
  substitution of $P$ for $\Box$ in $M$.
\end{definition}

$\meaningof{-} : L \to \mathcal{P}(\pi)$

\begin{mathpar}
  \inferrule* [lab=collection] {} {\meaningof{true} = \pi, \and \meaningof{~E} = \pi \setminus \meaningof{E}, \and \meaningof{E_{1} \& E_{2}} = \meaningof{E_{1}} \cap \meaningof{E_{2}}}
\end{mathpar}

\begin{mathpar}
  \inferrule* [lab=structure] {} {\meaningof{0} = \{ P \in \pi | P \equiv 0 \}, \and \\ \meaningof{E_1 | E_2} = \{ P \in \pi | P \equiv P_{1} | P_{2}, P_{1} \in \meaningof{E_{1}}, P_{2} \in \meaningof{E_2}\} }
\end{mathpar}

\begin{mathpar}
 \inferrule* [lab=behavior] {} {\meaningof{\langle a?b \rangle E} = \{ P \in \pi | P \equiv Q | u?(y)P', \\ \and \\\\ \and \\ \;\;\; u \in \meaningof{a}, \forall z.P'\{z/y\} \in \meaningof{E\{z/b\}}\}, \and \\ \meaningof{a!E} = \{ P \in \pi | P \equiv Q | x!\langle P' \rangle, x \in \meaningof{a} P' \in \meaningof{E}\} }
\end{mathpar}

\begin{mathpar}
 \inferrule* [lab=nominal] {} {\meaningof{\quotep{E}} = \{ \quotep{P} \in \quotep{\pi} | P \in \meaningof{E} \}, \and \meaningof{\quotep{P}} = \{ \quotep{Q} \in \quotep{\pi} | P \equiv Q \} \and \\ \meaningof{@\quotep{E}} = \{ P \in \pi | P \equiv @x, x \in \meaningof{E} \}}
\end{mathpar}

\begin{eqnarray*}
  \\
  \meaningof{-} : TS \to ST
\end{eqnarray*}

\begin{eqnarray*}
  \\
  L : TS \to ST
\end{eqnarray*}

\begin{eqnarray*}
  \\
  P \models E \iff P \in \meaningof{E}
\end{eqnarray*}

\begin{eqnarray*}
  P \approx_{L} Q \iff \forall E \in L. P \models E \iff Q \models E
\end{eqnarray*}

\begin{eqnarray*}
  P \approx_{K} Q
\end{eqnarray*}

\begin{eqnarray*}
  P \approx Q
\end{eqnarray*}

$\approx_{K} = \approx = \approx_{L}$

\subsubsection{Contextual duality}

Note that contexts extend the quotation operation to a family of
operations from processes to names. Given a context, $M$, we can
define a \emph{nominal context}, $\quotep{M}$ by $\quotep{M}[P] :=
\quotep{M[P]}$. To foreshadow what is to come we observe that these
operations enjoy a duality with processes very much like the duality
between vectors and maps from vectors to scalars.

Further, because the calculus is essentially higher-order, we have a
correspondence between contexts and processes. More specifically,
given a name $x$ and a context $M$ we can construct $M^{*}_{x}$ such
that 

\begin{mathpar}
  M^{*}_{x} | \lift{x}{P} \red M[P]
\end{mathpar}

namely,

\begin{mathpar}
  M^{*}_{x} := x?(u).M[\dropn{u}]
\end{mathpar}

The dependence of $M^{*}_{x}$ on a name makes it an abstraction, 

\begin{mathpar}
  M^{*} := (x)x?(u).M[\dropn{u}]
\end{mathpar}

\subsection{Additional notation}

It will sometimes be convenient to denote the process a name
quotes. We already have the notation $x = \quotep{P}$, but it will be
convenient to introduce an alternate notation, $\procn{x}$, when we
want to emphasize the connection to the use of the name. Note that, by
virtue of name equivalence, $\quotep{\procn{x}} \nameeq x$; so, the
notation is consistent with previous definitions.

Further, because names have structure it is possible to effect
substitutions on the basis of that structure. This means we need to
upgrade our notation for substitutions, which we accomplish by
adapting comprehension notation. Thus,

\begin{mathpar}
  P\{ y / x : x \in S \}
\end{mathpar}

is interpreted to mean the process derived from P by replacing (in a
capture-avoiding manner) each occurrence of $x$ in $S$ by $y$. For example,

\begin{mathpar}
  P\{ \quotep{\procn{x}|\procn{x}} / x : x \in \freenames{P} \}
\end{mathpar}

will replace each (occurrence) of a free name $x$ in $P$ by
$\quotep{\procn{x}|\procn{x}}$.

Also, we will avail ourselves of the notation $x^{L}$ and $x^{R}$ to
denote injections of a name into disjoint copies of the name
space. There are numerous ways to accomplish this. One example can be
found in \cite{MeredithR05}. This notation overloads to vectors of
names: $\vec{x}^{\pi} := (x_{i}^{\pi} \; : \; 0 \leq i < |\vec{x}| )$ where $\pi \in \{L,R\}$.

We also use $P^{\Box} := P|\Box$.

In \cite{MeredithR05} an interpretation of the new operator is
given. It turns out that there are several possible interpretations
all enjoying the requisite algebraic properties of the operator (see
\cite{milner91polyadicpi}). We will therefore make liberal use of
$(\nu\; \vec{x})P$.

% subsection the_syntax_and_semantics_of_the_notation_system (end)   

\input{qm2pi.qmops} 

\input{qm2pi.sterngerlach} 

\input{qm2pi.metric} 

% section concurrent_process_calculi (end)

%\input{qm2pi.proofsketch}

% section proof sketch (end)

%\input{qm2pi.slviaknots} 

% section spatial logic via knots (end)

\input{qm2pi.conclusion}

% section conclusion (end)

%\input{qm2pi.dtcodes} 

% section wiring algorithm (end)

\input{qm2pi.ack} 

% section acknowledgments (end)

\newpage


\bibliographystyle{plain}   
\bibliography{../../biblios/main.bib}

\input{qm2pi.rhodetails}

\end{document}

 

\documentclass[12pt]{llncs}
%\documentclass{jktr}

\usepackage[pdftex]{hyperref}                   
\usepackage {listings}
\usepackage {mathpartir}
\usepackage{bcprules}
%\usepackage{listings}
                       
\usepackage{graphicx} 
%\usepackage[margins=2.5cm,nohead,nofoot]{geometry}
%\usepackage{geometry}
\usepackage{amsfonts}
\usepackage{amstext}
\usepackage{latexsym}
\usepackage{amssymb}
\usepackage{color}


%\include{myPreamble}
\include{qm2pi.local} 

%\ifpdf
%\usepackage[pdftex]{graphicx}
%\else
%\usepackage{graphicx}
%\fi

 % \ifpdf
%  \usepackage{pdfsync}
%  \if


%\title{Brief Article}
%\author{David F. Snyder}
%\author{L.G. Meredith}

%\address{Dept. of Math., Texas State University--San Marcos, San Marcos, TX 78666}
       
\pagestyle{empty}


\begin{document}

\lstset{language=[Objective]Caml,frame=shadowbox}

\input{qm2pi.front}

% section front matter (end)

\input{qm2pi.intro} 
 
% section introduction (end)

% \input{qm2pi.knotations} 

% section notation (end)

\input{qm2pi.process.calculi} 

% section concurrent_process_calculi_and_spatial_logics_ (end)
    
%\input{qm2pi.knots2pi} 

%\input{qm2pi.trefoil} 

%\input{qm2pi.mainthm} 

% subsection basic_interpretation (end)

%\input{qm2pi.rho.presentation} 
\subsection{The syntax and semantics of the notation system}\label{sub:the_syntax_and_semantics_of_the_notation_system} % (fold)

We now summarize a technical presentation of the calculus that
embodies our theory of dynamics. The typical presentation of such a
calculus follows the style of giving generators and relations on
them. The grammar, below, describing term constructors, freely
generates the set of processes, $\Proc$. This set is then quotiented
by a relation known as structural congruence and it is over this set
that the notion of dynamics is expressed. This presentation is
essentially that of \cite{MeredithR05} with the addition of
polyadicity and summation. For readability we have relegated some of
the technical subtleties to an appendix.

\subsubsection{Process grammar}\label{subsub:process_grammar}

\begin{mathpar}
  \inferrule* [lab=synchronization] {} {{M} \bc \pzero \;|\; x?F \;|\; x!C }
  \and
  \inferrule* [lab=abstraction] {} {{F} \bc (x)P}
  \and
  \inferrule* [lab=concretion] {} {{C} \bc \langle Q \rangle}
  \and
  \inferrule* [lab=process] {} {{P,Q} \bc M \;| \;P|Q \;|\; @{x}}
  \and
  \inferrule* [lab=name] {} {{x} \bc \quotep{P}}
\end{mathpar} 

Note that $\vec{x}$ (resp. $\vec{P}$) denotes a vector of names
(resp. processes) of length $|\vec{x}|$ (resp. $|\vec{P}|$). We adopt
the following useful abbreviations.

\begin{mathpar}
   x?(\vec{y}).P := x.(\vec{y})P \and  x\clift{\vec{P}} := x.\clift{\vec{P}}
   \and x!(y) := \lift{x}{\dropn{y}}
   \and \Pi_{i=0}^{n-1}P_i := P_0 | \ldots | P_{n-1}
\end{mathpar}

\subsubsection{Structural congruence}

\paragraph{Free and bound names and alpha-equivalence.} At the
core of structural equivalence is alpha-equivalence which identifies
process that are the same up to a change of variable. Formally, we
recognize the distinction between free and bound names. The free names
of a process, $\freenames{P}$, may be calculated recursively as
follows:

\begin{mathpar}
\freenames{\pzero} := \emptyset
  \and \\
  \freenames{x?(y).P} := \{ x \} \cup (\freenames{P} \setminus \{ y \})
  \and 
  \freenames{x!\langle P \rangle} := \{ x \} \cup \{ P \} 
  \and \\
  \freenames{P|Q} := \freenames{P} \cup \freenames{Q}
  \and \\
  \freenames{@{x}} := \{ x \}
\end{mathpar}

$\pi$
$\quotep{\pi}$

$\freenames{-} : \pi \to \mathcal{P}(\quotep{\pi})$

\begin{eqnarray*}
  \freenames{\pzero} & := & \emptyset \\
  \freenames{x?(y).P} & := & \{ x \} \cup (\freenames{P} \setminus \{ y \}) \\
  \freenames{x!\langle P \rangle} & := & \{ x \} \cup \{ P \} \\
  \freenames{P|Q} & := & \freenames{P} \cup \freenames{Q} \\
  \freenames{\dropn{x}} & := & \{ x \}
\end{eqnarray*}

The bound names of a process, $\boundnames{P}$, are those names occurring in $P$
that are not free. For example, in $x?(y).0$, the name $x$ is free, while $y$ is bound.

\begin{mathpar}
  \inferrule* [lab=monoidal-laws] {} { P|Q \equiv Q|P \and P|0 \equiv P \and P|(Q|R) \equiv (P|Q)|R }
\end{mathpar}

\begin{mathpar}
  \inferrule* [lab=alpha-equivalence] {} { (x)P \equiv (y)P\{y/x\} \and y \not\in \freenames{P} }
\end{mathpar}

\begin{definition}
Then two processes, $P,Q$, are alpha-equivalent if $P = Q\{\vec{y}/\vec{x}\}$ for
some $\vec{x} \in \boundnames{Q},\vec{y} \in \boundnames{P}$, where $Q\{\vec{y}/\vec{x}\}$
denotes the capture-avoiding substitution of $\vec{y}$ for $\vec{x}$ in $Q$.
\end{definition}

\begin{definition}
  The {\em structural congruence} \cite{SangiorgiWalker} , $\equiv$,
  between processes is the least congruence containing
  alpha-equivalence, satisfying the abelian monoid laws
  (associativity, commutativity and $\pzero$ as identity) for parallel
  composition $|$ and for summation $+$.
\end{definition}

\subsection{Name equivalence}

We take name equivalence, written $\nameeq$, to be the smallest
equivalence relation generated by the following rules.

\begin{mathpar}
\inferrule*[lab=Quote-drop]
{ }
{ \quotep{@{x}} \nameeq x }

\inferrule*[lab=Struct-equiv]
{ P \scong Q }
{ \quotep{P} \nameeq \quotep{Q} }
\end{mathpar}

The astute reader will have noticed that the mutual recursion of names
and processes imposes a mutual recursion on alpha-equivalence and
structural equivalence via name-equivalence. Fortunately, all of this
works out pleasantly and we may calculate in the natural way, free of
concern. The reader interested in the details is referred to the
appendix \ref{appendix:rho_details}.

\subsection{Substitution}

We use $\Proc$ for the set of processes, $\QProc$ for the set of
names, and $\id{\{}\vec{y} / \vec{x} \id{\}}$ to denote partial maps,
$s : \QProc \rightarrow \QProc$. A map, $s$ lifts, uniquely, to a map
on process terms, $\widehat{s} : \Proc \rightarrow \Proc$ by the
following equations.

\begin{mathpar}
  (0) \psubstp{Q}{P} := 0 \\
  (R \juxtap S) \psubstp{Q}{P}
  :=    
  (R)\psubstp{Q}{P} \juxtap (S) \psubstp{Q}{P} \\
  (x?(y).R) \psubstp{Q}{P}    
  :=    
  (x)\substp{Q}{P} (z)\concat( (R \psubstn{z}{y}) \psubstp{Q}{P} ) \\
  (\lift{x}{R}) \psubstp{Q}{P}  
  :=
  \lift{(x)\substp{Q}{P}}{ R \psubstp{Q}{P} } \\
%   (\dropn{x})  \psubstp{Q}{P}       
%   := 
%   \left\{ 
%     \begin{array}{ccc} 
%       \dropn{\quotep{Q}} & & x \nameeq \quotep{P} \\
%       \dropn{x} & & otherwise \\
%     \end{array}
%   \right. 
  (\dropn{x})  \psubstp{Q}{P}       
  := 
  \left\{ 
    \begin{array}{ccc} 
      Q & & x \nameeq \quotep{P} \\
      \dropn{x} & & otherwise \\
    \end{array}
  \right.
\end{mathpar}
 

where

\begin{eqnarray}
  (x)\id{\{} \lpquote Q \rpquote / \lpquote P \rpquote \id{\}}            = 
  \left\{ 
    \begin{array}{ccc}
      \lpquote Q \rpquote & & x \nameeq \lpquote P \rpquote \\
      x & & otherwise \\
    \end{array}
  \right. \nonumber
\end{eqnarray}

and $z$ is chosen distinct from $\quotep{P}$, $\quotep{Q}$, the free
names in $Q$, and all the names in $R$. Our $\alpha$-equivalence will
be built in the standard way from this substitution.

\begin{remark}\label{rem:no_self_referential_names}
  One consequence of these definitions is that $\forall P. \quotep{P}
  \not\in \freenames{P}$.
\end{remark}

\subsection{ Dynamic quote: an example }

Anticipating something of what's to come, consider applying the
substitution, $\widehat{\id{\{}u / z \id{\}}}$, to the following pair
of processes, $\lift{w}{y!(z)}$ and $w[ \lpquote y!(z) \rpquote ]$.

\begin{eqnarray}
	\lift{w}{y!(z)}\widehat{\id{\{}u / z \id{\}}}
		& = &
		\lift{w}{y!(u)} \nonumber\\
	w[ \lpquote y!(z) \rpquote ] \widehat{ \id{\{}u / z \id{\}} }
		& = &
		w[ \lpquote y!(z) \rpquote ] \nonumber
\end{eqnarray}

Because the body of the process between quotes is impervious to
substitution, we get radically different answers. In fact, by
examining the first process in an input context,
e.g. $x?(z).\lift{w}{y!(z)}$, we see that the process under the lift
operator may be shaped by prefixed inputs binding a name inside it. In
this sense, the lift operator will be seen as a way to dynamically
construct processes before reifying them as names.

Finally equipped with these standard features we can present the
dynamics of the calculus.

\subsubsection{Operational semantics} 

Finally, we introduce the computational dynamics. What marks these
algebras as distinct from other more traditionally studied algebraic
structures, e.g. vector spaces or polynomial rings, is the manner in
which dynamics is captured. In traditional structures, dynamics is typically
expressed through morphisms between such structures, as in linear maps
between vector spaces or morphisms between rings. In algebras
associated with the semantics of computation, the dynamics is
expressed as part of the algebraic structure itself, through a
reduction reduction relation typically denoted by $\red$. Below, we
give a recursive presentation of this relation for the calculus used
in the encoding.

$\red \subseteq \pi \times \pi$
$\red : \pi \to \mathcal{P}(\pi)$

\begin{mathpar}
  \inferrule* [lab=Comm] { \textsf{match}( x_{src}, x_{trgt} ) } { x_{trgt}?(y)P \; | \; x_{src}!\langle {Q} \rangle \red P\{\quotep{Q}/y}\} }
  \and \\
  \inferrule* [lab=Par] {{P} \red {P}'} {{{P} | {Q}} \red {{P}' | {Q}}}
  \and
  \inferrule* [lab=Equiv]{{{P} \scong {P}'} \andalso {{P}' \red {Q}'} \andalso {{Q}' \scong {Q}}}{{P} \red {Q}}
\end{mathpar}

\begin{eqnarray*}
  match_{\equiv} (\quotep{P},\quotep{Q}) & := & P \equiv Q \\
  match_{\dagger}(\quotep{P},\quotep{Q}) & := & \forall R. P|Q \red^{*} R => R \red^{*} 0 \\
  match_{K}(\quotep{P},\quotep{Q}) & := & K \mbox{ for some context } K
\end{eqnarray*}

$u?(x)P | u!\langle Q \rangle \red P\{\quotep{Q}/x\}$

%We write $\wred$ for $\red^*$, and $P\red$ if $\exists Q $ such that $ P \red Q$.
We write $P\red$ if $\exists Q $ such that $ P \red Q$ and $P\not\red$, otherwise.

\section{Replication}

As mentioned before, it is known that replication (and hence
recursion) can be implemented in a higher-order process algebra
\cite{SangiorgiWalker}. As our first example of calculation with the
machinery thus far presented we give the construction explicitly in
the {\rhoc}.

\begin{eqnarray}
	D_{x} & := & \prefix{x}{y}{(\binpar{\outputp{x}{y}}{@{y}})} \nonumber\\
	\bangp_{x}{P} & := & \binpar{{x}!\langle{\binpar{D_{x}}{P}}\rangle}{D_{x}} \nonumber
\end{eqnarray}

\begin{eqnarray}
	\bangp_{x}{P} & & \nonumber\\
	=
	& {x}!\langle{(\prefix{x}{y}{(\outputp{x}{y} | @{y})) | P}}\rangle 
	      | \prefix{x}{y}{(\outputp{x}{y} | @{y})} & \nonumber\\
	\red
	& (\outputp{x}{y} | @{y})\substn{\quotep{(\prefix{x}{y}{(@{y} | \outputp{x}{y})) | P}}}{y} & \nonumber\\
	=
	& \outputp{x}{\quotep{(\prefix{x}{y}{(\outputp{x}{y} | @{y})) | P}}}
	  | {(\prefix{x}{y}{(\outputp{x}{y} | @{y})) | P}} & \nonumber\\
	\red
	& \ldots & \nonumber\\
	\red^*
	& P | P | \ldots & \nonumber
\end{eqnarray}

Of course, this encoding, as an implementation, runs away, unfolding
$\bangp{P}$ eagerly. A lazier and more implementable replication
operator, restricted to input-guarded processes, may be obtained as follows.

\begin{eqnarray}
\bangp{\prefix{u}{v}{P}} 
	:= 
	\binpar{\lift{x}{\prefix{u}{v}{(\binpar{D(x)}{P})}}}{D(x)} \nonumber
\end{eqnarray}

\begin{remark}
  Note that the lazier definition still does not deal with summation
  or mixed summation (i.e. sums over input and output). The reader is
  invited to construct definitions of replication that deal with these
  features. 

  Further, the definitions are parameterized in a name, $x$. Can you,
  gentle reader, make a definition that eliminates this parameter and
  guarantees no accidental interaction between the replication
  machinery and the process being replicated -- i.e. no accidental
  sharing of names used by the process to get its work done and the
  name(s) used by the replication to effect copying. This latter
  revision of the definition of replication is crucial to obtaining
  the expected identity $!!P \sim !P$.
\end{remark}

\begin{remark}\label{rem:paradoxical_combinator}
  The reader familiar with the lambda calculus will have noticed the
  similarity between $D$ and the paradoxical combinator.

  [Ed. note: the existence of this seems to suggest we have to be more
  restrictive on the set of processes and names we admit if we are to
  support no-cloning.]
\end{remark}

\subsubsection{Bisimulation}

The computational dynamics gives rise to another kind of equivalence,
the equivalence of computational behavior. As previously mentioned
this is typically captured \emph{via} some form of bisimulation.

% The notion we use in this paper is weak barbed bisimulation
% \cite{milner91polyadicpi}.

The notion we use in this paper is derived from weak barbed
bisimulation \cite{milner91polyadicpi}. 

\begin{definition}
An \emph{observation relation}, $\downarrow_{\mathcal N}$, over a set
of names, $\mathcal N$, is the smallest relation satisfying the rules
below.

\infrule[Out-barb]{y \in {\mathcal N}, \; x \nameeq y}
		  {\outputp{x}{v} \downarrow_{\mathcal N} x}
\infrule[Par-barb]{\mbox{$P\downarrow_{\mathcal N} x$ or $Q\downarrow_{\mathcal N} x$}}
		  {\binpar{P}{Q} \downarrow_{\mathcal N} x}

We write $P \Downarrow_{\mathcal N} x$ if there is $Q$ such that 
$P \wred Q$ and $Q \downarrow_{\mathcal N} x$.
\end{definition}

\begin{definition}
%\label{def.bbisim}
An  ${\mathcal N}$-\emph{barbed bisimulation} over a set of names, ${\mathcal N}$, is a symmetric binary relation 
${\mathcal S}_{\mathcal N}$ between agents such that $P\rel{S}_{\mathcal N}Q$ implies:
\begin{enumerate}
\item If $P \red P'$ then $Q \wred Q'$ and $P'\rel{S}_{\mathcal N} Q'$.
\item If $P\downarrow_{\mathcal N} x$, then $Q\Downarrow_{\mathcal N} x$.
\end{enumerate}
$P$ is ${\mathcal N}$-barbed bisimilar to $Q$, written
$P \wbbisim_{\mathcal N} Q$, if $P \rel{S}_{\mathcal N} Q$ for some ${\mathcal N}$-barbed bisimulation ${\mathcal S}_{\mathcal N}$.
\end{definition}

$\mathcal{R} \subseteq \pi \times \pi$

$P \mathcal{R} Q => \forall P'. P \red P' \Rightarrow \exists Q'. Q \red Q', P' \mathcal{R} Q'$

$P \vdash x \Rightarrow Q \vdash x$

\begin{mathpar}
  \inferrule*[lab=Out-barb]{x \nameeq y}{{y}!\langle{Q}\rangle \vdash x}
  \and
  \inferrule*[lab=Par-barb]{\mbox{$P\vdash x$ or $Q\vdash x$}}{\binpar{P}{Q} \vdash x}
\end{mathpar}

\subsubsection{Contexts}

One of the principle advantages of computational calculi like the
$\pi$-calculus is a well-defined notion of context,
contextual-equivalence and a correlation between
contextual-equivalence and notions of bisimulation. The notion of
context allows the decomposition of a process into (sub-)process and
its syntactic environment, its context. Thus, a context may be
thought of as a process with a ``hole'' (written $\Box$) in it. The
application of a context $M$ to a process $P$, written $M[P]$, is
tantamount to filling the hole in $M$ with $P$. In this paper we do
not need the full weight of this theory, but do make use of the notion
of context in the proof the main theorem. 

\begin{mathpar}
  \inferrule* [lab=summation] {} {{M_{M},M_{N}} \bc \Box \;|\; x.M_{A} \;|\; M_{M}+M_{N}}
  \and
  \inferrule* [lab=agent] {} {{M_{A}} \bc (\vec{x})M_{P} \;| \; \clift{P_0,\ldots,M_{P},\ldots,P_N}}
  \and \\
  \inferrule* [lab=process] {} {{M_{P}} \bc M_{N} \;| \;P|M_{P} }
\end{mathpar} 

\begin{mathpar}
  \inferrule* [lab=sychronization] {} {M_{N} \bc \Box \;|\; x?M_{F} \;|\; x!M_{C}}
  \and
  \inferrule* [lab=abstraction] {} {{M_{F}} \bc (x)M_{P} }
  \and
  \inferrule* [lab=concretion] {} {{M_{C}} \bc \langle M_{P} \rangle }
  \and \\
  \inferrule* [lab=process] {} {{M_{P}} \bc M_{N} \;| \;P|M_{P} }
\end{mathpar}

\begin{definition}[contextual application] Given a context $M$, and
  process $P$, we define the \emph{contextual application}, $M[P] :=
  M\{P/\Box\}$. That is, the contextual application of M to P is the
  substitution of $P$ for $\Box$ in $M$.
\end{definition}

$\meaningof{-} : L \to \mathcal{P}(\pi)$

\begin{mathpar}
  \inferrule* [lab=collection] {} {\meaningof{true} = \pi, \and \meaningof{~E} = \pi \setminus \meaningof{E}, \and \meaningof{E_{1} \& E_{2}} = \meaningof{E_{1}} \cap \meaningof{E_{2}}}
\end{mathpar}

\begin{mathpar}
  \inferrule* [lab=structure] {} {\meaningof{0} = \{ P \in \pi | P \equiv 0 \}, \and \\ \meaningof{E_1 | E_2} = \{ P \in \pi | P \equiv P_{1} | P_{2}, P_{1} \in \meaningof{E_{1}}, P_{2} \in \meaningof{E_2}\} }
\end{mathpar}

\begin{mathpar}
 \inferrule* [lab=behavior] {} {\meaningof{\langle a?b \rangle E} = \{ P \in \pi | P \equiv Q | u?(y)P', \\ \and \\\\ \and \\ \;\;\; u \in \meaningof{a}, \forall z.P'\{z/y\} \in \meaningof{E\{z/b\}}\}, \and \\ \meaningof{a!E} = \{ P \in \pi | P \equiv Q | x!\langle P' \rangle, x \in \meaningof{a} P' \in \meaningof{E}\} }
\end{mathpar}

\begin{mathpar}
 \inferrule* [lab=nominal] {} {\meaningof{\quotep{E}} = \{ \quotep{P} \in \quotep{\pi} | P \in \meaningof{E} \}, \and \meaningof{\quotep{P}} = \{ \quotep{Q} \in \quotep{\pi} | P \equiv Q \} \and \\ \meaningof{@\quotep{E}} = \{ P \in \pi | P \equiv @x, x \in \meaningof{E} \}}
\end{mathpar}

\begin{eqnarray*}
  \\
  \meaningof{-} : TS \to ST
\end{eqnarray*}

\begin{eqnarray*}
  \\
  L : TS \to ST
\end{eqnarray*}

\begin{eqnarray*}
  \\
  P \models E \iff P \in \meaningof{E}
\end{eqnarray*}

\begin{eqnarray*}
  P \approx_{L} Q \iff \forall E \in L. P \models E \iff Q \models E
\end{eqnarray*}

\begin{eqnarray*}
  P \approx_{K} Q
\end{eqnarray*}

\begin{eqnarray*}
  P \approx Q
\end{eqnarray*}

$\approx_{K} = \approx = \approx_{L}$

\subsubsection{Contextual duality}

Note that contexts extend the quotation operation to a family of
operations from processes to names. Given a context, $M$, we can
define a \emph{nominal context}, $\quotep{M}$ by $\quotep{M}[P] :=
\quotep{M[P]}$. To foreshadow what is to come we observe that these
operations enjoy a duality with processes very much like the duality
between vectors and maps from vectors to scalars.

Further, because the calculus is essentially higher-order, we have a
correspondence between contexts and processes. More specifically,
given a name $x$ and a context $M$ we can construct $M^{*}_{x}$ such
that 

\begin{mathpar}
  M^{*}_{x} | \lift{x}{P} \red M[P]
\end{mathpar}

namely,

\begin{mathpar}
  M^{*}_{x} := x?(u).M[\dropn{u}]
\end{mathpar}

The dependence of $M^{*}_{x}$ on a name makes it an abstraction, 

\begin{mathpar}
  M^{*} := (x)x?(u).M[\dropn{u}]
\end{mathpar}

\subsection{Additional notation}

It will sometimes be convenient to denote the process a name
quotes. We already have the notation $x = \quotep{P}$, but it will be
convenient to introduce an alternate notation, $\procn{x}$, when we
want to emphasize the connection to the use of the name. Note that, by
virtue of name equivalence, $\quotep{\procn{x}} \nameeq x$; so, the
notation is consistent with previous definitions.

Further, because names have structure it is possible to effect
substitutions on the basis of that structure. This means we need to
upgrade our notation for substitutions, which we accomplish by
adapting comprehension notation. Thus,

\begin{mathpar}
  P\{ y / x : x \in S \}
\end{mathpar}

is interpreted to mean the process derived from P by replacing (in a
capture-avoiding manner) each occurrence of $x$ in $S$ by $y$. For example,

\begin{mathpar}
  P\{ \quotep{\procn{x}|\procn{x}} / x : x \in \freenames{P} \}
\end{mathpar}

will replace each (occurrence) of a free name $x$ in $P$ by
$\quotep{\procn{x}|\procn{x}}$.

Also, we will avail ourselves of the notation $x^{L}$ and $x^{R}$ to
denote injections of a name into disjoint copies of the name
space. There are numerous ways to accomplish this. One example can be
found in \cite{MeredithR05}. This notation overloads to vectors of
names: $\vec{x}^{\pi} := (x_{i}^{\pi} \; : \; 0 \leq i < |\vec{x}| )$ where $\pi \in \{L,R\}$.

We also use $P^{\Box} := P|\Box$.

In \cite{MeredithR05} an interpretation of the new operator is
given. It turns out that there are several possible interpretations
all enjoying the requisite algebraic properties of the operator (see
\cite{milner91polyadicpi}). We will therefore make liberal use of
$(\nu\; \vec{x})P$.

% subsection the_syntax_and_semantics_of_the_notation_system (end)   

\input{qm2pi.qmops} 

\input{qm2pi.sterngerlach} 

\input{qm2pi.metric} 

% section concurrent_process_calculi (end)

%\input{qm2pi.proofsketch}

% section proof sketch (end)

%\input{qm2pi.slviaknots} 

% section spatial logic via knots (end)

\input{qm2pi.conclusion}

% section conclusion (end)

%\input{qm2pi.dtcodes} 

% section wiring algorithm (end)

\input{qm2pi.ack} 

% section acknowledgments (end)

\newpage


\bibliographystyle{plain}   
\bibliography{../../biblios/main.bib}

\input{qm2pi.rhodetails}

\end{document}

 

% section concurrent_process_calculi (end)

%\documentclass[12pt]{llncs}
%\documentclass{jktr}

\usepackage[pdftex]{hyperref}                   
\usepackage {listings}
\usepackage {mathpartir}
\usepackage{bcprules}
%\usepackage{listings}
                       
\usepackage{graphicx} 
%\usepackage[margins=2.5cm,nohead,nofoot]{geometry}
%\usepackage{geometry}
\usepackage{amsfonts}
\usepackage{amstext}
\usepackage{latexsym}
\usepackage{amssymb}
\usepackage{color}


%\include{myPreamble}
\include{qm2pi.local} 

%\ifpdf
%\usepackage[pdftex]{graphicx}
%\else
%\usepackage{graphicx}
%\fi

 % \ifpdf
%  \usepackage{pdfsync}
%  \if


%\title{Brief Article}
%\author{David F. Snyder}
%\author{L.G. Meredith}

%\address{Dept. of Math., Texas State University--San Marcos, San Marcos, TX 78666}
       
\pagestyle{empty}


\begin{document}

\lstset{language=[Objective]Caml,frame=shadowbox}

\input{qm2pi.front}

% section front matter (end)

\input{qm2pi.intro} 
 
% section introduction (end)

% \input{qm2pi.knotations} 

% section notation (end)

\input{qm2pi.process.calculi} 

% section concurrent_process_calculi_and_spatial_logics_ (end)
    
%\input{qm2pi.knots2pi} 

%\input{qm2pi.trefoil} 

%\input{qm2pi.mainthm} 

% subsection basic_interpretation (end)

%\input{qm2pi.rho.presentation} 
\subsection{The syntax and semantics of the notation system}\label{sub:the_syntax_and_semantics_of_the_notation_system} % (fold)

We now summarize a technical presentation of the calculus that
embodies our theory of dynamics. The typical presentation of such a
calculus follows the style of giving generators and relations on
them. The grammar, below, describing term constructors, freely
generates the set of processes, $\Proc$. This set is then quotiented
by a relation known as structural congruence and it is over this set
that the notion of dynamics is expressed. This presentation is
essentially that of \cite{MeredithR05} with the addition of
polyadicity and summation. For readability we have relegated some of
the technical subtleties to an appendix.

\subsubsection{Process grammar}\label{subsub:process_grammar}

\begin{mathpar}
  \inferrule* [lab=synchronization] {} {{M} \bc \pzero \;|\; x?F \;|\; x!C }
  \and
  \inferrule* [lab=abstraction] {} {{F} \bc (x)P}
  \and
  \inferrule* [lab=concretion] {} {{C} \bc \langle Q \rangle}
  \and
  \inferrule* [lab=process] {} {{P,Q} \bc M \;| \;P|Q \;|\; @{x}}
  \and
  \inferrule* [lab=name] {} {{x} \bc \quotep{P}}
\end{mathpar} 

Note that $\vec{x}$ (resp. $\vec{P}$) denotes a vector of names
(resp. processes) of length $|\vec{x}|$ (resp. $|\vec{P}|$). We adopt
the following useful abbreviations.

\begin{mathpar}
   x?(\vec{y}).P := x.(\vec{y})P \and  x\clift{\vec{P}} := x.\clift{\vec{P}}
   \and x!(y) := \lift{x}{\dropn{y}}
   \and \Pi_{i=0}^{n-1}P_i := P_0 | \ldots | P_{n-1}
\end{mathpar}

\subsubsection{Structural congruence}

\paragraph{Free and bound names and alpha-equivalence.} At the
core of structural equivalence is alpha-equivalence which identifies
process that are the same up to a change of variable. Formally, we
recognize the distinction between free and bound names. The free names
of a process, $\freenames{P}$, may be calculated recursively as
follows:

\begin{mathpar}
\freenames{\pzero} := \emptyset
  \and \\
  \freenames{x?(y).P} := \{ x \} \cup (\freenames{P} \setminus \{ y \})
  \and 
  \freenames{x!\langle P \rangle} := \{ x \} \cup \{ P \} 
  \and \\
  \freenames{P|Q} := \freenames{P} \cup \freenames{Q}
  \and \\
  \freenames{@{x}} := \{ x \}
\end{mathpar}

$\pi$
$\quotep{\pi}$

$\freenames{-} : \pi \to \mathcal{P}(\quotep{\pi})$

\begin{eqnarray*}
  \freenames{\pzero} & := & \emptyset \\
  \freenames{x?(y).P} & := & \{ x \} \cup (\freenames{P} \setminus \{ y \}) \\
  \freenames{x!\langle P \rangle} & := & \{ x \} \cup \{ P \} \\
  \freenames{P|Q} & := & \freenames{P} \cup \freenames{Q} \\
  \freenames{\dropn{x}} & := & \{ x \}
\end{eqnarray*}

The bound names of a process, $\boundnames{P}$, are those names occurring in $P$
that are not free. For example, in $x?(y).0$, the name $x$ is free, while $y$ is bound.

\begin{mathpar}
  \inferrule* [lab=monoidal-laws] {} { P|Q \equiv Q|P \and P|0 \equiv P \and P|(Q|R) \equiv (P|Q)|R }
\end{mathpar}

\begin{mathpar}
  \inferrule* [lab=alpha-equivalence] {} { (x)P \equiv (y)P\{y/x\} \and y \not\in \freenames{P} }
\end{mathpar}

\begin{definition}
Then two processes, $P,Q$, are alpha-equivalent if $P = Q\{\vec{y}/\vec{x}\}$ for
some $\vec{x} \in \boundnames{Q},\vec{y} \in \boundnames{P}$, where $Q\{\vec{y}/\vec{x}\}$
denotes the capture-avoiding substitution of $\vec{y}$ for $\vec{x}$ in $Q$.
\end{definition}

\begin{definition}
  The {\em structural congruence} \cite{SangiorgiWalker} , $\equiv$,
  between processes is the least congruence containing
  alpha-equivalence, satisfying the abelian monoid laws
  (associativity, commutativity and $\pzero$ as identity) for parallel
  composition $|$ and for summation $+$.
\end{definition}

\subsection{Name equivalence}

We take name equivalence, written $\nameeq$, to be the smallest
equivalence relation generated by the following rules.

\begin{mathpar}
\inferrule*[lab=Quote-drop]
{ }
{ \quotep{@{x}} \nameeq x }

\inferrule*[lab=Struct-equiv]
{ P \scong Q }
{ \quotep{P} \nameeq \quotep{Q} }
\end{mathpar}

The astute reader will have noticed that the mutual recursion of names
and processes imposes a mutual recursion on alpha-equivalence and
structural equivalence via name-equivalence. Fortunately, all of this
works out pleasantly and we may calculate in the natural way, free of
concern. The reader interested in the details is referred to the
appendix \ref{appendix:rho_details}.

\subsection{Substitution}

We use $\Proc$ for the set of processes, $\QProc$ for the set of
names, and $\id{\{}\vec{y} / \vec{x} \id{\}}$ to denote partial maps,
$s : \QProc \rightarrow \QProc$. A map, $s$ lifts, uniquely, to a map
on process terms, $\widehat{s} : \Proc \rightarrow \Proc$ by the
following equations.

\begin{mathpar}
  (0) \psubstp{Q}{P} := 0 \\
  (R \juxtap S) \psubstp{Q}{P}
  :=    
  (R)\psubstp{Q}{P} \juxtap (S) \psubstp{Q}{P} \\
  (x?(y).R) \psubstp{Q}{P}    
  :=    
  (x)\substp{Q}{P} (z)\concat( (R \psubstn{z}{y}) \psubstp{Q}{P} ) \\
  (\lift{x}{R}) \psubstp{Q}{P}  
  :=
  \lift{(x)\substp{Q}{P}}{ R \psubstp{Q}{P} } \\
%   (\dropn{x})  \psubstp{Q}{P}       
%   := 
%   \left\{ 
%     \begin{array}{ccc} 
%       \dropn{\quotep{Q}} & & x \nameeq \quotep{P} \\
%       \dropn{x} & & otherwise \\
%     \end{array}
%   \right. 
  (\dropn{x})  \psubstp{Q}{P}       
  := 
  \left\{ 
    \begin{array}{ccc} 
      Q & & x \nameeq \quotep{P} \\
      \dropn{x} & & otherwise \\
    \end{array}
  \right.
\end{mathpar}
 

where

\begin{eqnarray}
  (x)\id{\{} \lpquote Q \rpquote / \lpquote P \rpquote \id{\}}            = 
  \left\{ 
    \begin{array}{ccc}
      \lpquote Q \rpquote & & x \nameeq \lpquote P \rpquote \\
      x & & otherwise \\
    \end{array}
  \right. \nonumber
\end{eqnarray}

and $z$ is chosen distinct from $\quotep{P}$, $\quotep{Q}$, the free
names in $Q$, and all the names in $R$. Our $\alpha$-equivalence will
be built in the standard way from this substitution.

\begin{remark}\label{rem:no_self_referential_names}
  One consequence of these definitions is that $\forall P. \quotep{P}
  \not\in \freenames{P}$.
\end{remark}

\subsection{ Dynamic quote: an example }

Anticipating something of what's to come, consider applying the
substitution, $\widehat{\id{\{}u / z \id{\}}}$, to the following pair
of processes, $\lift{w}{y!(z)}$ and $w[ \lpquote y!(z) \rpquote ]$.

\begin{eqnarray}
	\lift{w}{y!(z)}\widehat{\id{\{}u / z \id{\}}}
		& = &
		\lift{w}{y!(u)} \nonumber\\
	w[ \lpquote y!(z) \rpquote ] \widehat{ \id{\{}u / z \id{\}} }
		& = &
		w[ \lpquote y!(z) \rpquote ] \nonumber
\end{eqnarray}

Because the body of the process between quotes is impervious to
substitution, we get radically different answers. In fact, by
examining the first process in an input context,
e.g. $x?(z).\lift{w}{y!(z)}$, we see that the process under the lift
operator may be shaped by prefixed inputs binding a name inside it. In
this sense, the lift operator will be seen as a way to dynamically
construct processes before reifying them as names.

Finally equipped with these standard features we can present the
dynamics of the calculus.

\subsubsection{Operational semantics} 

Finally, we introduce the computational dynamics. What marks these
algebras as distinct from other more traditionally studied algebraic
structures, e.g. vector spaces or polynomial rings, is the manner in
which dynamics is captured. In traditional structures, dynamics is typically
expressed through morphisms between such structures, as in linear maps
between vector spaces or morphisms between rings. In algebras
associated with the semantics of computation, the dynamics is
expressed as part of the algebraic structure itself, through a
reduction reduction relation typically denoted by $\red$. Below, we
give a recursive presentation of this relation for the calculus used
in the encoding.

$\red \subseteq \pi \times \pi$
$\red : \pi \to \mathcal{P}(\pi)$

\begin{mathpar}
  \inferrule* [lab=Comm] { \textsf{match}( x_{src}, x_{trgt} ) } { x_{trgt}?(y)P \; | \; x_{src}!\langle {Q} \rangle \red P\{\quotep{Q}/y}\} }
  \and \\
  \inferrule* [lab=Par] {{P} \red {P}'} {{{P} | {Q}} \red {{P}' | {Q}}}
  \and
  \inferrule* [lab=Equiv]{{{P} \scong {P}'} \andalso {{P}' \red {Q}'} \andalso {{Q}' \scong {Q}}}{{P} \red {Q}}
\end{mathpar}

\begin{eqnarray*}
  match_{\equiv} (\quotep{P},\quotep{Q}) & := & P \equiv Q \\
  match_{\dagger}(\quotep{P},\quotep{Q}) & := & \forall R. P|Q \red^{*} R => R \red^{*} 0 \\
  match_{K}(\quotep{P},\quotep{Q}) & := & K \mbox{ for some context } K
\end{eqnarray*}

$u?(x)P | u!\langle Q \rangle \red P\{\quotep{Q}/x\}$

%We write $\wred$ for $\red^*$, and $P\red$ if $\exists Q $ such that $ P \red Q$.
We write $P\red$ if $\exists Q $ such that $ P \red Q$ and $P\not\red$, otherwise.

\section{Replication}

As mentioned before, it is known that replication (and hence
recursion) can be implemented in a higher-order process algebra
\cite{SangiorgiWalker}. As our first example of calculation with the
machinery thus far presented we give the construction explicitly in
the {\rhoc}.

\begin{eqnarray}
	D_{x} & := & \prefix{x}{y}{(\binpar{\outputp{x}{y}}{@{y}})} \nonumber\\
	\bangp_{x}{P} & := & \binpar{{x}!\langle{\binpar{D_{x}}{P}}\rangle}{D_{x}} \nonumber
\end{eqnarray}

\begin{eqnarray}
	\bangp_{x}{P} & & \nonumber\\
	=
	& {x}!\langle{(\prefix{x}{y}{(\outputp{x}{y} | @{y})) | P}}\rangle 
	      | \prefix{x}{y}{(\outputp{x}{y} | @{y})} & \nonumber\\
	\red
	& (\outputp{x}{y} | @{y})\substn{\quotep{(\prefix{x}{y}{(@{y} | \outputp{x}{y})) | P}}}{y} & \nonumber\\
	=
	& \outputp{x}{\quotep{(\prefix{x}{y}{(\outputp{x}{y} | @{y})) | P}}}
	  | {(\prefix{x}{y}{(\outputp{x}{y} | @{y})) | P}} & \nonumber\\
	\red
	& \ldots & \nonumber\\
	\red^*
	& P | P | \ldots & \nonumber
\end{eqnarray}

Of course, this encoding, as an implementation, runs away, unfolding
$\bangp{P}$ eagerly. A lazier and more implementable replication
operator, restricted to input-guarded processes, may be obtained as follows.

\begin{eqnarray}
\bangp{\prefix{u}{v}{P}} 
	:= 
	\binpar{\lift{x}{\prefix{u}{v}{(\binpar{D(x)}{P})}}}{D(x)} \nonumber
\end{eqnarray}

\begin{remark}
  Note that the lazier definition still does not deal with summation
  or mixed summation (i.e. sums over input and output). The reader is
  invited to construct definitions of replication that deal with these
  features. 

  Further, the definitions are parameterized in a name, $x$. Can you,
  gentle reader, make a definition that eliminates this parameter and
  guarantees no accidental interaction between the replication
  machinery and the process being replicated -- i.e. no accidental
  sharing of names used by the process to get its work done and the
  name(s) used by the replication to effect copying. This latter
  revision of the definition of replication is crucial to obtaining
  the expected identity $!!P \sim !P$.
\end{remark}

\begin{remark}\label{rem:paradoxical_combinator}
  The reader familiar with the lambda calculus will have noticed the
  similarity between $D$ and the paradoxical combinator.

  [Ed. note: the existence of this seems to suggest we have to be more
  restrictive on the set of processes and names we admit if we are to
  support no-cloning.]
\end{remark}

\subsubsection{Bisimulation}

The computational dynamics gives rise to another kind of equivalence,
the equivalence of computational behavior. As previously mentioned
this is typically captured \emph{via} some form of bisimulation.

% The notion we use in this paper is weak barbed bisimulation
% \cite{milner91polyadicpi}.

The notion we use in this paper is derived from weak barbed
bisimulation \cite{milner91polyadicpi}. 

\begin{definition}
An \emph{observation relation}, $\downarrow_{\mathcal N}$, over a set
of names, $\mathcal N$, is the smallest relation satisfying the rules
below.

\infrule[Out-barb]{y \in {\mathcal N}, \; x \nameeq y}
		  {\outputp{x}{v} \downarrow_{\mathcal N} x}
\infrule[Par-barb]{\mbox{$P\downarrow_{\mathcal N} x$ or $Q\downarrow_{\mathcal N} x$}}
		  {\binpar{P}{Q} \downarrow_{\mathcal N} x}

We write $P \Downarrow_{\mathcal N} x$ if there is $Q$ such that 
$P \wred Q$ and $Q \downarrow_{\mathcal N} x$.
\end{definition}

\begin{definition}
%\label{def.bbisim}
An  ${\mathcal N}$-\emph{barbed bisimulation} over a set of names, ${\mathcal N}$, is a symmetric binary relation 
${\mathcal S}_{\mathcal N}$ between agents such that $P\rel{S}_{\mathcal N}Q$ implies:
\begin{enumerate}
\item If $P \red P'$ then $Q \wred Q'$ and $P'\rel{S}_{\mathcal N} Q'$.
\item If $P\downarrow_{\mathcal N} x$, then $Q\Downarrow_{\mathcal N} x$.
\end{enumerate}
$P$ is ${\mathcal N}$-barbed bisimilar to $Q$, written
$P \wbbisim_{\mathcal N} Q$, if $P \rel{S}_{\mathcal N} Q$ for some ${\mathcal N}$-barbed bisimulation ${\mathcal S}_{\mathcal N}$.
\end{definition}

$\mathcal{R} \subseteq \pi \times \pi$

$P \mathcal{R} Q => \forall P'. P \red P' \Rightarrow \exists Q'. Q \red Q', P' \mathcal{R} Q'$

$P \vdash x \Rightarrow Q \vdash x$

\begin{mathpar}
  \inferrule*[lab=Out-barb]{x \nameeq y}{{y}!\langle{Q}\rangle \vdash x}
  \and
  \inferrule*[lab=Par-barb]{\mbox{$P\vdash x$ or $Q\vdash x$}}{\binpar{P}{Q} \vdash x}
\end{mathpar}

\subsubsection{Contexts}

One of the principle advantages of computational calculi like the
$\pi$-calculus is a well-defined notion of context,
contextual-equivalence and a correlation between
contextual-equivalence and notions of bisimulation. The notion of
context allows the decomposition of a process into (sub-)process and
its syntactic environment, its context. Thus, a context may be
thought of as a process with a ``hole'' (written $\Box$) in it. The
application of a context $M$ to a process $P$, written $M[P]$, is
tantamount to filling the hole in $M$ with $P$. In this paper we do
not need the full weight of this theory, but do make use of the notion
of context in the proof the main theorem. 

\begin{mathpar}
  \inferrule* [lab=summation] {} {{M_{M},M_{N}} \bc \Box \;|\; x.M_{A} \;|\; M_{M}+M_{N}}
  \and
  \inferrule* [lab=agent] {} {{M_{A}} \bc (\vec{x})M_{P} \;| \; \clift{P_0,\ldots,M_{P},\ldots,P_N}}
  \and \\
  \inferrule* [lab=process] {} {{M_{P}} \bc M_{N} \;| \;P|M_{P} }
\end{mathpar} 

\begin{mathpar}
  \inferrule* [lab=sychronization] {} {M_{N} \bc \Box \;|\; x?M_{F} \;|\; x!M_{C}}
  \and
  \inferrule* [lab=abstraction] {} {{M_{F}} \bc (x)M_{P} }
  \and
  \inferrule* [lab=concretion] {} {{M_{C}} \bc \langle M_{P} \rangle }
  \and \\
  \inferrule* [lab=process] {} {{M_{P}} \bc M_{N} \;| \;P|M_{P} }
\end{mathpar}

\begin{definition}[contextual application] Given a context $M$, and
  process $P$, we define the \emph{contextual application}, $M[P] :=
  M\{P/\Box\}$. That is, the contextual application of M to P is the
  substitution of $P$ for $\Box$ in $M$.
\end{definition}

$\meaningof{-} : L \to \mathcal{P}(\pi)$

\begin{mathpar}
  \inferrule* [lab=collection] {} {\meaningof{true} = \pi, \and \meaningof{~E} = \pi \setminus \meaningof{E}, \and \meaningof{E_{1} \& E_{2}} = \meaningof{E_{1}} \cap \meaningof{E_{2}}}
\end{mathpar}

\begin{mathpar}
  \inferrule* [lab=structure] {} {\meaningof{0} = \{ P \in \pi | P \equiv 0 \}, \and \\ \meaningof{E_1 | E_2} = \{ P \in \pi | P \equiv P_{1} | P_{2}, P_{1} \in \meaningof{E_{1}}, P_{2} \in \meaningof{E_2}\} }
\end{mathpar}

\begin{mathpar}
 \inferrule* [lab=behavior] {} {\meaningof{\langle a?b \rangle E} = \{ P \in \pi | P \equiv Q | u?(y)P', \\ \and \\\\ \and \\ \;\;\; u \in \meaningof{a}, \forall z.P'\{z/y\} \in \meaningof{E\{z/b\}}\}, \and \\ \meaningof{a!E} = \{ P \in \pi | P \equiv Q | x!\langle P' \rangle, x \in \meaningof{a} P' \in \meaningof{E}\} }
\end{mathpar}

\begin{mathpar}
 \inferrule* [lab=nominal] {} {\meaningof{\quotep{E}} = \{ \quotep{P} \in \quotep{\pi} | P \in \meaningof{E} \}, \and \meaningof{\quotep{P}} = \{ \quotep{Q} \in \quotep{\pi} | P \equiv Q \} \and \\ \meaningof{@\quotep{E}} = \{ P \in \pi | P \equiv @x, x \in \meaningof{E} \}}
\end{mathpar}

\begin{eqnarray*}
  \\
  \meaningof{-} : TS \to ST
\end{eqnarray*}

\begin{eqnarray*}
  \\
  L : TS \to ST
\end{eqnarray*}

\begin{eqnarray*}
  \\
  P \models E \iff P \in \meaningof{E}
\end{eqnarray*}

\begin{eqnarray*}
  P \approx_{L} Q \iff \forall E \in L. P \models E \iff Q \models E
\end{eqnarray*}

\begin{eqnarray*}
  P \approx_{K} Q
\end{eqnarray*}

\begin{eqnarray*}
  P \approx Q
\end{eqnarray*}

$\approx_{K} = \approx = \approx_{L}$

\subsubsection{Contextual duality}

Note that contexts extend the quotation operation to a family of
operations from processes to names. Given a context, $M$, we can
define a \emph{nominal context}, $\quotep{M}$ by $\quotep{M}[P] :=
\quotep{M[P]}$. To foreshadow what is to come we observe that these
operations enjoy a duality with processes very much like the duality
between vectors and maps from vectors to scalars.

Further, because the calculus is essentially higher-order, we have a
correspondence between contexts and processes. More specifically,
given a name $x$ and a context $M$ we can construct $M^{*}_{x}$ such
that 

\begin{mathpar}
  M^{*}_{x} | \lift{x}{P} \red M[P]
\end{mathpar}

namely,

\begin{mathpar}
  M^{*}_{x} := x?(u).M[\dropn{u}]
\end{mathpar}

The dependence of $M^{*}_{x}$ on a name makes it an abstraction, 

\begin{mathpar}
  M^{*} := (x)x?(u).M[\dropn{u}]
\end{mathpar}

\subsection{Additional notation}

It will sometimes be convenient to denote the process a name
quotes. We already have the notation $x = \quotep{P}$, but it will be
convenient to introduce an alternate notation, $\procn{x}$, when we
want to emphasize the connection to the use of the name. Note that, by
virtue of name equivalence, $\quotep{\procn{x}} \nameeq x$; so, the
notation is consistent with previous definitions.

Further, because names have structure it is possible to effect
substitutions on the basis of that structure. This means we need to
upgrade our notation for substitutions, which we accomplish by
adapting comprehension notation. Thus,

\begin{mathpar}
  P\{ y / x : x \in S \}
\end{mathpar}

is interpreted to mean the process derived from P by replacing (in a
capture-avoiding manner) each occurrence of $x$ in $S$ by $y$. For example,

\begin{mathpar}
  P\{ \quotep{\procn{x}|\procn{x}} / x : x \in \freenames{P} \}
\end{mathpar}

will replace each (occurrence) of a free name $x$ in $P$ by
$\quotep{\procn{x}|\procn{x}}$.

Also, we will avail ourselves of the notation $x^{L}$ and $x^{R}$ to
denote injections of a name into disjoint copies of the name
space. There are numerous ways to accomplish this. One example can be
found in \cite{MeredithR05}. This notation overloads to vectors of
names: $\vec{x}^{\pi} := (x_{i}^{\pi} \; : \; 0 \leq i < |\vec{x}| )$ where $\pi \in \{L,R\}$.

We also use $P^{\Box} := P|\Box$.

In \cite{MeredithR05} an interpretation of the new operator is
given. It turns out that there are several possible interpretations
all enjoying the requisite algebraic properties of the operator (see
\cite{milner91polyadicpi}). We will therefore make liberal use of
$(\nu\; \vec{x})P$.

% subsection the_syntax_and_semantics_of_the_notation_system (end)   

\input{qm2pi.qmops} 

\input{qm2pi.sterngerlach} 

\input{qm2pi.metric} 

% section concurrent_process_calculi (end)

%\input{qm2pi.proofsketch}

% section proof sketch (end)

%\input{qm2pi.slviaknots} 

% section spatial logic via knots (end)

\input{qm2pi.conclusion}

% section conclusion (end)

%\input{qm2pi.dtcodes} 

% section wiring algorithm (end)

\input{qm2pi.ack} 

% section acknowledgments (end)

\newpage


\bibliographystyle{plain}   
\bibliography{../../biblios/main.bib}

\input{qm2pi.rhodetails}

\end{document}



% section proof sketch (end)

%\section{Unlikely characters: spatial logic for
  knots}\label{sub:characteristic_formulae} % (fold)

Associated to the mobile process calculi are a family of logics known
as the Hennessy-Milner logics. These logics typically enjoy a
semantics interpreting formulae as sets of processes that when
factored through the encoding outlined above allows an identification
of classes of knots with logical formulae. In the context of this
encoding the sub-family known as the spatial logics \cite{CairesC03}
\cite{CairesC04} \cite{Caires04} are of particular interest providing
several important features for expressing and reasoning about
properties (i.e. classes) of knots. We hint here at how this may be done.

%\begin{description}
%\item [structural connectives] 
\subsubsection{Structural connectives} The spatial logics enjoy
structural connectives corresponding, at the logical level, to the
parallel composition ($P | Q$) and new name ($(\nu \; x)P$)
connectives for processes. As illustrated in the examples below, these
connectives are extremely expressive given the shape of our encoding.
%\item [decideable satisfaction]

\subsubsection{Decideable satisfaction}
In \cite{Caires04} the satisfaction relation is shown to be decideable
for a rich class of processes. It further turns out that the image of
the our encoding is a proper subset of that class. This result
provides the basis for an algorithm by which to search for knots
enjoying a given property.
%\item [characteristic formulae]

\subsubsection{Characteristic formulae}
In the same paper \cite{Caires04} , Caires presents a means of calculating
characteristic formulae, selecting equivalence classes of processes
up to a pre--specified depth limit on the support set of names. Composed with our
encoding, this characteristic formula can be used to select
characteristic formulae for knots.
%\end{description}

\subsubsection{Spatial logic formulae}

The grammar below (segmented for comprehension) summarizes the syntax
of spatial logic formulae. We employ illustrative examples in the
sequel to provide an intuitive understanding of their meaning
referring the reader to \cite{Caires04} for a more detailed explication
of the semantics.

\begin{mathpar}
  \inferrule* [lab=boolean] {} {{A,B} \bc T \;|\; \neg A \;|\; A \wedge B \;|\; \eta = \eta'}
  \and
  \inferrule* [lab=spatial] {} {|\; \pzero \;|\; A | B \;|\; x \text{\textregistered} A \;|\; \forall x . A \;|\;  H x . A}
  \and
  \inferrule* [lab=behavioral] {} {|\; \alpha . A}
  \and 
  \inferrule* [lab=recursion] {} {|\; X(\vec{u}) \;|\; \mu X(\vec{u}) . A}
  \and
  \inferrule* [lab=action] {} {\alpha \bc \langle x?(\vec{y}) \rangle \;|\; \langle x!(\vec{y}) \rangle \;|\; \langle \tau \rangle}
  \and 
  \inferrule* [lab=name] {} {\eta \bc x \;|\; \tau}
\end{mathpar} 

% subsection characteristic_formulae (end)   	 

\subsection{Example formulae}\label{sub:example_formulae_} % (fold)

\subsubsection{Crossing as formula.}
% 
% \begin{align*}
%   \frac{d}{dx} \sin x &= \cos x 
%   & \frac{d}{dx} e^x &= e^x \\
%   \frac{d}{dx} \cos x &= - \sin x 
%   & \frac{d}{dx} \log x &= \frac{1}{x} \\
% \end{align*} 

\begin{align*}
 \mu C(x_{0},x_{1},y_{0},y_{1},u).&(\langle x_{0}?(z) \rangle(\langle u! \rangle\langle y_{1}!z \rangle C(x_{0},x_{1},y_{0},y_{1},u)) & \\
  & \wedge \langle y_{1}?(z) \rangle (\langle u! \rangle \langle x_{0}!z \rangle C(x_{0},x_{1},y_{0},y_{1},u)) & \\
  & \wedge \langle x_{1}?(z) \rangle (\langle u? \rangle \langle y_{0}!z \rangle C(x_{0},x_{1},y_{0},y_{1},u)) & \\
  & \wedge \langle y_{0}?(z) \rangle (\langle u? \rangle \langle x_{1}!z \rangle C(x_{0},x_{1},y_{0},y_{1},u))) &
\end{align*}

The lexicographical similarity between the shape of this formulae and
the shape of definition of the process representing a crossing reveals
the intuitive meaning of this formulae. It describes the capabilities
of a process that has the right to represent a crossing. For example
it picks out processes that may perform an input on the port $x_0$ in
its initial menu of capabilities. What differentiates the formula
from the process, however, is that the crossing process is the
smallest candidate to satisfy the formula. Infinitely many other
processes -- with internal behavior hidden behind this interface, so
to speak -- also satisfy this formula. Even this simple formula,
then, can be seen to open a new view onto knots, providing a
computational interpretation of \emph{virtual} knots.

Note that this formula is derived by hand. A similar formula can be
derived by employing Caires' calculation of characteristic formula
\cite{Caires04} to the process representing a crossing. In light of
this discussion, we let
$\meaningof{C}_{\phi}(x0,x1,y0,y1,u)$ denote a formula specifying the
dynamics we wish to capture of a crossing. To guarantee we preserve
the shape of the interface and minimal semantics we demand that
$\meaningof{C}_{\phi}(x0,x1,y0,y1,u) \Rightarrow
\textbf{C}(x0,x1,y0,y1,u)$ where $\textbf{C}(x0,x1,y0,y1,u)$ denotes
the formula above.
                            
\subsubsection{Crossing number constraints.}
The moral content of the context lemma (Lemma \ref{context}) is that the notion of
``locality'' in the Reidemeister moves is effectively captured by the
parallel composition operator of the process calculus. This intuition
extends through the logic. Given a formula,
$\meaningof{C}_{\phi}(x0,x1,y0,y1,u)$, we can use the structural
connectives to specify constraints on crossing numbers, such as at
least $n$ crossings, or exactly $n$ crossings.
\begin{mathpar}
  \inferrule* [lab=at-least-n] {} { K^{\geq n}_{\phi}(\vec{xs},\vec{ys}) := \Pi_{i=0}^{n-1} Hu . \meaningof{C}_{\phi}(xs_i,ys_i,u) | T }
  \and 
  \inferrule* [lab=exactly-n] {} { K^{= n}_{\phi}(\vec{xs},\vec{ys}) := \Pi_{i=0}^{n-1} Hu . \meaningof{C}_{\phi}(xs_i,ys_i,u) | \neg (\forall x_0,y_0,x_1,y_1,u . \meaningof{C}_{\phi}(x_0,y_0,x_1,y_1,u) | T) }
\end{mathpar}

To round out this section, recall that the encoding of an $n$-crossing
knot decomposes into a parallel composition of $n$ \emph{copies} of a
crossing process together with a wiring harness. To specify different
knot classes with the same crossing number amounts to specifying
logical constraints on the wiring harness. In the interest of space,
we defer examples to a forthcoming paper. Suffice it to say that both
the conditions ``alternating knot'' and ``contains the tangle
corresponding to 5/3'' are expressible. For example, it is possible to
calculate the characteristic formula of a process corresponding to the
tangle 5/3 and conjoin it into the classifying formula via the
composition connective of the logic.

Finally, we wish to observe that it is entirely within reason to
contemplate a more domain-specific version of spatial logic tailored
to the shape of processes in the image of the encoding. Such a
domain-specific logic would have a better claim to the title formal
language of knot properties.

% subsection example_formulae_ (end)

% section knots_as_processes (end) 

% section spatial logic via knots (end)

\section{Conclusions and future work}

\paragraph{Testing physical space}
You, gentle reader, may wonder why of all the theorems to be proved
given this set up we pick the one above. In some sense it's hardly
central to quantum mechanics. We see it as central in the sense that
it firmly establishes a notion of physical space arising from a notion
of the equivalence of behavior. Relating bisimulation to a metric is a
big step forward, but one is faced with interpreting the relationship
of that metric space to something more physical. Quantum mechanical
notions of ``physical'' space are still far from intuitive, but by
relating this idea of distance as testing to calculations that predict
physical circumstances we are making a not insignificant step forward
toward an understanding of the physical space we inhabit as
essentially dynamic.

\paragraph{Effectivity and simulation}
One of the observations we have yet to make is that the entire program
spelled out here is effective. We have built various interpreters for
the reflective calculus at work in this interpretation. In principle,
then, we can simulate quantum mechanics on a computer. The place where
the simulation may lose fidelity is the infinitely branching summation
for the annihilator.

In this connection i also want to point out that the evaluation style
calculation of the inner product puts the non-determinism of the
summation right at the heart of measurement. This suggests that
Milner's original reduction-based formulation of the dynamics of his
calculi in terms of sums was not just notationally suggestive of a
notion of measure-and-continue but captured some significant part of
the physics.

\paragraph{Quantum continuations}
In light of this last observation i want to point out that the
predominant account of quantum mechanics is missing a key aspect of a
truly compositional story of the physical situation. In a real lab,
when a measurement is made the observation can be made to feed into
another device that then makes another measurement conditioned on the
results of the first. This means that after the superposition was
collapsed the entire experimental set up remained in
superposition. While QM offers a means of writing this down it doesn't
quite line up well with the well-trodden formulation of computation
and continuation that we see so succinctly expressed in Milner's
calculi. This suggests that there might be advantages to this account
of dynamics waiting to be explored.

\paragraph{Quantum logic}
In this connection, we also note that by virtue of having the
Hennessy-Milner construction, we can pull the construction through the
interpretation of QM. This gives us a natural candidate for a quantum
logic that enjoys an extremely tight connection with it's domain of
interpretation, making the construction much less ad hoc (rather it is
the image of functor!).

\paragraph{Quantum probabiity}
i have questions about the basis of the interpretation of inner
product as probability amplitude. In particular, using which
axiomatization of probability theory does the notion of probability
amplitude earn the right to be so dubbed? In other words, where is the
proof that the operation for calculating a probability amplitude (and
then squaring) satisfies the axioms of what it means to calculate a
probability? Even if such a proof exists (i have yet to find it in the
literature), i wonder if it might not be possible to turn things on
their heads. Can we view the calculation of the probability amplitude
as an axiomatization of probability? If so, then the definition we
give for calculating probability amplitude may provide the basis for
an \emph{effective} theory of probability.

\paragraph{Quantum vs ``biological'' information}
Finally, i want to conclude with a more philosophical observation. At
a recent workshop in which QM was a predominant topic i noticed
something about quantum information. The speaker was giving a riveting
discussion of axiomatic QM and showing how properties of ``no
cloning'' and ``no deleting'' emerged as consequences of the
axiomatization. Theorems of this form are necessary to give us a sense
of confidence that our axioms characterize the physical theory. What
struck me, though, was that if quantum information is neither erasable
nor replicable it is markedly different from \emph{life}. Two of the
things we know about life is that

\begin{itemize}
  \item it ends;
  \item to gain some measure of persistence, to transcend it's
    finitude it is imminently copyable.
\end{itemize}

Both of these qualities are summarized succinctly in the aphorism: all
flesh is grass. For me these two kinds of ``information'' -- call them
quantum and biological -- are end points on a spectrum of strategies
for persistence. At one end, we have those curious entities that enjoy
uniqueness and permanence; at the other, we have those who in the face
of a certain end and an uncertain present make a go of passing
something on. To me one of the more remarkable aspects of the latter
strategy is that in the presence of noise (and certain features of
copying) we get a kind of dynamism, a chance for improvement against a
given persistent condition.

% subsection other_calculi_other_bisimulations_and_geometry_as_behavior (end)




% section conclusion (end)

%\documentclass[12pt]{llncs}
%\documentclass{jktr}

\usepackage[pdftex]{hyperref}                   
\usepackage {listings}
\usepackage {mathpartir}
\usepackage{bcprules}
%\usepackage{listings}
                       
\usepackage{graphicx} 
%\usepackage[margins=2.5cm,nohead,nofoot]{geometry}
%\usepackage{geometry}
\usepackage{amsfonts}
\usepackage{amstext}
\usepackage{latexsym}
\usepackage{amssymb}
\usepackage{color}


%\include{myPreamble}
\include{qm2pi.local} 

%\ifpdf
%\usepackage[pdftex]{graphicx}
%\else
%\usepackage{graphicx}
%\fi

 % \ifpdf
%  \usepackage{pdfsync}
%  \if


%\title{Brief Article}
%\author{David F. Snyder}
%\author{L.G. Meredith}

%\address{Dept. of Math., Texas State University--San Marcos, San Marcos, TX 78666}
       
\pagestyle{empty}


\begin{document}

\lstset{language=[Objective]Caml,frame=shadowbox}

\input{qm2pi.front}

% section front matter (end)

\input{qm2pi.intro} 
 
% section introduction (end)

% \input{qm2pi.knotations} 

% section notation (end)

\input{qm2pi.process.calculi} 

% section concurrent_process_calculi_and_spatial_logics_ (end)
    
%\input{qm2pi.knots2pi} 

%\input{qm2pi.trefoil} 

%\input{qm2pi.mainthm} 

% subsection basic_interpretation (end)

%\input{qm2pi.rho.presentation} 
\subsection{The syntax and semantics of the notation system}\label{sub:the_syntax_and_semantics_of_the_notation_system} % (fold)

We now summarize a technical presentation of the calculus that
embodies our theory of dynamics. The typical presentation of such a
calculus follows the style of giving generators and relations on
them. The grammar, below, describing term constructors, freely
generates the set of processes, $\Proc$. This set is then quotiented
by a relation known as structural congruence and it is over this set
that the notion of dynamics is expressed. This presentation is
essentially that of \cite{MeredithR05} with the addition of
polyadicity and summation. For readability we have relegated some of
the technical subtleties to an appendix.

\subsubsection{Process grammar}\label{subsub:process_grammar}

\begin{mathpar}
  \inferrule* [lab=synchronization] {} {{M} \bc \pzero \;|\; x?F \;|\; x!C }
  \and
  \inferrule* [lab=abstraction] {} {{F} \bc (x)P}
  \and
  \inferrule* [lab=concretion] {} {{C} \bc \langle Q \rangle}
  \and
  \inferrule* [lab=process] {} {{P,Q} \bc M \;| \;P|Q \;|\; @{x}}
  \and
  \inferrule* [lab=name] {} {{x} \bc \quotep{P}}
\end{mathpar} 

Note that $\vec{x}$ (resp. $\vec{P}$) denotes a vector of names
(resp. processes) of length $|\vec{x}|$ (resp. $|\vec{P}|$). We adopt
the following useful abbreviations.

\begin{mathpar}
   x?(\vec{y}).P := x.(\vec{y})P \and  x\clift{\vec{P}} := x.\clift{\vec{P}}
   \and x!(y) := \lift{x}{\dropn{y}}
   \and \Pi_{i=0}^{n-1}P_i := P_0 | \ldots | P_{n-1}
\end{mathpar}

\subsubsection{Structural congruence}

\paragraph{Free and bound names and alpha-equivalence.} At the
core of structural equivalence is alpha-equivalence which identifies
process that are the same up to a change of variable. Formally, we
recognize the distinction between free and bound names. The free names
of a process, $\freenames{P}$, may be calculated recursively as
follows:

\begin{mathpar}
\freenames{\pzero} := \emptyset
  \and \\
  \freenames{x?(y).P} := \{ x \} \cup (\freenames{P} \setminus \{ y \})
  \and 
  \freenames{x!\langle P \rangle} := \{ x \} \cup \{ P \} 
  \and \\
  \freenames{P|Q} := \freenames{P} \cup \freenames{Q}
  \and \\
  \freenames{@{x}} := \{ x \}
\end{mathpar}

$\pi$
$\quotep{\pi}$

$\freenames{-} : \pi \to \mathcal{P}(\quotep{\pi})$

\begin{eqnarray*}
  \freenames{\pzero} & := & \emptyset \\
  \freenames{x?(y).P} & := & \{ x \} \cup (\freenames{P} \setminus \{ y \}) \\
  \freenames{x!\langle P \rangle} & := & \{ x \} \cup \{ P \} \\
  \freenames{P|Q} & := & \freenames{P} \cup \freenames{Q} \\
  \freenames{\dropn{x}} & := & \{ x \}
\end{eqnarray*}

The bound names of a process, $\boundnames{P}$, are those names occurring in $P$
that are not free. For example, in $x?(y).0$, the name $x$ is free, while $y$ is bound.

\begin{mathpar}
  \inferrule* [lab=monoidal-laws] {} { P|Q \equiv Q|P \and P|0 \equiv P \and P|(Q|R) \equiv (P|Q)|R }
\end{mathpar}

\begin{mathpar}
  \inferrule* [lab=alpha-equivalence] {} { (x)P \equiv (y)P\{y/x\} \and y \not\in \freenames{P} }
\end{mathpar}

\begin{definition}
Then two processes, $P,Q$, are alpha-equivalent if $P = Q\{\vec{y}/\vec{x}\}$ for
some $\vec{x} \in \boundnames{Q},\vec{y} \in \boundnames{P}$, where $Q\{\vec{y}/\vec{x}\}$
denotes the capture-avoiding substitution of $\vec{y}$ for $\vec{x}$ in $Q$.
\end{definition}

\begin{definition}
  The {\em structural congruence} \cite{SangiorgiWalker} , $\equiv$,
  between processes is the least congruence containing
  alpha-equivalence, satisfying the abelian monoid laws
  (associativity, commutativity and $\pzero$ as identity) for parallel
  composition $|$ and for summation $+$.
\end{definition}

\subsection{Name equivalence}

We take name equivalence, written $\nameeq$, to be the smallest
equivalence relation generated by the following rules.

\begin{mathpar}
\inferrule*[lab=Quote-drop]
{ }
{ \quotep{@{x}} \nameeq x }

\inferrule*[lab=Struct-equiv]
{ P \scong Q }
{ \quotep{P} \nameeq \quotep{Q} }
\end{mathpar}

The astute reader will have noticed that the mutual recursion of names
and processes imposes a mutual recursion on alpha-equivalence and
structural equivalence via name-equivalence. Fortunately, all of this
works out pleasantly and we may calculate in the natural way, free of
concern. The reader interested in the details is referred to the
appendix \ref{appendix:rho_details}.

\subsection{Substitution}

We use $\Proc$ for the set of processes, $\QProc$ for the set of
names, and $\id{\{}\vec{y} / \vec{x} \id{\}}$ to denote partial maps,
$s : \QProc \rightarrow \QProc$. A map, $s$ lifts, uniquely, to a map
on process terms, $\widehat{s} : \Proc \rightarrow \Proc$ by the
following equations.

\begin{mathpar}
  (0) \psubstp{Q}{P} := 0 \\
  (R \juxtap S) \psubstp{Q}{P}
  :=    
  (R)\psubstp{Q}{P} \juxtap (S) \psubstp{Q}{P} \\
  (x?(y).R) \psubstp{Q}{P}    
  :=    
  (x)\substp{Q}{P} (z)\concat( (R \psubstn{z}{y}) \psubstp{Q}{P} ) \\
  (\lift{x}{R}) \psubstp{Q}{P}  
  :=
  \lift{(x)\substp{Q}{P}}{ R \psubstp{Q}{P} } \\
%   (\dropn{x})  \psubstp{Q}{P}       
%   := 
%   \left\{ 
%     \begin{array}{ccc} 
%       \dropn{\quotep{Q}} & & x \nameeq \quotep{P} \\
%       \dropn{x} & & otherwise \\
%     \end{array}
%   \right. 
  (\dropn{x})  \psubstp{Q}{P}       
  := 
  \left\{ 
    \begin{array}{ccc} 
      Q & & x \nameeq \quotep{P} \\
      \dropn{x} & & otherwise \\
    \end{array}
  \right.
\end{mathpar}
 

where

\begin{eqnarray}
  (x)\id{\{} \lpquote Q \rpquote / \lpquote P \rpquote \id{\}}            = 
  \left\{ 
    \begin{array}{ccc}
      \lpquote Q \rpquote & & x \nameeq \lpquote P \rpquote \\
      x & & otherwise \\
    \end{array}
  \right. \nonumber
\end{eqnarray}

and $z$ is chosen distinct from $\quotep{P}$, $\quotep{Q}$, the free
names in $Q$, and all the names in $R$. Our $\alpha$-equivalence will
be built in the standard way from this substitution.

\begin{remark}\label{rem:no_self_referential_names}
  One consequence of these definitions is that $\forall P. \quotep{P}
  \not\in \freenames{P}$.
\end{remark}

\subsection{ Dynamic quote: an example }

Anticipating something of what's to come, consider applying the
substitution, $\widehat{\id{\{}u / z \id{\}}}$, to the following pair
of processes, $\lift{w}{y!(z)}$ and $w[ \lpquote y!(z) \rpquote ]$.

\begin{eqnarray}
	\lift{w}{y!(z)}\widehat{\id{\{}u / z \id{\}}}
		& = &
		\lift{w}{y!(u)} \nonumber\\
	w[ \lpquote y!(z) \rpquote ] \widehat{ \id{\{}u / z \id{\}} }
		& = &
		w[ \lpquote y!(z) \rpquote ] \nonumber
\end{eqnarray}

Because the body of the process between quotes is impervious to
substitution, we get radically different answers. In fact, by
examining the first process in an input context,
e.g. $x?(z).\lift{w}{y!(z)}$, we see that the process under the lift
operator may be shaped by prefixed inputs binding a name inside it. In
this sense, the lift operator will be seen as a way to dynamically
construct processes before reifying them as names.

Finally equipped with these standard features we can present the
dynamics of the calculus.

\subsubsection{Operational semantics} 

Finally, we introduce the computational dynamics. What marks these
algebras as distinct from other more traditionally studied algebraic
structures, e.g. vector spaces or polynomial rings, is the manner in
which dynamics is captured. In traditional structures, dynamics is typically
expressed through morphisms between such structures, as in linear maps
between vector spaces or morphisms between rings. In algebras
associated with the semantics of computation, the dynamics is
expressed as part of the algebraic structure itself, through a
reduction reduction relation typically denoted by $\red$. Below, we
give a recursive presentation of this relation for the calculus used
in the encoding.

$\red \subseteq \pi \times \pi$
$\red : \pi \to \mathcal{P}(\pi)$

\begin{mathpar}
  \inferrule* [lab=Comm] { \textsf{match}( x_{src}, x_{trgt} ) } { x_{trgt}?(y)P \; | \; x_{src}!\langle {Q} \rangle \red P\{\quotep{Q}/y}\} }
  \and \\
  \inferrule* [lab=Par] {{P} \red {P}'} {{{P} | {Q}} \red {{P}' | {Q}}}
  \and
  \inferrule* [lab=Equiv]{{{P} \scong {P}'} \andalso {{P}' \red {Q}'} \andalso {{Q}' \scong {Q}}}{{P} \red {Q}}
\end{mathpar}

\begin{eqnarray*}
  match_{\equiv} (\quotep{P},\quotep{Q}) & := & P \equiv Q \\
  match_{\dagger}(\quotep{P},\quotep{Q}) & := & \forall R. P|Q \red^{*} R => R \red^{*} 0 \\
  match_{K}(\quotep{P},\quotep{Q}) & := & K \mbox{ for some context } K
\end{eqnarray*}

$u?(x)P | u!\langle Q \rangle \red P\{\quotep{Q}/x\}$

%We write $\wred$ for $\red^*$, and $P\red$ if $\exists Q $ such that $ P \red Q$.
We write $P\red$ if $\exists Q $ such that $ P \red Q$ and $P\not\red$, otherwise.

\section{Replication}

As mentioned before, it is known that replication (and hence
recursion) can be implemented in a higher-order process algebra
\cite{SangiorgiWalker}. As our first example of calculation with the
machinery thus far presented we give the construction explicitly in
the {\rhoc}.

\begin{eqnarray}
	D_{x} & := & \prefix{x}{y}{(\binpar{\outputp{x}{y}}{@{y}})} \nonumber\\
	\bangp_{x}{P} & := & \binpar{{x}!\langle{\binpar{D_{x}}{P}}\rangle}{D_{x}} \nonumber
\end{eqnarray}

\begin{eqnarray}
	\bangp_{x}{P} & & \nonumber\\
	=
	& {x}!\langle{(\prefix{x}{y}{(\outputp{x}{y} | @{y})) | P}}\rangle 
	      | \prefix{x}{y}{(\outputp{x}{y} | @{y})} & \nonumber\\
	\red
	& (\outputp{x}{y} | @{y})\substn{\quotep{(\prefix{x}{y}{(@{y} | \outputp{x}{y})) | P}}}{y} & \nonumber\\
	=
	& \outputp{x}{\quotep{(\prefix{x}{y}{(\outputp{x}{y} | @{y})) | P}}}
	  | {(\prefix{x}{y}{(\outputp{x}{y} | @{y})) | P}} & \nonumber\\
	\red
	& \ldots & \nonumber\\
	\red^*
	& P | P | \ldots & \nonumber
\end{eqnarray}

Of course, this encoding, as an implementation, runs away, unfolding
$\bangp{P}$ eagerly. A lazier and more implementable replication
operator, restricted to input-guarded processes, may be obtained as follows.

\begin{eqnarray}
\bangp{\prefix{u}{v}{P}} 
	:= 
	\binpar{\lift{x}{\prefix{u}{v}{(\binpar{D(x)}{P})}}}{D(x)} \nonumber
\end{eqnarray}

\begin{remark}
  Note that the lazier definition still does not deal with summation
  or mixed summation (i.e. sums over input and output). The reader is
  invited to construct definitions of replication that deal with these
  features. 

  Further, the definitions are parameterized in a name, $x$. Can you,
  gentle reader, make a definition that eliminates this parameter and
  guarantees no accidental interaction between the replication
  machinery and the process being replicated -- i.e. no accidental
  sharing of names used by the process to get its work done and the
  name(s) used by the replication to effect copying. This latter
  revision of the definition of replication is crucial to obtaining
  the expected identity $!!P \sim !P$.
\end{remark}

\begin{remark}\label{rem:paradoxical_combinator}
  The reader familiar with the lambda calculus will have noticed the
  similarity between $D$ and the paradoxical combinator.

  [Ed. note: the existence of this seems to suggest we have to be more
  restrictive on the set of processes and names we admit if we are to
  support no-cloning.]
\end{remark}

\subsubsection{Bisimulation}

The computational dynamics gives rise to another kind of equivalence,
the equivalence of computational behavior. As previously mentioned
this is typically captured \emph{via} some form of bisimulation.

% The notion we use in this paper is weak barbed bisimulation
% \cite{milner91polyadicpi}.

The notion we use in this paper is derived from weak barbed
bisimulation \cite{milner91polyadicpi}. 

\begin{definition}
An \emph{observation relation}, $\downarrow_{\mathcal N}$, over a set
of names, $\mathcal N$, is the smallest relation satisfying the rules
below.

\infrule[Out-barb]{y \in {\mathcal N}, \; x \nameeq y}
		  {\outputp{x}{v} \downarrow_{\mathcal N} x}
\infrule[Par-barb]{\mbox{$P\downarrow_{\mathcal N} x$ or $Q\downarrow_{\mathcal N} x$}}
		  {\binpar{P}{Q} \downarrow_{\mathcal N} x}

We write $P \Downarrow_{\mathcal N} x$ if there is $Q$ such that 
$P \wred Q$ and $Q \downarrow_{\mathcal N} x$.
\end{definition}

\begin{definition}
%\label{def.bbisim}
An  ${\mathcal N}$-\emph{barbed bisimulation} over a set of names, ${\mathcal N}$, is a symmetric binary relation 
${\mathcal S}_{\mathcal N}$ between agents such that $P\rel{S}_{\mathcal N}Q$ implies:
\begin{enumerate}
\item If $P \red P'$ then $Q \wred Q'$ and $P'\rel{S}_{\mathcal N} Q'$.
\item If $P\downarrow_{\mathcal N} x$, then $Q\Downarrow_{\mathcal N} x$.
\end{enumerate}
$P$ is ${\mathcal N}$-barbed bisimilar to $Q$, written
$P \wbbisim_{\mathcal N} Q$, if $P \rel{S}_{\mathcal N} Q$ for some ${\mathcal N}$-barbed bisimulation ${\mathcal S}_{\mathcal N}$.
\end{definition}

$\mathcal{R} \subseteq \pi \times \pi$

$P \mathcal{R} Q => \forall P'. P \red P' \Rightarrow \exists Q'. Q \red Q', P' \mathcal{R} Q'$

$P \vdash x \Rightarrow Q \vdash x$

\begin{mathpar}
  \inferrule*[lab=Out-barb]{x \nameeq y}{{y}!\langle{Q}\rangle \vdash x}
  \and
  \inferrule*[lab=Par-barb]{\mbox{$P\vdash x$ or $Q\vdash x$}}{\binpar{P}{Q} \vdash x}
\end{mathpar}

\subsubsection{Contexts}

One of the principle advantages of computational calculi like the
$\pi$-calculus is a well-defined notion of context,
contextual-equivalence and a correlation between
contextual-equivalence and notions of bisimulation. The notion of
context allows the decomposition of a process into (sub-)process and
its syntactic environment, its context. Thus, a context may be
thought of as a process with a ``hole'' (written $\Box$) in it. The
application of a context $M$ to a process $P$, written $M[P]$, is
tantamount to filling the hole in $M$ with $P$. In this paper we do
not need the full weight of this theory, but do make use of the notion
of context in the proof the main theorem. 

\begin{mathpar}
  \inferrule* [lab=summation] {} {{M_{M},M_{N}} \bc \Box \;|\; x.M_{A} \;|\; M_{M}+M_{N}}
  \and
  \inferrule* [lab=agent] {} {{M_{A}} \bc (\vec{x})M_{P} \;| \; \clift{P_0,\ldots,M_{P},\ldots,P_N}}
  \and \\
  \inferrule* [lab=process] {} {{M_{P}} \bc M_{N} \;| \;P|M_{P} }
\end{mathpar} 

\begin{mathpar}
  \inferrule* [lab=sychronization] {} {M_{N} \bc \Box \;|\; x?M_{F} \;|\; x!M_{C}}
  \and
  \inferrule* [lab=abstraction] {} {{M_{F}} \bc (x)M_{P} }
  \and
  \inferrule* [lab=concretion] {} {{M_{C}} \bc \langle M_{P} \rangle }
  \and \\
  \inferrule* [lab=process] {} {{M_{P}} \bc M_{N} \;| \;P|M_{P} }
\end{mathpar}

\begin{definition}[contextual application] Given a context $M$, and
  process $P$, we define the \emph{contextual application}, $M[P] :=
  M\{P/\Box\}$. That is, the contextual application of M to P is the
  substitution of $P$ for $\Box$ in $M$.
\end{definition}

$\meaningof{-} : L \to \mathcal{P}(\pi)$

\begin{mathpar}
  \inferrule* [lab=collection] {} {\meaningof{true} = \pi, \and \meaningof{~E} = \pi \setminus \meaningof{E}, \and \meaningof{E_{1} \& E_{2}} = \meaningof{E_{1}} \cap \meaningof{E_{2}}}
\end{mathpar}

\begin{mathpar}
  \inferrule* [lab=structure] {} {\meaningof{0} = \{ P \in \pi | P \equiv 0 \}, \and \\ \meaningof{E_1 | E_2} = \{ P \in \pi | P \equiv P_{1} | P_{2}, P_{1} \in \meaningof{E_{1}}, P_{2} \in \meaningof{E_2}\} }
\end{mathpar}

\begin{mathpar}
 \inferrule* [lab=behavior] {} {\meaningof{\langle a?b \rangle E} = \{ P \in \pi | P \equiv Q | u?(y)P', \\ \and \\\\ \and \\ \;\;\; u \in \meaningof{a}, \forall z.P'\{z/y\} \in \meaningof{E\{z/b\}}\}, \and \\ \meaningof{a!E} = \{ P \in \pi | P \equiv Q | x!\langle P' \rangle, x \in \meaningof{a} P' \in \meaningof{E}\} }
\end{mathpar}

\begin{mathpar}
 \inferrule* [lab=nominal] {} {\meaningof{\quotep{E}} = \{ \quotep{P} \in \quotep{\pi} | P \in \meaningof{E} \}, \and \meaningof{\quotep{P}} = \{ \quotep{Q} \in \quotep{\pi} | P \equiv Q \} \and \\ \meaningof{@\quotep{E}} = \{ P \in \pi | P \equiv @x, x \in \meaningof{E} \}}
\end{mathpar}

\begin{eqnarray*}
  \\
  \meaningof{-} : TS \to ST
\end{eqnarray*}

\begin{eqnarray*}
  \\
  L : TS \to ST
\end{eqnarray*}

\begin{eqnarray*}
  \\
  P \models E \iff P \in \meaningof{E}
\end{eqnarray*}

\begin{eqnarray*}
  P \approx_{L} Q \iff \forall E \in L. P \models E \iff Q \models E
\end{eqnarray*}

\begin{eqnarray*}
  P \approx_{K} Q
\end{eqnarray*}

\begin{eqnarray*}
  P \approx Q
\end{eqnarray*}

$\approx_{K} = \approx = \approx_{L}$

\subsubsection{Contextual duality}

Note that contexts extend the quotation operation to a family of
operations from processes to names. Given a context, $M$, we can
define a \emph{nominal context}, $\quotep{M}$ by $\quotep{M}[P] :=
\quotep{M[P]}$. To foreshadow what is to come we observe that these
operations enjoy a duality with processes very much like the duality
between vectors and maps from vectors to scalars.

Further, because the calculus is essentially higher-order, we have a
correspondence between contexts and processes. More specifically,
given a name $x$ and a context $M$ we can construct $M^{*}_{x}$ such
that 

\begin{mathpar}
  M^{*}_{x} | \lift{x}{P} \red M[P]
\end{mathpar}

namely,

\begin{mathpar}
  M^{*}_{x} := x?(u).M[\dropn{u}]
\end{mathpar}

The dependence of $M^{*}_{x}$ on a name makes it an abstraction, 

\begin{mathpar}
  M^{*} := (x)x?(u).M[\dropn{u}]
\end{mathpar}

\subsection{Additional notation}

It will sometimes be convenient to denote the process a name
quotes. We already have the notation $x = \quotep{P}$, but it will be
convenient to introduce an alternate notation, $\procn{x}$, when we
want to emphasize the connection to the use of the name. Note that, by
virtue of name equivalence, $\quotep{\procn{x}} \nameeq x$; so, the
notation is consistent with previous definitions.

Further, because names have structure it is possible to effect
substitutions on the basis of that structure. This means we need to
upgrade our notation for substitutions, which we accomplish by
adapting comprehension notation. Thus,

\begin{mathpar}
  P\{ y / x : x \in S \}
\end{mathpar}

is interpreted to mean the process derived from P by replacing (in a
capture-avoiding manner) each occurrence of $x$ in $S$ by $y$. For example,

\begin{mathpar}
  P\{ \quotep{\procn{x}|\procn{x}} / x : x \in \freenames{P} \}
\end{mathpar}

will replace each (occurrence) of a free name $x$ in $P$ by
$\quotep{\procn{x}|\procn{x}}$.

Also, we will avail ourselves of the notation $x^{L}$ and $x^{R}$ to
denote injections of a name into disjoint copies of the name
space. There are numerous ways to accomplish this. One example can be
found in \cite{MeredithR05}. This notation overloads to vectors of
names: $\vec{x}^{\pi} := (x_{i}^{\pi} \; : \; 0 \leq i < |\vec{x}| )$ where $\pi \in \{L,R\}$.

We also use $P^{\Box} := P|\Box$.

In \cite{MeredithR05} an interpretation of the new operator is
given. It turns out that there are several possible interpretations
all enjoying the requisite algebraic properties of the operator (see
\cite{milner91polyadicpi}). We will therefore make liberal use of
$(\nu\; \vec{x})P$.

% subsection the_syntax_and_semantics_of_the_notation_system (end)   

\input{qm2pi.qmops} 

\input{qm2pi.sterngerlach} 

\input{qm2pi.metric} 

% section concurrent_process_calculi (end)

%\input{qm2pi.proofsketch}

% section proof sketch (end)

%\input{qm2pi.slviaknots} 

% section spatial logic via knots (end)

\input{qm2pi.conclusion}

% section conclusion (end)

%\input{qm2pi.dtcodes} 

% section wiring algorithm (end)

\input{qm2pi.ack} 

% section acknowledgments (end)

\newpage


\bibliographystyle{plain}   
\bibliography{../../biblios/main.bib}

\input{qm2pi.rhodetails}

\end{document}

 

% section wiring algorithm (end)

\documentclass[12pt]{llncs}
%\documentclass{jktr}

\usepackage[pdftex]{hyperref}                   
\usepackage {listings}
\usepackage {mathpartir}
\usepackage{bcprules}
%\usepackage{listings}
                       
\usepackage{graphicx} 
%\usepackage[margins=2.5cm,nohead,nofoot]{geometry}
%\usepackage{geometry}
\usepackage{amsfonts}
\usepackage{amstext}
\usepackage{latexsym}
\usepackage{amssymb}
\usepackage{color}


%\include{myPreamble}
\include{qm2pi.local} 

%\ifpdf
%\usepackage[pdftex]{graphicx}
%\else
%\usepackage{graphicx}
%\fi

 % \ifpdf
%  \usepackage{pdfsync}
%  \if


%\title{Brief Article}
%\author{David F. Snyder}
%\author{L.G. Meredith}

%\address{Dept. of Math., Texas State University--San Marcos, San Marcos, TX 78666}
       
\pagestyle{empty}


\begin{document}

\lstset{language=[Objective]Caml,frame=shadowbox}

\input{qm2pi.front}

% section front matter (end)

\input{qm2pi.intro} 
 
% section introduction (end)

% \input{qm2pi.knotations} 

% section notation (end)

\input{qm2pi.process.calculi} 

% section concurrent_process_calculi_and_spatial_logics_ (end)
    
%\input{qm2pi.knots2pi} 

%\input{qm2pi.trefoil} 

%\input{qm2pi.mainthm} 

% subsection basic_interpretation (end)

%\input{qm2pi.rho.presentation} 
\subsection{The syntax and semantics of the notation system}\label{sub:the_syntax_and_semantics_of_the_notation_system} % (fold)

We now summarize a technical presentation of the calculus that
embodies our theory of dynamics. The typical presentation of such a
calculus follows the style of giving generators and relations on
them. The grammar, below, describing term constructors, freely
generates the set of processes, $\Proc$. This set is then quotiented
by a relation known as structural congruence and it is over this set
that the notion of dynamics is expressed. This presentation is
essentially that of \cite{MeredithR05} with the addition of
polyadicity and summation. For readability we have relegated some of
the technical subtleties to an appendix.

\subsubsection{Process grammar}\label{subsub:process_grammar}

\begin{mathpar}
  \inferrule* [lab=synchronization] {} {{M} \bc \pzero \;|\; x?F \;|\; x!C }
  \and
  \inferrule* [lab=abstraction] {} {{F} \bc (x)P}
  \and
  \inferrule* [lab=concretion] {} {{C} \bc \langle Q \rangle}
  \and
  \inferrule* [lab=process] {} {{P,Q} \bc M \;| \;P|Q \;|\; @{x}}
  \and
  \inferrule* [lab=name] {} {{x} \bc \quotep{P}}
\end{mathpar} 

Note that $\vec{x}$ (resp. $\vec{P}$) denotes a vector of names
(resp. processes) of length $|\vec{x}|$ (resp. $|\vec{P}|$). We adopt
the following useful abbreviations.

\begin{mathpar}
   x?(\vec{y}).P := x.(\vec{y})P \and  x\clift{\vec{P}} := x.\clift{\vec{P}}
   \and x!(y) := \lift{x}{\dropn{y}}
   \and \Pi_{i=0}^{n-1}P_i := P_0 | \ldots | P_{n-1}
\end{mathpar}

\subsubsection{Structural congruence}

\paragraph{Free and bound names and alpha-equivalence.} At the
core of structural equivalence is alpha-equivalence which identifies
process that are the same up to a change of variable. Formally, we
recognize the distinction between free and bound names. The free names
of a process, $\freenames{P}$, may be calculated recursively as
follows:

\begin{mathpar}
\freenames{\pzero} := \emptyset
  \and \\
  \freenames{x?(y).P} := \{ x \} \cup (\freenames{P} \setminus \{ y \})
  \and 
  \freenames{x!\langle P \rangle} := \{ x \} \cup \{ P \} 
  \and \\
  \freenames{P|Q} := \freenames{P} \cup \freenames{Q}
  \and \\
  \freenames{@{x}} := \{ x \}
\end{mathpar}

$\pi$
$\quotep{\pi}$

$\freenames{-} : \pi \to \mathcal{P}(\quotep{\pi})$

\begin{eqnarray*}
  \freenames{\pzero} & := & \emptyset \\
  \freenames{x?(y).P} & := & \{ x \} \cup (\freenames{P} \setminus \{ y \}) \\
  \freenames{x!\langle P \rangle} & := & \{ x \} \cup \{ P \} \\
  \freenames{P|Q} & := & \freenames{P} \cup \freenames{Q} \\
  \freenames{\dropn{x}} & := & \{ x \}
\end{eqnarray*}

The bound names of a process, $\boundnames{P}$, are those names occurring in $P$
that are not free. For example, in $x?(y).0$, the name $x$ is free, while $y$ is bound.

\begin{mathpar}
  \inferrule* [lab=monoidal-laws] {} { P|Q \equiv Q|P \and P|0 \equiv P \and P|(Q|R) \equiv (P|Q)|R }
\end{mathpar}

\begin{mathpar}
  \inferrule* [lab=alpha-equivalence] {} { (x)P \equiv (y)P\{y/x\} \and y \not\in \freenames{P} }
\end{mathpar}

\begin{definition}
Then two processes, $P,Q$, are alpha-equivalent if $P = Q\{\vec{y}/\vec{x}\}$ for
some $\vec{x} \in \boundnames{Q},\vec{y} \in \boundnames{P}$, where $Q\{\vec{y}/\vec{x}\}$
denotes the capture-avoiding substitution of $\vec{y}$ for $\vec{x}$ in $Q$.
\end{definition}

\begin{definition}
  The {\em structural congruence} \cite{SangiorgiWalker} , $\equiv$,
  between processes is the least congruence containing
  alpha-equivalence, satisfying the abelian monoid laws
  (associativity, commutativity and $\pzero$ as identity) for parallel
  composition $|$ and for summation $+$.
\end{definition}

\subsection{Name equivalence}

We take name equivalence, written $\nameeq$, to be the smallest
equivalence relation generated by the following rules.

\begin{mathpar}
\inferrule*[lab=Quote-drop]
{ }
{ \quotep{@{x}} \nameeq x }

\inferrule*[lab=Struct-equiv]
{ P \scong Q }
{ \quotep{P} \nameeq \quotep{Q} }
\end{mathpar}

The astute reader will have noticed that the mutual recursion of names
and processes imposes a mutual recursion on alpha-equivalence and
structural equivalence via name-equivalence. Fortunately, all of this
works out pleasantly and we may calculate in the natural way, free of
concern. The reader interested in the details is referred to the
appendix \ref{appendix:rho_details}.

\subsection{Substitution}

We use $\Proc$ for the set of processes, $\QProc$ for the set of
names, and $\id{\{}\vec{y} / \vec{x} \id{\}}$ to denote partial maps,
$s : \QProc \rightarrow \QProc$. A map, $s$ lifts, uniquely, to a map
on process terms, $\widehat{s} : \Proc \rightarrow \Proc$ by the
following equations.

\begin{mathpar}
  (0) \psubstp{Q}{P} := 0 \\
  (R \juxtap S) \psubstp{Q}{P}
  :=    
  (R)\psubstp{Q}{P} \juxtap (S) \psubstp{Q}{P} \\
  (x?(y).R) \psubstp{Q}{P}    
  :=    
  (x)\substp{Q}{P} (z)\concat( (R \psubstn{z}{y}) \psubstp{Q}{P} ) \\
  (\lift{x}{R}) \psubstp{Q}{P}  
  :=
  \lift{(x)\substp{Q}{P}}{ R \psubstp{Q}{P} } \\
%   (\dropn{x})  \psubstp{Q}{P}       
%   := 
%   \left\{ 
%     \begin{array}{ccc} 
%       \dropn{\quotep{Q}} & & x \nameeq \quotep{P} \\
%       \dropn{x} & & otherwise \\
%     \end{array}
%   \right. 
  (\dropn{x})  \psubstp{Q}{P}       
  := 
  \left\{ 
    \begin{array}{ccc} 
      Q & & x \nameeq \quotep{P} \\
      \dropn{x} & & otherwise \\
    \end{array}
  \right.
\end{mathpar}
 

where

\begin{eqnarray}
  (x)\id{\{} \lpquote Q \rpquote / \lpquote P \rpquote \id{\}}            = 
  \left\{ 
    \begin{array}{ccc}
      \lpquote Q \rpquote & & x \nameeq \lpquote P \rpquote \\
      x & & otherwise \\
    \end{array}
  \right. \nonumber
\end{eqnarray}

and $z$ is chosen distinct from $\quotep{P}$, $\quotep{Q}$, the free
names in $Q$, and all the names in $R$. Our $\alpha$-equivalence will
be built in the standard way from this substitution.

\begin{remark}\label{rem:no_self_referential_names}
  One consequence of these definitions is that $\forall P. \quotep{P}
  \not\in \freenames{P}$.
\end{remark}

\subsection{ Dynamic quote: an example }

Anticipating something of what's to come, consider applying the
substitution, $\widehat{\id{\{}u / z \id{\}}}$, to the following pair
of processes, $\lift{w}{y!(z)}$ and $w[ \lpquote y!(z) \rpquote ]$.

\begin{eqnarray}
	\lift{w}{y!(z)}\widehat{\id{\{}u / z \id{\}}}
		& = &
		\lift{w}{y!(u)} \nonumber\\
	w[ \lpquote y!(z) \rpquote ] \widehat{ \id{\{}u / z \id{\}} }
		& = &
		w[ \lpquote y!(z) \rpquote ] \nonumber
\end{eqnarray}

Because the body of the process between quotes is impervious to
substitution, we get radically different answers. In fact, by
examining the first process in an input context,
e.g. $x?(z).\lift{w}{y!(z)}$, we see that the process under the lift
operator may be shaped by prefixed inputs binding a name inside it. In
this sense, the lift operator will be seen as a way to dynamically
construct processes before reifying them as names.

Finally equipped with these standard features we can present the
dynamics of the calculus.

\subsubsection{Operational semantics} 

Finally, we introduce the computational dynamics. What marks these
algebras as distinct from other more traditionally studied algebraic
structures, e.g. vector spaces or polynomial rings, is the manner in
which dynamics is captured. In traditional structures, dynamics is typically
expressed through morphisms between such structures, as in linear maps
between vector spaces or morphisms between rings. In algebras
associated with the semantics of computation, the dynamics is
expressed as part of the algebraic structure itself, through a
reduction reduction relation typically denoted by $\red$. Below, we
give a recursive presentation of this relation for the calculus used
in the encoding.

$\red \subseteq \pi \times \pi$
$\red : \pi \to \mathcal{P}(\pi)$

\begin{mathpar}
  \inferrule* [lab=Comm] { \textsf{match}( x_{src}, x_{trgt} ) } { x_{trgt}?(y)P \; | \; x_{src}!\langle {Q} \rangle \red P\{\quotep{Q}/y}\} }
  \and \\
  \inferrule* [lab=Par] {{P} \red {P}'} {{{P} | {Q}} \red {{P}' | {Q}}}
  \and
  \inferrule* [lab=Equiv]{{{P} \scong {P}'} \andalso {{P}' \red {Q}'} \andalso {{Q}' \scong {Q}}}{{P} \red {Q}}
\end{mathpar}

\begin{eqnarray*}
  match_{\equiv} (\quotep{P},\quotep{Q}) & := & P \equiv Q \\
  match_{\dagger}(\quotep{P},\quotep{Q}) & := & \forall R. P|Q \red^{*} R => R \red^{*} 0 \\
  match_{K}(\quotep{P},\quotep{Q}) & := & K \mbox{ for some context } K
\end{eqnarray*}

$u?(x)P | u!\langle Q \rangle \red P\{\quotep{Q}/x\}$

%We write $\wred$ for $\red^*$, and $P\red$ if $\exists Q $ such that $ P \red Q$.
We write $P\red$ if $\exists Q $ such that $ P \red Q$ and $P\not\red$, otherwise.

\section{Replication}

As mentioned before, it is known that replication (and hence
recursion) can be implemented in a higher-order process algebra
\cite{SangiorgiWalker}. As our first example of calculation with the
machinery thus far presented we give the construction explicitly in
the {\rhoc}.

\begin{eqnarray}
	D_{x} & := & \prefix{x}{y}{(\binpar{\outputp{x}{y}}{@{y}})} \nonumber\\
	\bangp_{x}{P} & := & \binpar{{x}!\langle{\binpar{D_{x}}{P}}\rangle}{D_{x}} \nonumber
\end{eqnarray}

\begin{eqnarray}
	\bangp_{x}{P} & & \nonumber\\
	=
	& {x}!\langle{(\prefix{x}{y}{(\outputp{x}{y} | @{y})) | P}}\rangle 
	      | \prefix{x}{y}{(\outputp{x}{y} | @{y})} & \nonumber\\
	\red
	& (\outputp{x}{y} | @{y})\substn{\quotep{(\prefix{x}{y}{(@{y} | \outputp{x}{y})) | P}}}{y} & \nonumber\\
	=
	& \outputp{x}{\quotep{(\prefix{x}{y}{(\outputp{x}{y} | @{y})) | P}}}
	  | {(\prefix{x}{y}{(\outputp{x}{y} | @{y})) | P}} & \nonumber\\
	\red
	& \ldots & \nonumber\\
	\red^*
	& P | P | \ldots & \nonumber
\end{eqnarray}

Of course, this encoding, as an implementation, runs away, unfolding
$\bangp{P}$ eagerly. A lazier and more implementable replication
operator, restricted to input-guarded processes, may be obtained as follows.

\begin{eqnarray}
\bangp{\prefix{u}{v}{P}} 
	:= 
	\binpar{\lift{x}{\prefix{u}{v}{(\binpar{D(x)}{P})}}}{D(x)} \nonumber
\end{eqnarray}

\begin{remark}
  Note that the lazier definition still does not deal with summation
  or mixed summation (i.e. sums over input and output). The reader is
  invited to construct definitions of replication that deal with these
  features. 

  Further, the definitions are parameterized in a name, $x$. Can you,
  gentle reader, make a definition that eliminates this parameter and
  guarantees no accidental interaction between the replication
  machinery and the process being replicated -- i.e. no accidental
  sharing of names used by the process to get its work done and the
  name(s) used by the replication to effect copying. This latter
  revision of the definition of replication is crucial to obtaining
  the expected identity $!!P \sim !P$.
\end{remark}

\begin{remark}\label{rem:paradoxical_combinator}
  The reader familiar with the lambda calculus will have noticed the
  similarity between $D$ and the paradoxical combinator.

  [Ed. note: the existence of this seems to suggest we have to be more
  restrictive on the set of processes and names we admit if we are to
  support no-cloning.]
\end{remark}

\subsubsection{Bisimulation}

The computational dynamics gives rise to another kind of equivalence,
the equivalence of computational behavior. As previously mentioned
this is typically captured \emph{via} some form of bisimulation.

% The notion we use in this paper is weak barbed bisimulation
% \cite{milner91polyadicpi}.

The notion we use in this paper is derived from weak barbed
bisimulation \cite{milner91polyadicpi}. 

\begin{definition}
An \emph{observation relation}, $\downarrow_{\mathcal N}$, over a set
of names, $\mathcal N$, is the smallest relation satisfying the rules
below.

\infrule[Out-barb]{y \in {\mathcal N}, \; x \nameeq y}
		  {\outputp{x}{v} \downarrow_{\mathcal N} x}
\infrule[Par-barb]{\mbox{$P\downarrow_{\mathcal N} x$ or $Q\downarrow_{\mathcal N} x$}}
		  {\binpar{P}{Q} \downarrow_{\mathcal N} x}

We write $P \Downarrow_{\mathcal N} x$ if there is $Q$ such that 
$P \wred Q$ and $Q \downarrow_{\mathcal N} x$.
\end{definition}

\begin{definition}
%\label{def.bbisim}
An  ${\mathcal N}$-\emph{barbed bisimulation} over a set of names, ${\mathcal N}$, is a symmetric binary relation 
${\mathcal S}_{\mathcal N}$ between agents such that $P\rel{S}_{\mathcal N}Q$ implies:
\begin{enumerate}
\item If $P \red P'$ then $Q \wred Q'$ and $P'\rel{S}_{\mathcal N} Q'$.
\item If $P\downarrow_{\mathcal N} x$, then $Q\Downarrow_{\mathcal N} x$.
\end{enumerate}
$P$ is ${\mathcal N}$-barbed bisimilar to $Q$, written
$P \wbbisim_{\mathcal N} Q$, if $P \rel{S}_{\mathcal N} Q$ for some ${\mathcal N}$-barbed bisimulation ${\mathcal S}_{\mathcal N}$.
\end{definition}

$\mathcal{R} \subseteq \pi \times \pi$

$P \mathcal{R} Q => \forall P'. P \red P' \Rightarrow \exists Q'. Q \red Q', P' \mathcal{R} Q'$

$P \vdash x \Rightarrow Q \vdash x$

\begin{mathpar}
  \inferrule*[lab=Out-barb]{x \nameeq y}{{y}!\langle{Q}\rangle \vdash x}
  \and
  \inferrule*[lab=Par-barb]{\mbox{$P\vdash x$ or $Q\vdash x$}}{\binpar{P}{Q} \vdash x}
\end{mathpar}

\subsubsection{Contexts}

One of the principle advantages of computational calculi like the
$\pi$-calculus is a well-defined notion of context,
contextual-equivalence and a correlation between
contextual-equivalence and notions of bisimulation. The notion of
context allows the decomposition of a process into (sub-)process and
its syntactic environment, its context. Thus, a context may be
thought of as a process with a ``hole'' (written $\Box$) in it. The
application of a context $M$ to a process $P$, written $M[P]$, is
tantamount to filling the hole in $M$ with $P$. In this paper we do
not need the full weight of this theory, but do make use of the notion
of context in the proof the main theorem. 

\begin{mathpar}
  \inferrule* [lab=summation] {} {{M_{M},M_{N}} \bc \Box \;|\; x.M_{A} \;|\; M_{M}+M_{N}}
  \and
  \inferrule* [lab=agent] {} {{M_{A}} \bc (\vec{x})M_{P} \;| \; \clift{P_0,\ldots,M_{P},\ldots,P_N}}
  \and \\
  \inferrule* [lab=process] {} {{M_{P}} \bc M_{N} \;| \;P|M_{P} }
\end{mathpar} 

\begin{mathpar}
  \inferrule* [lab=sychronization] {} {M_{N} \bc \Box \;|\; x?M_{F} \;|\; x!M_{C}}
  \and
  \inferrule* [lab=abstraction] {} {{M_{F}} \bc (x)M_{P} }
  \and
  \inferrule* [lab=concretion] {} {{M_{C}} \bc \langle M_{P} \rangle }
  \and \\
  \inferrule* [lab=process] {} {{M_{P}} \bc M_{N} \;| \;P|M_{P} }
\end{mathpar}

\begin{definition}[contextual application] Given a context $M$, and
  process $P$, we define the \emph{contextual application}, $M[P] :=
  M\{P/\Box\}$. That is, the contextual application of M to P is the
  substitution of $P$ for $\Box$ in $M$.
\end{definition}

$\meaningof{-} : L \to \mathcal{P}(\pi)$

\begin{mathpar}
  \inferrule* [lab=collection] {} {\meaningof{true} = \pi, \and \meaningof{~E} = \pi \setminus \meaningof{E}, \and \meaningof{E_{1} \& E_{2}} = \meaningof{E_{1}} \cap \meaningof{E_{2}}}
\end{mathpar}

\begin{mathpar}
  \inferrule* [lab=structure] {} {\meaningof{0} = \{ P \in \pi | P \equiv 0 \}, \and \\ \meaningof{E_1 | E_2} = \{ P \in \pi | P \equiv P_{1} | P_{2}, P_{1} \in \meaningof{E_{1}}, P_{2} \in \meaningof{E_2}\} }
\end{mathpar}

\begin{mathpar}
 \inferrule* [lab=behavior] {} {\meaningof{\langle a?b \rangle E} = \{ P \in \pi | P \equiv Q | u?(y)P', \\ \and \\\\ \and \\ \;\;\; u \in \meaningof{a}, \forall z.P'\{z/y\} \in \meaningof{E\{z/b\}}\}, \and \\ \meaningof{a!E} = \{ P \in \pi | P \equiv Q | x!\langle P' \rangle, x \in \meaningof{a} P' \in \meaningof{E}\} }
\end{mathpar}

\begin{mathpar}
 \inferrule* [lab=nominal] {} {\meaningof{\quotep{E}} = \{ \quotep{P} \in \quotep{\pi} | P \in \meaningof{E} \}, \and \meaningof{\quotep{P}} = \{ \quotep{Q} \in \quotep{\pi} | P \equiv Q \} \and \\ \meaningof{@\quotep{E}} = \{ P \in \pi | P \equiv @x, x \in \meaningof{E} \}}
\end{mathpar}

\begin{eqnarray*}
  \\
  \meaningof{-} : TS \to ST
\end{eqnarray*}

\begin{eqnarray*}
  \\
  L : TS \to ST
\end{eqnarray*}

\begin{eqnarray*}
  \\
  P \models E \iff P \in \meaningof{E}
\end{eqnarray*}

\begin{eqnarray*}
  P \approx_{L} Q \iff \forall E \in L. P \models E \iff Q \models E
\end{eqnarray*}

\begin{eqnarray*}
  P \approx_{K} Q
\end{eqnarray*}

\begin{eqnarray*}
  P \approx Q
\end{eqnarray*}

$\approx_{K} = \approx = \approx_{L}$

\subsubsection{Contextual duality}

Note that contexts extend the quotation operation to a family of
operations from processes to names. Given a context, $M$, we can
define a \emph{nominal context}, $\quotep{M}$ by $\quotep{M}[P] :=
\quotep{M[P]}$. To foreshadow what is to come we observe that these
operations enjoy a duality with processes very much like the duality
between vectors and maps from vectors to scalars.

Further, because the calculus is essentially higher-order, we have a
correspondence between contexts and processes. More specifically,
given a name $x$ and a context $M$ we can construct $M^{*}_{x}$ such
that 

\begin{mathpar}
  M^{*}_{x} | \lift{x}{P} \red M[P]
\end{mathpar}

namely,

\begin{mathpar}
  M^{*}_{x} := x?(u).M[\dropn{u}]
\end{mathpar}

The dependence of $M^{*}_{x}$ on a name makes it an abstraction, 

\begin{mathpar}
  M^{*} := (x)x?(u).M[\dropn{u}]
\end{mathpar}

\subsection{Additional notation}

It will sometimes be convenient to denote the process a name
quotes. We already have the notation $x = \quotep{P}$, but it will be
convenient to introduce an alternate notation, $\procn{x}$, when we
want to emphasize the connection to the use of the name. Note that, by
virtue of name equivalence, $\quotep{\procn{x}} \nameeq x$; so, the
notation is consistent with previous definitions.

Further, because names have structure it is possible to effect
substitutions on the basis of that structure. This means we need to
upgrade our notation for substitutions, which we accomplish by
adapting comprehension notation. Thus,

\begin{mathpar}
  P\{ y / x : x \in S \}
\end{mathpar}

is interpreted to mean the process derived from P by replacing (in a
capture-avoiding manner) each occurrence of $x$ in $S$ by $y$. For example,

\begin{mathpar}
  P\{ \quotep{\procn{x}|\procn{x}} / x : x \in \freenames{P} \}
\end{mathpar}

will replace each (occurrence) of a free name $x$ in $P$ by
$\quotep{\procn{x}|\procn{x}}$.

Also, we will avail ourselves of the notation $x^{L}$ and $x^{R}$ to
denote injections of a name into disjoint copies of the name
space. There are numerous ways to accomplish this. One example can be
found in \cite{MeredithR05}. This notation overloads to vectors of
names: $\vec{x}^{\pi} := (x_{i}^{\pi} \; : \; 0 \leq i < |\vec{x}| )$ where $\pi \in \{L,R\}$.

We also use $P^{\Box} := P|\Box$.

In \cite{MeredithR05} an interpretation of the new operator is
given. It turns out that there are several possible interpretations
all enjoying the requisite algebraic properties of the operator (see
\cite{milner91polyadicpi}). We will therefore make liberal use of
$(\nu\; \vec{x})P$.

% subsection the_syntax_and_semantics_of_the_notation_system (end)   

\input{qm2pi.qmops} 

\input{qm2pi.sterngerlach} 

\input{qm2pi.metric} 

% section concurrent_process_calculi (end)

%\input{qm2pi.proofsketch}

% section proof sketch (end)

%\input{qm2pi.slviaknots} 

% section spatial logic via knots (end)

\input{qm2pi.conclusion}

% section conclusion (end)

%\input{qm2pi.dtcodes} 

% section wiring algorithm (end)

\input{qm2pi.ack} 

% section acknowledgments (end)

\newpage


\bibliographystyle{plain}   
\bibliography{../../biblios/main.bib}

\input{qm2pi.rhodetails}

\end{document}

 

% section acknowledgments (end)

\newpage


\bibliographystyle{plain}   
\bibliography{../../biblios/main.bib}

\documentclass[12pt]{llncs}
%\documentclass{jktr}

\usepackage[pdftex]{hyperref}                   
\usepackage {listings}
\usepackage {mathpartir}
\usepackage{bcprules}
%\usepackage{listings}
                       
\usepackage{graphicx} 
%\usepackage[margins=2.5cm,nohead,nofoot]{geometry}
%\usepackage{geometry}
\usepackage{amsfonts}
\usepackage{amstext}
\usepackage{latexsym}
\usepackage{amssymb}
\usepackage{color}


%\include{myPreamble}
\include{qm2pi.local} 

%\ifpdf
%\usepackage[pdftex]{graphicx}
%\else
%\usepackage{graphicx}
%\fi

 % \ifpdf
%  \usepackage{pdfsync}
%  \if


%\title{Brief Article}
%\author{David F. Snyder}
%\author{L.G. Meredith}

%\address{Dept. of Math., Texas State University--San Marcos, San Marcos, TX 78666}
       
\pagestyle{empty}


\begin{document}

\lstset{language=[Objective]Caml,frame=shadowbox}

\input{qm2pi.front}

% section front matter (end)

\input{qm2pi.intro} 
 
% section introduction (end)

% \input{qm2pi.knotations} 

% section notation (end)

\input{qm2pi.process.calculi} 

% section concurrent_process_calculi_and_spatial_logics_ (end)
    
%\input{qm2pi.knots2pi} 

%\input{qm2pi.trefoil} 

%\input{qm2pi.mainthm} 

% subsection basic_interpretation (end)

%\input{qm2pi.rho.presentation} 
\subsection{The syntax and semantics of the notation system}\label{sub:the_syntax_and_semantics_of_the_notation_system} % (fold)

We now summarize a technical presentation of the calculus that
embodies our theory of dynamics. The typical presentation of such a
calculus follows the style of giving generators and relations on
them. The grammar, below, describing term constructors, freely
generates the set of processes, $\Proc$. This set is then quotiented
by a relation known as structural congruence and it is over this set
that the notion of dynamics is expressed. This presentation is
essentially that of \cite{MeredithR05} with the addition of
polyadicity and summation. For readability we have relegated some of
the technical subtleties to an appendix.

\subsubsection{Process grammar}\label{subsub:process_grammar}

\begin{mathpar}
  \inferrule* [lab=synchronization] {} {{M} \bc \pzero \;|\; x?F \;|\; x!C }
  \and
  \inferrule* [lab=abstraction] {} {{F} \bc (x)P}
  \and
  \inferrule* [lab=concretion] {} {{C} \bc \langle Q \rangle}
  \and
  \inferrule* [lab=process] {} {{P,Q} \bc M \;| \;P|Q \;|\; @{x}}
  \and
  \inferrule* [lab=name] {} {{x} \bc \quotep{P}}
\end{mathpar} 

Note that $\vec{x}$ (resp. $\vec{P}$) denotes a vector of names
(resp. processes) of length $|\vec{x}|$ (resp. $|\vec{P}|$). We adopt
the following useful abbreviations.

\begin{mathpar}
   x?(\vec{y}).P := x.(\vec{y})P \and  x\clift{\vec{P}} := x.\clift{\vec{P}}
   \and x!(y) := \lift{x}{\dropn{y}}
   \and \Pi_{i=0}^{n-1}P_i := P_0 | \ldots | P_{n-1}
\end{mathpar}

\subsubsection{Structural congruence}

\paragraph{Free and bound names and alpha-equivalence.} At the
core of structural equivalence is alpha-equivalence which identifies
process that are the same up to a change of variable. Formally, we
recognize the distinction between free and bound names. The free names
of a process, $\freenames{P}$, may be calculated recursively as
follows:

\begin{mathpar}
\freenames{\pzero} := \emptyset
  \and \\
  \freenames{x?(y).P} := \{ x \} \cup (\freenames{P} \setminus \{ y \})
  \and 
  \freenames{x!\langle P \rangle} := \{ x \} \cup \{ P \} 
  \and \\
  \freenames{P|Q} := \freenames{P} \cup \freenames{Q}
  \and \\
  \freenames{@{x}} := \{ x \}
\end{mathpar}

$\pi$
$\quotep{\pi}$

$\freenames{-} : \pi \to \mathcal{P}(\quotep{\pi})$

\begin{eqnarray*}
  \freenames{\pzero} & := & \emptyset \\
  \freenames{x?(y).P} & := & \{ x \} \cup (\freenames{P} \setminus \{ y \}) \\
  \freenames{x!\langle P \rangle} & := & \{ x \} \cup \{ P \} \\
  \freenames{P|Q} & := & \freenames{P} \cup \freenames{Q} \\
  \freenames{\dropn{x}} & := & \{ x \}
\end{eqnarray*}

The bound names of a process, $\boundnames{P}$, are those names occurring in $P$
that are not free. For example, in $x?(y).0$, the name $x$ is free, while $y$ is bound.

\begin{mathpar}
  \inferrule* [lab=monoidal-laws] {} { P|Q \equiv Q|P \and P|0 \equiv P \and P|(Q|R) \equiv (P|Q)|R }
\end{mathpar}

\begin{mathpar}
  \inferrule* [lab=alpha-equivalence] {} { (x)P \equiv (y)P\{y/x\} \and y \not\in \freenames{P} }
\end{mathpar}

\begin{definition}
Then two processes, $P,Q$, are alpha-equivalent if $P = Q\{\vec{y}/\vec{x}\}$ for
some $\vec{x} \in \boundnames{Q},\vec{y} \in \boundnames{P}$, where $Q\{\vec{y}/\vec{x}\}$
denotes the capture-avoiding substitution of $\vec{y}$ for $\vec{x}$ in $Q$.
\end{definition}

\begin{definition}
  The {\em structural congruence} \cite{SangiorgiWalker} , $\equiv$,
  between processes is the least congruence containing
  alpha-equivalence, satisfying the abelian monoid laws
  (associativity, commutativity and $\pzero$ as identity) for parallel
  composition $|$ and for summation $+$.
\end{definition}

\subsection{Name equivalence}

We take name equivalence, written $\nameeq$, to be the smallest
equivalence relation generated by the following rules.

\begin{mathpar}
\inferrule*[lab=Quote-drop]
{ }
{ \quotep{@{x}} \nameeq x }

\inferrule*[lab=Struct-equiv]
{ P \scong Q }
{ \quotep{P} \nameeq \quotep{Q} }
\end{mathpar}

The astute reader will have noticed that the mutual recursion of names
and processes imposes a mutual recursion on alpha-equivalence and
structural equivalence via name-equivalence. Fortunately, all of this
works out pleasantly and we may calculate in the natural way, free of
concern. The reader interested in the details is referred to the
appendix \ref{appendix:rho_details}.

\subsection{Substitution}

We use $\Proc$ for the set of processes, $\QProc$ for the set of
names, and $\id{\{}\vec{y} / \vec{x} \id{\}}$ to denote partial maps,
$s : \QProc \rightarrow \QProc$. A map, $s$ lifts, uniquely, to a map
on process terms, $\widehat{s} : \Proc \rightarrow \Proc$ by the
following equations.

\begin{mathpar}
  (0) \psubstp{Q}{P} := 0 \\
  (R \juxtap S) \psubstp{Q}{P}
  :=    
  (R)\psubstp{Q}{P} \juxtap (S) \psubstp{Q}{P} \\
  (x?(y).R) \psubstp{Q}{P}    
  :=    
  (x)\substp{Q}{P} (z)\concat( (R \psubstn{z}{y}) \psubstp{Q}{P} ) \\
  (\lift{x}{R}) \psubstp{Q}{P}  
  :=
  \lift{(x)\substp{Q}{P}}{ R \psubstp{Q}{P} } \\
%   (\dropn{x})  \psubstp{Q}{P}       
%   := 
%   \left\{ 
%     \begin{array}{ccc} 
%       \dropn{\quotep{Q}} & & x \nameeq \quotep{P} \\
%       \dropn{x} & & otherwise \\
%     \end{array}
%   \right. 
  (\dropn{x})  \psubstp{Q}{P}       
  := 
  \left\{ 
    \begin{array}{ccc} 
      Q & & x \nameeq \quotep{P} \\
      \dropn{x} & & otherwise \\
    \end{array}
  \right.
\end{mathpar}
 

where

\begin{eqnarray}
  (x)\id{\{} \lpquote Q \rpquote / \lpquote P \rpquote \id{\}}            = 
  \left\{ 
    \begin{array}{ccc}
      \lpquote Q \rpquote & & x \nameeq \lpquote P \rpquote \\
      x & & otherwise \\
    \end{array}
  \right. \nonumber
\end{eqnarray}

and $z$ is chosen distinct from $\quotep{P}$, $\quotep{Q}$, the free
names in $Q$, and all the names in $R$. Our $\alpha$-equivalence will
be built in the standard way from this substitution.

\begin{remark}\label{rem:no_self_referential_names}
  One consequence of these definitions is that $\forall P. \quotep{P}
  \not\in \freenames{P}$.
\end{remark}

\subsection{ Dynamic quote: an example }

Anticipating something of what's to come, consider applying the
substitution, $\widehat{\id{\{}u / z \id{\}}}$, to the following pair
of processes, $\lift{w}{y!(z)}$ and $w[ \lpquote y!(z) \rpquote ]$.

\begin{eqnarray}
	\lift{w}{y!(z)}\widehat{\id{\{}u / z \id{\}}}
		& = &
		\lift{w}{y!(u)} \nonumber\\
	w[ \lpquote y!(z) \rpquote ] \widehat{ \id{\{}u / z \id{\}} }
		& = &
		w[ \lpquote y!(z) \rpquote ] \nonumber
\end{eqnarray}

Because the body of the process between quotes is impervious to
substitution, we get radically different answers. In fact, by
examining the first process in an input context,
e.g. $x?(z).\lift{w}{y!(z)}$, we see that the process under the lift
operator may be shaped by prefixed inputs binding a name inside it. In
this sense, the lift operator will be seen as a way to dynamically
construct processes before reifying them as names.

Finally equipped with these standard features we can present the
dynamics of the calculus.

\subsubsection{Operational semantics} 

Finally, we introduce the computational dynamics. What marks these
algebras as distinct from other more traditionally studied algebraic
structures, e.g. vector spaces or polynomial rings, is the manner in
which dynamics is captured. In traditional structures, dynamics is typically
expressed through morphisms between such structures, as in linear maps
between vector spaces or morphisms between rings. In algebras
associated with the semantics of computation, the dynamics is
expressed as part of the algebraic structure itself, through a
reduction reduction relation typically denoted by $\red$. Below, we
give a recursive presentation of this relation for the calculus used
in the encoding.

$\red \subseteq \pi \times \pi$
$\red : \pi \to \mathcal{P}(\pi)$

\begin{mathpar}
  \inferrule* [lab=Comm] { \textsf{match}( x_{src}, x_{trgt} ) } { x_{trgt}?(y)P \; | \; x_{src}!\langle {Q} \rangle \red P\{\quotep{Q}/y}\} }
  \and \\
  \inferrule* [lab=Par] {{P} \red {P}'} {{{P} | {Q}} \red {{P}' | {Q}}}
  \and
  \inferrule* [lab=Equiv]{{{P} \scong {P}'} \andalso {{P}' \red {Q}'} \andalso {{Q}' \scong {Q}}}{{P} \red {Q}}
\end{mathpar}

\begin{eqnarray*}
  match_{\equiv} (\quotep{P},\quotep{Q}) & := & P \equiv Q \\
  match_{\dagger}(\quotep{P},\quotep{Q}) & := & \forall R. P|Q \red^{*} R => R \red^{*} 0 \\
  match_{K}(\quotep{P},\quotep{Q}) & := & K \mbox{ for some context } K
\end{eqnarray*}

$u?(x)P | u!\langle Q \rangle \red P\{\quotep{Q}/x\}$

%We write $\wred$ for $\red^*$, and $P\red$ if $\exists Q $ such that $ P \red Q$.
We write $P\red$ if $\exists Q $ such that $ P \red Q$ and $P\not\red$, otherwise.

\section{Replication}

As mentioned before, it is known that replication (and hence
recursion) can be implemented in a higher-order process algebra
\cite{SangiorgiWalker}. As our first example of calculation with the
machinery thus far presented we give the construction explicitly in
the {\rhoc}.

\begin{eqnarray}
	D_{x} & := & \prefix{x}{y}{(\binpar{\outputp{x}{y}}{@{y}})} \nonumber\\
	\bangp_{x}{P} & := & \binpar{{x}!\langle{\binpar{D_{x}}{P}}\rangle}{D_{x}} \nonumber
\end{eqnarray}

\begin{eqnarray}
	\bangp_{x}{P} & & \nonumber\\
	=
	& {x}!\langle{(\prefix{x}{y}{(\outputp{x}{y} | @{y})) | P}}\rangle 
	      | \prefix{x}{y}{(\outputp{x}{y} | @{y})} & \nonumber\\
	\red
	& (\outputp{x}{y} | @{y})\substn{\quotep{(\prefix{x}{y}{(@{y} | \outputp{x}{y})) | P}}}{y} & \nonumber\\
	=
	& \outputp{x}{\quotep{(\prefix{x}{y}{(\outputp{x}{y} | @{y})) | P}}}
	  | {(\prefix{x}{y}{(\outputp{x}{y} | @{y})) | P}} & \nonumber\\
	\red
	& \ldots & \nonumber\\
	\red^*
	& P | P | \ldots & \nonumber
\end{eqnarray}

Of course, this encoding, as an implementation, runs away, unfolding
$\bangp{P}$ eagerly. A lazier and more implementable replication
operator, restricted to input-guarded processes, may be obtained as follows.

\begin{eqnarray}
\bangp{\prefix{u}{v}{P}} 
	:= 
	\binpar{\lift{x}{\prefix{u}{v}{(\binpar{D(x)}{P})}}}{D(x)} \nonumber
\end{eqnarray}

\begin{remark}
  Note that the lazier definition still does not deal with summation
  or mixed summation (i.e. sums over input and output). The reader is
  invited to construct definitions of replication that deal with these
  features. 

  Further, the definitions are parameterized in a name, $x$. Can you,
  gentle reader, make a definition that eliminates this parameter and
  guarantees no accidental interaction between the replication
  machinery and the process being replicated -- i.e. no accidental
  sharing of names used by the process to get its work done and the
  name(s) used by the replication to effect copying. This latter
  revision of the definition of replication is crucial to obtaining
  the expected identity $!!P \sim !P$.
\end{remark}

\begin{remark}\label{rem:paradoxical_combinator}
  The reader familiar with the lambda calculus will have noticed the
  similarity between $D$ and the paradoxical combinator.

  [Ed. note: the existence of this seems to suggest we have to be more
  restrictive on the set of processes and names we admit if we are to
  support no-cloning.]
\end{remark}

\subsubsection{Bisimulation}

The computational dynamics gives rise to another kind of equivalence,
the equivalence of computational behavior. As previously mentioned
this is typically captured \emph{via} some form of bisimulation.

% The notion we use in this paper is weak barbed bisimulation
% \cite{milner91polyadicpi}.

The notion we use in this paper is derived from weak barbed
bisimulation \cite{milner91polyadicpi}. 

\begin{definition}
An \emph{observation relation}, $\downarrow_{\mathcal N}$, over a set
of names, $\mathcal N$, is the smallest relation satisfying the rules
below.

\infrule[Out-barb]{y \in {\mathcal N}, \; x \nameeq y}
		  {\outputp{x}{v} \downarrow_{\mathcal N} x}
\infrule[Par-barb]{\mbox{$P\downarrow_{\mathcal N} x$ or $Q\downarrow_{\mathcal N} x$}}
		  {\binpar{P}{Q} \downarrow_{\mathcal N} x}

We write $P \Downarrow_{\mathcal N} x$ if there is $Q$ such that 
$P \wred Q$ and $Q \downarrow_{\mathcal N} x$.
\end{definition}

\begin{definition}
%\label{def.bbisim}
An  ${\mathcal N}$-\emph{barbed bisimulation} over a set of names, ${\mathcal N}$, is a symmetric binary relation 
${\mathcal S}_{\mathcal N}$ between agents such that $P\rel{S}_{\mathcal N}Q$ implies:
\begin{enumerate}
\item If $P \red P'$ then $Q \wred Q'$ and $P'\rel{S}_{\mathcal N} Q'$.
\item If $P\downarrow_{\mathcal N} x$, then $Q\Downarrow_{\mathcal N} x$.
\end{enumerate}
$P$ is ${\mathcal N}$-barbed bisimilar to $Q$, written
$P \wbbisim_{\mathcal N} Q$, if $P \rel{S}_{\mathcal N} Q$ for some ${\mathcal N}$-barbed bisimulation ${\mathcal S}_{\mathcal N}$.
\end{definition}

$\mathcal{R} \subseteq \pi \times \pi$

$P \mathcal{R} Q => \forall P'. P \red P' \Rightarrow \exists Q'. Q \red Q', P' \mathcal{R} Q'$

$P \vdash x \Rightarrow Q \vdash x$

\begin{mathpar}
  \inferrule*[lab=Out-barb]{x \nameeq y}{{y}!\langle{Q}\rangle \vdash x}
  \and
  \inferrule*[lab=Par-barb]{\mbox{$P\vdash x$ or $Q\vdash x$}}{\binpar{P}{Q} \vdash x}
\end{mathpar}

\subsubsection{Contexts}

One of the principle advantages of computational calculi like the
$\pi$-calculus is a well-defined notion of context,
contextual-equivalence and a correlation between
contextual-equivalence and notions of bisimulation. The notion of
context allows the decomposition of a process into (sub-)process and
its syntactic environment, its context. Thus, a context may be
thought of as a process with a ``hole'' (written $\Box$) in it. The
application of a context $M$ to a process $P$, written $M[P]$, is
tantamount to filling the hole in $M$ with $P$. In this paper we do
not need the full weight of this theory, but do make use of the notion
of context in the proof the main theorem. 

\begin{mathpar}
  \inferrule* [lab=summation] {} {{M_{M},M_{N}} \bc \Box \;|\; x.M_{A} \;|\; M_{M}+M_{N}}
  \and
  \inferrule* [lab=agent] {} {{M_{A}} \bc (\vec{x})M_{P} \;| \; \clift{P_0,\ldots,M_{P},\ldots,P_N}}
  \and \\
  \inferrule* [lab=process] {} {{M_{P}} \bc M_{N} \;| \;P|M_{P} }
\end{mathpar} 

\begin{mathpar}
  \inferrule* [lab=sychronization] {} {M_{N} \bc \Box \;|\; x?M_{F} \;|\; x!M_{C}}
  \and
  \inferrule* [lab=abstraction] {} {{M_{F}} \bc (x)M_{P} }
  \and
  \inferrule* [lab=concretion] {} {{M_{C}} \bc \langle M_{P} \rangle }
  \and \\
  \inferrule* [lab=process] {} {{M_{P}} \bc M_{N} \;| \;P|M_{P} }
\end{mathpar}

\begin{definition}[contextual application] Given a context $M$, and
  process $P$, we define the \emph{contextual application}, $M[P] :=
  M\{P/\Box\}$. That is, the contextual application of M to P is the
  substitution of $P$ for $\Box$ in $M$.
\end{definition}

$\meaningof{-} : L \to \mathcal{P}(\pi)$

\begin{mathpar}
  \inferrule* [lab=collection] {} {\meaningof{true} = \pi, \and \meaningof{~E} = \pi \setminus \meaningof{E}, \and \meaningof{E_{1} \& E_{2}} = \meaningof{E_{1}} \cap \meaningof{E_{2}}}
\end{mathpar}

\begin{mathpar}
  \inferrule* [lab=structure] {} {\meaningof{0} = \{ P \in \pi | P \equiv 0 \}, \and \\ \meaningof{E_1 | E_2} = \{ P \in \pi | P \equiv P_{1} | P_{2}, P_{1} \in \meaningof{E_{1}}, P_{2} \in \meaningof{E_2}\} }
\end{mathpar}

\begin{mathpar}
 \inferrule* [lab=behavior] {} {\meaningof{\langle a?b \rangle E} = \{ P \in \pi | P \equiv Q | u?(y)P', \\ \and \\\\ \and \\ \;\;\; u \in \meaningof{a}, \forall z.P'\{z/y\} \in \meaningof{E\{z/b\}}\}, \and \\ \meaningof{a!E} = \{ P \in \pi | P \equiv Q | x!\langle P' \rangle, x \in \meaningof{a} P' \in \meaningof{E}\} }
\end{mathpar}

\begin{mathpar}
 \inferrule* [lab=nominal] {} {\meaningof{\quotep{E}} = \{ \quotep{P} \in \quotep{\pi} | P \in \meaningof{E} \}, \and \meaningof{\quotep{P}} = \{ \quotep{Q} \in \quotep{\pi} | P \equiv Q \} \and \\ \meaningof{@\quotep{E}} = \{ P \in \pi | P \equiv @x, x \in \meaningof{E} \}}
\end{mathpar}

\begin{eqnarray*}
  \\
  \meaningof{-} : TS \to ST
\end{eqnarray*}

\begin{eqnarray*}
  \\
  L : TS \to ST
\end{eqnarray*}

\begin{eqnarray*}
  \\
  P \models E \iff P \in \meaningof{E}
\end{eqnarray*}

\begin{eqnarray*}
  P \approx_{L} Q \iff \forall E \in L. P \models E \iff Q \models E
\end{eqnarray*}

\begin{eqnarray*}
  P \approx_{K} Q
\end{eqnarray*}

\begin{eqnarray*}
  P \approx Q
\end{eqnarray*}

$\approx_{K} = \approx = \approx_{L}$

\subsubsection{Contextual duality}

Note that contexts extend the quotation operation to a family of
operations from processes to names. Given a context, $M$, we can
define a \emph{nominal context}, $\quotep{M}$ by $\quotep{M}[P] :=
\quotep{M[P]}$. To foreshadow what is to come we observe that these
operations enjoy a duality with processes very much like the duality
between vectors and maps from vectors to scalars.

Further, because the calculus is essentially higher-order, we have a
correspondence between contexts and processes. More specifically,
given a name $x$ and a context $M$ we can construct $M^{*}_{x}$ such
that 

\begin{mathpar}
  M^{*}_{x} | \lift{x}{P} \red M[P]
\end{mathpar}

namely,

\begin{mathpar}
  M^{*}_{x} := x?(u).M[\dropn{u}]
\end{mathpar}

The dependence of $M^{*}_{x}$ on a name makes it an abstraction, 

\begin{mathpar}
  M^{*} := (x)x?(u).M[\dropn{u}]
\end{mathpar}

\subsection{Additional notation}

It will sometimes be convenient to denote the process a name
quotes. We already have the notation $x = \quotep{P}$, but it will be
convenient to introduce an alternate notation, $\procn{x}$, when we
want to emphasize the connection to the use of the name. Note that, by
virtue of name equivalence, $\quotep{\procn{x}} \nameeq x$; so, the
notation is consistent with previous definitions.

Further, because names have structure it is possible to effect
substitutions on the basis of that structure. This means we need to
upgrade our notation for substitutions, which we accomplish by
adapting comprehension notation. Thus,

\begin{mathpar}
  P\{ y / x : x \in S \}
\end{mathpar}

is interpreted to mean the process derived from P by replacing (in a
capture-avoiding manner) each occurrence of $x$ in $S$ by $y$. For example,

\begin{mathpar}
  P\{ \quotep{\procn{x}|\procn{x}} / x : x \in \freenames{P} \}
\end{mathpar}

will replace each (occurrence) of a free name $x$ in $P$ by
$\quotep{\procn{x}|\procn{x}}$.

Also, we will avail ourselves of the notation $x^{L}$ and $x^{R}$ to
denote injections of a name into disjoint copies of the name
space. There are numerous ways to accomplish this. One example can be
found in \cite{MeredithR05}. This notation overloads to vectors of
names: $\vec{x}^{\pi} := (x_{i}^{\pi} \; : \; 0 \leq i < |\vec{x}| )$ where $\pi \in \{L,R\}$.

We also use $P^{\Box} := P|\Box$.

In \cite{MeredithR05} an interpretation of the new operator is
given. It turns out that there are several possible interpretations
all enjoying the requisite algebraic properties of the operator (see
\cite{milner91polyadicpi}). We will therefore make liberal use of
$(\nu\; \vec{x})P$.

% subsection the_syntax_and_semantics_of_the_notation_system (end)   

\input{qm2pi.qmops} 

\input{qm2pi.sterngerlach} 

\input{qm2pi.metric} 

% section concurrent_process_calculi (end)

%\input{qm2pi.proofsketch}

% section proof sketch (end)

%\input{qm2pi.slviaknots} 

% section spatial logic via knots (end)

\input{qm2pi.conclusion}

% section conclusion (end)

%\input{qm2pi.dtcodes} 

% section wiring algorithm (end)

\input{qm2pi.ack} 

% section acknowledgments (end)

\newpage


\bibliographystyle{plain}   
\bibliography{../../biblios/main.bib}

\input{qm2pi.rhodetails}

\end{document}



\end{document}

 

\documentclass[12pt]{llncs}
%\documentclass{jktr}

\usepackage[pdftex]{hyperref}                   
\usepackage {listings}
\usepackage {mathpartir}
\usepackage{bcprules}
%\usepackage{listings}
                       
\usepackage{graphicx} 
%\usepackage[margins=2.5cm,nohead,nofoot]{geometry}
%\usepackage{geometry}
\usepackage{amsfonts}
\usepackage{amstext}
\usepackage{latexsym}
\usepackage{amssymb}
\usepackage{color}


%\include{myPreamble}
\documentclass[12pt]{llncs}
%\documentclass{jktr}

\usepackage[pdftex]{hyperref}                   
\usepackage {listings}
\usepackage {mathpartir}
\usepackage{bcprules}
%\usepackage{listings}
                       
\usepackage{graphicx} 
%\usepackage[margins=2.5cm,nohead,nofoot]{geometry}
%\usepackage{geometry}
\usepackage{amsfonts}
\usepackage{amstext}
\usepackage{latexsym}
\usepackage{amssymb}
\usepackage{color}


%\include{myPreamble}
\include{qm2pi.local} 

%\ifpdf
%\usepackage[pdftex]{graphicx}
%\else
%\usepackage{graphicx}
%\fi

 % \ifpdf
%  \usepackage{pdfsync}
%  \if


%\title{Brief Article}
%\author{David F. Snyder}
%\author{L.G. Meredith}

%\address{Dept. of Math., Texas State University--San Marcos, San Marcos, TX 78666}
       
\pagestyle{empty}


\begin{document}

\lstset{language=[Objective]Caml,frame=shadowbox}

\input{qm2pi.front}

% section front matter (end)

\input{qm2pi.intro} 
 
% section introduction (end)

% \input{qm2pi.knotations} 

% section notation (end)

\input{qm2pi.process.calculi} 

% section concurrent_process_calculi_and_spatial_logics_ (end)
    
%\input{qm2pi.knots2pi} 

%\input{qm2pi.trefoil} 

%\input{qm2pi.mainthm} 

% subsection basic_interpretation (end)

%\input{qm2pi.rho.presentation} 
\subsection{The syntax and semantics of the notation system}\label{sub:the_syntax_and_semantics_of_the_notation_system} % (fold)

We now summarize a technical presentation of the calculus that
embodies our theory of dynamics. The typical presentation of such a
calculus follows the style of giving generators and relations on
them. The grammar, below, describing term constructors, freely
generates the set of processes, $\Proc$. This set is then quotiented
by a relation known as structural congruence and it is over this set
that the notion of dynamics is expressed. This presentation is
essentially that of \cite{MeredithR05} with the addition of
polyadicity and summation. For readability we have relegated some of
the technical subtleties to an appendix.

\subsubsection{Process grammar}\label{subsub:process_grammar}

\begin{mathpar}
  \inferrule* [lab=synchronization] {} {{M} \bc \pzero \;|\; x?F \;|\; x!C }
  \and
  \inferrule* [lab=abstraction] {} {{F} \bc (x)P}
  \and
  \inferrule* [lab=concretion] {} {{C} \bc \langle Q \rangle}
  \and
  \inferrule* [lab=process] {} {{P,Q} \bc M \;| \;P|Q \;|\; @{x}}
  \and
  \inferrule* [lab=name] {} {{x} \bc \quotep{P}}
\end{mathpar} 

Note that $\vec{x}$ (resp. $\vec{P}$) denotes a vector of names
(resp. processes) of length $|\vec{x}|$ (resp. $|\vec{P}|$). We adopt
the following useful abbreviations.

\begin{mathpar}
   x?(\vec{y}).P := x.(\vec{y})P \and  x\clift{\vec{P}} := x.\clift{\vec{P}}
   \and x!(y) := \lift{x}{\dropn{y}}
   \and \Pi_{i=0}^{n-1}P_i := P_0 | \ldots | P_{n-1}
\end{mathpar}

\subsubsection{Structural congruence}

\paragraph{Free and bound names and alpha-equivalence.} At the
core of structural equivalence is alpha-equivalence which identifies
process that are the same up to a change of variable. Formally, we
recognize the distinction between free and bound names. The free names
of a process, $\freenames{P}$, may be calculated recursively as
follows:

\begin{mathpar}
\freenames{\pzero} := \emptyset
  \and \\
  \freenames{x?(y).P} := \{ x \} \cup (\freenames{P} \setminus \{ y \})
  \and 
  \freenames{x!\langle P \rangle} := \{ x \} \cup \{ P \} 
  \and \\
  \freenames{P|Q} := \freenames{P} \cup \freenames{Q}
  \and \\
  \freenames{@{x}} := \{ x \}
\end{mathpar}

$\pi$
$\quotep{\pi}$

$\freenames{-} : \pi \to \mathcal{P}(\quotep{\pi})$

\begin{eqnarray*}
  \freenames{\pzero} & := & \emptyset \\
  \freenames{x?(y).P} & := & \{ x \} \cup (\freenames{P} \setminus \{ y \}) \\
  \freenames{x!\langle P \rangle} & := & \{ x \} \cup \{ P \} \\
  \freenames{P|Q} & := & \freenames{P} \cup \freenames{Q} \\
  \freenames{\dropn{x}} & := & \{ x \}
\end{eqnarray*}

The bound names of a process, $\boundnames{P}$, are those names occurring in $P$
that are not free. For example, in $x?(y).0$, the name $x$ is free, while $y$ is bound.

\begin{mathpar}
  \inferrule* [lab=monoidal-laws] {} { P|Q \equiv Q|P \and P|0 \equiv P \and P|(Q|R) \equiv (P|Q)|R }
\end{mathpar}

\begin{mathpar}
  \inferrule* [lab=alpha-equivalence] {} { (x)P \equiv (y)P\{y/x\} \and y \not\in \freenames{P} }
\end{mathpar}

\begin{definition}
Then two processes, $P,Q$, are alpha-equivalent if $P = Q\{\vec{y}/\vec{x}\}$ for
some $\vec{x} \in \boundnames{Q},\vec{y} \in \boundnames{P}$, where $Q\{\vec{y}/\vec{x}\}$
denotes the capture-avoiding substitution of $\vec{y}$ for $\vec{x}$ in $Q$.
\end{definition}

\begin{definition}
  The {\em structural congruence} \cite{SangiorgiWalker} , $\equiv$,
  between processes is the least congruence containing
  alpha-equivalence, satisfying the abelian monoid laws
  (associativity, commutativity and $\pzero$ as identity) for parallel
  composition $|$ and for summation $+$.
\end{definition}

\subsection{Name equivalence}

We take name equivalence, written $\nameeq$, to be the smallest
equivalence relation generated by the following rules.

\begin{mathpar}
\inferrule*[lab=Quote-drop]
{ }
{ \quotep{@{x}} \nameeq x }

\inferrule*[lab=Struct-equiv]
{ P \scong Q }
{ \quotep{P} \nameeq \quotep{Q} }
\end{mathpar}

The astute reader will have noticed that the mutual recursion of names
and processes imposes a mutual recursion on alpha-equivalence and
structural equivalence via name-equivalence. Fortunately, all of this
works out pleasantly and we may calculate in the natural way, free of
concern. The reader interested in the details is referred to the
appendix \ref{appendix:rho_details}.

\subsection{Substitution}

We use $\Proc$ for the set of processes, $\QProc$ for the set of
names, and $\id{\{}\vec{y} / \vec{x} \id{\}}$ to denote partial maps,
$s : \QProc \rightarrow \QProc$. A map, $s$ lifts, uniquely, to a map
on process terms, $\widehat{s} : \Proc \rightarrow \Proc$ by the
following equations.

\begin{mathpar}
  (0) \psubstp{Q}{P} := 0 \\
  (R \juxtap S) \psubstp{Q}{P}
  :=    
  (R)\psubstp{Q}{P} \juxtap (S) \psubstp{Q}{P} \\
  (x?(y).R) \psubstp{Q}{P}    
  :=    
  (x)\substp{Q}{P} (z)\concat( (R \psubstn{z}{y}) \psubstp{Q}{P} ) \\
  (\lift{x}{R}) \psubstp{Q}{P}  
  :=
  \lift{(x)\substp{Q}{P}}{ R \psubstp{Q}{P} } \\
%   (\dropn{x})  \psubstp{Q}{P}       
%   := 
%   \left\{ 
%     \begin{array}{ccc} 
%       \dropn{\quotep{Q}} & & x \nameeq \quotep{P} \\
%       \dropn{x} & & otherwise \\
%     \end{array}
%   \right. 
  (\dropn{x})  \psubstp{Q}{P}       
  := 
  \left\{ 
    \begin{array}{ccc} 
      Q & & x \nameeq \quotep{P} \\
      \dropn{x} & & otherwise \\
    \end{array}
  \right.
\end{mathpar}
 

where

\begin{eqnarray}
  (x)\id{\{} \lpquote Q \rpquote / \lpquote P \rpquote \id{\}}            = 
  \left\{ 
    \begin{array}{ccc}
      \lpquote Q \rpquote & & x \nameeq \lpquote P \rpquote \\
      x & & otherwise \\
    \end{array}
  \right. \nonumber
\end{eqnarray}

and $z$ is chosen distinct from $\quotep{P}$, $\quotep{Q}$, the free
names in $Q$, and all the names in $R$. Our $\alpha$-equivalence will
be built in the standard way from this substitution.

\begin{remark}\label{rem:no_self_referential_names}
  One consequence of these definitions is that $\forall P. \quotep{P}
  \not\in \freenames{P}$.
\end{remark}

\subsection{ Dynamic quote: an example }

Anticipating something of what's to come, consider applying the
substitution, $\widehat{\id{\{}u / z \id{\}}}$, to the following pair
of processes, $\lift{w}{y!(z)}$ and $w[ \lpquote y!(z) \rpquote ]$.

\begin{eqnarray}
	\lift{w}{y!(z)}\widehat{\id{\{}u / z \id{\}}}
		& = &
		\lift{w}{y!(u)} \nonumber\\
	w[ \lpquote y!(z) \rpquote ] \widehat{ \id{\{}u / z \id{\}} }
		& = &
		w[ \lpquote y!(z) \rpquote ] \nonumber
\end{eqnarray}

Because the body of the process between quotes is impervious to
substitution, we get radically different answers. In fact, by
examining the first process in an input context,
e.g. $x?(z).\lift{w}{y!(z)}$, we see that the process under the lift
operator may be shaped by prefixed inputs binding a name inside it. In
this sense, the lift operator will be seen as a way to dynamically
construct processes before reifying them as names.

Finally equipped with these standard features we can present the
dynamics of the calculus.

\subsubsection{Operational semantics} 

Finally, we introduce the computational dynamics. What marks these
algebras as distinct from other more traditionally studied algebraic
structures, e.g. vector spaces or polynomial rings, is the manner in
which dynamics is captured. In traditional structures, dynamics is typically
expressed through morphisms between such structures, as in linear maps
between vector spaces or morphisms between rings. In algebras
associated with the semantics of computation, the dynamics is
expressed as part of the algebraic structure itself, through a
reduction reduction relation typically denoted by $\red$. Below, we
give a recursive presentation of this relation for the calculus used
in the encoding.

$\red \subseteq \pi \times \pi$
$\red : \pi \to \mathcal{P}(\pi)$

\begin{mathpar}
  \inferrule* [lab=Comm] { \textsf{match}( x_{src}, x_{trgt} ) } { x_{trgt}?(y)P \; | \; x_{src}!\langle {Q} \rangle \red P\{\quotep{Q}/y}\} }
  \and \\
  \inferrule* [lab=Par] {{P} \red {P}'} {{{P} | {Q}} \red {{P}' | {Q}}}
  \and
  \inferrule* [lab=Equiv]{{{P} \scong {P}'} \andalso {{P}' \red {Q}'} \andalso {{Q}' \scong {Q}}}{{P} \red {Q}}
\end{mathpar}

\begin{eqnarray*}
  match_{\equiv} (\quotep{P},\quotep{Q}) & := & P \equiv Q \\
  match_{\dagger}(\quotep{P},\quotep{Q}) & := & \forall R. P|Q \red^{*} R => R \red^{*} 0 \\
  match_{K}(\quotep{P},\quotep{Q}) & := & K \mbox{ for some context } K
\end{eqnarray*}

$u?(x)P | u!\langle Q \rangle \red P\{\quotep{Q}/x\}$

%We write $\wred$ for $\red^*$, and $P\red$ if $\exists Q $ such that $ P \red Q$.
We write $P\red$ if $\exists Q $ such that $ P \red Q$ and $P\not\red$, otherwise.

\section{Replication}

As mentioned before, it is known that replication (and hence
recursion) can be implemented in a higher-order process algebra
\cite{SangiorgiWalker}. As our first example of calculation with the
machinery thus far presented we give the construction explicitly in
the {\rhoc}.

\begin{eqnarray}
	D_{x} & := & \prefix{x}{y}{(\binpar{\outputp{x}{y}}{@{y}})} \nonumber\\
	\bangp_{x}{P} & := & \binpar{{x}!\langle{\binpar{D_{x}}{P}}\rangle}{D_{x}} \nonumber
\end{eqnarray}

\begin{eqnarray}
	\bangp_{x}{P} & & \nonumber\\
	=
	& {x}!\langle{(\prefix{x}{y}{(\outputp{x}{y} | @{y})) | P}}\rangle 
	      | \prefix{x}{y}{(\outputp{x}{y} | @{y})} & \nonumber\\
	\red
	& (\outputp{x}{y} | @{y})\substn{\quotep{(\prefix{x}{y}{(@{y} | \outputp{x}{y})) | P}}}{y} & \nonumber\\
	=
	& \outputp{x}{\quotep{(\prefix{x}{y}{(\outputp{x}{y} | @{y})) | P}}}
	  | {(\prefix{x}{y}{(\outputp{x}{y} | @{y})) | P}} & \nonumber\\
	\red
	& \ldots & \nonumber\\
	\red^*
	& P | P | \ldots & \nonumber
\end{eqnarray}

Of course, this encoding, as an implementation, runs away, unfolding
$\bangp{P}$ eagerly. A lazier and more implementable replication
operator, restricted to input-guarded processes, may be obtained as follows.

\begin{eqnarray}
\bangp{\prefix{u}{v}{P}} 
	:= 
	\binpar{\lift{x}{\prefix{u}{v}{(\binpar{D(x)}{P})}}}{D(x)} \nonumber
\end{eqnarray}

\begin{remark}
  Note that the lazier definition still does not deal with summation
  or mixed summation (i.e. sums over input and output). The reader is
  invited to construct definitions of replication that deal with these
  features. 

  Further, the definitions are parameterized in a name, $x$. Can you,
  gentle reader, make a definition that eliminates this parameter and
  guarantees no accidental interaction between the replication
  machinery and the process being replicated -- i.e. no accidental
  sharing of names used by the process to get its work done and the
  name(s) used by the replication to effect copying. This latter
  revision of the definition of replication is crucial to obtaining
  the expected identity $!!P \sim !P$.
\end{remark}

\begin{remark}\label{rem:paradoxical_combinator}
  The reader familiar with the lambda calculus will have noticed the
  similarity between $D$ and the paradoxical combinator.

  [Ed. note: the existence of this seems to suggest we have to be more
  restrictive on the set of processes and names we admit if we are to
  support no-cloning.]
\end{remark}

\subsubsection{Bisimulation}

The computational dynamics gives rise to another kind of equivalence,
the equivalence of computational behavior. As previously mentioned
this is typically captured \emph{via} some form of bisimulation.

% The notion we use in this paper is weak barbed bisimulation
% \cite{milner91polyadicpi}.

The notion we use in this paper is derived from weak barbed
bisimulation \cite{milner91polyadicpi}. 

\begin{definition}
An \emph{observation relation}, $\downarrow_{\mathcal N}$, over a set
of names, $\mathcal N$, is the smallest relation satisfying the rules
below.

\infrule[Out-barb]{y \in {\mathcal N}, \; x \nameeq y}
		  {\outputp{x}{v} \downarrow_{\mathcal N} x}
\infrule[Par-barb]{\mbox{$P\downarrow_{\mathcal N} x$ or $Q\downarrow_{\mathcal N} x$}}
		  {\binpar{P}{Q} \downarrow_{\mathcal N} x}

We write $P \Downarrow_{\mathcal N} x$ if there is $Q$ such that 
$P \wred Q$ and $Q \downarrow_{\mathcal N} x$.
\end{definition}

\begin{definition}
%\label{def.bbisim}
An  ${\mathcal N}$-\emph{barbed bisimulation} over a set of names, ${\mathcal N}$, is a symmetric binary relation 
${\mathcal S}_{\mathcal N}$ between agents such that $P\rel{S}_{\mathcal N}Q$ implies:
\begin{enumerate}
\item If $P \red P'$ then $Q \wred Q'$ and $P'\rel{S}_{\mathcal N} Q'$.
\item If $P\downarrow_{\mathcal N} x$, then $Q\Downarrow_{\mathcal N} x$.
\end{enumerate}
$P$ is ${\mathcal N}$-barbed bisimilar to $Q$, written
$P \wbbisim_{\mathcal N} Q$, if $P \rel{S}_{\mathcal N} Q$ for some ${\mathcal N}$-barbed bisimulation ${\mathcal S}_{\mathcal N}$.
\end{definition}

$\mathcal{R} \subseteq \pi \times \pi$

$P \mathcal{R} Q => \forall P'. P \red P' \Rightarrow \exists Q'. Q \red Q', P' \mathcal{R} Q'$

$P \vdash x \Rightarrow Q \vdash x$

\begin{mathpar}
  \inferrule*[lab=Out-barb]{x \nameeq y}{{y}!\langle{Q}\rangle \vdash x}
  \and
  \inferrule*[lab=Par-barb]{\mbox{$P\vdash x$ or $Q\vdash x$}}{\binpar{P}{Q} \vdash x}
\end{mathpar}

\subsubsection{Contexts}

One of the principle advantages of computational calculi like the
$\pi$-calculus is a well-defined notion of context,
contextual-equivalence and a correlation between
contextual-equivalence and notions of bisimulation. The notion of
context allows the decomposition of a process into (sub-)process and
its syntactic environment, its context. Thus, a context may be
thought of as a process with a ``hole'' (written $\Box$) in it. The
application of a context $M$ to a process $P$, written $M[P]$, is
tantamount to filling the hole in $M$ with $P$. In this paper we do
not need the full weight of this theory, but do make use of the notion
of context in the proof the main theorem. 

\begin{mathpar}
  \inferrule* [lab=summation] {} {{M_{M},M_{N}} \bc \Box \;|\; x.M_{A} \;|\; M_{M}+M_{N}}
  \and
  \inferrule* [lab=agent] {} {{M_{A}} \bc (\vec{x})M_{P} \;| \; \clift{P_0,\ldots,M_{P},\ldots,P_N}}
  \and \\
  \inferrule* [lab=process] {} {{M_{P}} \bc M_{N} \;| \;P|M_{P} }
\end{mathpar} 

\begin{mathpar}
  \inferrule* [lab=sychronization] {} {M_{N} \bc \Box \;|\; x?M_{F} \;|\; x!M_{C}}
  \and
  \inferrule* [lab=abstraction] {} {{M_{F}} \bc (x)M_{P} }
  \and
  \inferrule* [lab=concretion] {} {{M_{C}} \bc \langle M_{P} \rangle }
  \and \\
  \inferrule* [lab=process] {} {{M_{P}} \bc M_{N} \;| \;P|M_{P} }
\end{mathpar}

\begin{definition}[contextual application] Given a context $M$, and
  process $P$, we define the \emph{contextual application}, $M[P] :=
  M\{P/\Box\}$. That is, the contextual application of M to P is the
  substitution of $P$ for $\Box$ in $M$.
\end{definition}

$\meaningof{-} : L \to \mathcal{P}(\pi)$

\begin{mathpar}
  \inferrule* [lab=collection] {} {\meaningof{true} = \pi, \and \meaningof{~E} = \pi \setminus \meaningof{E}, \and \meaningof{E_{1} \& E_{2}} = \meaningof{E_{1}} \cap \meaningof{E_{2}}}
\end{mathpar}

\begin{mathpar}
  \inferrule* [lab=structure] {} {\meaningof{0} = \{ P \in \pi | P \equiv 0 \}, \and \\ \meaningof{E_1 | E_2} = \{ P \in \pi | P \equiv P_{1} | P_{2}, P_{1} \in \meaningof{E_{1}}, P_{2} \in \meaningof{E_2}\} }
\end{mathpar}

\begin{mathpar}
 \inferrule* [lab=behavior] {} {\meaningof{\langle a?b \rangle E} = \{ P \in \pi | P \equiv Q | u?(y)P', \\ \and \\\\ \and \\ \;\;\; u \in \meaningof{a}, \forall z.P'\{z/y\} \in \meaningof{E\{z/b\}}\}, \and \\ \meaningof{a!E} = \{ P \in \pi | P \equiv Q | x!\langle P' \rangle, x \in \meaningof{a} P' \in \meaningof{E}\} }
\end{mathpar}

\begin{mathpar}
 \inferrule* [lab=nominal] {} {\meaningof{\quotep{E}} = \{ \quotep{P} \in \quotep{\pi} | P \in \meaningof{E} \}, \and \meaningof{\quotep{P}} = \{ \quotep{Q} \in \quotep{\pi} | P \equiv Q \} \and \\ \meaningof{@\quotep{E}} = \{ P \in \pi | P \equiv @x, x \in \meaningof{E} \}}
\end{mathpar}

\begin{eqnarray*}
  \\
  \meaningof{-} : TS \to ST
\end{eqnarray*}

\begin{eqnarray*}
  \\
  L : TS \to ST
\end{eqnarray*}

\begin{eqnarray*}
  \\
  P \models E \iff P \in \meaningof{E}
\end{eqnarray*}

\begin{eqnarray*}
  P \approx_{L} Q \iff \forall E \in L. P \models E \iff Q \models E
\end{eqnarray*}

\begin{eqnarray*}
  P \approx_{K} Q
\end{eqnarray*}

\begin{eqnarray*}
  P \approx Q
\end{eqnarray*}

$\approx_{K} = \approx = \approx_{L}$

\subsubsection{Contextual duality}

Note that contexts extend the quotation operation to a family of
operations from processes to names. Given a context, $M$, we can
define a \emph{nominal context}, $\quotep{M}$ by $\quotep{M}[P] :=
\quotep{M[P]}$. To foreshadow what is to come we observe that these
operations enjoy a duality with processes very much like the duality
between vectors and maps from vectors to scalars.

Further, because the calculus is essentially higher-order, we have a
correspondence between contexts and processes. More specifically,
given a name $x$ and a context $M$ we can construct $M^{*}_{x}$ such
that 

\begin{mathpar}
  M^{*}_{x} | \lift{x}{P} \red M[P]
\end{mathpar}

namely,

\begin{mathpar}
  M^{*}_{x} := x?(u).M[\dropn{u}]
\end{mathpar}

The dependence of $M^{*}_{x}$ on a name makes it an abstraction, 

\begin{mathpar}
  M^{*} := (x)x?(u).M[\dropn{u}]
\end{mathpar}

\subsection{Additional notation}

It will sometimes be convenient to denote the process a name
quotes. We already have the notation $x = \quotep{P}$, but it will be
convenient to introduce an alternate notation, $\procn{x}$, when we
want to emphasize the connection to the use of the name. Note that, by
virtue of name equivalence, $\quotep{\procn{x}} \nameeq x$; so, the
notation is consistent with previous definitions.

Further, because names have structure it is possible to effect
substitutions on the basis of that structure. This means we need to
upgrade our notation for substitutions, which we accomplish by
adapting comprehension notation. Thus,

\begin{mathpar}
  P\{ y / x : x \in S \}
\end{mathpar}

is interpreted to mean the process derived from P by replacing (in a
capture-avoiding manner) each occurrence of $x$ in $S$ by $y$. For example,

\begin{mathpar}
  P\{ \quotep{\procn{x}|\procn{x}} / x : x \in \freenames{P} \}
\end{mathpar}

will replace each (occurrence) of a free name $x$ in $P$ by
$\quotep{\procn{x}|\procn{x}}$.

Also, we will avail ourselves of the notation $x^{L}$ and $x^{R}$ to
denote injections of a name into disjoint copies of the name
space. There are numerous ways to accomplish this. One example can be
found in \cite{MeredithR05}. This notation overloads to vectors of
names: $\vec{x}^{\pi} := (x_{i}^{\pi} \; : \; 0 \leq i < |\vec{x}| )$ where $\pi \in \{L,R\}$.

We also use $P^{\Box} := P|\Box$.

In \cite{MeredithR05} an interpretation of the new operator is
given. It turns out that there are several possible interpretations
all enjoying the requisite algebraic properties of the operator (see
\cite{milner91polyadicpi}). We will therefore make liberal use of
$(\nu\; \vec{x})P$.

% subsection the_syntax_and_semantics_of_the_notation_system (end)   

\input{qm2pi.qmops} 

\input{qm2pi.sterngerlach} 

\input{qm2pi.metric} 

% section concurrent_process_calculi (end)

%\input{qm2pi.proofsketch}

% section proof sketch (end)

%\input{qm2pi.slviaknots} 

% section spatial logic via knots (end)

\input{qm2pi.conclusion}

% section conclusion (end)

%\input{qm2pi.dtcodes} 

% section wiring algorithm (end)

\input{qm2pi.ack} 

% section acknowledgments (end)

\newpage


\bibliographystyle{plain}   
\bibliography{../../biblios/main.bib}

\input{qm2pi.rhodetails}

\end{document}

 

%\ifpdf
%\usepackage[pdftex]{graphicx}
%\else
%\usepackage{graphicx}
%\fi

 % \ifpdf
%  \usepackage{pdfsync}
%  \if


%\title{Brief Article}
%\author{David F. Snyder}
%\author{L.G. Meredith}

%\address{Dept. of Math., Texas State University--San Marcos, San Marcos, TX 78666}
       
\pagestyle{empty}


\begin{document}

\lstset{language=[Objective]Caml,frame=shadowbox}

\documentclass[12pt]{llncs}
%\documentclass{jktr}

\usepackage[pdftex]{hyperref}                   
\usepackage {listings}
\usepackage {mathpartir}
\usepackage{bcprules}
%\usepackage{listings}
                       
\usepackage{graphicx} 
%\usepackage[margins=2.5cm,nohead,nofoot]{geometry}
%\usepackage{geometry}
\usepackage{amsfonts}
\usepackage{amstext}
\usepackage{latexsym}
\usepackage{amssymb}
\usepackage{color}


%\include{myPreamble}
\include{qm2pi.local} 

%\ifpdf
%\usepackage[pdftex]{graphicx}
%\else
%\usepackage{graphicx}
%\fi

 % \ifpdf
%  \usepackage{pdfsync}
%  \if


%\title{Brief Article}
%\author{David F. Snyder}
%\author{L.G. Meredith}

%\address{Dept. of Math., Texas State University--San Marcos, San Marcos, TX 78666}
       
\pagestyle{empty}


\begin{document}

\lstset{language=[Objective]Caml,frame=shadowbox}

\input{qm2pi.front}

% section front matter (end)

\input{qm2pi.intro} 
 
% section introduction (end)

% \input{qm2pi.knotations} 

% section notation (end)

\input{qm2pi.process.calculi} 

% section concurrent_process_calculi_and_spatial_logics_ (end)
    
%\input{qm2pi.knots2pi} 

%\input{qm2pi.trefoil} 

%\input{qm2pi.mainthm} 

% subsection basic_interpretation (end)

%\input{qm2pi.rho.presentation} 
\subsection{The syntax and semantics of the notation system}\label{sub:the_syntax_and_semantics_of_the_notation_system} % (fold)

We now summarize a technical presentation of the calculus that
embodies our theory of dynamics. The typical presentation of such a
calculus follows the style of giving generators and relations on
them. The grammar, below, describing term constructors, freely
generates the set of processes, $\Proc$. This set is then quotiented
by a relation known as structural congruence and it is over this set
that the notion of dynamics is expressed. This presentation is
essentially that of \cite{MeredithR05} with the addition of
polyadicity and summation. For readability we have relegated some of
the technical subtleties to an appendix.

\subsubsection{Process grammar}\label{subsub:process_grammar}

\begin{mathpar}
  \inferrule* [lab=synchronization] {} {{M} \bc \pzero \;|\; x?F \;|\; x!C }
  \and
  \inferrule* [lab=abstraction] {} {{F} \bc (x)P}
  \and
  \inferrule* [lab=concretion] {} {{C} \bc \langle Q \rangle}
  \and
  \inferrule* [lab=process] {} {{P,Q} \bc M \;| \;P|Q \;|\; @{x}}
  \and
  \inferrule* [lab=name] {} {{x} \bc \quotep{P}}
\end{mathpar} 

Note that $\vec{x}$ (resp. $\vec{P}$) denotes a vector of names
(resp. processes) of length $|\vec{x}|$ (resp. $|\vec{P}|$). We adopt
the following useful abbreviations.

\begin{mathpar}
   x?(\vec{y}).P := x.(\vec{y})P \and  x\clift{\vec{P}} := x.\clift{\vec{P}}
   \and x!(y) := \lift{x}{\dropn{y}}
   \and \Pi_{i=0}^{n-1}P_i := P_0 | \ldots | P_{n-1}
\end{mathpar}

\subsubsection{Structural congruence}

\paragraph{Free and bound names and alpha-equivalence.} At the
core of structural equivalence is alpha-equivalence which identifies
process that are the same up to a change of variable. Formally, we
recognize the distinction between free and bound names. The free names
of a process, $\freenames{P}$, may be calculated recursively as
follows:

\begin{mathpar}
\freenames{\pzero} := \emptyset
  \and \\
  \freenames{x?(y).P} := \{ x \} \cup (\freenames{P} \setminus \{ y \})
  \and 
  \freenames{x!\langle P \rangle} := \{ x \} \cup \{ P \} 
  \and \\
  \freenames{P|Q} := \freenames{P} \cup \freenames{Q}
  \and \\
  \freenames{@{x}} := \{ x \}
\end{mathpar}

$\pi$
$\quotep{\pi}$

$\freenames{-} : \pi \to \mathcal{P}(\quotep{\pi})$

\begin{eqnarray*}
  \freenames{\pzero} & := & \emptyset \\
  \freenames{x?(y).P} & := & \{ x \} \cup (\freenames{P} \setminus \{ y \}) \\
  \freenames{x!\langle P \rangle} & := & \{ x \} \cup \{ P \} \\
  \freenames{P|Q} & := & \freenames{P} \cup \freenames{Q} \\
  \freenames{\dropn{x}} & := & \{ x \}
\end{eqnarray*}

The bound names of a process, $\boundnames{P}$, are those names occurring in $P$
that are not free. For example, in $x?(y).0$, the name $x$ is free, while $y$ is bound.

\begin{mathpar}
  \inferrule* [lab=monoidal-laws] {} { P|Q \equiv Q|P \and P|0 \equiv P \and P|(Q|R) \equiv (P|Q)|R }
\end{mathpar}

\begin{mathpar}
  \inferrule* [lab=alpha-equivalence] {} { (x)P \equiv (y)P\{y/x\} \and y \not\in \freenames{P} }
\end{mathpar}

\begin{definition}
Then two processes, $P,Q$, are alpha-equivalent if $P = Q\{\vec{y}/\vec{x}\}$ for
some $\vec{x} \in \boundnames{Q},\vec{y} \in \boundnames{P}$, where $Q\{\vec{y}/\vec{x}\}$
denotes the capture-avoiding substitution of $\vec{y}$ for $\vec{x}$ in $Q$.
\end{definition}

\begin{definition}
  The {\em structural congruence} \cite{SangiorgiWalker} , $\equiv$,
  between processes is the least congruence containing
  alpha-equivalence, satisfying the abelian monoid laws
  (associativity, commutativity and $\pzero$ as identity) for parallel
  composition $|$ and for summation $+$.
\end{definition}

\subsection{Name equivalence}

We take name equivalence, written $\nameeq$, to be the smallest
equivalence relation generated by the following rules.

\begin{mathpar}
\inferrule*[lab=Quote-drop]
{ }
{ \quotep{@{x}} \nameeq x }

\inferrule*[lab=Struct-equiv]
{ P \scong Q }
{ \quotep{P} \nameeq \quotep{Q} }
\end{mathpar}

The astute reader will have noticed that the mutual recursion of names
and processes imposes a mutual recursion on alpha-equivalence and
structural equivalence via name-equivalence. Fortunately, all of this
works out pleasantly and we may calculate in the natural way, free of
concern. The reader interested in the details is referred to the
appendix \ref{appendix:rho_details}.

\subsection{Substitution}

We use $\Proc$ for the set of processes, $\QProc$ for the set of
names, and $\id{\{}\vec{y} / \vec{x} \id{\}}$ to denote partial maps,
$s : \QProc \rightarrow \QProc$. A map, $s$ lifts, uniquely, to a map
on process terms, $\widehat{s} : \Proc \rightarrow \Proc$ by the
following equations.

\begin{mathpar}
  (0) \psubstp{Q}{P} := 0 \\
  (R \juxtap S) \psubstp{Q}{P}
  :=    
  (R)\psubstp{Q}{P} \juxtap (S) \psubstp{Q}{P} \\
  (x?(y).R) \psubstp{Q}{P}    
  :=    
  (x)\substp{Q}{P} (z)\concat( (R \psubstn{z}{y}) \psubstp{Q}{P} ) \\
  (\lift{x}{R}) \psubstp{Q}{P}  
  :=
  \lift{(x)\substp{Q}{P}}{ R \psubstp{Q}{P} } \\
%   (\dropn{x})  \psubstp{Q}{P}       
%   := 
%   \left\{ 
%     \begin{array}{ccc} 
%       \dropn{\quotep{Q}} & & x \nameeq \quotep{P} \\
%       \dropn{x} & & otherwise \\
%     \end{array}
%   \right. 
  (\dropn{x})  \psubstp{Q}{P}       
  := 
  \left\{ 
    \begin{array}{ccc} 
      Q & & x \nameeq \quotep{P} \\
      \dropn{x} & & otherwise \\
    \end{array}
  \right.
\end{mathpar}
 

where

\begin{eqnarray}
  (x)\id{\{} \lpquote Q \rpquote / \lpquote P \rpquote \id{\}}            = 
  \left\{ 
    \begin{array}{ccc}
      \lpquote Q \rpquote & & x \nameeq \lpquote P \rpquote \\
      x & & otherwise \\
    \end{array}
  \right. \nonumber
\end{eqnarray}

and $z$ is chosen distinct from $\quotep{P}$, $\quotep{Q}$, the free
names in $Q$, and all the names in $R$. Our $\alpha$-equivalence will
be built in the standard way from this substitution.

\begin{remark}\label{rem:no_self_referential_names}
  One consequence of these definitions is that $\forall P. \quotep{P}
  \not\in \freenames{P}$.
\end{remark}

\subsection{ Dynamic quote: an example }

Anticipating something of what's to come, consider applying the
substitution, $\widehat{\id{\{}u / z \id{\}}}$, to the following pair
of processes, $\lift{w}{y!(z)}$ and $w[ \lpquote y!(z) \rpquote ]$.

\begin{eqnarray}
	\lift{w}{y!(z)}\widehat{\id{\{}u / z \id{\}}}
		& = &
		\lift{w}{y!(u)} \nonumber\\
	w[ \lpquote y!(z) \rpquote ] \widehat{ \id{\{}u / z \id{\}} }
		& = &
		w[ \lpquote y!(z) \rpquote ] \nonumber
\end{eqnarray}

Because the body of the process between quotes is impervious to
substitution, we get radically different answers. In fact, by
examining the first process in an input context,
e.g. $x?(z).\lift{w}{y!(z)}$, we see that the process under the lift
operator may be shaped by prefixed inputs binding a name inside it. In
this sense, the lift operator will be seen as a way to dynamically
construct processes before reifying them as names.

Finally equipped with these standard features we can present the
dynamics of the calculus.

\subsubsection{Operational semantics} 

Finally, we introduce the computational dynamics. What marks these
algebras as distinct from other more traditionally studied algebraic
structures, e.g. vector spaces or polynomial rings, is the manner in
which dynamics is captured. In traditional structures, dynamics is typically
expressed through morphisms between such structures, as in linear maps
between vector spaces or morphisms between rings. In algebras
associated with the semantics of computation, the dynamics is
expressed as part of the algebraic structure itself, through a
reduction reduction relation typically denoted by $\red$. Below, we
give a recursive presentation of this relation for the calculus used
in the encoding.

$\red \subseteq \pi \times \pi$
$\red : \pi \to \mathcal{P}(\pi)$

\begin{mathpar}
  \inferrule* [lab=Comm] { \textsf{match}( x_{src}, x_{trgt} ) } { x_{trgt}?(y)P \; | \; x_{src}!\langle {Q} \rangle \red P\{\quotep{Q}/y}\} }
  \and \\
  \inferrule* [lab=Par] {{P} \red {P}'} {{{P} | {Q}} \red {{P}' | {Q}}}
  \and
  \inferrule* [lab=Equiv]{{{P} \scong {P}'} \andalso {{P}' \red {Q}'} \andalso {{Q}' \scong {Q}}}{{P} \red {Q}}
\end{mathpar}

\begin{eqnarray*}
  match_{\equiv} (\quotep{P},\quotep{Q}) & := & P \equiv Q \\
  match_{\dagger}(\quotep{P},\quotep{Q}) & := & \forall R. P|Q \red^{*} R => R \red^{*} 0 \\
  match_{K}(\quotep{P},\quotep{Q}) & := & K \mbox{ for some context } K
\end{eqnarray*}

$u?(x)P | u!\langle Q \rangle \red P\{\quotep{Q}/x\}$

%We write $\wred$ for $\red^*$, and $P\red$ if $\exists Q $ such that $ P \red Q$.
We write $P\red$ if $\exists Q $ such that $ P \red Q$ and $P\not\red$, otherwise.

\section{Replication}

As mentioned before, it is known that replication (and hence
recursion) can be implemented in a higher-order process algebra
\cite{SangiorgiWalker}. As our first example of calculation with the
machinery thus far presented we give the construction explicitly in
the {\rhoc}.

\begin{eqnarray}
	D_{x} & := & \prefix{x}{y}{(\binpar{\outputp{x}{y}}{@{y}})} \nonumber\\
	\bangp_{x}{P} & := & \binpar{{x}!\langle{\binpar{D_{x}}{P}}\rangle}{D_{x}} \nonumber
\end{eqnarray}

\begin{eqnarray}
	\bangp_{x}{P} & & \nonumber\\
	=
	& {x}!\langle{(\prefix{x}{y}{(\outputp{x}{y} | @{y})) | P}}\rangle 
	      | \prefix{x}{y}{(\outputp{x}{y} | @{y})} & \nonumber\\
	\red
	& (\outputp{x}{y} | @{y})\substn{\quotep{(\prefix{x}{y}{(@{y} | \outputp{x}{y})) | P}}}{y} & \nonumber\\
	=
	& \outputp{x}{\quotep{(\prefix{x}{y}{(\outputp{x}{y} | @{y})) | P}}}
	  | {(\prefix{x}{y}{(\outputp{x}{y} | @{y})) | P}} & \nonumber\\
	\red
	& \ldots & \nonumber\\
	\red^*
	& P | P | \ldots & \nonumber
\end{eqnarray}

Of course, this encoding, as an implementation, runs away, unfolding
$\bangp{P}$ eagerly. A lazier and more implementable replication
operator, restricted to input-guarded processes, may be obtained as follows.

\begin{eqnarray}
\bangp{\prefix{u}{v}{P}} 
	:= 
	\binpar{\lift{x}{\prefix{u}{v}{(\binpar{D(x)}{P})}}}{D(x)} \nonumber
\end{eqnarray}

\begin{remark}
  Note that the lazier definition still does not deal with summation
  or mixed summation (i.e. sums over input and output). The reader is
  invited to construct definitions of replication that deal with these
  features. 

  Further, the definitions are parameterized in a name, $x$. Can you,
  gentle reader, make a definition that eliminates this parameter and
  guarantees no accidental interaction between the replication
  machinery and the process being replicated -- i.e. no accidental
  sharing of names used by the process to get its work done and the
  name(s) used by the replication to effect copying. This latter
  revision of the definition of replication is crucial to obtaining
  the expected identity $!!P \sim !P$.
\end{remark}

\begin{remark}\label{rem:paradoxical_combinator}
  The reader familiar with the lambda calculus will have noticed the
  similarity between $D$ and the paradoxical combinator.

  [Ed. note: the existence of this seems to suggest we have to be more
  restrictive on the set of processes and names we admit if we are to
  support no-cloning.]
\end{remark}

\subsubsection{Bisimulation}

The computational dynamics gives rise to another kind of equivalence,
the equivalence of computational behavior. As previously mentioned
this is typically captured \emph{via} some form of bisimulation.

% The notion we use in this paper is weak barbed bisimulation
% \cite{milner91polyadicpi}.

The notion we use in this paper is derived from weak barbed
bisimulation \cite{milner91polyadicpi}. 

\begin{definition}
An \emph{observation relation}, $\downarrow_{\mathcal N}$, over a set
of names, $\mathcal N$, is the smallest relation satisfying the rules
below.

\infrule[Out-barb]{y \in {\mathcal N}, \; x \nameeq y}
		  {\outputp{x}{v} \downarrow_{\mathcal N} x}
\infrule[Par-barb]{\mbox{$P\downarrow_{\mathcal N} x$ or $Q\downarrow_{\mathcal N} x$}}
		  {\binpar{P}{Q} \downarrow_{\mathcal N} x}

We write $P \Downarrow_{\mathcal N} x$ if there is $Q$ such that 
$P \wred Q$ and $Q \downarrow_{\mathcal N} x$.
\end{definition}

\begin{definition}
%\label{def.bbisim}
An  ${\mathcal N}$-\emph{barbed bisimulation} over a set of names, ${\mathcal N}$, is a symmetric binary relation 
${\mathcal S}_{\mathcal N}$ between agents such that $P\rel{S}_{\mathcal N}Q$ implies:
\begin{enumerate}
\item If $P \red P'$ then $Q \wred Q'$ and $P'\rel{S}_{\mathcal N} Q'$.
\item If $P\downarrow_{\mathcal N} x$, then $Q\Downarrow_{\mathcal N} x$.
\end{enumerate}
$P$ is ${\mathcal N}$-barbed bisimilar to $Q$, written
$P \wbbisim_{\mathcal N} Q$, if $P \rel{S}_{\mathcal N} Q$ for some ${\mathcal N}$-barbed bisimulation ${\mathcal S}_{\mathcal N}$.
\end{definition}

$\mathcal{R} \subseteq \pi \times \pi$

$P \mathcal{R} Q => \forall P'. P \red P' \Rightarrow \exists Q'. Q \red Q', P' \mathcal{R} Q'$

$P \vdash x \Rightarrow Q \vdash x$

\begin{mathpar}
  \inferrule*[lab=Out-barb]{x \nameeq y}{{y}!\langle{Q}\rangle \vdash x}
  \and
  \inferrule*[lab=Par-barb]{\mbox{$P\vdash x$ or $Q\vdash x$}}{\binpar{P}{Q} \vdash x}
\end{mathpar}

\subsubsection{Contexts}

One of the principle advantages of computational calculi like the
$\pi$-calculus is a well-defined notion of context,
contextual-equivalence and a correlation between
contextual-equivalence and notions of bisimulation. The notion of
context allows the decomposition of a process into (sub-)process and
its syntactic environment, its context. Thus, a context may be
thought of as a process with a ``hole'' (written $\Box$) in it. The
application of a context $M$ to a process $P$, written $M[P]$, is
tantamount to filling the hole in $M$ with $P$. In this paper we do
not need the full weight of this theory, but do make use of the notion
of context in the proof the main theorem. 

\begin{mathpar}
  \inferrule* [lab=summation] {} {{M_{M},M_{N}} \bc \Box \;|\; x.M_{A} \;|\; M_{M}+M_{N}}
  \and
  \inferrule* [lab=agent] {} {{M_{A}} \bc (\vec{x})M_{P} \;| \; \clift{P_0,\ldots,M_{P},\ldots,P_N}}
  \and \\
  \inferrule* [lab=process] {} {{M_{P}} \bc M_{N} \;| \;P|M_{P} }
\end{mathpar} 

\begin{mathpar}
  \inferrule* [lab=sychronization] {} {M_{N} \bc \Box \;|\; x?M_{F} \;|\; x!M_{C}}
  \and
  \inferrule* [lab=abstraction] {} {{M_{F}} \bc (x)M_{P} }
  \and
  \inferrule* [lab=concretion] {} {{M_{C}} \bc \langle M_{P} \rangle }
  \and \\
  \inferrule* [lab=process] {} {{M_{P}} \bc M_{N} \;| \;P|M_{P} }
\end{mathpar}

\begin{definition}[contextual application] Given a context $M$, and
  process $P$, we define the \emph{contextual application}, $M[P] :=
  M\{P/\Box\}$. That is, the contextual application of M to P is the
  substitution of $P$ for $\Box$ in $M$.
\end{definition}

$\meaningof{-} : L \to \mathcal{P}(\pi)$

\begin{mathpar}
  \inferrule* [lab=collection] {} {\meaningof{true} = \pi, \and \meaningof{~E} = \pi \setminus \meaningof{E}, \and \meaningof{E_{1} \& E_{2}} = \meaningof{E_{1}} \cap \meaningof{E_{2}}}
\end{mathpar}

\begin{mathpar}
  \inferrule* [lab=structure] {} {\meaningof{0} = \{ P \in \pi | P \equiv 0 \}, \and \\ \meaningof{E_1 | E_2} = \{ P \in \pi | P \equiv P_{1} | P_{2}, P_{1} \in \meaningof{E_{1}}, P_{2} \in \meaningof{E_2}\} }
\end{mathpar}

\begin{mathpar}
 \inferrule* [lab=behavior] {} {\meaningof{\langle a?b \rangle E} = \{ P \in \pi | P \equiv Q | u?(y)P', \\ \and \\\\ \and \\ \;\;\; u \in \meaningof{a}, \forall z.P'\{z/y\} \in \meaningof{E\{z/b\}}\}, \and \\ \meaningof{a!E} = \{ P \in \pi | P \equiv Q | x!\langle P' \rangle, x \in \meaningof{a} P' \in \meaningof{E}\} }
\end{mathpar}

\begin{mathpar}
 \inferrule* [lab=nominal] {} {\meaningof{\quotep{E}} = \{ \quotep{P} \in \quotep{\pi} | P \in \meaningof{E} \}, \and \meaningof{\quotep{P}} = \{ \quotep{Q} \in \quotep{\pi} | P \equiv Q \} \and \\ \meaningof{@\quotep{E}} = \{ P \in \pi | P \equiv @x, x \in \meaningof{E} \}}
\end{mathpar}

\begin{eqnarray*}
  \\
  \meaningof{-} : TS \to ST
\end{eqnarray*}

\begin{eqnarray*}
  \\
  L : TS \to ST
\end{eqnarray*}

\begin{eqnarray*}
  \\
  P \models E \iff P \in \meaningof{E}
\end{eqnarray*}

\begin{eqnarray*}
  P \approx_{L} Q \iff \forall E \in L. P \models E \iff Q \models E
\end{eqnarray*}

\begin{eqnarray*}
  P \approx_{K} Q
\end{eqnarray*}

\begin{eqnarray*}
  P \approx Q
\end{eqnarray*}

$\approx_{K} = \approx = \approx_{L}$

\subsubsection{Contextual duality}

Note that contexts extend the quotation operation to a family of
operations from processes to names. Given a context, $M$, we can
define a \emph{nominal context}, $\quotep{M}$ by $\quotep{M}[P] :=
\quotep{M[P]}$. To foreshadow what is to come we observe that these
operations enjoy a duality with processes very much like the duality
between vectors and maps from vectors to scalars.

Further, because the calculus is essentially higher-order, we have a
correspondence between contexts and processes. More specifically,
given a name $x$ and a context $M$ we can construct $M^{*}_{x}$ such
that 

\begin{mathpar}
  M^{*}_{x} | \lift{x}{P} \red M[P]
\end{mathpar}

namely,

\begin{mathpar}
  M^{*}_{x} := x?(u).M[\dropn{u}]
\end{mathpar}

The dependence of $M^{*}_{x}$ on a name makes it an abstraction, 

\begin{mathpar}
  M^{*} := (x)x?(u).M[\dropn{u}]
\end{mathpar}

\subsection{Additional notation}

It will sometimes be convenient to denote the process a name
quotes. We already have the notation $x = \quotep{P}$, but it will be
convenient to introduce an alternate notation, $\procn{x}$, when we
want to emphasize the connection to the use of the name. Note that, by
virtue of name equivalence, $\quotep{\procn{x}} \nameeq x$; so, the
notation is consistent with previous definitions.

Further, because names have structure it is possible to effect
substitutions on the basis of that structure. This means we need to
upgrade our notation for substitutions, which we accomplish by
adapting comprehension notation. Thus,

\begin{mathpar}
  P\{ y / x : x \in S \}
\end{mathpar}

is interpreted to mean the process derived from P by replacing (in a
capture-avoiding manner) each occurrence of $x$ in $S$ by $y$. For example,

\begin{mathpar}
  P\{ \quotep{\procn{x}|\procn{x}} / x : x \in \freenames{P} \}
\end{mathpar}

will replace each (occurrence) of a free name $x$ in $P$ by
$\quotep{\procn{x}|\procn{x}}$.

Also, we will avail ourselves of the notation $x^{L}$ and $x^{R}$ to
denote injections of a name into disjoint copies of the name
space. There are numerous ways to accomplish this. One example can be
found in \cite{MeredithR05}. This notation overloads to vectors of
names: $\vec{x}^{\pi} := (x_{i}^{\pi} \; : \; 0 \leq i < |\vec{x}| )$ where $\pi \in \{L,R\}$.

We also use $P^{\Box} := P|\Box$.

In \cite{MeredithR05} an interpretation of the new operator is
given. It turns out that there are several possible interpretations
all enjoying the requisite algebraic properties of the operator (see
\cite{milner91polyadicpi}). We will therefore make liberal use of
$(\nu\; \vec{x})P$.

% subsection the_syntax_and_semantics_of_the_notation_system (end)   

\input{qm2pi.qmops} 

\input{qm2pi.sterngerlach} 

\input{qm2pi.metric} 

% section concurrent_process_calculi (end)

%\input{qm2pi.proofsketch}

% section proof sketch (end)

%\input{qm2pi.slviaknots} 

% section spatial logic via knots (end)

\input{qm2pi.conclusion}

% section conclusion (end)

%\input{qm2pi.dtcodes} 

% section wiring algorithm (end)

\input{qm2pi.ack} 

% section acknowledgments (end)

\newpage


\bibliographystyle{plain}   
\bibliography{../../biblios/main.bib}

\input{qm2pi.rhodetails}

\end{document}



% section front matter (end)

\section{Introduction}\label{sec:introduction} % (fold)
In this draft of the material i am going to have to dispense with the
usual writing conventions adopted in papers on these topics. i'm going
to have adopt whatever tone i need at the time i'm writing up the
calculations. Sometimes this may be very conversational; others it may
be the barest mathematical grunts; others still it may be that i have
lifted text from one of my other papers because the exposition of some
point was better said there. i hope that my readers are not unduly put
out by this decision. i'm not doing this to flout convention or be
rebellious. i find these calculations very technically challenging. To
keep everything going technically, something has to give; i have to
let go of some cognitive burden. So, the academic writing style --
with all of its trade-offs in terms of facilitating technical
communication -- is what i'm letting go of. Perhaps subsequent drafts
can be tightened and polished, but for now, i'm going to speak as if
we were sitting together in a coffee shop with a laptop, wifi and a
pad of paper and a pencil.

So, here's what i have to say. We -- you and i, comfortably ensconced
in our coffee shop and well-equipped with our tools -- can realize and
carry out the calculations of quantum mechanics over a very different
formal theory of dynamics, a formal theory of dynamics that
corresponds to a theory of concurrent computation with
\emph{reflection}. It has the advantage that the underlying theory is
already `quantized', but supports analogues all of the continuuous
operations. Strikingly, this underlying theory has recently been
connected with a notion of metric that we can show, by calculating
together, coincides with the metric induced by the inner product.

There are a lot of reasons why you might be interested in seeing
calculations of this form. Here's why i'm interested. For the past
several centuries there has been no competitor to the ``Newtonian''
account of dynamics. As a result the predominant share of accounts of
dynamical systems and situations have had to be formulated in terms of
the Newtonian machinery. i view this as an intellectually dangerous
position to occupy. Everything, despite it's intrinsic shape, turns
into a nail to be hit with this hammer. Recently, however, the theory
of computation has matured to the point where we have candidates for
theories of dynamics that offer very different perspective on
reasoning about dynamical systems and situations. Testing these
candidates against very successful accounts of dynamical situations,
like quantum mechanics, is going to give us some sense of how mature
they are and some measure of the quality of these accounts of
dynamics.

\subsection{Summary of contributions and outline of paper}

So, we're going to develop an interpretation of the operations of
quantum mechanics normally interpreted by Hilbert spaces and
operators. We're going to do this over a theory of computation. Note
that this is very different than the usual quantum computation program
which develops notions of computation over quantum mechanics. Rather,
we are developing a story that aligns with Wheeler's slogan: It from
Bit. To do this we will first provide an account of the theory of
computation at play here. Then we will dive into a calculation-driven
interpretation of the operations of quantum mechanics.

The reason we take this approach is that -- until very recently --
there hasn't been an axiomatic account of quantum mechanics. As a
result there has been no sharp delineation of the mathematical theory
supporting interpretation of the physical theory and the physical
theory, itself. So, ambient features of the maths are free to be
exploited (or supressed) without a real accounting of their physical
relevance. There is no sharp statement ``here's the physical theory''
qua \emph{theory} and ``here's the mathematical interpretation''
enabling a judgment of how faithful the interpretation is -- apart
from experimental observation. When there is an axiomatic account we
can judge how well a given mathematical formalism supports an
interpretation of the axioms, independent of
experimentation. Likewise, we can judge how well we have captured our
physical evidence and experience with our axiomatics, independent of
any specific mathematical implementation, with accidental detail that
may or may not have physical significance. 

In lieu of a fully fleshed out and vetted axiomatic account of quantum
mechanics, interpreting the operational notions in service of modeling
physical systems will have to suffice. In other words, we are not in
the business of providing a model of Hilbert spaces and operators. We
are in the business of providing a model of quantum mechanics because
we are motivated by testing our notions of dynamics against physical
theory; and, the predictive calculations of the physical theory must
serve as the best formulation -- shy of a fully fleshed out axiomatic
account -- of the physical theory itself (as they have for scientific
theories since time immemorial). Put another way, despite a
whole-hearted commitment to an It-from-Bit ontology, we are firmly
aligned with the shut-up-and-calculate camp as the best way to obtain
results either from the physical perspective or as a quality assurance
measure of our fledgling theory of dynamics.

In detail, we present a reflective process calculus. Then we develop
intuitive correspondences between the notions available in this
calculus and the usual physical notions supporting quantum mechanical
calculations. Thus, 

\begin{table}[htp]
  \center{
    \fbox{
      \begin{tabular}{c|c}
        quantum mechanics & process calculus \\
        \hline
        scalar & name \\
        state vector & process \\
        dual & contextual duals \\
        matrix & formal sums of process-context-dual pairs \\
        orthogonality & process annihilation \\
        inner product & execution-formula + quoting
      \end{tabular}
    }
  }
  \caption{QM - process calculi correspondences}
\end{table}

Then we tighten up these intuitions to operational definitions. We
employ the Dirac notation as the best proxy we can find for an
abstract syntax of the quantum mechanical notions. The definitions we
develop put us in contact with equational constraints coming from the
theory that we demonstrate the definitions and calculations satisfy.

This puts us in a position to shut up and calculate for the
Stern-Gerlach experimental set up, showing how these predictive
calculations become calculations on processes in our theory of a
reflective process calculus.

Penultimately, we demonstrate that the notion of metric coming from
the inner product coincides with the notion of metric available from
the theory of bisimulation. This demonstration gives us the right to
think of space as arising from behavior. Finally, we consider where we
might go from the new vantage point we have obtained.

% section introduction (end) 
 
% section introduction (end)

% \documentclass[12pt]{llncs}
%\documentclass{jktr}

\usepackage[pdftex]{hyperref}                   
\usepackage {listings}
\usepackage {mathpartir}
\usepackage{bcprules}
%\usepackage{listings}
                       
\usepackage{graphicx} 
%\usepackage[margins=2.5cm,nohead,nofoot]{geometry}
%\usepackage{geometry}
\usepackage{amsfonts}
\usepackage{amstext}
\usepackage{latexsym}
\usepackage{amssymb}
\usepackage{color}


%\include{myPreamble}
\include{qm2pi.local} 

%\ifpdf
%\usepackage[pdftex]{graphicx}
%\else
%\usepackage{graphicx}
%\fi

 % \ifpdf
%  \usepackage{pdfsync}
%  \if


%\title{Brief Article}
%\author{David F. Snyder}
%\author{L.G. Meredith}

%\address{Dept. of Math., Texas State University--San Marcos, San Marcos, TX 78666}
       
\pagestyle{empty}


\begin{document}

\lstset{language=[Objective]Caml,frame=shadowbox}

\input{qm2pi.front}

% section front matter (end)

\input{qm2pi.intro} 
 
% section introduction (end)

% \input{qm2pi.knotations} 

% section notation (end)

\input{qm2pi.process.calculi} 

% section concurrent_process_calculi_and_spatial_logics_ (end)
    
%\input{qm2pi.knots2pi} 

%\input{qm2pi.trefoil} 

%\input{qm2pi.mainthm} 

% subsection basic_interpretation (end)

%\input{qm2pi.rho.presentation} 
\subsection{The syntax and semantics of the notation system}\label{sub:the_syntax_and_semantics_of_the_notation_system} % (fold)

We now summarize a technical presentation of the calculus that
embodies our theory of dynamics. The typical presentation of such a
calculus follows the style of giving generators and relations on
them. The grammar, below, describing term constructors, freely
generates the set of processes, $\Proc$. This set is then quotiented
by a relation known as structural congruence and it is over this set
that the notion of dynamics is expressed. This presentation is
essentially that of \cite{MeredithR05} with the addition of
polyadicity and summation. For readability we have relegated some of
the technical subtleties to an appendix.

\subsubsection{Process grammar}\label{subsub:process_grammar}

\begin{mathpar}
  \inferrule* [lab=synchronization] {} {{M} \bc \pzero \;|\; x?F \;|\; x!C }
  \and
  \inferrule* [lab=abstraction] {} {{F} \bc (x)P}
  \and
  \inferrule* [lab=concretion] {} {{C} \bc \langle Q \rangle}
  \and
  \inferrule* [lab=process] {} {{P,Q} \bc M \;| \;P|Q \;|\; @{x}}
  \and
  \inferrule* [lab=name] {} {{x} \bc \quotep{P}}
\end{mathpar} 

Note that $\vec{x}$ (resp. $\vec{P}$) denotes a vector of names
(resp. processes) of length $|\vec{x}|$ (resp. $|\vec{P}|$). We adopt
the following useful abbreviations.

\begin{mathpar}
   x?(\vec{y}).P := x.(\vec{y})P \and  x\clift{\vec{P}} := x.\clift{\vec{P}}
   \and x!(y) := \lift{x}{\dropn{y}}
   \and \Pi_{i=0}^{n-1}P_i := P_0 | \ldots | P_{n-1}
\end{mathpar}

\subsubsection{Structural congruence}

\paragraph{Free and bound names and alpha-equivalence.} At the
core of structural equivalence is alpha-equivalence which identifies
process that are the same up to a change of variable. Formally, we
recognize the distinction between free and bound names. The free names
of a process, $\freenames{P}$, may be calculated recursively as
follows:

\begin{mathpar}
\freenames{\pzero} := \emptyset
  \and \\
  \freenames{x?(y).P} := \{ x \} \cup (\freenames{P} \setminus \{ y \})
  \and 
  \freenames{x!\langle P \rangle} := \{ x \} \cup \{ P \} 
  \and \\
  \freenames{P|Q} := \freenames{P} \cup \freenames{Q}
  \and \\
  \freenames{@{x}} := \{ x \}
\end{mathpar}

$\pi$
$\quotep{\pi}$

$\freenames{-} : \pi \to \mathcal{P}(\quotep{\pi})$

\begin{eqnarray*}
  \freenames{\pzero} & := & \emptyset \\
  \freenames{x?(y).P} & := & \{ x \} \cup (\freenames{P} \setminus \{ y \}) \\
  \freenames{x!\langle P \rangle} & := & \{ x \} \cup \{ P \} \\
  \freenames{P|Q} & := & \freenames{P} \cup \freenames{Q} \\
  \freenames{\dropn{x}} & := & \{ x \}
\end{eqnarray*}

The bound names of a process, $\boundnames{P}$, are those names occurring in $P$
that are not free. For example, in $x?(y).0$, the name $x$ is free, while $y$ is bound.

\begin{mathpar}
  \inferrule* [lab=monoidal-laws] {} { P|Q \equiv Q|P \and P|0 \equiv P \and P|(Q|R) \equiv (P|Q)|R }
\end{mathpar}

\begin{mathpar}
  \inferrule* [lab=alpha-equivalence] {} { (x)P \equiv (y)P\{y/x\} \and y \not\in \freenames{P} }
\end{mathpar}

\begin{definition}
Then two processes, $P,Q$, are alpha-equivalent if $P = Q\{\vec{y}/\vec{x}\}$ for
some $\vec{x} \in \boundnames{Q},\vec{y} \in \boundnames{P}$, where $Q\{\vec{y}/\vec{x}\}$
denotes the capture-avoiding substitution of $\vec{y}$ for $\vec{x}$ in $Q$.
\end{definition}

\begin{definition}
  The {\em structural congruence} \cite{SangiorgiWalker} , $\equiv$,
  between processes is the least congruence containing
  alpha-equivalence, satisfying the abelian monoid laws
  (associativity, commutativity and $\pzero$ as identity) for parallel
  composition $|$ and for summation $+$.
\end{definition}

\subsection{Name equivalence}

We take name equivalence, written $\nameeq$, to be the smallest
equivalence relation generated by the following rules.

\begin{mathpar}
\inferrule*[lab=Quote-drop]
{ }
{ \quotep{@{x}} \nameeq x }

\inferrule*[lab=Struct-equiv]
{ P \scong Q }
{ \quotep{P} \nameeq \quotep{Q} }
\end{mathpar}

The astute reader will have noticed that the mutual recursion of names
and processes imposes a mutual recursion on alpha-equivalence and
structural equivalence via name-equivalence. Fortunately, all of this
works out pleasantly and we may calculate in the natural way, free of
concern. The reader interested in the details is referred to the
appendix \ref{appendix:rho_details}.

\subsection{Substitution}

We use $\Proc$ for the set of processes, $\QProc$ for the set of
names, and $\id{\{}\vec{y} / \vec{x} \id{\}}$ to denote partial maps,
$s : \QProc \rightarrow \QProc$. A map, $s$ lifts, uniquely, to a map
on process terms, $\widehat{s} : \Proc \rightarrow \Proc$ by the
following equations.

\begin{mathpar}
  (0) \psubstp{Q}{P} := 0 \\
  (R \juxtap S) \psubstp{Q}{P}
  :=    
  (R)\psubstp{Q}{P} \juxtap (S) \psubstp{Q}{P} \\
  (x?(y).R) \psubstp{Q}{P}    
  :=    
  (x)\substp{Q}{P} (z)\concat( (R \psubstn{z}{y}) \psubstp{Q}{P} ) \\
  (\lift{x}{R}) \psubstp{Q}{P}  
  :=
  \lift{(x)\substp{Q}{P}}{ R \psubstp{Q}{P} } \\
%   (\dropn{x})  \psubstp{Q}{P}       
%   := 
%   \left\{ 
%     \begin{array}{ccc} 
%       \dropn{\quotep{Q}} & & x \nameeq \quotep{P} \\
%       \dropn{x} & & otherwise \\
%     \end{array}
%   \right. 
  (\dropn{x})  \psubstp{Q}{P}       
  := 
  \left\{ 
    \begin{array}{ccc} 
      Q & & x \nameeq \quotep{P} \\
      \dropn{x} & & otherwise \\
    \end{array}
  \right.
\end{mathpar}
 

where

\begin{eqnarray}
  (x)\id{\{} \lpquote Q \rpquote / \lpquote P \rpquote \id{\}}            = 
  \left\{ 
    \begin{array}{ccc}
      \lpquote Q \rpquote & & x \nameeq \lpquote P \rpquote \\
      x & & otherwise \\
    \end{array}
  \right. \nonumber
\end{eqnarray}

and $z$ is chosen distinct from $\quotep{P}$, $\quotep{Q}$, the free
names in $Q$, and all the names in $R$. Our $\alpha$-equivalence will
be built in the standard way from this substitution.

\begin{remark}\label{rem:no_self_referential_names}
  One consequence of these definitions is that $\forall P. \quotep{P}
  \not\in \freenames{P}$.
\end{remark}

\subsection{ Dynamic quote: an example }

Anticipating something of what's to come, consider applying the
substitution, $\widehat{\id{\{}u / z \id{\}}}$, to the following pair
of processes, $\lift{w}{y!(z)}$ and $w[ \lpquote y!(z) \rpquote ]$.

\begin{eqnarray}
	\lift{w}{y!(z)}\widehat{\id{\{}u / z \id{\}}}
		& = &
		\lift{w}{y!(u)} \nonumber\\
	w[ \lpquote y!(z) \rpquote ] \widehat{ \id{\{}u / z \id{\}} }
		& = &
		w[ \lpquote y!(z) \rpquote ] \nonumber
\end{eqnarray}

Because the body of the process between quotes is impervious to
substitution, we get radically different answers. In fact, by
examining the first process in an input context,
e.g. $x?(z).\lift{w}{y!(z)}$, we see that the process under the lift
operator may be shaped by prefixed inputs binding a name inside it. In
this sense, the lift operator will be seen as a way to dynamically
construct processes before reifying them as names.

Finally equipped with these standard features we can present the
dynamics of the calculus.

\subsubsection{Operational semantics} 

Finally, we introduce the computational dynamics. What marks these
algebras as distinct from other more traditionally studied algebraic
structures, e.g. vector spaces or polynomial rings, is the manner in
which dynamics is captured. In traditional structures, dynamics is typically
expressed through morphisms between such structures, as in linear maps
between vector spaces or morphisms between rings. In algebras
associated with the semantics of computation, the dynamics is
expressed as part of the algebraic structure itself, through a
reduction reduction relation typically denoted by $\red$. Below, we
give a recursive presentation of this relation for the calculus used
in the encoding.

$\red \subseteq \pi \times \pi$
$\red : \pi \to \mathcal{P}(\pi)$

\begin{mathpar}
  \inferrule* [lab=Comm] { \textsf{match}( x_{src}, x_{trgt} ) } { x_{trgt}?(y)P \; | \; x_{src}!\langle {Q} \rangle \red P\{\quotep{Q}/y}\} }
  \and \\
  \inferrule* [lab=Par] {{P} \red {P}'} {{{P} | {Q}} \red {{P}' | {Q}}}
  \and
  \inferrule* [lab=Equiv]{{{P} \scong {P}'} \andalso {{P}' \red {Q}'} \andalso {{Q}' \scong {Q}}}{{P} \red {Q}}
\end{mathpar}

\begin{eqnarray*}
  match_{\equiv} (\quotep{P},\quotep{Q}) & := & P \equiv Q \\
  match_{\dagger}(\quotep{P},\quotep{Q}) & := & \forall R. P|Q \red^{*} R => R \red^{*} 0 \\
  match_{K}(\quotep{P},\quotep{Q}) & := & K \mbox{ for some context } K
\end{eqnarray*}

$u?(x)P | u!\langle Q \rangle \red P\{\quotep{Q}/x\}$

%We write $\wred$ for $\red^*$, and $P\red$ if $\exists Q $ such that $ P \red Q$.
We write $P\red$ if $\exists Q $ such that $ P \red Q$ and $P\not\red$, otherwise.

\section{Replication}

As mentioned before, it is known that replication (and hence
recursion) can be implemented in a higher-order process algebra
\cite{SangiorgiWalker}. As our first example of calculation with the
machinery thus far presented we give the construction explicitly in
the {\rhoc}.

\begin{eqnarray}
	D_{x} & := & \prefix{x}{y}{(\binpar{\outputp{x}{y}}{@{y}})} \nonumber\\
	\bangp_{x}{P} & := & \binpar{{x}!\langle{\binpar{D_{x}}{P}}\rangle}{D_{x}} \nonumber
\end{eqnarray}

\begin{eqnarray}
	\bangp_{x}{P} & & \nonumber\\
	=
	& {x}!\langle{(\prefix{x}{y}{(\outputp{x}{y} | @{y})) | P}}\rangle 
	      | \prefix{x}{y}{(\outputp{x}{y} | @{y})} & \nonumber\\
	\red
	& (\outputp{x}{y} | @{y})\substn{\quotep{(\prefix{x}{y}{(@{y} | \outputp{x}{y})) | P}}}{y} & \nonumber\\
	=
	& \outputp{x}{\quotep{(\prefix{x}{y}{(\outputp{x}{y} | @{y})) | P}}}
	  | {(\prefix{x}{y}{(\outputp{x}{y} | @{y})) | P}} & \nonumber\\
	\red
	& \ldots & \nonumber\\
	\red^*
	& P | P | \ldots & \nonumber
\end{eqnarray}

Of course, this encoding, as an implementation, runs away, unfolding
$\bangp{P}$ eagerly. A lazier and more implementable replication
operator, restricted to input-guarded processes, may be obtained as follows.

\begin{eqnarray}
\bangp{\prefix{u}{v}{P}} 
	:= 
	\binpar{\lift{x}{\prefix{u}{v}{(\binpar{D(x)}{P})}}}{D(x)} \nonumber
\end{eqnarray}

\begin{remark}
  Note that the lazier definition still does not deal with summation
  or mixed summation (i.e. sums over input and output). The reader is
  invited to construct definitions of replication that deal with these
  features. 

  Further, the definitions are parameterized in a name, $x$. Can you,
  gentle reader, make a definition that eliminates this parameter and
  guarantees no accidental interaction between the replication
  machinery and the process being replicated -- i.e. no accidental
  sharing of names used by the process to get its work done and the
  name(s) used by the replication to effect copying. This latter
  revision of the definition of replication is crucial to obtaining
  the expected identity $!!P \sim !P$.
\end{remark}

\begin{remark}\label{rem:paradoxical_combinator}
  The reader familiar with the lambda calculus will have noticed the
  similarity between $D$ and the paradoxical combinator.

  [Ed. note: the existence of this seems to suggest we have to be more
  restrictive on the set of processes and names we admit if we are to
  support no-cloning.]
\end{remark}

\subsubsection{Bisimulation}

The computational dynamics gives rise to another kind of equivalence,
the equivalence of computational behavior. As previously mentioned
this is typically captured \emph{via} some form of bisimulation.

% The notion we use in this paper is weak barbed bisimulation
% \cite{milner91polyadicpi}.

The notion we use in this paper is derived from weak barbed
bisimulation \cite{milner91polyadicpi}. 

\begin{definition}
An \emph{observation relation}, $\downarrow_{\mathcal N}$, over a set
of names, $\mathcal N$, is the smallest relation satisfying the rules
below.

\infrule[Out-barb]{y \in {\mathcal N}, \; x \nameeq y}
		  {\outputp{x}{v} \downarrow_{\mathcal N} x}
\infrule[Par-barb]{\mbox{$P\downarrow_{\mathcal N} x$ or $Q\downarrow_{\mathcal N} x$}}
		  {\binpar{P}{Q} \downarrow_{\mathcal N} x}

We write $P \Downarrow_{\mathcal N} x$ if there is $Q$ such that 
$P \wred Q$ and $Q \downarrow_{\mathcal N} x$.
\end{definition}

\begin{definition}
%\label{def.bbisim}
An  ${\mathcal N}$-\emph{barbed bisimulation} over a set of names, ${\mathcal N}$, is a symmetric binary relation 
${\mathcal S}_{\mathcal N}$ between agents such that $P\rel{S}_{\mathcal N}Q$ implies:
\begin{enumerate}
\item If $P \red P'$ then $Q \wred Q'$ and $P'\rel{S}_{\mathcal N} Q'$.
\item If $P\downarrow_{\mathcal N} x$, then $Q\Downarrow_{\mathcal N} x$.
\end{enumerate}
$P$ is ${\mathcal N}$-barbed bisimilar to $Q$, written
$P \wbbisim_{\mathcal N} Q$, if $P \rel{S}_{\mathcal N} Q$ for some ${\mathcal N}$-barbed bisimulation ${\mathcal S}_{\mathcal N}$.
\end{definition}

$\mathcal{R} \subseteq \pi \times \pi$

$P \mathcal{R} Q => \forall P'. P \red P' \Rightarrow \exists Q'. Q \red Q', P' \mathcal{R} Q'$

$P \vdash x \Rightarrow Q \vdash x$

\begin{mathpar}
  \inferrule*[lab=Out-barb]{x \nameeq y}{{y}!\langle{Q}\rangle \vdash x}
  \and
  \inferrule*[lab=Par-barb]{\mbox{$P\vdash x$ or $Q\vdash x$}}{\binpar{P}{Q} \vdash x}
\end{mathpar}

\subsubsection{Contexts}

One of the principle advantages of computational calculi like the
$\pi$-calculus is a well-defined notion of context,
contextual-equivalence and a correlation between
contextual-equivalence and notions of bisimulation. The notion of
context allows the decomposition of a process into (sub-)process and
its syntactic environment, its context. Thus, a context may be
thought of as a process with a ``hole'' (written $\Box$) in it. The
application of a context $M$ to a process $P$, written $M[P]$, is
tantamount to filling the hole in $M$ with $P$. In this paper we do
not need the full weight of this theory, but do make use of the notion
of context in the proof the main theorem. 

\begin{mathpar}
  \inferrule* [lab=summation] {} {{M_{M},M_{N}} \bc \Box \;|\; x.M_{A} \;|\; M_{M}+M_{N}}
  \and
  \inferrule* [lab=agent] {} {{M_{A}} \bc (\vec{x})M_{P} \;| \; \clift{P_0,\ldots,M_{P},\ldots,P_N}}
  \and \\
  \inferrule* [lab=process] {} {{M_{P}} \bc M_{N} \;| \;P|M_{P} }
\end{mathpar} 

\begin{mathpar}
  \inferrule* [lab=sychronization] {} {M_{N} \bc \Box \;|\; x?M_{F} \;|\; x!M_{C}}
  \and
  \inferrule* [lab=abstraction] {} {{M_{F}} \bc (x)M_{P} }
  \and
  \inferrule* [lab=concretion] {} {{M_{C}} \bc \langle M_{P} \rangle }
  \and \\
  \inferrule* [lab=process] {} {{M_{P}} \bc M_{N} \;| \;P|M_{P} }
\end{mathpar}

\begin{definition}[contextual application] Given a context $M$, and
  process $P$, we define the \emph{contextual application}, $M[P] :=
  M\{P/\Box\}$. That is, the contextual application of M to P is the
  substitution of $P$ for $\Box$ in $M$.
\end{definition}

$\meaningof{-} : L \to \mathcal{P}(\pi)$

\begin{mathpar}
  \inferrule* [lab=collection] {} {\meaningof{true} = \pi, \and \meaningof{~E} = \pi \setminus \meaningof{E}, \and \meaningof{E_{1} \& E_{2}} = \meaningof{E_{1}} \cap \meaningof{E_{2}}}
\end{mathpar}

\begin{mathpar}
  \inferrule* [lab=structure] {} {\meaningof{0} = \{ P \in \pi | P \equiv 0 \}, \and \\ \meaningof{E_1 | E_2} = \{ P \in \pi | P \equiv P_{1} | P_{2}, P_{1} \in \meaningof{E_{1}}, P_{2} \in \meaningof{E_2}\} }
\end{mathpar}

\begin{mathpar}
 \inferrule* [lab=behavior] {} {\meaningof{\langle a?b \rangle E} = \{ P \in \pi | P \equiv Q | u?(y)P', \\ \and \\\\ \and \\ \;\;\; u \in \meaningof{a}, \forall z.P'\{z/y\} \in \meaningof{E\{z/b\}}\}, \and \\ \meaningof{a!E} = \{ P \in \pi | P \equiv Q | x!\langle P' \rangle, x \in \meaningof{a} P' \in \meaningof{E}\} }
\end{mathpar}

\begin{mathpar}
 \inferrule* [lab=nominal] {} {\meaningof{\quotep{E}} = \{ \quotep{P} \in \quotep{\pi} | P \in \meaningof{E} \}, \and \meaningof{\quotep{P}} = \{ \quotep{Q} \in \quotep{\pi} | P \equiv Q \} \and \\ \meaningof{@\quotep{E}} = \{ P \in \pi | P \equiv @x, x \in \meaningof{E} \}}
\end{mathpar}

\begin{eqnarray*}
  \\
  \meaningof{-} : TS \to ST
\end{eqnarray*}

\begin{eqnarray*}
  \\
  L : TS \to ST
\end{eqnarray*}

\begin{eqnarray*}
  \\
  P \models E \iff P \in \meaningof{E}
\end{eqnarray*}

\begin{eqnarray*}
  P \approx_{L} Q \iff \forall E \in L. P \models E \iff Q \models E
\end{eqnarray*}

\begin{eqnarray*}
  P \approx_{K} Q
\end{eqnarray*}

\begin{eqnarray*}
  P \approx Q
\end{eqnarray*}

$\approx_{K} = \approx = \approx_{L}$

\subsubsection{Contextual duality}

Note that contexts extend the quotation operation to a family of
operations from processes to names. Given a context, $M$, we can
define a \emph{nominal context}, $\quotep{M}$ by $\quotep{M}[P] :=
\quotep{M[P]}$. To foreshadow what is to come we observe that these
operations enjoy a duality with processes very much like the duality
between vectors and maps from vectors to scalars.

Further, because the calculus is essentially higher-order, we have a
correspondence between contexts and processes. More specifically,
given a name $x$ and a context $M$ we can construct $M^{*}_{x}$ such
that 

\begin{mathpar}
  M^{*}_{x} | \lift{x}{P} \red M[P]
\end{mathpar}

namely,

\begin{mathpar}
  M^{*}_{x} := x?(u).M[\dropn{u}]
\end{mathpar}

The dependence of $M^{*}_{x}$ on a name makes it an abstraction, 

\begin{mathpar}
  M^{*} := (x)x?(u).M[\dropn{u}]
\end{mathpar}

\subsection{Additional notation}

It will sometimes be convenient to denote the process a name
quotes. We already have the notation $x = \quotep{P}$, but it will be
convenient to introduce an alternate notation, $\procn{x}$, when we
want to emphasize the connection to the use of the name. Note that, by
virtue of name equivalence, $\quotep{\procn{x}} \nameeq x$; so, the
notation is consistent with previous definitions.

Further, because names have structure it is possible to effect
substitutions on the basis of that structure. This means we need to
upgrade our notation for substitutions, which we accomplish by
adapting comprehension notation. Thus,

\begin{mathpar}
  P\{ y / x : x \in S \}
\end{mathpar}

is interpreted to mean the process derived from P by replacing (in a
capture-avoiding manner) each occurrence of $x$ in $S$ by $y$. For example,

\begin{mathpar}
  P\{ \quotep{\procn{x}|\procn{x}} / x : x \in \freenames{P} \}
\end{mathpar}

will replace each (occurrence) of a free name $x$ in $P$ by
$\quotep{\procn{x}|\procn{x}}$.

Also, we will avail ourselves of the notation $x^{L}$ and $x^{R}$ to
denote injections of a name into disjoint copies of the name
space. There are numerous ways to accomplish this. One example can be
found in \cite{MeredithR05}. This notation overloads to vectors of
names: $\vec{x}^{\pi} := (x_{i}^{\pi} \; : \; 0 \leq i < |\vec{x}| )$ where $\pi \in \{L,R\}$.

We also use $P^{\Box} := P|\Box$.

In \cite{MeredithR05} an interpretation of the new operator is
given. It turns out that there are several possible interpretations
all enjoying the requisite algebraic properties of the operator (see
\cite{milner91polyadicpi}). We will therefore make liberal use of
$(\nu\; \vec{x})P$.

% subsection the_syntax_and_semantics_of_the_notation_system (end)   

\input{qm2pi.qmops} 

\input{qm2pi.sterngerlach} 

\input{qm2pi.metric} 

% section concurrent_process_calculi (end)

%\input{qm2pi.proofsketch}

% section proof sketch (end)

%\input{qm2pi.slviaknots} 

% section spatial logic via knots (end)

\input{qm2pi.conclusion}

% section conclusion (end)

%\input{qm2pi.dtcodes} 

% section wiring algorithm (end)

\input{qm2pi.ack} 

% section acknowledgments (end)

\newpage


\bibliographystyle{plain}   
\bibliography{../../biblios/main.bib}

\input{qm2pi.rhodetails}

\end{document}

 

% section notation (end)

\input{qm2pi.process.calculi} 

% section concurrent_process_calculi_and_spatial_logics_ (end)
    
%\documentclass[12pt]{llncs}
%\documentclass{jktr}

\usepackage[pdftex]{hyperref}                   
\usepackage {listings}
\usepackage {mathpartir}
\usepackage{bcprules}
%\usepackage{listings}
                       
\usepackage{graphicx} 
%\usepackage[margins=2.5cm,nohead,nofoot]{geometry}
%\usepackage{geometry}
\usepackage{amsfonts}
\usepackage{amstext}
\usepackage{latexsym}
\usepackage{amssymb}
\usepackage{color}


%\include{myPreamble}
\include{qm2pi.local} 

%\ifpdf
%\usepackage[pdftex]{graphicx}
%\else
%\usepackage{graphicx}
%\fi

 % \ifpdf
%  \usepackage{pdfsync}
%  \if


%\title{Brief Article}
%\author{David F. Snyder}
%\author{L.G. Meredith}

%\address{Dept. of Math., Texas State University--San Marcos, San Marcos, TX 78666}
       
\pagestyle{empty}


\begin{document}

\lstset{language=[Objective]Caml,frame=shadowbox}

\input{qm2pi.front}

% section front matter (end)

\input{qm2pi.intro} 
 
% section introduction (end)

% \input{qm2pi.knotations} 

% section notation (end)

\input{qm2pi.process.calculi} 

% section concurrent_process_calculi_and_spatial_logics_ (end)
    
%\input{qm2pi.knots2pi} 

%\input{qm2pi.trefoil} 

%\input{qm2pi.mainthm} 

% subsection basic_interpretation (end)

%\input{qm2pi.rho.presentation} 
\subsection{The syntax and semantics of the notation system}\label{sub:the_syntax_and_semantics_of_the_notation_system} % (fold)

We now summarize a technical presentation of the calculus that
embodies our theory of dynamics. The typical presentation of such a
calculus follows the style of giving generators and relations on
them. The grammar, below, describing term constructors, freely
generates the set of processes, $\Proc$. This set is then quotiented
by a relation known as structural congruence and it is over this set
that the notion of dynamics is expressed. This presentation is
essentially that of \cite{MeredithR05} with the addition of
polyadicity and summation. For readability we have relegated some of
the technical subtleties to an appendix.

\subsubsection{Process grammar}\label{subsub:process_grammar}

\begin{mathpar}
  \inferrule* [lab=synchronization] {} {{M} \bc \pzero \;|\; x?F \;|\; x!C }
  \and
  \inferrule* [lab=abstraction] {} {{F} \bc (x)P}
  \and
  \inferrule* [lab=concretion] {} {{C} \bc \langle Q \rangle}
  \and
  \inferrule* [lab=process] {} {{P,Q} \bc M \;| \;P|Q \;|\; @{x}}
  \and
  \inferrule* [lab=name] {} {{x} \bc \quotep{P}}
\end{mathpar} 

Note that $\vec{x}$ (resp. $\vec{P}$) denotes a vector of names
(resp. processes) of length $|\vec{x}|$ (resp. $|\vec{P}|$). We adopt
the following useful abbreviations.

\begin{mathpar}
   x?(\vec{y}).P := x.(\vec{y})P \and  x\clift{\vec{P}} := x.\clift{\vec{P}}
   \and x!(y) := \lift{x}{\dropn{y}}
   \and \Pi_{i=0}^{n-1}P_i := P_0 | \ldots | P_{n-1}
\end{mathpar}

\subsubsection{Structural congruence}

\paragraph{Free and bound names and alpha-equivalence.} At the
core of structural equivalence is alpha-equivalence which identifies
process that are the same up to a change of variable. Formally, we
recognize the distinction between free and bound names. The free names
of a process, $\freenames{P}$, may be calculated recursively as
follows:

\begin{mathpar}
\freenames{\pzero} := \emptyset
  \and \\
  \freenames{x?(y).P} := \{ x \} \cup (\freenames{P} \setminus \{ y \})
  \and 
  \freenames{x!\langle P \rangle} := \{ x \} \cup \{ P \} 
  \and \\
  \freenames{P|Q} := \freenames{P} \cup \freenames{Q}
  \and \\
  \freenames{@{x}} := \{ x \}
\end{mathpar}

$\pi$
$\quotep{\pi}$

$\freenames{-} : \pi \to \mathcal{P}(\quotep{\pi})$

\begin{eqnarray*}
  \freenames{\pzero} & := & \emptyset \\
  \freenames{x?(y).P} & := & \{ x \} \cup (\freenames{P} \setminus \{ y \}) \\
  \freenames{x!\langle P \rangle} & := & \{ x \} \cup \{ P \} \\
  \freenames{P|Q} & := & \freenames{P} \cup \freenames{Q} \\
  \freenames{\dropn{x}} & := & \{ x \}
\end{eqnarray*}

The bound names of a process, $\boundnames{P}$, are those names occurring in $P$
that are not free. For example, in $x?(y).0$, the name $x$ is free, while $y$ is bound.

\begin{mathpar}
  \inferrule* [lab=monoidal-laws] {} { P|Q \equiv Q|P \and P|0 \equiv P \and P|(Q|R) \equiv (P|Q)|R }
\end{mathpar}

\begin{mathpar}
  \inferrule* [lab=alpha-equivalence] {} { (x)P \equiv (y)P\{y/x\} \and y \not\in \freenames{P} }
\end{mathpar}

\begin{definition}
Then two processes, $P,Q$, are alpha-equivalent if $P = Q\{\vec{y}/\vec{x}\}$ for
some $\vec{x} \in \boundnames{Q},\vec{y} \in \boundnames{P}$, where $Q\{\vec{y}/\vec{x}\}$
denotes the capture-avoiding substitution of $\vec{y}$ for $\vec{x}$ in $Q$.
\end{definition}

\begin{definition}
  The {\em structural congruence} \cite{SangiorgiWalker} , $\equiv$,
  between processes is the least congruence containing
  alpha-equivalence, satisfying the abelian monoid laws
  (associativity, commutativity and $\pzero$ as identity) for parallel
  composition $|$ and for summation $+$.
\end{definition}

\subsection{Name equivalence}

We take name equivalence, written $\nameeq$, to be the smallest
equivalence relation generated by the following rules.

\begin{mathpar}
\inferrule*[lab=Quote-drop]
{ }
{ \quotep{@{x}} \nameeq x }

\inferrule*[lab=Struct-equiv]
{ P \scong Q }
{ \quotep{P} \nameeq \quotep{Q} }
\end{mathpar}

The astute reader will have noticed that the mutual recursion of names
and processes imposes a mutual recursion on alpha-equivalence and
structural equivalence via name-equivalence. Fortunately, all of this
works out pleasantly and we may calculate in the natural way, free of
concern. The reader interested in the details is referred to the
appendix \ref{appendix:rho_details}.

\subsection{Substitution}

We use $\Proc$ for the set of processes, $\QProc$ for the set of
names, and $\id{\{}\vec{y} / \vec{x} \id{\}}$ to denote partial maps,
$s : \QProc \rightarrow \QProc$. A map, $s$ lifts, uniquely, to a map
on process terms, $\widehat{s} : \Proc \rightarrow \Proc$ by the
following equations.

\begin{mathpar}
  (0) \psubstp{Q}{P} := 0 \\
  (R \juxtap S) \psubstp{Q}{P}
  :=    
  (R)\psubstp{Q}{P} \juxtap (S) \psubstp{Q}{P} \\
  (x?(y).R) \psubstp{Q}{P}    
  :=    
  (x)\substp{Q}{P} (z)\concat( (R \psubstn{z}{y}) \psubstp{Q}{P} ) \\
  (\lift{x}{R}) \psubstp{Q}{P}  
  :=
  \lift{(x)\substp{Q}{P}}{ R \psubstp{Q}{P} } \\
%   (\dropn{x})  \psubstp{Q}{P}       
%   := 
%   \left\{ 
%     \begin{array}{ccc} 
%       \dropn{\quotep{Q}} & & x \nameeq \quotep{P} \\
%       \dropn{x} & & otherwise \\
%     \end{array}
%   \right. 
  (\dropn{x})  \psubstp{Q}{P}       
  := 
  \left\{ 
    \begin{array}{ccc} 
      Q & & x \nameeq \quotep{P} \\
      \dropn{x} & & otherwise \\
    \end{array}
  \right.
\end{mathpar}
 

where

\begin{eqnarray}
  (x)\id{\{} \lpquote Q \rpquote / \lpquote P \rpquote \id{\}}            = 
  \left\{ 
    \begin{array}{ccc}
      \lpquote Q \rpquote & & x \nameeq \lpquote P \rpquote \\
      x & & otherwise \\
    \end{array}
  \right. \nonumber
\end{eqnarray}

and $z$ is chosen distinct from $\quotep{P}$, $\quotep{Q}$, the free
names in $Q$, and all the names in $R$. Our $\alpha$-equivalence will
be built in the standard way from this substitution.

\begin{remark}\label{rem:no_self_referential_names}
  One consequence of these definitions is that $\forall P. \quotep{P}
  \not\in \freenames{P}$.
\end{remark}

\subsection{ Dynamic quote: an example }

Anticipating something of what's to come, consider applying the
substitution, $\widehat{\id{\{}u / z \id{\}}}$, to the following pair
of processes, $\lift{w}{y!(z)}$ and $w[ \lpquote y!(z) \rpquote ]$.

\begin{eqnarray}
	\lift{w}{y!(z)}\widehat{\id{\{}u / z \id{\}}}
		& = &
		\lift{w}{y!(u)} \nonumber\\
	w[ \lpquote y!(z) \rpquote ] \widehat{ \id{\{}u / z \id{\}} }
		& = &
		w[ \lpquote y!(z) \rpquote ] \nonumber
\end{eqnarray}

Because the body of the process between quotes is impervious to
substitution, we get radically different answers. In fact, by
examining the first process in an input context,
e.g. $x?(z).\lift{w}{y!(z)}$, we see that the process under the lift
operator may be shaped by prefixed inputs binding a name inside it. In
this sense, the lift operator will be seen as a way to dynamically
construct processes before reifying them as names.

Finally equipped with these standard features we can present the
dynamics of the calculus.

\subsubsection{Operational semantics} 

Finally, we introduce the computational dynamics. What marks these
algebras as distinct from other more traditionally studied algebraic
structures, e.g. vector spaces or polynomial rings, is the manner in
which dynamics is captured. In traditional structures, dynamics is typically
expressed through morphisms between such structures, as in linear maps
between vector spaces or morphisms between rings. In algebras
associated with the semantics of computation, the dynamics is
expressed as part of the algebraic structure itself, through a
reduction reduction relation typically denoted by $\red$. Below, we
give a recursive presentation of this relation for the calculus used
in the encoding.

$\red \subseteq \pi \times \pi$
$\red : \pi \to \mathcal{P}(\pi)$

\begin{mathpar}
  \inferrule* [lab=Comm] { \textsf{match}( x_{src}, x_{trgt} ) } { x_{trgt}?(y)P \; | \; x_{src}!\langle {Q} \rangle \red P\{\quotep{Q}/y}\} }
  \and \\
  \inferrule* [lab=Par] {{P} \red {P}'} {{{P} | {Q}} \red {{P}' | {Q}}}
  \and
  \inferrule* [lab=Equiv]{{{P} \scong {P}'} \andalso {{P}' \red {Q}'} \andalso {{Q}' \scong {Q}}}{{P} \red {Q}}
\end{mathpar}

\begin{eqnarray*}
  match_{\equiv} (\quotep{P},\quotep{Q}) & := & P \equiv Q \\
  match_{\dagger}(\quotep{P},\quotep{Q}) & := & \forall R. P|Q \red^{*} R => R \red^{*} 0 \\
  match_{K}(\quotep{P},\quotep{Q}) & := & K \mbox{ for some context } K
\end{eqnarray*}

$u?(x)P | u!\langle Q \rangle \red P\{\quotep{Q}/x\}$

%We write $\wred$ for $\red^*$, and $P\red$ if $\exists Q $ such that $ P \red Q$.
We write $P\red$ if $\exists Q $ such that $ P \red Q$ and $P\not\red$, otherwise.

\section{Replication}

As mentioned before, it is known that replication (and hence
recursion) can be implemented in a higher-order process algebra
\cite{SangiorgiWalker}. As our first example of calculation with the
machinery thus far presented we give the construction explicitly in
the {\rhoc}.

\begin{eqnarray}
	D_{x} & := & \prefix{x}{y}{(\binpar{\outputp{x}{y}}{@{y}})} \nonumber\\
	\bangp_{x}{P} & := & \binpar{{x}!\langle{\binpar{D_{x}}{P}}\rangle}{D_{x}} \nonumber
\end{eqnarray}

\begin{eqnarray}
	\bangp_{x}{P} & & \nonumber\\
	=
	& {x}!\langle{(\prefix{x}{y}{(\outputp{x}{y} | @{y})) | P}}\rangle 
	      | \prefix{x}{y}{(\outputp{x}{y} | @{y})} & \nonumber\\
	\red
	& (\outputp{x}{y} | @{y})\substn{\quotep{(\prefix{x}{y}{(@{y} | \outputp{x}{y})) | P}}}{y} & \nonumber\\
	=
	& \outputp{x}{\quotep{(\prefix{x}{y}{(\outputp{x}{y} | @{y})) | P}}}
	  | {(\prefix{x}{y}{(\outputp{x}{y} | @{y})) | P}} & \nonumber\\
	\red
	& \ldots & \nonumber\\
	\red^*
	& P | P | \ldots & \nonumber
\end{eqnarray}

Of course, this encoding, as an implementation, runs away, unfolding
$\bangp{P}$ eagerly. A lazier and more implementable replication
operator, restricted to input-guarded processes, may be obtained as follows.

\begin{eqnarray}
\bangp{\prefix{u}{v}{P}} 
	:= 
	\binpar{\lift{x}{\prefix{u}{v}{(\binpar{D(x)}{P})}}}{D(x)} \nonumber
\end{eqnarray}

\begin{remark}
  Note that the lazier definition still does not deal with summation
  or mixed summation (i.e. sums over input and output). The reader is
  invited to construct definitions of replication that deal with these
  features. 

  Further, the definitions are parameterized in a name, $x$. Can you,
  gentle reader, make a definition that eliminates this parameter and
  guarantees no accidental interaction between the replication
  machinery and the process being replicated -- i.e. no accidental
  sharing of names used by the process to get its work done and the
  name(s) used by the replication to effect copying. This latter
  revision of the definition of replication is crucial to obtaining
  the expected identity $!!P \sim !P$.
\end{remark}

\begin{remark}\label{rem:paradoxical_combinator}
  The reader familiar with the lambda calculus will have noticed the
  similarity between $D$ and the paradoxical combinator.

  [Ed. note: the existence of this seems to suggest we have to be more
  restrictive on the set of processes and names we admit if we are to
  support no-cloning.]
\end{remark}

\subsubsection{Bisimulation}

The computational dynamics gives rise to another kind of equivalence,
the equivalence of computational behavior. As previously mentioned
this is typically captured \emph{via} some form of bisimulation.

% The notion we use in this paper is weak barbed bisimulation
% \cite{milner91polyadicpi}.

The notion we use in this paper is derived from weak barbed
bisimulation \cite{milner91polyadicpi}. 

\begin{definition}
An \emph{observation relation}, $\downarrow_{\mathcal N}$, over a set
of names, $\mathcal N$, is the smallest relation satisfying the rules
below.

\infrule[Out-barb]{y \in {\mathcal N}, \; x \nameeq y}
		  {\outputp{x}{v} \downarrow_{\mathcal N} x}
\infrule[Par-barb]{\mbox{$P\downarrow_{\mathcal N} x$ or $Q\downarrow_{\mathcal N} x$}}
		  {\binpar{P}{Q} \downarrow_{\mathcal N} x}

We write $P \Downarrow_{\mathcal N} x$ if there is $Q$ such that 
$P \wred Q$ and $Q \downarrow_{\mathcal N} x$.
\end{definition}

\begin{definition}
%\label{def.bbisim}
An  ${\mathcal N}$-\emph{barbed bisimulation} over a set of names, ${\mathcal N}$, is a symmetric binary relation 
${\mathcal S}_{\mathcal N}$ between agents such that $P\rel{S}_{\mathcal N}Q$ implies:
\begin{enumerate}
\item If $P \red P'$ then $Q \wred Q'$ and $P'\rel{S}_{\mathcal N} Q'$.
\item If $P\downarrow_{\mathcal N} x$, then $Q\Downarrow_{\mathcal N} x$.
\end{enumerate}
$P$ is ${\mathcal N}$-barbed bisimilar to $Q$, written
$P \wbbisim_{\mathcal N} Q$, if $P \rel{S}_{\mathcal N} Q$ for some ${\mathcal N}$-barbed bisimulation ${\mathcal S}_{\mathcal N}$.
\end{definition}

$\mathcal{R} \subseteq \pi \times \pi$

$P \mathcal{R} Q => \forall P'. P \red P' \Rightarrow \exists Q'. Q \red Q', P' \mathcal{R} Q'$

$P \vdash x \Rightarrow Q \vdash x$

\begin{mathpar}
  \inferrule*[lab=Out-barb]{x \nameeq y}{{y}!\langle{Q}\rangle \vdash x}
  \and
  \inferrule*[lab=Par-barb]{\mbox{$P\vdash x$ or $Q\vdash x$}}{\binpar{P}{Q} \vdash x}
\end{mathpar}

\subsubsection{Contexts}

One of the principle advantages of computational calculi like the
$\pi$-calculus is a well-defined notion of context,
contextual-equivalence and a correlation between
contextual-equivalence and notions of bisimulation. The notion of
context allows the decomposition of a process into (sub-)process and
its syntactic environment, its context. Thus, a context may be
thought of as a process with a ``hole'' (written $\Box$) in it. The
application of a context $M$ to a process $P$, written $M[P]$, is
tantamount to filling the hole in $M$ with $P$. In this paper we do
not need the full weight of this theory, but do make use of the notion
of context in the proof the main theorem. 

\begin{mathpar}
  \inferrule* [lab=summation] {} {{M_{M},M_{N}} \bc \Box \;|\; x.M_{A} \;|\; M_{M}+M_{N}}
  \and
  \inferrule* [lab=agent] {} {{M_{A}} \bc (\vec{x})M_{P} \;| \; \clift{P_0,\ldots,M_{P},\ldots,P_N}}
  \and \\
  \inferrule* [lab=process] {} {{M_{P}} \bc M_{N} \;| \;P|M_{P} }
\end{mathpar} 

\begin{mathpar}
  \inferrule* [lab=sychronization] {} {M_{N} \bc \Box \;|\; x?M_{F} \;|\; x!M_{C}}
  \and
  \inferrule* [lab=abstraction] {} {{M_{F}} \bc (x)M_{P} }
  \and
  \inferrule* [lab=concretion] {} {{M_{C}} \bc \langle M_{P} \rangle }
  \and \\
  \inferrule* [lab=process] {} {{M_{P}} \bc M_{N} \;| \;P|M_{P} }
\end{mathpar}

\begin{definition}[contextual application] Given a context $M$, and
  process $P$, we define the \emph{contextual application}, $M[P] :=
  M\{P/\Box\}$. That is, the contextual application of M to P is the
  substitution of $P$ for $\Box$ in $M$.
\end{definition}

$\meaningof{-} : L \to \mathcal{P}(\pi)$

\begin{mathpar}
  \inferrule* [lab=collection] {} {\meaningof{true} = \pi, \and \meaningof{~E} = \pi \setminus \meaningof{E}, \and \meaningof{E_{1} \& E_{2}} = \meaningof{E_{1}} \cap \meaningof{E_{2}}}
\end{mathpar}

\begin{mathpar}
  \inferrule* [lab=structure] {} {\meaningof{0} = \{ P \in \pi | P \equiv 0 \}, \and \\ \meaningof{E_1 | E_2} = \{ P \in \pi | P \equiv P_{1} | P_{2}, P_{1} \in \meaningof{E_{1}}, P_{2} \in \meaningof{E_2}\} }
\end{mathpar}

\begin{mathpar}
 \inferrule* [lab=behavior] {} {\meaningof{\langle a?b \rangle E} = \{ P \in \pi | P \equiv Q | u?(y)P', \\ \and \\\\ \and \\ \;\;\; u \in \meaningof{a}, \forall z.P'\{z/y\} \in \meaningof{E\{z/b\}}\}, \and \\ \meaningof{a!E} = \{ P \in \pi | P \equiv Q | x!\langle P' \rangle, x \in \meaningof{a} P' \in \meaningof{E}\} }
\end{mathpar}

\begin{mathpar}
 \inferrule* [lab=nominal] {} {\meaningof{\quotep{E}} = \{ \quotep{P} \in \quotep{\pi} | P \in \meaningof{E} \}, \and \meaningof{\quotep{P}} = \{ \quotep{Q} \in \quotep{\pi} | P \equiv Q \} \and \\ \meaningof{@\quotep{E}} = \{ P \in \pi | P \equiv @x, x \in \meaningof{E} \}}
\end{mathpar}

\begin{eqnarray*}
  \\
  \meaningof{-} : TS \to ST
\end{eqnarray*}

\begin{eqnarray*}
  \\
  L : TS \to ST
\end{eqnarray*}

\begin{eqnarray*}
  \\
  P \models E \iff P \in \meaningof{E}
\end{eqnarray*}

\begin{eqnarray*}
  P \approx_{L} Q \iff \forall E \in L. P \models E \iff Q \models E
\end{eqnarray*}

\begin{eqnarray*}
  P \approx_{K} Q
\end{eqnarray*}

\begin{eqnarray*}
  P \approx Q
\end{eqnarray*}

$\approx_{K} = \approx = \approx_{L}$

\subsubsection{Contextual duality}

Note that contexts extend the quotation operation to a family of
operations from processes to names. Given a context, $M$, we can
define a \emph{nominal context}, $\quotep{M}$ by $\quotep{M}[P] :=
\quotep{M[P]}$. To foreshadow what is to come we observe that these
operations enjoy a duality with processes very much like the duality
between vectors and maps from vectors to scalars.

Further, because the calculus is essentially higher-order, we have a
correspondence between contexts and processes. More specifically,
given a name $x$ and a context $M$ we can construct $M^{*}_{x}$ such
that 

\begin{mathpar}
  M^{*}_{x} | \lift{x}{P} \red M[P]
\end{mathpar}

namely,

\begin{mathpar}
  M^{*}_{x} := x?(u).M[\dropn{u}]
\end{mathpar}

The dependence of $M^{*}_{x}$ on a name makes it an abstraction, 

\begin{mathpar}
  M^{*} := (x)x?(u).M[\dropn{u}]
\end{mathpar}

\subsection{Additional notation}

It will sometimes be convenient to denote the process a name
quotes. We already have the notation $x = \quotep{P}$, but it will be
convenient to introduce an alternate notation, $\procn{x}$, when we
want to emphasize the connection to the use of the name. Note that, by
virtue of name equivalence, $\quotep{\procn{x}} \nameeq x$; so, the
notation is consistent with previous definitions.

Further, because names have structure it is possible to effect
substitutions on the basis of that structure. This means we need to
upgrade our notation for substitutions, which we accomplish by
adapting comprehension notation. Thus,

\begin{mathpar}
  P\{ y / x : x \in S \}
\end{mathpar}

is interpreted to mean the process derived from P by replacing (in a
capture-avoiding manner) each occurrence of $x$ in $S$ by $y$. For example,

\begin{mathpar}
  P\{ \quotep{\procn{x}|\procn{x}} / x : x \in \freenames{P} \}
\end{mathpar}

will replace each (occurrence) of a free name $x$ in $P$ by
$\quotep{\procn{x}|\procn{x}}$.

Also, we will avail ourselves of the notation $x^{L}$ and $x^{R}$ to
denote injections of a name into disjoint copies of the name
space. There are numerous ways to accomplish this. One example can be
found in \cite{MeredithR05}. This notation overloads to vectors of
names: $\vec{x}^{\pi} := (x_{i}^{\pi} \; : \; 0 \leq i < |\vec{x}| )$ where $\pi \in \{L,R\}$.

We also use $P^{\Box} := P|\Box$.

In \cite{MeredithR05} an interpretation of the new operator is
given. It turns out that there are several possible interpretations
all enjoying the requisite algebraic properties of the operator (see
\cite{milner91polyadicpi}). We will therefore make liberal use of
$(\nu\; \vec{x})P$.

% subsection the_syntax_and_semantics_of_the_notation_system (end)   

\input{qm2pi.qmops} 

\input{qm2pi.sterngerlach} 

\input{qm2pi.metric} 

% section concurrent_process_calculi (end)

%\input{qm2pi.proofsketch}

% section proof sketch (end)

%\input{qm2pi.slviaknots} 

% section spatial logic via knots (end)

\input{qm2pi.conclusion}

% section conclusion (end)

%\input{qm2pi.dtcodes} 

% section wiring algorithm (end)

\input{qm2pi.ack} 

% section acknowledgments (end)

\newpage


\bibliographystyle{plain}   
\bibliography{../../biblios/main.bib}

\input{qm2pi.rhodetails}

\end{document}

 

%\documentclass[12pt]{llncs}
%\documentclass{jktr}

\usepackage[pdftex]{hyperref}                   
\usepackage {listings}
\usepackage {mathpartir}
\usepackage{bcprules}
%\usepackage{listings}
                       
\usepackage{graphicx} 
%\usepackage[margins=2.5cm,nohead,nofoot]{geometry}
%\usepackage{geometry}
\usepackage{amsfonts}
\usepackage{amstext}
\usepackage{latexsym}
\usepackage{amssymb}
\usepackage{color}


%\include{myPreamble}
\include{qm2pi.local} 

%\ifpdf
%\usepackage[pdftex]{graphicx}
%\else
%\usepackage{graphicx}
%\fi

 % \ifpdf
%  \usepackage{pdfsync}
%  \if


%\title{Brief Article}
%\author{David F. Snyder}
%\author{L.G. Meredith}

%\address{Dept. of Math., Texas State University--San Marcos, San Marcos, TX 78666}
       
\pagestyle{empty}


\begin{document}

\lstset{language=[Objective]Caml,frame=shadowbox}

\input{qm2pi.front}

% section front matter (end)

\input{qm2pi.intro} 
 
% section introduction (end)

% \input{qm2pi.knotations} 

% section notation (end)

\input{qm2pi.process.calculi} 

% section concurrent_process_calculi_and_spatial_logics_ (end)
    
%\input{qm2pi.knots2pi} 

%\input{qm2pi.trefoil} 

%\input{qm2pi.mainthm} 

% subsection basic_interpretation (end)

%\input{qm2pi.rho.presentation} 
\subsection{The syntax and semantics of the notation system}\label{sub:the_syntax_and_semantics_of_the_notation_system} % (fold)

We now summarize a technical presentation of the calculus that
embodies our theory of dynamics. The typical presentation of such a
calculus follows the style of giving generators and relations on
them. The grammar, below, describing term constructors, freely
generates the set of processes, $\Proc$. This set is then quotiented
by a relation known as structural congruence and it is over this set
that the notion of dynamics is expressed. This presentation is
essentially that of \cite{MeredithR05} with the addition of
polyadicity and summation. For readability we have relegated some of
the technical subtleties to an appendix.

\subsubsection{Process grammar}\label{subsub:process_grammar}

\begin{mathpar}
  \inferrule* [lab=synchronization] {} {{M} \bc \pzero \;|\; x?F \;|\; x!C }
  \and
  \inferrule* [lab=abstraction] {} {{F} \bc (x)P}
  \and
  \inferrule* [lab=concretion] {} {{C} \bc \langle Q \rangle}
  \and
  \inferrule* [lab=process] {} {{P,Q} \bc M \;| \;P|Q \;|\; @{x}}
  \and
  \inferrule* [lab=name] {} {{x} \bc \quotep{P}}
\end{mathpar} 

Note that $\vec{x}$ (resp. $\vec{P}$) denotes a vector of names
(resp. processes) of length $|\vec{x}|$ (resp. $|\vec{P}|$). We adopt
the following useful abbreviations.

\begin{mathpar}
   x?(\vec{y}).P := x.(\vec{y})P \and  x\clift{\vec{P}} := x.\clift{\vec{P}}
   \and x!(y) := \lift{x}{\dropn{y}}
   \and \Pi_{i=0}^{n-1}P_i := P_0 | \ldots | P_{n-1}
\end{mathpar}

\subsubsection{Structural congruence}

\paragraph{Free and bound names and alpha-equivalence.} At the
core of structural equivalence is alpha-equivalence which identifies
process that are the same up to a change of variable. Formally, we
recognize the distinction between free and bound names. The free names
of a process, $\freenames{P}$, may be calculated recursively as
follows:

\begin{mathpar}
\freenames{\pzero} := \emptyset
  \and \\
  \freenames{x?(y).P} := \{ x \} \cup (\freenames{P} \setminus \{ y \})
  \and 
  \freenames{x!\langle P \rangle} := \{ x \} \cup \{ P \} 
  \and \\
  \freenames{P|Q} := \freenames{P} \cup \freenames{Q}
  \and \\
  \freenames{@{x}} := \{ x \}
\end{mathpar}

$\pi$
$\quotep{\pi}$

$\freenames{-} : \pi \to \mathcal{P}(\quotep{\pi})$

\begin{eqnarray*}
  \freenames{\pzero} & := & \emptyset \\
  \freenames{x?(y).P} & := & \{ x \} \cup (\freenames{P} \setminus \{ y \}) \\
  \freenames{x!\langle P \rangle} & := & \{ x \} \cup \{ P \} \\
  \freenames{P|Q} & := & \freenames{P} \cup \freenames{Q} \\
  \freenames{\dropn{x}} & := & \{ x \}
\end{eqnarray*}

The bound names of a process, $\boundnames{P}$, are those names occurring in $P$
that are not free. For example, in $x?(y).0$, the name $x$ is free, while $y$ is bound.

\begin{mathpar}
  \inferrule* [lab=monoidal-laws] {} { P|Q \equiv Q|P \and P|0 \equiv P \and P|(Q|R) \equiv (P|Q)|R }
\end{mathpar}

\begin{mathpar}
  \inferrule* [lab=alpha-equivalence] {} { (x)P \equiv (y)P\{y/x\} \and y \not\in \freenames{P} }
\end{mathpar}

\begin{definition}
Then two processes, $P,Q$, are alpha-equivalent if $P = Q\{\vec{y}/\vec{x}\}$ for
some $\vec{x} \in \boundnames{Q},\vec{y} \in \boundnames{P}$, where $Q\{\vec{y}/\vec{x}\}$
denotes the capture-avoiding substitution of $\vec{y}$ for $\vec{x}$ in $Q$.
\end{definition}

\begin{definition}
  The {\em structural congruence} \cite{SangiorgiWalker} , $\equiv$,
  between processes is the least congruence containing
  alpha-equivalence, satisfying the abelian monoid laws
  (associativity, commutativity and $\pzero$ as identity) for parallel
  composition $|$ and for summation $+$.
\end{definition}

\subsection{Name equivalence}

We take name equivalence, written $\nameeq$, to be the smallest
equivalence relation generated by the following rules.

\begin{mathpar}
\inferrule*[lab=Quote-drop]
{ }
{ \quotep{@{x}} \nameeq x }

\inferrule*[lab=Struct-equiv]
{ P \scong Q }
{ \quotep{P} \nameeq \quotep{Q} }
\end{mathpar}

The astute reader will have noticed that the mutual recursion of names
and processes imposes a mutual recursion on alpha-equivalence and
structural equivalence via name-equivalence. Fortunately, all of this
works out pleasantly and we may calculate in the natural way, free of
concern. The reader interested in the details is referred to the
appendix \ref{appendix:rho_details}.

\subsection{Substitution}

We use $\Proc$ for the set of processes, $\QProc$ for the set of
names, and $\id{\{}\vec{y} / \vec{x} \id{\}}$ to denote partial maps,
$s : \QProc \rightarrow \QProc$. A map, $s$ lifts, uniquely, to a map
on process terms, $\widehat{s} : \Proc \rightarrow \Proc$ by the
following equations.

\begin{mathpar}
  (0) \psubstp{Q}{P} := 0 \\
  (R \juxtap S) \psubstp{Q}{P}
  :=    
  (R)\psubstp{Q}{P} \juxtap (S) \psubstp{Q}{P} \\
  (x?(y).R) \psubstp{Q}{P}    
  :=    
  (x)\substp{Q}{P} (z)\concat( (R \psubstn{z}{y}) \psubstp{Q}{P} ) \\
  (\lift{x}{R}) \psubstp{Q}{P}  
  :=
  \lift{(x)\substp{Q}{P}}{ R \psubstp{Q}{P} } \\
%   (\dropn{x})  \psubstp{Q}{P}       
%   := 
%   \left\{ 
%     \begin{array}{ccc} 
%       \dropn{\quotep{Q}} & & x \nameeq \quotep{P} \\
%       \dropn{x} & & otherwise \\
%     \end{array}
%   \right. 
  (\dropn{x})  \psubstp{Q}{P}       
  := 
  \left\{ 
    \begin{array}{ccc} 
      Q & & x \nameeq \quotep{P} \\
      \dropn{x} & & otherwise \\
    \end{array}
  \right.
\end{mathpar}
 

where

\begin{eqnarray}
  (x)\id{\{} \lpquote Q \rpquote / \lpquote P \rpquote \id{\}}            = 
  \left\{ 
    \begin{array}{ccc}
      \lpquote Q \rpquote & & x \nameeq \lpquote P \rpquote \\
      x & & otherwise \\
    \end{array}
  \right. \nonumber
\end{eqnarray}

and $z$ is chosen distinct from $\quotep{P}$, $\quotep{Q}$, the free
names in $Q$, and all the names in $R$. Our $\alpha$-equivalence will
be built in the standard way from this substitution.

\begin{remark}\label{rem:no_self_referential_names}
  One consequence of these definitions is that $\forall P. \quotep{P}
  \not\in \freenames{P}$.
\end{remark}

\subsection{ Dynamic quote: an example }

Anticipating something of what's to come, consider applying the
substitution, $\widehat{\id{\{}u / z \id{\}}}$, to the following pair
of processes, $\lift{w}{y!(z)}$ and $w[ \lpquote y!(z) \rpquote ]$.

\begin{eqnarray}
	\lift{w}{y!(z)}\widehat{\id{\{}u / z \id{\}}}
		& = &
		\lift{w}{y!(u)} \nonumber\\
	w[ \lpquote y!(z) \rpquote ] \widehat{ \id{\{}u / z \id{\}} }
		& = &
		w[ \lpquote y!(z) \rpquote ] \nonumber
\end{eqnarray}

Because the body of the process between quotes is impervious to
substitution, we get radically different answers. In fact, by
examining the first process in an input context,
e.g. $x?(z).\lift{w}{y!(z)}$, we see that the process under the lift
operator may be shaped by prefixed inputs binding a name inside it. In
this sense, the lift operator will be seen as a way to dynamically
construct processes before reifying them as names.

Finally equipped with these standard features we can present the
dynamics of the calculus.

\subsubsection{Operational semantics} 

Finally, we introduce the computational dynamics. What marks these
algebras as distinct from other more traditionally studied algebraic
structures, e.g. vector spaces or polynomial rings, is the manner in
which dynamics is captured. In traditional structures, dynamics is typically
expressed through morphisms between such structures, as in linear maps
between vector spaces or morphisms between rings. In algebras
associated with the semantics of computation, the dynamics is
expressed as part of the algebraic structure itself, through a
reduction reduction relation typically denoted by $\red$. Below, we
give a recursive presentation of this relation for the calculus used
in the encoding.

$\red \subseteq \pi \times \pi$
$\red : \pi \to \mathcal{P}(\pi)$

\begin{mathpar}
  \inferrule* [lab=Comm] { \textsf{match}( x_{src}, x_{trgt} ) } { x_{trgt}?(y)P \; | \; x_{src}!\langle {Q} \rangle \red P\{\quotep{Q}/y}\} }
  \and \\
  \inferrule* [lab=Par] {{P} \red {P}'} {{{P} | {Q}} \red {{P}' | {Q}}}
  \and
  \inferrule* [lab=Equiv]{{{P} \scong {P}'} \andalso {{P}' \red {Q}'} \andalso {{Q}' \scong {Q}}}{{P} \red {Q}}
\end{mathpar}

\begin{eqnarray*}
  match_{\equiv} (\quotep{P},\quotep{Q}) & := & P \equiv Q \\
  match_{\dagger}(\quotep{P},\quotep{Q}) & := & \forall R. P|Q \red^{*} R => R \red^{*} 0 \\
  match_{K}(\quotep{P},\quotep{Q}) & := & K \mbox{ for some context } K
\end{eqnarray*}

$u?(x)P | u!\langle Q \rangle \red P\{\quotep{Q}/x\}$

%We write $\wred$ for $\red^*$, and $P\red$ if $\exists Q $ such that $ P \red Q$.
We write $P\red$ if $\exists Q $ such that $ P \red Q$ and $P\not\red$, otherwise.

\section{Replication}

As mentioned before, it is known that replication (and hence
recursion) can be implemented in a higher-order process algebra
\cite{SangiorgiWalker}. As our first example of calculation with the
machinery thus far presented we give the construction explicitly in
the {\rhoc}.

\begin{eqnarray}
	D_{x} & := & \prefix{x}{y}{(\binpar{\outputp{x}{y}}{@{y}})} \nonumber\\
	\bangp_{x}{P} & := & \binpar{{x}!\langle{\binpar{D_{x}}{P}}\rangle}{D_{x}} \nonumber
\end{eqnarray}

\begin{eqnarray}
	\bangp_{x}{P} & & \nonumber\\
	=
	& {x}!\langle{(\prefix{x}{y}{(\outputp{x}{y} | @{y})) | P}}\rangle 
	      | \prefix{x}{y}{(\outputp{x}{y} | @{y})} & \nonumber\\
	\red
	& (\outputp{x}{y} | @{y})\substn{\quotep{(\prefix{x}{y}{(@{y} | \outputp{x}{y})) | P}}}{y} & \nonumber\\
	=
	& \outputp{x}{\quotep{(\prefix{x}{y}{(\outputp{x}{y} | @{y})) | P}}}
	  | {(\prefix{x}{y}{(\outputp{x}{y} | @{y})) | P}} & \nonumber\\
	\red
	& \ldots & \nonumber\\
	\red^*
	& P | P | \ldots & \nonumber
\end{eqnarray}

Of course, this encoding, as an implementation, runs away, unfolding
$\bangp{P}$ eagerly. A lazier and more implementable replication
operator, restricted to input-guarded processes, may be obtained as follows.

\begin{eqnarray}
\bangp{\prefix{u}{v}{P}} 
	:= 
	\binpar{\lift{x}{\prefix{u}{v}{(\binpar{D(x)}{P})}}}{D(x)} \nonumber
\end{eqnarray}

\begin{remark}
  Note that the lazier definition still does not deal with summation
  or mixed summation (i.e. sums over input and output). The reader is
  invited to construct definitions of replication that deal with these
  features. 

  Further, the definitions are parameterized in a name, $x$. Can you,
  gentle reader, make a definition that eliminates this parameter and
  guarantees no accidental interaction between the replication
  machinery and the process being replicated -- i.e. no accidental
  sharing of names used by the process to get its work done and the
  name(s) used by the replication to effect copying. This latter
  revision of the definition of replication is crucial to obtaining
  the expected identity $!!P \sim !P$.
\end{remark}

\begin{remark}\label{rem:paradoxical_combinator}
  The reader familiar with the lambda calculus will have noticed the
  similarity between $D$ and the paradoxical combinator.

  [Ed. note: the existence of this seems to suggest we have to be more
  restrictive on the set of processes and names we admit if we are to
  support no-cloning.]
\end{remark}

\subsubsection{Bisimulation}

The computational dynamics gives rise to another kind of equivalence,
the equivalence of computational behavior. As previously mentioned
this is typically captured \emph{via} some form of bisimulation.

% The notion we use in this paper is weak barbed bisimulation
% \cite{milner91polyadicpi}.

The notion we use in this paper is derived from weak barbed
bisimulation \cite{milner91polyadicpi}. 

\begin{definition}
An \emph{observation relation}, $\downarrow_{\mathcal N}$, over a set
of names, $\mathcal N$, is the smallest relation satisfying the rules
below.

\infrule[Out-barb]{y \in {\mathcal N}, \; x \nameeq y}
		  {\outputp{x}{v} \downarrow_{\mathcal N} x}
\infrule[Par-barb]{\mbox{$P\downarrow_{\mathcal N} x$ or $Q\downarrow_{\mathcal N} x$}}
		  {\binpar{P}{Q} \downarrow_{\mathcal N} x}

We write $P \Downarrow_{\mathcal N} x$ if there is $Q$ such that 
$P \wred Q$ and $Q \downarrow_{\mathcal N} x$.
\end{definition}

\begin{definition}
%\label{def.bbisim}
An  ${\mathcal N}$-\emph{barbed bisimulation} over a set of names, ${\mathcal N}$, is a symmetric binary relation 
${\mathcal S}_{\mathcal N}$ between agents such that $P\rel{S}_{\mathcal N}Q$ implies:
\begin{enumerate}
\item If $P \red P'$ then $Q \wred Q'$ and $P'\rel{S}_{\mathcal N} Q'$.
\item If $P\downarrow_{\mathcal N} x$, then $Q\Downarrow_{\mathcal N} x$.
\end{enumerate}
$P$ is ${\mathcal N}$-barbed bisimilar to $Q$, written
$P \wbbisim_{\mathcal N} Q$, if $P \rel{S}_{\mathcal N} Q$ for some ${\mathcal N}$-barbed bisimulation ${\mathcal S}_{\mathcal N}$.
\end{definition}

$\mathcal{R} \subseteq \pi \times \pi$

$P \mathcal{R} Q => \forall P'. P \red P' \Rightarrow \exists Q'. Q \red Q', P' \mathcal{R} Q'$

$P \vdash x \Rightarrow Q \vdash x$

\begin{mathpar}
  \inferrule*[lab=Out-barb]{x \nameeq y}{{y}!\langle{Q}\rangle \vdash x}
  \and
  \inferrule*[lab=Par-barb]{\mbox{$P\vdash x$ or $Q\vdash x$}}{\binpar{P}{Q} \vdash x}
\end{mathpar}

\subsubsection{Contexts}

One of the principle advantages of computational calculi like the
$\pi$-calculus is a well-defined notion of context,
contextual-equivalence and a correlation between
contextual-equivalence and notions of bisimulation. The notion of
context allows the decomposition of a process into (sub-)process and
its syntactic environment, its context. Thus, a context may be
thought of as a process with a ``hole'' (written $\Box$) in it. The
application of a context $M$ to a process $P$, written $M[P]$, is
tantamount to filling the hole in $M$ with $P$. In this paper we do
not need the full weight of this theory, but do make use of the notion
of context in the proof the main theorem. 

\begin{mathpar}
  \inferrule* [lab=summation] {} {{M_{M},M_{N}} \bc \Box \;|\; x.M_{A} \;|\; M_{M}+M_{N}}
  \and
  \inferrule* [lab=agent] {} {{M_{A}} \bc (\vec{x})M_{P} \;| \; \clift{P_0,\ldots,M_{P},\ldots,P_N}}
  \and \\
  \inferrule* [lab=process] {} {{M_{P}} \bc M_{N} \;| \;P|M_{P} }
\end{mathpar} 

\begin{mathpar}
  \inferrule* [lab=sychronization] {} {M_{N} \bc \Box \;|\; x?M_{F} \;|\; x!M_{C}}
  \and
  \inferrule* [lab=abstraction] {} {{M_{F}} \bc (x)M_{P} }
  \and
  \inferrule* [lab=concretion] {} {{M_{C}} \bc \langle M_{P} \rangle }
  \and \\
  \inferrule* [lab=process] {} {{M_{P}} \bc M_{N} \;| \;P|M_{P} }
\end{mathpar}

\begin{definition}[contextual application] Given a context $M$, and
  process $P$, we define the \emph{contextual application}, $M[P] :=
  M\{P/\Box\}$. That is, the contextual application of M to P is the
  substitution of $P$ for $\Box$ in $M$.
\end{definition}

$\meaningof{-} : L \to \mathcal{P}(\pi)$

\begin{mathpar}
  \inferrule* [lab=collection] {} {\meaningof{true} = \pi, \and \meaningof{~E} = \pi \setminus \meaningof{E}, \and \meaningof{E_{1} \& E_{2}} = \meaningof{E_{1}} \cap \meaningof{E_{2}}}
\end{mathpar}

\begin{mathpar}
  \inferrule* [lab=structure] {} {\meaningof{0} = \{ P \in \pi | P \equiv 0 \}, \and \\ \meaningof{E_1 | E_2} = \{ P \in \pi | P \equiv P_{1} | P_{2}, P_{1} \in \meaningof{E_{1}}, P_{2} \in \meaningof{E_2}\} }
\end{mathpar}

\begin{mathpar}
 \inferrule* [lab=behavior] {} {\meaningof{\langle a?b \rangle E} = \{ P \in \pi | P \equiv Q | u?(y)P', \\ \and \\\\ \and \\ \;\;\; u \in \meaningof{a}, \forall z.P'\{z/y\} \in \meaningof{E\{z/b\}}\}, \and \\ \meaningof{a!E} = \{ P \in \pi | P \equiv Q | x!\langle P' \rangle, x \in \meaningof{a} P' \in \meaningof{E}\} }
\end{mathpar}

\begin{mathpar}
 \inferrule* [lab=nominal] {} {\meaningof{\quotep{E}} = \{ \quotep{P} \in \quotep{\pi} | P \in \meaningof{E} \}, \and \meaningof{\quotep{P}} = \{ \quotep{Q} \in \quotep{\pi} | P \equiv Q \} \and \\ \meaningof{@\quotep{E}} = \{ P \in \pi | P \equiv @x, x \in \meaningof{E} \}}
\end{mathpar}

\begin{eqnarray*}
  \\
  \meaningof{-} : TS \to ST
\end{eqnarray*}

\begin{eqnarray*}
  \\
  L : TS \to ST
\end{eqnarray*}

\begin{eqnarray*}
  \\
  P \models E \iff P \in \meaningof{E}
\end{eqnarray*}

\begin{eqnarray*}
  P \approx_{L} Q \iff \forall E \in L. P \models E \iff Q \models E
\end{eqnarray*}

\begin{eqnarray*}
  P \approx_{K} Q
\end{eqnarray*}

\begin{eqnarray*}
  P \approx Q
\end{eqnarray*}

$\approx_{K} = \approx = \approx_{L}$

\subsubsection{Contextual duality}

Note that contexts extend the quotation operation to a family of
operations from processes to names. Given a context, $M$, we can
define a \emph{nominal context}, $\quotep{M}$ by $\quotep{M}[P] :=
\quotep{M[P]}$. To foreshadow what is to come we observe that these
operations enjoy a duality with processes very much like the duality
between vectors and maps from vectors to scalars.

Further, because the calculus is essentially higher-order, we have a
correspondence between contexts and processes. More specifically,
given a name $x$ and a context $M$ we can construct $M^{*}_{x}$ such
that 

\begin{mathpar}
  M^{*}_{x} | \lift{x}{P} \red M[P]
\end{mathpar}

namely,

\begin{mathpar}
  M^{*}_{x} := x?(u).M[\dropn{u}]
\end{mathpar}

The dependence of $M^{*}_{x}$ on a name makes it an abstraction, 

\begin{mathpar}
  M^{*} := (x)x?(u).M[\dropn{u}]
\end{mathpar}

\subsection{Additional notation}

It will sometimes be convenient to denote the process a name
quotes. We already have the notation $x = \quotep{P}$, but it will be
convenient to introduce an alternate notation, $\procn{x}$, when we
want to emphasize the connection to the use of the name. Note that, by
virtue of name equivalence, $\quotep{\procn{x}} \nameeq x$; so, the
notation is consistent with previous definitions.

Further, because names have structure it is possible to effect
substitutions on the basis of that structure. This means we need to
upgrade our notation for substitutions, which we accomplish by
adapting comprehension notation. Thus,

\begin{mathpar}
  P\{ y / x : x \in S \}
\end{mathpar}

is interpreted to mean the process derived from P by replacing (in a
capture-avoiding manner) each occurrence of $x$ in $S$ by $y$. For example,

\begin{mathpar}
  P\{ \quotep{\procn{x}|\procn{x}} / x : x \in \freenames{P} \}
\end{mathpar}

will replace each (occurrence) of a free name $x$ in $P$ by
$\quotep{\procn{x}|\procn{x}}$.

Also, we will avail ourselves of the notation $x^{L}$ and $x^{R}$ to
denote injections of a name into disjoint copies of the name
space. There are numerous ways to accomplish this. One example can be
found in \cite{MeredithR05}. This notation overloads to vectors of
names: $\vec{x}^{\pi} := (x_{i}^{\pi} \; : \; 0 \leq i < |\vec{x}| )$ where $\pi \in \{L,R\}$.

We also use $P^{\Box} := P|\Box$.

In \cite{MeredithR05} an interpretation of the new operator is
given. It turns out that there are several possible interpretations
all enjoying the requisite algebraic properties of the operator (see
\cite{milner91polyadicpi}). We will therefore make liberal use of
$(\nu\; \vec{x})P$.

% subsection the_syntax_and_semantics_of_the_notation_system (end)   

\input{qm2pi.qmops} 

\input{qm2pi.sterngerlach} 

\input{qm2pi.metric} 

% section concurrent_process_calculi (end)

%\input{qm2pi.proofsketch}

% section proof sketch (end)

%\input{qm2pi.slviaknots} 

% section spatial logic via knots (end)

\input{qm2pi.conclusion}

% section conclusion (end)

%\input{qm2pi.dtcodes} 

% section wiring algorithm (end)

\input{qm2pi.ack} 

% section acknowledgments (end)

\newpage


\bibliographystyle{plain}   
\bibliography{../../biblios/main.bib}

\input{qm2pi.rhodetails}

\end{document}

 

%\documentclass[12pt]{llncs}
%\documentclass{jktr}

\usepackage[pdftex]{hyperref}                   
\usepackage {listings}
\usepackage {mathpartir}
\usepackage{bcprules}
%\usepackage{listings}
                       
\usepackage{graphicx} 
%\usepackage[margins=2.5cm,nohead,nofoot]{geometry}
%\usepackage{geometry}
\usepackage{amsfonts}
\usepackage{amstext}
\usepackage{latexsym}
\usepackage{amssymb}
\usepackage{color}


%\include{myPreamble}
\include{qm2pi.local} 

%\ifpdf
%\usepackage[pdftex]{graphicx}
%\else
%\usepackage{graphicx}
%\fi

 % \ifpdf
%  \usepackage{pdfsync}
%  \if


%\title{Brief Article}
%\author{David F. Snyder}
%\author{L.G. Meredith}

%\address{Dept. of Math., Texas State University--San Marcos, San Marcos, TX 78666}
       
\pagestyle{empty}


\begin{document}

\lstset{language=[Objective]Caml,frame=shadowbox}

\input{qm2pi.front}

% section front matter (end)

\input{qm2pi.intro} 
 
% section introduction (end)

% \input{qm2pi.knotations} 

% section notation (end)

\input{qm2pi.process.calculi} 

% section concurrent_process_calculi_and_spatial_logics_ (end)
    
%\input{qm2pi.knots2pi} 

%\input{qm2pi.trefoil} 

%\input{qm2pi.mainthm} 

% subsection basic_interpretation (end)

%\input{qm2pi.rho.presentation} 
\subsection{The syntax and semantics of the notation system}\label{sub:the_syntax_and_semantics_of_the_notation_system} % (fold)

We now summarize a technical presentation of the calculus that
embodies our theory of dynamics. The typical presentation of such a
calculus follows the style of giving generators and relations on
them. The grammar, below, describing term constructors, freely
generates the set of processes, $\Proc$. This set is then quotiented
by a relation known as structural congruence and it is over this set
that the notion of dynamics is expressed. This presentation is
essentially that of \cite{MeredithR05} with the addition of
polyadicity and summation. For readability we have relegated some of
the technical subtleties to an appendix.

\subsubsection{Process grammar}\label{subsub:process_grammar}

\begin{mathpar}
  \inferrule* [lab=synchronization] {} {{M} \bc \pzero \;|\; x?F \;|\; x!C }
  \and
  \inferrule* [lab=abstraction] {} {{F} \bc (x)P}
  \and
  \inferrule* [lab=concretion] {} {{C} \bc \langle Q \rangle}
  \and
  \inferrule* [lab=process] {} {{P,Q} \bc M \;| \;P|Q \;|\; @{x}}
  \and
  \inferrule* [lab=name] {} {{x} \bc \quotep{P}}
\end{mathpar} 

Note that $\vec{x}$ (resp. $\vec{P}$) denotes a vector of names
(resp. processes) of length $|\vec{x}|$ (resp. $|\vec{P}|$). We adopt
the following useful abbreviations.

\begin{mathpar}
   x?(\vec{y}).P := x.(\vec{y})P \and  x\clift{\vec{P}} := x.\clift{\vec{P}}
   \and x!(y) := \lift{x}{\dropn{y}}
   \and \Pi_{i=0}^{n-1}P_i := P_0 | \ldots | P_{n-1}
\end{mathpar}

\subsubsection{Structural congruence}

\paragraph{Free and bound names and alpha-equivalence.} At the
core of structural equivalence is alpha-equivalence which identifies
process that are the same up to a change of variable. Formally, we
recognize the distinction between free and bound names. The free names
of a process, $\freenames{P}$, may be calculated recursively as
follows:

\begin{mathpar}
\freenames{\pzero} := \emptyset
  \and \\
  \freenames{x?(y).P} := \{ x \} \cup (\freenames{P} \setminus \{ y \})
  \and 
  \freenames{x!\langle P \rangle} := \{ x \} \cup \{ P \} 
  \and \\
  \freenames{P|Q} := \freenames{P} \cup \freenames{Q}
  \and \\
  \freenames{@{x}} := \{ x \}
\end{mathpar}

$\pi$
$\quotep{\pi}$

$\freenames{-} : \pi \to \mathcal{P}(\quotep{\pi})$

\begin{eqnarray*}
  \freenames{\pzero} & := & \emptyset \\
  \freenames{x?(y).P} & := & \{ x \} \cup (\freenames{P} \setminus \{ y \}) \\
  \freenames{x!\langle P \rangle} & := & \{ x \} \cup \{ P \} \\
  \freenames{P|Q} & := & \freenames{P} \cup \freenames{Q} \\
  \freenames{\dropn{x}} & := & \{ x \}
\end{eqnarray*}

The bound names of a process, $\boundnames{P}$, are those names occurring in $P$
that are not free. For example, in $x?(y).0$, the name $x$ is free, while $y$ is bound.

\begin{mathpar}
  \inferrule* [lab=monoidal-laws] {} { P|Q \equiv Q|P \and P|0 \equiv P \and P|(Q|R) \equiv (P|Q)|R }
\end{mathpar}

\begin{mathpar}
  \inferrule* [lab=alpha-equivalence] {} { (x)P \equiv (y)P\{y/x\} \and y \not\in \freenames{P} }
\end{mathpar}

\begin{definition}
Then two processes, $P,Q$, are alpha-equivalent if $P = Q\{\vec{y}/\vec{x}\}$ for
some $\vec{x} \in \boundnames{Q},\vec{y} \in \boundnames{P}$, where $Q\{\vec{y}/\vec{x}\}$
denotes the capture-avoiding substitution of $\vec{y}$ for $\vec{x}$ in $Q$.
\end{definition}

\begin{definition}
  The {\em structural congruence} \cite{SangiorgiWalker} , $\equiv$,
  between processes is the least congruence containing
  alpha-equivalence, satisfying the abelian monoid laws
  (associativity, commutativity and $\pzero$ as identity) for parallel
  composition $|$ and for summation $+$.
\end{definition}

\subsection{Name equivalence}

We take name equivalence, written $\nameeq$, to be the smallest
equivalence relation generated by the following rules.

\begin{mathpar}
\inferrule*[lab=Quote-drop]
{ }
{ \quotep{@{x}} \nameeq x }

\inferrule*[lab=Struct-equiv]
{ P \scong Q }
{ \quotep{P} \nameeq \quotep{Q} }
\end{mathpar}

The astute reader will have noticed that the mutual recursion of names
and processes imposes a mutual recursion on alpha-equivalence and
structural equivalence via name-equivalence. Fortunately, all of this
works out pleasantly and we may calculate in the natural way, free of
concern. The reader interested in the details is referred to the
appendix \ref{appendix:rho_details}.

\subsection{Substitution}

We use $\Proc$ for the set of processes, $\QProc$ for the set of
names, and $\id{\{}\vec{y} / \vec{x} \id{\}}$ to denote partial maps,
$s : \QProc \rightarrow \QProc$. A map, $s$ lifts, uniquely, to a map
on process terms, $\widehat{s} : \Proc \rightarrow \Proc$ by the
following equations.

\begin{mathpar}
  (0) \psubstp{Q}{P} := 0 \\
  (R \juxtap S) \psubstp{Q}{P}
  :=    
  (R)\psubstp{Q}{P} \juxtap (S) \psubstp{Q}{P} \\
  (x?(y).R) \psubstp{Q}{P}    
  :=    
  (x)\substp{Q}{P} (z)\concat( (R \psubstn{z}{y}) \psubstp{Q}{P} ) \\
  (\lift{x}{R}) \psubstp{Q}{P}  
  :=
  \lift{(x)\substp{Q}{P}}{ R \psubstp{Q}{P} } \\
%   (\dropn{x})  \psubstp{Q}{P}       
%   := 
%   \left\{ 
%     \begin{array}{ccc} 
%       \dropn{\quotep{Q}} & & x \nameeq \quotep{P} \\
%       \dropn{x} & & otherwise \\
%     \end{array}
%   \right. 
  (\dropn{x})  \psubstp{Q}{P}       
  := 
  \left\{ 
    \begin{array}{ccc} 
      Q & & x \nameeq \quotep{P} \\
      \dropn{x} & & otherwise \\
    \end{array}
  \right.
\end{mathpar}
 

where

\begin{eqnarray}
  (x)\id{\{} \lpquote Q \rpquote / \lpquote P \rpquote \id{\}}            = 
  \left\{ 
    \begin{array}{ccc}
      \lpquote Q \rpquote & & x \nameeq \lpquote P \rpquote \\
      x & & otherwise \\
    \end{array}
  \right. \nonumber
\end{eqnarray}

and $z$ is chosen distinct from $\quotep{P}$, $\quotep{Q}$, the free
names in $Q$, and all the names in $R$. Our $\alpha$-equivalence will
be built in the standard way from this substitution.

\begin{remark}\label{rem:no_self_referential_names}
  One consequence of these definitions is that $\forall P. \quotep{P}
  \not\in \freenames{P}$.
\end{remark}

\subsection{ Dynamic quote: an example }

Anticipating something of what's to come, consider applying the
substitution, $\widehat{\id{\{}u / z \id{\}}}$, to the following pair
of processes, $\lift{w}{y!(z)}$ and $w[ \lpquote y!(z) \rpquote ]$.

\begin{eqnarray}
	\lift{w}{y!(z)}\widehat{\id{\{}u / z \id{\}}}
		& = &
		\lift{w}{y!(u)} \nonumber\\
	w[ \lpquote y!(z) \rpquote ] \widehat{ \id{\{}u / z \id{\}} }
		& = &
		w[ \lpquote y!(z) \rpquote ] \nonumber
\end{eqnarray}

Because the body of the process between quotes is impervious to
substitution, we get radically different answers. In fact, by
examining the first process in an input context,
e.g. $x?(z).\lift{w}{y!(z)}$, we see that the process under the lift
operator may be shaped by prefixed inputs binding a name inside it. In
this sense, the lift operator will be seen as a way to dynamically
construct processes before reifying them as names.

Finally equipped with these standard features we can present the
dynamics of the calculus.

\subsubsection{Operational semantics} 

Finally, we introduce the computational dynamics. What marks these
algebras as distinct from other more traditionally studied algebraic
structures, e.g. vector spaces or polynomial rings, is the manner in
which dynamics is captured. In traditional structures, dynamics is typically
expressed through morphisms between such structures, as in linear maps
between vector spaces or morphisms between rings. In algebras
associated with the semantics of computation, the dynamics is
expressed as part of the algebraic structure itself, through a
reduction reduction relation typically denoted by $\red$. Below, we
give a recursive presentation of this relation for the calculus used
in the encoding.

$\red \subseteq \pi \times \pi$
$\red : \pi \to \mathcal{P}(\pi)$

\begin{mathpar}
  \inferrule* [lab=Comm] { \textsf{match}( x_{src}, x_{trgt} ) } { x_{trgt}?(y)P \; | \; x_{src}!\langle {Q} \rangle \red P\{\quotep{Q}/y}\} }
  \and \\
  \inferrule* [lab=Par] {{P} \red {P}'} {{{P} | {Q}} \red {{P}' | {Q}}}
  \and
  \inferrule* [lab=Equiv]{{{P} \scong {P}'} \andalso {{P}' \red {Q}'} \andalso {{Q}' \scong {Q}}}{{P} \red {Q}}
\end{mathpar}

\begin{eqnarray*}
  match_{\equiv} (\quotep{P},\quotep{Q}) & := & P \equiv Q \\
  match_{\dagger}(\quotep{P},\quotep{Q}) & := & \forall R. P|Q \red^{*} R => R \red^{*} 0 \\
  match_{K}(\quotep{P},\quotep{Q}) & := & K \mbox{ for some context } K
\end{eqnarray*}

$u?(x)P | u!\langle Q \rangle \red P\{\quotep{Q}/x\}$

%We write $\wred$ for $\red^*$, and $P\red$ if $\exists Q $ such that $ P \red Q$.
We write $P\red$ if $\exists Q $ such that $ P \red Q$ and $P\not\red$, otherwise.

\section{Replication}

As mentioned before, it is known that replication (and hence
recursion) can be implemented in a higher-order process algebra
\cite{SangiorgiWalker}. As our first example of calculation with the
machinery thus far presented we give the construction explicitly in
the {\rhoc}.

\begin{eqnarray}
	D_{x} & := & \prefix{x}{y}{(\binpar{\outputp{x}{y}}{@{y}})} \nonumber\\
	\bangp_{x}{P} & := & \binpar{{x}!\langle{\binpar{D_{x}}{P}}\rangle}{D_{x}} \nonumber
\end{eqnarray}

\begin{eqnarray}
	\bangp_{x}{P} & & \nonumber\\
	=
	& {x}!\langle{(\prefix{x}{y}{(\outputp{x}{y} | @{y})) | P}}\rangle 
	      | \prefix{x}{y}{(\outputp{x}{y} | @{y})} & \nonumber\\
	\red
	& (\outputp{x}{y} | @{y})\substn{\quotep{(\prefix{x}{y}{(@{y} | \outputp{x}{y})) | P}}}{y} & \nonumber\\
	=
	& \outputp{x}{\quotep{(\prefix{x}{y}{(\outputp{x}{y} | @{y})) | P}}}
	  | {(\prefix{x}{y}{(\outputp{x}{y} | @{y})) | P}} & \nonumber\\
	\red
	& \ldots & \nonumber\\
	\red^*
	& P | P | \ldots & \nonumber
\end{eqnarray}

Of course, this encoding, as an implementation, runs away, unfolding
$\bangp{P}$ eagerly. A lazier and more implementable replication
operator, restricted to input-guarded processes, may be obtained as follows.

\begin{eqnarray}
\bangp{\prefix{u}{v}{P}} 
	:= 
	\binpar{\lift{x}{\prefix{u}{v}{(\binpar{D(x)}{P})}}}{D(x)} \nonumber
\end{eqnarray}

\begin{remark}
  Note that the lazier definition still does not deal with summation
  or mixed summation (i.e. sums over input and output). The reader is
  invited to construct definitions of replication that deal with these
  features. 

  Further, the definitions are parameterized in a name, $x$. Can you,
  gentle reader, make a definition that eliminates this parameter and
  guarantees no accidental interaction between the replication
  machinery and the process being replicated -- i.e. no accidental
  sharing of names used by the process to get its work done and the
  name(s) used by the replication to effect copying. This latter
  revision of the definition of replication is crucial to obtaining
  the expected identity $!!P \sim !P$.
\end{remark}

\begin{remark}\label{rem:paradoxical_combinator}
  The reader familiar with the lambda calculus will have noticed the
  similarity between $D$ and the paradoxical combinator.

  [Ed. note: the existence of this seems to suggest we have to be more
  restrictive on the set of processes and names we admit if we are to
  support no-cloning.]
\end{remark}

\subsubsection{Bisimulation}

The computational dynamics gives rise to another kind of equivalence,
the equivalence of computational behavior. As previously mentioned
this is typically captured \emph{via} some form of bisimulation.

% The notion we use in this paper is weak barbed bisimulation
% \cite{milner91polyadicpi}.

The notion we use in this paper is derived from weak barbed
bisimulation \cite{milner91polyadicpi}. 

\begin{definition}
An \emph{observation relation}, $\downarrow_{\mathcal N}$, over a set
of names, $\mathcal N$, is the smallest relation satisfying the rules
below.

\infrule[Out-barb]{y \in {\mathcal N}, \; x \nameeq y}
		  {\outputp{x}{v} \downarrow_{\mathcal N} x}
\infrule[Par-barb]{\mbox{$P\downarrow_{\mathcal N} x$ or $Q\downarrow_{\mathcal N} x$}}
		  {\binpar{P}{Q} \downarrow_{\mathcal N} x}

We write $P \Downarrow_{\mathcal N} x$ if there is $Q$ such that 
$P \wred Q$ and $Q \downarrow_{\mathcal N} x$.
\end{definition}

\begin{definition}
%\label{def.bbisim}
An  ${\mathcal N}$-\emph{barbed bisimulation} over a set of names, ${\mathcal N}$, is a symmetric binary relation 
${\mathcal S}_{\mathcal N}$ between agents such that $P\rel{S}_{\mathcal N}Q$ implies:
\begin{enumerate}
\item If $P \red P'$ then $Q \wred Q'$ and $P'\rel{S}_{\mathcal N} Q'$.
\item If $P\downarrow_{\mathcal N} x$, then $Q\Downarrow_{\mathcal N} x$.
\end{enumerate}
$P$ is ${\mathcal N}$-barbed bisimilar to $Q$, written
$P \wbbisim_{\mathcal N} Q$, if $P \rel{S}_{\mathcal N} Q$ for some ${\mathcal N}$-barbed bisimulation ${\mathcal S}_{\mathcal N}$.
\end{definition}

$\mathcal{R} \subseteq \pi \times \pi$

$P \mathcal{R} Q => \forall P'. P \red P' \Rightarrow \exists Q'. Q \red Q', P' \mathcal{R} Q'$

$P \vdash x \Rightarrow Q \vdash x$

\begin{mathpar}
  \inferrule*[lab=Out-barb]{x \nameeq y}{{y}!\langle{Q}\rangle \vdash x}
  \and
  \inferrule*[lab=Par-barb]{\mbox{$P\vdash x$ or $Q\vdash x$}}{\binpar{P}{Q} \vdash x}
\end{mathpar}

\subsubsection{Contexts}

One of the principle advantages of computational calculi like the
$\pi$-calculus is a well-defined notion of context,
contextual-equivalence and a correlation between
contextual-equivalence and notions of bisimulation. The notion of
context allows the decomposition of a process into (sub-)process and
its syntactic environment, its context. Thus, a context may be
thought of as a process with a ``hole'' (written $\Box$) in it. The
application of a context $M$ to a process $P$, written $M[P]$, is
tantamount to filling the hole in $M$ with $P$. In this paper we do
not need the full weight of this theory, but do make use of the notion
of context in the proof the main theorem. 

\begin{mathpar}
  \inferrule* [lab=summation] {} {{M_{M},M_{N}} \bc \Box \;|\; x.M_{A} \;|\; M_{M}+M_{N}}
  \and
  \inferrule* [lab=agent] {} {{M_{A}} \bc (\vec{x})M_{P} \;| \; \clift{P_0,\ldots,M_{P},\ldots,P_N}}
  \and \\
  \inferrule* [lab=process] {} {{M_{P}} \bc M_{N} \;| \;P|M_{P} }
\end{mathpar} 

\begin{mathpar}
  \inferrule* [lab=sychronization] {} {M_{N} \bc \Box \;|\; x?M_{F} \;|\; x!M_{C}}
  \and
  \inferrule* [lab=abstraction] {} {{M_{F}} \bc (x)M_{P} }
  \and
  \inferrule* [lab=concretion] {} {{M_{C}} \bc \langle M_{P} \rangle }
  \and \\
  \inferrule* [lab=process] {} {{M_{P}} \bc M_{N} \;| \;P|M_{P} }
\end{mathpar}

\begin{definition}[contextual application] Given a context $M$, and
  process $P$, we define the \emph{contextual application}, $M[P] :=
  M\{P/\Box\}$. That is, the contextual application of M to P is the
  substitution of $P$ for $\Box$ in $M$.
\end{definition}

$\meaningof{-} : L \to \mathcal{P}(\pi)$

\begin{mathpar}
  \inferrule* [lab=collection] {} {\meaningof{true} = \pi, \and \meaningof{~E} = \pi \setminus \meaningof{E}, \and \meaningof{E_{1} \& E_{2}} = \meaningof{E_{1}} \cap \meaningof{E_{2}}}
\end{mathpar}

\begin{mathpar}
  \inferrule* [lab=structure] {} {\meaningof{0} = \{ P \in \pi | P \equiv 0 \}, \and \\ \meaningof{E_1 | E_2} = \{ P \in \pi | P \equiv P_{1} | P_{2}, P_{1} \in \meaningof{E_{1}}, P_{2} \in \meaningof{E_2}\} }
\end{mathpar}

\begin{mathpar}
 \inferrule* [lab=behavior] {} {\meaningof{\langle a?b \rangle E} = \{ P \in \pi | P \equiv Q | u?(y)P', \\ \and \\\\ \and \\ \;\;\; u \in \meaningof{a}, \forall z.P'\{z/y\} \in \meaningof{E\{z/b\}}\}, \and \\ \meaningof{a!E} = \{ P \in \pi | P \equiv Q | x!\langle P' \rangle, x \in \meaningof{a} P' \in \meaningof{E}\} }
\end{mathpar}

\begin{mathpar}
 \inferrule* [lab=nominal] {} {\meaningof{\quotep{E}} = \{ \quotep{P} \in \quotep{\pi} | P \in \meaningof{E} \}, \and \meaningof{\quotep{P}} = \{ \quotep{Q} \in \quotep{\pi} | P \equiv Q \} \and \\ \meaningof{@\quotep{E}} = \{ P \in \pi | P \equiv @x, x \in \meaningof{E} \}}
\end{mathpar}

\begin{eqnarray*}
  \\
  \meaningof{-} : TS \to ST
\end{eqnarray*}

\begin{eqnarray*}
  \\
  L : TS \to ST
\end{eqnarray*}

\begin{eqnarray*}
  \\
  P \models E \iff P \in \meaningof{E}
\end{eqnarray*}

\begin{eqnarray*}
  P \approx_{L} Q \iff \forall E \in L. P \models E \iff Q \models E
\end{eqnarray*}

\begin{eqnarray*}
  P \approx_{K} Q
\end{eqnarray*}

\begin{eqnarray*}
  P \approx Q
\end{eqnarray*}

$\approx_{K} = \approx = \approx_{L}$

\subsubsection{Contextual duality}

Note that contexts extend the quotation operation to a family of
operations from processes to names. Given a context, $M$, we can
define a \emph{nominal context}, $\quotep{M}$ by $\quotep{M}[P] :=
\quotep{M[P]}$. To foreshadow what is to come we observe that these
operations enjoy a duality with processes very much like the duality
between vectors and maps from vectors to scalars.

Further, because the calculus is essentially higher-order, we have a
correspondence between contexts and processes. More specifically,
given a name $x$ and a context $M$ we can construct $M^{*}_{x}$ such
that 

\begin{mathpar}
  M^{*}_{x} | \lift{x}{P} \red M[P]
\end{mathpar}

namely,

\begin{mathpar}
  M^{*}_{x} := x?(u).M[\dropn{u}]
\end{mathpar}

The dependence of $M^{*}_{x}$ on a name makes it an abstraction, 

\begin{mathpar}
  M^{*} := (x)x?(u).M[\dropn{u}]
\end{mathpar}

\subsection{Additional notation}

It will sometimes be convenient to denote the process a name
quotes. We already have the notation $x = \quotep{P}$, but it will be
convenient to introduce an alternate notation, $\procn{x}$, when we
want to emphasize the connection to the use of the name. Note that, by
virtue of name equivalence, $\quotep{\procn{x}} \nameeq x$; so, the
notation is consistent with previous definitions.

Further, because names have structure it is possible to effect
substitutions on the basis of that structure. This means we need to
upgrade our notation for substitutions, which we accomplish by
adapting comprehension notation. Thus,

\begin{mathpar}
  P\{ y / x : x \in S \}
\end{mathpar}

is interpreted to mean the process derived from P by replacing (in a
capture-avoiding manner) each occurrence of $x$ in $S$ by $y$. For example,

\begin{mathpar}
  P\{ \quotep{\procn{x}|\procn{x}} / x : x \in \freenames{P} \}
\end{mathpar}

will replace each (occurrence) of a free name $x$ in $P$ by
$\quotep{\procn{x}|\procn{x}}$.

Also, we will avail ourselves of the notation $x^{L}$ and $x^{R}$ to
denote injections of a name into disjoint copies of the name
space. There are numerous ways to accomplish this. One example can be
found in \cite{MeredithR05}. This notation overloads to vectors of
names: $\vec{x}^{\pi} := (x_{i}^{\pi} \; : \; 0 \leq i < |\vec{x}| )$ where $\pi \in \{L,R\}$.

We also use $P^{\Box} := P|\Box$.

In \cite{MeredithR05} an interpretation of the new operator is
given. It turns out that there are several possible interpretations
all enjoying the requisite algebraic properties of the operator (see
\cite{milner91polyadicpi}). We will therefore make liberal use of
$(\nu\; \vec{x})P$.

% subsection the_syntax_and_semantics_of_the_notation_system (end)   

\input{qm2pi.qmops} 

\input{qm2pi.sterngerlach} 

\input{qm2pi.metric} 

% section concurrent_process_calculi (end)

%\input{qm2pi.proofsketch}

% section proof sketch (end)

%\input{qm2pi.slviaknots} 

% section spatial logic via knots (end)

\input{qm2pi.conclusion}

% section conclusion (end)

%\input{qm2pi.dtcodes} 

% section wiring algorithm (end)

\input{qm2pi.ack} 

% section acknowledgments (end)

\newpage


\bibliographystyle{plain}   
\bibliography{../../biblios/main.bib}

\input{qm2pi.rhodetails}

\end{document}

 

% subsection basic_interpretation (end)

%\input{qm2pi.rho.presentation} 
\subsection{The syntax and semantics of the notation system}\label{sub:the_syntax_and_semantics_of_the_notation_system} % (fold)

We now summarize a technical presentation of the calculus that
embodies our theory of dynamics. The typical presentation of such a
calculus follows the style of giving generators and relations on
them. The grammar, below, describing term constructors, freely
generates the set of processes, $\Proc$. This set is then quotiented
by a relation known as structural congruence and it is over this set
that the notion of dynamics is expressed. This presentation is
essentially that of \cite{MeredithR05} with the addition of
polyadicity and summation. For readability we have relegated some of
the technical subtleties to an appendix.

\subsubsection{Process grammar}\label{subsub:process_grammar}

\begin{mathpar}
  \inferrule* [lab=synchronization] {} {{M} \bc \pzero \;|\; x?F \;|\; x!C }
  \and
  \inferrule* [lab=abstraction] {} {{F} \bc (x)P}
  \and
  \inferrule* [lab=concretion] {} {{C} \bc \langle Q \rangle}
  \and
  \inferrule* [lab=process] {} {{P,Q} \bc M \;| \;P|Q \;|\; @{x}}
  \and
  \inferrule* [lab=name] {} {{x} \bc \quotep{P}}
\end{mathpar} 

Note that $\vec{x}$ (resp. $\vec{P}$) denotes a vector of names
(resp. processes) of length $|\vec{x}|$ (resp. $|\vec{P}|$). We adopt
the following useful abbreviations.

\begin{mathpar}
   x?(\vec{y}).P := x.(\vec{y})P \and  x\clift{\vec{P}} := x.\clift{\vec{P}}
   \and x!(y) := \lift{x}{\dropn{y}}
   \and \Pi_{i=0}^{n-1}P_i := P_0 | \ldots | P_{n-1}
\end{mathpar}

\subsubsection{Structural congruence}

\paragraph{Free and bound names and alpha-equivalence.} At the
core of structural equivalence is alpha-equivalence which identifies
process that are the same up to a change of variable. Formally, we
recognize the distinction between free and bound names. The free names
of a process, $\freenames{P}$, may be calculated recursively as
follows:

\begin{mathpar}
\freenames{\pzero} := \emptyset
  \and \\
  \freenames{x?(y).P} := \{ x \} \cup (\freenames{P} \setminus \{ y \})
  \and 
  \freenames{x!\langle P \rangle} := \{ x \} \cup \{ P \} 
  \and \\
  \freenames{P|Q} := \freenames{P} \cup \freenames{Q}
  \and \\
  \freenames{@{x}} := \{ x \}
\end{mathpar}

$\pi$
$\quotep{\pi}$

$\freenames{-} : \pi \to \mathcal{P}(\quotep{\pi})$

\begin{eqnarray*}
  \freenames{\pzero} & := & \emptyset \\
  \freenames{x?(y).P} & := & \{ x \} \cup (\freenames{P} \setminus \{ y \}) \\
  \freenames{x!\langle P \rangle} & := & \{ x \} \cup \{ P \} \\
  \freenames{P|Q} & := & \freenames{P} \cup \freenames{Q} \\
  \freenames{\dropn{x}} & := & \{ x \}
\end{eqnarray*}

The bound names of a process, $\boundnames{P}$, are those names occurring in $P$
that are not free. For example, in $x?(y).0$, the name $x$ is free, while $y$ is bound.

\begin{mathpar}
  \inferrule* [lab=monoidal-laws] {} { P|Q \equiv Q|P \and P|0 \equiv P \and P|(Q|R) \equiv (P|Q)|R }
\end{mathpar}

\begin{mathpar}
  \inferrule* [lab=alpha-equivalence] {} { (x)P \equiv (y)P\{y/x\} \and y \not\in \freenames{P} }
\end{mathpar}

\begin{definition}
Then two processes, $P,Q$, are alpha-equivalent if $P = Q\{\vec{y}/\vec{x}\}$ for
some $\vec{x} \in \boundnames{Q},\vec{y} \in \boundnames{P}$, where $Q\{\vec{y}/\vec{x}\}$
denotes the capture-avoiding substitution of $\vec{y}$ for $\vec{x}$ in $Q$.
\end{definition}

\begin{definition}
  The {\em structural congruence} \cite{SangiorgiWalker} , $\equiv$,
  between processes is the least congruence containing
  alpha-equivalence, satisfying the abelian monoid laws
  (associativity, commutativity and $\pzero$ as identity) for parallel
  composition $|$ and for summation $+$.
\end{definition}

\subsection{Name equivalence}

We take name equivalence, written $\nameeq$, to be the smallest
equivalence relation generated by the following rules.

\begin{mathpar}
\inferrule*[lab=Quote-drop]
{ }
{ \quotep{@{x}} \nameeq x }

\inferrule*[lab=Struct-equiv]
{ P \scong Q }
{ \quotep{P} \nameeq \quotep{Q} }
\end{mathpar}

The astute reader will have noticed that the mutual recursion of names
and processes imposes a mutual recursion on alpha-equivalence and
structural equivalence via name-equivalence. Fortunately, all of this
works out pleasantly and we may calculate in the natural way, free of
concern. The reader interested in the details is referred to the
appendix \ref{appendix:rho_details}.

\subsection{Substitution}

We use $\Proc$ for the set of processes, $\QProc$ for the set of
names, and $\id{\{}\vec{y} / \vec{x} \id{\}}$ to denote partial maps,
$s : \QProc \rightarrow \QProc$. A map, $s$ lifts, uniquely, to a map
on process terms, $\widehat{s} : \Proc \rightarrow \Proc$ by the
following equations.

\begin{mathpar}
  (0) \psubstp{Q}{P} := 0 \\
  (R \juxtap S) \psubstp{Q}{P}
  :=    
  (R)\psubstp{Q}{P} \juxtap (S) \psubstp{Q}{P} \\
  (x?(y).R) \psubstp{Q}{P}    
  :=    
  (x)\substp{Q}{P} (z)\concat( (R \psubstn{z}{y}) \psubstp{Q}{P} ) \\
  (\lift{x}{R}) \psubstp{Q}{P}  
  :=
  \lift{(x)\substp{Q}{P}}{ R \psubstp{Q}{P} } \\
%   (\dropn{x})  \psubstp{Q}{P}       
%   := 
%   \left\{ 
%     \begin{array}{ccc} 
%       \dropn{\quotep{Q}} & & x \nameeq \quotep{P} \\
%       \dropn{x} & & otherwise \\
%     \end{array}
%   \right. 
  (\dropn{x})  \psubstp{Q}{P}       
  := 
  \left\{ 
    \begin{array}{ccc} 
      Q & & x \nameeq \quotep{P} \\
      \dropn{x} & & otherwise \\
    \end{array}
  \right.
\end{mathpar}
 

where

\begin{eqnarray}
  (x)\id{\{} \lpquote Q \rpquote / \lpquote P \rpquote \id{\}}            = 
  \left\{ 
    \begin{array}{ccc}
      \lpquote Q \rpquote & & x \nameeq \lpquote P \rpquote \\
      x & & otherwise \\
    \end{array}
  \right. \nonumber
\end{eqnarray}

and $z$ is chosen distinct from $\quotep{P}$, $\quotep{Q}$, the free
names in $Q$, and all the names in $R$. Our $\alpha$-equivalence will
be built in the standard way from this substitution.

\begin{remark}\label{rem:no_self_referential_names}
  One consequence of these definitions is that $\forall P. \quotep{P}
  \not\in \freenames{P}$.
\end{remark}

\subsection{ Dynamic quote: an example }

Anticipating something of what's to come, consider applying the
substitution, $\widehat{\id{\{}u / z \id{\}}}$, to the following pair
of processes, $\lift{w}{y!(z)}$ and $w[ \lpquote y!(z) \rpquote ]$.

\begin{eqnarray}
	\lift{w}{y!(z)}\widehat{\id{\{}u / z \id{\}}}
		& = &
		\lift{w}{y!(u)} \nonumber\\
	w[ \lpquote y!(z) \rpquote ] \widehat{ \id{\{}u / z \id{\}} }
		& = &
		w[ \lpquote y!(z) \rpquote ] \nonumber
\end{eqnarray}

Because the body of the process between quotes is impervious to
substitution, we get radically different answers. In fact, by
examining the first process in an input context,
e.g. $x?(z).\lift{w}{y!(z)}$, we see that the process under the lift
operator may be shaped by prefixed inputs binding a name inside it. In
this sense, the lift operator will be seen as a way to dynamically
construct processes before reifying them as names.

Finally equipped with these standard features we can present the
dynamics of the calculus.

\subsubsection{Operational semantics} 

Finally, we introduce the computational dynamics. What marks these
algebras as distinct from other more traditionally studied algebraic
structures, e.g. vector spaces or polynomial rings, is the manner in
which dynamics is captured. In traditional structures, dynamics is typically
expressed through morphisms between such structures, as in linear maps
between vector spaces or morphisms between rings. In algebras
associated with the semantics of computation, the dynamics is
expressed as part of the algebraic structure itself, through a
reduction reduction relation typically denoted by $\red$. Below, we
give a recursive presentation of this relation for the calculus used
in the encoding.

$\red \subseteq \pi \times \pi$
$\red : \pi \to \mathcal{P}(\pi)$

\begin{mathpar}
  \inferrule* [lab=Comm] { \textsf{match}( x_{src}, x_{trgt} ) } { x_{trgt}?(y)P \; | \; x_{src}!\langle {Q} \rangle \red P\{\quotep{Q}/y}\} }
  \and \\
  \inferrule* [lab=Par] {{P} \red {P}'} {{{P} | {Q}} \red {{P}' | {Q}}}
  \and
  \inferrule* [lab=Equiv]{{{P} \scong {P}'} \andalso {{P}' \red {Q}'} \andalso {{Q}' \scong {Q}}}{{P} \red {Q}}
\end{mathpar}

\begin{eqnarray*}
  match_{\equiv} (\quotep{P},\quotep{Q}) & := & P \equiv Q \\
  match_{\dagger}(\quotep{P},\quotep{Q}) & := & \forall R. P|Q \red^{*} R => R \red^{*} 0 \\
  match_{K}(\quotep{P},\quotep{Q}) & := & K \mbox{ for some context } K
\end{eqnarray*}

$u?(x)P | u!\langle Q \rangle \red P\{\quotep{Q}/x\}$

%We write $\wred$ for $\red^*$, and $P\red$ if $\exists Q $ such that $ P \red Q$.
We write $P\red$ if $\exists Q $ such that $ P \red Q$ and $P\not\red$, otherwise.

\section{Replication}

As mentioned before, it is known that replication (and hence
recursion) can be implemented in a higher-order process algebra
\cite{SangiorgiWalker}. As our first example of calculation with the
machinery thus far presented we give the construction explicitly in
the {\rhoc}.

\begin{eqnarray}
	D_{x} & := & \prefix{x}{y}{(\binpar{\outputp{x}{y}}{@{y}})} \nonumber\\
	\bangp_{x}{P} & := & \binpar{{x}!\langle{\binpar{D_{x}}{P}}\rangle}{D_{x}} \nonumber
\end{eqnarray}

\begin{eqnarray}
	\bangp_{x}{P} & & \nonumber\\
	=
	& {x}!\langle{(\prefix{x}{y}{(\outputp{x}{y} | @{y})) | P}}\rangle 
	      | \prefix{x}{y}{(\outputp{x}{y} | @{y})} & \nonumber\\
	\red
	& (\outputp{x}{y} | @{y})\substn{\quotep{(\prefix{x}{y}{(@{y} | \outputp{x}{y})) | P}}}{y} & \nonumber\\
	=
	& \outputp{x}{\quotep{(\prefix{x}{y}{(\outputp{x}{y} | @{y})) | P}}}
	  | {(\prefix{x}{y}{(\outputp{x}{y} | @{y})) | P}} & \nonumber\\
	\red
	& \ldots & \nonumber\\
	\red^*
	& P | P | \ldots & \nonumber
\end{eqnarray}

Of course, this encoding, as an implementation, runs away, unfolding
$\bangp{P}$ eagerly. A lazier and more implementable replication
operator, restricted to input-guarded processes, may be obtained as follows.

\begin{eqnarray}
\bangp{\prefix{u}{v}{P}} 
	:= 
	\binpar{\lift{x}{\prefix{u}{v}{(\binpar{D(x)}{P})}}}{D(x)} \nonumber
\end{eqnarray}

\begin{remark}
  Note that the lazier definition still does not deal with summation
  or mixed summation (i.e. sums over input and output). The reader is
  invited to construct definitions of replication that deal with these
  features. 

  Further, the definitions are parameterized in a name, $x$. Can you,
  gentle reader, make a definition that eliminates this parameter and
  guarantees no accidental interaction between the replication
  machinery and the process being replicated -- i.e. no accidental
  sharing of names used by the process to get its work done and the
  name(s) used by the replication to effect copying. This latter
  revision of the definition of replication is crucial to obtaining
  the expected identity $!!P \sim !P$.
\end{remark}

\begin{remark}\label{rem:paradoxical_combinator}
  The reader familiar with the lambda calculus will have noticed the
  similarity between $D$ and the paradoxical combinator.

  [Ed. note: the existence of this seems to suggest we have to be more
  restrictive on the set of processes and names we admit if we are to
  support no-cloning.]
\end{remark}

\subsubsection{Bisimulation}

The computational dynamics gives rise to another kind of equivalence,
the equivalence of computational behavior. As previously mentioned
this is typically captured \emph{via} some form of bisimulation.

% The notion we use in this paper is weak barbed bisimulation
% \cite{milner91polyadicpi}.

The notion we use in this paper is derived from weak barbed
bisimulation \cite{milner91polyadicpi}. 

\begin{definition}
An \emph{observation relation}, $\downarrow_{\mathcal N}$, over a set
of names, $\mathcal N$, is the smallest relation satisfying the rules
below.

\infrule[Out-barb]{y \in {\mathcal N}, \; x \nameeq y}
		  {\outputp{x}{v} \downarrow_{\mathcal N} x}
\infrule[Par-barb]{\mbox{$P\downarrow_{\mathcal N} x$ or $Q\downarrow_{\mathcal N} x$}}
		  {\binpar{P}{Q} \downarrow_{\mathcal N} x}

We write $P \Downarrow_{\mathcal N} x$ if there is $Q$ such that 
$P \wred Q$ and $Q \downarrow_{\mathcal N} x$.
\end{definition}

\begin{definition}
%\label{def.bbisim}
An  ${\mathcal N}$-\emph{barbed bisimulation} over a set of names, ${\mathcal N}$, is a symmetric binary relation 
${\mathcal S}_{\mathcal N}$ between agents such that $P\rel{S}_{\mathcal N}Q$ implies:
\begin{enumerate}
\item If $P \red P'$ then $Q \wred Q'$ and $P'\rel{S}_{\mathcal N} Q'$.
\item If $P\downarrow_{\mathcal N} x$, then $Q\Downarrow_{\mathcal N} x$.
\end{enumerate}
$P$ is ${\mathcal N}$-barbed bisimilar to $Q$, written
$P \wbbisim_{\mathcal N} Q$, if $P \rel{S}_{\mathcal N} Q$ for some ${\mathcal N}$-barbed bisimulation ${\mathcal S}_{\mathcal N}$.
\end{definition}

$\mathcal{R} \subseteq \pi \times \pi$

$P \mathcal{R} Q => \forall P'. P \red P' \Rightarrow \exists Q'. Q \red Q', P' \mathcal{R} Q'$

$P \vdash x \Rightarrow Q \vdash x$

\begin{mathpar}
  \inferrule*[lab=Out-barb]{x \nameeq y}{{y}!\langle{Q}\rangle \vdash x}
  \and
  \inferrule*[lab=Par-barb]{\mbox{$P\vdash x$ or $Q\vdash x$}}{\binpar{P}{Q} \vdash x}
\end{mathpar}

\subsubsection{Contexts}

One of the principle advantages of computational calculi like the
$\pi$-calculus is a well-defined notion of context,
contextual-equivalence and a correlation between
contextual-equivalence and notions of bisimulation. The notion of
context allows the decomposition of a process into (sub-)process and
its syntactic environment, its context. Thus, a context may be
thought of as a process with a ``hole'' (written $\Box$) in it. The
application of a context $M$ to a process $P$, written $M[P]$, is
tantamount to filling the hole in $M$ with $P$. In this paper we do
not need the full weight of this theory, but do make use of the notion
of context in the proof the main theorem. 

\begin{mathpar}
  \inferrule* [lab=summation] {} {{M_{M},M_{N}} \bc \Box \;|\; x.M_{A} \;|\; M_{M}+M_{N}}
  \and
  \inferrule* [lab=agent] {} {{M_{A}} \bc (\vec{x})M_{P} \;| \; \clift{P_0,\ldots,M_{P},\ldots,P_N}}
  \and \\
  \inferrule* [lab=process] {} {{M_{P}} \bc M_{N} \;| \;P|M_{P} }
\end{mathpar} 

\begin{mathpar}
  \inferrule* [lab=sychronization] {} {M_{N} \bc \Box \;|\; x?M_{F} \;|\; x!M_{C}}
  \and
  \inferrule* [lab=abstraction] {} {{M_{F}} \bc (x)M_{P} }
  \and
  \inferrule* [lab=concretion] {} {{M_{C}} \bc \langle M_{P} \rangle }
  \and \\
  \inferrule* [lab=process] {} {{M_{P}} \bc M_{N} \;| \;P|M_{P} }
\end{mathpar}

\begin{definition}[contextual application] Given a context $M$, and
  process $P$, we define the \emph{contextual application}, $M[P] :=
  M\{P/\Box\}$. That is, the contextual application of M to P is the
  substitution of $P$ for $\Box$ in $M$.
\end{definition}

$\meaningof{-} : L \to \mathcal{P}(\pi)$

\begin{mathpar}
  \inferrule* [lab=collection] {} {\meaningof{true} = \pi, \and \meaningof{~E} = \pi \setminus \meaningof{E}, \and \meaningof{E_{1} \& E_{2}} = \meaningof{E_{1}} \cap \meaningof{E_{2}}}
\end{mathpar}

\begin{mathpar}
  \inferrule* [lab=structure] {} {\meaningof{0} = \{ P \in \pi | P \equiv 0 \}, \and \\ \meaningof{E_1 | E_2} = \{ P \in \pi | P \equiv P_{1} | P_{2}, P_{1} \in \meaningof{E_{1}}, P_{2} \in \meaningof{E_2}\} }
\end{mathpar}

\begin{mathpar}
 \inferrule* [lab=behavior] {} {\meaningof{\langle a?b \rangle E} = \{ P \in \pi | P \equiv Q | u?(y)P', \\ \and \\\\ \and \\ \;\;\; u \in \meaningof{a}, \forall z.P'\{z/y\} \in \meaningof{E\{z/b\}}\}, \and \\ \meaningof{a!E} = \{ P \in \pi | P \equiv Q | x!\langle P' \rangle, x \in \meaningof{a} P' \in \meaningof{E}\} }
\end{mathpar}

\begin{mathpar}
 \inferrule* [lab=nominal] {} {\meaningof{\quotep{E}} = \{ \quotep{P} \in \quotep{\pi} | P \in \meaningof{E} \}, \and \meaningof{\quotep{P}} = \{ \quotep{Q} \in \quotep{\pi} | P \equiv Q \} \and \\ \meaningof{@\quotep{E}} = \{ P \in \pi | P \equiv @x, x \in \meaningof{E} \}}
\end{mathpar}

\begin{eqnarray*}
  \\
  \meaningof{-} : TS \to ST
\end{eqnarray*}

\begin{eqnarray*}
  \\
  L : TS \to ST
\end{eqnarray*}

\begin{eqnarray*}
  \\
  P \models E \iff P \in \meaningof{E}
\end{eqnarray*}

\begin{eqnarray*}
  P \approx_{L} Q \iff \forall E \in L. P \models E \iff Q \models E
\end{eqnarray*}

\begin{eqnarray*}
  P \approx_{K} Q
\end{eqnarray*}

\begin{eqnarray*}
  P \approx Q
\end{eqnarray*}

$\approx_{K} = \approx = \approx_{L}$

\subsubsection{Contextual duality}

Note that contexts extend the quotation operation to a family of
operations from processes to names. Given a context, $M$, we can
define a \emph{nominal context}, $\quotep{M}$ by $\quotep{M}[P] :=
\quotep{M[P]}$. To foreshadow what is to come we observe that these
operations enjoy a duality with processes very much like the duality
between vectors and maps from vectors to scalars.

Further, because the calculus is essentially higher-order, we have a
correspondence between contexts and processes. More specifically,
given a name $x$ and a context $M$ we can construct $M^{*}_{x}$ such
that 

\begin{mathpar}
  M^{*}_{x} | \lift{x}{P} \red M[P]
\end{mathpar}

namely,

\begin{mathpar}
  M^{*}_{x} := x?(u).M[\dropn{u}]
\end{mathpar}

The dependence of $M^{*}_{x}$ on a name makes it an abstraction, 

\begin{mathpar}
  M^{*} := (x)x?(u).M[\dropn{u}]
\end{mathpar}

\subsection{Additional notation}

It will sometimes be convenient to denote the process a name
quotes. We already have the notation $x = \quotep{P}$, but it will be
convenient to introduce an alternate notation, $\procn{x}$, when we
want to emphasize the connection to the use of the name. Note that, by
virtue of name equivalence, $\quotep{\procn{x}} \nameeq x$; so, the
notation is consistent with previous definitions.

Further, because names have structure it is possible to effect
substitutions on the basis of that structure. This means we need to
upgrade our notation for substitutions, which we accomplish by
adapting comprehension notation. Thus,

\begin{mathpar}
  P\{ y / x : x \in S \}
\end{mathpar}

is interpreted to mean the process derived from P by replacing (in a
capture-avoiding manner) each occurrence of $x$ in $S$ by $y$. For example,

\begin{mathpar}
  P\{ \quotep{\procn{x}|\procn{x}} / x : x \in \freenames{P} \}
\end{mathpar}

will replace each (occurrence) of a free name $x$ in $P$ by
$\quotep{\procn{x}|\procn{x}}$.

Also, we will avail ourselves of the notation $x^{L}$ and $x^{R}$ to
denote injections of a name into disjoint copies of the name
space. There are numerous ways to accomplish this. One example can be
found in \cite{MeredithR05}. This notation overloads to vectors of
names: $\vec{x}^{\pi} := (x_{i}^{\pi} \; : \; 0 \leq i < |\vec{x}| )$ where $\pi \in \{L,R\}$.

We also use $P^{\Box} := P|\Box$.

In \cite{MeredithR05} an interpretation of the new operator is
given. It turns out that there are several possible interpretations
all enjoying the requisite algebraic properties of the operator (see
\cite{milner91polyadicpi}). We will therefore make liberal use of
$(\nu\; \vec{x})P$.

% subsection the_syntax_and_semantics_of_the_notation_system (end)   

\section{Interpretation of QM}
\subsection{Supporting definitions}
\subsubsection{Multiplication}
\begin{mathpar}
  \quotep{Q} \cdot \quotep{R} := \quotep{Q|R}
  \and \\
  \quotep{Q} \cdot P := P\{ \quotep{Q|R} / \quotep{R} : \quotep{R} \in \freenames{P} \}
\end{mathpar}

\paragraph{Discussion}
The first line needs little explanation. The second line says that
each free name of the process is replaced with the multiplication of
that name by the scalar. Multiplication of a scalar (name) by a state
(process) results in a process all the names of which have been `moved
over' by parallel composition with the process the scalar
quotes. There is a subtlety that the bound names have to be
manipulated so that multiplied names aren't accidentally
captured. There are many ways to achieve this.

\begin{remark}\label{rem:multiplication_identities}
  The reader is invited to verify that for all $x,y,z \in \QProc$ and $P \in \Proc$
  \begin{mathpar}
    x \cdot \quotep{0} \equiv x 
    \and
    x \cdot y \equiv y \cdot x
    \and
    x \cdot (y \cdot z) \equiv (x \cdot y) \cdot z
    \and \\
    \quotep{0} \cdot P \equiv P
    \and \\
    x \cdot (y \cdot P) \equiv (x \cdot y) \cdot P
    \and \\
    x \cdot (P|Q) \equiv (x \cdot P) | (x \cdot Q)
    \and \\    
  \end{mathpar}
\end{remark}

\subsubsection{Tensor product}

We define a tensor product on processes by structural induction.

\paragraph{Tensor of sums} First note that all summations, including
$\pzero$ and sequence, can be written $\Sigma_{i} x_{i}.A_{i} +
\Sigma_{j} x_{j}.C_{j}$, where we have grouped input-guarded processes
together and output-guarded processes together.

Thus, we can define the tensor product of two summations, $N_{1}\otimes N_{2}$, where

\begin{mathpar}
  N_{1} := \Sigma_{i} x_{i}.A_{i} + \Sigma_{j} x_{j}.C_{j}
  \and
  N_{2} := \Sigma_{i'} y_{i'}.B_{i'} + \Sigma_{j'} y_{j'}.D_{j'} 
\end{mathpar}

as follows.

\begin{mathpar}
  \Sigma_{i} x_{i}.A_{i} + \Sigma_{j} x_{j}.C_{j} \otimes \Sigma_{i'}
  y_{i'}.B_{i'} + \Sigma_{j'} y_{j'}.D_{j'} 
  \and \\
  := \; \Sigma_{i} \Sigma_{i'} \quotep{\stackrel{\vee}{x_{i}}| \stackrel{\vee}{y_{i'}}}.(A_{i}\otimes B_{i'}) \; | \; \Sigma_{i'} \Sigma_{i} \quotep{\stackrel{\vee}{y_{i'}}|\stackrel{\vee}{x_{i}}}.(B_{i'}\otimes A_{i})
  \and
  \;\; | \;\; \Sigma_{j} \Sigma_{j'} \quotep{\stackrel{\vee}{x_{j}}|\stackrel{\vee}{y_{j'}}}.(A_{j}\otimes B_{j'}) \; | \; \Sigma_{j'} \Sigma_{j} \quotep{\stackrel{\vee}{y_{j'}}|\stackrel{\vee}{x_{j}}}.(B_{j'}\otimes A_{j})
\end{mathpar}

\begin{remark}
  Do we need to $x^{L}$ and $y^{R}$ for this construction as well?
\end{remark}

\paragraph{Tensor of parallel compositions} Next, we distribute tensor
over par.

\begin{mathpar}
  P_{1}|P_{2} \otimes Q_{1}|Q_{2} := (P_{1} \otimes Q_{1}) | (P_{1}
  \otimes Q_{2}) | (P_{2} \otimes Q_{1}) | (P_{2} \otimes Q_{2})
\end{mathpar}

\paragraph{Tensor with dropped names} We treat tensor of a
process with a dropped name as parallel composition.

\begin{mathpar}
  P \otimes \dropn{x} := P | \dropn{x}
\end{mathpar}

\paragraph{Tensor of agents}

Finally, we need to define tensor on agents. Note that the definition
of tensor on normal products only tensors inputs with inputs and
outputs with outputs. Thus, we only have to define the operation on
``homogeneous'' pairings.

\begin{mathpar}
  (\vec{x})P \otimes (\vec{y})Q
  \and \\
  := (x_{0}^{L}|y_{0}^{R},\ldots,x_{0}^{L}|y_{n}^{R},\ldots,x_{m}^{L}|y_{0}^{R},\ldots,x_{m}^{L}|y_{n}^R)(P\{ \vec{x}^{L}/\vec{x}\} \otimes Q \{ \vec{y}^{R}/\vec{y}\})
  \and \\
  \clift{\vec{P}} \otimes \clift{\vec{Q}}
  \and \\
  := \clift{P_{0}\otimes Q_{0},\ldots,P_{0}\otimes Q_{n},\ldots,P_{m}\otimes Q_{0},\ldots,P_{m}\otimes Q_{n}}
\end{mathpar}

\begin{remark}
  Observe that arities of tensored abstractions matches arities of
  tensored concretions if the original arities matched. Note also that
  the length of the arities corresponds to the increase in dimension
  we see in ordinary vector space tensor product.
\end{remark}

\begin{remark}
  Operationally, this definition distributes the tensor down to
  components ``linked'' by summation. Tensor over summation is
  intriguing in that it mixes names. Moreover, as a consequence of the
  way it mixes names we have the identities for all $x \in \QProc$ and
  $P,Q \in \Proc$

  \begin{mathpar}
    (x \cdot P) \otimes Q \equiv x \cdot (P \otimes Q) \equiv P \otimes (x \cdot Q)
    \and
    P \otimes \pzero \equiv P
  \end{mathpar}

  that the reader is invited to verify.
\end{remark}

\subsubsection{Annihilation}
\begin{mathpar}
  P^{\perp} := \{ Q | \forall R. P|Q \red^{*} R \Rightarrow R \red^{*} \pzero \}
  \and \\
  P^{\underline{\perp}} := \Sigma_{Q \in P^{\perp}} \quotep{Q}?(y).(\dropn{y}|Q) | \Sigma_{Q \in P^{\perp}} \quotep{Q}\clift{\Box}
\end{mathpar}

\paragraph{Discussion} The reader will note that $P^{\perp}$ is a
\emph{set} of processes, while $P^{\underline{\perp}}$ is a
\emph{context}. We call the set $P^{\perp}$ the \emph{annihilators} of
$P$. The parallel composition of a process in the annihilators of $P$
with $P$ will result in a process, the state space of which has all
paths eventually leading to $\pzero$. Execution may endure loops; but
under reasonable conditions of fairness (naturally guaranteed under
most notions of bisimulation) such a composite process cannot get
stuck in such a loop and will, eventually pop out and terminate.

The context $P^{\underline{\perp}}$ is ready and willing to ``take the
$P$ out of'' the process to which it is applied. It will effectively
transmit the code of the process to which it is applied to one of the
annihilators and run the process against it.

\subsubsection{Evaluation}
We fix $M$ a domain of fully abstract interpretation with an equality
coincident with bisimulation. We take $\meaningof{\cdot} : \Proc \to
M$ to be the map interpreting processes and $\nmeaningof{\cdot} : \M
\to Proc$ to be the map running the other way. Then we define

\begin{mathpar}
  \int P := \nmeaningof{\meaningof{P}}
\end{mathpar}

\paragraph{Discussion}
There are many fully abstract interpretations of Milner's
$\pi$-calculus. Any of them can be used as a basis for interpreting
the reflective calculus here. Equipped with such a domain it is
largely a matter of grinding through to check that the Yoneda
construction for the normalization-by-evaluation program can be
extended to this setting.

\begin{remark}
  The reader is invited to verify that $\int (P^{\underline{\perp}}[P]) = 0$.
\end{remark}

\subsection{Quantum mechanics}

Table \ref{tbl:core_qm_op_defns} gives the core operational definitions

\begin{table}[htp]\label{tbl:core_qm_op_defns}
  \center{
    \fbox{
      \begin{tabular}{c|c}
        quantum mechanics & process calculus \\
        \hline
        scalar & $x := \quotep{P}$ \\
        state vector & $\state{P} := P$ \\
        dual & $\state{P}^{*} := \event{P^{\underline{\perp}}} := \quotep{P^{\underline{\perp}}}[-]$ \\
        matrix & $ \Sigma_{\alpha} \state{P_{\alpha}}x_{\alpha}\event{Q_{\alpha}}$ \\
        vector addition & $\state{P} + \state{Q} := \state{P | Q}$ \\
        tensor product & $\state{P} \otimes \state{Q} := \state{P \otimes Q}$ \\
        inner product & $\innerprod{P}{Q} := \quotep{\int P^{\underline{\perp}}[Q]}$ \\
      \end{tabular}
    }
  }
  \caption{QM - operational definitions}
\end{table}

where

\begin{mathpar}
  \prmatrix{P}{Q} := \fprmatrix{P}{\quotep{\pzero}}{Q}
  \and
  \fprmatrix{P}{x}{Q} := (\state{P},x,\event{Q})
  \and
  (\fprmatrix{P}{x}{Q})(\state{R}) := x \cdot \innerprod{Q}{R} \cdot \state{P}
  \and
  (\fprmatrix{P}{x}{Q})(\event{R}) := x \cdot \innerprod{R}{P} \cdot \event{Q}
\end{mathpar}

\paragraph{Discussion}
As promised: vectors (aka states) are represented as processes; duals
as contextual duals; inner product definition should be compared with
standard inner product definition for ....

\begin{remark}
  Assuming $\int (P^{\underline{\perp}}[P]) = 0$, the reader is
  invited to verify that $(\fprmatrix{P}{x}{P})(\state{P}) = x \cdot \state{P}$.
\end{remark}

\begin{remark}
  The reader is invited to verify that $\innerprod{P}{Q}$ could
  equally well have been written $\quotep{\int \stackrel{\vee}{x}}$
  where $x = \event{P^{\underline{\perp}}}(Q)$.

  One of the motivations for this remark is that there is another way
  to factor these operations. We could package up evaluation in the dual:

  \begin{mathpar}
    \state{P}^{*} := \event{\int P^{\underline{\perp}}} := \quotep{\int P^{\underline{\perp}}}[-]
  \end{mathpar}

  and then have inner product defined by
  
  \begin{mathpar}
    \innerprod{P}{Q} := \event{P}(Q)
  \end{mathpar}

  Hopefully, experience with the calculations will provide guidance on
  the best factoring.
\end{remark}

\begin{remark}
  Assuming $\int (P^{\underline{\perp}}[P]) = 0$, the reader is
  invited to verify that $\forall P,Q. (\prmatrix{0}{Q})(\state{0}) =
  \state{0}$ and dually $(\prmatrix{P}{0})(\event{0}) = \event{0}$.
\end{remark}

\begin{remark}
  i'm a little worried that i don't (yet) have proper support for
  complex conjugacy. But, the observation above may give us a
  clue. According to Abramsky, it must be the case that the scalars
  are iso to the homset of the identity for the tensor -- which the
  observation above characterizes. 

  For now, we will simply bookmark the notion with $\overline{x}$.
\end{remark}

\subsubsection{Adjointness}

We need to give a definition of $(\cdot)^{\dagger}$ for matrices. The
obvious candidate definition is
\begin{mathpar}
(\Sigma_{\alpha}\fprmatrix{P_{\alpha}}{x_{\alpha}}{Q_{\alpha}})^{\dagger}
= \Sigma_{\alpha}\fprmatrix{(Q_{\alpha}^{\underline{\perp}})^{*}}{\overline{x}_{\alpha}}{P_{\alpha}^{\underline{\perp}}} 
\end{mathpar}

But, $(Q_{\alpha}^{\underline{\perp}})^{*}$ requires a name along
which to communicate the process to achieve the context application.

\subsubsection{Basis for a basis}
If processes label states and ``addition'' of states (a.k.a. vector
addition) is interpreted as parallel composition, what corresponds to
notions of linear independence and basis? Here, we recall that Yoshida
has developed a set of \emph{combinators} for an asynchronous verison
of Milner's $\pi$-calculus. These are a finite set of processes such
any process can be expressed as parallel composition of these
combinators together with liberal uses of the new operator and
replication. We can simply give a translation of these into the
present calculus and have reasonable expectation that the property
carries over. That is, that the resultant set allows to express all
processes via parallel composition. Note, however, that there is no
new operator or replication in this calculus. As a result, we expect
that the corresponding set is actually infinite. That is, we expect
that the space is actually infinite dimensional.

\begin{remark}
  The attentive reader may be a bit concerned. Certainly, the
  collection $S$, $K$ and $I$ is a finite set of
  combinators. Shouldn't we expect to see a finite set of combinators
  for an effectively equivalent system? i am very sympathetic to this
  critique and feel it warrants full attention. On the other hand, i
  also have in mind the following analogy. The natural numbers, as a
  monoid under addition, has exactly $1$ generator, while the natural
  numbers, as a monoid under multiplication, has countably many
  generators (the primes). We observe that the application of the
  lambda calculus is much less resource sensitive than the parallel
  composition of the $\pi$-calculus. Could it be the case that we have
  an analogy of the form
  
  \begin{mathpar}
    m + n : MN :: m*n : M|N
  \end{mathpar}

  giving a similar blow up in the set of ``primes''?  This is such a
  wonderful thought that, even if it's not true, i think it's worth
  writing down.
\end{remark}
 

\documentclass[12pt]{llncs}
%\documentclass{jktr}

\usepackage[pdftex]{hyperref}                   
\usepackage {listings}
\usepackage {mathpartir}
\usepackage{bcprules}
%\usepackage{listings}
                       
\usepackage{graphicx} 
%\usepackage[margins=2.5cm,nohead,nofoot]{geometry}
%\usepackage{geometry}
\usepackage{amsfonts}
\usepackage{amstext}
\usepackage{latexsym}
\usepackage{amssymb}
\usepackage{color}


%\include{myPreamble}
\include{qm2pi.local} 

%\ifpdf
%\usepackage[pdftex]{graphicx}
%\else
%\usepackage{graphicx}
%\fi

 % \ifpdf
%  \usepackage{pdfsync}
%  \if


%\title{Brief Article}
%\author{David F. Snyder}
%\author{L.G. Meredith}

%\address{Dept. of Math., Texas State University--San Marcos, San Marcos, TX 78666}
       
\pagestyle{empty}


\begin{document}

\lstset{language=[Objective]Caml,frame=shadowbox}

\input{qm2pi.front}

% section front matter (end)

\input{qm2pi.intro} 
 
% section introduction (end)

% \input{qm2pi.knotations} 

% section notation (end)

\input{qm2pi.process.calculi} 

% section concurrent_process_calculi_and_spatial_logics_ (end)
    
%\input{qm2pi.knots2pi} 

%\input{qm2pi.trefoil} 

%\input{qm2pi.mainthm} 

% subsection basic_interpretation (end)

%\input{qm2pi.rho.presentation} 
\subsection{The syntax and semantics of the notation system}\label{sub:the_syntax_and_semantics_of_the_notation_system} % (fold)

We now summarize a technical presentation of the calculus that
embodies our theory of dynamics. The typical presentation of such a
calculus follows the style of giving generators and relations on
them. The grammar, below, describing term constructors, freely
generates the set of processes, $\Proc$. This set is then quotiented
by a relation known as structural congruence and it is over this set
that the notion of dynamics is expressed. This presentation is
essentially that of \cite{MeredithR05} with the addition of
polyadicity and summation. For readability we have relegated some of
the technical subtleties to an appendix.

\subsubsection{Process grammar}\label{subsub:process_grammar}

\begin{mathpar}
  \inferrule* [lab=synchronization] {} {{M} \bc \pzero \;|\; x?F \;|\; x!C }
  \and
  \inferrule* [lab=abstraction] {} {{F} \bc (x)P}
  \and
  \inferrule* [lab=concretion] {} {{C} \bc \langle Q \rangle}
  \and
  \inferrule* [lab=process] {} {{P,Q} \bc M \;| \;P|Q \;|\; @{x}}
  \and
  \inferrule* [lab=name] {} {{x} \bc \quotep{P}}
\end{mathpar} 

Note that $\vec{x}$ (resp. $\vec{P}$) denotes a vector of names
(resp. processes) of length $|\vec{x}|$ (resp. $|\vec{P}|$). We adopt
the following useful abbreviations.

\begin{mathpar}
   x?(\vec{y}).P := x.(\vec{y})P \and  x\clift{\vec{P}} := x.\clift{\vec{P}}
   \and x!(y) := \lift{x}{\dropn{y}}
   \and \Pi_{i=0}^{n-1}P_i := P_0 | \ldots | P_{n-1}
\end{mathpar}

\subsubsection{Structural congruence}

\paragraph{Free and bound names and alpha-equivalence.} At the
core of structural equivalence is alpha-equivalence which identifies
process that are the same up to a change of variable. Formally, we
recognize the distinction between free and bound names. The free names
of a process, $\freenames{P}$, may be calculated recursively as
follows:

\begin{mathpar}
\freenames{\pzero} := \emptyset
  \and \\
  \freenames{x?(y).P} := \{ x \} \cup (\freenames{P} \setminus \{ y \})
  \and 
  \freenames{x!\langle P \rangle} := \{ x \} \cup \{ P \} 
  \and \\
  \freenames{P|Q} := \freenames{P} \cup \freenames{Q}
  \and \\
  \freenames{@{x}} := \{ x \}
\end{mathpar}

$\pi$
$\quotep{\pi}$

$\freenames{-} : \pi \to \mathcal{P}(\quotep{\pi})$

\begin{eqnarray*}
  \freenames{\pzero} & := & \emptyset \\
  \freenames{x?(y).P} & := & \{ x \} \cup (\freenames{P} \setminus \{ y \}) \\
  \freenames{x!\langle P \rangle} & := & \{ x \} \cup \{ P \} \\
  \freenames{P|Q} & := & \freenames{P} \cup \freenames{Q} \\
  \freenames{\dropn{x}} & := & \{ x \}
\end{eqnarray*}

The bound names of a process, $\boundnames{P}$, are those names occurring in $P$
that are not free. For example, in $x?(y).0$, the name $x$ is free, while $y$ is bound.

\begin{mathpar}
  \inferrule* [lab=monoidal-laws] {} { P|Q \equiv Q|P \and P|0 \equiv P \and P|(Q|R) \equiv (P|Q)|R }
\end{mathpar}

\begin{mathpar}
  \inferrule* [lab=alpha-equivalence] {} { (x)P \equiv (y)P\{y/x\} \and y \not\in \freenames{P} }
\end{mathpar}

\begin{definition}
Then two processes, $P,Q$, are alpha-equivalent if $P = Q\{\vec{y}/\vec{x}\}$ for
some $\vec{x} \in \boundnames{Q},\vec{y} \in \boundnames{P}$, where $Q\{\vec{y}/\vec{x}\}$
denotes the capture-avoiding substitution of $\vec{y}$ for $\vec{x}$ in $Q$.
\end{definition}

\begin{definition}
  The {\em structural congruence} \cite{SangiorgiWalker} , $\equiv$,
  between processes is the least congruence containing
  alpha-equivalence, satisfying the abelian monoid laws
  (associativity, commutativity and $\pzero$ as identity) for parallel
  composition $|$ and for summation $+$.
\end{definition}

\subsection{Name equivalence}

We take name equivalence, written $\nameeq$, to be the smallest
equivalence relation generated by the following rules.

\begin{mathpar}
\inferrule*[lab=Quote-drop]
{ }
{ \quotep{@{x}} \nameeq x }

\inferrule*[lab=Struct-equiv]
{ P \scong Q }
{ \quotep{P} \nameeq \quotep{Q} }
\end{mathpar}

The astute reader will have noticed that the mutual recursion of names
and processes imposes a mutual recursion on alpha-equivalence and
structural equivalence via name-equivalence. Fortunately, all of this
works out pleasantly and we may calculate in the natural way, free of
concern. The reader interested in the details is referred to the
appendix \ref{appendix:rho_details}.

\subsection{Substitution}

We use $\Proc$ for the set of processes, $\QProc$ for the set of
names, and $\id{\{}\vec{y} / \vec{x} \id{\}}$ to denote partial maps,
$s : \QProc \rightarrow \QProc$. A map, $s$ lifts, uniquely, to a map
on process terms, $\widehat{s} : \Proc \rightarrow \Proc$ by the
following equations.

\begin{mathpar}
  (0) \psubstp{Q}{P} := 0 \\
  (R \juxtap S) \psubstp{Q}{P}
  :=    
  (R)\psubstp{Q}{P} \juxtap (S) \psubstp{Q}{P} \\
  (x?(y).R) \psubstp{Q}{P}    
  :=    
  (x)\substp{Q}{P} (z)\concat( (R \psubstn{z}{y}) \psubstp{Q}{P} ) \\
  (\lift{x}{R}) \psubstp{Q}{P}  
  :=
  \lift{(x)\substp{Q}{P}}{ R \psubstp{Q}{P} } \\
%   (\dropn{x})  \psubstp{Q}{P}       
%   := 
%   \left\{ 
%     \begin{array}{ccc} 
%       \dropn{\quotep{Q}} & & x \nameeq \quotep{P} \\
%       \dropn{x} & & otherwise \\
%     \end{array}
%   \right. 
  (\dropn{x})  \psubstp{Q}{P}       
  := 
  \left\{ 
    \begin{array}{ccc} 
      Q & & x \nameeq \quotep{P} \\
      \dropn{x} & & otherwise \\
    \end{array}
  \right.
\end{mathpar}
 

where

\begin{eqnarray}
  (x)\id{\{} \lpquote Q \rpquote / \lpquote P \rpquote \id{\}}            = 
  \left\{ 
    \begin{array}{ccc}
      \lpquote Q \rpquote & & x \nameeq \lpquote P \rpquote \\
      x & & otherwise \\
    \end{array}
  \right. \nonumber
\end{eqnarray}

and $z$ is chosen distinct from $\quotep{P}$, $\quotep{Q}$, the free
names in $Q$, and all the names in $R$. Our $\alpha$-equivalence will
be built in the standard way from this substitution.

\begin{remark}\label{rem:no_self_referential_names}
  One consequence of these definitions is that $\forall P. \quotep{P}
  \not\in \freenames{P}$.
\end{remark}

\subsection{ Dynamic quote: an example }

Anticipating something of what's to come, consider applying the
substitution, $\widehat{\id{\{}u / z \id{\}}}$, to the following pair
of processes, $\lift{w}{y!(z)}$ and $w[ \lpquote y!(z) \rpquote ]$.

\begin{eqnarray}
	\lift{w}{y!(z)}\widehat{\id{\{}u / z \id{\}}}
		& = &
		\lift{w}{y!(u)} \nonumber\\
	w[ \lpquote y!(z) \rpquote ] \widehat{ \id{\{}u / z \id{\}} }
		& = &
		w[ \lpquote y!(z) \rpquote ] \nonumber
\end{eqnarray}

Because the body of the process between quotes is impervious to
substitution, we get radically different answers. In fact, by
examining the first process in an input context,
e.g. $x?(z).\lift{w}{y!(z)}$, we see that the process under the lift
operator may be shaped by prefixed inputs binding a name inside it. In
this sense, the lift operator will be seen as a way to dynamically
construct processes before reifying them as names.

Finally equipped with these standard features we can present the
dynamics of the calculus.

\subsubsection{Operational semantics} 

Finally, we introduce the computational dynamics. What marks these
algebras as distinct from other more traditionally studied algebraic
structures, e.g. vector spaces or polynomial rings, is the manner in
which dynamics is captured. In traditional structures, dynamics is typically
expressed through morphisms between such structures, as in linear maps
between vector spaces or morphisms between rings. In algebras
associated with the semantics of computation, the dynamics is
expressed as part of the algebraic structure itself, through a
reduction reduction relation typically denoted by $\red$. Below, we
give a recursive presentation of this relation for the calculus used
in the encoding.

$\red \subseteq \pi \times \pi$
$\red : \pi \to \mathcal{P}(\pi)$

\begin{mathpar}
  \inferrule* [lab=Comm] { \textsf{match}( x_{src}, x_{trgt} ) } { x_{trgt}?(y)P \; | \; x_{src}!\langle {Q} \rangle \red P\{\quotep{Q}/y}\} }
  \and \\
  \inferrule* [lab=Par] {{P} \red {P}'} {{{P} | {Q}} \red {{P}' | {Q}}}
  \and
  \inferrule* [lab=Equiv]{{{P} \scong {P}'} \andalso {{P}' \red {Q}'} \andalso {{Q}' \scong {Q}}}{{P} \red {Q}}
\end{mathpar}

\begin{eqnarray*}
  match_{\equiv} (\quotep{P},\quotep{Q}) & := & P \equiv Q \\
  match_{\dagger}(\quotep{P},\quotep{Q}) & := & \forall R. P|Q \red^{*} R => R \red^{*} 0 \\
  match_{K}(\quotep{P},\quotep{Q}) & := & K \mbox{ for some context } K
\end{eqnarray*}

$u?(x)P | u!\langle Q \rangle \red P\{\quotep{Q}/x\}$

%We write $\wred$ for $\red^*$, and $P\red$ if $\exists Q $ such that $ P \red Q$.
We write $P\red$ if $\exists Q $ such that $ P \red Q$ and $P\not\red$, otherwise.

\section{Replication}

As mentioned before, it is known that replication (and hence
recursion) can be implemented in a higher-order process algebra
\cite{SangiorgiWalker}. As our first example of calculation with the
machinery thus far presented we give the construction explicitly in
the {\rhoc}.

\begin{eqnarray}
	D_{x} & := & \prefix{x}{y}{(\binpar{\outputp{x}{y}}{@{y}})} \nonumber\\
	\bangp_{x}{P} & := & \binpar{{x}!\langle{\binpar{D_{x}}{P}}\rangle}{D_{x}} \nonumber
\end{eqnarray}

\begin{eqnarray}
	\bangp_{x}{P} & & \nonumber\\
	=
	& {x}!\langle{(\prefix{x}{y}{(\outputp{x}{y} | @{y})) | P}}\rangle 
	      | \prefix{x}{y}{(\outputp{x}{y} | @{y})} & \nonumber\\
	\red
	& (\outputp{x}{y} | @{y})\substn{\quotep{(\prefix{x}{y}{(@{y} | \outputp{x}{y})) | P}}}{y} & \nonumber\\
	=
	& \outputp{x}{\quotep{(\prefix{x}{y}{(\outputp{x}{y} | @{y})) | P}}}
	  | {(\prefix{x}{y}{(\outputp{x}{y} | @{y})) | P}} & \nonumber\\
	\red
	& \ldots & \nonumber\\
	\red^*
	& P | P | \ldots & \nonumber
\end{eqnarray}

Of course, this encoding, as an implementation, runs away, unfolding
$\bangp{P}$ eagerly. A lazier and more implementable replication
operator, restricted to input-guarded processes, may be obtained as follows.

\begin{eqnarray}
\bangp{\prefix{u}{v}{P}} 
	:= 
	\binpar{\lift{x}{\prefix{u}{v}{(\binpar{D(x)}{P})}}}{D(x)} \nonumber
\end{eqnarray}

\begin{remark}
  Note that the lazier definition still does not deal with summation
  or mixed summation (i.e. sums over input and output). The reader is
  invited to construct definitions of replication that deal with these
  features. 

  Further, the definitions are parameterized in a name, $x$. Can you,
  gentle reader, make a definition that eliminates this parameter and
  guarantees no accidental interaction between the replication
  machinery and the process being replicated -- i.e. no accidental
  sharing of names used by the process to get its work done and the
  name(s) used by the replication to effect copying. This latter
  revision of the definition of replication is crucial to obtaining
  the expected identity $!!P \sim !P$.
\end{remark}

\begin{remark}\label{rem:paradoxical_combinator}
  The reader familiar with the lambda calculus will have noticed the
  similarity between $D$ and the paradoxical combinator.

  [Ed. note: the existence of this seems to suggest we have to be more
  restrictive on the set of processes and names we admit if we are to
  support no-cloning.]
\end{remark}

\subsubsection{Bisimulation}

The computational dynamics gives rise to another kind of equivalence,
the equivalence of computational behavior. As previously mentioned
this is typically captured \emph{via} some form of bisimulation.

% The notion we use in this paper is weak barbed bisimulation
% \cite{milner91polyadicpi}.

The notion we use in this paper is derived from weak barbed
bisimulation \cite{milner91polyadicpi}. 

\begin{definition}
An \emph{observation relation}, $\downarrow_{\mathcal N}$, over a set
of names, $\mathcal N$, is the smallest relation satisfying the rules
below.

\infrule[Out-barb]{y \in {\mathcal N}, \; x \nameeq y}
		  {\outputp{x}{v} \downarrow_{\mathcal N} x}
\infrule[Par-barb]{\mbox{$P\downarrow_{\mathcal N} x$ or $Q\downarrow_{\mathcal N} x$}}
		  {\binpar{P}{Q} \downarrow_{\mathcal N} x}

We write $P \Downarrow_{\mathcal N} x$ if there is $Q$ such that 
$P \wred Q$ and $Q \downarrow_{\mathcal N} x$.
\end{definition}

\begin{definition}
%\label{def.bbisim}
An  ${\mathcal N}$-\emph{barbed bisimulation} over a set of names, ${\mathcal N}$, is a symmetric binary relation 
${\mathcal S}_{\mathcal N}$ between agents such that $P\rel{S}_{\mathcal N}Q$ implies:
\begin{enumerate}
\item If $P \red P'$ then $Q \wred Q'$ and $P'\rel{S}_{\mathcal N} Q'$.
\item If $P\downarrow_{\mathcal N} x$, then $Q\Downarrow_{\mathcal N} x$.
\end{enumerate}
$P$ is ${\mathcal N}$-barbed bisimilar to $Q$, written
$P \wbbisim_{\mathcal N} Q$, if $P \rel{S}_{\mathcal N} Q$ for some ${\mathcal N}$-barbed bisimulation ${\mathcal S}_{\mathcal N}$.
\end{definition}

$\mathcal{R} \subseteq \pi \times \pi$

$P \mathcal{R} Q => \forall P'. P \red P' \Rightarrow \exists Q'. Q \red Q', P' \mathcal{R} Q'$

$P \vdash x \Rightarrow Q \vdash x$

\begin{mathpar}
  \inferrule*[lab=Out-barb]{x \nameeq y}{{y}!\langle{Q}\rangle \vdash x}
  \and
  \inferrule*[lab=Par-barb]{\mbox{$P\vdash x$ or $Q\vdash x$}}{\binpar{P}{Q} \vdash x}
\end{mathpar}

\subsubsection{Contexts}

One of the principle advantages of computational calculi like the
$\pi$-calculus is a well-defined notion of context,
contextual-equivalence and a correlation between
contextual-equivalence and notions of bisimulation. The notion of
context allows the decomposition of a process into (sub-)process and
its syntactic environment, its context. Thus, a context may be
thought of as a process with a ``hole'' (written $\Box$) in it. The
application of a context $M$ to a process $P$, written $M[P]$, is
tantamount to filling the hole in $M$ with $P$. In this paper we do
not need the full weight of this theory, but do make use of the notion
of context in the proof the main theorem. 

\begin{mathpar}
  \inferrule* [lab=summation] {} {{M_{M},M_{N}} \bc \Box \;|\; x.M_{A} \;|\; M_{M}+M_{N}}
  \and
  \inferrule* [lab=agent] {} {{M_{A}} \bc (\vec{x})M_{P} \;| \; \clift{P_0,\ldots,M_{P},\ldots,P_N}}
  \and \\
  \inferrule* [lab=process] {} {{M_{P}} \bc M_{N} \;| \;P|M_{P} }
\end{mathpar} 

\begin{mathpar}
  \inferrule* [lab=sychronization] {} {M_{N} \bc \Box \;|\; x?M_{F} \;|\; x!M_{C}}
  \and
  \inferrule* [lab=abstraction] {} {{M_{F}} \bc (x)M_{P} }
  \and
  \inferrule* [lab=concretion] {} {{M_{C}} \bc \langle M_{P} \rangle }
  \and \\
  \inferrule* [lab=process] {} {{M_{P}} \bc M_{N} \;| \;P|M_{P} }
\end{mathpar}

\begin{definition}[contextual application] Given a context $M$, and
  process $P$, we define the \emph{contextual application}, $M[P] :=
  M\{P/\Box\}$. That is, the contextual application of M to P is the
  substitution of $P$ for $\Box$ in $M$.
\end{definition}

$\meaningof{-} : L \to \mathcal{P}(\pi)$

\begin{mathpar}
  \inferrule* [lab=collection] {} {\meaningof{true} = \pi, \and \meaningof{~E} = \pi \setminus \meaningof{E}, \and \meaningof{E_{1} \& E_{2}} = \meaningof{E_{1}} \cap \meaningof{E_{2}}}
\end{mathpar}

\begin{mathpar}
  \inferrule* [lab=structure] {} {\meaningof{0} = \{ P \in \pi | P \equiv 0 \}, \and \\ \meaningof{E_1 | E_2} = \{ P \in \pi | P \equiv P_{1} | P_{2}, P_{1} \in \meaningof{E_{1}}, P_{2} \in \meaningof{E_2}\} }
\end{mathpar}

\begin{mathpar}
 \inferrule* [lab=behavior] {} {\meaningof{\langle a?b \rangle E} = \{ P \in \pi | P \equiv Q | u?(y)P', \\ \and \\\\ \and \\ \;\;\; u \in \meaningof{a}, \forall z.P'\{z/y\} \in \meaningof{E\{z/b\}}\}, \and \\ \meaningof{a!E} = \{ P \in \pi | P \equiv Q | x!\langle P' \rangle, x \in \meaningof{a} P' \in \meaningof{E}\} }
\end{mathpar}

\begin{mathpar}
 \inferrule* [lab=nominal] {} {\meaningof{\quotep{E}} = \{ \quotep{P} \in \quotep{\pi} | P \in \meaningof{E} \}, \and \meaningof{\quotep{P}} = \{ \quotep{Q} \in \quotep{\pi} | P \equiv Q \} \and \\ \meaningof{@\quotep{E}} = \{ P \in \pi | P \equiv @x, x \in \meaningof{E} \}}
\end{mathpar}

\begin{eqnarray*}
  \\
  \meaningof{-} : TS \to ST
\end{eqnarray*}

\begin{eqnarray*}
  \\
  L : TS \to ST
\end{eqnarray*}

\begin{eqnarray*}
  \\
  P \models E \iff P \in \meaningof{E}
\end{eqnarray*}

\begin{eqnarray*}
  P \approx_{L} Q \iff \forall E \in L. P \models E \iff Q \models E
\end{eqnarray*}

\begin{eqnarray*}
  P \approx_{K} Q
\end{eqnarray*}

\begin{eqnarray*}
  P \approx Q
\end{eqnarray*}

$\approx_{K} = \approx = \approx_{L}$

\subsubsection{Contextual duality}

Note that contexts extend the quotation operation to a family of
operations from processes to names. Given a context, $M$, we can
define a \emph{nominal context}, $\quotep{M}$ by $\quotep{M}[P] :=
\quotep{M[P]}$. To foreshadow what is to come we observe that these
operations enjoy a duality with processes very much like the duality
between vectors and maps from vectors to scalars.

Further, because the calculus is essentially higher-order, we have a
correspondence between contexts and processes. More specifically,
given a name $x$ and a context $M$ we can construct $M^{*}_{x}$ such
that 

\begin{mathpar}
  M^{*}_{x} | \lift{x}{P} \red M[P]
\end{mathpar}

namely,

\begin{mathpar}
  M^{*}_{x} := x?(u).M[\dropn{u}]
\end{mathpar}

The dependence of $M^{*}_{x}$ on a name makes it an abstraction, 

\begin{mathpar}
  M^{*} := (x)x?(u).M[\dropn{u}]
\end{mathpar}

\subsection{Additional notation}

It will sometimes be convenient to denote the process a name
quotes. We already have the notation $x = \quotep{P}$, but it will be
convenient to introduce an alternate notation, $\procn{x}$, when we
want to emphasize the connection to the use of the name. Note that, by
virtue of name equivalence, $\quotep{\procn{x}} \nameeq x$; so, the
notation is consistent with previous definitions.

Further, because names have structure it is possible to effect
substitutions on the basis of that structure. This means we need to
upgrade our notation for substitutions, which we accomplish by
adapting comprehension notation. Thus,

\begin{mathpar}
  P\{ y / x : x \in S \}
\end{mathpar}

is interpreted to mean the process derived from P by replacing (in a
capture-avoiding manner) each occurrence of $x$ in $S$ by $y$. For example,

\begin{mathpar}
  P\{ \quotep{\procn{x}|\procn{x}} / x : x \in \freenames{P} \}
\end{mathpar}

will replace each (occurrence) of a free name $x$ in $P$ by
$\quotep{\procn{x}|\procn{x}}$.

Also, we will avail ourselves of the notation $x^{L}$ and $x^{R}$ to
denote injections of a name into disjoint copies of the name
space. There are numerous ways to accomplish this. One example can be
found in \cite{MeredithR05}. This notation overloads to vectors of
names: $\vec{x}^{\pi} := (x_{i}^{\pi} \; : \; 0 \leq i < |\vec{x}| )$ where $\pi \in \{L,R\}$.

We also use $P^{\Box} := P|\Box$.

In \cite{MeredithR05} an interpretation of the new operator is
given. It turns out that there are several possible interpretations
all enjoying the requisite algebraic properties of the operator (see
\cite{milner91polyadicpi}). We will therefore make liberal use of
$(\nu\; \vec{x})P$.

% subsection the_syntax_and_semantics_of_the_notation_system (end)   

\input{qm2pi.qmops} 

\input{qm2pi.sterngerlach} 

\input{qm2pi.metric} 

% section concurrent_process_calculi (end)

%\input{qm2pi.proofsketch}

% section proof sketch (end)

%\input{qm2pi.slviaknots} 

% section spatial logic via knots (end)

\input{qm2pi.conclusion}

% section conclusion (end)

%\input{qm2pi.dtcodes} 

% section wiring algorithm (end)

\input{qm2pi.ack} 

% section acknowledgments (end)

\newpage


\bibliographystyle{plain}   
\bibliography{../../biblios/main.bib}

\input{qm2pi.rhodetails}

\end{document}

 

\documentclass[12pt]{llncs}
%\documentclass{jktr}

\usepackage[pdftex]{hyperref}                   
\usepackage {listings}
\usepackage {mathpartir}
\usepackage{bcprules}
%\usepackage{listings}
                       
\usepackage{graphicx} 
%\usepackage[margins=2.5cm,nohead,nofoot]{geometry}
%\usepackage{geometry}
\usepackage{amsfonts}
\usepackage{amstext}
\usepackage{latexsym}
\usepackage{amssymb}
\usepackage{color}


%\include{myPreamble}
\include{qm2pi.local} 

%\ifpdf
%\usepackage[pdftex]{graphicx}
%\else
%\usepackage{graphicx}
%\fi

 % \ifpdf
%  \usepackage{pdfsync}
%  \if


%\title{Brief Article}
%\author{David F. Snyder}
%\author{L.G. Meredith}

%\address{Dept. of Math., Texas State University--San Marcos, San Marcos, TX 78666}
       
\pagestyle{empty}


\begin{document}

\lstset{language=[Objective]Caml,frame=shadowbox}

\input{qm2pi.front}

% section front matter (end)

\input{qm2pi.intro} 
 
% section introduction (end)

% \input{qm2pi.knotations} 

% section notation (end)

\input{qm2pi.process.calculi} 

% section concurrent_process_calculi_and_spatial_logics_ (end)
    
%\input{qm2pi.knots2pi} 

%\input{qm2pi.trefoil} 

%\input{qm2pi.mainthm} 

% subsection basic_interpretation (end)

%\input{qm2pi.rho.presentation} 
\subsection{The syntax and semantics of the notation system}\label{sub:the_syntax_and_semantics_of_the_notation_system} % (fold)

We now summarize a technical presentation of the calculus that
embodies our theory of dynamics. The typical presentation of such a
calculus follows the style of giving generators and relations on
them. The grammar, below, describing term constructors, freely
generates the set of processes, $\Proc$. This set is then quotiented
by a relation known as structural congruence and it is over this set
that the notion of dynamics is expressed. This presentation is
essentially that of \cite{MeredithR05} with the addition of
polyadicity and summation. For readability we have relegated some of
the technical subtleties to an appendix.

\subsubsection{Process grammar}\label{subsub:process_grammar}

\begin{mathpar}
  \inferrule* [lab=synchronization] {} {{M} \bc \pzero \;|\; x?F \;|\; x!C }
  \and
  \inferrule* [lab=abstraction] {} {{F} \bc (x)P}
  \and
  \inferrule* [lab=concretion] {} {{C} \bc \langle Q \rangle}
  \and
  \inferrule* [lab=process] {} {{P,Q} \bc M \;| \;P|Q \;|\; @{x}}
  \and
  \inferrule* [lab=name] {} {{x} \bc \quotep{P}}
\end{mathpar} 

Note that $\vec{x}$ (resp. $\vec{P}$) denotes a vector of names
(resp. processes) of length $|\vec{x}|$ (resp. $|\vec{P}|$). We adopt
the following useful abbreviations.

\begin{mathpar}
   x?(\vec{y}).P := x.(\vec{y})P \and  x\clift{\vec{P}} := x.\clift{\vec{P}}
   \and x!(y) := \lift{x}{\dropn{y}}
   \and \Pi_{i=0}^{n-1}P_i := P_0 | \ldots | P_{n-1}
\end{mathpar}

\subsubsection{Structural congruence}

\paragraph{Free and bound names and alpha-equivalence.} At the
core of structural equivalence is alpha-equivalence which identifies
process that are the same up to a change of variable. Formally, we
recognize the distinction between free and bound names. The free names
of a process, $\freenames{P}$, may be calculated recursively as
follows:

\begin{mathpar}
\freenames{\pzero} := \emptyset
  \and \\
  \freenames{x?(y).P} := \{ x \} \cup (\freenames{P} \setminus \{ y \})
  \and 
  \freenames{x!\langle P \rangle} := \{ x \} \cup \{ P \} 
  \and \\
  \freenames{P|Q} := \freenames{P} \cup \freenames{Q}
  \and \\
  \freenames{@{x}} := \{ x \}
\end{mathpar}

$\pi$
$\quotep{\pi}$

$\freenames{-} : \pi \to \mathcal{P}(\quotep{\pi})$

\begin{eqnarray*}
  \freenames{\pzero} & := & \emptyset \\
  \freenames{x?(y).P} & := & \{ x \} \cup (\freenames{P} \setminus \{ y \}) \\
  \freenames{x!\langle P \rangle} & := & \{ x \} \cup \{ P \} \\
  \freenames{P|Q} & := & \freenames{P} \cup \freenames{Q} \\
  \freenames{\dropn{x}} & := & \{ x \}
\end{eqnarray*}

The bound names of a process, $\boundnames{P}$, are those names occurring in $P$
that are not free. For example, in $x?(y).0$, the name $x$ is free, while $y$ is bound.

\begin{mathpar}
  \inferrule* [lab=monoidal-laws] {} { P|Q \equiv Q|P \and P|0 \equiv P \and P|(Q|R) \equiv (P|Q)|R }
\end{mathpar}

\begin{mathpar}
  \inferrule* [lab=alpha-equivalence] {} { (x)P \equiv (y)P\{y/x\} \and y \not\in \freenames{P} }
\end{mathpar}

\begin{definition}
Then two processes, $P,Q$, are alpha-equivalent if $P = Q\{\vec{y}/\vec{x}\}$ for
some $\vec{x} \in \boundnames{Q},\vec{y} \in \boundnames{P}$, where $Q\{\vec{y}/\vec{x}\}$
denotes the capture-avoiding substitution of $\vec{y}$ for $\vec{x}$ in $Q$.
\end{definition}

\begin{definition}
  The {\em structural congruence} \cite{SangiorgiWalker} , $\equiv$,
  between processes is the least congruence containing
  alpha-equivalence, satisfying the abelian monoid laws
  (associativity, commutativity and $\pzero$ as identity) for parallel
  composition $|$ and for summation $+$.
\end{definition}

\subsection{Name equivalence}

We take name equivalence, written $\nameeq$, to be the smallest
equivalence relation generated by the following rules.

\begin{mathpar}
\inferrule*[lab=Quote-drop]
{ }
{ \quotep{@{x}} \nameeq x }

\inferrule*[lab=Struct-equiv]
{ P \scong Q }
{ \quotep{P} \nameeq \quotep{Q} }
\end{mathpar}

The astute reader will have noticed that the mutual recursion of names
and processes imposes a mutual recursion on alpha-equivalence and
structural equivalence via name-equivalence. Fortunately, all of this
works out pleasantly and we may calculate in the natural way, free of
concern. The reader interested in the details is referred to the
appendix \ref{appendix:rho_details}.

\subsection{Substitution}

We use $\Proc$ for the set of processes, $\QProc$ for the set of
names, and $\id{\{}\vec{y} / \vec{x} \id{\}}$ to denote partial maps,
$s : \QProc \rightarrow \QProc$. A map, $s$ lifts, uniquely, to a map
on process terms, $\widehat{s} : \Proc \rightarrow \Proc$ by the
following equations.

\begin{mathpar}
  (0) \psubstp{Q}{P} := 0 \\
  (R \juxtap S) \psubstp{Q}{P}
  :=    
  (R)\psubstp{Q}{P} \juxtap (S) \psubstp{Q}{P} \\
  (x?(y).R) \psubstp{Q}{P}    
  :=    
  (x)\substp{Q}{P} (z)\concat( (R \psubstn{z}{y}) \psubstp{Q}{P} ) \\
  (\lift{x}{R}) \psubstp{Q}{P}  
  :=
  \lift{(x)\substp{Q}{P}}{ R \psubstp{Q}{P} } \\
%   (\dropn{x})  \psubstp{Q}{P}       
%   := 
%   \left\{ 
%     \begin{array}{ccc} 
%       \dropn{\quotep{Q}} & & x \nameeq \quotep{P} \\
%       \dropn{x} & & otherwise \\
%     \end{array}
%   \right. 
  (\dropn{x})  \psubstp{Q}{P}       
  := 
  \left\{ 
    \begin{array}{ccc} 
      Q & & x \nameeq \quotep{P} \\
      \dropn{x} & & otherwise \\
    \end{array}
  \right.
\end{mathpar}
 

where

\begin{eqnarray}
  (x)\id{\{} \lpquote Q \rpquote / \lpquote P \rpquote \id{\}}            = 
  \left\{ 
    \begin{array}{ccc}
      \lpquote Q \rpquote & & x \nameeq \lpquote P \rpquote \\
      x & & otherwise \\
    \end{array}
  \right. \nonumber
\end{eqnarray}

and $z$ is chosen distinct from $\quotep{P}$, $\quotep{Q}$, the free
names in $Q$, and all the names in $R$. Our $\alpha$-equivalence will
be built in the standard way from this substitution.

\begin{remark}\label{rem:no_self_referential_names}
  One consequence of these definitions is that $\forall P. \quotep{P}
  \not\in \freenames{P}$.
\end{remark}

\subsection{ Dynamic quote: an example }

Anticipating something of what's to come, consider applying the
substitution, $\widehat{\id{\{}u / z \id{\}}}$, to the following pair
of processes, $\lift{w}{y!(z)}$ and $w[ \lpquote y!(z) \rpquote ]$.

\begin{eqnarray}
	\lift{w}{y!(z)}\widehat{\id{\{}u / z \id{\}}}
		& = &
		\lift{w}{y!(u)} \nonumber\\
	w[ \lpquote y!(z) \rpquote ] \widehat{ \id{\{}u / z \id{\}} }
		& = &
		w[ \lpquote y!(z) \rpquote ] \nonumber
\end{eqnarray}

Because the body of the process between quotes is impervious to
substitution, we get radically different answers. In fact, by
examining the first process in an input context,
e.g. $x?(z).\lift{w}{y!(z)}$, we see that the process under the lift
operator may be shaped by prefixed inputs binding a name inside it. In
this sense, the lift operator will be seen as a way to dynamically
construct processes before reifying them as names.

Finally equipped with these standard features we can present the
dynamics of the calculus.

\subsubsection{Operational semantics} 

Finally, we introduce the computational dynamics. What marks these
algebras as distinct from other more traditionally studied algebraic
structures, e.g. vector spaces or polynomial rings, is the manner in
which dynamics is captured. In traditional structures, dynamics is typically
expressed through morphisms between such structures, as in linear maps
between vector spaces or morphisms between rings. In algebras
associated with the semantics of computation, the dynamics is
expressed as part of the algebraic structure itself, through a
reduction reduction relation typically denoted by $\red$. Below, we
give a recursive presentation of this relation for the calculus used
in the encoding.

$\red \subseteq \pi \times \pi$
$\red : \pi \to \mathcal{P}(\pi)$

\begin{mathpar}
  \inferrule* [lab=Comm] { \textsf{match}( x_{src}, x_{trgt} ) } { x_{trgt}?(y)P \; | \; x_{src}!\langle {Q} \rangle \red P\{\quotep{Q}/y}\} }
  \and \\
  \inferrule* [lab=Par] {{P} \red {P}'} {{{P} | {Q}} \red {{P}' | {Q}}}
  \and
  \inferrule* [lab=Equiv]{{{P} \scong {P}'} \andalso {{P}' \red {Q}'} \andalso {{Q}' \scong {Q}}}{{P} \red {Q}}
\end{mathpar}

\begin{eqnarray*}
  match_{\equiv} (\quotep{P},\quotep{Q}) & := & P \equiv Q \\
  match_{\dagger}(\quotep{P},\quotep{Q}) & := & \forall R. P|Q \red^{*} R => R \red^{*} 0 \\
  match_{K}(\quotep{P},\quotep{Q}) & := & K \mbox{ for some context } K
\end{eqnarray*}

$u?(x)P | u!\langle Q \rangle \red P\{\quotep{Q}/x\}$

%We write $\wred$ for $\red^*$, and $P\red$ if $\exists Q $ such that $ P \red Q$.
We write $P\red$ if $\exists Q $ such that $ P \red Q$ and $P\not\red$, otherwise.

\section{Replication}

As mentioned before, it is known that replication (and hence
recursion) can be implemented in a higher-order process algebra
\cite{SangiorgiWalker}. As our first example of calculation with the
machinery thus far presented we give the construction explicitly in
the {\rhoc}.

\begin{eqnarray}
	D_{x} & := & \prefix{x}{y}{(\binpar{\outputp{x}{y}}{@{y}})} \nonumber\\
	\bangp_{x}{P} & := & \binpar{{x}!\langle{\binpar{D_{x}}{P}}\rangle}{D_{x}} \nonumber
\end{eqnarray}

\begin{eqnarray}
	\bangp_{x}{P} & & \nonumber\\
	=
	& {x}!\langle{(\prefix{x}{y}{(\outputp{x}{y} | @{y})) | P}}\rangle 
	      | \prefix{x}{y}{(\outputp{x}{y} | @{y})} & \nonumber\\
	\red
	& (\outputp{x}{y} | @{y})\substn{\quotep{(\prefix{x}{y}{(@{y} | \outputp{x}{y})) | P}}}{y} & \nonumber\\
	=
	& \outputp{x}{\quotep{(\prefix{x}{y}{(\outputp{x}{y} | @{y})) | P}}}
	  | {(\prefix{x}{y}{(\outputp{x}{y} | @{y})) | P}} & \nonumber\\
	\red
	& \ldots & \nonumber\\
	\red^*
	& P | P | \ldots & \nonumber
\end{eqnarray}

Of course, this encoding, as an implementation, runs away, unfolding
$\bangp{P}$ eagerly. A lazier and more implementable replication
operator, restricted to input-guarded processes, may be obtained as follows.

\begin{eqnarray}
\bangp{\prefix{u}{v}{P}} 
	:= 
	\binpar{\lift{x}{\prefix{u}{v}{(\binpar{D(x)}{P})}}}{D(x)} \nonumber
\end{eqnarray}

\begin{remark}
  Note that the lazier definition still does not deal with summation
  or mixed summation (i.e. sums over input and output). The reader is
  invited to construct definitions of replication that deal with these
  features. 

  Further, the definitions are parameterized in a name, $x$. Can you,
  gentle reader, make a definition that eliminates this parameter and
  guarantees no accidental interaction between the replication
  machinery and the process being replicated -- i.e. no accidental
  sharing of names used by the process to get its work done and the
  name(s) used by the replication to effect copying. This latter
  revision of the definition of replication is crucial to obtaining
  the expected identity $!!P \sim !P$.
\end{remark}

\begin{remark}\label{rem:paradoxical_combinator}
  The reader familiar with the lambda calculus will have noticed the
  similarity between $D$ and the paradoxical combinator.

  [Ed. note: the existence of this seems to suggest we have to be more
  restrictive on the set of processes and names we admit if we are to
  support no-cloning.]
\end{remark}

\subsubsection{Bisimulation}

The computational dynamics gives rise to another kind of equivalence,
the equivalence of computational behavior. As previously mentioned
this is typically captured \emph{via} some form of bisimulation.

% The notion we use in this paper is weak barbed bisimulation
% \cite{milner91polyadicpi}.

The notion we use in this paper is derived from weak barbed
bisimulation \cite{milner91polyadicpi}. 

\begin{definition}
An \emph{observation relation}, $\downarrow_{\mathcal N}$, over a set
of names, $\mathcal N$, is the smallest relation satisfying the rules
below.

\infrule[Out-barb]{y \in {\mathcal N}, \; x \nameeq y}
		  {\outputp{x}{v} \downarrow_{\mathcal N} x}
\infrule[Par-barb]{\mbox{$P\downarrow_{\mathcal N} x$ or $Q\downarrow_{\mathcal N} x$}}
		  {\binpar{P}{Q} \downarrow_{\mathcal N} x}

We write $P \Downarrow_{\mathcal N} x$ if there is $Q$ such that 
$P \wred Q$ and $Q \downarrow_{\mathcal N} x$.
\end{definition}

\begin{definition}
%\label{def.bbisim}
An  ${\mathcal N}$-\emph{barbed bisimulation} over a set of names, ${\mathcal N}$, is a symmetric binary relation 
${\mathcal S}_{\mathcal N}$ between agents such that $P\rel{S}_{\mathcal N}Q$ implies:
\begin{enumerate}
\item If $P \red P'$ then $Q \wred Q'$ and $P'\rel{S}_{\mathcal N} Q'$.
\item If $P\downarrow_{\mathcal N} x$, then $Q\Downarrow_{\mathcal N} x$.
\end{enumerate}
$P$ is ${\mathcal N}$-barbed bisimilar to $Q$, written
$P \wbbisim_{\mathcal N} Q$, if $P \rel{S}_{\mathcal N} Q$ for some ${\mathcal N}$-barbed bisimulation ${\mathcal S}_{\mathcal N}$.
\end{definition}

$\mathcal{R} \subseteq \pi \times \pi$

$P \mathcal{R} Q => \forall P'. P \red P' \Rightarrow \exists Q'. Q \red Q', P' \mathcal{R} Q'$

$P \vdash x \Rightarrow Q \vdash x$

\begin{mathpar}
  \inferrule*[lab=Out-barb]{x \nameeq y}{{y}!\langle{Q}\rangle \vdash x}
  \and
  \inferrule*[lab=Par-barb]{\mbox{$P\vdash x$ or $Q\vdash x$}}{\binpar{P}{Q} \vdash x}
\end{mathpar}

\subsubsection{Contexts}

One of the principle advantages of computational calculi like the
$\pi$-calculus is a well-defined notion of context,
contextual-equivalence and a correlation between
contextual-equivalence and notions of bisimulation. The notion of
context allows the decomposition of a process into (sub-)process and
its syntactic environment, its context. Thus, a context may be
thought of as a process with a ``hole'' (written $\Box$) in it. The
application of a context $M$ to a process $P$, written $M[P]$, is
tantamount to filling the hole in $M$ with $P$. In this paper we do
not need the full weight of this theory, but do make use of the notion
of context in the proof the main theorem. 

\begin{mathpar}
  \inferrule* [lab=summation] {} {{M_{M},M_{N}} \bc \Box \;|\; x.M_{A} \;|\; M_{M}+M_{N}}
  \and
  \inferrule* [lab=agent] {} {{M_{A}} \bc (\vec{x})M_{P} \;| \; \clift{P_0,\ldots,M_{P},\ldots,P_N}}
  \and \\
  \inferrule* [lab=process] {} {{M_{P}} \bc M_{N} \;| \;P|M_{P} }
\end{mathpar} 

\begin{mathpar}
  \inferrule* [lab=sychronization] {} {M_{N} \bc \Box \;|\; x?M_{F} \;|\; x!M_{C}}
  \and
  \inferrule* [lab=abstraction] {} {{M_{F}} \bc (x)M_{P} }
  \and
  \inferrule* [lab=concretion] {} {{M_{C}} \bc \langle M_{P} \rangle }
  \and \\
  \inferrule* [lab=process] {} {{M_{P}} \bc M_{N} \;| \;P|M_{P} }
\end{mathpar}

\begin{definition}[contextual application] Given a context $M$, and
  process $P$, we define the \emph{contextual application}, $M[P] :=
  M\{P/\Box\}$. That is, the contextual application of M to P is the
  substitution of $P$ for $\Box$ in $M$.
\end{definition}

$\meaningof{-} : L \to \mathcal{P}(\pi)$

\begin{mathpar}
  \inferrule* [lab=collection] {} {\meaningof{true} = \pi, \and \meaningof{~E} = \pi \setminus \meaningof{E}, \and \meaningof{E_{1} \& E_{2}} = \meaningof{E_{1}} \cap \meaningof{E_{2}}}
\end{mathpar}

\begin{mathpar}
  \inferrule* [lab=structure] {} {\meaningof{0} = \{ P \in \pi | P \equiv 0 \}, \and \\ \meaningof{E_1 | E_2} = \{ P \in \pi | P \equiv P_{1} | P_{2}, P_{1} \in \meaningof{E_{1}}, P_{2} \in \meaningof{E_2}\} }
\end{mathpar}

\begin{mathpar}
 \inferrule* [lab=behavior] {} {\meaningof{\langle a?b \rangle E} = \{ P \in \pi | P \equiv Q | u?(y)P', \\ \and \\\\ \and \\ \;\;\; u \in \meaningof{a}, \forall z.P'\{z/y\} \in \meaningof{E\{z/b\}}\}, \and \\ \meaningof{a!E} = \{ P \in \pi | P \equiv Q | x!\langle P' \rangle, x \in \meaningof{a} P' \in \meaningof{E}\} }
\end{mathpar}

\begin{mathpar}
 \inferrule* [lab=nominal] {} {\meaningof{\quotep{E}} = \{ \quotep{P} \in \quotep{\pi} | P \in \meaningof{E} \}, \and \meaningof{\quotep{P}} = \{ \quotep{Q} \in \quotep{\pi} | P \equiv Q \} \and \\ \meaningof{@\quotep{E}} = \{ P \in \pi | P \equiv @x, x \in \meaningof{E} \}}
\end{mathpar}

\begin{eqnarray*}
  \\
  \meaningof{-} : TS \to ST
\end{eqnarray*}

\begin{eqnarray*}
  \\
  L : TS \to ST
\end{eqnarray*}

\begin{eqnarray*}
  \\
  P \models E \iff P \in \meaningof{E}
\end{eqnarray*}

\begin{eqnarray*}
  P \approx_{L} Q \iff \forall E \in L. P \models E \iff Q \models E
\end{eqnarray*}

\begin{eqnarray*}
  P \approx_{K} Q
\end{eqnarray*}

\begin{eqnarray*}
  P \approx Q
\end{eqnarray*}

$\approx_{K} = \approx = \approx_{L}$

\subsubsection{Contextual duality}

Note that contexts extend the quotation operation to a family of
operations from processes to names. Given a context, $M$, we can
define a \emph{nominal context}, $\quotep{M}$ by $\quotep{M}[P] :=
\quotep{M[P]}$. To foreshadow what is to come we observe that these
operations enjoy a duality with processes very much like the duality
between vectors and maps from vectors to scalars.

Further, because the calculus is essentially higher-order, we have a
correspondence between contexts and processes. More specifically,
given a name $x$ and a context $M$ we can construct $M^{*}_{x}$ such
that 

\begin{mathpar}
  M^{*}_{x} | \lift{x}{P} \red M[P]
\end{mathpar}

namely,

\begin{mathpar}
  M^{*}_{x} := x?(u).M[\dropn{u}]
\end{mathpar}

The dependence of $M^{*}_{x}$ on a name makes it an abstraction, 

\begin{mathpar}
  M^{*} := (x)x?(u).M[\dropn{u}]
\end{mathpar}

\subsection{Additional notation}

It will sometimes be convenient to denote the process a name
quotes. We already have the notation $x = \quotep{P}$, but it will be
convenient to introduce an alternate notation, $\procn{x}$, when we
want to emphasize the connection to the use of the name. Note that, by
virtue of name equivalence, $\quotep{\procn{x}} \nameeq x$; so, the
notation is consistent with previous definitions.

Further, because names have structure it is possible to effect
substitutions on the basis of that structure. This means we need to
upgrade our notation for substitutions, which we accomplish by
adapting comprehension notation. Thus,

\begin{mathpar}
  P\{ y / x : x \in S \}
\end{mathpar}

is interpreted to mean the process derived from P by replacing (in a
capture-avoiding manner) each occurrence of $x$ in $S$ by $y$. For example,

\begin{mathpar}
  P\{ \quotep{\procn{x}|\procn{x}} / x : x \in \freenames{P} \}
\end{mathpar}

will replace each (occurrence) of a free name $x$ in $P$ by
$\quotep{\procn{x}|\procn{x}}$.

Also, we will avail ourselves of the notation $x^{L}$ and $x^{R}$ to
denote injections of a name into disjoint copies of the name
space. There are numerous ways to accomplish this. One example can be
found in \cite{MeredithR05}. This notation overloads to vectors of
names: $\vec{x}^{\pi} := (x_{i}^{\pi} \; : \; 0 \leq i < |\vec{x}| )$ where $\pi \in \{L,R\}$.

We also use $P^{\Box} := P|\Box$.

In \cite{MeredithR05} an interpretation of the new operator is
given. It turns out that there are several possible interpretations
all enjoying the requisite algebraic properties of the operator (see
\cite{milner91polyadicpi}). We will therefore make liberal use of
$(\nu\; \vec{x})P$.

% subsection the_syntax_and_semantics_of_the_notation_system (end)   

\input{qm2pi.qmops} 

\input{qm2pi.sterngerlach} 

\input{qm2pi.metric} 

% section concurrent_process_calculi (end)

%\input{qm2pi.proofsketch}

% section proof sketch (end)

%\input{qm2pi.slviaknots} 

% section spatial logic via knots (end)

\input{qm2pi.conclusion}

% section conclusion (end)

%\input{qm2pi.dtcodes} 

% section wiring algorithm (end)

\input{qm2pi.ack} 

% section acknowledgments (end)

\newpage


\bibliographystyle{plain}   
\bibliography{../../biblios/main.bib}

\input{qm2pi.rhodetails}

\end{document}

 

% section concurrent_process_calculi (end)

%\documentclass[12pt]{llncs}
%\documentclass{jktr}

\usepackage[pdftex]{hyperref}                   
\usepackage {listings}
\usepackage {mathpartir}
\usepackage{bcprules}
%\usepackage{listings}
                       
\usepackage{graphicx} 
%\usepackage[margins=2.5cm,nohead,nofoot]{geometry}
%\usepackage{geometry}
\usepackage{amsfonts}
\usepackage{amstext}
\usepackage{latexsym}
\usepackage{amssymb}
\usepackage{color}


%\include{myPreamble}
\include{qm2pi.local} 

%\ifpdf
%\usepackage[pdftex]{graphicx}
%\else
%\usepackage{graphicx}
%\fi

 % \ifpdf
%  \usepackage{pdfsync}
%  \if


%\title{Brief Article}
%\author{David F. Snyder}
%\author{L.G. Meredith}

%\address{Dept. of Math., Texas State University--San Marcos, San Marcos, TX 78666}
       
\pagestyle{empty}


\begin{document}

\lstset{language=[Objective]Caml,frame=shadowbox}

\input{qm2pi.front}

% section front matter (end)

\input{qm2pi.intro} 
 
% section introduction (end)

% \input{qm2pi.knotations} 

% section notation (end)

\input{qm2pi.process.calculi} 

% section concurrent_process_calculi_and_spatial_logics_ (end)
    
%\input{qm2pi.knots2pi} 

%\input{qm2pi.trefoil} 

%\input{qm2pi.mainthm} 

% subsection basic_interpretation (end)

%\input{qm2pi.rho.presentation} 
\subsection{The syntax and semantics of the notation system}\label{sub:the_syntax_and_semantics_of_the_notation_system} % (fold)

We now summarize a technical presentation of the calculus that
embodies our theory of dynamics. The typical presentation of such a
calculus follows the style of giving generators and relations on
them. The grammar, below, describing term constructors, freely
generates the set of processes, $\Proc$. This set is then quotiented
by a relation known as structural congruence and it is over this set
that the notion of dynamics is expressed. This presentation is
essentially that of \cite{MeredithR05} with the addition of
polyadicity and summation. For readability we have relegated some of
the technical subtleties to an appendix.

\subsubsection{Process grammar}\label{subsub:process_grammar}

\begin{mathpar}
  \inferrule* [lab=synchronization] {} {{M} \bc \pzero \;|\; x?F \;|\; x!C }
  \and
  \inferrule* [lab=abstraction] {} {{F} \bc (x)P}
  \and
  \inferrule* [lab=concretion] {} {{C} \bc \langle Q \rangle}
  \and
  \inferrule* [lab=process] {} {{P,Q} \bc M \;| \;P|Q \;|\; @{x}}
  \and
  \inferrule* [lab=name] {} {{x} \bc \quotep{P}}
\end{mathpar} 

Note that $\vec{x}$ (resp. $\vec{P}$) denotes a vector of names
(resp. processes) of length $|\vec{x}|$ (resp. $|\vec{P}|$). We adopt
the following useful abbreviations.

\begin{mathpar}
   x?(\vec{y}).P := x.(\vec{y})P \and  x\clift{\vec{P}} := x.\clift{\vec{P}}
   \and x!(y) := \lift{x}{\dropn{y}}
   \and \Pi_{i=0}^{n-1}P_i := P_0 | \ldots | P_{n-1}
\end{mathpar}

\subsubsection{Structural congruence}

\paragraph{Free and bound names and alpha-equivalence.} At the
core of structural equivalence is alpha-equivalence which identifies
process that are the same up to a change of variable. Formally, we
recognize the distinction between free and bound names. The free names
of a process, $\freenames{P}$, may be calculated recursively as
follows:

\begin{mathpar}
\freenames{\pzero} := \emptyset
  \and \\
  \freenames{x?(y).P} := \{ x \} \cup (\freenames{P} \setminus \{ y \})
  \and 
  \freenames{x!\langle P \rangle} := \{ x \} \cup \{ P \} 
  \and \\
  \freenames{P|Q} := \freenames{P} \cup \freenames{Q}
  \and \\
  \freenames{@{x}} := \{ x \}
\end{mathpar}

$\pi$
$\quotep{\pi}$

$\freenames{-} : \pi \to \mathcal{P}(\quotep{\pi})$

\begin{eqnarray*}
  \freenames{\pzero} & := & \emptyset \\
  \freenames{x?(y).P} & := & \{ x \} \cup (\freenames{P} \setminus \{ y \}) \\
  \freenames{x!\langle P \rangle} & := & \{ x \} \cup \{ P \} \\
  \freenames{P|Q} & := & \freenames{P} \cup \freenames{Q} \\
  \freenames{\dropn{x}} & := & \{ x \}
\end{eqnarray*}

The bound names of a process, $\boundnames{P}$, are those names occurring in $P$
that are not free. For example, in $x?(y).0$, the name $x$ is free, while $y$ is bound.

\begin{mathpar}
  \inferrule* [lab=monoidal-laws] {} { P|Q \equiv Q|P \and P|0 \equiv P \and P|(Q|R) \equiv (P|Q)|R }
\end{mathpar}

\begin{mathpar}
  \inferrule* [lab=alpha-equivalence] {} { (x)P \equiv (y)P\{y/x\} \and y \not\in \freenames{P} }
\end{mathpar}

\begin{definition}
Then two processes, $P,Q$, are alpha-equivalent if $P = Q\{\vec{y}/\vec{x}\}$ for
some $\vec{x} \in \boundnames{Q},\vec{y} \in \boundnames{P}$, where $Q\{\vec{y}/\vec{x}\}$
denotes the capture-avoiding substitution of $\vec{y}$ for $\vec{x}$ in $Q$.
\end{definition}

\begin{definition}
  The {\em structural congruence} \cite{SangiorgiWalker} , $\equiv$,
  between processes is the least congruence containing
  alpha-equivalence, satisfying the abelian monoid laws
  (associativity, commutativity and $\pzero$ as identity) for parallel
  composition $|$ and for summation $+$.
\end{definition}

\subsection{Name equivalence}

We take name equivalence, written $\nameeq$, to be the smallest
equivalence relation generated by the following rules.

\begin{mathpar}
\inferrule*[lab=Quote-drop]
{ }
{ \quotep{@{x}} \nameeq x }

\inferrule*[lab=Struct-equiv]
{ P \scong Q }
{ \quotep{P} \nameeq \quotep{Q} }
\end{mathpar}

The astute reader will have noticed that the mutual recursion of names
and processes imposes a mutual recursion on alpha-equivalence and
structural equivalence via name-equivalence. Fortunately, all of this
works out pleasantly and we may calculate in the natural way, free of
concern. The reader interested in the details is referred to the
appendix \ref{appendix:rho_details}.

\subsection{Substitution}

We use $\Proc$ for the set of processes, $\QProc$ for the set of
names, and $\id{\{}\vec{y} / \vec{x} \id{\}}$ to denote partial maps,
$s : \QProc \rightarrow \QProc$. A map, $s$ lifts, uniquely, to a map
on process terms, $\widehat{s} : \Proc \rightarrow \Proc$ by the
following equations.

\begin{mathpar}
  (0) \psubstp{Q}{P} := 0 \\
  (R \juxtap S) \psubstp{Q}{P}
  :=    
  (R)\psubstp{Q}{P} \juxtap (S) \psubstp{Q}{P} \\
  (x?(y).R) \psubstp{Q}{P}    
  :=    
  (x)\substp{Q}{P} (z)\concat( (R \psubstn{z}{y}) \psubstp{Q}{P} ) \\
  (\lift{x}{R}) \psubstp{Q}{P}  
  :=
  \lift{(x)\substp{Q}{P}}{ R \psubstp{Q}{P} } \\
%   (\dropn{x})  \psubstp{Q}{P}       
%   := 
%   \left\{ 
%     \begin{array}{ccc} 
%       \dropn{\quotep{Q}} & & x \nameeq \quotep{P} \\
%       \dropn{x} & & otherwise \\
%     \end{array}
%   \right. 
  (\dropn{x})  \psubstp{Q}{P}       
  := 
  \left\{ 
    \begin{array}{ccc} 
      Q & & x \nameeq \quotep{P} \\
      \dropn{x} & & otherwise \\
    \end{array}
  \right.
\end{mathpar}
 

where

\begin{eqnarray}
  (x)\id{\{} \lpquote Q \rpquote / \lpquote P \rpquote \id{\}}            = 
  \left\{ 
    \begin{array}{ccc}
      \lpquote Q \rpquote & & x \nameeq \lpquote P \rpquote \\
      x & & otherwise \\
    \end{array}
  \right. \nonumber
\end{eqnarray}

and $z$ is chosen distinct from $\quotep{P}$, $\quotep{Q}$, the free
names in $Q$, and all the names in $R$. Our $\alpha$-equivalence will
be built in the standard way from this substitution.

\begin{remark}\label{rem:no_self_referential_names}
  One consequence of these definitions is that $\forall P. \quotep{P}
  \not\in \freenames{P}$.
\end{remark}

\subsection{ Dynamic quote: an example }

Anticipating something of what's to come, consider applying the
substitution, $\widehat{\id{\{}u / z \id{\}}}$, to the following pair
of processes, $\lift{w}{y!(z)}$ and $w[ \lpquote y!(z) \rpquote ]$.

\begin{eqnarray}
	\lift{w}{y!(z)}\widehat{\id{\{}u / z \id{\}}}
		& = &
		\lift{w}{y!(u)} \nonumber\\
	w[ \lpquote y!(z) \rpquote ] \widehat{ \id{\{}u / z \id{\}} }
		& = &
		w[ \lpquote y!(z) \rpquote ] \nonumber
\end{eqnarray}

Because the body of the process between quotes is impervious to
substitution, we get radically different answers. In fact, by
examining the first process in an input context,
e.g. $x?(z).\lift{w}{y!(z)}$, we see that the process under the lift
operator may be shaped by prefixed inputs binding a name inside it. In
this sense, the lift operator will be seen as a way to dynamically
construct processes before reifying them as names.

Finally equipped with these standard features we can present the
dynamics of the calculus.

\subsubsection{Operational semantics} 

Finally, we introduce the computational dynamics. What marks these
algebras as distinct from other more traditionally studied algebraic
structures, e.g. vector spaces or polynomial rings, is the manner in
which dynamics is captured. In traditional structures, dynamics is typically
expressed through morphisms between such structures, as in linear maps
between vector spaces or morphisms between rings. In algebras
associated with the semantics of computation, the dynamics is
expressed as part of the algebraic structure itself, through a
reduction reduction relation typically denoted by $\red$. Below, we
give a recursive presentation of this relation for the calculus used
in the encoding.

$\red \subseteq \pi \times \pi$
$\red : \pi \to \mathcal{P}(\pi)$

\begin{mathpar}
  \inferrule* [lab=Comm] { \textsf{match}( x_{src}, x_{trgt} ) } { x_{trgt}?(y)P \; | \; x_{src}!\langle {Q} \rangle \red P\{\quotep{Q}/y}\} }
  \and \\
  \inferrule* [lab=Par] {{P} \red {P}'} {{{P} | {Q}} \red {{P}' | {Q}}}
  \and
  \inferrule* [lab=Equiv]{{{P} \scong {P}'} \andalso {{P}' \red {Q}'} \andalso {{Q}' \scong {Q}}}{{P} \red {Q}}
\end{mathpar}

\begin{eqnarray*}
  match_{\equiv} (\quotep{P},\quotep{Q}) & := & P \equiv Q \\
  match_{\dagger}(\quotep{P},\quotep{Q}) & := & \forall R. P|Q \red^{*} R => R \red^{*} 0 \\
  match_{K}(\quotep{P},\quotep{Q}) & := & K \mbox{ for some context } K
\end{eqnarray*}

$u?(x)P | u!\langle Q \rangle \red P\{\quotep{Q}/x\}$

%We write $\wred$ for $\red^*$, and $P\red$ if $\exists Q $ such that $ P \red Q$.
We write $P\red$ if $\exists Q $ such that $ P \red Q$ and $P\not\red$, otherwise.

\section{Replication}

As mentioned before, it is known that replication (and hence
recursion) can be implemented in a higher-order process algebra
\cite{SangiorgiWalker}. As our first example of calculation with the
machinery thus far presented we give the construction explicitly in
the {\rhoc}.

\begin{eqnarray}
	D_{x} & := & \prefix{x}{y}{(\binpar{\outputp{x}{y}}{@{y}})} \nonumber\\
	\bangp_{x}{P} & := & \binpar{{x}!\langle{\binpar{D_{x}}{P}}\rangle}{D_{x}} \nonumber
\end{eqnarray}

\begin{eqnarray}
	\bangp_{x}{P} & & \nonumber\\
	=
	& {x}!\langle{(\prefix{x}{y}{(\outputp{x}{y} | @{y})) | P}}\rangle 
	      | \prefix{x}{y}{(\outputp{x}{y} | @{y})} & \nonumber\\
	\red
	& (\outputp{x}{y} | @{y})\substn{\quotep{(\prefix{x}{y}{(@{y} | \outputp{x}{y})) | P}}}{y} & \nonumber\\
	=
	& \outputp{x}{\quotep{(\prefix{x}{y}{(\outputp{x}{y} | @{y})) | P}}}
	  | {(\prefix{x}{y}{(\outputp{x}{y} | @{y})) | P}} & \nonumber\\
	\red
	& \ldots & \nonumber\\
	\red^*
	& P | P | \ldots & \nonumber
\end{eqnarray}

Of course, this encoding, as an implementation, runs away, unfolding
$\bangp{P}$ eagerly. A lazier and more implementable replication
operator, restricted to input-guarded processes, may be obtained as follows.

\begin{eqnarray}
\bangp{\prefix{u}{v}{P}} 
	:= 
	\binpar{\lift{x}{\prefix{u}{v}{(\binpar{D(x)}{P})}}}{D(x)} \nonumber
\end{eqnarray}

\begin{remark}
  Note that the lazier definition still does not deal with summation
  or mixed summation (i.e. sums over input and output). The reader is
  invited to construct definitions of replication that deal with these
  features. 

  Further, the definitions are parameterized in a name, $x$. Can you,
  gentle reader, make a definition that eliminates this parameter and
  guarantees no accidental interaction between the replication
  machinery and the process being replicated -- i.e. no accidental
  sharing of names used by the process to get its work done and the
  name(s) used by the replication to effect copying. This latter
  revision of the definition of replication is crucial to obtaining
  the expected identity $!!P \sim !P$.
\end{remark}

\begin{remark}\label{rem:paradoxical_combinator}
  The reader familiar with the lambda calculus will have noticed the
  similarity between $D$ and the paradoxical combinator.

  [Ed. note: the existence of this seems to suggest we have to be more
  restrictive on the set of processes and names we admit if we are to
  support no-cloning.]
\end{remark}

\subsubsection{Bisimulation}

The computational dynamics gives rise to another kind of equivalence,
the equivalence of computational behavior. As previously mentioned
this is typically captured \emph{via} some form of bisimulation.

% The notion we use in this paper is weak barbed bisimulation
% \cite{milner91polyadicpi}.

The notion we use in this paper is derived from weak barbed
bisimulation \cite{milner91polyadicpi}. 

\begin{definition}
An \emph{observation relation}, $\downarrow_{\mathcal N}$, over a set
of names, $\mathcal N$, is the smallest relation satisfying the rules
below.

\infrule[Out-barb]{y \in {\mathcal N}, \; x \nameeq y}
		  {\outputp{x}{v} \downarrow_{\mathcal N} x}
\infrule[Par-barb]{\mbox{$P\downarrow_{\mathcal N} x$ or $Q\downarrow_{\mathcal N} x$}}
		  {\binpar{P}{Q} \downarrow_{\mathcal N} x}

We write $P \Downarrow_{\mathcal N} x$ if there is $Q$ such that 
$P \wred Q$ and $Q \downarrow_{\mathcal N} x$.
\end{definition}

\begin{definition}
%\label{def.bbisim}
An  ${\mathcal N}$-\emph{barbed bisimulation} over a set of names, ${\mathcal N}$, is a symmetric binary relation 
${\mathcal S}_{\mathcal N}$ between agents such that $P\rel{S}_{\mathcal N}Q$ implies:
\begin{enumerate}
\item If $P \red P'$ then $Q \wred Q'$ and $P'\rel{S}_{\mathcal N} Q'$.
\item If $P\downarrow_{\mathcal N} x$, then $Q\Downarrow_{\mathcal N} x$.
\end{enumerate}
$P$ is ${\mathcal N}$-barbed bisimilar to $Q$, written
$P \wbbisim_{\mathcal N} Q$, if $P \rel{S}_{\mathcal N} Q$ for some ${\mathcal N}$-barbed bisimulation ${\mathcal S}_{\mathcal N}$.
\end{definition}

$\mathcal{R} \subseteq \pi \times \pi$

$P \mathcal{R} Q => \forall P'. P \red P' \Rightarrow \exists Q'. Q \red Q', P' \mathcal{R} Q'$

$P \vdash x \Rightarrow Q \vdash x$

\begin{mathpar}
  \inferrule*[lab=Out-barb]{x \nameeq y}{{y}!\langle{Q}\rangle \vdash x}
  \and
  \inferrule*[lab=Par-barb]{\mbox{$P\vdash x$ or $Q\vdash x$}}{\binpar{P}{Q} \vdash x}
\end{mathpar}

\subsubsection{Contexts}

One of the principle advantages of computational calculi like the
$\pi$-calculus is a well-defined notion of context,
contextual-equivalence and a correlation between
contextual-equivalence and notions of bisimulation. The notion of
context allows the decomposition of a process into (sub-)process and
its syntactic environment, its context. Thus, a context may be
thought of as a process with a ``hole'' (written $\Box$) in it. The
application of a context $M$ to a process $P$, written $M[P]$, is
tantamount to filling the hole in $M$ with $P$. In this paper we do
not need the full weight of this theory, but do make use of the notion
of context in the proof the main theorem. 

\begin{mathpar}
  \inferrule* [lab=summation] {} {{M_{M},M_{N}} \bc \Box \;|\; x.M_{A} \;|\; M_{M}+M_{N}}
  \and
  \inferrule* [lab=agent] {} {{M_{A}} \bc (\vec{x})M_{P} \;| \; \clift{P_0,\ldots,M_{P},\ldots,P_N}}
  \and \\
  \inferrule* [lab=process] {} {{M_{P}} \bc M_{N} \;| \;P|M_{P} }
\end{mathpar} 

\begin{mathpar}
  \inferrule* [lab=sychronization] {} {M_{N} \bc \Box \;|\; x?M_{F} \;|\; x!M_{C}}
  \and
  \inferrule* [lab=abstraction] {} {{M_{F}} \bc (x)M_{P} }
  \and
  \inferrule* [lab=concretion] {} {{M_{C}} \bc \langle M_{P} \rangle }
  \and \\
  \inferrule* [lab=process] {} {{M_{P}} \bc M_{N} \;| \;P|M_{P} }
\end{mathpar}

\begin{definition}[contextual application] Given a context $M$, and
  process $P$, we define the \emph{contextual application}, $M[P] :=
  M\{P/\Box\}$. That is, the contextual application of M to P is the
  substitution of $P$ for $\Box$ in $M$.
\end{definition}

$\meaningof{-} : L \to \mathcal{P}(\pi)$

\begin{mathpar}
  \inferrule* [lab=collection] {} {\meaningof{true} = \pi, \and \meaningof{~E} = \pi \setminus \meaningof{E}, \and \meaningof{E_{1} \& E_{2}} = \meaningof{E_{1}} \cap \meaningof{E_{2}}}
\end{mathpar}

\begin{mathpar}
  \inferrule* [lab=structure] {} {\meaningof{0} = \{ P \in \pi | P \equiv 0 \}, \and \\ \meaningof{E_1 | E_2} = \{ P \in \pi | P \equiv P_{1} | P_{2}, P_{1} \in \meaningof{E_{1}}, P_{2} \in \meaningof{E_2}\} }
\end{mathpar}

\begin{mathpar}
 \inferrule* [lab=behavior] {} {\meaningof{\langle a?b \rangle E} = \{ P \in \pi | P \equiv Q | u?(y)P', \\ \and \\\\ \and \\ \;\;\; u \in \meaningof{a}, \forall z.P'\{z/y\} \in \meaningof{E\{z/b\}}\}, \and \\ \meaningof{a!E} = \{ P \in \pi | P \equiv Q | x!\langle P' \rangle, x \in \meaningof{a} P' \in \meaningof{E}\} }
\end{mathpar}

\begin{mathpar}
 \inferrule* [lab=nominal] {} {\meaningof{\quotep{E}} = \{ \quotep{P} \in \quotep{\pi} | P \in \meaningof{E} \}, \and \meaningof{\quotep{P}} = \{ \quotep{Q} \in \quotep{\pi} | P \equiv Q \} \and \\ \meaningof{@\quotep{E}} = \{ P \in \pi | P \equiv @x, x \in \meaningof{E} \}}
\end{mathpar}

\begin{eqnarray*}
  \\
  \meaningof{-} : TS \to ST
\end{eqnarray*}

\begin{eqnarray*}
  \\
  L : TS \to ST
\end{eqnarray*}

\begin{eqnarray*}
  \\
  P \models E \iff P \in \meaningof{E}
\end{eqnarray*}

\begin{eqnarray*}
  P \approx_{L} Q \iff \forall E \in L. P \models E \iff Q \models E
\end{eqnarray*}

\begin{eqnarray*}
  P \approx_{K} Q
\end{eqnarray*}

\begin{eqnarray*}
  P \approx Q
\end{eqnarray*}

$\approx_{K} = \approx = \approx_{L}$

\subsubsection{Contextual duality}

Note that contexts extend the quotation operation to a family of
operations from processes to names. Given a context, $M$, we can
define a \emph{nominal context}, $\quotep{M}$ by $\quotep{M}[P] :=
\quotep{M[P]}$. To foreshadow what is to come we observe that these
operations enjoy a duality with processes very much like the duality
between vectors and maps from vectors to scalars.

Further, because the calculus is essentially higher-order, we have a
correspondence between contexts and processes. More specifically,
given a name $x$ and a context $M$ we can construct $M^{*}_{x}$ such
that 

\begin{mathpar}
  M^{*}_{x} | \lift{x}{P} \red M[P]
\end{mathpar}

namely,

\begin{mathpar}
  M^{*}_{x} := x?(u).M[\dropn{u}]
\end{mathpar}

The dependence of $M^{*}_{x}$ on a name makes it an abstraction, 

\begin{mathpar}
  M^{*} := (x)x?(u).M[\dropn{u}]
\end{mathpar}

\subsection{Additional notation}

It will sometimes be convenient to denote the process a name
quotes. We already have the notation $x = \quotep{P}$, but it will be
convenient to introduce an alternate notation, $\procn{x}$, when we
want to emphasize the connection to the use of the name. Note that, by
virtue of name equivalence, $\quotep{\procn{x}} \nameeq x$; so, the
notation is consistent with previous definitions.

Further, because names have structure it is possible to effect
substitutions on the basis of that structure. This means we need to
upgrade our notation for substitutions, which we accomplish by
adapting comprehension notation. Thus,

\begin{mathpar}
  P\{ y / x : x \in S \}
\end{mathpar}

is interpreted to mean the process derived from P by replacing (in a
capture-avoiding manner) each occurrence of $x$ in $S$ by $y$. For example,

\begin{mathpar}
  P\{ \quotep{\procn{x}|\procn{x}} / x : x \in \freenames{P} \}
\end{mathpar}

will replace each (occurrence) of a free name $x$ in $P$ by
$\quotep{\procn{x}|\procn{x}}$.

Also, we will avail ourselves of the notation $x^{L}$ and $x^{R}$ to
denote injections of a name into disjoint copies of the name
space. There are numerous ways to accomplish this. One example can be
found in \cite{MeredithR05}. This notation overloads to vectors of
names: $\vec{x}^{\pi} := (x_{i}^{\pi} \; : \; 0 \leq i < |\vec{x}| )$ where $\pi \in \{L,R\}$.

We also use $P^{\Box} := P|\Box$.

In \cite{MeredithR05} an interpretation of the new operator is
given. It turns out that there are several possible interpretations
all enjoying the requisite algebraic properties of the operator (see
\cite{milner91polyadicpi}). We will therefore make liberal use of
$(\nu\; \vec{x})P$.

% subsection the_syntax_and_semantics_of_the_notation_system (end)   

\input{qm2pi.qmops} 

\input{qm2pi.sterngerlach} 

\input{qm2pi.metric} 

% section concurrent_process_calculi (end)

%\input{qm2pi.proofsketch}

% section proof sketch (end)

%\input{qm2pi.slviaknots} 

% section spatial logic via knots (end)

\input{qm2pi.conclusion}

% section conclusion (end)

%\input{qm2pi.dtcodes} 

% section wiring algorithm (end)

\input{qm2pi.ack} 

% section acknowledgments (end)

\newpage


\bibliographystyle{plain}   
\bibliography{../../biblios/main.bib}

\input{qm2pi.rhodetails}

\end{document}



% section proof sketch (end)

%\section{Unlikely characters: spatial logic for
  knots}\label{sub:characteristic_formulae} % (fold)

Associated to the mobile process calculi are a family of logics known
as the Hennessy-Milner logics. These logics typically enjoy a
semantics interpreting formulae as sets of processes that when
factored through the encoding outlined above allows an identification
of classes of knots with logical formulae. In the context of this
encoding the sub-family known as the spatial logics \cite{CairesC03}
\cite{CairesC04} \cite{Caires04} are of particular interest providing
several important features for expressing and reasoning about
properties (i.e. classes) of knots. We hint here at how this may be done.

%\begin{description}
%\item [structural connectives] 
\subsubsection{Structural connectives} The spatial logics enjoy
structural connectives corresponding, at the logical level, to the
parallel composition ($P | Q$) and new name ($(\nu \; x)P$)
connectives for processes. As illustrated in the examples below, these
connectives are extremely expressive given the shape of our encoding.
%\item [decideable satisfaction]

\subsubsection{Decideable satisfaction}
In \cite{Caires04} the satisfaction relation is shown to be decideable
for a rich class of processes. It further turns out that the image of
the our encoding is a proper subset of that class. This result
provides the basis for an algorithm by which to search for knots
enjoying a given property.
%\item [characteristic formulae]

\subsubsection{Characteristic formulae}
In the same paper \cite{Caires04} , Caires presents a means of calculating
characteristic formulae, selecting equivalence classes of processes
up to a pre--specified depth limit on the support set of names. Composed with our
encoding, this characteristic formula can be used to select
characteristic formulae for knots.
%\end{description}

\subsubsection{Spatial logic formulae}

The grammar below (segmented for comprehension) summarizes the syntax
of spatial logic formulae. We employ illustrative examples in the
sequel to provide an intuitive understanding of their meaning
referring the reader to \cite{Caires04} for a more detailed explication
of the semantics.

\begin{mathpar}
  \inferrule* [lab=boolean] {} {{A,B} \bc T \;|\; \neg A \;|\; A \wedge B \;|\; \eta = \eta'}
  \and
  \inferrule* [lab=spatial] {} {|\; \pzero \;|\; A | B \;|\; x \text{\textregistered} A \;|\; \forall x . A \;|\;  H x . A}
  \and
  \inferrule* [lab=behavioral] {} {|\; \alpha . A}
  \and 
  \inferrule* [lab=recursion] {} {|\; X(\vec{u}) \;|\; \mu X(\vec{u}) . A}
  \and
  \inferrule* [lab=action] {} {\alpha \bc \langle x?(\vec{y}) \rangle \;|\; \langle x!(\vec{y}) \rangle \;|\; \langle \tau \rangle}
  \and 
  \inferrule* [lab=name] {} {\eta \bc x \;|\; \tau}
\end{mathpar} 

% subsection characteristic_formulae (end)   	 

\subsection{Example formulae}\label{sub:example_formulae_} % (fold)

\subsubsection{Crossing as formula.}
% 
% \begin{align*}
%   \frac{d}{dx} \sin x &= \cos x 
%   & \frac{d}{dx} e^x &= e^x \\
%   \frac{d}{dx} \cos x &= - \sin x 
%   & \frac{d}{dx} \log x &= \frac{1}{x} \\
% \end{align*} 

\begin{align*}
 \mu C(x_{0},x_{1},y_{0},y_{1},u).&(\langle x_{0}?(z) \rangle(\langle u! \rangle\langle y_{1}!z \rangle C(x_{0},x_{1},y_{0},y_{1},u)) & \\
  & \wedge \langle y_{1}?(z) \rangle (\langle u! \rangle \langle x_{0}!z \rangle C(x_{0},x_{1},y_{0},y_{1},u)) & \\
  & \wedge \langle x_{1}?(z) \rangle (\langle u? \rangle \langle y_{0}!z \rangle C(x_{0},x_{1},y_{0},y_{1},u)) & \\
  & \wedge \langle y_{0}?(z) \rangle (\langle u? \rangle \langle x_{1}!z \rangle C(x_{0},x_{1},y_{0},y_{1},u))) &
\end{align*}

The lexicographical similarity between the shape of this formulae and
the shape of definition of the process representing a crossing reveals
the intuitive meaning of this formulae. It describes the capabilities
of a process that has the right to represent a crossing. For example
it picks out processes that may perform an input on the port $x_0$ in
its initial menu of capabilities. What differentiates the formula
from the process, however, is that the crossing process is the
smallest candidate to satisfy the formula. Infinitely many other
processes -- with internal behavior hidden behind this interface, so
to speak -- also satisfy this formula. Even this simple formula,
then, can be seen to open a new view onto knots, providing a
computational interpretation of \emph{virtual} knots.

Note that this formula is derived by hand. A similar formula can be
derived by employing Caires' calculation of characteristic formula
\cite{Caires04} to the process representing a crossing. In light of
this discussion, we let
$\meaningof{C}_{\phi}(x0,x1,y0,y1,u)$ denote a formula specifying the
dynamics we wish to capture of a crossing. To guarantee we preserve
the shape of the interface and minimal semantics we demand that
$\meaningof{C}_{\phi}(x0,x1,y0,y1,u) \Rightarrow
\textbf{C}(x0,x1,y0,y1,u)$ where $\textbf{C}(x0,x1,y0,y1,u)$ denotes
the formula above.
                            
\subsubsection{Crossing number constraints.}
The moral content of the context lemma (Lemma \ref{context}) is that the notion of
``locality'' in the Reidemeister moves is effectively captured by the
parallel composition operator of the process calculus. This intuition
extends through the logic. Given a formula,
$\meaningof{C}_{\phi}(x0,x1,y0,y1,u)$, we can use the structural
connectives to specify constraints on crossing numbers, such as at
least $n$ crossings, or exactly $n$ crossings.
\begin{mathpar}
  \inferrule* [lab=at-least-n] {} { K^{\geq n}_{\phi}(\vec{xs},\vec{ys}) := \Pi_{i=0}^{n-1} Hu . \meaningof{C}_{\phi}(xs_i,ys_i,u) | T }
  \and 
  \inferrule* [lab=exactly-n] {} { K^{= n}_{\phi}(\vec{xs},\vec{ys}) := \Pi_{i=0}^{n-1} Hu . \meaningof{C}_{\phi}(xs_i,ys_i,u) | \neg (\forall x_0,y_0,x_1,y_1,u . \meaningof{C}_{\phi}(x_0,y_0,x_1,y_1,u) | T) }
\end{mathpar}

To round out this section, recall that the encoding of an $n$-crossing
knot decomposes into a parallel composition of $n$ \emph{copies} of a
crossing process together with a wiring harness. To specify different
knot classes with the same crossing number amounts to specifying
logical constraints on the wiring harness. In the interest of space,
we defer examples to a forthcoming paper. Suffice it to say that both
the conditions ``alternating knot'' and ``contains the tangle
corresponding to 5/3'' are expressible. For example, it is possible to
calculate the characteristic formula of a process corresponding to the
tangle 5/3 and conjoin it into the classifying formula via the
composition connective of the logic.

Finally, we wish to observe that it is entirely within reason to
contemplate a more domain-specific version of spatial logic tailored
to the shape of processes in the image of the encoding. Such a
domain-specific logic would have a better claim to the title formal
language of knot properties.

% subsection example_formulae_ (end)

% section knots_as_processes (end) 

% section spatial logic via knots (end)

\section{Conclusions and future work}

\paragraph{Testing physical space}
You, gentle reader, may wonder why of all the theorems to be proved
given this set up we pick the one above. In some sense it's hardly
central to quantum mechanics. We see it as central in the sense that
it firmly establishes a notion of physical space arising from a notion
of the equivalence of behavior. Relating bisimulation to a metric is a
big step forward, but one is faced with interpreting the relationship
of that metric space to something more physical. Quantum mechanical
notions of ``physical'' space are still far from intuitive, but by
relating this idea of distance as testing to calculations that predict
physical circumstances we are making a not insignificant step forward
toward an understanding of the physical space we inhabit as
essentially dynamic.

\paragraph{Effectivity and simulation}
One of the observations we have yet to make is that the entire program
spelled out here is effective. We have built various interpreters for
the reflective calculus at work in this interpretation. In principle,
then, we can simulate quantum mechanics on a computer. The place where
the simulation may lose fidelity is the infinitely branching summation
for the annihilator.

In this connection i also want to point out that the evaluation style
calculation of the inner product puts the non-determinism of the
summation right at the heart of measurement. This suggests that
Milner's original reduction-based formulation of the dynamics of his
calculi in terms of sums was not just notationally suggestive of a
notion of measure-and-continue but captured some significant part of
the physics.

\paragraph{Quantum continuations}
In light of this last observation i want to point out that the
predominant account of quantum mechanics is missing a key aspect of a
truly compositional story of the physical situation. In a real lab,
when a measurement is made the observation can be made to feed into
another device that then makes another measurement conditioned on the
results of the first. This means that after the superposition was
collapsed the entire experimental set up remained in
superposition. While QM offers a means of writing this down it doesn't
quite line up well with the well-trodden formulation of computation
and continuation that we see so succinctly expressed in Milner's
calculi. This suggests that there might be advantages to this account
of dynamics waiting to be explored.

\paragraph{Quantum logic}
In this connection, we also note that by virtue of having the
Hennessy-Milner construction, we can pull the construction through the
interpretation of QM. This gives us a natural candidate for a quantum
logic that enjoys an extremely tight connection with it's domain of
interpretation, making the construction much less ad hoc (rather it is
the image of functor!).

\paragraph{Quantum probabiity}
i have questions about the basis of the interpretation of inner
product as probability amplitude. In particular, using which
axiomatization of probability theory does the notion of probability
amplitude earn the right to be so dubbed? In other words, where is the
proof that the operation for calculating a probability amplitude (and
then squaring) satisfies the axioms of what it means to calculate a
probability? Even if such a proof exists (i have yet to find it in the
literature), i wonder if it might not be possible to turn things on
their heads. Can we view the calculation of the probability amplitude
as an axiomatization of probability? If so, then the definition we
give for calculating probability amplitude may provide the basis for
an \emph{effective} theory of probability.

\paragraph{Quantum vs ``biological'' information}
Finally, i want to conclude with a more philosophical observation. At
a recent workshop in which QM was a predominant topic i noticed
something about quantum information. The speaker was giving a riveting
discussion of axiomatic QM and showing how properties of ``no
cloning'' and ``no deleting'' emerged as consequences of the
axiomatization. Theorems of this form are necessary to give us a sense
of confidence that our axioms characterize the physical theory. What
struck me, though, was that if quantum information is neither erasable
nor replicable it is markedly different from \emph{life}. Two of the
things we know about life is that

\begin{itemize}
  \item it ends;
  \item to gain some measure of persistence, to transcend it's
    finitude it is imminently copyable.
\end{itemize}

Both of these qualities are summarized succinctly in the aphorism: all
flesh is grass. For me these two kinds of ``information'' -- call them
quantum and biological -- are end points on a spectrum of strategies
for persistence. At one end, we have those curious entities that enjoy
uniqueness and permanence; at the other, we have those who in the face
of a certain end and an uncertain present make a go of passing
something on. To me one of the more remarkable aspects of the latter
strategy is that in the presence of noise (and certain features of
copying) we get a kind of dynamism, a chance for improvement against a
given persistent condition.

% subsection other_calculi_other_bisimulations_and_geometry_as_behavior (end)




% section conclusion (end)

%\documentclass[12pt]{llncs}
%\documentclass{jktr}

\usepackage[pdftex]{hyperref}                   
\usepackage {listings}
\usepackage {mathpartir}
\usepackage{bcprules}
%\usepackage{listings}
                       
\usepackage{graphicx} 
%\usepackage[margins=2.5cm,nohead,nofoot]{geometry}
%\usepackage{geometry}
\usepackage{amsfonts}
\usepackage{amstext}
\usepackage{latexsym}
\usepackage{amssymb}
\usepackage{color}


%\include{myPreamble}
\include{qm2pi.local} 

%\ifpdf
%\usepackage[pdftex]{graphicx}
%\else
%\usepackage{graphicx}
%\fi

 % \ifpdf
%  \usepackage{pdfsync}
%  \if


%\title{Brief Article}
%\author{David F. Snyder}
%\author{L.G. Meredith}

%\address{Dept. of Math., Texas State University--San Marcos, San Marcos, TX 78666}
       
\pagestyle{empty}


\begin{document}

\lstset{language=[Objective]Caml,frame=shadowbox}

\input{qm2pi.front}

% section front matter (end)

\input{qm2pi.intro} 
 
% section introduction (end)

% \input{qm2pi.knotations} 

% section notation (end)

\input{qm2pi.process.calculi} 

% section concurrent_process_calculi_and_spatial_logics_ (end)
    
%\input{qm2pi.knots2pi} 

%\input{qm2pi.trefoil} 

%\input{qm2pi.mainthm} 

% subsection basic_interpretation (end)

%\input{qm2pi.rho.presentation} 
\subsection{The syntax and semantics of the notation system}\label{sub:the_syntax_and_semantics_of_the_notation_system} % (fold)

We now summarize a technical presentation of the calculus that
embodies our theory of dynamics. The typical presentation of such a
calculus follows the style of giving generators and relations on
them. The grammar, below, describing term constructors, freely
generates the set of processes, $\Proc$. This set is then quotiented
by a relation known as structural congruence and it is over this set
that the notion of dynamics is expressed. This presentation is
essentially that of \cite{MeredithR05} with the addition of
polyadicity and summation. For readability we have relegated some of
the technical subtleties to an appendix.

\subsubsection{Process grammar}\label{subsub:process_grammar}

\begin{mathpar}
  \inferrule* [lab=synchronization] {} {{M} \bc \pzero \;|\; x?F \;|\; x!C }
  \and
  \inferrule* [lab=abstraction] {} {{F} \bc (x)P}
  \and
  \inferrule* [lab=concretion] {} {{C} \bc \langle Q \rangle}
  \and
  \inferrule* [lab=process] {} {{P,Q} \bc M \;| \;P|Q \;|\; @{x}}
  \and
  \inferrule* [lab=name] {} {{x} \bc \quotep{P}}
\end{mathpar} 

Note that $\vec{x}$ (resp. $\vec{P}$) denotes a vector of names
(resp. processes) of length $|\vec{x}|$ (resp. $|\vec{P}|$). We adopt
the following useful abbreviations.

\begin{mathpar}
   x?(\vec{y}).P := x.(\vec{y})P \and  x\clift{\vec{P}} := x.\clift{\vec{P}}
   \and x!(y) := \lift{x}{\dropn{y}}
   \and \Pi_{i=0}^{n-1}P_i := P_0 | \ldots | P_{n-1}
\end{mathpar}

\subsubsection{Structural congruence}

\paragraph{Free and bound names and alpha-equivalence.} At the
core of structural equivalence is alpha-equivalence which identifies
process that are the same up to a change of variable. Formally, we
recognize the distinction between free and bound names. The free names
of a process, $\freenames{P}$, may be calculated recursively as
follows:

\begin{mathpar}
\freenames{\pzero} := \emptyset
  \and \\
  \freenames{x?(y).P} := \{ x \} \cup (\freenames{P} \setminus \{ y \})
  \and 
  \freenames{x!\langle P \rangle} := \{ x \} \cup \{ P \} 
  \and \\
  \freenames{P|Q} := \freenames{P} \cup \freenames{Q}
  \and \\
  \freenames{@{x}} := \{ x \}
\end{mathpar}

$\pi$
$\quotep{\pi}$

$\freenames{-} : \pi \to \mathcal{P}(\quotep{\pi})$

\begin{eqnarray*}
  \freenames{\pzero} & := & \emptyset \\
  \freenames{x?(y).P} & := & \{ x \} \cup (\freenames{P} \setminus \{ y \}) \\
  \freenames{x!\langle P \rangle} & := & \{ x \} \cup \{ P \} \\
  \freenames{P|Q} & := & \freenames{P} \cup \freenames{Q} \\
  \freenames{\dropn{x}} & := & \{ x \}
\end{eqnarray*}

The bound names of a process, $\boundnames{P}$, are those names occurring in $P$
that are not free. For example, in $x?(y).0$, the name $x$ is free, while $y$ is bound.

\begin{mathpar}
  \inferrule* [lab=monoidal-laws] {} { P|Q \equiv Q|P \and P|0 \equiv P \and P|(Q|R) \equiv (P|Q)|R }
\end{mathpar}

\begin{mathpar}
  \inferrule* [lab=alpha-equivalence] {} { (x)P \equiv (y)P\{y/x\} \and y \not\in \freenames{P} }
\end{mathpar}

\begin{definition}
Then two processes, $P,Q$, are alpha-equivalent if $P = Q\{\vec{y}/\vec{x}\}$ for
some $\vec{x} \in \boundnames{Q},\vec{y} \in \boundnames{P}$, where $Q\{\vec{y}/\vec{x}\}$
denotes the capture-avoiding substitution of $\vec{y}$ for $\vec{x}$ in $Q$.
\end{definition}

\begin{definition}
  The {\em structural congruence} \cite{SangiorgiWalker} , $\equiv$,
  between processes is the least congruence containing
  alpha-equivalence, satisfying the abelian monoid laws
  (associativity, commutativity and $\pzero$ as identity) for parallel
  composition $|$ and for summation $+$.
\end{definition}

\subsection{Name equivalence}

We take name equivalence, written $\nameeq$, to be the smallest
equivalence relation generated by the following rules.

\begin{mathpar}
\inferrule*[lab=Quote-drop]
{ }
{ \quotep{@{x}} \nameeq x }

\inferrule*[lab=Struct-equiv]
{ P \scong Q }
{ \quotep{P} \nameeq \quotep{Q} }
\end{mathpar}

The astute reader will have noticed that the mutual recursion of names
and processes imposes a mutual recursion on alpha-equivalence and
structural equivalence via name-equivalence. Fortunately, all of this
works out pleasantly and we may calculate in the natural way, free of
concern. The reader interested in the details is referred to the
appendix \ref{appendix:rho_details}.

\subsection{Substitution}

We use $\Proc$ for the set of processes, $\QProc$ for the set of
names, and $\id{\{}\vec{y} / \vec{x} \id{\}}$ to denote partial maps,
$s : \QProc \rightarrow \QProc$. A map, $s$ lifts, uniquely, to a map
on process terms, $\widehat{s} : \Proc \rightarrow \Proc$ by the
following equations.

\begin{mathpar}
  (0) \psubstp{Q}{P} := 0 \\
  (R \juxtap S) \psubstp{Q}{P}
  :=    
  (R)\psubstp{Q}{P} \juxtap (S) \psubstp{Q}{P} \\
  (x?(y).R) \psubstp{Q}{P}    
  :=    
  (x)\substp{Q}{P} (z)\concat( (R \psubstn{z}{y}) \psubstp{Q}{P} ) \\
  (\lift{x}{R}) \psubstp{Q}{P}  
  :=
  \lift{(x)\substp{Q}{P}}{ R \psubstp{Q}{P} } \\
%   (\dropn{x})  \psubstp{Q}{P}       
%   := 
%   \left\{ 
%     \begin{array}{ccc} 
%       \dropn{\quotep{Q}} & & x \nameeq \quotep{P} \\
%       \dropn{x} & & otherwise \\
%     \end{array}
%   \right. 
  (\dropn{x})  \psubstp{Q}{P}       
  := 
  \left\{ 
    \begin{array}{ccc} 
      Q & & x \nameeq \quotep{P} \\
      \dropn{x} & & otherwise \\
    \end{array}
  \right.
\end{mathpar}
 

where

\begin{eqnarray}
  (x)\id{\{} \lpquote Q \rpquote / \lpquote P \rpquote \id{\}}            = 
  \left\{ 
    \begin{array}{ccc}
      \lpquote Q \rpquote & & x \nameeq \lpquote P \rpquote \\
      x & & otherwise \\
    \end{array}
  \right. \nonumber
\end{eqnarray}

and $z$ is chosen distinct from $\quotep{P}$, $\quotep{Q}$, the free
names in $Q$, and all the names in $R$. Our $\alpha$-equivalence will
be built in the standard way from this substitution.

\begin{remark}\label{rem:no_self_referential_names}
  One consequence of these definitions is that $\forall P. \quotep{P}
  \not\in \freenames{P}$.
\end{remark}

\subsection{ Dynamic quote: an example }

Anticipating something of what's to come, consider applying the
substitution, $\widehat{\id{\{}u / z \id{\}}}$, to the following pair
of processes, $\lift{w}{y!(z)}$ and $w[ \lpquote y!(z) \rpquote ]$.

\begin{eqnarray}
	\lift{w}{y!(z)}\widehat{\id{\{}u / z \id{\}}}
		& = &
		\lift{w}{y!(u)} \nonumber\\
	w[ \lpquote y!(z) \rpquote ] \widehat{ \id{\{}u / z \id{\}} }
		& = &
		w[ \lpquote y!(z) \rpquote ] \nonumber
\end{eqnarray}

Because the body of the process between quotes is impervious to
substitution, we get radically different answers. In fact, by
examining the first process in an input context,
e.g. $x?(z).\lift{w}{y!(z)}$, we see that the process under the lift
operator may be shaped by prefixed inputs binding a name inside it. In
this sense, the lift operator will be seen as a way to dynamically
construct processes before reifying them as names.

Finally equipped with these standard features we can present the
dynamics of the calculus.

\subsubsection{Operational semantics} 

Finally, we introduce the computational dynamics. What marks these
algebras as distinct from other more traditionally studied algebraic
structures, e.g. vector spaces or polynomial rings, is the manner in
which dynamics is captured. In traditional structures, dynamics is typically
expressed through morphisms between such structures, as in linear maps
between vector spaces or morphisms between rings. In algebras
associated with the semantics of computation, the dynamics is
expressed as part of the algebraic structure itself, through a
reduction reduction relation typically denoted by $\red$. Below, we
give a recursive presentation of this relation for the calculus used
in the encoding.

$\red \subseteq \pi \times \pi$
$\red : \pi \to \mathcal{P}(\pi)$

\begin{mathpar}
  \inferrule* [lab=Comm] { \textsf{match}( x_{src}, x_{trgt} ) } { x_{trgt}?(y)P \; | \; x_{src}!\langle {Q} \rangle \red P\{\quotep{Q}/y}\} }
  \and \\
  \inferrule* [lab=Par] {{P} \red {P}'} {{{P} | {Q}} \red {{P}' | {Q}}}
  \and
  \inferrule* [lab=Equiv]{{{P} \scong {P}'} \andalso {{P}' \red {Q}'} \andalso {{Q}' \scong {Q}}}{{P} \red {Q}}
\end{mathpar}

\begin{eqnarray*}
  match_{\equiv} (\quotep{P},\quotep{Q}) & := & P \equiv Q \\
  match_{\dagger}(\quotep{P},\quotep{Q}) & := & \forall R. P|Q \red^{*} R => R \red^{*} 0 \\
  match_{K}(\quotep{P},\quotep{Q}) & := & K \mbox{ for some context } K
\end{eqnarray*}

$u?(x)P | u!\langle Q \rangle \red P\{\quotep{Q}/x\}$

%We write $\wred$ for $\red^*$, and $P\red$ if $\exists Q $ such that $ P \red Q$.
We write $P\red$ if $\exists Q $ such that $ P \red Q$ and $P\not\red$, otherwise.

\section{Replication}

As mentioned before, it is known that replication (and hence
recursion) can be implemented in a higher-order process algebra
\cite{SangiorgiWalker}. As our first example of calculation with the
machinery thus far presented we give the construction explicitly in
the {\rhoc}.

\begin{eqnarray}
	D_{x} & := & \prefix{x}{y}{(\binpar{\outputp{x}{y}}{@{y}})} \nonumber\\
	\bangp_{x}{P} & := & \binpar{{x}!\langle{\binpar{D_{x}}{P}}\rangle}{D_{x}} \nonumber
\end{eqnarray}

\begin{eqnarray}
	\bangp_{x}{P} & & \nonumber\\
	=
	& {x}!\langle{(\prefix{x}{y}{(\outputp{x}{y} | @{y})) | P}}\rangle 
	      | \prefix{x}{y}{(\outputp{x}{y} | @{y})} & \nonumber\\
	\red
	& (\outputp{x}{y} | @{y})\substn{\quotep{(\prefix{x}{y}{(@{y} | \outputp{x}{y})) | P}}}{y} & \nonumber\\
	=
	& \outputp{x}{\quotep{(\prefix{x}{y}{(\outputp{x}{y} | @{y})) | P}}}
	  | {(\prefix{x}{y}{(\outputp{x}{y} | @{y})) | P}} & \nonumber\\
	\red
	& \ldots & \nonumber\\
	\red^*
	& P | P | \ldots & \nonumber
\end{eqnarray}

Of course, this encoding, as an implementation, runs away, unfolding
$\bangp{P}$ eagerly. A lazier and more implementable replication
operator, restricted to input-guarded processes, may be obtained as follows.

\begin{eqnarray}
\bangp{\prefix{u}{v}{P}} 
	:= 
	\binpar{\lift{x}{\prefix{u}{v}{(\binpar{D(x)}{P})}}}{D(x)} \nonumber
\end{eqnarray}

\begin{remark}
  Note that the lazier definition still does not deal with summation
  or mixed summation (i.e. sums over input and output). The reader is
  invited to construct definitions of replication that deal with these
  features. 

  Further, the definitions are parameterized in a name, $x$. Can you,
  gentle reader, make a definition that eliminates this parameter and
  guarantees no accidental interaction between the replication
  machinery and the process being replicated -- i.e. no accidental
  sharing of names used by the process to get its work done and the
  name(s) used by the replication to effect copying. This latter
  revision of the definition of replication is crucial to obtaining
  the expected identity $!!P \sim !P$.
\end{remark}

\begin{remark}\label{rem:paradoxical_combinator}
  The reader familiar with the lambda calculus will have noticed the
  similarity between $D$ and the paradoxical combinator.

  [Ed. note: the existence of this seems to suggest we have to be more
  restrictive on the set of processes and names we admit if we are to
  support no-cloning.]
\end{remark}

\subsubsection{Bisimulation}

The computational dynamics gives rise to another kind of equivalence,
the equivalence of computational behavior. As previously mentioned
this is typically captured \emph{via} some form of bisimulation.

% The notion we use in this paper is weak barbed bisimulation
% \cite{milner91polyadicpi}.

The notion we use in this paper is derived from weak barbed
bisimulation \cite{milner91polyadicpi}. 

\begin{definition}
An \emph{observation relation}, $\downarrow_{\mathcal N}$, over a set
of names, $\mathcal N$, is the smallest relation satisfying the rules
below.

\infrule[Out-barb]{y \in {\mathcal N}, \; x \nameeq y}
		  {\outputp{x}{v} \downarrow_{\mathcal N} x}
\infrule[Par-barb]{\mbox{$P\downarrow_{\mathcal N} x$ or $Q\downarrow_{\mathcal N} x$}}
		  {\binpar{P}{Q} \downarrow_{\mathcal N} x}

We write $P \Downarrow_{\mathcal N} x$ if there is $Q$ such that 
$P \wred Q$ and $Q \downarrow_{\mathcal N} x$.
\end{definition}

\begin{definition}
%\label{def.bbisim}
An  ${\mathcal N}$-\emph{barbed bisimulation} over a set of names, ${\mathcal N}$, is a symmetric binary relation 
${\mathcal S}_{\mathcal N}$ between agents such that $P\rel{S}_{\mathcal N}Q$ implies:
\begin{enumerate}
\item If $P \red P'$ then $Q \wred Q'$ and $P'\rel{S}_{\mathcal N} Q'$.
\item If $P\downarrow_{\mathcal N} x$, then $Q\Downarrow_{\mathcal N} x$.
\end{enumerate}
$P$ is ${\mathcal N}$-barbed bisimilar to $Q$, written
$P \wbbisim_{\mathcal N} Q$, if $P \rel{S}_{\mathcal N} Q$ for some ${\mathcal N}$-barbed bisimulation ${\mathcal S}_{\mathcal N}$.
\end{definition}

$\mathcal{R} \subseteq \pi \times \pi$

$P \mathcal{R} Q => \forall P'. P \red P' \Rightarrow \exists Q'. Q \red Q', P' \mathcal{R} Q'$

$P \vdash x \Rightarrow Q \vdash x$

\begin{mathpar}
  \inferrule*[lab=Out-barb]{x \nameeq y}{{y}!\langle{Q}\rangle \vdash x}
  \and
  \inferrule*[lab=Par-barb]{\mbox{$P\vdash x$ or $Q\vdash x$}}{\binpar{P}{Q} \vdash x}
\end{mathpar}

\subsubsection{Contexts}

One of the principle advantages of computational calculi like the
$\pi$-calculus is a well-defined notion of context,
contextual-equivalence and a correlation between
contextual-equivalence and notions of bisimulation. The notion of
context allows the decomposition of a process into (sub-)process and
its syntactic environment, its context. Thus, a context may be
thought of as a process with a ``hole'' (written $\Box$) in it. The
application of a context $M$ to a process $P$, written $M[P]$, is
tantamount to filling the hole in $M$ with $P$. In this paper we do
not need the full weight of this theory, but do make use of the notion
of context in the proof the main theorem. 

\begin{mathpar}
  \inferrule* [lab=summation] {} {{M_{M},M_{N}} \bc \Box \;|\; x.M_{A} \;|\; M_{M}+M_{N}}
  \and
  \inferrule* [lab=agent] {} {{M_{A}} \bc (\vec{x})M_{P} \;| \; \clift{P_0,\ldots,M_{P},\ldots,P_N}}
  \and \\
  \inferrule* [lab=process] {} {{M_{P}} \bc M_{N} \;| \;P|M_{P} }
\end{mathpar} 

\begin{mathpar}
  \inferrule* [lab=sychronization] {} {M_{N} \bc \Box \;|\; x?M_{F} \;|\; x!M_{C}}
  \and
  \inferrule* [lab=abstraction] {} {{M_{F}} \bc (x)M_{P} }
  \and
  \inferrule* [lab=concretion] {} {{M_{C}} \bc \langle M_{P} \rangle }
  \and \\
  \inferrule* [lab=process] {} {{M_{P}} \bc M_{N} \;| \;P|M_{P} }
\end{mathpar}

\begin{definition}[contextual application] Given a context $M$, and
  process $P$, we define the \emph{contextual application}, $M[P] :=
  M\{P/\Box\}$. That is, the contextual application of M to P is the
  substitution of $P$ for $\Box$ in $M$.
\end{definition}

$\meaningof{-} : L \to \mathcal{P}(\pi)$

\begin{mathpar}
  \inferrule* [lab=collection] {} {\meaningof{true} = \pi, \and \meaningof{~E} = \pi \setminus \meaningof{E}, \and \meaningof{E_{1} \& E_{2}} = \meaningof{E_{1}} \cap \meaningof{E_{2}}}
\end{mathpar}

\begin{mathpar}
  \inferrule* [lab=structure] {} {\meaningof{0} = \{ P \in \pi | P \equiv 0 \}, \and \\ \meaningof{E_1 | E_2} = \{ P \in \pi | P \equiv P_{1} | P_{2}, P_{1} \in \meaningof{E_{1}}, P_{2} \in \meaningof{E_2}\} }
\end{mathpar}

\begin{mathpar}
 \inferrule* [lab=behavior] {} {\meaningof{\langle a?b \rangle E} = \{ P \in \pi | P \equiv Q | u?(y)P', \\ \and \\\\ \and \\ \;\;\; u \in \meaningof{a}, \forall z.P'\{z/y\} \in \meaningof{E\{z/b\}}\}, \and \\ \meaningof{a!E} = \{ P \in \pi | P \equiv Q | x!\langle P' \rangle, x \in \meaningof{a} P' \in \meaningof{E}\} }
\end{mathpar}

\begin{mathpar}
 \inferrule* [lab=nominal] {} {\meaningof{\quotep{E}} = \{ \quotep{P} \in \quotep{\pi} | P \in \meaningof{E} \}, \and \meaningof{\quotep{P}} = \{ \quotep{Q} \in \quotep{\pi} | P \equiv Q \} \and \\ \meaningof{@\quotep{E}} = \{ P \in \pi | P \equiv @x, x \in \meaningof{E} \}}
\end{mathpar}

\begin{eqnarray*}
  \\
  \meaningof{-} : TS \to ST
\end{eqnarray*}

\begin{eqnarray*}
  \\
  L : TS \to ST
\end{eqnarray*}

\begin{eqnarray*}
  \\
  P \models E \iff P \in \meaningof{E}
\end{eqnarray*}

\begin{eqnarray*}
  P \approx_{L} Q \iff \forall E \in L. P \models E \iff Q \models E
\end{eqnarray*}

\begin{eqnarray*}
  P \approx_{K} Q
\end{eqnarray*}

\begin{eqnarray*}
  P \approx Q
\end{eqnarray*}

$\approx_{K} = \approx = \approx_{L}$

\subsubsection{Contextual duality}

Note that contexts extend the quotation operation to a family of
operations from processes to names. Given a context, $M$, we can
define a \emph{nominal context}, $\quotep{M}$ by $\quotep{M}[P] :=
\quotep{M[P]}$. To foreshadow what is to come we observe that these
operations enjoy a duality with processes very much like the duality
between vectors and maps from vectors to scalars.

Further, because the calculus is essentially higher-order, we have a
correspondence between contexts and processes. More specifically,
given a name $x$ and a context $M$ we can construct $M^{*}_{x}$ such
that 

\begin{mathpar}
  M^{*}_{x} | \lift{x}{P} \red M[P]
\end{mathpar}

namely,

\begin{mathpar}
  M^{*}_{x} := x?(u).M[\dropn{u}]
\end{mathpar}

The dependence of $M^{*}_{x}$ on a name makes it an abstraction, 

\begin{mathpar}
  M^{*} := (x)x?(u).M[\dropn{u}]
\end{mathpar}

\subsection{Additional notation}

It will sometimes be convenient to denote the process a name
quotes. We already have the notation $x = \quotep{P}$, but it will be
convenient to introduce an alternate notation, $\procn{x}$, when we
want to emphasize the connection to the use of the name. Note that, by
virtue of name equivalence, $\quotep{\procn{x}} \nameeq x$; so, the
notation is consistent with previous definitions.

Further, because names have structure it is possible to effect
substitutions on the basis of that structure. This means we need to
upgrade our notation for substitutions, which we accomplish by
adapting comprehension notation. Thus,

\begin{mathpar}
  P\{ y / x : x \in S \}
\end{mathpar}

is interpreted to mean the process derived from P by replacing (in a
capture-avoiding manner) each occurrence of $x$ in $S$ by $y$. For example,

\begin{mathpar}
  P\{ \quotep{\procn{x}|\procn{x}} / x : x \in \freenames{P} \}
\end{mathpar}

will replace each (occurrence) of a free name $x$ in $P$ by
$\quotep{\procn{x}|\procn{x}}$.

Also, we will avail ourselves of the notation $x^{L}$ and $x^{R}$ to
denote injections of a name into disjoint copies of the name
space. There are numerous ways to accomplish this. One example can be
found in \cite{MeredithR05}. This notation overloads to vectors of
names: $\vec{x}^{\pi} := (x_{i}^{\pi} \; : \; 0 \leq i < |\vec{x}| )$ where $\pi \in \{L,R\}$.

We also use $P^{\Box} := P|\Box$.

In \cite{MeredithR05} an interpretation of the new operator is
given. It turns out that there are several possible interpretations
all enjoying the requisite algebraic properties of the operator (see
\cite{milner91polyadicpi}). We will therefore make liberal use of
$(\nu\; \vec{x})P$.

% subsection the_syntax_and_semantics_of_the_notation_system (end)   

\input{qm2pi.qmops} 

\input{qm2pi.sterngerlach} 

\input{qm2pi.metric} 

% section concurrent_process_calculi (end)

%\input{qm2pi.proofsketch}

% section proof sketch (end)

%\input{qm2pi.slviaknots} 

% section spatial logic via knots (end)

\input{qm2pi.conclusion}

% section conclusion (end)

%\input{qm2pi.dtcodes} 

% section wiring algorithm (end)

\input{qm2pi.ack} 

% section acknowledgments (end)

\newpage


\bibliographystyle{plain}   
\bibliography{../../biblios/main.bib}

\input{qm2pi.rhodetails}

\end{document}

 

% section wiring algorithm (end)

\documentclass[12pt]{llncs}
%\documentclass{jktr}

\usepackage[pdftex]{hyperref}                   
\usepackage {listings}
\usepackage {mathpartir}
\usepackage{bcprules}
%\usepackage{listings}
                       
\usepackage{graphicx} 
%\usepackage[margins=2.5cm,nohead,nofoot]{geometry}
%\usepackage{geometry}
\usepackage{amsfonts}
\usepackage{amstext}
\usepackage{latexsym}
\usepackage{amssymb}
\usepackage{color}


%\include{myPreamble}
\include{qm2pi.local} 

%\ifpdf
%\usepackage[pdftex]{graphicx}
%\else
%\usepackage{graphicx}
%\fi

 % \ifpdf
%  \usepackage{pdfsync}
%  \if


%\title{Brief Article}
%\author{David F. Snyder}
%\author{L.G. Meredith}

%\address{Dept. of Math., Texas State University--San Marcos, San Marcos, TX 78666}
       
\pagestyle{empty}


\begin{document}

\lstset{language=[Objective]Caml,frame=shadowbox}

\input{qm2pi.front}

% section front matter (end)

\input{qm2pi.intro} 
 
% section introduction (end)

% \input{qm2pi.knotations} 

% section notation (end)

\input{qm2pi.process.calculi} 

% section concurrent_process_calculi_and_spatial_logics_ (end)
    
%\input{qm2pi.knots2pi} 

%\input{qm2pi.trefoil} 

%\input{qm2pi.mainthm} 

% subsection basic_interpretation (end)

%\input{qm2pi.rho.presentation} 
\subsection{The syntax and semantics of the notation system}\label{sub:the_syntax_and_semantics_of_the_notation_system} % (fold)

We now summarize a technical presentation of the calculus that
embodies our theory of dynamics. The typical presentation of such a
calculus follows the style of giving generators and relations on
them. The grammar, below, describing term constructors, freely
generates the set of processes, $\Proc$. This set is then quotiented
by a relation known as structural congruence and it is over this set
that the notion of dynamics is expressed. This presentation is
essentially that of \cite{MeredithR05} with the addition of
polyadicity and summation. For readability we have relegated some of
the technical subtleties to an appendix.

\subsubsection{Process grammar}\label{subsub:process_grammar}

\begin{mathpar}
  \inferrule* [lab=synchronization] {} {{M} \bc \pzero \;|\; x?F \;|\; x!C }
  \and
  \inferrule* [lab=abstraction] {} {{F} \bc (x)P}
  \and
  \inferrule* [lab=concretion] {} {{C} \bc \langle Q \rangle}
  \and
  \inferrule* [lab=process] {} {{P,Q} \bc M \;| \;P|Q \;|\; @{x}}
  \and
  \inferrule* [lab=name] {} {{x} \bc \quotep{P}}
\end{mathpar} 

Note that $\vec{x}$ (resp. $\vec{P}$) denotes a vector of names
(resp. processes) of length $|\vec{x}|$ (resp. $|\vec{P}|$). We adopt
the following useful abbreviations.

\begin{mathpar}
   x?(\vec{y}).P := x.(\vec{y})P \and  x\clift{\vec{P}} := x.\clift{\vec{P}}
   \and x!(y) := \lift{x}{\dropn{y}}
   \and \Pi_{i=0}^{n-1}P_i := P_0 | \ldots | P_{n-1}
\end{mathpar}

\subsubsection{Structural congruence}

\paragraph{Free and bound names and alpha-equivalence.} At the
core of structural equivalence is alpha-equivalence which identifies
process that are the same up to a change of variable. Formally, we
recognize the distinction between free and bound names. The free names
of a process, $\freenames{P}$, may be calculated recursively as
follows:

\begin{mathpar}
\freenames{\pzero} := \emptyset
  \and \\
  \freenames{x?(y).P} := \{ x \} \cup (\freenames{P} \setminus \{ y \})
  \and 
  \freenames{x!\langle P \rangle} := \{ x \} \cup \{ P \} 
  \and \\
  \freenames{P|Q} := \freenames{P} \cup \freenames{Q}
  \and \\
  \freenames{@{x}} := \{ x \}
\end{mathpar}

$\pi$
$\quotep{\pi}$

$\freenames{-} : \pi \to \mathcal{P}(\quotep{\pi})$

\begin{eqnarray*}
  \freenames{\pzero} & := & \emptyset \\
  \freenames{x?(y).P} & := & \{ x \} \cup (\freenames{P} \setminus \{ y \}) \\
  \freenames{x!\langle P \rangle} & := & \{ x \} \cup \{ P \} \\
  \freenames{P|Q} & := & \freenames{P} \cup \freenames{Q} \\
  \freenames{\dropn{x}} & := & \{ x \}
\end{eqnarray*}

The bound names of a process, $\boundnames{P}$, are those names occurring in $P$
that are not free. For example, in $x?(y).0$, the name $x$ is free, while $y$ is bound.

\begin{mathpar}
  \inferrule* [lab=monoidal-laws] {} { P|Q \equiv Q|P \and P|0 \equiv P \and P|(Q|R) \equiv (P|Q)|R }
\end{mathpar}

\begin{mathpar}
  \inferrule* [lab=alpha-equivalence] {} { (x)P \equiv (y)P\{y/x\} \and y \not\in \freenames{P} }
\end{mathpar}

\begin{definition}
Then two processes, $P,Q$, are alpha-equivalent if $P = Q\{\vec{y}/\vec{x}\}$ for
some $\vec{x} \in \boundnames{Q},\vec{y} \in \boundnames{P}$, where $Q\{\vec{y}/\vec{x}\}$
denotes the capture-avoiding substitution of $\vec{y}$ for $\vec{x}$ in $Q$.
\end{definition}

\begin{definition}
  The {\em structural congruence} \cite{SangiorgiWalker} , $\equiv$,
  between processes is the least congruence containing
  alpha-equivalence, satisfying the abelian monoid laws
  (associativity, commutativity and $\pzero$ as identity) for parallel
  composition $|$ and for summation $+$.
\end{definition}

\subsection{Name equivalence}

We take name equivalence, written $\nameeq$, to be the smallest
equivalence relation generated by the following rules.

\begin{mathpar}
\inferrule*[lab=Quote-drop]
{ }
{ \quotep{@{x}} \nameeq x }

\inferrule*[lab=Struct-equiv]
{ P \scong Q }
{ \quotep{P} \nameeq \quotep{Q} }
\end{mathpar}

The astute reader will have noticed that the mutual recursion of names
and processes imposes a mutual recursion on alpha-equivalence and
structural equivalence via name-equivalence. Fortunately, all of this
works out pleasantly and we may calculate in the natural way, free of
concern. The reader interested in the details is referred to the
appendix \ref{appendix:rho_details}.

\subsection{Substitution}

We use $\Proc$ for the set of processes, $\QProc$ for the set of
names, and $\id{\{}\vec{y} / \vec{x} \id{\}}$ to denote partial maps,
$s : \QProc \rightarrow \QProc$. A map, $s$ lifts, uniquely, to a map
on process terms, $\widehat{s} : \Proc \rightarrow \Proc$ by the
following equations.

\begin{mathpar}
  (0) \psubstp{Q}{P} := 0 \\
  (R \juxtap S) \psubstp{Q}{P}
  :=    
  (R)\psubstp{Q}{P} \juxtap (S) \psubstp{Q}{P} \\
  (x?(y).R) \psubstp{Q}{P}    
  :=    
  (x)\substp{Q}{P} (z)\concat( (R \psubstn{z}{y}) \psubstp{Q}{P} ) \\
  (\lift{x}{R}) \psubstp{Q}{P}  
  :=
  \lift{(x)\substp{Q}{P}}{ R \psubstp{Q}{P} } \\
%   (\dropn{x})  \psubstp{Q}{P}       
%   := 
%   \left\{ 
%     \begin{array}{ccc} 
%       \dropn{\quotep{Q}} & & x \nameeq \quotep{P} \\
%       \dropn{x} & & otherwise \\
%     \end{array}
%   \right. 
  (\dropn{x})  \psubstp{Q}{P}       
  := 
  \left\{ 
    \begin{array}{ccc} 
      Q & & x \nameeq \quotep{P} \\
      \dropn{x} & & otherwise \\
    \end{array}
  \right.
\end{mathpar}
 

where

\begin{eqnarray}
  (x)\id{\{} \lpquote Q \rpquote / \lpquote P \rpquote \id{\}}            = 
  \left\{ 
    \begin{array}{ccc}
      \lpquote Q \rpquote & & x \nameeq \lpquote P \rpquote \\
      x & & otherwise \\
    \end{array}
  \right. \nonumber
\end{eqnarray}

and $z$ is chosen distinct from $\quotep{P}$, $\quotep{Q}$, the free
names in $Q$, and all the names in $R$. Our $\alpha$-equivalence will
be built in the standard way from this substitution.

\begin{remark}\label{rem:no_self_referential_names}
  One consequence of these definitions is that $\forall P. \quotep{P}
  \not\in \freenames{P}$.
\end{remark}

\subsection{ Dynamic quote: an example }

Anticipating something of what's to come, consider applying the
substitution, $\widehat{\id{\{}u / z \id{\}}}$, to the following pair
of processes, $\lift{w}{y!(z)}$ and $w[ \lpquote y!(z) \rpquote ]$.

\begin{eqnarray}
	\lift{w}{y!(z)}\widehat{\id{\{}u / z \id{\}}}
		& = &
		\lift{w}{y!(u)} \nonumber\\
	w[ \lpquote y!(z) \rpquote ] \widehat{ \id{\{}u / z \id{\}} }
		& = &
		w[ \lpquote y!(z) \rpquote ] \nonumber
\end{eqnarray}

Because the body of the process between quotes is impervious to
substitution, we get radically different answers. In fact, by
examining the first process in an input context,
e.g. $x?(z).\lift{w}{y!(z)}$, we see that the process under the lift
operator may be shaped by prefixed inputs binding a name inside it. In
this sense, the lift operator will be seen as a way to dynamically
construct processes before reifying them as names.

Finally equipped with these standard features we can present the
dynamics of the calculus.

\subsubsection{Operational semantics} 

Finally, we introduce the computational dynamics. What marks these
algebras as distinct from other more traditionally studied algebraic
structures, e.g. vector spaces or polynomial rings, is the manner in
which dynamics is captured. In traditional structures, dynamics is typically
expressed through morphisms between such structures, as in linear maps
between vector spaces or morphisms between rings. In algebras
associated with the semantics of computation, the dynamics is
expressed as part of the algebraic structure itself, through a
reduction reduction relation typically denoted by $\red$. Below, we
give a recursive presentation of this relation for the calculus used
in the encoding.

$\red \subseteq \pi \times \pi$
$\red : \pi \to \mathcal{P}(\pi)$

\begin{mathpar}
  \inferrule* [lab=Comm] { \textsf{match}( x_{src}, x_{trgt} ) } { x_{trgt}?(y)P \; | \; x_{src}!\langle {Q} \rangle \red P\{\quotep{Q}/y}\} }
  \and \\
  \inferrule* [lab=Par] {{P} \red {P}'} {{{P} | {Q}} \red {{P}' | {Q}}}
  \and
  \inferrule* [lab=Equiv]{{{P} \scong {P}'} \andalso {{P}' \red {Q}'} \andalso {{Q}' \scong {Q}}}{{P} \red {Q}}
\end{mathpar}

\begin{eqnarray*}
  match_{\equiv} (\quotep{P},\quotep{Q}) & := & P \equiv Q \\
  match_{\dagger}(\quotep{P},\quotep{Q}) & := & \forall R. P|Q \red^{*} R => R \red^{*} 0 \\
  match_{K}(\quotep{P},\quotep{Q}) & := & K \mbox{ for some context } K
\end{eqnarray*}

$u?(x)P | u!\langle Q \rangle \red P\{\quotep{Q}/x\}$

%We write $\wred$ for $\red^*$, and $P\red$ if $\exists Q $ such that $ P \red Q$.
We write $P\red$ if $\exists Q $ such that $ P \red Q$ and $P\not\red$, otherwise.

\section{Replication}

As mentioned before, it is known that replication (and hence
recursion) can be implemented in a higher-order process algebra
\cite{SangiorgiWalker}. As our first example of calculation with the
machinery thus far presented we give the construction explicitly in
the {\rhoc}.

\begin{eqnarray}
	D_{x} & := & \prefix{x}{y}{(\binpar{\outputp{x}{y}}{@{y}})} \nonumber\\
	\bangp_{x}{P} & := & \binpar{{x}!\langle{\binpar{D_{x}}{P}}\rangle}{D_{x}} \nonumber
\end{eqnarray}

\begin{eqnarray}
	\bangp_{x}{P} & & \nonumber\\
	=
	& {x}!\langle{(\prefix{x}{y}{(\outputp{x}{y} | @{y})) | P}}\rangle 
	      | \prefix{x}{y}{(\outputp{x}{y} | @{y})} & \nonumber\\
	\red
	& (\outputp{x}{y} | @{y})\substn{\quotep{(\prefix{x}{y}{(@{y} | \outputp{x}{y})) | P}}}{y} & \nonumber\\
	=
	& \outputp{x}{\quotep{(\prefix{x}{y}{(\outputp{x}{y} | @{y})) | P}}}
	  | {(\prefix{x}{y}{(\outputp{x}{y} | @{y})) | P}} & \nonumber\\
	\red
	& \ldots & \nonumber\\
	\red^*
	& P | P | \ldots & \nonumber
\end{eqnarray}

Of course, this encoding, as an implementation, runs away, unfolding
$\bangp{P}$ eagerly. A lazier and more implementable replication
operator, restricted to input-guarded processes, may be obtained as follows.

\begin{eqnarray}
\bangp{\prefix{u}{v}{P}} 
	:= 
	\binpar{\lift{x}{\prefix{u}{v}{(\binpar{D(x)}{P})}}}{D(x)} \nonumber
\end{eqnarray}

\begin{remark}
  Note that the lazier definition still does not deal with summation
  or mixed summation (i.e. sums over input and output). The reader is
  invited to construct definitions of replication that deal with these
  features. 

  Further, the definitions are parameterized in a name, $x$. Can you,
  gentle reader, make a definition that eliminates this parameter and
  guarantees no accidental interaction between the replication
  machinery and the process being replicated -- i.e. no accidental
  sharing of names used by the process to get its work done and the
  name(s) used by the replication to effect copying. This latter
  revision of the definition of replication is crucial to obtaining
  the expected identity $!!P \sim !P$.
\end{remark}

\begin{remark}\label{rem:paradoxical_combinator}
  The reader familiar with the lambda calculus will have noticed the
  similarity between $D$ and the paradoxical combinator.

  [Ed. note: the existence of this seems to suggest we have to be more
  restrictive on the set of processes and names we admit if we are to
  support no-cloning.]
\end{remark}

\subsubsection{Bisimulation}

The computational dynamics gives rise to another kind of equivalence,
the equivalence of computational behavior. As previously mentioned
this is typically captured \emph{via} some form of bisimulation.

% The notion we use in this paper is weak barbed bisimulation
% \cite{milner91polyadicpi}.

The notion we use in this paper is derived from weak barbed
bisimulation \cite{milner91polyadicpi}. 

\begin{definition}
An \emph{observation relation}, $\downarrow_{\mathcal N}$, over a set
of names, $\mathcal N$, is the smallest relation satisfying the rules
below.

\infrule[Out-barb]{y \in {\mathcal N}, \; x \nameeq y}
		  {\outputp{x}{v} \downarrow_{\mathcal N} x}
\infrule[Par-barb]{\mbox{$P\downarrow_{\mathcal N} x$ or $Q\downarrow_{\mathcal N} x$}}
		  {\binpar{P}{Q} \downarrow_{\mathcal N} x}

We write $P \Downarrow_{\mathcal N} x$ if there is $Q$ such that 
$P \wred Q$ and $Q \downarrow_{\mathcal N} x$.
\end{definition}

\begin{definition}
%\label{def.bbisim}
An  ${\mathcal N}$-\emph{barbed bisimulation} over a set of names, ${\mathcal N}$, is a symmetric binary relation 
${\mathcal S}_{\mathcal N}$ between agents such that $P\rel{S}_{\mathcal N}Q$ implies:
\begin{enumerate}
\item If $P \red P'$ then $Q \wred Q'$ and $P'\rel{S}_{\mathcal N} Q'$.
\item If $P\downarrow_{\mathcal N} x$, then $Q\Downarrow_{\mathcal N} x$.
\end{enumerate}
$P$ is ${\mathcal N}$-barbed bisimilar to $Q$, written
$P \wbbisim_{\mathcal N} Q$, if $P \rel{S}_{\mathcal N} Q$ for some ${\mathcal N}$-barbed bisimulation ${\mathcal S}_{\mathcal N}$.
\end{definition}

$\mathcal{R} \subseteq \pi \times \pi$

$P \mathcal{R} Q => \forall P'. P \red P' \Rightarrow \exists Q'. Q \red Q', P' \mathcal{R} Q'$

$P \vdash x \Rightarrow Q \vdash x$

\begin{mathpar}
  \inferrule*[lab=Out-barb]{x \nameeq y}{{y}!\langle{Q}\rangle \vdash x}
  \and
  \inferrule*[lab=Par-barb]{\mbox{$P\vdash x$ or $Q\vdash x$}}{\binpar{P}{Q} \vdash x}
\end{mathpar}

\subsubsection{Contexts}

One of the principle advantages of computational calculi like the
$\pi$-calculus is a well-defined notion of context,
contextual-equivalence and a correlation between
contextual-equivalence and notions of bisimulation. The notion of
context allows the decomposition of a process into (sub-)process and
its syntactic environment, its context. Thus, a context may be
thought of as a process with a ``hole'' (written $\Box$) in it. The
application of a context $M$ to a process $P$, written $M[P]$, is
tantamount to filling the hole in $M$ with $P$. In this paper we do
not need the full weight of this theory, but do make use of the notion
of context in the proof the main theorem. 

\begin{mathpar}
  \inferrule* [lab=summation] {} {{M_{M},M_{N}} \bc \Box \;|\; x.M_{A} \;|\; M_{M}+M_{N}}
  \and
  \inferrule* [lab=agent] {} {{M_{A}} \bc (\vec{x})M_{P} \;| \; \clift{P_0,\ldots,M_{P},\ldots,P_N}}
  \and \\
  \inferrule* [lab=process] {} {{M_{P}} \bc M_{N} \;| \;P|M_{P} }
\end{mathpar} 

\begin{mathpar}
  \inferrule* [lab=sychronization] {} {M_{N} \bc \Box \;|\; x?M_{F} \;|\; x!M_{C}}
  \and
  \inferrule* [lab=abstraction] {} {{M_{F}} \bc (x)M_{P} }
  \and
  \inferrule* [lab=concretion] {} {{M_{C}} \bc \langle M_{P} \rangle }
  \and \\
  \inferrule* [lab=process] {} {{M_{P}} \bc M_{N} \;| \;P|M_{P} }
\end{mathpar}

\begin{definition}[contextual application] Given a context $M$, and
  process $P$, we define the \emph{contextual application}, $M[P] :=
  M\{P/\Box\}$. That is, the contextual application of M to P is the
  substitution of $P$ for $\Box$ in $M$.
\end{definition}

$\meaningof{-} : L \to \mathcal{P}(\pi)$

\begin{mathpar}
  \inferrule* [lab=collection] {} {\meaningof{true} = \pi, \and \meaningof{~E} = \pi \setminus \meaningof{E}, \and \meaningof{E_{1} \& E_{2}} = \meaningof{E_{1}} \cap \meaningof{E_{2}}}
\end{mathpar}

\begin{mathpar}
  \inferrule* [lab=structure] {} {\meaningof{0} = \{ P \in \pi | P \equiv 0 \}, \and \\ \meaningof{E_1 | E_2} = \{ P \in \pi | P \equiv P_{1} | P_{2}, P_{1} \in \meaningof{E_{1}}, P_{2} \in \meaningof{E_2}\} }
\end{mathpar}

\begin{mathpar}
 \inferrule* [lab=behavior] {} {\meaningof{\langle a?b \rangle E} = \{ P \in \pi | P \equiv Q | u?(y)P', \\ \and \\\\ \and \\ \;\;\; u \in \meaningof{a}, \forall z.P'\{z/y\} \in \meaningof{E\{z/b\}}\}, \and \\ \meaningof{a!E} = \{ P \in \pi | P \equiv Q | x!\langle P' \rangle, x \in \meaningof{a} P' \in \meaningof{E}\} }
\end{mathpar}

\begin{mathpar}
 \inferrule* [lab=nominal] {} {\meaningof{\quotep{E}} = \{ \quotep{P} \in \quotep{\pi} | P \in \meaningof{E} \}, \and \meaningof{\quotep{P}} = \{ \quotep{Q} \in \quotep{\pi} | P \equiv Q \} \and \\ \meaningof{@\quotep{E}} = \{ P \in \pi | P \equiv @x, x \in \meaningof{E} \}}
\end{mathpar}

\begin{eqnarray*}
  \\
  \meaningof{-} : TS \to ST
\end{eqnarray*}

\begin{eqnarray*}
  \\
  L : TS \to ST
\end{eqnarray*}

\begin{eqnarray*}
  \\
  P \models E \iff P \in \meaningof{E}
\end{eqnarray*}

\begin{eqnarray*}
  P \approx_{L} Q \iff \forall E \in L. P \models E \iff Q \models E
\end{eqnarray*}

\begin{eqnarray*}
  P \approx_{K} Q
\end{eqnarray*}

\begin{eqnarray*}
  P \approx Q
\end{eqnarray*}

$\approx_{K} = \approx = \approx_{L}$

\subsubsection{Contextual duality}

Note that contexts extend the quotation operation to a family of
operations from processes to names. Given a context, $M$, we can
define a \emph{nominal context}, $\quotep{M}$ by $\quotep{M}[P] :=
\quotep{M[P]}$. To foreshadow what is to come we observe that these
operations enjoy a duality with processes very much like the duality
between vectors and maps from vectors to scalars.

Further, because the calculus is essentially higher-order, we have a
correspondence between contexts and processes. More specifically,
given a name $x$ and a context $M$ we can construct $M^{*}_{x}$ such
that 

\begin{mathpar}
  M^{*}_{x} | \lift{x}{P} \red M[P]
\end{mathpar}

namely,

\begin{mathpar}
  M^{*}_{x} := x?(u).M[\dropn{u}]
\end{mathpar}

The dependence of $M^{*}_{x}$ on a name makes it an abstraction, 

\begin{mathpar}
  M^{*} := (x)x?(u).M[\dropn{u}]
\end{mathpar}

\subsection{Additional notation}

It will sometimes be convenient to denote the process a name
quotes. We already have the notation $x = \quotep{P}$, but it will be
convenient to introduce an alternate notation, $\procn{x}$, when we
want to emphasize the connection to the use of the name. Note that, by
virtue of name equivalence, $\quotep{\procn{x}} \nameeq x$; so, the
notation is consistent with previous definitions.

Further, because names have structure it is possible to effect
substitutions on the basis of that structure. This means we need to
upgrade our notation for substitutions, which we accomplish by
adapting comprehension notation. Thus,

\begin{mathpar}
  P\{ y / x : x \in S \}
\end{mathpar}

is interpreted to mean the process derived from P by replacing (in a
capture-avoiding manner) each occurrence of $x$ in $S$ by $y$. For example,

\begin{mathpar}
  P\{ \quotep{\procn{x}|\procn{x}} / x : x \in \freenames{P} \}
\end{mathpar}

will replace each (occurrence) of a free name $x$ in $P$ by
$\quotep{\procn{x}|\procn{x}}$.

Also, we will avail ourselves of the notation $x^{L}$ and $x^{R}$ to
denote injections of a name into disjoint copies of the name
space. There are numerous ways to accomplish this. One example can be
found in \cite{MeredithR05}. This notation overloads to vectors of
names: $\vec{x}^{\pi} := (x_{i}^{\pi} \; : \; 0 \leq i < |\vec{x}| )$ where $\pi \in \{L,R\}$.

We also use $P^{\Box} := P|\Box$.

In \cite{MeredithR05} an interpretation of the new operator is
given. It turns out that there are several possible interpretations
all enjoying the requisite algebraic properties of the operator (see
\cite{milner91polyadicpi}). We will therefore make liberal use of
$(\nu\; \vec{x})P$.

% subsection the_syntax_and_semantics_of_the_notation_system (end)   

\input{qm2pi.qmops} 

\input{qm2pi.sterngerlach} 

\input{qm2pi.metric} 

% section concurrent_process_calculi (end)

%\input{qm2pi.proofsketch}

% section proof sketch (end)

%\input{qm2pi.slviaknots} 

% section spatial logic via knots (end)

\input{qm2pi.conclusion}

% section conclusion (end)

%\input{qm2pi.dtcodes} 

% section wiring algorithm (end)

\input{qm2pi.ack} 

% section acknowledgments (end)

\newpage


\bibliographystyle{plain}   
\bibliography{../../biblios/main.bib}

\input{qm2pi.rhodetails}

\end{document}

 

% section acknowledgments (end)

\newpage


\bibliographystyle{plain}   
\bibliography{../../biblios/main.bib}

\documentclass[12pt]{llncs}
%\documentclass{jktr}

\usepackage[pdftex]{hyperref}                   
\usepackage {listings}
\usepackage {mathpartir}
\usepackage{bcprules}
%\usepackage{listings}
                       
\usepackage{graphicx} 
%\usepackage[margins=2.5cm,nohead,nofoot]{geometry}
%\usepackage{geometry}
\usepackage{amsfonts}
\usepackage{amstext}
\usepackage{latexsym}
\usepackage{amssymb}
\usepackage{color}


%\include{myPreamble}
\include{qm2pi.local} 

%\ifpdf
%\usepackage[pdftex]{graphicx}
%\else
%\usepackage{graphicx}
%\fi

 % \ifpdf
%  \usepackage{pdfsync}
%  \if


%\title{Brief Article}
%\author{David F. Snyder}
%\author{L.G. Meredith}

%\address{Dept. of Math., Texas State University--San Marcos, San Marcos, TX 78666}
       
\pagestyle{empty}


\begin{document}

\lstset{language=[Objective]Caml,frame=shadowbox}

\input{qm2pi.front}

% section front matter (end)

\input{qm2pi.intro} 
 
% section introduction (end)

% \input{qm2pi.knotations} 

% section notation (end)

\input{qm2pi.process.calculi} 

% section concurrent_process_calculi_and_spatial_logics_ (end)
    
%\input{qm2pi.knots2pi} 

%\input{qm2pi.trefoil} 

%\input{qm2pi.mainthm} 

% subsection basic_interpretation (end)

%\input{qm2pi.rho.presentation} 
\subsection{The syntax and semantics of the notation system}\label{sub:the_syntax_and_semantics_of_the_notation_system} % (fold)

We now summarize a technical presentation of the calculus that
embodies our theory of dynamics. The typical presentation of such a
calculus follows the style of giving generators and relations on
them. The grammar, below, describing term constructors, freely
generates the set of processes, $\Proc$. This set is then quotiented
by a relation known as structural congruence and it is over this set
that the notion of dynamics is expressed. This presentation is
essentially that of \cite{MeredithR05} with the addition of
polyadicity and summation. For readability we have relegated some of
the technical subtleties to an appendix.

\subsubsection{Process grammar}\label{subsub:process_grammar}

\begin{mathpar}
  \inferrule* [lab=synchronization] {} {{M} \bc \pzero \;|\; x?F \;|\; x!C }
  \and
  \inferrule* [lab=abstraction] {} {{F} \bc (x)P}
  \and
  \inferrule* [lab=concretion] {} {{C} \bc \langle Q \rangle}
  \and
  \inferrule* [lab=process] {} {{P,Q} \bc M \;| \;P|Q \;|\; @{x}}
  \and
  \inferrule* [lab=name] {} {{x} \bc \quotep{P}}
\end{mathpar} 

Note that $\vec{x}$ (resp. $\vec{P}$) denotes a vector of names
(resp. processes) of length $|\vec{x}|$ (resp. $|\vec{P}|$). We adopt
the following useful abbreviations.

\begin{mathpar}
   x?(\vec{y}).P := x.(\vec{y})P \and  x\clift{\vec{P}} := x.\clift{\vec{P}}
   \and x!(y) := \lift{x}{\dropn{y}}
   \and \Pi_{i=0}^{n-1}P_i := P_0 | \ldots | P_{n-1}
\end{mathpar}

\subsubsection{Structural congruence}

\paragraph{Free and bound names and alpha-equivalence.} At the
core of structural equivalence is alpha-equivalence which identifies
process that are the same up to a change of variable. Formally, we
recognize the distinction between free and bound names. The free names
of a process, $\freenames{P}$, may be calculated recursively as
follows:

\begin{mathpar}
\freenames{\pzero} := \emptyset
  \and \\
  \freenames{x?(y).P} := \{ x \} \cup (\freenames{P} \setminus \{ y \})
  \and 
  \freenames{x!\langle P \rangle} := \{ x \} \cup \{ P \} 
  \and \\
  \freenames{P|Q} := \freenames{P} \cup \freenames{Q}
  \and \\
  \freenames{@{x}} := \{ x \}
\end{mathpar}

$\pi$
$\quotep{\pi}$

$\freenames{-} : \pi \to \mathcal{P}(\quotep{\pi})$

\begin{eqnarray*}
  \freenames{\pzero} & := & \emptyset \\
  \freenames{x?(y).P} & := & \{ x \} \cup (\freenames{P} \setminus \{ y \}) \\
  \freenames{x!\langle P \rangle} & := & \{ x \} \cup \{ P \} \\
  \freenames{P|Q} & := & \freenames{P} \cup \freenames{Q} \\
  \freenames{\dropn{x}} & := & \{ x \}
\end{eqnarray*}

The bound names of a process, $\boundnames{P}$, are those names occurring in $P$
that are not free. For example, in $x?(y).0$, the name $x$ is free, while $y$ is bound.

\begin{mathpar}
  \inferrule* [lab=monoidal-laws] {} { P|Q \equiv Q|P \and P|0 \equiv P \and P|(Q|R) \equiv (P|Q)|R }
\end{mathpar}

\begin{mathpar}
  \inferrule* [lab=alpha-equivalence] {} { (x)P \equiv (y)P\{y/x\} \and y \not\in \freenames{P} }
\end{mathpar}

\begin{definition}
Then two processes, $P,Q$, are alpha-equivalent if $P = Q\{\vec{y}/\vec{x}\}$ for
some $\vec{x} \in \boundnames{Q},\vec{y} \in \boundnames{P}$, where $Q\{\vec{y}/\vec{x}\}$
denotes the capture-avoiding substitution of $\vec{y}$ for $\vec{x}$ in $Q$.
\end{definition}

\begin{definition}
  The {\em structural congruence} \cite{SangiorgiWalker} , $\equiv$,
  between processes is the least congruence containing
  alpha-equivalence, satisfying the abelian monoid laws
  (associativity, commutativity and $\pzero$ as identity) for parallel
  composition $|$ and for summation $+$.
\end{definition}

\subsection{Name equivalence}

We take name equivalence, written $\nameeq$, to be the smallest
equivalence relation generated by the following rules.

\begin{mathpar}
\inferrule*[lab=Quote-drop]
{ }
{ \quotep{@{x}} \nameeq x }

\inferrule*[lab=Struct-equiv]
{ P \scong Q }
{ \quotep{P} \nameeq \quotep{Q} }
\end{mathpar}

The astute reader will have noticed that the mutual recursion of names
and processes imposes a mutual recursion on alpha-equivalence and
structural equivalence via name-equivalence. Fortunately, all of this
works out pleasantly and we may calculate in the natural way, free of
concern. The reader interested in the details is referred to the
appendix \ref{appendix:rho_details}.

\subsection{Substitution}

We use $\Proc$ for the set of processes, $\QProc$ for the set of
names, and $\id{\{}\vec{y} / \vec{x} \id{\}}$ to denote partial maps,
$s : \QProc \rightarrow \QProc$. A map, $s$ lifts, uniquely, to a map
on process terms, $\widehat{s} : \Proc \rightarrow \Proc$ by the
following equations.

\begin{mathpar}
  (0) \psubstp{Q}{P} := 0 \\
  (R \juxtap S) \psubstp{Q}{P}
  :=    
  (R)\psubstp{Q}{P} \juxtap (S) \psubstp{Q}{P} \\
  (x?(y).R) \psubstp{Q}{P}    
  :=    
  (x)\substp{Q}{P} (z)\concat( (R \psubstn{z}{y}) \psubstp{Q}{P} ) \\
  (\lift{x}{R}) \psubstp{Q}{P}  
  :=
  \lift{(x)\substp{Q}{P}}{ R \psubstp{Q}{P} } \\
%   (\dropn{x})  \psubstp{Q}{P}       
%   := 
%   \left\{ 
%     \begin{array}{ccc} 
%       \dropn{\quotep{Q}} & & x \nameeq \quotep{P} \\
%       \dropn{x} & & otherwise \\
%     \end{array}
%   \right. 
  (\dropn{x})  \psubstp{Q}{P}       
  := 
  \left\{ 
    \begin{array}{ccc} 
      Q & & x \nameeq \quotep{P} \\
      \dropn{x} & & otherwise \\
    \end{array}
  \right.
\end{mathpar}
 

where

\begin{eqnarray}
  (x)\id{\{} \lpquote Q \rpquote / \lpquote P \rpquote \id{\}}            = 
  \left\{ 
    \begin{array}{ccc}
      \lpquote Q \rpquote & & x \nameeq \lpquote P \rpquote \\
      x & & otherwise \\
    \end{array}
  \right. \nonumber
\end{eqnarray}

and $z$ is chosen distinct from $\quotep{P}$, $\quotep{Q}$, the free
names in $Q$, and all the names in $R$. Our $\alpha$-equivalence will
be built in the standard way from this substitution.

\begin{remark}\label{rem:no_self_referential_names}
  One consequence of these definitions is that $\forall P. \quotep{P}
  \not\in \freenames{P}$.
\end{remark}

\subsection{ Dynamic quote: an example }

Anticipating something of what's to come, consider applying the
substitution, $\widehat{\id{\{}u / z \id{\}}}$, to the following pair
of processes, $\lift{w}{y!(z)}$ and $w[ \lpquote y!(z) \rpquote ]$.

\begin{eqnarray}
	\lift{w}{y!(z)}\widehat{\id{\{}u / z \id{\}}}
		& = &
		\lift{w}{y!(u)} \nonumber\\
	w[ \lpquote y!(z) \rpquote ] \widehat{ \id{\{}u / z \id{\}} }
		& = &
		w[ \lpquote y!(z) \rpquote ] \nonumber
\end{eqnarray}

Because the body of the process between quotes is impervious to
substitution, we get radically different answers. In fact, by
examining the first process in an input context,
e.g. $x?(z).\lift{w}{y!(z)}$, we see that the process under the lift
operator may be shaped by prefixed inputs binding a name inside it. In
this sense, the lift operator will be seen as a way to dynamically
construct processes before reifying them as names.

Finally equipped with these standard features we can present the
dynamics of the calculus.

\subsubsection{Operational semantics} 

Finally, we introduce the computational dynamics. What marks these
algebras as distinct from other more traditionally studied algebraic
structures, e.g. vector spaces or polynomial rings, is the manner in
which dynamics is captured. In traditional structures, dynamics is typically
expressed through morphisms between such structures, as in linear maps
between vector spaces or morphisms between rings. In algebras
associated with the semantics of computation, the dynamics is
expressed as part of the algebraic structure itself, through a
reduction reduction relation typically denoted by $\red$. Below, we
give a recursive presentation of this relation for the calculus used
in the encoding.

$\red \subseteq \pi \times \pi$
$\red : \pi \to \mathcal{P}(\pi)$

\begin{mathpar}
  \inferrule* [lab=Comm] { \textsf{match}( x_{src}, x_{trgt} ) } { x_{trgt}?(y)P \; | \; x_{src}!\langle {Q} \rangle \red P\{\quotep{Q}/y}\} }
  \and \\
  \inferrule* [lab=Par] {{P} \red {P}'} {{{P} | {Q}} \red {{P}' | {Q}}}
  \and
  \inferrule* [lab=Equiv]{{{P} \scong {P}'} \andalso {{P}' \red {Q}'} \andalso {{Q}' \scong {Q}}}{{P} \red {Q}}
\end{mathpar}

\begin{eqnarray*}
  match_{\equiv} (\quotep{P},\quotep{Q}) & := & P \equiv Q \\
  match_{\dagger}(\quotep{P},\quotep{Q}) & := & \forall R. P|Q \red^{*} R => R \red^{*} 0 \\
  match_{K}(\quotep{P},\quotep{Q}) & := & K \mbox{ for some context } K
\end{eqnarray*}

$u?(x)P | u!\langle Q \rangle \red P\{\quotep{Q}/x\}$

%We write $\wred$ for $\red^*$, and $P\red$ if $\exists Q $ such that $ P \red Q$.
We write $P\red$ if $\exists Q $ such that $ P \red Q$ and $P\not\red$, otherwise.

\section{Replication}

As mentioned before, it is known that replication (and hence
recursion) can be implemented in a higher-order process algebra
\cite{SangiorgiWalker}. As our first example of calculation with the
machinery thus far presented we give the construction explicitly in
the {\rhoc}.

\begin{eqnarray}
	D_{x} & := & \prefix{x}{y}{(\binpar{\outputp{x}{y}}{@{y}})} \nonumber\\
	\bangp_{x}{P} & := & \binpar{{x}!\langle{\binpar{D_{x}}{P}}\rangle}{D_{x}} \nonumber
\end{eqnarray}

\begin{eqnarray}
	\bangp_{x}{P} & & \nonumber\\
	=
	& {x}!\langle{(\prefix{x}{y}{(\outputp{x}{y} | @{y})) | P}}\rangle 
	      | \prefix{x}{y}{(\outputp{x}{y} | @{y})} & \nonumber\\
	\red
	& (\outputp{x}{y} | @{y})\substn{\quotep{(\prefix{x}{y}{(@{y} | \outputp{x}{y})) | P}}}{y} & \nonumber\\
	=
	& \outputp{x}{\quotep{(\prefix{x}{y}{(\outputp{x}{y} | @{y})) | P}}}
	  | {(\prefix{x}{y}{(\outputp{x}{y} | @{y})) | P}} & \nonumber\\
	\red
	& \ldots & \nonumber\\
	\red^*
	& P | P | \ldots & \nonumber
\end{eqnarray}

Of course, this encoding, as an implementation, runs away, unfolding
$\bangp{P}$ eagerly. A lazier and more implementable replication
operator, restricted to input-guarded processes, may be obtained as follows.

\begin{eqnarray}
\bangp{\prefix{u}{v}{P}} 
	:= 
	\binpar{\lift{x}{\prefix{u}{v}{(\binpar{D(x)}{P})}}}{D(x)} \nonumber
\end{eqnarray}

\begin{remark}
  Note that the lazier definition still does not deal with summation
  or mixed summation (i.e. sums over input and output). The reader is
  invited to construct definitions of replication that deal with these
  features. 

  Further, the definitions are parameterized in a name, $x$. Can you,
  gentle reader, make a definition that eliminates this parameter and
  guarantees no accidental interaction between the replication
  machinery and the process being replicated -- i.e. no accidental
  sharing of names used by the process to get its work done and the
  name(s) used by the replication to effect copying. This latter
  revision of the definition of replication is crucial to obtaining
  the expected identity $!!P \sim !P$.
\end{remark}

\begin{remark}\label{rem:paradoxical_combinator}
  The reader familiar with the lambda calculus will have noticed the
  similarity between $D$ and the paradoxical combinator.

  [Ed. note: the existence of this seems to suggest we have to be more
  restrictive on the set of processes and names we admit if we are to
  support no-cloning.]
\end{remark}

\subsubsection{Bisimulation}

The computational dynamics gives rise to another kind of equivalence,
the equivalence of computational behavior. As previously mentioned
this is typically captured \emph{via} some form of bisimulation.

% The notion we use in this paper is weak barbed bisimulation
% \cite{milner91polyadicpi}.

The notion we use in this paper is derived from weak barbed
bisimulation \cite{milner91polyadicpi}. 

\begin{definition}
An \emph{observation relation}, $\downarrow_{\mathcal N}$, over a set
of names, $\mathcal N$, is the smallest relation satisfying the rules
below.

\infrule[Out-barb]{y \in {\mathcal N}, \; x \nameeq y}
		  {\outputp{x}{v} \downarrow_{\mathcal N} x}
\infrule[Par-barb]{\mbox{$P\downarrow_{\mathcal N} x$ or $Q\downarrow_{\mathcal N} x$}}
		  {\binpar{P}{Q} \downarrow_{\mathcal N} x}

We write $P \Downarrow_{\mathcal N} x$ if there is $Q$ such that 
$P \wred Q$ and $Q \downarrow_{\mathcal N} x$.
\end{definition}

\begin{definition}
%\label{def.bbisim}
An  ${\mathcal N}$-\emph{barbed bisimulation} over a set of names, ${\mathcal N}$, is a symmetric binary relation 
${\mathcal S}_{\mathcal N}$ between agents such that $P\rel{S}_{\mathcal N}Q$ implies:
\begin{enumerate}
\item If $P \red P'$ then $Q \wred Q'$ and $P'\rel{S}_{\mathcal N} Q'$.
\item If $P\downarrow_{\mathcal N} x$, then $Q\Downarrow_{\mathcal N} x$.
\end{enumerate}
$P$ is ${\mathcal N}$-barbed bisimilar to $Q$, written
$P \wbbisim_{\mathcal N} Q$, if $P \rel{S}_{\mathcal N} Q$ for some ${\mathcal N}$-barbed bisimulation ${\mathcal S}_{\mathcal N}$.
\end{definition}

$\mathcal{R} \subseteq \pi \times \pi$

$P \mathcal{R} Q => \forall P'. P \red P' \Rightarrow \exists Q'. Q \red Q', P' \mathcal{R} Q'$

$P \vdash x \Rightarrow Q \vdash x$

\begin{mathpar}
  \inferrule*[lab=Out-barb]{x \nameeq y}{{y}!\langle{Q}\rangle \vdash x}
  \and
  \inferrule*[lab=Par-barb]{\mbox{$P\vdash x$ or $Q\vdash x$}}{\binpar{P}{Q} \vdash x}
\end{mathpar}

\subsubsection{Contexts}

One of the principle advantages of computational calculi like the
$\pi$-calculus is a well-defined notion of context,
contextual-equivalence and a correlation between
contextual-equivalence and notions of bisimulation. The notion of
context allows the decomposition of a process into (sub-)process and
its syntactic environment, its context. Thus, a context may be
thought of as a process with a ``hole'' (written $\Box$) in it. The
application of a context $M$ to a process $P$, written $M[P]$, is
tantamount to filling the hole in $M$ with $P$. In this paper we do
not need the full weight of this theory, but do make use of the notion
of context in the proof the main theorem. 

\begin{mathpar}
  \inferrule* [lab=summation] {} {{M_{M},M_{N}} \bc \Box \;|\; x.M_{A} \;|\; M_{M}+M_{N}}
  \and
  \inferrule* [lab=agent] {} {{M_{A}} \bc (\vec{x})M_{P} \;| \; \clift{P_0,\ldots,M_{P},\ldots,P_N}}
  \and \\
  \inferrule* [lab=process] {} {{M_{P}} \bc M_{N} \;| \;P|M_{P} }
\end{mathpar} 

\begin{mathpar}
  \inferrule* [lab=sychronization] {} {M_{N} \bc \Box \;|\; x?M_{F} \;|\; x!M_{C}}
  \and
  \inferrule* [lab=abstraction] {} {{M_{F}} \bc (x)M_{P} }
  \and
  \inferrule* [lab=concretion] {} {{M_{C}} \bc \langle M_{P} \rangle }
  \and \\
  \inferrule* [lab=process] {} {{M_{P}} \bc M_{N} \;| \;P|M_{P} }
\end{mathpar}

\begin{definition}[contextual application] Given a context $M$, and
  process $P$, we define the \emph{contextual application}, $M[P] :=
  M\{P/\Box\}$. That is, the contextual application of M to P is the
  substitution of $P$ for $\Box$ in $M$.
\end{definition}

$\meaningof{-} : L \to \mathcal{P}(\pi)$

\begin{mathpar}
  \inferrule* [lab=collection] {} {\meaningof{true} = \pi, \and \meaningof{~E} = \pi \setminus \meaningof{E}, \and \meaningof{E_{1} \& E_{2}} = \meaningof{E_{1}} \cap \meaningof{E_{2}}}
\end{mathpar}

\begin{mathpar}
  \inferrule* [lab=structure] {} {\meaningof{0} = \{ P \in \pi | P \equiv 0 \}, \and \\ \meaningof{E_1 | E_2} = \{ P \in \pi | P \equiv P_{1} | P_{2}, P_{1} \in \meaningof{E_{1}}, P_{2} \in \meaningof{E_2}\} }
\end{mathpar}

\begin{mathpar}
 \inferrule* [lab=behavior] {} {\meaningof{\langle a?b \rangle E} = \{ P \in \pi | P \equiv Q | u?(y)P', \\ \and \\\\ \and \\ \;\;\; u \in \meaningof{a}, \forall z.P'\{z/y\} \in \meaningof{E\{z/b\}}\}, \and \\ \meaningof{a!E} = \{ P \in \pi | P \equiv Q | x!\langle P' \rangle, x \in \meaningof{a} P' \in \meaningof{E}\} }
\end{mathpar}

\begin{mathpar}
 \inferrule* [lab=nominal] {} {\meaningof{\quotep{E}} = \{ \quotep{P} \in \quotep{\pi} | P \in \meaningof{E} \}, \and \meaningof{\quotep{P}} = \{ \quotep{Q} \in \quotep{\pi} | P \equiv Q \} \and \\ \meaningof{@\quotep{E}} = \{ P \in \pi | P \equiv @x, x \in \meaningof{E} \}}
\end{mathpar}

\begin{eqnarray*}
  \\
  \meaningof{-} : TS \to ST
\end{eqnarray*}

\begin{eqnarray*}
  \\
  L : TS \to ST
\end{eqnarray*}

\begin{eqnarray*}
  \\
  P \models E \iff P \in \meaningof{E}
\end{eqnarray*}

\begin{eqnarray*}
  P \approx_{L} Q \iff \forall E \in L. P \models E \iff Q \models E
\end{eqnarray*}

\begin{eqnarray*}
  P \approx_{K} Q
\end{eqnarray*}

\begin{eqnarray*}
  P \approx Q
\end{eqnarray*}

$\approx_{K} = \approx = \approx_{L}$

\subsubsection{Contextual duality}

Note that contexts extend the quotation operation to a family of
operations from processes to names. Given a context, $M$, we can
define a \emph{nominal context}, $\quotep{M}$ by $\quotep{M}[P] :=
\quotep{M[P]}$. To foreshadow what is to come we observe that these
operations enjoy a duality with processes very much like the duality
between vectors and maps from vectors to scalars.

Further, because the calculus is essentially higher-order, we have a
correspondence between contexts and processes. More specifically,
given a name $x$ and a context $M$ we can construct $M^{*}_{x}$ such
that 

\begin{mathpar}
  M^{*}_{x} | \lift{x}{P} \red M[P]
\end{mathpar}

namely,

\begin{mathpar}
  M^{*}_{x} := x?(u).M[\dropn{u}]
\end{mathpar}

The dependence of $M^{*}_{x}$ on a name makes it an abstraction, 

\begin{mathpar}
  M^{*} := (x)x?(u).M[\dropn{u}]
\end{mathpar}

\subsection{Additional notation}

It will sometimes be convenient to denote the process a name
quotes. We already have the notation $x = \quotep{P}$, but it will be
convenient to introduce an alternate notation, $\procn{x}$, when we
want to emphasize the connection to the use of the name. Note that, by
virtue of name equivalence, $\quotep{\procn{x}} \nameeq x$; so, the
notation is consistent with previous definitions.

Further, because names have structure it is possible to effect
substitutions on the basis of that structure. This means we need to
upgrade our notation for substitutions, which we accomplish by
adapting comprehension notation. Thus,

\begin{mathpar}
  P\{ y / x : x \in S \}
\end{mathpar}

is interpreted to mean the process derived from P by replacing (in a
capture-avoiding manner) each occurrence of $x$ in $S$ by $y$. For example,

\begin{mathpar}
  P\{ \quotep{\procn{x}|\procn{x}} / x : x \in \freenames{P} \}
\end{mathpar}

will replace each (occurrence) of a free name $x$ in $P$ by
$\quotep{\procn{x}|\procn{x}}$.

Also, we will avail ourselves of the notation $x^{L}$ and $x^{R}$ to
denote injections of a name into disjoint copies of the name
space. There are numerous ways to accomplish this. One example can be
found in \cite{MeredithR05}. This notation overloads to vectors of
names: $\vec{x}^{\pi} := (x_{i}^{\pi} \; : \; 0 \leq i < |\vec{x}| )$ where $\pi \in \{L,R\}$.

We also use $P^{\Box} := P|\Box$.

In \cite{MeredithR05} an interpretation of the new operator is
given. It turns out that there are several possible interpretations
all enjoying the requisite algebraic properties of the operator (see
\cite{milner91polyadicpi}). We will therefore make liberal use of
$(\nu\; \vec{x})P$.

% subsection the_syntax_and_semantics_of_the_notation_system (end)   

\input{qm2pi.qmops} 

\input{qm2pi.sterngerlach} 

\input{qm2pi.metric} 

% section concurrent_process_calculi (end)

%\input{qm2pi.proofsketch}

% section proof sketch (end)

%\input{qm2pi.slviaknots} 

% section spatial logic via knots (end)

\input{qm2pi.conclusion}

% section conclusion (end)

%\input{qm2pi.dtcodes} 

% section wiring algorithm (end)

\input{qm2pi.ack} 

% section acknowledgments (end)

\newpage


\bibliographystyle{plain}   
\bibliography{../../biblios/main.bib}

\input{qm2pi.rhodetails}

\end{document}



\end{document}

 

% section concurrent_process_calculi (end)

%\documentclass[12pt]{llncs}
%\documentclass{jktr}

\usepackage[pdftex]{hyperref}                   
\usepackage {listings}
\usepackage {mathpartir}
\usepackage{bcprules}
%\usepackage{listings}
                       
\usepackage{graphicx} 
%\usepackage[margins=2.5cm,nohead,nofoot]{geometry}
%\usepackage{geometry}
\usepackage{amsfonts}
\usepackage{amstext}
\usepackage{latexsym}
\usepackage{amssymb}
\usepackage{color}


%\include{myPreamble}
\documentclass[12pt]{llncs}
%\documentclass{jktr}

\usepackage[pdftex]{hyperref}                   
\usepackage {listings}
\usepackage {mathpartir}
\usepackage{bcprules}
%\usepackage{listings}
                       
\usepackage{graphicx} 
%\usepackage[margins=2.5cm,nohead,nofoot]{geometry}
%\usepackage{geometry}
\usepackage{amsfonts}
\usepackage{amstext}
\usepackage{latexsym}
\usepackage{amssymb}
\usepackage{color}


%\include{myPreamble}
\include{qm2pi.local} 

%\ifpdf
%\usepackage[pdftex]{graphicx}
%\else
%\usepackage{graphicx}
%\fi

 % \ifpdf
%  \usepackage{pdfsync}
%  \if


%\title{Brief Article}
%\author{David F. Snyder}
%\author{L.G. Meredith}

%\address{Dept. of Math., Texas State University--San Marcos, San Marcos, TX 78666}
       
\pagestyle{empty}


\begin{document}

\lstset{language=[Objective]Caml,frame=shadowbox}

\input{qm2pi.front}

% section front matter (end)

\input{qm2pi.intro} 
 
% section introduction (end)

% \input{qm2pi.knotations} 

% section notation (end)

\input{qm2pi.process.calculi} 

% section concurrent_process_calculi_and_spatial_logics_ (end)
    
%\input{qm2pi.knots2pi} 

%\input{qm2pi.trefoil} 

%\input{qm2pi.mainthm} 

% subsection basic_interpretation (end)

%\input{qm2pi.rho.presentation} 
\subsection{The syntax and semantics of the notation system}\label{sub:the_syntax_and_semantics_of_the_notation_system} % (fold)

We now summarize a technical presentation of the calculus that
embodies our theory of dynamics. The typical presentation of such a
calculus follows the style of giving generators and relations on
them. The grammar, below, describing term constructors, freely
generates the set of processes, $\Proc$. This set is then quotiented
by a relation known as structural congruence and it is over this set
that the notion of dynamics is expressed. This presentation is
essentially that of \cite{MeredithR05} with the addition of
polyadicity and summation. For readability we have relegated some of
the technical subtleties to an appendix.

\subsubsection{Process grammar}\label{subsub:process_grammar}

\begin{mathpar}
  \inferrule* [lab=synchronization] {} {{M} \bc \pzero \;|\; x?F \;|\; x!C }
  \and
  \inferrule* [lab=abstraction] {} {{F} \bc (x)P}
  \and
  \inferrule* [lab=concretion] {} {{C} \bc \langle Q \rangle}
  \and
  \inferrule* [lab=process] {} {{P,Q} \bc M \;| \;P|Q \;|\; @{x}}
  \and
  \inferrule* [lab=name] {} {{x} \bc \quotep{P}}
\end{mathpar} 

Note that $\vec{x}$ (resp. $\vec{P}$) denotes a vector of names
(resp. processes) of length $|\vec{x}|$ (resp. $|\vec{P}|$). We adopt
the following useful abbreviations.

\begin{mathpar}
   x?(\vec{y}).P := x.(\vec{y})P \and  x\clift{\vec{P}} := x.\clift{\vec{P}}
   \and x!(y) := \lift{x}{\dropn{y}}
   \and \Pi_{i=0}^{n-1}P_i := P_0 | \ldots | P_{n-1}
\end{mathpar}

\subsubsection{Structural congruence}

\paragraph{Free and bound names and alpha-equivalence.} At the
core of structural equivalence is alpha-equivalence which identifies
process that are the same up to a change of variable. Formally, we
recognize the distinction between free and bound names. The free names
of a process, $\freenames{P}$, may be calculated recursively as
follows:

\begin{mathpar}
\freenames{\pzero} := \emptyset
  \and \\
  \freenames{x?(y).P} := \{ x \} \cup (\freenames{P} \setminus \{ y \})
  \and 
  \freenames{x!\langle P \rangle} := \{ x \} \cup \{ P \} 
  \and \\
  \freenames{P|Q} := \freenames{P} \cup \freenames{Q}
  \and \\
  \freenames{@{x}} := \{ x \}
\end{mathpar}

$\pi$
$\quotep{\pi}$

$\freenames{-} : \pi \to \mathcal{P}(\quotep{\pi})$

\begin{eqnarray*}
  \freenames{\pzero} & := & \emptyset \\
  \freenames{x?(y).P} & := & \{ x \} \cup (\freenames{P} \setminus \{ y \}) \\
  \freenames{x!\langle P \rangle} & := & \{ x \} \cup \{ P \} \\
  \freenames{P|Q} & := & \freenames{P} \cup \freenames{Q} \\
  \freenames{\dropn{x}} & := & \{ x \}
\end{eqnarray*}

The bound names of a process, $\boundnames{P}$, are those names occurring in $P$
that are not free. For example, in $x?(y).0$, the name $x$ is free, while $y$ is bound.

\begin{mathpar}
  \inferrule* [lab=monoidal-laws] {} { P|Q \equiv Q|P \and P|0 \equiv P \and P|(Q|R) \equiv (P|Q)|R }
\end{mathpar}

\begin{mathpar}
  \inferrule* [lab=alpha-equivalence] {} { (x)P \equiv (y)P\{y/x\} \and y \not\in \freenames{P} }
\end{mathpar}

\begin{definition}
Then two processes, $P,Q$, are alpha-equivalent if $P = Q\{\vec{y}/\vec{x}\}$ for
some $\vec{x} \in \boundnames{Q},\vec{y} \in \boundnames{P}$, where $Q\{\vec{y}/\vec{x}\}$
denotes the capture-avoiding substitution of $\vec{y}$ for $\vec{x}$ in $Q$.
\end{definition}

\begin{definition}
  The {\em structural congruence} \cite{SangiorgiWalker} , $\equiv$,
  between processes is the least congruence containing
  alpha-equivalence, satisfying the abelian monoid laws
  (associativity, commutativity and $\pzero$ as identity) for parallel
  composition $|$ and for summation $+$.
\end{definition}

\subsection{Name equivalence}

We take name equivalence, written $\nameeq$, to be the smallest
equivalence relation generated by the following rules.

\begin{mathpar}
\inferrule*[lab=Quote-drop]
{ }
{ \quotep{@{x}} \nameeq x }

\inferrule*[lab=Struct-equiv]
{ P \scong Q }
{ \quotep{P} \nameeq \quotep{Q} }
\end{mathpar}

The astute reader will have noticed that the mutual recursion of names
and processes imposes a mutual recursion on alpha-equivalence and
structural equivalence via name-equivalence. Fortunately, all of this
works out pleasantly and we may calculate in the natural way, free of
concern. The reader interested in the details is referred to the
appendix \ref{appendix:rho_details}.

\subsection{Substitution}

We use $\Proc$ for the set of processes, $\QProc$ for the set of
names, and $\id{\{}\vec{y} / \vec{x} \id{\}}$ to denote partial maps,
$s : \QProc \rightarrow \QProc$. A map, $s$ lifts, uniquely, to a map
on process terms, $\widehat{s} : \Proc \rightarrow \Proc$ by the
following equations.

\begin{mathpar}
  (0) \psubstp{Q}{P} := 0 \\
  (R \juxtap S) \psubstp{Q}{P}
  :=    
  (R)\psubstp{Q}{P} \juxtap (S) \psubstp{Q}{P} \\
  (x?(y).R) \psubstp{Q}{P}    
  :=    
  (x)\substp{Q}{P} (z)\concat( (R \psubstn{z}{y}) \psubstp{Q}{P} ) \\
  (\lift{x}{R}) \psubstp{Q}{P}  
  :=
  \lift{(x)\substp{Q}{P}}{ R \psubstp{Q}{P} } \\
%   (\dropn{x})  \psubstp{Q}{P}       
%   := 
%   \left\{ 
%     \begin{array}{ccc} 
%       \dropn{\quotep{Q}} & & x \nameeq \quotep{P} \\
%       \dropn{x} & & otherwise \\
%     \end{array}
%   \right. 
  (\dropn{x})  \psubstp{Q}{P}       
  := 
  \left\{ 
    \begin{array}{ccc} 
      Q & & x \nameeq \quotep{P} \\
      \dropn{x} & & otherwise \\
    \end{array}
  \right.
\end{mathpar}
 

where

\begin{eqnarray}
  (x)\id{\{} \lpquote Q \rpquote / \lpquote P \rpquote \id{\}}            = 
  \left\{ 
    \begin{array}{ccc}
      \lpquote Q \rpquote & & x \nameeq \lpquote P \rpquote \\
      x & & otherwise \\
    \end{array}
  \right. \nonumber
\end{eqnarray}

and $z$ is chosen distinct from $\quotep{P}$, $\quotep{Q}$, the free
names in $Q$, and all the names in $R$. Our $\alpha$-equivalence will
be built in the standard way from this substitution.

\begin{remark}\label{rem:no_self_referential_names}
  One consequence of these definitions is that $\forall P. \quotep{P}
  \not\in \freenames{P}$.
\end{remark}

\subsection{ Dynamic quote: an example }

Anticipating something of what's to come, consider applying the
substitution, $\widehat{\id{\{}u / z \id{\}}}$, to the following pair
of processes, $\lift{w}{y!(z)}$ and $w[ \lpquote y!(z) \rpquote ]$.

\begin{eqnarray}
	\lift{w}{y!(z)}\widehat{\id{\{}u / z \id{\}}}
		& = &
		\lift{w}{y!(u)} \nonumber\\
	w[ \lpquote y!(z) \rpquote ] \widehat{ \id{\{}u / z \id{\}} }
		& = &
		w[ \lpquote y!(z) \rpquote ] \nonumber
\end{eqnarray}

Because the body of the process between quotes is impervious to
substitution, we get radically different answers. In fact, by
examining the first process in an input context,
e.g. $x?(z).\lift{w}{y!(z)}$, we see that the process under the lift
operator may be shaped by prefixed inputs binding a name inside it. In
this sense, the lift operator will be seen as a way to dynamically
construct processes before reifying them as names.

Finally equipped with these standard features we can present the
dynamics of the calculus.

\subsubsection{Operational semantics} 

Finally, we introduce the computational dynamics. What marks these
algebras as distinct from other more traditionally studied algebraic
structures, e.g. vector spaces or polynomial rings, is the manner in
which dynamics is captured. In traditional structures, dynamics is typically
expressed through morphisms between such structures, as in linear maps
between vector spaces or morphisms between rings. In algebras
associated with the semantics of computation, the dynamics is
expressed as part of the algebraic structure itself, through a
reduction reduction relation typically denoted by $\red$. Below, we
give a recursive presentation of this relation for the calculus used
in the encoding.

$\red \subseteq \pi \times \pi$
$\red : \pi \to \mathcal{P}(\pi)$

\begin{mathpar}
  \inferrule* [lab=Comm] { \textsf{match}( x_{src}, x_{trgt} ) } { x_{trgt}?(y)P \; | \; x_{src}!\langle {Q} \rangle \red P\{\quotep{Q}/y}\} }
  \and \\
  \inferrule* [lab=Par] {{P} \red {P}'} {{{P} | {Q}} \red {{P}' | {Q}}}
  \and
  \inferrule* [lab=Equiv]{{{P} \scong {P}'} \andalso {{P}' \red {Q}'} \andalso {{Q}' \scong {Q}}}{{P} \red {Q}}
\end{mathpar}

\begin{eqnarray*}
  match_{\equiv} (\quotep{P},\quotep{Q}) & := & P \equiv Q \\
  match_{\dagger}(\quotep{P},\quotep{Q}) & := & \forall R. P|Q \red^{*} R => R \red^{*} 0 \\
  match_{K}(\quotep{P},\quotep{Q}) & := & K \mbox{ for some context } K
\end{eqnarray*}

$u?(x)P | u!\langle Q \rangle \red P\{\quotep{Q}/x\}$

%We write $\wred$ for $\red^*$, and $P\red$ if $\exists Q $ such that $ P \red Q$.
We write $P\red$ if $\exists Q $ such that $ P \red Q$ and $P\not\red$, otherwise.

\section{Replication}

As mentioned before, it is known that replication (and hence
recursion) can be implemented in a higher-order process algebra
\cite{SangiorgiWalker}. As our first example of calculation with the
machinery thus far presented we give the construction explicitly in
the {\rhoc}.

\begin{eqnarray}
	D_{x} & := & \prefix{x}{y}{(\binpar{\outputp{x}{y}}{@{y}})} \nonumber\\
	\bangp_{x}{P} & := & \binpar{{x}!\langle{\binpar{D_{x}}{P}}\rangle}{D_{x}} \nonumber
\end{eqnarray}

\begin{eqnarray}
	\bangp_{x}{P} & & \nonumber\\
	=
	& {x}!\langle{(\prefix{x}{y}{(\outputp{x}{y} | @{y})) | P}}\rangle 
	      | \prefix{x}{y}{(\outputp{x}{y} | @{y})} & \nonumber\\
	\red
	& (\outputp{x}{y} | @{y})\substn{\quotep{(\prefix{x}{y}{(@{y} | \outputp{x}{y})) | P}}}{y} & \nonumber\\
	=
	& \outputp{x}{\quotep{(\prefix{x}{y}{(\outputp{x}{y} | @{y})) | P}}}
	  | {(\prefix{x}{y}{(\outputp{x}{y} | @{y})) | P}} & \nonumber\\
	\red
	& \ldots & \nonumber\\
	\red^*
	& P | P | \ldots & \nonumber
\end{eqnarray}

Of course, this encoding, as an implementation, runs away, unfolding
$\bangp{P}$ eagerly. A lazier and more implementable replication
operator, restricted to input-guarded processes, may be obtained as follows.

\begin{eqnarray}
\bangp{\prefix{u}{v}{P}} 
	:= 
	\binpar{\lift{x}{\prefix{u}{v}{(\binpar{D(x)}{P})}}}{D(x)} \nonumber
\end{eqnarray}

\begin{remark}
  Note that the lazier definition still does not deal with summation
  or mixed summation (i.e. sums over input and output). The reader is
  invited to construct definitions of replication that deal with these
  features. 

  Further, the definitions are parameterized in a name, $x$. Can you,
  gentle reader, make a definition that eliminates this parameter and
  guarantees no accidental interaction between the replication
  machinery and the process being replicated -- i.e. no accidental
  sharing of names used by the process to get its work done and the
  name(s) used by the replication to effect copying. This latter
  revision of the definition of replication is crucial to obtaining
  the expected identity $!!P \sim !P$.
\end{remark}

\begin{remark}\label{rem:paradoxical_combinator}
  The reader familiar with the lambda calculus will have noticed the
  similarity between $D$ and the paradoxical combinator.

  [Ed. note: the existence of this seems to suggest we have to be more
  restrictive on the set of processes and names we admit if we are to
  support no-cloning.]
\end{remark}

\subsubsection{Bisimulation}

The computational dynamics gives rise to another kind of equivalence,
the equivalence of computational behavior. As previously mentioned
this is typically captured \emph{via} some form of bisimulation.

% The notion we use in this paper is weak barbed bisimulation
% \cite{milner91polyadicpi}.

The notion we use in this paper is derived from weak barbed
bisimulation \cite{milner91polyadicpi}. 

\begin{definition}
An \emph{observation relation}, $\downarrow_{\mathcal N}$, over a set
of names, $\mathcal N$, is the smallest relation satisfying the rules
below.

\infrule[Out-barb]{y \in {\mathcal N}, \; x \nameeq y}
		  {\outputp{x}{v} \downarrow_{\mathcal N} x}
\infrule[Par-barb]{\mbox{$P\downarrow_{\mathcal N} x$ or $Q\downarrow_{\mathcal N} x$}}
		  {\binpar{P}{Q} \downarrow_{\mathcal N} x}

We write $P \Downarrow_{\mathcal N} x$ if there is $Q$ such that 
$P \wred Q$ and $Q \downarrow_{\mathcal N} x$.
\end{definition}

\begin{definition}
%\label{def.bbisim}
An  ${\mathcal N}$-\emph{barbed bisimulation} over a set of names, ${\mathcal N}$, is a symmetric binary relation 
${\mathcal S}_{\mathcal N}$ between agents such that $P\rel{S}_{\mathcal N}Q$ implies:
\begin{enumerate}
\item If $P \red P'$ then $Q \wred Q'$ and $P'\rel{S}_{\mathcal N} Q'$.
\item If $P\downarrow_{\mathcal N} x$, then $Q\Downarrow_{\mathcal N} x$.
\end{enumerate}
$P$ is ${\mathcal N}$-barbed bisimilar to $Q$, written
$P \wbbisim_{\mathcal N} Q$, if $P \rel{S}_{\mathcal N} Q$ for some ${\mathcal N}$-barbed bisimulation ${\mathcal S}_{\mathcal N}$.
\end{definition}

$\mathcal{R} \subseteq \pi \times \pi$

$P \mathcal{R} Q => \forall P'. P \red P' \Rightarrow \exists Q'. Q \red Q', P' \mathcal{R} Q'$

$P \vdash x \Rightarrow Q \vdash x$

\begin{mathpar}
  \inferrule*[lab=Out-barb]{x \nameeq y}{{y}!\langle{Q}\rangle \vdash x}
  \and
  \inferrule*[lab=Par-barb]{\mbox{$P\vdash x$ or $Q\vdash x$}}{\binpar{P}{Q} \vdash x}
\end{mathpar}

\subsubsection{Contexts}

One of the principle advantages of computational calculi like the
$\pi$-calculus is a well-defined notion of context,
contextual-equivalence and a correlation between
contextual-equivalence and notions of bisimulation. The notion of
context allows the decomposition of a process into (sub-)process and
its syntactic environment, its context. Thus, a context may be
thought of as a process with a ``hole'' (written $\Box$) in it. The
application of a context $M$ to a process $P$, written $M[P]$, is
tantamount to filling the hole in $M$ with $P$. In this paper we do
not need the full weight of this theory, but do make use of the notion
of context in the proof the main theorem. 

\begin{mathpar}
  \inferrule* [lab=summation] {} {{M_{M},M_{N}} \bc \Box \;|\; x.M_{A} \;|\; M_{M}+M_{N}}
  \and
  \inferrule* [lab=agent] {} {{M_{A}} \bc (\vec{x})M_{P} \;| \; \clift{P_0,\ldots,M_{P},\ldots,P_N}}
  \and \\
  \inferrule* [lab=process] {} {{M_{P}} \bc M_{N} \;| \;P|M_{P} }
\end{mathpar} 

\begin{mathpar}
  \inferrule* [lab=sychronization] {} {M_{N} \bc \Box \;|\; x?M_{F} \;|\; x!M_{C}}
  \and
  \inferrule* [lab=abstraction] {} {{M_{F}} \bc (x)M_{P} }
  \and
  \inferrule* [lab=concretion] {} {{M_{C}} \bc \langle M_{P} \rangle }
  \and \\
  \inferrule* [lab=process] {} {{M_{P}} \bc M_{N} \;| \;P|M_{P} }
\end{mathpar}

\begin{definition}[contextual application] Given a context $M$, and
  process $P$, we define the \emph{contextual application}, $M[P] :=
  M\{P/\Box\}$. That is, the contextual application of M to P is the
  substitution of $P$ for $\Box$ in $M$.
\end{definition}

$\meaningof{-} : L \to \mathcal{P}(\pi)$

\begin{mathpar}
  \inferrule* [lab=collection] {} {\meaningof{true} = \pi, \and \meaningof{~E} = \pi \setminus \meaningof{E}, \and \meaningof{E_{1} \& E_{2}} = \meaningof{E_{1}} \cap \meaningof{E_{2}}}
\end{mathpar}

\begin{mathpar}
  \inferrule* [lab=structure] {} {\meaningof{0} = \{ P \in \pi | P \equiv 0 \}, \and \\ \meaningof{E_1 | E_2} = \{ P \in \pi | P \equiv P_{1} | P_{2}, P_{1} \in \meaningof{E_{1}}, P_{2} \in \meaningof{E_2}\} }
\end{mathpar}

\begin{mathpar}
 \inferrule* [lab=behavior] {} {\meaningof{\langle a?b \rangle E} = \{ P \in \pi | P \equiv Q | u?(y)P', \\ \and \\\\ \and \\ \;\;\; u \in \meaningof{a}, \forall z.P'\{z/y\} \in \meaningof{E\{z/b\}}\}, \and \\ \meaningof{a!E} = \{ P \in \pi | P \equiv Q | x!\langle P' \rangle, x \in \meaningof{a} P' \in \meaningof{E}\} }
\end{mathpar}

\begin{mathpar}
 \inferrule* [lab=nominal] {} {\meaningof{\quotep{E}} = \{ \quotep{P} \in \quotep{\pi} | P \in \meaningof{E} \}, \and \meaningof{\quotep{P}} = \{ \quotep{Q} \in \quotep{\pi} | P \equiv Q \} \and \\ \meaningof{@\quotep{E}} = \{ P \in \pi | P \equiv @x, x \in \meaningof{E} \}}
\end{mathpar}

\begin{eqnarray*}
  \\
  \meaningof{-} : TS \to ST
\end{eqnarray*}

\begin{eqnarray*}
  \\
  L : TS \to ST
\end{eqnarray*}

\begin{eqnarray*}
  \\
  P \models E \iff P \in \meaningof{E}
\end{eqnarray*}

\begin{eqnarray*}
  P \approx_{L} Q \iff \forall E \in L. P \models E \iff Q \models E
\end{eqnarray*}

\begin{eqnarray*}
  P \approx_{K} Q
\end{eqnarray*}

\begin{eqnarray*}
  P \approx Q
\end{eqnarray*}

$\approx_{K} = \approx = \approx_{L}$

\subsubsection{Contextual duality}

Note that contexts extend the quotation operation to a family of
operations from processes to names. Given a context, $M$, we can
define a \emph{nominal context}, $\quotep{M}$ by $\quotep{M}[P] :=
\quotep{M[P]}$. To foreshadow what is to come we observe that these
operations enjoy a duality with processes very much like the duality
between vectors and maps from vectors to scalars.

Further, because the calculus is essentially higher-order, we have a
correspondence between contexts and processes. More specifically,
given a name $x$ and a context $M$ we can construct $M^{*}_{x}$ such
that 

\begin{mathpar}
  M^{*}_{x} | \lift{x}{P} \red M[P]
\end{mathpar}

namely,

\begin{mathpar}
  M^{*}_{x} := x?(u).M[\dropn{u}]
\end{mathpar}

The dependence of $M^{*}_{x}$ on a name makes it an abstraction, 

\begin{mathpar}
  M^{*} := (x)x?(u).M[\dropn{u}]
\end{mathpar}

\subsection{Additional notation}

It will sometimes be convenient to denote the process a name
quotes. We already have the notation $x = \quotep{P}$, but it will be
convenient to introduce an alternate notation, $\procn{x}$, when we
want to emphasize the connection to the use of the name. Note that, by
virtue of name equivalence, $\quotep{\procn{x}} \nameeq x$; so, the
notation is consistent with previous definitions.

Further, because names have structure it is possible to effect
substitutions on the basis of that structure. This means we need to
upgrade our notation for substitutions, which we accomplish by
adapting comprehension notation. Thus,

\begin{mathpar}
  P\{ y / x : x \in S \}
\end{mathpar}

is interpreted to mean the process derived from P by replacing (in a
capture-avoiding manner) each occurrence of $x$ in $S$ by $y$. For example,

\begin{mathpar}
  P\{ \quotep{\procn{x}|\procn{x}} / x : x \in \freenames{P} \}
\end{mathpar}

will replace each (occurrence) of a free name $x$ in $P$ by
$\quotep{\procn{x}|\procn{x}}$.

Also, we will avail ourselves of the notation $x^{L}$ and $x^{R}$ to
denote injections of a name into disjoint copies of the name
space. There are numerous ways to accomplish this. One example can be
found in \cite{MeredithR05}. This notation overloads to vectors of
names: $\vec{x}^{\pi} := (x_{i}^{\pi} \; : \; 0 \leq i < |\vec{x}| )$ where $\pi \in \{L,R\}$.

We also use $P^{\Box} := P|\Box$.

In \cite{MeredithR05} an interpretation of the new operator is
given. It turns out that there are several possible interpretations
all enjoying the requisite algebraic properties of the operator (see
\cite{milner91polyadicpi}). We will therefore make liberal use of
$(\nu\; \vec{x})P$.

% subsection the_syntax_and_semantics_of_the_notation_system (end)   

\input{qm2pi.qmops} 

\input{qm2pi.sterngerlach} 

\input{qm2pi.metric} 

% section concurrent_process_calculi (end)

%\input{qm2pi.proofsketch}

% section proof sketch (end)

%\input{qm2pi.slviaknots} 

% section spatial logic via knots (end)

\input{qm2pi.conclusion}

% section conclusion (end)

%\input{qm2pi.dtcodes} 

% section wiring algorithm (end)

\input{qm2pi.ack} 

% section acknowledgments (end)

\newpage


\bibliographystyle{plain}   
\bibliography{../../biblios/main.bib}

\input{qm2pi.rhodetails}

\end{document}

 

%\ifpdf
%\usepackage[pdftex]{graphicx}
%\else
%\usepackage{graphicx}
%\fi

 % \ifpdf
%  \usepackage{pdfsync}
%  \if


%\title{Brief Article}
%\author{David F. Snyder}
%\author{L.G. Meredith}

%\address{Dept. of Math., Texas State University--San Marcos, San Marcos, TX 78666}
       
\pagestyle{empty}


\begin{document}

\lstset{language=[Objective]Caml,frame=shadowbox}

\documentclass[12pt]{llncs}
%\documentclass{jktr}

\usepackage[pdftex]{hyperref}                   
\usepackage {listings}
\usepackage {mathpartir}
\usepackage{bcprules}
%\usepackage{listings}
                       
\usepackage{graphicx} 
%\usepackage[margins=2.5cm,nohead,nofoot]{geometry}
%\usepackage{geometry}
\usepackage{amsfonts}
\usepackage{amstext}
\usepackage{latexsym}
\usepackage{amssymb}
\usepackage{color}


%\include{myPreamble}
\include{qm2pi.local} 

%\ifpdf
%\usepackage[pdftex]{graphicx}
%\else
%\usepackage{graphicx}
%\fi

 % \ifpdf
%  \usepackage{pdfsync}
%  \if


%\title{Brief Article}
%\author{David F. Snyder}
%\author{L.G. Meredith}

%\address{Dept. of Math., Texas State University--San Marcos, San Marcos, TX 78666}
       
\pagestyle{empty}


\begin{document}

\lstset{language=[Objective]Caml,frame=shadowbox}

\input{qm2pi.front}

% section front matter (end)

\input{qm2pi.intro} 
 
% section introduction (end)

% \input{qm2pi.knotations} 

% section notation (end)

\input{qm2pi.process.calculi} 

% section concurrent_process_calculi_and_spatial_logics_ (end)
    
%\input{qm2pi.knots2pi} 

%\input{qm2pi.trefoil} 

%\input{qm2pi.mainthm} 

% subsection basic_interpretation (end)

%\input{qm2pi.rho.presentation} 
\subsection{The syntax and semantics of the notation system}\label{sub:the_syntax_and_semantics_of_the_notation_system} % (fold)

We now summarize a technical presentation of the calculus that
embodies our theory of dynamics. The typical presentation of such a
calculus follows the style of giving generators and relations on
them. The grammar, below, describing term constructors, freely
generates the set of processes, $\Proc$. This set is then quotiented
by a relation known as structural congruence and it is over this set
that the notion of dynamics is expressed. This presentation is
essentially that of \cite{MeredithR05} with the addition of
polyadicity and summation. For readability we have relegated some of
the technical subtleties to an appendix.

\subsubsection{Process grammar}\label{subsub:process_grammar}

\begin{mathpar}
  \inferrule* [lab=synchronization] {} {{M} \bc \pzero \;|\; x?F \;|\; x!C }
  \and
  \inferrule* [lab=abstraction] {} {{F} \bc (x)P}
  \and
  \inferrule* [lab=concretion] {} {{C} \bc \langle Q \rangle}
  \and
  \inferrule* [lab=process] {} {{P,Q} \bc M \;| \;P|Q \;|\; @{x}}
  \and
  \inferrule* [lab=name] {} {{x} \bc \quotep{P}}
\end{mathpar} 

Note that $\vec{x}$ (resp. $\vec{P}$) denotes a vector of names
(resp. processes) of length $|\vec{x}|$ (resp. $|\vec{P}|$). We adopt
the following useful abbreviations.

\begin{mathpar}
   x?(\vec{y}).P := x.(\vec{y})P \and  x\clift{\vec{P}} := x.\clift{\vec{P}}
   \and x!(y) := \lift{x}{\dropn{y}}
   \and \Pi_{i=0}^{n-1}P_i := P_0 | \ldots | P_{n-1}
\end{mathpar}

\subsubsection{Structural congruence}

\paragraph{Free and bound names and alpha-equivalence.} At the
core of structural equivalence is alpha-equivalence which identifies
process that are the same up to a change of variable. Formally, we
recognize the distinction between free and bound names. The free names
of a process, $\freenames{P}$, may be calculated recursively as
follows:

\begin{mathpar}
\freenames{\pzero} := \emptyset
  \and \\
  \freenames{x?(y).P} := \{ x \} \cup (\freenames{P} \setminus \{ y \})
  \and 
  \freenames{x!\langle P \rangle} := \{ x \} \cup \{ P \} 
  \and \\
  \freenames{P|Q} := \freenames{P} \cup \freenames{Q}
  \and \\
  \freenames{@{x}} := \{ x \}
\end{mathpar}

$\pi$
$\quotep{\pi}$

$\freenames{-} : \pi \to \mathcal{P}(\quotep{\pi})$

\begin{eqnarray*}
  \freenames{\pzero} & := & \emptyset \\
  \freenames{x?(y).P} & := & \{ x \} \cup (\freenames{P} \setminus \{ y \}) \\
  \freenames{x!\langle P \rangle} & := & \{ x \} \cup \{ P \} \\
  \freenames{P|Q} & := & \freenames{P} \cup \freenames{Q} \\
  \freenames{\dropn{x}} & := & \{ x \}
\end{eqnarray*}

The bound names of a process, $\boundnames{P}$, are those names occurring in $P$
that are not free. For example, in $x?(y).0$, the name $x$ is free, while $y$ is bound.

\begin{mathpar}
  \inferrule* [lab=monoidal-laws] {} { P|Q \equiv Q|P \and P|0 \equiv P \and P|(Q|R) \equiv (P|Q)|R }
\end{mathpar}

\begin{mathpar}
  \inferrule* [lab=alpha-equivalence] {} { (x)P \equiv (y)P\{y/x\} \and y \not\in \freenames{P} }
\end{mathpar}

\begin{definition}
Then two processes, $P,Q$, are alpha-equivalent if $P = Q\{\vec{y}/\vec{x}\}$ for
some $\vec{x} \in \boundnames{Q},\vec{y} \in \boundnames{P}$, where $Q\{\vec{y}/\vec{x}\}$
denotes the capture-avoiding substitution of $\vec{y}$ for $\vec{x}$ in $Q$.
\end{definition}

\begin{definition}
  The {\em structural congruence} \cite{SangiorgiWalker} , $\equiv$,
  between processes is the least congruence containing
  alpha-equivalence, satisfying the abelian monoid laws
  (associativity, commutativity and $\pzero$ as identity) for parallel
  composition $|$ and for summation $+$.
\end{definition}

\subsection{Name equivalence}

We take name equivalence, written $\nameeq$, to be the smallest
equivalence relation generated by the following rules.

\begin{mathpar}
\inferrule*[lab=Quote-drop]
{ }
{ \quotep{@{x}} \nameeq x }

\inferrule*[lab=Struct-equiv]
{ P \scong Q }
{ \quotep{P} \nameeq \quotep{Q} }
\end{mathpar}

The astute reader will have noticed that the mutual recursion of names
and processes imposes a mutual recursion on alpha-equivalence and
structural equivalence via name-equivalence. Fortunately, all of this
works out pleasantly and we may calculate in the natural way, free of
concern. The reader interested in the details is referred to the
appendix \ref{appendix:rho_details}.

\subsection{Substitution}

We use $\Proc$ for the set of processes, $\QProc$ for the set of
names, and $\id{\{}\vec{y} / \vec{x} \id{\}}$ to denote partial maps,
$s : \QProc \rightarrow \QProc$. A map, $s$ lifts, uniquely, to a map
on process terms, $\widehat{s} : \Proc \rightarrow \Proc$ by the
following equations.

\begin{mathpar}
  (0) \psubstp{Q}{P} := 0 \\
  (R \juxtap S) \psubstp{Q}{P}
  :=    
  (R)\psubstp{Q}{P} \juxtap (S) \psubstp{Q}{P} \\
  (x?(y).R) \psubstp{Q}{P}    
  :=    
  (x)\substp{Q}{P} (z)\concat( (R \psubstn{z}{y}) \psubstp{Q}{P} ) \\
  (\lift{x}{R}) \psubstp{Q}{P}  
  :=
  \lift{(x)\substp{Q}{P}}{ R \psubstp{Q}{P} } \\
%   (\dropn{x})  \psubstp{Q}{P}       
%   := 
%   \left\{ 
%     \begin{array}{ccc} 
%       \dropn{\quotep{Q}} & & x \nameeq \quotep{P} \\
%       \dropn{x} & & otherwise \\
%     \end{array}
%   \right. 
  (\dropn{x})  \psubstp{Q}{P}       
  := 
  \left\{ 
    \begin{array}{ccc} 
      Q & & x \nameeq \quotep{P} \\
      \dropn{x} & & otherwise \\
    \end{array}
  \right.
\end{mathpar}
 

where

\begin{eqnarray}
  (x)\id{\{} \lpquote Q \rpquote / \lpquote P \rpquote \id{\}}            = 
  \left\{ 
    \begin{array}{ccc}
      \lpquote Q \rpquote & & x \nameeq \lpquote P \rpquote \\
      x & & otherwise \\
    \end{array}
  \right. \nonumber
\end{eqnarray}

and $z$ is chosen distinct from $\quotep{P}$, $\quotep{Q}$, the free
names in $Q$, and all the names in $R$. Our $\alpha$-equivalence will
be built in the standard way from this substitution.

\begin{remark}\label{rem:no_self_referential_names}
  One consequence of these definitions is that $\forall P. \quotep{P}
  \not\in \freenames{P}$.
\end{remark}

\subsection{ Dynamic quote: an example }

Anticipating something of what's to come, consider applying the
substitution, $\widehat{\id{\{}u / z \id{\}}}$, to the following pair
of processes, $\lift{w}{y!(z)}$ and $w[ \lpquote y!(z) \rpquote ]$.

\begin{eqnarray}
	\lift{w}{y!(z)}\widehat{\id{\{}u / z \id{\}}}
		& = &
		\lift{w}{y!(u)} \nonumber\\
	w[ \lpquote y!(z) \rpquote ] \widehat{ \id{\{}u / z \id{\}} }
		& = &
		w[ \lpquote y!(z) \rpquote ] \nonumber
\end{eqnarray}

Because the body of the process between quotes is impervious to
substitution, we get radically different answers. In fact, by
examining the first process in an input context,
e.g. $x?(z).\lift{w}{y!(z)}$, we see that the process under the lift
operator may be shaped by prefixed inputs binding a name inside it. In
this sense, the lift operator will be seen as a way to dynamically
construct processes before reifying them as names.

Finally equipped with these standard features we can present the
dynamics of the calculus.

\subsubsection{Operational semantics} 

Finally, we introduce the computational dynamics. What marks these
algebras as distinct from other more traditionally studied algebraic
structures, e.g. vector spaces or polynomial rings, is the manner in
which dynamics is captured. In traditional structures, dynamics is typically
expressed through morphisms between such structures, as in linear maps
between vector spaces or morphisms between rings. In algebras
associated with the semantics of computation, the dynamics is
expressed as part of the algebraic structure itself, through a
reduction reduction relation typically denoted by $\red$. Below, we
give a recursive presentation of this relation for the calculus used
in the encoding.

$\red \subseteq \pi \times \pi$
$\red : \pi \to \mathcal{P}(\pi)$

\begin{mathpar}
  \inferrule* [lab=Comm] { \textsf{match}( x_{src}, x_{trgt} ) } { x_{trgt}?(y)P \; | \; x_{src}!\langle {Q} \rangle \red P\{\quotep{Q}/y}\} }
  \and \\
  \inferrule* [lab=Par] {{P} \red {P}'} {{{P} | {Q}} \red {{P}' | {Q}}}
  \and
  \inferrule* [lab=Equiv]{{{P} \scong {P}'} \andalso {{P}' \red {Q}'} \andalso {{Q}' \scong {Q}}}{{P} \red {Q}}
\end{mathpar}

\begin{eqnarray*}
  match_{\equiv} (\quotep{P},\quotep{Q}) & := & P \equiv Q \\
  match_{\dagger}(\quotep{P},\quotep{Q}) & := & \forall R. P|Q \red^{*} R => R \red^{*} 0 \\
  match_{K}(\quotep{P},\quotep{Q}) & := & K \mbox{ for some context } K
\end{eqnarray*}

$u?(x)P | u!\langle Q \rangle \red P\{\quotep{Q}/x\}$

%We write $\wred$ for $\red^*$, and $P\red$ if $\exists Q $ such that $ P \red Q$.
We write $P\red$ if $\exists Q $ such that $ P \red Q$ and $P\not\red$, otherwise.

\section{Replication}

As mentioned before, it is known that replication (and hence
recursion) can be implemented in a higher-order process algebra
\cite{SangiorgiWalker}. As our first example of calculation with the
machinery thus far presented we give the construction explicitly in
the {\rhoc}.

\begin{eqnarray}
	D_{x} & := & \prefix{x}{y}{(\binpar{\outputp{x}{y}}{@{y}})} \nonumber\\
	\bangp_{x}{P} & := & \binpar{{x}!\langle{\binpar{D_{x}}{P}}\rangle}{D_{x}} \nonumber
\end{eqnarray}

\begin{eqnarray}
	\bangp_{x}{P} & & \nonumber\\
	=
	& {x}!\langle{(\prefix{x}{y}{(\outputp{x}{y} | @{y})) | P}}\rangle 
	      | \prefix{x}{y}{(\outputp{x}{y} | @{y})} & \nonumber\\
	\red
	& (\outputp{x}{y} | @{y})\substn{\quotep{(\prefix{x}{y}{(@{y} | \outputp{x}{y})) | P}}}{y} & \nonumber\\
	=
	& \outputp{x}{\quotep{(\prefix{x}{y}{(\outputp{x}{y} | @{y})) | P}}}
	  | {(\prefix{x}{y}{(\outputp{x}{y} | @{y})) | P}} & \nonumber\\
	\red
	& \ldots & \nonumber\\
	\red^*
	& P | P | \ldots & \nonumber
\end{eqnarray}

Of course, this encoding, as an implementation, runs away, unfolding
$\bangp{P}$ eagerly. A lazier and more implementable replication
operator, restricted to input-guarded processes, may be obtained as follows.

\begin{eqnarray}
\bangp{\prefix{u}{v}{P}} 
	:= 
	\binpar{\lift{x}{\prefix{u}{v}{(\binpar{D(x)}{P})}}}{D(x)} \nonumber
\end{eqnarray}

\begin{remark}
  Note that the lazier definition still does not deal with summation
  or mixed summation (i.e. sums over input and output). The reader is
  invited to construct definitions of replication that deal with these
  features. 

  Further, the definitions are parameterized in a name, $x$. Can you,
  gentle reader, make a definition that eliminates this parameter and
  guarantees no accidental interaction between the replication
  machinery and the process being replicated -- i.e. no accidental
  sharing of names used by the process to get its work done and the
  name(s) used by the replication to effect copying. This latter
  revision of the definition of replication is crucial to obtaining
  the expected identity $!!P \sim !P$.
\end{remark}

\begin{remark}\label{rem:paradoxical_combinator}
  The reader familiar with the lambda calculus will have noticed the
  similarity between $D$ and the paradoxical combinator.

  [Ed. note: the existence of this seems to suggest we have to be more
  restrictive on the set of processes and names we admit if we are to
  support no-cloning.]
\end{remark}

\subsubsection{Bisimulation}

The computational dynamics gives rise to another kind of equivalence,
the equivalence of computational behavior. As previously mentioned
this is typically captured \emph{via} some form of bisimulation.

% The notion we use in this paper is weak barbed bisimulation
% \cite{milner91polyadicpi}.

The notion we use in this paper is derived from weak barbed
bisimulation \cite{milner91polyadicpi}. 

\begin{definition}
An \emph{observation relation}, $\downarrow_{\mathcal N}$, over a set
of names, $\mathcal N$, is the smallest relation satisfying the rules
below.

\infrule[Out-barb]{y \in {\mathcal N}, \; x \nameeq y}
		  {\outputp{x}{v} \downarrow_{\mathcal N} x}
\infrule[Par-barb]{\mbox{$P\downarrow_{\mathcal N} x$ or $Q\downarrow_{\mathcal N} x$}}
		  {\binpar{P}{Q} \downarrow_{\mathcal N} x}

We write $P \Downarrow_{\mathcal N} x$ if there is $Q$ such that 
$P \wred Q$ and $Q \downarrow_{\mathcal N} x$.
\end{definition}

\begin{definition}
%\label{def.bbisim}
An  ${\mathcal N}$-\emph{barbed bisimulation} over a set of names, ${\mathcal N}$, is a symmetric binary relation 
${\mathcal S}_{\mathcal N}$ between agents such that $P\rel{S}_{\mathcal N}Q$ implies:
\begin{enumerate}
\item If $P \red P'$ then $Q \wred Q'$ and $P'\rel{S}_{\mathcal N} Q'$.
\item If $P\downarrow_{\mathcal N} x$, then $Q\Downarrow_{\mathcal N} x$.
\end{enumerate}
$P$ is ${\mathcal N}$-barbed bisimilar to $Q$, written
$P \wbbisim_{\mathcal N} Q$, if $P \rel{S}_{\mathcal N} Q$ for some ${\mathcal N}$-barbed bisimulation ${\mathcal S}_{\mathcal N}$.
\end{definition}

$\mathcal{R} \subseteq \pi \times \pi$

$P \mathcal{R} Q => \forall P'. P \red P' \Rightarrow \exists Q'. Q \red Q', P' \mathcal{R} Q'$

$P \vdash x \Rightarrow Q \vdash x$

\begin{mathpar}
  \inferrule*[lab=Out-barb]{x \nameeq y}{{y}!\langle{Q}\rangle \vdash x}
  \and
  \inferrule*[lab=Par-barb]{\mbox{$P\vdash x$ or $Q\vdash x$}}{\binpar{P}{Q} \vdash x}
\end{mathpar}

\subsubsection{Contexts}

One of the principle advantages of computational calculi like the
$\pi$-calculus is a well-defined notion of context,
contextual-equivalence and a correlation between
contextual-equivalence and notions of bisimulation. The notion of
context allows the decomposition of a process into (sub-)process and
its syntactic environment, its context. Thus, a context may be
thought of as a process with a ``hole'' (written $\Box$) in it. The
application of a context $M$ to a process $P$, written $M[P]$, is
tantamount to filling the hole in $M$ with $P$. In this paper we do
not need the full weight of this theory, but do make use of the notion
of context in the proof the main theorem. 

\begin{mathpar}
  \inferrule* [lab=summation] {} {{M_{M},M_{N}} \bc \Box \;|\; x.M_{A} \;|\; M_{M}+M_{N}}
  \and
  \inferrule* [lab=agent] {} {{M_{A}} \bc (\vec{x})M_{P} \;| \; \clift{P_0,\ldots,M_{P},\ldots,P_N}}
  \and \\
  \inferrule* [lab=process] {} {{M_{P}} \bc M_{N} \;| \;P|M_{P} }
\end{mathpar} 

\begin{mathpar}
  \inferrule* [lab=sychronization] {} {M_{N} \bc \Box \;|\; x?M_{F} \;|\; x!M_{C}}
  \and
  \inferrule* [lab=abstraction] {} {{M_{F}} \bc (x)M_{P} }
  \and
  \inferrule* [lab=concretion] {} {{M_{C}} \bc \langle M_{P} \rangle }
  \and \\
  \inferrule* [lab=process] {} {{M_{P}} \bc M_{N} \;| \;P|M_{P} }
\end{mathpar}

\begin{definition}[contextual application] Given a context $M$, and
  process $P$, we define the \emph{contextual application}, $M[P] :=
  M\{P/\Box\}$. That is, the contextual application of M to P is the
  substitution of $P$ for $\Box$ in $M$.
\end{definition}

$\meaningof{-} : L \to \mathcal{P}(\pi)$

\begin{mathpar}
  \inferrule* [lab=collection] {} {\meaningof{true} = \pi, \and \meaningof{~E} = \pi \setminus \meaningof{E}, \and \meaningof{E_{1} \& E_{2}} = \meaningof{E_{1}} \cap \meaningof{E_{2}}}
\end{mathpar}

\begin{mathpar}
  \inferrule* [lab=structure] {} {\meaningof{0} = \{ P \in \pi | P \equiv 0 \}, \and \\ \meaningof{E_1 | E_2} = \{ P \in \pi | P \equiv P_{1} | P_{2}, P_{1} \in \meaningof{E_{1}}, P_{2} \in \meaningof{E_2}\} }
\end{mathpar}

\begin{mathpar}
 \inferrule* [lab=behavior] {} {\meaningof{\langle a?b \rangle E} = \{ P \in \pi | P \equiv Q | u?(y)P', \\ \and \\\\ \and \\ \;\;\; u \in \meaningof{a}, \forall z.P'\{z/y\} \in \meaningof{E\{z/b\}}\}, \and \\ \meaningof{a!E} = \{ P \in \pi | P \equiv Q | x!\langle P' \rangle, x \in \meaningof{a} P' \in \meaningof{E}\} }
\end{mathpar}

\begin{mathpar}
 \inferrule* [lab=nominal] {} {\meaningof{\quotep{E}} = \{ \quotep{P} \in \quotep{\pi} | P \in \meaningof{E} \}, \and \meaningof{\quotep{P}} = \{ \quotep{Q} \in \quotep{\pi} | P \equiv Q \} \and \\ \meaningof{@\quotep{E}} = \{ P \in \pi | P \equiv @x, x \in \meaningof{E} \}}
\end{mathpar}

\begin{eqnarray*}
  \\
  \meaningof{-} : TS \to ST
\end{eqnarray*}

\begin{eqnarray*}
  \\
  L : TS \to ST
\end{eqnarray*}

\begin{eqnarray*}
  \\
  P \models E \iff P \in \meaningof{E}
\end{eqnarray*}

\begin{eqnarray*}
  P \approx_{L} Q \iff \forall E \in L. P \models E \iff Q \models E
\end{eqnarray*}

\begin{eqnarray*}
  P \approx_{K} Q
\end{eqnarray*}

\begin{eqnarray*}
  P \approx Q
\end{eqnarray*}

$\approx_{K} = \approx = \approx_{L}$

\subsubsection{Contextual duality}

Note that contexts extend the quotation operation to a family of
operations from processes to names. Given a context, $M$, we can
define a \emph{nominal context}, $\quotep{M}$ by $\quotep{M}[P] :=
\quotep{M[P]}$. To foreshadow what is to come we observe that these
operations enjoy a duality with processes very much like the duality
between vectors and maps from vectors to scalars.

Further, because the calculus is essentially higher-order, we have a
correspondence between contexts and processes. More specifically,
given a name $x$ and a context $M$ we can construct $M^{*}_{x}$ such
that 

\begin{mathpar}
  M^{*}_{x} | \lift{x}{P} \red M[P]
\end{mathpar}

namely,

\begin{mathpar}
  M^{*}_{x} := x?(u).M[\dropn{u}]
\end{mathpar}

The dependence of $M^{*}_{x}$ on a name makes it an abstraction, 

\begin{mathpar}
  M^{*} := (x)x?(u).M[\dropn{u}]
\end{mathpar}

\subsection{Additional notation}

It will sometimes be convenient to denote the process a name
quotes. We already have the notation $x = \quotep{P}$, but it will be
convenient to introduce an alternate notation, $\procn{x}$, when we
want to emphasize the connection to the use of the name. Note that, by
virtue of name equivalence, $\quotep{\procn{x}} \nameeq x$; so, the
notation is consistent with previous definitions.

Further, because names have structure it is possible to effect
substitutions on the basis of that structure. This means we need to
upgrade our notation for substitutions, which we accomplish by
adapting comprehension notation. Thus,

\begin{mathpar}
  P\{ y / x : x \in S \}
\end{mathpar}

is interpreted to mean the process derived from P by replacing (in a
capture-avoiding manner) each occurrence of $x$ in $S$ by $y$. For example,

\begin{mathpar}
  P\{ \quotep{\procn{x}|\procn{x}} / x : x \in \freenames{P} \}
\end{mathpar}

will replace each (occurrence) of a free name $x$ in $P$ by
$\quotep{\procn{x}|\procn{x}}$.

Also, we will avail ourselves of the notation $x^{L}$ and $x^{R}$ to
denote injections of a name into disjoint copies of the name
space. There are numerous ways to accomplish this. One example can be
found in \cite{MeredithR05}. This notation overloads to vectors of
names: $\vec{x}^{\pi} := (x_{i}^{\pi} \; : \; 0 \leq i < |\vec{x}| )$ where $\pi \in \{L,R\}$.

We also use $P^{\Box} := P|\Box$.

In \cite{MeredithR05} an interpretation of the new operator is
given. It turns out that there are several possible interpretations
all enjoying the requisite algebraic properties of the operator (see
\cite{milner91polyadicpi}). We will therefore make liberal use of
$(\nu\; \vec{x})P$.

% subsection the_syntax_and_semantics_of_the_notation_system (end)   

\input{qm2pi.qmops} 

\input{qm2pi.sterngerlach} 

\input{qm2pi.metric} 

% section concurrent_process_calculi (end)

%\input{qm2pi.proofsketch}

% section proof sketch (end)

%\input{qm2pi.slviaknots} 

% section spatial logic via knots (end)

\input{qm2pi.conclusion}

% section conclusion (end)

%\input{qm2pi.dtcodes} 

% section wiring algorithm (end)

\input{qm2pi.ack} 

% section acknowledgments (end)

\newpage


\bibliographystyle{plain}   
\bibliography{../../biblios/main.bib}

\input{qm2pi.rhodetails}

\end{document}



% section front matter (end)

\section{Introduction}\label{sec:introduction} % (fold)
In this draft of the material i am going to have to dispense with the
usual writing conventions adopted in papers on these topics. i'm going
to have adopt whatever tone i need at the time i'm writing up the
calculations. Sometimes this may be very conversational; others it may
be the barest mathematical grunts; others still it may be that i have
lifted text from one of my other papers because the exposition of some
point was better said there. i hope that my readers are not unduly put
out by this decision. i'm not doing this to flout convention or be
rebellious. i find these calculations very technically challenging. To
keep everything going technically, something has to give; i have to
let go of some cognitive burden. So, the academic writing style --
with all of its trade-offs in terms of facilitating technical
communication -- is what i'm letting go of. Perhaps subsequent drafts
can be tightened and polished, but for now, i'm going to speak as if
we were sitting together in a coffee shop with a laptop, wifi and a
pad of paper and a pencil.

So, here's what i have to say. We -- you and i, comfortably ensconced
in our coffee shop and well-equipped with our tools -- can realize and
carry out the calculations of quantum mechanics over a very different
formal theory of dynamics, a formal theory of dynamics that
corresponds to a theory of concurrent computation with
\emph{reflection}. It has the advantage that the underlying theory is
already `quantized', but supports analogues all of the continuuous
operations. Strikingly, this underlying theory has recently been
connected with a notion of metric that we can show, by calculating
together, coincides with the metric induced by the inner product.

There are a lot of reasons why you might be interested in seeing
calculations of this form. Here's why i'm interested. For the past
several centuries there has been no competitor to the ``Newtonian''
account of dynamics. As a result the predominant share of accounts of
dynamical systems and situations have had to be formulated in terms of
the Newtonian machinery. i view this as an intellectually dangerous
position to occupy. Everything, despite it's intrinsic shape, turns
into a nail to be hit with this hammer. Recently, however, the theory
of computation has matured to the point where we have candidates for
theories of dynamics that offer very different perspective on
reasoning about dynamical systems and situations. Testing these
candidates against very successful accounts of dynamical situations,
like quantum mechanics, is going to give us some sense of how mature
they are and some measure of the quality of these accounts of
dynamics.

\subsection{Summary of contributions and outline of paper}

So, we're going to develop an interpretation of the operations of
quantum mechanics normally interpreted by Hilbert spaces and
operators. We're going to do this over a theory of computation. Note
that this is very different than the usual quantum computation program
which develops notions of computation over quantum mechanics. Rather,
we are developing a story that aligns with Wheeler's slogan: It from
Bit. To do this we will first provide an account of the theory of
computation at play here. Then we will dive into a calculation-driven
interpretation of the operations of quantum mechanics.

The reason we take this approach is that -- until very recently --
there hasn't been an axiomatic account of quantum mechanics. As a
result there has been no sharp delineation of the mathematical theory
supporting interpretation of the physical theory and the physical
theory, itself. So, ambient features of the maths are free to be
exploited (or supressed) without a real accounting of their physical
relevance. There is no sharp statement ``here's the physical theory''
qua \emph{theory} and ``here's the mathematical interpretation''
enabling a judgment of how faithful the interpretation is -- apart
from experimental observation. When there is an axiomatic account we
can judge how well a given mathematical formalism supports an
interpretation of the axioms, independent of
experimentation. Likewise, we can judge how well we have captured our
physical evidence and experience with our axiomatics, independent of
any specific mathematical implementation, with accidental detail that
may or may not have physical significance. 

In lieu of a fully fleshed out and vetted axiomatic account of quantum
mechanics, interpreting the operational notions in service of modeling
physical systems will have to suffice. In other words, we are not in
the business of providing a model of Hilbert spaces and operators. We
are in the business of providing a model of quantum mechanics because
we are motivated by testing our notions of dynamics against physical
theory; and, the predictive calculations of the physical theory must
serve as the best formulation -- shy of a fully fleshed out axiomatic
account -- of the physical theory itself (as they have for scientific
theories since time immemorial). Put another way, despite a
whole-hearted commitment to an It-from-Bit ontology, we are firmly
aligned with the shut-up-and-calculate camp as the best way to obtain
results either from the physical perspective or as a quality assurance
measure of our fledgling theory of dynamics.

In detail, we present a reflective process calculus. Then we develop
intuitive correspondences between the notions available in this
calculus and the usual physical notions supporting quantum mechanical
calculations. Thus, 

\begin{table}[htp]
  \center{
    \fbox{
      \begin{tabular}{c|c}
        quantum mechanics & process calculus \\
        \hline
        scalar & name \\
        state vector & process \\
        dual & contextual duals \\
        matrix & formal sums of process-context-dual pairs \\
        orthogonality & process annihilation \\
        inner product & execution-formula + quoting
      \end{tabular}
    }
  }
  \caption{QM - process calculi correspondences}
\end{table}

Then we tighten up these intuitions to operational definitions. We
employ the Dirac notation as the best proxy we can find for an
abstract syntax of the quantum mechanical notions. The definitions we
develop put us in contact with equational constraints coming from the
theory that we demonstrate the definitions and calculations satisfy.

This puts us in a position to shut up and calculate for the
Stern-Gerlach experimental set up, showing how these predictive
calculations become calculations on processes in our theory of a
reflective process calculus.

Penultimately, we demonstrate that the notion of metric coming from
the inner product coincides with the notion of metric available from
the theory of bisimulation. This demonstration gives us the right to
think of space as arising from behavior. Finally, we consider where we
might go from the new vantage point we have obtained.

% section introduction (end) 
 
% section introduction (end)

% \documentclass[12pt]{llncs}
%\documentclass{jktr}

\usepackage[pdftex]{hyperref}                   
\usepackage {listings}
\usepackage {mathpartir}
\usepackage{bcprules}
%\usepackage{listings}
                       
\usepackage{graphicx} 
%\usepackage[margins=2.5cm,nohead,nofoot]{geometry}
%\usepackage{geometry}
\usepackage{amsfonts}
\usepackage{amstext}
\usepackage{latexsym}
\usepackage{amssymb}
\usepackage{color}


%\include{myPreamble}
\include{qm2pi.local} 

%\ifpdf
%\usepackage[pdftex]{graphicx}
%\else
%\usepackage{graphicx}
%\fi

 % \ifpdf
%  \usepackage{pdfsync}
%  \if


%\title{Brief Article}
%\author{David F. Snyder}
%\author{L.G. Meredith}

%\address{Dept. of Math., Texas State University--San Marcos, San Marcos, TX 78666}
       
\pagestyle{empty}


\begin{document}

\lstset{language=[Objective]Caml,frame=shadowbox}

\input{qm2pi.front}

% section front matter (end)

\input{qm2pi.intro} 
 
% section introduction (end)

% \input{qm2pi.knotations} 

% section notation (end)

\input{qm2pi.process.calculi} 

% section concurrent_process_calculi_and_spatial_logics_ (end)
    
%\input{qm2pi.knots2pi} 

%\input{qm2pi.trefoil} 

%\input{qm2pi.mainthm} 

% subsection basic_interpretation (end)

%\input{qm2pi.rho.presentation} 
\subsection{The syntax and semantics of the notation system}\label{sub:the_syntax_and_semantics_of_the_notation_system} % (fold)

We now summarize a technical presentation of the calculus that
embodies our theory of dynamics. The typical presentation of such a
calculus follows the style of giving generators and relations on
them. The grammar, below, describing term constructors, freely
generates the set of processes, $\Proc$. This set is then quotiented
by a relation known as structural congruence and it is over this set
that the notion of dynamics is expressed. This presentation is
essentially that of \cite{MeredithR05} with the addition of
polyadicity and summation. For readability we have relegated some of
the technical subtleties to an appendix.

\subsubsection{Process grammar}\label{subsub:process_grammar}

\begin{mathpar}
  \inferrule* [lab=synchronization] {} {{M} \bc \pzero \;|\; x?F \;|\; x!C }
  \and
  \inferrule* [lab=abstraction] {} {{F} \bc (x)P}
  \and
  \inferrule* [lab=concretion] {} {{C} \bc \langle Q \rangle}
  \and
  \inferrule* [lab=process] {} {{P,Q} \bc M \;| \;P|Q \;|\; @{x}}
  \and
  \inferrule* [lab=name] {} {{x} \bc \quotep{P}}
\end{mathpar} 

Note that $\vec{x}$ (resp. $\vec{P}$) denotes a vector of names
(resp. processes) of length $|\vec{x}|$ (resp. $|\vec{P}|$). We adopt
the following useful abbreviations.

\begin{mathpar}
   x?(\vec{y}).P := x.(\vec{y})P \and  x\clift{\vec{P}} := x.\clift{\vec{P}}
   \and x!(y) := \lift{x}{\dropn{y}}
   \and \Pi_{i=0}^{n-1}P_i := P_0 | \ldots | P_{n-1}
\end{mathpar}

\subsubsection{Structural congruence}

\paragraph{Free and bound names and alpha-equivalence.} At the
core of structural equivalence is alpha-equivalence which identifies
process that are the same up to a change of variable. Formally, we
recognize the distinction between free and bound names. The free names
of a process, $\freenames{P}$, may be calculated recursively as
follows:

\begin{mathpar}
\freenames{\pzero} := \emptyset
  \and \\
  \freenames{x?(y).P} := \{ x \} \cup (\freenames{P} \setminus \{ y \})
  \and 
  \freenames{x!\langle P \rangle} := \{ x \} \cup \{ P \} 
  \and \\
  \freenames{P|Q} := \freenames{P} \cup \freenames{Q}
  \and \\
  \freenames{@{x}} := \{ x \}
\end{mathpar}

$\pi$
$\quotep{\pi}$

$\freenames{-} : \pi \to \mathcal{P}(\quotep{\pi})$

\begin{eqnarray*}
  \freenames{\pzero} & := & \emptyset \\
  \freenames{x?(y).P} & := & \{ x \} \cup (\freenames{P} \setminus \{ y \}) \\
  \freenames{x!\langle P \rangle} & := & \{ x \} \cup \{ P \} \\
  \freenames{P|Q} & := & \freenames{P} \cup \freenames{Q} \\
  \freenames{\dropn{x}} & := & \{ x \}
\end{eqnarray*}

The bound names of a process, $\boundnames{P}$, are those names occurring in $P$
that are not free. For example, in $x?(y).0$, the name $x$ is free, while $y$ is bound.

\begin{mathpar}
  \inferrule* [lab=monoidal-laws] {} { P|Q \equiv Q|P \and P|0 \equiv P \and P|(Q|R) \equiv (P|Q)|R }
\end{mathpar}

\begin{mathpar}
  \inferrule* [lab=alpha-equivalence] {} { (x)P \equiv (y)P\{y/x\} \and y \not\in \freenames{P} }
\end{mathpar}

\begin{definition}
Then two processes, $P,Q$, are alpha-equivalent if $P = Q\{\vec{y}/\vec{x}\}$ for
some $\vec{x} \in \boundnames{Q},\vec{y} \in \boundnames{P}$, where $Q\{\vec{y}/\vec{x}\}$
denotes the capture-avoiding substitution of $\vec{y}$ for $\vec{x}$ in $Q$.
\end{definition}

\begin{definition}
  The {\em structural congruence} \cite{SangiorgiWalker} , $\equiv$,
  between processes is the least congruence containing
  alpha-equivalence, satisfying the abelian monoid laws
  (associativity, commutativity and $\pzero$ as identity) for parallel
  composition $|$ and for summation $+$.
\end{definition}

\subsection{Name equivalence}

We take name equivalence, written $\nameeq$, to be the smallest
equivalence relation generated by the following rules.

\begin{mathpar}
\inferrule*[lab=Quote-drop]
{ }
{ \quotep{@{x}} \nameeq x }

\inferrule*[lab=Struct-equiv]
{ P \scong Q }
{ \quotep{P} \nameeq \quotep{Q} }
\end{mathpar}

The astute reader will have noticed that the mutual recursion of names
and processes imposes a mutual recursion on alpha-equivalence and
structural equivalence via name-equivalence. Fortunately, all of this
works out pleasantly and we may calculate in the natural way, free of
concern. The reader interested in the details is referred to the
appendix \ref{appendix:rho_details}.

\subsection{Substitution}

We use $\Proc$ for the set of processes, $\QProc$ for the set of
names, and $\id{\{}\vec{y} / \vec{x} \id{\}}$ to denote partial maps,
$s : \QProc \rightarrow \QProc$. A map, $s$ lifts, uniquely, to a map
on process terms, $\widehat{s} : \Proc \rightarrow \Proc$ by the
following equations.

\begin{mathpar}
  (0) \psubstp{Q}{P} := 0 \\
  (R \juxtap S) \psubstp{Q}{P}
  :=    
  (R)\psubstp{Q}{P} \juxtap (S) \psubstp{Q}{P} \\
  (x?(y).R) \psubstp{Q}{P}    
  :=    
  (x)\substp{Q}{P} (z)\concat( (R \psubstn{z}{y}) \psubstp{Q}{P} ) \\
  (\lift{x}{R}) \psubstp{Q}{P}  
  :=
  \lift{(x)\substp{Q}{P}}{ R \psubstp{Q}{P} } \\
%   (\dropn{x})  \psubstp{Q}{P}       
%   := 
%   \left\{ 
%     \begin{array}{ccc} 
%       \dropn{\quotep{Q}} & & x \nameeq \quotep{P} \\
%       \dropn{x} & & otherwise \\
%     \end{array}
%   \right. 
  (\dropn{x})  \psubstp{Q}{P}       
  := 
  \left\{ 
    \begin{array}{ccc} 
      Q & & x \nameeq \quotep{P} \\
      \dropn{x} & & otherwise \\
    \end{array}
  \right.
\end{mathpar}
 

where

\begin{eqnarray}
  (x)\id{\{} \lpquote Q \rpquote / \lpquote P \rpquote \id{\}}            = 
  \left\{ 
    \begin{array}{ccc}
      \lpquote Q \rpquote & & x \nameeq \lpquote P \rpquote \\
      x & & otherwise \\
    \end{array}
  \right. \nonumber
\end{eqnarray}

and $z$ is chosen distinct from $\quotep{P}$, $\quotep{Q}$, the free
names in $Q$, and all the names in $R$. Our $\alpha$-equivalence will
be built in the standard way from this substitution.

\begin{remark}\label{rem:no_self_referential_names}
  One consequence of these definitions is that $\forall P. \quotep{P}
  \not\in \freenames{P}$.
\end{remark}

\subsection{ Dynamic quote: an example }

Anticipating something of what's to come, consider applying the
substitution, $\widehat{\id{\{}u / z \id{\}}}$, to the following pair
of processes, $\lift{w}{y!(z)}$ and $w[ \lpquote y!(z) \rpquote ]$.

\begin{eqnarray}
	\lift{w}{y!(z)}\widehat{\id{\{}u / z \id{\}}}
		& = &
		\lift{w}{y!(u)} \nonumber\\
	w[ \lpquote y!(z) \rpquote ] \widehat{ \id{\{}u / z \id{\}} }
		& = &
		w[ \lpquote y!(z) \rpquote ] \nonumber
\end{eqnarray}

Because the body of the process between quotes is impervious to
substitution, we get radically different answers. In fact, by
examining the first process in an input context,
e.g. $x?(z).\lift{w}{y!(z)}$, we see that the process under the lift
operator may be shaped by prefixed inputs binding a name inside it. In
this sense, the lift operator will be seen as a way to dynamically
construct processes before reifying them as names.

Finally equipped with these standard features we can present the
dynamics of the calculus.

\subsubsection{Operational semantics} 

Finally, we introduce the computational dynamics. What marks these
algebras as distinct from other more traditionally studied algebraic
structures, e.g. vector spaces or polynomial rings, is the manner in
which dynamics is captured. In traditional structures, dynamics is typically
expressed through morphisms between such structures, as in linear maps
between vector spaces or morphisms between rings. In algebras
associated with the semantics of computation, the dynamics is
expressed as part of the algebraic structure itself, through a
reduction reduction relation typically denoted by $\red$. Below, we
give a recursive presentation of this relation for the calculus used
in the encoding.

$\red \subseteq \pi \times \pi$
$\red : \pi \to \mathcal{P}(\pi)$

\begin{mathpar}
  \inferrule* [lab=Comm] { \textsf{match}( x_{src}, x_{trgt} ) } { x_{trgt}?(y)P \; | \; x_{src}!\langle {Q} \rangle \red P\{\quotep{Q}/y}\} }
  \and \\
  \inferrule* [lab=Par] {{P} \red {P}'} {{{P} | {Q}} \red {{P}' | {Q}}}
  \and
  \inferrule* [lab=Equiv]{{{P} \scong {P}'} \andalso {{P}' \red {Q}'} \andalso {{Q}' \scong {Q}}}{{P} \red {Q}}
\end{mathpar}

\begin{eqnarray*}
  match_{\equiv} (\quotep{P},\quotep{Q}) & := & P \equiv Q \\
  match_{\dagger}(\quotep{P},\quotep{Q}) & := & \forall R. P|Q \red^{*} R => R \red^{*} 0 \\
  match_{K}(\quotep{P},\quotep{Q}) & := & K \mbox{ for some context } K
\end{eqnarray*}

$u?(x)P | u!\langle Q \rangle \red P\{\quotep{Q}/x\}$

%We write $\wred$ for $\red^*$, and $P\red$ if $\exists Q $ such that $ P \red Q$.
We write $P\red$ if $\exists Q $ such that $ P \red Q$ and $P\not\red$, otherwise.

\section{Replication}

As mentioned before, it is known that replication (and hence
recursion) can be implemented in a higher-order process algebra
\cite{SangiorgiWalker}. As our first example of calculation with the
machinery thus far presented we give the construction explicitly in
the {\rhoc}.

\begin{eqnarray}
	D_{x} & := & \prefix{x}{y}{(\binpar{\outputp{x}{y}}{@{y}})} \nonumber\\
	\bangp_{x}{P} & := & \binpar{{x}!\langle{\binpar{D_{x}}{P}}\rangle}{D_{x}} \nonumber
\end{eqnarray}

\begin{eqnarray}
	\bangp_{x}{P} & & \nonumber\\
	=
	& {x}!\langle{(\prefix{x}{y}{(\outputp{x}{y} | @{y})) | P}}\rangle 
	      | \prefix{x}{y}{(\outputp{x}{y} | @{y})} & \nonumber\\
	\red
	& (\outputp{x}{y} | @{y})\substn{\quotep{(\prefix{x}{y}{(@{y} | \outputp{x}{y})) | P}}}{y} & \nonumber\\
	=
	& \outputp{x}{\quotep{(\prefix{x}{y}{(\outputp{x}{y} | @{y})) | P}}}
	  | {(\prefix{x}{y}{(\outputp{x}{y} | @{y})) | P}} & \nonumber\\
	\red
	& \ldots & \nonumber\\
	\red^*
	& P | P | \ldots & \nonumber
\end{eqnarray}

Of course, this encoding, as an implementation, runs away, unfolding
$\bangp{P}$ eagerly. A lazier and more implementable replication
operator, restricted to input-guarded processes, may be obtained as follows.

\begin{eqnarray}
\bangp{\prefix{u}{v}{P}} 
	:= 
	\binpar{\lift{x}{\prefix{u}{v}{(\binpar{D(x)}{P})}}}{D(x)} \nonumber
\end{eqnarray}

\begin{remark}
  Note that the lazier definition still does not deal with summation
  or mixed summation (i.e. sums over input and output). The reader is
  invited to construct definitions of replication that deal with these
  features. 

  Further, the definitions are parameterized in a name, $x$. Can you,
  gentle reader, make a definition that eliminates this parameter and
  guarantees no accidental interaction between the replication
  machinery and the process being replicated -- i.e. no accidental
  sharing of names used by the process to get its work done and the
  name(s) used by the replication to effect copying. This latter
  revision of the definition of replication is crucial to obtaining
  the expected identity $!!P \sim !P$.
\end{remark}

\begin{remark}\label{rem:paradoxical_combinator}
  The reader familiar with the lambda calculus will have noticed the
  similarity between $D$ and the paradoxical combinator.

  [Ed. note: the existence of this seems to suggest we have to be more
  restrictive on the set of processes and names we admit if we are to
  support no-cloning.]
\end{remark}

\subsubsection{Bisimulation}

The computational dynamics gives rise to another kind of equivalence,
the equivalence of computational behavior. As previously mentioned
this is typically captured \emph{via} some form of bisimulation.

% The notion we use in this paper is weak barbed bisimulation
% \cite{milner91polyadicpi}.

The notion we use in this paper is derived from weak barbed
bisimulation \cite{milner91polyadicpi}. 

\begin{definition}
An \emph{observation relation}, $\downarrow_{\mathcal N}$, over a set
of names, $\mathcal N$, is the smallest relation satisfying the rules
below.

\infrule[Out-barb]{y \in {\mathcal N}, \; x \nameeq y}
		  {\outputp{x}{v} \downarrow_{\mathcal N} x}
\infrule[Par-barb]{\mbox{$P\downarrow_{\mathcal N} x$ or $Q\downarrow_{\mathcal N} x$}}
		  {\binpar{P}{Q} \downarrow_{\mathcal N} x}

We write $P \Downarrow_{\mathcal N} x$ if there is $Q$ such that 
$P \wred Q$ and $Q \downarrow_{\mathcal N} x$.
\end{definition}

\begin{definition}
%\label{def.bbisim}
An  ${\mathcal N}$-\emph{barbed bisimulation} over a set of names, ${\mathcal N}$, is a symmetric binary relation 
${\mathcal S}_{\mathcal N}$ between agents such that $P\rel{S}_{\mathcal N}Q$ implies:
\begin{enumerate}
\item If $P \red P'$ then $Q \wred Q'$ and $P'\rel{S}_{\mathcal N} Q'$.
\item If $P\downarrow_{\mathcal N} x$, then $Q\Downarrow_{\mathcal N} x$.
\end{enumerate}
$P$ is ${\mathcal N}$-barbed bisimilar to $Q$, written
$P \wbbisim_{\mathcal N} Q$, if $P \rel{S}_{\mathcal N} Q$ for some ${\mathcal N}$-barbed bisimulation ${\mathcal S}_{\mathcal N}$.
\end{definition}

$\mathcal{R} \subseteq \pi \times \pi$

$P \mathcal{R} Q => \forall P'. P \red P' \Rightarrow \exists Q'. Q \red Q', P' \mathcal{R} Q'$

$P \vdash x \Rightarrow Q \vdash x$

\begin{mathpar}
  \inferrule*[lab=Out-barb]{x \nameeq y}{{y}!\langle{Q}\rangle \vdash x}
  \and
  \inferrule*[lab=Par-barb]{\mbox{$P\vdash x$ or $Q\vdash x$}}{\binpar{P}{Q} \vdash x}
\end{mathpar}

\subsubsection{Contexts}

One of the principle advantages of computational calculi like the
$\pi$-calculus is a well-defined notion of context,
contextual-equivalence and a correlation between
contextual-equivalence and notions of bisimulation. The notion of
context allows the decomposition of a process into (sub-)process and
its syntactic environment, its context. Thus, a context may be
thought of as a process with a ``hole'' (written $\Box$) in it. The
application of a context $M$ to a process $P$, written $M[P]$, is
tantamount to filling the hole in $M$ with $P$. In this paper we do
not need the full weight of this theory, but do make use of the notion
of context in the proof the main theorem. 

\begin{mathpar}
  \inferrule* [lab=summation] {} {{M_{M},M_{N}} \bc \Box \;|\; x.M_{A} \;|\; M_{M}+M_{N}}
  \and
  \inferrule* [lab=agent] {} {{M_{A}} \bc (\vec{x})M_{P} \;| \; \clift{P_0,\ldots,M_{P},\ldots,P_N}}
  \and \\
  \inferrule* [lab=process] {} {{M_{P}} \bc M_{N} \;| \;P|M_{P} }
\end{mathpar} 

\begin{mathpar}
  \inferrule* [lab=sychronization] {} {M_{N} \bc \Box \;|\; x?M_{F} \;|\; x!M_{C}}
  \and
  \inferrule* [lab=abstraction] {} {{M_{F}} \bc (x)M_{P} }
  \and
  \inferrule* [lab=concretion] {} {{M_{C}} \bc \langle M_{P} \rangle }
  \and \\
  \inferrule* [lab=process] {} {{M_{P}} \bc M_{N} \;| \;P|M_{P} }
\end{mathpar}

\begin{definition}[contextual application] Given a context $M$, and
  process $P$, we define the \emph{contextual application}, $M[P] :=
  M\{P/\Box\}$. That is, the contextual application of M to P is the
  substitution of $P$ for $\Box$ in $M$.
\end{definition}

$\meaningof{-} : L \to \mathcal{P}(\pi)$

\begin{mathpar}
  \inferrule* [lab=collection] {} {\meaningof{true} = \pi, \and \meaningof{~E} = \pi \setminus \meaningof{E}, \and \meaningof{E_{1} \& E_{2}} = \meaningof{E_{1}} \cap \meaningof{E_{2}}}
\end{mathpar}

\begin{mathpar}
  \inferrule* [lab=structure] {} {\meaningof{0} = \{ P \in \pi | P \equiv 0 \}, \and \\ \meaningof{E_1 | E_2} = \{ P \in \pi | P \equiv P_{1} | P_{2}, P_{1} \in \meaningof{E_{1}}, P_{2} \in \meaningof{E_2}\} }
\end{mathpar}

\begin{mathpar}
 \inferrule* [lab=behavior] {} {\meaningof{\langle a?b \rangle E} = \{ P \in \pi | P \equiv Q | u?(y)P', \\ \and \\\\ \and \\ \;\;\; u \in \meaningof{a}, \forall z.P'\{z/y\} \in \meaningof{E\{z/b\}}\}, \and \\ \meaningof{a!E} = \{ P \in \pi | P \equiv Q | x!\langle P' \rangle, x \in \meaningof{a} P' \in \meaningof{E}\} }
\end{mathpar}

\begin{mathpar}
 \inferrule* [lab=nominal] {} {\meaningof{\quotep{E}} = \{ \quotep{P} \in \quotep{\pi} | P \in \meaningof{E} \}, \and \meaningof{\quotep{P}} = \{ \quotep{Q} \in \quotep{\pi} | P \equiv Q \} \and \\ \meaningof{@\quotep{E}} = \{ P \in \pi | P \equiv @x, x \in \meaningof{E} \}}
\end{mathpar}

\begin{eqnarray*}
  \\
  \meaningof{-} : TS \to ST
\end{eqnarray*}

\begin{eqnarray*}
  \\
  L : TS \to ST
\end{eqnarray*}

\begin{eqnarray*}
  \\
  P \models E \iff P \in \meaningof{E}
\end{eqnarray*}

\begin{eqnarray*}
  P \approx_{L} Q \iff \forall E \in L. P \models E \iff Q \models E
\end{eqnarray*}

\begin{eqnarray*}
  P \approx_{K} Q
\end{eqnarray*}

\begin{eqnarray*}
  P \approx Q
\end{eqnarray*}

$\approx_{K} = \approx = \approx_{L}$

\subsubsection{Contextual duality}

Note that contexts extend the quotation operation to a family of
operations from processes to names. Given a context, $M$, we can
define a \emph{nominal context}, $\quotep{M}$ by $\quotep{M}[P] :=
\quotep{M[P]}$. To foreshadow what is to come we observe that these
operations enjoy a duality with processes very much like the duality
between vectors and maps from vectors to scalars.

Further, because the calculus is essentially higher-order, we have a
correspondence between contexts and processes. More specifically,
given a name $x$ and a context $M$ we can construct $M^{*}_{x}$ such
that 

\begin{mathpar}
  M^{*}_{x} | \lift{x}{P} \red M[P]
\end{mathpar}

namely,

\begin{mathpar}
  M^{*}_{x} := x?(u).M[\dropn{u}]
\end{mathpar}

The dependence of $M^{*}_{x}$ on a name makes it an abstraction, 

\begin{mathpar}
  M^{*} := (x)x?(u).M[\dropn{u}]
\end{mathpar}

\subsection{Additional notation}

It will sometimes be convenient to denote the process a name
quotes. We already have the notation $x = \quotep{P}$, but it will be
convenient to introduce an alternate notation, $\procn{x}$, when we
want to emphasize the connection to the use of the name. Note that, by
virtue of name equivalence, $\quotep{\procn{x}} \nameeq x$; so, the
notation is consistent with previous definitions.

Further, because names have structure it is possible to effect
substitutions on the basis of that structure. This means we need to
upgrade our notation for substitutions, which we accomplish by
adapting comprehension notation. Thus,

\begin{mathpar}
  P\{ y / x : x \in S \}
\end{mathpar}

is interpreted to mean the process derived from P by replacing (in a
capture-avoiding manner) each occurrence of $x$ in $S$ by $y$. For example,

\begin{mathpar}
  P\{ \quotep{\procn{x}|\procn{x}} / x : x \in \freenames{P} \}
\end{mathpar}

will replace each (occurrence) of a free name $x$ in $P$ by
$\quotep{\procn{x}|\procn{x}}$.

Also, we will avail ourselves of the notation $x^{L}$ and $x^{R}$ to
denote injections of a name into disjoint copies of the name
space. There are numerous ways to accomplish this. One example can be
found in \cite{MeredithR05}. This notation overloads to vectors of
names: $\vec{x}^{\pi} := (x_{i}^{\pi} \; : \; 0 \leq i < |\vec{x}| )$ where $\pi \in \{L,R\}$.

We also use $P^{\Box} := P|\Box$.

In \cite{MeredithR05} an interpretation of the new operator is
given. It turns out that there are several possible interpretations
all enjoying the requisite algebraic properties of the operator (see
\cite{milner91polyadicpi}). We will therefore make liberal use of
$(\nu\; \vec{x})P$.

% subsection the_syntax_and_semantics_of_the_notation_system (end)   

\input{qm2pi.qmops} 

\input{qm2pi.sterngerlach} 

\input{qm2pi.metric} 

% section concurrent_process_calculi (end)

%\input{qm2pi.proofsketch}

% section proof sketch (end)

%\input{qm2pi.slviaknots} 

% section spatial logic via knots (end)

\input{qm2pi.conclusion}

% section conclusion (end)

%\input{qm2pi.dtcodes} 

% section wiring algorithm (end)

\input{qm2pi.ack} 

% section acknowledgments (end)

\newpage


\bibliographystyle{plain}   
\bibliography{../../biblios/main.bib}

\input{qm2pi.rhodetails}

\end{document}

 

% section notation (end)

\input{qm2pi.process.calculi} 

% section concurrent_process_calculi_and_spatial_logics_ (end)
    
%\documentclass[12pt]{llncs}
%\documentclass{jktr}

\usepackage[pdftex]{hyperref}                   
\usepackage {listings}
\usepackage {mathpartir}
\usepackage{bcprules}
%\usepackage{listings}
                       
\usepackage{graphicx} 
%\usepackage[margins=2.5cm,nohead,nofoot]{geometry}
%\usepackage{geometry}
\usepackage{amsfonts}
\usepackage{amstext}
\usepackage{latexsym}
\usepackage{amssymb}
\usepackage{color}


%\include{myPreamble}
\include{qm2pi.local} 

%\ifpdf
%\usepackage[pdftex]{graphicx}
%\else
%\usepackage{graphicx}
%\fi

 % \ifpdf
%  \usepackage{pdfsync}
%  \if


%\title{Brief Article}
%\author{David F. Snyder}
%\author{L.G. Meredith}

%\address{Dept. of Math., Texas State University--San Marcos, San Marcos, TX 78666}
       
\pagestyle{empty}


\begin{document}

\lstset{language=[Objective]Caml,frame=shadowbox}

\input{qm2pi.front}

% section front matter (end)

\input{qm2pi.intro} 
 
% section introduction (end)

% \input{qm2pi.knotations} 

% section notation (end)

\input{qm2pi.process.calculi} 

% section concurrent_process_calculi_and_spatial_logics_ (end)
    
%\input{qm2pi.knots2pi} 

%\input{qm2pi.trefoil} 

%\input{qm2pi.mainthm} 

% subsection basic_interpretation (end)

%\input{qm2pi.rho.presentation} 
\subsection{The syntax and semantics of the notation system}\label{sub:the_syntax_and_semantics_of_the_notation_system} % (fold)

We now summarize a technical presentation of the calculus that
embodies our theory of dynamics. The typical presentation of such a
calculus follows the style of giving generators and relations on
them. The grammar, below, describing term constructors, freely
generates the set of processes, $\Proc$. This set is then quotiented
by a relation known as structural congruence and it is over this set
that the notion of dynamics is expressed. This presentation is
essentially that of \cite{MeredithR05} with the addition of
polyadicity and summation. For readability we have relegated some of
the technical subtleties to an appendix.

\subsubsection{Process grammar}\label{subsub:process_grammar}

\begin{mathpar}
  \inferrule* [lab=synchronization] {} {{M} \bc \pzero \;|\; x?F \;|\; x!C }
  \and
  \inferrule* [lab=abstraction] {} {{F} \bc (x)P}
  \and
  \inferrule* [lab=concretion] {} {{C} \bc \langle Q \rangle}
  \and
  \inferrule* [lab=process] {} {{P,Q} \bc M \;| \;P|Q \;|\; @{x}}
  \and
  \inferrule* [lab=name] {} {{x} \bc \quotep{P}}
\end{mathpar} 

Note that $\vec{x}$ (resp. $\vec{P}$) denotes a vector of names
(resp. processes) of length $|\vec{x}|$ (resp. $|\vec{P}|$). We adopt
the following useful abbreviations.

\begin{mathpar}
   x?(\vec{y}).P := x.(\vec{y})P \and  x\clift{\vec{P}} := x.\clift{\vec{P}}
   \and x!(y) := \lift{x}{\dropn{y}}
   \and \Pi_{i=0}^{n-1}P_i := P_0 | \ldots | P_{n-1}
\end{mathpar}

\subsubsection{Structural congruence}

\paragraph{Free and bound names and alpha-equivalence.} At the
core of structural equivalence is alpha-equivalence which identifies
process that are the same up to a change of variable. Formally, we
recognize the distinction between free and bound names. The free names
of a process, $\freenames{P}$, may be calculated recursively as
follows:

\begin{mathpar}
\freenames{\pzero} := \emptyset
  \and \\
  \freenames{x?(y).P} := \{ x \} \cup (\freenames{P} \setminus \{ y \})
  \and 
  \freenames{x!\langle P \rangle} := \{ x \} \cup \{ P \} 
  \and \\
  \freenames{P|Q} := \freenames{P} \cup \freenames{Q}
  \and \\
  \freenames{@{x}} := \{ x \}
\end{mathpar}

$\pi$
$\quotep{\pi}$

$\freenames{-} : \pi \to \mathcal{P}(\quotep{\pi})$

\begin{eqnarray*}
  \freenames{\pzero} & := & \emptyset \\
  \freenames{x?(y).P} & := & \{ x \} \cup (\freenames{P} \setminus \{ y \}) \\
  \freenames{x!\langle P \rangle} & := & \{ x \} \cup \{ P \} \\
  \freenames{P|Q} & := & \freenames{P} \cup \freenames{Q} \\
  \freenames{\dropn{x}} & := & \{ x \}
\end{eqnarray*}

The bound names of a process, $\boundnames{P}$, are those names occurring in $P$
that are not free. For example, in $x?(y).0$, the name $x$ is free, while $y$ is bound.

\begin{mathpar}
  \inferrule* [lab=monoidal-laws] {} { P|Q \equiv Q|P \and P|0 \equiv P \and P|(Q|R) \equiv (P|Q)|R }
\end{mathpar}

\begin{mathpar}
  \inferrule* [lab=alpha-equivalence] {} { (x)P \equiv (y)P\{y/x\} \and y \not\in \freenames{P} }
\end{mathpar}

\begin{definition}
Then two processes, $P,Q$, are alpha-equivalent if $P = Q\{\vec{y}/\vec{x}\}$ for
some $\vec{x} \in \boundnames{Q},\vec{y} \in \boundnames{P}$, where $Q\{\vec{y}/\vec{x}\}$
denotes the capture-avoiding substitution of $\vec{y}$ for $\vec{x}$ in $Q$.
\end{definition}

\begin{definition}
  The {\em structural congruence} \cite{SangiorgiWalker} , $\equiv$,
  between processes is the least congruence containing
  alpha-equivalence, satisfying the abelian monoid laws
  (associativity, commutativity and $\pzero$ as identity) for parallel
  composition $|$ and for summation $+$.
\end{definition}

\subsection{Name equivalence}

We take name equivalence, written $\nameeq$, to be the smallest
equivalence relation generated by the following rules.

\begin{mathpar}
\inferrule*[lab=Quote-drop]
{ }
{ \quotep{@{x}} \nameeq x }

\inferrule*[lab=Struct-equiv]
{ P \scong Q }
{ \quotep{P} \nameeq \quotep{Q} }
\end{mathpar}

The astute reader will have noticed that the mutual recursion of names
and processes imposes a mutual recursion on alpha-equivalence and
structural equivalence via name-equivalence. Fortunately, all of this
works out pleasantly and we may calculate in the natural way, free of
concern. The reader interested in the details is referred to the
appendix \ref{appendix:rho_details}.

\subsection{Substitution}

We use $\Proc$ for the set of processes, $\QProc$ for the set of
names, and $\id{\{}\vec{y} / \vec{x} \id{\}}$ to denote partial maps,
$s : \QProc \rightarrow \QProc$. A map, $s$ lifts, uniquely, to a map
on process terms, $\widehat{s} : \Proc \rightarrow \Proc$ by the
following equations.

\begin{mathpar}
  (0) \psubstp{Q}{P} := 0 \\
  (R \juxtap S) \psubstp{Q}{P}
  :=    
  (R)\psubstp{Q}{P} \juxtap (S) \psubstp{Q}{P} \\
  (x?(y).R) \psubstp{Q}{P}    
  :=    
  (x)\substp{Q}{P} (z)\concat( (R \psubstn{z}{y}) \psubstp{Q}{P} ) \\
  (\lift{x}{R}) \psubstp{Q}{P}  
  :=
  \lift{(x)\substp{Q}{P}}{ R \psubstp{Q}{P} } \\
%   (\dropn{x})  \psubstp{Q}{P}       
%   := 
%   \left\{ 
%     \begin{array}{ccc} 
%       \dropn{\quotep{Q}} & & x \nameeq \quotep{P} \\
%       \dropn{x} & & otherwise \\
%     \end{array}
%   \right. 
  (\dropn{x})  \psubstp{Q}{P}       
  := 
  \left\{ 
    \begin{array}{ccc} 
      Q & & x \nameeq \quotep{P} \\
      \dropn{x} & & otherwise \\
    \end{array}
  \right.
\end{mathpar}
 

where

\begin{eqnarray}
  (x)\id{\{} \lpquote Q \rpquote / \lpquote P \rpquote \id{\}}            = 
  \left\{ 
    \begin{array}{ccc}
      \lpquote Q \rpquote & & x \nameeq \lpquote P \rpquote \\
      x & & otherwise \\
    \end{array}
  \right. \nonumber
\end{eqnarray}

and $z$ is chosen distinct from $\quotep{P}$, $\quotep{Q}$, the free
names in $Q$, and all the names in $R$. Our $\alpha$-equivalence will
be built in the standard way from this substitution.

\begin{remark}\label{rem:no_self_referential_names}
  One consequence of these definitions is that $\forall P. \quotep{P}
  \not\in \freenames{P}$.
\end{remark}

\subsection{ Dynamic quote: an example }

Anticipating something of what's to come, consider applying the
substitution, $\widehat{\id{\{}u / z \id{\}}}$, to the following pair
of processes, $\lift{w}{y!(z)}$ and $w[ \lpquote y!(z) \rpquote ]$.

\begin{eqnarray}
	\lift{w}{y!(z)}\widehat{\id{\{}u / z \id{\}}}
		& = &
		\lift{w}{y!(u)} \nonumber\\
	w[ \lpquote y!(z) \rpquote ] \widehat{ \id{\{}u / z \id{\}} }
		& = &
		w[ \lpquote y!(z) \rpquote ] \nonumber
\end{eqnarray}

Because the body of the process between quotes is impervious to
substitution, we get radically different answers. In fact, by
examining the first process in an input context,
e.g. $x?(z).\lift{w}{y!(z)}$, we see that the process under the lift
operator may be shaped by prefixed inputs binding a name inside it. In
this sense, the lift operator will be seen as a way to dynamically
construct processes before reifying them as names.

Finally equipped with these standard features we can present the
dynamics of the calculus.

\subsubsection{Operational semantics} 

Finally, we introduce the computational dynamics. What marks these
algebras as distinct from other more traditionally studied algebraic
structures, e.g. vector spaces or polynomial rings, is the manner in
which dynamics is captured. In traditional structures, dynamics is typically
expressed through morphisms between such structures, as in linear maps
between vector spaces or morphisms between rings. In algebras
associated with the semantics of computation, the dynamics is
expressed as part of the algebraic structure itself, through a
reduction reduction relation typically denoted by $\red$. Below, we
give a recursive presentation of this relation for the calculus used
in the encoding.

$\red \subseteq \pi \times \pi$
$\red : \pi \to \mathcal{P}(\pi)$

\begin{mathpar}
  \inferrule* [lab=Comm] { \textsf{match}( x_{src}, x_{trgt} ) } { x_{trgt}?(y)P \; | \; x_{src}!\langle {Q} \rangle \red P\{\quotep{Q}/y}\} }
  \and \\
  \inferrule* [lab=Par] {{P} \red {P}'} {{{P} | {Q}} \red {{P}' | {Q}}}
  \and
  \inferrule* [lab=Equiv]{{{P} \scong {P}'} \andalso {{P}' \red {Q}'} \andalso {{Q}' \scong {Q}}}{{P} \red {Q}}
\end{mathpar}

\begin{eqnarray*}
  match_{\equiv} (\quotep{P},\quotep{Q}) & := & P \equiv Q \\
  match_{\dagger}(\quotep{P},\quotep{Q}) & := & \forall R. P|Q \red^{*} R => R \red^{*} 0 \\
  match_{K}(\quotep{P},\quotep{Q}) & := & K \mbox{ for some context } K
\end{eqnarray*}

$u?(x)P | u!\langle Q \rangle \red P\{\quotep{Q}/x\}$

%We write $\wred$ for $\red^*$, and $P\red$ if $\exists Q $ such that $ P \red Q$.
We write $P\red$ if $\exists Q $ such that $ P \red Q$ and $P\not\red$, otherwise.

\section{Replication}

As mentioned before, it is known that replication (and hence
recursion) can be implemented in a higher-order process algebra
\cite{SangiorgiWalker}. As our first example of calculation with the
machinery thus far presented we give the construction explicitly in
the {\rhoc}.

\begin{eqnarray}
	D_{x} & := & \prefix{x}{y}{(\binpar{\outputp{x}{y}}{@{y}})} \nonumber\\
	\bangp_{x}{P} & := & \binpar{{x}!\langle{\binpar{D_{x}}{P}}\rangle}{D_{x}} \nonumber
\end{eqnarray}

\begin{eqnarray}
	\bangp_{x}{P} & & \nonumber\\
	=
	& {x}!\langle{(\prefix{x}{y}{(\outputp{x}{y} | @{y})) | P}}\rangle 
	      | \prefix{x}{y}{(\outputp{x}{y} | @{y})} & \nonumber\\
	\red
	& (\outputp{x}{y} | @{y})\substn{\quotep{(\prefix{x}{y}{(@{y} | \outputp{x}{y})) | P}}}{y} & \nonumber\\
	=
	& \outputp{x}{\quotep{(\prefix{x}{y}{(\outputp{x}{y} | @{y})) | P}}}
	  | {(\prefix{x}{y}{(\outputp{x}{y} | @{y})) | P}} & \nonumber\\
	\red
	& \ldots & \nonumber\\
	\red^*
	& P | P | \ldots & \nonumber
\end{eqnarray}

Of course, this encoding, as an implementation, runs away, unfolding
$\bangp{P}$ eagerly. A lazier and more implementable replication
operator, restricted to input-guarded processes, may be obtained as follows.

\begin{eqnarray}
\bangp{\prefix{u}{v}{P}} 
	:= 
	\binpar{\lift{x}{\prefix{u}{v}{(\binpar{D(x)}{P})}}}{D(x)} \nonumber
\end{eqnarray}

\begin{remark}
  Note that the lazier definition still does not deal with summation
  or mixed summation (i.e. sums over input and output). The reader is
  invited to construct definitions of replication that deal with these
  features. 

  Further, the definitions are parameterized in a name, $x$. Can you,
  gentle reader, make a definition that eliminates this parameter and
  guarantees no accidental interaction between the replication
  machinery and the process being replicated -- i.e. no accidental
  sharing of names used by the process to get its work done and the
  name(s) used by the replication to effect copying. This latter
  revision of the definition of replication is crucial to obtaining
  the expected identity $!!P \sim !P$.
\end{remark}

\begin{remark}\label{rem:paradoxical_combinator}
  The reader familiar with the lambda calculus will have noticed the
  similarity between $D$ and the paradoxical combinator.

  [Ed. note: the existence of this seems to suggest we have to be more
  restrictive on the set of processes and names we admit if we are to
  support no-cloning.]
\end{remark}

\subsubsection{Bisimulation}

The computational dynamics gives rise to another kind of equivalence,
the equivalence of computational behavior. As previously mentioned
this is typically captured \emph{via} some form of bisimulation.

% The notion we use in this paper is weak barbed bisimulation
% \cite{milner91polyadicpi}.

The notion we use in this paper is derived from weak barbed
bisimulation \cite{milner91polyadicpi}. 

\begin{definition}
An \emph{observation relation}, $\downarrow_{\mathcal N}$, over a set
of names, $\mathcal N$, is the smallest relation satisfying the rules
below.

\infrule[Out-barb]{y \in {\mathcal N}, \; x \nameeq y}
		  {\outputp{x}{v} \downarrow_{\mathcal N} x}
\infrule[Par-barb]{\mbox{$P\downarrow_{\mathcal N} x$ or $Q\downarrow_{\mathcal N} x$}}
		  {\binpar{P}{Q} \downarrow_{\mathcal N} x}

We write $P \Downarrow_{\mathcal N} x$ if there is $Q$ such that 
$P \wred Q$ and $Q \downarrow_{\mathcal N} x$.
\end{definition}

\begin{definition}
%\label{def.bbisim}
An  ${\mathcal N}$-\emph{barbed bisimulation} over a set of names, ${\mathcal N}$, is a symmetric binary relation 
${\mathcal S}_{\mathcal N}$ between agents such that $P\rel{S}_{\mathcal N}Q$ implies:
\begin{enumerate}
\item If $P \red P'$ then $Q \wred Q'$ and $P'\rel{S}_{\mathcal N} Q'$.
\item If $P\downarrow_{\mathcal N} x$, then $Q\Downarrow_{\mathcal N} x$.
\end{enumerate}
$P$ is ${\mathcal N}$-barbed bisimilar to $Q$, written
$P \wbbisim_{\mathcal N} Q$, if $P \rel{S}_{\mathcal N} Q$ for some ${\mathcal N}$-barbed bisimulation ${\mathcal S}_{\mathcal N}$.
\end{definition}

$\mathcal{R} \subseteq \pi \times \pi$

$P \mathcal{R} Q => \forall P'. P \red P' \Rightarrow \exists Q'. Q \red Q', P' \mathcal{R} Q'$

$P \vdash x \Rightarrow Q \vdash x$

\begin{mathpar}
  \inferrule*[lab=Out-barb]{x \nameeq y}{{y}!\langle{Q}\rangle \vdash x}
  \and
  \inferrule*[lab=Par-barb]{\mbox{$P\vdash x$ or $Q\vdash x$}}{\binpar{P}{Q} \vdash x}
\end{mathpar}

\subsubsection{Contexts}

One of the principle advantages of computational calculi like the
$\pi$-calculus is a well-defined notion of context,
contextual-equivalence and a correlation between
contextual-equivalence and notions of bisimulation. The notion of
context allows the decomposition of a process into (sub-)process and
its syntactic environment, its context. Thus, a context may be
thought of as a process with a ``hole'' (written $\Box$) in it. The
application of a context $M$ to a process $P$, written $M[P]$, is
tantamount to filling the hole in $M$ with $P$. In this paper we do
not need the full weight of this theory, but do make use of the notion
of context in the proof the main theorem. 

\begin{mathpar}
  \inferrule* [lab=summation] {} {{M_{M},M_{N}} \bc \Box \;|\; x.M_{A} \;|\; M_{M}+M_{N}}
  \and
  \inferrule* [lab=agent] {} {{M_{A}} \bc (\vec{x})M_{P} \;| \; \clift{P_0,\ldots,M_{P},\ldots,P_N}}
  \and \\
  \inferrule* [lab=process] {} {{M_{P}} \bc M_{N} \;| \;P|M_{P} }
\end{mathpar} 

\begin{mathpar}
  \inferrule* [lab=sychronization] {} {M_{N} \bc \Box \;|\; x?M_{F} \;|\; x!M_{C}}
  \and
  \inferrule* [lab=abstraction] {} {{M_{F}} \bc (x)M_{P} }
  \and
  \inferrule* [lab=concretion] {} {{M_{C}} \bc \langle M_{P} \rangle }
  \and \\
  \inferrule* [lab=process] {} {{M_{P}} \bc M_{N} \;| \;P|M_{P} }
\end{mathpar}

\begin{definition}[contextual application] Given a context $M$, and
  process $P$, we define the \emph{contextual application}, $M[P] :=
  M\{P/\Box\}$. That is, the contextual application of M to P is the
  substitution of $P$ for $\Box$ in $M$.
\end{definition}

$\meaningof{-} : L \to \mathcal{P}(\pi)$

\begin{mathpar}
  \inferrule* [lab=collection] {} {\meaningof{true} = \pi, \and \meaningof{~E} = \pi \setminus \meaningof{E}, \and \meaningof{E_{1} \& E_{2}} = \meaningof{E_{1}} \cap \meaningof{E_{2}}}
\end{mathpar}

\begin{mathpar}
  \inferrule* [lab=structure] {} {\meaningof{0} = \{ P \in \pi | P \equiv 0 \}, \and \\ \meaningof{E_1 | E_2} = \{ P \in \pi | P \equiv P_{1} | P_{2}, P_{1} \in \meaningof{E_{1}}, P_{2} \in \meaningof{E_2}\} }
\end{mathpar}

\begin{mathpar}
 \inferrule* [lab=behavior] {} {\meaningof{\langle a?b \rangle E} = \{ P \in \pi | P \equiv Q | u?(y)P', \\ \and \\\\ \and \\ \;\;\; u \in \meaningof{a}, \forall z.P'\{z/y\} \in \meaningof{E\{z/b\}}\}, \and \\ \meaningof{a!E} = \{ P \in \pi | P \equiv Q | x!\langle P' \rangle, x \in \meaningof{a} P' \in \meaningof{E}\} }
\end{mathpar}

\begin{mathpar}
 \inferrule* [lab=nominal] {} {\meaningof{\quotep{E}} = \{ \quotep{P} \in \quotep{\pi} | P \in \meaningof{E} \}, \and \meaningof{\quotep{P}} = \{ \quotep{Q} \in \quotep{\pi} | P \equiv Q \} \and \\ \meaningof{@\quotep{E}} = \{ P \in \pi | P \equiv @x, x \in \meaningof{E} \}}
\end{mathpar}

\begin{eqnarray*}
  \\
  \meaningof{-} : TS \to ST
\end{eqnarray*}

\begin{eqnarray*}
  \\
  L : TS \to ST
\end{eqnarray*}

\begin{eqnarray*}
  \\
  P \models E \iff P \in \meaningof{E}
\end{eqnarray*}

\begin{eqnarray*}
  P \approx_{L} Q \iff \forall E \in L. P \models E \iff Q \models E
\end{eqnarray*}

\begin{eqnarray*}
  P \approx_{K} Q
\end{eqnarray*}

\begin{eqnarray*}
  P \approx Q
\end{eqnarray*}

$\approx_{K} = \approx = \approx_{L}$

\subsubsection{Contextual duality}

Note that contexts extend the quotation operation to a family of
operations from processes to names. Given a context, $M$, we can
define a \emph{nominal context}, $\quotep{M}$ by $\quotep{M}[P] :=
\quotep{M[P]}$. To foreshadow what is to come we observe that these
operations enjoy a duality with processes very much like the duality
between vectors and maps from vectors to scalars.

Further, because the calculus is essentially higher-order, we have a
correspondence between contexts and processes. More specifically,
given a name $x$ and a context $M$ we can construct $M^{*}_{x}$ such
that 

\begin{mathpar}
  M^{*}_{x} | \lift{x}{P} \red M[P]
\end{mathpar}

namely,

\begin{mathpar}
  M^{*}_{x} := x?(u).M[\dropn{u}]
\end{mathpar}

The dependence of $M^{*}_{x}$ on a name makes it an abstraction, 

\begin{mathpar}
  M^{*} := (x)x?(u).M[\dropn{u}]
\end{mathpar}

\subsection{Additional notation}

It will sometimes be convenient to denote the process a name
quotes. We already have the notation $x = \quotep{P}$, but it will be
convenient to introduce an alternate notation, $\procn{x}$, when we
want to emphasize the connection to the use of the name. Note that, by
virtue of name equivalence, $\quotep{\procn{x}} \nameeq x$; so, the
notation is consistent with previous definitions.

Further, because names have structure it is possible to effect
substitutions on the basis of that structure. This means we need to
upgrade our notation for substitutions, which we accomplish by
adapting comprehension notation. Thus,

\begin{mathpar}
  P\{ y / x : x \in S \}
\end{mathpar}

is interpreted to mean the process derived from P by replacing (in a
capture-avoiding manner) each occurrence of $x$ in $S$ by $y$. For example,

\begin{mathpar}
  P\{ \quotep{\procn{x}|\procn{x}} / x : x \in \freenames{P} \}
\end{mathpar}

will replace each (occurrence) of a free name $x$ in $P$ by
$\quotep{\procn{x}|\procn{x}}$.

Also, we will avail ourselves of the notation $x^{L}$ and $x^{R}$ to
denote injections of a name into disjoint copies of the name
space. There are numerous ways to accomplish this. One example can be
found in \cite{MeredithR05}. This notation overloads to vectors of
names: $\vec{x}^{\pi} := (x_{i}^{\pi} \; : \; 0 \leq i < |\vec{x}| )$ where $\pi \in \{L,R\}$.

We also use $P^{\Box} := P|\Box$.

In \cite{MeredithR05} an interpretation of the new operator is
given. It turns out that there are several possible interpretations
all enjoying the requisite algebraic properties of the operator (see
\cite{milner91polyadicpi}). We will therefore make liberal use of
$(\nu\; \vec{x})P$.

% subsection the_syntax_and_semantics_of_the_notation_system (end)   

\input{qm2pi.qmops} 

\input{qm2pi.sterngerlach} 

\input{qm2pi.metric} 

% section concurrent_process_calculi (end)

%\input{qm2pi.proofsketch}

% section proof sketch (end)

%\input{qm2pi.slviaknots} 

% section spatial logic via knots (end)

\input{qm2pi.conclusion}

% section conclusion (end)

%\input{qm2pi.dtcodes} 

% section wiring algorithm (end)

\input{qm2pi.ack} 

% section acknowledgments (end)

\newpage


\bibliographystyle{plain}   
\bibliography{../../biblios/main.bib}

\input{qm2pi.rhodetails}

\end{document}

 

%\documentclass[12pt]{llncs}
%\documentclass{jktr}

\usepackage[pdftex]{hyperref}                   
\usepackage {listings}
\usepackage {mathpartir}
\usepackage{bcprules}
%\usepackage{listings}
                       
\usepackage{graphicx} 
%\usepackage[margins=2.5cm,nohead,nofoot]{geometry}
%\usepackage{geometry}
\usepackage{amsfonts}
\usepackage{amstext}
\usepackage{latexsym}
\usepackage{amssymb}
\usepackage{color}


%\include{myPreamble}
\include{qm2pi.local} 

%\ifpdf
%\usepackage[pdftex]{graphicx}
%\else
%\usepackage{graphicx}
%\fi

 % \ifpdf
%  \usepackage{pdfsync}
%  \if


%\title{Brief Article}
%\author{David F. Snyder}
%\author{L.G. Meredith}

%\address{Dept. of Math., Texas State University--San Marcos, San Marcos, TX 78666}
       
\pagestyle{empty}


\begin{document}

\lstset{language=[Objective]Caml,frame=shadowbox}

\input{qm2pi.front}

% section front matter (end)

\input{qm2pi.intro} 
 
% section introduction (end)

% \input{qm2pi.knotations} 

% section notation (end)

\input{qm2pi.process.calculi} 

% section concurrent_process_calculi_and_spatial_logics_ (end)
    
%\input{qm2pi.knots2pi} 

%\input{qm2pi.trefoil} 

%\input{qm2pi.mainthm} 

% subsection basic_interpretation (end)

%\input{qm2pi.rho.presentation} 
\subsection{The syntax and semantics of the notation system}\label{sub:the_syntax_and_semantics_of_the_notation_system} % (fold)

We now summarize a technical presentation of the calculus that
embodies our theory of dynamics. The typical presentation of such a
calculus follows the style of giving generators and relations on
them. The grammar, below, describing term constructors, freely
generates the set of processes, $\Proc$. This set is then quotiented
by a relation known as structural congruence and it is over this set
that the notion of dynamics is expressed. This presentation is
essentially that of \cite{MeredithR05} with the addition of
polyadicity and summation. For readability we have relegated some of
the technical subtleties to an appendix.

\subsubsection{Process grammar}\label{subsub:process_grammar}

\begin{mathpar}
  \inferrule* [lab=synchronization] {} {{M} \bc \pzero \;|\; x?F \;|\; x!C }
  \and
  \inferrule* [lab=abstraction] {} {{F} \bc (x)P}
  \and
  \inferrule* [lab=concretion] {} {{C} \bc \langle Q \rangle}
  \and
  \inferrule* [lab=process] {} {{P,Q} \bc M \;| \;P|Q \;|\; @{x}}
  \and
  \inferrule* [lab=name] {} {{x} \bc \quotep{P}}
\end{mathpar} 

Note that $\vec{x}$ (resp. $\vec{P}$) denotes a vector of names
(resp. processes) of length $|\vec{x}|$ (resp. $|\vec{P}|$). We adopt
the following useful abbreviations.

\begin{mathpar}
   x?(\vec{y}).P := x.(\vec{y})P \and  x\clift{\vec{P}} := x.\clift{\vec{P}}
   \and x!(y) := \lift{x}{\dropn{y}}
   \and \Pi_{i=0}^{n-1}P_i := P_0 | \ldots | P_{n-1}
\end{mathpar}

\subsubsection{Structural congruence}

\paragraph{Free and bound names and alpha-equivalence.} At the
core of structural equivalence is alpha-equivalence which identifies
process that are the same up to a change of variable. Formally, we
recognize the distinction between free and bound names. The free names
of a process, $\freenames{P}$, may be calculated recursively as
follows:

\begin{mathpar}
\freenames{\pzero} := \emptyset
  \and \\
  \freenames{x?(y).P} := \{ x \} \cup (\freenames{P} \setminus \{ y \})
  \and 
  \freenames{x!\langle P \rangle} := \{ x \} \cup \{ P \} 
  \and \\
  \freenames{P|Q} := \freenames{P} \cup \freenames{Q}
  \and \\
  \freenames{@{x}} := \{ x \}
\end{mathpar}

$\pi$
$\quotep{\pi}$

$\freenames{-} : \pi \to \mathcal{P}(\quotep{\pi})$

\begin{eqnarray*}
  \freenames{\pzero} & := & \emptyset \\
  \freenames{x?(y).P} & := & \{ x \} \cup (\freenames{P} \setminus \{ y \}) \\
  \freenames{x!\langle P \rangle} & := & \{ x \} \cup \{ P \} \\
  \freenames{P|Q} & := & \freenames{P} \cup \freenames{Q} \\
  \freenames{\dropn{x}} & := & \{ x \}
\end{eqnarray*}

The bound names of a process, $\boundnames{P}$, are those names occurring in $P$
that are not free. For example, in $x?(y).0$, the name $x$ is free, while $y$ is bound.

\begin{mathpar}
  \inferrule* [lab=monoidal-laws] {} { P|Q \equiv Q|P \and P|0 \equiv P \and P|(Q|R) \equiv (P|Q)|R }
\end{mathpar}

\begin{mathpar}
  \inferrule* [lab=alpha-equivalence] {} { (x)P \equiv (y)P\{y/x\} \and y \not\in \freenames{P} }
\end{mathpar}

\begin{definition}
Then two processes, $P,Q$, are alpha-equivalent if $P = Q\{\vec{y}/\vec{x}\}$ for
some $\vec{x} \in \boundnames{Q},\vec{y} \in \boundnames{P}$, where $Q\{\vec{y}/\vec{x}\}$
denotes the capture-avoiding substitution of $\vec{y}$ for $\vec{x}$ in $Q$.
\end{definition}

\begin{definition}
  The {\em structural congruence} \cite{SangiorgiWalker} , $\equiv$,
  between processes is the least congruence containing
  alpha-equivalence, satisfying the abelian monoid laws
  (associativity, commutativity and $\pzero$ as identity) for parallel
  composition $|$ and for summation $+$.
\end{definition}

\subsection{Name equivalence}

We take name equivalence, written $\nameeq$, to be the smallest
equivalence relation generated by the following rules.

\begin{mathpar}
\inferrule*[lab=Quote-drop]
{ }
{ \quotep{@{x}} \nameeq x }

\inferrule*[lab=Struct-equiv]
{ P \scong Q }
{ \quotep{P} \nameeq \quotep{Q} }
\end{mathpar}

The astute reader will have noticed that the mutual recursion of names
and processes imposes a mutual recursion on alpha-equivalence and
structural equivalence via name-equivalence. Fortunately, all of this
works out pleasantly and we may calculate in the natural way, free of
concern. The reader interested in the details is referred to the
appendix \ref{appendix:rho_details}.

\subsection{Substitution}

We use $\Proc$ for the set of processes, $\QProc$ for the set of
names, and $\id{\{}\vec{y} / \vec{x} \id{\}}$ to denote partial maps,
$s : \QProc \rightarrow \QProc$. A map, $s$ lifts, uniquely, to a map
on process terms, $\widehat{s} : \Proc \rightarrow \Proc$ by the
following equations.

\begin{mathpar}
  (0) \psubstp{Q}{P} := 0 \\
  (R \juxtap S) \psubstp{Q}{P}
  :=    
  (R)\psubstp{Q}{P} \juxtap (S) \psubstp{Q}{P} \\
  (x?(y).R) \psubstp{Q}{P}    
  :=    
  (x)\substp{Q}{P} (z)\concat( (R \psubstn{z}{y}) \psubstp{Q}{P} ) \\
  (\lift{x}{R}) \psubstp{Q}{P}  
  :=
  \lift{(x)\substp{Q}{P}}{ R \psubstp{Q}{P} } \\
%   (\dropn{x})  \psubstp{Q}{P}       
%   := 
%   \left\{ 
%     \begin{array}{ccc} 
%       \dropn{\quotep{Q}} & & x \nameeq \quotep{P} \\
%       \dropn{x} & & otherwise \\
%     \end{array}
%   \right. 
  (\dropn{x})  \psubstp{Q}{P}       
  := 
  \left\{ 
    \begin{array}{ccc} 
      Q & & x \nameeq \quotep{P} \\
      \dropn{x} & & otherwise \\
    \end{array}
  \right.
\end{mathpar}
 

where

\begin{eqnarray}
  (x)\id{\{} \lpquote Q \rpquote / \lpquote P \rpquote \id{\}}            = 
  \left\{ 
    \begin{array}{ccc}
      \lpquote Q \rpquote & & x \nameeq \lpquote P \rpquote \\
      x & & otherwise \\
    \end{array}
  \right. \nonumber
\end{eqnarray}

and $z$ is chosen distinct from $\quotep{P}$, $\quotep{Q}$, the free
names in $Q$, and all the names in $R$. Our $\alpha$-equivalence will
be built in the standard way from this substitution.

\begin{remark}\label{rem:no_self_referential_names}
  One consequence of these definitions is that $\forall P. \quotep{P}
  \not\in \freenames{P}$.
\end{remark}

\subsection{ Dynamic quote: an example }

Anticipating something of what's to come, consider applying the
substitution, $\widehat{\id{\{}u / z \id{\}}}$, to the following pair
of processes, $\lift{w}{y!(z)}$ and $w[ \lpquote y!(z) \rpquote ]$.

\begin{eqnarray}
	\lift{w}{y!(z)}\widehat{\id{\{}u / z \id{\}}}
		& = &
		\lift{w}{y!(u)} \nonumber\\
	w[ \lpquote y!(z) \rpquote ] \widehat{ \id{\{}u / z \id{\}} }
		& = &
		w[ \lpquote y!(z) \rpquote ] \nonumber
\end{eqnarray}

Because the body of the process between quotes is impervious to
substitution, we get radically different answers. In fact, by
examining the first process in an input context,
e.g. $x?(z).\lift{w}{y!(z)}$, we see that the process under the lift
operator may be shaped by prefixed inputs binding a name inside it. In
this sense, the lift operator will be seen as a way to dynamically
construct processes before reifying them as names.

Finally equipped with these standard features we can present the
dynamics of the calculus.

\subsubsection{Operational semantics} 

Finally, we introduce the computational dynamics. What marks these
algebras as distinct from other more traditionally studied algebraic
structures, e.g. vector spaces or polynomial rings, is the manner in
which dynamics is captured. In traditional structures, dynamics is typically
expressed through morphisms between such structures, as in linear maps
between vector spaces or morphisms between rings. In algebras
associated with the semantics of computation, the dynamics is
expressed as part of the algebraic structure itself, through a
reduction reduction relation typically denoted by $\red$. Below, we
give a recursive presentation of this relation for the calculus used
in the encoding.

$\red \subseteq \pi \times \pi$
$\red : \pi \to \mathcal{P}(\pi)$

\begin{mathpar}
  \inferrule* [lab=Comm] { \textsf{match}( x_{src}, x_{trgt} ) } { x_{trgt}?(y)P \; | \; x_{src}!\langle {Q} \rangle \red P\{\quotep{Q}/y}\} }
  \and \\
  \inferrule* [lab=Par] {{P} \red {P}'} {{{P} | {Q}} \red {{P}' | {Q}}}
  \and
  \inferrule* [lab=Equiv]{{{P} \scong {P}'} \andalso {{P}' \red {Q}'} \andalso {{Q}' \scong {Q}}}{{P} \red {Q}}
\end{mathpar}

\begin{eqnarray*}
  match_{\equiv} (\quotep{P},\quotep{Q}) & := & P \equiv Q \\
  match_{\dagger}(\quotep{P},\quotep{Q}) & := & \forall R. P|Q \red^{*} R => R \red^{*} 0 \\
  match_{K}(\quotep{P},\quotep{Q}) & := & K \mbox{ for some context } K
\end{eqnarray*}

$u?(x)P | u!\langle Q \rangle \red P\{\quotep{Q}/x\}$

%We write $\wred$ for $\red^*$, and $P\red$ if $\exists Q $ such that $ P \red Q$.
We write $P\red$ if $\exists Q $ such that $ P \red Q$ and $P\not\red$, otherwise.

\section{Replication}

As mentioned before, it is known that replication (and hence
recursion) can be implemented in a higher-order process algebra
\cite{SangiorgiWalker}. As our first example of calculation with the
machinery thus far presented we give the construction explicitly in
the {\rhoc}.

\begin{eqnarray}
	D_{x} & := & \prefix{x}{y}{(\binpar{\outputp{x}{y}}{@{y}})} \nonumber\\
	\bangp_{x}{P} & := & \binpar{{x}!\langle{\binpar{D_{x}}{P}}\rangle}{D_{x}} \nonumber
\end{eqnarray}

\begin{eqnarray}
	\bangp_{x}{P} & & \nonumber\\
	=
	& {x}!\langle{(\prefix{x}{y}{(\outputp{x}{y} | @{y})) | P}}\rangle 
	      | \prefix{x}{y}{(\outputp{x}{y} | @{y})} & \nonumber\\
	\red
	& (\outputp{x}{y} | @{y})\substn{\quotep{(\prefix{x}{y}{(@{y} | \outputp{x}{y})) | P}}}{y} & \nonumber\\
	=
	& \outputp{x}{\quotep{(\prefix{x}{y}{(\outputp{x}{y} | @{y})) | P}}}
	  | {(\prefix{x}{y}{(\outputp{x}{y} | @{y})) | P}} & \nonumber\\
	\red
	& \ldots & \nonumber\\
	\red^*
	& P | P | \ldots & \nonumber
\end{eqnarray}

Of course, this encoding, as an implementation, runs away, unfolding
$\bangp{P}$ eagerly. A lazier and more implementable replication
operator, restricted to input-guarded processes, may be obtained as follows.

\begin{eqnarray}
\bangp{\prefix{u}{v}{P}} 
	:= 
	\binpar{\lift{x}{\prefix{u}{v}{(\binpar{D(x)}{P})}}}{D(x)} \nonumber
\end{eqnarray}

\begin{remark}
  Note that the lazier definition still does not deal with summation
  or mixed summation (i.e. sums over input and output). The reader is
  invited to construct definitions of replication that deal with these
  features. 

  Further, the definitions are parameterized in a name, $x$. Can you,
  gentle reader, make a definition that eliminates this parameter and
  guarantees no accidental interaction between the replication
  machinery and the process being replicated -- i.e. no accidental
  sharing of names used by the process to get its work done and the
  name(s) used by the replication to effect copying. This latter
  revision of the definition of replication is crucial to obtaining
  the expected identity $!!P \sim !P$.
\end{remark}

\begin{remark}\label{rem:paradoxical_combinator}
  The reader familiar with the lambda calculus will have noticed the
  similarity between $D$ and the paradoxical combinator.

  [Ed. note: the existence of this seems to suggest we have to be more
  restrictive on the set of processes and names we admit if we are to
  support no-cloning.]
\end{remark}

\subsubsection{Bisimulation}

The computational dynamics gives rise to another kind of equivalence,
the equivalence of computational behavior. As previously mentioned
this is typically captured \emph{via} some form of bisimulation.

% The notion we use in this paper is weak barbed bisimulation
% \cite{milner91polyadicpi}.

The notion we use in this paper is derived from weak barbed
bisimulation \cite{milner91polyadicpi}. 

\begin{definition}
An \emph{observation relation}, $\downarrow_{\mathcal N}$, over a set
of names, $\mathcal N$, is the smallest relation satisfying the rules
below.

\infrule[Out-barb]{y \in {\mathcal N}, \; x \nameeq y}
		  {\outputp{x}{v} \downarrow_{\mathcal N} x}
\infrule[Par-barb]{\mbox{$P\downarrow_{\mathcal N} x$ or $Q\downarrow_{\mathcal N} x$}}
		  {\binpar{P}{Q} \downarrow_{\mathcal N} x}

We write $P \Downarrow_{\mathcal N} x$ if there is $Q$ such that 
$P \wred Q$ and $Q \downarrow_{\mathcal N} x$.
\end{definition}

\begin{definition}
%\label{def.bbisim}
An  ${\mathcal N}$-\emph{barbed bisimulation} over a set of names, ${\mathcal N}$, is a symmetric binary relation 
${\mathcal S}_{\mathcal N}$ between agents such that $P\rel{S}_{\mathcal N}Q$ implies:
\begin{enumerate}
\item If $P \red P'$ then $Q \wred Q'$ and $P'\rel{S}_{\mathcal N} Q'$.
\item If $P\downarrow_{\mathcal N} x$, then $Q\Downarrow_{\mathcal N} x$.
\end{enumerate}
$P$ is ${\mathcal N}$-barbed bisimilar to $Q$, written
$P \wbbisim_{\mathcal N} Q$, if $P \rel{S}_{\mathcal N} Q$ for some ${\mathcal N}$-barbed bisimulation ${\mathcal S}_{\mathcal N}$.
\end{definition}

$\mathcal{R} \subseteq \pi \times \pi$

$P \mathcal{R} Q => \forall P'. P \red P' \Rightarrow \exists Q'. Q \red Q', P' \mathcal{R} Q'$

$P \vdash x \Rightarrow Q \vdash x$

\begin{mathpar}
  \inferrule*[lab=Out-barb]{x \nameeq y}{{y}!\langle{Q}\rangle \vdash x}
  \and
  \inferrule*[lab=Par-barb]{\mbox{$P\vdash x$ or $Q\vdash x$}}{\binpar{P}{Q} \vdash x}
\end{mathpar}

\subsubsection{Contexts}

One of the principle advantages of computational calculi like the
$\pi$-calculus is a well-defined notion of context,
contextual-equivalence and a correlation between
contextual-equivalence and notions of bisimulation. The notion of
context allows the decomposition of a process into (sub-)process and
its syntactic environment, its context. Thus, a context may be
thought of as a process with a ``hole'' (written $\Box$) in it. The
application of a context $M$ to a process $P$, written $M[P]$, is
tantamount to filling the hole in $M$ with $P$. In this paper we do
not need the full weight of this theory, but do make use of the notion
of context in the proof the main theorem. 

\begin{mathpar}
  \inferrule* [lab=summation] {} {{M_{M},M_{N}} \bc \Box \;|\; x.M_{A} \;|\; M_{M}+M_{N}}
  \and
  \inferrule* [lab=agent] {} {{M_{A}} \bc (\vec{x})M_{P} \;| \; \clift{P_0,\ldots,M_{P},\ldots,P_N}}
  \and \\
  \inferrule* [lab=process] {} {{M_{P}} \bc M_{N} \;| \;P|M_{P} }
\end{mathpar} 

\begin{mathpar}
  \inferrule* [lab=sychronization] {} {M_{N} \bc \Box \;|\; x?M_{F} \;|\; x!M_{C}}
  \and
  \inferrule* [lab=abstraction] {} {{M_{F}} \bc (x)M_{P} }
  \and
  \inferrule* [lab=concretion] {} {{M_{C}} \bc \langle M_{P} \rangle }
  \and \\
  \inferrule* [lab=process] {} {{M_{P}} \bc M_{N} \;| \;P|M_{P} }
\end{mathpar}

\begin{definition}[contextual application] Given a context $M$, and
  process $P$, we define the \emph{contextual application}, $M[P] :=
  M\{P/\Box\}$. That is, the contextual application of M to P is the
  substitution of $P$ for $\Box$ in $M$.
\end{definition}

$\meaningof{-} : L \to \mathcal{P}(\pi)$

\begin{mathpar}
  \inferrule* [lab=collection] {} {\meaningof{true} = \pi, \and \meaningof{~E} = \pi \setminus \meaningof{E}, \and \meaningof{E_{1} \& E_{2}} = \meaningof{E_{1}} \cap \meaningof{E_{2}}}
\end{mathpar}

\begin{mathpar}
  \inferrule* [lab=structure] {} {\meaningof{0} = \{ P \in \pi | P \equiv 0 \}, \and \\ \meaningof{E_1 | E_2} = \{ P \in \pi | P \equiv P_{1} | P_{2}, P_{1} \in \meaningof{E_{1}}, P_{2} \in \meaningof{E_2}\} }
\end{mathpar}

\begin{mathpar}
 \inferrule* [lab=behavior] {} {\meaningof{\langle a?b \rangle E} = \{ P \in \pi | P \equiv Q | u?(y)P', \\ \and \\\\ \and \\ \;\;\; u \in \meaningof{a}, \forall z.P'\{z/y\} \in \meaningof{E\{z/b\}}\}, \and \\ \meaningof{a!E} = \{ P \in \pi | P \equiv Q | x!\langle P' \rangle, x \in \meaningof{a} P' \in \meaningof{E}\} }
\end{mathpar}

\begin{mathpar}
 \inferrule* [lab=nominal] {} {\meaningof{\quotep{E}} = \{ \quotep{P} \in \quotep{\pi} | P \in \meaningof{E} \}, \and \meaningof{\quotep{P}} = \{ \quotep{Q} \in \quotep{\pi} | P \equiv Q \} \and \\ \meaningof{@\quotep{E}} = \{ P \in \pi | P \equiv @x, x \in \meaningof{E} \}}
\end{mathpar}

\begin{eqnarray*}
  \\
  \meaningof{-} : TS \to ST
\end{eqnarray*}

\begin{eqnarray*}
  \\
  L : TS \to ST
\end{eqnarray*}

\begin{eqnarray*}
  \\
  P \models E \iff P \in \meaningof{E}
\end{eqnarray*}

\begin{eqnarray*}
  P \approx_{L} Q \iff \forall E \in L. P \models E \iff Q \models E
\end{eqnarray*}

\begin{eqnarray*}
  P \approx_{K} Q
\end{eqnarray*}

\begin{eqnarray*}
  P \approx Q
\end{eqnarray*}

$\approx_{K} = \approx = \approx_{L}$

\subsubsection{Contextual duality}

Note that contexts extend the quotation operation to a family of
operations from processes to names. Given a context, $M$, we can
define a \emph{nominal context}, $\quotep{M}$ by $\quotep{M}[P] :=
\quotep{M[P]}$. To foreshadow what is to come we observe that these
operations enjoy a duality with processes very much like the duality
between vectors and maps from vectors to scalars.

Further, because the calculus is essentially higher-order, we have a
correspondence between contexts and processes. More specifically,
given a name $x$ and a context $M$ we can construct $M^{*}_{x}$ such
that 

\begin{mathpar}
  M^{*}_{x} | \lift{x}{P} \red M[P]
\end{mathpar}

namely,

\begin{mathpar}
  M^{*}_{x} := x?(u).M[\dropn{u}]
\end{mathpar}

The dependence of $M^{*}_{x}$ on a name makes it an abstraction, 

\begin{mathpar}
  M^{*} := (x)x?(u).M[\dropn{u}]
\end{mathpar}

\subsection{Additional notation}

It will sometimes be convenient to denote the process a name
quotes. We already have the notation $x = \quotep{P}$, but it will be
convenient to introduce an alternate notation, $\procn{x}$, when we
want to emphasize the connection to the use of the name. Note that, by
virtue of name equivalence, $\quotep{\procn{x}} \nameeq x$; so, the
notation is consistent with previous definitions.

Further, because names have structure it is possible to effect
substitutions on the basis of that structure. This means we need to
upgrade our notation for substitutions, which we accomplish by
adapting comprehension notation. Thus,

\begin{mathpar}
  P\{ y / x : x \in S \}
\end{mathpar}

is interpreted to mean the process derived from P by replacing (in a
capture-avoiding manner) each occurrence of $x$ in $S$ by $y$. For example,

\begin{mathpar}
  P\{ \quotep{\procn{x}|\procn{x}} / x : x \in \freenames{P} \}
\end{mathpar}

will replace each (occurrence) of a free name $x$ in $P$ by
$\quotep{\procn{x}|\procn{x}}$.

Also, we will avail ourselves of the notation $x^{L}$ and $x^{R}$ to
denote injections of a name into disjoint copies of the name
space. There are numerous ways to accomplish this. One example can be
found in \cite{MeredithR05}. This notation overloads to vectors of
names: $\vec{x}^{\pi} := (x_{i}^{\pi} \; : \; 0 \leq i < |\vec{x}| )$ where $\pi \in \{L,R\}$.

We also use $P^{\Box} := P|\Box$.

In \cite{MeredithR05} an interpretation of the new operator is
given. It turns out that there are several possible interpretations
all enjoying the requisite algebraic properties of the operator (see
\cite{milner91polyadicpi}). We will therefore make liberal use of
$(\nu\; \vec{x})P$.

% subsection the_syntax_and_semantics_of_the_notation_system (end)   

\input{qm2pi.qmops} 

\input{qm2pi.sterngerlach} 

\input{qm2pi.metric} 

% section concurrent_process_calculi (end)

%\input{qm2pi.proofsketch}

% section proof sketch (end)

%\input{qm2pi.slviaknots} 

% section spatial logic via knots (end)

\input{qm2pi.conclusion}

% section conclusion (end)

%\input{qm2pi.dtcodes} 

% section wiring algorithm (end)

\input{qm2pi.ack} 

% section acknowledgments (end)

\newpage


\bibliographystyle{plain}   
\bibliography{../../biblios/main.bib}

\input{qm2pi.rhodetails}

\end{document}

 

%\documentclass[12pt]{llncs}
%\documentclass{jktr}

\usepackage[pdftex]{hyperref}                   
\usepackage {listings}
\usepackage {mathpartir}
\usepackage{bcprules}
%\usepackage{listings}
                       
\usepackage{graphicx} 
%\usepackage[margins=2.5cm,nohead,nofoot]{geometry}
%\usepackage{geometry}
\usepackage{amsfonts}
\usepackage{amstext}
\usepackage{latexsym}
\usepackage{amssymb}
\usepackage{color}


%\include{myPreamble}
\include{qm2pi.local} 

%\ifpdf
%\usepackage[pdftex]{graphicx}
%\else
%\usepackage{graphicx}
%\fi

 % \ifpdf
%  \usepackage{pdfsync}
%  \if


%\title{Brief Article}
%\author{David F. Snyder}
%\author{L.G. Meredith}

%\address{Dept. of Math., Texas State University--San Marcos, San Marcos, TX 78666}
       
\pagestyle{empty}


\begin{document}

\lstset{language=[Objective]Caml,frame=shadowbox}

\input{qm2pi.front}

% section front matter (end)

\input{qm2pi.intro} 
 
% section introduction (end)

% \input{qm2pi.knotations} 

% section notation (end)

\input{qm2pi.process.calculi} 

% section concurrent_process_calculi_and_spatial_logics_ (end)
    
%\input{qm2pi.knots2pi} 

%\input{qm2pi.trefoil} 

%\input{qm2pi.mainthm} 

% subsection basic_interpretation (end)

%\input{qm2pi.rho.presentation} 
\subsection{The syntax and semantics of the notation system}\label{sub:the_syntax_and_semantics_of_the_notation_system} % (fold)

We now summarize a technical presentation of the calculus that
embodies our theory of dynamics. The typical presentation of such a
calculus follows the style of giving generators and relations on
them. The grammar, below, describing term constructors, freely
generates the set of processes, $\Proc$. This set is then quotiented
by a relation known as structural congruence and it is over this set
that the notion of dynamics is expressed. This presentation is
essentially that of \cite{MeredithR05} with the addition of
polyadicity and summation. For readability we have relegated some of
the technical subtleties to an appendix.

\subsubsection{Process grammar}\label{subsub:process_grammar}

\begin{mathpar}
  \inferrule* [lab=synchronization] {} {{M} \bc \pzero \;|\; x?F \;|\; x!C }
  \and
  \inferrule* [lab=abstraction] {} {{F} \bc (x)P}
  \and
  \inferrule* [lab=concretion] {} {{C} \bc \langle Q \rangle}
  \and
  \inferrule* [lab=process] {} {{P,Q} \bc M \;| \;P|Q \;|\; @{x}}
  \and
  \inferrule* [lab=name] {} {{x} \bc \quotep{P}}
\end{mathpar} 

Note that $\vec{x}$ (resp. $\vec{P}$) denotes a vector of names
(resp. processes) of length $|\vec{x}|$ (resp. $|\vec{P}|$). We adopt
the following useful abbreviations.

\begin{mathpar}
   x?(\vec{y}).P := x.(\vec{y})P \and  x\clift{\vec{P}} := x.\clift{\vec{P}}
   \and x!(y) := \lift{x}{\dropn{y}}
   \and \Pi_{i=0}^{n-1}P_i := P_0 | \ldots | P_{n-1}
\end{mathpar}

\subsubsection{Structural congruence}

\paragraph{Free and bound names and alpha-equivalence.} At the
core of structural equivalence is alpha-equivalence which identifies
process that are the same up to a change of variable. Formally, we
recognize the distinction between free and bound names. The free names
of a process, $\freenames{P}$, may be calculated recursively as
follows:

\begin{mathpar}
\freenames{\pzero} := \emptyset
  \and \\
  \freenames{x?(y).P} := \{ x \} \cup (\freenames{P} \setminus \{ y \})
  \and 
  \freenames{x!\langle P \rangle} := \{ x \} \cup \{ P \} 
  \and \\
  \freenames{P|Q} := \freenames{P} \cup \freenames{Q}
  \and \\
  \freenames{@{x}} := \{ x \}
\end{mathpar}

$\pi$
$\quotep{\pi}$

$\freenames{-} : \pi \to \mathcal{P}(\quotep{\pi})$

\begin{eqnarray*}
  \freenames{\pzero} & := & \emptyset \\
  \freenames{x?(y).P} & := & \{ x \} \cup (\freenames{P} \setminus \{ y \}) \\
  \freenames{x!\langle P \rangle} & := & \{ x \} \cup \{ P \} \\
  \freenames{P|Q} & := & \freenames{P} \cup \freenames{Q} \\
  \freenames{\dropn{x}} & := & \{ x \}
\end{eqnarray*}

The bound names of a process, $\boundnames{P}$, are those names occurring in $P$
that are not free. For example, in $x?(y).0$, the name $x$ is free, while $y$ is bound.

\begin{mathpar}
  \inferrule* [lab=monoidal-laws] {} { P|Q \equiv Q|P \and P|0 \equiv P \and P|(Q|R) \equiv (P|Q)|R }
\end{mathpar}

\begin{mathpar}
  \inferrule* [lab=alpha-equivalence] {} { (x)P \equiv (y)P\{y/x\} \and y \not\in \freenames{P} }
\end{mathpar}

\begin{definition}
Then two processes, $P,Q$, are alpha-equivalent if $P = Q\{\vec{y}/\vec{x}\}$ for
some $\vec{x} \in \boundnames{Q},\vec{y} \in \boundnames{P}$, where $Q\{\vec{y}/\vec{x}\}$
denotes the capture-avoiding substitution of $\vec{y}$ for $\vec{x}$ in $Q$.
\end{definition}

\begin{definition}
  The {\em structural congruence} \cite{SangiorgiWalker} , $\equiv$,
  between processes is the least congruence containing
  alpha-equivalence, satisfying the abelian monoid laws
  (associativity, commutativity and $\pzero$ as identity) for parallel
  composition $|$ and for summation $+$.
\end{definition}

\subsection{Name equivalence}

We take name equivalence, written $\nameeq$, to be the smallest
equivalence relation generated by the following rules.

\begin{mathpar}
\inferrule*[lab=Quote-drop]
{ }
{ \quotep{@{x}} \nameeq x }

\inferrule*[lab=Struct-equiv]
{ P \scong Q }
{ \quotep{P} \nameeq \quotep{Q} }
\end{mathpar}

The astute reader will have noticed that the mutual recursion of names
and processes imposes a mutual recursion on alpha-equivalence and
structural equivalence via name-equivalence. Fortunately, all of this
works out pleasantly and we may calculate in the natural way, free of
concern. The reader interested in the details is referred to the
appendix \ref{appendix:rho_details}.

\subsection{Substitution}

We use $\Proc$ for the set of processes, $\QProc$ for the set of
names, and $\id{\{}\vec{y} / \vec{x} \id{\}}$ to denote partial maps,
$s : \QProc \rightarrow \QProc$. A map, $s$ lifts, uniquely, to a map
on process terms, $\widehat{s} : \Proc \rightarrow \Proc$ by the
following equations.

\begin{mathpar}
  (0) \psubstp{Q}{P} := 0 \\
  (R \juxtap S) \psubstp{Q}{P}
  :=    
  (R)\psubstp{Q}{P} \juxtap (S) \psubstp{Q}{P} \\
  (x?(y).R) \psubstp{Q}{P}    
  :=    
  (x)\substp{Q}{P} (z)\concat( (R \psubstn{z}{y}) \psubstp{Q}{P} ) \\
  (\lift{x}{R}) \psubstp{Q}{P}  
  :=
  \lift{(x)\substp{Q}{P}}{ R \psubstp{Q}{P} } \\
%   (\dropn{x})  \psubstp{Q}{P}       
%   := 
%   \left\{ 
%     \begin{array}{ccc} 
%       \dropn{\quotep{Q}} & & x \nameeq \quotep{P} \\
%       \dropn{x} & & otherwise \\
%     \end{array}
%   \right. 
  (\dropn{x})  \psubstp{Q}{P}       
  := 
  \left\{ 
    \begin{array}{ccc} 
      Q & & x \nameeq \quotep{P} \\
      \dropn{x} & & otherwise \\
    \end{array}
  \right.
\end{mathpar}
 

where

\begin{eqnarray}
  (x)\id{\{} \lpquote Q \rpquote / \lpquote P \rpquote \id{\}}            = 
  \left\{ 
    \begin{array}{ccc}
      \lpquote Q \rpquote & & x \nameeq \lpquote P \rpquote \\
      x & & otherwise \\
    \end{array}
  \right. \nonumber
\end{eqnarray}

and $z$ is chosen distinct from $\quotep{P}$, $\quotep{Q}$, the free
names in $Q$, and all the names in $R$. Our $\alpha$-equivalence will
be built in the standard way from this substitution.

\begin{remark}\label{rem:no_self_referential_names}
  One consequence of these definitions is that $\forall P. \quotep{P}
  \not\in \freenames{P}$.
\end{remark}

\subsection{ Dynamic quote: an example }

Anticipating something of what's to come, consider applying the
substitution, $\widehat{\id{\{}u / z \id{\}}}$, to the following pair
of processes, $\lift{w}{y!(z)}$ and $w[ \lpquote y!(z) \rpquote ]$.

\begin{eqnarray}
	\lift{w}{y!(z)}\widehat{\id{\{}u / z \id{\}}}
		& = &
		\lift{w}{y!(u)} \nonumber\\
	w[ \lpquote y!(z) \rpquote ] \widehat{ \id{\{}u / z \id{\}} }
		& = &
		w[ \lpquote y!(z) \rpquote ] \nonumber
\end{eqnarray}

Because the body of the process between quotes is impervious to
substitution, we get radically different answers. In fact, by
examining the first process in an input context,
e.g. $x?(z).\lift{w}{y!(z)}$, we see that the process under the lift
operator may be shaped by prefixed inputs binding a name inside it. In
this sense, the lift operator will be seen as a way to dynamically
construct processes before reifying them as names.

Finally equipped with these standard features we can present the
dynamics of the calculus.

\subsubsection{Operational semantics} 

Finally, we introduce the computational dynamics. What marks these
algebras as distinct from other more traditionally studied algebraic
structures, e.g. vector spaces or polynomial rings, is the manner in
which dynamics is captured. In traditional structures, dynamics is typically
expressed through morphisms between such structures, as in linear maps
between vector spaces or morphisms between rings. In algebras
associated with the semantics of computation, the dynamics is
expressed as part of the algebraic structure itself, through a
reduction reduction relation typically denoted by $\red$. Below, we
give a recursive presentation of this relation for the calculus used
in the encoding.

$\red \subseteq \pi \times \pi$
$\red : \pi \to \mathcal{P}(\pi)$

\begin{mathpar}
  \inferrule* [lab=Comm] { \textsf{match}( x_{src}, x_{trgt} ) } { x_{trgt}?(y)P \; | \; x_{src}!\langle {Q} \rangle \red P\{\quotep{Q}/y}\} }
  \and \\
  \inferrule* [lab=Par] {{P} \red {P}'} {{{P} | {Q}} \red {{P}' | {Q}}}
  \and
  \inferrule* [lab=Equiv]{{{P} \scong {P}'} \andalso {{P}' \red {Q}'} \andalso {{Q}' \scong {Q}}}{{P} \red {Q}}
\end{mathpar}

\begin{eqnarray*}
  match_{\equiv} (\quotep{P},\quotep{Q}) & := & P \equiv Q \\
  match_{\dagger}(\quotep{P},\quotep{Q}) & := & \forall R. P|Q \red^{*} R => R \red^{*} 0 \\
  match_{K}(\quotep{P},\quotep{Q}) & := & K \mbox{ for some context } K
\end{eqnarray*}

$u?(x)P | u!\langle Q \rangle \red P\{\quotep{Q}/x\}$

%We write $\wred$ for $\red^*$, and $P\red$ if $\exists Q $ such that $ P \red Q$.
We write $P\red$ if $\exists Q $ such that $ P \red Q$ and $P\not\red$, otherwise.

\section{Replication}

As mentioned before, it is known that replication (and hence
recursion) can be implemented in a higher-order process algebra
\cite{SangiorgiWalker}. As our first example of calculation with the
machinery thus far presented we give the construction explicitly in
the {\rhoc}.

\begin{eqnarray}
	D_{x} & := & \prefix{x}{y}{(\binpar{\outputp{x}{y}}{@{y}})} \nonumber\\
	\bangp_{x}{P} & := & \binpar{{x}!\langle{\binpar{D_{x}}{P}}\rangle}{D_{x}} \nonumber
\end{eqnarray}

\begin{eqnarray}
	\bangp_{x}{P} & & \nonumber\\
	=
	& {x}!\langle{(\prefix{x}{y}{(\outputp{x}{y} | @{y})) | P}}\rangle 
	      | \prefix{x}{y}{(\outputp{x}{y} | @{y})} & \nonumber\\
	\red
	& (\outputp{x}{y} | @{y})\substn{\quotep{(\prefix{x}{y}{(@{y} | \outputp{x}{y})) | P}}}{y} & \nonumber\\
	=
	& \outputp{x}{\quotep{(\prefix{x}{y}{(\outputp{x}{y} | @{y})) | P}}}
	  | {(\prefix{x}{y}{(\outputp{x}{y} | @{y})) | P}} & \nonumber\\
	\red
	& \ldots & \nonumber\\
	\red^*
	& P | P | \ldots & \nonumber
\end{eqnarray}

Of course, this encoding, as an implementation, runs away, unfolding
$\bangp{P}$ eagerly. A lazier and more implementable replication
operator, restricted to input-guarded processes, may be obtained as follows.

\begin{eqnarray}
\bangp{\prefix{u}{v}{P}} 
	:= 
	\binpar{\lift{x}{\prefix{u}{v}{(\binpar{D(x)}{P})}}}{D(x)} \nonumber
\end{eqnarray}

\begin{remark}
  Note that the lazier definition still does not deal with summation
  or mixed summation (i.e. sums over input and output). The reader is
  invited to construct definitions of replication that deal with these
  features. 

  Further, the definitions are parameterized in a name, $x$. Can you,
  gentle reader, make a definition that eliminates this parameter and
  guarantees no accidental interaction between the replication
  machinery and the process being replicated -- i.e. no accidental
  sharing of names used by the process to get its work done and the
  name(s) used by the replication to effect copying. This latter
  revision of the definition of replication is crucial to obtaining
  the expected identity $!!P \sim !P$.
\end{remark}

\begin{remark}\label{rem:paradoxical_combinator}
  The reader familiar with the lambda calculus will have noticed the
  similarity between $D$ and the paradoxical combinator.

  [Ed. note: the existence of this seems to suggest we have to be more
  restrictive on the set of processes and names we admit if we are to
  support no-cloning.]
\end{remark}

\subsubsection{Bisimulation}

The computational dynamics gives rise to another kind of equivalence,
the equivalence of computational behavior. As previously mentioned
this is typically captured \emph{via} some form of bisimulation.

% The notion we use in this paper is weak barbed bisimulation
% \cite{milner91polyadicpi}.

The notion we use in this paper is derived from weak barbed
bisimulation \cite{milner91polyadicpi}. 

\begin{definition}
An \emph{observation relation}, $\downarrow_{\mathcal N}$, over a set
of names, $\mathcal N$, is the smallest relation satisfying the rules
below.

\infrule[Out-barb]{y \in {\mathcal N}, \; x \nameeq y}
		  {\outputp{x}{v} \downarrow_{\mathcal N} x}
\infrule[Par-barb]{\mbox{$P\downarrow_{\mathcal N} x$ or $Q\downarrow_{\mathcal N} x$}}
		  {\binpar{P}{Q} \downarrow_{\mathcal N} x}

We write $P \Downarrow_{\mathcal N} x$ if there is $Q$ such that 
$P \wred Q$ and $Q \downarrow_{\mathcal N} x$.
\end{definition}

\begin{definition}
%\label{def.bbisim}
An  ${\mathcal N}$-\emph{barbed bisimulation} over a set of names, ${\mathcal N}$, is a symmetric binary relation 
${\mathcal S}_{\mathcal N}$ between agents such that $P\rel{S}_{\mathcal N}Q$ implies:
\begin{enumerate}
\item If $P \red P'$ then $Q \wred Q'$ and $P'\rel{S}_{\mathcal N} Q'$.
\item If $P\downarrow_{\mathcal N} x$, then $Q\Downarrow_{\mathcal N} x$.
\end{enumerate}
$P$ is ${\mathcal N}$-barbed bisimilar to $Q$, written
$P \wbbisim_{\mathcal N} Q$, if $P \rel{S}_{\mathcal N} Q$ for some ${\mathcal N}$-barbed bisimulation ${\mathcal S}_{\mathcal N}$.
\end{definition}

$\mathcal{R} \subseteq \pi \times \pi$

$P \mathcal{R} Q => \forall P'. P \red P' \Rightarrow \exists Q'. Q \red Q', P' \mathcal{R} Q'$

$P \vdash x \Rightarrow Q \vdash x$

\begin{mathpar}
  \inferrule*[lab=Out-barb]{x \nameeq y}{{y}!\langle{Q}\rangle \vdash x}
  \and
  \inferrule*[lab=Par-barb]{\mbox{$P\vdash x$ or $Q\vdash x$}}{\binpar{P}{Q} \vdash x}
\end{mathpar}

\subsubsection{Contexts}

One of the principle advantages of computational calculi like the
$\pi$-calculus is a well-defined notion of context,
contextual-equivalence and a correlation between
contextual-equivalence and notions of bisimulation. The notion of
context allows the decomposition of a process into (sub-)process and
its syntactic environment, its context. Thus, a context may be
thought of as a process with a ``hole'' (written $\Box$) in it. The
application of a context $M$ to a process $P$, written $M[P]$, is
tantamount to filling the hole in $M$ with $P$. In this paper we do
not need the full weight of this theory, but do make use of the notion
of context in the proof the main theorem. 

\begin{mathpar}
  \inferrule* [lab=summation] {} {{M_{M},M_{N}} \bc \Box \;|\; x.M_{A} \;|\; M_{M}+M_{N}}
  \and
  \inferrule* [lab=agent] {} {{M_{A}} \bc (\vec{x})M_{P} \;| \; \clift{P_0,\ldots,M_{P},\ldots,P_N}}
  \and \\
  \inferrule* [lab=process] {} {{M_{P}} \bc M_{N} \;| \;P|M_{P} }
\end{mathpar} 

\begin{mathpar}
  \inferrule* [lab=sychronization] {} {M_{N} \bc \Box \;|\; x?M_{F} \;|\; x!M_{C}}
  \and
  \inferrule* [lab=abstraction] {} {{M_{F}} \bc (x)M_{P} }
  \and
  \inferrule* [lab=concretion] {} {{M_{C}} \bc \langle M_{P} \rangle }
  \and \\
  \inferrule* [lab=process] {} {{M_{P}} \bc M_{N} \;| \;P|M_{P} }
\end{mathpar}

\begin{definition}[contextual application] Given a context $M$, and
  process $P$, we define the \emph{contextual application}, $M[P] :=
  M\{P/\Box\}$. That is, the contextual application of M to P is the
  substitution of $P$ for $\Box$ in $M$.
\end{definition}

$\meaningof{-} : L \to \mathcal{P}(\pi)$

\begin{mathpar}
  \inferrule* [lab=collection] {} {\meaningof{true} = \pi, \and \meaningof{~E} = \pi \setminus \meaningof{E}, \and \meaningof{E_{1} \& E_{2}} = \meaningof{E_{1}} \cap \meaningof{E_{2}}}
\end{mathpar}

\begin{mathpar}
  \inferrule* [lab=structure] {} {\meaningof{0} = \{ P \in \pi | P \equiv 0 \}, \and \\ \meaningof{E_1 | E_2} = \{ P \in \pi | P \equiv P_{1} | P_{2}, P_{1} \in \meaningof{E_{1}}, P_{2} \in \meaningof{E_2}\} }
\end{mathpar}

\begin{mathpar}
 \inferrule* [lab=behavior] {} {\meaningof{\langle a?b \rangle E} = \{ P \in \pi | P \equiv Q | u?(y)P', \\ \and \\\\ \and \\ \;\;\; u \in \meaningof{a}, \forall z.P'\{z/y\} \in \meaningof{E\{z/b\}}\}, \and \\ \meaningof{a!E} = \{ P \in \pi | P \equiv Q | x!\langle P' \rangle, x \in \meaningof{a} P' \in \meaningof{E}\} }
\end{mathpar}

\begin{mathpar}
 \inferrule* [lab=nominal] {} {\meaningof{\quotep{E}} = \{ \quotep{P} \in \quotep{\pi} | P \in \meaningof{E} \}, \and \meaningof{\quotep{P}} = \{ \quotep{Q} \in \quotep{\pi} | P \equiv Q \} \and \\ \meaningof{@\quotep{E}} = \{ P \in \pi | P \equiv @x, x \in \meaningof{E} \}}
\end{mathpar}

\begin{eqnarray*}
  \\
  \meaningof{-} : TS \to ST
\end{eqnarray*}

\begin{eqnarray*}
  \\
  L : TS \to ST
\end{eqnarray*}

\begin{eqnarray*}
  \\
  P \models E \iff P \in \meaningof{E}
\end{eqnarray*}

\begin{eqnarray*}
  P \approx_{L} Q \iff \forall E \in L. P \models E \iff Q \models E
\end{eqnarray*}

\begin{eqnarray*}
  P \approx_{K} Q
\end{eqnarray*}

\begin{eqnarray*}
  P \approx Q
\end{eqnarray*}

$\approx_{K} = \approx = \approx_{L}$

\subsubsection{Contextual duality}

Note that contexts extend the quotation operation to a family of
operations from processes to names. Given a context, $M$, we can
define a \emph{nominal context}, $\quotep{M}$ by $\quotep{M}[P] :=
\quotep{M[P]}$. To foreshadow what is to come we observe that these
operations enjoy a duality with processes very much like the duality
between vectors and maps from vectors to scalars.

Further, because the calculus is essentially higher-order, we have a
correspondence between contexts and processes. More specifically,
given a name $x$ and a context $M$ we can construct $M^{*}_{x}$ such
that 

\begin{mathpar}
  M^{*}_{x} | \lift{x}{P} \red M[P]
\end{mathpar}

namely,

\begin{mathpar}
  M^{*}_{x} := x?(u).M[\dropn{u}]
\end{mathpar}

The dependence of $M^{*}_{x}$ on a name makes it an abstraction, 

\begin{mathpar}
  M^{*} := (x)x?(u).M[\dropn{u}]
\end{mathpar}

\subsection{Additional notation}

It will sometimes be convenient to denote the process a name
quotes. We already have the notation $x = \quotep{P}$, but it will be
convenient to introduce an alternate notation, $\procn{x}$, when we
want to emphasize the connection to the use of the name. Note that, by
virtue of name equivalence, $\quotep{\procn{x}} \nameeq x$; so, the
notation is consistent with previous definitions.

Further, because names have structure it is possible to effect
substitutions on the basis of that structure. This means we need to
upgrade our notation for substitutions, which we accomplish by
adapting comprehension notation. Thus,

\begin{mathpar}
  P\{ y / x : x \in S \}
\end{mathpar}

is interpreted to mean the process derived from P by replacing (in a
capture-avoiding manner) each occurrence of $x$ in $S$ by $y$. For example,

\begin{mathpar}
  P\{ \quotep{\procn{x}|\procn{x}} / x : x \in \freenames{P} \}
\end{mathpar}

will replace each (occurrence) of a free name $x$ in $P$ by
$\quotep{\procn{x}|\procn{x}}$.

Also, we will avail ourselves of the notation $x^{L}$ and $x^{R}$ to
denote injections of a name into disjoint copies of the name
space. There are numerous ways to accomplish this. One example can be
found in \cite{MeredithR05}. This notation overloads to vectors of
names: $\vec{x}^{\pi} := (x_{i}^{\pi} \; : \; 0 \leq i < |\vec{x}| )$ where $\pi \in \{L,R\}$.

We also use $P^{\Box} := P|\Box$.

In \cite{MeredithR05} an interpretation of the new operator is
given. It turns out that there are several possible interpretations
all enjoying the requisite algebraic properties of the operator (see
\cite{milner91polyadicpi}). We will therefore make liberal use of
$(\nu\; \vec{x})P$.

% subsection the_syntax_and_semantics_of_the_notation_system (end)   

\input{qm2pi.qmops} 

\input{qm2pi.sterngerlach} 

\input{qm2pi.metric} 

% section concurrent_process_calculi (end)

%\input{qm2pi.proofsketch}

% section proof sketch (end)

%\input{qm2pi.slviaknots} 

% section spatial logic via knots (end)

\input{qm2pi.conclusion}

% section conclusion (end)

%\input{qm2pi.dtcodes} 

% section wiring algorithm (end)

\input{qm2pi.ack} 

% section acknowledgments (end)

\newpage


\bibliographystyle{plain}   
\bibliography{../../biblios/main.bib}

\input{qm2pi.rhodetails}

\end{document}

 

% subsection basic_interpretation (end)

%\input{qm2pi.rho.presentation} 
\subsection{The syntax and semantics of the notation system}\label{sub:the_syntax_and_semantics_of_the_notation_system} % (fold)

We now summarize a technical presentation of the calculus that
embodies our theory of dynamics. The typical presentation of such a
calculus follows the style of giving generators and relations on
them. The grammar, below, describing term constructors, freely
generates the set of processes, $\Proc$. This set is then quotiented
by a relation known as structural congruence and it is over this set
that the notion of dynamics is expressed. This presentation is
essentially that of \cite{MeredithR05} with the addition of
polyadicity and summation. For readability we have relegated some of
the technical subtleties to an appendix.

\subsubsection{Process grammar}\label{subsub:process_grammar}

\begin{mathpar}
  \inferrule* [lab=synchronization] {} {{M} \bc \pzero \;|\; x?F \;|\; x!C }
  \and
  \inferrule* [lab=abstraction] {} {{F} \bc (x)P}
  \and
  \inferrule* [lab=concretion] {} {{C} \bc \langle Q \rangle}
  \and
  \inferrule* [lab=process] {} {{P,Q} \bc M \;| \;P|Q \;|\; @{x}}
  \and
  \inferrule* [lab=name] {} {{x} \bc \quotep{P}}
\end{mathpar} 

Note that $\vec{x}$ (resp. $\vec{P}$) denotes a vector of names
(resp. processes) of length $|\vec{x}|$ (resp. $|\vec{P}|$). We adopt
the following useful abbreviations.

\begin{mathpar}
   x?(\vec{y}).P := x.(\vec{y})P \and  x\clift{\vec{P}} := x.\clift{\vec{P}}
   \and x!(y) := \lift{x}{\dropn{y}}
   \and \Pi_{i=0}^{n-1}P_i := P_0 | \ldots | P_{n-1}
\end{mathpar}

\subsubsection{Structural congruence}

\paragraph{Free and bound names and alpha-equivalence.} At the
core of structural equivalence is alpha-equivalence which identifies
process that are the same up to a change of variable. Formally, we
recognize the distinction between free and bound names. The free names
of a process, $\freenames{P}$, may be calculated recursively as
follows:

\begin{mathpar}
\freenames{\pzero} := \emptyset
  \and \\
  \freenames{x?(y).P} := \{ x \} \cup (\freenames{P} \setminus \{ y \})
  \and 
  \freenames{x!\langle P \rangle} := \{ x \} \cup \{ P \} 
  \and \\
  \freenames{P|Q} := \freenames{P} \cup \freenames{Q}
  \and \\
  \freenames{@{x}} := \{ x \}
\end{mathpar}

$\pi$
$\quotep{\pi}$

$\freenames{-} : \pi \to \mathcal{P}(\quotep{\pi})$

\begin{eqnarray*}
  \freenames{\pzero} & := & \emptyset \\
  \freenames{x?(y).P} & := & \{ x \} \cup (\freenames{P} \setminus \{ y \}) \\
  \freenames{x!\langle P \rangle} & := & \{ x \} \cup \{ P \} \\
  \freenames{P|Q} & := & \freenames{P} \cup \freenames{Q} \\
  \freenames{\dropn{x}} & := & \{ x \}
\end{eqnarray*}

The bound names of a process, $\boundnames{P}$, are those names occurring in $P$
that are not free. For example, in $x?(y).0$, the name $x$ is free, while $y$ is bound.

\begin{mathpar}
  \inferrule* [lab=monoidal-laws] {} { P|Q \equiv Q|P \and P|0 \equiv P \and P|(Q|R) \equiv (P|Q)|R }
\end{mathpar}

\begin{mathpar}
  \inferrule* [lab=alpha-equivalence] {} { (x)P \equiv (y)P\{y/x\} \and y \not\in \freenames{P} }
\end{mathpar}

\begin{definition}
Then two processes, $P,Q$, are alpha-equivalent if $P = Q\{\vec{y}/\vec{x}\}$ for
some $\vec{x} \in \boundnames{Q},\vec{y} \in \boundnames{P}$, where $Q\{\vec{y}/\vec{x}\}$
denotes the capture-avoiding substitution of $\vec{y}$ for $\vec{x}$ in $Q$.
\end{definition}

\begin{definition}
  The {\em structural congruence} \cite{SangiorgiWalker} , $\equiv$,
  between processes is the least congruence containing
  alpha-equivalence, satisfying the abelian monoid laws
  (associativity, commutativity and $\pzero$ as identity) for parallel
  composition $|$ and for summation $+$.
\end{definition}

\subsection{Name equivalence}

We take name equivalence, written $\nameeq$, to be the smallest
equivalence relation generated by the following rules.

\begin{mathpar}
\inferrule*[lab=Quote-drop]
{ }
{ \quotep{@{x}} \nameeq x }

\inferrule*[lab=Struct-equiv]
{ P \scong Q }
{ \quotep{P} \nameeq \quotep{Q} }
\end{mathpar}

The astute reader will have noticed that the mutual recursion of names
and processes imposes a mutual recursion on alpha-equivalence and
structural equivalence via name-equivalence. Fortunately, all of this
works out pleasantly and we may calculate in the natural way, free of
concern. The reader interested in the details is referred to the
appendix \ref{appendix:rho_details}.

\subsection{Substitution}

We use $\Proc$ for the set of processes, $\QProc$ for the set of
names, and $\id{\{}\vec{y} / \vec{x} \id{\}}$ to denote partial maps,
$s : \QProc \rightarrow \QProc$. A map, $s$ lifts, uniquely, to a map
on process terms, $\widehat{s} : \Proc \rightarrow \Proc$ by the
following equations.

\begin{mathpar}
  (0) \psubstp{Q}{P} := 0 \\
  (R \juxtap S) \psubstp{Q}{P}
  :=    
  (R)\psubstp{Q}{P} \juxtap (S) \psubstp{Q}{P} \\
  (x?(y).R) \psubstp{Q}{P}    
  :=    
  (x)\substp{Q}{P} (z)\concat( (R \psubstn{z}{y}) \psubstp{Q}{P} ) \\
  (\lift{x}{R}) \psubstp{Q}{P}  
  :=
  \lift{(x)\substp{Q}{P}}{ R \psubstp{Q}{P} } \\
%   (\dropn{x})  \psubstp{Q}{P}       
%   := 
%   \left\{ 
%     \begin{array}{ccc} 
%       \dropn{\quotep{Q}} & & x \nameeq \quotep{P} \\
%       \dropn{x} & & otherwise \\
%     \end{array}
%   \right. 
  (\dropn{x})  \psubstp{Q}{P}       
  := 
  \left\{ 
    \begin{array}{ccc} 
      Q & & x \nameeq \quotep{P} \\
      \dropn{x} & & otherwise \\
    \end{array}
  \right.
\end{mathpar}
 

where

\begin{eqnarray}
  (x)\id{\{} \lpquote Q \rpquote / \lpquote P \rpquote \id{\}}            = 
  \left\{ 
    \begin{array}{ccc}
      \lpquote Q \rpquote & & x \nameeq \lpquote P \rpquote \\
      x & & otherwise \\
    \end{array}
  \right. \nonumber
\end{eqnarray}

and $z$ is chosen distinct from $\quotep{P}$, $\quotep{Q}$, the free
names in $Q$, and all the names in $R$. Our $\alpha$-equivalence will
be built in the standard way from this substitution.

\begin{remark}\label{rem:no_self_referential_names}
  One consequence of these definitions is that $\forall P. \quotep{P}
  \not\in \freenames{P}$.
\end{remark}

\subsection{ Dynamic quote: an example }

Anticipating something of what's to come, consider applying the
substitution, $\widehat{\id{\{}u / z \id{\}}}$, to the following pair
of processes, $\lift{w}{y!(z)}$ and $w[ \lpquote y!(z) \rpquote ]$.

\begin{eqnarray}
	\lift{w}{y!(z)}\widehat{\id{\{}u / z \id{\}}}
		& = &
		\lift{w}{y!(u)} \nonumber\\
	w[ \lpquote y!(z) \rpquote ] \widehat{ \id{\{}u / z \id{\}} }
		& = &
		w[ \lpquote y!(z) \rpquote ] \nonumber
\end{eqnarray}

Because the body of the process between quotes is impervious to
substitution, we get radically different answers. In fact, by
examining the first process in an input context,
e.g. $x?(z).\lift{w}{y!(z)}$, we see that the process under the lift
operator may be shaped by prefixed inputs binding a name inside it. In
this sense, the lift operator will be seen as a way to dynamically
construct processes before reifying them as names.

Finally equipped with these standard features we can present the
dynamics of the calculus.

\subsubsection{Operational semantics} 

Finally, we introduce the computational dynamics. What marks these
algebras as distinct from other more traditionally studied algebraic
structures, e.g. vector spaces or polynomial rings, is the manner in
which dynamics is captured. In traditional structures, dynamics is typically
expressed through morphisms between such structures, as in linear maps
between vector spaces or morphisms between rings. In algebras
associated with the semantics of computation, the dynamics is
expressed as part of the algebraic structure itself, through a
reduction reduction relation typically denoted by $\red$. Below, we
give a recursive presentation of this relation for the calculus used
in the encoding.

$\red \subseteq \pi \times \pi$
$\red : \pi \to \mathcal{P}(\pi)$

\begin{mathpar}
  \inferrule* [lab=Comm] { \textsf{match}( x_{src}, x_{trgt} ) } { x_{trgt}?(y)P \; | \; x_{src}!\langle {Q} \rangle \red P\{\quotep{Q}/y}\} }
  \and \\
  \inferrule* [lab=Par] {{P} \red {P}'} {{{P} | {Q}} \red {{P}' | {Q}}}
  \and
  \inferrule* [lab=Equiv]{{{P} \scong {P}'} \andalso {{P}' \red {Q}'} \andalso {{Q}' \scong {Q}}}{{P} \red {Q}}
\end{mathpar}

\begin{eqnarray*}
  match_{\equiv} (\quotep{P},\quotep{Q}) & := & P \equiv Q \\
  match_{\dagger}(\quotep{P},\quotep{Q}) & := & \forall R. P|Q \red^{*} R => R \red^{*} 0 \\
  match_{K}(\quotep{P},\quotep{Q}) & := & K \mbox{ for some context } K
\end{eqnarray*}

$u?(x)P | u!\langle Q \rangle \red P\{\quotep{Q}/x\}$

%We write $\wred$ for $\red^*$, and $P\red$ if $\exists Q $ such that $ P \red Q$.
We write $P\red$ if $\exists Q $ such that $ P \red Q$ and $P\not\red$, otherwise.

\section{Replication}

As mentioned before, it is known that replication (and hence
recursion) can be implemented in a higher-order process algebra
\cite{SangiorgiWalker}. As our first example of calculation with the
machinery thus far presented we give the construction explicitly in
the {\rhoc}.

\begin{eqnarray}
	D_{x} & := & \prefix{x}{y}{(\binpar{\outputp{x}{y}}{@{y}})} \nonumber\\
	\bangp_{x}{P} & := & \binpar{{x}!\langle{\binpar{D_{x}}{P}}\rangle}{D_{x}} \nonumber
\end{eqnarray}

\begin{eqnarray}
	\bangp_{x}{P} & & \nonumber\\
	=
	& {x}!\langle{(\prefix{x}{y}{(\outputp{x}{y} | @{y})) | P}}\rangle 
	      | \prefix{x}{y}{(\outputp{x}{y} | @{y})} & \nonumber\\
	\red
	& (\outputp{x}{y} | @{y})\substn{\quotep{(\prefix{x}{y}{(@{y} | \outputp{x}{y})) | P}}}{y} & \nonumber\\
	=
	& \outputp{x}{\quotep{(\prefix{x}{y}{(\outputp{x}{y} | @{y})) | P}}}
	  | {(\prefix{x}{y}{(\outputp{x}{y} | @{y})) | P}} & \nonumber\\
	\red
	& \ldots & \nonumber\\
	\red^*
	& P | P | \ldots & \nonumber
\end{eqnarray}

Of course, this encoding, as an implementation, runs away, unfolding
$\bangp{P}$ eagerly. A lazier and more implementable replication
operator, restricted to input-guarded processes, may be obtained as follows.

\begin{eqnarray}
\bangp{\prefix{u}{v}{P}} 
	:= 
	\binpar{\lift{x}{\prefix{u}{v}{(\binpar{D(x)}{P})}}}{D(x)} \nonumber
\end{eqnarray}

\begin{remark}
  Note that the lazier definition still does not deal with summation
  or mixed summation (i.e. sums over input and output). The reader is
  invited to construct definitions of replication that deal with these
  features. 

  Further, the definitions are parameterized in a name, $x$. Can you,
  gentle reader, make a definition that eliminates this parameter and
  guarantees no accidental interaction between the replication
  machinery and the process being replicated -- i.e. no accidental
  sharing of names used by the process to get its work done and the
  name(s) used by the replication to effect copying. This latter
  revision of the definition of replication is crucial to obtaining
  the expected identity $!!P \sim !P$.
\end{remark}

\begin{remark}\label{rem:paradoxical_combinator}
  The reader familiar with the lambda calculus will have noticed the
  similarity between $D$ and the paradoxical combinator.

  [Ed. note: the existence of this seems to suggest we have to be more
  restrictive on the set of processes and names we admit if we are to
  support no-cloning.]
\end{remark}

\subsubsection{Bisimulation}

The computational dynamics gives rise to another kind of equivalence,
the equivalence of computational behavior. As previously mentioned
this is typically captured \emph{via} some form of bisimulation.

% The notion we use in this paper is weak barbed bisimulation
% \cite{milner91polyadicpi}.

The notion we use in this paper is derived from weak barbed
bisimulation \cite{milner91polyadicpi}. 

\begin{definition}
An \emph{observation relation}, $\downarrow_{\mathcal N}$, over a set
of names, $\mathcal N$, is the smallest relation satisfying the rules
below.

\infrule[Out-barb]{y \in {\mathcal N}, \; x \nameeq y}
		  {\outputp{x}{v} \downarrow_{\mathcal N} x}
\infrule[Par-barb]{\mbox{$P\downarrow_{\mathcal N} x$ or $Q\downarrow_{\mathcal N} x$}}
		  {\binpar{P}{Q} \downarrow_{\mathcal N} x}

We write $P \Downarrow_{\mathcal N} x$ if there is $Q$ such that 
$P \wred Q$ and $Q \downarrow_{\mathcal N} x$.
\end{definition}

\begin{definition}
%\label{def.bbisim}
An  ${\mathcal N}$-\emph{barbed bisimulation} over a set of names, ${\mathcal N}$, is a symmetric binary relation 
${\mathcal S}_{\mathcal N}$ between agents such that $P\rel{S}_{\mathcal N}Q$ implies:
\begin{enumerate}
\item If $P \red P'$ then $Q \wred Q'$ and $P'\rel{S}_{\mathcal N} Q'$.
\item If $P\downarrow_{\mathcal N} x$, then $Q\Downarrow_{\mathcal N} x$.
\end{enumerate}
$P$ is ${\mathcal N}$-barbed bisimilar to $Q$, written
$P \wbbisim_{\mathcal N} Q$, if $P \rel{S}_{\mathcal N} Q$ for some ${\mathcal N}$-barbed bisimulation ${\mathcal S}_{\mathcal N}$.
\end{definition}

$\mathcal{R} \subseteq \pi \times \pi$

$P \mathcal{R} Q => \forall P'. P \red P' \Rightarrow \exists Q'. Q \red Q', P' \mathcal{R} Q'$

$P \vdash x \Rightarrow Q \vdash x$

\begin{mathpar}
  \inferrule*[lab=Out-barb]{x \nameeq y}{{y}!\langle{Q}\rangle \vdash x}
  \and
  \inferrule*[lab=Par-barb]{\mbox{$P\vdash x$ or $Q\vdash x$}}{\binpar{P}{Q} \vdash x}
\end{mathpar}

\subsubsection{Contexts}

One of the principle advantages of computational calculi like the
$\pi$-calculus is a well-defined notion of context,
contextual-equivalence and a correlation between
contextual-equivalence and notions of bisimulation. The notion of
context allows the decomposition of a process into (sub-)process and
its syntactic environment, its context. Thus, a context may be
thought of as a process with a ``hole'' (written $\Box$) in it. The
application of a context $M$ to a process $P$, written $M[P]$, is
tantamount to filling the hole in $M$ with $P$. In this paper we do
not need the full weight of this theory, but do make use of the notion
of context in the proof the main theorem. 

\begin{mathpar}
  \inferrule* [lab=summation] {} {{M_{M},M_{N}} \bc \Box \;|\; x.M_{A} \;|\; M_{M}+M_{N}}
  \and
  \inferrule* [lab=agent] {} {{M_{A}} \bc (\vec{x})M_{P} \;| \; \clift{P_0,\ldots,M_{P},\ldots,P_N}}
  \and \\
  \inferrule* [lab=process] {} {{M_{P}} \bc M_{N} \;| \;P|M_{P} }
\end{mathpar} 

\begin{mathpar}
  \inferrule* [lab=sychronization] {} {M_{N} \bc \Box \;|\; x?M_{F} \;|\; x!M_{C}}
  \and
  \inferrule* [lab=abstraction] {} {{M_{F}} \bc (x)M_{P} }
  \and
  \inferrule* [lab=concretion] {} {{M_{C}} \bc \langle M_{P} \rangle }
  \and \\
  \inferrule* [lab=process] {} {{M_{P}} \bc M_{N} \;| \;P|M_{P} }
\end{mathpar}

\begin{definition}[contextual application] Given a context $M$, and
  process $P$, we define the \emph{contextual application}, $M[P] :=
  M\{P/\Box\}$. That is, the contextual application of M to P is the
  substitution of $P$ for $\Box$ in $M$.
\end{definition}

$\meaningof{-} : L \to \mathcal{P}(\pi)$

\begin{mathpar}
  \inferrule* [lab=collection] {} {\meaningof{true} = \pi, \and \meaningof{~E} = \pi \setminus \meaningof{E}, \and \meaningof{E_{1} \& E_{2}} = \meaningof{E_{1}} \cap \meaningof{E_{2}}}
\end{mathpar}

\begin{mathpar}
  \inferrule* [lab=structure] {} {\meaningof{0} = \{ P \in \pi | P \equiv 0 \}, \and \\ \meaningof{E_1 | E_2} = \{ P \in \pi | P \equiv P_{1} | P_{2}, P_{1} \in \meaningof{E_{1}}, P_{2} \in \meaningof{E_2}\} }
\end{mathpar}

\begin{mathpar}
 \inferrule* [lab=behavior] {} {\meaningof{\langle a?b \rangle E} = \{ P \in \pi | P \equiv Q | u?(y)P', \\ \and \\\\ \and \\ \;\;\; u \in \meaningof{a}, \forall z.P'\{z/y\} \in \meaningof{E\{z/b\}}\}, \and \\ \meaningof{a!E} = \{ P \in \pi | P \equiv Q | x!\langle P' \rangle, x \in \meaningof{a} P' \in \meaningof{E}\} }
\end{mathpar}

\begin{mathpar}
 \inferrule* [lab=nominal] {} {\meaningof{\quotep{E}} = \{ \quotep{P} \in \quotep{\pi} | P \in \meaningof{E} \}, \and \meaningof{\quotep{P}} = \{ \quotep{Q} \in \quotep{\pi} | P \equiv Q \} \and \\ \meaningof{@\quotep{E}} = \{ P \in \pi | P \equiv @x, x \in \meaningof{E} \}}
\end{mathpar}

\begin{eqnarray*}
  \\
  \meaningof{-} : TS \to ST
\end{eqnarray*}

\begin{eqnarray*}
  \\
  L : TS \to ST
\end{eqnarray*}

\begin{eqnarray*}
  \\
  P \models E \iff P \in \meaningof{E}
\end{eqnarray*}

\begin{eqnarray*}
  P \approx_{L} Q \iff \forall E \in L. P \models E \iff Q \models E
\end{eqnarray*}

\begin{eqnarray*}
  P \approx_{K} Q
\end{eqnarray*}

\begin{eqnarray*}
  P \approx Q
\end{eqnarray*}

$\approx_{K} = \approx = \approx_{L}$

\subsubsection{Contextual duality}

Note that contexts extend the quotation operation to a family of
operations from processes to names. Given a context, $M$, we can
define a \emph{nominal context}, $\quotep{M}$ by $\quotep{M}[P] :=
\quotep{M[P]}$. To foreshadow what is to come we observe that these
operations enjoy a duality with processes very much like the duality
between vectors and maps from vectors to scalars.

Further, because the calculus is essentially higher-order, we have a
correspondence between contexts and processes. More specifically,
given a name $x$ and a context $M$ we can construct $M^{*}_{x}$ such
that 

\begin{mathpar}
  M^{*}_{x} | \lift{x}{P} \red M[P]
\end{mathpar}

namely,

\begin{mathpar}
  M^{*}_{x} := x?(u).M[\dropn{u}]
\end{mathpar}

The dependence of $M^{*}_{x}$ on a name makes it an abstraction, 

\begin{mathpar}
  M^{*} := (x)x?(u).M[\dropn{u}]
\end{mathpar}

\subsection{Additional notation}

It will sometimes be convenient to denote the process a name
quotes. We already have the notation $x = \quotep{P}$, but it will be
convenient to introduce an alternate notation, $\procn{x}$, when we
want to emphasize the connection to the use of the name. Note that, by
virtue of name equivalence, $\quotep{\procn{x}} \nameeq x$; so, the
notation is consistent with previous definitions.

Further, because names have structure it is possible to effect
substitutions on the basis of that structure. This means we need to
upgrade our notation for substitutions, which we accomplish by
adapting comprehension notation. Thus,

\begin{mathpar}
  P\{ y / x : x \in S \}
\end{mathpar}

is interpreted to mean the process derived from P by replacing (in a
capture-avoiding manner) each occurrence of $x$ in $S$ by $y$. For example,

\begin{mathpar}
  P\{ \quotep{\procn{x}|\procn{x}} / x : x \in \freenames{P} \}
\end{mathpar}

will replace each (occurrence) of a free name $x$ in $P$ by
$\quotep{\procn{x}|\procn{x}}$.

Also, we will avail ourselves of the notation $x^{L}$ and $x^{R}$ to
denote injections of a name into disjoint copies of the name
space. There are numerous ways to accomplish this. One example can be
found in \cite{MeredithR05}. This notation overloads to vectors of
names: $\vec{x}^{\pi} := (x_{i}^{\pi} \; : \; 0 \leq i < |\vec{x}| )$ where $\pi \in \{L,R\}$.

We also use $P^{\Box} := P|\Box$.

In \cite{MeredithR05} an interpretation of the new operator is
given. It turns out that there are several possible interpretations
all enjoying the requisite algebraic properties of the operator (see
\cite{milner91polyadicpi}). We will therefore make liberal use of
$(\nu\; \vec{x})P$.

% subsection the_syntax_and_semantics_of_the_notation_system (end)   

\section{Interpretation of QM}
\subsection{Supporting definitions}
\subsubsection{Multiplication}
\begin{mathpar}
  \quotep{Q} \cdot \quotep{R} := \quotep{Q|R}
  \and \\
  \quotep{Q} \cdot P := P\{ \quotep{Q|R} / \quotep{R} : \quotep{R} \in \freenames{P} \}
\end{mathpar}

\paragraph{Discussion}
The first line needs little explanation. The second line says that
each free name of the process is replaced with the multiplication of
that name by the scalar. Multiplication of a scalar (name) by a state
(process) results in a process all the names of which have been `moved
over' by parallel composition with the process the scalar
quotes. There is a subtlety that the bound names have to be
manipulated so that multiplied names aren't accidentally
captured. There are many ways to achieve this.

\begin{remark}\label{rem:multiplication_identities}
  The reader is invited to verify that for all $x,y,z \in \QProc$ and $P \in \Proc$
  \begin{mathpar}
    x \cdot \quotep{0} \equiv x 
    \and
    x \cdot y \equiv y \cdot x
    \and
    x \cdot (y \cdot z) \equiv (x \cdot y) \cdot z
    \and \\
    \quotep{0} \cdot P \equiv P
    \and \\
    x \cdot (y \cdot P) \equiv (x \cdot y) \cdot P
    \and \\
    x \cdot (P|Q) \equiv (x \cdot P) | (x \cdot Q)
    \and \\    
  \end{mathpar}
\end{remark}

\subsubsection{Tensor product}

We define a tensor product on processes by structural induction.

\paragraph{Tensor of sums} First note that all summations, including
$\pzero$ and sequence, can be written $\Sigma_{i} x_{i}.A_{i} +
\Sigma_{j} x_{j}.C_{j}$, where we have grouped input-guarded processes
together and output-guarded processes together.

Thus, we can define the tensor product of two summations, $N_{1}\otimes N_{2}$, where

\begin{mathpar}
  N_{1} := \Sigma_{i} x_{i}.A_{i} + \Sigma_{j} x_{j}.C_{j}
  \and
  N_{2} := \Sigma_{i'} y_{i'}.B_{i'} + \Sigma_{j'} y_{j'}.D_{j'} 
\end{mathpar}

as follows.

\begin{mathpar}
  \Sigma_{i} x_{i}.A_{i} + \Sigma_{j} x_{j}.C_{j} \otimes \Sigma_{i'}
  y_{i'}.B_{i'} + \Sigma_{j'} y_{j'}.D_{j'} 
  \and \\
  := \; \Sigma_{i} \Sigma_{i'} \quotep{\stackrel{\vee}{x_{i}}| \stackrel{\vee}{y_{i'}}}.(A_{i}\otimes B_{i'}) \; | \; \Sigma_{i'} \Sigma_{i} \quotep{\stackrel{\vee}{y_{i'}}|\stackrel{\vee}{x_{i}}}.(B_{i'}\otimes A_{i})
  \and
  \;\; | \;\; \Sigma_{j} \Sigma_{j'} \quotep{\stackrel{\vee}{x_{j}}|\stackrel{\vee}{y_{j'}}}.(A_{j}\otimes B_{j'}) \; | \; \Sigma_{j'} \Sigma_{j} \quotep{\stackrel{\vee}{y_{j'}}|\stackrel{\vee}{x_{j}}}.(B_{j'}\otimes A_{j})
\end{mathpar}

\begin{remark}
  Do we need to $x^{L}$ and $y^{R}$ for this construction as well?
\end{remark}

\paragraph{Tensor of parallel compositions} Next, we distribute tensor
over par.

\begin{mathpar}
  P_{1}|P_{2} \otimes Q_{1}|Q_{2} := (P_{1} \otimes Q_{1}) | (P_{1}
  \otimes Q_{2}) | (P_{2} \otimes Q_{1}) | (P_{2} \otimes Q_{2})
\end{mathpar}

\paragraph{Tensor with dropped names} We treat tensor of a
process with a dropped name as parallel composition.

\begin{mathpar}
  P \otimes \dropn{x} := P | \dropn{x}
\end{mathpar}

\paragraph{Tensor of agents}

Finally, we need to define tensor on agents. Note that the definition
of tensor on normal products only tensors inputs with inputs and
outputs with outputs. Thus, we only have to define the operation on
``homogeneous'' pairings.

\begin{mathpar}
  (\vec{x})P \otimes (\vec{y})Q
  \and \\
  := (x_{0}^{L}|y_{0}^{R},\ldots,x_{0}^{L}|y_{n}^{R},\ldots,x_{m}^{L}|y_{0}^{R},\ldots,x_{m}^{L}|y_{n}^R)(P\{ \vec{x}^{L}/\vec{x}\} \otimes Q \{ \vec{y}^{R}/\vec{y}\})
  \and \\
  \clift{\vec{P}} \otimes \clift{\vec{Q}}
  \and \\
  := \clift{P_{0}\otimes Q_{0},\ldots,P_{0}\otimes Q_{n},\ldots,P_{m}\otimes Q_{0},\ldots,P_{m}\otimes Q_{n}}
\end{mathpar}

\begin{remark}
  Observe that arities of tensored abstractions matches arities of
  tensored concretions if the original arities matched. Note also that
  the length of the arities corresponds to the increase in dimension
  we see in ordinary vector space tensor product.
\end{remark}

\begin{remark}
  Operationally, this definition distributes the tensor down to
  components ``linked'' by summation. Tensor over summation is
  intriguing in that it mixes names. Moreover, as a consequence of the
  way it mixes names we have the identities for all $x \in \QProc$ and
  $P,Q \in \Proc$

  \begin{mathpar}
    (x \cdot P) \otimes Q \equiv x \cdot (P \otimes Q) \equiv P \otimes (x \cdot Q)
    \and
    P \otimes \pzero \equiv P
  \end{mathpar}

  that the reader is invited to verify.
\end{remark}

\subsubsection{Annihilation}
\begin{mathpar}
  P^{\perp} := \{ Q | \forall R. P|Q \red^{*} R \Rightarrow R \red^{*} \pzero \}
  \and \\
  P^{\underline{\perp}} := \Sigma_{Q \in P^{\perp}} \quotep{Q}?(y).(\dropn{y}|Q) | \Sigma_{Q \in P^{\perp}} \quotep{Q}\clift{\Box}
\end{mathpar}

\paragraph{Discussion} The reader will note that $P^{\perp}$ is a
\emph{set} of processes, while $P^{\underline{\perp}}$ is a
\emph{context}. We call the set $P^{\perp}$ the \emph{annihilators} of
$P$. The parallel composition of a process in the annihilators of $P$
with $P$ will result in a process, the state space of which has all
paths eventually leading to $\pzero$. Execution may endure loops; but
under reasonable conditions of fairness (naturally guaranteed under
most notions of bisimulation) such a composite process cannot get
stuck in such a loop and will, eventually pop out and terminate.

The context $P^{\underline{\perp}}$ is ready and willing to ``take the
$P$ out of'' the process to which it is applied. It will effectively
transmit the code of the process to which it is applied to one of the
annihilators and run the process against it.

\subsubsection{Evaluation}
We fix $M$ a domain of fully abstract interpretation with an equality
coincident with bisimulation. We take $\meaningof{\cdot} : \Proc \to
M$ to be the map interpreting processes and $\nmeaningof{\cdot} : \M
\to Proc$ to be the map running the other way. Then we define

\begin{mathpar}
  \int P := \nmeaningof{\meaningof{P}}
\end{mathpar}

\paragraph{Discussion}
There are many fully abstract interpretations of Milner's
$\pi$-calculus. Any of them can be used as a basis for interpreting
the reflective calculus here. Equipped with such a domain it is
largely a matter of grinding through to check that the Yoneda
construction for the normalization-by-evaluation program can be
extended to this setting.

\begin{remark}
  The reader is invited to verify that $\int (P^{\underline{\perp}}[P]) = 0$.
\end{remark}

\subsection{Quantum mechanics}

Table \ref{tbl:core_qm_op_defns} gives the core operational definitions

\begin{table}[htp]\label{tbl:core_qm_op_defns}
  \center{
    \fbox{
      \begin{tabular}{c|c}
        quantum mechanics & process calculus \\
        \hline
        scalar & $x := \quotep{P}$ \\
        state vector & $\state{P} := P$ \\
        dual & $\state{P}^{*} := \event{P^{\underline{\perp}}} := \quotep{P^{\underline{\perp}}}[-]$ \\
        matrix & $ \Sigma_{\alpha} \state{P_{\alpha}}x_{\alpha}\event{Q_{\alpha}}$ \\
        vector addition & $\state{P} + \state{Q} := \state{P | Q}$ \\
        tensor product & $\state{P} \otimes \state{Q} := \state{P \otimes Q}$ \\
        inner product & $\innerprod{P}{Q} := \quotep{\int P^{\underline{\perp}}[Q]}$ \\
      \end{tabular}
    }
  }
  \caption{QM - operational definitions}
\end{table}

where

\begin{mathpar}
  \prmatrix{P}{Q} := \fprmatrix{P}{\quotep{\pzero}}{Q}
  \and
  \fprmatrix{P}{x}{Q} := (\state{P},x,\event{Q})
  \and
  (\fprmatrix{P}{x}{Q})(\state{R}) := x \cdot \innerprod{Q}{R} \cdot \state{P}
  \and
  (\fprmatrix{P}{x}{Q})(\event{R}) := x \cdot \innerprod{R}{P} \cdot \event{Q}
\end{mathpar}

\paragraph{Discussion}
As promised: vectors (aka states) are represented as processes; duals
as contextual duals; inner product definition should be compared with
standard inner product definition for ....

\begin{remark}
  Assuming $\int (P^{\underline{\perp}}[P]) = 0$, the reader is
  invited to verify that $(\fprmatrix{P}{x}{P})(\state{P}) = x \cdot \state{P}$.
\end{remark}

\begin{remark}
  The reader is invited to verify that $\innerprod{P}{Q}$ could
  equally well have been written $\quotep{\int \stackrel{\vee}{x}}$
  where $x = \event{P^{\underline{\perp}}}(Q)$.

  One of the motivations for this remark is that there is another way
  to factor these operations. We could package up evaluation in the dual:

  \begin{mathpar}
    \state{P}^{*} := \event{\int P^{\underline{\perp}}} := \quotep{\int P^{\underline{\perp}}}[-]
  \end{mathpar}

  and then have inner product defined by
  
  \begin{mathpar}
    \innerprod{P}{Q} := \event{P}(Q)
  \end{mathpar}

  Hopefully, experience with the calculations will provide guidance on
  the best factoring.
\end{remark}

\begin{remark}
  Assuming $\int (P^{\underline{\perp}}[P]) = 0$, the reader is
  invited to verify that $\forall P,Q. (\prmatrix{0}{Q})(\state{0}) =
  \state{0}$ and dually $(\prmatrix{P}{0})(\event{0}) = \event{0}$.
\end{remark}

\begin{remark}
  i'm a little worried that i don't (yet) have proper support for
  complex conjugacy. But, the observation above may give us a
  clue. According to Abramsky, it must be the case that the scalars
  are iso to the homset of the identity for the tensor -- which the
  observation above characterizes. 

  For now, we will simply bookmark the notion with $\overline{x}$.
\end{remark}

\subsubsection{Adjointness}

We need to give a definition of $(\cdot)^{\dagger}$ for matrices. The
obvious candidate definition is
\begin{mathpar}
(\Sigma_{\alpha}\fprmatrix{P_{\alpha}}{x_{\alpha}}{Q_{\alpha}})^{\dagger}
= \Sigma_{\alpha}\fprmatrix{(Q_{\alpha}^{\underline{\perp}})^{*}}{\overline{x}_{\alpha}}{P_{\alpha}^{\underline{\perp}}} 
\end{mathpar}

But, $(Q_{\alpha}^{\underline{\perp}})^{*}$ requires a name along
which to communicate the process to achieve the context application.

\subsubsection{Basis for a basis}
If processes label states and ``addition'' of states (a.k.a. vector
addition) is interpreted as parallel composition, what corresponds to
notions of linear independence and basis? Here, we recall that Yoshida
has developed a set of \emph{combinators} for an asynchronous verison
of Milner's $\pi$-calculus. These are a finite set of processes such
any process can be expressed as parallel composition of these
combinators together with liberal uses of the new operator and
replication. We can simply give a translation of these into the
present calculus and have reasonable expectation that the property
carries over. That is, that the resultant set allows to express all
processes via parallel composition. Note, however, that there is no
new operator or replication in this calculus. As a result, we expect
that the corresponding set is actually infinite. That is, we expect
that the space is actually infinite dimensional.

\begin{remark}
  The attentive reader may be a bit concerned. Certainly, the
  collection $S$, $K$ and $I$ is a finite set of
  combinators. Shouldn't we expect to see a finite set of combinators
  for an effectively equivalent system? i am very sympathetic to this
  critique and feel it warrants full attention. On the other hand, i
  also have in mind the following analogy. The natural numbers, as a
  monoid under addition, has exactly $1$ generator, while the natural
  numbers, as a monoid under multiplication, has countably many
  generators (the primes). We observe that the application of the
  lambda calculus is much less resource sensitive than the parallel
  composition of the $\pi$-calculus. Could it be the case that we have
  an analogy of the form
  
  \begin{mathpar}
    m + n : MN :: m*n : M|N
  \end{mathpar}

  giving a similar blow up in the set of ``primes''?  This is such a
  wonderful thought that, even if it's not true, i think it's worth
  writing down.
\end{remark}
 

\documentclass[12pt]{llncs}
%\documentclass{jktr}

\usepackage[pdftex]{hyperref}                   
\usepackage {listings}
\usepackage {mathpartir}
\usepackage{bcprules}
%\usepackage{listings}
                       
\usepackage{graphicx} 
%\usepackage[margins=2.5cm,nohead,nofoot]{geometry}
%\usepackage{geometry}
\usepackage{amsfonts}
\usepackage{amstext}
\usepackage{latexsym}
\usepackage{amssymb}
\usepackage{color}


%\include{myPreamble}
\include{qm2pi.local} 

%\ifpdf
%\usepackage[pdftex]{graphicx}
%\else
%\usepackage{graphicx}
%\fi

 % \ifpdf
%  \usepackage{pdfsync}
%  \if


%\title{Brief Article}
%\author{David F. Snyder}
%\author{L.G. Meredith}

%\address{Dept. of Math., Texas State University--San Marcos, San Marcos, TX 78666}
       
\pagestyle{empty}


\begin{document}

\lstset{language=[Objective]Caml,frame=shadowbox}

\input{qm2pi.front}

% section front matter (end)

\input{qm2pi.intro} 
 
% section introduction (end)

% \input{qm2pi.knotations} 

% section notation (end)

\input{qm2pi.process.calculi} 

% section concurrent_process_calculi_and_spatial_logics_ (end)
    
%\input{qm2pi.knots2pi} 

%\input{qm2pi.trefoil} 

%\input{qm2pi.mainthm} 

% subsection basic_interpretation (end)

%\input{qm2pi.rho.presentation} 
\subsection{The syntax and semantics of the notation system}\label{sub:the_syntax_and_semantics_of_the_notation_system} % (fold)

We now summarize a technical presentation of the calculus that
embodies our theory of dynamics. The typical presentation of such a
calculus follows the style of giving generators and relations on
them. The grammar, below, describing term constructors, freely
generates the set of processes, $\Proc$. This set is then quotiented
by a relation known as structural congruence and it is over this set
that the notion of dynamics is expressed. This presentation is
essentially that of \cite{MeredithR05} with the addition of
polyadicity and summation. For readability we have relegated some of
the technical subtleties to an appendix.

\subsubsection{Process grammar}\label{subsub:process_grammar}

\begin{mathpar}
  \inferrule* [lab=synchronization] {} {{M} \bc \pzero \;|\; x?F \;|\; x!C }
  \and
  \inferrule* [lab=abstraction] {} {{F} \bc (x)P}
  \and
  \inferrule* [lab=concretion] {} {{C} \bc \langle Q \rangle}
  \and
  \inferrule* [lab=process] {} {{P,Q} \bc M \;| \;P|Q \;|\; @{x}}
  \and
  \inferrule* [lab=name] {} {{x} \bc \quotep{P}}
\end{mathpar} 

Note that $\vec{x}$ (resp. $\vec{P}$) denotes a vector of names
(resp. processes) of length $|\vec{x}|$ (resp. $|\vec{P}|$). We adopt
the following useful abbreviations.

\begin{mathpar}
   x?(\vec{y}).P := x.(\vec{y})P \and  x\clift{\vec{P}} := x.\clift{\vec{P}}
   \and x!(y) := \lift{x}{\dropn{y}}
   \and \Pi_{i=0}^{n-1}P_i := P_0 | \ldots | P_{n-1}
\end{mathpar}

\subsubsection{Structural congruence}

\paragraph{Free and bound names and alpha-equivalence.} At the
core of structural equivalence is alpha-equivalence which identifies
process that are the same up to a change of variable. Formally, we
recognize the distinction between free and bound names. The free names
of a process, $\freenames{P}$, may be calculated recursively as
follows:

\begin{mathpar}
\freenames{\pzero} := \emptyset
  \and \\
  \freenames{x?(y).P} := \{ x \} \cup (\freenames{P} \setminus \{ y \})
  \and 
  \freenames{x!\langle P \rangle} := \{ x \} \cup \{ P \} 
  \and \\
  \freenames{P|Q} := \freenames{P} \cup \freenames{Q}
  \and \\
  \freenames{@{x}} := \{ x \}
\end{mathpar}

$\pi$
$\quotep{\pi}$

$\freenames{-} : \pi \to \mathcal{P}(\quotep{\pi})$

\begin{eqnarray*}
  \freenames{\pzero} & := & \emptyset \\
  \freenames{x?(y).P} & := & \{ x \} \cup (\freenames{P} \setminus \{ y \}) \\
  \freenames{x!\langle P \rangle} & := & \{ x \} \cup \{ P \} \\
  \freenames{P|Q} & := & \freenames{P} \cup \freenames{Q} \\
  \freenames{\dropn{x}} & := & \{ x \}
\end{eqnarray*}

The bound names of a process, $\boundnames{P}$, are those names occurring in $P$
that are not free. For example, in $x?(y).0$, the name $x$ is free, while $y$ is bound.

\begin{mathpar}
  \inferrule* [lab=monoidal-laws] {} { P|Q \equiv Q|P \and P|0 \equiv P \and P|(Q|R) \equiv (P|Q)|R }
\end{mathpar}

\begin{mathpar}
  \inferrule* [lab=alpha-equivalence] {} { (x)P \equiv (y)P\{y/x\} \and y \not\in \freenames{P} }
\end{mathpar}

\begin{definition}
Then two processes, $P,Q$, are alpha-equivalent if $P = Q\{\vec{y}/\vec{x}\}$ for
some $\vec{x} \in \boundnames{Q},\vec{y} \in \boundnames{P}$, where $Q\{\vec{y}/\vec{x}\}$
denotes the capture-avoiding substitution of $\vec{y}$ for $\vec{x}$ in $Q$.
\end{definition}

\begin{definition}
  The {\em structural congruence} \cite{SangiorgiWalker} , $\equiv$,
  between processes is the least congruence containing
  alpha-equivalence, satisfying the abelian monoid laws
  (associativity, commutativity and $\pzero$ as identity) for parallel
  composition $|$ and for summation $+$.
\end{definition}

\subsection{Name equivalence}

We take name equivalence, written $\nameeq$, to be the smallest
equivalence relation generated by the following rules.

\begin{mathpar}
\inferrule*[lab=Quote-drop]
{ }
{ \quotep{@{x}} \nameeq x }

\inferrule*[lab=Struct-equiv]
{ P \scong Q }
{ \quotep{P} \nameeq \quotep{Q} }
\end{mathpar}

The astute reader will have noticed that the mutual recursion of names
and processes imposes a mutual recursion on alpha-equivalence and
structural equivalence via name-equivalence. Fortunately, all of this
works out pleasantly and we may calculate in the natural way, free of
concern. The reader interested in the details is referred to the
appendix \ref{appendix:rho_details}.

\subsection{Substitution}

We use $\Proc$ for the set of processes, $\QProc$ for the set of
names, and $\id{\{}\vec{y} / \vec{x} \id{\}}$ to denote partial maps,
$s : \QProc \rightarrow \QProc$. A map, $s$ lifts, uniquely, to a map
on process terms, $\widehat{s} : \Proc \rightarrow \Proc$ by the
following equations.

\begin{mathpar}
  (0) \psubstp{Q}{P} := 0 \\
  (R \juxtap S) \psubstp{Q}{P}
  :=    
  (R)\psubstp{Q}{P} \juxtap (S) \psubstp{Q}{P} \\
  (x?(y).R) \psubstp{Q}{P}    
  :=    
  (x)\substp{Q}{P} (z)\concat( (R \psubstn{z}{y}) \psubstp{Q}{P} ) \\
  (\lift{x}{R}) \psubstp{Q}{P}  
  :=
  \lift{(x)\substp{Q}{P}}{ R \psubstp{Q}{P} } \\
%   (\dropn{x})  \psubstp{Q}{P}       
%   := 
%   \left\{ 
%     \begin{array}{ccc} 
%       \dropn{\quotep{Q}} & & x \nameeq \quotep{P} \\
%       \dropn{x} & & otherwise \\
%     \end{array}
%   \right. 
  (\dropn{x})  \psubstp{Q}{P}       
  := 
  \left\{ 
    \begin{array}{ccc} 
      Q & & x \nameeq \quotep{P} \\
      \dropn{x} & & otherwise \\
    \end{array}
  \right.
\end{mathpar}
 

where

\begin{eqnarray}
  (x)\id{\{} \lpquote Q \rpquote / \lpquote P \rpquote \id{\}}            = 
  \left\{ 
    \begin{array}{ccc}
      \lpquote Q \rpquote & & x \nameeq \lpquote P \rpquote \\
      x & & otherwise \\
    \end{array}
  \right. \nonumber
\end{eqnarray}

and $z$ is chosen distinct from $\quotep{P}$, $\quotep{Q}$, the free
names in $Q$, and all the names in $R$. Our $\alpha$-equivalence will
be built in the standard way from this substitution.

\begin{remark}\label{rem:no_self_referential_names}
  One consequence of these definitions is that $\forall P. \quotep{P}
  \not\in \freenames{P}$.
\end{remark}

\subsection{ Dynamic quote: an example }

Anticipating something of what's to come, consider applying the
substitution, $\widehat{\id{\{}u / z \id{\}}}$, to the following pair
of processes, $\lift{w}{y!(z)}$ and $w[ \lpquote y!(z) \rpquote ]$.

\begin{eqnarray}
	\lift{w}{y!(z)}\widehat{\id{\{}u / z \id{\}}}
		& = &
		\lift{w}{y!(u)} \nonumber\\
	w[ \lpquote y!(z) \rpquote ] \widehat{ \id{\{}u / z \id{\}} }
		& = &
		w[ \lpquote y!(z) \rpquote ] \nonumber
\end{eqnarray}

Because the body of the process between quotes is impervious to
substitution, we get radically different answers. In fact, by
examining the first process in an input context,
e.g. $x?(z).\lift{w}{y!(z)}$, we see that the process under the lift
operator may be shaped by prefixed inputs binding a name inside it. In
this sense, the lift operator will be seen as a way to dynamically
construct processes before reifying them as names.

Finally equipped with these standard features we can present the
dynamics of the calculus.

\subsubsection{Operational semantics} 

Finally, we introduce the computational dynamics. What marks these
algebras as distinct from other more traditionally studied algebraic
structures, e.g. vector spaces or polynomial rings, is the manner in
which dynamics is captured. In traditional structures, dynamics is typically
expressed through morphisms between such structures, as in linear maps
between vector spaces or morphisms between rings. In algebras
associated with the semantics of computation, the dynamics is
expressed as part of the algebraic structure itself, through a
reduction reduction relation typically denoted by $\red$. Below, we
give a recursive presentation of this relation for the calculus used
in the encoding.

$\red \subseteq \pi \times \pi$
$\red : \pi \to \mathcal{P}(\pi)$

\begin{mathpar}
  \inferrule* [lab=Comm] { \textsf{match}( x_{src}, x_{trgt} ) } { x_{trgt}?(y)P \; | \; x_{src}!\langle {Q} \rangle \red P\{\quotep{Q}/y}\} }
  \and \\
  \inferrule* [lab=Par] {{P} \red {P}'} {{{P} | {Q}} \red {{P}' | {Q}}}
  \and
  \inferrule* [lab=Equiv]{{{P} \scong {P}'} \andalso {{P}' \red {Q}'} \andalso {{Q}' \scong {Q}}}{{P} \red {Q}}
\end{mathpar}

\begin{eqnarray*}
  match_{\equiv} (\quotep{P},\quotep{Q}) & := & P \equiv Q \\
  match_{\dagger}(\quotep{P},\quotep{Q}) & := & \forall R. P|Q \red^{*} R => R \red^{*} 0 \\
  match_{K}(\quotep{P},\quotep{Q}) & := & K \mbox{ for some context } K
\end{eqnarray*}

$u?(x)P | u!\langle Q \rangle \red P\{\quotep{Q}/x\}$

%We write $\wred$ for $\red^*$, and $P\red$ if $\exists Q $ such that $ P \red Q$.
We write $P\red$ if $\exists Q $ such that $ P \red Q$ and $P\not\red$, otherwise.

\section{Replication}

As mentioned before, it is known that replication (and hence
recursion) can be implemented in a higher-order process algebra
\cite{SangiorgiWalker}. As our first example of calculation with the
machinery thus far presented we give the construction explicitly in
the {\rhoc}.

\begin{eqnarray}
	D_{x} & := & \prefix{x}{y}{(\binpar{\outputp{x}{y}}{@{y}})} \nonumber\\
	\bangp_{x}{P} & := & \binpar{{x}!\langle{\binpar{D_{x}}{P}}\rangle}{D_{x}} \nonumber
\end{eqnarray}

\begin{eqnarray}
	\bangp_{x}{P} & & \nonumber\\
	=
	& {x}!\langle{(\prefix{x}{y}{(\outputp{x}{y} | @{y})) | P}}\rangle 
	      | \prefix{x}{y}{(\outputp{x}{y} | @{y})} & \nonumber\\
	\red
	& (\outputp{x}{y} | @{y})\substn{\quotep{(\prefix{x}{y}{(@{y} | \outputp{x}{y})) | P}}}{y} & \nonumber\\
	=
	& \outputp{x}{\quotep{(\prefix{x}{y}{(\outputp{x}{y} | @{y})) | P}}}
	  | {(\prefix{x}{y}{(\outputp{x}{y} | @{y})) | P}} & \nonumber\\
	\red
	& \ldots & \nonumber\\
	\red^*
	& P | P | \ldots & \nonumber
\end{eqnarray}

Of course, this encoding, as an implementation, runs away, unfolding
$\bangp{P}$ eagerly. A lazier and more implementable replication
operator, restricted to input-guarded processes, may be obtained as follows.

\begin{eqnarray}
\bangp{\prefix{u}{v}{P}} 
	:= 
	\binpar{\lift{x}{\prefix{u}{v}{(\binpar{D(x)}{P})}}}{D(x)} \nonumber
\end{eqnarray}

\begin{remark}
  Note that the lazier definition still does not deal with summation
  or mixed summation (i.e. sums over input and output). The reader is
  invited to construct definitions of replication that deal with these
  features. 

  Further, the definitions are parameterized in a name, $x$. Can you,
  gentle reader, make a definition that eliminates this parameter and
  guarantees no accidental interaction between the replication
  machinery and the process being replicated -- i.e. no accidental
  sharing of names used by the process to get its work done and the
  name(s) used by the replication to effect copying. This latter
  revision of the definition of replication is crucial to obtaining
  the expected identity $!!P \sim !P$.
\end{remark}

\begin{remark}\label{rem:paradoxical_combinator}
  The reader familiar with the lambda calculus will have noticed the
  similarity between $D$ and the paradoxical combinator.

  [Ed. note: the existence of this seems to suggest we have to be more
  restrictive on the set of processes and names we admit if we are to
  support no-cloning.]
\end{remark}

\subsubsection{Bisimulation}

The computational dynamics gives rise to another kind of equivalence,
the equivalence of computational behavior. As previously mentioned
this is typically captured \emph{via} some form of bisimulation.

% The notion we use in this paper is weak barbed bisimulation
% \cite{milner91polyadicpi}.

The notion we use in this paper is derived from weak barbed
bisimulation \cite{milner91polyadicpi}. 

\begin{definition}
An \emph{observation relation}, $\downarrow_{\mathcal N}$, over a set
of names, $\mathcal N$, is the smallest relation satisfying the rules
below.

\infrule[Out-barb]{y \in {\mathcal N}, \; x \nameeq y}
		  {\outputp{x}{v} \downarrow_{\mathcal N} x}
\infrule[Par-barb]{\mbox{$P\downarrow_{\mathcal N} x$ or $Q\downarrow_{\mathcal N} x$}}
		  {\binpar{P}{Q} \downarrow_{\mathcal N} x}

We write $P \Downarrow_{\mathcal N} x$ if there is $Q$ such that 
$P \wred Q$ and $Q \downarrow_{\mathcal N} x$.
\end{definition}

\begin{definition}
%\label{def.bbisim}
An  ${\mathcal N}$-\emph{barbed bisimulation} over a set of names, ${\mathcal N}$, is a symmetric binary relation 
${\mathcal S}_{\mathcal N}$ between agents such that $P\rel{S}_{\mathcal N}Q$ implies:
\begin{enumerate}
\item If $P \red P'$ then $Q \wred Q'$ and $P'\rel{S}_{\mathcal N} Q'$.
\item If $P\downarrow_{\mathcal N} x$, then $Q\Downarrow_{\mathcal N} x$.
\end{enumerate}
$P$ is ${\mathcal N}$-barbed bisimilar to $Q$, written
$P \wbbisim_{\mathcal N} Q$, if $P \rel{S}_{\mathcal N} Q$ for some ${\mathcal N}$-barbed bisimulation ${\mathcal S}_{\mathcal N}$.
\end{definition}

$\mathcal{R} \subseteq \pi \times \pi$

$P \mathcal{R} Q => \forall P'. P \red P' \Rightarrow \exists Q'. Q \red Q', P' \mathcal{R} Q'$

$P \vdash x \Rightarrow Q \vdash x$

\begin{mathpar}
  \inferrule*[lab=Out-barb]{x \nameeq y}{{y}!\langle{Q}\rangle \vdash x}
  \and
  \inferrule*[lab=Par-barb]{\mbox{$P\vdash x$ or $Q\vdash x$}}{\binpar{P}{Q} \vdash x}
\end{mathpar}

\subsubsection{Contexts}

One of the principle advantages of computational calculi like the
$\pi$-calculus is a well-defined notion of context,
contextual-equivalence and a correlation between
contextual-equivalence and notions of bisimulation. The notion of
context allows the decomposition of a process into (sub-)process and
its syntactic environment, its context. Thus, a context may be
thought of as a process with a ``hole'' (written $\Box$) in it. The
application of a context $M$ to a process $P$, written $M[P]$, is
tantamount to filling the hole in $M$ with $P$. In this paper we do
not need the full weight of this theory, but do make use of the notion
of context in the proof the main theorem. 

\begin{mathpar}
  \inferrule* [lab=summation] {} {{M_{M},M_{N}} \bc \Box \;|\; x.M_{A} \;|\; M_{M}+M_{N}}
  \and
  \inferrule* [lab=agent] {} {{M_{A}} \bc (\vec{x})M_{P} \;| \; \clift{P_0,\ldots,M_{P},\ldots,P_N}}
  \and \\
  \inferrule* [lab=process] {} {{M_{P}} \bc M_{N} \;| \;P|M_{P} }
\end{mathpar} 

\begin{mathpar}
  \inferrule* [lab=sychronization] {} {M_{N} \bc \Box \;|\; x?M_{F} \;|\; x!M_{C}}
  \and
  \inferrule* [lab=abstraction] {} {{M_{F}} \bc (x)M_{P} }
  \and
  \inferrule* [lab=concretion] {} {{M_{C}} \bc \langle M_{P} \rangle }
  \and \\
  \inferrule* [lab=process] {} {{M_{P}} \bc M_{N} \;| \;P|M_{P} }
\end{mathpar}

\begin{definition}[contextual application] Given a context $M$, and
  process $P$, we define the \emph{contextual application}, $M[P] :=
  M\{P/\Box\}$. That is, the contextual application of M to P is the
  substitution of $P$ for $\Box$ in $M$.
\end{definition}

$\meaningof{-} : L \to \mathcal{P}(\pi)$

\begin{mathpar}
  \inferrule* [lab=collection] {} {\meaningof{true} = \pi, \and \meaningof{~E} = \pi \setminus \meaningof{E}, \and \meaningof{E_{1} \& E_{2}} = \meaningof{E_{1}} \cap \meaningof{E_{2}}}
\end{mathpar}

\begin{mathpar}
  \inferrule* [lab=structure] {} {\meaningof{0} = \{ P \in \pi | P \equiv 0 \}, \and \\ \meaningof{E_1 | E_2} = \{ P \in \pi | P \equiv P_{1} | P_{2}, P_{1} \in \meaningof{E_{1}}, P_{2} \in \meaningof{E_2}\} }
\end{mathpar}

\begin{mathpar}
 \inferrule* [lab=behavior] {} {\meaningof{\langle a?b \rangle E} = \{ P \in \pi | P \equiv Q | u?(y)P', \\ \and \\\\ \and \\ \;\;\; u \in \meaningof{a}, \forall z.P'\{z/y\} \in \meaningof{E\{z/b\}}\}, \and \\ \meaningof{a!E} = \{ P \in \pi | P \equiv Q | x!\langle P' \rangle, x \in \meaningof{a} P' \in \meaningof{E}\} }
\end{mathpar}

\begin{mathpar}
 \inferrule* [lab=nominal] {} {\meaningof{\quotep{E}} = \{ \quotep{P} \in \quotep{\pi} | P \in \meaningof{E} \}, \and \meaningof{\quotep{P}} = \{ \quotep{Q} \in \quotep{\pi} | P \equiv Q \} \and \\ \meaningof{@\quotep{E}} = \{ P \in \pi | P \equiv @x, x \in \meaningof{E} \}}
\end{mathpar}

\begin{eqnarray*}
  \\
  \meaningof{-} : TS \to ST
\end{eqnarray*}

\begin{eqnarray*}
  \\
  L : TS \to ST
\end{eqnarray*}

\begin{eqnarray*}
  \\
  P \models E \iff P \in \meaningof{E}
\end{eqnarray*}

\begin{eqnarray*}
  P \approx_{L} Q \iff \forall E \in L. P \models E \iff Q \models E
\end{eqnarray*}

\begin{eqnarray*}
  P \approx_{K} Q
\end{eqnarray*}

\begin{eqnarray*}
  P \approx Q
\end{eqnarray*}

$\approx_{K} = \approx = \approx_{L}$

\subsubsection{Contextual duality}

Note that contexts extend the quotation operation to a family of
operations from processes to names. Given a context, $M$, we can
define a \emph{nominal context}, $\quotep{M}$ by $\quotep{M}[P] :=
\quotep{M[P]}$. To foreshadow what is to come we observe that these
operations enjoy a duality with processes very much like the duality
between vectors and maps from vectors to scalars.

Further, because the calculus is essentially higher-order, we have a
correspondence between contexts and processes. More specifically,
given a name $x$ and a context $M$ we can construct $M^{*}_{x}$ such
that 

\begin{mathpar}
  M^{*}_{x} | \lift{x}{P} \red M[P]
\end{mathpar}

namely,

\begin{mathpar}
  M^{*}_{x} := x?(u).M[\dropn{u}]
\end{mathpar}

The dependence of $M^{*}_{x}$ on a name makes it an abstraction, 

\begin{mathpar}
  M^{*} := (x)x?(u).M[\dropn{u}]
\end{mathpar}

\subsection{Additional notation}

It will sometimes be convenient to denote the process a name
quotes. We already have the notation $x = \quotep{P}$, but it will be
convenient to introduce an alternate notation, $\procn{x}$, when we
want to emphasize the connection to the use of the name. Note that, by
virtue of name equivalence, $\quotep{\procn{x}} \nameeq x$; so, the
notation is consistent with previous definitions.

Further, because names have structure it is possible to effect
substitutions on the basis of that structure. This means we need to
upgrade our notation for substitutions, which we accomplish by
adapting comprehension notation. Thus,

\begin{mathpar}
  P\{ y / x : x \in S \}
\end{mathpar}

is interpreted to mean the process derived from P by replacing (in a
capture-avoiding manner) each occurrence of $x$ in $S$ by $y$. For example,

\begin{mathpar}
  P\{ \quotep{\procn{x}|\procn{x}} / x : x \in \freenames{P} \}
\end{mathpar}

will replace each (occurrence) of a free name $x$ in $P$ by
$\quotep{\procn{x}|\procn{x}}$.

Also, we will avail ourselves of the notation $x^{L}$ and $x^{R}$ to
denote injections of a name into disjoint copies of the name
space. There are numerous ways to accomplish this. One example can be
found in \cite{MeredithR05}. This notation overloads to vectors of
names: $\vec{x}^{\pi} := (x_{i}^{\pi} \; : \; 0 \leq i < |\vec{x}| )$ where $\pi \in \{L,R\}$.

We also use $P^{\Box} := P|\Box$.

In \cite{MeredithR05} an interpretation of the new operator is
given. It turns out that there are several possible interpretations
all enjoying the requisite algebraic properties of the operator (see
\cite{milner91polyadicpi}). We will therefore make liberal use of
$(\nu\; \vec{x})P$.

% subsection the_syntax_and_semantics_of_the_notation_system (end)   

\input{qm2pi.qmops} 

\input{qm2pi.sterngerlach} 

\input{qm2pi.metric} 

% section concurrent_process_calculi (end)

%\input{qm2pi.proofsketch}

% section proof sketch (end)

%\input{qm2pi.slviaknots} 

% section spatial logic via knots (end)

\input{qm2pi.conclusion}

% section conclusion (end)

%\input{qm2pi.dtcodes} 

% section wiring algorithm (end)

\input{qm2pi.ack} 

% section acknowledgments (end)

\newpage


\bibliographystyle{plain}   
\bibliography{../../biblios/main.bib}

\input{qm2pi.rhodetails}

\end{document}

 

\documentclass[12pt]{llncs}
%\documentclass{jktr}

\usepackage[pdftex]{hyperref}                   
\usepackage {listings}
\usepackage {mathpartir}
\usepackage{bcprules}
%\usepackage{listings}
                       
\usepackage{graphicx} 
%\usepackage[margins=2.5cm,nohead,nofoot]{geometry}
%\usepackage{geometry}
\usepackage{amsfonts}
\usepackage{amstext}
\usepackage{latexsym}
\usepackage{amssymb}
\usepackage{color}


%\include{myPreamble}
\include{qm2pi.local} 

%\ifpdf
%\usepackage[pdftex]{graphicx}
%\else
%\usepackage{graphicx}
%\fi

 % \ifpdf
%  \usepackage{pdfsync}
%  \if


%\title{Brief Article}
%\author{David F. Snyder}
%\author{L.G. Meredith}

%\address{Dept. of Math., Texas State University--San Marcos, San Marcos, TX 78666}
       
\pagestyle{empty}


\begin{document}

\lstset{language=[Objective]Caml,frame=shadowbox}

\input{qm2pi.front}

% section front matter (end)

\input{qm2pi.intro} 
 
% section introduction (end)

% \input{qm2pi.knotations} 

% section notation (end)

\input{qm2pi.process.calculi} 

% section concurrent_process_calculi_and_spatial_logics_ (end)
    
%\input{qm2pi.knots2pi} 

%\input{qm2pi.trefoil} 

%\input{qm2pi.mainthm} 

% subsection basic_interpretation (end)

%\input{qm2pi.rho.presentation} 
\subsection{The syntax and semantics of the notation system}\label{sub:the_syntax_and_semantics_of_the_notation_system} % (fold)

We now summarize a technical presentation of the calculus that
embodies our theory of dynamics. The typical presentation of such a
calculus follows the style of giving generators and relations on
them. The grammar, below, describing term constructors, freely
generates the set of processes, $\Proc$. This set is then quotiented
by a relation known as structural congruence and it is over this set
that the notion of dynamics is expressed. This presentation is
essentially that of \cite{MeredithR05} with the addition of
polyadicity and summation. For readability we have relegated some of
the technical subtleties to an appendix.

\subsubsection{Process grammar}\label{subsub:process_grammar}

\begin{mathpar}
  \inferrule* [lab=synchronization] {} {{M} \bc \pzero \;|\; x?F \;|\; x!C }
  \and
  \inferrule* [lab=abstraction] {} {{F} \bc (x)P}
  \and
  \inferrule* [lab=concretion] {} {{C} \bc \langle Q \rangle}
  \and
  \inferrule* [lab=process] {} {{P,Q} \bc M \;| \;P|Q \;|\; @{x}}
  \and
  \inferrule* [lab=name] {} {{x} \bc \quotep{P}}
\end{mathpar} 

Note that $\vec{x}$ (resp. $\vec{P}$) denotes a vector of names
(resp. processes) of length $|\vec{x}|$ (resp. $|\vec{P}|$). We adopt
the following useful abbreviations.

\begin{mathpar}
   x?(\vec{y}).P := x.(\vec{y})P \and  x\clift{\vec{P}} := x.\clift{\vec{P}}
   \and x!(y) := \lift{x}{\dropn{y}}
   \and \Pi_{i=0}^{n-1}P_i := P_0 | \ldots | P_{n-1}
\end{mathpar}

\subsubsection{Structural congruence}

\paragraph{Free and bound names and alpha-equivalence.} At the
core of structural equivalence is alpha-equivalence which identifies
process that are the same up to a change of variable. Formally, we
recognize the distinction between free and bound names. The free names
of a process, $\freenames{P}$, may be calculated recursively as
follows:

\begin{mathpar}
\freenames{\pzero} := \emptyset
  \and \\
  \freenames{x?(y).P} := \{ x \} \cup (\freenames{P} \setminus \{ y \})
  \and 
  \freenames{x!\langle P \rangle} := \{ x \} \cup \{ P \} 
  \and \\
  \freenames{P|Q} := \freenames{P} \cup \freenames{Q}
  \and \\
  \freenames{@{x}} := \{ x \}
\end{mathpar}

$\pi$
$\quotep{\pi}$

$\freenames{-} : \pi \to \mathcal{P}(\quotep{\pi})$

\begin{eqnarray*}
  \freenames{\pzero} & := & \emptyset \\
  \freenames{x?(y).P} & := & \{ x \} \cup (\freenames{P} \setminus \{ y \}) \\
  \freenames{x!\langle P \rangle} & := & \{ x \} \cup \{ P \} \\
  \freenames{P|Q} & := & \freenames{P} \cup \freenames{Q} \\
  \freenames{\dropn{x}} & := & \{ x \}
\end{eqnarray*}

The bound names of a process, $\boundnames{P}$, are those names occurring in $P$
that are not free. For example, in $x?(y).0$, the name $x$ is free, while $y$ is bound.

\begin{mathpar}
  \inferrule* [lab=monoidal-laws] {} { P|Q \equiv Q|P \and P|0 \equiv P \and P|(Q|R) \equiv (P|Q)|R }
\end{mathpar}

\begin{mathpar}
  \inferrule* [lab=alpha-equivalence] {} { (x)P \equiv (y)P\{y/x\} \and y \not\in \freenames{P} }
\end{mathpar}

\begin{definition}
Then two processes, $P,Q$, are alpha-equivalent if $P = Q\{\vec{y}/\vec{x}\}$ for
some $\vec{x} \in \boundnames{Q},\vec{y} \in \boundnames{P}$, where $Q\{\vec{y}/\vec{x}\}$
denotes the capture-avoiding substitution of $\vec{y}$ for $\vec{x}$ in $Q$.
\end{definition}

\begin{definition}
  The {\em structural congruence} \cite{SangiorgiWalker} , $\equiv$,
  between processes is the least congruence containing
  alpha-equivalence, satisfying the abelian monoid laws
  (associativity, commutativity and $\pzero$ as identity) for parallel
  composition $|$ and for summation $+$.
\end{definition}

\subsection{Name equivalence}

We take name equivalence, written $\nameeq$, to be the smallest
equivalence relation generated by the following rules.

\begin{mathpar}
\inferrule*[lab=Quote-drop]
{ }
{ \quotep{@{x}} \nameeq x }

\inferrule*[lab=Struct-equiv]
{ P \scong Q }
{ \quotep{P} \nameeq \quotep{Q} }
\end{mathpar}

The astute reader will have noticed that the mutual recursion of names
and processes imposes a mutual recursion on alpha-equivalence and
structural equivalence via name-equivalence. Fortunately, all of this
works out pleasantly and we may calculate in the natural way, free of
concern. The reader interested in the details is referred to the
appendix \ref{appendix:rho_details}.

\subsection{Substitution}

We use $\Proc$ for the set of processes, $\QProc$ for the set of
names, and $\id{\{}\vec{y} / \vec{x} \id{\}}$ to denote partial maps,
$s : \QProc \rightarrow \QProc$. A map, $s$ lifts, uniquely, to a map
on process terms, $\widehat{s} : \Proc \rightarrow \Proc$ by the
following equations.

\begin{mathpar}
  (0) \psubstp{Q}{P} := 0 \\
  (R \juxtap S) \psubstp{Q}{P}
  :=    
  (R)\psubstp{Q}{P} \juxtap (S) \psubstp{Q}{P} \\
  (x?(y).R) \psubstp{Q}{P}    
  :=    
  (x)\substp{Q}{P} (z)\concat( (R \psubstn{z}{y}) \psubstp{Q}{P} ) \\
  (\lift{x}{R}) \psubstp{Q}{P}  
  :=
  \lift{(x)\substp{Q}{P}}{ R \psubstp{Q}{P} } \\
%   (\dropn{x})  \psubstp{Q}{P}       
%   := 
%   \left\{ 
%     \begin{array}{ccc} 
%       \dropn{\quotep{Q}} & & x \nameeq \quotep{P} \\
%       \dropn{x} & & otherwise \\
%     \end{array}
%   \right. 
  (\dropn{x})  \psubstp{Q}{P}       
  := 
  \left\{ 
    \begin{array}{ccc} 
      Q & & x \nameeq \quotep{P} \\
      \dropn{x} & & otherwise \\
    \end{array}
  \right.
\end{mathpar}
 

where

\begin{eqnarray}
  (x)\id{\{} \lpquote Q \rpquote / \lpquote P \rpquote \id{\}}            = 
  \left\{ 
    \begin{array}{ccc}
      \lpquote Q \rpquote & & x \nameeq \lpquote P \rpquote \\
      x & & otherwise \\
    \end{array}
  \right. \nonumber
\end{eqnarray}

and $z$ is chosen distinct from $\quotep{P}$, $\quotep{Q}$, the free
names in $Q$, and all the names in $R$. Our $\alpha$-equivalence will
be built in the standard way from this substitution.

\begin{remark}\label{rem:no_self_referential_names}
  One consequence of these definitions is that $\forall P. \quotep{P}
  \not\in \freenames{P}$.
\end{remark}

\subsection{ Dynamic quote: an example }

Anticipating something of what's to come, consider applying the
substitution, $\widehat{\id{\{}u / z \id{\}}}$, to the following pair
of processes, $\lift{w}{y!(z)}$ and $w[ \lpquote y!(z) \rpquote ]$.

\begin{eqnarray}
	\lift{w}{y!(z)}\widehat{\id{\{}u / z \id{\}}}
		& = &
		\lift{w}{y!(u)} \nonumber\\
	w[ \lpquote y!(z) \rpquote ] \widehat{ \id{\{}u / z \id{\}} }
		& = &
		w[ \lpquote y!(z) \rpquote ] \nonumber
\end{eqnarray}

Because the body of the process between quotes is impervious to
substitution, we get radically different answers. In fact, by
examining the first process in an input context,
e.g. $x?(z).\lift{w}{y!(z)}$, we see that the process under the lift
operator may be shaped by prefixed inputs binding a name inside it. In
this sense, the lift operator will be seen as a way to dynamically
construct processes before reifying them as names.

Finally equipped with these standard features we can present the
dynamics of the calculus.

\subsubsection{Operational semantics} 

Finally, we introduce the computational dynamics. What marks these
algebras as distinct from other more traditionally studied algebraic
structures, e.g. vector spaces or polynomial rings, is the manner in
which dynamics is captured. In traditional structures, dynamics is typically
expressed through morphisms between such structures, as in linear maps
between vector spaces or morphisms between rings. In algebras
associated with the semantics of computation, the dynamics is
expressed as part of the algebraic structure itself, through a
reduction reduction relation typically denoted by $\red$. Below, we
give a recursive presentation of this relation for the calculus used
in the encoding.

$\red \subseteq \pi \times \pi$
$\red : \pi \to \mathcal{P}(\pi)$

\begin{mathpar}
  \inferrule* [lab=Comm] { \textsf{match}( x_{src}, x_{trgt} ) } { x_{trgt}?(y)P \; | \; x_{src}!\langle {Q} \rangle \red P\{\quotep{Q}/y}\} }
  \and \\
  \inferrule* [lab=Par] {{P} \red {P}'} {{{P} | {Q}} \red {{P}' | {Q}}}
  \and
  \inferrule* [lab=Equiv]{{{P} \scong {P}'} \andalso {{P}' \red {Q}'} \andalso {{Q}' \scong {Q}}}{{P} \red {Q}}
\end{mathpar}

\begin{eqnarray*}
  match_{\equiv} (\quotep{P},\quotep{Q}) & := & P \equiv Q \\
  match_{\dagger}(\quotep{P},\quotep{Q}) & := & \forall R. P|Q \red^{*} R => R \red^{*} 0 \\
  match_{K}(\quotep{P},\quotep{Q}) & := & K \mbox{ for some context } K
\end{eqnarray*}

$u?(x)P | u!\langle Q \rangle \red P\{\quotep{Q}/x\}$

%We write $\wred$ for $\red^*$, and $P\red$ if $\exists Q $ such that $ P \red Q$.
We write $P\red$ if $\exists Q $ such that $ P \red Q$ and $P\not\red$, otherwise.

\section{Replication}

As mentioned before, it is known that replication (and hence
recursion) can be implemented in a higher-order process algebra
\cite{SangiorgiWalker}. As our first example of calculation with the
machinery thus far presented we give the construction explicitly in
the {\rhoc}.

\begin{eqnarray}
	D_{x} & := & \prefix{x}{y}{(\binpar{\outputp{x}{y}}{@{y}})} \nonumber\\
	\bangp_{x}{P} & := & \binpar{{x}!\langle{\binpar{D_{x}}{P}}\rangle}{D_{x}} \nonumber
\end{eqnarray}

\begin{eqnarray}
	\bangp_{x}{P} & & \nonumber\\
	=
	& {x}!\langle{(\prefix{x}{y}{(\outputp{x}{y} | @{y})) | P}}\rangle 
	      | \prefix{x}{y}{(\outputp{x}{y} | @{y})} & \nonumber\\
	\red
	& (\outputp{x}{y} | @{y})\substn{\quotep{(\prefix{x}{y}{(@{y} | \outputp{x}{y})) | P}}}{y} & \nonumber\\
	=
	& \outputp{x}{\quotep{(\prefix{x}{y}{(\outputp{x}{y} | @{y})) | P}}}
	  | {(\prefix{x}{y}{(\outputp{x}{y} | @{y})) | P}} & \nonumber\\
	\red
	& \ldots & \nonumber\\
	\red^*
	& P | P | \ldots & \nonumber
\end{eqnarray}

Of course, this encoding, as an implementation, runs away, unfolding
$\bangp{P}$ eagerly. A lazier and more implementable replication
operator, restricted to input-guarded processes, may be obtained as follows.

\begin{eqnarray}
\bangp{\prefix{u}{v}{P}} 
	:= 
	\binpar{\lift{x}{\prefix{u}{v}{(\binpar{D(x)}{P})}}}{D(x)} \nonumber
\end{eqnarray}

\begin{remark}
  Note that the lazier definition still does not deal with summation
  or mixed summation (i.e. sums over input and output). The reader is
  invited to construct definitions of replication that deal with these
  features. 

  Further, the definitions are parameterized in a name, $x$. Can you,
  gentle reader, make a definition that eliminates this parameter and
  guarantees no accidental interaction between the replication
  machinery and the process being replicated -- i.e. no accidental
  sharing of names used by the process to get its work done and the
  name(s) used by the replication to effect copying. This latter
  revision of the definition of replication is crucial to obtaining
  the expected identity $!!P \sim !P$.
\end{remark}

\begin{remark}\label{rem:paradoxical_combinator}
  The reader familiar with the lambda calculus will have noticed the
  similarity between $D$ and the paradoxical combinator.

  [Ed. note: the existence of this seems to suggest we have to be more
  restrictive on the set of processes and names we admit if we are to
  support no-cloning.]
\end{remark}

\subsubsection{Bisimulation}

The computational dynamics gives rise to another kind of equivalence,
the equivalence of computational behavior. As previously mentioned
this is typically captured \emph{via} some form of bisimulation.

% The notion we use in this paper is weak barbed bisimulation
% \cite{milner91polyadicpi}.

The notion we use in this paper is derived from weak barbed
bisimulation \cite{milner91polyadicpi}. 

\begin{definition}
An \emph{observation relation}, $\downarrow_{\mathcal N}$, over a set
of names, $\mathcal N$, is the smallest relation satisfying the rules
below.

\infrule[Out-barb]{y \in {\mathcal N}, \; x \nameeq y}
		  {\outputp{x}{v} \downarrow_{\mathcal N} x}
\infrule[Par-barb]{\mbox{$P\downarrow_{\mathcal N} x$ or $Q\downarrow_{\mathcal N} x$}}
		  {\binpar{P}{Q} \downarrow_{\mathcal N} x}

We write $P \Downarrow_{\mathcal N} x$ if there is $Q$ such that 
$P \wred Q$ and $Q \downarrow_{\mathcal N} x$.
\end{definition}

\begin{definition}
%\label{def.bbisim}
An  ${\mathcal N}$-\emph{barbed bisimulation} over a set of names, ${\mathcal N}$, is a symmetric binary relation 
${\mathcal S}_{\mathcal N}$ between agents such that $P\rel{S}_{\mathcal N}Q$ implies:
\begin{enumerate}
\item If $P \red P'$ then $Q \wred Q'$ and $P'\rel{S}_{\mathcal N} Q'$.
\item If $P\downarrow_{\mathcal N} x$, then $Q\Downarrow_{\mathcal N} x$.
\end{enumerate}
$P$ is ${\mathcal N}$-barbed bisimilar to $Q$, written
$P \wbbisim_{\mathcal N} Q$, if $P \rel{S}_{\mathcal N} Q$ for some ${\mathcal N}$-barbed bisimulation ${\mathcal S}_{\mathcal N}$.
\end{definition}

$\mathcal{R} \subseteq \pi \times \pi$

$P \mathcal{R} Q => \forall P'. P \red P' \Rightarrow \exists Q'. Q \red Q', P' \mathcal{R} Q'$

$P \vdash x \Rightarrow Q \vdash x$

\begin{mathpar}
  \inferrule*[lab=Out-barb]{x \nameeq y}{{y}!\langle{Q}\rangle \vdash x}
  \and
  \inferrule*[lab=Par-barb]{\mbox{$P\vdash x$ or $Q\vdash x$}}{\binpar{P}{Q} \vdash x}
\end{mathpar}

\subsubsection{Contexts}

One of the principle advantages of computational calculi like the
$\pi$-calculus is a well-defined notion of context,
contextual-equivalence and a correlation between
contextual-equivalence and notions of bisimulation. The notion of
context allows the decomposition of a process into (sub-)process and
its syntactic environment, its context. Thus, a context may be
thought of as a process with a ``hole'' (written $\Box$) in it. The
application of a context $M$ to a process $P$, written $M[P]$, is
tantamount to filling the hole in $M$ with $P$. In this paper we do
not need the full weight of this theory, but do make use of the notion
of context in the proof the main theorem. 

\begin{mathpar}
  \inferrule* [lab=summation] {} {{M_{M},M_{N}} \bc \Box \;|\; x.M_{A} \;|\; M_{M}+M_{N}}
  \and
  \inferrule* [lab=agent] {} {{M_{A}} \bc (\vec{x})M_{P} \;| \; \clift{P_0,\ldots,M_{P},\ldots,P_N}}
  \and \\
  \inferrule* [lab=process] {} {{M_{P}} \bc M_{N} \;| \;P|M_{P} }
\end{mathpar} 

\begin{mathpar}
  \inferrule* [lab=sychronization] {} {M_{N} \bc \Box \;|\; x?M_{F} \;|\; x!M_{C}}
  \and
  \inferrule* [lab=abstraction] {} {{M_{F}} \bc (x)M_{P} }
  \and
  \inferrule* [lab=concretion] {} {{M_{C}} \bc \langle M_{P} \rangle }
  \and \\
  \inferrule* [lab=process] {} {{M_{P}} \bc M_{N} \;| \;P|M_{P} }
\end{mathpar}

\begin{definition}[contextual application] Given a context $M$, and
  process $P$, we define the \emph{contextual application}, $M[P] :=
  M\{P/\Box\}$. That is, the contextual application of M to P is the
  substitution of $P$ for $\Box$ in $M$.
\end{definition}

$\meaningof{-} : L \to \mathcal{P}(\pi)$

\begin{mathpar}
  \inferrule* [lab=collection] {} {\meaningof{true} = \pi, \and \meaningof{~E} = \pi \setminus \meaningof{E}, \and \meaningof{E_{1} \& E_{2}} = \meaningof{E_{1}} \cap \meaningof{E_{2}}}
\end{mathpar}

\begin{mathpar}
  \inferrule* [lab=structure] {} {\meaningof{0} = \{ P \in \pi | P \equiv 0 \}, \and \\ \meaningof{E_1 | E_2} = \{ P \in \pi | P \equiv P_{1} | P_{2}, P_{1} \in \meaningof{E_{1}}, P_{2} \in \meaningof{E_2}\} }
\end{mathpar}

\begin{mathpar}
 \inferrule* [lab=behavior] {} {\meaningof{\langle a?b \rangle E} = \{ P \in \pi | P \equiv Q | u?(y)P', \\ \and \\\\ \and \\ \;\;\; u \in \meaningof{a}, \forall z.P'\{z/y\} \in \meaningof{E\{z/b\}}\}, \and \\ \meaningof{a!E} = \{ P \in \pi | P \equiv Q | x!\langle P' \rangle, x \in \meaningof{a} P' \in \meaningof{E}\} }
\end{mathpar}

\begin{mathpar}
 \inferrule* [lab=nominal] {} {\meaningof{\quotep{E}} = \{ \quotep{P} \in \quotep{\pi} | P \in \meaningof{E} \}, \and \meaningof{\quotep{P}} = \{ \quotep{Q} \in \quotep{\pi} | P \equiv Q \} \and \\ \meaningof{@\quotep{E}} = \{ P \in \pi | P \equiv @x, x \in \meaningof{E} \}}
\end{mathpar}

\begin{eqnarray*}
  \\
  \meaningof{-} : TS \to ST
\end{eqnarray*}

\begin{eqnarray*}
  \\
  L : TS \to ST
\end{eqnarray*}

\begin{eqnarray*}
  \\
  P \models E \iff P \in \meaningof{E}
\end{eqnarray*}

\begin{eqnarray*}
  P \approx_{L} Q \iff \forall E \in L. P \models E \iff Q \models E
\end{eqnarray*}

\begin{eqnarray*}
  P \approx_{K} Q
\end{eqnarray*}

\begin{eqnarray*}
  P \approx Q
\end{eqnarray*}

$\approx_{K} = \approx = \approx_{L}$

\subsubsection{Contextual duality}

Note that contexts extend the quotation operation to a family of
operations from processes to names. Given a context, $M$, we can
define a \emph{nominal context}, $\quotep{M}$ by $\quotep{M}[P] :=
\quotep{M[P]}$. To foreshadow what is to come we observe that these
operations enjoy a duality with processes very much like the duality
between vectors and maps from vectors to scalars.

Further, because the calculus is essentially higher-order, we have a
correspondence between contexts and processes. More specifically,
given a name $x$ and a context $M$ we can construct $M^{*}_{x}$ such
that 

\begin{mathpar}
  M^{*}_{x} | \lift{x}{P} \red M[P]
\end{mathpar}

namely,

\begin{mathpar}
  M^{*}_{x} := x?(u).M[\dropn{u}]
\end{mathpar}

The dependence of $M^{*}_{x}$ on a name makes it an abstraction, 

\begin{mathpar}
  M^{*} := (x)x?(u).M[\dropn{u}]
\end{mathpar}

\subsection{Additional notation}

It will sometimes be convenient to denote the process a name
quotes. We already have the notation $x = \quotep{P}$, but it will be
convenient to introduce an alternate notation, $\procn{x}$, when we
want to emphasize the connection to the use of the name. Note that, by
virtue of name equivalence, $\quotep{\procn{x}} \nameeq x$; so, the
notation is consistent with previous definitions.

Further, because names have structure it is possible to effect
substitutions on the basis of that structure. This means we need to
upgrade our notation for substitutions, which we accomplish by
adapting comprehension notation. Thus,

\begin{mathpar}
  P\{ y / x : x \in S \}
\end{mathpar}

is interpreted to mean the process derived from P by replacing (in a
capture-avoiding manner) each occurrence of $x$ in $S$ by $y$. For example,

\begin{mathpar}
  P\{ \quotep{\procn{x}|\procn{x}} / x : x \in \freenames{P} \}
\end{mathpar}

will replace each (occurrence) of a free name $x$ in $P$ by
$\quotep{\procn{x}|\procn{x}}$.

Also, we will avail ourselves of the notation $x^{L}$ and $x^{R}$ to
denote injections of a name into disjoint copies of the name
space. There are numerous ways to accomplish this. One example can be
found in \cite{MeredithR05}. This notation overloads to vectors of
names: $\vec{x}^{\pi} := (x_{i}^{\pi} \; : \; 0 \leq i < |\vec{x}| )$ where $\pi \in \{L,R\}$.

We also use $P^{\Box} := P|\Box$.

In \cite{MeredithR05} an interpretation of the new operator is
given. It turns out that there are several possible interpretations
all enjoying the requisite algebraic properties of the operator (see
\cite{milner91polyadicpi}). We will therefore make liberal use of
$(\nu\; \vec{x})P$.

% subsection the_syntax_and_semantics_of_the_notation_system (end)   

\input{qm2pi.qmops} 

\input{qm2pi.sterngerlach} 

\input{qm2pi.metric} 

% section concurrent_process_calculi (end)

%\input{qm2pi.proofsketch}

% section proof sketch (end)

%\input{qm2pi.slviaknots} 

% section spatial logic via knots (end)

\input{qm2pi.conclusion}

% section conclusion (end)

%\input{qm2pi.dtcodes} 

% section wiring algorithm (end)

\input{qm2pi.ack} 

% section acknowledgments (end)

\newpage


\bibliographystyle{plain}   
\bibliography{../../biblios/main.bib}

\input{qm2pi.rhodetails}

\end{document}

 

% section concurrent_process_calculi (end)

%\documentclass[12pt]{llncs}
%\documentclass{jktr}

\usepackage[pdftex]{hyperref}                   
\usepackage {listings}
\usepackage {mathpartir}
\usepackage{bcprules}
%\usepackage{listings}
                       
\usepackage{graphicx} 
%\usepackage[margins=2.5cm,nohead,nofoot]{geometry}
%\usepackage{geometry}
\usepackage{amsfonts}
\usepackage{amstext}
\usepackage{latexsym}
\usepackage{amssymb}
\usepackage{color}


%\include{myPreamble}
\include{qm2pi.local} 

%\ifpdf
%\usepackage[pdftex]{graphicx}
%\else
%\usepackage{graphicx}
%\fi

 % \ifpdf
%  \usepackage{pdfsync}
%  \if


%\title{Brief Article}
%\author{David F. Snyder}
%\author{L.G. Meredith}

%\address{Dept. of Math., Texas State University--San Marcos, San Marcos, TX 78666}
       
\pagestyle{empty}


\begin{document}

\lstset{language=[Objective]Caml,frame=shadowbox}

\input{qm2pi.front}

% section front matter (end)

\input{qm2pi.intro} 
 
% section introduction (end)

% \input{qm2pi.knotations} 

% section notation (end)

\input{qm2pi.process.calculi} 

% section concurrent_process_calculi_and_spatial_logics_ (end)
    
%\input{qm2pi.knots2pi} 

%\input{qm2pi.trefoil} 

%\input{qm2pi.mainthm} 

% subsection basic_interpretation (end)

%\input{qm2pi.rho.presentation} 
\subsection{The syntax and semantics of the notation system}\label{sub:the_syntax_and_semantics_of_the_notation_system} % (fold)

We now summarize a technical presentation of the calculus that
embodies our theory of dynamics. The typical presentation of such a
calculus follows the style of giving generators and relations on
them. The grammar, below, describing term constructors, freely
generates the set of processes, $\Proc$. This set is then quotiented
by a relation known as structural congruence and it is over this set
that the notion of dynamics is expressed. This presentation is
essentially that of \cite{MeredithR05} with the addition of
polyadicity and summation. For readability we have relegated some of
the technical subtleties to an appendix.

\subsubsection{Process grammar}\label{subsub:process_grammar}

\begin{mathpar}
  \inferrule* [lab=synchronization] {} {{M} \bc \pzero \;|\; x?F \;|\; x!C }
  \and
  \inferrule* [lab=abstraction] {} {{F} \bc (x)P}
  \and
  \inferrule* [lab=concretion] {} {{C} \bc \langle Q \rangle}
  \and
  \inferrule* [lab=process] {} {{P,Q} \bc M \;| \;P|Q \;|\; @{x}}
  \and
  \inferrule* [lab=name] {} {{x} \bc \quotep{P}}
\end{mathpar} 

Note that $\vec{x}$ (resp. $\vec{P}$) denotes a vector of names
(resp. processes) of length $|\vec{x}|$ (resp. $|\vec{P}|$). We adopt
the following useful abbreviations.

\begin{mathpar}
   x?(\vec{y}).P := x.(\vec{y})P \and  x\clift{\vec{P}} := x.\clift{\vec{P}}
   \and x!(y) := \lift{x}{\dropn{y}}
   \and \Pi_{i=0}^{n-1}P_i := P_0 | \ldots | P_{n-1}
\end{mathpar}

\subsubsection{Structural congruence}

\paragraph{Free and bound names and alpha-equivalence.} At the
core of structural equivalence is alpha-equivalence which identifies
process that are the same up to a change of variable. Formally, we
recognize the distinction between free and bound names. The free names
of a process, $\freenames{P}$, may be calculated recursively as
follows:

\begin{mathpar}
\freenames{\pzero} := \emptyset
  \and \\
  \freenames{x?(y).P} := \{ x \} \cup (\freenames{P} \setminus \{ y \})
  \and 
  \freenames{x!\langle P \rangle} := \{ x \} \cup \{ P \} 
  \and \\
  \freenames{P|Q} := \freenames{P} \cup \freenames{Q}
  \and \\
  \freenames{@{x}} := \{ x \}
\end{mathpar}

$\pi$
$\quotep{\pi}$

$\freenames{-} : \pi \to \mathcal{P}(\quotep{\pi})$

\begin{eqnarray*}
  \freenames{\pzero} & := & \emptyset \\
  \freenames{x?(y).P} & := & \{ x \} \cup (\freenames{P} \setminus \{ y \}) \\
  \freenames{x!\langle P \rangle} & := & \{ x \} \cup \{ P \} \\
  \freenames{P|Q} & := & \freenames{P} \cup \freenames{Q} \\
  \freenames{\dropn{x}} & := & \{ x \}
\end{eqnarray*}

The bound names of a process, $\boundnames{P}$, are those names occurring in $P$
that are not free. For example, in $x?(y).0$, the name $x$ is free, while $y$ is bound.

\begin{mathpar}
  \inferrule* [lab=monoidal-laws] {} { P|Q \equiv Q|P \and P|0 \equiv P \and P|(Q|R) \equiv (P|Q)|R }
\end{mathpar}

\begin{mathpar}
  \inferrule* [lab=alpha-equivalence] {} { (x)P \equiv (y)P\{y/x\} \and y \not\in \freenames{P} }
\end{mathpar}

\begin{definition}
Then two processes, $P,Q$, are alpha-equivalent if $P = Q\{\vec{y}/\vec{x}\}$ for
some $\vec{x} \in \boundnames{Q},\vec{y} \in \boundnames{P}$, where $Q\{\vec{y}/\vec{x}\}$
denotes the capture-avoiding substitution of $\vec{y}$ for $\vec{x}$ in $Q$.
\end{definition}

\begin{definition}
  The {\em structural congruence} \cite{SangiorgiWalker} , $\equiv$,
  between processes is the least congruence containing
  alpha-equivalence, satisfying the abelian monoid laws
  (associativity, commutativity and $\pzero$ as identity) for parallel
  composition $|$ and for summation $+$.
\end{definition}

\subsection{Name equivalence}

We take name equivalence, written $\nameeq$, to be the smallest
equivalence relation generated by the following rules.

\begin{mathpar}
\inferrule*[lab=Quote-drop]
{ }
{ \quotep{@{x}} \nameeq x }

\inferrule*[lab=Struct-equiv]
{ P \scong Q }
{ \quotep{P} \nameeq \quotep{Q} }
\end{mathpar}

The astute reader will have noticed that the mutual recursion of names
and processes imposes a mutual recursion on alpha-equivalence and
structural equivalence via name-equivalence. Fortunately, all of this
works out pleasantly and we may calculate in the natural way, free of
concern. The reader interested in the details is referred to the
appendix \ref{appendix:rho_details}.

\subsection{Substitution}

We use $\Proc$ for the set of processes, $\QProc$ for the set of
names, and $\id{\{}\vec{y} / \vec{x} \id{\}}$ to denote partial maps,
$s : \QProc \rightarrow \QProc$. A map, $s$ lifts, uniquely, to a map
on process terms, $\widehat{s} : \Proc \rightarrow \Proc$ by the
following equations.

\begin{mathpar}
  (0) \psubstp{Q}{P} := 0 \\
  (R \juxtap S) \psubstp{Q}{P}
  :=    
  (R)\psubstp{Q}{P} \juxtap (S) \psubstp{Q}{P} \\
  (x?(y).R) \psubstp{Q}{P}    
  :=    
  (x)\substp{Q}{P} (z)\concat( (R \psubstn{z}{y}) \psubstp{Q}{P} ) \\
  (\lift{x}{R}) \psubstp{Q}{P}  
  :=
  \lift{(x)\substp{Q}{P}}{ R \psubstp{Q}{P} } \\
%   (\dropn{x})  \psubstp{Q}{P}       
%   := 
%   \left\{ 
%     \begin{array}{ccc} 
%       \dropn{\quotep{Q}} & & x \nameeq \quotep{P} \\
%       \dropn{x} & & otherwise \\
%     \end{array}
%   \right. 
  (\dropn{x})  \psubstp{Q}{P}       
  := 
  \left\{ 
    \begin{array}{ccc} 
      Q & & x \nameeq \quotep{P} \\
      \dropn{x} & & otherwise \\
    \end{array}
  \right.
\end{mathpar}
 

where

\begin{eqnarray}
  (x)\id{\{} \lpquote Q \rpquote / \lpquote P \rpquote \id{\}}            = 
  \left\{ 
    \begin{array}{ccc}
      \lpquote Q \rpquote & & x \nameeq \lpquote P \rpquote \\
      x & & otherwise \\
    \end{array}
  \right. \nonumber
\end{eqnarray}

and $z$ is chosen distinct from $\quotep{P}$, $\quotep{Q}$, the free
names in $Q$, and all the names in $R$. Our $\alpha$-equivalence will
be built in the standard way from this substitution.

\begin{remark}\label{rem:no_self_referential_names}
  One consequence of these definitions is that $\forall P. \quotep{P}
  \not\in \freenames{P}$.
\end{remark}

\subsection{ Dynamic quote: an example }

Anticipating something of what's to come, consider applying the
substitution, $\widehat{\id{\{}u / z \id{\}}}$, to the following pair
of processes, $\lift{w}{y!(z)}$ and $w[ \lpquote y!(z) \rpquote ]$.

\begin{eqnarray}
	\lift{w}{y!(z)}\widehat{\id{\{}u / z \id{\}}}
		& = &
		\lift{w}{y!(u)} \nonumber\\
	w[ \lpquote y!(z) \rpquote ] \widehat{ \id{\{}u / z \id{\}} }
		& = &
		w[ \lpquote y!(z) \rpquote ] \nonumber
\end{eqnarray}

Because the body of the process between quotes is impervious to
substitution, we get radically different answers. In fact, by
examining the first process in an input context,
e.g. $x?(z).\lift{w}{y!(z)}$, we see that the process under the lift
operator may be shaped by prefixed inputs binding a name inside it. In
this sense, the lift operator will be seen as a way to dynamically
construct processes before reifying them as names.

Finally equipped with these standard features we can present the
dynamics of the calculus.

\subsubsection{Operational semantics} 

Finally, we introduce the computational dynamics. What marks these
algebras as distinct from other more traditionally studied algebraic
structures, e.g. vector spaces or polynomial rings, is the manner in
which dynamics is captured. In traditional structures, dynamics is typically
expressed through morphisms between such structures, as in linear maps
between vector spaces or morphisms between rings. In algebras
associated with the semantics of computation, the dynamics is
expressed as part of the algebraic structure itself, through a
reduction reduction relation typically denoted by $\red$. Below, we
give a recursive presentation of this relation for the calculus used
in the encoding.

$\red \subseteq \pi \times \pi$
$\red : \pi \to \mathcal{P}(\pi)$

\begin{mathpar}
  \inferrule* [lab=Comm] { \textsf{match}( x_{src}, x_{trgt} ) } { x_{trgt}?(y)P \; | \; x_{src}!\langle {Q} \rangle \red P\{\quotep{Q}/y}\} }
  \and \\
  \inferrule* [lab=Par] {{P} \red {P}'} {{{P} | {Q}} \red {{P}' | {Q}}}
  \and
  \inferrule* [lab=Equiv]{{{P} \scong {P}'} \andalso {{P}' \red {Q}'} \andalso {{Q}' \scong {Q}}}{{P} \red {Q}}
\end{mathpar}

\begin{eqnarray*}
  match_{\equiv} (\quotep{P},\quotep{Q}) & := & P \equiv Q \\
  match_{\dagger}(\quotep{P},\quotep{Q}) & := & \forall R. P|Q \red^{*} R => R \red^{*} 0 \\
  match_{K}(\quotep{P},\quotep{Q}) & := & K \mbox{ for some context } K
\end{eqnarray*}

$u?(x)P | u!\langle Q \rangle \red P\{\quotep{Q}/x\}$

%We write $\wred$ for $\red^*$, and $P\red$ if $\exists Q $ such that $ P \red Q$.
We write $P\red$ if $\exists Q $ such that $ P \red Q$ and $P\not\red$, otherwise.

\section{Replication}

As mentioned before, it is known that replication (and hence
recursion) can be implemented in a higher-order process algebra
\cite{SangiorgiWalker}. As our first example of calculation with the
machinery thus far presented we give the construction explicitly in
the {\rhoc}.

\begin{eqnarray}
	D_{x} & := & \prefix{x}{y}{(\binpar{\outputp{x}{y}}{@{y}})} \nonumber\\
	\bangp_{x}{P} & := & \binpar{{x}!\langle{\binpar{D_{x}}{P}}\rangle}{D_{x}} \nonumber
\end{eqnarray}

\begin{eqnarray}
	\bangp_{x}{P} & & \nonumber\\
	=
	& {x}!\langle{(\prefix{x}{y}{(\outputp{x}{y} | @{y})) | P}}\rangle 
	      | \prefix{x}{y}{(\outputp{x}{y} | @{y})} & \nonumber\\
	\red
	& (\outputp{x}{y} | @{y})\substn{\quotep{(\prefix{x}{y}{(@{y} | \outputp{x}{y})) | P}}}{y} & \nonumber\\
	=
	& \outputp{x}{\quotep{(\prefix{x}{y}{(\outputp{x}{y} | @{y})) | P}}}
	  | {(\prefix{x}{y}{(\outputp{x}{y} | @{y})) | P}} & \nonumber\\
	\red
	& \ldots & \nonumber\\
	\red^*
	& P | P | \ldots & \nonumber
\end{eqnarray}

Of course, this encoding, as an implementation, runs away, unfolding
$\bangp{P}$ eagerly. A lazier and more implementable replication
operator, restricted to input-guarded processes, may be obtained as follows.

\begin{eqnarray}
\bangp{\prefix{u}{v}{P}} 
	:= 
	\binpar{\lift{x}{\prefix{u}{v}{(\binpar{D(x)}{P})}}}{D(x)} \nonumber
\end{eqnarray}

\begin{remark}
  Note that the lazier definition still does not deal with summation
  or mixed summation (i.e. sums over input and output). The reader is
  invited to construct definitions of replication that deal with these
  features. 

  Further, the definitions are parameterized in a name, $x$. Can you,
  gentle reader, make a definition that eliminates this parameter and
  guarantees no accidental interaction between the replication
  machinery and the process being replicated -- i.e. no accidental
  sharing of names used by the process to get its work done and the
  name(s) used by the replication to effect copying. This latter
  revision of the definition of replication is crucial to obtaining
  the expected identity $!!P \sim !P$.
\end{remark}

\begin{remark}\label{rem:paradoxical_combinator}
  The reader familiar with the lambda calculus will have noticed the
  similarity between $D$ and the paradoxical combinator.

  [Ed. note: the existence of this seems to suggest we have to be more
  restrictive on the set of processes and names we admit if we are to
  support no-cloning.]
\end{remark}

\subsubsection{Bisimulation}

The computational dynamics gives rise to another kind of equivalence,
the equivalence of computational behavior. As previously mentioned
this is typically captured \emph{via} some form of bisimulation.

% The notion we use in this paper is weak barbed bisimulation
% \cite{milner91polyadicpi}.

The notion we use in this paper is derived from weak barbed
bisimulation \cite{milner91polyadicpi}. 

\begin{definition}
An \emph{observation relation}, $\downarrow_{\mathcal N}$, over a set
of names, $\mathcal N$, is the smallest relation satisfying the rules
below.

\infrule[Out-barb]{y \in {\mathcal N}, \; x \nameeq y}
		  {\outputp{x}{v} \downarrow_{\mathcal N} x}
\infrule[Par-barb]{\mbox{$P\downarrow_{\mathcal N} x$ or $Q\downarrow_{\mathcal N} x$}}
		  {\binpar{P}{Q} \downarrow_{\mathcal N} x}

We write $P \Downarrow_{\mathcal N} x$ if there is $Q$ such that 
$P \wred Q$ and $Q \downarrow_{\mathcal N} x$.
\end{definition}

\begin{definition}
%\label{def.bbisim}
An  ${\mathcal N}$-\emph{barbed bisimulation} over a set of names, ${\mathcal N}$, is a symmetric binary relation 
${\mathcal S}_{\mathcal N}$ between agents such that $P\rel{S}_{\mathcal N}Q$ implies:
\begin{enumerate}
\item If $P \red P'$ then $Q \wred Q'$ and $P'\rel{S}_{\mathcal N} Q'$.
\item If $P\downarrow_{\mathcal N} x$, then $Q\Downarrow_{\mathcal N} x$.
\end{enumerate}
$P$ is ${\mathcal N}$-barbed bisimilar to $Q$, written
$P \wbbisim_{\mathcal N} Q$, if $P \rel{S}_{\mathcal N} Q$ for some ${\mathcal N}$-barbed bisimulation ${\mathcal S}_{\mathcal N}$.
\end{definition}

$\mathcal{R} \subseteq \pi \times \pi$

$P \mathcal{R} Q => \forall P'. P \red P' \Rightarrow \exists Q'. Q \red Q', P' \mathcal{R} Q'$

$P \vdash x \Rightarrow Q \vdash x$

\begin{mathpar}
  \inferrule*[lab=Out-barb]{x \nameeq y}{{y}!\langle{Q}\rangle \vdash x}
  \and
  \inferrule*[lab=Par-barb]{\mbox{$P\vdash x$ or $Q\vdash x$}}{\binpar{P}{Q} \vdash x}
\end{mathpar}

\subsubsection{Contexts}

One of the principle advantages of computational calculi like the
$\pi$-calculus is a well-defined notion of context,
contextual-equivalence and a correlation between
contextual-equivalence and notions of bisimulation. The notion of
context allows the decomposition of a process into (sub-)process and
its syntactic environment, its context. Thus, a context may be
thought of as a process with a ``hole'' (written $\Box$) in it. The
application of a context $M$ to a process $P$, written $M[P]$, is
tantamount to filling the hole in $M$ with $P$. In this paper we do
not need the full weight of this theory, but do make use of the notion
of context in the proof the main theorem. 

\begin{mathpar}
  \inferrule* [lab=summation] {} {{M_{M},M_{N}} \bc \Box \;|\; x.M_{A} \;|\; M_{M}+M_{N}}
  \and
  \inferrule* [lab=agent] {} {{M_{A}} \bc (\vec{x})M_{P} \;| \; \clift{P_0,\ldots,M_{P},\ldots,P_N}}
  \and \\
  \inferrule* [lab=process] {} {{M_{P}} \bc M_{N} \;| \;P|M_{P} }
\end{mathpar} 

\begin{mathpar}
  \inferrule* [lab=sychronization] {} {M_{N} \bc \Box \;|\; x?M_{F} \;|\; x!M_{C}}
  \and
  \inferrule* [lab=abstraction] {} {{M_{F}} \bc (x)M_{P} }
  \and
  \inferrule* [lab=concretion] {} {{M_{C}} \bc \langle M_{P} \rangle }
  \and \\
  \inferrule* [lab=process] {} {{M_{P}} \bc M_{N} \;| \;P|M_{P} }
\end{mathpar}

\begin{definition}[contextual application] Given a context $M$, and
  process $P$, we define the \emph{contextual application}, $M[P] :=
  M\{P/\Box\}$. That is, the contextual application of M to P is the
  substitution of $P$ for $\Box$ in $M$.
\end{definition}

$\meaningof{-} : L \to \mathcal{P}(\pi)$

\begin{mathpar}
  \inferrule* [lab=collection] {} {\meaningof{true} = \pi, \and \meaningof{~E} = \pi \setminus \meaningof{E}, \and \meaningof{E_{1} \& E_{2}} = \meaningof{E_{1}} \cap \meaningof{E_{2}}}
\end{mathpar}

\begin{mathpar}
  \inferrule* [lab=structure] {} {\meaningof{0} = \{ P \in \pi | P \equiv 0 \}, \and \\ \meaningof{E_1 | E_2} = \{ P \in \pi | P \equiv P_{1} | P_{2}, P_{1} \in \meaningof{E_{1}}, P_{2} \in \meaningof{E_2}\} }
\end{mathpar}

\begin{mathpar}
 \inferrule* [lab=behavior] {} {\meaningof{\langle a?b \rangle E} = \{ P \in \pi | P \equiv Q | u?(y)P', \\ \and \\\\ \and \\ \;\;\; u \in \meaningof{a}, \forall z.P'\{z/y\} \in \meaningof{E\{z/b\}}\}, \and \\ \meaningof{a!E} = \{ P \in \pi | P \equiv Q | x!\langle P' \rangle, x \in \meaningof{a} P' \in \meaningof{E}\} }
\end{mathpar}

\begin{mathpar}
 \inferrule* [lab=nominal] {} {\meaningof{\quotep{E}} = \{ \quotep{P} \in \quotep{\pi} | P \in \meaningof{E} \}, \and \meaningof{\quotep{P}} = \{ \quotep{Q} \in \quotep{\pi} | P \equiv Q \} \and \\ \meaningof{@\quotep{E}} = \{ P \in \pi | P \equiv @x, x \in \meaningof{E} \}}
\end{mathpar}

\begin{eqnarray*}
  \\
  \meaningof{-} : TS \to ST
\end{eqnarray*}

\begin{eqnarray*}
  \\
  L : TS \to ST
\end{eqnarray*}

\begin{eqnarray*}
  \\
  P \models E \iff P \in \meaningof{E}
\end{eqnarray*}

\begin{eqnarray*}
  P \approx_{L} Q \iff \forall E \in L. P \models E \iff Q \models E
\end{eqnarray*}

\begin{eqnarray*}
  P \approx_{K} Q
\end{eqnarray*}

\begin{eqnarray*}
  P \approx Q
\end{eqnarray*}

$\approx_{K} = \approx = \approx_{L}$

\subsubsection{Contextual duality}

Note that contexts extend the quotation operation to a family of
operations from processes to names. Given a context, $M$, we can
define a \emph{nominal context}, $\quotep{M}$ by $\quotep{M}[P] :=
\quotep{M[P]}$. To foreshadow what is to come we observe that these
operations enjoy a duality with processes very much like the duality
between vectors and maps from vectors to scalars.

Further, because the calculus is essentially higher-order, we have a
correspondence between contexts and processes. More specifically,
given a name $x$ and a context $M$ we can construct $M^{*}_{x}$ such
that 

\begin{mathpar}
  M^{*}_{x} | \lift{x}{P} \red M[P]
\end{mathpar}

namely,

\begin{mathpar}
  M^{*}_{x} := x?(u).M[\dropn{u}]
\end{mathpar}

The dependence of $M^{*}_{x}$ on a name makes it an abstraction, 

\begin{mathpar}
  M^{*} := (x)x?(u).M[\dropn{u}]
\end{mathpar}

\subsection{Additional notation}

It will sometimes be convenient to denote the process a name
quotes. We already have the notation $x = \quotep{P}$, but it will be
convenient to introduce an alternate notation, $\procn{x}$, when we
want to emphasize the connection to the use of the name. Note that, by
virtue of name equivalence, $\quotep{\procn{x}} \nameeq x$; so, the
notation is consistent with previous definitions.

Further, because names have structure it is possible to effect
substitutions on the basis of that structure. This means we need to
upgrade our notation for substitutions, which we accomplish by
adapting comprehension notation. Thus,

\begin{mathpar}
  P\{ y / x : x \in S \}
\end{mathpar}

is interpreted to mean the process derived from P by replacing (in a
capture-avoiding manner) each occurrence of $x$ in $S$ by $y$. For example,

\begin{mathpar}
  P\{ \quotep{\procn{x}|\procn{x}} / x : x \in \freenames{P} \}
\end{mathpar}

will replace each (occurrence) of a free name $x$ in $P$ by
$\quotep{\procn{x}|\procn{x}}$.

Also, we will avail ourselves of the notation $x^{L}$ and $x^{R}$ to
denote injections of a name into disjoint copies of the name
space. There are numerous ways to accomplish this. One example can be
found in \cite{MeredithR05}. This notation overloads to vectors of
names: $\vec{x}^{\pi} := (x_{i}^{\pi} \; : \; 0 \leq i < |\vec{x}| )$ where $\pi \in \{L,R\}$.

We also use $P^{\Box} := P|\Box$.

In \cite{MeredithR05} an interpretation of the new operator is
given. It turns out that there are several possible interpretations
all enjoying the requisite algebraic properties of the operator (see
\cite{milner91polyadicpi}). We will therefore make liberal use of
$(\nu\; \vec{x})P$.

% subsection the_syntax_and_semantics_of_the_notation_system (end)   

\input{qm2pi.qmops} 

\input{qm2pi.sterngerlach} 

\input{qm2pi.metric} 

% section concurrent_process_calculi (end)

%\input{qm2pi.proofsketch}

% section proof sketch (end)

%\input{qm2pi.slviaknots} 

% section spatial logic via knots (end)

\input{qm2pi.conclusion}

% section conclusion (end)

%\input{qm2pi.dtcodes} 

% section wiring algorithm (end)

\input{qm2pi.ack} 

% section acknowledgments (end)

\newpage


\bibliographystyle{plain}   
\bibliography{../../biblios/main.bib}

\input{qm2pi.rhodetails}

\end{document}



% section proof sketch (end)

%\section{Unlikely characters: spatial logic for
  knots}\label{sub:characteristic_formulae} % (fold)

Associated to the mobile process calculi are a family of logics known
as the Hennessy-Milner logics. These logics typically enjoy a
semantics interpreting formulae as sets of processes that when
factored through the encoding outlined above allows an identification
of classes of knots with logical formulae. In the context of this
encoding the sub-family known as the spatial logics \cite{CairesC03}
\cite{CairesC04} \cite{Caires04} are of particular interest providing
several important features for expressing and reasoning about
properties (i.e. classes) of knots. We hint here at how this may be done.

%\begin{description}
%\item [structural connectives] 
\subsubsection{Structural connectives} The spatial logics enjoy
structural connectives corresponding, at the logical level, to the
parallel composition ($P | Q$) and new name ($(\nu \; x)P$)
connectives for processes. As illustrated in the examples below, these
connectives are extremely expressive given the shape of our encoding.
%\item [decideable satisfaction]

\subsubsection{Decideable satisfaction}
In \cite{Caires04} the satisfaction relation is shown to be decideable
for a rich class of processes. It further turns out that the image of
the our encoding is a proper subset of that class. This result
provides the basis for an algorithm by which to search for knots
enjoying a given property.
%\item [characteristic formulae]

\subsubsection{Characteristic formulae}
In the same paper \cite{Caires04} , Caires presents a means of calculating
characteristic formulae, selecting equivalence classes of processes
up to a pre--specified depth limit on the support set of names. Composed with our
encoding, this characteristic formula can be used to select
characteristic formulae for knots.
%\end{description}

\subsubsection{Spatial logic formulae}

The grammar below (segmented for comprehension) summarizes the syntax
of spatial logic formulae. We employ illustrative examples in the
sequel to provide an intuitive understanding of their meaning
referring the reader to \cite{Caires04} for a more detailed explication
of the semantics.

\begin{mathpar}
  \inferrule* [lab=boolean] {} {{A,B} \bc T \;|\; \neg A \;|\; A \wedge B \;|\; \eta = \eta'}
  \and
  \inferrule* [lab=spatial] {} {|\; \pzero \;|\; A | B \;|\; x \text{\textregistered} A \;|\; \forall x . A \;|\;  H x . A}
  \and
  \inferrule* [lab=behavioral] {} {|\; \alpha . A}
  \and 
  \inferrule* [lab=recursion] {} {|\; X(\vec{u}) \;|\; \mu X(\vec{u}) . A}
  \and
  \inferrule* [lab=action] {} {\alpha \bc \langle x?(\vec{y}) \rangle \;|\; \langle x!(\vec{y}) \rangle \;|\; \langle \tau \rangle}
  \and 
  \inferrule* [lab=name] {} {\eta \bc x \;|\; \tau}
\end{mathpar} 

% subsection characteristic_formulae (end)   	 

\subsection{Example formulae}\label{sub:example_formulae_} % (fold)

\subsubsection{Crossing as formula.}
% 
% \begin{align*}
%   \frac{d}{dx} \sin x &= \cos x 
%   & \frac{d}{dx} e^x &= e^x \\
%   \frac{d}{dx} \cos x &= - \sin x 
%   & \frac{d}{dx} \log x &= \frac{1}{x} \\
% \end{align*} 

\begin{align*}
 \mu C(x_{0},x_{1},y_{0},y_{1},u).&(\langle x_{0}?(z) \rangle(\langle u! \rangle\langle y_{1}!z \rangle C(x_{0},x_{1},y_{0},y_{1},u)) & \\
  & \wedge \langle y_{1}?(z) \rangle (\langle u! \rangle \langle x_{0}!z \rangle C(x_{0},x_{1},y_{0},y_{1},u)) & \\
  & \wedge \langle x_{1}?(z) \rangle (\langle u? \rangle \langle y_{0}!z \rangle C(x_{0},x_{1},y_{0},y_{1},u)) & \\
  & \wedge \langle y_{0}?(z) \rangle (\langle u? \rangle \langle x_{1}!z \rangle C(x_{0},x_{1},y_{0},y_{1},u))) &
\end{align*}

The lexicographical similarity between the shape of this formulae and
the shape of definition of the process representing a crossing reveals
the intuitive meaning of this formulae. It describes the capabilities
of a process that has the right to represent a crossing. For example
it picks out processes that may perform an input on the port $x_0$ in
its initial menu of capabilities. What differentiates the formula
from the process, however, is that the crossing process is the
smallest candidate to satisfy the formula. Infinitely many other
processes -- with internal behavior hidden behind this interface, so
to speak -- also satisfy this formula. Even this simple formula,
then, can be seen to open a new view onto knots, providing a
computational interpretation of \emph{virtual} knots.

Note that this formula is derived by hand. A similar formula can be
derived by employing Caires' calculation of characteristic formula
\cite{Caires04} to the process representing a crossing. In light of
this discussion, we let
$\meaningof{C}_{\phi}(x0,x1,y0,y1,u)$ denote a formula specifying the
dynamics we wish to capture of a crossing. To guarantee we preserve
the shape of the interface and minimal semantics we demand that
$\meaningof{C}_{\phi}(x0,x1,y0,y1,u) \Rightarrow
\textbf{C}(x0,x1,y0,y1,u)$ where $\textbf{C}(x0,x1,y0,y1,u)$ denotes
the formula above.
                            
\subsubsection{Crossing number constraints.}
The moral content of the context lemma (Lemma \ref{context}) is that the notion of
``locality'' in the Reidemeister moves is effectively captured by the
parallel composition operator of the process calculus. This intuition
extends through the logic. Given a formula,
$\meaningof{C}_{\phi}(x0,x1,y0,y1,u)$, we can use the structural
connectives to specify constraints on crossing numbers, such as at
least $n$ crossings, or exactly $n$ crossings.
\begin{mathpar}
  \inferrule* [lab=at-least-n] {} { K^{\geq n}_{\phi}(\vec{xs},\vec{ys}) := \Pi_{i=0}^{n-1} Hu . \meaningof{C}_{\phi}(xs_i,ys_i,u) | T }
  \and 
  \inferrule* [lab=exactly-n] {} { K^{= n}_{\phi}(\vec{xs},\vec{ys}) := \Pi_{i=0}^{n-1} Hu . \meaningof{C}_{\phi}(xs_i,ys_i,u) | \neg (\forall x_0,y_0,x_1,y_1,u . \meaningof{C}_{\phi}(x_0,y_0,x_1,y_1,u) | T) }
\end{mathpar}

To round out this section, recall that the encoding of an $n$-crossing
knot decomposes into a parallel composition of $n$ \emph{copies} of a
crossing process together with a wiring harness. To specify different
knot classes with the same crossing number amounts to specifying
logical constraints on the wiring harness. In the interest of space,
we defer examples to a forthcoming paper. Suffice it to say that both
the conditions ``alternating knot'' and ``contains the tangle
corresponding to 5/3'' are expressible. For example, it is possible to
calculate the characteristic formula of a process corresponding to the
tangle 5/3 and conjoin it into the classifying formula via the
composition connective of the logic.

Finally, we wish to observe that it is entirely within reason to
contemplate a more domain-specific version of spatial logic tailored
to the shape of processes in the image of the encoding. Such a
domain-specific logic would have a better claim to the title formal
language of knot properties.

% subsection example_formulae_ (end)

% section knots_as_processes (end) 

% section spatial logic via knots (end)

\section{Conclusions and future work}

\paragraph{Testing physical space}
You, gentle reader, may wonder why of all the theorems to be proved
given this set up we pick the one above. In some sense it's hardly
central to quantum mechanics. We see it as central in the sense that
it firmly establishes a notion of physical space arising from a notion
of the equivalence of behavior. Relating bisimulation to a metric is a
big step forward, but one is faced with interpreting the relationship
of that metric space to something more physical. Quantum mechanical
notions of ``physical'' space are still far from intuitive, but by
relating this idea of distance as testing to calculations that predict
physical circumstances we are making a not insignificant step forward
toward an understanding of the physical space we inhabit as
essentially dynamic.

\paragraph{Effectivity and simulation}
One of the observations we have yet to make is that the entire program
spelled out here is effective. We have built various interpreters for
the reflective calculus at work in this interpretation. In principle,
then, we can simulate quantum mechanics on a computer. The place where
the simulation may lose fidelity is the infinitely branching summation
for the annihilator.

In this connection i also want to point out that the evaluation style
calculation of the inner product puts the non-determinism of the
summation right at the heart of measurement. This suggests that
Milner's original reduction-based formulation of the dynamics of his
calculi in terms of sums was not just notationally suggestive of a
notion of measure-and-continue but captured some significant part of
the physics.

\paragraph{Quantum continuations}
In light of this last observation i want to point out that the
predominant account of quantum mechanics is missing a key aspect of a
truly compositional story of the physical situation. In a real lab,
when a measurement is made the observation can be made to feed into
another device that then makes another measurement conditioned on the
results of the first. This means that after the superposition was
collapsed the entire experimental set up remained in
superposition. While QM offers a means of writing this down it doesn't
quite line up well with the well-trodden formulation of computation
and continuation that we see so succinctly expressed in Milner's
calculi. This suggests that there might be advantages to this account
of dynamics waiting to be explored.

\paragraph{Quantum logic}
In this connection, we also note that by virtue of having the
Hennessy-Milner construction, we can pull the construction through the
interpretation of QM. This gives us a natural candidate for a quantum
logic that enjoys an extremely tight connection with it's domain of
interpretation, making the construction much less ad hoc (rather it is
the image of functor!).

\paragraph{Quantum probabiity}
i have questions about the basis of the interpretation of inner
product as probability amplitude. In particular, using which
axiomatization of probability theory does the notion of probability
amplitude earn the right to be so dubbed? In other words, where is the
proof that the operation for calculating a probability amplitude (and
then squaring) satisfies the axioms of what it means to calculate a
probability? Even if such a proof exists (i have yet to find it in the
literature), i wonder if it might not be possible to turn things on
their heads. Can we view the calculation of the probability amplitude
as an axiomatization of probability? If so, then the definition we
give for calculating probability amplitude may provide the basis for
an \emph{effective} theory of probability.

\paragraph{Quantum vs ``biological'' information}
Finally, i want to conclude with a more philosophical observation. At
a recent workshop in which QM was a predominant topic i noticed
something about quantum information. The speaker was giving a riveting
discussion of axiomatic QM and showing how properties of ``no
cloning'' and ``no deleting'' emerged as consequences of the
axiomatization. Theorems of this form are necessary to give us a sense
of confidence that our axioms characterize the physical theory. What
struck me, though, was that if quantum information is neither erasable
nor replicable it is markedly different from \emph{life}. Two of the
things we know about life is that

\begin{itemize}
  \item it ends;
  \item to gain some measure of persistence, to transcend it's
    finitude it is imminently copyable.
\end{itemize}

Both of these qualities are summarized succinctly in the aphorism: all
flesh is grass. For me these two kinds of ``information'' -- call them
quantum and biological -- are end points on a spectrum of strategies
for persistence. At one end, we have those curious entities that enjoy
uniqueness and permanence; at the other, we have those who in the face
of a certain end and an uncertain present make a go of passing
something on. To me one of the more remarkable aspects of the latter
strategy is that in the presence of noise (and certain features of
copying) we get a kind of dynamism, a chance for improvement against a
given persistent condition.

% subsection other_calculi_other_bisimulations_and_geometry_as_behavior (end)




% section conclusion (end)

%\documentclass[12pt]{llncs}
%\documentclass{jktr}

\usepackage[pdftex]{hyperref}                   
\usepackage {listings}
\usepackage {mathpartir}
\usepackage{bcprules}
%\usepackage{listings}
                       
\usepackage{graphicx} 
%\usepackage[margins=2.5cm,nohead,nofoot]{geometry}
%\usepackage{geometry}
\usepackage{amsfonts}
\usepackage{amstext}
\usepackage{latexsym}
\usepackage{amssymb}
\usepackage{color}


%\include{myPreamble}
\include{qm2pi.local} 

%\ifpdf
%\usepackage[pdftex]{graphicx}
%\else
%\usepackage{graphicx}
%\fi

 % \ifpdf
%  \usepackage{pdfsync}
%  \if


%\title{Brief Article}
%\author{David F. Snyder}
%\author{L.G. Meredith}

%\address{Dept. of Math., Texas State University--San Marcos, San Marcos, TX 78666}
       
\pagestyle{empty}


\begin{document}

\lstset{language=[Objective]Caml,frame=shadowbox}

\input{qm2pi.front}

% section front matter (end)

\input{qm2pi.intro} 
 
% section introduction (end)

% \input{qm2pi.knotations} 

% section notation (end)

\input{qm2pi.process.calculi} 

% section concurrent_process_calculi_and_spatial_logics_ (end)
    
%\input{qm2pi.knots2pi} 

%\input{qm2pi.trefoil} 

%\input{qm2pi.mainthm} 

% subsection basic_interpretation (end)

%\input{qm2pi.rho.presentation} 
\subsection{The syntax and semantics of the notation system}\label{sub:the_syntax_and_semantics_of_the_notation_system} % (fold)

We now summarize a technical presentation of the calculus that
embodies our theory of dynamics. The typical presentation of such a
calculus follows the style of giving generators and relations on
them. The grammar, below, describing term constructors, freely
generates the set of processes, $\Proc$. This set is then quotiented
by a relation known as structural congruence and it is over this set
that the notion of dynamics is expressed. This presentation is
essentially that of \cite{MeredithR05} with the addition of
polyadicity and summation. For readability we have relegated some of
the technical subtleties to an appendix.

\subsubsection{Process grammar}\label{subsub:process_grammar}

\begin{mathpar}
  \inferrule* [lab=synchronization] {} {{M} \bc \pzero \;|\; x?F \;|\; x!C }
  \and
  \inferrule* [lab=abstraction] {} {{F} \bc (x)P}
  \and
  \inferrule* [lab=concretion] {} {{C} \bc \langle Q \rangle}
  \and
  \inferrule* [lab=process] {} {{P,Q} \bc M \;| \;P|Q \;|\; @{x}}
  \and
  \inferrule* [lab=name] {} {{x} \bc \quotep{P}}
\end{mathpar} 

Note that $\vec{x}$ (resp. $\vec{P}$) denotes a vector of names
(resp. processes) of length $|\vec{x}|$ (resp. $|\vec{P}|$). We adopt
the following useful abbreviations.

\begin{mathpar}
   x?(\vec{y}).P := x.(\vec{y})P \and  x\clift{\vec{P}} := x.\clift{\vec{P}}
   \and x!(y) := \lift{x}{\dropn{y}}
   \and \Pi_{i=0}^{n-1}P_i := P_0 | \ldots | P_{n-1}
\end{mathpar}

\subsubsection{Structural congruence}

\paragraph{Free and bound names and alpha-equivalence.} At the
core of structural equivalence is alpha-equivalence which identifies
process that are the same up to a change of variable. Formally, we
recognize the distinction between free and bound names. The free names
of a process, $\freenames{P}$, may be calculated recursively as
follows:

\begin{mathpar}
\freenames{\pzero} := \emptyset
  \and \\
  \freenames{x?(y).P} := \{ x \} \cup (\freenames{P} \setminus \{ y \})
  \and 
  \freenames{x!\langle P \rangle} := \{ x \} \cup \{ P \} 
  \and \\
  \freenames{P|Q} := \freenames{P} \cup \freenames{Q}
  \and \\
  \freenames{@{x}} := \{ x \}
\end{mathpar}

$\pi$
$\quotep{\pi}$

$\freenames{-} : \pi \to \mathcal{P}(\quotep{\pi})$

\begin{eqnarray*}
  \freenames{\pzero} & := & \emptyset \\
  \freenames{x?(y).P} & := & \{ x \} \cup (\freenames{P} \setminus \{ y \}) \\
  \freenames{x!\langle P \rangle} & := & \{ x \} \cup \{ P \} \\
  \freenames{P|Q} & := & \freenames{P} \cup \freenames{Q} \\
  \freenames{\dropn{x}} & := & \{ x \}
\end{eqnarray*}

The bound names of a process, $\boundnames{P}$, are those names occurring in $P$
that are not free. For example, in $x?(y).0$, the name $x$ is free, while $y$ is bound.

\begin{mathpar}
  \inferrule* [lab=monoidal-laws] {} { P|Q \equiv Q|P \and P|0 \equiv P \and P|(Q|R) \equiv (P|Q)|R }
\end{mathpar}

\begin{mathpar}
  \inferrule* [lab=alpha-equivalence] {} { (x)P \equiv (y)P\{y/x\} \and y \not\in \freenames{P} }
\end{mathpar}

\begin{definition}
Then two processes, $P,Q$, are alpha-equivalent if $P = Q\{\vec{y}/\vec{x}\}$ for
some $\vec{x} \in \boundnames{Q},\vec{y} \in \boundnames{P}$, where $Q\{\vec{y}/\vec{x}\}$
denotes the capture-avoiding substitution of $\vec{y}$ for $\vec{x}$ in $Q$.
\end{definition}

\begin{definition}
  The {\em structural congruence} \cite{SangiorgiWalker} , $\equiv$,
  between processes is the least congruence containing
  alpha-equivalence, satisfying the abelian monoid laws
  (associativity, commutativity and $\pzero$ as identity) for parallel
  composition $|$ and for summation $+$.
\end{definition}

\subsection{Name equivalence}

We take name equivalence, written $\nameeq$, to be the smallest
equivalence relation generated by the following rules.

\begin{mathpar}
\inferrule*[lab=Quote-drop]
{ }
{ \quotep{@{x}} \nameeq x }

\inferrule*[lab=Struct-equiv]
{ P \scong Q }
{ \quotep{P} \nameeq \quotep{Q} }
\end{mathpar}

The astute reader will have noticed that the mutual recursion of names
and processes imposes a mutual recursion on alpha-equivalence and
structural equivalence via name-equivalence. Fortunately, all of this
works out pleasantly and we may calculate in the natural way, free of
concern. The reader interested in the details is referred to the
appendix \ref{appendix:rho_details}.

\subsection{Substitution}

We use $\Proc$ for the set of processes, $\QProc$ for the set of
names, and $\id{\{}\vec{y} / \vec{x} \id{\}}$ to denote partial maps,
$s : \QProc \rightarrow \QProc$. A map, $s$ lifts, uniquely, to a map
on process terms, $\widehat{s} : \Proc \rightarrow \Proc$ by the
following equations.

\begin{mathpar}
  (0) \psubstp{Q}{P} := 0 \\
  (R \juxtap S) \psubstp{Q}{P}
  :=    
  (R)\psubstp{Q}{P} \juxtap (S) \psubstp{Q}{P} \\
  (x?(y).R) \psubstp{Q}{P}    
  :=    
  (x)\substp{Q}{P} (z)\concat( (R \psubstn{z}{y}) \psubstp{Q}{P} ) \\
  (\lift{x}{R}) \psubstp{Q}{P}  
  :=
  \lift{(x)\substp{Q}{P}}{ R \psubstp{Q}{P} } \\
%   (\dropn{x})  \psubstp{Q}{P}       
%   := 
%   \left\{ 
%     \begin{array}{ccc} 
%       \dropn{\quotep{Q}} & & x \nameeq \quotep{P} \\
%       \dropn{x} & & otherwise \\
%     \end{array}
%   \right. 
  (\dropn{x})  \psubstp{Q}{P}       
  := 
  \left\{ 
    \begin{array}{ccc} 
      Q & & x \nameeq \quotep{P} \\
      \dropn{x} & & otherwise \\
    \end{array}
  \right.
\end{mathpar}
 

where

\begin{eqnarray}
  (x)\id{\{} \lpquote Q \rpquote / \lpquote P \rpquote \id{\}}            = 
  \left\{ 
    \begin{array}{ccc}
      \lpquote Q \rpquote & & x \nameeq \lpquote P \rpquote \\
      x & & otherwise \\
    \end{array}
  \right. \nonumber
\end{eqnarray}

and $z$ is chosen distinct from $\quotep{P}$, $\quotep{Q}$, the free
names in $Q$, and all the names in $R$. Our $\alpha$-equivalence will
be built in the standard way from this substitution.

\begin{remark}\label{rem:no_self_referential_names}
  One consequence of these definitions is that $\forall P. \quotep{P}
  \not\in \freenames{P}$.
\end{remark}

\subsection{ Dynamic quote: an example }

Anticipating something of what's to come, consider applying the
substitution, $\widehat{\id{\{}u / z \id{\}}}$, to the following pair
of processes, $\lift{w}{y!(z)}$ and $w[ \lpquote y!(z) \rpquote ]$.

\begin{eqnarray}
	\lift{w}{y!(z)}\widehat{\id{\{}u / z \id{\}}}
		& = &
		\lift{w}{y!(u)} \nonumber\\
	w[ \lpquote y!(z) \rpquote ] \widehat{ \id{\{}u / z \id{\}} }
		& = &
		w[ \lpquote y!(z) \rpquote ] \nonumber
\end{eqnarray}

Because the body of the process between quotes is impervious to
substitution, we get radically different answers. In fact, by
examining the first process in an input context,
e.g. $x?(z).\lift{w}{y!(z)}$, we see that the process under the lift
operator may be shaped by prefixed inputs binding a name inside it. In
this sense, the lift operator will be seen as a way to dynamically
construct processes before reifying them as names.

Finally equipped with these standard features we can present the
dynamics of the calculus.

\subsubsection{Operational semantics} 

Finally, we introduce the computational dynamics. What marks these
algebras as distinct from other more traditionally studied algebraic
structures, e.g. vector spaces or polynomial rings, is the manner in
which dynamics is captured. In traditional structures, dynamics is typically
expressed through morphisms between such structures, as in linear maps
between vector spaces or morphisms between rings. In algebras
associated with the semantics of computation, the dynamics is
expressed as part of the algebraic structure itself, through a
reduction reduction relation typically denoted by $\red$. Below, we
give a recursive presentation of this relation for the calculus used
in the encoding.

$\red \subseteq \pi \times \pi$
$\red : \pi \to \mathcal{P}(\pi)$

\begin{mathpar}
  \inferrule* [lab=Comm] { \textsf{match}( x_{src}, x_{trgt} ) } { x_{trgt}?(y)P \; | \; x_{src}!\langle {Q} \rangle \red P\{\quotep{Q}/y}\} }
  \and \\
  \inferrule* [lab=Par] {{P} \red {P}'} {{{P} | {Q}} \red {{P}' | {Q}}}
  \and
  \inferrule* [lab=Equiv]{{{P} \scong {P}'} \andalso {{P}' \red {Q}'} \andalso {{Q}' \scong {Q}}}{{P} \red {Q}}
\end{mathpar}

\begin{eqnarray*}
  match_{\equiv} (\quotep{P},\quotep{Q}) & := & P \equiv Q \\
  match_{\dagger}(\quotep{P},\quotep{Q}) & := & \forall R. P|Q \red^{*} R => R \red^{*} 0 \\
  match_{K}(\quotep{P},\quotep{Q}) & := & K \mbox{ for some context } K
\end{eqnarray*}

$u?(x)P | u!\langle Q \rangle \red P\{\quotep{Q}/x\}$

%We write $\wred$ for $\red^*$, and $P\red$ if $\exists Q $ such that $ P \red Q$.
We write $P\red$ if $\exists Q $ such that $ P \red Q$ and $P\not\red$, otherwise.

\section{Replication}

As mentioned before, it is known that replication (and hence
recursion) can be implemented in a higher-order process algebra
\cite{SangiorgiWalker}. As our first example of calculation with the
machinery thus far presented we give the construction explicitly in
the {\rhoc}.

\begin{eqnarray}
	D_{x} & := & \prefix{x}{y}{(\binpar{\outputp{x}{y}}{@{y}})} \nonumber\\
	\bangp_{x}{P} & := & \binpar{{x}!\langle{\binpar{D_{x}}{P}}\rangle}{D_{x}} \nonumber
\end{eqnarray}

\begin{eqnarray}
	\bangp_{x}{P} & & \nonumber\\
	=
	& {x}!\langle{(\prefix{x}{y}{(\outputp{x}{y} | @{y})) | P}}\rangle 
	      | \prefix{x}{y}{(\outputp{x}{y} | @{y})} & \nonumber\\
	\red
	& (\outputp{x}{y} | @{y})\substn{\quotep{(\prefix{x}{y}{(@{y} | \outputp{x}{y})) | P}}}{y} & \nonumber\\
	=
	& \outputp{x}{\quotep{(\prefix{x}{y}{(\outputp{x}{y} | @{y})) | P}}}
	  | {(\prefix{x}{y}{(\outputp{x}{y} | @{y})) | P}} & \nonumber\\
	\red
	& \ldots & \nonumber\\
	\red^*
	& P | P | \ldots & \nonumber
\end{eqnarray}

Of course, this encoding, as an implementation, runs away, unfolding
$\bangp{P}$ eagerly. A lazier and more implementable replication
operator, restricted to input-guarded processes, may be obtained as follows.

\begin{eqnarray}
\bangp{\prefix{u}{v}{P}} 
	:= 
	\binpar{\lift{x}{\prefix{u}{v}{(\binpar{D(x)}{P})}}}{D(x)} \nonumber
\end{eqnarray}

\begin{remark}
  Note that the lazier definition still does not deal with summation
  or mixed summation (i.e. sums over input and output). The reader is
  invited to construct definitions of replication that deal with these
  features. 

  Further, the definitions are parameterized in a name, $x$. Can you,
  gentle reader, make a definition that eliminates this parameter and
  guarantees no accidental interaction between the replication
  machinery and the process being replicated -- i.e. no accidental
  sharing of names used by the process to get its work done and the
  name(s) used by the replication to effect copying. This latter
  revision of the definition of replication is crucial to obtaining
  the expected identity $!!P \sim !P$.
\end{remark}

\begin{remark}\label{rem:paradoxical_combinator}
  The reader familiar with the lambda calculus will have noticed the
  similarity between $D$ and the paradoxical combinator.

  [Ed. note: the existence of this seems to suggest we have to be more
  restrictive on the set of processes and names we admit if we are to
  support no-cloning.]
\end{remark}

\subsubsection{Bisimulation}

The computational dynamics gives rise to another kind of equivalence,
the equivalence of computational behavior. As previously mentioned
this is typically captured \emph{via} some form of bisimulation.

% The notion we use in this paper is weak barbed bisimulation
% \cite{milner91polyadicpi}.

The notion we use in this paper is derived from weak barbed
bisimulation \cite{milner91polyadicpi}. 

\begin{definition}
An \emph{observation relation}, $\downarrow_{\mathcal N}$, over a set
of names, $\mathcal N$, is the smallest relation satisfying the rules
below.

\infrule[Out-barb]{y \in {\mathcal N}, \; x \nameeq y}
		  {\outputp{x}{v} \downarrow_{\mathcal N} x}
\infrule[Par-barb]{\mbox{$P\downarrow_{\mathcal N} x$ or $Q\downarrow_{\mathcal N} x$}}
		  {\binpar{P}{Q} \downarrow_{\mathcal N} x}

We write $P \Downarrow_{\mathcal N} x$ if there is $Q$ such that 
$P \wred Q$ and $Q \downarrow_{\mathcal N} x$.
\end{definition}

\begin{definition}
%\label{def.bbisim}
An  ${\mathcal N}$-\emph{barbed bisimulation} over a set of names, ${\mathcal N}$, is a symmetric binary relation 
${\mathcal S}_{\mathcal N}$ between agents such that $P\rel{S}_{\mathcal N}Q$ implies:
\begin{enumerate}
\item If $P \red P'$ then $Q \wred Q'$ and $P'\rel{S}_{\mathcal N} Q'$.
\item If $P\downarrow_{\mathcal N} x$, then $Q\Downarrow_{\mathcal N} x$.
\end{enumerate}
$P$ is ${\mathcal N}$-barbed bisimilar to $Q$, written
$P \wbbisim_{\mathcal N} Q$, if $P \rel{S}_{\mathcal N} Q$ for some ${\mathcal N}$-barbed bisimulation ${\mathcal S}_{\mathcal N}$.
\end{definition}

$\mathcal{R} \subseteq \pi \times \pi$

$P \mathcal{R} Q => \forall P'. P \red P' \Rightarrow \exists Q'. Q \red Q', P' \mathcal{R} Q'$

$P \vdash x \Rightarrow Q \vdash x$

\begin{mathpar}
  \inferrule*[lab=Out-barb]{x \nameeq y}{{y}!\langle{Q}\rangle \vdash x}
  \and
  \inferrule*[lab=Par-barb]{\mbox{$P\vdash x$ or $Q\vdash x$}}{\binpar{P}{Q} \vdash x}
\end{mathpar}

\subsubsection{Contexts}

One of the principle advantages of computational calculi like the
$\pi$-calculus is a well-defined notion of context,
contextual-equivalence and a correlation between
contextual-equivalence and notions of bisimulation. The notion of
context allows the decomposition of a process into (sub-)process and
its syntactic environment, its context. Thus, a context may be
thought of as a process with a ``hole'' (written $\Box$) in it. The
application of a context $M$ to a process $P$, written $M[P]$, is
tantamount to filling the hole in $M$ with $P$. In this paper we do
not need the full weight of this theory, but do make use of the notion
of context in the proof the main theorem. 

\begin{mathpar}
  \inferrule* [lab=summation] {} {{M_{M},M_{N}} \bc \Box \;|\; x.M_{A} \;|\; M_{M}+M_{N}}
  \and
  \inferrule* [lab=agent] {} {{M_{A}} \bc (\vec{x})M_{P} \;| \; \clift{P_0,\ldots,M_{P},\ldots,P_N}}
  \and \\
  \inferrule* [lab=process] {} {{M_{P}} \bc M_{N} \;| \;P|M_{P} }
\end{mathpar} 

\begin{mathpar}
  \inferrule* [lab=sychronization] {} {M_{N} \bc \Box \;|\; x?M_{F} \;|\; x!M_{C}}
  \and
  \inferrule* [lab=abstraction] {} {{M_{F}} \bc (x)M_{P} }
  \and
  \inferrule* [lab=concretion] {} {{M_{C}} \bc \langle M_{P} \rangle }
  \and \\
  \inferrule* [lab=process] {} {{M_{P}} \bc M_{N} \;| \;P|M_{P} }
\end{mathpar}

\begin{definition}[contextual application] Given a context $M$, and
  process $P$, we define the \emph{contextual application}, $M[P] :=
  M\{P/\Box\}$. That is, the contextual application of M to P is the
  substitution of $P$ for $\Box$ in $M$.
\end{definition}

$\meaningof{-} : L \to \mathcal{P}(\pi)$

\begin{mathpar}
  \inferrule* [lab=collection] {} {\meaningof{true} = \pi, \and \meaningof{~E} = \pi \setminus \meaningof{E}, \and \meaningof{E_{1} \& E_{2}} = \meaningof{E_{1}} \cap \meaningof{E_{2}}}
\end{mathpar}

\begin{mathpar}
  \inferrule* [lab=structure] {} {\meaningof{0} = \{ P \in \pi | P \equiv 0 \}, \and \\ \meaningof{E_1 | E_2} = \{ P \in \pi | P \equiv P_{1} | P_{2}, P_{1} \in \meaningof{E_{1}}, P_{2} \in \meaningof{E_2}\} }
\end{mathpar}

\begin{mathpar}
 \inferrule* [lab=behavior] {} {\meaningof{\langle a?b \rangle E} = \{ P \in \pi | P \equiv Q | u?(y)P', \\ \and \\\\ \and \\ \;\;\; u \in \meaningof{a}, \forall z.P'\{z/y\} \in \meaningof{E\{z/b\}}\}, \and \\ \meaningof{a!E} = \{ P \in \pi | P \equiv Q | x!\langle P' \rangle, x \in \meaningof{a} P' \in \meaningof{E}\} }
\end{mathpar}

\begin{mathpar}
 \inferrule* [lab=nominal] {} {\meaningof{\quotep{E}} = \{ \quotep{P} \in \quotep{\pi} | P \in \meaningof{E} \}, \and \meaningof{\quotep{P}} = \{ \quotep{Q} \in \quotep{\pi} | P \equiv Q \} \and \\ \meaningof{@\quotep{E}} = \{ P \in \pi | P \equiv @x, x \in \meaningof{E} \}}
\end{mathpar}

\begin{eqnarray*}
  \\
  \meaningof{-} : TS \to ST
\end{eqnarray*}

\begin{eqnarray*}
  \\
  L : TS \to ST
\end{eqnarray*}

\begin{eqnarray*}
  \\
  P \models E \iff P \in \meaningof{E}
\end{eqnarray*}

\begin{eqnarray*}
  P \approx_{L} Q \iff \forall E \in L. P \models E \iff Q \models E
\end{eqnarray*}

\begin{eqnarray*}
  P \approx_{K} Q
\end{eqnarray*}

\begin{eqnarray*}
  P \approx Q
\end{eqnarray*}

$\approx_{K} = \approx = \approx_{L}$

\subsubsection{Contextual duality}

Note that contexts extend the quotation operation to a family of
operations from processes to names. Given a context, $M$, we can
define a \emph{nominal context}, $\quotep{M}$ by $\quotep{M}[P] :=
\quotep{M[P]}$. To foreshadow what is to come we observe that these
operations enjoy a duality with processes very much like the duality
between vectors and maps from vectors to scalars.

Further, because the calculus is essentially higher-order, we have a
correspondence between contexts and processes. More specifically,
given a name $x$ and a context $M$ we can construct $M^{*}_{x}$ such
that 

\begin{mathpar}
  M^{*}_{x} | \lift{x}{P} \red M[P]
\end{mathpar}

namely,

\begin{mathpar}
  M^{*}_{x} := x?(u).M[\dropn{u}]
\end{mathpar}

The dependence of $M^{*}_{x}$ on a name makes it an abstraction, 

\begin{mathpar}
  M^{*} := (x)x?(u).M[\dropn{u}]
\end{mathpar}

\subsection{Additional notation}

It will sometimes be convenient to denote the process a name
quotes. We already have the notation $x = \quotep{P}$, but it will be
convenient to introduce an alternate notation, $\procn{x}$, when we
want to emphasize the connection to the use of the name. Note that, by
virtue of name equivalence, $\quotep{\procn{x}} \nameeq x$; so, the
notation is consistent with previous definitions.

Further, because names have structure it is possible to effect
substitutions on the basis of that structure. This means we need to
upgrade our notation for substitutions, which we accomplish by
adapting comprehension notation. Thus,

\begin{mathpar}
  P\{ y / x : x \in S \}
\end{mathpar}

is interpreted to mean the process derived from P by replacing (in a
capture-avoiding manner) each occurrence of $x$ in $S$ by $y$. For example,

\begin{mathpar}
  P\{ \quotep{\procn{x}|\procn{x}} / x : x \in \freenames{P} \}
\end{mathpar}

will replace each (occurrence) of a free name $x$ in $P$ by
$\quotep{\procn{x}|\procn{x}}$.

Also, we will avail ourselves of the notation $x^{L}$ and $x^{R}$ to
denote injections of a name into disjoint copies of the name
space. There are numerous ways to accomplish this. One example can be
found in \cite{MeredithR05}. This notation overloads to vectors of
names: $\vec{x}^{\pi} := (x_{i}^{\pi} \; : \; 0 \leq i < |\vec{x}| )$ where $\pi \in \{L,R\}$.

We also use $P^{\Box} := P|\Box$.

In \cite{MeredithR05} an interpretation of the new operator is
given. It turns out that there are several possible interpretations
all enjoying the requisite algebraic properties of the operator (see
\cite{milner91polyadicpi}). We will therefore make liberal use of
$(\nu\; \vec{x})P$.

% subsection the_syntax_and_semantics_of_the_notation_system (end)   

\input{qm2pi.qmops} 

\input{qm2pi.sterngerlach} 

\input{qm2pi.metric} 

% section concurrent_process_calculi (end)

%\input{qm2pi.proofsketch}

% section proof sketch (end)

%\input{qm2pi.slviaknots} 

% section spatial logic via knots (end)

\input{qm2pi.conclusion}

% section conclusion (end)

%\input{qm2pi.dtcodes} 

% section wiring algorithm (end)

\input{qm2pi.ack} 

% section acknowledgments (end)

\newpage


\bibliographystyle{plain}   
\bibliography{../../biblios/main.bib}

\input{qm2pi.rhodetails}

\end{document}

 

% section wiring algorithm (end)

\documentclass[12pt]{llncs}
%\documentclass{jktr}

\usepackage[pdftex]{hyperref}                   
\usepackage {listings}
\usepackage {mathpartir}
\usepackage{bcprules}
%\usepackage{listings}
                       
\usepackage{graphicx} 
%\usepackage[margins=2.5cm,nohead,nofoot]{geometry}
%\usepackage{geometry}
\usepackage{amsfonts}
\usepackage{amstext}
\usepackage{latexsym}
\usepackage{amssymb}
\usepackage{color}


%\include{myPreamble}
\include{qm2pi.local} 

%\ifpdf
%\usepackage[pdftex]{graphicx}
%\else
%\usepackage{graphicx}
%\fi

 % \ifpdf
%  \usepackage{pdfsync}
%  \if


%\title{Brief Article}
%\author{David F. Snyder}
%\author{L.G. Meredith}

%\address{Dept. of Math., Texas State University--San Marcos, San Marcos, TX 78666}
       
\pagestyle{empty}


\begin{document}

\lstset{language=[Objective]Caml,frame=shadowbox}

\input{qm2pi.front}

% section front matter (end)

\input{qm2pi.intro} 
 
% section introduction (end)

% \input{qm2pi.knotations} 

% section notation (end)

\input{qm2pi.process.calculi} 

% section concurrent_process_calculi_and_spatial_logics_ (end)
    
%\input{qm2pi.knots2pi} 

%\input{qm2pi.trefoil} 

%\input{qm2pi.mainthm} 

% subsection basic_interpretation (end)

%\input{qm2pi.rho.presentation} 
\subsection{The syntax and semantics of the notation system}\label{sub:the_syntax_and_semantics_of_the_notation_system} % (fold)

We now summarize a technical presentation of the calculus that
embodies our theory of dynamics. The typical presentation of such a
calculus follows the style of giving generators and relations on
them. The grammar, below, describing term constructors, freely
generates the set of processes, $\Proc$. This set is then quotiented
by a relation known as structural congruence and it is over this set
that the notion of dynamics is expressed. This presentation is
essentially that of \cite{MeredithR05} with the addition of
polyadicity and summation. For readability we have relegated some of
the technical subtleties to an appendix.

\subsubsection{Process grammar}\label{subsub:process_grammar}

\begin{mathpar}
  \inferrule* [lab=synchronization] {} {{M} \bc \pzero \;|\; x?F \;|\; x!C }
  \and
  \inferrule* [lab=abstraction] {} {{F} \bc (x)P}
  \and
  \inferrule* [lab=concretion] {} {{C} \bc \langle Q \rangle}
  \and
  \inferrule* [lab=process] {} {{P,Q} \bc M \;| \;P|Q \;|\; @{x}}
  \and
  \inferrule* [lab=name] {} {{x} \bc \quotep{P}}
\end{mathpar} 

Note that $\vec{x}$ (resp. $\vec{P}$) denotes a vector of names
(resp. processes) of length $|\vec{x}|$ (resp. $|\vec{P}|$). We adopt
the following useful abbreviations.

\begin{mathpar}
   x?(\vec{y}).P := x.(\vec{y})P \and  x\clift{\vec{P}} := x.\clift{\vec{P}}
   \and x!(y) := \lift{x}{\dropn{y}}
   \and \Pi_{i=0}^{n-1}P_i := P_0 | \ldots | P_{n-1}
\end{mathpar}

\subsubsection{Structural congruence}

\paragraph{Free and bound names and alpha-equivalence.} At the
core of structural equivalence is alpha-equivalence which identifies
process that are the same up to a change of variable. Formally, we
recognize the distinction between free and bound names. The free names
of a process, $\freenames{P}$, may be calculated recursively as
follows:

\begin{mathpar}
\freenames{\pzero} := \emptyset
  \and \\
  \freenames{x?(y).P} := \{ x \} \cup (\freenames{P} \setminus \{ y \})
  \and 
  \freenames{x!\langle P \rangle} := \{ x \} \cup \{ P \} 
  \and \\
  \freenames{P|Q} := \freenames{P} \cup \freenames{Q}
  \and \\
  \freenames{@{x}} := \{ x \}
\end{mathpar}

$\pi$
$\quotep{\pi}$

$\freenames{-} : \pi \to \mathcal{P}(\quotep{\pi})$

\begin{eqnarray*}
  \freenames{\pzero} & := & \emptyset \\
  \freenames{x?(y).P} & := & \{ x \} \cup (\freenames{P} \setminus \{ y \}) \\
  \freenames{x!\langle P \rangle} & := & \{ x \} \cup \{ P \} \\
  \freenames{P|Q} & := & \freenames{P} \cup \freenames{Q} \\
  \freenames{\dropn{x}} & := & \{ x \}
\end{eqnarray*}

The bound names of a process, $\boundnames{P}$, are those names occurring in $P$
that are not free. For example, in $x?(y).0$, the name $x$ is free, while $y$ is bound.

\begin{mathpar}
  \inferrule* [lab=monoidal-laws] {} { P|Q \equiv Q|P \and P|0 \equiv P \and P|(Q|R) \equiv (P|Q)|R }
\end{mathpar}

\begin{mathpar}
  \inferrule* [lab=alpha-equivalence] {} { (x)P \equiv (y)P\{y/x\} \and y \not\in \freenames{P} }
\end{mathpar}

\begin{definition}
Then two processes, $P,Q$, are alpha-equivalent if $P = Q\{\vec{y}/\vec{x}\}$ for
some $\vec{x} \in \boundnames{Q},\vec{y} \in \boundnames{P}$, where $Q\{\vec{y}/\vec{x}\}$
denotes the capture-avoiding substitution of $\vec{y}$ for $\vec{x}$ in $Q$.
\end{definition}

\begin{definition}
  The {\em structural congruence} \cite{SangiorgiWalker} , $\equiv$,
  between processes is the least congruence containing
  alpha-equivalence, satisfying the abelian monoid laws
  (associativity, commutativity and $\pzero$ as identity) for parallel
  composition $|$ and for summation $+$.
\end{definition}

\subsection{Name equivalence}

We take name equivalence, written $\nameeq$, to be the smallest
equivalence relation generated by the following rules.

\begin{mathpar}
\inferrule*[lab=Quote-drop]
{ }
{ \quotep{@{x}} \nameeq x }

\inferrule*[lab=Struct-equiv]
{ P \scong Q }
{ \quotep{P} \nameeq \quotep{Q} }
\end{mathpar}

The astute reader will have noticed that the mutual recursion of names
and processes imposes a mutual recursion on alpha-equivalence and
structural equivalence via name-equivalence. Fortunately, all of this
works out pleasantly and we may calculate in the natural way, free of
concern. The reader interested in the details is referred to the
appendix \ref{appendix:rho_details}.

\subsection{Substitution}

We use $\Proc$ for the set of processes, $\QProc$ for the set of
names, and $\id{\{}\vec{y} / \vec{x} \id{\}}$ to denote partial maps,
$s : \QProc \rightarrow \QProc$. A map, $s$ lifts, uniquely, to a map
on process terms, $\widehat{s} : \Proc \rightarrow \Proc$ by the
following equations.

\begin{mathpar}
  (0) \psubstp{Q}{P} := 0 \\
  (R \juxtap S) \psubstp{Q}{P}
  :=    
  (R)\psubstp{Q}{P} \juxtap (S) \psubstp{Q}{P} \\
  (x?(y).R) \psubstp{Q}{P}    
  :=    
  (x)\substp{Q}{P} (z)\concat( (R \psubstn{z}{y}) \psubstp{Q}{P} ) \\
  (\lift{x}{R}) \psubstp{Q}{P}  
  :=
  \lift{(x)\substp{Q}{P}}{ R \psubstp{Q}{P} } \\
%   (\dropn{x})  \psubstp{Q}{P}       
%   := 
%   \left\{ 
%     \begin{array}{ccc} 
%       \dropn{\quotep{Q}} & & x \nameeq \quotep{P} \\
%       \dropn{x} & & otherwise \\
%     \end{array}
%   \right. 
  (\dropn{x})  \psubstp{Q}{P}       
  := 
  \left\{ 
    \begin{array}{ccc} 
      Q & & x \nameeq \quotep{P} \\
      \dropn{x} & & otherwise \\
    \end{array}
  \right.
\end{mathpar}
 

where

\begin{eqnarray}
  (x)\id{\{} \lpquote Q \rpquote / \lpquote P \rpquote \id{\}}            = 
  \left\{ 
    \begin{array}{ccc}
      \lpquote Q \rpquote & & x \nameeq \lpquote P \rpquote \\
      x & & otherwise \\
    \end{array}
  \right. \nonumber
\end{eqnarray}

and $z$ is chosen distinct from $\quotep{P}$, $\quotep{Q}$, the free
names in $Q$, and all the names in $R$. Our $\alpha$-equivalence will
be built in the standard way from this substitution.

\begin{remark}\label{rem:no_self_referential_names}
  One consequence of these definitions is that $\forall P. \quotep{P}
  \not\in \freenames{P}$.
\end{remark}

\subsection{ Dynamic quote: an example }

Anticipating something of what's to come, consider applying the
substitution, $\widehat{\id{\{}u / z \id{\}}}$, to the following pair
of processes, $\lift{w}{y!(z)}$ and $w[ \lpquote y!(z) \rpquote ]$.

\begin{eqnarray}
	\lift{w}{y!(z)}\widehat{\id{\{}u / z \id{\}}}
		& = &
		\lift{w}{y!(u)} \nonumber\\
	w[ \lpquote y!(z) \rpquote ] \widehat{ \id{\{}u / z \id{\}} }
		& = &
		w[ \lpquote y!(z) \rpquote ] \nonumber
\end{eqnarray}

Because the body of the process between quotes is impervious to
substitution, we get radically different answers. In fact, by
examining the first process in an input context,
e.g. $x?(z).\lift{w}{y!(z)}$, we see that the process under the lift
operator may be shaped by prefixed inputs binding a name inside it. In
this sense, the lift operator will be seen as a way to dynamically
construct processes before reifying them as names.

Finally equipped with these standard features we can present the
dynamics of the calculus.

\subsubsection{Operational semantics} 

Finally, we introduce the computational dynamics. What marks these
algebras as distinct from other more traditionally studied algebraic
structures, e.g. vector spaces or polynomial rings, is the manner in
which dynamics is captured. In traditional structures, dynamics is typically
expressed through morphisms between such structures, as in linear maps
between vector spaces or morphisms between rings. In algebras
associated with the semantics of computation, the dynamics is
expressed as part of the algebraic structure itself, through a
reduction reduction relation typically denoted by $\red$. Below, we
give a recursive presentation of this relation for the calculus used
in the encoding.

$\red \subseteq \pi \times \pi$
$\red : \pi \to \mathcal{P}(\pi)$

\begin{mathpar}
  \inferrule* [lab=Comm] { \textsf{match}( x_{src}, x_{trgt} ) } { x_{trgt}?(y)P \; | \; x_{src}!\langle {Q} \rangle \red P\{\quotep{Q}/y}\} }
  \and \\
  \inferrule* [lab=Par] {{P} \red {P}'} {{{P} | {Q}} \red {{P}' | {Q}}}
  \and
  \inferrule* [lab=Equiv]{{{P} \scong {P}'} \andalso {{P}' \red {Q}'} \andalso {{Q}' \scong {Q}}}{{P} \red {Q}}
\end{mathpar}

\begin{eqnarray*}
  match_{\equiv} (\quotep{P},\quotep{Q}) & := & P \equiv Q \\
  match_{\dagger}(\quotep{P},\quotep{Q}) & := & \forall R. P|Q \red^{*} R => R \red^{*} 0 \\
  match_{K}(\quotep{P},\quotep{Q}) & := & K \mbox{ for some context } K
\end{eqnarray*}

$u?(x)P | u!\langle Q \rangle \red P\{\quotep{Q}/x\}$

%We write $\wred$ for $\red^*$, and $P\red$ if $\exists Q $ such that $ P \red Q$.
We write $P\red$ if $\exists Q $ such that $ P \red Q$ and $P\not\red$, otherwise.

\section{Replication}

As mentioned before, it is known that replication (and hence
recursion) can be implemented in a higher-order process algebra
\cite{SangiorgiWalker}. As our first example of calculation with the
machinery thus far presented we give the construction explicitly in
the {\rhoc}.

\begin{eqnarray}
	D_{x} & := & \prefix{x}{y}{(\binpar{\outputp{x}{y}}{@{y}})} \nonumber\\
	\bangp_{x}{P} & := & \binpar{{x}!\langle{\binpar{D_{x}}{P}}\rangle}{D_{x}} \nonumber
\end{eqnarray}

\begin{eqnarray}
	\bangp_{x}{P} & & \nonumber\\
	=
	& {x}!\langle{(\prefix{x}{y}{(\outputp{x}{y} | @{y})) | P}}\rangle 
	      | \prefix{x}{y}{(\outputp{x}{y} | @{y})} & \nonumber\\
	\red
	& (\outputp{x}{y} | @{y})\substn{\quotep{(\prefix{x}{y}{(@{y} | \outputp{x}{y})) | P}}}{y} & \nonumber\\
	=
	& \outputp{x}{\quotep{(\prefix{x}{y}{(\outputp{x}{y} | @{y})) | P}}}
	  | {(\prefix{x}{y}{(\outputp{x}{y} | @{y})) | P}} & \nonumber\\
	\red
	& \ldots & \nonumber\\
	\red^*
	& P | P | \ldots & \nonumber
\end{eqnarray}

Of course, this encoding, as an implementation, runs away, unfolding
$\bangp{P}$ eagerly. A lazier and more implementable replication
operator, restricted to input-guarded processes, may be obtained as follows.

\begin{eqnarray}
\bangp{\prefix{u}{v}{P}} 
	:= 
	\binpar{\lift{x}{\prefix{u}{v}{(\binpar{D(x)}{P})}}}{D(x)} \nonumber
\end{eqnarray}

\begin{remark}
  Note that the lazier definition still does not deal with summation
  or mixed summation (i.e. sums over input and output). The reader is
  invited to construct definitions of replication that deal with these
  features. 

  Further, the definitions are parameterized in a name, $x$. Can you,
  gentle reader, make a definition that eliminates this parameter and
  guarantees no accidental interaction between the replication
  machinery and the process being replicated -- i.e. no accidental
  sharing of names used by the process to get its work done and the
  name(s) used by the replication to effect copying. This latter
  revision of the definition of replication is crucial to obtaining
  the expected identity $!!P \sim !P$.
\end{remark}

\begin{remark}\label{rem:paradoxical_combinator}
  The reader familiar with the lambda calculus will have noticed the
  similarity between $D$ and the paradoxical combinator.

  [Ed. note: the existence of this seems to suggest we have to be more
  restrictive on the set of processes and names we admit if we are to
  support no-cloning.]
\end{remark}

\subsubsection{Bisimulation}

The computational dynamics gives rise to another kind of equivalence,
the equivalence of computational behavior. As previously mentioned
this is typically captured \emph{via} some form of bisimulation.

% The notion we use in this paper is weak barbed bisimulation
% \cite{milner91polyadicpi}.

The notion we use in this paper is derived from weak barbed
bisimulation \cite{milner91polyadicpi}. 

\begin{definition}
An \emph{observation relation}, $\downarrow_{\mathcal N}$, over a set
of names, $\mathcal N$, is the smallest relation satisfying the rules
below.

\infrule[Out-barb]{y \in {\mathcal N}, \; x \nameeq y}
		  {\outputp{x}{v} \downarrow_{\mathcal N} x}
\infrule[Par-barb]{\mbox{$P\downarrow_{\mathcal N} x$ or $Q\downarrow_{\mathcal N} x$}}
		  {\binpar{P}{Q} \downarrow_{\mathcal N} x}

We write $P \Downarrow_{\mathcal N} x$ if there is $Q$ such that 
$P \wred Q$ and $Q \downarrow_{\mathcal N} x$.
\end{definition}

\begin{definition}
%\label{def.bbisim}
An  ${\mathcal N}$-\emph{barbed bisimulation} over a set of names, ${\mathcal N}$, is a symmetric binary relation 
${\mathcal S}_{\mathcal N}$ between agents such that $P\rel{S}_{\mathcal N}Q$ implies:
\begin{enumerate}
\item If $P \red P'$ then $Q \wred Q'$ and $P'\rel{S}_{\mathcal N} Q'$.
\item If $P\downarrow_{\mathcal N} x$, then $Q\Downarrow_{\mathcal N} x$.
\end{enumerate}
$P$ is ${\mathcal N}$-barbed bisimilar to $Q$, written
$P \wbbisim_{\mathcal N} Q$, if $P \rel{S}_{\mathcal N} Q$ for some ${\mathcal N}$-barbed bisimulation ${\mathcal S}_{\mathcal N}$.
\end{definition}

$\mathcal{R} \subseteq \pi \times \pi$

$P \mathcal{R} Q => \forall P'. P \red P' \Rightarrow \exists Q'. Q \red Q', P' \mathcal{R} Q'$

$P \vdash x \Rightarrow Q \vdash x$

\begin{mathpar}
  \inferrule*[lab=Out-barb]{x \nameeq y}{{y}!\langle{Q}\rangle \vdash x}
  \and
  \inferrule*[lab=Par-barb]{\mbox{$P\vdash x$ or $Q\vdash x$}}{\binpar{P}{Q} \vdash x}
\end{mathpar}

\subsubsection{Contexts}

One of the principle advantages of computational calculi like the
$\pi$-calculus is a well-defined notion of context,
contextual-equivalence and a correlation between
contextual-equivalence and notions of bisimulation. The notion of
context allows the decomposition of a process into (sub-)process and
its syntactic environment, its context. Thus, a context may be
thought of as a process with a ``hole'' (written $\Box$) in it. The
application of a context $M$ to a process $P$, written $M[P]$, is
tantamount to filling the hole in $M$ with $P$. In this paper we do
not need the full weight of this theory, but do make use of the notion
of context in the proof the main theorem. 

\begin{mathpar}
  \inferrule* [lab=summation] {} {{M_{M},M_{N}} \bc \Box \;|\; x.M_{A} \;|\; M_{M}+M_{N}}
  \and
  \inferrule* [lab=agent] {} {{M_{A}} \bc (\vec{x})M_{P} \;| \; \clift{P_0,\ldots,M_{P},\ldots,P_N}}
  \and \\
  \inferrule* [lab=process] {} {{M_{P}} \bc M_{N} \;| \;P|M_{P} }
\end{mathpar} 

\begin{mathpar}
  \inferrule* [lab=sychronization] {} {M_{N} \bc \Box \;|\; x?M_{F} \;|\; x!M_{C}}
  \and
  \inferrule* [lab=abstraction] {} {{M_{F}} \bc (x)M_{P} }
  \and
  \inferrule* [lab=concretion] {} {{M_{C}} \bc \langle M_{P} \rangle }
  \and \\
  \inferrule* [lab=process] {} {{M_{P}} \bc M_{N} \;| \;P|M_{P} }
\end{mathpar}

\begin{definition}[contextual application] Given a context $M$, and
  process $P$, we define the \emph{contextual application}, $M[P] :=
  M\{P/\Box\}$. That is, the contextual application of M to P is the
  substitution of $P$ for $\Box$ in $M$.
\end{definition}

$\meaningof{-} : L \to \mathcal{P}(\pi)$

\begin{mathpar}
  \inferrule* [lab=collection] {} {\meaningof{true} = \pi, \and \meaningof{~E} = \pi \setminus \meaningof{E}, \and \meaningof{E_{1} \& E_{2}} = \meaningof{E_{1}} \cap \meaningof{E_{2}}}
\end{mathpar}

\begin{mathpar}
  \inferrule* [lab=structure] {} {\meaningof{0} = \{ P \in \pi | P \equiv 0 \}, \and \\ \meaningof{E_1 | E_2} = \{ P \in \pi | P \equiv P_{1} | P_{2}, P_{1} \in \meaningof{E_{1}}, P_{2} \in \meaningof{E_2}\} }
\end{mathpar}

\begin{mathpar}
 \inferrule* [lab=behavior] {} {\meaningof{\langle a?b \rangle E} = \{ P \in \pi | P \equiv Q | u?(y)P', \\ \and \\\\ \and \\ \;\;\; u \in \meaningof{a}, \forall z.P'\{z/y\} \in \meaningof{E\{z/b\}}\}, \and \\ \meaningof{a!E} = \{ P \in \pi | P \equiv Q | x!\langle P' \rangle, x \in \meaningof{a} P' \in \meaningof{E}\} }
\end{mathpar}

\begin{mathpar}
 \inferrule* [lab=nominal] {} {\meaningof{\quotep{E}} = \{ \quotep{P} \in \quotep{\pi} | P \in \meaningof{E} \}, \and \meaningof{\quotep{P}} = \{ \quotep{Q} \in \quotep{\pi} | P \equiv Q \} \and \\ \meaningof{@\quotep{E}} = \{ P \in \pi | P \equiv @x, x \in \meaningof{E} \}}
\end{mathpar}

\begin{eqnarray*}
  \\
  \meaningof{-} : TS \to ST
\end{eqnarray*}

\begin{eqnarray*}
  \\
  L : TS \to ST
\end{eqnarray*}

\begin{eqnarray*}
  \\
  P \models E \iff P \in \meaningof{E}
\end{eqnarray*}

\begin{eqnarray*}
  P \approx_{L} Q \iff \forall E \in L. P \models E \iff Q \models E
\end{eqnarray*}

\begin{eqnarray*}
  P \approx_{K} Q
\end{eqnarray*}

\begin{eqnarray*}
  P \approx Q
\end{eqnarray*}

$\approx_{K} = \approx = \approx_{L}$

\subsubsection{Contextual duality}

Note that contexts extend the quotation operation to a family of
operations from processes to names. Given a context, $M$, we can
define a \emph{nominal context}, $\quotep{M}$ by $\quotep{M}[P] :=
\quotep{M[P]}$. To foreshadow what is to come we observe that these
operations enjoy a duality with processes very much like the duality
between vectors and maps from vectors to scalars.

Further, because the calculus is essentially higher-order, we have a
correspondence between contexts and processes. More specifically,
given a name $x$ and a context $M$ we can construct $M^{*}_{x}$ such
that 

\begin{mathpar}
  M^{*}_{x} | \lift{x}{P} \red M[P]
\end{mathpar}

namely,

\begin{mathpar}
  M^{*}_{x} := x?(u).M[\dropn{u}]
\end{mathpar}

The dependence of $M^{*}_{x}$ on a name makes it an abstraction, 

\begin{mathpar}
  M^{*} := (x)x?(u).M[\dropn{u}]
\end{mathpar}

\subsection{Additional notation}

It will sometimes be convenient to denote the process a name
quotes. We already have the notation $x = \quotep{P}$, but it will be
convenient to introduce an alternate notation, $\procn{x}$, when we
want to emphasize the connection to the use of the name. Note that, by
virtue of name equivalence, $\quotep{\procn{x}} \nameeq x$; so, the
notation is consistent with previous definitions.

Further, because names have structure it is possible to effect
substitutions on the basis of that structure. This means we need to
upgrade our notation for substitutions, which we accomplish by
adapting comprehension notation. Thus,

\begin{mathpar}
  P\{ y / x : x \in S \}
\end{mathpar}

is interpreted to mean the process derived from P by replacing (in a
capture-avoiding manner) each occurrence of $x$ in $S$ by $y$. For example,

\begin{mathpar}
  P\{ \quotep{\procn{x}|\procn{x}} / x : x \in \freenames{P} \}
\end{mathpar}

will replace each (occurrence) of a free name $x$ in $P$ by
$\quotep{\procn{x}|\procn{x}}$.

Also, we will avail ourselves of the notation $x^{L}$ and $x^{R}$ to
denote injections of a name into disjoint copies of the name
space. There are numerous ways to accomplish this. One example can be
found in \cite{MeredithR05}. This notation overloads to vectors of
names: $\vec{x}^{\pi} := (x_{i}^{\pi} \; : \; 0 \leq i < |\vec{x}| )$ where $\pi \in \{L,R\}$.

We also use $P^{\Box} := P|\Box$.

In \cite{MeredithR05} an interpretation of the new operator is
given. It turns out that there are several possible interpretations
all enjoying the requisite algebraic properties of the operator (see
\cite{milner91polyadicpi}). We will therefore make liberal use of
$(\nu\; \vec{x})P$.

% subsection the_syntax_and_semantics_of_the_notation_system (end)   

\input{qm2pi.qmops} 

\input{qm2pi.sterngerlach} 

\input{qm2pi.metric} 

% section concurrent_process_calculi (end)

%\input{qm2pi.proofsketch}

% section proof sketch (end)

%\input{qm2pi.slviaknots} 

% section spatial logic via knots (end)

\input{qm2pi.conclusion}

% section conclusion (end)

%\input{qm2pi.dtcodes} 

% section wiring algorithm (end)

\input{qm2pi.ack} 

% section acknowledgments (end)

\newpage


\bibliographystyle{plain}   
\bibliography{../../biblios/main.bib}

\input{qm2pi.rhodetails}

\end{document}

 

% section acknowledgments (end)

\newpage


\bibliographystyle{plain}   
\bibliography{../../biblios/main.bib}

\documentclass[12pt]{llncs}
%\documentclass{jktr}

\usepackage[pdftex]{hyperref}                   
\usepackage {listings}
\usepackage {mathpartir}
\usepackage{bcprules}
%\usepackage{listings}
                       
\usepackage{graphicx} 
%\usepackage[margins=2.5cm,nohead,nofoot]{geometry}
%\usepackage{geometry}
\usepackage{amsfonts}
\usepackage{amstext}
\usepackage{latexsym}
\usepackage{amssymb}
\usepackage{color}


%\include{myPreamble}
\include{qm2pi.local} 

%\ifpdf
%\usepackage[pdftex]{graphicx}
%\else
%\usepackage{graphicx}
%\fi

 % \ifpdf
%  \usepackage{pdfsync}
%  \if


%\title{Brief Article}
%\author{David F. Snyder}
%\author{L.G. Meredith}

%\address{Dept. of Math., Texas State University--San Marcos, San Marcos, TX 78666}
       
\pagestyle{empty}


\begin{document}

\lstset{language=[Objective]Caml,frame=shadowbox}

\input{qm2pi.front}

% section front matter (end)

\input{qm2pi.intro} 
 
% section introduction (end)

% \input{qm2pi.knotations} 

% section notation (end)

\input{qm2pi.process.calculi} 

% section concurrent_process_calculi_and_spatial_logics_ (end)
    
%\input{qm2pi.knots2pi} 

%\input{qm2pi.trefoil} 

%\input{qm2pi.mainthm} 

% subsection basic_interpretation (end)

%\input{qm2pi.rho.presentation} 
\subsection{The syntax and semantics of the notation system}\label{sub:the_syntax_and_semantics_of_the_notation_system} % (fold)

We now summarize a technical presentation of the calculus that
embodies our theory of dynamics. The typical presentation of such a
calculus follows the style of giving generators and relations on
them. The grammar, below, describing term constructors, freely
generates the set of processes, $\Proc$. This set is then quotiented
by a relation known as structural congruence and it is over this set
that the notion of dynamics is expressed. This presentation is
essentially that of \cite{MeredithR05} with the addition of
polyadicity and summation. For readability we have relegated some of
the technical subtleties to an appendix.

\subsubsection{Process grammar}\label{subsub:process_grammar}

\begin{mathpar}
  \inferrule* [lab=synchronization] {} {{M} \bc \pzero \;|\; x?F \;|\; x!C }
  \and
  \inferrule* [lab=abstraction] {} {{F} \bc (x)P}
  \and
  \inferrule* [lab=concretion] {} {{C} \bc \langle Q \rangle}
  \and
  \inferrule* [lab=process] {} {{P,Q} \bc M \;| \;P|Q \;|\; @{x}}
  \and
  \inferrule* [lab=name] {} {{x} \bc \quotep{P}}
\end{mathpar} 

Note that $\vec{x}$ (resp. $\vec{P}$) denotes a vector of names
(resp. processes) of length $|\vec{x}|$ (resp. $|\vec{P}|$). We adopt
the following useful abbreviations.

\begin{mathpar}
   x?(\vec{y}).P := x.(\vec{y})P \and  x\clift{\vec{P}} := x.\clift{\vec{P}}
   \and x!(y) := \lift{x}{\dropn{y}}
   \and \Pi_{i=0}^{n-1}P_i := P_0 | \ldots | P_{n-1}
\end{mathpar}

\subsubsection{Structural congruence}

\paragraph{Free and bound names and alpha-equivalence.} At the
core of structural equivalence is alpha-equivalence which identifies
process that are the same up to a change of variable. Formally, we
recognize the distinction between free and bound names. The free names
of a process, $\freenames{P}$, may be calculated recursively as
follows:

\begin{mathpar}
\freenames{\pzero} := \emptyset
  \and \\
  \freenames{x?(y).P} := \{ x \} \cup (\freenames{P} \setminus \{ y \})
  \and 
  \freenames{x!\langle P \rangle} := \{ x \} \cup \{ P \} 
  \and \\
  \freenames{P|Q} := \freenames{P} \cup \freenames{Q}
  \and \\
  \freenames{@{x}} := \{ x \}
\end{mathpar}

$\pi$
$\quotep{\pi}$

$\freenames{-} : \pi \to \mathcal{P}(\quotep{\pi})$

\begin{eqnarray*}
  \freenames{\pzero} & := & \emptyset \\
  \freenames{x?(y).P} & := & \{ x \} \cup (\freenames{P} \setminus \{ y \}) \\
  \freenames{x!\langle P \rangle} & := & \{ x \} \cup \{ P \} \\
  \freenames{P|Q} & := & \freenames{P} \cup \freenames{Q} \\
  \freenames{\dropn{x}} & := & \{ x \}
\end{eqnarray*}

The bound names of a process, $\boundnames{P}$, are those names occurring in $P$
that are not free. For example, in $x?(y).0$, the name $x$ is free, while $y$ is bound.

\begin{mathpar}
  \inferrule* [lab=monoidal-laws] {} { P|Q \equiv Q|P \and P|0 \equiv P \and P|(Q|R) \equiv (P|Q)|R }
\end{mathpar}

\begin{mathpar}
  \inferrule* [lab=alpha-equivalence] {} { (x)P \equiv (y)P\{y/x\} \and y \not\in \freenames{P} }
\end{mathpar}

\begin{definition}
Then two processes, $P,Q$, are alpha-equivalent if $P = Q\{\vec{y}/\vec{x}\}$ for
some $\vec{x} \in \boundnames{Q},\vec{y} \in \boundnames{P}$, where $Q\{\vec{y}/\vec{x}\}$
denotes the capture-avoiding substitution of $\vec{y}$ for $\vec{x}$ in $Q$.
\end{definition}

\begin{definition}
  The {\em structural congruence} \cite{SangiorgiWalker} , $\equiv$,
  between processes is the least congruence containing
  alpha-equivalence, satisfying the abelian monoid laws
  (associativity, commutativity and $\pzero$ as identity) for parallel
  composition $|$ and for summation $+$.
\end{definition}

\subsection{Name equivalence}

We take name equivalence, written $\nameeq$, to be the smallest
equivalence relation generated by the following rules.

\begin{mathpar}
\inferrule*[lab=Quote-drop]
{ }
{ \quotep{@{x}} \nameeq x }

\inferrule*[lab=Struct-equiv]
{ P \scong Q }
{ \quotep{P} \nameeq \quotep{Q} }
\end{mathpar}

The astute reader will have noticed that the mutual recursion of names
and processes imposes a mutual recursion on alpha-equivalence and
structural equivalence via name-equivalence. Fortunately, all of this
works out pleasantly and we may calculate in the natural way, free of
concern. The reader interested in the details is referred to the
appendix \ref{appendix:rho_details}.

\subsection{Substitution}

We use $\Proc$ for the set of processes, $\QProc$ for the set of
names, and $\id{\{}\vec{y} / \vec{x} \id{\}}$ to denote partial maps,
$s : \QProc \rightarrow \QProc$. A map, $s$ lifts, uniquely, to a map
on process terms, $\widehat{s} : \Proc \rightarrow \Proc$ by the
following equations.

\begin{mathpar}
  (0) \psubstp{Q}{P} := 0 \\
  (R \juxtap S) \psubstp{Q}{P}
  :=    
  (R)\psubstp{Q}{P} \juxtap (S) \psubstp{Q}{P} \\
  (x?(y).R) \psubstp{Q}{P}    
  :=    
  (x)\substp{Q}{P} (z)\concat( (R \psubstn{z}{y}) \psubstp{Q}{P} ) \\
  (\lift{x}{R}) \psubstp{Q}{P}  
  :=
  \lift{(x)\substp{Q}{P}}{ R \psubstp{Q}{P} } \\
%   (\dropn{x})  \psubstp{Q}{P}       
%   := 
%   \left\{ 
%     \begin{array}{ccc} 
%       \dropn{\quotep{Q}} & & x \nameeq \quotep{P} \\
%       \dropn{x} & & otherwise \\
%     \end{array}
%   \right. 
  (\dropn{x})  \psubstp{Q}{P}       
  := 
  \left\{ 
    \begin{array}{ccc} 
      Q & & x \nameeq \quotep{P} \\
      \dropn{x} & & otherwise \\
    \end{array}
  \right.
\end{mathpar}
 

where

\begin{eqnarray}
  (x)\id{\{} \lpquote Q \rpquote / \lpquote P \rpquote \id{\}}            = 
  \left\{ 
    \begin{array}{ccc}
      \lpquote Q \rpquote & & x \nameeq \lpquote P \rpquote \\
      x & & otherwise \\
    \end{array}
  \right. \nonumber
\end{eqnarray}

and $z$ is chosen distinct from $\quotep{P}$, $\quotep{Q}$, the free
names in $Q$, and all the names in $R$. Our $\alpha$-equivalence will
be built in the standard way from this substitution.

\begin{remark}\label{rem:no_self_referential_names}
  One consequence of these definitions is that $\forall P. \quotep{P}
  \not\in \freenames{P}$.
\end{remark}

\subsection{ Dynamic quote: an example }

Anticipating something of what's to come, consider applying the
substitution, $\widehat{\id{\{}u / z \id{\}}}$, to the following pair
of processes, $\lift{w}{y!(z)}$ and $w[ \lpquote y!(z) \rpquote ]$.

\begin{eqnarray}
	\lift{w}{y!(z)}\widehat{\id{\{}u / z \id{\}}}
		& = &
		\lift{w}{y!(u)} \nonumber\\
	w[ \lpquote y!(z) \rpquote ] \widehat{ \id{\{}u / z \id{\}} }
		& = &
		w[ \lpquote y!(z) \rpquote ] \nonumber
\end{eqnarray}

Because the body of the process between quotes is impervious to
substitution, we get radically different answers. In fact, by
examining the first process in an input context,
e.g. $x?(z).\lift{w}{y!(z)}$, we see that the process under the lift
operator may be shaped by prefixed inputs binding a name inside it. In
this sense, the lift operator will be seen as a way to dynamically
construct processes before reifying them as names.

Finally equipped with these standard features we can present the
dynamics of the calculus.

\subsubsection{Operational semantics} 

Finally, we introduce the computational dynamics. What marks these
algebras as distinct from other more traditionally studied algebraic
structures, e.g. vector spaces or polynomial rings, is the manner in
which dynamics is captured. In traditional structures, dynamics is typically
expressed through morphisms between such structures, as in linear maps
between vector spaces or morphisms between rings. In algebras
associated with the semantics of computation, the dynamics is
expressed as part of the algebraic structure itself, through a
reduction reduction relation typically denoted by $\red$. Below, we
give a recursive presentation of this relation for the calculus used
in the encoding.

$\red \subseteq \pi \times \pi$
$\red : \pi \to \mathcal{P}(\pi)$

\begin{mathpar}
  \inferrule* [lab=Comm] { \textsf{match}( x_{src}, x_{trgt} ) } { x_{trgt}?(y)P \; | \; x_{src}!\langle {Q} \rangle \red P\{\quotep{Q}/y}\} }
  \and \\
  \inferrule* [lab=Par] {{P} \red {P}'} {{{P} | {Q}} \red {{P}' | {Q}}}
  \and
  \inferrule* [lab=Equiv]{{{P} \scong {P}'} \andalso {{P}' \red {Q}'} \andalso {{Q}' \scong {Q}}}{{P} \red {Q}}
\end{mathpar}

\begin{eqnarray*}
  match_{\equiv} (\quotep{P},\quotep{Q}) & := & P \equiv Q \\
  match_{\dagger}(\quotep{P},\quotep{Q}) & := & \forall R. P|Q \red^{*} R => R \red^{*} 0 \\
  match_{K}(\quotep{P},\quotep{Q}) & := & K \mbox{ for some context } K
\end{eqnarray*}

$u?(x)P | u!\langle Q \rangle \red P\{\quotep{Q}/x\}$

%We write $\wred$ for $\red^*$, and $P\red$ if $\exists Q $ such that $ P \red Q$.
We write $P\red$ if $\exists Q $ such that $ P \red Q$ and $P\not\red$, otherwise.

\section{Replication}

As mentioned before, it is known that replication (and hence
recursion) can be implemented in a higher-order process algebra
\cite{SangiorgiWalker}. As our first example of calculation with the
machinery thus far presented we give the construction explicitly in
the {\rhoc}.

\begin{eqnarray}
	D_{x} & := & \prefix{x}{y}{(\binpar{\outputp{x}{y}}{@{y}})} \nonumber\\
	\bangp_{x}{P} & := & \binpar{{x}!\langle{\binpar{D_{x}}{P}}\rangle}{D_{x}} \nonumber
\end{eqnarray}

\begin{eqnarray}
	\bangp_{x}{P} & & \nonumber\\
	=
	& {x}!\langle{(\prefix{x}{y}{(\outputp{x}{y} | @{y})) | P}}\rangle 
	      | \prefix{x}{y}{(\outputp{x}{y} | @{y})} & \nonumber\\
	\red
	& (\outputp{x}{y} | @{y})\substn{\quotep{(\prefix{x}{y}{(@{y} | \outputp{x}{y})) | P}}}{y} & \nonumber\\
	=
	& \outputp{x}{\quotep{(\prefix{x}{y}{(\outputp{x}{y} | @{y})) | P}}}
	  | {(\prefix{x}{y}{(\outputp{x}{y} | @{y})) | P}} & \nonumber\\
	\red
	& \ldots & \nonumber\\
	\red^*
	& P | P | \ldots & \nonumber
\end{eqnarray}

Of course, this encoding, as an implementation, runs away, unfolding
$\bangp{P}$ eagerly. A lazier and more implementable replication
operator, restricted to input-guarded processes, may be obtained as follows.

\begin{eqnarray}
\bangp{\prefix{u}{v}{P}} 
	:= 
	\binpar{\lift{x}{\prefix{u}{v}{(\binpar{D(x)}{P})}}}{D(x)} \nonumber
\end{eqnarray}

\begin{remark}
  Note that the lazier definition still does not deal with summation
  or mixed summation (i.e. sums over input and output). The reader is
  invited to construct definitions of replication that deal with these
  features. 

  Further, the definitions are parameterized in a name, $x$. Can you,
  gentle reader, make a definition that eliminates this parameter and
  guarantees no accidental interaction between the replication
  machinery and the process being replicated -- i.e. no accidental
  sharing of names used by the process to get its work done and the
  name(s) used by the replication to effect copying. This latter
  revision of the definition of replication is crucial to obtaining
  the expected identity $!!P \sim !P$.
\end{remark}

\begin{remark}\label{rem:paradoxical_combinator}
  The reader familiar with the lambda calculus will have noticed the
  similarity between $D$ and the paradoxical combinator.

  [Ed. note: the existence of this seems to suggest we have to be more
  restrictive on the set of processes and names we admit if we are to
  support no-cloning.]
\end{remark}

\subsubsection{Bisimulation}

The computational dynamics gives rise to another kind of equivalence,
the equivalence of computational behavior. As previously mentioned
this is typically captured \emph{via} some form of bisimulation.

% The notion we use in this paper is weak barbed bisimulation
% \cite{milner91polyadicpi}.

The notion we use in this paper is derived from weak barbed
bisimulation \cite{milner91polyadicpi}. 

\begin{definition}
An \emph{observation relation}, $\downarrow_{\mathcal N}$, over a set
of names, $\mathcal N$, is the smallest relation satisfying the rules
below.

\infrule[Out-barb]{y \in {\mathcal N}, \; x \nameeq y}
		  {\outputp{x}{v} \downarrow_{\mathcal N} x}
\infrule[Par-barb]{\mbox{$P\downarrow_{\mathcal N} x$ or $Q\downarrow_{\mathcal N} x$}}
		  {\binpar{P}{Q} \downarrow_{\mathcal N} x}

We write $P \Downarrow_{\mathcal N} x$ if there is $Q$ such that 
$P \wred Q$ and $Q \downarrow_{\mathcal N} x$.
\end{definition}

\begin{definition}
%\label{def.bbisim}
An  ${\mathcal N}$-\emph{barbed bisimulation} over a set of names, ${\mathcal N}$, is a symmetric binary relation 
${\mathcal S}_{\mathcal N}$ between agents such that $P\rel{S}_{\mathcal N}Q$ implies:
\begin{enumerate}
\item If $P \red P'$ then $Q \wred Q'$ and $P'\rel{S}_{\mathcal N} Q'$.
\item If $P\downarrow_{\mathcal N} x$, then $Q\Downarrow_{\mathcal N} x$.
\end{enumerate}
$P$ is ${\mathcal N}$-barbed bisimilar to $Q$, written
$P \wbbisim_{\mathcal N} Q$, if $P \rel{S}_{\mathcal N} Q$ for some ${\mathcal N}$-barbed bisimulation ${\mathcal S}_{\mathcal N}$.
\end{definition}

$\mathcal{R} \subseteq \pi \times \pi$

$P \mathcal{R} Q => \forall P'. P \red P' \Rightarrow \exists Q'. Q \red Q', P' \mathcal{R} Q'$

$P \vdash x \Rightarrow Q \vdash x$

\begin{mathpar}
  \inferrule*[lab=Out-barb]{x \nameeq y}{{y}!\langle{Q}\rangle \vdash x}
  \and
  \inferrule*[lab=Par-barb]{\mbox{$P\vdash x$ or $Q\vdash x$}}{\binpar{P}{Q} \vdash x}
\end{mathpar}

\subsubsection{Contexts}

One of the principle advantages of computational calculi like the
$\pi$-calculus is a well-defined notion of context,
contextual-equivalence and a correlation between
contextual-equivalence and notions of bisimulation. The notion of
context allows the decomposition of a process into (sub-)process and
its syntactic environment, its context. Thus, a context may be
thought of as a process with a ``hole'' (written $\Box$) in it. The
application of a context $M$ to a process $P$, written $M[P]$, is
tantamount to filling the hole in $M$ with $P$. In this paper we do
not need the full weight of this theory, but do make use of the notion
of context in the proof the main theorem. 

\begin{mathpar}
  \inferrule* [lab=summation] {} {{M_{M},M_{N}} \bc \Box \;|\; x.M_{A} \;|\; M_{M}+M_{N}}
  \and
  \inferrule* [lab=agent] {} {{M_{A}} \bc (\vec{x})M_{P} \;| \; \clift{P_0,\ldots,M_{P},\ldots,P_N}}
  \and \\
  \inferrule* [lab=process] {} {{M_{P}} \bc M_{N} \;| \;P|M_{P} }
\end{mathpar} 

\begin{mathpar}
  \inferrule* [lab=sychronization] {} {M_{N} \bc \Box \;|\; x?M_{F} \;|\; x!M_{C}}
  \and
  \inferrule* [lab=abstraction] {} {{M_{F}} \bc (x)M_{P} }
  \and
  \inferrule* [lab=concretion] {} {{M_{C}} \bc \langle M_{P} \rangle }
  \and \\
  \inferrule* [lab=process] {} {{M_{P}} \bc M_{N} \;| \;P|M_{P} }
\end{mathpar}

\begin{definition}[contextual application] Given a context $M$, and
  process $P$, we define the \emph{contextual application}, $M[P] :=
  M\{P/\Box\}$. That is, the contextual application of M to P is the
  substitution of $P$ for $\Box$ in $M$.
\end{definition}

$\meaningof{-} : L \to \mathcal{P}(\pi)$

\begin{mathpar}
  \inferrule* [lab=collection] {} {\meaningof{true} = \pi, \and \meaningof{~E} = \pi \setminus \meaningof{E}, \and \meaningof{E_{1} \& E_{2}} = \meaningof{E_{1}} \cap \meaningof{E_{2}}}
\end{mathpar}

\begin{mathpar}
  \inferrule* [lab=structure] {} {\meaningof{0} = \{ P \in \pi | P \equiv 0 \}, \and \\ \meaningof{E_1 | E_2} = \{ P \in \pi | P \equiv P_{1} | P_{2}, P_{1} \in \meaningof{E_{1}}, P_{2} \in \meaningof{E_2}\} }
\end{mathpar}

\begin{mathpar}
 \inferrule* [lab=behavior] {} {\meaningof{\langle a?b \rangle E} = \{ P \in \pi | P \equiv Q | u?(y)P', \\ \and \\\\ \and \\ \;\;\; u \in \meaningof{a}, \forall z.P'\{z/y\} \in \meaningof{E\{z/b\}}\}, \and \\ \meaningof{a!E} = \{ P \in \pi | P \equiv Q | x!\langle P' \rangle, x \in \meaningof{a} P' \in \meaningof{E}\} }
\end{mathpar}

\begin{mathpar}
 \inferrule* [lab=nominal] {} {\meaningof{\quotep{E}} = \{ \quotep{P} \in \quotep{\pi} | P \in \meaningof{E} \}, \and \meaningof{\quotep{P}} = \{ \quotep{Q} \in \quotep{\pi} | P \equiv Q \} \and \\ \meaningof{@\quotep{E}} = \{ P \in \pi | P \equiv @x, x \in \meaningof{E} \}}
\end{mathpar}

\begin{eqnarray*}
  \\
  \meaningof{-} : TS \to ST
\end{eqnarray*}

\begin{eqnarray*}
  \\
  L : TS \to ST
\end{eqnarray*}

\begin{eqnarray*}
  \\
  P \models E \iff P \in \meaningof{E}
\end{eqnarray*}

\begin{eqnarray*}
  P \approx_{L} Q \iff \forall E \in L. P \models E \iff Q \models E
\end{eqnarray*}

\begin{eqnarray*}
  P \approx_{K} Q
\end{eqnarray*}

\begin{eqnarray*}
  P \approx Q
\end{eqnarray*}

$\approx_{K} = \approx = \approx_{L}$

\subsubsection{Contextual duality}

Note that contexts extend the quotation operation to a family of
operations from processes to names. Given a context, $M$, we can
define a \emph{nominal context}, $\quotep{M}$ by $\quotep{M}[P] :=
\quotep{M[P]}$. To foreshadow what is to come we observe that these
operations enjoy a duality with processes very much like the duality
between vectors and maps from vectors to scalars.

Further, because the calculus is essentially higher-order, we have a
correspondence between contexts and processes. More specifically,
given a name $x$ and a context $M$ we can construct $M^{*}_{x}$ such
that 

\begin{mathpar}
  M^{*}_{x} | \lift{x}{P} \red M[P]
\end{mathpar}

namely,

\begin{mathpar}
  M^{*}_{x} := x?(u).M[\dropn{u}]
\end{mathpar}

The dependence of $M^{*}_{x}$ on a name makes it an abstraction, 

\begin{mathpar}
  M^{*} := (x)x?(u).M[\dropn{u}]
\end{mathpar}

\subsection{Additional notation}

It will sometimes be convenient to denote the process a name
quotes. We already have the notation $x = \quotep{P}$, but it will be
convenient to introduce an alternate notation, $\procn{x}$, when we
want to emphasize the connection to the use of the name. Note that, by
virtue of name equivalence, $\quotep{\procn{x}} \nameeq x$; so, the
notation is consistent with previous definitions.

Further, because names have structure it is possible to effect
substitutions on the basis of that structure. This means we need to
upgrade our notation for substitutions, which we accomplish by
adapting comprehension notation. Thus,

\begin{mathpar}
  P\{ y / x : x \in S \}
\end{mathpar}

is interpreted to mean the process derived from P by replacing (in a
capture-avoiding manner) each occurrence of $x$ in $S$ by $y$. For example,

\begin{mathpar}
  P\{ \quotep{\procn{x}|\procn{x}} / x : x \in \freenames{P} \}
\end{mathpar}

will replace each (occurrence) of a free name $x$ in $P$ by
$\quotep{\procn{x}|\procn{x}}$.

Also, we will avail ourselves of the notation $x^{L}$ and $x^{R}$ to
denote injections of a name into disjoint copies of the name
space. There are numerous ways to accomplish this. One example can be
found in \cite{MeredithR05}. This notation overloads to vectors of
names: $\vec{x}^{\pi} := (x_{i}^{\pi} \; : \; 0 \leq i < |\vec{x}| )$ where $\pi \in \{L,R\}$.

We also use $P^{\Box} := P|\Box$.

In \cite{MeredithR05} an interpretation of the new operator is
given. It turns out that there are several possible interpretations
all enjoying the requisite algebraic properties of the operator (see
\cite{milner91polyadicpi}). We will therefore make liberal use of
$(\nu\; \vec{x})P$.

% subsection the_syntax_and_semantics_of_the_notation_system (end)   

\input{qm2pi.qmops} 

\input{qm2pi.sterngerlach} 

\input{qm2pi.metric} 

% section concurrent_process_calculi (end)

%\input{qm2pi.proofsketch}

% section proof sketch (end)

%\input{qm2pi.slviaknots} 

% section spatial logic via knots (end)

\input{qm2pi.conclusion}

% section conclusion (end)

%\input{qm2pi.dtcodes} 

% section wiring algorithm (end)

\input{qm2pi.ack} 

% section acknowledgments (end)

\newpage


\bibliographystyle{plain}   
\bibliography{../../biblios/main.bib}

\input{qm2pi.rhodetails}

\end{document}



\end{document}



% section proof sketch (end)

%\section{Unlikely characters: spatial logic for
  knots}\label{sub:characteristic_formulae} % (fold)

Associated to the mobile process calculi are a family of logics known
as the Hennessy-Milner logics. These logics typically enjoy a
semantics interpreting formulae as sets of processes that when
factored through the encoding outlined above allows an identification
of classes of knots with logical formulae. In the context of this
encoding the sub-family known as the spatial logics \cite{CairesC03}
\cite{CairesC04} \cite{Caires04} are of particular interest providing
several important features for expressing and reasoning about
properties (i.e. classes) of knots. We hint here at how this may be done.

%\begin{description}
%\item [structural connectives] 
\subsubsection{Structural connectives} The spatial logics enjoy
structural connectives corresponding, at the logical level, to the
parallel composition ($P | Q$) and new name ($(\nu \; x)P$)
connectives for processes. As illustrated in the examples below, these
connectives are extremely expressive given the shape of our encoding.
%\item [decideable satisfaction]

\subsubsection{Decideable satisfaction}
In \cite{Caires04} the satisfaction relation is shown to be decideable
for a rich class of processes. It further turns out that the image of
the our encoding is a proper subset of that class. This result
provides the basis for an algorithm by which to search for knots
enjoying a given property.
%\item [characteristic formulae]

\subsubsection{Characteristic formulae}
In the same paper \cite{Caires04} , Caires presents a means of calculating
characteristic formulae, selecting equivalence classes of processes
up to a pre--specified depth limit on the support set of names. Composed with our
encoding, this characteristic formula can be used to select
characteristic formulae for knots.
%\end{description}

\subsubsection{Spatial logic formulae}

The grammar below (segmented for comprehension) summarizes the syntax
of spatial logic formulae. We employ illustrative examples in the
sequel to provide an intuitive understanding of their meaning
referring the reader to \cite{Caires04} for a more detailed explication
of the semantics.

\begin{mathpar}
  \inferrule* [lab=boolean] {} {{A,B} \bc T \;|\; \neg A \;|\; A \wedge B \;|\; \eta = \eta'}
  \and
  \inferrule* [lab=spatial] {} {|\; \pzero \;|\; A | B \;|\; x \text{\textregistered} A \;|\; \forall x . A \;|\;  H x . A}
  \and
  \inferrule* [lab=behavioral] {} {|\; \alpha . A}
  \and 
  \inferrule* [lab=recursion] {} {|\; X(\vec{u}) \;|\; \mu X(\vec{u}) . A}
  \and
  \inferrule* [lab=action] {} {\alpha \bc \langle x?(\vec{y}) \rangle \;|\; \langle x!(\vec{y}) \rangle \;|\; \langle \tau \rangle}
  \and 
  \inferrule* [lab=name] {} {\eta \bc x \;|\; \tau}
\end{mathpar} 

% subsection characteristic_formulae (end)   	 

\subsection{Example formulae}\label{sub:example_formulae_} % (fold)

\subsubsection{Crossing as formula.}
% 
% \begin{align*}
%   \frac{d}{dx} \sin x &= \cos x 
%   & \frac{d}{dx} e^x &= e^x \\
%   \frac{d}{dx} \cos x &= - \sin x 
%   & \frac{d}{dx} \log x &= \frac{1}{x} \\
% \end{align*} 

\begin{align*}
 \mu C(x_{0},x_{1},y_{0},y_{1},u).&(\langle x_{0}?(z) \rangle(\langle u! \rangle\langle y_{1}!z \rangle C(x_{0},x_{1},y_{0},y_{1},u)) & \\
  & \wedge \langle y_{1}?(z) \rangle (\langle u! \rangle \langle x_{0}!z \rangle C(x_{0},x_{1},y_{0},y_{1},u)) & \\
  & \wedge \langle x_{1}?(z) \rangle (\langle u? \rangle \langle y_{0}!z \rangle C(x_{0},x_{1},y_{0},y_{1},u)) & \\
  & \wedge \langle y_{0}?(z) \rangle (\langle u? \rangle \langle x_{1}!z \rangle C(x_{0},x_{1},y_{0},y_{1},u))) &
\end{align*}

The lexicographical similarity between the shape of this formulae and
the shape of definition of the process representing a crossing reveals
the intuitive meaning of this formulae. It describes the capabilities
of a process that has the right to represent a crossing. For example
it picks out processes that may perform an input on the port $x_0$ in
its initial menu of capabilities. What differentiates the formula
from the process, however, is that the crossing process is the
smallest candidate to satisfy the formula. Infinitely many other
processes -- with internal behavior hidden behind this interface, so
to speak -- also satisfy this formula. Even this simple formula,
then, can be seen to open a new view onto knots, providing a
computational interpretation of \emph{virtual} knots.

Note that this formula is derived by hand. A similar formula can be
derived by employing Caires' calculation of characteristic formula
\cite{Caires04} to the process representing a crossing. In light of
this discussion, we let
$\meaningof{C}_{\phi}(x0,x1,y0,y1,u)$ denote a formula specifying the
dynamics we wish to capture of a crossing. To guarantee we preserve
the shape of the interface and minimal semantics we demand that
$\meaningof{C}_{\phi}(x0,x1,y0,y1,u) \Rightarrow
\textbf{C}(x0,x1,y0,y1,u)$ where $\textbf{C}(x0,x1,y0,y1,u)$ denotes
the formula above.
                            
\subsubsection{Crossing number constraints.}
The moral content of the context lemma (Lemma \ref{context}) is that the notion of
``locality'' in the Reidemeister moves is effectively captured by the
parallel composition operator of the process calculus. This intuition
extends through the logic. Given a formula,
$\meaningof{C}_{\phi}(x0,x1,y0,y1,u)$, we can use the structural
connectives to specify constraints on crossing numbers, such as at
least $n$ crossings, or exactly $n$ crossings.
\begin{mathpar}
  \inferrule* [lab=at-least-n] {} { K^{\geq n}_{\phi}(\vec{xs},\vec{ys}) := \Pi_{i=0}^{n-1} Hu . \meaningof{C}_{\phi}(xs_i,ys_i,u) | T }
  \and 
  \inferrule* [lab=exactly-n] {} { K^{= n}_{\phi}(\vec{xs},\vec{ys}) := \Pi_{i=0}^{n-1} Hu . \meaningof{C}_{\phi}(xs_i,ys_i,u) | \neg (\forall x_0,y_0,x_1,y_1,u . \meaningof{C}_{\phi}(x_0,y_0,x_1,y_1,u) | T) }
\end{mathpar}

To round out this section, recall that the encoding of an $n$-crossing
knot decomposes into a parallel composition of $n$ \emph{copies} of a
crossing process together with a wiring harness. To specify different
knot classes with the same crossing number amounts to specifying
logical constraints on the wiring harness. In the interest of space,
we defer examples to a forthcoming paper. Suffice it to say that both
the conditions ``alternating knot'' and ``contains the tangle
corresponding to 5/3'' are expressible. For example, it is possible to
calculate the characteristic formula of a process corresponding to the
tangle 5/3 and conjoin it into the classifying formula via the
composition connective of the logic.

Finally, we wish to observe that it is entirely within reason to
contemplate a more domain-specific version of spatial logic tailored
to the shape of processes in the image of the encoding. Such a
domain-specific logic would have a better claim to the title formal
language of knot properties.

% subsection example_formulae_ (end)

% section knots_as_processes (end) 

% section spatial logic via knots (end)

\section{Conclusions and future work}

\paragraph{Testing physical space}
You, gentle reader, may wonder why of all the theorems to be proved
given this set up we pick the one above. In some sense it's hardly
central to quantum mechanics. We see it as central in the sense that
it firmly establishes a notion of physical space arising from a notion
of the equivalence of behavior. Relating bisimulation to a metric is a
big step forward, but one is faced with interpreting the relationship
of that metric space to something more physical. Quantum mechanical
notions of ``physical'' space are still far from intuitive, but by
relating this idea of distance as testing to calculations that predict
physical circumstances we are making a not insignificant step forward
toward an understanding of the physical space we inhabit as
essentially dynamic.

\paragraph{Effectivity and simulation}
One of the observations we have yet to make is that the entire program
spelled out here is effective. We have built various interpreters for
the reflective calculus at work in this interpretation. In principle,
then, we can simulate quantum mechanics on a computer. The place where
the simulation may lose fidelity is the infinitely branching summation
for the annihilator.

In this connection i also want to point out that the evaluation style
calculation of the inner product puts the non-determinism of the
summation right at the heart of measurement. This suggests that
Milner's original reduction-based formulation of the dynamics of his
calculi in terms of sums was not just notationally suggestive of a
notion of measure-and-continue but captured some significant part of
the physics.

\paragraph{Quantum continuations}
In light of this last observation i want to point out that the
predominant account of quantum mechanics is missing a key aspect of a
truly compositional story of the physical situation. In a real lab,
when a measurement is made the observation can be made to feed into
another device that then makes another measurement conditioned on the
results of the first. This means that after the superposition was
collapsed the entire experimental set up remained in
superposition. While QM offers a means of writing this down it doesn't
quite line up well with the well-trodden formulation of computation
and continuation that we see so succinctly expressed in Milner's
calculi. This suggests that there might be advantages to this account
of dynamics waiting to be explored.

\paragraph{Quantum logic}
In this connection, we also note that by virtue of having the
Hennessy-Milner construction, we can pull the construction through the
interpretation of QM. This gives us a natural candidate for a quantum
logic that enjoys an extremely tight connection with it's domain of
interpretation, making the construction much less ad hoc (rather it is
the image of functor!).

\paragraph{Quantum probabiity}
i have questions about the basis of the interpretation of inner
product as probability amplitude. In particular, using which
axiomatization of probability theory does the notion of probability
amplitude earn the right to be so dubbed? In other words, where is the
proof that the operation for calculating a probability amplitude (and
then squaring) satisfies the axioms of what it means to calculate a
probability? Even if such a proof exists (i have yet to find it in the
literature), i wonder if it might not be possible to turn things on
their heads. Can we view the calculation of the probability amplitude
as an axiomatization of probability? If so, then the definition we
give for calculating probability amplitude may provide the basis for
an \emph{effective} theory of probability.

\paragraph{Quantum vs ``biological'' information}
Finally, i want to conclude with a more philosophical observation. At
a recent workshop in which QM was a predominant topic i noticed
something about quantum information. The speaker was giving a riveting
discussion of axiomatic QM and showing how properties of ``no
cloning'' and ``no deleting'' emerged as consequences of the
axiomatization. Theorems of this form are necessary to give us a sense
of confidence that our axioms characterize the physical theory. What
struck me, though, was that if quantum information is neither erasable
nor replicable it is markedly different from \emph{life}. Two of the
things we know about life is that

\begin{itemize}
  \item it ends;
  \item to gain some measure of persistence, to transcend it's
    finitude it is imminently copyable.
\end{itemize}

Both of these qualities are summarized succinctly in the aphorism: all
flesh is grass. For me these two kinds of ``information'' -- call them
quantum and biological -- are end points on a spectrum of strategies
for persistence. At one end, we have those curious entities that enjoy
uniqueness and permanence; at the other, we have those who in the face
of a certain end and an uncertain present make a go of passing
something on. To me one of the more remarkable aspects of the latter
strategy is that in the presence of noise (and certain features of
copying) we get a kind of dynamism, a chance for improvement against a
given persistent condition.

% subsection other_calculi_other_bisimulations_and_geometry_as_behavior (end)




% section conclusion (end)

%\documentclass[12pt]{llncs}
%\documentclass{jktr}

\usepackage[pdftex]{hyperref}                   
\usepackage {listings}
\usepackage {mathpartir}
\usepackage{bcprules}
%\usepackage{listings}
                       
\usepackage{graphicx} 
%\usepackage[margins=2.5cm,nohead,nofoot]{geometry}
%\usepackage{geometry}
\usepackage{amsfonts}
\usepackage{amstext}
\usepackage{latexsym}
\usepackage{amssymb}
\usepackage{color}


%\include{myPreamble}
\documentclass[12pt]{llncs}
%\documentclass{jktr}

\usepackage[pdftex]{hyperref}                   
\usepackage {listings}
\usepackage {mathpartir}
\usepackage{bcprules}
%\usepackage{listings}
                       
\usepackage{graphicx} 
%\usepackage[margins=2.5cm,nohead,nofoot]{geometry}
%\usepackage{geometry}
\usepackage{amsfonts}
\usepackage{amstext}
\usepackage{latexsym}
\usepackage{amssymb}
\usepackage{color}


%\include{myPreamble}
\include{qm2pi.local} 

%\ifpdf
%\usepackage[pdftex]{graphicx}
%\else
%\usepackage{graphicx}
%\fi

 % \ifpdf
%  \usepackage{pdfsync}
%  \if


%\title{Brief Article}
%\author{David F. Snyder}
%\author{L.G. Meredith}

%\address{Dept. of Math., Texas State University--San Marcos, San Marcos, TX 78666}
       
\pagestyle{empty}


\begin{document}

\lstset{language=[Objective]Caml,frame=shadowbox}

\input{qm2pi.front}

% section front matter (end)

\input{qm2pi.intro} 
 
% section introduction (end)

% \input{qm2pi.knotations} 

% section notation (end)

\input{qm2pi.process.calculi} 

% section concurrent_process_calculi_and_spatial_logics_ (end)
    
%\input{qm2pi.knots2pi} 

%\input{qm2pi.trefoil} 

%\input{qm2pi.mainthm} 

% subsection basic_interpretation (end)

%\input{qm2pi.rho.presentation} 
\subsection{The syntax and semantics of the notation system}\label{sub:the_syntax_and_semantics_of_the_notation_system} % (fold)

We now summarize a technical presentation of the calculus that
embodies our theory of dynamics. The typical presentation of such a
calculus follows the style of giving generators and relations on
them. The grammar, below, describing term constructors, freely
generates the set of processes, $\Proc$. This set is then quotiented
by a relation known as structural congruence and it is over this set
that the notion of dynamics is expressed. This presentation is
essentially that of \cite{MeredithR05} with the addition of
polyadicity and summation. For readability we have relegated some of
the technical subtleties to an appendix.

\subsubsection{Process grammar}\label{subsub:process_grammar}

\begin{mathpar}
  \inferrule* [lab=synchronization] {} {{M} \bc \pzero \;|\; x?F \;|\; x!C }
  \and
  \inferrule* [lab=abstraction] {} {{F} \bc (x)P}
  \and
  \inferrule* [lab=concretion] {} {{C} \bc \langle Q \rangle}
  \and
  \inferrule* [lab=process] {} {{P,Q} \bc M \;| \;P|Q \;|\; @{x}}
  \and
  \inferrule* [lab=name] {} {{x} \bc \quotep{P}}
\end{mathpar} 

Note that $\vec{x}$ (resp. $\vec{P}$) denotes a vector of names
(resp. processes) of length $|\vec{x}|$ (resp. $|\vec{P}|$). We adopt
the following useful abbreviations.

\begin{mathpar}
   x?(\vec{y}).P := x.(\vec{y})P \and  x\clift{\vec{P}} := x.\clift{\vec{P}}
   \and x!(y) := \lift{x}{\dropn{y}}
   \and \Pi_{i=0}^{n-1}P_i := P_0 | \ldots | P_{n-1}
\end{mathpar}

\subsubsection{Structural congruence}

\paragraph{Free and bound names and alpha-equivalence.} At the
core of structural equivalence is alpha-equivalence which identifies
process that are the same up to a change of variable. Formally, we
recognize the distinction between free and bound names. The free names
of a process, $\freenames{P}$, may be calculated recursively as
follows:

\begin{mathpar}
\freenames{\pzero} := \emptyset
  \and \\
  \freenames{x?(y).P} := \{ x \} \cup (\freenames{P} \setminus \{ y \})
  \and 
  \freenames{x!\langle P \rangle} := \{ x \} \cup \{ P \} 
  \and \\
  \freenames{P|Q} := \freenames{P} \cup \freenames{Q}
  \and \\
  \freenames{@{x}} := \{ x \}
\end{mathpar}

$\pi$
$\quotep{\pi}$

$\freenames{-} : \pi \to \mathcal{P}(\quotep{\pi})$

\begin{eqnarray*}
  \freenames{\pzero} & := & \emptyset \\
  \freenames{x?(y).P} & := & \{ x \} \cup (\freenames{P} \setminus \{ y \}) \\
  \freenames{x!\langle P \rangle} & := & \{ x \} \cup \{ P \} \\
  \freenames{P|Q} & := & \freenames{P} \cup \freenames{Q} \\
  \freenames{\dropn{x}} & := & \{ x \}
\end{eqnarray*}

The bound names of a process, $\boundnames{P}$, are those names occurring in $P$
that are not free. For example, in $x?(y).0$, the name $x$ is free, while $y$ is bound.

\begin{mathpar}
  \inferrule* [lab=monoidal-laws] {} { P|Q \equiv Q|P \and P|0 \equiv P \and P|(Q|R) \equiv (P|Q)|R }
\end{mathpar}

\begin{mathpar}
  \inferrule* [lab=alpha-equivalence] {} { (x)P \equiv (y)P\{y/x\} \and y \not\in \freenames{P} }
\end{mathpar}

\begin{definition}
Then two processes, $P,Q$, are alpha-equivalent if $P = Q\{\vec{y}/\vec{x}\}$ for
some $\vec{x} \in \boundnames{Q},\vec{y} \in \boundnames{P}$, where $Q\{\vec{y}/\vec{x}\}$
denotes the capture-avoiding substitution of $\vec{y}$ for $\vec{x}$ in $Q$.
\end{definition}

\begin{definition}
  The {\em structural congruence} \cite{SangiorgiWalker} , $\equiv$,
  between processes is the least congruence containing
  alpha-equivalence, satisfying the abelian monoid laws
  (associativity, commutativity and $\pzero$ as identity) for parallel
  composition $|$ and for summation $+$.
\end{definition}

\subsection{Name equivalence}

We take name equivalence, written $\nameeq$, to be the smallest
equivalence relation generated by the following rules.

\begin{mathpar}
\inferrule*[lab=Quote-drop]
{ }
{ \quotep{@{x}} \nameeq x }

\inferrule*[lab=Struct-equiv]
{ P \scong Q }
{ \quotep{P} \nameeq \quotep{Q} }
\end{mathpar}

The astute reader will have noticed that the mutual recursion of names
and processes imposes a mutual recursion on alpha-equivalence and
structural equivalence via name-equivalence. Fortunately, all of this
works out pleasantly and we may calculate in the natural way, free of
concern. The reader interested in the details is referred to the
appendix \ref{appendix:rho_details}.

\subsection{Substitution}

We use $\Proc$ for the set of processes, $\QProc$ for the set of
names, and $\id{\{}\vec{y} / \vec{x} \id{\}}$ to denote partial maps,
$s : \QProc \rightarrow \QProc$. A map, $s$ lifts, uniquely, to a map
on process terms, $\widehat{s} : \Proc \rightarrow \Proc$ by the
following equations.

\begin{mathpar}
  (0) \psubstp{Q}{P} := 0 \\
  (R \juxtap S) \psubstp{Q}{P}
  :=    
  (R)\psubstp{Q}{P} \juxtap (S) \psubstp{Q}{P} \\
  (x?(y).R) \psubstp{Q}{P}    
  :=    
  (x)\substp{Q}{P} (z)\concat( (R \psubstn{z}{y}) \psubstp{Q}{P} ) \\
  (\lift{x}{R}) \psubstp{Q}{P}  
  :=
  \lift{(x)\substp{Q}{P}}{ R \psubstp{Q}{P} } \\
%   (\dropn{x})  \psubstp{Q}{P}       
%   := 
%   \left\{ 
%     \begin{array}{ccc} 
%       \dropn{\quotep{Q}} & & x \nameeq \quotep{P} \\
%       \dropn{x} & & otherwise \\
%     \end{array}
%   \right. 
  (\dropn{x})  \psubstp{Q}{P}       
  := 
  \left\{ 
    \begin{array}{ccc} 
      Q & & x \nameeq \quotep{P} \\
      \dropn{x} & & otherwise \\
    \end{array}
  \right.
\end{mathpar}
 

where

\begin{eqnarray}
  (x)\id{\{} \lpquote Q \rpquote / \lpquote P \rpquote \id{\}}            = 
  \left\{ 
    \begin{array}{ccc}
      \lpquote Q \rpquote & & x \nameeq \lpquote P \rpquote \\
      x & & otherwise \\
    \end{array}
  \right. \nonumber
\end{eqnarray}

and $z$ is chosen distinct from $\quotep{P}$, $\quotep{Q}$, the free
names in $Q$, and all the names in $R$. Our $\alpha$-equivalence will
be built in the standard way from this substitution.

\begin{remark}\label{rem:no_self_referential_names}
  One consequence of these definitions is that $\forall P. \quotep{P}
  \not\in \freenames{P}$.
\end{remark}

\subsection{ Dynamic quote: an example }

Anticipating something of what's to come, consider applying the
substitution, $\widehat{\id{\{}u / z \id{\}}}$, to the following pair
of processes, $\lift{w}{y!(z)}$ and $w[ \lpquote y!(z) \rpquote ]$.

\begin{eqnarray}
	\lift{w}{y!(z)}\widehat{\id{\{}u / z \id{\}}}
		& = &
		\lift{w}{y!(u)} \nonumber\\
	w[ \lpquote y!(z) \rpquote ] \widehat{ \id{\{}u / z \id{\}} }
		& = &
		w[ \lpquote y!(z) \rpquote ] \nonumber
\end{eqnarray}

Because the body of the process between quotes is impervious to
substitution, we get radically different answers. In fact, by
examining the first process in an input context,
e.g. $x?(z).\lift{w}{y!(z)}$, we see that the process under the lift
operator may be shaped by prefixed inputs binding a name inside it. In
this sense, the lift operator will be seen as a way to dynamically
construct processes before reifying them as names.

Finally equipped with these standard features we can present the
dynamics of the calculus.

\subsubsection{Operational semantics} 

Finally, we introduce the computational dynamics. What marks these
algebras as distinct from other more traditionally studied algebraic
structures, e.g. vector spaces or polynomial rings, is the manner in
which dynamics is captured. In traditional structures, dynamics is typically
expressed through morphisms between such structures, as in linear maps
between vector spaces or morphisms between rings. In algebras
associated with the semantics of computation, the dynamics is
expressed as part of the algebraic structure itself, through a
reduction reduction relation typically denoted by $\red$. Below, we
give a recursive presentation of this relation for the calculus used
in the encoding.

$\red \subseteq \pi \times \pi$
$\red : \pi \to \mathcal{P}(\pi)$

\begin{mathpar}
  \inferrule* [lab=Comm] { \textsf{match}( x_{src}, x_{trgt} ) } { x_{trgt}?(y)P \; | \; x_{src}!\langle {Q} \rangle \red P\{\quotep{Q}/y}\} }
  \and \\
  \inferrule* [lab=Par] {{P} \red {P}'} {{{P} | {Q}} \red {{P}' | {Q}}}
  \and
  \inferrule* [lab=Equiv]{{{P} \scong {P}'} \andalso {{P}' \red {Q}'} \andalso {{Q}' \scong {Q}}}{{P} \red {Q}}
\end{mathpar}

\begin{eqnarray*}
  match_{\equiv} (\quotep{P},\quotep{Q}) & := & P \equiv Q \\
  match_{\dagger}(\quotep{P},\quotep{Q}) & := & \forall R. P|Q \red^{*} R => R \red^{*} 0 \\
  match_{K}(\quotep{P},\quotep{Q}) & := & K \mbox{ for some context } K
\end{eqnarray*}

$u?(x)P | u!\langle Q \rangle \red P\{\quotep{Q}/x\}$

%We write $\wred$ for $\red^*$, and $P\red$ if $\exists Q $ such that $ P \red Q$.
We write $P\red$ if $\exists Q $ such that $ P \red Q$ and $P\not\red$, otherwise.

\section{Replication}

As mentioned before, it is known that replication (and hence
recursion) can be implemented in a higher-order process algebra
\cite{SangiorgiWalker}. As our first example of calculation with the
machinery thus far presented we give the construction explicitly in
the {\rhoc}.

\begin{eqnarray}
	D_{x} & := & \prefix{x}{y}{(\binpar{\outputp{x}{y}}{@{y}})} \nonumber\\
	\bangp_{x}{P} & := & \binpar{{x}!\langle{\binpar{D_{x}}{P}}\rangle}{D_{x}} \nonumber
\end{eqnarray}

\begin{eqnarray}
	\bangp_{x}{P} & & \nonumber\\
	=
	& {x}!\langle{(\prefix{x}{y}{(\outputp{x}{y} | @{y})) | P}}\rangle 
	      | \prefix{x}{y}{(\outputp{x}{y} | @{y})} & \nonumber\\
	\red
	& (\outputp{x}{y} | @{y})\substn{\quotep{(\prefix{x}{y}{(@{y} | \outputp{x}{y})) | P}}}{y} & \nonumber\\
	=
	& \outputp{x}{\quotep{(\prefix{x}{y}{(\outputp{x}{y} | @{y})) | P}}}
	  | {(\prefix{x}{y}{(\outputp{x}{y} | @{y})) | P}} & \nonumber\\
	\red
	& \ldots & \nonumber\\
	\red^*
	& P | P | \ldots & \nonumber
\end{eqnarray}

Of course, this encoding, as an implementation, runs away, unfolding
$\bangp{P}$ eagerly. A lazier and more implementable replication
operator, restricted to input-guarded processes, may be obtained as follows.

\begin{eqnarray}
\bangp{\prefix{u}{v}{P}} 
	:= 
	\binpar{\lift{x}{\prefix{u}{v}{(\binpar{D(x)}{P})}}}{D(x)} \nonumber
\end{eqnarray}

\begin{remark}
  Note that the lazier definition still does not deal with summation
  or mixed summation (i.e. sums over input and output). The reader is
  invited to construct definitions of replication that deal with these
  features. 

  Further, the definitions are parameterized in a name, $x$. Can you,
  gentle reader, make a definition that eliminates this parameter and
  guarantees no accidental interaction between the replication
  machinery and the process being replicated -- i.e. no accidental
  sharing of names used by the process to get its work done and the
  name(s) used by the replication to effect copying. This latter
  revision of the definition of replication is crucial to obtaining
  the expected identity $!!P \sim !P$.
\end{remark}

\begin{remark}\label{rem:paradoxical_combinator}
  The reader familiar with the lambda calculus will have noticed the
  similarity between $D$ and the paradoxical combinator.

  [Ed. note: the existence of this seems to suggest we have to be more
  restrictive on the set of processes and names we admit if we are to
  support no-cloning.]
\end{remark}

\subsubsection{Bisimulation}

The computational dynamics gives rise to another kind of equivalence,
the equivalence of computational behavior. As previously mentioned
this is typically captured \emph{via} some form of bisimulation.

% The notion we use in this paper is weak barbed bisimulation
% \cite{milner91polyadicpi}.

The notion we use in this paper is derived from weak barbed
bisimulation \cite{milner91polyadicpi}. 

\begin{definition}
An \emph{observation relation}, $\downarrow_{\mathcal N}$, over a set
of names, $\mathcal N$, is the smallest relation satisfying the rules
below.

\infrule[Out-barb]{y \in {\mathcal N}, \; x \nameeq y}
		  {\outputp{x}{v} \downarrow_{\mathcal N} x}
\infrule[Par-barb]{\mbox{$P\downarrow_{\mathcal N} x$ or $Q\downarrow_{\mathcal N} x$}}
		  {\binpar{P}{Q} \downarrow_{\mathcal N} x}

We write $P \Downarrow_{\mathcal N} x$ if there is $Q$ such that 
$P \wred Q$ and $Q \downarrow_{\mathcal N} x$.
\end{definition}

\begin{definition}
%\label{def.bbisim}
An  ${\mathcal N}$-\emph{barbed bisimulation} over a set of names, ${\mathcal N}$, is a symmetric binary relation 
${\mathcal S}_{\mathcal N}$ between agents such that $P\rel{S}_{\mathcal N}Q$ implies:
\begin{enumerate}
\item If $P \red P'$ then $Q \wred Q'$ and $P'\rel{S}_{\mathcal N} Q'$.
\item If $P\downarrow_{\mathcal N} x$, then $Q\Downarrow_{\mathcal N} x$.
\end{enumerate}
$P$ is ${\mathcal N}$-barbed bisimilar to $Q$, written
$P \wbbisim_{\mathcal N} Q$, if $P \rel{S}_{\mathcal N} Q$ for some ${\mathcal N}$-barbed bisimulation ${\mathcal S}_{\mathcal N}$.
\end{definition}

$\mathcal{R} \subseteq \pi \times \pi$

$P \mathcal{R} Q => \forall P'. P \red P' \Rightarrow \exists Q'. Q \red Q', P' \mathcal{R} Q'$

$P \vdash x \Rightarrow Q \vdash x$

\begin{mathpar}
  \inferrule*[lab=Out-barb]{x \nameeq y}{{y}!\langle{Q}\rangle \vdash x}
  \and
  \inferrule*[lab=Par-barb]{\mbox{$P\vdash x$ or $Q\vdash x$}}{\binpar{P}{Q} \vdash x}
\end{mathpar}

\subsubsection{Contexts}

One of the principle advantages of computational calculi like the
$\pi$-calculus is a well-defined notion of context,
contextual-equivalence and a correlation between
contextual-equivalence and notions of bisimulation. The notion of
context allows the decomposition of a process into (sub-)process and
its syntactic environment, its context. Thus, a context may be
thought of as a process with a ``hole'' (written $\Box$) in it. The
application of a context $M$ to a process $P$, written $M[P]$, is
tantamount to filling the hole in $M$ with $P$. In this paper we do
not need the full weight of this theory, but do make use of the notion
of context in the proof the main theorem. 

\begin{mathpar}
  \inferrule* [lab=summation] {} {{M_{M},M_{N}} \bc \Box \;|\; x.M_{A} \;|\; M_{M}+M_{N}}
  \and
  \inferrule* [lab=agent] {} {{M_{A}} \bc (\vec{x})M_{P} \;| \; \clift{P_0,\ldots,M_{P},\ldots,P_N}}
  \and \\
  \inferrule* [lab=process] {} {{M_{P}} \bc M_{N} \;| \;P|M_{P} }
\end{mathpar} 

\begin{mathpar}
  \inferrule* [lab=sychronization] {} {M_{N} \bc \Box \;|\; x?M_{F} \;|\; x!M_{C}}
  \and
  \inferrule* [lab=abstraction] {} {{M_{F}} \bc (x)M_{P} }
  \and
  \inferrule* [lab=concretion] {} {{M_{C}} \bc \langle M_{P} \rangle }
  \and \\
  \inferrule* [lab=process] {} {{M_{P}} \bc M_{N} \;| \;P|M_{P} }
\end{mathpar}

\begin{definition}[contextual application] Given a context $M$, and
  process $P$, we define the \emph{contextual application}, $M[P] :=
  M\{P/\Box\}$. That is, the contextual application of M to P is the
  substitution of $P$ for $\Box$ in $M$.
\end{definition}

$\meaningof{-} : L \to \mathcal{P}(\pi)$

\begin{mathpar}
  \inferrule* [lab=collection] {} {\meaningof{true} = \pi, \and \meaningof{~E} = \pi \setminus \meaningof{E}, \and \meaningof{E_{1} \& E_{2}} = \meaningof{E_{1}} \cap \meaningof{E_{2}}}
\end{mathpar}

\begin{mathpar}
  \inferrule* [lab=structure] {} {\meaningof{0} = \{ P \in \pi | P \equiv 0 \}, \and \\ \meaningof{E_1 | E_2} = \{ P \in \pi | P \equiv P_{1} | P_{2}, P_{1} \in \meaningof{E_{1}}, P_{2} \in \meaningof{E_2}\} }
\end{mathpar}

\begin{mathpar}
 \inferrule* [lab=behavior] {} {\meaningof{\langle a?b \rangle E} = \{ P \in \pi | P \equiv Q | u?(y)P', \\ \and \\\\ \and \\ \;\;\; u \in \meaningof{a}, \forall z.P'\{z/y\} \in \meaningof{E\{z/b\}}\}, \and \\ \meaningof{a!E} = \{ P \in \pi | P \equiv Q | x!\langle P' \rangle, x \in \meaningof{a} P' \in \meaningof{E}\} }
\end{mathpar}

\begin{mathpar}
 \inferrule* [lab=nominal] {} {\meaningof{\quotep{E}} = \{ \quotep{P} \in \quotep{\pi} | P \in \meaningof{E} \}, \and \meaningof{\quotep{P}} = \{ \quotep{Q} \in \quotep{\pi} | P \equiv Q \} \and \\ \meaningof{@\quotep{E}} = \{ P \in \pi | P \equiv @x, x \in \meaningof{E} \}}
\end{mathpar}

\begin{eqnarray*}
  \\
  \meaningof{-} : TS \to ST
\end{eqnarray*}

\begin{eqnarray*}
  \\
  L : TS \to ST
\end{eqnarray*}

\begin{eqnarray*}
  \\
  P \models E \iff P \in \meaningof{E}
\end{eqnarray*}

\begin{eqnarray*}
  P \approx_{L} Q \iff \forall E \in L. P \models E \iff Q \models E
\end{eqnarray*}

\begin{eqnarray*}
  P \approx_{K} Q
\end{eqnarray*}

\begin{eqnarray*}
  P \approx Q
\end{eqnarray*}

$\approx_{K} = \approx = \approx_{L}$

\subsubsection{Contextual duality}

Note that contexts extend the quotation operation to a family of
operations from processes to names. Given a context, $M$, we can
define a \emph{nominal context}, $\quotep{M}$ by $\quotep{M}[P] :=
\quotep{M[P]}$. To foreshadow what is to come we observe that these
operations enjoy a duality with processes very much like the duality
between vectors and maps from vectors to scalars.

Further, because the calculus is essentially higher-order, we have a
correspondence between contexts and processes. More specifically,
given a name $x$ and a context $M$ we can construct $M^{*}_{x}$ such
that 

\begin{mathpar}
  M^{*}_{x} | \lift{x}{P} \red M[P]
\end{mathpar}

namely,

\begin{mathpar}
  M^{*}_{x} := x?(u).M[\dropn{u}]
\end{mathpar}

The dependence of $M^{*}_{x}$ on a name makes it an abstraction, 

\begin{mathpar}
  M^{*} := (x)x?(u).M[\dropn{u}]
\end{mathpar}

\subsection{Additional notation}

It will sometimes be convenient to denote the process a name
quotes. We already have the notation $x = \quotep{P}$, but it will be
convenient to introduce an alternate notation, $\procn{x}$, when we
want to emphasize the connection to the use of the name. Note that, by
virtue of name equivalence, $\quotep{\procn{x}} \nameeq x$; so, the
notation is consistent with previous definitions.

Further, because names have structure it is possible to effect
substitutions on the basis of that structure. This means we need to
upgrade our notation for substitutions, which we accomplish by
adapting comprehension notation. Thus,

\begin{mathpar}
  P\{ y / x : x \in S \}
\end{mathpar}

is interpreted to mean the process derived from P by replacing (in a
capture-avoiding manner) each occurrence of $x$ in $S$ by $y$. For example,

\begin{mathpar}
  P\{ \quotep{\procn{x}|\procn{x}} / x : x \in \freenames{P} \}
\end{mathpar}

will replace each (occurrence) of a free name $x$ in $P$ by
$\quotep{\procn{x}|\procn{x}}$.

Also, we will avail ourselves of the notation $x^{L}$ and $x^{R}$ to
denote injections of a name into disjoint copies of the name
space. There are numerous ways to accomplish this. One example can be
found in \cite{MeredithR05}. This notation overloads to vectors of
names: $\vec{x}^{\pi} := (x_{i}^{\pi} \; : \; 0 \leq i < |\vec{x}| )$ where $\pi \in \{L,R\}$.

We also use $P^{\Box} := P|\Box$.

In \cite{MeredithR05} an interpretation of the new operator is
given. It turns out that there are several possible interpretations
all enjoying the requisite algebraic properties of the operator (see
\cite{milner91polyadicpi}). We will therefore make liberal use of
$(\nu\; \vec{x})P$.

% subsection the_syntax_and_semantics_of_the_notation_system (end)   

\input{qm2pi.qmops} 

\input{qm2pi.sterngerlach} 

\input{qm2pi.metric} 

% section concurrent_process_calculi (end)

%\input{qm2pi.proofsketch}

% section proof sketch (end)

%\input{qm2pi.slviaknots} 

% section spatial logic via knots (end)

\input{qm2pi.conclusion}

% section conclusion (end)

%\input{qm2pi.dtcodes} 

% section wiring algorithm (end)

\input{qm2pi.ack} 

% section acknowledgments (end)

\newpage


\bibliographystyle{plain}   
\bibliography{../../biblios/main.bib}

\input{qm2pi.rhodetails}

\end{document}

 

%\ifpdf
%\usepackage[pdftex]{graphicx}
%\else
%\usepackage{graphicx}
%\fi

 % \ifpdf
%  \usepackage{pdfsync}
%  \if


%\title{Brief Article}
%\author{David F. Snyder}
%\author{L.G. Meredith}

%\address{Dept. of Math., Texas State University--San Marcos, San Marcos, TX 78666}
       
\pagestyle{empty}


\begin{document}

\lstset{language=[Objective]Caml,frame=shadowbox}

\documentclass[12pt]{llncs}
%\documentclass{jktr}

\usepackage[pdftex]{hyperref}                   
\usepackage {listings}
\usepackage {mathpartir}
\usepackage{bcprules}
%\usepackage{listings}
                       
\usepackage{graphicx} 
%\usepackage[margins=2.5cm,nohead,nofoot]{geometry}
%\usepackage{geometry}
\usepackage{amsfonts}
\usepackage{amstext}
\usepackage{latexsym}
\usepackage{amssymb}
\usepackage{color}


%\include{myPreamble}
\include{qm2pi.local} 

%\ifpdf
%\usepackage[pdftex]{graphicx}
%\else
%\usepackage{graphicx}
%\fi

 % \ifpdf
%  \usepackage{pdfsync}
%  \if


%\title{Brief Article}
%\author{David F. Snyder}
%\author{L.G. Meredith}

%\address{Dept. of Math., Texas State University--San Marcos, San Marcos, TX 78666}
       
\pagestyle{empty}


\begin{document}

\lstset{language=[Objective]Caml,frame=shadowbox}

\input{qm2pi.front}

% section front matter (end)

\input{qm2pi.intro} 
 
% section introduction (end)

% \input{qm2pi.knotations} 

% section notation (end)

\input{qm2pi.process.calculi} 

% section concurrent_process_calculi_and_spatial_logics_ (end)
    
%\input{qm2pi.knots2pi} 

%\input{qm2pi.trefoil} 

%\input{qm2pi.mainthm} 

% subsection basic_interpretation (end)

%\input{qm2pi.rho.presentation} 
\subsection{The syntax and semantics of the notation system}\label{sub:the_syntax_and_semantics_of_the_notation_system} % (fold)

We now summarize a technical presentation of the calculus that
embodies our theory of dynamics. The typical presentation of such a
calculus follows the style of giving generators and relations on
them. The grammar, below, describing term constructors, freely
generates the set of processes, $\Proc$. This set is then quotiented
by a relation known as structural congruence and it is over this set
that the notion of dynamics is expressed. This presentation is
essentially that of \cite{MeredithR05} with the addition of
polyadicity and summation. For readability we have relegated some of
the technical subtleties to an appendix.

\subsubsection{Process grammar}\label{subsub:process_grammar}

\begin{mathpar}
  \inferrule* [lab=synchronization] {} {{M} \bc \pzero \;|\; x?F \;|\; x!C }
  \and
  \inferrule* [lab=abstraction] {} {{F} \bc (x)P}
  \and
  \inferrule* [lab=concretion] {} {{C} \bc \langle Q \rangle}
  \and
  \inferrule* [lab=process] {} {{P,Q} \bc M \;| \;P|Q \;|\; @{x}}
  \and
  \inferrule* [lab=name] {} {{x} \bc \quotep{P}}
\end{mathpar} 

Note that $\vec{x}$ (resp. $\vec{P}$) denotes a vector of names
(resp. processes) of length $|\vec{x}|$ (resp. $|\vec{P}|$). We adopt
the following useful abbreviations.

\begin{mathpar}
   x?(\vec{y}).P := x.(\vec{y})P \and  x\clift{\vec{P}} := x.\clift{\vec{P}}
   \and x!(y) := \lift{x}{\dropn{y}}
   \and \Pi_{i=0}^{n-1}P_i := P_0 | \ldots | P_{n-1}
\end{mathpar}

\subsubsection{Structural congruence}

\paragraph{Free and bound names and alpha-equivalence.} At the
core of structural equivalence is alpha-equivalence which identifies
process that are the same up to a change of variable. Formally, we
recognize the distinction between free and bound names. The free names
of a process, $\freenames{P}$, may be calculated recursively as
follows:

\begin{mathpar}
\freenames{\pzero} := \emptyset
  \and \\
  \freenames{x?(y).P} := \{ x \} \cup (\freenames{P} \setminus \{ y \})
  \and 
  \freenames{x!\langle P \rangle} := \{ x \} \cup \{ P \} 
  \and \\
  \freenames{P|Q} := \freenames{P} \cup \freenames{Q}
  \and \\
  \freenames{@{x}} := \{ x \}
\end{mathpar}

$\pi$
$\quotep{\pi}$

$\freenames{-} : \pi \to \mathcal{P}(\quotep{\pi})$

\begin{eqnarray*}
  \freenames{\pzero} & := & \emptyset \\
  \freenames{x?(y).P} & := & \{ x \} \cup (\freenames{P} \setminus \{ y \}) \\
  \freenames{x!\langle P \rangle} & := & \{ x \} \cup \{ P \} \\
  \freenames{P|Q} & := & \freenames{P} \cup \freenames{Q} \\
  \freenames{\dropn{x}} & := & \{ x \}
\end{eqnarray*}

The bound names of a process, $\boundnames{P}$, are those names occurring in $P$
that are not free. For example, in $x?(y).0$, the name $x$ is free, while $y$ is bound.

\begin{mathpar}
  \inferrule* [lab=monoidal-laws] {} { P|Q \equiv Q|P \and P|0 \equiv P \and P|(Q|R) \equiv (P|Q)|R }
\end{mathpar}

\begin{mathpar}
  \inferrule* [lab=alpha-equivalence] {} { (x)P \equiv (y)P\{y/x\} \and y \not\in \freenames{P} }
\end{mathpar}

\begin{definition}
Then two processes, $P,Q$, are alpha-equivalent if $P = Q\{\vec{y}/\vec{x}\}$ for
some $\vec{x} \in \boundnames{Q},\vec{y} \in \boundnames{P}$, where $Q\{\vec{y}/\vec{x}\}$
denotes the capture-avoiding substitution of $\vec{y}$ for $\vec{x}$ in $Q$.
\end{definition}

\begin{definition}
  The {\em structural congruence} \cite{SangiorgiWalker} , $\equiv$,
  between processes is the least congruence containing
  alpha-equivalence, satisfying the abelian monoid laws
  (associativity, commutativity and $\pzero$ as identity) for parallel
  composition $|$ and for summation $+$.
\end{definition}

\subsection{Name equivalence}

We take name equivalence, written $\nameeq$, to be the smallest
equivalence relation generated by the following rules.

\begin{mathpar}
\inferrule*[lab=Quote-drop]
{ }
{ \quotep{@{x}} \nameeq x }

\inferrule*[lab=Struct-equiv]
{ P \scong Q }
{ \quotep{P} \nameeq \quotep{Q} }
\end{mathpar}

The astute reader will have noticed that the mutual recursion of names
and processes imposes a mutual recursion on alpha-equivalence and
structural equivalence via name-equivalence. Fortunately, all of this
works out pleasantly and we may calculate in the natural way, free of
concern. The reader interested in the details is referred to the
appendix \ref{appendix:rho_details}.

\subsection{Substitution}

We use $\Proc$ for the set of processes, $\QProc$ for the set of
names, and $\id{\{}\vec{y} / \vec{x} \id{\}}$ to denote partial maps,
$s : \QProc \rightarrow \QProc$. A map, $s$ lifts, uniquely, to a map
on process terms, $\widehat{s} : \Proc \rightarrow \Proc$ by the
following equations.

\begin{mathpar}
  (0) \psubstp{Q}{P} := 0 \\
  (R \juxtap S) \psubstp{Q}{P}
  :=    
  (R)\psubstp{Q}{P} \juxtap (S) \psubstp{Q}{P} \\
  (x?(y).R) \psubstp{Q}{P}    
  :=    
  (x)\substp{Q}{P} (z)\concat( (R \psubstn{z}{y}) \psubstp{Q}{P} ) \\
  (\lift{x}{R}) \psubstp{Q}{P}  
  :=
  \lift{(x)\substp{Q}{P}}{ R \psubstp{Q}{P} } \\
%   (\dropn{x})  \psubstp{Q}{P}       
%   := 
%   \left\{ 
%     \begin{array}{ccc} 
%       \dropn{\quotep{Q}} & & x \nameeq \quotep{P} \\
%       \dropn{x} & & otherwise \\
%     \end{array}
%   \right. 
  (\dropn{x})  \psubstp{Q}{P}       
  := 
  \left\{ 
    \begin{array}{ccc} 
      Q & & x \nameeq \quotep{P} \\
      \dropn{x} & & otherwise \\
    \end{array}
  \right.
\end{mathpar}
 

where

\begin{eqnarray}
  (x)\id{\{} \lpquote Q \rpquote / \lpquote P \rpquote \id{\}}            = 
  \left\{ 
    \begin{array}{ccc}
      \lpquote Q \rpquote & & x \nameeq \lpquote P \rpquote \\
      x & & otherwise \\
    \end{array}
  \right. \nonumber
\end{eqnarray}

and $z$ is chosen distinct from $\quotep{P}$, $\quotep{Q}$, the free
names in $Q$, and all the names in $R$. Our $\alpha$-equivalence will
be built in the standard way from this substitution.

\begin{remark}\label{rem:no_self_referential_names}
  One consequence of these definitions is that $\forall P. \quotep{P}
  \not\in \freenames{P}$.
\end{remark}

\subsection{ Dynamic quote: an example }

Anticipating something of what's to come, consider applying the
substitution, $\widehat{\id{\{}u / z \id{\}}}$, to the following pair
of processes, $\lift{w}{y!(z)}$ and $w[ \lpquote y!(z) \rpquote ]$.

\begin{eqnarray}
	\lift{w}{y!(z)}\widehat{\id{\{}u / z \id{\}}}
		& = &
		\lift{w}{y!(u)} \nonumber\\
	w[ \lpquote y!(z) \rpquote ] \widehat{ \id{\{}u / z \id{\}} }
		& = &
		w[ \lpquote y!(z) \rpquote ] \nonumber
\end{eqnarray}

Because the body of the process between quotes is impervious to
substitution, we get radically different answers. In fact, by
examining the first process in an input context,
e.g. $x?(z).\lift{w}{y!(z)}$, we see that the process under the lift
operator may be shaped by prefixed inputs binding a name inside it. In
this sense, the lift operator will be seen as a way to dynamically
construct processes before reifying them as names.

Finally equipped with these standard features we can present the
dynamics of the calculus.

\subsubsection{Operational semantics} 

Finally, we introduce the computational dynamics. What marks these
algebras as distinct from other more traditionally studied algebraic
structures, e.g. vector spaces or polynomial rings, is the manner in
which dynamics is captured. In traditional structures, dynamics is typically
expressed through morphisms between such structures, as in linear maps
between vector spaces or morphisms between rings. In algebras
associated with the semantics of computation, the dynamics is
expressed as part of the algebraic structure itself, through a
reduction reduction relation typically denoted by $\red$. Below, we
give a recursive presentation of this relation for the calculus used
in the encoding.

$\red \subseteq \pi \times \pi$
$\red : \pi \to \mathcal{P}(\pi)$

\begin{mathpar}
  \inferrule* [lab=Comm] { \textsf{match}( x_{src}, x_{trgt} ) } { x_{trgt}?(y)P \; | \; x_{src}!\langle {Q} \rangle \red P\{\quotep{Q}/y}\} }
  \and \\
  \inferrule* [lab=Par] {{P} \red {P}'} {{{P} | {Q}} \red {{P}' | {Q}}}
  \and
  \inferrule* [lab=Equiv]{{{P} \scong {P}'} \andalso {{P}' \red {Q}'} \andalso {{Q}' \scong {Q}}}{{P} \red {Q}}
\end{mathpar}

\begin{eqnarray*}
  match_{\equiv} (\quotep{P},\quotep{Q}) & := & P \equiv Q \\
  match_{\dagger}(\quotep{P},\quotep{Q}) & := & \forall R. P|Q \red^{*} R => R \red^{*} 0 \\
  match_{K}(\quotep{P},\quotep{Q}) & := & K \mbox{ for some context } K
\end{eqnarray*}

$u?(x)P | u!\langle Q \rangle \red P\{\quotep{Q}/x\}$

%We write $\wred$ for $\red^*$, and $P\red$ if $\exists Q $ such that $ P \red Q$.
We write $P\red$ if $\exists Q $ such that $ P \red Q$ and $P\not\red$, otherwise.

\section{Replication}

As mentioned before, it is known that replication (and hence
recursion) can be implemented in a higher-order process algebra
\cite{SangiorgiWalker}. As our first example of calculation with the
machinery thus far presented we give the construction explicitly in
the {\rhoc}.

\begin{eqnarray}
	D_{x} & := & \prefix{x}{y}{(\binpar{\outputp{x}{y}}{@{y}})} \nonumber\\
	\bangp_{x}{P} & := & \binpar{{x}!\langle{\binpar{D_{x}}{P}}\rangle}{D_{x}} \nonumber
\end{eqnarray}

\begin{eqnarray}
	\bangp_{x}{P} & & \nonumber\\
	=
	& {x}!\langle{(\prefix{x}{y}{(\outputp{x}{y} | @{y})) | P}}\rangle 
	      | \prefix{x}{y}{(\outputp{x}{y} | @{y})} & \nonumber\\
	\red
	& (\outputp{x}{y} | @{y})\substn{\quotep{(\prefix{x}{y}{(@{y} | \outputp{x}{y})) | P}}}{y} & \nonumber\\
	=
	& \outputp{x}{\quotep{(\prefix{x}{y}{(\outputp{x}{y} | @{y})) | P}}}
	  | {(\prefix{x}{y}{(\outputp{x}{y} | @{y})) | P}} & \nonumber\\
	\red
	& \ldots & \nonumber\\
	\red^*
	& P | P | \ldots & \nonumber
\end{eqnarray}

Of course, this encoding, as an implementation, runs away, unfolding
$\bangp{P}$ eagerly. A lazier and more implementable replication
operator, restricted to input-guarded processes, may be obtained as follows.

\begin{eqnarray}
\bangp{\prefix{u}{v}{P}} 
	:= 
	\binpar{\lift{x}{\prefix{u}{v}{(\binpar{D(x)}{P})}}}{D(x)} \nonumber
\end{eqnarray}

\begin{remark}
  Note that the lazier definition still does not deal with summation
  or mixed summation (i.e. sums over input and output). The reader is
  invited to construct definitions of replication that deal with these
  features. 

  Further, the definitions are parameterized in a name, $x$. Can you,
  gentle reader, make a definition that eliminates this parameter and
  guarantees no accidental interaction between the replication
  machinery and the process being replicated -- i.e. no accidental
  sharing of names used by the process to get its work done and the
  name(s) used by the replication to effect copying. This latter
  revision of the definition of replication is crucial to obtaining
  the expected identity $!!P \sim !P$.
\end{remark}

\begin{remark}\label{rem:paradoxical_combinator}
  The reader familiar with the lambda calculus will have noticed the
  similarity between $D$ and the paradoxical combinator.

  [Ed. note: the existence of this seems to suggest we have to be more
  restrictive on the set of processes and names we admit if we are to
  support no-cloning.]
\end{remark}

\subsubsection{Bisimulation}

The computational dynamics gives rise to another kind of equivalence,
the equivalence of computational behavior. As previously mentioned
this is typically captured \emph{via} some form of bisimulation.

% The notion we use in this paper is weak barbed bisimulation
% \cite{milner91polyadicpi}.

The notion we use in this paper is derived from weak barbed
bisimulation \cite{milner91polyadicpi}. 

\begin{definition}
An \emph{observation relation}, $\downarrow_{\mathcal N}$, over a set
of names, $\mathcal N$, is the smallest relation satisfying the rules
below.

\infrule[Out-barb]{y \in {\mathcal N}, \; x \nameeq y}
		  {\outputp{x}{v} \downarrow_{\mathcal N} x}
\infrule[Par-barb]{\mbox{$P\downarrow_{\mathcal N} x$ or $Q\downarrow_{\mathcal N} x$}}
		  {\binpar{P}{Q} \downarrow_{\mathcal N} x}

We write $P \Downarrow_{\mathcal N} x$ if there is $Q$ such that 
$P \wred Q$ and $Q \downarrow_{\mathcal N} x$.
\end{definition}

\begin{definition}
%\label{def.bbisim}
An  ${\mathcal N}$-\emph{barbed bisimulation} over a set of names, ${\mathcal N}$, is a symmetric binary relation 
${\mathcal S}_{\mathcal N}$ between agents such that $P\rel{S}_{\mathcal N}Q$ implies:
\begin{enumerate}
\item If $P \red P'$ then $Q \wred Q'$ and $P'\rel{S}_{\mathcal N} Q'$.
\item If $P\downarrow_{\mathcal N} x$, then $Q\Downarrow_{\mathcal N} x$.
\end{enumerate}
$P$ is ${\mathcal N}$-barbed bisimilar to $Q$, written
$P \wbbisim_{\mathcal N} Q$, if $P \rel{S}_{\mathcal N} Q$ for some ${\mathcal N}$-barbed bisimulation ${\mathcal S}_{\mathcal N}$.
\end{definition}

$\mathcal{R} \subseteq \pi \times \pi$

$P \mathcal{R} Q => \forall P'. P \red P' \Rightarrow \exists Q'. Q \red Q', P' \mathcal{R} Q'$

$P \vdash x \Rightarrow Q \vdash x$

\begin{mathpar}
  \inferrule*[lab=Out-barb]{x \nameeq y}{{y}!\langle{Q}\rangle \vdash x}
  \and
  \inferrule*[lab=Par-barb]{\mbox{$P\vdash x$ or $Q\vdash x$}}{\binpar{P}{Q} \vdash x}
\end{mathpar}

\subsubsection{Contexts}

One of the principle advantages of computational calculi like the
$\pi$-calculus is a well-defined notion of context,
contextual-equivalence and a correlation between
contextual-equivalence and notions of bisimulation. The notion of
context allows the decomposition of a process into (sub-)process and
its syntactic environment, its context. Thus, a context may be
thought of as a process with a ``hole'' (written $\Box$) in it. The
application of a context $M$ to a process $P$, written $M[P]$, is
tantamount to filling the hole in $M$ with $P$. In this paper we do
not need the full weight of this theory, but do make use of the notion
of context in the proof the main theorem. 

\begin{mathpar}
  \inferrule* [lab=summation] {} {{M_{M},M_{N}} \bc \Box \;|\; x.M_{A} \;|\; M_{M}+M_{N}}
  \and
  \inferrule* [lab=agent] {} {{M_{A}} \bc (\vec{x})M_{P} \;| \; \clift{P_0,\ldots,M_{P},\ldots,P_N}}
  \and \\
  \inferrule* [lab=process] {} {{M_{P}} \bc M_{N} \;| \;P|M_{P} }
\end{mathpar} 

\begin{mathpar}
  \inferrule* [lab=sychronization] {} {M_{N} \bc \Box \;|\; x?M_{F} \;|\; x!M_{C}}
  \and
  \inferrule* [lab=abstraction] {} {{M_{F}} \bc (x)M_{P} }
  \and
  \inferrule* [lab=concretion] {} {{M_{C}} \bc \langle M_{P} \rangle }
  \and \\
  \inferrule* [lab=process] {} {{M_{P}} \bc M_{N} \;| \;P|M_{P} }
\end{mathpar}

\begin{definition}[contextual application] Given a context $M$, and
  process $P$, we define the \emph{contextual application}, $M[P] :=
  M\{P/\Box\}$. That is, the contextual application of M to P is the
  substitution of $P$ for $\Box$ in $M$.
\end{definition}

$\meaningof{-} : L \to \mathcal{P}(\pi)$

\begin{mathpar}
  \inferrule* [lab=collection] {} {\meaningof{true} = \pi, \and \meaningof{~E} = \pi \setminus \meaningof{E}, \and \meaningof{E_{1} \& E_{2}} = \meaningof{E_{1}} \cap \meaningof{E_{2}}}
\end{mathpar}

\begin{mathpar}
  \inferrule* [lab=structure] {} {\meaningof{0} = \{ P \in \pi | P \equiv 0 \}, \and \\ \meaningof{E_1 | E_2} = \{ P \in \pi | P \equiv P_{1} | P_{2}, P_{1} \in \meaningof{E_{1}}, P_{2} \in \meaningof{E_2}\} }
\end{mathpar}

\begin{mathpar}
 \inferrule* [lab=behavior] {} {\meaningof{\langle a?b \rangle E} = \{ P \in \pi | P \equiv Q | u?(y)P', \\ \and \\\\ \and \\ \;\;\; u \in \meaningof{a}, \forall z.P'\{z/y\} \in \meaningof{E\{z/b\}}\}, \and \\ \meaningof{a!E} = \{ P \in \pi | P \equiv Q | x!\langle P' \rangle, x \in \meaningof{a} P' \in \meaningof{E}\} }
\end{mathpar}

\begin{mathpar}
 \inferrule* [lab=nominal] {} {\meaningof{\quotep{E}} = \{ \quotep{P} \in \quotep{\pi} | P \in \meaningof{E} \}, \and \meaningof{\quotep{P}} = \{ \quotep{Q} \in \quotep{\pi} | P \equiv Q \} \and \\ \meaningof{@\quotep{E}} = \{ P \in \pi | P \equiv @x, x \in \meaningof{E} \}}
\end{mathpar}

\begin{eqnarray*}
  \\
  \meaningof{-} : TS \to ST
\end{eqnarray*}

\begin{eqnarray*}
  \\
  L : TS \to ST
\end{eqnarray*}

\begin{eqnarray*}
  \\
  P \models E \iff P \in \meaningof{E}
\end{eqnarray*}

\begin{eqnarray*}
  P \approx_{L} Q \iff \forall E \in L. P \models E \iff Q \models E
\end{eqnarray*}

\begin{eqnarray*}
  P \approx_{K} Q
\end{eqnarray*}

\begin{eqnarray*}
  P \approx Q
\end{eqnarray*}

$\approx_{K} = \approx = \approx_{L}$

\subsubsection{Contextual duality}

Note that contexts extend the quotation operation to a family of
operations from processes to names. Given a context, $M$, we can
define a \emph{nominal context}, $\quotep{M}$ by $\quotep{M}[P] :=
\quotep{M[P]}$. To foreshadow what is to come we observe that these
operations enjoy a duality with processes very much like the duality
between vectors and maps from vectors to scalars.

Further, because the calculus is essentially higher-order, we have a
correspondence between contexts and processes. More specifically,
given a name $x$ and a context $M$ we can construct $M^{*}_{x}$ such
that 

\begin{mathpar}
  M^{*}_{x} | \lift{x}{P} \red M[P]
\end{mathpar}

namely,

\begin{mathpar}
  M^{*}_{x} := x?(u).M[\dropn{u}]
\end{mathpar}

The dependence of $M^{*}_{x}$ on a name makes it an abstraction, 

\begin{mathpar}
  M^{*} := (x)x?(u).M[\dropn{u}]
\end{mathpar}

\subsection{Additional notation}

It will sometimes be convenient to denote the process a name
quotes. We already have the notation $x = \quotep{P}$, but it will be
convenient to introduce an alternate notation, $\procn{x}$, when we
want to emphasize the connection to the use of the name. Note that, by
virtue of name equivalence, $\quotep{\procn{x}} \nameeq x$; so, the
notation is consistent with previous definitions.

Further, because names have structure it is possible to effect
substitutions on the basis of that structure. This means we need to
upgrade our notation for substitutions, which we accomplish by
adapting comprehension notation. Thus,

\begin{mathpar}
  P\{ y / x : x \in S \}
\end{mathpar}

is interpreted to mean the process derived from P by replacing (in a
capture-avoiding manner) each occurrence of $x$ in $S$ by $y$. For example,

\begin{mathpar}
  P\{ \quotep{\procn{x}|\procn{x}} / x : x \in \freenames{P} \}
\end{mathpar}

will replace each (occurrence) of a free name $x$ in $P$ by
$\quotep{\procn{x}|\procn{x}}$.

Also, we will avail ourselves of the notation $x^{L}$ and $x^{R}$ to
denote injections of a name into disjoint copies of the name
space. There are numerous ways to accomplish this. One example can be
found in \cite{MeredithR05}. This notation overloads to vectors of
names: $\vec{x}^{\pi} := (x_{i}^{\pi} \; : \; 0 \leq i < |\vec{x}| )$ where $\pi \in \{L,R\}$.

We also use $P^{\Box} := P|\Box$.

In \cite{MeredithR05} an interpretation of the new operator is
given. It turns out that there are several possible interpretations
all enjoying the requisite algebraic properties of the operator (see
\cite{milner91polyadicpi}). We will therefore make liberal use of
$(\nu\; \vec{x})P$.

% subsection the_syntax_and_semantics_of_the_notation_system (end)   

\input{qm2pi.qmops} 

\input{qm2pi.sterngerlach} 

\input{qm2pi.metric} 

% section concurrent_process_calculi (end)

%\input{qm2pi.proofsketch}

% section proof sketch (end)

%\input{qm2pi.slviaknots} 

% section spatial logic via knots (end)

\input{qm2pi.conclusion}

% section conclusion (end)

%\input{qm2pi.dtcodes} 

% section wiring algorithm (end)

\input{qm2pi.ack} 

% section acknowledgments (end)

\newpage


\bibliographystyle{plain}   
\bibliography{../../biblios/main.bib}

\input{qm2pi.rhodetails}

\end{document}



% section front matter (end)

\section{Introduction}\label{sec:introduction} % (fold)
In this draft of the material i am going to have to dispense with the
usual writing conventions adopted in papers on these topics. i'm going
to have adopt whatever tone i need at the time i'm writing up the
calculations. Sometimes this may be very conversational; others it may
be the barest mathematical grunts; others still it may be that i have
lifted text from one of my other papers because the exposition of some
point was better said there. i hope that my readers are not unduly put
out by this decision. i'm not doing this to flout convention or be
rebellious. i find these calculations very technically challenging. To
keep everything going technically, something has to give; i have to
let go of some cognitive burden. So, the academic writing style --
with all of its trade-offs in terms of facilitating technical
communication -- is what i'm letting go of. Perhaps subsequent drafts
can be tightened and polished, but for now, i'm going to speak as if
we were sitting together in a coffee shop with a laptop, wifi and a
pad of paper and a pencil.

So, here's what i have to say. We -- you and i, comfortably ensconced
in our coffee shop and well-equipped with our tools -- can realize and
carry out the calculations of quantum mechanics over a very different
formal theory of dynamics, a formal theory of dynamics that
corresponds to a theory of concurrent computation with
\emph{reflection}. It has the advantage that the underlying theory is
already `quantized', but supports analogues all of the continuuous
operations. Strikingly, this underlying theory has recently been
connected with a notion of metric that we can show, by calculating
together, coincides with the metric induced by the inner product.

There are a lot of reasons why you might be interested in seeing
calculations of this form. Here's why i'm interested. For the past
several centuries there has been no competitor to the ``Newtonian''
account of dynamics. As a result the predominant share of accounts of
dynamical systems and situations have had to be formulated in terms of
the Newtonian machinery. i view this as an intellectually dangerous
position to occupy. Everything, despite it's intrinsic shape, turns
into a nail to be hit with this hammer. Recently, however, the theory
of computation has matured to the point where we have candidates for
theories of dynamics that offer very different perspective on
reasoning about dynamical systems and situations. Testing these
candidates against very successful accounts of dynamical situations,
like quantum mechanics, is going to give us some sense of how mature
they are and some measure of the quality of these accounts of
dynamics.

\subsection{Summary of contributions and outline of paper}

So, we're going to develop an interpretation of the operations of
quantum mechanics normally interpreted by Hilbert spaces and
operators. We're going to do this over a theory of computation. Note
that this is very different than the usual quantum computation program
which develops notions of computation over quantum mechanics. Rather,
we are developing a story that aligns with Wheeler's slogan: It from
Bit. To do this we will first provide an account of the theory of
computation at play here. Then we will dive into a calculation-driven
interpretation of the operations of quantum mechanics.

The reason we take this approach is that -- until very recently --
there hasn't been an axiomatic account of quantum mechanics. As a
result there has been no sharp delineation of the mathematical theory
supporting interpretation of the physical theory and the physical
theory, itself. So, ambient features of the maths are free to be
exploited (or supressed) without a real accounting of their physical
relevance. There is no sharp statement ``here's the physical theory''
qua \emph{theory} and ``here's the mathematical interpretation''
enabling a judgment of how faithful the interpretation is -- apart
from experimental observation. When there is an axiomatic account we
can judge how well a given mathematical formalism supports an
interpretation of the axioms, independent of
experimentation. Likewise, we can judge how well we have captured our
physical evidence and experience with our axiomatics, independent of
any specific mathematical implementation, with accidental detail that
may or may not have physical significance. 

In lieu of a fully fleshed out and vetted axiomatic account of quantum
mechanics, interpreting the operational notions in service of modeling
physical systems will have to suffice. In other words, we are not in
the business of providing a model of Hilbert spaces and operators. We
are in the business of providing a model of quantum mechanics because
we are motivated by testing our notions of dynamics against physical
theory; and, the predictive calculations of the physical theory must
serve as the best formulation -- shy of a fully fleshed out axiomatic
account -- of the physical theory itself (as they have for scientific
theories since time immemorial). Put another way, despite a
whole-hearted commitment to an It-from-Bit ontology, we are firmly
aligned with the shut-up-and-calculate camp as the best way to obtain
results either from the physical perspective or as a quality assurance
measure of our fledgling theory of dynamics.

In detail, we present a reflective process calculus. Then we develop
intuitive correspondences between the notions available in this
calculus and the usual physical notions supporting quantum mechanical
calculations. Thus, 

\begin{table}[htp]
  \center{
    \fbox{
      \begin{tabular}{c|c}
        quantum mechanics & process calculus \\
        \hline
        scalar & name \\
        state vector & process \\
        dual & contextual duals \\
        matrix & formal sums of process-context-dual pairs \\
        orthogonality & process annihilation \\
        inner product & execution-formula + quoting
      \end{tabular}
    }
  }
  \caption{QM - process calculi correspondences}
\end{table}

Then we tighten up these intuitions to operational definitions. We
employ the Dirac notation as the best proxy we can find for an
abstract syntax of the quantum mechanical notions. The definitions we
develop put us in contact with equational constraints coming from the
theory that we demonstrate the definitions and calculations satisfy.

This puts us in a position to shut up and calculate for the
Stern-Gerlach experimental set up, showing how these predictive
calculations become calculations on processes in our theory of a
reflective process calculus.

Penultimately, we demonstrate that the notion of metric coming from
the inner product coincides with the notion of metric available from
the theory of bisimulation. This demonstration gives us the right to
think of space as arising from behavior. Finally, we consider where we
might go from the new vantage point we have obtained.

% section introduction (end) 
 
% section introduction (end)

% \documentclass[12pt]{llncs}
%\documentclass{jktr}

\usepackage[pdftex]{hyperref}                   
\usepackage {listings}
\usepackage {mathpartir}
\usepackage{bcprules}
%\usepackage{listings}
                       
\usepackage{graphicx} 
%\usepackage[margins=2.5cm,nohead,nofoot]{geometry}
%\usepackage{geometry}
\usepackage{amsfonts}
\usepackage{amstext}
\usepackage{latexsym}
\usepackage{amssymb}
\usepackage{color}


%\include{myPreamble}
\include{qm2pi.local} 

%\ifpdf
%\usepackage[pdftex]{graphicx}
%\else
%\usepackage{graphicx}
%\fi

 % \ifpdf
%  \usepackage{pdfsync}
%  \if


%\title{Brief Article}
%\author{David F. Snyder}
%\author{L.G. Meredith}

%\address{Dept. of Math., Texas State University--San Marcos, San Marcos, TX 78666}
       
\pagestyle{empty}


\begin{document}

\lstset{language=[Objective]Caml,frame=shadowbox}

\input{qm2pi.front}

% section front matter (end)

\input{qm2pi.intro} 
 
% section introduction (end)

% \input{qm2pi.knotations} 

% section notation (end)

\input{qm2pi.process.calculi} 

% section concurrent_process_calculi_and_spatial_logics_ (end)
    
%\input{qm2pi.knots2pi} 

%\input{qm2pi.trefoil} 

%\input{qm2pi.mainthm} 

% subsection basic_interpretation (end)

%\input{qm2pi.rho.presentation} 
\subsection{The syntax and semantics of the notation system}\label{sub:the_syntax_and_semantics_of_the_notation_system} % (fold)

We now summarize a technical presentation of the calculus that
embodies our theory of dynamics. The typical presentation of such a
calculus follows the style of giving generators and relations on
them. The grammar, below, describing term constructors, freely
generates the set of processes, $\Proc$. This set is then quotiented
by a relation known as structural congruence and it is over this set
that the notion of dynamics is expressed. This presentation is
essentially that of \cite{MeredithR05} with the addition of
polyadicity and summation. For readability we have relegated some of
the technical subtleties to an appendix.

\subsubsection{Process grammar}\label{subsub:process_grammar}

\begin{mathpar}
  \inferrule* [lab=synchronization] {} {{M} \bc \pzero \;|\; x?F \;|\; x!C }
  \and
  \inferrule* [lab=abstraction] {} {{F} \bc (x)P}
  \and
  \inferrule* [lab=concretion] {} {{C} \bc \langle Q \rangle}
  \and
  \inferrule* [lab=process] {} {{P,Q} \bc M \;| \;P|Q \;|\; @{x}}
  \and
  \inferrule* [lab=name] {} {{x} \bc \quotep{P}}
\end{mathpar} 

Note that $\vec{x}$ (resp. $\vec{P}$) denotes a vector of names
(resp. processes) of length $|\vec{x}|$ (resp. $|\vec{P}|$). We adopt
the following useful abbreviations.

\begin{mathpar}
   x?(\vec{y}).P := x.(\vec{y})P \and  x\clift{\vec{P}} := x.\clift{\vec{P}}
   \and x!(y) := \lift{x}{\dropn{y}}
   \and \Pi_{i=0}^{n-1}P_i := P_0 | \ldots | P_{n-1}
\end{mathpar}

\subsubsection{Structural congruence}

\paragraph{Free and bound names and alpha-equivalence.} At the
core of structural equivalence is alpha-equivalence which identifies
process that are the same up to a change of variable. Formally, we
recognize the distinction between free and bound names. The free names
of a process, $\freenames{P}$, may be calculated recursively as
follows:

\begin{mathpar}
\freenames{\pzero} := \emptyset
  \and \\
  \freenames{x?(y).P} := \{ x \} \cup (\freenames{P} \setminus \{ y \})
  \and 
  \freenames{x!\langle P \rangle} := \{ x \} \cup \{ P \} 
  \and \\
  \freenames{P|Q} := \freenames{P} \cup \freenames{Q}
  \and \\
  \freenames{@{x}} := \{ x \}
\end{mathpar}

$\pi$
$\quotep{\pi}$

$\freenames{-} : \pi \to \mathcal{P}(\quotep{\pi})$

\begin{eqnarray*}
  \freenames{\pzero} & := & \emptyset \\
  \freenames{x?(y).P} & := & \{ x \} \cup (\freenames{P} \setminus \{ y \}) \\
  \freenames{x!\langle P \rangle} & := & \{ x \} \cup \{ P \} \\
  \freenames{P|Q} & := & \freenames{P} \cup \freenames{Q} \\
  \freenames{\dropn{x}} & := & \{ x \}
\end{eqnarray*}

The bound names of a process, $\boundnames{P}$, are those names occurring in $P$
that are not free. For example, in $x?(y).0$, the name $x$ is free, while $y$ is bound.

\begin{mathpar}
  \inferrule* [lab=monoidal-laws] {} { P|Q \equiv Q|P \and P|0 \equiv P \and P|(Q|R) \equiv (P|Q)|R }
\end{mathpar}

\begin{mathpar}
  \inferrule* [lab=alpha-equivalence] {} { (x)P \equiv (y)P\{y/x\} \and y \not\in \freenames{P} }
\end{mathpar}

\begin{definition}
Then two processes, $P,Q$, are alpha-equivalent if $P = Q\{\vec{y}/\vec{x}\}$ for
some $\vec{x} \in \boundnames{Q},\vec{y} \in \boundnames{P}$, where $Q\{\vec{y}/\vec{x}\}$
denotes the capture-avoiding substitution of $\vec{y}$ for $\vec{x}$ in $Q$.
\end{definition}

\begin{definition}
  The {\em structural congruence} \cite{SangiorgiWalker} , $\equiv$,
  between processes is the least congruence containing
  alpha-equivalence, satisfying the abelian monoid laws
  (associativity, commutativity and $\pzero$ as identity) for parallel
  composition $|$ and for summation $+$.
\end{definition}

\subsection{Name equivalence}

We take name equivalence, written $\nameeq$, to be the smallest
equivalence relation generated by the following rules.

\begin{mathpar}
\inferrule*[lab=Quote-drop]
{ }
{ \quotep{@{x}} \nameeq x }

\inferrule*[lab=Struct-equiv]
{ P \scong Q }
{ \quotep{P} \nameeq \quotep{Q} }
\end{mathpar}

The astute reader will have noticed that the mutual recursion of names
and processes imposes a mutual recursion on alpha-equivalence and
structural equivalence via name-equivalence. Fortunately, all of this
works out pleasantly and we may calculate in the natural way, free of
concern. The reader interested in the details is referred to the
appendix \ref{appendix:rho_details}.

\subsection{Substitution}

We use $\Proc$ for the set of processes, $\QProc$ for the set of
names, and $\id{\{}\vec{y} / \vec{x} \id{\}}$ to denote partial maps,
$s : \QProc \rightarrow \QProc$. A map, $s$ lifts, uniquely, to a map
on process terms, $\widehat{s} : \Proc \rightarrow \Proc$ by the
following equations.

\begin{mathpar}
  (0) \psubstp{Q}{P} := 0 \\
  (R \juxtap S) \psubstp{Q}{P}
  :=    
  (R)\psubstp{Q}{P} \juxtap (S) \psubstp{Q}{P} \\
  (x?(y).R) \psubstp{Q}{P}    
  :=    
  (x)\substp{Q}{P} (z)\concat( (R \psubstn{z}{y}) \psubstp{Q}{P} ) \\
  (\lift{x}{R}) \psubstp{Q}{P}  
  :=
  \lift{(x)\substp{Q}{P}}{ R \psubstp{Q}{P} } \\
%   (\dropn{x})  \psubstp{Q}{P}       
%   := 
%   \left\{ 
%     \begin{array}{ccc} 
%       \dropn{\quotep{Q}} & & x \nameeq \quotep{P} \\
%       \dropn{x} & & otherwise \\
%     \end{array}
%   \right. 
  (\dropn{x})  \psubstp{Q}{P}       
  := 
  \left\{ 
    \begin{array}{ccc} 
      Q & & x \nameeq \quotep{P} \\
      \dropn{x} & & otherwise \\
    \end{array}
  \right.
\end{mathpar}
 

where

\begin{eqnarray}
  (x)\id{\{} \lpquote Q \rpquote / \lpquote P \rpquote \id{\}}            = 
  \left\{ 
    \begin{array}{ccc}
      \lpquote Q \rpquote & & x \nameeq \lpquote P \rpquote \\
      x & & otherwise \\
    \end{array}
  \right. \nonumber
\end{eqnarray}

and $z$ is chosen distinct from $\quotep{P}$, $\quotep{Q}$, the free
names in $Q$, and all the names in $R$. Our $\alpha$-equivalence will
be built in the standard way from this substitution.

\begin{remark}\label{rem:no_self_referential_names}
  One consequence of these definitions is that $\forall P. \quotep{P}
  \not\in \freenames{P}$.
\end{remark}

\subsection{ Dynamic quote: an example }

Anticipating something of what's to come, consider applying the
substitution, $\widehat{\id{\{}u / z \id{\}}}$, to the following pair
of processes, $\lift{w}{y!(z)}$ and $w[ \lpquote y!(z) \rpquote ]$.

\begin{eqnarray}
	\lift{w}{y!(z)}\widehat{\id{\{}u / z \id{\}}}
		& = &
		\lift{w}{y!(u)} \nonumber\\
	w[ \lpquote y!(z) \rpquote ] \widehat{ \id{\{}u / z \id{\}} }
		& = &
		w[ \lpquote y!(z) \rpquote ] \nonumber
\end{eqnarray}

Because the body of the process between quotes is impervious to
substitution, we get radically different answers. In fact, by
examining the first process in an input context,
e.g. $x?(z).\lift{w}{y!(z)}$, we see that the process under the lift
operator may be shaped by prefixed inputs binding a name inside it. In
this sense, the lift operator will be seen as a way to dynamically
construct processes before reifying them as names.

Finally equipped with these standard features we can present the
dynamics of the calculus.

\subsubsection{Operational semantics} 

Finally, we introduce the computational dynamics. What marks these
algebras as distinct from other more traditionally studied algebraic
structures, e.g. vector spaces or polynomial rings, is the manner in
which dynamics is captured. In traditional structures, dynamics is typically
expressed through morphisms between such structures, as in linear maps
between vector spaces or morphisms between rings. In algebras
associated with the semantics of computation, the dynamics is
expressed as part of the algebraic structure itself, through a
reduction reduction relation typically denoted by $\red$. Below, we
give a recursive presentation of this relation for the calculus used
in the encoding.

$\red \subseteq \pi \times \pi$
$\red : \pi \to \mathcal{P}(\pi)$

\begin{mathpar}
  \inferrule* [lab=Comm] { \textsf{match}( x_{src}, x_{trgt} ) } { x_{trgt}?(y)P \; | \; x_{src}!\langle {Q} \rangle \red P\{\quotep{Q}/y}\} }
  \and \\
  \inferrule* [lab=Par] {{P} \red {P}'} {{{P} | {Q}} \red {{P}' | {Q}}}
  \and
  \inferrule* [lab=Equiv]{{{P} \scong {P}'} \andalso {{P}' \red {Q}'} \andalso {{Q}' \scong {Q}}}{{P} \red {Q}}
\end{mathpar}

\begin{eqnarray*}
  match_{\equiv} (\quotep{P},\quotep{Q}) & := & P \equiv Q \\
  match_{\dagger}(\quotep{P},\quotep{Q}) & := & \forall R. P|Q \red^{*} R => R \red^{*} 0 \\
  match_{K}(\quotep{P},\quotep{Q}) & := & K \mbox{ for some context } K
\end{eqnarray*}

$u?(x)P | u!\langle Q \rangle \red P\{\quotep{Q}/x\}$

%We write $\wred$ for $\red^*$, and $P\red$ if $\exists Q $ such that $ P \red Q$.
We write $P\red$ if $\exists Q $ such that $ P \red Q$ and $P\not\red$, otherwise.

\section{Replication}

As mentioned before, it is known that replication (and hence
recursion) can be implemented in a higher-order process algebra
\cite{SangiorgiWalker}. As our first example of calculation with the
machinery thus far presented we give the construction explicitly in
the {\rhoc}.

\begin{eqnarray}
	D_{x} & := & \prefix{x}{y}{(\binpar{\outputp{x}{y}}{@{y}})} \nonumber\\
	\bangp_{x}{P} & := & \binpar{{x}!\langle{\binpar{D_{x}}{P}}\rangle}{D_{x}} \nonumber
\end{eqnarray}

\begin{eqnarray}
	\bangp_{x}{P} & & \nonumber\\
	=
	& {x}!\langle{(\prefix{x}{y}{(\outputp{x}{y} | @{y})) | P}}\rangle 
	      | \prefix{x}{y}{(\outputp{x}{y} | @{y})} & \nonumber\\
	\red
	& (\outputp{x}{y} | @{y})\substn{\quotep{(\prefix{x}{y}{(@{y} | \outputp{x}{y})) | P}}}{y} & \nonumber\\
	=
	& \outputp{x}{\quotep{(\prefix{x}{y}{(\outputp{x}{y} | @{y})) | P}}}
	  | {(\prefix{x}{y}{(\outputp{x}{y} | @{y})) | P}} & \nonumber\\
	\red
	& \ldots & \nonumber\\
	\red^*
	& P | P | \ldots & \nonumber
\end{eqnarray}

Of course, this encoding, as an implementation, runs away, unfolding
$\bangp{P}$ eagerly. A lazier and more implementable replication
operator, restricted to input-guarded processes, may be obtained as follows.

\begin{eqnarray}
\bangp{\prefix{u}{v}{P}} 
	:= 
	\binpar{\lift{x}{\prefix{u}{v}{(\binpar{D(x)}{P})}}}{D(x)} \nonumber
\end{eqnarray}

\begin{remark}
  Note that the lazier definition still does not deal with summation
  or mixed summation (i.e. sums over input and output). The reader is
  invited to construct definitions of replication that deal with these
  features. 

  Further, the definitions are parameterized in a name, $x$. Can you,
  gentle reader, make a definition that eliminates this parameter and
  guarantees no accidental interaction between the replication
  machinery and the process being replicated -- i.e. no accidental
  sharing of names used by the process to get its work done and the
  name(s) used by the replication to effect copying. This latter
  revision of the definition of replication is crucial to obtaining
  the expected identity $!!P \sim !P$.
\end{remark}

\begin{remark}\label{rem:paradoxical_combinator}
  The reader familiar with the lambda calculus will have noticed the
  similarity between $D$ and the paradoxical combinator.

  [Ed. note: the existence of this seems to suggest we have to be more
  restrictive on the set of processes and names we admit if we are to
  support no-cloning.]
\end{remark}

\subsubsection{Bisimulation}

The computational dynamics gives rise to another kind of equivalence,
the equivalence of computational behavior. As previously mentioned
this is typically captured \emph{via} some form of bisimulation.

% The notion we use in this paper is weak barbed bisimulation
% \cite{milner91polyadicpi}.

The notion we use in this paper is derived from weak barbed
bisimulation \cite{milner91polyadicpi}. 

\begin{definition}
An \emph{observation relation}, $\downarrow_{\mathcal N}$, over a set
of names, $\mathcal N$, is the smallest relation satisfying the rules
below.

\infrule[Out-barb]{y \in {\mathcal N}, \; x \nameeq y}
		  {\outputp{x}{v} \downarrow_{\mathcal N} x}
\infrule[Par-barb]{\mbox{$P\downarrow_{\mathcal N} x$ or $Q\downarrow_{\mathcal N} x$}}
		  {\binpar{P}{Q} \downarrow_{\mathcal N} x}

We write $P \Downarrow_{\mathcal N} x$ if there is $Q$ such that 
$P \wred Q$ and $Q \downarrow_{\mathcal N} x$.
\end{definition}

\begin{definition}
%\label{def.bbisim}
An  ${\mathcal N}$-\emph{barbed bisimulation} over a set of names, ${\mathcal N}$, is a symmetric binary relation 
${\mathcal S}_{\mathcal N}$ between agents such that $P\rel{S}_{\mathcal N}Q$ implies:
\begin{enumerate}
\item If $P \red P'$ then $Q \wred Q'$ and $P'\rel{S}_{\mathcal N} Q'$.
\item If $P\downarrow_{\mathcal N} x$, then $Q\Downarrow_{\mathcal N} x$.
\end{enumerate}
$P$ is ${\mathcal N}$-barbed bisimilar to $Q$, written
$P \wbbisim_{\mathcal N} Q$, if $P \rel{S}_{\mathcal N} Q$ for some ${\mathcal N}$-barbed bisimulation ${\mathcal S}_{\mathcal N}$.
\end{definition}

$\mathcal{R} \subseteq \pi \times \pi$

$P \mathcal{R} Q => \forall P'. P \red P' \Rightarrow \exists Q'. Q \red Q', P' \mathcal{R} Q'$

$P \vdash x \Rightarrow Q \vdash x$

\begin{mathpar}
  \inferrule*[lab=Out-barb]{x \nameeq y}{{y}!\langle{Q}\rangle \vdash x}
  \and
  \inferrule*[lab=Par-barb]{\mbox{$P\vdash x$ or $Q\vdash x$}}{\binpar{P}{Q} \vdash x}
\end{mathpar}

\subsubsection{Contexts}

One of the principle advantages of computational calculi like the
$\pi$-calculus is a well-defined notion of context,
contextual-equivalence and a correlation between
contextual-equivalence and notions of bisimulation. The notion of
context allows the decomposition of a process into (sub-)process and
its syntactic environment, its context. Thus, a context may be
thought of as a process with a ``hole'' (written $\Box$) in it. The
application of a context $M$ to a process $P$, written $M[P]$, is
tantamount to filling the hole in $M$ with $P$. In this paper we do
not need the full weight of this theory, but do make use of the notion
of context in the proof the main theorem. 

\begin{mathpar}
  \inferrule* [lab=summation] {} {{M_{M},M_{N}} \bc \Box \;|\; x.M_{A} \;|\; M_{M}+M_{N}}
  \and
  \inferrule* [lab=agent] {} {{M_{A}} \bc (\vec{x})M_{P} \;| \; \clift{P_0,\ldots,M_{P},\ldots,P_N}}
  \and \\
  \inferrule* [lab=process] {} {{M_{P}} \bc M_{N} \;| \;P|M_{P} }
\end{mathpar} 

\begin{mathpar}
  \inferrule* [lab=sychronization] {} {M_{N} \bc \Box \;|\; x?M_{F} \;|\; x!M_{C}}
  \and
  \inferrule* [lab=abstraction] {} {{M_{F}} \bc (x)M_{P} }
  \and
  \inferrule* [lab=concretion] {} {{M_{C}} \bc \langle M_{P} \rangle }
  \and \\
  \inferrule* [lab=process] {} {{M_{P}} \bc M_{N} \;| \;P|M_{P} }
\end{mathpar}

\begin{definition}[contextual application] Given a context $M$, and
  process $P$, we define the \emph{contextual application}, $M[P] :=
  M\{P/\Box\}$. That is, the contextual application of M to P is the
  substitution of $P$ for $\Box$ in $M$.
\end{definition}

$\meaningof{-} : L \to \mathcal{P}(\pi)$

\begin{mathpar}
  \inferrule* [lab=collection] {} {\meaningof{true} = \pi, \and \meaningof{~E} = \pi \setminus \meaningof{E}, \and \meaningof{E_{1} \& E_{2}} = \meaningof{E_{1}} \cap \meaningof{E_{2}}}
\end{mathpar}

\begin{mathpar}
  \inferrule* [lab=structure] {} {\meaningof{0} = \{ P \in \pi | P \equiv 0 \}, \and \\ \meaningof{E_1 | E_2} = \{ P \in \pi | P \equiv P_{1} | P_{2}, P_{1} \in \meaningof{E_{1}}, P_{2} \in \meaningof{E_2}\} }
\end{mathpar}

\begin{mathpar}
 \inferrule* [lab=behavior] {} {\meaningof{\langle a?b \rangle E} = \{ P \in \pi | P \equiv Q | u?(y)P', \\ \and \\\\ \and \\ \;\;\; u \in \meaningof{a}, \forall z.P'\{z/y\} \in \meaningof{E\{z/b\}}\}, \and \\ \meaningof{a!E} = \{ P \in \pi | P \equiv Q | x!\langle P' \rangle, x \in \meaningof{a} P' \in \meaningof{E}\} }
\end{mathpar}

\begin{mathpar}
 \inferrule* [lab=nominal] {} {\meaningof{\quotep{E}} = \{ \quotep{P} \in \quotep{\pi} | P \in \meaningof{E} \}, \and \meaningof{\quotep{P}} = \{ \quotep{Q} \in \quotep{\pi} | P \equiv Q \} \and \\ \meaningof{@\quotep{E}} = \{ P \in \pi | P \equiv @x, x \in \meaningof{E} \}}
\end{mathpar}

\begin{eqnarray*}
  \\
  \meaningof{-} : TS \to ST
\end{eqnarray*}

\begin{eqnarray*}
  \\
  L : TS \to ST
\end{eqnarray*}

\begin{eqnarray*}
  \\
  P \models E \iff P \in \meaningof{E}
\end{eqnarray*}

\begin{eqnarray*}
  P \approx_{L} Q \iff \forall E \in L. P \models E \iff Q \models E
\end{eqnarray*}

\begin{eqnarray*}
  P \approx_{K} Q
\end{eqnarray*}

\begin{eqnarray*}
  P \approx Q
\end{eqnarray*}

$\approx_{K} = \approx = \approx_{L}$

\subsubsection{Contextual duality}

Note that contexts extend the quotation operation to a family of
operations from processes to names. Given a context, $M$, we can
define a \emph{nominal context}, $\quotep{M}$ by $\quotep{M}[P] :=
\quotep{M[P]}$. To foreshadow what is to come we observe that these
operations enjoy a duality with processes very much like the duality
between vectors and maps from vectors to scalars.

Further, because the calculus is essentially higher-order, we have a
correspondence between contexts and processes. More specifically,
given a name $x$ and a context $M$ we can construct $M^{*}_{x}$ such
that 

\begin{mathpar}
  M^{*}_{x} | \lift{x}{P} \red M[P]
\end{mathpar}

namely,

\begin{mathpar}
  M^{*}_{x} := x?(u).M[\dropn{u}]
\end{mathpar}

The dependence of $M^{*}_{x}$ on a name makes it an abstraction, 

\begin{mathpar}
  M^{*} := (x)x?(u).M[\dropn{u}]
\end{mathpar}

\subsection{Additional notation}

It will sometimes be convenient to denote the process a name
quotes. We already have the notation $x = \quotep{P}$, but it will be
convenient to introduce an alternate notation, $\procn{x}$, when we
want to emphasize the connection to the use of the name. Note that, by
virtue of name equivalence, $\quotep{\procn{x}} \nameeq x$; so, the
notation is consistent with previous definitions.

Further, because names have structure it is possible to effect
substitutions on the basis of that structure. This means we need to
upgrade our notation for substitutions, which we accomplish by
adapting comprehension notation. Thus,

\begin{mathpar}
  P\{ y / x : x \in S \}
\end{mathpar}

is interpreted to mean the process derived from P by replacing (in a
capture-avoiding manner) each occurrence of $x$ in $S$ by $y$. For example,

\begin{mathpar}
  P\{ \quotep{\procn{x}|\procn{x}} / x : x \in \freenames{P} \}
\end{mathpar}

will replace each (occurrence) of a free name $x$ in $P$ by
$\quotep{\procn{x}|\procn{x}}$.

Also, we will avail ourselves of the notation $x^{L}$ and $x^{R}$ to
denote injections of a name into disjoint copies of the name
space. There are numerous ways to accomplish this. One example can be
found in \cite{MeredithR05}. This notation overloads to vectors of
names: $\vec{x}^{\pi} := (x_{i}^{\pi} \; : \; 0 \leq i < |\vec{x}| )$ where $\pi \in \{L,R\}$.

We also use $P^{\Box} := P|\Box$.

In \cite{MeredithR05} an interpretation of the new operator is
given. It turns out that there are several possible interpretations
all enjoying the requisite algebraic properties of the operator (see
\cite{milner91polyadicpi}). We will therefore make liberal use of
$(\nu\; \vec{x})P$.

% subsection the_syntax_and_semantics_of_the_notation_system (end)   

\input{qm2pi.qmops} 

\input{qm2pi.sterngerlach} 

\input{qm2pi.metric} 

% section concurrent_process_calculi (end)

%\input{qm2pi.proofsketch}

% section proof sketch (end)

%\input{qm2pi.slviaknots} 

% section spatial logic via knots (end)

\input{qm2pi.conclusion}

% section conclusion (end)

%\input{qm2pi.dtcodes} 

% section wiring algorithm (end)

\input{qm2pi.ack} 

% section acknowledgments (end)

\newpage


\bibliographystyle{plain}   
\bibliography{../../biblios/main.bib}

\input{qm2pi.rhodetails}

\end{document}

 

% section notation (end)

\input{qm2pi.process.calculi} 

% section concurrent_process_calculi_and_spatial_logics_ (end)
    
%\documentclass[12pt]{llncs}
%\documentclass{jktr}

\usepackage[pdftex]{hyperref}                   
\usepackage {listings}
\usepackage {mathpartir}
\usepackage{bcprules}
%\usepackage{listings}
                       
\usepackage{graphicx} 
%\usepackage[margins=2.5cm,nohead,nofoot]{geometry}
%\usepackage{geometry}
\usepackage{amsfonts}
\usepackage{amstext}
\usepackage{latexsym}
\usepackage{amssymb}
\usepackage{color}


%\include{myPreamble}
\include{qm2pi.local} 

%\ifpdf
%\usepackage[pdftex]{graphicx}
%\else
%\usepackage{graphicx}
%\fi

 % \ifpdf
%  \usepackage{pdfsync}
%  \if


%\title{Brief Article}
%\author{David F. Snyder}
%\author{L.G. Meredith}

%\address{Dept. of Math., Texas State University--San Marcos, San Marcos, TX 78666}
       
\pagestyle{empty}


\begin{document}

\lstset{language=[Objective]Caml,frame=shadowbox}

\input{qm2pi.front}

% section front matter (end)

\input{qm2pi.intro} 
 
% section introduction (end)

% \input{qm2pi.knotations} 

% section notation (end)

\input{qm2pi.process.calculi} 

% section concurrent_process_calculi_and_spatial_logics_ (end)
    
%\input{qm2pi.knots2pi} 

%\input{qm2pi.trefoil} 

%\input{qm2pi.mainthm} 

% subsection basic_interpretation (end)

%\input{qm2pi.rho.presentation} 
\subsection{The syntax and semantics of the notation system}\label{sub:the_syntax_and_semantics_of_the_notation_system} % (fold)

We now summarize a technical presentation of the calculus that
embodies our theory of dynamics. The typical presentation of such a
calculus follows the style of giving generators and relations on
them. The grammar, below, describing term constructors, freely
generates the set of processes, $\Proc$. This set is then quotiented
by a relation known as structural congruence and it is over this set
that the notion of dynamics is expressed. This presentation is
essentially that of \cite{MeredithR05} with the addition of
polyadicity and summation. For readability we have relegated some of
the technical subtleties to an appendix.

\subsubsection{Process grammar}\label{subsub:process_grammar}

\begin{mathpar}
  \inferrule* [lab=synchronization] {} {{M} \bc \pzero \;|\; x?F \;|\; x!C }
  \and
  \inferrule* [lab=abstraction] {} {{F} \bc (x)P}
  \and
  \inferrule* [lab=concretion] {} {{C} \bc \langle Q \rangle}
  \and
  \inferrule* [lab=process] {} {{P,Q} \bc M \;| \;P|Q \;|\; @{x}}
  \and
  \inferrule* [lab=name] {} {{x} \bc \quotep{P}}
\end{mathpar} 

Note that $\vec{x}$ (resp. $\vec{P}$) denotes a vector of names
(resp. processes) of length $|\vec{x}|$ (resp. $|\vec{P}|$). We adopt
the following useful abbreviations.

\begin{mathpar}
   x?(\vec{y}).P := x.(\vec{y})P \and  x\clift{\vec{P}} := x.\clift{\vec{P}}
   \and x!(y) := \lift{x}{\dropn{y}}
   \and \Pi_{i=0}^{n-1}P_i := P_0 | \ldots | P_{n-1}
\end{mathpar}

\subsubsection{Structural congruence}

\paragraph{Free and bound names and alpha-equivalence.} At the
core of structural equivalence is alpha-equivalence which identifies
process that are the same up to a change of variable. Formally, we
recognize the distinction between free and bound names. The free names
of a process, $\freenames{P}$, may be calculated recursively as
follows:

\begin{mathpar}
\freenames{\pzero} := \emptyset
  \and \\
  \freenames{x?(y).P} := \{ x \} \cup (\freenames{P} \setminus \{ y \})
  \and 
  \freenames{x!\langle P \rangle} := \{ x \} \cup \{ P \} 
  \and \\
  \freenames{P|Q} := \freenames{P} \cup \freenames{Q}
  \and \\
  \freenames{@{x}} := \{ x \}
\end{mathpar}

$\pi$
$\quotep{\pi}$

$\freenames{-} : \pi \to \mathcal{P}(\quotep{\pi})$

\begin{eqnarray*}
  \freenames{\pzero} & := & \emptyset \\
  \freenames{x?(y).P} & := & \{ x \} \cup (\freenames{P} \setminus \{ y \}) \\
  \freenames{x!\langle P \rangle} & := & \{ x \} \cup \{ P \} \\
  \freenames{P|Q} & := & \freenames{P} \cup \freenames{Q} \\
  \freenames{\dropn{x}} & := & \{ x \}
\end{eqnarray*}

The bound names of a process, $\boundnames{P}$, are those names occurring in $P$
that are not free. For example, in $x?(y).0$, the name $x$ is free, while $y$ is bound.

\begin{mathpar}
  \inferrule* [lab=monoidal-laws] {} { P|Q \equiv Q|P \and P|0 \equiv P \and P|(Q|R) \equiv (P|Q)|R }
\end{mathpar}

\begin{mathpar}
  \inferrule* [lab=alpha-equivalence] {} { (x)P \equiv (y)P\{y/x\} \and y \not\in \freenames{P} }
\end{mathpar}

\begin{definition}
Then two processes, $P,Q$, are alpha-equivalent if $P = Q\{\vec{y}/\vec{x}\}$ for
some $\vec{x} \in \boundnames{Q},\vec{y} \in \boundnames{P}$, where $Q\{\vec{y}/\vec{x}\}$
denotes the capture-avoiding substitution of $\vec{y}$ for $\vec{x}$ in $Q$.
\end{definition}

\begin{definition}
  The {\em structural congruence} \cite{SangiorgiWalker} , $\equiv$,
  between processes is the least congruence containing
  alpha-equivalence, satisfying the abelian monoid laws
  (associativity, commutativity and $\pzero$ as identity) for parallel
  composition $|$ and for summation $+$.
\end{definition}

\subsection{Name equivalence}

We take name equivalence, written $\nameeq$, to be the smallest
equivalence relation generated by the following rules.

\begin{mathpar}
\inferrule*[lab=Quote-drop]
{ }
{ \quotep{@{x}} \nameeq x }

\inferrule*[lab=Struct-equiv]
{ P \scong Q }
{ \quotep{P} \nameeq \quotep{Q} }
\end{mathpar}

The astute reader will have noticed that the mutual recursion of names
and processes imposes a mutual recursion on alpha-equivalence and
structural equivalence via name-equivalence. Fortunately, all of this
works out pleasantly and we may calculate in the natural way, free of
concern. The reader interested in the details is referred to the
appendix \ref{appendix:rho_details}.

\subsection{Substitution}

We use $\Proc$ for the set of processes, $\QProc$ for the set of
names, and $\id{\{}\vec{y} / \vec{x} \id{\}}$ to denote partial maps,
$s : \QProc \rightarrow \QProc$. A map, $s$ lifts, uniquely, to a map
on process terms, $\widehat{s} : \Proc \rightarrow \Proc$ by the
following equations.

\begin{mathpar}
  (0) \psubstp{Q}{P} := 0 \\
  (R \juxtap S) \psubstp{Q}{P}
  :=    
  (R)\psubstp{Q}{P} \juxtap (S) \psubstp{Q}{P} \\
  (x?(y).R) \psubstp{Q}{P}    
  :=    
  (x)\substp{Q}{P} (z)\concat( (R \psubstn{z}{y}) \psubstp{Q}{P} ) \\
  (\lift{x}{R}) \psubstp{Q}{P}  
  :=
  \lift{(x)\substp{Q}{P}}{ R \psubstp{Q}{P} } \\
%   (\dropn{x})  \psubstp{Q}{P}       
%   := 
%   \left\{ 
%     \begin{array}{ccc} 
%       \dropn{\quotep{Q}} & & x \nameeq \quotep{P} \\
%       \dropn{x} & & otherwise \\
%     \end{array}
%   \right. 
  (\dropn{x})  \psubstp{Q}{P}       
  := 
  \left\{ 
    \begin{array}{ccc} 
      Q & & x \nameeq \quotep{P} \\
      \dropn{x} & & otherwise \\
    \end{array}
  \right.
\end{mathpar}
 

where

\begin{eqnarray}
  (x)\id{\{} \lpquote Q \rpquote / \lpquote P \rpquote \id{\}}            = 
  \left\{ 
    \begin{array}{ccc}
      \lpquote Q \rpquote & & x \nameeq \lpquote P \rpquote \\
      x & & otherwise \\
    \end{array}
  \right. \nonumber
\end{eqnarray}

and $z$ is chosen distinct from $\quotep{P}$, $\quotep{Q}$, the free
names in $Q$, and all the names in $R$. Our $\alpha$-equivalence will
be built in the standard way from this substitution.

\begin{remark}\label{rem:no_self_referential_names}
  One consequence of these definitions is that $\forall P. \quotep{P}
  \not\in \freenames{P}$.
\end{remark}

\subsection{ Dynamic quote: an example }

Anticipating something of what's to come, consider applying the
substitution, $\widehat{\id{\{}u / z \id{\}}}$, to the following pair
of processes, $\lift{w}{y!(z)}$ and $w[ \lpquote y!(z) \rpquote ]$.

\begin{eqnarray}
	\lift{w}{y!(z)}\widehat{\id{\{}u / z \id{\}}}
		& = &
		\lift{w}{y!(u)} \nonumber\\
	w[ \lpquote y!(z) \rpquote ] \widehat{ \id{\{}u / z \id{\}} }
		& = &
		w[ \lpquote y!(z) \rpquote ] \nonumber
\end{eqnarray}

Because the body of the process between quotes is impervious to
substitution, we get radically different answers. In fact, by
examining the first process in an input context,
e.g. $x?(z).\lift{w}{y!(z)}$, we see that the process under the lift
operator may be shaped by prefixed inputs binding a name inside it. In
this sense, the lift operator will be seen as a way to dynamically
construct processes before reifying them as names.

Finally equipped with these standard features we can present the
dynamics of the calculus.

\subsubsection{Operational semantics} 

Finally, we introduce the computational dynamics. What marks these
algebras as distinct from other more traditionally studied algebraic
structures, e.g. vector spaces or polynomial rings, is the manner in
which dynamics is captured. In traditional structures, dynamics is typically
expressed through morphisms between such structures, as in linear maps
between vector spaces or morphisms between rings. In algebras
associated with the semantics of computation, the dynamics is
expressed as part of the algebraic structure itself, through a
reduction reduction relation typically denoted by $\red$. Below, we
give a recursive presentation of this relation for the calculus used
in the encoding.

$\red \subseteq \pi \times \pi$
$\red : \pi \to \mathcal{P}(\pi)$

\begin{mathpar}
  \inferrule* [lab=Comm] { \textsf{match}( x_{src}, x_{trgt} ) } { x_{trgt}?(y)P \; | \; x_{src}!\langle {Q} \rangle \red P\{\quotep{Q}/y}\} }
  \and \\
  \inferrule* [lab=Par] {{P} \red {P}'} {{{P} | {Q}} \red {{P}' | {Q}}}
  \and
  \inferrule* [lab=Equiv]{{{P} \scong {P}'} \andalso {{P}' \red {Q}'} \andalso {{Q}' \scong {Q}}}{{P} \red {Q}}
\end{mathpar}

\begin{eqnarray*}
  match_{\equiv} (\quotep{P},\quotep{Q}) & := & P \equiv Q \\
  match_{\dagger}(\quotep{P},\quotep{Q}) & := & \forall R. P|Q \red^{*} R => R \red^{*} 0 \\
  match_{K}(\quotep{P},\quotep{Q}) & := & K \mbox{ for some context } K
\end{eqnarray*}

$u?(x)P | u!\langle Q \rangle \red P\{\quotep{Q}/x\}$

%We write $\wred$ for $\red^*$, and $P\red$ if $\exists Q $ such that $ P \red Q$.
We write $P\red$ if $\exists Q $ such that $ P \red Q$ and $P\not\red$, otherwise.

\section{Replication}

As mentioned before, it is known that replication (and hence
recursion) can be implemented in a higher-order process algebra
\cite{SangiorgiWalker}. As our first example of calculation with the
machinery thus far presented we give the construction explicitly in
the {\rhoc}.

\begin{eqnarray}
	D_{x} & := & \prefix{x}{y}{(\binpar{\outputp{x}{y}}{@{y}})} \nonumber\\
	\bangp_{x}{P} & := & \binpar{{x}!\langle{\binpar{D_{x}}{P}}\rangle}{D_{x}} \nonumber
\end{eqnarray}

\begin{eqnarray}
	\bangp_{x}{P} & & \nonumber\\
	=
	& {x}!\langle{(\prefix{x}{y}{(\outputp{x}{y} | @{y})) | P}}\rangle 
	      | \prefix{x}{y}{(\outputp{x}{y} | @{y})} & \nonumber\\
	\red
	& (\outputp{x}{y} | @{y})\substn{\quotep{(\prefix{x}{y}{(@{y} | \outputp{x}{y})) | P}}}{y} & \nonumber\\
	=
	& \outputp{x}{\quotep{(\prefix{x}{y}{(\outputp{x}{y} | @{y})) | P}}}
	  | {(\prefix{x}{y}{(\outputp{x}{y} | @{y})) | P}} & \nonumber\\
	\red
	& \ldots & \nonumber\\
	\red^*
	& P | P | \ldots & \nonumber
\end{eqnarray}

Of course, this encoding, as an implementation, runs away, unfolding
$\bangp{P}$ eagerly. A lazier and more implementable replication
operator, restricted to input-guarded processes, may be obtained as follows.

\begin{eqnarray}
\bangp{\prefix{u}{v}{P}} 
	:= 
	\binpar{\lift{x}{\prefix{u}{v}{(\binpar{D(x)}{P})}}}{D(x)} \nonumber
\end{eqnarray}

\begin{remark}
  Note that the lazier definition still does not deal with summation
  or mixed summation (i.e. sums over input and output). The reader is
  invited to construct definitions of replication that deal with these
  features. 

  Further, the definitions are parameterized in a name, $x$. Can you,
  gentle reader, make a definition that eliminates this parameter and
  guarantees no accidental interaction between the replication
  machinery and the process being replicated -- i.e. no accidental
  sharing of names used by the process to get its work done and the
  name(s) used by the replication to effect copying. This latter
  revision of the definition of replication is crucial to obtaining
  the expected identity $!!P \sim !P$.
\end{remark}

\begin{remark}\label{rem:paradoxical_combinator}
  The reader familiar with the lambda calculus will have noticed the
  similarity between $D$ and the paradoxical combinator.

  [Ed. note: the existence of this seems to suggest we have to be more
  restrictive on the set of processes and names we admit if we are to
  support no-cloning.]
\end{remark}

\subsubsection{Bisimulation}

The computational dynamics gives rise to another kind of equivalence,
the equivalence of computational behavior. As previously mentioned
this is typically captured \emph{via} some form of bisimulation.

% The notion we use in this paper is weak barbed bisimulation
% \cite{milner91polyadicpi}.

The notion we use in this paper is derived from weak barbed
bisimulation \cite{milner91polyadicpi}. 

\begin{definition}
An \emph{observation relation}, $\downarrow_{\mathcal N}$, over a set
of names, $\mathcal N$, is the smallest relation satisfying the rules
below.

\infrule[Out-barb]{y \in {\mathcal N}, \; x \nameeq y}
		  {\outputp{x}{v} \downarrow_{\mathcal N} x}
\infrule[Par-barb]{\mbox{$P\downarrow_{\mathcal N} x$ or $Q\downarrow_{\mathcal N} x$}}
		  {\binpar{P}{Q} \downarrow_{\mathcal N} x}

We write $P \Downarrow_{\mathcal N} x$ if there is $Q$ such that 
$P \wred Q$ and $Q \downarrow_{\mathcal N} x$.
\end{definition}

\begin{definition}
%\label{def.bbisim}
An  ${\mathcal N}$-\emph{barbed bisimulation} over a set of names, ${\mathcal N}$, is a symmetric binary relation 
${\mathcal S}_{\mathcal N}$ between agents such that $P\rel{S}_{\mathcal N}Q$ implies:
\begin{enumerate}
\item If $P \red P'$ then $Q \wred Q'$ and $P'\rel{S}_{\mathcal N} Q'$.
\item If $P\downarrow_{\mathcal N} x$, then $Q\Downarrow_{\mathcal N} x$.
\end{enumerate}
$P$ is ${\mathcal N}$-barbed bisimilar to $Q$, written
$P \wbbisim_{\mathcal N} Q$, if $P \rel{S}_{\mathcal N} Q$ for some ${\mathcal N}$-barbed bisimulation ${\mathcal S}_{\mathcal N}$.
\end{definition}

$\mathcal{R} \subseteq \pi \times \pi$

$P \mathcal{R} Q => \forall P'. P \red P' \Rightarrow \exists Q'. Q \red Q', P' \mathcal{R} Q'$

$P \vdash x \Rightarrow Q \vdash x$

\begin{mathpar}
  \inferrule*[lab=Out-barb]{x \nameeq y}{{y}!\langle{Q}\rangle \vdash x}
  \and
  \inferrule*[lab=Par-barb]{\mbox{$P\vdash x$ or $Q\vdash x$}}{\binpar{P}{Q} \vdash x}
\end{mathpar}

\subsubsection{Contexts}

One of the principle advantages of computational calculi like the
$\pi$-calculus is a well-defined notion of context,
contextual-equivalence and a correlation between
contextual-equivalence and notions of bisimulation. The notion of
context allows the decomposition of a process into (sub-)process and
its syntactic environment, its context. Thus, a context may be
thought of as a process with a ``hole'' (written $\Box$) in it. The
application of a context $M$ to a process $P$, written $M[P]$, is
tantamount to filling the hole in $M$ with $P$. In this paper we do
not need the full weight of this theory, but do make use of the notion
of context in the proof the main theorem. 

\begin{mathpar}
  \inferrule* [lab=summation] {} {{M_{M},M_{N}} \bc \Box \;|\; x.M_{A} \;|\; M_{M}+M_{N}}
  \and
  \inferrule* [lab=agent] {} {{M_{A}} \bc (\vec{x})M_{P} \;| \; \clift{P_0,\ldots,M_{P},\ldots,P_N}}
  \and \\
  \inferrule* [lab=process] {} {{M_{P}} \bc M_{N} \;| \;P|M_{P} }
\end{mathpar} 

\begin{mathpar}
  \inferrule* [lab=sychronization] {} {M_{N} \bc \Box \;|\; x?M_{F} \;|\; x!M_{C}}
  \and
  \inferrule* [lab=abstraction] {} {{M_{F}} \bc (x)M_{P} }
  \and
  \inferrule* [lab=concretion] {} {{M_{C}} \bc \langle M_{P} \rangle }
  \and \\
  \inferrule* [lab=process] {} {{M_{P}} \bc M_{N} \;| \;P|M_{P} }
\end{mathpar}

\begin{definition}[contextual application] Given a context $M$, and
  process $P$, we define the \emph{contextual application}, $M[P] :=
  M\{P/\Box\}$. That is, the contextual application of M to P is the
  substitution of $P$ for $\Box$ in $M$.
\end{definition}

$\meaningof{-} : L \to \mathcal{P}(\pi)$

\begin{mathpar}
  \inferrule* [lab=collection] {} {\meaningof{true} = \pi, \and \meaningof{~E} = \pi \setminus \meaningof{E}, \and \meaningof{E_{1} \& E_{2}} = \meaningof{E_{1}} \cap \meaningof{E_{2}}}
\end{mathpar}

\begin{mathpar}
  \inferrule* [lab=structure] {} {\meaningof{0} = \{ P \in \pi | P \equiv 0 \}, \and \\ \meaningof{E_1 | E_2} = \{ P \in \pi | P \equiv P_{1} | P_{2}, P_{1} \in \meaningof{E_{1}}, P_{2} \in \meaningof{E_2}\} }
\end{mathpar}

\begin{mathpar}
 \inferrule* [lab=behavior] {} {\meaningof{\langle a?b \rangle E} = \{ P \in \pi | P \equiv Q | u?(y)P', \\ \and \\\\ \and \\ \;\;\; u \in \meaningof{a}, \forall z.P'\{z/y\} \in \meaningof{E\{z/b\}}\}, \and \\ \meaningof{a!E} = \{ P \in \pi | P \equiv Q | x!\langle P' \rangle, x \in \meaningof{a} P' \in \meaningof{E}\} }
\end{mathpar}

\begin{mathpar}
 \inferrule* [lab=nominal] {} {\meaningof{\quotep{E}} = \{ \quotep{P} \in \quotep{\pi} | P \in \meaningof{E} \}, \and \meaningof{\quotep{P}} = \{ \quotep{Q} \in \quotep{\pi} | P \equiv Q \} \and \\ \meaningof{@\quotep{E}} = \{ P \in \pi | P \equiv @x, x \in \meaningof{E} \}}
\end{mathpar}

\begin{eqnarray*}
  \\
  \meaningof{-} : TS \to ST
\end{eqnarray*}

\begin{eqnarray*}
  \\
  L : TS \to ST
\end{eqnarray*}

\begin{eqnarray*}
  \\
  P \models E \iff P \in \meaningof{E}
\end{eqnarray*}

\begin{eqnarray*}
  P \approx_{L} Q \iff \forall E \in L. P \models E \iff Q \models E
\end{eqnarray*}

\begin{eqnarray*}
  P \approx_{K} Q
\end{eqnarray*}

\begin{eqnarray*}
  P \approx Q
\end{eqnarray*}

$\approx_{K} = \approx = \approx_{L}$

\subsubsection{Contextual duality}

Note that contexts extend the quotation operation to a family of
operations from processes to names. Given a context, $M$, we can
define a \emph{nominal context}, $\quotep{M}$ by $\quotep{M}[P] :=
\quotep{M[P]}$. To foreshadow what is to come we observe that these
operations enjoy a duality with processes very much like the duality
between vectors and maps from vectors to scalars.

Further, because the calculus is essentially higher-order, we have a
correspondence between contexts and processes. More specifically,
given a name $x$ and a context $M$ we can construct $M^{*}_{x}$ such
that 

\begin{mathpar}
  M^{*}_{x} | \lift{x}{P} \red M[P]
\end{mathpar}

namely,

\begin{mathpar}
  M^{*}_{x} := x?(u).M[\dropn{u}]
\end{mathpar}

The dependence of $M^{*}_{x}$ on a name makes it an abstraction, 

\begin{mathpar}
  M^{*} := (x)x?(u).M[\dropn{u}]
\end{mathpar}

\subsection{Additional notation}

It will sometimes be convenient to denote the process a name
quotes. We already have the notation $x = \quotep{P}$, but it will be
convenient to introduce an alternate notation, $\procn{x}$, when we
want to emphasize the connection to the use of the name. Note that, by
virtue of name equivalence, $\quotep{\procn{x}} \nameeq x$; so, the
notation is consistent with previous definitions.

Further, because names have structure it is possible to effect
substitutions on the basis of that structure. This means we need to
upgrade our notation for substitutions, which we accomplish by
adapting comprehension notation. Thus,

\begin{mathpar}
  P\{ y / x : x \in S \}
\end{mathpar}

is interpreted to mean the process derived from P by replacing (in a
capture-avoiding manner) each occurrence of $x$ in $S$ by $y$. For example,

\begin{mathpar}
  P\{ \quotep{\procn{x}|\procn{x}} / x : x \in \freenames{P} \}
\end{mathpar}

will replace each (occurrence) of a free name $x$ in $P$ by
$\quotep{\procn{x}|\procn{x}}$.

Also, we will avail ourselves of the notation $x^{L}$ and $x^{R}$ to
denote injections of a name into disjoint copies of the name
space. There are numerous ways to accomplish this. One example can be
found in \cite{MeredithR05}. This notation overloads to vectors of
names: $\vec{x}^{\pi} := (x_{i}^{\pi} \; : \; 0 \leq i < |\vec{x}| )$ where $\pi \in \{L,R\}$.

We also use $P^{\Box} := P|\Box$.

In \cite{MeredithR05} an interpretation of the new operator is
given. It turns out that there are several possible interpretations
all enjoying the requisite algebraic properties of the operator (see
\cite{milner91polyadicpi}). We will therefore make liberal use of
$(\nu\; \vec{x})P$.

% subsection the_syntax_and_semantics_of_the_notation_system (end)   

\input{qm2pi.qmops} 

\input{qm2pi.sterngerlach} 

\input{qm2pi.metric} 

% section concurrent_process_calculi (end)

%\input{qm2pi.proofsketch}

% section proof sketch (end)

%\input{qm2pi.slviaknots} 

% section spatial logic via knots (end)

\input{qm2pi.conclusion}

% section conclusion (end)

%\input{qm2pi.dtcodes} 

% section wiring algorithm (end)

\input{qm2pi.ack} 

% section acknowledgments (end)

\newpage


\bibliographystyle{plain}   
\bibliography{../../biblios/main.bib}

\input{qm2pi.rhodetails}

\end{document}

 

%\documentclass[12pt]{llncs}
%\documentclass{jktr}

\usepackage[pdftex]{hyperref}                   
\usepackage {listings}
\usepackage {mathpartir}
\usepackage{bcprules}
%\usepackage{listings}
                       
\usepackage{graphicx} 
%\usepackage[margins=2.5cm,nohead,nofoot]{geometry}
%\usepackage{geometry}
\usepackage{amsfonts}
\usepackage{amstext}
\usepackage{latexsym}
\usepackage{amssymb}
\usepackage{color}


%\include{myPreamble}
\include{qm2pi.local} 

%\ifpdf
%\usepackage[pdftex]{graphicx}
%\else
%\usepackage{graphicx}
%\fi

 % \ifpdf
%  \usepackage{pdfsync}
%  \if


%\title{Brief Article}
%\author{David F. Snyder}
%\author{L.G. Meredith}

%\address{Dept. of Math., Texas State University--San Marcos, San Marcos, TX 78666}
       
\pagestyle{empty}


\begin{document}

\lstset{language=[Objective]Caml,frame=shadowbox}

\input{qm2pi.front}

% section front matter (end)

\input{qm2pi.intro} 
 
% section introduction (end)

% \input{qm2pi.knotations} 

% section notation (end)

\input{qm2pi.process.calculi} 

% section concurrent_process_calculi_and_spatial_logics_ (end)
    
%\input{qm2pi.knots2pi} 

%\input{qm2pi.trefoil} 

%\input{qm2pi.mainthm} 

% subsection basic_interpretation (end)

%\input{qm2pi.rho.presentation} 
\subsection{The syntax and semantics of the notation system}\label{sub:the_syntax_and_semantics_of_the_notation_system} % (fold)

We now summarize a technical presentation of the calculus that
embodies our theory of dynamics. The typical presentation of such a
calculus follows the style of giving generators and relations on
them. The grammar, below, describing term constructors, freely
generates the set of processes, $\Proc$. This set is then quotiented
by a relation known as structural congruence and it is over this set
that the notion of dynamics is expressed. This presentation is
essentially that of \cite{MeredithR05} with the addition of
polyadicity and summation. For readability we have relegated some of
the technical subtleties to an appendix.

\subsubsection{Process grammar}\label{subsub:process_grammar}

\begin{mathpar}
  \inferrule* [lab=synchronization] {} {{M} \bc \pzero \;|\; x?F \;|\; x!C }
  \and
  \inferrule* [lab=abstraction] {} {{F} \bc (x)P}
  \and
  \inferrule* [lab=concretion] {} {{C} \bc \langle Q \rangle}
  \and
  \inferrule* [lab=process] {} {{P,Q} \bc M \;| \;P|Q \;|\; @{x}}
  \and
  \inferrule* [lab=name] {} {{x} \bc \quotep{P}}
\end{mathpar} 

Note that $\vec{x}$ (resp. $\vec{P}$) denotes a vector of names
(resp. processes) of length $|\vec{x}|$ (resp. $|\vec{P}|$). We adopt
the following useful abbreviations.

\begin{mathpar}
   x?(\vec{y}).P := x.(\vec{y})P \and  x\clift{\vec{P}} := x.\clift{\vec{P}}
   \and x!(y) := \lift{x}{\dropn{y}}
   \and \Pi_{i=0}^{n-1}P_i := P_0 | \ldots | P_{n-1}
\end{mathpar}

\subsubsection{Structural congruence}

\paragraph{Free and bound names and alpha-equivalence.} At the
core of structural equivalence is alpha-equivalence which identifies
process that are the same up to a change of variable. Formally, we
recognize the distinction between free and bound names. The free names
of a process, $\freenames{P}$, may be calculated recursively as
follows:

\begin{mathpar}
\freenames{\pzero} := \emptyset
  \and \\
  \freenames{x?(y).P} := \{ x \} \cup (\freenames{P} \setminus \{ y \})
  \and 
  \freenames{x!\langle P \rangle} := \{ x \} \cup \{ P \} 
  \and \\
  \freenames{P|Q} := \freenames{P} \cup \freenames{Q}
  \and \\
  \freenames{@{x}} := \{ x \}
\end{mathpar}

$\pi$
$\quotep{\pi}$

$\freenames{-} : \pi \to \mathcal{P}(\quotep{\pi})$

\begin{eqnarray*}
  \freenames{\pzero} & := & \emptyset \\
  \freenames{x?(y).P} & := & \{ x \} \cup (\freenames{P} \setminus \{ y \}) \\
  \freenames{x!\langle P \rangle} & := & \{ x \} \cup \{ P \} \\
  \freenames{P|Q} & := & \freenames{P} \cup \freenames{Q} \\
  \freenames{\dropn{x}} & := & \{ x \}
\end{eqnarray*}

The bound names of a process, $\boundnames{P}$, are those names occurring in $P$
that are not free. For example, in $x?(y).0$, the name $x$ is free, while $y$ is bound.

\begin{mathpar}
  \inferrule* [lab=monoidal-laws] {} { P|Q \equiv Q|P \and P|0 \equiv P \and P|(Q|R) \equiv (P|Q)|R }
\end{mathpar}

\begin{mathpar}
  \inferrule* [lab=alpha-equivalence] {} { (x)P \equiv (y)P\{y/x\} \and y \not\in \freenames{P} }
\end{mathpar}

\begin{definition}
Then two processes, $P,Q$, are alpha-equivalent if $P = Q\{\vec{y}/\vec{x}\}$ for
some $\vec{x} \in \boundnames{Q},\vec{y} \in \boundnames{P}$, where $Q\{\vec{y}/\vec{x}\}$
denotes the capture-avoiding substitution of $\vec{y}$ for $\vec{x}$ in $Q$.
\end{definition}

\begin{definition}
  The {\em structural congruence} \cite{SangiorgiWalker} , $\equiv$,
  between processes is the least congruence containing
  alpha-equivalence, satisfying the abelian monoid laws
  (associativity, commutativity and $\pzero$ as identity) for parallel
  composition $|$ and for summation $+$.
\end{definition}

\subsection{Name equivalence}

We take name equivalence, written $\nameeq$, to be the smallest
equivalence relation generated by the following rules.

\begin{mathpar}
\inferrule*[lab=Quote-drop]
{ }
{ \quotep{@{x}} \nameeq x }

\inferrule*[lab=Struct-equiv]
{ P \scong Q }
{ \quotep{P} \nameeq \quotep{Q} }
\end{mathpar}

The astute reader will have noticed that the mutual recursion of names
and processes imposes a mutual recursion on alpha-equivalence and
structural equivalence via name-equivalence. Fortunately, all of this
works out pleasantly and we may calculate in the natural way, free of
concern. The reader interested in the details is referred to the
appendix \ref{appendix:rho_details}.

\subsection{Substitution}

We use $\Proc$ for the set of processes, $\QProc$ for the set of
names, and $\id{\{}\vec{y} / \vec{x} \id{\}}$ to denote partial maps,
$s : \QProc \rightarrow \QProc$. A map, $s$ lifts, uniquely, to a map
on process terms, $\widehat{s} : \Proc \rightarrow \Proc$ by the
following equations.

\begin{mathpar}
  (0) \psubstp{Q}{P} := 0 \\
  (R \juxtap S) \psubstp{Q}{P}
  :=    
  (R)\psubstp{Q}{P} \juxtap (S) \psubstp{Q}{P} \\
  (x?(y).R) \psubstp{Q}{P}    
  :=    
  (x)\substp{Q}{P} (z)\concat( (R \psubstn{z}{y}) \psubstp{Q}{P} ) \\
  (\lift{x}{R}) \psubstp{Q}{P}  
  :=
  \lift{(x)\substp{Q}{P}}{ R \psubstp{Q}{P} } \\
%   (\dropn{x})  \psubstp{Q}{P}       
%   := 
%   \left\{ 
%     \begin{array}{ccc} 
%       \dropn{\quotep{Q}} & & x \nameeq \quotep{P} \\
%       \dropn{x} & & otherwise \\
%     \end{array}
%   \right. 
  (\dropn{x})  \psubstp{Q}{P}       
  := 
  \left\{ 
    \begin{array}{ccc} 
      Q & & x \nameeq \quotep{P} \\
      \dropn{x} & & otherwise \\
    \end{array}
  \right.
\end{mathpar}
 

where

\begin{eqnarray}
  (x)\id{\{} \lpquote Q \rpquote / \lpquote P \rpquote \id{\}}            = 
  \left\{ 
    \begin{array}{ccc}
      \lpquote Q \rpquote & & x \nameeq \lpquote P \rpquote \\
      x & & otherwise \\
    \end{array}
  \right. \nonumber
\end{eqnarray}

and $z$ is chosen distinct from $\quotep{P}$, $\quotep{Q}$, the free
names in $Q$, and all the names in $R$. Our $\alpha$-equivalence will
be built in the standard way from this substitution.

\begin{remark}\label{rem:no_self_referential_names}
  One consequence of these definitions is that $\forall P. \quotep{P}
  \not\in \freenames{P}$.
\end{remark}

\subsection{ Dynamic quote: an example }

Anticipating something of what's to come, consider applying the
substitution, $\widehat{\id{\{}u / z \id{\}}}$, to the following pair
of processes, $\lift{w}{y!(z)}$ and $w[ \lpquote y!(z) \rpquote ]$.

\begin{eqnarray}
	\lift{w}{y!(z)}\widehat{\id{\{}u / z \id{\}}}
		& = &
		\lift{w}{y!(u)} \nonumber\\
	w[ \lpquote y!(z) \rpquote ] \widehat{ \id{\{}u / z \id{\}} }
		& = &
		w[ \lpquote y!(z) \rpquote ] \nonumber
\end{eqnarray}

Because the body of the process between quotes is impervious to
substitution, we get radically different answers. In fact, by
examining the first process in an input context,
e.g. $x?(z).\lift{w}{y!(z)}$, we see that the process under the lift
operator may be shaped by prefixed inputs binding a name inside it. In
this sense, the lift operator will be seen as a way to dynamically
construct processes before reifying them as names.

Finally equipped with these standard features we can present the
dynamics of the calculus.

\subsubsection{Operational semantics} 

Finally, we introduce the computational dynamics. What marks these
algebras as distinct from other more traditionally studied algebraic
structures, e.g. vector spaces or polynomial rings, is the manner in
which dynamics is captured. In traditional structures, dynamics is typically
expressed through morphisms between such structures, as in linear maps
between vector spaces or morphisms between rings. In algebras
associated with the semantics of computation, the dynamics is
expressed as part of the algebraic structure itself, through a
reduction reduction relation typically denoted by $\red$. Below, we
give a recursive presentation of this relation for the calculus used
in the encoding.

$\red \subseteq \pi \times \pi$
$\red : \pi \to \mathcal{P}(\pi)$

\begin{mathpar}
  \inferrule* [lab=Comm] { \textsf{match}( x_{src}, x_{trgt} ) } { x_{trgt}?(y)P \; | \; x_{src}!\langle {Q} \rangle \red P\{\quotep{Q}/y}\} }
  \and \\
  \inferrule* [lab=Par] {{P} \red {P}'} {{{P} | {Q}} \red {{P}' | {Q}}}
  \and
  \inferrule* [lab=Equiv]{{{P} \scong {P}'} \andalso {{P}' \red {Q}'} \andalso {{Q}' \scong {Q}}}{{P} \red {Q}}
\end{mathpar}

\begin{eqnarray*}
  match_{\equiv} (\quotep{P},\quotep{Q}) & := & P \equiv Q \\
  match_{\dagger}(\quotep{P},\quotep{Q}) & := & \forall R. P|Q \red^{*} R => R \red^{*} 0 \\
  match_{K}(\quotep{P},\quotep{Q}) & := & K \mbox{ for some context } K
\end{eqnarray*}

$u?(x)P | u!\langle Q \rangle \red P\{\quotep{Q}/x\}$

%We write $\wred$ for $\red^*$, and $P\red$ if $\exists Q $ such that $ P \red Q$.
We write $P\red$ if $\exists Q $ such that $ P \red Q$ and $P\not\red$, otherwise.

\section{Replication}

As mentioned before, it is known that replication (and hence
recursion) can be implemented in a higher-order process algebra
\cite{SangiorgiWalker}. As our first example of calculation with the
machinery thus far presented we give the construction explicitly in
the {\rhoc}.

\begin{eqnarray}
	D_{x} & := & \prefix{x}{y}{(\binpar{\outputp{x}{y}}{@{y}})} \nonumber\\
	\bangp_{x}{P} & := & \binpar{{x}!\langle{\binpar{D_{x}}{P}}\rangle}{D_{x}} \nonumber
\end{eqnarray}

\begin{eqnarray}
	\bangp_{x}{P} & & \nonumber\\
	=
	& {x}!\langle{(\prefix{x}{y}{(\outputp{x}{y} | @{y})) | P}}\rangle 
	      | \prefix{x}{y}{(\outputp{x}{y} | @{y})} & \nonumber\\
	\red
	& (\outputp{x}{y} | @{y})\substn{\quotep{(\prefix{x}{y}{(@{y} | \outputp{x}{y})) | P}}}{y} & \nonumber\\
	=
	& \outputp{x}{\quotep{(\prefix{x}{y}{(\outputp{x}{y} | @{y})) | P}}}
	  | {(\prefix{x}{y}{(\outputp{x}{y} | @{y})) | P}} & \nonumber\\
	\red
	& \ldots & \nonumber\\
	\red^*
	& P | P | \ldots & \nonumber
\end{eqnarray}

Of course, this encoding, as an implementation, runs away, unfolding
$\bangp{P}$ eagerly. A lazier and more implementable replication
operator, restricted to input-guarded processes, may be obtained as follows.

\begin{eqnarray}
\bangp{\prefix{u}{v}{P}} 
	:= 
	\binpar{\lift{x}{\prefix{u}{v}{(\binpar{D(x)}{P})}}}{D(x)} \nonumber
\end{eqnarray}

\begin{remark}
  Note that the lazier definition still does not deal with summation
  or mixed summation (i.e. sums over input and output). The reader is
  invited to construct definitions of replication that deal with these
  features. 

  Further, the definitions are parameterized in a name, $x$. Can you,
  gentle reader, make a definition that eliminates this parameter and
  guarantees no accidental interaction between the replication
  machinery and the process being replicated -- i.e. no accidental
  sharing of names used by the process to get its work done and the
  name(s) used by the replication to effect copying. This latter
  revision of the definition of replication is crucial to obtaining
  the expected identity $!!P \sim !P$.
\end{remark}

\begin{remark}\label{rem:paradoxical_combinator}
  The reader familiar with the lambda calculus will have noticed the
  similarity between $D$ and the paradoxical combinator.

  [Ed. note: the existence of this seems to suggest we have to be more
  restrictive on the set of processes and names we admit if we are to
  support no-cloning.]
\end{remark}

\subsubsection{Bisimulation}

The computational dynamics gives rise to another kind of equivalence,
the equivalence of computational behavior. As previously mentioned
this is typically captured \emph{via} some form of bisimulation.

% The notion we use in this paper is weak barbed bisimulation
% \cite{milner91polyadicpi}.

The notion we use in this paper is derived from weak barbed
bisimulation \cite{milner91polyadicpi}. 

\begin{definition}
An \emph{observation relation}, $\downarrow_{\mathcal N}$, over a set
of names, $\mathcal N$, is the smallest relation satisfying the rules
below.

\infrule[Out-barb]{y \in {\mathcal N}, \; x \nameeq y}
		  {\outputp{x}{v} \downarrow_{\mathcal N} x}
\infrule[Par-barb]{\mbox{$P\downarrow_{\mathcal N} x$ or $Q\downarrow_{\mathcal N} x$}}
		  {\binpar{P}{Q} \downarrow_{\mathcal N} x}

We write $P \Downarrow_{\mathcal N} x$ if there is $Q$ such that 
$P \wred Q$ and $Q \downarrow_{\mathcal N} x$.
\end{definition}

\begin{definition}
%\label{def.bbisim}
An  ${\mathcal N}$-\emph{barbed bisimulation} over a set of names, ${\mathcal N}$, is a symmetric binary relation 
${\mathcal S}_{\mathcal N}$ between agents such that $P\rel{S}_{\mathcal N}Q$ implies:
\begin{enumerate}
\item If $P \red P'$ then $Q \wred Q'$ and $P'\rel{S}_{\mathcal N} Q'$.
\item If $P\downarrow_{\mathcal N} x$, then $Q\Downarrow_{\mathcal N} x$.
\end{enumerate}
$P$ is ${\mathcal N}$-barbed bisimilar to $Q$, written
$P \wbbisim_{\mathcal N} Q$, if $P \rel{S}_{\mathcal N} Q$ for some ${\mathcal N}$-barbed bisimulation ${\mathcal S}_{\mathcal N}$.
\end{definition}

$\mathcal{R} \subseteq \pi \times \pi$

$P \mathcal{R} Q => \forall P'. P \red P' \Rightarrow \exists Q'. Q \red Q', P' \mathcal{R} Q'$

$P \vdash x \Rightarrow Q \vdash x$

\begin{mathpar}
  \inferrule*[lab=Out-barb]{x \nameeq y}{{y}!\langle{Q}\rangle \vdash x}
  \and
  \inferrule*[lab=Par-barb]{\mbox{$P\vdash x$ or $Q\vdash x$}}{\binpar{P}{Q} \vdash x}
\end{mathpar}

\subsubsection{Contexts}

One of the principle advantages of computational calculi like the
$\pi$-calculus is a well-defined notion of context,
contextual-equivalence and a correlation between
contextual-equivalence and notions of bisimulation. The notion of
context allows the decomposition of a process into (sub-)process and
its syntactic environment, its context. Thus, a context may be
thought of as a process with a ``hole'' (written $\Box$) in it. The
application of a context $M$ to a process $P$, written $M[P]$, is
tantamount to filling the hole in $M$ with $P$. In this paper we do
not need the full weight of this theory, but do make use of the notion
of context in the proof the main theorem. 

\begin{mathpar}
  \inferrule* [lab=summation] {} {{M_{M},M_{N}} \bc \Box \;|\; x.M_{A} \;|\; M_{M}+M_{N}}
  \and
  \inferrule* [lab=agent] {} {{M_{A}} \bc (\vec{x})M_{P} \;| \; \clift{P_0,\ldots,M_{P},\ldots,P_N}}
  \and \\
  \inferrule* [lab=process] {} {{M_{P}} \bc M_{N} \;| \;P|M_{P} }
\end{mathpar} 

\begin{mathpar}
  \inferrule* [lab=sychronization] {} {M_{N} \bc \Box \;|\; x?M_{F} \;|\; x!M_{C}}
  \and
  \inferrule* [lab=abstraction] {} {{M_{F}} \bc (x)M_{P} }
  \and
  \inferrule* [lab=concretion] {} {{M_{C}} \bc \langle M_{P} \rangle }
  \and \\
  \inferrule* [lab=process] {} {{M_{P}} \bc M_{N} \;| \;P|M_{P} }
\end{mathpar}

\begin{definition}[contextual application] Given a context $M$, and
  process $P$, we define the \emph{contextual application}, $M[P] :=
  M\{P/\Box\}$. That is, the contextual application of M to P is the
  substitution of $P$ for $\Box$ in $M$.
\end{definition}

$\meaningof{-} : L \to \mathcal{P}(\pi)$

\begin{mathpar}
  \inferrule* [lab=collection] {} {\meaningof{true} = \pi, \and \meaningof{~E} = \pi \setminus \meaningof{E}, \and \meaningof{E_{1} \& E_{2}} = \meaningof{E_{1}} \cap \meaningof{E_{2}}}
\end{mathpar}

\begin{mathpar}
  \inferrule* [lab=structure] {} {\meaningof{0} = \{ P \in \pi | P \equiv 0 \}, \and \\ \meaningof{E_1 | E_2} = \{ P \in \pi | P \equiv P_{1} | P_{2}, P_{1} \in \meaningof{E_{1}}, P_{2} \in \meaningof{E_2}\} }
\end{mathpar}

\begin{mathpar}
 \inferrule* [lab=behavior] {} {\meaningof{\langle a?b \rangle E} = \{ P \in \pi | P \equiv Q | u?(y)P', \\ \and \\\\ \and \\ \;\;\; u \in \meaningof{a}, \forall z.P'\{z/y\} \in \meaningof{E\{z/b\}}\}, \and \\ \meaningof{a!E} = \{ P \in \pi | P \equiv Q | x!\langle P' \rangle, x \in \meaningof{a} P' \in \meaningof{E}\} }
\end{mathpar}

\begin{mathpar}
 \inferrule* [lab=nominal] {} {\meaningof{\quotep{E}} = \{ \quotep{P} \in \quotep{\pi} | P \in \meaningof{E} \}, \and \meaningof{\quotep{P}} = \{ \quotep{Q} \in \quotep{\pi} | P \equiv Q \} \and \\ \meaningof{@\quotep{E}} = \{ P \in \pi | P \equiv @x, x \in \meaningof{E} \}}
\end{mathpar}

\begin{eqnarray*}
  \\
  \meaningof{-} : TS \to ST
\end{eqnarray*}

\begin{eqnarray*}
  \\
  L : TS \to ST
\end{eqnarray*}

\begin{eqnarray*}
  \\
  P \models E \iff P \in \meaningof{E}
\end{eqnarray*}

\begin{eqnarray*}
  P \approx_{L} Q \iff \forall E \in L. P \models E \iff Q \models E
\end{eqnarray*}

\begin{eqnarray*}
  P \approx_{K} Q
\end{eqnarray*}

\begin{eqnarray*}
  P \approx Q
\end{eqnarray*}

$\approx_{K} = \approx = \approx_{L}$

\subsubsection{Contextual duality}

Note that contexts extend the quotation operation to a family of
operations from processes to names. Given a context, $M$, we can
define a \emph{nominal context}, $\quotep{M}$ by $\quotep{M}[P] :=
\quotep{M[P]}$. To foreshadow what is to come we observe that these
operations enjoy a duality with processes very much like the duality
between vectors and maps from vectors to scalars.

Further, because the calculus is essentially higher-order, we have a
correspondence between contexts and processes. More specifically,
given a name $x$ and a context $M$ we can construct $M^{*}_{x}$ such
that 

\begin{mathpar}
  M^{*}_{x} | \lift{x}{P} \red M[P]
\end{mathpar}

namely,

\begin{mathpar}
  M^{*}_{x} := x?(u).M[\dropn{u}]
\end{mathpar}

The dependence of $M^{*}_{x}$ on a name makes it an abstraction, 

\begin{mathpar}
  M^{*} := (x)x?(u).M[\dropn{u}]
\end{mathpar}

\subsection{Additional notation}

It will sometimes be convenient to denote the process a name
quotes. We already have the notation $x = \quotep{P}$, but it will be
convenient to introduce an alternate notation, $\procn{x}$, when we
want to emphasize the connection to the use of the name. Note that, by
virtue of name equivalence, $\quotep{\procn{x}} \nameeq x$; so, the
notation is consistent with previous definitions.

Further, because names have structure it is possible to effect
substitutions on the basis of that structure. This means we need to
upgrade our notation for substitutions, which we accomplish by
adapting comprehension notation. Thus,

\begin{mathpar}
  P\{ y / x : x \in S \}
\end{mathpar}

is interpreted to mean the process derived from P by replacing (in a
capture-avoiding manner) each occurrence of $x$ in $S$ by $y$. For example,

\begin{mathpar}
  P\{ \quotep{\procn{x}|\procn{x}} / x : x \in \freenames{P} \}
\end{mathpar}

will replace each (occurrence) of a free name $x$ in $P$ by
$\quotep{\procn{x}|\procn{x}}$.

Also, we will avail ourselves of the notation $x^{L}$ and $x^{R}$ to
denote injections of a name into disjoint copies of the name
space. There are numerous ways to accomplish this. One example can be
found in \cite{MeredithR05}. This notation overloads to vectors of
names: $\vec{x}^{\pi} := (x_{i}^{\pi} \; : \; 0 \leq i < |\vec{x}| )$ where $\pi \in \{L,R\}$.

We also use $P^{\Box} := P|\Box$.

In \cite{MeredithR05} an interpretation of the new operator is
given. It turns out that there are several possible interpretations
all enjoying the requisite algebraic properties of the operator (see
\cite{milner91polyadicpi}). We will therefore make liberal use of
$(\nu\; \vec{x})P$.

% subsection the_syntax_and_semantics_of_the_notation_system (end)   

\input{qm2pi.qmops} 

\input{qm2pi.sterngerlach} 

\input{qm2pi.metric} 

% section concurrent_process_calculi (end)

%\input{qm2pi.proofsketch}

% section proof sketch (end)

%\input{qm2pi.slviaknots} 

% section spatial logic via knots (end)

\input{qm2pi.conclusion}

% section conclusion (end)

%\input{qm2pi.dtcodes} 

% section wiring algorithm (end)

\input{qm2pi.ack} 

% section acknowledgments (end)

\newpage


\bibliographystyle{plain}   
\bibliography{../../biblios/main.bib}

\input{qm2pi.rhodetails}

\end{document}

 

%\documentclass[12pt]{llncs}
%\documentclass{jktr}

\usepackage[pdftex]{hyperref}                   
\usepackage {listings}
\usepackage {mathpartir}
\usepackage{bcprules}
%\usepackage{listings}
                       
\usepackage{graphicx} 
%\usepackage[margins=2.5cm,nohead,nofoot]{geometry}
%\usepackage{geometry}
\usepackage{amsfonts}
\usepackage{amstext}
\usepackage{latexsym}
\usepackage{amssymb}
\usepackage{color}


%\include{myPreamble}
\include{qm2pi.local} 

%\ifpdf
%\usepackage[pdftex]{graphicx}
%\else
%\usepackage{graphicx}
%\fi

 % \ifpdf
%  \usepackage{pdfsync}
%  \if


%\title{Brief Article}
%\author{David F. Snyder}
%\author{L.G. Meredith}

%\address{Dept. of Math., Texas State University--San Marcos, San Marcos, TX 78666}
       
\pagestyle{empty}


\begin{document}

\lstset{language=[Objective]Caml,frame=shadowbox}

\input{qm2pi.front}

% section front matter (end)

\input{qm2pi.intro} 
 
% section introduction (end)

% \input{qm2pi.knotations} 

% section notation (end)

\input{qm2pi.process.calculi} 

% section concurrent_process_calculi_and_spatial_logics_ (end)
    
%\input{qm2pi.knots2pi} 

%\input{qm2pi.trefoil} 

%\input{qm2pi.mainthm} 

% subsection basic_interpretation (end)

%\input{qm2pi.rho.presentation} 
\subsection{The syntax and semantics of the notation system}\label{sub:the_syntax_and_semantics_of_the_notation_system} % (fold)

We now summarize a technical presentation of the calculus that
embodies our theory of dynamics. The typical presentation of such a
calculus follows the style of giving generators and relations on
them. The grammar, below, describing term constructors, freely
generates the set of processes, $\Proc$. This set is then quotiented
by a relation known as structural congruence and it is over this set
that the notion of dynamics is expressed. This presentation is
essentially that of \cite{MeredithR05} with the addition of
polyadicity and summation. For readability we have relegated some of
the technical subtleties to an appendix.

\subsubsection{Process grammar}\label{subsub:process_grammar}

\begin{mathpar}
  \inferrule* [lab=synchronization] {} {{M} \bc \pzero \;|\; x?F \;|\; x!C }
  \and
  \inferrule* [lab=abstraction] {} {{F} \bc (x)P}
  \and
  \inferrule* [lab=concretion] {} {{C} \bc \langle Q \rangle}
  \and
  \inferrule* [lab=process] {} {{P,Q} \bc M \;| \;P|Q \;|\; @{x}}
  \and
  \inferrule* [lab=name] {} {{x} \bc \quotep{P}}
\end{mathpar} 

Note that $\vec{x}$ (resp. $\vec{P}$) denotes a vector of names
(resp. processes) of length $|\vec{x}|$ (resp. $|\vec{P}|$). We adopt
the following useful abbreviations.

\begin{mathpar}
   x?(\vec{y}).P := x.(\vec{y})P \and  x\clift{\vec{P}} := x.\clift{\vec{P}}
   \and x!(y) := \lift{x}{\dropn{y}}
   \and \Pi_{i=0}^{n-1}P_i := P_0 | \ldots | P_{n-1}
\end{mathpar}

\subsubsection{Structural congruence}

\paragraph{Free and bound names and alpha-equivalence.} At the
core of structural equivalence is alpha-equivalence which identifies
process that are the same up to a change of variable. Formally, we
recognize the distinction between free and bound names. The free names
of a process, $\freenames{P}$, may be calculated recursively as
follows:

\begin{mathpar}
\freenames{\pzero} := \emptyset
  \and \\
  \freenames{x?(y).P} := \{ x \} \cup (\freenames{P} \setminus \{ y \})
  \and 
  \freenames{x!\langle P \rangle} := \{ x \} \cup \{ P \} 
  \and \\
  \freenames{P|Q} := \freenames{P} \cup \freenames{Q}
  \and \\
  \freenames{@{x}} := \{ x \}
\end{mathpar}

$\pi$
$\quotep{\pi}$

$\freenames{-} : \pi \to \mathcal{P}(\quotep{\pi})$

\begin{eqnarray*}
  \freenames{\pzero} & := & \emptyset \\
  \freenames{x?(y).P} & := & \{ x \} \cup (\freenames{P} \setminus \{ y \}) \\
  \freenames{x!\langle P \rangle} & := & \{ x \} \cup \{ P \} \\
  \freenames{P|Q} & := & \freenames{P} \cup \freenames{Q} \\
  \freenames{\dropn{x}} & := & \{ x \}
\end{eqnarray*}

The bound names of a process, $\boundnames{P}$, are those names occurring in $P$
that are not free. For example, in $x?(y).0$, the name $x$ is free, while $y$ is bound.

\begin{mathpar}
  \inferrule* [lab=monoidal-laws] {} { P|Q \equiv Q|P \and P|0 \equiv P \and P|(Q|R) \equiv (P|Q)|R }
\end{mathpar}

\begin{mathpar}
  \inferrule* [lab=alpha-equivalence] {} { (x)P \equiv (y)P\{y/x\} \and y \not\in \freenames{P} }
\end{mathpar}

\begin{definition}
Then two processes, $P,Q$, are alpha-equivalent if $P = Q\{\vec{y}/\vec{x}\}$ for
some $\vec{x} \in \boundnames{Q},\vec{y} \in \boundnames{P}$, where $Q\{\vec{y}/\vec{x}\}$
denotes the capture-avoiding substitution of $\vec{y}$ for $\vec{x}$ in $Q$.
\end{definition}

\begin{definition}
  The {\em structural congruence} \cite{SangiorgiWalker} , $\equiv$,
  between processes is the least congruence containing
  alpha-equivalence, satisfying the abelian monoid laws
  (associativity, commutativity and $\pzero$ as identity) for parallel
  composition $|$ and for summation $+$.
\end{definition}

\subsection{Name equivalence}

We take name equivalence, written $\nameeq$, to be the smallest
equivalence relation generated by the following rules.

\begin{mathpar}
\inferrule*[lab=Quote-drop]
{ }
{ \quotep{@{x}} \nameeq x }

\inferrule*[lab=Struct-equiv]
{ P \scong Q }
{ \quotep{P} \nameeq \quotep{Q} }
\end{mathpar}

The astute reader will have noticed that the mutual recursion of names
and processes imposes a mutual recursion on alpha-equivalence and
structural equivalence via name-equivalence. Fortunately, all of this
works out pleasantly and we may calculate in the natural way, free of
concern. The reader interested in the details is referred to the
appendix \ref{appendix:rho_details}.

\subsection{Substitution}

We use $\Proc$ for the set of processes, $\QProc$ for the set of
names, and $\id{\{}\vec{y} / \vec{x} \id{\}}$ to denote partial maps,
$s : \QProc \rightarrow \QProc$. A map, $s$ lifts, uniquely, to a map
on process terms, $\widehat{s} : \Proc \rightarrow \Proc$ by the
following equations.

\begin{mathpar}
  (0) \psubstp{Q}{P} := 0 \\
  (R \juxtap S) \psubstp{Q}{P}
  :=    
  (R)\psubstp{Q}{P} \juxtap (S) \psubstp{Q}{P} \\
  (x?(y).R) \psubstp{Q}{P}    
  :=    
  (x)\substp{Q}{P} (z)\concat( (R \psubstn{z}{y}) \psubstp{Q}{P} ) \\
  (\lift{x}{R}) \psubstp{Q}{P}  
  :=
  \lift{(x)\substp{Q}{P}}{ R \psubstp{Q}{P} } \\
%   (\dropn{x})  \psubstp{Q}{P}       
%   := 
%   \left\{ 
%     \begin{array}{ccc} 
%       \dropn{\quotep{Q}} & & x \nameeq \quotep{P} \\
%       \dropn{x} & & otherwise \\
%     \end{array}
%   \right. 
  (\dropn{x})  \psubstp{Q}{P}       
  := 
  \left\{ 
    \begin{array}{ccc} 
      Q & & x \nameeq \quotep{P} \\
      \dropn{x} & & otherwise \\
    \end{array}
  \right.
\end{mathpar}
 

where

\begin{eqnarray}
  (x)\id{\{} \lpquote Q \rpquote / \lpquote P \rpquote \id{\}}            = 
  \left\{ 
    \begin{array}{ccc}
      \lpquote Q \rpquote & & x \nameeq \lpquote P \rpquote \\
      x & & otherwise \\
    \end{array}
  \right. \nonumber
\end{eqnarray}

and $z$ is chosen distinct from $\quotep{P}$, $\quotep{Q}$, the free
names in $Q$, and all the names in $R$. Our $\alpha$-equivalence will
be built in the standard way from this substitution.

\begin{remark}\label{rem:no_self_referential_names}
  One consequence of these definitions is that $\forall P. \quotep{P}
  \not\in \freenames{P}$.
\end{remark}

\subsection{ Dynamic quote: an example }

Anticipating something of what's to come, consider applying the
substitution, $\widehat{\id{\{}u / z \id{\}}}$, to the following pair
of processes, $\lift{w}{y!(z)}$ and $w[ \lpquote y!(z) \rpquote ]$.

\begin{eqnarray}
	\lift{w}{y!(z)}\widehat{\id{\{}u / z \id{\}}}
		& = &
		\lift{w}{y!(u)} \nonumber\\
	w[ \lpquote y!(z) \rpquote ] \widehat{ \id{\{}u / z \id{\}} }
		& = &
		w[ \lpquote y!(z) \rpquote ] \nonumber
\end{eqnarray}

Because the body of the process between quotes is impervious to
substitution, we get radically different answers. In fact, by
examining the first process in an input context,
e.g. $x?(z).\lift{w}{y!(z)}$, we see that the process under the lift
operator may be shaped by prefixed inputs binding a name inside it. In
this sense, the lift operator will be seen as a way to dynamically
construct processes before reifying them as names.

Finally equipped with these standard features we can present the
dynamics of the calculus.

\subsubsection{Operational semantics} 

Finally, we introduce the computational dynamics. What marks these
algebras as distinct from other more traditionally studied algebraic
structures, e.g. vector spaces or polynomial rings, is the manner in
which dynamics is captured. In traditional structures, dynamics is typically
expressed through morphisms between such structures, as in linear maps
between vector spaces or morphisms between rings. In algebras
associated with the semantics of computation, the dynamics is
expressed as part of the algebraic structure itself, through a
reduction reduction relation typically denoted by $\red$. Below, we
give a recursive presentation of this relation for the calculus used
in the encoding.

$\red \subseteq \pi \times \pi$
$\red : \pi \to \mathcal{P}(\pi)$

\begin{mathpar}
  \inferrule* [lab=Comm] { \textsf{match}( x_{src}, x_{trgt} ) } { x_{trgt}?(y)P \; | \; x_{src}!\langle {Q} \rangle \red P\{\quotep{Q}/y}\} }
  \and \\
  \inferrule* [lab=Par] {{P} \red {P}'} {{{P} | {Q}} \red {{P}' | {Q}}}
  \and
  \inferrule* [lab=Equiv]{{{P} \scong {P}'} \andalso {{P}' \red {Q}'} \andalso {{Q}' \scong {Q}}}{{P} \red {Q}}
\end{mathpar}

\begin{eqnarray*}
  match_{\equiv} (\quotep{P},\quotep{Q}) & := & P \equiv Q \\
  match_{\dagger}(\quotep{P},\quotep{Q}) & := & \forall R. P|Q \red^{*} R => R \red^{*} 0 \\
  match_{K}(\quotep{P},\quotep{Q}) & := & K \mbox{ for some context } K
\end{eqnarray*}

$u?(x)P | u!\langle Q \rangle \red P\{\quotep{Q}/x\}$

%We write $\wred$ for $\red^*$, and $P\red$ if $\exists Q $ such that $ P \red Q$.
We write $P\red$ if $\exists Q $ such that $ P \red Q$ and $P\not\red$, otherwise.

\section{Replication}

As mentioned before, it is known that replication (and hence
recursion) can be implemented in a higher-order process algebra
\cite{SangiorgiWalker}. As our first example of calculation with the
machinery thus far presented we give the construction explicitly in
the {\rhoc}.

\begin{eqnarray}
	D_{x} & := & \prefix{x}{y}{(\binpar{\outputp{x}{y}}{@{y}})} \nonumber\\
	\bangp_{x}{P} & := & \binpar{{x}!\langle{\binpar{D_{x}}{P}}\rangle}{D_{x}} \nonumber
\end{eqnarray}

\begin{eqnarray}
	\bangp_{x}{P} & & \nonumber\\
	=
	& {x}!\langle{(\prefix{x}{y}{(\outputp{x}{y} | @{y})) | P}}\rangle 
	      | \prefix{x}{y}{(\outputp{x}{y} | @{y})} & \nonumber\\
	\red
	& (\outputp{x}{y} | @{y})\substn{\quotep{(\prefix{x}{y}{(@{y} | \outputp{x}{y})) | P}}}{y} & \nonumber\\
	=
	& \outputp{x}{\quotep{(\prefix{x}{y}{(\outputp{x}{y} | @{y})) | P}}}
	  | {(\prefix{x}{y}{(\outputp{x}{y} | @{y})) | P}} & \nonumber\\
	\red
	& \ldots & \nonumber\\
	\red^*
	& P | P | \ldots & \nonumber
\end{eqnarray}

Of course, this encoding, as an implementation, runs away, unfolding
$\bangp{P}$ eagerly. A lazier and more implementable replication
operator, restricted to input-guarded processes, may be obtained as follows.

\begin{eqnarray}
\bangp{\prefix{u}{v}{P}} 
	:= 
	\binpar{\lift{x}{\prefix{u}{v}{(\binpar{D(x)}{P})}}}{D(x)} \nonumber
\end{eqnarray}

\begin{remark}
  Note that the lazier definition still does not deal with summation
  or mixed summation (i.e. sums over input and output). The reader is
  invited to construct definitions of replication that deal with these
  features. 

  Further, the definitions are parameterized in a name, $x$. Can you,
  gentle reader, make a definition that eliminates this parameter and
  guarantees no accidental interaction between the replication
  machinery and the process being replicated -- i.e. no accidental
  sharing of names used by the process to get its work done and the
  name(s) used by the replication to effect copying. This latter
  revision of the definition of replication is crucial to obtaining
  the expected identity $!!P \sim !P$.
\end{remark}

\begin{remark}\label{rem:paradoxical_combinator}
  The reader familiar with the lambda calculus will have noticed the
  similarity between $D$ and the paradoxical combinator.

  [Ed. note: the existence of this seems to suggest we have to be more
  restrictive on the set of processes and names we admit if we are to
  support no-cloning.]
\end{remark}

\subsubsection{Bisimulation}

The computational dynamics gives rise to another kind of equivalence,
the equivalence of computational behavior. As previously mentioned
this is typically captured \emph{via} some form of bisimulation.

% The notion we use in this paper is weak barbed bisimulation
% \cite{milner91polyadicpi}.

The notion we use in this paper is derived from weak barbed
bisimulation \cite{milner91polyadicpi}. 

\begin{definition}
An \emph{observation relation}, $\downarrow_{\mathcal N}$, over a set
of names, $\mathcal N$, is the smallest relation satisfying the rules
below.

\infrule[Out-barb]{y \in {\mathcal N}, \; x \nameeq y}
		  {\outputp{x}{v} \downarrow_{\mathcal N} x}
\infrule[Par-barb]{\mbox{$P\downarrow_{\mathcal N} x$ or $Q\downarrow_{\mathcal N} x$}}
		  {\binpar{P}{Q} \downarrow_{\mathcal N} x}

We write $P \Downarrow_{\mathcal N} x$ if there is $Q$ such that 
$P \wred Q$ and $Q \downarrow_{\mathcal N} x$.
\end{definition}

\begin{definition}
%\label{def.bbisim}
An  ${\mathcal N}$-\emph{barbed bisimulation} over a set of names, ${\mathcal N}$, is a symmetric binary relation 
${\mathcal S}_{\mathcal N}$ between agents such that $P\rel{S}_{\mathcal N}Q$ implies:
\begin{enumerate}
\item If $P \red P'$ then $Q \wred Q'$ and $P'\rel{S}_{\mathcal N} Q'$.
\item If $P\downarrow_{\mathcal N} x$, then $Q\Downarrow_{\mathcal N} x$.
\end{enumerate}
$P$ is ${\mathcal N}$-barbed bisimilar to $Q$, written
$P \wbbisim_{\mathcal N} Q$, if $P \rel{S}_{\mathcal N} Q$ for some ${\mathcal N}$-barbed bisimulation ${\mathcal S}_{\mathcal N}$.
\end{definition}

$\mathcal{R} \subseteq \pi \times \pi$

$P \mathcal{R} Q => \forall P'. P \red P' \Rightarrow \exists Q'. Q \red Q', P' \mathcal{R} Q'$

$P \vdash x \Rightarrow Q \vdash x$

\begin{mathpar}
  \inferrule*[lab=Out-barb]{x \nameeq y}{{y}!\langle{Q}\rangle \vdash x}
  \and
  \inferrule*[lab=Par-barb]{\mbox{$P\vdash x$ or $Q\vdash x$}}{\binpar{P}{Q} \vdash x}
\end{mathpar}

\subsubsection{Contexts}

One of the principle advantages of computational calculi like the
$\pi$-calculus is a well-defined notion of context,
contextual-equivalence and a correlation between
contextual-equivalence and notions of bisimulation. The notion of
context allows the decomposition of a process into (sub-)process and
its syntactic environment, its context. Thus, a context may be
thought of as a process with a ``hole'' (written $\Box$) in it. The
application of a context $M$ to a process $P$, written $M[P]$, is
tantamount to filling the hole in $M$ with $P$. In this paper we do
not need the full weight of this theory, but do make use of the notion
of context in the proof the main theorem. 

\begin{mathpar}
  \inferrule* [lab=summation] {} {{M_{M},M_{N}} \bc \Box \;|\; x.M_{A} \;|\; M_{M}+M_{N}}
  \and
  \inferrule* [lab=agent] {} {{M_{A}} \bc (\vec{x})M_{P} \;| \; \clift{P_0,\ldots,M_{P},\ldots,P_N}}
  \and \\
  \inferrule* [lab=process] {} {{M_{P}} \bc M_{N} \;| \;P|M_{P} }
\end{mathpar} 

\begin{mathpar}
  \inferrule* [lab=sychronization] {} {M_{N} \bc \Box \;|\; x?M_{F} \;|\; x!M_{C}}
  \and
  \inferrule* [lab=abstraction] {} {{M_{F}} \bc (x)M_{P} }
  \and
  \inferrule* [lab=concretion] {} {{M_{C}} \bc \langle M_{P} \rangle }
  \and \\
  \inferrule* [lab=process] {} {{M_{P}} \bc M_{N} \;| \;P|M_{P} }
\end{mathpar}

\begin{definition}[contextual application] Given a context $M$, and
  process $P$, we define the \emph{contextual application}, $M[P] :=
  M\{P/\Box\}$. That is, the contextual application of M to P is the
  substitution of $P$ for $\Box$ in $M$.
\end{definition}

$\meaningof{-} : L \to \mathcal{P}(\pi)$

\begin{mathpar}
  \inferrule* [lab=collection] {} {\meaningof{true} = \pi, \and \meaningof{~E} = \pi \setminus \meaningof{E}, \and \meaningof{E_{1} \& E_{2}} = \meaningof{E_{1}} \cap \meaningof{E_{2}}}
\end{mathpar}

\begin{mathpar}
  \inferrule* [lab=structure] {} {\meaningof{0} = \{ P \in \pi | P \equiv 0 \}, \and \\ \meaningof{E_1 | E_2} = \{ P \in \pi | P \equiv P_{1} | P_{2}, P_{1} \in \meaningof{E_{1}}, P_{2} \in \meaningof{E_2}\} }
\end{mathpar}

\begin{mathpar}
 \inferrule* [lab=behavior] {} {\meaningof{\langle a?b \rangle E} = \{ P \in \pi | P \equiv Q | u?(y)P', \\ \and \\\\ \and \\ \;\;\; u \in \meaningof{a}, \forall z.P'\{z/y\} \in \meaningof{E\{z/b\}}\}, \and \\ \meaningof{a!E} = \{ P \in \pi | P \equiv Q | x!\langle P' \rangle, x \in \meaningof{a} P' \in \meaningof{E}\} }
\end{mathpar}

\begin{mathpar}
 \inferrule* [lab=nominal] {} {\meaningof{\quotep{E}} = \{ \quotep{P} \in \quotep{\pi} | P \in \meaningof{E} \}, \and \meaningof{\quotep{P}} = \{ \quotep{Q} \in \quotep{\pi} | P \equiv Q \} \and \\ \meaningof{@\quotep{E}} = \{ P \in \pi | P \equiv @x, x \in \meaningof{E} \}}
\end{mathpar}

\begin{eqnarray*}
  \\
  \meaningof{-} : TS \to ST
\end{eqnarray*}

\begin{eqnarray*}
  \\
  L : TS \to ST
\end{eqnarray*}

\begin{eqnarray*}
  \\
  P \models E \iff P \in \meaningof{E}
\end{eqnarray*}

\begin{eqnarray*}
  P \approx_{L} Q \iff \forall E \in L. P \models E \iff Q \models E
\end{eqnarray*}

\begin{eqnarray*}
  P \approx_{K} Q
\end{eqnarray*}

\begin{eqnarray*}
  P \approx Q
\end{eqnarray*}

$\approx_{K} = \approx = \approx_{L}$

\subsubsection{Contextual duality}

Note that contexts extend the quotation operation to a family of
operations from processes to names. Given a context, $M$, we can
define a \emph{nominal context}, $\quotep{M}$ by $\quotep{M}[P] :=
\quotep{M[P]}$. To foreshadow what is to come we observe that these
operations enjoy a duality with processes very much like the duality
between vectors and maps from vectors to scalars.

Further, because the calculus is essentially higher-order, we have a
correspondence between contexts and processes. More specifically,
given a name $x$ and a context $M$ we can construct $M^{*}_{x}$ such
that 

\begin{mathpar}
  M^{*}_{x} | \lift{x}{P} \red M[P]
\end{mathpar}

namely,

\begin{mathpar}
  M^{*}_{x} := x?(u).M[\dropn{u}]
\end{mathpar}

The dependence of $M^{*}_{x}$ on a name makes it an abstraction, 

\begin{mathpar}
  M^{*} := (x)x?(u).M[\dropn{u}]
\end{mathpar}

\subsection{Additional notation}

It will sometimes be convenient to denote the process a name
quotes. We already have the notation $x = \quotep{P}$, but it will be
convenient to introduce an alternate notation, $\procn{x}$, when we
want to emphasize the connection to the use of the name. Note that, by
virtue of name equivalence, $\quotep{\procn{x}} \nameeq x$; so, the
notation is consistent with previous definitions.

Further, because names have structure it is possible to effect
substitutions on the basis of that structure. This means we need to
upgrade our notation for substitutions, which we accomplish by
adapting comprehension notation. Thus,

\begin{mathpar}
  P\{ y / x : x \in S \}
\end{mathpar}

is interpreted to mean the process derived from P by replacing (in a
capture-avoiding manner) each occurrence of $x$ in $S$ by $y$. For example,

\begin{mathpar}
  P\{ \quotep{\procn{x}|\procn{x}} / x : x \in \freenames{P} \}
\end{mathpar}

will replace each (occurrence) of a free name $x$ in $P$ by
$\quotep{\procn{x}|\procn{x}}$.

Also, we will avail ourselves of the notation $x^{L}$ and $x^{R}$ to
denote injections of a name into disjoint copies of the name
space. There are numerous ways to accomplish this. One example can be
found in \cite{MeredithR05}. This notation overloads to vectors of
names: $\vec{x}^{\pi} := (x_{i}^{\pi} \; : \; 0 \leq i < |\vec{x}| )$ where $\pi \in \{L,R\}$.

We also use $P^{\Box} := P|\Box$.

In \cite{MeredithR05} an interpretation of the new operator is
given. It turns out that there are several possible interpretations
all enjoying the requisite algebraic properties of the operator (see
\cite{milner91polyadicpi}). We will therefore make liberal use of
$(\nu\; \vec{x})P$.

% subsection the_syntax_and_semantics_of_the_notation_system (end)   

\input{qm2pi.qmops} 

\input{qm2pi.sterngerlach} 

\input{qm2pi.metric} 

% section concurrent_process_calculi (end)

%\input{qm2pi.proofsketch}

% section proof sketch (end)

%\input{qm2pi.slviaknots} 

% section spatial logic via knots (end)

\input{qm2pi.conclusion}

% section conclusion (end)

%\input{qm2pi.dtcodes} 

% section wiring algorithm (end)

\input{qm2pi.ack} 

% section acknowledgments (end)

\newpage


\bibliographystyle{plain}   
\bibliography{../../biblios/main.bib}

\input{qm2pi.rhodetails}

\end{document}

 

% subsection basic_interpretation (end)

%\input{qm2pi.rho.presentation} 
\subsection{The syntax and semantics of the notation system}\label{sub:the_syntax_and_semantics_of_the_notation_system} % (fold)

We now summarize a technical presentation of the calculus that
embodies our theory of dynamics. The typical presentation of such a
calculus follows the style of giving generators and relations on
them. The grammar, below, describing term constructors, freely
generates the set of processes, $\Proc$. This set is then quotiented
by a relation known as structural congruence and it is over this set
that the notion of dynamics is expressed. This presentation is
essentially that of \cite{MeredithR05} with the addition of
polyadicity and summation. For readability we have relegated some of
the technical subtleties to an appendix.

\subsubsection{Process grammar}\label{subsub:process_grammar}

\begin{mathpar}
  \inferrule* [lab=synchronization] {} {{M} \bc \pzero \;|\; x?F \;|\; x!C }
  \and
  \inferrule* [lab=abstraction] {} {{F} \bc (x)P}
  \and
  \inferrule* [lab=concretion] {} {{C} \bc \langle Q \rangle}
  \and
  \inferrule* [lab=process] {} {{P,Q} \bc M \;| \;P|Q \;|\; @{x}}
  \and
  \inferrule* [lab=name] {} {{x} \bc \quotep{P}}
\end{mathpar} 

Note that $\vec{x}$ (resp. $\vec{P}$) denotes a vector of names
(resp. processes) of length $|\vec{x}|$ (resp. $|\vec{P}|$). We adopt
the following useful abbreviations.

\begin{mathpar}
   x?(\vec{y}).P := x.(\vec{y})P \and  x\clift{\vec{P}} := x.\clift{\vec{P}}
   \and x!(y) := \lift{x}{\dropn{y}}
   \and \Pi_{i=0}^{n-1}P_i := P_0 | \ldots | P_{n-1}
\end{mathpar}

\subsubsection{Structural congruence}

\paragraph{Free and bound names and alpha-equivalence.} At the
core of structural equivalence is alpha-equivalence which identifies
process that are the same up to a change of variable. Formally, we
recognize the distinction between free and bound names. The free names
of a process, $\freenames{P}$, may be calculated recursively as
follows:

\begin{mathpar}
\freenames{\pzero} := \emptyset
  \and \\
  \freenames{x?(y).P} := \{ x \} \cup (\freenames{P} \setminus \{ y \})
  \and 
  \freenames{x!\langle P \rangle} := \{ x \} \cup \{ P \} 
  \and \\
  \freenames{P|Q} := \freenames{P} \cup \freenames{Q}
  \and \\
  \freenames{@{x}} := \{ x \}
\end{mathpar}

$\pi$
$\quotep{\pi}$

$\freenames{-} : \pi \to \mathcal{P}(\quotep{\pi})$

\begin{eqnarray*}
  \freenames{\pzero} & := & \emptyset \\
  \freenames{x?(y).P} & := & \{ x \} \cup (\freenames{P} \setminus \{ y \}) \\
  \freenames{x!\langle P \rangle} & := & \{ x \} \cup \{ P \} \\
  \freenames{P|Q} & := & \freenames{P} \cup \freenames{Q} \\
  \freenames{\dropn{x}} & := & \{ x \}
\end{eqnarray*}

The bound names of a process, $\boundnames{P}$, are those names occurring in $P$
that are not free. For example, in $x?(y).0$, the name $x$ is free, while $y$ is bound.

\begin{mathpar}
  \inferrule* [lab=monoidal-laws] {} { P|Q \equiv Q|P \and P|0 \equiv P \and P|(Q|R) \equiv (P|Q)|R }
\end{mathpar}

\begin{mathpar}
  \inferrule* [lab=alpha-equivalence] {} { (x)P \equiv (y)P\{y/x\} \and y \not\in \freenames{P} }
\end{mathpar}

\begin{definition}
Then two processes, $P,Q$, are alpha-equivalent if $P = Q\{\vec{y}/\vec{x}\}$ for
some $\vec{x} \in \boundnames{Q},\vec{y} \in \boundnames{P}$, where $Q\{\vec{y}/\vec{x}\}$
denotes the capture-avoiding substitution of $\vec{y}$ for $\vec{x}$ in $Q$.
\end{definition}

\begin{definition}
  The {\em structural congruence} \cite{SangiorgiWalker} , $\equiv$,
  between processes is the least congruence containing
  alpha-equivalence, satisfying the abelian monoid laws
  (associativity, commutativity and $\pzero$ as identity) for parallel
  composition $|$ and for summation $+$.
\end{definition}

\subsection{Name equivalence}

We take name equivalence, written $\nameeq$, to be the smallest
equivalence relation generated by the following rules.

\begin{mathpar}
\inferrule*[lab=Quote-drop]
{ }
{ \quotep{@{x}} \nameeq x }

\inferrule*[lab=Struct-equiv]
{ P \scong Q }
{ \quotep{P} \nameeq \quotep{Q} }
\end{mathpar}

The astute reader will have noticed that the mutual recursion of names
and processes imposes a mutual recursion on alpha-equivalence and
structural equivalence via name-equivalence. Fortunately, all of this
works out pleasantly and we may calculate in the natural way, free of
concern. The reader interested in the details is referred to the
appendix \ref{appendix:rho_details}.

\subsection{Substitution}

We use $\Proc$ for the set of processes, $\QProc$ for the set of
names, and $\id{\{}\vec{y} / \vec{x} \id{\}}$ to denote partial maps,
$s : \QProc \rightarrow \QProc$. A map, $s$ lifts, uniquely, to a map
on process terms, $\widehat{s} : \Proc \rightarrow \Proc$ by the
following equations.

\begin{mathpar}
  (0) \psubstp{Q}{P} := 0 \\
  (R \juxtap S) \psubstp{Q}{P}
  :=    
  (R)\psubstp{Q}{P} \juxtap (S) \psubstp{Q}{P} \\
  (x?(y).R) \psubstp{Q}{P}    
  :=    
  (x)\substp{Q}{P} (z)\concat( (R \psubstn{z}{y}) \psubstp{Q}{P} ) \\
  (\lift{x}{R}) \psubstp{Q}{P}  
  :=
  \lift{(x)\substp{Q}{P}}{ R \psubstp{Q}{P} } \\
%   (\dropn{x})  \psubstp{Q}{P}       
%   := 
%   \left\{ 
%     \begin{array}{ccc} 
%       \dropn{\quotep{Q}} & & x \nameeq \quotep{P} \\
%       \dropn{x} & & otherwise \\
%     \end{array}
%   \right. 
  (\dropn{x})  \psubstp{Q}{P}       
  := 
  \left\{ 
    \begin{array}{ccc} 
      Q & & x \nameeq \quotep{P} \\
      \dropn{x} & & otherwise \\
    \end{array}
  \right.
\end{mathpar}
 

where

\begin{eqnarray}
  (x)\id{\{} \lpquote Q \rpquote / \lpquote P \rpquote \id{\}}            = 
  \left\{ 
    \begin{array}{ccc}
      \lpquote Q \rpquote & & x \nameeq \lpquote P \rpquote \\
      x & & otherwise \\
    \end{array}
  \right. \nonumber
\end{eqnarray}

and $z$ is chosen distinct from $\quotep{P}$, $\quotep{Q}$, the free
names in $Q$, and all the names in $R$. Our $\alpha$-equivalence will
be built in the standard way from this substitution.

\begin{remark}\label{rem:no_self_referential_names}
  One consequence of these definitions is that $\forall P. \quotep{P}
  \not\in \freenames{P}$.
\end{remark}

\subsection{ Dynamic quote: an example }

Anticipating something of what's to come, consider applying the
substitution, $\widehat{\id{\{}u / z \id{\}}}$, to the following pair
of processes, $\lift{w}{y!(z)}$ and $w[ \lpquote y!(z) \rpquote ]$.

\begin{eqnarray}
	\lift{w}{y!(z)}\widehat{\id{\{}u / z \id{\}}}
		& = &
		\lift{w}{y!(u)} \nonumber\\
	w[ \lpquote y!(z) \rpquote ] \widehat{ \id{\{}u / z \id{\}} }
		& = &
		w[ \lpquote y!(z) \rpquote ] \nonumber
\end{eqnarray}

Because the body of the process between quotes is impervious to
substitution, we get radically different answers. In fact, by
examining the first process in an input context,
e.g. $x?(z).\lift{w}{y!(z)}$, we see that the process under the lift
operator may be shaped by prefixed inputs binding a name inside it. In
this sense, the lift operator will be seen as a way to dynamically
construct processes before reifying them as names.

Finally equipped with these standard features we can present the
dynamics of the calculus.

\subsubsection{Operational semantics} 

Finally, we introduce the computational dynamics. What marks these
algebras as distinct from other more traditionally studied algebraic
structures, e.g. vector spaces or polynomial rings, is the manner in
which dynamics is captured. In traditional structures, dynamics is typically
expressed through morphisms between such structures, as in linear maps
between vector spaces or morphisms between rings. In algebras
associated with the semantics of computation, the dynamics is
expressed as part of the algebraic structure itself, through a
reduction reduction relation typically denoted by $\red$. Below, we
give a recursive presentation of this relation for the calculus used
in the encoding.

$\red \subseteq \pi \times \pi$
$\red : \pi \to \mathcal{P}(\pi)$

\begin{mathpar}
  \inferrule* [lab=Comm] { \textsf{match}( x_{src}, x_{trgt} ) } { x_{trgt}?(y)P \; | \; x_{src}!\langle {Q} \rangle \red P\{\quotep{Q}/y}\} }
  \and \\
  \inferrule* [lab=Par] {{P} \red {P}'} {{{P} | {Q}} \red {{P}' | {Q}}}
  \and
  \inferrule* [lab=Equiv]{{{P} \scong {P}'} \andalso {{P}' \red {Q}'} \andalso {{Q}' \scong {Q}}}{{P} \red {Q}}
\end{mathpar}

\begin{eqnarray*}
  match_{\equiv} (\quotep{P},\quotep{Q}) & := & P \equiv Q \\
  match_{\dagger}(\quotep{P},\quotep{Q}) & := & \forall R. P|Q \red^{*} R => R \red^{*} 0 \\
  match_{K}(\quotep{P},\quotep{Q}) & := & K \mbox{ for some context } K
\end{eqnarray*}

$u?(x)P | u!\langle Q \rangle \red P\{\quotep{Q}/x\}$

%We write $\wred$ for $\red^*$, and $P\red$ if $\exists Q $ such that $ P \red Q$.
We write $P\red$ if $\exists Q $ such that $ P \red Q$ and $P\not\red$, otherwise.

\section{Replication}

As mentioned before, it is known that replication (and hence
recursion) can be implemented in a higher-order process algebra
\cite{SangiorgiWalker}. As our first example of calculation with the
machinery thus far presented we give the construction explicitly in
the {\rhoc}.

\begin{eqnarray}
	D_{x} & := & \prefix{x}{y}{(\binpar{\outputp{x}{y}}{@{y}})} \nonumber\\
	\bangp_{x}{P} & := & \binpar{{x}!\langle{\binpar{D_{x}}{P}}\rangle}{D_{x}} \nonumber
\end{eqnarray}

\begin{eqnarray}
	\bangp_{x}{P} & & \nonumber\\
	=
	& {x}!\langle{(\prefix{x}{y}{(\outputp{x}{y} | @{y})) | P}}\rangle 
	      | \prefix{x}{y}{(\outputp{x}{y} | @{y})} & \nonumber\\
	\red
	& (\outputp{x}{y} | @{y})\substn{\quotep{(\prefix{x}{y}{(@{y} | \outputp{x}{y})) | P}}}{y} & \nonumber\\
	=
	& \outputp{x}{\quotep{(\prefix{x}{y}{(\outputp{x}{y} | @{y})) | P}}}
	  | {(\prefix{x}{y}{(\outputp{x}{y} | @{y})) | P}} & \nonumber\\
	\red
	& \ldots & \nonumber\\
	\red^*
	& P | P | \ldots & \nonumber
\end{eqnarray}

Of course, this encoding, as an implementation, runs away, unfolding
$\bangp{P}$ eagerly. A lazier and more implementable replication
operator, restricted to input-guarded processes, may be obtained as follows.

\begin{eqnarray}
\bangp{\prefix{u}{v}{P}} 
	:= 
	\binpar{\lift{x}{\prefix{u}{v}{(\binpar{D(x)}{P})}}}{D(x)} \nonumber
\end{eqnarray}

\begin{remark}
  Note that the lazier definition still does not deal with summation
  or mixed summation (i.e. sums over input and output). The reader is
  invited to construct definitions of replication that deal with these
  features. 

  Further, the definitions are parameterized in a name, $x$. Can you,
  gentle reader, make a definition that eliminates this parameter and
  guarantees no accidental interaction between the replication
  machinery and the process being replicated -- i.e. no accidental
  sharing of names used by the process to get its work done and the
  name(s) used by the replication to effect copying. This latter
  revision of the definition of replication is crucial to obtaining
  the expected identity $!!P \sim !P$.
\end{remark}

\begin{remark}\label{rem:paradoxical_combinator}
  The reader familiar with the lambda calculus will have noticed the
  similarity between $D$ and the paradoxical combinator.

  [Ed. note: the existence of this seems to suggest we have to be more
  restrictive on the set of processes and names we admit if we are to
  support no-cloning.]
\end{remark}

\subsubsection{Bisimulation}

The computational dynamics gives rise to another kind of equivalence,
the equivalence of computational behavior. As previously mentioned
this is typically captured \emph{via} some form of bisimulation.

% The notion we use in this paper is weak barbed bisimulation
% \cite{milner91polyadicpi}.

The notion we use in this paper is derived from weak barbed
bisimulation \cite{milner91polyadicpi}. 

\begin{definition}
An \emph{observation relation}, $\downarrow_{\mathcal N}$, over a set
of names, $\mathcal N$, is the smallest relation satisfying the rules
below.

\infrule[Out-barb]{y \in {\mathcal N}, \; x \nameeq y}
		  {\outputp{x}{v} \downarrow_{\mathcal N} x}
\infrule[Par-barb]{\mbox{$P\downarrow_{\mathcal N} x$ or $Q\downarrow_{\mathcal N} x$}}
		  {\binpar{P}{Q} \downarrow_{\mathcal N} x}

We write $P \Downarrow_{\mathcal N} x$ if there is $Q$ such that 
$P \wred Q$ and $Q \downarrow_{\mathcal N} x$.
\end{definition}

\begin{definition}
%\label{def.bbisim}
An  ${\mathcal N}$-\emph{barbed bisimulation} over a set of names, ${\mathcal N}$, is a symmetric binary relation 
${\mathcal S}_{\mathcal N}$ between agents such that $P\rel{S}_{\mathcal N}Q$ implies:
\begin{enumerate}
\item If $P \red P'$ then $Q \wred Q'$ and $P'\rel{S}_{\mathcal N} Q'$.
\item If $P\downarrow_{\mathcal N} x$, then $Q\Downarrow_{\mathcal N} x$.
\end{enumerate}
$P$ is ${\mathcal N}$-barbed bisimilar to $Q$, written
$P \wbbisim_{\mathcal N} Q$, if $P \rel{S}_{\mathcal N} Q$ for some ${\mathcal N}$-barbed bisimulation ${\mathcal S}_{\mathcal N}$.
\end{definition}

$\mathcal{R} \subseteq \pi \times \pi$

$P \mathcal{R} Q => \forall P'. P \red P' \Rightarrow \exists Q'. Q \red Q', P' \mathcal{R} Q'$

$P \vdash x \Rightarrow Q \vdash x$

\begin{mathpar}
  \inferrule*[lab=Out-barb]{x \nameeq y}{{y}!\langle{Q}\rangle \vdash x}
  \and
  \inferrule*[lab=Par-barb]{\mbox{$P\vdash x$ or $Q\vdash x$}}{\binpar{P}{Q} \vdash x}
\end{mathpar}

\subsubsection{Contexts}

One of the principle advantages of computational calculi like the
$\pi$-calculus is a well-defined notion of context,
contextual-equivalence and a correlation between
contextual-equivalence and notions of bisimulation. The notion of
context allows the decomposition of a process into (sub-)process and
its syntactic environment, its context. Thus, a context may be
thought of as a process with a ``hole'' (written $\Box$) in it. The
application of a context $M$ to a process $P$, written $M[P]$, is
tantamount to filling the hole in $M$ with $P$. In this paper we do
not need the full weight of this theory, but do make use of the notion
of context in the proof the main theorem. 

\begin{mathpar}
  \inferrule* [lab=summation] {} {{M_{M},M_{N}} \bc \Box \;|\; x.M_{A} \;|\; M_{M}+M_{N}}
  \and
  \inferrule* [lab=agent] {} {{M_{A}} \bc (\vec{x})M_{P} \;| \; \clift{P_0,\ldots,M_{P},\ldots,P_N}}
  \and \\
  \inferrule* [lab=process] {} {{M_{P}} \bc M_{N} \;| \;P|M_{P} }
\end{mathpar} 

\begin{mathpar}
  \inferrule* [lab=sychronization] {} {M_{N} \bc \Box \;|\; x?M_{F} \;|\; x!M_{C}}
  \and
  \inferrule* [lab=abstraction] {} {{M_{F}} \bc (x)M_{P} }
  \and
  \inferrule* [lab=concretion] {} {{M_{C}} \bc \langle M_{P} \rangle }
  \and \\
  \inferrule* [lab=process] {} {{M_{P}} \bc M_{N} \;| \;P|M_{P} }
\end{mathpar}

\begin{definition}[contextual application] Given a context $M$, and
  process $P$, we define the \emph{contextual application}, $M[P] :=
  M\{P/\Box\}$. That is, the contextual application of M to P is the
  substitution of $P$ for $\Box$ in $M$.
\end{definition}

$\meaningof{-} : L \to \mathcal{P}(\pi)$

\begin{mathpar}
  \inferrule* [lab=collection] {} {\meaningof{true} = \pi, \and \meaningof{~E} = \pi \setminus \meaningof{E}, \and \meaningof{E_{1} \& E_{2}} = \meaningof{E_{1}} \cap \meaningof{E_{2}}}
\end{mathpar}

\begin{mathpar}
  \inferrule* [lab=structure] {} {\meaningof{0} = \{ P \in \pi | P \equiv 0 \}, \and \\ \meaningof{E_1 | E_2} = \{ P \in \pi | P \equiv P_{1} | P_{2}, P_{1} \in \meaningof{E_{1}}, P_{2} \in \meaningof{E_2}\} }
\end{mathpar}

\begin{mathpar}
 \inferrule* [lab=behavior] {} {\meaningof{\langle a?b \rangle E} = \{ P \in \pi | P \equiv Q | u?(y)P', \\ \and \\\\ \and \\ \;\;\; u \in \meaningof{a}, \forall z.P'\{z/y\} \in \meaningof{E\{z/b\}}\}, \and \\ \meaningof{a!E} = \{ P \in \pi | P \equiv Q | x!\langle P' \rangle, x \in \meaningof{a} P' \in \meaningof{E}\} }
\end{mathpar}

\begin{mathpar}
 \inferrule* [lab=nominal] {} {\meaningof{\quotep{E}} = \{ \quotep{P} \in \quotep{\pi} | P \in \meaningof{E} \}, \and \meaningof{\quotep{P}} = \{ \quotep{Q} \in \quotep{\pi} | P \equiv Q \} \and \\ \meaningof{@\quotep{E}} = \{ P \in \pi | P \equiv @x, x \in \meaningof{E} \}}
\end{mathpar}

\begin{eqnarray*}
  \\
  \meaningof{-} : TS \to ST
\end{eqnarray*}

\begin{eqnarray*}
  \\
  L : TS \to ST
\end{eqnarray*}

\begin{eqnarray*}
  \\
  P \models E \iff P \in \meaningof{E}
\end{eqnarray*}

\begin{eqnarray*}
  P \approx_{L} Q \iff \forall E \in L. P \models E \iff Q \models E
\end{eqnarray*}

\begin{eqnarray*}
  P \approx_{K} Q
\end{eqnarray*}

\begin{eqnarray*}
  P \approx Q
\end{eqnarray*}

$\approx_{K} = \approx = \approx_{L}$

\subsubsection{Contextual duality}

Note that contexts extend the quotation operation to a family of
operations from processes to names. Given a context, $M$, we can
define a \emph{nominal context}, $\quotep{M}$ by $\quotep{M}[P] :=
\quotep{M[P]}$. To foreshadow what is to come we observe that these
operations enjoy a duality with processes very much like the duality
between vectors and maps from vectors to scalars.

Further, because the calculus is essentially higher-order, we have a
correspondence between contexts and processes. More specifically,
given a name $x$ and a context $M$ we can construct $M^{*}_{x}$ such
that 

\begin{mathpar}
  M^{*}_{x} | \lift{x}{P} \red M[P]
\end{mathpar}

namely,

\begin{mathpar}
  M^{*}_{x} := x?(u).M[\dropn{u}]
\end{mathpar}

The dependence of $M^{*}_{x}$ on a name makes it an abstraction, 

\begin{mathpar}
  M^{*} := (x)x?(u).M[\dropn{u}]
\end{mathpar}

\subsection{Additional notation}

It will sometimes be convenient to denote the process a name
quotes. We already have the notation $x = \quotep{P}$, but it will be
convenient to introduce an alternate notation, $\procn{x}$, when we
want to emphasize the connection to the use of the name. Note that, by
virtue of name equivalence, $\quotep{\procn{x}} \nameeq x$; so, the
notation is consistent with previous definitions.

Further, because names have structure it is possible to effect
substitutions on the basis of that structure. This means we need to
upgrade our notation for substitutions, which we accomplish by
adapting comprehension notation. Thus,

\begin{mathpar}
  P\{ y / x : x \in S \}
\end{mathpar}

is interpreted to mean the process derived from P by replacing (in a
capture-avoiding manner) each occurrence of $x$ in $S$ by $y$. For example,

\begin{mathpar}
  P\{ \quotep{\procn{x}|\procn{x}} / x : x \in \freenames{P} \}
\end{mathpar}

will replace each (occurrence) of a free name $x$ in $P$ by
$\quotep{\procn{x}|\procn{x}}$.

Also, we will avail ourselves of the notation $x^{L}$ and $x^{R}$ to
denote injections of a name into disjoint copies of the name
space. There are numerous ways to accomplish this. One example can be
found in \cite{MeredithR05}. This notation overloads to vectors of
names: $\vec{x}^{\pi} := (x_{i}^{\pi} \; : \; 0 \leq i < |\vec{x}| )$ where $\pi \in \{L,R\}$.

We also use $P^{\Box} := P|\Box$.

In \cite{MeredithR05} an interpretation of the new operator is
given. It turns out that there are several possible interpretations
all enjoying the requisite algebraic properties of the operator (see
\cite{milner91polyadicpi}). We will therefore make liberal use of
$(\nu\; \vec{x})P$.

% subsection the_syntax_and_semantics_of_the_notation_system (end)   

\section{Interpretation of QM}
\subsection{Supporting definitions}
\subsubsection{Multiplication}
\begin{mathpar}
  \quotep{Q} \cdot \quotep{R} := \quotep{Q|R}
  \and \\
  \quotep{Q} \cdot P := P\{ \quotep{Q|R} / \quotep{R} : \quotep{R} \in \freenames{P} \}
\end{mathpar}

\paragraph{Discussion}
The first line needs little explanation. The second line says that
each free name of the process is replaced with the multiplication of
that name by the scalar. Multiplication of a scalar (name) by a state
(process) results in a process all the names of which have been `moved
over' by parallel composition with the process the scalar
quotes. There is a subtlety that the bound names have to be
manipulated so that multiplied names aren't accidentally
captured. There are many ways to achieve this.

\begin{remark}\label{rem:multiplication_identities}
  The reader is invited to verify that for all $x,y,z \in \QProc$ and $P \in \Proc$
  \begin{mathpar}
    x \cdot \quotep{0} \equiv x 
    \and
    x \cdot y \equiv y \cdot x
    \and
    x \cdot (y \cdot z) \equiv (x \cdot y) \cdot z
    \and \\
    \quotep{0} \cdot P \equiv P
    \and \\
    x \cdot (y \cdot P) \equiv (x \cdot y) \cdot P
    \and \\
    x \cdot (P|Q) \equiv (x \cdot P) | (x \cdot Q)
    \and \\    
  \end{mathpar}
\end{remark}

\subsubsection{Tensor product}

We define a tensor product on processes by structural induction.

\paragraph{Tensor of sums} First note that all summations, including
$\pzero$ and sequence, can be written $\Sigma_{i} x_{i}.A_{i} +
\Sigma_{j} x_{j}.C_{j}$, where we have grouped input-guarded processes
together and output-guarded processes together.

Thus, we can define the tensor product of two summations, $N_{1}\otimes N_{2}$, where

\begin{mathpar}
  N_{1} := \Sigma_{i} x_{i}.A_{i} + \Sigma_{j} x_{j}.C_{j}
  \and
  N_{2} := \Sigma_{i'} y_{i'}.B_{i'} + \Sigma_{j'} y_{j'}.D_{j'} 
\end{mathpar}

as follows.

\begin{mathpar}
  \Sigma_{i} x_{i}.A_{i} + \Sigma_{j} x_{j}.C_{j} \otimes \Sigma_{i'}
  y_{i'}.B_{i'} + \Sigma_{j'} y_{j'}.D_{j'} 
  \and \\
  := \; \Sigma_{i} \Sigma_{i'} \quotep{\stackrel{\vee}{x_{i}}| \stackrel{\vee}{y_{i'}}}.(A_{i}\otimes B_{i'}) \; | \; \Sigma_{i'} \Sigma_{i} \quotep{\stackrel{\vee}{y_{i'}}|\stackrel{\vee}{x_{i}}}.(B_{i'}\otimes A_{i})
  \and
  \;\; | \;\; \Sigma_{j} \Sigma_{j'} \quotep{\stackrel{\vee}{x_{j}}|\stackrel{\vee}{y_{j'}}}.(A_{j}\otimes B_{j'}) \; | \; \Sigma_{j'} \Sigma_{j} \quotep{\stackrel{\vee}{y_{j'}}|\stackrel{\vee}{x_{j}}}.(B_{j'}\otimes A_{j})
\end{mathpar}

\begin{remark}
  Do we need to $x^{L}$ and $y^{R}$ for this construction as well?
\end{remark}

\paragraph{Tensor of parallel compositions} Next, we distribute tensor
over par.

\begin{mathpar}
  P_{1}|P_{2} \otimes Q_{1}|Q_{2} := (P_{1} \otimes Q_{1}) | (P_{1}
  \otimes Q_{2}) | (P_{2} \otimes Q_{1}) | (P_{2} \otimes Q_{2})
\end{mathpar}

\paragraph{Tensor with dropped names} We treat tensor of a
process with a dropped name as parallel composition.

\begin{mathpar}
  P \otimes \dropn{x} := P | \dropn{x}
\end{mathpar}

\paragraph{Tensor of agents}

Finally, we need to define tensor on agents. Note that the definition
of tensor on normal products only tensors inputs with inputs and
outputs with outputs. Thus, we only have to define the operation on
``homogeneous'' pairings.

\begin{mathpar}
  (\vec{x})P \otimes (\vec{y})Q
  \and \\
  := (x_{0}^{L}|y_{0}^{R},\ldots,x_{0}^{L}|y_{n}^{R},\ldots,x_{m}^{L}|y_{0}^{R},\ldots,x_{m}^{L}|y_{n}^R)(P\{ \vec{x}^{L}/\vec{x}\} \otimes Q \{ \vec{y}^{R}/\vec{y}\})
  \and \\
  \clift{\vec{P}} \otimes \clift{\vec{Q}}
  \and \\
  := \clift{P_{0}\otimes Q_{0},\ldots,P_{0}\otimes Q_{n},\ldots,P_{m}\otimes Q_{0},\ldots,P_{m}\otimes Q_{n}}
\end{mathpar}

\begin{remark}
  Observe that arities of tensored abstractions matches arities of
  tensored concretions if the original arities matched. Note also that
  the length of the arities corresponds to the increase in dimension
  we see in ordinary vector space tensor product.
\end{remark}

\begin{remark}
  Operationally, this definition distributes the tensor down to
  components ``linked'' by summation. Tensor over summation is
  intriguing in that it mixes names. Moreover, as a consequence of the
  way it mixes names we have the identities for all $x \in \QProc$ and
  $P,Q \in \Proc$

  \begin{mathpar}
    (x \cdot P) \otimes Q \equiv x \cdot (P \otimes Q) \equiv P \otimes (x \cdot Q)
    \and
    P \otimes \pzero \equiv P
  \end{mathpar}

  that the reader is invited to verify.
\end{remark}

\subsubsection{Annihilation}
\begin{mathpar}
  P^{\perp} := \{ Q | \forall R. P|Q \red^{*} R \Rightarrow R \red^{*} \pzero \}
  \and \\
  P^{\underline{\perp}} := \Sigma_{Q \in P^{\perp}} \quotep{Q}?(y).(\dropn{y}|Q) | \Sigma_{Q \in P^{\perp}} \quotep{Q}\clift{\Box}
\end{mathpar}

\paragraph{Discussion} The reader will note that $P^{\perp}$ is a
\emph{set} of processes, while $P^{\underline{\perp}}$ is a
\emph{context}. We call the set $P^{\perp}$ the \emph{annihilators} of
$P$. The parallel composition of a process in the annihilators of $P$
with $P$ will result in a process, the state space of which has all
paths eventually leading to $\pzero$. Execution may endure loops; but
under reasonable conditions of fairness (naturally guaranteed under
most notions of bisimulation) such a composite process cannot get
stuck in such a loop and will, eventually pop out and terminate.

The context $P^{\underline{\perp}}$ is ready and willing to ``take the
$P$ out of'' the process to which it is applied. It will effectively
transmit the code of the process to which it is applied to one of the
annihilators and run the process against it.

\subsubsection{Evaluation}
We fix $M$ a domain of fully abstract interpretation with an equality
coincident with bisimulation. We take $\meaningof{\cdot} : \Proc \to
M$ to be the map interpreting processes and $\nmeaningof{\cdot} : \M
\to Proc$ to be the map running the other way. Then we define

\begin{mathpar}
  \int P := \nmeaningof{\meaningof{P}}
\end{mathpar}

\paragraph{Discussion}
There are many fully abstract interpretations of Milner's
$\pi$-calculus. Any of them can be used as a basis for interpreting
the reflective calculus here. Equipped with such a domain it is
largely a matter of grinding through to check that the Yoneda
construction for the normalization-by-evaluation program can be
extended to this setting.

\begin{remark}
  The reader is invited to verify that $\int (P^{\underline{\perp}}[P]) = 0$.
\end{remark}

\subsection{Quantum mechanics}

Table \ref{tbl:core_qm_op_defns} gives the core operational definitions

\begin{table}[htp]\label{tbl:core_qm_op_defns}
  \center{
    \fbox{
      \begin{tabular}{c|c}
        quantum mechanics & process calculus \\
        \hline
        scalar & $x := \quotep{P}$ \\
        state vector & $\state{P} := P$ \\
        dual & $\state{P}^{*} := \event{P^{\underline{\perp}}} := \quotep{P^{\underline{\perp}}}[-]$ \\
        matrix & $ \Sigma_{\alpha} \state{P_{\alpha}}x_{\alpha}\event{Q_{\alpha}}$ \\
        vector addition & $\state{P} + \state{Q} := \state{P | Q}$ \\
        tensor product & $\state{P} \otimes \state{Q} := \state{P \otimes Q}$ \\
        inner product & $\innerprod{P}{Q} := \quotep{\int P^{\underline{\perp}}[Q]}$ \\
      \end{tabular}
    }
  }
  \caption{QM - operational definitions}
\end{table}

where

\begin{mathpar}
  \prmatrix{P}{Q} := \fprmatrix{P}{\quotep{\pzero}}{Q}
  \and
  \fprmatrix{P}{x}{Q} := (\state{P},x,\event{Q})
  \and
  (\fprmatrix{P}{x}{Q})(\state{R}) := x \cdot \innerprod{Q}{R} \cdot \state{P}
  \and
  (\fprmatrix{P}{x}{Q})(\event{R}) := x \cdot \innerprod{R}{P} \cdot \event{Q}
\end{mathpar}

\paragraph{Discussion}
As promised: vectors (aka states) are represented as processes; duals
as contextual duals; inner product definition should be compared with
standard inner product definition for ....

\begin{remark}
  Assuming $\int (P^{\underline{\perp}}[P]) = 0$, the reader is
  invited to verify that $(\fprmatrix{P}{x}{P})(\state{P}) = x \cdot \state{P}$.
\end{remark}

\begin{remark}
  The reader is invited to verify that $\innerprod{P}{Q}$ could
  equally well have been written $\quotep{\int \stackrel{\vee}{x}}$
  where $x = \event{P^{\underline{\perp}}}(Q)$.

  One of the motivations for this remark is that there is another way
  to factor these operations. We could package up evaluation in the dual:

  \begin{mathpar}
    \state{P}^{*} := \event{\int P^{\underline{\perp}}} := \quotep{\int P^{\underline{\perp}}}[-]
  \end{mathpar}

  and then have inner product defined by
  
  \begin{mathpar}
    \innerprod{P}{Q} := \event{P}(Q)
  \end{mathpar}

  Hopefully, experience with the calculations will provide guidance on
  the best factoring.
\end{remark}

\begin{remark}
  Assuming $\int (P^{\underline{\perp}}[P]) = 0$, the reader is
  invited to verify that $\forall P,Q. (\prmatrix{0}{Q})(\state{0}) =
  \state{0}$ and dually $(\prmatrix{P}{0})(\event{0}) = \event{0}$.
\end{remark}

\begin{remark}
  i'm a little worried that i don't (yet) have proper support for
  complex conjugacy. But, the observation above may give us a
  clue. According to Abramsky, it must be the case that the scalars
  are iso to the homset of the identity for the tensor -- which the
  observation above characterizes. 

  For now, we will simply bookmark the notion with $\overline{x}$.
\end{remark}

\subsubsection{Adjointness}

We need to give a definition of $(\cdot)^{\dagger}$ for matrices. The
obvious candidate definition is
\begin{mathpar}
(\Sigma_{\alpha}\fprmatrix{P_{\alpha}}{x_{\alpha}}{Q_{\alpha}})^{\dagger}
= \Sigma_{\alpha}\fprmatrix{(Q_{\alpha}^{\underline{\perp}})^{*}}{\overline{x}_{\alpha}}{P_{\alpha}^{\underline{\perp}}} 
\end{mathpar}

But, $(Q_{\alpha}^{\underline{\perp}})^{*}$ requires a name along
which to communicate the process to achieve the context application.

\subsubsection{Basis for a basis}
If processes label states and ``addition'' of states (a.k.a. vector
addition) is interpreted as parallel composition, what corresponds to
notions of linear independence and basis? Here, we recall that Yoshida
has developed a set of \emph{combinators} for an asynchronous verison
of Milner's $\pi$-calculus. These are a finite set of processes such
any process can be expressed as parallel composition of these
combinators together with liberal uses of the new operator and
replication. We can simply give a translation of these into the
present calculus and have reasonable expectation that the property
carries over. That is, that the resultant set allows to express all
processes via parallel composition. Note, however, that there is no
new operator or replication in this calculus. As a result, we expect
that the corresponding set is actually infinite. That is, we expect
that the space is actually infinite dimensional.

\begin{remark}
  The attentive reader may be a bit concerned. Certainly, the
  collection $S$, $K$ and $I$ is a finite set of
  combinators. Shouldn't we expect to see a finite set of combinators
  for an effectively equivalent system? i am very sympathetic to this
  critique and feel it warrants full attention. On the other hand, i
  also have in mind the following analogy. The natural numbers, as a
  monoid under addition, has exactly $1$ generator, while the natural
  numbers, as a monoid under multiplication, has countably many
  generators (the primes). We observe that the application of the
  lambda calculus is much less resource sensitive than the parallel
  composition of the $\pi$-calculus. Could it be the case that we have
  an analogy of the form
  
  \begin{mathpar}
    m + n : MN :: m*n : M|N
  \end{mathpar}

  giving a similar blow up in the set of ``primes''?  This is such a
  wonderful thought that, even if it's not true, i think it's worth
  writing down.
\end{remark}
 

\documentclass[12pt]{llncs}
%\documentclass{jktr}

\usepackage[pdftex]{hyperref}                   
\usepackage {listings}
\usepackage {mathpartir}
\usepackage{bcprules}
%\usepackage{listings}
                       
\usepackage{graphicx} 
%\usepackage[margins=2.5cm,nohead,nofoot]{geometry}
%\usepackage{geometry}
\usepackage{amsfonts}
\usepackage{amstext}
\usepackage{latexsym}
\usepackage{amssymb}
\usepackage{color}


%\include{myPreamble}
\include{qm2pi.local} 

%\ifpdf
%\usepackage[pdftex]{graphicx}
%\else
%\usepackage{graphicx}
%\fi

 % \ifpdf
%  \usepackage{pdfsync}
%  \if


%\title{Brief Article}
%\author{David F. Snyder}
%\author{L.G. Meredith}

%\address{Dept. of Math., Texas State University--San Marcos, San Marcos, TX 78666}
       
\pagestyle{empty}


\begin{document}

\lstset{language=[Objective]Caml,frame=shadowbox}

\input{qm2pi.front}

% section front matter (end)

\input{qm2pi.intro} 
 
% section introduction (end)

% \input{qm2pi.knotations} 

% section notation (end)

\input{qm2pi.process.calculi} 

% section concurrent_process_calculi_and_spatial_logics_ (end)
    
%\input{qm2pi.knots2pi} 

%\input{qm2pi.trefoil} 

%\input{qm2pi.mainthm} 

% subsection basic_interpretation (end)

%\input{qm2pi.rho.presentation} 
\subsection{The syntax and semantics of the notation system}\label{sub:the_syntax_and_semantics_of_the_notation_system} % (fold)

We now summarize a technical presentation of the calculus that
embodies our theory of dynamics. The typical presentation of such a
calculus follows the style of giving generators and relations on
them. The grammar, below, describing term constructors, freely
generates the set of processes, $\Proc$. This set is then quotiented
by a relation known as structural congruence and it is over this set
that the notion of dynamics is expressed. This presentation is
essentially that of \cite{MeredithR05} with the addition of
polyadicity and summation. For readability we have relegated some of
the technical subtleties to an appendix.

\subsubsection{Process grammar}\label{subsub:process_grammar}

\begin{mathpar}
  \inferrule* [lab=synchronization] {} {{M} \bc \pzero \;|\; x?F \;|\; x!C }
  \and
  \inferrule* [lab=abstraction] {} {{F} \bc (x)P}
  \and
  \inferrule* [lab=concretion] {} {{C} \bc \langle Q \rangle}
  \and
  \inferrule* [lab=process] {} {{P,Q} \bc M \;| \;P|Q \;|\; @{x}}
  \and
  \inferrule* [lab=name] {} {{x} \bc \quotep{P}}
\end{mathpar} 

Note that $\vec{x}$ (resp. $\vec{P}$) denotes a vector of names
(resp. processes) of length $|\vec{x}|$ (resp. $|\vec{P}|$). We adopt
the following useful abbreviations.

\begin{mathpar}
   x?(\vec{y}).P := x.(\vec{y})P \and  x\clift{\vec{P}} := x.\clift{\vec{P}}
   \and x!(y) := \lift{x}{\dropn{y}}
   \and \Pi_{i=0}^{n-1}P_i := P_0 | \ldots | P_{n-1}
\end{mathpar}

\subsubsection{Structural congruence}

\paragraph{Free and bound names and alpha-equivalence.} At the
core of structural equivalence is alpha-equivalence which identifies
process that are the same up to a change of variable. Formally, we
recognize the distinction between free and bound names. The free names
of a process, $\freenames{P}$, may be calculated recursively as
follows:

\begin{mathpar}
\freenames{\pzero} := \emptyset
  \and \\
  \freenames{x?(y).P} := \{ x \} \cup (\freenames{P} \setminus \{ y \})
  \and 
  \freenames{x!\langle P \rangle} := \{ x \} \cup \{ P \} 
  \and \\
  \freenames{P|Q} := \freenames{P} \cup \freenames{Q}
  \and \\
  \freenames{@{x}} := \{ x \}
\end{mathpar}

$\pi$
$\quotep{\pi}$

$\freenames{-} : \pi \to \mathcal{P}(\quotep{\pi})$

\begin{eqnarray*}
  \freenames{\pzero} & := & \emptyset \\
  \freenames{x?(y).P} & := & \{ x \} \cup (\freenames{P} \setminus \{ y \}) \\
  \freenames{x!\langle P \rangle} & := & \{ x \} \cup \{ P \} \\
  \freenames{P|Q} & := & \freenames{P} \cup \freenames{Q} \\
  \freenames{\dropn{x}} & := & \{ x \}
\end{eqnarray*}

The bound names of a process, $\boundnames{P}$, are those names occurring in $P$
that are not free. For example, in $x?(y).0$, the name $x$ is free, while $y$ is bound.

\begin{mathpar}
  \inferrule* [lab=monoidal-laws] {} { P|Q \equiv Q|P \and P|0 \equiv P \and P|(Q|R) \equiv (P|Q)|R }
\end{mathpar}

\begin{mathpar}
  \inferrule* [lab=alpha-equivalence] {} { (x)P \equiv (y)P\{y/x\} \and y \not\in \freenames{P} }
\end{mathpar}

\begin{definition}
Then two processes, $P,Q$, are alpha-equivalent if $P = Q\{\vec{y}/\vec{x}\}$ for
some $\vec{x} \in \boundnames{Q},\vec{y} \in \boundnames{P}$, where $Q\{\vec{y}/\vec{x}\}$
denotes the capture-avoiding substitution of $\vec{y}$ for $\vec{x}$ in $Q$.
\end{definition}

\begin{definition}
  The {\em structural congruence} \cite{SangiorgiWalker} , $\equiv$,
  between processes is the least congruence containing
  alpha-equivalence, satisfying the abelian monoid laws
  (associativity, commutativity and $\pzero$ as identity) for parallel
  composition $|$ and for summation $+$.
\end{definition}

\subsection{Name equivalence}

We take name equivalence, written $\nameeq$, to be the smallest
equivalence relation generated by the following rules.

\begin{mathpar}
\inferrule*[lab=Quote-drop]
{ }
{ \quotep{@{x}} \nameeq x }

\inferrule*[lab=Struct-equiv]
{ P \scong Q }
{ \quotep{P} \nameeq \quotep{Q} }
\end{mathpar}

The astute reader will have noticed that the mutual recursion of names
and processes imposes a mutual recursion on alpha-equivalence and
structural equivalence via name-equivalence. Fortunately, all of this
works out pleasantly and we may calculate in the natural way, free of
concern. The reader interested in the details is referred to the
appendix \ref{appendix:rho_details}.

\subsection{Substitution}

We use $\Proc$ for the set of processes, $\QProc$ for the set of
names, and $\id{\{}\vec{y} / \vec{x} \id{\}}$ to denote partial maps,
$s : \QProc \rightarrow \QProc$. A map, $s$ lifts, uniquely, to a map
on process terms, $\widehat{s} : \Proc \rightarrow \Proc$ by the
following equations.

\begin{mathpar}
  (0) \psubstp{Q}{P} := 0 \\
  (R \juxtap S) \psubstp{Q}{P}
  :=    
  (R)\psubstp{Q}{P} \juxtap (S) \psubstp{Q}{P} \\
  (x?(y).R) \psubstp{Q}{P}    
  :=    
  (x)\substp{Q}{P} (z)\concat( (R \psubstn{z}{y}) \psubstp{Q}{P} ) \\
  (\lift{x}{R}) \psubstp{Q}{P}  
  :=
  \lift{(x)\substp{Q}{P}}{ R \psubstp{Q}{P} } \\
%   (\dropn{x})  \psubstp{Q}{P}       
%   := 
%   \left\{ 
%     \begin{array}{ccc} 
%       \dropn{\quotep{Q}} & & x \nameeq \quotep{P} \\
%       \dropn{x} & & otherwise \\
%     \end{array}
%   \right. 
  (\dropn{x})  \psubstp{Q}{P}       
  := 
  \left\{ 
    \begin{array}{ccc} 
      Q & & x \nameeq \quotep{P} \\
      \dropn{x} & & otherwise \\
    \end{array}
  \right.
\end{mathpar}
 

where

\begin{eqnarray}
  (x)\id{\{} \lpquote Q \rpquote / \lpquote P \rpquote \id{\}}            = 
  \left\{ 
    \begin{array}{ccc}
      \lpquote Q \rpquote & & x \nameeq \lpquote P \rpquote \\
      x & & otherwise \\
    \end{array}
  \right. \nonumber
\end{eqnarray}

and $z$ is chosen distinct from $\quotep{P}$, $\quotep{Q}$, the free
names in $Q$, and all the names in $R$. Our $\alpha$-equivalence will
be built in the standard way from this substitution.

\begin{remark}\label{rem:no_self_referential_names}
  One consequence of these definitions is that $\forall P. \quotep{P}
  \not\in \freenames{P}$.
\end{remark}

\subsection{ Dynamic quote: an example }

Anticipating something of what's to come, consider applying the
substitution, $\widehat{\id{\{}u / z \id{\}}}$, to the following pair
of processes, $\lift{w}{y!(z)}$ and $w[ \lpquote y!(z) \rpquote ]$.

\begin{eqnarray}
	\lift{w}{y!(z)}\widehat{\id{\{}u / z \id{\}}}
		& = &
		\lift{w}{y!(u)} \nonumber\\
	w[ \lpquote y!(z) \rpquote ] \widehat{ \id{\{}u / z \id{\}} }
		& = &
		w[ \lpquote y!(z) \rpquote ] \nonumber
\end{eqnarray}

Because the body of the process between quotes is impervious to
substitution, we get radically different answers. In fact, by
examining the first process in an input context,
e.g. $x?(z).\lift{w}{y!(z)}$, we see that the process under the lift
operator may be shaped by prefixed inputs binding a name inside it. In
this sense, the lift operator will be seen as a way to dynamically
construct processes before reifying them as names.

Finally equipped with these standard features we can present the
dynamics of the calculus.

\subsubsection{Operational semantics} 

Finally, we introduce the computational dynamics. What marks these
algebras as distinct from other more traditionally studied algebraic
structures, e.g. vector spaces or polynomial rings, is the manner in
which dynamics is captured. In traditional structures, dynamics is typically
expressed through morphisms between such structures, as in linear maps
between vector spaces or morphisms between rings. In algebras
associated with the semantics of computation, the dynamics is
expressed as part of the algebraic structure itself, through a
reduction reduction relation typically denoted by $\red$. Below, we
give a recursive presentation of this relation for the calculus used
in the encoding.

$\red \subseteq \pi \times \pi$
$\red : \pi \to \mathcal{P}(\pi)$

\begin{mathpar}
  \inferrule* [lab=Comm] { \textsf{match}( x_{src}, x_{trgt} ) } { x_{trgt}?(y)P \; | \; x_{src}!\langle {Q} \rangle \red P\{\quotep{Q}/y}\} }
  \and \\
  \inferrule* [lab=Par] {{P} \red {P}'} {{{P} | {Q}} \red {{P}' | {Q}}}
  \and
  \inferrule* [lab=Equiv]{{{P} \scong {P}'} \andalso {{P}' \red {Q}'} \andalso {{Q}' \scong {Q}}}{{P} \red {Q}}
\end{mathpar}

\begin{eqnarray*}
  match_{\equiv} (\quotep{P},\quotep{Q}) & := & P \equiv Q \\
  match_{\dagger}(\quotep{P},\quotep{Q}) & := & \forall R. P|Q \red^{*} R => R \red^{*} 0 \\
  match_{K}(\quotep{P},\quotep{Q}) & := & K \mbox{ for some context } K
\end{eqnarray*}

$u?(x)P | u!\langle Q \rangle \red P\{\quotep{Q}/x\}$

%We write $\wred$ for $\red^*$, and $P\red$ if $\exists Q $ such that $ P \red Q$.
We write $P\red$ if $\exists Q $ such that $ P \red Q$ and $P\not\red$, otherwise.

\section{Replication}

As mentioned before, it is known that replication (and hence
recursion) can be implemented in a higher-order process algebra
\cite{SangiorgiWalker}. As our first example of calculation with the
machinery thus far presented we give the construction explicitly in
the {\rhoc}.

\begin{eqnarray}
	D_{x} & := & \prefix{x}{y}{(\binpar{\outputp{x}{y}}{@{y}})} \nonumber\\
	\bangp_{x}{P} & := & \binpar{{x}!\langle{\binpar{D_{x}}{P}}\rangle}{D_{x}} \nonumber
\end{eqnarray}

\begin{eqnarray}
	\bangp_{x}{P} & & \nonumber\\
	=
	& {x}!\langle{(\prefix{x}{y}{(\outputp{x}{y} | @{y})) | P}}\rangle 
	      | \prefix{x}{y}{(\outputp{x}{y} | @{y})} & \nonumber\\
	\red
	& (\outputp{x}{y} | @{y})\substn{\quotep{(\prefix{x}{y}{(@{y} | \outputp{x}{y})) | P}}}{y} & \nonumber\\
	=
	& \outputp{x}{\quotep{(\prefix{x}{y}{(\outputp{x}{y} | @{y})) | P}}}
	  | {(\prefix{x}{y}{(\outputp{x}{y} | @{y})) | P}} & \nonumber\\
	\red
	& \ldots & \nonumber\\
	\red^*
	& P | P | \ldots & \nonumber
\end{eqnarray}

Of course, this encoding, as an implementation, runs away, unfolding
$\bangp{P}$ eagerly. A lazier and more implementable replication
operator, restricted to input-guarded processes, may be obtained as follows.

\begin{eqnarray}
\bangp{\prefix{u}{v}{P}} 
	:= 
	\binpar{\lift{x}{\prefix{u}{v}{(\binpar{D(x)}{P})}}}{D(x)} \nonumber
\end{eqnarray}

\begin{remark}
  Note that the lazier definition still does not deal with summation
  or mixed summation (i.e. sums over input and output). The reader is
  invited to construct definitions of replication that deal with these
  features. 

  Further, the definitions are parameterized in a name, $x$. Can you,
  gentle reader, make a definition that eliminates this parameter and
  guarantees no accidental interaction between the replication
  machinery and the process being replicated -- i.e. no accidental
  sharing of names used by the process to get its work done and the
  name(s) used by the replication to effect copying. This latter
  revision of the definition of replication is crucial to obtaining
  the expected identity $!!P \sim !P$.
\end{remark}

\begin{remark}\label{rem:paradoxical_combinator}
  The reader familiar with the lambda calculus will have noticed the
  similarity between $D$ and the paradoxical combinator.

  [Ed. note: the existence of this seems to suggest we have to be more
  restrictive on the set of processes and names we admit if we are to
  support no-cloning.]
\end{remark}

\subsubsection{Bisimulation}

The computational dynamics gives rise to another kind of equivalence,
the equivalence of computational behavior. As previously mentioned
this is typically captured \emph{via} some form of bisimulation.

% The notion we use in this paper is weak barbed bisimulation
% \cite{milner91polyadicpi}.

The notion we use in this paper is derived from weak barbed
bisimulation \cite{milner91polyadicpi}. 

\begin{definition}
An \emph{observation relation}, $\downarrow_{\mathcal N}$, over a set
of names, $\mathcal N$, is the smallest relation satisfying the rules
below.

\infrule[Out-barb]{y \in {\mathcal N}, \; x \nameeq y}
		  {\outputp{x}{v} \downarrow_{\mathcal N} x}
\infrule[Par-barb]{\mbox{$P\downarrow_{\mathcal N} x$ or $Q\downarrow_{\mathcal N} x$}}
		  {\binpar{P}{Q} \downarrow_{\mathcal N} x}

We write $P \Downarrow_{\mathcal N} x$ if there is $Q$ such that 
$P \wred Q$ and $Q \downarrow_{\mathcal N} x$.
\end{definition}

\begin{definition}
%\label{def.bbisim}
An  ${\mathcal N}$-\emph{barbed bisimulation} over a set of names, ${\mathcal N}$, is a symmetric binary relation 
${\mathcal S}_{\mathcal N}$ between agents such that $P\rel{S}_{\mathcal N}Q$ implies:
\begin{enumerate}
\item If $P \red P'$ then $Q \wred Q'$ and $P'\rel{S}_{\mathcal N} Q'$.
\item If $P\downarrow_{\mathcal N} x$, then $Q\Downarrow_{\mathcal N} x$.
\end{enumerate}
$P$ is ${\mathcal N}$-barbed bisimilar to $Q$, written
$P \wbbisim_{\mathcal N} Q$, if $P \rel{S}_{\mathcal N} Q$ for some ${\mathcal N}$-barbed bisimulation ${\mathcal S}_{\mathcal N}$.
\end{definition}

$\mathcal{R} \subseteq \pi \times \pi$

$P \mathcal{R} Q => \forall P'. P \red P' \Rightarrow \exists Q'. Q \red Q', P' \mathcal{R} Q'$

$P \vdash x \Rightarrow Q \vdash x$

\begin{mathpar}
  \inferrule*[lab=Out-barb]{x \nameeq y}{{y}!\langle{Q}\rangle \vdash x}
  \and
  \inferrule*[lab=Par-barb]{\mbox{$P\vdash x$ or $Q\vdash x$}}{\binpar{P}{Q} \vdash x}
\end{mathpar}

\subsubsection{Contexts}

One of the principle advantages of computational calculi like the
$\pi$-calculus is a well-defined notion of context,
contextual-equivalence and a correlation between
contextual-equivalence and notions of bisimulation. The notion of
context allows the decomposition of a process into (sub-)process and
its syntactic environment, its context. Thus, a context may be
thought of as a process with a ``hole'' (written $\Box$) in it. The
application of a context $M$ to a process $P$, written $M[P]$, is
tantamount to filling the hole in $M$ with $P$. In this paper we do
not need the full weight of this theory, but do make use of the notion
of context in the proof the main theorem. 

\begin{mathpar}
  \inferrule* [lab=summation] {} {{M_{M},M_{N}} \bc \Box \;|\; x.M_{A} \;|\; M_{M}+M_{N}}
  \and
  \inferrule* [lab=agent] {} {{M_{A}} \bc (\vec{x})M_{P} \;| \; \clift{P_0,\ldots,M_{P},\ldots,P_N}}
  \and \\
  \inferrule* [lab=process] {} {{M_{P}} \bc M_{N} \;| \;P|M_{P} }
\end{mathpar} 

\begin{mathpar}
  \inferrule* [lab=sychronization] {} {M_{N} \bc \Box \;|\; x?M_{F} \;|\; x!M_{C}}
  \and
  \inferrule* [lab=abstraction] {} {{M_{F}} \bc (x)M_{P} }
  \and
  \inferrule* [lab=concretion] {} {{M_{C}} \bc \langle M_{P} \rangle }
  \and \\
  \inferrule* [lab=process] {} {{M_{P}} \bc M_{N} \;| \;P|M_{P} }
\end{mathpar}

\begin{definition}[contextual application] Given a context $M$, and
  process $P$, we define the \emph{contextual application}, $M[P] :=
  M\{P/\Box\}$. That is, the contextual application of M to P is the
  substitution of $P$ for $\Box$ in $M$.
\end{definition}

$\meaningof{-} : L \to \mathcal{P}(\pi)$

\begin{mathpar}
  \inferrule* [lab=collection] {} {\meaningof{true} = \pi, \and \meaningof{~E} = \pi \setminus \meaningof{E}, \and \meaningof{E_{1} \& E_{2}} = \meaningof{E_{1}} \cap \meaningof{E_{2}}}
\end{mathpar}

\begin{mathpar}
  \inferrule* [lab=structure] {} {\meaningof{0} = \{ P \in \pi | P \equiv 0 \}, \and \\ \meaningof{E_1 | E_2} = \{ P \in \pi | P \equiv P_{1} | P_{2}, P_{1} \in \meaningof{E_{1}}, P_{2} \in \meaningof{E_2}\} }
\end{mathpar}

\begin{mathpar}
 \inferrule* [lab=behavior] {} {\meaningof{\langle a?b \rangle E} = \{ P \in \pi | P \equiv Q | u?(y)P', \\ \and \\\\ \and \\ \;\;\; u \in \meaningof{a}, \forall z.P'\{z/y\} \in \meaningof{E\{z/b\}}\}, \and \\ \meaningof{a!E} = \{ P \in \pi | P \equiv Q | x!\langle P' \rangle, x \in \meaningof{a} P' \in \meaningof{E}\} }
\end{mathpar}

\begin{mathpar}
 \inferrule* [lab=nominal] {} {\meaningof{\quotep{E}} = \{ \quotep{P} \in \quotep{\pi} | P \in \meaningof{E} \}, \and \meaningof{\quotep{P}} = \{ \quotep{Q} \in \quotep{\pi} | P \equiv Q \} \and \\ \meaningof{@\quotep{E}} = \{ P \in \pi | P \equiv @x, x \in \meaningof{E} \}}
\end{mathpar}

\begin{eqnarray*}
  \\
  \meaningof{-} : TS \to ST
\end{eqnarray*}

\begin{eqnarray*}
  \\
  L : TS \to ST
\end{eqnarray*}

\begin{eqnarray*}
  \\
  P \models E \iff P \in \meaningof{E}
\end{eqnarray*}

\begin{eqnarray*}
  P \approx_{L} Q \iff \forall E \in L. P \models E \iff Q \models E
\end{eqnarray*}

\begin{eqnarray*}
  P \approx_{K} Q
\end{eqnarray*}

\begin{eqnarray*}
  P \approx Q
\end{eqnarray*}

$\approx_{K} = \approx = \approx_{L}$

\subsubsection{Contextual duality}

Note that contexts extend the quotation operation to a family of
operations from processes to names. Given a context, $M$, we can
define a \emph{nominal context}, $\quotep{M}$ by $\quotep{M}[P] :=
\quotep{M[P]}$. To foreshadow what is to come we observe that these
operations enjoy a duality with processes very much like the duality
between vectors and maps from vectors to scalars.

Further, because the calculus is essentially higher-order, we have a
correspondence between contexts and processes. More specifically,
given a name $x$ and a context $M$ we can construct $M^{*}_{x}$ such
that 

\begin{mathpar}
  M^{*}_{x} | \lift{x}{P} \red M[P]
\end{mathpar}

namely,

\begin{mathpar}
  M^{*}_{x} := x?(u).M[\dropn{u}]
\end{mathpar}

The dependence of $M^{*}_{x}$ on a name makes it an abstraction, 

\begin{mathpar}
  M^{*} := (x)x?(u).M[\dropn{u}]
\end{mathpar}

\subsection{Additional notation}

It will sometimes be convenient to denote the process a name
quotes. We already have the notation $x = \quotep{P}$, but it will be
convenient to introduce an alternate notation, $\procn{x}$, when we
want to emphasize the connection to the use of the name. Note that, by
virtue of name equivalence, $\quotep{\procn{x}} \nameeq x$; so, the
notation is consistent with previous definitions.

Further, because names have structure it is possible to effect
substitutions on the basis of that structure. This means we need to
upgrade our notation for substitutions, which we accomplish by
adapting comprehension notation. Thus,

\begin{mathpar}
  P\{ y / x : x \in S \}
\end{mathpar}

is interpreted to mean the process derived from P by replacing (in a
capture-avoiding manner) each occurrence of $x$ in $S$ by $y$. For example,

\begin{mathpar}
  P\{ \quotep{\procn{x}|\procn{x}} / x : x \in \freenames{P} \}
\end{mathpar}

will replace each (occurrence) of a free name $x$ in $P$ by
$\quotep{\procn{x}|\procn{x}}$.

Also, we will avail ourselves of the notation $x^{L}$ and $x^{R}$ to
denote injections of a name into disjoint copies of the name
space. There are numerous ways to accomplish this. One example can be
found in \cite{MeredithR05}. This notation overloads to vectors of
names: $\vec{x}^{\pi} := (x_{i}^{\pi} \; : \; 0 \leq i < |\vec{x}| )$ where $\pi \in \{L,R\}$.

We also use $P^{\Box} := P|\Box$.

In \cite{MeredithR05} an interpretation of the new operator is
given. It turns out that there are several possible interpretations
all enjoying the requisite algebraic properties of the operator (see
\cite{milner91polyadicpi}). We will therefore make liberal use of
$(\nu\; \vec{x})P$.

% subsection the_syntax_and_semantics_of_the_notation_system (end)   

\input{qm2pi.qmops} 

\input{qm2pi.sterngerlach} 

\input{qm2pi.metric} 

% section concurrent_process_calculi (end)

%\input{qm2pi.proofsketch}

% section proof sketch (end)

%\input{qm2pi.slviaknots} 

% section spatial logic via knots (end)

\input{qm2pi.conclusion}

% section conclusion (end)

%\input{qm2pi.dtcodes} 

% section wiring algorithm (end)

\input{qm2pi.ack} 

% section acknowledgments (end)

\newpage


\bibliographystyle{plain}   
\bibliography{../../biblios/main.bib}

\input{qm2pi.rhodetails}

\end{document}

 

\documentclass[12pt]{llncs}
%\documentclass{jktr}

\usepackage[pdftex]{hyperref}                   
\usepackage {listings}
\usepackage {mathpartir}
\usepackage{bcprules}
%\usepackage{listings}
                       
\usepackage{graphicx} 
%\usepackage[margins=2.5cm,nohead,nofoot]{geometry}
%\usepackage{geometry}
\usepackage{amsfonts}
\usepackage{amstext}
\usepackage{latexsym}
\usepackage{amssymb}
\usepackage{color}


%\include{myPreamble}
\include{qm2pi.local} 

%\ifpdf
%\usepackage[pdftex]{graphicx}
%\else
%\usepackage{graphicx}
%\fi

 % \ifpdf
%  \usepackage{pdfsync}
%  \if


%\title{Brief Article}
%\author{David F. Snyder}
%\author{L.G. Meredith}

%\address{Dept. of Math., Texas State University--San Marcos, San Marcos, TX 78666}
       
\pagestyle{empty}


\begin{document}

\lstset{language=[Objective]Caml,frame=shadowbox}

\input{qm2pi.front}

% section front matter (end)

\input{qm2pi.intro} 
 
% section introduction (end)

% \input{qm2pi.knotations} 

% section notation (end)

\input{qm2pi.process.calculi} 

% section concurrent_process_calculi_and_spatial_logics_ (end)
    
%\input{qm2pi.knots2pi} 

%\input{qm2pi.trefoil} 

%\input{qm2pi.mainthm} 

% subsection basic_interpretation (end)

%\input{qm2pi.rho.presentation} 
\subsection{The syntax and semantics of the notation system}\label{sub:the_syntax_and_semantics_of_the_notation_system} % (fold)

We now summarize a technical presentation of the calculus that
embodies our theory of dynamics. The typical presentation of such a
calculus follows the style of giving generators and relations on
them. The grammar, below, describing term constructors, freely
generates the set of processes, $\Proc$. This set is then quotiented
by a relation known as structural congruence and it is over this set
that the notion of dynamics is expressed. This presentation is
essentially that of \cite{MeredithR05} with the addition of
polyadicity and summation. For readability we have relegated some of
the technical subtleties to an appendix.

\subsubsection{Process grammar}\label{subsub:process_grammar}

\begin{mathpar}
  \inferrule* [lab=synchronization] {} {{M} \bc \pzero \;|\; x?F \;|\; x!C }
  \and
  \inferrule* [lab=abstraction] {} {{F} \bc (x)P}
  \and
  \inferrule* [lab=concretion] {} {{C} \bc \langle Q \rangle}
  \and
  \inferrule* [lab=process] {} {{P,Q} \bc M \;| \;P|Q \;|\; @{x}}
  \and
  \inferrule* [lab=name] {} {{x} \bc \quotep{P}}
\end{mathpar} 

Note that $\vec{x}$ (resp. $\vec{P}$) denotes a vector of names
(resp. processes) of length $|\vec{x}|$ (resp. $|\vec{P}|$). We adopt
the following useful abbreviations.

\begin{mathpar}
   x?(\vec{y}).P := x.(\vec{y})P \and  x\clift{\vec{P}} := x.\clift{\vec{P}}
   \and x!(y) := \lift{x}{\dropn{y}}
   \and \Pi_{i=0}^{n-1}P_i := P_0 | \ldots | P_{n-1}
\end{mathpar}

\subsubsection{Structural congruence}

\paragraph{Free and bound names and alpha-equivalence.} At the
core of structural equivalence is alpha-equivalence which identifies
process that are the same up to a change of variable. Formally, we
recognize the distinction between free and bound names. The free names
of a process, $\freenames{P}$, may be calculated recursively as
follows:

\begin{mathpar}
\freenames{\pzero} := \emptyset
  \and \\
  \freenames{x?(y).P} := \{ x \} \cup (\freenames{P} \setminus \{ y \})
  \and 
  \freenames{x!\langle P \rangle} := \{ x \} \cup \{ P \} 
  \and \\
  \freenames{P|Q} := \freenames{P} \cup \freenames{Q}
  \and \\
  \freenames{@{x}} := \{ x \}
\end{mathpar}

$\pi$
$\quotep{\pi}$

$\freenames{-} : \pi \to \mathcal{P}(\quotep{\pi})$

\begin{eqnarray*}
  \freenames{\pzero} & := & \emptyset \\
  \freenames{x?(y).P} & := & \{ x \} \cup (\freenames{P} \setminus \{ y \}) \\
  \freenames{x!\langle P \rangle} & := & \{ x \} \cup \{ P \} \\
  \freenames{P|Q} & := & \freenames{P} \cup \freenames{Q} \\
  \freenames{\dropn{x}} & := & \{ x \}
\end{eqnarray*}

The bound names of a process, $\boundnames{P}$, are those names occurring in $P$
that are not free. For example, in $x?(y).0$, the name $x$ is free, while $y$ is bound.

\begin{mathpar}
  \inferrule* [lab=monoidal-laws] {} { P|Q \equiv Q|P \and P|0 \equiv P \and P|(Q|R) \equiv (P|Q)|R }
\end{mathpar}

\begin{mathpar}
  \inferrule* [lab=alpha-equivalence] {} { (x)P \equiv (y)P\{y/x\} \and y \not\in \freenames{P} }
\end{mathpar}

\begin{definition}
Then two processes, $P,Q$, are alpha-equivalent if $P = Q\{\vec{y}/\vec{x}\}$ for
some $\vec{x} \in \boundnames{Q},\vec{y} \in \boundnames{P}$, where $Q\{\vec{y}/\vec{x}\}$
denotes the capture-avoiding substitution of $\vec{y}$ for $\vec{x}$ in $Q$.
\end{definition}

\begin{definition}
  The {\em structural congruence} \cite{SangiorgiWalker} , $\equiv$,
  between processes is the least congruence containing
  alpha-equivalence, satisfying the abelian monoid laws
  (associativity, commutativity and $\pzero$ as identity) for parallel
  composition $|$ and for summation $+$.
\end{definition}

\subsection{Name equivalence}

We take name equivalence, written $\nameeq$, to be the smallest
equivalence relation generated by the following rules.

\begin{mathpar}
\inferrule*[lab=Quote-drop]
{ }
{ \quotep{@{x}} \nameeq x }

\inferrule*[lab=Struct-equiv]
{ P \scong Q }
{ \quotep{P} \nameeq \quotep{Q} }
\end{mathpar}

The astute reader will have noticed that the mutual recursion of names
and processes imposes a mutual recursion on alpha-equivalence and
structural equivalence via name-equivalence. Fortunately, all of this
works out pleasantly and we may calculate in the natural way, free of
concern. The reader interested in the details is referred to the
appendix \ref{appendix:rho_details}.

\subsection{Substitution}

We use $\Proc$ for the set of processes, $\QProc$ for the set of
names, and $\id{\{}\vec{y} / \vec{x} \id{\}}$ to denote partial maps,
$s : \QProc \rightarrow \QProc$. A map, $s$ lifts, uniquely, to a map
on process terms, $\widehat{s} : \Proc \rightarrow \Proc$ by the
following equations.

\begin{mathpar}
  (0) \psubstp{Q}{P} := 0 \\
  (R \juxtap S) \psubstp{Q}{P}
  :=    
  (R)\psubstp{Q}{P} \juxtap (S) \psubstp{Q}{P} \\
  (x?(y).R) \psubstp{Q}{P}    
  :=    
  (x)\substp{Q}{P} (z)\concat( (R \psubstn{z}{y}) \psubstp{Q}{P} ) \\
  (\lift{x}{R}) \psubstp{Q}{P}  
  :=
  \lift{(x)\substp{Q}{P}}{ R \psubstp{Q}{P} } \\
%   (\dropn{x})  \psubstp{Q}{P}       
%   := 
%   \left\{ 
%     \begin{array}{ccc} 
%       \dropn{\quotep{Q}} & & x \nameeq \quotep{P} \\
%       \dropn{x} & & otherwise \\
%     \end{array}
%   \right. 
  (\dropn{x})  \psubstp{Q}{P}       
  := 
  \left\{ 
    \begin{array}{ccc} 
      Q & & x \nameeq \quotep{P} \\
      \dropn{x} & & otherwise \\
    \end{array}
  \right.
\end{mathpar}
 

where

\begin{eqnarray}
  (x)\id{\{} \lpquote Q \rpquote / \lpquote P \rpquote \id{\}}            = 
  \left\{ 
    \begin{array}{ccc}
      \lpquote Q \rpquote & & x \nameeq \lpquote P \rpquote \\
      x & & otherwise \\
    \end{array}
  \right. \nonumber
\end{eqnarray}

and $z$ is chosen distinct from $\quotep{P}$, $\quotep{Q}$, the free
names in $Q$, and all the names in $R$. Our $\alpha$-equivalence will
be built in the standard way from this substitution.

\begin{remark}\label{rem:no_self_referential_names}
  One consequence of these definitions is that $\forall P. \quotep{P}
  \not\in \freenames{P}$.
\end{remark}

\subsection{ Dynamic quote: an example }

Anticipating something of what's to come, consider applying the
substitution, $\widehat{\id{\{}u / z \id{\}}}$, to the following pair
of processes, $\lift{w}{y!(z)}$ and $w[ \lpquote y!(z) \rpquote ]$.

\begin{eqnarray}
	\lift{w}{y!(z)}\widehat{\id{\{}u / z \id{\}}}
		& = &
		\lift{w}{y!(u)} \nonumber\\
	w[ \lpquote y!(z) \rpquote ] \widehat{ \id{\{}u / z \id{\}} }
		& = &
		w[ \lpquote y!(z) \rpquote ] \nonumber
\end{eqnarray}

Because the body of the process between quotes is impervious to
substitution, we get radically different answers. In fact, by
examining the first process in an input context,
e.g. $x?(z).\lift{w}{y!(z)}$, we see that the process under the lift
operator may be shaped by prefixed inputs binding a name inside it. In
this sense, the lift operator will be seen as a way to dynamically
construct processes before reifying them as names.

Finally equipped with these standard features we can present the
dynamics of the calculus.

\subsubsection{Operational semantics} 

Finally, we introduce the computational dynamics. What marks these
algebras as distinct from other more traditionally studied algebraic
structures, e.g. vector spaces or polynomial rings, is the manner in
which dynamics is captured. In traditional structures, dynamics is typically
expressed through morphisms between such structures, as in linear maps
between vector spaces or morphisms between rings. In algebras
associated with the semantics of computation, the dynamics is
expressed as part of the algebraic structure itself, through a
reduction reduction relation typically denoted by $\red$. Below, we
give a recursive presentation of this relation for the calculus used
in the encoding.

$\red \subseteq \pi \times \pi$
$\red : \pi \to \mathcal{P}(\pi)$

\begin{mathpar}
  \inferrule* [lab=Comm] { \textsf{match}( x_{src}, x_{trgt} ) } { x_{trgt}?(y)P \; | \; x_{src}!\langle {Q} \rangle \red P\{\quotep{Q}/y}\} }
  \and \\
  \inferrule* [lab=Par] {{P} \red {P}'} {{{P} | {Q}} \red {{P}' | {Q}}}
  \and
  \inferrule* [lab=Equiv]{{{P} \scong {P}'} \andalso {{P}' \red {Q}'} \andalso {{Q}' \scong {Q}}}{{P} \red {Q}}
\end{mathpar}

\begin{eqnarray*}
  match_{\equiv} (\quotep{P},\quotep{Q}) & := & P \equiv Q \\
  match_{\dagger}(\quotep{P},\quotep{Q}) & := & \forall R. P|Q \red^{*} R => R \red^{*} 0 \\
  match_{K}(\quotep{P},\quotep{Q}) & := & K \mbox{ for some context } K
\end{eqnarray*}

$u?(x)P | u!\langle Q \rangle \red P\{\quotep{Q}/x\}$

%We write $\wred$ for $\red^*$, and $P\red$ if $\exists Q $ such that $ P \red Q$.
We write $P\red$ if $\exists Q $ such that $ P \red Q$ and $P\not\red$, otherwise.

\section{Replication}

As mentioned before, it is known that replication (and hence
recursion) can be implemented in a higher-order process algebra
\cite{SangiorgiWalker}. As our first example of calculation with the
machinery thus far presented we give the construction explicitly in
the {\rhoc}.

\begin{eqnarray}
	D_{x} & := & \prefix{x}{y}{(\binpar{\outputp{x}{y}}{@{y}})} \nonumber\\
	\bangp_{x}{P} & := & \binpar{{x}!\langle{\binpar{D_{x}}{P}}\rangle}{D_{x}} \nonumber
\end{eqnarray}

\begin{eqnarray}
	\bangp_{x}{P} & & \nonumber\\
	=
	& {x}!\langle{(\prefix{x}{y}{(\outputp{x}{y} | @{y})) | P}}\rangle 
	      | \prefix{x}{y}{(\outputp{x}{y} | @{y})} & \nonumber\\
	\red
	& (\outputp{x}{y} | @{y})\substn{\quotep{(\prefix{x}{y}{(@{y} | \outputp{x}{y})) | P}}}{y} & \nonumber\\
	=
	& \outputp{x}{\quotep{(\prefix{x}{y}{(\outputp{x}{y} | @{y})) | P}}}
	  | {(\prefix{x}{y}{(\outputp{x}{y} | @{y})) | P}} & \nonumber\\
	\red
	& \ldots & \nonumber\\
	\red^*
	& P | P | \ldots & \nonumber
\end{eqnarray}

Of course, this encoding, as an implementation, runs away, unfolding
$\bangp{P}$ eagerly. A lazier and more implementable replication
operator, restricted to input-guarded processes, may be obtained as follows.

\begin{eqnarray}
\bangp{\prefix{u}{v}{P}} 
	:= 
	\binpar{\lift{x}{\prefix{u}{v}{(\binpar{D(x)}{P})}}}{D(x)} \nonumber
\end{eqnarray}

\begin{remark}
  Note that the lazier definition still does not deal with summation
  or mixed summation (i.e. sums over input and output). The reader is
  invited to construct definitions of replication that deal with these
  features. 

  Further, the definitions are parameterized in a name, $x$. Can you,
  gentle reader, make a definition that eliminates this parameter and
  guarantees no accidental interaction between the replication
  machinery and the process being replicated -- i.e. no accidental
  sharing of names used by the process to get its work done and the
  name(s) used by the replication to effect copying. This latter
  revision of the definition of replication is crucial to obtaining
  the expected identity $!!P \sim !P$.
\end{remark}

\begin{remark}\label{rem:paradoxical_combinator}
  The reader familiar with the lambda calculus will have noticed the
  similarity between $D$ and the paradoxical combinator.

  [Ed. note: the existence of this seems to suggest we have to be more
  restrictive on the set of processes and names we admit if we are to
  support no-cloning.]
\end{remark}

\subsubsection{Bisimulation}

The computational dynamics gives rise to another kind of equivalence,
the equivalence of computational behavior. As previously mentioned
this is typically captured \emph{via} some form of bisimulation.

% The notion we use in this paper is weak barbed bisimulation
% \cite{milner91polyadicpi}.

The notion we use in this paper is derived from weak barbed
bisimulation \cite{milner91polyadicpi}. 

\begin{definition}
An \emph{observation relation}, $\downarrow_{\mathcal N}$, over a set
of names, $\mathcal N$, is the smallest relation satisfying the rules
below.

\infrule[Out-barb]{y \in {\mathcal N}, \; x \nameeq y}
		  {\outputp{x}{v} \downarrow_{\mathcal N} x}
\infrule[Par-barb]{\mbox{$P\downarrow_{\mathcal N} x$ or $Q\downarrow_{\mathcal N} x$}}
		  {\binpar{P}{Q} \downarrow_{\mathcal N} x}

We write $P \Downarrow_{\mathcal N} x$ if there is $Q$ such that 
$P \wred Q$ and $Q \downarrow_{\mathcal N} x$.
\end{definition}

\begin{definition}
%\label{def.bbisim}
An  ${\mathcal N}$-\emph{barbed bisimulation} over a set of names, ${\mathcal N}$, is a symmetric binary relation 
${\mathcal S}_{\mathcal N}$ between agents such that $P\rel{S}_{\mathcal N}Q$ implies:
\begin{enumerate}
\item If $P \red P'$ then $Q \wred Q'$ and $P'\rel{S}_{\mathcal N} Q'$.
\item If $P\downarrow_{\mathcal N} x$, then $Q\Downarrow_{\mathcal N} x$.
\end{enumerate}
$P$ is ${\mathcal N}$-barbed bisimilar to $Q$, written
$P \wbbisim_{\mathcal N} Q$, if $P \rel{S}_{\mathcal N} Q$ for some ${\mathcal N}$-barbed bisimulation ${\mathcal S}_{\mathcal N}$.
\end{definition}

$\mathcal{R} \subseteq \pi \times \pi$

$P \mathcal{R} Q => \forall P'. P \red P' \Rightarrow \exists Q'. Q \red Q', P' \mathcal{R} Q'$

$P \vdash x \Rightarrow Q \vdash x$

\begin{mathpar}
  \inferrule*[lab=Out-barb]{x \nameeq y}{{y}!\langle{Q}\rangle \vdash x}
  \and
  \inferrule*[lab=Par-barb]{\mbox{$P\vdash x$ or $Q\vdash x$}}{\binpar{P}{Q} \vdash x}
\end{mathpar}

\subsubsection{Contexts}

One of the principle advantages of computational calculi like the
$\pi$-calculus is a well-defined notion of context,
contextual-equivalence and a correlation between
contextual-equivalence and notions of bisimulation. The notion of
context allows the decomposition of a process into (sub-)process and
its syntactic environment, its context. Thus, a context may be
thought of as a process with a ``hole'' (written $\Box$) in it. The
application of a context $M$ to a process $P$, written $M[P]$, is
tantamount to filling the hole in $M$ with $P$. In this paper we do
not need the full weight of this theory, but do make use of the notion
of context in the proof the main theorem. 

\begin{mathpar}
  \inferrule* [lab=summation] {} {{M_{M},M_{N}} \bc \Box \;|\; x.M_{A} \;|\; M_{M}+M_{N}}
  \and
  \inferrule* [lab=agent] {} {{M_{A}} \bc (\vec{x})M_{P} \;| \; \clift{P_0,\ldots,M_{P},\ldots,P_N}}
  \and \\
  \inferrule* [lab=process] {} {{M_{P}} \bc M_{N} \;| \;P|M_{P} }
\end{mathpar} 

\begin{mathpar}
  \inferrule* [lab=sychronization] {} {M_{N} \bc \Box \;|\; x?M_{F} \;|\; x!M_{C}}
  \and
  \inferrule* [lab=abstraction] {} {{M_{F}} \bc (x)M_{P} }
  \and
  \inferrule* [lab=concretion] {} {{M_{C}} \bc \langle M_{P} \rangle }
  \and \\
  \inferrule* [lab=process] {} {{M_{P}} \bc M_{N} \;| \;P|M_{P} }
\end{mathpar}

\begin{definition}[contextual application] Given a context $M$, and
  process $P$, we define the \emph{contextual application}, $M[P] :=
  M\{P/\Box\}$. That is, the contextual application of M to P is the
  substitution of $P$ for $\Box$ in $M$.
\end{definition}

$\meaningof{-} : L \to \mathcal{P}(\pi)$

\begin{mathpar}
  \inferrule* [lab=collection] {} {\meaningof{true} = \pi, \and \meaningof{~E} = \pi \setminus \meaningof{E}, \and \meaningof{E_{1} \& E_{2}} = \meaningof{E_{1}} \cap \meaningof{E_{2}}}
\end{mathpar}

\begin{mathpar}
  \inferrule* [lab=structure] {} {\meaningof{0} = \{ P \in \pi | P \equiv 0 \}, \and \\ \meaningof{E_1 | E_2} = \{ P \in \pi | P \equiv P_{1} | P_{2}, P_{1} \in \meaningof{E_{1}}, P_{2} \in \meaningof{E_2}\} }
\end{mathpar}

\begin{mathpar}
 \inferrule* [lab=behavior] {} {\meaningof{\langle a?b \rangle E} = \{ P \in \pi | P \equiv Q | u?(y)P', \\ \and \\\\ \and \\ \;\;\; u \in \meaningof{a}, \forall z.P'\{z/y\} \in \meaningof{E\{z/b\}}\}, \and \\ \meaningof{a!E} = \{ P \in \pi | P \equiv Q | x!\langle P' \rangle, x \in \meaningof{a} P' \in \meaningof{E}\} }
\end{mathpar}

\begin{mathpar}
 \inferrule* [lab=nominal] {} {\meaningof{\quotep{E}} = \{ \quotep{P} \in \quotep{\pi} | P \in \meaningof{E} \}, \and \meaningof{\quotep{P}} = \{ \quotep{Q} \in \quotep{\pi} | P \equiv Q \} \and \\ \meaningof{@\quotep{E}} = \{ P \in \pi | P \equiv @x, x \in \meaningof{E} \}}
\end{mathpar}

\begin{eqnarray*}
  \\
  \meaningof{-} : TS \to ST
\end{eqnarray*}

\begin{eqnarray*}
  \\
  L : TS \to ST
\end{eqnarray*}

\begin{eqnarray*}
  \\
  P \models E \iff P \in \meaningof{E}
\end{eqnarray*}

\begin{eqnarray*}
  P \approx_{L} Q \iff \forall E \in L. P \models E \iff Q \models E
\end{eqnarray*}

\begin{eqnarray*}
  P \approx_{K} Q
\end{eqnarray*}

\begin{eqnarray*}
  P \approx Q
\end{eqnarray*}

$\approx_{K} = \approx = \approx_{L}$

\subsubsection{Contextual duality}

Note that contexts extend the quotation operation to a family of
operations from processes to names. Given a context, $M$, we can
define a \emph{nominal context}, $\quotep{M}$ by $\quotep{M}[P] :=
\quotep{M[P]}$. To foreshadow what is to come we observe that these
operations enjoy a duality with processes very much like the duality
between vectors and maps from vectors to scalars.

Further, because the calculus is essentially higher-order, we have a
correspondence between contexts and processes. More specifically,
given a name $x$ and a context $M$ we can construct $M^{*}_{x}$ such
that 

\begin{mathpar}
  M^{*}_{x} | \lift{x}{P} \red M[P]
\end{mathpar}

namely,

\begin{mathpar}
  M^{*}_{x} := x?(u).M[\dropn{u}]
\end{mathpar}

The dependence of $M^{*}_{x}$ on a name makes it an abstraction, 

\begin{mathpar}
  M^{*} := (x)x?(u).M[\dropn{u}]
\end{mathpar}

\subsection{Additional notation}

It will sometimes be convenient to denote the process a name
quotes. We already have the notation $x = \quotep{P}$, but it will be
convenient to introduce an alternate notation, $\procn{x}$, when we
want to emphasize the connection to the use of the name. Note that, by
virtue of name equivalence, $\quotep{\procn{x}} \nameeq x$; so, the
notation is consistent with previous definitions.

Further, because names have structure it is possible to effect
substitutions on the basis of that structure. This means we need to
upgrade our notation for substitutions, which we accomplish by
adapting comprehension notation. Thus,

\begin{mathpar}
  P\{ y / x : x \in S \}
\end{mathpar}

is interpreted to mean the process derived from P by replacing (in a
capture-avoiding manner) each occurrence of $x$ in $S$ by $y$. For example,

\begin{mathpar}
  P\{ \quotep{\procn{x}|\procn{x}} / x : x \in \freenames{P} \}
\end{mathpar}

will replace each (occurrence) of a free name $x$ in $P$ by
$\quotep{\procn{x}|\procn{x}}$.

Also, we will avail ourselves of the notation $x^{L}$ and $x^{R}$ to
denote injections of a name into disjoint copies of the name
space. There are numerous ways to accomplish this. One example can be
found in \cite{MeredithR05}. This notation overloads to vectors of
names: $\vec{x}^{\pi} := (x_{i}^{\pi} \; : \; 0 \leq i < |\vec{x}| )$ where $\pi \in \{L,R\}$.

We also use $P^{\Box} := P|\Box$.

In \cite{MeredithR05} an interpretation of the new operator is
given. It turns out that there are several possible interpretations
all enjoying the requisite algebraic properties of the operator (see
\cite{milner91polyadicpi}). We will therefore make liberal use of
$(\nu\; \vec{x})P$.

% subsection the_syntax_and_semantics_of_the_notation_system (end)   

\input{qm2pi.qmops} 

\input{qm2pi.sterngerlach} 

\input{qm2pi.metric} 

% section concurrent_process_calculi (end)

%\input{qm2pi.proofsketch}

% section proof sketch (end)

%\input{qm2pi.slviaknots} 

% section spatial logic via knots (end)

\input{qm2pi.conclusion}

% section conclusion (end)

%\input{qm2pi.dtcodes} 

% section wiring algorithm (end)

\input{qm2pi.ack} 

% section acknowledgments (end)

\newpage


\bibliographystyle{plain}   
\bibliography{../../biblios/main.bib}

\input{qm2pi.rhodetails}

\end{document}

 

% section concurrent_process_calculi (end)

%\documentclass[12pt]{llncs}
%\documentclass{jktr}

\usepackage[pdftex]{hyperref}                   
\usepackage {listings}
\usepackage {mathpartir}
\usepackage{bcprules}
%\usepackage{listings}
                       
\usepackage{graphicx} 
%\usepackage[margins=2.5cm,nohead,nofoot]{geometry}
%\usepackage{geometry}
\usepackage{amsfonts}
\usepackage{amstext}
\usepackage{latexsym}
\usepackage{amssymb}
\usepackage{color}


%\include{myPreamble}
\include{qm2pi.local} 

%\ifpdf
%\usepackage[pdftex]{graphicx}
%\else
%\usepackage{graphicx}
%\fi

 % \ifpdf
%  \usepackage{pdfsync}
%  \if


%\title{Brief Article}
%\author{David F. Snyder}
%\author{L.G. Meredith}

%\address{Dept. of Math., Texas State University--San Marcos, San Marcos, TX 78666}
       
\pagestyle{empty}


\begin{document}

\lstset{language=[Objective]Caml,frame=shadowbox}

\input{qm2pi.front}

% section front matter (end)

\input{qm2pi.intro} 
 
% section introduction (end)

% \input{qm2pi.knotations} 

% section notation (end)

\input{qm2pi.process.calculi} 

% section concurrent_process_calculi_and_spatial_logics_ (end)
    
%\input{qm2pi.knots2pi} 

%\input{qm2pi.trefoil} 

%\input{qm2pi.mainthm} 

% subsection basic_interpretation (end)

%\input{qm2pi.rho.presentation} 
\subsection{The syntax and semantics of the notation system}\label{sub:the_syntax_and_semantics_of_the_notation_system} % (fold)

We now summarize a technical presentation of the calculus that
embodies our theory of dynamics. The typical presentation of such a
calculus follows the style of giving generators and relations on
them. The grammar, below, describing term constructors, freely
generates the set of processes, $\Proc$. This set is then quotiented
by a relation known as structural congruence and it is over this set
that the notion of dynamics is expressed. This presentation is
essentially that of \cite{MeredithR05} with the addition of
polyadicity and summation. For readability we have relegated some of
the technical subtleties to an appendix.

\subsubsection{Process grammar}\label{subsub:process_grammar}

\begin{mathpar}
  \inferrule* [lab=synchronization] {} {{M} \bc \pzero \;|\; x?F \;|\; x!C }
  \and
  \inferrule* [lab=abstraction] {} {{F} \bc (x)P}
  \and
  \inferrule* [lab=concretion] {} {{C} \bc \langle Q \rangle}
  \and
  \inferrule* [lab=process] {} {{P,Q} \bc M \;| \;P|Q \;|\; @{x}}
  \and
  \inferrule* [lab=name] {} {{x} \bc \quotep{P}}
\end{mathpar} 

Note that $\vec{x}$ (resp. $\vec{P}$) denotes a vector of names
(resp. processes) of length $|\vec{x}|$ (resp. $|\vec{P}|$). We adopt
the following useful abbreviations.

\begin{mathpar}
   x?(\vec{y}).P := x.(\vec{y})P \and  x\clift{\vec{P}} := x.\clift{\vec{P}}
   \and x!(y) := \lift{x}{\dropn{y}}
   \and \Pi_{i=0}^{n-1}P_i := P_0 | \ldots | P_{n-1}
\end{mathpar}

\subsubsection{Structural congruence}

\paragraph{Free and bound names and alpha-equivalence.} At the
core of structural equivalence is alpha-equivalence which identifies
process that are the same up to a change of variable. Formally, we
recognize the distinction between free and bound names. The free names
of a process, $\freenames{P}$, may be calculated recursively as
follows:

\begin{mathpar}
\freenames{\pzero} := \emptyset
  \and \\
  \freenames{x?(y).P} := \{ x \} \cup (\freenames{P} \setminus \{ y \})
  \and 
  \freenames{x!\langle P \rangle} := \{ x \} \cup \{ P \} 
  \and \\
  \freenames{P|Q} := \freenames{P} \cup \freenames{Q}
  \and \\
  \freenames{@{x}} := \{ x \}
\end{mathpar}

$\pi$
$\quotep{\pi}$

$\freenames{-} : \pi \to \mathcal{P}(\quotep{\pi})$

\begin{eqnarray*}
  \freenames{\pzero} & := & \emptyset \\
  \freenames{x?(y).P} & := & \{ x \} \cup (\freenames{P} \setminus \{ y \}) \\
  \freenames{x!\langle P \rangle} & := & \{ x \} \cup \{ P \} \\
  \freenames{P|Q} & := & \freenames{P} \cup \freenames{Q} \\
  \freenames{\dropn{x}} & := & \{ x \}
\end{eqnarray*}

The bound names of a process, $\boundnames{P}$, are those names occurring in $P$
that are not free. For example, in $x?(y).0$, the name $x$ is free, while $y$ is bound.

\begin{mathpar}
  \inferrule* [lab=monoidal-laws] {} { P|Q \equiv Q|P \and P|0 \equiv P \and P|(Q|R) \equiv (P|Q)|R }
\end{mathpar}

\begin{mathpar}
  \inferrule* [lab=alpha-equivalence] {} { (x)P \equiv (y)P\{y/x\} \and y \not\in \freenames{P} }
\end{mathpar}

\begin{definition}
Then two processes, $P,Q$, are alpha-equivalent if $P = Q\{\vec{y}/\vec{x}\}$ for
some $\vec{x} \in \boundnames{Q},\vec{y} \in \boundnames{P}$, where $Q\{\vec{y}/\vec{x}\}$
denotes the capture-avoiding substitution of $\vec{y}$ for $\vec{x}$ in $Q$.
\end{definition}

\begin{definition}
  The {\em structural congruence} \cite{SangiorgiWalker} , $\equiv$,
  between processes is the least congruence containing
  alpha-equivalence, satisfying the abelian monoid laws
  (associativity, commutativity and $\pzero$ as identity) for parallel
  composition $|$ and for summation $+$.
\end{definition}

\subsection{Name equivalence}

We take name equivalence, written $\nameeq$, to be the smallest
equivalence relation generated by the following rules.

\begin{mathpar}
\inferrule*[lab=Quote-drop]
{ }
{ \quotep{@{x}} \nameeq x }

\inferrule*[lab=Struct-equiv]
{ P \scong Q }
{ \quotep{P} \nameeq \quotep{Q} }
\end{mathpar}

The astute reader will have noticed that the mutual recursion of names
and processes imposes a mutual recursion on alpha-equivalence and
structural equivalence via name-equivalence. Fortunately, all of this
works out pleasantly and we may calculate in the natural way, free of
concern. The reader interested in the details is referred to the
appendix \ref{appendix:rho_details}.

\subsection{Substitution}

We use $\Proc$ for the set of processes, $\QProc$ for the set of
names, and $\id{\{}\vec{y} / \vec{x} \id{\}}$ to denote partial maps,
$s : \QProc \rightarrow \QProc$. A map, $s$ lifts, uniquely, to a map
on process terms, $\widehat{s} : \Proc \rightarrow \Proc$ by the
following equations.

\begin{mathpar}
  (0) \psubstp{Q}{P} := 0 \\
  (R \juxtap S) \psubstp{Q}{P}
  :=    
  (R)\psubstp{Q}{P} \juxtap (S) \psubstp{Q}{P} \\
  (x?(y).R) \psubstp{Q}{P}    
  :=    
  (x)\substp{Q}{P} (z)\concat( (R \psubstn{z}{y}) \psubstp{Q}{P} ) \\
  (\lift{x}{R}) \psubstp{Q}{P}  
  :=
  \lift{(x)\substp{Q}{P}}{ R \psubstp{Q}{P} } \\
%   (\dropn{x})  \psubstp{Q}{P}       
%   := 
%   \left\{ 
%     \begin{array}{ccc} 
%       \dropn{\quotep{Q}} & & x \nameeq \quotep{P} \\
%       \dropn{x} & & otherwise \\
%     \end{array}
%   \right. 
  (\dropn{x})  \psubstp{Q}{P}       
  := 
  \left\{ 
    \begin{array}{ccc} 
      Q & & x \nameeq \quotep{P} \\
      \dropn{x} & & otherwise \\
    \end{array}
  \right.
\end{mathpar}
 

where

\begin{eqnarray}
  (x)\id{\{} \lpquote Q \rpquote / \lpquote P \rpquote \id{\}}            = 
  \left\{ 
    \begin{array}{ccc}
      \lpquote Q \rpquote & & x \nameeq \lpquote P \rpquote \\
      x & & otherwise \\
    \end{array}
  \right. \nonumber
\end{eqnarray}

and $z$ is chosen distinct from $\quotep{P}$, $\quotep{Q}$, the free
names in $Q$, and all the names in $R$. Our $\alpha$-equivalence will
be built in the standard way from this substitution.

\begin{remark}\label{rem:no_self_referential_names}
  One consequence of these definitions is that $\forall P. \quotep{P}
  \not\in \freenames{P}$.
\end{remark}

\subsection{ Dynamic quote: an example }

Anticipating something of what's to come, consider applying the
substitution, $\widehat{\id{\{}u / z \id{\}}}$, to the following pair
of processes, $\lift{w}{y!(z)}$ and $w[ \lpquote y!(z) \rpquote ]$.

\begin{eqnarray}
	\lift{w}{y!(z)}\widehat{\id{\{}u / z \id{\}}}
		& = &
		\lift{w}{y!(u)} \nonumber\\
	w[ \lpquote y!(z) \rpquote ] \widehat{ \id{\{}u / z \id{\}} }
		& = &
		w[ \lpquote y!(z) \rpquote ] \nonumber
\end{eqnarray}

Because the body of the process between quotes is impervious to
substitution, we get radically different answers. In fact, by
examining the first process in an input context,
e.g. $x?(z).\lift{w}{y!(z)}$, we see that the process under the lift
operator may be shaped by prefixed inputs binding a name inside it. In
this sense, the lift operator will be seen as a way to dynamically
construct processes before reifying them as names.

Finally equipped with these standard features we can present the
dynamics of the calculus.

\subsubsection{Operational semantics} 

Finally, we introduce the computational dynamics. What marks these
algebras as distinct from other more traditionally studied algebraic
structures, e.g. vector spaces or polynomial rings, is the manner in
which dynamics is captured. In traditional structures, dynamics is typically
expressed through morphisms between such structures, as in linear maps
between vector spaces or morphisms between rings. In algebras
associated with the semantics of computation, the dynamics is
expressed as part of the algebraic structure itself, through a
reduction reduction relation typically denoted by $\red$. Below, we
give a recursive presentation of this relation for the calculus used
in the encoding.

$\red \subseteq \pi \times \pi$
$\red : \pi \to \mathcal{P}(\pi)$

\begin{mathpar}
  \inferrule* [lab=Comm] { \textsf{match}( x_{src}, x_{trgt} ) } { x_{trgt}?(y)P \; | \; x_{src}!\langle {Q} \rangle \red P\{\quotep{Q}/y}\} }
  \and \\
  \inferrule* [lab=Par] {{P} \red {P}'} {{{P} | {Q}} \red {{P}' | {Q}}}
  \and
  \inferrule* [lab=Equiv]{{{P} \scong {P}'} \andalso {{P}' \red {Q}'} \andalso {{Q}' \scong {Q}}}{{P} \red {Q}}
\end{mathpar}

\begin{eqnarray*}
  match_{\equiv} (\quotep{P},\quotep{Q}) & := & P \equiv Q \\
  match_{\dagger}(\quotep{P},\quotep{Q}) & := & \forall R. P|Q \red^{*} R => R \red^{*} 0 \\
  match_{K}(\quotep{P},\quotep{Q}) & := & K \mbox{ for some context } K
\end{eqnarray*}

$u?(x)P | u!\langle Q \rangle \red P\{\quotep{Q}/x\}$

%We write $\wred$ for $\red^*$, and $P\red$ if $\exists Q $ such that $ P \red Q$.
We write $P\red$ if $\exists Q $ such that $ P \red Q$ and $P\not\red$, otherwise.

\section{Replication}

As mentioned before, it is known that replication (and hence
recursion) can be implemented in a higher-order process algebra
\cite{SangiorgiWalker}. As our first example of calculation with the
machinery thus far presented we give the construction explicitly in
the {\rhoc}.

\begin{eqnarray}
	D_{x} & := & \prefix{x}{y}{(\binpar{\outputp{x}{y}}{@{y}})} \nonumber\\
	\bangp_{x}{P} & := & \binpar{{x}!\langle{\binpar{D_{x}}{P}}\rangle}{D_{x}} \nonumber
\end{eqnarray}

\begin{eqnarray}
	\bangp_{x}{P} & & \nonumber\\
	=
	& {x}!\langle{(\prefix{x}{y}{(\outputp{x}{y} | @{y})) | P}}\rangle 
	      | \prefix{x}{y}{(\outputp{x}{y} | @{y})} & \nonumber\\
	\red
	& (\outputp{x}{y} | @{y})\substn{\quotep{(\prefix{x}{y}{(@{y} | \outputp{x}{y})) | P}}}{y} & \nonumber\\
	=
	& \outputp{x}{\quotep{(\prefix{x}{y}{(\outputp{x}{y} | @{y})) | P}}}
	  | {(\prefix{x}{y}{(\outputp{x}{y} | @{y})) | P}} & \nonumber\\
	\red
	& \ldots & \nonumber\\
	\red^*
	& P | P | \ldots & \nonumber
\end{eqnarray}

Of course, this encoding, as an implementation, runs away, unfolding
$\bangp{P}$ eagerly. A lazier and more implementable replication
operator, restricted to input-guarded processes, may be obtained as follows.

\begin{eqnarray}
\bangp{\prefix{u}{v}{P}} 
	:= 
	\binpar{\lift{x}{\prefix{u}{v}{(\binpar{D(x)}{P})}}}{D(x)} \nonumber
\end{eqnarray}

\begin{remark}
  Note that the lazier definition still does not deal with summation
  or mixed summation (i.e. sums over input and output). The reader is
  invited to construct definitions of replication that deal with these
  features. 

  Further, the definitions are parameterized in a name, $x$. Can you,
  gentle reader, make a definition that eliminates this parameter and
  guarantees no accidental interaction between the replication
  machinery and the process being replicated -- i.e. no accidental
  sharing of names used by the process to get its work done and the
  name(s) used by the replication to effect copying. This latter
  revision of the definition of replication is crucial to obtaining
  the expected identity $!!P \sim !P$.
\end{remark}

\begin{remark}\label{rem:paradoxical_combinator}
  The reader familiar with the lambda calculus will have noticed the
  similarity between $D$ and the paradoxical combinator.

  [Ed. note: the existence of this seems to suggest we have to be more
  restrictive on the set of processes and names we admit if we are to
  support no-cloning.]
\end{remark}

\subsubsection{Bisimulation}

The computational dynamics gives rise to another kind of equivalence,
the equivalence of computational behavior. As previously mentioned
this is typically captured \emph{via} some form of bisimulation.

% The notion we use in this paper is weak barbed bisimulation
% \cite{milner91polyadicpi}.

The notion we use in this paper is derived from weak barbed
bisimulation \cite{milner91polyadicpi}. 

\begin{definition}
An \emph{observation relation}, $\downarrow_{\mathcal N}$, over a set
of names, $\mathcal N$, is the smallest relation satisfying the rules
below.

\infrule[Out-barb]{y \in {\mathcal N}, \; x \nameeq y}
		  {\outputp{x}{v} \downarrow_{\mathcal N} x}
\infrule[Par-barb]{\mbox{$P\downarrow_{\mathcal N} x$ or $Q\downarrow_{\mathcal N} x$}}
		  {\binpar{P}{Q} \downarrow_{\mathcal N} x}

We write $P \Downarrow_{\mathcal N} x$ if there is $Q$ such that 
$P \wred Q$ and $Q \downarrow_{\mathcal N} x$.
\end{definition}

\begin{definition}
%\label{def.bbisim}
An  ${\mathcal N}$-\emph{barbed bisimulation} over a set of names, ${\mathcal N}$, is a symmetric binary relation 
${\mathcal S}_{\mathcal N}$ between agents such that $P\rel{S}_{\mathcal N}Q$ implies:
\begin{enumerate}
\item If $P \red P'$ then $Q \wred Q'$ and $P'\rel{S}_{\mathcal N} Q'$.
\item If $P\downarrow_{\mathcal N} x$, then $Q\Downarrow_{\mathcal N} x$.
\end{enumerate}
$P$ is ${\mathcal N}$-barbed bisimilar to $Q$, written
$P \wbbisim_{\mathcal N} Q$, if $P \rel{S}_{\mathcal N} Q$ for some ${\mathcal N}$-barbed bisimulation ${\mathcal S}_{\mathcal N}$.
\end{definition}

$\mathcal{R} \subseteq \pi \times \pi$

$P \mathcal{R} Q => \forall P'. P \red P' \Rightarrow \exists Q'. Q \red Q', P' \mathcal{R} Q'$

$P \vdash x \Rightarrow Q \vdash x$

\begin{mathpar}
  \inferrule*[lab=Out-barb]{x \nameeq y}{{y}!\langle{Q}\rangle \vdash x}
  \and
  \inferrule*[lab=Par-barb]{\mbox{$P\vdash x$ or $Q\vdash x$}}{\binpar{P}{Q} \vdash x}
\end{mathpar}

\subsubsection{Contexts}

One of the principle advantages of computational calculi like the
$\pi$-calculus is a well-defined notion of context,
contextual-equivalence and a correlation between
contextual-equivalence and notions of bisimulation. The notion of
context allows the decomposition of a process into (sub-)process and
its syntactic environment, its context. Thus, a context may be
thought of as a process with a ``hole'' (written $\Box$) in it. The
application of a context $M$ to a process $P$, written $M[P]$, is
tantamount to filling the hole in $M$ with $P$. In this paper we do
not need the full weight of this theory, but do make use of the notion
of context in the proof the main theorem. 

\begin{mathpar}
  \inferrule* [lab=summation] {} {{M_{M},M_{N}} \bc \Box \;|\; x.M_{A} \;|\; M_{M}+M_{N}}
  \and
  \inferrule* [lab=agent] {} {{M_{A}} \bc (\vec{x})M_{P} \;| \; \clift{P_0,\ldots,M_{P},\ldots,P_N}}
  \and \\
  \inferrule* [lab=process] {} {{M_{P}} \bc M_{N} \;| \;P|M_{P} }
\end{mathpar} 

\begin{mathpar}
  \inferrule* [lab=sychronization] {} {M_{N} \bc \Box \;|\; x?M_{F} \;|\; x!M_{C}}
  \and
  \inferrule* [lab=abstraction] {} {{M_{F}} \bc (x)M_{P} }
  \and
  \inferrule* [lab=concretion] {} {{M_{C}} \bc \langle M_{P} \rangle }
  \and \\
  \inferrule* [lab=process] {} {{M_{P}} \bc M_{N} \;| \;P|M_{P} }
\end{mathpar}

\begin{definition}[contextual application] Given a context $M$, and
  process $P$, we define the \emph{contextual application}, $M[P] :=
  M\{P/\Box\}$. That is, the contextual application of M to P is the
  substitution of $P$ for $\Box$ in $M$.
\end{definition}

$\meaningof{-} : L \to \mathcal{P}(\pi)$

\begin{mathpar}
  \inferrule* [lab=collection] {} {\meaningof{true} = \pi, \and \meaningof{~E} = \pi \setminus \meaningof{E}, \and \meaningof{E_{1} \& E_{2}} = \meaningof{E_{1}} \cap \meaningof{E_{2}}}
\end{mathpar}

\begin{mathpar}
  \inferrule* [lab=structure] {} {\meaningof{0} = \{ P \in \pi | P \equiv 0 \}, \and \\ \meaningof{E_1 | E_2} = \{ P \in \pi | P \equiv P_{1} | P_{2}, P_{1} \in \meaningof{E_{1}}, P_{2} \in \meaningof{E_2}\} }
\end{mathpar}

\begin{mathpar}
 \inferrule* [lab=behavior] {} {\meaningof{\langle a?b \rangle E} = \{ P \in \pi | P \equiv Q | u?(y)P', \\ \and \\\\ \and \\ \;\;\; u \in \meaningof{a}, \forall z.P'\{z/y\} \in \meaningof{E\{z/b\}}\}, \and \\ \meaningof{a!E} = \{ P \in \pi | P \equiv Q | x!\langle P' \rangle, x \in \meaningof{a} P' \in \meaningof{E}\} }
\end{mathpar}

\begin{mathpar}
 \inferrule* [lab=nominal] {} {\meaningof{\quotep{E}} = \{ \quotep{P} \in \quotep{\pi} | P \in \meaningof{E} \}, \and \meaningof{\quotep{P}} = \{ \quotep{Q} \in \quotep{\pi} | P \equiv Q \} \and \\ \meaningof{@\quotep{E}} = \{ P \in \pi | P \equiv @x, x \in \meaningof{E} \}}
\end{mathpar}

\begin{eqnarray*}
  \\
  \meaningof{-} : TS \to ST
\end{eqnarray*}

\begin{eqnarray*}
  \\
  L : TS \to ST
\end{eqnarray*}

\begin{eqnarray*}
  \\
  P \models E \iff P \in \meaningof{E}
\end{eqnarray*}

\begin{eqnarray*}
  P \approx_{L} Q \iff \forall E \in L. P \models E \iff Q \models E
\end{eqnarray*}

\begin{eqnarray*}
  P \approx_{K} Q
\end{eqnarray*}

\begin{eqnarray*}
  P \approx Q
\end{eqnarray*}

$\approx_{K} = \approx = \approx_{L}$

\subsubsection{Contextual duality}

Note that contexts extend the quotation operation to a family of
operations from processes to names. Given a context, $M$, we can
define a \emph{nominal context}, $\quotep{M}$ by $\quotep{M}[P] :=
\quotep{M[P]}$. To foreshadow what is to come we observe that these
operations enjoy a duality with processes very much like the duality
between vectors and maps from vectors to scalars.

Further, because the calculus is essentially higher-order, we have a
correspondence between contexts and processes. More specifically,
given a name $x$ and a context $M$ we can construct $M^{*}_{x}$ such
that 

\begin{mathpar}
  M^{*}_{x} | \lift{x}{P} \red M[P]
\end{mathpar}

namely,

\begin{mathpar}
  M^{*}_{x} := x?(u).M[\dropn{u}]
\end{mathpar}

The dependence of $M^{*}_{x}$ on a name makes it an abstraction, 

\begin{mathpar}
  M^{*} := (x)x?(u).M[\dropn{u}]
\end{mathpar}

\subsection{Additional notation}

It will sometimes be convenient to denote the process a name
quotes. We already have the notation $x = \quotep{P}$, but it will be
convenient to introduce an alternate notation, $\procn{x}$, when we
want to emphasize the connection to the use of the name. Note that, by
virtue of name equivalence, $\quotep{\procn{x}} \nameeq x$; so, the
notation is consistent with previous definitions.

Further, because names have structure it is possible to effect
substitutions on the basis of that structure. This means we need to
upgrade our notation for substitutions, which we accomplish by
adapting comprehension notation. Thus,

\begin{mathpar}
  P\{ y / x : x \in S \}
\end{mathpar}

is interpreted to mean the process derived from P by replacing (in a
capture-avoiding manner) each occurrence of $x$ in $S$ by $y$. For example,

\begin{mathpar}
  P\{ \quotep{\procn{x}|\procn{x}} / x : x \in \freenames{P} \}
\end{mathpar}

will replace each (occurrence) of a free name $x$ in $P$ by
$\quotep{\procn{x}|\procn{x}}$.

Also, we will avail ourselves of the notation $x^{L}$ and $x^{R}$ to
denote injections of a name into disjoint copies of the name
space. There are numerous ways to accomplish this. One example can be
found in \cite{MeredithR05}. This notation overloads to vectors of
names: $\vec{x}^{\pi} := (x_{i}^{\pi} \; : \; 0 \leq i < |\vec{x}| )$ where $\pi \in \{L,R\}$.

We also use $P^{\Box} := P|\Box$.

In \cite{MeredithR05} an interpretation of the new operator is
given. It turns out that there are several possible interpretations
all enjoying the requisite algebraic properties of the operator (see
\cite{milner91polyadicpi}). We will therefore make liberal use of
$(\nu\; \vec{x})P$.

% subsection the_syntax_and_semantics_of_the_notation_system (end)   

\input{qm2pi.qmops} 

\input{qm2pi.sterngerlach} 

\input{qm2pi.metric} 

% section concurrent_process_calculi (end)

%\input{qm2pi.proofsketch}

% section proof sketch (end)

%\input{qm2pi.slviaknots} 

% section spatial logic via knots (end)

\input{qm2pi.conclusion}

% section conclusion (end)

%\input{qm2pi.dtcodes} 

% section wiring algorithm (end)

\input{qm2pi.ack} 

% section acknowledgments (end)

\newpage


\bibliographystyle{plain}   
\bibliography{../../biblios/main.bib}

\input{qm2pi.rhodetails}

\end{document}



% section proof sketch (end)

%\section{Unlikely characters: spatial logic for
  knots}\label{sub:characteristic_formulae} % (fold)

Associated to the mobile process calculi are a family of logics known
as the Hennessy-Milner logics. These logics typically enjoy a
semantics interpreting formulae as sets of processes that when
factored through the encoding outlined above allows an identification
of classes of knots with logical formulae. In the context of this
encoding the sub-family known as the spatial logics \cite{CairesC03}
\cite{CairesC04} \cite{Caires04} are of particular interest providing
several important features for expressing and reasoning about
properties (i.e. classes) of knots. We hint here at how this may be done.

%\begin{description}
%\item [structural connectives] 
\subsubsection{Structural connectives} The spatial logics enjoy
structural connectives corresponding, at the logical level, to the
parallel composition ($P | Q$) and new name ($(\nu \; x)P$)
connectives for processes. As illustrated in the examples below, these
connectives are extremely expressive given the shape of our encoding.
%\item [decideable satisfaction]

\subsubsection{Decideable satisfaction}
In \cite{Caires04} the satisfaction relation is shown to be decideable
for a rich class of processes. It further turns out that the image of
the our encoding is a proper subset of that class. This result
provides the basis for an algorithm by which to search for knots
enjoying a given property.
%\item [characteristic formulae]

\subsubsection{Characteristic formulae}
In the same paper \cite{Caires04} , Caires presents a means of calculating
characteristic formulae, selecting equivalence classes of processes
up to a pre--specified depth limit on the support set of names. Composed with our
encoding, this characteristic formula can be used to select
characteristic formulae for knots.
%\end{description}

\subsubsection{Spatial logic formulae}

The grammar below (segmented for comprehension) summarizes the syntax
of spatial logic formulae. We employ illustrative examples in the
sequel to provide an intuitive understanding of their meaning
referring the reader to \cite{Caires04} for a more detailed explication
of the semantics.

\begin{mathpar}
  \inferrule* [lab=boolean] {} {{A,B} \bc T \;|\; \neg A \;|\; A \wedge B \;|\; \eta = \eta'}
  \and
  \inferrule* [lab=spatial] {} {|\; \pzero \;|\; A | B \;|\; x \text{\textregistered} A \;|\; \forall x . A \;|\;  H x . A}
  \and
  \inferrule* [lab=behavioral] {} {|\; \alpha . A}
  \and 
  \inferrule* [lab=recursion] {} {|\; X(\vec{u}) \;|\; \mu X(\vec{u}) . A}
  \and
  \inferrule* [lab=action] {} {\alpha \bc \langle x?(\vec{y}) \rangle \;|\; \langle x!(\vec{y}) \rangle \;|\; \langle \tau \rangle}
  \and 
  \inferrule* [lab=name] {} {\eta \bc x \;|\; \tau}
\end{mathpar} 

% subsection characteristic_formulae (end)   	 

\subsection{Example formulae}\label{sub:example_formulae_} % (fold)

\subsubsection{Crossing as formula.}
% 
% \begin{align*}
%   \frac{d}{dx} \sin x &= \cos x 
%   & \frac{d}{dx} e^x &= e^x \\
%   \frac{d}{dx} \cos x &= - \sin x 
%   & \frac{d}{dx} \log x &= \frac{1}{x} \\
% \end{align*} 

\begin{align*}
 \mu C(x_{0},x_{1},y_{0},y_{1},u).&(\langle x_{0}?(z) \rangle(\langle u! \rangle\langle y_{1}!z \rangle C(x_{0},x_{1},y_{0},y_{1},u)) & \\
  & \wedge \langle y_{1}?(z) \rangle (\langle u! \rangle \langle x_{0}!z \rangle C(x_{0},x_{1},y_{0},y_{1},u)) & \\
  & \wedge \langle x_{1}?(z) \rangle (\langle u? \rangle \langle y_{0}!z \rangle C(x_{0},x_{1},y_{0},y_{1},u)) & \\
  & \wedge \langle y_{0}?(z) \rangle (\langle u? \rangle \langle x_{1}!z \rangle C(x_{0},x_{1},y_{0},y_{1},u))) &
\end{align*}

The lexicographical similarity between the shape of this formulae and
the shape of definition of the process representing a crossing reveals
the intuitive meaning of this formulae. It describes the capabilities
of a process that has the right to represent a crossing. For example
it picks out processes that may perform an input on the port $x_0$ in
its initial menu of capabilities. What differentiates the formula
from the process, however, is that the crossing process is the
smallest candidate to satisfy the formula. Infinitely many other
processes -- with internal behavior hidden behind this interface, so
to speak -- also satisfy this formula. Even this simple formula,
then, can be seen to open a new view onto knots, providing a
computational interpretation of \emph{virtual} knots.

Note that this formula is derived by hand. A similar formula can be
derived by employing Caires' calculation of characteristic formula
\cite{Caires04} to the process representing a crossing. In light of
this discussion, we let
$\meaningof{C}_{\phi}(x0,x1,y0,y1,u)$ denote a formula specifying the
dynamics we wish to capture of a crossing. To guarantee we preserve
the shape of the interface and minimal semantics we demand that
$\meaningof{C}_{\phi}(x0,x1,y0,y1,u) \Rightarrow
\textbf{C}(x0,x1,y0,y1,u)$ where $\textbf{C}(x0,x1,y0,y1,u)$ denotes
the formula above.
                            
\subsubsection{Crossing number constraints.}
The moral content of the context lemma (Lemma \ref{context}) is that the notion of
``locality'' in the Reidemeister moves is effectively captured by the
parallel composition operator of the process calculus. This intuition
extends through the logic. Given a formula,
$\meaningof{C}_{\phi}(x0,x1,y0,y1,u)$, we can use the structural
connectives to specify constraints on crossing numbers, such as at
least $n$ crossings, or exactly $n$ crossings.
\begin{mathpar}
  \inferrule* [lab=at-least-n] {} { K^{\geq n}_{\phi}(\vec{xs},\vec{ys}) := \Pi_{i=0}^{n-1} Hu . \meaningof{C}_{\phi}(xs_i,ys_i,u) | T }
  \and 
  \inferrule* [lab=exactly-n] {} { K^{= n}_{\phi}(\vec{xs},\vec{ys}) := \Pi_{i=0}^{n-1} Hu . \meaningof{C}_{\phi}(xs_i,ys_i,u) | \neg (\forall x_0,y_0,x_1,y_1,u . \meaningof{C}_{\phi}(x_0,y_0,x_1,y_1,u) | T) }
\end{mathpar}

To round out this section, recall that the encoding of an $n$-crossing
knot decomposes into a parallel composition of $n$ \emph{copies} of a
crossing process together with a wiring harness. To specify different
knot classes with the same crossing number amounts to specifying
logical constraints on the wiring harness. In the interest of space,
we defer examples to a forthcoming paper. Suffice it to say that both
the conditions ``alternating knot'' and ``contains the tangle
corresponding to 5/3'' are expressible. For example, it is possible to
calculate the characteristic formula of a process corresponding to the
tangle 5/3 and conjoin it into the classifying formula via the
composition connective of the logic.

Finally, we wish to observe that it is entirely within reason to
contemplate a more domain-specific version of spatial logic tailored
to the shape of processes in the image of the encoding. Such a
domain-specific logic would have a better claim to the title formal
language of knot properties.

% subsection example_formulae_ (end)

% section knots_as_processes (end) 

% section spatial logic via knots (end)

\section{Conclusions and future work}

\paragraph{Testing physical space}
You, gentle reader, may wonder why of all the theorems to be proved
given this set up we pick the one above. In some sense it's hardly
central to quantum mechanics. We see it as central in the sense that
it firmly establishes a notion of physical space arising from a notion
of the equivalence of behavior. Relating bisimulation to a metric is a
big step forward, but one is faced with interpreting the relationship
of that metric space to something more physical. Quantum mechanical
notions of ``physical'' space are still far from intuitive, but by
relating this idea of distance as testing to calculations that predict
physical circumstances we are making a not insignificant step forward
toward an understanding of the physical space we inhabit as
essentially dynamic.

\paragraph{Effectivity and simulation}
One of the observations we have yet to make is that the entire program
spelled out here is effective. We have built various interpreters for
the reflective calculus at work in this interpretation. In principle,
then, we can simulate quantum mechanics on a computer. The place where
the simulation may lose fidelity is the infinitely branching summation
for the annihilator.

In this connection i also want to point out that the evaluation style
calculation of the inner product puts the non-determinism of the
summation right at the heart of measurement. This suggests that
Milner's original reduction-based formulation of the dynamics of his
calculi in terms of sums was not just notationally suggestive of a
notion of measure-and-continue but captured some significant part of
the physics.

\paragraph{Quantum continuations}
In light of this last observation i want to point out that the
predominant account of quantum mechanics is missing a key aspect of a
truly compositional story of the physical situation. In a real lab,
when a measurement is made the observation can be made to feed into
another device that then makes another measurement conditioned on the
results of the first. This means that after the superposition was
collapsed the entire experimental set up remained in
superposition. While QM offers a means of writing this down it doesn't
quite line up well with the well-trodden formulation of computation
and continuation that we see so succinctly expressed in Milner's
calculi. This suggests that there might be advantages to this account
of dynamics waiting to be explored.

\paragraph{Quantum logic}
In this connection, we also note that by virtue of having the
Hennessy-Milner construction, we can pull the construction through the
interpretation of QM. This gives us a natural candidate for a quantum
logic that enjoys an extremely tight connection with it's domain of
interpretation, making the construction much less ad hoc (rather it is
the image of functor!).

\paragraph{Quantum probabiity}
i have questions about the basis of the interpretation of inner
product as probability amplitude. In particular, using which
axiomatization of probability theory does the notion of probability
amplitude earn the right to be so dubbed? In other words, where is the
proof that the operation for calculating a probability amplitude (and
then squaring) satisfies the axioms of what it means to calculate a
probability? Even if such a proof exists (i have yet to find it in the
literature), i wonder if it might not be possible to turn things on
their heads. Can we view the calculation of the probability amplitude
as an axiomatization of probability? If so, then the definition we
give for calculating probability amplitude may provide the basis for
an \emph{effective} theory of probability.

\paragraph{Quantum vs ``biological'' information}
Finally, i want to conclude with a more philosophical observation. At
a recent workshop in which QM was a predominant topic i noticed
something about quantum information. The speaker was giving a riveting
discussion of axiomatic QM and showing how properties of ``no
cloning'' and ``no deleting'' emerged as consequences of the
axiomatization. Theorems of this form are necessary to give us a sense
of confidence that our axioms characterize the physical theory. What
struck me, though, was that if quantum information is neither erasable
nor replicable it is markedly different from \emph{life}. Two of the
things we know about life is that

\begin{itemize}
  \item it ends;
  \item to gain some measure of persistence, to transcend it's
    finitude it is imminently copyable.
\end{itemize}

Both of these qualities are summarized succinctly in the aphorism: all
flesh is grass. For me these two kinds of ``information'' -- call them
quantum and biological -- are end points on a spectrum of strategies
for persistence. At one end, we have those curious entities that enjoy
uniqueness and permanence; at the other, we have those who in the face
of a certain end and an uncertain present make a go of passing
something on. To me one of the more remarkable aspects of the latter
strategy is that in the presence of noise (and certain features of
copying) we get a kind of dynamism, a chance for improvement against a
given persistent condition.

% subsection other_calculi_other_bisimulations_and_geometry_as_behavior (end)




% section conclusion (end)

%\documentclass[12pt]{llncs}
%\documentclass{jktr}

\usepackage[pdftex]{hyperref}                   
\usepackage {listings}
\usepackage {mathpartir}
\usepackage{bcprules}
%\usepackage{listings}
                       
\usepackage{graphicx} 
%\usepackage[margins=2.5cm,nohead,nofoot]{geometry}
%\usepackage{geometry}
\usepackage{amsfonts}
\usepackage{amstext}
\usepackage{latexsym}
\usepackage{amssymb}
\usepackage{color}


%\include{myPreamble}
\include{qm2pi.local} 

%\ifpdf
%\usepackage[pdftex]{graphicx}
%\else
%\usepackage{graphicx}
%\fi

 % \ifpdf
%  \usepackage{pdfsync}
%  \if


%\title{Brief Article}
%\author{David F. Snyder}
%\author{L.G. Meredith}

%\address{Dept. of Math., Texas State University--San Marcos, San Marcos, TX 78666}
       
\pagestyle{empty}


\begin{document}

\lstset{language=[Objective]Caml,frame=shadowbox}

\input{qm2pi.front}

% section front matter (end)

\input{qm2pi.intro} 
 
% section introduction (end)

% \input{qm2pi.knotations} 

% section notation (end)

\input{qm2pi.process.calculi} 

% section concurrent_process_calculi_and_spatial_logics_ (end)
    
%\input{qm2pi.knots2pi} 

%\input{qm2pi.trefoil} 

%\input{qm2pi.mainthm} 

% subsection basic_interpretation (end)

%\input{qm2pi.rho.presentation} 
\subsection{The syntax and semantics of the notation system}\label{sub:the_syntax_and_semantics_of_the_notation_system} % (fold)

We now summarize a technical presentation of the calculus that
embodies our theory of dynamics. The typical presentation of such a
calculus follows the style of giving generators and relations on
them. The grammar, below, describing term constructors, freely
generates the set of processes, $\Proc$. This set is then quotiented
by a relation known as structural congruence and it is over this set
that the notion of dynamics is expressed. This presentation is
essentially that of \cite{MeredithR05} with the addition of
polyadicity and summation. For readability we have relegated some of
the technical subtleties to an appendix.

\subsubsection{Process grammar}\label{subsub:process_grammar}

\begin{mathpar}
  \inferrule* [lab=synchronization] {} {{M} \bc \pzero \;|\; x?F \;|\; x!C }
  \and
  \inferrule* [lab=abstraction] {} {{F} \bc (x)P}
  \and
  \inferrule* [lab=concretion] {} {{C} \bc \langle Q \rangle}
  \and
  \inferrule* [lab=process] {} {{P,Q} \bc M \;| \;P|Q \;|\; @{x}}
  \and
  \inferrule* [lab=name] {} {{x} \bc \quotep{P}}
\end{mathpar} 

Note that $\vec{x}$ (resp. $\vec{P}$) denotes a vector of names
(resp. processes) of length $|\vec{x}|$ (resp. $|\vec{P}|$). We adopt
the following useful abbreviations.

\begin{mathpar}
   x?(\vec{y}).P := x.(\vec{y})P \and  x\clift{\vec{P}} := x.\clift{\vec{P}}
   \and x!(y) := \lift{x}{\dropn{y}}
   \and \Pi_{i=0}^{n-1}P_i := P_0 | \ldots | P_{n-1}
\end{mathpar}

\subsubsection{Structural congruence}

\paragraph{Free and bound names and alpha-equivalence.} At the
core of structural equivalence is alpha-equivalence which identifies
process that are the same up to a change of variable. Formally, we
recognize the distinction between free and bound names. The free names
of a process, $\freenames{P}$, may be calculated recursively as
follows:

\begin{mathpar}
\freenames{\pzero} := \emptyset
  \and \\
  \freenames{x?(y).P} := \{ x \} \cup (\freenames{P} \setminus \{ y \})
  \and 
  \freenames{x!\langle P \rangle} := \{ x \} \cup \{ P \} 
  \and \\
  \freenames{P|Q} := \freenames{P} \cup \freenames{Q}
  \and \\
  \freenames{@{x}} := \{ x \}
\end{mathpar}

$\pi$
$\quotep{\pi}$

$\freenames{-} : \pi \to \mathcal{P}(\quotep{\pi})$

\begin{eqnarray*}
  \freenames{\pzero} & := & \emptyset \\
  \freenames{x?(y).P} & := & \{ x \} \cup (\freenames{P} \setminus \{ y \}) \\
  \freenames{x!\langle P \rangle} & := & \{ x \} \cup \{ P \} \\
  \freenames{P|Q} & := & \freenames{P} \cup \freenames{Q} \\
  \freenames{\dropn{x}} & := & \{ x \}
\end{eqnarray*}

The bound names of a process, $\boundnames{P}$, are those names occurring in $P$
that are not free. For example, in $x?(y).0$, the name $x$ is free, while $y$ is bound.

\begin{mathpar}
  \inferrule* [lab=monoidal-laws] {} { P|Q \equiv Q|P \and P|0 \equiv P \and P|(Q|R) \equiv (P|Q)|R }
\end{mathpar}

\begin{mathpar}
  \inferrule* [lab=alpha-equivalence] {} { (x)P \equiv (y)P\{y/x\} \and y \not\in \freenames{P} }
\end{mathpar}

\begin{definition}
Then two processes, $P,Q$, are alpha-equivalent if $P = Q\{\vec{y}/\vec{x}\}$ for
some $\vec{x} \in \boundnames{Q},\vec{y} \in \boundnames{P}$, where $Q\{\vec{y}/\vec{x}\}$
denotes the capture-avoiding substitution of $\vec{y}$ for $\vec{x}$ in $Q$.
\end{definition}

\begin{definition}
  The {\em structural congruence} \cite{SangiorgiWalker} , $\equiv$,
  between processes is the least congruence containing
  alpha-equivalence, satisfying the abelian monoid laws
  (associativity, commutativity and $\pzero$ as identity) for parallel
  composition $|$ and for summation $+$.
\end{definition}

\subsection{Name equivalence}

We take name equivalence, written $\nameeq$, to be the smallest
equivalence relation generated by the following rules.

\begin{mathpar}
\inferrule*[lab=Quote-drop]
{ }
{ \quotep{@{x}} \nameeq x }

\inferrule*[lab=Struct-equiv]
{ P \scong Q }
{ \quotep{P} \nameeq \quotep{Q} }
\end{mathpar}

The astute reader will have noticed that the mutual recursion of names
and processes imposes a mutual recursion on alpha-equivalence and
structural equivalence via name-equivalence. Fortunately, all of this
works out pleasantly and we may calculate in the natural way, free of
concern. The reader interested in the details is referred to the
appendix \ref{appendix:rho_details}.

\subsection{Substitution}

We use $\Proc$ for the set of processes, $\QProc$ for the set of
names, and $\id{\{}\vec{y} / \vec{x} \id{\}}$ to denote partial maps,
$s : \QProc \rightarrow \QProc$. A map, $s$ lifts, uniquely, to a map
on process terms, $\widehat{s} : \Proc \rightarrow \Proc$ by the
following equations.

\begin{mathpar}
  (0) \psubstp{Q}{P} := 0 \\
  (R \juxtap S) \psubstp{Q}{P}
  :=    
  (R)\psubstp{Q}{P} \juxtap (S) \psubstp{Q}{P} \\
  (x?(y).R) \psubstp{Q}{P}    
  :=    
  (x)\substp{Q}{P} (z)\concat( (R \psubstn{z}{y}) \psubstp{Q}{P} ) \\
  (\lift{x}{R}) \psubstp{Q}{P}  
  :=
  \lift{(x)\substp{Q}{P}}{ R \psubstp{Q}{P} } \\
%   (\dropn{x})  \psubstp{Q}{P}       
%   := 
%   \left\{ 
%     \begin{array}{ccc} 
%       \dropn{\quotep{Q}} & & x \nameeq \quotep{P} \\
%       \dropn{x} & & otherwise \\
%     \end{array}
%   \right. 
  (\dropn{x})  \psubstp{Q}{P}       
  := 
  \left\{ 
    \begin{array}{ccc} 
      Q & & x \nameeq \quotep{P} \\
      \dropn{x} & & otherwise \\
    \end{array}
  \right.
\end{mathpar}
 

where

\begin{eqnarray}
  (x)\id{\{} \lpquote Q \rpquote / \lpquote P \rpquote \id{\}}            = 
  \left\{ 
    \begin{array}{ccc}
      \lpquote Q \rpquote & & x \nameeq \lpquote P \rpquote \\
      x & & otherwise \\
    \end{array}
  \right. \nonumber
\end{eqnarray}

and $z$ is chosen distinct from $\quotep{P}$, $\quotep{Q}$, the free
names in $Q$, and all the names in $R$. Our $\alpha$-equivalence will
be built in the standard way from this substitution.

\begin{remark}\label{rem:no_self_referential_names}
  One consequence of these definitions is that $\forall P. \quotep{P}
  \not\in \freenames{P}$.
\end{remark}

\subsection{ Dynamic quote: an example }

Anticipating something of what's to come, consider applying the
substitution, $\widehat{\id{\{}u / z \id{\}}}$, to the following pair
of processes, $\lift{w}{y!(z)}$ and $w[ \lpquote y!(z) \rpquote ]$.

\begin{eqnarray}
	\lift{w}{y!(z)}\widehat{\id{\{}u / z \id{\}}}
		& = &
		\lift{w}{y!(u)} \nonumber\\
	w[ \lpquote y!(z) \rpquote ] \widehat{ \id{\{}u / z \id{\}} }
		& = &
		w[ \lpquote y!(z) \rpquote ] \nonumber
\end{eqnarray}

Because the body of the process between quotes is impervious to
substitution, we get radically different answers. In fact, by
examining the first process in an input context,
e.g. $x?(z).\lift{w}{y!(z)}$, we see that the process under the lift
operator may be shaped by prefixed inputs binding a name inside it. In
this sense, the lift operator will be seen as a way to dynamically
construct processes before reifying them as names.

Finally equipped with these standard features we can present the
dynamics of the calculus.

\subsubsection{Operational semantics} 

Finally, we introduce the computational dynamics. What marks these
algebras as distinct from other more traditionally studied algebraic
structures, e.g. vector spaces or polynomial rings, is the manner in
which dynamics is captured. In traditional structures, dynamics is typically
expressed through morphisms between such structures, as in linear maps
between vector spaces or morphisms between rings. In algebras
associated with the semantics of computation, the dynamics is
expressed as part of the algebraic structure itself, through a
reduction reduction relation typically denoted by $\red$. Below, we
give a recursive presentation of this relation for the calculus used
in the encoding.

$\red \subseteq \pi \times \pi$
$\red : \pi \to \mathcal{P}(\pi)$

\begin{mathpar}
  \inferrule* [lab=Comm] { \textsf{match}( x_{src}, x_{trgt} ) } { x_{trgt}?(y)P \; | \; x_{src}!\langle {Q} \rangle \red P\{\quotep{Q}/y}\} }
  \and \\
  \inferrule* [lab=Par] {{P} \red {P}'} {{{P} | {Q}} \red {{P}' | {Q}}}
  \and
  \inferrule* [lab=Equiv]{{{P} \scong {P}'} \andalso {{P}' \red {Q}'} \andalso {{Q}' \scong {Q}}}{{P} \red {Q}}
\end{mathpar}

\begin{eqnarray*}
  match_{\equiv} (\quotep{P},\quotep{Q}) & := & P \equiv Q \\
  match_{\dagger}(\quotep{P},\quotep{Q}) & := & \forall R. P|Q \red^{*} R => R \red^{*} 0 \\
  match_{K}(\quotep{P},\quotep{Q}) & := & K \mbox{ for some context } K
\end{eqnarray*}

$u?(x)P | u!\langle Q \rangle \red P\{\quotep{Q}/x\}$

%We write $\wred$ for $\red^*$, and $P\red$ if $\exists Q $ such that $ P \red Q$.
We write $P\red$ if $\exists Q $ such that $ P \red Q$ and $P\not\red$, otherwise.

\section{Replication}

As mentioned before, it is known that replication (and hence
recursion) can be implemented in a higher-order process algebra
\cite{SangiorgiWalker}. As our first example of calculation with the
machinery thus far presented we give the construction explicitly in
the {\rhoc}.

\begin{eqnarray}
	D_{x} & := & \prefix{x}{y}{(\binpar{\outputp{x}{y}}{@{y}})} \nonumber\\
	\bangp_{x}{P} & := & \binpar{{x}!\langle{\binpar{D_{x}}{P}}\rangle}{D_{x}} \nonumber
\end{eqnarray}

\begin{eqnarray}
	\bangp_{x}{P} & & \nonumber\\
	=
	& {x}!\langle{(\prefix{x}{y}{(\outputp{x}{y} | @{y})) | P}}\rangle 
	      | \prefix{x}{y}{(\outputp{x}{y} | @{y})} & \nonumber\\
	\red
	& (\outputp{x}{y} | @{y})\substn{\quotep{(\prefix{x}{y}{(@{y} | \outputp{x}{y})) | P}}}{y} & \nonumber\\
	=
	& \outputp{x}{\quotep{(\prefix{x}{y}{(\outputp{x}{y} | @{y})) | P}}}
	  | {(\prefix{x}{y}{(\outputp{x}{y} | @{y})) | P}} & \nonumber\\
	\red
	& \ldots & \nonumber\\
	\red^*
	& P | P | \ldots & \nonumber
\end{eqnarray}

Of course, this encoding, as an implementation, runs away, unfolding
$\bangp{P}$ eagerly. A lazier and more implementable replication
operator, restricted to input-guarded processes, may be obtained as follows.

\begin{eqnarray}
\bangp{\prefix{u}{v}{P}} 
	:= 
	\binpar{\lift{x}{\prefix{u}{v}{(\binpar{D(x)}{P})}}}{D(x)} \nonumber
\end{eqnarray}

\begin{remark}
  Note that the lazier definition still does not deal with summation
  or mixed summation (i.e. sums over input and output). The reader is
  invited to construct definitions of replication that deal with these
  features. 

  Further, the definitions are parameterized in a name, $x$. Can you,
  gentle reader, make a definition that eliminates this parameter and
  guarantees no accidental interaction between the replication
  machinery and the process being replicated -- i.e. no accidental
  sharing of names used by the process to get its work done and the
  name(s) used by the replication to effect copying. This latter
  revision of the definition of replication is crucial to obtaining
  the expected identity $!!P \sim !P$.
\end{remark}

\begin{remark}\label{rem:paradoxical_combinator}
  The reader familiar with the lambda calculus will have noticed the
  similarity between $D$ and the paradoxical combinator.

  [Ed. note: the existence of this seems to suggest we have to be more
  restrictive on the set of processes and names we admit if we are to
  support no-cloning.]
\end{remark}

\subsubsection{Bisimulation}

The computational dynamics gives rise to another kind of equivalence,
the equivalence of computational behavior. As previously mentioned
this is typically captured \emph{via} some form of bisimulation.

% The notion we use in this paper is weak barbed bisimulation
% \cite{milner91polyadicpi}.

The notion we use in this paper is derived from weak barbed
bisimulation \cite{milner91polyadicpi}. 

\begin{definition}
An \emph{observation relation}, $\downarrow_{\mathcal N}$, over a set
of names, $\mathcal N$, is the smallest relation satisfying the rules
below.

\infrule[Out-barb]{y \in {\mathcal N}, \; x \nameeq y}
		  {\outputp{x}{v} \downarrow_{\mathcal N} x}
\infrule[Par-barb]{\mbox{$P\downarrow_{\mathcal N} x$ or $Q\downarrow_{\mathcal N} x$}}
		  {\binpar{P}{Q} \downarrow_{\mathcal N} x}

We write $P \Downarrow_{\mathcal N} x$ if there is $Q$ such that 
$P \wred Q$ and $Q \downarrow_{\mathcal N} x$.
\end{definition}

\begin{definition}
%\label{def.bbisim}
An  ${\mathcal N}$-\emph{barbed bisimulation} over a set of names, ${\mathcal N}$, is a symmetric binary relation 
${\mathcal S}_{\mathcal N}$ between agents such that $P\rel{S}_{\mathcal N}Q$ implies:
\begin{enumerate}
\item If $P \red P'$ then $Q \wred Q'$ and $P'\rel{S}_{\mathcal N} Q'$.
\item If $P\downarrow_{\mathcal N} x$, then $Q\Downarrow_{\mathcal N} x$.
\end{enumerate}
$P$ is ${\mathcal N}$-barbed bisimilar to $Q$, written
$P \wbbisim_{\mathcal N} Q$, if $P \rel{S}_{\mathcal N} Q$ for some ${\mathcal N}$-barbed bisimulation ${\mathcal S}_{\mathcal N}$.
\end{definition}

$\mathcal{R} \subseteq \pi \times \pi$

$P \mathcal{R} Q => \forall P'. P \red P' \Rightarrow \exists Q'. Q \red Q', P' \mathcal{R} Q'$

$P \vdash x \Rightarrow Q \vdash x$

\begin{mathpar}
  \inferrule*[lab=Out-barb]{x \nameeq y}{{y}!\langle{Q}\rangle \vdash x}
  \and
  \inferrule*[lab=Par-barb]{\mbox{$P\vdash x$ or $Q\vdash x$}}{\binpar{P}{Q} \vdash x}
\end{mathpar}

\subsubsection{Contexts}

One of the principle advantages of computational calculi like the
$\pi$-calculus is a well-defined notion of context,
contextual-equivalence and a correlation between
contextual-equivalence and notions of bisimulation. The notion of
context allows the decomposition of a process into (sub-)process and
its syntactic environment, its context. Thus, a context may be
thought of as a process with a ``hole'' (written $\Box$) in it. The
application of a context $M$ to a process $P$, written $M[P]$, is
tantamount to filling the hole in $M$ with $P$. In this paper we do
not need the full weight of this theory, but do make use of the notion
of context in the proof the main theorem. 

\begin{mathpar}
  \inferrule* [lab=summation] {} {{M_{M},M_{N}} \bc \Box \;|\; x.M_{A} \;|\; M_{M}+M_{N}}
  \and
  \inferrule* [lab=agent] {} {{M_{A}} \bc (\vec{x})M_{P} \;| \; \clift{P_0,\ldots,M_{P},\ldots,P_N}}
  \and \\
  \inferrule* [lab=process] {} {{M_{P}} \bc M_{N} \;| \;P|M_{P} }
\end{mathpar} 

\begin{mathpar}
  \inferrule* [lab=sychronization] {} {M_{N} \bc \Box \;|\; x?M_{F} \;|\; x!M_{C}}
  \and
  \inferrule* [lab=abstraction] {} {{M_{F}} \bc (x)M_{P} }
  \and
  \inferrule* [lab=concretion] {} {{M_{C}} \bc \langle M_{P} \rangle }
  \and \\
  \inferrule* [lab=process] {} {{M_{P}} \bc M_{N} \;| \;P|M_{P} }
\end{mathpar}

\begin{definition}[contextual application] Given a context $M$, and
  process $P$, we define the \emph{contextual application}, $M[P] :=
  M\{P/\Box\}$. That is, the contextual application of M to P is the
  substitution of $P$ for $\Box$ in $M$.
\end{definition}

$\meaningof{-} : L \to \mathcal{P}(\pi)$

\begin{mathpar}
  \inferrule* [lab=collection] {} {\meaningof{true} = \pi, \and \meaningof{~E} = \pi \setminus \meaningof{E}, \and \meaningof{E_{1} \& E_{2}} = \meaningof{E_{1}} \cap \meaningof{E_{2}}}
\end{mathpar}

\begin{mathpar}
  \inferrule* [lab=structure] {} {\meaningof{0} = \{ P \in \pi | P \equiv 0 \}, \and \\ \meaningof{E_1 | E_2} = \{ P \in \pi | P \equiv P_{1} | P_{2}, P_{1} \in \meaningof{E_{1}}, P_{2} \in \meaningof{E_2}\} }
\end{mathpar}

\begin{mathpar}
 \inferrule* [lab=behavior] {} {\meaningof{\langle a?b \rangle E} = \{ P \in \pi | P \equiv Q | u?(y)P', \\ \and \\\\ \and \\ \;\;\; u \in \meaningof{a}, \forall z.P'\{z/y\} \in \meaningof{E\{z/b\}}\}, \and \\ \meaningof{a!E} = \{ P \in \pi | P \equiv Q | x!\langle P' \rangle, x \in \meaningof{a} P' \in \meaningof{E}\} }
\end{mathpar}

\begin{mathpar}
 \inferrule* [lab=nominal] {} {\meaningof{\quotep{E}} = \{ \quotep{P} \in \quotep{\pi} | P \in \meaningof{E} \}, \and \meaningof{\quotep{P}} = \{ \quotep{Q} \in \quotep{\pi} | P \equiv Q \} \and \\ \meaningof{@\quotep{E}} = \{ P \in \pi | P \equiv @x, x \in \meaningof{E} \}}
\end{mathpar}

\begin{eqnarray*}
  \\
  \meaningof{-} : TS \to ST
\end{eqnarray*}

\begin{eqnarray*}
  \\
  L : TS \to ST
\end{eqnarray*}

\begin{eqnarray*}
  \\
  P \models E \iff P \in \meaningof{E}
\end{eqnarray*}

\begin{eqnarray*}
  P \approx_{L} Q \iff \forall E \in L. P \models E \iff Q \models E
\end{eqnarray*}

\begin{eqnarray*}
  P \approx_{K} Q
\end{eqnarray*}

\begin{eqnarray*}
  P \approx Q
\end{eqnarray*}

$\approx_{K} = \approx = \approx_{L}$

\subsubsection{Contextual duality}

Note that contexts extend the quotation operation to a family of
operations from processes to names. Given a context, $M$, we can
define a \emph{nominal context}, $\quotep{M}$ by $\quotep{M}[P] :=
\quotep{M[P]}$. To foreshadow what is to come we observe that these
operations enjoy a duality with processes very much like the duality
between vectors and maps from vectors to scalars.

Further, because the calculus is essentially higher-order, we have a
correspondence between contexts and processes. More specifically,
given a name $x$ and a context $M$ we can construct $M^{*}_{x}$ such
that 

\begin{mathpar}
  M^{*}_{x} | \lift{x}{P} \red M[P]
\end{mathpar}

namely,

\begin{mathpar}
  M^{*}_{x} := x?(u).M[\dropn{u}]
\end{mathpar}

The dependence of $M^{*}_{x}$ on a name makes it an abstraction, 

\begin{mathpar}
  M^{*} := (x)x?(u).M[\dropn{u}]
\end{mathpar}

\subsection{Additional notation}

It will sometimes be convenient to denote the process a name
quotes. We already have the notation $x = \quotep{P}$, but it will be
convenient to introduce an alternate notation, $\procn{x}$, when we
want to emphasize the connection to the use of the name. Note that, by
virtue of name equivalence, $\quotep{\procn{x}} \nameeq x$; so, the
notation is consistent with previous definitions.

Further, because names have structure it is possible to effect
substitutions on the basis of that structure. This means we need to
upgrade our notation for substitutions, which we accomplish by
adapting comprehension notation. Thus,

\begin{mathpar}
  P\{ y / x : x \in S \}
\end{mathpar}

is interpreted to mean the process derived from P by replacing (in a
capture-avoiding manner) each occurrence of $x$ in $S$ by $y$. For example,

\begin{mathpar}
  P\{ \quotep{\procn{x}|\procn{x}} / x : x \in \freenames{P} \}
\end{mathpar}

will replace each (occurrence) of a free name $x$ in $P$ by
$\quotep{\procn{x}|\procn{x}}$.

Also, we will avail ourselves of the notation $x^{L}$ and $x^{R}$ to
denote injections of a name into disjoint copies of the name
space. There are numerous ways to accomplish this. One example can be
found in \cite{MeredithR05}. This notation overloads to vectors of
names: $\vec{x}^{\pi} := (x_{i}^{\pi} \; : \; 0 \leq i < |\vec{x}| )$ where $\pi \in \{L,R\}$.

We also use $P^{\Box} := P|\Box$.

In \cite{MeredithR05} an interpretation of the new operator is
given. It turns out that there are several possible interpretations
all enjoying the requisite algebraic properties of the operator (see
\cite{milner91polyadicpi}). We will therefore make liberal use of
$(\nu\; \vec{x})P$.

% subsection the_syntax_and_semantics_of_the_notation_system (end)   

\input{qm2pi.qmops} 

\input{qm2pi.sterngerlach} 

\input{qm2pi.metric} 

% section concurrent_process_calculi (end)

%\input{qm2pi.proofsketch}

% section proof sketch (end)

%\input{qm2pi.slviaknots} 

% section spatial logic via knots (end)

\input{qm2pi.conclusion}

% section conclusion (end)

%\input{qm2pi.dtcodes} 

% section wiring algorithm (end)

\input{qm2pi.ack} 

% section acknowledgments (end)

\newpage


\bibliographystyle{plain}   
\bibliography{../../biblios/main.bib}

\input{qm2pi.rhodetails}

\end{document}

 

% section wiring algorithm (end)

\documentclass[12pt]{llncs}
%\documentclass{jktr}

\usepackage[pdftex]{hyperref}                   
\usepackage {listings}
\usepackage {mathpartir}
\usepackage{bcprules}
%\usepackage{listings}
                       
\usepackage{graphicx} 
%\usepackage[margins=2.5cm,nohead,nofoot]{geometry}
%\usepackage{geometry}
\usepackage{amsfonts}
\usepackage{amstext}
\usepackage{latexsym}
\usepackage{amssymb}
\usepackage{color}


%\include{myPreamble}
\include{qm2pi.local} 

%\ifpdf
%\usepackage[pdftex]{graphicx}
%\else
%\usepackage{graphicx}
%\fi

 % \ifpdf
%  \usepackage{pdfsync}
%  \if


%\title{Brief Article}
%\author{David F. Snyder}
%\author{L.G. Meredith}

%\address{Dept. of Math., Texas State University--San Marcos, San Marcos, TX 78666}
       
\pagestyle{empty}


\begin{document}

\lstset{language=[Objective]Caml,frame=shadowbox}

\input{qm2pi.front}

% section front matter (end)

\input{qm2pi.intro} 
 
% section introduction (end)

% \input{qm2pi.knotations} 

% section notation (end)

\input{qm2pi.process.calculi} 

% section concurrent_process_calculi_and_spatial_logics_ (end)
    
%\input{qm2pi.knots2pi} 

%\input{qm2pi.trefoil} 

%\input{qm2pi.mainthm} 

% subsection basic_interpretation (end)

%\input{qm2pi.rho.presentation} 
\subsection{The syntax and semantics of the notation system}\label{sub:the_syntax_and_semantics_of_the_notation_system} % (fold)

We now summarize a technical presentation of the calculus that
embodies our theory of dynamics. The typical presentation of such a
calculus follows the style of giving generators and relations on
them. The grammar, below, describing term constructors, freely
generates the set of processes, $\Proc$. This set is then quotiented
by a relation known as structural congruence and it is over this set
that the notion of dynamics is expressed. This presentation is
essentially that of \cite{MeredithR05} with the addition of
polyadicity and summation. For readability we have relegated some of
the technical subtleties to an appendix.

\subsubsection{Process grammar}\label{subsub:process_grammar}

\begin{mathpar}
  \inferrule* [lab=synchronization] {} {{M} \bc \pzero \;|\; x?F \;|\; x!C }
  \and
  \inferrule* [lab=abstraction] {} {{F} \bc (x)P}
  \and
  \inferrule* [lab=concretion] {} {{C} \bc \langle Q \rangle}
  \and
  \inferrule* [lab=process] {} {{P,Q} \bc M \;| \;P|Q \;|\; @{x}}
  \and
  \inferrule* [lab=name] {} {{x} \bc \quotep{P}}
\end{mathpar} 

Note that $\vec{x}$ (resp. $\vec{P}$) denotes a vector of names
(resp. processes) of length $|\vec{x}|$ (resp. $|\vec{P}|$). We adopt
the following useful abbreviations.

\begin{mathpar}
   x?(\vec{y}).P := x.(\vec{y})P \and  x\clift{\vec{P}} := x.\clift{\vec{P}}
   \and x!(y) := \lift{x}{\dropn{y}}
   \and \Pi_{i=0}^{n-1}P_i := P_0 | \ldots | P_{n-1}
\end{mathpar}

\subsubsection{Structural congruence}

\paragraph{Free and bound names and alpha-equivalence.} At the
core of structural equivalence is alpha-equivalence which identifies
process that are the same up to a change of variable. Formally, we
recognize the distinction between free and bound names. The free names
of a process, $\freenames{P}$, may be calculated recursively as
follows:

\begin{mathpar}
\freenames{\pzero} := \emptyset
  \and \\
  \freenames{x?(y).P} := \{ x \} \cup (\freenames{P} \setminus \{ y \})
  \and 
  \freenames{x!\langle P \rangle} := \{ x \} \cup \{ P \} 
  \and \\
  \freenames{P|Q} := \freenames{P} \cup \freenames{Q}
  \and \\
  \freenames{@{x}} := \{ x \}
\end{mathpar}

$\pi$
$\quotep{\pi}$

$\freenames{-} : \pi \to \mathcal{P}(\quotep{\pi})$

\begin{eqnarray*}
  \freenames{\pzero} & := & \emptyset \\
  \freenames{x?(y).P} & := & \{ x \} \cup (\freenames{P} \setminus \{ y \}) \\
  \freenames{x!\langle P \rangle} & := & \{ x \} \cup \{ P \} \\
  \freenames{P|Q} & := & \freenames{P} \cup \freenames{Q} \\
  \freenames{\dropn{x}} & := & \{ x \}
\end{eqnarray*}

The bound names of a process, $\boundnames{P}$, are those names occurring in $P$
that are not free. For example, in $x?(y).0$, the name $x$ is free, while $y$ is bound.

\begin{mathpar}
  \inferrule* [lab=monoidal-laws] {} { P|Q \equiv Q|P \and P|0 \equiv P \and P|(Q|R) \equiv (P|Q)|R }
\end{mathpar}

\begin{mathpar}
  \inferrule* [lab=alpha-equivalence] {} { (x)P \equiv (y)P\{y/x\} \and y \not\in \freenames{P} }
\end{mathpar}

\begin{definition}
Then two processes, $P,Q$, are alpha-equivalent if $P = Q\{\vec{y}/\vec{x}\}$ for
some $\vec{x} \in \boundnames{Q},\vec{y} \in \boundnames{P}$, where $Q\{\vec{y}/\vec{x}\}$
denotes the capture-avoiding substitution of $\vec{y}$ for $\vec{x}$ in $Q$.
\end{definition}

\begin{definition}
  The {\em structural congruence} \cite{SangiorgiWalker} , $\equiv$,
  between processes is the least congruence containing
  alpha-equivalence, satisfying the abelian monoid laws
  (associativity, commutativity and $\pzero$ as identity) for parallel
  composition $|$ and for summation $+$.
\end{definition}

\subsection{Name equivalence}

We take name equivalence, written $\nameeq$, to be the smallest
equivalence relation generated by the following rules.

\begin{mathpar}
\inferrule*[lab=Quote-drop]
{ }
{ \quotep{@{x}} \nameeq x }

\inferrule*[lab=Struct-equiv]
{ P \scong Q }
{ \quotep{P} \nameeq \quotep{Q} }
\end{mathpar}

The astute reader will have noticed that the mutual recursion of names
and processes imposes a mutual recursion on alpha-equivalence and
structural equivalence via name-equivalence. Fortunately, all of this
works out pleasantly and we may calculate in the natural way, free of
concern. The reader interested in the details is referred to the
appendix \ref{appendix:rho_details}.

\subsection{Substitution}

We use $\Proc$ for the set of processes, $\QProc$ for the set of
names, and $\id{\{}\vec{y} / \vec{x} \id{\}}$ to denote partial maps,
$s : \QProc \rightarrow \QProc$. A map, $s$ lifts, uniquely, to a map
on process terms, $\widehat{s} : \Proc \rightarrow \Proc$ by the
following equations.

\begin{mathpar}
  (0) \psubstp{Q}{P} := 0 \\
  (R \juxtap S) \psubstp{Q}{P}
  :=    
  (R)\psubstp{Q}{P} \juxtap (S) \psubstp{Q}{P} \\
  (x?(y).R) \psubstp{Q}{P}    
  :=    
  (x)\substp{Q}{P} (z)\concat( (R \psubstn{z}{y}) \psubstp{Q}{P} ) \\
  (\lift{x}{R}) \psubstp{Q}{P}  
  :=
  \lift{(x)\substp{Q}{P}}{ R \psubstp{Q}{P} } \\
%   (\dropn{x})  \psubstp{Q}{P}       
%   := 
%   \left\{ 
%     \begin{array}{ccc} 
%       \dropn{\quotep{Q}} & & x \nameeq \quotep{P} \\
%       \dropn{x} & & otherwise \\
%     \end{array}
%   \right. 
  (\dropn{x})  \psubstp{Q}{P}       
  := 
  \left\{ 
    \begin{array}{ccc} 
      Q & & x \nameeq \quotep{P} \\
      \dropn{x} & & otherwise \\
    \end{array}
  \right.
\end{mathpar}
 

where

\begin{eqnarray}
  (x)\id{\{} \lpquote Q \rpquote / \lpquote P \rpquote \id{\}}            = 
  \left\{ 
    \begin{array}{ccc}
      \lpquote Q \rpquote & & x \nameeq \lpquote P \rpquote \\
      x & & otherwise \\
    \end{array}
  \right. \nonumber
\end{eqnarray}

and $z$ is chosen distinct from $\quotep{P}$, $\quotep{Q}$, the free
names in $Q$, and all the names in $R$. Our $\alpha$-equivalence will
be built in the standard way from this substitution.

\begin{remark}\label{rem:no_self_referential_names}
  One consequence of these definitions is that $\forall P. \quotep{P}
  \not\in \freenames{P}$.
\end{remark}

\subsection{ Dynamic quote: an example }

Anticipating something of what's to come, consider applying the
substitution, $\widehat{\id{\{}u / z \id{\}}}$, to the following pair
of processes, $\lift{w}{y!(z)}$ and $w[ \lpquote y!(z) \rpquote ]$.

\begin{eqnarray}
	\lift{w}{y!(z)}\widehat{\id{\{}u / z \id{\}}}
		& = &
		\lift{w}{y!(u)} \nonumber\\
	w[ \lpquote y!(z) \rpquote ] \widehat{ \id{\{}u / z \id{\}} }
		& = &
		w[ \lpquote y!(z) \rpquote ] \nonumber
\end{eqnarray}

Because the body of the process between quotes is impervious to
substitution, we get radically different answers. In fact, by
examining the first process in an input context,
e.g. $x?(z).\lift{w}{y!(z)}$, we see that the process under the lift
operator may be shaped by prefixed inputs binding a name inside it. In
this sense, the lift operator will be seen as a way to dynamically
construct processes before reifying them as names.

Finally equipped with these standard features we can present the
dynamics of the calculus.

\subsubsection{Operational semantics} 

Finally, we introduce the computational dynamics. What marks these
algebras as distinct from other more traditionally studied algebraic
structures, e.g. vector spaces or polynomial rings, is the manner in
which dynamics is captured. In traditional structures, dynamics is typically
expressed through morphisms between such structures, as in linear maps
between vector spaces or morphisms between rings. In algebras
associated with the semantics of computation, the dynamics is
expressed as part of the algebraic structure itself, through a
reduction reduction relation typically denoted by $\red$. Below, we
give a recursive presentation of this relation for the calculus used
in the encoding.

$\red \subseteq \pi \times \pi$
$\red : \pi \to \mathcal{P}(\pi)$

\begin{mathpar}
  \inferrule* [lab=Comm] { \textsf{match}( x_{src}, x_{trgt} ) } { x_{trgt}?(y)P \; | \; x_{src}!\langle {Q} \rangle \red P\{\quotep{Q}/y}\} }
  \and \\
  \inferrule* [lab=Par] {{P} \red {P}'} {{{P} | {Q}} \red {{P}' | {Q}}}
  \and
  \inferrule* [lab=Equiv]{{{P} \scong {P}'} \andalso {{P}' \red {Q}'} \andalso {{Q}' \scong {Q}}}{{P} \red {Q}}
\end{mathpar}

\begin{eqnarray*}
  match_{\equiv} (\quotep{P},\quotep{Q}) & := & P \equiv Q \\
  match_{\dagger}(\quotep{P},\quotep{Q}) & := & \forall R. P|Q \red^{*} R => R \red^{*} 0 \\
  match_{K}(\quotep{P},\quotep{Q}) & := & K \mbox{ for some context } K
\end{eqnarray*}

$u?(x)P | u!\langle Q \rangle \red P\{\quotep{Q}/x\}$

%We write $\wred$ for $\red^*$, and $P\red$ if $\exists Q $ such that $ P \red Q$.
We write $P\red$ if $\exists Q $ such that $ P \red Q$ and $P\not\red$, otherwise.

\section{Replication}

As mentioned before, it is known that replication (and hence
recursion) can be implemented in a higher-order process algebra
\cite{SangiorgiWalker}. As our first example of calculation with the
machinery thus far presented we give the construction explicitly in
the {\rhoc}.

\begin{eqnarray}
	D_{x} & := & \prefix{x}{y}{(\binpar{\outputp{x}{y}}{@{y}})} \nonumber\\
	\bangp_{x}{P} & := & \binpar{{x}!\langle{\binpar{D_{x}}{P}}\rangle}{D_{x}} \nonumber
\end{eqnarray}

\begin{eqnarray}
	\bangp_{x}{P} & & \nonumber\\
	=
	& {x}!\langle{(\prefix{x}{y}{(\outputp{x}{y} | @{y})) | P}}\rangle 
	      | \prefix{x}{y}{(\outputp{x}{y} | @{y})} & \nonumber\\
	\red
	& (\outputp{x}{y} | @{y})\substn{\quotep{(\prefix{x}{y}{(@{y} | \outputp{x}{y})) | P}}}{y} & \nonumber\\
	=
	& \outputp{x}{\quotep{(\prefix{x}{y}{(\outputp{x}{y} | @{y})) | P}}}
	  | {(\prefix{x}{y}{(\outputp{x}{y} | @{y})) | P}} & \nonumber\\
	\red
	& \ldots & \nonumber\\
	\red^*
	& P | P | \ldots & \nonumber
\end{eqnarray}

Of course, this encoding, as an implementation, runs away, unfolding
$\bangp{P}$ eagerly. A lazier and more implementable replication
operator, restricted to input-guarded processes, may be obtained as follows.

\begin{eqnarray}
\bangp{\prefix{u}{v}{P}} 
	:= 
	\binpar{\lift{x}{\prefix{u}{v}{(\binpar{D(x)}{P})}}}{D(x)} \nonumber
\end{eqnarray}

\begin{remark}
  Note that the lazier definition still does not deal with summation
  or mixed summation (i.e. sums over input and output). The reader is
  invited to construct definitions of replication that deal with these
  features. 

  Further, the definitions are parameterized in a name, $x$. Can you,
  gentle reader, make a definition that eliminates this parameter and
  guarantees no accidental interaction between the replication
  machinery and the process being replicated -- i.e. no accidental
  sharing of names used by the process to get its work done and the
  name(s) used by the replication to effect copying. This latter
  revision of the definition of replication is crucial to obtaining
  the expected identity $!!P \sim !P$.
\end{remark}

\begin{remark}\label{rem:paradoxical_combinator}
  The reader familiar with the lambda calculus will have noticed the
  similarity between $D$ and the paradoxical combinator.

  [Ed. note: the existence of this seems to suggest we have to be more
  restrictive on the set of processes and names we admit if we are to
  support no-cloning.]
\end{remark}

\subsubsection{Bisimulation}

The computational dynamics gives rise to another kind of equivalence,
the equivalence of computational behavior. As previously mentioned
this is typically captured \emph{via} some form of bisimulation.

% The notion we use in this paper is weak barbed bisimulation
% \cite{milner91polyadicpi}.

The notion we use in this paper is derived from weak barbed
bisimulation \cite{milner91polyadicpi}. 

\begin{definition}
An \emph{observation relation}, $\downarrow_{\mathcal N}$, over a set
of names, $\mathcal N$, is the smallest relation satisfying the rules
below.

\infrule[Out-barb]{y \in {\mathcal N}, \; x \nameeq y}
		  {\outputp{x}{v} \downarrow_{\mathcal N} x}
\infrule[Par-barb]{\mbox{$P\downarrow_{\mathcal N} x$ or $Q\downarrow_{\mathcal N} x$}}
		  {\binpar{P}{Q} \downarrow_{\mathcal N} x}

We write $P \Downarrow_{\mathcal N} x$ if there is $Q$ such that 
$P \wred Q$ and $Q \downarrow_{\mathcal N} x$.
\end{definition}

\begin{definition}
%\label{def.bbisim}
An  ${\mathcal N}$-\emph{barbed bisimulation} over a set of names, ${\mathcal N}$, is a symmetric binary relation 
${\mathcal S}_{\mathcal N}$ between agents such that $P\rel{S}_{\mathcal N}Q$ implies:
\begin{enumerate}
\item If $P \red P'$ then $Q \wred Q'$ and $P'\rel{S}_{\mathcal N} Q'$.
\item If $P\downarrow_{\mathcal N} x$, then $Q\Downarrow_{\mathcal N} x$.
\end{enumerate}
$P$ is ${\mathcal N}$-barbed bisimilar to $Q$, written
$P \wbbisim_{\mathcal N} Q$, if $P \rel{S}_{\mathcal N} Q$ for some ${\mathcal N}$-barbed bisimulation ${\mathcal S}_{\mathcal N}$.
\end{definition}

$\mathcal{R} \subseteq \pi \times \pi$

$P \mathcal{R} Q => \forall P'. P \red P' \Rightarrow \exists Q'. Q \red Q', P' \mathcal{R} Q'$

$P \vdash x \Rightarrow Q \vdash x$

\begin{mathpar}
  \inferrule*[lab=Out-barb]{x \nameeq y}{{y}!\langle{Q}\rangle \vdash x}
  \and
  \inferrule*[lab=Par-barb]{\mbox{$P\vdash x$ or $Q\vdash x$}}{\binpar{P}{Q} \vdash x}
\end{mathpar}

\subsubsection{Contexts}

One of the principle advantages of computational calculi like the
$\pi$-calculus is a well-defined notion of context,
contextual-equivalence and a correlation between
contextual-equivalence and notions of bisimulation. The notion of
context allows the decomposition of a process into (sub-)process and
its syntactic environment, its context. Thus, a context may be
thought of as a process with a ``hole'' (written $\Box$) in it. The
application of a context $M$ to a process $P$, written $M[P]$, is
tantamount to filling the hole in $M$ with $P$. In this paper we do
not need the full weight of this theory, but do make use of the notion
of context in the proof the main theorem. 

\begin{mathpar}
  \inferrule* [lab=summation] {} {{M_{M},M_{N}} \bc \Box \;|\; x.M_{A} \;|\; M_{M}+M_{N}}
  \and
  \inferrule* [lab=agent] {} {{M_{A}} \bc (\vec{x})M_{P} \;| \; \clift{P_0,\ldots,M_{P},\ldots,P_N}}
  \and \\
  \inferrule* [lab=process] {} {{M_{P}} \bc M_{N} \;| \;P|M_{P} }
\end{mathpar} 

\begin{mathpar}
  \inferrule* [lab=sychronization] {} {M_{N} \bc \Box \;|\; x?M_{F} \;|\; x!M_{C}}
  \and
  \inferrule* [lab=abstraction] {} {{M_{F}} \bc (x)M_{P} }
  \and
  \inferrule* [lab=concretion] {} {{M_{C}} \bc \langle M_{P} \rangle }
  \and \\
  \inferrule* [lab=process] {} {{M_{P}} \bc M_{N} \;| \;P|M_{P} }
\end{mathpar}

\begin{definition}[contextual application] Given a context $M$, and
  process $P$, we define the \emph{contextual application}, $M[P] :=
  M\{P/\Box\}$. That is, the contextual application of M to P is the
  substitution of $P$ for $\Box$ in $M$.
\end{definition}

$\meaningof{-} : L \to \mathcal{P}(\pi)$

\begin{mathpar}
  \inferrule* [lab=collection] {} {\meaningof{true} = \pi, \and \meaningof{~E} = \pi \setminus \meaningof{E}, \and \meaningof{E_{1} \& E_{2}} = \meaningof{E_{1}} \cap \meaningof{E_{2}}}
\end{mathpar}

\begin{mathpar}
  \inferrule* [lab=structure] {} {\meaningof{0} = \{ P \in \pi | P \equiv 0 \}, \and \\ \meaningof{E_1 | E_2} = \{ P \in \pi | P \equiv P_{1} | P_{2}, P_{1} \in \meaningof{E_{1}}, P_{2} \in \meaningof{E_2}\} }
\end{mathpar}

\begin{mathpar}
 \inferrule* [lab=behavior] {} {\meaningof{\langle a?b \rangle E} = \{ P \in \pi | P \equiv Q | u?(y)P', \\ \and \\\\ \and \\ \;\;\; u \in \meaningof{a}, \forall z.P'\{z/y\} \in \meaningof{E\{z/b\}}\}, \and \\ \meaningof{a!E} = \{ P \in \pi | P \equiv Q | x!\langle P' \rangle, x \in \meaningof{a} P' \in \meaningof{E}\} }
\end{mathpar}

\begin{mathpar}
 \inferrule* [lab=nominal] {} {\meaningof{\quotep{E}} = \{ \quotep{P} \in \quotep{\pi} | P \in \meaningof{E} \}, \and \meaningof{\quotep{P}} = \{ \quotep{Q} \in \quotep{\pi} | P \equiv Q \} \and \\ \meaningof{@\quotep{E}} = \{ P \in \pi | P \equiv @x, x \in \meaningof{E} \}}
\end{mathpar}

\begin{eqnarray*}
  \\
  \meaningof{-} : TS \to ST
\end{eqnarray*}

\begin{eqnarray*}
  \\
  L : TS \to ST
\end{eqnarray*}

\begin{eqnarray*}
  \\
  P \models E \iff P \in \meaningof{E}
\end{eqnarray*}

\begin{eqnarray*}
  P \approx_{L} Q \iff \forall E \in L. P \models E \iff Q \models E
\end{eqnarray*}

\begin{eqnarray*}
  P \approx_{K} Q
\end{eqnarray*}

\begin{eqnarray*}
  P \approx Q
\end{eqnarray*}

$\approx_{K} = \approx = \approx_{L}$

\subsubsection{Contextual duality}

Note that contexts extend the quotation operation to a family of
operations from processes to names. Given a context, $M$, we can
define a \emph{nominal context}, $\quotep{M}$ by $\quotep{M}[P] :=
\quotep{M[P]}$. To foreshadow what is to come we observe that these
operations enjoy a duality with processes very much like the duality
between vectors and maps from vectors to scalars.

Further, because the calculus is essentially higher-order, we have a
correspondence between contexts and processes. More specifically,
given a name $x$ and a context $M$ we can construct $M^{*}_{x}$ such
that 

\begin{mathpar}
  M^{*}_{x} | \lift{x}{P} \red M[P]
\end{mathpar}

namely,

\begin{mathpar}
  M^{*}_{x} := x?(u).M[\dropn{u}]
\end{mathpar}

The dependence of $M^{*}_{x}$ on a name makes it an abstraction, 

\begin{mathpar}
  M^{*} := (x)x?(u).M[\dropn{u}]
\end{mathpar}

\subsection{Additional notation}

It will sometimes be convenient to denote the process a name
quotes. We already have the notation $x = \quotep{P}$, but it will be
convenient to introduce an alternate notation, $\procn{x}$, when we
want to emphasize the connection to the use of the name. Note that, by
virtue of name equivalence, $\quotep{\procn{x}} \nameeq x$; so, the
notation is consistent with previous definitions.

Further, because names have structure it is possible to effect
substitutions on the basis of that structure. This means we need to
upgrade our notation for substitutions, which we accomplish by
adapting comprehension notation. Thus,

\begin{mathpar}
  P\{ y / x : x \in S \}
\end{mathpar}

is interpreted to mean the process derived from P by replacing (in a
capture-avoiding manner) each occurrence of $x$ in $S$ by $y$. For example,

\begin{mathpar}
  P\{ \quotep{\procn{x}|\procn{x}} / x : x \in \freenames{P} \}
\end{mathpar}

will replace each (occurrence) of a free name $x$ in $P$ by
$\quotep{\procn{x}|\procn{x}}$.

Also, we will avail ourselves of the notation $x^{L}$ and $x^{R}$ to
denote injections of a name into disjoint copies of the name
space. There are numerous ways to accomplish this. One example can be
found in \cite{MeredithR05}. This notation overloads to vectors of
names: $\vec{x}^{\pi} := (x_{i}^{\pi} \; : \; 0 \leq i < |\vec{x}| )$ where $\pi \in \{L,R\}$.

We also use $P^{\Box} := P|\Box$.

In \cite{MeredithR05} an interpretation of the new operator is
given. It turns out that there are several possible interpretations
all enjoying the requisite algebraic properties of the operator (see
\cite{milner91polyadicpi}). We will therefore make liberal use of
$(\nu\; \vec{x})P$.

% subsection the_syntax_and_semantics_of_the_notation_system (end)   

\input{qm2pi.qmops} 

\input{qm2pi.sterngerlach} 

\input{qm2pi.metric} 

% section concurrent_process_calculi (end)

%\input{qm2pi.proofsketch}

% section proof sketch (end)

%\input{qm2pi.slviaknots} 

% section spatial logic via knots (end)

\input{qm2pi.conclusion}

% section conclusion (end)

%\input{qm2pi.dtcodes} 

% section wiring algorithm (end)

\input{qm2pi.ack} 

% section acknowledgments (end)

\newpage


\bibliographystyle{plain}   
\bibliography{../../biblios/main.bib}

\input{qm2pi.rhodetails}

\end{document}

 

% section acknowledgments (end)

\newpage


\bibliographystyle{plain}   
\bibliography{../../biblios/main.bib}

\documentclass[12pt]{llncs}
%\documentclass{jktr}

\usepackage[pdftex]{hyperref}                   
\usepackage {listings}
\usepackage {mathpartir}
\usepackage{bcprules}
%\usepackage{listings}
                       
\usepackage{graphicx} 
%\usepackage[margins=2.5cm,nohead,nofoot]{geometry}
%\usepackage{geometry}
\usepackage{amsfonts}
\usepackage{amstext}
\usepackage{latexsym}
\usepackage{amssymb}
\usepackage{color}


%\include{myPreamble}
\include{qm2pi.local} 

%\ifpdf
%\usepackage[pdftex]{graphicx}
%\else
%\usepackage{graphicx}
%\fi

 % \ifpdf
%  \usepackage{pdfsync}
%  \if


%\title{Brief Article}
%\author{David F. Snyder}
%\author{L.G. Meredith}

%\address{Dept. of Math., Texas State University--San Marcos, San Marcos, TX 78666}
       
\pagestyle{empty}


\begin{document}

\lstset{language=[Objective]Caml,frame=shadowbox}

\input{qm2pi.front}

% section front matter (end)

\input{qm2pi.intro} 
 
% section introduction (end)

% \input{qm2pi.knotations} 

% section notation (end)

\input{qm2pi.process.calculi} 

% section concurrent_process_calculi_and_spatial_logics_ (end)
    
%\input{qm2pi.knots2pi} 

%\input{qm2pi.trefoil} 

%\input{qm2pi.mainthm} 

% subsection basic_interpretation (end)

%\input{qm2pi.rho.presentation} 
\subsection{The syntax and semantics of the notation system}\label{sub:the_syntax_and_semantics_of_the_notation_system} % (fold)

We now summarize a technical presentation of the calculus that
embodies our theory of dynamics. The typical presentation of such a
calculus follows the style of giving generators and relations on
them. The grammar, below, describing term constructors, freely
generates the set of processes, $\Proc$. This set is then quotiented
by a relation known as structural congruence and it is over this set
that the notion of dynamics is expressed. This presentation is
essentially that of \cite{MeredithR05} with the addition of
polyadicity and summation. For readability we have relegated some of
the technical subtleties to an appendix.

\subsubsection{Process grammar}\label{subsub:process_grammar}

\begin{mathpar}
  \inferrule* [lab=synchronization] {} {{M} \bc \pzero \;|\; x?F \;|\; x!C }
  \and
  \inferrule* [lab=abstraction] {} {{F} \bc (x)P}
  \and
  \inferrule* [lab=concretion] {} {{C} \bc \langle Q \rangle}
  \and
  \inferrule* [lab=process] {} {{P,Q} \bc M \;| \;P|Q \;|\; @{x}}
  \and
  \inferrule* [lab=name] {} {{x} \bc \quotep{P}}
\end{mathpar} 

Note that $\vec{x}$ (resp. $\vec{P}$) denotes a vector of names
(resp. processes) of length $|\vec{x}|$ (resp. $|\vec{P}|$). We adopt
the following useful abbreviations.

\begin{mathpar}
   x?(\vec{y}).P := x.(\vec{y})P \and  x\clift{\vec{P}} := x.\clift{\vec{P}}
   \and x!(y) := \lift{x}{\dropn{y}}
   \and \Pi_{i=0}^{n-1}P_i := P_0 | \ldots | P_{n-1}
\end{mathpar}

\subsubsection{Structural congruence}

\paragraph{Free and bound names and alpha-equivalence.} At the
core of structural equivalence is alpha-equivalence which identifies
process that are the same up to a change of variable. Formally, we
recognize the distinction between free and bound names. The free names
of a process, $\freenames{P}$, may be calculated recursively as
follows:

\begin{mathpar}
\freenames{\pzero} := \emptyset
  \and \\
  \freenames{x?(y).P} := \{ x \} \cup (\freenames{P} \setminus \{ y \})
  \and 
  \freenames{x!\langle P \rangle} := \{ x \} \cup \{ P \} 
  \and \\
  \freenames{P|Q} := \freenames{P} \cup \freenames{Q}
  \and \\
  \freenames{@{x}} := \{ x \}
\end{mathpar}

$\pi$
$\quotep{\pi}$

$\freenames{-} : \pi \to \mathcal{P}(\quotep{\pi})$

\begin{eqnarray*}
  \freenames{\pzero} & := & \emptyset \\
  \freenames{x?(y).P} & := & \{ x \} \cup (\freenames{P} \setminus \{ y \}) \\
  \freenames{x!\langle P \rangle} & := & \{ x \} \cup \{ P \} \\
  \freenames{P|Q} & := & \freenames{P} \cup \freenames{Q} \\
  \freenames{\dropn{x}} & := & \{ x \}
\end{eqnarray*}

The bound names of a process, $\boundnames{P}$, are those names occurring in $P$
that are not free. For example, in $x?(y).0$, the name $x$ is free, while $y$ is bound.

\begin{mathpar}
  \inferrule* [lab=monoidal-laws] {} { P|Q \equiv Q|P \and P|0 \equiv P \and P|(Q|R) \equiv (P|Q)|R }
\end{mathpar}

\begin{mathpar}
  \inferrule* [lab=alpha-equivalence] {} { (x)P \equiv (y)P\{y/x\} \and y \not\in \freenames{P} }
\end{mathpar}

\begin{definition}
Then two processes, $P,Q$, are alpha-equivalent if $P = Q\{\vec{y}/\vec{x}\}$ for
some $\vec{x} \in \boundnames{Q},\vec{y} \in \boundnames{P}$, where $Q\{\vec{y}/\vec{x}\}$
denotes the capture-avoiding substitution of $\vec{y}$ for $\vec{x}$ in $Q$.
\end{definition}

\begin{definition}
  The {\em structural congruence} \cite{SangiorgiWalker} , $\equiv$,
  between processes is the least congruence containing
  alpha-equivalence, satisfying the abelian monoid laws
  (associativity, commutativity and $\pzero$ as identity) for parallel
  composition $|$ and for summation $+$.
\end{definition}

\subsection{Name equivalence}

We take name equivalence, written $\nameeq$, to be the smallest
equivalence relation generated by the following rules.

\begin{mathpar}
\inferrule*[lab=Quote-drop]
{ }
{ \quotep{@{x}} \nameeq x }

\inferrule*[lab=Struct-equiv]
{ P \scong Q }
{ \quotep{P} \nameeq \quotep{Q} }
\end{mathpar}

The astute reader will have noticed that the mutual recursion of names
and processes imposes a mutual recursion on alpha-equivalence and
structural equivalence via name-equivalence. Fortunately, all of this
works out pleasantly and we may calculate in the natural way, free of
concern. The reader interested in the details is referred to the
appendix \ref{appendix:rho_details}.

\subsection{Substitution}

We use $\Proc$ for the set of processes, $\QProc$ for the set of
names, and $\id{\{}\vec{y} / \vec{x} \id{\}}$ to denote partial maps,
$s : \QProc \rightarrow \QProc$. A map, $s$ lifts, uniquely, to a map
on process terms, $\widehat{s} : \Proc \rightarrow \Proc$ by the
following equations.

\begin{mathpar}
  (0) \psubstp{Q}{P} := 0 \\
  (R \juxtap S) \psubstp{Q}{P}
  :=    
  (R)\psubstp{Q}{P} \juxtap (S) \psubstp{Q}{P} \\
  (x?(y).R) \psubstp{Q}{P}    
  :=    
  (x)\substp{Q}{P} (z)\concat( (R \psubstn{z}{y}) \psubstp{Q}{P} ) \\
  (\lift{x}{R}) \psubstp{Q}{P}  
  :=
  \lift{(x)\substp{Q}{P}}{ R \psubstp{Q}{P} } \\
%   (\dropn{x})  \psubstp{Q}{P}       
%   := 
%   \left\{ 
%     \begin{array}{ccc} 
%       \dropn{\quotep{Q}} & & x \nameeq \quotep{P} \\
%       \dropn{x} & & otherwise \\
%     \end{array}
%   \right. 
  (\dropn{x})  \psubstp{Q}{P}       
  := 
  \left\{ 
    \begin{array}{ccc} 
      Q & & x \nameeq \quotep{P} \\
      \dropn{x} & & otherwise \\
    \end{array}
  \right.
\end{mathpar}
 

where

\begin{eqnarray}
  (x)\id{\{} \lpquote Q \rpquote / \lpquote P \rpquote \id{\}}            = 
  \left\{ 
    \begin{array}{ccc}
      \lpquote Q \rpquote & & x \nameeq \lpquote P \rpquote \\
      x & & otherwise \\
    \end{array}
  \right. \nonumber
\end{eqnarray}

and $z$ is chosen distinct from $\quotep{P}$, $\quotep{Q}$, the free
names in $Q$, and all the names in $R$. Our $\alpha$-equivalence will
be built in the standard way from this substitution.

\begin{remark}\label{rem:no_self_referential_names}
  One consequence of these definitions is that $\forall P. \quotep{P}
  \not\in \freenames{P}$.
\end{remark}

\subsection{ Dynamic quote: an example }

Anticipating something of what's to come, consider applying the
substitution, $\widehat{\id{\{}u / z \id{\}}}$, to the following pair
of processes, $\lift{w}{y!(z)}$ and $w[ \lpquote y!(z) \rpquote ]$.

\begin{eqnarray}
	\lift{w}{y!(z)}\widehat{\id{\{}u / z \id{\}}}
		& = &
		\lift{w}{y!(u)} \nonumber\\
	w[ \lpquote y!(z) \rpquote ] \widehat{ \id{\{}u / z \id{\}} }
		& = &
		w[ \lpquote y!(z) \rpquote ] \nonumber
\end{eqnarray}

Because the body of the process between quotes is impervious to
substitution, we get radically different answers. In fact, by
examining the first process in an input context,
e.g. $x?(z).\lift{w}{y!(z)}$, we see that the process under the lift
operator may be shaped by prefixed inputs binding a name inside it. In
this sense, the lift operator will be seen as a way to dynamically
construct processes before reifying them as names.

Finally equipped with these standard features we can present the
dynamics of the calculus.

\subsubsection{Operational semantics} 

Finally, we introduce the computational dynamics. What marks these
algebras as distinct from other more traditionally studied algebraic
structures, e.g. vector spaces or polynomial rings, is the manner in
which dynamics is captured. In traditional structures, dynamics is typically
expressed through morphisms between such structures, as in linear maps
between vector spaces or morphisms between rings. In algebras
associated with the semantics of computation, the dynamics is
expressed as part of the algebraic structure itself, through a
reduction reduction relation typically denoted by $\red$. Below, we
give a recursive presentation of this relation for the calculus used
in the encoding.

$\red \subseteq \pi \times \pi$
$\red : \pi \to \mathcal{P}(\pi)$

\begin{mathpar}
  \inferrule* [lab=Comm] { \textsf{match}( x_{src}, x_{trgt} ) } { x_{trgt}?(y)P \; | \; x_{src}!\langle {Q} \rangle \red P\{\quotep{Q}/y}\} }
  \and \\
  \inferrule* [lab=Par] {{P} \red {P}'} {{{P} | {Q}} \red {{P}' | {Q}}}
  \and
  \inferrule* [lab=Equiv]{{{P} \scong {P}'} \andalso {{P}' \red {Q}'} \andalso {{Q}' \scong {Q}}}{{P} \red {Q}}
\end{mathpar}

\begin{eqnarray*}
  match_{\equiv} (\quotep{P},\quotep{Q}) & := & P \equiv Q \\
  match_{\dagger}(\quotep{P},\quotep{Q}) & := & \forall R. P|Q \red^{*} R => R \red^{*} 0 \\
  match_{K}(\quotep{P},\quotep{Q}) & := & K \mbox{ for some context } K
\end{eqnarray*}

$u?(x)P | u!\langle Q \rangle \red P\{\quotep{Q}/x\}$

%We write $\wred$ for $\red^*$, and $P\red$ if $\exists Q $ such that $ P \red Q$.
We write $P\red$ if $\exists Q $ such that $ P \red Q$ and $P\not\red$, otherwise.

\section{Replication}

As mentioned before, it is known that replication (and hence
recursion) can be implemented in a higher-order process algebra
\cite{SangiorgiWalker}. As our first example of calculation with the
machinery thus far presented we give the construction explicitly in
the {\rhoc}.

\begin{eqnarray}
	D_{x} & := & \prefix{x}{y}{(\binpar{\outputp{x}{y}}{@{y}})} \nonumber\\
	\bangp_{x}{P} & := & \binpar{{x}!\langle{\binpar{D_{x}}{P}}\rangle}{D_{x}} \nonumber
\end{eqnarray}

\begin{eqnarray}
	\bangp_{x}{P} & & \nonumber\\
	=
	& {x}!\langle{(\prefix{x}{y}{(\outputp{x}{y} | @{y})) | P}}\rangle 
	      | \prefix{x}{y}{(\outputp{x}{y} | @{y})} & \nonumber\\
	\red
	& (\outputp{x}{y} | @{y})\substn{\quotep{(\prefix{x}{y}{(@{y} | \outputp{x}{y})) | P}}}{y} & \nonumber\\
	=
	& \outputp{x}{\quotep{(\prefix{x}{y}{(\outputp{x}{y} | @{y})) | P}}}
	  | {(\prefix{x}{y}{(\outputp{x}{y} | @{y})) | P}} & \nonumber\\
	\red
	& \ldots & \nonumber\\
	\red^*
	& P | P | \ldots & \nonumber
\end{eqnarray}

Of course, this encoding, as an implementation, runs away, unfolding
$\bangp{P}$ eagerly. A lazier and more implementable replication
operator, restricted to input-guarded processes, may be obtained as follows.

\begin{eqnarray}
\bangp{\prefix{u}{v}{P}} 
	:= 
	\binpar{\lift{x}{\prefix{u}{v}{(\binpar{D(x)}{P})}}}{D(x)} \nonumber
\end{eqnarray}

\begin{remark}
  Note that the lazier definition still does not deal with summation
  or mixed summation (i.e. sums over input and output). The reader is
  invited to construct definitions of replication that deal with these
  features. 

  Further, the definitions are parameterized in a name, $x$. Can you,
  gentle reader, make a definition that eliminates this parameter and
  guarantees no accidental interaction between the replication
  machinery and the process being replicated -- i.e. no accidental
  sharing of names used by the process to get its work done and the
  name(s) used by the replication to effect copying. This latter
  revision of the definition of replication is crucial to obtaining
  the expected identity $!!P \sim !P$.
\end{remark}

\begin{remark}\label{rem:paradoxical_combinator}
  The reader familiar with the lambda calculus will have noticed the
  similarity between $D$ and the paradoxical combinator.

  [Ed. note: the existence of this seems to suggest we have to be more
  restrictive on the set of processes and names we admit if we are to
  support no-cloning.]
\end{remark}

\subsubsection{Bisimulation}

The computational dynamics gives rise to another kind of equivalence,
the equivalence of computational behavior. As previously mentioned
this is typically captured \emph{via} some form of bisimulation.

% The notion we use in this paper is weak barbed bisimulation
% \cite{milner91polyadicpi}.

The notion we use in this paper is derived from weak barbed
bisimulation \cite{milner91polyadicpi}. 

\begin{definition}
An \emph{observation relation}, $\downarrow_{\mathcal N}$, over a set
of names, $\mathcal N$, is the smallest relation satisfying the rules
below.

\infrule[Out-barb]{y \in {\mathcal N}, \; x \nameeq y}
		  {\outputp{x}{v} \downarrow_{\mathcal N} x}
\infrule[Par-barb]{\mbox{$P\downarrow_{\mathcal N} x$ or $Q\downarrow_{\mathcal N} x$}}
		  {\binpar{P}{Q} \downarrow_{\mathcal N} x}

We write $P \Downarrow_{\mathcal N} x$ if there is $Q$ such that 
$P \wred Q$ and $Q \downarrow_{\mathcal N} x$.
\end{definition}

\begin{definition}
%\label{def.bbisim}
An  ${\mathcal N}$-\emph{barbed bisimulation} over a set of names, ${\mathcal N}$, is a symmetric binary relation 
${\mathcal S}_{\mathcal N}$ between agents such that $P\rel{S}_{\mathcal N}Q$ implies:
\begin{enumerate}
\item If $P \red P'$ then $Q \wred Q'$ and $P'\rel{S}_{\mathcal N} Q'$.
\item If $P\downarrow_{\mathcal N} x$, then $Q\Downarrow_{\mathcal N} x$.
\end{enumerate}
$P$ is ${\mathcal N}$-barbed bisimilar to $Q$, written
$P \wbbisim_{\mathcal N} Q$, if $P \rel{S}_{\mathcal N} Q$ for some ${\mathcal N}$-barbed bisimulation ${\mathcal S}_{\mathcal N}$.
\end{definition}

$\mathcal{R} \subseteq \pi \times \pi$

$P \mathcal{R} Q => \forall P'. P \red P' \Rightarrow \exists Q'. Q \red Q', P' \mathcal{R} Q'$

$P \vdash x \Rightarrow Q \vdash x$

\begin{mathpar}
  \inferrule*[lab=Out-barb]{x \nameeq y}{{y}!\langle{Q}\rangle \vdash x}
  \and
  \inferrule*[lab=Par-barb]{\mbox{$P\vdash x$ or $Q\vdash x$}}{\binpar{P}{Q} \vdash x}
\end{mathpar}

\subsubsection{Contexts}

One of the principle advantages of computational calculi like the
$\pi$-calculus is a well-defined notion of context,
contextual-equivalence and a correlation between
contextual-equivalence and notions of bisimulation. The notion of
context allows the decomposition of a process into (sub-)process and
its syntactic environment, its context. Thus, a context may be
thought of as a process with a ``hole'' (written $\Box$) in it. The
application of a context $M$ to a process $P$, written $M[P]$, is
tantamount to filling the hole in $M$ with $P$. In this paper we do
not need the full weight of this theory, but do make use of the notion
of context in the proof the main theorem. 

\begin{mathpar}
  \inferrule* [lab=summation] {} {{M_{M},M_{N}} \bc \Box \;|\; x.M_{A} \;|\; M_{M}+M_{N}}
  \and
  \inferrule* [lab=agent] {} {{M_{A}} \bc (\vec{x})M_{P} \;| \; \clift{P_0,\ldots,M_{P},\ldots,P_N}}
  \and \\
  \inferrule* [lab=process] {} {{M_{P}} \bc M_{N} \;| \;P|M_{P} }
\end{mathpar} 

\begin{mathpar}
  \inferrule* [lab=sychronization] {} {M_{N} \bc \Box \;|\; x?M_{F} \;|\; x!M_{C}}
  \and
  \inferrule* [lab=abstraction] {} {{M_{F}} \bc (x)M_{P} }
  \and
  \inferrule* [lab=concretion] {} {{M_{C}} \bc \langle M_{P} \rangle }
  \and \\
  \inferrule* [lab=process] {} {{M_{P}} \bc M_{N} \;| \;P|M_{P} }
\end{mathpar}

\begin{definition}[contextual application] Given a context $M$, and
  process $P$, we define the \emph{contextual application}, $M[P] :=
  M\{P/\Box\}$. That is, the contextual application of M to P is the
  substitution of $P$ for $\Box$ in $M$.
\end{definition}

$\meaningof{-} : L \to \mathcal{P}(\pi)$

\begin{mathpar}
  \inferrule* [lab=collection] {} {\meaningof{true} = \pi, \and \meaningof{~E} = \pi \setminus \meaningof{E}, \and \meaningof{E_{1} \& E_{2}} = \meaningof{E_{1}} \cap \meaningof{E_{2}}}
\end{mathpar}

\begin{mathpar}
  \inferrule* [lab=structure] {} {\meaningof{0} = \{ P \in \pi | P \equiv 0 \}, \and \\ \meaningof{E_1 | E_2} = \{ P \in \pi | P \equiv P_{1} | P_{2}, P_{1} \in \meaningof{E_{1}}, P_{2} \in \meaningof{E_2}\} }
\end{mathpar}

\begin{mathpar}
 \inferrule* [lab=behavior] {} {\meaningof{\langle a?b \rangle E} = \{ P \in \pi | P \equiv Q | u?(y)P', \\ \and \\\\ \and \\ \;\;\; u \in \meaningof{a}, \forall z.P'\{z/y\} \in \meaningof{E\{z/b\}}\}, \and \\ \meaningof{a!E} = \{ P \in \pi | P \equiv Q | x!\langle P' \rangle, x \in \meaningof{a} P' \in \meaningof{E}\} }
\end{mathpar}

\begin{mathpar}
 \inferrule* [lab=nominal] {} {\meaningof{\quotep{E}} = \{ \quotep{P} \in \quotep{\pi} | P \in \meaningof{E} \}, \and \meaningof{\quotep{P}} = \{ \quotep{Q} \in \quotep{\pi} | P \equiv Q \} \and \\ \meaningof{@\quotep{E}} = \{ P \in \pi | P \equiv @x, x \in \meaningof{E} \}}
\end{mathpar}

\begin{eqnarray*}
  \\
  \meaningof{-} : TS \to ST
\end{eqnarray*}

\begin{eqnarray*}
  \\
  L : TS \to ST
\end{eqnarray*}

\begin{eqnarray*}
  \\
  P \models E \iff P \in \meaningof{E}
\end{eqnarray*}

\begin{eqnarray*}
  P \approx_{L} Q \iff \forall E \in L. P \models E \iff Q \models E
\end{eqnarray*}

\begin{eqnarray*}
  P \approx_{K} Q
\end{eqnarray*}

\begin{eqnarray*}
  P \approx Q
\end{eqnarray*}

$\approx_{K} = \approx = \approx_{L}$

\subsubsection{Contextual duality}

Note that contexts extend the quotation operation to a family of
operations from processes to names. Given a context, $M$, we can
define a \emph{nominal context}, $\quotep{M}$ by $\quotep{M}[P] :=
\quotep{M[P]}$. To foreshadow what is to come we observe that these
operations enjoy a duality with processes very much like the duality
between vectors and maps from vectors to scalars.

Further, because the calculus is essentially higher-order, we have a
correspondence between contexts and processes. More specifically,
given a name $x$ and a context $M$ we can construct $M^{*}_{x}$ such
that 

\begin{mathpar}
  M^{*}_{x} | \lift{x}{P} \red M[P]
\end{mathpar}

namely,

\begin{mathpar}
  M^{*}_{x} := x?(u).M[\dropn{u}]
\end{mathpar}

The dependence of $M^{*}_{x}$ on a name makes it an abstraction, 

\begin{mathpar}
  M^{*} := (x)x?(u).M[\dropn{u}]
\end{mathpar}

\subsection{Additional notation}

It will sometimes be convenient to denote the process a name
quotes. We already have the notation $x = \quotep{P}$, but it will be
convenient to introduce an alternate notation, $\procn{x}$, when we
want to emphasize the connection to the use of the name. Note that, by
virtue of name equivalence, $\quotep{\procn{x}} \nameeq x$; so, the
notation is consistent with previous definitions.

Further, because names have structure it is possible to effect
substitutions on the basis of that structure. This means we need to
upgrade our notation for substitutions, which we accomplish by
adapting comprehension notation. Thus,

\begin{mathpar}
  P\{ y / x : x \in S \}
\end{mathpar}

is interpreted to mean the process derived from P by replacing (in a
capture-avoiding manner) each occurrence of $x$ in $S$ by $y$. For example,

\begin{mathpar}
  P\{ \quotep{\procn{x}|\procn{x}} / x : x \in \freenames{P} \}
\end{mathpar}

will replace each (occurrence) of a free name $x$ in $P$ by
$\quotep{\procn{x}|\procn{x}}$.

Also, we will avail ourselves of the notation $x^{L}$ and $x^{R}$ to
denote injections of a name into disjoint copies of the name
space. There are numerous ways to accomplish this. One example can be
found in \cite{MeredithR05}. This notation overloads to vectors of
names: $\vec{x}^{\pi} := (x_{i}^{\pi} \; : \; 0 \leq i < |\vec{x}| )$ where $\pi \in \{L,R\}$.

We also use $P^{\Box} := P|\Box$.

In \cite{MeredithR05} an interpretation of the new operator is
given. It turns out that there are several possible interpretations
all enjoying the requisite algebraic properties of the operator (see
\cite{milner91polyadicpi}). We will therefore make liberal use of
$(\nu\; \vec{x})P$.

% subsection the_syntax_and_semantics_of_the_notation_system (end)   

\input{qm2pi.qmops} 

\input{qm2pi.sterngerlach} 

\input{qm2pi.metric} 

% section concurrent_process_calculi (end)

%\input{qm2pi.proofsketch}

% section proof sketch (end)

%\input{qm2pi.slviaknots} 

% section spatial logic via knots (end)

\input{qm2pi.conclusion}

% section conclusion (end)

%\input{qm2pi.dtcodes} 

% section wiring algorithm (end)

\input{qm2pi.ack} 

% section acknowledgments (end)

\newpage


\bibliographystyle{plain}   
\bibliography{../../biblios/main.bib}

\input{qm2pi.rhodetails}

\end{document}



\end{document}

 

% section wiring algorithm (end)

\documentclass[12pt]{llncs}
%\documentclass{jktr}

\usepackage[pdftex]{hyperref}                   
\usepackage {listings}
\usepackage {mathpartir}
\usepackage{bcprules}
%\usepackage{listings}
                       
\usepackage{graphicx} 
%\usepackage[margins=2.5cm,nohead,nofoot]{geometry}
%\usepackage{geometry}
\usepackage{amsfonts}
\usepackage{amstext}
\usepackage{latexsym}
\usepackage{amssymb}
\usepackage{color}


%\include{myPreamble}
\documentclass[12pt]{llncs}
%\documentclass{jktr}

\usepackage[pdftex]{hyperref}                   
\usepackage {listings}
\usepackage {mathpartir}
\usepackage{bcprules}
%\usepackage{listings}
                       
\usepackage{graphicx} 
%\usepackage[margins=2.5cm,nohead,nofoot]{geometry}
%\usepackage{geometry}
\usepackage{amsfonts}
\usepackage{amstext}
\usepackage{latexsym}
\usepackage{amssymb}
\usepackage{color}


%\include{myPreamble}
\include{qm2pi.local} 

%\ifpdf
%\usepackage[pdftex]{graphicx}
%\else
%\usepackage{graphicx}
%\fi

 % \ifpdf
%  \usepackage{pdfsync}
%  \if


%\title{Brief Article}
%\author{David F. Snyder}
%\author{L.G. Meredith}

%\address{Dept. of Math., Texas State University--San Marcos, San Marcos, TX 78666}
       
\pagestyle{empty}


\begin{document}

\lstset{language=[Objective]Caml,frame=shadowbox}

\input{qm2pi.front}

% section front matter (end)

\input{qm2pi.intro} 
 
% section introduction (end)

% \input{qm2pi.knotations} 

% section notation (end)

\input{qm2pi.process.calculi} 

% section concurrent_process_calculi_and_spatial_logics_ (end)
    
%\input{qm2pi.knots2pi} 

%\input{qm2pi.trefoil} 

%\input{qm2pi.mainthm} 

% subsection basic_interpretation (end)

%\input{qm2pi.rho.presentation} 
\subsection{The syntax and semantics of the notation system}\label{sub:the_syntax_and_semantics_of_the_notation_system} % (fold)

We now summarize a technical presentation of the calculus that
embodies our theory of dynamics. The typical presentation of such a
calculus follows the style of giving generators and relations on
them. The grammar, below, describing term constructors, freely
generates the set of processes, $\Proc$. This set is then quotiented
by a relation known as structural congruence and it is over this set
that the notion of dynamics is expressed. This presentation is
essentially that of \cite{MeredithR05} with the addition of
polyadicity and summation. For readability we have relegated some of
the technical subtleties to an appendix.

\subsubsection{Process grammar}\label{subsub:process_grammar}

\begin{mathpar}
  \inferrule* [lab=synchronization] {} {{M} \bc \pzero \;|\; x?F \;|\; x!C }
  \and
  \inferrule* [lab=abstraction] {} {{F} \bc (x)P}
  \and
  \inferrule* [lab=concretion] {} {{C} \bc \langle Q \rangle}
  \and
  \inferrule* [lab=process] {} {{P,Q} \bc M \;| \;P|Q \;|\; @{x}}
  \and
  \inferrule* [lab=name] {} {{x} \bc \quotep{P}}
\end{mathpar} 

Note that $\vec{x}$ (resp. $\vec{P}$) denotes a vector of names
(resp. processes) of length $|\vec{x}|$ (resp. $|\vec{P}|$). We adopt
the following useful abbreviations.

\begin{mathpar}
   x?(\vec{y}).P := x.(\vec{y})P \and  x\clift{\vec{P}} := x.\clift{\vec{P}}
   \and x!(y) := \lift{x}{\dropn{y}}
   \and \Pi_{i=0}^{n-1}P_i := P_0 | \ldots | P_{n-1}
\end{mathpar}

\subsubsection{Structural congruence}

\paragraph{Free and bound names and alpha-equivalence.} At the
core of structural equivalence is alpha-equivalence which identifies
process that are the same up to a change of variable. Formally, we
recognize the distinction between free and bound names. The free names
of a process, $\freenames{P}$, may be calculated recursively as
follows:

\begin{mathpar}
\freenames{\pzero} := \emptyset
  \and \\
  \freenames{x?(y).P} := \{ x \} \cup (\freenames{P} \setminus \{ y \})
  \and 
  \freenames{x!\langle P \rangle} := \{ x \} \cup \{ P \} 
  \and \\
  \freenames{P|Q} := \freenames{P} \cup \freenames{Q}
  \and \\
  \freenames{@{x}} := \{ x \}
\end{mathpar}

$\pi$
$\quotep{\pi}$

$\freenames{-} : \pi \to \mathcal{P}(\quotep{\pi})$

\begin{eqnarray*}
  \freenames{\pzero} & := & \emptyset \\
  \freenames{x?(y).P} & := & \{ x \} \cup (\freenames{P} \setminus \{ y \}) \\
  \freenames{x!\langle P \rangle} & := & \{ x \} \cup \{ P \} \\
  \freenames{P|Q} & := & \freenames{P} \cup \freenames{Q} \\
  \freenames{\dropn{x}} & := & \{ x \}
\end{eqnarray*}

The bound names of a process, $\boundnames{P}$, are those names occurring in $P$
that are not free. For example, in $x?(y).0$, the name $x$ is free, while $y$ is bound.

\begin{mathpar}
  \inferrule* [lab=monoidal-laws] {} { P|Q \equiv Q|P \and P|0 \equiv P \and P|(Q|R) \equiv (P|Q)|R }
\end{mathpar}

\begin{mathpar}
  \inferrule* [lab=alpha-equivalence] {} { (x)P \equiv (y)P\{y/x\} \and y \not\in \freenames{P} }
\end{mathpar}

\begin{definition}
Then two processes, $P,Q$, are alpha-equivalent if $P = Q\{\vec{y}/\vec{x}\}$ for
some $\vec{x} \in \boundnames{Q},\vec{y} \in \boundnames{P}$, where $Q\{\vec{y}/\vec{x}\}$
denotes the capture-avoiding substitution of $\vec{y}$ for $\vec{x}$ in $Q$.
\end{definition}

\begin{definition}
  The {\em structural congruence} \cite{SangiorgiWalker} , $\equiv$,
  between processes is the least congruence containing
  alpha-equivalence, satisfying the abelian monoid laws
  (associativity, commutativity and $\pzero$ as identity) for parallel
  composition $|$ and for summation $+$.
\end{definition}

\subsection{Name equivalence}

We take name equivalence, written $\nameeq$, to be the smallest
equivalence relation generated by the following rules.

\begin{mathpar}
\inferrule*[lab=Quote-drop]
{ }
{ \quotep{@{x}} \nameeq x }

\inferrule*[lab=Struct-equiv]
{ P \scong Q }
{ \quotep{P} \nameeq \quotep{Q} }
\end{mathpar}

The astute reader will have noticed that the mutual recursion of names
and processes imposes a mutual recursion on alpha-equivalence and
structural equivalence via name-equivalence. Fortunately, all of this
works out pleasantly and we may calculate in the natural way, free of
concern. The reader interested in the details is referred to the
appendix \ref{appendix:rho_details}.

\subsection{Substitution}

We use $\Proc$ for the set of processes, $\QProc$ for the set of
names, and $\id{\{}\vec{y} / \vec{x} \id{\}}$ to denote partial maps,
$s : \QProc \rightarrow \QProc$. A map, $s$ lifts, uniquely, to a map
on process terms, $\widehat{s} : \Proc \rightarrow \Proc$ by the
following equations.

\begin{mathpar}
  (0) \psubstp{Q}{P} := 0 \\
  (R \juxtap S) \psubstp{Q}{P}
  :=    
  (R)\psubstp{Q}{P} \juxtap (S) \psubstp{Q}{P} \\
  (x?(y).R) \psubstp{Q}{P}    
  :=    
  (x)\substp{Q}{P} (z)\concat( (R \psubstn{z}{y}) \psubstp{Q}{P} ) \\
  (\lift{x}{R}) \psubstp{Q}{P}  
  :=
  \lift{(x)\substp{Q}{P}}{ R \psubstp{Q}{P} } \\
%   (\dropn{x})  \psubstp{Q}{P}       
%   := 
%   \left\{ 
%     \begin{array}{ccc} 
%       \dropn{\quotep{Q}} & & x \nameeq \quotep{P} \\
%       \dropn{x} & & otherwise \\
%     \end{array}
%   \right. 
  (\dropn{x})  \psubstp{Q}{P}       
  := 
  \left\{ 
    \begin{array}{ccc} 
      Q & & x \nameeq \quotep{P} \\
      \dropn{x} & & otherwise \\
    \end{array}
  \right.
\end{mathpar}
 

where

\begin{eqnarray}
  (x)\id{\{} \lpquote Q \rpquote / \lpquote P \rpquote \id{\}}            = 
  \left\{ 
    \begin{array}{ccc}
      \lpquote Q \rpquote & & x \nameeq \lpquote P \rpquote \\
      x & & otherwise \\
    \end{array}
  \right. \nonumber
\end{eqnarray}

and $z$ is chosen distinct from $\quotep{P}$, $\quotep{Q}$, the free
names in $Q$, and all the names in $R$. Our $\alpha$-equivalence will
be built in the standard way from this substitution.

\begin{remark}\label{rem:no_self_referential_names}
  One consequence of these definitions is that $\forall P. \quotep{P}
  \not\in \freenames{P}$.
\end{remark}

\subsection{ Dynamic quote: an example }

Anticipating something of what's to come, consider applying the
substitution, $\widehat{\id{\{}u / z \id{\}}}$, to the following pair
of processes, $\lift{w}{y!(z)}$ and $w[ \lpquote y!(z) \rpquote ]$.

\begin{eqnarray}
	\lift{w}{y!(z)}\widehat{\id{\{}u / z \id{\}}}
		& = &
		\lift{w}{y!(u)} \nonumber\\
	w[ \lpquote y!(z) \rpquote ] \widehat{ \id{\{}u / z \id{\}} }
		& = &
		w[ \lpquote y!(z) \rpquote ] \nonumber
\end{eqnarray}

Because the body of the process between quotes is impervious to
substitution, we get radically different answers. In fact, by
examining the first process in an input context,
e.g. $x?(z).\lift{w}{y!(z)}$, we see that the process under the lift
operator may be shaped by prefixed inputs binding a name inside it. In
this sense, the lift operator will be seen as a way to dynamically
construct processes before reifying them as names.

Finally equipped with these standard features we can present the
dynamics of the calculus.

\subsubsection{Operational semantics} 

Finally, we introduce the computational dynamics. What marks these
algebras as distinct from other more traditionally studied algebraic
structures, e.g. vector spaces or polynomial rings, is the manner in
which dynamics is captured. In traditional structures, dynamics is typically
expressed through morphisms between such structures, as in linear maps
between vector spaces or morphisms between rings. In algebras
associated with the semantics of computation, the dynamics is
expressed as part of the algebraic structure itself, through a
reduction reduction relation typically denoted by $\red$. Below, we
give a recursive presentation of this relation for the calculus used
in the encoding.

$\red \subseteq \pi \times \pi$
$\red : \pi \to \mathcal{P}(\pi)$

\begin{mathpar}
  \inferrule* [lab=Comm] { \textsf{match}( x_{src}, x_{trgt} ) } { x_{trgt}?(y)P \; | \; x_{src}!\langle {Q} \rangle \red P\{\quotep{Q}/y}\} }
  \and \\
  \inferrule* [lab=Par] {{P} \red {P}'} {{{P} | {Q}} \red {{P}' | {Q}}}
  \and
  \inferrule* [lab=Equiv]{{{P} \scong {P}'} \andalso {{P}' \red {Q}'} \andalso {{Q}' \scong {Q}}}{{P} \red {Q}}
\end{mathpar}

\begin{eqnarray*}
  match_{\equiv} (\quotep{P},\quotep{Q}) & := & P \equiv Q \\
  match_{\dagger}(\quotep{P},\quotep{Q}) & := & \forall R. P|Q \red^{*} R => R \red^{*} 0 \\
  match_{K}(\quotep{P},\quotep{Q}) & := & K \mbox{ for some context } K
\end{eqnarray*}

$u?(x)P | u!\langle Q \rangle \red P\{\quotep{Q}/x\}$

%We write $\wred$ for $\red^*$, and $P\red$ if $\exists Q $ such that $ P \red Q$.
We write $P\red$ if $\exists Q $ such that $ P \red Q$ and $P\not\red$, otherwise.

\section{Replication}

As mentioned before, it is known that replication (and hence
recursion) can be implemented in a higher-order process algebra
\cite{SangiorgiWalker}. As our first example of calculation with the
machinery thus far presented we give the construction explicitly in
the {\rhoc}.

\begin{eqnarray}
	D_{x} & := & \prefix{x}{y}{(\binpar{\outputp{x}{y}}{@{y}})} \nonumber\\
	\bangp_{x}{P} & := & \binpar{{x}!\langle{\binpar{D_{x}}{P}}\rangle}{D_{x}} \nonumber
\end{eqnarray}

\begin{eqnarray}
	\bangp_{x}{P} & & \nonumber\\
	=
	& {x}!\langle{(\prefix{x}{y}{(\outputp{x}{y} | @{y})) | P}}\rangle 
	      | \prefix{x}{y}{(\outputp{x}{y} | @{y})} & \nonumber\\
	\red
	& (\outputp{x}{y} | @{y})\substn{\quotep{(\prefix{x}{y}{(@{y} | \outputp{x}{y})) | P}}}{y} & \nonumber\\
	=
	& \outputp{x}{\quotep{(\prefix{x}{y}{(\outputp{x}{y} | @{y})) | P}}}
	  | {(\prefix{x}{y}{(\outputp{x}{y} | @{y})) | P}} & \nonumber\\
	\red
	& \ldots & \nonumber\\
	\red^*
	& P | P | \ldots & \nonumber
\end{eqnarray}

Of course, this encoding, as an implementation, runs away, unfolding
$\bangp{P}$ eagerly. A lazier and more implementable replication
operator, restricted to input-guarded processes, may be obtained as follows.

\begin{eqnarray}
\bangp{\prefix{u}{v}{P}} 
	:= 
	\binpar{\lift{x}{\prefix{u}{v}{(\binpar{D(x)}{P})}}}{D(x)} \nonumber
\end{eqnarray}

\begin{remark}
  Note that the lazier definition still does not deal with summation
  or mixed summation (i.e. sums over input and output). The reader is
  invited to construct definitions of replication that deal with these
  features. 

  Further, the definitions are parameterized in a name, $x$. Can you,
  gentle reader, make a definition that eliminates this parameter and
  guarantees no accidental interaction between the replication
  machinery and the process being replicated -- i.e. no accidental
  sharing of names used by the process to get its work done and the
  name(s) used by the replication to effect copying. This latter
  revision of the definition of replication is crucial to obtaining
  the expected identity $!!P \sim !P$.
\end{remark}

\begin{remark}\label{rem:paradoxical_combinator}
  The reader familiar with the lambda calculus will have noticed the
  similarity between $D$ and the paradoxical combinator.

  [Ed. note: the existence of this seems to suggest we have to be more
  restrictive on the set of processes and names we admit if we are to
  support no-cloning.]
\end{remark}

\subsubsection{Bisimulation}

The computational dynamics gives rise to another kind of equivalence,
the equivalence of computational behavior. As previously mentioned
this is typically captured \emph{via} some form of bisimulation.

% The notion we use in this paper is weak barbed bisimulation
% \cite{milner91polyadicpi}.

The notion we use in this paper is derived from weak barbed
bisimulation \cite{milner91polyadicpi}. 

\begin{definition}
An \emph{observation relation}, $\downarrow_{\mathcal N}$, over a set
of names, $\mathcal N$, is the smallest relation satisfying the rules
below.

\infrule[Out-barb]{y \in {\mathcal N}, \; x \nameeq y}
		  {\outputp{x}{v} \downarrow_{\mathcal N} x}
\infrule[Par-barb]{\mbox{$P\downarrow_{\mathcal N} x$ or $Q\downarrow_{\mathcal N} x$}}
		  {\binpar{P}{Q} \downarrow_{\mathcal N} x}

We write $P \Downarrow_{\mathcal N} x$ if there is $Q$ such that 
$P \wred Q$ and $Q \downarrow_{\mathcal N} x$.
\end{definition}

\begin{definition}
%\label{def.bbisim}
An  ${\mathcal N}$-\emph{barbed bisimulation} over a set of names, ${\mathcal N}$, is a symmetric binary relation 
${\mathcal S}_{\mathcal N}$ between agents such that $P\rel{S}_{\mathcal N}Q$ implies:
\begin{enumerate}
\item If $P \red P'$ then $Q \wred Q'$ and $P'\rel{S}_{\mathcal N} Q'$.
\item If $P\downarrow_{\mathcal N} x$, then $Q\Downarrow_{\mathcal N} x$.
\end{enumerate}
$P$ is ${\mathcal N}$-barbed bisimilar to $Q$, written
$P \wbbisim_{\mathcal N} Q$, if $P \rel{S}_{\mathcal N} Q$ for some ${\mathcal N}$-barbed bisimulation ${\mathcal S}_{\mathcal N}$.
\end{definition}

$\mathcal{R} \subseteq \pi \times \pi$

$P \mathcal{R} Q => \forall P'. P \red P' \Rightarrow \exists Q'. Q \red Q', P' \mathcal{R} Q'$

$P \vdash x \Rightarrow Q \vdash x$

\begin{mathpar}
  \inferrule*[lab=Out-barb]{x \nameeq y}{{y}!\langle{Q}\rangle \vdash x}
  \and
  \inferrule*[lab=Par-barb]{\mbox{$P\vdash x$ or $Q\vdash x$}}{\binpar{P}{Q} \vdash x}
\end{mathpar}

\subsubsection{Contexts}

One of the principle advantages of computational calculi like the
$\pi$-calculus is a well-defined notion of context,
contextual-equivalence and a correlation between
contextual-equivalence and notions of bisimulation. The notion of
context allows the decomposition of a process into (sub-)process and
its syntactic environment, its context. Thus, a context may be
thought of as a process with a ``hole'' (written $\Box$) in it. The
application of a context $M$ to a process $P$, written $M[P]$, is
tantamount to filling the hole in $M$ with $P$. In this paper we do
not need the full weight of this theory, but do make use of the notion
of context in the proof the main theorem. 

\begin{mathpar}
  \inferrule* [lab=summation] {} {{M_{M},M_{N}} \bc \Box \;|\; x.M_{A} \;|\; M_{M}+M_{N}}
  \and
  \inferrule* [lab=agent] {} {{M_{A}} \bc (\vec{x})M_{P} \;| \; \clift{P_0,\ldots,M_{P},\ldots,P_N}}
  \and \\
  \inferrule* [lab=process] {} {{M_{P}} \bc M_{N} \;| \;P|M_{P} }
\end{mathpar} 

\begin{mathpar}
  \inferrule* [lab=sychronization] {} {M_{N} \bc \Box \;|\; x?M_{F} \;|\; x!M_{C}}
  \and
  \inferrule* [lab=abstraction] {} {{M_{F}} \bc (x)M_{P} }
  \and
  \inferrule* [lab=concretion] {} {{M_{C}} \bc \langle M_{P} \rangle }
  \and \\
  \inferrule* [lab=process] {} {{M_{P}} \bc M_{N} \;| \;P|M_{P} }
\end{mathpar}

\begin{definition}[contextual application] Given a context $M$, and
  process $P$, we define the \emph{contextual application}, $M[P] :=
  M\{P/\Box\}$. That is, the contextual application of M to P is the
  substitution of $P$ for $\Box$ in $M$.
\end{definition}

$\meaningof{-} : L \to \mathcal{P}(\pi)$

\begin{mathpar}
  \inferrule* [lab=collection] {} {\meaningof{true} = \pi, \and \meaningof{~E} = \pi \setminus \meaningof{E}, \and \meaningof{E_{1} \& E_{2}} = \meaningof{E_{1}} \cap \meaningof{E_{2}}}
\end{mathpar}

\begin{mathpar}
  \inferrule* [lab=structure] {} {\meaningof{0} = \{ P \in \pi | P \equiv 0 \}, \and \\ \meaningof{E_1 | E_2} = \{ P \in \pi | P \equiv P_{1} | P_{2}, P_{1} \in \meaningof{E_{1}}, P_{2} \in \meaningof{E_2}\} }
\end{mathpar}

\begin{mathpar}
 \inferrule* [lab=behavior] {} {\meaningof{\langle a?b \rangle E} = \{ P \in \pi | P \equiv Q | u?(y)P', \\ \and \\\\ \and \\ \;\;\; u \in \meaningof{a}, \forall z.P'\{z/y\} \in \meaningof{E\{z/b\}}\}, \and \\ \meaningof{a!E} = \{ P \in \pi | P \equiv Q | x!\langle P' \rangle, x \in \meaningof{a} P' \in \meaningof{E}\} }
\end{mathpar}

\begin{mathpar}
 \inferrule* [lab=nominal] {} {\meaningof{\quotep{E}} = \{ \quotep{P} \in \quotep{\pi} | P \in \meaningof{E} \}, \and \meaningof{\quotep{P}} = \{ \quotep{Q} \in \quotep{\pi} | P \equiv Q \} \and \\ \meaningof{@\quotep{E}} = \{ P \in \pi | P \equiv @x, x \in \meaningof{E} \}}
\end{mathpar}

\begin{eqnarray*}
  \\
  \meaningof{-} : TS \to ST
\end{eqnarray*}

\begin{eqnarray*}
  \\
  L : TS \to ST
\end{eqnarray*}

\begin{eqnarray*}
  \\
  P \models E \iff P \in \meaningof{E}
\end{eqnarray*}

\begin{eqnarray*}
  P \approx_{L} Q \iff \forall E \in L. P \models E \iff Q \models E
\end{eqnarray*}

\begin{eqnarray*}
  P \approx_{K} Q
\end{eqnarray*}

\begin{eqnarray*}
  P \approx Q
\end{eqnarray*}

$\approx_{K} = \approx = \approx_{L}$

\subsubsection{Contextual duality}

Note that contexts extend the quotation operation to a family of
operations from processes to names. Given a context, $M$, we can
define a \emph{nominal context}, $\quotep{M}$ by $\quotep{M}[P] :=
\quotep{M[P]}$. To foreshadow what is to come we observe that these
operations enjoy a duality with processes very much like the duality
between vectors and maps from vectors to scalars.

Further, because the calculus is essentially higher-order, we have a
correspondence between contexts and processes. More specifically,
given a name $x$ and a context $M$ we can construct $M^{*}_{x}$ such
that 

\begin{mathpar}
  M^{*}_{x} | \lift{x}{P} \red M[P]
\end{mathpar}

namely,

\begin{mathpar}
  M^{*}_{x} := x?(u).M[\dropn{u}]
\end{mathpar}

The dependence of $M^{*}_{x}$ on a name makes it an abstraction, 

\begin{mathpar}
  M^{*} := (x)x?(u).M[\dropn{u}]
\end{mathpar}

\subsection{Additional notation}

It will sometimes be convenient to denote the process a name
quotes. We already have the notation $x = \quotep{P}$, but it will be
convenient to introduce an alternate notation, $\procn{x}$, when we
want to emphasize the connection to the use of the name. Note that, by
virtue of name equivalence, $\quotep{\procn{x}} \nameeq x$; so, the
notation is consistent with previous definitions.

Further, because names have structure it is possible to effect
substitutions on the basis of that structure. This means we need to
upgrade our notation for substitutions, which we accomplish by
adapting comprehension notation. Thus,

\begin{mathpar}
  P\{ y / x : x \in S \}
\end{mathpar}

is interpreted to mean the process derived from P by replacing (in a
capture-avoiding manner) each occurrence of $x$ in $S$ by $y$. For example,

\begin{mathpar}
  P\{ \quotep{\procn{x}|\procn{x}} / x : x \in \freenames{P} \}
\end{mathpar}

will replace each (occurrence) of a free name $x$ in $P$ by
$\quotep{\procn{x}|\procn{x}}$.

Also, we will avail ourselves of the notation $x^{L}$ and $x^{R}$ to
denote injections of a name into disjoint copies of the name
space. There are numerous ways to accomplish this. One example can be
found in \cite{MeredithR05}. This notation overloads to vectors of
names: $\vec{x}^{\pi} := (x_{i}^{\pi} \; : \; 0 \leq i < |\vec{x}| )$ where $\pi \in \{L,R\}$.

We also use $P^{\Box} := P|\Box$.

In \cite{MeredithR05} an interpretation of the new operator is
given. It turns out that there are several possible interpretations
all enjoying the requisite algebraic properties of the operator (see
\cite{milner91polyadicpi}). We will therefore make liberal use of
$(\nu\; \vec{x})P$.

% subsection the_syntax_and_semantics_of_the_notation_system (end)   

\input{qm2pi.qmops} 

\input{qm2pi.sterngerlach} 

\input{qm2pi.metric} 

% section concurrent_process_calculi (end)

%\input{qm2pi.proofsketch}

% section proof sketch (end)

%\input{qm2pi.slviaknots} 

% section spatial logic via knots (end)

\input{qm2pi.conclusion}

% section conclusion (end)

%\input{qm2pi.dtcodes} 

% section wiring algorithm (end)

\input{qm2pi.ack} 

% section acknowledgments (end)

\newpage


\bibliographystyle{plain}   
\bibliography{../../biblios/main.bib}

\input{qm2pi.rhodetails}

\end{document}

 

%\ifpdf
%\usepackage[pdftex]{graphicx}
%\else
%\usepackage{graphicx}
%\fi

 % \ifpdf
%  \usepackage{pdfsync}
%  \if


%\title{Brief Article}
%\author{David F. Snyder}
%\author{L.G. Meredith}

%\address{Dept. of Math., Texas State University--San Marcos, San Marcos, TX 78666}
       
\pagestyle{empty}


\begin{document}

\lstset{language=[Objective]Caml,frame=shadowbox}

\documentclass[12pt]{llncs}
%\documentclass{jktr}

\usepackage[pdftex]{hyperref}                   
\usepackage {listings}
\usepackage {mathpartir}
\usepackage{bcprules}
%\usepackage{listings}
                       
\usepackage{graphicx} 
%\usepackage[margins=2.5cm,nohead,nofoot]{geometry}
%\usepackage{geometry}
\usepackage{amsfonts}
\usepackage{amstext}
\usepackage{latexsym}
\usepackage{amssymb}
\usepackage{color}


%\include{myPreamble}
\include{qm2pi.local} 

%\ifpdf
%\usepackage[pdftex]{graphicx}
%\else
%\usepackage{graphicx}
%\fi

 % \ifpdf
%  \usepackage{pdfsync}
%  \if


%\title{Brief Article}
%\author{David F. Snyder}
%\author{L.G. Meredith}

%\address{Dept. of Math., Texas State University--San Marcos, San Marcos, TX 78666}
       
\pagestyle{empty}


\begin{document}

\lstset{language=[Objective]Caml,frame=shadowbox}

\input{qm2pi.front}

% section front matter (end)

\input{qm2pi.intro} 
 
% section introduction (end)

% \input{qm2pi.knotations} 

% section notation (end)

\input{qm2pi.process.calculi} 

% section concurrent_process_calculi_and_spatial_logics_ (end)
    
%\input{qm2pi.knots2pi} 

%\input{qm2pi.trefoil} 

%\input{qm2pi.mainthm} 

% subsection basic_interpretation (end)

%\input{qm2pi.rho.presentation} 
\subsection{The syntax and semantics of the notation system}\label{sub:the_syntax_and_semantics_of_the_notation_system} % (fold)

We now summarize a technical presentation of the calculus that
embodies our theory of dynamics. The typical presentation of such a
calculus follows the style of giving generators and relations on
them. The grammar, below, describing term constructors, freely
generates the set of processes, $\Proc$. This set is then quotiented
by a relation known as structural congruence and it is over this set
that the notion of dynamics is expressed. This presentation is
essentially that of \cite{MeredithR05} with the addition of
polyadicity and summation. For readability we have relegated some of
the technical subtleties to an appendix.

\subsubsection{Process grammar}\label{subsub:process_grammar}

\begin{mathpar}
  \inferrule* [lab=synchronization] {} {{M} \bc \pzero \;|\; x?F \;|\; x!C }
  \and
  \inferrule* [lab=abstraction] {} {{F} \bc (x)P}
  \and
  \inferrule* [lab=concretion] {} {{C} \bc \langle Q \rangle}
  \and
  \inferrule* [lab=process] {} {{P,Q} \bc M \;| \;P|Q \;|\; @{x}}
  \and
  \inferrule* [lab=name] {} {{x} \bc \quotep{P}}
\end{mathpar} 

Note that $\vec{x}$ (resp. $\vec{P}$) denotes a vector of names
(resp. processes) of length $|\vec{x}|$ (resp. $|\vec{P}|$). We adopt
the following useful abbreviations.

\begin{mathpar}
   x?(\vec{y}).P := x.(\vec{y})P \and  x\clift{\vec{P}} := x.\clift{\vec{P}}
   \and x!(y) := \lift{x}{\dropn{y}}
   \and \Pi_{i=0}^{n-1}P_i := P_0 | \ldots | P_{n-1}
\end{mathpar}

\subsubsection{Structural congruence}

\paragraph{Free and bound names and alpha-equivalence.} At the
core of structural equivalence is alpha-equivalence which identifies
process that are the same up to a change of variable. Formally, we
recognize the distinction between free and bound names. The free names
of a process, $\freenames{P}$, may be calculated recursively as
follows:

\begin{mathpar}
\freenames{\pzero} := \emptyset
  \and \\
  \freenames{x?(y).P} := \{ x \} \cup (\freenames{P} \setminus \{ y \})
  \and 
  \freenames{x!\langle P \rangle} := \{ x \} \cup \{ P \} 
  \and \\
  \freenames{P|Q} := \freenames{P} \cup \freenames{Q}
  \and \\
  \freenames{@{x}} := \{ x \}
\end{mathpar}

$\pi$
$\quotep{\pi}$

$\freenames{-} : \pi \to \mathcal{P}(\quotep{\pi})$

\begin{eqnarray*}
  \freenames{\pzero} & := & \emptyset \\
  \freenames{x?(y).P} & := & \{ x \} \cup (\freenames{P} \setminus \{ y \}) \\
  \freenames{x!\langle P \rangle} & := & \{ x \} \cup \{ P \} \\
  \freenames{P|Q} & := & \freenames{P} \cup \freenames{Q} \\
  \freenames{\dropn{x}} & := & \{ x \}
\end{eqnarray*}

The bound names of a process, $\boundnames{P}$, are those names occurring in $P$
that are not free. For example, in $x?(y).0$, the name $x$ is free, while $y$ is bound.

\begin{mathpar}
  \inferrule* [lab=monoidal-laws] {} { P|Q \equiv Q|P \and P|0 \equiv P \and P|(Q|R) \equiv (P|Q)|R }
\end{mathpar}

\begin{mathpar}
  \inferrule* [lab=alpha-equivalence] {} { (x)P \equiv (y)P\{y/x\} \and y \not\in \freenames{P} }
\end{mathpar}

\begin{definition}
Then two processes, $P,Q$, are alpha-equivalent if $P = Q\{\vec{y}/\vec{x}\}$ for
some $\vec{x} \in \boundnames{Q},\vec{y} \in \boundnames{P}$, where $Q\{\vec{y}/\vec{x}\}$
denotes the capture-avoiding substitution of $\vec{y}$ for $\vec{x}$ in $Q$.
\end{definition}

\begin{definition}
  The {\em structural congruence} \cite{SangiorgiWalker} , $\equiv$,
  between processes is the least congruence containing
  alpha-equivalence, satisfying the abelian monoid laws
  (associativity, commutativity and $\pzero$ as identity) for parallel
  composition $|$ and for summation $+$.
\end{definition}

\subsection{Name equivalence}

We take name equivalence, written $\nameeq$, to be the smallest
equivalence relation generated by the following rules.

\begin{mathpar}
\inferrule*[lab=Quote-drop]
{ }
{ \quotep{@{x}} \nameeq x }

\inferrule*[lab=Struct-equiv]
{ P \scong Q }
{ \quotep{P} \nameeq \quotep{Q} }
\end{mathpar}

The astute reader will have noticed that the mutual recursion of names
and processes imposes a mutual recursion on alpha-equivalence and
structural equivalence via name-equivalence. Fortunately, all of this
works out pleasantly and we may calculate in the natural way, free of
concern. The reader interested in the details is referred to the
appendix \ref{appendix:rho_details}.

\subsection{Substitution}

We use $\Proc$ for the set of processes, $\QProc$ for the set of
names, and $\id{\{}\vec{y} / \vec{x} \id{\}}$ to denote partial maps,
$s : \QProc \rightarrow \QProc$. A map, $s$ lifts, uniquely, to a map
on process terms, $\widehat{s} : \Proc \rightarrow \Proc$ by the
following equations.

\begin{mathpar}
  (0) \psubstp{Q}{P} := 0 \\
  (R \juxtap S) \psubstp{Q}{P}
  :=    
  (R)\psubstp{Q}{P} \juxtap (S) \psubstp{Q}{P} \\
  (x?(y).R) \psubstp{Q}{P}    
  :=    
  (x)\substp{Q}{P} (z)\concat( (R \psubstn{z}{y}) \psubstp{Q}{P} ) \\
  (\lift{x}{R}) \psubstp{Q}{P}  
  :=
  \lift{(x)\substp{Q}{P}}{ R \psubstp{Q}{P} } \\
%   (\dropn{x})  \psubstp{Q}{P}       
%   := 
%   \left\{ 
%     \begin{array}{ccc} 
%       \dropn{\quotep{Q}} & & x \nameeq \quotep{P} \\
%       \dropn{x} & & otherwise \\
%     \end{array}
%   \right. 
  (\dropn{x})  \psubstp{Q}{P}       
  := 
  \left\{ 
    \begin{array}{ccc} 
      Q & & x \nameeq \quotep{P} \\
      \dropn{x} & & otherwise \\
    \end{array}
  \right.
\end{mathpar}
 

where

\begin{eqnarray}
  (x)\id{\{} \lpquote Q \rpquote / \lpquote P \rpquote \id{\}}            = 
  \left\{ 
    \begin{array}{ccc}
      \lpquote Q \rpquote & & x \nameeq \lpquote P \rpquote \\
      x & & otherwise \\
    \end{array}
  \right. \nonumber
\end{eqnarray}

and $z$ is chosen distinct from $\quotep{P}$, $\quotep{Q}$, the free
names in $Q$, and all the names in $R$. Our $\alpha$-equivalence will
be built in the standard way from this substitution.

\begin{remark}\label{rem:no_self_referential_names}
  One consequence of these definitions is that $\forall P. \quotep{P}
  \not\in \freenames{P}$.
\end{remark}

\subsection{ Dynamic quote: an example }

Anticipating something of what's to come, consider applying the
substitution, $\widehat{\id{\{}u / z \id{\}}}$, to the following pair
of processes, $\lift{w}{y!(z)}$ and $w[ \lpquote y!(z) \rpquote ]$.

\begin{eqnarray}
	\lift{w}{y!(z)}\widehat{\id{\{}u / z \id{\}}}
		& = &
		\lift{w}{y!(u)} \nonumber\\
	w[ \lpquote y!(z) \rpquote ] \widehat{ \id{\{}u / z \id{\}} }
		& = &
		w[ \lpquote y!(z) \rpquote ] \nonumber
\end{eqnarray}

Because the body of the process between quotes is impervious to
substitution, we get radically different answers. In fact, by
examining the first process in an input context,
e.g. $x?(z).\lift{w}{y!(z)}$, we see that the process under the lift
operator may be shaped by prefixed inputs binding a name inside it. In
this sense, the lift operator will be seen as a way to dynamically
construct processes before reifying them as names.

Finally equipped with these standard features we can present the
dynamics of the calculus.

\subsubsection{Operational semantics} 

Finally, we introduce the computational dynamics. What marks these
algebras as distinct from other more traditionally studied algebraic
structures, e.g. vector spaces or polynomial rings, is the manner in
which dynamics is captured. In traditional structures, dynamics is typically
expressed through morphisms between such structures, as in linear maps
between vector spaces or morphisms between rings. In algebras
associated with the semantics of computation, the dynamics is
expressed as part of the algebraic structure itself, through a
reduction reduction relation typically denoted by $\red$. Below, we
give a recursive presentation of this relation for the calculus used
in the encoding.

$\red \subseteq \pi \times \pi$
$\red : \pi \to \mathcal{P}(\pi)$

\begin{mathpar}
  \inferrule* [lab=Comm] { \textsf{match}( x_{src}, x_{trgt} ) } { x_{trgt}?(y)P \; | \; x_{src}!\langle {Q} \rangle \red P\{\quotep{Q}/y}\} }
  \and \\
  \inferrule* [lab=Par] {{P} \red {P}'} {{{P} | {Q}} \red {{P}' | {Q}}}
  \and
  \inferrule* [lab=Equiv]{{{P} \scong {P}'} \andalso {{P}' \red {Q}'} \andalso {{Q}' \scong {Q}}}{{P} \red {Q}}
\end{mathpar}

\begin{eqnarray*}
  match_{\equiv} (\quotep{P},\quotep{Q}) & := & P \equiv Q \\
  match_{\dagger}(\quotep{P},\quotep{Q}) & := & \forall R. P|Q \red^{*} R => R \red^{*} 0 \\
  match_{K}(\quotep{P},\quotep{Q}) & := & K \mbox{ for some context } K
\end{eqnarray*}

$u?(x)P | u!\langle Q \rangle \red P\{\quotep{Q}/x\}$

%We write $\wred$ for $\red^*$, and $P\red$ if $\exists Q $ such that $ P \red Q$.
We write $P\red$ if $\exists Q $ such that $ P \red Q$ and $P\not\red$, otherwise.

\section{Replication}

As mentioned before, it is known that replication (and hence
recursion) can be implemented in a higher-order process algebra
\cite{SangiorgiWalker}. As our first example of calculation with the
machinery thus far presented we give the construction explicitly in
the {\rhoc}.

\begin{eqnarray}
	D_{x} & := & \prefix{x}{y}{(\binpar{\outputp{x}{y}}{@{y}})} \nonumber\\
	\bangp_{x}{P} & := & \binpar{{x}!\langle{\binpar{D_{x}}{P}}\rangle}{D_{x}} \nonumber
\end{eqnarray}

\begin{eqnarray}
	\bangp_{x}{P} & & \nonumber\\
	=
	& {x}!\langle{(\prefix{x}{y}{(\outputp{x}{y} | @{y})) | P}}\rangle 
	      | \prefix{x}{y}{(\outputp{x}{y} | @{y})} & \nonumber\\
	\red
	& (\outputp{x}{y} | @{y})\substn{\quotep{(\prefix{x}{y}{(@{y} | \outputp{x}{y})) | P}}}{y} & \nonumber\\
	=
	& \outputp{x}{\quotep{(\prefix{x}{y}{(\outputp{x}{y} | @{y})) | P}}}
	  | {(\prefix{x}{y}{(\outputp{x}{y} | @{y})) | P}} & \nonumber\\
	\red
	& \ldots & \nonumber\\
	\red^*
	& P | P | \ldots & \nonumber
\end{eqnarray}

Of course, this encoding, as an implementation, runs away, unfolding
$\bangp{P}$ eagerly. A lazier and more implementable replication
operator, restricted to input-guarded processes, may be obtained as follows.

\begin{eqnarray}
\bangp{\prefix{u}{v}{P}} 
	:= 
	\binpar{\lift{x}{\prefix{u}{v}{(\binpar{D(x)}{P})}}}{D(x)} \nonumber
\end{eqnarray}

\begin{remark}
  Note that the lazier definition still does not deal with summation
  or mixed summation (i.e. sums over input and output). The reader is
  invited to construct definitions of replication that deal with these
  features. 

  Further, the definitions are parameterized in a name, $x$. Can you,
  gentle reader, make a definition that eliminates this parameter and
  guarantees no accidental interaction between the replication
  machinery and the process being replicated -- i.e. no accidental
  sharing of names used by the process to get its work done and the
  name(s) used by the replication to effect copying. This latter
  revision of the definition of replication is crucial to obtaining
  the expected identity $!!P \sim !P$.
\end{remark}

\begin{remark}\label{rem:paradoxical_combinator}
  The reader familiar with the lambda calculus will have noticed the
  similarity between $D$ and the paradoxical combinator.

  [Ed. note: the existence of this seems to suggest we have to be more
  restrictive on the set of processes and names we admit if we are to
  support no-cloning.]
\end{remark}

\subsubsection{Bisimulation}

The computational dynamics gives rise to another kind of equivalence,
the equivalence of computational behavior. As previously mentioned
this is typically captured \emph{via} some form of bisimulation.

% The notion we use in this paper is weak barbed bisimulation
% \cite{milner91polyadicpi}.

The notion we use in this paper is derived from weak barbed
bisimulation \cite{milner91polyadicpi}. 

\begin{definition}
An \emph{observation relation}, $\downarrow_{\mathcal N}$, over a set
of names, $\mathcal N$, is the smallest relation satisfying the rules
below.

\infrule[Out-barb]{y \in {\mathcal N}, \; x \nameeq y}
		  {\outputp{x}{v} \downarrow_{\mathcal N} x}
\infrule[Par-barb]{\mbox{$P\downarrow_{\mathcal N} x$ or $Q\downarrow_{\mathcal N} x$}}
		  {\binpar{P}{Q} \downarrow_{\mathcal N} x}

We write $P \Downarrow_{\mathcal N} x$ if there is $Q$ such that 
$P \wred Q$ and $Q \downarrow_{\mathcal N} x$.
\end{definition}

\begin{definition}
%\label{def.bbisim}
An  ${\mathcal N}$-\emph{barbed bisimulation} over a set of names, ${\mathcal N}$, is a symmetric binary relation 
${\mathcal S}_{\mathcal N}$ between agents such that $P\rel{S}_{\mathcal N}Q$ implies:
\begin{enumerate}
\item If $P \red P'$ then $Q \wred Q'$ and $P'\rel{S}_{\mathcal N} Q'$.
\item If $P\downarrow_{\mathcal N} x$, then $Q\Downarrow_{\mathcal N} x$.
\end{enumerate}
$P$ is ${\mathcal N}$-barbed bisimilar to $Q$, written
$P \wbbisim_{\mathcal N} Q$, if $P \rel{S}_{\mathcal N} Q$ for some ${\mathcal N}$-barbed bisimulation ${\mathcal S}_{\mathcal N}$.
\end{definition}

$\mathcal{R} \subseteq \pi \times \pi$

$P \mathcal{R} Q => \forall P'. P \red P' \Rightarrow \exists Q'. Q \red Q', P' \mathcal{R} Q'$

$P \vdash x \Rightarrow Q \vdash x$

\begin{mathpar}
  \inferrule*[lab=Out-barb]{x \nameeq y}{{y}!\langle{Q}\rangle \vdash x}
  \and
  \inferrule*[lab=Par-barb]{\mbox{$P\vdash x$ or $Q\vdash x$}}{\binpar{P}{Q} \vdash x}
\end{mathpar}

\subsubsection{Contexts}

One of the principle advantages of computational calculi like the
$\pi$-calculus is a well-defined notion of context,
contextual-equivalence and a correlation between
contextual-equivalence and notions of bisimulation. The notion of
context allows the decomposition of a process into (sub-)process and
its syntactic environment, its context. Thus, a context may be
thought of as a process with a ``hole'' (written $\Box$) in it. The
application of a context $M$ to a process $P$, written $M[P]$, is
tantamount to filling the hole in $M$ with $P$. In this paper we do
not need the full weight of this theory, but do make use of the notion
of context in the proof the main theorem. 

\begin{mathpar}
  \inferrule* [lab=summation] {} {{M_{M},M_{N}} \bc \Box \;|\; x.M_{A} \;|\; M_{M}+M_{N}}
  \and
  \inferrule* [lab=agent] {} {{M_{A}} \bc (\vec{x})M_{P} \;| \; \clift{P_0,\ldots,M_{P},\ldots,P_N}}
  \and \\
  \inferrule* [lab=process] {} {{M_{P}} \bc M_{N} \;| \;P|M_{P} }
\end{mathpar} 

\begin{mathpar}
  \inferrule* [lab=sychronization] {} {M_{N} \bc \Box \;|\; x?M_{F} \;|\; x!M_{C}}
  \and
  \inferrule* [lab=abstraction] {} {{M_{F}} \bc (x)M_{P} }
  \and
  \inferrule* [lab=concretion] {} {{M_{C}} \bc \langle M_{P} \rangle }
  \and \\
  \inferrule* [lab=process] {} {{M_{P}} \bc M_{N} \;| \;P|M_{P} }
\end{mathpar}

\begin{definition}[contextual application] Given a context $M$, and
  process $P$, we define the \emph{contextual application}, $M[P] :=
  M\{P/\Box\}$. That is, the contextual application of M to P is the
  substitution of $P$ for $\Box$ in $M$.
\end{definition}

$\meaningof{-} : L \to \mathcal{P}(\pi)$

\begin{mathpar}
  \inferrule* [lab=collection] {} {\meaningof{true} = \pi, \and \meaningof{~E} = \pi \setminus \meaningof{E}, \and \meaningof{E_{1} \& E_{2}} = \meaningof{E_{1}} \cap \meaningof{E_{2}}}
\end{mathpar}

\begin{mathpar}
  \inferrule* [lab=structure] {} {\meaningof{0} = \{ P \in \pi | P \equiv 0 \}, \and \\ \meaningof{E_1 | E_2} = \{ P \in \pi | P \equiv P_{1} | P_{2}, P_{1} \in \meaningof{E_{1}}, P_{2} \in \meaningof{E_2}\} }
\end{mathpar}

\begin{mathpar}
 \inferrule* [lab=behavior] {} {\meaningof{\langle a?b \rangle E} = \{ P \in \pi | P \equiv Q | u?(y)P', \\ \and \\\\ \and \\ \;\;\; u \in \meaningof{a}, \forall z.P'\{z/y\} \in \meaningof{E\{z/b\}}\}, \and \\ \meaningof{a!E} = \{ P \in \pi | P \equiv Q | x!\langle P' \rangle, x \in \meaningof{a} P' \in \meaningof{E}\} }
\end{mathpar}

\begin{mathpar}
 \inferrule* [lab=nominal] {} {\meaningof{\quotep{E}} = \{ \quotep{P} \in \quotep{\pi} | P \in \meaningof{E} \}, \and \meaningof{\quotep{P}} = \{ \quotep{Q} \in \quotep{\pi} | P \equiv Q \} \and \\ \meaningof{@\quotep{E}} = \{ P \in \pi | P \equiv @x, x \in \meaningof{E} \}}
\end{mathpar}

\begin{eqnarray*}
  \\
  \meaningof{-} : TS \to ST
\end{eqnarray*}

\begin{eqnarray*}
  \\
  L : TS \to ST
\end{eqnarray*}

\begin{eqnarray*}
  \\
  P \models E \iff P \in \meaningof{E}
\end{eqnarray*}

\begin{eqnarray*}
  P \approx_{L} Q \iff \forall E \in L. P \models E \iff Q \models E
\end{eqnarray*}

\begin{eqnarray*}
  P \approx_{K} Q
\end{eqnarray*}

\begin{eqnarray*}
  P \approx Q
\end{eqnarray*}

$\approx_{K} = \approx = \approx_{L}$

\subsubsection{Contextual duality}

Note that contexts extend the quotation operation to a family of
operations from processes to names. Given a context, $M$, we can
define a \emph{nominal context}, $\quotep{M}$ by $\quotep{M}[P] :=
\quotep{M[P]}$. To foreshadow what is to come we observe that these
operations enjoy a duality with processes very much like the duality
between vectors and maps from vectors to scalars.

Further, because the calculus is essentially higher-order, we have a
correspondence between contexts and processes. More specifically,
given a name $x$ and a context $M$ we can construct $M^{*}_{x}$ such
that 

\begin{mathpar}
  M^{*}_{x} | \lift{x}{P} \red M[P]
\end{mathpar}

namely,

\begin{mathpar}
  M^{*}_{x} := x?(u).M[\dropn{u}]
\end{mathpar}

The dependence of $M^{*}_{x}$ on a name makes it an abstraction, 

\begin{mathpar}
  M^{*} := (x)x?(u).M[\dropn{u}]
\end{mathpar}

\subsection{Additional notation}

It will sometimes be convenient to denote the process a name
quotes. We already have the notation $x = \quotep{P}$, but it will be
convenient to introduce an alternate notation, $\procn{x}$, when we
want to emphasize the connection to the use of the name. Note that, by
virtue of name equivalence, $\quotep{\procn{x}} \nameeq x$; so, the
notation is consistent with previous definitions.

Further, because names have structure it is possible to effect
substitutions on the basis of that structure. This means we need to
upgrade our notation for substitutions, which we accomplish by
adapting comprehension notation. Thus,

\begin{mathpar}
  P\{ y / x : x \in S \}
\end{mathpar}

is interpreted to mean the process derived from P by replacing (in a
capture-avoiding manner) each occurrence of $x$ in $S$ by $y$. For example,

\begin{mathpar}
  P\{ \quotep{\procn{x}|\procn{x}} / x : x \in \freenames{P} \}
\end{mathpar}

will replace each (occurrence) of a free name $x$ in $P$ by
$\quotep{\procn{x}|\procn{x}}$.

Also, we will avail ourselves of the notation $x^{L}$ and $x^{R}$ to
denote injections of a name into disjoint copies of the name
space. There are numerous ways to accomplish this. One example can be
found in \cite{MeredithR05}. This notation overloads to vectors of
names: $\vec{x}^{\pi} := (x_{i}^{\pi} \; : \; 0 \leq i < |\vec{x}| )$ where $\pi \in \{L,R\}$.

We also use $P^{\Box} := P|\Box$.

In \cite{MeredithR05} an interpretation of the new operator is
given. It turns out that there are several possible interpretations
all enjoying the requisite algebraic properties of the operator (see
\cite{milner91polyadicpi}). We will therefore make liberal use of
$(\nu\; \vec{x})P$.

% subsection the_syntax_and_semantics_of_the_notation_system (end)   

\input{qm2pi.qmops} 

\input{qm2pi.sterngerlach} 

\input{qm2pi.metric} 

% section concurrent_process_calculi (end)

%\input{qm2pi.proofsketch}

% section proof sketch (end)

%\input{qm2pi.slviaknots} 

% section spatial logic via knots (end)

\input{qm2pi.conclusion}

% section conclusion (end)

%\input{qm2pi.dtcodes} 

% section wiring algorithm (end)

\input{qm2pi.ack} 

% section acknowledgments (end)

\newpage


\bibliographystyle{plain}   
\bibliography{../../biblios/main.bib}

\input{qm2pi.rhodetails}

\end{document}



% section front matter (end)

\section{Introduction}\label{sec:introduction} % (fold)
In this draft of the material i am going to have to dispense with the
usual writing conventions adopted in papers on these topics. i'm going
to have adopt whatever tone i need at the time i'm writing up the
calculations. Sometimes this may be very conversational; others it may
be the barest mathematical grunts; others still it may be that i have
lifted text from one of my other papers because the exposition of some
point was better said there. i hope that my readers are not unduly put
out by this decision. i'm not doing this to flout convention or be
rebellious. i find these calculations very technically challenging. To
keep everything going technically, something has to give; i have to
let go of some cognitive burden. So, the academic writing style --
with all of its trade-offs in terms of facilitating technical
communication -- is what i'm letting go of. Perhaps subsequent drafts
can be tightened and polished, but for now, i'm going to speak as if
we were sitting together in a coffee shop with a laptop, wifi and a
pad of paper and a pencil.

So, here's what i have to say. We -- you and i, comfortably ensconced
in our coffee shop and well-equipped with our tools -- can realize and
carry out the calculations of quantum mechanics over a very different
formal theory of dynamics, a formal theory of dynamics that
corresponds to a theory of concurrent computation with
\emph{reflection}. It has the advantage that the underlying theory is
already `quantized', but supports analogues all of the continuuous
operations. Strikingly, this underlying theory has recently been
connected with a notion of metric that we can show, by calculating
together, coincides with the metric induced by the inner product.

There are a lot of reasons why you might be interested in seeing
calculations of this form. Here's why i'm interested. For the past
several centuries there has been no competitor to the ``Newtonian''
account of dynamics. As a result the predominant share of accounts of
dynamical systems and situations have had to be formulated in terms of
the Newtonian machinery. i view this as an intellectually dangerous
position to occupy. Everything, despite it's intrinsic shape, turns
into a nail to be hit with this hammer. Recently, however, the theory
of computation has matured to the point where we have candidates for
theories of dynamics that offer very different perspective on
reasoning about dynamical systems and situations. Testing these
candidates against very successful accounts of dynamical situations,
like quantum mechanics, is going to give us some sense of how mature
they are and some measure of the quality of these accounts of
dynamics.

\subsection{Summary of contributions and outline of paper}

So, we're going to develop an interpretation of the operations of
quantum mechanics normally interpreted by Hilbert spaces and
operators. We're going to do this over a theory of computation. Note
that this is very different than the usual quantum computation program
which develops notions of computation over quantum mechanics. Rather,
we are developing a story that aligns with Wheeler's slogan: It from
Bit. To do this we will first provide an account of the theory of
computation at play here. Then we will dive into a calculation-driven
interpretation of the operations of quantum mechanics.

The reason we take this approach is that -- until very recently --
there hasn't been an axiomatic account of quantum mechanics. As a
result there has been no sharp delineation of the mathematical theory
supporting interpretation of the physical theory and the physical
theory, itself. So, ambient features of the maths are free to be
exploited (or supressed) without a real accounting of their physical
relevance. There is no sharp statement ``here's the physical theory''
qua \emph{theory} and ``here's the mathematical interpretation''
enabling a judgment of how faithful the interpretation is -- apart
from experimental observation. When there is an axiomatic account we
can judge how well a given mathematical formalism supports an
interpretation of the axioms, independent of
experimentation. Likewise, we can judge how well we have captured our
physical evidence and experience with our axiomatics, independent of
any specific mathematical implementation, with accidental detail that
may or may not have physical significance. 

In lieu of a fully fleshed out and vetted axiomatic account of quantum
mechanics, interpreting the operational notions in service of modeling
physical systems will have to suffice. In other words, we are not in
the business of providing a model of Hilbert spaces and operators. We
are in the business of providing a model of quantum mechanics because
we are motivated by testing our notions of dynamics against physical
theory; and, the predictive calculations of the physical theory must
serve as the best formulation -- shy of a fully fleshed out axiomatic
account -- of the physical theory itself (as they have for scientific
theories since time immemorial). Put another way, despite a
whole-hearted commitment to an It-from-Bit ontology, we are firmly
aligned with the shut-up-and-calculate camp as the best way to obtain
results either from the physical perspective or as a quality assurance
measure of our fledgling theory of dynamics.

In detail, we present a reflective process calculus. Then we develop
intuitive correspondences between the notions available in this
calculus and the usual physical notions supporting quantum mechanical
calculations. Thus, 

\begin{table}[htp]
  \center{
    \fbox{
      \begin{tabular}{c|c}
        quantum mechanics & process calculus \\
        \hline
        scalar & name \\
        state vector & process \\
        dual & contextual duals \\
        matrix & formal sums of process-context-dual pairs \\
        orthogonality & process annihilation \\
        inner product & execution-formula + quoting
      \end{tabular}
    }
  }
  \caption{QM - process calculi correspondences}
\end{table}

Then we tighten up these intuitions to operational definitions. We
employ the Dirac notation as the best proxy we can find for an
abstract syntax of the quantum mechanical notions. The definitions we
develop put us in contact with equational constraints coming from the
theory that we demonstrate the definitions and calculations satisfy.

This puts us in a position to shut up and calculate for the
Stern-Gerlach experimental set up, showing how these predictive
calculations become calculations on processes in our theory of a
reflective process calculus.

Penultimately, we demonstrate that the notion of metric coming from
the inner product coincides with the notion of metric available from
the theory of bisimulation. This demonstration gives us the right to
think of space as arising from behavior. Finally, we consider where we
might go from the new vantage point we have obtained.

% section introduction (end) 
 
% section introduction (end)

% \documentclass[12pt]{llncs}
%\documentclass{jktr}

\usepackage[pdftex]{hyperref}                   
\usepackage {listings}
\usepackage {mathpartir}
\usepackage{bcprules}
%\usepackage{listings}
                       
\usepackage{graphicx} 
%\usepackage[margins=2.5cm,nohead,nofoot]{geometry}
%\usepackage{geometry}
\usepackage{amsfonts}
\usepackage{amstext}
\usepackage{latexsym}
\usepackage{amssymb}
\usepackage{color}


%\include{myPreamble}
\include{qm2pi.local} 

%\ifpdf
%\usepackage[pdftex]{graphicx}
%\else
%\usepackage{graphicx}
%\fi

 % \ifpdf
%  \usepackage{pdfsync}
%  \if


%\title{Brief Article}
%\author{David F. Snyder}
%\author{L.G. Meredith}

%\address{Dept. of Math., Texas State University--San Marcos, San Marcos, TX 78666}
       
\pagestyle{empty}


\begin{document}

\lstset{language=[Objective]Caml,frame=shadowbox}

\input{qm2pi.front}

% section front matter (end)

\input{qm2pi.intro} 
 
% section introduction (end)

% \input{qm2pi.knotations} 

% section notation (end)

\input{qm2pi.process.calculi} 

% section concurrent_process_calculi_and_spatial_logics_ (end)
    
%\input{qm2pi.knots2pi} 

%\input{qm2pi.trefoil} 

%\input{qm2pi.mainthm} 

% subsection basic_interpretation (end)

%\input{qm2pi.rho.presentation} 
\subsection{The syntax and semantics of the notation system}\label{sub:the_syntax_and_semantics_of_the_notation_system} % (fold)

We now summarize a technical presentation of the calculus that
embodies our theory of dynamics. The typical presentation of such a
calculus follows the style of giving generators and relations on
them. The grammar, below, describing term constructors, freely
generates the set of processes, $\Proc$. This set is then quotiented
by a relation known as structural congruence and it is over this set
that the notion of dynamics is expressed. This presentation is
essentially that of \cite{MeredithR05} with the addition of
polyadicity and summation. For readability we have relegated some of
the technical subtleties to an appendix.

\subsubsection{Process grammar}\label{subsub:process_grammar}

\begin{mathpar}
  \inferrule* [lab=synchronization] {} {{M} \bc \pzero \;|\; x?F \;|\; x!C }
  \and
  \inferrule* [lab=abstraction] {} {{F} \bc (x)P}
  \and
  \inferrule* [lab=concretion] {} {{C} \bc \langle Q \rangle}
  \and
  \inferrule* [lab=process] {} {{P,Q} \bc M \;| \;P|Q \;|\; @{x}}
  \and
  \inferrule* [lab=name] {} {{x} \bc \quotep{P}}
\end{mathpar} 

Note that $\vec{x}$ (resp. $\vec{P}$) denotes a vector of names
(resp. processes) of length $|\vec{x}|$ (resp. $|\vec{P}|$). We adopt
the following useful abbreviations.

\begin{mathpar}
   x?(\vec{y}).P := x.(\vec{y})P \and  x\clift{\vec{P}} := x.\clift{\vec{P}}
   \and x!(y) := \lift{x}{\dropn{y}}
   \and \Pi_{i=0}^{n-1}P_i := P_0 | \ldots | P_{n-1}
\end{mathpar}

\subsubsection{Structural congruence}

\paragraph{Free and bound names and alpha-equivalence.} At the
core of structural equivalence is alpha-equivalence which identifies
process that are the same up to a change of variable. Formally, we
recognize the distinction between free and bound names. The free names
of a process, $\freenames{P}$, may be calculated recursively as
follows:

\begin{mathpar}
\freenames{\pzero} := \emptyset
  \and \\
  \freenames{x?(y).P} := \{ x \} \cup (\freenames{P} \setminus \{ y \})
  \and 
  \freenames{x!\langle P \rangle} := \{ x \} \cup \{ P \} 
  \and \\
  \freenames{P|Q} := \freenames{P} \cup \freenames{Q}
  \and \\
  \freenames{@{x}} := \{ x \}
\end{mathpar}

$\pi$
$\quotep{\pi}$

$\freenames{-} : \pi \to \mathcal{P}(\quotep{\pi})$

\begin{eqnarray*}
  \freenames{\pzero} & := & \emptyset \\
  \freenames{x?(y).P} & := & \{ x \} \cup (\freenames{P} \setminus \{ y \}) \\
  \freenames{x!\langle P \rangle} & := & \{ x \} \cup \{ P \} \\
  \freenames{P|Q} & := & \freenames{P} \cup \freenames{Q} \\
  \freenames{\dropn{x}} & := & \{ x \}
\end{eqnarray*}

The bound names of a process, $\boundnames{P}$, are those names occurring in $P$
that are not free. For example, in $x?(y).0$, the name $x$ is free, while $y$ is bound.

\begin{mathpar}
  \inferrule* [lab=monoidal-laws] {} { P|Q \equiv Q|P \and P|0 \equiv P \and P|(Q|R) \equiv (P|Q)|R }
\end{mathpar}

\begin{mathpar}
  \inferrule* [lab=alpha-equivalence] {} { (x)P \equiv (y)P\{y/x\} \and y \not\in \freenames{P} }
\end{mathpar}

\begin{definition}
Then two processes, $P,Q$, are alpha-equivalent if $P = Q\{\vec{y}/\vec{x}\}$ for
some $\vec{x} \in \boundnames{Q},\vec{y} \in \boundnames{P}$, where $Q\{\vec{y}/\vec{x}\}$
denotes the capture-avoiding substitution of $\vec{y}$ for $\vec{x}$ in $Q$.
\end{definition}

\begin{definition}
  The {\em structural congruence} \cite{SangiorgiWalker} , $\equiv$,
  between processes is the least congruence containing
  alpha-equivalence, satisfying the abelian monoid laws
  (associativity, commutativity and $\pzero$ as identity) for parallel
  composition $|$ and for summation $+$.
\end{definition}

\subsection{Name equivalence}

We take name equivalence, written $\nameeq$, to be the smallest
equivalence relation generated by the following rules.

\begin{mathpar}
\inferrule*[lab=Quote-drop]
{ }
{ \quotep{@{x}} \nameeq x }

\inferrule*[lab=Struct-equiv]
{ P \scong Q }
{ \quotep{P} \nameeq \quotep{Q} }
\end{mathpar}

The astute reader will have noticed that the mutual recursion of names
and processes imposes a mutual recursion on alpha-equivalence and
structural equivalence via name-equivalence. Fortunately, all of this
works out pleasantly and we may calculate in the natural way, free of
concern. The reader interested in the details is referred to the
appendix \ref{appendix:rho_details}.

\subsection{Substitution}

We use $\Proc$ for the set of processes, $\QProc$ for the set of
names, and $\id{\{}\vec{y} / \vec{x} \id{\}}$ to denote partial maps,
$s : \QProc \rightarrow \QProc$. A map, $s$ lifts, uniquely, to a map
on process terms, $\widehat{s} : \Proc \rightarrow \Proc$ by the
following equations.

\begin{mathpar}
  (0) \psubstp{Q}{P} := 0 \\
  (R \juxtap S) \psubstp{Q}{P}
  :=    
  (R)\psubstp{Q}{P} \juxtap (S) \psubstp{Q}{P} \\
  (x?(y).R) \psubstp{Q}{P}    
  :=    
  (x)\substp{Q}{P} (z)\concat( (R \psubstn{z}{y}) \psubstp{Q}{P} ) \\
  (\lift{x}{R}) \psubstp{Q}{P}  
  :=
  \lift{(x)\substp{Q}{P}}{ R \psubstp{Q}{P} } \\
%   (\dropn{x})  \psubstp{Q}{P}       
%   := 
%   \left\{ 
%     \begin{array}{ccc} 
%       \dropn{\quotep{Q}} & & x \nameeq \quotep{P} \\
%       \dropn{x} & & otherwise \\
%     \end{array}
%   \right. 
  (\dropn{x})  \psubstp{Q}{P}       
  := 
  \left\{ 
    \begin{array}{ccc} 
      Q & & x \nameeq \quotep{P} \\
      \dropn{x} & & otherwise \\
    \end{array}
  \right.
\end{mathpar}
 

where

\begin{eqnarray}
  (x)\id{\{} \lpquote Q \rpquote / \lpquote P \rpquote \id{\}}            = 
  \left\{ 
    \begin{array}{ccc}
      \lpquote Q \rpquote & & x \nameeq \lpquote P \rpquote \\
      x & & otherwise \\
    \end{array}
  \right. \nonumber
\end{eqnarray}

and $z$ is chosen distinct from $\quotep{P}$, $\quotep{Q}$, the free
names in $Q$, and all the names in $R$. Our $\alpha$-equivalence will
be built in the standard way from this substitution.

\begin{remark}\label{rem:no_self_referential_names}
  One consequence of these definitions is that $\forall P. \quotep{P}
  \not\in \freenames{P}$.
\end{remark}

\subsection{ Dynamic quote: an example }

Anticipating something of what's to come, consider applying the
substitution, $\widehat{\id{\{}u / z \id{\}}}$, to the following pair
of processes, $\lift{w}{y!(z)}$ and $w[ \lpquote y!(z) \rpquote ]$.

\begin{eqnarray}
	\lift{w}{y!(z)}\widehat{\id{\{}u / z \id{\}}}
		& = &
		\lift{w}{y!(u)} \nonumber\\
	w[ \lpquote y!(z) \rpquote ] \widehat{ \id{\{}u / z \id{\}} }
		& = &
		w[ \lpquote y!(z) \rpquote ] \nonumber
\end{eqnarray}

Because the body of the process between quotes is impervious to
substitution, we get radically different answers. In fact, by
examining the first process in an input context,
e.g. $x?(z).\lift{w}{y!(z)}$, we see that the process under the lift
operator may be shaped by prefixed inputs binding a name inside it. In
this sense, the lift operator will be seen as a way to dynamically
construct processes before reifying them as names.

Finally equipped with these standard features we can present the
dynamics of the calculus.

\subsubsection{Operational semantics} 

Finally, we introduce the computational dynamics. What marks these
algebras as distinct from other more traditionally studied algebraic
structures, e.g. vector spaces or polynomial rings, is the manner in
which dynamics is captured. In traditional structures, dynamics is typically
expressed through morphisms between such structures, as in linear maps
between vector spaces or morphisms between rings. In algebras
associated with the semantics of computation, the dynamics is
expressed as part of the algebraic structure itself, through a
reduction reduction relation typically denoted by $\red$. Below, we
give a recursive presentation of this relation for the calculus used
in the encoding.

$\red \subseteq \pi \times \pi$
$\red : \pi \to \mathcal{P}(\pi)$

\begin{mathpar}
  \inferrule* [lab=Comm] { \textsf{match}( x_{src}, x_{trgt} ) } { x_{trgt}?(y)P \; | \; x_{src}!\langle {Q} \rangle \red P\{\quotep{Q}/y}\} }
  \and \\
  \inferrule* [lab=Par] {{P} \red {P}'} {{{P} | {Q}} \red {{P}' | {Q}}}
  \and
  \inferrule* [lab=Equiv]{{{P} \scong {P}'} \andalso {{P}' \red {Q}'} \andalso {{Q}' \scong {Q}}}{{P} \red {Q}}
\end{mathpar}

\begin{eqnarray*}
  match_{\equiv} (\quotep{P},\quotep{Q}) & := & P \equiv Q \\
  match_{\dagger}(\quotep{P},\quotep{Q}) & := & \forall R. P|Q \red^{*} R => R \red^{*} 0 \\
  match_{K}(\quotep{P},\quotep{Q}) & := & K \mbox{ for some context } K
\end{eqnarray*}

$u?(x)P | u!\langle Q \rangle \red P\{\quotep{Q}/x\}$

%We write $\wred$ for $\red^*$, and $P\red$ if $\exists Q $ such that $ P \red Q$.
We write $P\red$ if $\exists Q $ such that $ P \red Q$ and $P\not\red$, otherwise.

\section{Replication}

As mentioned before, it is known that replication (and hence
recursion) can be implemented in a higher-order process algebra
\cite{SangiorgiWalker}. As our first example of calculation with the
machinery thus far presented we give the construction explicitly in
the {\rhoc}.

\begin{eqnarray}
	D_{x} & := & \prefix{x}{y}{(\binpar{\outputp{x}{y}}{@{y}})} \nonumber\\
	\bangp_{x}{P} & := & \binpar{{x}!\langle{\binpar{D_{x}}{P}}\rangle}{D_{x}} \nonumber
\end{eqnarray}

\begin{eqnarray}
	\bangp_{x}{P} & & \nonumber\\
	=
	& {x}!\langle{(\prefix{x}{y}{(\outputp{x}{y} | @{y})) | P}}\rangle 
	      | \prefix{x}{y}{(\outputp{x}{y} | @{y})} & \nonumber\\
	\red
	& (\outputp{x}{y} | @{y})\substn{\quotep{(\prefix{x}{y}{(@{y} | \outputp{x}{y})) | P}}}{y} & \nonumber\\
	=
	& \outputp{x}{\quotep{(\prefix{x}{y}{(\outputp{x}{y} | @{y})) | P}}}
	  | {(\prefix{x}{y}{(\outputp{x}{y} | @{y})) | P}} & \nonumber\\
	\red
	& \ldots & \nonumber\\
	\red^*
	& P | P | \ldots & \nonumber
\end{eqnarray}

Of course, this encoding, as an implementation, runs away, unfolding
$\bangp{P}$ eagerly. A lazier and more implementable replication
operator, restricted to input-guarded processes, may be obtained as follows.

\begin{eqnarray}
\bangp{\prefix{u}{v}{P}} 
	:= 
	\binpar{\lift{x}{\prefix{u}{v}{(\binpar{D(x)}{P})}}}{D(x)} \nonumber
\end{eqnarray}

\begin{remark}
  Note that the lazier definition still does not deal with summation
  or mixed summation (i.e. sums over input and output). The reader is
  invited to construct definitions of replication that deal with these
  features. 

  Further, the definitions are parameterized in a name, $x$. Can you,
  gentle reader, make a definition that eliminates this parameter and
  guarantees no accidental interaction between the replication
  machinery and the process being replicated -- i.e. no accidental
  sharing of names used by the process to get its work done and the
  name(s) used by the replication to effect copying. This latter
  revision of the definition of replication is crucial to obtaining
  the expected identity $!!P \sim !P$.
\end{remark}

\begin{remark}\label{rem:paradoxical_combinator}
  The reader familiar with the lambda calculus will have noticed the
  similarity between $D$ and the paradoxical combinator.

  [Ed. note: the existence of this seems to suggest we have to be more
  restrictive on the set of processes and names we admit if we are to
  support no-cloning.]
\end{remark}

\subsubsection{Bisimulation}

The computational dynamics gives rise to another kind of equivalence,
the equivalence of computational behavior. As previously mentioned
this is typically captured \emph{via} some form of bisimulation.

% The notion we use in this paper is weak barbed bisimulation
% \cite{milner91polyadicpi}.

The notion we use in this paper is derived from weak barbed
bisimulation \cite{milner91polyadicpi}. 

\begin{definition}
An \emph{observation relation}, $\downarrow_{\mathcal N}$, over a set
of names, $\mathcal N$, is the smallest relation satisfying the rules
below.

\infrule[Out-barb]{y \in {\mathcal N}, \; x \nameeq y}
		  {\outputp{x}{v} \downarrow_{\mathcal N} x}
\infrule[Par-barb]{\mbox{$P\downarrow_{\mathcal N} x$ or $Q\downarrow_{\mathcal N} x$}}
		  {\binpar{P}{Q} \downarrow_{\mathcal N} x}

We write $P \Downarrow_{\mathcal N} x$ if there is $Q$ such that 
$P \wred Q$ and $Q \downarrow_{\mathcal N} x$.
\end{definition}

\begin{definition}
%\label{def.bbisim}
An  ${\mathcal N}$-\emph{barbed bisimulation} over a set of names, ${\mathcal N}$, is a symmetric binary relation 
${\mathcal S}_{\mathcal N}$ between agents such that $P\rel{S}_{\mathcal N}Q$ implies:
\begin{enumerate}
\item If $P \red P'$ then $Q \wred Q'$ and $P'\rel{S}_{\mathcal N} Q'$.
\item If $P\downarrow_{\mathcal N} x$, then $Q\Downarrow_{\mathcal N} x$.
\end{enumerate}
$P$ is ${\mathcal N}$-barbed bisimilar to $Q$, written
$P \wbbisim_{\mathcal N} Q$, if $P \rel{S}_{\mathcal N} Q$ for some ${\mathcal N}$-barbed bisimulation ${\mathcal S}_{\mathcal N}$.
\end{definition}

$\mathcal{R} \subseteq \pi \times \pi$

$P \mathcal{R} Q => \forall P'. P \red P' \Rightarrow \exists Q'. Q \red Q', P' \mathcal{R} Q'$

$P \vdash x \Rightarrow Q \vdash x$

\begin{mathpar}
  \inferrule*[lab=Out-barb]{x \nameeq y}{{y}!\langle{Q}\rangle \vdash x}
  \and
  \inferrule*[lab=Par-barb]{\mbox{$P\vdash x$ or $Q\vdash x$}}{\binpar{P}{Q} \vdash x}
\end{mathpar}

\subsubsection{Contexts}

One of the principle advantages of computational calculi like the
$\pi$-calculus is a well-defined notion of context,
contextual-equivalence and a correlation between
contextual-equivalence and notions of bisimulation. The notion of
context allows the decomposition of a process into (sub-)process and
its syntactic environment, its context. Thus, a context may be
thought of as a process with a ``hole'' (written $\Box$) in it. The
application of a context $M$ to a process $P$, written $M[P]$, is
tantamount to filling the hole in $M$ with $P$. In this paper we do
not need the full weight of this theory, but do make use of the notion
of context in the proof the main theorem. 

\begin{mathpar}
  \inferrule* [lab=summation] {} {{M_{M},M_{N}} \bc \Box \;|\; x.M_{A} \;|\; M_{M}+M_{N}}
  \and
  \inferrule* [lab=agent] {} {{M_{A}} \bc (\vec{x})M_{P} \;| \; \clift{P_0,\ldots,M_{P},\ldots,P_N}}
  \and \\
  \inferrule* [lab=process] {} {{M_{P}} \bc M_{N} \;| \;P|M_{P} }
\end{mathpar} 

\begin{mathpar}
  \inferrule* [lab=sychronization] {} {M_{N} \bc \Box \;|\; x?M_{F} \;|\; x!M_{C}}
  \and
  \inferrule* [lab=abstraction] {} {{M_{F}} \bc (x)M_{P} }
  \and
  \inferrule* [lab=concretion] {} {{M_{C}} \bc \langle M_{P} \rangle }
  \and \\
  \inferrule* [lab=process] {} {{M_{P}} \bc M_{N} \;| \;P|M_{P} }
\end{mathpar}

\begin{definition}[contextual application] Given a context $M$, and
  process $P$, we define the \emph{contextual application}, $M[P] :=
  M\{P/\Box\}$. That is, the contextual application of M to P is the
  substitution of $P$ for $\Box$ in $M$.
\end{definition}

$\meaningof{-} : L \to \mathcal{P}(\pi)$

\begin{mathpar}
  \inferrule* [lab=collection] {} {\meaningof{true} = \pi, \and \meaningof{~E} = \pi \setminus \meaningof{E}, \and \meaningof{E_{1} \& E_{2}} = \meaningof{E_{1}} \cap \meaningof{E_{2}}}
\end{mathpar}

\begin{mathpar}
  \inferrule* [lab=structure] {} {\meaningof{0} = \{ P \in \pi | P \equiv 0 \}, \and \\ \meaningof{E_1 | E_2} = \{ P \in \pi | P \equiv P_{1} | P_{2}, P_{1} \in \meaningof{E_{1}}, P_{2} \in \meaningof{E_2}\} }
\end{mathpar}

\begin{mathpar}
 \inferrule* [lab=behavior] {} {\meaningof{\langle a?b \rangle E} = \{ P \in \pi | P \equiv Q | u?(y)P', \\ \and \\\\ \and \\ \;\;\; u \in \meaningof{a}, \forall z.P'\{z/y\} \in \meaningof{E\{z/b\}}\}, \and \\ \meaningof{a!E} = \{ P \in \pi | P \equiv Q | x!\langle P' \rangle, x \in \meaningof{a} P' \in \meaningof{E}\} }
\end{mathpar}

\begin{mathpar}
 \inferrule* [lab=nominal] {} {\meaningof{\quotep{E}} = \{ \quotep{P} \in \quotep{\pi} | P \in \meaningof{E} \}, \and \meaningof{\quotep{P}} = \{ \quotep{Q} \in \quotep{\pi} | P \equiv Q \} \and \\ \meaningof{@\quotep{E}} = \{ P \in \pi | P \equiv @x, x \in \meaningof{E} \}}
\end{mathpar}

\begin{eqnarray*}
  \\
  \meaningof{-} : TS \to ST
\end{eqnarray*}

\begin{eqnarray*}
  \\
  L : TS \to ST
\end{eqnarray*}

\begin{eqnarray*}
  \\
  P \models E \iff P \in \meaningof{E}
\end{eqnarray*}

\begin{eqnarray*}
  P \approx_{L} Q \iff \forall E \in L. P \models E \iff Q \models E
\end{eqnarray*}

\begin{eqnarray*}
  P \approx_{K} Q
\end{eqnarray*}

\begin{eqnarray*}
  P \approx Q
\end{eqnarray*}

$\approx_{K} = \approx = \approx_{L}$

\subsubsection{Contextual duality}

Note that contexts extend the quotation operation to a family of
operations from processes to names. Given a context, $M$, we can
define a \emph{nominal context}, $\quotep{M}$ by $\quotep{M}[P] :=
\quotep{M[P]}$. To foreshadow what is to come we observe that these
operations enjoy a duality with processes very much like the duality
between vectors and maps from vectors to scalars.

Further, because the calculus is essentially higher-order, we have a
correspondence between contexts and processes. More specifically,
given a name $x$ and a context $M$ we can construct $M^{*}_{x}$ such
that 

\begin{mathpar}
  M^{*}_{x} | \lift{x}{P} \red M[P]
\end{mathpar}

namely,

\begin{mathpar}
  M^{*}_{x} := x?(u).M[\dropn{u}]
\end{mathpar}

The dependence of $M^{*}_{x}$ on a name makes it an abstraction, 

\begin{mathpar}
  M^{*} := (x)x?(u).M[\dropn{u}]
\end{mathpar}

\subsection{Additional notation}

It will sometimes be convenient to denote the process a name
quotes. We already have the notation $x = \quotep{P}$, but it will be
convenient to introduce an alternate notation, $\procn{x}$, when we
want to emphasize the connection to the use of the name. Note that, by
virtue of name equivalence, $\quotep{\procn{x}} \nameeq x$; so, the
notation is consistent with previous definitions.

Further, because names have structure it is possible to effect
substitutions on the basis of that structure. This means we need to
upgrade our notation for substitutions, which we accomplish by
adapting comprehension notation. Thus,

\begin{mathpar}
  P\{ y / x : x \in S \}
\end{mathpar}

is interpreted to mean the process derived from P by replacing (in a
capture-avoiding manner) each occurrence of $x$ in $S$ by $y$. For example,

\begin{mathpar}
  P\{ \quotep{\procn{x}|\procn{x}} / x : x \in \freenames{P} \}
\end{mathpar}

will replace each (occurrence) of a free name $x$ in $P$ by
$\quotep{\procn{x}|\procn{x}}$.

Also, we will avail ourselves of the notation $x^{L}$ and $x^{R}$ to
denote injections of a name into disjoint copies of the name
space. There are numerous ways to accomplish this. One example can be
found in \cite{MeredithR05}. This notation overloads to vectors of
names: $\vec{x}^{\pi} := (x_{i}^{\pi} \; : \; 0 \leq i < |\vec{x}| )$ where $\pi \in \{L,R\}$.

We also use $P^{\Box} := P|\Box$.

In \cite{MeredithR05} an interpretation of the new operator is
given. It turns out that there are several possible interpretations
all enjoying the requisite algebraic properties of the operator (see
\cite{milner91polyadicpi}). We will therefore make liberal use of
$(\nu\; \vec{x})P$.

% subsection the_syntax_and_semantics_of_the_notation_system (end)   

\input{qm2pi.qmops} 

\input{qm2pi.sterngerlach} 

\input{qm2pi.metric} 

% section concurrent_process_calculi (end)

%\input{qm2pi.proofsketch}

% section proof sketch (end)

%\input{qm2pi.slviaknots} 

% section spatial logic via knots (end)

\input{qm2pi.conclusion}

% section conclusion (end)

%\input{qm2pi.dtcodes} 

% section wiring algorithm (end)

\input{qm2pi.ack} 

% section acknowledgments (end)

\newpage


\bibliographystyle{plain}   
\bibliography{../../biblios/main.bib}

\input{qm2pi.rhodetails}

\end{document}

 

% section notation (end)

\input{qm2pi.process.calculi} 

% section concurrent_process_calculi_and_spatial_logics_ (end)
    
%\documentclass[12pt]{llncs}
%\documentclass{jktr}

\usepackage[pdftex]{hyperref}                   
\usepackage {listings}
\usepackage {mathpartir}
\usepackage{bcprules}
%\usepackage{listings}
                       
\usepackage{graphicx} 
%\usepackage[margins=2.5cm,nohead,nofoot]{geometry}
%\usepackage{geometry}
\usepackage{amsfonts}
\usepackage{amstext}
\usepackage{latexsym}
\usepackage{amssymb}
\usepackage{color}


%\include{myPreamble}
\include{qm2pi.local} 

%\ifpdf
%\usepackage[pdftex]{graphicx}
%\else
%\usepackage{graphicx}
%\fi

 % \ifpdf
%  \usepackage{pdfsync}
%  \if


%\title{Brief Article}
%\author{David F. Snyder}
%\author{L.G. Meredith}

%\address{Dept. of Math., Texas State University--San Marcos, San Marcos, TX 78666}
       
\pagestyle{empty}


\begin{document}

\lstset{language=[Objective]Caml,frame=shadowbox}

\input{qm2pi.front}

% section front matter (end)

\input{qm2pi.intro} 
 
% section introduction (end)

% \input{qm2pi.knotations} 

% section notation (end)

\input{qm2pi.process.calculi} 

% section concurrent_process_calculi_and_spatial_logics_ (end)
    
%\input{qm2pi.knots2pi} 

%\input{qm2pi.trefoil} 

%\input{qm2pi.mainthm} 

% subsection basic_interpretation (end)

%\input{qm2pi.rho.presentation} 
\subsection{The syntax and semantics of the notation system}\label{sub:the_syntax_and_semantics_of_the_notation_system} % (fold)

We now summarize a technical presentation of the calculus that
embodies our theory of dynamics. The typical presentation of such a
calculus follows the style of giving generators and relations on
them. The grammar, below, describing term constructors, freely
generates the set of processes, $\Proc$. This set is then quotiented
by a relation known as structural congruence and it is over this set
that the notion of dynamics is expressed. This presentation is
essentially that of \cite{MeredithR05} with the addition of
polyadicity and summation. For readability we have relegated some of
the technical subtleties to an appendix.

\subsubsection{Process grammar}\label{subsub:process_grammar}

\begin{mathpar}
  \inferrule* [lab=synchronization] {} {{M} \bc \pzero \;|\; x?F \;|\; x!C }
  \and
  \inferrule* [lab=abstraction] {} {{F} \bc (x)P}
  \and
  \inferrule* [lab=concretion] {} {{C} \bc \langle Q \rangle}
  \and
  \inferrule* [lab=process] {} {{P,Q} \bc M \;| \;P|Q \;|\; @{x}}
  \and
  \inferrule* [lab=name] {} {{x} \bc \quotep{P}}
\end{mathpar} 

Note that $\vec{x}$ (resp. $\vec{P}$) denotes a vector of names
(resp. processes) of length $|\vec{x}|$ (resp. $|\vec{P}|$). We adopt
the following useful abbreviations.

\begin{mathpar}
   x?(\vec{y}).P := x.(\vec{y})P \and  x\clift{\vec{P}} := x.\clift{\vec{P}}
   \and x!(y) := \lift{x}{\dropn{y}}
   \and \Pi_{i=0}^{n-1}P_i := P_0 | \ldots | P_{n-1}
\end{mathpar}

\subsubsection{Structural congruence}

\paragraph{Free and bound names and alpha-equivalence.} At the
core of structural equivalence is alpha-equivalence which identifies
process that are the same up to a change of variable. Formally, we
recognize the distinction between free and bound names. The free names
of a process, $\freenames{P}$, may be calculated recursively as
follows:

\begin{mathpar}
\freenames{\pzero} := \emptyset
  \and \\
  \freenames{x?(y).P} := \{ x \} \cup (\freenames{P} \setminus \{ y \})
  \and 
  \freenames{x!\langle P \rangle} := \{ x \} \cup \{ P \} 
  \and \\
  \freenames{P|Q} := \freenames{P} \cup \freenames{Q}
  \and \\
  \freenames{@{x}} := \{ x \}
\end{mathpar}

$\pi$
$\quotep{\pi}$

$\freenames{-} : \pi \to \mathcal{P}(\quotep{\pi})$

\begin{eqnarray*}
  \freenames{\pzero} & := & \emptyset \\
  \freenames{x?(y).P} & := & \{ x \} \cup (\freenames{P} \setminus \{ y \}) \\
  \freenames{x!\langle P \rangle} & := & \{ x \} \cup \{ P \} \\
  \freenames{P|Q} & := & \freenames{P} \cup \freenames{Q} \\
  \freenames{\dropn{x}} & := & \{ x \}
\end{eqnarray*}

The bound names of a process, $\boundnames{P}$, are those names occurring in $P$
that are not free. For example, in $x?(y).0$, the name $x$ is free, while $y$ is bound.

\begin{mathpar}
  \inferrule* [lab=monoidal-laws] {} { P|Q \equiv Q|P \and P|0 \equiv P \and P|(Q|R) \equiv (P|Q)|R }
\end{mathpar}

\begin{mathpar}
  \inferrule* [lab=alpha-equivalence] {} { (x)P \equiv (y)P\{y/x\} \and y \not\in \freenames{P} }
\end{mathpar}

\begin{definition}
Then two processes, $P,Q$, are alpha-equivalent if $P = Q\{\vec{y}/\vec{x}\}$ for
some $\vec{x} \in \boundnames{Q},\vec{y} \in \boundnames{P}$, where $Q\{\vec{y}/\vec{x}\}$
denotes the capture-avoiding substitution of $\vec{y}$ for $\vec{x}$ in $Q$.
\end{definition}

\begin{definition}
  The {\em structural congruence} \cite{SangiorgiWalker} , $\equiv$,
  between processes is the least congruence containing
  alpha-equivalence, satisfying the abelian monoid laws
  (associativity, commutativity and $\pzero$ as identity) for parallel
  composition $|$ and for summation $+$.
\end{definition}

\subsection{Name equivalence}

We take name equivalence, written $\nameeq$, to be the smallest
equivalence relation generated by the following rules.

\begin{mathpar}
\inferrule*[lab=Quote-drop]
{ }
{ \quotep{@{x}} \nameeq x }

\inferrule*[lab=Struct-equiv]
{ P \scong Q }
{ \quotep{P} \nameeq \quotep{Q} }
\end{mathpar}

The astute reader will have noticed that the mutual recursion of names
and processes imposes a mutual recursion on alpha-equivalence and
structural equivalence via name-equivalence. Fortunately, all of this
works out pleasantly and we may calculate in the natural way, free of
concern. The reader interested in the details is referred to the
appendix \ref{appendix:rho_details}.

\subsection{Substitution}

We use $\Proc$ for the set of processes, $\QProc$ for the set of
names, and $\id{\{}\vec{y} / \vec{x} \id{\}}$ to denote partial maps,
$s : \QProc \rightarrow \QProc$. A map, $s$ lifts, uniquely, to a map
on process terms, $\widehat{s} : \Proc \rightarrow \Proc$ by the
following equations.

\begin{mathpar}
  (0) \psubstp{Q}{P} := 0 \\
  (R \juxtap S) \psubstp{Q}{P}
  :=    
  (R)\psubstp{Q}{P} \juxtap (S) \psubstp{Q}{P} \\
  (x?(y).R) \psubstp{Q}{P}    
  :=    
  (x)\substp{Q}{P} (z)\concat( (R \psubstn{z}{y}) \psubstp{Q}{P} ) \\
  (\lift{x}{R}) \psubstp{Q}{P}  
  :=
  \lift{(x)\substp{Q}{P}}{ R \psubstp{Q}{P} } \\
%   (\dropn{x})  \psubstp{Q}{P}       
%   := 
%   \left\{ 
%     \begin{array}{ccc} 
%       \dropn{\quotep{Q}} & & x \nameeq \quotep{P} \\
%       \dropn{x} & & otherwise \\
%     \end{array}
%   \right. 
  (\dropn{x})  \psubstp{Q}{P}       
  := 
  \left\{ 
    \begin{array}{ccc} 
      Q & & x \nameeq \quotep{P} \\
      \dropn{x} & & otherwise \\
    \end{array}
  \right.
\end{mathpar}
 

where

\begin{eqnarray}
  (x)\id{\{} \lpquote Q \rpquote / \lpquote P \rpquote \id{\}}            = 
  \left\{ 
    \begin{array}{ccc}
      \lpquote Q \rpquote & & x \nameeq \lpquote P \rpquote \\
      x & & otherwise \\
    \end{array}
  \right. \nonumber
\end{eqnarray}

and $z$ is chosen distinct from $\quotep{P}$, $\quotep{Q}$, the free
names in $Q$, and all the names in $R$. Our $\alpha$-equivalence will
be built in the standard way from this substitution.

\begin{remark}\label{rem:no_self_referential_names}
  One consequence of these definitions is that $\forall P. \quotep{P}
  \not\in \freenames{P}$.
\end{remark}

\subsection{ Dynamic quote: an example }

Anticipating something of what's to come, consider applying the
substitution, $\widehat{\id{\{}u / z \id{\}}}$, to the following pair
of processes, $\lift{w}{y!(z)}$ and $w[ \lpquote y!(z) \rpquote ]$.

\begin{eqnarray}
	\lift{w}{y!(z)}\widehat{\id{\{}u / z \id{\}}}
		& = &
		\lift{w}{y!(u)} \nonumber\\
	w[ \lpquote y!(z) \rpquote ] \widehat{ \id{\{}u / z \id{\}} }
		& = &
		w[ \lpquote y!(z) \rpquote ] \nonumber
\end{eqnarray}

Because the body of the process between quotes is impervious to
substitution, we get radically different answers. In fact, by
examining the first process in an input context,
e.g. $x?(z).\lift{w}{y!(z)}$, we see that the process under the lift
operator may be shaped by prefixed inputs binding a name inside it. In
this sense, the lift operator will be seen as a way to dynamically
construct processes before reifying them as names.

Finally equipped with these standard features we can present the
dynamics of the calculus.

\subsubsection{Operational semantics} 

Finally, we introduce the computational dynamics. What marks these
algebras as distinct from other more traditionally studied algebraic
structures, e.g. vector spaces or polynomial rings, is the manner in
which dynamics is captured. In traditional structures, dynamics is typically
expressed through morphisms between such structures, as in linear maps
between vector spaces or morphisms between rings. In algebras
associated with the semantics of computation, the dynamics is
expressed as part of the algebraic structure itself, through a
reduction reduction relation typically denoted by $\red$. Below, we
give a recursive presentation of this relation for the calculus used
in the encoding.

$\red \subseteq \pi \times \pi$
$\red : \pi \to \mathcal{P}(\pi)$

\begin{mathpar}
  \inferrule* [lab=Comm] { \textsf{match}( x_{src}, x_{trgt} ) } { x_{trgt}?(y)P \; | \; x_{src}!\langle {Q} \rangle \red P\{\quotep{Q}/y}\} }
  \and \\
  \inferrule* [lab=Par] {{P} \red {P}'} {{{P} | {Q}} \red {{P}' | {Q}}}
  \and
  \inferrule* [lab=Equiv]{{{P} \scong {P}'} \andalso {{P}' \red {Q}'} \andalso {{Q}' \scong {Q}}}{{P} \red {Q}}
\end{mathpar}

\begin{eqnarray*}
  match_{\equiv} (\quotep{P},\quotep{Q}) & := & P \equiv Q \\
  match_{\dagger}(\quotep{P},\quotep{Q}) & := & \forall R. P|Q \red^{*} R => R \red^{*} 0 \\
  match_{K}(\quotep{P},\quotep{Q}) & := & K \mbox{ for some context } K
\end{eqnarray*}

$u?(x)P | u!\langle Q \rangle \red P\{\quotep{Q}/x\}$

%We write $\wred$ for $\red^*$, and $P\red$ if $\exists Q $ such that $ P \red Q$.
We write $P\red$ if $\exists Q $ such that $ P \red Q$ and $P\not\red$, otherwise.

\section{Replication}

As mentioned before, it is known that replication (and hence
recursion) can be implemented in a higher-order process algebra
\cite{SangiorgiWalker}. As our first example of calculation with the
machinery thus far presented we give the construction explicitly in
the {\rhoc}.

\begin{eqnarray}
	D_{x} & := & \prefix{x}{y}{(\binpar{\outputp{x}{y}}{@{y}})} \nonumber\\
	\bangp_{x}{P} & := & \binpar{{x}!\langle{\binpar{D_{x}}{P}}\rangle}{D_{x}} \nonumber
\end{eqnarray}

\begin{eqnarray}
	\bangp_{x}{P} & & \nonumber\\
	=
	& {x}!\langle{(\prefix{x}{y}{(\outputp{x}{y} | @{y})) | P}}\rangle 
	      | \prefix{x}{y}{(\outputp{x}{y} | @{y})} & \nonumber\\
	\red
	& (\outputp{x}{y} | @{y})\substn{\quotep{(\prefix{x}{y}{(@{y} | \outputp{x}{y})) | P}}}{y} & \nonumber\\
	=
	& \outputp{x}{\quotep{(\prefix{x}{y}{(\outputp{x}{y} | @{y})) | P}}}
	  | {(\prefix{x}{y}{(\outputp{x}{y} | @{y})) | P}} & \nonumber\\
	\red
	& \ldots & \nonumber\\
	\red^*
	& P | P | \ldots & \nonumber
\end{eqnarray}

Of course, this encoding, as an implementation, runs away, unfolding
$\bangp{P}$ eagerly. A lazier and more implementable replication
operator, restricted to input-guarded processes, may be obtained as follows.

\begin{eqnarray}
\bangp{\prefix{u}{v}{P}} 
	:= 
	\binpar{\lift{x}{\prefix{u}{v}{(\binpar{D(x)}{P})}}}{D(x)} \nonumber
\end{eqnarray}

\begin{remark}
  Note that the lazier definition still does not deal with summation
  or mixed summation (i.e. sums over input and output). The reader is
  invited to construct definitions of replication that deal with these
  features. 

  Further, the definitions are parameterized in a name, $x$. Can you,
  gentle reader, make a definition that eliminates this parameter and
  guarantees no accidental interaction between the replication
  machinery and the process being replicated -- i.e. no accidental
  sharing of names used by the process to get its work done and the
  name(s) used by the replication to effect copying. This latter
  revision of the definition of replication is crucial to obtaining
  the expected identity $!!P \sim !P$.
\end{remark}

\begin{remark}\label{rem:paradoxical_combinator}
  The reader familiar with the lambda calculus will have noticed the
  similarity between $D$ and the paradoxical combinator.

  [Ed. note: the existence of this seems to suggest we have to be more
  restrictive on the set of processes and names we admit if we are to
  support no-cloning.]
\end{remark}

\subsubsection{Bisimulation}

The computational dynamics gives rise to another kind of equivalence,
the equivalence of computational behavior. As previously mentioned
this is typically captured \emph{via} some form of bisimulation.

% The notion we use in this paper is weak barbed bisimulation
% \cite{milner91polyadicpi}.

The notion we use in this paper is derived from weak barbed
bisimulation \cite{milner91polyadicpi}. 

\begin{definition}
An \emph{observation relation}, $\downarrow_{\mathcal N}$, over a set
of names, $\mathcal N$, is the smallest relation satisfying the rules
below.

\infrule[Out-barb]{y \in {\mathcal N}, \; x \nameeq y}
		  {\outputp{x}{v} \downarrow_{\mathcal N} x}
\infrule[Par-barb]{\mbox{$P\downarrow_{\mathcal N} x$ or $Q\downarrow_{\mathcal N} x$}}
		  {\binpar{P}{Q} \downarrow_{\mathcal N} x}

We write $P \Downarrow_{\mathcal N} x$ if there is $Q$ such that 
$P \wred Q$ and $Q \downarrow_{\mathcal N} x$.
\end{definition}

\begin{definition}
%\label{def.bbisim}
An  ${\mathcal N}$-\emph{barbed bisimulation} over a set of names, ${\mathcal N}$, is a symmetric binary relation 
${\mathcal S}_{\mathcal N}$ between agents such that $P\rel{S}_{\mathcal N}Q$ implies:
\begin{enumerate}
\item If $P \red P'$ then $Q \wred Q'$ and $P'\rel{S}_{\mathcal N} Q'$.
\item If $P\downarrow_{\mathcal N} x$, then $Q\Downarrow_{\mathcal N} x$.
\end{enumerate}
$P$ is ${\mathcal N}$-barbed bisimilar to $Q$, written
$P \wbbisim_{\mathcal N} Q$, if $P \rel{S}_{\mathcal N} Q$ for some ${\mathcal N}$-barbed bisimulation ${\mathcal S}_{\mathcal N}$.
\end{definition}

$\mathcal{R} \subseteq \pi \times \pi$

$P \mathcal{R} Q => \forall P'. P \red P' \Rightarrow \exists Q'. Q \red Q', P' \mathcal{R} Q'$

$P \vdash x \Rightarrow Q \vdash x$

\begin{mathpar}
  \inferrule*[lab=Out-barb]{x \nameeq y}{{y}!\langle{Q}\rangle \vdash x}
  \and
  \inferrule*[lab=Par-barb]{\mbox{$P\vdash x$ or $Q\vdash x$}}{\binpar{P}{Q} \vdash x}
\end{mathpar}

\subsubsection{Contexts}

One of the principle advantages of computational calculi like the
$\pi$-calculus is a well-defined notion of context,
contextual-equivalence and a correlation between
contextual-equivalence and notions of bisimulation. The notion of
context allows the decomposition of a process into (sub-)process and
its syntactic environment, its context. Thus, a context may be
thought of as a process with a ``hole'' (written $\Box$) in it. The
application of a context $M$ to a process $P$, written $M[P]$, is
tantamount to filling the hole in $M$ with $P$. In this paper we do
not need the full weight of this theory, but do make use of the notion
of context in the proof the main theorem. 

\begin{mathpar}
  \inferrule* [lab=summation] {} {{M_{M},M_{N}} \bc \Box \;|\; x.M_{A} \;|\; M_{M}+M_{N}}
  \and
  \inferrule* [lab=agent] {} {{M_{A}} \bc (\vec{x})M_{P} \;| \; \clift{P_0,\ldots,M_{P},\ldots,P_N}}
  \and \\
  \inferrule* [lab=process] {} {{M_{P}} \bc M_{N} \;| \;P|M_{P} }
\end{mathpar} 

\begin{mathpar}
  \inferrule* [lab=sychronization] {} {M_{N} \bc \Box \;|\; x?M_{F} \;|\; x!M_{C}}
  \and
  \inferrule* [lab=abstraction] {} {{M_{F}} \bc (x)M_{P} }
  \and
  \inferrule* [lab=concretion] {} {{M_{C}} \bc \langle M_{P} \rangle }
  \and \\
  \inferrule* [lab=process] {} {{M_{P}} \bc M_{N} \;| \;P|M_{P} }
\end{mathpar}

\begin{definition}[contextual application] Given a context $M$, and
  process $P$, we define the \emph{contextual application}, $M[P] :=
  M\{P/\Box\}$. That is, the contextual application of M to P is the
  substitution of $P$ for $\Box$ in $M$.
\end{definition}

$\meaningof{-} : L \to \mathcal{P}(\pi)$

\begin{mathpar}
  \inferrule* [lab=collection] {} {\meaningof{true} = \pi, \and \meaningof{~E} = \pi \setminus \meaningof{E}, \and \meaningof{E_{1} \& E_{2}} = \meaningof{E_{1}} \cap \meaningof{E_{2}}}
\end{mathpar}

\begin{mathpar}
  \inferrule* [lab=structure] {} {\meaningof{0} = \{ P \in \pi | P \equiv 0 \}, \and \\ \meaningof{E_1 | E_2} = \{ P \in \pi | P \equiv P_{1} | P_{2}, P_{1} \in \meaningof{E_{1}}, P_{2} \in \meaningof{E_2}\} }
\end{mathpar}

\begin{mathpar}
 \inferrule* [lab=behavior] {} {\meaningof{\langle a?b \rangle E} = \{ P \in \pi | P \equiv Q | u?(y)P', \\ \and \\\\ \and \\ \;\;\; u \in \meaningof{a}, \forall z.P'\{z/y\} \in \meaningof{E\{z/b\}}\}, \and \\ \meaningof{a!E} = \{ P \in \pi | P \equiv Q | x!\langle P' \rangle, x \in \meaningof{a} P' \in \meaningof{E}\} }
\end{mathpar}

\begin{mathpar}
 \inferrule* [lab=nominal] {} {\meaningof{\quotep{E}} = \{ \quotep{P} \in \quotep{\pi} | P \in \meaningof{E} \}, \and \meaningof{\quotep{P}} = \{ \quotep{Q} \in \quotep{\pi} | P \equiv Q \} \and \\ \meaningof{@\quotep{E}} = \{ P \in \pi | P \equiv @x, x \in \meaningof{E} \}}
\end{mathpar}

\begin{eqnarray*}
  \\
  \meaningof{-} : TS \to ST
\end{eqnarray*}

\begin{eqnarray*}
  \\
  L : TS \to ST
\end{eqnarray*}

\begin{eqnarray*}
  \\
  P \models E \iff P \in \meaningof{E}
\end{eqnarray*}

\begin{eqnarray*}
  P \approx_{L} Q \iff \forall E \in L. P \models E \iff Q \models E
\end{eqnarray*}

\begin{eqnarray*}
  P \approx_{K} Q
\end{eqnarray*}

\begin{eqnarray*}
  P \approx Q
\end{eqnarray*}

$\approx_{K} = \approx = \approx_{L}$

\subsubsection{Contextual duality}

Note that contexts extend the quotation operation to a family of
operations from processes to names. Given a context, $M$, we can
define a \emph{nominal context}, $\quotep{M}$ by $\quotep{M}[P] :=
\quotep{M[P]}$. To foreshadow what is to come we observe that these
operations enjoy a duality with processes very much like the duality
between vectors and maps from vectors to scalars.

Further, because the calculus is essentially higher-order, we have a
correspondence between contexts and processes. More specifically,
given a name $x$ and a context $M$ we can construct $M^{*}_{x}$ such
that 

\begin{mathpar}
  M^{*}_{x} | \lift{x}{P} \red M[P]
\end{mathpar}

namely,

\begin{mathpar}
  M^{*}_{x} := x?(u).M[\dropn{u}]
\end{mathpar}

The dependence of $M^{*}_{x}$ on a name makes it an abstraction, 

\begin{mathpar}
  M^{*} := (x)x?(u).M[\dropn{u}]
\end{mathpar}

\subsection{Additional notation}

It will sometimes be convenient to denote the process a name
quotes. We already have the notation $x = \quotep{P}$, but it will be
convenient to introduce an alternate notation, $\procn{x}$, when we
want to emphasize the connection to the use of the name. Note that, by
virtue of name equivalence, $\quotep{\procn{x}} \nameeq x$; so, the
notation is consistent with previous definitions.

Further, because names have structure it is possible to effect
substitutions on the basis of that structure. This means we need to
upgrade our notation for substitutions, which we accomplish by
adapting comprehension notation. Thus,

\begin{mathpar}
  P\{ y / x : x \in S \}
\end{mathpar}

is interpreted to mean the process derived from P by replacing (in a
capture-avoiding manner) each occurrence of $x$ in $S$ by $y$. For example,

\begin{mathpar}
  P\{ \quotep{\procn{x}|\procn{x}} / x : x \in \freenames{P} \}
\end{mathpar}

will replace each (occurrence) of a free name $x$ in $P$ by
$\quotep{\procn{x}|\procn{x}}$.

Also, we will avail ourselves of the notation $x^{L}$ and $x^{R}$ to
denote injections of a name into disjoint copies of the name
space. There are numerous ways to accomplish this. One example can be
found in \cite{MeredithR05}. This notation overloads to vectors of
names: $\vec{x}^{\pi} := (x_{i}^{\pi} \; : \; 0 \leq i < |\vec{x}| )$ where $\pi \in \{L,R\}$.

We also use $P^{\Box} := P|\Box$.

In \cite{MeredithR05} an interpretation of the new operator is
given. It turns out that there are several possible interpretations
all enjoying the requisite algebraic properties of the operator (see
\cite{milner91polyadicpi}). We will therefore make liberal use of
$(\nu\; \vec{x})P$.

% subsection the_syntax_and_semantics_of_the_notation_system (end)   

\input{qm2pi.qmops} 

\input{qm2pi.sterngerlach} 

\input{qm2pi.metric} 

% section concurrent_process_calculi (end)

%\input{qm2pi.proofsketch}

% section proof sketch (end)

%\input{qm2pi.slviaknots} 

% section spatial logic via knots (end)

\input{qm2pi.conclusion}

% section conclusion (end)

%\input{qm2pi.dtcodes} 

% section wiring algorithm (end)

\input{qm2pi.ack} 

% section acknowledgments (end)

\newpage


\bibliographystyle{plain}   
\bibliography{../../biblios/main.bib}

\input{qm2pi.rhodetails}

\end{document}

 

%\documentclass[12pt]{llncs}
%\documentclass{jktr}

\usepackage[pdftex]{hyperref}                   
\usepackage {listings}
\usepackage {mathpartir}
\usepackage{bcprules}
%\usepackage{listings}
                       
\usepackage{graphicx} 
%\usepackage[margins=2.5cm,nohead,nofoot]{geometry}
%\usepackage{geometry}
\usepackage{amsfonts}
\usepackage{amstext}
\usepackage{latexsym}
\usepackage{amssymb}
\usepackage{color}


%\include{myPreamble}
\include{qm2pi.local} 

%\ifpdf
%\usepackage[pdftex]{graphicx}
%\else
%\usepackage{graphicx}
%\fi

 % \ifpdf
%  \usepackage{pdfsync}
%  \if


%\title{Brief Article}
%\author{David F. Snyder}
%\author{L.G. Meredith}

%\address{Dept. of Math., Texas State University--San Marcos, San Marcos, TX 78666}
       
\pagestyle{empty}


\begin{document}

\lstset{language=[Objective]Caml,frame=shadowbox}

\input{qm2pi.front}

% section front matter (end)

\input{qm2pi.intro} 
 
% section introduction (end)

% \input{qm2pi.knotations} 

% section notation (end)

\input{qm2pi.process.calculi} 

% section concurrent_process_calculi_and_spatial_logics_ (end)
    
%\input{qm2pi.knots2pi} 

%\input{qm2pi.trefoil} 

%\input{qm2pi.mainthm} 

% subsection basic_interpretation (end)

%\input{qm2pi.rho.presentation} 
\subsection{The syntax and semantics of the notation system}\label{sub:the_syntax_and_semantics_of_the_notation_system} % (fold)

We now summarize a technical presentation of the calculus that
embodies our theory of dynamics. The typical presentation of such a
calculus follows the style of giving generators and relations on
them. The grammar, below, describing term constructors, freely
generates the set of processes, $\Proc$. This set is then quotiented
by a relation known as structural congruence and it is over this set
that the notion of dynamics is expressed. This presentation is
essentially that of \cite{MeredithR05} with the addition of
polyadicity and summation. For readability we have relegated some of
the technical subtleties to an appendix.

\subsubsection{Process grammar}\label{subsub:process_grammar}

\begin{mathpar}
  \inferrule* [lab=synchronization] {} {{M} \bc \pzero \;|\; x?F \;|\; x!C }
  \and
  \inferrule* [lab=abstraction] {} {{F} \bc (x)P}
  \and
  \inferrule* [lab=concretion] {} {{C} \bc \langle Q \rangle}
  \and
  \inferrule* [lab=process] {} {{P,Q} \bc M \;| \;P|Q \;|\; @{x}}
  \and
  \inferrule* [lab=name] {} {{x} \bc \quotep{P}}
\end{mathpar} 

Note that $\vec{x}$ (resp. $\vec{P}$) denotes a vector of names
(resp. processes) of length $|\vec{x}|$ (resp. $|\vec{P}|$). We adopt
the following useful abbreviations.

\begin{mathpar}
   x?(\vec{y}).P := x.(\vec{y})P \and  x\clift{\vec{P}} := x.\clift{\vec{P}}
   \and x!(y) := \lift{x}{\dropn{y}}
   \and \Pi_{i=0}^{n-1}P_i := P_0 | \ldots | P_{n-1}
\end{mathpar}

\subsubsection{Structural congruence}

\paragraph{Free and bound names and alpha-equivalence.} At the
core of structural equivalence is alpha-equivalence which identifies
process that are the same up to a change of variable. Formally, we
recognize the distinction between free and bound names. The free names
of a process, $\freenames{P}$, may be calculated recursively as
follows:

\begin{mathpar}
\freenames{\pzero} := \emptyset
  \and \\
  \freenames{x?(y).P} := \{ x \} \cup (\freenames{P} \setminus \{ y \})
  \and 
  \freenames{x!\langle P \rangle} := \{ x \} \cup \{ P \} 
  \and \\
  \freenames{P|Q} := \freenames{P} \cup \freenames{Q}
  \and \\
  \freenames{@{x}} := \{ x \}
\end{mathpar}

$\pi$
$\quotep{\pi}$

$\freenames{-} : \pi \to \mathcal{P}(\quotep{\pi})$

\begin{eqnarray*}
  \freenames{\pzero} & := & \emptyset \\
  \freenames{x?(y).P} & := & \{ x \} \cup (\freenames{P} \setminus \{ y \}) \\
  \freenames{x!\langle P \rangle} & := & \{ x \} \cup \{ P \} \\
  \freenames{P|Q} & := & \freenames{P} \cup \freenames{Q} \\
  \freenames{\dropn{x}} & := & \{ x \}
\end{eqnarray*}

The bound names of a process, $\boundnames{P}$, are those names occurring in $P$
that are not free. For example, in $x?(y).0$, the name $x$ is free, while $y$ is bound.

\begin{mathpar}
  \inferrule* [lab=monoidal-laws] {} { P|Q \equiv Q|P \and P|0 \equiv P \and P|(Q|R) \equiv (P|Q)|R }
\end{mathpar}

\begin{mathpar}
  \inferrule* [lab=alpha-equivalence] {} { (x)P \equiv (y)P\{y/x\} \and y \not\in \freenames{P} }
\end{mathpar}

\begin{definition}
Then two processes, $P,Q$, are alpha-equivalent if $P = Q\{\vec{y}/\vec{x}\}$ for
some $\vec{x} \in \boundnames{Q},\vec{y} \in \boundnames{P}$, where $Q\{\vec{y}/\vec{x}\}$
denotes the capture-avoiding substitution of $\vec{y}$ for $\vec{x}$ in $Q$.
\end{definition}

\begin{definition}
  The {\em structural congruence} \cite{SangiorgiWalker} , $\equiv$,
  between processes is the least congruence containing
  alpha-equivalence, satisfying the abelian monoid laws
  (associativity, commutativity and $\pzero$ as identity) for parallel
  composition $|$ and for summation $+$.
\end{definition}

\subsection{Name equivalence}

We take name equivalence, written $\nameeq$, to be the smallest
equivalence relation generated by the following rules.

\begin{mathpar}
\inferrule*[lab=Quote-drop]
{ }
{ \quotep{@{x}} \nameeq x }

\inferrule*[lab=Struct-equiv]
{ P \scong Q }
{ \quotep{P} \nameeq \quotep{Q} }
\end{mathpar}

The astute reader will have noticed that the mutual recursion of names
and processes imposes a mutual recursion on alpha-equivalence and
structural equivalence via name-equivalence. Fortunately, all of this
works out pleasantly and we may calculate in the natural way, free of
concern. The reader interested in the details is referred to the
appendix \ref{appendix:rho_details}.

\subsection{Substitution}

We use $\Proc$ for the set of processes, $\QProc$ for the set of
names, and $\id{\{}\vec{y} / \vec{x} \id{\}}$ to denote partial maps,
$s : \QProc \rightarrow \QProc$. A map, $s$ lifts, uniquely, to a map
on process terms, $\widehat{s} : \Proc \rightarrow \Proc$ by the
following equations.

\begin{mathpar}
  (0) \psubstp{Q}{P} := 0 \\
  (R \juxtap S) \psubstp{Q}{P}
  :=    
  (R)\psubstp{Q}{P} \juxtap (S) \psubstp{Q}{P} \\
  (x?(y).R) \psubstp{Q}{P}    
  :=    
  (x)\substp{Q}{P} (z)\concat( (R \psubstn{z}{y}) \psubstp{Q}{P} ) \\
  (\lift{x}{R}) \psubstp{Q}{P}  
  :=
  \lift{(x)\substp{Q}{P}}{ R \psubstp{Q}{P} } \\
%   (\dropn{x})  \psubstp{Q}{P}       
%   := 
%   \left\{ 
%     \begin{array}{ccc} 
%       \dropn{\quotep{Q}} & & x \nameeq \quotep{P} \\
%       \dropn{x} & & otherwise \\
%     \end{array}
%   \right. 
  (\dropn{x})  \psubstp{Q}{P}       
  := 
  \left\{ 
    \begin{array}{ccc} 
      Q & & x \nameeq \quotep{P} \\
      \dropn{x} & & otherwise \\
    \end{array}
  \right.
\end{mathpar}
 

where

\begin{eqnarray}
  (x)\id{\{} \lpquote Q \rpquote / \lpquote P \rpquote \id{\}}            = 
  \left\{ 
    \begin{array}{ccc}
      \lpquote Q \rpquote & & x \nameeq \lpquote P \rpquote \\
      x & & otherwise \\
    \end{array}
  \right. \nonumber
\end{eqnarray}

and $z$ is chosen distinct from $\quotep{P}$, $\quotep{Q}$, the free
names in $Q$, and all the names in $R$. Our $\alpha$-equivalence will
be built in the standard way from this substitution.

\begin{remark}\label{rem:no_self_referential_names}
  One consequence of these definitions is that $\forall P. \quotep{P}
  \not\in \freenames{P}$.
\end{remark}

\subsection{ Dynamic quote: an example }

Anticipating something of what's to come, consider applying the
substitution, $\widehat{\id{\{}u / z \id{\}}}$, to the following pair
of processes, $\lift{w}{y!(z)}$ and $w[ \lpquote y!(z) \rpquote ]$.

\begin{eqnarray}
	\lift{w}{y!(z)}\widehat{\id{\{}u / z \id{\}}}
		& = &
		\lift{w}{y!(u)} \nonumber\\
	w[ \lpquote y!(z) \rpquote ] \widehat{ \id{\{}u / z \id{\}} }
		& = &
		w[ \lpquote y!(z) \rpquote ] \nonumber
\end{eqnarray}

Because the body of the process between quotes is impervious to
substitution, we get radically different answers. In fact, by
examining the first process in an input context,
e.g. $x?(z).\lift{w}{y!(z)}$, we see that the process under the lift
operator may be shaped by prefixed inputs binding a name inside it. In
this sense, the lift operator will be seen as a way to dynamically
construct processes before reifying them as names.

Finally equipped with these standard features we can present the
dynamics of the calculus.

\subsubsection{Operational semantics} 

Finally, we introduce the computational dynamics. What marks these
algebras as distinct from other more traditionally studied algebraic
structures, e.g. vector spaces or polynomial rings, is the manner in
which dynamics is captured. In traditional structures, dynamics is typically
expressed through morphisms between such structures, as in linear maps
between vector spaces or morphisms between rings. In algebras
associated with the semantics of computation, the dynamics is
expressed as part of the algebraic structure itself, through a
reduction reduction relation typically denoted by $\red$. Below, we
give a recursive presentation of this relation for the calculus used
in the encoding.

$\red \subseteq \pi \times \pi$
$\red : \pi \to \mathcal{P}(\pi)$

\begin{mathpar}
  \inferrule* [lab=Comm] { \textsf{match}( x_{src}, x_{trgt} ) } { x_{trgt}?(y)P \; | \; x_{src}!\langle {Q} \rangle \red P\{\quotep{Q}/y}\} }
  \and \\
  \inferrule* [lab=Par] {{P} \red {P}'} {{{P} | {Q}} \red {{P}' | {Q}}}
  \and
  \inferrule* [lab=Equiv]{{{P} \scong {P}'} \andalso {{P}' \red {Q}'} \andalso {{Q}' \scong {Q}}}{{P} \red {Q}}
\end{mathpar}

\begin{eqnarray*}
  match_{\equiv} (\quotep{P},\quotep{Q}) & := & P \equiv Q \\
  match_{\dagger}(\quotep{P},\quotep{Q}) & := & \forall R. P|Q \red^{*} R => R \red^{*} 0 \\
  match_{K}(\quotep{P},\quotep{Q}) & := & K \mbox{ for some context } K
\end{eqnarray*}

$u?(x)P | u!\langle Q \rangle \red P\{\quotep{Q}/x\}$

%We write $\wred$ for $\red^*$, and $P\red$ if $\exists Q $ such that $ P \red Q$.
We write $P\red$ if $\exists Q $ such that $ P \red Q$ and $P\not\red$, otherwise.

\section{Replication}

As mentioned before, it is known that replication (and hence
recursion) can be implemented in a higher-order process algebra
\cite{SangiorgiWalker}. As our first example of calculation with the
machinery thus far presented we give the construction explicitly in
the {\rhoc}.

\begin{eqnarray}
	D_{x} & := & \prefix{x}{y}{(\binpar{\outputp{x}{y}}{@{y}})} \nonumber\\
	\bangp_{x}{P} & := & \binpar{{x}!\langle{\binpar{D_{x}}{P}}\rangle}{D_{x}} \nonumber
\end{eqnarray}

\begin{eqnarray}
	\bangp_{x}{P} & & \nonumber\\
	=
	& {x}!\langle{(\prefix{x}{y}{(\outputp{x}{y} | @{y})) | P}}\rangle 
	      | \prefix{x}{y}{(\outputp{x}{y} | @{y})} & \nonumber\\
	\red
	& (\outputp{x}{y} | @{y})\substn{\quotep{(\prefix{x}{y}{(@{y} | \outputp{x}{y})) | P}}}{y} & \nonumber\\
	=
	& \outputp{x}{\quotep{(\prefix{x}{y}{(\outputp{x}{y} | @{y})) | P}}}
	  | {(\prefix{x}{y}{(\outputp{x}{y} | @{y})) | P}} & \nonumber\\
	\red
	& \ldots & \nonumber\\
	\red^*
	& P | P | \ldots & \nonumber
\end{eqnarray}

Of course, this encoding, as an implementation, runs away, unfolding
$\bangp{P}$ eagerly. A lazier and more implementable replication
operator, restricted to input-guarded processes, may be obtained as follows.

\begin{eqnarray}
\bangp{\prefix{u}{v}{P}} 
	:= 
	\binpar{\lift{x}{\prefix{u}{v}{(\binpar{D(x)}{P})}}}{D(x)} \nonumber
\end{eqnarray}

\begin{remark}
  Note that the lazier definition still does not deal with summation
  or mixed summation (i.e. sums over input and output). The reader is
  invited to construct definitions of replication that deal with these
  features. 

  Further, the definitions are parameterized in a name, $x$. Can you,
  gentle reader, make a definition that eliminates this parameter and
  guarantees no accidental interaction between the replication
  machinery and the process being replicated -- i.e. no accidental
  sharing of names used by the process to get its work done and the
  name(s) used by the replication to effect copying. This latter
  revision of the definition of replication is crucial to obtaining
  the expected identity $!!P \sim !P$.
\end{remark}

\begin{remark}\label{rem:paradoxical_combinator}
  The reader familiar with the lambda calculus will have noticed the
  similarity between $D$ and the paradoxical combinator.

  [Ed. note: the existence of this seems to suggest we have to be more
  restrictive on the set of processes and names we admit if we are to
  support no-cloning.]
\end{remark}

\subsubsection{Bisimulation}

The computational dynamics gives rise to another kind of equivalence,
the equivalence of computational behavior. As previously mentioned
this is typically captured \emph{via} some form of bisimulation.

% The notion we use in this paper is weak barbed bisimulation
% \cite{milner91polyadicpi}.

The notion we use in this paper is derived from weak barbed
bisimulation \cite{milner91polyadicpi}. 

\begin{definition}
An \emph{observation relation}, $\downarrow_{\mathcal N}$, over a set
of names, $\mathcal N$, is the smallest relation satisfying the rules
below.

\infrule[Out-barb]{y \in {\mathcal N}, \; x \nameeq y}
		  {\outputp{x}{v} \downarrow_{\mathcal N} x}
\infrule[Par-barb]{\mbox{$P\downarrow_{\mathcal N} x$ or $Q\downarrow_{\mathcal N} x$}}
		  {\binpar{P}{Q} \downarrow_{\mathcal N} x}

We write $P \Downarrow_{\mathcal N} x$ if there is $Q$ such that 
$P \wred Q$ and $Q \downarrow_{\mathcal N} x$.
\end{definition}

\begin{definition}
%\label{def.bbisim}
An  ${\mathcal N}$-\emph{barbed bisimulation} over a set of names, ${\mathcal N}$, is a symmetric binary relation 
${\mathcal S}_{\mathcal N}$ between agents such that $P\rel{S}_{\mathcal N}Q$ implies:
\begin{enumerate}
\item If $P \red P'$ then $Q \wred Q'$ and $P'\rel{S}_{\mathcal N} Q'$.
\item If $P\downarrow_{\mathcal N} x$, then $Q\Downarrow_{\mathcal N} x$.
\end{enumerate}
$P$ is ${\mathcal N}$-barbed bisimilar to $Q$, written
$P \wbbisim_{\mathcal N} Q$, if $P \rel{S}_{\mathcal N} Q$ for some ${\mathcal N}$-barbed bisimulation ${\mathcal S}_{\mathcal N}$.
\end{definition}

$\mathcal{R} \subseteq \pi \times \pi$

$P \mathcal{R} Q => \forall P'. P \red P' \Rightarrow \exists Q'. Q \red Q', P' \mathcal{R} Q'$

$P \vdash x \Rightarrow Q \vdash x$

\begin{mathpar}
  \inferrule*[lab=Out-barb]{x \nameeq y}{{y}!\langle{Q}\rangle \vdash x}
  \and
  \inferrule*[lab=Par-barb]{\mbox{$P\vdash x$ or $Q\vdash x$}}{\binpar{P}{Q} \vdash x}
\end{mathpar}

\subsubsection{Contexts}

One of the principle advantages of computational calculi like the
$\pi$-calculus is a well-defined notion of context,
contextual-equivalence and a correlation between
contextual-equivalence and notions of bisimulation. The notion of
context allows the decomposition of a process into (sub-)process and
its syntactic environment, its context. Thus, a context may be
thought of as a process with a ``hole'' (written $\Box$) in it. The
application of a context $M$ to a process $P$, written $M[P]$, is
tantamount to filling the hole in $M$ with $P$. In this paper we do
not need the full weight of this theory, but do make use of the notion
of context in the proof the main theorem. 

\begin{mathpar}
  \inferrule* [lab=summation] {} {{M_{M},M_{N}} \bc \Box \;|\; x.M_{A} \;|\; M_{M}+M_{N}}
  \and
  \inferrule* [lab=agent] {} {{M_{A}} \bc (\vec{x})M_{P} \;| \; \clift{P_0,\ldots,M_{P},\ldots,P_N}}
  \and \\
  \inferrule* [lab=process] {} {{M_{P}} \bc M_{N} \;| \;P|M_{P} }
\end{mathpar} 

\begin{mathpar}
  \inferrule* [lab=sychronization] {} {M_{N} \bc \Box \;|\; x?M_{F} \;|\; x!M_{C}}
  \and
  \inferrule* [lab=abstraction] {} {{M_{F}} \bc (x)M_{P} }
  \and
  \inferrule* [lab=concretion] {} {{M_{C}} \bc \langle M_{P} \rangle }
  \and \\
  \inferrule* [lab=process] {} {{M_{P}} \bc M_{N} \;| \;P|M_{P} }
\end{mathpar}

\begin{definition}[contextual application] Given a context $M$, and
  process $P$, we define the \emph{contextual application}, $M[P] :=
  M\{P/\Box\}$. That is, the contextual application of M to P is the
  substitution of $P$ for $\Box$ in $M$.
\end{definition}

$\meaningof{-} : L \to \mathcal{P}(\pi)$

\begin{mathpar}
  \inferrule* [lab=collection] {} {\meaningof{true} = \pi, \and \meaningof{~E} = \pi \setminus \meaningof{E}, \and \meaningof{E_{1} \& E_{2}} = \meaningof{E_{1}} \cap \meaningof{E_{2}}}
\end{mathpar}

\begin{mathpar}
  \inferrule* [lab=structure] {} {\meaningof{0} = \{ P \in \pi | P \equiv 0 \}, \and \\ \meaningof{E_1 | E_2} = \{ P \in \pi | P \equiv P_{1} | P_{2}, P_{1} \in \meaningof{E_{1}}, P_{2} \in \meaningof{E_2}\} }
\end{mathpar}

\begin{mathpar}
 \inferrule* [lab=behavior] {} {\meaningof{\langle a?b \rangle E} = \{ P \in \pi | P \equiv Q | u?(y)P', \\ \and \\\\ \and \\ \;\;\; u \in \meaningof{a}, \forall z.P'\{z/y\} \in \meaningof{E\{z/b\}}\}, \and \\ \meaningof{a!E} = \{ P \in \pi | P \equiv Q | x!\langle P' \rangle, x \in \meaningof{a} P' \in \meaningof{E}\} }
\end{mathpar}

\begin{mathpar}
 \inferrule* [lab=nominal] {} {\meaningof{\quotep{E}} = \{ \quotep{P} \in \quotep{\pi} | P \in \meaningof{E} \}, \and \meaningof{\quotep{P}} = \{ \quotep{Q} \in \quotep{\pi} | P \equiv Q \} \and \\ \meaningof{@\quotep{E}} = \{ P \in \pi | P \equiv @x, x \in \meaningof{E} \}}
\end{mathpar}

\begin{eqnarray*}
  \\
  \meaningof{-} : TS \to ST
\end{eqnarray*}

\begin{eqnarray*}
  \\
  L : TS \to ST
\end{eqnarray*}

\begin{eqnarray*}
  \\
  P \models E \iff P \in \meaningof{E}
\end{eqnarray*}

\begin{eqnarray*}
  P \approx_{L} Q \iff \forall E \in L. P \models E \iff Q \models E
\end{eqnarray*}

\begin{eqnarray*}
  P \approx_{K} Q
\end{eqnarray*}

\begin{eqnarray*}
  P \approx Q
\end{eqnarray*}

$\approx_{K} = \approx = \approx_{L}$

\subsubsection{Contextual duality}

Note that contexts extend the quotation operation to a family of
operations from processes to names. Given a context, $M$, we can
define a \emph{nominal context}, $\quotep{M}$ by $\quotep{M}[P] :=
\quotep{M[P]}$. To foreshadow what is to come we observe that these
operations enjoy a duality with processes very much like the duality
between vectors and maps from vectors to scalars.

Further, because the calculus is essentially higher-order, we have a
correspondence between contexts and processes. More specifically,
given a name $x$ and a context $M$ we can construct $M^{*}_{x}$ such
that 

\begin{mathpar}
  M^{*}_{x} | \lift{x}{P} \red M[P]
\end{mathpar}

namely,

\begin{mathpar}
  M^{*}_{x} := x?(u).M[\dropn{u}]
\end{mathpar}

The dependence of $M^{*}_{x}$ on a name makes it an abstraction, 

\begin{mathpar}
  M^{*} := (x)x?(u).M[\dropn{u}]
\end{mathpar}

\subsection{Additional notation}

It will sometimes be convenient to denote the process a name
quotes. We already have the notation $x = \quotep{P}$, but it will be
convenient to introduce an alternate notation, $\procn{x}$, when we
want to emphasize the connection to the use of the name. Note that, by
virtue of name equivalence, $\quotep{\procn{x}} \nameeq x$; so, the
notation is consistent with previous definitions.

Further, because names have structure it is possible to effect
substitutions on the basis of that structure. This means we need to
upgrade our notation for substitutions, which we accomplish by
adapting comprehension notation. Thus,

\begin{mathpar}
  P\{ y / x : x \in S \}
\end{mathpar}

is interpreted to mean the process derived from P by replacing (in a
capture-avoiding manner) each occurrence of $x$ in $S$ by $y$. For example,

\begin{mathpar}
  P\{ \quotep{\procn{x}|\procn{x}} / x : x \in \freenames{P} \}
\end{mathpar}

will replace each (occurrence) of a free name $x$ in $P$ by
$\quotep{\procn{x}|\procn{x}}$.

Also, we will avail ourselves of the notation $x^{L}$ and $x^{R}$ to
denote injections of a name into disjoint copies of the name
space. There are numerous ways to accomplish this. One example can be
found in \cite{MeredithR05}. This notation overloads to vectors of
names: $\vec{x}^{\pi} := (x_{i}^{\pi} \; : \; 0 \leq i < |\vec{x}| )$ where $\pi \in \{L,R\}$.

We also use $P^{\Box} := P|\Box$.

In \cite{MeredithR05} an interpretation of the new operator is
given. It turns out that there are several possible interpretations
all enjoying the requisite algebraic properties of the operator (see
\cite{milner91polyadicpi}). We will therefore make liberal use of
$(\nu\; \vec{x})P$.

% subsection the_syntax_and_semantics_of_the_notation_system (end)   

\input{qm2pi.qmops} 

\input{qm2pi.sterngerlach} 

\input{qm2pi.metric} 

% section concurrent_process_calculi (end)

%\input{qm2pi.proofsketch}

% section proof sketch (end)

%\input{qm2pi.slviaknots} 

% section spatial logic via knots (end)

\input{qm2pi.conclusion}

% section conclusion (end)

%\input{qm2pi.dtcodes} 

% section wiring algorithm (end)

\input{qm2pi.ack} 

% section acknowledgments (end)

\newpage


\bibliographystyle{plain}   
\bibliography{../../biblios/main.bib}

\input{qm2pi.rhodetails}

\end{document}

 

%\documentclass[12pt]{llncs}
%\documentclass{jktr}

\usepackage[pdftex]{hyperref}                   
\usepackage {listings}
\usepackage {mathpartir}
\usepackage{bcprules}
%\usepackage{listings}
                       
\usepackage{graphicx} 
%\usepackage[margins=2.5cm,nohead,nofoot]{geometry}
%\usepackage{geometry}
\usepackage{amsfonts}
\usepackage{amstext}
\usepackage{latexsym}
\usepackage{amssymb}
\usepackage{color}


%\include{myPreamble}
\include{qm2pi.local} 

%\ifpdf
%\usepackage[pdftex]{graphicx}
%\else
%\usepackage{graphicx}
%\fi

 % \ifpdf
%  \usepackage{pdfsync}
%  \if


%\title{Brief Article}
%\author{David F. Snyder}
%\author{L.G. Meredith}

%\address{Dept. of Math., Texas State University--San Marcos, San Marcos, TX 78666}
       
\pagestyle{empty}


\begin{document}

\lstset{language=[Objective]Caml,frame=shadowbox}

\input{qm2pi.front}

% section front matter (end)

\input{qm2pi.intro} 
 
% section introduction (end)

% \input{qm2pi.knotations} 

% section notation (end)

\input{qm2pi.process.calculi} 

% section concurrent_process_calculi_and_spatial_logics_ (end)
    
%\input{qm2pi.knots2pi} 

%\input{qm2pi.trefoil} 

%\input{qm2pi.mainthm} 

% subsection basic_interpretation (end)

%\input{qm2pi.rho.presentation} 
\subsection{The syntax and semantics of the notation system}\label{sub:the_syntax_and_semantics_of_the_notation_system} % (fold)

We now summarize a technical presentation of the calculus that
embodies our theory of dynamics. The typical presentation of such a
calculus follows the style of giving generators and relations on
them. The grammar, below, describing term constructors, freely
generates the set of processes, $\Proc$. This set is then quotiented
by a relation known as structural congruence and it is over this set
that the notion of dynamics is expressed. This presentation is
essentially that of \cite{MeredithR05} with the addition of
polyadicity and summation. For readability we have relegated some of
the technical subtleties to an appendix.

\subsubsection{Process grammar}\label{subsub:process_grammar}

\begin{mathpar}
  \inferrule* [lab=synchronization] {} {{M} \bc \pzero \;|\; x?F \;|\; x!C }
  \and
  \inferrule* [lab=abstraction] {} {{F} \bc (x)P}
  \and
  \inferrule* [lab=concretion] {} {{C} \bc \langle Q \rangle}
  \and
  \inferrule* [lab=process] {} {{P,Q} \bc M \;| \;P|Q \;|\; @{x}}
  \and
  \inferrule* [lab=name] {} {{x} \bc \quotep{P}}
\end{mathpar} 

Note that $\vec{x}$ (resp. $\vec{P}$) denotes a vector of names
(resp. processes) of length $|\vec{x}|$ (resp. $|\vec{P}|$). We adopt
the following useful abbreviations.

\begin{mathpar}
   x?(\vec{y}).P := x.(\vec{y})P \and  x\clift{\vec{P}} := x.\clift{\vec{P}}
   \and x!(y) := \lift{x}{\dropn{y}}
   \and \Pi_{i=0}^{n-1}P_i := P_0 | \ldots | P_{n-1}
\end{mathpar}

\subsubsection{Structural congruence}

\paragraph{Free and bound names and alpha-equivalence.} At the
core of structural equivalence is alpha-equivalence which identifies
process that are the same up to a change of variable. Formally, we
recognize the distinction between free and bound names. The free names
of a process, $\freenames{P}$, may be calculated recursively as
follows:

\begin{mathpar}
\freenames{\pzero} := \emptyset
  \and \\
  \freenames{x?(y).P} := \{ x \} \cup (\freenames{P} \setminus \{ y \})
  \and 
  \freenames{x!\langle P \rangle} := \{ x \} \cup \{ P \} 
  \and \\
  \freenames{P|Q} := \freenames{P} \cup \freenames{Q}
  \and \\
  \freenames{@{x}} := \{ x \}
\end{mathpar}

$\pi$
$\quotep{\pi}$

$\freenames{-} : \pi \to \mathcal{P}(\quotep{\pi})$

\begin{eqnarray*}
  \freenames{\pzero} & := & \emptyset \\
  \freenames{x?(y).P} & := & \{ x \} \cup (\freenames{P} \setminus \{ y \}) \\
  \freenames{x!\langle P \rangle} & := & \{ x \} \cup \{ P \} \\
  \freenames{P|Q} & := & \freenames{P} \cup \freenames{Q} \\
  \freenames{\dropn{x}} & := & \{ x \}
\end{eqnarray*}

The bound names of a process, $\boundnames{P}$, are those names occurring in $P$
that are not free. For example, in $x?(y).0$, the name $x$ is free, while $y$ is bound.

\begin{mathpar}
  \inferrule* [lab=monoidal-laws] {} { P|Q \equiv Q|P \and P|0 \equiv P \and P|(Q|R) \equiv (P|Q)|R }
\end{mathpar}

\begin{mathpar}
  \inferrule* [lab=alpha-equivalence] {} { (x)P \equiv (y)P\{y/x\} \and y \not\in \freenames{P} }
\end{mathpar}

\begin{definition}
Then two processes, $P,Q$, are alpha-equivalent if $P = Q\{\vec{y}/\vec{x}\}$ for
some $\vec{x} \in \boundnames{Q},\vec{y} \in \boundnames{P}$, where $Q\{\vec{y}/\vec{x}\}$
denotes the capture-avoiding substitution of $\vec{y}$ for $\vec{x}$ in $Q$.
\end{definition}

\begin{definition}
  The {\em structural congruence} \cite{SangiorgiWalker} , $\equiv$,
  between processes is the least congruence containing
  alpha-equivalence, satisfying the abelian monoid laws
  (associativity, commutativity and $\pzero$ as identity) for parallel
  composition $|$ and for summation $+$.
\end{definition}

\subsection{Name equivalence}

We take name equivalence, written $\nameeq$, to be the smallest
equivalence relation generated by the following rules.

\begin{mathpar}
\inferrule*[lab=Quote-drop]
{ }
{ \quotep{@{x}} \nameeq x }

\inferrule*[lab=Struct-equiv]
{ P \scong Q }
{ \quotep{P} \nameeq \quotep{Q} }
\end{mathpar}

The astute reader will have noticed that the mutual recursion of names
and processes imposes a mutual recursion on alpha-equivalence and
structural equivalence via name-equivalence. Fortunately, all of this
works out pleasantly and we may calculate in the natural way, free of
concern. The reader interested in the details is referred to the
appendix \ref{appendix:rho_details}.

\subsection{Substitution}

We use $\Proc$ for the set of processes, $\QProc$ for the set of
names, and $\id{\{}\vec{y} / \vec{x} \id{\}}$ to denote partial maps,
$s : \QProc \rightarrow \QProc$. A map, $s$ lifts, uniquely, to a map
on process terms, $\widehat{s} : \Proc \rightarrow \Proc$ by the
following equations.

\begin{mathpar}
  (0) \psubstp{Q}{P} := 0 \\
  (R \juxtap S) \psubstp{Q}{P}
  :=    
  (R)\psubstp{Q}{P} \juxtap (S) \psubstp{Q}{P} \\
  (x?(y).R) \psubstp{Q}{P}    
  :=    
  (x)\substp{Q}{P} (z)\concat( (R \psubstn{z}{y}) \psubstp{Q}{P} ) \\
  (\lift{x}{R}) \psubstp{Q}{P}  
  :=
  \lift{(x)\substp{Q}{P}}{ R \psubstp{Q}{P} } \\
%   (\dropn{x})  \psubstp{Q}{P}       
%   := 
%   \left\{ 
%     \begin{array}{ccc} 
%       \dropn{\quotep{Q}} & & x \nameeq \quotep{P} \\
%       \dropn{x} & & otherwise \\
%     \end{array}
%   \right. 
  (\dropn{x})  \psubstp{Q}{P}       
  := 
  \left\{ 
    \begin{array}{ccc} 
      Q & & x \nameeq \quotep{P} \\
      \dropn{x} & & otherwise \\
    \end{array}
  \right.
\end{mathpar}
 

where

\begin{eqnarray}
  (x)\id{\{} \lpquote Q \rpquote / \lpquote P \rpquote \id{\}}            = 
  \left\{ 
    \begin{array}{ccc}
      \lpquote Q \rpquote & & x \nameeq \lpquote P \rpquote \\
      x & & otherwise \\
    \end{array}
  \right. \nonumber
\end{eqnarray}

and $z$ is chosen distinct from $\quotep{P}$, $\quotep{Q}$, the free
names in $Q$, and all the names in $R$. Our $\alpha$-equivalence will
be built in the standard way from this substitution.

\begin{remark}\label{rem:no_self_referential_names}
  One consequence of these definitions is that $\forall P. \quotep{P}
  \not\in \freenames{P}$.
\end{remark}

\subsection{ Dynamic quote: an example }

Anticipating something of what's to come, consider applying the
substitution, $\widehat{\id{\{}u / z \id{\}}}$, to the following pair
of processes, $\lift{w}{y!(z)}$ and $w[ \lpquote y!(z) \rpquote ]$.

\begin{eqnarray}
	\lift{w}{y!(z)}\widehat{\id{\{}u / z \id{\}}}
		& = &
		\lift{w}{y!(u)} \nonumber\\
	w[ \lpquote y!(z) \rpquote ] \widehat{ \id{\{}u / z \id{\}} }
		& = &
		w[ \lpquote y!(z) \rpquote ] \nonumber
\end{eqnarray}

Because the body of the process between quotes is impervious to
substitution, we get radically different answers. In fact, by
examining the first process in an input context,
e.g. $x?(z).\lift{w}{y!(z)}$, we see that the process under the lift
operator may be shaped by prefixed inputs binding a name inside it. In
this sense, the lift operator will be seen as a way to dynamically
construct processes before reifying them as names.

Finally equipped with these standard features we can present the
dynamics of the calculus.

\subsubsection{Operational semantics} 

Finally, we introduce the computational dynamics. What marks these
algebras as distinct from other more traditionally studied algebraic
structures, e.g. vector spaces or polynomial rings, is the manner in
which dynamics is captured. In traditional structures, dynamics is typically
expressed through morphisms between such structures, as in linear maps
between vector spaces or morphisms between rings. In algebras
associated with the semantics of computation, the dynamics is
expressed as part of the algebraic structure itself, through a
reduction reduction relation typically denoted by $\red$. Below, we
give a recursive presentation of this relation for the calculus used
in the encoding.

$\red \subseteq \pi \times \pi$
$\red : \pi \to \mathcal{P}(\pi)$

\begin{mathpar}
  \inferrule* [lab=Comm] { \textsf{match}( x_{src}, x_{trgt} ) } { x_{trgt}?(y)P \; | \; x_{src}!\langle {Q} \rangle \red P\{\quotep{Q}/y}\} }
  \and \\
  \inferrule* [lab=Par] {{P} \red {P}'} {{{P} | {Q}} \red {{P}' | {Q}}}
  \and
  \inferrule* [lab=Equiv]{{{P} \scong {P}'} \andalso {{P}' \red {Q}'} \andalso {{Q}' \scong {Q}}}{{P} \red {Q}}
\end{mathpar}

\begin{eqnarray*}
  match_{\equiv} (\quotep{P},\quotep{Q}) & := & P \equiv Q \\
  match_{\dagger}(\quotep{P},\quotep{Q}) & := & \forall R. P|Q \red^{*} R => R \red^{*} 0 \\
  match_{K}(\quotep{P},\quotep{Q}) & := & K \mbox{ for some context } K
\end{eqnarray*}

$u?(x)P | u!\langle Q \rangle \red P\{\quotep{Q}/x\}$

%We write $\wred$ for $\red^*$, and $P\red$ if $\exists Q $ such that $ P \red Q$.
We write $P\red$ if $\exists Q $ such that $ P \red Q$ and $P\not\red$, otherwise.

\section{Replication}

As mentioned before, it is known that replication (and hence
recursion) can be implemented in a higher-order process algebra
\cite{SangiorgiWalker}. As our first example of calculation with the
machinery thus far presented we give the construction explicitly in
the {\rhoc}.

\begin{eqnarray}
	D_{x} & := & \prefix{x}{y}{(\binpar{\outputp{x}{y}}{@{y}})} \nonumber\\
	\bangp_{x}{P} & := & \binpar{{x}!\langle{\binpar{D_{x}}{P}}\rangle}{D_{x}} \nonumber
\end{eqnarray}

\begin{eqnarray}
	\bangp_{x}{P} & & \nonumber\\
	=
	& {x}!\langle{(\prefix{x}{y}{(\outputp{x}{y} | @{y})) | P}}\rangle 
	      | \prefix{x}{y}{(\outputp{x}{y} | @{y})} & \nonumber\\
	\red
	& (\outputp{x}{y} | @{y})\substn{\quotep{(\prefix{x}{y}{(@{y} | \outputp{x}{y})) | P}}}{y} & \nonumber\\
	=
	& \outputp{x}{\quotep{(\prefix{x}{y}{(\outputp{x}{y} | @{y})) | P}}}
	  | {(\prefix{x}{y}{(\outputp{x}{y} | @{y})) | P}} & \nonumber\\
	\red
	& \ldots & \nonumber\\
	\red^*
	& P | P | \ldots & \nonumber
\end{eqnarray}

Of course, this encoding, as an implementation, runs away, unfolding
$\bangp{P}$ eagerly. A lazier and more implementable replication
operator, restricted to input-guarded processes, may be obtained as follows.

\begin{eqnarray}
\bangp{\prefix{u}{v}{P}} 
	:= 
	\binpar{\lift{x}{\prefix{u}{v}{(\binpar{D(x)}{P})}}}{D(x)} \nonumber
\end{eqnarray}

\begin{remark}
  Note that the lazier definition still does not deal with summation
  or mixed summation (i.e. sums over input and output). The reader is
  invited to construct definitions of replication that deal with these
  features. 

  Further, the definitions are parameterized in a name, $x$. Can you,
  gentle reader, make a definition that eliminates this parameter and
  guarantees no accidental interaction between the replication
  machinery and the process being replicated -- i.e. no accidental
  sharing of names used by the process to get its work done and the
  name(s) used by the replication to effect copying. This latter
  revision of the definition of replication is crucial to obtaining
  the expected identity $!!P \sim !P$.
\end{remark}

\begin{remark}\label{rem:paradoxical_combinator}
  The reader familiar with the lambda calculus will have noticed the
  similarity between $D$ and the paradoxical combinator.

  [Ed. note: the existence of this seems to suggest we have to be more
  restrictive on the set of processes and names we admit if we are to
  support no-cloning.]
\end{remark}

\subsubsection{Bisimulation}

The computational dynamics gives rise to another kind of equivalence,
the equivalence of computational behavior. As previously mentioned
this is typically captured \emph{via} some form of bisimulation.

% The notion we use in this paper is weak barbed bisimulation
% \cite{milner91polyadicpi}.

The notion we use in this paper is derived from weak barbed
bisimulation \cite{milner91polyadicpi}. 

\begin{definition}
An \emph{observation relation}, $\downarrow_{\mathcal N}$, over a set
of names, $\mathcal N$, is the smallest relation satisfying the rules
below.

\infrule[Out-barb]{y \in {\mathcal N}, \; x \nameeq y}
		  {\outputp{x}{v} \downarrow_{\mathcal N} x}
\infrule[Par-barb]{\mbox{$P\downarrow_{\mathcal N} x$ or $Q\downarrow_{\mathcal N} x$}}
		  {\binpar{P}{Q} \downarrow_{\mathcal N} x}

We write $P \Downarrow_{\mathcal N} x$ if there is $Q$ such that 
$P \wred Q$ and $Q \downarrow_{\mathcal N} x$.
\end{definition}

\begin{definition}
%\label{def.bbisim}
An  ${\mathcal N}$-\emph{barbed bisimulation} over a set of names, ${\mathcal N}$, is a symmetric binary relation 
${\mathcal S}_{\mathcal N}$ between agents such that $P\rel{S}_{\mathcal N}Q$ implies:
\begin{enumerate}
\item If $P \red P'$ then $Q \wred Q'$ and $P'\rel{S}_{\mathcal N} Q'$.
\item If $P\downarrow_{\mathcal N} x$, then $Q\Downarrow_{\mathcal N} x$.
\end{enumerate}
$P$ is ${\mathcal N}$-barbed bisimilar to $Q$, written
$P \wbbisim_{\mathcal N} Q$, if $P \rel{S}_{\mathcal N} Q$ for some ${\mathcal N}$-barbed bisimulation ${\mathcal S}_{\mathcal N}$.
\end{definition}

$\mathcal{R} \subseteq \pi \times \pi$

$P \mathcal{R} Q => \forall P'. P \red P' \Rightarrow \exists Q'. Q \red Q', P' \mathcal{R} Q'$

$P \vdash x \Rightarrow Q \vdash x$

\begin{mathpar}
  \inferrule*[lab=Out-barb]{x \nameeq y}{{y}!\langle{Q}\rangle \vdash x}
  \and
  \inferrule*[lab=Par-barb]{\mbox{$P\vdash x$ or $Q\vdash x$}}{\binpar{P}{Q} \vdash x}
\end{mathpar}

\subsubsection{Contexts}

One of the principle advantages of computational calculi like the
$\pi$-calculus is a well-defined notion of context,
contextual-equivalence and a correlation between
contextual-equivalence and notions of bisimulation. The notion of
context allows the decomposition of a process into (sub-)process and
its syntactic environment, its context. Thus, a context may be
thought of as a process with a ``hole'' (written $\Box$) in it. The
application of a context $M$ to a process $P$, written $M[P]$, is
tantamount to filling the hole in $M$ with $P$. In this paper we do
not need the full weight of this theory, but do make use of the notion
of context in the proof the main theorem. 

\begin{mathpar}
  \inferrule* [lab=summation] {} {{M_{M},M_{N}} \bc \Box \;|\; x.M_{A} \;|\; M_{M}+M_{N}}
  \and
  \inferrule* [lab=agent] {} {{M_{A}} \bc (\vec{x})M_{P} \;| \; \clift{P_0,\ldots,M_{P},\ldots,P_N}}
  \and \\
  \inferrule* [lab=process] {} {{M_{P}} \bc M_{N} \;| \;P|M_{P} }
\end{mathpar} 

\begin{mathpar}
  \inferrule* [lab=sychronization] {} {M_{N} \bc \Box \;|\; x?M_{F} \;|\; x!M_{C}}
  \and
  \inferrule* [lab=abstraction] {} {{M_{F}} \bc (x)M_{P} }
  \and
  \inferrule* [lab=concretion] {} {{M_{C}} \bc \langle M_{P} \rangle }
  \and \\
  \inferrule* [lab=process] {} {{M_{P}} \bc M_{N} \;| \;P|M_{P} }
\end{mathpar}

\begin{definition}[contextual application] Given a context $M$, and
  process $P$, we define the \emph{contextual application}, $M[P] :=
  M\{P/\Box\}$. That is, the contextual application of M to P is the
  substitution of $P$ for $\Box$ in $M$.
\end{definition}

$\meaningof{-} : L \to \mathcal{P}(\pi)$

\begin{mathpar}
  \inferrule* [lab=collection] {} {\meaningof{true} = \pi, \and \meaningof{~E} = \pi \setminus \meaningof{E}, \and \meaningof{E_{1} \& E_{2}} = \meaningof{E_{1}} \cap \meaningof{E_{2}}}
\end{mathpar}

\begin{mathpar}
  \inferrule* [lab=structure] {} {\meaningof{0} = \{ P \in \pi | P \equiv 0 \}, \and \\ \meaningof{E_1 | E_2} = \{ P \in \pi | P \equiv P_{1} | P_{2}, P_{1} \in \meaningof{E_{1}}, P_{2} \in \meaningof{E_2}\} }
\end{mathpar}

\begin{mathpar}
 \inferrule* [lab=behavior] {} {\meaningof{\langle a?b \rangle E} = \{ P \in \pi | P \equiv Q | u?(y)P', \\ \and \\\\ \and \\ \;\;\; u \in \meaningof{a}, \forall z.P'\{z/y\} \in \meaningof{E\{z/b\}}\}, \and \\ \meaningof{a!E} = \{ P \in \pi | P \equiv Q | x!\langle P' \rangle, x \in \meaningof{a} P' \in \meaningof{E}\} }
\end{mathpar}

\begin{mathpar}
 \inferrule* [lab=nominal] {} {\meaningof{\quotep{E}} = \{ \quotep{P} \in \quotep{\pi} | P \in \meaningof{E} \}, \and \meaningof{\quotep{P}} = \{ \quotep{Q} \in \quotep{\pi} | P \equiv Q \} \and \\ \meaningof{@\quotep{E}} = \{ P \in \pi | P \equiv @x, x \in \meaningof{E} \}}
\end{mathpar}

\begin{eqnarray*}
  \\
  \meaningof{-} : TS \to ST
\end{eqnarray*}

\begin{eqnarray*}
  \\
  L : TS \to ST
\end{eqnarray*}

\begin{eqnarray*}
  \\
  P \models E \iff P \in \meaningof{E}
\end{eqnarray*}

\begin{eqnarray*}
  P \approx_{L} Q \iff \forall E \in L. P \models E \iff Q \models E
\end{eqnarray*}

\begin{eqnarray*}
  P \approx_{K} Q
\end{eqnarray*}

\begin{eqnarray*}
  P \approx Q
\end{eqnarray*}

$\approx_{K} = \approx = \approx_{L}$

\subsubsection{Contextual duality}

Note that contexts extend the quotation operation to a family of
operations from processes to names. Given a context, $M$, we can
define a \emph{nominal context}, $\quotep{M}$ by $\quotep{M}[P] :=
\quotep{M[P]}$. To foreshadow what is to come we observe that these
operations enjoy a duality with processes very much like the duality
between vectors and maps from vectors to scalars.

Further, because the calculus is essentially higher-order, we have a
correspondence between contexts and processes. More specifically,
given a name $x$ and a context $M$ we can construct $M^{*}_{x}$ such
that 

\begin{mathpar}
  M^{*}_{x} | \lift{x}{P} \red M[P]
\end{mathpar}

namely,

\begin{mathpar}
  M^{*}_{x} := x?(u).M[\dropn{u}]
\end{mathpar}

The dependence of $M^{*}_{x}$ on a name makes it an abstraction, 

\begin{mathpar}
  M^{*} := (x)x?(u).M[\dropn{u}]
\end{mathpar}

\subsection{Additional notation}

It will sometimes be convenient to denote the process a name
quotes. We already have the notation $x = \quotep{P}$, but it will be
convenient to introduce an alternate notation, $\procn{x}$, when we
want to emphasize the connection to the use of the name. Note that, by
virtue of name equivalence, $\quotep{\procn{x}} \nameeq x$; so, the
notation is consistent with previous definitions.

Further, because names have structure it is possible to effect
substitutions on the basis of that structure. This means we need to
upgrade our notation for substitutions, which we accomplish by
adapting comprehension notation. Thus,

\begin{mathpar}
  P\{ y / x : x \in S \}
\end{mathpar}

is interpreted to mean the process derived from P by replacing (in a
capture-avoiding manner) each occurrence of $x$ in $S$ by $y$. For example,

\begin{mathpar}
  P\{ \quotep{\procn{x}|\procn{x}} / x : x \in \freenames{P} \}
\end{mathpar}

will replace each (occurrence) of a free name $x$ in $P$ by
$\quotep{\procn{x}|\procn{x}}$.

Also, we will avail ourselves of the notation $x^{L}$ and $x^{R}$ to
denote injections of a name into disjoint copies of the name
space. There are numerous ways to accomplish this. One example can be
found in \cite{MeredithR05}. This notation overloads to vectors of
names: $\vec{x}^{\pi} := (x_{i}^{\pi} \; : \; 0 \leq i < |\vec{x}| )$ where $\pi \in \{L,R\}$.

We also use $P^{\Box} := P|\Box$.

In \cite{MeredithR05} an interpretation of the new operator is
given. It turns out that there are several possible interpretations
all enjoying the requisite algebraic properties of the operator (see
\cite{milner91polyadicpi}). We will therefore make liberal use of
$(\nu\; \vec{x})P$.

% subsection the_syntax_and_semantics_of_the_notation_system (end)   

\input{qm2pi.qmops} 

\input{qm2pi.sterngerlach} 

\input{qm2pi.metric} 

% section concurrent_process_calculi (end)

%\input{qm2pi.proofsketch}

% section proof sketch (end)

%\input{qm2pi.slviaknots} 

% section spatial logic via knots (end)

\input{qm2pi.conclusion}

% section conclusion (end)

%\input{qm2pi.dtcodes} 

% section wiring algorithm (end)

\input{qm2pi.ack} 

% section acknowledgments (end)

\newpage


\bibliographystyle{plain}   
\bibliography{../../biblios/main.bib}

\input{qm2pi.rhodetails}

\end{document}

 

% subsection basic_interpretation (end)

%\input{qm2pi.rho.presentation} 
\subsection{The syntax and semantics of the notation system}\label{sub:the_syntax_and_semantics_of_the_notation_system} % (fold)

We now summarize a technical presentation of the calculus that
embodies our theory of dynamics. The typical presentation of such a
calculus follows the style of giving generators and relations on
them. The grammar, below, describing term constructors, freely
generates the set of processes, $\Proc$. This set is then quotiented
by a relation known as structural congruence and it is over this set
that the notion of dynamics is expressed. This presentation is
essentially that of \cite{MeredithR05} with the addition of
polyadicity and summation. For readability we have relegated some of
the technical subtleties to an appendix.

\subsubsection{Process grammar}\label{subsub:process_grammar}

\begin{mathpar}
  \inferrule* [lab=synchronization] {} {{M} \bc \pzero \;|\; x?F \;|\; x!C }
  \and
  \inferrule* [lab=abstraction] {} {{F} \bc (x)P}
  \and
  \inferrule* [lab=concretion] {} {{C} \bc \langle Q \rangle}
  \and
  \inferrule* [lab=process] {} {{P,Q} \bc M \;| \;P|Q \;|\; @{x}}
  \and
  \inferrule* [lab=name] {} {{x} \bc \quotep{P}}
\end{mathpar} 

Note that $\vec{x}$ (resp. $\vec{P}$) denotes a vector of names
(resp. processes) of length $|\vec{x}|$ (resp. $|\vec{P}|$). We adopt
the following useful abbreviations.

\begin{mathpar}
   x?(\vec{y}).P := x.(\vec{y})P \and  x\clift{\vec{P}} := x.\clift{\vec{P}}
   \and x!(y) := \lift{x}{\dropn{y}}
   \and \Pi_{i=0}^{n-1}P_i := P_0 | \ldots | P_{n-1}
\end{mathpar}

\subsubsection{Structural congruence}

\paragraph{Free and bound names and alpha-equivalence.} At the
core of structural equivalence is alpha-equivalence which identifies
process that are the same up to a change of variable. Formally, we
recognize the distinction between free and bound names. The free names
of a process, $\freenames{P}$, may be calculated recursively as
follows:

\begin{mathpar}
\freenames{\pzero} := \emptyset
  \and \\
  \freenames{x?(y).P} := \{ x \} \cup (\freenames{P} \setminus \{ y \})
  \and 
  \freenames{x!\langle P \rangle} := \{ x \} \cup \{ P \} 
  \and \\
  \freenames{P|Q} := \freenames{P} \cup \freenames{Q}
  \and \\
  \freenames{@{x}} := \{ x \}
\end{mathpar}

$\pi$
$\quotep{\pi}$

$\freenames{-} : \pi \to \mathcal{P}(\quotep{\pi})$

\begin{eqnarray*}
  \freenames{\pzero} & := & \emptyset \\
  \freenames{x?(y).P} & := & \{ x \} \cup (\freenames{P} \setminus \{ y \}) \\
  \freenames{x!\langle P \rangle} & := & \{ x \} \cup \{ P \} \\
  \freenames{P|Q} & := & \freenames{P} \cup \freenames{Q} \\
  \freenames{\dropn{x}} & := & \{ x \}
\end{eqnarray*}

The bound names of a process, $\boundnames{P}$, are those names occurring in $P$
that are not free. For example, in $x?(y).0$, the name $x$ is free, while $y$ is bound.

\begin{mathpar}
  \inferrule* [lab=monoidal-laws] {} { P|Q \equiv Q|P \and P|0 \equiv P \and P|(Q|R) \equiv (P|Q)|R }
\end{mathpar}

\begin{mathpar}
  \inferrule* [lab=alpha-equivalence] {} { (x)P \equiv (y)P\{y/x\} \and y \not\in \freenames{P} }
\end{mathpar}

\begin{definition}
Then two processes, $P,Q$, are alpha-equivalent if $P = Q\{\vec{y}/\vec{x}\}$ for
some $\vec{x} \in \boundnames{Q},\vec{y} \in \boundnames{P}$, where $Q\{\vec{y}/\vec{x}\}$
denotes the capture-avoiding substitution of $\vec{y}$ for $\vec{x}$ in $Q$.
\end{definition}

\begin{definition}
  The {\em structural congruence} \cite{SangiorgiWalker} , $\equiv$,
  between processes is the least congruence containing
  alpha-equivalence, satisfying the abelian monoid laws
  (associativity, commutativity and $\pzero$ as identity) for parallel
  composition $|$ and for summation $+$.
\end{definition}

\subsection{Name equivalence}

We take name equivalence, written $\nameeq$, to be the smallest
equivalence relation generated by the following rules.

\begin{mathpar}
\inferrule*[lab=Quote-drop]
{ }
{ \quotep{@{x}} \nameeq x }

\inferrule*[lab=Struct-equiv]
{ P \scong Q }
{ \quotep{P} \nameeq \quotep{Q} }
\end{mathpar}

The astute reader will have noticed that the mutual recursion of names
and processes imposes a mutual recursion on alpha-equivalence and
structural equivalence via name-equivalence. Fortunately, all of this
works out pleasantly and we may calculate in the natural way, free of
concern. The reader interested in the details is referred to the
appendix \ref{appendix:rho_details}.

\subsection{Substitution}

We use $\Proc$ for the set of processes, $\QProc$ for the set of
names, and $\id{\{}\vec{y} / \vec{x} \id{\}}$ to denote partial maps,
$s : \QProc \rightarrow \QProc$. A map, $s$ lifts, uniquely, to a map
on process terms, $\widehat{s} : \Proc \rightarrow \Proc$ by the
following equations.

\begin{mathpar}
  (0) \psubstp{Q}{P} := 0 \\
  (R \juxtap S) \psubstp{Q}{P}
  :=    
  (R)\psubstp{Q}{P} \juxtap (S) \psubstp{Q}{P} \\
  (x?(y).R) \psubstp{Q}{P}    
  :=    
  (x)\substp{Q}{P} (z)\concat( (R \psubstn{z}{y}) \psubstp{Q}{P} ) \\
  (\lift{x}{R}) \psubstp{Q}{P}  
  :=
  \lift{(x)\substp{Q}{P}}{ R \psubstp{Q}{P} } \\
%   (\dropn{x})  \psubstp{Q}{P}       
%   := 
%   \left\{ 
%     \begin{array}{ccc} 
%       \dropn{\quotep{Q}} & & x \nameeq \quotep{P} \\
%       \dropn{x} & & otherwise \\
%     \end{array}
%   \right. 
  (\dropn{x})  \psubstp{Q}{P}       
  := 
  \left\{ 
    \begin{array}{ccc} 
      Q & & x \nameeq \quotep{P} \\
      \dropn{x} & & otherwise \\
    \end{array}
  \right.
\end{mathpar}
 

where

\begin{eqnarray}
  (x)\id{\{} \lpquote Q \rpquote / \lpquote P \rpquote \id{\}}            = 
  \left\{ 
    \begin{array}{ccc}
      \lpquote Q \rpquote & & x \nameeq \lpquote P \rpquote \\
      x & & otherwise \\
    \end{array}
  \right. \nonumber
\end{eqnarray}

and $z$ is chosen distinct from $\quotep{P}$, $\quotep{Q}$, the free
names in $Q$, and all the names in $R$. Our $\alpha$-equivalence will
be built in the standard way from this substitution.

\begin{remark}\label{rem:no_self_referential_names}
  One consequence of these definitions is that $\forall P. \quotep{P}
  \not\in \freenames{P}$.
\end{remark}

\subsection{ Dynamic quote: an example }

Anticipating something of what's to come, consider applying the
substitution, $\widehat{\id{\{}u / z \id{\}}}$, to the following pair
of processes, $\lift{w}{y!(z)}$ and $w[ \lpquote y!(z) \rpquote ]$.

\begin{eqnarray}
	\lift{w}{y!(z)}\widehat{\id{\{}u / z \id{\}}}
		& = &
		\lift{w}{y!(u)} \nonumber\\
	w[ \lpquote y!(z) \rpquote ] \widehat{ \id{\{}u / z \id{\}} }
		& = &
		w[ \lpquote y!(z) \rpquote ] \nonumber
\end{eqnarray}

Because the body of the process between quotes is impervious to
substitution, we get radically different answers. In fact, by
examining the first process in an input context,
e.g. $x?(z).\lift{w}{y!(z)}$, we see that the process under the lift
operator may be shaped by prefixed inputs binding a name inside it. In
this sense, the lift operator will be seen as a way to dynamically
construct processes before reifying them as names.

Finally equipped with these standard features we can present the
dynamics of the calculus.

\subsubsection{Operational semantics} 

Finally, we introduce the computational dynamics. What marks these
algebras as distinct from other more traditionally studied algebraic
structures, e.g. vector spaces or polynomial rings, is the manner in
which dynamics is captured. In traditional structures, dynamics is typically
expressed through morphisms between such structures, as in linear maps
between vector spaces or morphisms between rings. In algebras
associated with the semantics of computation, the dynamics is
expressed as part of the algebraic structure itself, through a
reduction reduction relation typically denoted by $\red$. Below, we
give a recursive presentation of this relation for the calculus used
in the encoding.

$\red \subseteq \pi \times \pi$
$\red : \pi \to \mathcal{P}(\pi)$

\begin{mathpar}
  \inferrule* [lab=Comm] { \textsf{match}( x_{src}, x_{trgt} ) } { x_{trgt}?(y)P \; | \; x_{src}!\langle {Q} \rangle \red P\{\quotep{Q}/y}\} }
  \and \\
  \inferrule* [lab=Par] {{P} \red {P}'} {{{P} | {Q}} \red {{P}' | {Q}}}
  \and
  \inferrule* [lab=Equiv]{{{P} \scong {P}'} \andalso {{P}' \red {Q}'} \andalso {{Q}' \scong {Q}}}{{P} \red {Q}}
\end{mathpar}

\begin{eqnarray*}
  match_{\equiv} (\quotep{P},\quotep{Q}) & := & P \equiv Q \\
  match_{\dagger}(\quotep{P},\quotep{Q}) & := & \forall R. P|Q \red^{*} R => R \red^{*} 0 \\
  match_{K}(\quotep{P},\quotep{Q}) & := & K \mbox{ for some context } K
\end{eqnarray*}

$u?(x)P | u!\langle Q \rangle \red P\{\quotep{Q}/x\}$

%We write $\wred$ for $\red^*$, and $P\red$ if $\exists Q $ such that $ P \red Q$.
We write $P\red$ if $\exists Q $ such that $ P \red Q$ and $P\not\red$, otherwise.

\section{Replication}

As mentioned before, it is known that replication (and hence
recursion) can be implemented in a higher-order process algebra
\cite{SangiorgiWalker}. As our first example of calculation with the
machinery thus far presented we give the construction explicitly in
the {\rhoc}.

\begin{eqnarray}
	D_{x} & := & \prefix{x}{y}{(\binpar{\outputp{x}{y}}{@{y}})} \nonumber\\
	\bangp_{x}{P} & := & \binpar{{x}!\langle{\binpar{D_{x}}{P}}\rangle}{D_{x}} \nonumber
\end{eqnarray}

\begin{eqnarray}
	\bangp_{x}{P} & & \nonumber\\
	=
	& {x}!\langle{(\prefix{x}{y}{(\outputp{x}{y} | @{y})) | P}}\rangle 
	      | \prefix{x}{y}{(\outputp{x}{y} | @{y})} & \nonumber\\
	\red
	& (\outputp{x}{y} | @{y})\substn{\quotep{(\prefix{x}{y}{(@{y} | \outputp{x}{y})) | P}}}{y} & \nonumber\\
	=
	& \outputp{x}{\quotep{(\prefix{x}{y}{(\outputp{x}{y} | @{y})) | P}}}
	  | {(\prefix{x}{y}{(\outputp{x}{y} | @{y})) | P}} & \nonumber\\
	\red
	& \ldots & \nonumber\\
	\red^*
	& P | P | \ldots & \nonumber
\end{eqnarray}

Of course, this encoding, as an implementation, runs away, unfolding
$\bangp{P}$ eagerly. A lazier and more implementable replication
operator, restricted to input-guarded processes, may be obtained as follows.

\begin{eqnarray}
\bangp{\prefix{u}{v}{P}} 
	:= 
	\binpar{\lift{x}{\prefix{u}{v}{(\binpar{D(x)}{P})}}}{D(x)} \nonumber
\end{eqnarray}

\begin{remark}
  Note that the lazier definition still does not deal with summation
  or mixed summation (i.e. sums over input and output). The reader is
  invited to construct definitions of replication that deal with these
  features. 

  Further, the definitions are parameterized in a name, $x$. Can you,
  gentle reader, make a definition that eliminates this parameter and
  guarantees no accidental interaction between the replication
  machinery and the process being replicated -- i.e. no accidental
  sharing of names used by the process to get its work done and the
  name(s) used by the replication to effect copying. This latter
  revision of the definition of replication is crucial to obtaining
  the expected identity $!!P \sim !P$.
\end{remark}

\begin{remark}\label{rem:paradoxical_combinator}
  The reader familiar with the lambda calculus will have noticed the
  similarity between $D$ and the paradoxical combinator.

  [Ed. note: the existence of this seems to suggest we have to be more
  restrictive on the set of processes and names we admit if we are to
  support no-cloning.]
\end{remark}

\subsubsection{Bisimulation}

The computational dynamics gives rise to another kind of equivalence,
the equivalence of computational behavior. As previously mentioned
this is typically captured \emph{via} some form of bisimulation.

% The notion we use in this paper is weak barbed bisimulation
% \cite{milner91polyadicpi}.

The notion we use in this paper is derived from weak barbed
bisimulation \cite{milner91polyadicpi}. 

\begin{definition}
An \emph{observation relation}, $\downarrow_{\mathcal N}$, over a set
of names, $\mathcal N$, is the smallest relation satisfying the rules
below.

\infrule[Out-barb]{y \in {\mathcal N}, \; x \nameeq y}
		  {\outputp{x}{v} \downarrow_{\mathcal N} x}
\infrule[Par-barb]{\mbox{$P\downarrow_{\mathcal N} x$ or $Q\downarrow_{\mathcal N} x$}}
		  {\binpar{P}{Q} \downarrow_{\mathcal N} x}

We write $P \Downarrow_{\mathcal N} x$ if there is $Q$ such that 
$P \wred Q$ and $Q \downarrow_{\mathcal N} x$.
\end{definition}

\begin{definition}
%\label{def.bbisim}
An  ${\mathcal N}$-\emph{barbed bisimulation} over a set of names, ${\mathcal N}$, is a symmetric binary relation 
${\mathcal S}_{\mathcal N}$ between agents such that $P\rel{S}_{\mathcal N}Q$ implies:
\begin{enumerate}
\item If $P \red P'$ then $Q \wred Q'$ and $P'\rel{S}_{\mathcal N} Q'$.
\item If $P\downarrow_{\mathcal N} x$, then $Q\Downarrow_{\mathcal N} x$.
\end{enumerate}
$P$ is ${\mathcal N}$-barbed bisimilar to $Q$, written
$P \wbbisim_{\mathcal N} Q$, if $P \rel{S}_{\mathcal N} Q$ for some ${\mathcal N}$-barbed bisimulation ${\mathcal S}_{\mathcal N}$.
\end{definition}

$\mathcal{R} \subseteq \pi \times \pi$

$P \mathcal{R} Q => \forall P'. P \red P' \Rightarrow \exists Q'. Q \red Q', P' \mathcal{R} Q'$

$P \vdash x \Rightarrow Q \vdash x$

\begin{mathpar}
  \inferrule*[lab=Out-barb]{x \nameeq y}{{y}!\langle{Q}\rangle \vdash x}
  \and
  \inferrule*[lab=Par-barb]{\mbox{$P\vdash x$ or $Q\vdash x$}}{\binpar{P}{Q} \vdash x}
\end{mathpar}

\subsubsection{Contexts}

One of the principle advantages of computational calculi like the
$\pi$-calculus is a well-defined notion of context,
contextual-equivalence and a correlation between
contextual-equivalence and notions of bisimulation. The notion of
context allows the decomposition of a process into (sub-)process and
its syntactic environment, its context. Thus, a context may be
thought of as a process with a ``hole'' (written $\Box$) in it. The
application of a context $M$ to a process $P$, written $M[P]$, is
tantamount to filling the hole in $M$ with $P$. In this paper we do
not need the full weight of this theory, but do make use of the notion
of context in the proof the main theorem. 

\begin{mathpar}
  \inferrule* [lab=summation] {} {{M_{M},M_{N}} \bc \Box \;|\; x.M_{A} \;|\; M_{M}+M_{N}}
  \and
  \inferrule* [lab=agent] {} {{M_{A}} \bc (\vec{x})M_{P} \;| \; \clift{P_0,\ldots,M_{P},\ldots,P_N}}
  \and \\
  \inferrule* [lab=process] {} {{M_{P}} \bc M_{N} \;| \;P|M_{P} }
\end{mathpar} 

\begin{mathpar}
  \inferrule* [lab=sychronization] {} {M_{N} \bc \Box \;|\; x?M_{F} \;|\; x!M_{C}}
  \and
  \inferrule* [lab=abstraction] {} {{M_{F}} \bc (x)M_{P} }
  \and
  \inferrule* [lab=concretion] {} {{M_{C}} \bc \langle M_{P} \rangle }
  \and \\
  \inferrule* [lab=process] {} {{M_{P}} \bc M_{N} \;| \;P|M_{P} }
\end{mathpar}

\begin{definition}[contextual application] Given a context $M$, and
  process $P$, we define the \emph{contextual application}, $M[P] :=
  M\{P/\Box\}$. That is, the contextual application of M to P is the
  substitution of $P$ for $\Box$ in $M$.
\end{definition}

$\meaningof{-} : L \to \mathcal{P}(\pi)$

\begin{mathpar}
  \inferrule* [lab=collection] {} {\meaningof{true} = \pi, \and \meaningof{~E} = \pi \setminus \meaningof{E}, \and \meaningof{E_{1} \& E_{2}} = \meaningof{E_{1}} \cap \meaningof{E_{2}}}
\end{mathpar}

\begin{mathpar}
  \inferrule* [lab=structure] {} {\meaningof{0} = \{ P \in \pi | P \equiv 0 \}, \and \\ \meaningof{E_1 | E_2} = \{ P \in \pi | P \equiv P_{1} | P_{2}, P_{1} \in \meaningof{E_{1}}, P_{2} \in \meaningof{E_2}\} }
\end{mathpar}

\begin{mathpar}
 \inferrule* [lab=behavior] {} {\meaningof{\langle a?b \rangle E} = \{ P \in \pi | P \equiv Q | u?(y)P', \\ \and \\\\ \and \\ \;\;\; u \in \meaningof{a}, \forall z.P'\{z/y\} \in \meaningof{E\{z/b\}}\}, \and \\ \meaningof{a!E} = \{ P \in \pi | P \equiv Q | x!\langle P' \rangle, x \in \meaningof{a} P' \in \meaningof{E}\} }
\end{mathpar}

\begin{mathpar}
 \inferrule* [lab=nominal] {} {\meaningof{\quotep{E}} = \{ \quotep{P} \in \quotep{\pi} | P \in \meaningof{E} \}, \and \meaningof{\quotep{P}} = \{ \quotep{Q} \in \quotep{\pi} | P \equiv Q \} \and \\ \meaningof{@\quotep{E}} = \{ P \in \pi | P \equiv @x, x \in \meaningof{E} \}}
\end{mathpar}

\begin{eqnarray*}
  \\
  \meaningof{-} : TS \to ST
\end{eqnarray*}

\begin{eqnarray*}
  \\
  L : TS \to ST
\end{eqnarray*}

\begin{eqnarray*}
  \\
  P \models E \iff P \in \meaningof{E}
\end{eqnarray*}

\begin{eqnarray*}
  P \approx_{L} Q \iff \forall E \in L. P \models E \iff Q \models E
\end{eqnarray*}

\begin{eqnarray*}
  P \approx_{K} Q
\end{eqnarray*}

\begin{eqnarray*}
  P \approx Q
\end{eqnarray*}

$\approx_{K} = \approx = \approx_{L}$

\subsubsection{Contextual duality}

Note that contexts extend the quotation operation to a family of
operations from processes to names. Given a context, $M$, we can
define a \emph{nominal context}, $\quotep{M}$ by $\quotep{M}[P] :=
\quotep{M[P]}$. To foreshadow what is to come we observe that these
operations enjoy a duality with processes very much like the duality
between vectors and maps from vectors to scalars.

Further, because the calculus is essentially higher-order, we have a
correspondence between contexts and processes. More specifically,
given a name $x$ and a context $M$ we can construct $M^{*}_{x}$ such
that 

\begin{mathpar}
  M^{*}_{x} | \lift{x}{P} \red M[P]
\end{mathpar}

namely,

\begin{mathpar}
  M^{*}_{x} := x?(u).M[\dropn{u}]
\end{mathpar}

The dependence of $M^{*}_{x}$ on a name makes it an abstraction, 

\begin{mathpar}
  M^{*} := (x)x?(u).M[\dropn{u}]
\end{mathpar}

\subsection{Additional notation}

It will sometimes be convenient to denote the process a name
quotes. We already have the notation $x = \quotep{P}$, but it will be
convenient to introduce an alternate notation, $\procn{x}$, when we
want to emphasize the connection to the use of the name. Note that, by
virtue of name equivalence, $\quotep{\procn{x}} \nameeq x$; so, the
notation is consistent with previous definitions.

Further, because names have structure it is possible to effect
substitutions on the basis of that structure. This means we need to
upgrade our notation for substitutions, which we accomplish by
adapting comprehension notation. Thus,

\begin{mathpar}
  P\{ y / x : x \in S \}
\end{mathpar}

is interpreted to mean the process derived from P by replacing (in a
capture-avoiding manner) each occurrence of $x$ in $S$ by $y$. For example,

\begin{mathpar}
  P\{ \quotep{\procn{x}|\procn{x}} / x : x \in \freenames{P} \}
\end{mathpar}

will replace each (occurrence) of a free name $x$ in $P$ by
$\quotep{\procn{x}|\procn{x}}$.

Also, we will avail ourselves of the notation $x^{L}$ and $x^{R}$ to
denote injections of a name into disjoint copies of the name
space. There are numerous ways to accomplish this. One example can be
found in \cite{MeredithR05}. This notation overloads to vectors of
names: $\vec{x}^{\pi} := (x_{i}^{\pi} \; : \; 0 \leq i < |\vec{x}| )$ where $\pi \in \{L,R\}$.

We also use $P^{\Box} := P|\Box$.

In \cite{MeredithR05} an interpretation of the new operator is
given. It turns out that there are several possible interpretations
all enjoying the requisite algebraic properties of the operator (see
\cite{milner91polyadicpi}). We will therefore make liberal use of
$(\nu\; \vec{x})P$.

% subsection the_syntax_and_semantics_of_the_notation_system (end)   

\section{Interpretation of QM}
\subsection{Supporting definitions}
\subsubsection{Multiplication}
\begin{mathpar}
  \quotep{Q} \cdot \quotep{R} := \quotep{Q|R}
  \and \\
  \quotep{Q} \cdot P := P\{ \quotep{Q|R} / \quotep{R} : \quotep{R} \in \freenames{P} \}
\end{mathpar}

\paragraph{Discussion}
The first line needs little explanation. The second line says that
each free name of the process is replaced with the multiplication of
that name by the scalar. Multiplication of a scalar (name) by a state
(process) results in a process all the names of which have been `moved
over' by parallel composition with the process the scalar
quotes. There is a subtlety that the bound names have to be
manipulated so that multiplied names aren't accidentally
captured. There are many ways to achieve this.

\begin{remark}\label{rem:multiplication_identities}
  The reader is invited to verify that for all $x,y,z \in \QProc$ and $P \in \Proc$
  \begin{mathpar}
    x \cdot \quotep{0} \equiv x 
    \and
    x \cdot y \equiv y \cdot x
    \and
    x \cdot (y \cdot z) \equiv (x \cdot y) \cdot z
    \and \\
    \quotep{0} \cdot P \equiv P
    \and \\
    x \cdot (y \cdot P) \equiv (x \cdot y) \cdot P
    \and \\
    x \cdot (P|Q) \equiv (x \cdot P) | (x \cdot Q)
    \and \\    
  \end{mathpar}
\end{remark}

\subsubsection{Tensor product}

We define a tensor product on processes by structural induction.

\paragraph{Tensor of sums} First note that all summations, including
$\pzero$ and sequence, can be written $\Sigma_{i} x_{i}.A_{i} +
\Sigma_{j} x_{j}.C_{j}$, where we have grouped input-guarded processes
together and output-guarded processes together.

Thus, we can define the tensor product of two summations, $N_{1}\otimes N_{2}$, where

\begin{mathpar}
  N_{1} := \Sigma_{i} x_{i}.A_{i} + \Sigma_{j} x_{j}.C_{j}
  \and
  N_{2} := \Sigma_{i'} y_{i'}.B_{i'} + \Sigma_{j'} y_{j'}.D_{j'} 
\end{mathpar}

as follows.

\begin{mathpar}
  \Sigma_{i} x_{i}.A_{i} + \Sigma_{j} x_{j}.C_{j} \otimes \Sigma_{i'}
  y_{i'}.B_{i'} + \Sigma_{j'} y_{j'}.D_{j'} 
  \and \\
  := \; \Sigma_{i} \Sigma_{i'} \quotep{\stackrel{\vee}{x_{i}}| \stackrel{\vee}{y_{i'}}}.(A_{i}\otimes B_{i'}) \; | \; \Sigma_{i'} \Sigma_{i} \quotep{\stackrel{\vee}{y_{i'}}|\stackrel{\vee}{x_{i}}}.(B_{i'}\otimes A_{i})
  \and
  \;\; | \;\; \Sigma_{j} \Sigma_{j'} \quotep{\stackrel{\vee}{x_{j}}|\stackrel{\vee}{y_{j'}}}.(A_{j}\otimes B_{j'}) \; | \; \Sigma_{j'} \Sigma_{j} \quotep{\stackrel{\vee}{y_{j'}}|\stackrel{\vee}{x_{j}}}.(B_{j'}\otimes A_{j})
\end{mathpar}

\begin{remark}
  Do we need to $x^{L}$ and $y^{R}$ for this construction as well?
\end{remark}

\paragraph{Tensor of parallel compositions} Next, we distribute tensor
over par.

\begin{mathpar}
  P_{1}|P_{2} \otimes Q_{1}|Q_{2} := (P_{1} \otimes Q_{1}) | (P_{1}
  \otimes Q_{2}) | (P_{2} \otimes Q_{1}) | (P_{2} \otimes Q_{2})
\end{mathpar}

\paragraph{Tensor with dropped names} We treat tensor of a
process with a dropped name as parallel composition.

\begin{mathpar}
  P \otimes \dropn{x} := P | \dropn{x}
\end{mathpar}

\paragraph{Tensor of agents}

Finally, we need to define tensor on agents. Note that the definition
of tensor on normal products only tensors inputs with inputs and
outputs with outputs. Thus, we only have to define the operation on
``homogeneous'' pairings.

\begin{mathpar}
  (\vec{x})P \otimes (\vec{y})Q
  \and \\
  := (x_{0}^{L}|y_{0}^{R},\ldots,x_{0}^{L}|y_{n}^{R},\ldots,x_{m}^{L}|y_{0}^{R},\ldots,x_{m}^{L}|y_{n}^R)(P\{ \vec{x}^{L}/\vec{x}\} \otimes Q \{ \vec{y}^{R}/\vec{y}\})
  \and \\
  \clift{\vec{P}} \otimes \clift{\vec{Q}}
  \and \\
  := \clift{P_{0}\otimes Q_{0},\ldots,P_{0}\otimes Q_{n},\ldots,P_{m}\otimes Q_{0},\ldots,P_{m}\otimes Q_{n}}
\end{mathpar}

\begin{remark}
  Observe that arities of tensored abstractions matches arities of
  tensored concretions if the original arities matched. Note also that
  the length of the arities corresponds to the increase in dimension
  we see in ordinary vector space tensor product.
\end{remark}

\begin{remark}
  Operationally, this definition distributes the tensor down to
  components ``linked'' by summation. Tensor over summation is
  intriguing in that it mixes names. Moreover, as a consequence of the
  way it mixes names we have the identities for all $x \in \QProc$ and
  $P,Q \in \Proc$

  \begin{mathpar}
    (x \cdot P) \otimes Q \equiv x \cdot (P \otimes Q) \equiv P \otimes (x \cdot Q)
    \and
    P \otimes \pzero \equiv P
  \end{mathpar}

  that the reader is invited to verify.
\end{remark}

\subsubsection{Annihilation}
\begin{mathpar}
  P^{\perp} := \{ Q | \forall R. P|Q \red^{*} R \Rightarrow R \red^{*} \pzero \}
  \and \\
  P^{\underline{\perp}} := \Sigma_{Q \in P^{\perp}} \quotep{Q}?(y).(\dropn{y}|Q) | \Sigma_{Q \in P^{\perp}} \quotep{Q}\clift{\Box}
\end{mathpar}

\paragraph{Discussion} The reader will note that $P^{\perp}$ is a
\emph{set} of processes, while $P^{\underline{\perp}}$ is a
\emph{context}. We call the set $P^{\perp}$ the \emph{annihilators} of
$P$. The parallel composition of a process in the annihilators of $P$
with $P$ will result in a process, the state space of which has all
paths eventually leading to $\pzero$. Execution may endure loops; but
under reasonable conditions of fairness (naturally guaranteed under
most notions of bisimulation) such a composite process cannot get
stuck in such a loop and will, eventually pop out and terminate.

The context $P^{\underline{\perp}}$ is ready and willing to ``take the
$P$ out of'' the process to which it is applied. It will effectively
transmit the code of the process to which it is applied to one of the
annihilators and run the process against it.

\subsubsection{Evaluation}
We fix $M$ a domain of fully abstract interpretation with an equality
coincident with bisimulation. We take $\meaningof{\cdot} : \Proc \to
M$ to be the map interpreting processes and $\nmeaningof{\cdot} : \M
\to Proc$ to be the map running the other way. Then we define

\begin{mathpar}
  \int P := \nmeaningof{\meaningof{P}}
\end{mathpar}

\paragraph{Discussion}
There are many fully abstract interpretations of Milner's
$\pi$-calculus. Any of them can be used as a basis for interpreting
the reflective calculus here. Equipped with such a domain it is
largely a matter of grinding through to check that the Yoneda
construction for the normalization-by-evaluation program can be
extended to this setting.

\begin{remark}
  The reader is invited to verify that $\int (P^{\underline{\perp}}[P]) = 0$.
\end{remark}

\subsection{Quantum mechanics}

Table \ref{tbl:core_qm_op_defns} gives the core operational definitions

\begin{table}[htp]\label{tbl:core_qm_op_defns}
  \center{
    \fbox{
      \begin{tabular}{c|c}
        quantum mechanics & process calculus \\
        \hline
        scalar & $x := \quotep{P}$ \\
        state vector & $\state{P} := P$ \\
        dual & $\state{P}^{*} := \event{P^{\underline{\perp}}} := \quotep{P^{\underline{\perp}}}[-]$ \\
        matrix & $ \Sigma_{\alpha} \state{P_{\alpha}}x_{\alpha}\event{Q_{\alpha}}$ \\
        vector addition & $\state{P} + \state{Q} := \state{P | Q}$ \\
        tensor product & $\state{P} \otimes \state{Q} := \state{P \otimes Q}$ \\
        inner product & $\innerprod{P}{Q} := \quotep{\int P^{\underline{\perp}}[Q]}$ \\
      \end{tabular}
    }
  }
  \caption{QM - operational definitions}
\end{table}

where

\begin{mathpar}
  \prmatrix{P}{Q} := \fprmatrix{P}{\quotep{\pzero}}{Q}
  \and
  \fprmatrix{P}{x}{Q} := (\state{P},x,\event{Q})
  \and
  (\fprmatrix{P}{x}{Q})(\state{R}) := x \cdot \innerprod{Q}{R} \cdot \state{P}
  \and
  (\fprmatrix{P}{x}{Q})(\event{R}) := x \cdot \innerprod{R}{P} \cdot \event{Q}
\end{mathpar}

\paragraph{Discussion}
As promised: vectors (aka states) are represented as processes; duals
as contextual duals; inner product definition should be compared with
standard inner product definition for ....

\begin{remark}
  Assuming $\int (P^{\underline{\perp}}[P]) = 0$, the reader is
  invited to verify that $(\fprmatrix{P}{x}{P})(\state{P}) = x \cdot \state{P}$.
\end{remark}

\begin{remark}
  The reader is invited to verify that $\innerprod{P}{Q}$ could
  equally well have been written $\quotep{\int \stackrel{\vee}{x}}$
  where $x = \event{P^{\underline{\perp}}}(Q)$.

  One of the motivations for this remark is that there is another way
  to factor these operations. We could package up evaluation in the dual:

  \begin{mathpar}
    \state{P}^{*} := \event{\int P^{\underline{\perp}}} := \quotep{\int P^{\underline{\perp}}}[-]
  \end{mathpar}

  and then have inner product defined by
  
  \begin{mathpar}
    \innerprod{P}{Q} := \event{P}(Q)
  \end{mathpar}

  Hopefully, experience with the calculations will provide guidance on
  the best factoring.
\end{remark}

\begin{remark}
  Assuming $\int (P^{\underline{\perp}}[P]) = 0$, the reader is
  invited to verify that $\forall P,Q. (\prmatrix{0}{Q})(\state{0}) =
  \state{0}$ and dually $(\prmatrix{P}{0})(\event{0}) = \event{0}$.
\end{remark}

\begin{remark}
  i'm a little worried that i don't (yet) have proper support for
  complex conjugacy. But, the observation above may give us a
  clue. According to Abramsky, it must be the case that the scalars
  are iso to the homset of the identity for the tensor -- which the
  observation above characterizes. 

  For now, we will simply bookmark the notion with $\overline{x}$.
\end{remark}

\subsubsection{Adjointness}

We need to give a definition of $(\cdot)^{\dagger}$ for matrices. The
obvious candidate definition is
\begin{mathpar}
(\Sigma_{\alpha}\fprmatrix{P_{\alpha}}{x_{\alpha}}{Q_{\alpha}})^{\dagger}
= \Sigma_{\alpha}\fprmatrix{(Q_{\alpha}^{\underline{\perp}})^{*}}{\overline{x}_{\alpha}}{P_{\alpha}^{\underline{\perp}}} 
\end{mathpar}

But, $(Q_{\alpha}^{\underline{\perp}})^{*}$ requires a name along
which to communicate the process to achieve the context application.

\subsubsection{Basis for a basis}
If processes label states and ``addition'' of states (a.k.a. vector
addition) is interpreted as parallel composition, what corresponds to
notions of linear independence and basis? Here, we recall that Yoshida
has developed a set of \emph{combinators} for an asynchronous verison
of Milner's $\pi$-calculus. These are a finite set of processes such
any process can be expressed as parallel composition of these
combinators together with liberal uses of the new operator and
replication. We can simply give a translation of these into the
present calculus and have reasonable expectation that the property
carries over. That is, that the resultant set allows to express all
processes via parallel composition. Note, however, that there is no
new operator or replication in this calculus. As a result, we expect
that the corresponding set is actually infinite. That is, we expect
that the space is actually infinite dimensional.

\begin{remark}
  The attentive reader may be a bit concerned. Certainly, the
  collection $S$, $K$ and $I$ is a finite set of
  combinators. Shouldn't we expect to see a finite set of combinators
  for an effectively equivalent system? i am very sympathetic to this
  critique and feel it warrants full attention. On the other hand, i
  also have in mind the following analogy. The natural numbers, as a
  monoid under addition, has exactly $1$ generator, while the natural
  numbers, as a monoid under multiplication, has countably many
  generators (the primes). We observe that the application of the
  lambda calculus is much less resource sensitive than the parallel
  composition of the $\pi$-calculus. Could it be the case that we have
  an analogy of the form
  
  \begin{mathpar}
    m + n : MN :: m*n : M|N
  \end{mathpar}

  giving a similar blow up in the set of ``primes''?  This is such a
  wonderful thought that, even if it's not true, i think it's worth
  writing down.
\end{remark}
 

\documentclass[12pt]{llncs}
%\documentclass{jktr}

\usepackage[pdftex]{hyperref}                   
\usepackage {listings}
\usepackage {mathpartir}
\usepackage{bcprules}
%\usepackage{listings}
                       
\usepackage{graphicx} 
%\usepackage[margins=2.5cm,nohead,nofoot]{geometry}
%\usepackage{geometry}
\usepackage{amsfonts}
\usepackage{amstext}
\usepackage{latexsym}
\usepackage{amssymb}
\usepackage{color}


%\include{myPreamble}
\include{qm2pi.local} 

%\ifpdf
%\usepackage[pdftex]{graphicx}
%\else
%\usepackage{graphicx}
%\fi

 % \ifpdf
%  \usepackage{pdfsync}
%  \if


%\title{Brief Article}
%\author{David F. Snyder}
%\author{L.G. Meredith}

%\address{Dept. of Math., Texas State University--San Marcos, San Marcos, TX 78666}
       
\pagestyle{empty}


\begin{document}

\lstset{language=[Objective]Caml,frame=shadowbox}

\input{qm2pi.front}

% section front matter (end)

\input{qm2pi.intro} 
 
% section introduction (end)

% \input{qm2pi.knotations} 

% section notation (end)

\input{qm2pi.process.calculi} 

% section concurrent_process_calculi_and_spatial_logics_ (end)
    
%\input{qm2pi.knots2pi} 

%\input{qm2pi.trefoil} 

%\input{qm2pi.mainthm} 

% subsection basic_interpretation (end)

%\input{qm2pi.rho.presentation} 
\subsection{The syntax and semantics of the notation system}\label{sub:the_syntax_and_semantics_of_the_notation_system} % (fold)

We now summarize a technical presentation of the calculus that
embodies our theory of dynamics. The typical presentation of such a
calculus follows the style of giving generators and relations on
them. The grammar, below, describing term constructors, freely
generates the set of processes, $\Proc$. This set is then quotiented
by a relation known as structural congruence and it is over this set
that the notion of dynamics is expressed. This presentation is
essentially that of \cite{MeredithR05} with the addition of
polyadicity and summation. For readability we have relegated some of
the technical subtleties to an appendix.

\subsubsection{Process grammar}\label{subsub:process_grammar}

\begin{mathpar}
  \inferrule* [lab=synchronization] {} {{M} \bc \pzero \;|\; x?F \;|\; x!C }
  \and
  \inferrule* [lab=abstraction] {} {{F} \bc (x)P}
  \and
  \inferrule* [lab=concretion] {} {{C} \bc \langle Q \rangle}
  \and
  \inferrule* [lab=process] {} {{P,Q} \bc M \;| \;P|Q \;|\; @{x}}
  \and
  \inferrule* [lab=name] {} {{x} \bc \quotep{P}}
\end{mathpar} 

Note that $\vec{x}$ (resp. $\vec{P}$) denotes a vector of names
(resp. processes) of length $|\vec{x}|$ (resp. $|\vec{P}|$). We adopt
the following useful abbreviations.

\begin{mathpar}
   x?(\vec{y}).P := x.(\vec{y})P \and  x\clift{\vec{P}} := x.\clift{\vec{P}}
   \and x!(y) := \lift{x}{\dropn{y}}
   \and \Pi_{i=0}^{n-1}P_i := P_0 | \ldots | P_{n-1}
\end{mathpar}

\subsubsection{Structural congruence}

\paragraph{Free and bound names and alpha-equivalence.} At the
core of structural equivalence is alpha-equivalence which identifies
process that are the same up to a change of variable. Formally, we
recognize the distinction between free and bound names. The free names
of a process, $\freenames{P}$, may be calculated recursively as
follows:

\begin{mathpar}
\freenames{\pzero} := \emptyset
  \and \\
  \freenames{x?(y).P} := \{ x \} \cup (\freenames{P} \setminus \{ y \})
  \and 
  \freenames{x!\langle P \rangle} := \{ x \} \cup \{ P \} 
  \and \\
  \freenames{P|Q} := \freenames{P} \cup \freenames{Q}
  \and \\
  \freenames{@{x}} := \{ x \}
\end{mathpar}

$\pi$
$\quotep{\pi}$

$\freenames{-} : \pi \to \mathcal{P}(\quotep{\pi})$

\begin{eqnarray*}
  \freenames{\pzero} & := & \emptyset \\
  \freenames{x?(y).P} & := & \{ x \} \cup (\freenames{P} \setminus \{ y \}) \\
  \freenames{x!\langle P \rangle} & := & \{ x \} \cup \{ P \} \\
  \freenames{P|Q} & := & \freenames{P} \cup \freenames{Q} \\
  \freenames{\dropn{x}} & := & \{ x \}
\end{eqnarray*}

The bound names of a process, $\boundnames{P}$, are those names occurring in $P$
that are not free. For example, in $x?(y).0$, the name $x$ is free, while $y$ is bound.

\begin{mathpar}
  \inferrule* [lab=monoidal-laws] {} { P|Q \equiv Q|P \and P|0 \equiv P \and P|(Q|R) \equiv (P|Q)|R }
\end{mathpar}

\begin{mathpar}
  \inferrule* [lab=alpha-equivalence] {} { (x)P \equiv (y)P\{y/x\} \and y \not\in \freenames{P} }
\end{mathpar}

\begin{definition}
Then two processes, $P,Q$, are alpha-equivalent if $P = Q\{\vec{y}/\vec{x}\}$ for
some $\vec{x} \in \boundnames{Q},\vec{y} \in \boundnames{P}$, where $Q\{\vec{y}/\vec{x}\}$
denotes the capture-avoiding substitution of $\vec{y}$ for $\vec{x}$ in $Q$.
\end{definition}

\begin{definition}
  The {\em structural congruence} \cite{SangiorgiWalker} , $\equiv$,
  between processes is the least congruence containing
  alpha-equivalence, satisfying the abelian monoid laws
  (associativity, commutativity and $\pzero$ as identity) for parallel
  composition $|$ and for summation $+$.
\end{definition}

\subsection{Name equivalence}

We take name equivalence, written $\nameeq$, to be the smallest
equivalence relation generated by the following rules.

\begin{mathpar}
\inferrule*[lab=Quote-drop]
{ }
{ \quotep{@{x}} \nameeq x }

\inferrule*[lab=Struct-equiv]
{ P \scong Q }
{ \quotep{P} \nameeq \quotep{Q} }
\end{mathpar}

The astute reader will have noticed that the mutual recursion of names
and processes imposes a mutual recursion on alpha-equivalence and
structural equivalence via name-equivalence. Fortunately, all of this
works out pleasantly and we may calculate in the natural way, free of
concern. The reader interested in the details is referred to the
appendix \ref{appendix:rho_details}.

\subsection{Substitution}

We use $\Proc$ for the set of processes, $\QProc$ for the set of
names, and $\id{\{}\vec{y} / \vec{x} \id{\}}$ to denote partial maps,
$s : \QProc \rightarrow \QProc$. A map, $s$ lifts, uniquely, to a map
on process terms, $\widehat{s} : \Proc \rightarrow \Proc$ by the
following equations.

\begin{mathpar}
  (0) \psubstp{Q}{P} := 0 \\
  (R \juxtap S) \psubstp{Q}{P}
  :=    
  (R)\psubstp{Q}{P} \juxtap (S) \psubstp{Q}{P} \\
  (x?(y).R) \psubstp{Q}{P}    
  :=    
  (x)\substp{Q}{P} (z)\concat( (R \psubstn{z}{y}) \psubstp{Q}{P} ) \\
  (\lift{x}{R}) \psubstp{Q}{P}  
  :=
  \lift{(x)\substp{Q}{P}}{ R \psubstp{Q}{P} } \\
%   (\dropn{x})  \psubstp{Q}{P}       
%   := 
%   \left\{ 
%     \begin{array}{ccc} 
%       \dropn{\quotep{Q}} & & x \nameeq \quotep{P} \\
%       \dropn{x} & & otherwise \\
%     \end{array}
%   \right. 
  (\dropn{x})  \psubstp{Q}{P}       
  := 
  \left\{ 
    \begin{array}{ccc} 
      Q & & x \nameeq \quotep{P} \\
      \dropn{x} & & otherwise \\
    \end{array}
  \right.
\end{mathpar}
 

where

\begin{eqnarray}
  (x)\id{\{} \lpquote Q \rpquote / \lpquote P \rpquote \id{\}}            = 
  \left\{ 
    \begin{array}{ccc}
      \lpquote Q \rpquote & & x \nameeq \lpquote P \rpquote \\
      x & & otherwise \\
    \end{array}
  \right. \nonumber
\end{eqnarray}

and $z$ is chosen distinct from $\quotep{P}$, $\quotep{Q}$, the free
names in $Q$, and all the names in $R$. Our $\alpha$-equivalence will
be built in the standard way from this substitution.

\begin{remark}\label{rem:no_self_referential_names}
  One consequence of these definitions is that $\forall P. \quotep{P}
  \not\in \freenames{P}$.
\end{remark}

\subsection{ Dynamic quote: an example }

Anticipating something of what's to come, consider applying the
substitution, $\widehat{\id{\{}u / z \id{\}}}$, to the following pair
of processes, $\lift{w}{y!(z)}$ and $w[ \lpquote y!(z) \rpquote ]$.

\begin{eqnarray}
	\lift{w}{y!(z)}\widehat{\id{\{}u / z \id{\}}}
		& = &
		\lift{w}{y!(u)} \nonumber\\
	w[ \lpquote y!(z) \rpquote ] \widehat{ \id{\{}u / z \id{\}} }
		& = &
		w[ \lpquote y!(z) \rpquote ] \nonumber
\end{eqnarray}

Because the body of the process between quotes is impervious to
substitution, we get radically different answers. In fact, by
examining the first process in an input context,
e.g. $x?(z).\lift{w}{y!(z)}$, we see that the process under the lift
operator may be shaped by prefixed inputs binding a name inside it. In
this sense, the lift operator will be seen as a way to dynamically
construct processes before reifying them as names.

Finally equipped with these standard features we can present the
dynamics of the calculus.

\subsubsection{Operational semantics} 

Finally, we introduce the computational dynamics. What marks these
algebras as distinct from other more traditionally studied algebraic
structures, e.g. vector spaces or polynomial rings, is the manner in
which dynamics is captured. In traditional structures, dynamics is typically
expressed through morphisms between such structures, as in linear maps
between vector spaces or morphisms between rings. In algebras
associated with the semantics of computation, the dynamics is
expressed as part of the algebraic structure itself, through a
reduction reduction relation typically denoted by $\red$. Below, we
give a recursive presentation of this relation for the calculus used
in the encoding.

$\red \subseteq \pi \times \pi$
$\red : \pi \to \mathcal{P}(\pi)$

\begin{mathpar}
  \inferrule* [lab=Comm] { \textsf{match}( x_{src}, x_{trgt} ) } { x_{trgt}?(y)P \; | \; x_{src}!\langle {Q} \rangle \red P\{\quotep{Q}/y}\} }
  \and \\
  \inferrule* [lab=Par] {{P} \red {P}'} {{{P} | {Q}} \red {{P}' | {Q}}}
  \and
  \inferrule* [lab=Equiv]{{{P} \scong {P}'} \andalso {{P}' \red {Q}'} \andalso {{Q}' \scong {Q}}}{{P} \red {Q}}
\end{mathpar}

\begin{eqnarray*}
  match_{\equiv} (\quotep{P},\quotep{Q}) & := & P \equiv Q \\
  match_{\dagger}(\quotep{P},\quotep{Q}) & := & \forall R. P|Q \red^{*} R => R \red^{*} 0 \\
  match_{K}(\quotep{P},\quotep{Q}) & := & K \mbox{ for some context } K
\end{eqnarray*}

$u?(x)P | u!\langle Q \rangle \red P\{\quotep{Q}/x\}$

%We write $\wred$ for $\red^*$, and $P\red$ if $\exists Q $ such that $ P \red Q$.
We write $P\red$ if $\exists Q $ such that $ P \red Q$ and $P\not\red$, otherwise.

\section{Replication}

As mentioned before, it is known that replication (and hence
recursion) can be implemented in a higher-order process algebra
\cite{SangiorgiWalker}. As our first example of calculation with the
machinery thus far presented we give the construction explicitly in
the {\rhoc}.

\begin{eqnarray}
	D_{x} & := & \prefix{x}{y}{(\binpar{\outputp{x}{y}}{@{y}})} \nonumber\\
	\bangp_{x}{P} & := & \binpar{{x}!\langle{\binpar{D_{x}}{P}}\rangle}{D_{x}} \nonumber
\end{eqnarray}

\begin{eqnarray}
	\bangp_{x}{P} & & \nonumber\\
	=
	& {x}!\langle{(\prefix{x}{y}{(\outputp{x}{y} | @{y})) | P}}\rangle 
	      | \prefix{x}{y}{(\outputp{x}{y} | @{y})} & \nonumber\\
	\red
	& (\outputp{x}{y} | @{y})\substn{\quotep{(\prefix{x}{y}{(@{y} | \outputp{x}{y})) | P}}}{y} & \nonumber\\
	=
	& \outputp{x}{\quotep{(\prefix{x}{y}{(\outputp{x}{y} | @{y})) | P}}}
	  | {(\prefix{x}{y}{(\outputp{x}{y} | @{y})) | P}} & \nonumber\\
	\red
	& \ldots & \nonumber\\
	\red^*
	& P | P | \ldots & \nonumber
\end{eqnarray}

Of course, this encoding, as an implementation, runs away, unfolding
$\bangp{P}$ eagerly. A lazier and more implementable replication
operator, restricted to input-guarded processes, may be obtained as follows.

\begin{eqnarray}
\bangp{\prefix{u}{v}{P}} 
	:= 
	\binpar{\lift{x}{\prefix{u}{v}{(\binpar{D(x)}{P})}}}{D(x)} \nonumber
\end{eqnarray}

\begin{remark}
  Note that the lazier definition still does not deal with summation
  or mixed summation (i.e. sums over input and output). The reader is
  invited to construct definitions of replication that deal with these
  features. 

  Further, the definitions are parameterized in a name, $x$. Can you,
  gentle reader, make a definition that eliminates this parameter and
  guarantees no accidental interaction between the replication
  machinery and the process being replicated -- i.e. no accidental
  sharing of names used by the process to get its work done and the
  name(s) used by the replication to effect copying. This latter
  revision of the definition of replication is crucial to obtaining
  the expected identity $!!P \sim !P$.
\end{remark}

\begin{remark}\label{rem:paradoxical_combinator}
  The reader familiar with the lambda calculus will have noticed the
  similarity between $D$ and the paradoxical combinator.

  [Ed. note: the existence of this seems to suggest we have to be more
  restrictive on the set of processes and names we admit if we are to
  support no-cloning.]
\end{remark}

\subsubsection{Bisimulation}

The computational dynamics gives rise to another kind of equivalence,
the equivalence of computational behavior. As previously mentioned
this is typically captured \emph{via} some form of bisimulation.

% The notion we use in this paper is weak barbed bisimulation
% \cite{milner91polyadicpi}.

The notion we use in this paper is derived from weak barbed
bisimulation \cite{milner91polyadicpi}. 

\begin{definition}
An \emph{observation relation}, $\downarrow_{\mathcal N}$, over a set
of names, $\mathcal N$, is the smallest relation satisfying the rules
below.

\infrule[Out-barb]{y \in {\mathcal N}, \; x \nameeq y}
		  {\outputp{x}{v} \downarrow_{\mathcal N} x}
\infrule[Par-barb]{\mbox{$P\downarrow_{\mathcal N} x$ or $Q\downarrow_{\mathcal N} x$}}
		  {\binpar{P}{Q} \downarrow_{\mathcal N} x}

We write $P \Downarrow_{\mathcal N} x$ if there is $Q$ such that 
$P \wred Q$ and $Q \downarrow_{\mathcal N} x$.
\end{definition}

\begin{definition}
%\label{def.bbisim}
An  ${\mathcal N}$-\emph{barbed bisimulation} over a set of names, ${\mathcal N}$, is a symmetric binary relation 
${\mathcal S}_{\mathcal N}$ between agents such that $P\rel{S}_{\mathcal N}Q$ implies:
\begin{enumerate}
\item If $P \red P'$ then $Q \wred Q'$ and $P'\rel{S}_{\mathcal N} Q'$.
\item If $P\downarrow_{\mathcal N} x$, then $Q\Downarrow_{\mathcal N} x$.
\end{enumerate}
$P$ is ${\mathcal N}$-barbed bisimilar to $Q$, written
$P \wbbisim_{\mathcal N} Q$, if $P \rel{S}_{\mathcal N} Q$ for some ${\mathcal N}$-barbed bisimulation ${\mathcal S}_{\mathcal N}$.
\end{definition}

$\mathcal{R} \subseteq \pi \times \pi$

$P \mathcal{R} Q => \forall P'. P \red P' \Rightarrow \exists Q'. Q \red Q', P' \mathcal{R} Q'$

$P \vdash x \Rightarrow Q \vdash x$

\begin{mathpar}
  \inferrule*[lab=Out-barb]{x \nameeq y}{{y}!\langle{Q}\rangle \vdash x}
  \and
  \inferrule*[lab=Par-barb]{\mbox{$P\vdash x$ or $Q\vdash x$}}{\binpar{P}{Q} \vdash x}
\end{mathpar}

\subsubsection{Contexts}

One of the principle advantages of computational calculi like the
$\pi$-calculus is a well-defined notion of context,
contextual-equivalence and a correlation between
contextual-equivalence and notions of bisimulation. The notion of
context allows the decomposition of a process into (sub-)process and
its syntactic environment, its context. Thus, a context may be
thought of as a process with a ``hole'' (written $\Box$) in it. The
application of a context $M$ to a process $P$, written $M[P]$, is
tantamount to filling the hole in $M$ with $P$. In this paper we do
not need the full weight of this theory, but do make use of the notion
of context in the proof the main theorem. 

\begin{mathpar}
  \inferrule* [lab=summation] {} {{M_{M},M_{N}} \bc \Box \;|\; x.M_{A} \;|\; M_{M}+M_{N}}
  \and
  \inferrule* [lab=agent] {} {{M_{A}} \bc (\vec{x})M_{P} \;| \; \clift{P_0,\ldots,M_{P},\ldots,P_N}}
  \and \\
  \inferrule* [lab=process] {} {{M_{P}} \bc M_{N} \;| \;P|M_{P} }
\end{mathpar} 

\begin{mathpar}
  \inferrule* [lab=sychronization] {} {M_{N} \bc \Box \;|\; x?M_{F} \;|\; x!M_{C}}
  \and
  \inferrule* [lab=abstraction] {} {{M_{F}} \bc (x)M_{P} }
  \and
  \inferrule* [lab=concretion] {} {{M_{C}} \bc \langle M_{P} \rangle }
  \and \\
  \inferrule* [lab=process] {} {{M_{P}} \bc M_{N} \;| \;P|M_{P} }
\end{mathpar}

\begin{definition}[contextual application] Given a context $M$, and
  process $P$, we define the \emph{contextual application}, $M[P] :=
  M\{P/\Box\}$. That is, the contextual application of M to P is the
  substitution of $P$ for $\Box$ in $M$.
\end{definition}

$\meaningof{-} : L \to \mathcal{P}(\pi)$

\begin{mathpar}
  \inferrule* [lab=collection] {} {\meaningof{true} = \pi, \and \meaningof{~E} = \pi \setminus \meaningof{E}, \and \meaningof{E_{1} \& E_{2}} = \meaningof{E_{1}} \cap \meaningof{E_{2}}}
\end{mathpar}

\begin{mathpar}
  \inferrule* [lab=structure] {} {\meaningof{0} = \{ P \in \pi | P \equiv 0 \}, \and \\ \meaningof{E_1 | E_2} = \{ P \in \pi | P \equiv P_{1} | P_{2}, P_{1} \in \meaningof{E_{1}}, P_{2} \in \meaningof{E_2}\} }
\end{mathpar}

\begin{mathpar}
 \inferrule* [lab=behavior] {} {\meaningof{\langle a?b \rangle E} = \{ P \in \pi | P \equiv Q | u?(y)P', \\ \and \\\\ \and \\ \;\;\; u \in \meaningof{a}, \forall z.P'\{z/y\} \in \meaningof{E\{z/b\}}\}, \and \\ \meaningof{a!E} = \{ P \in \pi | P \equiv Q | x!\langle P' \rangle, x \in \meaningof{a} P' \in \meaningof{E}\} }
\end{mathpar}

\begin{mathpar}
 \inferrule* [lab=nominal] {} {\meaningof{\quotep{E}} = \{ \quotep{P} \in \quotep{\pi} | P \in \meaningof{E} \}, \and \meaningof{\quotep{P}} = \{ \quotep{Q} \in \quotep{\pi} | P \equiv Q \} \and \\ \meaningof{@\quotep{E}} = \{ P \in \pi | P \equiv @x, x \in \meaningof{E} \}}
\end{mathpar}

\begin{eqnarray*}
  \\
  \meaningof{-} : TS \to ST
\end{eqnarray*}

\begin{eqnarray*}
  \\
  L : TS \to ST
\end{eqnarray*}

\begin{eqnarray*}
  \\
  P \models E \iff P \in \meaningof{E}
\end{eqnarray*}

\begin{eqnarray*}
  P \approx_{L} Q \iff \forall E \in L. P \models E \iff Q \models E
\end{eqnarray*}

\begin{eqnarray*}
  P \approx_{K} Q
\end{eqnarray*}

\begin{eqnarray*}
  P \approx Q
\end{eqnarray*}

$\approx_{K} = \approx = \approx_{L}$

\subsubsection{Contextual duality}

Note that contexts extend the quotation operation to a family of
operations from processes to names. Given a context, $M$, we can
define a \emph{nominal context}, $\quotep{M}$ by $\quotep{M}[P] :=
\quotep{M[P]}$. To foreshadow what is to come we observe that these
operations enjoy a duality with processes very much like the duality
between vectors and maps from vectors to scalars.

Further, because the calculus is essentially higher-order, we have a
correspondence between contexts and processes. More specifically,
given a name $x$ and a context $M$ we can construct $M^{*}_{x}$ such
that 

\begin{mathpar}
  M^{*}_{x} | \lift{x}{P} \red M[P]
\end{mathpar}

namely,

\begin{mathpar}
  M^{*}_{x} := x?(u).M[\dropn{u}]
\end{mathpar}

The dependence of $M^{*}_{x}$ on a name makes it an abstraction, 

\begin{mathpar}
  M^{*} := (x)x?(u).M[\dropn{u}]
\end{mathpar}

\subsection{Additional notation}

It will sometimes be convenient to denote the process a name
quotes. We already have the notation $x = \quotep{P}$, but it will be
convenient to introduce an alternate notation, $\procn{x}$, when we
want to emphasize the connection to the use of the name. Note that, by
virtue of name equivalence, $\quotep{\procn{x}} \nameeq x$; so, the
notation is consistent with previous definitions.

Further, because names have structure it is possible to effect
substitutions on the basis of that structure. This means we need to
upgrade our notation for substitutions, which we accomplish by
adapting comprehension notation. Thus,

\begin{mathpar}
  P\{ y / x : x \in S \}
\end{mathpar}

is interpreted to mean the process derived from P by replacing (in a
capture-avoiding manner) each occurrence of $x$ in $S$ by $y$. For example,

\begin{mathpar}
  P\{ \quotep{\procn{x}|\procn{x}} / x : x \in \freenames{P} \}
\end{mathpar}

will replace each (occurrence) of a free name $x$ in $P$ by
$\quotep{\procn{x}|\procn{x}}$.

Also, we will avail ourselves of the notation $x^{L}$ and $x^{R}$ to
denote injections of a name into disjoint copies of the name
space. There are numerous ways to accomplish this. One example can be
found in \cite{MeredithR05}. This notation overloads to vectors of
names: $\vec{x}^{\pi} := (x_{i}^{\pi} \; : \; 0 \leq i < |\vec{x}| )$ where $\pi \in \{L,R\}$.

We also use $P^{\Box} := P|\Box$.

In \cite{MeredithR05} an interpretation of the new operator is
given. It turns out that there are several possible interpretations
all enjoying the requisite algebraic properties of the operator (see
\cite{milner91polyadicpi}). We will therefore make liberal use of
$(\nu\; \vec{x})P$.

% subsection the_syntax_and_semantics_of_the_notation_system (end)   

\input{qm2pi.qmops} 

\input{qm2pi.sterngerlach} 

\input{qm2pi.metric} 

% section concurrent_process_calculi (end)

%\input{qm2pi.proofsketch}

% section proof sketch (end)

%\input{qm2pi.slviaknots} 

% section spatial logic via knots (end)

\input{qm2pi.conclusion}

% section conclusion (end)

%\input{qm2pi.dtcodes} 

% section wiring algorithm (end)

\input{qm2pi.ack} 

% section acknowledgments (end)

\newpage


\bibliographystyle{plain}   
\bibliography{../../biblios/main.bib}

\input{qm2pi.rhodetails}

\end{document}

 

\documentclass[12pt]{llncs}
%\documentclass{jktr}

\usepackage[pdftex]{hyperref}                   
\usepackage {listings}
\usepackage {mathpartir}
\usepackage{bcprules}
%\usepackage{listings}
                       
\usepackage{graphicx} 
%\usepackage[margins=2.5cm,nohead,nofoot]{geometry}
%\usepackage{geometry}
\usepackage{amsfonts}
\usepackage{amstext}
\usepackage{latexsym}
\usepackage{amssymb}
\usepackage{color}


%\include{myPreamble}
\include{qm2pi.local} 

%\ifpdf
%\usepackage[pdftex]{graphicx}
%\else
%\usepackage{graphicx}
%\fi

 % \ifpdf
%  \usepackage{pdfsync}
%  \if


%\title{Brief Article}
%\author{David F. Snyder}
%\author{L.G. Meredith}

%\address{Dept. of Math., Texas State University--San Marcos, San Marcos, TX 78666}
       
\pagestyle{empty}


\begin{document}

\lstset{language=[Objective]Caml,frame=shadowbox}

\input{qm2pi.front}

% section front matter (end)

\input{qm2pi.intro} 
 
% section introduction (end)

% \input{qm2pi.knotations} 

% section notation (end)

\input{qm2pi.process.calculi} 

% section concurrent_process_calculi_and_spatial_logics_ (end)
    
%\input{qm2pi.knots2pi} 

%\input{qm2pi.trefoil} 

%\input{qm2pi.mainthm} 

% subsection basic_interpretation (end)

%\input{qm2pi.rho.presentation} 
\subsection{The syntax and semantics of the notation system}\label{sub:the_syntax_and_semantics_of_the_notation_system} % (fold)

We now summarize a technical presentation of the calculus that
embodies our theory of dynamics. The typical presentation of such a
calculus follows the style of giving generators and relations on
them. The grammar, below, describing term constructors, freely
generates the set of processes, $\Proc$. This set is then quotiented
by a relation known as structural congruence and it is over this set
that the notion of dynamics is expressed. This presentation is
essentially that of \cite{MeredithR05} with the addition of
polyadicity and summation. For readability we have relegated some of
the technical subtleties to an appendix.

\subsubsection{Process grammar}\label{subsub:process_grammar}

\begin{mathpar}
  \inferrule* [lab=synchronization] {} {{M} \bc \pzero \;|\; x?F \;|\; x!C }
  \and
  \inferrule* [lab=abstraction] {} {{F} \bc (x)P}
  \and
  \inferrule* [lab=concretion] {} {{C} \bc \langle Q \rangle}
  \and
  \inferrule* [lab=process] {} {{P,Q} \bc M \;| \;P|Q \;|\; @{x}}
  \and
  \inferrule* [lab=name] {} {{x} \bc \quotep{P}}
\end{mathpar} 

Note that $\vec{x}$ (resp. $\vec{P}$) denotes a vector of names
(resp. processes) of length $|\vec{x}|$ (resp. $|\vec{P}|$). We adopt
the following useful abbreviations.

\begin{mathpar}
   x?(\vec{y}).P := x.(\vec{y})P \and  x\clift{\vec{P}} := x.\clift{\vec{P}}
   \and x!(y) := \lift{x}{\dropn{y}}
   \and \Pi_{i=0}^{n-1}P_i := P_0 | \ldots | P_{n-1}
\end{mathpar}

\subsubsection{Structural congruence}

\paragraph{Free and bound names and alpha-equivalence.} At the
core of structural equivalence is alpha-equivalence which identifies
process that are the same up to a change of variable. Formally, we
recognize the distinction between free and bound names. The free names
of a process, $\freenames{P}$, may be calculated recursively as
follows:

\begin{mathpar}
\freenames{\pzero} := \emptyset
  \and \\
  \freenames{x?(y).P} := \{ x \} \cup (\freenames{P} \setminus \{ y \})
  \and 
  \freenames{x!\langle P \rangle} := \{ x \} \cup \{ P \} 
  \and \\
  \freenames{P|Q} := \freenames{P} \cup \freenames{Q}
  \and \\
  \freenames{@{x}} := \{ x \}
\end{mathpar}

$\pi$
$\quotep{\pi}$

$\freenames{-} : \pi \to \mathcal{P}(\quotep{\pi})$

\begin{eqnarray*}
  \freenames{\pzero} & := & \emptyset \\
  \freenames{x?(y).P} & := & \{ x \} \cup (\freenames{P} \setminus \{ y \}) \\
  \freenames{x!\langle P \rangle} & := & \{ x \} \cup \{ P \} \\
  \freenames{P|Q} & := & \freenames{P} \cup \freenames{Q} \\
  \freenames{\dropn{x}} & := & \{ x \}
\end{eqnarray*}

The bound names of a process, $\boundnames{P}$, are those names occurring in $P$
that are not free. For example, in $x?(y).0$, the name $x$ is free, while $y$ is bound.

\begin{mathpar}
  \inferrule* [lab=monoidal-laws] {} { P|Q \equiv Q|P \and P|0 \equiv P \and P|(Q|R) \equiv (P|Q)|R }
\end{mathpar}

\begin{mathpar}
  \inferrule* [lab=alpha-equivalence] {} { (x)P \equiv (y)P\{y/x\} \and y \not\in \freenames{P} }
\end{mathpar}

\begin{definition}
Then two processes, $P,Q$, are alpha-equivalent if $P = Q\{\vec{y}/\vec{x}\}$ for
some $\vec{x} \in \boundnames{Q},\vec{y} \in \boundnames{P}$, where $Q\{\vec{y}/\vec{x}\}$
denotes the capture-avoiding substitution of $\vec{y}$ for $\vec{x}$ in $Q$.
\end{definition}

\begin{definition}
  The {\em structural congruence} \cite{SangiorgiWalker} , $\equiv$,
  between processes is the least congruence containing
  alpha-equivalence, satisfying the abelian monoid laws
  (associativity, commutativity and $\pzero$ as identity) for parallel
  composition $|$ and for summation $+$.
\end{definition}

\subsection{Name equivalence}

We take name equivalence, written $\nameeq$, to be the smallest
equivalence relation generated by the following rules.

\begin{mathpar}
\inferrule*[lab=Quote-drop]
{ }
{ \quotep{@{x}} \nameeq x }

\inferrule*[lab=Struct-equiv]
{ P \scong Q }
{ \quotep{P} \nameeq \quotep{Q} }
\end{mathpar}

The astute reader will have noticed that the mutual recursion of names
and processes imposes a mutual recursion on alpha-equivalence and
structural equivalence via name-equivalence. Fortunately, all of this
works out pleasantly and we may calculate in the natural way, free of
concern. The reader interested in the details is referred to the
appendix \ref{appendix:rho_details}.

\subsection{Substitution}

We use $\Proc$ for the set of processes, $\QProc$ for the set of
names, and $\id{\{}\vec{y} / \vec{x} \id{\}}$ to denote partial maps,
$s : \QProc \rightarrow \QProc$. A map, $s$ lifts, uniquely, to a map
on process terms, $\widehat{s} : \Proc \rightarrow \Proc$ by the
following equations.

\begin{mathpar}
  (0) \psubstp{Q}{P} := 0 \\
  (R \juxtap S) \psubstp{Q}{P}
  :=    
  (R)\psubstp{Q}{P} \juxtap (S) \psubstp{Q}{P} \\
  (x?(y).R) \psubstp{Q}{P}    
  :=    
  (x)\substp{Q}{P} (z)\concat( (R \psubstn{z}{y}) \psubstp{Q}{P} ) \\
  (\lift{x}{R}) \psubstp{Q}{P}  
  :=
  \lift{(x)\substp{Q}{P}}{ R \psubstp{Q}{P} } \\
%   (\dropn{x})  \psubstp{Q}{P}       
%   := 
%   \left\{ 
%     \begin{array}{ccc} 
%       \dropn{\quotep{Q}} & & x \nameeq \quotep{P} \\
%       \dropn{x} & & otherwise \\
%     \end{array}
%   \right. 
  (\dropn{x})  \psubstp{Q}{P}       
  := 
  \left\{ 
    \begin{array}{ccc} 
      Q & & x \nameeq \quotep{P} \\
      \dropn{x} & & otherwise \\
    \end{array}
  \right.
\end{mathpar}
 

where

\begin{eqnarray}
  (x)\id{\{} \lpquote Q \rpquote / \lpquote P \rpquote \id{\}}            = 
  \left\{ 
    \begin{array}{ccc}
      \lpquote Q \rpquote & & x \nameeq \lpquote P \rpquote \\
      x & & otherwise \\
    \end{array}
  \right. \nonumber
\end{eqnarray}

and $z$ is chosen distinct from $\quotep{P}$, $\quotep{Q}$, the free
names in $Q$, and all the names in $R$. Our $\alpha$-equivalence will
be built in the standard way from this substitution.

\begin{remark}\label{rem:no_self_referential_names}
  One consequence of these definitions is that $\forall P. \quotep{P}
  \not\in \freenames{P}$.
\end{remark}

\subsection{ Dynamic quote: an example }

Anticipating something of what's to come, consider applying the
substitution, $\widehat{\id{\{}u / z \id{\}}}$, to the following pair
of processes, $\lift{w}{y!(z)}$ and $w[ \lpquote y!(z) \rpquote ]$.

\begin{eqnarray}
	\lift{w}{y!(z)}\widehat{\id{\{}u / z \id{\}}}
		& = &
		\lift{w}{y!(u)} \nonumber\\
	w[ \lpquote y!(z) \rpquote ] \widehat{ \id{\{}u / z \id{\}} }
		& = &
		w[ \lpquote y!(z) \rpquote ] \nonumber
\end{eqnarray}

Because the body of the process between quotes is impervious to
substitution, we get radically different answers. In fact, by
examining the first process in an input context,
e.g. $x?(z).\lift{w}{y!(z)}$, we see that the process under the lift
operator may be shaped by prefixed inputs binding a name inside it. In
this sense, the lift operator will be seen as a way to dynamically
construct processes before reifying them as names.

Finally equipped with these standard features we can present the
dynamics of the calculus.

\subsubsection{Operational semantics} 

Finally, we introduce the computational dynamics. What marks these
algebras as distinct from other more traditionally studied algebraic
structures, e.g. vector spaces or polynomial rings, is the manner in
which dynamics is captured. In traditional structures, dynamics is typically
expressed through morphisms between such structures, as in linear maps
between vector spaces or morphisms between rings. In algebras
associated with the semantics of computation, the dynamics is
expressed as part of the algebraic structure itself, through a
reduction reduction relation typically denoted by $\red$. Below, we
give a recursive presentation of this relation for the calculus used
in the encoding.

$\red \subseteq \pi \times \pi$
$\red : \pi \to \mathcal{P}(\pi)$

\begin{mathpar}
  \inferrule* [lab=Comm] { \textsf{match}( x_{src}, x_{trgt} ) } { x_{trgt}?(y)P \; | \; x_{src}!\langle {Q} \rangle \red P\{\quotep{Q}/y}\} }
  \and \\
  \inferrule* [lab=Par] {{P} \red {P}'} {{{P} | {Q}} \red {{P}' | {Q}}}
  \and
  \inferrule* [lab=Equiv]{{{P} \scong {P}'} \andalso {{P}' \red {Q}'} \andalso {{Q}' \scong {Q}}}{{P} \red {Q}}
\end{mathpar}

\begin{eqnarray*}
  match_{\equiv} (\quotep{P},\quotep{Q}) & := & P \equiv Q \\
  match_{\dagger}(\quotep{P},\quotep{Q}) & := & \forall R. P|Q \red^{*} R => R \red^{*} 0 \\
  match_{K}(\quotep{P},\quotep{Q}) & := & K \mbox{ for some context } K
\end{eqnarray*}

$u?(x)P | u!\langle Q \rangle \red P\{\quotep{Q}/x\}$

%We write $\wred$ for $\red^*$, and $P\red$ if $\exists Q $ such that $ P \red Q$.
We write $P\red$ if $\exists Q $ such that $ P \red Q$ and $P\not\red$, otherwise.

\section{Replication}

As mentioned before, it is known that replication (and hence
recursion) can be implemented in a higher-order process algebra
\cite{SangiorgiWalker}. As our first example of calculation with the
machinery thus far presented we give the construction explicitly in
the {\rhoc}.

\begin{eqnarray}
	D_{x} & := & \prefix{x}{y}{(\binpar{\outputp{x}{y}}{@{y}})} \nonumber\\
	\bangp_{x}{P} & := & \binpar{{x}!\langle{\binpar{D_{x}}{P}}\rangle}{D_{x}} \nonumber
\end{eqnarray}

\begin{eqnarray}
	\bangp_{x}{P} & & \nonumber\\
	=
	& {x}!\langle{(\prefix{x}{y}{(\outputp{x}{y} | @{y})) | P}}\rangle 
	      | \prefix{x}{y}{(\outputp{x}{y} | @{y})} & \nonumber\\
	\red
	& (\outputp{x}{y} | @{y})\substn{\quotep{(\prefix{x}{y}{(@{y} | \outputp{x}{y})) | P}}}{y} & \nonumber\\
	=
	& \outputp{x}{\quotep{(\prefix{x}{y}{(\outputp{x}{y} | @{y})) | P}}}
	  | {(\prefix{x}{y}{(\outputp{x}{y} | @{y})) | P}} & \nonumber\\
	\red
	& \ldots & \nonumber\\
	\red^*
	& P | P | \ldots & \nonumber
\end{eqnarray}

Of course, this encoding, as an implementation, runs away, unfolding
$\bangp{P}$ eagerly. A lazier and more implementable replication
operator, restricted to input-guarded processes, may be obtained as follows.

\begin{eqnarray}
\bangp{\prefix{u}{v}{P}} 
	:= 
	\binpar{\lift{x}{\prefix{u}{v}{(\binpar{D(x)}{P})}}}{D(x)} \nonumber
\end{eqnarray}

\begin{remark}
  Note that the lazier definition still does not deal with summation
  or mixed summation (i.e. sums over input and output). The reader is
  invited to construct definitions of replication that deal with these
  features. 

  Further, the definitions are parameterized in a name, $x$. Can you,
  gentle reader, make a definition that eliminates this parameter and
  guarantees no accidental interaction between the replication
  machinery and the process being replicated -- i.e. no accidental
  sharing of names used by the process to get its work done and the
  name(s) used by the replication to effect copying. This latter
  revision of the definition of replication is crucial to obtaining
  the expected identity $!!P \sim !P$.
\end{remark}

\begin{remark}\label{rem:paradoxical_combinator}
  The reader familiar with the lambda calculus will have noticed the
  similarity between $D$ and the paradoxical combinator.

  [Ed. note: the existence of this seems to suggest we have to be more
  restrictive on the set of processes and names we admit if we are to
  support no-cloning.]
\end{remark}

\subsubsection{Bisimulation}

The computational dynamics gives rise to another kind of equivalence,
the equivalence of computational behavior. As previously mentioned
this is typically captured \emph{via} some form of bisimulation.

% The notion we use in this paper is weak barbed bisimulation
% \cite{milner91polyadicpi}.

The notion we use in this paper is derived from weak barbed
bisimulation \cite{milner91polyadicpi}. 

\begin{definition}
An \emph{observation relation}, $\downarrow_{\mathcal N}$, over a set
of names, $\mathcal N$, is the smallest relation satisfying the rules
below.

\infrule[Out-barb]{y \in {\mathcal N}, \; x \nameeq y}
		  {\outputp{x}{v} \downarrow_{\mathcal N} x}
\infrule[Par-barb]{\mbox{$P\downarrow_{\mathcal N} x$ or $Q\downarrow_{\mathcal N} x$}}
		  {\binpar{P}{Q} \downarrow_{\mathcal N} x}

We write $P \Downarrow_{\mathcal N} x$ if there is $Q$ such that 
$P \wred Q$ and $Q \downarrow_{\mathcal N} x$.
\end{definition}

\begin{definition}
%\label{def.bbisim}
An  ${\mathcal N}$-\emph{barbed bisimulation} over a set of names, ${\mathcal N}$, is a symmetric binary relation 
${\mathcal S}_{\mathcal N}$ between agents such that $P\rel{S}_{\mathcal N}Q$ implies:
\begin{enumerate}
\item If $P \red P'$ then $Q \wred Q'$ and $P'\rel{S}_{\mathcal N} Q'$.
\item If $P\downarrow_{\mathcal N} x$, then $Q\Downarrow_{\mathcal N} x$.
\end{enumerate}
$P$ is ${\mathcal N}$-barbed bisimilar to $Q$, written
$P \wbbisim_{\mathcal N} Q$, if $P \rel{S}_{\mathcal N} Q$ for some ${\mathcal N}$-barbed bisimulation ${\mathcal S}_{\mathcal N}$.
\end{definition}

$\mathcal{R} \subseteq \pi \times \pi$

$P \mathcal{R} Q => \forall P'. P \red P' \Rightarrow \exists Q'. Q \red Q', P' \mathcal{R} Q'$

$P \vdash x \Rightarrow Q \vdash x$

\begin{mathpar}
  \inferrule*[lab=Out-barb]{x \nameeq y}{{y}!\langle{Q}\rangle \vdash x}
  \and
  \inferrule*[lab=Par-barb]{\mbox{$P\vdash x$ or $Q\vdash x$}}{\binpar{P}{Q} \vdash x}
\end{mathpar}

\subsubsection{Contexts}

One of the principle advantages of computational calculi like the
$\pi$-calculus is a well-defined notion of context,
contextual-equivalence and a correlation between
contextual-equivalence and notions of bisimulation. The notion of
context allows the decomposition of a process into (sub-)process and
its syntactic environment, its context. Thus, a context may be
thought of as a process with a ``hole'' (written $\Box$) in it. The
application of a context $M$ to a process $P$, written $M[P]$, is
tantamount to filling the hole in $M$ with $P$. In this paper we do
not need the full weight of this theory, but do make use of the notion
of context in the proof the main theorem. 

\begin{mathpar}
  \inferrule* [lab=summation] {} {{M_{M},M_{N}} \bc \Box \;|\; x.M_{A} \;|\; M_{M}+M_{N}}
  \and
  \inferrule* [lab=agent] {} {{M_{A}} \bc (\vec{x})M_{P} \;| \; \clift{P_0,\ldots,M_{P},\ldots,P_N}}
  \and \\
  \inferrule* [lab=process] {} {{M_{P}} \bc M_{N} \;| \;P|M_{P} }
\end{mathpar} 

\begin{mathpar}
  \inferrule* [lab=sychronization] {} {M_{N} \bc \Box \;|\; x?M_{F} \;|\; x!M_{C}}
  \and
  \inferrule* [lab=abstraction] {} {{M_{F}} \bc (x)M_{P} }
  \and
  \inferrule* [lab=concretion] {} {{M_{C}} \bc \langle M_{P} \rangle }
  \and \\
  \inferrule* [lab=process] {} {{M_{P}} \bc M_{N} \;| \;P|M_{P} }
\end{mathpar}

\begin{definition}[contextual application] Given a context $M$, and
  process $P$, we define the \emph{contextual application}, $M[P] :=
  M\{P/\Box\}$. That is, the contextual application of M to P is the
  substitution of $P$ for $\Box$ in $M$.
\end{definition}

$\meaningof{-} : L \to \mathcal{P}(\pi)$

\begin{mathpar}
  \inferrule* [lab=collection] {} {\meaningof{true} = \pi, \and \meaningof{~E} = \pi \setminus \meaningof{E}, \and \meaningof{E_{1} \& E_{2}} = \meaningof{E_{1}} \cap \meaningof{E_{2}}}
\end{mathpar}

\begin{mathpar}
  \inferrule* [lab=structure] {} {\meaningof{0} = \{ P \in \pi | P \equiv 0 \}, \and \\ \meaningof{E_1 | E_2} = \{ P \in \pi | P \equiv P_{1} | P_{2}, P_{1} \in \meaningof{E_{1}}, P_{2} \in \meaningof{E_2}\} }
\end{mathpar}

\begin{mathpar}
 \inferrule* [lab=behavior] {} {\meaningof{\langle a?b \rangle E} = \{ P \in \pi | P \equiv Q | u?(y)P', \\ \and \\\\ \and \\ \;\;\; u \in \meaningof{a}, \forall z.P'\{z/y\} \in \meaningof{E\{z/b\}}\}, \and \\ \meaningof{a!E} = \{ P \in \pi | P \equiv Q | x!\langle P' \rangle, x \in \meaningof{a} P' \in \meaningof{E}\} }
\end{mathpar}

\begin{mathpar}
 \inferrule* [lab=nominal] {} {\meaningof{\quotep{E}} = \{ \quotep{P} \in \quotep{\pi} | P \in \meaningof{E} \}, \and \meaningof{\quotep{P}} = \{ \quotep{Q} \in \quotep{\pi} | P \equiv Q \} \and \\ \meaningof{@\quotep{E}} = \{ P \in \pi | P \equiv @x, x \in \meaningof{E} \}}
\end{mathpar}

\begin{eqnarray*}
  \\
  \meaningof{-} : TS \to ST
\end{eqnarray*}

\begin{eqnarray*}
  \\
  L : TS \to ST
\end{eqnarray*}

\begin{eqnarray*}
  \\
  P \models E \iff P \in \meaningof{E}
\end{eqnarray*}

\begin{eqnarray*}
  P \approx_{L} Q \iff \forall E \in L. P \models E \iff Q \models E
\end{eqnarray*}

\begin{eqnarray*}
  P \approx_{K} Q
\end{eqnarray*}

\begin{eqnarray*}
  P \approx Q
\end{eqnarray*}

$\approx_{K} = \approx = \approx_{L}$

\subsubsection{Contextual duality}

Note that contexts extend the quotation operation to a family of
operations from processes to names. Given a context, $M$, we can
define a \emph{nominal context}, $\quotep{M}$ by $\quotep{M}[P] :=
\quotep{M[P]}$. To foreshadow what is to come we observe that these
operations enjoy a duality with processes very much like the duality
between vectors and maps from vectors to scalars.

Further, because the calculus is essentially higher-order, we have a
correspondence between contexts and processes. More specifically,
given a name $x$ and a context $M$ we can construct $M^{*}_{x}$ such
that 

\begin{mathpar}
  M^{*}_{x} | \lift{x}{P} \red M[P]
\end{mathpar}

namely,

\begin{mathpar}
  M^{*}_{x} := x?(u).M[\dropn{u}]
\end{mathpar}

The dependence of $M^{*}_{x}$ on a name makes it an abstraction, 

\begin{mathpar}
  M^{*} := (x)x?(u).M[\dropn{u}]
\end{mathpar}

\subsection{Additional notation}

It will sometimes be convenient to denote the process a name
quotes. We already have the notation $x = \quotep{P}$, but it will be
convenient to introduce an alternate notation, $\procn{x}$, when we
want to emphasize the connection to the use of the name. Note that, by
virtue of name equivalence, $\quotep{\procn{x}} \nameeq x$; so, the
notation is consistent with previous definitions.

Further, because names have structure it is possible to effect
substitutions on the basis of that structure. This means we need to
upgrade our notation for substitutions, which we accomplish by
adapting comprehension notation. Thus,

\begin{mathpar}
  P\{ y / x : x \in S \}
\end{mathpar}

is interpreted to mean the process derived from P by replacing (in a
capture-avoiding manner) each occurrence of $x$ in $S$ by $y$. For example,

\begin{mathpar}
  P\{ \quotep{\procn{x}|\procn{x}} / x : x \in \freenames{P} \}
\end{mathpar}

will replace each (occurrence) of a free name $x$ in $P$ by
$\quotep{\procn{x}|\procn{x}}$.

Also, we will avail ourselves of the notation $x^{L}$ and $x^{R}$ to
denote injections of a name into disjoint copies of the name
space. There are numerous ways to accomplish this. One example can be
found in \cite{MeredithR05}. This notation overloads to vectors of
names: $\vec{x}^{\pi} := (x_{i}^{\pi} \; : \; 0 \leq i < |\vec{x}| )$ where $\pi \in \{L,R\}$.

We also use $P^{\Box} := P|\Box$.

In \cite{MeredithR05} an interpretation of the new operator is
given. It turns out that there are several possible interpretations
all enjoying the requisite algebraic properties of the operator (see
\cite{milner91polyadicpi}). We will therefore make liberal use of
$(\nu\; \vec{x})P$.

% subsection the_syntax_and_semantics_of_the_notation_system (end)   

\input{qm2pi.qmops} 

\input{qm2pi.sterngerlach} 

\input{qm2pi.metric} 

% section concurrent_process_calculi (end)

%\input{qm2pi.proofsketch}

% section proof sketch (end)

%\input{qm2pi.slviaknots} 

% section spatial logic via knots (end)

\input{qm2pi.conclusion}

% section conclusion (end)

%\input{qm2pi.dtcodes} 

% section wiring algorithm (end)

\input{qm2pi.ack} 

% section acknowledgments (end)

\newpage


\bibliographystyle{plain}   
\bibliography{../../biblios/main.bib}

\input{qm2pi.rhodetails}

\end{document}

 

% section concurrent_process_calculi (end)

%\documentclass[12pt]{llncs}
%\documentclass{jktr}

\usepackage[pdftex]{hyperref}                   
\usepackage {listings}
\usepackage {mathpartir}
\usepackage{bcprules}
%\usepackage{listings}
                       
\usepackage{graphicx} 
%\usepackage[margins=2.5cm,nohead,nofoot]{geometry}
%\usepackage{geometry}
\usepackage{amsfonts}
\usepackage{amstext}
\usepackage{latexsym}
\usepackage{amssymb}
\usepackage{color}


%\include{myPreamble}
\include{qm2pi.local} 

%\ifpdf
%\usepackage[pdftex]{graphicx}
%\else
%\usepackage{graphicx}
%\fi

 % \ifpdf
%  \usepackage{pdfsync}
%  \if


%\title{Brief Article}
%\author{David F. Snyder}
%\author{L.G. Meredith}

%\address{Dept. of Math., Texas State University--San Marcos, San Marcos, TX 78666}
       
\pagestyle{empty}


\begin{document}

\lstset{language=[Objective]Caml,frame=shadowbox}

\input{qm2pi.front}

% section front matter (end)

\input{qm2pi.intro} 
 
% section introduction (end)

% \input{qm2pi.knotations} 

% section notation (end)

\input{qm2pi.process.calculi} 

% section concurrent_process_calculi_and_spatial_logics_ (end)
    
%\input{qm2pi.knots2pi} 

%\input{qm2pi.trefoil} 

%\input{qm2pi.mainthm} 

% subsection basic_interpretation (end)

%\input{qm2pi.rho.presentation} 
\subsection{The syntax and semantics of the notation system}\label{sub:the_syntax_and_semantics_of_the_notation_system} % (fold)

We now summarize a technical presentation of the calculus that
embodies our theory of dynamics. The typical presentation of such a
calculus follows the style of giving generators and relations on
them. The grammar, below, describing term constructors, freely
generates the set of processes, $\Proc$. This set is then quotiented
by a relation known as structural congruence and it is over this set
that the notion of dynamics is expressed. This presentation is
essentially that of \cite{MeredithR05} with the addition of
polyadicity and summation. For readability we have relegated some of
the technical subtleties to an appendix.

\subsubsection{Process grammar}\label{subsub:process_grammar}

\begin{mathpar}
  \inferrule* [lab=synchronization] {} {{M} \bc \pzero \;|\; x?F \;|\; x!C }
  \and
  \inferrule* [lab=abstraction] {} {{F} \bc (x)P}
  \and
  \inferrule* [lab=concretion] {} {{C} \bc \langle Q \rangle}
  \and
  \inferrule* [lab=process] {} {{P,Q} \bc M \;| \;P|Q \;|\; @{x}}
  \and
  \inferrule* [lab=name] {} {{x} \bc \quotep{P}}
\end{mathpar} 

Note that $\vec{x}$ (resp. $\vec{P}$) denotes a vector of names
(resp. processes) of length $|\vec{x}|$ (resp. $|\vec{P}|$). We adopt
the following useful abbreviations.

\begin{mathpar}
   x?(\vec{y}).P := x.(\vec{y})P \and  x\clift{\vec{P}} := x.\clift{\vec{P}}
   \and x!(y) := \lift{x}{\dropn{y}}
   \and \Pi_{i=0}^{n-1}P_i := P_0 | \ldots | P_{n-1}
\end{mathpar}

\subsubsection{Structural congruence}

\paragraph{Free and bound names and alpha-equivalence.} At the
core of structural equivalence is alpha-equivalence which identifies
process that are the same up to a change of variable. Formally, we
recognize the distinction between free and bound names. The free names
of a process, $\freenames{P}$, may be calculated recursively as
follows:

\begin{mathpar}
\freenames{\pzero} := \emptyset
  \and \\
  \freenames{x?(y).P} := \{ x \} \cup (\freenames{P} \setminus \{ y \})
  \and 
  \freenames{x!\langle P \rangle} := \{ x \} \cup \{ P \} 
  \and \\
  \freenames{P|Q} := \freenames{P} \cup \freenames{Q}
  \and \\
  \freenames{@{x}} := \{ x \}
\end{mathpar}

$\pi$
$\quotep{\pi}$

$\freenames{-} : \pi \to \mathcal{P}(\quotep{\pi})$

\begin{eqnarray*}
  \freenames{\pzero} & := & \emptyset \\
  \freenames{x?(y).P} & := & \{ x \} \cup (\freenames{P} \setminus \{ y \}) \\
  \freenames{x!\langle P \rangle} & := & \{ x \} \cup \{ P \} \\
  \freenames{P|Q} & := & \freenames{P} \cup \freenames{Q} \\
  \freenames{\dropn{x}} & := & \{ x \}
\end{eqnarray*}

The bound names of a process, $\boundnames{P}$, are those names occurring in $P$
that are not free. For example, in $x?(y).0$, the name $x$ is free, while $y$ is bound.

\begin{mathpar}
  \inferrule* [lab=monoidal-laws] {} { P|Q \equiv Q|P \and P|0 \equiv P \and P|(Q|R) \equiv (P|Q)|R }
\end{mathpar}

\begin{mathpar}
  \inferrule* [lab=alpha-equivalence] {} { (x)P \equiv (y)P\{y/x\} \and y \not\in \freenames{P} }
\end{mathpar}

\begin{definition}
Then two processes, $P,Q$, are alpha-equivalent if $P = Q\{\vec{y}/\vec{x}\}$ for
some $\vec{x} \in \boundnames{Q},\vec{y} \in \boundnames{P}$, where $Q\{\vec{y}/\vec{x}\}$
denotes the capture-avoiding substitution of $\vec{y}$ for $\vec{x}$ in $Q$.
\end{definition}

\begin{definition}
  The {\em structural congruence} \cite{SangiorgiWalker} , $\equiv$,
  between processes is the least congruence containing
  alpha-equivalence, satisfying the abelian monoid laws
  (associativity, commutativity and $\pzero$ as identity) for parallel
  composition $|$ and for summation $+$.
\end{definition}

\subsection{Name equivalence}

We take name equivalence, written $\nameeq$, to be the smallest
equivalence relation generated by the following rules.

\begin{mathpar}
\inferrule*[lab=Quote-drop]
{ }
{ \quotep{@{x}} \nameeq x }

\inferrule*[lab=Struct-equiv]
{ P \scong Q }
{ \quotep{P} \nameeq \quotep{Q} }
\end{mathpar}

The astute reader will have noticed that the mutual recursion of names
and processes imposes a mutual recursion on alpha-equivalence and
structural equivalence via name-equivalence. Fortunately, all of this
works out pleasantly and we may calculate in the natural way, free of
concern. The reader interested in the details is referred to the
appendix \ref{appendix:rho_details}.

\subsection{Substitution}

We use $\Proc$ for the set of processes, $\QProc$ for the set of
names, and $\id{\{}\vec{y} / \vec{x} \id{\}}$ to denote partial maps,
$s : \QProc \rightarrow \QProc$. A map, $s$ lifts, uniquely, to a map
on process terms, $\widehat{s} : \Proc \rightarrow \Proc$ by the
following equations.

\begin{mathpar}
  (0) \psubstp{Q}{P} := 0 \\
  (R \juxtap S) \psubstp{Q}{P}
  :=    
  (R)\psubstp{Q}{P} \juxtap (S) \psubstp{Q}{P} \\
  (x?(y).R) \psubstp{Q}{P}    
  :=    
  (x)\substp{Q}{P} (z)\concat( (R \psubstn{z}{y}) \psubstp{Q}{P} ) \\
  (\lift{x}{R}) \psubstp{Q}{P}  
  :=
  \lift{(x)\substp{Q}{P}}{ R \psubstp{Q}{P} } \\
%   (\dropn{x})  \psubstp{Q}{P}       
%   := 
%   \left\{ 
%     \begin{array}{ccc} 
%       \dropn{\quotep{Q}} & & x \nameeq \quotep{P} \\
%       \dropn{x} & & otherwise \\
%     \end{array}
%   \right. 
  (\dropn{x})  \psubstp{Q}{P}       
  := 
  \left\{ 
    \begin{array}{ccc} 
      Q & & x \nameeq \quotep{P} \\
      \dropn{x} & & otherwise \\
    \end{array}
  \right.
\end{mathpar}
 

where

\begin{eqnarray}
  (x)\id{\{} \lpquote Q \rpquote / \lpquote P \rpquote \id{\}}            = 
  \left\{ 
    \begin{array}{ccc}
      \lpquote Q \rpquote & & x \nameeq \lpquote P \rpquote \\
      x & & otherwise \\
    \end{array}
  \right. \nonumber
\end{eqnarray}

and $z$ is chosen distinct from $\quotep{P}$, $\quotep{Q}$, the free
names in $Q$, and all the names in $R$. Our $\alpha$-equivalence will
be built in the standard way from this substitution.

\begin{remark}\label{rem:no_self_referential_names}
  One consequence of these definitions is that $\forall P. \quotep{P}
  \not\in \freenames{P}$.
\end{remark}

\subsection{ Dynamic quote: an example }

Anticipating something of what's to come, consider applying the
substitution, $\widehat{\id{\{}u / z \id{\}}}$, to the following pair
of processes, $\lift{w}{y!(z)}$ and $w[ \lpquote y!(z) \rpquote ]$.

\begin{eqnarray}
	\lift{w}{y!(z)}\widehat{\id{\{}u / z \id{\}}}
		& = &
		\lift{w}{y!(u)} \nonumber\\
	w[ \lpquote y!(z) \rpquote ] \widehat{ \id{\{}u / z \id{\}} }
		& = &
		w[ \lpquote y!(z) \rpquote ] \nonumber
\end{eqnarray}

Because the body of the process between quotes is impervious to
substitution, we get radically different answers. In fact, by
examining the first process in an input context,
e.g. $x?(z).\lift{w}{y!(z)}$, we see that the process under the lift
operator may be shaped by prefixed inputs binding a name inside it. In
this sense, the lift operator will be seen as a way to dynamically
construct processes before reifying them as names.

Finally equipped with these standard features we can present the
dynamics of the calculus.

\subsubsection{Operational semantics} 

Finally, we introduce the computational dynamics. What marks these
algebras as distinct from other more traditionally studied algebraic
structures, e.g. vector spaces or polynomial rings, is the manner in
which dynamics is captured. In traditional structures, dynamics is typically
expressed through morphisms between such structures, as in linear maps
between vector spaces or morphisms between rings. In algebras
associated with the semantics of computation, the dynamics is
expressed as part of the algebraic structure itself, through a
reduction reduction relation typically denoted by $\red$. Below, we
give a recursive presentation of this relation for the calculus used
in the encoding.

$\red \subseteq \pi \times \pi$
$\red : \pi \to \mathcal{P}(\pi)$

\begin{mathpar}
  \inferrule* [lab=Comm] { \textsf{match}( x_{src}, x_{trgt} ) } { x_{trgt}?(y)P \; | \; x_{src}!\langle {Q} \rangle \red P\{\quotep{Q}/y}\} }
  \and \\
  \inferrule* [lab=Par] {{P} \red {P}'} {{{P} | {Q}} \red {{P}' | {Q}}}
  \and
  \inferrule* [lab=Equiv]{{{P} \scong {P}'} \andalso {{P}' \red {Q}'} \andalso {{Q}' \scong {Q}}}{{P} \red {Q}}
\end{mathpar}

\begin{eqnarray*}
  match_{\equiv} (\quotep{P},\quotep{Q}) & := & P \equiv Q \\
  match_{\dagger}(\quotep{P},\quotep{Q}) & := & \forall R. P|Q \red^{*} R => R \red^{*} 0 \\
  match_{K}(\quotep{P},\quotep{Q}) & := & K \mbox{ for some context } K
\end{eqnarray*}

$u?(x)P | u!\langle Q \rangle \red P\{\quotep{Q}/x\}$

%We write $\wred$ for $\red^*$, and $P\red$ if $\exists Q $ such that $ P \red Q$.
We write $P\red$ if $\exists Q $ such that $ P \red Q$ and $P\not\red$, otherwise.

\section{Replication}

As mentioned before, it is known that replication (and hence
recursion) can be implemented in a higher-order process algebra
\cite{SangiorgiWalker}. As our first example of calculation with the
machinery thus far presented we give the construction explicitly in
the {\rhoc}.

\begin{eqnarray}
	D_{x} & := & \prefix{x}{y}{(\binpar{\outputp{x}{y}}{@{y}})} \nonumber\\
	\bangp_{x}{P} & := & \binpar{{x}!\langle{\binpar{D_{x}}{P}}\rangle}{D_{x}} \nonumber
\end{eqnarray}

\begin{eqnarray}
	\bangp_{x}{P} & & \nonumber\\
	=
	& {x}!\langle{(\prefix{x}{y}{(\outputp{x}{y} | @{y})) | P}}\rangle 
	      | \prefix{x}{y}{(\outputp{x}{y} | @{y})} & \nonumber\\
	\red
	& (\outputp{x}{y} | @{y})\substn{\quotep{(\prefix{x}{y}{(@{y} | \outputp{x}{y})) | P}}}{y} & \nonumber\\
	=
	& \outputp{x}{\quotep{(\prefix{x}{y}{(\outputp{x}{y} | @{y})) | P}}}
	  | {(\prefix{x}{y}{(\outputp{x}{y} | @{y})) | P}} & \nonumber\\
	\red
	& \ldots & \nonumber\\
	\red^*
	& P | P | \ldots & \nonumber
\end{eqnarray}

Of course, this encoding, as an implementation, runs away, unfolding
$\bangp{P}$ eagerly. A lazier and more implementable replication
operator, restricted to input-guarded processes, may be obtained as follows.

\begin{eqnarray}
\bangp{\prefix{u}{v}{P}} 
	:= 
	\binpar{\lift{x}{\prefix{u}{v}{(\binpar{D(x)}{P})}}}{D(x)} \nonumber
\end{eqnarray}

\begin{remark}
  Note that the lazier definition still does not deal with summation
  or mixed summation (i.e. sums over input and output). The reader is
  invited to construct definitions of replication that deal with these
  features. 

  Further, the definitions are parameterized in a name, $x$. Can you,
  gentle reader, make a definition that eliminates this parameter and
  guarantees no accidental interaction between the replication
  machinery and the process being replicated -- i.e. no accidental
  sharing of names used by the process to get its work done and the
  name(s) used by the replication to effect copying. This latter
  revision of the definition of replication is crucial to obtaining
  the expected identity $!!P \sim !P$.
\end{remark}

\begin{remark}\label{rem:paradoxical_combinator}
  The reader familiar with the lambda calculus will have noticed the
  similarity between $D$ and the paradoxical combinator.

  [Ed. note: the existence of this seems to suggest we have to be more
  restrictive on the set of processes and names we admit if we are to
  support no-cloning.]
\end{remark}

\subsubsection{Bisimulation}

The computational dynamics gives rise to another kind of equivalence,
the equivalence of computational behavior. As previously mentioned
this is typically captured \emph{via} some form of bisimulation.

% The notion we use in this paper is weak barbed bisimulation
% \cite{milner91polyadicpi}.

The notion we use in this paper is derived from weak barbed
bisimulation \cite{milner91polyadicpi}. 

\begin{definition}
An \emph{observation relation}, $\downarrow_{\mathcal N}$, over a set
of names, $\mathcal N$, is the smallest relation satisfying the rules
below.

\infrule[Out-barb]{y \in {\mathcal N}, \; x \nameeq y}
		  {\outputp{x}{v} \downarrow_{\mathcal N} x}
\infrule[Par-barb]{\mbox{$P\downarrow_{\mathcal N} x$ or $Q\downarrow_{\mathcal N} x$}}
		  {\binpar{P}{Q} \downarrow_{\mathcal N} x}

We write $P \Downarrow_{\mathcal N} x$ if there is $Q$ such that 
$P \wred Q$ and $Q \downarrow_{\mathcal N} x$.
\end{definition}

\begin{definition}
%\label{def.bbisim}
An  ${\mathcal N}$-\emph{barbed bisimulation} over a set of names, ${\mathcal N}$, is a symmetric binary relation 
${\mathcal S}_{\mathcal N}$ between agents such that $P\rel{S}_{\mathcal N}Q$ implies:
\begin{enumerate}
\item If $P \red P'$ then $Q \wred Q'$ and $P'\rel{S}_{\mathcal N} Q'$.
\item If $P\downarrow_{\mathcal N} x$, then $Q\Downarrow_{\mathcal N} x$.
\end{enumerate}
$P$ is ${\mathcal N}$-barbed bisimilar to $Q$, written
$P \wbbisim_{\mathcal N} Q$, if $P \rel{S}_{\mathcal N} Q$ for some ${\mathcal N}$-barbed bisimulation ${\mathcal S}_{\mathcal N}$.
\end{definition}

$\mathcal{R} \subseteq \pi \times \pi$

$P \mathcal{R} Q => \forall P'. P \red P' \Rightarrow \exists Q'. Q \red Q', P' \mathcal{R} Q'$

$P \vdash x \Rightarrow Q \vdash x$

\begin{mathpar}
  \inferrule*[lab=Out-barb]{x \nameeq y}{{y}!\langle{Q}\rangle \vdash x}
  \and
  \inferrule*[lab=Par-barb]{\mbox{$P\vdash x$ or $Q\vdash x$}}{\binpar{P}{Q} \vdash x}
\end{mathpar}

\subsubsection{Contexts}

One of the principle advantages of computational calculi like the
$\pi$-calculus is a well-defined notion of context,
contextual-equivalence and a correlation between
contextual-equivalence and notions of bisimulation. The notion of
context allows the decomposition of a process into (sub-)process and
its syntactic environment, its context. Thus, a context may be
thought of as a process with a ``hole'' (written $\Box$) in it. The
application of a context $M$ to a process $P$, written $M[P]$, is
tantamount to filling the hole in $M$ with $P$. In this paper we do
not need the full weight of this theory, but do make use of the notion
of context in the proof the main theorem. 

\begin{mathpar}
  \inferrule* [lab=summation] {} {{M_{M},M_{N}} \bc \Box \;|\; x.M_{A} \;|\; M_{M}+M_{N}}
  \and
  \inferrule* [lab=agent] {} {{M_{A}} \bc (\vec{x})M_{P} \;| \; \clift{P_0,\ldots,M_{P},\ldots,P_N}}
  \and \\
  \inferrule* [lab=process] {} {{M_{P}} \bc M_{N} \;| \;P|M_{P} }
\end{mathpar} 

\begin{mathpar}
  \inferrule* [lab=sychronization] {} {M_{N} \bc \Box \;|\; x?M_{F} \;|\; x!M_{C}}
  \and
  \inferrule* [lab=abstraction] {} {{M_{F}} \bc (x)M_{P} }
  \and
  \inferrule* [lab=concretion] {} {{M_{C}} \bc \langle M_{P} \rangle }
  \and \\
  \inferrule* [lab=process] {} {{M_{P}} \bc M_{N} \;| \;P|M_{P} }
\end{mathpar}

\begin{definition}[contextual application] Given a context $M$, and
  process $P$, we define the \emph{contextual application}, $M[P] :=
  M\{P/\Box\}$. That is, the contextual application of M to P is the
  substitution of $P$ for $\Box$ in $M$.
\end{definition}

$\meaningof{-} : L \to \mathcal{P}(\pi)$

\begin{mathpar}
  \inferrule* [lab=collection] {} {\meaningof{true} = \pi, \and \meaningof{~E} = \pi \setminus \meaningof{E}, \and \meaningof{E_{1} \& E_{2}} = \meaningof{E_{1}} \cap \meaningof{E_{2}}}
\end{mathpar}

\begin{mathpar}
  \inferrule* [lab=structure] {} {\meaningof{0} = \{ P \in \pi | P \equiv 0 \}, \and \\ \meaningof{E_1 | E_2} = \{ P \in \pi | P \equiv P_{1} | P_{2}, P_{1} \in \meaningof{E_{1}}, P_{2} \in \meaningof{E_2}\} }
\end{mathpar}

\begin{mathpar}
 \inferrule* [lab=behavior] {} {\meaningof{\langle a?b \rangle E} = \{ P \in \pi | P \equiv Q | u?(y)P', \\ \and \\\\ \and \\ \;\;\; u \in \meaningof{a}, \forall z.P'\{z/y\} \in \meaningof{E\{z/b\}}\}, \and \\ \meaningof{a!E} = \{ P \in \pi | P \equiv Q | x!\langle P' \rangle, x \in \meaningof{a} P' \in \meaningof{E}\} }
\end{mathpar}

\begin{mathpar}
 \inferrule* [lab=nominal] {} {\meaningof{\quotep{E}} = \{ \quotep{P} \in \quotep{\pi} | P \in \meaningof{E} \}, \and \meaningof{\quotep{P}} = \{ \quotep{Q} \in \quotep{\pi} | P \equiv Q \} \and \\ \meaningof{@\quotep{E}} = \{ P \in \pi | P \equiv @x, x \in \meaningof{E} \}}
\end{mathpar}

\begin{eqnarray*}
  \\
  \meaningof{-} : TS \to ST
\end{eqnarray*}

\begin{eqnarray*}
  \\
  L : TS \to ST
\end{eqnarray*}

\begin{eqnarray*}
  \\
  P \models E \iff P \in \meaningof{E}
\end{eqnarray*}

\begin{eqnarray*}
  P \approx_{L} Q \iff \forall E \in L. P \models E \iff Q \models E
\end{eqnarray*}

\begin{eqnarray*}
  P \approx_{K} Q
\end{eqnarray*}

\begin{eqnarray*}
  P \approx Q
\end{eqnarray*}

$\approx_{K} = \approx = \approx_{L}$

\subsubsection{Contextual duality}

Note that contexts extend the quotation operation to a family of
operations from processes to names. Given a context, $M$, we can
define a \emph{nominal context}, $\quotep{M}$ by $\quotep{M}[P] :=
\quotep{M[P]}$. To foreshadow what is to come we observe that these
operations enjoy a duality with processes very much like the duality
between vectors and maps from vectors to scalars.

Further, because the calculus is essentially higher-order, we have a
correspondence between contexts and processes. More specifically,
given a name $x$ and a context $M$ we can construct $M^{*}_{x}$ such
that 

\begin{mathpar}
  M^{*}_{x} | \lift{x}{P} \red M[P]
\end{mathpar}

namely,

\begin{mathpar}
  M^{*}_{x} := x?(u).M[\dropn{u}]
\end{mathpar}

The dependence of $M^{*}_{x}$ on a name makes it an abstraction, 

\begin{mathpar}
  M^{*} := (x)x?(u).M[\dropn{u}]
\end{mathpar}

\subsection{Additional notation}

It will sometimes be convenient to denote the process a name
quotes. We already have the notation $x = \quotep{P}$, but it will be
convenient to introduce an alternate notation, $\procn{x}$, when we
want to emphasize the connection to the use of the name. Note that, by
virtue of name equivalence, $\quotep{\procn{x}} \nameeq x$; so, the
notation is consistent with previous definitions.

Further, because names have structure it is possible to effect
substitutions on the basis of that structure. This means we need to
upgrade our notation for substitutions, which we accomplish by
adapting comprehension notation. Thus,

\begin{mathpar}
  P\{ y / x : x \in S \}
\end{mathpar}

is interpreted to mean the process derived from P by replacing (in a
capture-avoiding manner) each occurrence of $x$ in $S$ by $y$. For example,

\begin{mathpar}
  P\{ \quotep{\procn{x}|\procn{x}} / x : x \in \freenames{P} \}
\end{mathpar}

will replace each (occurrence) of a free name $x$ in $P$ by
$\quotep{\procn{x}|\procn{x}}$.

Also, we will avail ourselves of the notation $x^{L}$ and $x^{R}$ to
denote injections of a name into disjoint copies of the name
space. There are numerous ways to accomplish this. One example can be
found in \cite{MeredithR05}. This notation overloads to vectors of
names: $\vec{x}^{\pi} := (x_{i}^{\pi} \; : \; 0 \leq i < |\vec{x}| )$ where $\pi \in \{L,R\}$.

We also use $P^{\Box} := P|\Box$.

In \cite{MeredithR05} an interpretation of the new operator is
given. It turns out that there are several possible interpretations
all enjoying the requisite algebraic properties of the operator (see
\cite{milner91polyadicpi}). We will therefore make liberal use of
$(\nu\; \vec{x})P$.

% subsection the_syntax_and_semantics_of_the_notation_system (end)   

\input{qm2pi.qmops} 

\input{qm2pi.sterngerlach} 

\input{qm2pi.metric} 

% section concurrent_process_calculi (end)

%\input{qm2pi.proofsketch}

% section proof sketch (end)

%\input{qm2pi.slviaknots} 

% section spatial logic via knots (end)

\input{qm2pi.conclusion}

% section conclusion (end)

%\input{qm2pi.dtcodes} 

% section wiring algorithm (end)

\input{qm2pi.ack} 

% section acknowledgments (end)

\newpage


\bibliographystyle{plain}   
\bibliography{../../biblios/main.bib}

\input{qm2pi.rhodetails}

\end{document}



% section proof sketch (end)

%\section{Unlikely characters: spatial logic for
  knots}\label{sub:characteristic_formulae} % (fold)

Associated to the mobile process calculi are a family of logics known
as the Hennessy-Milner logics. These logics typically enjoy a
semantics interpreting formulae as sets of processes that when
factored through the encoding outlined above allows an identification
of classes of knots with logical formulae. In the context of this
encoding the sub-family known as the spatial logics \cite{CairesC03}
\cite{CairesC04} \cite{Caires04} are of particular interest providing
several important features for expressing and reasoning about
properties (i.e. classes) of knots. We hint here at how this may be done.

%\begin{description}
%\item [structural connectives] 
\subsubsection{Structural connectives} The spatial logics enjoy
structural connectives corresponding, at the logical level, to the
parallel composition ($P | Q$) and new name ($(\nu \; x)P$)
connectives for processes. As illustrated in the examples below, these
connectives are extremely expressive given the shape of our encoding.
%\item [decideable satisfaction]

\subsubsection{Decideable satisfaction}
In \cite{Caires04} the satisfaction relation is shown to be decideable
for a rich class of processes. It further turns out that the image of
the our encoding is a proper subset of that class. This result
provides the basis for an algorithm by which to search for knots
enjoying a given property.
%\item [characteristic formulae]

\subsubsection{Characteristic formulae}
In the same paper \cite{Caires04} , Caires presents a means of calculating
characteristic formulae, selecting equivalence classes of processes
up to a pre--specified depth limit on the support set of names. Composed with our
encoding, this characteristic formula can be used to select
characteristic formulae for knots.
%\end{description}

\subsubsection{Spatial logic formulae}

The grammar below (segmented for comprehension) summarizes the syntax
of spatial logic formulae. We employ illustrative examples in the
sequel to provide an intuitive understanding of their meaning
referring the reader to \cite{Caires04} for a more detailed explication
of the semantics.

\begin{mathpar}
  \inferrule* [lab=boolean] {} {{A,B} \bc T \;|\; \neg A \;|\; A \wedge B \;|\; \eta = \eta'}
  \and
  \inferrule* [lab=spatial] {} {|\; \pzero \;|\; A | B \;|\; x \text{\textregistered} A \;|\; \forall x . A \;|\;  H x . A}
  \and
  \inferrule* [lab=behavioral] {} {|\; \alpha . A}
  \and 
  \inferrule* [lab=recursion] {} {|\; X(\vec{u}) \;|\; \mu X(\vec{u}) . A}
  \and
  \inferrule* [lab=action] {} {\alpha \bc \langle x?(\vec{y}) \rangle \;|\; \langle x!(\vec{y}) \rangle \;|\; \langle \tau \rangle}
  \and 
  \inferrule* [lab=name] {} {\eta \bc x \;|\; \tau}
\end{mathpar} 

% subsection characteristic_formulae (end)   	 

\subsection{Example formulae}\label{sub:example_formulae_} % (fold)

\subsubsection{Crossing as formula.}
% 
% \begin{align*}
%   \frac{d}{dx} \sin x &= \cos x 
%   & \frac{d}{dx} e^x &= e^x \\
%   \frac{d}{dx} \cos x &= - \sin x 
%   & \frac{d}{dx} \log x &= \frac{1}{x} \\
% \end{align*} 

\begin{align*}
 \mu C(x_{0},x_{1},y_{0},y_{1},u).&(\langle x_{0}?(z) \rangle(\langle u! \rangle\langle y_{1}!z \rangle C(x_{0},x_{1},y_{0},y_{1},u)) & \\
  & \wedge \langle y_{1}?(z) \rangle (\langle u! \rangle \langle x_{0}!z \rangle C(x_{0},x_{1},y_{0},y_{1},u)) & \\
  & \wedge \langle x_{1}?(z) \rangle (\langle u? \rangle \langle y_{0}!z \rangle C(x_{0},x_{1},y_{0},y_{1},u)) & \\
  & \wedge \langle y_{0}?(z) \rangle (\langle u? \rangle \langle x_{1}!z \rangle C(x_{0},x_{1},y_{0},y_{1},u))) &
\end{align*}

The lexicographical similarity between the shape of this formulae and
the shape of definition of the process representing a crossing reveals
the intuitive meaning of this formulae. It describes the capabilities
of a process that has the right to represent a crossing. For example
it picks out processes that may perform an input on the port $x_0$ in
its initial menu of capabilities. What differentiates the formula
from the process, however, is that the crossing process is the
smallest candidate to satisfy the formula. Infinitely many other
processes -- with internal behavior hidden behind this interface, so
to speak -- also satisfy this formula. Even this simple formula,
then, can be seen to open a new view onto knots, providing a
computational interpretation of \emph{virtual} knots.

Note that this formula is derived by hand. A similar formula can be
derived by employing Caires' calculation of characteristic formula
\cite{Caires04} to the process representing a crossing. In light of
this discussion, we let
$\meaningof{C}_{\phi}(x0,x1,y0,y1,u)$ denote a formula specifying the
dynamics we wish to capture of a crossing. To guarantee we preserve
the shape of the interface and minimal semantics we demand that
$\meaningof{C}_{\phi}(x0,x1,y0,y1,u) \Rightarrow
\textbf{C}(x0,x1,y0,y1,u)$ where $\textbf{C}(x0,x1,y0,y1,u)$ denotes
the formula above.
                            
\subsubsection{Crossing number constraints.}
The moral content of the context lemma (Lemma \ref{context}) is that the notion of
``locality'' in the Reidemeister moves is effectively captured by the
parallel composition operator of the process calculus. This intuition
extends through the logic. Given a formula,
$\meaningof{C}_{\phi}(x0,x1,y0,y1,u)$, we can use the structural
connectives to specify constraints on crossing numbers, such as at
least $n$ crossings, or exactly $n$ crossings.
\begin{mathpar}
  \inferrule* [lab=at-least-n] {} { K^{\geq n}_{\phi}(\vec{xs},\vec{ys}) := \Pi_{i=0}^{n-1} Hu . \meaningof{C}_{\phi}(xs_i,ys_i,u) | T }
  \and 
  \inferrule* [lab=exactly-n] {} { K^{= n}_{\phi}(\vec{xs},\vec{ys}) := \Pi_{i=0}^{n-1} Hu . \meaningof{C}_{\phi}(xs_i,ys_i,u) | \neg (\forall x_0,y_0,x_1,y_1,u . \meaningof{C}_{\phi}(x_0,y_0,x_1,y_1,u) | T) }
\end{mathpar}

To round out this section, recall that the encoding of an $n$-crossing
knot decomposes into a parallel composition of $n$ \emph{copies} of a
crossing process together with a wiring harness. To specify different
knot classes with the same crossing number amounts to specifying
logical constraints on the wiring harness. In the interest of space,
we defer examples to a forthcoming paper. Suffice it to say that both
the conditions ``alternating knot'' and ``contains the tangle
corresponding to 5/3'' are expressible. For example, it is possible to
calculate the characteristic formula of a process corresponding to the
tangle 5/3 and conjoin it into the classifying formula via the
composition connective of the logic.

Finally, we wish to observe that it is entirely within reason to
contemplate a more domain-specific version of spatial logic tailored
to the shape of processes in the image of the encoding. Such a
domain-specific logic would have a better claim to the title formal
language of knot properties.

% subsection example_formulae_ (end)

% section knots_as_processes (end) 

% section spatial logic via knots (end)

\section{Conclusions and future work}

\paragraph{Testing physical space}
You, gentle reader, may wonder why of all the theorems to be proved
given this set up we pick the one above. In some sense it's hardly
central to quantum mechanics. We see it as central in the sense that
it firmly establishes a notion of physical space arising from a notion
of the equivalence of behavior. Relating bisimulation to a metric is a
big step forward, but one is faced with interpreting the relationship
of that metric space to something more physical. Quantum mechanical
notions of ``physical'' space are still far from intuitive, but by
relating this idea of distance as testing to calculations that predict
physical circumstances we are making a not insignificant step forward
toward an understanding of the physical space we inhabit as
essentially dynamic.

\paragraph{Effectivity and simulation}
One of the observations we have yet to make is that the entire program
spelled out here is effective. We have built various interpreters for
the reflective calculus at work in this interpretation. In principle,
then, we can simulate quantum mechanics on a computer. The place where
the simulation may lose fidelity is the infinitely branching summation
for the annihilator.

In this connection i also want to point out that the evaluation style
calculation of the inner product puts the non-determinism of the
summation right at the heart of measurement. This suggests that
Milner's original reduction-based formulation of the dynamics of his
calculi in terms of sums was not just notationally suggestive of a
notion of measure-and-continue but captured some significant part of
the physics.

\paragraph{Quantum continuations}
In light of this last observation i want to point out that the
predominant account of quantum mechanics is missing a key aspect of a
truly compositional story of the physical situation. In a real lab,
when a measurement is made the observation can be made to feed into
another device that then makes another measurement conditioned on the
results of the first. This means that after the superposition was
collapsed the entire experimental set up remained in
superposition. While QM offers a means of writing this down it doesn't
quite line up well with the well-trodden formulation of computation
and continuation that we see so succinctly expressed in Milner's
calculi. This suggests that there might be advantages to this account
of dynamics waiting to be explored.

\paragraph{Quantum logic}
In this connection, we also note that by virtue of having the
Hennessy-Milner construction, we can pull the construction through the
interpretation of QM. This gives us a natural candidate for a quantum
logic that enjoys an extremely tight connection with it's domain of
interpretation, making the construction much less ad hoc (rather it is
the image of functor!).

\paragraph{Quantum probabiity}
i have questions about the basis of the interpretation of inner
product as probability amplitude. In particular, using which
axiomatization of probability theory does the notion of probability
amplitude earn the right to be so dubbed? In other words, where is the
proof that the operation for calculating a probability amplitude (and
then squaring) satisfies the axioms of what it means to calculate a
probability? Even if such a proof exists (i have yet to find it in the
literature), i wonder if it might not be possible to turn things on
their heads. Can we view the calculation of the probability amplitude
as an axiomatization of probability? If so, then the definition we
give for calculating probability amplitude may provide the basis for
an \emph{effective} theory of probability.

\paragraph{Quantum vs ``biological'' information}
Finally, i want to conclude with a more philosophical observation. At
a recent workshop in which QM was a predominant topic i noticed
something about quantum information. The speaker was giving a riveting
discussion of axiomatic QM and showing how properties of ``no
cloning'' and ``no deleting'' emerged as consequences of the
axiomatization. Theorems of this form are necessary to give us a sense
of confidence that our axioms characterize the physical theory. What
struck me, though, was that if quantum information is neither erasable
nor replicable it is markedly different from \emph{life}. Two of the
things we know about life is that

\begin{itemize}
  \item it ends;
  \item to gain some measure of persistence, to transcend it's
    finitude it is imminently copyable.
\end{itemize}

Both of these qualities are summarized succinctly in the aphorism: all
flesh is grass. For me these two kinds of ``information'' -- call them
quantum and biological -- are end points on a spectrum of strategies
for persistence. At one end, we have those curious entities that enjoy
uniqueness and permanence; at the other, we have those who in the face
of a certain end and an uncertain present make a go of passing
something on. To me one of the more remarkable aspects of the latter
strategy is that in the presence of noise (and certain features of
copying) we get a kind of dynamism, a chance for improvement against a
given persistent condition.

% subsection other_calculi_other_bisimulations_and_geometry_as_behavior (end)




% section conclusion (end)

%\documentclass[12pt]{llncs}
%\documentclass{jktr}

\usepackage[pdftex]{hyperref}                   
\usepackage {listings}
\usepackage {mathpartir}
\usepackage{bcprules}
%\usepackage{listings}
                       
\usepackage{graphicx} 
%\usepackage[margins=2.5cm,nohead,nofoot]{geometry}
%\usepackage{geometry}
\usepackage{amsfonts}
\usepackage{amstext}
\usepackage{latexsym}
\usepackage{amssymb}
\usepackage{color}


%\include{myPreamble}
\include{qm2pi.local} 

%\ifpdf
%\usepackage[pdftex]{graphicx}
%\else
%\usepackage{graphicx}
%\fi

 % \ifpdf
%  \usepackage{pdfsync}
%  \if


%\title{Brief Article}
%\author{David F. Snyder}
%\author{L.G. Meredith}

%\address{Dept. of Math., Texas State University--San Marcos, San Marcos, TX 78666}
       
\pagestyle{empty}


\begin{document}

\lstset{language=[Objective]Caml,frame=shadowbox}

\input{qm2pi.front}

% section front matter (end)

\input{qm2pi.intro} 
 
% section introduction (end)

% \input{qm2pi.knotations} 

% section notation (end)

\input{qm2pi.process.calculi} 

% section concurrent_process_calculi_and_spatial_logics_ (end)
    
%\input{qm2pi.knots2pi} 

%\input{qm2pi.trefoil} 

%\input{qm2pi.mainthm} 

% subsection basic_interpretation (end)

%\input{qm2pi.rho.presentation} 
\subsection{The syntax and semantics of the notation system}\label{sub:the_syntax_and_semantics_of_the_notation_system} % (fold)

We now summarize a technical presentation of the calculus that
embodies our theory of dynamics. The typical presentation of such a
calculus follows the style of giving generators and relations on
them. The grammar, below, describing term constructors, freely
generates the set of processes, $\Proc$. This set is then quotiented
by a relation known as structural congruence and it is over this set
that the notion of dynamics is expressed. This presentation is
essentially that of \cite{MeredithR05} with the addition of
polyadicity and summation. For readability we have relegated some of
the technical subtleties to an appendix.

\subsubsection{Process grammar}\label{subsub:process_grammar}

\begin{mathpar}
  \inferrule* [lab=synchronization] {} {{M} \bc \pzero \;|\; x?F \;|\; x!C }
  \and
  \inferrule* [lab=abstraction] {} {{F} \bc (x)P}
  \and
  \inferrule* [lab=concretion] {} {{C} \bc \langle Q \rangle}
  \and
  \inferrule* [lab=process] {} {{P,Q} \bc M \;| \;P|Q \;|\; @{x}}
  \and
  \inferrule* [lab=name] {} {{x} \bc \quotep{P}}
\end{mathpar} 

Note that $\vec{x}$ (resp. $\vec{P}$) denotes a vector of names
(resp. processes) of length $|\vec{x}|$ (resp. $|\vec{P}|$). We adopt
the following useful abbreviations.

\begin{mathpar}
   x?(\vec{y}).P := x.(\vec{y})P \and  x\clift{\vec{P}} := x.\clift{\vec{P}}
   \and x!(y) := \lift{x}{\dropn{y}}
   \and \Pi_{i=0}^{n-1}P_i := P_0 | \ldots | P_{n-1}
\end{mathpar}

\subsubsection{Structural congruence}

\paragraph{Free and bound names and alpha-equivalence.} At the
core of structural equivalence is alpha-equivalence which identifies
process that are the same up to a change of variable. Formally, we
recognize the distinction between free and bound names. The free names
of a process, $\freenames{P}$, may be calculated recursively as
follows:

\begin{mathpar}
\freenames{\pzero} := \emptyset
  \and \\
  \freenames{x?(y).P} := \{ x \} \cup (\freenames{P} \setminus \{ y \})
  \and 
  \freenames{x!\langle P \rangle} := \{ x \} \cup \{ P \} 
  \and \\
  \freenames{P|Q} := \freenames{P} \cup \freenames{Q}
  \and \\
  \freenames{@{x}} := \{ x \}
\end{mathpar}

$\pi$
$\quotep{\pi}$

$\freenames{-} : \pi \to \mathcal{P}(\quotep{\pi})$

\begin{eqnarray*}
  \freenames{\pzero} & := & \emptyset \\
  \freenames{x?(y).P} & := & \{ x \} \cup (\freenames{P} \setminus \{ y \}) \\
  \freenames{x!\langle P \rangle} & := & \{ x \} \cup \{ P \} \\
  \freenames{P|Q} & := & \freenames{P} \cup \freenames{Q} \\
  \freenames{\dropn{x}} & := & \{ x \}
\end{eqnarray*}

The bound names of a process, $\boundnames{P}$, are those names occurring in $P$
that are not free. For example, in $x?(y).0$, the name $x$ is free, while $y$ is bound.

\begin{mathpar}
  \inferrule* [lab=monoidal-laws] {} { P|Q \equiv Q|P \and P|0 \equiv P \and P|(Q|R) \equiv (P|Q)|R }
\end{mathpar}

\begin{mathpar}
  \inferrule* [lab=alpha-equivalence] {} { (x)P \equiv (y)P\{y/x\} \and y \not\in \freenames{P} }
\end{mathpar}

\begin{definition}
Then two processes, $P,Q$, are alpha-equivalent if $P = Q\{\vec{y}/\vec{x}\}$ for
some $\vec{x} \in \boundnames{Q},\vec{y} \in \boundnames{P}$, where $Q\{\vec{y}/\vec{x}\}$
denotes the capture-avoiding substitution of $\vec{y}$ for $\vec{x}$ in $Q$.
\end{definition}

\begin{definition}
  The {\em structural congruence} \cite{SangiorgiWalker} , $\equiv$,
  between processes is the least congruence containing
  alpha-equivalence, satisfying the abelian monoid laws
  (associativity, commutativity and $\pzero$ as identity) for parallel
  composition $|$ and for summation $+$.
\end{definition}

\subsection{Name equivalence}

We take name equivalence, written $\nameeq$, to be the smallest
equivalence relation generated by the following rules.

\begin{mathpar}
\inferrule*[lab=Quote-drop]
{ }
{ \quotep{@{x}} \nameeq x }

\inferrule*[lab=Struct-equiv]
{ P \scong Q }
{ \quotep{P} \nameeq \quotep{Q} }
\end{mathpar}

The astute reader will have noticed that the mutual recursion of names
and processes imposes a mutual recursion on alpha-equivalence and
structural equivalence via name-equivalence. Fortunately, all of this
works out pleasantly and we may calculate in the natural way, free of
concern. The reader interested in the details is referred to the
appendix \ref{appendix:rho_details}.

\subsection{Substitution}

We use $\Proc$ for the set of processes, $\QProc$ for the set of
names, and $\id{\{}\vec{y} / \vec{x} \id{\}}$ to denote partial maps,
$s : \QProc \rightarrow \QProc$. A map, $s$ lifts, uniquely, to a map
on process terms, $\widehat{s} : \Proc \rightarrow \Proc$ by the
following equations.

\begin{mathpar}
  (0) \psubstp{Q}{P} := 0 \\
  (R \juxtap S) \psubstp{Q}{P}
  :=    
  (R)\psubstp{Q}{P} \juxtap (S) \psubstp{Q}{P} \\
  (x?(y).R) \psubstp{Q}{P}    
  :=    
  (x)\substp{Q}{P} (z)\concat( (R \psubstn{z}{y}) \psubstp{Q}{P} ) \\
  (\lift{x}{R}) \psubstp{Q}{P}  
  :=
  \lift{(x)\substp{Q}{P}}{ R \psubstp{Q}{P} } \\
%   (\dropn{x})  \psubstp{Q}{P}       
%   := 
%   \left\{ 
%     \begin{array}{ccc} 
%       \dropn{\quotep{Q}} & & x \nameeq \quotep{P} \\
%       \dropn{x} & & otherwise \\
%     \end{array}
%   \right. 
  (\dropn{x})  \psubstp{Q}{P}       
  := 
  \left\{ 
    \begin{array}{ccc} 
      Q & & x \nameeq \quotep{P} \\
      \dropn{x} & & otherwise \\
    \end{array}
  \right.
\end{mathpar}
 

where

\begin{eqnarray}
  (x)\id{\{} \lpquote Q \rpquote / \lpquote P \rpquote \id{\}}            = 
  \left\{ 
    \begin{array}{ccc}
      \lpquote Q \rpquote & & x \nameeq \lpquote P \rpquote \\
      x & & otherwise \\
    \end{array}
  \right. \nonumber
\end{eqnarray}

and $z$ is chosen distinct from $\quotep{P}$, $\quotep{Q}$, the free
names in $Q$, and all the names in $R$. Our $\alpha$-equivalence will
be built in the standard way from this substitution.

\begin{remark}\label{rem:no_self_referential_names}
  One consequence of these definitions is that $\forall P. \quotep{P}
  \not\in \freenames{P}$.
\end{remark}

\subsection{ Dynamic quote: an example }

Anticipating something of what's to come, consider applying the
substitution, $\widehat{\id{\{}u / z \id{\}}}$, to the following pair
of processes, $\lift{w}{y!(z)}$ and $w[ \lpquote y!(z) \rpquote ]$.

\begin{eqnarray}
	\lift{w}{y!(z)}\widehat{\id{\{}u / z \id{\}}}
		& = &
		\lift{w}{y!(u)} \nonumber\\
	w[ \lpquote y!(z) \rpquote ] \widehat{ \id{\{}u / z \id{\}} }
		& = &
		w[ \lpquote y!(z) \rpquote ] \nonumber
\end{eqnarray}

Because the body of the process between quotes is impervious to
substitution, we get radically different answers. In fact, by
examining the first process in an input context,
e.g. $x?(z).\lift{w}{y!(z)}$, we see that the process under the lift
operator may be shaped by prefixed inputs binding a name inside it. In
this sense, the lift operator will be seen as a way to dynamically
construct processes before reifying them as names.

Finally equipped with these standard features we can present the
dynamics of the calculus.

\subsubsection{Operational semantics} 

Finally, we introduce the computational dynamics. What marks these
algebras as distinct from other more traditionally studied algebraic
structures, e.g. vector spaces or polynomial rings, is the manner in
which dynamics is captured. In traditional structures, dynamics is typically
expressed through morphisms between such structures, as in linear maps
between vector spaces or morphisms between rings. In algebras
associated with the semantics of computation, the dynamics is
expressed as part of the algebraic structure itself, through a
reduction reduction relation typically denoted by $\red$. Below, we
give a recursive presentation of this relation for the calculus used
in the encoding.

$\red \subseteq \pi \times \pi$
$\red : \pi \to \mathcal{P}(\pi)$

\begin{mathpar}
  \inferrule* [lab=Comm] { \textsf{match}( x_{src}, x_{trgt} ) } { x_{trgt}?(y)P \; | \; x_{src}!\langle {Q} \rangle \red P\{\quotep{Q}/y}\} }
  \and \\
  \inferrule* [lab=Par] {{P} \red {P}'} {{{P} | {Q}} \red {{P}' | {Q}}}
  \and
  \inferrule* [lab=Equiv]{{{P} \scong {P}'} \andalso {{P}' \red {Q}'} \andalso {{Q}' \scong {Q}}}{{P} \red {Q}}
\end{mathpar}

\begin{eqnarray*}
  match_{\equiv} (\quotep{P},\quotep{Q}) & := & P \equiv Q \\
  match_{\dagger}(\quotep{P},\quotep{Q}) & := & \forall R. P|Q \red^{*} R => R \red^{*} 0 \\
  match_{K}(\quotep{P},\quotep{Q}) & := & K \mbox{ for some context } K
\end{eqnarray*}

$u?(x)P | u!\langle Q \rangle \red P\{\quotep{Q}/x\}$

%We write $\wred$ for $\red^*$, and $P\red$ if $\exists Q $ such that $ P \red Q$.
We write $P\red$ if $\exists Q $ such that $ P \red Q$ and $P\not\red$, otherwise.

\section{Replication}

As mentioned before, it is known that replication (and hence
recursion) can be implemented in a higher-order process algebra
\cite{SangiorgiWalker}. As our first example of calculation with the
machinery thus far presented we give the construction explicitly in
the {\rhoc}.

\begin{eqnarray}
	D_{x} & := & \prefix{x}{y}{(\binpar{\outputp{x}{y}}{@{y}})} \nonumber\\
	\bangp_{x}{P} & := & \binpar{{x}!\langle{\binpar{D_{x}}{P}}\rangle}{D_{x}} \nonumber
\end{eqnarray}

\begin{eqnarray}
	\bangp_{x}{P} & & \nonumber\\
	=
	& {x}!\langle{(\prefix{x}{y}{(\outputp{x}{y} | @{y})) | P}}\rangle 
	      | \prefix{x}{y}{(\outputp{x}{y} | @{y})} & \nonumber\\
	\red
	& (\outputp{x}{y} | @{y})\substn{\quotep{(\prefix{x}{y}{(@{y} | \outputp{x}{y})) | P}}}{y} & \nonumber\\
	=
	& \outputp{x}{\quotep{(\prefix{x}{y}{(\outputp{x}{y} | @{y})) | P}}}
	  | {(\prefix{x}{y}{(\outputp{x}{y} | @{y})) | P}} & \nonumber\\
	\red
	& \ldots & \nonumber\\
	\red^*
	& P | P | \ldots & \nonumber
\end{eqnarray}

Of course, this encoding, as an implementation, runs away, unfolding
$\bangp{P}$ eagerly. A lazier and more implementable replication
operator, restricted to input-guarded processes, may be obtained as follows.

\begin{eqnarray}
\bangp{\prefix{u}{v}{P}} 
	:= 
	\binpar{\lift{x}{\prefix{u}{v}{(\binpar{D(x)}{P})}}}{D(x)} \nonumber
\end{eqnarray}

\begin{remark}
  Note that the lazier definition still does not deal with summation
  or mixed summation (i.e. sums over input and output). The reader is
  invited to construct definitions of replication that deal with these
  features. 

  Further, the definitions are parameterized in a name, $x$. Can you,
  gentle reader, make a definition that eliminates this parameter and
  guarantees no accidental interaction between the replication
  machinery and the process being replicated -- i.e. no accidental
  sharing of names used by the process to get its work done and the
  name(s) used by the replication to effect copying. This latter
  revision of the definition of replication is crucial to obtaining
  the expected identity $!!P \sim !P$.
\end{remark}

\begin{remark}\label{rem:paradoxical_combinator}
  The reader familiar with the lambda calculus will have noticed the
  similarity between $D$ and the paradoxical combinator.

  [Ed. note: the existence of this seems to suggest we have to be more
  restrictive on the set of processes and names we admit if we are to
  support no-cloning.]
\end{remark}

\subsubsection{Bisimulation}

The computational dynamics gives rise to another kind of equivalence,
the equivalence of computational behavior. As previously mentioned
this is typically captured \emph{via} some form of bisimulation.

% The notion we use in this paper is weak barbed bisimulation
% \cite{milner91polyadicpi}.

The notion we use in this paper is derived from weak barbed
bisimulation \cite{milner91polyadicpi}. 

\begin{definition}
An \emph{observation relation}, $\downarrow_{\mathcal N}$, over a set
of names, $\mathcal N$, is the smallest relation satisfying the rules
below.

\infrule[Out-barb]{y \in {\mathcal N}, \; x \nameeq y}
		  {\outputp{x}{v} \downarrow_{\mathcal N} x}
\infrule[Par-barb]{\mbox{$P\downarrow_{\mathcal N} x$ or $Q\downarrow_{\mathcal N} x$}}
		  {\binpar{P}{Q} \downarrow_{\mathcal N} x}

We write $P \Downarrow_{\mathcal N} x$ if there is $Q$ such that 
$P \wred Q$ and $Q \downarrow_{\mathcal N} x$.
\end{definition}

\begin{definition}
%\label{def.bbisim}
An  ${\mathcal N}$-\emph{barbed bisimulation} over a set of names, ${\mathcal N}$, is a symmetric binary relation 
${\mathcal S}_{\mathcal N}$ between agents such that $P\rel{S}_{\mathcal N}Q$ implies:
\begin{enumerate}
\item If $P \red P'$ then $Q \wred Q'$ and $P'\rel{S}_{\mathcal N} Q'$.
\item If $P\downarrow_{\mathcal N} x$, then $Q\Downarrow_{\mathcal N} x$.
\end{enumerate}
$P$ is ${\mathcal N}$-barbed bisimilar to $Q$, written
$P \wbbisim_{\mathcal N} Q$, if $P \rel{S}_{\mathcal N} Q$ for some ${\mathcal N}$-barbed bisimulation ${\mathcal S}_{\mathcal N}$.
\end{definition}

$\mathcal{R} \subseteq \pi \times \pi$

$P \mathcal{R} Q => \forall P'. P \red P' \Rightarrow \exists Q'. Q \red Q', P' \mathcal{R} Q'$

$P \vdash x \Rightarrow Q \vdash x$

\begin{mathpar}
  \inferrule*[lab=Out-barb]{x \nameeq y}{{y}!\langle{Q}\rangle \vdash x}
  \and
  \inferrule*[lab=Par-barb]{\mbox{$P\vdash x$ or $Q\vdash x$}}{\binpar{P}{Q} \vdash x}
\end{mathpar}

\subsubsection{Contexts}

One of the principle advantages of computational calculi like the
$\pi$-calculus is a well-defined notion of context,
contextual-equivalence and a correlation between
contextual-equivalence and notions of bisimulation. The notion of
context allows the decomposition of a process into (sub-)process and
its syntactic environment, its context. Thus, a context may be
thought of as a process with a ``hole'' (written $\Box$) in it. The
application of a context $M$ to a process $P$, written $M[P]$, is
tantamount to filling the hole in $M$ with $P$. In this paper we do
not need the full weight of this theory, but do make use of the notion
of context in the proof the main theorem. 

\begin{mathpar}
  \inferrule* [lab=summation] {} {{M_{M},M_{N}} \bc \Box \;|\; x.M_{A} \;|\; M_{M}+M_{N}}
  \and
  \inferrule* [lab=agent] {} {{M_{A}} \bc (\vec{x})M_{P} \;| \; \clift{P_0,\ldots,M_{P},\ldots,P_N}}
  \and \\
  \inferrule* [lab=process] {} {{M_{P}} \bc M_{N} \;| \;P|M_{P} }
\end{mathpar} 

\begin{mathpar}
  \inferrule* [lab=sychronization] {} {M_{N} \bc \Box \;|\; x?M_{F} \;|\; x!M_{C}}
  \and
  \inferrule* [lab=abstraction] {} {{M_{F}} \bc (x)M_{P} }
  \and
  \inferrule* [lab=concretion] {} {{M_{C}} \bc \langle M_{P} \rangle }
  \and \\
  \inferrule* [lab=process] {} {{M_{P}} \bc M_{N} \;| \;P|M_{P} }
\end{mathpar}

\begin{definition}[contextual application] Given a context $M$, and
  process $P$, we define the \emph{contextual application}, $M[P] :=
  M\{P/\Box\}$. That is, the contextual application of M to P is the
  substitution of $P$ for $\Box$ in $M$.
\end{definition}

$\meaningof{-} : L \to \mathcal{P}(\pi)$

\begin{mathpar}
  \inferrule* [lab=collection] {} {\meaningof{true} = \pi, \and \meaningof{~E} = \pi \setminus \meaningof{E}, \and \meaningof{E_{1} \& E_{2}} = \meaningof{E_{1}} \cap \meaningof{E_{2}}}
\end{mathpar}

\begin{mathpar}
  \inferrule* [lab=structure] {} {\meaningof{0} = \{ P \in \pi | P \equiv 0 \}, \and \\ \meaningof{E_1 | E_2} = \{ P \in \pi | P \equiv P_{1} | P_{2}, P_{1} \in \meaningof{E_{1}}, P_{2} \in \meaningof{E_2}\} }
\end{mathpar}

\begin{mathpar}
 \inferrule* [lab=behavior] {} {\meaningof{\langle a?b \rangle E} = \{ P \in \pi | P \equiv Q | u?(y)P', \\ \and \\\\ \and \\ \;\;\; u \in \meaningof{a}, \forall z.P'\{z/y\} \in \meaningof{E\{z/b\}}\}, \and \\ \meaningof{a!E} = \{ P \in \pi | P \equiv Q | x!\langle P' \rangle, x \in \meaningof{a} P' \in \meaningof{E}\} }
\end{mathpar}

\begin{mathpar}
 \inferrule* [lab=nominal] {} {\meaningof{\quotep{E}} = \{ \quotep{P} \in \quotep{\pi} | P \in \meaningof{E} \}, \and \meaningof{\quotep{P}} = \{ \quotep{Q} \in \quotep{\pi} | P \equiv Q \} \and \\ \meaningof{@\quotep{E}} = \{ P \in \pi | P \equiv @x, x \in \meaningof{E} \}}
\end{mathpar}

\begin{eqnarray*}
  \\
  \meaningof{-} : TS \to ST
\end{eqnarray*}

\begin{eqnarray*}
  \\
  L : TS \to ST
\end{eqnarray*}

\begin{eqnarray*}
  \\
  P \models E \iff P \in \meaningof{E}
\end{eqnarray*}

\begin{eqnarray*}
  P \approx_{L} Q \iff \forall E \in L. P \models E \iff Q \models E
\end{eqnarray*}

\begin{eqnarray*}
  P \approx_{K} Q
\end{eqnarray*}

\begin{eqnarray*}
  P \approx Q
\end{eqnarray*}

$\approx_{K} = \approx = \approx_{L}$

\subsubsection{Contextual duality}

Note that contexts extend the quotation operation to a family of
operations from processes to names. Given a context, $M$, we can
define a \emph{nominal context}, $\quotep{M}$ by $\quotep{M}[P] :=
\quotep{M[P]}$. To foreshadow what is to come we observe that these
operations enjoy a duality with processes very much like the duality
between vectors and maps from vectors to scalars.

Further, because the calculus is essentially higher-order, we have a
correspondence between contexts and processes. More specifically,
given a name $x$ and a context $M$ we can construct $M^{*}_{x}$ such
that 

\begin{mathpar}
  M^{*}_{x} | \lift{x}{P} \red M[P]
\end{mathpar}

namely,

\begin{mathpar}
  M^{*}_{x} := x?(u).M[\dropn{u}]
\end{mathpar}

The dependence of $M^{*}_{x}$ on a name makes it an abstraction, 

\begin{mathpar}
  M^{*} := (x)x?(u).M[\dropn{u}]
\end{mathpar}

\subsection{Additional notation}

It will sometimes be convenient to denote the process a name
quotes. We already have the notation $x = \quotep{P}$, but it will be
convenient to introduce an alternate notation, $\procn{x}$, when we
want to emphasize the connection to the use of the name. Note that, by
virtue of name equivalence, $\quotep{\procn{x}} \nameeq x$; so, the
notation is consistent with previous definitions.

Further, because names have structure it is possible to effect
substitutions on the basis of that structure. This means we need to
upgrade our notation for substitutions, which we accomplish by
adapting comprehension notation. Thus,

\begin{mathpar}
  P\{ y / x : x \in S \}
\end{mathpar}

is interpreted to mean the process derived from P by replacing (in a
capture-avoiding manner) each occurrence of $x$ in $S$ by $y$. For example,

\begin{mathpar}
  P\{ \quotep{\procn{x}|\procn{x}} / x : x \in \freenames{P} \}
\end{mathpar}

will replace each (occurrence) of a free name $x$ in $P$ by
$\quotep{\procn{x}|\procn{x}}$.

Also, we will avail ourselves of the notation $x^{L}$ and $x^{R}$ to
denote injections of a name into disjoint copies of the name
space. There are numerous ways to accomplish this. One example can be
found in \cite{MeredithR05}. This notation overloads to vectors of
names: $\vec{x}^{\pi} := (x_{i}^{\pi} \; : \; 0 \leq i < |\vec{x}| )$ where $\pi \in \{L,R\}$.

We also use $P^{\Box} := P|\Box$.

In \cite{MeredithR05} an interpretation of the new operator is
given. It turns out that there are several possible interpretations
all enjoying the requisite algebraic properties of the operator (see
\cite{milner91polyadicpi}). We will therefore make liberal use of
$(\nu\; \vec{x})P$.

% subsection the_syntax_and_semantics_of_the_notation_system (end)   

\input{qm2pi.qmops} 

\input{qm2pi.sterngerlach} 

\input{qm2pi.metric} 

% section concurrent_process_calculi (end)

%\input{qm2pi.proofsketch}

% section proof sketch (end)

%\input{qm2pi.slviaknots} 

% section spatial logic via knots (end)

\input{qm2pi.conclusion}

% section conclusion (end)

%\input{qm2pi.dtcodes} 

% section wiring algorithm (end)

\input{qm2pi.ack} 

% section acknowledgments (end)

\newpage


\bibliographystyle{plain}   
\bibliography{../../biblios/main.bib}

\input{qm2pi.rhodetails}

\end{document}

 

% section wiring algorithm (end)

\documentclass[12pt]{llncs}
%\documentclass{jktr}

\usepackage[pdftex]{hyperref}                   
\usepackage {listings}
\usepackage {mathpartir}
\usepackage{bcprules}
%\usepackage{listings}
                       
\usepackage{graphicx} 
%\usepackage[margins=2.5cm,nohead,nofoot]{geometry}
%\usepackage{geometry}
\usepackage{amsfonts}
\usepackage{amstext}
\usepackage{latexsym}
\usepackage{amssymb}
\usepackage{color}


%\include{myPreamble}
\include{qm2pi.local} 

%\ifpdf
%\usepackage[pdftex]{graphicx}
%\else
%\usepackage{graphicx}
%\fi

 % \ifpdf
%  \usepackage{pdfsync}
%  \if


%\title{Brief Article}
%\author{David F. Snyder}
%\author{L.G. Meredith}

%\address{Dept. of Math., Texas State University--San Marcos, San Marcos, TX 78666}
       
\pagestyle{empty}


\begin{document}

\lstset{language=[Objective]Caml,frame=shadowbox}

\input{qm2pi.front}

% section front matter (end)

\input{qm2pi.intro} 
 
% section introduction (end)

% \input{qm2pi.knotations} 

% section notation (end)

\input{qm2pi.process.calculi} 

% section concurrent_process_calculi_and_spatial_logics_ (end)
    
%\input{qm2pi.knots2pi} 

%\input{qm2pi.trefoil} 

%\input{qm2pi.mainthm} 

% subsection basic_interpretation (end)

%\input{qm2pi.rho.presentation} 
\subsection{The syntax and semantics of the notation system}\label{sub:the_syntax_and_semantics_of_the_notation_system} % (fold)

We now summarize a technical presentation of the calculus that
embodies our theory of dynamics. The typical presentation of such a
calculus follows the style of giving generators and relations on
them. The grammar, below, describing term constructors, freely
generates the set of processes, $\Proc$. This set is then quotiented
by a relation known as structural congruence and it is over this set
that the notion of dynamics is expressed. This presentation is
essentially that of \cite{MeredithR05} with the addition of
polyadicity and summation. For readability we have relegated some of
the technical subtleties to an appendix.

\subsubsection{Process grammar}\label{subsub:process_grammar}

\begin{mathpar}
  \inferrule* [lab=synchronization] {} {{M} \bc \pzero \;|\; x?F \;|\; x!C }
  \and
  \inferrule* [lab=abstraction] {} {{F} \bc (x)P}
  \and
  \inferrule* [lab=concretion] {} {{C} \bc \langle Q \rangle}
  \and
  \inferrule* [lab=process] {} {{P,Q} \bc M \;| \;P|Q \;|\; @{x}}
  \and
  \inferrule* [lab=name] {} {{x} \bc \quotep{P}}
\end{mathpar} 

Note that $\vec{x}$ (resp. $\vec{P}$) denotes a vector of names
(resp. processes) of length $|\vec{x}|$ (resp. $|\vec{P}|$). We adopt
the following useful abbreviations.

\begin{mathpar}
   x?(\vec{y}).P := x.(\vec{y})P \and  x\clift{\vec{P}} := x.\clift{\vec{P}}
   \and x!(y) := \lift{x}{\dropn{y}}
   \and \Pi_{i=0}^{n-1}P_i := P_0 | \ldots | P_{n-1}
\end{mathpar}

\subsubsection{Structural congruence}

\paragraph{Free and bound names and alpha-equivalence.} At the
core of structural equivalence is alpha-equivalence which identifies
process that are the same up to a change of variable. Formally, we
recognize the distinction between free and bound names. The free names
of a process, $\freenames{P}$, may be calculated recursively as
follows:

\begin{mathpar}
\freenames{\pzero} := \emptyset
  \and \\
  \freenames{x?(y).P} := \{ x \} \cup (\freenames{P} \setminus \{ y \})
  \and 
  \freenames{x!\langle P \rangle} := \{ x \} \cup \{ P \} 
  \and \\
  \freenames{P|Q} := \freenames{P} \cup \freenames{Q}
  \and \\
  \freenames{@{x}} := \{ x \}
\end{mathpar}

$\pi$
$\quotep{\pi}$

$\freenames{-} : \pi \to \mathcal{P}(\quotep{\pi})$

\begin{eqnarray*}
  \freenames{\pzero} & := & \emptyset \\
  \freenames{x?(y).P} & := & \{ x \} \cup (\freenames{P} \setminus \{ y \}) \\
  \freenames{x!\langle P \rangle} & := & \{ x \} \cup \{ P \} \\
  \freenames{P|Q} & := & \freenames{P} \cup \freenames{Q} \\
  \freenames{\dropn{x}} & := & \{ x \}
\end{eqnarray*}

The bound names of a process, $\boundnames{P}$, are those names occurring in $P$
that are not free. For example, in $x?(y).0$, the name $x$ is free, while $y$ is bound.

\begin{mathpar}
  \inferrule* [lab=monoidal-laws] {} { P|Q \equiv Q|P \and P|0 \equiv P \and P|(Q|R) \equiv (P|Q)|R }
\end{mathpar}

\begin{mathpar}
  \inferrule* [lab=alpha-equivalence] {} { (x)P \equiv (y)P\{y/x\} \and y \not\in \freenames{P} }
\end{mathpar}

\begin{definition}
Then two processes, $P,Q$, are alpha-equivalent if $P = Q\{\vec{y}/\vec{x}\}$ for
some $\vec{x} \in \boundnames{Q},\vec{y} \in \boundnames{P}$, where $Q\{\vec{y}/\vec{x}\}$
denotes the capture-avoiding substitution of $\vec{y}$ for $\vec{x}$ in $Q$.
\end{definition}

\begin{definition}
  The {\em structural congruence} \cite{SangiorgiWalker} , $\equiv$,
  between processes is the least congruence containing
  alpha-equivalence, satisfying the abelian monoid laws
  (associativity, commutativity and $\pzero$ as identity) for parallel
  composition $|$ and for summation $+$.
\end{definition}

\subsection{Name equivalence}

We take name equivalence, written $\nameeq$, to be the smallest
equivalence relation generated by the following rules.

\begin{mathpar}
\inferrule*[lab=Quote-drop]
{ }
{ \quotep{@{x}} \nameeq x }

\inferrule*[lab=Struct-equiv]
{ P \scong Q }
{ \quotep{P} \nameeq \quotep{Q} }
\end{mathpar}

The astute reader will have noticed that the mutual recursion of names
and processes imposes a mutual recursion on alpha-equivalence and
structural equivalence via name-equivalence. Fortunately, all of this
works out pleasantly and we may calculate in the natural way, free of
concern. The reader interested in the details is referred to the
appendix \ref{appendix:rho_details}.

\subsection{Substitution}

We use $\Proc$ for the set of processes, $\QProc$ for the set of
names, and $\id{\{}\vec{y} / \vec{x} \id{\}}$ to denote partial maps,
$s : \QProc \rightarrow \QProc$. A map, $s$ lifts, uniquely, to a map
on process terms, $\widehat{s} : \Proc \rightarrow \Proc$ by the
following equations.

\begin{mathpar}
  (0) \psubstp{Q}{P} := 0 \\
  (R \juxtap S) \psubstp{Q}{P}
  :=    
  (R)\psubstp{Q}{P} \juxtap (S) \psubstp{Q}{P} \\
  (x?(y).R) \psubstp{Q}{P}    
  :=    
  (x)\substp{Q}{P} (z)\concat( (R \psubstn{z}{y}) \psubstp{Q}{P} ) \\
  (\lift{x}{R}) \psubstp{Q}{P}  
  :=
  \lift{(x)\substp{Q}{P}}{ R \psubstp{Q}{P} } \\
%   (\dropn{x})  \psubstp{Q}{P}       
%   := 
%   \left\{ 
%     \begin{array}{ccc} 
%       \dropn{\quotep{Q}} & & x \nameeq \quotep{P} \\
%       \dropn{x} & & otherwise \\
%     \end{array}
%   \right. 
  (\dropn{x})  \psubstp{Q}{P}       
  := 
  \left\{ 
    \begin{array}{ccc} 
      Q & & x \nameeq \quotep{P} \\
      \dropn{x} & & otherwise \\
    \end{array}
  \right.
\end{mathpar}
 

where

\begin{eqnarray}
  (x)\id{\{} \lpquote Q \rpquote / \lpquote P \rpquote \id{\}}            = 
  \left\{ 
    \begin{array}{ccc}
      \lpquote Q \rpquote & & x \nameeq \lpquote P \rpquote \\
      x & & otherwise \\
    \end{array}
  \right. \nonumber
\end{eqnarray}

and $z$ is chosen distinct from $\quotep{P}$, $\quotep{Q}$, the free
names in $Q$, and all the names in $R$. Our $\alpha$-equivalence will
be built in the standard way from this substitution.

\begin{remark}\label{rem:no_self_referential_names}
  One consequence of these definitions is that $\forall P. \quotep{P}
  \not\in \freenames{P}$.
\end{remark}

\subsection{ Dynamic quote: an example }

Anticipating something of what's to come, consider applying the
substitution, $\widehat{\id{\{}u / z \id{\}}}$, to the following pair
of processes, $\lift{w}{y!(z)}$ and $w[ \lpquote y!(z) \rpquote ]$.

\begin{eqnarray}
	\lift{w}{y!(z)}\widehat{\id{\{}u / z \id{\}}}
		& = &
		\lift{w}{y!(u)} \nonumber\\
	w[ \lpquote y!(z) \rpquote ] \widehat{ \id{\{}u / z \id{\}} }
		& = &
		w[ \lpquote y!(z) \rpquote ] \nonumber
\end{eqnarray}

Because the body of the process between quotes is impervious to
substitution, we get radically different answers. In fact, by
examining the first process in an input context,
e.g. $x?(z).\lift{w}{y!(z)}$, we see that the process under the lift
operator may be shaped by prefixed inputs binding a name inside it. In
this sense, the lift operator will be seen as a way to dynamically
construct processes before reifying them as names.

Finally equipped with these standard features we can present the
dynamics of the calculus.

\subsubsection{Operational semantics} 

Finally, we introduce the computational dynamics. What marks these
algebras as distinct from other more traditionally studied algebraic
structures, e.g. vector spaces or polynomial rings, is the manner in
which dynamics is captured. In traditional structures, dynamics is typically
expressed through morphisms between such structures, as in linear maps
between vector spaces or morphisms between rings. In algebras
associated with the semantics of computation, the dynamics is
expressed as part of the algebraic structure itself, through a
reduction reduction relation typically denoted by $\red$. Below, we
give a recursive presentation of this relation for the calculus used
in the encoding.

$\red \subseteq \pi \times \pi$
$\red : \pi \to \mathcal{P}(\pi)$

\begin{mathpar}
  \inferrule* [lab=Comm] { \textsf{match}( x_{src}, x_{trgt} ) } { x_{trgt}?(y)P \; | \; x_{src}!\langle {Q} \rangle \red P\{\quotep{Q}/y}\} }
  \and \\
  \inferrule* [lab=Par] {{P} \red {P}'} {{{P} | {Q}} \red {{P}' | {Q}}}
  \and
  \inferrule* [lab=Equiv]{{{P} \scong {P}'} \andalso {{P}' \red {Q}'} \andalso {{Q}' \scong {Q}}}{{P} \red {Q}}
\end{mathpar}

\begin{eqnarray*}
  match_{\equiv} (\quotep{P},\quotep{Q}) & := & P \equiv Q \\
  match_{\dagger}(\quotep{P},\quotep{Q}) & := & \forall R. P|Q \red^{*} R => R \red^{*} 0 \\
  match_{K}(\quotep{P},\quotep{Q}) & := & K \mbox{ for some context } K
\end{eqnarray*}

$u?(x)P | u!\langle Q \rangle \red P\{\quotep{Q}/x\}$

%We write $\wred$ for $\red^*$, and $P\red$ if $\exists Q $ such that $ P \red Q$.
We write $P\red$ if $\exists Q $ such that $ P \red Q$ and $P\not\red$, otherwise.

\section{Replication}

As mentioned before, it is known that replication (and hence
recursion) can be implemented in a higher-order process algebra
\cite{SangiorgiWalker}. As our first example of calculation with the
machinery thus far presented we give the construction explicitly in
the {\rhoc}.

\begin{eqnarray}
	D_{x} & := & \prefix{x}{y}{(\binpar{\outputp{x}{y}}{@{y}})} \nonumber\\
	\bangp_{x}{P} & := & \binpar{{x}!\langle{\binpar{D_{x}}{P}}\rangle}{D_{x}} \nonumber
\end{eqnarray}

\begin{eqnarray}
	\bangp_{x}{P} & & \nonumber\\
	=
	& {x}!\langle{(\prefix{x}{y}{(\outputp{x}{y} | @{y})) | P}}\rangle 
	      | \prefix{x}{y}{(\outputp{x}{y} | @{y})} & \nonumber\\
	\red
	& (\outputp{x}{y} | @{y})\substn{\quotep{(\prefix{x}{y}{(@{y} | \outputp{x}{y})) | P}}}{y} & \nonumber\\
	=
	& \outputp{x}{\quotep{(\prefix{x}{y}{(\outputp{x}{y} | @{y})) | P}}}
	  | {(\prefix{x}{y}{(\outputp{x}{y} | @{y})) | P}} & \nonumber\\
	\red
	& \ldots & \nonumber\\
	\red^*
	& P | P | \ldots & \nonumber
\end{eqnarray}

Of course, this encoding, as an implementation, runs away, unfolding
$\bangp{P}$ eagerly. A lazier and more implementable replication
operator, restricted to input-guarded processes, may be obtained as follows.

\begin{eqnarray}
\bangp{\prefix{u}{v}{P}} 
	:= 
	\binpar{\lift{x}{\prefix{u}{v}{(\binpar{D(x)}{P})}}}{D(x)} \nonumber
\end{eqnarray}

\begin{remark}
  Note that the lazier definition still does not deal with summation
  or mixed summation (i.e. sums over input and output). The reader is
  invited to construct definitions of replication that deal with these
  features. 

  Further, the definitions are parameterized in a name, $x$. Can you,
  gentle reader, make a definition that eliminates this parameter and
  guarantees no accidental interaction between the replication
  machinery and the process being replicated -- i.e. no accidental
  sharing of names used by the process to get its work done and the
  name(s) used by the replication to effect copying. This latter
  revision of the definition of replication is crucial to obtaining
  the expected identity $!!P \sim !P$.
\end{remark}

\begin{remark}\label{rem:paradoxical_combinator}
  The reader familiar with the lambda calculus will have noticed the
  similarity between $D$ and the paradoxical combinator.

  [Ed. note: the existence of this seems to suggest we have to be more
  restrictive on the set of processes and names we admit if we are to
  support no-cloning.]
\end{remark}

\subsubsection{Bisimulation}

The computational dynamics gives rise to another kind of equivalence,
the equivalence of computational behavior. As previously mentioned
this is typically captured \emph{via} some form of bisimulation.

% The notion we use in this paper is weak barbed bisimulation
% \cite{milner91polyadicpi}.

The notion we use in this paper is derived from weak barbed
bisimulation \cite{milner91polyadicpi}. 

\begin{definition}
An \emph{observation relation}, $\downarrow_{\mathcal N}$, over a set
of names, $\mathcal N$, is the smallest relation satisfying the rules
below.

\infrule[Out-barb]{y \in {\mathcal N}, \; x \nameeq y}
		  {\outputp{x}{v} \downarrow_{\mathcal N} x}
\infrule[Par-barb]{\mbox{$P\downarrow_{\mathcal N} x$ or $Q\downarrow_{\mathcal N} x$}}
		  {\binpar{P}{Q} \downarrow_{\mathcal N} x}

We write $P \Downarrow_{\mathcal N} x$ if there is $Q$ such that 
$P \wred Q$ and $Q \downarrow_{\mathcal N} x$.
\end{definition}

\begin{definition}
%\label{def.bbisim}
An  ${\mathcal N}$-\emph{barbed bisimulation} over a set of names, ${\mathcal N}$, is a symmetric binary relation 
${\mathcal S}_{\mathcal N}$ between agents such that $P\rel{S}_{\mathcal N}Q$ implies:
\begin{enumerate}
\item If $P \red P'$ then $Q \wred Q'$ and $P'\rel{S}_{\mathcal N} Q'$.
\item If $P\downarrow_{\mathcal N} x$, then $Q\Downarrow_{\mathcal N} x$.
\end{enumerate}
$P$ is ${\mathcal N}$-barbed bisimilar to $Q$, written
$P \wbbisim_{\mathcal N} Q$, if $P \rel{S}_{\mathcal N} Q$ for some ${\mathcal N}$-barbed bisimulation ${\mathcal S}_{\mathcal N}$.
\end{definition}

$\mathcal{R} \subseteq \pi \times \pi$

$P \mathcal{R} Q => \forall P'. P \red P' \Rightarrow \exists Q'. Q \red Q', P' \mathcal{R} Q'$

$P \vdash x \Rightarrow Q \vdash x$

\begin{mathpar}
  \inferrule*[lab=Out-barb]{x \nameeq y}{{y}!\langle{Q}\rangle \vdash x}
  \and
  \inferrule*[lab=Par-barb]{\mbox{$P\vdash x$ or $Q\vdash x$}}{\binpar{P}{Q} \vdash x}
\end{mathpar}

\subsubsection{Contexts}

One of the principle advantages of computational calculi like the
$\pi$-calculus is a well-defined notion of context,
contextual-equivalence and a correlation between
contextual-equivalence and notions of bisimulation. The notion of
context allows the decomposition of a process into (sub-)process and
its syntactic environment, its context. Thus, a context may be
thought of as a process with a ``hole'' (written $\Box$) in it. The
application of a context $M$ to a process $P$, written $M[P]$, is
tantamount to filling the hole in $M$ with $P$. In this paper we do
not need the full weight of this theory, but do make use of the notion
of context in the proof the main theorem. 

\begin{mathpar}
  \inferrule* [lab=summation] {} {{M_{M},M_{N}} \bc \Box \;|\; x.M_{A} \;|\; M_{M}+M_{N}}
  \and
  \inferrule* [lab=agent] {} {{M_{A}} \bc (\vec{x})M_{P} \;| \; \clift{P_0,\ldots,M_{P},\ldots,P_N}}
  \and \\
  \inferrule* [lab=process] {} {{M_{P}} \bc M_{N} \;| \;P|M_{P} }
\end{mathpar} 

\begin{mathpar}
  \inferrule* [lab=sychronization] {} {M_{N} \bc \Box \;|\; x?M_{F} \;|\; x!M_{C}}
  \and
  \inferrule* [lab=abstraction] {} {{M_{F}} \bc (x)M_{P} }
  \and
  \inferrule* [lab=concretion] {} {{M_{C}} \bc \langle M_{P} \rangle }
  \and \\
  \inferrule* [lab=process] {} {{M_{P}} \bc M_{N} \;| \;P|M_{P} }
\end{mathpar}

\begin{definition}[contextual application] Given a context $M$, and
  process $P$, we define the \emph{contextual application}, $M[P] :=
  M\{P/\Box\}$. That is, the contextual application of M to P is the
  substitution of $P$ for $\Box$ in $M$.
\end{definition}

$\meaningof{-} : L \to \mathcal{P}(\pi)$

\begin{mathpar}
  \inferrule* [lab=collection] {} {\meaningof{true} = \pi, \and \meaningof{~E} = \pi \setminus \meaningof{E}, \and \meaningof{E_{1} \& E_{2}} = \meaningof{E_{1}} \cap \meaningof{E_{2}}}
\end{mathpar}

\begin{mathpar}
  \inferrule* [lab=structure] {} {\meaningof{0} = \{ P \in \pi | P \equiv 0 \}, \and \\ \meaningof{E_1 | E_2} = \{ P \in \pi | P \equiv P_{1} | P_{2}, P_{1} \in \meaningof{E_{1}}, P_{2} \in \meaningof{E_2}\} }
\end{mathpar}

\begin{mathpar}
 \inferrule* [lab=behavior] {} {\meaningof{\langle a?b \rangle E} = \{ P \in \pi | P \equiv Q | u?(y)P', \\ \and \\\\ \and \\ \;\;\; u \in \meaningof{a}, \forall z.P'\{z/y\} \in \meaningof{E\{z/b\}}\}, \and \\ \meaningof{a!E} = \{ P \in \pi | P \equiv Q | x!\langle P' \rangle, x \in \meaningof{a} P' \in \meaningof{E}\} }
\end{mathpar}

\begin{mathpar}
 \inferrule* [lab=nominal] {} {\meaningof{\quotep{E}} = \{ \quotep{P} \in \quotep{\pi} | P \in \meaningof{E} \}, \and \meaningof{\quotep{P}} = \{ \quotep{Q} \in \quotep{\pi} | P \equiv Q \} \and \\ \meaningof{@\quotep{E}} = \{ P \in \pi | P \equiv @x, x \in \meaningof{E} \}}
\end{mathpar}

\begin{eqnarray*}
  \\
  \meaningof{-} : TS \to ST
\end{eqnarray*}

\begin{eqnarray*}
  \\
  L : TS \to ST
\end{eqnarray*}

\begin{eqnarray*}
  \\
  P \models E \iff P \in \meaningof{E}
\end{eqnarray*}

\begin{eqnarray*}
  P \approx_{L} Q \iff \forall E \in L. P \models E \iff Q \models E
\end{eqnarray*}

\begin{eqnarray*}
  P \approx_{K} Q
\end{eqnarray*}

\begin{eqnarray*}
  P \approx Q
\end{eqnarray*}

$\approx_{K} = \approx = \approx_{L}$

\subsubsection{Contextual duality}

Note that contexts extend the quotation operation to a family of
operations from processes to names. Given a context, $M$, we can
define a \emph{nominal context}, $\quotep{M}$ by $\quotep{M}[P] :=
\quotep{M[P]}$. To foreshadow what is to come we observe that these
operations enjoy a duality with processes very much like the duality
between vectors and maps from vectors to scalars.

Further, because the calculus is essentially higher-order, we have a
correspondence between contexts and processes. More specifically,
given a name $x$ and a context $M$ we can construct $M^{*}_{x}$ such
that 

\begin{mathpar}
  M^{*}_{x} | \lift{x}{P} \red M[P]
\end{mathpar}

namely,

\begin{mathpar}
  M^{*}_{x} := x?(u).M[\dropn{u}]
\end{mathpar}

The dependence of $M^{*}_{x}$ on a name makes it an abstraction, 

\begin{mathpar}
  M^{*} := (x)x?(u).M[\dropn{u}]
\end{mathpar}

\subsection{Additional notation}

It will sometimes be convenient to denote the process a name
quotes. We already have the notation $x = \quotep{P}$, but it will be
convenient to introduce an alternate notation, $\procn{x}$, when we
want to emphasize the connection to the use of the name. Note that, by
virtue of name equivalence, $\quotep{\procn{x}} \nameeq x$; so, the
notation is consistent with previous definitions.

Further, because names have structure it is possible to effect
substitutions on the basis of that structure. This means we need to
upgrade our notation for substitutions, which we accomplish by
adapting comprehension notation. Thus,

\begin{mathpar}
  P\{ y / x : x \in S \}
\end{mathpar}

is interpreted to mean the process derived from P by replacing (in a
capture-avoiding manner) each occurrence of $x$ in $S$ by $y$. For example,

\begin{mathpar}
  P\{ \quotep{\procn{x}|\procn{x}} / x : x \in \freenames{P} \}
\end{mathpar}

will replace each (occurrence) of a free name $x$ in $P$ by
$\quotep{\procn{x}|\procn{x}}$.

Also, we will avail ourselves of the notation $x^{L}$ and $x^{R}$ to
denote injections of a name into disjoint copies of the name
space. There are numerous ways to accomplish this. One example can be
found in \cite{MeredithR05}. This notation overloads to vectors of
names: $\vec{x}^{\pi} := (x_{i}^{\pi} \; : \; 0 \leq i < |\vec{x}| )$ where $\pi \in \{L,R\}$.

We also use $P^{\Box} := P|\Box$.

In \cite{MeredithR05} an interpretation of the new operator is
given. It turns out that there are several possible interpretations
all enjoying the requisite algebraic properties of the operator (see
\cite{milner91polyadicpi}). We will therefore make liberal use of
$(\nu\; \vec{x})P$.

% subsection the_syntax_and_semantics_of_the_notation_system (end)   

\input{qm2pi.qmops} 

\input{qm2pi.sterngerlach} 

\input{qm2pi.metric} 

% section concurrent_process_calculi (end)

%\input{qm2pi.proofsketch}

% section proof sketch (end)

%\input{qm2pi.slviaknots} 

% section spatial logic via knots (end)

\input{qm2pi.conclusion}

% section conclusion (end)

%\input{qm2pi.dtcodes} 

% section wiring algorithm (end)

\input{qm2pi.ack} 

% section acknowledgments (end)

\newpage


\bibliographystyle{plain}   
\bibliography{../../biblios/main.bib}

\input{qm2pi.rhodetails}

\end{document}

 

% section acknowledgments (end)

\newpage


\bibliographystyle{plain}   
\bibliography{../../biblios/main.bib}

\documentclass[12pt]{llncs}
%\documentclass{jktr}

\usepackage[pdftex]{hyperref}                   
\usepackage {listings}
\usepackage {mathpartir}
\usepackage{bcprules}
%\usepackage{listings}
                       
\usepackage{graphicx} 
%\usepackage[margins=2.5cm,nohead,nofoot]{geometry}
%\usepackage{geometry}
\usepackage{amsfonts}
\usepackage{amstext}
\usepackage{latexsym}
\usepackage{amssymb}
\usepackage{color}


%\include{myPreamble}
\include{qm2pi.local} 

%\ifpdf
%\usepackage[pdftex]{graphicx}
%\else
%\usepackage{graphicx}
%\fi

 % \ifpdf
%  \usepackage{pdfsync}
%  \if


%\title{Brief Article}
%\author{David F. Snyder}
%\author{L.G. Meredith}

%\address{Dept. of Math., Texas State University--San Marcos, San Marcos, TX 78666}
       
\pagestyle{empty}


\begin{document}

\lstset{language=[Objective]Caml,frame=shadowbox}

\input{qm2pi.front}

% section front matter (end)

\input{qm2pi.intro} 
 
% section introduction (end)

% \input{qm2pi.knotations} 

% section notation (end)

\input{qm2pi.process.calculi} 

% section concurrent_process_calculi_and_spatial_logics_ (end)
    
%\input{qm2pi.knots2pi} 

%\input{qm2pi.trefoil} 

%\input{qm2pi.mainthm} 

% subsection basic_interpretation (end)

%\input{qm2pi.rho.presentation} 
\subsection{The syntax and semantics of the notation system}\label{sub:the_syntax_and_semantics_of_the_notation_system} % (fold)

We now summarize a technical presentation of the calculus that
embodies our theory of dynamics. The typical presentation of such a
calculus follows the style of giving generators and relations on
them. The grammar, below, describing term constructors, freely
generates the set of processes, $\Proc$. This set is then quotiented
by a relation known as structural congruence and it is over this set
that the notion of dynamics is expressed. This presentation is
essentially that of \cite{MeredithR05} with the addition of
polyadicity and summation. For readability we have relegated some of
the technical subtleties to an appendix.

\subsubsection{Process grammar}\label{subsub:process_grammar}

\begin{mathpar}
  \inferrule* [lab=synchronization] {} {{M} \bc \pzero \;|\; x?F \;|\; x!C }
  \and
  \inferrule* [lab=abstraction] {} {{F} \bc (x)P}
  \and
  \inferrule* [lab=concretion] {} {{C} \bc \langle Q \rangle}
  \and
  \inferrule* [lab=process] {} {{P,Q} \bc M \;| \;P|Q \;|\; @{x}}
  \and
  \inferrule* [lab=name] {} {{x} \bc \quotep{P}}
\end{mathpar} 

Note that $\vec{x}$ (resp. $\vec{P}$) denotes a vector of names
(resp. processes) of length $|\vec{x}|$ (resp. $|\vec{P}|$). We adopt
the following useful abbreviations.

\begin{mathpar}
   x?(\vec{y}).P := x.(\vec{y})P \and  x\clift{\vec{P}} := x.\clift{\vec{P}}
   \and x!(y) := \lift{x}{\dropn{y}}
   \and \Pi_{i=0}^{n-1}P_i := P_0 | \ldots | P_{n-1}
\end{mathpar}

\subsubsection{Structural congruence}

\paragraph{Free and bound names and alpha-equivalence.} At the
core of structural equivalence is alpha-equivalence which identifies
process that are the same up to a change of variable. Formally, we
recognize the distinction between free and bound names. The free names
of a process, $\freenames{P}$, may be calculated recursively as
follows:

\begin{mathpar}
\freenames{\pzero} := \emptyset
  \and \\
  \freenames{x?(y).P} := \{ x \} \cup (\freenames{P} \setminus \{ y \})
  \and 
  \freenames{x!\langle P \rangle} := \{ x \} \cup \{ P \} 
  \and \\
  \freenames{P|Q} := \freenames{P} \cup \freenames{Q}
  \and \\
  \freenames{@{x}} := \{ x \}
\end{mathpar}

$\pi$
$\quotep{\pi}$

$\freenames{-} : \pi \to \mathcal{P}(\quotep{\pi})$

\begin{eqnarray*}
  \freenames{\pzero} & := & \emptyset \\
  \freenames{x?(y).P} & := & \{ x \} \cup (\freenames{P} \setminus \{ y \}) \\
  \freenames{x!\langle P \rangle} & := & \{ x \} \cup \{ P \} \\
  \freenames{P|Q} & := & \freenames{P} \cup \freenames{Q} \\
  \freenames{\dropn{x}} & := & \{ x \}
\end{eqnarray*}

The bound names of a process, $\boundnames{P}$, are those names occurring in $P$
that are not free. For example, in $x?(y).0$, the name $x$ is free, while $y$ is bound.

\begin{mathpar}
  \inferrule* [lab=monoidal-laws] {} { P|Q \equiv Q|P \and P|0 \equiv P \and P|(Q|R) \equiv (P|Q)|R }
\end{mathpar}

\begin{mathpar}
  \inferrule* [lab=alpha-equivalence] {} { (x)P \equiv (y)P\{y/x\} \and y \not\in \freenames{P} }
\end{mathpar}

\begin{definition}
Then two processes, $P,Q$, are alpha-equivalent if $P = Q\{\vec{y}/\vec{x}\}$ for
some $\vec{x} \in \boundnames{Q},\vec{y} \in \boundnames{P}$, where $Q\{\vec{y}/\vec{x}\}$
denotes the capture-avoiding substitution of $\vec{y}$ for $\vec{x}$ in $Q$.
\end{definition}

\begin{definition}
  The {\em structural congruence} \cite{SangiorgiWalker} , $\equiv$,
  between processes is the least congruence containing
  alpha-equivalence, satisfying the abelian monoid laws
  (associativity, commutativity and $\pzero$ as identity) for parallel
  composition $|$ and for summation $+$.
\end{definition}

\subsection{Name equivalence}

We take name equivalence, written $\nameeq$, to be the smallest
equivalence relation generated by the following rules.

\begin{mathpar}
\inferrule*[lab=Quote-drop]
{ }
{ \quotep{@{x}} \nameeq x }

\inferrule*[lab=Struct-equiv]
{ P \scong Q }
{ \quotep{P} \nameeq \quotep{Q} }
\end{mathpar}

The astute reader will have noticed that the mutual recursion of names
and processes imposes a mutual recursion on alpha-equivalence and
structural equivalence via name-equivalence. Fortunately, all of this
works out pleasantly and we may calculate in the natural way, free of
concern. The reader interested in the details is referred to the
appendix \ref{appendix:rho_details}.

\subsection{Substitution}

We use $\Proc$ for the set of processes, $\QProc$ for the set of
names, and $\id{\{}\vec{y} / \vec{x} \id{\}}$ to denote partial maps,
$s : \QProc \rightarrow \QProc$. A map, $s$ lifts, uniquely, to a map
on process terms, $\widehat{s} : \Proc \rightarrow \Proc$ by the
following equations.

\begin{mathpar}
  (0) \psubstp{Q}{P} := 0 \\
  (R \juxtap S) \psubstp{Q}{P}
  :=    
  (R)\psubstp{Q}{P} \juxtap (S) \psubstp{Q}{P} \\
  (x?(y).R) \psubstp{Q}{P}    
  :=    
  (x)\substp{Q}{P} (z)\concat( (R \psubstn{z}{y}) \psubstp{Q}{P} ) \\
  (\lift{x}{R}) \psubstp{Q}{P}  
  :=
  \lift{(x)\substp{Q}{P}}{ R \psubstp{Q}{P} } \\
%   (\dropn{x})  \psubstp{Q}{P}       
%   := 
%   \left\{ 
%     \begin{array}{ccc} 
%       \dropn{\quotep{Q}} & & x \nameeq \quotep{P} \\
%       \dropn{x} & & otherwise \\
%     \end{array}
%   \right. 
  (\dropn{x})  \psubstp{Q}{P}       
  := 
  \left\{ 
    \begin{array}{ccc} 
      Q & & x \nameeq \quotep{P} \\
      \dropn{x} & & otherwise \\
    \end{array}
  \right.
\end{mathpar}
 

where

\begin{eqnarray}
  (x)\id{\{} \lpquote Q \rpquote / \lpquote P \rpquote \id{\}}            = 
  \left\{ 
    \begin{array}{ccc}
      \lpquote Q \rpquote & & x \nameeq \lpquote P \rpquote \\
      x & & otherwise \\
    \end{array}
  \right. \nonumber
\end{eqnarray}

and $z$ is chosen distinct from $\quotep{P}$, $\quotep{Q}$, the free
names in $Q$, and all the names in $R$. Our $\alpha$-equivalence will
be built in the standard way from this substitution.

\begin{remark}\label{rem:no_self_referential_names}
  One consequence of these definitions is that $\forall P. \quotep{P}
  \not\in \freenames{P}$.
\end{remark}

\subsection{ Dynamic quote: an example }

Anticipating something of what's to come, consider applying the
substitution, $\widehat{\id{\{}u / z \id{\}}}$, to the following pair
of processes, $\lift{w}{y!(z)}$ and $w[ \lpquote y!(z) \rpquote ]$.

\begin{eqnarray}
	\lift{w}{y!(z)}\widehat{\id{\{}u / z \id{\}}}
		& = &
		\lift{w}{y!(u)} \nonumber\\
	w[ \lpquote y!(z) \rpquote ] \widehat{ \id{\{}u / z \id{\}} }
		& = &
		w[ \lpquote y!(z) \rpquote ] \nonumber
\end{eqnarray}

Because the body of the process between quotes is impervious to
substitution, we get radically different answers. In fact, by
examining the first process in an input context,
e.g. $x?(z).\lift{w}{y!(z)}$, we see that the process under the lift
operator may be shaped by prefixed inputs binding a name inside it. In
this sense, the lift operator will be seen as a way to dynamically
construct processes before reifying them as names.

Finally equipped with these standard features we can present the
dynamics of the calculus.

\subsubsection{Operational semantics} 

Finally, we introduce the computational dynamics. What marks these
algebras as distinct from other more traditionally studied algebraic
structures, e.g. vector spaces or polynomial rings, is the manner in
which dynamics is captured. In traditional structures, dynamics is typically
expressed through morphisms between such structures, as in linear maps
between vector spaces or morphisms between rings. In algebras
associated with the semantics of computation, the dynamics is
expressed as part of the algebraic structure itself, through a
reduction reduction relation typically denoted by $\red$. Below, we
give a recursive presentation of this relation for the calculus used
in the encoding.

$\red \subseteq \pi \times \pi$
$\red : \pi \to \mathcal{P}(\pi)$

\begin{mathpar}
  \inferrule* [lab=Comm] { \textsf{match}( x_{src}, x_{trgt} ) } { x_{trgt}?(y)P \; | \; x_{src}!\langle {Q} \rangle \red P\{\quotep{Q}/y}\} }
  \and \\
  \inferrule* [lab=Par] {{P} \red {P}'} {{{P} | {Q}} \red {{P}' | {Q}}}
  \and
  \inferrule* [lab=Equiv]{{{P} \scong {P}'} \andalso {{P}' \red {Q}'} \andalso {{Q}' \scong {Q}}}{{P} \red {Q}}
\end{mathpar}

\begin{eqnarray*}
  match_{\equiv} (\quotep{P},\quotep{Q}) & := & P \equiv Q \\
  match_{\dagger}(\quotep{P},\quotep{Q}) & := & \forall R. P|Q \red^{*} R => R \red^{*} 0 \\
  match_{K}(\quotep{P},\quotep{Q}) & := & K \mbox{ for some context } K
\end{eqnarray*}

$u?(x)P | u!\langle Q \rangle \red P\{\quotep{Q}/x\}$

%We write $\wred$ for $\red^*$, and $P\red$ if $\exists Q $ such that $ P \red Q$.
We write $P\red$ if $\exists Q $ such that $ P \red Q$ and $P\not\red$, otherwise.

\section{Replication}

As mentioned before, it is known that replication (and hence
recursion) can be implemented in a higher-order process algebra
\cite{SangiorgiWalker}. As our first example of calculation with the
machinery thus far presented we give the construction explicitly in
the {\rhoc}.

\begin{eqnarray}
	D_{x} & := & \prefix{x}{y}{(\binpar{\outputp{x}{y}}{@{y}})} \nonumber\\
	\bangp_{x}{P} & := & \binpar{{x}!\langle{\binpar{D_{x}}{P}}\rangle}{D_{x}} \nonumber
\end{eqnarray}

\begin{eqnarray}
	\bangp_{x}{P} & & \nonumber\\
	=
	& {x}!\langle{(\prefix{x}{y}{(\outputp{x}{y} | @{y})) | P}}\rangle 
	      | \prefix{x}{y}{(\outputp{x}{y} | @{y})} & \nonumber\\
	\red
	& (\outputp{x}{y} | @{y})\substn{\quotep{(\prefix{x}{y}{(@{y} | \outputp{x}{y})) | P}}}{y} & \nonumber\\
	=
	& \outputp{x}{\quotep{(\prefix{x}{y}{(\outputp{x}{y} | @{y})) | P}}}
	  | {(\prefix{x}{y}{(\outputp{x}{y} | @{y})) | P}} & \nonumber\\
	\red
	& \ldots & \nonumber\\
	\red^*
	& P | P | \ldots & \nonumber
\end{eqnarray}

Of course, this encoding, as an implementation, runs away, unfolding
$\bangp{P}$ eagerly. A lazier and more implementable replication
operator, restricted to input-guarded processes, may be obtained as follows.

\begin{eqnarray}
\bangp{\prefix{u}{v}{P}} 
	:= 
	\binpar{\lift{x}{\prefix{u}{v}{(\binpar{D(x)}{P})}}}{D(x)} \nonumber
\end{eqnarray}

\begin{remark}
  Note that the lazier definition still does not deal with summation
  or mixed summation (i.e. sums over input and output). The reader is
  invited to construct definitions of replication that deal with these
  features. 

  Further, the definitions are parameterized in a name, $x$. Can you,
  gentle reader, make a definition that eliminates this parameter and
  guarantees no accidental interaction between the replication
  machinery and the process being replicated -- i.e. no accidental
  sharing of names used by the process to get its work done and the
  name(s) used by the replication to effect copying. This latter
  revision of the definition of replication is crucial to obtaining
  the expected identity $!!P \sim !P$.
\end{remark}

\begin{remark}\label{rem:paradoxical_combinator}
  The reader familiar with the lambda calculus will have noticed the
  similarity between $D$ and the paradoxical combinator.

  [Ed. note: the existence of this seems to suggest we have to be more
  restrictive on the set of processes and names we admit if we are to
  support no-cloning.]
\end{remark}

\subsubsection{Bisimulation}

The computational dynamics gives rise to another kind of equivalence,
the equivalence of computational behavior. As previously mentioned
this is typically captured \emph{via} some form of bisimulation.

% The notion we use in this paper is weak barbed bisimulation
% \cite{milner91polyadicpi}.

The notion we use in this paper is derived from weak barbed
bisimulation \cite{milner91polyadicpi}. 

\begin{definition}
An \emph{observation relation}, $\downarrow_{\mathcal N}$, over a set
of names, $\mathcal N$, is the smallest relation satisfying the rules
below.

\infrule[Out-barb]{y \in {\mathcal N}, \; x \nameeq y}
		  {\outputp{x}{v} \downarrow_{\mathcal N} x}
\infrule[Par-barb]{\mbox{$P\downarrow_{\mathcal N} x$ or $Q\downarrow_{\mathcal N} x$}}
		  {\binpar{P}{Q} \downarrow_{\mathcal N} x}

We write $P \Downarrow_{\mathcal N} x$ if there is $Q$ such that 
$P \wred Q$ and $Q \downarrow_{\mathcal N} x$.
\end{definition}

\begin{definition}
%\label{def.bbisim}
An  ${\mathcal N}$-\emph{barbed bisimulation} over a set of names, ${\mathcal N}$, is a symmetric binary relation 
${\mathcal S}_{\mathcal N}$ between agents such that $P\rel{S}_{\mathcal N}Q$ implies:
\begin{enumerate}
\item If $P \red P'$ then $Q \wred Q'$ and $P'\rel{S}_{\mathcal N} Q'$.
\item If $P\downarrow_{\mathcal N} x$, then $Q\Downarrow_{\mathcal N} x$.
\end{enumerate}
$P$ is ${\mathcal N}$-barbed bisimilar to $Q$, written
$P \wbbisim_{\mathcal N} Q$, if $P \rel{S}_{\mathcal N} Q$ for some ${\mathcal N}$-barbed bisimulation ${\mathcal S}_{\mathcal N}$.
\end{definition}

$\mathcal{R} \subseteq \pi \times \pi$

$P \mathcal{R} Q => \forall P'. P \red P' \Rightarrow \exists Q'. Q \red Q', P' \mathcal{R} Q'$

$P \vdash x \Rightarrow Q \vdash x$

\begin{mathpar}
  \inferrule*[lab=Out-barb]{x \nameeq y}{{y}!\langle{Q}\rangle \vdash x}
  \and
  \inferrule*[lab=Par-barb]{\mbox{$P\vdash x$ or $Q\vdash x$}}{\binpar{P}{Q} \vdash x}
\end{mathpar}

\subsubsection{Contexts}

One of the principle advantages of computational calculi like the
$\pi$-calculus is a well-defined notion of context,
contextual-equivalence and a correlation between
contextual-equivalence and notions of bisimulation. The notion of
context allows the decomposition of a process into (sub-)process and
its syntactic environment, its context. Thus, a context may be
thought of as a process with a ``hole'' (written $\Box$) in it. The
application of a context $M$ to a process $P$, written $M[P]$, is
tantamount to filling the hole in $M$ with $P$. In this paper we do
not need the full weight of this theory, but do make use of the notion
of context in the proof the main theorem. 

\begin{mathpar}
  \inferrule* [lab=summation] {} {{M_{M},M_{N}} \bc \Box \;|\; x.M_{A} \;|\; M_{M}+M_{N}}
  \and
  \inferrule* [lab=agent] {} {{M_{A}} \bc (\vec{x})M_{P} \;| \; \clift{P_0,\ldots,M_{P},\ldots,P_N}}
  \and \\
  \inferrule* [lab=process] {} {{M_{P}} \bc M_{N} \;| \;P|M_{P} }
\end{mathpar} 

\begin{mathpar}
  \inferrule* [lab=sychronization] {} {M_{N} \bc \Box \;|\; x?M_{F} \;|\; x!M_{C}}
  \and
  \inferrule* [lab=abstraction] {} {{M_{F}} \bc (x)M_{P} }
  \and
  \inferrule* [lab=concretion] {} {{M_{C}} \bc \langle M_{P} \rangle }
  \and \\
  \inferrule* [lab=process] {} {{M_{P}} \bc M_{N} \;| \;P|M_{P} }
\end{mathpar}

\begin{definition}[contextual application] Given a context $M$, and
  process $P$, we define the \emph{contextual application}, $M[P] :=
  M\{P/\Box\}$. That is, the contextual application of M to P is the
  substitution of $P$ for $\Box$ in $M$.
\end{definition}

$\meaningof{-} : L \to \mathcal{P}(\pi)$

\begin{mathpar}
  \inferrule* [lab=collection] {} {\meaningof{true} = \pi, \and \meaningof{~E} = \pi \setminus \meaningof{E}, \and \meaningof{E_{1} \& E_{2}} = \meaningof{E_{1}} \cap \meaningof{E_{2}}}
\end{mathpar}

\begin{mathpar}
  \inferrule* [lab=structure] {} {\meaningof{0} = \{ P \in \pi | P \equiv 0 \}, \and \\ \meaningof{E_1 | E_2} = \{ P \in \pi | P \equiv P_{1} | P_{2}, P_{1} \in \meaningof{E_{1}}, P_{2} \in \meaningof{E_2}\} }
\end{mathpar}

\begin{mathpar}
 \inferrule* [lab=behavior] {} {\meaningof{\langle a?b \rangle E} = \{ P \in \pi | P \equiv Q | u?(y)P', \\ \and \\\\ \and \\ \;\;\; u \in \meaningof{a}, \forall z.P'\{z/y\} \in \meaningof{E\{z/b\}}\}, \and \\ \meaningof{a!E} = \{ P \in \pi | P \equiv Q | x!\langle P' \rangle, x \in \meaningof{a} P' \in \meaningof{E}\} }
\end{mathpar}

\begin{mathpar}
 \inferrule* [lab=nominal] {} {\meaningof{\quotep{E}} = \{ \quotep{P} \in \quotep{\pi} | P \in \meaningof{E} \}, \and \meaningof{\quotep{P}} = \{ \quotep{Q} \in \quotep{\pi} | P \equiv Q \} \and \\ \meaningof{@\quotep{E}} = \{ P \in \pi | P \equiv @x, x \in \meaningof{E} \}}
\end{mathpar}

\begin{eqnarray*}
  \\
  \meaningof{-} : TS \to ST
\end{eqnarray*}

\begin{eqnarray*}
  \\
  L : TS \to ST
\end{eqnarray*}

\begin{eqnarray*}
  \\
  P \models E \iff P \in \meaningof{E}
\end{eqnarray*}

\begin{eqnarray*}
  P \approx_{L} Q \iff \forall E \in L. P \models E \iff Q \models E
\end{eqnarray*}

\begin{eqnarray*}
  P \approx_{K} Q
\end{eqnarray*}

\begin{eqnarray*}
  P \approx Q
\end{eqnarray*}

$\approx_{K} = \approx = \approx_{L}$

\subsubsection{Contextual duality}

Note that contexts extend the quotation operation to a family of
operations from processes to names. Given a context, $M$, we can
define a \emph{nominal context}, $\quotep{M}$ by $\quotep{M}[P] :=
\quotep{M[P]}$. To foreshadow what is to come we observe that these
operations enjoy a duality with processes very much like the duality
between vectors and maps from vectors to scalars.

Further, because the calculus is essentially higher-order, we have a
correspondence between contexts and processes. More specifically,
given a name $x$ and a context $M$ we can construct $M^{*}_{x}$ such
that 

\begin{mathpar}
  M^{*}_{x} | \lift{x}{P} \red M[P]
\end{mathpar}

namely,

\begin{mathpar}
  M^{*}_{x} := x?(u).M[\dropn{u}]
\end{mathpar}

The dependence of $M^{*}_{x}$ on a name makes it an abstraction, 

\begin{mathpar}
  M^{*} := (x)x?(u).M[\dropn{u}]
\end{mathpar}

\subsection{Additional notation}

It will sometimes be convenient to denote the process a name
quotes. We already have the notation $x = \quotep{P}$, but it will be
convenient to introduce an alternate notation, $\procn{x}$, when we
want to emphasize the connection to the use of the name. Note that, by
virtue of name equivalence, $\quotep{\procn{x}} \nameeq x$; so, the
notation is consistent with previous definitions.

Further, because names have structure it is possible to effect
substitutions on the basis of that structure. This means we need to
upgrade our notation for substitutions, which we accomplish by
adapting comprehension notation. Thus,

\begin{mathpar}
  P\{ y / x : x \in S \}
\end{mathpar}

is interpreted to mean the process derived from P by replacing (in a
capture-avoiding manner) each occurrence of $x$ in $S$ by $y$. For example,

\begin{mathpar}
  P\{ \quotep{\procn{x}|\procn{x}} / x : x \in \freenames{P} \}
\end{mathpar}

will replace each (occurrence) of a free name $x$ in $P$ by
$\quotep{\procn{x}|\procn{x}}$.

Also, we will avail ourselves of the notation $x^{L}$ and $x^{R}$ to
denote injections of a name into disjoint copies of the name
space. There are numerous ways to accomplish this. One example can be
found in \cite{MeredithR05}. This notation overloads to vectors of
names: $\vec{x}^{\pi} := (x_{i}^{\pi} \; : \; 0 \leq i < |\vec{x}| )$ where $\pi \in \{L,R\}$.

We also use $P^{\Box} := P|\Box$.

In \cite{MeredithR05} an interpretation of the new operator is
given. It turns out that there are several possible interpretations
all enjoying the requisite algebraic properties of the operator (see
\cite{milner91polyadicpi}). We will therefore make liberal use of
$(\nu\; \vec{x})P$.

% subsection the_syntax_and_semantics_of_the_notation_system (end)   

\input{qm2pi.qmops} 

\input{qm2pi.sterngerlach} 

\input{qm2pi.metric} 

% section concurrent_process_calculi (end)

%\input{qm2pi.proofsketch}

% section proof sketch (end)

%\input{qm2pi.slviaknots} 

% section spatial logic via knots (end)

\input{qm2pi.conclusion}

% section conclusion (end)

%\input{qm2pi.dtcodes} 

% section wiring algorithm (end)

\input{qm2pi.ack} 

% section acknowledgments (end)

\newpage


\bibliographystyle{plain}   
\bibliography{../../biblios/main.bib}

\input{qm2pi.rhodetails}

\end{document}



\end{document}

 

% section acknowledgments (end)

\newpage


\bibliographystyle{plain}   
\bibliography{../../biblios/main.bib}

\documentclass[12pt]{llncs}
%\documentclass{jktr}

\usepackage[pdftex]{hyperref}                   
\usepackage {listings}
\usepackage {mathpartir}
\usepackage{bcprules}
%\usepackage{listings}
                       
\usepackage{graphicx} 
%\usepackage[margins=2.5cm,nohead,nofoot]{geometry}
%\usepackage{geometry}
\usepackage{amsfonts}
\usepackage{amstext}
\usepackage{latexsym}
\usepackage{amssymb}
\usepackage{color}


%\include{myPreamble}
\documentclass[12pt]{llncs}
%\documentclass{jktr}

\usepackage[pdftex]{hyperref}                   
\usepackage {listings}
\usepackage {mathpartir}
\usepackage{bcprules}
%\usepackage{listings}
                       
\usepackage{graphicx} 
%\usepackage[margins=2.5cm,nohead,nofoot]{geometry}
%\usepackage{geometry}
\usepackage{amsfonts}
\usepackage{amstext}
\usepackage{latexsym}
\usepackage{amssymb}
\usepackage{color}


%\include{myPreamble}
\include{qm2pi.local} 

%\ifpdf
%\usepackage[pdftex]{graphicx}
%\else
%\usepackage{graphicx}
%\fi

 % \ifpdf
%  \usepackage{pdfsync}
%  \if


%\title{Brief Article}
%\author{David F. Snyder}
%\author{L.G. Meredith}

%\address{Dept. of Math., Texas State University--San Marcos, San Marcos, TX 78666}
       
\pagestyle{empty}


\begin{document}

\lstset{language=[Objective]Caml,frame=shadowbox}

\input{qm2pi.front}

% section front matter (end)

\input{qm2pi.intro} 
 
% section introduction (end)

% \input{qm2pi.knotations} 

% section notation (end)

\input{qm2pi.process.calculi} 

% section concurrent_process_calculi_and_spatial_logics_ (end)
    
%\input{qm2pi.knots2pi} 

%\input{qm2pi.trefoil} 

%\input{qm2pi.mainthm} 

% subsection basic_interpretation (end)

%\input{qm2pi.rho.presentation} 
\subsection{The syntax and semantics of the notation system}\label{sub:the_syntax_and_semantics_of_the_notation_system} % (fold)

We now summarize a technical presentation of the calculus that
embodies our theory of dynamics. The typical presentation of such a
calculus follows the style of giving generators and relations on
them. The grammar, below, describing term constructors, freely
generates the set of processes, $\Proc$. This set is then quotiented
by a relation known as structural congruence and it is over this set
that the notion of dynamics is expressed. This presentation is
essentially that of \cite{MeredithR05} with the addition of
polyadicity and summation. For readability we have relegated some of
the technical subtleties to an appendix.

\subsubsection{Process grammar}\label{subsub:process_grammar}

\begin{mathpar}
  \inferrule* [lab=synchronization] {} {{M} \bc \pzero \;|\; x?F \;|\; x!C }
  \and
  \inferrule* [lab=abstraction] {} {{F} \bc (x)P}
  \and
  \inferrule* [lab=concretion] {} {{C} \bc \langle Q \rangle}
  \and
  \inferrule* [lab=process] {} {{P,Q} \bc M \;| \;P|Q \;|\; @{x}}
  \and
  \inferrule* [lab=name] {} {{x} \bc \quotep{P}}
\end{mathpar} 

Note that $\vec{x}$ (resp. $\vec{P}$) denotes a vector of names
(resp. processes) of length $|\vec{x}|$ (resp. $|\vec{P}|$). We adopt
the following useful abbreviations.

\begin{mathpar}
   x?(\vec{y}).P := x.(\vec{y})P \and  x\clift{\vec{P}} := x.\clift{\vec{P}}
   \and x!(y) := \lift{x}{\dropn{y}}
   \and \Pi_{i=0}^{n-1}P_i := P_0 | \ldots | P_{n-1}
\end{mathpar}

\subsubsection{Structural congruence}

\paragraph{Free and bound names and alpha-equivalence.} At the
core of structural equivalence is alpha-equivalence which identifies
process that are the same up to a change of variable. Formally, we
recognize the distinction between free and bound names. The free names
of a process, $\freenames{P}$, may be calculated recursively as
follows:

\begin{mathpar}
\freenames{\pzero} := \emptyset
  \and \\
  \freenames{x?(y).P} := \{ x \} \cup (\freenames{P} \setminus \{ y \})
  \and 
  \freenames{x!\langle P \rangle} := \{ x \} \cup \{ P \} 
  \and \\
  \freenames{P|Q} := \freenames{P} \cup \freenames{Q}
  \and \\
  \freenames{@{x}} := \{ x \}
\end{mathpar}

$\pi$
$\quotep{\pi}$

$\freenames{-} : \pi \to \mathcal{P}(\quotep{\pi})$

\begin{eqnarray*}
  \freenames{\pzero} & := & \emptyset \\
  \freenames{x?(y).P} & := & \{ x \} \cup (\freenames{P} \setminus \{ y \}) \\
  \freenames{x!\langle P \rangle} & := & \{ x \} \cup \{ P \} \\
  \freenames{P|Q} & := & \freenames{P} \cup \freenames{Q} \\
  \freenames{\dropn{x}} & := & \{ x \}
\end{eqnarray*}

The bound names of a process, $\boundnames{P}$, are those names occurring in $P$
that are not free. For example, in $x?(y).0$, the name $x$ is free, while $y$ is bound.

\begin{mathpar}
  \inferrule* [lab=monoidal-laws] {} { P|Q \equiv Q|P \and P|0 \equiv P \and P|(Q|R) \equiv (P|Q)|R }
\end{mathpar}

\begin{mathpar}
  \inferrule* [lab=alpha-equivalence] {} { (x)P \equiv (y)P\{y/x\} \and y \not\in \freenames{P} }
\end{mathpar}

\begin{definition}
Then two processes, $P,Q$, are alpha-equivalent if $P = Q\{\vec{y}/\vec{x}\}$ for
some $\vec{x} \in \boundnames{Q},\vec{y} \in \boundnames{P}$, where $Q\{\vec{y}/\vec{x}\}$
denotes the capture-avoiding substitution of $\vec{y}$ for $\vec{x}$ in $Q$.
\end{definition}

\begin{definition}
  The {\em structural congruence} \cite{SangiorgiWalker} , $\equiv$,
  between processes is the least congruence containing
  alpha-equivalence, satisfying the abelian monoid laws
  (associativity, commutativity and $\pzero$ as identity) for parallel
  composition $|$ and for summation $+$.
\end{definition}

\subsection{Name equivalence}

We take name equivalence, written $\nameeq$, to be the smallest
equivalence relation generated by the following rules.

\begin{mathpar}
\inferrule*[lab=Quote-drop]
{ }
{ \quotep{@{x}} \nameeq x }

\inferrule*[lab=Struct-equiv]
{ P \scong Q }
{ \quotep{P} \nameeq \quotep{Q} }
\end{mathpar}

The astute reader will have noticed that the mutual recursion of names
and processes imposes a mutual recursion on alpha-equivalence and
structural equivalence via name-equivalence. Fortunately, all of this
works out pleasantly and we may calculate in the natural way, free of
concern. The reader interested in the details is referred to the
appendix \ref{appendix:rho_details}.

\subsection{Substitution}

We use $\Proc$ for the set of processes, $\QProc$ for the set of
names, and $\id{\{}\vec{y} / \vec{x} \id{\}}$ to denote partial maps,
$s : \QProc \rightarrow \QProc$. A map, $s$ lifts, uniquely, to a map
on process terms, $\widehat{s} : \Proc \rightarrow \Proc$ by the
following equations.

\begin{mathpar}
  (0) \psubstp{Q}{P} := 0 \\
  (R \juxtap S) \psubstp{Q}{P}
  :=    
  (R)\psubstp{Q}{P} \juxtap (S) \psubstp{Q}{P} \\
  (x?(y).R) \psubstp{Q}{P}    
  :=    
  (x)\substp{Q}{P} (z)\concat( (R \psubstn{z}{y}) \psubstp{Q}{P} ) \\
  (\lift{x}{R}) \psubstp{Q}{P}  
  :=
  \lift{(x)\substp{Q}{P}}{ R \psubstp{Q}{P} } \\
%   (\dropn{x})  \psubstp{Q}{P}       
%   := 
%   \left\{ 
%     \begin{array}{ccc} 
%       \dropn{\quotep{Q}} & & x \nameeq \quotep{P} \\
%       \dropn{x} & & otherwise \\
%     \end{array}
%   \right. 
  (\dropn{x})  \psubstp{Q}{P}       
  := 
  \left\{ 
    \begin{array}{ccc} 
      Q & & x \nameeq \quotep{P} \\
      \dropn{x} & & otherwise \\
    \end{array}
  \right.
\end{mathpar}
 

where

\begin{eqnarray}
  (x)\id{\{} \lpquote Q \rpquote / \lpquote P \rpquote \id{\}}            = 
  \left\{ 
    \begin{array}{ccc}
      \lpquote Q \rpquote & & x \nameeq \lpquote P \rpquote \\
      x & & otherwise \\
    \end{array}
  \right. \nonumber
\end{eqnarray}

and $z$ is chosen distinct from $\quotep{P}$, $\quotep{Q}$, the free
names in $Q$, and all the names in $R$. Our $\alpha$-equivalence will
be built in the standard way from this substitution.

\begin{remark}\label{rem:no_self_referential_names}
  One consequence of these definitions is that $\forall P. \quotep{P}
  \not\in \freenames{P}$.
\end{remark}

\subsection{ Dynamic quote: an example }

Anticipating something of what's to come, consider applying the
substitution, $\widehat{\id{\{}u / z \id{\}}}$, to the following pair
of processes, $\lift{w}{y!(z)}$ and $w[ \lpquote y!(z) \rpquote ]$.

\begin{eqnarray}
	\lift{w}{y!(z)}\widehat{\id{\{}u / z \id{\}}}
		& = &
		\lift{w}{y!(u)} \nonumber\\
	w[ \lpquote y!(z) \rpquote ] \widehat{ \id{\{}u / z \id{\}} }
		& = &
		w[ \lpquote y!(z) \rpquote ] \nonumber
\end{eqnarray}

Because the body of the process between quotes is impervious to
substitution, we get radically different answers. In fact, by
examining the first process in an input context,
e.g. $x?(z).\lift{w}{y!(z)}$, we see that the process under the lift
operator may be shaped by prefixed inputs binding a name inside it. In
this sense, the lift operator will be seen as a way to dynamically
construct processes before reifying them as names.

Finally equipped with these standard features we can present the
dynamics of the calculus.

\subsubsection{Operational semantics} 

Finally, we introduce the computational dynamics. What marks these
algebras as distinct from other more traditionally studied algebraic
structures, e.g. vector spaces or polynomial rings, is the manner in
which dynamics is captured. In traditional structures, dynamics is typically
expressed through morphisms between such structures, as in linear maps
between vector spaces or morphisms between rings. In algebras
associated with the semantics of computation, the dynamics is
expressed as part of the algebraic structure itself, through a
reduction reduction relation typically denoted by $\red$. Below, we
give a recursive presentation of this relation for the calculus used
in the encoding.

$\red \subseteq \pi \times \pi$
$\red : \pi \to \mathcal{P}(\pi)$

\begin{mathpar}
  \inferrule* [lab=Comm] { \textsf{match}( x_{src}, x_{trgt} ) } { x_{trgt}?(y)P \; | \; x_{src}!\langle {Q} \rangle \red P\{\quotep{Q}/y}\} }
  \and \\
  \inferrule* [lab=Par] {{P} \red {P}'} {{{P} | {Q}} \red {{P}' | {Q}}}
  \and
  \inferrule* [lab=Equiv]{{{P} \scong {P}'} \andalso {{P}' \red {Q}'} \andalso {{Q}' \scong {Q}}}{{P} \red {Q}}
\end{mathpar}

\begin{eqnarray*}
  match_{\equiv} (\quotep{P},\quotep{Q}) & := & P \equiv Q \\
  match_{\dagger}(\quotep{P},\quotep{Q}) & := & \forall R. P|Q \red^{*} R => R \red^{*} 0 \\
  match_{K}(\quotep{P},\quotep{Q}) & := & K \mbox{ for some context } K
\end{eqnarray*}

$u?(x)P | u!\langle Q \rangle \red P\{\quotep{Q}/x\}$

%We write $\wred$ for $\red^*$, and $P\red$ if $\exists Q $ such that $ P \red Q$.
We write $P\red$ if $\exists Q $ such that $ P \red Q$ and $P\not\red$, otherwise.

\section{Replication}

As mentioned before, it is known that replication (and hence
recursion) can be implemented in a higher-order process algebra
\cite{SangiorgiWalker}. As our first example of calculation with the
machinery thus far presented we give the construction explicitly in
the {\rhoc}.

\begin{eqnarray}
	D_{x} & := & \prefix{x}{y}{(\binpar{\outputp{x}{y}}{@{y}})} \nonumber\\
	\bangp_{x}{P} & := & \binpar{{x}!\langle{\binpar{D_{x}}{P}}\rangle}{D_{x}} \nonumber
\end{eqnarray}

\begin{eqnarray}
	\bangp_{x}{P} & & \nonumber\\
	=
	& {x}!\langle{(\prefix{x}{y}{(\outputp{x}{y} | @{y})) | P}}\rangle 
	      | \prefix{x}{y}{(\outputp{x}{y} | @{y})} & \nonumber\\
	\red
	& (\outputp{x}{y} | @{y})\substn{\quotep{(\prefix{x}{y}{(@{y} | \outputp{x}{y})) | P}}}{y} & \nonumber\\
	=
	& \outputp{x}{\quotep{(\prefix{x}{y}{(\outputp{x}{y} | @{y})) | P}}}
	  | {(\prefix{x}{y}{(\outputp{x}{y} | @{y})) | P}} & \nonumber\\
	\red
	& \ldots & \nonumber\\
	\red^*
	& P | P | \ldots & \nonumber
\end{eqnarray}

Of course, this encoding, as an implementation, runs away, unfolding
$\bangp{P}$ eagerly. A lazier and more implementable replication
operator, restricted to input-guarded processes, may be obtained as follows.

\begin{eqnarray}
\bangp{\prefix{u}{v}{P}} 
	:= 
	\binpar{\lift{x}{\prefix{u}{v}{(\binpar{D(x)}{P})}}}{D(x)} \nonumber
\end{eqnarray}

\begin{remark}
  Note that the lazier definition still does not deal with summation
  or mixed summation (i.e. sums over input and output). The reader is
  invited to construct definitions of replication that deal with these
  features. 

  Further, the definitions are parameterized in a name, $x$. Can you,
  gentle reader, make a definition that eliminates this parameter and
  guarantees no accidental interaction between the replication
  machinery and the process being replicated -- i.e. no accidental
  sharing of names used by the process to get its work done and the
  name(s) used by the replication to effect copying. This latter
  revision of the definition of replication is crucial to obtaining
  the expected identity $!!P \sim !P$.
\end{remark}

\begin{remark}\label{rem:paradoxical_combinator}
  The reader familiar with the lambda calculus will have noticed the
  similarity between $D$ and the paradoxical combinator.

  [Ed. note: the existence of this seems to suggest we have to be more
  restrictive on the set of processes and names we admit if we are to
  support no-cloning.]
\end{remark}

\subsubsection{Bisimulation}

The computational dynamics gives rise to another kind of equivalence,
the equivalence of computational behavior. As previously mentioned
this is typically captured \emph{via} some form of bisimulation.

% The notion we use in this paper is weak barbed bisimulation
% \cite{milner91polyadicpi}.

The notion we use in this paper is derived from weak barbed
bisimulation \cite{milner91polyadicpi}. 

\begin{definition}
An \emph{observation relation}, $\downarrow_{\mathcal N}$, over a set
of names, $\mathcal N$, is the smallest relation satisfying the rules
below.

\infrule[Out-barb]{y \in {\mathcal N}, \; x \nameeq y}
		  {\outputp{x}{v} \downarrow_{\mathcal N} x}
\infrule[Par-barb]{\mbox{$P\downarrow_{\mathcal N} x$ or $Q\downarrow_{\mathcal N} x$}}
		  {\binpar{P}{Q} \downarrow_{\mathcal N} x}

We write $P \Downarrow_{\mathcal N} x$ if there is $Q$ such that 
$P \wred Q$ and $Q \downarrow_{\mathcal N} x$.
\end{definition}

\begin{definition}
%\label{def.bbisim}
An  ${\mathcal N}$-\emph{barbed bisimulation} over a set of names, ${\mathcal N}$, is a symmetric binary relation 
${\mathcal S}_{\mathcal N}$ between agents such that $P\rel{S}_{\mathcal N}Q$ implies:
\begin{enumerate}
\item If $P \red P'$ then $Q \wred Q'$ and $P'\rel{S}_{\mathcal N} Q'$.
\item If $P\downarrow_{\mathcal N} x$, then $Q\Downarrow_{\mathcal N} x$.
\end{enumerate}
$P$ is ${\mathcal N}$-barbed bisimilar to $Q$, written
$P \wbbisim_{\mathcal N} Q$, if $P \rel{S}_{\mathcal N} Q$ for some ${\mathcal N}$-barbed bisimulation ${\mathcal S}_{\mathcal N}$.
\end{definition}

$\mathcal{R} \subseteq \pi \times \pi$

$P \mathcal{R} Q => \forall P'. P \red P' \Rightarrow \exists Q'. Q \red Q', P' \mathcal{R} Q'$

$P \vdash x \Rightarrow Q \vdash x$

\begin{mathpar}
  \inferrule*[lab=Out-barb]{x \nameeq y}{{y}!\langle{Q}\rangle \vdash x}
  \and
  \inferrule*[lab=Par-barb]{\mbox{$P\vdash x$ or $Q\vdash x$}}{\binpar{P}{Q} \vdash x}
\end{mathpar}

\subsubsection{Contexts}

One of the principle advantages of computational calculi like the
$\pi$-calculus is a well-defined notion of context,
contextual-equivalence and a correlation between
contextual-equivalence and notions of bisimulation. The notion of
context allows the decomposition of a process into (sub-)process and
its syntactic environment, its context. Thus, a context may be
thought of as a process with a ``hole'' (written $\Box$) in it. The
application of a context $M$ to a process $P$, written $M[P]$, is
tantamount to filling the hole in $M$ with $P$. In this paper we do
not need the full weight of this theory, but do make use of the notion
of context in the proof the main theorem. 

\begin{mathpar}
  \inferrule* [lab=summation] {} {{M_{M},M_{N}} \bc \Box \;|\; x.M_{A} \;|\; M_{M}+M_{N}}
  \and
  \inferrule* [lab=agent] {} {{M_{A}} \bc (\vec{x})M_{P} \;| \; \clift{P_0,\ldots,M_{P},\ldots,P_N}}
  \and \\
  \inferrule* [lab=process] {} {{M_{P}} \bc M_{N} \;| \;P|M_{P} }
\end{mathpar} 

\begin{mathpar}
  \inferrule* [lab=sychronization] {} {M_{N} \bc \Box \;|\; x?M_{F} \;|\; x!M_{C}}
  \and
  \inferrule* [lab=abstraction] {} {{M_{F}} \bc (x)M_{P} }
  \and
  \inferrule* [lab=concretion] {} {{M_{C}} \bc \langle M_{P} \rangle }
  \and \\
  \inferrule* [lab=process] {} {{M_{P}} \bc M_{N} \;| \;P|M_{P} }
\end{mathpar}

\begin{definition}[contextual application] Given a context $M$, and
  process $P$, we define the \emph{contextual application}, $M[P] :=
  M\{P/\Box\}$. That is, the contextual application of M to P is the
  substitution of $P$ for $\Box$ in $M$.
\end{definition}

$\meaningof{-} : L \to \mathcal{P}(\pi)$

\begin{mathpar}
  \inferrule* [lab=collection] {} {\meaningof{true} = \pi, \and \meaningof{~E} = \pi \setminus \meaningof{E}, \and \meaningof{E_{1} \& E_{2}} = \meaningof{E_{1}} \cap \meaningof{E_{2}}}
\end{mathpar}

\begin{mathpar}
  \inferrule* [lab=structure] {} {\meaningof{0} = \{ P \in \pi | P \equiv 0 \}, \and \\ \meaningof{E_1 | E_2} = \{ P \in \pi | P \equiv P_{1} | P_{2}, P_{1} \in \meaningof{E_{1}}, P_{2} \in \meaningof{E_2}\} }
\end{mathpar}

\begin{mathpar}
 \inferrule* [lab=behavior] {} {\meaningof{\langle a?b \rangle E} = \{ P \in \pi | P \equiv Q | u?(y)P', \\ \and \\\\ \and \\ \;\;\; u \in \meaningof{a}, \forall z.P'\{z/y\} \in \meaningof{E\{z/b\}}\}, \and \\ \meaningof{a!E} = \{ P \in \pi | P \equiv Q | x!\langle P' \rangle, x \in \meaningof{a} P' \in \meaningof{E}\} }
\end{mathpar}

\begin{mathpar}
 \inferrule* [lab=nominal] {} {\meaningof{\quotep{E}} = \{ \quotep{P} \in \quotep{\pi} | P \in \meaningof{E} \}, \and \meaningof{\quotep{P}} = \{ \quotep{Q} \in \quotep{\pi} | P \equiv Q \} \and \\ \meaningof{@\quotep{E}} = \{ P \in \pi | P \equiv @x, x \in \meaningof{E} \}}
\end{mathpar}

\begin{eqnarray*}
  \\
  \meaningof{-} : TS \to ST
\end{eqnarray*}

\begin{eqnarray*}
  \\
  L : TS \to ST
\end{eqnarray*}

\begin{eqnarray*}
  \\
  P \models E \iff P \in \meaningof{E}
\end{eqnarray*}

\begin{eqnarray*}
  P \approx_{L} Q \iff \forall E \in L. P \models E \iff Q \models E
\end{eqnarray*}

\begin{eqnarray*}
  P \approx_{K} Q
\end{eqnarray*}

\begin{eqnarray*}
  P \approx Q
\end{eqnarray*}

$\approx_{K} = \approx = \approx_{L}$

\subsubsection{Contextual duality}

Note that contexts extend the quotation operation to a family of
operations from processes to names. Given a context, $M$, we can
define a \emph{nominal context}, $\quotep{M}$ by $\quotep{M}[P] :=
\quotep{M[P]}$. To foreshadow what is to come we observe that these
operations enjoy a duality with processes very much like the duality
between vectors and maps from vectors to scalars.

Further, because the calculus is essentially higher-order, we have a
correspondence between contexts and processes. More specifically,
given a name $x$ and a context $M$ we can construct $M^{*}_{x}$ such
that 

\begin{mathpar}
  M^{*}_{x} | \lift{x}{P} \red M[P]
\end{mathpar}

namely,

\begin{mathpar}
  M^{*}_{x} := x?(u).M[\dropn{u}]
\end{mathpar}

The dependence of $M^{*}_{x}$ on a name makes it an abstraction, 

\begin{mathpar}
  M^{*} := (x)x?(u).M[\dropn{u}]
\end{mathpar}

\subsection{Additional notation}

It will sometimes be convenient to denote the process a name
quotes. We already have the notation $x = \quotep{P}$, but it will be
convenient to introduce an alternate notation, $\procn{x}$, when we
want to emphasize the connection to the use of the name. Note that, by
virtue of name equivalence, $\quotep{\procn{x}} \nameeq x$; so, the
notation is consistent with previous definitions.

Further, because names have structure it is possible to effect
substitutions on the basis of that structure. This means we need to
upgrade our notation for substitutions, which we accomplish by
adapting comprehension notation. Thus,

\begin{mathpar}
  P\{ y / x : x \in S \}
\end{mathpar}

is interpreted to mean the process derived from P by replacing (in a
capture-avoiding manner) each occurrence of $x$ in $S$ by $y$. For example,

\begin{mathpar}
  P\{ \quotep{\procn{x}|\procn{x}} / x : x \in \freenames{P} \}
\end{mathpar}

will replace each (occurrence) of a free name $x$ in $P$ by
$\quotep{\procn{x}|\procn{x}}$.

Also, we will avail ourselves of the notation $x^{L}$ and $x^{R}$ to
denote injections of a name into disjoint copies of the name
space. There are numerous ways to accomplish this. One example can be
found in \cite{MeredithR05}. This notation overloads to vectors of
names: $\vec{x}^{\pi} := (x_{i}^{\pi} \; : \; 0 \leq i < |\vec{x}| )$ where $\pi \in \{L,R\}$.

We also use $P^{\Box} := P|\Box$.

In \cite{MeredithR05} an interpretation of the new operator is
given. It turns out that there are several possible interpretations
all enjoying the requisite algebraic properties of the operator (see
\cite{milner91polyadicpi}). We will therefore make liberal use of
$(\nu\; \vec{x})P$.

% subsection the_syntax_and_semantics_of_the_notation_system (end)   

\input{qm2pi.qmops} 

\input{qm2pi.sterngerlach} 

\input{qm2pi.metric} 

% section concurrent_process_calculi (end)

%\input{qm2pi.proofsketch}

% section proof sketch (end)

%\input{qm2pi.slviaknots} 

% section spatial logic via knots (end)

\input{qm2pi.conclusion}

% section conclusion (end)

%\input{qm2pi.dtcodes} 

% section wiring algorithm (end)

\input{qm2pi.ack} 

% section acknowledgments (end)

\newpage


\bibliographystyle{plain}   
\bibliography{../../biblios/main.bib}

\input{qm2pi.rhodetails}

\end{document}

 

%\ifpdf
%\usepackage[pdftex]{graphicx}
%\else
%\usepackage{graphicx}
%\fi

 % \ifpdf
%  \usepackage{pdfsync}
%  \if


%\title{Brief Article}
%\author{David F. Snyder}
%\author{L.G. Meredith}

%\address{Dept. of Math., Texas State University--San Marcos, San Marcos, TX 78666}
       
\pagestyle{empty}


\begin{document}

\lstset{language=[Objective]Caml,frame=shadowbox}

\documentclass[12pt]{llncs}
%\documentclass{jktr}

\usepackage[pdftex]{hyperref}                   
\usepackage {listings}
\usepackage {mathpartir}
\usepackage{bcprules}
%\usepackage{listings}
                       
\usepackage{graphicx} 
%\usepackage[margins=2.5cm,nohead,nofoot]{geometry}
%\usepackage{geometry}
\usepackage{amsfonts}
\usepackage{amstext}
\usepackage{latexsym}
\usepackage{amssymb}
\usepackage{color}


%\include{myPreamble}
\include{qm2pi.local} 

%\ifpdf
%\usepackage[pdftex]{graphicx}
%\else
%\usepackage{graphicx}
%\fi

 % \ifpdf
%  \usepackage{pdfsync}
%  \if


%\title{Brief Article}
%\author{David F. Snyder}
%\author{L.G. Meredith}

%\address{Dept. of Math., Texas State University--San Marcos, San Marcos, TX 78666}
       
\pagestyle{empty}


\begin{document}

\lstset{language=[Objective]Caml,frame=shadowbox}

\input{qm2pi.front}

% section front matter (end)

\input{qm2pi.intro} 
 
% section introduction (end)

% \input{qm2pi.knotations} 

% section notation (end)

\input{qm2pi.process.calculi} 

% section concurrent_process_calculi_and_spatial_logics_ (end)
    
%\input{qm2pi.knots2pi} 

%\input{qm2pi.trefoil} 

%\input{qm2pi.mainthm} 

% subsection basic_interpretation (end)

%\input{qm2pi.rho.presentation} 
\subsection{The syntax and semantics of the notation system}\label{sub:the_syntax_and_semantics_of_the_notation_system} % (fold)

We now summarize a technical presentation of the calculus that
embodies our theory of dynamics. The typical presentation of such a
calculus follows the style of giving generators and relations on
them. The grammar, below, describing term constructors, freely
generates the set of processes, $\Proc$. This set is then quotiented
by a relation known as structural congruence and it is over this set
that the notion of dynamics is expressed. This presentation is
essentially that of \cite{MeredithR05} with the addition of
polyadicity and summation. For readability we have relegated some of
the technical subtleties to an appendix.

\subsubsection{Process grammar}\label{subsub:process_grammar}

\begin{mathpar}
  \inferrule* [lab=synchronization] {} {{M} \bc \pzero \;|\; x?F \;|\; x!C }
  \and
  \inferrule* [lab=abstraction] {} {{F} \bc (x)P}
  \and
  \inferrule* [lab=concretion] {} {{C} \bc \langle Q \rangle}
  \and
  \inferrule* [lab=process] {} {{P,Q} \bc M \;| \;P|Q \;|\; @{x}}
  \and
  \inferrule* [lab=name] {} {{x} \bc \quotep{P}}
\end{mathpar} 

Note that $\vec{x}$ (resp. $\vec{P}$) denotes a vector of names
(resp. processes) of length $|\vec{x}|$ (resp. $|\vec{P}|$). We adopt
the following useful abbreviations.

\begin{mathpar}
   x?(\vec{y}).P := x.(\vec{y})P \and  x\clift{\vec{P}} := x.\clift{\vec{P}}
   \and x!(y) := \lift{x}{\dropn{y}}
   \and \Pi_{i=0}^{n-1}P_i := P_0 | \ldots | P_{n-1}
\end{mathpar}

\subsubsection{Structural congruence}

\paragraph{Free and bound names and alpha-equivalence.} At the
core of structural equivalence is alpha-equivalence which identifies
process that are the same up to a change of variable. Formally, we
recognize the distinction between free and bound names. The free names
of a process, $\freenames{P}$, may be calculated recursively as
follows:

\begin{mathpar}
\freenames{\pzero} := \emptyset
  \and \\
  \freenames{x?(y).P} := \{ x \} \cup (\freenames{P} \setminus \{ y \})
  \and 
  \freenames{x!\langle P \rangle} := \{ x \} \cup \{ P \} 
  \and \\
  \freenames{P|Q} := \freenames{P} \cup \freenames{Q}
  \and \\
  \freenames{@{x}} := \{ x \}
\end{mathpar}

$\pi$
$\quotep{\pi}$

$\freenames{-} : \pi \to \mathcal{P}(\quotep{\pi})$

\begin{eqnarray*}
  \freenames{\pzero} & := & \emptyset \\
  \freenames{x?(y).P} & := & \{ x \} \cup (\freenames{P} \setminus \{ y \}) \\
  \freenames{x!\langle P \rangle} & := & \{ x \} \cup \{ P \} \\
  \freenames{P|Q} & := & \freenames{P} \cup \freenames{Q} \\
  \freenames{\dropn{x}} & := & \{ x \}
\end{eqnarray*}

The bound names of a process, $\boundnames{P}$, are those names occurring in $P$
that are not free. For example, in $x?(y).0$, the name $x$ is free, while $y$ is bound.

\begin{mathpar}
  \inferrule* [lab=monoidal-laws] {} { P|Q \equiv Q|P \and P|0 \equiv P \and P|(Q|R) \equiv (P|Q)|R }
\end{mathpar}

\begin{mathpar}
  \inferrule* [lab=alpha-equivalence] {} { (x)P \equiv (y)P\{y/x\} \and y \not\in \freenames{P} }
\end{mathpar}

\begin{definition}
Then two processes, $P,Q$, are alpha-equivalent if $P = Q\{\vec{y}/\vec{x}\}$ for
some $\vec{x} \in \boundnames{Q},\vec{y} \in \boundnames{P}$, where $Q\{\vec{y}/\vec{x}\}$
denotes the capture-avoiding substitution of $\vec{y}$ for $\vec{x}$ in $Q$.
\end{definition}

\begin{definition}
  The {\em structural congruence} \cite{SangiorgiWalker} , $\equiv$,
  between processes is the least congruence containing
  alpha-equivalence, satisfying the abelian monoid laws
  (associativity, commutativity and $\pzero$ as identity) for parallel
  composition $|$ and for summation $+$.
\end{definition}

\subsection{Name equivalence}

We take name equivalence, written $\nameeq$, to be the smallest
equivalence relation generated by the following rules.

\begin{mathpar}
\inferrule*[lab=Quote-drop]
{ }
{ \quotep{@{x}} \nameeq x }

\inferrule*[lab=Struct-equiv]
{ P \scong Q }
{ \quotep{P} \nameeq \quotep{Q} }
\end{mathpar}

The astute reader will have noticed that the mutual recursion of names
and processes imposes a mutual recursion on alpha-equivalence and
structural equivalence via name-equivalence. Fortunately, all of this
works out pleasantly and we may calculate in the natural way, free of
concern. The reader interested in the details is referred to the
appendix \ref{appendix:rho_details}.

\subsection{Substitution}

We use $\Proc$ for the set of processes, $\QProc$ for the set of
names, and $\id{\{}\vec{y} / \vec{x} \id{\}}$ to denote partial maps,
$s : \QProc \rightarrow \QProc$. A map, $s$ lifts, uniquely, to a map
on process terms, $\widehat{s} : \Proc \rightarrow \Proc$ by the
following equations.

\begin{mathpar}
  (0) \psubstp{Q}{P} := 0 \\
  (R \juxtap S) \psubstp{Q}{P}
  :=    
  (R)\psubstp{Q}{P} \juxtap (S) \psubstp{Q}{P} \\
  (x?(y).R) \psubstp{Q}{P}    
  :=    
  (x)\substp{Q}{P} (z)\concat( (R \psubstn{z}{y}) \psubstp{Q}{P} ) \\
  (\lift{x}{R}) \psubstp{Q}{P}  
  :=
  \lift{(x)\substp{Q}{P}}{ R \psubstp{Q}{P} } \\
%   (\dropn{x})  \psubstp{Q}{P}       
%   := 
%   \left\{ 
%     \begin{array}{ccc} 
%       \dropn{\quotep{Q}} & & x \nameeq \quotep{P} \\
%       \dropn{x} & & otherwise \\
%     \end{array}
%   \right. 
  (\dropn{x})  \psubstp{Q}{P}       
  := 
  \left\{ 
    \begin{array}{ccc} 
      Q & & x \nameeq \quotep{P} \\
      \dropn{x} & & otherwise \\
    \end{array}
  \right.
\end{mathpar}
 

where

\begin{eqnarray}
  (x)\id{\{} \lpquote Q \rpquote / \lpquote P \rpquote \id{\}}            = 
  \left\{ 
    \begin{array}{ccc}
      \lpquote Q \rpquote & & x \nameeq \lpquote P \rpquote \\
      x & & otherwise \\
    \end{array}
  \right. \nonumber
\end{eqnarray}

and $z$ is chosen distinct from $\quotep{P}$, $\quotep{Q}$, the free
names in $Q$, and all the names in $R$. Our $\alpha$-equivalence will
be built in the standard way from this substitution.

\begin{remark}\label{rem:no_self_referential_names}
  One consequence of these definitions is that $\forall P. \quotep{P}
  \not\in \freenames{P}$.
\end{remark}

\subsection{ Dynamic quote: an example }

Anticipating something of what's to come, consider applying the
substitution, $\widehat{\id{\{}u / z \id{\}}}$, to the following pair
of processes, $\lift{w}{y!(z)}$ and $w[ \lpquote y!(z) \rpquote ]$.

\begin{eqnarray}
	\lift{w}{y!(z)}\widehat{\id{\{}u / z \id{\}}}
		& = &
		\lift{w}{y!(u)} \nonumber\\
	w[ \lpquote y!(z) \rpquote ] \widehat{ \id{\{}u / z \id{\}} }
		& = &
		w[ \lpquote y!(z) \rpquote ] \nonumber
\end{eqnarray}

Because the body of the process between quotes is impervious to
substitution, we get radically different answers. In fact, by
examining the first process in an input context,
e.g. $x?(z).\lift{w}{y!(z)}$, we see that the process under the lift
operator may be shaped by prefixed inputs binding a name inside it. In
this sense, the lift operator will be seen as a way to dynamically
construct processes before reifying them as names.

Finally equipped with these standard features we can present the
dynamics of the calculus.

\subsubsection{Operational semantics} 

Finally, we introduce the computational dynamics. What marks these
algebras as distinct from other more traditionally studied algebraic
structures, e.g. vector spaces or polynomial rings, is the manner in
which dynamics is captured. In traditional structures, dynamics is typically
expressed through morphisms between such structures, as in linear maps
between vector spaces or morphisms between rings. In algebras
associated with the semantics of computation, the dynamics is
expressed as part of the algebraic structure itself, through a
reduction reduction relation typically denoted by $\red$. Below, we
give a recursive presentation of this relation for the calculus used
in the encoding.

$\red \subseteq \pi \times \pi$
$\red : \pi \to \mathcal{P}(\pi)$

\begin{mathpar}
  \inferrule* [lab=Comm] { \textsf{match}( x_{src}, x_{trgt} ) } { x_{trgt}?(y)P \; | \; x_{src}!\langle {Q} \rangle \red P\{\quotep{Q}/y}\} }
  \and \\
  \inferrule* [lab=Par] {{P} \red {P}'} {{{P} | {Q}} \red {{P}' | {Q}}}
  \and
  \inferrule* [lab=Equiv]{{{P} \scong {P}'} \andalso {{P}' \red {Q}'} \andalso {{Q}' \scong {Q}}}{{P} \red {Q}}
\end{mathpar}

\begin{eqnarray*}
  match_{\equiv} (\quotep{P},\quotep{Q}) & := & P \equiv Q \\
  match_{\dagger}(\quotep{P},\quotep{Q}) & := & \forall R. P|Q \red^{*} R => R \red^{*} 0 \\
  match_{K}(\quotep{P},\quotep{Q}) & := & K \mbox{ for some context } K
\end{eqnarray*}

$u?(x)P | u!\langle Q \rangle \red P\{\quotep{Q}/x\}$

%We write $\wred$ for $\red^*$, and $P\red$ if $\exists Q $ such that $ P \red Q$.
We write $P\red$ if $\exists Q $ such that $ P \red Q$ and $P\not\red$, otherwise.

\section{Replication}

As mentioned before, it is known that replication (and hence
recursion) can be implemented in a higher-order process algebra
\cite{SangiorgiWalker}. As our first example of calculation with the
machinery thus far presented we give the construction explicitly in
the {\rhoc}.

\begin{eqnarray}
	D_{x} & := & \prefix{x}{y}{(\binpar{\outputp{x}{y}}{@{y}})} \nonumber\\
	\bangp_{x}{P} & := & \binpar{{x}!\langle{\binpar{D_{x}}{P}}\rangle}{D_{x}} \nonumber
\end{eqnarray}

\begin{eqnarray}
	\bangp_{x}{P} & & \nonumber\\
	=
	& {x}!\langle{(\prefix{x}{y}{(\outputp{x}{y} | @{y})) | P}}\rangle 
	      | \prefix{x}{y}{(\outputp{x}{y} | @{y})} & \nonumber\\
	\red
	& (\outputp{x}{y} | @{y})\substn{\quotep{(\prefix{x}{y}{(@{y} | \outputp{x}{y})) | P}}}{y} & \nonumber\\
	=
	& \outputp{x}{\quotep{(\prefix{x}{y}{(\outputp{x}{y} | @{y})) | P}}}
	  | {(\prefix{x}{y}{(\outputp{x}{y} | @{y})) | P}} & \nonumber\\
	\red
	& \ldots & \nonumber\\
	\red^*
	& P | P | \ldots & \nonumber
\end{eqnarray}

Of course, this encoding, as an implementation, runs away, unfolding
$\bangp{P}$ eagerly. A lazier and more implementable replication
operator, restricted to input-guarded processes, may be obtained as follows.

\begin{eqnarray}
\bangp{\prefix{u}{v}{P}} 
	:= 
	\binpar{\lift{x}{\prefix{u}{v}{(\binpar{D(x)}{P})}}}{D(x)} \nonumber
\end{eqnarray}

\begin{remark}
  Note that the lazier definition still does not deal with summation
  or mixed summation (i.e. sums over input and output). The reader is
  invited to construct definitions of replication that deal with these
  features. 

  Further, the definitions are parameterized in a name, $x$. Can you,
  gentle reader, make a definition that eliminates this parameter and
  guarantees no accidental interaction between the replication
  machinery and the process being replicated -- i.e. no accidental
  sharing of names used by the process to get its work done and the
  name(s) used by the replication to effect copying. This latter
  revision of the definition of replication is crucial to obtaining
  the expected identity $!!P \sim !P$.
\end{remark}

\begin{remark}\label{rem:paradoxical_combinator}
  The reader familiar with the lambda calculus will have noticed the
  similarity between $D$ and the paradoxical combinator.

  [Ed. note: the existence of this seems to suggest we have to be more
  restrictive on the set of processes and names we admit if we are to
  support no-cloning.]
\end{remark}

\subsubsection{Bisimulation}

The computational dynamics gives rise to another kind of equivalence,
the equivalence of computational behavior. As previously mentioned
this is typically captured \emph{via} some form of bisimulation.

% The notion we use in this paper is weak barbed bisimulation
% \cite{milner91polyadicpi}.

The notion we use in this paper is derived from weak barbed
bisimulation \cite{milner91polyadicpi}. 

\begin{definition}
An \emph{observation relation}, $\downarrow_{\mathcal N}$, over a set
of names, $\mathcal N$, is the smallest relation satisfying the rules
below.

\infrule[Out-barb]{y \in {\mathcal N}, \; x \nameeq y}
		  {\outputp{x}{v} \downarrow_{\mathcal N} x}
\infrule[Par-barb]{\mbox{$P\downarrow_{\mathcal N} x$ or $Q\downarrow_{\mathcal N} x$}}
		  {\binpar{P}{Q} \downarrow_{\mathcal N} x}

We write $P \Downarrow_{\mathcal N} x$ if there is $Q$ such that 
$P \wred Q$ and $Q \downarrow_{\mathcal N} x$.
\end{definition}

\begin{definition}
%\label{def.bbisim}
An  ${\mathcal N}$-\emph{barbed bisimulation} over a set of names, ${\mathcal N}$, is a symmetric binary relation 
${\mathcal S}_{\mathcal N}$ between agents such that $P\rel{S}_{\mathcal N}Q$ implies:
\begin{enumerate}
\item If $P \red P'$ then $Q \wred Q'$ and $P'\rel{S}_{\mathcal N} Q'$.
\item If $P\downarrow_{\mathcal N} x$, then $Q\Downarrow_{\mathcal N} x$.
\end{enumerate}
$P$ is ${\mathcal N}$-barbed bisimilar to $Q$, written
$P \wbbisim_{\mathcal N} Q$, if $P \rel{S}_{\mathcal N} Q$ for some ${\mathcal N}$-barbed bisimulation ${\mathcal S}_{\mathcal N}$.
\end{definition}

$\mathcal{R} \subseteq \pi \times \pi$

$P \mathcal{R} Q => \forall P'. P \red P' \Rightarrow \exists Q'. Q \red Q', P' \mathcal{R} Q'$

$P \vdash x \Rightarrow Q \vdash x$

\begin{mathpar}
  \inferrule*[lab=Out-barb]{x \nameeq y}{{y}!\langle{Q}\rangle \vdash x}
  \and
  \inferrule*[lab=Par-barb]{\mbox{$P\vdash x$ or $Q\vdash x$}}{\binpar{P}{Q} \vdash x}
\end{mathpar}

\subsubsection{Contexts}

One of the principle advantages of computational calculi like the
$\pi$-calculus is a well-defined notion of context,
contextual-equivalence and a correlation between
contextual-equivalence and notions of bisimulation. The notion of
context allows the decomposition of a process into (sub-)process and
its syntactic environment, its context. Thus, a context may be
thought of as a process with a ``hole'' (written $\Box$) in it. The
application of a context $M$ to a process $P$, written $M[P]$, is
tantamount to filling the hole in $M$ with $P$. In this paper we do
not need the full weight of this theory, but do make use of the notion
of context in the proof the main theorem. 

\begin{mathpar}
  \inferrule* [lab=summation] {} {{M_{M},M_{N}} \bc \Box \;|\; x.M_{A} \;|\; M_{M}+M_{N}}
  \and
  \inferrule* [lab=agent] {} {{M_{A}} \bc (\vec{x})M_{P} \;| \; \clift{P_0,\ldots,M_{P},\ldots,P_N}}
  \and \\
  \inferrule* [lab=process] {} {{M_{P}} \bc M_{N} \;| \;P|M_{P} }
\end{mathpar} 

\begin{mathpar}
  \inferrule* [lab=sychronization] {} {M_{N} \bc \Box \;|\; x?M_{F} \;|\; x!M_{C}}
  \and
  \inferrule* [lab=abstraction] {} {{M_{F}} \bc (x)M_{P} }
  \and
  \inferrule* [lab=concretion] {} {{M_{C}} \bc \langle M_{P} \rangle }
  \and \\
  \inferrule* [lab=process] {} {{M_{P}} \bc M_{N} \;| \;P|M_{P} }
\end{mathpar}

\begin{definition}[contextual application] Given a context $M$, and
  process $P$, we define the \emph{contextual application}, $M[P] :=
  M\{P/\Box\}$. That is, the contextual application of M to P is the
  substitution of $P$ for $\Box$ in $M$.
\end{definition}

$\meaningof{-} : L \to \mathcal{P}(\pi)$

\begin{mathpar}
  \inferrule* [lab=collection] {} {\meaningof{true} = \pi, \and \meaningof{~E} = \pi \setminus \meaningof{E}, \and \meaningof{E_{1} \& E_{2}} = \meaningof{E_{1}} \cap \meaningof{E_{2}}}
\end{mathpar}

\begin{mathpar}
  \inferrule* [lab=structure] {} {\meaningof{0} = \{ P \in \pi | P \equiv 0 \}, \and \\ \meaningof{E_1 | E_2} = \{ P \in \pi | P \equiv P_{1} | P_{2}, P_{1} \in \meaningof{E_{1}}, P_{2} \in \meaningof{E_2}\} }
\end{mathpar}

\begin{mathpar}
 \inferrule* [lab=behavior] {} {\meaningof{\langle a?b \rangle E} = \{ P \in \pi | P \equiv Q | u?(y)P', \\ \and \\\\ \and \\ \;\;\; u \in \meaningof{a}, \forall z.P'\{z/y\} \in \meaningof{E\{z/b\}}\}, \and \\ \meaningof{a!E} = \{ P \in \pi | P \equiv Q | x!\langle P' \rangle, x \in \meaningof{a} P' \in \meaningof{E}\} }
\end{mathpar}

\begin{mathpar}
 \inferrule* [lab=nominal] {} {\meaningof{\quotep{E}} = \{ \quotep{P} \in \quotep{\pi} | P \in \meaningof{E} \}, \and \meaningof{\quotep{P}} = \{ \quotep{Q} \in \quotep{\pi} | P \equiv Q \} \and \\ \meaningof{@\quotep{E}} = \{ P \in \pi | P \equiv @x, x \in \meaningof{E} \}}
\end{mathpar}

\begin{eqnarray*}
  \\
  \meaningof{-} : TS \to ST
\end{eqnarray*}

\begin{eqnarray*}
  \\
  L : TS \to ST
\end{eqnarray*}

\begin{eqnarray*}
  \\
  P \models E \iff P \in \meaningof{E}
\end{eqnarray*}

\begin{eqnarray*}
  P \approx_{L} Q \iff \forall E \in L. P \models E \iff Q \models E
\end{eqnarray*}

\begin{eqnarray*}
  P \approx_{K} Q
\end{eqnarray*}

\begin{eqnarray*}
  P \approx Q
\end{eqnarray*}

$\approx_{K} = \approx = \approx_{L}$

\subsubsection{Contextual duality}

Note that contexts extend the quotation operation to a family of
operations from processes to names. Given a context, $M$, we can
define a \emph{nominal context}, $\quotep{M}$ by $\quotep{M}[P] :=
\quotep{M[P]}$. To foreshadow what is to come we observe that these
operations enjoy a duality with processes very much like the duality
between vectors and maps from vectors to scalars.

Further, because the calculus is essentially higher-order, we have a
correspondence between contexts and processes. More specifically,
given a name $x$ and a context $M$ we can construct $M^{*}_{x}$ such
that 

\begin{mathpar}
  M^{*}_{x} | \lift{x}{P} \red M[P]
\end{mathpar}

namely,

\begin{mathpar}
  M^{*}_{x} := x?(u).M[\dropn{u}]
\end{mathpar}

The dependence of $M^{*}_{x}$ on a name makes it an abstraction, 

\begin{mathpar}
  M^{*} := (x)x?(u).M[\dropn{u}]
\end{mathpar}

\subsection{Additional notation}

It will sometimes be convenient to denote the process a name
quotes. We already have the notation $x = \quotep{P}$, but it will be
convenient to introduce an alternate notation, $\procn{x}$, when we
want to emphasize the connection to the use of the name. Note that, by
virtue of name equivalence, $\quotep{\procn{x}} \nameeq x$; so, the
notation is consistent with previous definitions.

Further, because names have structure it is possible to effect
substitutions on the basis of that structure. This means we need to
upgrade our notation for substitutions, which we accomplish by
adapting comprehension notation. Thus,

\begin{mathpar}
  P\{ y / x : x \in S \}
\end{mathpar}

is interpreted to mean the process derived from P by replacing (in a
capture-avoiding manner) each occurrence of $x$ in $S$ by $y$. For example,

\begin{mathpar}
  P\{ \quotep{\procn{x}|\procn{x}} / x : x \in \freenames{P} \}
\end{mathpar}

will replace each (occurrence) of a free name $x$ in $P$ by
$\quotep{\procn{x}|\procn{x}}$.

Also, we will avail ourselves of the notation $x^{L}$ and $x^{R}$ to
denote injections of a name into disjoint copies of the name
space. There are numerous ways to accomplish this. One example can be
found in \cite{MeredithR05}. This notation overloads to vectors of
names: $\vec{x}^{\pi} := (x_{i}^{\pi} \; : \; 0 \leq i < |\vec{x}| )$ where $\pi \in \{L,R\}$.

We also use $P^{\Box} := P|\Box$.

In \cite{MeredithR05} an interpretation of the new operator is
given. It turns out that there are several possible interpretations
all enjoying the requisite algebraic properties of the operator (see
\cite{milner91polyadicpi}). We will therefore make liberal use of
$(\nu\; \vec{x})P$.

% subsection the_syntax_and_semantics_of_the_notation_system (end)   

\input{qm2pi.qmops} 

\input{qm2pi.sterngerlach} 

\input{qm2pi.metric} 

% section concurrent_process_calculi (end)

%\input{qm2pi.proofsketch}

% section proof sketch (end)

%\input{qm2pi.slviaknots} 

% section spatial logic via knots (end)

\input{qm2pi.conclusion}

% section conclusion (end)

%\input{qm2pi.dtcodes} 

% section wiring algorithm (end)

\input{qm2pi.ack} 

% section acknowledgments (end)

\newpage


\bibliographystyle{plain}   
\bibliography{../../biblios/main.bib}

\input{qm2pi.rhodetails}

\end{document}



% section front matter (end)

\section{Introduction}\label{sec:introduction} % (fold)
In this draft of the material i am going to have to dispense with the
usual writing conventions adopted in papers on these topics. i'm going
to have adopt whatever tone i need at the time i'm writing up the
calculations. Sometimes this may be very conversational; others it may
be the barest mathematical grunts; others still it may be that i have
lifted text from one of my other papers because the exposition of some
point was better said there. i hope that my readers are not unduly put
out by this decision. i'm not doing this to flout convention or be
rebellious. i find these calculations very technically challenging. To
keep everything going technically, something has to give; i have to
let go of some cognitive burden. So, the academic writing style --
with all of its trade-offs in terms of facilitating technical
communication -- is what i'm letting go of. Perhaps subsequent drafts
can be tightened and polished, but for now, i'm going to speak as if
we were sitting together in a coffee shop with a laptop, wifi and a
pad of paper and a pencil.

So, here's what i have to say. We -- you and i, comfortably ensconced
in our coffee shop and well-equipped with our tools -- can realize and
carry out the calculations of quantum mechanics over a very different
formal theory of dynamics, a formal theory of dynamics that
corresponds to a theory of concurrent computation with
\emph{reflection}. It has the advantage that the underlying theory is
already `quantized', but supports analogues all of the continuuous
operations. Strikingly, this underlying theory has recently been
connected with a notion of metric that we can show, by calculating
together, coincides with the metric induced by the inner product.

There are a lot of reasons why you might be interested in seeing
calculations of this form. Here's why i'm interested. For the past
several centuries there has been no competitor to the ``Newtonian''
account of dynamics. As a result the predominant share of accounts of
dynamical systems and situations have had to be formulated in terms of
the Newtonian machinery. i view this as an intellectually dangerous
position to occupy. Everything, despite it's intrinsic shape, turns
into a nail to be hit with this hammer. Recently, however, the theory
of computation has matured to the point where we have candidates for
theories of dynamics that offer very different perspective on
reasoning about dynamical systems and situations. Testing these
candidates against very successful accounts of dynamical situations,
like quantum mechanics, is going to give us some sense of how mature
they are and some measure of the quality of these accounts of
dynamics.

\subsection{Summary of contributions and outline of paper}

So, we're going to develop an interpretation of the operations of
quantum mechanics normally interpreted by Hilbert spaces and
operators. We're going to do this over a theory of computation. Note
that this is very different than the usual quantum computation program
which develops notions of computation over quantum mechanics. Rather,
we are developing a story that aligns with Wheeler's slogan: It from
Bit. To do this we will first provide an account of the theory of
computation at play here. Then we will dive into a calculation-driven
interpretation of the operations of quantum mechanics.

The reason we take this approach is that -- until very recently --
there hasn't been an axiomatic account of quantum mechanics. As a
result there has been no sharp delineation of the mathematical theory
supporting interpretation of the physical theory and the physical
theory, itself. So, ambient features of the maths are free to be
exploited (or supressed) without a real accounting of their physical
relevance. There is no sharp statement ``here's the physical theory''
qua \emph{theory} and ``here's the mathematical interpretation''
enabling a judgment of how faithful the interpretation is -- apart
from experimental observation. When there is an axiomatic account we
can judge how well a given mathematical formalism supports an
interpretation of the axioms, independent of
experimentation. Likewise, we can judge how well we have captured our
physical evidence and experience with our axiomatics, independent of
any specific mathematical implementation, with accidental detail that
may or may not have physical significance. 

In lieu of a fully fleshed out and vetted axiomatic account of quantum
mechanics, interpreting the operational notions in service of modeling
physical systems will have to suffice. In other words, we are not in
the business of providing a model of Hilbert spaces and operators. We
are in the business of providing a model of quantum mechanics because
we are motivated by testing our notions of dynamics against physical
theory; and, the predictive calculations of the physical theory must
serve as the best formulation -- shy of a fully fleshed out axiomatic
account -- of the physical theory itself (as they have for scientific
theories since time immemorial). Put another way, despite a
whole-hearted commitment to an It-from-Bit ontology, we are firmly
aligned with the shut-up-and-calculate camp as the best way to obtain
results either from the physical perspective or as a quality assurance
measure of our fledgling theory of dynamics.

In detail, we present a reflective process calculus. Then we develop
intuitive correspondences between the notions available in this
calculus and the usual physical notions supporting quantum mechanical
calculations. Thus, 

\begin{table}[htp]
  \center{
    \fbox{
      \begin{tabular}{c|c}
        quantum mechanics & process calculus \\
        \hline
        scalar & name \\
        state vector & process \\
        dual & contextual duals \\
        matrix & formal sums of process-context-dual pairs \\
        orthogonality & process annihilation \\
        inner product & execution-formula + quoting
      \end{tabular}
    }
  }
  \caption{QM - process calculi correspondences}
\end{table}

Then we tighten up these intuitions to operational definitions. We
employ the Dirac notation as the best proxy we can find for an
abstract syntax of the quantum mechanical notions. The definitions we
develop put us in contact with equational constraints coming from the
theory that we demonstrate the definitions and calculations satisfy.

This puts us in a position to shut up and calculate for the
Stern-Gerlach experimental set up, showing how these predictive
calculations become calculations on processes in our theory of a
reflective process calculus.

Penultimately, we demonstrate that the notion of metric coming from
the inner product coincides with the notion of metric available from
the theory of bisimulation. This demonstration gives us the right to
think of space as arising from behavior. Finally, we consider where we
might go from the new vantage point we have obtained.

% section introduction (end) 
 
% section introduction (end)

% \documentclass[12pt]{llncs}
%\documentclass{jktr}

\usepackage[pdftex]{hyperref}                   
\usepackage {listings}
\usepackage {mathpartir}
\usepackage{bcprules}
%\usepackage{listings}
                       
\usepackage{graphicx} 
%\usepackage[margins=2.5cm,nohead,nofoot]{geometry}
%\usepackage{geometry}
\usepackage{amsfonts}
\usepackage{amstext}
\usepackage{latexsym}
\usepackage{amssymb}
\usepackage{color}


%\include{myPreamble}
\include{qm2pi.local} 

%\ifpdf
%\usepackage[pdftex]{graphicx}
%\else
%\usepackage{graphicx}
%\fi

 % \ifpdf
%  \usepackage{pdfsync}
%  \if


%\title{Brief Article}
%\author{David F. Snyder}
%\author{L.G. Meredith}

%\address{Dept. of Math., Texas State University--San Marcos, San Marcos, TX 78666}
       
\pagestyle{empty}


\begin{document}

\lstset{language=[Objective]Caml,frame=shadowbox}

\input{qm2pi.front}

% section front matter (end)

\input{qm2pi.intro} 
 
% section introduction (end)

% \input{qm2pi.knotations} 

% section notation (end)

\input{qm2pi.process.calculi} 

% section concurrent_process_calculi_and_spatial_logics_ (end)
    
%\input{qm2pi.knots2pi} 

%\input{qm2pi.trefoil} 

%\input{qm2pi.mainthm} 

% subsection basic_interpretation (end)

%\input{qm2pi.rho.presentation} 
\subsection{The syntax and semantics of the notation system}\label{sub:the_syntax_and_semantics_of_the_notation_system} % (fold)

We now summarize a technical presentation of the calculus that
embodies our theory of dynamics. The typical presentation of such a
calculus follows the style of giving generators and relations on
them. The grammar, below, describing term constructors, freely
generates the set of processes, $\Proc$. This set is then quotiented
by a relation known as structural congruence and it is over this set
that the notion of dynamics is expressed. This presentation is
essentially that of \cite{MeredithR05} with the addition of
polyadicity and summation. For readability we have relegated some of
the technical subtleties to an appendix.

\subsubsection{Process grammar}\label{subsub:process_grammar}

\begin{mathpar}
  \inferrule* [lab=synchronization] {} {{M} \bc \pzero \;|\; x?F \;|\; x!C }
  \and
  \inferrule* [lab=abstraction] {} {{F} \bc (x)P}
  \and
  \inferrule* [lab=concretion] {} {{C} \bc \langle Q \rangle}
  \and
  \inferrule* [lab=process] {} {{P,Q} \bc M \;| \;P|Q \;|\; @{x}}
  \and
  \inferrule* [lab=name] {} {{x} \bc \quotep{P}}
\end{mathpar} 

Note that $\vec{x}$ (resp. $\vec{P}$) denotes a vector of names
(resp. processes) of length $|\vec{x}|$ (resp. $|\vec{P}|$). We adopt
the following useful abbreviations.

\begin{mathpar}
   x?(\vec{y}).P := x.(\vec{y})P \and  x\clift{\vec{P}} := x.\clift{\vec{P}}
   \and x!(y) := \lift{x}{\dropn{y}}
   \and \Pi_{i=0}^{n-1}P_i := P_0 | \ldots | P_{n-1}
\end{mathpar}

\subsubsection{Structural congruence}

\paragraph{Free and bound names and alpha-equivalence.} At the
core of structural equivalence is alpha-equivalence which identifies
process that are the same up to a change of variable. Formally, we
recognize the distinction between free and bound names. The free names
of a process, $\freenames{P}$, may be calculated recursively as
follows:

\begin{mathpar}
\freenames{\pzero} := \emptyset
  \and \\
  \freenames{x?(y).P} := \{ x \} \cup (\freenames{P} \setminus \{ y \})
  \and 
  \freenames{x!\langle P \rangle} := \{ x \} \cup \{ P \} 
  \and \\
  \freenames{P|Q} := \freenames{P} \cup \freenames{Q}
  \and \\
  \freenames{@{x}} := \{ x \}
\end{mathpar}

$\pi$
$\quotep{\pi}$

$\freenames{-} : \pi \to \mathcal{P}(\quotep{\pi})$

\begin{eqnarray*}
  \freenames{\pzero} & := & \emptyset \\
  \freenames{x?(y).P} & := & \{ x \} \cup (\freenames{P} \setminus \{ y \}) \\
  \freenames{x!\langle P \rangle} & := & \{ x \} \cup \{ P \} \\
  \freenames{P|Q} & := & \freenames{P} \cup \freenames{Q} \\
  \freenames{\dropn{x}} & := & \{ x \}
\end{eqnarray*}

The bound names of a process, $\boundnames{P}$, are those names occurring in $P$
that are not free. For example, in $x?(y).0$, the name $x$ is free, while $y$ is bound.

\begin{mathpar}
  \inferrule* [lab=monoidal-laws] {} { P|Q \equiv Q|P \and P|0 \equiv P \and P|(Q|R) \equiv (P|Q)|R }
\end{mathpar}

\begin{mathpar}
  \inferrule* [lab=alpha-equivalence] {} { (x)P \equiv (y)P\{y/x\} \and y \not\in \freenames{P} }
\end{mathpar}

\begin{definition}
Then two processes, $P,Q$, are alpha-equivalent if $P = Q\{\vec{y}/\vec{x}\}$ for
some $\vec{x} \in \boundnames{Q},\vec{y} \in \boundnames{P}$, where $Q\{\vec{y}/\vec{x}\}$
denotes the capture-avoiding substitution of $\vec{y}$ for $\vec{x}$ in $Q$.
\end{definition}

\begin{definition}
  The {\em structural congruence} \cite{SangiorgiWalker} , $\equiv$,
  between processes is the least congruence containing
  alpha-equivalence, satisfying the abelian monoid laws
  (associativity, commutativity and $\pzero$ as identity) for parallel
  composition $|$ and for summation $+$.
\end{definition}

\subsection{Name equivalence}

We take name equivalence, written $\nameeq$, to be the smallest
equivalence relation generated by the following rules.

\begin{mathpar}
\inferrule*[lab=Quote-drop]
{ }
{ \quotep{@{x}} \nameeq x }

\inferrule*[lab=Struct-equiv]
{ P \scong Q }
{ \quotep{P} \nameeq \quotep{Q} }
\end{mathpar}

The astute reader will have noticed that the mutual recursion of names
and processes imposes a mutual recursion on alpha-equivalence and
structural equivalence via name-equivalence. Fortunately, all of this
works out pleasantly and we may calculate in the natural way, free of
concern. The reader interested in the details is referred to the
appendix \ref{appendix:rho_details}.

\subsection{Substitution}

We use $\Proc$ for the set of processes, $\QProc$ for the set of
names, and $\id{\{}\vec{y} / \vec{x} \id{\}}$ to denote partial maps,
$s : \QProc \rightarrow \QProc$. A map, $s$ lifts, uniquely, to a map
on process terms, $\widehat{s} : \Proc \rightarrow \Proc$ by the
following equations.

\begin{mathpar}
  (0) \psubstp{Q}{P} := 0 \\
  (R \juxtap S) \psubstp{Q}{P}
  :=    
  (R)\psubstp{Q}{P} \juxtap (S) \psubstp{Q}{P} \\
  (x?(y).R) \psubstp{Q}{P}    
  :=    
  (x)\substp{Q}{P} (z)\concat( (R \psubstn{z}{y}) \psubstp{Q}{P} ) \\
  (\lift{x}{R}) \psubstp{Q}{P}  
  :=
  \lift{(x)\substp{Q}{P}}{ R \psubstp{Q}{P} } \\
%   (\dropn{x})  \psubstp{Q}{P}       
%   := 
%   \left\{ 
%     \begin{array}{ccc} 
%       \dropn{\quotep{Q}} & & x \nameeq \quotep{P} \\
%       \dropn{x} & & otherwise \\
%     \end{array}
%   \right. 
  (\dropn{x})  \psubstp{Q}{P}       
  := 
  \left\{ 
    \begin{array}{ccc} 
      Q & & x \nameeq \quotep{P} \\
      \dropn{x} & & otherwise \\
    \end{array}
  \right.
\end{mathpar}
 

where

\begin{eqnarray}
  (x)\id{\{} \lpquote Q \rpquote / \lpquote P \rpquote \id{\}}            = 
  \left\{ 
    \begin{array}{ccc}
      \lpquote Q \rpquote & & x \nameeq \lpquote P \rpquote \\
      x & & otherwise \\
    \end{array}
  \right. \nonumber
\end{eqnarray}

and $z$ is chosen distinct from $\quotep{P}$, $\quotep{Q}$, the free
names in $Q$, and all the names in $R$. Our $\alpha$-equivalence will
be built in the standard way from this substitution.

\begin{remark}\label{rem:no_self_referential_names}
  One consequence of these definitions is that $\forall P. \quotep{P}
  \not\in \freenames{P}$.
\end{remark}

\subsection{ Dynamic quote: an example }

Anticipating something of what's to come, consider applying the
substitution, $\widehat{\id{\{}u / z \id{\}}}$, to the following pair
of processes, $\lift{w}{y!(z)}$ and $w[ \lpquote y!(z) \rpquote ]$.

\begin{eqnarray}
	\lift{w}{y!(z)}\widehat{\id{\{}u / z \id{\}}}
		& = &
		\lift{w}{y!(u)} \nonumber\\
	w[ \lpquote y!(z) \rpquote ] \widehat{ \id{\{}u / z \id{\}} }
		& = &
		w[ \lpquote y!(z) \rpquote ] \nonumber
\end{eqnarray}

Because the body of the process between quotes is impervious to
substitution, we get radically different answers. In fact, by
examining the first process in an input context,
e.g. $x?(z).\lift{w}{y!(z)}$, we see that the process under the lift
operator may be shaped by prefixed inputs binding a name inside it. In
this sense, the lift operator will be seen as a way to dynamically
construct processes before reifying them as names.

Finally equipped with these standard features we can present the
dynamics of the calculus.

\subsubsection{Operational semantics} 

Finally, we introduce the computational dynamics. What marks these
algebras as distinct from other more traditionally studied algebraic
structures, e.g. vector spaces or polynomial rings, is the manner in
which dynamics is captured. In traditional structures, dynamics is typically
expressed through morphisms between such structures, as in linear maps
between vector spaces or morphisms between rings. In algebras
associated with the semantics of computation, the dynamics is
expressed as part of the algebraic structure itself, through a
reduction reduction relation typically denoted by $\red$. Below, we
give a recursive presentation of this relation for the calculus used
in the encoding.

$\red \subseteq \pi \times \pi$
$\red : \pi \to \mathcal{P}(\pi)$

\begin{mathpar}
  \inferrule* [lab=Comm] { \textsf{match}( x_{src}, x_{trgt} ) } { x_{trgt}?(y)P \; | \; x_{src}!\langle {Q} \rangle \red P\{\quotep{Q}/y}\} }
  \and \\
  \inferrule* [lab=Par] {{P} \red {P}'} {{{P} | {Q}} \red {{P}' | {Q}}}
  \and
  \inferrule* [lab=Equiv]{{{P} \scong {P}'} \andalso {{P}' \red {Q}'} \andalso {{Q}' \scong {Q}}}{{P} \red {Q}}
\end{mathpar}

\begin{eqnarray*}
  match_{\equiv} (\quotep{P},\quotep{Q}) & := & P \equiv Q \\
  match_{\dagger}(\quotep{P},\quotep{Q}) & := & \forall R. P|Q \red^{*} R => R \red^{*} 0 \\
  match_{K}(\quotep{P},\quotep{Q}) & := & K \mbox{ for some context } K
\end{eqnarray*}

$u?(x)P | u!\langle Q \rangle \red P\{\quotep{Q}/x\}$

%We write $\wred$ for $\red^*$, and $P\red$ if $\exists Q $ such that $ P \red Q$.
We write $P\red$ if $\exists Q $ such that $ P \red Q$ and $P\not\red$, otherwise.

\section{Replication}

As mentioned before, it is known that replication (and hence
recursion) can be implemented in a higher-order process algebra
\cite{SangiorgiWalker}. As our first example of calculation with the
machinery thus far presented we give the construction explicitly in
the {\rhoc}.

\begin{eqnarray}
	D_{x} & := & \prefix{x}{y}{(\binpar{\outputp{x}{y}}{@{y}})} \nonumber\\
	\bangp_{x}{P} & := & \binpar{{x}!\langle{\binpar{D_{x}}{P}}\rangle}{D_{x}} \nonumber
\end{eqnarray}

\begin{eqnarray}
	\bangp_{x}{P} & & \nonumber\\
	=
	& {x}!\langle{(\prefix{x}{y}{(\outputp{x}{y} | @{y})) | P}}\rangle 
	      | \prefix{x}{y}{(\outputp{x}{y} | @{y})} & \nonumber\\
	\red
	& (\outputp{x}{y} | @{y})\substn{\quotep{(\prefix{x}{y}{(@{y} | \outputp{x}{y})) | P}}}{y} & \nonumber\\
	=
	& \outputp{x}{\quotep{(\prefix{x}{y}{(\outputp{x}{y} | @{y})) | P}}}
	  | {(\prefix{x}{y}{(\outputp{x}{y} | @{y})) | P}} & \nonumber\\
	\red
	& \ldots & \nonumber\\
	\red^*
	& P | P | \ldots & \nonumber
\end{eqnarray}

Of course, this encoding, as an implementation, runs away, unfolding
$\bangp{P}$ eagerly. A lazier and more implementable replication
operator, restricted to input-guarded processes, may be obtained as follows.

\begin{eqnarray}
\bangp{\prefix{u}{v}{P}} 
	:= 
	\binpar{\lift{x}{\prefix{u}{v}{(\binpar{D(x)}{P})}}}{D(x)} \nonumber
\end{eqnarray}

\begin{remark}
  Note that the lazier definition still does not deal with summation
  or mixed summation (i.e. sums over input and output). The reader is
  invited to construct definitions of replication that deal with these
  features. 

  Further, the definitions are parameterized in a name, $x$. Can you,
  gentle reader, make a definition that eliminates this parameter and
  guarantees no accidental interaction between the replication
  machinery and the process being replicated -- i.e. no accidental
  sharing of names used by the process to get its work done and the
  name(s) used by the replication to effect copying. This latter
  revision of the definition of replication is crucial to obtaining
  the expected identity $!!P \sim !P$.
\end{remark}

\begin{remark}\label{rem:paradoxical_combinator}
  The reader familiar with the lambda calculus will have noticed the
  similarity between $D$ and the paradoxical combinator.

  [Ed. note: the existence of this seems to suggest we have to be more
  restrictive on the set of processes and names we admit if we are to
  support no-cloning.]
\end{remark}

\subsubsection{Bisimulation}

The computational dynamics gives rise to another kind of equivalence,
the equivalence of computational behavior. As previously mentioned
this is typically captured \emph{via} some form of bisimulation.

% The notion we use in this paper is weak barbed bisimulation
% \cite{milner91polyadicpi}.

The notion we use in this paper is derived from weak barbed
bisimulation \cite{milner91polyadicpi}. 

\begin{definition}
An \emph{observation relation}, $\downarrow_{\mathcal N}$, over a set
of names, $\mathcal N$, is the smallest relation satisfying the rules
below.

\infrule[Out-barb]{y \in {\mathcal N}, \; x \nameeq y}
		  {\outputp{x}{v} \downarrow_{\mathcal N} x}
\infrule[Par-barb]{\mbox{$P\downarrow_{\mathcal N} x$ or $Q\downarrow_{\mathcal N} x$}}
		  {\binpar{P}{Q} \downarrow_{\mathcal N} x}

We write $P \Downarrow_{\mathcal N} x$ if there is $Q$ such that 
$P \wred Q$ and $Q \downarrow_{\mathcal N} x$.
\end{definition}

\begin{definition}
%\label{def.bbisim}
An  ${\mathcal N}$-\emph{barbed bisimulation} over a set of names, ${\mathcal N}$, is a symmetric binary relation 
${\mathcal S}_{\mathcal N}$ between agents such that $P\rel{S}_{\mathcal N}Q$ implies:
\begin{enumerate}
\item If $P \red P'$ then $Q \wred Q'$ and $P'\rel{S}_{\mathcal N} Q'$.
\item If $P\downarrow_{\mathcal N} x$, then $Q\Downarrow_{\mathcal N} x$.
\end{enumerate}
$P$ is ${\mathcal N}$-barbed bisimilar to $Q$, written
$P \wbbisim_{\mathcal N} Q$, if $P \rel{S}_{\mathcal N} Q$ for some ${\mathcal N}$-barbed bisimulation ${\mathcal S}_{\mathcal N}$.
\end{definition}

$\mathcal{R} \subseteq \pi \times \pi$

$P \mathcal{R} Q => \forall P'. P \red P' \Rightarrow \exists Q'. Q \red Q', P' \mathcal{R} Q'$

$P \vdash x \Rightarrow Q \vdash x$

\begin{mathpar}
  \inferrule*[lab=Out-barb]{x \nameeq y}{{y}!\langle{Q}\rangle \vdash x}
  \and
  \inferrule*[lab=Par-barb]{\mbox{$P\vdash x$ or $Q\vdash x$}}{\binpar{P}{Q} \vdash x}
\end{mathpar}

\subsubsection{Contexts}

One of the principle advantages of computational calculi like the
$\pi$-calculus is a well-defined notion of context,
contextual-equivalence and a correlation between
contextual-equivalence and notions of bisimulation. The notion of
context allows the decomposition of a process into (sub-)process and
its syntactic environment, its context. Thus, a context may be
thought of as a process with a ``hole'' (written $\Box$) in it. The
application of a context $M$ to a process $P$, written $M[P]$, is
tantamount to filling the hole in $M$ with $P$. In this paper we do
not need the full weight of this theory, but do make use of the notion
of context in the proof the main theorem. 

\begin{mathpar}
  \inferrule* [lab=summation] {} {{M_{M},M_{N}} \bc \Box \;|\; x.M_{A} \;|\; M_{M}+M_{N}}
  \and
  \inferrule* [lab=agent] {} {{M_{A}} \bc (\vec{x})M_{P} \;| \; \clift{P_0,\ldots,M_{P},\ldots,P_N}}
  \and \\
  \inferrule* [lab=process] {} {{M_{P}} \bc M_{N} \;| \;P|M_{P} }
\end{mathpar} 

\begin{mathpar}
  \inferrule* [lab=sychronization] {} {M_{N} \bc \Box \;|\; x?M_{F} \;|\; x!M_{C}}
  \and
  \inferrule* [lab=abstraction] {} {{M_{F}} \bc (x)M_{P} }
  \and
  \inferrule* [lab=concretion] {} {{M_{C}} \bc \langle M_{P} \rangle }
  \and \\
  \inferrule* [lab=process] {} {{M_{P}} \bc M_{N} \;| \;P|M_{P} }
\end{mathpar}

\begin{definition}[contextual application] Given a context $M$, and
  process $P$, we define the \emph{contextual application}, $M[P] :=
  M\{P/\Box\}$. That is, the contextual application of M to P is the
  substitution of $P$ for $\Box$ in $M$.
\end{definition}

$\meaningof{-} : L \to \mathcal{P}(\pi)$

\begin{mathpar}
  \inferrule* [lab=collection] {} {\meaningof{true} = \pi, \and \meaningof{~E} = \pi \setminus \meaningof{E}, \and \meaningof{E_{1} \& E_{2}} = \meaningof{E_{1}} \cap \meaningof{E_{2}}}
\end{mathpar}

\begin{mathpar}
  \inferrule* [lab=structure] {} {\meaningof{0} = \{ P \in \pi | P \equiv 0 \}, \and \\ \meaningof{E_1 | E_2} = \{ P \in \pi | P \equiv P_{1} | P_{2}, P_{1} \in \meaningof{E_{1}}, P_{2} \in \meaningof{E_2}\} }
\end{mathpar}

\begin{mathpar}
 \inferrule* [lab=behavior] {} {\meaningof{\langle a?b \rangle E} = \{ P \in \pi | P \equiv Q | u?(y)P', \\ \and \\\\ \and \\ \;\;\; u \in \meaningof{a}, \forall z.P'\{z/y\} \in \meaningof{E\{z/b\}}\}, \and \\ \meaningof{a!E} = \{ P \in \pi | P \equiv Q | x!\langle P' \rangle, x \in \meaningof{a} P' \in \meaningof{E}\} }
\end{mathpar}

\begin{mathpar}
 \inferrule* [lab=nominal] {} {\meaningof{\quotep{E}} = \{ \quotep{P} \in \quotep{\pi} | P \in \meaningof{E} \}, \and \meaningof{\quotep{P}} = \{ \quotep{Q} \in \quotep{\pi} | P \equiv Q \} \and \\ \meaningof{@\quotep{E}} = \{ P \in \pi | P \equiv @x, x \in \meaningof{E} \}}
\end{mathpar}

\begin{eqnarray*}
  \\
  \meaningof{-} : TS \to ST
\end{eqnarray*}

\begin{eqnarray*}
  \\
  L : TS \to ST
\end{eqnarray*}

\begin{eqnarray*}
  \\
  P \models E \iff P \in \meaningof{E}
\end{eqnarray*}

\begin{eqnarray*}
  P \approx_{L} Q \iff \forall E \in L. P \models E \iff Q \models E
\end{eqnarray*}

\begin{eqnarray*}
  P \approx_{K} Q
\end{eqnarray*}

\begin{eqnarray*}
  P \approx Q
\end{eqnarray*}

$\approx_{K} = \approx = \approx_{L}$

\subsubsection{Contextual duality}

Note that contexts extend the quotation operation to a family of
operations from processes to names. Given a context, $M$, we can
define a \emph{nominal context}, $\quotep{M}$ by $\quotep{M}[P] :=
\quotep{M[P]}$. To foreshadow what is to come we observe that these
operations enjoy a duality with processes very much like the duality
between vectors and maps from vectors to scalars.

Further, because the calculus is essentially higher-order, we have a
correspondence between contexts and processes. More specifically,
given a name $x$ and a context $M$ we can construct $M^{*}_{x}$ such
that 

\begin{mathpar}
  M^{*}_{x} | \lift{x}{P} \red M[P]
\end{mathpar}

namely,

\begin{mathpar}
  M^{*}_{x} := x?(u).M[\dropn{u}]
\end{mathpar}

The dependence of $M^{*}_{x}$ on a name makes it an abstraction, 

\begin{mathpar}
  M^{*} := (x)x?(u).M[\dropn{u}]
\end{mathpar}

\subsection{Additional notation}

It will sometimes be convenient to denote the process a name
quotes. We already have the notation $x = \quotep{P}$, but it will be
convenient to introduce an alternate notation, $\procn{x}$, when we
want to emphasize the connection to the use of the name. Note that, by
virtue of name equivalence, $\quotep{\procn{x}} \nameeq x$; so, the
notation is consistent with previous definitions.

Further, because names have structure it is possible to effect
substitutions on the basis of that structure. This means we need to
upgrade our notation for substitutions, which we accomplish by
adapting comprehension notation. Thus,

\begin{mathpar}
  P\{ y / x : x \in S \}
\end{mathpar}

is interpreted to mean the process derived from P by replacing (in a
capture-avoiding manner) each occurrence of $x$ in $S$ by $y$. For example,

\begin{mathpar}
  P\{ \quotep{\procn{x}|\procn{x}} / x : x \in \freenames{P} \}
\end{mathpar}

will replace each (occurrence) of a free name $x$ in $P$ by
$\quotep{\procn{x}|\procn{x}}$.

Also, we will avail ourselves of the notation $x^{L}$ and $x^{R}$ to
denote injections of a name into disjoint copies of the name
space. There are numerous ways to accomplish this. One example can be
found in \cite{MeredithR05}. This notation overloads to vectors of
names: $\vec{x}^{\pi} := (x_{i}^{\pi} \; : \; 0 \leq i < |\vec{x}| )$ where $\pi \in \{L,R\}$.

We also use $P^{\Box} := P|\Box$.

In \cite{MeredithR05} an interpretation of the new operator is
given. It turns out that there are several possible interpretations
all enjoying the requisite algebraic properties of the operator (see
\cite{milner91polyadicpi}). We will therefore make liberal use of
$(\nu\; \vec{x})P$.

% subsection the_syntax_and_semantics_of_the_notation_system (end)   

\input{qm2pi.qmops} 

\input{qm2pi.sterngerlach} 

\input{qm2pi.metric} 

% section concurrent_process_calculi (end)

%\input{qm2pi.proofsketch}

% section proof sketch (end)

%\input{qm2pi.slviaknots} 

% section spatial logic via knots (end)

\input{qm2pi.conclusion}

% section conclusion (end)

%\input{qm2pi.dtcodes} 

% section wiring algorithm (end)

\input{qm2pi.ack} 

% section acknowledgments (end)

\newpage


\bibliographystyle{plain}   
\bibliography{../../biblios/main.bib}

\input{qm2pi.rhodetails}

\end{document}

 

% section notation (end)

\input{qm2pi.process.calculi} 

% section concurrent_process_calculi_and_spatial_logics_ (end)
    
%\documentclass[12pt]{llncs}
%\documentclass{jktr}

\usepackage[pdftex]{hyperref}                   
\usepackage {listings}
\usepackage {mathpartir}
\usepackage{bcprules}
%\usepackage{listings}
                       
\usepackage{graphicx} 
%\usepackage[margins=2.5cm,nohead,nofoot]{geometry}
%\usepackage{geometry}
\usepackage{amsfonts}
\usepackage{amstext}
\usepackage{latexsym}
\usepackage{amssymb}
\usepackage{color}


%\include{myPreamble}
\include{qm2pi.local} 

%\ifpdf
%\usepackage[pdftex]{graphicx}
%\else
%\usepackage{graphicx}
%\fi

 % \ifpdf
%  \usepackage{pdfsync}
%  \if


%\title{Brief Article}
%\author{David F. Snyder}
%\author{L.G. Meredith}

%\address{Dept. of Math., Texas State University--San Marcos, San Marcos, TX 78666}
       
\pagestyle{empty}


\begin{document}

\lstset{language=[Objective]Caml,frame=shadowbox}

\input{qm2pi.front}

% section front matter (end)

\input{qm2pi.intro} 
 
% section introduction (end)

% \input{qm2pi.knotations} 

% section notation (end)

\input{qm2pi.process.calculi} 

% section concurrent_process_calculi_and_spatial_logics_ (end)
    
%\input{qm2pi.knots2pi} 

%\input{qm2pi.trefoil} 

%\input{qm2pi.mainthm} 

% subsection basic_interpretation (end)

%\input{qm2pi.rho.presentation} 
\subsection{The syntax and semantics of the notation system}\label{sub:the_syntax_and_semantics_of_the_notation_system} % (fold)

We now summarize a technical presentation of the calculus that
embodies our theory of dynamics. The typical presentation of such a
calculus follows the style of giving generators and relations on
them. The grammar, below, describing term constructors, freely
generates the set of processes, $\Proc$. This set is then quotiented
by a relation known as structural congruence and it is over this set
that the notion of dynamics is expressed. This presentation is
essentially that of \cite{MeredithR05} with the addition of
polyadicity and summation. For readability we have relegated some of
the technical subtleties to an appendix.

\subsubsection{Process grammar}\label{subsub:process_grammar}

\begin{mathpar}
  \inferrule* [lab=synchronization] {} {{M} \bc \pzero \;|\; x?F \;|\; x!C }
  \and
  \inferrule* [lab=abstraction] {} {{F} \bc (x)P}
  \and
  \inferrule* [lab=concretion] {} {{C} \bc \langle Q \rangle}
  \and
  \inferrule* [lab=process] {} {{P,Q} \bc M \;| \;P|Q \;|\; @{x}}
  \and
  \inferrule* [lab=name] {} {{x} \bc \quotep{P}}
\end{mathpar} 

Note that $\vec{x}$ (resp. $\vec{P}$) denotes a vector of names
(resp. processes) of length $|\vec{x}|$ (resp. $|\vec{P}|$). We adopt
the following useful abbreviations.

\begin{mathpar}
   x?(\vec{y}).P := x.(\vec{y})P \and  x\clift{\vec{P}} := x.\clift{\vec{P}}
   \and x!(y) := \lift{x}{\dropn{y}}
   \and \Pi_{i=0}^{n-1}P_i := P_0 | \ldots | P_{n-1}
\end{mathpar}

\subsubsection{Structural congruence}

\paragraph{Free and bound names and alpha-equivalence.} At the
core of structural equivalence is alpha-equivalence which identifies
process that are the same up to a change of variable. Formally, we
recognize the distinction between free and bound names. The free names
of a process, $\freenames{P}$, may be calculated recursively as
follows:

\begin{mathpar}
\freenames{\pzero} := \emptyset
  \and \\
  \freenames{x?(y).P} := \{ x \} \cup (\freenames{P} \setminus \{ y \})
  \and 
  \freenames{x!\langle P \rangle} := \{ x \} \cup \{ P \} 
  \and \\
  \freenames{P|Q} := \freenames{P} \cup \freenames{Q}
  \and \\
  \freenames{@{x}} := \{ x \}
\end{mathpar}

$\pi$
$\quotep{\pi}$

$\freenames{-} : \pi \to \mathcal{P}(\quotep{\pi})$

\begin{eqnarray*}
  \freenames{\pzero} & := & \emptyset \\
  \freenames{x?(y).P} & := & \{ x \} \cup (\freenames{P} \setminus \{ y \}) \\
  \freenames{x!\langle P \rangle} & := & \{ x \} \cup \{ P \} \\
  \freenames{P|Q} & := & \freenames{P} \cup \freenames{Q} \\
  \freenames{\dropn{x}} & := & \{ x \}
\end{eqnarray*}

The bound names of a process, $\boundnames{P}$, are those names occurring in $P$
that are not free. For example, in $x?(y).0$, the name $x$ is free, while $y$ is bound.

\begin{mathpar}
  \inferrule* [lab=monoidal-laws] {} { P|Q \equiv Q|P \and P|0 \equiv P \and P|(Q|R) \equiv (P|Q)|R }
\end{mathpar}

\begin{mathpar}
  \inferrule* [lab=alpha-equivalence] {} { (x)P \equiv (y)P\{y/x\} \and y \not\in \freenames{P} }
\end{mathpar}

\begin{definition}
Then two processes, $P,Q$, are alpha-equivalent if $P = Q\{\vec{y}/\vec{x}\}$ for
some $\vec{x} \in \boundnames{Q},\vec{y} \in \boundnames{P}$, where $Q\{\vec{y}/\vec{x}\}$
denotes the capture-avoiding substitution of $\vec{y}$ for $\vec{x}$ in $Q$.
\end{definition}

\begin{definition}
  The {\em structural congruence} \cite{SangiorgiWalker} , $\equiv$,
  between processes is the least congruence containing
  alpha-equivalence, satisfying the abelian monoid laws
  (associativity, commutativity and $\pzero$ as identity) for parallel
  composition $|$ and for summation $+$.
\end{definition}

\subsection{Name equivalence}

We take name equivalence, written $\nameeq$, to be the smallest
equivalence relation generated by the following rules.

\begin{mathpar}
\inferrule*[lab=Quote-drop]
{ }
{ \quotep{@{x}} \nameeq x }

\inferrule*[lab=Struct-equiv]
{ P \scong Q }
{ \quotep{P} \nameeq \quotep{Q} }
\end{mathpar}

The astute reader will have noticed that the mutual recursion of names
and processes imposes a mutual recursion on alpha-equivalence and
structural equivalence via name-equivalence. Fortunately, all of this
works out pleasantly and we may calculate in the natural way, free of
concern. The reader interested in the details is referred to the
appendix \ref{appendix:rho_details}.

\subsection{Substitution}

We use $\Proc$ for the set of processes, $\QProc$ for the set of
names, and $\id{\{}\vec{y} / \vec{x} \id{\}}$ to denote partial maps,
$s : \QProc \rightarrow \QProc$. A map, $s$ lifts, uniquely, to a map
on process terms, $\widehat{s} : \Proc \rightarrow \Proc$ by the
following equations.

\begin{mathpar}
  (0) \psubstp{Q}{P} := 0 \\
  (R \juxtap S) \psubstp{Q}{P}
  :=    
  (R)\psubstp{Q}{P} \juxtap (S) \psubstp{Q}{P} \\
  (x?(y).R) \psubstp{Q}{P}    
  :=    
  (x)\substp{Q}{P} (z)\concat( (R \psubstn{z}{y}) \psubstp{Q}{P} ) \\
  (\lift{x}{R}) \psubstp{Q}{P}  
  :=
  \lift{(x)\substp{Q}{P}}{ R \psubstp{Q}{P} } \\
%   (\dropn{x})  \psubstp{Q}{P}       
%   := 
%   \left\{ 
%     \begin{array}{ccc} 
%       \dropn{\quotep{Q}} & & x \nameeq \quotep{P} \\
%       \dropn{x} & & otherwise \\
%     \end{array}
%   \right. 
  (\dropn{x})  \psubstp{Q}{P}       
  := 
  \left\{ 
    \begin{array}{ccc} 
      Q & & x \nameeq \quotep{P} \\
      \dropn{x} & & otherwise \\
    \end{array}
  \right.
\end{mathpar}
 

where

\begin{eqnarray}
  (x)\id{\{} \lpquote Q \rpquote / \lpquote P \rpquote \id{\}}            = 
  \left\{ 
    \begin{array}{ccc}
      \lpquote Q \rpquote & & x \nameeq \lpquote P \rpquote \\
      x & & otherwise \\
    \end{array}
  \right. \nonumber
\end{eqnarray}

and $z$ is chosen distinct from $\quotep{P}$, $\quotep{Q}$, the free
names in $Q$, and all the names in $R$. Our $\alpha$-equivalence will
be built in the standard way from this substitution.

\begin{remark}\label{rem:no_self_referential_names}
  One consequence of these definitions is that $\forall P. \quotep{P}
  \not\in \freenames{P}$.
\end{remark}

\subsection{ Dynamic quote: an example }

Anticipating something of what's to come, consider applying the
substitution, $\widehat{\id{\{}u / z \id{\}}}$, to the following pair
of processes, $\lift{w}{y!(z)}$ and $w[ \lpquote y!(z) \rpquote ]$.

\begin{eqnarray}
	\lift{w}{y!(z)}\widehat{\id{\{}u / z \id{\}}}
		& = &
		\lift{w}{y!(u)} \nonumber\\
	w[ \lpquote y!(z) \rpquote ] \widehat{ \id{\{}u / z \id{\}} }
		& = &
		w[ \lpquote y!(z) \rpquote ] \nonumber
\end{eqnarray}

Because the body of the process between quotes is impervious to
substitution, we get radically different answers. In fact, by
examining the first process in an input context,
e.g. $x?(z).\lift{w}{y!(z)}$, we see that the process under the lift
operator may be shaped by prefixed inputs binding a name inside it. In
this sense, the lift operator will be seen as a way to dynamically
construct processes before reifying them as names.

Finally equipped with these standard features we can present the
dynamics of the calculus.

\subsubsection{Operational semantics} 

Finally, we introduce the computational dynamics. What marks these
algebras as distinct from other more traditionally studied algebraic
structures, e.g. vector spaces or polynomial rings, is the manner in
which dynamics is captured. In traditional structures, dynamics is typically
expressed through morphisms between such structures, as in linear maps
between vector spaces or morphisms between rings. In algebras
associated with the semantics of computation, the dynamics is
expressed as part of the algebraic structure itself, through a
reduction reduction relation typically denoted by $\red$. Below, we
give a recursive presentation of this relation for the calculus used
in the encoding.

$\red \subseteq \pi \times \pi$
$\red : \pi \to \mathcal{P}(\pi)$

\begin{mathpar}
  \inferrule* [lab=Comm] { \textsf{match}( x_{src}, x_{trgt} ) } { x_{trgt}?(y)P \; | \; x_{src}!\langle {Q} \rangle \red P\{\quotep{Q}/y}\} }
  \and \\
  \inferrule* [lab=Par] {{P} \red {P}'} {{{P} | {Q}} \red {{P}' | {Q}}}
  \and
  \inferrule* [lab=Equiv]{{{P} \scong {P}'} \andalso {{P}' \red {Q}'} \andalso {{Q}' \scong {Q}}}{{P} \red {Q}}
\end{mathpar}

\begin{eqnarray*}
  match_{\equiv} (\quotep{P},\quotep{Q}) & := & P \equiv Q \\
  match_{\dagger}(\quotep{P},\quotep{Q}) & := & \forall R. P|Q \red^{*} R => R \red^{*} 0 \\
  match_{K}(\quotep{P},\quotep{Q}) & := & K \mbox{ for some context } K
\end{eqnarray*}

$u?(x)P | u!\langle Q \rangle \red P\{\quotep{Q}/x\}$

%We write $\wred$ for $\red^*$, and $P\red$ if $\exists Q $ such that $ P \red Q$.
We write $P\red$ if $\exists Q $ such that $ P \red Q$ and $P\not\red$, otherwise.

\section{Replication}

As mentioned before, it is known that replication (and hence
recursion) can be implemented in a higher-order process algebra
\cite{SangiorgiWalker}. As our first example of calculation with the
machinery thus far presented we give the construction explicitly in
the {\rhoc}.

\begin{eqnarray}
	D_{x} & := & \prefix{x}{y}{(\binpar{\outputp{x}{y}}{@{y}})} \nonumber\\
	\bangp_{x}{P} & := & \binpar{{x}!\langle{\binpar{D_{x}}{P}}\rangle}{D_{x}} \nonumber
\end{eqnarray}

\begin{eqnarray}
	\bangp_{x}{P} & & \nonumber\\
	=
	& {x}!\langle{(\prefix{x}{y}{(\outputp{x}{y} | @{y})) | P}}\rangle 
	      | \prefix{x}{y}{(\outputp{x}{y} | @{y})} & \nonumber\\
	\red
	& (\outputp{x}{y} | @{y})\substn{\quotep{(\prefix{x}{y}{(@{y} | \outputp{x}{y})) | P}}}{y} & \nonumber\\
	=
	& \outputp{x}{\quotep{(\prefix{x}{y}{(\outputp{x}{y} | @{y})) | P}}}
	  | {(\prefix{x}{y}{(\outputp{x}{y} | @{y})) | P}} & \nonumber\\
	\red
	& \ldots & \nonumber\\
	\red^*
	& P | P | \ldots & \nonumber
\end{eqnarray}

Of course, this encoding, as an implementation, runs away, unfolding
$\bangp{P}$ eagerly. A lazier and more implementable replication
operator, restricted to input-guarded processes, may be obtained as follows.

\begin{eqnarray}
\bangp{\prefix{u}{v}{P}} 
	:= 
	\binpar{\lift{x}{\prefix{u}{v}{(\binpar{D(x)}{P})}}}{D(x)} \nonumber
\end{eqnarray}

\begin{remark}
  Note that the lazier definition still does not deal with summation
  or mixed summation (i.e. sums over input and output). The reader is
  invited to construct definitions of replication that deal with these
  features. 

  Further, the definitions are parameterized in a name, $x$. Can you,
  gentle reader, make a definition that eliminates this parameter and
  guarantees no accidental interaction between the replication
  machinery and the process being replicated -- i.e. no accidental
  sharing of names used by the process to get its work done and the
  name(s) used by the replication to effect copying. This latter
  revision of the definition of replication is crucial to obtaining
  the expected identity $!!P \sim !P$.
\end{remark}

\begin{remark}\label{rem:paradoxical_combinator}
  The reader familiar with the lambda calculus will have noticed the
  similarity between $D$ and the paradoxical combinator.

  [Ed. note: the existence of this seems to suggest we have to be more
  restrictive on the set of processes and names we admit if we are to
  support no-cloning.]
\end{remark}

\subsubsection{Bisimulation}

The computational dynamics gives rise to another kind of equivalence,
the equivalence of computational behavior. As previously mentioned
this is typically captured \emph{via} some form of bisimulation.

% The notion we use in this paper is weak barbed bisimulation
% \cite{milner91polyadicpi}.

The notion we use in this paper is derived from weak barbed
bisimulation \cite{milner91polyadicpi}. 

\begin{definition}
An \emph{observation relation}, $\downarrow_{\mathcal N}$, over a set
of names, $\mathcal N$, is the smallest relation satisfying the rules
below.

\infrule[Out-barb]{y \in {\mathcal N}, \; x \nameeq y}
		  {\outputp{x}{v} \downarrow_{\mathcal N} x}
\infrule[Par-barb]{\mbox{$P\downarrow_{\mathcal N} x$ or $Q\downarrow_{\mathcal N} x$}}
		  {\binpar{P}{Q} \downarrow_{\mathcal N} x}

We write $P \Downarrow_{\mathcal N} x$ if there is $Q$ such that 
$P \wred Q$ and $Q \downarrow_{\mathcal N} x$.
\end{definition}

\begin{definition}
%\label{def.bbisim}
An  ${\mathcal N}$-\emph{barbed bisimulation} over a set of names, ${\mathcal N}$, is a symmetric binary relation 
${\mathcal S}_{\mathcal N}$ between agents such that $P\rel{S}_{\mathcal N}Q$ implies:
\begin{enumerate}
\item If $P \red P'$ then $Q \wred Q'$ and $P'\rel{S}_{\mathcal N} Q'$.
\item If $P\downarrow_{\mathcal N} x$, then $Q\Downarrow_{\mathcal N} x$.
\end{enumerate}
$P$ is ${\mathcal N}$-barbed bisimilar to $Q$, written
$P \wbbisim_{\mathcal N} Q$, if $P \rel{S}_{\mathcal N} Q$ for some ${\mathcal N}$-barbed bisimulation ${\mathcal S}_{\mathcal N}$.
\end{definition}

$\mathcal{R} \subseteq \pi \times \pi$

$P \mathcal{R} Q => \forall P'. P \red P' \Rightarrow \exists Q'. Q \red Q', P' \mathcal{R} Q'$

$P \vdash x \Rightarrow Q \vdash x$

\begin{mathpar}
  \inferrule*[lab=Out-barb]{x \nameeq y}{{y}!\langle{Q}\rangle \vdash x}
  \and
  \inferrule*[lab=Par-barb]{\mbox{$P\vdash x$ or $Q\vdash x$}}{\binpar{P}{Q} \vdash x}
\end{mathpar}

\subsubsection{Contexts}

One of the principle advantages of computational calculi like the
$\pi$-calculus is a well-defined notion of context,
contextual-equivalence and a correlation between
contextual-equivalence and notions of bisimulation. The notion of
context allows the decomposition of a process into (sub-)process and
its syntactic environment, its context. Thus, a context may be
thought of as a process with a ``hole'' (written $\Box$) in it. The
application of a context $M$ to a process $P$, written $M[P]$, is
tantamount to filling the hole in $M$ with $P$. In this paper we do
not need the full weight of this theory, but do make use of the notion
of context in the proof the main theorem. 

\begin{mathpar}
  \inferrule* [lab=summation] {} {{M_{M},M_{N}} \bc \Box \;|\; x.M_{A} \;|\; M_{M}+M_{N}}
  \and
  \inferrule* [lab=agent] {} {{M_{A}} \bc (\vec{x})M_{P} \;| \; \clift{P_0,\ldots,M_{P},\ldots,P_N}}
  \and \\
  \inferrule* [lab=process] {} {{M_{P}} \bc M_{N} \;| \;P|M_{P} }
\end{mathpar} 

\begin{mathpar}
  \inferrule* [lab=sychronization] {} {M_{N} \bc \Box \;|\; x?M_{F} \;|\; x!M_{C}}
  \and
  \inferrule* [lab=abstraction] {} {{M_{F}} \bc (x)M_{P} }
  \and
  \inferrule* [lab=concretion] {} {{M_{C}} \bc \langle M_{P} \rangle }
  \and \\
  \inferrule* [lab=process] {} {{M_{P}} \bc M_{N} \;| \;P|M_{P} }
\end{mathpar}

\begin{definition}[contextual application] Given a context $M$, and
  process $P$, we define the \emph{contextual application}, $M[P] :=
  M\{P/\Box\}$. That is, the contextual application of M to P is the
  substitution of $P$ for $\Box$ in $M$.
\end{definition}

$\meaningof{-} : L \to \mathcal{P}(\pi)$

\begin{mathpar}
  \inferrule* [lab=collection] {} {\meaningof{true} = \pi, \and \meaningof{~E} = \pi \setminus \meaningof{E}, \and \meaningof{E_{1} \& E_{2}} = \meaningof{E_{1}} \cap \meaningof{E_{2}}}
\end{mathpar}

\begin{mathpar}
  \inferrule* [lab=structure] {} {\meaningof{0} = \{ P \in \pi | P \equiv 0 \}, \and \\ \meaningof{E_1 | E_2} = \{ P \in \pi | P \equiv P_{1} | P_{2}, P_{1} \in \meaningof{E_{1}}, P_{2} \in \meaningof{E_2}\} }
\end{mathpar}

\begin{mathpar}
 \inferrule* [lab=behavior] {} {\meaningof{\langle a?b \rangle E} = \{ P \in \pi | P \equiv Q | u?(y)P', \\ \and \\\\ \and \\ \;\;\; u \in \meaningof{a}, \forall z.P'\{z/y\} \in \meaningof{E\{z/b\}}\}, \and \\ \meaningof{a!E} = \{ P \in \pi | P \equiv Q | x!\langle P' \rangle, x \in \meaningof{a} P' \in \meaningof{E}\} }
\end{mathpar}

\begin{mathpar}
 \inferrule* [lab=nominal] {} {\meaningof{\quotep{E}} = \{ \quotep{P} \in \quotep{\pi} | P \in \meaningof{E} \}, \and \meaningof{\quotep{P}} = \{ \quotep{Q} \in \quotep{\pi} | P \equiv Q \} \and \\ \meaningof{@\quotep{E}} = \{ P \in \pi | P \equiv @x, x \in \meaningof{E} \}}
\end{mathpar}

\begin{eqnarray*}
  \\
  \meaningof{-} : TS \to ST
\end{eqnarray*}

\begin{eqnarray*}
  \\
  L : TS \to ST
\end{eqnarray*}

\begin{eqnarray*}
  \\
  P \models E \iff P \in \meaningof{E}
\end{eqnarray*}

\begin{eqnarray*}
  P \approx_{L} Q \iff \forall E \in L. P \models E \iff Q \models E
\end{eqnarray*}

\begin{eqnarray*}
  P \approx_{K} Q
\end{eqnarray*}

\begin{eqnarray*}
  P \approx Q
\end{eqnarray*}

$\approx_{K} = \approx = \approx_{L}$

\subsubsection{Contextual duality}

Note that contexts extend the quotation operation to a family of
operations from processes to names. Given a context, $M$, we can
define a \emph{nominal context}, $\quotep{M}$ by $\quotep{M}[P] :=
\quotep{M[P]}$. To foreshadow what is to come we observe that these
operations enjoy a duality with processes very much like the duality
between vectors and maps from vectors to scalars.

Further, because the calculus is essentially higher-order, we have a
correspondence between contexts and processes. More specifically,
given a name $x$ and a context $M$ we can construct $M^{*}_{x}$ such
that 

\begin{mathpar}
  M^{*}_{x} | \lift{x}{P} \red M[P]
\end{mathpar}

namely,

\begin{mathpar}
  M^{*}_{x} := x?(u).M[\dropn{u}]
\end{mathpar}

The dependence of $M^{*}_{x}$ on a name makes it an abstraction, 

\begin{mathpar}
  M^{*} := (x)x?(u).M[\dropn{u}]
\end{mathpar}

\subsection{Additional notation}

It will sometimes be convenient to denote the process a name
quotes. We already have the notation $x = \quotep{P}$, but it will be
convenient to introduce an alternate notation, $\procn{x}$, when we
want to emphasize the connection to the use of the name. Note that, by
virtue of name equivalence, $\quotep{\procn{x}} \nameeq x$; so, the
notation is consistent with previous definitions.

Further, because names have structure it is possible to effect
substitutions on the basis of that structure. This means we need to
upgrade our notation for substitutions, which we accomplish by
adapting comprehension notation. Thus,

\begin{mathpar}
  P\{ y / x : x \in S \}
\end{mathpar}

is interpreted to mean the process derived from P by replacing (in a
capture-avoiding manner) each occurrence of $x$ in $S$ by $y$. For example,

\begin{mathpar}
  P\{ \quotep{\procn{x}|\procn{x}} / x : x \in \freenames{P} \}
\end{mathpar}

will replace each (occurrence) of a free name $x$ in $P$ by
$\quotep{\procn{x}|\procn{x}}$.

Also, we will avail ourselves of the notation $x^{L}$ and $x^{R}$ to
denote injections of a name into disjoint copies of the name
space. There are numerous ways to accomplish this. One example can be
found in \cite{MeredithR05}. This notation overloads to vectors of
names: $\vec{x}^{\pi} := (x_{i}^{\pi} \; : \; 0 \leq i < |\vec{x}| )$ where $\pi \in \{L,R\}$.

We also use $P^{\Box} := P|\Box$.

In \cite{MeredithR05} an interpretation of the new operator is
given. It turns out that there are several possible interpretations
all enjoying the requisite algebraic properties of the operator (see
\cite{milner91polyadicpi}). We will therefore make liberal use of
$(\nu\; \vec{x})P$.

% subsection the_syntax_and_semantics_of_the_notation_system (end)   

\input{qm2pi.qmops} 

\input{qm2pi.sterngerlach} 

\input{qm2pi.metric} 

% section concurrent_process_calculi (end)

%\input{qm2pi.proofsketch}

% section proof sketch (end)

%\input{qm2pi.slviaknots} 

% section spatial logic via knots (end)

\input{qm2pi.conclusion}

% section conclusion (end)

%\input{qm2pi.dtcodes} 

% section wiring algorithm (end)

\input{qm2pi.ack} 

% section acknowledgments (end)

\newpage


\bibliographystyle{plain}   
\bibliography{../../biblios/main.bib}

\input{qm2pi.rhodetails}

\end{document}

 

%\documentclass[12pt]{llncs}
%\documentclass{jktr}

\usepackage[pdftex]{hyperref}                   
\usepackage {listings}
\usepackage {mathpartir}
\usepackage{bcprules}
%\usepackage{listings}
                       
\usepackage{graphicx} 
%\usepackage[margins=2.5cm,nohead,nofoot]{geometry}
%\usepackage{geometry}
\usepackage{amsfonts}
\usepackage{amstext}
\usepackage{latexsym}
\usepackage{amssymb}
\usepackage{color}


%\include{myPreamble}
\include{qm2pi.local} 

%\ifpdf
%\usepackage[pdftex]{graphicx}
%\else
%\usepackage{graphicx}
%\fi

 % \ifpdf
%  \usepackage{pdfsync}
%  \if


%\title{Brief Article}
%\author{David F. Snyder}
%\author{L.G. Meredith}

%\address{Dept. of Math., Texas State University--San Marcos, San Marcos, TX 78666}
       
\pagestyle{empty}


\begin{document}

\lstset{language=[Objective]Caml,frame=shadowbox}

\input{qm2pi.front}

% section front matter (end)

\input{qm2pi.intro} 
 
% section introduction (end)

% \input{qm2pi.knotations} 

% section notation (end)

\input{qm2pi.process.calculi} 

% section concurrent_process_calculi_and_spatial_logics_ (end)
    
%\input{qm2pi.knots2pi} 

%\input{qm2pi.trefoil} 

%\input{qm2pi.mainthm} 

% subsection basic_interpretation (end)

%\input{qm2pi.rho.presentation} 
\subsection{The syntax and semantics of the notation system}\label{sub:the_syntax_and_semantics_of_the_notation_system} % (fold)

We now summarize a technical presentation of the calculus that
embodies our theory of dynamics. The typical presentation of such a
calculus follows the style of giving generators and relations on
them. The grammar, below, describing term constructors, freely
generates the set of processes, $\Proc$. This set is then quotiented
by a relation known as structural congruence and it is over this set
that the notion of dynamics is expressed. This presentation is
essentially that of \cite{MeredithR05} with the addition of
polyadicity and summation. For readability we have relegated some of
the technical subtleties to an appendix.

\subsubsection{Process grammar}\label{subsub:process_grammar}

\begin{mathpar}
  \inferrule* [lab=synchronization] {} {{M} \bc \pzero \;|\; x?F \;|\; x!C }
  \and
  \inferrule* [lab=abstraction] {} {{F} \bc (x)P}
  \and
  \inferrule* [lab=concretion] {} {{C} \bc \langle Q \rangle}
  \and
  \inferrule* [lab=process] {} {{P,Q} \bc M \;| \;P|Q \;|\; @{x}}
  \and
  \inferrule* [lab=name] {} {{x} \bc \quotep{P}}
\end{mathpar} 

Note that $\vec{x}$ (resp. $\vec{P}$) denotes a vector of names
(resp. processes) of length $|\vec{x}|$ (resp. $|\vec{P}|$). We adopt
the following useful abbreviations.

\begin{mathpar}
   x?(\vec{y}).P := x.(\vec{y})P \and  x\clift{\vec{P}} := x.\clift{\vec{P}}
   \and x!(y) := \lift{x}{\dropn{y}}
   \and \Pi_{i=0}^{n-1}P_i := P_0 | \ldots | P_{n-1}
\end{mathpar}

\subsubsection{Structural congruence}

\paragraph{Free and bound names and alpha-equivalence.} At the
core of structural equivalence is alpha-equivalence which identifies
process that are the same up to a change of variable. Formally, we
recognize the distinction between free and bound names. The free names
of a process, $\freenames{P}$, may be calculated recursively as
follows:

\begin{mathpar}
\freenames{\pzero} := \emptyset
  \and \\
  \freenames{x?(y).P} := \{ x \} \cup (\freenames{P} \setminus \{ y \})
  \and 
  \freenames{x!\langle P \rangle} := \{ x \} \cup \{ P \} 
  \and \\
  \freenames{P|Q} := \freenames{P} \cup \freenames{Q}
  \and \\
  \freenames{@{x}} := \{ x \}
\end{mathpar}

$\pi$
$\quotep{\pi}$

$\freenames{-} : \pi \to \mathcal{P}(\quotep{\pi})$

\begin{eqnarray*}
  \freenames{\pzero} & := & \emptyset \\
  \freenames{x?(y).P} & := & \{ x \} \cup (\freenames{P} \setminus \{ y \}) \\
  \freenames{x!\langle P \rangle} & := & \{ x \} \cup \{ P \} \\
  \freenames{P|Q} & := & \freenames{P} \cup \freenames{Q} \\
  \freenames{\dropn{x}} & := & \{ x \}
\end{eqnarray*}

The bound names of a process, $\boundnames{P}$, are those names occurring in $P$
that are not free. For example, in $x?(y).0$, the name $x$ is free, while $y$ is bound.

\begin{mathpar}
  \inferrule* [lab=monoidal-laws] {} { P|Q \equiv Q|P \and P|0 \equiv P \and P|(Q|R) \equiv (P|Q)|R }
\end{mathpar}

\begin{mathpar}
  \inferrule* [lab=alpha-equivalence] {} { (x)P \equiv (y)P\{y/x\} \and y \not\in \freenames{P} }
\end{mathpar}

\begin{definition}
Then two processes, $P,Q$, are alpha-equivalent if $P = Q\{\vec{y}/\vec{x}\}$ for
some $\vec{x} \in \boundnames{Q},\vec{y} \in \boundnames{P}$, where $Q\{\vec{y}/\vec{x}\}$
denotes the capture-avoiding substitution of $\vec{y}$ for $\vec{x}$ in $Q$.
\end{definition}

\begin{definition}
  The {\em structural congruence} \cite{SangiorgiWalker} , $\equiv$,
  between processes is the least congruence containing
  alpha-equivalence, satisfying the abelian monoid laws
  (associativity, commutativity and $\pzero$ as identity) for parallel
  composition $|$ and for summation $+$.
\end{definition}

\subsection{Name equivalence}

We take name equivalence, written $\nameeq$, to be the smallest
equivalence relation generated by the following rules.

\begin{mathpar}
\inferrule*[lab=Quote-drop]
{ }
{ \quotep{@{x}} \nameeq x }

\inferrule*[lab=Struct-equiv]
{ P \scong Q }
{ \quotep{P} \nameeq \quotep{Q} }
\end{mathpar}

The astute reader will have noticed that the mutual recursion of names
and processes imposes a mutual recursion on alpha-equivalence and
structural equivalence via name-equivalence. Fortunately, all of this
works out pleasantly and we may calculate in the natural way, free of
concern. The reader interested in the details is referred to the
appendix \ref{appendix:rho_details}.

\subsection{Substitution}

We use $\Proc$ for the set of processes, $\QProc$ for the set of
names, and $\id{\{}\vec{y} / \vec{x} \id{\}}$ to denote partial maps,
$s : \QProc \rightarrow \QProc$. A map, $s$ lifts, uniquely, to a map
on process terms, $\widehat{s} : \Proc \rightarrow \Proc$ by the
following equations.

\begin{mathpar}
  (0) \psubstp{Q}{P} := 0 \\
  (R \juxtap S) \psubstp{Q}{P}
  :=    
  (R)\psubstp{Q}{P} \juxtap (S) \psubstp{Q}{P} \\
  (x?(y).R) \psubstp{Q}{P}    
  :=    
  (x)\substp{Q}{P} (z)\concat( (R \psubstn{z}{y}) \psubstp{Q}{P} ) \\
  (\lift{x}{R}) \psubstp{Q}{P}  
  :=
  \lift{(x)\substp{Q}{P}}{ R \psubstp{Q}{P} } \\
%   (\dropn{x})  \psubstp{Q}{P}       
%   := 
%   \left\{ 
%     \begin{array}{ccc} 
%       \dropn{\quotep{Q}} & & x \nameeq \quotep{P} \\
%       \dropn{x} & & otherwise \\
%     \end{array}
%   \right. 
  (\dropn{x})  \psubstp{Q}{P}       
  := 
  \left\{ 
    \begin{array}{ccc} 
      Q & & x \nameeq \quotep{P} \\
      \dropn{x} & & otherwise \\
    \end{array}
  \right.
\end{mathpar}
 

where

\begin{eqnarray}
  (x)\id{\{} \lpquote Q \rpquote / \lpquote P \rpquote \id{\}}            = 
  \left\{ 
    \begin{array}{ccc}
      \lpquote Q \rpquote & & x \nameeq \lpquote P \rpquote \\
      x & & otherwise \\
    \end{array}
  \right. \nonumber
\end{eqnarray}

and $z$ is chosen distinct from $\quotep{P}$, $\quotep{Q}$, the free
names in $Q$, and all the names in $R$. Our $\alpha$-equivalence will
be built in the standard way from this substitution.

\begin{remark}\label{rem:no_self_referential_names}
  One consequence of these definitions is that $\forall P. \quotep{P}
  \not\in \freenames{P}$.
\end{remark}

\subsection{ Dynamic quote: an example }

Anticipating something of what's to come, consider applying the
substitution, $\widehat{\id{\{}u / z \id{\}}}$, to the following pair
of processes, $\lift{w}{y!(z)}$ and $w[ \lpquote y!(z) \rpquote ]$.

\begin{eqnarray}
	\lift{w}{y!(z)}\widehat{\id{\{}u / z \id{\}}}
		& = &
		\lift{w}{y!(u)} \nonumber\\
	w[ \lpquote y!(z) \rpquote ] \widehat{ \id{\{}u / z \id{\}} }
		& = &
		w[ \lpquote y!(z) \rpquote ] \nonumber
\end{eqnarray}

Because the body of the process between quotes is impervious to
substitution, we get radically different answers. In fact, by
examining the first process in an input context,
e.g. $x?(z).\lift{w}{y!(z)}$, we see that the process under the lift
operator may be shaped by prefixed inputs binding a name inside it. In
this sense, the lift operator will be seen as a way to dynamically
construct processes before reifying them as names.

Finally equipped with these standard features we can present the
dynamics of the calculus.

\subsubsection{Operational semantics} 

Finally, we introduce the computational dynamics. What marks these
algebras as distinct from other more traditionally studied algebraic
structures, e.g. vector spaces or polynomial rings, is the manner in
which dynamics is captured. In traditional structures, dynamics is typically
expressed through morphisms between such structures, as in linear maps
between vector spaces or morphisms between rings. In algebras
associated with the semantics of computation, the dynamics is
expressed as part of the algebraic structure itself, through a
reduction reduction relation typically denoted by $\red$. Below, we
give a recursive presentation of this relation for the calculus used
in the encoding.

$\red \subseteq \pi \times \pi$
$\red : \pi \to \mathcal{P}(\pi)$

\begin{mathpar}
  \inferrule* [lab=Comm] { \textsf{match}( x_{src}, x_{trgt} ) } { x_{trgt}?(y)P \; | \; x_{src}!\langle {Q} \rangle \red P\{\quotep{Q}/y}\} }
  \and \\
  \inferrule* [lab=Par] {{P} \red {P}'} {{{P} | {Q}} \red {{P}' | {Q}}}
  \and
  \inferrule* [lab=Equiv]{{{P} \scong {P}'} \andalso {{P}' \red {Q}'} \andalso {{Q}' \scong {Q}}}{{P} \red {Q}}
\end{mathpar}

\begin{eqnarray*}
  match_{\equiv} (\quotep{P},\quotep{Q}) & := & P \equiv Q \\
  match_{\dagger}(\quotep{P},\quotep{Q}) & := & \forall R. P|Q \red^{*} R => R \red^{*} 0 \\
  match_{K}(\quotep{P},\quotep{Q}) & := & K \mbox{ for some context } K
\end{eqnarray*}

$u?(x)P | u!\langle Q \rangle \red P\{\quotep{Q}/x\}$

%We write $\wred$ for $\red^*$, and $P\red$ if $\exists Q $ such that $ P \red Q$.
We write $P\red$ if $\exists Q $ such that $ P \red Q$ and $P\not\red$, otherwise.

\section{Replication}

As mentioned before, it is known that replication (and hence
recursion) can be implemented in a higher-order process algebra
\cite{SangiorgiWalker}. As our first example of calculation with the
machinery thus far presented we give the construction explicitly in
the {\rhoc}.

\begin{eqnarray}
	D_{x} & := & \prefix{x}{y}{(\binpar{\outputp{x}{y}}{@{y}})} \nonumber\\
	\bangp_{x}{P} & := & \binpar{{x}!\langle{\binpar{D_{x}}{P}}\rangle}{D_{x}} \nonumber
\end{eqnarray}

\begin{eqnarray}
	\bangp_{x}{P} & & \nonumber\\
	=
	& {x}!\langle{(\prefix{x}{y}{(\outputp{x}{y} | @{y})) | P}}\rangle 
	      | \prefix{x}{y}{(\outputp{x}{y} | @{y})} & \nonumber\\
	\red
	& (\outputp{x}{y} | @{y})\substn{\quotep{(\prefix{x}{y}{(@{y} | \outputp{x}{y})) | P}}}{y} & \nonumber\\
	=
	& \outputp{x}{\quotep{(\prefix{x}{y}{(\outputp{x}{y} | @{y})) | P}}}
	  | {(\prefix{x}{y}{(\outputp{x}{y} | @{y})) | P}} & \nonumber\\
	\red
	& \ldots & \nonumber\\
	\red^*
	& P | P | \ldots & \nonumber
\end{eqnarray}

Of course, this encoding, as an implementation, runs away, unfolding
$\bangp{P}$ eagerly. A lazier and more implementable replication
operator, restricted to input-guarded processes, may be obtained as follows.

\begin{eqnarray}
\bangp{\prefix{u}{v}{P}} 
	:= 
	\binpar{\lift{x}{\prefix{u}{v}{(\binpar{D(x)}{P})}}}{D(x)} \nonumber
\end{eqnarray}

\begin{remark}
  Note that the lazier definition still does not deal with summation
  or mixed summation (i.e. sums over input and output). The reader is
  invited to construct definitions of replication that deal with these
  features. 

  Further, the definitions are parameterized in a name, $x$. Can you,
  gentle reader, make a definition that eliminates this parameter and
  guarantees no accidental interaction between the replication
  machinery and the process being replicated -- i.e. no accidental
  sharing of names used by the process to get its work done and the
  name(s) used by the replication to effect copying. This latter
  revision of the definition of replication is crucial to obtaining
  the expected identity $!!P \sim !P$.
\end{remark}

\begin{remark}\label{rem:paradoxical_combinator}
  The reader familiar with the lambda calculus will have noticed the
  similarity between $D$ and the paradoxical combinator.

  [Ed. note: the existence of this seems to suggest we have to be more
  restrictive on the set of processes and names we admit if we are to
  support no-cloning.]
\end{remark}

\subsubsection{Bisimulation}

The computational dynamics gives rise to another kind of equivalence,
the equivalence of computational behavior. As previously mentioned
this is typically captured \emph{via} some form of bisimulation.

% The notion we use in this paper is weak barbed bisimulation
% \cite{milner91polyadicpi}.

The notion we use in this paper is derived from weak barbed
bisimulation \cite{milner91polyadicpi}. 

\begin{definition}
An \emph{observation relation}, $\downarrow_{\mathcal N}$, over a set
of names, $\mathcal N$, is the smallest relation satisfying the rules
below.

\infrule[Out-barb]{y \in {\mathcal N}, \; x \nameeq y}
		  {\outputp{x}{v} \downarrow_{\mathcal N} x}
\infrule[Par-barb]{\mbox{$P\downarrow_{\mathcal N} x$ or $Q\downarrow_{\mathcal N} x$}}
		  {\binpar{P}{Q} \downarrow_{\mathcal N} x}

We write $P \Downarrow_{\mathcal N} x$ if there is $Q$ such that 
$P \wred Q$ and $Q \downarrow_{\mathcal N} x$.
\end{definition}

\begin{definition}
%\label{def.bbisim}
An  ${\mathcal N}$-\emph{barbed bisimulation} over a set of names, ${\mathcal N}$, is a symmetric binary relation 
${\mathcal S}_{\mathcal N}$ between agents such that $P\rel{S}_{\mathcal N}Q$ implies:
\begin{enumerate}
\item If $P \red P'$ then $Q \wred Q'$ and $P'\rel{S}_{\mathcal N} Q'$.
\item If $P\downarrow_{\mathcal N} x$, then $Q\Downarrow_{\mathcal N} x$.
\end{enumerate}
$P$ is ${\mathcal N}$-barbed bisimilar to $Q$, written
$P \wbbisim_{\mathcal N} Q$, if $P \rel{S}_{\mathcal N} Q$ for some ${\mathcal N}$-barbed bisimulation ${\mathcal S}_{\mathcal N}$.
\end{definition}

$\mathcal{R} \subseteq \pi \times \pi$

$P \mathcal{R} Q => \forall P'. P \red P' \Rightarrow \exists Q'. Q \red Q', P' \mathcal{R} Q'$

$P \vdash x \Rightarrow Q \vdash x$

\begin{mathpar}
  \inferrule*[lab=Out-barb]{x \nameeq y}{{y}!\langle{Q}\rangle \vdash x}
  \and
  \inferrule*[lab=Par-barb]{\mbox{$P\vdash x$ or $Q\vdash x$}}{\binpar{P}{Q} \vdash x}
\end{mathpar}

\subsubsection{Contexts}

One of the principle advantages of computational calculi like the
$\pi$-calculus is a well-defined notion of context,
contextual-equivalence and a correlation between
contextual-equivalence and notions of bisimulation. The notion of
context allows the decomposition of a process into (sub-)process and
its syntactic environment, its context. Thus, a context may be
thought of as a process with a ``hole'' (written $\Box$) in it. The
application of a context $M$ to a process $P$, written $M[P]$, is
tantamount to filling the hole in $M$ with $P$. In this paper we do
not need the full weight of this theory, but do make use of the notion
of context in the proof the main theorem. 

\begin{mathpar}
  \inferrule* [lab=summation] {} {{M_{M},M_{N}} \bc \Box \;|\; x.M_{A} \;|\; M_{M}+M_{N}}
  \and
  \inferrule* [lab=agent] {} {{M_{A}} \bc (\vec{x})M_{P} \;| \; \clift{P_0,\ldots,M_{P},\ldots,P_N}}
  \and \\
  \inferrule* [lab=process] {} {{M_{P}} \bc M_{N} \;| \;P|M_{P} }
\end{mathpar} 

\begin{mathpar}
  \inferrule* [lab=sychronization] {} {M_{N} \bc \Box \;|\; x?M_{F} \;|\; x!M_{C}}
  \and
  \inferrule* [lab=abstraction] {} {{M_{F}} \bc (x)M_{P} }
  \and
  \inferrule* [lab=concretion] {} {{M_{C}} \bc \langle M_{P} \rangle }
  \and \\
  \inferrule* [lab=process] {} {{M_{P}} \bc M_{N} \;| \;P|M_{P} }
\end{mathpar}

\begin{definition}[contextual application] Given a context $M$, and
  process $P$, we define the \emph{contextual application}, $M[P] :=
  M\{P/\Box\}$. That is, the contextual application of M to P is the
  substitution of $P$ for $\Box$ in $M$.
\end{definition}

$\meaningof{-} : L \to \mathcal{P}(\pi)$

\begin{mathpar}
  \inferrule* [lab=collection] {} {\meaningof{true} = \pi, \and \meaningof{~E} = \pi \setminus \meaningof{E}, \and \meaningof{E_{1} \& E_{2}} = \meaningof{E_{1}} \cap \meaningof{E_{2}}}
\end{mathpar}

\begin{mathpar}
  \inferrule* [lab=structure] {} {\meaningof{0} = \{ P \in \pi | P \equiv 0 \}, \and \\ \meaningof{E_1 | E_2} = \{ P \in \pi | P \equiv P_{1} | P_{2}, P_{1} \in \meaningof{E_{1}}, P_{2} \in \meaningof{E_2}\} }
\end{mathpar}

\begin{mathpar}
 \inferrule* [lab=behavior] {} {\meaningof{\langle a?b \rangle E} = \{ P \in \pi | P \equiv Q | u?(y)P', \\ \and \\\\ \and \\ \;\;\; u \in \meaningof{a}, \forall z.P'\{z/y\} \in \meaningof{E\{z/b\}}\}, \and \\ \meaningof{a!E} = \{ P \in \pi | P \equiv Q | x!\langle P' \rangle, x \in \meaningof{a} P' \in \meaningof{E}\} }
\end{mathpar}

\begin{mathpar}
 \inferrule* [lab=nominal] {} {\meaningof{\quotep{E}} = \{ \quotep{P} \in \quotep{\pi} | P \in \meaningof{E} \}, \and \meaningof{\quotep{P}} = \{ \quotep{Q} \in \quotep{\pi} | P \equiv Q \} \and \\ \meaningof{@\quotep{E}} = \{ P \in \pi | P \equiv @x, x \in \meaningof{E} \}}
\end{mathpar}

\begin{eqnarray*}
  \\
  \meaningof{-} : TS \to ST
\end{eqnarray*}

\begin{eqnarray*}
  \\
  L : TS \to ST
\end{eqnarray*}

\begin{eqnarray*}
  \\
  P \models E \iff P \in \meaningof{E}
\end{eqnarray*}

\begin{eqnarray*}
  P \approx_{L} Q \iff \forall E \in L. P \models E \iff Q \models E
\end{eqnarray*}

\begin{eqnarray*}
  P \approx_{K} Q
\end{eqnarray*}

\begin{eqnarray*}
  P \approx Q
\end{eqnarray*}

$\approx_{K} = \approx = \approx_{L}$

\subsubsection{Contextual duality}

Note that contexts extend the quotation operation to a family of
operations from processes to names. Given a context, $M$, we can
define a \emph{nominal context}, $\quotep{M}$ by $\quotep{M}[P] :=
\quotep{M[P]}$. To foreshadow what is to come we observe that these
operations enjoy a duality with processes very much like the duality
between vectors and maps from vectors to scalars.

Further, because the calculus is essentially higher-order, we have a
correspondence between contexts and processes. More specifically,
given a name $x$ and a context $M$ we can construct $M^{*}_{x}$ such
that 

\begin{mathpar}
  M^{*}_{x} | \lift{x}{P} \red M[P]
\end{mathpar}

namely,

\begin{mathpar}
  M^{*}_{x} := x?(u).M[\dropn{u}]
\end{mathpar}

The dependence of $M^{*}_{x}$ on a name makes it an abstraction, 

\begin{mathpar}
  M^{*} := (x)x?(u).M[\dropn{u}]
\end{mathpar}

\subsection{Additional notation}

It will sometimes be convenient to denote the process a name
quotes. We already have the notation $x = \quotep{P}$, but it will be
convenient to introduce an alternate notation, $\procn{x}$, when we
want to emphasize the connection to the use of the name. Note that, by
virtue of name equivalence, $\quotep{\procn{x}} \nameeq x$; so, the
notation is consistent with previous definitions.

Further, because names have structure it is possible to effect
substitutions on the basis of that structure. This means we need to
upgrade our notation for substitutions, which we accomplish by
adapting comprehension notation. Thus,

\begin{mathpar}
  P\{ y / x : x \in S \}
\end{mathpar}

is interpreted to mean the process derived from P by replacing (in a
capture-avoiding manner) each occurrence of $x$ in $S$ by $y$. For example,

\begin{mathpar}
  P\{ \quotep{\procn{x}|\procn{x}} / x : x \in \freenames{P} \}
\end{mathpar}

will replace each (occurrence) of a free name $x$ in $P$ by
$\quotep{\procn{x}|\procn{x}}$.

Also, we will avail ourselves of the notation $x^{L}$ and $x^{R}$ to
denote injections of a name into disjoint copies of the name
space. There are numerous ways to accomplish this. One example can be
found in \cite{MeredithR05}. This notation overloads to vectors of
names: $\vec{x}^{\pi} := (x_{i}^{\pi} \; : \; 0 \leq i < |\vec{x}| )$ where $\pi \in \{L,R\}$.

We also use $P^{\Box} := P|\Box$.

In \cite{MeredithR05} an interpretation of the new operator is
given. It turns out that there are several possible interpretations
all enjoying the requisite algebraic properties of the operator (see
\cite{milner91polyadicpi}). We will therefore make liberal use of
$(\nu\; \vec{x})P$.

% subsection the_syntax_and_semantics_of_the_notation_system (end)   

\input{qm2pi.qmops} 

\input{qm2pi.sterngerlach} 

\input{qm2pi.metric} 

% section concurrent_process_calculi (end)

%\input{qm2pi.proofsketch}

% section proof sketch (end)

%\input{qm2pi.slviaknots} 

% section spatial logic via knots (end)

\input{qm2pi.conclusion}

% section conclusion (end)

%\input{qm2pi.dtcodes} 

% section wiring algorithm (end)

\input{qm2pi.ack} 

% section acknowledgments (end)

\newpage


\bibliographystyle{plain}   
\bibliography{../../biblios/main.bib}

\input{qm2pi.rhodetails}

\end{document}

 

%\documentclass[12pt]{llncs}
%\documentclass{jktr}

\usepackage[pdftex]{hyperref}                   
\usepackage {listings}
\usepackage {mathpartir}
\usepackage{bcprules}
%\usepackage{listings}
                       
\usepackage{graphicx} 
%\usepackage[margins=2.5cm,nohead,nofoot]{geometry}
%\usepackage{geometry}
\usepackage{amsfonts}
\usepackage{amstext}
\usepackage{latexsym}
\usepackage{amssymb}
\usepackage{color}


%\include{myPreamble}
\include{qm2pi.local} 

%\ifpdf
%\usepackage[pdftex]{graphicx}
%\else
%\usepackage{graphicx}
%\fi

 % \ifpdf
%  \usepackage{pdfsync}
%  \if


%\title{Brief Article}
%\author{David F. Snyder}
%\author{L.G. Meredith}

%\address{Dept. of Math., Texas State University--San Marcos, San Marcos, TX 78666}
       
\pagestyle{empty}


\begin{document}

\lstset{language=[Objective]Caml,frame=shadowbox}

\input{qm2pi.front}

% section front matter (end)

\input{qm2pi.intro} 
 
% section introduction (end)

% \input{qm2pi.knotations} 

% section notation (end)

\input{qm2pi.process.calculi} 

% section concurrent_process_calculi_and_spatial_logics_ (end)
    
%\input{qm2pi.knots2pi} 

%\input{qm2pi.trefoil} 

%\input{qm2pi.mainthm} 

% subsection basic_interpretation (end)

%\input{qm2pi.rho.presentation} 
\subsection{The syntax and semantics of the notation system}\label{sub:the_syntax_and_semantics_of_the_notation_system} % (fold)

We now summarize a technical presentation of the calculus that
embodies our theory of dynamics. The typical presentation of such a
calculus follows the style of giving generators and relations on
them. The grammar, below, describing term constructors, freely
generates the set of processes, $\Proc$. This set is then quotiented
by a relation known as structural congruence and it is over this set
that the notion of dynamics is expressed. This presentation is
essentially that of \cite{MeredithR05} with the addition of
polyadicity and summation. For readability we have relegated some of
the technical subtleties to an appendix.

\subsubsection{Process grammar}\label{subsub:process_grammar}

\begin{mathpar}
  \inferrule* [lab=synchronization] {} {{M} \bc \pzero \;|\; x?F \;|\; x!C }
  \and
  \inferrule* [lab=abstraction] {} {{F} \bc (x)P}
  \and
  \inferrule* [lab=concretion] {} {{C} \bc \langle Q \rangle}
  \and
  \inferrule* [lab=process] {} {{P,Q} \bc M \;| \;P|Q \;|\; @{x}}
  \and
  \inferrule* [lab=name] {} {{x} \bc \quotep{P}}
\end{mathpar} 

Note that $\vec{x}$ (resp. $\vec{P}$) denotes a vector of names
(resp. processes) of length $|\vec{x}|$ (resp. $|\vec{P}|$). We adopt
the following useful abbreviations.

\begin{mathpar}
   x?(\vec{y}).P := x.(\vec{y})P \and  x\clift{\vec{P}} := x.\clift{\vec{P}}
   \and x!(y) := \lift{x}{\dropn{y}}
   \and \Pi_{i=0}^{n-1}P_i := P_0 | \ldots | P_{n-1}
\end{mathpar}

\subsubsection{Structural congruence}

\paragraph{Free and bound names and alpha-equivalence.} At the
core of structural equivalence is alpha-equivalence which identifies
process that are the same up to a change of variable. Formally, we
recognize the distinction between free and bound names. The free names
of a process, $\freenames{P}$, may be calculated recursively as
follows:

\begin{mathpar}
\freenames{\pzero} := \emptyset
  \and \\
  \freenames{x?(y).P} := \{ x \} \cup (\freenames{P} \setminus \{ y \})
  \and 
  \freenames{x!\langle P \rangle} := \{ x \} \cup \{ P \} 
  \and \\
  \freenames{P|Q} := \freenames{P} \cup \freenames{Q}
  \and \\
  \freenames{@{x}} := \{ x \}
\end{mathpar}

$\pi$
$\quotep{\pi}$

$\freenames{-} : \pi \to \mathcal{P}(\quotep{\pi})$

\begin{eqnarray*}
  \freenames{\pzero} & := & \emptyset \\
  \freenames{x?(y).P} & := & \{ x \} \cup (\freenames{P} \setminus \{ y \}) \\
  \freenames{x!\langle P \rangle} & := & \{ x \} \cup \{ P \} \\
  \freenames{P|Q} & := & \freenames{P} \cup \freenames{Q} \\
  \freenames{\dropn{x}} & := & \{ x \}
\end{eqnarray*}

The bound names of a process, $\boundnames{P}$, are those names occurring in $P$
that are not free. For example, in $x?(y).0$, the name $x$ is free, while $y$ is bound.

\begin{mathpar}
  \inferrule* [lab=monoidal-laws] {} { P|Q \equiv Q|P \and P|0 \equiv P \and P|(Q|R) \equiv (P|Q)|R }
\end{mathpar}

\begin{mathpar}
  \inferrule* [lab=alpha-equivalence] {} { (x)P \equiv (y)P\{y/x\} \and y \not\in \freenames{P} }
\end{mathpar}

\begin{definition}
Then two processes, $P,Q$, are alpha-equivalent if $P = Q\{\vec{y}/\vec{x}\}$ for
some $\vec{x} \in \boundnames{Q},\vec{y} \in \boundnames{P}$, where $Q\{\vec{y}/\vec{x}\}$
denotes the capture-avoiding substitution of $\vec{y}$ for $\vec{x}$ in $Q$.
\end{definition}

\begin{definition}
  The {\em structural congruence} \cite{SangiorgiWalker} , $\equiv$,
  between processes is the least congruence containing
  alpha-equivalence, satisfying the abelian monoid laws
  (associativity, commutativity and $\pzero$ as identity) for parallel
  composition $|$ and for summation $+$.
\end{definition}

\subsection{Name equivalence}

We take name equivalence, written $\nameeq$, to be the smallest
equivalence relation generated by the following rules.

\begin{mathpar}
\inferrule*[lab=Quote-drop]
{ }
{ \quotep{@{x}} \nameeq x }

\inferrule*[lab=Struct-equiv]
{ P \scong Q }
{ \quotep{P} \nameeq \quotep{Q} }
\end{mathpar}

The astute reader will have noticed that the mutual recursion of names
and processes imposes a mutual recursion on alpha-equivalence and
structural equivalence via name-equivalence. Fortunately, all of this
works out pleasantly and we may calculate in the natural way, free of
concern. The reader interested in the details is referred to the
appendix \ref{appendix:rho_details}.

\subsection{Substitution}

We use $\Proc$ for the set of processes, $\QProc$ for the set of
names, and $\id{\{}\vec{y} / \vec{x} \id{\}}$ to denote partial maps,
$s : \QProc \rightarrow \QProc$. A map, $s$ lifts, uniquely, to a map
on process terms, $\widehat{s} : \Proc \rightarrow \Proc$ by the
following equations.

\begin{mathpar}
  (0) \psubstp{Q}{P} := 0 \\
  (R \juxtap S) \psubstp{Q}{P}
  :=    
  (R)\psubstp{Q}{P} \juxtap (S) \psubstp{Q}{P} \\
  (x?(y).R) \psubstp{Q}{P}    
  :=    
  (x)\substp{Q}{P} (z)\concat( (R \psubstn{z}{y}) \psubstp{Q}{P} ) \\
  (\lift{x}{R}) \psubstp{Q}{P}  
  :=
  \lift{(x)\substp{Q}{P}}{ R \psubstp{Q}{P} } \\
%   (\dropn{x})  \psubstp{Q}{P}       
%   := 
%   \left\{ 
%     \begin{array}{ccc} 
%       \dropn{\quotep{Q}} & & x \nameeq \quotep{P} \\
%       \dropn{x} & & otherwise \\
%     \end{array}
%   \right. 
  (\dropn{x})  \psubstp{Q}{P}       
  := 
  \left\{ 
    \begin{array}{ccc} 
      Q & & x \nameeq \quotep{P} \\
      \dropn{x} & & otherwise \\
    \end{array}
  \right.
\end{mathpar}
 

where

\begin{eqnarray}
  (x)\id{\{} \lpquote Q \rpquote / \lpquote P \rpquote \id{\}}            = 
  \left\{ 
    \begin{array}{ccc}
      \lpquote Q \rpquote & & x \nameeq \lpquote P \rpquote \\
      x & & otherwise \\
    \end{array}
  \right. \nonumber
\end{eqnarray}

and $z$ is chosen distinct from $\quotep{P}$, $\quotep{Q}$, the free
names in $Q$, and all the names in $R$. Our $\alpha$-equivalence will
be built in the standard way from this substitution.

\begin{remark}\label{rem:no_self_referential_names}
  One consequence of these definitions is that $\forall P. \quotep{P}
  \not\in \freenames{P}$.
\end{remark}

\subsection{ Dynamic quote: an example }

Anticipating something of what's to come, consider applying the
substitution, $\widehat{\id{\{}u / z \id{\}}}$, to the following pair
of processes, $\lift{w}{y!(z)}$ and $w[ \lpquote y!(z) \rpquote ]$.

\begin{eqnarray}
	\lift{w}{y!(z)}\widehat{\id{\{}u / z \id{\}}}
		& = &
		\lift{w}{y!(u)} \nonumber\\
	w[ \lpquote y!(z) \rpquote ] \widehat{ \id{\{}u / z \id{\}} }
		& = &
		w[ \lpquote y!(z) \rpquote ] \nonumber
\end{eqnarray}

Because the body of the process between quotes is impervious to
substitution, we get radically different answers. In fact, by
examining the first process in an input context,
e.g. $x?(z).\lift{w}{y!(z)}$, we see that the process under the lift
operator may be shaped by prefixed inputs binding a name inside it. In
this sense, the lift operator will be seen as a way to dynamically
construct processes before reifying them as names.

Finally equipped with these standard features we can present the
dynamics of the calculus.

\subsubsection{Operational semantics} 

Finally, we introduce the computational dynamics. What marks these
algebras as distinct from other more traditionally studied algebraic
structures, e.g. vector spaces or polynomial rings, is the manner in
which dynamics is captured. In traditional structures, dynamics is typically
expressed through morphisms between such structures, as in linear maps
between vector spaces or morphisms between rings. In algebras
associated with the semantics of computation, the dynamics is
expressed as part of the algebraic structure itself, through a
reduction reduction relation typically denoted by $\red$. Below, we
give a recursive presentation of this relation for the calculus used
in the encoding.

$\red \subseteq \pi \times \pi$
$\red : \pi \to \mathcal{P}(\pi)$

\begin{mathpar}
  \inferrule* [lab=Comm] { \textsf{match}( x_{src}, x_{trgt} ) } { x_{trgt}?(y)P \; | \; x_{src}!\langle {Q} \rangle \red P\{\quotep{Q}/y}\} }
  \and \\
  \inferrule* [lab=Par] {{P} \red {P}'} {{{P} | {Q}} \red {{P}' | {Q}}}
  \and
  \inferrule* [lab=Equiv]{{{P} \scong {P}'} \andalso {{P}' \red {Q}'} \andalso {{Q}' \scong {Q}}}{{P} \red {Q}}
\end{mathpar}

\begin{eqnarray*}
  match_{\equiv} (\quotep{P},\quotep{Q}) & := & P \equiv Q \\
  match_{\dagger}(\quotep{P},\quotep{Q}) & := & \forall R. P|Q \red^{*} R => R \red^{*} 0 \\
  match_{K}(\quotep{P},\quotep{Q}) & := & K \mbox{ for some context } K
\end{eqnarray*}

$u?(x)P | u!\langle Q \rangle \red P\{\quotep{Q}/x\}$

%We write $\wred$ for $\red^*$, and $P\red$ if $\exists Q $ such that $ P \red Q$.
We write $P\red$ if $\exists Q $ such that $ P \red Q$ and $P\not\red$, otherwise.

\section{Replication}

As mentioned before, it is known that replication (and hence
recursion) can be implemented in a higher-order process algebra
\cite{SangiorgiWalker}. As our first example of calculation with the
machinery thus far presented we give the construction explicitly in
the {\rhoc}.

\begin{eqnarray}
	D_{x} & := & \prefix{x}{y}{(\binpar{\outputp{x}{y}}{@{y}})} \nonumber\\
	\bangp_{x}{P} & := & \binpar{{x}!\langle{\binpar{D_{x}}{P}}\rangle}{D_{x}} \nonumber
\end{eqnarray}

\begin{eqnarray}
	\bangp_{x}{P} & & \nonumber\\
	=
	& {x}!\langle{(\prefix{x}{y}{(\outputp{x}{y} | @{y})) | P}}\rangle 
	      | \prefix{x}{y}{(\outputp{x}{y} | @{y})} & \nonumber\\
	\red
	& (\outputp{x}{y} | @{y})\substn{\quotep{(\prefix{x}{y}{(@{y} | \outputp{x}{y})) | P}}}{y} & \nonumber\\
	=
	& \outputp{x}{\quotep{(\prefix{x}{y}{(\outputp{x}{y} | @{y})) | P}}}
	  | {(\prefix{x}{y}{(\outputp{x}{y} | @{y})) | P}} & \nonumber\\
	\red
	& \ldots & \nonumber\\
	\red^*
	& P | P | \ldots & \nonumber
\end{eqnarray}

Of course, this encoding, as an implementation, runs away, unfolding
$\bangp{P}$ eagerly. A lazier and more implementable replication
operator, restricted to input-guarded processes, may be obtained as follows.

\begin{eqnarray}
\bangp{\prefix{u}{v}{P}} 
	:= 
	\binpar{\lift{x}{\prefix{u}{v}{(\binpar{D(x)}{P})}}}{D(x)} \nonumber
\end{eqnarray}

\begin{remark}
  Note that the lazier definition still does not deal with summation
  or mixed summation (i.e. sums over input and output). The reader is
  invited to construct definitions of replication that deal with these
  features. 

  Further, the definitions are parameterized in a name, $x$. Can you,
  gentle reader, make a definition that eliminates this parameter and
  guarantees no accidental interaction between the replication
  machinery and the process being replicated -- i.e. no accidental
  sharing of names used by the process to get its work done and the
  name(s) used by the replication to effect copying. This latter
  revision of the definition of replication is crucial to obtaining
  the expected identity $!!P \sim !P$.
\end{remark}

\begin{remark}\label{rem:paradoxical_combinator}
  The reader familiar with the lambda calculus will have noticed the
  similarity between $D$ and the paradoxical combinator.

  [Ed. note: the existence of this seems to suggest we have to be more
  restrictive on the set of processes and names we admit if we are to
  support no-cloning.]
\end{remark}

\subsubsection{Bisimulation}

The computational dynamics gives rise to another kind of equivalence,
the equivalence of computational behavior. As previously mentioned
this is typically captured \emph{via} some form of bisimulation.

% The notion we use in this paper is weak barbed bisimulation
% \cite{milner91polyadicpi}.

The notion we use in this paper is derived from weak barbed
bisimulation \cite{milner91polyadicpi}. 

\begin{definition}
An \emph{observation relation}, $\downarrow_{\mathcal N}$, over a set
of names, $\mathcal N$, is the smallest relation satisfying the rules
below.

\infrule[Out-barb]{y \in {\mathcal N}, \; x \nameeq y}
		  {\outputp{x}{v} \downarrow_{\mathcal N} x}
\infrule[Par-barb]{\mbox{$P\downarrow_{\mathcal N} x$ or $Q\downarrow_{\mathcal N} x$}}
		  {\binpar{P}{Q} \downarrow_{\mathcal N} x}

We write $P \Downarrow_{\mathcal N} x$ if there is $Q$ such that 
$P \wred Q$ and $Q \downarrow_{\mathcal N} x$.
\end{definition}

\begin{definition}
%\label{def.bbisim}
An  ${\mathcal N}$-\emph{barbed bisimulation} over a set of names, ${\mathcal N}$, is a symmetric binary relation 
${\mathcal S}_{\mathcal N}$ between agents such that $P\rel{S}_{\mathcal N}Q$ implies:
\begin{enumerate}
\item If $P \red P'$ then $Q \wred Q'$ and $P'\rel{S}_{\mathcal N} Q'$.
\item If $P\downarrow_{\mathcal N} x$, then $Q\Downarrow_{\mathcal N} x$.
\end{enumerate}
$P$ is ${\mathcal N}$-barbed bisimilar to $Q$, written
$P \wbbisim_{\mathcal N} Q$, if $P \rel{S}_{\mathcal N} Q$ for some ${\mathcal N}$-barbed bisimulation ${\mathcal S}_{\mathcal N}$.
\end{definition}

$\mathcal{R} \subseteq \pi \times \pi$

$P \mathcal{R} Q => \forall P'. P \red P' \Rightarrow \exists Q'. Q \red Q', P' \mathcal{R} Q'$

$P \vdash x \Rightarrow Q \vdash x$

\begin{mathpar}
  \inferrule*[lab=Out-barb]{x \nameeq y}{{y}!\langle{Q}\rangle \vdash x}
  \and
  \inferrule*[lab=Par-barb]{\mbox{$P\vdash x$ or $Q\vdash x$}}{\binpar{P}{Q} \vdash x}
\end{mathpar}

\subsubsection{Contexts}

One of the principle advantages of computational calculi like the
$\pi$-calculus is a well-defined notion of context,
contextual-equivalence and a correlation between
contextual-equivalence and notions of bisimulation. The notion of
context allows the decomposition of a process into (sub-)process and
its syntactic environment, its context. Thus, a context may be
thought of as a process with a ``hole'' (written $\Box$) in it. The
application of a context $M$ to a process $P$, written $M[P]$, is
tantamount to filling the hole in $M$ with $P$. In this paper we do
not need the full weight of this theory, but do make use of the notion
of context in the proof the main theorem. 

\begin{mathpar}
  \inferrule* [lab=summation] {} {{M_{M},M_{N}} \bc \Box \;|\; x.M_{A} \;|\; M_{M}+M_{N}}
  \and
  \inferrule* [lab=agent] {} {{M_{A}} \bc (\vec{x})M_{P} \;| \; \clift{P_0,\ldots,M_{P},\ldots,P_N}}
  \and \\
  \inferrule* [lab=process] {} {{M_{P}} \bc M_{N} \;| \;P|M_{P} }
\end{mathpar} 

\begin{mathpar}
  \inferrule* [lab=sychronization] {} {M_{N} \bc \Box \;|\; x?M_{F} \;|\; x!M_{C}}
  \and
  \inferrule* [lab=abstraction] {} {{M_{F}} \bc (x)M_{P} }
  \and
  \inferrule* [lab=concretion] {} {{M_{C}} \bc \langle M_{P} \rangle }
  \and \\
  \inferrule* [lab=process] {} {{M_{P}} \bc M_{N} \;| \;P|M_{P} }
\end{mathpar}

\begin{definition}[contextual application] Given a context $M$, and
  process $P$, we define the \emph{contextual application}, $M[P] :=
  M\{P/\Box\}$. That is, the contextual application of M to P is the
  substitution of $P$ for $\Box$ in $M$.
\end{definition}

$\meaningof{-} : L \to \mathcal{P}(\pi)$

\begin{mathpar}
  \inferrule* [lab=collection] {} {\meaningof{true} = \pi, \and \meaningof{~E} = \pi \setminus \meaningof{E}, \and \meaningof{E_{1} \& E_{2}} = \meaningof{E_{1}} \cap \meaningof{E_{2}}}
\end{mathpar}

\begin{mathpar}
  \inferrule* [lab=structure] {} {\meaningof{0} = \{ P \in \pi | P \equiv 0 \}, \and \\ \meaningof{E_1 | E_2} = \{ P \in \pi | P \equiv P_{1} | P_{2}, P_{1} \in \meaningof{E_{1}}, P_{2} \in \meaningof{E_2}\} }
\end{mathpar}

\begin{mathpar}
 \inferrule* [lab=behavior] {} {\meaningof{\langle a?b \rangle E} = \{ P \in \pi | P \equiv Q | u?(y)P', \\ \and \\\\ \and \\ \;\;\; u \in \meaningof{a}, \forall z.P'\{z/y\} \in \meaningof{E\{z/b\}}\}, \and \\ \meaningof{a!E} = \{ P \in \pi | P \equiv Q | x!\langle P' \rangle, x \in \meaningof{a} P' \in \meaningof{E}\} }
\end{mathpar}

\begin{mathpar}
 \inferrule* [lab=nominal] {} {\meaningof{\quotep{E}} = \{ \quotep{P} \in \quotep{\pi} | P \in \meaningof{E} \}, \and \meaningof{\quotep{P}} = \{ \quotep{Q} \in \quotep{\pi} | P \equiv Q \} \and \\ \meaningof{@\quotep{E}} = \{ P \in \pi | P \equiv @x, x \in \meaningof{E} \}}
\end{mathpar}

\begin{eqnarray*}
  \\
  \meaningof{-} : TS \to ST
\end{eqnarray*}

\begin{eqnarray*}
  \\
  L : TS \to ST
\end{eqnarray*}

\begin{eqnarray*}
  \\
  P \models E \iff P \in \meaningof{E}
\end{eqnarray*}

\begin{eqnarray*}
  P \approx_{L} Q \iff \forall E \in L. P \models E \iff Q \models E
\end{eqnarray*}

\begin{eqnarray*}
  P \approx_{K} Q
\end{eqnarray*}

\begin{eqnarray*}
  P \approx Q
\end{eqnarray*}

$\approx_{K} = \approx = \approx_{L}$

\subsubsection{Contextual duality}

Note that contexts extend the quotation operation to a family of
operations from processes to names. Given a context, $M$, we can
define a \emph{nominal context}, $\quotep{M}$ by $\quotep{M}[P] :=
\quotep{M[P]}$. To foreshadow what is to come we observe that these
operations enjoy a duality with processes very much like the duality
between vectors and maps from vectors to scalars.

Further, because the calculus is essentially higher-order, we have a
correspondence between contexts and processes. More specifically,
given a name $x$ and a context $M$ we can construct $M^{*}_{x}$ such
that 

\begin{mathpar}
  M^{*}_{x} | \lift{x}{P} \red M[P]
\end{mathpar}

namely,

\begin{mathpar}
  M^{*}_{x} := x?(u).M[\dropn{u}]
\end{mathpar}

The dependence of $M^{*}_{x}$ on a name makes it an abstraction, 

\begin{mathpar}
  M^{*} := (x)x?(u).M[\dropn{u}]
\end{mathpar}

\subsection{Additional notation}

It will sometimes be convenient to denote the process a name
quotes. We already have the notation $x = \quotep{P}$, but it will be
convenient to introduce an alternate notation, $\procn{x}$, when we
want to emphasize the connection to the use of the name. Note that, by
virtue of name equivalence, $\quotep{\procn{x}} \nameeq x$; so, the
notation is consistent with previous definitions.

Further, because names have structure it is possible to effect
substitutions on the basis of that structure. This means we need to
upgrade our notation for substitutions, which we accomplish by
adapting comprehension notation. Thus,

\begin{mathpar}
  P\{ y / x : x \in S \}
\end{mathpar}

is interpreted to mean the process derived from P by replacing (in a
capture-avoiding manner) each occurrence of $x$ in $S$ by $y$. For example,

\begin{mathpar}
  P\{ \quotep{\procn{x}|\procn{x}} / x : x \in \freenames{P} \}
\end{mathpar}

will replace each (occurrence) of a free name $x$ in $P$ by
$\quotep{\procn{x}|\procn{x}}$.

Also, we will avail ourselves of the notation $x^{L}$ and $x^{R}$ to
denote injections of a name into disjoint copies of the name
space. There are numerous ways to accomplish this. One example can be
found in \cite{MeredithR05}. This notation overloads to vectors of
names: $\vec{x}^{\pi} := (x_{i}^{\pi} \; : \; 0 \leq i < |\vec{x}| )$ where $\pi \in \{L,R\}$.

We also use $P^{\Box} := P|\Box$.

In \cite{MeredithR05} an interpretation of the new operator is
given. It turns out that there are several possible interpretations
all enjoying the requisite algebraic properties of the operator (see
\cite{milner91polyadicpi}). We will therefore make liberal use of
$(\nu\; \vec{x})P$.

% subsection the_syntax_and_semantics_of_the_notation_system (end)   

\input{qm2pi.qmops} 

\input{qm2pi.sterngerlach} 

\input{qm2pi.metric} 

% section concurrent_process_calculi (end)

%\input{qm2pi.proofsketch}

% section proof sketch (end)

%\input{qm2pi.slviaknots} 

% section spatial logic via knots (end)

\input{qm2pi.conclusion}

% section conclusion (end)

%\input{qm2pi.dtcodes} 

% section wiring algorithm (end)

\input{qm2pi.ack} 

% section acknowledgments (end)

\newpage


\bibliographystyle{plain}   
\bibliography{../../biblios/main.bib}

\input{qm2pi.rhodetails}

\end{document}

 

% subsection basic_interpretation (end)

%\input{qm2pi.rho.presentation} 
\subsection{The syntax and semantics of the notation system}\label{sub:the_syntax_and_semantics_of_the_notation_system} % (fold)

We now summarize a technical presentation of the calculus that
embodies our theory of dynamics. The typical presentation of such a
calculus follows the style of giving generators and relations on
them. The grammar, below, describing term constructors, freely
generates the set of processes, $\Proc$. This set is then quotiented
by a relation known as structural congruence and it is over this set
that the notion of dynamics is expressed. This presentation is
essentially that of \cite{MeredithR05} with the addition of
polyadicity and summation. For readability we have relegated some of
the technical subtleties to an appendix.

\subsubsection{Process grammar}\label{subsub:process_grammar}

\begin{mathpar}
  \inferrule* [lab=synchronization] {} {{M} \bc \pzero \;|\; x?F \;|\; x!C }
  \and
  \inferrule* [lab=abstraction] {} {{F} \bc (x)P}
  \and
  \inferrule* [lab=concretion] {} {{C} \bc \langle Q \rangle}
  \and
  \inferrule* [lab=process] {} {{P,Q} \bc M \;| \;P|Q \;|\; @{x}}
  \and
  \inferrule* [lab=name] {} {{x} \bc \quotep{P}}
\end{mathpar} 

Note that $\vec{x}$ (resp. $\vec{P}$) denotes a vector of names
(resp. processes) of length $|\vec{x}|$ (resp. $|\vec{P}|$). We adopt
the following useful abbreviations.

\begin{mathpar}
   x?(\vec{y}).P := x.(\vec{y})P \and  x\clift{\vec{P}} := x.\clift{\vec{P}}
   \and x!(y) := \lift{x}{\dropn{y}}
   \and \Pi_{i=0}^{n-1}P_i := P_0 | \ldots | P_{n-1}
\end{mathpar}

\subsubsection{Structural congruence}

\paragraph{Free and bound names and alpha-equivalence.} At the
core of structural equivalence is alpha-equivalence which identifies
process that are the same up to a change of variable. Formally, we
recognize the distinction between free and bound names. The free names
of a process, $\freenames{P}$, may be calculated recursively as
follows:

\begin{mathpar}
\freenames{\pzero} := \emptyset
  \and \\
  \freenames{x?(y).P} := \{ x \} \cup (\freenames{P} \setminus \{ y \})
  \and 
  \freenames{x!\langle P \rangle} := \{ x \} \cup \{ P \} 
  \and \\
  \freenames{P|Q} := \freenames{P} \cup \freenames{Q}
  \and \\
  \freenames{@{x}} := \{ x \}
\end{mathpar}

$\pi$
$\quotep{\pi}$

$\freenames{-} : \pi \to \mathcal{P}(\quotep{\pi})$

\begin{eqnarray*}
  \freenames{\pzero} & := & \emptyset \\
  \freenames{x?(y).P} & := & \{ x \} \cup (\freenames{P} \setminus \{ y \}) \\
  \freenames{x!\langle P \rangle} & := & \{ x \} \cup \{ P \} \\
  \freenames{P|Q} & := & \freenames{P} \cup \freenames{Q} \\
  \freenames{\dropn{x}} & := & \{ x \}
\end{eqnarray*}

The bound names of a process, $\boundnames{P}$, are those names occurring in $P$
that are not free. For example, in $x?(y).0$, the name $x$ is free, while $y$ is bound.

\begin{mathpar}
  \inferrule* [lab=monoidal-laws] {} { P|Q \equiv Q|P \and P|0 \equiv P \and P|(Q|R) \equiv (P|Q)|R }
\end{mathpar}

\begin{mathpar}
  \inferrule* [lab=alpha-equivalence] {} { (x)P \equiv (y)P\{y/x\} \and y \not\in \freenames{P} }
\end{mathpar}

\begin{definition}
Then two processes, $P,Q$, are alpha-equivalent if $P = Q\{\vec{y}/\vec{x}\}$ for
some $\vec{x} \in \boundnames{Q},\vec{y} \in \boundnames{P}$, where $Q\{\vec{y}/\vec{x}\}$
denotes the capture-avoiding substitution of $\vec{y}$ for $\vec{x}$ in $Q$.
\end{definition}

\begin{definition}
  The {\em structural congruence} \cite{SangiorgiWalker} , $\equiv$,
  between processes is the least congruence containing
  alpha-equivalence, satisfying the abelian monoid laws
  (associativity, commutativity and $\pzero$ as identity) for parallel
  composition $|$ and for summation $+$.
\end{definition}

\subsection{Name equivalence}

We take name equivalence, written $\nameeq$, to be the smallest
equivalence relation generated by the following rules.

\begin{mathpar}
\inferrule*[lab=Quote-drop]
{ }
{ \quotep{@{x}} \nameeq x }

\inferrule*[lab=Struct-equiv]
{ P \scong Q }
{ \quotep{P} \nameeq \quotep{Q} }
\end{mathpar}

The astute reader will have noticed that the mutual recursion of names
and processes imposes a mutual recursion on alpha-equivalence and
structural equivalence via name-equivalence. Fortunately, all of this
works out pleasantly and we may calculate in the natural way, free of
concern. The reader interested in the details is referred to the
appendix \ref{appendix:rho_details}.

\subsection{Substitution}

We use $\Proc$ for the set of processes, $\QProc$ for the set of
names, and $\id{\{}\vec{y} / \vec{x} \id{\}}$ to denote partial maps,
$s : \QProc \rightarrow \QProc$. A map, $s$ lifts, uniquely, to a map
on process terms, $\widehat{s} : \Proc \rightarrow \Proc$ by the
following equations.

\begin{mathpar}
  (0) \psubstp{Q}{P} := 0 \\
  (R \juxtap S) \psubstp{Q}{P}
  :=    
  (R)\psubstp{Q}{P} \juxtap (S) \psubstp{Q}{P} \\
  (x?(y).R) \psubstp{Q}{P}    
  :=    
  (x)\substp{Q}{P} (z)\concat( (R \psubstn{z}{y}) \psubstp{Q}{P} ) \\
  (\lift{x}{R}) \psubstp{Q}{P}  
  :=
  \lift{(x)\substp{Q}{P}}{ R \psubstp{Q}{P} } \\
%   (\dropn{x})  \psubstp{Q}{P}       
%   := 
%   \left\{ 
%     \begin{array}{ccc} 
%       \dropn{\quotep{Q}} & & x \nameeq \quotep{P} \\
%       \dropn{x} & & otherwise \\
%     \end{array}
%   \right. 
  (\dropn{x})  \psubstp{Q}{P}       
  := 
  \left\{ 
    \begin{array}{ccc} 
      Q & & x \nameeq \quotep{P} \\
      \dropn{x} & & otherwise \\
    \end{array}
  \right.
\end{mathpar}
 

where

\begin{eqnarray}
  (x)\id{\{} \lpquote Q \rpquote / \lpquote P \rpquote \id{\}}            = 
  \left\{ 
    \begin{array}{ccc}
      \lpquote Q \rpquote & & x \nameeq \lpquote P \rpquote \\
      x & & otherwise \\
    \end{array}
  \right. \nonumber
\end{eqnarray}

and $z$ is chosen distinct from $\quotep{P}$, $\quotep{Q}$, the free
names in $Q$, and all the names in $R$. Our $\alpha$-equivalence will
be built in the standard way from this substitution.

\begin{remark}\label{rem:no_self_referential_names}
  One consequence of these definitions is that $\forall P. \quotep{P}
  \not\in \freenames{P}$.
\end{remark}

\subsection{ Dynamic quote: an example }

Anticipating something of what's to come, consider applying the
substitution, $\widehat{\id{\{}u / z \id{\}}}$, to the following pair
of processes, $\lift{w}{y!(z)}$ and $w[ \lpquote y!(z) \rpquote ]$.

\begin{eqnarray}
	\lift{w}{y!(z)}\widehat{\id{\{}u / z \id{\}}}
		& = &
		\lift{w}{y!(u)} \nonumber\\
	w[ \lpquote y!(z) \rpquote ] \widehat{ \id{\{}u / z \id{\}} }
		& = &
		w[ \lpquote y!(z) \rpquote ] \nonumber
\end{eqnarray}

Because the body of the process between quotes is impervious to
substitution, we get radically different answers. In fact, by
examining the first process in an input context,
e.g. $x?(z).\lift{w}{y!(z)}$, we see that the process under the lift
operator may be shaped by prefixed inputs binding a name inside it. In
this sense, the lift operator will be seen as a way to dynamically
construct processes before reifying them as names.

Finally equipped with these standard features we can present the
dynamics of the calculus.

\subsubsection{Operational semantics} 

Finally, we introduce the computational dynamics. What marks these
algebras as distinct from other more traditionally studied algebraic
structures, e.g. vector spaces or polynomial rings, is the manner in
which dynamics is captured. In traditional structures, dynamics is typically
expressed through morphisms between such structures, as in linear maps
between vector spaces or morphisms between rings. In algebras
associated with the semantics of computation, the dynamics is
expressed as part of the algebraic structure itself, through a
reduction reduction relation typically denoted by $\red$. Below, we
give a recursive presentation of this relation for the calculus used
in the encoding.

$\red \subseteq \pi \times \pi$
$\red : \pi \to \mathcal{P}(\pi)$

\begin{mathpar}
  \inferrule* [lab=Comm] { \textsf{match}( x_{src}, x_{trgt} ) } { x_{trgt}?(y)P \; | \; x_{src}!\langle {Q} \rangle \red P\{\quotep{Q}/y}\} }
  \and \\
  \inferrule* [lab=Par] {{P} \red {P}'} {{{P} | {Q}} \red {{P}' | {Q}}}
  \and
  \inferrule* [lab=Equiv]{{{P} \scong {P}'} \andalso {{P}' \red {Q}'} \andalso {{Q}' \scong {Q}}}{{P} \red {Q}}
\end{mathpar}

\begin{eqnarray*}
  match_{\equiv} (\quotep{P},\quotep{Q}) & := & P \equiv Q \\
  match_{\dagger}(\quotep{P},\quotep{Q}) & := & \forall R. P|Q \red^{*} R => R \red^{*} 0 \\
  match_{K}(\quotep{P},\quotep{Q}) & := & K \mbox{ for some context } K
\end{eqnarray*}

$u?(x)P | u!\langle Q \rangle \red P\{\quotep{Q}/x\}$

%We write $\wred$ for $\red^*$, and $P\red$ if $\exists Q $ such that $ P \red Q$.
We write $P\red$ if $\exists Q $ such that $ P \red Q$ and $P\not\red$, otherwise.

\section{Replication}

As mentioned before, it is known that replication (and hence
recursion) can be implemented in a higher-order process algebra
\cite{SangiorgiWalker}. As our first example of calculation with the
machinery thus far presented we give the construction explicitly in
the {\rhoc}.

\begin{eqnarray}
	D_{x} & := & \prefix{x}{y}{(\binpar{\outputp{x}{y}}{@{y}})} \nonumber\\
	\bangp_{x}{P} & := & \binpar{{x}!\langle{\binpar{D_{x}}{P}}\rangle}{D_{x}} \nonumber
\end{eqnarray}

\begin{eqnarray}
	\bangp_{x}{P} & & \nonumber\\
	=
	& {x}!\langle{(\prefix{x}{y}{(\outputp{x}{y} | @{y})) | P}}\rangle 
	      | \prefix{x}{y}{(\outputp{x}{y} | @{y})} & \nonumber\\
	\red
	& (\outputp{x}{y} | @{y})\substn{\quotep{(\prefix{x}{y}{(@{y} | \outputp{x}{y})) | P}}}{y} & \nonumber\\
	=
	& \outputp{x}{\quotep{(\prefix{x}{y}{(\outputp{x}{y} | @{y})) | P}}}
	  | {(\prefix{x}{y}{(\outputp{x}{y} | @{y})) | P}} & \nonumber\\
	\red
	& \ldots & \nonumber\\
	\red^*
	& P | P | \ldots & \nonumber
\end{eqnarray}

Of course, this encoding, as an implementation, runs away, unfolding
$\bangp{P}$ eagerly. A lazier and more implementable replication
operator, restricted to input-guarded processes, may be obtained as follows.

\begin{eqnarray}
\bangp{\prefix{u}{v}{P}} 
	:= 
	\binpar{\lift{x}{\prefix{u}{v}{(\binpar{D(x)}{P})}}}{D(x)} \nonumber
\end{eqnarray}

\begin{remark}
  Note that the lazier definition still does not deal with summation
  or mixed summation (i.e. sums over input and output). The reader is
  invited to construct definitions of replication that deal with these
  features. 

  Further, the definitions are parameterized in a name, $x$. Can you,
  gentle reader, make a definition that eliminates this parameter and
  guarantees no accidental interaction between the replication
  machinery and the process being replicated -- i.e. no accidental
  sharing of names used by the process to get its work done and the
  name(s) used by the replication to effect copying. This latter
  revision of the definition of replication is crucial to obtaining
  the expected identity $!!P \sim !P$.
\end{remark}

\begin{remark}\label{rem:paradoxical_combinator}
  The reader familiar with the lambda calculus will have noticed the
  similarity between $D$ and the paradoxical combinator.

  [Ed. note: the existence of this seems to suggest we have to be more
  restrictive on the set of processes and names we admit if we are to
  support no-cloning.]
\end{remark}

\subsubsection{Bisimulation}

The computational dynamics gives rise to another kind of equivalence,
the equivalence of computational behavior. As previously mentioned
this is typically captured \emph{via} some form of bisimulation.

% The notion we use in this paper is weak barbed bisimulation
% \cite{milner91polyadicpi}.

The notion we use in this paper is derived from weak barbed
bisimulation \cite{milner91polyadicpi}. 

\begin{definition}
An \emph{observation relation}, $\downarrow_{\mathcal N}$, over a set
of names, $\mathcal N$, is the smallest relation satisfying the rules
below.

\infrule[Out-barb]{y \in {\mathcal N}, \; x \nameeq y}
		  {\outputp{x}{v} \downarrow_{\mathcal N} x}
\infrule[Par-barb]{\mbox{$P\downarrow_{\mathcal N} x$ or $Q\downarrow_{\mathcal N} x$}}
		  {\binpar{P}{Q} \downarrow_{\mathcal N} x}

We write $P \Downarrow_{\mathcal N} x$ if there is $Q$ such that 
$P \wred Q$ and $Q \downarrow_{\mathcal N} x$.
\end{definition}

\begin{definition}
%\label{def.bbisim}
An  ${\mathcal N}$-\emph{barbed bisimulation} over a set of names, ${\mathcal N}$, is a symmetric binary relation 
${\mathcal S}_{\mathcal N}$ between agents such that $P\rel{S}_{\mathcal N}Q$ implies:
\begin{enumerate}
\item If $P \red P'$ then $Q \wred Q'$ and $P'\rel{S}_{\mathcal N} Q'$.
\item If $P\downarrow_{\mathcal N} x$, then $Q\Downarrow_{\mathcal N} x$.
\end{enumerate}
$P$ is ${\mathcal N}$-barbed bisimilar to $Q$, written
$P \wbbisim_{\mathcal N} Q$, if $P \rel{S}_{\mathcal N} Q$ for some ${\mathcal N}$-barbed bisimulation ${\mathcal S}_{\mathcal N}$.
\end{definition}

$\mathcal{R} \subseteq \pi \times \pi$

$P \mathcal{R} Q => \forall P'. P \red P' \Rightarrow \exists Q'. Q \red Q', P' \mathcal{R} Q'$

$P \vdash x \Rightarrow Q \vdash x$

\begin{mathpar}
  \inferrule*[lab=Out-barb]{x \nameeq y}{{y}!\langle{Q}\rangle \vdash x}
  \and
  \inferrule*[lab=Par-barb]{\mbox{$P\vdash x$ or $Q\vdash x$}}{\binpar{P}{Q} \vdash x}
\end{mathpar}

\subsubsection{Contexts}

One of the principle advantages of computational calculi like the
$\pi$-calculus is a well-defined notion of context,
contextual-equivalence and a correlation between
contextual-equivalence and notions of bisimulation. The notion of
context allows the decomposition of a process into (sub-)process and
its syntactic environment, its context. Thus, a context may be
thought of as a process with a ``hole'' (written $\Box$) in it. The
application of a context $M$ to a process $P$, written $M[P]$, is
tantamount to filling the hole in $M$ with $P$. In this paper we do
not need the full weight of this theory, but do make use of the notion
of context in the proof the main theorem. 

\begin{mathpar}
  \inferrule* [lab=summation] {} {{M_{M},M_{N}} \bc \Box \;|\; x.M_{A} \;|\; M_{M}+M_{N}}
  \and
  \inferrule* [lab=agent] {} {{M_{A}} \bc (\vec{x})M_{P} \;| \; \clift{P_0,\ldots,M_{P},\ldots,P_N}}
  \and \\
  \inferrule* [lab=process] {} {{M_{P}} \bc M_{N} \;| \;P|M_{P} }
\end{mathpar} 

\begin{mathpar}
  \inferrule* [lab=sychronization] {} {M_{N} \bc \Box \;|\; x?M_{F} \;|\; x!M_{C}}
  \and
  \inferrule* [lab=abstraction] {} {{M_{F}} \bc (x)M_{P} }
  \and
  \inferrule* [lab=concretion] {} {{M_{C}} \bc \langle M_{P} \rangle }
  \and \\
  \inferrule* [lab=process] {} {{M_{P}} \bc M_{N} \;| \;P|M_{P} }
\end{mathpar}

\begin{definition}[contextual application] Given a context $M$, and
  process $P$, we define the \emph{contextual application}, $M[P] :=
  M\{P/\Box\}$. That is, the contextual application of M to P is the
  substitution of $P$ for $\Box$ in $M$.
\end{definition}

$\meaningof{-} : L \to \mathcal{P}(\pi)$

\begin{mathpar}
  \inferrule* [lab=collection] {} {\meaningof{true} = \pi, \and \meaningof{~E} = \pi \setminus \meaningof{E}, \and \meaningof{E_{1} \& E_{2}} = \meaningof{E_{1}} \cap \meaningof{E_{2}}}
\end{mathpar}

\begin{mathpar}
  \inferrule* [lab=structure] {} {\meaningof{0} = \{ P \in \pi | P \equiv 0 \}, \and \\ \meaningof{E_1 | E_2} = \{ P \in \pi | P \equiv P_{1} | P_{2}, P_{1} \in \meaningof{E_{1}}, P_{2} \in \meaningof{E_2}\} }
\end{mathpar}

\begin{mathpar}
 \inferrule* [lab=behavior] {} {\meaningof{\langle a?b \rangle E} = \{ P \in \pi | P \equiv Q | u?(y)P', \\ \and \\\\ \and \\ \;\;\; u \in \meaningof{a}, \forall z.P'\{z/y\} \in \meaningof{E\{z/b\}}\}, \and \\ \meaningof{a!E} = \{ P \in \pi | P \equiv Q | x!\langle P' \rangle, x \in \meaningof{a} P' \in \meaningof{E}\} }
\end{mathpar}

\begin{mathpar}
 \inferrule* [lab=nominal] {} {\meaningof{\quotep{E}} = \{ \quotep{P} \in \quotep{\pi} | P \in \meaningof{E} \}, \and \meaningof{\quotep{P}} = \{ \quotep{Q} \in \quotep{\pi} | P \equiv Q \} \and \\ \meaningof{@\quotep{E}} = \{ P \in \pi | P \equiv @x, x \in \meaningof{E} \}}
\end{mathpar}

\begin{eqnarray*}
  \\
  \meaningof{-} : TS \to ST
\end{eqnarray*}

\begin{eqnarray*}
  \\
  L : TS \to ST
\end{eqnarray*}

\begin{eqnarray*}
  \\
  P \models E \iff P \in \meaningof{E}
\end{eqnarray*}

\begin{eqnarray*}
  P \approx_{L} Q \iff \forall E \in L. P \models E \iff Q \models E
\end{eqnarray*}

\begin{eqnarray*}
  P \approx_{K} Q
\end{eqnarray*}

\begin{eqnarray*}
  P \approx Q
\end{eqnarray*}

$\approx_{K} = \approx = \approx_{L}$

\subsubsection{Contextual duality}

Note that contexts extend the quotation operation to a family of
operations from processes to names. Given a context, $M$, we can
define a \emph{nominal context}, $\quotep{M}$ by $\quotep{M}[P] :=
\quotep{M[P]}$. To foreshadow what is to come we observe that these
operations enjoy a duality with processes very much like the duality
between vectors and maps from vectors to scalars.

Further, because the calculus is essentially higher-order, we have a
correspondence between contexts and processes. More specifically,
given a name $x$ and a context $M$ we can construct $M^{*}_{x}$ such
that 

\begin{mathpar}
  M^{*}_{x} | \lift{x}{P} \red M[P]
\end{mathpar}

namely,

\begin{mathpar}
  M^{*}_{x} := x?(u).M[\dropn{u}]
\end{mathpar}

The dependence of $M^{*}_{x}$ on a name makes it an abstraction, 

\begin{mathpar}
  M^{*} := (x)x?(u).M[\dropn{u}]
\end{mathpar}

\subsection{Additional notation}

It will sometimes be convenient to denote the process a name
quotes. We already have the notation $x = \quotep{P}$, but it will be
convenient to introduce an alternate notation, $\procn{x}$, when we
want to emphasize the connection to the use of the name. Note that, by
virtue of name equivalence, $\quotep{\procn{x}} \nameeq x$; so, the
notation is consistent with previous definitions.

Further, because names have structure it is possible to effect
substitutions on the basis of that structure. This means we need to
upgrade our notation for substitutions, which we accomplish by
adapting comprehension notation. Thus,

\begin{mathpar}
  P\{ y / x : x \in S \}
\end{mathpar}

is interpreted to mean the process derived from P by replacing (in a
capture-avoiding manner) each occurrence of $x$ in $S$ by $y$. For example,

\begin{mathpar}
  P\{ \quotep{\procn{x}|\procn{x}} / x : x \in \freenames{P} \}
\end{mathpar}

will replace each (occurrence) of a free name $x$ in $P$ by
$\quotep{\procn{x}|\procn{x}}$.

Also, we will avail ourselves of the notation $x^{L}$ and $x^{R}$ to
denote injections of a name into disjoint copies of the name
space. There are numerous ways to accomplish this. One example can be
found in \cite{MeredithR05}. This notation overloads to vectors of
names: $\vec{x}^{\pi} := (x_{i}^{\pi} \; : \; 0 \leq i < |\vec{x}| )$ where $\pi \in \{L,R\}$.

We also use $P^{\Box} := P|\Box$.

In \cite{MeredithR05} an interpretation of the new operator is
given. It turns out that there are several possible interpretations
all enjoying the requisite algebraic properties of the operator (see
\cite{milner91polyadicpi}). We will therefore make liberal use of
$(\nu\; \vec{x})P$.

% subsection the_syntax_and_semantics_of_the_notation_system (end)   

\section{Interpretation of QM}
\subsection{Supporting definitions}
\subsubsection{Multiplication}
\begin{mathpar}
  \quotep{Q} \cdot \quotep{R} := \quotep{Q|R}
  \and \\
  \quotep{Q} \cdot P := P\{ \quotep{Q|R} / \quotep{R} : \quotep{R} \in \freenames{P} \}
\end{mathpar}

\paragraph{Discussion}
The first line needs little explanation. The second line says that
each free name of the process is replaced with the multiplication of
that name by the scalar. Multiplication of a scalar (name) by a state
(process) results in a process all the names of which have been `moved
over' by parallel composition with the process the scalar
quotes. There is a subtlety that the bound names have to be
manipulated so that multiplied names aren't accidentally
captured. There are many ways to achieve this.

\begin{remark}\label{rem:multiplication_identities}
  The reader is invited to verify that for all $x,y,z \in \QProc$ and $P \in \Proc$
  \begin{mathpar}
    x \cdot \quotep{0} \equiv x 
    \and
    x \cdot y \equiv y \cdot x
    \and
    x \cdot (y \cdot z) \equiv (x \cdot y) \cdot z
    \and \\
    \quotep{0} \cdot P \equiv P
    \and \\
    x \cdot (y \cdot P) \equiv (x \cdot y) \cdot P
    \and \\
    x \cdot (P|Q) \equiv (x \cdot P) | (x \cdot Q)
    \and \\    
  \end{mathpar}
\end{remark}

\subsubsection{Tensor product}

We define a tensor product on processes by structural induction.

\paragraph{Tensor of sums} First note that all summations, including
$\pzero$ and sequence, can be written $\Sigma_{i} x_{i}.A_{i} +
\Sigma_{j} x_{j}.C_{j}$, where we have grouped input-guarded processes
together and output-guarded processes together.

Thus, we can define the tensor product of two summations, $N_{1}\otimes N_{2}$, where

\begin{mathpar}
  N_{1} := \Sigma_{i} x_{i}.A_{i} + \Sigma_{j} x_{j}.C_{j}
  \and
  N_{2} := \Sigma_{i'} y_{i'}.B_{i'} + \Sigma_{j'} y_{j'}.D_{j'} 
\end{mathpar}

as follows.

\begin{mathpar}
  \Sigma_{i} x_{i}.A_{i} + \Sigma_{j} x_{j}.C_{j} \otimes \Sigma_{i'}
  y_{i'}.B_{i'} + \Sigma_{j'} y_{j'}.D_{j'} 
  \and \\
  := \; \Sigma_{i} \Sigma_{i'} \quotep{\stackrel{\vee}{x_{i}}| \stackrel{\vee}{y_{i'}}}.(A_{i}\otimes B_{i'}) \; | \; \Sigma_{i'} \Sigma_{i} \quotep{\stackrel{\vee}{y_{i'}}|\stackrel{\vee}{x_{i}}}.(B_{i'}\otimes A_{i})
  \and
  \;\; | \;\; \Sigma_{j} \Sigma_{j'} \quotep{\stackrel{\vee}{x_{j}}|\stackrel{\vee}{y_{j'}}}.(A_{j}\otimes B_{j'}) \; | \; \Sigma_{j'} \Sigma_{j} \quotep{\stackrel{\vee}{y_{j'}}|\stackrel{\vee}{x_{j}}}.(B_{j'}\otimes A_{j})
\end{mathpar}

\begin{remark}
  Do we need to $x^{L}$ and $y^{R}$ for this construction as well?
\end{remark}

\paragraph{Tensor of parallel compositions} Next, we distribute tensor
over par.

\begin{mathpar}
  P_{1}|P_{2} \otimes Q_{1}|Q_{2} := (P_{1} \otimes Q_{1}) | (P_{1}
  \otimes Q_{2}) | (P_{2} \otimes Q_{1}) | (P_{2} \otimes Q_{2})
\end{mathpar}

\paragraph{Tensor with dropped names} We treat tensor of a
process with a dropped name as parallel composition.

\begin{mathpar}
  P \otimes \dropn{x} := P | \dropn{x}
\end{mathpar}

\paragraph{Tensor of agents}

Finally, we need to define tensor on agents. Note that the definition
of tensor on normal products only tensors inputs with inputs and
outputs with outputs. Thus, we only have to define the operation on
``homogeneous'' pairings.

\begin{mathpar}
  (\vec{x})P \otimes (\vec{y})Q
  \and \\
  := (x_{0}^{L}|y_{0}^{R},\ldots,x_{0}^{L}|y_{n}^{R},\ldots,x_{m}^{L}|y_{0}^{R},\ldots,x_{m}^{L}|y_{n}^R)(P\{ \vec{x}^{L}/\vec{x}\} \otimes Q \{ \vec{y}^{R}/\vec{y}\})
  \and \\
  \clift{\vec{P}} \otimes \clift{\vec{Q}}
  \and \\
  := \clift{P_{0}\otimes Q_{0},\ldots,P_{0}\otimes Q_{n},\ldots,P_{m}\otimes Q_{0},\ldots,P_{m}\otimes Q_{n}}
\end{mathpar}

\begin{remark}
  Observe that arities of tensored abstractions matches arities of
  tensored concretions if the original arities matched. Note also that
  the length of the arities corresponds to the increase in dimension
  we see in ordinary vector space tensor product.
\end{remark}

\begin{remark}
  Operationally, this definition distributes the tensor down to
  components ``linked'' by summation. Tensor over summation is
  intriguing in that it mixes names. Moreover, as a consequence of the
  way it mixes names we have the identities for all $x \in \QProc$ and
  $P,Q \in \Proc$

  \begin{mathpar}
    (x \cdot P) \otimes Q \equiv x \cdot (P \otimes Q) \equiv P \otimes (x \cdot Q)
    \and
    P \otimes \pzero \equiv P
  \end{mathpar}

  that the reader is invited to verify.
\end{remark}

\subsubsection{Annihilation}
\begin{mathpar}
  P^{\perp} := \{ Q | \forall R. P|Q \red^{*} R \Rightarrow R \red^{*} \pzero \}
  \and \\
  P^{\underline{\perp}} := \Sigma_{Q \in P^{\perp}} \quotep{Q}?(y).(\dropn{y}|Q) | \Sigma_{Q \in P^{\perp}} \quotep{Q}\clift{\Box}
\end{mathpar}

\paragraph{Discussion} The reader will note that $P^{\perp}$ is a
\emph{set} of processes, while $P^{\underline{\perp}}$ is a
\emph{context}. We call the set $P^{\perp}$ the \emph{annihilators} of
$P$. The parallel composition of a process in the annihilators of $P$
with $P$ will result in a process, the state space of which has all
paths eventually leading to $\pzero$. Execution may endure loops; but
under reasonable conditions of fairness (naturally guaranteed under
most notions of bisimulation) such a composite process cannot get
stuck in such a loop and will, eventually pop out and terminate.

The context $P^{\underline{\perp}}$ is ready and willing to ``take the
$P$ out of'' the process to which it is applied. It will effectively
transmit the code of the process to which it is applied to one of the
annihilators and run the process against it.

\subsubsection{Evaluation}
We fix $M$ a domain of fully abstract interpretation with an equality
coincident with bisimulation. We take $\meaningof{\cdot} : \Proc \to
M$ to be the map interpreting processes and $\nmeaningof{\cdot} : \M
\to Proc$ to be the map running the other way. Then we define

\begin{mathpar}
  \int P := \nmeaningof{\meaningof{P}}
\end{mathpar}

\paragraph{Discussion}
There are many fully abstract interpretations of Milner's
$\pi$-calculus. Any of them can be used as a basis for interpreting
the reflective calculus here. Equipped with such a domain it is
largely a matter of grinding through to check that the Yoneda
construction for the normalization-by-evaluation program can be
extended to this setting.

\begin{remark}
  The reader is invited to verify that $\int (P^{\underline{\perp}}[P]) = 0$.
\end{remark}

\subsection{Quantum mechanics}

Table \ref{tbl:core_qm_op_defns} gives the core operational definitions

\begin{table}[htp]\label{tbl:core_qm_op_defns}
  \center{
    \fbox{
      \begin{tabular}{c|c}
        quantum mechanics & process calculus \\
        \hline
        scalar & $x := \quotep{P}$ \\
        state vector & $\state{P} := P$ \\
        dual & $\state{P}^{*} := \event{P^{\underline{\perp}}} := \quotep{P^{\underline{\perp}}}[-]$ \\
        matrix & $ \Sigma_{\alpha} \state{P_{\alpha}}x_{\alpha}\event{Q_{\alpha}}$ \\
        vector addition & $\state{P} + \state{Q} := \state{P | Q}$ \\
        tensor product & $\state{P} \otimes \state{Q} := \state{P \otimes Q}$ \\
        inner product & $\innerprod{P}{Q} := \quotep{\int P^{\underline{\perp}}[Q]}$ \\
      \end{tabular}
    }
  }
  \caption{QM - operational definitions}
\end{table}

where

\begin{mathpar}
  \prmatrix{P}{Q} := \fprmatrix{P}{\quotep{\pzero}}{Q}
  \and
  \fprmatrix{P}{x}{Q} := (\state{P},x,\event{Q})
  \and
  (\fprmatrix{P}{x}{Q})(\state{R}) := x \cdot \innerprod{Q}{R} \cdot \state{P}
  \and
  (\fprmatrix{P}{x}{Q})(\event{R}) := x \cdot \innerprod{R}{P} \cdot \event{Q}
\end{mathpar}

\paragraph{Discussion}
As promised: vectors (aka states) are represented as processes; duals
as contextual duals; inner product definition should be compared with
standard inner product definition for ....

\begin{remark}
  Assuming $\int (P^{\underline{\perp}}[P]) = 0$, the reader is
  invited to verify that $(\fprmatrix{P}{x}{P})(\state{P}) = x \cdot \state{P}$.
\end{remark}

\begin{remark}
  The reader is invited to verify that $\innerprod{P}{Q}$ could
  equally well have been written $\quotep{\int \stackrel{\vee}{x}}$
  where $x = \event{P^{\underline{\perp}}}(Q)$.

  One of the motivations for this remark is that there is another way
  to factor these operations. We could package up evaluation in the dual:

  \begin{mathpar}
    \state{P}^{*} := \event{\int P^{\underline{\perp}}} := \quotep{\int P^{\underline{\perp}}}[-]
  \end{mathpar}

  and then have inner product defined by
  
  \begin{mathpar}
    \innerprod{P}{Q} := \event{P}(Q)
  \end{mathpar}

  Hopefully, experience with the calculations will provide guidance on
  the best factoring.
\end{remark}

\begin{remark}
  Assuming $\int (P^{\underline{\perp}}[P]) = 0$, the reader is
  invited to verify that $\forall P,Q. (\prmatrix{0}{Q})(\state{0}) =
  \state{0}$ and dually $(\prmatrix{P}{0})(\event{0}) = \event{0}$.
\end{remark}

\begin{remark}
  i'm a little worried that i don't (yet) have proper support for
  complex conjugacy. But, the observation above may give us a
  clue. According to Abramsky, it must be the case that the scalars
  are iso to the homset of the identity for the tensor -- which the
  observation above characterizes. 

  For now, we will simply bookmark the notion with $\overline{x}$.
\end{remark}

\subsubsection{Adjointness}

We need to give a definition of $(\cdot)^{\dagger}$ for matrices. The
obvious candidate definition is
\begin{mathpar}
(\Sigma_{\alpha}\fprmatrix{P_{\alpha}}{x_{\alpha}}{Q_{\alpha}})^{\dagger}
= \Sigma_{\alpha}\fprmatrix{(Q_{\alpha}^{\underline{\perp}})^{*}}{\overline{x}_{\alpha}}{P_{\alpha}^{\underline{\perp}}} 
\end{mathpar}

But, $(Q_{\alpha}^{\underline{\perp}})^{*}$ requires a name along
which to communicate the process to achieve the context application.

\subsubsection{Basis for a basis}
If processes label states and ``addition'' of states (a.k.a. vector
addition) is interpreted as parallel composition, what corresponds to
notions of linear independence and basis? Here, we recall that Yoshida
has developed a set of \emph{combinators} for an asynchronous verison
of Milner's $\pi$-calculus. These are a finite set of processes such
any process can be expressed as parallel composition of these
combinators together with liberal uses of the new operator and
replication. We can simply give a translation of these into the
present calculus and have reasonable expectation that the property
carries over. That is, that the resultant set allows to express all
processes via parallel composition. Note, however, that there is no
new operator or replication in this calculus. As a result, we expect
that the corresponding set is actually infinite. That is, we expect
that the space is actually infinite dimensional.

\begin{remark}
  The attentive reader may be a bit concerned. Certainly, the
  collection $S$, $K$ and $I$ is a finite set of
  combinators. Shouldn't we expect to see a finite set of combinators
  for an effectively equivalent system? i am very sympathetic to this
  critique and feel it warrants full attention. On the other hand, i
  also have in mind the following analogy. The natural numbers, as a
  monoid under addition, has exactly $1$ generator, while the natural
  numbers, as a monoid under multiplication, has countably many
  generators (the primes). We observe that the application of the
  lambda calculus is much less resource sensitive than the parallel
  composition of the $\pi$-calculus. Could it be the case that we have
  an analogy of the form
  
  \begin{mathpar}
    m + n : MN :: m*n : M|N
  \end{mathpar}

  giving a similar blow up in the set of ``primes''?  This is such a
  wonderful thought that, even if it's not true, i think it's worth
  writing down.
\end{remark}
 

\documentclass[12pt]{llncs}
%\documentclass{jktr}

\usepackage[pdftex]{hyperref}                   
\usepackage {listings}
\usepackage {mathpartir}
\usepackage{bcprules}
%\usepackage{listings}
                       
\usepackage{graphicx} 
%\usepackage[margins=2.5cm,nohead,nofoot]{geometry}
%\usepackage{geometry}
\usepackage{amsfonts}
\usepackage{amstext}
\usepackage{latexsym}
\usepackage{amssymb}
\usepackage{color}


%\include{myPreamble}
\include{qm2pi.local} 

%\ifpdf
%\usepackage[pdftex]{graphicx}
%\else
%\usepackage{graphicx}
%\fi

 % \ifpdf
%  \usepackage{pdfsync}
%  \if


%\title{Brief Article}
%\author{David F. Snyder}
%\author{L.G. Meredith}

%\address{Dept. of Math., Texas State University--San Marcos, San Marcos, TX 78666}
       
\pagestyle{empty}


\begin{document}

\lstset{language=[Objective]Caml,frame=shadowbox}

\input{qm2pi.front}

% section front matter (end)

\input{qm2pi.intro} 
 
% section introduction (end)

% \input{qm2pi.knotations} 

% section notation (end)

\input{qm2pi.process.calculi} 

% section concurrent_process_calculi_and_spatial_logics_ (end)
    
%\input{qm2pi.knots2pi} 

%\input{qm2pi.trefoil} 

%\input{qm2pi.mainthm} 

% subsection basic_interpretation (end)

%\input{qm2pi.rho.presentation} 
\subsection{The syntax and semantics of the notation system}\label{sub:the_syntax_and_semantics_of_the_notation_system} % (fold)

We now summarize a technical presentation of the calculus that
embodies our theory of dynamics. The typical presentation of such a
calculus follows the style of giving generators and relations on
them. The grammar, below, describing term constructors, freely
generates the set of processes, $\Proc$. This set is then quotiented
by a relation known as structural congruence and it is over this set
that the notion of dynamics is expressed. This presentation is
essentially that of \cite{MeredithR05} with the addition of
polyadicity and summation. For readability we have relegated some of
the technical subtleties to an appendix.

\subsubsection{Process grammar}\label{subsub:process_grammar}

\begin{mathpar}
  \inferrule* [lab=synchronization] {} {{M} \bc \pzero \;|\; x?F \;|\; x!C }
  \and
  \inferrule* [lab=abstraction] {} {{F} \bc (x)P}
  \and
  \inferrule* [lab=concretion] {} {{C} \bc \langle Q \rangle}
  \and
  \inferrule* [lab=process] {} {{P,Q} \bc M \;| \;P|Q \;|\; @{x}}
  \and
  \inferrule* [lab=name] {} {{x} \bc \quotep{P}}
\end{mathpar} 

Note that $\vec{x}$ (resp. $\vec{P}$) denotes a vector of names
(resp. processes) of length $|\vec{x}|$ (resp. $|\vec{P}|$). We adopt
the following useful abbreviations.

\begin{mathpar}
   x?(\vec{y}).P := x.(\vec{y})P \and  x\clift{\vec{P}} := x.\clift{\vec{P}}
   \and x!(y) := \lift{x}{\dropn{y}}
   \and \Pi_{i=0}^{n-1}P_i := P_0 | \ldots | P_{n-1}
\end{mathpar}

\subsubsection{Structural congruence}

\paragraph{Free and bound names and alpha-equivalence.} At the
core of structural equivalence is alpha-equivalence which identifies
process that are the same up to a change of variable. Formally, we
recognize the distinction between free and bound names. The free names
of a process, $\freenames{P}$, may be calculated recursively as
follows:

\begin{mathpar}
\freenames{\pzero} := \emptyset
  \and \\
  \freenames{x?(y).P} := \{ x \} \cup (\freenames{P} \setminus \{ y \})
  \and 
  \freenames{x!\langle P \rangle} := \{ x \} \cup \{ P \} 
  \and \\
  \freenames{P|Q} := \freenames{P} \cup \freenames{Q}
  \and \\
  \freenames{@{x}} := \{ x \}
\end{mathpar}

$\pi$
$\quotep{\pi}$

$\freenames{-} : \pi \to \mathcal{P}(\quotep{\pi})$

\begin{eqnarray*}
  \freenames{\pzero} & := & \emptyset \\
  \freenames{x?(y).P} & := & \{ x \} \cup (\freenames{P} \setminus \{ y \}) \\
  \freenames{x!\langle P \rangle} & := & \{ x \} \cup \{ P \} \\
  \freenames{P|Q} & := & \freenames{P} \cup \freenames{Q} \\
  \freenames{\dropn{x}} & := & \{ x \}
\end{eqnarray*}

The bound names of a process, $\boundnames{P}$, are those names occurring in $P$
that are not free. For example, in $x?(y).0$, the name $x$ is free, while $y$ is bound.

\begin{mathpar}
  \inferrule* [lab=monoidal-laws] {} { P|Q \equiv Q|P \and P|0 \equiv P \and P|(Q|R) \equiv (P|Q)|R }
\end{mathpar}

\begin{mathpar}
  \inferrule* [lab=alpha-equivalence] {} { (x)P \equiv (y)P\{y/x\} \and y \not\in \freenames{P} }
\end{mathpar}

\begin{definition}
Then two processes, $P,Q$, are alpha-equivalent if $P = Q\{\vec{y}/\vec{x}\}$ for
some $\vec{x} \in \boundnames{Q},\vec{y} \in \boundnames{P}$, where $Q\{\vec{y}/\vec{x}\}$
denotes the capture-avoiding substitution of $\vec{y}$ for $\vec{x}$ in $Q$.
\end{definition}

\begin{definition}
  The {\em structural congruence} \cite{SangiorgiWalker} , $\equiv$,
  between processes is the least congruence containing
  alpha-equivalence, satisfying the abelian monoid laws
  (associativity, commutativity and $\pzero$ as identity) for parallel
  composition $|$ and for summation $+$.
\end{definition}

\subsection{Name equivalence}

We take name equivalence, written $\nameeq$, to be the smallest
equivalence relation generated by the following rules.

\begin{mathpar}
\inferrule*[lab=Quote-drop]
{ }
{ \quotep{@{x}} \nameeq x }

\inferrule*[lab=Struct-equiv]
{ P \scong Q }
{ \quotep{P} \nameeq \quotep{Q} }
\end{mathpar}

The astute reader will have noticed that the mutual recursion of names
and processes imposes a mutual recursion on alpha-equivalence and
structural equivalence via name-equivalence. Fortunately, all of this
works out pleasantly and we may calculate in the natural way, free of
concern. The reader interested in the details is referred to the
appendix \ref{appendix:rho_details}.

\subsection{Substitution}

We use $\Proc$ for the set of processes, $\QProc$ for the set of
names, and $\id{\{}\vec{y} / \vec{x} \id{\}}$ to denote partial maps,
$s : \QProc \rightarrow \QProc$. A map, $s$ lifts, uniquely, to a map
on process terms, $\widehat{s} : \Proc \rightarrow \Proc$ by the
following equations.

\begin{mathpar}
  (0) \psubstp{Q}{P} := 0 \\
  (R \juxtap S) \psubstp{Q}{P}
  :=    
  (R)\psubstp{Q}{P} \juxtap (S) \psubstp{Q}{P} \\
  (x?(y).R) \psubstp{Q}{P}    
  :=    
  (x)\substp{Q}{P} (z)\concat( (R \psubstn{z}{y}) \psubstp{Q}{P} ) \\
  (\lift{x}{R}) \psubstp{Q}{P}  
  :=
  \lift{(x)\substp{Q}{P}}{ R \psubstp{Q}{P} } \\
%   (\dropn{x})  \psubstp{Q}{P}       
%   := 
%   \left\{ 
%     \begin{array}{ccc} 
%       \dropn{\quotep{Q}} & & x \nameeq \quotep{P} \\
%       \dropn{x} & & otherwise \\
%     \end{array}
%   \right. 
  (\dropn{x})  \psubstp{Q}{P}       
  := 
  \left\{ 
    \begin{array}{ccc} 
      Q & & x \nameeq \quotep{P} \\
      \dropn{x} & & otherwise \\
    \end{array}
  \right.
\end{mathpar}
 

where

\begin{eqnarray}
  (x)\id{\{} \lpquote Q \rpquote / \lpquote P \rpquote \id{\}}            = 
  \left\{ 
    \begin{array}{ccc}
      \lpquote Q \rpquote & & x \nameeq \lpquote P \rpquote \\
      x & & otherwise \\
    \end{array}
  \right. \nonumber
\end{eqnarray}

and $z$ is chosen distinct from $\quotep{P}$, $\quotep{Q}$, the free
names in $Q$, and all the names in $R$. Our $\alpha$-equivalence will
be built in the standard way from this substitution.

\begin{remark}\label{rem:no_self_referential_names}
  One consequence of these definitions is that $\forall P. \quotep{P}
  \not\in \freenames{P}$.
\end{remark}

\subsection{ Dynamic quote: an example }

Anticipating something of what's to come, consider applying the
substitution, $\widehat{\id{\{}u / z \id{\}}}$, to the following pair
of processes, $\lift{w}{y!(z)}$ and $w[ \lpquote y!(z) \rpquote ]$.

\begin{eqnarray}
	\lift{w}{y!(z)}\widehat{\id{\{}u / z \id{\}}}
		& = &
		\lift{w}{y!(u)} \nonumber\\
	w[ \lpquote y!(z) \rpquote ] \widehat{ \id{\{}u / z \id{\}} }
		& = &
		w[ \lpquote y!(z) \rpquote ] \nonumber
\end{eqnarray}

Because the body of the process between quotes is impervious to
substitution, we get radically different answers. In fact, by
examining the first process in an input context,
e.g. $x?(z).\lift{w}{y!(z)}$, we see that the process under the lift
operator may be shaped by prefixed inputs binding a name inside it. In
this sense, the lift operator will be seen as a way to dynamically
construct processes before reifying them as names.

Finally equipped with these standard features we can present the
dynamics of the calculus.

\subsubsection{Operational semantics} 

Finally, we introduce the computational dynamics. What marks these
algebras as distinct from other more traditionally studied algebraic
structures, e.g. vector spaces or polynomial rings, is the manner in
which dynamics is captured. In traditional structures, dynamics is typically
expressed through morphisms between such structures, as in linear maps
between vector spaces or morphisms between rings. In algebras
associated with the semantics of computation, the dynamics is
expressed as part of the algebraic structure itself, through a
reduction reduction relation typically denoted by $\red$. Below, we
give a recursive presentation of this relation for the calculus used
in the encoding.

$\red \subseteq \pi \times \pi$
$\red : \pi \to \mathcal{P}(\pi)$

\begin{mathpar}
  \inferrule* [lab=Comm] { \textsf{match}( x_{src}, x_{trgt} ) } { x_{trgt}?(y)P \; | \; x_{src}!\langle {Q} \rangle \red P\{\quotep{Q}/y}\} }
  \and \\
  \inferrule* [lab=Par] {{P} \red {P}'} {{{P} | {Q}} \red {{P}' | {Q}}}
  \and
  \inferrule* [lab=Equiv]{{{P} \scong {P}'} \andalso {{P}' \red {Q}'} \andalso {{Q}' \scong {Q}}}{{P} \red {Q}}
\end{mathpar}

\begin{eqnarray*}
  match_{\equiv} (\quotep{P},\quotep{Q}) & := & P \equiv Q \\
  match_{\dagger}(\quotep{P},\quotep{Q}) & := & \forall R. P|Q \red^{*} R => R \red^{*} 0 \\
  match_{K}(\quotep{P},\quotep{Q}) & := & K \mbox{ for some context } K
\end{eqnarray*}

$u?(x)P | u!\langle Q \rangle \red P\{\quotep{Q}/x\}$

%We write $\wred$ for $\red^*$, and $P\red$ if $\exists Q $ such that $ P \red Q$.
We write $P\red$ if $\exists Q $ such that $ P \red Q$ and $P\not\red$, otherwise.

\section{Replication}

As mentioned before, it is known that replication (and hence
recursion) can be implemented in a higher-order process algebra
\cite{SangiorgiWalker}. As our first example of calculation with the
machinery thus far presented we give the construction explicitly in
the {\rhoc}.

\begin{eqnarray}
	D_{x} & := & \prefix{x}{y}{(\binpar{\outputp{x}{y}}{@{y}})} \nonumber\\
	\bangp_{x}{P} & := & \binpar{{x}!\langle{\binpar{D_{x}}{P}}\rangle}{D_{x}} \nonumber
\end{eqnarray}

\begin{eqnarray}
	\bangp_{x}{P} & & \nonumber\\
	=
	& {x}!\langle{(\prefix{x}{y}{(\outputp{x}{y} | @{y})) | P}}\rangle 
	      | \prefix{x}{y}{(\outputp{x}{y} | @{y})} & \nonumber\\
	\red
	& (\outputp{x}{y} | @{y})\substn{\quotep{(\prefix{x}{y}{(@{y} | \outputp{x}{y})) | P}}}{y} & \nonumber\\
	=
	& \outputp{x}{\quotep{(\prefix{x}{y}{(\outputp{x}{y} | @{y})) | P}}}
	  | {(\prefix{x}{y}{(\outputp{x}{y} | @{y})) | P}} & \nonumber\\
	\red
	& \ldots & \nonumber\\
	\red^*
	& P | P | \ldots & \nonumber
\end{eqnarray}

Of course, this encoding, as an implementation, runs away, unfolding
$\bangp{P}$ eagerly. A lazier and more implementable replication
operator, restricted to input-guarded processes, may be obtained as follows.

\begin{eqnarray}
\bangp{\prefix{u}{v}{P}} 
	:= 
	\binpar{\lift{x}{\prefix{u}{v}{(\binpar{D(x)}{P})}}}{D(x)} \nonumber
\end{eqnarray}

\begin{remark}
  Note that the lazier definition still does not deal with summation
  or mixed summation (i.e. sums over input and output). The reader is
  invited to construct definitions of replication that deal with these
  features. 

  Further, the definitions are parameterized in a name, $x$. Can you,
  gentle reader, make a definition that eliminates this parameter and
  guarantees no accidental interaction between the replication
  machinery and the process being replicated -- i.e. no accidental
  sharing of names used by the process to get its work done and the
  name(s) used by the replication to effect copying. This latter
  revision of the definition of replication is crucial to obtaining
  the expected identity $!!P \sim !P$.
\end{remark}

\begin{remark}\label{rem:paradoxical_combinator}
  The reader familiar with the lambda calculus will have noticed the
  similarity between $D$ and the paradoxical combinator.

  [Ed. note: the existence of this seems to suggest we have to be more
  restrictive on the set of processes and names we admit if we are to
  support no-cloning.]
\end{remark}

\subsubsection{Bisimulation}

The computational dynamics gives rise to another kind of equivalence,
the equivalence of computational behavior. As previously mentioned
this is typically captured \emph{via} some form of bisimulation.

% The notion we use in this paper is weak barbed bisimulation
% \cite{milner91polyadicpi}.

The notion we use in this paper is derived from weak barbed
bisimulation \cite{milner91polyadicpi}. 

\begin{definition}
An \emph{observation relation}, $\downarrow_{\mathcal N}$, over a set
of names, $\mathcal N$, is the smallest relation satisfying the rules
below.

\infrule[Out-barb]{y \in {\mathcal N}, \; x \nameeq y}
		  {\outputp{x}{v} \downarrow_{\mathcal N} x}
\infrule[Par-barb]{\mbox{$P\downarrow_{\mathcal N} x$ or $Q\downarrow_{\mathcal N} x$}}
		  {\binpar{P}{Q} \downarrow_{\mathcal N} x}

We write $P \Downarrow_{\mathcal N} x$ if there is $Q$ such that 
$P \wred Q$ and $Q \downarrow_{\mathcal N} x$.
\end{definition}

\begin{definition}
%\label{def.bbisim}
An  ${\mathcal N}$-\emph{barbed bisimulation} over a set of names, ${\mathcal N}$, is a symmetric binary relation 
${\mathcal S}_{\mathcal N}$ between agents such that $P\rel{S}_{\mathcal N}Q$ implies:
\begin{enumerate}
\item If $P \red P'$ then $Q \wred Q'$ and $P'\rel{S}_{\mathcal N} Q'$.
\item If $P\downarrow_{\mathcal N} x$, then $Q\Downarrow_{\mathcal N} x$.
\end{enumerate}
$P$ is ${\mathcal N}$-barbed bisimilar to $Q$, written
$P \wbbisim_{\mathcal N} Q$, if $P \rel{S}_{\mathcal N} Q$ for some ${\mathcal N}$-barbed bisimulation ${\mathcal S}_{\mathcal N}$.
\end{definition}

$\mathcal{R} \subseteq \pi \times \pi$

$P \mathcal{R} Q => \forall P'. P \red P' \Rightarrow \exists Q'. Q \red Q', P' \mathcal{R} Q'$

$P \vdash x \Rightarrow Q \vdash x$

\begin{mathpar}
  \inferrule*[lab=Out-barb]{x \nameeq y}{{y}!\langle{Q}\rangle \vdash x}
  \and
  \inferrule*[lab=Par-barb]{\mbox{$P\vdash x$ or $Q\vdash x$}}{\binpar{P}{Q} \vdash x}
\end{mathpar}

\subsubsection{Contexts}

One of the principle advantages of computational calculi like the
$\pi$-calculus is a well-defined notion of context,
contextual-equivalence and a correlation between
contextual-equivalence and notions of bisimulation. The notion of
context allows the decomposition of a process into (sub-)process and
its syntactic environment, its context. Thus, a context may be
thought of as a process with a ``hole'' (written $\Box$) in it. The
application of a context $M$ to a process $P$, written $M[P]$, is
tantamount to filling the hole in $M$ with $P$. In this paper we do
not need the full weight of this theory, but do make use of the notion
of context in the proof the main theorem. 

\begin{mathpar}
  \inferrule* [lab=summation] {} {{M_{M},M_{N}} \bc \Box \;|\; x.M_{A} \;|\; M_{M}+M_{N}}
  \and
  \inferrule* [lab=agent] {} {{M_{A}} \bc (\vec{x})M_{P} \;| \; \clift{P_0,\ldots,M_{P},\ldots,P_N}}
  \and \\
  \inferrule* [lab=process] {} {{M_{P}} \bc M_{N} \;| \;P|M_{P} }
\end{mathpar} 

\begin{mathpar}
  \inferrule* [lab=sychronization] {} {M_{N} \bc \Box \;|\; x?M_{F} \;|\; x!M_{C}}
  \and
  \inferrule* [lab=abstraction] {} {{M_{F}} \bc (x)M_{P} }
  \and
  \inferrule* [lab=concretion] {} {{M_{C}} \bc \langle M_{P} \rangle }
  \and \\
  \inferrule* [lab=process] {} {{M_{P}} \bc M_{N} \;| \;P|M_{P} }
\end{mathpar}

\begin{definition}[contextual application] Given a context $M$, and
  process $P$, we define the \emph{contextual application}, $M[P] :=
  M\{P/\Box\}$. That is, the contextual application of M to P is the
  substitution of $P$ for $\Box$ in $M$.
\end{definition}

$\meaningof{-} : L \to \mathcal{P}(\pi)$

\begin{mathpar}
  \inferrule* [lab=collection] {} {\meaningof{true} = \pi, \and \meaningof{~E} = \pi \setminus \meaningof{E}, \and \meaningof{E_{1} \& E_{2}} = \meaningof{E_{1}} \cap \meaningof{E_{2}}}
\end{mathpar}

\begin{mathpar}
  \inferrule* [lab=structure] {} {\meaningof{0} = \{ P \in \pi | P \equiv 0 \}, \and \\ \meaningof{E_1 | E_2} = \{ P \in \pi | P \equiv P_{1} | P_{2}, P_{1} \in \meaningof{E_{1}}, P_{2} \in \meaningof{E_2}\} }
\end{mathpar}

\begin{mathpar}
 \inferrule* [lab=behavior] {} {\meaningof{\langle a?b \rangle E} = \{ P \in \pi | P \equiv Q | u?(y)P', \\ \and \\\\ \and \\ \;\;\; u \in \meaningof{a}, \forall z.P'\{z/y\} \in \meaningof{E\{z/b\}}\}, \and \\ \meaningof{a!E} = \{ P \in \pi | P \equiv Q | x!\langle P' \rangle, x \in \meaningof{a} P' \in \meaningof{E}\} }
\end{mathpar}

\begin{mathpar}
 \inferrule* [lab=nominal] {} {\meaningof{\quotep{E}} = \{ \quotep{P} \in \quotep{\pi} | P \in \meaningof{E} \}, \and \meaningof{\quotep{P}} = \{ \quotep{Q} \in \quotep{\pi} | P \equiv Q \} \and \\ \meaningof{@\quotep{E}} = \{ P \in \pi | P \equiv @x, x \in \meaningof{E} \}}
\end{mathpar}

\begin{eqnarray*}
  \\
  \meaningof{-} : TS \to ST
\end{eqnarray*}

\begin{eqnarray*}
  \\
  L : TS \to ST
\end{eqnarray*}

\begin{eqnarray*}
  \\
  P \models E \iff P \in \meaningof{E}
\end{eqnarray*}

\begin{eqnarray*}
  P \approx_{L} Q \iff \forall E \in L. P \models E \iff Q \models E
\end{eqnarray*}

\begin{eqnarray*}
  P \approx_{K} Q
\end{eqnarray*}

\begin{eqnarray*}
  P \approx Q
\end{eqnarray*}

$\approx_{K} = \approx = \approx_{L}$

\subsubsection{Contextual duality}

Note that contexts extend the quotation operation to a family of
operations from processes to names. Given a context, $M$, we can
define a \emph{nominal context}, $\quotep{M}$ by $\quotep{M}[P] :=
\quotep{M[P]}$. To foreshadow what is to come we observe that these
operations enjoy a duality with processes very much like the duality
between vectors and maps from vectors to scalars.

Further, because the calculus is essentially higher-order, we have a
correspondence between contexts and processes. More specifically,
given a name $x$ and a context $M$ we can construct $M^{*}_{x}$ such
that 

\begin{mathpar}
  M^{*}_{x} | \lift{x}{P} \red M[P]
\end{mathpar}

namely,

\begin{mathpar}
  M^{*}_{x} := x?(u).M[\dropn{u}]
\end{mathpar}

The dependence of $M^{*}_{x}$ on a name makes it an abstraction, 

\begin{mathpar}
  M^{*} := (x)x?(u).M[\dropn{u}]
\end{mathpar}

\subsection{Additional notation}

It will sometimes be convenient to denote the process a name
quotes. We already have the notation $x = \quotep{P}$, but it will be
convenient to introduce an alternate notation, $\procn{x}$, when we
want to emphasize the connection to the use of the name. Note that, by
virtue of name equivalence, $\quotep{\procn{x}} \nameeq x$; so, the
notation is consistent with previous definitions.

Further, because names have structure it is possible to effect
substitutions on the basis of that structure. This means we need to
upgrade our notation for substitutions, which we accomplish by
adapting comprehension notation. Thus,

\begin{mathpar}
  P\{ y / x : x \in S \}
\end{mathpar}

is interpreted to mean the process derived from P by replacing (in a
capture-avoiding manner) each occurrence of $x$ in $S$ by $y$. For example,

\begin{mathpar}
  P\{ \quotep{\procn{x}|\procn{x}} / x : x \in \freenames{P} \}
\end{mathpar}

will replace each (occurrence) of a free name $x$ in $P$ by
$\quotep{\procn{x}|\procn{x}}$.

Also, we will avail ourselves of the notation $x^{L}$ and $x^{R}$ to
denote injections of a name into disjoint copies of the name
space. There are numerous ways to accomplish this. One example can be
found in \cite{MeredithR05}. This notation overloads to vectors of
names: $\vec{x}^{\pi} := (x_{i}^{\pi} \; : \; 0 \leq i < |\vec{x}| )$ where $\pi \in \{L,R\}$.

We also use $P^{\Box} := P|\Box$.

In \cite{MeredithR05} an interpretation of the new operator is
given. It turns out that there are several possible interpretations
all enjoying the requisite algebraic properties of the operator (see
\cite{milner91polyadicpi}). We will therefore make liberal use of
$(\nu\; \vec{x})P$.

% subsection the_syntax_and_semantics_of_the_notation_system (end)   

\input{qm2pi.qmops} 

\input{qm2pi.sterngerlach} 

\input{qm2pi.metric} 

% section concurrent_process_calculi (end)

%\input{qm2pi.proofsketch}

% section proof sketch (end)

%\input{qm2pi.slviaknots} 

% section spatial logic via knots (end)

\input{qm2pi.conclusion}

% section conclusion (end)

%\input{qm2pi.dtcodes} 

% section wiring algorithm (end)

\input{qm2pi.ack} 

% section acknowledgments (end)

\newpage


\bibliographystyle{plain}   
\bibliography{../../biblios/main.bib}

\input{qm2pi.rhodetails}

\end{document}

 

\documentclass[12pt]{llncs}
%\documentclass{jktr}

\usepackage[pdftex]{hyperref}                   
\usepackage {listings}
\usepackage {mathpartir}
\usepackage{bcprules}
%\usepackage{listings}
                       
\usepackage{graphicx} 
%\usepackage[margins=2.5cm,nohead,nofoot]{geometry}
%\usepackage{geometry}
\usepackage{amsfonts}
\usepackage{amstext}
\usepackage{latexsym}
\usepackage{amssymb}
\usepackage{color}


%\include{myPreamble}
\include{qm2pi.local} 

%\ifpdf
%\usepackage[pdftex]{graphicx}
%\else
%\usepackage{graphicx}
%\fi

 % \ifpdf
%  \usepackage{pdfsync}
%  \if


%\title{Brief Article}
%\author{David F. Snyder}
%\author{L.G. Meredith}

%\address{Dept. of Math., Texas State University--San Marcos, San Marcos, TX 78666}
       
\pagestyle{empty}


\begin{document}

\lstset{language=[Objective]Caml,frame=shadowbox}

\input{qm2pi.front}

% section front matter (end)

\input{qm2pi.intro} 
 
% section introduction (end)

% \input{qm2pi.knotations} 

% section notation (end)

\input{qm2pi.process.calculi} 

% section concurrent_process_calculi_and_spatial_logics_ (end)
    
%\input{qm2pi.knots2pi} 

%\input{qm2pi.trefoil} 

%\input{qm2pi.mainthm} 

% subsection basic_interpretation (end)

%\input{qm2pi.rho.presentation} 
\subsection{The syntax and semantics of the notation system}\label{sub:the_syntax_and_semantics_of_the_notation_system} % (fold)

We now summarize a technical presentation of the calculus that
embodies our theory of dynamics. The typical presentation of such a
calculus follows the style of giving generators and relations on
them. The grammar, below, describing term constructors, freely
generates the set of processes, $\Proc$. This set is then quotiented
by a relation known as structural congruence and it is over this set
that the notion of dynamics is expressed. This presentation is
essentially that of \cite{MeredithR05} with the addition of
polyadicity and summation. For readability we have relegated some of
the technical subtleties to an appendix.

\subsubsection{Process grammar}\label{subsub:process_grammar}

\begin{mathpar}
  \inferrule* [lab=synchronization] {} {{M} \bc \pzero \;|\; x?F \;|\; x!C }
  \and
  \inferrule* [lab=abstraction] {} {{F} \bc (x)P}
  \and
  \inferrule* [lab=concretion] {} {{C} \bc \langle Q \rangle}
  \and
  \inferrule* [lab=process] {} {{P,Q} \bc M \;| \;P|Q \;|\; @{x}}
  \and
  \inferrule* [lab=name] {} {{x} \bc \quotep{P}}
\end{mathpar} 

Note that $\vec{x}$ (resp. $\vec{P}$) denotes a vector of names
(resp. processes) of length $|\vec{x}|$ (resp. $|\vec{P}|$). We adopt
the following useful abbreviations.

\begin{mathpar}
   x?(\vec{y}).P := x.(\vec{y})P \and  x\clift{\vec{P}} := x.\clift{\vec{P}}
   \and x!(y) := \lift{x}{\dropn{y}}
   \and \Pi_{i=0}^{n-1}P_i := P_0 | \ldots | P_{n-1}
\end{mathpar}

\subsubsection{Structural congruence}

\paragraph{Free and bound names and alpha-equivalence.} At the
core of structural equivalence is alpha-equivalence which identifies
process that are the same up to a change of variable. Formally, we
recognize the distinction between free and bound names. The free names
of a process, $\freenames{P}$, may be calculated recursively as
follows:

\begin{mathpar}
\freenames{\pzero} := \emptyset
  \and \\
  \freenames{x?(y).P} := \{ x \} \cup (\freenames{P} \setminus \{ y \})
  \and 
  \freenames{x!\langle P \rangle} := \{ x \} \cup \{ P \} 
  \and \\
  \freenames{P|Q} := \freenames{P} \cup \freenames{Q}
  \and \\
  \freenames{@{x}} := \{ x \}
\end{mathpar}

$\pi$
$\quotep{\pi}$

$\freenames{-} : \pi \to \mathcal{P}(\quotep{\pi})$

\begin{eqnarray*}
  \freenames{\pzero} & := & \emptyset \\
  \freenames{x?(y).P} & := & \{ x \} \cup (\freenames{P} \setminus \{ y \}) \\
  \freenames{x!\langle P \rangle} & := & \{ x \} \cup \{ P \} \\
  \freenames{P|Q} & := & \freenames{P} \cup \freenames{Q} \\
  \freenames{\dropn{x}} & := & \{ x \}
\end{eqnarray*}

The bound names of a process, $\boundnames{P}$, are those names occurring in $P$
that are not free. For example, in $x?(y).0$, the name $x$ is free, while $y$ is bound.

\begin{mathpar}
  \inferrule* [lab=monoidal-laws] {} { P|Q \equiv Q|P \and P|0 \equiv P \and P|(Q|R) \equiv (P|Q)|R }
\end{mathpar}

\begin{mathpar}
  \inferrule* [lab=alpha-equivalence] {} { (x)P \equiv (y)P\{y/x\} \and y \not\in \freenames{P} }
\end{mathpar}

\begin{definition}
Then two processes, $P,Q$, are alpha-equivalent if $P = Q\{\vec{y}/\vec{x}\}$ for
some $\vec{x} \in \boundnames{Q},\vec{y} \in \boundnames{P}$, where $Q\{\vec{y}/\vec{x}\}$
denotes the capture-avoiding substitution of $\vec{y}$ for $\vec{x}$ in $Q$.
\end{definition}

\begin{definition}
  The {\em structural congruence} \cite{SangiorgiWalker} , $\equiv$,
  between processes is the least congruence containing
  alpha-equivalence, satisfying the abelian monoid laws
  (associativity, commutativity and $\pzero$ as identity) for parallel
  composition $|$ and for summation $+$.
\end{definition}

\subsection{Name equivalence}

We take name equivalence, written $\nameeq$, to be the smallest
equivalence relation generated by the following rules.

\begin{mathpar}
\inferrule*[lab=Quote-drop]
{ }
{ \quotep{@{x}} \nameeq x }

\inferrule*[lab=Struct-equiv]
{ P \scong Q }
{ \quotep{P} \nameeq \quotep{Q} }
\end{mathpar}

The astute reader will have noticed that the mutual recursion of names
and processes imposes a mutual recursion on alpha-equivalence and
structural equivalence via name-equivalence. Fortunately, all of this
works out pleasantly and we may calculate in the natural way, free of
concern. The reader interested in the details is referred to the
appendix \ref{appendix:rho_details}.

\subsection{Substitution}

We use $\Proc$ for the set of processes, $\QProc$ for the set of
names, and $\id{\{}\vec{y} / \vec{x} \id{\}}$ to denote partial maps,
$s : \QProc \rightarrow \QProc$. A map, $s$ lifts, uniquely, to a map
on process terms, $\widehat{s} : \Proc \rightarrow \Proc$ by the
following equations.

\begin{mathpar}
  (0) \psubstp{Q}{P} := 0 \\
  (R \juxtap S) \psubstp{Q}{P}
  :=    
  (R)\psubstp{Q}{P} \juxtap (S) \psubstp{Q}{P} \\
  (x?(y).R) \psubstp{Q}{P}    
  :=    
  (x)\substp{Q}{P} (z)\concat( (R \psubstn{z}{y}) \psubstp{Q}{P} ) \\
  (\lift{x}{R}) \psubstp{Q}{P}  
  :=
  \lift{(x)\substp{Q}{P}}{ R \psubstp{Q}{P} } \\
%   (\dropn{x})  \psubstp{Q}{P}       
%   := 
%   \left\{ 
%     \begin{array}{ccc} 
%       \dropn{\quotep{Q}} & & x \nameeq \quotep{P} \\
%       \dropn{x} & & otherwise \\
%     \end{array}
%   \right. 
  (\dropn{x})  \psubstp{Q}{P}       
  := 
  \left\{ 
    \begin{array}{ccc} 
      Q & & x \nameeq \quotep{P} \\
      \dropn{x} & & otherwise \\
    \end{array}
  \right.
\end{mathpar}
 

where

\begin{eqnarray}
  (x)\id{\{} \lpquote Q \rpquote / \lpquote P \rpquote \id{\}}            = 
  \left\{ 
    \begin{array}{ccc}
      \lpquote Q \rpquote & & x \nameeq \lpquote P \rpquote \\
      x & & otherwise \\
    \end{array}
  \right. \nonumber
\end{eqnarray}

and $z$ is chosen distinct from $\quotep{P}$, $\quotep{Q}$, the free
names in $Q$, and all the names in $R$. Our $\alpha$-equivalence will
be built in the standard way from this substitution.

\begin{remark}\label{rem:no_self_referential_names}
  One consequence of these definitions is that $\forall P. \quotep{P}
  \not\in \freenames{P}$.
\end{remark}

\subsection{ Dynamic quote: an example }

Anticipating something of what's to come, consider applying the
substitution, $\widehat{\id{\{}u / z \id{\}}}$, to the following pair
of processes, $\lift{w}{y!(z)}$ and $w[ \lpquote y!(z) \rpquote ]$.

\begin{eqnarray}
	\lift{w}{y!(z)}\widehat{\id{\{}u / z \id{\}}}
		& = &
		\lift{w}{y!(u)} \nonumber\\
	w[ \lpquote y!(z) \rpquote ] \widehat{ \id{\{}u / z \id{\}} }
		& = &
		w[ \lpquote y!(z) \rpquote ] \nonumber
\end{eqnarray}

Because the body of the process between quotes is impervious to
substitution, we get radically different answers. In fact, by
examining the first process in an input context,
e.g. $x?(z).\lift{w}{y!(z)}$, we see that the process under the lift
operator may be shaped by prefixed inputs binding a name inside it. In
this sense, the lift operator will be seen as a way to dynamically
construct processes before reifying them as names.

Finally equipped with these standard features we can present the
dynamics of the calculus.

\subsubsection{Operational semantics} 

Finally, we introduce the computational dynamics. What marks these
algebras as distinct from other more traditionally studied algebraic
structures, e.g. vector spaces or polynomial rings, is the manner in
which dynamics is captured. In traditional structures, dynamics is typically
expressed through morphisms between such structures, as in linear maps
between vector spaces or morphisms between rings. In algebras
associated with the semantics of computation, the dynamics is
expressed as part of the algebraic structure itself, through a
reduction reduction relation typically denoted by $\red$. Below, we
give a recursive presentation of this relation for the calculus used
in the encoding.

$\red \subseteq \pi \times \pi$
$\red : \pi \to \mathcal{P}(\pi)$

\begin{mathpar}
  \inferrule* [lab=Comm] { \textsf{match}( x_{src}, x_{trgt} ) } { x_{trgt}?(y)P \; | \; x_{src}!\langle {Q} \rangle \red P\{\quotep{Q}/y}\} }
  \and \\
  \inferrule* [lab=Par] {{P} \red {P}'} {{{P} | {Q}} \red {{P}' | {Q}}}
  \and
  \inferrule* [lab=Equiv]{{{P} \scong {P}'} \andalso {{P}' \red {Q}'} \andalso {{Q}' \scong {Q}}}{{P} \red {Q}}
\end{mathpar}

\begin{eqnarray*}
  match_{\equiv} (\quotep{P},\quotep{Q}) & := & P \equiv Q \\
  match_{\dagger}(\quotep{P},\quotep{Q}) & := & \forall R. P|Q \red^{*} R => R \red^{*} 0 \\
  match_{K}(\quotep{P},\quotep{Q}) & := & K \mbox{ for some context } K
\end{eqnarray*}

$u?(x)P | u!\langle Q \rangle \red P\{\quotep{Q}/x\}$

%We write $\wred$ for $\red^*$, and $P\red$ if $\exists Q $ such that $ P \red Q$.
We write $P\red$ if $\exists Q $ such that $ P \red Q$ and $P\not\red$, otherwise.

\section{Replication}

As mentioned before, it is known that replication (and hence
recursion) can be implemented in a higher-order process algebra
\cite{SangiorgiWalker}. As our first example of calculation with the
machinery thus far presented we give the construction explicitly in
the {\rhoc}.

\begin{eqnarray}
	D_{x} & := & \prefix{x}{y}{(\binpar{\outputp{x}{y}}{@{y}})} \nonumber\\
	\bangp_{x}{P} & := & \binpar{{x}!\langle{\binpar{D_{x}}{P}}\rangle}{D_{x}} \nonumber
\end{eqnarray}

\begin{eqnarray}
	\bangp_{x}{P} & & \nonumber\\
	=
	& {x}!\langle{(\prefix{x}{y}{(\outputp{x}{y} | @{y})) | P}}\rangle 
	      | \prefix{x}{y}{(\outputp{x}{y} | @{y})} & \nonumber\\
	\red
	& (\outputp{x}{y} | @{y})\substn{\quotep{(\prefix{x}{y}{(@{y} | \outputp{x}{y})) | P}}}{y} & \nonumber\\
	=
	& \outputp{x}{\quotep{(\prefix{x}{y}{(\outputp{x}{y} | @{y})) | P}}}
	  | {(\prefix{x}{y}{(\outputp{x}{y} | @{y})) | P}} & \nonumber\\
	\red
	& \ldots & \nonumber\\
	\red^*
	& P | P | \ldots & \nonumber
\end{eqnarray}

Of course, this encoding, as an implementation, runs away, unfolding
$\bangp{P}$ eagerly. A lazier and more implementable replication
operator, restricted to input-guarded processes, may be obtained as follows.

\begin{eqnarray}
\bangp{\prefix{u}{v}{P}} 
	:= 
	\binpar{\lift{x}{\prefix{u}{v}{(\binpar{D(x)}{P})}}}{D(x)} \nonumber
\end{eqnarray}

\begin{remark}
  Note that the lazier definition still does not deal with summation
  or mixed summation (i.e. sums over input and output). The reader is
  invited to construct definitions of replication that deal with these
  features. 

  Further, the definitions are parameterized in a name, $x$. Can you,
  gentle reader, make a definition that eliminates this parameter and
  guarantees no accidental interaction between the replication
  machinery and the process being replicated -- i.e. no accidental
  sharing of names used by the process to get its work done and the
  name(s) used by the replication to effect copying. This latter
  revision of the definition of replication is crucial to obtaining
  the expected identity $!!P \sim !P$.
\end{remark}

\begin{remark}\label{rem:paradoxical_combinator}
  The reader familiar with the lambda calculus will have noticed the
  similarity between $D$ and the paradoxical combinator.

  [Ed. note: the existence of this seems to suggest we have to be more
  restrictive on the set of processes and names we admit if we are to
  support no-cloning.]
\end{remark}

\subsubsection{Bisimulation}

The computational dynamics gives rise to another kind of equivalence,
the equivalence of computational behavior. As previously mentioned
this is typically captured \emph{via} some form of bisimulation.

% The notion we use in this paper is weak barbed bisimulation
% \cite{milner91polyadicpi}.

The notion we use in this paper is derived from weak barbed
bisimulation \cite{milner91polyadicpi}. 

\begin{definition}
An \emph{observation relation}, $\downarrow_{\mathcal N}$, over a set
of names, $\mathcal N$, is the smallest relation satisfying the rules
below.

\infrule[Out-barb]{y \in {\mathcal N}, \; x \nameeq y}
		  {\outputp{x}{v} \downarrow_{\mathcal N} x}
\infrule[Par-barb]{\mbox{$P\downarrow_{\mathcal N} x$ or $Q\downarrow_{\mathcal N} x$}}
		  {\binpar{P}{Q} \downarrow_{\mathcal N} x}

We write $P \Downarrow_{\mathcal N} x$ if there is $Q$ such that 
$P \wred Q$ and $Q \downarrow_{\mathcal N} x$.
\end{definition}

\begin{definition}
%\label{def.bbisim}
An  ${\mathcal N}$-\emph{barbed bisimulation} over a set of names, ${\mathcal N}$, is a symmetric binary relation 
${\mathcal S}_{\mathcal N}$ between agents such that $P\rel{S}_{\mathcal N}Q$ implies:
\begin{enumerate}
\item If $P \red P'$ then $Q \wred Q'$ and $P'\rel{S}_{\mathcal N} Q'$.
\item If $P\downarrow_{\mathcal N} x$, then $Q\Downarrow_{\mathcal N} x$.
\end{enumerate}
$P$ is ${\mathcal N}$-barbed bisimilar to $Q$, written
$P \wbbisim_{\mathcal N} Q$, if $P \rel{S}_{\mathcal N} Q$ for some ${\mathcal N}$-barbed bisimulation ${\mathcal S}_{\mathcal N}$.
\end{definition}

$\mathcal{R} \subseteq \pi \times \pi$

$P \mathcal{R} Q => \forall P'. P \red P' \Rightarrow \exists Q'. Q \red Q', P' \mathcal{R} Q'$

$P \vdash x \Rightarrow Q \vdash x$

\begin{mathpar}
  \inferrule*[lab=Out-barb]{x \nameeq y}{{y}!\langle{Q}\rangle \vdash x}
  \and
  \inferrule*[lab=Par-barb]{\mbox{$P\vdash x$ or $Q\vdash x$}}{\binpar{P}{Q} \vdash x}
\end{mathpar}

\subsubsection{Contexts}

One of the principle advantages of computational calculi like the
$\pi$-calculus is a well-defined notion of context,
contextual-equivalence and a correlation between
contextual-equivalence and notions of bisimulation. The notion of
context allows the decomposition of a process into (sub-)process and
its syntactic environment, its context. Thus, a context may be
thought of as a process with a ``hole'' (written $\Box$) in it. The
application of a context $M$ to a process $P$, written $M[P]$, is
tantamount to filling the hole in $M$ with $P$. In this paper we do
not need the full weight of this theory, but do make use of the notion
of context in the proof the main theorem. 

\begin{mathpar}
  \inferrule* [lab=summation] {} {{M_{M},M_{N}} \bc \Box \;|\; x.M_{A} \;|\; M_{M}+M_{N}}
  \and
  \inferrule* [lab=agent] {} {{M_{A}} \bc (\vec{x})M_{P} \;| \; \clift{P_0,\ldots,M_{P},\ldots,P_N}}
  \and \\
  \inferrule* [lab=process] {} {{M_{P}} \bc M_{N} \;| \;P|M_{P} }
\end{mathpar} 

\begin{mathpar}
  \inferrule* [lab=sychronization] {} {M_{N} \bc \Box \;|\; x?M_{F} \;|\; x!M_{C}}
  \and
  \inferrule* [lab=abstraction] {} {{M_{F}} \bc (x)M_{P} }
  \and
  \inferrule* [lab=concretion] {} {{M_{C}} \bc \langle M_{P} \rangle }
  \and \\
  \inferrule* [lab=process] {} {{M_{P}} \bc M_{N} \;| \;P|M_{P} }
\end{mathpar}

\begin{definition}[contextual application] Given a context $M$, and
  process $P$, we define the \emph{contextual application}, $M[P] :=
  M\{P/\Box\}$. That is, the contextual application of M to P is the
  substitution of $P$ for $\Box$ in $M$.
\end{definition}

$\meaningof{-} : L \to \mathcal{P}(\pi)$

\begin{mathpar}
  \inferrule* [lab=collection] {} {\meaningof{true} = \pi, \and \meaningof{~E} = \pi \setminus \meaningof{E}, \and \meaningof{E_{1} \& E_{2}} = \meaningof{E_{1}} \cap \meaningof{E_{2}}}
\end{mathpar}

\begin{mathpar}
  \inferrule* [lab=structure] {} {\meaningof{0} = \{ P \in \pi | P \equiv 0 \}, \and \\ \meaningof{E_1 | E_2} = \{ P \in \pi | P \equiv P_{1} | P_{2}, P_{1} \in \meaningof{E_{1}}, P_{2} \in \meaningof{E_2}\} }
\end{mathpar}

\begin{mathpar}
 \inferrule* [lab=behavior] {} {\meaningof{\langle a?b \rangle E} = \{ P \in \pi | P \equiv Q | u?(y)P', \\ \and \\\\ \and \\ \;\;\; u \in \meaningof{a}, \forall z.P'\{z/y\} \in \meaningof{E\{z/b\}}\}, \and \\ \meaningof{a!E} = \{ P \in \pi | P \equiv Q | x!\langle P' \rangle, x \in \meaningof{a} P' \in \meaningof{E}\} }
\end{mathpar}

\begin{mathpar}
 \inferrule* [lab=nominal] {} {\meaningof{\quotep{E}} = \{ \quotep{P} \in \quotep{\pi} | P \in \meaningof{E} \}, \and \meaningof{\quotep{P}} = \{ \quotep{Q} \in \quotep{\pi} | P \equiv Q \} \and \\ \meaningof{@\quotep{E}} = \{ P \in \pi | P \equiv @x, x \in \meaningof{E} \}}
\end{mathpar}

\begin{eqnarray*}
  \\
  \meaningof{-} : TS \to ST
\end{eqnarray*}

\begin{eqnarray*}
  \\
  L : TS \to ST
\end{eqnarray*}

\begin{eqnarray*}
  \\
  P \models E \iff P \in \meaningof{E}
\end{eqnarray*}

\begin{eqnarray*}
  P \approx_{L} Q \iff \forall E \in L. P \models E \iff Q \models E
\end{eqnarray*}

\begin{eqnarray*}
  P \approx_{K} Q
\end{eqnarray*}

\begin{eqnarray*}
  P \approx Q
\end{eqnarray*}

$\approx_{K} = \approx = \approx_{L}$

\subsubsection{Contextual duality}

Note that contexts extend the quotation operation to a family of
operations from processes to names. Given a context, $M$, we can
define a \emph{nominal context}, $\quotep{M}$ by $\quotep{M}[P] :=
\quotep{M[P]}$. To foreshadow what is to come we observe that these
operations enjoy a duality with processes very much like the duality
between vectors and maps from vectors to scalars.

Further, because the calculus is essentially higher-order, we have a
correspondence between contexts and processes. More specifically,
given a name $x$ and a context $M$ we can construct $M^{*}_{x}$ such
that 

\begin{mathpar}
  M^{*}_{x} | \lift{x}{P} \red M[P]
\end{mathpar}

namely,

\begin{mathpar}
  M^{*}_{x} := x?(u).M[\dropn{u}]
\end{mathpar}

The dependence of $M^{*}_{x}$ on a name makes it an abstraction, 

\begin{mathpar}
  M^{*} := (x)x?(u).M[\dropn{u}]
\end{mathpar}

\subsection{Additional notation}

It will sometimes be convenient to denote the process a name
quotes. We already have the notation $x = \quotep{P}$, but it will be
convenient to introduce an alternate notation, $\procn{x}$, when we
want to emphasize the connection to the use of the name. Note that, by
virtue of name equivalence, $\quotep{\procn{x}} \nameeq x$; so, the
notation is consistent with previous definitions.

Further, because names have structure it is possible to effect
substitutions on the basis of that structure. This means we need to
upgrade our notation for substitutions, which we accomplish by
adapting comprehension notation. Thus,

\begin{mathpar}
  P\{ y / x : x \in S \}
\end{mathpar}

is interpreted to mean the process derived from P by replacing (in a
capture-avoiding manner) each occurrence of $x$ in $S$ by $y$. For example,

\begin{mathpar}
  P\{ \quotep{\procn{x}|\procn{x}} / x : x \in \freenames{P} \}
\end{mathpar}

will replace each (occurrence) of a free name $x$ in $P$ by
$\quotep{\procn{x}|\procn{x}}$.

Also, we will avail ourselves of the notation $x^{L}$ and $x^{R}$ to
denote injections of a name into disjoint copies of the name
space. There are numerous ways to accomplish this. One example can be
found in \cite{MeredithR05}. This notation overloads to vectors of
names: $\vec{x}^{\pi} := (x_{i}^{\pi} \; : \; 0 \leq i < |\vec{x}| )$ where $\pi \in \{L,R\}$.

We also use $P^{\Box} := P|\Box$.

In \cite{MeredithR05} an interpretation of the new operator is
given. It turns out that there are several possible interpretations
all enjoying the requisite algebraic properties of the operator (see
\cite{milner91polyadicpi}). We will therefore make liberal use of
$(\nu\; \vec{x})P$.

% subsection the_syntax_and_semantics_of_the_notation_system (end)   

\input{qm2pi.qmops} 

\input{qm2pi.sterngerlach} 

\input{qm2pi.metric} 

% section concurrent_process_calculi (end)

%\input{qm2pi.proofsketch}

% section proof sketch (end)

%\input{qm2pi.slviaknots} 

% section spatial logic via knots (end)

\input{qm2pi.conclusion}

% section conclusion (end)

%\input{qm2pi.dtcodes} 

% section wiring algorithm (end)

\input{qm2pi.ack} 

% section acknowledgments (end)

\newpage


\bibliographystyle{plain}   
\bibliography{../../biblios/main.bib}

\input{qm2pi.rhodetails}

\end{document}

 

% section concurrent_process_calculi (end)

%\documentclass[12pt]{llncs}
%\documentclass{jktr}

\usepackage[pdftex]{hyperref}                   
\usepackage {listings}
\usepackage {mathpartir}
\usepackage{bcprules}
%\usepackage{listings}
                       
\usepackage{graphicx} 
%\usepackage[margins=2.5cm,nohead,nofoot]{geometry}
%\usepackage{geometry}
\usepackage{amsfonts}
\usepackage{amstext}
\usepackage{latexsym}
\usepackage{amssymb}
\usepackage{color}


%\include{myPreamble}
\include{qm2pi.local} 

%\ifpdf
%\usepackage[pdftex]{graphicx}
%\else
%\usepackage{graphicx}
%\fi

 % \ifpdf
%  \usepackage{pdfsync}
%  \if


%\title{Brief Article}
%\author{David F. Snyder}
%\author{L.G. Meredith}

%\address{Dept. of Math., Texas State University--San Marcos, San Marcos, TX 78666}
       
\pagestyle{empty}


\begin{document}

\lstset{language=[Objective]Caml,frame=shadowbox}

\input{qm2pi.front}

% section front matter (end)

\input{qm2pi.intro} 
 
% section introduction (end)

% \input{qm2pi.knotations} 

% section notation (end)

\input{qm2pi.process.calculi} 

% section concurrent_process_calculi_and_spatial_logics_ (end)
    
%\input{qm2pi.knots2pi} 

%\input{qm2pi.trefoil} 

%\input{qm2pi.mainthm} 

% subsection basic_interpretation (end)

%\input{qm2pi.rho.presentation} 
\subsection{The syntax and semantics of the notation system}\label{sub:the_syntax_and_semantics_of_the_notation_system} % (fold)

We now summarize a technical presentation of the calculus that
embodies our theory of dynamics. The typical presentation of such a
calculus follows the style of giving generators and relations on
them. The grammar, below, describing term constructors, freely
generates the set of processes, $\Proc$. This set is then quotiented
by a relation known as structural congruence and it is over this set
that the notion of dynamics is expressed. This presentation is
essentially that of \cite{MeredithR05} with the addition of
polyadicity and summation. For readability we have relegated some of
the technical subtleties to an appendix.

\subsubsection{Process grammar}\label{subsub:process_grammar}

\begin{mathpar}
  \inferrule* [lab=synchronization] {} {{M} \bc \pzero \;|\; x?F \;|\; x!C }
  \and
  \inferrule* [lab=abstraction] {} {{F} \bc (x)P}
  \and
  \inferrule* [lab=concretion] {} {{C} \bc \langle Q \rangle}
  \and
  \inferrule* [lab=process] {} {{P,Q} \bc M \;| \;P|Q \;|\; @{x}}
  \and
  \inferrule* [lab=name] {} {{x} \bc \quotep{P}}
\end{mathpar} 

Note that $\vec{x}$ (resp. $\vec{P}$) denotes a vector of names
(resp. processes) of length $|\vec{x}|$ (resp. $|\vec{P}|$). We adopt
the following useful abbreviations.

\begin{mathpar}
   x?(\vec{y}).P := x.(\vec{y})P \and  x\clift{\vec{P}} := x.\clift{\vec{P}}
   \and x!(y) := \lift{x}{\dropn{y}}
   \and \Pi_{i=0}^{n-1}P_i := P_0 | \ldots | P_{n-1}
\end{mathpar}

\subsubsection{Structural congruence}

\paragraph{Free and bound names and alpha-equivalence.} At the
core of structural equivalence is alpha-equivalence which identifies
process that are the same up to a change of variable. Formally, we
recognize the distinction between free and bound names. The free names
of a process, $\freenames{P}$, may be calculated recursively as
follows:

\begin{mathpar}
\freenames{\pzero} := \emptyset
  \and \\
  \freenames{x?(y).P} := \{ x \} \cup (\freenames{P} \setminus \{ y \})
  \and 
  \freenames{x!\langle P \rangle} := \{ x \} \cup \{ P \} 
  \and \\
  \freenames{P|Q} := \freenames{P} \cup \freenames{Q}
  \and \\
  \freenames{@{x}} := \{ x \}
\end{mathpar}

$\pi$
$\quotep{\pi}$

$\freenames{-} : \pi \to \mathcal{P}(\quotep{\pi})$

\begin{eqnarray*}
  \freenames{\pzero} & := & \emptyset \\
  \freenames{x?(y).P} & := & \{ x \} \cup (\freenames{P} \setminus \{ y \}) \\
  \freenames{x!\langle P \rangle} & := & \{ x \} \cup \{ P \} \\
  \freenames{P|Q} & := & \freenames{P} \cup \freenames{Q} \\
  \freenames{\dropn{x}} & := & \{ x \}
\end{eqnarray*}

The bound names of a process, $\boundnames{P}$, are those names occurring in $P$
that are not free. For example, in $x?(y).0$, the name $x$ is free, while $y$ is bound.

\begin{mathpar}
  \inferrule* [lab=monoidal-laws] {} { P|Q \equiv Q|P \and P|0 \equiv P \and P|(Q|R) \equiv (P|Q)|R }
\end{mathpar}

\begin{mathpar}
  \inferrule* [lab=alpha-equivalence] {} { (x)P \equiv (y)P\{y/x\} \and y \not\in \freenames{P} }
\end{mathpar}

\begin{definition}
Then two processes, $P,Q$, are alpha-equivalent if $P = Q\{\vec{y}/\vec{x}\}$ for
some $\vec{x} \in \boundnames{Q},\vec{y} \in \boundnames{P}$, where $Q\{\vec{y}/\vec{x}\}$
denotes the capture-avoiding substitution of $\vec{y}$ for $\vec{x}$ in $Q$.
\end{definition}

\begin{definition}
  The {\em structural congruence} \cite{SangiorgiWalker} , $\equiv$,
  between processes is the least congruence containing
  alpha-equivalence, satisfying the abelian monoid laws
  (associativity, commutativity and $\pzero$ as identity) for parallel
  composition $|$ and for summation $+$.
\end{definition}

\subsection{Name equivalence}

We take name equivalence, written $\nameeq$, to be the smallest
equivalence relation generated by the following rules.

\begin{mathpar}
\inferrule*[lab=Quote-drop]
{ }
{ \quotep{@{x}} \nameeq x }

\inferrule*[lab=Struct-equiv]
{ P \scong Q }
{ \quotep{P} \nameeq \quotep{Q} }
\end{mathpar}

The astute reader will have noticed that the mutual recursion of names
and processes imposes a mutual recursion on alpha-equivalence and
structural equivalence via name-equivalence. Fortunately, all of this
works out pleasantly and we may calculate in the natural way, free of
concern. The reader interested in the details is referred to the
appendix \ref{appendix:rho_details}.

\subsection{Substitution}

We use $\Proc$ for the set of processes, $\QProc$ for the set of
names, and $\id{\{}\vec{y} / \vec{x} \id{\}}$ to denote partial maps,
$s : \QProc \rightarrow \QProc$. A map, $s$ lifts, uniquely, to a map
on process terms, $\widehat{s} : \Proc \rightarrow \Proc$ by the
following equations.

\begin{mathpar}
  (0) \psubstp{Q}{P} := 0 \\
  (R \juxtap S) \psubstp{Q}{P}
  :=    
  (R)\psubstp{Q}{P} \juxtap (S) \psubstp{Q}{P} \\
  (x?(y).R) \psubstp{Q}{P}    
  :=    
  (x)\substp{Q}{P} (z)\concat( (R \psubstn{z}{y}) \psubstp{Q}{P} ) \\
  (\lift{x}{R}) \psubstp{Q}{P}  
  :=
  \lift{(x)\substp{Q}{P}}{ R \psubstp{Q}{P} } \\
%   (\dropn{x})  \psubstp{Q}{P}       
%   := 
%   \left\{ 
%     \begin{array}{ccc} 
%       \dropn{\quotep{Q}} & & x \nameeq \quotep{P} \\
%       \dropn{x} & & otherwise \\
%     \end{array}
%   \right. 
  (\dropn{x})  \psubstp{Q}{P}       
  := 
  \left\{ 
    \begin{array}{ccc} 
      Q & & x \nameeq \quotep{P} \\
      \dropn{x} & & otherwise \\
    \end{array}
  \right.
\end{mathpar}
 

where

\begin{eqnarray}
  (x)\id{\{} \lpquote Q \rpquote / \lpquote P \rpquote \id{\}}            = 
  \left\{ 
    \begin{array}{ccc}
      \lpquote Q \rpquote & & x \nameeq \lpquote P \rpquote \\
      x & & otherwise \\
    \end{array}
  \right. \nonumber
\end{eqnarray}

and $z$ is chosen distinct from $\quotep{P}$, $\quotep{Q}$, the free
names in $Q$, and all the names in $R$. Our $\alpha$-equivalence will
be built in the standard way from this substitution.

\begin{remark}\label{rem:no_self_referential_names}
  One consequence of these definitions is that $\forall P. \quotep{P}
  \not\in \freenames{P}$.
\end{remark}

\subsection{ Dynamic quote: an example }

Anticipating something of what's to come, consider applying the
substitution, $\widehat{\id{\{}u / z \id{\}}}$, to the following pair
of processes, $\lift{w}{y!(z)}$ and $w[ \lpquote y!(z) \rpquote ]$.

\begin{eqnarray}
	\lift{w}{y!(z)}\widehat{\id{\{}u / z \id{\}}}
		& = &
		\lift{w}{y!(u)} \nonumber\\
	w[ \lpquote y!(z) \rpquote ] \widehat{ \id{\{}u / z \id{\}} }
		& = &
		w[ \lpquote y!(z) \rpquote ] \nonumber
\end{eqnarray}

Because the body of the process between quotes is impervious to
substitution, we get radically different answers. In fact, by
examining the first process in an input context,
e.g. $x?(z).\lift{w}{y!(z)}$, we see that the process under the lift
operator may be shaped by prefixed inputs binding a name inside it. In
this sense, the lift operator will be seen as a way to dynamically
construct processes before reifying them as names.

Finally equipped with these standard features we can present the
dynamics of the calculus.

\subsubsection{Operational semantics} 

Finally, we introduce the computational dynamics. What marks these
algebras as distinct from other more traditionally studied algebraic
structures, e.g. vector spaces or polynomial rings, is the manner in
which dynamics is captured. In traditional structures, dynamics is typically
expressed through morphisms between such structures, as in linear maps
between vector spaces or morphisms between rings. In algebras
associated with the semantics of computation, the dynamics is
expressed as part of the algebraic structure itself, through a
reduction reduction relation typically denoted by $\red$. Below, we
give a recursive presentation of this relation for the calculus used
in the encoding.

$\red \subseteq \pi \times \pi$
$\red : \pi \to \mathcal{P}(\pi)$

\begin{mathpar}
  \inferrule* [lab=Comm] { \textsf{match}( x_{src}, x_{trgt} ) } { x_{trgt}?(y)P \; | \; x_{src}!\langle {Q} \rangle \red P\{\quotep{Q}/y}\} }
  \and \\
  \inferrule* [lab=Par] {{P} \red {P}'} {{{P} | {Q}} \red {{P}' | {Q}}}
  \and
  \inferrule* [lab=Equiv]{{{P} \scong {P}'} \andalso {{P}' \red {Q}'} \andalso {{Q}' \scong {Q}}}{{P} \red {Q}}
\end{mathpar}

\begin{eqnarray*}
  match_{\equiv} (\quotep{P},\quotep{Q}) & := & P \equiv Q \\
  match_{\dagger}(\quotep{P},\quotep{Q}) & := & \forall R. P|Q \red^{*} R => R \red^{*} 0 \\
  match_{K}(\quotep{P},\quotep{Q}) & := & K \mbox{ for some context } K
\end{eqnarray*}

$u?(x)P | u!\langle Q \rangle \red P\{\quotep{Q}/x\}$

%We write $\wred$ for $\red^*$, and $P\red$ if $\exists Q $ such that $ P \red Q$.
We write $P\red$ if $\exists Q $ such that $ P \red Q$ and $P\not\red$, otherwise.

\section{Replication}

As mentioned before, it is known that replication (and hence
recursion) can be implemented in a higher-order process algebra
\cite{SangiorgiWalker}. As our first example of calculation with the
machinery thus far presented we give the construction explicitly in
the {\rhoc}.

\begin{eqnarray}
	D_{x} & := & \prefix{x}{y}{(\binpar{\outputp{x}{y}}{@{y}})} \nonumber\\
	\bangp_{x}{P} & := & \binpar{{x}!\langle{\binpar{D_{x}}{P}}\rangle}{D_{x}} \nonumber
\end{eqnarray}

\begin{eqnarray}
	\bangp_{x}{P} & & \nonumber\\
	=
	& {x}!\langle{(\prefix{x}{y}{(\outputp{x}{y} | @{y})) | P}}\rangle 
	      | \prefix{x}{y}{(\outputp{x}{y} | @{y})} & \nonumber\\
	\red
	& (\outputp{x}{y} | @{y})\substn{\quotep{(\prefix{x}{y}{(@{y} | \outputp{x}{y})) | P}}}{y} & \nonumber\\
	=
	& \outputp{x}{\quotep{(\prefix{x}{y}{(\outputp{x}{y} | @{y})) | P}}}
	  | {(\prefix{x}{y}{(\outputp{x}{y} | @{y})) | P}} & \nonumber\\
	\red
	& \ldots & \nonumber\\
	\red^*
	& P | P | \ldots & \nonumber
\end{eqnarray}

Of course, this encoding, as an implementation, runs away, unfolding
$\bangp{P}$ eagerly. A lazier and more implementable replication
operator, restricted to input-guarded processes, may be obtained as follows.

\begin{eqnarray}
\bangp{\prefix{u}{v}{P}} 
	:= 
	\binpar{\lift{x}{\prefix{u}{v}{(\binpar{D(x)}{P})}}}{D(x)} \nonumber
\end{eqnarray}

\begin{remark}
  Note that the lazier definition still does not deal with summation
  or mixed summation (i.e. sums over input and output). The reader is
  invited to construct definitions of replication that deal with these
  features. 

  Further, the definitions are parameterized in a name, $x$. Can you,
  gentle reader, make a definition that eliminates this parameter and
  guarantees no accidental interaction between the replication
  machinery and the process being replicated -- i.e. no accidental
  sharing of names used by the process to get its work done and the
  name(s) used by the replication to effect copying. This latter
  revision of the definition of replication is crucial to obtaining
  the expected identity $!!P \sim !P$.
\end{remark}

\begin{remark}\label{rem:paradoxical_combinator}
  The reader familiar with the lambda calculus will have noticed the
  similarity between $D$ and the paradoxical combinator.

  [Ed. note: the existence of this seems to suggest we have to be more
  restrictive on the set of processes and names we admit if we are to
  support no-cloning.]
\end{remark}

\subsubsection{Bisimulation}

The computational dynamics gives rise to another kind of equivalence,
the equivalence of computational behavior. As previously mentioned
this is typically captured \emph{via} some form of bisimulation.

% The notion we use in this paper is weak barbed bisimulation
% \cite{milner91polyadicpi}.

The notion we use in this paper is derived from weak barbed
bisimulation \cite{milner91polyadicpi}. 

\begin{definition}
An \emph{observation relation}, $\downarrow_{\mathcal N}$, over a set
of names, $\mathcal N$, is the smallest relation satisfying the rules
below.

\infrule[Out-barb]{y \in {\mathcal N}, \; x \nameeq y}
		  {\outputp{x}{v} \downarrow_{\mathcal N} x}
\infrule[Par-barb]{\mbox{$P\downarrow_{\mathcal N} x$ or $Q\downarrow_{\mathcal N} x$}}
		  {\binpar{P}{Q} \downarrow_{\mathcal N} x}

We write $P \Downarrow_{\mathcal N} x$ if there is $Q$ such that 
$P \wred Q$ and $Q \downarrow_{\mathcal N} x$.
\end{definition}

\begin{definition}
%\label{def.bbisim}
An  ${\mathcal N}$-\emph{barbed bisimulation} over a set of names, ${\mathcal N}$, is a symmetric binary relation 
${\mathcal S}_{\mathcal N}$ between agents such that $P\rel{S}_{\mathcal N}Q$ implies:
\begin{enumerate}
\item If $P \red P'$ then $Q \wred Q'$ and $P'\rel{S}_{\mathcal N} Q'$.
\item If $P\downarrow_{\mathcal N} x$, then $Q\Downarrow_{\mathcal N} x$.
\end{enumerate}
$P$ is ${\mathcal N}$-barbed bisimilar to $Q$, written
$P \wbbisim_{\mathcal N} Q$, if $P \rel{S}_{\mathcal N} Q$ for some ${\mathcal N}$-barbed bisimulation ${\mathcal S}_{\mathcal N}$.
\end{definition}

$\mathcal{R} \subseteq \pi \times \pi$

$P \mathcal{R} Q => \forall P'. P \red P' \Rightarrow \exists Q'. Q \red Q', P' \mathcal{R} Q'$

$P \vdash x \Rightarrow Q \vdash x$

\begin{mathpar}
  \inferrule*[lab=Out-barb]{x \nameeq y}{{y}!\langle{Q}\rangle \vdash x}
  \and
  \inferrule*[lab=Par-barb]{\mbox{$P\vdash x$ or $Q\vdash x$}}{\binpar{P}{Q} \vdash x}
\end{mathpar}

\subsubsection{Contexts}

One of the principle advantages of computational calculi like the
$\pi$-calculus is a well-defined notion of context,
contextual-equivalence and a correlation between
contextual-equivalence and notions of bisimulation. The notion of
context allows the decomposition of a process into (sub-)process and
its syntactic environment, its context. Thus, a context may be
thought of as a process with a ``hole'' (written $\Box$) in it. The
application of a context $M$ to a process $P$, written $M[P]$, is
tantamount to filling the hole in $M$ with $P$. In this paper we do
not need the full weight of this theory, but do make use of the notion
of context in the proof the main theorem. 

\begin{mathpar}
  \inferrule* [lab=summation] {} {{M_{M},M_{N}} \bc \Box \;|\; x.M_{A} \;|\; M_{M}+M_{N}}
  \and
  \inferrule* [lab=agent] {} {{M_{A}} \bc (\vec{x})M_{P} \;| \; \clift{P_0,\ldots,M_{P},\ldots,P_N}}
  \and \\
  \inferrule* [lab=process] {} {{M_{P}} \bc M_{N} \;| \;P|M_{P} }
\end{mathpar} 

\begin{mathpar}
  \inferrule* [lab=sychronization] {} {M_{N} \bc \Box \;|\; x?M_{F} \;|\; x!M_{C}}
  \and
  \inferrule* [lab=abstraction] {} {{M_{F}} \bc (x)M_{P} }
  \and
  \inferrule* [lab=concretion] {} {{M_{C}} \bc \langle M_{P} \rangle }
  \and \\
  \inferrule* [lab=process] {} {{M_{P}} \bc M_{N} \;| \;P|M_{P} }
\end{mathpar}

\begin{definition}[contextual application] Given a context $M$, and
  process $P$, we define the \emph{contextual application}, $M[P] :=
  M\{P/\Box\}$. That is, the contextual application of M to P is the
  substitution of $P$ for $\Box$ in $M$.
\end{definition}

$\meaningof{-} : L \to \mathcal{P}(\pi)$

\begin{mathpar}
  \inferrule* [lab=collection] {} {\meaningof{true} = \pi, \and \meaningof{~E} = \pi \setminus \meaningof{E}, \and \meaningof{E_{1} \& E_{2}} = \meaningof{E_{1}} \cap \meaningof{E_{2}}}
\end{mathpar}

\begin{mathpar}
  \inferrule* [lab=structure] {} {\meaningof{0} = \{ P \in \pi | P \equiv 0 \}, \and \\ \meaningof{E_1 | E_2} = \{ P \in \pi | P \equiv P_{1} | P_{2}, P_{1} \in \meaningof{E_{1}}, P_{2} \in \meaningof{E_2}\} }
\end{mathpar}

\begin{mathpar}
 \inferrule* [lab=behavior] {} {\meaningof{\langle a?b \rangle E} = \{ P \in \pi | P \equiv Q | u?(y)P', \\ \and \\\\ \and \\ \;\;\; u \in \meaningof{a}, \forall z.P'\{z/y\} \in \meaningof{E\{z/b\}}\}, \and \\ \meaningof{a!E} = \{ P \in \pi | P \equiv Q | x!\langle P' \rangle, x \in \meaningof{a} P' \in \meaningof{E}\} }
\end{mathpar}

\begin{mathpar}
 \inferrule* [lab=nominal] {} {\meaningof{\quotep{E}} = \{ \quotep{P} \in \quotep{\pi} | P \in \meaningof{E} \}, \and \meaningof{\quotep{P}} = \{ \quotep{Q} \in \quotep{\pi} | P \equiv Q \} \and \\ \meaningof{@\quotep{E}} = \{ P \in \pi | P \equiv @x, x \in \meaningof{E} \}}
\end{mathpar}

\begin{eqnarray*}
  \\
  \meaningof{-} : TS \to ST
\end{eqnarray*}

\begin{eqnarray*}
  \\
  L : TS \to ST
\end{eqnarray*}

\begin{eqnarray*}
  \\
  P \models E \iff P \in \meaningof{E}
\end{eqnarray*}

\begin{eqnarray*}
  P \approx_{L} Q \iff \forall E \in L. P \models E \iff Q \models E
\end{eqnarray*}

\begin{eqnarray*}
  P \approx_{K} Q
\end{eqnarray*}

\begin{eqnarray*}
  P \approx Q
\end{eqnarray*}

$\approx_{K} = \approx = \approx_{L}$

\subsubsection{Contextual duality}

Note that contexts extend the quotation operation to a family of
operations from processes to names. Given a context, $M$, we can
define a \emph{nominal context}, $\quotep{M}$ by $\quotep{M}[P] :=
\quotep{M[P]}$. To foreshadow what is to come we observe that these
operations enjoy a duality with processes very much like the duality
between vectors and maps from vectors to scalars.

Further, because the calculus is essentially higher-order, we have a
correspondence between contexts and processes. More specifically,
given a name $x$ and a context $M$ we can construct $M^{*}_{x}$ such
that 

\begin{mathpar}
  M^{*}_{x} | \lift{x}{P} \red M[P]
\end{mathpar}

namely,

\begin{mathpar}
  M^{*}_{x} := x?(u).M[\dropn{u}]
\end{mathpar}

The dependence of $M^{*}_{x}$ on a name makes it an abstraction, 

\begin{mathpar}
  M^{*} := (x)x?(u).M[\dropn{u}]
\end{mathpar}

\subsection{Additional notation}

It will sometimes be convenient to denote the process a name
quotes. We already have the notation $x = \quotep{P}$, but it will be
convenient to introduce an alternate notation, $\procn{x}$, when we
want to emphasize the connection to the use of the name. Note that, by
virtue of name equivalence, $\quotep{\procn{x}} \nameeq x$; so, the
notation is consistent with previous definitions.

Further, because names have structure it is possible to effect
substitutions on the basis of that structure. This means we need to
upgrade our notation for substitutions, which we accomplish by
adapting comprehension notation. Thus,

\begin{mathpar}
  P\{ y / x : x \in S \}
\end{mathpar}

is interpreted to mean the process derived from P by replacing (in a
capture-avoiding manner) each occurrence of $x$ in $S$ by $y$. For example,

\begin{mathpar}
  P\{ \quotep{\procn{x}|\procn{x}} / x : x \in \freenames{P} \}
\end{mathpar}

will replace each (occurrence) of a free name $x$ in $P$ by
$\quotep{\procn{x}|\procn{x}}$.

Also, we will avail ourselves of the notation $x^{L}$ and $x^{R}$ to
denote injections of a name into disjoint copies of the name
space. There are numerous ways to accomplish this. One example can be
found in \cite{MeredithR05}. This notation overloads to vectors of
names: $\vec{x}^{\pi} := (x_{i}^{\pi} \; : \; 0 \leq i < |\vec{x}| )$ where $\pi \in \{L,R\}$.

We also use $P^{\Box} := P|\Box$.

In \cite{MeredithR05} an interpretation of the new operator is
given. It turns out that there are several possible interpretations
all enjoying the requisite algebraic properties of the operator (see
\cite{milner91polyadicpi}). We will therefore make liberal use of
$(\nu\; \vec{x})P$.

% subsection the_syntax_and_semantics_of_the_notation_system (end)   

\input{qm2pi.qmops} 

\input{qm2pi.sterngerlach} 

\input{qm2pi.metric} 

% section concurrent_process_calculi (end)

%\input{qm2pi.proofsketch}

% section proof sketch (end)

%\input{qm2pi.slviaknots} 

% section spatial logic via knots (end)

\input{qm2pi.conclusion}

% section conclusion (end)

%\input{qm2pi.dtcodes} 

% section wiring algorithm (end)

\input{qm2pi.ack} 

% section acknowledgments (end)

\newpage


\bibliographystyle{plain}   
\bibliography{../../biblios/main.bib}

\input{qm2pi.rhodetails}

\end{document}



% section proof sketch (end)

%\section{Unlikely characters: spatial logic for
  knots}\label{sub:characteristic_formulae} % (fold)

Associated to the mobile process calculi are a family of logics known
as the Hennessy-Milner logics. These logics typically enjoy a
semantics interpreting formulae as sets of processes that when
factored through the encoding outlined above allows an identification
of classes of knots with logical formulae. In the context of this
encoding the sub-family known as the spatial logics \cite{CairesC03}
\cite{CairesC04} \cite{Caires04} are of particular interest providing
several important features for expressing and reasoning about
properties (i.e. classes) of knots. We hint here at how this may be done.

%\begin{description}
%\item [structural connectives] 
\subsubsection{Structural connectives} The spatial logics enjoy
structural connectives corresponding, at the logical level, to the
parallel composition ($P | Q$) and new name ($(\nu \; x)P$)
connectives for processes. As illustrated in the examples below, these
connectives are extremely expressive given the shape of our encoding.
%\item [decideable satisfaction]

\subsubsection{Decideable satisfaction}
In \cite{Caires04} the satisfaction relation is shown to be decideable
for a rich class of processes. It further turns out that the image of
the our encoding is a proper subset of that class. This result
provides the basis for an algorithm by which to search for knots
enjoying a given property.
%\item [characteristic formulae]

\subsubsection{Characteristic formulae}
In the same paper \cite{Caires04} , Caires presents a means of calculating
characteristic formulae, selecting equivalence classes of processes
up to a pre--specified depth limit on the support set of names. Composed with our
encoding, this characteristic formula can be used to select
characteristic formulae for knots.
%\end{description}

\subsubsection{Spatial logic formulae}

The grammar below (segmented for comprehension) summarizes the syntax
of spatial logic formulae. We employ illustrative examples in the
sequel to provide an intuitive understanding of their meaning
referring the reader to \cite{Caires04} for a more detailed explication
of the semantics.

\begin{mathpar}
  \inferrule* [lab=boolean] {} {{A,B} \bc T \;|\; \neg A \;|\; A \wedge B \;|\; \eta = \eta'}
  \and
  \inferrule* [lab=spatial] {} {|\; \pzero \;|\; A | B \;|\; x \text{\textregistered} A \;|\; \forall x . A \;|\;  H x . A}
  \and
  \inferrule* [lab=behavioral] {} {|\; \alpha . A}
  \and 
  \inferrule* [lab=recursion] {} {|\; X(\vec{u}) \;|\; \mu X(\vec{u}) . A}
  \and
  \inferrule* [lab=action] {} {\alpha \bc \langle x?(\vec{y}) \rangle \;|\; \langle x!(\vec{y}) \rangle \;|\; \langle \tau \rangle}
  \and 
  \inferrule* [lab=name] {} {\eta \bc x \;|\; \tau}
\end{mathpar} 

% subsection characteristic_formulae (end)   	 

\subsection{Example formulae}\label{sub:example_formulae_} % (fold)

\subsubsection{Crossing as formula.}
% 
% \begin{align*}
%   \frac{d}{dx} \sin x &= \cos x 
%   & \frac{d}{dx} e^x &= e^x \\
%   \frac{d}{dx} \cos x &= - \sin x 
%   & \frac{d}{dx} \log x &= \frac{1}{x} \\
% \end{align*} 

\begin{align*}
 \mu C(x_{0},x_{1},y_{0},y_{1},u).&(\langle x_{0}?(z) \rangle(\langle u! \rangle\langle y_{1}!z \rangle C(x_{0},x_{1},y_{0},y_{1},u)) & \\
  & \wedge \langle y_{1}?(z) \rangle (\langle u! \rangle \langle x_{0}!z \rangle C(x_{0},x_{1},y_{0},y_{1},u)) & \\
  & \wedge \langle x_{1}?(z) \rangle (\langle u? \rangle \langle y_{0}!z \rangle C(x_{0},x_{1},y_{0},y_{1},u)) & \\
  & \wedge \langle y_{0}?(z) \rangle (\langle u? \rangle \langle x_{1}!z \rangle C(x_{0},x_{1},y_{0},y_{1},u))) &
\end{align*}

The lexicographical similarity between the shape of this formulae and
the shape of definition of the process representing a crossing reveals
the intuitive meaning of this formulae. It describes the capabilities
of a process that has the right to represent a crossing. For example
it picks out processes that may perform an input on the port $x_0$ in
its initial menu of capabilities. What differentiates the formula
from the process, however, is that the crossing process is the
smallest candidate to satisfy the formula. Infinitely many other
processes -- with internal behavior hidden behind this interface, so
to speak -- also satisfy this formula. Even this simple formula,
then, can be seen to open a new view onto knots, providing a
computational interpretation of \emph{virtual} knots.

Note that this formula is derived by hand. A similar formula can be
derived by employing Caires' calculation of characteristic formula
\cite{Caires04} to the process representing a crossing. In light of
this discussion, we let
$\meaningof{C}_{\phi}(x0,x1,y0,y1,u)$ denote a formula specifying the
dynamics we wish to capture of a crossing. To guarantee we preserve
the shape of the interface and minimal semantics we demand that
$\meaningof{C}_{\phi}(x0,x1,y0,y1,u) \Rightarrow
\textbf{C}(x0,x1,y0,y1,u)$ where $\textbf{C}(x0,x1,y0,y1,u)$ denotes
the formula above.
                            
\subsubsection{Crossing number constraints.}
The moral content of the context lemma (Lemma \ref{context}) is that the notion of
``locality'' in the Reidemeister moves is effectively captured by the
parallel composition operator of the process calculus. This intuition
extends through the logic. Given a formula,
$\meaningof{C}_{\phi}(x0,x1,y0,y1,u)$, we can use the structural
connectives to specify constraints on crossing numbers, such as at
least $n$ crossings, or exactly $n$ crossings.
\begin{mathpar}
  \inferrule* [lab=at-least-n] {} { K^{\geq n}_{\phi}(\vec{xs},\vec{ys}) := \Pi_{i=0}^{n-1} Hu . \meaningof{C}_{\phi}(xs_i,ys_i,u) | T }
  \and 
  \inferrule* [lab=exactly-n] {} { K^{= n}_{\phi}(\vec{xs},\vec{ys}) := \Pi_{i=0}^{n-1} Hu . \meaningof{C}_{\phi}(xs_i,ys_i,u) | \neg (\forall x_0,y_0,x_1,y_1,u . \meaningof{C}_{\phi}(x_0,y_0,x_1,y_1,u) | T) }
\end{mathpar}

To round out this section, recall that the encoding of an $n$-crossing
knot decomposes into a parallel composition of $n$ \emph{copies} of a
crossing process together with a wiring harness. To specify different
knot classes with the same crossing number amounts to specifying
logical constraints on the wiring harness. In the interest of space,
we defer examples to a forthcoming paper. Suffice it to say that both
the conditions ``alternating knot'' and ``contains the tangle
corresponding to 5/3'' are expressible. For example, it is possible to
calculate the characteristic formula of a process corresponding to the
tangle 5/3 and conjoin it into the classifying formula via the
composition connective of the logic.

Finally, we wish to observe that it is entirely within reason to
contemplate a more domain-specific version of spatial logic tailored
to the shape of processes in the image of the encoding. Such a
domain-specific logic would have a better claim to the title formal
language of knot properties.

% subsection example_formulae_ (end)

% section knots_as_processes (end) 

% section spatial logic via knots (end)

\section{Conclusions and future work}

\paragraph{Testing physical space}
You, gentle reader, may wonder why of all the theorems to be proved
given this set up we pick the one above. In some sense it's hardly
central to quantum mechanics. We see it as central in the sense that
it firmly establishes a notion of physical space arising from a notion
of the equivalence of behavior. Relating bisimulation to a metric is a
big step forward, but one is faced with interpreting the relationship
of that metric space to something more physical. Quantum mechanical
notions of ``physical'' space are still far from intuitive, but by
relating this idea of distance as testing to calculations that predict
physical circumstances we are making a not insignificant step forward
toward an understanding of the physical space we inhabit as
essentially dynamic.

\paragraph{Effectivity and simulation}
One of the observations we have yet to make is that the entire program
spelled out here is effective. We have built various interpreters for
the reflective calculus at work in this interpretation. In principle,
then, we can simulate quantum mechanics on a computer. The place where
the simulation may lose fidelity is the infinitely branching summation
for the annihilator.

In this connection i also want to point out that the evaluation style
calculation of the inner product puts the non-determinism of the
summation right at the heart of measurement. This suggests that
Milner's original reduction-based formulation of the dynamics of his
calculi in terms of sums was not just notationally suggestive of a
notion of measure-and-continue but captured some significant part of
the physics.

\paragraph{Quantum continuations}
In light of this last observation i want to point out that the
predominant account of quantum mechanics is missing a key aspect of a
truly compositional story of the physical situation. In a real lab,
when a measurement is made the observation can be made to feed into
another device that then makes another measurement conditioned on the
results of the first. This means that after the superposition was
collapsed the entire experimental set up remained in
superposition. While QM offers a means of writing this down it doesn't
quite line up well with the well-trodden formulation of computation
and continuation that we see so succinctly expressed in Milner's
calculi. This suggests that there might be advantages to this account
of dynamics waiting to be explored.

\paragraph{Quantum logic}
In this connection, we also note that by virtue of having the
Hennessy-Milner construction, we can pull the construction through the
interpretation of QM. This gives us a natural candidate for a quantum
logic that enjoys an extremely tight connection with it's domain of
interpretation, making the construction much less ad hoc (rather it is
the image of functor!).

\paragraph{Quantum probabiity}
i have questions about the basis of the interpretation of inner
product as probability amplitude. In particular, using which
axiomatization of probability theory does the notion of probability
amplitude earn the right to be so dubbed? In other words, where is the
proof that the operation for calculating a probability amplitude (and
then squaring) satisfies the axioms of what it means to calculate a
probability? Even if such a proof exists (i have yet to find it in the
literature), i wonder if it might not be possible to turn things on
their heads. Can we view the calculation of the probability amplitude
as an axiomatization of probability? If so, then the definition we
give for calculating probability amplitude may provide the basis for
an \emph{effective} theory of probability.

\paragraph{Quantum vs ``biological'' information}
Finally, i want to conclude with a more philosophical observation. At
a recent workshop in which QM was a predominant topic i noticed
something about quantum information. The speaker was giving a riveting
discussion of axiomatic QM and showing how properties of ``no
cloning'' and ``no deleting'' emerged as consequences of the
axiomatization. Theorems of this form are necessary to give us a sense
of confidence that our axioms characterize the physical theory. What
struck me, though, was that if quantum information is neither erasable
nor replicable it is markedly different from \emph{life}. Two of the
things we know about life is that

\begin{itemize}
  \item it ends;
  \item to gain some measure of persistence, to transcend it's
    finitude it is imminently copyable.
\end{itemize}

Both of these qualities are summarized succinctly in the aphorism: all
flesh is grass. For me these two kinds of ``information'' -- call them
quantum and biological -- are end points on a spectrum of strategies
for persistence. At one end, we have those curious entities that enjoy
uniqueness and permanence; at the other, we have those who in the face
of a certain end and an uncertain present make a go of passing
something on. To me one of the more remarkable aspects of the latter
strategy is that in the presence of noise (and certain features of
copying) we get a kind of dynamism, a chance for improvement against a
given persistent condition.

% subsection other_calculi_other_bisimulations_and_geometry_as_behavior (end)




% section conclusion (end)

%\documentclass[12pt]{llncs}
%\documentclass{jktr}

\usepackage[pdftex]{hyperref}                   
\usepackage {listings}
\usepackage {mathpartir}
\usepackage{bcprules}
%\usepackage{listings}
                       
\usepackage{graphicx} 
%\usepackage[margins=2.5cm,nohead,nofoot]{geometry}
%\usepackage{geometry}
\usepackage{amsfonts}
\usepackage{amstext}
\usepackage{latexsym}
\usepackage{amssymb}
\usepackage{color}


%\include{myPreamble}
\include{qm2pi.local} 

%\ifpdf
%\usepackage[pdftex]{graphicx}
%\else
%\usepackage{graphicx}
%\fi

 % \ifpdf
%  \usepackage{pdfsync}
%  \if


%\title{Brief Article}
%\author{David F. Snyder}
%\author{L.G. Meredith}

%\address{Dept. of Math., Texas State University--San Marcos, San Marcos, TX 78666}
       
\pagestyle{empty}


\begin{document}

\lstset{language=[Objective]Caml,frame=shadowbox}

\input{qm2pi.front}

% section front matter (end)

\input{qm2pi.intro} 
 
% section introduction (end)

% \input{qm2pi.knotations} 

% section notation (end)

\input{qm2pi.process.calculi} 

% section concurrent_process_calculi_and_spatial_logics_ (end)
    
%\input{qm2pi.knots2pi} 

%\input{qm2pi.trefoil} 

%\input{qm2pi.mainthm} 

% subsection basic_interpretation (end)

%\input{qm2pi.rho.presentation} 
\subsection{The syntax and semantics of the notation system}\label{sub:the_syntax_and_semantics_of_the_notation_system} % (fold)

We now summarize a technical presentation of the calculus that
embodies our theory of dynamics. The typical presentation of such a
calculus follows the style of giving generators and relations on
them. The grammar, below, describing term constructors, freely
generates the set of processes, $\Proc$. This set is then quotiented
by a relation known as structural congruence and it is over this set
that the notion of dynamics is expressed. This presentation is
essentially that of \cite{MeredithR05} with the addition of
polyadicity and summation. For readability we have relegated some of
the technical subtleties to an appendix.

\subsubsection{Process grammar}\label{subsub:process_grammar}

\begin{mathpar}
  \inferrule* [lab=synchronization] {} {{M} \bc \pzero \;|\; x?F \;|\; x!C }
  \and
  \inferrule* [lab=abstraction] {} {{F} \bc (x)P}
  \and
  \inferrule* [lab=concretion] {} {{C} \bc \langle Q \rangle}
  \and
  \inferrule* [lab=process] {} {{P,Q} \bc M \;| \;P|Q \;|\; @{x}}
  \and
  \inferrule* [lab=name] {} {{x} \bc \quotep{P}}
\end{mathpar} 

Note that $\vec{x}$ (resp. $\vec{P}$) denotes a vector of names
(resp. processes) of length $|\vec{x}|$ (resp. $|\vec{P}|$). We adopt
the following useful abbreviations.

\begin{mathpar}
   x?(\vec{y}).P := x.(\vec{y})P \and  x\clift{\vec{P}} := x.\clift{\vec{P}}
   \and x!(y) := \lift{x}{\dropn{y}}
   \and \Pi_{i=0}^{n-1}P_i := P_0 | \ldots | P_{n-1}
\end{mathpar}

\subsubsection{Structural congruence}

\paragraph{Free and bound names and alpha-equivalence.} At the
core of structural equivalence is alpha-equivalence which identifies
process that are the same up to a change of variable. Formally, we
recognize the distinction between free and bound names. The free names
of a process, $\freenames{P}$, may be calculated recursively as
follows:

\begin{mathpar}
\freenames{\pzero} := \emptyset
  \and \\
  \freenames{x?(y).P} := \{ x \} \cup (\freenames{P} \setminus \{ y \})
  \and 
  \freenames{x!\langle P \rangle} := \{ x \} \cup \{ P \} 
  \and \\
  \freenames{P|Q} := \freenames{P} \cup \freenames{Q}
  \and \\
  \freenames{@{x}} := \{ x \}
\end{mathpar}

$\pi$
$\quotep{\pi}$

$\freenames{-} : \pi \to \mathcal{P}(\quotep{\pi})$

\begin{eqnarray*}
  \freenames{\pzero} & := & \emptyset \\
  \freenames{x?(y).P} & := & \{ x \} \cup (\freenames{P} \setminus \{ y \}) \\
  \freenames{x!\langle P \rangle} & := & \{ x \} \cup \{ P \} \\
  \freenames{P|Q} & := & \freenames{P} \cup \freenames{Q} \\
  \freenames{\dropn{x}} & := & \{ x \}
\end{eqnarray*}

The bound names of a process, $\boundnames{P}$, are those names occurring in $P$
that are not free. For example, in $x?(y).0$, the name $x$ is free, while $y$ is bound.

\begin{mathpar}
  \inferrule* [lab=monoidal-laws] {} { P|Q \equiv Q|P \and P|0 \equiv P \and P|(Q|R) \equiv (P|Q)|R }
\end{mathpar}

\begin{mathpar}
  \inferrule* [lab=alpha-equivalence] {} { (x)P \equiv (y)P\{y/x\} \and y \not\in \freenames{P} }
\end{mathpar}

\begin{definition}
Then two processes, $P,Q$, are alpha-equivalent if $P = Q\{\vec{y}/\vec{x}\}$ for
some $\vec{x} \in \boundnames{Q},\vec{y} \in \boundnames{P}$, where $Q\{\vec{y}/\vec{x}\}$
denotes the capture-avoiding substitution of $\vec{y}$ for $\vec{x}$ in $Q$.
\end{definition}

\begin{definition}
  The {\em structural congruence} \cite{SangiorgiWalker} , $\equiv$,
  between processes is the least congruence containing
  alpha-equivalence, satisfying the abelian monoid laws
  (associativity, commutativity and $\pzero$ as identity) for parallel
  composition $|$ and for summation $+$.
\end{definition}

\subsection{Name equivalence}

We take name equivalence, written $\nameeq$, to be the smallest
equivalence relation generated by the following rules.

\begin{mathpar}
\inferrule*[lab=Quote-drop]
{ }
{ \quotep{@{x}} \nameeq x }

\inferrule*[lab=Struct-equiv]
{ P \scong Q }
{ \quotep{P} \nameeq \quotep{Q} }
\end{mathpar}

The astute reader will have noticed that the mutual recursion of names
and processes imposes a mutual recursion on alpha-equivalence and
structural equivalence via name-equivalence. Fortunately, all of this
works out pleasantly and we may calculate in the natural way, free of
concern. The reader interested in the details is referred to the
appendix \ref{appendix:rho_details}.

\subsection{Substitution}

We use $\Proc$ for the set of processes, $\QProc$ for the set of
names, and $\id{\{}\vec{y} / \vec{x} \id{\}}$ to denote partial maps,
$s : \QProc \rightarrow \QProc$. A map, $s$ lifts, uniquely, to a map
on process terms, $\widehat{s} : \Proc \rightarrow \Proc$ by the
following equations.

\begin{mathpar}
  (0) \psubstp{Q}{P} := 0 \\
  (R \juxtap S) \psubstp{Q}{P}
  :=    
  (R)\psubstp{Q}{P} \juxtap (S) \psubstp{Q}{P} \\
  (x?(y).R) \psubstp{Q}{P}    
  :=    
  (x)\substp{Q}{P} (z)\concat( (R \psubstn{z}{y}) \psubstp{Q}{P} ) \\
  (\lift{x}{R}) \psubstp{Q}{P}  
  :=
  \lift{(x)\substp{Q}{P}}{ R \psubstp{Q}{P} } \\
%   (\dropn{x})  \psubstp{Q}{P}       
%   := 
%   \left\{ 
%     \begin{array}{ccc} 
%       \dropn{\quotep{Q}} & & x \nameeq \quotep{P} \\
%       \dropn{x} & & otherwise \\
%     \end{array}
%   \right. 
  (\dropn{x})  \psubstp{Q}{P}       
  := 
  \left\{ 
    \begin{array}{ccc} 
      Q & & x \nameeq \quotep{P} \\
      \dropn{x} & & otherwise \\
    \end{array}
  \right.
\end{mathpar}
 

where

\begin{eqnarray}
  (x)\id{\{} \lpquote Q \rpquote / \lpquote P \rpquote \id{\}}            = 
  \left\{ 
    \begin{array}{ccc}
      \lpquote Q \rpquote & & x \nameeq \lpquote P \rpquote \\
      x & & otherwise \\
    \end{array}
  \right. \nonumber
\end{eqnarray}

and $z$ is chosen distinct from $\quotep{P}$, $\quotep{Q}$, the free
names in $Q$, and all the names in $R$. Our $\alpha$-equivalence will
be built in the standard way from this substitution.

\begin{remark}\label{rem:no_self_referential_names}
  One consequence of these definitions is that $\forall P. \quotep{P}
  \not\in \freenames{P}$.
\end{remark}

\subsection{ Dynamic quote: an example }

Anticipating something of what's to come, consider applying the
substitution, $\widehat{\id{\{}u / z \id{\}}}$, to the following pair
of processes, $\lift{w}{y!(z)}$ and $w[ \lpquote y!(z) \rpquote ]$.

\begin{eqnarray}
	\lift{w}{y!(z)}\widehat{\id{\{}u / z \id{\}}}
		& = &
		\lift{w}{y!(u)} \nonumber\\
	w[ \lpquote y!(z) \rpquote ] \widehat{ \id{\{}u / z \id{\}} }
		& = &
		w[ \lpquote y!(z) \rpquote ] \nonumber
\end{eqnarray}

Because the body of the process between quotes is impervious to
substitution, we get radically different answers. In fact, by
examining the first process in an input context,
e.g. $x?(z).\lift{w}{y!(z)}$, we see that the process under the lift
operator may be shaped by prefixed inputs binding a name inside it. In
this sense, the lift operator will be seen as a way to dynamically
construct processes before reifying them as names.

Finally equipped with these standard features we can present the
dynamics of the calculus.

\subsubsection{Operational semantics} 

Finally, we introduce the computational dynamics. What marks these
algebras as distinct from other more traditionally studied algebraic
structures, e.g. vector spaces or polynomial rings, is the manner in
which dynamics is captured. In traditional structures, dynamics is typically
expressed through morphisms between such structures, as in linear maps
between vector spaces or morphisms between rings. In algebras
associated with the semantics of computation, the dynamics is
expressed as part of the algebraic structure itself, through a
reduction reduction relation typically denoted by $\red$. Below, we
give a recursive presentation of this relation for the calculus used
in the encoding.

$\red \subseteq \pi \times \pi$
$\red : \pi \to \mathcal{P}(\pi)$

\begin{mathpar}
  \inferrule* [lab=Comm] { \textsf{match}( x_{src}, x_{trgt} ) } { x_{trgt}?(y)P \; | \; x_{src}!\langle {Q} \rangle \red P\{\quotep{Q}/y}\} }
  \and \\
  \inferrule* [lab=Par] {{P} \red {P}'} {{{P} | {Q}} \red {{P}' | {Q}}}
  \and
  \inferrule* [lab=Equiv]{{{P} \scong {P}'} \andalso {{P}' \red {Q}'} \andalso {{Q}' \scong {Q}}}{{P} \red {Q}}
\end{mathpar}

\begin{eqnarray*}
  match_{\equiv} (\quotep{P},\quotep{Q}) & := & P \equiv Q \\
  match_{\dagger}(\quotep{P},\quotep{Q}) & := & \forall R. P|Q \red^{*} R => R \red^{*} 0 \\
  match_{K}(\quotep{P},\quotep{Q}) & := & K \mbox{ for some context } K
\end{eqnarray*}

$u?(x)P | u!\langle Q \rangle \red P\{\quotep{Q}/x\}$

%We write $\wred$ for $\red^*$, and $P\red$ if $\exists Q $ such that $ P \red Q$.
We write $P\red$ if $\exists Q $ such that $ P \red Q$ and $P\not\red$, otherwise.

\section{Replication}

As mentioned before, it is known that replication (and hence
recursion) can be implemented in a higher-order process algebra
\cite{SangiorgiWalker}. As our first example of calculation with the
machinery thus far presented we give the construction explicitly in
the {\rhoc}.

\begin{eqnarray}
	D_{x} & := & \prefix{x}{y}{(\binpar{\outputp{x}{y}}{@{y}})} \nonumber\\
	\bangp_{x}{P} & := & \binpar{{x}!\langle{\binpar{D_{x}}{P}}\rangle}{D_{x}} \nonumber
\end{eqnarray}

\begin{eqnarray}
	\bangp_{x}{P} & & \nonumber\\
	=
	& {x}!\langle{(\prefix{x}{y}{(\outputp{x}{y} | @{y})) | P}}\rangle 
	      | \prefix{x}{y}{(\outputp{x}{y} | @{y})} & \nonumber\\
	\red
	& (\outputp{x}{y} | @{y})\substn{\quotep{(\prefix{x}{y}{(@{y} | \outputp{x}{y})) | P}}}{y} & \nonumber\\
	=
	& \outputp{x}{\quotep{(\prefix{x}{y}{(\outputp{x}{y} | @{y})) | P}}}
	  | {(\prefix{x}{y}{(\outputp{x}{y} | @{y})) | P}} & \nonumber\\
	\red
	& \ldots & \nonumber\\
	\red^*
	& P | P | \ldots & \nonumber
\end{eqnarray}

Of course, this encoding, as an implementation, runs away, unfolding
$\bangp{P}$ eagerly. A lazier and more implementable replication
operator, restricted to input-guarded processes, may be obtained as follows.

\begin{eqnarray}
\bangp{\prefix{u}{v}{P}} 
	:= 
	\binpar{\lift{x}{\prefix{u}{v}{(\binpar{D(x)}{P})}}}{D(x)} \nonumber
\end{eqnarray}

\begin{remark}
  Note that the lazier definition still does not deal with summation
  or mixed summation (i.e. sums over input and output). The reader is
  invited to construct definitions of replication that deal with these
  features. 

  Further, the definitions are parameterized in a name, $x$. Can you,
  gentle reader, make a definition that eliminates this parameter and
  guarantees no accidental interaction between the replication
  machinery and the process being replicated -- i.e. no accidental
  sharing of names used by the process to get its work done and the
  name(s) used by the replication to effect copying. This latter
  revision of the definition of replication is crucial to obtaining
  the expected identity $!!P \sim !P$.
\end{remark}

\begin{remark}\label{rem:paradoxical_combinator}
  The reader familiar with the lambda calculus will have noticed the
  similarity between $D$ and the paradoxical combinator.

  [Ed. note: the existence of this seems to suggest we have to be more
  restrictive on the set of processes and names we admit if we are to
  support no-cloning.]
\end{remark}

\subsubsection{Bisimulation}

The computational dynamics gives rise to another kind of equivalence,
the equivalence of computational behavior. As previously mentioned
this is typically captured \emph{via} some form of bisimulation.

% The notion we use in this paper is weak barbed bisimulation
% \cite{milner91polyadicpi}.

The notion we use in this paper is derived from weak barbed
bisimulation \cite{milner91polyadicpi}. 

\begin{definition}
An \emph{observation relation}, $\downarrow_{\mathcal N}$, over a set
of names, $\mathcal N$, is the smallest relation satisfying the rules
below.

\infrule[Out-barb]{y \in {\mathcal N}, \; x \nameeq y}
		  {\outputp{x}{v} \downarrow_{\mathcal N} x}
\infrule[Par-barb]{\mbox{$P\downarrow_{\mathcal N} x$ or $Q\downarrow_{\mathcal N} x$}}
		  {\binpar{P}{Q} \downarrow_{\mathcal N} x}

We write $P \Downarrow_{\mathcal N} x$ if there is $Q$ such that 
$P \wred Q$ and $Q \downarrow_{\mathcal N} x$.
\end{definition}

\begin{definition}
%\label{def.bbisim}
An  ${\mathcal N}$-\emph{barbed bisimulation} over a set of names, ${\mathcal N}$, is a symmetric binary relation 
${\mathcal S}_{\mathcal N}$ between agents such that $P\rel{S}_{\mathcal N}Q$ implies:
\begin{enumerate}
\item If $P \red P'$ then $Q \wred Q'$ and $P'\rel{S}_{\mathcal N} Q'$.
\item If $P\downarrow_{\mathcal N} x$, then $Q\Downarrow_{\mathcal N} x$.
\end{enumerate}
$P$ is ${\mathcal N}$-barbed bisimilar to $Q$, written
$P \wbbisim_{\mathcal N} Q$, if $P \rel{S}_{\mathcal N} Q$ for some ${\mathcal N}$-barbed bisimulation ${\mathcal S}_{\mathcal N}$.
\end{definition}

$\mathcal{R} \subseteq \pi \times \pi$

$P \mathcal{R} Q => \forall P'. P \red P' \Rightarrow \exists Q'. Q \red Q', P' \mathcal{R} Q'$

$P \vdash x \Rightarrow Q \vdash x$

\begin{mathpar}
  \inferrule*[lab=Out-barb]{x \nameeq y}{{y}!\langle{Q}\rangle \vdash x}
  \and
  \inferrule*[lab=Par-barb]{\mbox{$P\vdash x$ or $Q\vdash x$}}{\binpar{P}{Q} \vdash x}
\end{mathpar}

\subsubsection{Contexts}

One of the principle advantages of computational calculi like the
$\pi$-calculus is a well-defined notion of context,
contextual-equivalence and a correlation between
contextual-equivalence and notions of bisimulation. The notion of
context allows the decomposition of a process into (sub-)process and
its syntactic environment, its context. Thus, a context may be
thought of as a process with a ``hole'' (written $\Box$) in it. The
application of a context $M$ to a process $P$, written $M[P]$, is
tantamount to filling the hole in $M$ with $P$. In this paper we do
not need the full weight of this theory, but do make use of the notion
of context in the proof the main theorem. 

\begin{mathpar}
  \inferrule* [lab=summation] {} {{M_{M},M_{N}} \bc \Box \;|\; x.M_{A} \;|\; M_{M}+M_{N}}
  \and
  \inferrule* [lab=agent] {} {{M_{A}} \bc (\vec{x})M_{P} \;| \; \clift{P_0,\ldots,M_{P},\ldots,P_N}}
  \and \\
  \inferrule* [lab=process] {} {{M_{P}} \bc M_{N} \;| \;P|M_{P} }
\end{mathpar} 

\begin{mathpar}
  \inferrule* [lab=sychronization] {} {M_{N} \bc \Box \;|\; x?M_{F} \;|\; x!M_{C}}
  \and
  \inferrule* [lab=abstraction] {} {{M_{F}} \bc (x)M_{P} }
  \and
  \inferrule* [lab=concretion] {} {{M_{C}} \bc \langle M_{P} \rangle }
  \and \\
  \inferrule* [lab=process] {} {{M_{P}} \bc M_{N} \;| \;P|M_{P} }
\end{mathpar}

\begin{definition}[contextual application] Given a context $M$, and
  process $P$, we define the \emph{contextual application}, $M[P] :=
  M\{P/\Box\}$. That is, the contextual application of M to P is the
  substitution of $P$ for $\Box$ in $M$.
\end{definition}

$\meaningof{-} : L \to \mathcal{P}(\pi)$

\begin{mathpar}
  \inferrule* [lab=collection] {} {\meaningof{true} = \pi, \and \meaningof{~E} = \pi \setminus \meaningof{E}, \and \meaningof{E_{1} \& E_{2}} = \meaningof{E_{1}} \cap \meaningof{E_{2}}}
\end{mathpar}

\begin{mathpar}
  \inferrule* [lab=structure] {} {\meaningof{0} = \{ P \in \pi | P \equiv 0 \}, \and \\ \meaningof{E_1 | E_2} = \{ P \in \pi | P \equiv P_{1} | P_{2}, P_{1} \in \meaningof{E_{1}}, P_{2} \in \meaningof{E_2}\} }
\end{mathpar}

\begin{mathpar}
 \inferrule* [lab=behavior] {} {\meaningof{\langle a?b \rangle E} = \{ P \in \pi | P \equiv Q | u?(y)P', \\ \and \\\\ \and \\ \;\;\; u \in \meaningof{a}, \forall z.P'\{z/y\} \in \meaningof{E\{z/b\}}\}, \and \\ \meaningof{a!E} = \{ P \in \pi | P \equiv Q | x!\langle P' \rangle, x \in \meaningof{a} P' \in \meaningof{E}\} }
\end{mathpar}

\begin{mathpar}
 \inferrule* [lab=nominal] {} {\meaningof{\quotep{E}} = \{ \quotep{P} \in \quotep{\pi} | P \in \meaningof{E} \}, \and \meaningof{\quotep{P}} = \{ \quotep{Q} \in \quotep{\pi} | P \equiv Q \} \and \\ \meaningof{@\quotep{E}} = \{ P \in \pi | P \equiv @x, x \in \meaningof{E} \}}
\end{mathpar}

\begin{eqnarray*}
  \\
  \meaningof{-} : TS \to ST
\end{eqnarray*}

\begin{eqnarray*}
  \\
  L : TS \to ST
\end{eqnarray*}

\begin{eqnarray*}
  \\
  P \models E \iff P \in \meaningof{E}
\end{eqnarray*}

\begin{eqnarray*}
  P \approx_{L} Q \iff \forall E \in L. P \models E \iff Q \models E
\end{eqnarray*}

\begin{eqnarray*}
  P \approx_{K} Q
\end{eqnarray*}

\begin{eqnarray*}
  P \approx Q
\end{eqnarray*}

$\approx_{K} = \approx = \approx_{L}$

\subsubsection{Contextual duality}

Note that contexts extend the quotation operation to a family of
operations from processes to names. Given a context, $M$, we can
define a \emph{nominal context}, $\quotep{M}$ by $\quotep{M}[P] :=
\quotep{M[P]}$. To foreshadow what is to come we observe that these
operations enjoy a duality with processes very much like the duality
between vectors and maps from vectors to scalars.

Further, because the calculus is essentially higher-order, we have a
correspondence between contexts and processes. More specifically,
given a name $x$ and a context $M$ we can construct $M^{*}_{x}$ such
that 

\begin{mathpar}
  M^{*}_{x} | \lift{x}{P} \red M[P]
\end{mathpar}

namely,

\begin{mathpar}
  M^{*}_{x} := x?(u).M[\dropn{u}]
\end{mathpar}

The dependence of $M^{*}_{x}$ on a name makes it an abstraction, 

\begin{mathpar}
  M^{*} := (x)x?(u).M[\dropn{u}]
\end{mathpar}

\subsection{Additional notation}

It will sometimes be convenient to denote the process a name
quotes. We already have the notation $x = \quotep{P}$, but it will be
convenient to introduce an alternate notation, $\procn{x}$, when we
want to emphasize the connection to the use of the name. Note that, by
virtue of name equivalence, $\quotep{\procn{x}} \nameeq x$; so, the
notation is consistent with previous definitions.

Further, because names have structure it is possible to effect
substitutions on the basis of that structure. This means we need to
upgrade our notation for substitutions, which we accomplish by
adapting comprehension notation. Thus,

\begin{mathpar}
  P\{ y / x : x \in S \}
\end{mathpar}

is interpreted to mean the process derived from P by replacing (in a
capture-avoiding manner) each occurrence of $x$ in $S$ by $y$. For example,

\begin{mathpar}
  P\{ \quotep{\procn{x}|\procn{x}} / x : x \in \freenames{P} \}
\end{mathpar}

will replace each (occurrence) of a free name $x$ in $P$ by
$\quotep{\procn{x}|\procn{x}}$.

Also, we will avail ourselves of the notation $x^{L}$ and $x^{R}$ to
denote injections of a name into disjoint copies of the name
space. There are numerous ways to accomplish this. One example can be
found in \cite{MeredithR05}. This notation overloads to vectors of
names: $\vec{x}^{\pi} := (x_{i}^{\pi} \; : \; 0 \leq i < |\vec{x}| )$ where $\pi \in \{L,R\}$.

We also use $P^{\Box} := P|\Box$.

In \cite{MeredithR05} an interpretation of the new operator is
given. It turns out that there are several possible interpretations
all enjoying the requisite algebraic properties of the operator (see
\cite{milner91polyadicpi}). We will therefore make liberal use of
$(\nu\; \vec{x})P$.

% subsection the_syntax_and_semantics_of_the_notation_system (end)   

\input{qm2pi.qmops} 

\input{qm2pi.sterngerlach} 

\input{qm2pi.metric} 

% section concurrent_process_calculi (end)

%\input{qm2pi.proofsketch}

% section proof sketch (end)

%\input{qm2pi.slviaknots} 

% section spatial logic via knots (end)

\input{qm2pi.conclusion}

% section conclusion (end)

%\input{qm2pi.dtcodes} 

% section wiring algorithm (end)

\input{qm2pi.ack} 

% section acknowledgments (end)

\newpage


\bibliographystyle{plain}   
\bibliography{../../biblios/main.bib}

\input{qm2pi.rhodetails}

\end{document}

 

% section wiring algorithm (end)

\documentclass[12pt]{llncs}
%\documentclass{jktr}

\usepackage[pdftex]{hyperref}                   
\usepackage {listings}
\usepackage {mathpartir}
\usepackage{bcprules}
%\usepackage{listings}
                       
\usepackage{graphicx} 
%\usepackage[margins=2.5cm,nohead,nofoot]{geometry}
%\usepackage{geometry}
\usepackage{amsfonts}
\usepackage{amstext}
\usepackage{latexsym}
\usepackage{amssymb}
\usepackage{color}


%\include{myPreamble}
\include{qm2pi.local} 

%\ifpdf
%\usepackage[pdftex]{graphicx}
%\else
%\usepackage{graphicx}
%\fi

 % \ifpdf
%  \usepackage{pdfsync}
%  \if


%\title{Brief Article}
%\author{David F. Snyder}
%\author{L.G. Meredith}

%\address{Dept. of Math., Texas State University--San Marcos, San Marcos, TX 78666}
       
\pagestyle{empty}


\begin{document}

\lstset{language=[Objective]Caml,frame=shadowbox}

\input{qm2pi.front}

% section front matter (end)

\input{qm2pi.intro} 
 
% section introduction (end)

% \input{qm2pi.knotations} 

% section notation (end)

\input{qm2pi.process.calculi} 

% section concurrent_process_calculi_and_spatial_logics_ (end)
    
%\input{qm2pi.knots2pi} 

%\input{qm2pi.trefoil} 

%\input{qm2pi.mainthm} 

% subsection basic_interpretation (end)

%\input{qm2pi.rho.presentation} 
\subsection{The syntax and semantics of the notation system}\label{sub:the_syntax_and_semantics_of_the_notation_system} % (fold)

We now summarize a technical presentation of the calculus that
embodies our theory of dynamics. The typical presentation of such a
calculus follows the style of giving generators and relations on
them. The grammar, below, describing term constructors, freely
generates the set of processes, $\Proc$. This set is then quotiented
by a relation known as structural congruence and it is over this set
that the notion of dynamics is expressed. This presentation is
essentially that of \cite{MeredithR05} with the addition of
polyadicity and summation. For readability we have relegated some of
the technical subtleties to an appendix.

\subsubsection{Process grammar}\label{subsub:process_grammar}

\begin{mathpar}
  \inferrule* [lab=synchronization] {} {{M} \bc \pzero \;|\; x?F \;|\; x!C }
  \and
  \inferrule* [lab=abstraction] {} {{F} \bc (x)P}
  \and
  \inferrule* [lab=concretion] {} {{C} \bc \langle Q \rangle}
  \and
  \inferrule* [lab=process] {} {{P,Q} \bc M \;| \;P|Q \;|\; @{x}}
  \and
  \inferrule* [lab=name] {} {{x} \bc \quotep{P}}
\end{mathpar} 

Note that $\vec{x}$ (resp. $\vec{P}$) denotes a vector of names
(resp. processes) of length $|\vec{x}|$ (resp. $|\vec{P}|$). We adopt
the following useful abbreviations.

\begin{mathpar}
   x?(\vec{y}).P := x.(\vec{y})P \and  x\clift{\vec{P}} := x.\clift{\vec{P}}
   \and x!(y) := \lift{x}{\dropn{y}}
   \and \Pi_{i=0}^{n-1}P_i := P_0 | \ldots | P_{n-1}
\end{mathpar}

\subsubsection{Structural congruence}

\paragraph{Free and bound names and alpha-equivalence.} At the
core of structural equivalence is alpha-equivalence which identifies
process that are the same up to a change of variable. Formally, we
recognize the distinction between free and bound names. The free names
of a process, $\freenames{P}$, may be calculated recursively as
follows:

\begin{mathpar}
\freenames{\pzero} := \emptyset
  \and \\
  \freenames{x?(y).P} := \{ x \} \cup (\freenames{P} \setminus \{ y \})
  \and 
  \freenames{x!\langle P \rangle} := \{ x \} \cup \{ P \} 
  \and \\
  \freenames{P|Q} := \freenames{P} \cup \freenames{Q}
  \and \\
  \freenames{@{x}} := \{ x \}
\end{mathpar}

$\pi$
$\quotep{\pi}$

$\freenames{-} : \pi \to \mathcal{P}(\quotep{\pi})$

\begin{eqnarray*}
  \freenames{\pzero} & := & \emptyset \\
  \freenames{x?(y).P} & := & \{ x \} \cup (\freenames{P} \setminus \{ y \}) \\
  \freenames{x!\langle P \rangle} & := & \{ x \} \cup \{ P \} \\
  \freenames{P|Q} & := & \freenames{P} \cup \freenames{Q} \\
  \freenames{\dropn{x}} & := & \{ x \}
\end{eqnarray*}

The bound names of a process, $\boundnames{P}$, are those names occurring in $P$
that are not free. For example, in $x?(y).0$, the name $x$ is free, while $y$ is bound.

\begin{mathpar}
  \inferrule* [lab=monoidal-laws] {} { P|Q \equiv Q|P \and P|0 \equiv P \and P|(Q|R) \equiv (P|Q)|R }
\end{mathpar}

\begin{mathpar}
  \inferrule* [lab=alpha-equivalence] {} { (x)P \equiv (y)P\{y/x\} \and y \not\in \freenames{P} }
\end{mathpar}

\begin{definition}
Then two processes, $P,Q$, are alpha-equivalent if $P = Q\{\vec{y}/\vec{x}\}$ for
some $\vec{x} \in \boundnames{Q},\vec{y} \in \boundnames{P}$, where $Q\{\vec{y}/\vec{x}\}$
denotes the capture-avoiding substitution of $\vec{y}$ for $\vec{x}$ in $Q$.
\end{definition}

\begin{definition}
  The {\em structural congruence} \cite{SangiorgiWalker} , $\equiv$,
  between processes is the least congruence containing
  alpha-equivalence, satisfying the abelian monoid laws
  (associativity, commutativity and $\pzero$ as identity) for parallel
  composition $|$ and for summation $+$.
\end{definition}

\subsection{Name equivalence}

We take name equivalence, written $\nameeq$, to be the smallest
equivalence relation generated by the following rules.

\begin{mathpar}
\inferrule*[lab=Quote-drop]
{ }
{ \quotep{@{x}} \nameeq x }

\inferrule*[lab=Struct-equiv]
{ P \scong Q }
{ \quotep{P} \nameeq \quotep{Q} }
\end{mathpar}

The astute reader will have noticed that the mutual recursion of names
and processes imposes a mutual recursion on alpha-equivalence and
structural equivalence via name-equivalence. Fortunately, all of this
works out pleasantly and we may calculate in the natural way, free of
concern. The reader interested in the details is referred to the
appendix \ref{appendix:rho_details}.

\subsection{Substitution}

We use $\Proc$ for the set of processes, $\QProc$ for the set of
names, and $\id{\{}\vec{y} / \vec{x} \id{\}}$ to denote partial maps,
$s : \QProc \rightarrow \QProc$. A map, $s$ lifts, uniquely, to a map
on process terms, $\widehat{s} : \Proc \rightarrow \Proc$ by the
following equations.

\begin{mathpar}
  (0) \psubstp{Q}{P} := 0 \\
  (R \juxtap S) \psubstp{Q}{P}
  :=    
  (R)\psubstp{Q}{P} \juxtap (S) \psubstp{Q}{P} \\
  (x?(y).R) \psubstp{Q}{P}    
  :=    
  (x)\substp{Q}{P} (z)\concat( (R \psubstn{z}{y}) \psubstp{Q}{P} ) \\
  (\lift{x}{R}) \psubstp{Q}{P}  
  :=
  \lift{(x)\substp{Q}{P}}{ R \psubstp{Q}{P} } \\
%   (\dropn{x})  \psubstp{Q}{P}       
%   := 
%   \left\{ 
%     \begin{array}{ccc} 
%       \dropn{\quotep{Q}} & & x \nameeq \quotep{P} \\
%       \dropn{x} & & otherwise \\
%     \end{array}
%   \right. 
  (\dropn{x})  \psubstp{Q}{P}       
  := 
  \left\{ 
    \begin{array}{ccc} 
      Q & & x \nameeq \quotep{P} \\
      \dropn{x} & & otherwise \\
    \end{array}
  \right.
\end{mathpar}
 

where

\begin{eqnarray}
  (x)\id{\{} \lpquote Q \rpquote / \lpquote P \rpquote \id{\}}            = 
  \left\{ 
    \begin{array}{ccc}
      \lpquote Q \rpquote & & x \nameeq \lpquote P \rpquote \\
      x & & otherwise \\
    \end{array}
  \right. \nonumber
\end{eqnarray}

and $z$ is chosen distinct from $\quotep{P}$, $\quotep{Q}$, the free
names in $Q$, and all the names in $R$. Our $\alpha$-equivalence will
be built in the standard way from this substitution.

\begin{remark}\label{rem:no_self_referential_names}
  One consequence of these definitions is that $\forall P. \quotep{P}
  \not\in \freenames{P}$.
\end{remark}

\subsection{ Dynamic quote: an example }

Anticipating something of what's to come, consider applying the
substitution, $\widehat{\id{\{}u / z \id{\}}}$, to the following pair
of processes, $\lift{w}{y!(z)}$ and $w[ \lpquote y!(z) \rpquote ]$.

\begin{eqnarray}
	\lift{w}{y!(z)}\widehat{\id{\{}u / z \id{\}}}
		& = &
		\lift{w}{y!(u)} \nonumber\\
	w[ \lpquote y!(z) \rpquote ] \widehat{ \id{\{}u / z \id{\}} }
		& = &
		w[ \lpquote y!(z) \rpquote ] \nonumber
\end{eqnarray}

Because the body of the process between quotes is impervious to
substitution, we get radically different answers. In fact, by
examining the first process in an input context,
e.g. $x?(z).\lift{w}{y!(z)}$, we see that the process under the lift
operator may be shaped by prefixed inputs binding a name inside it. In
this sense, the lift operator will be seen as a way to dynamically
construct processes before reifying them as names.

Finally equipped with these standard features we can present the
dynamics of the calculus.

\subsubsection{Operational semantics} 

Finally, we introduce the computational dynamics. What marks these
algebras as distinct from other more traditionally studied algebraic
structures, e.g. vector spaces or polynomial rings, is the manner in
which dynamics is captured. In traditional structures, dynamics is typically
expressed through morphisms between such structures, as in linear maps
between vector spaces or morphisms between rings. In algebras
associated with the semantics of computation, the dynamics is
expressed as part of the algebraic structure itself, through a
reduction reduction relation typically denoted by $\red$. Below, we
give a recursive presentation of this relation for the calculus used
in the encoding.

$\red \subseteq \pi \times \pi$
$\red : \pi \to \mathcal{P}(\pi)$

\begin{mathpar}
  \inferrule* [lab=Comm] { \textsf{match}( x_{src}, x_{trgt} ) } { x_{trgt}?(y)P \; | \; x_{src}!\langle {Q} \rangle \red P\{\quotep{Q}/y}\} }
  \and \\
  \inferrule* [lab=Par] {{P} \red {P}'} {{{P} | {Q}} \red {{P}' | {Q}}}
  \and
  \inferrule* [lab=Equiv]{{{P} \scong {P}'} \andalso {{P}' \red {Q}'} \andalso {{Q}' \scong {Q}}}{{P} \red {Q}}
\end{mathpar}

\begin{eqnarray*}
  match_{\equiv} (\quotep{P},\quotep{Q}) & := & P \equiv Q \\
  match_{\dagger}(\quotep{P},\quotep{Q}) & := & \forall R. P|Q \red^{*} R => R \red^{*} 0 \\
  match_{K}(\quotep{P},\quotep{Q}) & := & K \mbox{ for some context } K
\end{eqnarray*}

$u?(x)P | u!\langle Q \rangle \red P\{\quotep{Q}/x\}$

%We write $\wred$ for $\red^*$, and $P\red$ if $\exists Q $ such that $ P \red Q$.
We write $P\red$ if $\exists Q $ such that $ P \red Q$ and $P\not\red$, otherwise.

\section{Replication}

As mentioned before, it is known that replication (and hence
recursion) can be implemented in a higher-order process algebra
\cite{SangiorgiWalker}. As our first example of calculation with the
machinery thus far presented we give the construction explicitly in
the {\rhoc}.

\begin{eqnarray}
	D_{x} & := & \prefix{x}{y}{(\binpar{\outputp{x}{y}}{@{y}})} \nonumber\\
	\bangp_{x}{P} & := & \binpar{{x}!\langle{\binpar{D_{x}}{P}}\rangle}{D_{x}} \nonumber
\end{eqnarray}

\begin{eqnarray}
	\bangp_{x}{P} & & \nonumber\\
	=
	& {x}!\langle{(\prefix{x}{y}{(\outputp{x}{y} | @{y})) | P}}\rangle 
	      | \prefix{x}{y}{(\outputp{x}{y} | @{y})} & \nonumber\\
	\red
	& (\outputp{x}{y} | @{y})\substn{\quotep{(\prefix{x}{y}{(@{y} | \outputp{x}{y})) | P}}}{y} & \nonumber\\
	=
	& \outputp{x}{\quotep{(\prefix{x}{y}{(\outputp{x}{y} | @{y})) | P}}}
	  | {(\prefix{x}{y}{(\outputp{x}{y} | @{y})) | P}} & \nonumber\\
	\red
	& \ldots & \nonumber\\
	\red^*
	& P | P | \ldots & \nonumber
\end{eqnarray}

Of course, this encoding, as an implementation, runs away, unfolding
$\bangp{P}$ eagerly. A lazier and more implementable replication
operator, restricted to input-guarded processes, may be obtained as follows.

\begin{eqnarray}
\bangp{\prefix{u}{v}{P}} 
	:= 
	\binpar{\lift{x}{\prefix{u}{v}{(\binpar{D(x)}{P})}}}{D(x)} \nonumber
\end{eqnarray}

\begin{remark}
  Note that the lazier definition still does not deal with summation
  or mixed summation (i.e. sums over input and output). The reader is
  invited to construct definitions of replication that deal with these
  features. 

  Further, the definitions are parameterized in a name, $x$. Can you,
  gentle reader, make a definition that eliminates this parameter and
  guarantees no accidental interaction between the replication
  machinery and the process being replicated -- i.e. no accidental
  sharing of names used by the process to get its work done and the
  name(s) used by the replication to effect copying. This latter
  revision of the definition of replication is crucial to obtaining
  the expected identity $!!P \sim !P$.
\end{remark}

\begin{remark}\label{rem:paradoxical_combinator}
  The reader familiar with the lambda calculus will have noticed the
  similarity between $D$ and the paradoxical combinator.

  [Ed. note: the existence of this seems to suggest we have to be more
  restrictive on the set of processes and names we admit if we are to
  support no-cloning.]
\end{remark}

\subsubsection{Bisimulation}

The computational dynamics gives rise to another kind of equivalence,
the equivalence of computational behavior. As previously mentioned
this is typically captured \emph{via} some form of bisimulation.

% The notion we use in this paper is weak barbed bisimulation
% \cite{milner91polyadicpi}.

The notion we use in this paper is derived from weak barbed
bisimulation \cite{milner91polyadicpi}. 

\begin{definition}
An \emph{observation relation}, $\downarrow_{\mathcal N}$, over a set
of names, $\mathcal N$, is the smallest relation satisfying the rules
below.

\infrule[Out-barb]{y \in {\mathcal N}, \; x \nameeq y}
		  {\outputp{x}{v} \downarrow_{\mathcal N} x}
\infrule[Par-barb]{\mbox{$P\downarrow_{\mathcal N} x$ or $Q\downarrow_{\mathcal N} x$}}
		  {\binpar{P}{Q} \downarrow_{\mathcal N} x}

We write $P \Downarrow_{\mathcal N} x$ if there is $Q$ such that 
$P \wred Q$ and $Q \downarrow_{\mathcal N} x$.
\end{definition}

\begin{definition}
%\label{def.bbisim}
An  ${\mathcal N}$-\emph{barbed bisimulation} over a set of names, ${\mathcal N}$, is a symmetric binary relation 
${\mathcal S}_{\mathcal N}$ between agents such that $P\rel{S}_{\mathcal N}Q$ implies:
\begin{enumerate}
\item If $P \red P'$ then $Q \wred Q'$ and $P'\rel{S}_{\mathcal N} Q'$.
\item If $P\downarrow_{\mathcal N} x$, then $Q\Downarrow_{\mathcal N} x$.
\end{enumerate}
$P$ is ${\mathcal N}$-barbed bisimilar to $Q$, written
$P \wbbisim_{\mathcal N} Q$, if $P \rel{S}_{\mathcal N} Q$ for some ${\mathcal N}$-barbed bisimulation ${\mathcal S}_{\mathcal N}$.
\end{definition}

$\mathcal{R} \subseteq \pi \times \pi$

$P \mathcal{R} Q => \forall P'. P \red P' \Rightarrow \exists Q'. Q \red Q', P' \mathcal{R} Q'$

$P \vdash x \Rightarrow Q \vdash x$

\begin{mathpar}
  \inferrule*[lab=Out-barb]{x \nameeq y}{{y}!\langle{Q}\rangle \vdash x}
  \and
  \inferrule*[lab=Par-barb]{\mbox{$P\vdash x$ or $Q\vdash x$}}{\binpar{P}{Q} \vdash x}
\end{mathpar}

\subsubsection{Contexts}

One of the principle advantages of computational calculi like the
$\pi$-calculus is a well-defined notion of context,
contextual-equivalence and a correlation between
contextual-equivalence and notions of bisimulation. The notion of
context allows the decomposition of a process into (sub-)process and
its syntactic environment, its context. Thus, a context may be
thought of as a process with a ``hole'' (written $\Box$) in it. The
application of a context $M$ to a process $P$, written $M[P]$, is
tantamount to filling the hole in $M$ with $P$. In this paper we do
not need the full weight of this theory, but do make use of the notion
of context in the proof the main theorem. 

\begin{mathpar}
  \inferrule* [lab=summation] {} {{M_{M},M_{N}} \bc \Box \;|\; x.M_{A} \;|\; M_{M}+M_{N}}
  \and
  \inferrule* [lab=agent] {} {{M_{A}} \bc (\vec{x})M_{P} \;| \; \clift{P_0,\ldots,M_{P},\ldots,P_N}}
  \and \\
  \inferrule* [lab=process] {} {{M_{P}} \bc M_{N} \;| \;P|M_{P} }
\end{mathpar} 

\begin{mathpar}
  \inferrule* [lab=sychronization] {} {M_{N} \bc \Box \;|\; x?M_{F} \;|\; x!M_{C}}
  \and
  \inferrule* [lab=abstraction] {} {{M_{F}} \bc (x)M_{P} }
  \and
  \inferrule* [lab=concretion] {} {{M_{C}} \bc \langle M_{P} \rangle }
  \and \\
  \inferrule* [lab=process] {} {{M_{P}} \bc M_{N} \;| \;P|M_{P} }
\end{mathpar}

\begin{definition}[contextual application] Given a context $M$, and
  process $P$, we define the \emph{contextual application}, $M[P] :=
  M\{P/\Box\}$. That is, the contextual application of M to P is the
  substitution of $P$ for $\Box$ in $M$.
\end{definition}

$\meaningof{-} : L \to \mathcal{P}(\pi)$

\begin{mathpar}
  \inferrule* [lab=collection] {} {\meaningof{true} = \pi, \and \meaningof{~E} = \pi \setminus \meaningof{E}, \and \meaningof{E_{1} \& E_{2}} = \meaningof{E_{1}} \cap \meaningof{E_{2}}}
\end{mathpar}

\begin{mathpar}
  \inferrule* [lab=structure] {} {\meaningof{0} = \{ P \in \pi | P \equiv 0 \}, \and \\ \meaningof{E_1 | E_2} = \{ P \in \pi | P \equiv P_{1} | P_{2}, P_{1} \in \meaningof{E_{1}}, P_{2} \in \meaningof{E_2}\} }
\end{mathpar}

\begin{mathpar}
 \inferrule* [lab=behavior] {} {\meaningof{\langle a?b \rangle E} = \{ P \in \pi | P \equiv Q | u?(y)P', \\ \and \\\\ \and \\ \;\;\; u \in \meaningof{a}, \forall z.P'\{z/y\} \in \meaningof{E\{z/b\}}\}, \and \\ \meaningof{a!E} = \{ P \in \pi | P \equiv Q | x!\langle P' \rangle, x \in \meaningof{a} P' \in \meaningof{E}\} }
\end{mathpar}

\begin{mathpar}
 \inferrule* [lab=nominal] {} {\meaningof{\quotep{E}} = \{ \quotep{P} \in \quotep{\pi} | P \in \meaningof{E} \}, \and \meaningof{\quotep{P}} = \{ \quotep{Q} \in \quotep{\pi} | P \equiv Q \} \and \\ \meaningof{@\quotep{E}} = \{ P \in \pi | P \equiv @x, x \in \meaningof{E} \}}
\end{mathpar}

\begin{eqnarray*}
  \\
  \meaningof{-} : TS \to ST
\end{eqnarray*}

\begin{eqnarray*}
  \\
  L : TS \to ST
\end{eqnarray*}

\begin{eqnarray*}
  \\
  P \models E \iff P \in \meaningof{E}
\end{eqnarray*}

\begin{eqnarray*}
  P \approx_{L} Q \iff \forall E \in L. P \models E \iff Q \models E
\end{eqnarray*}

\begin{eqnarray*}
  P \approx_{K} Q
\end{eqnarray*}

\begin{eqnarray*}
  P \approx Q
\end{eqnarray*}

$\approx_{K} = \approx = \approx_{L}$

\subsubsection{Contextual duality}

Note that contexts extend the quotation operation to a family of
operations from processes to names. Given a context, $M$, we can
define a \emph{nominal context}, $\quotep{M}$ by $\quotep{M}[P] :=
\quotep{M[P]}$. To foreshadow what is to come we observe that these
operations enjoy a duality with processes very much like the duality
between vectors and maps from vectors to scalars.

Further, because the calculus is essentially higher-order, we have a
correspondence between contexts and processes. More specifically,
given a name $x$ and a context $M$ we can construct $M^{*}_{x}$ such
that 

\begin{mathpar}
  M^{*}_{x} | \lift{x}{P} \red M[P]
\end{mathpar}

namely,

\begin{mathpar}
  M^{*}_{x} := x?(u).M[\dropn{u}]
\end{mathpar}

The dependence of $M^{*}_{x}$ on a name makes it an abstraction, 

\begin{mathpar}
  M^{*} := (x)x?(u).M[\dropn{u}]
\end{mathpar}

\subsection{Additional notation}

It will sometimes be convenient to denote the process a name
quotes. We already have the notation $x = \quotep{P}$, but it will be
convenient to introduce an alternate notation, $\procn{x}$, when we
want to emphasize the connection to the use of the name. Note that, by
virtue of name equivalence, $\quotep{\procn{x}} \nameeq x$; so, the
notation is consistent with previous definitions.

Further, because names have structure it is possible to effect
substitutions on the basis of that structure. This means we need to
upgrade our notation for substitutions, which we accomplish by
adapting comprehension notation. Thus,

\begin{mathpar}
  P\{ y / x : x \in S \}
\end{mathpar}

is interpreted to mean the process derived from P by replacing (in a
capture-avoiding manner) each occurrence of $x$ in $S$ by $y$. For example,

\begin{mathpar}
  P\{ \quotep{\procn{x}|\procn{x}} / x : x \in \freenames{P} \}
\end{mathpar}

will replace each (occurrence) of a free name $x$ in $P$ by
$\quotep{\procn{x}|\procn{x}}$.

Also, we will avail ourselves of the notation $x^{L}$ and $x^{R}$ to
denote injections of a name into disjoint copies of the name
space. There are numerous ways to accomplish this. One example can be
found in \cite{MeredithR05}. This notation overloads to vectors of
names: $\vec{x}^{\pi} := (x_{i}^{\pi} \; : \; 0 \leq i < |\vec{x}| )$ where $\pi \in \{L,R\}$.

We also use $P^{\Box} := P|\Box$.

In \cite{MeredithR05} an interpretation of the new operator is
given. It turns out that there are several possible interpretations
all enjoying the requisite algebraic properties of the operator (see
\cite{milner91polyadicpi}). We will therefore make liberal use of
$(\nu\; \vec{x})P$.

% subsection the_syntax_and_semantics_of_the_notation_system (end)   

\input{qm2pi.qmops} 

\input{qm2pi.sterngerlach} 

\input{qm2pi.metric} 

% section concurrent_process_calculi (end)

%\input{qm2pi.proofsketch}

% section proof sketch (end)

%\input{qm2pi.slviaknots} 

% section spatial logic via knots (end)

\input{qm2pi.conclusion}

% section conclusion (end)

%\input{qm2pi.dtcodes} 

% section wiring algorithm (end)

\input{qm2pi.ack} 

% section acknowledgments (end)

\newpage


\bibliographystyle{plain}   
\bibliography{../../biblios/main.bib}

\input{qm2pi.rhodetails}

\end{document}

 

% section acknowledgments (end)

\newpage


\bibliographystyle{plain}   
\bibliography{../../biblios/main.bib}

\documentclass[12pt]{llncs}
%\documentclass{jktr}

\usepackage[pdftex]{hyperref}                   
\usepackage {listings}
\usepackage {mathpartir}
\usepackage{bcprules}
%\usepackage{listings}
                       
\usepackage{graphicx} 
%\usepackage[margins=2.5cm,nohead,nofoot]{geometry}
%\usepackage{geometry}
\usepackage{amsfonts}
\usepackage{amstext}
\usepackage{latexsym}
\usepackage{amssymb}
\usepackage{color}


%\include{myPreamble}
\include{qm2pi.local} 

%\ifpdf
%\usepackage[pdftex]{graphicx}
%\else
%\usepackage{graphicx}
%\fi

 % \ifpdf
%  \usepackage{pdfsync}
%  \if


%\title{Brief Article}
%\author{David F. Snyder}
%\author{L.G. Meredith}

%\address{Dept. of Math., Texas State University--San Marcos, San Marcos, TX 78666}
       
\pagestyle{empty}


\begin{document}

\lstset{language=[Objective]Caml,frame=shadowbox}

\input{qm2pi.front}

% section front matter (end)

\input{qm2pi.intro} 
 
% section introduction (end)

% \input{qm2pi.knotations} 

% section notation (end)

\input{qm2pi.process.calculi} 

% section concurrent_process_calculi_and_spatial_logics_ (end)
    
%\input{qm2pi.knots2pi} 

%\input{qm2pi.trefoil} 

%\input{qm2pi.mainthm} 

% subsection basic_interpretation (end)

%\input{qm2pi.rho.presentation} 
\subsection{The syntax and semantics of the notation system}\label{sub:the_syntax_and_semantics_of_the_notation_system} % (fold)

We now summarize a technical presentation of the calculus that
embodies our theory of dynamics. The typical presentation of such a
calculus follows the style of giving generators and relations on
them. The grammar, below, describing term constructors, freely
generates the set of processes, $\Proc$. This set is then quotiented
by a relation known as structural congruence and it is over this set
that the notion of dynamics is expressed. This presentation is
essentially that of \cite{MeredithR05} with the addition of
polyadicity and summation. For readability we have relegated some of
the technical subtleties to an appendix.

\subsubsection{Process grammar}\label{subsub:process_grammar}

\begin{mathpar}
  \inferrule* [lab=synchronization] {} {{M} \bc \pzero \;|\; x?F \;|\; x!C }
  \and
  \inferrule* [lab=abstraction] {} {{F} \bc (x)P}
  \and
  \inferrule* [lab=concretion] {} {{C} \bc \langle Q \rangle}
  \and
  \inferrule* [lab=process] {} {{P,Q} \bc M \;| \;P|Q \;|\; @{x}}
  \and
  \inferrule* [lab=name] {} {{x} \bc \quotep{P}}
\end{mathpar} 

Note that $\vec{x}$ (resp. $\vec{P}$) denotes a vector of names
(resp. processes) of length $|\vec{x}|$ (resp. $|\vec{P}|$). We adopt
the following useful abbreviations.

\begin{mathpar}
   x?(\vec{y}).P := x.(\vec{y})P \and  x\clift{\vec{P}} := x.\clift{\vec{P}}
   \and x!(y) := \lift{x}{\dropn{y}}
   \and \Pi_{i=0}^{n-1}P_i := P_0 | \ldots | P_{n-1}
\end{mathpar}

\subsubsection{Structural congruence}

\paragraph{Free and bound names and alpha-equivalence.} At the
core of structural equivalence is alpha-equivalence which identifies
process that are the same up to a change of variable. Formally, we
recognize the distinction between free and bound names. The free names
of a process, $\freenames{P}$, may be calculated recursively as
follows:

\begin{mathpar}
\freenames{\pzero} := \emptyset
  \and \\
  \freenames{x?(y).P} := \{ x \} \cup (\freenames{P} \setminus \{ y \})
  \and 
  \freenames{x!\langle P \rangle} := \{ x \} \cup \{ P \} 
  \and \\
  \freenames{P|Q} := \freenames{P} \cup \freenames{Q}
  \and \\
  \freenames{@{x}} := \{ x \}
\end{mathpar}

$\pi$
$\quotep{\pi}$

$\freenames{-} : \pi \to \mathcal{P}(\quotep{\pi})$

\begin{eqnarray*}
  \freenames{\pzero} & := & \emptyset \\
  \freenames{x?(y).P} & := & \{ x \} \cup (\freenames{P} \setminus \{ y \}) \\
  \freenames{x!\langle P \rangle} & := & \{ x \} \cup \{ P \} \\
  \freenames{P|Q} & := & \freenames{P} \cup \freenames{Q} \\
  \freenames{\dropn{x}} & := & \{ x \}
\end{eqnarray*}

The bound names of a process, $\boundnames{P}$, are those names occurring in $P$
that are not free. For example, in $x?(y).0$, the name $x$ is free, while $y$ is bound.

\begin{mathpar}
  \inferrule* [lab=monoidal-laws] {} { P|Q \equiv Q|P \and P|0 \equiv P \and P|(Q|R) \equiv (P|Q)|R }
\end{mathpar}

\begin{mathpar}
  \inferrule* [lab=alpha-equivalence] {} { (x)P \equiv (y)P\{y/x\} \and y \not\in \freenames{P} }
\end{mathpar}

\begin{definition}
Then two processes, $P,Q$, are alpha-equivalent if $P = Q\{\vec{y}/\vec{x}\}$ for
some $\vec{x} \in \boundnames{Q},\vec{y} \in \boundnames{P}$, where $Q\{\vec{y}/\vec{x}\}$
denotes the capture-avoiding substitution of $\vec{y}$ for $\vec{x}$ in $Q$.
\end{definition}

\begin{definition}
  The {\em structural congruence} \cite{SangiorgiWalker} , $\equiv$,
  between processes is the least congruence containing
  alpha-equivalence, satisfying the abelian monoid laws
  (associativity, commutativity and $\pzero$ as identity) for parallel
  composition $|$ and for summation $+$.
\end{definition}

\subsection{Name equivalence}

We take name equivalence, written $\nameeq$, to be the smallest
equivalence relation generated by the following rules.

\begin{mathpar}
\inferrule*[lab=Quote-drop]
{ }
{ \quotep{@{x}} \nameeq x }

\inferrule*[lab=Struct-equiv]
{ P \scong Q }
{ \quotep{P} \nameeq \quotep{Q} }
\end{mathpar}

The astute reader will have noticed that the mutual recursion of names
and processes imposes a mutual recursion on alpha-equivalence and
structural equivalence via name-equivalence. Fortunately, all of this
works out pleasantly and we may calculate in the natural way, free of
concern. The reader interested in the details is referred to the
appendix \ref{appendix:rho_details}.

\subsection{Substitution}

We use $\Proc$ for the set of processes, $\QProc$ for the set of
names, and $\id{\{}\vec{y} / \vec{x} \id{\}}$ to denote partial maps,
$s : \QProc \rightarrow \QProc$. A map, $s$ lifts, uniquely, to a map
on process terms, $\widehat{s} : \Proc \rightarrow \Proc$ by the
following equations.

\begin{mathpar}
  (0) \psubstp{Q}{P} := 0 \\
  (R \juxtap S) \psubstp{Q}{P}
  :=    
  (R)\psubstp{Q}{P} \juxtap (S) \psubstp{Q}{P} \\
  (x?(y).R) \psubstp{Q}{P}    
  :=    
  (x)\substp{Q}{P} (z)\concat( (R \psubstn{z}{y}) \psubstp{Q}{P} ) \\
  (\lift{x}{R}) \psubstp{Q}{P}  
  :=
  \lift{(x)\substp{Q}{P}}{ R \psubstp{Q}{P} } \\
%   (\dropn{x})  \psubstp{Q}{P}       
%   := 
%   \left\{ 
%     \begin{array}{ccc} 
%       \dropn{\quotep{Q}} & & x \nameeq \quotep{P} \\
%       \dropn{x} & & otherwise \\
%     \end{array}
%   \right. 
  (\dropn{x})  \psubstp{Q}{P}       
  := 
  \left\{ 
    \begin{array}{ccc} 
      Q & & x \nameeq \quotep{P} \\
      \dropn{x} & & otherwise \\
    \end{array}
  \right.
\end{mathpar}
 

where

\begin{eqnarray}
  (x)\id{\{} \lpquote Q \rpquote / \lpquote P \rpquote \id{\}}            = 
  \left\{ 
    \begin{array}{ccc}
      \lpquote Q \rpquote & & x \nameeq \lpquote P \rpquote \\
      x & & otherwise \\
    \end{array}
  \right. \nonumber
\end{eqnarray}

and $z$ is chosen distinct from $\quotep{P}$, $\quotep{Q}$, the free
names in $Q$, and all the names in $R$. Our $\alpha$-equivalence will
be built in the standard way from this substitution.

\begin{remark}\label{rem:no_self_referential_names}
  One consequence of these definitions is that $\forall P. \quotep{P}
  \not\in \freenames{P}$.
\end{remark}

\subsection{ Dynamic quote: an example }

Anticipating something of what's to come, consider applying the
substitution, $\widehat{\id{\{}u / z \id{\}}}$, to the following pair
of processes, $\lift{w}{y!(z)}$ and $w[ \lpquote y!(z) \rpquote ]$.

\begin{eqnarray}
	\lift{w}{y!(z)}\widehat{\id{\{}u / z \id{\}}}
		& = &
		\lift{w}{y!(u)} \nonumber\\
	w[ \lpquote y!(z) \rpquote ] \widehat{ \id{\{}u / z \id{\}} }
		& = &
		w[ \lpquote y!(z) \rpquote ] \nonumber
\end{eqnarray}

Because the body of the process between quotes is impervious to
substitution, we get radically different answers. In fact, by
examining the first process in an input context,
e.g. $x?(z).\lift{w}{y!(z)}$, we see that the process under the lift
operator may be shaped by prefixed inputs binding a name inside it. In
this sense, the lift operator will be seen as a way to dynamically
construct processes before reifying them as names.

Finally equipped with these standard features we can present the
dynamics of the calculus.

\subsubsection{Operational semantics} 

Finally, we introduce the computational dynamics. What marks these
algebras as distinct from other more traditionally studied algebraic
structures, e.g. vector spaces or polynomial rings, is the manner in
which dynamics is captured. In traditional structures, dynamics is typically
expressed through morphisms between such structures, as in linear maps
between vector spaces or morphisms between rings. In algebras
associated with the semantics of computation, the dynamics is
expressed as part of the algebraic structure itself, through a
reduction reduction relation typically denoted by $\red$. Below, we
give a recursive presentation of this relation for the calculus used
in the encoding.

$\red \subseteq \pi \times \pi$
$\red : \pi \to \mathcal{P}(\pi)$

\begin{mathpar}
  \inferrule* [lab=Comm] { \textsf{match}( x_{src}, x_{trgt} ) } { x_{trgt}?(y)P \; | \; x_{src}!\langle {Q} \rangle \red P\{\quotep{Q}/y}\} }
  \and \\
  \inferrule* [lab=Par] {{P} \red {P}'} {{{P} | {Q}} \red {{P}' | {Q}}}
  \and
  \inferrule* [lab=Equiv]{{{P} \scong {P}'} \andalso {{P}' \red {Q}'} \andalso {{Q}' \scong {Q}}}{{P} \red {Q}}
\end{mathpar}

\begin{eqnarray*}
  match_{\equiv} (\quotep{P},\quotep{Q}) & := & P \equiv Q \\
  match_{\dagger}(\quotep{P},\quotep{Q}) & := & \forall R. P|Q \red^{*} R => R \red^{*} 0 \\
  match_{K}(\quotep{P},\quotep{Q}) & := & K \mbox{ for some context } K
\end{eqnarray*}

$u?(x)P | u!\langle Q \rangle \red P\{\quotep{Q}/x\}$

%We write $\wred$ for $\red^*$, and $P\red$ if $\exists Q $ such that $ P \red Q$.
We write $P\red$ if $\exists Q $ such that $ P \red Q$ and $P\not\red$, otherwise.

\section{Replication}

As mentioned before, it is known that replication (and hence
recursion) can be implemented in a higher-order process algebra
\cite{SangiorgiWalker}. As our first example of calculation with the
machinery thus far presented we give the construction explicitly in
the {\rhoc}.

\begin{eqnarray}
	D_{x} & := & \prefix{x}{y}{(\binpar{\outputp{x}{y}}{@{y}})} \nonumber\\
	\bangp_{x}{P} & := & \binpar{{x}!\langle{\binpar{D_{x}}{P}}\rangle}{D_{x}} \nonumber
\end{eqnarray}

\begin{eqnarray}
	\bangp_{x}{P} & & \nonumber\\
	=
	& {x}!\langle{(\prefix{x}{y}{(\outputp{x}{y} | @{y})) | P}}\rangle 
	      | \prefix{x}{y}{(\outputp{x}{y} | @{y})} & \nonumber\\
	\red
	& (\outputp{x}{y} | @{y})\substn{\quotep{(\prefix{x}{y}{(@{y} | \outputp{x}{y})) | P}}}{y} & \nonumber\\
	=
	& \outputp{x}{\quotep{(\prefix{x}{y}{(\outputp{x}{y} | @{y})) | P}}}
	  | {(\prefix{x}{y}{(\outputp{x}{y} | @{y})) | P}} & \nonumber\\
	\red
	& \ldots & \nonumber\\
	\red^*
	& P | P | \ldots & \nonumber
\end{eqnarray}

Of course, this encoding, as an implementation, runs away, unfolding
$\bangp{P}$ eagerly. A lazier and more implementable replication
operator, restricted to input-guarded processes, may be obtained as follows.

\begin{eqnarray}
\bangp{\prefix{u}{v}{P}} 
	:= 
	\binpar{\lift{x}{\prefix{u}{v}{(\binpar{D(x)}{P})}}}{D(x)} \nonumber
\end{eqnarray}

\begin{remark}
  Note that the lazier definition still does not deal with summation
  or mixed summation (i.e. sums over input and output). The reader is
  invited to construct definitions of replication that deal with these
  features. 

  Further, the definitions are parameterized in a name, $x$. Can you,
  gentle reader, make a definition that eliminates this parameter and
  guarantees no accidental interaction between the replication
  machinery and the process being replicated -- i.e. no accidental
  sharing of names used by the process to get its work done and the
  name(s) used by the replication to effect copying. This latter
  revision of the definition of replication is crucial to obtaining
  the expected identity $!!P \sim !P$.
\end{remark}

\begin{remark}\label{rem:paradoxical_combinator}
  The reader familiar with the lambda calculus will have noticed the
  similarity between $D$ and the paradoxical combinator.

  [Ed. note: the existence of this seems to suggest we have to be more
  restrictive on the set of processes and names we admit if we are to
  support no-cloning.]
\end{remark}

\subsubsection{Bisimulation}

The computational dynamics gives rise to another kind of equivalence,
the equivalence of computational behavior. As previously mentioned
this is typically captured \emph{via} some form of bisimulation.

% The notion we use in this paper is weak barbed bisimulation
% \cite{milner91polyadicpi}.

The notion we use in this paper is derived from weak barbed
bisimulation \cite{milner91polyadicpi}. 

\begin{definition}
An \emph{observation relation}, $\downarrow_{\mathcal N}$, over a set
of names, $\mathcal N$, is the smallest relation satisfying the rules
below.

\infrule[Out-barb]{y \in {\mathcal N}, \; x \nameeq y}
		  {\outputp{x}{v} \downarrow_{\mathcal N} x}
\infrule[Par-barb]{\mbox{$P\downarrow_{\mathcal N} x$ or $Q\downarrow_{\mathcal N} x$}}
		  {\binpar{P}{Q} \downarrow_{\mathcal N} x}

We write $P \Downarrow_{\mathcal N} x$ if there is $Q$ such that 
$P \wred Q$ and $Q \downarrow_{\mathcal N} x$.
\end{definition}

\begin{definition}
%\label{def.bbisim}
An  ${\mathcal N}$-\emph{barbed bisimulation} over a set of names, ${\mathcal N}$, is a symmetric binary relation 
${\mathcal S}_{\mathcal N}$ between agents such that $P\rel{S}_{\mathcal N}Q$ implies:
\begin{enumerate}
\item If $P \red P'$ then $Q \wred Q'$ and $P'\rel{S}_{\mathcal N} Q'$.
\item If $P\downarrow_{\mathcal N} x$, then $Q\Downarrow_{\mathcal N} x$.
\end{enumerate}
$P$ is ${\mathcal N}$-barbed bisimilar to $Q$, written
$P \wbbisim_{\mathcal N} Q$, if $P \rel{S}_{\mathcal N} Q$ for some ${\mathcal N}$-barbed bisimulation ${\mathcal S}_{\mathcal N}$.
\end{definition}

$\mathcal{R} \subseteq \pi \times \pi$

$P \mathcal{R} Q => \forall P'. P \red P' \Rightarrow \exists Q'. Q \red Q', P' \mathcal{R} Q'$

$P \vdash x \Rightarrow Q \vdash x$

\begin{mathpar}
  \inferrule*[lab=Out-barb]{x \nameeq y}{{y}!\langle{Q}\rangle \vdash x}
  \and
  \inferrule*[lab=Par-barb]{\mbox{$P\vdash x$ or $Q\vdash x$}}{\binpar{P}{Q} \vdash x}
\end{mathpar}

\subsubsection{Contexts}

One of the principle advantages of computational calculi like the
$\pi$-calculus is a well-defined notion of context,
contextual-equivalence and a correlation between
contextual-equivalence and notions of bisimulation. The notion of
context allows the decomposition of a process into (sub-)process and
its syntactic environment, its context. Thus, a context may be
thought of as a process with a ``hole'' (written $\Box$) in it. The
application of a context $M$ to a process $P$, written $M[P]$, is
tantamount to filling the hole in $M$ with $P$. In this paper we do
not need the full weight of this theory, but do make use of the notion
of context in the proof the main theorem. 

\begin{mathpar}
  \inferrule* [lab=summation] {} {{M_{M},M_{N}} \bc \Box \;|\; x.M_{A} \;|\; M_{M}+M_{N}}
  \and
  \inferrule* [lab=agent] {} {{M_{A}} \bc (\vec{x})M_{P} \;| \; \clift{P_0,\ldots,M_{P},\ldots,P_N}}
  \and \\
  \inferrule* [lab=process] {} {{M_{P}} \bc M_{N} \;| \;P|M_{P} }
\end{mathpar} 

\begin{mathpar}
  \inferrule* [lab=sychronization] {} {M_{N} \bc \Box \;|\; x?M_{F} \;|\; x!M_{C}}
  \and
  \inferrule* [lab=abstraction] {} {{M_{F}} \bc (x)M_{P} }
  \and
  \inferrule* [lab=concretion] {} {{M_{C}} \bc \langle M_{P} \rangle }
  \and \\
  \inferrule* [lab=process] {} {{M_{P}} \bc M_{N} \;| \;P|M_{P} }
\end{mathpar}

\begin{definition}[contextual application] Given a context $M$, and
  process $P$, we define the \emph{contextual application}, $M[P] :=
  M\{P/\Box\}$. That is, the contextual application of M to P is the
  substitution of $P$ for $\Box$ in $M$.
\end{definition}

$\meaningof{-} : L \to \mathcal{P}(\pi)$

\begin{mathpar}
  \inferrule* [lab=collection] {} {\meaningof{true} = \pi, \and \meaningof{~E} = \pi \setminus \meaningof{E}, \and \meaningof{E_{1} \& E_{2}} = \meaningof{E_{1}} \cap \meaningof{E_{2}}}
\end{mathpar}

\begin{mathpar}
  \inferrule* [lab=structure] {} {\meaningof{0} = \{ P \in \pi | P \equiv 0 \}, \and \\ \meaningof{E_1 | E_2} = \{ P \in \pi | P \equiv P_{1} | P_{2}, P_{1} \in \meaningof{E_{1}}, P_{2} \in \meaningof{E_2}\} }
\end{mathpar}

\begin{mathpar}
 \inferrule* [lab=behavior] {} {\meaningof{\langle a?b \rangle E} = \{ P \in \pi | P \equiv Q | u?(y)P', \\ \and \\\\ \and \\ \;\;\; u \in \meaningof{a}, \forall z.P'\{z/y\} \in \meaningof{E\{z/b\}}\}, \and \\ \meaningof{a!E} = \{ P \in \pi | P \equiv Q | x!\langle P' \rangle, x \in \meaningof{a} P' \in \meaningof{E}\} }
\end{mathpar}

\begin{mathpar}
 \inferrule* [lab=nominal] {} {\meaningof{\quotep{E}} = \{ \quotep{P} \in \quotep{\pi} | P \in \meaningof{E} \}, \and \meaningof{\quotep{P}} = \{ \quotep{Q} \in \quotep{\pi} | P \equiv Q \} \and \\ \meaningof{@\quotep{E}} = \{ P \in \pi | P \equiv @x, x \in \meaningof{E} \}}
\end{mathpar}

\begin{eqnarray*}
  \\
  \meaningof{-} : TS \to ST
\end{eqnarray*}

\begin{eqnarray*}
  \\
  L : TS \to ST
\end{eqnarray*}

\begin{eqnarray*}
  \\
  P \models E \iff P \in \meaningof{E}
\end{eqnarray*}

\begin{eqnarray*}
  P \approx_{L} Q \iff \forall E \in L. P \models E \iff Q \models E
\end{eqnarray*}

\begin{eqnarray*}
  P \approx_{K} Q
\end{eqnarray*}

\begin{eqnarray*}
  P \approx Q
\end{eqnarray*}

$\approx_{K} = \approx = \approx_{L}$

\subsubsection{Contextual duality}

Note that contexts extend the quotation operation to a family of
operations from processes to names. Given a context, $M$, we can
define a \emph{nominal context}, $\quotep{M}$ by $\quotep{M}[P] :=
\quotep{M[P]}$. To foreshadow what is to come we observe that these
operations enjoy a duality with processes very much like the duality
between vectors and maps from vectors to scalars.

Further, because the calculus is essentially higher-order, we have a
correspondence between contexts and processes. More specifically,
given a name $x$ and a context $M$ we can construct $M^{*}_{x}$ such
that 

\begin{mathpar}
  M^{*}_{x} | \lift{x}{P} \red M[P]
\end{mathpar}

namely,

\begin{mathpar}
  M^{*}_{x} := x?(u).M[\dropn{u}]
\end{mathpar}

The dependence of $M^{*}_{x}$ on a name makes it an abstraction, 

\begin{mathpar}
  M^{*} := (x)x?(u).M[\dropn{u}]
\end{mathpar}

\subsection{Additional notation}

It will sometimes be convenient to denote the process a name
quotes. We already have the notation $x = \quotep{P}$, but it will be
convenient to introduce an alternate notation, $\procn{x}$, when we
want to emphasize the connection to the use of the name. Note that, by
virtue of name equivalence, $\quotep{\procn{x}} \nameeq x$; so, the
notation is consistent with previous definitions.

Further, because names have structure it is possible to effect
substitutions on the basis of that structure. This means we need to
upgrade our notation for substitutions, which we accomplish by
adapting comprehension notation. Thus,

\begin{mathpar}
  P\{ y / x : x \in S \}
\end{mathpar}

is interpreted to mean the process derived from P by replacing (in a
capture-avoiding manner) each occurrence of $x$ in $S$ by $y$. For example,

\begin{mathpar}
  P\{ \quotep{\procn{x}|\procn{x}} / x : x \in \freenames{P} \}
\end{mathpar}

will replace each (occurrence) of a free name $x$ in $P$ by
$\quotep{\procn{x}|\procn{x}}$.

Also, we will avail ourselves of the notation $x^{L}$ and $x^{R}$ to
denote injections of a name into disjoint copies of the name
space. There are numerous ways to accomplish this. One example can be
found in \cite{MeredithR05}. This notation overloads to vectors of
names: $\vec{x}^{\pi} := (x_{i}^{\pi} \; : \; 0 \leq i < |\vec{x}| )$ where $\pi \in \{L,R\}$.

We also use $P^{\Box} := P|\Box$.

In \cite{MeredithR05} an interpretation of the new operator is
given. It turns out that there are several possible interpretations
all enjoying the requisite algebraic properties of the operator (see
\cite{milner91polyadicpi}). We will therefore make liberal use of
$(\nu\; \vec{x})P$.

% subsection the_syntax_and_semantics_of_the_notation_system (end)   

\input{qm2pi.qmops} 

\input{qm2pi.sterngerlach} 

\input{qm2pi.metric} 

% section concurrent_process_calculi (end)

%\input{qm2pi.proofsketch}

% section proof sketch (end)

%\input{qm2pi.slviaknots} 

% section spatial logic via knots (end)

\input{qm2pi.conclusion}

% section conclusion (end)

%\input{qm2pi.dtcodes} 

% section wiring algorithm (end)

\input{qm2pi.ack} 

% section acknowledgments (end)

\newpage


\bibliographystyle{plain}   
\bibliography{../../biblios/main.bib}

\input{qm2pi.rhodetails}

\end{document}



\end{document}



\end{document}

 

%\ifpdf
%\usepackage[pdftex]{graphicx}
%\else
%\usepackage{graphicx}
%\fi

 % \ifpdf
%  \usepackage{pdfsync}
%  \if


%\title{Brief Article}
%\author{David F. Snyder}
%\author{L.G. Meredith}

%\address{Dept. of Math., Texas State University--San Marcos, San Marcos, TX 78666}
       
\pagestyle{empty}


\begin{document}

\lstset{language=[Objective]Caml,frame=shadowbox}

\documentclass[12pt]{llncs}
%\documentclass{jktr}

\usepackage[pdftex]{hyperref}                   
\usepackage {listings}
\usepackage {mathpartir}
\usepackage{bcprules}
%\usepackage{listings}
                       
\usepackage{graphicx} 
%\usepackage[margins=2.5cm,nohead,nofoot]{geometry}
%\usepackage{geometry}
\usepackage{amsfonts}
\usepackage{amstext}
\usepackage{latexsym}
\usepackage{amssymb}
\usepackage{color}


%\include{myPreamble}
\documentclass[12pt]{llncs}
%\documentclass{jktr}

\usepackage[pdftex]{hyperref}                   
\usepackage {listings}
\usepackage {mathpartir}
\usepackage{bcprules}
%\usepackage{listings}
                       
\usepackage{graphicx} 
%\usepackage[margins=2.5cm,nohead,nofoot]{geometry}
%\usepackage{geometry}
\usepackage{amsfonts}
\usepackage{amstext}
\usepackage{latexsym}
\usepackage{amssymb}
\usepackage{color}


%\include{myPreamble}
\documentclass[12pt]{llncs}
%\documentclass{jktr}

\usepackage[pdftex]{hyperref}                   
\usepackage {listings}
\usepackage {mathpartir}
\usepackage{bcprules}
%\usepackage{listings}
                       
\usepackage{graphicx} 
%\usepackage[margins=2.5cm,nohead,nofoot]{geometry}
%\usepackage{geometry}
\usepackage{amsfonts}
\usepackage{amstext}
\usepackage{latexsym}
\usepackage{amssymb}
\usepackage{color}


%\include{myPreamble}
\include{qm2pi.local} 

%\ifpdf
%\usepackage[pdftex]{graphicx}
%\else
%\usepackage{graphicx}
%\fi

 % \ifpdf
%  \usepackage{pdfsync}
%  \if


%\title{Brief Article}
%\author{David F. Snyder}
%\author{L.G. Meredith}

%\address{Dept. of Math., Texas State University--San Marcos, San Marcos, TX 78666}
       
\pagestyle{empty}


\begin{document}

\lstset{language=[Objective]Caml,frame=shadowbox}

\input{qm2pi.front}

% section front matter (end)

\input{qm2pi.intro} 
 
% section introduction (end)

% \input{qm2pi.knotations} 

% section notation (end)

\input{qm2pi.process.calculi} 

% section concurrent_process_calculi_and_spatial_logics_ (end)
    
%\input{qm2pi.knots2pi} 

%\input{qm2pi.trefoil} 

%\input{qm2pi.mainthm} 

% subsection basic_interpretation (end)

%\input{qm2pi.rho.presentation} 
\subsection{The syntax and semantics of the notation system}\label{sub:the_syntax_and_semantics_of_the_notation_system} % (fold)

We now summarize a technical presentation of the calculus that
embodies our theory of dynamics. The typical presentation of such a
calculus follows the style of giving generators and relations on
them. The grammar, below, describing term constructors, freely
generates the set of processes, $\Proc$. This set is then quotiented
by a relation known as structural congruence and it is over this set
that the notion of dynamics is expressed. This presentation is
essentially that of \cite{MeredithR05} with the addition of
polyadicity and summation. For readability we have relegated some of
the technical subtleties to an appendix.

\subsubsection{Process grammar}\label{subsub:process_grammar}

\begin{mathpar}
  \inferrule* [lab=synchronization] {} {{M} \bc \pzero \;|\; x?F \;|\; x!C }
  \and
  \inferrule* [lab=abstraction] {} {{F} \bc (x)P}
  \and
  \inferrule* [lab=concretion] {} {{C} \bc \langle Q \rangle}
  \and
  \inferrule* [lab=process] {} {{P,Q} \bc M \;| \;P|Q \;|\; @{x}}
  \and
  \inferrule* [lab=name] {} {{x} \bc \quotep{P}}
\end{mathpar} 

Note that $\vec{x}$ (resp. $\vec{P}$) denotes a vector of names
(resp. processes) of length $|\vec{x}|$ (resp. $|\vec{P}|$). We adopt
the following useful abbreviations.

\begin{mathpar}
   x?(\vec{y}).P := x.(\vec{y})P \and  x\clift{\vec{P}} := x.\clift{\vec{P}}
   \and x!(y) := \lift{x}{\dropn{y}}
   \and \Pi_{i=0}^{n-1}P_i := P_0 | \ldots | P_{n-1}
\end{mathpar}

\subsubsection{Structural congruence}

\paragraph{Free and bound names and alpha-equivalence.} At the
core of structural equivalence is alpha-equivalence which identifies
process that are the same up to a change of variable. Formally, we
recognize the distinction between free and bound names. The free names
of a process, $\freenames{P}$, may be calculated recursively as
follows:

\begin{mathpar}
\freenames{\pzero} := \emptyset
  \and \\
  \freenames{x?(y).P} := \{ x \} \cup (\freenames{P} \setminus \{ y \})
  \and 
  \freenames{x!\langle P \rangle} := \{ x \} \cup \{ P \} 
  \and \\
  \freenames{P|Q} := \freenames{P} \cup \freenames{Q}
  \and \\
  \freenames{@{x}} := \{ x \}
\end{mathpar}

$\pi$
$\quotep{\pi}$

$\freenames{-} : \pi \to \mathcal{P}(\quotep{\pi})$

\begin{eqnarray*}
  \freenames{\pzero} & := & \emptyset \\
  \freenames{x?(y).P} & := & \{ x \} \cup (\freenames{P} \setminus \{ y \}) \\
  \freenames{x!\langle P \rangle} & := & \{ x \} \cup \{ P \} \\
  \freenames{P|Q} & := & \freenames{P} \cup \freenames{Q} \\
  \freenames{\dropn{x}} & := & \{ x \}
\end{eqnarray*}

The bound names of a process, $\boundnames{P}$, are those names occurring in $P$
that are not free. For example, in $x?(y).0$, the name $x$ is free, while $y$ is bound.

\begin{mathpar}
  \inferrule* [lab=monoidal-laws] {} { P|Q \equiv Q|P \and P|0 \equiv P \and P|(Q|R) \equiv (P|Q)|R }
\end{mathpar}

\begin{mathpar}
  \inferrule* [lab=alpha-equivalence] {} { (x)P \equiv (y)P\{y/x\} \and y \not\in \freenames{P} }
\end{mathpar}

\begin{definition}
Then two processes, $P,Q$, are alpha-equivalent if $P = Q\{\vec{y}/\vec{x}\}$ for
some $\vec{x} \in \boundnames{Q},\vec{y} \in \boundnames{P}$, where $Q\{\vec{y}/\vec{x}\}$
denotes the capture-avoiding substitution of $\vec{y}$ for $\vec{x}$ in $Q$.
\end{definition}

\begin{definition}
  The {\em structural congruence} \cite{SangiorgiWalker} , $\equiv$,
  between processes is the least congruence containing
  alpha-equivalence, satisfying the abelian monoid laws
  (associativity, commutativity and $\pzero$ as identity) for parallel
  composition $|$ and for summation $+$.
\end{definition}

\subsection{Name equivalence}

We take name equivalence, written $\nameeq$, to be the smallest
equivalence relation generated by the following rules.

\begin{mathpar}
\inferrule*[lab=Quote-drop]
{ }
{ \quotep{@{x}} \nameeq x }

\inferrule*[lab=Struct-equiv]
{ P \scong Q }
{ \quotep{P} \nameeq \quotep{Q} }
\end{mathpar}

The astute reader will have noticed that the mutual recursion of names
and processes imposes a mutual recursion on alpha-equivalence and
structural equivalence via name-equivalence. Fortunately, all of this
works out pleasantly and we may calculate in the natural way, free of
concern. The reader interested in the details is referred to the
appendix \ref{appendix:rho_details}.

\subsection{Substitution}

We use $\Proc$ for the set of processes, $\QProc$ for the set of
names, and $\id{\{}\vec{y} / \vec{x} \id{\}}$ to denote partial maps,
$s : \QProc \rightarrow \QProc$. A map, $s$ lifts, uniquely, to a map
on process terms, $\widehat{s} : \Proc \rightarrow \Proc$ by the
following equations.

\begin{mathpar}
  (0) \psubstp{Q}{P} := 0 \\
  (R \juxtap S) \psubstp{Q}{P}
  :=    
  (R)\psubstp{Q}{P} \juxtap (S) \psubstp{Q}{P} \\
  (x?(y).R) \psubstp{Q}{P}    
  :=    
  (x)\substp{Q}{P} (z)\concat( (R \psubstn{z}{y}) \psubstp{Q}{P} ) \\
  (\lift{x}{R}) \psubstp{Q}{P}  
  :=
  \lift{(x)\substp{Q}{P}}{ R \psubstp{Q}{P} } \\
%   (\dropn{x})  \psubstp{Q}{P}       
%   := 
%   \left\{ 
%     \begin{array}{ccc} 
%       \dropn{\quotep{Q}} & & x \nameeq \quotep{P} \\
%       \dropn{x} & & otherwise \\
%     \end{array}
%   \right. 
  (\dropn{x})  \psubstp{Q}{P}       
  := 
  \left\{ 
    \begin{array}{ccc} 
      Q & & x \nameeq \quotep{P} \\
      \dropn{x} & & otherwise \\
    \end{array}
  \right.
\end{mathpar}
 

where

\begin{eqnarray}
  (x)\id{\{} \lpquote Q \rpquote / \lpquote P \rpquote \id{\}}            = 
  \left\{ 
    \begin{array}{ccc}
      \lpquote Q \rpquote & & x \nameeq \lpquote P \rpquote \\
      x & & otherwise \\
    \end{array}
  \right. \nonumber
\end{eqnarray}

and $z$ is chosen distinct from $\quotep{P}$, $\quotep{Q}$, the free
names in $Q$, and all the names in $R$. Our $\alpha$-equivalence will
be built in the standard way from this substitution.

\begin{remark}\label{rem:no_self_referential_names}
  One consequence of these definitions is that $\forall P. \quotep{P}
  \not\in \freenames{P}$.
\end{remark}

\subsection{ Dynamic quote: an example }

Anticipating something of what's to come, consider applying the
substitution, $\widehat{\id{\{}u / z \id{\}}}$, to the following pair
of processes, $\lift{w}{y!(z)}$ and $w[ \lpquote y!(z) \rpquote ]$.

\begin{eqnarray}
	\lift{w}{y!(z)}\widehat{\id{\{}u / z \id{\}}}
		& = &
		\lift{w}{y!(u)} \nonumber\\
	w[ \lpquote y!(z) \rpquote ] \widehat{ \id{\{}u / z \id{\}} }
		& = &
		w[ \lpquote y!(z) \rpquote ] \nonumber
\end{eqnarray}

Because the body of the process between quotes is impervious to
substitution, we get radically different answers. In fact, by
examining the first process in an input context,
e.g. $x?(z).\lift{w}{y!(z)}$, we see that the process under the lift
operator may be shaped by prefixed inputs binding a name inside it. In
this sense, the lift operator will be seen as a way to dynamically
construct processes before reifying them as names.

Finally equipped with these standard features we can present the
dynamics of the calculus.

\subsubsection{Operational semantics} 

Finally, we introduce the computational dynamics. What marks these
algebras as distinct from other more traditionally studied algebraic
structures, e.g. vector spaces or polynomial rings, is the manner in
which dynamics is captured. In traditional structures, dynamics is typically
expressed through morphisms between such structures, as in linear maps
between vector spaces or morphisms between rings. In algebras
associated with the semantics of computation, the dynamics is
expressed as part of the algebraic structure itself, through a
reduction reduction relation typically denoted by $\red$. Below, we
give a recursive presentation of this relation for the calculus used
in the encoding.

$\red \subseteq \pi \times \pi$
$\red : \pi \to \mathcal{P}(\pi)$

\begin{mathpar}
  \inferrule* [lab=Comm] { \textsf{match}( x_{src}, x_{trgt} ) } { x_{trgt}?(y)P \; | \; x_{src}!\langle {Q} \rangle \red P\{\quotep{Q}/y}\} }
  \and \\
  \inferrule* [lab=Par] {{P} \red {P}'} {{{P} | {Q}} \red {{P}' | {Q}}}
  \and
  \inferrule* [lab=Equiv]{{{P} \scong {P}'} \andalso {{P}' \red {Q}'} \andalso {{Q}' \scong {Q}}}{{P} \red {Q}}
\end{mathpar}

\begin{eqnarray*}
  match_{\equiv} (\quotep{P},\quotep{Q}) & := & P \equiv Q \\
  match_{\dagger}(\quotep{P},\quotep{Q}) & := & \forall R. P|Q \red^{*} R => R \red^{*} 0 \\
  match_{K}(\quotep{P},\quotep{Q}) & := & K \mbox{ for some context } K
\end{eqnarray*}

$u?(x)P | u!\langle Q \rangle \red P\{\quotep{Q}/x\}$

%We write $\wred$ for $\red^*$, and $P\red$ if $\exists Q $ such that $ P \red Q$.
We write $P\red$ if $\exists Q $ such that $ P \red Q$ and $P\not\red$, otherwise.

\section{Replication}

As mentioned before, it is known that replication (and hence
recursion) can be implemented in a higher-order process algebra
\cite{SangiorgiWalker}. As our first example of calculation with the
machinery thus far presented we give the construction explicitly in
the {\rhoc}.

\begin{eqnarray}
	D_{x} & := & \prefix{x}{y}{(\binpar{\outputp{x}{y}}{@{y}})} \nonumber\\
	\bangp_{x}{P} & := & \binpar{{x}!\langle{\binpar{D_{x}}{P}}\rangle}{D_{x}} \nonumber
\end{eqnarray}

\begin{eqnarray}
	\bangp_{x}{P} & & \nonumber\\
	=
	& {x}!\langle{(\prefix{x}{y}{(\outputp{x}{y} | @{y})) | P}}\rangle 
	      | \prefix{x}{y}{(\outputp{x}{y} | @{y})} & \nonumber\\
	\red
	& (\outputp{x}{y} | @{y})\substn{\quotep{(\prefix{x}{y}{(@{y} | \outputp{x}{y})) | P}}}{y} & \nonumber\\
	=
	& \outputp{x}{\quotep{(\prefix{x}{y}{(\outputp{x}{y} | @{y})) | P}}}
	  | {(\prefix{x}{y}{(\outputp{x}{y} | @{y})) | P}} & \nonumber\\
	\red
	& \ldots & \nonumber\\
	\red^*
	& P | P | \ldots & \nonumber
\end{eqnarray}

Of course, this encoding, as an implementation, runs away, unfolding
$\bangp{P}$ eagerly. A lazier and more implementable replication
operator, restricted to input-guarded processes, may be obtained as follows.

\begin{eqnarray}
\bangp{\prefix{u}{v}{P}} 
	:= 
	\binpar{\lift{x}{\prefix{u}{v}{(\binpar{D(x)}{P})}}}{D(x)} \nonumber
\end{eqnarray}

\begin{remark}
  Note that the lazier definition still does not deal with summation
  or mixed summation (i.e. sums over input and output). The reader is
  invited to construct definitions of replication that deal with these
  features. 

  Further, the definitions are parameterized in a name, $x$. Can you,
  gentle reader, make a definition that eliminates this parameter and
  guarantees no accidental interaction between the replication
  machinery and the process being replicated -- i.e. no accidental
  sharing of names used by the process to get its work done and the
  name(s) used by the replication to effect copying. This latter
  revision of the definition of replication is crucial to obtaining
  the expected identity $!!P \sim !P$.
\end{remark}

\begin{remark}\label{rem:paradoxical_combinator}
  The reader familiar with the lambda calculus will have noticed the
  similarity between $D$ and the paradoxical combinator.

  [Ed. note: the existence of this seems to suggest we have to be more
  restrictive on the set of processes and names we admit if we are to
  support no-cloning.]
\end{remark}

\subsubsection{Bisimulation}

The computational dynamics gives rise to another kind of equivalence,
the equivalence of computational behavior. As previously mentioned
this is typically captured \emph{via} some form of bisimulation.

% The notion we use in this paper is weak barbed bisimulation
% \cite{milner91polyadicpi}.

The notion we use in this paper is derived from weak barbed
bisimulation \cite{milner91polyadicpi}. 

\begin{definition}
An \emph{observation relation}, $\downarrow_{\mathcal N}$, over a set
of names, $\mathcal N$, is the smallest relation satisfying the rules
below.

\infrule[Out-barb]{y \in {\mathcal N}, \; x \nameeq y}
		  {\outputp{x}{v} \downarrow_{\mathcal N} x}
\infrule[Par-barb]{\mbox{$P\downarrow_{\mathcal N} x$ or $Q\downarrow_{\mathcal N} x$}}
		  {\binpar{P}{Q} \downarrow_{\mathcal N} x}

We write $P \Downarrow_{\mathcal N} x$ if there is $Q$ such that 
$P \wred Q$ and $Q \downarrow_{\mathcal N} x$.
\end{definition}

\begin{definition}
%\label{def.bbisim}
An  ${\mathcal N}$-\emph{barbed bisimulation} over a set of names, ${\mathcal N}$, is a symmetric binary relation 
${\mathcal S}_{\mathcal N}$ between agents such that $P\rel{S}_{\mathcal N}Q$ implies:
\begin{enumerate}
\item If $P \red P'$ then $Q \wred Q'$ and $P'\rel{S}_{\mathcal N} Q'$.
\item If $P\downarrow_{\mathcal N} x$, then $Q\Downarrow_{\mathcal N} x$.
\end{enumerate}
$P$ is ${\mathcal N}$-barbed bisimilar to $Q$, written
$P \wbbisim_{\mathcal N} Q$, if $P \rel{S}_{\mathcal N} Q$ for some ${\mathcal N}$-barbed bisimulation ${\mathcal S}_{\mathcal N}$.
\end{definition}

$\mathcal{R} \subseteq \pi \times \pi$

$P \mathcal{R} Q => \forall P'. P \red P' \Rightarrow \exists Q'. Q \red Q', P' \mathcal{R} Q'$

$P \vdash x \Rightarrow Q \vdash x$

\begin{mathpar}
  \inferrule*[lab=Out-barb]{x \nameeq y}{{y}!\langle{Q}\rangle \vdash x}
  \and
  \inferrule*[lab=Par-barb]{\mbox{$P\vdash x$ or $Q\vdash x$}}{\binpar{P}{Q} \vdash x}
\end{mathpar}

\subsubsection{Contexts}

One of the principle advantages of computational calculi like the
$\pi$-calculus is a well-defined notion of context,
contextual-equivalence and a correlation between
contextual-equivalence and notions of bisimulation. The notion of
context allows the decomposition of a process into (sub-)process and
its syntactic environment, its context. Thus, a context may be
thought of as a process with a ``hole'' (written $\Box$) in it. The
application of a context $M$ to a process $P$, written $M[P]$, is
tantamount to filling the hole in $M$ with $P$. In this paper we do
not need the full weight of this theory, but do make use of the notion
of context in the proof the main theorem. 

\begin{mathpar}
  \inferrule* [lab=summation] {} {{M_{M},M_{N}} \bc \Box \;|\; x.M_{A} \;|\; M_{M}+M_{N}}
  \and
  \inferrule* [lab=agent] {} {{M_{A}} \bc (\vec{x})M_{P} \;| \; \clift{P_0,\ldots,M_{P},\ldots,P_N}}
  \and \\
  \inferrule* [lab=process] {} {{M_{P}} \bc M_{N} \;| \;P|M_{P} }
\end{mathpar} 

\begin{mathpar}
  \inferrule* [lab=sychronization] {} {M_{N} \bc \Box \;|\; x?M_{F} \;|\; x!M_{C}}
  \and
  \inferrule* [lab=abstraction] {} {{M_{F}} \bc (x)M_{P} }
  \and
  \inferrule* [lab=concretion] {} {{M_{C}} \bc \langle M_{P} \rangle }
  \and \\
  \inferrule* [lab=process] {} {{M_{P}} \bc M_{N} \;| \;P|M_{P} }
\end{mathpar}

\begin{definition}[contextual application] Given a context $M$, and
  process $P$, we define the \emph{contextual application}, $M[P] :=
  M\{P/\Box\}$. That is, the contextual application of M to P is the
  substitution of $P$ for $\Box$ in $M$.
\end{definition}

$\meaningof{-} : L \to \mathcal{P}(\pi)$

\begin{mathpar}
  \inferrule* [lab=collection] {} {\meaningof{true} = \pi, \and \meaningof{~E} = \pi \setminus \meaningof{E}, \and \meaningof{E_{1} \& E_{2}} = \meaningof{E_{1}} \cap \meaningof{E_{2}}}
\end{mathpar}

\begin{mathpar}
  \inferrule* [lab=structure] {} {\meaningof{0} = \{ P \in \pi | P \equiv 0 \}, \and \\ \meaningof{E_1 | E_2} = \{ P \in \pi | P \equiv P_{1} | P_{2}, P_{1} \in \meaningof{E_{1}}, P_{2} \in \meaningof{E_2}\} }
\end{mathpar}

\begin{mathpar}
 \inferrule* [lab=behavior] {} {\meaningof{\langle a?b \rangle E} = \{ P \in \pi | P \equiv Q | u?(y)P', \\ \and \\\\ \and \\ \;\;\; u \in \meaningof{a}, \forall z.P'\{z/y\} \in \meaningof{E\{z/b\}}\}, \and \\ \meaningof{a!E} = \{ P \in \pi | P \equiv Q | x!\langle P' \rangle, x \in \meaningof{a} P' \in \meaningof{E}\} }
\end{mathpar}

\begin{mathpar}
 \inferrule* [lab=nominal] {} {\meaningof{\quotep{E}} = \{ \quotep{P} \in \quotep{\pi} | P \in \meaningof{E} \}, \and \meaningof{\quotep{P}} = \{ \quotep{Q} \in \quotep{\pi} | P \equiv Q \} \and \\ \meaningof{@\quotep{E}} = \{ P \in \pi | P \equiv @x, x \in \meaningof{E} \}}
\end{mathpar}

\begin{eqnarray*}
  \\
  \meaningof{-} : TS \to ST
\end{eqnarray*}

\begin{eqnarray*}
  \\
  L : TS \to ST
\end{eqnarray*}

\begin{eqnarray*}
  \\
  P \models E \iff P \in \meaningof{E}
\end{eqnarray*}

\begin{eqnarray*}
  P \approx_{L} Q \iff \forall E \in L. P \models E \iff Q \models E
\end{eqnarray*}

\begin{eqnarray*}
  P \approx_{K} Q
\end{eqnarray*}

\begin{eqnarray*}
  P \approx Q
\end{eqnarray*}

$\approx_{K} = \approx = \approx_{L}$

\subsubsection{Contextual duality}

Note that contexts extend the quotation operation to a family of
operations from processes to names. Given a context, $M$, we can
define a \emph{nominal context}, $\quotep{M}$ by $\quotep{M}[P] :=
\quotep{M[P]}$. To foreshadow what is to come we observe that these
operations enjoy a duality with processes very much like the duality
between vectors and maps from vectors to scalars.

Further, because the calculus is essentially higher-order, we have a
correspondence between contexts and processes. More specifically,
given a name $x$ and a context $M$ we can construct $M^{*}_{x}$ such
that 

\begin{mathpar}
  M^{*}_{x} | \lift{x}{P} \red M[P]
\end{mathpar}

namely,

\begin{mathpar}
  M^{*}_{x} := x?(u).M[\dropn{u}]
\end{mathpar}

The dependence of $M^{*}_{x}$ on a name makes it an abstraction, 

\begin{mathpar}
  M^{*} := (x)x?(u).M[\dropn{u}]
\end{mathpar}

\subsection{Additional notation}

It will sometimes be convenient to denote the process a name
quotes. We already have the notation $x = \quotep{P}$, but it will be
convenient to introduce an alternate notation, $\procn{x}$, when we
want to emphasize the connection to the use of the name. Note that, by
virtue of name equivalence, $\quotep{\procn{x}} \nameeq x$; so, the
notation is consistent with previous definitions.

Further, because names have structure it is possible to effect
substitutions on the basis of that structure. This means we need to
upgrade our notation for substitutions, which we accomplish by
adapting comprehension notation. Thus,

\begin{mathpar}
  P\{ y / x : x \in S \}
\end{mathpar}

is interpreted to mean the process derived from P by replacing (in a
capture-avoiding manner) each occurrence of $x$ in $S$ by $y$. For example,

\begin{mathpar}
  P\{ \quotep{\procn{x}|\procn{x}} / x : x \in \freenames{P} \}
\end{mathpar}

will replace each (occurrence) of a free name $x$ in $P$ by
$\quotep{\procn{x}|\procn{x}}$.

Also, we will avail ourselves of the notation $x^{L}$ and $x^{R}$ to
denote injections of a name into disjoint copies of the name
space. There are numerous ways to accomplish this. One example can be
found in \cite{MeredithR05}. This notation overloads to vectors of
names: $\vec{x}^{\pi} := (x_{i}^{\pi} \; : \; 0 \leq i < |\vec{x}| )$ where $\pi \in \{L,R\}$.

We also use $P^{\Box} := P|\Box$.

In \cite{MeredithR05} an interpretation of the new operator is
given. It turns out that there are several possible interpretations
all enjoying the requisite algebraic properties of the operator (see
\cite{milner91polyadicpi}). We will therefore make liberal use of
$(\nu\; \vec{x})P$.

% subsection the_syntax_and_semantics_of_the_notation_system (end)   

\input{qm2pi.qmops} 

\input{qm2pi.sterngerlach} 

\input{qm2pi.metric} 

% section concurrent_process_calculi (end)

%\input{qm2pi.proofsketch}

% section proof sketch (end)

%\input{qm2pi.slviaknots} 

% section spatial logic via knots (end)

\input{qm2pi.conclusion}

% section conclusion (end)

%\input{qm2pi.dtcodes} 

% section wiring algorithm (end)

\input{qm2pi.ack} 

% section acknowledgments (end)

\newpage


\bibliographystyle{plain}   
\bibliography{../../biblios/main.bib}

\input{qm2pi.rhodetails}

\end{document}

 

%\ifpdf
%\usepackage[pdftex]{graphicx}
%\else
%\usepackage{graphicx}
%\fi

 % \ifpdf
%  \usepackage{pdfsync}
%  \if


%\title{Brief Article}
%\author{David F. Snyder}
%\author{L.G. Meredith}

%\address{Dept. of Math., Texas State University--San Marcos, San Marcos, TX 78666}
       
\pagestyle{empty}


\begin{document}

\lstset{language=[Objective]Caml,frame=shadowbox}

\documentclass[12pt]{llncs}
%\documentclass{jktr}

\usepackage[pdftex]{hyperref}                   
\usepackage {listings}
\usepackage {mathpartir}
\usepackage{bcprules}
%\usepackage{listings}
                       
\usepackage{graphicx} 
%\usepackage[margins=2.5cm,nohead,nofoot]{geometry}
%\usepackage{geometry}
\usepackage{amsfonts}
\usepackage{amstext}
\usepackage{latexsym}
\usepackage{amssymb}
\usepackage{color}


%\include{myPreamble}
\include{qm2pi.local} 

%\ifpdf
%\usepackage[pdftex]{graphicx}
%\else
%\usepackage{graphicx}
%\fi

 % \ifpdf
%  \usepackage{pdfsync}
%  \if


%\title{Brief Article}
%\author{David F. Snyder}
%\author{L.G. Meredith}

%\address{Dept. of Math., Texas State University--San Marcos, San Marcos, TX 78666}
       
\pagestyle{empty}


\begin{document}

\lstset{language=[Objective]Caml,frame=shadowbox}

\input{qm2pi.front}

% section front matter (end)

\input{qm2pi.intro} 
 
% section introduction (end)

% \input{qm2pi.knotations} 

% section notation (end)

\input{qm2pi.process.calculi} 

% section concurrent_process_calculi_and_spatial_logics_ (end)
    
%\input{qm2pi.knots2pi} 

%\input{qm2pi.trefoil} 

%\input{qm2pi.mainthm} 

% subsection basic_interpretation (end)

%\input{qm2pi.rho.presentation} 
\subsection{The syntax and semantics of the notation system}\label{sub:the_syntax_and_semantics_of_the_notation_system} % (fold)

We now summarize a technical presentation of the calculus that
embodies our theory of dynamics. The typical presentation of such a
calculus follows the style of giving generators and relations on
them. The grammar, below, describing term constructors, freely
generates the set of processes, $\Proc$. This set is then quotiented
by a relation known as structural congruence and it is over this set
that the notion of dynamics is expressed. This presentation is
essentially that of \cite{MeredithR05} with the addition of
polyadicity and summation. For readability we have relegated some of
the technical subtleties to an appendix.

\subsubsection{Process grammar}\label{subsub:process_grammar}

\begin{mathpar}
  \inferrule* [lab=synchronization] {} {{M} \bc \pzero \;|\; x?F \;|\; x!C }
  \and
  \inferrule* [lab=abstraction] {} {{F} \bc (x)P}
  \and
  \inferrule* [lab=concretion] {} {{C} \bc \langle Q \rangle}
  \and
  \inferrule* [lab=process] {} {{P,Q} \bc M \;| \;P|Q \;|\; @{x}}
  \and
  \inferrule* [lab=name] {} {{x} \bc \quotep{P}}
\end{mathpar} 

Note that $\vec{x}$ (resp. $\vec{P}$) denotes a vector of names
(resp. processes) of length $|\vec{x}|$ (resp. $|\vec{P}|$). We adopt
the following useful abbreviations.

\begin{mathpar}
   x?(\vec{y}).P := x.(\vec{y})P \and  x\clift{\vec{P}} := x.\clift{\vec{P}}
   \and x!(y) := \lift{x}{\dropn{y}}
   \and \Pi_{i=0}^{n-1}P_i := P_0 | \ldots | P_{n-1}
\end{mathpar}

\subsubsection{Structural congruence}

\paragraph{Free and bound names and alpha-equivalence.} At the
core of structural equivalence is alpha-equivalence which identifies
process that are the same up to a change of variable. Formally, we
recognize the distinction between free and bound names. The free names
of a process, $\freenames{P}$, may be calculated recursively as
follows:

\begin{mathpar}
\freenames{\pzero} := \emptyset
  \and \\
  \freenames{x?(y).P} := \{ x \} \cup (\freenames{P} \setminus \{ y \})
  \and 
  \freenames{x!\langle P \rangle} := \{ x \} \cup \{ P \} 
  \and \\
  \freenames{P|Q} := \freenames{P} \cup \freenames{Q}
  \and \\
  \freenames{@{x}} := \{ x \}
\end{mathpar}

$\pi$
$\quotep{\pi}$

$\freenames{-} : \pi \to \mathcal{P}(\quotep{\pi})$

\begin{eqnarray*}
  \freenames{\pzero} & := & \emptyset \\
  \freenames{x?(y).P} & := & \{ x \} \cup (\freenames{P} \setminus \{ y \}) \\
  \freenames{x!\langle P \rangle} & := & \{ x \} \cup \{ P \} \\
  \freenames{P|Q} & := & \freenames{P} \cup \freenames{Q} \\
  \freenames{\dropn{x}} & := & \{ x \}
\end{eqnarray*}

The bound names of a process, $\boundnames{P}$, are those names occurring in $P$
that are not free. For example, in $x?(y).0$, the name $x$ is free, while $y$ is bound.

\begin{mathpar}
  \inferrule* [lab=monoidal-laws] {} { P|Q \equiv Q|P \and P|0 \equiv P \and P|(Q|R) \equiv (P|Q)|R }
\end{mathpar}

\begin{mathpar}
  \inferrule* [lab=alpha-equivalence] {} { (x)P \equiv (y)P\{y/x\} \and y \not\in \freenames{P} }
\end{mathpar}

\begin{definition}
Then two processes, $P,Q$, are alpha-equivalent if $P = Q\{\vec{y}/\vec{x}\}$ for
some $\vec{x} \in \boundnames{Q},\vec{y} \in \boundnames{P}$, where $Q\{\vec{y}/\vec{x}\}$
denotes the capture-avoiding substitution of $\vec{y}$ for $\vec{x}$ in $Q$.
\end{definition}

\begin{definition}
  The {\em structural congruence} \cite{SangiorgiWalker} , $\equiv$,
  between processes is the least congruence containing
  alpha-equivalence, satisfying the abelian monoid laws
  (associativity, commutativity and $\pzero$ as identity) for parallel
  composition $|$ and for summation $+$.
\end{definition}

\subsection{Name equivalence}

We take name equivalence, written $\nameeq$, to be the smallest
equivalence relation generated by the following rules.

\begin{mathpar}
\inferrule*[lab=Quote-drop]
{ }
{ \quotep{@{x}} \nameeq x }

\inferrule*[lab=Struct-equiv]
{ P \scong Q }
{ \quotep{P} \nameeq \quotep{Q} }
\end{mathpar}

The astute reader will have noticed that the mutual recursion of names
and processes imposes a mutual recursion on alpha-equivalence and
structural equivalence via name-equivalence. Fortunately, all of this
works out pleasantly and we may calculate in the natural way, free of
concern. The reader interested in the details is referred to the
appendix \ref{appendix:rho_details}.

\subsection{Substitution}

We use $\Proc$ for the set of processes, $\QProc$ for the set of
names, and $\id{\{}\vec{y} / \vec{x} \id{\}}$ to denote partial maps,
$s : \QProc \rightarrow \QProc$. A map, $s$ lifts, uniquely, to a map
on process terms, $\widehat{s} : \Proc \rightarrow \Proc$ by the
following equations.

\begin{mathpar}
  (0) \psubstp{Q}{P} := 0 \\
  (R \juxtap S) \psubstp{Q}{P}
  :=    
  (R)\psubstp{Q}{P} \juxtap (S) \psubstp{Q}{P} \\
  (x?(y).R) \psubstp{Q}{P}    
  :=    
  (x)\substp{Q}{P} (z)\concat( (R \psubstn{z}{y}) \psubstp{Q}{P} ) \\
  (\lift{x}{R}) \psubstp{Q}{P}  
  :=
  \lift{(x)\substp{Q}{P}}{ R \psubstp{Q}{P} } \\
%   (\dropn{x})  \psubstp{Q}{P}       
%   := 
%   \left\{ 
%     \begin{array}{ccc} 
%       \dropn{\quotep{Q}} & & x \nameeq \quotep{P} \\
%       \dropn{x} & & otherwise \\
%     \end{array}
%   \right. 
  (\dropn{x})  \psubstp{Q}{P}       
  := 
  \left\{ 
    \begin{array}{ccc} 
      Q & & x \nameeq \quotep{P} \\
      \dropn{x} & & otherwise \\
    \end{array}
  \right.
\end{mathpar}
 

where

\begin{eqnarray}
  (x)\id{\{} \lpquote Q \rpquote / \lpquote P \rpquote \id{\}}            = 
  \left\{ 
    \begin{array}{ccc}
      \lpquote Q \rpquote & & x \nameeq \lpquote P \rpquote \\
      x & & otherwise \\
    \end{array}
  \right. \nonumber
\end{eqnarray}

and $z$ is chosen distinct from $\quotep{P}$, $\quotep{Q}$, the free
names in $Q$, and all the names in $R$. Our $\alpha$-equivalence will
be built in the standard way from this substitution.

\begin{remark}\label{rem:no_self_referential_names}
  One consequence of these definitions is that $\forall P. \quotep{P}
  \not\in \freenames{P}$.
\end{remark}

\subsection{ Dynamic quote: an example }

Anticipating something of what's to come, consider applying the
substitution, $\widehat{\id{\{}u / z \id{\}}}$, to the following pair
of processes, $\lift{w}{y!(z)}$ and $w[ \lpquote y!(z) \rpquote ]$.

\begin{eqnarray}
	\lift{w}{y!(z)}\widehat{\id{\{}u / z \id{\}}}
		& = &
		\lift{w}{y!(u)} \nonumber\\
	w[ \lpquote y!(z) \rpquote ] \widehat{ \id{\{}u / z \id{\}} }
		& = &
		w[ \lpquote y!(z) \rpquote ] \nonumber
\end{eqnarray}

Because the body of the process between quotes is impervious to
substitution, we get radically different answers. In fact, by
examining the first process in an input context,
e.g. $x?(z).\lift{w}{y!(z)}$, we see that the process under the lift
operator may be shaped by prefixed inputs binding a name inside it. In
this sense, the lift operator will be seen as a way to dynamically
construct processes before reifying them as names.

Finally equipped with these standard features we can present the
dynamics of the calculus.

\subsubsection{Operational semantics} 

Finally, we introduce the computational dynamics. What marks these
algebras as distinct from other more traditionally studied algebraic
structures, e.g. vector spaces or polynomial rings, is the manner in
which dynamics is captured. In traditional structures, dynamics is typically
expressed through morphisms between such structures, as in linear maps
between vector spaces or morphisms between rings. In algebras
associated with the semantics of computation, the dynamics is
expressed as part of the algebraic structure itself, through a
reduction reduction relation typically denoted by $\red$. Below, we
give a recursive presentation of this relation for the calculus used
in the encoding.

$\red \subseteq \pi \times \pi$
$\red : \pi \to \mathcal{P}(\pi)$

\begin{mathpar}
  \inferrule* [lab=Comm] { \textsf{match}( x_{src}, x_{trgt} ) } { x_{trgt}?(y)P \; | \; x_{src}!\langle {Q} \rangle \red P\{\quotep{Q}/y}\} }
  \and \\
  \inferrule* [lab=Par] {{P} \red {P}'} {{{P} | {Q}} \red {{P}' | {Q}}}
  \and
  \inferrule* [lab=Equiv]{{{P} \scong {P}'} \andalso {{P}' \red {Q}'} \andalso {{Q}' \scong {Q}}}{{P} \red {Q}}
\end{mathpar}

\begin{eqnarray*}
  match_{\equiv} (\quotep{P},\quotep{Q}) & := & P \equiv Q \\
  match_{\dagger}(\quotep{P},\quotep{Q}) & := & \forall R. P|Q \red^{*} R => R \red^{*} 0 \\
  match_{K}(\quotep{P},\quotep{Q}) & := & K \mbox{ for some context } K
\end{eqnarray*}

$u?(x)P | u!\langle Q \rangle \red P\{\quotep{Q}/x\}$

%We write $\wred$ for $\red^*$, and $P\red$ if $\exists Q $ such that $ P \red Q$.
We write $P\red$ if $\exists Q $ such that $ P \red Q$ and $P\not\red$, otherwise.

\section{Replication}

As mentioned before, it is known that replication (and hence
recursion) can be implemented in a higher-order process algebra
\cite{SangiorgiWalker}. As our first example of calculation with the
machinery thus far presented we give the construction explicitly in
the {\rhoc}.

\begin{eqnarray}
	D_{x} & := & \prefix{x}{y}{(\binpar{\outputp{x}{y}}{@{y}})} \nonumber\\
	\bangp_{x}{P} & := & \binpar{{x}!\langle{\binpar{D_{x}}{P}}\rangle}{D_{x}} \nonumber
\end{eqnarray}

\begin{eqnarray}
	\bangp_{x}{P} & & \nonumber\\
	=
	& {x}!\langle{(\prefix{x}{y}{(\outputp{x}{y} | @{y})) | P}}\rangle 
	      | \prefix{x}{y}{(\outputp{x}{y} | @{y})} & \nonumber\\
	\red
	& (\outputp{x}{y} | @{y})\substn{\quotep{(\prefix{x}{y}{(@{y} | \outputp{x}{y})) | P}}}{y} & \nonumber\\
	=
	& \outputp{x}{\quotep{(\prefix{x}{y}{(\outputp{x}{y} | @{y})) | P}}}
	  | {(\prefix{x}{y}{(\outputp{x}{y} | @{y})) | P}} & \nonumber\\
	\red
	& \ldots & \nonumber\\
	\red^*
	& P | P | \ldots & \nonumber
\end{eqnarray}

Of course, this encoding, as an implementation, runs away, unfolding
$\bangp{P}$ eagerly. A lazier and more implementable replication
operator, restricted to input-guarded processes, may be obtained as follows.

\begin{eqnarray}
\bangp{\prefix{u}{v}{P}} 
	:= 
	\binpar{\lift{x}{\prefix{u}{v}{(\binpar{D(x)}{P})}}}{D(x)} \nonumber
\end{eqnarray}

\begin{remark}
  Note that the lazier definition still does not deal with summation
  or mixed summation (i.e. sums over input and output). The reader is
  invited to construct definitions of replication that deal with these
  features. 

  Further, the definitions are parameterized in a name, $x$. Can you,
  gentle reader, make a definition that eliminates this parameter and
  guarantees no accidental interaction between the replication
  machinery and the process being replicated -- i.e. no accidental
  sharing of names used by the process to get its work done and the
  name(s) used by the replication to effect copying. This latter
  revision of the definition of replication is crucial to obtaining
  the expected identity $!!P \sim !P$.
\end{remark}

\begin{remark}\label{rem:paradoxical_combinator}
  The reader familiar with the lambda calculus will have noticed the
  similarity between $D$ and the paradoxical combinator.

  [Ed. note: the existence of this seems to suggest we have to be more
  restrictive on the set of processes and names we admit if we are to
  support no-cloning.]
\end{remark}

\subsubsection{Bisimulation}

The computational dynamics gives rise to another kind of equivalence,
the equivalence of computational behavior. As previously mentioned
this is typically captured \emph{via} some form of bisimulation.

% The notion we use in this paper is weak barbed bisimulation
% \cite{milner91polyadicpi}.

The notion we use in this paper is derived from weak barbed
bisimulation \cite{milner91polyadicpi}. 

\begin{definition}
An \emph{observation relation}, $\downarrow_{\mathcal N}$, over a set
of names, $\mathcal N$, is the smallest relation satisfying the rules
below.

\infrule[Out-barb]{y \in {\mathcal N}, \; x \nameeq y}
		  {\outputp{x}{v} \downarrow_{\mathcal N} x}
\infrule[Par-barb]{\mbox{$P\downarrow_{\mathcal N} x$ or $Q\downarrow_{\mathcal N} x$}}
		  {\binpar{P}{Q} \downarrow_{\mathcal N} x}

We write $P \Downarrow_{\mathcal N} x$ if there is $Q$ such that 
$P \wred Q$ and $Q \downarrow_{\mathcal N} x$.
\end{definition}

\begin{definition}
%\label{def.bbisim}
An  ${\mathcal N}$-\emph{barbed bisimulation} over a set of names, ${\mathcal N}$, is a symmetric binary relation 
${\mathcal S}_{\mathcal N}$ between agents such that $P\rel{S}_{\mathcal N}Q$ implies:
\begin{enumerate}
\item If $P \red P'$ then $Q \wred Q'$ and $P'\rel{S}_{\mathcal N} Q'$.
\item If $P\downarrow_{\mathcal N} x$, then $Q\Downarrow_{\mathcal N} x$.
\end{enumerate}
$P$ is ${\mathcal N}$-barbed bisimilar to $Q$, written
$P \wbbisim_{\mathcal N} Q$, if $P \rel{S}_{\mathcal N} Q$ for some ${\mathcal N}$-barbed bisimulation ${\mathcal S}_{\mathcal N}$.
\end{definition}

$\mathcal{R} \subseteq \pi \times \pi$

$P \mathcal{R} Q => \forall P'. P \red P' \Rightarrow \exists Q'. Q \red Q', P' \mathcal{R} Q'$

$P \vdash x \Rightarrow Q \vdash x$

\begin{mathpar}
  \inferrule*[lab=Out-barb]{x \nameeq y}{{y}!\langle{Q}\rangle \vdash x}
  \and
  \inferrule*[lab=Par-barb]{\mbox{$P\vdash x$ or $Q\vdash x$}}{\binpar{P}{Q} \vdash x}
\end{mathpar}

\subsubsection{Contexts}

One of the principle advantages of computational calculi like the
$\pi$-calculus is a well-defined notion of context,
contextual-equivalence and a correlation between
contextual-equivalence and notions of bisimulation. The notion of
context allows the decomposition of a process into (sub-)process and
its syntactic environment, its context. Thus, a context may be
thought of as a process with a ``hole'' (written $\Box$) in it. The
application of a context $M$ to a process $P$, written $M[P]$, is
tantamount to filling the hole in $M$ with $P$. In this paper we do
not need the full weight of this theory, but do make use of the notion
of context in the proof the main theorem. 

\begin{mathpar}
  \inferrule* [lab=summation] {} {{M_{M},M_{N}} \bc \Box \;|\; x.M_{A} \;|\; M_{M}+M_{N}}
  \and
  \inferrule* [lab=agent] {} {{M_{A}} \bc (\vec{x})M_{P} \;| \; \clift{P_0,\ldots,M_{P},\ldots,P_N}}
  \and \\
  \inferrule* [lab=process] {} {{M_{P}} \bc M_{N} \;| \;P|M_{P} }
\end{mathpar} 

\begin{mathpar}
  \inferrule* [lab=sychronization] {} {M_{N} \bc \Box \;|\; x?M_{F} \;|\; x!M_{C}}
  \and
  \inferrule* [lab=abstraction] {} {{M_{F}} \bc (x)M_{P} }
  \and
  \inferrule* [lab=concretion] {} {{M_{C}} \bc \langle M_{P} \rangle }
  \and \\
  \inferrule* [lab=process] {} {{M_{P}} \bc M_{N} \;| \;P|M_{P} }
\end{mathpar}

\begin{definition}[contextual application] Given a context $M$, and
  process $P$, we define the \emph{contextual application}, $M[P] :=
  M\{P/\Box\}$. That is, the contextual application of M to P is the
  substitution of $P$ for $\Box$ in $M$.
\end{definition}

$\meaningof{-} : L \to \mathcal{P}(\pi)$

\begin{mathpar}
  \inferrule* [lab=collection] {} {\meaningof{true} = \pi, \and \meaningof{~E} = \pi \setminus \meaningof{E}, \and \meaningof{E_{1} \& E_{2}} = \meaningof{E_{1}} \cap \meaningof{E_{2}}}
\end{mathpar}

\begin{mathpar}
  \inferrule* [lab=structure] {} {\meaningof{0} = \{ P \in \pi | P \equiv 0 \}, \and \\ \meaningof{E_1 | E_2} = \{ P \in \pi | P \equiv P_{1} | P_{2}, P_{1} \in \meaningof{E_{1}}, P_{2} \in \meaningof{E_2}\} }
\end{mathpar}

\begin{mathpar}
 \inferrule* [lab=behavior] {} {\meaningof{\langle a?b \rangle E} = \{ P \in \pi | P \equiv Q | u?(y)P', \\ \and \\\\ \and \\ \;\;\; u \in \meaningof{a}, \forall z.P'\{z/y\} \in \meaningof{E\{z/b\}}\}, \and \\ \meaningof{a!E} = \{ P \in \pi | P \equiv Q | x!\langle P' \rangle, x \in \meaningof{a} P' \in \meaningof{E}\} }
\end{mathpar}

\begin{mathpar}
 \inferrule* [lab=nominal] {} {\meaningof{\quotep{E}} = \{ \quotep{P} \in \quotep{\pi} | P \in \meaningof{E} \}, \and \meaningof{\quotep{P}} = \{ \quotep{Q} \in \quotep{\pi} | P \equiv Q \} \and \\ \meaningof{@\quotep{E}} = \{ P \in \pi | P \equiv @x, x \in \meaningof{E} \}}
\end{mathpar}

\begin{eqnarray*}
  \\
  \meaningof{-} : TS \to ST
\end{eqnarray*}

\begin{eqnarray*}
  \\
  L : TS \to ST
\end{eqnarray*}

\begin{eqnarray*}
  \\
  P \models E \iff P \in \meaningof{E}
\end{eqnarray*}

\begin{eqnarray*}
  P \approx_{L} Q \iff \forall E \in L. P \models E \iff Q \models E
\end{eqnarray*}

\begin{eqnarray*}
  P \approx_{K} Q
\end{eqnarray*}

\begin{eqnarray*}
  P \approx Q
\end{eqnarray*}

$\approx_{K} = \approx = \approx_{L}$

\subsubsection{Contextual duality}

Note that contexts extend the quotation operation to a family of
operations from processes to names. Given a context, $M$, we can
define a \emph{nominal context}, $\quotep{M}$ by $\quotep{M}[P] :=
\quotep{M[P]}$. To foreshadow what is to come we observe that these
operations enjoy a duality with processes very much like the duality
between vectors and maps from vectors to scalars.

Further, because the calculus is essentially higher-order, we have a
correspondence between contexts and processes. More specifically,
given a name $x$ and a context $M$ we can construct $M^{*}_{x}$ such
that 

\begin{mathpar}
  M^{*}_{x} | \lift{x}{P} \red M[P]
\end{mathpar}

namely,

\begin{mathpar}
  M^{*}_{x} := x?(u).M[\dropn{u}]
\end{mathpar}

The dependence of $M^{*}_{x}$ on a name makes it an abstraction, 

\begin{mathpar}
  M^{*} := (x)x?(u).M[\dropn{u}]
\end{mathpar}

\subsection{Additional notation}

It will sometimes be convenient to denote the process a name
quotes. We already have the notation $x = \quotep{P}$, but it will be
convenient to introduce an alternate notation, $\procn{x}$, when we
want to emphasize the connection to the use of the name. Note that, by
virtue of name equivalence, $\quotep{\procn{x}} \nameeq x$; so, the
notation is consistent with previous definitions.

Further, because names have structure it is possible to effect
substitutions on the basis of that structure. This means we need to
upgrade our notation for substitutions, which we accomplish by
adapting comprehension notation. Thus,

\begin{mathpar}
  P\{ y / x : x \in S \}
\end{mathpar}

is interpreted to mean the process derived from P by replacing (in a
capture-avoiding manner) each occurrence of $x$ in $S$ by $y$. For example,

\begin{mathpar}
  P\{ \quotep{\procn{x}|\procn{x}} / x : x \in \freenames{P} \}
\end{mathpar}

will replace each (occurrence) of a free name $x$ in $P$ by
$\quotep{\procn{x}|\procn{x}}$.

Also, we will avail ourselves of the notation $x^{L}$ and $x^{R}$ to
denote injections of a name into disjoint copies of the name
space. There are numerous ways to accomplish this. One example can be
found in \cite{MeredithR05}. This notation overloads to vectors of
names: $\vec{x}^{\pi} := (x_{i}^{\pi} \; : \; 0 \leq i < |\vec{x}| )$ where $\pi \in \{L,R\}$.

We also use $P^{\Box} := P|\Box$.

In \cite{MeredithR05} an interpretation of the new operator is
given. It turns out that there are several possible interpretations
all enjoying the requisite algebraic properties of the operator (see
\cite{milner91polyadicpi}). We will therefore make liberal use of
$(\nu\; \vec{x})P$.

% subsection the_syntax_and_semantics_of_the_notation_system (end)   

\input{qm2pi.qmops} 

\input{qm2pi.sterngerlach} 

\input{qm2pi.metric} 

% section concurrent_process_calculi (end)

%\input{qm2pi.proofsketch}

% section proof sketch (end)

%\input{qm2pi.slviaknots} 

% section spatial logic via knots (end)

\input{qm2pi.conclusion}

% section conclusion (end)

%\input{qm2pi.dtcodes} 

% section wiring algorithm (end)

\input{qm2pi.ack} 

% section acknowledgments (end)

\newpage


\bibliographystyle{plain}   
\bibliography{../../biblios/main.bib}

\input{qm2pi.rhodetails}

\end{document}



% section front matter (end)

\section{Introduction}\label{sec:introduction} % (fold)
In this draft of the material i am going to have to dispense with the
usual writing conventions adopted in papers on these topics. i'm going
to have adopt whatever tone i need at the time i'm writing up the
calculations. Sometimes this may be very conversational; others it may
be the barest mathematical grunts; others still it may be that i have
lifted text from one of my other papers because the exposition of some
point was better said there. i hope that my readers are not unduly put
out by this decision. i'm not doing this to flout convention or be
rebellious. i find these calculations very technically challenging. To
keep everything going technically, something has to give; i have to
let go of some cognitive burden. So, the academic writing style --
with all of its trade-offs in terms of facilitating technical
communication -- is what i'm letting go of. Perhaps subsequent drafts
can be tightened and polished, but for now, i'm going to speak as if
we were sitting together in a coffee shop with a laptop, wifi and a
pad of paper and a pencil.

So, here's what i have to say. We -- you and i, comfortably ensconced
in our coffee shop and well-equipped with our tools -- can realize and
carry out the calculations of quantum mechanics over a very different
formal theory of dynamics, a formal theory of dynamics that
corresponds to a theory of concurrent computation with
\emph{reflection}. It has the advantage that the underlying theory is
already `quantized', but supports analogues all of the continuuous
operations. Strikingly, this underlying theory has recently been
connected with a notion of metric that we can show, by calculating
together, coincides with the metric induced by the inner product.

There are a lot of reasons why you might be interested in seeing
calculations of this form. Here's why i'm interested. For the past
several centuries there has been no competitor to the ``Newtonian''
account of dynamics. As a result the predominant share of accounts of
dynamical systems and situations have had to be formulated in terms of
the Newtonian machinery. i view this as an intellectually dangerous
position to occupy. Everything, despite it's intrinsic shape, turns
into a nail to be hit with this hammer. Recently, however, the theory
of computation has matured to the point where we have candidates for
theories of dynamics that offer very different perspective on
reasoning about dynamical systems and situations. Testing these
candidates against very successful accounts of dynamical situations,
like quantum mechanics, is going to give us some sense of how mature
they are and some measure of the quality of these accounts of
dynamics.

\subsection{Summary of contributions and outline of paper}

So, we're going to develop an interpretation of the operations of
quantum mechanics normally interpreted by Hilbert spaces and
operators. We're going to do this over a theory of computation. Note
that this is very different than the usual quantum computation program
which develops notions of computation over quantum mechanics. Rather,
we are developing a story that aligns with Wheeler's slogan: It from
Bit. To do this we will first provide an account of the theory of
computation at play here. Then we will dive into a calculation-driven
interpretation of the operations of quantum mechanics.

The reason we take this approach is that -- until very recently --
there hasn't been an axiomatic account of quantum mechanics. As a
result there has been no sharp delineation of the mathematical theory
supporting interpretation of the physical theory and the physical
theory, itself. So, ambient features of the maths are free to be
exploited (or supressed) without a real accounting of their physical
relevance. There is no sharp statement ``here's the physical theory''
qua \emph{theory} and ``here's the mathematical interpretation''
enabling a judgment of how faithful the interpretation is -- apart
from experimental observation. When there is an axiomatic account we
can judge how well a given mathematical formalism supports an
interpretation of the axioms, independent of
experimentation. Likewise, we can judge how well we have captured our
physical evidence and experience with our axiomatics, independent of
any specific mathematical implementation, with accidental detail that
may or may not have physical significance. 

In lieu of a fully fleshed out and vetted axiomatic account of quantum
mechanics, interpreting the operational notions in service of modeling
physical systems will have to suffice. In other words, we are not in
the business of providing a model of Hilbert spaces and operators. We
are in the business of providing a model of quantum mechanics because
we are motivated by testing our notions of dynamics against physical
theory; and, the predictive calculations of the physical theory must
serve as the best formulation -- shy of a fully fleshed out axiomatic
account -- of the physical theory itself (as they have for scientific
theories since time immemorial). Put another way, despite a
whole-hearted commitment to an It-from-Bit ontology, we are firmly
aligned with the shut-up-and-calculate camp as the best way to obtain
results either from the physical perspective or as a quality assurance
measure of our fledgling theory of dynamics.

In detail, we present a reflective process calculus. Then we develop
intuitive correspondences between the notions available in this
calculus and the usual physical notions supporting quantum mechanical
calculations. Thus, 

\begin{table}[htp]
  \center{
    \fbox{
      \begin{tabular}{c|c}
        quantum mechanics & process calculus \\
        \hline
        scalar & name \\
        state vector & process \\
        dual & contextual duals \\
        matrix & formal sums of process-context-dual pairs \\
        orthogonality & process annihilation \\
        inner product & execution-formula + quoting
      \end{tabular}
    }
  }
  \caption{QM - process calculi correspondences}
\end{table}

Then we tighten up these intuitions to operational definitions. We
employ the Dirac notation as the best proxy we can find for an
abstract syntax of the quantum mechanical notions. The definitions we
develop put us in contact with equational constraints coming from the
theory that we demonstrate the definitions and calculations satisfy.

This puts us in a position to shut up and calculate for the
Stern-Gerlach experimental set up, showing how these predictive
calculations become calculations on processes in our theory of a
reflective process calculus.

Penultimately, we demonstrate that the notion of metric coming from
the inner product coincides with the notion of metric available from
the theory of bisimulation. This demonstration gives us the right to
think of space as arising from behavior. Finally, we consider where we
might go from the new vantage point we have obtained.

% section introduction (end) 
 
% section introduction (end)

% \documentclass[12pt]{llncs}
%\documentclass{jktr}

\usepackage[pdftex]{hyperref}                   
\usepackage {listings}
\usepackage {mathpartir}
\usepackage{bcprules}
%\usepackage{listings}
                       
\usepackage{graphicx} 
%\usepackage[margins=2.5cm,nohead,nofoot]{geometry}
%\usepackage{geometry}
\usepackage{amsfonts}
\usepackage{amstext}
\usepackage{latexsym}
\usepackage{amssymb}
\usepackage{color}


%\include{myPreamble}
\include{qm2pi.local} 

%\ifpdf
%\usepackage[pdftex]{graphicx}
%\else
%\usepackage{graphicx}
%\fi

 % \ifpdf
%  \usepackage{pdfsync}
%  \if


%\title{Brief Article}
%\author{David F. Snyder}
%\author{L.G. Meredith}

%\address{Dept. of Math., Texas State University--San Marcos, San Marcos, TX 78666}
       
\pagestyle{empty}


\begin{document}

\lstset{language=[Objective]Caml,frame=shadowbox}

\input{qm2pi.front}

% section front matter (end)

\input{qm2pi.intro} 
 
% section introduction (end)

% \input{qm2pi.knotations} 

% section notation (end)

\input{qm2pi.process.calculi} 

% section concurrent_process_calculi_and_spatial_logics_ (end)
    
%\input{qm2pi.knots2pi} 

%\input{qm2pi.trefoil} 

%\input{qm2pi.mainthm} 

% subsection basic_interpretation (end)

%\input{qm2pi.rho.presentation} 
\subsection{The syntax and semantics of the notation system}\label{sub:the_syntax_and_semantics_of_the_notation_system} % (fold)

We now summarize a technical presentation of the calculus that
embodies our theory of dynamics. The typical presentation of such a
calculus follows the style of giving generators and relations on
them. The grammar, below, describing term constructors, freely
generates the set of processes, $\Proc$. This set is then quotiented
by a relation known as structural congruence and it is over this set
that the notion of dynamics is expressed. This presentation is
essentially that of \cite{MeredithR05} with the addition of
polyadicity and summation. For readability we have relegated some of
the technical subtleties to an appendix.

\subsubsection{Process grammar}\label{subsub:process_grammar}

\begin{mathpar}
  \inferrule* [lab=synchronization] {} {{M} \bc \pzero \;|\; x?F \;|\; x!C }
  \and
  \inferrule* [lab=abstraction] {} {{F} \bc (x)P}
  \and
  \inferrule* [lab=concretion] {} {{C} \bc \langle Q \rangle}
  \and
  \inferrule* [lab=process] {} {{P,Q} \bc M \;| \;P|Q \;|\; @{x}}
  \and
  \inferrule* [lab=name] {} {{x} \bc \quotep{P}}
\end{mathpar} 

Note that $\vec{x}$ (resp. $\vec{P}$) denotes a vector of names
(resp. processes) of length $|\vec{x}|$ (resp. $|\vec{P}|$). We adopt
the following useful abbreviations.

\begin{mathpar}
   x?(\vec{y}).P := x.(\vec{y})P \and  x\clift{\vec{P}} := x.\clift{\vec{P}}
   \and x!(y) := \lift{x}{\dropn{y}}
   \and \Pi_{i=0}^{n-1}P_i := P_0 | \ldots | P_{n-1}
\end{mathpar}

\subsubsection{Structural congruence}

\paragraph{Free and bound names and alpha-equivalence.} At the
core of structural equivalence is alpha-equivalence which identifies
process that are the same up to a change of variable. Formally, we
recognize the distinction between free and bound names. The free names
of a process, $\freenames{P}$, may be calculated recursively as
follows:

\begin{mathpar}
\freenames{\pzero} := \emptyset
  \and \\
  \freenames{x?(y).P} := \{ x \} \cup (\freenames{P} \setminus \{ y \})
  \and 
  \freenames{x!\langle P \rangle} := \{ x \} \cup \{ P \} 
  \and \\
  \freenames{P|Q} := \freenames{P} \cup \freenames{Q}
  \and \\
  \freenames{@{x}} := \{ x \}
\end{mathpar}

$\pi$
$\quotep{\pi}$

$\freenames{-} : \pi \to \mathcal{P}(\quotep{\pi})$

\begin{eqnarray*}
  \freenames{\pzero} & := & \emptyset \\
  \freenames{x?(y).P} & := & \{ x \} \cup (\freenames{P} \setminus \{ y \}) \\
  \freenames{x!\langle P \rangle} & := & \{ x \} \cup \{ P \} \\
  \freenames{P|Q} & := & \freenames{P} \cup \freenames{Q} \\
  \freenames{\dropn{x}} & := & \{ x \}
\end{eqnarray*}

The bound names of a process, $\boundnames{P}$, are those names occurring in $P$
that are not free. For example, in $x?(y).0$, the name $x$ is free, while $y$ is bound.

\begin{mathpar}
  \inferrule* [lab=monoidal-laws] {} { P|Q \equiv Q|P \and P|0 \equiv P \and P|(Q|R) \equiv (P|Q)|R }
\end{mathpar}

\begin{mathpar}
  \inferrule* [lab=alpha-equivalence] {} { (x)P \equiv (y)P\{y/x\} \and y \not\in \freenames{P} }
\end{mathpar}

\begin{definition}
Then two processes, $P,Q$, are alpha-equivalent if $P = Q\{\vec{y}/\vec{x}\}$ for
some $\vec{x} \in \boundnames{Q},\vec{y} \in \boundnames{P}$, where $Q\{\vec{y}/\vec{x}\}$
denotes the capture-avoiding substitution of $\vec{y}$ for $\vec{x}$ in $Q$.
\end{definition}

\begin{definition}
  The {\em structural congruence} \cite{SangiorgiWalker} , $\equiv$,
  between processes is the least congruence containing
  alpha-equivalence, satisfying the abelian monoid laws
  (associativity, commutativity and $\pzero$ as identity) for parallel
  composition $|$ and for summation $+$.
\end{definition}

\subsection{Name equivalence}

We take name equivalence, written $\nameeq$, to be the smallest
equivalence relation generated by the following rules.

\begin{mathpar}
\inferrule*[lab=Quote-drop]
{ }
{ \quotep{@{x}} \nameeq x }

\inferrule*[lab=Struct-equiv]
{ P \scong Q }
{ \quotep{P} \nameeq \quotep{Q} }
\end{mathpar}

The astute reader will have noticed that the mutual recursion of names
and processes imposes a mutual recursion on alpha-equivalence and
structural equivalence via name-equivalence. Fortunately, all of this
works out pleasantly and we may calculate in the natural way, free of
concern. The reader interested in the details is referred to the
appendix \ref{appendix:rho_details}.

\subsection{Substitution}

We use $\Proc$ for the set of processes, $\QProc$ for the set of
names, and $\id{\{}\vec{y} / \vec{x} \id{\}}$ to denote partial maps,
$s : \QProc \rightarrow \QProc$. A map, $s$ lifts, uniquely, to a map
on process terms, $\widehat{s} : \Proc \rightarrow \Proc$ by the
following equations.

\begin{mathpar}
  (0) \psubstp{Q}{P} := 0 \\
  (R \juxtap S) \psubstp{Q}{P}
  :=    
  (R)\psubstp{Q}{P} \juxtap (S) \psubstp{Q}{P} \\
  (x?(y).R) \psubstp{Q}{P}    
  :=    
  (x)\substp{Q}{P} (z)\concat( (R \psubstn{z}{y}) \psubstp{Q}{P} ) \\
  (\lift{x}{R}) \psubstp{Q}{P}  
  :=
  \lift{(x)\substp{Q}{P}}{ R \psubstp{Q}{P} } \\
%   (\dropn{x})  \psubstp{Q}{P}       
%   := 
%   \left\{ 
%     \begin{array}{ccc} 
%       \dropn{\quotep{Q}} & & x \nameeq \quotep{P} \\
%       \dropn{x} & & otherwise \\
%     \end{array}
%   \right. 
  (\dropn{x})  \psubstp{Q}{P}       
  := 
  \left\{ 
    \begin{array}{ccc} 
      Q & & x \nameeq \quotep{P} \\
      \dropn{x} & & otherwise \\
    \end{array}
  \right.
\end{mathpar}
 

where

\begin{eqnarray}
  (x)\id{\{} \lpquote Q \rpquote / \lpquote P \rpquote \id{\}}            = 
  \left\{ 
    \begin{array}{ccc}
      \lpquote Q \rpquote & & x \nameeq \lpquote P \rpquote \\
      x & & otherwise \\
    \end{array}
  \right. \nonumber
\end{eqnarray}

and $z$ is chosen distinct from $\quotep{P}$, $\quotep{Q}$, the free
names in $Q$, and all the names in $R$. Our $\alpha$-equivalence will
be built in the standard way from this substitution.

\begin{remark}\label{rem:no_self_referential_names}
  One consequence of these definitions is that $\forall P. \quotep{P}
  \not\in \freenames{P}$.
\end{remark}

\subsection{ Dynamic quote: an example }

Anticipating something of what's to come, consider applying the
substitution, $\widehat{\id{\{}u / z \id{\}}}$, to the following pair
of processes, $\lift{w}{y!(z)}$ and $w[ \lpquote y!(z) \rpquote ]$.

\begin{eqnarray}
	\lift{w}{y!(z)}\widehat{\id{\{}u / z \id{\}}}
		& = &
		\lift{w}{y!(u)} \nonumber\\
	w[ \lpquote y!(z) \rpquote ] \widehat{ \id{\{}u / z \id{\}} }
		& = &
		w[ \lpquote y!(z) \rpquote ] \nonumber
\end{eqnarray}

Because the body of the process between quotes is impervious to
substitution, we get radically different answers. In fact, by
examining the first process in an input context,
e.g. $x?(z).\lift{w}{y!(z)}$, we see that the process under the lift
operator may be shaped by prefixed inputs binding a name inside it. In
this sense, the lift operator will be seen as a way to dynamically
construct processes before reifying them as names.

Finally equipped with these standard features we can present the
dynamics of the calculus.

\subsubsection{Operational semantics} 

Finally, we introduce the computational dynamics. What marks these
algebras as distinct from other more traditionally studied algebraic
structures, e.g. vector spaces or polynomial rings, is the manner in
which dynamics is captured. In traditional structures, dynamics is typically
expressed through morphisms between such structures, as in linear maps
between vector spaces or morphisms between rings. In algebras
associated with the semantics of computation, the dynamics is
expressed as part of the algebraic structure itself, through a
reduction reduction relation typically denoted by $\red$. Below, we
give a recursive presentation of this relation for the calculus used
in the encoding.

$\red \subseteq \pi \times \pi$
$\red : \pi \to \mathcal{P}(\pi)$

\begin{mathpar}
  \inferrule* [lab=Comm] { \textsf{match}( x_{src}, x_{trgt} ) } { x_{trgt}?(y)P \; | \; x_{src}!\langle {Q} \rangle \red P\{\quotep{Q}/y}\} }
  \and \\
  \inferrule* [lab=Par] {{P} \red {P}'} {{{P} | {Q}} \red {{P}' | {Q}}}
  \and
  \inferrule* [lab=Equiv]{{{P} \scong {P}'} \andalso {{P}' \red {Q}'} \andalso {{Q}' \scong {Q}}}{{P} \red {Q}}
\end{mathpar}

\begin{eqnarray*}
  match_{\equiv} (\quotep{P},\quotep{Q}) & := & P \equiv Q \\
  match_{\dagger}(\quotep{P},\quotep{Q}) & := & \forall R. P|Q \red^{*} R => R \red^{*} 0 \\
  match_{K}(\quotep{P},\quotep{Q}) & := & K \mbox{ for some context } K
\end{eqnarray*}

$u?(x)P | u!\langle Q \rangle \red P\{\quotep{Q}/x\}$

%We write $\wred$ for $\red^*$, and $P\red$ if $\exists Q $ such that $ P \red Q$.
We write $P\red$ if $\exists Q $ such that $ P \red Q$ and $P\not\red$, otherwise.

\section{Replication}

As mentioned before, it is known that replication (and hence
recursion) can be implemented in a higher-order process algebra
\cite{SangiorgiWalker}. As our first example of calculation with the
machinery thus far presented we give the construction explicitly in
the {\rhoc}.

\begin{eqnarray}
	D_{x} & := & \prefix{x}{y}{(\binpar{\outputp{x}{y}}{@{y}})} \nonumber\\
	\bangp_{x}{P} & := & \binpar{{x}!\langle{\binpar{D_{x}}{P}}\rangle}{D_{x}} \nonumber
\end{eqnarray}

\begin{eqnarray}
	\bangp_{x}{P} & & \nonumber\\
	=
	& {x}!\langle{(\prefix{x}{y}{(\outputp{x}{y} | @{y})) | P}}\rangle 
	      | \prefix{x}{y}{(\outputp{x}{y} | @{y})} & \nonumber\\
	\red
	& (\outputp{x}{y} | @{y})\substn{\quotep{(\prefix{x}{y}{(@{y} | \outputp{x}{y})) | P}}}{y} & \nonumber\\
	=
	& \outputp{x}{\quotep{(\prefix{x}{y}{(\outputp{x}{y} | @{y})) | P}}}
	  | {(\prefix{x}{y}{(\outputp{x}{y} | @{y})) | P}} & \nonumber\\
	\red
	& \ldots & \nonumber\\
	\red^*
	& P | P | \ldots & \nonumber
\end{eqnarray}

Of course, this encoding, as an implementation, runs away, unfolding
$\bangp{P}$ eagerly. A lazier and more implementable replication
operator, restricted to input-guarded processes, may be obtained as follows.

\begin{eqnarray}
\bangp{\prefix{u}{v}{P}} 
	:= 
	\binpar{\lift{x}{\prefix{u}{v}{(\binpar{D(x)}{P})}}}{D(x)} \nonumber
\end{eqnarray}

\begin{remark}
  Note that the lazier definition still does not deal with summation
  or mixed summation (i.e. sums over input and output). The reader is
  invited to construct definitions of replication that deal with these
  features. 

  Further, the definitions are parameterized in a name, $x$. Can you,
  gentle reader, make a definition that eliminates this parameter and
  guarantees no accidental interaction between the replication
  machinery and the process being replicated -- i.e. no accidental
  sharing of names used by the process to get its work done and the
  name(s) used by the replication to effect copying. This latter
  revision of the definition of replication is crucial to obtaining
  the expected identity $!!P \sim !P$.
\end{remark}

\begin{remark}\label{rem:paradoxical_combinator}
  The reader familiar with the lambda calculus will have noticed the
  similarity between $D$ and the paradoxical combinator.

  [Ed. note: the existence of this seems to suggest we have to be more
  restrictive on the set of processes and names we admit if we are to
  support no-cloning.]
\end{remark}

\subsubsection{Bisimulation}

The computational dynamics gives rise to another kind of equivalence,
the equivalence of computational behavior. As previously mentioned
this is typically captured \emph{via} some form of bisimulation.

% The notion we use in this paper is weak barbed bisimulation
% \cite{milner91polyadicpi}.

The notion we use in this paper is derived from weak barbed
bisimulation \cite{milner91polyadicpi}. 

\begin{definition}
An \emph{observation relation}, $\downarrow_{\mathcal N}$, over a set
of names, $\mathcal N$, is the smallest relation satisfying the rules
below.

\infrule[Out-barb]{y \in {\mathcal N}, \; x \nameeq y}
		  {\outputp{x}{v} \downarrow_{\mathcal N} x}
\infrule[Par-barb]{\mbox{$P\downarrow_{\mathcal N} x$ or $Q\downarrow_{\mathcal N} x$}}
		  {\binpar{P}{Q} \downarrow_{\mathcal N} x}

We write $P \Downarrow_{\mathcal N} x$ if there is $Q$ such that 
$P \wred Q$ and $Q \downarrow_{\mathcal N} x$.
\end{definition}

\begin{definition}
%\label{def.bbisim}
An  ${\mathcal N}$-\emph{barbed bisimulation} over a set of names, ${\mathcal N}$, is a symmetric binary relation 
${\mathcal S}_{\mathcal N}$ between agents such that $P\rel{S}_{\mathcal N}Q$ implies:
\begin{enumerate}
\item If $P \red P'$ then $Q \wred Q'$ and $P'\rel{S}_{\mathcal N} Q'$.
\item If $P\downarrow_{\mathcal N} x$, then $Q\Downarrow_{\mathcal N} x$.
\end{enumerate}
$P$ is ${\mathcal N}$-barbed bisimilar to $Q$, written
$P \wbbisim_{\mathcal N} Q$, if $P \rel{S}_{\mathcal N} Q$ for some ${\mathcal N}$-barbed bisimulation ${\mathcal S}_{\mathcal N}$.
\end{definition}

$\mathcal{R} \subseteq \pi \times \pi$

$P \mathcal{R} Q => \forall P'. P \red P' \Rightarrow \exists Q'. Q \red Q', P' \mathcal{R} Q'$

$P \vdash x \Rightarrow Q \vdash x$

\begin{mathpar}
  \inferrule*[lab=Out-barb]{x \nameeq y}{{y}!\langle{Q}\rangle \vdash x}
  \and
  \inferrule*[lab=Par-barb]{\mbox{$P\vdash x$ or $Q\vdash x$}}{\binpar{P}{Q} \vdash x}
\end{mathpar}

\subsubsection{Contexts}

One of the principle advantages of computational calculi like the
$\pi$-calculus is a well-defined notion of context,
contextual-equivalence and a correlation between
contextual-equivalence and notions of bisimulation. The notion of
context allows the decomposition of a process into (sub-)process and
its syntactic environment, its context. Thus, a context may be
thought of as a process with a ``hole'' (written $\Box$) in it. The
application of a context $M$ to a process $P$, written $M[P]$, is
tantamount to filling the hole in $M$ with $P$. In this paper we do
not need the full weight of this theory, but do make use of the notion
of context in the proof the main theorem. 

\begin{mathpar}
  \inferrule* [lab=summation] {} {{M_{M},M_{N}} \bc \Box \;|\; x.M_{A} \;|\; M_{M}+M_{N}}
  \and
  \inferrule* [lab=agent] {} {{M_{A}} \bc (\vec{x})M_{P} \;| \; \clift{P_0,\ldots,M_{P},\ldots,P_N}}
  \and \\
  \inferrule* [lab=process] {} {{M_{P}} \bc M_{N} \;| \;P|M_{P} }
\end{mathpar} 

\begin{mathpar}
  \inferrule* [lab=sychronization] {} {M_{N} \bc \Box \;|\; x?M_{F} \;|\; x!M_{C}}
  \and
  \inferrule* [lab=abstraction] {} {{M_{F}} \bc (x)M_{P} }
  \and
  \inferrule* [lab=concretion] {} {{M_{C}} \bc \langle M_{P} \rangle }
  \and \\
  \inferrule* [lab=process] {} {{M_{P}} \bc M_{N} \;| \;P|M_{P} }
\end{mathpar}

\begin{definition}[contextual application] Given a context $M$, and
  process $P$, we define the \emph{contextual application}, $M[P] :=
  M\{P/\Box\}$. That is, the contextual application of M to P is the
  substitution of $P$ for $\Box$ in $M$.
\end{definition}

$\meaningof{-} : L \to \mathcal{P}(\pi)$

\begin{mathpar}
  \inferrule* [lab=collection] {} {\meaningof{true} = \pi, \and \meaningof{~E} = \pi \setminus \meaningof{E}, \and \meaningof{E_{1} \& E_{2}} = \meaningof{E_{1}} \cap \meaningof{E_{2}}}
\end{mathpar}

\begin{mathpar}
  \inferrule* [lab=structure] {} {\meaningof{0} = \{ P \in \pi | P \equiv 0 \}, \and \\ \meaningof{E_1 | E_2} = \{ P \in \pi | P \equiv P_{1} | P_{2}, P_{1} \in \meaningof{E_{1}}, P_{2} \in \meaningof{E_2}\} }
\end{mathpar}

\begin{mathpar}
 \inferrule* [lab=behavior] {} {\meaningof{\langle a?b \rangle E} = \{ P \in \pi | P \equiv Q | u?(y)P', \\ \and \\\\ \and \\ \;\;\; u \in \meaningof{a}, \forall z.P'\{z/y\} \in \meaningof{E\{z/b\}}\}, \and \\ \meaningof{a!E} = \{ P \in \pi | P \equiv Q | x!\langle P' \rangle, x \in \meaningof{a} P' \in \meaningof{E}\} }
\end{mathpar}

\begin{mathpar}
 \inferrule* [lab=nominal] {} {\meaningof{\quotep{E}} = \{ \quotep{P} \in \quotep{\pi} | P \in \meaningof{E} \}, \and \meaningof{\quotep{P}} = \{ \quotep{Q} \in \quotep{\pi} | P \equiv Q \} \and \\ \meaningof{@\quotep{E}} = \{ P \in \pi | P \equiv @x, x \in \meaningof{E} \}}
\end{mathpar}

\begin{eqnarray*}
  \\
  \meaningof{-} : TS \to ST
\end{eqnarray*}

\begin{eqnarray*}
  \\
  L : TS \to ST
\end{eqnarray*}

\begin{eqnarray*}
  \\
  P \models E \iff P \in \meaningof{E}
\end{eqnarray*}

\begin{eqnarray*}
  P \approx_{L} Q \iff \forall E \in L. P \models E \iff Q \models E
\end{eqnarray*}

\begin{eqnarray*}
  P \approx_{K} Q
\end{eqnarray*}

\begin{eqnarray*}
  P \approx Q
\end{eqnarray*}

$\approx_{K} = \approx = \approx_{L}$

\subsubsection{Contextual duality}

Note that contexts extend the quotation operation to a family of
operations from processes to names. Given a context, $M$, we can
define a \emph{nominal context}, $\quotep{M}$ by $\quotep{M}[P] :=
\quotep{M[P]}$. To foreshadow what is to come we observe that these
operations enjoy a duality with processes very much like the duality
between vectors and maps from vectors to scalars.

Further, because the calculus is essentially higher-order, we have a
correspondence between contexts and processes. More specifically,
given a name $x$ and a context $M$ we can construct $M^{*}_{x}$ such
that 

\begin{mathpar}
  M^{*}_{x} | \lift{x}{P} \red M[P]
\end{mathpar}

namely,

\begin{mathpar}
  M^{*}_{x} := x?(u).M[\dropn{u}]
\end{mathpar}

The dependence of $M^{*}_{x}$ on a name makes it an abstraction, 

\begin{mathpar}
  M^{*} := (x)x?(u).M[\dropn{u}]
\end{mathpar}

\subsection{Additional notation}

It will sometimes be convenient to denote the process a name
quotes. We already have the notation $x = \quotep{P}$, but it will be
convenient to introduce an alternate notation, $\procn{x}$, when we
want to emphasize the connection to the use of the name. Note that, by
virtue of name equivalence, $\quotep{\procn{x}} \nameeq x$; so, the
notation is consistent with previous definitions.

Further, because names have structure it is possible to effect
substitutions on the basis of that structure. This means we need to
upgrade our notation for substitutions, which we accomplish by
adapting comprehension notation. Thus,

\begin{mathpar}
  P\{ y / x : x \in S \}
\end{mathpar}

is interpreted to mean the process derived from P by replacing (in a
capture-avoiding manner) each occurrence of $x$ in $S$ by $y$. For example,

\begin{mathpar}
  P\{ \quotep{\procn{x}|\procn{x}} / x : x \in \freenames{P} \}
\end{mathpar}

will replace each (occurrence) of a free name $x$ in $P$ by
$\quotep{\procn{x}|\procn{x}}$.

Also, we will avail ourselves of the notation $x^{L}$ and $x^{R}$ to
denote injections of a name into disjoint copies of the name
space. There are numerous ways to accomplish this. One example can be
found in \cite{MeredithR05}. This notation overloads to vectors of
names: $\vec{x}^{\pi} := (x_{i}^{\pi} \; : \; 0 \leq i < |\vec{x}| )$ where $\pi \in \{L,R\}$.

We also use $P^{\Box} := P|\Box$.

In \cite{MeredithR05} an interpretation of the new operator is
given. It turns out that there are several possible interpretations
all enjoying the requisite algebraic properties of the operator (see
\cite{milner91polyadicpi}). We will therefore make liberal use of
$(\nu\; \vec{x})P$.

% subsection the_syntax_and_semantics_of_the_notation_system (end)   

\input{qm2pi.qmops} 

\input{qm2pi.sterngerlach} 

\input{qm2pi.metric} 

% section concurrent_process_calculi (end)

%\input{qm2pi.proofsketch}

% section proof sketch (end)

%\input{qm2pi.slviaknots} 

% section spatial logic via knots (end)

\input{qm2pi.conclusion}

% section conclusion (end)

%\input{qm2pi.dtcodes} 

% section wiring algorithm (end)

\input{qm2pi.ack} 

% section acknowledgments (end)

\newpage


\bibliographystyle{plain}   
\bibliography{../../biblios/main.bib}

\input{qm2pi.rhodetails}

\end{document}

 

% section notation (end)

\input{qm2pi.process.calculi} 

% section concurrent_process_calculi_and_spatial_logics_ (end)
    
%\documentclass[12pt]{llncs}
%\documentclass{jktr}

\usepackage[pdftex]{hyperref}                   
\usepackage {listings}
\usepackage {mathpartir}
\usepackage{bcprules}
%\usepackage{listings}
                       
\usepackage{graphicx} 
%\usepackage[margins=2.5cm,nohead,nofoot]{geometry}
%\usepackage{geometry}
\usepackage{amsfonts}
\usepackage{amstext}
\usepackage{latexsym}
\usepackage{amssymb}
\usepackage{color}


%\include{myPreamble}
\include{qm2pi.local} 

%\ifpdf
%\usepackage[pdftex]{graphicx}
%\else
%\usepackage{graphicx}
%\fi

 % \ifpdf
%  \usepackage{pdfsync}
%  \if


%\title{Brief Article}
%\author{David F. Snyder}
%\author{L.G. Meredith}

%\address{Dept. of Math., Texas State University--San Marcos, San Marcos, TX 78666}
       
\pagestyle{empty}


\begin{document}

\lstset{language=[Objective]Caml,frame=shadowbox}

\input{qm2pi.front}

% section front matter (end)

\input{qm2pi.intro} 
 
% section introduction (end)

% \input{qm2pi.knotations} 

% section notation (end)

\input{qm2pi.process.calculi} 

% section concurrent_process_calculi_and_spatial_logics_ (end)
    
%\input{qm2pi.knots2pi} 

%\input{qm2pi.trefoil} 

%\input{qm2pi.mainthm} 

% subsection basic_interpretation (end)

%\input{qm2pi.rho.presentation} 
\subsection{The syntax and semantics of the notation system}\label{sub:the_syntax_and_semantics_of_the_notation_system} % (fold)

We now summarize a technical presentation of the calculus that
embodies our theory of dynamics. The typical presentation of such a
calculus follows the style of giving generators and relations on
them. The grammar, below, describing term constructors, freely
generates the set of processes, $\Proc$. This set is then quotiented
by a relation known as structural congruence and it is over this set
that the notion of dynamics is expressed. This presentation is
essentially that of \cite{MeredithR05} with the addition of
polyadicity and summation. For readability we have relegated some of
the technical subtleties to an appendix.

\subsubsection{Process grammar}\label{subsub:process_grammar}

\begin{mathpar}
  \inferrule* [lab=synchronization] {} {{M} \bc \pzero \;|\; x?F \;|\; x!C }
  \and
  \inferrule* [lab=abstraction] {} {{F} \bc (x)P}
  \and
  \inferrule* [lab=concretion] {} {{C} \bc \langle Q \rangle}
  \and
  \inferrule* [lab=process] {} {{P,Q} \bc M \;| \;P|Q \;|\; @{x}}
  \and
  \inferrule* [lab=name] {} {{x} \bc \quotep{P}}
\end{mathpar} 

Note that $\vec{x}$ (resp. $\vec{P}$) denotes a vector of names
(resp. processes) of length $|\vec{x}|$ (resp. $|\vec{P}|$). We adopt
the following useful abbreviations.

\begin{mathpar}
   x?(\vec{y}).P := x.(\vec{y})P \and  x\clift{\vec{P}} := x.\clift{\vec{P}}
   \and x!(y) := \lift{x}{\dropn{y}}
   \and \Pi_{i=0}^{n-1}P_i := P_0 | \ldots | P_{n-1}
\end{mathpar}

\subsubsection{Structural congruence}

\paragraph{Free and bound names and alpha-equivalence.} At the
core of structural equivalence is alpha-equivalence which identifies
process that are the same up to a change of variable. Formally, we
recognize the distinction between free and bound names. The free names
of a process, $\freenames{P}$, may be calculated recursively as
follows:

\begin{mathpar}
\freenames{\pzero} := \emptyset
  \and \\
  \freenames{x?(y).P} := \{ x \} \cup (\freenames{P} \setminus \{ y \})
  \and 
  \freenames{x!\langle P \rangle} := \{ x \} \cup \{ P \} 
  \and \\
  \freenames{P|Q} := \freenames{P} \cup \freenames{Q}
  \and \\
  \freenames{@{x}} := \{ x \}
\end{mathpar}

$\pi$
$\quotep{\pi}$

$\freenames{-} : \pi \to \mathcal{P}(\quotep{\pi})$

\begin{eqnarray*}
  \freenames{\pzero} & := & \emptyset \\
  \freenames{x?(y).P} & := & \{ x \} \cup (\freenames{P} \setminus \{ y \}) \\
  \freenames{x!\langle P \rangle} & := & \{ x \} \cup \{ P \} \\
  \freenames{P|Q} & := & \freenames{P} \cup \freenames{Q} \\
  \freenames{\dropn{x}} & := & \{ x \}
\end{eqnarray*}

The bound names of a process, $\boundnames{P}$, are those names occurring in $P$
that are not free. For example, in $x?(y).0$, the name $x$ is free, while $y$ is bound.

\begin{mathpar}
  \inferrule* [lab=monoidal-laws] {} { P|Q \equiv Q|P \and P|0 \equiv P \and P|(Q|R) \equiv (P|Q)|R }
\end{mathpar}

\begin{mathpar}
  \inferrule* [lab=alpha-equivalence] {} { (x)P \equiv (y)P\{y/x\} \and y \not\in \freenames{P} }
\end{mathpar}

\begin{definition}
Then two processes, $P,Q$, are alpha-equivalent if $P = Q\{\vec{y}/\vec{x}\}$ for
some $\vec{x} \in \boundnames{Q},\vec{y} \in \boundnames{P}$, where $Q\{\vec{y}/\vec{x}\}$
denotes the capture-avoiding substitution of $\vec{y}$ for $\vec{x}$ in $Q$.
\end{definition}

\begin{definition}
  The {\em structural congruence} \cite{SangiorgiWalker} , $\equiv$,
  between processes is the least congruence containing
  alpha-equivalence, satisfying the abelian monoid laws
  (associativity, commutativity and $\pzero$ as identity) for parallel
  composition $|$ and for summation $+$.
\end{definition}

\subsection{Name equivalence}

We take name equivalence, written $\nameeq$, to be the smallest
equivalence relation generated by the following rules.

\begin{mathpar}
\inferrule*[lab=Quote-drop]
{ }
{ \quotep{@{x}} \nameeq x }

\inferrule*[lab=Struct-equiv]
{ P \scong Q }
{ \quotep{P} \nameeq \quotep{Q} }
\end{mathpar}

The astute reader will have noticed that the mutual recursion of names
and processes imposes a mutual recursion on alpha-equivalence and
structural equivalence via name-equivalence. Fortunately, all of this
works out pleasantly and we may calculate in the natural way, free of
concern. The reader interested in the details is referred to the
appendix \ref{appendix:rho_details}.

\subsection{Substitution}

We use $\Proc$ for the set of processes, $\QProc$ for the set of
names, and $\id{\{}\vec{y} / \vec{x} \id{\}}$ to denote partial maps,
$s : \QProc \rightarrow \QProc$. A map, $s$ lifts, uniquely, to a map
on process terms, $\widehat{s} : \Proc \rightarrow \Proc$ by the
following equations.

\begin{mathpar}
  (0) \psubstp{Q}{P} := 0 \\
  (R \juxtap S) \psubstp{Q}{P}
  :=    
  (R)\psubstp{Q}{P} \juxtap (S) \psubstp{Q}{P} \\
  (x?(y).R) \psubstp{Q}{P}    
  :=    
  (x)\substp{Q}{P} (z)\concat( (R \psubstn{z}{y}) \psubstp{Q}{P} ) \\
  (\lift{x}{R}) \psubstp{Q}{P}  
  :=
  \lift{(x)\substp{Q}{P}}{ R \psubstp{Q}{P} } \\
%   (\dropn{x})  \psubstp{Q}{P}       
%   := 
%   \left\{ 
%     \begin{array}{ccc} 
%       \dropn{\quotep{Q}} & & x \nameeq \quotep{P} \\
%       \dropn{x} & & otherwise \\
%     \end{array}
%   \right. 
  (\dropn{x})  \psubstp{Q}{P}       
  := 
  \left\{ 
    \begin{array}{ccc} 
      Q & & x \nameeq \quotep{P} \\
      \dropn{x} & & otherwise \\
    \end{array}
  \right.
\end{mathpar}
 

where

\begin{eqnarray}
  (x)\id{\{} \lpquote Q \rpquote / \lpquote P \rpquote \id{\}}            = 
  \left\{ 
    \begin{array}{ccc}
      \lpquote Q \rpquote & & x \nameeq \lpquote P \rpquote \\
      x & & otherwise \\
    \end{array}
  \right. \nonumber
\end{eqnarray}

and $z$ is chosen distinct from $\quotep{P}$, $\quotep{Q}$, the free
names in $Q$, and all the names in $R$. Our $\alpha$-equivalence will
be built in the standard way from this substitution.

\begin{remark}\label{rem:no_self_referential_names}
  One consequence of these definitions is that $\forall P. \quotep{P}
  \not\in \freenames{P}$.
\end{remark}

\subsection{ Dynamic quote: an example }

Anticipating something of what's to come, consider applying the
substitution, $\widehat{\id{\{}u / z \id{\}}}$, to the following pair
of processes, $\lift{w}{y!(z)}$ and $w[ \lpquote y!(z) \rpquote ]$.

\begin{eqnarray}
	\lift{w}{y!(z)}\widehat{\id{\{}u / z \id{\}}}
		& = &
		\lift{w}{y!(u)} \nonumber\\
	w[ \lpquote y!(z) \rpquote ] \widehat{ \id{\{}u / z \id{\}} }
		& = &
		w[ \lpquote y!(z) \rpquote ] \nonumber
\end{eqnarray}

Because the body of the process between quotes is impervious to
substitution, we get radically different answers. In fact, by
examining the first process in an input context,
e.g. $x?(z).\lift{w}{y!(z)}$, we see that the process under the lift
operator may be shaped by prefixed inputs binding a name inside it. In
this sense, the lift operator will be seen as a way to dynamically
construct processes before reifying them as names.

Finally equipped with these standard features we can present the
dynamics of the calculus.

\subsubsection{Operational semantics} 

Finally, we introduce the computational dynamics. What marks these
algebras as distinct from other more traditionally studied algebraic
structures, e.g. vector spaces or polynomial rings, is the manner in
which dynamics is captured. In traditional structures, dynamics is typically
expressed through morphisms between such structures, as in linear maps
between vector spaces or morphisms between rings. In algebras
associated with the semantics of computation, the dynamics is
expressed as part of the algebraic structure itself, through a
reduction reduction relation typically denoted by $\red$. Below, we
give a recursive presentation of this relation for the calculus used
in the encoding.

$\red \subseteq \pi \times \pi$
$\red : \pi \to \mathcal{P}(\pi)$

\begin{mathpar}
  \inferrule* [lab=Comm] { \textsf{match}( x_{src}, x_{trgt} ) } { x_{trgt}?(y)P \; | \; x_{src}!\langle {Q} \rangle \red P\{\quotep{Q}/y}\} }
  \and \\
  \inferrule* [lab=Par] {{P} \red {P}'} {{{P} | {Q}} \red {{P}' | {Q}}}
  \and
  \inferrule* [lab=Equiv]{{{P} \scong {P}'} \andalso {{P}' \red {Q}'} \andalso {{Q}' \scong {Q}}}{{P} \red {Q}}
\end{mathpar}

\begin{eqnarray*}
  match_{\equiv} (\quotep{P},\quotep{Q}) & := & P \equiv Q \\
  match_{\dagger}(\quotep{P},\quotep{Q}) & := & \forall R. P|Q \red^{*} R => R \red^{*} 0 \\
  match_{K}(\quotep{P},\quotep{Q}) & := & K \mbox{ for some context } K
\end{eqnarray*}

$u?(x)P | u!\langle Q \rangle \red P\{\quotep{Q}/x\}$

%We write $\wred$ for $\red^*$, and $P\red$ if $\exists Q $ such that $ P \red Q$.
We write $P\red$ if $\exists Q $ such that $ P \red Q$ and $P\not\red$, otherwise.

\section{Replication}

As mentioned before, it is known that replication (and hence
recursion) can be implemented in a higher-order process algebra
\cite{SangiorgiWalker}. As our first example of calculation with the
machinery thus far presented we give the construction explicitly in
the {\rhoc}.

\begin{eqnarray}
	D_{x} & := & \prefix{x}{y}{(\binpar{\outputp{x}{y}}{@{y}})} \nonumber\\
	\bangp_{x}{P} & := & \binpar{{x}!\langle{\binpar{D_{x}}{P}}\rangle}{D_{x}} \nonumber
\end{eqnarray}

\begin{eqnarray}
	\bangp_{x}{P} & & \nonumber\\
	=
	& {x}!\langle{(\prefix{x}{y}{(\outputp{x}{y} | @{y})) | P}}\rangle 
	      | \prefix{x}{y}{(\outputp{x}{y} | @{y})} & \nonumber\\
	\red
	& (\outputp{x}{y} | @{y})\substn{\quotep{(\prefix{x}{y}{(@{y} | \outputp{x}{y})) | P}}}{y} & \nonumber\\
	=
	& \outputp{x}{\quotep{(\prefix{x}{y}{(\outputp{x}{y} | @{y})) | P}}}
	  | {(\prefix{x}{y}{(\outputp{x}{y} | @{y})) | P}} & \nonumber\\
	\red
	& \ldots & \nonumber\\
	\red^*
	& P | P | \ldots & \nonumber
\end{eqnarray}

Of course, this encoding, as an implementation, runs away, unfolding
$\bangp{P}$ eagerly. A lazier and more implementable replication
operator, restricted to input-guarded processes, may be obtained as follows.

\begin{eqnarray}
\bangp{\prefix{u}{v}{P}} 
	:= 
	\binpar{\lift{x}{\prefix{u}{v}{(\binpar{D(x)}{P})}}}{D(x)} \nonumber
\end{eqnarray}

\begin{remark}
  Note that the lazier definition still does not deal with summation
  or mixed summation (i.e. sums over input and output). The reader is
  invited to construct definitions of replication that deal with these
  features. 

  Further, the definitions are parameterized in a name, $x$. Can you,
  gentle reader, make a definition that eliminates this parameter and
  guarantees no accidental interaction between the replication
  machinery and the process being replicated -- i.e. no accidental
  sharing of names used by the process to get its work done and the
  name(s) used by the replication to effect copying. This latter
  revision of the definition of replication is crucial to obtaining
  the expected identity $!!P \sim !P$.
\end{remark}

\begin{remark}\label{rem:paradoxical_combinator}
  The reader familiar with the lambda calculus will have noticed the
  similarity between $D$ and the paradoxical combinator.

  [Ed. note: the existence of this seems to suggest we have to be more
  restrictive on the set of processes and names we admit if we are to
  support no-cloning.]
\end{remark}

\subsubsection{Bisimulation}

The computational dynamics gives rise to another kind of equivalence,
the equivalence of computational behavior. As previously mentioned
this is typically captured \emph{via} some form of bisimulation.

% The notion we use in this paper is weak barbed bisimulation
% \cite{milner91polyadicpi}.

The notion we use in this paper is derived from weak barbed
bisimulation \cite{milner91polyadicpi}. 

\begin{definition}
An \emph{observation relation}, $\downarrow_{\mathcal N}$, over a set
of names, $\mathcal N$, is the smallest relation satisfying the rules
below.

\infrule[Out-barb]{y \in {\mathcal N}, \; x \nameeq y}
		  {\outputp{x}{v} \downarrow_{\mathcal N} x}
\infrule[Par-barb]{\mbox{$P\downarrow_{\mathcal N} x$ or $Q\downarrow_{\mathcal N} x$}}
		  {\binpar{P}{Q} \downarrow_{\mathcal N} x}

We write $P \Downarrow_{\mathcal N} x$ if there is $Q$ such that 
$P \wred Q$ and $Q \downarrow_{\mathcal N} x$.
\end{definition}

\begin{definition}
%\label{def.bbisim}
An  ${\mathcal N}$-\emph{barbed bisimulation} over a set of names, ${\mathcal N}$, is a symmetric binary relation 
${\mathcal S}_{\mathcal N}$ between agents such that $P\rel{S}_{\mathcal N}Q$ implies:
\begin{enumerate}
\item If $P \red P'$ then $Q \wred Q'$ and $P'\rel{S}_{\mathcal N} Q'$.
\item If $P\downarrow_{\mathcal N} x$, then $Q\Downarrow_{\mathcal N} x$.
\end{enumerate}
$P$ is ${\mathcal N}$-barbed bisimilar to $Q$, written
$P \wbbisim_{\mathcal N} Q$, if $P \rel{S}_{\mathcal N} Q$ for some ${\mathcal N}$-barbed bisimulation ${\mathcal S}_{\mathcal N}$.
\end{definition}

$\mathcal{R} \subseteq \pi \times \pi$

$P \mathcal{R} Q => \forall P'. P \red P' \Rightarrow \exists Q'. Q \red Q', P' \mathcal{R} Q'$

$P \vdash x \Rightarrow Q \vdash x$

\begin{mathpar}
  \inferrule*[lab=Out-barb]{x \nameeq y}{{y}!\langle{Q}\rangle \vdash x}
  \and
  \inferrule*[lab=Par-barb]{\mbox{$P\vdash x$ or $Q\vdash x$}}{\binpar{P}{Q} \vdash x}
\end{mathpar}

\subsubsection{Contexts}

One of the principle advantages of computational calculi like the
$\pi$-calculus is a well-defined notion of context,
contextual-equivalence and a correlation between
contextual-equivalence and notions of bisimulation. The notion of
context allows the decomposition of a process into (sub-)process and
its syntactic environment, its context. Thus, a context may be
thought of as a process with a ``hole'' (written $\Box$) in it. The
application of a context $M$ to a process $P$, written $M[P]$, is
tantamount to filling the hole in $M$ with $P$. In this paper we do
not need the full weight of this theory, but do make use of the notion
of context in the proof the main theorem. 

\begin{mathpar}
  \inferrule* [lab=summation] {} {{M_{M},M_{N}} \bc \Box \;|\; x.M_{A} \;|\; M_{M}+M_{N}}
  \and
  \inferrule* [lab=agent] {} {{M_{A}} \bc (\vec{x})M_{P} \;| \; \clift{P_0,\ldots,M_{P},\ldots,P_N}}
  \and \\
  \inferrule* [lab=process] {} {{M_{P}} \bc M_{N} \;| \;P|M_{P} }
\end{mathpar} 

\begin{mathpar}
  \inferrule* [lab=sychronization] {} {M_{N} \bc \Box \;|\; x?M_{F} \;|\; x!M_{C}}
  \and
  \inferrule* [lab=abstraction] {} {{M_{F}} \bc (x)M_{P} }
  \and
  \inferrule* [lab=concretion] {} {{M_{C}} \bc \langle M_{P} \rangle }
  \and \\
  \inferrule* [lab=process] {} {{M_{P}} \bc M_{N} \;| \;P|M_{P} }
\end{mathpar}

\begin{definition}[contextual application] Given a context $M$, and
  process $P$, we define the \emph{contextual application}, $M[P] :=
  M\{P/\Box\}$. That is, the contextual application of M to P is the
  substitution of $P$ for $\Box$ in $M$.
\end{definition}

$\meaningof{-} : L \to \mathcal{P}(\pi)$

\begin{mathpar}
  \inferrule* [lab=collection] {} {\meaningof{true} = \pi, \and \meaningof{~E} = \pi \setminus \meaningof{E}, \and \meaningof{E_{1} \& E_{2}} = \meaningof{E_{1}} \cap \meaningof{E_{2}}}
\end{mathpar}

\begin{mathpar}
  \inferrule* [lab=structure] {} {\meaningof{0} = \{ P \in \pi | P \equiv 0 \}, \and \\ \meaningof{E_1 | E_2} = \{ P \in \pi | P \equiv P_{1} | P_{2}, P_{1} \in \meaningof{E_{1}}, P_{2} \in \meaningof{E_2}\} }
\end{mathpar}

\begin{mathpar}
 \inferrule* [lab=behavior] {} {\meaningof{\langle a?b \rangle E} = \{ P \in \pi | P \equiv Q | u?(y)P', \\ \and \\\\ \and \\ \;\;\; u \in \meaningof{a}, \forall z.P'\{z/y\} \in \meaningof{E\{z/b\}}\}, \and \\ \meaningof{a!E} = \{ P \in \pi | P \equiv Q | x!\langle P' \rangle, x \in \meaningof{a} P' \in \meaningof{E}\} }
\end{mathpar}

\begin{mathpar}
 \inferrule* [lab=nominal] {} {\meaningof{\quotep{E}} = \{ \quotep{P} \in \quotep{\pi} | P \in \meaningof{E} \}, \and \meaningof{\quotep{P}} = \{ \quotep{Q} \in \quotep{\pi} | P \equiv Q \} \and \\ \meaningof{@\quotep{E}} = \{ P \in \pi | P \equiv @x, x \in \meaningof{E} \}}
\end{mathpar}

\begin{eqnarray*}
  \\
  \meaningof{-} : TS \to ST
\end{eqnarray*}

\begin{eqnarray*}
  \\
  L : TS \to ST
\end{eqnarray*}

\begin{eqnarray*}
  \\
  P \models E \iff P \in \meaningof{E}
\end{eqnarray*}

\begin{eqnarray*}
  P \approx_{L} Q \iff \forall E \in L. P \models E \iff Q \models E
\end{eqnarray*}

\begin{eqnarray*}
  P \approx_{K} Q
\end{eqnarray*}

\begin{eqnarray*}
  P \approx Q
\end{eqnarray*}

$\approx_{K} = \approx = \approx_{L}$

\subsubsection{Contextual duality}

Note that contexts extend the quotation operation to a family of
operations from processes to names. Given a context, $M$, we can
define a \emph{nominal context}, $\quotep{M}$ by $\quotep{M}[P] :=
\quotep{M[P]}$. To foreshadow what is to come we observe that these
operations enjoy a duality with processes very much like the duality
between vectors and maps from vectors to scalars.

Further, because the calculus is essentially higher-order, we have a
correspondence between contexts and processes. More specifically,
given a name $x$ and a context $M$ we can construct $M^{*}_{x}$ such
that 

\begin{mathpar}
  M^{*}_{x} | \lift{x}{P} \red M[P]
\end{mathpar}

namely,

\begin{mathpar}
  M^{*}_{x} := x?(u).M[\dropn{u}]
\end{mathpar}

The dependence of $M^{*}_{x}$ on a name makes it an abstraction, 

\begin{mathpar}
  M^{*} := (x)x?(u).M[\dropn{u}]
\end{mathpar}

\subsection{Additional notation}

It will sometimes be convenient to denote the process a name
quotes. We already have the notation $x = \quotep{P}$, but it will be
convenient to introduce an alternate notation, $\procn{x}$, when we
want to emphasize the connection to the use of the name. Note that, by
virtue of name equivalence, $\quotep{\procn{x}} \nameeq x$; so, the
notation is consistent with previous definitions.

Further, because names have structure it is possible to effect
substitutions on the basis of that structure. This means we need to
upgrade our notation for substitutions, which we accomplish by
adapting comprehension notation. Thus,

\begin{mathpar}
  P\{ y / x : x \in S \}
\end{mathpar}

is interpreted to mean the process derived from P by replacing (in a
capture-avoiding manner) each occurrence of $x$ in $S$ by $y$. For example,

\begin{mathpar}
  P\{ \quotep{\procn{x}|\procn{x}} / x : x \in \freenames{P} \}
\end{mathpar}

will replace each (occurrence) of a free name $x$ in $P$ by
$\quotep{\procn{x}|\procn{x}}$.

Also, we will avail ourselves of the notation $x^{L}$ and $x^{R}$ to
denote injections of a name into disjoint copies of the name
space. There are numerous ways to accomplish this. One example can be
found in \cite{MeredithR05}. This notation overloads to vectors of
names: $\vec{x}^{\pi} := (x_{i}^{\pi} \; : \; 0 \leq i < |\vec{x}| )$ where $\pi \in \{L,R\}$.

We also use $P^{\Box} := P|\Box$.

In \cite{MeredithR05} an interpretation of the new operator is
given. It turns out that there are several possible interpretations
all enjoying the requisite algebraic properties of the operator (see
\cite{milner91polyadicpi}). We will therefore make liberal use of
$(\nu\; \vec{x})P$.

% subsection the_syntax_and_semantics_of_the_notation_system (end)   

\input{qm2pi.qmops} 

\input{qm2pi.sterngerlach} 

\input{qm2pi.metric} 

% section concurrent_process_calculi (end)

%\input{qm2pi.proofsketch}

% section proof sketch (end)

%\input{qm2pi.slviaknots} 

% section spatial logic via knots (end)

\input{qm2pi.conclusion}

% section conclusion (end)

%\input{qm2pi.dtcodes} 

% section wiring algorithm (end)

\input{qm2pi.ack} 

% section acknowledgments (end)

\newpage


\bibliographystyle{plain}   
\bibliography{../../biblios/main.bib}

\input{qm2pi.rhodetails}

\end{document}

 

%\documentclass[12pt]{llncs}
%\documentclass{jktr}

\usepackage[pdftex]{hyperref}                   
\usepackage {listings}
\usepackage {mathpartir}
\usepackage{bcprules}
%\usepackage{listings}
                       
\usepackage{graphicx} 
%\usepackage[margins=2.5cm,nohead,nofoot]{geometry}
%\usepackage{geometry}
\usepackage{amsfonts}
\usepackage{amstext}
\usepackage{latexsym}
\usepackage{amssymb}
\usepackage{color}


%\include{myPreamble}
\include{qm2pi.local} 

%\ifpdf
%\usepackage[pdftex]{graphicx}
%\else
%\usepackage{graphicx}
%\fi

 % \ifpdf
%  \usepackage{pdfsync}
%  \if


%\title{Brief Article}
%\author{David F. Snyder}
%\author{L.G. Meredith}

%\address{Dept. of Math., Texas State University--San Marcos, San Marcos, TX 78666}
       
\pagestyle{empty}


\begin{document}

\lstset{language=[Objective]Caml,frame=shadowbox}

\input{qm2pi.front}

% section front matter (end)

\input{qm2pi.intro} 
 
% section introduction (end)

% \input{qm2pi.knotations} 

% section notation (end)

\input{qm2pi.process.calculi} 

% section concurrent_process_calculi_and_spatial_logics_ (end)
    
%\input{qm2pi.knots2pi} 

%\input{qm2pi.trefoil} 

%\input{qm2pi.mainthm} 

% subsection basic_interpretation (end)

%\input{qm2pi.rho.presentation} 
\subsection{The syntax and semantics of the notation system}\label{sub:the_syntax_and_semantics_of_the_notation_system} % (fold)

We now summarize a technical presentation of the calculus that
embodies our theory of dynamics. The typical presentation of such a
calculus follows the style of giving generators and relations on
them. The grammar, below, describing term constructors, freely
generates the set of processes, $\Proc$. This set is then quotiented
by a relation known as structural congruence and it is over this set
that the notion of dynamics is expressed. This presentation is
essentially that of \cite{MeredithR05} with the addition of
polyadicity and summation. For readability we have relegated some of
the technical subtleties to an appendix.

\subsubsection{Process grammar}\label{subsub:process_grammar}

\begin{mathpar}
  \inferrule* [lab=synchronization] {} {{M} \bc \pzero \;|\; x?F \;|\; x!C }
  \and
  \inferrule* [lab=abstraction] {} {{F} \bc (x)P}
  \and
  \inferrule* [lab=concretion] {} {{C} \bc \langle Q \rangle}
  \and
  \inferrule* [lab=process] {} {{P,Q} \bc M \;| \;P|Q \;|\; @{x}}
  \and
  \inferrule* [lab=name] {} {{x} \bc \quotep{P}}
\end{mathpar} 

Note that $\vec{x}$ (resp. $\vec{P}$) denotes a vector of names
(resp. processes) of length $|\vec{x}|$ (resp. $|\vec{P}|$). We adopt
the following useful abbreviations.

\begin{mathpar}
   x?(\vec{y}).P := x.(\vec{y})P \and  x\clift{\vec{P}} := x.\clift{\vec{P}}
   \and x!(y) := \lift{x}{\dropn{y}}
   \and \Pi_{i=0}^{n-1}P_i := P_0 | \ldots | P_{n-1}
\end{mathpar}

\subsubsection{Structural congruence}

\paragraph{Free and bound names and alpha-equivalence.} At the
core of structural equivalence is alpha-equivalence which identifies
process that are the same up to a change of variable. Formally, we
recognize the distinction between free and bound names. The free names
of a process, $\freenames{P}$, may be calculated recursively as
follows:

\begin{mathpar}
\freenames{\pzero} := \emptyset
  \and \\
  \freenames{x?(y).P} := \{ x \} \cup (\freenames{P} \setminus \{ y \})
  \and 
  \freenames{x!\langle P \rangle} := \{ x \} \cup \{ P \} 
  \and \\
  \freenames{P|Q} := \freenames{P} \cup \freenames{Q}
  \and \\
  \freenames{@{x}} := \{ x \}
\end{mathpar}

$\pi$
$\quotep{\pi}$

$\freenames{-} : \pi \to \mathcal{P}(\quotep{\pi})$

\begin{eqnarray*}
  \freenames{\pzero} & := & \emptyset \\
  \freenames{x?(y).P} & := & \{ x \} \cup (\freenames{P} \setminus \{ y \}) \\
  \freenames{x!\langle P \rangle} & := & \{ x \} \cup \{ P \} \\
  \freenames{P|Q} & := & \freenames{P} \cup \freenames{Q} \\
  \freenames{\dropn{x}} & := & \{ x \}
\end{eqnarray*}

The bound names of a process, $\boundnames{P}$, are those names occurring in $P$
that are not free. For example, in $x?(y).0$, the name $x$ is free, while $y$ is bound.

\begin{mathpar}
  \inferrule* [lab=monoidal-laws] {} { P|Q \equiv Q|P \and P|0 \equiv P \and P|(Q|R) \equiv (P|Q)|R }
\end{mathpar}

\begin{mathpar}
  \inferrule* [lab=alpha-equivalence] {} { (x)P \equiv (y)P\{y/x\} \and y \not\in \freenames{P} }
\end{mathpar}

\begin{definition}
Then two processes, $P,Q$, are alpha-equivalent if $P = Q\{\vec{y}/\vec{x}\}$ for
some $\vec{x} \in \boundnames{Q},\vec{y} \in \boundnames{P}$, where $Q\{\vec{y}/\vec{x}\}$
denotes the capture-avoiding substitution of $\vec{y}$ for $\vec{x}$ in $Q$.
\end{definition}

\begin{definition}
  The {\em structural congruence} \cite{SangiorgiWalker} , $\equiv$,
  between processes is the least congruence containing
  alpha-equivalence, satisfying the abelian monoid laws
  (associativity, commutativity and $\pzero$ as identity) for parallel
  composition $|$ and for summation $+$.
\end{definition}

\subsection{Name equivalence}

We take name equivalence, written $\nameeq$, to be the smallest
equivalence relation generated by the following rules.

\begin{mathpar}
\inferrule*[lab=Quote-drop]
{ }
{ \quotep{@{x}} \nameeq x }

\inferrule*[lab=Struct-equiv]
{ P \scong Q }
{ \quotep{P} \nameeq \quotep{Q} }
\end{mathpar}

The astute reader will have noticed that the mutual recursion of names
and processes imposes a mutual recursion on alpha-equivalence and
structural equivalence via name-equivalence. Fortunately, all of this
works out pleasantly and we may calculate in the natural way, free of
concern. The reader interested in the details is referred to the
appendix \ref{appendix:rho_details}.

\subsection{Substitution}

We use $\Proc$ for the set of processes, $\QProc$ for the set of
names, and $\id{\{}\vec{y} / \vec{x} \id{\}}$ to denote partial maps,
$s : \QProc \rightarrow \QProc$. A map, $s$ lifts, uniquely, to a map
on process terms, $\widehat{s} : \Proc \rightarrow \Proc$ by the
following equations.

\begin{mathpar}
  (0) \psubstp{Q}{P} := 0 \\
  (R \juxtap S) \psubstp{Q}{P}
  :=    
  (R)\psubstp{Q}{P} \juxtap (S) \psubstp{Q}{P} \\
  (x?(y).R) \psubstp{Q}{P}    
  :=    
  (x)\substp{Q}{P} (z)\concat( (R \psubstn{z}{y}) \psubstp{Q}{P} ) \\
  (\lift{x}{R}) \psubstp{Q}{P}  
  :=
  \lift{(x)\substp{Q}{P}}{ R \psubstp{Q}{P} } \\
%   (\dropn{x})  \psubstp{Q}{P}       
%   := 
%   \left\{ 
%     \begin{array}{ccc} 
%       \dropn{\quotep{Q}} & & x \nameeq \quotep{P} \\
%       \dropn{x} & & otherwise \\
%     \end{array}
%   \right. 
  (\dropn{x})  \psubstp{Q}{P}       
  := 
  \left\{ 
    \begin{array}{ccc} 
      Q & & x \nameeq \quotep{P} \\
      \dropn{x} & & otherwise \\
    \end{array}
  \right.
\end{mathpar}
 

where

\begin{eqnarray}
  (x)\id{\{} \lpquote Q \rpquote / \lpquote P \rpquote \id{\}}            = 
  \left\{ 
    \begin{array}{ccc}
      \lpquote Q \rpquote & & x \nameeq \lpquote P \rpquote \\
      x & & otherwise \\
    \end{array}
  \right. \nonumber
\end{eqnarray}

and $z$ is chosen distinct from $\quotep{P}$, $\quotep{Q}$, the free
names in $Q$, and all the names in $R$. Our $\alpha$-equivalence will
be built in the standard way from this substitution.

\begin{remark}\label{rem:no_self_referential_names}
  One consequence of these definitions is that $\forall P. \quotep{P}
  \not\in \freenames{P}$.
\end{remark}

\subsection{ Dynamic quote: an example }

Anticipating something of what's to come, consider applying the
substitution, $\widehat{\id{\{}u / z \id{\}}}$, to the following pair
of processes, $\lift{w}{y!(z)}$ and $w[ \lpquote y!(z) \rpquote ]$.

\begin{eqnarray}
	\lift{w}{y!(z)}\widehat{\id{\{}u / z \id{\}}}
		& = &
		\lift{w}{y!(u)} \nonumber\\
	w[ \lpquote y!(z) \rpquote ] \widehat{ \id{\{}u / z \id{\}} }
		& = &
		w[ \lpquote y!(z) \rpquote ] \nonumber
\end{eqnarray}

Because the body of the process between quotes is impervious to
substitution, we get radically different answers. In fact, by
examining the first process in an input context,
e.g. $x?(z).\lift{w}{y!(z)}$, we see that the process under the lift
operator may be shaped by prefixed inputs binding a name inside it. In
this sense, the lift operator will be seen as a way to dynamically
construct processes before reifying them as names.

Finally equipped with these standard features we can present the
dynamics of the calculus.

\subsubsection{Operational semantics} 

Finally, we introduce the computational dynamics. What marks these
algebras as distinct from other more traditionally studied algebraic
structures, e.g. vector spaces or polynomial rings, is the manner in
which dynamics is captured. In traditional structures, dynamics is typically
expressed through morphisms between such structures, as in linear maps
between vector spaces or morphisms between rings. In algebras
associated with the semantics of computation, the dynamics is
expressed as part of the algebraic structure itself, through a
reduction reduction relation typically denoted by $\red$. Below, we
give a recursive presentation of this relation for the calculus used
in the encoding.

$\red \subseteq \pi \times \pi$
$\red : \pi \to \mathcal{P}(\pi)$

\begin{mathpar}
  \inferrule* [lab=Comm] { \textsf{match}( x_{src}, x_{trgt} ) } { x_{trgt}?(y)P \; | \; x_{src}!\langle {Q} \rangle \red P\{\quotep{Q}/y}\} }
  \and \\
  \inferrule* [lab=Par] {{P} \red {P}'} {{{P} | {Q}} \red {{P}' | {Q}}}
  \and
  \inferrule* [lab=Equiv]{{{P} \scong {P}'} \andalso {{P}' \red {Q}'} \andalso {{Q}' \scong {Q}}}{{P} \red {Q}}
\end{mathpar}

\begin{eqnarray*}
  match_{\equiv} (\quotep{P},\quotep{Q}) & := & P \equiv Q \\
  match_{\dagger}(\quotep{P},\quotep{Q}) & := & \forall R. P|Q \red^{*} R => R \red^{*} 0 \\
  match_{K}(\quotep{P},\quotep{Q}) & := & K \mbox{ for some context } K
\end{eqnarray*}

$u?(x)P | u!\langle Q \rangle \red P\{\quotep{Q}/x\}$

%We write $\wred$ for $\red^*$, and $P\red$ if $\exists Q $ such that $ P \red Q$.
We write $P\red$ if $\exists Q $ such that $ P \red Q$ and $P\not\red$, otherwise.

\section{Replication}

As mentioned before, it is known that replication (and hence
recursion) can be implemented in a higher-order process algebra
\cite{SangiorgiWalker}. As our first example of calculation with the
machinery thus far presented we give the construction explicitly in
the {\rhoc}.

\begin{eqnarray}
	D_{x} & := & \prefix{x}{y}{(\binpar{\outputp{x}{y}}{@{y}})} \nonumber\\
	\bangp_{x}{P} & := & \binpar{{x}!\langle{\binpar{D_{x}}{P}}\rangle}{D_{x}} \nonumber
\end{eqnarray}

\begin{eqnarray}
	\bangp_{x}{P} & & \nonumber\\
	=
	& {x}!\langle{(\prefix{x}{y}{(\outputp{x}{y} | @{y})) | P}}\rangle 
	      | \prefix{x}{y}{(\outputp{x}{y} | @{y})} & \nonumber\\
	\red
	& (\outputp{x}{y} | @{y})\substn{\quotep{(\prefix{x}{y}{(@{y} | \outputp{x}{y})) | P}}}{y} & \nonumber\\
	=
	& \outputp{x}{\quotep{(\prefix{x}{y}{(\outputp{x}{y} | @{y})) | P}}}
	  | {(\prefix{x}{y}{(\outputp{x}{y} | @{y})) | P}} & \nonumber\\
	\red
	& \ldots & \nonumber\\
	\red^*
	& P | P | \ldots & \nonumber
\end{eqnarray}

Of course, this encoding, as an implementation, runs away, unfolding
$\bangp{P}$ eagerly. A lazier and more implementable replication
operator, restricted to input-guarded processes, may be obtained as follows.

\begin{eqnarray}
\bangp{\prefix{u}{v}{P}} 
	:= 
	\binpar{\lift{x}{\prefix{u}{v}{(\binpar{D(x)}{P})}}}{D(x)} \nonumber
\end{eqnarray}

\begin{remark}
  Note that the lazier definition still does not deal with summation
  or mixed summation (i.e. sums over input and output). The reader is
  invited to construct definitions of replication that deal with these
  features. 

  Further, the definitions are parameterized in a name, $x$. Can you,
  gentle reader, make a definition that eliminates this parameter and
  guarantees no accidental interaction between the replication
  machinery and the process being replicated -- i.e. no accidental
  sharing of names used by the process to get its work done and the
  name(s) used by the replication to effect copying. This latter
  revision of the definition of replication is crucial to obtaining
  the expected identity $!!P \sim !P$.
\end{remark}

\begin{remark}\label{rem:paradoxical_combinator}
  The reader familiar with the lambda calculus will have noticed the
  similarity between $D$ and the paradoxical combinator.

  [Ed. note: the existence of this seems to suggest we have to be more
  restrictive on the set of processes and names we admit if we are to
  support no-cloning.]
\end{remark}

\subsubsection{Bisimulation}

The computational dynamics gives rise to another kind of equivalence,
the equivalence of computational behavior. As previously mentioned
this is typically captured \emph{via} some form of bisimulation.

% The notion we use in this paper is weak barbed bisimulation
% \cite{milner91polyadicpi}.

The notion we use in this paper is derived from weak barbed
bisimulation \cite{milner91polyadicpi}. 

\begin{definition}
An \emph{observation relation}, $\downarrow_{\mathcal N}$, over a set
of names, $\mathcal N$, is the smallest relation satisfying the rules
below.

\infrule[Out-barb]{y \in {\mathcal N}, \; x \nameeq y}
		  {\outputp{x}{v} \downarrow_{\mathcal N} x}
\infrule[Par-barb]{\mbox{$P\downarrow_{\mathcal N} x$ or $Q\downarrow_{\mathcal N} x$}}
		  {\binpar{P}{Q} \downarrow_{\mathcal N} x}

We write $P \Downarrow_{\mathcal N} x$ if there is $Q$ such that 
$P \wred Q$ and $Q \downarrow_{\mathcal N} x$.
\end{definition}

\begin{definition}
%\label{def.bbisim}
An  ${\mathcal N}$-\emph{barbed bisimulation} over a set of names, ${\mathcal N}$, is a symmetric binary relation 
${\mathcal S}_{\mathcal N}$ between agents such that $P\rel{S}_{\mathcal N}Q$ implies:
\begin{enumerate}
\item If $P \red P'$ then $Q \wred Q'$ and $P'\rel{S}_{\mathcal N} Q'$.
\item If $P\downarrow_{\mathcal N} x$, then $Q\Downarrow_{\mathcal N} x$.
\end{enumerate}
$P$ is ${\mathcal N}$-barbed bisimilar to $Q$, written
$P \wbbisim_{\mathcal N} Q$, if $P \rel{S}_{\mathcal N} Q$ for some ${\mathcal N}$-barbed bisimulation ${\mathcal S}_{\mathcal N}$.
\end{definition}

$\mathcal{R} \subseteq \pi \times \pi$

$P \mathcal{R} Q => \forall P'. P \red P' \Rightarrow \exists Q'. Q \red Q', P' \mathcal{R} Q'$

$P \vdash x \Rightarrow Q \vdash x$

\begin{mathpar}
  \inferrule*[lab=Out-barb]{x \nameeq y}{{y}!\langle{Q}\rangle \vdash x}
  \and
  \inferrule*[lab=Par-barb]{\mbox{$P\vdash x$ or $Q\vdash x$}}{\binpar{P}{Q} \vdash x}
\end{mathpar}

\subsubsection{Contexts}

One of the principle advantages of computational calculi like the
$\pi$-calculus is a well-defined notion of context,
contextual-equivalence and a correlation between
contextual-equivalence and notions of bisimulation. The notion of
context allows the decomposition of a process into (sub-)process and
its syntactic environment, its context. Thus, a context may be
thought of as a process with a ``hole'' (written $\Box$) in it. The
application of a context $M$ to a process $P$, written $M[P]$, is
tantamount to filling the hole in $M$ with $P$. In this paper we do
not need the full weight of this theory, but do make use of the notion
of context in the proof the main theorem. 

\begin{mathpar}
  \inferrule* [lab=summation] {} {{M_{M},M_{N}} \bc \Box \;|\; x.M_{A} \;|\; M_{M}+M_{N}}
  \and
  \inferrule* [lab=agent] {} {{M_{A}} \bc (\vec{x})M_{P} \;| \; \clift{P_0,\ldots,M_{P},\ldots,P_N}}
  \and \\
  \inferrule* [lab=process] {} {{M_{P}} \bc M_{N} \;| \;P|M_{P} }
\end{mathpar} 

\begin{mathpar}
  \inferrule* [lab=sychronization] {} {M_{N} \bc \Box \;|\; x?M_{F} \;|\; x!M_{C}}
  \and
  \inferrule* [lab=abstraction] {} {{M_{F}} \bc (x)M_{P} }
  \and
  \inferrule* [lab=concretion] {} {{M_{C}} \bc \langle M_{P} \rangle }
  \and \\
  \inferrule* [lab=process] {} {{M_{P}} \bc M_{N} \;| \;P|M_{P} }
\end{mathpar}

\begin{definition}[contextual application] Given a context $M$, and
  process $P$, we define the \emph{contextual application}, $M[P] :=
  M\{P/\Box\}$. That is, the contextual application of M to P is the
  substitution of $P$ for $\Box$ in $M$.
\end{definition}

$\meaningof{-} : L \to \mathcal{P}(\pi)$

\begin{mathpar}
  \inferrule* [lab=collection] {} {\meaningof{true} = \pi, \and \meaningof{~E} = \pi \setminus \meaningof{E}, \and \meaningof{E_{1} \& E_{2}} = \meaningof{E_{1}} \cap \meaningof{E_{2}}}
\end{mathpar}

\begin{mathpar}
  \inferrule* [lab=structure] {} {\meaningof{0} = \{ P \in \pi | P \equiv 0 \}, \and \\ \meaningof{E_1 | E_2} = \{ P \in \pi | P \equiv P_{1} | P_{2}, P_{1} \in \meaningof{E_{1}}, P_{2} \in \meaningof{E_2}\} }
\end{mathpar}

\begin{mathpar}
 \inferrule* [lab=behavior] {} {\meaningof{\langle a?b \rangle E} = \{ P \in \pi | P \equiv Q | u?(y)P', \\ \and \\\\ \and \\ \;\;\; u \in \meaningof{a}, \forall z.P'\{z/y\} \in \meaningof{E\{z/b\}}\}, \and \\ \meaningof{a!E} = \{ P \in \pi | P \equiv Q | x!\langle P' \rangle, x \in \meaningof{a} P' \in \meaningof{E}\} }
\end{mathpar}

\begin{mathpar}
 \inferrule* [lab=nominal] {} {\meaningof{\quotep{E}} = \{ \quotep{P} \in \quotep{\pi} | P \in \meaningof{E} \}, \and \meaningof{\quotep{P}} = \{ \quotep{Q} \in \quotep{\pi} | P \equiv Q \} \and \\ \meaningof{@\quotep{E}} = \{ P \in \pi | P \equiv @x, x \in \meaningof{E} \}}
\end{mathpar}

\begin{eqnarray*}
  \\
  \meaningof{-} : TS \to ST
\end{eqnarray*}

\begin{eqnarray*}
  \\
  L : TS \to ST
\end{eqnarray*}

\begin{eqnarray*}
  \\
  P \models E \iff P \in \meaningof{E}
\end{eqnarray*}

\begin{eqnarray*}
  P \approx_{L} Q \iff \forall E \in L. P \models E \iff Q \models E
\end{eqnarray*}

\begin{eqnarray*}
  P \approx_{K} Q
\end{eqnarray*}

\begin{eqnarray*}
  P \approx Q
\end{eqnarray*}

$\approx_{K} = \approx = \approx_{L}$

\subsubsection{Contextual duality}

Note that contexts extend the quotation operation to a family of
operations from processes to names. Given a context, $M$, we can
define a \emph{nominal context}, $\quotep{M}$ by $\quotep{M}[P] :=
\quotep{M[P]}$. To foreshadow what is to come we observe that these
operations enjoy a duality with processes very much like the duality
between vectors and maps from vectors to scalars.

Further, because the calculus is essentially higher-order, we have a
correspondence between contexts and processes. More specifically,
given a name $x$ and a context $M$ we can construct $M^{*}_{x}$ such
that 

\begin{mathpar}
  M^{*}_{x} | \lift{x}{P} \red M[P]
\end{mathpar}

namely,

\begin{mathpar}
  M^{*}_{x} := x?(u).M[\dropn{u}]
\end{mathpar}

The dependence of $M^{*}_{x}$ on a name makes it an abstraction, 

\begin{mathpar}
  M^{*} := (x)x?(u).M[\dropn{u}]
\end{mathpar}

\subsection{Additional notation}

It will sometimes be convenient to denote the process a name
quotes. We already have the notation $x = \quotep{P}$, but it will be
convenient to introduce an alternate notation, $\procn{x}$, when we
want to emphasize the connection to the use of the name. Note that, by
virtue of name equivalence, $\quotep{\procn{x}} \nameeq x$; so, the
notation is consistent with previous definitions.

Further, because names have structure it is possible to effect
substitutions on the basis of that structure. This means we need to
upgrade our notation for substitutions, which we accomplish by
adapting comprehension notation. Thus,

\begin{mathpar}
  P\{ y / x : x \in S \}
\end{mathpar}

is interpreted to mean the process derived from P by replacing (in a
capture-avoiding manner) each occurrence of $x$ in $S$ by $y$. For example,

\begin{mathpar}
  P\{ \quotep{\procn{x}|\procn{x}} / x : x \in \freenames{P} \}
\end{mathpar}

will replace each (occurrence) of a free name $x$ in $P$ by
$\quotep{\procn{x}|\procn{x}}$.

Also, we will avail ourselves of the notation $x^{L}$ and $x^{R}$ to
denote injections of a name into disjoint copies of the name
space. There are numerous ways to accomplish this. One example can be
found in \cite{MeredithR05}. This notation overloads to vectors of
names: $\vec{x}^{\pi} := (x_{i}^{\pi} \; : \; 0 \leq i < |\vec{x}| )$ where $\pi \in \{L,R\}$.

We also use $P^{\Box} := P|\Box$.

In \cite{MeredithR05} an interpretation of the new operator is
given. It turns out that there are several possible interpretations
all enjoying the requisite algebraic properties of the operator (see
\cite{milner91polyadicpi}). We will therefore make liberal use of
$(\nu\; \vec{x})P$.

% subsection the_syntax_and_semantics_of_the_notation_system (end)   

\input{qm2pi.qmops} 

\input{qm2pi.sterngerlach} 

\input{qm2pi.metric} 

% section concurrent_process_calculi (end)

%\input{qm2pi.proofsketch}

% section proof sketch (end)

%\input{qm2pi.slviaknots} 

% section spatial logic via knots (end)

\input{qm2pi.conclusion}

% section conclusion (end)

%\input{qm2pi.dtcodes} 

% section wiring algorithm (end)

\input{qm2pi.ack} 

% section acknowledgments (end)

\newpage


\bibliographystyle{plain}   
\bibliography{../../biblios/main.bib}

\input{qm2pi.rhodetails}

\end{document}

 

%\documentclass[12pt]{llncs}
%\documentclass{jktr}

\usepackage[pdftex]{hyperref}                   
\usepackage {listings}
\usepackage {mathpartir}
\usepackage{bcprules}
%\usepackage{listings}
                       
\usepackage{graphicx} 
%\usepackage[margins=2.5cm,nohead,nofoot]{geometry}
%\usepackage{geometry}
\usepackage{amsfonts}
\usepackage{amstext}
\usepackage{latexsym}
\usepackage{amssymb}
\usepackage{color}


%\include{myPreamble}
\include{qm2pi.local} 

%\ifpdf
%\usepackage[pdftex]{graphicx}
%\else
%\usepackage{graphicx}
%\fi

 % \ifpdf
%  \usepackage{pdfsync}
%  \if


%\title{Brief Article}
%\author{David F. Snyder}
%\author{L.G. Meredith}

%\address{Dept. of Math., Texas State University--San Marcos, San Marcos, TX 78666}
       
\pagestyle{empty}


\begin{document}

\lstset{language=[Objective]Caml,frame=shadowbox}

\input{qm2pi.front}

% section front matter (end)

\input{qm2pi.intro} 
 
% section introduction (end)

% \input{qm2pi.knotations} 

% section notation (end)

\input{qm2pi.process.calculi} 

% section concurrent_process_calculi_and_spatial_logics_ (end)
    
%\input{qm2pi.knots2pi} 

%\input{qm2pi.trefoil} 

%\input{qm2pi.mainthm} 

% subsection basic_interpretation (end)

%\input{qm2pi.rho.presentation} 
\subsection{The syntax and semantics of the notation system}\label{sub:the_syntax_and_semantics_of_the_notation_system} % (fold)

We now summarize a technical presentation of the calculus that
embodies our theory of dynamics. The typical presentation of such a
calculus follows the style of giving generators and relations on
them. The grammar, below, describing term constructors, freely
generates the set of processes, $\Proc$. This set is then quotiented
by a relation known as structural congruence and it is over this set
that the notion of dynamics is expressed. This presentation is
essentially that of \cite{MeredithR05} with the addition of
polyadicity and summation. For readability we have relegated some of
the technical subtleties to an appendix.

\subsubsection{Process grammar}\label{subsub:process_grammar}

\begin{mathpar}
  \inferrule* [lab=synchronization] {} {{M} \bc \pzero \;|\; x?F \;|\; x!C }
  \and
  \inferrule* [lab=abstraction] {} {{F} \bc (x)P}
  \and
  \inferrule* [lab=concretion] {} {{C} \bc \langle Q \rangle}
  \and
  \inferrule* [lab=process] {} {{P,Q} \bc M \;| \;P|Q \;|\; @{x}}
  \and
  \inferrule* [lab=name] {} {{x} \bc \quotep{P}}
\end{mathpar} 

Note that $\vec{x}$ (resp. $\vec{P}$) denotes a vector of names
(resp. processes) of length $|\vec{x}|$ (resp. $|\vec{P}|$). We adopt
the following useful abbreviations.

\begin{mathpar}
   x?(\vec{y}).P := x.(\vec{y})P \and  x\clift{\vec{P}} := x.\clift{\vec{P}}
   \and x!(y) := \lift{x}{\dropn{y}}
   \and \Pi_{i=0}^{n-1}P_i := P_0 | \ldots | P_{n-1}
\end{mathpar}

\subsubsection{Structural congruence}

\paragraph{Free and bound names and alpha-equivalence.} At the
core of structural equivalence is alpha-equivalence which identifies
process that are the same up to a change of variable. Formally, we
recognize the distinction between free and bound names. The free names
of a process, $\freenames{P}$, may be calculated recursively as
follows:

\begin{mathpar}
\freenames{\pzero} := \emptyset
  \and \\
  \freenames{x?(y).P} := \{ x \} \cup (\freenames{P} \setminus \{ y \})
  \and 
  \freenames{x!\langle P \rangle} := \{ x \} \cup \{ P \} 
  \and \\
  \freenames{P|Q} := \freenames{P} \cup \freenames{Q}
  \and \\
  \freenames{@{x}} := \{ x \}
\end{mathpar}

$\pi$
$\quotep{\pi}$

$\freenames{-} : \pi \to \mathcal{P}(\quotep{\pi})$

\begin{eqnarray*}
  \freenames{\pzero} & := & \emptyset \\
  \freenames{x?(y).P} & := & \{ x \} \cup (\freenames{P} \setminus \{ y \}) \\
  \freenames{x!\langle P \rangle} & := & \{ x \} \cup \{ P \} \\
  \freenames{P|Q} & := & \freenames{P} \cup \freenames{Q} \\
  \freenames{\dropn{x}} & := & \{ x \}
\end{eqnarray*}

The bound names of a process, $\boundnames{P}$, are those names occurring in $P$
that are not free. For example, in $x?(y).0$, the name $x$ is free, while $y$ is bound.

\begin{mathpar}
  \inferrule* [lab=monoidal-laws] {} { P|Q \equiv Q|P \and P|0 \equiv P \and P|(Q|R) \equiv (P|Q)|R }
\end{mathpar}

\begin{mathpar}
  \inferrule* [lab=alpha-equivalence] {} { (x)P \equiv (y)P\{y/x\} \and y \not\in \freenames{P} }
\end{mathpar}

\begin{definition}
Then two processes, $P,Q$, are alpha-equivalent if $P = Q\{\vec{y}/\vec{x}\}$ for
some $\vec{x} \in \boundnames{Q},\vec{y} \in \boundnames{P}$, where $Q\{\vec{y}/\vec{x}\}$
denotes the capture-avoiding substitution of $\vec{y}$ for $\vec{x}$ in $Q$.
\end{definition}

\begin{definition}
  The {\em structural congruence} \cite{SangiorgiWalker} , $\equiv$,
  between processes is the least congruence containing
  alpha-equivalence, satisfying the abelian monoid laws
  (associativity, commutativity and $\pzero$ as identity) for parallel
  composition $|$ and for summation $+$.
\end{definition}

\subsection{Name equivalence}

We take name equivalence, written $\nameeq$, to be the smallest
equivalence relation generated by the following rules.

\begin{mathpar}
\inferrule*[lab=Quote-drop]
{ }
{ \quotep{@{x}} \nameeq x }

\inferrule*[lab=Struct-equiv]
{ P \scong Q }
{ \quotep{P} \nameeq \quotep{Q} }
\end{mathpar}

The astute reader will have noticed that the mutual recursion of names
and processes imposes a mutual recursion on alpha-equivalence and
structural equivalence via name-equivalence. Fortunately, all of this
works out pleasantly and we may calculate in the natural way, free of
concern. The reader interested in the details is referred to the
appendix \ref{appendix:rho_details}.

\subsection{Substitution}

We use $\Proc$ for the set of processes, $\QProc$ for the set of
names, and $\id{\{}\vec{y} / \vec{x} \id{\}}$ to denote partial maps,
$s : \QProc \rightarrow \QProc$. A map, $s$ lifts, uniquely, to a map
on process terms, $\widehat{s} : \Proc \rightarrow \Proc$ by the
following equations.

\begin{mathpar}
  (0) \psubstp{Q}{P} := 0 \\
  (R \juxtap S) \psubstp{Q}{P}
  :=    
  (R)\psubstp{Q}{P} \juxtap (S) \psubstp{Q}{P} \\
  (x?(y).R) \psubstp{Q}{P}    
  :=    
  (x)\substp{Q}{P} (z)\concat( (R \psubstn{z}{y}) \psubstp{Q}{P} ) \\
  (\lift{x}{R}) \psubstp{Q}{P}  
  :=
  \lift{(x)\substp{Q}{P}}{ R \psubstp{Q}{P} } \\
%   (\dropn{x})  \psubstp{Q}{P}       
%   := 
%   \left\{ 
%     \begin{array}{ccc} 
%       \dropn{\quotep{Q}} & & x \nameeq \quotep{P} \\
%       \dropn{x} & & otherwise \\
%     \end{array}
%   \right. 
  (\dropn{x})  \psubstp{Q}{P}       
  := 
  \left\{ 
    \begin{array}{ccc} 
      Q & & x \nameeq \quotep{P} \\
      \dropn{x} & & otherwise \\
    \end{array}
  \right.
\end{mathpar}
 

where

\begin{eqnarray}
  (x)\id{\{} \lpquote Q \rpquote / \lpquote P \rpquote \id{\}}            = 
  \left\{ 
    \begin{array}{ccc}
      \lpquote Q \rpquote & & x \nameeq \lpquote P \rpquote \\
      x & & otherwise \\
    \end{array}
  \right. \nonumber
\end{eqnarray}

and $z$ is chosen distinct from $\quotep{P}$, $\quotep{Q}$, the free
names in $Q$, and all the names in $R$. Our $\alpha$-equivalence will
be built in the standard way from this substitution.

\begin{remark}\label{rem:no_self_referential_names}
  One consequence of these definitions is that $\forall P. \quotep{P}
  \not\in \freenames{P}$.
\end{remark}

\subsection{ Dynamic quote: an example }

Anticipating something of what's to come, consider applying the
substitution, $\widehat{\id{\{}u / z \id{\}}}$, to the following pair
of processes, $\lift{w}{y!(z)}$ and $w[ \lpquote y!(z) \rpquote ]$.

\begin{eqnarray}
	\lift{w}{y!(z)}\widehat{\id{\{}u / z \id{\}}}
		& = &
		\lift{w}{y!(u)} \nonumber\\
	w[ \lpquote y!(z) \rpquote ] \widehat{ \id{\{}u / z \id{\}} }
		& = &
		w[ \lpquote y!(z) \rpquote ] \nonumber
\end{eqnarray}

Because the body of the process between quotes is impervious to
substitution, we get radically different answers. In fact, by
examining the first process in an input context,
e.g. $x?(z).\lift{w}{y!(z)}$, we see that the process under the lift
operator may be shaped by prefixed inputs binding a name inside it. In
this sense, the lift operator will be seen as a way to dynamically
construct processes before reifying them as names.

Finally equipped with these standard features we can present the
dynamics of the calculus.

\subsubsection{Operational semantics} 

Finally, we introduce the computational dynamics. What marks these
algebras as distinct from other more traditionally studied algebraic
structures, e.g. vector spaces or polynomial rings, is the manner in
which dynamics is captured. In traditional structures, dynamics is typically
expressed through morphisms between such structures, as in linear maps
between vector spaces or morphisms between rings. In algebras
associated with the semantics of computation, the dynamics is
expressed as part of the algebraic structure itself, through a
reduction reduction relation typically denoted by $\red$. Below, we
give a recursive presentation of this relation for the calculus used
in the encoding.

$\red \subseteq \pi \times \pi$
$\red : \pi \to \mathcal{P}(\pi)$

\begin{mathpar}
  \inferrule* [lab=Comm] { \textsf{match}( x_{src}, x_{trgt} ) } { x_{trgt}?(y)P \; | \; x_{src}!\langle {Q} \rangle \red P\{\quotep{Q}/y}\} }
  \and \\
  \inferrule* [lab=Par] {{P} \red {P}'} {{{P} | {Q}} \red {{P}' | {Q}}}
  \and
  \inferrule* [lab=Equiv]{{{P} \scong {P}'} \andalso {{P}' \red {Q}'} \andalso {{Q}' \scong {Q}}}{{P} \red {Q}}
\end{mathpar}

\begin{eqnarray*}
  match_{\equiv} (\quotep{P},\quotep{Q}) & := & P \equiv Q \\
  match_{\dagger}(\quotep{P},\quotep{Q}) & := & \forall R. P|Q \red^{*} R => R \red^{*} 0 \\
  match_{K}(\quotep{P},\quotep{Q}) & := & K \mbox{ for some context } K
\end{eqnarray*}

$u?(x)P | u!\langle Q \rangle \red P\{\quotep{Q}/x\}$

%We write $\wred$ for $\red^*$, and $P\red$ if $\exists Q $ such that $ P \red Q$.
We write $P\red$ if $\exists Q $ such that $ P \red Q$ and $P\not\red$, otherwise.

\section{Replication}

As mentioned before, it is known that replication (and hence
recursion) can be implemented in a higher-order process algebra
\cite{SangiorgiWalker}. As our first example of calculation with the
machinery thus far presented we give the construction explicitly in
the {\rhoc}.

\begin{eqnarray}
	D_{x} & := & \prefix{x}{y}{(\binpar{\outputp{x}{y}}{@{y}})} \nonumber\\
	\bangp_{x}{P} & := & \binpar{{x}!\langle{\binpar{D_{x}}{P}}\rangle}{D_{x}} \nonumber
\end{eqnarray}

\begin{eqnarray}
	\bangp_{x}{P} & & \nonumber\\
	=
	& {x}!\langle{(\prefix{x}{y}{(\outputp{x}{y} | @{y})) | P}}\rangle 
	      | \prefix{x}{y}{(\outputp{x}{y} | @{y})} & \nonumber\\
	\red
	& (\outputp{x}{y} | @{y})\substn{\quotep{(\prefix{x}{y}{(@{y} | \outputp{x}{y})) | P}}}{y} & \nonumber\\
	=
	& \outputp{x}{\quotep{(\prefix{x}{y}{(\outputp{x}{y} | @{y})) | P}}}
	  | {(\prefix{x}{y}{(\outputp{x}{y} | @{y})) | P}} & \nonumber\\
	\red
	& \ldots & \nonumber\\
	\red^*
	& P | P | \ldots & \nonumber
\end{eqnarray}

Of course, this encoding, as an implementation, runs away, unfolding
$\bangp{P}$ eagerly. A lazier and more implementable replication
operator, restricted to input-guarded processes, may be obtained as follows.

\begin{eqnarray}
\bangp{\prefix{u}{v}{P}} 
	:= 
	\binpar{\lift{x}{\prefix{u}{v}{(\binpar{D(x)}{P})}}}{D(x)} \nonumber
\end{eqnarray}

\begin{remark}
  Note that the lazier definition still does not deal with summation
  or mixed summation (i.e. sums over input and output). The reader is
  invited to construct definitions of replication that deal with these
  features. 

  Further, the definitions are parameterized in a name, $x$. Can you,
  gentle reader, make a definition that eliminates this parameter and
  guarantees no accidental interaction between the replication
  machinery and the process being replicated -- i.e. no accidental
  sharing of names used by the process to get its work done and the
  name(s) used by the replication to effect copying. This latter
  revision of the definition of replication is crucial to obtaining
  the expected identity $!!P \sim !P$.
\end{remark}

\begin{remark}\label{rem:paradoxical_combinator}
  The reader familiar with the lambda calculus will have noticed the
  similarity between $D$ and the paradoxical combinator.

  [Ed. note: the existence of this seems to suggest we have to be more
  restrictive on the set of processes and names we admit if we are to
  support no-cloning.]
\end{remark}

\subsubsection{Bisimulation}

The computational dynamics gives rise to another kind of equivalence,
the equivalence of computational behavior. As previously mentioned
this is typically captured \emph{via} some form of bisimulation.

% The notion we use in this paper is weak barbed bisimulation
% \cite{milner91polyadicpi}.

The notion we use in this paper is derived from weak barbed
bisimulation \cite{milner91polyadicpi}. 

\begin{definition}
An \emph{observation relation}, $\downarrow_{\mathcal N}$, over a set
of names, $\mathcal N$, is the smallest relation satisfying the rules
below.

\infrule[Out-barb]{y \in {\mathcal N}, \; x \nameeq y}
		  {\outputp{x}{v} \downarrow_{\mathcal N} x}
\infrule[Par-barb]{\mbox{$P\downarrow_{\mathcal N} x$ or $Q\downarrow_{\mathcal N} x$}}
		  {\binpar{P}{Q} \downarrow_{\mathcal N} x}

We write $P \Downarrow_{\mathcal N} x$ if there is $Q$ such that 
$P \wred Q$ and $Q \downarrow_{\mathcal N} x$.
\end{definition}

\begin{definition}
%\label{def.bbisim}
An  ${\mathcal N}$-\emph{barbed bisimulation} over a set of names, ${\mathcal N}$, is a symmetric binary relation 
${\mathcal S}_{\mathcal N}$ between agents such that $P\rel{S}_{\mathcal N}Q$ implies:
\begin{enumerate}
\item If $P \red P'$ then $Q \wred Q'$ and $P'\rel{S}_{\mathcal N} Q'$.
\item If $P\downarrow_{\mathcal N} x$, then $Q\Downarrow_{\mathcal N} x$.
\end{enumerate}
$P$ is ${\mathcal N}$-barbed bisimilar to $Q$, written
$P \wbbisim_{\mathcal N} Q$, if $P \rel{S}_{\mathcal N} Q$ for some ${\mathcal N}$-barbed bisimulation ${\mathcal S}_{\mathcal N}$.
\end{definition}

$\mathcal{R} \subseteq \pi \times \pi$

$P \mathcal{R} Q => \forall P'. P \red P' \Rightarrow \exists Q'. Q \red Q', P' \mathcal{R} Q'$

$P \vdash x \Rightarrow Q \vdash x$

\begin{mathpar}
  \inferrule*[lab=Out-barb]{x \nameeq y}{{y}!\langle{Q}\rangle \vdash x}
  \and
  \inferrule*[lab=Par-barb]{\mbox{$P\vdash x$ or $Q\vdash x$}}{\binpar{P}{Q} \vdash x}
\end{mathpar}

\subsubsection{Contexts}

One of the principle advantages of computational calculi like the
$\pi$-calculus is a well-defined notion of context,
contextual-equivalence and a correlation between
contextual-equivalence and notions of bisimulation. The notion of
context allows the decomposition of a process into (sub-)process and
its syntactic environment, its context. Thus, a context may be
thought of as a process with a ``hole'' (written $\Box$) in it. The
application of a context $M$ to a process $P$, written $M[P]$, is
tantamount to filling the hole in $M$ with $P$. In this paper we do
not need the full weight of this theory, but do make use of the notion
of context in the proof the main theorem. 

\begin{mathpar}
  \inferrule* [lab=summation] {} {{M_{M},M_{N}} \bc \Box \;|\; x.M_{A} \;|\; M_{M}+M_{N}}
  \and
  \inferrule* [lab=agent] {} {{M_{A}} \bc (\vec{x})M_{P} \;| \; \clift{P_0,\ldots,M_{P},\ldots,P_N}}
  \and \\
  \inferrule* [lab=process] {} {{M_{P}} \bc M_{N} \;| \;P|M_{P} }
\end{mathpar} 

\begin{mathpar}
  \inferrule* [lab=sychronization] {} {M_{N} \bc \Box \;|\; x?M_{F} \;|\; x!M_{C}}
  \and
  \inferrule* [lab=abstraction] {} {{M_{F}} \bc (x)M_{P} }
  \and
  \inferrule* [lab=concretion] {} {{M_{C}} \bc \langle M_{P} \rangle }
  \and \\
  \inferrule* [lab=process] {} {{M_{P}} \bc M_{N} \;| \;P|M_{P} }
\end{mathpar}

\begin{definition}[contextual application] Given a context $M$, and
  process $P$, we define the \emph{contextual application}, $M[P] :=
  M\{P/\Box\}$. That is, the contextual application of M to P is the
  substitution of $P$ for $\Box$ in $M$.
\end{definition}

$\meaningof{-} : L \to \mathcal{P}(\pi)$

\begin{mathpar}
  \inferrule* [lab=collection] {} {\meaningof{true} = \pi, \and \meaningof{~E} = \pi \setminus \meaningof{E}, \and \meaningof{E_{1} \& E_{2}} = \meaningof{E_{1}} \cap \meaningof{E_{2}}}
\end{mathpar}

\begin{mathpar}
  \inferrule* [lab=structure] {} {\meaningof{0} = \{ P \in \pi | P \equiv 0 \}, \and \\ \meaningof{E_1 | E_2} = \{ P \in \pi | P \equiv P_{1} | P_{2}, P_{1} \in \meaningof{E_{1}}, P_{2} \in \meaningof{E_2}\} }
\end{mathpar}

\begin{mathpar}
 \inferrule* [lab=behavior] {} {\meaningof{\langle a?b \rangle E} = \{ P \in \pi | P \equiv Q | u?(y)P', \\ \and \\\\ \and \\ \;\;\; u \in \meaningof{a}, \forall z.P'\{z/y\} \in \meaningof{E\{z/b\}}\}, \and \\ \meaningof{a!E} = \{ P \in \pi | P \equiv Q | x!\langle P' \rangle, x \in \meaningof{a} P' \in \meaningof{E}\} }
\end{mathpar}

\begin{mathpar}
 \inferrule* [lab=nominal] {} {\meaningof{\quotep{E}} = \{ \quotep{P} \in \quotep{\pi} | P \in \meaningof{E} \}, \and \meaningof{\quotep{P}} = \{ \quotep{Q} \in \quotep{\pi} | P \equiv Q \} \and \\ \meaningof{@\quotep{E}} = \{ P \in \pi | P \equiv @x, x \in \meaningof{E} \}}
\end{mathpar}

\begin{eqnarray*}
  \\
  \meaningof{-} : TS \to ST
\end{eqnarray*}

\begin{eqnarray*}
  \\
  L : TS \to ST
\end{eqnarray*}

\begin{eqnarray*}
  \\
  P \models E \iff P \in \meaningof{E}
\end{eqnarray*}

\begin{eqnarray*}
  P \approx_{L} Q \iff \forall E \in L. P \models E \iff Q \models E
\end{eqnarray*}

\begin{eqnarray*}
  P \approx_{K} Q
\end{eqnarray*}

\begin{eqnarray*}
  P \approx Q
\end{eqnarray*}

$\approx_{K} = \approx = \approx_{L}$

\subsubsection{Contextual duality}

Note that contexts extend the quotation operation to a family of
operations from processes to names. Given a context, $M$, we can
define a \emph{nominal context}, $\quotep{M}$ by $\quotep{M}[P] :=
\quotep{M[P]}$. To foreshadow what is to come we observe that these
operations enjoy a duality with processes very much like the duality
between vectors and maps from vectors to scalars.

Further, because the calculus is essentially higher-order, we have a
correspondence between contexts and processes. More specifically,
given a name $x$ and a context $M$ we can construct $M^{*}_{x}$ such
that 

\begin{mathpar}
  M^{*}_{x} | \lift{x}{P} \red M[P]
\end{mathpar}

namely,

\begin{mathpar}
  M^{*}_{x} := x?(u).M[\dropn{u}]
\end{mathpar}

The dependence of $M^{*}_{x}$ on a name makes it an abstraction, 

\begin{mathpar}
  M^{*} := (x)x?(u).M[\dropn{u}]
\end{mathpar}

\subsection{Additional notation}

It will sometimes be convenient to denote the process a name
quotes. We already have the notation $x = \quotep{P}$, but it will be
convenient to introduce an alternate notation, $\procn{x}$, when we
want to emphasize the connection to the use of the name. Note that, by
virtue of name equivalence, $\quotep{\procn{x}} \nameeq x$; so, the
notation is consistent with previous definitions.

Further, because names have structure it is possible to effect
substitutions on the basis of that structure. This means we need to
upgrade our notation for substitutions, which we accomplish by
adapting comprehension notation. Thus,

\begin{mathpar}
  P\{ y / x : x \in S \}
\end{mathpar}

is interpreted to mean the process derived from P by replacing (in a
capture-avoiding manner) each occurrence of $x$ in $S$ by $y$. For example,

\begin{mathpar}
  P\{ \quotep{\procn{x}|\procn{x}} / x : x \in \freenames{P} \}
\end{mathpar}

will replace each (occurrence) of a free name $x$ in $P$ by
$\quotep{\procn{x}|\procn{x}}$.

Also, we will avail ourselves of the notation $x^{L}$ and $x^{R}$ to
denote injections of a name into disjoint copies of the name
space. There are numerous ways to accomplish this. One example can be
found in \cite{MeredithR05}. This notation overloads to vectors of
names: $\vec{x}^{\pi} := (x_{i}^{\pi} \; : \; 0 \leq i < |\vec{x}| )$ where $\pi \in \{L,R\}$.

We also use $P^{\Box} := P|\Box$.

In \cite{MeredithR05} an interpretation of the new operator is
given. It turns out that there are several possible interpretations
all enjoying the requisite algebraic properties of the operator (see
\cite{milner91polyadicpi}). We will therefore make liberal use of
$(\nu\; \vec{x})P$.

% subsection the_syntax_and_semantics_of_the_notation_system (end)   

\input{qm2pi.qmops} 

\input{qm2pi.sterngerlach} 

\input{qm2pi.metric} 

% section concurrent_process_calculi (end)

%\input{qm2pi.proofsketch}

% section proof sketch (end)

%\input{qm2pi.slviaknots} 

% section spatial logic via knots (end)

\input{qm2pi.conclusion}

% section conclusion (end)

%\input{qm2pi.dtcodes} 

% section wiring algorithm (end)

\input{qm2pi.ack} 

% section acknowledgments (end)

\newpage


\bibliographystyle{plain}   
\bibliography{../../biblios/main.bib}

\input{qm2pi.rhodetails}

\end{document}

 

% subsection basic_interpretation (end)

%\input{qm2pi.rho.presentation} 
\subsection{The syntax and semantics of the notation system}\label{sub:the_syntax_and_semantics_of_the_notation_system} % (fold)

We now summarize a technical presentation of the calculus that
embodies our theory of dynamics. The typical presentation of such a
calculus follows the style of giving generators and relations on
them. The grammar, below, describing term constructors, freely
generates the set of processes, $\Proc$. This set is then quotiented
by a relation known as structural congruence and it is over this set
that the notion of dynamics is expressed. This presentation is
essentially that of \cite{MeredithR05} with the addition of
polyadicity and summation. For readability we have relegated some of
the technical subtleties to an appendix.

\subsubsection{Process grammar}\label{subsub:process_grammar}

\begin{mathpar}
  \inferrule* [lab=synchronization] {} {{M} \bc \pzero \;|\; x?F \;|\; x!C }
  \and
  \inferrule* [lab=abstraction] {} {{F} \bc (x)P}
  \and
  \inferrule* [lab=concretion] {} {{C} \bc \langle Q \rangle}
  \and
  \inferrule* [lab=process] {} {{P,Q} \bc M \;| \;P|Q \;|\; @{x}}
  \and
  \inferrule* [lab=name] {} {{x} \bc \quotep{P}}
\end{mathpar} 

Note that $\vec{x}$ (resp. $\vec{P}$) denotes a vector of names
(resp. processes) of length $|\vec{x}|$ (resp. $|\vec{P}|$). We adopt
the following useful abbreviations.

\begin{mathpar}
   x?(\vec{y}).P := x.(\vec{y})P \and  x\clift{\vec{P}} := x.\clift{\vec{P}}
   \and x!(y) := \lift{x}{\dropn{y}}
   \and \Pi_{i=0}^{n-1}P_i := P_0 | \ldots | P_{n-1}
\end{mathpar}

\subsubsection{Structural congruence}

\paragraph{Free and bound names and alpha-equivalence.} At the
core of structural equivalence is alpha-equivalence which identifies
process that are the same up to a change of variable. Formally, we
recognize the distinction between free and bound names. The free names
of a process, $\freenames{P}$, may be calculated recursively as
follows:

\begin{mathpar}
\freenames{\pzero} := \emptyset
  \and \\
  \freenames{x?(y).P} := \{ x \} \cup (\freenames{P} \setminus \{ y \})
  \and 
  \freenames{x!\langle P \rangle} := \{ x \} \cup \{ P \} 
  \and \\
  \freenames{P|Q} := \freenames{P} \cup \freenames{Q}
  \and \\
  \freenames{@{x}} := \{ x \}
\end{mathpar}

$\pi$
$\quotep{\pi}$

$\freenames{-} : \pi \to \mathcal{P}(\quotep{\pi})$

\begin{eqnarray*}
  \freenames{\pzero} & := & \emptyset \\
  \freenames{x?(y).P} & := & \{ x \} \cup (\freenames{P} \setminus \{ y \}) \\
  \freenames{x!\langle P \rangle} & := & \{ x \} \cup \{ P \} \\
  \freenames{P|Q} & := & \freenames{P} \cup \freenames{Q} \\
  \freenames{\dropn{x}} & := & \{ x \}
\end{eqnarray*}

The bound names of a process, $\boundnames{P}$, are those names occurring in $P$
that are not free. For example, in $x?(y).0$, the name $x$ is free, while $y$ is bound.

\begin{mathpar}
  \inferrule* [lab=monoidal-laws] {} { P|Q \equiv Q|P \and P|0 \equiv P \and P|(Q|R) \equiv (P|Q)|R }
\end{mathpar}

\begin{mathpar}
  \inferrule* [lab=alpha-equivalence] {} { (x)P \equiv (y)P\{y/x\} \and y \not\in \freenames{P} }
\end{mathpar}

\begin{definition}
Then two processes, $P,Q$, are alpha-equivalent if $P = Q\{\vec{y}/\vec{x}\}$ for
some $\vec{x} \in \boundnames{Q},\vec{y} \in \boundnames{P}$, where $Q\{\vec{y}/\vec{x}\}$
denotes the capture-avoiding substitution of $\vec{y}$ for $\vec{x}$ in $Q$.
\end{definition}

\begin{definition}
  The {\em structural congruence} \cite{SangiorgiWalker} , $\equiv$,
  between processes is the least congruence containing
  alpha-equivalence, satisfying the abelian monoid laws
  (associativity, commutativity and $\pzero$ as identity) for parallel
  composition $|$ and for summation $+$.
\end{definition}

\subsection{Name equivalence}

We take name equivalence, written $\nameeq$, to be the smallest
equivalence relation generated by the following rules.

\begin{mathpar}
\inferrule*[lab=Quote-drop]
{ }
{ \quotep{@{x}} \nameeq x }

\inferrule*[lab=Struct-equiv]
{ P \scong Q }
{ \quotep{P} \nameeq \quotep{Q} }
\end{mathpar}

The astute reader will have noticed that the mutual recursion of names
and processes imposes a mutual recursion on alpha-equivalence and
structural equivalence via name-equivalence. Fortunately, all of this
works out pleasantly and we may calculate in the natural way, free of
concern. The reader interested in the details is referred to the
appendix \ref{appendix:rho_details}.

\subsection{Substitution}

We use $\Proc$ for the set of processes, $\QProc$ for the set of
names, and $\id{\{}\vec{y} / \vec{x} \id{\}}$ to denote partial maps,
$s : \QProc \rightarrow \QProc$. A map, $s$ lifts, uniquely, to a map
on process terms, $\widehat{s} : \Proc \rightarrow \Proc$ by the
following equations.

\begin{mathpar}
  (0) \psubstp{Q}{P} := 0 \\
  (R \juxtap S) \psubstp{Q}{P}
  :=    
  (R)\psubstp{Q}{P} \juxtap (S) \psubstp{Q}{P} \\
  (x?(y).R) \psubstp{Q}{P}    
  :=    
  (x)\substp{Q}{P} (z)\concat( (R \psubstn{z}{y}) \psubstp{Q}{P} ) \\
  (\lift{x}{R}) \psubstp{Q}{P}  
  :=
  \lift{(x)\substp{Q}{P}}{ R \psubstp{Q}{P} } \\
%   (\dropn{x})  \psubstp{Q}{P}       
%   := 
%   \left\{ 
%     \begin{array}{ccc} 
%       \dropn{\quotep{Q}} & & x \nameeq \quotep{P} \\
%       \dropn{x} & & otherwise \\
%     \end{array}
%   \right. 
  (\dropn{x})  \psubstp{Q}{P}       
  := 
  \left\{ 
    \begin{array}{ccc} 
      Q & & x \nameeq \quotep{P} \\
      \dropn{x} & & otherwise \\
    \end{array}
  \right.
\end{mathpar}
 

where

\begin{eqnarray}
  (x)\id{\{} \lpquote Q \rpquote / \lpquote P \rpquote \id{\}}            = 
  \left\{ 
    \begin{array}{ccc}
      \lpquote Q \rpquote & & x \nameeq \lpquote P \rpquote \\
      x & & otherwise \\
    \end{array}
  \right. \nonumber
\end{eqnarray}

and $z$ is chosen distinct from $\quotep{P}$, $\quotep{Q}$, the free
names in $Q$, and all the names in $R$. Our $\alpha$-equivalence will
be built in the standard way from this substitution.

\begin{remark}\label{rem:no_self_referential_names}
  One consequence of these definitions is that $\forall P. \quotep{P}
  \not\in \freenames{P}$.
\end{remark}

\subsection{ Dynamic quote: an example }

Anticipating something of what's to come, consider applying the
substitution, $\widehat{\id{\{}u / z \id{\}}}$, to the following pair
of processes, $\lift{w}{y!(z)}$ and $w[ \lpquote y!(z) \rpquote ]$.

\begin{eqnarray}
	\lift{w}{y!(z)}\widehat{\id{\{}u / z \id{\}}}
		& = &
		\lift{w}{y!(u)} \nonumber\\
	w[ \lpquote y!(z) \rpquote ] \widehat{ \id{\{}u / z \id{\}} }
		& = &
		w[ \lpquote y!(z) \rpquote ] \nonumber
\end{eqnarray}

Because the body of the process between quotes is impervious to
substitution, we get radically different answers. In fact, by
examining the first process in an input context,
e.g. $x?(z).\lift{w}{y!(z)}$, we see that the process under the lift
operator may be shaped by prefixed inputs binding a name inside it. In
this sense, the lift operator will be seen as a way to dynamically
construct processes before reifying them as names.

Finally equipped with these standard features we can present the
dynamics of the calculus.

\subsubsection{Operational semantics} 

Finally, we introduce the computational dynamics. What marks these
algebras as distinct from other more traditionally studied algebraic
structures, e.g. vector spaces or polynomial rings, is the manner in
which dynamics is captured. In traditional structures, dynamics is typically
expressed through morphisms between such structures, as in linear maps
between vector spaces or morphisms between rings. In algebras
associated with the semantics of computation, the dynamics is
expressed as part of the algebraic structure itself, through a
reduction reduction relation typically denoted by $\red$. Below, we
give a recursive presentation of this relation for the calculus used
in the encoding.

$\red \subseteq \pi \times \pi$
$\red : \pi \to \mathcal{P}(\pi)$

\begin{mathpar}
  \inferrule* [lab=Comm] { \textsf{match}( x_{src}, x_{trgt} ) } { x_{trgt}?(y)P \; | \; x_{src}!\langle {Q} \rangle \red P\{\quotep{Q}/y}\} }
  \and \\
  \inferrule* [lab=Par] {{P} \red {P}'} {{{P} | {Q}} \red {{P}' | {Q}}}
  \and
  \inferrule* [lab=Equiv]{{{P} \scong {P}'} \andalso {{P}' \red {Q}'} \andalso {{Q}' \scong {Q}}}{{P} \red {Q}}
\end{mathpar}

\begin{eqnarray*}
  match_{\equiv} (\quotep{P},\quotep{Q}) & := & P \equiv Q \\
  match_{\dagger}(\quotep{P},\quotep{Q}) & := & \forall R. P|Q \red^{*} R => R \red^{*} 0 \\
  match_{K}(\quotep{P},\quotep{Q}) & := & K \mbox{ for some context } K
\end{eqnarray*}

$u?(x)P | u!\langle Q \rangle \red P\{\quotep{Q}/x\}$

%We write $\wred$ for $\red^*$, and $P\red$ if $\exists Q $ such that $ P \red Q$.
We write $P\red$ if $\exists Q $ such that $ P \red Q$ and $P\not\red$, otherwise.

\section{Replication}

As mentioned before, it is known that replication (and hence
recursion) can be implemented in a higher-order process algebra
\cite{SangiorgiWalker}. As our first example of calculation with the
machinery thus far presented we give the construction explicitly in
the {\rhoc}.

\begin{eqnarray}
	D_{x} & := & \prefix{x}{y}{(\binpar{\outputp{x}{y}}{@{y}})} \nonumber\\
	\bangp_{x}{P} & := & \binpar{{x}!\langle{\binpar{D_{x}}{P}}\rangle}{D_{x}} \nonumber
\end{eqnarray}

\begin{eqnarray}
	\bangp_{x}{P} & & \nonumber\\
	=
	& {x}!\langle{(\prefix{x}{y}{(\outputp{x}{y} | @{y})) | P}}\rangle 
	      | \prefix{x}{y}{(\outputp{x}{y} | @{y})} & \nonumber\\
	\red
	& (\outputp{x}{y} | @{y})\substn{\quotep{(\prefix{x}{y}{(@{y} | \outputp{x}{y})) | P}}}{y} & \nonumber\\
	=
	& \outputp{x}{\quotep{(\prefix{x}{y}{(\outputp{x}{y} | @{y})) | P}}}
	  | {(\prefix{x}{y}{(\outputp{x}{y} | @{y})) | P}} & \nonumber\\
	\red
	& \ldots & \nonumber\\
	\red^*
	& P | P | \ldots & \nonumber
\end{eqnarray}

Of course, this encoding, as an implementation, runs away, unfolding
$\bangp{P}$ eagerly. A lazier and more implementable replication
operator, restricted to input-guarded processes, may be obtained as follows.

\begin{eqnarray}
\bangp{\prefix{u}{v}{P}} 
	:= 
	\binpar{\lift{x}{\prefix{u}{v}{(\binpar{D(x)}{P})}}}{D(x)} \nonumber
\end{eqnarray}

\begin{remark}
  Note that the lazier definition still does not deal with summation
  or mixed summation (i.e. sums over input and output). The reader is
  invited to construct definitions of replication that deal with these
  features. 

  Further, the definitions are parameterized in a name, $x$. Can you,
  gentle reader, make a definition that eliminates this parameter and
  guarantees no accidental interaction between the replication
  machinery and the process being replicated -- i.e. no accidental
  sharing of names used by the process to get its work done and the
  name(s) used by the replication to effect copying. This latter
  revision of the definition of replication is crucial to obtaining
  the expected identity $!!P \sim !P$.
\end{remark}

\begin{remark}\label{rem:paradoxical_combinator}
  The reader familiar with the lambda calculus will have noticed the
  similarity between $D$ and the paradoxical combinator.

  [Ed. note: the existence of this seems to suggest we have to be more
  restrictive on the set of processes and names we admit if we are to
  support no-cloning.]
\end{remark}

\subsubsection{Bisimulation}

The computational dynamics gives rise to another kind of equivalence,
the equivalence of computational behavior. As previously mentioned
this is typically captured \emph{via} some form of bisimulation.

% The notion we use in this paper is weak barbed bisimulation
% \cite{milner91polyadicpi}.

The notion we use in this paper is derived from weak barbed
bisimulation \cite{milner91polyadicpi}. 

\begin{definition}
An \emph{observation relation}, $\downarrow_{\mathcal N}$, over a set
of names, $\mathcal N$, is the smallest relation satisfying the rules
below.

\infrule[Out-barb]{y \in {\mathcal N}, \; x \nameeq y}
		  {\outputp{x}{v} \downarrow_{\mathcal N} x}
\infrule[Par-barb]{\mbox{$P\downarrow_{\mathcal N} x$ or $Q\downarrow_{\mathcal N} x$}}
		  {\binpar{P}{Q} \downarrow_{\mathcal N} x}

We write $P \Downarrow_{\mathcal N} x$ if there is $Q$ such that 
$P \wred Q$ and $Q \downarrow_{\mathcal N} x$.
\end{definition}

\begin{definition}
%\label{def.bbisim}
An  ${\mathcal N}$-\emph{barbed bisimulation} over a set of names, ${\mathcal N}$, is a symmetric binary relation 
${\mathcal S}_{\mathcal N}$ between agents such that $P\rel{S}_{\mathcal N}Q$ implies:
\begin{enumerate}
\item If $P \red P'$ then $Q \wred Q'$ and $P'\rel{S}_{\mathcal N} Q'$.
\item If $P\downarrow_{\mathcal N} x$, then $Q\Downarrow_{\mathcal N} x$.
\end{enumerate}
$P$ is ${\mathcal N}$-barbed bisimilar to $Q$, written
$P \wbbisim_{\mathcal N} Q$, if $P \rel{S}_{\mathcal N} Q$ for some ${\mathcal N}$-barbed bisimulation ${\mathcal S}_{\mathcal N}$.
\end{definition}

$\mathcal{R} \subseteq \pi \times \pi$

$P \mathcal{R} Q => \forall P'. P \red P' \Rightarrow \exists Q'. Q \red Q', P' \mathcal{R} Q'$

$P \vdash x \Rightarrow Q \vdash x$

\begin{mathpar}
  \inferrule*[lab=Out-barb]{x \nameeq y}{{y}!\langle{Q}\rangle \vdash x}
  \and
  \inferrule*[lab=Par-barb]{\mbox{$P\vdash x$ or $Q\vdash x$}}{\binpar{P}{Q} \vdash x}
\end{mathpar}

\subsubsection{Contexts}

One of the principle advantages of computational calculi like the
$\pi$-calculus is a well-defined notion of context,
contextual-equivalence and a correlation between
contextual-equivalence and notions of bisimulation. The notion of
context allows the decomposition of a process into (sub-)process and
its syntactic environment, its context. Thus, a context may be
thought of as a process with a ``hole'' (written $\Box$) in it. The
application of a context $M$ to a process $P$, written $M[P]$, is
tantamount to filling the hole in $M$ with $P$. In this paper we do
not need the full weight of this theory, but do make use of the notion
of context in the proof the main theorem. 

\begin{mathpar}
  \inferrule* [lab=summation] {} {{M_{M},M_{N}} \bc \Box \;|\; x.M_{A} \;|\; M_{M}+M_{N}}
  \and
  \inferrule* [lab=agent] {} {{M_{A}} \bc (\vec{x})M_{P} \;| \; \clift{P_0,\ldots,M_{P},\ldots,P_N}}
  \and \\
  \inferrule* [lab=process] {} {{M_{P}} \bc M_{N} \;| \;P|M_{P} }
\end{mathpar} 

\begin{mathpar}
  \inferrule* [lab=sychronization] {} {M_{N} \bc \Box \;|\; x?M_{F} \;|\; x!M_{C}}
  \and
  \inferrule* [lab=abstraction] {} {{M_{F}} \bc (x)M_{P} }
  \and
  \inferrule* [lab=concretion] {} {{M_{C}} \bc \langle M_{P} \rangle }
  \and \\
  \inferrule* [lab=process] {} {{M_{P}} \bc M_{N} \;| \;P|M_{P} }
\end{mathpar}

\begin{definition}[contextual application] Given a context $M$, and
  process $P$, we define the \emph{contextual application}, $M[P] :=
  M\{P/\Box\}$. That is, the contextual application of M to P is the
  substitution of $P$ for $\Box$ in $M$.
\end{definition}

$\meaningof{-} : L \to \mathcal{P}(\pi)$

\begin{mathpar}
  \inferrule* [lab=collection] {} {\meaningof{true} = \pi, \and \meaningof{~E} = \pi \setminus \meaningof{E}, \and \meaningof{E_{1} \& E_{2}} = \meaningof{E_{1}} \cap \meaningof{E_{2}}}
\end{mathpar}

\begin{mathpar}
  \inferrule* [lab=structure] {} {\meaningof{0} = \{ P \in \pi | P \equiv 0 \}, \and \\ \meaningof{E_1 | E_2} = \{ P \in \pi | P \equiv P_{1} | P_{2}, P_{1} \in \meaningof{E_{1}}, P_{2} \in \meaningof{E_2}\} }
\end{mathpar}

\begin{mathpar}
 \inferrule* [lab=behavior] {} {\meaningof{\langle a?b \rangle E} = \{ P \in \pi | P \equiv Q | u?(y)P', \\ \and \\\\ \and \\ \;\;\; u \in \meaningof{a}, \forall z.P'\{z/y\} \in \meaningof{E\{z/b\}}\}, \and \\ \meaningof{a!E} = \{ P \in \pi | P \equiv Q | x!\langle P' \rangle, x \in \meaningof{a} P' \in \meaningof{E}\} }
\end{mathpar}

\begin{mathpar}
 \inferrule* [lab=nominal] {} {\meaningof{\quotep{E}} = \{ \quotep{P} \in \quotep{\pi} | P \in \meaningof{E} \}, \and \meaningof{\quotep{P}} = \{ \quotep{Q} \in \quotep{\pi} | P \equiv Q \} \and \\ \meaningof{@\quotep{E}} = \{ P \in \pi | P \equiv @x, x \in \meaningof{E} \}}
\end{mathpar}

\begin{eqnarray*}
  \\
  \meaningof{-} : TS \to ST
\end{eqnarray*}

\begin{eqnarray*}
  \\
  L : TS \to ST
\end{eqnarray*}

\begin{eqnarray*}
  \\
  P \models E \iff P \in \meaningof{E}
\end{eqnarray*}

\begin{eqnarray*}
  P \approx_{L} Q \iff \forall E \in L. P \models E \iff Q \models E
\end{eqnarray*}

\begin{eqnarray*}
  P \approx_{K} Q
\end{eqnarray*}

\begin{eqnarray*}
  P \approx Q
\end{eqnarray*}

$\approx_{K} = \approx = \approx_{L}$

\subsubsection{Contextual duality}

Note that contexts extend the quotation operation to a family of
operations from processes to names. Given a context, $M$, we can
define a \emph{nominal context}, $\quotep{M}$ by $\quotep{M}[P] :=
\quotep{M[P]}$. To foreshadow what is to come we observe that these
operations enjoy a duality with processes very much like the duality
between vectors and maps from vectors to scalars.

Further, because the calculus is essentially higher-order, we have a
correspondence between contexts and processes. More specifically,
given a name $x$ and a context $M$ we can construct $M^{*}_{x}$ such
that 

\begin{mathpar}
  M^{*}_{x} | \lift{x}{P} \red M[P]
\end{mathpar}

namely,

\begin{mathpar}
  M^{*}_{x} := x?(u).M[\dropn{u}]
\end{mathpar}

The dependence of $M^{*}_{x}$ on a name makes it an abstraction, 

\begin{mathpar}
  M^{*} := (x)x?(u).M[\dropn{u}]
\end{mathpar}

\subsection{Additional notation}

It will sometimes be convenient to denote the process a name
quotes. We already have the notation $x = \quotep{P}$, but it will be
convenient to introduce an alternate notation, $\procn{x}$, when we
want to emphasize the connection to the use of the name. Note that, by
virtue of name equivalence, $\quotep{\procn{x}} \nameeq x$; so, the
notation is consistent with previous definitions.

Further, because names have structure it is possible to effect
substitutions on the basis of that structure. This means we need to
upgrade our notation for substitutions, which we accomplish by
adapting comprehension notation. Thus,

\begin{mathpar}
  P\{ y / x : x \in S \}
\end{mathpar}

is interpreted to mean the process derived from P by replacing (in a
capture-avoiding manner) each occurrence of $x$ in $S$ by $y$. For example,

\begin{mathpar}
  P\{ \quotep{\procn{x}|\procn{x}} / x : x \in \freenames{P} \}
\end{mathpar}

will replace each (occurrence) of a free name $x$ in $P$ by
$\quotep{\procn{x}|\procn{x}}$.

Also, we will avail ourselves of the notation $x^{L}$ and $x^{R}$ to
denote injections of a name into disjoint copies of the name
space. There are numerous ways to accomplish this. One example can be
found in \cite{MeredithR05}. This notation overloads to vectors of
names: $\vec{x}^{\pi} := (x_{i}^{\pi} \; : \; 0 \leq i < |\vec{x}| )$ where $\pi \in \{L,R\}$.

We also use $P^{\Box} := P|\Box$.

In \cite{MeredithR05} an interpretation of the new operator is
given. It turns out that there are several possible interpretations
all enjoying the requisite algebraic properties of the operator (see
\cite{milner91polyadicpi}). We will therefore make liberal use of
$(\nu\; \vec{x})P$.

% subsection the_syntax_and_semantics_of_the_notation_system (end)   

\section{Interpretation of QM}
\subsection{Supporting definitions}
\subsubsection{Multiplication}
\begin{mathpar}
  \quotep{Q} \cdot \quotep{R} := \quotep{Q|R}
  \and \\
  \quotep{Q} \cdot P := P\{ \quotep{Q|R} / \quotep{R} : \quotep{R} \in \freenames{P} \}
\end{mathpar}

\paragraph{Discussion}
The first line needs little explanation. The second line says that
each free name of the process is replaced with the multiplication of
that name by the scalar. Multiplication of a scalar (name) by a state
(process) results in a process all the names of which have been `moved
over' by parallel composition with the process the scalar
quotes. There is a subtlety that the bound names have to be
manipulated so that multiplied names aren't accidentally
captured. There are many ways to achieve this.

\begin{remark}\label{rem:multiplication_identities}
  The reader is invited to verify that for all $x,y,z \in \QProc$ and $P \in \Proc$
  \begin{mathpar}
    x \cdot \quotep{0} \equiv x 
    \and
    x \cdot y \equiv y \cdot x
    \and
    x \cdot (y \cdot z) \equiv (x \cdot y) \cdot z
    \and \\
    \quotep{0} \cdot P \equiv P
    \and \\
    x \cdot (y \cdot P) \equiv (x \cdot y) \cdot P
    \and \\
    x \cdot (P|Q) \equiv (x \cdot P) | (x \cdot Q)
    \and \\    
  \end{mathpar}
\end{remark}

\subsubsection{Tensor product}

We define a tensor product on processes by structural induction.

\paragraph{Tensor of sums} First note that all summations, including
$\pzero$ and sequence, can be written $\Sigma_{i} x_{i}.A_{i} +
\Sigma_{j} x_{j}.C_{j}$, where we have grouped input-guarded processes
together and output-guarded processes together.

Thus, we can define the tensor product of two summations, $N_{1}\otimes N_{2}$, where

\begin{mathpar}
  N_{1} := \Sigma_{i} x_{i}.A_{i} + \Sigma_{j} x_{j}.C_{j}
  \and
  N_{2} := \Sigma_{i'} y_{i'}.B_{i'} + \Sigma_{j'} y_{j'}.D_{j'} 
\end{mathpar}

as follows.

\begin{mathpar}
  \Sigma_{i} x_{i}.A_{i} + \Sigma_{j} x_{j}.C_{j} \otimes \Sigma_{i'}
  y_{i'}.B_{i'} + \Sigma_{j'} y_{j'}.D_{j'} 
  \and \\
  := \; \Sigma_{i} \Sigma_{i'} \quotep{\stackrel{\vee}{x_{i}}| \stackrel{\vee}{y_{i'}}}.(A_{i}\otimes B_{i'}) \; | \; \Sigma_{i'} \Sigma_{i} \quotep{\stackrel{\vee}{y_{i'}}|\stackrel{\vee}{x_{i}}}.(B_{i'}\otimes A_{i})
  \and
  \;\; | \;\; \Sigma_{j} \Sigma_{j'} \quotep{\stackrel{\vee}{x_{j}}|\stackrel{\vee}{y_{j'}}}.(A_{j}\otimes B_{j'}) \; | \; \Sigma_{j'} \Sigma_{j} \quotep{\stackrel{\vee}{y_{j'}}|\stackrel{\vee}{x_{j}}}.(B_{j'}\otimes A_{j})
\end{mathpar}

\begin{remark}
  Do we need to $x^{L}$ and $y^{R}$ for this construction as well?
\end{remark}

\paragraph{Tensor of parallel compositions} Next, we distribute tensor
over par.

\begin{mathpar}
  P_{1}|P_{2} \otimes Q_{1}|Q_{2} := (P_{1} \otimes Q_{1}) | (P_{1}
  \otimes Q_{2}) | (P_{2} \otimes Q_{1}) | (P_{2} \otimes Q_{2})
\end{mathpar}

\paragraph{Tensor with dropped names} We treat tensor of a
process with a dropped name as parallel composition.

\begin{mathpar}
  P \otimes \dropn{x} := P | \dropn{x}
\end{mathpar}

\paragraph{Tensor of agents}

Finally, we need to define tensor on agents. Note that the definition
of tensor on normal products only tensors inputs with inputs and
outputs with outputs. Thus, we only have to define the operation on
``homogeneous'' pairings.

\begin{mathpar}
  (\vec{x})P \otimes (\vec{y})Q
  \and \\
  := (x_{0}^{L}|y_{0}^{R},\ldots,x_{0}^{L}|y_{n}^{R},\ldots,x_{m}^{L}|y_{0}^{R},\ldots,x_{m}^{L}|y_{n}^R)(P\{ \vec{x}^{L}/\vec{x}\} \otimes Q \{ \vec{y}^{R}/\vec{y}\})
  \and \\
  \clift{\vec{P}} \otimes \clift{\vec{Q}}
  \and \\
  := \clift{P_{0}\otimes Q_{0},\ldots,P_{0}\otimes Q_{n},\ldots,P_{m}\otimes Q_{0},\ldots,P_{m}\otimes Q_{n}}
\end{mathpar}

\begin{remark}
  Observe that arities of tensored abstractions matches arities of
  tensored concretions if the original arities matched. Note also that
  the length of the arities corresponds to the increase in dimension
  we see in ordinary vector space tensor product.
\end{remark}

\begin{remark}
  Operationally, this definition distributes the tensor down to
  components ``linked'' by summation. Tensor over summation is
  intriguing in that it mixes names. Moreover, as a consequence of the
  way it mixes names we have the identities for all $x \in \QProc$ and
  $P,Q \in \Proc$

  \begin{mathpar}
    (x \cdot P) \otimes Q \equiv x \cdot (P \otimes Q) \equiv P \otimes (x \cdot Q)
    \and
    P \otimes \pzero \equiv P
  \end{mathpar}

  that the reader is invited to verify.
\end{remark}

\subsubsection{Annihilation}
\begin{mathpar}
  P^{\perp} := \{ Q | \forall R. P|Q \red^{*} R \Rightarrow R \red^{*} \pzero \}
  \and \\
  P^{\underline{\perp}} := \Sigma_{Q \in P^{\perp}} \quotep{Q}?(y).(\dropn{y}|Q) | \Sigma_{Q \in P^{\perp}} \quotep{Q}\clift{\Box}
\end{mathpar}

\paragraph{Discussion} The reader will note that $P^{\perp}$ is a
\emph{set} of processes, while $P^{\underline{\perp}}$ is a
\emph{context}. We call the set $P^{\perp}$ the \emph{annihilators} of
$P$. The parallel composition of a process in the annihilators of $P$
with $P$ will result in a process, the state space of which has all
paths eventually leading to $\pzero$. Execution may endure loops; but
under reasonable conditions of fairness (naturally guaranteed under
most notions of bisimulation) such a composite process cannot get
stuck in such a loop and will, eventually pop out and terminate.

The context $P^{\underline{\perp}}$ is ready and willing to ``take the
$P$ out of'' the process to which it is applied. It will effectively
transmit the code of the process to which it is applied to one of the
annihilators and run the process against it.

\subsubsection{Evaluation}
We fix $M$ a domain of fully abstract interpretation with an equality
coincident with bisimulation. We take $\meaningof{\cdot} : \Proc \to
M$ to be the map interpreting processes and $\nmeaningof{\cdot} : \M
\to Proc$ to be the map running the other way. Then we define

\begin{mathpar}
  \int P := \nmeaningof{\meaningof{P}}
\end{mathpar}

\paragraph{Discussion}
There are many fully abstract interpretations of Milner's
$\pi$-calculus. Any of them can be used as a basis for interpreting
the reflective calculus here. Equipped with such a domain it is
largely a matter of grinding through to check that the Yoneda
construction for the normalization-by-evaluation program can be
extended to this setting.

\begin{remark}
  The reader is invited to verify that $\int (P^{\underline{\perp}}[P]) = 0$.
\end{remark}

\subsection{Quantum mechanics}

Table \ref{tbl:core_qm_op_defns} gives the core operational definitions

\begin{table}[htp]\label{tbl:core_qm_op_defns}
  \center{
    \fbox{
      \begin{tabular}{c|c}
        quantum mechanics & process calculus \\
        \hline
        scalar & $x := \quotep{P}$ \\
        state vector & $\state{P} := P$ \\
        dual & $\state{P}^{*} := \event{P^{\underline{\perp}}} := \quotep{P^{\underline{\perp}}}[-]$ \\
        matrix & $ \Sigma_{\alpha} \state{P_{\alpha}}x_{\alpha}\event{Q_{\alpha}}$ \\
        vector addition & $\state{P} + \state{Q} := \state{P | Q}$ \\
        tensor product & $\state{P} \otimes \state{Q} := \state{P \otimes Q}$ \\
        inner product & $\innerprod{P}{Q} := \quotep{\int P^{\underline{\perp}}[Q]}$ \\
      \end{tabular}
    }
  }
  \caption{QM - operational definitions}
\end{table}

where

\begin{mathpar}
  \prmatrix{P}{Q} := \fprmatrix{P}{\quotep{\pzero}}{Q}
  \and
  \fprmatrix{P}{x}{Q} := (\state{P},x,\event{Q})
  \and
  (\fprmatrix{P}{x}{Q})(\state{R}) := x \cdot \innerprod{Q}{R} \cdot \state{P}
  \and
  (\fprmatrix{P}{x}{Q})(\event{R}) := x \cdot \innerprod{R}{P} \cdot \event{Q}
\end{mathpar}

\paragraph{Discussion}
As promised: vectors (aka states) are represented as processes; duals
as contextual duals; inner product definition should be compared with
standard inner product definition for ....

\begin{remark}
  Assuming $\int (P^{\underline{\perp}}[P]) = 0$, the reader is
  invited to verify that $(\fprmatrix{P}{x}{P})(\state{P}) = x \cdot \state{P}$.
\end{remark}

\begin{remark}
  The reader is invited to verify that $\innerprod{P}{Q}$ could
  equally well have been written $\quotep{\int \stackrel{\vee}{x}}$
  where $x = \event{P^{\underline{\perp}}}(Q)$.

  One of the motivations for this remark is that there is another way
  to factor these operations. We could package up evaluation in the dual:

  \begin{mathpar}
    \state{P}^{*} := \event{\int P^{\underline{\perp}}} := \quotep{\int P^{\underline{\perp}}}[-]
  \end{mathpar}

  and then have inner product defined by
  
  \begin{mathpar}
    \innerprod{P}{Q} := \event{P}(Q)
  \end{mathpar}

  Hopefully, experience with the calculations will provide guidance on
  the best factoring.
\end{remark}

\begin{remark}
  Assuming $\int (P^{\underline{\perp}}[P]) = 0$, the reader is
  invited to verify that $\forall P,Q. (\prmatrix{0}{Q})(\state{0}) =
  \state{0}$ and dually $(\prmatrix{P}{0})(\event{0}) = \event{0}$.
\end{remark}

\begin{remark}
  i'm a little worried that i don't (yet) have proper support for
  complex conjugacy. But, the observation above may give us a
  clue. According to Abramsky, it must be the case that the scalars
  are iso to the homset of the identity for the tensor -- which the
  observation above characterizes. 

  For now, we will simply bookmark the notion with $\overline{x}$.
\end{remark}

\subsubsection{Adjointness}

We need to give a definition of $(\cdot)^{\dagger}$ for matrices. The
obvious candidate definition is
\begin{mathpar}
(\Sigma_{\alpha}\fprmatrix{P_{\alpha}}{x_{\alpha}}{Q_{\alpha}})^{\dagger}
= \Sigma_{\alpha}\fprmatrix{(Q_{\alpha}^{\underline{\perp}})^{*}}{\overline{x}_{\alpha}}{P_{\alpha}^{\underline{\perp}}} 
\end{mathpar}

But, $(Q_{\alpha}^{\underline{\perp}})^{*}$ requires a name along
which to communicate the process to achieve the context application.

\subsubsection{Basis for a basis}
If processes label states and ``addition'' of states (a.k.a. vector
addition) is interpreted as parallel composition, what corresponds to
notions of linear independence and basis? Here, we recall that Yoshida
has developed a set of \emph{combinators} for an asynchronous verison
of Milner's $\pi$-calculus. These are a finite set of processes such
any process can be expressed as parallel composition of these
combinators together with liberal uses of the new operator and
replication. We can simply give a translation of these into the
present calculus and have reasonable expectation that the property
carries over. That is, that the resultant set allows to express all
processes via parallel composition. Note, however, that there is no
new operator or replication in this calculus. As a result, we expect
that the corresponding set is actually infinite. That is, we expect
that the space is actually infinite dimensional.

\begin{remark}
  The attentive reader may be a bit concerned. Certainly, the
  collection $S$, $K$ and $I$ is a finite set of
  combinators. Shouldn't we expect to see a finite set of combinators
  for an effectively equivalent system? i am very sympathetic to this
  critique and feel it warrants full attention. On the other hand, i
  also have in mind the following analogy. The natural numbers, as a
  monoid under addition, has exactly $1$ generator, while the natural
  numbers, as a monoid under multiplication, has countably many
  generators (the primes). We observe that the application of the
  lambda calculus is much less resource sensitive than the parallel
  composition of the $\pi$-calculus. Could it be the case that we have
  an analogy of the form
  
  \begin{mathpar}
    m + n : MN :: m*n : M|N
  \end{mathpar}

  giving a similar blow up in the set of ``primes''?  This is such a
  wonderful thought that, even if it's not true, i think it's worth
  writing down.
\end{remark}
 

\documentclass[12pt]{llncs}
%\documentclass{jktr}

\usepackage[pdftex]{hyperref}                   
\usepackage {listings}
\usepackage {mathpartir}
\usepackage{bcprules}
%\usepackage{listings}
                       
\usepackage{graphicx} 
%\usepackage[margins=2.5cm,nohead,nofoot]{geometry}
%\usepackage{geometry}
\usepackage{amsfonts}
\usepackage{amstext}
\usepackage{latexsym}
\usepackage{amssymb}
\usepackage{color}


%\include{myPreamble}
\include{qm2pi.local} 

%\ifpdf
%\usepackage[pdftex]{graphicx}
%\else
%\usepackage{graphicx}
%\fi

 % \ifpdf
%  \usepackage{pdfsync}
%  \if


%\title{Brief Article}
%\author{David F. Snyder}
%\author{L.G. Meredith}

%\address{Dept. of Math., Texas State University--San Marcos, San Marcos, TX 78666}
       
\pagestyle{empty}


\begin{document}

\lstset{language=[Objective]Caml,frame=shadowbox}

\input{qm2pi.front}

% section front matter (end)

\input{qm2pi.intro} 
 
% section introduction (end)

% \input{qm2pi.knotations} 

% section notation (end)

\input{qm2pi.process.calculi} 

% section concurrent_process_calculi_and_spatial_logics_ (end)
    
%\input{qm2pi.knots2pi} 

%\input{qm2pi.trefoil} 

%\input{qm2pi.mainthm} 

% subsection basic_interpretation (end)

%\input{qm2pi.rho.presentation} 
\subsection{The syntax and semantics of the notation system}\label{sub:the_syntax_and_semantics_of_the_notation_system} % (fold)

We now summarize a technical presentation of the calculus that
embodies our theory of dynamics. The typical presentation of such a
calculus follows the style of giving generators and relations on
them. The grammar, below, describing term constructors, freely
generates the set of processes, $\Proc$. This set is then quotiented
by a relation known as structural congruence and it is over this set
that the notion of dynamics is expressed. This presentation is
essentially that of \cite{MeredithR05} with the addition of
polyadicity and summation. For readability we have relegated some of
the technical subtleties to an appendix.

\subsubsection{Process grammar}\label{subsub:process_grammar}

\begin{mathpar}
  \inferrule* [lab=synchronization] {} {{M} \bc \pzero \;|\; x?F \;|\; x!C }
  \and
  \inferrule* [lab=abstraction] {} {{F} \bc (x)P}
  \and
  \inferrule* [lab=concretion] {} {{C} \bc \langle Q \rangle}
  \and
  \inferrule* [lab=process] {} {{P,Q} \bc M \;| \;P|Q \;|\; @{x}}
  \and
  \inferrule* [lab=name] {} {{x} \bc \quotep{P}}
\end{mathpar} 

Note that $\vec{x}$ (resp. $\vec{P}$) denotes a vector of names
(resp. processes) of length $|\vec{x}|$ (resp. $|\vec{P}|$). We adopt
the following useful abbreviations.

\begin{mathpar}
   x?(\vec{y}).P := x.(\vec{y})P \and  x\clift{\vec{P}} := x.\clift{\vec{P}}
   \and x!(y) := \lift{x}{\dropn{y}}
   \and \Pi_{i=0}^{n-1}P_i := P_0 | \ldots | P_{n-1}
\end{mathpar}

\subsubsection{Structural congruence}

\paragraph{Free and bound names and alpha-equivalence.} At the
core of structural equivalence is alpha-equivalence which identifies
process that are the same up to a change of variable. Formally, we
recognize the distinction between free and bound names. The free names
of a process, $\freenames{P}$, may be calculated recursively as
follows:

\begin{mathpar}
\freenames{\pzero} := \emptyset
  \and \\
  \freenames{x?(y).P} := \{ x \} \cup (\freenames{P} \setminus \{ y \})
  \and 
  \freenames{x!\langle P \rangle} := \{ x \} \cup \{ P \} 
  \and \\
  \freenames{P|Q} := \freenames{P} \cup \freenames{Q}
  \and \\
  \freenames{@{x}} := \{ x \}
\end{mathpar}

$\pi$
$\quotep{\pi}$

$\freenames{-} : \pi \to \mathcal{P}(\quotep{\pi})$

\begin{eqnarray*}
  \freenames{\pzero} & := & \emptyset \\
  \freenames{x?(y).P} & := & \{ x \} \cup (\freenames{P} \setminus \{ y \}) \\
  \freenames{x!\langle P \rangle} & := & \{ x \} \cup \{ P \} \\
  \freenames{P|Q} & := & \freenames{P} \cup \freenames{Q} \\
  \freenames{\dropn{x}} & := & \{ x \}
\end{eqnarray*}

The bound names of a process, $\boundnames{P}$, are those names occurring in $P$
that are not free. For example, in $x?(y).0$, the name $x$ is free, while $y$ is bound.

\begin{mathpar}
  \inferrule* [lab=monoidal-laws] {} { P|Q \equiv Q|P \and P|0 \equiv P \and P|(Q|R) \equiv (P|Q)|R }
\end{mathpar}

\begin{mathpar}
  \inferrule* [lab=alpha-equivalence] {} { (x)P \equiv (y)P\{y/x\} \and y \not\in \freenames{P} }
\end{mathpar}

\begin{definition}
Then two processes, $P,Q$, are alpha-equivalent if $P = Q\{\vec{y}/\vec{x}\}$ for
some $\vec{x} \in \boundnames{Q},\vec{y} \in \boundnames{P}$, where $Q\{\vec{y}/\vec{x}\}$
denotes the capture-avoiding substitution of $\vec{y}$ for $\vec{x}$ in $Q$.
\end{definition}

\begin{definition}
  The {\em structural congruence} \cite{SangiorgiWalker} , $\equiv$,
  between processes is the least congruence containing
  alpha-equivalence, satisfying the abelian monoid laws
  (associativity, commutativity and $\pzero$ as identity) for parallel
  composition $|$ and for summation $+$.
\end{definition}

\subsection{Name equivalence}

We take name equivalence, written $\nameeq$, to be the smallest
equivalence relation generated by the following rules.

\begin{mathpar}
\inferrule*[lab=Quote-drop]
{ }
{ \quotep{@{x}} \nameeq x }

\inferrule*[lab=Struct-equiv]
{ P \scong Q }
{ \quotep{P} \nameeq \quotep{Q} }
\end{mathpar}

The astute reader will have noticed that the mutual recursion of names
and processes imposes a mutual recursion on alpha-equivalence and
structural equivalence via name-equivalence. Fortunately, all of this
works out pleasantly and we may calculate in the natural way, free of
concern. The reader interested in the details is referred to the
appendix \ref{appendix:rho_details}.

\subsection{Substitution}

We use $\Proc$ for the set of processes, $\QProc$ for the set of
names, and $\id{\{}\vec{y} / \vec{x} \id{\}}$ to denote partial maps,
$s : \QProc \rightarrow \QProc$. A map, $s$ lifts, uniquely, to a map
on process terms, $\widehat{s} : \Proc \rightarrow \Proc$ by the
following equations.

\begin{mathpar}
  (0) \psubstp{Q}{P} := 0 \\
  (R \juxtap S) \psubstp{Q}{P}
  :=    
  (R)\psubstp{Q}{P} \juxtap (S) \psubstp{Q}{P} \\
  (x?(y).R) \psubstp{Q}{P}    
  :=    
  (x)\substp{Q}{P} (z)\concat( (R \psubstn{z}{y}) \psubstp{Q}{P} ) \\
  (\lift{x}{R}) \psubstp{Q}{P}  
  :=
  \lift{(x)\substp{Q}{P}}{ R \psubstp{Q}{P} } \\
%   (\dropn{x})  \psubstp{Q}{P}       
%   := 
%   \left\{ 
%     \begin{array}{ccc} 
%       \dropn{\quotep{Q}} & & x \nameeq \quotep{P} \\
%       \dropn{x} & & otherwise \\
%     \end{array}
%   \right. 
  (\dropn{x})  \psubstp{Q}{P}       
  := 
  \left\{ 
    \begin{array}{ccc} 
      Q & & x \nameeq \quotep{P} \\
      \dropn{x} & & otherwise \\
    \end{array}
  \right.
\end{mathpar}
 

where

\begin{eqnarray}
  (x)\id{\{} \lpquote Q \rpquote / \lpquote P \rpquote \id{\}}            = 
  \left\{ 
    \begin{array}{ccc}
      \lpquote Q \rpquote & & x \nameeq \lpquote P \rpquote \\
      x & & otherwise \\
    \end{array}
  \right. \nonumber
\end{eqnarray}

and $z$ is chosen distinct from $\quotep{P}$, $\quotep{Q}$, the free
names in $Q$, and all the names in $R$. Our $\alpha$-equivalence will
be built in the standard way from this substitution.

\begin{remark}\label{rem:no_self_referential_names}
  One consequence of these definitions is that $\forall P. \quotep{P}
  \not\in \freenames{P}$.
\end{remark}

\subsection{ Dynamic quote: an example }

Anticipating something of what's to come, consider applying the
substitution, $\widehat{\id{\{}u / z \id{\}}}$, to the following pair
of processes, $\lift{w}{y!(z)}$ and $w[ \lpquote y!(z) \rpquote ]$.

\begin{eqnarray}
	\lift{w}{y!(z)}\widehat{\id{\{}u / z \id{\}}}
		& = &
		\lift{w}{y!(u)} \nonumber\\
	w[ \lpquote y!(z) \rpquote ] \widehat{ \id{\{}u / z \id{\}} }
		& = &
		w[ \lpquote y!(z) \rpquote ] \nonumber
\end{eqnarray}

Because the body of the process between quotes is impervious to
substitution, we get radically different answers. In fact, by
examining the first process in an input context,
e.g. $x?(z).\lift{w}{y!(z)}$, we see that the process under the lift
operator may be shaped by prefixed inputs binding a name inside it. In
this sense, the lift operator will be seen as a way to dynamically
construct processes before reifying them as names.

Finally equipped with these standard features we can present the
dynamics of the calculus.

\subsubsection{Operational semantics} 

Finally, we introduce the computational dynamics. What marks these
algebras as distinct from other more traditionally studied algebraic
structures, e.g. vector spaces or polynomial rings, is the manner in
which dynamics is captured. In traditional structures, dynamics is typically
expressed through morphisms between such structures, as in linear maps
between vector spaces or morphisms between rings. In algebras
associated with the semantics of computation, the dynamics is
expressed as part of the algebraic structure itself, through a
reduction reduction relation typically denoted by $\red$. Below, we
give a recursive presentation of this relation for the calculus used
in the encoding.

$\red \subseteq \pi \times \pi$
$\red : \pi \to \mathcal{P}(\pi)$

\begin{mathpar}
  \inferrule* [lab=Comm] { \textsf{match}( x_{src}, x_{trgt} ) } { x_{trgt}?(y)P \; | \; x_{src}!\langle {Q} \rangle \red P\{\quotep{Q}/y}\} }
  \and \\
  \inferrule* [lab=Par] {{P} \red {P}'} {{{P} | {Q}} \red {{P}' | {Q}}}
  \and
  \inferrule* [lab=Equiv]{{{P} \scong {P}'} \andalso {{P}' \red {Q}'} \andalso {{Q}' \scong {Q}}}{{P} \red {Q}}
\end{mathpar}

\begin{eqnarray*}
  match_{\equiv} (\quotep{P},\quotep{Q}) & := & P \equiv Q \\
  match_{\dagger}(\quotep{P},\quotep{Q}) & := & \forall R. P|Q \red^{*} R => R \red^{*} 0 \\
  match_{K}(\quotep{P},\quotep{Q}) & := & K \mbox{ for some context } K
\end{eqnarray*}

$u?(x)P | u!\langle Q \rangle \red P\{\quotep{Q}/x\}$

%We write $\wred$ for $\red^*$, and $P\red$ if $\exists Q $ such that $ P \red Q$.
We write $P\red$ if $\exists Q $ such that $ P \red Q$ and $P\not\red$, otherwise.

\section{Replication}

As mentioned before, it is known that replication (and hence
recursion) can be implemented in a higher-order process algebra
\cite{SangiorgiWalker}. As our first example of calculation with the
machinery thus far presented we give the construction explicitly in
the {\rhoc}.

\begin{eqnarray}
	D_{x} & := & \prefix{x}{y}{(\binpar{\outputp{x}{y}}{@{y}})} \nonumber\\
	\bangp_{x}{P} & := & \binpar{{x}!\langle{\binpar{D_{x}}{P}}\rangle}{D_{x}} \nonumber
\end{eqnarray}

\begin{eqnarray}
	\bangp_{x}{P} & & \nonumber\\
	=
	& {x}!\langle{(\prefix{x}{y}{(\outputp{x}{y} | @{y})) | P}}\rangle 
	      | \prefix{x}{y}{(\outputp{x}{y} | @{y})} & \nonumber\\
	\red
	& (\outputp{x}{y} | @{y})\substn{\quotep{(\prefix{x}{y}{(@{y} | \outputp{x}{y})) | P}}}{y} & \nonumber\\
	=
	& \outputp{x}{\quotep{(\prefix{x}{y}{(\outputp{x}{y} | @{y})) | P}}}
	  | {(\prefix{x}{y}{(\outputp{x}{y} | @{y})) | P}} & \nonumber\\
	\red
	& \ldots & \nonumber\\
	\red^*
	& P | P | \ldots & \nonumber
\end{eqnarray}

Of course, this encoding, as an implementation, runs away, unfolding
$\bangp{P}$ eagerly. A lazier and more implementable replication
operator, restricted to input-guarded processes, may be obtained as follows.

\begin{eqnarray}
\bangp{\prefix{u}{v}{P}} 
	:= 
	\binpar{\lift{x}{\prefix{u}{v}{(\binpar{D(x)}{P})}}}{D(x)} \nonumber
\end{eqnarray}

\begin{remark}
  Note that the lazier definition still does not deal with summation
  or mixed summation (i.e. sums over input and output). The reader is
  invited to construct definitions of replication that deal with these
  features. 

  Further, the definitions are parameterized in a name, $x$. Can you,
  gentle reader, make a definition that eliminates this parameter and
  guarantees no accidental interaction between the replication
  machinery and the process being replicated -- i.e. no accidental
  sharing of names used by the process to get its work done and the
  name(s) used by the replication to effect copying. This latter
  revision of the definition of replication is crucial to obtaining
  the expected identity $!!P \sim !P$.
\end{remark}

\begin{remark}\label{rem:paradoxical_combinator}
  The reader familiar with the lambda calculus will have noticed the
  similarity between $D$ and the paradoxical combinator.

  [Ed. note: the existence of this seems to suggest we have to be more
  restrictive on the set of processes and names we admit if we are to
  support no-cloning.]
\end{remark}

\subsubsection{Bisimulation}

The computational dynamics gives rise to another kind of equivalence,
the equivalence of computational behavior. As previously mentioned
this is typically captured \emph{via} some form of bisimulation.

% The notion we use in this paper is weak barbed bisimulation
% \cite{milner91polyadicpi}.

The notion we use in this paper is derived from weak barbed
bisimulation \cite{milner91polyadicpi}. 

\begin{definition}
An \emph{observation relation}, $\downarrow_{\mathcal N}$, over a set
of names, $\mathcal N$, is the smallest relation satisfying the rules
below.

\infrule[Out-barb]{y \in {\mathcal N}, \; x \nameeq y}
		  {\outputp{x}{v} \downarrow_{\mathcal N} x}
\infrule[Par-barb]{\mbox{$P\downarrow_{\mathcal N} x$ or $Q\downarrow_{\mathcal N} x$}}
		  {\binpar{P}{Q} \downarrow_{\mathcal N} x}

We write $P \Downarrow_{\mathcal N} x$ if there is $Q$ such that 
$P \wred Q$ and $Q \downarrow_{\mathcal N} x$.
\end{definition}

\begin{definition}
%\label{def.bbisim}
An  ${\mathcal N}$-\emph{barbed bisimulation} over a set of names, ${\mathcal N}$, is a symmetric binary relation 
${\mathcal S}_{\mathcal N}$ between agents such that $P\rel{S}_{\mathcal N}Q$ implies:
\begin{enumerate}
\item If $P \red P'$ then $Q \wred Q'$ and $P'\rel{S}_{\mathcal N} Q'$.
\item If $P\downarrow_{\mathcal N} x$, then $Q\Downarrow_{\mathcal N} x$.
\end{enumerate}
$P$ is ${\mathcal N}$-barbed bisimilar to $Q$, written
$P \wbbisim_{\mathcal N} Q$, if $P \rel{S}_{\mathcal N} Q$ for some ${\mathcal N}$-barbed bisimulation ${\mathcal S}_{\mathcal N}$.
\end{definition}

$\mathcal{R} \subseteq \pi \times \pi$

$P \mathcal{R} Q => \forall P'. P \red P' \Rightarrow \exists Q'. Q \red Q', P' \mathcal{R} Q'$

$P \vdash x \Rightarrow Q \vdash x$

\begin{mathpar}
  \inferrule*[lab=Out-barb]{x \nameeq y}{{y}!\langle{Q}\rangle \vdash x}
  \and
  \inferrule*[lab=Par-barb]{\mbox{$P\vdash x$ or $Q\vdash x$}}{\binpar{P}{Q} \vdash x}
\end{mathpar}

\subsubsection{Contexts}

One of the principle advantages of computational calculi like the
$\pi$-calculus is a well-defined notion of context,
contextual-equivalence and a correlation between
contextual-equivalence and notions of bisimulation. The notion of
context allows the decomposition of a process into (sub-)process and
its syntactic environment, its context. Thus, a context may be
thought of as a process with a ``hole'' (written $\Box$) in it. The
application of a context $M$ to a process $P$, written $M[P]$, is
tantamount to filling the hole in $M$ with $P$. In this paper we do
not need the full weight of this theory, but do make use of the notion
of context in the proof the main theorem. 

\begin{mathpar}
  \inferrule* [lab=summation] {} {{M_{M},M_{N}} \bc \Box \;|\; x.M_{A} \;|\; M_{M}+M_{N}}
  \and
  \inferrule* [lab=agent] {} {{M_{A}} \bc (\vec{x})M_{P} \;| \; \clift{P_0,\ldots,M_{P},\ldots,P_N}}
  \and \\
  \inferrule* [lab=process] {} {{M_{P}} \bc M_{N} \;| \;P|M_{P} }
\end{mathpar} 

\begin{mathpar}
  \inferrule* [lab=sychronization] {} {M_{N} \bc \Box \;|\; x?M_{F} \;|\; x!M_{C}}
  \and
  \inferrule* [lab=abstraction] {} {{M_{F}} \bc (x)M_{P} }
  \and
  \inferrule* [lab=concretion] {} {{M_{C}} \bc \langle M_{P} \rangle }
  \and \\
  \inferrule* [lab=process] {} {{M_{P}} \bc M_{N} \;| \;P|M_{P} }
\end{mathpar}

\begin{definition}[contextual application] Given a context $M$, and
  process $P$, we define the \emph{contextual application}, $M[P] :=
  M\{P/\Box\}$. That is, the contextual application of M to P is the
  substitution of $P$ for $\Box$ in $M$.
\end{definition}

$\meaningof{-} : L \to \mathcal{P}(\pi)$

\begin{mathpar}
  \inferrule* [lab=collection] {} {\meaningof{true} = \pi, \and \meaningof{~E} = \pi \setminus \meaningof{E}, \and \meaningof{E_{1} \& E_{2}} = \meaningof{E_{1}} \cap \meaningof{E_{2}}}
\end{mathpar}

\begin{mathpar}
  \inferrule* [lab=structure] {} {\meaningof{0} = \{ P \in \pi | P \equiv 0 \}, \and \\ \meaningof{E_1 | E_2} = \{ P \in \pi | P \equiv P_{1} | P_{2}, P_{1} \in \meaningof{E_{1}}, P_{2} \in \meaningof{E_2}\} }
\end{mathpar}

\begin{mathpar}
 \inferrule* [lab=behavior] {} {\meaningof{\langle a?b \rangle E} = \{ P \in \pi | P \equiv Q | u?(y)P', \\ \and \\\\ \and \\ \;\;\; u \in \meaningof{a}, \forall z.P'\{z/y\} \in \meaningof{E\{z/b\}}\}, \and \\ \meaningof{a!E} = \{ P \in \pi | P \equiv Q | x!\langle P' \rangle, x \in \meaningof{a} P' \in \meaningof{E}\} }
\end{mathpar}

\begin{mathpar}
 \inferrule* [lab=nominal] {} {\meaningof{\quotep{E}} = \{ \quotep{P} \in \quotep{\pi} | P \in \meaningof{E} \}, \and \meaningof{\quotep{P}} = \{ \quotep{Q} \in \quotep{\pi} | P \equiv Q \} \and \\ \meaningof{@\quotep{E}} = \{ P \in \pi | P \equiv @x, x \in \meaningof{E} \}}
\end{mathpar}

\begin{eqnarray*}
  \\
  \meaningof{-} : TS \to ST
\end{eqnarray*}

\begin{eqnarray*}
  \\
  L : TS \to ST
\end{eqnarray*}

\begin{eqnarray*}
  \\
  P \models E \iff P \in \meaningof{E}
\end{eqnarray*}

\begin{eqnarray*}
  P \approx_{L} Q \iff \forall E \in L. P \models E \iff Q \models E
\end{eqnarray*}

\begin{eqnarray*}
  P \approx_{K} Q
\end{eqnarray*}

\begin{eqnarray*}
  P \approx Q
\end{eqnarray*}

$\approx_{K} = \approx = \approx_{L}$

\subsubsection{Contextual duality}

Note that contexts extend the quotation operation to a family of
operations from processes to names. Given a context, $M$, we can
define a \emph{nominal context}, $\quotep{M}$ by $\quotep{M}[P] :=
\quotep{M[P]}$. To foreshadow what is to come we observe that these
operations enjoy a duality with processes very much like the duality
between vectors and maps from vectors to scalars.

Further, because the calculus is essentially higher-order, we have a
correspondence between contexts and processes. More specifically,
given a name $x$ and a context $M$ we can construct $M^{*}_{x}$ such
that 

\begin{mathpar}
  M^{*}_{x} | \lift{x}{P} \red M[P]
\end{mathpar}

namely,

\begin{mathpar}
  M^{*}_{x} := x?(u).M[\dropn{u}]
\end{mathpar}

The dependence of $M^{*}_{x}$ on a name makes it an abstraction, 

\begin{mathpar}
  M^{*} := (x)x?(u).M[\dropn{u}]
\end{mathpar}

\subsection{Additional notation}

It will sometimes be convenient to denote the process a name
quotes. We already have the notation $x = \quotep{P}$, but it will be
convenient to introduce an alternate notation, $\procn{x}$, when we
want to emphasize the connection to the use of the name. Note that, by
virtue of name equivalence, $\quotep{\procn{x}} \nameeq x$; so, the
notation is consistent with previous definitions.

Further, because names have structure it is possible to effect
substitutions on the basis of that structure. This means we need to
upgrade our notation for substitutions, which we accomplish by
adapting comprehension notation. Thus,

\begin{mathpar}
  P\{ y / x : x \in S \}
\end{mathpar}

is interpreted to mean the process derived from P by replacing (in a
capture-avoiding manner) each occurrence of $x$ in $S$ by $y$. For example,

\begin{mathpar}
  P\{ \quotep{\procn{x}|\procn{x}} / x : x \in \freenames{P} \}
\end{mathpar}

will replace each (occurrence) of a free name $x$ in $P$ by
$\quotep{\procn{x}|\procn{x}}$.

Also, we will avail ourselves of the notation $x^{L}$ and $x^{R}$ to
denote injections of a name into disjoint copies of the name
space. There are numerous ways to accomplish this. One example can be
found in \cite{MeredithR05}. This notation overloads to vectors of
names: $\vec{x}^{\pi} := (x_{i}^{\pi} \; : \; 0 \leq i < |\vec{x}| )$ where $\pi \in \{L,R\}$.

We also use $P^{\Box} := P|\Box$.

In \cite{MeredithR05} an interpretation of the new operator is
given. It turns out that there are several possible interpretations
all enjoying the requisite algebraic properties of the operator (see
\cite{milner91polyadicpi}). We will therefore make liberal use of
$(\nu\; \vec{x})P$.

% subsection the_syntax_and_semantics_of_the_notation_system (end)   

\input{qm2pi.qmops} 

\input{qm2pi.sterngerlach} 

\input{qm2pi.metric} 

% section concurrent_process_calculi (end)

%\input{qm2pi.proofsketch}

% section proof sketch (end)

%\input{qm2pi.slviaknots} 

% section spatial logic via knots (end)

\input{qm2pi.conclusion}

% section conclusion (end)

%\input{qm2pi.dtcodes} 

% section wiring algorithm (end)

\input{qm2pi.ack} 

% section acknowledgments (end)

\newpage


\bibliographystyle{plain}   
\bibliography{../../biblios/main.bib}

\input{qm2pi.rhodetails}

\end{document}

 

\documentclass[12pt]{llncs}
%\documentclass{jktr}

\usepackage[pdftex]{hyperref}                   
\usepackage {listings}
\usepackage {mathpartir}
\usepackage{bcprules}
%\usepackage{listings}
                       
\usepackage{graphicx} 
%\usepackage[margins=2.5cm,nohead,nofoot]{geometry}
%\usepackage{geometry}
\usepackage{amsfonts}
\usepackage{amstext}
\usepackage{latexsym}
\usepackage{amssymb}
\usepackage{color}


%\include{myPreamble}
\include{qm2pi.local} 

%\ifpdf
%\usepackage[pdftex]{graphicx}
%\else
%\usepackage{graphicx}
%\fi

 % \ifpdf
%  \usepackage{pdfsync}
%  \if


%\title{Brief Article}
%\author{David F. Snyder}
%\author{L.G. Meredith}

%\address{Dept. of Math., Texas State University--San Marcos, San Marcos, TX 78666}
       
\pagestyle{empty}


\begin{document}

\lstset{language=[Objective]Caml,frame=shadowbox}

\input{qm2pi.front}

% section front matter (end)

\input{qm2pi.intro} 
 
% section introduction (end)

% \input{qm2pi.knotations} 

% section notation (end)

\input{qm2pi.process.calculi} 

% section concurrent_process_calculi_and_spatial_logics_ (end)
    
%\input{qm2pi.knots2pi} 

%\input{qm2pi.trefoil} 

%\input{qm2pi.mainthm} 

% subsection basic_interpretation (end)

%\input{qm2pi.rho.presentation} 
\subsection{The syntax and semantics of the notation system}\label{sub:the_syntax_and_semantics_of_the_notation_system} % (fold)

We now summarize a technical presentation of the calculus that
embodies our theory of dynamics. The typical presentation of such a
calculus follows the style of giving generators and relations on
them. The grammar, below, describing term constructors, freely
generates the set of processes, $\Proc$. This set is then quotiented
by a relation known as structural congruence and it is over this set
that the notion of dynamics is expressed. This presentation is
essentially that of \cite{MeredithR05} with the addition of
polyadicity and summation. For readability we have relegated some of
the technical subtleties to an appendix.

\subsubsection{Process grammar}\label{subsub:process_grammar}

\begin{mathpar}
  \inferrule* [lab=synchronization] {} {{M} \bc \pzero \;|\; x?F \;|\; x!C }
  \and
  \inferrule* [lab=abstraction] {} {{F} \bc (x)P}
  \and
  \inferrule* [lab=concretion] {} {{C} \bc \langle Q \rangle}
  \and
  \inferrule* [lab=process] {} {{P,Q} \bc M \;| \;P|Q \;|\; @{x}}
  \and
  \inferrule* [lab=name] {} {{x} \bc \quotep{P}}
\end{mathpar} 

Note that $\vec{x}$ (resp. $\vec{P}$) denotes a vector of names
(resp. processes) of length $|\vec{x}|$ (resp. $|\vec{P}|$). We adopt
the following useful abbreviations.

\begin{mathpar}
   x?(\vec{y}).P := x.(\vec{y})P \and  x\clift{\vec{P}} := x.\clift{\vec{P}}
   \and x!(y) := \lift{x}{\dropn{y}}
   \and \Pi_{i=0}^{n-1}P_i := P_0 | \ldots | P_{n-1}
\end{mathpar}

\subsubsection{Structural congruence}

\paragraph{Free and bound names and alpha-equivalence.} At the
core of structural equivalence is alpha-equivalence which identifies
process that are the same up to a change of variable. Formally, we
recognize the distinction between free and bound names. The free names
of a process, $\freenames{P}$, may be calculated recursively as
follows:

\begin{mathpar}
\freenames{\pzero} := \emptyset
  \and \\
  \freenames{x?(y).P} := \{ x \} \cup (\freenames{P} \setminus \{ y \})
  \and 
  \freenames{x!\langle P \rangle} := \{ x \} \cup \{ P \} 
  \and \\
  \freenames{P|Q} := \freenames{P} \cup \freenames{Q}
  \and \\
  \freenames{@{x}} := \{ x \}
\end{mathpar}

$\pi$
$\quotep{\pi}$

$\freenames{-} : \pi \to \mathcal{P}(\quotep{\pi})$

\begin{eqnarray*}
  \freenames{\pzero} & := & \emptyset \\
  \freenames{x?(y).P} & := & \{ x \} \cup (\freenames{P} \setminus \{ y \}) \\
  \freenames{x!\langle P \rangle} & := & \{ x \} \cup \{ P \} \\
  \freenames{P|Q} & := & \freenames{P} \cup \freenames{Q} \\
  \freenames{\dropn{x}} & := & \{ x \}
\end{eqnarray*}

The bound names of a process, $\boundnames{P}$, are those names occurring in $P$
that are not free. For example, in $x?(y).0$, the name $x$ is free, while $y$ is bound.

\begin{mathpar}
  \inferrule* [lab=monoidal-laws] {} { P|Q \equiv Q|P \and P|0 \equiv P \and P|(Q|R) \equiv (P|Q)|R }
\end{mathpar}

\begin{mathpar}
  \inferrule* [lab=alpha-equivalence] {} { (x)P \equiv (y)P\{y/x\} \and y \not\in \freenames{P} }
\end{mathpar}

\begin{definition}
Then two processes, $P,Q$, are alpha-equivalent if $P = Q\{\vec{y}/\vec{x}\}$ for
some $\vec{x} \in \boundnames{Q},\vec{y} \in \boundnames{P}$, where $Q\{\vec{y}/\vec{x}\}$
denotes the capture-avoiding substitution of $\vec{y}$ for $\vec{x}$ in $Q$.
\end{definition}

\begin{definition}
  The {\em structural congruence} \cite{SangiorgiWalker} , $\equiv$,
  between processes is the least congruence containing
  alpha-equivalence, satisfying the abelian monoid laws
  (associativity, commutativity and $\pzero$ as identity) for parallel
  composition $|$ and for summation $+$.
\end{definition}

\subsection{Name equivalence}

We take name equivalence, written $\nameeq$, to be the smallest
equivalence relation generated by the following rules.

\begin{mathpar}
\inferrule*[lab=Quote-drop]
{ }
{ \quotep{@{x}} \nameeq x }

\inferrule*[lab=Struct-equiv]
{ P \scong Q }
{ \quotep{P} \nameeq \quotep{Q} }
\end{mathpar}

The astute reader will have noticed that the mutual recursion of names
and processes imposes a mutual recursion on alpha-equivalence and
structural equivalence via name-equivalence. Fortunately, all of this
works out pleasantly and we may calculate in the natural way, free of
concern. The reader interested in the details is referred to the
appendix \ref{appendix:rho_details}.

\subsection{Substitution}

We use $\Proc$ for the set of processes, $\QProc$ for the set of
names, and $\id{\{}\vec{y} / \vec{x} \id{\}}$ to denote partial maps,
$s : \QProc \rightarrow \QProc$. A map, $s$ lifts, uniquely, to a map
on process terms, $\widehat{s} : \Proc \rightarrow \Proc$ by the
following equations.

\begin{mathpar}
  (0) \psubstp{Q}{P} := 0 \\
  (R \juxtap S) \psubstp{Q}{P}
  :=    
  (R)\psubstp{Q}{P} \juxtap (S) \psubstp{Q}{P} \\
  (x?(y).R) \psubstp{Q}{P}    
  :=    
  (x)\substp{Q}{P} (z)\concat( (R \psubstn{z}{y}) \psubstp{Q}{P} ) \\
  (\lift{x}{R}) \psubstp{Q}{P}  
  :=
  \lift{(x)\substp{Q}{P}}{ R \psubstp{Q}{P} } \\
%   (\dropn{x})  \psubstp{Q}{P}       
%   := 
%   \left\{ 
%     \begin{array}{ccc} 
%       \dropn{\quotep{Q}} & & x \nameeq \quotep{P} \\
%       \dropn{x} & & otherwise \\
%     \end{array}
%   \right. 
  (\dropn{x})  \psubstp{Q}{P}       
  := 
  \left\{ 
    \begin{array}{ccc} 
      Q & & x \nameeq \quotep{P} \\
      \dropn{x} & & otherwise \\
    \end{array}
  \right.
\end{mathpar}
 

where

\begin{eqnarray}
  (x)\id{\{} \lpquote Q \rpquote / \lpquote P \rpquote \id{\}}            = 
  \left\{ 
    \begin{array}{ccc}
      \lpquote Q \rpquote & & x \nameeq \lpquote P \rpquote \\
      x & & otherwise \\
    \end{array}
  \right. \nonumber
\end{eqnarray}

and $z$ is chosen distinct from $\quotep{P}$, $\quotep{Q}$, the free
names in $Q$, and all the names in $R$. Our $\alpha$-equivalence will
be built in the standard way from this substitution.

\begin{remark}\label{rem:no_self_referential_names}
  One consequence of these definitions is that $\forall P. \quotep{P}
  \not\in \freenames{P}$.
\end{remark}

\subsection{ Dynamic quote: an example }

Anticipating something of what's to come, consider applying the
substitution, $\widehat{\id{\{}u / z \id{\}}}$, to the following pair
of processes, $\lift{w}{y!(z)}$ and $w[ \lpquote y!(z) \rpquote ]$.

\begin{eqnarray}
	\lift{w}{y!(z)}\widehat{\id{\{}u / z \id{\}}}
		& = &
		\lift{w}{y!(u)} \nonumber\\
	w[ \lpquote y!(z) \rpquote ] \widehat{ \id{\{}u / z \id{\}} }
		& = &
		w[ \lpquote y!(z) \rpquote ] \nonumber
\end{eqnarray}

Because the body of the process between quotes is impervious to
substitution, we get radically different answers. In fact, by
examining the first process in an input context,
e.g. $x?(z).\lift{w}{y!(z)}$, we see that the process under the lift
operator may be shaped by prefixed inputs binding a name inside it. In
this sense, the lift operator will be seen as a way to dynamically
construct processes before reifying them as names.

Finally equipped with these standard features we can present the
dynamics of the calculus.

\subsubsection{Operational semantics} 

Finally, we introduce the computational dynamics. What marks these
algebras as distinct from other more traditionally studied algebraic
structures, e.g. vector spaces or polynomial rings, is the manner in
which dynamics is captured. In traditional structures, dynamics is typically
expressed through morphisms between such structures, as in linear maps
between vector spaces or morphisms between rings. In algebras
associated with the semantics of computation, the dynamics is
expressed as part of the algebraic structure itself, through a
reduction reduction relation typically denoted by $\red$. Below, we
give a recursive presentation of this relation for the calculus used
in the encoding.

$\red \subseteq \pi \times \pi$
$\red : \pi \to \mathcal{P}(\pi)$

\begin{mathpar}
  \inferrule* [lab=Comm] { \textsf{match}( x_{src}, x_{trgt} ) } { x_{trgt}?(y)P \; | \; x_{src}!\langle {Q} \rangle \red P\{\quotep{Q}/y}\} }
  \and \\
  \inferrule* [lab=Par] {{P} \red {P}'} {{{P} | {Q}} \red {{P}' | {Q}}}
  \and
  \inferrule* [lab=Equiv]{{{P} \scong {P}'} \andalso {{P}' \red {Q}'} \andalso {{Q}' \scong {Q}}}{{P} \red {Q}}
\end{mathpar}

\begin{eqnarray*}
  match_{\equiv} (\quotep{P},\quotep{Q}) & := & P \equiv Q \\
  match_{\dagger}(\quotep{P},\quotep{Q}) & := & \forall R. P|Q \red^{*} R => R \red^{*} 0 \\
  match_{K}(\quotep{P},\quotep{Q}) & := & K \mbox{ for some context } K
\end{eqnarray*}

$u?(x)P | u!\langle Q \rangle \red P\{\quotep{Q}/x\}$

%We write $\wred$ for $\red^*$, and $P\red$ if $\exists Q $ such that $ P \red Q$.
We write $P\red$ if $\exists Q $ such that $ P \red Q$ and $P\not\red$, otherwise.

\section{Replication}

As mentioned before, it is known that replication (and hence
recursion) can be implemented in a higher-order process algebra
\cite{SangiorgiWalker}. As our first example of calculation with the
machinery thus far presented we give the construction explicitly in
the {\rhoc}.

\begin{eqnarray}
	D_{x} & := & \prefix{x}{y}{(\binpar{\outputp{x}{y}}{@{y}})} \nonumber\\
	\bangp_{x}{P} & := & \binpar{{x}!\langle{\binpar{D_{x}}{P}}\rangle}{D_{x}} \nonumber
\end{eqnarray}

\begin{eqnarray}
	\bangp_{x}{P} & & \nonumber\\
	=
	& {x}!\langle{(\prefix{x}{y}{(\outputp{x}{y} | @{y})) | P}}\rangle 
	      | \prefix{x}{y}{(\outputp{x}{y} | @{y})} & \nonumber\\
	\red
	& (\outputp{x}{y} | @{y})\substn{\quotep{(\prefix{x}{y}{(@{y} | \outputp{x}{y})) | P}}}{y} & \nonumber\\
	=
	& \outputp{x}{\quotep{(\prefix{x}{y}{(\outputp{x}{y} | @{y})) | P}}}
	  | {(\prefix{x}{y}{(\outputp{x}{y} | @{y})) | P}} & \nonumber\\
	\red
	& \ldots & \nonumber\\
	\red^*
	& P | P | \ldots & \nonumber
\end{eqnarray}

Of course, this encoding, as an implementation, runs away, unfolding
$\bangp{P}$ eagerly. A lazier and more implementable replication
operator, restricted to input-guarded processes, may be obtained as follows.

\begin{eqnarray}
\bangp{\prefix{u}{v}{P}} 
	:= 
	\binpar{\lift{x}{\prefix{u}{v}{(\binpar{D(x)}{P})}}}{D(x)} \nonumber
\end{eqnarray}

\begin{remark}
  Note that the lazier definition still does not deal with summation
  or mixed summation (i.e. sums over input and output). The reader is
  invited to construct definitions of replication that deal with these
  features. 

  Further, the definitions are parameterized in a name, $x$. Can you,
  gentle reader, make a definition that eliminates this parameter and
  guarantees no accidental interaction between the replication
  machinery and the process being replicated -- i.e. no accidental
  sharing of names used by the process to get its work done and the
  name(s) used by the replication to effect copying. This latter
  revision of the definition of replication is crucial to obtaining
  the expected identity $!!P \sim !P$.
\end{remark}

\begin{remark}\label{rem:paradoxical_combinator}
  The reader familiar with the lambda calculus will have noticed the
  similarity between $D$ and the paradoxical combinator.

  [Ed. note: the existence of this seems to suggest we have to be more
  restrictive on the set of processes and names we admit if we are to
  support no-cloning.]
\end{remark}

\subsubsection{Bisimulation}

The computational dynamics gives rise to another kind of equivalence,
the equivalence of computational behavior. As previously mentioned
this is typically captured \emph{via} some form of bisimulation.

% The notion we use in this paper is weak barbed bisimulation
% \cite{milner91polyadicpi}.

The notion we use in this paper is derived from weak barbed
bisimulation \cite{milner91polyadicpi}. 

\begin{definition}
An \emph{observation relation}, $\downarrow_{\mathcal N}$, over a set
of names, $\mathcal N$, is the smallest relation satisfying the rules
below.

\infrule[Out-barb]{y \in {\mathcal N}, \; x \nameeq y}
		  {\outputp{x}{v} \downarrow_{\mathcal N} x}
\infrule[Par-barb]{\mbox{$P\downarrow_{\mathcal N} x$ or $Q\downarrow_{\mathcal N} x$}}
		  {\binpar{P}{Q} \downarrow_{\mathcal N} x}

We write $P \Downarrow_{\mathcal N} x$ if there is $Q$ such that 
$P \wred Q$ and $Q \downarrow_{\mathcal N} x$.
\end{definition}

\begin{definition}
%\label{def.bbisim}
An  ${\mathcal N}$-\emph{barbed bisimulation} over a set of names, ${\mathcal N}$, is a symmetric binary relation 
${\mathcal S}_{\mathcal N}$ between agents such that $P\rel{S}_{\mathcal N}Q$ implies:
\begin{enumerate}
\item If $P \red P'$ then $Q \wred Q'$ and $P'\rel{S}_{\mathcal N} Q'$.
\item If $P\downarrow_{\mathcal N} x$, then $Q\Downarrow_{\mathcal N} x$.
\end{enumerate}
$P$ is ${\mathcal N}$-barbed bisimilar to $Q$, written
$P \wbbisim_{\mathcal N} Q$, if $P \rel{S}_{\mathcal N} Q$ for some ${\mathcal N}$-barbed bisimulation ${\mathcal S}_{\mathcal N}$.
\end{definition}

$\mathcal{R} \subseteq \pi \times \pi$

$P \mathcal{R} Q => \forall P'. P \red P' \Rightarrow \exists Q'. Q \red Q', P' \mathcal{R} Q'$

$P \vdash x \Rightarrow Q \vdash x$

\begin{mathpar}
  \inferrule*[lab=Out-barb]{x \nameeq y}{{y}!\langle{Q}\rangle \vdash x}
  \and
  \inferrule*[lab=Par-barb]{\mbox{$P\vdash x$ or $Q\vdash x$}}{\binpar{P}{Q} \vdash x}
\end{mathpar}

\subsubsection{Contexts}

One of the principle advantages of computational calculi like the
$\pi$-calculus is a well-defined notion of context,
contextual-equivalence and a correlation between
contextual-equivalence and notions of bisimulation. The notion of
context allows the decomposition of a process into (sub-)process and
its syntactic environment, its context. Thus, a context may be
thought of as a process with a ``hole'' (written $\Box$) in it. The
application of a context $M$ to a process $P$, written $M[P]$, is
tantamount to filling the hole in $M$ with $P$. In this paper we do
not need the full weight of this theory, but do make use of the notion
of context in the proof the main theorem. 

\begin{mathpar}
  \inferrule* [lab=summation] {} {{M_{M},M_{N}} \bc \Box \;|\; x.M_{A} \;|\; M_{M}+M_{N}}
  \and
  \inferrule* [lab=agent] {} {{M_{A}} \bc (\vec{x})M_{P} \;| \; \clift{P_0,\ldots,M_{P},\ldots,P_N}}
  \and \\
  \inferrule* [lab=process] {} {{M_{P}} \bc M_{N} \;| \;P|M_{P} }
\end{mathpar} 

\begin{mathpar}
  \inferrule* [lab=sychronization] {} {M_{N} \bc \Box \;|\; x?M_{F} \;|\; x!M_{C}}
  \and
  \inferrule* [lab=abstraction] {} {{M_{F}} \bc (x)M_{P} }
  \and
  \inferrule* [lab=concretion] {} {{M_{C}} \bc \langle M_{P} \rangle }
  \and \\
  \inferrule* [lab=process] {} {{M_{P}} \bc M_{N} \;| \;P|M_{P} }
\end{mathpar}

\begin{definition}[contextual application] Given a context $M$, and
  process $P$, we define the \emph{contextual application}, $M[P] :=
  M\{P/\Box\}$. That is, the contextual application of M to P is the
  substitution of $P$ for $\Box$ in $M$.
\end{definition}

$\meaningof{-} : L \to \mathcal{P}(\pi)$

\begin{mathpar}
  \inferrule* [lab=collection] {} {\meaningof{true} = \pi, \and \meaningof{~E} = \pi \setminus \meaningof{E}, \and \meaningof{E_{1} \& E_{2}} = \meaningof{E_{1}} \cap \meaningof{E_{2}}}
\end{mathpar}

\begin{mathpar}
  \inferrule* [lab=structure] {} {\meaningof{0} = \{ P \in \pi | P \equiv 0 \}, \and \\ \meaningof{E_1 | E_2} = \{ P \in \pi | P \equiv P_{1} | P_{2}, P_{1} \in \meaningof{E_{1}}, P_{2} \in \meaningof{E_2}\} }
\end{mathpar}

\begin{mathpar}
 \inferrule* [lab=behavior] {} {\meaningof{\langle a?b \rangle E} = \{ P \in \pi | P \equiv Q | u?(y)P', \\ \and \\\\ \and \\ \;\;\; u \in \meaningof{a}, \forall z.P'\{z/y\} \in \meaningof{E\{z/b\}}\}, \and \\ \meaningof{a!E} = \{ P \in \pi | P \equiv Q | x!\langle P' \rangle, x \in \meaningof{a} P' \in \meaningof{E}\} }
\end{mathpar}

\begin{mathpar}
 \inferrule* [lab=nominal] {} {\meaningof{\quotep{E}} = \{ \quotep{P} \in \quotep{\pi} | P \in \meaningof{E} \}, \and \meaningof{\quotep{P}} = \{ \quotep{Q} \in \quotep{\pi} | P \equiv Q \} \and \\ \meaningof{@\quotep{E}} = \{ P \in \pi | P \equiv @x, x \in \meaningof{E} \}}
\end{mathpar}

\begin{eqnarray*}
  \\
  \meaningof{-} : TS \to ST
\end{eqnarray*}

\begin{eqnarray*}
  \\
  L : TS \to ST
\end{eqnarray*}

\begin{eqnarray*}
  \\
  P \models E \iff P \in \meaningof{E}
\end{eqnarray*}

\begin{eqnarray*}
  P \approx_{L} Q \iff \forall E \in L. P \models E \iff Q \models E
\end{eqnarray*}

\begin{eqnarray*}
  P \approx_{K} Q
\end{eqnarray*}

\begin{eqnarray*}
  P \approx Q
\end{eqnarray*}

$\approx_{K} = \approx = \approx_{L}$

\subsubsection{Contextual duality}

Note that contexts extend the quotation operation to a family of
operations from processes to names. Given a context, $M$, we can
define a \emph{nominal context}, $\quotep{M}$ by $\quotep{M}[P] :=
\quotep{M[P]}$. To foreshadow what is to come we observe that these
operations enjoy a duality with processes very much like the duality
between vectors and maps from vectors to scalars.

Further, because the calculus is essentially higher-order, we have a
correspondence between contexts and processes. More specifically,
given a name $x$ and a context $M$ we can construct $M^{*}_{x}$ such
that 

\begin{mathpar}
  M^{*}_{x} | \lift{x}{P} \red M[P]
\end{mathpar}

namely,

\begin{mathpar}
  M^{*}_{x} := x?(u).M[\dropn{u}]
\end{mathpar}

The dependence of $M^{*}_{x}$ on a name makes it an abstraction, 

\begin{mathpar}
  M^{*} := (x)x?(u).M[\dropn{u}]
\end{mathpar}

\subsection{Additional notation}

It will sometimes be convenient to denote the process a name
quotes. We already have the notation $x = \quotep{P}$, but it will be
convenient to introduce an alternate notation, $\procn{x}$, when we
want to emphasize the connection to the use of the name. Note that, by
virtue of name equivalence, $\quotep{\procn{x}} \nameeq x$; so, the
notation is consistent with previous definitions.

Further, because names have structure it is possible to effect
substitutions on the basis of that structure. This means we need to
upgrade our notation for substitutions, which we accomplish by
adapting comprehension notation. Thus,

\begin{mathpar}
  P\{ y / x : x \in S \}
\end{mathpar}

is interpreted to mean the process derived from P by replacing (in a
capture-avoiding manner) each occurrence of $x$ in $S$ by $y$. For example,

\begin{mathpar}
  P\{ \quotep{\procn{x}|\procn{x}} / x : x \in \freenames{P} \}
\end{mathpar}

will replace each (occurrence) of a free name $x$ in $P$ by
$\quotep{\procn{x}|\procn{x}}$.

Also, we will avail ourselves of the notation $x^{L}$ and $x^{R}$ to
denote injections of a name into disjoint copies of the name
space. There are numerous ways to accomplish this. One example can be
found in \cite{MeredithR05}. This notation overloads to vectors of
names: $\vec{x}^{\pi} := (x_{i}^{\pi} \; : \; 0 \leq i < |\vec{x}| )$ where $\pi \in \{L,R\}$.

We also use $P^{\Box} := P|\Box$.

In \cite{MeredithR05} an interpretation of the new operator is
given. It turns out that there are several possible interpretations
all enjoying the requisite algebraic properties of the operator (see
\cite{milner91polyadicpi}). We will therefore make liberal use of
$(\nu\; \vec{x})P$.

% subsection the_syntax_and_semantics_of_the_notation_system (end)   

\input{qm2pi.qmops} 

\input{qm2pi.sterngerlach} 

\input{qm2pi.metric} 

% section concurrent_process_calculi (end)

%\input{qm2pi.proofsketch}

% section proof sketch (end)

%\input{qm2pi.slviaknots} 

% section spatial logic via knots (end)

\input{qm2pi.conclusion}

% section conclusion (end)

%\input{qm2pi.dtcodes} 

% section wiring algorithm (end)

\input{qm2pi.ack} 

% section acknowledgments (end)

\newpage


\bibliographystyle{plain}   
\bibliography{../../biblios/main.bib}

\input{qm2pi.rhodetails}

\end{document}

 

% section concurrent_process_calculi (end)

%\documentclass[12pt]{llncs}
%\documentclass{jktr}

\usepackage[pdftex]{hyperref}                   
\usepackage {listings}
\usepackage {mathpartir}
\usepackage{bcprules}
%\usepackage{listings}
                       
\usepackage{graphicx} 
%\usepackage[margins=2.5cm,nohead,nofoot]{geometry}
%\usepackage{geometry}
\usepackage{amsfonts}
\usepackage{amstext}
\usepackage{latexsym}
\usepackage{amssymb}
\usepackage{color}


%\include{myPreamble}
\include{qm2pi.local} 

%\ifpdf
%\usepackage[pdftex]{graphicx}
%\else
%\usepackage{graphicx}
%\fi

 % \ifpdf
%  \usepackage{pdfsync}
%  \if


%\title{Brief Article}
%\author{David F. Snyder}
%\author{L.G. Meredith}

%\address{Dept. of Math., Texas State University--San Marcos, San Marcos, TX 78666}
       
\pagestyle{empty}


\begin{document}

\lstset{language=[Objective]Caml,frame=shadowbox}

\input{qm2pi.front}

% section front matter (end)

\input{qm2pi.intro} 
 
% section introduction (end)

% \input{qm2pi.knotations} 

% section notation (end)

\input{qm2pi.process.calculi} 

% section concurrent_process_calculi_and_spatial_logics_ (end)
    
%\input{qm2pi.knots2pi} 

%\input{qm2pi.trefoil} 

%\input{qm2pi.mainthm} 

% subsection basic_interpretation (end)

%\input{qm2pi.rho.presentation} 
\subsection{The syntax and semantics of the notation system}\label{sub:the_syntax_and_semantics_of_the_notation_system} % (fold)

We now summarize a technical presentation of the calculus that
embodies our theory of dynamics. The typical presentation of such a
calculus follows the style of giving generators and relations on
them. The grammar, below, describing term constructors, freely
generates the set of processes, $\Proc$. This set is then quotiented
by a relation known as structural congruence and it is over this set
that the notion of dynamics is expressed. This presentation is
essentially that of \cite{MeredithR05} with the addition of
polyadicity and summation. For readability we have relegated some of
the technical subtleties to an appendix.

\subsubsection{Process grammar}\label{subsub:process_grammar}

\begin{mathpar}
  \inferrule* [lab=synchronization] {} {{M} \bc \pzero \;|\; x?F \;|\; x!C }
  \and
  \inferrule* [lab=abstraction] {} {{F} \bc (x)P}
  \and
  \inferrule* [lab=concretion] {} {{C} \bc \langle Q \rangle}
  \and
  \inferrule* [lab=process] {} {{P,Q} \bc M \;| \;P|Q \;|\; @{x}}
  \and
  \inferrule* [lab=name] {} {{x} \bc \quotep{P}}
\end{mathpar} 

Note that $\vec{x}$ (resp. $\vec{P}$) denotes a vector of names
(resp. processes) of length $|\vec{x}|$ (resp. $|\vec{P}|$). We adopt
the following useful abbreviations.

\begin{mathpar}
   x?(\vec{y}).P := x.(\vec{y})P \and  x\clift{\vec{P}} := x.\clift{\vec{P}}
   \and x!(y) := \lift{x}{\dropn{y}}
   \and \Pi_{i=0}^{n-1}P_i := P_0 | \ldots | P_{n-1}
\end{mathpar}

\subsubsection{Structural congruence}

\paragraph{Free and bound names and alpha-equivalence.} At the
core of structural equivalence is alpha-equivalence which identifies
process that are the same up to a change of variable. Formally, we
recognize the distinction between free and bound names. The free names
of a process, $\freenames{P}$, may be calculated recursively as
follows:

\begin{mathpar}
\freenames{\pzero} := \emptyset
  \and \\
  \freenames{x?(y).P} := \{ x \} \cup (\freenames{P} \setminus \{ y \})
  \and 
  \freenames{x!\langle P \rangle} := \{ x \} \cup \{ P \} 
  \and \\
  \freenames{P|Q} := \freenames{P} \cup \freenames{Q}
  \and \\
  \freenames{@{x}} := \{ x \}
\end{mathpar}

$\pi$
$\quotep{\pi}$

$\freenames{-} : \pi \to \mathcal{P}(\quotep{\pi})$

\begin{eqnarray*}
  \freenames{\pzero} & := & \emptyset \\
  \freenames{x?(y).P} & := & \{ x \} \cup (\freenames{P} \setminus \{ y \}) \\
  \freenames{x!\langle P \rangle} & := & \{ x \} \cup \{ P \} \\
  \freenames{P|Q} & := & \freenames{P} \cup \freenames{Q} \\
  \freenames{\dropn{x}} & := & \{ x \}
\end{eqnarray*}

The bound names of a process, $\boundnames{P}$, are those names occurring in $P$
that are not free. For example, in $x?(y).0$, the name $x$ is free, while $y$ is bound.

\begin{mathpar}
  \inferrule* [lab=monoidal-laws] {} { P|Q \equiv Q|P \and P|0 \equiv P \and P|(Q|R) \equiv (P|Q)|R }
\end{mathpar}

\begin{mathpar}
  \inferrule* [lab=alpha-equivalence] {} { (x)P \equiv (y)P\{y/x\} \and y \not\in \freenames{P} }
\end{mathpar}

\begin{definition}
Then two processes, $P,Q$, are alpha-equivalent if $P = Q\{\vec{y}/\vec{x}\}$ for
some $\vec{x} \in \boundnames{Q},\vec{y} \in \boundnames{P}$, where $Q\{\vec{y}/\vec{x}\}$
denotes the capture-avoiding substitution of $\vec{y}$ for $\vec{x}$ in $Q$.
\end{definition}

\begin{definition}
  The {\em structural congruence} \cite{SangiorgiWalker} , $\equiv$,
  between processes is the least congruence containing
  alpha-equivalence, satisfying the abelian monoid laws
  (associativity, commutativity and $\pzero$ as identity) for parallel
  composition $|$ and for summation $+$.
\end{definition}

\subsection{Name equivalence}

We take name equivalence, written $\nameeq$, to be the smallest
equivalence relation generated by the following rules.

\begin{mathpar}
\inferrule*[lab=Quote-drop]
{ }
{ \quotep{@{x}} \nameeq x }

\inferrule*[lab=Struct-equiv]
{ P \scong Q }
{ \quotep{P} \nameeq \quotep{Q} }
\end{mathpar}

The astute reader will have noticed that the mutual recursion of names
and processes imposes a mutual recursion on alpha-equivalence and
structural equivalence via name-equivalence. Fortunately, all of this
works out pleasantly and we may calculate in the natural way, free of
concern. The reader interested in the details is referred to the
appendix \ref{appendix:rho_details}.

\subsection{Substitution}

We use $\Proc$ for the set of processes, $\QProc$ for the set of
names, and $\id{\{}\vec{y} / \vec{x} \id{\}}$ to denote partial maps,
$s : \QProc \rightarrow \QProc$. A map, $s$ lifts, uniquely, to a map
on process terms, $\widehat{s} : \Proc \rightarrow \Proc$ by the
following equations.

\begin{mathpar}
  (0) \psubstp{Q}{P} := 0 \\
  (R \juxtap S) \psubstp{Q}{P}
  :=    
  (R)\psubstp{Q}{P} \juxtap (S) \psubstp{Q}{P} \\
  (x?(y).R) \psubstp{Q}{P}    
  :=    
  (x)\substp{Q}{P} (z)\concat( (R \psubstn{z}{y}) \psubstp{Q}{P} ) \\
  (\lift{x}{R}) \psubstp{Q}{P}  
  :=
  \lift{(x)\substp{Q}{P}}{ R \psubstp{Q}{P} } \\
%   (\dropn{x})  \psubstp{Q}{P}       
%   := 
%   \left\{ 
%     \begin{array}{ccc} 
%       \dropn{\quotep{Q}} & & x \nameeq \quotep{P} \\
%       \dropn{x} & & otherwise \\
%     \end{array}
%   \right. 
  (\dropn{x})  \psubstp{Q}{P}       
  := 
  \left\{ 
    \begin{array}{ccc} 
      Q & & x \nameeq \quotep{P} \\
      \dropn{x} & & otherwise \\
    \end{array}
  \right.
\end{mathpar}
 

where

\begin{eqnarray}
  (x)\id{\{} \lpquote Q \rpquote / \lpquote P \rpquote \id{\}}            = 
  \left\{ 
    \begin{array}{ccc}
      \lpquote Q \rpquote & & x \nameeq \lpquote P \rpquote \\
      x & & otherwise \\
    \end{array}
  \right. \nonumber
\end{eqnarray}

and $z$ is chosen distinct from $\quotep{P}$, $\quotep{Q}$, the free
names in $Q$, and all the names in $R$. Our $\alpha$-equivalence will
be built in the standard way from this substitution.

\begin{remark}\label{rem:no_self_referential_names}
  One consequence of these definitions is that $\forall P. \quotep{P}
  \not\in \freenames{P}$.
\end{remark}

\subsection{ Dynamic quote: an example }

Anticipating something of what's to come, consider applying the
substitution, $\widehat{\id{\{}u / z \id{\}}}$, to the following pair
of processes, $\lift{w}{y!(z)}$ and $w[ \lpquote y!(z) \rpquote ]$.

\begin{eqnarray}
	\lift{w}{y!(z)}\widehat{\id{\{}u / z \id{\}}}
		& = &
		\lift{w}{y!(u)} \nonumber\\
	w[ \lpquote y!(z) \rpquote ] \widehat{ \id{\{}u / z \id{\}} }
		& = &
		w[ \lpquote y!(z) \rpquote ] \nonumber
\end{eqnarray}

Because the body of the process between quotes is impervious to
substitution, we get radically different answers. In fact, by
examining the first process in an input context,
e.g. $x?(z).\lift{w}{y!(z)}$, we see that the process under the lift
operator may be shaped by prefixed inputs binding a name inside it. In
this sense, the lift operator will be seen as a way to dynamically
construct processes before reifying them as names.

Finally equipped with these standard features we can present the
dynamics of the calculus.

\subsubsection{Operational semantics} 

Finally, we introduce the computational dynamics. What marks these
algebras as distinct from other more traditionally studied algebraic
structures, e.g. vector spaces or polynomial rings, is the manner in
which dynamics is captured. In traditional structures, dynamics is typically
expressed through morphisms between such structures, as in linear maps
between vector spaces or morphisms between rings. In algebras
associated with the semantics of computation, the dynamics is
expressed as part of the algebraic structure itself, through a
reduction reduction relation typically denoted by $\red$. Below, we
give a recursive presentation of this relation for the calculus used
in the encoding.

$\red \subseteq \pi \times \pi$
$\red : \pi \to \mathcal{P}(\pi)$

\begin{mathpar}
  \inferrule* [lab=Comm] { \textsf{match}( x_{src}, x_{trgt} ) } { x_{trgt}?(y)P \; | \; x_{src}!\langle {Q} \rangle \red P\{\quotep{Q}/y}\} }
  \and \\
  \inferrule* [lab=Par] {{P} \red {P}'} {{{P} | {Q}} \red {{P}' | {Q}}}
  \and
  \inferrule* [lab=Equiv]{{{P} \scong {P}'} \andalso {{P}' \red {Q}'} \andalso {{Q}' \scong {Q}}}{{P} \red {Q}}
\end{mathpar}

\begin{eqnarray*}
  match_{\equiv} (\quotep{P},\quotep{Q}) & := & P \equiv Q \\
  match_{\dagger}(\quotep{P},\quotep{Q}) & := & \forall R. P|Q \red^{*} R => R \red^{*} 0 \\
  match_{K}(\quotep{P},\quotep{Q}) & := & K \mbox{ for some context } K
\end{eqnarray*}

$u?(x)P | u!\langle Q \rangle \red P\{\quotep{Q}/x\}$

%We write $\wred$ for $\red^*$, and $P\red$ if $\exists Q $ such that $ P \red Q$.
We write $P\red$ if $\exists Q $ such that $ P \red Q$ and $P\not\red$, otherwise.

\section{Replication}

As mentioned before, it is known that replication (and hence
recursion) can be implemented in a higher-order process algebra
\cite{SangiorgiWalker}. As our first example of calculation with the
machinery thus far presented we give the construction explicitly in
the {\rhoc}.

\begin{eqnarray}
	D_{x} & := & \prefix{x}{y}{(\binpar{\outputp{x}{y}}{@{y}})} \nonumber\\
	\bangp_{x}{P} & := & \binpar{{x}!\langle{\binpar{D_{x}}{P}}\rangle}{D_{x}} \nonumber
\end{eqnarray}

\begin{eqnarray}
	\bangp_{x}{P} & & \nonumber\\
	=
	& {x}!\langle{(\prefix{x}{y}{(\outputp{x}{y} | @{y})) | P}}\rangle 
	      | \prefix{x}{y}{(\outputp{x}{y} | @{y})} & \nonumber\\
	\red
	& (\outputp{x}{y} | @{y})\substn{\quotep{(\prefix{x}{y}{(@{y} | \outputp{x}{y})) | P}}}{y} & \nonumber\\
	=
	& \outputp{x}{\quotep{(\prefix{x}{y}{(\outputp{x}{y} | @{y})) | P}}}
	  | {(\prefix{x}{y}{(\outputp{x}{y} | @{y})) | P}} & \nonumber\\
	\red
	& \ldots & \nonumber\\
	\red^*
	& P | P | \ldots & \nonumber
\end{eqnarray}

Of course, this encoding, as an implementation, runs away, unfolding
$\bangp{P}$ eagerly. A lazier and more implementable replication
operator, restricted to input-guarded processes, may be obtained as follows.

\begin{eqnarray}
\bangp{\prefix{u}{v}{P}} 
	:= 
	\binpar{\lift{x}{\prefix{u}{v}{(\binpar{D(x)}{P})}}}{D(x)} \nonumber
\end{eqnarray}

\begin{remark}
  Note that the lazier definition still does not deal with summation
  or mixed summation (i.e. sums over input and output). The reader is
  invited to construct definitions of replication that deal with these
  features. 

  Further, the definitions are parameterized in a name, $x$. Can you,
  gentle reader, make a definition that eliminates this parameter and
  guarantees no accidental interaction between the replication
  machinery and the process being replicated -- i.e. no accidental
  sharing of names used by the process to get its work done and the
  name(s) used by the replication to effect copying. This latter
  revision of the definition of replication is crucial to obtaining
  the expected identity $!!P \sim !P$.
\end{remark}

\begin{remark}\label{rem:paradoxical_combinator}
  The reader familiar with the lambda calculus will have noticed the
  similarity between $D$ and the paradoxical combinator.

  [Ed. note: the existence of this seems to suggest we have to be more
  restrictive on the set of processes and names we admit if we are to
  support no-cloning.]
\end{remark}

\subsubsection{Bisimulation}

The computational dynamics gives rise to another kind of equivalence,
the equivalence of computational behavior. As previously mentioned
this is typically captured \emph{via} some form of bisimulation.

% The notion we use in this paper is weak barbed bisimulation
% \cite{milner91polyadicpi}.

The notion we use in this paper is derived from weak barbed
bisimulation \cite{milner91polyadicpi}. 

\begin{definition}
An \emph{observation relation}, $\downarrow_{\mathcal N}$, over a set
of names, $\mathcal N$, is the smallest relation satisfying the rules
below.

\infrule[Out-barb]{y \in {\mathcal N}, \; x \nameeq y}
		  {\outputp{x}{v} \downarrow_{\mathcal N} x}
\infrule[Par-barb]{\mbox{$P\downarrow_{\mathcal N} x$ or $Q\downarrow_{\mathcal N} x$}}
		  {\binpar{P}{Q} \downarrow_{\mathcal N} x}

We write $P \Downarrow_{\mathcal N} x$ if there is $Q$ such that 
$P \wred Q$ and $Q \downarrow_{\mathcal N} x$.
\end{definition}

\begin{definition}
%\label{def.bbisim}
An  ${\mathcal N}$-\emph{barbed bisimulation} over a set of names, ${\mathcal N}$, is a symmetric binary relation 
${\mathcal S}_{\mathcal N}$ between agents such that $P\rel{S}_{\mathcal N}Q$ implies:
\begin{enumerate}
\item If $P \red P'$ then $Q \wred Q'$ and $P'\rel{S}_{\mathcal N} Q'$.
\item If $P\downarrow_{\mathcal N} x$, then $Q\Downarrow_{\mathcal N} x$.
\end{enumerate}
$P$ is ${\mathcal N}$-barbed bisimilar to $Q$, written
$P \wbbisim_{\mathcal N} Q$, if $P \rel{S}_{\mathcal N} Q$ for some ${\mathcal N}$-barbed bisimulation ${\mathcal S}_{\mathcal N}$.
\end{definition}

$\mathcal{R} \subseteq \pi \times \pi$

$P \mathcal{R} Q => \forall P'. P \red P' \Rightarrow \exists Q'. Q \red Q', P' \mathcal{R} Q'$

$P \vdash x \Rightarrow Q \vdash x$

\begin{mathpar}
  \inferrule*[lab=Out-barb]{x \nameeq y}{{y}!\langle{Q}\rangle \vdash x}
  \and
  \inferrule*[lab=Par-barb]{\mbox{$P\vdash x$ or $Q\vdash x$}}{\binpar{P}{Q} \vdash x}
\end{mathpar}

\subsubsection{Contexts}

One of the principle advantages of computational calculi like the
$\pi$-calculus is a well-defined notion of context,
contextual-equivalence and a correlation between
contextual-equivalence and notions of bisimulation. The notion of
context allows the decomposition of a process into (sub-)process and
its syntactic environment, its context. Thus, a context may be
thought of as a process with a ``hole'' (written $\Box$) in it. The
application of a context $M$ to a process $P$, written $M[P]$, is
tantamount to filling the hole in $M$ with $P$. In this paper we do
not need the full weight of this theory, but do make use of the notion
of context in the proof the main theorem. 

\begin{mathpar}
  \inferrule* [lab=summation] {} {{M_{M},M_{N}} \bc \Box \;|\; x.M_{A} \;|\; M_{M}+M_{N}}
  \and
  \inferrule* [lab=agent] {} {{M_{A}} \bc (\vec{x})M_{P} \;| \; \clift{P_0,\ldots,M_{P},\ldots,P_N}}
  \and \\
  \inferrule* [lab=process] {} {{M_{P}} \bc M_{N} \;| \;P|M_{P} }
\end{mathpar} 

\begin{mathpar}
  \inferrule* [lab=sychronization] {} {M_{N} \bc \Box \;|\; x?M_{F} \;|\; x!M_{C}}
  \and
  \inferrule* [lab=abstraction] {} {{M_{F}} \bc (x)M_{P} }
  \and
  \inferrule* [lab=concretion] {} {{M_{C}} \bc \langle M_{P} \rangle }
  \and \\
  \inferrule* [lab=process] {} {{M_{P}} \bc M_{N} \;| \;P|M_{P} }
\end{mathpar}

\begin{definition}[contextual application] Given a context $M$, and
  process $P$, we define the \emph{contextual application}, $M[P] :=
  M\{P/\Box\}$. That is, the contextual application of M to P is the
  substitution of $P$ for $\Box$ in $M$.
\end{definition}

$\meaningof{-} : L \to \mathcal{P}(\pi)$

\begin{mathpar}
  \inferrule* [lab=collection] {} {\meaningof{true} = \pi, \and \meaningof{~E} = \pi \setminus \meaningof{E}, \and \meaningof{E_{1} \& E_{2}} = \meaningof{E_{1}} \cap \meaningof{E_{2}}}
\end{mathpar}

\begin{mathpar}
  \inferrule* [lab=structure] {} {\meaningof{0} = \{ P \in \pi | P \equiv 0 \}, \and \\ \meaningof{E_1 | E_2} = \{ P \in \pi | P \equiv P_{1} | P_{2}, P_{1} \in \meaningof{E_{1}}, P_{2} \in \meaningof{E_2}\} }
\end{mathpar}

\begin{mathpar}
 \inferrule* [lab=behavior] {} {\meaningof{\langle a?b \rangle E} = \{ P \in \pi | P \equiv Q | u?(y)P', \\ \and \\\\ \and \\ \;\;\; u \in \meaningof{a}, \forall z.P'\{z/y\} \in \meaningof{E\{z/b\}}\}, \and \\ \meaningof{a!E} = \{ P \in \pi | P \equiv Q | x!\langle P' \rangle, x \in \meaningof{a} P' \in \meaningof{E}\} }
\end{mathpar}

\begin{mathpar}
 \inferrule* [lab=nominal] {} {\meaningof{\quotep{E}} = \{ \quotep{P} \in \quotep{\pi} | P \in \meaningof{E} \}, \and \meaningof{\quotep{P}} = \{ \quotep{Q} \in \quotep{\pi} | P \equiv Q \} \and \\ \meaningof{@\quotep{E}} = \{ P \in \pi | P \equiv @x, x \in \meaningof{E} \}}
\end{mathpar}

\begin{eqnarray*}
  \\
  \meaningof{-} : TS \to ST
\end{eqnarray*}

\begin{eqnarray*}
  \\
  L : TS \to ST
\end{eqnarray*}

\begin{eqnarray*}
  \\
  P \models E \iff P \in \meaningof{E}
\end{eqnarray*}

\begin{eqnarray*}
  P \approx_{L} Q \iff \forall E \in L. P \models E \iff Q \models E
\end{eqnarray*}

\begin{eqnarray*}
  P \approx_{K} Q
\end{eqnarray*}

\begin{eqnarray*}
  P \approx Q
\end{eqnarray*}

$\approx_{K} = \approx = \approx_{L}$

\subsubsection{Contextual duality}

Note that contexts extend the quotation operation to a family of
operations from processes to names. Given a context, $M$, we can
define a \emph{nominal context}, $\quotep{M}$ by $\quotep{M}[P] :=
\quotep{M[P]}$. To foreshadow what is to come we observe that these
operations enjoy a duality with processes very much like the duality
between vectors and maps from vectors to scalars.

Further, because the calculus is essentially higher-order, we have a
correspondence between contexts and processes. More specifically,
given a name $x$ and a context $M$ we can construct $M^{*}_{x}$ such
that 

\begin{mathpar}
  M^{*}_{x} | \lift{x}{P} \red M[P]
\end{mathpar}

namely,

\begin{mathpar}
  M^{*}_{x} := x?(u).M[\dropn{u}]
\end{mathpar}

The dependence of $M^{*}_{x}$ on a name makes it an abstraction, 

\begin{mathpar}
  M^{*} := (x)x?(u).M[\dropn{u}]
\end{mathpar}

\subsection{Additional notation}

It will sometimes be convenient to denote the process a name
quotes. We already have the notation $x = \quotep{P}$, but it will be
convenient to introduce an alternate notation, $\procn{x}$, when we
want to emphasize the connection to the use of the name. Note that, by
virtue of name equivalence, $\quotep{\procn{x}} \nameeq x$; so, the
notation is consistent with previous definitions.

Further, because names have structure it is possible to effect
substitutions on the basis of that structure. This means we need to
upgrade our notation for substitutions, which we accomplish by
adapting comprehension notation. Thus,

\begin{mathpar}
  P\{ y / x : x \in S \}
\end{mathpar}

is interpreted to mean the process derived from P by replacing (in a
capture-avoiding manner) each occurrence of $x$ in $S$ by $y$. For example,

\begin{mathpar}
  P\{ \quotep{\procn{x}|\procn{x}} / x : x \in \freenames{P} \}
\end{mathpar}

will replace each (occurrence) of a free name $x$ in $P$ by
$\quotep{\procn{x}|\procn{x}}$.

Also, we will avail ourselves of the notation $x^{L}$ and $x^{R}$ to
denote injections of a name into disjoint copies of the name
space. There are numerous ways to accomplish this. One example can be
found in \cite{MeredithR05}. This notation overloads to vectors of
names: $\vec{x}^{\pi} := (x_{i}^{\pi} \; : \; 0 \leq i < |\vec{x}| )$ where $\pi \in \{L,R\}$.

We also use $P^{\Box} := P|\Box$.

In \cite{MeredithR05} an interpretation of the new operator is
given. It turns out that there are several possible interpretations
all enjoying the requisite algebraic properties of the operator (see
\cite{milner91polyadicpi}). We will therefore make liberal use of
$(\nu\; \vec{x})P$.

% subsection the_syntax_and_semantics_of_the_notation_system (end)   

\input{qm2pi.qmops} 

\input{qm2pi.sterngerlach} 

\input{qm2pi.metric} 

% section concurrent_process_calculi (end)

%\input{qm2pi.proofsketch}

% section proof sketch (end)

%\input{qm2pi.slviaknots} 

% section spatial logic via knots (end)

\input{qm2pi.conclusion}

% section conclusion (end)

%\input{qm2pi.dtcodes} 

% section wiring algorithm (end)

\input{qm2pi.ack} 

% section acknowledgments (end)

\newpage


\bibliographystyle{plain}   
\bibliography{../../biblios/main.bib}

\input{qm2pi.rhodetails}

\end{document}



% section proof sketch (end)

%\section{Unlikely characters: spatial logic for
  knots}\label{sub:characteristic_formulae} % (fold)

Associated to the mobile process calculi are a family of logics known
as the Hennessy-Milner logics. These logics typically enjoy a
semantics interpreting formulae as sets of processes that when
factored through the encoding outlined above allows an identification
of classes of knots with logical formulae. In the context of this
encoding the sub-family known as the spatial logics \cite{CairesC03}
\cite{CairesC04} \cite{Caires04} are of particular interest providing
several important features for expressing and reasoning about
properties (i.e. classes) of knots. We hint here at how this may be done.

%\begin{description}
%\item [structural connectives] 
\subsubsection{Structural connectives} The spatial logics enjoy
structural connectives corresponding, at the logical level, to the
parallel composition ($P | Q$) and new name ($(\nu \; x)P$)
connectives for processes. As illustrated in the examples below, these
connectives are extremely expressive given the shape of our encoding.
%\item [decideable satisfaction]

\subsubsection{Decideable satisfaction}
In \cite{Caires04} the satisfaction relation is shown to be decideable
for a rich class of processes. It further turns out that the image of
the our encoding is a proper subset of that class. This result
provides the basis for an algorithm by which to search for knots
enjoying a given property.
%\item [characteristic formulae]

\subsubsection{Characteristic formulae}
In the same paper \cite{Caires04} , Caires presents a means of calculating
characteristic formulae, selecting equivalence classes of processes
up to a pre--specified depth limit on the support set of names. Composed with our
encoding, this characteristic formula can be used to select
characteristic formulae for knots.
%\end{description}

\subsubsection{Spatial logic formulae}

The grammar below (segmented for comprehension) summarizes the syntax
of spatial logic formulae. We employ illustrative examples in the
sequel to provide an intuitive understanding of their meaning
referring the reader to \cite{Caires04} for a more detailed explication
of the semantics.

\begin{mathpar}
  \inferrule* [lab=boolean] {} {{A,B} \bc T \;|\; \neg A \;|\; A \wedge B \;|\; \eta = \eta'}
  \and
  \inferrule* [lab=spatial] {} {|\; \pzero \;|\; A | B \;|\; x \text{\textregistered} A \;|\; \forall x . A \;|\;  H x . A}
  \and
  \inferrule* [lab=behavioral] {} {|\; \alpha . A}
  \and 
  \inferrule* [lab=recursion] {} {|\; X(\vec{u}) \;|\; \mu X(\vec{u}) . A}
  \and
  \inferrule* [lab=action] {} {\alpha \bc \langle x?(\vec{y}) \rangle \;|\; \langle x!(\vec{y}) \rangle \;|\; \langle \tau \rangle}
  \and 
  \inferrule* [lab=name] {} {\eta \bc x \;|\; \tau}
\end{mathpar} 

% subsection characteristic_formulae (end)   	 

\subsection{Example formulae}\label{sub:example_formulae_} % (fold)

\subsubsection{Crossing as formula.}
% 
% \begin{align*}
%   \frac{d}{dx} \sin x &= \cos x 
%   & \frac{d}{dx} e^x &= e^x \\
%   \frac{d}{dx} \cos x &= - \sin x 
%   & \frac{d}{dx} \log x &= \frac{1}{x} \\
% \end{align*} 

\begin{align*}
 \mu C(x_{0},x_{1},y_{0},y_{1},u).&(\langle x_{0}?(z) \rangle(\langle u! \rangle\langle y_{1}!z \rangle C(x_{0},x_{1},y_{0},y_{1},u)) & \\
  & \wedge \langle y_{1}?(z) \rangle (\langle u! \rangle \langle x_{0}!z \rangle C(x_{0},x_{1},y_{0},y_{1},u)) & \\
  & \wedge \langle x_{1}?(z) \rangle (\langle u? \rangle \langle y_{0}!z \rangle C(x_{0},x_{1},y_{0},y_{1},u)) & \\
  & \wedge \langle y_{0}?(z) \rangle (\langle u? \rangle \langle x_{1}!z \rangle C(x_{0},x_{1},y_{0},y_{1},u))) &
\end{align*}

The lexicographical similarity between the shape of this formulae and
the shape of definition of the process representing a crossing reveals
the intuitive meaning of this formulae. It describes the capabilities
of a process that has the right to represent a crossing. For example
it picks out processes that may perform an input on the port $x_0$ in
its initial menu of capabilities. What differentiates the formula
from the process, however, is that the crossing process is the
smallest candidate to satisfy the formula. Infinitely many other
processes -- with internal behavior hidden behind this interface, so
to speak -- also satisfy this formula. Even this simple formula,
then, can be seen to open a new view onto knots, providing a
computational interpretation of \emph{virtual} knots.

Note that this formula is derived by hand. A similar formula can be
derived by employing Caires' calculation of characteristic formula
\cite{Caires04} to the process representing a crossing. In light of
this discussion, we let
$\meaningof{C}_{\phi}(x0,x1,y0,y1,u)$ denote a formula specifying the
dynamics we wish to capture of a crossing. To guarantee we preserve
the shape of the interface and minimal semantics we demand that
$\meaningof{C}_{\phi}(x0,x1,y0,y1,u) \Rightarrow
\textbf{C}(x0,x1,y0,y1,u)$ where $\textbf{C}(x0,x1,y0,y1,u)$ denotes
the formula above.
                            
\subsubsection{Crossing number constraints.}
The moral content of the context lemma (Lemma \ref{context}) is that the notion of
``locality'' in the Reidemeister moves is effectively captured by the
parallel composition operator of the process calculus. This intuition
extends through the logic. Given a formula,
$\meaningof{C}_{\phi}(x0,x1,y0,y1,u)$, we can use the structural
connectives to specify constraints on crossing numbers, such as at
least $n$ crossings, or exactly $n$ crossings.
\begin{mathpar}
  \inferrule* [lab=at-least-n] {} { K^{\geq n}_{\phi}(\vec{xs},\vec{ys}) := \Pi_{i=0}^{n-1} Hu . \meaningof{C}_{\phi}(xs_i,ys_i,u) | T }
  \and 
  \inferrule* [lab=exactly-n] {} { K^{= n}_{\phi}(\vec{xs},\vec{ys}) := \Pi_{i=0}^{n-1} Hu . \meaningof{C}_{\phi}(xs_i,ys_i,u) | \neg (\forall x_0,y_0,x_1,y_1,u . \meaningof{C}_{\phi}(x_0,y_0,x_1,y_1,u) | T) }
\end{mathpar}

To round out this section, recall that the encoding of an $n$-crossing
knot decomposes into a parallel composition of $n$ \emph{copies} of a
crossing process together with a wiring harness. To specify different
knot classes with the same crossing number amounts to specifying
logical constraints on the wiring harness. In the interest of space,
we defer examples to a forthcoming paper. Suffice it to say that both
the conditions ``alternating knot'' and ``contains the tangle
corresponding to 5/3'' are expressible. For example, it is possible to
calculate the characteristic formula of a process corresponding to the
tangle 5/3 and conjoin it into the classifying formula via the
composition connective of the logic.

Finally, we wish to observe that it is entirely within reason to
contemplate a more domain-specific version of spatial logic tailored
to the shape of processes in the image of the encoding. Such a
domain-specific logic would have a better claim to the title formal
language of knot properties.

% subsection example_formulae_ (end)

% section knots_as_processes (end) 

% section spatial logic via knots (end)

\section{Conclusions and future work}

\paragraph{Testing physical space}
You, gentle reader, may wonder why of all the theorems to be proved
given this set up we pick the one above. In some sense it's hardly
central to quantum mechanics. We see it as central in the sense that
it firmly establishes a notion of physical space arising from a notion
of the equivalence of behavior. Relating bisimulation to a metric is a
big step forward, but one is faced with interpreting the relationship
of that metric space to something more physical. Quantum mechanical
notions of ``physical'' space are still far from intuitive, but by
relating this idea of distance as testing to calculations that predict
physical circumstances we are making a not insignificant step forward
toward an understanding of the physical space we inhabit as
essentially dynamic.

\paragraph{Effectivity and simulation}
One of the observations we have yet to make is that the entire program
spelled out here is effective. We have built various interpreters for
the reflective calculus at work in this interpretation. In principle,
then, we can simulate quantum mechanics on a computer. The place where
the simulation may lose fidelity is the infinitely branching summation
for the annihilator.

In this connection i also want to point out that the evaluation style
calculation of the inner product puts the non-determinism of the
summation right at the heart of measurement. This suggests that
Milner's original reduction-based formulation of the dynamics of his
calculi in terms of sums was not just notationally suggestive of a
notion of measure-and-continue but captured some significant part of
the physics.

\paragraph{Quantum continuations}
In light of this last observation i want to point out that the
predominant account of quantum mechanics is missing a key aspect of a
truly compositional story of the physical situation. In a real lab,
when a measurement is made the observation can be made to feed into
another device that then makes another measurement conditioned on the
results of the first. This means that after the superposition was
collapsed the entire experimental set up remained in
superposition. While QM offers a means of writing this down it doesn't
quite line up well with the well-trodden formulation of computation
and continuation that we see so succinctly expressed in Milner's
calculi. This suggests that there might be advantages to this account
of dynamics waiting to be explored.

\paragraph{Quantum logic}
In this connection, we also note that by virtue of having the
Hennessy-Milner construction, we can pull the construction through the
interpretation of QM. This gives us a natural candidate for a quantum
logic that enjoys an extremely tight connection with it's domain of
interpretation, making the construction much less ad hoc (rather it is
the image of functor!).

\paragraph{Quantum probabiity}
i have questions about the basis of the interpretation of inner
product as probability amplitude. In particular, using which
axiomatization of probability theory does the notion of probability
amplitude earn the right to be so dubbed? In other words, where is the
proof that the operation for calculating a probability amplitude (and
then squaring) satisfies the axioms of what it means to calculate a
probability? Even if such a proof exists (i have yet to find it in the
literature), i wonder if it might not be possible to turn things on
their heads. Can we view the calculation of the probability amplitude
as an axiomatization of probability? If so, then the definition we
give for calculating probability amplitude may provide the basis for
an \emph{effective} theory of probability.

\paragraph{Quantum vs ``biological'' information}
Finally, i want to conclude with a more philosophical observation. At
a recent workshop in which QM was a predominant topic i noticed
something about quantum information. The speaker was giving a riveting
discussion of axiomatic QM and showing how properties of ``no
cloning'' and ``no deleting'' emerged as consequences of the
axiomatization. Theorems of this form are necessary to give us a sense
of confidence that our axioms characterize the physical theory. What
struck me, though, was that if quantum information is neither erasable
nor replicable it is markedly different from \emph{life}. Two of the
things we know about life is that

\begin{itemize}
  \item it ends;
  \item to gain some measure of persistence, to transcend it's
    finitude it is imminently copyable.
\end{itemize}

Both of these qualities are summarized succinctly in the aphorism: all
flesh is grass. For me these two kinds of ``information'' -- call them
quantum and biological -- are end points on a spectrum of strategies
for persistence. At one end, we have those curious entities that enjoy
uniqueness and permanence; at the other, we have those who in the face
of a certain end and an uncertain present make a go of passing
something on. To me one of the more remarkable aspects of the latter
strategy is that in the presence of noise (and certain features of
copying) we get a kind of dynamism, a chance for improvement against a
given persistent condition.

% subsection other_calculi_other_bisimulations_and_geometry_as_behavior (end)




% section conclusion (end)

%\documentclass[12pt]{llncs}
%\documentclass{jktr}

\usepackage[pdftex]{hyperref}                   
\usepackage {listings}
\usepackage {mathpartir}
\usepackage{bcprules}
%\usepackage{listings}
                       
\usepackage{graphicx} 
%\usepackage[margins=2.5cm,nohead,nofoot]{geometry}
%\usepackage{geometry}
\usepackage{amsfonts}
\usepackage{amstext}
\usepackage{latexsym}
\usepackage{amssymb}
\usepackage{color}


%\include{myPreamble}
\include{qm2pi.local} 

%\ifpdf
%\usepackage[pdftex]{graphicx}
%\else
%\usepackage{graphicx}
%\fi

 % \ifpdf
%  \usepackage{pdfsync}
%  \if


%\title{Brief Article}
%\author{David F. Snyder}
%\author{L.G. Meredith}

%\address{Dept. of Math., Texas State University--San Marcos, San Marcos, TX 78666}
       
\pagestyle{empty}


\begin{document}

\lstset{language=[Objective]Caml,frame=shadowbox}

\input{qm2pi.front}

% section front matter (end)

\input{qm2pi.intro} 
 
% section introduction (end)

% \input{qm2pi.knotations} 

% section notation (end)

\input{qm2pi.process.calculi} 

% section concurrent_process_calculi_and_spatial_logics_ (end)
    
%\input{qm2pi.knots2pi} 

%\input{qm2pi.trefoil} 

%\input{qm2pi.mainthm} 

% subsection basic_interpretation (end)

%\input{qm2pi.rho.presentation} 
\subsection{The syntax and semantics of the notation system}\label{sub:the_syntax_and_semantics_of_the_notation_system} % (fold)

We now summarize a technical presentation of the calculus that
embodies our theory of dynamics. The typical presentation of such a
calculus follows the style of giving generators and relations on
them. The grammar, below, describing term constructors, freely
generates the set of processes, $\Proc$. This set is then quotiented
by a relation known as structural congruence and it is over this set
that the notion of dynamics is expressed. This presentation is
essentially that of \cite{MeredithR05} with the addition of
polyadicity and summation. For readability we have relegated some of
the technical subtleties to an appendix.

\subsubsection{Process grammar}\label{subsub:process_grammar}

\begin{mathpar}
  \inferrule* [lab=synchronization] {} {{M} \bc \pzero \;|\; x?F \;|\; x!C }
  \and
  \inferrule* [lab=abstraction] {} {{F} \bc (x)P}
  \and
  \inferrule* [lab=concretion] {} {{C} \bc \langle Q \rangle}
  \and
  \inferrule* [lab=process] {} {{P,Q} \bc M \;| \;P|Q \;|\; @{x}}
  \and
  \inferrule* [lab=name] {} {{x} \bc \quotep{P}}
\end{mathpar} 

Note that $\vec{x}$ (resp. $\vec{P}$) denotes a vector of names
(resp. processes) of length $|\vec{x}|$ (resp. $|\vec{P}|$). We adopt
the following useful abbreviations.

\begin{mathpar}
   x?(\vec{y}).P := x.(\vec{y})P \and  x\clift{\vec{P}} := x.\clift{\vec{P}}
   \and x!(y) := \lift{x}{\dropn{y}}
   \and \Pi_{i=0}^{n-1}P_i := P_0 | \ldots | P_{n-1}
\end{mathpar}

\subsubsection{Structural congruence}

\paragraph{Free and bound names and alpha-equivalence.} At the
core of structural equivalence is alpha-equivalence which identifies
process that are the same up to a change of variable. Formally, we
recognize the distinction between free and bound names. The free names
of a process, $\freenames{P}$, may be calculated recursively as
follows:

\begin{mathpar}
\freenames{\pzero} := \emptyset
  \and \\
  \freenames{x?(y).P} := \{ x \} \cup (\freenames{P} \setminus \{ y \})
  \and 
  \freenames{x!\langle P \rangle} := \{ x \} \cup \{ P \} 
  \and \\
  \freenames{P|Q} := \freenames{P} \cup \freenames{Q}
  \and \\
  \freenames{@{x}} := \{ x \}
\end{mathpar}

$\pi$
$\quotep{\pi}$

$\freenames{-} : \pi \to \mathcal{P}(\quotep{\pi})$

\begin{eqnarray*}
  \freenames{\pzero} & := & \emptyset \\
  \freenames{x?(y).P} & := & \{ x \} \cup (\freenames{P} \setminus \{ y \}) \\
  \freenames{x!\langle P \rangle} & := & \{ x \} \cup \{ P \} \\
  \freenames{P|Q} & := & \freenames{P} \cup \freenames{Q} \\
  \freenames{\dropn{x}} & := & \{ x \}
\end{eqnarray*}

The bound names of a process, $\boundnames{P}$, are those names occurring in $P$
that are not free. For example, in $x?(y).0$, the name $x$ is free, while $y$ is bound.

\begin{mathpar}
  \inferrule* [lab=monoidal-laws] {} { P|Q \equiv Q|P \and P|0 \equiv P \and P|(Q|R) \equiv (P|Q)|R }
\end{mathpar}

\begin{mathpar}
  \inferrule* [lab=alpha-equivalence] {} { (x)P \equiv (y)P\{y/x\} \and y \not\in \freenames{P} }
\end{mathpar}

\begin{definition}
Then two processes, $P,Q$, are alpha-equivalent if $P = Q\{\vec{y}/\vec{x}\}$ for
some $\vec{x} \in \boundnames{Q},\vec{y} \in \boundnames{P}$, where $Q\{\vec{y}/\vec{x}\}$
denotes the capture-avoiding substitution of $\vec{y}$ for $\vec{x}$ in $Q$.
\end{definition}

\begin{definition}
  The {\em structural congruence} \cite{SangiorgiWalker} , $\equiv$,
  between processes is the least congruence containing
  alpha-equivalence, satisfying the abelian monoid laws
  (associativity, commutativity and $\pzero$ as identity) for parallel
  composition $|$ and for summation $+$.
\end{definition}

\subsection{Name equivalence}

We take name equivalence, written $\nameeq$, to be the smallest
equivalence relation generated by the following rules.

\begin{mathpar}
\inferrule*[lab=Quote-drop]
{ }
{ \quotep{@{x}} \nameeq x }

\inferrule*[lab=Struct-equiv]
{ P \scong Q }
{ \quotep{P} \nameeq \quotep{Q} }
\end{mathpar}

The astute reader will have noticed that the mutual recursion of names
and processes imposes a mutual recursion on alpha-equivalence and
structural equivalence via name-equivalence. Fortunately, all of this
works out pleasantly and we may calculate in the natural way, free of
concern. The reader interested in the details is referred to the
appendix \ref{appendix:rho_details}.

\subsection{Substitution}

We use $\Proc$ for the set of processes, $\QProc$ for the set of
names, and $\id{\{}\vec{y} / \vec{x} \id{\}}$ to denote partial maps,
$s : \QProc \rightarrow \QProc$. A map, $s$ lifts, uniquely, to a map
on process terms, $\widehat{s} : \Proc \rightarrow \Proc$ by the
following equations.

\begin{mathpar}
  (0) \psubstp{Q}{P} := 0 \\
  (R \juxtap S) \psubstp{Q}{P}
  :=    
  (R)\psubstp{Q}{P} \juxtap (S) \psubstp{Q}{P} \\
  (x?(y).R) \psubstp{Q}{P}    
  :=    
  (x)\substp{Q}{P} (z)\concat( (R \psubstn{z}{y}) \psubstp{Q}{P} ) \\
  (\lift{x}{R}) \psubstp{Q}{P}  
  :=
  \lift{(x)\substp{Q}{P}}{ R \psubstp{Q}{P} } \\
%   (\dropn{x})  \psubstp{Q}{P}       
%   := 
%   \left\{ 
%     \begin{array}{ccc} 
%       \dropn{\quotep{Q}} & & x \nameeq \quotep{P} \\
%       \dropn{x} & & otherwise \\
%     \end{array}
%   \right. 
  (\dropn{x})  \psubstp{Q}{P}       
  := 
  \left\{ 
    \begin{array}{ccc} 
      Q & & x \nameeq \quotep{P} \\
      \dropn{x} & & otherwise \\
    \end{array}
  \right.
\end{mathpar}
 

where

\begin{eqnarray}
  (x)\id{\{} \lpquote Q \rpquote / \lpquote P \rpquote \id{\}}            = 
  \left\{ 
    \begin{array}{ccc}
      \lpquote Q \rpquote & & x \nameeq \lpquote P \rpquote \\
      x & & otherwise \\
    \end{array}
  \right. \nonumber
\end{eqnarray}

and $z$ is chosen distinct from $\quotep{P}$, $\quotep{Q}$, the free
names in $Q$, and all the names in $R$. Our $\alpha$-equivalence will
be built in the standard way from this substitution.

\begin{remark}\label{rem:no_self_referential_names}
  One consequence of these definitions is that $\forall P. \quotep{P}
  \not\in \freenames{P}$.
\end{remark}

\subsection{ Dynamic quote: an example }

Anticipating something of what's to come, consider applying the
substitution, $\widehat{\id{\{}u / z \id{\}}}$, to the following pair
of processes, $\lift{w}{y!(z)}$ and $w[ \lpquote y!(z) \rpquote ]$.

\begin{eqnarray}
	\lift{w}{y!(z)}\widehat{\id{\{}u / z \id{\}}}
		& = &
		\lift{w}{y!(u)} \nonumber\\
	w[ \lpquote y!(z) \rpquote ] \widehat{ \id{\{}u / z \id{\}} }
		& = &
		w[ \lpquote y!(z) \rpquote ] \nonumber
\end{eqnarray}

Because the body of the process between quotes is impervious to
substitution, we get radically different answers. In fact, by
examining the first process in an input context,
e.g. $x?(z).\lift{w}{y!(z)}$, we see that the process under the lift
operator may be shaped by prefixed inputs binding a name inside it. In
this sense, the lift operator will be seen as a way to dynamically
construct processes before reifying them as names.

Finally equipped with these standard features we can present the
dynamics of the calculus.

\subsubsection{Operational semantics} 

Finally, we introduce the computational dynamics. What marks these
algebras as distinct from other more traditionally studied algebraic
structures, e.g. vector spaces or polynomial rings, is the manner in
which dynamics is captured. In traditional structures, dynamics is typically
expressed through morphisms between such structures, as in linear maps
between vector spaces or morphisms between rings. In algebras
associated with the semantics of computation, the dynamics is
expressed as part of the algebraic structure itself, through a
reduction reduction relation typically denoted by $\red$. Below, we
give a recursive presentation of this relation for the calculus used
in the encoding.

$\red \subseteq \pi \times \pi$
$\red : \pi \to \mathcal{P}(\pi)$

\begin{mathpar}
  \inferrule* [lab=Comm] { \textsf{match}( x_{src}, x_{trgt} ) } { x_{trgt}?(y)P \; | \; x_{src}!\langle {Q} \rangle \red P\{\quotep{Q}/y}\} }
  \and \\
  \inferrule* [lab=Par] {{P} \red {P}'} {{{P} | {Q}} \red {{P}' | {Q}}}
  \and
  \inferrule* [lab=Equiv]{{{P} \scong {P}'} \andalso {{P}' \red {Q}'} \andalso {{Q}' \scong {Q}}}{{P} \red {Q}}
\end{mathpar}

\begin{eqnarray*}
  match_{\equiv} (\quotep{P},\quotep{Q}) & := & P \equiv Q \\
  match_{\dagger}(\quotep{P},\quotep{Q}) & := & \forall R. P|Q \red^{*} R => R \red^{*} 0 \\
  match_{K}(\quotep{P},\quotep{Q}) & := & K \mbox{ for some context } K
\end{eqnarray*}

$u?(x)P | u!\langle Q \rangle \red P\{\quotep{Q}/x\}$

%We write $\wred$ for $\red^*$, and $P\red$ if $\exists Q $ such that $ P \red Q$.
We write $P\red$ if $\exists Q $ such that $ P \red Q$ and $P\not\red$, otherwise.

\section{Replication}

As mentioned before, it is known that replication (and hence
recursion) can be implemented in a higher-order process algebra
\cite{SangiorgiWalker}. As our first example of calculation with the
machinery thus far presented we give the construction explicitly in
the {\rhoc}.

\begin{eqnarray}
	D_{x} & := & \prefix{x}{y}{(\binpar{\outputp{x}{y}}{@{y}})} \nonumber\\
	\bangp_{x}{P} & := & \binpar{{x}!\langle{\binpar{D_{x}}{P}}\rangle}{D_{x}} \nonumber
\end{eqnarray}

\begin{eqnarray}
	\bangp_{x}{P} & & \nonumber\\
	=
	& {x}!\langle{(\prefix{x}{y}{(\outputp{x}{y} | @{y})) | P}}\rangle 
	      | \prefix{x}{y}{(\outputp{x}{y} | @{y})} & \nonumber\\
	\red
	& (\outputp{x}{y} | @{y})\substn{\quotep{(\prefix{x}{y}{(@{y} | \outputp{x}{y})) | P}}}{y} & \nonumber\\
	=
	& \outputp{x}{\quotep{(\prefix{x}{y}{(\outputp{x}{y} | @{y})) | P}}}
	  | {(\prefix{x}{y}{(\outputp{x}{y} | @{y})) | P}} & \nonumber\\
	\red
	& \ldots & \nonumber\\
	\red^*
	& P | P | \ldots & \nonumber
\end{eqnarray}

Of course, this encoding, as an implementation, runs away, unfolding
$\bangp{P}$ eagerly. A lazier and more implementable replication
operator, restricted to input-guarded processes, may be obtained as follows.

\begin{eqnarray}
\bangp{\prefix{u}{v}{P}} 
	:= 
	\binpar{\lift{x}{\prefix{u}{v}{(\binpar{D(x)}{P})}}}{D(x)} \nonumber
\end{eqnarray}

\begin{remark}
  Note that the lazier definition still does not deal with summation
  or mixed summation (i.e. sums over input and output). The reader is
  invited to construct definitions of replication that deal with these
  features. 

  Further, the definitions are parameterized in a name, $x$. Can you,
  gentle reader, make a definition that eliminates this parameter and
  guarantees no accidental interaction between the replication
  machinery and the process being replicated -- i.e. no accidental
  sharing of names used by the process to get its work done and the
  name(s) used by the replication to effect copying. This latter
  revision of the definition of replication is crucial to obtaining
  the expected identity $!!P \sim !P$.
\end{remark}

\begin{remark}\label{rem:paradoxical_combinator}
  The reader familiar with the lambda calculus will have noticed the
  similarity between $D$ and the paradoxical combinator.

  [Ed. note: the existence of this seems to suggest we have to be more
  restrictive on the set of processes and names we admit if we are to
  support no-cloning.]
\end{remark}

\subsubsection{Bisimulation}

The computational dynamics gives rise to another kind of equivalence,
the equivalence of computational behavior. As previously mentioned
this is typically captured \emph{via} some form of bisimulation.

% The notion we use in this paper is weak barbed bisimulation
% \cite{milner91polyadicpi}.

The notion we use in this paper is derived from weak barbed
bisimulation \cite{milner91polyadicpi}. 

\begin{definition}
An \emph{observation relation}, $\downarrow_{\mathcal N}$, over a set
of names, $\mathcal N$, is the smallest relation satisfying the rules
below.

\infrule[Out-barb]{y \in {\mathcal N}, \; x \nameeq y}
		  {\outputp{x}{v} \downarrow_{\mathcal N} x}
\infrule[Par-barb]{\mbox{$P\downarrow_{\mathcal N} x$ or $Q\downarrow_{\mathcal N} x$}}
		  {\binpar{P}{Q} \downarrow_{\mathcal N} x}

We write $P \Downarrow_{\mathcal N} x$ if there is $Q$ such that 
$P \wred Q$ and $Q \downarrow_{\mathcal N} x$.
\end{definition}

\begin{definition}
%\label{def.bbisim}
An  ${\mathcal N}$-\emph{barbed bisimulation} over a set of names, ${\mathcal N}$, is a symmetric binary relation 
${\mathcal S}_{\mathcal N}$ between agents such that $P\rel{S}_{\mathcal N}Q$ implies:
\begin{enumerate}
\item If $P \red P'$ then $Q \wred Q'$ and $P'\rel{S}_{\mathcal N} Q'$.
\item If $P\downarrow_{\mathcal N} x$, then $Q\Downarrow_{\mathcal N} x$.
\end{enumerate}
$P$ is ${\mathcal N}$-barbed bisimilar to $Q$, written
$P \wbbisim_{\mathcal N} Q$, if $P \rel{S}_{\mathcal N} Q$ for some ${\mathcal N}$-barbed bisimulation ${\mathcal S}_{\mathcal N}$.
\end{definition}

$\mathcal{R} \subseteq \pi \times \pi$

$P \mathcal{R} Q => \forall P'. P \red P' \Rightarrow \exists Q'. Q \red Q', P' \mathcal{R} Q'$

$P \vdash x \Rightarrow Q \vdash x$

\begin{mathpar}
  \inferrule*[lab=Out-barb]{x \nameeq y}{{y}!\langle{Q}\rangle \vdash x}
  \and
  \inferrule*[lab=Par-barb]{\mbox{$P\vdash x$ or $Q\vdash x$}}{\binpar{P}{Q} \vdash x}
\end{mathpar}

\subsubsection{Contexts}

One of the principle advantages of computational calculi like the
$\pi$-calculus is a well-defined notion of context,
contextual-equivalence and a correlation between
contextual-equivalence and notions of bisimulation. The notion of
context allows the decomposition of a process into (sub-)process and
its syntactic environment, its context. Thus, a context may be
thought of as a process with a ``hole'' (written $\Box$) in it. The
application of a context $M$ to a process $P$, written $M[P]$, is
tantamount to filling the hole in $M$ with $P$. In this paper we do
not need the full weight of this theory, but do make use of the notion
of context in the proof the main theorem. 

\begin{mathpar}
  \inferrule* [lab=summation] {} {{M_{M},M_{N}} \bc \Box \;|\; x.M_{A} \;|\; M_{M}+M_{N}}
  \and
  \inferrule* [lab=agent] {} {{M_{A}} \bc (\vec{x})M_{P} \;| \; \clift{P_0,\ldots,M_{P},\ldots,P_N}}
  \and \\
  \inferrule* [lab=process] {} {{M_{P}} \bc M_{N} \;| \;P|M_{P} }
\end{mathpar} 

\begin{mathpar}
  \inferrule* [lab=sychronization] {} {M_{N} \bc \Box \;|\; x?M_{F} \;|\; x!M_{C}}
  \and
  \inferrule* [lab=abstraction] {} {{M_{F}} \bc (x)M_{P} }
  \and
  \inferrule* [lab=concretion] {} {{M_{C}} \bc \langle M_{P} \rangle }
  \and \\
  \inferrule* [lab=process] {} {{M_{P}} \bc M_{N} \;| \;P|M_{P} }
\end{mathpar}

\begin{definition}[contextual application] Given a context $M$, and
  process $P$, we define the \emph{contextual application}, $M[P] :=
  M\{P/\Box\}$. That is, the contextual application of M to P is the
  substitution of $P$ for $\Box$ in $M$.
\end{definition}

$\meaningof{-} : L \to \mathcal{P}(\pi)$

\begin{mathpar}
  \inferrule* [lab=collection] {} {\meaningof{true} = \pi, \and \meaningof{~E} = \pi \setminus \meaningof{E}, \and \meaningof{E_{1} \& E_{2}} = \meaningof{E_{1}} \cap \meaningof{E_{2}}}
\end{mathpar}

\begin{mathpar}
  \inferrule* [lab=structure] {} {\meaningof{0} = \{ P \in \pi | P \equiv 0 \}, \and \\ \meaningof{E_1 | E_2} = \{ P \in \pi | P \equiv P_{1} | P_{2}, P_{1} \in \meaningof{E_{1}}, P_{2} \in \meaningof{E_2}\} }
\end{mathpar}

\begin{mathpar}
 \inferrule* [lab=behavior] {} {\meaningof{\langle a?b \rangle E} = \{ P \in \pi | P \equiv Q | u?(y)P', \\ \and \\\\ \and \\ \;\;\; u \in \meaningof{a}, \forall z.P'\{z/y\} \in \meaningof{E\{z/b\}}\}, \and \\ \meaningof{a!E} = \{ P \in \pi | P \equiv Q | x!\langle P' \rangle, x \in \meaningof{a} P' \in \meaningof{E}\} }
\end{mathpar}

\begin{mathpar}
 \inferrule* [lab=nominal] {} {\meaningof{\quotep{E}} = \{ \quotep{P} \in \quotep{\pi} | P \in \meaningof{E} \}, \and \meaningof{\quotep{P}} = \{ \quotep{Q} \in \quotep{\pi} | P \equiv Q \} \and \\ \meaningof{@\quotep{E}} = \{ P \in \pi | P \equiv @x, x \in \meaningof{E} \}}
\end{mathpar}

\begin{eqnarray*}
  \\
  \meaningof{-} : TS \to ST
\end{eqnarray*}

\begin{eqnarray*}
  \\
  L : TS \to ST
\end{eqnarray*}

\begin{eqnarray*}
  \\
  P \models E \iff P \in \meaningof{E}
\end{eqnarray*}

\begin{eqnarray*}
  P \approx_{L} Q \iff \forall E \in L. P \models E \iff Q \models E
\end{eqnarray*}

\begin{eqnarray*}
  P \approx_{K} Q
\end{eqnarray*}

\begin{eqnarray*}
  P \approx Q
\end{eqnarray*}

$\approx_{K} = \approx = \approx_{L}$

\subsubsection{Contextual duality}

Note that contexts extend the quotation operation to a family of
operations from processes to names. Given a context, $M$, we can
define a \emph{nominal context}, $\quotep{M}$ by $\quotep{M}[P] :=
\quotep{M[P]}$. To foreshadow what is to come we observe that these
operations enjoy a duality with processes very much like the duality
between vectors and maps from vectors to scalars.

Further, because the calculus is essentially higher-order, we have a
correspondence between contexts and processes. More specifically,
given a name $x$ and a context $M$ we can construct $M^{*}_{x}$ such
that 

\begin{mathpar}
  M^{*}_{x} | \lift{x}{P} \red M[P]
\end{mathpar}

namely,

\begin{mathpar}
  M^{*}_{x} := x?(u).M[\dropn{u}]
\end{mathpar}

The dependence of $M^{*}_{x}$ on a name makes it an abstraction, 

\begin{mathpar}
  M^{*} := (x)x?(u).M[\dropn{u}]
\end{mathpar}

\subsection{Additional notation}

It will sometimes be convenient to denote the process a name
quotes. We already have the notation $x = \quotep{P}$, but it will be
convenient to introduce an alternate notation, $\procn{x}$, when we
want to emphasize the connection to the use of the name. Note that, by
virtue of name equivalence, $\quotep{\procn{x}} \nameeq x$; so, the
notation is consistent with previous definitions.

Further, because names have structure it is possible to effect
substitutions on the basis of that structure. This means we need to
upgrade our notation for substitutions, which we accomplish by
adapting comprehension notation. Thus,

\begin{mathpar}
  P\{ y / x : x \in S \}
\end{mathpar}

is interpreted to mean the process derived from P by replacing (in a
capture-avoiding manner) each occurrence of $x$ in $S$ by $y$. For example,

\begin{mathpar}
  P\{ \quotep{\procn{x}|\procn{x}} / x : x \in \freenames{P} \}
\end{mathpar}

will replace each (occurrence) of a free name $x$ in $P$ by
$\quotep{\procn{x}|\procn{x}}$.

Also, we will avail ourselves of the notation $x^{L}$ and $x^{R}$ to
denote injections of a name into disjoint copies of the name
space. There are numerous ways to accomplish this. One example can be
found in \cite{MeredithR05}. This notation overloads to vectors of
names: $\vec{x}^{\pi} := (x_{i}^{\pi} \; : \; 0 \leq i < |\vec{x}| )$ where $\pi \in \{L,R\}$.

We also use $P^{\Box} := P|\Box$.

In \cite{MeredithR05} an interpretation of the new operator is
given. It turns out that there are several possible interpretations
all enjoying the requisite algebraic properties of the operator (see
\cite{milner91polyadicpi}). We will therefore make liberal use of
$(\nu\; \vec{x})P$.

% subsection the_syntax_and_semantics_of_the_notation_system (end)   

\input{qm2pi.qmops} 

\input{qm2pi.sterngerlach} 

\input{qm2pi.metric} 

% section concurrent_process_calculi (end)

%\input{qm2pi.proofsketch}

% section proof sketch (end)

%\input{qm2pi.slviaknots} 

% section spatial logic via knots (end)

\input{qm2pi.conclusion}

% section conclusion (end)

%\input{qm2pi.dtcodes} 

% section wiring algorithm (end)

\input{qm2pi.ack} 

% section acknowledgments (end)

\newpage


\bibliographystyle{plain}   
\bibliography{../../biblios/main.bib}

\input{qm2pi.rhodetails}

\end{document}

 

% section wiring algorithm (end)

\documentclass[12pt]{llncs}
%\documentclass{jktr}

\usepackage[pdftex]{hyperref}                   
\usepackage {listings}
\usepackage {mathpartir}
\usepackage{bcprules}
%\usepackage{listings}
                       
\usepackage{graphicx} 
%\usepackage[margins=2.5cm,nohead,nofoot]{geometry}
%\usepackage{geometry}
\usepackage{amsfonts}
\usepackage{amstext}
\usepackage{latexsym}
\usepackage{amssymb}
\usepackage{color}


%\include{myPreamble}
\include{qm2pi.local} 

%\ifpdf
%\usepackage[pdftex]{graphicx}
%\else
%\usepackage{graphicx}
%\fi

 % \ifpdf
%  \usepackage{pdfsync}
%  \if


%\title{Brief Article}
%\author{David F. Snyder}
%\author{L.G. Meredith}

%\address{Dept. of Math., Texas State University--San Marcos, San Marcos, TX 78666}
       
\pagestyle{empty}


\begin{document}

\lstset{language=[Objective]Caml,frame=shadowbox}

\input{qm2pi.front}

% section front matter (end)

\input{qm2pi.intro} 
 
% section introduction (end)

% \input{qm2pi.knotations} 

% section notation (end)

\input{qm2pi.process.calculi} 

% section concurrent_process_calculi_and_spatial_logics_ (end)
    
%\input{qm2pi.knots2pi} 

%\input{qm2pi.trefoil} 

%\input{qm2pi.mainthm} 

% subsection basic_interpretation (end)

%\input{qm2pi.rho.presentation} 
\subsection{The syntax and semantics of the notation system}\label{sub:the_syntax_and_semantics_of_the_notation_system} % (fold)

We now summarize a technical presentation of the calculus that
embodies our theory of dynamics. The typical presentation of such a
calculus follows the style of giving generators and relations on
them. The grammar, below, describing term constructors, freely
generates the set of processes, $\Proc$. This set is then quotiented
by a relation known as structural congruence and it is over this set
that the notion of dynamics is expressed. This presentation is
essentially that of \cite{MeredithR05} with the addition of
polyadicity and summation. For readability we have relegated some of
the technical subtleties to an appendix.

\subsubsection{Process grammar}\label{subsub:process_grammar}

\begin{mathpar}
  \inferrule* [lab=synchronization] {} {{M} \bc \pzero \;|\; x?F \;|\; x!C }
  \and
  \inferrule* [lab=abstraction] {} {{F} \bc (x)P}
  \and
  \inferrule* [lab=concretion] {} {{C} \bc \langle Q \rangle}
  \and
  \inferrule* [lab=process] {} {{P,Q} \bc M \;| \;P|Q \;|\; @{x}}
  \and
  \inferrule* [lab=name] {} {{x} \bc \quotep{P}}
\end{mathpar} 

Note that $\vec{x}$ (resp. $\vec{P}$) denotes a vector of names
(resp. processes) of length $|\vec{x}|$ (resp. $|\vec{P}|$). We adopt
the following useful abbreviations.

\begin{mathpar}
   x?(\vec{y}).P := x.(\vec{y})P \and  x\clift{\vec{P}} := x.\clift{\vec{P}}
   \and x!(y) := \lift{x}{\dropn{y}}
   \and \Pi_{i=0}^{n-1}P_i := P_0 | \ldots | P_{n-1}
\end{mathpar}

\subsubsection{Structural congruence}

\paragraph{Free and bound names and alpha-equivalence.} At the
core of structural equivalence is alpha-equivalence which identifies
process that are the same up to a change of variable. Formally, we
recognize the distinction between free and bound names. The free names
of a process, $\freenames{P}$, may be calculated recursively as
follows:

\begin{mathpar}
\freenames{\pzero} := \emptyset
  \and \\
  \freenames{x?(y).P} := \{ x \} \cup (\freenames{P} \setminus \{ y \})
  \and 
  \freenames{x!\langle P \rangle} := \{ x \} \cup \{ P \} 
  \and \\
  \freenames{P|Q} := \freenames{P} \cup \freenames{Q}
  \and \\
  \freenames{@{x}} := \{ x \}
\end{mathpar}

$\pi$
$\quotep{\pi}$

$\freenames{-} : \pi \to \mathcal{P}(\quotep{\pi})$

\begin{eqnarray*}
  \freenames{\pzero} & := & \emptyset \\
  \freenames{x?(y).P} & := & \{ x \} \cup (\freenames{P} \setminus \{ y \}) \\
  \freenames{x!\langle P \rangle} & := & \{ x \} \cup \{ P \} \\
  \freenames{P|Q} & := & \freenames{P} \cup \freenames{Q} \\
  \freenames{\dropn{x}} & := & \{ x \}
\end{eqnarray*}

The bound names of a process, $\boundnames{P}$, are those names occurring in $P$
that are not free. For example, in $x?(y).0$, the name $x$ is free, while $y$ is bound.

\begin{mathpar}
  \inferrule* [lab=monoidal-laws] {} { P|Q \equiv Q|P \and P|0 \equiv P \and P|(Q|R) \equiv (P|Q)|R }
\end{mathpar}

\begin{mathpar}
  \inferrule* [lab=alpha-equivalence] {} { (x)P \equiv (y)P\{y/x\} \and y \not\in \freenames{P} }
\end{mathpar}

\begin{definition}
Then two processes, $P,Q$, are alpha-equivalent if $P = Q\{\vec{y}/\vec{x}\}$ for
some $\vec{x} \in \boundnames{Q},\vec{y} \in \boundnames{P}$, where $Q\{\vec{y}/\vec{x}\}$
denotes the capture-avoiding substitution of $\vec{y}$ for $\vec{x}$ in $Q$.
\end{definition}

\begin{definition}
  The {\em structural congruence} \cite{SangiorgiWalker} , $\equiv$,
  between processes is the least congruence containing
  alpha-equivalence, satisfying the abelian monoid laws
  (associativity, commutativity and $\pzero$ as identity) for parallel
  composition $|$ and for summation $+$.
\end{definition}

\subsection{Name equivalence}

We take name equivalence, written $\nameeq$, to be the smallest
equivalence relation generated by the following rules.

\begin{mathpar}
\inferrule*[lab=Quote-drop]
{ }
{ \quotep{@{x}} \nameeq x }

\inferrule*[lab=Struct-equiv]
{ P \scong Q }
{ \quotep{P} \nameeq \quotep{Q} }
\end{mathpar}

The astute reader will have noticed that the mutual recursion of names
and processes imposes a mutual recursion on alpha-equivalence and
structural equivalence via name-equivalence. Fortunately, all of this
works out pleasantly and we may calculate in the natural way, free of
concern. The reader interested in the details is referred to the
appendix \ref{appendix:rho_details}.

\subsection{Substitution}

We use $\Proc$ for the set of processes, $\QProc$ for the set of
names, and $\id{\{}\vec{y} / \vec{x} \id{\}}$ to denote partial maps,
$s : \QProc \rightarrow \QProc$. A map, $s$ lifts, uniquely, to a map
on process terms, $\widehat{s} : \Proc \rightarrow \Proc$ by the
following equations.

\begin{mathpar}
  (0) \psubstp{Q}{P} := 0 \\
  (R \juxtap S) \psubstp{Q}{P}
  :=    
  (R)\psubstp{Q}{P} \juxtap (S) \psubstp{Q}{P} \\
  (x?(y).R) \psubstp{Q}{P}    
  :=    
  (x)\substp{Q}{P} (z)\concat( (R \psubstn{z}{y}) \psubstp{Q}{P} ) \\
  (\lift{x}{R}) \psubstp{Q}{P}  
  :=
  \lift{(x)\substp{Q}{P}}{ R \psubstp{Q}{P} } \\
%   (\dropn{x})  \psubstp{Q}{P}       
%   := 
%   \left\{ 
%     \begin{array}{ccc} 
%       \dropn{\quotep{Q}} & & x \nameeq \quotep{P} \\
%       \dropn{x} & & otherwise \\
%     \end{array}
%   \right. 
  (\dropn{x})  \psubstp{Q}{P}       
  := 
  \left\{ 
    \begin{array}{ccc} 
      Q & & x \nameeq \quotep{P} \\
      \dropn{x} & & otherwise \\
    \end{array}
  \right.
\end{mathpar}
 

where

\begin{eqnarray}
  (x)\id{\{} \lpquote Q \rpquote / \lpquote P \rpquote \id{\}}            = 
  \left\{ 
    \begin{array}{ccc}
      \lpquote Q \rpquote & & x \nameeq \lpquote P \rpquote \\
      x & & otherwise \\
    \end{array}
  \right. \nonumber
\end{eqnarray}

and $z$ is chosen distinct from $\quotep{P}$, $\quotep{Q}$, the free
names in $Q$, and all the names in $R$. Our $\alpha$-equivalence will
be built in the standard way from this substitution.

\begin{remark}\label{rem:no_self_referential_names}
  One consequence of these definitions is that $\forall P. \quotep{P}
  \not\in \freenames{P}$.
\end{remark}

\subsection{ Dynamic quote: an example }

Anticipating something of what's to come, consider applying the
substitution, $\widehat{\id{\{}u / z \id{\}}}$, to the following pair
of processes, $\lift{w}{y!(z)}$ and $w[ \lpquote y!(z) \rpquote ]$.

\begin{eqnarray}
	\lift{w}{y!(z)}\widehat{\id{\{}u / z \id{\}}}
		& = &
		\lift{w}{y!(u)} \nonumber\\
	w[ \lpquote y!(z) \rpquote ] \widehat{ \id{\{}u / z \id{\}} }
		& = &
		w[ \lpquote y!(z) \rpquote ] \nonumber
\end{eqnarray}

Because the body of the process between quotes is impervious to
substitution, we get radically different answers. In fact, by
examining the first process in an input context,
e.g. $x?(z).\lift{w}{y!(z)}$, we see that the process under the lift
operator may be shaped by prefixed inputs binding a name inside it. In
this sense, the lift operator will be seen as a way to dynamically
construct processes before reifying them as names.

Finally equipped with these standard features we can present the
dynamics of the calculus.

\subsubsection{Operational semantics} 

Finally, we introduce the computational dynamics. What marks these
algebras as distinct from other more traditionally studied algebraic
structures, e.g. vector spaces or polynomial rings, is the manner in
which dynamics is captured. In traditional structures, dynamics is typically
expressed through morphisms between such structures, as in linear maps
between vector spaces or morphisms between rings. In algebras
associated with the semantics of computation, the dynamics is
expressed as part of the algebraic structure itself, through a
reduction reduction relation typically denoted by $\red$. Below, we
give a recursive presentation of this relation for the calculus used
in the encoding.

$\red \subseteq \pi \times \pi$
$\red : \pi \to \mathcal{P}(\pi)$

\begin{mathpar}
  \inferrule* [lab=Comm] { \textsf{match}( x_{src}, x_{trgt} ) } { x_{trgt}?(y)P \; | \; x_{src}!\langle {Q} \rangle \red P\{\quotep{Q}/y}\} }
  \and \\
  \inferrule* [lab=Par] {{P} \red {P}'} {{{P} | {Q}} \red {{P}' | {Q}}}
  \and
  \inferrule* [lab=Equiv]{{{P} \scong {P}'} \andalso {{P}' \red {Q}'} \andalso {{Q}' \scong {Q}}}{{P} \red {Q}}
\end{mathpar}

\begin{eqnarray*}
  match_{\equiv} (\quotep{P},\quotep{Q}) & := & P \equiv Q \\
  match_{\dagger}(\quotep{P},\quotep{Q}) & := & \forall R. P|Q \red^{*} R => R \red^{*} 0 \\
  match_{K}(\quotep{P},\quotep{Q}) & := & K \mbox{ for some context } K
\end{eqnarray*}

$u?(x)P | u!\langle Q \rangle \red P\{\quotep{Q}/x\}$

%We write $\wred$ for $\red^*$, and $P\red$ if $\exists Q $ such that $ P \red Q$.
We write $P\red$ if $\exists Q $ such that $ P \red Q$ and $P\not\red$, otherwise.

\section{Replication}

As mentioned before, it is known that replication (and hence
recursion) can be implemented in a higher-order process algebra
\cite{SangiorgiWalker}. As our first example of calculation with the
machinery thus far presented we give the construction explicitly in
the {\rhoc}.

\begin{eqnarray}
	D_{x} & := & \prefix{x}{y}{(\binpar{\outputp{x}{y}}{@{y}})} \nonumber\\
	\bangp_{x}{P} & := & \binpar{{x}!\langle{\binpar{D_{x}}{P}}\rangle}{D_{x}} \nonumber
\end{eqnarray}

\begin{eqnarray}
	\bangp_{x}{P} & & \nonumber\\
	=
	& {x}!\langle{(\prefix{x}{y}{(\outputp{x}{y} | @{y})) | P}}\rangle 
	      | \prefix{x}{y}{(\outputp{x}{y} | @{y})} & \nonumber\\
	\red
	& (\outputp{x}{y} | @{y})\substn{\quotep{(\prefix{x}{y}{(@{y} | \outputp{x}{y})) | P}}}{y} & \nonumber\\
	=
	& \outputp{x}{\quotep{(\prefix{x}{y}{(\outputp{x}{y} | @{y})) | P}}}
	  | {(\prefix{x}{y}{(\outputp{x}{y} | @{y})) | P}} & \nonumber\\
	\red
	& \ldots & \nonumber\\
	\red^*
	& P | P | \ldots & \nonumber
\end{eqnarray}

Of course, this encoding, as an implementation, runs away, unfolding
$\bangp{P}$ eagerly. A lazier and more implementable replication
operator, restricted to input-guarded processes, may be obtained as follows.

\begin{eqnarray}
\bangp{\prefix{u}{v}{P}} 
	:= 
	\binpar{\lift{x}{\prefix{u}{v}{(\binpar{D(x)}{P})}}}{D(x)} \nonumber
\end{eqnarray}

\begin{remark}
  Note that the lazier definition still does not deal with summation
  or mixed summation (i.e. sums over input and output). The reader is
  invited to construct definitions of replication that deal with these
  features. 

  Further, the definitions are parameterized in a name, $x$. Can you,
  gentle reader, make a definition that eliminates this parameter and
  guarantees no accidental interaction between the replication
  machinery and the process being replicated -- i.e. no accidental
  sharing of names used by the process to get its work done and the
  name(s) used by the replication to effect copying. This latter
  revision of the definition of replication is crucial to obtaining
  the expected identity $!!P \sim !P$.
\end{remark}

\begin{remark}\label{rem:paradoxical_combinator}
  The reader familiar with the lambda calculus will have noticed the
  similarity between $D$ and the paradoxical combinator.

  [Ed. note: the existence of this seems to suggest we have to be more
  restrictive on the set of processes and names we admit if we are to
  support no-cloning.]
\end{remark}

\subsubsection{Bisimulation}

The computational dynamics gives rise to another kind of equivalence,
the equivalence of computational behavior. As previously mentioned
this is typically captured \emph{via} some form of bisimulation.

% The notion we use in this paper is weak barbed bisimulation
% \cite{milner91polyadicpi}.

The notion we use in this paper is derived from weak barbed
bisimulation \cite{milner91polyadicpi}. 

\begin{definition}
An \emph{observation relation}, $\downarrow_{\mathcal N}$, over a set
of names, $\mathcal N$, is the smallest relation satisfying the rules
below.

\infrule[Out-barb]{y \in {\mathcal N}, \; x \nameeq y}
		  {\outputp{x}{v} \downarrow_{\mathcal N} x}
\infrule[Par-barb]{\mbox{$P\downarrow_{\mathcal N} x$ or $Q\downarrow_{\mathcal N} x$}}
		  {\binpar{P}{Q} \downarrow_{\mathcal N} x}

We write $P \Downarrow_{\mathcal N} x$ if there is $Q$ such that 
$P \wred Q$ and $Q \downarrow_{\mathcal N} x$.
\end{definition}

\begin{definition}
%\label{def.bbisim}
An  ${\mathcal N}$-\emph{barbed bisimulation} over a set of names, ${\mathcal N}$, is a symmetric binary relation 
${\mathcal S}_{\mathcal N}$ between agents such that $P\rel{S}_{\mathcal N}Q$ implies:
\begin{enumerate}
\item If $P \red P'$ then $Q \wred Q'$ and $P'\rel{S}_{\mathcal N} Q'$.
\item If $P\downarrow_{\mathcal N} x$, then $Q\Downarrow_{\mathcal N} x$.
\end{enumerate}
$P$ is ${\mathcal N}$-barbed bisimilar to $Q$, written
$P \wbbisim_{\mathcal N} Q$, if $P \rel{S}_{\mathcal N} Q$ for some ${\mathcal N}$-barbed bisimulation ${\mathcal S}_{\mathcal N}$.
\end{definition}

$\mathcal{R} \subseteq \pi \times \pi$

$P \mathcal{R} Q => \forall P'. P \red P' \Rightarrow \exists Q'. Q \red Q', P' \mathcal{R} Q'$

$P \vdash x \Rightarrow Q \vdash x$

\begin{mathpar}
  \inferrule*[lab=Out-barb]{x \nameeq y}{{y}!\langle{Q}\rangle \vdash x}
  \and
  \inferrule*[lab=Par-barb]{\mbox{$P\vdash x$ or $Q\vdash x$}}{\binpar{P}{Q} \vdash x}
\end{mathpar}

\subsubsection{Contexts}

One of the principle advantages of computational calculi like the
$\pi$-calculus is a well-defined notion of context,
contextual-equivalence and a correlation between
contextual-equivalence and notions of bisimulation. The notion of
context allows the decomposition of a process into (sub-)process and
its syntactic environment, its context. Thus, a context may be
thought of as a process with a ``hole'' (written $\Box$) in it. The
application of a context $M$ to a process $P$, written $M[P]$, is
tantamount to filling the hole in $M$ with $P$. In this paper we do
not need the full weight of this theory, but do make use of the notion
of context in the proof the main theorem. 

\begin{mathpar}
  \inferrule* [lab=summation] {} {{M_{M},M_{N}} \bc \Box \;|\; x.M_{A} \;|\; M_{M}+M_{N}}
  \and
  \inferrule* [lab=agent] {} {{M_{A}} \bc (\vec{x})M_{P} \;| \; \clift{P_0,\ldots,M_{P},\ldots,P_N}}
  \and \\
  \inferrule* [lab=process] {} {{M_{P}} \bc M_{N} \;| \;P|M_{P} }
\end{mathpar} 

\begin{mathpar}
  \inferrule* [lab=sychronization] {} {M_{N} \bc \Box \;|\; x?M_{F} \;|\; x!M_{C}}
  \and
  \inferrule* [lab=abstraction] {} {{M_{F}} \bc (x)M_{P} }
  \and
  \inferrule* [lab=concretion] {} {{M_{C}} \bc \langle M_{P} \rangle }
  \and \\
  \inferrule* [lab=process] {} {{M_{P}} \bc M_{N} \;| \;P|M_{P} }
\end{mathpar}

\begin{definition}[contextual application] Given a context $M$, and
  process $P$, we define the \emph{contextual application}, $M[P] :=
  M\{P/\Box\}$. That is, the contextual application of M to P is the
  substitution of $P$ for $\Box$ in $M$.
\end{definition}

$\meaningof{-} : L \to \mathcal{P}(\pi)$

\begin{mathpar}
  \inferrule* [lab=collection] {} {\meaningof{true} = \pi, \and \meaningof{~E} = \pi \setminus \meaningof{E}, \and \meaningof{E_{1} \& E_{2}} = \meaningof{E_{1}} \cap \meaningof{E_{2}}}
\end{mathpar}

\begin{mathpar}
  \inferrule* [lab=structure] {} {\meaningof{0} = \{ P \in \pi | P \equiv 0 \}, \and \\ \meaningof{E_1 | E_2} = \{ P \in \pi | P \equiv P_{1} | P_{2}, P_{1} \in \meaningof{E_{1}}, P_{2} \in \meaningof{E_2}\} }
\end{mathpar}

\begin{mathpar}
 \inferrule* [lab=behavior] {} {\meaningof{\langle a?b \rangle E} = \{ P \in \pi | P \equiv Q | u?(y)P', \\ \and \\\\ \and \\ \;\;\; u \in \meaningof{a}, \forall z.P'\{z/y\} \in \meaningof{E\{z/b\}}\}, \and \\ \meaningof{a!E} = \{ P \in \pi | P \equiv Q | x!\langle P' \rangle, x \in \meaningof{a} P' \in \meaningof{E}\} }
\end{mathpar}

\begin{mathpar}
 \inferrule* [lab=nominal] {} {\meaningof{\quotep{E}} = \{ \quotep{P} \in \quotep{\pi} | P \in \meaningof{E} \}, \and \meaningof{\quotep{P}} = \{ \quotep{Q} \in \quotep{\pi} | P \equiv Q \} \and \\ \meaningof{@\quotep{E}} = \{ P \in \pi | P \equiv @x, x \in \meaningof{E} \}}
\end{mathpar}

\begin{eqnarray*}
  \\
  \meaningof{-} : TS \to ST
\end{eqnarray*}

\begin{eqnarray*}
  \\
  L : TS \to ST
\end{eqnarray*}

\begin{eqnarray*}
  \\
  P \models E \iff P \in \meaningof{E}
\end{eqnarray*}

\begin{eqnarray*}
  P \approx_{L} Q \iff \forall E \in L. P \models E \iff Q \models E
\end{eqnarray*}

\begin{eqnarray*}
  P \approx_{K} Q
\end{eqnarray*}

\begin{eqnarray*}
  P \approx Q
\end{eqnarray*}

$\approx_{K} = \approx = \approx_{L}$

\subsubsection{Contextual duality}

Note that contexts extend the quotation operation to a family of
operations from processes to names. Given a context, $M$, we can
define a \emph{nominal context}, $\quotep{M}$ by $\quotep{M}[P] :=
\quotep{M[P]}$. To foreshadow what is to come we observe that these
operations enjoy a duality with processes very much like the duality
between vectors and maps from vectors to scalars.

Further, because the calculus is essentially higher-order, we have a
correspondence between contexts and processes. More specifically,
given a name $x$ and a context $M$ we can construct $M^{*}_{x}$ such
that 

\begin{mathpar}
  M^{*}_{x} | \lift{x}{P} \red M[P]
\end{mathpar}

namely,

\begin{mathpar}
  M^{*}_{x} := x?(u).M[\dropn{u}]
\end{mathpar}

The dependence of $M^{*}_{x}$ on a name makes it an abstraction, 

\begin{mathpar}
  M^{*} := (x)x?(u).M[\dropn{u}]
\end{mathpar}

\subsection{Additional notation}

It will sometimes be convenient to denote the process a name
quotes. We already have the notation $x = \quotep{P}$, but it will be
convenient to introduce an alternate notation, $\procn{x}$, when we
want to emphasize the connection to the use of the name. Note that, by
virtue of name equivalence, $\quotep{\procn{x}} \nameeq x$; so, the
notation is consistent with previous definitions.

Further, because names have structure it is possible to effect
substitutions on the basis of that structure. This means we need to
upgrade our notation for substitutions, which we accomplish by
adapting comprehension notation. Thus,

\begin{mathpar}
  P\{ y / x : x \in S \}
\end{mathpar}

is interpreted to mean the process derived from P by replacing (in a
capture-avoiding manner) each occurrence of $x$ in $S$ by $y$. For example,

\begin{mathpar}
  P\{ \quotep{\procn{x}|\procn{x}} / x : x \in \freenames{P} \}
\end{mathpar}

will replace each (occurrence) of a free name $x$ in $P$ by
$\quotep{\procn{x}|\procn{x}}$.

Also, we will avail ourselves of the notation $x^{L}$ and $x^{R}$ to
denote injections of a name into disjoint copies of the name
space. There are numerous ways to accomplish this. One example can be
found in \cite{MeredithR05}. This notation overloads to vectors of
names: $\vec{x}^{\pi} := (x_{i}^{\pi} \; : \; 0 \leq i < |\vec{x}| )$ where $\pi \in \{L,R\}$.

We also use $P^{\Box} := P|\Box$.

In \cite{MeredithR05} an interpretation of the new operator is
given. It turns out that there are several possible interpretations
all enjoying the requisite algebraic properties of the operator (see
\cite{milner91polyadicpi}). We will therefore make liberal use of
$(\nu\; \vec{x})P$.

% subsection the_syntax_and_semantics_of_the_notation_system (end)   

\input{qm2pi.qmops} 

\input{qm2pi.sterngerlach} 

\input{qm2pi.metric} 

% section concurrent_process_calculi (end)

%\input{qm2pi.proofsketch}

% section proof sketch (end)

%\input{qm2pi.slviaknots} 

% section spatial logic via knots (end)

\input{qm2pi.conclusion}

% section conclusion (end)

%\input{qm2pi.dtcodes} 

% section wiring algorithm (end)

\input{qm2pi.ack} 

% section acknowledgments (end)

\newpage


\bibliographystyle{plain}   
\bibliography{../../biblios/main.bib}

\input{qm2pi.rhodetails}

\end{document}

 

% section acknowledgments (end)

\newpage


\bibliographystyle{plain}   
\bibliography{../../biblios/main.bib}

\documentclass[12pt]{llncs}
%\documentclass{jktr}

\usepackage[pdftex]{hyperref}                   
\usepackage {listings}
\usepackage {mathpartir}
\usepackage{bcprules}
%\usepackage{listings}
                       
\usepackage{graphicx} 
%\usepackage[margins=2.5cm,nohead,nofoot]{geometry}
%\usepackage{geometry}
\usepackage{amsfonts}
\usepackage{amstext}
\usepackage{latexsym}
\usepackage{amssymb}
\usepackage{color}


%\include{myPreamble}
\include{qm2pi.local} 

%\ifpdf
%\usepackage[pdftex]{graphicx}
%\else
%\usepackage{graphicx}
%\fi

 % \ifpdf
%  \usepackage{pdfsync}
%  \if


%\title{Brief Article}
%\author{David F. Snyder}
%\author{L.G. Meredith}

%\address{Dept. of Math., Texas State University--San Marcos, San Marcos, TX 78666}
       
\pagestyle{empty}


\begin{document}

\lstset{language=[Objective]Caml,frame=shadowbox}

\input{qm2pi.front}

% section front matter (end)

\input{qm2pi.intro} 
 
% section introduction (end)

% \input{qm2pi.knotations} 

% section notation (end)

\input{qm2pi.process.calculi} 

% section concurrent_process_calculi_and_spatial_logics_ (end)
    
%\input{qm2pi.knots2pi} 

%\input{qm2pi.trefoil} 

%\input{qm2pi.mainthm} 

% subsection basic_interpretation (end)

%\input{qm2pi.rho.presentation} 
\subsection{The syntax and semantics of the notation system}\label{sub:the_syntax_and_semantics_of_the_notation_system} % (fold)

We now summarize a technical presentation of the calculus that
embodies our theory of dynamics. The typical presentation of such a
calculus follows the style of giving generators and relations on
them. The grammar, below, describing term constructors, freely
generates the set of processes, $\Proc$. This set is then quotiented
by a relation known as structural congruence and it is over this set
that the notion of dynamics is expressed. This presentation is
essentially that of \cite{MeredithR05} with the addition of
polyadicity and summation. For readability we have relegated some of
the technical subtleties to an appendix.

\subsubsection{Process grammar}\label{subsub:process_grammar}

\begin{mathpar}
  \inferrule* [lab=synchronization] {} {{M} \bc \pzero \;|\; x?F \;|\; x!C }
  \and
  \inferrule* [lab=abstraction] {} {{F} \bc (x)P}
  \and
  \inferrule* [lab=concretion] {} {{C} \bc \langle Q \rangle}
  \and
  \inferrule* [lab=process] {} {{P,Q} \bc M \;| \;P|Q \;|\; @{x}}
  \and
  \inferrule* [lab=name] {} {{x} \bc \quotep{P}}
\end{mathpar} 

Note that $\vec{x}$ (resp. $\vec{P}$) denotes a vector of names
(resp. processes) of length $|\vec{x}|$ (resp. $|\vec{P}|$). We adopt
the following useful abbreviations.

\begin{mathpar}
   x?(\vec{y}).P := x.(\vec{y})P \and  x\clift{\vec{P}} := x.\clift{\vec{P}}
   \and x!(y) := \lift{x}{\dropn{y}}
   \and \Pi_{i=0}^{n-1}P_i := P_0 | \ldots | P_{n-1}
\end{mathpar}

\subsubsection{Structural congruence}

\paragraph{Free and bound names and alpha-equivalence.} At the
core of structural equivalence is alpha-equivalence which identifies
process that are the same up to a change of variable. Formally, we
recognize the distinction between free and bound names. The free names
of a process, $\freenames{P}$, may be calculated recursively as
follows:

\begin{mathpar}
\freenames{\pzero} := \emptyset
  \and \\
  \freenames{x?(y).P} := \{ x \} \cup (\freenames{P} \setminus \{ y \})
  \and 
  \freenames{x!\langle P \rangle} := \{ x \} \cup \{ P \} 
  \and \\
  \freenames{P|Q} := \freenames{P} \cup \freenames{Q}
  \and \\
  \freenames{@{x}} := \{ x \}
\end{mathpar}

$\pi$
$\quotep{\pi}$

$\freenames{-} : \pi \to \mathcal{P}(\quotep{\pi})$

\begin{eqnarray*}
  \freenames{\pzero} & := & \emptyset \\
  \freenames{x?(y).P} & := & \{ x \} \cup (\freenames{P} \setminus \{ y \}) \\
  \freenames{x!\langle P \rangle} & := & \{ x \} \cup \{ P \} \\
  \freenames{P|Q} & := & \freenames{P} \cup \freenames{Q} \\
  \freenames{\dropn{x}} & := & \{ x \}
\end{eqnarray*}

The bound names of a process, $\boundnames{P}$, are those names occurring in $P$
that are not free. For example, in $x?(y).0$, the name $x$ is free, while $y$ is bound.

\begin{mathpar}
  \inferrule* [lab=monoidal-laws] {} { P|Q \equiv Q|P \and P|0 \equiv P \and P|(Q|R) \equiv (P|Q)|R }
\end{mathpar}

\begin{mathpar}
  \inferrule* [lab=alpha-equivalence] {} { (x)P \equiv (y)P\{y/x\} \and y \not\in \freenames{P} }
\end{mathpar}

\begin{definition}
Then two processes, $P,Q$, are alpha-equivalent if $P = Q\{\vec{y}/\vec{x}\}$ for
some $\vec{x} \in \boundnames{Q},\vec{y} \in \boundnames{P}$, where $Q\{\vec{y}/\vec{x}\}$
denotes the capture-avoiding substitution of $\vec{y}$ for $\vec{x}$ in $Q$.
\end{definition}

\begin{definition}
  The {\em structural congruence} \cite{SangiorgiWalker} , $\equiv$,
  between processes is the least congruence containing
  alpha-equivalence, satisfying the abelian monoid laws
  (associativity, commutativity and $\pzero$ as identity) for parallel
  composition $|$ and for summation $+$.
\end{definition}

\subsection{Name equivalence}

We take name equivalence, written $\nameeq$, to be the smallest
equivalence relation generated by the following rules.

\begin{mathpar}
\inferrule*[lab=Quote-drop]
{ }
{ \quotep{@{x}} \nameeq x }

\inferrule*[lab=Struct-equiv]
{ P \scong Q }
{ \quotep{P} \nameeq \quotep{Q} }
\end{mathpar}

The astute reader will have noticed that the mutual recursion of names
and processes imposes a mutual recursion on alpha-equivalence and
structural equivalence via name-equivalence. Fortunately, all of this
works out pleasantly and we may calculate in the natural way, free of
concern. The reader interested in the details is referred to the
appendix \ref{appendix:rho_details}.

\subsection{Substitution}

We use $\Proc$ for the set of processes, $\QProc$ for the set of
names, and $\id{\{}\vec{y} / \vec{x} \id{\}}$ to denote partial maps,
$s : \QProc \rightarrow \QProc$. A map, $s$ lifts, uniquely, to a map
on process terms, $\widehat{s} : \Proc \rightarrow \Proc$ by the
following equations.

\begin{mathpar}
  (0) \psubstp{Q}{P} := 0 \\
  (R \juxtap S) \psubstp{Q}{P}
  :=    
  (R)\psubstp{Q}{P} \juxtap (S) \psubstp{Q}{P} \\
  (x?(y).R) \psubstp{Q}{P}    
  :=    
  (x)\substp{Q}{P} (z)\concat( (R \psubstn{z}{y}) \psubstp{Q}{P} ) \\
  (\lift{x}{R}) \psubstp{Q}{P}  
  :=
  \lift{(x)\substp{Q}{P}}{ R \psubstp{Q}{P} } \\
%   (\dropn{x})  \psubstp{Q}{P}       
%   := 
%   \left\{ 
%     \begin{array}{ccc} 
%       \dropn{\quotep{Q}} & & x \nameeq \quotep{P} \\
%       \dropn{x} & & otherwise \\
%     \end{array}
%   \right. 
  (\dropn{x})  \psubstp{Q}{P}       
  := 
  \left\{ 
    \begin{array}{ccc} 
      Q & & x \nameeq \quotep{P} \\
      \dropn{x} & & otherwise \\
    \end{array}
  \right.
\end{mathpar}
 

where

\begin{eqnarray}
  (x)\id{\{} \lpquote Q \rpquote / \lpquote P \rpquote \id{\}}            = 
  \left\{ 
    \begin{array}{ccc}
      \lpquote Q \rpquote & & x \nameeq \lpquote P \rpquote \\
      x & & otherwise \\
    \end{array}
  \right. \nonumber
\end{eqnarray}

and $z$ is chosen distinct from $\quotep{P}$, $\quotep{Q}$, the free
names in $Q$, and all the names in $R$. Our $\alpha$-equivalence will
be built in the standard way from this substitution.

\begin{remark}\label{rem:no_self_referential_names}
  One consequence of these definitions is that $\forall P. \quotep{P}
  \not\in \freenames{P}$.
\end{remark}

\subsection{ Dynamic quote: an example }

Anticipating something of what's to come, consider applying the
substitution, $\widehat{\id{\{}u / z \id{\}}}$, to the following pair
of processes, $\lift{w}{y!(z)}$ and $w[ \lpquote y!(z) \rpquote ]$.

\begin{eqnarray}
	\lift{w}{y!(z)}\widehat{\id{\{}u / z \id{\}}}
		& = &
		\lift{w}{y!(u)} \nonumber\\
	w[ \lpquote y!(z) \rpquote ] \widehat{ \id{\{}u / z \id{\}} }
		& = &
		w[ \lpquote y!(z) \rpquote ] \nonumber
\end{eqnarray}

Because the body of the process between quotes is impervious to
substitution, we get radically different answers. In fact, by
examining the first process in an input context,
e.g. $x?(z).\lift{w}{y!(z)}$, we see that the process under the lift
operator may be shaped by prefixed inputs binding a name inside it. In
this sense, the lift operator will be seen as a way to dynamically
construct processes before reifying them as names.

Finally equipped with these standard features we can present the
dynamics of the calculus.

\subsubsection{Operational semantics} 

Finally, we introduce the computational dynamics. What marks these
algebras as distinct from other more traditionally studied algebraic
structures, e.g. vector spaces or polynomial rings, is the manner in
which dynamics is captured. In traditional structures, dynamics is typically
expressed through morphisms between such structures, as in linear maps
between vector spaces or morphisms between rings. In algebras
associated with the semantics of computation, the dynamics is
expressed as part of the algebraic structure itself, through a
reduction reduction relation typically denoted by $\red$. Below, we
give a recursive presentation of this relation for the calculus used
in the encoding.

$\red \subseteq \pi \times \pi$
$\red : \pi \to \mathcal{P}(\pi)$

\begin{mathpar}
  \inferrule* [lab=Comm] { \textsf{match}( x_{src}, x_{trgt} ) } { x_{trgt}?(y)P \; | \; x_{src}!\langle {Q} \rangle \red P\{\quotep{Q}/y}\} }
  \and \\
  \inferrule* [lab=Par] {{P} \red {P}'} {{{P} | {Q}} \red {{P}' | {Q}}}
  \and
  \inferrule* [lab=Equiv]{{{P} \scong {P}'} \andalso {{P}' \red {Q}'} \andalso {{Q}' \scong {Q}}}{{P} \red {Q}}
\end{mathpar}

\begin{eqnarray*}
  match_{\equiv} (\quotep{P},\quotep{Q}) & := & P \equiv Q \\
  match_{\dagger}(\quotep{P},\quotep{Q}) & := & \forall R. P|Q \red^{*} R => R \red^{*} 0 \\
  match_{K}(\quotep{P},\quotep{Q}) & := & K \mbox{ for some context } K
\end{eqnarray*}

$u?(x)P | u!\langle Q \rangle \red P\{\quotep{Q}/x\}$

%We write $\wred$ for $\red^*$, and $P\red$ if $\exists Q $ such that $ P \red Q$.
We write $P\red$ if $\exists Q $ such that $ P \red Q$ and $P\not\red$, otherwise.

\section{Replication}

As mentioned before, it is known that replication (and hence
recursion) can be implemented in a higher-order process algebra
\cite{SangiorgiWalker}. As our first example of calculation with the
machinery thus far presented we give the construction explicitly in
the {\rhoc}.

\begin{eqnarray}
	D_{x} & := & \prefix{x}{y}{(\binpar{\outputp{x}{y}}{@{y}})} \nonumber\\
	\bangp_{x}{P} & := & \binpar{{x}!\langle{\binpar{D_{x}}{P}}\rangle}{D_{x}} \nonumber
\end{eqnarray}

\begin{eqnarray}
	\bangp_{x}{P} & & \nonumber\\
	=
	& {x}!\langle{(\prefix{x}{y}{(\outputp{x}{y} | @{y})) | P}}\rangle 
	      | \prefix{x}{y}{(\outputp{x}{y} | @{y})} & \nonumber\\
	\red
	& (\outputp{x}{y} | @{y})\substn{\quotep{(\prefix{x}{y}{(@{y} | \outputp{x}{y})) | P}}}{y} & \nonumber\\
	=
	& \outputp{x}{\quotep{(\prefix{x}{y}{(\outputp{x}{y} | @{y})) | P}}}
	  | {(\prefix{x}{y}{(\outputp{x}{y} | @{y})) | P}} & \nonumber\\
	\red
	& \ldots & \nonumber\\
	\red^*
	& P | P | \ldots & \nonumber
\end{eqnarray}

Of course, this encoding, as an implementation, runs away, unfolding
$\bangp{P}$ eagerly. A lazier and more implementable replication
operator, restricted to input-guarded processes, may be obtained as follows.

\begin{eqnarray}
\bangp{\prefix{u}{v}{P}} 
	:= 
	\binpar{\lift{x}{\prefix{u}{v}{(\binpar{D(x)}{P})}}}{D(x)} \nonumber
\end{eqnarray}

\begin{remark}
  Note that the lazier definition still does not deal with summation
  or mixed summation (i.e. sums over input and output). The reader is
  invited to construct definitions of replication that deal with these
  features. 

  Further, the definitions are parameterized in a name, $x$. Can you,
  gentle reader, make a definition that eliminates this parameter and
  guarantees no accidental interaction between the replication
  machinery and the process being replicated -- i.e. no accidental
  sharing of names used by the process to get its work done and the
  name(s) used by the replication to effect copying. This latter
  revision of the definition of replication is crucial to obtaining
  the expected identity $!!P \sim !P$.
\end{remark}

\begin{remark}\label{rem:paradoxical_combinator}
  The reader familiar with the lambda calculus will have noticed the
  similarity between $D$ and the paradoxical combinator.

  [Ed. note: the existence of this seems to suggest we have to be more
  restrictive on the set of processes and names we admit if we are to
  support no-cloning.]
\end{remark}

\subsubsection{Bisimulation}

The computational dynamics gives rise to another kind of equivalence,
the equivalence of computational behavior. As previously mentioned
this is typically captured \emph{via} some form of bisimulation.

% The notion we use in this paper is weak barbed bisimulation
% \cite{milner91polyadicpi}.

The notion we use in this paper is derived from weak barbed
bisimulation \cite{milner91polyadicpi}. 

\begin{definition}
An \emph{observation relation}, $\downarrow_{\mathcal N}$, over a set
of names, $\mathcal N$, is the smallest relation satisfying the rules
below.

\infrule[Out-barb]{y \in {\mathcal N}, \; x \nameeq y}
		  {\outputp{x}{v} \downarrow_{\mathcal N} x}
\infrule[Par-barb]{\mbox{$P\downarrow_{\mathcal N} x$ or $Q\downarrow_{\mathcal N} x$}}
		  {\binpar{P}{Q} \downarrow_{\mathcal N} x}

We write $P \Downarrow_{\mathcal N} x$ if there is $Q$ such that 
$P \wred Q$ and $Q \downarrow_{\mathcal N} x$.
\end{definition}

\begin{definition}
%\label{def.bbisim}
An  ${\mathcal N}$-\emph{barbed bisimulation} over a set of names, ${\mathcal N}$, is a symmetric binary relation 
${\mathcal S}_{\mathcal N}$ between agents such that $P\rel{S}_{\mathcal N}Q$ implies:
\begin{enumerate}
\item If $P \red P'$ then $Q \wred Q'$ and $P'\rel{S}_{\mathcal N} Q'$.
\item If $P\downarrow_{\mathcal N} x$, then $Q\Downarrow_{\mathcal N} x$.
\end{enumerate}
$P$ is ${\mathcal N}$-barbed bisimilar to $Q$, written
$P \wbbisim_{\mathcal N} Q$, if $P \rel{S}_{\mathcal N} Q$ for some ${\mathcal N}$-barbed bisimulation ${\mathcal S}_{\mathcal N}$.
\end{definition}

$\mathcal{R} \subseteq \pi \times \pi$

$P \mathcal{R} Q => \forall P'. P \red P' \Rightarrow \exists Q'. Q \red Q', P' \mathcal{R} Q'$

$P \vdash x \Rightarrow Q \vdash x$

\begin{mathpar}
  \inferrule*[lab=Out-barb]{x \nameeq y}{{y}!\langle{Q}\rangle \vdash x}
  \and
  \inferrule*[lab=Par-barb]{\mbox{$P\vdash x$ or $Q\vdash x$}}{\binpar{P}{Q} \vdash x}
\end{mathpar}

\subsubsection{Contexts}

One of the principle advantages of computational calculi like the
$\pi$-calculus is a well-defined notion of context,
contextual-equivalence and a correlation between
contextual-equivalence and notions of bisimulation. The notion of
context allows the decomposition of a process into (sub-)process and
its syntactic environment, its context. Thus, a context may be
thought of as a process with a ``hole'' (written $\Box$) in it. The
application of a context $M$ to a process $P$, written $M[P]$, is
tantamount to filling the hole in $M$ with $P$. In this paper we do
not need the full weight of this theory, but do make use of the notion
of context in the proof the main theorem. 

\begin{mathpar}
  \inferrule* [lab=summation] {} {{M_{M},M_{N}} \bc \Box \;|\; x.M_{A} \;|\; M_{M}+M_{N}}
  \and
  \inferrule* [lab=agent] {} {{M_{A}} \bc (\vec{x})M_{P} \;| \; \clift{P_0,\ldots,M_{P},\ldots,P_N}}
  \and \\
  \inferrule* [lab=process] {} {{M_{P}} \bc M_{N} \;| \;P|M_{P} }
\end{mathpar} 

\begin{mathpar}
  \inferrule* [lab=sychronization] {} {M_{N} \bc \Box \;|\; x?M_{F} \;|\; x!M_{C}}
  \and
  \inferrule* [lab=abstraction] {} {{M_{F}} \bc (x)M_{P} }
  \and
  \inferrule* [lab=concretion] {} {{M_{C}} \bc \langle M_{P} \rangle }
  \and \\
  \inferrule* [lab=process] {} {{M_{P}} \bc M_{N} \;| \;P|M_{P} }
\end{mathpar}

\begin{definition}[contextual application] Given a context $M$, and
  process $P$, we define the \emph{contextual application}, $M[P] :=
  M\{P/\Box\}$. That is, the contextual application of M to P is the
  substitution of $P$ for $\Box$ in $M$.
\end{definition}

$\meaningof{-} : L \to \mathcal{P}(\pi)$

\begin{mathpar}
  \inferrule* [lab=collection] {} {\meaningof{true} = \pi, \and \meaningof{~E} = \pi \setminus \meaningof{E}, \and \meaningof{E_{1} \& E_{2}} = \meaningof{E_{1}} \cap \meaningof{E_{2}}}
\end{mathpar}

\begin{mathpar}
  \inferrule* [lab=structure] {} {\meaningof{0} = \{ P \in \pi | P \equiv 0 \}, \and \\ \meaningof{E_1 | E_2} = \{ P \in \pi | P \equiv P_{1} | P_{2}, P_{1} \in \meaningof{E_{1}}, P_{2} \in \meaningof{E_2}\} }
\end{mathpar}

\begin{mathpar}
 \inferrule* [lab=behavior] {} {\meaningof{\langle a?b \rangle E} = \{ P \in \pi | P \equiv Q | u?(y)P', \\ \and \\\\ \and \\ \;\;\; u \in \meaningof{a}, \forall z.P'\{z/y\} \in \meaningof{E\{z/b\}}\}, \and \\ \meaningof{a!E} = \{ P \in \pi | P \equiv Q | x!\langle P' \rangle, x \in \meaningof{a} P' \in \meaningof{E}\} }
\end{mathpar}

\begin{mathpar}
 \inferrule* [lab=nominal] {} {\meaningof{\quotep{E}} = \{ \quotep{P} \in \quotep{\pi} | P \in \meaningof{E} \}, \and \meaningof{\quotep{P}} = \{ \quotep{Q} \in \quotep{\pi} | P \equiv Q \} \and \\ \meaningof{@\quotep{E}} = \{ P \in \pi | P \equiv @x, x \in \meaningof{E} \}}
\end{mathpar}

\begin{eqnarray*}
  \\
  \meaningof{-} : TS \to ST
\end{eqnarray*}

\begin{eqnarray*}
  \\
  L : TS \to ST
\end{eqnarray*}

\begin{eqnarray*}
  \\
  P \models E \iff P \in \meaningof{E}
\end{eqnarray*}

\begin{eqnarray*}
  P \approx_{L} Q \iff \forall E \in L. P \models E \iff Q \models E
\end{eqnarray*}

\begin{eqnarray*}
  P \approx_{K} Q
\end{eqnarray*}

\begin{eqnarray*}
  P \approx Q
\end{eqnarray*}

$\approx_{K} = \approx = \approx_{L}$

\subsubsection{Contextual duality}

Note that contexts extend the quotation operation to a family of
operations from processes to names. Given a context, $M$, we can
define a \emph{nominal context}, $\quotep{M}$ by $\quotep{M}[P] :=
\quotep{M[P]}$. To foreshadow what is to come we observe that these
operations enjoy a duality with processes very much like the duality
between vectors and maps from vectors to scalars.

Further, because the calculus is essentially higher-order, we have a
correspondence between contexts and processes. More specifically,
given a name $x$ and a context $M$ we can construct $M^{*}_{x}$ such
that 

\begin{mathpar}
  M^{*}_{x} | \lift{x}{P} \red M[P]
\end{mathpar}

namely,

\begin{mathpar}
  M^{*}_{x} := x?(u).M[\dropn{u}]
\end{mathpar}

The dependence of $M^{*}_{x}$ on a name makes it an abstraction, 

\begin{mathpar}
  M^{*} := (x)x?(u).M[\dropn{u}]
\end{mathpar}

\subsection{Additional notation}

It will sometimes be convenient to denote the process a name
quotes. We already have the notation $x = \quotep{P}$, but it will be
convenient to introduce an alternate notation, $\procn{x}$, when we
want to emphasize the connection to the use of the name. Note that, by
virtue of name equivalence, $\quotep{\procn{x}} \nameeq x$; so, the
notation is consistent with previous definitions.

Further, because names have structure it is possible to effect
substitutions on the basis of that structure. This means we need to
upgrade our notation for substitutions, which we accomplish by
adapting comprehension notation. Thus,

\begin{mathpar}
  P\{ y / x : x \in S \}
\end{mathpar}

is interpreted to mean the process derived from P by replacing (in a
capture-avoiding manner) each occurrence of $x$ in $S$ by $y$. For example,

\begin{mathpar}
  P\{ \quotep{\procn{x}|\procn{x}} / x : x \in \freenames{P} \}
\end{mathpar}

will replace each (occurrence) of a free name $x$ in $P$ by
$\quotep{\procn{x}|\procn{x}}$.

Also, we will avail ourselves of the notation $x^{L}$ and $x^{R}$ to
denote injections of a name into disjoint copies of the name
space. There are numerous ways to accomplish this. One example can be
found in \cite{MeredithR05}. This notation overloads to vectors of
names: $\vec{x}^{\pi} := (x_{i}^{\pi} \; : \; 0 \leq i < |\vec{x}| )$ where $\pi \in \{L,R\}$.

We also use $P^{\Box} := P|\Box$.

In \cite{MeredithR05} an interpretation of the new operator is
given. It turns out that there are several possible interpretations
all enjoying the requisite algebraic properties of the operator (see
\cite{milner91polyadicpi}). We will therefore make liberal use of
$(\nu\; \vec{x})P$.

% subsection the_syntax_and_semantics_of_the_notation_system (end)   

\input{qm2pi.qmops} 

\input{qm2pi.sterngerlach} 

\input{qm2pi.metric} 

% section concurrent_process_calculi (end)

%\input{qm2pi.proofsketch}

% section proof sketch (end)

%\input{qm2pi.slviaknots} 

% section spatial logic via knots (end)

\input{qm2pi.conclusion}

% section conclusion (end)

%\input{qm2pi.dtcodes} 

% section wiring algorithm (end)

\input{qm2pi.ack} 

% section acknowledgments (end)

\newpage


\bibliographystyle{plain}   
\bibliography{../../biblios/main.bib}

\input{qm2pi.rhodetails}

\end{document}



\end{document}

 

%\ifpdf
%\usepackage[pdftex]{graphicx}
%\else
%\usepackage{graphicx}
%\fi

 % \ifpdf
%  \usepackage{pdfsync}
%  \if


%\title{Brief Article}
%\author{David F. Snyder}
%\author{L.G. Meredith}

%\address{Dept. of Math., Texas State University--San Marcos, San Marcos, TX 78666}
       
\pagestyle{empty}


\begin{document}

\lstset{language=[Objective]Caml,frame=shadowbox}

\documentclass[12pt]{llncs}
%\documentclass{jktr}

\usepackage[pdftex]{hyperref}                   
\usepackage {listings}
\usepackage {mathpartir}
\usepackage{bcprules}
%\usepackage{listings}
                       
\usepackage{graphicx} 
%\usepackage[margins=2.5cm,nohead,nofoot]{geometry}
%\usepackage{geometry}
\usepackage{amsfonts}
\usepackage{amstext}
\usepackage{latexsym}
\usepackage{amssymb}
\usepackage{color}


%\include{myPreamble}
\documentclass[12pt]{llncs}
%\documentclass{jktr}

\usepackage[pdftex]{hyperref}                   
\usepackage {listings}
\usepackage {mathpartir}
\usepackage{bcprules}
%\usepackage{listings}
                       
\usepackage{graphicx} 
%\usepackage[margins=2.5cm,nohead,nofoot]{geometry}
%\usepackage{geometry}
\usepackage{amsfonts}
\usepackage{amstext}
\usepackage{latexsym}
\usepackage{amssymb}
\usepackage{color}


%\include{myPreamble}
\include{qm2pi.local} 

%\ifpdf
%\usepackage[pdftex]{graphicx}
%\else
%\usepackage{graphicx}
%\fi

 % \ifpdf
%  \usepackage{pdfsync}
%  \if


%\title{Brief Article}
%\author{David F. Snyder}
%\author{L.G. Meredith}

%\address{Dept. of Math., Texas State University--San Marcos, San Marcos, TX 78666}
       
\pagestyle{empty}


\begin{document}

\lstset{language=[Objective]Caml,frame=shadowbox}

\input{qm2pi.front}

% section front matter (end)

\input{qm2pi.intro} 
 
% section introduction (end)

% \input{qm2pi.knotations} 

% section notation (end)

\input{qm2pi.process.calculi} 

% section concurrent_process_calculi_and_spatial_logics_ (end)
    
%\input{qm2pi.knots2pi} 

%\input{qm2pi.trefoil} 

%\input{qm2pi.mainthm} 

% subsection basic_interpretation (end)

%\input{qm2pi.rho.presentation} 
\subsection{The syntax and semantics of the notation system}\label{sub:the_syntax_and_semantics_of_the_notation_system} % (fold)

We now summarize a technical presentation of the calculus that
embodies our theory of dynamics. The typical presentation of such a
calculus follows the style of giving generators and relations on
them. The grammar, below, describing term constructors, freely
generates the set of processes, $\Proc$. This set is then quotiented
by a relation known as structural congruence and it is over this set
that the notion of dynamics is expressed. This presentation is
essentially that of \cite{MeredithR05} with the addition of
polyadicity and summation. For readability we have relegated some of
the technical subtleties to an appendix.

\subsubsection{Process grammar}\label{subsub:process_grammar}

\begin{mathpar}
  \inferrule* [lab=synchronization] {} {{M} \bc \pzero \;|\; x?F \;|\; x!C }
  \and
  \inferrule* [lab=abstraction] {} {{F} \bc (x)P}
  \and
  \inferrule* [lab=concretion] {} {{C} \bc \langle Q \rangle}
  \and
  \inferrule* [lab=process] {} {{P,Q} \bc M \;| \;P|Q \;|\; @{x}}
  \and
  \inferrule* [lab=name] {} {{x} \bc \quotep{P}}
\end{mathpar} 

Note that $\vec{x}$ (resp. $\vec{P}$) denotes a vector of names
(resp. processes) of length $|\vec{x}|$ (resp. $|\vec{P}|$). We adopt
the following useful abbreviations.

\begin{mathpar}
   x?(\vec{y}).P := x.(\vec{y})P \and  x\clift{\vec{P}} := x.\clift{\vec{P}}
   \and x!(y) := \lift{x}{\dropn{y}}
   \and \Pi_{i=0}^{n-1}P_i := P_0 | \ldots | P_{n-1}
\end{mathpar}

\subsubsection{Structural congruence}

\paragraph{Free and bound names and alpha-equivalence.} At the
core of structural equivalence is alpha-equivalence which identifies
process that are the same up to a change of variable. Formally, we
recognize the distinction between free and bound names. The free names
of a process, $\freenames{P}$, may be calculated recursively as
follows:

\begin{mathpar}
\freenames{\pzero} := \emptyset
  \and \\
  \freenames{x?(y).P} := \{ x \} \cup (\freenames{P} \setminus \{ y \})
  \and 
  \freenames{x!\langle P \rangle} := \{ x \} \cup \{ P \} 
  \and \\
  \freenames{P|Q} := \freenames{P} \cup \freenames{Q}
  \and \\
  \freenames{@{x}} := \{ x \}
\end{mathpar}

$\pi$
$\quotep{\pi}$

$\freenames{-} : \pi \to \mathcal{P}(\quotep{\pi})$

\begin{eqnarray*}
  \freenames{\pzero} & := & \emptyset \\
  \freenames{x?(y).P} & := & \{ x \} \cup (\freenames{P} \setminus \{ y \}) \\
  \freenames{x!\langle P \rangle} & := & \{ x \} \cup \{ P \} \\
  \freenames{P|Q} & := & \freenames{P} \cup \freenames{Q} \\
  \freenames{\dropn{x}} & := & \{ x \}
\end{eqnarray*}

The bound names of a process, $\boundnames{P}$, are those names occurring in $P$
that are not free. For example, in $x?(y).0$, the name $x$ is free, while $y$ is bound.

\begin{mathpar}
  \inferrule* [lab=monoidal-laws] {} { P|Q \equiv Q|P \and P|0 \equiv P \and P|(Q|R) \equiv (P|Q)|R }
\end{mathpar}

\begin{mathpar}
  \inferrule* [lab=alpha-equivalence] {} { (x)P \equiv (y)P\{y/x\} \and y \not\in \freenames{P} }
\end{mathpar}

\begin{definition}
Then two processes, $P,Q$, are alpha-equivalent if $P = Q\{\vec{y}/\vec{x}\}$ for
some $\vec{x} \in \boundnames{Q},\vec{y} \in \boundnames{P}$, where $Q\{\vec{y}/\vec{x}\}$
denotes the capture-avoiding substitution of $\vec{y}$ for $\vec{x}$ in $Q$.
\end{definition}

\begin{definition}
  The {\em structural congruence} \cite{SangiorgiWalker} , $\equiv$,
  between processes is the least congruence containing
  alpha-equivalence, satisfying the abelian monoid laws
  (associativity, commutativity and $\pzero$ as identity) for parallel
  composition $|$ and for summation $+$.
\end{definition}

\subsection{Name equivalence}

We take name equivalence, written $\nameeq$, to be the smallest
equivalence relation generated by the following rules.

\begin{mathpar}
\inferrule*[lab=Quote-drop]
{ }
{ \quotep{@{x}} \nameeq x }

\inferrule*[lab=Struct-equiv]
{ P \scong Q }
{ \quotep{P} \nameeq \quotep{Q} }
\end{mathpar}

The astute reader will have noticed that the mutual recursion of names
and processes imposes a mutual recursion on alpha-equivalence and
structural equivalence via name-equivalence. Fortunately, all of this
works out pleasantly and we may calculate in the natural way, free of
concern. The reader interested in the details is referred to the
appendix \ref{appendix:rho_details}.

\subsection{Substitution}

We use $\Proc$ for the set of processes, $\QProc$ for the set of
names, and $\id{\{}\vec{y} / \vec{x} \id{\}}$ to denote partial maps,
$s : \QProc \rightarrow \QProc$. A map, $s$ lifts, uniquely, to a map
on process terms, $\widehat{s} : \Proc \rightarrow \Proc$ by the
following equations.

\begin{mathpar}
  (0) \psubstp{Q}{P} := 0 \\
  (R \juxtap S) \psubstp{Q}{P}
  :=    
  (R)\psubstp{Q}{P} \juxtap (S) \psubstp{Q}{P} \\
  (x?(y).R) \psubstp{Q}{P}    
  :=    
  (x)\substp{Q}{P} (z)\concat( (R \psubstn{z}{y}) \psubstp{Q}{P} ) \\
  (\lift{x}{R}) \psubstp{Q}{P}  
  :=
  \lift{(x)\substp{Q}{P}}{ R \psubstp{Q}{P} } \\
%   (\dropn{x})  \psubstp{Q}{P}       
%   := 
%   \left\{ 
%     \begin{array}{ccc} 
%       \dropn{\quotep{Q}} & & x \nameeq \quotep{P} \\
%       \dropn{x} & & otherwise \\
%     \end{array}
%   \right. 
  (\dropn{x})  \psubstp{Q}{P}       
  := 
  \left\{ 
    \begin{array}{ccc} 
      Q & & x \nameeq \quotep{P} \\
      \dropn{x} & & otherwise \\
    \end{array}
  \right.
\end{mathpar}
 

where

\begin{eqnarray}
  (x)\id{\{} \lpquote Q \rpquote / \lpquote P \rpquote \id{\}}            = 
  \left\{ 
    \begin{array}{ccc}
      \lpquote Q \rpquote & & x \nameeq \lpquote P \rpquote \\
      x & & otherwise \\
    \end{array}
  \right. \nonumber
\end{eqnarray}

and $z$ is chosen distinct from $\quotep{P}$, $\quotep{Q}$, the free
names in $Q$, and all the names in $R$. Our $\alpha$-equivalence will
be built in the standard way from this substitution.

\begin{remark}\label{rem:no_self_referential_names}
  One consequence of these definitions is that $\forall P. \quotep{P}
  \not\in \freenames{P}$.
\end{remark}

\subsection{ Dynamic quote: an example }

Anticipating something of what's to come, consider applying the
substitution, $\widehat{\id{\{}u / z \id{\}}}$, to the following pair
of processes, $\lift{w}{y!(z)}$ and $w[ \lpquote y!(z) \rpquote ]$.

\begin{eqnarray}
	\lift{w}{y!(z)}\widehat{\id{\{}u / z \id{\}}}
		& = &
		\lift{w}{y!(u)} \nonumber\\
	w[ \lpquote y!(z) \rpquote ] \widehat{ \id{\{}u / z \id{\}} }
		& = &
		w[ \lpquote y!(z) \rpquote ] \nonumber
\end{eqnarray}

Because the body of the process between quotes is impervious to
substitution, we get radically different answers. In fact, by
examining the first process in an input context,
e.g. $x?(z).\lift{w}{y!(z)}$, we see that the process under the lift
operator may be shaped by prefixed inputs binding a name inside it. In
this sense, the lift operator will be seen as a way to dynamically
construct processes before reifying them as names.

Finally equipped with these standard features we can present the
dynamics of the calculus.

\subsubsection{Operational semantics} 

Finally, we introduce the computational dynamics. What marks these
algebras as distinct from other more traditionally studied algebraic
structures, e.g. vector spaces or polynomial rings, is the manner in
which dynamics is captured. In traditional structures, dynamics is typically
expressed through morphisms between such structures, as in linear maps
between vector spaces or morphisms between rings. In algebras
associated with the semantics of computation, the dynamics is
expressed as part of the algebraic structure itself, through a
reduction reduction relation typically denoted by $\red$. Below, we
give a recursive presentation of this relation for the calculus used
in the encoding.

$\red \subseteq \pi \times \pi$
$\red : \pi \to \mathcal{P}(\pi)$

\begin{mathpar}
  \inferrule* [lab=Comm] { \textsf{match}( x_{src}, x_{trgt} ) } { x_{trgt}?(y)P \; | \; x_{src}!\langle {Q} \rangle \red P\{\quotep{Q}/y}\} }
  \and \\
  \inferrule* [lab=Par] {{P} \red {P}'} {{{P} | {Q}} \red {{P}' | {Q}}}
  \and
  \inferrule* [lab=Equiv]{{{P} \scong {P}'} \andalso {{P}' \red {Q}'} \andalso {{Q}' \scong {Q}}}{{P} \red {Q}}
\end{mathpar}

\begin{eqnarray*}
  match_{\equiv} (\quotep{P},\quotep{Q}) & := & P \equiv Q \\
  match_{\dagger}(\quotep{P},\quotep{Q}) & := & \forall R. P|Q \red^{*} R => R \red^{*} 0 \\
  match_{K}(\quotep{P},\quotep{Q}) & := & K \mbox{ for some context } K
\end{eqnarray*}

$u?(x)P | u!\langle Q \rangle \red P\{\quotep{Q}/x\}$

%We write $\wred$ for $\red^*$, and $P\red$ if $\exists Q $ such that $ P \red Q$.
We write $P\red$ if $\exists Q $ such that $ P \red Q$ and $P\not\red$, otherwise.

\section{Replication}

As mentioned before, it is known that replication (and hence
recursion) can be implemented in a higher-order process algebra
\cite{SangiorgiWalker}. As our first example of calculation with the
machinery thus far presented we give the construction explicitly in
the {\rhoc}.

\begin{eqnarray}
	D_{x} & := & \prefix{x}{y}{(\binpar{\outputp{x}{y}}{@{y}})} \nonumber\\
	\bangp_{x}{P} & := & \binpar{{x}!\langle{\binpar{D_{x}}{P}}\rangle}{D_{x}} \nonumber
\end{eqnarray}

\begin{eqnarray}
	\bangp_{x}{P} & & \nonumber\\
	=
	& {x}!\langle{(\prefix{x}{y}{(\outputp{x}{y} | @{y})) | P}}\rangle 
	      | \prefix{x}{y}{(\outputp{x}{y} | @{y})} & \nonumber\\
	\red
	& (\outputp{x}{y} | @{y})\substn{\quotep{(\prefix{x}{y}{(@{y} | \outputp{x}{y})) | P}}}{y} & \nonumber\\
	=
	& \outputp{x}{\quotep{(\prefix{x}{y}{(\outputp{x}{y} | @{y})) | P}}}
	  | {(\prefix{x}{y}{(\outputp{x}{y} | @{y})) | P}} & \nonumber\\
	\red
	& \ldots & \nonumber\\
	\red^*
	& P | P | \ldots & \nonumber
\end{eqnarray}

Of course, this encoding, as an implementation, runs away, unfolding
$\bangp{P}$ eagerly. A lazier and more implementable replication
operator, restricted to input-guarded processes, may be obtained as follows.

\begin{eqnarray}
\bangp{\prefix{u}{v}{P}} 
	:= 
	\binpar{\lift{x}{\prefix{u}{v}{(\binpar{D(x)}{P})}}}{D(x)} \nonumber
\end{eqnarray}

\begin{remark}
  Note that the lazier definition still does not deal with summation
  or mixed summation (i.e. sums over input and output). The reader is
  invited to construct definitions of replication that deal with these
  features. 

  Further, the definitions are parameterized in a name, $x$. Can you,
  gentle reader, make a definition that eliminates this parameter and
  guarantees no accidental interaction between the replication
  machinery and the process being replicated -- i.e. no accidental
  sharing of names used by the process to get its work done and the
  name(s) used by the replication to effect copying. This latter
  revision of the definition of replication is crucial to obtaining
  the expected identity $!!P \sim !P$.
\end{remark}

\begin{remark}\label{rem:paradoxical_combinator}
  The reader familiar with the lambda calculus will have noticed the
  similarity between $D$ and the paradoxical combinator.

  [Ed. note: the existence of this seems to suggest we have to be more
  restrictive on the set of processes and names we admit if we are to
  support no-cloning.]
\end{remark}

\subsubsection{Bisimulation}

The computational dynamics gives rise to another kind of equivalence,
the equivalence of computational behavior. As previously mentioned
this is typically captured \emph{via} some form of bisimulation.

% The notion we use in this paper is weak barbed bisimulation
% \cite{milner91polyadicpi}.

The notion we use in this paper is derived from weak barbed
bisimulation \cite{milner91polyadicpi}. 

\begin{definition}
An \emph{observation relation}, $\downarrow_{\mathcal N}$, over a set
of names, $\mathcal N$, is the smallest relation satisfying the rules
below.

\infrule[Out-barb]{y \in {\mathcal N}, \; x \nameeq y}
		  {\outputp{x}{v} \downarrow_{\mathcal N} x}
\infrule[Par-barb]{\mbox{$P\downarrow_{\mathcal N} x$ or $Q\downarrow_{\mathcal N} x$}}
		  {\binpar{P}{Q} \downarrow_{\mathcal N} x}

We write $P \Downarrow_{\mathcal N} x$ if there is $Q$ such that 
$P \wred Q$ and $Q \downarrow_{\mathcal N} x$.
\end{definition}

\begin{definition}
%\label{def.bbisim}
An  ${\mathcal N}$-\emph{barbed bisimulation} over a set of names, ${\mathcal N}$, is a symmetric binary relation 
${\mathcal S}_{\mathcal N}$ between agents such that $P\rel{S}_{\mathcal N}Q$ implies:
\begin{enumerate}
\item If $P \red P'$ then $Q \wred Q'$ and $P'\rel{S}_{\mathcal N} Q'$.
\item If $P\downarrow_{\mathcal N} x$, then $Q\Downarrow_{\mathcal N} x$.
\end{enumerate}
$P$ is ${\mathcal N}$-barbed bisimilar to $Q$, written
$P \wbbisim_{\mathcal N} Q$, if $P \rel{S}_{\mathcal N} Q$ for some ${\mathcal N}$-barbed bisimulation ${\mathcal S}_{\mathcal N}$.
\end{definition}

$\mathcal{R} \subseteq \pi \times \pi$

$P \mathcal{R} Q => \forall P'. P \red P' \Rightarrow \exists Q'. Q \red Q', P' \mathcal{R} Q'$

$P \vdash x \Rightarrow Q \vdash x$

\begin{mathpar}
  \inferrule*[lab=Out-barb]{x \nameeq y}{{y}!\langle{Q}\rangle \vdash x}
  \and
  \inferrule*[lab=Par-barb]{\mbox{$P\vdash x$ or $Q\vdash x$}}{\binpar{P}{Q} \vdash x}
\end{mathpar}

\subsubsection{Contexts}

One of the principle advantages of computational calculi like the
$\pi$-calculus is a well-defined notion of context,
contextual-equivalence and a correlation between
contextual-equivalence and notions of bisimulation. The notion of
context allows the decomposition of a process into (sub-)process and
its syntactic environment, its context. Thus, a context may be
thought of as a process with a ``hole'' (written $\Box$) in it. The
application of a context $M$ to a process $P$, written $M[P]$, is
tantamount to filling the hole in $M$ with $P$. In this paper we do
not need the full weight of this theory, but do make use of the notion
of context in the proof the main theorem. 

\begin{mathpar}
  \inferrule* [lab=summation] {} {{M_{M},M_{N}} \bc \Box \;|\; x.M_{A} \;|\; M_{M}+M_{N}}
  \and
  \inferrule* [lab=agent] {} {{M_{A}} \bc (\vec{x})M_{P} \;| \; \clift{P_0,\ldots,M_{P},\ldots,P_N}}
  \and \\
  \inferrule* [lab=process] {} {{M_{P}} \bc M_{N} \;| \;P|M_{P} }
\end{mathpar} 

\begin{mathpar}
  \inferrule* [lab=sychronization] {} {M_{N} \bc \Box \;|\; x?M_{F} \;|\; x!M_{C}}
  \and
  \inferrule* [lab=abstraction] {} {{M_{F}} \bc (x)M_{P} }
  \and
  \inferrule* [lab=concretion] {} {{M_{C}} \bc \langle M_{P} \rangle }
  \and \\
  \inferrule* [lab=process] {} {{M_{P}} \bc M_{N} \;| \;P|M_{P} }
\end{mathpar}

\begin{definition}[contextual application] Given a context $M$, and
  process $P$, we define the \emph{contextual application}, $M[P] :=
  M\{P/\Box\}$. That is, the contextual application of M to P is the
  substitution of $P$ for $\Box$ in $M$.
\end{definition}

$\meaningof{-} : L \to \mathcal{P}(\pi)$

\begin{mathpar}
  \inferrule* [lab=collection] {} {\meaningof{true} = \pi, \and \meaningof{~E} = \pi \setminus \meaningof{E}, \and \meaningof{E_{1} \& E_{2}} = \meaningof{E_{1}} \cap \meaningof{E_{2}}}
\end{mathpar}

\begin{mathpar}
  \inferrule* [lab=structure] {} {\meaningof{0} = \{ P \in \pi | P \equiv 0 \}, \and \\ \meaningof{E_1 | E_2} = \{ P \in \pi | P \equiv P_{1} | P_{2}, P_{1} \in \meaningof{E_{1}}, P_{2} \in \meaningof{E_2}\} }
\end{mathpar}

\begin{mathpar}
 \inferrule* [lab=behavior] {} {\meaningof{\langle a?b \rangle E} = \{ P \in \pi | P \equiv Q | u?(y)P', \\ \and \\\\ \and \\ \;\;\; u \in \meaningof{a}, \forall z.P'\{z/y\} \in \meaningof{E\{z/b\}}\}, \and \\ \meaningof{a!E} = \{ P \in \pi | P \equiv Q | x!\langle P' \rangle, x \in \meaningof{a} P' \in \meaningof{E}\} }
\end{mathpar}

\begin{mathpar}
 \inferrule* [lab=nominal] {} {\meaningof{\quotep{E}} = \{ \quotep{P} \in \quotep{\pi} | P \in \meaningof{E} \}, \and \meaningof{\quotep{P}} = \{ \quotep{Q} \in \quotep{\pi} | P \equiv Q \} \and \\ \meaningof{@\quotep{E}} = \{ P \in \pi | P \equiv @x, x \in \meaningof{E} \}}
\end{mathpar}

\begin{eqnarray*}
  \\
  \meaningof{-} : TS \to ST
\end{eqnarray*}

\begin{eqnarray*}
  \\
  L : TS \to ST
\end{eqnarray*}

\begin{eqnarray*}
  \\
  P \models E \iff P \in \meaningof{E}
\end{eqnarray*}

\begin{eqnarray*}
  P \approx_{L} Q \iff \forall E \in L. P \models E \iff Q \models E
\end{eqnarray*}

\begin{eqnarray*}
  P \approx_{K} Q
\end{eqnarray*}

\begin{eqnarray*}
  P \approx Q
\end{eqnarray*}

$\approx_{K} = \approx = \approx_{L}$

\subsubsection{Contextual duality}

Note that contexts extend the quotation operation to a family of
operations from processes to names. Given a context, $M$, we can
define a \emph{nominal context}, $\quotep{M}$ by $\quotep{M}[P] :=
\quotep{M[P]}$. To foreshadow what is to come we observe that these
operations enjoy a duality with processes very much like the duality
between vectors and maps from vectors to scalars.

Further, because the calculus is essentially higher-order, we have a
correspondence between contexts and processes. More specifically,
given a name $x$ and a context $M$ we can construct $M^{*}_{x}$ such
that 

\begin{mathpar}
  M^{*}_{x} | \lift{x}{P} \red M[P]
\end{mathpar}

namely,

\begin{mathpar}
  M^{*}_{x} := x?(u).M[\dropn{u}]
\end{mathpar}

The dependence of $M^{*}_{x}$ on a name makes it an abstraction, 

\begin{mathpar}
  M^{*} := (x)x?(u).M[\dropn{u}]
\end{mathpar}

\subsection{Additional notation}

It will sometimes be convenient to denote the process a name
quotes. We already have the notation $x = \quotep{P}$, but it will be
convenient to introduce an alternate notation, $\procn{x}$, when we
want to emphasize the connection to the use of the name. Note that, by
virtue of name equivalence, $\quotep{\procn{x}} \nameeq x$; so, the
notation is consistent with previous definitions.

Further, because names have structure it is possible to effect
substitutions on the basis of that structure. This means we need to
upgrade our notation for substitutions, which we accomplish by
adapting comprehension notation. Thus,

\begin{mathpar}
  P\{ y / x : x \in S \}
\end{mathpar}

is interpreted to mean the process derived from P by replacing (in a
capture-avoiding manner) each occurrence of $x$ in $S$ by $y$. For example,

\begin{mathpar}
  P\{ \quotep{\procn{x}|\procn{x}} / x : x \in \freenames{P} \}
\end{mathpar}

will replace each (occurrence) of a free name $x$ in $P$ by
$\quotep{\procn{x}|\procn{x}}$.

Also, we will avail ourselves of the notation $x^{L}$ and $x^{R}$ to
denote injections of a name into disjoint copies of the name
space. There are numerous ways to accomplish this. One example can be
found in \cite{MeredithR05}. This notation overloads to vectors of
names: $\vec{x}^{\pi} := (x_{i}^{\pi} \; : \; 0 \leq i < |\vec{x}| )$ where $\pi \in \{L,R\}$.

We also use $P^{\Box} := P|\Box$.

In \cite{MeredithR05} an interpretation of the new operator is
given. It turns out that there are several possible interpretations
all enjoying the requisite algebraic properties of the operator (see
\cite{milner91polyadicpi}). We will therefore make liberal use of
$(\nu\; \vec{x})P$.

% subsection the_syntax_and_semantics_of_the_notation_system (end)   

\input{qm2pi.qmops} 

\input{qm2pi.sterngerlach} 

\input{qm2pi.metric} 

% section concurrent_process_calculi (end)

%\input{qm2pi.proofsketch}

% section proof sketch (end)

%\input{qm2pi.slviaknots} 

% section spatial logic via knots (end)

\input{qm2pi.conclusion}

% section conclusion (end)

%\input{qm2pi.dtcodes} 

% section wiring algorithm (end)

\input{qm2pi.ack} 

% section acknowledgments (end)

\newpage


\bibliographystyle{plain}   
\bibliography{../../biblios/main.bib}

\input{qm2pi.rhodetails}

\end{document}

 

%\ifpdf
%\usepackage[pdftex]{graphicx}
%\else
%\usepackage{graphicx}
%\fi

 % \ifpdf
%  \usepackage{pdfsync}
%  \if


%\title{Brief Article}
%\author{David F. Snyder}
%\author{L.G. Meredith}

%\address{Dept. of Math., Texas State University--San Marcos, San Marcos, TX 78666}
       
\pagestyle{empty}


\begin{document}

\lstset{language=[Objective]Caml,frame=shadowbox}

\documentclass[12pt]{llncs}
%\documentclass{jktr}

\usepackage[pdftex]{hyperref}                   
\usepackage {listings}
\usepackage {mathpartir}
\usepackage{bcprules}
%\usepackage{listings}
                       
\usepackage{graphicx} 
%\usepackage[margins=2.5cm,nohead,nofoot]{geometry}
%\usepackage{geometry}
\usepackage{amsfonts}
\usepackage{amstext}
\usepackage{latexsym}
\usepackage{amssymb}
\usepackage{color}


%\include{myPreamble}
\include{qm2pi.local} 

%\ifpdf
%\usepackage[pdftex]{graphicx}
%\else
%\usepackage{graphicx}
%\fi

 % \ifpdf
%  \usepackage{pdfsync}
%  \if


%\title{Brief Article}
%\author{David F. Snyder}
%\author{L.G. Meredith}

%\address{Dept. of Math., Texas State University--San Marcos, San Marcos, TX 78666}
       
\pagestyle{empty}


\begin{document}

\lstset{language=[Objective]Caml,frame=shadowbox}

\input{qm2pi.front}

% section front matter (end)

\input{qm2pi.intro} 
 
% section introduction (end)

% \input{qm2pi.knotations} 

% section notation (end)

\input{qm2pi.process.calculi} 

% section concurrent_process_calculi_and_spatial_logics_ (end)
    
%\input{qm2pi.knots2pi} 

%\input{qm2pi.trefoil} 

%\input{qm2pi.mainthm} 

% subsection basic_interpretation (end)

%\input{qm2pi.rho.presentation} 
\subsection{The syntax and semantics of the notation system}\label{sub:the_syntax_and_semantics_of_the_notation_system} % (fold)

We now summarize a technical presentation of the calculus that
embodies our theory of dynamics. The typical presentation of such a
calculus follows the style of giving generators and relations on
them. The grammar, below, describing term constructors, freely
generates the set of processes, $\Proc$. This set is then quotiented
by a relation known as structural congruence and it is over this set
that the notion of dynamics is expressed. This presentation is
essentially that of \cite{MeredithR05} with the addition of
polyadicity and summation. For readability we have relegated some of
the technical subtleties to an appendix.

\subsubsection{Process grammar}\label{subsub:process_grammar}

\begin{mathpar}
  \inferrule* [lab=synchronization] {} {{M} \bc \pzero \;|\; x?F \;|\; x!C }
  \and
  \inferrule* [lab=abstraction] {} {{F} \bc (x)P}
  \and
  \inferrule* [lab=concretion] {} {{C} \bc \langle Q \rangle}
  \and
  \inferrule* [lab=process] {} {{P,Q} \bc M \;| \;P|Q \;|\; @{x}}
  \and
  \inferrule* [lab=name] {} {{x} \bc \quotep{P}}
\end{mathpar} 

Note that $\vec{x}$ (resp. $\vec{P}$) denotes a vector of names
(resp. processes) of length $|\vec{x}|$ (resp. $|\vec{P}|$). We adopt
the following useful abbreviations.

\begin{mathpar}
   x?(\vec{y}).P := x.(\vec{y})P \and  x\clift{\vec{P}} := x.\clift{\vec{P}}
   \and x!(y) := \lift{x}{\dropn{y}}
   \and \Pi_{i=0}^{n-1}P_i := P_0 | \ldots | P_{n-1}
\end{mathpar}

\subsubsection{Structural congruence}

\paragraph{Free and bound names and alpha-equivalence.} At the
core of structural equivalence is alpha-equivalence which identifies
process that are the same up to a change of variable. Formally, we
recognize the distinction between free and bound names. The free names
of a process, $\freenames{P}$, may be calculated recursively as
follows:

\begin{mathpar}
\freenames{\pzero} := \emptyset
  \and \\
  \freenames{x?(y).P} := \{ x \} \cup (\freenames{P} \setminus \{ y \})
  \and 
  \freenames{x!\langle P \rangle} := \{ x \} \cup \{ P \} 
  \and \\
  \freenames{P|Q} := \freenames{P} \cup \freenames{Q}
  \and \\
  \freenames{@{x}} := \{ x \}
\end{mathpar}

$\pi$
$\quotep{\pi}$

$\freenames{-} : \pi \to \mathcal{P}(\quotep{\pi})$

\begin{eqnarray*}
  \freenames{\pzero} & := & \emptyset \\
  \freenames{x?(y).P} & := & \{ x \} \cup (\freenames{P} \setminus \{ y \}) \\
  \freenames{x!\langle P \rangle} & := & \{ x \} \cup \{ P \} \\
  \freenames{P|Q} & := & \freenames{P} \cup \freenames{Q} \\
  \freenames{\dropn{x}} & := & \{ x \}
\end{eqnarray*}

The bound names of a process, $\boundnames{P}$, are those names occurring in $P$
that are not free. For example, in $x?(y).0$, the name $x$ is free, while $y$ is bound.

\begin{mathpar}
  \inferrule* [lab=monoidal-laws] {} { P|Q \equiv Q|P \and P|0 \equiv P \and P|(Q|R) \equiv (P|Q)|R }
\end{mathpar}

\begin{mathpar}
  \inferrule* [lab=alpha-equivalence] {} { (x)P \equiv (y)P\{y/x\} \and y \not\in \freenames{P} }
\end{mathpar}

\begin{definition}
Then two processes, $P,Q$, are alpha-equivalent if $P = Q\{\vec{y}/\vec{x}\}$ for
some $\vec{x} \in \boundnames{Q},\vec{y} \in \boundnames{P}$, where $Q\{\vec{y}/\vec{x}\}$
denotes the capture-avoiding substitution of $\vec{y}$ for $\vec{x}$ in $Q$.
\end{definition}

\begin{definition}
  The {\em structural congruence} \cite{SangiorgiWalker} , $\equiv$,
  between processes is the least congruence containing
  alpha-equivalence, satisfying the abelian monoid laws
  (associativity, commutativity and $\pzero$ as identity) for parallel
  composition $|$ and for summation $+$.
\end{definition}

\subsection{Name equivalence}

We take name equivalence, written $\nameeq$, to be the smallest
equivalence relation generated by the following rules.

\begin{mathpar}
\inferrule*[lab=Quote-drop]
{ }
{ \quotep{@{x}} \nameeq x }

\inferrule*[lab=Struct-equiv]
{ P \scong Q }
{ \quotep{P} \nameeq \quotep{Q} }
\end{mathpar}

The astute reader will have noticed that the mutual recursion of names
and processes imposes a mutual recursion on alpha-equivalence and
structural equivalence via name-equivalence. Fortunately, all of this
works out pleasantly and we may calculate in the natural way, free of
concern. The reader interested in the details is referred to the
appendix \ref{appendix:rho_details}.

\subsection{Substitution}

We use $\Proc$ for the set of processes, $\QProc$ for the set of
names, and $\id{\{}\vec{y} / \vec{x} \id{\}}$ to denote partial maps,
$s : \QProc \rightarrow \QProc$. A map, $s$ lifts, uniquely, to a map
on process terms, $\widehat{s} : \Proc \rightarrow \Proc$ by the
following equations.

\begin{mathpar}
  (0) \psubstp{Q}{P} := 0 \\
  (R \juxtap S) \psubstp{Q}{P}
  :=    
  (R)\psubstp{Q}{P} \juxtap (S) \psubstp{Q}{P} \\
  (x?(y).R) \psubstp{Q}{P}    
  :=    
  (x)\substp{Q}{P} (z)\concat( (R \psubstn{z}{y}) \psubstp{Q}{P} ) \\
  (\lift{x}{R}) \psubstp{Q}{P}  
  :=
  \lift{(x)\substp{Q}{P}}{ R \psubstp{Q}{P} } \\
%   (\dropn{x})  \psubstp{Q}{P}       
%   := 
%   \left\{ 
%     \begin{array}{ccc} 
%       \dropn{\quotep{Q}} & & x \nameeq \quotep{P} \\
%       \dropn{x} & & otherwise \\
%     \end{array}
%   \right. 
  (\dropn{x})  \psubstp{Q}{P}       
  := 
  \left\{ 
    \begin{array}{ccc} 
      Q & & x \nameeq \quotep{P} \\
      \dropn{x} & & otherwise \\
    \end{array}
  \right.
\end{mathpar}
 

where

\begin{eqnarray}
  (x)\id{\{} \lpquote Q \rpquote / \lpquote P \rpquote \id{\}}            = 
  \left\{ 
    \begin{array}{ccc}
      \lpquote Q \rpquote & & x \nameeq \lpquote P \rpquote \\
      x & & otherwise \\
    \end{array}
  \right. \nonumber
\end{eqnarray}

and $z$ is chosen distinct from $\quotep{P}$, $\quotep{Q}$, the free
names in $Q$, and all the names in $R$. Our $\alpha$-equivalence will
be built in the standard way from this substitution.

\begin{remark}\label{rem:no_self_referential_names}
  One consequence of these definitions is that $\forall P. \quotep{P}
  \not\in \freenames{P}$.
\end{remark}

\subsection{ Dynamic quote: an example }

Anticipating something of what's to come, consider applying the
substitution, $\widehat{\id{\{}u / z \id{\}}}$, to the following pair
of processes, $\lift{w}{y!(z)}$ and $w[ \lpquote y!(z) \rpquote ]$.

\begin{eqnarray}
	\lift{w}{y!(z)}\widehat{\id{\{}u / z \id{\}}}
		& = &
		\lift{w}{y!(u)} \nonumber\\
	w[ \lpquote y!(z) \rpquote ] \widehat{ \id{\{}u / z \id{\}} }
		& = &
		w[ \lpquote y!(z) \rpquote ] \nonumber
\end{eqnarray}

Because the body of the process between quotes is impervious to
substitution, we get radically different answers. In fact, by
examining the first process in an input context,
e.g. $x?(z).\lift{w}{y!(z)}$, we see that the process under the lift
operator may be shaped by prefixed inputs binding a name inside it. In
this sense, the lift operator will be seen as a way to dynamically
construct processes before reifying them as names.

Finally equipped with these standard features we can present the
dynamics of the calculus.

\subsubsection{Operational semantics} 

Finally, we introduce the computational dynamics. What marks these
algebras as distinct from other more traditionally studied algebraic
structures, e.g. vector spaces or polynomial rings, is the manner in
which dynamics is captured. In traditional structures, dynamics is typically
expressed through morphisms between such structures, as in linear maps
between vector spaces or morphisms between rings. In algebras
associated with the semantics of computation, the dynamics is
expressed as part of the algebraic structure itself, through a
reduction reduction relation typically denoted by $\red$. Below, we
give a recursive presentation of this relation for the calculus used
in the encoding.

$\red \subseteq \pi \times \pi$
$\red : \pi \to \mathcal{P}(\pi)$

\begin{mathpar}
  \inferrule* [lab=Comm] { \textsf{match}( x_{src}, x_{trgt} ) } { x_{trgt}?(y)P \; | \; x_{src}!\langle {Q} \rangle \red P\{\quotep{Q}/y}\} }
  \and \\
  \inferrule* [lab=Par] {{P} \red {P}'} {{{P} | {Q}} \red {{P}' | {Q}}}
  \and
  \inferrule* [lab=Equiv]{{{P} \scong {P}'} \andalso {{P}' \red {Q}'} \andalso {{Q}' \scong {Q}}}{{P} \red {Q}}
\end{mathpar}

\begin{eqnarray*}
  match_{\equiv} (\quotep{P},\quotep{Q}) & := & P \equiv Q \\
  match_{\dagger}(\quotep{P},\quotep{Q}) & := & \forall R. P|Q \red^{*} R => R \red^{*} 0 \\
  match_{K}(\quotep{P},\quotep{Q}) & := & K \mbox{ for some context } K
\end{eqnarray*}

$u?(x)P | u!\langle Q \rangle \red P\{\quotep{Q}/x\}$

%We write $\wred$ for $\red^*$, and $P\red$ if $\exists Q $ such that $ P \red Q$.
We write $P\red$ if $\exists Q $ such that $ P \red Q$ and $P\not\red$, otherwise.

\section{Replication}

As mentioned before, it is known that replication (and hence
recursion) can be implemented in a higher-order process algebra
\cite{SangiorgiWalker}. As our first example of calculation with the
machinery thus far presented we give the construction explicitly in
the {\rhoc}.

\begin{eqnarray}
	D_{x} & := & \prefix{x}{y}{(\binpar{\outputp{x}{y}}{@{y}})} \nonumber\\
	\bangp_{x}{P} & := & \binpar{{x}!\langle{\binpar{D_{x}}{P}}\rangle}{D_{x}} \nonumber
\end{eqnarray}

\begin{eqnarray}
	\bangp_{x}{P} & & \nonumber\\
	=
	& {x}!\langle{(\prefix{x}{y}{(\outputp{x}{y} | @{y})) | P}}\rangle 
	      | \prefix{x}{y}{(\outputp{x}{y} | @{y})} & \nonumber\\
	\red
	& (\outputp{x}{y} | @{y})\substn{\quotep{(\prefix{x}{y}{(@{y} | \outputp{x}{y})) | P}}}{y} & \nonumber\\
	=
	& \outputp{x}{\quotep{(\prefix{x}{y}{(\outputp{x}{y} | @{y})) | P}}}
	  | {(\prefix{x}{y}{(\outputp{x}{y} | @{y})) | P}} & \nonumber\\
	\red
	& \ldots & \nonumber\\
	\red^*
	& P | P | \ldots & \nonumber
\end{eqnarray}

Of course, this encoding, as an implementation, runs away, unfolding
$\bangp{P}$ eagerly. A lazier and more implementable replication
operator, restricted to input-guarded processes, may be obtained as follows.

\begin{eqnarray}
\bangp{\prefix{u}{v}{P}} 
	:= 
	\binpar{\lift{x}{\prefix{u}{v}{(\binpar{D(x)}{P})}}}{D(x)} \nonumber
\end{eqnarray}

\begin{remark}
  Note that the lazier definition still does not deal with summation
  or mixed summation (i.e. sums over input and output). The reader is
  invited to construct definitions of replication that deal with these
  features. 

  Further, the definitions are parameterized in a name, $x$. Can you,
  gentle reader, make a definition that eliminates this parameter and
  guarantees no accidental interaction between the replication
  machinery and the process being replicated -- i.e. no accidental
  sharing of names used by the process to get its work done and the
  name(s) used by the replication to effect copying. This latter
  revision of the definition of replication is crucial to obtaining
  the expected identity $!!P \sim !P$.
\end{remark}

\begin{remark}\label{rem:paradoxical_combinator}
  The reader familiar with the lambda calculus will have noticed the
  similarity between $D$ and the paradoxical combinator.

  [Ed. note: the existence of this seems to suggest we have to be more
  restrictive on the set of processes and names we admit if we are to
  support no-cloning.]
\end{remark}

\subsubsection{Bisimulation}

The computational dynamics gives rise to another kind of equivalence,
the equivalence of computational behavior. As previously mentioned
this is typically captured \emph{via} some form of bisimulation.

% The notion we use in this paper is weak barbed bisimulation
% \cite{milner91polyadicpi}.

The notion we use in this paper is derived from weak barbed
bisimulation \cite{milner91polyadicpi}. 

\begin{definition}
An \emph{observation relation}, $\downarrow_{\mathcal N}$, over a set
of names, $\mathcal N$, is the smallest relation satisfying the rules
below.

\infrule[Out-barb]{y \in {\mathcal N}, \; x \nameeq y}
		  {\outputp{x}{v} \downarrow_{\mathcal N} x}
\infrule[Par-barb]{\mbox{$P\downarrow_{\mathcal N} x$ or $Q\downarrow_{\mathcal N} x$}}
		  {\binpar{P}{Q} \downarrow_{\mathcal N} x}

We write $P \Downarrow_{\mathcal N} x$ if there is $Q$ such that 
$P \wred Q$ and $Q \downarrow_{\mathcal N} x$.
\end{definition}

\begin{definition}
%\label{def.bbisim}
An  ${\mathcal N}$-\emph{barbed bisimulation} over a set of names, ${\mathcal N}$, is a symmetric binary relation 
${\mathcal S}_{\mathcal N}$ between agents such that $P\rel{S}_{\mathcal N}Q$ implies:
\begin{enumerate}
\item If $P \red P'$ then $Q \wred Q'$ and $P'\rel{S}_{\mathcal N} Q'$.
\item If $P\downarrow_{\mathcal N} x$, then $Q\Downarrow_{\mathcal N} x$.
\end{enumerate}
$P$ is ${\mathcal N}$-barbed bisimilar to $Q$, written
$P \wbbisim_{\mathcal N} Q$, if $P \rel{S}_{\mathcal N} Q$ for some ${\mathcal N}$-barbed bisimulation ${\mathcal S}_{\mathcal N}$.
\end{definition}

$\mathcal{R} \subseteq \pi \times \pi$

$P \mathcal{R} Q => \forall P'. P \red P' \Rightarrow \exists Q'. Q \red Q', P' \mathcal{R} Q'$

$P \vdash x \Rightarrow Q \vdash x$

\begin{mathpar}
  \inferrule*[lab=Out-barb]{x \nameeq y}{{y}!\langle{Q}\rangle \vdash x}
  \and
  \inferrule*[lab=Par-barb]{\mbox{$P\vdash x$ or $Q\vdash x$}}{\binpar{P}{Q} \vdash x}
\end{mathpar}

\subsubsection{Contexts}

One of the principle advantages of computational calculi like the
$\pi$-calculus is a well-defined notion of context,
contextual-equivalence and a correlation between
contextual-equivalence and notions of bisimulation. The notion of
context allows the decomposition of a process into (sub-)process and
its syntactic environment, its context. Thus, a context may be
thought of as a process with a ``hole'' (written $\Box$) in it. The
application of a context $M$ to a process $P$, written $M[P]$, is
tantamount to filling the hole in $M$ with $P$. In this paper we do
not need the full weight of this theory, but do make use of the notion
of context in the proof the main theorem. 

\begin{mathpar}
  \inferrule* [lab=summation] {} {{M_{M},M_{N}} \bc \Box \;|\; x.M_{A} \;|\; M_{M}+M_{N}}
  \and
  \inferrule* [lab=agent] {} {{M_{A}} \bc (\vec{x})M_{P} \;| \; \clift{P_0,\ldots,M_{P},\ldots,P_N}}
  \and \\
  \inferrule* [lab=process] {} {{M_{P}} \bc M_{N} \;| \;P|M_{P} }
\end{mathpar} 

\begin{mathpar}
  \inferrule* [lab=sychronization] {} {M_{N} \bc \Box \;|\; x?M_{F} \;|\; x!M_{C}}
  \and
  \inferrule* [lab=abstraction] {} {{M_{F}} \bc (x)M_{P} }
  \and
  \inferrule* [lab=concretion] {} {{M_{C}} \bc \langle M_{P} \rangle }
  \and \\
  \inferrule* [lab=process] {} {{M_{P}} \bc M_{N} \;| \;P|M_{P} }
\end{mathpar}

\begin{definition}[contextual application] Given a context $M$, and
  process $P$, we define the \emph{contextual application}, $M[P] :=
  M\{P/\Box\}$. That is, the contextual application of M to P is the
  substitution of $P$ for $\Box$ in $M$.
\end{definition}

$\meaningof{-} : L \to \mathcal{P}(\pi)$

\begin{mathpar}
  \inferrule* [lab=collection] {} {\meaningof{true} = \pi, \and \meaningof{~E} = \pi \setminus \meaningof{E}, \and \meaningof{E_{1} \& E_{2}} = \meaningof{E_{1}} \cap \meaningof{E_{2}}}
\end{mathpar}

\begin{mathpar}
  \inferrule* [lab=structure] {} {\meaningof{0} = \{ P \in \pi | P \equiv 0 \}, \and \\ \meaningof{E_1 | E_2} = \{ P \in \pi | P \equiv P_{1} | P_{2}, P_{1} \in \meaningof{E_{1}}, P_{2} \in \meaningof{E_2}\} }
\end{mathpar}

\begin{mathpar}
 \inferrule* [lab=behavior] {} {\meaningof{\langle a?b \rangle E} = \{ P \in \pi | P \equiv Q | u?(y)P', \\ \and \\\\ \and \\ \;\;\; u \in \meaningof{a}, \forall z.P'\{z/y\} \in \meaningof{E\{z/b\}}\}, \and \\ \meaningof{a!E} = \{ P \in \pi | P \equiv Q | x!\langle P' \rangle, x \in \meaningof{a} P' \in \meaningof{E}\} }
\end{mathpar}

\begin{mathpar}
 \inferrule* [lab=nominal] {} {\meaningof{\quotep{E}} = \{ \quotep{P} \in \quotep{\pi} | P \in \meaningof{E} \}, \and \meaningof{\quotep{P}} = \{ \quotep{Q} \in \quotep{\pi} | P \equiv Q \} \and \\ \meaningof{@\quotep{E}} = \{ P \in \pi | P \equiv @x, x \in \meaningof{E} \}}
\end{mathpar}

\begin{eqnarray*}
  \\
  \meaningof{-} : TS \to ST
\end{eqnarray*}

\begin{eqnarray*}
  \\
  L : TS \to ST
\end{eqnarray*}

\begin{eqnarray*}
  \\
  P \models E \iff P \in \meaningof{E}
\end{eqnarray*}

\begin{eqnarray*}
  P \approx_{L} Q \iff \forall E \in L. P \models E \iff Q \models E
\end{eqnarray*}

\begin{eqnarray*}
  P \approx_{K} Q
\end{eqnarray*}

\begin{eqnarray*}
  P \approx Q
\end{eqnarray*}

$\approx_{K} = \approx = \approx_{L}$

\subsubsection{Contextual duality}

Note that contexts extend the quotation operation to a family of
operations from processes to names. Given a context, $M$, we can
define a \emph{nominal context}, $\quotep{M}$ by $\quotep{M}[P] :=
\quotep{M[P]}$. To foreshadow what is to come we observe that these
operations enjoy a duality with processes very much like the duality
between vectors and maps from vectors to scalars.

Further, because the calculus is essentially higher-order, we have a
correspondence between contexts and processes. More specifically,
given a name $x$ and a context $M$ we can construct $M^{*}_{x}$ such
that 

\begin{mathpar}
  M^{*}_{x} | \lift{x}{P} \red M[P]
\end{mathpar}

namely,

\begin{mathpar}
  M^{*}_{x} := x?(u).M[\dropn{u}]
\end{mathpar}

The dependence of $M^{*}_{x}$ on a name makes it an abstraction, 

\begin{mathpar}
  M^{*} := (x)x?(u).M[\dropn{u}]
\end{mathpar}

\subsection{Additional notation}

It will sometimes be convenient to denote the process a name
quotes. We already have the notation $x = \quotep{P}$, but it will be
convenient to introduce an alternate notation, $\procn{x}$, when we
want to emphasize the connection to the use of the name. Note that, by
virtue of name equivalence, $\quotep{\procn{x}} \nameeq x$; so, the
notation is consistent with previous definitions.

Further, because names have structure it is possible to effect
substitutions on the basis of that structure. This means we need to
upgrade our notation for substitutions, which we accomplish by
adapting comprehension notation. Thus,

\begin{mathpar}
  P\{ y / x : x \in S \}
\end{mathpar}

is interpreted to mean the process derived from P by replacing (in a
capture-avoiding manner) each occurrence of $x$ in $S$ by $y$. For example,

\begin{mathpar}
  P\{ \quotep{\procn{x}|\procn{x}} / x : x \in \freenames{P} \}
\end{mathpar}

will replace each (occurrence) of a free name $x$ in $P$ by
$\quotep{\procn{x}|\procn{x}}$.

Also, we will avail ourselves of the notation $x^{L}$ and $x^{R}$ to
denote injections of a name into disjoint copies of the name
space. There are numerous ways to accomplish this. One example can be
found in \cite{MeredithR05}. This notation overloads to vectors of
names: $\vec{x}^{\pi} := (x_{i}^{\pi} \; : \; 0 \leq i < |\vec{x}| )$ where $\pi \in \{L,R\}$.

We also use $P^{\Box} := P|\Box$.

In \cite{MeredithR05} an interpretation of the new operator is
given. It turns out that there are several possible interpretations
all enjoying the requisite algebraic properties of the operator (see
\cite{milner91polyadicpi}). We will therefore make liberal use of
$(\nu\; \vec{x})P$.

% subsection the_syntax_and_semantics_of_the_notation_system (end)   

\input{qm2pi.qmops} 

\input{qm2pi.sterngerlach} 

\input{qm2pi.metric} 

% section concurrent_process_calculi (end)

%\input{qm2pi.proofsketch}

% section proof sketch (end)

%\input{qm2pi.slviaknots} 

% section spatial logic via knots (end)

\input{qm2pi.conclusion}

% section conclusion (end)

%\input{qm2pi.dtcodes} 

% section wiring algorithm (end)

\input{qm2pi.ack} 

% section acknowledgments (end)

\newpage


\bibliographystyle{plain}   
\bibliography{../../biblios/main.bib}

\input{qm2pi.rhodetails}

\end{document}



% section front matter (end)

\section{Introduction}\label{sec:introduction} % (fold)
In this draft of the material i am going to have to dispense with the
usual writing conventions adopted in papers on these topics. i'm going
to have adopt whatever tone i need at the time i'm writing up the
calculations. Sometimes this may be very conversational; others it may
be the barest mathematical grunts; others still it may be that i have
lifted text from one of my other papers because the exposition of some
point was better said there. i hope that my readers are not unduly put
out by this decision. i'm not doing this to flout convention or be
rebellious. i find these calculations very technically challenging. To
keep everything going technically, something has to give; i have to
let go of some cognitive burden. So, the academic writing style --
with all of its trade-offs in terms of facilitating technical
communication -- is what i'm letting go of. Perhaps subsequent drafts
can be tightened and polished, but for now, i'm going to speak as if
we were sitting together in a coffee shop with a laptop, wifi and a
pad of paper and a pencil.

So, here's what i have to say. We -- you and i, comfortably ensconced
in our coffee shop and well-equipped with our tools -- can realize and
carry out the calculations of quantum mechanics over a very different
formal theory of dynamics, a formal theory of dynamics that
corresponds to a theory of concurrent computation with
\emph{reflection}. It has the advantage that the underlying theory is
already `quantized', but supports analogues all of the continuuous
operations. Strikingly, this underlying theory has recently been
connected with a notion of metric that we can show, by calculating
together, coincides with the metric induced by the inner product.

There are a lot of reasons why you might be interested in seeing
calculations of this form. Here's why i'm interested. For the past
several centuries there has been no competitor to the ``Newtonian''
account of dynamics. As a result the predominant share of accounts of
dynamical systems and situations have had to be formulated in terms of
the Newtonian machinery. i view this as an intellectually dangerous
position to occupy. Everything, despite it's intrinsic shape, turns
into a nail to be hit with this hammer. Recently, however, the theory
of computation has matured to the point where we have candidates for
theories of dynamics that offer very different perspective on
reasoning about dynamical systems and situations. Testing these
candidates against very successful accounts of dynamical situations,
like quantum mechanics, is going to give us some sense of how mature
they are and some measure of the quality of these accounts of
dynamics.

\subsection{Summary of contributions and outline of paper}

So, we're going to develop an interpretation of the operations of
quantum mechanics normally interpreted by Hilbert spaces and
operators. We're going to do this over a theory of computation. Note
that this is very different than the usual quantum computation program
which develops notions of computation over quantum mechanics. Rather,
we are developing a story that aligns with Wheeler's slogan: It from
Bit. To do this we will first provide an account of the theory of
computation at play here. Then we will dive into a calculation-driven
interpretation of the operations of quantum mechanics.

The reason we take this approach is that -- until very recently --
there hasn't been an axiomatic account of quantum mechanics. As a
result there has been no sharp delineation of the mathematical theory
supporting interpretation of the physical theory and the physical
theory, itself. So, ambient features of the maths are free to be
exploited (or supressed) without a real accounting of their physical
relevance. There is no sharp statement ``here's the physical theory''
qua \emph{theory} and ``here's the mathematical interpretation''
enabling a judgment of how faithful the interpretation is -- apart
from experimental observation. When there is an axiomatic account we
can judge how well a given mathematical formalism supports an
interpretation of the axioms, independent of
experimentation. Likewise, we can judge how well we have captured our
physical evidence and experience with our axiomatics, independent of
any specific mathematical implementation, with accidental detail that
may or may not have physical significance. 

In lieu of a fully fleshed out and vetted axiomatic account of quantum
mechanics, interpreting the operational notions in service of modeling
physical systems will have to suffice. In other words, we are not in
the business of providing a model of Hilbert spaces and operators. We
are in the business of providing a model of quantum mechanics because
we are motivated by testing our notions of dynamics against physical
theory; and, the predictive calculations of the physical theory must
serve as the best formulation -- shy of a fully fleshed out axiomatic
account -- of the physical theory itself (as they have for scientific
theories since time immemorial). Put another way, despite a
whole-hearted commitment to an It-from-Bit ontology, we are firmly
aligned with the shut-up-and-calculate camp as the best way to obtain
results either from the physical perspective or as a quality assurance
measure of our fledgling theory of dynamics.

In detail, we present a reflective process calculus. Then we develop
intuitive correspondences between the notions available in this
calculus and the usual physical notions supporting quantum mechanical
calculations. Thus, 

\begin{table}[htp]
  \center{
    \fbox{
      \begin{tabular}{c|c}
        quantum mechanics & process calculus \\
        \hline
        scalar & name \\
        state vector & process \\
        dual & contextual duals \\
        matrix & formal sums of process-context-dual pairs \\
        orthogonality & process annihilation \\
        inner product & execution-formula + quoting
      \end{tabular}
    }
  }
  \caption{QM - process calculi correspondences}
\end{table}

Then we tighten up these intuitions to operational definitions. We
employ the Dirac notation as the best proxy we can find for an
abstract syntax of the quantum mechanical notions. The definitions we
develop put us in contact with equational constraints coming from the
theory that we demonstrate the definitions and calculations satisfy.

This puts us in a position to shut up and calculate for the
Stern-Gerlach experimental set up, showing how these predictive
calculations become calculations on processes in our theory of a
reflective process calculus.

Penultimately, we demonstrate that the notion of metric coming from
the inner product coincides with the notion of metric available from
the theory of bisimulation. This demonstration gives us the right to
think of space as arising from behavior. Finally, we consider where we
might go from the new vantage point we have obtained.

% section introduction (end) 
 
% section introduction (end)

% \documentclass[12pt]{llncs}
%\documentclass{jktr}

\usepackage[pdftex]{hyperref}                   
\usepackage {listings}
\usepackage {mathpartir}
\usepackage{bcprules}
%\usepackage{listings}
                       
\usepackage{graphicx} 
%\usepackage[margins=2.5cm,nohead,nofoot]{geometry}
%\usepackage{geometry}
\usepackage{amsfonts}
\usepackage{amstext}
\usepackage{latexsym}
\usepackage{amssymb}
\usepackage{color}


%\include{myPreamble}
\include{qm2pi.local} 

%\ifpdf
%\usepackage[pdftex]{graphicx}
%\else
%\usepackage{graphicx}
%\fi

 % \ifpdf
%  \usepackage{pdfsync}
%  \if


%\title{Brief Article}
%\author{David F. Snyder}
%\author{L.G. Meredith}

%\address{Dept. of Math., Texas State University--San Marcos, San Marcos, TX 78666}
       
\pagestyle{empty}


\begin{document}

\lstset{language=[Objective]Caml,frame=shadowbox}

\input{qm2pi.front}

% section front matter (end)

\input{qm2pi.intro} 
 
% section introduction (end)

% \input{qm2pi.knotations} 

% section notation (end)

\input{qm2pi.process.calculi} 

% section concurrent_process_calculi_and_spatial_logics_ (end)
    
%\input{qm2pi.knots2pi} 

%\input{qm2pi.trefoil} 

%\input{qm2pi.mainthm} 

% subsection basic_interpretation (end)

%\input{qm2pi.rho.presentation} 
\subsection{The syntax and semantics of the notation system}\label{sub:the_syntax_and_semantics_of_the_notation_system} % (fold)

We now summarize a technical presentation of the calculus that
embodies our theory of dynamics. The typical presentation of such a
calculus follows the style of giving generators and relations on
them. The grammar, below, describing term constructors, freely
generates the set of processes, $\Proc$. This set is then quotiented
by a relation known as structural congruence and it is over this set
that the notion of dynamics is expressed. This presentation is
essentially that of \cite{MeredithR05} with the addition of
polyadicity and summation. For readability we have relegated some of
the technical subtleties to an appendix.

\subsubsection{Process grammar}\label{subsub:process_grammar}

\begin{mathpar}
  \inferrule* [lab=synchronization] {} {{M} \bc \pzero \;|\; x?F \;|\; x!C }
  \and
  \inferrule* [lab=abstraction] {} {{F} \bc (x)P}
  \and
  \inferrule* [lab=concretion] {} {{C} \bc \langle Q \rangle}
  \and
  \inferrule* [lab=process] {} {{P,Q} \bc M \;| \;P|Q \;|\; @{x}}
  \and
  \inferrule* [lab=name] {} {{x} \bc \quotep{P}}
\end{mathpar} 

Note that $\vec{x}$ (resp. $\vec{P}$) denotes a vector of names
(resp. processes) of length $|\vec{x}|$ (resp. $|\vec{P}|$). We adopt
the following useful abbreviations.

\begin{mathpar}
   x?(\vec{y}).P := x.(\vec{y})P \and  x\clift{\vec{P}} := x.\clift{\vec{P}}
   \and x!(y) := \lift{x}{\dropn{y}}
   \and \Pi_{i=0}^{n-1}P_i := P_0 | \ldots | P_{n-1}
\end{mathpar}

\subsubsection{Structural congruence}

\paragraph{Free and bound names and alpha-equivalence.} At the
core of structural equivalence is alpha-equivalence which identifies
process that are the same up to a change of variable. Formally, we
recognize the distinction between free and bound names. The free names
of a process, $\freenames{P}$, may be calculated recursively as
follows:

\begin{mathpar}
\freenames{\pzero} := \emptyset
  \and \\
  \freenames{x?(y).P} := \{ x \} \cup (\freenames{P} \setminus \{ y \})
  \and 
  \freenames{x!\langle P \rangle} := \{ x \} \cup \{ P \} 
  \and \\
  \freenames{P|Q} := \freenames{P} \cup \freenames{Q}
  \and \\
  \freenames{@{x}} := \{ x \}
\end{mathpar}

$\pi$
$\quotep{\pi}$

$\freenames{-} : \pi \to \mathcal{P}(\quotep{\pi})$

\begin{eqnarray*}
  \freenames{\pzero} & := & \emptyset \\
  \freenames{x?(y).P} & := & \{ x \} \cup (\freenames{P} \setminus \{ y \}) \\
  \freenames{x!\langle P \rangle} & := & \{ x \} \cup \{ P \} \\
  \freenames{P|Q} & := & \freenames{P} \cup \freenames{Q} \\
  \freenames{\dropn{x}} & := & \{ x \}
\end{eqnarray*}

The bound names of a process, $\boundnames{P}$, are those names occurring in $P$
that are not free. For example, in $x?(y).0$, the name $x$ is free, while $y$ is bound.

\begin{mathpar}
  \inferrule* [lab=monoidal-laws] {} { P|Q \equiv Q|P \and P|0 \equiv P \and P|(Q|R) \equiv (P|Q)|R }
\end{mathpar}

\begin{mathpar}
  \inferrule* [lab=alpha-equivalence] {} { (x)P \equiv (y)P\{y/x\} \and y \not\in \freenames{P} }
\end{mathpar}

\begin{definition}
Then two processes, $P,Q$, are alpha-equivalent if $P = Q\{\vec{y}/\vec{x}\}$ for
some $\vec{x} \in \boundnames{Q},\vec{y} \in \boundnames{P}$, where $Q\{\vec{y}/\vec{x}\}$
denotes the capture-avoiding substitution of $\vec{y}$ for $\vec{x}$ in $Q$.
\end{definition}

\begin{definition}
  The {\em structural congruence} \cite{SangiorgiWalker} , $\equiv$,
  between processes is the least congruence containing
  alpha-equivalence, satisfying the abelian monoid laws
  (associativity, commutativity and $\pzero$ as identity) for parallel
  composition $|$ and for summation $+$.
\end{definition}

\subsection{Name equivalence}

We take name equivalence, written $\nameeq$, to be the smallest
equivalence relation generated by the following rules.

\begin{mathpar}
\inferrule*[lab=Quote-drop]
{ }
{ \quotep{@{x}} \nameeq x }

\inferrule*[lab=Struct-equiv]
{ P \scong Q }
{ \quotep{P} \nameeq \quotep{Q} }
\end{mathpar}

The astute reader will have noticed that the mutual recursion of names
and processes imposes a mutual recursion on alpha-equivalence and
structural equivalence via name-equivalence. Fortunately, all of this
works out pleasantly and we may calculate in the natural way, free of
concern. The reader interested in the details is referred to the
appendix \ref{appendix:rho_details}.

\subsection{Substitution}

We use $\Proc$ for the set of processes, $\QProc$ for the set of
names, and $\id{\{}\vec{y} / \vec{x} \id{\}}$ to denote partial maps,
$s : \QProc \rightarrow \QProc$. A map, $s$ lifts, uniquely, to a map
on process terms, $\widehat{s} : \Proc \rightarrow \Proc$ by the
following equations.

\begin{mathpar}
  (0) \psubstp{Q}{P} := 0 \\
  (R \juxtap S) \psubstp{Q}{P}
  :=    
  (R)\psubstp{Q}{P} \juxtap (S) \psubstp{Q}{P} \\
  (x?(y).R) \psubstp{Q}{P}    
  :=    
  (x)\substp{Q}{P} (z)\concat( (R \psubstn{z}{y}) \psubstp{Q}{P} ) \\
  (\lift{x}{R}) \psubstp{Q}{P}  
  :=
  \lift{(x)\substp{Q}{P}}{ R \psubstp{Q}{P} } \\
%   (\dropn{x})  \psubstp{Q}{P}       
%   := 
%   \left\{ 
%     \begin{array}{ccc} 
%       \dropn{\quotep{Q}} & & x \nameeq \quotep{P} \\
%       \dropn{x} & & otherwise \\
%     \end{array}
%   \right. 
  (\dropn{x})  \psubstp{Q}{P}       
  := 
  \left\{ 
    \begin{array}{ccc} 
      Q & & x \nameeq \quotep{P} \\
      \dropn{x} & & otherwise \\
    \end{array}
  \right.
\end{mathpar}
 

where

\begin{eqnarray}
  (x)\id{\{} \lpquote Q \rpquote / \lpquote P \rpquote \id{\}}            = 
  \left\{ 
    \begin{array}{ccc}
      \lpquote Q \rpquote & & x \nameeq \lpquote P \rpquote \\
      x & & otherwise \\
    \end{array}
  \right. \nonumber
\end{eqnarray}

and $z$ is chosen distinct from $\quotep{P}$, $\quotep{Q}$, the free
names in $Q$, and all the names in $R$. Our $\alpha$-equivalence will
be built in the standard way from this substitution.

\begin{remark}\label{rem:no_self_referential_names}
  One consequence of these definitions is that $\forall P. \quotep{P}
  \not\in \freenames{P}$.
\end{remark}

\subsection{ Dynamic quote: an example }

Anticipating something of what's to come, consider applying the
substitution, $\widehat{\id{\{}u / z \id{\}}}$, to the following pair
of processes, $\lift{w}{y!(z)}$ and $w[ \lpquote y!(z) \rpquote ]$.

\begin{eqnarray}
	\lift{w}{y!(z)}\widehat{\id{\{}u / z \id{\}}}
		& = &
		\lift{w}{y!(u)} \nonumber\\
	w[ \lpquote y!(z) \rpquote ] \widehat{ \id{\{}u / z \id{\}} }
		& = &
		w[ \lpquote y!(z) \rpquote ] \nonumber
\end{eqnarray}

Because the body of the process between quotes is impervious to
substitution, we get radically different answers. In fact, by
examining the first process in an input context,
e.g. $x?(z).\lift{w}{y!(z)}$, we see that the process under the lift
operator may be shaped by prefixed inputs binding a name inside it. In
this sense, the lift operator will be seen as a way to dynamically
construct processes before reifying them as names.

Finally equipped with these standard features we can present the
dynamics of the calculus.

\subsubsection{Operational semantics} 

Finally, we introduce the computational dynamics. What marks these
algebras as distinct from other more traditionally studied algebraic
structures, e.g. vector spaces or polynomial rings, is the manner in
which dynamics is captured. In traditional structures, dynamics is typically
expressed through morphisms between such structures, as in linear maps
between vector spaces or morphisms between rings. In algebras
associated with the semantics of computation, the dynamics is
expressed as part of the algebraic structure itself, through a
reduction reduction relation typically denoted by $\red$. Below, we
give a recursive presentation of this relation for the calculus used
in the encoding.

$\red \subseteq \pi \times \pi$
$\red : \pi \to \mathcal{P}(\pi)$

\begin{mathpar}
  \inferrule* [lab=Comm] { \textsf{match}( x_{src}, x_{trgt} ) } { x_{trgt}?(y)P \; | \; x_{src}!\langle {Q} \rangle \red P\{\quotep{Q}/y}\} }
  \and \\
  \inferrule* [lab=Par] {{P} \red {P}'} {{{P} | {Q}} \red {{P}' | {Q}}}
  \and
  \inferrule* [lab=Equiv]{{{P} \scong {P}'} \andalso {{P}' \red {Q}'} \andalso {{Q}' \scong {Q}}}{{P} \red {Q}}
\end{mathpar}

\begin{eqnarray*}
  match_{\equiv} (\quotep{P},\quotep{Q}) & := & P \equiv Q \\
  match_{\dagger}(\quotep{P},\quotep{Q}) & := & \forall R. P|Q \red^{*} R => R \red^{*} 0 \\
  match_{K}(\quotep{P},\quotep{Q}) & := & K \mbox{ for some context } K
\end{eqnarray*}

$u?(x)P | u!\langle Q \rangle \red P\{\quotep{Q}/x\}$

%We write $\wred$ for $\red^*$, and $P\red$ if $\exists Q $ such that $ P \red Q$.
We write $P\red$ if $\exists Q $ such that $ P \red Q$ and $P\not\red$, otherwise.

\section{Replication}

As mentioned before, it is known that replication (and hence
recursion) can be implemented in a higher-order process algebra
\cite{SangiorgiWalker}. As our first example of calculation with the
machinery thus far presented we give the construction explicitly in
the {\rhoc}.

\begin{eqnarray}
	D_{x} & := & \prefix{x}{y}{(\binpar{\outputp{x}{y}}{@{y}})} \nonumber\\
	\bangp_{x}{P} & := & \binpar{{x}!\langle{\binpar{D_{x}}{P}}\rangle}{D_{x}} \nonumber
\end{eqnarray}

\begin{eqnarray}
	\bangp_{x}{P} & & \nonumber\\
	=
	& {x}!\langle{(\prefix{x}{y}{(\outputp{x}{y} | @{y})) | P}}\rangle 
	      | \prefix{x}{y}{(\outputp{x}{y} | @{y})} & \nonumber\\
	\red
	& (\outputp{x}{y} | @{y})\substn{\quotep{(\prefix{x}{y}{(@{y} | \outputp{x}{y})) | P}}}{y} & \nonumber\\
	=
	& \outputp{x}{\quotep{(\prefix{x}{y}{(\outputp{x}{y} | @{y})) | P}}}
	  | {(\prefix{x}{y}{(\outputp{x}{y} | @{y})) | P}} & \nonumber\\
	\red
	& \ldots & \nonumber\\
	\red^*
	& P | P | \ldots & \nonumber
\end{eqnarray}

Of course, this encoding, as an implementation, runs away, unfolding
$\bangp{P}$ eagerly. A lazier and more implementable replication
operator, restricted to input-guarded processes, may be obtained as follows.

\begin{eqnarray}
\bangp{\prefix{u}{v}{P}} 
	:= 
	\binpar{\lift{x}{\prefix{u}{v}{(\binpar{D(x)}{P})}}}{D(x)} \nonumber
\end{eqnarray}

\begin{remark}
  Note that the lazier definition still does not deal with summation
  or mixed summation (i.e. sums over input and output). The reader is
  invited to construct definitions of replication that deal with these
  features. 

  Further, the definitions are parameterized in a name, $x$. Can you,
  gentle reader, make a definition that eliminates this parameter and
  guarantees no accidental interaction between the replication
  machinery and the process being replicated -- i.e. no accidental
  sharing of names used by the process to get its work done and the
  name(s) used by the replication to effect copying. This latter
  revision of the definition of replication is crucial to obtaining
  the expected identity $!!P \sim !P$.
\end{remark}

\begin{remark}\label{rem:paradoxical_combinator}
  The reader familiar with the lambda calculus will have noticed the
  similarity between $D$ and the paradoxical combinator.

  [Ed. note: the existence of this seems to suggest we have to be more
  restrictive on the set of processes and names we admit if we are to
  support no-cloning.]
\end{remark}

\subsubsection{Bisimulation}

The computational dynamics gives rise to another kind of equivalence,
the equivalence of computational behavior. As previously mentioned
this is typically captured \emph{via} some form of bisimulation.

% The notion we use in this paper is weak barbed bisimulation
% \cite{milner91polyadicpi}.

The notion we use in this paper is derived from weak barbed
bisimulation \cite{milner91polyadicpi}. 

\begin{definition}
An \emph{observation relation}, $\downarrow_{\mathcal N}$, over a set
of names, $\mathcal N$, is the smallest relation satisfying the rules
below.

\infrule[Out-barb]{y \in {\mathcal N}, \; x \nameeq y}
		  {\outputp{x}{v} \downarrow_{\mathcal N} x}
\infrule[Par-barb]{\mbox{$P\downarrow_{\mathcal N} x$ or $Q\downarrow_{\mathcal N} x$}}
		  {\binpar{P}{Q} \downarrow_{\mathcal N} x}

We write $P \Downarrow_{\mathcal N} x$ if there is $Q$ such that 
$P \wred Q$ and $Q \downarrow_{\mathcal N} x$.
\end{definition}

\begin{definition}
%\label{def.bbisim}
An  ${\mathcal N}$-\emph{barbed bisimulation} over a set of names, ${\mathcal N}$, is a symmetric binary relation 
${\mathcal S}_{\mathcal N}$ between agents such that $P\rel{S}_{\mathcal N}Q$ implies:
\begin{enumerate}
\item If $P \red P'$ then $Q \wred Q'$ and $P'\rel{S}_{\mathcal N} Q'$.
\item If $P\downarrow_{\mathcal N} x$, then $Q\Downarrow_{\mathcal N} x$.
\end{enumerate}
$P$ is ${\mathcal N}$-barbed bisimilar to $Q$, written
$P \wbbisim_{\mathcal N} Q$, if $P \rel{S}_{\mathcal N} Q$ for some ${\mathcal N}$-barbed bisimulation ${\mathcal S}_{\mathcal N}$.
\end{definition}

$\mathcal{R} \subseteq \pi \times \pi$

$P \mathcal{R} Q => \forall P'. P \red P' \Rightarrow \exists Q'. Q \red Q', P' \mathcal{R} Q'$

$P \vdash x \Rightarrow Q \vdash x$

\begin{mathpar}
  \inferrule*[lab=Out-barb]{x \nameeq y}{{y}!\langle{Q}\rangle \vdash x}
  \and
  \inferrule*[lab=Par-barb]{\mbox{$P\vdash x$ or $Q\vdash x$}}{\binpar{P}{Q} \vdash x}
\end{mathpar}

\subsubsection{Contexts}

One of the principle advantages of computational calculi like the
$\pi$-calculus is a well-defined notion of context,
contextual-equivalence and a correlation between
contextual-equivalence and notions of bisimulation. The notion of
context allows the decomposition of a process into (sub-)process and
its syntactic environment, its context. Thus, a context may be
thought of as a process with a ``hole'' (written $\Box$) in it. The
application of a context $M$ to a process $P$, written $M[P]$, is
tantamount to filling the hole in $M$ with $P$. In this paper we do
not need the full weight of this theory, but do make use of the notion
of context in the proof the main theorem. 

\begin{mathpar}
  \inferrule* [lab=summation] {} {{M_{M},M_{N}} \bc \Box \;|\; x.M_{A} \;|\; M_{M}+M_{N}}
  \and
  \inferrule* [lab=agent] {} {{M_{A}} \bc (\vec{x})M_{P} \;| \; \clift{P_0,\ldots,M_{P},\ldots,P_N}}
  \and \\
  \inferrule* [lab=process] {} {{M_{P}} \bc M_{N} \;| \;P|M_{P} }
\end{mathpar} 

\begin{mathpar}
  \inferrule* [lab=sychronization] {} {M_{N} \bc \Box \;|\; x?M_{F} \;|\; x!M_{C}}
  \and
  \inferrule* [lab=abstraction] {} {{M_{F}} \bc (x)M_{P} }
  \and
  \inferrule* [lab=concretion] {} {{M_{C}} \bc \langle M_{P} \rangle }
  \and \\
  \inferrule* [lab=process] {} {{M_{P}} \bc M_{N} \;| \;P|M_{P} }
\end{mathpar}

\begin{definition}[contextual application] Given a context $M$, and
  process $P$, we define the \emph{contextual application}, $M[P] :=
  M\{P/\Box\}$. That is, the contextual application of M to P is the
  substitution of $P$ for $\Box$ in $M$.
\end{definition}

$\meaningof{-} : L \to \mathcal{P}(\pi)$

\begin{mathpar}
  \inferrule* [lab=collection] {} {\meaningof{true} = \pi, \and \meaningof{~E} = \pi \setminus \meaningof{E}, \and \meaningof{E_{1} \& E_{2}} = \meaningof{E_{1}} \cap \meaningof{E_{2}}}
\end{mathpar}

\begin{mathpar}
  \inferrule* [lab=structure] {} {\meaningof{0} = \{ P \in \pi | P \equiv 0 \}, \and \\ \meaningof{E_1 | E_2} = \{ P \in \pi | P \equiv P_{1} | P_{2}, P_{1} \in \meaningof{E_{1}}, P_{2} \in \meaningof{E_2}\} }
\end{mathpar}

\begin{mathpar}
 \inferrule* [lab=behavior] {} {\meaningof{\langle a?b \rangle E} = \{ P \in \pi | P \equiv Q | u?(y)P', \\ \and \\\\ \and \\ \;\;\; u \in \meaningof{a}, \forall z.P'\{z/y\} \in \meaningof{E\{z/b\}}\}, \and \\ \meaningof{a!E} = \{ P \in \pi | P \equiv Q | x!\langle P' \rangle, x \in \meaningof{a} P' \in \meaningof{E}\} }
\end{mathpar}

\begin{mathpar}
 \inferrule* [lab=nominal] {} {\meaningof{\quotep{E}} = \{ \quotep{P} \in \quotep{\pi} | P \in \meaningof{E} \}, \and \meaningof{\quotep{P}} = \{ \quotep{Q} \in \quotep{\pi} | P \equiv Q \} \and \\ \meaningof{@\quotep{E}} = \{ P \in \pi | P \equiv @x, x \in \meaningof{E} \}}
\end{mathpar}

\begin{eqnarray*}
  \\
  \meaningof{-} : TS \to ST
\end{eqnarray*}

\begin{eqnarray*}
  \\
  L : TS \to ST
\end{eqnarray*}

\begin{eqnarray*}
  \\
  P \models E \iff P \in \meaningof{E}
\end{eqnarray*}

\begin{eqnarray*}
  P \approx_{L} Q \iff \forall E \in L. P \models E \iff Q \models E
\end{eqnarray*}

\begin{eqnarray*}
  P \approx_{K} Q
\end{eqnarray*}

\begin{eqnarray*}
  P \approx Q
\end{eqnarray*}

$\approx_{K} = \approx = \approx_{L}$

\subsubsection{Contextual duality}

Note that contexts extend the quotation operation to a family of
operations from processes to names. Given a context, $M$, we can
define a \emph{nominal context}, $\quotep{M}$ by $\quotep{M}[P] :=
\quotep{M[P]}$. To foreshadow what is to come we observe that these
operations enjoy a duality with processes very much like the duality
between vectors and maps from vectors to scalars.

Further, because the calculus is essentially higher-order, we have a
correspondence between contexts and processes. More specifically,
given a name $x$ and a context $M$ we can construct $M^{*}_{x}$ such
that 

\begin{mathpar}
  M^{*}_{x} | \lift{x}{P} \red M[P]
\end{mathpar}

namely,

\begin{mathpar}
  M^{*}_{x} := x?(u).M[\dropn{u}]
\end{mathpar}

The dependence of $M^{*}_{x}$ on a name makes it an abstraction, 

\begin{mathpar}
  M^{*} := (x)x?(u).M[\dropn{u}]
\end{mathpar}

\subsection{Additional notation}

It will sometimes be convenient to denote the process a name
quotes. We already have the notation $x = \quotep{P}$, but it will be
convenient to introduce an alternate notation, $\procn{x}$, when we
want to emphasize the connection to the use of the name. Note that, by
virtue of name equivalence, $\quotep{\procn{x}} \nameeq x$; so, the
notation is consistent with previous definitions.

Further, because names have structure it is possible to effect
substitutions on the basis of that structure. This means we need to
upgrade our notation for substitutions, which we accomplish by
adapting comprehension notation. Thus,

\begin{mathpar}
  P\{ y / x : x \in S \}
\end{mathpar}

is interpreted to mean the process derived from P by replacing (in a
capture-avoiding manner) each occurrence of $x$ in $S$ by $y$. For example,

\begin{mathpar}
  P\{ \quotep{\procn{x}|\procn{x}} / x : x \in \freenames{P} \}
\end{mathpar}

will replace each (occurrence) of a free name $x$ in $P$ by
$\quotep{\procn{x}|\procn{x}}$.

Also, we will avail ourselves of the notation $x^{L}$ and $x^{R}$ to
denote injections of a name into disjoint copies of the name
space. There are numerous ways to accomplish this. One example can be
found in \cite{MeredithR05}. This notation overloads to vectors of
names: $\vec{x}^{\pi} := (x_{i}^{\pi} \; : \; 0 \leq i < |\vec{x}| )$ where $\pi \in \{L,R\}$.

We also use $P^{\Box} := P|\Box$.

In \cite{MeredithR05} an interpretation of the new operator is
given. It turns out that there are several possible interpretations
all enjoying the requisite algebraic properties of the operator (see
\cite{milner91polyadicpi}). We will therefore make liberal use of
$(\nu\; \vec{x})P$.

% subsection the_syntax_and_semantics_of_the_notation_system (end)   

\input{qm2pi.qmops} 

\input{qm2pi.sterngerlach} 

\input{qm2pi.metric} 

% section concurrent_process_calculi (end)

%\input{qm2pi.proofsketch}

% section proof sketch (end)

%\input{qm2pi.slviaknots} 

% section spatial logic via knots (end)

\input{qm2pi.conclusion}

% section conclusion (end)

%\input{qm2pi.dtcodes} 

% section wiring algorithm (end)

\input{qm2pi.ack} 

% section acknowledgments (end)

\newpage


\bibliographystyle{plain}   
\bibliography{../../biblios/main.bib}

\input{qm2pi.rhodetails}

\end{document}

 

% section notation (end)

\input{qm2pi.process.calculi} 

% section concurrent_process_calculi_and_spatial_logics_ (end)
    
%\documentclass[12pt]{llncs}
%\documentclass{jktr}

\usepackage[pdftex]{hyperref}                   
\usepackage {listings}
\usepackage {mathpartir}
\usepackage{bcprules}
%\usepackage{listings}
                       
\usepackage{graphicx} 
%\usepackage[margins=2.5cm,nohead,nofoot]{geometry}
%\usepackage{geometry}
\usepackage{amsfonts}
\usepackage{amstext}
\usepackage{latexsym}
\usepackage{amssymb}
\usepackage{color}


%\include{myPreamble}
\include{qm2pi.local} 

%\ifpdf
%\usepackage[pdftex]{graphicx}
%\else
%\usepackage{graphicx}
%\fi

 % \ifpdf
%  \usepackage{pdfsync}
%  \if


%\title{Brief Article}
%\author{David F. Snyder}
%\author{L.G. Meredith}

%\address{Dept. of Math., Texas State University--San Marcos, San Marcos, TX 78666}
       
\pagestyle{empty}


\begin{document}

\lstset{language=[Objective]Caml,frame=shadowbox}

\input{qm2pi.front}

% section front matter (end)

\input{qm2pi.intro} 
 
% section introduction (end)

% \input{qm2pi.knotations} 

% section notation (end)

\input{qm2pi.process.calculi} 

% section concurrent_process_calculi_and_spatial_logics_ (end)
    
%\input{qm2pi.knots2pi} 

%\input{qm2pi.trefoil} 

%\input{qm2pi.mainthm} 

% subsection basic_interpretation (end)

%\input{qm2pi.rho.presentation} 
\subsection{The syntax and semantics of the notation system}\label{sub:the_syntax_and_semantics_of_the_notation_system} % (fold)

We now summarize a technical presentation of the calculus that
embodies our theory of dynamics. The typical presentation of such a
calculus follows the style of giving generators and relations on
them. The grammar, below, describing term constructors, freely
generates the set of processes, $\Proc$. This set is then quotiented
by a relation known as structural congruence and it is over this set
that the notion of dynamics is expressed. This presentation is
essentially that of \cite{MeredithR05} with the addition of
polyadicity and summation. For readability we have relegated some of
the technical subtleties to an appendix.

\subsubsection{Process grammar}\label{subsub:process_grammar}

\begin{mathpar}
  \inferrule* [lab=synchronization] {} {{M} \bc \pzero \;|\; x?F \;|\; x!C }
  \and
  \inferrule* [lab=abstraction] {} {{F} \bc (x)P}
  \and
  \inferrule* [lab=concretion] {} {{C} \bc \langle Q \rangle}
  \and
  \inferrule* [lab=process] {} {{P,Q} \bc M \;| \;P|Q \;|\; @{x}}
  \and
  \inferrule* [lab=name] {} {{x} \bc \quotep{P}}
\end{mathpar} 

Note that $\vec{x}$ (resp. $\vec{P}$) denotes a vector of names
(resp. processes) of length $|\vec{x}|$ (resp. $|\vec{P}|$). We adopt
the following useful abbreviations.

\begin{mathpar}
   x?(\vec{y}).P := x.(\vec{y})P \and  x\clift{\vec{P}} := x.\clift{\vec{P}}
   \and x!(y) := \lift{x}{\dropn{y}}
   \and \Pi_{i=0}^{n-1}P_i := P_0 | \ldots | P_{n-1}
\end{mathpar}

\subsubsection{Structural congruence}

\paragraph{Free and bound names and alpha-equivalence.} At the
core of structural equivalence is alpha-equivalence which identifies
process that are the same up to a change of variable. Formally, we
recognize the distinction between free and bound names. The free names
of a process, $\freenames{P}$, may be calculated recursively as
follows:

\begin{mathpar}
\freenames{\pzero} := \emptyset
  \and \\
  \freenames{x?(y).P} := \{ x \} \cup (\freenames{P} \setminus \{ y \})
  \and 
  \freenames{x!\langle P \rangle} := \{ x \} \cup \{ P \} 
  \and \\
  \freenames{P|Q} := \freenames{P} \cup \freenames{Q}
  \and \\
  \freenames{@{x}} := \{ x \}
\end{mathpar}

$\pi$
$\quotep{\pi}$

$\freenames{-} : \pi \to \mathcal{P}(\quotep{\pi})$

\begin{eqnarray*}
  \freenames{\pzero} & := & \emptyset \\
  \freenames{x?(y).P} & := & \{ x \} \cup (\freenames{P} \setminus \{ y \}) \\
  \freenames{x!\langle P \rangle} & := & \{ x \} \cup \{ P \} \\
  \freenames{P|Q} & := & \freenames{P} \cup \freenames{Q} \\
  \freenames{\dropn{x}} & := & \{ x \}
\end{eqnarray*}

The bound names of a process, $\boundnames{P}$, are those names occurring in $P$
that are not free. For example, in $x?(y).0$, the name $x$ is free, while $y$ is bound.

\begin{mathpar}
  \inferrule* [lab=monoidal-laws] {} { P|Q \equiv Q|P \and P|0 \equiv P \and P|(Q|R) \equiv (P|Q)|R }
\end{mathpar}

\begin{mathpar}
  \inferrule* [lab=alpha-equivalence] {} { (x)P \equiv (y)P\{y/x\} \and y \not\in \freenames{P} }
\end{mathpar}

\begin{definition}
Then two processes, $P,Q$, are alpha-equivalent if $P = Q\{\vec{y}/\vec{x}\}$ for
some $\vec{x} \in \boundnames{Q},\vec{y} \in \boundnames{P}$, where $Q\{\vec{y}/\vec{x}\}$
denotes the capture-avoiding substitution of $\vec{y}$ for $\vec{x}$ in $Q$.
\end{definition}

\begin{definition}
  The {\em structural congruence} \cite{SangiorgiWalker} , $\equiv$,
  between processes is the least congruence containing
  alpha-equivalence, satisfying the abelian monoid laws
  (associativity, commutativity and $\pzero$ as identity) for parallel
  composition $|$ and for summation $+$.
\end{definition}

\subsection{Name equivalence}

We take name equivalence, written $\nameeq$, to be the smallest
equivalence relation generated by the following rules.

\begin{mathpar}
\inferrule*[lab=Quote-drop]
{ }
{ \quotep{@{x}} \nameeq x }

\inferrule*[lab=Struct-equiv]
{ P \scong Q }
{ \quotep{P} \nameeq \quotep{Q} }
\end{mathpar}

The astute reader will have noticed that the mutual recursion of names
and processes imposes a mutual recursion on alpha-equivalence and
structural equivalence via name-equivalence. Fortunately, all of this
works out pleasantly and we may calculate in the natural way, free of
concern. The reader interested in the details is referred to the
appendix \ref{appendix:rho_details}.

\subsection{Substitution}

We use $\Proc$ for the set of processes, $\QProc$ for the set of
names, and $\id{\{}\vec{y} / \vec{x} \id{\}}$ to denote partial maps,
$s : \QProc \rightarrow \QProc$. A map, $s$ lifts, uniquely, to a map
on process terms, $\widehat{s} : \Proc \rightarrow \Proc$ by the
following equations.

\begin{mathpar}
  (0) \psubstp{Q}{P} := 0 \\
  (R \juxtap S) \psubstp{Q}{P}
  :=    
  (R)\psubstp{Q}{P} \juxtap (S) \psubstp{Q}{P} \\
  (x?(y).R) \psubstp{Q}{P}    
  :=    
  (x)\substp{Q}{P} (z)\concat( (R \psubstn{z}{y}) \psubstp{Q}{P} ) \\
  (\lift{x}{R}) \psubstp{Q}{P}  
  :=
  \lift{(x)\substp{Q}{P}}{ R \psubstp{Q}{P} } \\
%   (\dropn{x})  \psubstp{Q}{P}       
%   := 
%   \left\{ 
%     \begin{array}{ccc} 
%       \dropn{\quotep{Q}} & & x \nameeq \quotep{P} \\
%       \dropn{x} & & otherwise \\
%     \end{array}
%   \right. 
  (\dropn{x})  \psubstp{Q}{P}       
  := 
  \left\{ 
    \begin{array}{ccc} 
      Q & & x \nameeq \quotep{P} \\
      \dropn{x} & & otherwise \\
    \end{array}
  \right.
\end{mathpar}
 

where

\begin{eqnarray}
  (x)\id{\{} \lpquote Q \rpquote / \lpquote P \rpquote \id{\}}            = 
  \left\{ 
    \begin{array}{ccc}
      \lpquote Q \rpquote & & x \nameeq \lpquote P \rpquote \\
      x & & otherwise \\
    \end{array}
  \right. \nonumber
\end{eqnarray}

and $z$ is chosen distinct from $\quotep{P}$, $\quotep{Q}$, the free
names in $Q$, and all the names in $R$. Our $\alpha$-equivalence will
be built in the standard way from this substitution.

\begin{remark}\label{rem:no_self_referential_names}
  One consequence of these definitions is that $\forall P. \quotep{P}
  \not\in \freenames{P}$.
\end{remark}

\subsection{ Dynamic quote: an example }

Anticipating something of what's to come, consider applying the
substitution, $\widehat{\id{\{}u / z \id{\}}}$, to the following pair
of processes, $\lift{w}{y!(z)}$ and $w[ \lpquote y!(z) \rpquote ]$.

\begin{eqnarray}
	\lift{w}{y!(z)}\widehat{\id{\{}u / z \id{\}}}
		& = &
		\lift{w}{y!(u)} \nonumber\\
	w[ \lpquote y!(z) \rpquote ] \widehat{ \id{\{}u / z \id{\}} }
		& = &
		w[ \lpquote y!(z) \rpquote ] \nonumber
\end{eqnarray}

Because the body of the process between quotes is impervious to
substitution, we get radically different answers. In fact, by
examining the first process in an input context,
e.g. $x?(z).\lift{w}{y!(z)}$, we see that the process under the lift
operator may be shaped by prefixed inputs binding a name inside it. In
this sense, the lift operator will be seen as a way to dynamically
construct processes before reifying them as names.

Finally equipped with these standard features we can present the
dynamics of the calculus.

\subsubsection{Operational semantics} 

Finally, we introduce the computational dynamics. What marks these
algebras as distinct from other more traditionally studied algebraic
structures, e.g. vector spaces or polynomial rings, is the manner in
which dynamics is captured. In traditional structures, dynamics is typically
expressed through morphisms between such structures, as in linear maps
between vector spaces or morphisms between rings. In algebras
associated with the semantics of computation, the dynamics is
expressed as part of the algebraic structure itself, through a
reduction reduction relation typically denoted by $\red$. Below, we
give a recursive presentation of this relation for the calculus used
in the encoding.

$\red \subseteq \pi \times \pi$
$\red : \pi \to \mathcal{P}(\pi)$

\begin{mathpar}
  \inferrule* [lab=Comm] { \textsf{match}( x_{src}, x_{trgt} ) } { x_{trgt}?(y)P \; | \; x_{src}!\langle {Q} \rangle \red P\{\quotep{Q}/y}\} }
  \and \\
  \inferrule* [lab=Par] {{P} \red {P}'} {{{P} | {Q}} \red {{P}' | {Q}}}
  \and
  \inferrule* [lab=Equiv]{{{P} \scong {P}'} \andalso {{P}' \red {Q}'} \andalso {{Q}' \scong {Q}}}{{P} \red {Q}}
\end{mathpar}

\begin{eqnarray*}
  match_{\equiv} (\quotep{P},\quotep{Q}) & := & P \equiv Q \\
  match_{\dagger}(\quotep{P},\quotep{Q}) & := & \forall R. P|Q \red^{*} R => R \red^{*} 0 \\
  match_{K}(\quotep{P},\quotep{Q}) & := & K \mbox{ for some context } K
\end{eqnarray*}

$u?(x)P | u!\langle Q \rangle \red P\{\quotep{Q}/x\}$

%We write $\wred$ for $\red^*$, and $P\red$ if $\exists Q $ such that $ P \red Q$.
We write $P\red$ if $\exists Q $ such that $ P \red Q$ and $P\not\red$, otherwise.

\section{Replication}

As mentioned before, it is known that replication (and hence
recursion) can be implemented in a higher-order process algebra
\cite{SangiorgiWalker}. As our first example of calculation with the
machinery thus far presented we give the construction explicitly in
the {\rhoc}.

\begin{eqnarray}
	D_{x} & := & \prefix{x}{y}{(\binpar{\outputp{x}{y}}{@{y}})} \nonumber\\
	\bangp_{x}{P} & := & \binpar{{x}!\langle{\binpar{D_{x}}{P}}\rangle}{D_{x}} \nonumber
\end{eqnarray}

\begin{eqnarray}
	\bangp_{x}{P} & & \nonumber\\
	=
	& {x}!\langle{(\prefix{x}{y}{(\outputp{x}{y} | @{y})) | P}}\rangle 
	      | \prefix{x}{y}{(\outputp{x}{y} | @{y})} & \nonumber\\
	\red
	& (\outputp{x}{y} | @{y})\substn{\quotep{(\prefix{x}{y}{(@{y} | \outputp{x}{y})) | P}}}{y} & \nonumber\\
	=
	& \outputp{x}{\quotep{(\prefix{x}{y}{(\outputp{x}{y} | @{y})) | P}}}
	  | {(\prefix{x}{y}{(\outputp{x}{y} | @{y})) | P}} & \nonumber\\
	\red
	& \ldots & \nonumber\\
	\red^*
	& P | P | \ldots & \nonumber
\end{eqnarray}

Of course, this encoding, as an implementation, runs away, unfolding
$\bangp{P}$ eagerly. A lazier and more implementable replication
operator, restricted to input-guarded processes, may be obtained as follows.

\begin{eqnarray}
\bangp{\prefix{u}{v}{P}} 
	:= 
	\binpar{\lift{x}{\prefix{u}{v}{(\binpar{D(x)}{P})}}}{D(x)} \nonumber
\end{eqnarray}

\begin{remark}
  Note that the lazier definition still does not deal with summation
  or mixed summation (i.e. sums over input and output). The reader is
  invited to construct definitions of replication that deal with these
  features. 

  Further, the definitions are parameterized in a name, $x$. Can you,
  gentle reader, make a definition that eliminates this parameter and
  guarantees no accidental interaction between the replication
  machinery and the process being replicated -- i.e. no accidental
  sharing of names used by the process to get its work done and the
  name(s) used by the replication to effect copying. This latter
  revision of the definition of replication is crucial to obtaining
  the expected identity $!!P \sim !P$.
\end{remark}

\begin{remark}\label{rem:paradoxical_combinator}
  The reader familiar with the lambda calculus will have noticed the
  similarity between $D$ and the paradoxical combinator.

  [Ed. note: the existence of this seems to suggest we have to be more
  restrictive on the set of processes and names we admit if we are to
  support no-cloning.]
\end{remark}

\subsubsection{Bisimulation}

The computational dynamics gives rise to another kind of equivalence,
the equivalence of computational behavior. As previously mentioned
this is typically captured \emph{via} some form of bisimulation.

% The notion we use in this paper is weak barbed bisimulation
% \cite{milner91polyadicpi}.

The notion we use in this paper is derived from weak barbed
bisimulation \cite{milner91polyadicpi}. 

\begin{definition}
An \emph{observation relation}, $\downarrow_{\mathcal N}$, over a set
of names, $\mathcal N$, is the smallest relation satisfying the rules
below.

\infrule[Out-barb]{y \in {\mathcal N}, \; x \nameeq y}
		  {\outputp{x}{v} \downarrow_{\mathcal N} x}
\infrule[Par-barb]{\mbox{$P\downarrow_{\mathcal N} x$ or $Q\downarrow_{\mathcal N} x$}}
		  {\binpar{P}{Q} \downarrow_{\mathcal N} x}

We write $P \Downarrow_{\mathcal N} x$ if there is $Q$ such that 
$P \wred Q$ and $Q \downarrow_{\mathcal N} x$.
\end{definition}

\begin{definition}
%\label{def.bbisim}
An  ${\mathcal N}$-\emph{barbed bisimulation} over a set of names, ${\mathcal N}$, is a symmetric binary relation 
${\mathcal S}_{\mathcal N}$ between agents such that $P\rel{S}_{\mathcal N}Q$ implies:
\begin{enumerate}
\item If $P \red P'$ then $Q \wred Q'$ and $P'\rel{S}_{\mathcal N} Q'$.
\item If $P\downarrow_{\mathcal N} x$, then $Q\Downarrow_{\mathcal N} x$.
\end{enumerate}
$P$ is ${\mathcal N}$-barbed bisimilar to $Q$, written
$P \wbbisim_{\mathcal N} Q$, if $P \rel{S}_{\mathcal N} Q$ for some ${\mathcal N}$-barbed bisimulation ${\mathcal S}_{\mathcal N}$.
\end{definition}

$\mathcal{R} \subseteq \pi \times \pi$

$P \mathcal{R} Q => \forall P'. P \red P' \Rightarrow \exists Q'. Q \red Q', P' \mathcal{R} Q'$

$P \vdash x \Rightarrow Q \vdash x$

\begin{mathpar}
  \inferrule*[lab=Out-barb]{x \nameeq y}{{y}!\langle{Q}\rangle \vdash x}
  \and
  \inferrule*[lab=Par-barb]{\mbox{$P\vdash x$ or $Q\vdash x$}}{\binpar{P}{Q} \vdash x}
\end{mathpar}

\subsubsection{Contexts}

One of the principle advantages of computational calculi like the
$\pi$-calculus is a well-defined notion of context,
contextual-equivalence and a correlation between
contextual-equivalence and notions of bisimulation. The notion of
context allows the decomposition of a process into (sub-)process and
its syntactic environment, its context. Thus, a context may be
thought of as a process with a ``hole'' (written $\Box$) in it. The
application of a context $M$ to a process $P$, written $M[P]$, is
tantamount to filling the hole in $M$ with $P$. In this paper we do
not need the full weight of this theory, but do make use of the notion
of context in the proof the main theorem. 

\begin{mathpar}
  \inferrule* [lab=summation] {} {{M_{M},M_{N}} \bc \Box \;|\; x.M_{A} \;|\; M_{M}+M_{N}}
  \and
  \inferrule* [lab=agent] {} {{M_{A}} \bc (\vec{x})M_{P} \;| \; \clift{P_0,\ldots,M_{P},\ldots,P_N}}
  \and \\
  \inferrule* [lab=process] {} {{M_{P}} \bc M_{N} \;| \;P|M_{P} }
\end{mathpar} 

\begin{mathpar}
  \inferrule* [lab=sychronization] {} {M_{N} \bc \Box \;|\; x?M_{F} \;|\; x!M_{C}}
  \and
  \inferrule* [lab=abstraction] {} {{M_{F}} \bc (x)M_{P} }
  \and
  \inferrule* [lab=concretion] {} {{M_{C}} \bc \langle M_{P} \rangle }
  \and \\
  \inferrule* [lab=process] {} {{M_{P}} \bc M_{N} \;| \;P|M_{P} }
\end{mathpar}

\begin{definition}[contextual application] Given a context $M$, and
  process $P$, we define the \emph{contextual application}, $M[P] :=
  M\{P/\Box\}$. That is, the contextual application of M to P is the
  substitution of $P$ for $\Box$ in $M$.
\end{definition}

$\meaningof{-} : L \to \mathcal{P}(\pi)$

\begin{mathpar}
  \inferrule* [lab=collection] {} {\meaningof{true} = \pi, \and \meaningof{~E} = \pi \setminus \meaningof{E}, \and \meaningof{E_{1} \& E_{2}} = \meaningof{E_{1}} \cap \meaningof{E_{2}}}
\end{mathpar}

\begin{mathpar}
  \inferrule* [lab=structure] {} {\meaningof{0} = \{ P \in \pi | P \equiv 0 \}, \and \\ \meaningof{E_1 | E_2} = \{ P \in \pi | P \equiv P_{1} | P_{2}, P_{1} \in \meaningof{E_{1}}, P_{2} \in \meaningof{E_2}\} }
\end{mathpar}

\begin{mathpar}
 \inferrule* [lab=behavior] {} {\meaningof{\langle a?b \rangle E} = \{ P \in \pi | P \equiv Q | u?(y)P', \\ \and \\\\ \and \\ \;\;\; u \in \meaningof{a}, \forall z.P'\{z/y\} \in \meaningof{E\{z/b\}}\}, \and \\ \meaningof{a!E} = \{ P \in \pi | P \equiv Q | x!\langle P' \rangle, x \in \meaningof{a} P' \in \meaningof{E}\} }
\end{mathpar}

\begin{mathpar}
 \inferrule* [lab=nominal] {} {\meaningof{\quotep{E}} = \{ \quotep{P} \in \quotep{\pi} | P \in \meaningof{E} \}, \and \meaningof{\quotep{P}} = \{ \quotep{Q} \in \quotep{\pi} | P \equiv Q \} \and \\ \meaningof{@\quotep{E}} = \{ P \in \pi | P \equiv @x, x \in \meaningof{E} \}}
\end{mathpar}

\begin{eqnarray*}
  \\
  \meaningof{-} : TS \to ST
\end{eqnarray*}

\begin{eqnarray*}
  \\
  L : TS \to ST
\end{eqnarray*}

\begin{eqnarray*}
  \\
  P \models E \iff P \in \meaningof{E}
\end{eqnarray*}

\begin{eqnarray*}
  P \approx_{L} Q \iff \forall E \in L. P \models E \iff Q \models E
\end{eqnarray*}

\begin{eqnarray*}
  P \approx_{K} Q
\end{eqnarray*}

\begin{eqnarray*}
  P \approx Q
\end{eqnarray*}

$\approx_{K} = \approx = \approx_{L}$

\subsubsection{Contextual duality}

Note that contexts extend the quotation operation to a family of
operations from processes to names. Given a context, $M$, we can
define a \emph{nominal context}, $\quotep{M}$ by $\quotep{M}[P] :=
\quotep{M[P]}$. To foreshadow what is to come we observe that these
operations enjoy a duality with processes very much like the duality
between vectors and maps from vectors to scalars.

Further, because the calculus is essentially higher-order, we have a
correspondence between contexts and processes. More specifically,
given a name $x$ and a context $M$ we can construct $M^{*}_{x}$ such
that 

\begin{mathpar}
  M^{*}_{x} | \lift{x}{P} \red M[P]
\end{mathpar}

namely,

\begin{mathpar}
  M^{*}_{x} := x?(u).M[\dropn{u}]
\end{mathpar}

The dependence of $M^{*}_{x}$ on a name makes it an abstraction, 

\begin{mathpar}
  M^{*} := (x)x?(u).M[\dropn{u}]
\end{mathpar}

\subsection{Additional notation}

It will sometimes be convenient to denote the process a name
quotes. We already have the notation $x = \quotep{P}$, but it will be
convenient to introduce an alternate notation, $\procn{x}$, when we
want to emphasize the connection to the use of the name. Note that, by
virtue of name equivalence, $\quotep{\procn{x}} \nameeq x$; so, the
notation is consistent with previous definitions.

Further, because names have structure it is possible to effect
substitutions on the basis of that structure. This means we need to
upgrade our notation for substitutions, which we accomplish by
adapting comprehension notation. Thus,

\begin{mathpar}
  P\{ y / x : x \in S \}
\end{mathpar}

is interpreted to mean the process derived from P by replacing (in a
capture-avoiding manner) each occurrence of $x$ in $S$ by $y$. For example,

\begin{mathpar}
  P\{ \quotep{\procn{x}|\procn{x}} / x : x \in \freenames{P} \}
\end{mathpar}

will replace each (occurrence) of a free name $x$ in $P$ by
$\quotep{\procn{x}|\procn{x}}$.

Also, we will avail ourselves of the notation $x^{L}$ and $x^{R}$ to
denote injections of a name into disjoint copies of the name
space. There are numerous ways to accomplish this. One example can be
found in \cite{MeredithR05}. This notation overloads to vectors of
names: $\vec{x}^{\pi} := (x_{i}^{\pi} \; : \; 0 \leq i < |\vec{x}| )$ where $\pi \in \{L,R\}$.

We also use $P^{\Box} := P|\Box$.

In \cite{MeredithR05} an interpretation of the new operator is
given. It turns out that there are several possible interpretations
all enjoying the requisite algebraic properties of the operator (see
\cite{milner91polyadicpi}). We will therefore make liberal use of
$(\nu\; \vec{x})P$.

% subsection the_syntax_and_semantics_of_the_notation_system (end)   

\input{qm2pi.qmops} 

\input{qm2pi.sterngerlach} 

\input{qm2pi.metric} 

% section concurrent_process_calculi (end)

%\input{qm2pi.proofsketch}

% section proof sketch (end)

%\input{qm2pi.slviaknots} 

% section spatial logic via knots (end)

\input{qm2pi.conclusion}

% section conclusion (end)

%\input{qm2pi.dtcodes} 

% section wiring algorithm (end)

\input{qm2pi.ack} 

% section acknowledgments (end)

\newpage


\bibliographystyle{plain}   
\bibliography{../../biblios/main.bib}

\input{qm2pi.rhodetails}

\end{document}

 

%\documentclass[12pt]{llncs}
%\documentclass{jktr}

\usepackage[pdftex]{hyperref}                   
\usepackage {listings}
\usepackage {mathpartir}
\usepackage{bcprules}
%\usepackage{listings}
                       
\usepackage{graphicx} 
%\usepackage[margins=2.5cm,nohead,nofoot]{geometry}
%\usepackage{geometry}
\usepackage{amsfonts}
\usepackage{amstext}
\usepackage{latexsym}
\usepackage{amssymb}
\usepackage{color}


%\include{myPreamble}
\include{qm2pi.local} 

%\ifpdf
%\usepackage[pdftex]{graphicx}
%\else
%\usepackage{graphicx}
%\fi

 % \ifpdf
%  \usepackage{pdfsync}
%  \if


%\title{Brief Article}
%\author{David F. Snyder}
%\author{L.G. Meredith}

%\address{Dept. of Math., Texas State University--San Marcos, San Marcos, TX 78666}
       
\pagestyle{empty}


\begin{document}

\lstset{language=[Objective]Caml,frame=shadowbox}

\input{qm2pi.front}

% section front matter (end)

\input{qm2pi.intro} 
 
% section introduction (end)

% \input{qm2pi.knotations} 

% section notation (end)

\input{qm2pi.process.calculi} 

% section concurrent_process_calculi_and_spatial_logics_ (end)
    
%\input{qm2pi.knots2pi} 

%\input{qm2pi.trefoil} 

%\input{qm2pi.mainthm} 

% subsection basic_interpretation (end)

%\input{qm2pi.rho.presentation} 
\subsection{The syntax and semantics of the notation system}\label{sub:the_syntax_and_semantics_of_the_notation_system} % (fold)

We now summarize a technical presentation of the calculus that
embodies our theory of dynamics. The typical presentation of such a
calculus follows the style of giving generators and relations on
them. The grammar, below, describing term constructors, freely
generates the set of processes, $\Proc$. This set is then quotiented
by a relation known as structural congruence and it is over this set
that the notion of dynamics is expressed. This presentation is
essentially that of \cite{MeredithR05} with the addition of
polyadicity and summation. For readability we have relegated some of
the technical subtleties to an appendix.

\subsubsection{Process grammar}\label{subsub:process_grammar}

\begin{mathpar}
  \inferrule* [lab=synchronization] {} {{M} \bc \pzero \;|\; x?F \;|\; x!C }
  \and
  \inferrule* [lab=abstraction] {} {{F} \bc (x)P}
  \and
  \inferrule* [lab=concretion] {} {{C} \bc \langle Q \rangle}
  \and
  \inferrule* [lab=process] {} {{P,Q} \bc M \;| \;P|Q \;|\; @{x}}
  \and
  \inferrule* [lab=name] {} {{x} \bc \quotep{P}}
\end{mathpar} 

Note that $\vec{x}$ (resp. $\vec{P}$) denotes a vector of names
(resp. processes) of length $|\vec{x}|$ (resp. $|\vec{P}|$). We adopt
the following useful abbreviations.

\begin{mathpar}
   x?(\vec{y}).P := x.(\vec{y})P \and  x\clift{\vec{P}} := x.\clift{\vec{P}}
   \and x!(y) := \lift{x}{\dropn{y}}
   \and \Pi_{i=0}^{n-1}P_i := P_0 | \ldots | P_{n-1}
\end{mathpar}

\subsubsection{Structural congruence}

\paragraph{Free and bound names and alpha-equivalence.} At the
core of structural equivalence is alpha-equivalence which identifies
process that are the same up to a change of variable. Formally, we
recognize the distinction between free and bound names. The free names
of a process, $\freenames{P}$, may be calculated recursively as
follows:

\begin{mathpar}
\freenames{\pzero} := \emptyset
  \and \\
  \freenames{x?(y).P} := \{ x \} \cup (\freenames{P} \setminus \{ y \})
  \and 
  \freenames{x!\langle P \rangle} := \{ x \} \cup \{ P \} 
  \and \\
  \freenames{P|Q} := \freenames{P} \cup \freenames{Q}
  \and \\
  \freenames{@{x}} := \{ x \}
\end{mathpar}

$\pi$
$\quotep{\pi}$

$\freenames{-} : \pi \to \mathcal{P}(\quotep{\pi})$

\begin{eqnarray*}
  \freenames{\pzero} & := & \emptyset \\
  \freenames{x?(y).P} & := & \{ x \} \cup (\freenames{P} \setminus \{ y \}) \\
  \freenames{x!\langle P \rangle} & := & \{ x \} \cup \{ P \} \\
  \freenames{P|Q} & := & \freenames{P} \cup \freenames{Q} \\
  \freenames{\dropn{x}} & := & \{ x \}
\end{eqnarray*}

The bound names of a process, $\boundnames{P}$, are those names occurring in $P$
that are not free. For example, in $x?(y).0$, the name $x$ is free, while $y$ is bound.

\begin{mathpar}
  \inferrule* [lab=monoidal-laws] {} { P|Q \equiv Q|P \and P|0 \equiv P \and P|(Q|R) \equiv (P|Q)|R }
\end{mathpar}

\begin{mathpar}
  \inferrule* [lab=alpha-equivalence] {} { (x)P \equiv (y)P\{y/x\} \and y \not\in \freenames{P} }
\end{mathpar}

\begin{definition}
Then two processes, $P,Q$, are alpha-equivalent if $P = Q\{\vec{y}/\vec{x}\}$ for
some $\vec{x} \in \boundnames{Q},\vec{y} \in \boundnames{P}$, where $Q\{\vec{y}/\vec{x}\}$
denotes the capture-avoiding substitution of $\vec{y}$ for $\vec{x}$ in $Q$.
\end{definition}

\begin{definition}
  The {\em structural congruence} \cite{SangiorgiWalker} , $\equiv$,
  between processes is the least congruence containing
  alpha-equivalence, satisfying the abelian monoid laws
  (associativity, commutativity and $\pzero$ as identity) for parallel
  composition $|$ and for summation $+$.
\end{definition}

\subsection{Name equivalence}

We take name equivalence, written $\nameeq$, to be the smallest
equivalence relation generated by the following rules.

\begin{mathpar}
\inferrule*[lab=Quote-drop]
{ }
{ \quotep{@{x}} \nameeq x }

\inferrule*[lab=Struct-equiv]
{ P \scong Q }
{ \quotep{P} \nameeq \quotep{Q} }
\end{mathpar}

The astute reader will have noticed that the mutual recursion of names
and processes imposes a mutual recursion on alpha-equivalence and
structural equivalence via name-equivalence. Fortunately, all of this
works out pleasantly and we may calculate in the natural way, free of
concern. The reader interested in the details is referred to the
appendix \ref{appendix:rho_details}.

\subsection{Substitution}

We use $\Proc$ for the set of processes, $\QProc$ for the set of
names, and $\id{\{}\vec{y} / \vec{x} \id{\}}$ to denote partial maps,
$s : \QProc \rightarrow \QProc$. A map, $s$ lifts, uniquely, to a map
on process terms, $\widehat{s} : \Proc \rightarrow \Proc$ by the
following equations.

\begin{mathpar}
  (0) \psubstp{Q}{P} := 0 \\
  (R \juxtap S) \psubstp{Q}{P}
  :=    
  (R)\psubstp{Q}{P} \juxtap (S) \psubstp{Q}{P} \\
  (x?(y).R) \psubstp{Q}{P}    
  :=    
  (x)\substp{Q}{P} (z)\concat( (R \psubstn{z}{y}) \psubstp{Q}{P} ) \\
  (\lift{x}{R}) \psubstp{Q}{P}  
  :=
  \lift{(x)\substp{Q}{P}}{ R \psubstp{Q}{P} } \\
%   (\dropn{x})  \psubstp{Q}{P}       
%   := 
%   \left\{ 
%     \begin{array}{ccc} 
%       \dropn{\quotep{Q}} & & x \nameeq \quotep{P} \\
%       \dropn{x} & & otherwise \\
%     \end{array}
%   \right. 
  (\dropn{x})  \psubstp{Q}{P}       
  := 
  \left\{ 
    \begin{array}{ccc} 
      Q & & x \nameeq \quotep{P} \\
      \dropn{x} & & otherwise \\
    \end{array}
  \right.
\end{mathpar}
 

where

\begin{eqnarray}
  (x)\id{\{} \lpquote Q \rpquote / \lpquote P \rpquote \id{\}}            = 
  \left\{ 
    \begin{array}{ccc}
      \lpquote Q \rpquote & & x \nameeq \lpquote P \rpquote \\
      x & & otherwise \\
    \end{array}
  \right. \nonumber
\end{eqnarray}

and $z$ is chosen distinct from $\quotep{P}$, $\quotep{Q}$, the free
names in $Q$, and all the names in $R$. Our $\alpha$-equivalence will
be built in the standard way from this substitution.

\begin{remark}\label{rem:no_self_referential_names}
  One consequence of these definitions is that $\forall P. \quotep{P}
  \not\in \freenames{P}$.
\end{remark}

\subsection{ Dynamic quote: an example }

Anticipating something of what's to come, consider applying the
substitution, $\widehat{\id{\{}u / z \id{\}}}$, to the following pair
of processes, $\lift{w}{y!(z)}$ and $w[ \lpquote y!(z) \rpquote ]$.

\begin{eqnarray}
	\lift{w}{y!(z)}\widehat{\id{\{}u / z \id{\}}}
		& = &
		\lift{w}{y!(u)} \nonumber\\
	w[ \lpquote y!(z) \rpquote ] \widehat{ \id{\{}u / z \id{\}} }
		& = &
		w[ \lpquote y!(z) \rpquote ] \nonumber
\end{eqnarray}

Because the body of the process between quotes is impervious to
substitution, we get radically different answers. In fact, by
examining the first process in an input context,
e.g. $x?(z).\lift{w}{y!(z)}$, we see that the process under the lift
operator may be shaped by prefixed inputs binding a name inside it. In
this sense, the lift operator will be seen as a way to dynamically
construct processes before reifying them as names.

Finally equipped with these standard features we can present the
dynamics of the calculus.

\subsubsection{Operational semantics} 

Finally, we introduce the computational dynamics. What marks these
algebras as distinct from other more traditionally studied algebraic
structures, e.g. vector spaces or polynomial rings, is the manner in
which dynamics is captured. In traditional structures, dynamics is typically
expressed through morphisms between such structures, as in linear maps
between vector spaces or morphisms between rings. In algebras
associated with the semantics of computation, the dynamics is
expressed as part of the algebraic structure itself, through a
reduction reduction relation typically denoted by $\red$. Below, we
give a recursive presentation of this relation for the calculus used
in the encoding.

$\red \subseteq \pi \times \pi$
$\red : \pi \to \mathcal{P}(\pi)$

\begin{mathpar}
  \inferrule* [lab=Comm] { \textsf{match}( x_{src}, x_{trgt} ) } { x_{trgt}?(y)P \; | \; x_{src}!\langle {Q} \rangle \red P\{\quotep{Q}/y}\} }
  \and \\
  \inferrule* [lab=Par] {{P} \red {P}'} {{{P} | {Q}} \red {{P}' | {Q}}}
  \and
  \inferrule* [lab=Equiv]{{{P} \scong {P}'} \andalso {{P}' \red {Q}'} \andalso {{Q}' \scong {Q}}}{{P} \red {Q}}
\end{mathpar}

\begin{eqnarray*}
  match_{\equiv} (\quotep{P},\quotep{Q}) & := & P \equiv Q \\
  match_{\dagger}(\quotep{P},\quotep{Q}) & := & \forall R. P|Q \red^{*} R => R \red^{*} 0 \\
  match_{K}(\quotep{P},\quotep{Q}) & := & K \mbox{ for some context } K
\end{eqnarray*}

$u?(x)P | u!\langle Q \rangle \red P\{\quotep{Q}/x\}$

%We write $\wred$ for $\red^*$, and $P\red$ if $\exists Q $ such that $ P \red Q$.
We write $P\red$ if $\exists Q $ such that $ P \red Q$ and $P\not\red$, otherwise.

\section{Replication}

As mentioned before, it is known that replication (and hence
recursion) can be implemented in a higher-order process algebra
\cite{SangiorgiWalker}. As our first example of calculation with the
machinery thus far presented we give the construction explicitly in
the {\rhoc}.

\begin{eqnarray}
	D_{x} & := & \prefix{x}{y}{(\binpar{\outputp{x}{y}}{@{y}})} \nonumber\\
	\bangp_{x}{P} & := & \binpar{{x}!\langle{\binpar{D_{x}}{P}}\rangle}{D_{x}} \nonumber
\end{eqnarray}

\begin{eqnarray}
	\bangp_{x}{P} & & \nonumber\\
	=
	& {x}!\langle{(\prefix{x}{y}{(\outputp{x}{y} | @{y})) | P}}\rangle 
	      | \prefix{x}{y}{(\outputp{x}{y} | @{y})} & \nonumber\\
	\red
	& (\outputp{x}{y} | @{y})\substn{\quotep{(\prefix{x}{y}{(@{y} | \outputp{x}{y})) | P}}}{y} & \nonumber\\
	=
	& \outputp{x}{\quotep{(\prefix{x}{y}{(\outputp{x}{y} | @{y})) | P}}}
	  | {(\prefix{x}{y}{(\outputp{x}{y} | @{y})) | P}} & \nonumber\\
	\red
	& \ldots & \nonumber\\
	\red^*
	& P | P | \ldots & \nonumber
\end{eqnarray}

Of course, this encoding, as an implementation, runs away, unfolding
$\bangp{P}$ eagerly. A lazier and more implementable replication
operator, restricted to input-guarded processes, may be obtained as follows.

\begin{eqnarray}
\bangp{\prefix{u}{v}{P}} 
	:= 
	\binpar{\lift{x}{\prefix{u}{v}{(\binpar{D(x)}{P})}}}{D(x)} \nonumber
\end{eqnarray}

\begin{remark}
  Note that the lazier definition still does not deal with summation
  or mixed summation (i.e. sums over input and output). The reader is
  invited to construct definitions of replication that deal with these
  features. 

  Further, the definitions are parameterized in a name, $x$. Can you,
  gentle reader, make a definition that eliminates this parameter and
  guarantees no accidental interaction between the replication
  machinery and the process being replicated -- i.e. no accidental
  sharing of names used by the process to get its work done and the
  name(s) used by the replication to effect copying. This latter
  revision of the definition of replication is crucial to obtaining
  the expected identity $!!P \sim !P$.
\end{remark}

\begin{remark}\label{rem:paradoxical_combinator}
  The reader familiar with the lambda calculus will have noticed the
  similarity between $D$ and the paradoxical combinator.

  [Ed. note: the existence of this seems to suggest we have to be more
  restrictive on the set of processes and names we admit if we are to
  support no-cloning.]
\end{remark}

\subsubsection{Bisimulation}

The computational dynamics gives rise to another kind of equivalence,
the equivalence of computational behavior. As previously mentioned
this is typically captured \emph{via} some form of bisimulation.

% The notion we use in this paper is weak barbed bisimulation
% \cite{milner91polyadicpi}.

The notion we use in this paper is derived from weak barbed
bisimulation \cite{milner91polyadicpi}. 

\begin{definition}
An \emph{observation relation}, $\downarrow_{\mathcal N}$, over a set
of names, $\mathcal N$, is the smallest relation satisfying the rules
below.

\infrule[Out-barb]{y \in {\mathcal N}, \; x \nameeq y}
		  {\outputp{x}{v} \downarrow_{\mathcal N} x}
\infrule[Par-barb]{\mbox{$P\downarrow_{\mathcal N} x$ or $Q\downarrow_{\mathcal N} x$}}
		  {\binpar{P}{Q} \downarrow_{\mathcal N} x}

We write $P \Downarrow_{\mathcal N} x$ if there is $Q$ such that 
$P \wred Q$ and $Q \downarrow_{\mathcal N} x$.
\end{definition}

\begin{definition}
%\label{def.bbisim}
An  ${\mathcal N}$-\emph{barbed bisimulation} over a set of names, ${\mathcal N}$, is a symmetric binary relation 
${\mathcal S}_{\mathcal N}$ between agents such that $P\rel{S}_{\mathcal N}Q$ implies:
\begin{enumerate}
\item If $P \red P'$ then $Q \wred Q'$ and $P'\rel{S}_{\mathcal N} Q'$.
\item If $P\downarrow_{\mathcal N} x$, then $Q\Downarrow_{\mathcal N} x$.
\end{enumerate}
$P$ is ${\mathcal N}$-barbed bisimilar to $Q$, written
$P \wbbisim_{\mathcal N} Q$, if $P \rel{S}_{\mathcal N} Q$ for some ${\mathcal N}$-barbed bisimulation ${\mathcal S}_{\mathcal N}$.
\end{definition}

$\mathcal{R} \subseteq \pi \times \pi$

$P \mathcal{R} Q => \forall P'. P \red P' \Rightarrow \exists Q'. Q \red Q', P' \mathcal{R} Q'$

$P \vdash x \Rightarrow Q \vdash x$

\begin{mathpar}
  \inferrule*[lab=Out-barb]{x \nameeq y}{{y}!\langle{Q}\rangle \vdash x}
  \and
  \inferrule*[lab=Par-barb]{\mbox{$P\vdash x$ or $Q\vdash x$}}{\binpar{P}{Q} \vdash x}
\end{mathpar}

\subsubsection{Contexts}

One of the principle advantages of computational calculi like the
$\pi$-calculus is a well-defined notion of context,
contextual-equivalence and a correlation between
contextual-equivalence and notions of bisimulation. The notion of
context allows the decomposition of a process into (sub-)process and
its syntactic environment, its context. Thus, a context may be
thought of as a process with a ``hole'' (written $\Box$) in it. The
application of a context $M$ to a process $P$, written $M[P]$, is
tantamount to filling the hole in $M$ with $P$. In this paper we do
not need the full weight of this theory, but do make use of the notion
of context in the proof the main theorem. 

\begin{mathpar}
  \inferrule* [lab=summation] {} {{M_{M},M_{N}} \bc \Box \;|\; x.M_{A} \;|\; M_{M}+M_{N}}
  \and
  \inferrule* [lab=agent] {} {{M_{A}} \bc (\vec{x})M_{P} \;| \; \clift{P_0,\ldots,M_{P},\ldots,P_N}}
  \and \\
  \inferrule* [lab=process] {} {{M_{P}} \bc M_{N} \;| \;P|M_{P} }
\end{mathpar} 

\begin{mathpar}
  \inferrule* [lab=sychronization] {} {M_{N} \bc \Box \;|\; x?M_{F} \;|\; x!M_{C}}
  \and
  \inferrule* [lab=abstraction] {} {{M_{F}} \bc (x)M_{P} }
  \and
  \inferrule* [lab=concretion] {} {{M_{C}} \bc \langle M_{P} \rangle }
  \and \\
  \inferrule* [lab=process] {} {{M_{P}} \bc M_{N} \;| \;P|M_{P} }
\end{mathpar}

\begin{definition}[contextual application] Given a context $M$, and
  process $P$, we define the \emph{contextual application}, $M[P] :=
  M\{P/\Box\}$. That is, the contextual application of M to P is the
  substitution of $P$ for $\Box$ in $M$.
\end{definition}

$\meaningof{-} : L \to \mathcal{P}(\pi)$

\begin{mathpar}
  \inferrule* [lab=collection] {} {\meaningof{true} = \pi, \and \meaningof{~E} = \pi \setminus \meaningof{E}, \and \meaningof{E_{1} \& E_{2}} = \meaningof{E_{1}} \cap \meaningof{E_{2}}}
\end{mathpar}

\begin{mathpar}
  \inferrule* [lab=structure] {} {\meaningof{0} = \{ P \in \pi | P \equiv 0 \}, \and \\ \meaningof{E_1 | E_2} = \{ P \in \pi | P \equiv P_{1} | P_{2}, P_{1} \in \meaningof{E_{1}}, P_{2} \in \meaningof{E_2}\} }
\end{mathpar}

\begin{mathpar}
 \inferrule* [lab=behavior] {} {\meaningof{\langle a?b \rangle E} = \{ P \in \pi | P \equiv Q | u?(y)P', \\ \and \\\\ \and \\ \;\;\; u \in \meaningof{a}, \forall z.P'\{z/y\} \in \meaningof{E\{z/b\}}\}, \and \\ \meaningof{a!E} = \{ P \in \pi | P \equiv Q | x!\langle P' \rangle, x \in \meaningof{a} P' \in \meaningof{E}\} }
\end{mathpar}

\begin{mathpar}
 \inferrule* [lab=nominal] {} {\meaningof{\quotep{E}} = \{ \quotep{P} \in \quotep{\pi} | P \in \meaningof{E} \}, \and \meaningof{\quotep{P}} = \{ \quotep{Q} \in \quotep{\pi} | P \equiv Q \} \and \\ \meaningof{@\quotep{E}} = \{ P \in \pi | P \equiv @x, x \in \meaningof{E} \}}
\end{mathpar}

\begin{eqnarray*}
  \\
  \meaningof{-} : TS \to ST
\end{eqnarray*}

\begin{eqnarray*}
  \\
  L : TS \to ST
\end{eqnarray*}

\begin{eqnarray*}
  \\
  P \models E \iff P \in \meaningof{E}
\end{eqnarray*}

\begin{eqnarray*}
  P \approx_{L} Q \iff \forall E \in L. P \models E \iff Q \models E
\end{eqnarray*}

\begin{eqnarray*}
  P \approx_{K} Q
\end{eqnarray*}

\begin{eqnarray*}
  P \approx Q
\end{eqnarray*}

$\approx_{K} = \approx = \approx_{L}$

\subsubsection{Contextual duality}

Note that contexts extend the quotation operation to a family of
operations from processes to names. Given a context, $M$, we can
define a \emph{nominal context}, $\quotep{M}$ by $\quotep{M}[P] :=
\quotep{M[P]}$. To foreshadow what is to come we observe that these
operations enjoy a duality with processes very much like the duality
between vectors and maps from vectors to scalars.

Further, because the calculus is essentially higher-order, we have a
correspondence between contexts and processes. More specifically,
given a name $x$ and a context $M$ we can construct $M^{*}_{x}$ such
that 

\begin{mathpar}
  M^{*}_{x} | \lift{x}{P} \red M[P]
\end{mathpar}

namely,

\begin{mathpar}
  M^{*}_{x} := x?(u).M[\dropn{u}]
\end{mathpar}

The dependence of $M^{*}_{x}$ on a name makes it an abstraction, 

\begin{mathpar}
  M^{*} := (x)x?(u).M[\dropn{u}]
\end{mathpar}

\subsection{Additional notation}

It will sometimes be convenient to denote the process a name
quotes. We already have the notation $x = \quotep{P}$, but it will be
convenient to introduce an alternate notation, $\procn{x}$, when we
want to emphasize the connection to the use of the name. Note that, by
virtue of name equivalence, $\quotep{\procn{x}} \nameeq x$; so, the
notation is consistent with previous definitions.

Further, because names have structure it is possible to effect
substitutions on the basis of that structure. This means we need to
upgrade our notation for substitutions, which we accomplish by
adapting comprehension notation. Thus,

\begin{mathpar}
  P\{ y / x : x \in S \}
\end{mathpar}

is interpreted to mean the process derived from P by replacing (in a
capture-avoiding manner) each occurrence of $x$ in $S$ by $y$. For example,

\begin{mathpar}
  P\{ \quotep{\procn{x}|\procn{x}} / x : x \in \freenames{P} \}
\end{mathpar}

will replace each (occurrence) of a free name $x$ in $P$ by
$\quotep{\procn{x}|\procn{x}}$.

Also, we will avail ourselves of the notation $x^{L}$ and $x^{R}$ to
denote injections of a name into disjoint copies of the name
space. There are numerous ways to accomplish this. One example can be
found in \cite{MeredithR05}. This notation overloads to vectors of
names: $\vec{x}^{\pi} := (x_{i}^{\pi} \; : \; 0 \leq i < |\vec{x}| )$ where $\pi \in \{L,R\}$.

We also use $P^{\Box} := P|\Box$.

In \cite{MeredithR05} an interpretation of the new operator is
given. It turns out that there are several possible interpretations
all enjoying the requisite algebraic properties of the operator (see
\cite{milner91polyadicpi}). We will therefore make liberal use of
$(\nu\; \vec{x})P$.

% subsection the_syntax_and_semantics_of_the_notation_system (end)   

\input{qm2pi.qmops} 

\input{qm2pi.sterngerlach} 

\input{qm2pi.metric} 

% section concurrent_process_calculi (end)

%\input{qm2pi.proofsketch}

% section proof sketch (end)

%\input{qm2pi.slviaknots} 

% section spatial logic via knots (end)

\input{qm2pi.conclusion}

% section conclusion (end)

%\input{qm2pi.dtcodes} 

% section wiring algorithm (end)

\input{qm2pi.ack} 

% section acknowledgments (end)

\newpage


\bibliographystyle{plain}   
\bibliography{../../biblios/main.bib}

\input{qm2pi.rhodetails}

\end{document}

 

%\documentclass[12pt]{llncs}
%\documentclass{jktr}

\usepackage[pdftex]{hyperref}                   
\usepackage {listings}
\usepackage {mathpartir}
\usepackage{bcprules}
%\usepackage{listings}
                       
\usepackage{graphicx} 
%\usepackage[margins=2.5cm,nohead,nofoot]{geometry}
%\usepackage{geometry}
\usepackage{amsfonts}
\usepackage{amstext}
\usepackage{latexsym}
\usepackage{amssymb}
\usepackage{color}


%\include{myPreamble}
\include{qm2pi.local} 

%\ifpdf
%\usepackage[pdftex]{graphicx}
%\else
%\usepackage{graphicx}
%\fi

 % \ifpdf
%  \usepackage{pdfsync}
%  \if


%\title{Brief Article}
%\author{David F. Snyder}
%\author{L.G. Meredith}

%\address{Dept. of Math., Texas State University--San Marcos, San Marcos, TX 78666}
       
\pagestyle{empty}


\begin{document}

\lstset{language=[Objective]Caml,frame=shadowbox}

\input{qm2pi.front}

% section front matter (end)

\input{qm2pi.intro} 
 
% section introduction (end)

% \input{qm2pi.knotations} 

% section notation (end)

\input{qm2pi.process.calculi} 

% section concurrent_process_calculi_and_spatial_logics_ (end)
    
%\input{qm2pi.knots2pi} 

%\input{qm2pi.trefoil} 

%\input{qm2pi.mainthm} 

% subsection basic_interpretation (end)

%\input{qm2pi.rho.presentation} 
\subsection{The syntax and semantics of the notation system}\label{sub:the_syntax_and_semantics_of_the_notation_system} % (fold)

We now summarize a technical presentation of the calculus that
embodies our theory of dynamics. The typical presentation of such a
calculus follows the style of giving generators and relations on
them. The grammar, below, describing term constructors, freely
generates the set of processes, $\Proc$. This set is then quotiented
by a relation known as structural congruence and it is over this set
that the notion of dynamics is expressed. This presentation is
essentially that of \cite{MeredithR05} with the addition of
polyadicity and summation. For readability we have relegated some of
the technical subtleties to an appendix.

\subsubsection{Process grammar}\label{subsub:process_grammar}

\begin{mathpar}
  \inferrule* [lab=synchronization] {} {{M} \bc \pzero \;|\; x?F \;|\; x!C }
  \and
  \inferrule* [lab=abstraction] {} {{F} \bc (x)P}
  \and
  \inferrule* [lab=concretion] {} {{C} \bc \langle Q \rangle}
  \and
  \inferrule* [lab=process] {} {{P,Q} \bc M \;| \;P|Q \;|\; @{x}}
  \and
  \inferrule* [lab=name] {} {{x} \bc \quotep{P}}
\end{mathpar} 

Note that $\vec{x}$ (resp. $\vec{P}$) denotes a vector of names
(resp. processes) of length $|\vec{x}|$ (resp. $|\vec{P}|$). We adopt
the following useful abbreviations.

\begin{mathpar}
   x?(\vec{y}).P := x.(\vec{y})P \and  x\clift{\vec{P}} := x.\clift{\vec{P}}
   \and x!(y) := \lift{x}{\dropn{y}}
   \and \Pi_{i=0}^{n-1}P_i := P_0 | \ldots | P_{n-1}
\end{mathpar}

\subsubsection{Structural congruence}

\paragraph{Free and bound names and alpha-equivalence.} At the
core of structural equivalence is alpha-equivalence which identifies
process that are the same up to a change of variable. Formally, we
recognize the distinction between free and bound names. The free names
of a process, $\freenames{P}$, may be calculated recursively as
follows:

\begin{mathpar}
\freenames{\pzero} := \emptyset
  \and \\
  \freenames{x?(y).P} := \{ x \} \cup (\freenames{P} \setminus \{ y \})
  \and 
  \freenames{x!\langle P \rangle} := \{ x \} \cup \{ P \} 
  \and \\
  \freenames{P|Q} := \freenames{P} \cup \freenames{Q}
  \and \\
  \freenames{@{x}} := \{ x \}
\end{mathpar}

$\pi$
$\quotep{\pi}$

$\freenames{-} : \pi \to \mathcal{P}(\quotep{\pi})$

\begin{eqnarray*}
  \freenames{\pzero} & := & \emptyset \\
  \freenames{x?(y).P} & := & \{ x \} \cup (\freenames{P} \setminus \{ y \}) \\
  \freenames{x!\langle P \rangle} & := & \{ x \} \cup \{ P \} \\
  \freenames{P|Q} & := & \freenames{P} \cup \freenames{Q} \\
  \freenames{\dropn{x}} & := & \{ x \}
\end{eqnarray*}

The bound names of a process, $\boundnames{P}$, are those names occurring in $P$
that are not free. For example, in $x?(y).0$, the name $x$ is free, while $y$ is bound.

\begin{mathpar}
  \inferrule* [lab=monoidal-laws] {} { P|Q \equiv Q|P \and P|0 \equiv P \and P|(Q|R) \equiv (P|Q)|R }
\end{mathpar}

\begin{mathpar}
  \inferrule* [lab=alpha-equivalence] {} { (x)P \equiv (y)P\{y/x\} \and y \not\in \freenames{P} }
\end{mathpar}

\begin{definition}
Then two processes, $P,Q$, are alpha-equivalent if $P = Q\{\vec{y}/\vec{x}\}$ for
some $\vec{x} \in \boundnames{Q},\vec{y} \in \boundnames{P}$, where $Q\{\vec{y}/\vec{x}\}$
denotes the capture-avoiding substitution of $\vec{y}$ for $\vec{x}$ in $Q$.
\end{definition}

\begin{definition}
  The {\em structural congruence} \cite{SangiorgiWalker} , $\equiv$,
  between processes is the least congruence containing
  alpha-equivalence, satisfying the abelian monoid laws
  (associativity, commutativity and $\pzero$ as identity) for parallel
  composition $|$ and for summation $+$.
\end{definition}

\subsection{Name equivalence}

We take name equivalence, written $\nameeq$, to be the smallest
equivalence relation generated by the following rules.

\begin{mathpar}
\inferrule*[lab=Quote-drop]
{ }
{ \quotep{@{x}} \nameeq x }

\inferrule*[lab=Struct-equiv]
{ P \scong Q }
{ \quotep{P} \nameeq \quotep{Q} }
\end{mathpar}

The astute reader will have noticed that the mutual recursion of names
and processes imposes a mutual recursion on alpha-equivalence and
structural equivalence via name-equivalence. Fortunately, all of this
works out pleasantly and we may calculate in the natural way, free of
concern. The reader interested in the details is referred to the
appendix \ref{appendix:rho_details}.

\subsection{Substitution}

We use $\Proc$ for the set of processes, $\QProc$ for the set of
names, and $\id{\{}\vec{y} / \vec{x} \id{\}}$ to denote partial maps,
$s : \QProc \rightarrow \QProc$. A map, $s$ lifts, uniquely, to a map
on process terms, $\widehat{s} : \Proc \rightarrow \Proc$ by the
following equations.

\begin{mathpar}
  (0) \psubstp{Q}{P} := 0 \\
  (R \juxtap S) \psubstp{Q}{P}
  :=    
  (R)\psubstp{Q}{P} \juxtap (S) \psubstp{Q}{P} \\
  (x?(y).R) \psubstp{Q}{P}    
  :=    
  (x)\substp{Q}{P} (z)\concat( (R \psubstn{z}{y}) \psubstp{Q}{P} ) \\
  (\lift{x}{R}) \psubstp{Q}{P}  
  :=
  \lift{(x)\substp{Q}{P}}{ R \psubstp{Q}{P} } \\
%   (\dropn{x})  \psubstp{Q}{P}       
%   := 
%   \left\{ 
%     \begin{array}{ccc} 
%       \dropn{\quotep{Q}} & & x \nameeq \quotep{P} \\
%       \dropn{x} & & otherwise \\
%     \end{array}
%   \right. 
  (\dropn{x})  \psubstp{Q}{P}       
  := 
  \left\{ 
    \begin{array}{ccc} 
      Q & & x \nameeq \quotep{P} \\
      \dropn{x} & & otherwise \\
    \end{array}
  \right.
\end{mathpar}
 

where

\begin{eqnarray}
  (x)\id{\{} \lpquote Q \rpquote / \lpquote P \rpquote \id{\}}            = 
  \left\{ 
    \begin{array}{ccc}
      \lpquote Q \rpquote & & x \nameeq \lpquote P \rpquote \\
      x & & otherwise \\
    \end{array}
  \right. \nonumber
\end{eqnarray}

and $z$ is chosen distinct from $\quotep{P}$, $\quotep{Q}$, the free
names in $Q$, and all the names in $R$. Our $\alpha$-equivalence will
be built in the standard way from this substitution.

\begin{remark}\label{rem:no_self_referential_names}
  One consequence of these definitions is that $\forall P. \quotep{P}
  \not\in \freenames{P}$.
\end{remark}

\subsection{ Dynamic quote: an example }

Anticipating something of what's to come, consider applying the
substitution, $\widehat{\id{\{}u / z \id{\}}}$, to the following pair
of processes, $\lift{w}{y!(z)}$ and $w[ \lpquote y!(z) \rpquote ]$.

\begin{eqnarray}
	\lift{w}{y!(z)}\widehat{\id{\{}u / z \id{\}}}
		& = &
		\lift{w}{y!(u)} \nonumber\\
	w[ \lpquote y!(z) \rpquote ] \widehat{ \id{\{}u / z \id{\}} }
		& = &
		w[ \lpquote y!(z) \rpquote ] \nonumber
\end{eqnarray}

Because the body of the process between quotes is impervious to
substitution, we get radically different answers. In fact, by
examining the first process in an input context,
e.g. $x?(z).\lift{w}{y!(z)}$, we see that the process under the lift
operator may be shaped by prefixed inputs binding a name inside it. In
this sense, the lift operator will be seen as a way to dynamically
construct processes before reifying them as names.

Finally equipped with these standard features we can present the
dynamics of the calculus.

\subsubsection{Operational semantics} 

Finally, we introduce the computational dynamics. What marks these
algebras as distinct from other more traditionally studied algebraic
structures, e.g. vector spaces or polynomial rings, is the manner in
which dynamics is captured. In traditional structures, dynamics is typically
expressed through morphisms between such structures, as in linear maps
between vector spaces or morphisms between rings. In algebras
associated with the semantics of computation, the dynamics is
expressed as part of the algebraic structure itself, through a
reduction reduction relation typically denoted by $\red$. Below, we
give a recursive presentation of this relation for the calculus used
in the encoding.

$\red \subseteq \pi \times \pi$
$\red : \pi \to \mathcal{P}(\pi)$

\begin{mathpar}
  \inferrule* [lab=Comm] { \textsf{match}( x_{src}, x_{trgt} ) } { x_{trgt}?(y)P \; | \; x_{src}!\langle {Q} \rangle \red P\{\quotep{Q}/y}\} }
  \and \\
  \inferrule* [lab=Par] {{P} \red {P}'} {{{P} | {Q}} \red {{P}' | {Q}}}
  \and
  \inferrule* [lab=Equiv]{{{P} \scong {P}'} \andalso {{P}' \red {Q}'} \andalso {{Q}' \scong {Q}}}{{P} \red {Q}}
\end{mathpar}

\begin{eqnarray*}
  match_{\equiv} (\quotep{P},\quotep{Q}) & := & P \equiv Q \\
  match_{\dagger}(\quotep{P},\quotep{Q}) & := & \forall R. P|Q \red^{*} R => R \red^{*} 0 \\
  match_{K}(\quotep{P},\quotep{Q}) & := & K \mbox{ for some context } K
\end{eqnarray*}

$u?(x)P | u!\langle Q \rangle \red P\{\quotep{Q}/x\}$

%We write $\wred$ for $\red^*$, and $P\red$ if $\exists Q $ such that $ P \red Q$.
We write $P\red$ if $\exists Q $ such that $ P \red Q$ and $P\not\red$, otherwise.

\section{Replication}

As mentioned before, it is known that replication (and hence
recursion) can be implemented in a higher-order process algebra
\cite{SangiorgiWalker}. As our first example of calculation with the
machinery thus far presented we give the construction explicitly in
the {\rhoc}.

\begin{eqnarray}
	D_{x} & := & \prefix{x}{y}{(\binpar{\outputp{x}{y}}{@{y}})} \nonumber\\
	\bangp_{x}{P} & := & \binpar{{x}!\langle{\binpar{D_{x}}{P}}\rangle}{D_{x}} \nonumber
\end{eqnarray}

\begin{eqnarray}
	\bangp_{x}{P} & & \nonumber\\
	=
	& {x}!\langle{(\prefix{x}{y}{(\outputp{x}{y} | @{y})) | P}}\rangle 
	      | \prefix{x}{y}{(\outputp{x}{y} | @{y})} & \nonumber\\
	\red
	& (\outputp{x}{y} | @{y})\substn{\quotep{(\prefix{x}{y}{(@{y} | \outputp{x}{y})) | P}}}{y} & \nonumber\\
	=
	& \outputp{x}{\quotep{(\prefix{x}{y}{(\outputp{x}{y} | @{y})) | P}}}
	  | {(\prefix{x}{y}{(\outputp{x}{y} | @{y})) | P}} & \nonumber\\
	\red
	& \ldots & \nonumber\\
	\red^*
	& P | P | \ldots & \nonumber
\end{eqnarray}

Of course, this encoding, as an implementation, runs away, unfolding
$\bangp{P}$ eagerly. A lazier and more implementable replication
operator, restricted to input-guarded processes, may be obtained as follows.

\begin{eqnarray}
\bangp{\prefix{u}{v}{P}} 
	:= 
	\binpar{\lift{x}{\prefix{u}{v}{(\binpar{D(x)}{P})}}}{D(x)} \nonumber
\end{eqnarray}

\begin{remark}
  Note that the lazier definition still does not deal with summation
  or mixed summation (i.e. sums over input and output). The reader is
  invited to construct definitions of replication that deal with these
  features. 

  Further, the definitions are parameterized in a name, $x$. Can you,
  gentle reader, make a definition that eliminates this parameter and
  guarantees no accidental interaction between the replication
  machinery and the process being replicated -- i.e. no accidental
  sharing of names used by the process to get its work done and the
  name(s) used by the replication to effect copying. This latter
  revision of the definition of replication is crucial to obtaining
  the expected identity $!!P \sim !P$.
\end{remark}

\begin{remark}\label{rem:paradoxical_combinator}
  The reader familiar with the lambda calculus will have noticed the
  similarity between $D$ and the paradoxical combinator.

  [Ed. note: the existence of this seems to suggest we have to be more
  restrictive on the set of processes and names we admit if we are to
  support no-cloning.]
\end{remark}

\subsubsection{Bisimulation}

The computational dynamics gives rise to another kind of equivalence,
the equivalence of computational behavior. As previously mentioned
this is typically captured \emph{via} some form of bisimulation.

% The notion we use in this paper is weak barbed bisimulation
% \cite{milner91polyadicpi}.

The notion we use in this paper is derived from weak barbed
bisimulation \cite{milner91polyadicpi}. 

\begin{definition}
An \emph{observation relation}, $\downarrow_{\mathcal N}$, over a set
of names, $\mathcal N$, is the smallest relation satisfying the rules
below.

\infrule[Out-barb]{y \in {\mathcal N}, \; x \nameeq y}
		  {\outputp{x}{v} \downarrow_{\mathcal N} x}
\infrule[Par-barb]{\mbox{$P\downarrow_{\mathcal N} x$ or $Q\downarrow_{\mathcal N} x$}}
		  {\binpar{P}{Q} \downarrow_{\mathcal N} x}

We write $P \Downarrow_{\mathcal N} x$ if there is $Q$ such that 
$P \wred Q$ and $Q \downarrow_{\mathcal N} x$.
\end{definition}

\begin{definition}
%\label{def.bbisim}
An  ${\mathcal N}$-\emph{barbed bisimulation} over a set of names, ${\mathcal N}$, is a symmetric binary relation 
${\mathcal S}_{\mathcal N}$ between agents such that $P\rel{S}_{\mathcal N}Q$ implies:
\begin{enumerate}
\item If $P \red P'$ then $Q \wred Q'$ and $P'\rel{S}_{\mathcal N} Q'$.
\item If $P\downarrow_{\mathcal N} x$, then $Q\Downarrow_{\mathcal N} x$.
\end{enumerate}
$P$ is ${\mathcal N}$-barbed bisimilar to $Q$, written
$P \wbbisim_{\mathcal N} Q$, if $P \rel{S}_{\mathcal N} Q$ for some ${\mathcal N}$-barbed bisimulation ${\mathcal S}_{\mathcal N}$.
\end{definition}

$\mathcal{R} \subseteq \pi \times \pi$

$P \mathcal{R} Q => \forall P'. P \red P' \Rightarrow \exists Q'. Q \red Q', P' \mathcal{R} Q'$

$P \vdash x \Rightarrow Q \vdash x$

\begin{mathpar}
  \inferrule*[lab=Out-barb]{x \nameeq y}{{y}!\langle{Q}\rangle \vdash x}
  \and
  \inferrule*[lab=Par-barb]{\mbox{$P\vdash x$ or $Q\vdash x$}}{\binpar{P}{Q} \vdash x}
\end{mathpar}

\subsubsection{Contexts}

One of the principle advantages of computational calculi like the
$\pi$-calculus is a well-defined notion of context,
contextual-equivalence and a correlation between
contextual-equivalence and notions of bisimulation. The notion of
context allows the decomposition of a process into (sub-)process and
its syntactic environment, its context. Thus, a context may be
thought of as a process with a ``hole'' (written $\Box$) in it. The
application of a context $M$ to a process $P$, written $M[P]$, is
tantamount to filling the hole in $M$ with $P$. In this paper we do
not need the full weight of this theory, but do make use of the notion
of context in the proof the main theorem. 

\begin{mathpar}
  \inferrule* [lab=summation] {} {{M_{M},M_{N}} \bc \Box \;|\; x.M_{A} \;|\; M_{M}+M_{N}}
  \and
  \inferrule* [lab=agent] {} {{M_{A}} \bc (\vec{x})M_{P} \;| \; \clift{P_0,\ldots,M_{P},\ldots,P_N}}
  \and \\
  \inferrule* [lab=process] {} {{M_{P}} \bc M_{N} \;| \;P|M_{P} }
\end{mathpar} 

\begin{mathpar}
  \inferrule* [lab=sychronization] {} {M_{N} \bc \Box \;|\; x?M_{F} \;|\; x!M_{C}}
  \and
  \inferrule* [lab=abstraction] {} {{M_{F}} \bc (x)M_{P} }
  \and
  \inferrule* [lab=concretion] {} {{M_{C}} \bc \langle M_{P} \rangle }
  \and \\
  \inferrule* [lab=process] {} {{M_{P}} \bc M_{N} \;| \;P|M_{P} }
\end{mathpar}

\begin{definition}[contextual application] Given a context $M$, and
  process $P$, we define the \emph{contextual application}, $M[P] :=
  M\{P/\Box\}$. That is, the contextual application of M to P is the
  substitution of $P$ for $\Box$ in $M$.
\end{definition}

$\meaningof{-} : L \to \mathcal{P}(\pi)$

\begin{mathpar}
  \inferrule* [lab=collection] {} {\meaningof{true} = \pi, \and \meaningof{~E} = \pi \setminus \meaningof{E}, \and \meaningof{E_{1} \& E_{2}} = \meaningof{E_{1}} \cap \meaningof{E_{2}}}
\end{mathpar}

\begin{mathpar}
  \inferrule* [lab=structure] {} {\meaningof{0} = \{ P \in \pi | P \equiv 0 \}, \and \\ \meaningof{E_1 | E_2} = \{ P \in \pi | P \equiv P_{1} | P_{2}, P_{1} \in \meaningof{E_{1}}, P_{2} \in \meaningof{E_2}\} }
\end{mathpar}

\begin{mathpar}
 \inferrule* [lab=behavior] {} {\meaningof{\langle a?b \rangle E} = \{ P \in \pi | P \equiv Q | u?(y)P', \\ \and \\\\ \and \\ \;\;\; u \in \meaningof{a}, \forall z.P'\{z/y\} \in \meaningof{E\{z/b\}}\}, \and \\ \meaningof{a!E} = \{ P \in \pi | P \equiv Q | x!\langle P' \rangle, x \in \meaningof{a} P' \in \meaningof{E}\} }
\end{mathpar}

\begin{mathpar}
 \inferrule* [lab=nominal] {} {\meaningof{\quotep{E}} = \{ \quotep{P} \in \quotep{\pi} | P \in \meaningof{E} \}, \and \meaningof{\quotep{P}} = \{ \quotep{Q} \in \quotep{\pi} | P \equiv Q \} \and \\ \meaningof{@\quotep{E}} = \{ P \in \pi | P \equiv @x, x \in \meaningof{E} \}}
\end{mathpar}

\begin{eqnarray*}
  \\
  \meaningof{-} : TS \to ST
\end{eqnarray*}

\begin{eqnarray*}
  \\
  L : TS \to ST
\end{eqnarray*}

\begin{eqnarray*}
  \\
  P \models E \iff P \in \meaningof{E}
\end{eqnarray*}

\begin{eqnarray*}
  P \approx_{L} Q \iff \forall E \in L. P \models E \iff Q \models E
\end{eqnarray*}

\begin{eqnarray*}
  P \approx_{K} Q
\end{eqnarray*}

\begin{eqnarray*}
  P \approx Q
\end{eqnarray*}

$\approx_{K} = \approx = \approx_{L}$

\subsubsection{Contextual duality}

Note that contexts extend the quotation operation to a family of
operations from processes to names. Given a context, $M$, we can
define a \emph{nominal context}, $\quotep{M}$ by $\quotep{M}[P] :=
\quotep{M[P]}$. To foreshadow what is to come we observe that these
operations enjoy a duality with processes very much like the duality
between vectors and maps from vectors to scalars.

Further, because the calculus is essentially higher-order, we have a
correspondence between contexts and processes. More specifically,
given a name $x$ and a context $M$ we can construct $M^{*}_{x}$ such
that 

\begin{mathpar}
  M^{*}_{x} | \lift{x}{P} \red M[P]
\end{mathpar}

namely,

\begin{mathpar}
  M^{*}_{x} := x?(u).M[\dropn{u}]
\end{mathpar}

The dependence of $M^{*}_{x}$ on a name makes it an abstraction, 

\begin{mathpar}
  M^{*} := (x)x?(u).M[\dropn{u}]
\end{mathpar}

\subsection{Additional notation}

It will sometimes be convenient to denote the process a name
quotes. We already have the notation $x = \quotep{P}$, but it will be
convenient to introduce an alternate notation, $\procn{x}$, when we
want to emphasize the connection to the use of the name. Note that, by
virtue of name equivalence, $\quotep{\procn{x}} \nameeq x$; so, the
notation is consistent with previous definitions.

Further, because names have structure it is possible to effect
substitutions on the basis of that structure. This means we need to
upgrade our notation for substitutions, which we accomplish by
adapting comprehension notation. Thus,

\begin{mathpar}
  P\{ y / x : x \in S \}
\end{mathpar}

is interpreted to mean the process derived from P by replacing (in a
capture-avoiding manner) each occurrence of $x$ in $S$ by $y$. For example,

\begin{mathpar}
  P\{ \quotep{\procn{x}|\procn{x}} / x : x \in \freenames{P} \}
\end{mathpar}

will replace each (occurrence) of a free name $x$ in $P$ by
$\quotep{\procn{x}|\procn{x}}$.

Also, we will avail ourselves of the notation $x^{L}$ and $x^{R}$ to
denote injections of a name into disjoint copies of the name
space. There are numerous ways to accomplish this. One example can be
found in \cite{MeredithR05}. This notation overloads to vectors of
names: $\vec{x}^{\pi} := (x_{i}^{\pi} \; : \; 0 \leq i < |\vec{x}| )$ where $\pi \in \{L,R\}$.

We also use $P^{\Box} := P|\Box$.

In \cite{MeredithR05} an interpretation of the new operator is
given. It turns out that there are several possible interpretations
all enjoying the requisite algebraic properties of the operator (see
\cite{milner91polyadicpi}). We will therefore make liberal use of
$(\nu\; \vec{x})P$.

% subsection the_syntax_and_semantics_of_the_notation_system (end)   

\input{qm2pi.qmops} 

\input{qm2pi.sterngerlach} 

\input{qm2pi.metric} 

% section concurrent_process_calculi (end)

%\input{qm2pi.proofsketch}

% section proof sketch (end)

%\input{qm2pi.slviaknots} 

% section spatial logic via knots (end)

\input{qm2pi.conclusion}

% section conclusion (end)

%\input{qm2pi.dtcodes} 

% section wiring algorithm (end)

\input{qm2pi.ack} 

% section acknowledgments (end)

\newpage


\bibliographystyle{plain}   
\bibliography{../../biblios/main.bib}

\input{qm2pi.rhodetails}

\end{document}

 

% subsection basic_interpretation (end)

%\input{qm2pi.rho.presentation} 
\subsection{The syntax and semantics of the notation system}\label{sub:the_syntax_and_semantics_of_the_notation_system} % (fold)

We now summarize a technical presentation of the calculus that
embodies our theory of dynamics. The typical presentation of such a
calculus follows the style of giving generators and relations on
them. The grammar, below, describing term constructors, freely
generates the set of processes, $\Proc$. This set is then quotiented
by a relation known as structural congruence and it is over this set
that the notion of dynamics is expressed. This presentation is
essentially that of \cite{MeredithR05} with the addition of
polyadicity and summation. For readability we have relegated some of
the technical subtleties to an appendix.

\subsubsection{Process grammar}\label{subsub:process_grammar}

\begin{mathpar}
  \inferrule* [lab=synchronization] {} {{M} \bc \pzero \;|\; x?F \;|\; x!C }
  \and
  \inferrule* [lab=abstraction] {} {{F} \bc (x)P}
  \and
  \inferrule* [lab=concretion] {} {{C} \bc \langle Q \rangle}
  \and
  \inferrule* [lab=process] {} {{P,Q} \bc M \;| \;P|Q \;|\; @{x}}
  \and
  \inferrule* [lab=name] {} {{x} \bc \quotep{P}}
\end{mathpar} 

Note that $\vec{x}$ (resp. $\vec{P}$) denotes a vector of names
(resp. processes) of length $|\vec{x}|$ (resp. $|\vec{P}|$). We adopt
the following useful abbreviations.

\begin{mathpar}
   x?(\vec{y}).P := x.(\vec{y})P \and  x\clift{\vec{P}} := x.\clift{\vec{P}}
   \and x!(y) := \lift{x}{\dropn{y}}
   \and \Pi_{i=0}^{n-1}P_i := P_0 | \ldots | P_{n-1}
\end{mathpar}

\subsubsection{Structural congruence}

\paragraph{Free and bound names and alpha-equivalence.} At the
core of structural equivalence is alpha-equivalence which identifies
process that are the same up to a change of variable. Formally, we
recognize the distinction between free and bound names. The free names
of a process, $\freenames{P}$, may be calculated recursively as
follows:

\begin{mathpar}
\freenames{\pzero} := \emptyset
  \and \\
  \freenames{x?(y).P} := \{ x \} \cup (\freenames{P} \setminus \{ y \})
  \and 
  \freenames{x!\langle P \rangle} := \{ x \} \cup \{ P \} 
  \and \\
  \freenames{P|Q} := \freenames{P} \cup \freenames{Q}
  \and \\
  \freenames{@{x}} := \{ x \}
\end{mathpar}

$\pi$
$\quotep{\pi}$

$\freenames{-} : \pi \to \mathcal{P}(\quotep{\pi})$

\begin{eqnarray*}
  \freenames{\pzero} & := & \emptyset \\
  \freenames{x?(y).P} & := & \{ x \} \cup (\freenames{P} \setminus \{ y \}) \\
  \freenames{x!\langle P \rangle} & := & \{ x \} \cup \{ P \} \\
  \freenames{P|Q} & := & \freenames{P} \cup \freenames{Q} \\
  \freenames{\dropn{x}} & := & \{ x \}
\end{eqnarray*}

The bound names of a process, $\boundnames{P}$, are those names occurring in $P$
that are not free. For example, in $x?(y).0$, the name $x$ is free, while $y$ is bound.

\begin{mathpar}
  \inferrule* [lab=monoidal-laws] {} { P|Q \equiv Q|P \and P|0 \equiv P \and P|(Q|R) \equiv (P|Q)|R }
\end{mathpar}

\begin{mathpar}
  \inferrule* [lab=alpha-equivalence] {} { (x)P \equiv (y)P\{y/x\} \and y \not\in \freenames{P} }
\end{mathpar}

\begin{definition}
Then two processes, $P,Q$, are alpha-equivalent if $P = Q\{\vec{y}/\vec{x}\}$ for
some $\vec{x} \in \boundnames{Q},\vec{y} \in \boundnames{P}$, where $Q\{\vec{y}/\vec{x}\}$
denotes the capture-avoiding substitution of $\vec{y}$ for $\vec{x}$ in $Q$.
\end{definition}

\begin{definition}
  The {\em structural congruence} \cite{SangiorgiWalker} , $\equiv$,
  between processes is the least congruence containing
  alpha-equivalence, satisfying the abelian monoid laws
  (associativity, commutativity and $\pzero$ as identity) for parallel
  composition $|$ and for summation $+$.
\end{definition}

\subsection{Name equivalence}

We take name equivalence, written $\nameeq$, to be the smallest
equivalence relation generated by the following rules.

\begin{mathpar}
\inferrule*[lab=Quote-drop]
{ }
{ \quotep{@{x}} \nameeq x }

\inferrule*[lab=Struct-equiv]
{ P \scong Q }
{ \quotep{P} \nameeq \quotep{Q} }
\end{mathpar}

The astute reader will have noticed that the mutual recursion of names
and processes imposes a mutual recursion on alpha-equivalence and
structural equivalence via name-equivalence. Fortunately, all of this
works out pleasantly and we may calculate in the natural way, free of
concern. The reader interested in the details is referred to the
appendix \ref{appendix:rho_details}.

\subsection{Substitution}

We use $\Proc$ for the set of processes, $\QProc$ for the set of
names, and $\id{\{}\vec{y} / \vec{x} \id{\}}$ to denote partial maps,
$s : \QProc \rightarrow \QProc$. A map, $s$ lifts, uniquely, to a map
on process terms, $\widehat{s} : \Proc \rightarrow \Proc$ by the
following equations.

\begin{mathpar}
  (0) \psubstp{Q}{P} := 0 \\
  (R \juxtap S) \psubstp{Q}{P}
  :=    
  (R)\psubstp{Q}{P} \juxtap (S) \psubstp{Q}{P} \\
  (x?(y).R) \psubstp{Q}{P}    
  :=    
  (x)\substp{Q}{P} (z)\concat( (R \psubstn{z}{y}) \psubstp{Q}{P} ) \\
  (\lift{x}{R}) \psubstp{Q}{P}  
  :=
  \lift{(x)\substp{Q}{P}}{ R \psubstp{Q}{P} } \\
%   (\dropn{x})  \psubstp{Q}{P}       
%   := 
%   \left\{ 
%     \begin{array}{ccc} 
%       \dropn{\quotep{Q}} & & x \nameeq \quotep{P} \\
%       \dropn{x} & & otherwise \\
%     \end{array}
%   \right. 
  (\dropn{x})  \psubstp{Q}{P}       
  := 
  \left\{ 
    \begin{array}{ccc} 
      Q & & x \nameeq \quotep{P} \\
      \dropn{x} & & otherwise \\
    \end{array}
  \right.
\end{mathpar}
 

where

\begin{eqnarray}
  (x)\id{\{} \lpquote Q \rpquote / \lpquote P \rpquote \id{\}}            = 
  \left\{ 
    \begin{array}{ccc}
      \lpquote Q \rpquote & & x \nameeq \lpquote P \rpquote \\
      x & & otherwise \\
    \end{array}
  \right. \nonumber
\end{eqnarray}

and $z$ is chosen distinct from $\quotep{P}$, $\quotep{Q}$, the free
names in $Q$, and all the names in $R$. Our $\alpha$-equivalence will
be built in the standard way from this substitution.

\begin{remark}\label{rem:no_self_referential_names}
  One consequence of these definitions is that $\forall P. \quotep{P}
  \not\in \freenames{P}$.
\end{remark}

\subsection{ Dynamic quote: an example }

Anticipating something of what's to come, consider applying the
substitution, $\widehat{\id{\{}u / z \id{\}}}$, to the following pair
of processes, $\lift{w}{y!(z)}$ and $w[ \lpquote y!(z) \rpquote ]$.

\begin{eqnarray}
	\lift{w}{y!(z)}\widehat{\id{\{}u / z \id{\}}}
		& = &
		\lift{w}{y!(u)} \nonumber\\
	w[ \lpquote y!(z) \rpquote ] \widehat{ \id{\{}u / z \id{\}} }
		& = &
		w[ \lpquote y!(z) \rpquote ] \nonumber
\end{eqnarray}

Because the body of the process between quotes is impervious to
substitution, we get radically different answers. In fact, by
examining the first process in an input context,
e.g. $x?(z).\lift{w}{y!(z)}$, we see that the process under the lift
operator may be shaped by prefixed inputs binding a name inside it. In
this sense, the lift operator will be seen as a way to dynamically
construct processes before reifying them as names.

Finally equipped with these standard features we can present the
dynamics of the calculus.

\subsubsection{Operational semantics} 

Finally, we introduce the computational dynamics. What marks these
algebras as distinct from other more traditionally studied algebraic
structures, e.g. vector spaces or polynomial rings, is the manner in
which dynamics is captured. In traditional structures, dynamics is typically
expressed through morphisms between such structures, as in linear maps
between vector spaces or morphisms between rings. In algebras
associated with the semantics of computation, the dynamics is
expressed as part of the algebraic structure itself, through a
reduction reduction relation typically denoted by $\red$. Below, we
give a recursive presentation of this relation for the calculus used
in the encoding.

$\red \subseteq \pi \times \pi$
$\red : \pi \to \mathcal{P}(\pi)$

\begin{mathpar}
  \inferrule* [lab=Comm] { \textsf{match}( x_{src}, x_{trgt} ) } { x_{trgt}?(y)P \; | \; x_{src}!\langle {Q} \rangle \red P\{\quotep{Q}/y}\} }
  \and \\
  \inferrule* [lab=Par] {{P} \red {P}'} {{{P} | {Q}} \red {{P}' | {Q}}}
  \and
  \inferrule* [lab=Equiv]{{{P} \scong {P}'} \andalso {{P}' \red {Q}'} \andalso {{Q}' \scong {Q}}}{{P} \red {Q}}
\end{mathpar}

\begin{eqnarray*}
  match_{\equiv} (\quotep{P},\quotep{Q}) & := & P \equiv Q \\
  match_{\dagger}(\quotep{P},\quotep{Q}) & := & \forall R. P|Q \red^{*} R => R \red^{*} 0 \\
  match_{K}(\quotep{P},\quotep{Q}) & := & K \mbox{ for some context } K
\end{eqnarray*}

$u?(x)P | u!\langle Q \rangle \red P\{\quotep{Q}/x\}$

%We write $\wred$ for $\red^*$, and $P\red$ if $\exists Q $ such that $ P \red Q$.
We write $P\red$ if $\exists Q $ such that $ P \red Q$ and $P\not\red$, otherwise.

\section{Replication}

As mentioned before, it is known that replication (and hence
recursion) can be implemented in a higher-order process algebra
\cite{SangiorgiWalker}. As our first example of calculation with the
machinery thus far presented we give the construction explicitly in
the {\rhoc}.

\begin{eqnarray}
	D_{x} & := & \prefix{x}{y}{(\binpar{\outputp{x}{y}}{@{y}})} \nonumber\\
	\bangp_{x}{P} & := & \binpar{{x}!\langle{\binpar{D_{x}}{P}}\rangle}{D_{x}} \nonumber
\end{eqnarray}

\begin{eqnarray}
	\bangp_{x}{P} & & \nonumber\\
	=
	& {x}!\langle{(\prefix{x}{y}{(\outputp{x}{y} | @{y})) | P}}\rangle 
	      | \prefix{x}{y}{(\outputp{x}{y} | @{y})} & \nonumber\\
	\red
	& (\outputp{x}{y} | @{y})\substn{\quotep{(\prefix{x}{y}{(@{y} | \outputp{x}{y})) | P}}}{y} & \nonumber\\
	=
	& \outputp{x}{\quotep{(\prefix{x}{y}{(\outputp{x}{y} | @{y})) | P}}}
	  | {(\prefix{x}{y}{(\outputp{x}{y} | @{y})) | P}} & \nonumber\\
	\red
	& \ldots & \nonumber\\
	\red^*
	& P | P | \ldots & \nonumber
\end{eqnarray}

Of course, this encoding, as an implementation, runs away, unfolding
$\bangp{P}$ eagerly. A lazier and more implementable replication
operator, restricted to input-guarded processes, may be obtained as follows.

\begin{eqnarray}
\bangp{\prefix{u}{v}{P}} 
	:= 
	\binpar{\lift{x}{\prefix{u}{v}{(\binpar{D(x)}{P})}}}{D(x)} \nonumber
\end{eqnarray}

\begin{remark}
  Note that the lazier definition still does not deal with summation
  or mixed summation (i.e. sums over input and output). The reader is
  invited to construct definitions of replication that deal with these
  features. 

  Further, the definitions are parameterized in a name, $x$. Can you,
  gentle reader, make a definition that eliminates this parameter and
  guarantees no accidental interaction between the replication
  machinery and the process being replicated -- i.e. no accidental
  sharing of names used by the process to get its work done and the
  name(s) used by the replication to effect copying. This latter
  revision of the definition of replication is crucial to obtaining
  the expected identity $!!P \sim !P$.
\end{remark}

\begin{remark}\label{rem:paradoxical_combinator}
  The reader familiar with the lambda calculus will have noticed the
  similarity between $D$ and the paradoxical combinator.

  [Ed. note: the existence of this seems to suggest we have to be more
  restrictive on the set of processes and names we admit if we are to
  support no-cloning.]
\end{remark}

\subsubsection{Bisimulation}

The computational dynamics gives rise to another kind of equivalence,
the equivalence of computational behavior. As previously mentioned
this is typically captured \emph{via} some form of bisimulation.

% The notion we use in this paper is weak barbed bisimulation
% \cite{milner91polyadicpi}.

The notion we use in this paper is derived from weak barbed
bisimulation \cite{milner91polyadicpi}. 

\begin{definition}
An \emph{observation relation}, $\downarrow_{\mathcal N}$, over a set
of names, $\mathcal N$, is the smallest relation satisfying the rules
below.

\infrule[Out-barb]{y \in {\mathcal N}, \; x \nameeq y}
		  {\outputp{x}{v} \downarrow_{\mathcal N} x}
\infrule[Par-barb]{\mbox{$P\downarrow_{\mathcal N} x$ or $Q\downarrow_{\mathcal N} x$}}
		  {\binpar{P}{Q} \downarrow_{\mathcal N} x}

We write $P \Downarrow_{\mathcal N} x$ if there is $Q$ such that 
$P \wred Q$ and $Q \downarrow_{\mathcal N} x$.
\end{definition}

\begin{definition}
%\label{def.bbisim}
An  ${\mathcal N}$-\emph{barbed bisimulation} over a set of names, ${\mathcal N}$, is a symmetric binary relation 
${\mathcal S}_{\mathcal N}$ between agents such that $P\rel{S}_{\mathcal N}Q$ implies:
\begin{enumerate}
\item If $P \red P'$ then $Q \wred Q'$ and $P'\rel{S}_{\mathcal N} Q'$.
\item If $P\downarrow_{\mathcal N} x$, then $Q\Downarrow_{\mathcal N} x$.
\end{enumerate}
$P$ is ${\mathcal N}$-barbed bisimilar to $Q$, written
$P \wbbisim_{\mathcal N} Q$, if $P \rel{S}_{\mathcal N} Q$ for some ${\mathcal N}$-barbed bisimulation ${\mathcal S}_{\mathcal N}$.
\end{definition}

$\mathcal{R} \subseteq \pi \times \pi$

$P \mathcal{R} Q => \forall P'. P \red P' \Rightarrow \exists Q'. Q \red Q', P' \mathcal{R} Q'$

$P \vdash x \Rightarrow Q \vdash x$

\begin{mathpar}
  \inferrule*[lab=Out-barb]{x \nameeq y}{{y}!\langle{Q}\rangle \vdash x}
  \and
  \inferrule*[lab=Par-barb]{\mbox{$P\vdash x$ or $Q\vdash x$}}{\binpar{P}{Q} \vdash x}
\end{mathpar}

\subsubsection{Contexts}

One of the principle advantages of computational calculi like the
$\pi$-calculus is a well-defined notion of context,
contextual-equivalence and a correlation between
contextual-equivalence and notions of bisimulation. The notion of
context allows the decomposition of a process into (sub-)process and
its syntactic environment, its context. Thus, a context may be
thought of as a process with a ``hole'' (written $\Box$) in it. The
application of a context $M$ to a process $P$, written $M[P]$, is
tantamount to filling the hole in $M$ with $P$. In this paper we do
not need the full weight of this theory, but do make use of the notion
of context in the proof the main theorem. 

\begin{mathpar}
  \inferrule* [lab=summation] {} {{M_{M},M_{N}} \bc \Box \;|\; x.M_{A} \;|\; M_{M}+M_{N}}
  \and
  \inferrule* [lab=agent] {} {{M_{A}} \bc (\vec{x})M_{P} \;| \; \clift{P_0,\ldots,M_{P},\ldots,P_N}}
  \and \\
  \inferrule* [lab=process] {} {{M_{P}} \bc M_{N} \;| \;P|M_{P} }
\end{mathpar} 

\begin{mathpar}
  \inferrule* [lab=sychronization] {} {M_{N} \bc \Box \;|\; x?M_{F} \;|\; x!M_{C}}
  \and
  \inferrule* [lab=abstraction] {} {{M_{F}} \bc (x)M_{P} }
  \and
  \inferrule* [lab=concretion] {} {{M_{C}} \bc \langle M_{P} \rangle }
  \and \\
  \inferrule* [lab=process] {} {{M_{P}} \bc M_{N} \;| \;P|M_{P} }
\end{mathpar}

\begin{definition}[contextual application] Given a context $M$, and
  process $P$, we define the \emph{contextual application}, $M[P] :=
  M\{P/\Box\}$. That is, the contextual application of M to P is the
  substitution of $P$ for $\Box$ in $M$.
\end{definition}

$\meaningof{-} : L \to \mathcal{P}(\pi)$

\begin{mathpar}
  \inferrule* [lab=collection] {} {\meaningof{true} = \pi, \and \meaningof{~E} = \pi \setminus \meaningof{E}, \and \meaningof{E_{1} \& E_{2}} = \meaningof{E_{1}} \cap \meaningof{E_{2}}}
\end{mathpar}

\begin{mathpar}
  \inferrule* [lab=structure] {} {\meaningof{0} = \{ P \in \pi | P \equiv 0 \}, \and \\ \meaningof{E_1 | E_2} = \{ P \in \pi | P \equiv P_{1} | P_{2}, P_{1} \in \meaningof{E_{1}}, P_{2} \in \meaningof{E_2}\} }
\end{mathpar}

\begin{mathpar}
 \inferrule* [lab=behavior] {} {\meaningof{\langle a?b \rangle E} = \{ P \in \pi | P \equiv Q | u?(y)P', \\ \and \\\\ \and \\ \;\;\; u \in \meaningof{a}, \forall z.P'\{z/y\} \in \meaningof{E\{z/b\}}\}, \and \\ \meaningof{a!E} = \{ P \in \pi | P \equiv Q | x!\langle P' \rangle, x \in \meaningof{a} P' \in \meaningof{E}\} }
\end{mathpar}

\begin{mathpar}
 \inferrule* [lab=nominal] {} {\meaningof{\quotep{E}} = \{ \quotep{P} \in \quotep{\pi} | P \in \meaningof{E} \}, \and \meaningof{\quotep{P}} = \{ \quotep{Q} \in \quotep{\pi} | P \equiv Q \} \and \\ \meaningof{@\quotep{E}} = \{ P \in \pi | P \equiv @x, x \in \meaningof{E} \}}
\end{mathpar}

\begin{eqnarray*}
  \\
  \meaningof{-} : TS \to ST
\end{eqnarray*}

\begin{eqnarray*}
  \\
  L : TS \to ST
\end{eqnarray*}

\begin{eqnarray*}
  \\
  P \models E \iff P \in \meaningof{E}
\end{eqnarray*}

\begin{eqnarray*}
  P \approx_{L} Q \iff \forall E \in L. P \models E \iff Q \models E
\end{eqnarray*}

\begin{eqnarray*}
  P \approx_{K} Q
\end{eqnarray*}

\begin{eqnarray*}
  P \approx Q
\end{eqnarray*}

$\approx_{K} = \approx = \approx_{L}$

\subsubsection{Contextual duality}

Note that contexts extend the quotation operation to a family of
operations from processes to names. Given a context, $M$, we can
define a \emph{nominal context}, $\quotep{M}$ by $\quotep{M}[P] :=
\quotep{M[P]}$. To foreshadow what is to come we observe that these
operations enjoy a duality with processes very much like the duality
between vectors and maps from vectors to scalars.

Further, because the calculus is essentially higher-order, we have a
correspondence between contexts and processes. More specifically,
given a name $x$ and a context $M$ we can construct $M^{*}_{x}$ such
that 

\begin{mathpar}
  M^{*}_{x} | \lift{x}{P} \red M[P]
\end{mathpar}

namely,

\begin{mathpar}
  M^{*}_{x} := x?(u).M[\dropn{u}]
\end{mathpar}

The dependence of $M^{*}_{x}$ on a name makes it an abstraction, 

\begin{mathpar}
  M^{*} := (x)x?(u).M[\dropn{u}]
\end{mathpar}

\subsection{Additional notation}

It will sometimes be convenient to denote the process a name
quotes. We already have the notation $x = \quotep{P}$, but it will be
convenient to introduce an alternate notation, $\procn{x}$, when we
want to emphasize the connection to the use of the name. Note that, by
virtue of name equivalence, $\quotep{\procn{x}} \nameeq x$; so, the
notation is consistent with previous definitions.

Further, because names have structure it is possible to effect
substitutions on the basis of that structure. This means we need to
upgrade our notation for substitutions, which we accomplish by
adapting comprehension notation. Thus,

\begin{mathpar}
  P\{ y / x : x \in S \}
\end{mathpar}

is interpreted to mean the process derived from P by replacing (in a
capture-avoiding manner) each occurrence of $x$ in $S$ by $y$. For example,

\begin{mathpar}
  P\{ \quotep{\procn{x}|\procn{x}} / x : x \in \freenames{P} \}
\end{mathpar}

will replace each (occurrence) of a free name $x$ in $P$ by
$\quotep{\procn{x}|\procn{x}}$.

Also, we will avail ourselves of the notation $x^{L}$ and $x^{R}$ to
denote injections of a name into disjoint copies of the name
space. There are numerous ways to accomplish this. One example can be
found in \cite{MeredithR05}. This notation overloads to vectors of
names: $\vec{x}^{\pi} := (x_{i}^{\pi} \; : \; 0 \leq i < |\vec{x}| )$ where $\pi \in \{L,R\}$.

We also use $P^{\Box} := P|\Box$.

In \cite{MeredithR05} an interpretation of the new operator is
given. It turns out that there are several possible interpretations
all enjoying the requisite algebraic properties of the operator (see
\cite{milner91polyadicpi}). We will therefore make liberal use of
$(\nu\; \vec{x})P$.

% subsection the_syntax_and_semantics_of_the_notation_system (end)   

\section{Interpretation of QM}
\subsection{Supporting definitions}
\subsubsection{Multiplication}
\begin{mathpar}
  \quotep{Q} \cdot \quotep{R} := \quotep{Q|R}
  \and \\
  \quotep{Q} \cdot P := P\{ \quotep{Q|R} / \quotep{R} : \quotep{R} \in \freenames{P} \}
\end{mathpar}

\paragraph{Discussion}
The first line needs little explanation. The second line says that
each free name of the process is replaced with the multiplication of
that name by the scalar. Multiplication of a scalar (name) by a state
(process) results in a process all the names of which have been `moved
over' by parallel composition with the process the scalar
quotes. There is a subtlety that the bound names have to be
manipulated so that multiplied names aren't accidentally
captured. There are many ways to achieve this.

\begin{remark}\label{rem:multiplication_identities}
  The reader is invited to verify that for all $x,y,z \in \QProc$ and $P \in \Proc$
  \begin{mathpar}
    x \cdot \quotep{0} \equiv x 
    \and
    x \cdot y \equiv y \cdot x
    \and
    x \cdot (y \cdot z) \equiv (x \cdot y) \cdot z
    \and \\
    \quotep{0} \cdot P \equiv P
    \and \\
    x \cdot (y \cdot P) \equiv (x \cdot y) \cdot P
    \and \\
    x \cdot (P|Q) \equiv (x \cdot P) | (x \cdot Q)
    \and \\    
  \end{mathpar}
\end{remark}

\subsubsection{Tensor product}

We define a tensor product on processes by structural induction.

\paragraph{Tensor of sums} First note that all summations, including
$\pzero$ and sequence, can be written $\Sigma_{i} x_{i}.A_{i} +
\Sigma_{j} x_{j}.C_{j}$, where we have grouped input-guarded processes
together and output-guarded processes together.

Thus, we can define the tensor product of two summations, $N_{1}\otimes N_{2}$, where

\begin{mathpar}
  N_{1} := \Sigma_{i} x_{i}.A_{i} + \Sigma_{j} x_{j}.C_{j}
  \and
  N_{2} := \Sigma_{i'} y_{i'}.B_{i'} + \Sigma_{j'} y_{j'}.D_{j'} 
\end{mathpar}

as follows.

\begin{mathpar}
  \Sigma_{i} x_{i}.A_{i} + \Sigma_{j} x_{j}.C_{j} \otimes \Sigma_{i'}
  y_{i'}.B_{i'} + \Sigma_{j'} y_{j'}.D_{j'} 
  \and \\
  := \; \Sigma_{i} \Sigma_{i'} \quotep{\stackrel{\vee}{x_{i}}| \stackrel{\vee}{y_{i'}}}.(A_{i}\otimes B_{i'}) \; | \; \Sigma_{i'} \Sigma_{i} \quotep{\stackrel{\vee}{y_{i'}}|\stackrel{\vee}{x_{i}}}.(B_{i'}\otimes A_{i})
  \and
  \;\; | \;\; \Sigma_{j} \Sigma_{j'} \quotep{\stackrel{\vee}{x_{j}}|\stackrel{\vee}{y_{j'}}}.(A_{j}\otimes B_{j'}) \; | \; \Sigma_{j'} \Sigma_{j} \quotep{\stackrel{\vee}{y_{j'}}|\stackrel{\vee}{x_{j}}}.(B_{j'}\otimes A_{j})
\end{mathpar}

\begin{remark}
  Do we need to $x^{L}$ and $y^{R}$ for this construction as well?
\end{remark}

\paragraph{Tensor of parallel compositions} Next, we distribute tensor
over par.

\begin{mathpar}
  P_{1}|P_{2} \otimes Q_{1}|Q_{2} := (P_{1} \otimes Q_{1}) | (P_{1}
  \otimes Q_{2}) | (P_{2} \otimes Q_{1}) | (P_{2} \otimes Q_{2})
\end{mathpar}

\paragraph{Tensor with dropped names} We treat tensor of a
process with a dropped name as parallel composition.

\begin{mathpar}
  P \otimes \dropn{x} := P | \dropn{x}
\end{mathpar}

\paragraph{Tensor of agents}

Finally, we need to define tensor on agents. Note that the definition
of tensor on normal products only tensors inputs with inputs and
outputs with outputs. Thus, we only have to define the operation on
``homogeneous'' pairings.

\begin{mathpar}
  (\vec{x})P \otimes (\vec{y})Q
  \and \\
  := (x_{0}^{L}|y_{0}^{R},\ldots,x_{0}^{L}|y_{n}^{R},\ldots,x_{m}^{L}|y_{0}^{R},\ldots,x_{m}^{L}|y_{n}^R)(P\{ \vec{x}^{L}/\vec{x}\} \otimes Q \{ \vec{y}^{R}/\vec{y}\})
  \and \\
  \clift{\vec{P}} \otimes \clift{\vec{Q}}
  \and \\
  := \clift{P_{0}\otimes Q_{0},\ldots,P_{0}\otimes Q_{n},\ldots,P_{m}\otimes Q_{0},\ldots,P_{m}\otimes Q_{n}}
\end{mathpar}

\begin{remark}
  Observe that arities of tensored abstractions matches arities of
  tensored concretions if the original arities matched. Note also that
  the length of the arities corresponds to the increase in dimension
  we see in ordinary vector space tensor product.
\end{remark}

\begin{remark}
  Operationally, this definition distributes the tensor down to
  components ``linked'' by summation. Tensor over summation is
  intriguing in that it mixes names. Moreover, as a consequence of the
  way it mixes names we have the identities for all $x \in \QProc$ and
  $P,Q \in \Proc$

  \begin{mathpar}
    (x \cdot P) \otimes Q \equiv x \cdot (P \otimes Q) \equiv P \otimes (x \cdot Q)
    \and
    P \otimes \pzero \equiv P
  \end{mathpar}

  that the reader is invited to verify.
\end{remark}

\subsubsection{Annihilation}
\begin{mathpar}
  P^{\perp} := \{ Q | \forall R. P|Q \red^{*} R \Rightarrow R \red^{*} \pzero \}
  \and \\
  P^{\underline{\perp}} := \Sigma_{Q \in P^{\perp}} \quotep{Q}?(y).(\dropn{y}|Q) | \Sigma_{Q \in P^{\perp}} \quotep{Q}\clift{\Box}
\end{mathpar}

\paragraph{Discussion} The reader will note that $P^{\perp}$ is a
\emph{set} of processes, while $P^{\underline{\perp}}$ is a
\emph{context}. We call the set $P^{\perp}$ the \emph{annihilators} of
$P$. The parallel composition of a process in the annihilators of $P$
with $P$ will result in a process, the state space of which has all
paths eventually leading to $\pzero$. Execution may endure loops; but
under reasonable conditions of fairness (naturally guaranteed under
most notions of bisimulation) such a composite process cannot get
stuck in such a loop and will, eventually pop out and terminate.

The context $P^{\underline{\perp}}$ is ready and willing to ``take the
$P$ out of'' the process to which it is applied. It will effectively
transmit the code of the process to which it is applied to one of the
annihilators and run the process against it.

\subsubsection{Evaluation}
We fix $M$ a domain of fully abstract interpretation with an equality
coincident with bisimulation. We take $\meaningof{\cdot} : \Proc \to
M$ to be the map interpreting processes and $\nmeaningof{\cdot} : \M
\to Proc$ to be the map running the other way. Then we define

\begin{mathpar}
  \int P := \nmeaningof{\meaningof{P}}
\end{mathpar}

\paragraph{Discussion}
There are many fully abstract interpretations of Milner's
$\pi$-calculus. Any of them can be used as a basis for interpreting
the reflective calculus here. Equipped with such a domain it is
largely a matter of grinding through to check that the Yoneda
construction for the normalization-by-evaluation program can be
extended to this setting.

\begin{remark}
  The reader is invited to verify that $\int (P^{\underline{\perp}}[P]) = 0$.
\end{remark}

\subsection{Quantum mechanics}

Table \ref{tbl:core_qm_op_defns} gives the core operational definitions

\begin{table}[htp]\label{tbl:core_qm_op_defns}
  \center{
    \fbox{
      \begin{tabular}{c|c}
        quantum mechanics & process calculus \\
        \hline
        scalar & $x := \quotep{P}$ \\
        state vector & $\state{P} := P$ \\
        dual & $\state{P}^{*} := \event{P^{\underline{\perp}}} := \quotep{P^{\underline{\perp}}}[-]$ \\
        matrix & $ \Sigma_{\alpha} \state{P_{\alpha}}x_{\alpha}\event{Q_{\alpha}}$ \\
        vector addition & $\state{P} + \state{Q} := \state{P | Q}$ \\
        tensor product & $\state{P} \otimes \state{Q} := \state{P \otimes Q}$ \\
        inner product & $\innerprod{P}{Q} := \quotep{\int P^{\underline{\perp}}[Q]}$ \\
      \end{tabular}
    }
  }
  \caption{QM - operational definitions}
\end{table}

where

\begin{mathpar}
  \prmatrix{P}{Q} := \fprmatrix{P}{\quotep{\pzero}}{Q}
  \and
  \fprmatrix{P}{x}{Q} := (\state{P},x,\event{Q})
  \and
  (\fprmatrix{P}{x}{Q})(\state{R}) := x \cdot \innerprod{Q}{R} \cdot \state{P}
  \and
  (\fprmatrix{P}{x}{Q})(\event{R}) := x \cdot \innerprod{R}{P} \cdot \event{Q}
\end{mathpar}

\paragraph{Discussion}
As promised: vectors (aka states) are represented as processes; duals
as contextual duals; inner product definition should be compared with
standard inner product definition for ....

\begin{remark}
  Assuming $\int (P^{\underline{\perp}}[P]) = 0$, the reader is
  invited to verify that $(\fprmatrix{P}{x}{P})(\state{P}) = x \cdot \state{P}$.
\end{remark}

\begin{remark}
  The reader is invited to verify that $\innerprod{P}{Q}$ could
  equally well have been written $\quotep{\int \stackrel{\vee}{x}}$
  where $x = \event{P^{\underline{\perp}}}(Q)$.

  One of the motivations for this remark is that there is another way
  to factor these operations. We could package up evaluation in the dual:

  \begin{mathpar}
    \state{P}^{*} := \event{\int P^{\underline{\perp}}} := \quotep{\int P^{\underline{\perp}}}[-]
  \end{mathpar}

  and then have inner product defined by
  
  \begin{mathpar}
    \innerprod{P}{Q} := \event{P}(Q)
  \end{mathpar}

  Hopefully, experience with the calculations will provide guidance on
  the best factoring.
\end{remark}

\begin{remark}
  Assuming $\int (P^{\underline{\perp}}[P]) = 0$, the reader is
  invited to verify that $\forall P,Q. (\prmatrix{0}{Q})(\state{0}) =
  \state{0}$ and dually $(\prmatrix{P}{0})(\event{0}) = \event{0}$.
\end{remark}

\begin{remark}
  i'm a little worried that i don't (yet) have proper support for
  complex conjugacy. But, the observation above may give us a
  clue. According to Abramsky, it must be the case that the scalars
  are iso to the homset of the identity for the tensor -- which the
  observation above characterizes. 

  For now, we will simply bookmark the notion with $\overline{x}$.
\end{remark}

\subsubsection{Adjointness}

We need to give a definition of $(\cdot)^{\dagger}$ for matrices. The
obvious candidate definition is
\begin{mathpar}
(\Sigma_{\alpha}\fprmatrix{P_{\alpha}}{x_{\alpha}}{Q_{\alpha}})^{\dagger}
= \Sigma_{\alpha}\fprmatrix{(Q_{\alpha}^{\underline{\perp}})^{*}}{\overline{x}_{\alpha}}{P_{\alpha}^{\underline{\perp}}} 
\end{mathpar}

But, $(Q_{\alpha}^{\underline{\perp}})^{*}$ requires a name along
which to communicate the process to achieve the context application.

\subsubsection{Basis for a basis}
If processes label states and ``addition'' of states (a.k.a. vector
addition) is interpreted as parallel composition, what corresponds to
notions of linear independence and basis? Here, we recall that Yoshida
has developed a set of \emph{combinators} for an asynchronous verison
of Milner's $\pi$-calculus. These are a finite set of processes such
any process can be expressed as parallel composition of these
combinators together with liberal uses of the new operator and
replication. We can simply give a translation of these into the
present calculus and have reasonable expectation that the property
carries over. That is, that the resultant set allows to express all
processes via parallel composition. Note, however, that there is no
new operator or replication in this calculus. As a result, we expect
that the corresponding set is actually infinite. That is, we expect
that the space is actually infinite dimensional.

\begin{remark}
  The attentive reader may be a bit concerned. Certainly, the
  collection $S$, $K$ and $I$ is a finite set of
  combinators. Shouldn't we expect to see a finite set of combinators
  for an effectively equivalent system? i am very sympathetic to this
  critique and feel it warrants full attention. On the other hand, i
  also have in mind the following analogy. The natural numbers, as a
  monoid under addition, has exactly $1$ generator, while the natural
  numbers, as a monoid under multiplication, has countably many
  generators (the primes). We observe that the application of the
  lambda calculus is much less resource sensitive than the parallel
  composition of the $\pi$-calculus. Could it be the case that we have
  an analogy of the form
  
  \begin{mathpar}
    m + n : MN :: m*n : M|N
  \end{mathpar}

  giving a similar blow up in the set of ``primes''?  This is such a
  wonderful thought that, even if it's not true, i think it's worth
  writing down.
\end{remark}
 

\documentclass[12pt]{llncs}
%\documentclass{jktr}

\usepackage[pdftex]{hyperref}                   
\usepackage {listings}
\usepackage {mathpartir}
\usepackage{bcprules}
%\usepackage{listings}
                       
\usepackage{graphicx} 
%\usepackage[margins=2.5cm,nohead,nofoot]{geometry}
%\usepackage{geometry}
\usepackage{amsfonts}
\usepackage{amstext}
\usepackage{latexsym}
\usepackage{amssymb}
\usepackage{color}


%\include{myPreamble}
\include{qm2pi.local} 

%\ifpdf
%\usepackage[pdftex]{graphicx}
%\else
%\usepackage{graphicx}
%\fi

 % \ifpdf
%  \usepackage{pdfsync}
%  \if


%\title{Brief Article}
%\author{David F. Snyder}
%\author{L.G. Meredith}

%\address{Dept. of Math., Texas State University--San Marcos, San Marcos, TX 78666}
       
\pagestyle{empty}


\begin{document}

\lstset{language=[Objective]Caml,frame=shadowbox}

\input{qm2pi.front}

% section front matter (end)

\input{qm2pi.intro} 
 
% section introduction (end)

% \input{qm2pi.knotations} 

% section notation (end)

\input{qm2pi.process.calculi} 

% section concurrent_process_calculi_and_spatial_logics_ (end)
    
%\input{qm2pi.knots2pi} 

%\input{qm2pi.trefoil} 

%\input{qm2pi.mainthm} 

% subsection basic_interpretation (end)

%\input{qm2pi.rho.presentation} 
\subsection{The syntax and semantics of the notation system}\label{sub:the_syntax_and_semantics_of_the_notation_system} % (fold)

We now summarize a technical presentation of the calculus that
embodies our theory of dynamics. The typical presentation of such a
calculus follows the style of giving generators and relations on
them. The grammar, below, describing term constructors, freely
generates the set of processes, $\Proc$. This set is then quotiented
by a relation known as structural congruence and it is over this set
that the notion of dynamics is expressed. This presentation is
essentially that of \cite{MeredithR05} with the addition of
polyadicity and summation. For readability we have relegated some of
the technical subtleties to an appendix.

\subsubsection{Process grammar}\label{subsub:process_grammar}

\begin{mathpar}
  \inferrule* [lab=synchronization] {} {{M} \bc \pzero \;|\; x?F \;|\; x!C }
  \and
  \inferrule* [lab=abstraction] {} {{F} \bc (x)P}
  \and
  \inferrule* [lab=concretion] {} {{C} \bc \langle Q \rangle}
  \and
  \inferrule* [lab=process] {} {{P,Q} \bc M \;| \;P|Q \;|\; @{x}}
  \and
  \inferrule* [lab=name] {} {{x} \bc \quotep{P}}
\end{mathpar} 

Note that $\vec{x}$ (resp. $\vec{P}$) denotes a vector of names
(resp. processes) of length $|\vec{x}|$ (resp. $|\vec{P}|$). We adopt
the following useful abbreviations.

\begin{mathpar}
   x?(\vec{y}).P := x.(\vec{y})P \and  x\clift{\vec{P}} := x.\clift{\vec{P}}
   \and x!(y) := \lift{x}{\dropn{y}}
   \and \Pi_{i=0}^{n-1}P_i := P_0 | \ldots | P_{n-1}
\end{mathpar}

\subsubsection{Structural congruence}

\paragraph{Free and bound names and alpha-equivalence.} At the
core of structural equivalence is alpha-equivalence which identifies
process that are the same up to a change of variable. Formally, we
recognize the distinction between free and bound names. The free names
of a process, $\freenames{P}$, may be calculated recursively as
follows:

\begin{mathpar}
\freenames{\pzero} := \emptyset
  \and \\
  \freenames{x?(y).P} := \{ x \} \cup (\freenames{P} \setminus \{ y \})
  \and 
  \freenames{x!\langle P \rangle} := \{ x \} \cup \{ P \} 
  \and \\
  \freenames{P|Q} := \freenames{P} \cup \freenames{Q}
  \and \\
  \freenames{@{x}} := \{ x \}
\end{mathpar}

$\pi$
$\quotep{\pi}$

$\freenames{-} : \pi \to \mathcal{P}(\quotep{\pi})$

\begin{eqnarray*}
  \freenames{\pzero} & := & \emptyset \\
  \freenames{x?(y).P} & := & \{ x \} \cup (\freenames{P} \setminus \{ y \}) \\
  \freenames{x!\langle P \rangle} & := & \{ x \} \cup \{ P \} \\
  \freenames{P|Q} & := & \freenames{P} \cup \freenames{Q} \\
  \freenames{\dropn{x}} & := & \{ x \}
\end{eqnarray*}

The bound names of a process, $\boundnames{P}$, are those names occurring in $P$
that are not free. For example, in $x?(y).0$, the name $x$ is free, while $y$ is bound.

\begin{mathpar}
  \inferrule* [lab=monoidal-laws] {} { P|Q \equiv Q|P \and P|0 \equiv P \and P|(Q|R) \equiv (P|Q)|R }
\end{mathpar}

\begin{mathpar}
  \inferrule* [lab=alpha-equivalence] {} { (x)P \equiv (y)P\{y/x\} \and y \not\in \freenames{P} }
\end{mathpar}

\begin{definition}
Then two processes, $P,Q$, are alpha-equivalent if $P = Q\{\vec{y}/\vec{x}\}$ for
some $\vec{x} \in \boundnames{Q},\vec{y} \in \boundnames{P}$, where $Q\{\vec{y}/\vec{x}\}$
denotes the capture-avoiding substitution of $\vec{y}$ for $\vec{x}$ in $Q$.
\end{definition}

\begin{definition}
  The {\em structural congruence} \cite{SangiorgiWalker} , $\equiv$,
  between processes is the least congruence containing
  alpha-equivalence, satisfying the abelian monoid laws
  (associativity, commutativity and $\pzero$ as identity) for parallel
  composition $|$ and for summation $+$.
\end{definition}

\subsection{Name equivalence}

We take name equivalence, written $\nameeq$, to be the smallest
equivalence relation generated by the following rules.

\begin{mathpar}
\inferrule*[lab=Quote-drop]
{ }
{ \quotep{@{x}} \nameeq x }

\inferrule*[lab=Struct-equiv]
{ P \scong Q }
{ \quotep{P} \nameeq \quotep{Q} }
\end{mathpar}

The astute reader will have noticed that the mutual recursion of names
and processes imposes a mutual recursion on alpha-equivalence and
structural equivalence via name-equivalence. Fortunately, all of this
works out pleasantly and we may calculate in the natural way, free of
concern. The reader interested in the details is referred to the
appendix \ref{appendix:rho_details}.

\subsection{Substitution}

We use $\Proc$ for the set of processes, $\QProc$ for the set of
names, and $\id{\{}\vec{y} / \vec{x} \id{\}}$ to denote partial maps,
$s : \QProc \rightarrow \QProc$. A map, $s$ lifts, uniquely, to a map
on process terms, $\widehat{s} : \Proc \rightarrow \Proc$ by the
following equations.

\begin{mathpar}
  (0) \psubstp{Q}{P} := 0 \\
  (R \juxtap S) \psubstp{Q}{P}
  :=    
  (R)\psubstp{Q}{P} \juxtap (S) \psubstp{Q}{P} \\
  (x?(y).R) \psubstp{Q}{P}    
  :=    
  (x)\substp{Q}{P} (z)\concat( (R \psubstn{z}{y}) \psubstp{Q}{P} ) \\
  (\lift{x}{R}) \psubstp{Q}{P}  
  :=
  \lift{(x)\substp{Q}{P}}{ R \psubstp{Q}{P} } \\
%   (\dropn{x})  \psubstp{Q}{P}       
%   := 
%   \left\{ 
%     \begin{array}{ccc} 
%       \dropn{\quotep{Q}} & & x \nameeq \quotep{P} \\
%       \dropn{x} & & otherwise \\
%     \end{array}
%   \right. 
  (\dropn{x})  \psubstp{Q}{P}       
  := 
  \left\{ 
    \begin{array}{ccc} 
      Q & & x \nameeq \quotep{P} \\
      \dropn{x} & & otherwise \\
    \end{array}
  \right.
\end{mathpar}
 

where

\begin{eqnarray}
  (x)\id{\{} \lpquote Q \rpquote / \lpquote P \rpquote \id{\}}            = 
  \left\{ 
    \begin{array}{ccc}
      \lpquote Q \rpquote & & x \nameeq \lpquote P \rpquote \\
      x & & otherwise \\
    \end{array}
  \right. \nonumber
\end{eqnarray}

and $z$ is chosen distinct from $\quotep{P}$, $\quotep{Q}$, the free
names in $Q$, and all the names in $R$. Our $\alpha$-equivalence will
be built in the standard way from this substitution.

\begin{remark}\label{rem:no_self_referential_names}
  One consequence of these definitions is that $\forall P. \quotep{P}
  \not\in \freenames{P}$.
\end{remark}

\subsection{ Dynamic quote: an example }

Anticipating something of what's to come, consider applying the
substitution, $\widehat{\id{\{}u / z \id{\}}}$, to the following pair
of processes, $\lift{w}{y!(z)}$ and $w[ \lpquote y!(z) \rpquote ]$.

\begin{eqnarray}
	\lift{w}{y!(z)}\widehat{\id{\{}u / z \id{\}}}
		& = &
		\lift{w}{y!(u)} \nonumber\\
	w[ \lpquote y!(z) \rpquote ] \widehat{ \id{\{}u / z \id{\}} }
		& = &
		w[ \lpquote y!(z) \rpquote ] \nonumber
\end{eqnarray}

Because the body of the process between quotes is impervious to
substitution, we get radically different answers. In fact, by
examining the first process in an input context,
e.g. $x?(z).\lift{w}{y!(z)}$, we see that the process under the lift
operator may be shaped by prefixed inputs binding a name inside it. In
this sense, the lift operator will be seen as a way to dynamically
construct processes before reifying them as names.

Finally equipped with these standard features we can present the
dynamics of the calculus.

\subsubsection{Operational semantics} 

Finally, we introduce the computational dynamics. What marks these
algebras as distinct from other more traditionally studied algebraic
structures, e.g. vector spaces or polynomial rings, is the manner in
which dynamics is captured. In traditional structures, dynamics is typically
expressed through morphisms between such structures, as in linear maps
between vector spaces or morphisms between rings. In algebras
associated with the semantics of computation, the dynamics is
expressed as part of the algebraic structure itself, through a
reduction reduction relation typically denoted by $\red$. Below, we
give a recursive presentation of this relation for the calculus used
in the encoding.

$\red \subseteq \pi \times \pi$
$\red : \pi \to \mathcal{P}(\pi)$

\begin{mathpar}
  \inferrule* [lab=Comm] { \textsf{match}( x_{src}, x_{trgt} ) } { x_{trgt}?(y)P \; | \; x_{src}!\langle {Q} \rangle \red P\{\quotep{Q}/y}\} }
  \and \\
  \inferrule* [lab=Par] {{P} \red {P}'} {{{P} | {Q}} \red {{P}' | {Q}}}
  \and
  \inferrule* [lab=Equiv]{{{P} \scong {P}'} \andalso {{P}' \red {Q}'} \andalso {{Q}' \scong {Q}}}{{P} \red {Q}}
\end{mathpar}

\begin{eqnarray*}
  match_{\equiv} (\quotep{P},\quotep{Q}) & := & P \equiv Q \\
  match_{\dagger}(\quotep{P},\quotep{Q}) & := & \forall R. P|Q \red^{*} R => R \red^{*} 0 \\
  match_{K}(\quotep{P},\quotep{Q}) & := & K \mbox{ for some context } K
\end{eqnarray*}

$u?(x)P | u!\langle Q \rangle \red P\{\quotep{Q}/x\}$

%We write $\wred$ for $\red^*$, and $P\red$ if $\exists Q $ such that $ P \red Q$.
We write $P\red$ if $\exists Q $ such that $ P \red Q$ and $P\not\red$, otherwise.

\section{Replication}

As mentioned before, it is known that replication (and hence
recursion) can be implemented in a higher-order process algebra
\cite{SangiorgiWalker}. As our first example of calculation with the
machinery thus far presented we give the construction explicitly in
the {\rhoc}.

\begin{eqnarray}
	D_{x} & := & \prefix{x}{y}{(\binpar{\outputp{x}{y}}{@{y}})} \nonumber\\
	\bangp_{x}{P} & := & \binpar{{x}!\langle{\binpar{D_{x}}{P}}\rangle}{D_{x}} \nonumber
\end{eqnarray}

\begin{eqnarray}
	\bangp_{x}{P} & & \nonumber\\
	=
	& {x}!\langle{(\prefix{x}{y}{(\outputp{x}{y} | @{y})) | P}}\rangle 
	      | \prefix{x}{y}{(\outputp{x}{y} | @{y})} & \nonumber\\
	\red
	& (\outputp{x}{y} | @{y})\substn{\quotep{(\prefix{x}{y}{(@{y} | \outputp{x}{y})) | P}}}{y} & \nonumber\\
	=
	& \outputp{x}{\quotep{(\prefix{x}{y}{(\outputp{x}{y} | @{y})) | P}}}
	  | {(\prefix{x}{y}{(\outputp{x}{y} | @{y})) | P}} & \nonumber\\
	\red
	& \ldots & \nonumber\\
	\red^*
	& P | P | \ldots & \nonumber
\end{eqnarray}

Of course, this encoding, as an implementation, runs away, unfolding
$\bangp{P}$ eagerly. A lazier and more implementable replication
operator, restricted to input-guarded processes, may be obtained as follows.

\begin{eqnarray}
\bangp{\prefix{u}{v}{P}} 
	:= 
	\binpar{\lift{x}{\prefix{u}{v}{(\binpar{D(x)}{P})}}}{D(x)} \nonumber
\end{eqnarray}

\begin{remark}
  Note that the lazier definition still does not deal with summation
  or mixed summation (i.e. sums over input and output). The reader is
  invited to construct definitions of replication that deal with these
  features. 

  Further, the definitions are parameterized in a name, $x$. Can you,
  gentle reader, make a definition that eliminates this parameter and
  guarantees no accidental interaction between the replication
  machinery and the process being replicated -- i.e. no accidental
  sharing of names used by the process to get its work done and the
  name(s) used by the replication to effect copying. This latter
  revision of the definition of replication is crucial to obtaining
  the expected identity $!!P \sim !P$.
\end{remark}

\begin{remark}\label{rem:paradoxical_combinator}
  The reader familiar with the lambda calculus will have noticed the
  similarity between $D$ and the paradoxical combinator.

  [Ed. note: the existence of this seems to suggest we have to be more
  restrictive on the set of processes and names we admit if we are to
  support no-cloning.]
\end{remark}

\subsubsection{Bisimulation}

The computational dynamics gives rise to another kind of equivalence,
the equivalence of computational behavior. As previously mentioned
this is typically captured \emph{via} some form of bisimulation.

% The notion we use in this paper is weak barbed bisimulation
% \cite{milner91polyadicpi}.

The notion we use in this paper is derived from weak barbed
bisimulation \cite{milner91polyadicpi}. 

\begin{definition}
An \emph{observation relation}, $\downarrow_{\mathcal N}$, over a set
of names, $\mathcal N$, is the smallest relation satisfying the rules
below.

\infrule[Out-barb]{y \in {\mathcal N}, \; x \nameeq y}
		  {\outputp{x}{v} \downarrow_{\mathcal N} x}
\infrule[Par-barb]{\mbox{$P\downarrow_{\mathcal N} x$ or $Q\downarrow_{\mathcal N} x$}}
		  {\binpar{P}{Q} \downarrow_{\mathcal N} x}

We write $P \Downarrow_{\mathcal N} x$ if there is $Q$ such that 
$P \wred Q$ and $Q \downarrow_{\mathcal N} x$.
\end{definition}

\begin{definition}
%\label{def.bbisim}
An  ${\mathcal N}$-\emph{barbed bisimulation} over a set of names, ${\mathcal N}$, is a symmetric binary relation 
${\mathcal S}_{\mathcal N}$ between agents such that $P\rel{S}_{\mathcal N}Q$ implies:
\begin{enumerate}
\item If $P \red P'$ then $Q \wred Q'$ and $P'\rel{S}_{\mathcal N} Q'$.
\item If $P\downarrow_{\mathcal N} x$, then $Q\Downarrow_{\mathcal N} x$.
\end{enumerate}
$P$ is ${\mathcal N}$-barbed bisimilar to $Q$, written
$P \wbbisim_{\mathcal N} Q$, if $P \rel{S}_{\mathcal N} Q$ for some ${\mathcal N}$-barbed bisimulation ${\mathcal S}_{\mathcal N}$.
\end{definition}

$\mathcal{R} \subseteq \pi \times \pi$

$P \mathcal{R} Q => \forall P'. P \red P' \Rightarrow \exists Q'. Q \red Q', P' \mathcal{R} Q'$

$P \vdash x \Rightarrow Q \vdash x$

\begin{mathpar}
  \inferrule*[lab=Out-barb]{x \nameeq y}{{y}!\langle{Q}\rangle \vdash x}
  \and
  \inferrule*[lab=Par-barb]{\mbox{$P\vdash x$ or $Q\vdash x$}}{\binpar{P}{Q} \vdash x}
\end{mathpar}

\subsubsection{Contexts}

One of the principle advantages of computational calculi like the
$\pi$-calculus is a well-defined notion of context,
contextual-equivalence and a correlation between
contextual-equivalence and notions of bisimulation. The notion of
context allows the decomposition of a process into (sub-)process and
its syntactic environment, its context. Thus, a context may be
thought of as a process with a ``hole'' (written $\Box$) in it. The
application of a context $M$ to a process $P$, written $M[P]$, is
tantamount to filling the hole in $M$ with $P$. In this paper we do
not need the full weight of this theory, but do make use of the notion
of context in the proof the main theorem. 

\begin{mathpar}
  \inferrule* [lab=summation] {} {{M_{M},M_{N}} \bc \Box \;|\; x.M_{A} \;|\; M_{M}+M_{N}}
  \and
  \inferrule* [lab=agent] {} {{M_{A}} \bc (\vec{x})M_{P} \;| \; \clift{P_0,\ldots,M_{P},\ldots,P_N}}
  \and \\
  \inferrule* [lab=process] {} {{M_{P}} \bc M_{N} \;| \;P|M_{P} }
\end{mathpar} 

\begin{mathpar}
  \inferrule* [lab=sychronization] {} {M_{N} \bc \Box \;|\; x?M_{F} \;|\; x!M_{C}}
  \and
  \inferrule* [lab=abstraction] {} {{M_{F}} \bc (x)M_{P} }
  \and
  \inferrule* [lab=concretion] {} {{M_{C}} \bc \langle M_{P} \rangle }
  \and \\
  \inferrule* [lab=process] {} {{M_{P}} \bc M_{N} \;| \;P|M_{P} }
\end{mathpar}

\begin{definition}[contextual application] Given a context $M$, and
  process $P$, we define the \emph{contextual application}, $M[P] :=
  M\{P/\Box\}$. That is, the contextual application of M to P is the
  substitution of $P$ for $\Box$ in $M$.
\end{definition}

$\meaningof{-} : L \to \mathcal{P}(\pi)$

\begin{mathpar}
  \inferrule* [lab=collection] {} {\meaningof{true} = \pi, \and \meaningof{~E} = \pi \setminus \meaningof{E}, \and \meaningof{E_{1} \& E_{2}} = \meaningof{E_{1}} \cap \meaningof{E_{2}}}
\end{mathpar}

\begin{mathpar}
  \inferrule* [lab=structure] {} {\meaningof{0} = \{ P \in \pi | P \equiv 0 \}, \and \\ \meaningof{E_1 | E_2} = \{ P \in \pi | P \equiv P_{1} | P_{2}, P_{1} \in \meaningof{E_{1}}, P_{2} \in \meaningof{E_2}\} }
\end{mathpar}

\begin{mathpar}
 \inferrule* [lab=behavior] {} {\meaningof{\langle a?b \rangle E} = \{ P \in \pi | P \equiv Q | u?(y)P', \\ \and \\\\ \and \\ \;\;\; u \in \meaningof{a}, \forall z.P'\{z/y\} \in \meaningof{E\{z/b\}}\}, \and \\ \meaningof{a!E} = \{ P \in \pi | P \equiv Q | x!\langle P' \rangle, x \in \meaningof{a} P' \in \meaningof{E}\} }
\end{mathpar}

\begin{mathpar}
 \inferrule* [lab=nominal] {} {\meaningof{\quotep{E}} = \{ \quotep{P} \in \quotep{\pi} | P \in \meaningof{E} \}, \and \meaningof{\quotep{P}} = \{ \quotep{Q} \in \quotep{\pi} | P \equiv Q \} \and \\ \meaningof{@\quotep{E}} = \{ P \in \pi | P \equiv @x, x \in \meaningof{E} \}}
\end{mathpar}

\begin{eqnarray*}
  \\
  \meaningof{-} : TS \to ST
\end{eqnarray*}

\begin{eqnarray*}
  \\
  L : TS \to ST
\end{eqnarray*}

\begin{eqnarray*}
  \\
  P \models E \iff P \in \meaningof{E}
\end{eqnarray*}

\begin{eqnarray*}
  P \approx_{L} Q \iff \forall E \in L. P \models E \iff Q \models E
\end{eqnarray*}

\begin{eqnarray*}
  P \approx_{K} Q
\end{eqnarray*}

\begin{eqnarray*}
  P \approx Q
\end{eqnarray*}

$\approx_{K} = \approx = \approx_{L}$

\subsubsection{Contextual duality}

Note that contexts extend the quotation operation to a family of
operations from processes to names. Given a context, $M$, we can
define a \emph{nominal context}, $\quotep{M}$ by $\quotep{M}[P] :=
\quotep{M[P]}$. To foreshadow what is to come we observe that these
operations enjoy a duality with processes very much like the duality
between vectors and maps from vectors to scalars.

Further, because the calculus is essentially higher-order, we have a
correspondence between contexts and processes. More specifically,
given a name $x$ and a context $M$ we can construct $M^{*}_{x}$ such
that 

\begin{mathpar}
  M^{*}_{x} | \lift{x}{P} \red M[P]
\end{mathpar}

namely,

\begin{mathpar}
  M^{*}_{x} := x?(u).M[\dropn{u}]
\end{mathpar}

The dependence of $M^{*}_{x}$ on a name makes it an abstraction, 

\begin{mathpar}
  M^{*} := (x)x?(u).M[\dropn{u}]
\end{mathpar}

\subsection{Additional notation}

It will sometimes be convenient to denote the process a name
quotes. We already have the notation $x = \quotep{P}$, but it will be
convenient to introduce an alternate notation, $\procn{x}$, when we
want to emphasize the connection to the use of the name. Note that, by
virtue of name equivalence, $\quotep{\procn{x}} \nameeq x$; so, the
notation is consistent with previous definitions.

Further, because names have structure it is possible to effect
substitutions on the basis of that structure. This means we need to
upgrade our notation for substitutions, which we accomplish by
adapting comprehension notation. Thus,

\begin{mathpar}
  P\{ y / x : x \in S \}
\end{mathpar}

is interpreted to mean the process derived from P by replacing (in a
capture-avoiding manner) each occurrence of $x$ in $S$ by $y$. For example,

\begin{mathpar}
  P\{ \quotep{\procn{x}|\procn{x}} / x : x \in \freenames{P} \}
\end{mathpar}

will replace each (occurrence) of a free name $x$ in $P$ by
$\quotep{\procn{x}|\procn{x}}$.

Also, we will avail ourselves of the notation $x^{L}$ and $x^{R}$ to
denote injections of a name into disjoint copies of the name
space. There are numerous ways to accomplish this. One example can be
found in \cite{MeredithR05}. This notation overloads to vectors of
names: $\vec{x}^{\pi} := (x_{i}^{\pi} \; : \; 0 \leq i < |\vec{x}| )$ where $\pi \in \{L,R\}$.

We also use $P^{\Box} := P|\Box$.

In \cite{MeredithR05} an interpretation of the new operator is
given. It turns out that there are several possible interpretations
all enjoying the requisite algebraic properties of the operator (see
\cite{milner91polyadicpi}). We will therefore make liberal use of
$(\nu\; \vec{x})P$.

% subsection the_syntax_and_semantics_of_the_notation_system (end)   

\input{qm2pi.qmops} 

\input{qm2pi.sterngerlach} 

\input{qm2pi.metric} 

% section concurrent_process_calculi (end)

%\input{qm2pi.proofsketch}

% section proof sketch (end)

%\input{qm2pi.slviaknots} 

% section spatial logic via knots (end)

\input{qm2pi.conclusion}

% section conclusion (end)

%\input{qm2pi.dtcodes} 

% section wiring algorithm (end)

\input{qm2pi.ack} 

% section acknowledgments (end)

\newpage


\bibliographystyle{plain}   
\bibliography{../../biblios/main.bib}

\input{qm2pi.rhodetails}

\end{document}

 

\documentclass[12pt]{llncs}
%\documentclass{jktr}

\usepackage[pdftex]{hyperref}                   
\usepackage {listings}
\usepackage {mathpartir}
\usepackage{bcprules}
%\usepackage{listings}
                       
\usepackage{graphicx} 
%\usepackage[margins=2.5cm,nohead,nofoot]{geometry}
%\usepackage{geometry}
\usepackage{amsfonts}
\usepackage{amstext}
\usepackage{latexsym}
\usepackage{amssymb}
\usepackage{color}


%\include{myPreamble}
\include{qm2pi.local} 

%\ifpdf
%\usepackage[pdftex]{graphicx}
%\else
%\usepackage{graphicx}
%\fi

 % \ifpdf
%  \usepackage{pdfsync}
%  \if


%\title{Brief Article}
%\author{David F. Snyder}
%\author{L.G. Meredith}

%\address{Dept. of Math., Texas State University--San Marcos, San Marcos, TX 78666}
       
\pagestyle{empty}


\begin{document}

\lstset{language=[Objective]Caml,frame=shadowbox}

\input{qm2pi.front}

% section front matter (end)

\input{qm2pi.intro} 
 
% section introduction (end)

% \input{qm2pi.knotations} 

% section notation (end)

\input{qm2pi.process.calculi} 

% section concurrent_process_calculi_and_spatial_logics_ (end)
    
%\input{qm2pi.knots2pi} 

%\input{qm2pi.trefoil} 

%\input{qm2pi.mainthm} 

% subsection basic_interpretation (end)

%\input{qm2pi.rho.presentation} 
\subsection{The syntax and semantics of the notation system}\label{sub:the_syntax_and_semantics_of_the_notation_system} % (fold)

We now summarize a technical presentation of the calculus that
embodies our theory of dynamics. The typical presentation of such a
calculus follows the style of giving generators and relations on
them. The grammar, below, describing term constructors, freely
generates the set of processes, $\Proc$. This set is then quotiented
by a relation known as structural congruence and it is over this set
that the notion of dynamics is expressed. This presentation is
essentially that of \cite{MeredithR05} with the addition of
polyadicity and summation. For readability we have relegated some of
the technical subtleties to an appendix.

\subsubsection{Process grammar}\label{subsub:process_grammar}

\begin{mathpar}
  \inferrule* [lab=synchronization] {} {{M} \bc \pzero \;|\; x?F \;|\; x!C }
  \and
  \inferrule* [lab=abstraction] {} {{F} \bc (x)P}
  \and
  \inferrule* [lab=concretion] {} {{C} \bc \langle Q \rangle}
  \and
  \inferrule* [lab=process] {} {{P,Q} \bc M \;| \;P|Q \;|\; @{x}}
  \and
  \inferrule* [lab=name] {} {{x} \bc \quotep{P}}
\end{mathpar} 

Note that $\vec{x}$ (resp. $\vec{P}$) denotes a vector of names
(resp. processes) of length $|\vec{x}|$ (resp. $|\vec{P}|$). We adopt
the following useful abbreviations.

\begin{mathpar}
   x?(\vec{y}).P := x.(\vec{y})P \and  x\clift{\vec{P}} := x.\clift{\vec{P}}
   \and x!(y) := \lift{x}{\dropn{y}}
   \and \Pi_{i=0}^{n-1}P_i := P_0 | \ldots | P_{n-1}
\end{mathpar}

\subsubsection{Structural congruence}

\paragraph{Free and bound names and alpha-equivalence.} At the
core of structural equivalence is alpha-equivalence which identifies
process that are the same up to a change of variable. Formally, we
recognize the distinction between free and bound names. The free names
of a process, $\freenames{P}$, may be calculated recursively as
follows:

\begin{mathpar}
\freenames{\pzero} := \emptyset
  \and \\
  \freenames{x?(y).P} := \{ x \} \cup (\freenames{P} \setminus \{ y \})
  \and 
  \freenames{x!\langle P \rangle} := \{ x \} \cup \{ P \} 
  \and \\
  \freenames{P|Q} := \freenames{P} \cup \freenames{Q}
  \and \\
  \freenames{@{x}} := \{ x \}
\end{mathpar}

$\pi$
$\quotep{\pi}$

$\freenames{-} : \pi \to \mathcal{P}(\quotep{\pi})$

\begin{eqnarray*}
  \freenames{\pzero} & := & \emptyset \\
  \freenames{x?(y).P} & := & \{ x \} \cup (\freenames{P} \setminus \{ y \}) \\
  \freenames{x!\langle P \rangle} & := & \{ x \} \cup \{ P \} \\
  \freenames{P|Q} & := & \freenames{P} \cup \freenames{Q} \\
  \freenames{\dropn{x}} & := & \{ x \}
\end{eqnarray*}

The bound names of a process, $\boundnames{P}$, are those names occurring in $P$
that are not free. For example, in $x?(y).0$, the name $x$ is free, while $y$ is bound.

\begin{mathpar}
  \inferrule* [lab=monoidal-laws] {} { P|Q \equiv Q|P \and P|0 \equiv P \and P|(Q|R) \equiv (P|Q)|R }
\end{mathpar}

\begin{mathpar}
  \inferrule* [lab=alpha-equivalence] {} { (x)P \equiv (y)P\{y/x\} \and y \not\in \freenames{P} }
\end{mathpar}

\begin{definition}
Then two processes, $P,Q$, are alpha-equivalent if $P = Q\{\vec{y}/\vec{x}\}$ for
some $\vec{x} \in \boundnames{Q},\vec{y} \in \boundnames{P}$, where $Q\{\vec{y}/\vec{x}\}$
denotes the capture-avoiding substitution of $\vec{y}$ for $\vec{x}$ in $Q$.
\end{definition}

\begin{definition}
  The {\em structural congruence} \cite{SangiorgiWalker} , $\equiv$,
  between processes is the least congruence containing
  alpha-equivalence, satisfying the abelian monoid laws
  (associativity, commutativity and $\pzero$ as identity) for parallel
  composition $|$ and for summation $+$.
\end{definition}

\subsection{Name equivalence}

We take name equivalence, written $\nameeq$, to be the smallest
equivalence relation generated by the following rules.

\begin{mathpar}
\inferrule*[lab=Quote-drop]
{ }
{ \quotep{@{x}} \nameeq x }

\inferrule*[lab=Struct-equiv]
{ P \scong Q }
{ \quotep{P} \nameeq \quotep{Q} }
\end{mathpar}

The astute reader will have noticed that the mutual recursion of names
and processes imposes a mutual recursion on alpha-equivalence and
structural equivalence via name-equivalence. Fortunately, all of this
works out pleasantly and we may calculate in the natural way, free of
concern. The reader interested in the details is referred to the
appendix \ref{appendix:rho_details}.

\subsection{Substitution}

We use $\Proc$ for the set of processes, $\QProc$ for the set of
names, and $\id{\{}\vec{y} / \vec{x} \id{\}}$ to denote partial maps,
$s : \QProc \rightarrow \QProc$. A map, $s$ lifts, uniquely, to a map
on process terms, $\widehat{s} : \Proc \rightarrow \Proc$ by the
following equations.

\begin{mathpar}
  (0) \psubstp{Q}{P} := 0 \\
  (R \juxtap S) \psubstp{Q}{P}
  :=    
  (R)\psubstp{Q}{P} \juxtap (S) \psubstp{Q}{P} \\
  (x?(y).R) \psubstp{Q}{P}    
  :=    
  (x)\substp{Q}{P} (z)\concat( (R \psubstn{z}{y}) \psubstp{Q}{P} ) \\
  (\lift{x}{R}) \psubstp{Q}{P}  
  :=
  \lift{(x)\substp{Q}{P}}{ R \psubstp{Q}{P} } \\
%   (\dropn{x})  \psubstp{Q}{P}       
%   := 
%   \left\{ 
%     \begin{array}{ccc} 
%       \dropn{\quotep{Q}} & & x \nameeq \quotep{P} \\
%       \dropn{x} & & otherwise \\
%     \end{array}
%   \right. 
  (\dropn{x})  \psubstp{Q}{P}       
  := 
  \left\{ 
    \begin{array}{ccc} 
      Q & & x \nameeq \quotep{P} \\
      \dropn{x} & & otherwise \\
    \end{array}
  \right.
\end{mathpar}
 

where

\begin{eqnarray}
  (x)\id{\{} \lpquote Q \rpquote / \lpquote P \rpquote \id{\}}            = 
  \left\{ 
    \begin{array}{ccc}
      \lpquote Q \rpquote & & x \nameeq \lpquote P \rpquote \\
      x & & otherwise \\
    \end{array}
  \right. \nonumber
\end{eqnarray}

and $z$ is chosen distinct from $\quotep{P}$, $\quotep{Q}$, the free
names in $Q$, and all the names in $R$. Our $\alpha$-equivalence will
be built in the standard way from this substitution.

\begin{remark}\label{rem:no_self_referential_names}
  One consequence of these definitions is that $\forall P. \quotep{P}
  \not\in \freenames{P}$.
\end{remark}

\subsection{ Dynamic quote: an example }

Anticipating something of what's to come, consider applying the
substitution, $\widehat{\id{\{}u / z \id{\}}}$, to the following pair
of processes, $\lift{w}{y!(z)}$ and $w[ \lpquote y!(z) \rpquote ]$.

\begin{eqnarray}
	\lift{w}{y!(z)}\widehat{\id{\{}u / z \id{\}}}
		& = &
		\lift{w}{y!(u)} \nonumber\\
	w[ \lpquote y!(z) \rpquote ] \widehat{ \id{\{}u / z \id{\}} }
		& = &
		w[ \lpquote y!(z) \rpquote ] \nonumber
\end{eqnarray}

Because the body of the process between quotes is impervious to
substitution, we get radically different answers. In fact, by
examining the first process in an input context,
e.g. $x?(z).\lift{w}{y!(z)}$, we see that the process under the lift
operator may be shaped by prefixed inputs binding a name inside it. In
this sense, the lift operator will be seen as a way to dynamically
construct processes before reifying them as names.

Finally equipped with these standard features we can present the
dynamics of the calculus.

\subsubsection{Operational semantics} 

Finally, we introduce the computational dynamics. What marks these
algebras as distinct from other more traditionally studied algebraic
structures, e.g. vector spaces or polynomial rings, is the manner in
which dynamics is captured. In traditional structures, dynamics is typically
expressed through morphisms between such structures, as in linear maps
between vector spaces or morphisms between rings. In algebras
associated with the semantics of computation, the dynamics is
expressed as part of the algebraic structure itself, through a
reduction reduction relation typically denoted by $\red$. Below, we
give a recursive presentation of this relation for the calculus used
in the encoding.

$\red \subseteq \pi \times \pi$
$\red : \pi \to \mathcal{P}(\pi)$

\begin{mathpar}
  \inferrule* [lab=Comm] { \textsf{match}( x_{src}, x_{trgt} ) } { x_{trgt}?(y)P \; | \; x_{src}!\langle {Q} \rangle \red P\{\quotep{Q}/y}\} }
  \and \\
  \inferrule* [lab=Par] {{P} \red {P}'} {{{P} | {Q}} \red {{P}' | {Q}}}
  \and
  \inferrule* [lab=Equiv]{{{P} \scong {P}'} \andalso {{P}' \red {Q}'} \andalso {{Q}' \scong {Q}}}{{P} \red {Q}}
\end{mathpar}

\begin{eqnarray*}
  match_{\equiv} (\quotep{P},\quotep{Q}) & := & P \equiv Q \\
  match_{\dagger}(\quotep{P},\quotep{Q}) & := & \forall R. P|Q \red^{*} R => R \red^{*} 0 \\
  match_{K}(\quotep{P},\quotep{Q}) & := & K \mbox{ for some context } K
\end{eqnarray*}

$u?(x)P | u!\langle Q \rangle \red P\{\quotep{Q}/x\}$

%We write $\wred$ for $\red^*$, and $P\red$ if $\exists Q $ such that $ P \red Q$.
We write $P\red$ if $\exists Q $ such that $ P \red Q$ and $P\not\red$, otherwise.

\section{Replication}

As mentioned before, it is known that replication (and hence
recursion) can be implemented in a higher-order process algebra
\cite{SangiorgiWalker}. As our first example of calculation with the
machinery thus far presented we give the construction explicitly in
the {\rhoc}.

\begin{eqnarray}
	D_{x} & := & \prefix{x}{y}{(\binpar{\outputp{x}{y}}{@{y}})} \nonumber\\
	\bangp_{x}{P} & := & \binpar{{x}!\langle{\binpar{D_{x}}{P}}\rangle}{D_{x}} \nonumber
\end{eqnarray}

\begin{eqnarray}
	\bangp_{x}{P} & & \nonumber\\
	=
	& {x}!\langle{(\prefix{x}{y}{(\outputp{x}{y} | @{y})) | P}}\rangle 
	      | \prefix{x}{y}{(\outputp{x}{y} | @{y})} & \nonumber\\
	\red
	& (\outputp{x}{y} | @{y})\substn{\quotep{(\prefix{x}{y}{(@{y} | \outputp{x}{y})) | P}}}{y} & \nonumber\\
	=
	& \outputp{x}{\quotep{(\prefix{x}{y}{(\outputp{x}{y} | @{y})) | P}}}
	  | {(\prefix{x}{y}{(\outputp{x}{y} | @{y})) | P}} & \nonumber\\
	\red
	& \ldots & \nonumber\\
	\red^*
	& P | P | \ldots & \nonumber
\end{eqnarray}

Of course, this encoding, as an implementation, runs away, unfolding
$\bangp{P}$ eagerly. A lazier and more implementable replication
operator, restricted to input-guarded processes, may be obtained as follows.

\begin{eqnarray}
\bangp{\prefix{u}{v}{P}} 
	:= 
	\binpar{\lift{x}{\prefix{u}{v}{(\binpar{D(x)}{P})}}}{D(x)} \nonumber
\end{eqnarray}

\begin{remark}
  Note that the lazier definition still does not deal with summation
  or mixed summation (i.e. sums over input and output). The reader is
  invited to construct definitions of replication that deal with these
  features. 

  Further, the definitions are parameterized in a name, $x$. Can you,
  gentle reader, make a definition that eliminates this parameter and
  guarantees no accidental interaction between the replication
  machinery and the process being replicated -- i.e. no accidental
  sharing of names used by the process to get its work done and the
  name(s) used by the replication to effect copying. This latter
  revision of the definition of replication is crucial to obtaining
  the expected identity $!!P \sim !P$.
\end{remark}

\begin{remark}\label{rem:paradoxical_combinator}
  The reader familiar with the lambda calculus will have noticed the
  similarity between $D$ and the paradoxical combinator.

  [Ed. note: the existence of this seems to suggest we have to be more
  restrictive on the set of processes and names we admit if we are to
  support no-cloning.]
\end{remark}

\subsubsection{Bisimulation}

The computational dynamics gives rise to another kind of equivalence,
the equivalence of computational behavior. As previously mentioned
this is typically captured \emph{via} some form of bisimulation.

% The notion we use in this paper is weak barbed bisimulation
% \cite{milner91polyadicpi}.

The notion we use in this paper is derived from weak barbed
bisimulation \cite{milner91polyadicpi}. 

\begin{definition}
An \emph{observation relation}, $\downarrow_{\mathcal N}$, over a set
of names, $\mathcal N$, is the smallest relation satisfying the rules
below.

\infrule[Out-barb]{y \in {\mathcal N}, \; x \nameeq y}
		  {\outputp{x}{v} \downarrow_{\mathcal N} x}
\infrule[Par-barb]{\mbox{$P\downarrow_{\mathcal N} x$ or $Q\downarrow_{\mathcal N} x$}}
		  {\binpar{P}{Q} \downarrow_{\mathcal N} x}

We write $P \Downarrow_{\mathcal N} x$ if there is $Q$ such that 
$P \wred Q$ and $Q \downarrow_{\mathcal N} x$.
\end{definition}

\begin{definition}
%\label{def.bbisim}
An  ${\mathcal N}$-\emph{barbed bisimulation} over a set of names, ${\mathcal N}$, is a symmetric binary relation 
${\mathcal S}_{\mathcal N}$ between agents such that $P\rel{S}_{\mathcal N}Q$ implies:
\begin{enumerate}
\item If $P \red P'$ then $Q \wred Q'$ and $P'\rel{S}_{\mathcal N} Q'$.
\item If $P\downarrow_{\mathcal N} x$, then $Q\Downarrow_{\mathcal N} x$.
\end{enumerate}
$P$ is ${\mathcal N}$-barbed bisimilar to $Q$, written
$P \wbbisim_{\mathcal N} Q$, if $P \rel{S}_{\mathcal N} Q$ for some ${\mathcal N}$-barbed bisimulation ${\mathcal S}_{\mathcal N}$.
\end{definition}

$\mathcal{R} \subseteq \pi \times \pi$

$P \mathcal{R} Q => \forall P'. P \red P' \Rightarrow \exists Q'. Q \red Q', P' \mathcal{R} Q'$

$P \vdash x \Rightarrow Q \vdash x$

\begin{mathpar}
  \inferrule*[lab=Out-barb]{x \nameeq y}{{y}!\langle{Q}\rangle \vdash x}
  \and
  \inferrule*[lab=Par-barb]{\mbox{$P\vdash x$ or $Q\vdash x$}}{\binpar{P}{Q} \vdash x}
\end{mathpar}

\subsubsection{Contexts}

One of the principle advantages of computational calculi like the
$\pi$-calculus is a well-defined notion of context,
contextual-equivalence and a correlation between
contextual-equivalence and notions of bisimulation. The notion of
context allows the decomposition of a process into (sub-)process and
its syntactic environment, its context. Thus, a context may be
thought of as a process with a ``hole'' (written $\Box$) in it. The
application of a context $M$ to a process $P$, written $M[P]$, is
tantamount to filling the hole in $M$ with $P$. In this paper we do
not need the full weight of this theory, but do make use of the notion
of context in the proof the main theorem. 

\begin{mathpar}
  \inferrule* [lab=summation] {} {{M_{M},M_{N}} \bc \Box \;|\; x.M_{A} \;|\; M_{M}+M_{N}}
  \and
  \inferrule* [lab=agent] {} {{M_{A}} \bc (\vec{x})M_{P} \;| \; \clift{P_0,\ldots,M_{P},\ldots,P_N}}
  \and \\
  \inferrule* [lab=process] {} {{M_{P}} \bc M_{N} \;| \;P|M_{P} }
\end{mathpar} 

\begin{mathpar}
  \inferrule* [lab=sychronization] {} {M_{N} \bc \Box \;|\; x?M_{F} \;|\; x!M_{C}}
  \and
  \inferrule* [lab=abstraction] {} {{M_{F}} \bc (x)M_{P} }
  \and
  \inferrule* [lab=concretion] {} {{M_{C}} \bc \langle M_{P} \rangle }
  \and \\
  \inferrule* [lab=process] {} {{M_{P}} \bc M_{N} \;| \;P|M_{P} }
\end{mathpar}

\begin{definition}[contextual application] Given a context $M$, and
  process $P$, we define the \emph{contextual application}, $M[P] :=
  M\{P/\Box\}$. That is, the contextual application of M to P is the
  substitution of $P$ for $\Box$ in $M$.
\end{definition}

$\meaningof{-} : L \to \mathcal{P}(\pi)$

\begin{mathpar}
  \inferrule* [lab=collection] {} {\meaningof{true} = \pi, \and \meaningof{~E} = \pi \setminus \meaningof{E}, \and \meaningof{E_{1} \& E_{2}} = \meaningof{E_{1}} \cap \meaningof{E_{2}}}
\end{mathpar}

\begin{mathpar}
  \inferrule* [lab=structure] {} {\meaningof{0} = \{ P \in \pi | P \equiv 0 \}, \and \\ \meaningof{E_1 | E_2} = \{ P \in \pi | P \equiv P_{1} | P_{2}, P_{1} \in \meaningof{E_{1}}, P_{2} \in \meaningof{E_2}\} }
\end{mathpar}

\begin{mathpar}
 \inferrule* [lab=behavior] {} {\meaningof{\langle a?b \rangle E} = \{ P \in \pi | P \equiv Q | u?(y)P', \\ \and \\\\ \and \\ \;\;\; u \in \meaningof{a}, \forall z.P'\{z/y\} \in \meaningof{E\{z/b\}}\}, \and \\ \meaningof{a!E} = \{ P \in \pi | P \equiv Q | x!\langle P' \rangle, x \in \meaningof{a} P' \in \meaningof{E}\} }
\end{mathpar}

\begin{mathpar}
 \inferrule* [lab=nominal] {} {\meaningof{\quotep{E}} = \{ \quotep{P} \in \quotep{\pi} | P \in \meaningof{E} \}, \and \meaningof{\quotep{P}} = \{ \quotep{Q} \in \quotep{\pi} | P \equiv Q \} \and \\ \meaningof{@\quotep{E}} = \{ P \in \pi | P \equiv @x, x \in \meaningof{E} \}}
\end{mathpar}

\begin{eqnarray*}
  \\
  \meaningof{-} : TS \to ST
\end{eqnarray*}

\begin{eqnarray*}
  \\
  L : TS \to ST
\end{eqnarray*}

\begin{eqnarray*}
  \\
  P \models E \iff P \in \meaningof{E}
\end{eqnarray*}

\begin{eqnarray*}
  P \approx_{L} Q \iff \forall E \in L. P \models E \iff Q \models E
\end{eqnarray*}

\begin{eqnarray*}
  P \approx_{K} Q
\end{eqnarray*}

\begin{eqnarray*}
  P \approx Q
\end{eqnarray*}

$\approx_{K} = \approx = \approx_{L}$

\subsubsection{Contextual duality}

Note that contexts extend the quotation operation to a family of
operations from processes to names. Given a context, $M$, we can
define a \emph{nominal context}, $\quotep{M}$ by $\quotep{M}[P] :=
\quotep{M[P]}$. To foreshadow what is to come we observe that these
operations enjoy a duality with processes very much like the duality
between vectors and maps from vectors to scalars.

Further, because the calculus is essentially higher-order, we have a
correspondence between contexts and processes. More specifically,
given a name $x$ and a context $M$ we can construct $M^{*}_{x}$ such
that 

\begin{mathpar}
  M^{*}_{x} | \lift{x}{P} \red M[P]
\end{mathpar}

namely,

\begin{mathpar}
  M^{*}_{x} := x?(u).M[\dropn{u}]
\end{mathpar}

The dependence of $M^{*}_{x}$ on a name makes it an abstraction, 

\begin{mathpar}
  M^{*} := (x)x?(u).M[\dropn{u}]
\end{mathpar}

\subsection{Additional notation}

It will sometimes be convenient to denote the process a name
quotes. We already have the notation $x = \quotep{P}$, but it will be
convenient to introduce an alternate notation, $\procn{x}$, when we
want to emphasize the connection to the use of the name. Note that, by
virtue of name equivalence, $\quotep{\procn{x}} \nameeq x$; so, the
notation is consistent with previous definitions.

Further, because names have structure it is possible to effect
substitutions on the basis of that structure. This means we need to
upgrade our notation for substitutions, which we accomplish by
adapting comprehension notation. Thus,

\begin{mathpar}
  P\{ y / x : x \in S \}
\end{mathpar}

is interpreted to mean the process derived from P by replacing (in a
capture-avoiding manner) each occurrence of $x$ in $S$ by $y$. For example,

\begin{mathpar}
  P\{ \quotep{\procn{x}|\procn{x}} / x : x \in \freenames{P} \}
\end{mathpar}

will replace each (occurrence) of a free name $x$ in $P$ by
$\quotep{\procn{x}|\procn{x}}$.

Also, we will avail ourselves of the notation $x^{L}$ and $x^{R}$ to
denote injections of a name into disjoint copies of the name
space. There are numerous ways to accomplish this. One example can be
found in \cite{MeredithR05}. This notation overloads to vectors of
names: $\vec{x}^{\pi} := (x_{i}^{\pi} \; : \; 0 \leq i < |\vec{x}| )$ where $\pi \in \{L,R\}$.

We also use $P^{\Box} := P|\Box$.

In \cite{MeredithR05} an interpretation of the new operator is
given. It turns out that there are several possible interpretations
all enjoying the requisite algebraic properties of the operator (see
\cite{milner91polyadicpi}). We will therefore make liberal use of
$(\nu\; \vec{x})P$.

% subsection the_syntax_and_semantics_of_the_notation_system (end)   

\input{qm2pi.qmops} 

\input{qm2pi.sterngerlach} 

\input{qm2pi.metric} 

% section concurrent_process_calculi (end)

%\input{qm2pi.proofsketch}

% section proof sketch (end)

%\input{qm2pi.slviaknots} 

% section spatial logic via knots (end)

\input{qm2pi.conclusion}

% section conclusion (end)

%\input{qm2pi.dtcodes} 

% section wiring algorithm (end)

\input{qm2pi.ack} 

% section acknowledgments (end)

\newpage


\bibliographystyle{plain}   
\bibliography{../../biblios/main.bib}

\input{qm2pi.rhodetails}

\end{document}

 

% section concurrent_process_calculi (end)

%\documentclass[12pt]{llncs}
%\documentclass{jktr}

\usepackage[pdftex]{hyperref}                   
\usepackage {listings}
\usepackage {mathpartir}
\usepackage{bcprules}
%\usepackage{listings}
                       
\usepackage{graphicx} 
%\usepackage[margins=2.5cm,nohead,nofoot]{geometry}
%\usepackage{geometry}
\usepackage{amsfonts}
\usepackage{amstext}
\usepackage{latexsym}
\usepackage{amssymb}
\usepackage{color}


%\include{myPreamble}
\include{qm2pi.local} 

%\ifpdf
%\usepackage[pdftex]{graphicx}
%\else
%\usepackage{graphicx}
%\fi

 % \ifpdf
%  \usepackage{pdfsync}
%  \if


%\title{Brief Article}
%\author{David F. Snyder}
%\author{L.G. Meredith}

%\address{Dept. of Math., Texas State University--San Marcos, San Marcos, TX 78666}
       
\pagestyle{empty}


\begin{document}

\lstset{language=[Objective]Caml,frame=shadowbox}

\input{qm2pi.front}

% section front matter (end)

\input{qm2pi.intro} 
 
% section introduction (end)

% \input{qm2pi.knotations} 

% section notation (end)

\input{qm2pi.process.calculi} 

% section concurrent_process_calculi_and_spatial_logics_ (end)
    
%\input{qm2pi.knots2pi} 

%\input{qm2pi.trefoil} 

%\input{qm2pi.mainthm} 

% subsection basic_interpretation (end)

%\input{qm2pi.rho.presentation} 
\subsection{The syntax and semantics of the notation system}\label{sub:the_syntax_and_semantics_of_the_notation_system} % (fold)

We now summarize a technical presentation of the calculus that
embodies our theory of dynamics. The typical presentation of such a
calculus follows the style of giving generators and relations on
them. The grammar, below, describing term constructors, freely
generates the set of processes, $\Proc$. This set is then quotiented
by a relation known as structural congruence and it is over this set
that the notion of dynamics is expressed. This presentation is
essentially that of \cite{MeredithR05} with the addition of
polyadicity and summation. For readability we have relegated some of
the technical subtleties to an appendix.

\subsubsection{Process grammar}\label{subsub:process_grammar}

\begin{mathpar}
  \inferrule* [lab=synchronization] {} {{M} \bc \pzero \;|\; x?F \;|\; x!C }
  \and
  \inferrule* [lab=abstraction] {} {{F} \bc (x)P}
  \and
  \inferrule* [lab=concretion] {} {{C} \bc \langle Q \rangle}
  \and
  \inferrule* [lab=process] {} {{P,Q} \bc M \;| \;P|Q \;|\; @{x}}
  \and
  \inferrule* [lab=name] {} {{x} \bc \quotep{P}}
\end{mathpar} 

Note that $\vec{x}$ (resp. $\vec{P}$) denotes a vector of names
(resp. processes) of length $|\vec{x}|$ (resp. $|\vec{P}|$). We adopt
the following useful abbreviations.

\begin{mathpar}
   x?(\vec{y}).P := x.(\vec{y})P \and  x\clift{\vec{P}} := x.\clift{\vec{P}}
   \and x!(y) := \lift{x}{\dropn{y}}
   \and \Pi_{i=0}^{n-1}P_i := P_0 | \ldots | P_{n-1}
\end{mathpar}

\subsubsection{Structural congruence}

\paragraph{Free and bound names and alpha-equivalence.} At the
core of structural equivalence is alpha-equivalence which identifies
process that are the same up to a change of variable. Formally, we
recognize the distinction between free and bound names. The free names
of a process, $\freenames{P}$, may be calculated recursively as
follows:

\begin{mathpar}
\freenames{\pzero} := \emptyset
  \and \\
  \freenames{x?(y).P} := \{ x \} \cup (\freenames{P} \setminus \{ y \})
  \and 
  \freenames{x!\langle P \rangle} := \{ x \} \cup \{ P \} 
  \and \\
  \freenames{P|Q} := \freenames{P} \cup \freenames{Q}
  \and \\
  \freenames{@{x}} := \{ x \}
\end{mathpar}

$\pi$
$\quotep{\pi}$

$\freenames{-} : \pi \to \mathcal{P}(\quotep{\pi})$

\begin{eqnarray*}
  \freenames{\pzero} & := & \emptyset \\
  \freenames{x?(y).P} & := & \{ x \} \cup (\freenames{P} \setminus \{ y \}) \\
  \freenames{x!\langle P \rangle} & := & \{ x \} \cup \{ P \} \\
  \freenames{P|Q} & := & \freenames{P} \cup \freenames{Q} \\
  \freenames{\dropn{x}} & := & \{ x \}
\end{eqnarray*}

The bound names of a process, $\boundnames{P}$, are those names occurring in $P$
that are not free. For example, in $x?(y).0$, the name $x$ is free, while $y$ is bound.

\begin{mathpar}
  \inferrule* [lab=monoidal-laws] {} { P|Q \equiv Q|P \and P|0 \equiv P \and P|(Q|R) \equiv (P|Q)|R }
\end{mathpar}

\begin{mathpar}
  \inferrule* [lab=alpha-equivalence] {} { (x)P \equiv (y)P\{y/x\} \and y \not\in \freenames{P} }
\end{mathpar}

\begin{definition}
Then two processes, $P,Q$, are alpha-equivalent if $P = Q\{\vec{y}/\vec{x}\}$ for
some $\vec{x} \in \boundnames{Q},\vec{y} \in \boundnames{P}$, where $Q\{\vec{y}/\vec{x}\}$
denotes the capture-avoiding substitution of $\vec{y}$ for $\vec{x}$ in $Q$.
\end{definition}

\begin{definition}
  The {\em structural congruence} \cite{SangiorgiWalker} , $\equiv$,
  between processes is the least congruence containing
  alpha-equivalence, satisfying the abelian monoid laws
  (associativity, commutativity and $\pzero$ as identity) for parallel
  composition $|$ and for summation $+$.
\end{definition}

\subsection{Name equivalence}

We take name equivalence, written $\nameeq$, to be the smallest
equivalence relation generated by the following rules.

\begin{mathpar}
\inferrule*[lab=Quote-drop]
{ }
{ \quotep{@{x}} \nameeq x }

\inferrule*[lab=Struct-equiv]
{ P \scong Q }
{ \quotep{P} \nameeq \quotep{Q} }
\end{mathpar}

The astute reader will have noticed that the mutual recursion of names
and processes imposes a mutual recursion on alpha-equivalence and
structural equivalence via name-equivalence. Fortunately, all of this
works out pleasantly and we may calculate in the natural way, free of
concern. The reader interested in the details is referred to the
appendix \ref{appendix:rho_details}.

\subsection{Substitution}

We use $\Proc$ for the set of processes, $\QProc$ for the set of
names, and $\id{\{}\vec{y} / \vec{x} \id{\}}$ to denote partial maps,
$s : \QProc \rightarrow \QProc$. A map, $s$ lifts, uniquely, to a map
on process terms, $\widehat{s} : \Proc \rightarrow \Proc$ by the
following equations.

\begin{mathpar}
  (0) \psubstp{Q}{P} := 0 \\
  (R \juxtap S) \psubstp{Q}{P}
  :=    
  (R)\psubstp{Q}{P} \juxtap (S) \psubstp{Q}{P} \\
  (x?(y).R) \psubstp{Q}{P}    
  :=    
  (x)\substp{Q}{P} (z)\concat( (R \psubstn{z}{y}) \psubstp{Q}{P} ) \\
  (\lift{x}{R}) \psubstp{Q}{P}  
  :=
  \lift{(x)\substp{Q}{P}}{ R \psubstp{Q}{P} } \\
%   (\dropn{x})  \psubstp{Q}{P}       
%   := 
%   \left\{ 
%     \begin{array}{ccc} 
%       \dropn{\quotep{Q}} & & x \nameeq \quotep{P} \\
%       \dropn{x} & & otherwise \\
%     \end{array}
%   \right. 
  (\dropn{x})  \psubstp{Q}{P}       
  := 
  \left\{ 
    \begin{array}{ccc} 
      Q & & x \nameeq \quotep{P} \\
      \dropn{x} & & otherwise \\
    \end{array}
  \right.
\end{mathpar}
 

where

\begin{eqnarray}
  (x)\id{\{} \lpquote Q \rpquote / \lpquote P \rpquote \id{\}}            = 
  \left\{ 
    \begin{array}{ccc}
      \lpquote Q \rpquote & & x \nameeq \lpquote P \rpquote \\
      x & & otherwise \\
    \end{array}
  \right. \nonumber
\end{eqnarray}

and $z$ is chosen distinct from $\quotep{P}$, $\quotep{Q}$, the free
names in $Q$, and all the names in $R$. Our $\alpha$-equivalence will
be built in the standard way from this substitution.

\begin{remark}\label{rem:no_self_referential_names}
  One consequence of these definitions is that $\forall P. \quotep{P}
  \not\in \freenames{P}$.
\end{remark}

\subsection{ Dynamic quote: an example }

Anticipating something of what's to come, consider applying the
substitution, $\widehat{\id{\{}u / z \id{\}}}$, to the following pair
of processes, $\lift{w}{y!(z)}$ and $w[ \lpquote y!(z) \rpquote ]$.

\begin{eqnarray}
	\lift{w}{y!(z)}\widehat{\id{\{}u / z \id{\}}}
		& = &
		\lift{w}{y!(u)} \nonumber\\
	w[ \lpquote y!(z) \rpquote ] \widehat{ \id{\{}u / z \id{\}} }
		& = &
		w[ \lpquote y!(z) \rpquote ] \nonumber
\end{eqnarray}

Because the body of the process between quotes is impervious to
substitution, we get radically different answers. In fact, by
examining the first process in an input context,
e.g. $x?(z).\lift{w}{y!(z)}$, we see that the process under the lift
operator may be shaped by prefixed inputs binding a name inside it. In
this sense, the lift operator will be seen as a way to dynamically
construct processes before reifying them as names.

Finally equipped with these standard features we can present the
dynamics of the calculus.

\subsubsection{Operational semantics} 

Finally, we introduce the computational dynamics. What marks these
algebras as distinct from other more traditionally studied algebraic
structures, e.g. vector spaces or polynomial rings, is the manner in
which dynamics is captured. In traditional structures, dynamics is typically
expressed through morphisms between such structures, as in linear maps
between vector spaces or morphisms between rings. In algebras
associated with the semantics of computation, the dynamics is
expressed as part of the algebraic structure itself, through a
reduction reduction relation typically denoted by $\red$. Below, we
give a recursive presentation of this relation for the calculus used
in the encoding.

$\red \subseteq \pi \times \pi$
$\red : \pi \to \mathcal{P}(\pi)$

\begin{mathpar}
  \inferrule* [lab=Comm] { \textsf{match}( x_{src}, x_{trgt} ) } { x_{trgt}?(y)P \; | \; x_{src}!\langle {Q} \rangle \red P\{\quotep{Q}/y}\} }
  \and \\
  \inferrule* [lab=Par] {{P} \red {P}'} {{{P} | {Q}} \red {{P}' | {Q}}}
  \and
  \inferrule* [lab=Equiv]{{{P} \scong {P}'} \andalso {{P}' \red {Q}'} \andalso {{Q}' \scong {Q}}}{{P} \red {Q}}
\end{mathpar}

\begin{eqnarray*}
  match_{\equiv} (\quotep{P},\quotep{Q}) & := & P \equiv Q \\
  match_{\dagger}(\quotep{P},\quotep{Q}) & := & \forall R. P|Q \red^{*} R => R \red^{*} 0 \\
  match_{K}(\quotep{P},\quotep{Q}) & := & K \mbox{ for some context } K
\end{eqnarray*}

$u?(x)P | u!\langle Q \rangle \red P\{\quotep{Q}/x\}$

%We write $\wred$ for $\red^*$, and $P\red$ if $\exists Q $ such that $ P \red Q$.
We write $P\red$ if $\exists Q $ such that $ P \red Q$ and $P\not\red$, otherwise.

\section{Replication}

As mentioned before, it is known that replication (and hence
recursion) can be implemented in a higher-order process algebra
\cite{SangiorgiWalker}. As our first example of calculation with the
machinery thus far presented we give the construction explicitly in
the {\rhoc}.

\begin{eqnarray}
	D_{x} & := & \prefix{x}{y}{(\binpar{\outputp{x}{y}}{@{y}})} \nonumber\\
	\bangp_{x}{P} & := & \binpar{{x}!\langle{\binpar{D_{x}}{P}}\rangle}{D_{x}} \nonumber
\end{eqnarray}

\begin{eqnarray}
	\bangp_{x}{P} & & \nonumber\\
	=
	& {x}!\langle{(\prefix{x}{y}{(\outputp{x}{y} | @{y})) | P}}\rangle 
	      | \prefix{x}{y}{(\outputp{x}{y} | @{y})} & \nonumber\\
	\red
	& (\outputp{x}{y} | @{y})\substn{\quotep{(\prefix{x}{y}{(@{y} | \outputp{x}{y})) | P}}}{y} & \nonumber\\
	=
	& \outputp{x}{\quotep{(\prefix{x}{y}{(\outputp{x}{y} | @{y})) | P}}}
	  | {(\prefix{x}{y}{(\outputp{x}{y} | @{y})) | P}} & \nonumber\\
	\red
	& \ldots & \nonumber\\
	\red^*
	& P | P | \ldots & \nonumber
\end{eqnarray}

Of course, this encoding, as an implementation, runs away, unfolding
$\bangp{P}$ eagerly. A lazier and more implementable replication
operator, restricted to input-guarded processes, may be obtained as follows.

\begin{eqnarray}
\bangp{\prefix{u}{v}{P}} 
	:= 
	\binpar{\lift{x}{\prefix{u}{v}{(\binpar{D(x)}{P})}}}{D(x)} \nonumber
\end{eqnarray}

\begin{remark}
  Note that the lazier definition still does not deal with summation
  or mixed summation (i.e. sums over input and output). The reader is
  invited to construct definitions of replication that deal with these
  features. 

  Further, the definitions are parameterized in a name, $x$. Can you,
  gentle reader, make a definition that eliminates this parameter and
  guarantees no accidental interaction between the replication
  machinery and the process being replicated -- i.e. no accidental
  sharing of names used by the process to get its work done and the
  name(s) used by the replication to effect copying. This latter
  revision of the definition of replication is crucial to obtaining
  the expected identity $!!P \sim !P$.
\end{remark}

\begin{remark}\label{rem:paradoxical_combinator}
  The reader familiar with the lambda calculus will have noticed the
  similarity between $D$ and the paradoxical combinator.

  [Ed. note: the existence of this seems to suggest we have to be more
  restrictive on the set of processes and names we admit if we are to
  support no-cloning.]
\end{remark}

\subsubsection{Bisimulation}

The computational dynamics gives rise to another kind of equivalence,
the equivalence of computational behavior. As previously mentioned
this is typically captured \emph{via} some form of bisimulation.

% The notion we use in this paper is weak barbed bisimulation
% \cite{milner91polyadicpi}.

The notion we use in this paper is derived from weak barbed
bisimulation \cite{milner91polyadicpi}. 

\begin{definition}
An \emph{observation relation}, $\downarrow_{\mathcal N}$, over a set
of names, $\mathcal N$, is the smallest relation satisfying the rules
below.

\infrule[Out-barb]{y \in {\mathcal N}, \; x \nameeq y}
		  {\outputp{x}{v} \downarrow_{\mathcal N} x}
\infrule[Par-barb]{\mbox{$P\downarrow_{\mathcal N} x$ or $Q\downarrow_{\mathcal N} x$}}
		  {\binpar{P}{Q} \downarrow_{\mathcal N} x}

We write $P \Downarrow_{\mathcal N} x$ if there is $Q$ such that 
$P \wred Q$ and $Q \downarrow_{\mathcal N} x$.
\end{definition}

\begin{definition}
%\label{def.bbisim}
An  ${\mathcal N}$-\emph{barbed bisimulation} over a set of names, ${\mathcal N}$, is a symmetric binary relation 
${\mathcal S}_{\mathcal N}$ between agents such that $P\rel{S}_{\mathcal N}Q$ implies:
\begin{enumerate}
\item If $P \red P'$ then $Q \wred Q'$ and $P'\rel{S}_{\mathcal N} Q'$.
\item If $P\downarrow_{\mathcal N} x$, then $Q\Downarrow_{\mathcal N} x$.
\end{enumerate}
$P$ is ${\mathcal N}$-barbed bisimilar to $Q$, written
$P \wbbisim_{\mathcal N} Q$, if $P \rel{S}_{\mathcal N} Q$ for some ${\mathcal N}$-barbed bisimulation ${\mathcal S}_{\mathcal N}$.
\end{definition}

$\mathcal{R} \subseteq \pi \times \pi$

$P \mathcal{R} Q => \forall P'. P \red P' \Rightarrow \exists Q'. Q \red Q', P' \mathcal{R} Q'$

$P \vdash x \Rightarrow Q \vdash x$

\begin{mathpar}
  \inferrule*[lab=Out-barb]{x \nameeq y}{{y}!\langle{Q}\rangle \vdash x}
  \and
  \inferrule*[lab=Par-barb]{\mbox{$P\vdash x$ or $Q\vdash x$}}{\binpar{P}{Q} \vdash x}
\end{mathpar}

\subsubsection{Contexts}

One of the principle advantages of computational calculi like the
$\pi$-calculus is a well-defined notion of context,
contextual-equivalence and a correlation between
contextual-equivalence and notions of bisimulation. The notion of
context allows the decomposition of a process into (sub-)process and
its syntactic environment, its context. Thus, a context may be
thought of as a process with a ``hole'' (written $\Box$) in it. The
application of a context $M$ to a process $P$, written $M[P]$, is
tantamount to filling the hole in $M$ with $P$. In this paper we do
not need the full weight of this theory, but do make use of the notion
of context in the proof the main theorem. 

\begin{mathpar}
  \inferrule* [lab=summation] {} {{M_{M},M_{N}} \bc \Box \;|\; x.M_{A} \;|\; M_{M}+M_{N}}
  \and
  \inferrule* [lab=agent] {} {{M_{A}} \bc (\vec{x})M_{P} \;| \; \clift{P_0,\ldots,M_{P},\ldots,P_N}}
  \and \\
  \inferrule* [lab=process] {} {{M_{P}} \bc M_{N} \;| \;P|M_{P} }
\end{mathpar} 

\begin{mathpar}
  \inferrule* [lab=sychronization] {} {M_{N} \bc \Box \;|\; x?M_{F} \;|\; x!M_{C}}
  \and
  \inferrule* [lab=abstraction] {} {{M_{F}} \bc (x)M_{P} }
  \and
  \inferrule* [lab=concretion] {} {{M_{C}} \bc \langle M_{P} \rangle }
  \and \\
  \inferrule* [lab=process] {} {{M_{P}} \bc M_{N} \;| \;P|M_{P} }
\end{mathpar}

\begin{definition}[contextual application] Given a context $M$, and
  process $P$, we define the \emph{contextual application}, $M[P] :=
  M\{P/\Box\}$. That is, the contextual application of M to P is the
  substitution of $P$ for $\Box$ in $M$.
\end{definition}

$\meaningof{-} : L \to \mathcal{P}(\pi)$

\begin{mathpar}
  \inferrule* [lab=collection] {} {\meaningof{true} = \pi, \and \meaningof{~E} = \pi \setminus \meaningof{E}, \and \meaningof{E_{1} \& E_{2}} = \meaningof{E_{1}} \cap \meaningof{E_{2}}}
\end{mathpar}

\begin{mathpar}
  \inferrule* [lab=structure] {} {\meaningof{0} = \{ P \in \pi | P \equiv 0 \}, \and \\ \meaningof{E_1 | E_2} = \{ P \in \pi | P \equiv P_{1} | P_{2}, P_{1} \in \meaningof{E_{1}}, P_{2} \in \meaningof{E_2}\} }
\end{mathpar}

\begin{mathpar}
 \inferrule* [lab=behavior] {} {\meaningof{\langle a?b \rangle E} = \{ P \in \pi | P \equiv Q | u?(y)P', \\ \and \\\\ \and \\ \;\;\; u \in \meaningof{a}, \forall z.P'\{z/y\} \in \meaningof{E\{z/b\}}\}, \and \\ \meaningof{a!E} = \{ P \in \pi | P \equiv Q | x!\langle P' \rangle, x \in \meaningof{a} P' \in \meaningof{E}\} }
\end{mathpar}

\begin{mathpar}
 \inferrule* [lab=nominal] {} {\meaningof{\quotep{E}} = \{ \quotep{P} \in \quotep{\pi} | P \in \meaningof{E} \}, \and \meaningof{\quotep{P}} = \{ \quotep{Q} \in \quotep{\pi} | P \equiv Q \} \and \\ \meaningof{@\quotep{E}} = \{ P \in \pi | P \equiv @x, x \in \meaningof{E} \}}
\end{mathpar}

\begin{eqnarray*}
  \\
  \meaningof{-} : TS \to ST
\end{eqnarray*}

\begin{eqnarray*}
  \\
  L : TS \to ST
\end{eqnarray*}

\begin{eqnarray*}
  \\
  P \models E \iff P \in \meaningof{E}
\end{eqnarray*}

\begin{eqnarray*}
  P \approx_{L} Q \iff \forall E \in L. P \models E \iff Q \models E
\end{eqnarray*}

\begin{eqnarray*}
  P \approx_{K} Q
\end{eqnarray*}

\begin{eqnarray*}
  P \approx Q
\end{eqnarray*}

$\approx_{K} = \approx = \approx_{L}$

\subsubsection{Contextual duality}

Note that contexts extend the quotation operation to a family of
operations from processes to names. Given a context, $M$, we can
define a \emph{nominal context}, $\quotep{M}$ by $\quotep{M}[P] :=
\quotep{M[P]}$. To foreshadow what is to come we observe that these
operations enjoy a duality with processes very much like the duality
between vectors and maps from vectors to scalars.

Further, because the calculus is essentially higher-order, we have a
correspondence between contexts and processes. More specifically,
given a name $x$ and a context $M$ we can construct $M^{*}_{x}$ such
that 

\begin{mathpar}
  M^{*}_{x} | \lift{x}{P} \red M[P]
\end{mathpar}

namely,

\begin{mathpar}
  M^{*}_{x} := x?(u).M[\dropn{u}]
\end{mathpar}

The dependence of $M^{*}_{x}$ on a name makes it an abstraction, 

\begin{mathpar}
  M^{*} := (x)x?(u).M[\dropn{u}]
\end{mathpar}

\subsection{Additional notation}

It will sometimes be convenient to denote the process a name
quotes. We already have the notation $x = \quotep{P}$, but it will be
convenient to introduce an alternate notation, $\procn{x}$, when we
want to emphasize the connection to the use of the name. Note that, by
virtue of name equivalence, $\quotep{\procn{x}} \nameeq x$; so, the
notation is consistent with previous definitions.

Further, because names have structure it is possible to effect
substitutions on the basis of that structure. This means we need to
upgrade our notation for substitutions, which we accomplish by
adapting comprehension notation. Thus,

\begin{mathpar}
  P\{ y / x : x \in S \}
\end{mathpar}

is interpreted to mean the process derived from P by replacing (in a
capture-avoiding manner) each occurrence of $x$ in $S$ by $y$. For example,

\begin{mathpar}
  P\{ \quotep{\procn{x}|\procn{x}} / x : x \in \freenames{P} \}
\end{mathpar}

will replace each (occurrence) of a free name $x$ in $P$ by
$\quotep{\procn{x}|\procn{x}}$.

Also, we will avail ourselves of the notation $x^{L}$ and $x^{R}$ to
denote injections of a name into disjoint copies of the name
space. There are numerous ways to accomplish this. One example can be
found in \cite{MeredithR05}. This notation overloads to vectors of
names: $\vec{x}^{\pi} := (x_{i}^{\pi} \; : \; 0 \leq i < |\vec{x}| )$ where $\pi \in \{L,R\}$.

We also use $P^{\Box} := P|\Box$.

In \cite{MeredithR05} an interpretation of the new operator is
given. It turns out that there are several possible interpretations
all enjoying the requisite algebraic properties of the operator (see
\cite{milner91polyadicpi}). We will therefore make liberal use of
$(\nu\; \vec{x})P$.

% subsection the_syntax_and_semantics_of_the_notation_system (end)   

\input{qm2pi.qmops} 

\input{qm2pi.sterngerlach} 

\input{qm2pi.metric} 

% section concurrent_process_calculi (end)

%\input{qm2pi.proofsketch}

% section proof sketch (end)

%\input{qm2pi.slviaknots} 

% section spatial logic via knots (end)

\input{qm2pi.conclusion}

% section conclusion (end)

%\input{qm2pi.dtcodes} 

% section wiring algorithm (end)

\input{qm2pi.ack} 

% section acknowledgments (end)

\newpage


\bibliographystyle{plain}   
\bibliography{../../biblios/main.bib}

\input{qm2pi.rhodetails}

\end{document}



% section proof sketch (end)

%\section{Unlikely characters: spatial logic for
  knots}\label{sub:characteristic_formulae} % (fold)

Associated to the mobile process calculi are a family of logics known
as the Hennessy-Milner logics. These logics typically enjoy a
semantics interpreting formulae as sets of processes that when
factored through the encoding outlined above allows an identification
of classes of knots with logical formulae. In the context of this
encoding the sub-family known as the spatial logics \cite{CairesC03}
\cite{CairesC04} \cite{Caires04} are of particular interest providing
several important features for expressing and reasoning about
properties (i.e. classes) of knots. We hint here at how this may be done.

%\begin{description}
%\item [structural connectives] 
\subsubsection{Structural connectives} The spatial logics enjoy
structural connectives corresponding, at the logical level, to the
parallel composition ($P | Q$) and new name ($(\nu \; x)P$)
connectives for processes. As illustrated in the examples below, these
connectives are extremely expressive given the shape of our encoding.
%\item [decideable satisfaction]

\subsubsection{Decideable satisfaction}
In \cite{Caires04} the satisfaction relation is shown to be decideable
for a rich class of processes. It further turns out that the image of
the our encoding is a proper subset of that class. This result
provides the basis for an algorithm by which to search for knots
enjoying a given property.
%\item [characteristic formulae]

\subsubsection{Characteristic formulae}
In the same paper \cite{Caires04} , Caires presents a means of calculating
characteristic formulae, selecting equivalence classes of processes
up to a pre--specified depth limit on the support set of names. Composed with our
encoding, this characteristic formula can be used to select
characteristic formulae for knots.
%\end{description}

\subsubsection{Spatial logic formulae}

The grammar below (segmented for comprehension) summarizes the syntax
of spatial logic formulae. We employ illustrative examples in the
sequel to provide an intuitive understanding of their meaning
referring the reader to \cite{Caires04} for a more detailed explication
of the semantics.

\begin{mathpar}
  \inferrule* [lab=boolean] {} {{A,B} \bc T \;|\; \neg A \;|\; A \wedge B \;|\; \eta = \eta'}
  \and
  \inferrule* [lab=spatial] {} {|\; \pzero \;|\; A | B \;|\; x \text{\textregistered} A \;|\; \forall x . A \;|\;  H x . A}
  \and
  \inferrule* [lab=behavioral] {} {|\; \alpha . A}
  \and 
  \inferrule* [lab=recursion] {} {|\; X(\vec{u}) \;|\; \mu X(\vec{u}) . A}
  \and
  \inferrule* [lab=action] {} {\alpha \bc \langle x?(\vec{y}) \rangle \;|\; \langle x!(\vec{y}) \rangle \;|\; \langle \tau \rangle}
  \and 
  \inferrule* [lab=name] {} {\eta \bc x \;|\; \tau}
\end{mathpar} 

% subsection characteristic_formulae (end)   	 

\subsection{Example formulae}\label{sub:example_formulae_} % (fold)

\subsubsection{Crossing as formula.}
% 
% \begin{align*}
%   \frac{d}{dx} \sin x &= \cos x 
%   & \frac{d}{dx} e^x &= e^x \\
%   \frac{d}{dx} \cos x &= - \sin x 
%   & \frac{d}{dx} \log x &= \frac{1}{x} \\
% \end{align*} 

\begin{align*}
 \mu C(x_{0},x_{1},y_{0},y_{1},u).&(\langle x_{0}?(z) \rangle(\langle u! \rangle\langle y_{1}!z \rangle C(x_{0},x_{1},y_{0},y_{1},u)) & \\
  & \wedge \langle y_{1}?(z) \rangle (\langle u! \rangle \langle x_{0}!z \rangle C(x_{0},x_{1},y_{0},y_{1},u)) & \\
  & \wedge \langle x_{1}?(z) \rangle (\langle u? \rangle \langle y_{0}!z \rangle C(x_{0},x_{1},y_{0},y_{1},u)) & \\
  & \wedge \langle y_{0}?(z) \rangle (\langle u? \rangle \langle x_{1}!z \rangle C(x_{0},x_{1},y_{0},y_{1},u))) &
\end{align*}

The lexicographical similarity between the shape of this formulae and
the shape of definition of the process representing a crossing reveals
the intuitive meaning of this formulae. It describes the capabilities
of a process that has the right to represent a crossing. For example
it picks out processes that may perform an input on the port $x_0$ in
its initial menu of capabilities. What differentiates the formula
from the process, however, is that the crossing process is the
smallest candidate to satisfy the formula. Infinitely many other
processes -- with internal behavior hidden behind this interface, so
to speak -- also satisfy this formula. Even this simple formula,
then, can be seen to open a new view onto knots, providing a
computational interpretation of \emph{virtual} knots.

Note that this formula is derived by hand. A similar formula can be
derived by employing Caires' calculation of characteristic formula
\cite{Caires04} to the process representing a crossing. In light of
this discussion, we let
$\meaningof{C}_{\phi}(x0,x1,y0,y1,u)$ denote a formula specifying the
dynamics we wish to capture of a crossing. To guarantee we preserve
the shape of the interface and minimal semantics we demand that
$\meaningof{C}_{\phi}(x0,x1,y0,y1,u) \Rightarrow
\textbf{C}(x0,x1,y0,y1,u)$ where $\textbf{C}(x0,x1,y0,y1,u)$ denotes
the formula above.
                            
\subsubsection{Crossing number constraints.}
The moral content of the context lemma (Lemma \ref{context}) is that the notion of
``locality'' in the Reidemeister moves is effectively captured by the
parallel composition operator of the process calculus. This intuition
extends through the logic. Given a formula,
$\meaningof{C}_{\phi}(x0,x1,y0,y1,u)$, we can use the structural
connectives to specify constraints on crossing numbers, such as at
least $n$ crossings, or exactly $n$ crossings.
\begin{mathpar}
  \inferrule* [lab=at-least-n] {} { K^{\geq n}_{\phi}(\vec{xs},\vec{ys}) := \Pi_{i=0}^{n-1} Hu . \meaningof{C}_{\phi}(xs_i,ys_i,u) | T }
  \and 
  \inferrule* [lab=exactly-n] {} { K^{= n}_{\phi}(\vec{xs},\vec{ys}) := \Pi_{i=0}^{n-1} Hu . \meaningof{C}_{\phi}(xs_i,ys_i,u) | \neg (\forall x_0,y_0,x_1,y_1,u . \meaningof{C}_{\phi}(x_0,y_0,x_1,y_1,u) | T) }
\end{mathpar}

To round out this section, recall that the encoding of an $n$-crossing
knot decomposes into a parallel composition of $n$ \emph{copies} of a
crossing process together with a wiring harness. To specify different
knot classes with the same crossing number amounts to specifying
logical constraints on the wiring harness. In the interest of space,
we defer examples to a forthcoming paper. Suffice it to say that both
the conditions ``alternating knot'' and ``contains the tangle
corresponding to 5/3'' are expressible. For example, it is possible to
calculate the characteristic formula of a process corresponding to the
tangle 5/3 and conjoin it into the classifying formula via the
composition connective of the logic.

Finally, we wish to observe that it is entirely within reason to
contemplate a more domain-specific version of spatial logic tailored
to the shape of processes in the image of the encoding. Such a
domain-specific logic would have a better claim to the title formal
language of knot properties.

% subsection example_formulae_ (end)

% section knots_as_processes (end) 

% section spatial logic via knots (end)

\section{Conclusions and future work}

\paragraph{Testing physical space}
You, gentle reader, may wonder why of all the theorems to be proved
given this set up we pick the one above. In some sense it's hardly
central to quantum mechanics. We see it as central in the sense that
it firmly establishes a notion of physical space arising from a notion
of the equivalence of behavior. Relating bisimulation to a metric is a
big step forward, but one is faced with interpreting the relationship
of that metric space to something more physical. Quantum mechanical
notions of ``physical'' space are still far from intuitive, but by
relating this idea of distance as testing to calculations that predict
physical circumstances we are making a not insignificant step forward
toward an understanding of the physical space we inhabit as
essentially dynamic.

\paragraph{Effectivity and simulation}
One of the observations we have yet to make is that the entire program
spelled out here is effective. We have built various interpreters for
the reflective calculus at work in this interpretation. In principle,
then, we can simulate quantum mechanics on a computer. The place where
the simulation may lose fidelity is the infinitely branching summation
for the annihilator.

In this connection i also want to point out that the evaluation style
calculation of the inner product puts the non-determinism of the
summation right at the heart of measurement. This suggests that
Milner's original reduction-based formulation of the dynamics of his
calculi in terms of sums was not just notationally suggestive of a
notion of measure-and-continue but captured some significant part of
the physics.

\paragraph{Quantum continuations}
In light of this last observation i want to point out that the
predominant account of quantum mechanics is missing a key aspect of a
truly compositional story of the physical situation. In a real lab,
when a measurement is made the observation can be made to feed into
another device that then makes another measurement conditioned on the
results of the first. This means that after the superposition was
collapsed the entire experimental set up remained in
superposition. While QM offers a means of writing this down it doesn't
quite line up well with the well-trodden formulation of computation
and continuation that we see so succinctly expressed in Milner's
calculi. This suggests that there might be advantages to this account
of dynamics waiting to be explored.

\paragraph{Quantum logic}
In this connection, we also note that by virtue of having the
Hennessy-Milner construction, we can pull the construction through the
interpretation of QM. This gives us a natural candidate for a quantum
logic that enjoys an extremely tight connection with it's domain of
interpretation, making the construction much less ad hoc (rather it is
the image of functor!).

\paragraph{Quantum probabiity}
i have questions about the basis of the interpretation of inner
product as probability amplitude. In particular, using which
axiomatization of probability theory does the notion of probability
amplitude earn the right to be so dubbed? In other words, where is the
proof that the operation for calculating a probability amplitude (and
then squaring) satisfies the axioms of what it means to calculate a
probability? Even if such a proof exists (i have yet to find it in the
literature), i wonder if it might not be possible to turn things on
their heads. Can we view the calculation of the probability amplitude
as an axiomatization of probability? If so, then the definition we
give for calculating probability amplitude may provide the basis for
an \emph{effective} theory of probability.

\paragraph{Quantum vs ``biological'' information}
Finally, i want to conclude with a more philosophical observation. At
a recent workshop in which QM was a predominant topic i noticed
something about quantum information. The speaker was giving a riveting
discussion of axiomatic QM and showing how properties of ``no
cloning'' and ``no deleting'' emerged as consequences of the
axiomatization. Theorems of this form are necessary to give us a sense
of confidence that our axioms characterize the physical theory. What
struck me, though, was that if quantum information is neither erasable
nor replicable it is markedly different from \emph{life}. Two of the
things we know about life is that

\begin{itemize}
  \item it ends;
  \item to gain some measure of persistence, to transcend it's
    finitude it is imminently copyable.
\end{itemize}

Both of these qualities are summarized succinctly in the aphorism: all
flesh is grass. For me these two kinds of ``information'' -- call them
quantum and biological -- are end points on a spectrum of strategies
for persistence. At one end, we have those curious entities that enjoy
uniqueness and permanence; at the other, we have those who in the face
of a certain end and an uncertain present make a go of passing
something on. To me one of the more remarkable aspects of the latter
strategy is that in the presence of noise (and certain features of
copying) we get a kind of dynamism, a chance for improvement against a
given persistent condition.

% subsection other_calculi_other_bisimulations_and_geometry_as_behavior (end)




% section conclusion (end)

%\documentclass[12pt]{llncs}
%\documentclass{jktr}

\usepackage[pdftex]{hyperref}                   
\usepackage {listings}
\usepackage {mathpartir}
\usepackage{bcprules}
%\usepackage{listings}
                       
\usepackage{graphicx} 
%\usepackage[margins=2.5cm,nohead,nofoot]{geometry}
%\usepackage{geometry}
\usepackage{amsfonts}
\usepackage{amstext}
\usepackage{latexsym}
\usepackage{amssymb}
\usepackage{color}


%\include{myPreamble}
\include{qm2pi.local} 

%\ifpdf
%\usepackage[pdftex]{graphicx}
%\else
%\usepackage{graphicx}
%\fi

 % \ifpdf
%  \usepackage{pdfsync}
%  \if


%\title{Brief Article}
%\author{David F. Snyder}
%\author{L.G. Meredith}

%\address{Dept. of Math., Texas State University--San Marcos, San Marcos, TX 78666}
       
\pagestyle{empty}


\begin{document}

\lstset{language=[Objective]Caml,frame=shadowbox}

\input{qm2pi.front}

% section front matter (end)

\input{qm2pi.intro} 
 
% section introduction (end)

% \input{qm2pi.knotations} 

% section notation (end)

\input{qm2pi.process.calculi} 

% section concurrent_process_calculi_and_spatial_logics_ (end)
    
%\input{qm2pi.knots2pi} 

%\input{qm2pi.trefoil} 

%\input{qm2pi.mainthm} 

% subsection basic_interpretation (end)

%\input{qm2pi.rho.presentation} 
\subsection{The syntax and semantics of the notation system}\label{sub:the_syntax_and_semantics_of_the_notation_system} % (fold)

We now summarize a technical presentation of the calculus that
embodies our theory of dynamics. The typical presentation of such a
calculus follows the style of giving generators and relations on
them. The grammar, below, describing term constructors, freely
generates the set of processes, $\Proc$. This set is then quotiented
by a relation known as structural congruence and it is over this set
that the notion of dynamics is expressed. This presentation is
essentially that of \cite{MeredithR05} with the addition of
polyadicity and summation. For readability we have relegated some of
the technical subtleties to an appendix.

\subsubsection{Process grammar}\label{subsub:process_grammar}

\begin{mathpar}
  \inferrule* [lab=synchronization] {} {{M} \bc \pzero \;|\; x?F \;|\; x!C }
  \and
  \inferrule* [lab=abstraction] {} {{F} \bc (x)P}
  \and
  \inferrule* [lab=concretion] {} {{C} \bc \langle Q \rangle}
  \and
  \inferrule* [lab=process] {} {{P,Q} \bc M \;| \;P|Q \;|\; @{x}}
  \and
  \inferrule* [lab=name] {} {{x} \bc \quotep{P}}
\end{mathpar} 

Note that $\vec{x}$ (resp. $\vec{P}$) denotes a vector of names
(resp. processes) of length $|\vec{x}|$ (resp. $|\vec{P}|$). We adopt
the following useful abbreviations.

\begin{mathpar}
   x?(\vec{y}).P := x.(\vec{y})P \and  x\clift{\vec{P}} := x.\clift{\vec{P}}
   \and x!(y) := \lift{x}{\dropn{y}}
   \and \Pi_{i=0}^{n-1}P_i := P_0 | \ldots | P_{n-1}
\end{mathpar}

\subsubsection{Structural congruence}

\paragraph{Free and bound names and alpha-equivalence.} At the
core of structural equivalence is alpha-equivalence which identifies
process that are the same up to a change of variable. Formally, we
recognize the distinction between free and bound names. The free names
of a process, $\freenames{P}$, may be calculated recursively as
follows:

\begin{mathpar}
\freenames{\pzero} := \emptyset
  \and \\
  \freenames{x?(y).P} := \{ x \} \cup (\freenames{P} \setminus \{ y \})
  \and 
  \freenames{x!\langle P \rangle} := \{ x \} \cup \{ P \} 
  \and \\
  \freenames{P|Q} := \freenames{P} \cup \freenames{Q}
  \and \\
  \freenames{@{x}} := \{ x \}
\end{mathpar}

$\pi$
$\quotep{\pi}$

$\freenames{-} : \pi \to \mathcal{P}(\quotep{\pi})$

\begin{eqnarray*}
  \freenames{\pzero} & := & \emptyset \\
  \freenames{x?(y).P} & := & \{ x \} \cup (\freenames{P} \setminus \{ y \}) \\
  \freenames{x!\langle P \rangle} & := & \{ x \} \cup \{ P \} \\
  \freenames{P|Q} & := & \freenames{P} \cup \freenames{Q} \\
  \freenames{\dropn{x}} & := & \{ x \}
\end{eqnarray*}

The bound names of a process, $\boundnames{P}$, are those names occurring in $P$
that are not free. For example, in $x?(y).0$, the name $x$ is free, while $y$ is bound.

\begin{mathpar}
  \inferrule* [lab=monoidal-laws] {} { P|Q \equiv Q|P \and P|0 \equiv P \and P|(Q|R) \equiv (P|Q)|R }
\end{mathpar}

\begin{mathpar}
  \inferrule* [lab=alpha-equivalence] {} { (x)P \equiv (y)P\{y/x\} \and y \not\in \freenames{P} }
\end{mathpar}

\begin{definition}
Then two processes, $P,Q$, are alpha-equivalent if $P = Q\{\vec{y}/\vec{x}\}$ for
some $\vec{x} \in \boundnames{Q},\vec{y} \in \boundnames{P}$, where $Q\{\vec{y}/\vec{x}\}$
denotes the capture-avoiding substitution of $\vec{y}$ for $\vec{x}$ in $Q$.
\end{definition}

\begin{definition}
  The {\em structural congruence} \cite{SangiorgiWalker} , $\equiv$,
  between processes is the least congruence containing
  alpha-equivalence, satisfying the abelian monoid laws
  (associativity, commutativity and $\pzero$ as identity) for parallel
  composition $|$ and for summation $+$.
\end{definition}

\subsection{Name equivalence}

We take name equivalence, written $\nameeq$, to be the smallest
equivalence relation generated by the following rules.

\begin{mathpar}
\inferrule*[lab=Quote-drop]
{ }
{ \quotep{@{x}} \nameeq x }

\inferrule*[lab=Struct-equiv]
{ P \scong Q }
{ \quotep{P} \nameeq \quotep{Q} }
\end{mathpar}

The astute reader will have noticed that the mutual recursion of names
and processes imposes a mutual recursion on alpha-equivalence and
structural equivalence via name-equivalence. Fortunately, all of this
works out pleasantly and we may calculate in the natural way, free of
concern. The reader interested in the details is referred to the
appendix \ref{appendix:rho_details}.

\subsection{Substitution}

We use $\Proc$ for the set of processes, $\QProc$ for the set of
names, and $\id{\{}\vec{y} / \vec{x} \id{\}}$ to denote partial maps,
$s : \QProc \rightarrow \QProc$. A map, $s$ lifts, uniquely, to a map
on process terms, $\widehat{s} : \Proc \rightarrow \Proc$ by the
following equations.

\begin{mathpar}
  (0) \psubstp{Q}{P} := 0 \\
  (R \juxtap S) \psubstp{Q}{P}
  :=    
  (R)\psubstp{Q}{P} \juxtap (S) \psubstp{Q}{P} \\
  (x?(y).R) \psubstp{Q}{P}    
  :=    
  (x)\substp{Q}{P} (z)\concat( (R \psubstn{z}{y}) \psubstp{Q}{P} ) \\
  (\lift{x}{R}) \psubstp{Q}{P}  
  :=
  \lift{(x)\substp{Q}{P}}{ R \psubstp{Q}{P} } \\
%   (\dropn{x})  \psubstp{Q}{P}       
%   := 
%   \left\{ 
%     \begin{array}{ccc} 
%       \dropn{\quotep{Q}} & & x \nameeq \quotep{P} \\
%       \dropn{x} & & otherwise \\
%     \end{array}
%   \right. 
  (\dropn{x})  \psubstp{Q}{P}       
  := 
  \left\{ 
    \begin{array}{ccc} 
      Q & & x \nameeq \quotep{P} \\
      \dropn{x} & & otherwise \\
    \end{array}
  \right.
\end{mathpar}
 

where

\begin{eqnarray}
  (x)\id{\{} \lpquote Q \rpquote / \lpquote P \rpquote \id{\}}            = 
  \left\{ 
    \begin{array}{ccc}
      \lpquote Q \rpquote & & x \nameeq \lpquote P \rpquote \\
      x & & otherwise \\
    \end{array}
  \right. \nonumber
\end{eqnarray}

and $z$ is chosen distinct from $\quotep{P}$, $\quotep{Q}$, the free
names in $Q$, and all the names in $R$. Our $\alpha$-equivalence will
be built in the standard way from this substitution.

\begin{remark}\label{rem:no_self_referential_names}
  One consequence of these definitions is that $\forall P. \quotep{P}
  \not\in \freenames{P}$.
\end{remark}

\subsection{ Dynamic quote: an example }

Anticipating something of what's to come, consider applying the
substitution, $\widehat{\id{\{}u / z \id{\}}}$, to the following pair
of processes, $\lift{w}{y!(z)}$ and $w[ \lpquote y!(z) \rpquote ]$.

\begin{eqnarray}
	\lift{w}{y!(z)}\widehat{\id{\{}u / z \id{\}}}
		& = &
		\lift{w}{y!(u)} \nonumber\\
	w[ \lpquote y!(z) \rpquote ] \widehat{ \id{\{}u / z \id{\}} }
		& = &
		w[ \lpquote y!(z) \rpquote ] \nonumber
\end{eqnarray}

Because the body of the process between quotes is impervious to
substitution, we get radically different answers. In fact, by
examining the first process in an input context,
e.g. $x?(z).\lift{w}{y!(z)}$, we see that the process under the lift
operator may be shaped by prefixed inputs binding a name inside it. In
this sense, the lift operator will be seen as a way to dynamically
construct processes before reifying them as names.

Finally equipped with these standard features we can present the
dynamics of the calculus.

\subsubsection{Operational semantics} 

Finally, we introduce the computational dynamics. What marks these
algebras as distinct from other more traditionally studied algebraic
structures, e.g. vector spaces or polynomial rings, is the manner in
which dynamics is captured. In traditional structures, dynamics is typically
expressed through morphisms between such structures, as in linear maps
between vector spaces or morphisms between rings. In algebras
associated with the semantics of computation, the dynamics is
expressed as part of the algebraic structure itself, through a
reduction reduction relation typically denoted by $\red$. Below, we
give a recursive presentation of this relation for the calculus used
in the encoding.

$\red \subseteq \pi \times \pi$
$\red : \pi \to \mathcal{P}(\pi)$

\begin{mathpar}
  \inferrule* [lab=Comm] { \textsf{match}( x_{src}, x_{trgt} ) } { x_{trgt}?(y)P \; | \; x_{src}!\langle {Q} \rangle \red P\{\quotep{Q}/y}\} }
  \and \\
  \inferrule* [lab=Par] {{P} \red {P}'} {{{P} | {Q}} \red {{P}' | {Q}}}
  \and
  \inferrule* [lab=Equiv]{{{P} \scong {P}'} \andalso {{P}' \red {Q}'} \andalso {{Q}' \scong {Q}}}{{P} \red {Q}}
\end{mathpar}

\begin{eqnarray*}
  match_{\equiv} (\quotep{P},\quotep{Q}) & := & P \equiv Q \\
  match_{\dagger}(\quotep{P},\quotep{Q}) & := & \forall R. P|Q \red^{*} R => R \red^{*} 0 \\
  match_{K}(\quotep{P},\quotep{Q}) & := & K \mbox{ for some context } K
\end{eqnarray*}

$u?(x)P | u!\langle Q \rangle \red P\{\quotep{Q}/x\}$

%We write $\wred$ for $\red^*$, and $P\red$ if $\exists Q $ such that $ P \red Q$.
We write $P\red$ if $\exists Q $ such that $ P \red Q$ and $P\not\red$, otherwise.

\section{Replication}

As mentioned before, it is known that replication (and hence
recursion) can be implemented in a higher-order process algebra
\cite{SangiorgiWalker}. As our first example of calculation with the
machinery thus far presented we give the construction explicitly in
the {\rhoc}.

\begin{eqnarray}
	D_{x} & := & \prefix{x}{y}{(\binpar{\outputp{x}{y}}{@{y}})} \nonumber\\
	\bangp_{x}{P} & := & \binpar{{x}!\langle{\binpar{D_{x}}{P}}\rangle}{D_{x}} \nonumber
\end{eqnarray}

\begin{eqnarray}
	\bangp_{x}{P} & & \nonumber\\
	=
	& {x}!\langle{(\prefix{x}{y}{(\outputp{x}{y} | @{y})) | P}}\rangle 
	      | \prefix{x}{y}{(\outputp{x}{y} | @{y})} & \nonumber\\
	\red
	& (\outputp{x}{y} | @{y})\substn{\quotep{(\prefix{x}{y}{(@{y} | \outputp{x}{y})) | P}}}{y} & \nonumber\\
	=
	& \outputp{x}{\quotep{(\prefix{x}{y}{(\outputp{x}{y} | @{y})) | P}}}
	  | {(\prefix{x}{y}{(\outputp{x}{y} | @{y})) | P}} & \nonumber\\
	\red
	& \ldots & \nonumber\\
	\red^*
	& P | P | \ldots & \nonumber
\end{eqnarray}

Of course, this encoding, as an implementation, runs away, unfolding
$\bangp{P}$ eagerly. A lazier and more implementable replication
operator, restricted to input-guarded processes, may be obtained as follows.

\begin{eqnarray}
\bangp{\prefix{u}{v}{P}} 
	:= 
	\binpar{\lift{x}{\prefix{u}{v}{(\binpar{D(x)}{P})}}}{D(x)} \nonumber
\end{eqnarray}

\begin{remark}
  Note that the lazier definition still does not deal with summation
  or mixed summation (i.e. sums over input and output). The reader is
  invited to construct definitions of replication that deal with these
  features. 

  Further, the definitions are parameterized in a name, $x$. Can you,
  gentle reader, make a definition that eliminates this parameter and
  guarantees no accidental interaction between the replication
  machinery and the process being replicated -- i.e. no accidental
  sharing of names used by the process to get its work done and the
  name(s) used by the replication to effect copying. This latter
  revision of the definition of replication is crucial to obtaining
  the expected identity $!!P \sim !P$.
\end{remark}

\begin{remark}\label{rem:paradoxical_combinator}
  The reader familiar with the lambda calculus will have noticed the
  similarity between $D$ and the paradoxical combinator.

  [Ed. note: the existence of this seems to suggest we have to be more
  restrictive on the set of processes and names we admit if we are to
  support no-cloning.]
\end{remark}

\subsubsection{Bisimulation}

The computational dynamics gives rise to another kind of equivalence,
the equivalence of computational behavior. As previously mentioned
this is typically captured \emph{via} some form of bisimulation.

% The notion we use in this paper is weak barbed bisimulation
% \cite{milner91polyadicpi}.

The notion we use in this paper is derived from weak barbed
bisimulation \cite{milner91polyadicpi}. 

\begin{definition}
An \emph{observation relation}, $\downarrow_{\mathcal N}$, over a set
of names, $\mathcal N$, is the smallest relation satisfying the rules
below.

\infrule[Out-barb]{y \in {\mathcal N}, \; x \nameeq y}
		  {\outputp{x}{v} \downarrow_{\mathcal N} x}
\infrule[Par-barb]{\mbox{$P\downarrow_{\mathcal N} x$ or $Q\downarrow_{\mathcal N} x$}}
		  {\binpar{P}{Q} \downarrow_{\mathcal N} x}

We write $P \Downarrow_{\mathcal N} x$ if there is $Q$ such that 
$P \wred Q$ and $Q \downarrow_{\mathcal N} x$.
\end{definition}

\begin{definition}
%\label{def.bbisim}
An  ${\mathcal N}$-\emph{barbed bisimulation} over a set of names, ${\mathcal N}$, is a symmetric binary relation 
${\mathcal S}_{\mathcal N}$ between agents such that $P\rel{S}_{\mathcal N}Q$ implies:
\begin{enumerate}
\item If $P \red P'$ then $Q \wred Q'$ and $P'\rel{S}_{\mathcal N} Q'$.
\item If $P\downarrow_{\mathcal N} x$, then $Q\Downarrow_{\mathcal N} x$.
\end{enumerate}
$P$ is ${\mathcal N}$-barbed bisimilar to $Q$, written
$P \wbbisim_{\mathcal N} Q$, if $P \rel{S}_{\mathcal N} Q$ for some ${\mathcal N}$-barbed bisimulation ${\mathcal S}_{\mathcal N}$.
\end{definition}

$\mathcal{R} \subseteq \pi \times \pi$

$P \mathcal{R} Q => \forall P'. P \red P' \Rightarrow \exists Q'. Q \red Q', P' \mathcal{R} Q'$

$P \vdash x \Rightarrow Q \vdash x$

\begin{mathpar}
  \inferrule*[lab=Out-barb]{x \nameeq y}{{y}!\langle{Q}\rangle \vdash x}
  \and
  \inferrule*[lab=Par-barb]{\mbox{$P\vdash x$ or $Q\vdash x$}}{\binpar{P}{Q} \vdash x}
\end{mathpar}

\subsubsection{Contexts}

One of the principle advantages of computational calculi like the
$\pi$-calculus is a well-defined notion of context,
contextual-equivalence and a correlation between
contextual-equivalence and notions of bisimulation. The notion of
context allows the decomposition of a process into (sub-)process and
its syntactic environment, its context. Thus, a context may be
thought of as a process with a ``hole'' (written $\Box$) in it. The
application of a context $M$ to a process $P$, written $M[P]$, is
tantamount to filling the hole in $M$ with $P$. In this paper we do
not need the full weight of this theory, but do make use of the notion
of context in the proof the main theorem. 

\begin{mathpar}
  \inferrule* [lab=summation] {} {{M_{M},M_{N}} \bc \Box \;|\; x.M_{A} \;|\; M_{M}+M_{N}}
  \and
  \inferrule* [lab=agent] {} {{M_{A}} \bc (\vec{x})M_{P} \;| \; \clift{P_0,\ldots,M_{P},\ldots,P_N}}
  \and \\
  \inferrule* [lab=process] {} {{M_{P}} \bc M_{N} \;| \;P|M_{P} }
\end{mathpar} 

\begin{mathpar}
  \inferrule* [lab=sychronization] {} {M_{N} \bc \Box \;|\; x?M_{F} \;|\; x!M_{C}}
  \and
  \inferrule* [lab=abstraction] {} {{M_{F}} \bc (x)M_{P} }
  \and
  \inferrule* [lab=concretion] {} {{M_{C}} \bc \langle M_{P} \rangle }
  \and \\
  \inferrule* [lab=process] {} {{M_{P}} \bc M_{N} \;| \;P|M_{P} }
\end{mathpar}

\begin{definition}[contextual application] Given a context $M$, and
  process $P$, we define the \emph{contextual application}, $M[P] :=
  M\{P/\Box\}$. That is, the contextual application of M to P is the
  substitution of $P$ for $\Box$ in $M$.
\end{definition}

$\meaningof{-} : L \to \mathcal{P}(\pi)$

\begin{mathpar}
  \inferrule* [lab=collection] {} {\meaningof{true} = \pi, \and \meaningof{~E} = \pi \setminus \meaningof{E}, \and \meaningof{E_{1} \& E_{2}} = \meaningof{E_{1}} \cap \meaningof{E_{2}}}
\end{mathpar}

\begin{mathpar}
  \inferrule* [lab=structure] {} {\meaningof{0} = \{ P \in \pi | P \equiv 0 \}, \and \\ \meaningof{E_1 | E_2} = \{ P \in \pi | P \equiv P_{1} | P_{2}, P_{1} \in \meaningof{E_{1}}, P_{2} \in \meaningof{E_2}\} }
\end{mathpar}

\begin{mathpar}
 \inferrule* [lab=behavior] {} {\meaningof{\langle a?b \rangle E} = \{ P \in \pi | P \equiv Q | u?(y)P', \\ \and \\\\ \and \\ \;\;\; u \in \meaningof{a}, \forall z.P'\{z/y\} \in \meaningof{E\{z/b\}}\}, \and \\ \meaningof{a!E} = \{ P \in \pi | P \equiv Q | x!\langle P' \rangle, x \in \meaningof{a} P' \in \meaningof{E}\} }
\end{mathpar}

\begin{mathpar}
 \inferrule* [lab=nominal] {} {\meaningof{\quotep{E}} = \{ \quotep{P} \in \quotep{\pi} | P \in \meaningof{E} \}, \and \meaningof{\quotep{P}} = \{ \quotep{Q} \in \quotep{\pi} | P \equiv Q \} \and \\ \meaningof{@\quotep{E}} = \{ P \in \pi | P \equiv @x, x \in \meaningof{E} \}}
\end{mathpar}

\begin{eqnarray*}
  \\
  \meaningof{-} : TS \to ST
\end{eqnarray*}

\begin{eqnarray*}
  \\
  L : TS \to ST
\end{eqnarray*}

\begin{eqnarray*}
  \\
  P \models E \iff P \in \meaningof{E}
\end{eqnarray*}

\begin{eqnarray*}
  P \approx_{L} Q \iff \forall E \in L. P \models E \iff Q \models E
\end{eqnarray*}

\begin{eqnarray*}
  P \approx_{K} Q
\end{eqnarray*}

\begin{eqnarray*}
  P \approx Q
\end{eqnarray*}

$\approx_{K} = \approx = \approx_{L}$

\subsubsection{Contextual duality}

Note that contexts extend the quotation operation to a family of
operations from processes to names. Given a context, $M$, we can
define a \emph{nominal context}, $\quotep{M}$ by $\quotep{M}[P] :=
\quotep{M[P]}$. To foreshadow what is to come we observe that these
operations enjoy a duality with processes very much like the duality
between vectors and maps from vectors to scalars.

Further, because the calculus is essentially higher-order, we have a
correspondence between contexts and processes. More specifically,
given a name $x$ and a context $M$ we can construct $M^{*}_{x}$ such
that 

\begin{mathpar}
  M^{*}_{x} | \lift{x}{P} \red M[P]
\end{mathpar}

namely,

\begin{mathpar}
  M^{*}_{x} := x?(u).M[\dropn{u}]
\end{mathpar}

The dependence of $M^{*}_{x}$ on a name makes it an abstraction, 

\begin{mathpar}
  M^{*} := (x)x?(u).M[\dropn{u}]
\end{mathpar}

\subsection{Additional notation}

It will sometimes be convenient to denote the process a name
quotes. We already have the notation $x = \quotep{P}$, but it will be
convenient to introduce an alternate notation, $\procn{x}$, when we
want to emphasize the connection to the use of the name. Note that, by
virtue of name equivalence, $\quotep{\procn{x}} \nameeq x$; so, the
notation is consistent with previous definitions.

Further, because names have structure it is possible to effect
substitutions on the basis of that structure. This means we need to
upgrade our notation for substitutions, which we accomplish by
adapting comprehension notation. Thus,

\begin{mathpar}
  P\{ y / x : x \in S \}
\end{mathpar}

is interpreted to mean the process derived from P by replacing (in a
capture-avoiding manner) each occurrence of $x$ in $S$ by $y$. For example,

\begin{mathpar}
  P\{ \quotep{\procn{x}|\procn{x}} / x : x \in \freenames{P} \}
\end{mathpar}

will replace each (occurrence) of a free name $x$ in $P$ by
$\quotep{\procn{x}|\procn{x}}$.

Also, we will avail ourselves of the notation $x^{L}$ and $x^{R}$ to
denote injections of a name into disjoint copies of the name
space. There are numerous ways to accomplish this. One example can be
found in \cite{MeredithR05}. This notation overloads to vectors of
names: $\vec{x}^{\pi} := (x_{i}^{\pi} \; : \; 0 \leq i < |\vec{x}| )$ where $\pi \in \{L,R\}$.

We also use $P^{\Box} := P|\Box$.

In \cite{MeredithR05} an interpretation of the new operator is
given. It turns out that there are several possible interpretations
all enjoying the requisite algebraic properties of the operator (see
\cite{milner91polyadicpi}). We will therefore make liberal use of
$(\nu\; \vec{x})P$.

% subsection the_syntax_and_semantics_of_the_notation_system (end)   

\input{qm2pi.qmops} 

\input{qm2pi.sterngerlach} 

\input{qm2pi.metric} 

% section concurrent_process_calculi (end)

%\input{qm2pi.proofsketch}

% section proof sketch (end)

%\input{qm2pi.slviaknots} 

% section spatial logic via knots (end)

\input{qm2pi.conclusion}

% section conclusion (end)

%\input{qm2pi.dtcodes} 

% section wiring algorithm (end)

\input{qm2pi.ack} 

% section acknowledgments (end)

\newpage


\bibliographystyle{plain}   
\bibliography{../../biblios/main.bib}

\input{qm2pi.rhodetails}

\end{document}

 

% section wiring algorithm (end)

\documentclass[12pt]{llncs}
%\documentclass{jktr}

\usepackage[pdftex]{hyperref}                   
\usepackage {listings}
\usepackage {mathpartir}
\usepackage{bcprules}
%\usepackage{listings}
                       
\usepackage{graphicx} 
%\usepackage[margins=2.5cm,nohead,nofoot]{geometry}
%\usepackage{geometry}
\usepackage{amsfonts}
\usepackage{amstext}
\usepackage{latexsym}
\usepackage{amssymb}
\usepackage{color}


%\include{myPreamble}
\include{qm2pi.local} 

%\ifpdf
%\usepackage[pdftex]{graphicx}
%\else
%\usepackage{graphicx}
%\fi

 % \ifpdf
%  \usepackage{pdfsync}
%  \if


%\title{Brief Article}
%\author{David F. Snyder}
%\author{L.G. Meredith}

%\address{Dept. of Math., Texas State University--San Marcos, San Marcos, TX 78666}
       
\pagestyle{empty}


\begin{document}

\lstset{language=[Objective]Caml,frame=shadowbox}

\input{qm2pi.front}

% section front matter (end)

\input{qm2pi.intro} 
 
% section introduction (end)

% \input{qm2pi.knotations} 

% section notation (end)

\input{qm2pi.process.calculi} 

% section concurrent_process_calculi_and_spatial_logics_ (end)
    
%\input{qm2pi.knots2pi} 

%\input{qm2pi.trefoil} 

%\input{qm2pi.mainthm} 

% subsection basic_interpretation (end)

%\input{qm2pi.rho.presentation} 
\subsection{The syntax and semantics of the notation system}\label{sub:the_syntax_and_semantics_of_the_notation_system} % (fold)

We now summarize a technical presentation of the calculus that
embodies our theory of dynamics. The typical presentation of such a
calculus follows the style of giving generators and relations on
them. The grammar, below, describing term constructors, freely
generates the set of processes, $\Proc$. This set is then quotiented
by a relation known as structural congruence and it is over this set
that the notion of dynamics is expressed. This presentation is
essentially that of \cite{MeredithR05} with the addition of
polyadicity and summation. For readability we have relegated some of
the technical subtleties to an appendix.

\subsubsection{Process grammar}\label{subsub:process_grammar}

\begin{mathpar}
  \inferrule* [lab=synchronization] {} {{M} \bc \pzero \;|\; x?F \;|\; x!C }
  \and
  \inferrule* [lab=abstraction] {} {{F} \bc (x)P}
  \and
  \inferrule* [lab=concretion] {} {{C} \bc \langle Q \rangle}
  \and
  \inferrule* [lab=process] {} {{P,Q} \bc M \;| \;P|Q \;|\; @{x}}
  \and
  \inferrule* [lab=name] {} {{x} \bc \quotep{P}}
\end{mathpar} 

Note that $\vec{x}$ (resp. $\vec{P}$) denotes a vector of names
(resp. processes) of length $|\vec{x}|$ (resp. $|\vec{P}|$). We adopt
the following useful abbreviations.

\begin{mathpar}
   x?(\vec{y}).P := x.(\vec{y})P \and  x\clift{\vec{P}} := x.\clift{\vec{P}}
   \and x!(y) := \lift{x}{\dropn{y}}
   \and \Pi_{i=0}^{n-1}P_i := P_0 | \ldots | P_{n-1}
\end{mathpar}

\subsubsection{Structural congruence}

\paragraph{Free and bound names and alpha-equivalence.} At the
core of structural equivalence is alpha-equivalence which identifies
process that are the same up to a change of variable. Formally, we
recognize the distinction between free and bound names. The free names
of a process, $\freenames{P}$, may be calculated recursively as
follows:

\begin{mathpar}
\freenames{\pzero} := \emptyset
  \and \\
  \freenames{x?(y).P} := \{ x \} \cup (\freenames{P} \setminus \{ y \})
  \and 
  \freenames{x!\langle P \rangle} := \{ x \} \cup \{ P \} 
  \and \\
  \freenames{P|Q} := \freenames{P} \cup \freenames{Q}
  \and \\
  \freenames{@{x}} := \{ x \}
\end{mathpar}

$\pi$
$\quotep{\pi}$

$\freenames{-} : \pi \to \mathcal{P}(\quotep{\pi})$

\begin{eqnarray*}
  \freenames{\pzero} & := & \emptyset \\
  \freenames{x?(y).P} & := & \{ x \} \cup (\freenames{P} \setminus \{ y \}) \\
  \freenames{x!\langle P \rangle} & := & \{ x \} \cup \{ P \} \\
  \freenames{P|Q} & := & \freenames{P} \cup \freenames{Q} \\
  \freenames{\dropn{x}} & := & \{ x \}
\end{eqnarray*}

The bound names of a process, $\boundnames{P}$, are those names occurring in $P$
that are not free. For example, in $x?(y).0$, the name $x$ is free, while $y$ is bound.

\begin{mathpar}
  \inferrule* [lab=monoidal-laws] {} { P|Q \equiv Q|P \and P|0 \equiv P \and P|(Q|R) \equiv (P|Q)|R }
\end{mathpar}

\begin{mathpar}
  \inferrule* [lab=alpha-equivalence] {} { (x)P \equiv (y)P\{y/x\} \and y \not\in \freenames{P} }
\end{mathpar}

\begin{definition}
Then two processes, $P,Q$, are alpha-equivalent if $P = Q\{\vec{y}/\vec{x}\}$ for
some $\vec{x} \in \boundnames{Q},\vec{y} \in \boundnames{P}$, where $Q\{\vec{y}/\vec{x}\}$
denotes the capture-avoiding substitution of $\vec{y}$ for $\vec{x}$ in $Q$.
\end{definition}

\begin{definition}
  The {\em structural congruence} \cite{SangiorgiWalker} , $\equiv$,
  between processes is the least congruence containing
  alpha-equivalence, satisfying the abelian monoid laws
  (associativity, commutativity and $\pzero$ as identity) for parallel
  composition $|$ and for summation $+$.
\end{definition}

\subsection{Name equivalence}

We take name equivalence, written $\nameeq$, to be the smallest
equivalence relation generated by the following rules.

\begin{mathpar}
\inferrule*[lab=Quote-drop]
{ }
{ \quotep{@{x}} \nameeq x }

\inferrule*[lab=Struct-equiv]
{ P \scong Q }
{ \quotep{P} \nameeq \quotep{Q} }
\end{mathpar}

The astute reader will have noticed that the mutual recursion of names
and processes imposes a mutual recursion on alpha-equivalence and
structural equivalence via name-equivalence. Fortunately, all of this
works out pleasantly and we may calculate in the natural way, free of
concern. The reader interested in the details is referred to the
appendix \ref{appendix:rho_details}.

\subsection{Substitution}

We use $\Proc$ for the set of processes, $\QProc$ for the set of
names, and $\id{\{}\vec{y} / \vec{x} \id{\}}$ to denote partial maps,
$s : \QProc \rightarrow \QProc$. A map, $s$ lifts, uniquely, to a map
on process terms, $\widehat{s} : \Proc \rightarrow \Proc$ by the
following equations.

\begin{mathpar}
  (0) \psubstp{Q}{P} := 0 \\
  (R \juxtap S) \psubstp{Q}{P}
  :=    
  (R)\psubstp{Q}{P} \juxtap (S) \psubstp{Q}{P} \\
  (x?(y).R) \psubstp{Q}{P}    
  :=    
  (x)\substp{Q}{P} (z)\concat( (R \psubstn{z}{y}) \psubstp{Q}{P} ) \\
  (\lift{x}{R}) \psubstp{Q}{P}  
  :=
  \lift{(x)\substp{Q}{P}}{ R \psubstp{Q}{P} } \\
%   (\dropn{x})  \psubstp{Q}{P}       
%   := 
%   \left\{ 
%     \begin{array}{ccc} 
%       \dropn{\quotep{Q}} & & x \nameeq \quotep{P} \\
%       \dropn{x} & & otherwise \\
%     \end{array}
%   \right. 
  (\dropn{x})  \psubstp{Q}{P}       
  := 
  \left\{ 
    \begin{array}{ccc} 
      Q & & x \nameeq \quotep{P} \\
      \dropn{x} & & otherwise \\
    \end{array}
  \right.
\end{mathpar}
 

where

\begin{eqnarray}
  (x)\id{\{} \lpquote Q \rpquote / \lpquote P \rpquote \id{\}}            = 
  \left\{ 
    \begin{array}{ccc}
      \lpquote Q \rpquote & & x \nameeq \lpquote P \rpquote \\
      x & & otherwise \\
    \end{array}
  \right. \nonumber
\end{eqnarray}

and $z$ is chosen distinct from $\quotep{P}$, $\quotep{Q}$, the free
names in $Q$, and all the names in $R$. Our $\alpha$-equivalence will
be built in the standard way from this substitution.

\begin{remark}\label{rem:no_self_referential_names}
  One consequence of these definitions is that $\forall P. \quotep{P}
  \not\in \freenames{P}$.
\end{remark}

\subsection{ Dynamic quote: an example }

Anticipating something of what's to come, consider applying the
substitution, $\widehat{\id{\{}u / z \id{\}}}$, to the following pair
of processes, $\lift{w}{y!(z)}$ and $w[ \lpquote y!(z) \rpquote ]$.

\begin{eqnarray}
	\lift{w}{y!(z)}\widehat{\id{\{}u / z \id{\}}}
		& = &
		\lift{w}{y!(u)} \nonumber\\
	w[ \lpquote y!(z) \rpquote ] \widehat{ \id{\{}u / z \id{\}} }
		& = &
		w[ \lpquote y!(z) \rpquote ] \nonumber
\end{eqnarray}

Because the body of the process between quotes is impervious to
substitution, we get radically different answers. In fact, by
examining the first process in an input context,
e.g. $x?(z).\lift{w}{y!(z)}$, we see that the process under the lift
operator may be shaped by prefixed inputs binding a name inside it. In
this sense, the lift operator will be seen as a way to dynamically
construct processes before reifying them as names.

Finally equipped with these standard features we can present the
dynamics of the calculus.

\subsubsection{Operational semantics} 

Finally, we introduce the computational dynamics. What marks these
algebras as distinct from other more traditionally studied algebraic
structures, e.g. vector spaces or polynomial rings, is the manner in
which dynamics is captured. In traditional structures, dynamics is typically
expressed through morphisms between such structures, as in linear maps
between vector spaces or morphisms between rings. In algebras
associated with the semantics of computation, the dynamics is
expressed as part of the algebraic structure itself, through a
reduction reduction relation typically denoted by $\red$. Below, we
give a recursive presentation of this relation for the calculus used
in the encoding.

$\red \subseteq \pi \times \pi$
$\red : \pi \to \mathcal{P}(\pi)$

\begin{mathpar}
  \inferrule* [lab=Comm] { \textsf{match}( x_{src}, x_{trgt} ) } { x_{trgt}?(y)P \; | \; x_{src}!\langle {Q} \rangle \red P\{\quotep{Q}/y}\} }
  \and \\
  \inferrule* [lab=Par] {{P} \red {P}'} {{{P} | {Q}} \red {{P}' | {Q}}}
  \and
  \inferrule* [lab=Equiv]{{{P} \scong {P}'} \andalso {{P}' \red {Q}'} \andalso {{Q}' \scong {Q}}}{{P} \red {Q}}
\end{mathpar}

\begin{eqnarray*}
  match_{\equiv} (\quotep{P},\quotep{Q}) & := & P \equiv Q \\
  match_{\dagger}(\quotep{P},\quotep{Q}) & := & \forall R. P|Q \red^{*} R => R \red^{*} 0 \\
  match_{K}(\quotep{P},\quotep{Q}) & := & K \mbox{ for some context } K
\end{eqnarray*}

$u?(x)P | u!\langle Q \rangle \red P\{\quotep{Q}/x\}$

%We write $\wred$ for $\red^*$, and $P\red$ if $\exists Q $ such that $ P \red Q$.
We write $P\red$ if $\exists Q $ such that $ P \red Q$ and $P\not\red$, otherwise.

\section{Replication}

As mentioned before, it is known that replication (and hence
recursion) can be implemented in a higher-order process algebra
\cite{SangiorgiWalker}. As our first example of calculation with the
machinery thus far presented we give the construction explicitly in
the {\rhoc}.

\begin{eqnarray}
	D_{x} & := & \prefix{x}{y}{(\binpar{\outputp{x}{y}}{@{y}})} \nonumber\\
	\bangp_{x}{P} & := & \binpar{{x}!\langle{\binpar{D_{x}}{P}}\rangle}{D_{x}} \nonumber
\end{eqnarray}

\begin{eqnarray}
	\bangp_{x}{P} & & \nonumber\\
	=
	& {x}!\langle{(\prefix{x}{y}{(\outputp{x}{y} | @{y})) | P}}\rangle 
	      | \prefix{x}{y}{(\outputp{x}{y} | @{y})} & \nonumber\\
	\red
	& (\outputp{x}{y} | @{y})\substn{\quotep{(\prefix{x}{y}{(@{y} | \outputp{x}{y})) | P}}}{y} & \nonumber\\
	=
	& \outputp{x}{\quotep{(\prefix{x}{y}{(\outputp{x}{y} | @{y})) | P}}}
	  | {(\prefix{x}{y}{(\outputp{x}{y} | @{y})) | P}} & \nonumber\\
	\red
	& \ldots & \nonumber\\
	\red^*
	& P | P | \ldots & \nonumber
\end{eqnarray}

Of course, this encoding, as an implementation, runs away, unfolding
$\bangp{P}$ eagerly. A lazier and more implementable replication
operator, restricted to input-guarded processes, may be obtained as follows.

\begin{eqnarray}
\bangp{\prefix{u}{v}{P}} 
	:= 
	\binpar{\lift{x}{\prefix{u}{v}{(\binpar{D(x)}{P})}}}{D(x)} \nonumber
\end{eqnarray}

\begin{remark}
  Note that the lazier definition still does not deal with summation
  or mixed summation (i.e. sums over input and output). The reader is
  invited to construct definitions of replication that deal with these
  features. 

  Further, the definitions are parameterized in a name, $x$. Can you,
  gentle reader, make a definition that eliminates this parameter and
  guarantees no accidental interaction between the replication
  machinery and the process being replicated -- i.e. no accidental
  sharing of names used by the process to get its work done and the
  name(s) used by the replication to effect copying. This latter
  revision of the definition of replication is crucial to obtaining
  the expected identity $!!P \sim !P$.
\end{remark}

\begin{remark}\label{rem:paradoxical_combinator}
  The reader familiar with the lambda calculus will have noticed the
  similarity between $D$ and the paradoxical combinator.

  [Ed. note: the existence of this seems to suggest we have to be more
  restrictive on the set of processes and names we admit if we are to
  support no-cloning.]
\end{remark}

\subsubsection{Bisimulation}

The computational dynamics gives rise to another kind of equivalence,
the equivalence of computational behavior. As previously mentioned
this is typically captured \emph{via} some form of bisimulation.

% The notion we use in this paper is weak barbed bisimulation
% \cite{milner91polyadicpi}.

The notion we use in this paper is derived from weak barbed
bisimulation \cite{milner91polyadicpi}. 

\begin{definition}
An \emph{observation relation}, $\downarrow_{\mathcal N}$, over a set
of names, $\mathcal N$, is the smallest relation satisfying the rules
below.

\infrule[Out-barb]{y \in {\mathcal N}, \; x \nameeq y}
		  {\outputp{x}{v} \downarrow_{\mathcal N} x}
\infrule[Par-barb]{\mbox{$P\downarrow_{\mathcal N} x$ or $Q\downarrow_{\mathcal N} x$}}
		  {\binpar{P}{Q} \downarrow_{\mathcal N} x}

We write $P \Downarrow_{\mathcal N} x$ if there is $Q$ such that 
$P \wred Q$ and $Q \downarrow_{\mathcal N} x$.
\end{definition}

\begin{definition}
%\label{def.bbisim}
An  ${\mathcal N}$-\emph{barbed bisimulation} over a set of names, ${\mathcal N}$, is a symmetric binary relation 
${\mathcal S}_{\mathcal N}$ between agents such that $P\rel{S}_{\mathcal N}Q$ implies:
\begin{enumerate}
\item If $P \red P'$ then $Q \wred Q'$ and $P'\rel{S}_{\mathcal N} Q'$.
\item If $P\downarrow_{\mathcal N} x$, then $Q\Downarrow_{\mathcal N} x$.
\end{enumerate}
$P$ is ${\mathcal N}$-barbed bisimilar to $Q$, written
$P \wbbisim_{\mathcal N} Q$, if $P \rel{S}_{\mathcal N} Q$ for some ${\mathcal N}$-barbed bisimulation ${\mathcal S}_{\mathcal N}$.
\end{definition}

$\mathcal{R} \subseteq \pi \times \pi$

$P \mathcal{R} Q => \forall P'. P \red P' \Rightarrow \exists Q'. Q \red Q', P' \mathcal{R} Q'$

$P \vdash x \Rightarrow Q \vdash x$

\begin{mathpar}
  \inferrule*[lab=Out-barb]{x \nameeq y}{{y}!\langle{Q}\rangle \vdash x}
  \and
  \inferrule*[lab=Par-barb]{\mbox{$P\vdash x$ or $Q\vdash x$}}{\binpar{P}{Q} \vdash x}
\end{mathpar}

\subsubsection{Contexts}

One of the principle advantages of computational calculi like the
$\pi$-calculus is a well-defined notion of context,
contextual-equivalence and a correlation between
contextual-equivalence and notions of bisimulation. The notion of
context allows the decomposition of a process into (sub-)process and
its syntactic environment, its context. Thus, a context may be
thought of as a process with a ``hole'' (written $\Box$) in it. The
application of a context $M$ to a process $P$, written $M[P]$, is
tantamount to filling the hole in $M$ with $P$. In this paper we do
not need the full weight of this theory, but do make use of the notion
of context in the proof the main theorem. 

\begin{mathpar}
  \inferrule* [lab=summation] {} {{M_{M},M_{N}} \bc \Box \;|\; x.M_{A} \;|\; M_{M}+M_{N}}
  \and
  \inferrule* [lab=agent] {} {{M_{A}} \bc (\vec{x})M_{P} \;| \; \clift{P_0,\ldots,M_{P},\ldots,P_N}}
  \and \\
  \inferrule* [lab=process] {} {{M_{P}} \bc M_{N} \;| \;P|M_{P} }
\end{mathpar} 

\begin{mathpar}
  \inferrule* [lab=sychronization] {} {M_{N} \bc \Box \;|\; x?M_{F} \;|\; x!M_{C}}
  \and
  \inferrule* [lab=abstraction] {} {{M_{F}} \bc (x)M_{P} }
  \and
  \inferrule* [lab=concretion] {} {{M_{C}} \bc \langle M_{P} \rangle }
  \and \\
  \inferrule* [lab=process] {} {{M_{P}} \bc M_{N} \;| \;P|M_{P} }
\end{mathpar}

\begin{definition}[contextual application] Given a context $M$, and
  process $P$, we define the \emph{contextual application}, $M[P] :=
  M\{P/\Box\}$. That is, the contextual application of M to P is the
  substitution of $P$ for $\Box$ in $M$.
\end{definition}

$\meaningof{-} : L \to \mathcal{P}(\pi)$

\begin{mathpar}
  \inferrule* [lab=collection] {} {\meaningof{true} = \pi, \and \meaningof{~E} = \pi \setminus \meaningof{E}, \and \meaningof{E_{1} \& E_{2}} = \meaningof{E_{1}} \cap \meaningof{E_{2}}}
\end{mathpar}

\begin{mathpar}
  \inferrule* [lab=structure] {} {\meaningof{0} = \{ P \in \pi | P \equiv 0 \}, \and \\ \meaningof{E_1 | E_2} = \{ P \in \pi | P \equiv P_{1} | P_{2}, P_{1} \in \meaningof{E_{1}}, P_{2} \in \meaningof{E_2}\} }
\end{mathpar}

\begin{mathpar}
 \inferrule* [lab=behavior] {} {\meaningof{\langle a?b \rangle E} = \{ P \in \pi | P \equiv Q | u?(y)P', \\ \and \\\\ \and \\ \;\;\; u \in \meaningof{a}, \forall z.P'\{z/y\} \in \meaningof{E\{z/b\}}\}, \and \\ \meaningof{a!E} = \{ P \in \pi | P \equiv Q | x!\langle P' \rangle, x \in \meaningof{a} P' \in \meaningof{E}\} }
\end{mathpar}

\begin{mathpar}
 \inferrule* [lab=nominal] {} {\meaningof{\quotep{E}} = \{ \quotep{P} \in \quotep{\pi} | P \in \meaningof{E} \}, \and \meaningof{\quotep{P}} = \{ \quotep{Q} \in \quotep{\pi} | P \equiv Q \} \and \\ \meaningof{@\quotep{E}} = \{ P \in \pi | P \equiv @x, x \in \meaningof{E} \}}
\end{mathpar}

\begin{eqnarray*}
  \\
  \meaningof{-} : TS \to ST
\end{eqnarray*}

\begin{eqnarray*}
  \\
  L : TS \to ST
\end{eqnarray*}

\begin{eqnarray*}
  \\
  P \models E \iff P \in \meaningof{E}
\end{eqnarray*}

\begin{eqnarray*}
  P \approx_{L} Q \iff \forall E \in L. P \models E \iff Q \models E
\end{eqnarray*}

\begin{eqnarray*}
  P \approx_{K} Q
\end{eqnarray*}

\begin{eqnarray*}
  P \approx Q
\end{eqnarray*}

$\approx_{K} = \approx = \approx_{L}$

\subsubsection{Contextual duality}

Note that contexts extend the quotation operation to a family of
operations from processes to names. Given a context, $M$, we can
define a \emph{nominal context}, $\quotep{M}$ by $\quotep{M}[P] :=
\quotep{M[P]}$. To foreshadow what is to come we observe that these
operations enjoy a duality with processes very much like the duality
between vectors and maps from vectors to scalars.

Further, because the calculus is essentially higher-order, we have a
correspondence between contexts and processes. More specifically,
given a name $x$ and a context $M$ we can construct $M^{*}_{x}$ such
that 

\begin{mathpar}
  M^{*}_{x} | \lift{x}{P} \red M[P]
\end{mathpar}

namely,

\begin{mathpar}
  M^{*}_{x} := x?(u).M[\dropn{u}]
\end{mathpar}

The dependence of $M^{*}_{x}$ on a name makes it an abstraction, 

\begin{mathpar}
  M^{*} := (x)x?(u).M[\dropn{u}]
\end{mathpar}

\subsection{Additional notation}

It will sometimes be convenient to denote the process a name
quotes. We already have the notation $x = \quotep{P}$, but it will be
convenient to introduce an alternate notation, $\procn{x}$, when we
want to emphasize the connection to the use of the name. Note that, by
virtue of name equivalence, $\quotep{\procn{x}} \nameeq x$; so, the
notation is consistent with previous definitions.

Further, because names have structure it is possible to effect
substitutions on the basis of that structure. This means we need to
upgrade our notation for substitutions, which we accomplish by
adapting comprehension notation. Thus,

\begin{mathpar}
  P\{ y / x : x \in S \}
\end{mathpar}

is interpreted to mean the process derived from P by replacing (in a
capture-avoiding manner) each occurrence of $x$ in $S$ by $y$. For example,

\begin{mathpar}
  P\{ \quotep{\procn{x}|\procn{x}} / x : x \in \freenames{P} \}
\end{mathpar}

will replace each (occurrence) of a free name $x$ in $P$ by
$\quotep{\procn{x}|\procn{x}}$.

Also, we will avail ourselves of the notation $x^{L}$ and $x^{R}$ to
denote injections of a name into disjoint copies of the name
space. There are numerous ways to accomplish this. One example can be
found in \cite{MeredithR05}. This notation overloads to vectors of
names: $\vec{x}^{\pi} := (x_{i}^{\pi} \; : \; 0 \leq i < |\vec{x}| )$ where $\pi \in \{L,R\}$.

We also use $P^{\Box} := P|\Box$.

In \cite{MeredithR05} an interpretation of the new operator is
given. It turns out that there are several possible interpretations
all enjoying the requisite algebraic properties of the operator (see
\cite{milner91polyadicpi}). We will therefore make liberal use of
$(\nu\; \vec{x})P$.

% subsection the_syntax_and_semantics_of_the_notation_system (end)   

\input{qm2pi.qmops} 

\input{qm2pi.sterngerlach} 

\input{qm2pi.metric} 

% section concurrent_process_calculi (end)

%\input{qm2pi.proofsketch}

% section proof sketch (end)

%\input{qm2pi.slviaknots} 

% section spatial logic via knots (end)

\input{qm2pi.conclusion}

% section conclusion (end)

%\input{qm2pi.dtcodes} 

% section wiring algorithm (end)

\input{qm2pi.ack} 

% section acknowledgments (end)

\newpage


\bibliographystyle{plain}   
\bibliography{../../biblios/main.bib}

\input{qm2pi.rhodetails}

\end{document}

 

% section acknowledgments (end)

\newpage


\bibliographystyle{plain}   
\bibliography{../../biblios/main.bib}

\documentclass[12pt]{llncs}
%\documentclass{jktr}

\usepackage[pdftex]{hyperref}                   
\usepackage {listings}
\usepackage {mathpartir}
\usepackage{bcprules}
%\usepackage{listings}
                       
\usepackage{graphicx} 
%\usepackage[margins=2.5cm,nohead,nofoot]{geometry}
%\usepackage{geometry}
\usepackage{amsfonts}
\usepackage{amstext}
\usepackage{latexsym}
\usepackage{amssymb}
\usepackage{color}


%\include{myPreamble}
\include{qm2pi.local} 

%\ifpdf
%\usepackage[pdftex]{graphicx}
%\else
%\usepackage{graphicx}
%\fi

 % \ifpdf
%  \usepackage{pdfsync}
%  \if


%\title{Brief Article}
%\author{David F. Snyder}
%\author{L.G. Meredith}

%\address{Dept. of Math., Texas State University--San Marcos, San Marcos, TX 78666}
       
\pagestyle{empty}


\begin{document}

\lstset{language=[Objective]Caml,frame=shadowbox}

\input{qm2pi.front}

% section front matter (end)

\input{qm2pi.intro} 
 
% section introduction (end)

% \input{qm2pi.knotations} 

% section notation (end)

\input{qm2pi.process.calculi} 

% section concurrent_process_calculi_and_spatial_logics_ (end)
    
%\input{qm2pi.knots2pi} 

%\input{qm2pi.trefoil} 

%\input{qm2pi.mainthm} 

% subsection basic_interpretation (end)

%\input{qm2pi.rho.presentation} 
\subsection{The syntax and semantics of the notation system}\label{sub:the_syntax_and_semantics_of_the_notation_system} % (fold)

We now summarize a technical presentation of the calculus that
embodies our theory of dynamics. The typical presentation of such a
calculus follows the style of giving generators and relations on
them. The grammar, below, describing term constructors, freely
generates the set of processes, $\Proc$. This set is then quotiented
by a relation known as structural congruence and it is over this set
that the notion of dynamics is expressed. This presentation is
essentially that of \cite{MeredithR05} with the addition of
polyadicity and summation. For readability we have relegated some of
the technical subtleties to an appendix.

\subsubsection{Process grammar}\label{subsub:process_grammar}

\begin{mathpar}
  \inferrule* [lab=synchronization] {} {{M} \bc \pzero \;|\; x?F \;|\; x!C }
  \and
  \inferrule* [lab=abstraction] {} {{F} \bc (x)P}
  \and
  \inferrule* [lab=concretion] {} {{C} \bc \langle Q \rangle}
  \and
  \inferrule* [lab=process] {} {{P,Q} \bc M \;| \;P|Q \;|\; @{x}}
  \and
  \inferrule* [lab=name] {} {{x} \bc \quotep{P}}
\end{mathpar} 

Note that $\vec{x}$ (resp. $\vec{P}$) denotes a vector of names
(resp. processes) of length $|\vec{x}|$ (resp. $|\vec{P}|$). We adopt
the following useful abbreviations.

\begin{mathpar}
   x?(\vec{y}).P := x.(\vec{y})P \and  x\clift{\vec{P}} := x.\clift{\vec{P}}
   \and x!(y) := \lift{x}{\dropn{y}}
   \and \Pi_{i=0}^{n-1}P_i := P_0 | \ldots | P_{n-1}
\end{mathpar}

\subsubsection{Structural congruence}

\paragraph{Free and bound names and alpha-equivalence.} At the
core of structural equivalence is alpha-equivalence which identifies
process that are the same up to a change of variable. Formally, we
recognize the distinction between free and bound names. The free names
of a process, $\freenames{P}$, may be calculated recursively as
follows:

\begin{mathpar}
\freenames{\pzero} := \emptyset
  \and \\
  \freenames{x?(y).P} := \{ x \} \cup (\freenames{P} \setminus \{ y \})
  \and 
  \freenames{x!\langle P \rangle} := \{ x \} \cup \{ P \} 
  \and \\
  \freenames{P|Q} := \freenames{P} \cup \freenames{Q}
  \and \\
  \freenames{@{x}} := \{ x \}
\end{mathpar}

$\pi$
$\quotep{\pi}$

$\freenames{-} : \pi \to \mathcal{P}(\quotep{\pi})$

\begin{eqnarray*}
  \freenames{\pzero} & := & \emptyset \\
  \freenames{x?(y).P} & := & \{ x \} \cup (\freenames{P} \setminus \{ y \}) \\
  \freenames{x!\langle P \rangle} & := & \{ x \} \cup \{ P \} \\
  \freenames{P|Q} & := & \freenames{P} \cup \freenames{Q} \\
  \freenames{\dropn{x}} & := & \{ x \}
\end{eqnarray*}

The bound names of a process, $\boundnames{P}$, are those names occurring in $P$
that are not free. For example, in $x?(y).0$, the name $x$ is free, while $y$ is bound.

\begin{mathpar}
  \inferrule* [lab=monoidal-laws] {} { P|Q \equiv Q|P \and P|0 \equiv P \and P|(Q|R) \equiv (P|Q)|R }
\end{mathpar}

\begin{mathpar}
  \inferrule* [lab=alpha-equivalence] {} { (x)P \equiv (y)P\{y/x\} \and y \not\in \freenames{P} }
\end{mathpar}

\begin{definition}
Then two processes, $P,Q$, are alpha-equivalent if $P = Q\{\vec{y}/\vec{x}\}$ for
some $\vec{x} \in \boundnames{Q},\vec{y} \in \boundnames{P}$, where $Q\{\vec{y}/\vec{x}\}$
denotes the capture-avoiding substitution of $\vec{y}$ for $\vec{x}$ in $Q$.
\end{definition}

\begin{definition}
  The {\em structural congruence} \cite{SangiorgiWalker} , $\equiv$,
  between processes is the least congruence containing
  alpha-equivalence, satisfying the abelian monoid laws
  (associativity, commutativity and $\pzero$ as identity) for parallel
  composition $|$ and for summation $+$.
\end{definition}

\subsection{Name equivalence}

We take name equivalence, written $\nameeq$, to be the smallest
equivalence relation generated by the following rules.

\begin{mathpar}
\inferrule*[lab=Quote-drop]
{ }
{ \quotep{@{x}} \nameeq x }

\inferrule*[lab=Struct-equiv]
{ P \scong Q }
{ \quotep{P} \nameeq \quotep{Q} }
\end{mathpar}

The astute reader will have noticed that the mutual recursion of names
and processes imposes a mutual recursion on alpha-equivalence and
structural equivalence via name-equivalence. Fortunately, all of this
works out pleasantly and we may calculate in the natural way, free of
concern. The reader interested in the details is referred to the
appendix \ref{appendix:rho_details}.

\subsection{Substitution}

We use $\Proc$ for the set of processes, $\QProc$ for the set of
names, and $\id{\{}\vec{y} / \vec{x} \id{\}}$ to denote partial maps,
$s : \QProc \rightarrow \QProc$. A map, $s$ lifts, uniquely, to a map
on process terms, $\widehat{s} : \Proc \rightarrow \Proc$ by the
following equations.

\begin{mathpar}
  (0) \psubstp{Q}{P} := 0 \\
  (R \juxtap S) \psubstp{Q}{P}
  :=    
  (R)\psubstp{Q}{P} \juxtap (S) \psubstp{Q}{P} \\
  (x?(y).R) \psubstp{Q}{P}    
  :=    
  (x)\substp{Q}{P} (z)\concat( (R \psubstn{z}{y}) \psubstp{Q}{P} ) \\
  (\lift{x}{R}) \psubstp{Q}{P}  
  :=
  \lift{(x)\substp{Q}{P}}{ R \psubstp{Q}{P} } \\
%   (\dropn{x})  \psubstp{Q}{P}       
%   := 
%   \left\{ 
%     \begin{array}{ccc} 
%       \dropn{\quotep{Q}} & & x \nameeq \quotep{P} \\
%       \dropn{x} & & otherwise \\
%     \end{array}
%   \right. 
  (\dropn{x})  \psubstp{Q}{P}       
  := 
  \left\{ 
    \begin{array}{ccc} 
      Q & & x \nameeq \quotep{P} \\
      \dropn{x} & & otherwise \\
    \end{array}
  \right.
\end{mathpar}
 

where

\begin{eqnarray}
  (x)\id{\{} \lpquote Q \rpquote / \lpquote P \rpquote \id{\}}            = 
  \left\{ 
    \begin{array}{ccc}
      \lpquote Q \rpquote & & x \nameeq \lpquote P \rpquote \\
      x & & otherwise \\
    \end{array}
  \right. \nonumber
\end{eqnarray}

and $z$ is chosen distinct from $\quotep{P}$, $\quotep{Q}$, the free
names in $Q$, and all the names in $R$. Our $\alpha$-equivalence will
be built in the standard way from this substitution.

\begin{remark}\label{rem:no_self_referential_names}
  One consequence of these definitions is that $\forall P. \quotep{P}
  \not\in \freenames{P}$.
\end{remark}

\subsection{ Dynamic quote: an example }

Anticipating something of what's to come, consider applying the
substitution, $\widehat{\id{\{}u / z \id{\}}}$, to the following pair
of processes, $\lift{w}{y!(z)}$ and $w[ \lpquote y!(z) \rpquote ]$.

\begin{eqnarray}
	\lift{w}{y!(z)}\widehat{\id{\{}u / z \id{\}}}
		& = &
		\lift{w}{y!(u)} \nonumber\\
	w[ \lpquote y!(z) \rpquote ] \widehat{ \id{\{}u / z \id{\}} }
		& = &
		w[ \lpquote y!(z) \rpquote ] \nonumber
\end{eqnarray}

Because the body of the process between quotes is impervious to
substitution, we get radically different answers. In fact, by
examining the first process in an input context,
e.g. $x?(z).\lift{w}{y!(z)}$, we see that the process under the lift
operator may be shaped by prefixed inputs binding a name inside it. In
this sense, the lift operator will be seen as a way to dynamically
construct processes before reifying them as names.

Finally equipped with these standard features we can present the
dynamics of the calculus.

\subsubsection{Operational semantics} 

Finally, we introduce the computational dynamics. What marks these
algebras as distinct from other more traditionally studied algebraic
structures, e.g. vector spaces or polynomial rings, is the manner in
which dynamics is captured. In traditional structures, dynamics is typically
expressed through morphisms between such structures, as in linear maps
between vector spaces or morphisms between rings. In algebras
associated with the semantics of computation, the dynamics is
expressed as part of the algebraic structure itself, through a
reduction reduction relation typically denoted by $\red$. Below, we
give a recursive presentation of this relation for the calculus used
in the encoding.

$\red \subseteq \pi \times \pi$
$\red : \pi \to \mathcal{P}(\pi)$

\begin{mathpar}
  \inferrule* [lab=Comm] { \textsf{match}( x_{src}, x_{trgt} ) } { x_{trgt}?(y)P \; | \; x_{src}!\langle {Q} \rangle \red P\{\quotep{Q}/y}\} }
  \and \\
  \inferrule* [lab=Par] {{P} \red {P}'} {{{P} | {Q}} \red {{P}' | {Q}}}
  \and
  \inferrule* [lab=Equiv]{{{P} \scong {P}'} \andalso {{P}' \red {Q}'} \andalso {{Q}' \scong {Q}}}{{P} \red {Q}}
\end{mathpar}

\begin{eqnarray*}
  match_{\equiv} (\quotep{P},\quotep{Q}) & := & P \equiv Q \\
  match_{\dagger}(\quotep{P},\quotep{Q}) & := & \forall R. P|Q \red^{*} R => R \red^{*} 0 \\
  match_{K}(\quotep{P},\quotep{Q}) & := & K \mbox{ for some context } K
\end{eqnarray*}

$u?(x)P | u!\langle Q \rangle \red P\{\quotep{Q}/x\}$

%We write $\wred$ for $\red^*$, and $P\red$ if $\exists Q $ such that $ P \red Q$.
We write $P\red$ if $\exists Q $ such that $ P \red Q$ and $P\not\red$, otherwise.

\section{Replication}

As mentioned before, it is known that replication (and hence
recursion) can be implemented in a higher-order process algebra
\cite{SangiorgiWalker}. As our first example of calculation with the
machinery thus far presented we give the construction explicitly in
the {\rhoc}.

\begin{eqnarray}
	D_{x} & := & \prefix{x}{y}{(\binpar{\outputp{x}{y}}{@{y}})} \nonumber\\
	\bangp_{x}{P} & := & \binpar{{x}!\langle{\binpar{D_{x}}{P}}\rangle}{D_{x}} \nonumber
\end{eqnarray}

\begin{eqnarray}
	\bangp_{x}{P} & & \nonumber\\
	=
	& {x}!\langle{(\prefix{x}{y}{(\outputp{x}{y} | @{y})) | P}}\rangle 
	      | \prefix{x}{y}{(\outputp{x}{y} | @{y})} & \nonumber\\
	\red
	& (\outputp{x}{y} | @{y})\substn{\quotep{(\prefix{x}{y}{(@{y} | \outputp{x}{y})) | P}}}{y} & \nonumber\\
	=
	& \outputp{x}{\quotep{(\prefix{x}{y}{(\outputp{x}{y} | @{y})) | P}}}
	  | {(\prefix{x}{y}{(\outputp{x}{y} | @{y})) | P}} & \nonumber\\
	\red
	& \ldots & \nonumber\\
	\red^*
	& P | P | \ldots & \nonumber
\end{eqnarray}

Of course, this encoding, as an implementation, runs away, unfolding
$\bangp{P}$ eagerly. A lazier and more implementable replication
operator, restricted to input-guarded processes, may be obtained as follows.

\begin{eqnarray}
\bangp{\prefix{u}{v}{P}} 
	:= 
	\binpar{\lift{x}{\prefix{u}{v}{(\binpar{D(x)}{P})}}}{D(x)} \nonumber
\end{eqnarray}

\begin{remark}
  Note that the lazier definition still does not deal with summation
  or mixed summation (i.e. sums over input and output). The reader is
  invited to construct definitions of replication that deal with these
  features. 

  Further, the definitions are parameterized in a name, $x$. Can you,
  gentle reader, make a definition that eliminates this parameter and
  guarantees no accidental interaction between the replication
  machinery and the process being replicated -- i.e. no accidental
  sharing of names used by the process to get its work done and the
  name(s) used by the replication to effect copying. This latter
  revision of the definition of replication is crucial to obtaining
  the expected identity $!!P \sim !P$.
\end{remark}

\begin{remark}\label{rem:paradoxical_combinator}
  The reader familiar with the lambda calculus will have noticed the
  similarity between $D$ and the paradoxical combinator.

  [Ed. note: the existence of this seems to suggest we have to be more
  restrictive on the set of processes and names we admit if we are to
  support no-cloning.]
\end{remark}

\subsubsection{Bisimulation}

The computational dynamics gives rise to another kind of equivalence,
the equivalence of computational behavior. As previously mentioned
this is typically captured \emph{via} some form of bisimulation.

% The notion we use in this paper is weak barbed bisimulation
% \cite{milner91polyadicpi}.

The notion we use in this paper is derived from weak barbed
bisimulation \cite{milner91polyadicpi}. 

\begin{definition}
An \emph{observation relation}, $\downarrow_{\mathcal N}$, over a set
of names, $\mathcal N$, is the smallest relation satisfying the rules
below.

\infrule[Out-barb]{y \in {\mathcal N}, \; x \nameeq y}
		  {\outputp{x}{v} \downarrow_{\mathcal N} x}
\infrule[Par-barb]{\mbox{$P\downarrow_{\mathcal N} x$ or $Q\downarrow_{\mathcal N} x$}}
		  {\binpar{P}{Q} \downarrow_{\mathcal N} x}

We write $P \Downarrow_{\mathcal N} x$ if there is $Q$ such that 
$P \wred Q$ and $Q \downarrow_{\mathcal N} x$.
\end{definition}

\begin{definition}
%\label{def.bbisim}
An  ${\mathcal N}$-\emph{barbed bisimulation} over a set of names, ${\mathcal N}$, is a symmetric binary relation 
${\mathcal S}_{\mathcal N}$ between agents such that $P\rel{S}_{\mathcal N}Q$ implies:
\begin{enumerate}
\item If $P \red P'$ then $Q \wred Q'$ and $P'\rel{S}_{\mathcal N} Q'$.
\item If $P\downarrow_{\mathcal N} x$, then $Q\Downarrow_{\mathcal N} x$.
\end{enumerate}
$P$ is ${\mathcal N}$-barbed bisimilar to $Q$, written
$P \wbbisim_{\mathcal N} Q$, if $P \rel{S}_{\mathcal N} Q$ for some ${\mathcal N}$-barbed bisimulation ${\mathcal S}_{\mathcal N}$.
\end{definition}

$\mathcal{R} \subseteq \pi \times \pi$

$P \mathcal{R} Q => \forall P'. P \red P' \Rightarrow \exists Q'. Q \red Q', P' \mathcal{R} Q'$

$P \vdash x \Rightarrow Q \vdash x$

\begin{mathpar}
  \inferrule*[lab=Out-barb]{x \nameeq y}{{y}!\langle{Q}\rangle \vdash x}
  \and
  \inferrule*[lab=Par-barb]{\mbox{$P\vdash x$ or $Q\vdash x$}}{\binpar{P}{Q} \vdash x}
\end{mathpar}

\subsubsection{Contexts}

One of the principle advantages of computational calculi like the
$\pi$-calculus is a well-defined notion of context,
contextual-equivalence and a correlation between
contextual-equivalence and notions of bisimulation. The notion of
context allows the decomposition of a process into (sub-)process and
its syntactic environment, its context. Thus, a context may be
thought of as a process with a ``hole'' (written $\Box$) in it. The
application of a context $M$ to a process $P$, written $M[P]$, is
tantamount to filling the hole in $M$ with $P$. In this paper we do
not need the full weight of this theory, but do make use of the notion
of context in the proof the main theorem. 

\begin{mathpar}
  \inferrule* [lab=summation] {} {{M_{M},M_{N}} \bc \Box \;|\; x.M_{A} \;|\; M_{M}+M_{N}}
  \and
  \inferrule* [lab=agent] {} {{M_{A}} \bc (\vec{x})M_{P} \;| \; \clift{P_0,\ldots,M_{P},\ldots,P_N}}
  \and \\
  \inferrule* [lab=process] {} {{M_{P}} \bc M_{N} \;| \;P|M_{P} }
\end{mathpar} 

\begin{mathpar}
  \inferrule* [lab=sychronization] {} {M_{N} \bc \Box \;|\; x?M_{F} \;|\; x!M_{C}}
  \and
  \inferrule* [lab=abstraction] {} {{M_{F}} \bc (x)M_{P} }
  \and
  \inferrule* [lab=concretion] {} {{M_{C}} \bc \langle M_{P} \rangle }
  \and \\
  \inferrule* [lab=process] {} {{M_{P}} \bc M_{N} \;| \;P|M_{P} }
\end{mathpar}

\begin{definition}[contextual application] Given a context $M$, and
  process $P$, we define the \emph{contextual application}, $M[P] :=
  M\{P/\Box\}$. That is, the contextual application of M to P is the
  substitution of $P$ for $\Box$ in $M$.
\end{definition}

$\meaningof{-} : L \to \mathcal{P}(\pi)$

\begin{mathpar}
  \inferrule* [lab=collection] {} {\meaningof{true} = \pi, \and \meaningof{~E} = \pi \setminus \meaningof{E}, \and \meaningof{E_{1} \& E_{2}} = \meaningof{E_{1}} \cap \meaningof{E_{2}}}
\end{mathpar}

\begin{mathpar}
  \inferrule* [lab=structure] {} {\meaningof{0} = \{ P \in \pi | P \equiv 0 \}, \and \\ \meaningof{E_1 | E_2} = \{ P \in \pi | P \equiv P_{1} | P_{2}, P_{1} \in \meaningof{E_{1}}, P_{2} \in \meaningof{E_2}\} }
\end{mathpar}

\begin{mathpar}
 \inferrule* [lab=behavior] {} {\meaningof{\langle a?b \rangle E} = \{ P \in \pi | P \equiv Q | u?(y)P', \\ \and \\\\ \and \\ \;\;\; u \in \meaningof{a}, \forall z.P'\{z/y\} \in \meaningof{E\{z/b\}}\}, \and \\ \meaningof{a!E} = \{ P \in \pi | P \equiv Q | x!\langle P' \rangle, x \in \meaningof{a} P' \in \meaningof{E}\} }
\end{mathpar}

\begin{mathpar}
 \inferrule* [lab=nominal] {} {\meaningof{\quotep{E}} = \{ \quotep{P} \in \quotep{\pi} | P \in \meaningof{E} \}, \and \meaningof{\quotep{P}} = \{ \quotep{Q} \in \quotep{\pi} | P \equiv Q \} \and \\ \meaningof{@\quotep{E}} = \{ P \in \pi | P \equiv @x, x \in \meaningof{E} \}}
\end{mathpar}

\begin{eqnarray*}
  \\
  \meaningof{-} : TS \to ST
\end{eqnarray*}

\begin{eqnarray*}
  \\
  L : TS \to ST
\end{eqnarray*}

\begin{eqnarray*}
  \\
  P \models E \iff P \in \meaningof{E}
\end{eqnarray*}

\begin{eqnarray*}
  P \approx_{L} Q \iff \forall E \in L. P \models E \iff Q \models E
\end{eqnarray*}

\begin{eqnarray*}
  P \approx_{K} Q
\end{eqnarray*}

\begin{eqnarray*}
  P \approx Q
\end{eqnarray*}

$\approx_{K} = \approx = \approx_{L}$

\subsubsection{Contextual duality}

Note that contexts extend the quotation operation to a family of
operations from processes to names. Given a context, $M$, we can
define a \emph{nominal context}, $\quotep{M}$ by $\quotep{M}[P] :=
\quotep{M[P]}$. To foreshadow what is to come we observe that these
operations enjoy a duality with processes very much like the duality
between vectors and maps from vectors to scalars.

Further, because the calculus is essentially higher-order, we have a
correspondence between contexts and processes. More specifically,
given a name $x$ and a context $M$ we can construct $M^{*}_{x}$ such
that 

\begin{mathpar}
  M^{*}_{x} | \lift{x}{P} \red M[P]
\end{mathpar}

namely,

\begin{mathpar}
  M^{*}_{x} := x?(u).M[\dropn{u}]
\end{mathpar}

The dependence of $M^{*}_{x}$ on a name makes it an abstraction, 

\begin{mathpar}
  M^{*} := (x)x?(u).M[\dropn{u}]
\end{mathpar}

\subsection{Additional notation}

It will sometimes be convenient to denote the process a name
quotes. We already have the notation $x = \quotep{P}$, but it will be
convenient to introduce an alternate notation, $\procn{x}$, when we
want to emphasize the connection to the use of the name. Note that, by
virtue of name equivalence, $\quotep{\procn{x}} \nameeq x$; so, the
notation is consistent with previous definitions.

Further, because names have structure it is possible to effect
substitutions on the basis of that structure. This means we need to
upgrade our notation for substitutions, which we accomplish by
adapting comprehension notation. Thus,

\begin{mathpar}
  P\{ y / x : x \in S \}
\end{mathpar}

is interpreted to mean the process derived from P by replacing (in a
capture-avoiding manner) each occurrence of $x$ in $S$ by $y$. For example,

\begin{mathpar}
  P\{ \quotep{\procn{x}|\procn{x}} / x : x \in \freenames{P} \}
\end{mathpar}

will replace each (occurrence) of a free name $x$ in $P$ by
$\quotep{\procn{x}|\procn{x}}$.

Also, we will avail ourselves of the notation $x^{L}$ and $x^{R}$ to
denote injections of a name into disjoint copies of the name
space. There are numerous ways to accomplish this. One example can be
found in \cite{MeredithR05}. This notation overloads to vectors of
names: $\vec{x}^{\pi} := (x_{i}^{\pi} \; : \; 0 \leq i < |\vec{x}| )$ where $\pi \in \{L,R\}$.

We also use $P^{\Box} := P|\Box$.

In \cite{MeredithR05} an interpretation of the new operator is
given. It turns out that there are several possible interpretations
all enjoying the requisite algebraic properties of the operator (see
\cite{milner91polyadicpi}). We will therefore make liberal use of
$(\nu\; \vec{x})P$.

% subsection the_syntax_and_semantics_of_the_notation_system (end)   

\input{qm2pi.qmops} 

\input{qm2pi.sterngerlach} 

\input{qm2pi.metric} 

% section concurrent_process_calculi (end)

%\input{qm2pi.proofsketch}

% section proof sketch (end)

%\input{qm2pi.slviaknots} 

% section spatial logic via knots (end)

\input{qm2pi.conclusion}

% section conclusion (end)

%\input{qm2pi.dtcodes} 

% section wiring algorithm (end)

\input{qm2pi.ack} 

% section acknowledgments (end)

\newpage


\bibliographystyle{plain}   
\bibliography{../../biblios/main.bib}

\input{qm2pi.rhodetails}

\end{document}



\end{document}



% section front matter (end)

\section{Introduction}\label{sec:introduction} % (fold)
In this draft of the material i am going to have to dispense with the
usual writing conventions adopted in papers on these topics. i'm going
to have adopt whatever tone i need at the time i'm writing up the
calculations. Sometimes this may be very conversational; others it may
be the barest mathematical grunts; others still it may be that i have
lifted text from one of my other papers because the exposition of some
point was better said there. i hope that my readers are not unduly put
out by this decision. i'm not doing this to flout convention or be
rebellious. i find these calculations very technically challenging. To
keep everything going technically, something has to give; i have to
let go of some cognitive burden. So, the academic writing style --
with all of its trade-offs in terms of facilitating technical
communication -- is what i'm letting go of. Perhaps subsequent drafts
can be tightened and polished, but for now, i'm going to speak as if
we were sitting together in a coffee shop with a laptop, wifi and a
pad of paper and a pencil.

So, here's what i have to say. We -- you and i, comfortably ensconced
in our coffee shop and well-equipped with our tools -- can realize and
carry out the calculations of quantum mechanics over a very different
formal theory of dynamics, a formal theory of dynamics that
corresponds to a theory of concurrent computation with
\emph{reflection}. It has the advantage that the underlying theory is
already `quantized', but supports analogues all of the continuuous
operations. Strikingly, this underlying theory has recently been
connected with a notion of metric that we can show, by calculating
together, coincides with the metric induced by the inner product.

There are a lot of reasons why you might be interested in seeing
calculations of this form. Here's why i'm interested. For the past
several centuries there has been no competitor to the ``Newtonian''
account of dynamics. As a result the predominant share of accounts of
dynamical systems and situations have had to be formulated in terms of
the Newtonian machinery. i view this as an intellectually dangerous
position to occupy. Everything, despite it's intrinsic shape, turns
into a nail to be hit with this hammer. Recently, however, the theory
of computation has matured to the point where we have candidates for
theories of dynamics that offer very different perspective on
reasoning about dynamical systems and situations. Testing these
candidates against very successful accounts of dynamical situations,
like quantum mechanics, is going to give us some sense of how mature
they are and some measure of the quality of these accounts of
dynamics.

\subsection{Summary of contributions and outline of paper}

So, we're going to develop an interpretation of the operations of
quantum mechanics normally interpreted by Hilbert spaces and
operators. We're going to do this over a theory of computation. Note
that this is very different than the usual quantum computation program
which develops notions of computation over quantum mechanics. Rather,
we are developing a story that aligns with Wheeler's slogan: It from
Bit. To do this we will first provide an account of the theory of
computation at play here. Then we will dive into a calculation-driven
interpretation of the operations of quantum mechanics.

The reason we take this approach is that -- until very recently --
there hasn't been an axiomatic account of quantum mechanics. As a
result there has been no sharp delineation of the mathematical theory
supporting interpretation of the physical theory and the physical
theory, itself. So, ambient features of the maths are free to be
exploited (or supressed) without a real accounting of their physical
relevance. There is no sharp statement ``here's the physical theory''
qua \emph{theory} and ``here's the mathematical interpretation''
enabling a judgment of how faithful the interpretation is -- apart
from experimental observation. When there is an axiomatic account we
can judge how well a given mathematical formalism supports an
interpretation of the axioms, independent of
experimentation. Likewise, we can judge how well we have captured our
physical evidence and experience with our axiomatics, independent of
any specific mathematical implementation, with accidental detail that
may or may not have physical significance. 

In lieu of a fully fleshed out and vetted axiomatic account of quantum
mechanics, interpreting the operational notions in service of modeling
physical systems will have to suffice. In other words, we are not in
the business of providing a model of Hilbert spaces and operators. We
are in the business of providing a model of quantum mechanics because
we are motivated by testing our notions of dynamics against physical
theory; and, the predictive calculations of the physical theory must
serve as the best formulation -- shy of a fully fleshed out axiomatic
account -- of the physical theory itself (as they have for scientific
theories since time immemorial). Put another way, despite a
whole-hearted commitment to an It-from-Bit ontology, we are firmly
aligned with the shut-up-and-calculate camp as the best way to obtain
results either from the physical perspective or as a quality assurance
measure of our fledgling theory of dynamics.

In detail, we present a reflective process calculus. Then we develop
intuitive correspondences between the notions available in this
calculus and the usual physical notions supporting quantum mechanical
calculations. Thus, 

\begin{table}[htp]
  \center{
    \fbox{
      \begin{tabular}{c|c}
        quantum mechanics & process calculus \\
        \hline
        scalar & name \\
        state vector & process \\
        dual & contextual duals \\
        matrix & formal sums of process-context-dual pairs \\
        orthogonality & process annihilation \\
        inner product & execution-formula + quoting
      \end{tabular}
    }
  }
  \caption{QM - process calculi correspondences}
\end{table}

Then we tighten up these intuitions to operational definitions. We
employ the Dirac notation as the best proxy we can find for an
abstract syntax of the quantum mechanical notions. The definitions we
develop put us in contact with equational constraints coming from the
theory that we demonstrate the definitions and calculations satisfy.

This puts us in a position to shut up and calculate for the
Stern-Gerlach experimental set up, showing how these predictive
calculations become calculations on processes in our theory of a
reflective process calculus.

Penultimately, we demonstrate that the notion of metric coming from
the inner product coincides with the notion of metric available from
the theory of bisimulation. This demonstration gives us the right to
think of space as arising from behavior. Finally, we consider where we
might go from the new vantage point we have obtained.

% section introduction (end) 
 
% section introduction (end)

% \documentclass[12pt]{llncs}
%\documentclass{jktr}

\usepackage[pdftex]{hyperref}                   
\usepackage {listings}
\usepackage {mathpartir}
\usepackage{bcprules}
%\usepackage{listings}
                       
\usepackage{graphicx} 
%\usepackage[margins=2.5cm,nohead,nofoot]{geometry}
%\usepackage{geometry}
\usepackage{amsfonts}
\usepackage{amstext}
\usepackage{latexsym}
\usepackage{amssymb}
\usepackage{color}


%\include{myPreamble}
\documentclass[12pt]{llncs}
%\documentclass{jktr}

\usepackage[pdftex]{hyperref}                   
\usepackage {listings}
\usepackage {mathpartir}
\usepackage{bcprules}
%\usepackage{listings}
                       
\usepackage{graphicx} 
%\usepackage[margins=2.5cm,nohead,nofoot]{geometry}
%\usepackage{geometry}
\usepackage{amsfonts}
\usepackage{amstext}
\usepackage{latexsym}
\usepackage{amssymb}
\usepackage{color}


%\include{myPreamble}
\include{qm2pi.local} 

%\ifpdf
%\usepackage[pdftex]{graphicx}
%\else
%\usepackage{graphicx}
%\fi

 % \ifpdf
%  \usepackage{pdfsync}
%  \if


%\title{Brief Article}
%\author{David F. Snyder}
%\author{L.G. Meredith}

%\address{Dept. of Math., Texas State University--San Marcos, San Marcos, TX 78666}
       
\pagestyle{empty}


\begin{document}

\lstset{language=[Objective]Caml,frame=shadowbox}

\input{qm2pi.front}

% section front matter (end)

\input{qm2pi.intro} 
 
% section introduction (end)

% \input{qm2pi.knotations} 

% section notation (end)

\input{qm2pi.process.calculi} 

% section concurrent_process_calculi_and_spatial_logics_ (end)
    
%\input{qm2pi.knots2pi} 

%\input{qm2pi.trefoil} 

%\input{qm2pi.mainthm} 

% subsection basic_interpretation (end)

%\input{qm2pi.rho.presentation} 
\subsection{The syntax and semantics of the notation system}\label{sub:the_syntax_and_semantics_of_the_notation_system} % (fold)

We now summarize a technical presentation of the calculus that
embodies our theory of dynamics. The typical presentation of such a
calculus follows the style of giving generators and relations on
them. The grammar, below, describing term constructors, freely
generates the set of processes, $\Proc$. This set is then quotiented
by a relation known as structural congruence and it is over this set
that the notion of dynamics is expressed. This presentation is
essentially that of \cite{MeredithR05} with the addition of
polyadicity and summation. For readability we have relegated some of
the technical subtleties to an appendix.

\subsubsection{Process grammar}\label{subsub:process_grammar}

\begin{mathpar}
  \inferrule* [lab=synchronization] {} {{M} \bc \pzero \;|\; x?F \;|\; x!C }
  \and
  \inferrule* [lab=abstraction] {} {{F} \bc (x)P}
  \and
  \inferrule* [lab=concretion] {} {{C} \bc \langle Q \rangle}
  \and
  \inferrule* [lab=process] {} {{P,Q} \bc M \;| \;P|Q \;|\; @{x}}
  \and
  \inferrule* [lab=name] {} {{x} \bc \quotep{P}}
\end{mathpar} 

Note that $\vec{x}$ (resp. $\vec{P}$) denotes a vector of names
(resp. processes) of length $|\vec{x}|$ (resp. $|\vec{P}|$). We adopt
the following useful abbreviations.

\begin{mathpar}
   x?(\vec{y}).P := x.(\vec{y})P \and  x\clift{\vec{P}} := x.\clift{\vec{P}}
   \and x!(y) := \lift{x}{\dropn{y}}
   \and \Pi_{i=0}^{n-1}P_i := P_0 | \ldots | P_{n-1}
\end{mathpar}

\subsubsection{Structural congruence}

\paragraph{Free and bound names and alpha-equivalence.} At the
core of structural equivalence is alpha-equivalence which identifies
process that are the same up to a change of variable. Formally, we
recognize the distinction between free and bound names. The free names
of a process, $\freenames{P}$, may be calculated recursively as
follows:

\begin{mathpar}
\freenames{\pzero} := \emptyset
  \and \\
  \freenames{x?(y).P} := \{ x \} \cup (\freenames{P} \setminus \{ y \})
  \and 
  \freenames{x!\langle P \rangle} := \{ x \} \cup \{ P \} 
  \and \\
  \freenames{P|Q} := \freenames{P} \cup \freenames{Q}
  \and \\
  \freenames{@{x}} := \{ x \}
\end{mathpar}

$\pi$
$\quotep{\pi}$

$\freenames{-} : \pi \to \mathcal{P}(\quotep{\pi})$

\begin{eqnarray*}
  \freenames{\pzero} & := & \emptyset \\
  \freenames{x?(y).P} & := & \{ x \} \cup (\freenames{P} \setminus \{ y \}) \\
  \freenames{x!\langle P \rangle} & := & \{ x \} \cup \{ P \} \\
  \freenames{P|Q} & := & \freenames{P} \cup \freenames{Q} \\
  \freenames{\dropn{x}} & := & \{ x \}
\end{eqnarray*}

The bound names of a process, $\boundnames{P}$, are those names occurring in $P$
that are not free. For example, in $x?(y).0$, the name $x$ is free, while $y$ is bound.

\begin{mathpar}
  \inferrule* [lab=monoidal-laws] {} { P|Q \equiv Q|P \and P|0 \equiv P \and P|(Q|R) \equiv (P|Q)|R }
\end{mathpar}

\begin{mathpar}
  \inferrule* [lab=alpha-equivalence] {} { (x)P \equiv (y)P\{y/x\} \and y \not\in \freenames{P} }
\end{mathpar}

\begin{definition}
Then two processes, $P,Q$, are alpha-equivalent if $P = Q\{\vec{y}/\vec{x}\}$ for
some $\vec{x} \in \boundnames{Q},\vec{y} \in \boundnames{P}$, where $Q\{\vec{y}/\vec{x}\}$
denotes the capture-avoiding substitution of $\vec{y}$ for $\vec{x}$ in $Q$.
\end{definition}

\begin{definition}
  The {\em structural congruence} \cite{SangiorgiWalker} , $\equiv$,
  between processes is the least congruence containing
  alpha-equivalence, satisfying the abelian monoid laws
  (associativity, commutativity and $\pzero$ as identity) for parallel
  composition $|$ and for summation $+$.
\end{definition}

\subsection{Name equivalence}

We take name equivalence, written $\nameeq$, to be the smallest
equivalence relation generated by the following rules.

\begin{mathpar}
\inferrule*[lab=Quote-drop]
{ }
{ \quotep{@{x}} \nameeq x }

\inferrule*[lab=Struct-equiv]
{ P \scong Q }
{ \quotep{P} \nameeq \quotep{Q} }
\end{mathpar}

The astute reader will have noticed that the mutual recursion of names
and processes imposes a mutual recursion on alpha-equivalence and
structural equivalence via name-equivalence. Fortunately, all of this
works out pleasantly and we may calculate in the natural way, free of
concern. The reader interested in the details is referred to the
appendix \ref{appendix:rho_details}.

\subsection{Substitution}

We use $\Proc$ for the set of processes, $\QProc$ for the set of
names, and $\id{\{}\vec{y} / \vec{x} \id{\}}$ to denote partial maps,
$s : \QProc \rightarrow \QProc$. A map, $s$ lifts, uniquely, to a map
on process terms, $\widehat{s} : \Proc \rightarrow \Proc$ by the
following equations.

\begin{mathpar}
  (0) \psubstp{Q}{P} := 0 \\
  (R \juxtap S) \psubstp{Q}{P}
  :=    
  (R)\psubstp{Q}{P} \juxtap (S) \psubstp{Q}{P} \\
  (x?(y).R) \psubstp{Q}{P}    
  :=    
  (x)\substp{Q}{P} (z)\concat( (R \psubstn{z}{y}) \psubstp{Q}{P} ) \\
  (\lift{x}{R}) \psubstp{Q}{P}  
  :=
  \lift{(x)\substp{Q}{P}}{ R \psubstp{Q}{P} } \\
%   (\dropn{x})  \psubstp{Q}{P}       
%   := 
%   \left\{ 
%     \begin{array}{ccc} 
%       \dropn{\quotep{Q}} & & x \nameeq \quotep{P} \\
%       \dropn{x} & & otherwise \\
%     \end{array}
%   \right. 
  (\dropn{x})  \psubstp{Q}{P}       
  := 
  \left\{ 
    \begin{array}{ccc} 
      Q & & x \nameeq \quotep{P} \\
      \dropn{x} & & otherwise \\
    \end{array}
  \right.
\end{mathpar}
 

where

\begin{eqnarray}
  (x)\id{\{} \lpquote Q \rpquote / \lpquote P \rpquote \id{\}}            = 
  \left\{ 
    \begin{array}{ccc}
      \lpquote Q \rpquote & & x \nameeq \lpquote P \rpquote \\
      x & & otherwise \\
    \end{array}
  \right. \nonumber
\end{eqnarray}

and $z$ is chosen distinct from $\quotep{P}$, $\quotep{Q}$, the free
names in $Q$, and all the names in $R$. Our $\alpha$-equivalence will
be built in the standard way from this substitution.

\begin{remark}\label{rem:no_self_referential_names}
  One consequence of these definitions is that $\forall P. \quotep{P}
  \not\in \freenames{P}$.
\end{remark}

\subsection{ Dynamic quote: an example }

Anticipating something of what's to come, consider applying the
substitution, $\widehat{\id{\{}u / z \id{\}}}$, to the following pair
of processes, $\lift{w}{y!(z)}$ and $w[ \lpquote y!(z) \rpquote ]$.

\begin{eqnarray}
	\lift{w}{y!(z)}\widehat{\id{\{}u / z \id{\}}}
		& = &
		\lift{w}{y!(u)} \nonumber\\
	w[ \lpquote y!(z) \rpquote ] \widehat{ \id{\{}u / z \id{\}} }
		& = &
		w[ \lpquote y!(z) \rpquote ] \nonumber
\end{eqnarray}

Because the body of the process between quotes is impervious to
substitution, we get radically different answers. In fact, by
examining the first process in an input context,
e.g. $x?(z).\lift{w}{y!(z)}$, we see that the process under the lift
operator may be shaped by prefixed inputs binding a name inside it. In
this sense, the lift operator will be seen as a way to dynamically
construct processes before reifying them as names.

Finally equipped with these standard features we can present the
dynamics of the calculus.

\subsubsection{Operational semantics} 

Finally, we introduce the computational dynamics. What marks these
algebras as distinct from other more traditionally studied algebraic
structures, e.g. vector spaces or polynomial rings, is the manner in
which dynamics is captured. In traditional structures, dynamics is typically
expressed through morphisms between such structures, as in linear maps
between vector spaces or morphisms between rings. In algebras
associated with the semantics of computation, the dynamics is
expressed as part of the algebraic structure itself, through a
reduction reduction relation typically denoted by $\red$. Below, we
give a recursive presentation of this relation for the calculus used
in the encoding.

$\red \subseteq \pi \times \pi$
$\red : \pi \to \mathcal{P}(\pi)$

\begin{mathpar}
  \inferrule* [lab=Comm] { \textsf{match}( x_{src}, x_{trgt} ) } { x_{trgt}?(y)P \; | \; x_{src}!\langle {Q} \rangle \red P\{\quotep{Q}/y}\} }
  \and \\
  \inferrule* [lab=Par] {{P} \red {P}'} {{{P} | {Q}} \red {{P}' | {Q}}}
  \and
  \inferrule* [lab=Equiv]{{{P} \scong {P}'} \andalso {{P}' \red {Q}'} \andalso {{Q}' \scong {Q}}}{{P} \red {Q}}
\end{mathpar}

\begin{eqnarray*}
  match_{\equiv} (\quotep{P},\quotep{Q}) & := & P \equiv Q \\
  match_{\dagger}(\quotep{P},\quotep{Q}) & := & \forall R. P|Q \red^{*} R => R \red^{*} 0 \\
  match_{K}(\quotep{P},\quotep{Q}) & := & K \mbox{ for some context } K
\end{eqnarray*}

$u?(x)P | u!\langle Q \rangle \red P\{\quotep{Q}/x\}$

%We write $\wred$ for $\red^*$, and $P\red$ if $\exists Q $ such that $ P \red Q$.
We write $P\red$ if $\exists Q $ such that $ P \red Q$ and $P\not\red$, otherwise.

\section{Replication}

As mentioned before, it is known that replication (and hence
recursion) can be implemented in a higher-order process algebra
\cite{SangiorgiWalker}. As our first example of calculation with the
machinery thus far presented we give the construction explicitly in
the {\rhoc}.

\begin{eqnarray}
	D_{x} & := & \prefix{x}{y}{(\binpar{\outputp{x}{y}}{@{y}})} \nonumber\\
	\bangp_{x}{P} & := & \binpar{{x}!\langle{\binpar{D_{x}}{P}}\rangle}{D_{x}} \nonumber
\end{eqnarray}

\begin{eqnarray}
	\bangp_{x}{P} & & \nonumber\\
	=
	& {x}!\langle{(\prefix{x}{y}{(\outputp{x}{y} | @{y})) | P}}\rangle 
	      | \prefix{x}{y}{(\outputp{x}{y} | @{y})} & \nonumber\\
	\red
	& (\outputp{x}{y} | @{y})\substn{\quotep{(\prefix{x}{y}{(@{y} | \outputp{x}{y})) | P}}}{y} & \nonumber\\
	=
	& \outputp{x}{\quotep{(\prefix{x}{y}{(\outputp{x}{y} | @{y})) | P}}}
	  | {(\prefix{x}{y}{(\outputp{x}{y} | @{y})) | P}} & \nonumber\\
	\red
	& \ldots & \nonumber\\
	\red^*
	& P | P | \ldots & \nonumber
\end{eqnarray}

Of course, this encoding, as an implementation, runs away, unfolding
$\bangp{P}$ eagerly. A lazier and more implementable replication
operator, restricted to input-guarded processes, may be obtained as follows.

\begin{eqnarray}
\bangp{\prefix{u}{v}{P}} 
	:= 
	\binpar{\lift{x}{\prefix{u}{v}{(\binpar{D(x)}{P})}}}{D(x)} \nonumber
\end{eqnarray}

\begin{remark}
  Note that the lazier definition still does not deal with summation
  or mixed summation (i.e. sums over input and output). The reader is
  invited to construct definitions of replication that deal with these
  features. 

  Further, the definitions are parameterized in a name, $x$. Can you,
  gentle reader, make a definition that eliminates this parameter and
  guarantees no accidental interaction between the replication
  machinery and the process being replicated -- i.e. no accidental
  sharing of names used by the process to get its work done and the
  name(s) used by the replication to effect copying. This latter
  revision of the definition of replication is crucial to obtaining
  the expected identity $!!P \sim !P$.
\end{remark}

\begin{remark}\label{rem:paradoxical_combinator}
  The reader familiar with the lambda calculus will have noticed the
  similarity between $D$ and the paradoxical combinator.

  [Ed. note: the existence of this seems to suggest we have to be more
  restrictive on the set of processes and names we admit if we are to
  support no-cloning.]
\end{remark}

\subsubsection{Bisimulation}

The computational dynamics gives rise to another kind of equivalence,
the equivalence of computational behavior. As previously mentioned
this is typically captured \emph{via} some form of bisimulation.

% The notion we use in this paper is weak barbed bisimulation
% \cite{milner91polyadicpi}.

The notion we use in this paper is derived from weak barbed
bisimulation \cite{milner91polyadicpi}. 

\begin{definition}
An \emph{observation relation}, $\downarrow_{\mathcal N}$, over a set
of names, $\mathcal N$, is the smallest relation satisfying the rules
below.

\infrule[Out-barb]{y \in {\mathcal N}, \; x \nameeq y}
		  {\outputp{x}{v} \downarrow_{\mathcal N} x}
\infrule[Par-barb]{\mbox{$P\downarrow_{\mathcal N} x$ or $Q\downarrow_{\mathcal N} x$}}
		  {\binpar{P}{Q} \downarrow_{\mathcal N} x}

We write $P \Downarrow_{\mathcal N} x$ if there is $Q$ such that 
$P \wred Q$ and $Q \downarrow_{\mathcal N} x$.
\end{definition}

\begin{definition}
%\label{def.bbisim}
An  ${\mathcal N}$-\emph{barbed bisimulation} over a set of names, ${\mathcal N}$, is a symmetric binary relation 
${\mathcal S}_{\mathcal N}$ between agents such that $P\rel{S}_{\mathcal N}Q$ implies:
\begin{enumerate}
\item If $P \red P'$ then $Q \wred Q'$ and $P'\rel{S}_{\mathcal N} Q'$.
\item If $P\downarrow_{\mathcal N} x$, then $Q\Downarrow_{\mathcal N} x$.
\end{enumerate}
$P$ is ${\mathcal N}$-barbed bisimilar to $Q$, written
$P \wbbisim_{\mathcal N} Q$, if $P \rel{S}_{\mathcal N} Q$ for some ${\mathcal N}$-barbed bisimulation ${\mathcal S}_{\mathcal N}$.
\end{definition}

$\mathcal{R} \subseteq \pi \times \pi$

$P \mathcal{R} Q => \forall P'. P \red P' \Rightarrow \exists Q'. Q \red Q', P' \mathcal{R} Q'$

$P \vdash x \Rightarrow Q \vdash x$

\begin{mathpar}
  \inferrule*[lab=Out-barb]{x \nameeq y}{{y}!\langle{Q}\rangle \vdash x}
  \and
  \inferrule*[lab=Par-barb]{\mbox{$P\vdash x$ or $Q\vdash x$}}{\binpar{P}{Q} \vdash x}
\end{mathpar}

\subsubsection{Contexts}

One of the principle advantages of computational calculi like the
$\pi$-calculus is a well-defined notion of context,
contextual-equivalence and a correlation between
contextual-equivalence and notions of bisimulation. The notion of
context allows the decomposition of a process into (sub-)process and
its syntactic environment, its context. Thus, a context may be
thought of as a process with a ``hole'' (written $\Box$) in it. The
application of a context $M$ to a process $P$, written $M[P]$, is
tantamount to filling the hole in $M$ with $P$. In this paper we do
not need the full weight of this theory, but do make use of the notion
of context in the proof the main theorem. 

\begin{mathpar}
  \inferrule* [lab=summation] {} {{M_{M},M_{N}} \bc \Box \;|\; x.M_{A} \;|\; M_{M}+M_{N}}
  \and
  \inferrule* [lab=agent] {} {{M_{A}} \bc (\vec{x})M_{P} \;| \; \clift{P_0,\ldots,M_{P},\ldots,P_N}}
  \and \\
  \inferrule* [lab=process] {} {{M_{P}} \bc M_{N} \;| \;P|M_{P} }
\end{mathpar} 

\begin{mathpar}
  \inferrule* [lab=sychronization] {} {M_{N} \bc \Box \;|\; x?M_{F} \;|\; x!M_{C}}
  \and
  \inferrule* [lab=abstraction] {} {{M_{F}} \bc (x)M_{P} }
  \and
  \inferrule* [lab=concretion] {} {{M_{C}} \bc \langle M_{P} \rangle }
  \and \\
  \inferrule* [lab=process] {} {{M_{P}} \bc M_{N} \;| \;P|M_{P} }
\end{mathpar}

\begin{definition}[contextual application] Given a context $M$, and
  process $P$, we define the \emph{contextual application}, $M[P] :=
  M\{P/\Box\}$. That is, the contextual application of M to P is the
  substitution of $P$ for $\Box$ in $M$.
\end{definition}

$\meaningof{-} : L \to \mathcal{P}(\pi)$

\begin{mathpar}
  \inferrule* [lab=collection] {} {\meaningof{true} = \pi, \and \meaningof{~E} = \pi \setminus \meaningof{E}, \and \meaningof{E_{1} \& E_{2}} = \meaningof{E_{1}} \cap \meaningof{E_{2}}}
\end{mathpar}

\begin{mathpar}
  \inferrule* [lab=structure] {} {\meaningof{0} = \{ P \in \pi | P \equiv 0 \}, \and \\ \meaningof{E_1 | E_2} = \{ P \in \pi | P \equiv P_{1} | P_{2}, P_{1} \in \meaningof{E_{1}}, P_{2} \in \meaningof{E_2}\} }
\end{mathpar}

\begin{mathpar}
 \inferrule* [lab=behavior] {} {\meaningof{\langle a?b \rangle E} = \{ P \in \pi | P \equiv Q | u?(y)P', \\ \and \\\\ \and \\ \;\;\; u \in \meaningof{a}, \forall z.P'\{z/y\} \in \meaningof{E\{z/b\}}\}, \and \\ \meaningof{a!E} = \{ P \in \pi | P \equiv Q | x!\langle P' \rangle, x \in \meaningof{a} P' \in \meaningof{E}\} }
\end{mathpar}

\begin{mathpar}
 \inferrule* [lab=nominal] {} {\meaningof{\quotep{E}} = \{ \quotep{P} \in \quotep{\pi} | P \in \meaningof{E} \}, \and \meaningof{\quotep{P}} = \{ \quotep{Q} \in \quotep{\pi} | P \equiv Q \} \and \\ \meaningof{@\quotep{E}} = \{ P \in \pi | P \equiv @x, x \in \meaningof{E} \}}
\end{mathpar}

\begin{eqnarray*}
  \\
  \meaningof{-} : TS \to ST
\end{eqnarray*}

\begin{eqnarray*}
  \\
  L : TS \to ST
\end{eqnarray*}

\begin{eqnarray*}
  \\
  P \models E \iff P \in \meaningof{E}
\end{eqnarray*}

\begin{eqnarray*}
  P \approx_{L} Q \iff \forall E \in L. P \models E \iff Q \models E
\end{eqnarray*}

\begin{eqnarray*}
  P \approx_{K} Q
\end{eqnarray*}

\begin{eqnarray*}
  P \approx Q
\end{eqnarray*}

$\approx_{K} = \approx = \approx_{L}$

\subsubsection{Contextual duality}

Note that contexts extend the quotation operation to a family of
operations from processes to names. Given a context, $M$, we can
define a \emph{nominal context}, $\quotep{M}$ by $\quotep{M}[P] :=
\quotep{M[P]}$. To foreshadow what is to come we observe that these
operations enjoy a duality with processes very much like the duality
between vectors and maps from vectors to scalars.

Further, because the calculus is essentially higher-order, we have a
correspondence between contexts and processes. More specifically,
given a name $x$ and a context $M$ we can construct $M^{*}_{x}$ such
that 

\begin{mathpar}
  M^{*}_{x} | \lift{x}{P} \red M[P]
\end{mathpar}

namely,

\begin{mathpar}
  M^{*}_{x} := x?(u).M[\dropn{u}]
\end{mathpar}

The dependence of $M^{*}_{x}$ on a name makes it an abstraction, 

\begin{mathpar}
  M^{*} := (x)x?(u).M[\dropn{u}]
\end{mathpar}

\subsection{Additional notation}

It will sometimes be convenient to denote the process a name
quotes. We already have the notation $x = \quotep{P}$, but it will be
convenient to introduce an alternate notation, $\procn{x}$, when we
want to emphasize the connection to the use of the name. Note that, by
virtue of name equivalence, $\quotep{\procn{x}} \nameeq x$; so, the
notation is consistent with previous definitions.

Further, because names have structure it is possible to effect
substitutions on the basis of that structure. This means we need to
upgrade our notation for substitutions, which we accomplish by
adapting comprehension notation. Thus,

\begin{mathpar}
  P\{ y / x : x \in S \}
\end{mathpar}

is interpreted to mean the process derived from P by replacing (in a
capture-avoiding manner) each occurrence of $x$ in $S$ by $y$. For example,

\begin{mathpar}
  P\{ \quotep{\procn{x}|\procn{x}} / x : x \in \freenames{P} \}
\end{mathpar}

will replace each (occurrence) of a free name $x$ in $P$ by
$\quotep{\procn{x}|\procn{x}}$.

Also, we will avail ourselves of the notation $x^{L}$ and $x^{R}$ to
denote injections of a name into disjoint copies of the name
space. There are numerous ways to accomplish this. One example can be
found in \cite{MeredithR05}. This notation overloads to vectors of
names: $\vec{x}^{\pi} := (x_{i}^{\pi} \; : \; 0 \leq i < |\vec{x}| )$ where $\pi \in \{L,R\}$.

We also use $P^{\Box} := P|\Box$.

In \cite{MeredithR05} an interpretation of the new operator is
given. It turns out that there are several possible interpretations
all enjoying the requisite algebraic properties of the operator (see
\cite{milner91polyadicpi}). We will therefore make liberal use of
$(\nu\; \vec{x})P$.

% subsection the_syntax_and_semantics_of_the_notation_system (end)   

\input{qm2pi.qmops} 

\input{qm2pi.sterngerlach} 

\input{qm2pi.metric} 

% section concurrent_process_calculi (end)

%\input{qm2pi.proofsketch}

% section proof sketch (end)

%\input{qm2pi.slviaknots} 

% section spatial logic via knots (end)

\input{qm2pi.conclusion}

% section conclusion (end)

%\input{qm2pi.dtcodes} 

% section wiring algorithm (end)

\input{qm2pi.ack} 

% section acknowledgments (end)

\newpage


\bibliographystyle{plain}   
\bibliography{../../biblios/main.bib}

\input{qm2pi.rhodetails}

\end{document}

 

%\ifpdf
%\usepackage[pdftex]{graphicx}
%\else
%\usepackage{graphicx}
%\fi

 % \ifpdf
%  \usepackage{pdfsync}
%  \if


%\title{Brief Article}
%\author{David F. Snyder}
%\author{L.G. Meredith}

%\address{Dept. of Math., Texas State University--San Marcos, San Marcos, TX 78666}
       
\pagestyle{empty}


\begin{document}

\lstset{language=[Objective]Caml,frame=shadowbox}

\documentclass[12pt]{llncs}
%\documentclass{jktr}

\usepackage[pdftex]{hyperref}                   
\usepackage {listings}
\usepackage {mathpartir}
\usepackage{bcprules}
%\usepackage{listings}
                       
\usepackage{graphicx} 
%\usepackage[margins=2.5cm,nohead,nofoot]{geometry}
%\usepackage{geometry}
\usepackage{amsfonts}
\usepackage{amstext}
\usepackage{latexsym}
\usepackage{amssymb}
\usepackage{color}


%\include{myPreamble}
\include{qm2pi.local} 

%\ifpdf
%\usepackage[pdftex]{graphicx}
%\else
%\usepackage{graphicx}
%\fi

 % \ifpdf
%  \usepackage{pdfsync}
%  \if


%\title{Brief Article}
%\author{David F. Snyder}
%\author{L.G. Meredith}

%\address{Dept. of Math., Texas State University--San Marcos, San Marcos, TX 78666}
       
\pagestyle{empty}


\begin{document}

\lstset{language=[Objective]Caml,frame=shadowbox}

\input{qm2pi.front}

% section front matter (end)

\input{qm2pi.intro} 
 
% section introduction (end)

% \input{qm2pi.knotations} 

% section notation (end)

\input{qm2pi.process.calculi} 

% section concurrent_process_calculi_and_spatial_logics_ (end)
    
%\input{qm2pi.knots2pi} 

%\input{qm2pi.trefoil} 

%\input{qm2pi.mainthm} 

% subsection basic_interpretation (end)

%\input{qm2pi.rho.presentation} 
\subsection{The syntax and semantics of the notation system}\label{sub:the_syntax_and_semantics_of_the_notation_system} % (fold)

We now summarize a technical presentation of the calculus that
embodies our theory of dynamics. The typical presentation of such a
calculus follows the style of giving generators and relations on
them. The grammar, below, describing term constructors, freely
generates the set of processes, $\Proc$. This set is then quotiented
by a relation known as structural congruence and it is over this set
that the notion of dynamics is expressed. This presentation is
essentially that of \cite{MeredithR05} with the addition of
polyadicity and summation. For readability we have relegated some of
the technical subtleties to an appendix.

\subsubsection{Process grammar}\label{subsub:process_grammar}

\begin{mathpar}
  \inferrule* [lab=synchronization] {} {{M} \bc \pzero \;|\; x?F \;|\; x!C }
  \and
  \inferrule* [lab=abstraction] {} {{F} \bc (x)P}
  \and
  \inferrule* [lab=concretion] {} {{C} \bc \langle Q \rangle}
  \and
  \inferrule* [lab=process] {} {{P,Q} \bc M \;| \;P|Q \;|\; @{x}}
  \and
  \inferrule* [lab=name] {} {{x} \bc \quotep{P}}
\end{mathpar} 

Note that $\vec{x}$ (resp. $\vec{P}$) denotes a vector of names
(resp. processes) of length $|\vec{x}|$ (resp. $|\vec{P}|$). We adopt
the following useful abbreviations.

\begin{mathpar}
   x?(\vec{y}).P := x.(\vec{y})P \and  x\clift{\vec{P}} := x.\clift{\vec{P}}
   \and x!(y) := \lift{x}{\dropn{y}}
   \and \Pi_{i=0}^{n-1}P_i := P_0 | \ldots | P_{n-1}
\end{mathpar}

\subsubsection{Structural congruence}

\paragraph{Free and bound names and alpha-equivalence.} At the
core of structural equivalence is alpha-equivalence which identifies
process that are the same up to a change of variable. Formally, we
recognize the distinction between free and bound names. The free names
of a process, $\freenames{P}$, may be calculated recursively as
follows:

\begin{mathpar}
\freenames{\pzero} := \emptyset
  \and \\
  \freenames{x?(y).P} := \{ x \} \cup (\freenames{P} \setminus \{ y \})
  \and 
  \freenames{x!\langle P \rangle} := \{ x \} \cup \{ P \} 
  \and \\
  \freenames{P|Q} := \freenames{P} \cup \freenames{Q}
  \and \\
  \freenames{@{x}} := \{ x \}
\end{mathpar}

$\pi$
$\quotep{\pi}$

$\freenames{-} : \pi \to \mathcal{P}(\quotep{\pi})$

\begin{eqnarray*}
  \freenames{\pzero} & := & \emptyset \\
  \freenames{x?(y).P} & := & \{ x \} \cup (\freenames{P} \setminus \{ y \}) \\
  \freenames{x!\langle P \rangle} & := & \{ x \} \cup \{ P \} \\
  \freenames{P|Q} & := & \freenames{P} \cup \freenames{Q} \\
  \freenames{\dropn{x}} & := & \{ x \}
\end{eqnarray*}

The bound names of a process, $\boundnames{P}$, are those names occurring in $P$
that are not free. For example, in $x?(y).0$, the name $x$ is free, while $y$ is bound.

\begin{mathpar}
  \inferrule* [lab=monoidal-laws] {} { P|Q \equiv Q|P \and P|0 \equiv P \and P|(Q|R) \equiv (P|Q)|R }
\end{mathpar}

\begin{mathpar}
  \inferrule* [lab=alpha-equivalence] {} { (x)P \equiv (y)P\{y/x\} \and y \not\in \freenames{P} }
\end{mathpar}

\begin{definition}
Then two processes, $P,Q$, are alpha-equivalent if $P = Q\{\vec{y}/\vec{x}\}$ for
some $\vec{x} \in \boundnames{Q},\vec{y} \in \boundnames{P}$, where $Q\{\vec{y}/\vec{x}\}$
denotes the capture-avoiding substitution of $\vec{y}$ for $\vec{x}$ in $Q$.
\end{definition}

\begin{definition}
  The {\em structural congruence} \cite{SangiorgiWalker} , $\equiv$,
  between processes is the least congruence containing
  alpha-equivalence, satisfying the abelian monoid laws
  (associativity, commutativity and $\pzero$ as identity) for parallel
  composition $|$ and for summation $+$.
\end{definition}

\subsection{Name equivalence}

We take name equivalence, written $\nameeq$, to be the smallest
equivalence relation generated by the following rules.

\begin{mathpar}
\inferrule*[lab=Quote-drop]
{ }
{ \quotep{@{x}} \nameeq x }

\inferrule*[lab=Struct-equiv]
{ P \scong Q }
{ \quotep{P} \nameeq \quotep{Q} }
\end{mathpar}

The astute reader will have noticed that the mutual recursion of names
and processes imposes a mutual recursion on alpha-equivalence and
structural equivalence via name-equivalence. Fortunately, all of this
works out pleasantly and we may calculate in the natural way, free of
concern. The reader interested in the details is referred to the
appendix \ref{appendix:rho_details}.

\subsection{Substitution}

We use $\Proc$ for the set of processes, $\QProc$ for the set of
names, and $\id{\{}\vec{y} / \vec{x} \id{\}}$ to denote partial maps,
$s : \QProc \rightarrow \QProc$. A map, $s$ lifts, uniquely, to a map
on process terms, $\widehat{s} : \Proc \rightarrow \Proc$ by the
following equations.

\begin{mathpar}
  (0) \psubstp{Q}{P} := 0 \\
  (R \juxtap S) \psubstp{Q}{P}
  :=    
  (R)\psubstp{Q}{P} \juxtap (S) \psubstp{Q}{P} \\
  (x?(y).R) \psubstp{Q}{P}    
  :=    
  (x)\substp{Q}{P} (z)\concat( (R \psubstn{z}{y}) \psubstp{Q}{P} ) \\
  (\lift{x}{R}) \psubstp{Q}{P}  
  :=
  \lift{(x)\substp{Q}{P}}{ R \psubstp{Q}{P} } \\
%   (\dropn{x})  \psubstp{Q}{P}       
%   := 
%   \left\{ 
%     \begin{array}{ccc} 
%       \dropn{\quotep{Q}} & & x \nameeq \quotep{P} \\
%       \dropn{x} & & otherwise \\
%     \end{array}
%   \right. 
  (\dropn{x})  \psubstp{Q}{P}       
  := 
  \left\{ 
    \begin{array}{ccc} 
      Q & & x \nameeq \quotep{P} \\
      \dropn{x} & & otherwise \\
    \end{array}
  \right.
\end{mathpar}
 

where

\begin{eqnarray}
  (x)\id{\{} \lpquote Q \rpquote / \lpquote P \rpquote \id{\}}            = 
  \left\{ 
    \begin{array}{ccc}
      \lpquote Q \rpquote & & x \nameeq \lpquote P \rpquote \\
      x & & otherwise \\
    \end{array}
  \right. \nonumber
\end{eqnarray}

and $z$ is chosen distinct from $\quotep{P}$, $\quotep{Q}$, the free
names in $Q$, and all the names in $R$. Our $\alpha$-equivalence will
be built in the standard way from this substitution.

\begin{remark}\label{rem:no_self_referential_names}
  One consequence of these definitions is that $\forall P. \quotep{P}
  \not\in \freenames{P}$.
\end{remark}

\subsection{ Dynamic quote: an example }

Anticipating something of what's to come, consider applying the
substitution, $\widehat{\id{\{}u / z \id{\}}}$, to the following pair
of processes, $\lift{w}{y!(z)}$ and $w[ \lpquote y!(z) \rpquote ]$.

\begin{eqnarray}
	\lift{w}{y!(z)}\widehat{\id{\{}u / z \id{\}}}
		& = &
		\lift{w}{y!(u)} \nonumber\\
	w[ \lpquote y!(z) \rpquote ] \widehat{ \id{\{}u / z \id{\}} }
		& = &
		w[ \lpquote y!(z) \rpquote ] \nonumber
\end{eqnarray}

Because the body of the process between quotes is impervious to
substitution, we get radically different answers. In fact, by
examining the first process in an input context,
e.g. $x?(z).\lift{w}{y!(z)}$, we see that the process under the lift
operator may be shaped by prefixed inputs binding a name inside it. In
this sense, the lift operator will be seen as a way to dynamically
construct processes before reifying them as names.

Finally equipped with these standard features we can present the
dynamics of the calculus.

\subsubsection{Operational semantics} 

Finally, we introduce the computational dynamics. What marks these
algebras as distinct from other more traditionally studied algebraic
structures, e.g. vector spaces or polynomial rings, is the manner in
which dynamics is captured. In traditional structures, dynamics is typically
expressed through morphisms between such structures, as in linear maps
between vector spaces or morphisms between rings. In algebras
associated with the semantics of computation, the dynamics is
expressed as part of the algebraic structure itself, through a
reduction reduction relation typically denoted by $\red$. Below, we
give a recursive presentation of this relation for the calculus used
in the encoding.

$\red \subseteq \pi \times \pi$
$\red : \pi \to \mathcal{P}(\pi)$

\begin{mathpar}
  \inferrule* [lab=Comm] { \textsf{match}( x_{src}, x_{trgt} ) } { x_{trgt}?(y)P \; | \; x_{src}!\langle {Q} \rangle \red P\{\quotep{Q}/y}\} }
  \and \\
  \inferrule* [lab=Par] {{P} \red {P}'} {{{P} | {Q}} \red {{P}' | {Q}}}
  \and
  \inferrule* [lab=Equiv]{{{P} \scong {P}'} \andalso {{P}' \red {Q}'} \andalso {{Q}' \scong {Q}}}{{P} \red {Q}}
\end{mathpar}

\begin{eqnarray*}
  match_{\equiv} (\quotep{P},\quotep{Q}) & := & P \equiv Q \\
  match_{\dagger}(\quotep{P},\quotep{Q}) & := & \forall R. P|Q \red^{*} R => R \red^{*} 0 \\
  match_{K}(\quotep{P},\quotep{Q}) & := & K \mbox{ for some context } K
\end{eqnarray*}

$u?(x)P | u!\langle Q \rangle \red P\{\quotep{Q}/x\}$

%We write $\wred$ for $\red^*$, and $P\red$ if $\exists Q $ such that $ P \red Q$.
We write $P\red$ if $\exists Q $ such that $ P \red Q$ and $P\not\red$, otherwise.

\section{Replication}

As mentioned before, it is known that replication (and hence
recursion) can be implemented in a higher-order process algebra
\cite{SangiorgiWalker}. As our first example of calculation with the
machinery thus far presented we give the construction explicitly in
the {\rhoc}.

\begin{eqnarray}
	D_{x} & := & \prefix{x}{y}{(\binpar{\outputp{x}{y}}{@{y}})} \nonumber\\
	\bangp_{x}{P} & := & \binpar{{x}!\langle{\binpar{D_{x}}{P}}\rangle}{D_{x}} \nonumber
\end{eqnarray}

\begin{eqnarray}
	\bangp_{x}{P} & & \nonumber\\
	=
	& {x}!\langle{(\prefix{x}{y}{(\outputp{x}{y} | @{y})) | P}}\rangle 
	      | \prefix{x}{y}{(\outputp{x}{y} | @{y})} & \nonumber\\
	\red
	& (\outputp{x}{y} | @{y})\substn{\quotep{(\prefix{x}{y}{(@{y} | \outputp{x}{y})) | P}}}{y} & \nonumber\\
	=
	& \outputp{x}{\quotep{(\prefix{x}{y}{(\outputp{x}{y} | @{y})) | P}}}
	  | {(\prefix{x}{y}{(\outputp{x}{y} | @{y})) | P}} & \nonumber\\
	\red
	& \ldots & \nonumber\\
	\red^*
	& P | P | \ldots & \nonumber
\end{eqnarray}

Of course, this encoding, as an implementation, runs away, unfolding
$\bangp{P}$ eagerly. A lazier and more implementable replication
operator, restricted to input-guarded processes, may be obtained as follows.

\begin{eqnarray}
\bangp{\prefix{u}{v}{P}} 
	:= 
	\binpar{\lift{x}{\prefix{u}{v}{(\binpar{D(x)}{P})}}}{D(x)} \nonumber
\end{eqnarray}

\begin{remark}
  Note that the lazier definition still does not deal with summation
  or mixed summation (i.e. sums over input and output). The reader is
  invited to construct definitions of replication that deal with these
  features. 

  Further, the definitions are parameterized in a name, $x$. Can you,
  gentle reader, make a definition that eliminates this parameter and
  guarantees no accidental interaction between the replication
  machinery and the process being replicated -- i.e. no accidental
  sharing of names used by the process to get its work done and the
  name(s) used by the replication to effect copying. This latter
  revision of the definition of replication is crucial to obtaining
  the expected identity $!!P \sim !P$.
\end{remark}

\begin{remark}\label{rem:paradoxical_combinator}
  The reader familiar with the lambda calculus will have noticed the
  similarity between $D$ and the paradoxical combinator.

  [Ed. note: the existence of this seems to suggest we have to be more
  restrictive on the set of processes and names we admit if we are to
  support no-cloning.]
\end{remark}

\subsubsection{Bisimulation}

The computational dynamics gives rise to another kind of equivalence,
the equivalence of computational behavior. As previously mentioned
this is typically captured \emph{via} some form of bisimulation.

% The notion we use in this paper is weak barbed bisimulation
% \cite{milner91polyadicpi}.

The notion we use in this paper is derived from weak barbed
bisimulation \cite{milner91polyadicpi}. 

\begin{definition}
An \emph{observation relation}, $\downarrow_{\mathcal N}$, over a set
of names, $\mathcal N$, is the smallest relation satisfying the rules
below.

\infrule[Out-barb]{y \in {\mathcal N}, \; x \nameeq y}
		  {\outputp{x}{v} \downarrow_{\mathcal N} x}
\infrule[Par-barb]{\mbox{$P\downarrow_{\mathcal N} x$ or $Q\downarrow_{\mathcal N} x$}}
		  {\binpar{P}{Q} \downarrow_{\mathcal N} x}

We write $P \Downarrow_{\mathcal N} x$ if there is $Q$ such that 
$P \wred Q$ and $Q \downarrow_{\mathcal N} x$.
\end{definition}

\begin{definition}
%\label{def.bbisim}
An  ${\mathcal N}$-\emph{barbed bisimulation} over a set of names, ${\mathcal N}$, is a symmetric binary relation 
${\mathcal S}_{\mathcal N}$ between agents such that $P\rel{S}_{\mathcal N}Q$ implies:
\begin{enumerate}
\item If $P \red P'$ then $Q \wred Q'$ and $P'\rel{S}_{\mathcal N} Q'$.
\item If $P\downarrow_{\mathcal N} x$, then $Q\Downarrow_{\mathcal N} x$.
\end{enumerate}
$P$ is ${\mathcal N}$-barbed bisimilar to $Q$, written
$P \wbbisim_{\mathcal N} Q$, if $P \rel{S}_{\mathcal N} Q$ for some ${\mathcal N}$-barbed bisimulation ${\mathcal S}_{\mathcal N}$.
\end{definition}

$\mathcal{R} \subseteq \pi \times \pi$

$P \mathcal{R} Q => \forall P'. P \red P' \Rightarrow \exists Q'. Q \red Q', P' \mathcal{R} Q'$

$P \vdash x \Rightarrow Q \vdash x$

\begin{mathpar}
  \inferrule*[lab=Out-barb]{x \nameeq y}{{y}!\langle{Q}\rangle \vdash x}
  \and
  \inferrule*[lab=Par-barb]{\mbox{$P\vdash x$ or $Q\vdash x$}}{\binpar{P}{Q} \vdash x}
\end{mathpar}

\subsubsection{Contexts}

One of the principle advantages of computational calculi like the
$\pi$-calculus is a well-defined notion of context,
contextual-equivalence and a correlation between
contextual-equivalence and notions of bisimulation. The notion of
context allows the decomposition of a process into (sub-)process and
its syntactic environment, its context. Thus, a context may be
thought of as a process with a ``hole'' (written $\Box$) in it. The
application of a context $M$ to a process $P$, written $M[P]$, is
tantamount to filling the hole in $M$ with $P$. In this paper we do
not need the full weight of this theory, but do make use of the notion
of context in the proof the main theorem. 

\begin{mathpar}
  \inferrule* [lab=summation] {} {{M_{M},M_{N}} \bc \Box \;|\; x.M_{A} \;|\; M_{M}+M_{N}}
  \and
  \inferrule* [lab=agent] {} {{M_{A}} \bc (\vec{x})M_{P} \;| \; \clift{P_0,\ldots,M_{P},\ldots,P_N}}
  \and \\
  \inferrule* [lab=process] {} {{M_{P}} \bc M_{N} \;| \;P|M_{P} }
\end{mathpar} 

\begin{mathpar}
  \inferrule* [lab=sychronization] {} {M_{N} \bc \Box \;|\; x?M_{F} \;|\; x!M_{C}}
  \and
  \inferrule* [lab=abstraction] {} {{M_{F}} \bc (x)M_{P} }
  \and
  \inferrule* [lab=concretion] {} {{M_{C}} \bc \langle M_{P} \rangle }
  \and \\
  \inferrule* [lab=process] {} {{M_{P}} \bc M_{N} \;| \;P|M_{P} }
\end{mathpar}

\begin{definition}[contextual application] Given a context $M$, and
  process $P$, we define the \emph{contextual application}, $M[P] :=
  M\{P/\Box\}$. That is, the contextual application of M to P is the
  substitution of $P$ for $\Box$ in $M$.
\end{definition}

$\meaningof{-} : L \to \mathcal{P}(\pi)$

\begin{mathpar}
  \inferrule* [lab=collection] {} {\meaningof{true} = \pi, \and \meaningof{~E} = \pi \setminus \meaningof{E}, \and \meaningof{E_{1} \& E_{2}} = \meaningof{E_{1}} \cap \meaningof{E_{2}}}
\end{mathpar}

\begin{mathpar}
  \inferrule* [lab=structure] {} {\meaningof{0} = \{ P \in \pi | P \equiv 0 \}, \and \\ \meaningof{E_1 | E_2} = \{ P \in \pi | P \equiv P_{1} | P_{2}, P_{1} \in \meaningof{E_{1}}, P_{2} \in \meaningof{E_2}\} }
\end{mathpar}

\begin{mathpar}
 \inferrule* [lab=behavior] {} {\meaningof{\langle a?b \rangle E} = \{ P \in \pi | P \equiv Q | u?(y)P', \\ \and \\\\ \and \\ \;\;\; u \in \meaningof{a}, \forall z.P'\{z/y\} \in \meaningof{E\{z/b\}}\}, \and \\ \meaningof{a!E} = \{ P \in \pi | P \equiv Q | x!\langle P' \rangle, x \in \meaningof{a} P' \in \meaningof{E}\} }
\end{mathpar}

\begin{mathpar}
 \inferrule* [lab=nominal] {} {\meaningof{\quotep{E}} = \{ \quotep{P} \in \quotep{\pi} | P \in \meaningof{E} \}, \and \meaningof{\quotep{P}} = \{ \quotep{Q} \in \quotep{\pi} | P \equiv Q \} \and \\ \meaningof{@\quotep{E}} = \{ P \in \pi | P \equiv @x, x \in \meaningof{E} \}}
\end{mathpar}

\begin{eqnarray*}
  \\
  \meaningof{-} : TS \to ST
\end{eqnarray*}

\begin{eqnarray*}
  \\
  L : TS \to ST
\end{eqnarray*}

\begin{eqnarray*}
  \\
  P \models E \iff P \in \meaningof{E}
\end{eqnarray*}

\begin{eqnarray*}
  P \approx_{L} Q \iff \forall E \in L. P \models E \iff Q \models E
\end{eqnarray*}

\begin{eqnarray*}
  P \approx_{K} Q
\end{eqnarray*}

\begin{eqnarray*}
  P \approx Q
\end{eqnarray*}

$\approx_{K} = \approx = \approx_{L}$

\subsubsection{Contextual duality}

Note that contexts extend the quotation operation to a family of
operations from processes to names. Given a context, $M$, we can
define a \emph{nominal context}, $\quotep{M}$ by $\quotep{M}[P] :=
\quotep{M[P]}$. To foreshadow what is to come we observe that these
operations enjoy a duality with processes very much like the duality
between vectors and maps from vectors to scalars.

Further, because the calculus is essentially higher-order, we have a
correspondence between contexts and processes. More specifically,
given a name $x$ and a context $M$ we can construct $M^{*}_{x}$ such
that 

\begin{mathpar}
  M^{*}_{x} | \lift{x}{P} \red M[P]
\end{mathpar}

namely,

\begin{mathpar}
  M^{*}_{x} := x?(u).M[\dropn{u}]
\end{mathpar}

The dependence of $M^{*}_{x}$ on a name makes it an abstraction, 

\begin{mathpar}
  M^{*} := (x)x?(u).M[\dropn{u}]
\end{mathpar}

\subsection{Additional notation}

It will sometimes be convenient to denote the process a name
quotes. We already have the notation $x = \quotep{P}$, but it will be
convenient to introduce an alternate notation, $\procn{x}$, when we
want to emphasize the connection to the use of the name. Note that, by
virtue of name equivalence, $\quotep{\procn{x}} \nameeq x$; so, the
notation is consistent with previous definitions.

Further, because names have structure it is possible to effect
substitutions on the basis of that structure. This means we need to
upgrade our notation for substitutions, which we accomplish by
adapting comprehension notation. Thus,

\begin{mathpar}
  P\{ y / x : x \in S \}
\end{mathpar}

is interpreted to mean the process derived from P by replacing (in a
capture-avoiding manner) each occurrence of $x$ in $S$ by $y$. For example,

\begin{mathpar}
  P\{ \quotep{\procn{x}|\procn{x}} / x : x \in \freenames{P} \}
\end{mathpar}

will replace each (occurrence) of a free name $x$ in $P$ by
$\quotep{\procn{x}|\procn{x}}$.

Also, we will avail ourselves of the notation $x^{L}$ and $x^{R}$ to
denote injections of a name into disjoint copies of the name
space. There are numerous ways to accomplish this. One example can be
found in \cite{MeredithR05}. This notation overloads to vectors of
names: $\vec{x}^{\pi} := (x_{i}^{\pi} \; : \; 0 \leq i < |\vec{x}| )$ where $\pi \in \{L,R\}$.

We also use $P^{\Box} := P|\Box$.

In \cite{MeredithR05} an interpretation of the new operator is
given. It turns out that there are several possible interpretations
all enjoying the requisite algebraic properties of the operator (see
\cite{milner91polyadicpi}). We will therefore make liberal use of
$(\nu\; \vec{x})P$.

% subsection the_syntax_and_semantics_of_the_notation_system (end)   

\input{qm2pi.qmops} 

\input{qm2pi.sterngerlach} 

\input{qm2pi.metric} 

% section concurrent_process_calculi (end)

%\input{qm2pi.proofsketch}

% section proof sketch (end)

%\input{qm2pi.slviaknots} 

% section spatial logic via knots (end)

\input{qm2pi.conclusion}

% section conclusion (end)

%\input{qm2pi.dtcodes} 

% section wiring algorithm (end)

\input{qm2pi.ack} 

% section acknowledgments (end)

\newpage


\bibliographystyle{plain}   
\bibliography{../../biblios/main.bib}

\input{qm2pi.rhodetails}

\end{document}



% section front matter (end)

\section{Introduction}\label{sec:introduction} % (fold)
In this draft of the material i am going to have to dispense with the
usual writing conventions adopted in papers on these topics. i'm going
to have adopt whatever tone i need at the time i'm writing up the
calculations. Sometimes this may be very conversational; others it may
be the barest mathematical grunts; others still it may be that i have
lifted text from one of my other papers because the exposition of some
point was better said there. i hope that my readers are not unduly put
out by this decision. i'm not doing this to flout convention or be
rebellious. i find these calculations very technically challenging. To
keep everything going technically, something has to give; i have to
let go of some cognitive burden. So, the academic writing style --
with all of its trade-offs in terms of facilitating technical
communication -- is what i'm letting go of. Perhaps subsequent drafts
can be tightened and polished, but for now, i'm going to speak as if
we were sitting together in a coffee shop with a laptop, wifi and a
pad of paper and a pencil.

So, here's what i have to say. We -- you and i, comfortably ensconced
in our coffee shop and well-equipped with our tools -- can realize and
carry out the calculations of quantum mechanics over a very different
formal theory of dynamics, a formal theory of dynamics that
corresponds to a theory of concurrent computation with
\emph{reflection}. It has the advantage that the underlying theory is
already `quantized', but supports analogues all of the continuuous
operations. Strikingly, this underlying theory has recently been
connected with a notion of metric that we can show, by calculating
together, coincides with the metric induced by the inner product.

There are a lot of reasons why you might be interested in seeing
calculations of this form. Here's why i'm interested. For the past
several centuries there has been no competitor to the ``Newtonian''
account of dynamics. As a result the predominant share of accounts of
dynamical systems and situations have had to be formulated in terms of
the Newtonian machinery. i view this as an intellectually dangerous
position to occupy. Everything, despite it's intrinsic shape, turns
into a nail to be hit with this hammer. Recently, however, the theory
of computation has matured to the point where we have candidates for
theories of dynamics that offer very different perspective on
reasoning about dynamical systems and situations. Testing these
candidates against very successful accounts of dynamical situations,
like quantum mechanics, is going to give us some sense of how mature
they are and some measure of the quality of these accounts of
dynamics.

\subsection{Summary of contributions and outline of paper}

So, we're going to develop an interpretation of the operations of
quantum mechanics normally interpreted by Hilbert spaces and
operators. We're going to do this over a theory of computation. Note
that this is very different than the usual quantum computation program
which develops notions of computation over quantum mechanics. Rather,
we are developing a story that aligns with Wheeler's slogan: It from
Bit. To do this we will first provide an account of the theory of
computation at play here. Then we will dive into a calculation-driven
interpretation of the operations of quantum mechanics.

The reason we take this approach is that -- until very recently --
there hasn't been an axiomatic account of quantum mechanics. As a
result there has been no sharp delineation of the mathematical theory
supporting interpretation of the physical theory and the physical
theory, itself. So, ambient features of the maths are free to be
exploited (or supressed) without a real accounting of their physical
relevance. There is no sharp statement ``here's the physical theory''
qua \emph{theory} and ``here's the mathematical interpretation''
enabling a judgment of how faithful the interpretation is -- apart
from experimental observation. When there is an axiomatic account we
can judge how well a given mathematical formalism supports an
interpretation of the axioms, independent of
experimentation. Likewise, we can judge how well we have captured our
physical evidence and experience with our axiomatics, independent of
any specific mathematical implementation, with accidental detail that
may or may not have physical significance. 

In lieu of a fully fleshed out and vetted axiomatic account of quantum
mechanics, interpreting the operational notions in service of modeling
physical systems will have to suffice. In other words, we are not in
the business of providing a model of Hilbert spaces and operators. We
are in the business of providing a model of quantum mechanics because
we are motivated by testing our notions of dynamics against physical
theory; and, the predictive calculations of the physical theory must
serve as the best formulation -- shy of a fully fleshed out axiomatic
account -- of the physical theory itself (as they have for scientific
theories since time immemorial). Put another way, despite a
whole-hearted commitment to an It-from-Bit ontology, we are firmly
aligned with the shut-up-and-calculate camp as the best way to obtain
results either from the physical perspective or as a quality assurance
measure of our fledgling theory of dynamics.

In detail, we present a reflective process calculus. Then we develop
intuitive correspondences between the notions available in this
calculus and the usual physical notions supporting quantum mechanical
calculations. Thus, 

\begin{table}[htp]
  \center{
    \fbox{
      \begin{tabular}{c|c}
        quantum mechanics & process calculus \\
        \hline
        scalar & name \\
        state vector & process \\
        dual & contextual duals \\
        matrix & formal sums of process-context-dual pairs \\
        orthogonality & process annihilation \\
        inner product & execution-formula + quoting
      \end{tabular}
    }
  }
  \caption{QM - process calculi correspondences}
\end{table}

Then we tighten up these intuitions to operational definitions. We
employ the Dirac notation as the best proxy we can find for an
abstract syntax of the quantum mechanical notions. The definitions we
develop put us in contact with equational constraints coming from the
theory that we demonstrate the definitions and calculations satisfy.

This puts us in a position to shut up and calculate for the
Stern-Gerlach experimental set up, showing how these predictive
calculations become calculations on processes in our theory of a
reflective process calculus.

Penultimately, we demonstrate that the notion of metric coming from
the inner product coincides with the notion of metric available from
the theory of bisimulation. This demonstration gives us the right to
think of space as arising from behavior. Finally, we consider where we
might go from the new vantage point we have obtained.

% section introduction (end) 
 
% section introduction (end)

% \documentclass[12pt]{llncs}
%\documentclass{jktr}

\usepackage[pdftex]{hyperref}                   
\usepackage {listings}
\usepackage {mathpartir}
\usepackage{bcprules}
%\usepackage{listings}
                       
\usepackage{graphicx} 
%\usepackage[margins=2.5cm,nohead,nofoot]{geometry}
%\usepackage{geometry}
\usepackage{amsfonts}
\usepackage{amstext}
\usepackage{latexsym}
\usepackage{amssymb}
\usepackage{color}


%\include{myPreamble}
\include{qm2pi.local} 

%\ifpdf
%\usepackage[pdftex]{graphicx}
%\else
%\usepackage{graphicx}
%\fi

 % \ifpdf
%  \usepackage{pdfsync}
%  \if


%\title{Brief Article}
%\author{David F. Snyder}
%\author{L.G. Meredith}

%\address{Dept. of Math., Texas State University--San Marcos, San Marcos, TX 78666}
       
\pagestyle{empty}


\begin{document}

\lstset{language=[Objective]Caml,frame=shadowbox}

\input{qm2pi.front}

% section front matter (end)

\input{qm2pi.intro} 
 
% section introduction (end)

% \input{qm2pi.knotations} 

% section notation (end)

\input{qm2pi.process.calculi} 

% section concurrent_process_calculi_and_spatial_logics_ (end)
    
%\input{qm2pi.knots2pi} 

%\input{qm2pi.trefoil} 

%\input{qm2pi.mainthm} 

% subsection basic_interpretation (end)

%\input{qm2pi.rho.presentation} 
\subsection{The syntax and semantics of the notation system}\label{sub:the_syntax_and_semantics_of_the_notation_system} % (fold)

We now summarize a technical presentation of the calculus that
embodies our theory of dynamics. The typical presentation of such a
calculus follows the style of giving generators and relations on
them. The grammar, below, describing term constructors, freely
generates the set of processes, $\Proc$. This set is then quotiented
by a relation known as structural congruence and it is over this set
that the notion of dynamics is expressed. This presentation is
essentially that of \cite{MeredithR05} with the addition of
polyadicity and summation. For readability we have relegated some of
the technical subtleties to an appendix.

\subsubsection{Process grammar}\label{subsub:process_grammar}

\begin{mathpar}
  \inferrule* [lab=synchronization] {} {{M} \bc \pzero \;|\; x?F \;|\; x!C }
  \and
  \inferrule* [lab=abstraction] {} {{F} \bc (x)P}
  \and
  \inferrule* [lab=concretion] {} {{C} \bc \langle Q \rangle}
  \and
  \inferrule* [lab=process] {} {{P,Q} \bc M \;| \;P|Q \;|\; @{x}}
  \and
  \inferrule* [lab=name] {} {{x} \bc \quotep{P}}
\end{mathpar} 

Note that $\vec{x}$ (resp. $\vec{P}$) denotes a vector of names
(resp. processes) of length $|\vec{x}|$ (resp. $|\vec{P}|$). We adopt
the following useful abbreviations.

\begin{mathpar}
   x?(\vec{y}).P := x.(\vec{y})P \and  x\clift{\vec{P}} := x.\clift{\vec{P}}
   \and x!(y) := \lift{x}{\dropn{y}}
   \and \Pi_{i=0}^{n-1}P_i := P_0 | \ldots | P_{n-1}
\end{mathpar}

\subsubsection{Structural congruence}

\paragraph{Free and bound names and alpha-equivalence.} At the
core of structural equivalence is alpha-equivalence which identifies
process that are the same up to a change of variable. Formally, we
recognize the distinction between free and bound names. The free names
of a process, $\freenames{P}$, may be calculated recursively as
follows:

\begin{mathpar}
\freenames{\pzero} := \emptyset
  \and \\
  \freenames{x?(y).P} := \{ x \} \cup (\freenames{P} \setminus \{ y \})
  \and 
  \freenames{x!\langle P \rangle} := \{ x \} \cup \{ P \} 
  \and \\
  \freenames{P|Q} := \freenames{P} \cup \freenames{Q}
  \and \\
  \freenames{@{x}} := \{ x \}
\end{mathpar}

$\pi$
$\quotep{\pi}$

$\freenames{-} : \pi \to \mathcal{P}(\quotep{\pi})$

\begin{eqnarray*}
  \freenames{\pzero} & := & \emptyset \\
  \freenames{x?(y).P} & := & \{ x \} \cup (\freenames{P} \setminus \{ y \}) \\
  \freenames{x!\langle P \rangle} & := & \{ x \} \cup \{ P \} \\
  \freenames{P|Q} & := & \freenames{P} \cup \freenames{Q} \\
  \freenames{\dropn{x}} & := & \{ x \}
\end{eqnarray*}

The bound names of a process, $\boundnames{P}$, are those names occurring in $P$
that are not free. For example, in $x?(y).0$, the name $x$ is free, while $y$ is bound.

\begin{mathpar}
  \inferrule* [lab=monoidal-laws] {} { P|Q \equiv Q|P \and P|0 \equiv P \and P|(Q|R) \equiv (P|Q)|R }
\end{mathpar}

\begin{mathpar}
  \inferrule* [lab=alpha-equivalence] {} { (x)P \equiv (y)P\{y/x\} \and y \not\in \freenames{P} }
\end{mathpar}

\begin{definition}
Then two processes, $P,Q$, are alpha-equivalent if $P = Q\{\vec{y}/\vec{x}\}$ for
some $\vec{x} \in \boundnames{Q},\vec{y} \in \boundnames{P}$, where $Q\{\vec{y}/\vec{x}\}$
denotes the capture-avoiding substitution of $\vec{y}$ for $\vec{x}$ in $Q$.
\end{definition}

\begin{definition}
  The {\em structural congruence} \cite{SangiorgiWalker} , $\equiv$,
  between processes is the least congruence containing
  alpha-equivalence, satisfying the abelian monoid laws
  (associativity, commutativity and $\pzero$ as identity) for parallel
  composition $|$ and for summation $+$.
\end{definition}

\subsection{Name equivalence}

We take name equivalence, written $\nameeq$, to be the smallest
equivalence relation generated by the following rules.

\begin{mathpar}
\inferrule*[lab=Quote-drop]
{ }
{ \quotep{@{x}} \nameeq x }

\inferrule*[lab=Struct-equiv]
{ P \scong Q }
{ \quotep{P} \nameeq \quotep{Q} }
\end{mathpar}

The astute reader will have noticed that the mutual recursion of names
and processes imposes a mutual recursion on alpha-equivalence and
structural equivalence via name-equivalence. Fortunately, all of this
works out pleasantly and we may calculate in the natural way, free of
concern. The reader interested in the details is referred to the
appendix \ref{appendix:rho_details}.

\subsection{Substitution}

We use $\Proc$ for the set of processes, $\QProc$ for the set of
names, and $\id{\{}\vec{y} / \vec{x} \id{\}}$ to denote partial maps,
$s : \QProc \rightarrow \QProc$. A map, $s$ lifts, uniquely, to a map
on process terms, $\widehat{s} : \Proc \rightarrow \Proc$ by the
following equations.

\begin{mathpar}
  (0) \psubstp{Q}{P} := 0 \\
  (R \juxtap S) \psubstp{Q}{P}
  :=    
  (R)\psubstp{Q}{P} \juxtap (S) \psubstp{Q}{P} \\
  (x?(y).R) \psubstp{Q}{P}    
  :=    
  (x)\substp{Q}{P} (z)\concat( (R \psubstn{z}{y}) \psubstp{Q}{P} ) \\
  (\lift{x}{R}) \psubstp{Q}{P}  
  :=
  \lift{(x)\substp{Q}{P}}{ R \psubstp{Q}{P} } \\
%   (\dropn{x})  \psubstp{Q}{P}       
%   := 
%   \left\{ 
%     \begin{array}{ccc} 
%       \dropn{\quotep{Q}} & & x \nameeq \quotep{P} \\
%       \dropn{x} & & otherwise \\
%     \end{array}
%   \right. 
  (\dropn{x})  \psubstp{Q}{P}       
  := 
  \left\{ 
    \begin{array}{ccc} 
      Q & & x \nameeq \quotep{P} \\
      \dropn{x} & & otherwise \\
    \end{array}
  \right.
\end{mathpar}
 

where

\begin{eqnarray}
  (x)\id{\{} \lpquote Q \rpquote / \lpquote P \rpquote \id{\}}            = 
  \left\{ 
    \begin{array}{ccc}
      \lpquote Q \rpquote & & x \nameeq \lpquote P \rpquote \\
      x & & otherwise \\
    \end{array}
  \right. \nonumber
\end{eqnarray}

and $z$ is chosen distinct from $\quotep{P}$, $\quotep{Q}$, the free
names in $Q$, and all the names in $R$. Our $\alpha$-equivalence will
be built in the standard way from this substitution.

\begin{remark}\label{rem:no_self_referential_names}
  One consequence of these definitions is that $\forall P. \quotep{P}
  \not\in \freenames{P}$.
\end{remark}

\subsection{ Dynamic quote: an example }

Anticipating something of what's to come, consider applying the
substitution, $\widehat{\id{\{}u / z \id{\}}}$, to the following pair
of processes, $\lift{w}{y!(z)}$ and $w[ \lpquote y!(z) \rpquote ]$.

\begin{eqnarray}
	\lift{w}{y!(z)}\widehat{\id{\{}u / z \id{\}}}
		& = &
		\lift{w}{y!(u)} \nonumber\\
	w[ \lpquote y!(z) \rpquote ] \widehat{ \id{\{}u / z \id{\}} }
		& = &
		w[ \lpquote y!(z) \rpquote ] \nonumber
\end{eqnarray}

Because the body of the process between quotes is impervious to
substitution, we get radically different answers. In fact, by
examining the first process in an input context,
e.g. $x?(z).\lift{w}{y!(z)}$, we see that the process under the lift
operator may be shaped by prefixed inputs binding a name inside it. In
this sense, the lift operator will be seen as a way to dynamically
construct processes before reifying them as names.

Finally equipped with these standard features we can present the
dynamics of the calculus.

\subsubsection{Operational semantics} 

Finally, we introduce the computational dynamics. What marks these
algebras as distinct from other more traditionally studied algebraic
structures, e.g. vector spaces or polynomial rings, is the manner in
which dynamics is captured. In traditional structures, dynamics is typically
expressed through morphisms between such structures, as in linear maps
between vector spaces or morphisms between rings. In algebras
associated with the semantics of computation, the dynamics is
expressed as part of the algebraic structure itself, through a
reduction reduction relation typically denoted by $\red$. Below, we
give a recursive presentation of this relation for the calculus used
in the encoding.

$\red \subseteq \pi \times \pi$
$\red : \pi \to \mathcal{P}(\pi)$

\begin{mathpar}
  \inferrule* [lab=Comm] { \textsf{match}( x_{src}, x_{trgt} ) } { x_{trgt}?(y)P \; | \; x_{src}!\langle {Q} \rangle \red P\{\quotep{Q}/y}\} }
  \and \\
  \inferrule* [lab=Par] {{P} \red {P}'} {{{P} | {Q}} \red {{P}' | {Q}}}
  \and
  \inferrule* [lab=Equiv]{{{P} \scong {P}'} \andalso {{P}' \red {Q}'} \andalso {{Q}' \scong {Q}}}{{P} \red {Q}}
\end{mathpar}

\begin{eqnarray*}
  match_{\equiv} (\quotep{P},\quotep{Q}) & := & P \equiv Q \\
  match_{\dagger}(\quotep{P},\quotep{Q}) & := & \forall R. P|Q \red^{*} R => R \red^{*} 0 \\
  match_{K}(\quotep{P},\quotep{Q}) & := & K \mbox{ for some context } K
\end{eqnarray*}

$u?(x)P | u!\langle Q \rangle \red P\{\quotep{Q}/x\}$

%We write $\wred$ for $\red^*$, and $P\red$ if $\exists Q $ such that $ P \red Q$.
We write $P\red$ if $\exists Q $ such that $ P \red Q$ and $P\not\red$, otherwise.

\section{Replication}

As mentioned before, it is known that replication (and hence
recursion) can be implemented in a higher-order process algebra
\cite{SangiorgiWalker}. As our first example of calculation with the
machinery thus far presented we give the construction explicitly in
the {\rhoc}.

\begin{eqnarray}
	D_{x} & := & \prefix{x}{y}{(\binpar{\outputp{x}{y}}{@{y}})} \nonumber\\
	\bangp_{x}{P} & := & \binpar{{x}!\langle{\binpar{D_{x}}{P}}\rangle}{D_{x}} \nonumber
\end{eqnarray}

\begin{eqnarray}
	\bangp_{x}{P} & & \nonumber\\
	=
	& {x}!\langle{(\prefix{x}{y}{(\outputp{x}{y} | @{y})) | P}}\rangle 
	      | \prefix{x}{y}{(\outputp{x}{y} | @{y})} & \nonumber\\
	\red
	& (\outputp{x}{y} | @{y})\substn{\quotep{(\prefix{x}{y}{(@{y} | \outputp{x}{y})) | P}}}{y} & \nonumber\\
	=
	& \outputp{x}{\quotep{(\prefix{x}{y}{(\outputp{x}{y} | @{y})) | P}}}
	  | {(\prefix{x}{y}{(\outputp{x}{y} | @{y})) | P}} & \nonumber\\
	\red
	& \ldots & \nonumber\\
	\red^*
	& P | P | \ldots & \nonumber
\end{eqnarray}

Of course, this encoding, as an implementation, runs away, unfolding
$\bangp{P}$ eagerly. A lazier and more implementable replication
operator, restricted to input-guarded processes, may be obtained as follows.

\begin{eqnarray}
\bangp{\prefix{u}{v}{P}} 
	:= 
	\binpar{\lift{x}{\prefix{u}{v}{(\binpar{D(x)}{P})}}}{D(x)} \nonumber
\end{eqnarray}

\begin{remark}
  Note that the lazier definition still does not deal with summation
  or mixed summation (i.e. sums over input and output). The reader is
  invited to construct definitions of replication that deal with these
  features. 

  Further, the definitions are parameterized in a name, $x$. Can you,
  gentle reader, make a definition that eliminates this parameter and
  guarantees no accidental interaction between the replication
  machinery and the process being replicated -- i.e. no accidental
  sharing of names used by the process to get its work done and the
  name(s) used by the replication to effect copying. This latter
  revision of the definition of replication is crucial to obtaining
  the expected identity $!!P \sim !P$.
\end{remark}

\begin{remark}\label{rem:paradoxical_combinator}
  The reader familiar with the lambda calculus will have noticed the
  similarity between $D$ and the paradoxical combinator.

  [Ed. note: the existence of this seems to suggest we have to be more
  restrictive on the set of processes and names we admit if we are to
  support no-cloning.]
\end{remark}

\subsubsection{Bisimulation}

The computational dynamics gives rise to another kind of equivalence,
the equivalence of computational behavior. As previously mentioned
this is typically captured \emph{via} some form of bisimulation.

% The notion we use in this paper is weak barbed bisimulation
% \cite{milner91polyadicpi}.

The notion we use in this paper is derived from weak barbed
bisimulation \cite{milner91polyadicpi}. 

\begin{definition}
An \emph{observation relation}, $\downarrow_{\mathcal N}$, over a set
of names, $\mathcal N$, is the smallest relation satisfying the rules
below.

\infrule[Out-barb]{y \in {\mathcal N}, \; x \nameeq y}
		  {\outputp{x}{v} \downarrow_{\mathcal N} x}
\infrule[Par-barb]{\mbox{$P\downarrow_{\mathcal N} x$ or $Q\downarrow_{\mathcal N} x$}}
		  {\binpar{P}{Q} \downarrow_{\mathcal N} x}

We write $P \Downarrow_{\mathcal N} x$ if there is $Q$ such that 
$P \wred Q$ and $Q \downarrow_{\mathcal N} x$.
\end{definition}

\begin{definition}
%\label{def.bbisim}
An  ${\mathcal N}$-\emph{barbed bisimulation} over a set of names, ${\mathcal N}$, is a symmetric binary relation 
${\mathcal S}_{\mathcal N}$ between agents such that $P\rel{S}_{\mathcal N}Q$ implies:
\begin{enumerate}
\item If $P \red P'$ then $Q \wred Q'$ and $P'\rel{S}_{\mathcal N} Q'$.
\item If $P\downarrow_{\mathcal N} x$, then $Q\Downarrow_{\mathcal N} x$.
\end{enumerate}
$P$ is ${\mathcal N}$-barbed bisimilar to $Q$, written
$P \wbbisim_{\mathcal N} Q$, if $P \rel{S}_{\mathcal N} Q$ for some ${\mathcal N}$-barbed bisimulation ${\mathcal S}_{\mathcal N}$.
\end{definition}

$\mathcal{R} \subseteq \pi \times \pi$

$P \mathcal{R} Q => \forall P'. P \red P' \Rightarrow \exists Q'. Q \red Q', P' \mathcal{R} Q'$

$P \vdash x \Rightarrow Q \vdash x$

\begin{mathpar}
  \inferrule*[lab=Out-barb]{x \nameeq y}{{y}!\langle{Q}\rangle \vdash x}
  \and
  \inferrule*[lab=Par-barb]{\mbox{$P\vdash x$ or $Q\vdash x$}}{\binpar{P}{Q} \vdash x}
\end{mathpar}

\subsubsection{Contexts}

One of the principle advantages of computational calculi like the
$\pi$-calculus is a well-defined notion of context,
contextual-equivalence and a correlation between
contextual-equivalence and notions of bisimulation. The notion of
context allows the decomposition of a process into (sub-)process and
its syntactic environment, its context. Thus, a context may be
thought of as a process with a ``hole'' (written $\Box$) in it. The
application of a context $M$ to a process $P$, written $M[P]$, is
tantamount to filling the hole in $M$ with $P$. In this paper we do
not need the full weight of this theory, but do make use of the notion
of context in the proof the main theorem. 

\begin{mathpar}
  \inferrule* [lab=summation] {} {{M_{M},M_{N}} \bc \Box \;|\; x.M_{A} \;|\; M_{M}+M_{N}}
  \and
  \inferrule* [lab=agent] {} {{M_{A}} \bc (\vec{x})M_{P} \;| \; \clift{P_0,\ldots,M_{P},\ldots,P_N}}
  \and \\
  \inferrule* [lab=process] {} {{M_{P}} \bc M_{N} \;| \;P|M_{P} }
\end{mathpar} 

\begin{mathpar}
  \inferrule* [lab=sychronization] {} {M_{N} \bc \Box \;|\; x?M_{F} \;|\; x!M_{C}}
  \and
  \inferrule* [lab=abstraction] {} {{M_{F}} \bc (x)M_{P} }
  \and
  \inferrule* [lab=concretion] {} {{M_{C}} \bc \langle M_{P} \rangle }
  \and \\
  \inferrule* [lab=process] {} {{M_{P}} \bc M_{N} \;| \;P|M_{P} }
\end{mathpar}

\begin{definition}[contextual application] Given a context $M$, and
  process $P$, we define the \emph{contextual application}, $M[P] :=
  M\{P/\Box\}$. That is, the contextual application of M to P is the
  substitution of $P$ for $\Box$ in $M$.
\end{definition}

$\meaningof{-} : L \to \mathcal{P}(\pi)$

\begin{mathpar}
  \inferrule* [lab=collection] {} {\meaningof{true} = \pi, \and \meaningof{~E} = \pi \setminus \meaningof{E}, \and \meaningof{E_{1} \& E_{2}} = \meaningof{E_{1}} \cap \meaningof{E_{2}}}
\end{mathpar}

\begin{mathpar}
  \inferrule* [lab=structure] {} {\meaningof{0} = \{ P \in \pi | P \equiv 0 \}, \and \\ \meaningof{E_1 | E_2} = \{ P \in \pi | P \equiv P_{1} | P_{2}, P_{1} \in \meaningof{E_{1}}, P_{2} \in \meaningof{E_2}\} }
\end{mathpar}

\begin{mathpar}
 \inferrule* [lab=behavior] {} {\meaningof{\langle a?b \rangle E} = \{ P \in \pi | P \equiv Q | u?(y)P', \\ \and \\\\ \and \\ \;\;\; u \in \meaningof{a}, \forall z.P'\{z/y\} \in \meaningof{E\{z/b\}}\}, \and \\ \meaningof{a!E} = \{ P \in \pi | P \equiv Q | x!\langle P' \rangle, x \in \meaningof{a} P' \in \meaningof{E}\} }
\end{mathpar}

\begin{mathpar}
 \inferrule* [lab=nominal] {} {\meaningof{\quotep{E}} = \{ \quotep{P} \in \quotep{\pi} | P \in \meaningof{E} \}, \and \meaningof{\quotep{P}} = \{ \quotep{Q} \in \quotep{\pi} | P \equiv Q \} \and \\ \meaningof{@\quotep{E}} = \{ P \in \pi | P \equiv @x, x \in \meaningof{E} \}}
\end{mathpar}

\begin{eqnarray*}
  \\
  \meaningof{-} : TS \to ST
\end{eqnarray*}

\begin{eqnarray*}
  \\
  L : TS \to ST
\end{eqnarray*}

\begin{eqnarray*}
  \\
  P \models E \iff P \in \meaningof{E}
\end{eqnarray*}

\begin{eqnarray*}
  P \approx_{L} Q \iff \forall E \in L. P \models E \iff Q \models E
\end{eqnarray*}

\begin{eqnarray*}
  P \approx_{K} Q
\end{eqnarray*}

\begin{eqnarray*}
  P \approx Q
\end{eqnarray*}

$\approx_{K} = \approx = \approx_{L}$

\subsubsection{Contextual duality}

Note that contexts extend the quotation operation to a family of
operations from processes to names. Given a context, $M$, we can
define a \emph{nominal context}, $\quotep{M}$ by $\quotep{M}[P] :=
\quotep{M[P]}$. To foreshadow what is to come we observe that these
operations enjoy a duality with processes very much like the duality
between vectors and maps from vectors to scalars.

Further, because the calculus is essentially higher-order, we have a
correspondence between contexts and processes. More specifically,
given a name $x$ and a context $M$ we can construct $M^{*}_{x}$ such
that 

\begin{mathpar}
  M^{*}_{x} | \lift{x}{P} \red M[P]
\end{mathpar}

namely,

\begin{mathpar}
  M^{*}_{x} := x?(u).M[\dropn{u}]
\end{mathpar}

The dependence of $M^{*}_{x}$ on a name makes it an abstraction, 

\begin{mathpar}
  M^{*} := (x)x?(u).M[\dropn{u}]
\end{mathpar}

\subsection{Additional notation}

It will sometimes be convenient to denote the process a name
quotes. We already have the notation $x = \quotep{P}$, but it will be
convenient to introduce an alternate notation, $\procn{x}$, when we
want to emphasize the connection to the use of the name. Note that, by
virtue of name equivalence, $\quotep{\procn{x}} \nameeq x$; so, the
notation is consistent with previous definitions.

Further, because names have structure it is possible to effect
substitutions on the basis of that structure. This means we need to
upgrade our notation for substitutions, which we accomplish by
adapting comprehension notation. Thus,

\begin{mathpar}
  P\{ y / x : x \in S \}
\end{mathpar}

is interpreted to mean the process derived from P by replacing (in a
capture-avoiding manner) each occurrence of $x$ in $S$ by $y$. For example,

\begin{mathpar}
  P\{ \quotep{\procn{x}|\procn{x}} / x : x \in \freenames{P} \}
\end{mathpar}

will replace each (occurrence) of a free name $x$ in $P$ by
$\quotep{\procn{x}|\procn{x}}$.

Also, we will avail ourselves of the notation $x^{L}$ and $x^{R}$ to
denote injections of a name into disjoint copies of the name
space. There are numerous ways to accomplish this. One example can be
found in \cite{MeredithR05}. This notation overloads to vectors of
names: $\vec{x}^{\pi} := (x_{i}^{\pi} \; : \; 0 \leq i < |\vec{x}| )$ where $\pi \in \{L,R\}$.

We also use $P^{\Box} := P|\Box$.

In \cite{MeredithR05} an interpretation of the new operator is
given. It turns out that there are several possible interpretations
all enjoying the requisite algebraic properties of the operator (see
\cite{milner91polyadicpi}). We will therefore make liberal use of
$(\nu\; \vec{x})P$.

% subsection the_syntax_and_semantics_of_the_notation_system (end)   

\input{qm2pi.qmops} 

\input{qm2pi.sterngerlach} 

\input{qm2pi.metric} 

% section concurrent_process_calculi (end)

%\input{qm2pi.proofsketch}

% section proof sketch (end)

%\input{qm2pi.slviaknots} 

% section spatial logic via knots (end)

\input{qm2pi.conclusion}

% section conclusion (end)

%\input{qm2pi.dtcodes} 

% section wiring algorithm (end)

\input{qm2pi.ack} 

% section acknowledgments (end)

\newpage


\bibliographystyle{plain}   
\bibliography{../../biblios/main.bib}

\input{qm2pi.rhodetails}

\end{document}

 

% section notation (end)

\input{qm2pi.process.calculi} 

% section concurrent_process_calculi_and_spatial_logics_ (end)
    
%\documentclass[12pt]{llncs}
%\documentclass{jktr}

\usepackage[pdftex]{hyperref}                   
\usepackage {listings}
\usepackage {mathpartir}
\usepackage{bcprules}
%\usepackage{listings}
                       
\usepackage{graphicx} 
%\usepackage[margins=2.5cm,nohead,nofoot]{geometry}
%\usepackage{geometry}
\usepackage{amsfonts}
\usepackage{amstext}
\usepackage{latexsym}
\usepackage{amssymb}
\usepackage{color}


%\include{myPreamble}
\include{qm2pi.local} 

%\ifpdf
%\usepackage[pdftex]{graphicx}
%\else
%\usepackage{graphicx}
%\fi

 % \ifpdf
%  \usepackage{pdfsync}
%  \if


%\title{Brief Article}
%\author{David F. Snyder}
%\author{L.G. Meredith}

%\address{Dept. of Math., Texas State University--San Marcos, San Marcos, TX 78666}
       
\pagestyle{empty}


\begin{document}

\lstset{language=[Objective]Caml,frame=shadowbox}

\input{qm2pi.front}

% section front matter (end)

\input{qm2pi.intro} 
 
% section introduction (end)

% \input{qm2pi.knotations} 

% section notation (end)

\input{qm2pi.process.calculi} 

% section concurrent_process_calculi_and_spatial_logics_ (end)
    
%\input{qm2pi.knots2pi} 

%\input{qm2pi.trefoil} 

%\input{qm2pi.mainthm} 

% subsection basic_interpretation (end)

%\input{qm2pi.rho.presentation} 
\subsection{The syntax and semantics of the notation system}\label{sub:the_syntax_and_semantics_of_the_notation_system} % (fold)

We now summarize a technical presentation of the calculus that
embodies our theory of dynamics. The typical presentation of such a
calculus follows the style of giving generators and relations on
them. The grammar, below, describing term constructors, freely
generates the set of processes, $\Proc$. This set is then quotiented
by a relation known as structural congruence and it is over this set
that the notion of dynamics is expressed. This presentation is
essentially that of \cite{MeredithR05} with the addition of
polyadicity and summation. For readability we have relegated some of
the technical subtleties to an appendix.

\subsubsection{Process grammar}\label{subsub:process_grammar}

\begin{mathpar}
  \inferrule* [lab=synchronization] {} {{M} \bc \pzero \;|\; x?F \;|\; x!C }
  \and
  \inferrule* [lab=abstraction] {} {{F} \bc (x)P}
  \and
  \inferrule* [lab=concretion] {} {{C} \bc \langle Q \rangle}
  \and
  \inferrule* [lab=process] {} {{P,Q} \bc M \;| \;P|Q \;|\; @{x}}
  \and
  \inferrule* [lab=name] {} {{x} \bc \quotep{P}}
\end{mathpar} 

Note that $\vec{x}$ (resp. $\vec{P}$) denotes a vector of names
(resp. processes) of length $|\vec{x}|$ (resp. $|\vec{P}|$). We adopt
the following useful abbreviations.

\begin{mathpar}
   x?(\vec{y}).P := x.(\vec{y})P \and  x\clift{\vec{P}} := x.\clift{\vec{P}}
   \and x!(y) := \lift{x}{\dropn{y}}
   \and \Pi_{i=0}^{n-1}P_i := P_0 | \ldots | P_{n-1}
\end{mathpar}

\subsubsection{Structural congruence}

\paragraph{Free and bound names and alpha-equivalence.} At the
core of structural equivalence is alpha-equivalence which identifies
process that are the same up to a change of variable. Formally, we
recognize the distinction between free and bound names. The free names
of a process, $\freenames{P}$, may be calculated recursively as
follows:

\begin{mathpar}
\freenames{\pzero} := \emptyset
  \and \\
  \freenames{x?(y).P} := \{ x \} \cup (\freenames{P} \setminus \{ y \})
  \and 
  \freenames{x!\langle P \rangle} := \{ x \} \cup \{ P \} 
  \and \\
  \freenames{P|Q} := \freenames{P} \cup \freenames{Q}
  \and \\
  \freenames{@{x}} := \{ x \}
\end{mathpar}

$\pi$
$\quotep{\pi}$

$\freenames{-} : \pi \to \mathcal{P}(\quotep{\pi})$

\begin{eqnarray*}
  \freenames{\pzero} & := & \emptyset \\
  \freenames{x?(y).P} & := & \{ x \} \cup (\freenames{P} \setminus \{ y \}) \\
  \freenames{x!\langle P \rangle} & := & \{ x \} \cup \{ P \} \\
  \freenames{P|Q} & := & \freenames{P} \cup \freenames{Q} \\
  \freenames{\dropn{x}} & := & \{ x \}
\end{eqnarray*}

The bound names of a process, $\boundnames{P}$, are those names occurring in $P$
that are not free. For example, in $x?(y).0$, the name $x$ is free, while $y$ is bound.

\begin{mathpar}
  \inferrule* [lab=monoidal-laws] {} { P|Q \equiv Q|P \and P|0 \equiv P \and P|(Q|R) \equiv (P|Q)|R }
\end{mathpar}

\begin{mathpar}
  \inferrule* [lab=alpha-equivalence] {} { (x)P \equiv (y)P\{y/x\} \and y \not\in \freenames{P} }
\end{mathpar}

\begin{definition}
Then two processes, $P,Q$, are alpha-equivalent if $P = Q\{\vec{y}/\vec{x}\}$ for
some $\vec{x} \in \boundnames{Q},\vec{y} \in \boundnames{P}$, where $Q\{\vec{y}/\vec{x}\}$
denotes the capture-avoiding substitution of $\vec{y}$ for $\vec{x}$ in $Q$.
\end{definition}

\begin{definition}
  The {\em structural congruence} \cite{SangiorgiWalker} , $\equiv$,
  between processes is the least congruence containing
  alpha-equivalence, satisfying the abelian monoid laws
  (associativity, commutativity and $\pzero$ as identity) for parallel
  composition $|$ and for summation $+$.
\end{definition}

\subsection{Name equivalence}

We take name equivalence, written $\nameeq$, to be the smallest
equivalence relation generated by the following rules.

\begin{mathpar}
\inferrule*[lab=Quote-drop]
{ }
{ \quotep{@{x}} \nameeq x }

\inferrule*[lab=Struct-equiv]
{ P \scong Q }
{ \quotep{P} \nameeq \quotep{Q} }
\end{mathpar}

The astute reader will have noticed that the mutual recursion of names
and processes imposes a mutual recursion on alpha-equivalence and
structural equivalence via name-equivalence. Fortunately, all of this
works out pleasantly and we may calculate in the natural way, free of
concern. The reader interested in the details is referred to the
appendix \ref{appendix:rho_details}.

\subsection{Substitution}

We use $\Proc$ for the set of processes, $\QProc$ for the set of
names, and $\id{\{}\vec{y} / \vec{x} \id{\}}$ to denote partial maps,
$s : \QProc \rightarrow \QProc$. A map, $s$ lifts, uniquely, to a map
on process terms, $\widehat{s} : \Proc \rightarrow \Proc$ by the
following equations.

\begin{mathpar}
  (0) \psubstp{Q}{P} := 0 \\
  (R \juxtap S) \psubstp{Q}{P}
  :=    
  (R)\psubstp{Q}{P} \juxtap (S) \psubstp{Q}{P} \\
  (x?(y).R) \psubstp{Q}{P}    
  :=    
  (x)\substp{Q}{P} (z)\concat( (R \psubstn{z}{y}) \psubstp{Q}{P} ) \\
  (\lift{x}{R}) \psubstp{Q}{P}  
  :=
  \lift{(x)\substp{Q}{P}}{ R \psubstp{Q}{P} } \\
%   (\dropn{x})  \psubstp{Q}{P}       
%   := 
%   \left\{ 
%     \begin{array}{ccc} 
%       \dropn{\quotep{Q}} & & x \nameeq \quotep{P} \\
%       \dropn{x} & & otherwise \\
%     \end{array}
%   \right. 
  (\dropn{x})  \psubstp{Q}{P}       
  := 
  \left\{ 
    \begin{array}{ccc} 
      Q & & x \nameeq \quotep{P} \\
      \dropn{x} & & otherwise \\
    \end{array}
  \right.
\end{mathpar}
 

where

\begin{eqnarray}
  (x)\id{\{} \lpquote Q \rpquote / \lpquote P \rpquote \id{\}}            = 
  \left\{ 
    \begin{array}{ccc}
      \lpquote Q \rpquote & & x \nameeq \lpquote P \rpquote \\
      x & & otherwise \\
    \end{array}
  \right. \nonumber
\end{eqnarray}

and $z$ is chosen distinct from $\quotep{P}$, $\quotep{Q}$, the free
names in $Q$, and all the names in $R$. Our $\alpha$-equivalence will
be built in the standard way from this substitution.

\begin{remark}\label{rem:no_self_referential_names}
  One consequence of these definitions is that $\forall P. \quotep{P}
  \not\in \freenames{P}$.
\end{remark}

\subsection{ Dynamic quote: an example }

Anticipating something of what's to come, consider applying the
substitution, $\widehat{\id{\{}u / z \id{\}}}$, to the following pair
of processes, $\lift{w}{y!(z)}$ and $w[ \lpquote y!(z) \rpquote ]$.

\begin{eqnarray}
	\lift{w}{y!(z)}\widehat{\id{\{}u / z \id{\}}}
		& = &
		\lift{w}{y!(u)} \nonumber\\
	w[ \lpquote y!(z) \rpquote ] \widehat{ \id{\{}u / z \id{\}} }
		& = &
		w[ \lpquote y!(z) \rpquote ] \nonumber
\end{eqnarray}

Because the body of the process between quotes is impervious to
substitution, we get radically different answers. In fact, by
examining the first process in an input context,
e.g. $x?(z).\lift{w}{y!(z)}$, we see that the process under the lift
operator may be shaped by prefixed inputs binding a name inside it. In
this sense, the lift operator will be seen as a way to dynamically
construct processes before reifying them as names.

Finally equipped with these standard features we can present the
dynamics of the calculus.

\subsubsection{Operational semantics} 

Finally, we introduce the computational dynamics. What marks these
algebras as distinct from other more traditionally studied algebraic
structures, e.g. vector spaces or polynomial rings, is the manner in
which dynamics is captured. In traditional structures, dynamics is typically
expressed through morphisms between such structures, as in linear maps
between vector spaces or morphisms between rings. In algebras
associated with the semantics of computation, the dynamics is
expressed as part of the algebraic structure itself, through a
reduction reduction relation typically denoted by $\red$. Below, we
give a recursive presentation of this relation for the calculus used
in the encoding.

$\red \subseteq \pi \times \pi$
$\red : \pi \to \mathcal{P}(\pi)$

\begin{mathpar}
  \inferrule* [lab=Comm] { \textsf{match}( x_{src}, x_{trgt} ) } { x_{trgt}?(y)P \; | \; x_{src}!\langle {Q} \rangle \red P\{\quotep{Q}/y}\} }
  \and \\
  \inferrule* [lab=Par] {{P} \red {P}'} {{{P} | {Q}} \red {{P}' | {Q}}}
  \and
  \inferrule* [lab=Equiv]{{{P} \scong {P}'} \andalso {{P}' \red {Q}'} \andalso {{Q}' \scong {Q}}}{{P} \red {Q}}
\end{mathpar}

\begin{eqnarray*}
  match_{\equiv} (\quotep{P},\quotep{Q}) & := & P \equiv Q \\
  match_{\dagger}(\quotep{P},\quotep{Q}) & := & \forall R. P|Q \red^{*} R => R \red^{*} 0 \\
  match_{K}(\quotep{P},\quotep{Q}) & := & K \mbox{ for some context } K
\end{eqnarray*}

$u?(x)P | u!\langle Q \rangle \red P\{\quotep{Q}/x\}$

%We write $\wred$ for $\red^*$, and $P\red$ if $\exists Q $ such that $ P \red Q$.
We write $P\red$ if $\exists Q $ such that $ P \red Q$ and $P\not\red$, otherwise.

\section{Replication}

As mentioned before, it is known that replication (and hence
recursion) can be implemented in a higher-order process algebra
\cite{SangiorgiWalker}. As our first example of calculation with the
machinery thus far presented we give the construction explicitly in
the {\rhoc}.

\begin{eqnarray}
	D_{x} & := & \prefix{x}{y}{(\binpar{\outputp{x}{y}}{@{y}})} \nonumber\\
	\bangp_{x}{P} & := & \binpar{{x}!\langle{\binpar{D_{x}}{P}}\rangle}{D_{x}} \nonumber
\end{eqnarray}

\begin{eqnarray}
	\bangp_{x}{P} & & \nonumber\\
	=
	& {x}!\langle{(\prefix{x}{y}{(\outputp{x}{y} | @{y})) | P}}\rangle 
	      | \prefix{x}{y}{(\outputp{x}{y} | @{y})} & \nonumber\\
	\red
	& (\outputp{x}{y} | @{y})\substn{\quotep{(\prefix{x}{y}{(@{y} | \outputp{x}{y})) | P}}}{y} & \nonumber\\
	=
	& \outputp{x}{\quotep{(\prefix{x}{y}{(\outputp{x}{y} | @{y})) | P}}}
	  | {(\prefix{x}{y}{(\outputp{x}{y} | @{y})) | P}} & \nonumber\\
	\red
	& \ldots & \nonumber\\
	\red^*
	& P | P | \ldots & \nonumber
\end{eqnarray}

Of course, this encoding, as an implementation, runs away, unfolding
$\bangp{P}$ eagerly. A lazier and more implementable replication
operator, restricted to input-guarded processes, may be obtained as follows.

\begin{eqnarray}
\bangp{\prefix{u}{v}{P}} 
	:= 
	\binpar{\lift{x}{\prefix{u}{v}{(\binpar{D(x)}{P})}}}{D(x)} \nonumber
\end{eqnarray}

\begin{remark}
  Note that the lazier definition still does not deal with summation
  or mixed summation (i.e. sums over input and output). The reader is
  invited to construct definitions of replication that deal with these
  features. 

  Further, the definitions are parameterized in a name, $x$. Can you,
  gentle reader, make a definition that eliminates this parameter and
  guarantees no accidental interaction between the replication
  machinery and the process being replicated -- i.e. no accidental
  sharing of names used by the process to get its work done and the
  name(s) used by the replication to effect copying. This latter
  revision of the definition of replication is crucial to obtaining
  the expected identity $!!P \sim !P$.
\end{remark}

\begin{remark}\label{rem:paradoxical_combinator}
  The reader familiar with the lambda calculus will have noticed the
  similarity between $D$ and the paradoxical combinator.

  [Ed. note: the existence of this seems to suggest we have to be more
  restrictive on the set of processes and names we admit if we are to
  support no-cloning.]
\end{remark}

\subsubsection{Bisimulation}

The computational dynamics gives rise to another kind of equivalence,
the equivalence of computational behavior. As previously mentioned
this is typically captured \emph{via} some form of bisimulation.

% The notion we use in this paper is weak barbed bisimulation
% \cite{milner91polyadicpi}.

The notion we use in this paper is derived from weak barbed
bisimulation \cite{milner91polyadicpi}. 

\begin{definition}
An \emph{observation relation}, $\downarrow_{\mathcal N}$, over a set
of names, $\mathcal N$, is the smallest relation satisfying the rules
below.

\infrule[Out-barb]{y \in {\mathcal N}, \; x \nameeq y}
		  {\outputp{x}{v} \downarrow_{\mathcal N} x}
\infrule[Par-barb]{\mbox{$P\downarrow_{\mathcal N} x$ or $Q\downarrow_{\mathcal N} x$}}
		  {\binpar{P}{Q} \downarrow_{\mathcal N} x}

We write $P \Downarrow_{\mathcal N} x$ if there is $Q$ such that 
$P \wred Q$ and $Q \downarrow_{\mathcal N} x$.
\end{definition}

\begin{definition}
%\label{def.bbisim}
An  ${\mathcal N}$-\emph{barbed bisimulation} over a set of names, ${\mathcal N}$, is a symmetric binary relation 
${\mathcal S}_{\mathcal N}$ between agents such that $P\rel{S}_{\mathcal N}Q$ implies:
\begin{enumerate}
\item If $P \red P'$ then $Q \wred Q'$ and $P'\rel{S}_{\mathcal N} Q'$.
\item If $P\downarrow_{\mathcal N} x$, then $Q\Downarrow_{\mathcal N} x$.
\end{enumerate}
$P$ is ${\mathcal N}$-barbed bisimilar to $Q$, written
$P \wbbisim_{\mathcal N} Q$, if $P \rel{S}_{\mathcal N} Q$ for some ${\mathcal N}$-barbed bisimulation ${\mathcal S}_{\mathcal N}$.
\end{definition}

$\mathcal{R} \subseteq \pi \times \pi$

$P \mathcal{R} Q => \forall P'. P \red P' \Rightarrow \exists Q'. Q \red Q', P' \mathcal{R} Q'$

$P \vdash x \Rightarrow Q \vdash x$

\begin{mathpar}
  \inferrule*[lab=Out-barb]{x \nameeq y}{{y}!\langle{Q}\rangle \vdash x}
  \and
  \inferrule*[lab=Par-barb]{\mbox{$P\vdash x$ or $Q\vdash x$}}{\binpar{P}{Q} \vdash x}
\end{mathpar}

\subsubsection{Contexts}

One of the principle advantages of computational calculi like the
$\pi$-calculus is a well-defined notion of context,
contextual-equivalence and a correlation between
contextual-equivalence and notions of bisimulation. The notion of
context allows the decomposition of a process into (sub-)process and
its syntactic environment, its context. Thus, a context may be
thought of as a process with a ``hole'' (written $\Box$) in it. The
application of a context $M$ to a process $P$, written $M[P]$, is
tantamount to filling the hole in $M$ with $P$. In this paper we do
not need the full weight of this theory, but do make use of the notion
of context in the proof the main theorem. 

\begin{mathpar}
  \inferrule* [lab=summation] {} {{M_{M},M_{N}} \bc \Box \;|\; x.M_{A} \;|\; M_{M}+M_{N}}
  \and
  \inferrule* [lab=agent] {} {{M_{A}} \bc (\vec{x})M_{P} \;| \; \clift{P_0,\ldots,M_{P},\ldots,P_N}}
  \and \\
  \inferrule* [lab=process] {} {{M_{P}} \bc M_{N} \;| \;P|M_{P} }
\end{mathpar} 

\begin{mathpar}
  \inferrule* [lab=sychronization] {} {M_{N} \bc \Box \;|\; x?M_{F} \;|\; x!M_{C}}
  \and
  \inferrule* [lab=abstraction] {} {{M_{F}} \bc (x)M_{P} }
  \and
  \inferrule* [lab=concretion] {} {{M_{C}} \bc \langle M_{P} \rangle }
  \and \\
  \inferrule* [lab=process] {} {{M_{P}} \bc M_{N} \;| \;P|M_{P} }
\end{mathpar}

\begin{definition}[contextual application] Given a context $M$, and
  process $P$, we define the \emph{contextual application}, $M[P] :=
  M\{P/\Box\}$. That is, the contextual application of M to P is the
  substitution of $P$ for $\Box$ in $M$.
\end{definition}

$\meaningof{-} : L \to \mathcal{P}(\pi)$

\begin{mathpar}
  \inferrule* [lab=collection] {} {\meaningof{true} = \pi, \and \meaningof{~E} = \pi \setminus \meaningof{E}, \and \meaningof{E_{1} \& E_{2}} = \meaningof{E_{1}} \cap \meaningof{E_{2}}}
\end{mathpar}

\begin{mathpar}
  \inferrule* [lab=structure] {} {\meaningof{0} = \{ P \in \pi | P \equiv 0 \}, \and \\ \meaningof{E_1 | E_2} = \{ P \in \pi | P \equiv P_{1} | P_{2}, P_{1} \in \meaningof{E_{1}}, P_{2} \in \meaningof{E_2}\} }
\end{mathpar}

\begin{mathpar}
 \inferrule* [lab=behavior] {} {\meaningof{\langle a?b \rangle E} = \{ P \in \pi | P \equiv Q | u?(y)P', \\ \and \\\\ \and \\ \;\;\; u \in \meaningof{a}, \forall z.P'\{z/y\} \in \meaningof{E\{z/b\}}\}, \and \\ \meaningof{a!E} = \{ P \in \pi | P \equiv Q | x!\langle P' \rangle, x \in \meaningof{a} P' \in \meaningof{E}\} }
\end{mathpar}

\begin{mathpar}
 \inferrule* [lab=nominal] {} {\meaningof{\quotep{E}} = \{ \quotep{P} \in \quotep{\pi} | P \in \meaningof{E} \}, \and \meaningof{\quotep{P}} = \{ \quotep{Q} \in \quotep{\pi} | P \equiv Q \} \and \\ \meaningof{@\quotep{E}} = \{ P \in \pi | P \equiv @x, x \in \meaningof{E} \}}
\end{mathpar}

\begin{eqnarray*}
  \\
  \meaningof{-} : TS \to ST
\end{eqnarray*}

\begin{eqnarray*}
  \\
  L : TS \to ST
\end{eqnarray*}

\begin{eqnarray*}
  \\
  P \models E \iff P \in \meaningof{E}
\end{eqnarray*}

\begin{eqnarray*}
  P \approx_{L} Q \iff \forall E \in L. P \models E \iff Q \models E
\end{eqnarray*}

\begin{eqnarray*}
  P \approx_{K} Q
\end{eqnarray*}

\begin{eqnarray*}
  P \approx Q
\end{eqnarray*}

$\approx_{K} = \approx = \approx_{L}$

\subsubsection{Contextual duality}

Note that contexts extend the quotation operation to a family of
operations from processes to names. Given a context, $M$, we can
define a \emph{nominal context}, $\quotep{M}$ by $\quotep{M}[P] :=
\quotep{M[P]}$. To foreshadow what is to come we observe that these
operations enjoy a duality with processes very much like the duality
between vectors and maps from vectors to scalars.

Further, because the calculus is essentially higher-order, we have a
correspondence between contexts and processes. More specifically,
given a name $x$ and a context $M$ we can construct $M^{*}_{x}$ such
that 

\begin{mathpar}
  M^{*}_{x} | \lift{x}{P} \red M[P]
\end{mathpar}

namely,

\begin{mathpar}
  M^{*}_{x} := x?(u).M[\dropn{u}]
\end{mathpar}

The dependence of $M^{*}_{x}$ on a name makes it an abstraction, 

\begin{mathpar}
  M^{*} := (x)x?(u).M[\dropn{u}]
\end{mathpar}

\subsection{Additional notation}

It will sometimes be convenient to denote the process a name
quotes. We already have the notation $x = \quotep{P}$, but it will be
convenient to introduce an alternate notation, $\procn{x}$, when we
want to emphasize the connection to the use of the name. Note that, by
virtue of name equivalence, $\quotep{\procn{x}} \nameeq x$; so, the
notation is consistent with previous definitions.

Further, because names have structure it is possible to effect
substitutions on the basis of that structure. This means we need to
upgrade our notation for substitutions, which we accomplish by
adapting comprehension notation. Thus,

\begin{mathpar}
  P\{ y / x : x \in S \}
\end{mathpar}

is interpreted to mean the process derived from P by replacing (in a
capture-avoiding manner) each occurrence of $x$ in $S$ by $y$. For example,

\begin{mathpar}
  P\{ \quotep{\procn{x}|\procn{x}} / x : x \in \freenames{P} \}
\end{mathpar}

will replace each (occurrence) of a free name $x$ in $P$ by
$\quotep{\procn{x}|\procn{x}}$.

Also, we will avail ourselves of the notation $x^{L}$ and $x^{R}$ to
denote injections of a name into disjoint copies of the name
space. There are numerous ways to accomplish this. One example can be
found in \cite{MeredithR05}. This notation overloads to vectors of
names: $\vec{x}^{\pi} := (x_{i}^{\pi} \; : \; 0 \leq i < |\vec{x}| )$ where $\pi \in \{L,R\}$.

We also use $P^{\Box} := P|\Box$.

In \cite{MeredithR05} an interpretation of the new operator is
given. It turns out that there are several possible interpretations
all enjoying the requisite algebraic properties of the operator (see
\cite{milner91polyadicpi}). We will therefore make liberal use of
$(\nu\; \vec{x})P$.

% subsection the_syntax_and_semantics_of_the_notation_system (end)   

\input{qm2pi.qmops} 

\input{qm2pi.sterngerlach} 

\input{qm2pi.metric} 

% section concurrent_process_calculi (end)

%\input{qm2pi.proofsketch}

% section proof sketch (end)

%\input{qm2pi.slviaknots} 

% section spatial logic via knots (end)

\input{qm2pi.conclusion}

% section conclusion (end)

%\input{qm2pi.dtcodes} 

% section wiring algorithm (end)

\input{qm2pi.ack} 

% section acknowledgments (end)

\newpage


\bibliographystyle{plain}   
\bibliography{../../biblios/main.bib}

\input{qm2pi.rhodetails}

\end{document}

 

%\documentclass[12pt]{llncs}
%\documentclass{jktr}

\usepackage[pdftex]{hyperref}                   
\usepackage {listings}
\usepackage {mathpartir}
\usepackage{bcprules}
%\usepackage{listings}
                       
\usepackage{graphicx} 
%\usepackage[margins=2.5cm,nohead,nofoot]{geometry}
%\usepackage{geometry}
\usepackage{amsfonts}
\usepackage{amstext}
\usepackage{latexsym}
\usepackage{amssymb}
\usepackage{color}


%\include{myPreamble}
\include{qm2pi.local} 

%\ifpdf
%\usepackage[pdftex]{graphicx}
%\else
%\usepackage{graphicx}
%\fi

 % \ifpdf
%  \usepackage{pdfsync}
%  \if


%\title{Brief Article}
%\author{David F. Snyder}
%\author{L.G. Meredith}

%\address{Dept. of Math., Texas State University--San Marcos, San Marcos, TX 78666}
       
\pagestyle{empty}


\begin{document}

\lstset{language=[Objective]Caml,frame=shadowbox}

\input{qm2pi.front}

% section front matter (end)

\input{qm2pi.intro} 
 
% section introduction (end)

% \input{qm2pi.knotations} 

% section notation (end)

\input{qm2pi.process.calculi} 

% section concurrent_process_calculi_and_spatial_logics_ (end)
    
%\input{qm2pi.knots2pi} 

%\input{qm2pi.trefoil} 

%\input{qm2pi.mainthm} 

% subsection basic_interpretation (end)

%\input{qm2pi.rho.presentation} 
\subsection{The syntax and semantics of the notation system}\label{sub:the_syntax_and_semantics_of_the_notation_system} % (fold)

We now summarize a technical presentation of the calculus that
embodies our theory of dynamics. The typical presentation of such a
calculus follows the style of giving generators and relations on
them. The grammar, below, describing term constructors, freely
generates the set of processes, $\Proc$. This set is then quotiented
by a relation known as structural congruence and it is over this set
that the notion of dynamics is expressed. This presentation is
essentially that of \cite{MeredithR05} with the addition of
polyadicity and summation. For readability we have relegated some of
the technical subtleties to an appendix.

\subsubsection{Process grammar}\label{subsub:process_grammar}

\begin{mathpar}
  \inferrule* [lab=synchronization] {} {{M} \bc \pzero \;|\; x?F \;|\; x!C }
  \and
  \inferrule* [lab=abstraction] {} {{F} \bc (x)P}
  \and
  \inferrule* [lab=concretion] {} {{C} \bc \langle Q \rangle}
  \and
  \inferrule* [lab=process] {} {{P,Q} \bc M \;| \;P|Q \;|\; @{x}}
  \and
  \inferrule* [lab=name] {} {{x} \bc \quotep{P}}
\end{mathpar} 

Note that $\vec{x}$ (resp. $\vec{P}$) denotes a vector of names
(resp. processes) of length $|\vec{x}|$ (resp. $|\vec{P}|$). We adopt
the following useful abbreviations.

\begin{mathpar}
   x?(\vec{y}).P := x.(\vec{y})P \and  x\clift{\vec{P}} := x.\clift{\vec{P}}
   \and x!(y) := \lift{x}{\dropn{y}}
   \and \Pi_{i=0}^{n-1}P_i := P_0 | \ldots | P_{n-1}
\end{mathpar}

\subsubsection{Structural congruence}

\paragraph{Free and bound names and alpha-equivalence.} At the
core of structural equivalence is alpha-equivalence which identifies
process that are the same up to a change of variable. Formally, we
recognize the distinction between free and bound names. The free names
of a process, $\freenames{P}$, may be calculated recursively as
follows:

\begin{mathpar}
\freenames{\pzero} := \emptyset
  \and \\
  \freenames{x?(y).P} := \{ x \} \cup (\freenames{P} \setminus \{ y \})
  \and 
  \freenames{x!\langle P \rangle} := \{ x \} \cup \{ P \} 
  \and \\
  \freenames{P|Q} := \freenames{P} \cup \freenames{Q}
  \and \\
  \freenames{@{x}} := \{ x \}
\end{mathpar}

$\pi$
$\quotep{\pi}$

$\freenames{-} : \pi \to \mathcal{P}(\quotep{\pi})$

\begin{eqnarray*}
  \freenames{\pzero} & := & \emptyset \\
  \freenames{x?(y).P} & := & \{ x \} \cup (\freenames{P} \setminus \{ y \}) \\
  \freenames{x!\langle P \rangle} & := & \{ x \} \cup \{ P \} \\
  \freenames{P|Q} & := & \freenames{P} \cup \freenames{Q} \\
  \freenames{\dropn{x}} & := & \{ x \}
\end{eqnarray*}

The bound names of a process, $\boundnames{P}$, are those names occurring in $P$
that are not free. For example, in $x?(y).0$, the name $x$ is free, while $y$ is bound.

\begin{mathpar}
  \inferrule* [lab=monoidal-laws] {} { P|Q \equiv Q|P \and P|0 \equiv P \and P|(Q|R) \equiv (P|Q)|R }
\end{mathpar}

\begin{mathpar}
  \inferrule* [lab=alpha-equivalence] {} { (x)P \equiv (y)P\{y/x\} \and y \not\in \freenames{P} }
\end{mathpar}

\begin{definition}
Then two processes, $P,Q$, are alpha-equivalent if $P = Q\{\vec{y}/\vec{x}\}$ for
some $\vec{x} \in \boundnames{Q},\vec{y} \in \boundnames{P}$, where $Q\{\vec{y}/\vec{x}\}$
denotes the capture-avoiding substitution of $\vec{y}$ for $\vec{x}$ in $Q$.
\end{definition}

\begin{definition}
  The {\em structural congruence} \cite{SangiorgiWalker} , $\equiv$,
  between processes is the least congruence containing
  alpha-equivalence, satisfying the abelian monoid laws
  (associativity, commutativity and $\pzero$ as identity) for parallel
  composition $|$ and for summation $+$.
\end{definition}

\subsection{Name equivalence}

We take name equivalence, written $\nameeq$, to be the smallest
equivalence relation generated by the following rules.

\begin{mathpar}
\inferrule*[lab=Quote-drop]
{ }
{ \quotep{@{x}} \nameeq x }

\inferrule*[lab=Struct-equiv]
{ P \scong Q }
{ \quotep{P} \nameeq \quotep{Q} }
\end{mathpar}

The astute reader will have noticed that the mutual recursion of names
and processes imposes a mutual recursion on alpha-equivalence and
structural equivalence via name-equivalence. Fortunately, all of this
works out pleasantly and we may calculate in the natural way, free of
concern. The reader interested in the details is referred to the
appendix \ref{appendix:rho_details}.

\subsection{Substitution}

We use $\Proc$ for the set of processes, $\QProc$ for the set of
names, and $\id{\{}\vec{y} / \vec{x} \id{\}}$ to denote partial maps,
$s : \QProc \rightarrow \QProc$. A map, $s$ lifts, uniquely, to a map
on process terms, $\widehat{s} : \Proc \rightarrow \Proc$ by the
following equations.

\begin{mathpar}
  (0) \psubstp{Q}{P} := 0 \\
  (R \juxtap S) \psubstp{Q}{P}
  :=    
  (R)\psubstp{Q}{P} \juxtap (S) \psubstp{Q}{P} \\
  (x?(y).R) \psubstp{Q}{P}    
  :=    
  (x)\substp{Q}{P} (z)\concat( (R \psubstn{z}{y}) \psubstp{Q}{P} ) \\
  (\lift{x}{R}) \psubstp{Q}{P}  
  :=
  \lift{(x)\substp{Q}{P}}{ R \psubstp{Q}{P} } \\
%   (\dropn{x})  \psubstp{Q}{P}       
%   := 
%   \left\{ 
%     \begin{array}{ccc} 
%       \dropn{\quotep{Q}} & & x \nameeq \quotep{P} \\
%       \dropn{x} & & otherwise \\
%     \end{array}
%   \right. 
  (\dropn{x})  \psubstp{Q}{P}       
  := 
  \left\{ 
    \begin{array}{ccc} 
      Q & & x \nameeq \quotep{P} \\
      \dropn{x} & & otherwise \\
    \end{array}
  \right.
\end{mathpar}
 

where

\begin{eqnarray}
  (x)\id{\{} \lpquote Q \rpquote / \lpquote P \rpquote \id{\}}            = 
  \left\{ 
    \begin{array}{ccc}
      \lpquote Q \rpquote & & x \nameeq \lpquote P \rpquote \\
      x & & otherwise \\
    \end{array}
  \right. \nonumber
\end{eqnarray}

and $z$ is chosen distinct from $\quotep{P}$, $\quotep{Q}$, the free
names in $Q$, and all the names in $R$. Our $\alpha$-equivalence will
be built in the standard way from this substitution.

\begin{remark}\label{rem:no_self_referential_names}
  One consequence of these definitions is that $\forall P. \quotep{P}
  \not\in \freenames{P}$.
\end{remark}

\subsection{ Dynamic quote: an example }

Anticipating something of what's to come, consider applying the
substitution, $\widehat{\id{\{}u / z \id{\}}}$, to the following pair
of processes, $\lift{w}{y!(z)}$ and $w[ \lpquote y!(z) \rpquote ]$.

\begin{eqnarray}
	\lift{w}{y!(z)}\widehat{\id{\{}u / z \id{\}}}
		& = &
		\lift{w}{y!(u)} \nonumber\\
	w[ \lpquote y!(z) \rpquote ] \widehat{ \id{\{}u / z \id{\}} }
		& = &
		w[ \lpquote y!(z) \rpquote ] \nonumber
\end{eqnarray}

Because the body of the process between quotes is impervious to
substitution, we get radically different answers. In fact, by
examining the first process in an input context,
e.g. $x?(z).\lift{w}{y!(z)}$, we see that the process under the lift
operator may be shaped by prefixed inputs binding a name inside it. In
this sense, the lift operator will be seen as a way to dynamically
construct processes before reifying them as names.

Finally equipped with these standard features we can present the
dynamics of the calculus.

\subsubsection{Operational semantics} 

Finally, we introduce the computational dynamics. What marks these
algebras as distinct from other more traditionally studied algebraic
structures, e.g. vector spaces or polynomial rings, is the manner in
which dynamics is captured. In traditional structures, dynamics is typically
expressed through morphisms between such structures, as in linear maps
between vector spaces or morphisms between rings. In algebras
associated with the semantics of computation, the dynamics is
expressed as part of the algebraic structure itself, through a
reduction reduction relation typically denoted by $\red$. Below, we
give a recursive presentation of this relation for the calculus used
in the encoding.

$\red \subseteq \pi \times \pi$
$\red : \pi \to \mathcal{P}(\pi)$

\begin{mathpar}
  \inferrule* [lab=Comm] { \textsf{match}( x_{src}, x_{trgt} ) } { x_{trgt}?(y)P \; | \; x_{src}!\langle {Q} \rangle \red P\{\quotep{Q}/y}\} }
  \and \\
  \inferrule* [lab=Par] {{P} \red {P}'} {{{P} | {Q}} \red {{P}' | {Q}}}
  \and
  \inferrule* [lab=Equiv]{{{P} \scong {P}'} \andalso {{P}' \red {Q}'} \andalso {{Q}' \scong {Q}}}{{P} \red {Q}}
\end{mathpar}

\begin{eqnarray*}
  match_{\equiv} (\quotep{P},\quotep{Q}) & := & P \equiv Q \\
  match_{\dagger}(\quotep{P},\quotep{Q}) & := & \forall R. P|Q \red^{*} R => R \red^{*} 0 \\
  match_{K}(\quotep{P},\quotep{Q}) & := & K \mbox{ for some context } K
\end{eqnarray*}

$u?(x)P | u!\langle Q \rangle \red P\{\quotep{Q}/x\}$

%We write $\wred$ for $\red^*$, and $P\red$ if $\exists Q $ such that $ P \red Q$.
We write $P\red$ if $\exists Q $ such that $ P \red Q$ and $P\not\red$, otherwise.

\section{Replication}

As mentioned before, it is known that replication (and hence
recursion) can be implemented in a higher-order process algebra
\cite{SangiorgiWalker}. As our first example of calculation with the
machinery thus far presented we give the construction explicitly in
the {\rhoc}.

\begin{eqnarray}
	D_{x} & := & \prefix{x}{y}{(\binpar{\outputp{x}{y}}{@{y}})} \nonumber\\
	\bangp_{x}{P} & := & \binpar{{x}!\langle{\binpar{D_{x}}{P}}\rangle}{D_{x}} \nonumber
\end{eqnarray}

\begin{eqnarray}
	\bangp_{x}{P} & & \nonumber\\
	=
	& {x}!\langle{(\prefix{x}{y}{(\outputp{x}{y} | @{y})) | P}}\rangle 
	      | \prefix{x}{y}{(\outputp{x}{y} | @{y})} & \nonumber\\
	\red
	& (\outputp{x}{y} | @{y})\substn{\quotep{(\prefix{x}{y}{(@{y} | \outputp{x}{y})) | P}}}{y} & \nonumber\\
	=
	& \outputp{x}{\quotep{(\prefix{x}{y}{(\outputp{x}{y} | @{y})) | P}}}
	  | {(\prefix{x}{y}{(\outputp{x}{y} | @{y})) | P}} & \nonumber\\
	\red
	& \ldots & \nonumber\\
	\red^*
	& P | P | \ldots & \nonumber
\end{eqnarray}

Of course, this encoding, as an implementation, runs away, unfolding
$\bangp{P}$ eagerly. A lazier and more implementable replication
operator, restricted to input-guarded processes, may be obtained as follows.

\begin{eqnarray}
\bangp{\prefix{u}{v}{P}} 
	:= 
	\binpar{\lift{x}{\prefix{u}{v}{(\binpar{D(x)}{P})}}}{D(x)} \nonumber
\end{eqnarray}

\begin{remark}
  Note that the lazier definition still does not deal with summation
  or mixed summation (i.e. sums over input and output). The reader is
  invited to construct definitions of replication that deal with these
  features. 

  Further, the definitions are parameterized in a name, $x$. Can you,
  gentle reader, make a definition that eliminates this parameter and
  guarantees no accidental interaction between the replication
  machinery and the process being replicated -- i.e. no accidental
  sharing of names used by the process to get its work done and the
  name(s) used by the replication to effect copying. This latter
  revision of the definition of replication is crucial to obtaining
  the expected identity $!!P \sim !P$.
\end{remark}

\begin{remark}\label{rem:paradoxical_combinator}
  The reader familiar with the lambda calculus will have noticed the
  similarity between $D$ and the paradoxical combinator.

  [Ed. note: the existence of this seems to suggest we have to be more
  restrictive on the set of processes and names we admit if we are to
  support no-cloning.]
\end{remark}

\subsubsection{Bisimulation}

The computational dynamics gives rise to another kind of equivalence,
the equivalence of computational behavior. As previously mentioned
this is typically captured \emph{via} some form of bisimulation.

% The notion we use in this paper is weak barbed bisimulation
% \cite{milner91polyadicpi}.

The notion we use in this paper is derived from weak barbed
bisimulation \cite{milner91polyadicpi}. 

\begin{definition}
An \emph{observation relation}, $\downarrow_{\mathcal N}$, over a set
of names, $\mathcal N$, is the smallest relation satisfying the rules
below.

\infrule[Out-barb]{y \in {\mathcal N}, \; x \nameeq y}
		  {\outputp{x}{v} \downarrow_{\mathcal N} x}
\infrule[Par-barb]{\mbox{$P\downarrow_{\mathcal N} x$ or $Q\downarrow_{\mathcal N} x$}}
		  {\binpar{P}{Q} \downarrow_{\mathcal N} x}

We write $P \Downarrow_{\mathcal N} x$ if there is $Q$ such that 
$P \wred Q$ and $Q \downarrow_{\mathcal N} x$.
\end{definition}

\begin{definition}
%\label{def.bbisim}
An  ${\mathcal N}$-\emph{barbed bisimulation} over a set of names, ${\mathcal N}$, is a symmetric binary relation 
${\mathcal S}_{\mathcal N}$ between agents such that $P\rel{S}_{\mathcal N}Q$ implies:
\begin{enumerate}
\item If $P \red P'$ then $Q \wred Q'$ and $P'\rel{S}_{\mathcal N} Q'$.
\item If $P\downarrow_{\mathcal N} x$, then $Q\Downarrow_{\mathcal N} x$.
\end{enumerate}
$P$ is ${\mathcal N}$-barbed bisimilar to $Q$, written
$P \wbbisim_{\mathcal N} Q$, if $P \rel{S}_{\mathcal N} Q$ for some ${\mathcal N}$-barbed bisimulation ${\mathcal S}_{\mathcal N}$.
\end{definition}

$\mathcal{R} \subseteq \pi \times \pi$

$P \mathcal{R} Q => \forall P'. P \red P' \Rightarrow \exists Q'. Q \red Q', P' \mathcal{R} Q'$

$P \vdash x \Rightarrow Q \vdash x$

\begin{mathpar}
  \inferrule*[lab=Out-barb]{x \nameeq y}{{y}!\langle{Q}\rangle \vdash x}
  \and
  \inferrule*[lab=Par-barb]{\mbox{$P\vdash x$ or $Q\vdash x$}}{\binpar{P}{Q} \vdash x}
\end{mathpar}

\subsubsection{Contexts}

One of the principle advantages of computational calculi like the
$\pi$-calculus is a well-defined notion of context,
contextual-equivalence and a correlation between
contextual-equivalence and notions of bisimulation. The notion of
context allows the decomposition of a process into (sub-)process and
its syntactic environment, its context. Thus, a context may be
thought of as a process with a ``hole'' (written $\Box$) in it. The
application of a context $M$ to a process $P$, written $M[P]$, is
tantamount to filling the hole in $M$ with $P$. In this paper we do
not need the full weight of this theory, but do make use of the notion
of context in the proof the main theorem. 

\begin{mathpar}
  \inferrule* [lab=summation] {} {{M_{M},M_{N}} \bc \Box \;|\; x.M_{A} \;|\; M_{M}+M_{N}}
  \and
  \inferrule* [lab=agent] {} {{M_{A}} \bc (\vec{x})M_{P} \;| \; \clift{P_0,\ldots,M_{P},\ldots,P_N}}
  \and \\
  \inferrule* [lab=process] {} {{M_{P}} \bc M_{N} \;| \;P|M_{P} }
\end{mathpar} 

\begin{mathpar}
  \inferrule* [lab=sychronization] {} {M_{N} \bc \Box \;|\; x?M_{F} \;|\; x!M_{C}}
  \and
  \inferrule* [lab=abstraction] {} {{M_{F}} \bc (x)M_{P} }
  \and
  \inferrule* [lab=concretion] {} {{M_{C}} \bc \langle M_{P} \rangle }
  \and \\
  \inferrule* [lab=process] {} {{M_{P}} \bc M_{N} \;| \;P|M_{P} }
\end{mathpar}

\begin{definition}[contextual application] Given a context $M$, and
  process $P$, we define the \emph{contextual application}, $M[P] :=
  M\{P/\Box\}$. That is, the contextual application of M to P is the
  substitution of $P$ for $\Box$ in $M$.
\end{definition}

$\meaningof{-} : L \to \mathcal{P}(\pi)$

\begin{mathpar}
  \inferrule* [lab=collection] {} {\meaningof{true} = \pi, \and \meaningof{~E} = \pi \setminus \meaningof{E}, \and \meaningof{E_{1} \& E_{2}} = \meaningof{E_{1}} \cap \meaningof{E_{2}}}
\end{mathpar}

\begin{mathpar}
  \inferrule* [lab=structure] {} {\meaningof{0} = \{ P \in \pi | P \equiv 0 \}, \and \\ \meaningof{E_1 | E_2} = \{ P \in \pi | P \equiv P_{1} | P_{2}, P_{1} \in \meaningof{E_{1}}, P_{2} \in \meaningof{E_2}\} }
\end{mathpar}

\begin{mathpar}
 \inferrule* [lab=behavior] {} {\meaningof{\langle a?b \rangle E} = \{ P \in \pi | P \equiv Q | u?(y)P', \\ \and \\\\ \and \\ \;\;\; u \in \meaningof{a}, \forall z.P'\{z/y\} \in \meaningof{E\{z/b\}}\}, \and \\ \meaningof{a!E} = \{ P \in \pi | P \equiv Q | x!\langle P' \rangle, x \in \meaningof{a} P' \in \meaningof{E}\} }
\end{mathpar}

\begin{mathpar}
 \inferrule* [lab=nominal] {} {\meaningof{\quotep{E}} = \{ \quotep{P} \in \quotep{\pi} | P \in \meaningof{E} \}, \and \meaningof{\quotep{P}} = \{ \quotep{Q} \in \quotep{\pi} | P \equiv Q \} \and \\ \meaningof{@\quotep{E}} = \{ P \in \pi | P \equiv @x, x \in \meaningof{E} \}}
\end{mathpar}

\begin{eqnarray*}
  \\
  \meaningof{-} : TS \to ST
\end{eqnarray*}

\begin{eqnarray*}
  \\
  L : TS \to ST
\end{eqnarray*}

\begin{eqnarray*}
  \\
  P \models E \iff P \in \meaningof{E}
\end{eqnarray*}

\begin{eqnarray*}
  P \approx_{L} Q \iff \forall E \in L. P \models E \iff Q \models E
\end{eqnarray*}

\begin{eqnarray*}
  P \approx_{K} Q
\end{eqnarray*}

\begin{eqnarray*}
  P \approx Q
\end{eqnarray*}

$\approx_{K} = \approx = \approx_{L}$

\subsubsection{Contextual duality}

Note that contexts extend the quotation operation to a family of
operations from processes to names. Given a context, $M$, we can
define a \emph{nominal context}, $\quotep{M}$ by $\quotep{M}[P] :=
\quotep{M[P]}$. To foreshadow what is to come we observe that these
operations enjoy a duality with processes very much like the duality
between vectors and maps from vectors to scalars.

Further, because the calculus is essentially higher-order, we have a
correspondence between contexts and processes. More specifically,
given a name $x$ and a context $M$ we can construct $M^{*}_{x}$ such
that 

\begin{mathpar}
  M^{*}_{x} | \lift{x}{P} \red M[P]
\end{mathpar}

namely,

\begin{mathpar}
  M^{*}_{x} := x?(u).M[\dropn{u}]
\end{mathpar}

The dependence of $M^{*}_{x}$ on a name makes it an abstraction, 

\begin{mathpar}
  M^{*} := (x)x?(u).M[\dropn{u}]
\end{mathpar}

\subsection{Additional notation}

It will sometimes be convenient to denote the process a name
quotes. We already have the notation $x = \quotep{P}$, but it will be
convenient to introduce an alternate notation, $\procn{x}$, when we
want to emphasize the connection to the use of the name. Note that, by
virtue of name equivalence, $\quotep{\procn{x}} \nameeq x$; so, the
notation is consistent with previous definitions.

Further, because names have structure it is possible to effect
substitutions on the basis of that structure. This means we need to
upgrade our notation for substitutions, which we accomplish by
adapting comprehension notation. Thus,

\begin{mathpar}
  P\{ y / x : x \in S \}
\end{mathpar}

is interpreted to mean the process derived from P by replacing (in a
capture-avoiding manner) each occurrence of $x$ in $S$ by $y$. For example,

\begin{mathpar}
  P\{ \quotep{\procn{x}|\procn{x}} / x : x \in \freenames{P} \}
\end{mathpar}

will replace each (occurrence) of a free name $x$ in $P$ by
$\quotep{\procn{x}|\procn{x}}$.

Also, we will avail ourselves of the notation $x^{L}$ and $x^{R}$ to
denote injections of a name into disjoint copies of the name
space. There are numerous ways to accomplish this. One example can be
found in \cite{MeredithR05}. This notation overloads to vectors of
names: $\vec{x}^{\pi} := (x_{i}^{\pi} \; : \; 0 \leq i < |\vec{x}| )$ where $\pi \in \{L,R\}$.

We also use $P^{\Box} := P|\Box$.

In \cite{MeredithR05} an interpretation of the new operator is
given. It turns out that there are several possible interpretations
all enjoying the requisite algebraic properties of the operator (see
\cite{milner91polyadicpi}). We will therefore make liberal use of
$(\nu\; \vec{x})P$.

% subsection the_syntax_and_semantics_of_the_notation_system (end)   

\input{qm2pi.qmops} 

\input{qm2pi.sterngerlach} 

\input{qm2pi.metric} 

% section concurrent_process_calculi (end)

%\input{qm2pi.proofsketch}

% section proof sketch (end)

%\input{qm2pi.slviaknots} 

% section spatial logic via knots (end)

\input{qm2pi.conclusion}

% section conclusion (end)

%\input{qm2pi.dtcodes} 

% section wiring algorithm (end)

\input{qm2pi.ack} 

% section acknowledgments (end)

\newpage


\bibliographystyle{plain}   
\bibliography{../../biblios/main.bib}

\input{qm2pi.rhodetails}

\end{document}

 

%\documentclass[12pt]{llncs}
%\documentclass{jktr}

\usepackage[pdftex]{hyperref}                   
\usepackage {listings}
\usepackage {mathpartir}
\usepackage{bcprules}
%\usepackage{listings}
                       
\usepackage{graphicx} 
%\usepackage[margins=2.5cm,nohead,nofoot]{geometry}
%\usepackage{geometry}
\usepackage{amsfonts}
\usepackage{amstext}
\usepackage{latexsym}
\usepackage{amssymb}
\usepackage{color}


%\include{myPreamble}
\include{qm2pi.local} 

%\ifpdf
%\usepackage[pdftex]{graphicx}
%\else
%\usepackage{graphicx}
%\fi

 % \ifpdf
%  \usepackage{pdfsync}
%  \if


%\title{Brief Article}
%\author{David F. Snyder}
%\author{L.G. Meredith}

%\address{Dept. of Math., Texas State University--San Marcos, San Marcos, TX 78666}
       
\pagestyle{empty}


\begin{document}

\lstset{language=[Objective]Caml,frame=shadowbox}

\input{qm2pi.front}

% section front matter (end)

\input{qm2pi.intro} 
 
% section introduction (end)

% \input{qm2pi.knotations} 

% section notation (end)

\input{qm2pi.process.calculi} 

% section concurrent_process_calculi_and_spatial_logics_ (end)
    
%\input{qm2pi.knots2pi} 

%\input{qm2pi.trefoil} 

%\input{qm2pi.mainthm} 

% subsection basic_interpretation (end)

%\input{qm2pi.rho.presentation} 
\subsection{The syntax and semantics of the notation system}\label{sub:the_syntax_and_semantics_of_the_notation_system} % (fold)

We now summarize a technical presentation of the calculus that
embodies our theory of dynamics. The typical presentation of such a
calculus follows the style of giving generators and relations on
them. The grammar, below, describing term constructors, freely
generates the set of processes, $\Proc$. This set is then quotiented
by a relation known as structural congruence and it is over this set
that the notion of dynamics is expressed. This presentation is
essentially that of \cite{MeredithR05} with the addition of
polyadicity and summation. For readability we have relegated some of
the technical subtleties to an appendix.

\subsubsection{Process grammar}\label{subsub:process_grammar}

\begin{mathpar}
  \inferrule* [lab=synchronization] {} {{M} \bc \pzero \;|\; x?F \;|\; x!C }
  \and
  \inferrule* [lab=abstraction] {} {{F} \bc (x)P}
  \and
  \inferrule* [lab=concretion] {} {{C} \bc \langle Q \rangle}
  \and
  \inferrule* [lab=process] {} {{P,Q} \bc M \;| \;P|Q \;|\; @{x}}
  \and
  \inferrule* [lab=name] {} {{x} \bc \quotep{P}}
\end{mathpar} 

Note that $\vec{x}$ (resp. $\vec{P}$) denotes a vector of names
(resp. processes) of length $|\vec{x}|$ (resp. $|\vec{P}|$). We adopt
the following useful abbreviations.

\begin{mathpar}
   x?(\vec{y}).P := x.(\vec{y})P \and  x\clift{\vec{P}} := x.\clift{\vec{P}}
   \and x!(y) := \lift{x}{\dropn{y}}
   \and \Pi_{i=0}^{n-1}P_i := P_0 | \ldots | P_{n-1}
\end{mathpar}

\subsubsection{Structural congruence}

\paragraph{Free and bound names and alpha-equivalence.} At the
core of structural equivalence is alpha-equivalence which identifies
process that are the same up to a change of variable. Formally, we
recognize the distinction between free and bound names. The free names
of a process, $\freenames{P}$, may be calculated recursively as
follows:

\begin{mathpar}
\freenames{\pzero} := \emptyset
  \and \\
  \freenames{x?(y).P} := \{ x \} \cup (\freenames{P} \setminus \{ y \})
  \and 
  \freenames{x!\langle P \rangle} := \{ x \} \cup \{ P \} 
  \and \\
  \freenames{P|Q} := \freenames{P} \cup \freenames{Q}
  \and \\
  \freenames{@{x}} := \{ x \}
\end{mathpar}

$\pi$
$\quotep{\pi}$

$\freenames{-} : \pi \to \mathcal{P}(\quotep{\pi})$

\begin{eqnarray*}
  \freenames{\pzero} & := & \emptyset \\
  \freenames{x?(y).P} & := & \{ x \} \cup (\freenames{P} \setminus \{ y \}) \\
  \freenames{x!\langle P \rangle} & := & \{ x \} \cup \{ P \} \\
  \freenames{P|Q} & := & \freenames{P} \cup \freenames{Q} \\
  \freenames{\dropn{x}} & := & \{ x \}
\end{eqnarray*}

The bound names of a process, $\boundnames{P}$, are those names occurring in $P$
that are not free. For example, in $x?(y).0$, the name $x$ is free, while $y$ is bound.

\begin{mathpar}
  \inferrule* [lab=monoidal-laws] {} { P|Q \equiv Q|P \and P|0 \equiv P \and P|(Q|R) \equiv (P|Q)|R }
\end{mathpar}

\begin{mathpar}
  \inferrule* [lab=alpha-equivalence] {} { (x)P \equiv (y)P\{y/x\} \and y \not\in \freenames{P} }
\end{mathpar}

\begin{definition}
Then two processes, $P,Q$, are alpha-equivalent if $P = Q\{\vec{y}/\vec{x}\}$ for
some $\vec{x} \in \boundnames{Q},\vec{y} \in \boundnames{P}$, where $Q\{\vec{y}/\vec{x}\}$
denotes the capture-avoiding substitution of $\vec{y}$ for $\vec{x}$ in $Q$.
\end{definition}

\begin{definition}
  The {\em structural congruence} \cite{SangiorgiWalker} , $\equiv$,
  between processes is the least congruence containing
  alpha-equivalence, satisfying the abelian monoid laws
  (associativity, commutativity and $\pzero$ as identity) for parallel
  composition $|$ and for summation $+$.
\end{definition}

\subsection{Name equivalence}

We take name equivalence, written $\nameeq$, to be the smallest
equivalence relation generated by the following rules.

\begin{mathpar}
\inferrule*[lab=Quote-drop]
{ }
{ \quotep{@{x}} \nameeq x }

\inferrule*[lab=Struct-equiv]
{ P \scong Q }
{ \quotep{P} \nameeq \quotep{Q} }
\end{mathpar}

The astute reader will have noticed that the mutual recursion of names
and processes imposes a mutual recursion on alpha-equivalence and
structural equivalence via name-equivalence. Fortunately, all of this
works out pleasantly and we may calculate in the natural way, free of
concern. The reader interested in the details is referred to the
appendix \ref{appendix:rho_details}.

\subsection{Substitution}

We use $\Proc$ for the set of processes, $\QProc$ for the set of
names, and $\id{\{}\vec{y} / \vec{x} \id{\}}$ to denote partial maps,
$s : \QProc \rightarrow \QProc$. A map, $s$ lifts, uniquely, to a map
on process terms, $\widehat{s} : \Proc \rightarrow \Proc$ by the
following equations.

\begin{mathpar}
  (0) \psubstp{Q}{P} := 0 \\
  (R \juxtap S) \psubstp{Q}{P}
  :=    
  (R)\psubstp{Q}{P} \juxtap (S) \psubstp{Q}{P} \\
  (x?(y).R) \psubstp{Q}{P}    
  :=    
  (x)\substp{Q}{P} (z)\concat( (R \psubstn{z}{y}) \psubstp{Q}{P} ) \\
  (\lift{x}{R}) \psubstp{Q}{P}  
  :=
  \lift{(x)\substp{Q}{P}}{ R \psubstp{Q}{P} } \\
%   (\dropn{x})  \psubstp{Q}{P}       
%   := 
%   \left\{ 
%     \begin{array}{ccc} 
%       \dropn{\quotep{Q}} & & x \nameeq \quotep{P} \\
%       \dropn{x} & & otherwise \\
%     \end{array}
%   \right. 
  (\dropn{x})  \psubstp{Q}{P}       
  := 
  \left\{ 
    \begin{array}{ccc} 
      Q & & x \nameeq \quotep{P} \\
      \dropn{x} & & otherwise \\
    \end{array}
  \right.
\end{mathpar}
 

where

\begin{eqnarray}
  (x)\id{\{} \lpquote Q \rpquote / \lpquote P \rpquote \id{\}}            = 
  \left\{ 
    \begin{array}{ccc}
      \lpquote Q \rpquote & & x \nameeq \lpquote P \rpquote \\
      x & & otherwise \\
    \end{array}
  \right. \nonumber
\end{eqnarray}

and $z$ is chosen distinct from $\quotep{P}$, $\quotep{Q}$, the free
names in $Q$, and all the names in $R$. Our $\alpha$-equivalence will
be built in the standard way from this substitution.

\begin{remark}\label{rem:no_self_referential_names}
  One consequence of these definitions is that $\forall P. \quotep{P}
  \not\in \freenames{P}$.
\end{remark}

\subsection{ Dynamic quote: an example }

Anticipating something of what's to come, consider applying the
substitution, $\widehat{\id{\{}u / z \id{\}}}$, to the following pair
of processes, $\lift{w}{y!(z)}$ and $w[ \lpquote y!(z) \rpquote ]$.

\begin{eqnarray}
	\lift{w}{y!(z)}\widehat{\id{\{}u / z \id{\}}}
		& = &
		\lift{w}{y!(u)} \nonumber\\
	w[ \lpquote y!(z) \rpquote ] \widehat{ \id{\{}u / z \id{\}} }
		& = &
		w[ \lpquote y!(z) \rpquote ] \nonumber
\end{eqnarray}

Because the body of the process between quotes is impervious to
substitution, we get radically different answers. In fact, by
examining the first process in an input context,
e.g. $x?(z).\lift{w}{y!(z)}$, we see that the process under the lift
operator may be shaped by prefixed inputs binding a name inside it. In
this sense, the lift operator will be seen as a way to dynamically
construct processes before reifying them as names.

Finally equipped with these standard features we can present the
dynamics of the calculus.

\subsubsection{Operational semantics} 

Finally, we introduce the computational dynamics. What marks these
algebras as distinct from other more traditionally studied algebraic
structures, e.g. vector spaces or polynomial rings, is the manner in
which dynamics is captured. In traditional structures, dynamics is typically
expressed through morphisms between such structures, as in linear maps
between vector spaces or morphisms between rings. In algebras
associated with the semantics of computation, the dynamics is
expressed as part of the algebraic structure itself, through a
reduction reduction relation typically denoted by $\red$. Below, we
give a recursive presentation of this relation for the calculus used
in the encoding.

$\red \subseteq \pi \times \pi$
$\red : \pi \to \mathcal{P}(\pi)$

\begin{mathpar}
  \inferrule* [lab=Comm] { \textsf{match}( x_{src}, x_{trgt} ) } { x_{trgt}?(y)P \; | \; x_{src}!\langle {Q} \rangle \red P\{\quotep{Q}/y}\} }
  \and \\
  \inferrule* [lab=Par] {{P} \red {P}'} {{{P} | {Q}} \red {{P}' | {Q}}}
  \and
  \inferrule* [lab=Equiv]{{{P} \scong {P}'} \andalso {{P}' \red {Q}'} \andalso {{Q}' \scong {Q}}}{{P} \red {Q}}
\end{mathpar}

\begin{eqnarray*}
  match_{\equiv} (\quotep{P},\quotep{Q}) & := & P \equiv Q \\
  match_{\dagger}(\quotep{P},\quotep{Q}) & := & \forall R. P|Q \red^{*} R => R \red^{*} 0 \\
  match_{K}(\quotep{P},\quotep{Q}) & := & K \mbox{ for some context } K
\end{eqnarray*}

$u?(x)P | u!\langle Q \rangle \red P\{\quotep{Q}/x\}$

%We write $\wred$ for $\red^*$, and $P\red$ if $\exists Q $ such that $ P \red Q$.
We write $P\red$ if $\exists Q $ such that $ P \red Q$ and $P\not\red$, otherwise.

\section{Replication}

As mentioned before, it is known that replication (and hence
recursion) can be implemented in a higher-order process algebra
\cite{SangiorgiWalker}. As our first example of calculation with the
machinery thus far presented we give the construction explicitly in
the {\rhoc}.

\begin{eqnarray}
	D_{x} & := & \prefix{x}{y}{(\binpar{\outputp{x}{y}}{@{y}})} \nonumber\\
	\bangp_{x}{P} & := & \binpar{{x}!\langle{\binpar{D_{x}}{P}}\rangle}{D_{x}} \nonumber
\end{eqnarray}

\begin{eqnarray}
	\bangp_{x}{P} & & \nonumber\\
	=
	& {x}!\langle{(\prefix{x}{y}{(\outputp{x}{y} | @{y})) | P}}\rangle 
	      | \prefix{x}{y}{(\outputp{x}{y} | @{y})} & \nonumber\\
	\red
	& (\outputp{x}{y} | @{y})\substn{\quotep{(\prefix{x}{y}{(@{y} | \outputp{x}{y})) | P}}}{y} & \nonumber\\
	=
	& \outputp{x}{\quotep{(\prefix{x}{y}{(\outputp{x}{y} | @{y})) | P}}}
	  | {(\prefix{x}{y}{(\outputp{x}{y} | @{y})) | P}} & \nonumber\\
	\red
	& \ldots & \nonumber\\
	\red^*
	& P | P | \ldots & \nonumber
\end{eqnarray}

Of course, this encoding, as an implementation, runs away, unfolding
$\bangp{P}$ eagerly. A lazier and more implementable replication
operator, restricted to input-guarded processes, may be obtained as follows.

\begin{eqnarray}
\bangp{\prefix{u}{v}{P}} 
	:= 
	\binpar{\lift{x}{\prefix{u}{v}{(\binpar{D(x)}{P})}}}{D(x)} \nonumber
\end{eqnarray}

\begin{remark}
  Note that the lazier definition still does not deal with summation
  or mixed summation (i.e. sums over input and output). The reader is
  invited to construct definitions of replication that deal with these
  features. 

  Further, the definitions are parameterized in a name, $x$. Can you,
  gentle reader, make a definition that eliminates this parameter and
  guarantees no accidental interaction between the replication
  machinery and the process being replicated -- i.e. no accidental
  sharing of names used by the process to get its work done and the
  name(s) used by the replication to effect copying. This latter
  revision of the definition of replication is crucial to obtaining
  the expected identity $!!P \sim !P$.
\end{remark}

\begin{remark}\label{rem:paradoxical_combinator}
  The reader familiar with the lambda calculus will have noticed the
  similarity between $D$ and the paradoxical combinator.

  [Ed. note: the existence of this seems to suggest we have to be more
  restrictive on the set of processes and names we admit if we are to
  support no-cloning.]
\end{remark}

\subsubsection{Bisimulation}

The computational dynamics gives rise to another kind of equivalence,
the equivalence of computational behavior. As previously mentioned
this is typically captured \emph{via} some form of bisimulation.

% The notion we use in this paper is weak barbed bisimulation
% \cite{milner91polyadicpi}.

The notion we use in this paper is derived from weak barbed
bisimulation \cite{milner91polyadicpi}. 

\begin{definition}
An \emph{observation relation}, $\downarrow_{\mathcal N}$, over a set
of names, $\mathcal N$, is the smallest relation satisfying the rules
below.

\infrule[Out-barb]{y \in {\mathcal N}, \; x \nameeq y}
		  {\outputp{x}{v} \downarrow_{\mathcal N} x}
\infrule[Par-barb]{\mbox{$P\downarrow_{\mathcal N} x$ or $Q\downarrow_{\mathcal N} x$}}
		  {\binpar{P}{Q} \downarrow_{\mathcal N} x}

We write $P \Downarrow_{\mathcal N} x$ if there is $Q$ such that 
$P \wred Q$ and $Q \downarrow_{\mathcal N} x$.
\end{definition}

\begin{definition}
%\label{def.bbisim}
An  ${\mathcal N}$-\emph{barbed bisimulation} over a set of names, ${\mathcal N}$, is a symmetric binary relation 
${\mathcal S}_{\mathcal N}$ between agents such that $P\rel{S}_{\mathcal N}Q$ implies:
\begin{enumerate}
\item If $P \red P'$ then $Q \wred Q'$ and $P'\rel{S}_{\mathcal N} Q'$.
\item If $P\downarrow_{\mathcal N} x$, then $Q\Downarrow_{\mathcal N} x$.
\end{enumerate}
$P$ is ${\mathcal N}$-barbed bisimilar to $Q$, written
$P \wbbisim_{\mathcal N} Q$, if $P \rel{S}_{\mathcal N} Q$ for some ${\mathcal N}$-barbed bisimulation ${\mathcal S}_{\mathcal N}$.
\end{definition}

$\mathcal{R} \subseteq \pi \times \pi$

$P \mathcal{R} Q => \forall P'. P \red P' \Rightarrow \exists Q'. Q \red Q', P' \mathcal{R} Q'$

$P \vdash x \Rightarrow Q \vdash x$

\begin{mathpar}
  \inferrule*[lab=Out-barb]{x \nameeq y}{{y}!\langle{Q}\rangle \vdash x}
  \and
  \inferrule*[lab=Par-barb]{\mbox{$P\vdash x$ or $Q\vdash x$}}{\binpar{P}{Q} \vdash x}
\end{mathpar}

\subsubsection{Contexts}

One of the principle advantages of computational calculi like the
$\pi$-calculus is a well-defined notion of context,
contextual-equivalence and a correlation between
contextual-equivalence and notions of bisimulation. The notion of
context allows the decomposition of a process into (sub-)process and
its syntactic environment, its context. Thus, a context may be
thought of as a process with a ``hole'' (written $\Box$) in it. The
application of a context $M$ to a process $P$, written $M[P]$, is
tantamount to filling the hole in $M$ with $P$. In this paper we do
not need the full weight of this theory, but do make use of the notion
of context in the proof the main theorem. 

\begin{mathpar}
  \inferrule* [lab=summation] {} {{M_{M},M_{N}} \bc \Box \;|\; x.M_{A} \;|\; M_{M}+M_{N}}
  \and
  \inferrule* [lab=agent] {} {{M_{A}} \bc (\vec{x})M_{P} \;| \; \clift{P_0,\ldots,M_{P},\ldots,P_N}}
  \and \\
  \inferrule* [lab=process] {} {{M_{P}} \bc M_{N} \;| \;P|M_{P} }
\end{mathpar} 

\begin{mathpar}
  \inferrule* [lab=sychronization] {} {M_{N} \bc \Box \;|\; x?M_{F} \;|\; x!M_{C}}
  \and
  \inferrule* [lab=abstraction] {} {{M_{F}} \bc (x)M_{P} }
  \and
  \inferrule* [lab=concretion] {} {{M_{C}} \bc \langle M_{P} \rangle }
  \and \\
  \inferrule* [lab=process] {} {{M_{P}} \bc M_{N} \;| \;P|M_{P} }
\end{mathpar}

\begin{definition}[contextual application] Given a context $M$, and
  process $P$, we define the \emph{contextual application}, $M[P] :=
  M\{P/\Box\}$. That is, the contextual application of M to P is the
  substitution of $P$ for $\Box$ in $M$.
\end{definition}

$\meaningof{-} : L \to \mathcal{P}(\pi)$

\begin{mathpar}
  \inferrule* [lab=collection] {} {\meaningof{true} = \pi, \and \meaningof{~E} = \pi \setminus \meaningof{E}, \and \meaningof{E_{1} \& E_{2}} = \meaningof{E_{1}} \cap \meaningof{E_{2}}}
\end{mathpar}

\begin{mathpar}
  \inferrule* [lab=structure] {} {\meaningof{0} = \{ P \in \pi | P \equiv 0 \}, \and \\ \meaningof{E_1 | E_2} = \{ P \in \pi | P \equiv P_{1} | P_{2}, P_{1} \in \meaningof{E_{1}}, P_{2} \in \meaningof{E_2}\} }
\end{mathpar}

\begin{mathpar}
 \inferrule* [lab=behavior] {} {\meaningof{\langle a?b \rangle E} = \{ P \in \pi | P \equiv Q | u?(y)P', \\ \and \\\\ \and \\ \;\;\; u \in \meaningof{a}, \forall z.P'\{z/y\} \in \meaningof{E\{z/b\}}\}, \and \\ \meaningof{a!E} = \{ P \in \pi | P \equiv Q | x!\langle P' \rangle, x \in \meaningof{a} P' \in \meaningof{E}\} }
\end{mathpar}

\begin{mathpar}
 \inferrule* [lab=nominal] {} {\meaningof{\quotep{E}} = \{ \quotep{P} \in \quotep{\pi} | P \in \meaningof{E} \}, \and \meaningof{\quotep{P}} = \{ \quotep{Q} \in \quotep{\pi} | P \equiv Q \} \and \\ \meaningof{@\quotep{E}} = \{ P \in \pi | P \equiv @x, x \in \meaningof{E} \}}
\end{mathpar}

\begin{eqnarray*}
  \\
  \meaningof{-} : TS \to ST
\end{eqnarray*}

\begin{eqnarray*}
  \\
  L : TS \to ST
\end{eqnarray*}

\begin{eqnarray*}
  \\
  P \models E \iff P \in \meaningof{E}
\end{eqnarray*}

\begin{eqnarray*}
  P \approx_{L} Q \iff \forall E \in L. P \models E \iff Q \models E
\end{eqnarray*}

\begin{eqnarray*}
  P \approx_{K} Q
\end{eqnarray*}

\begin{eqnarray*}
  P \approx Q
\end{eqnarray*}

$\approx_{K} = \approx = \approx_{L}$

\subsubsection{Contextual duality}

Note that contexts extend the quotation operation to a family of
operations from processes to names. Given a context, $M$, we can
define a \emph{nominal context}, $\quotep{M}$ by $\quotep{M}[P] :=
\quotep{M[P]}$. To foreshadow what is to come we observe that these
operations enjoy a duality with processes very much like the duality
between vectors and maps from vectors to scalars.

Further, because the calculus is essentially higher-order, we have a
correspondence between contexts and processes. More specifically,
given a name $x$ and a context $M$ we can construct $M^{*}_{x}$ such
that 

\begin{mathpar}
  M^{*}_{x} | \lift{x}{P} \red M[P]
\end{mathpar}

namely,

\begin{mathpar}
  M^{*}_{x} := x?(u).M[\dropn{u}]
\end{mathpar}

The dependence of $M^{*}_{x}$ on a name makes it an abstraction, 

\begin{mathpar}
  M^{*} := (x)x?(u).M[\dropn{u}]
\end{mathpar}

\subsection{Additional notation}

It will sometimes be convenient to denote the process a name
quotes. We already have the notation $x = \quotep{P}$, but it will be
convenient to introduce an alternate notation, $\procn{x}$, when we
want to emphasize the connection to the use of the name. Note that, by
virtue of name equivalence, $\quotep{\procn{x}} \nameeq x$; so, the
notation is consistent with previous definitions.

Further, because names have structure it is possible to effect
substitutions on the basis of that structure. This means we need to
upgrade our notation for substitutions, which we accomplish by
adapting comprehension notation. Thus,

\begin{mathpar}
  P\{ y / x : x \in S \}
\end{mathpar}

is interpreted to mean the process derived from P by replacing (in a
capture-avoiding manner) each occurrence of $x$ in $S$ by $y$. For example,

\begin{mathpar}
  P\{ \quotep{\procn{x}|\procn{x}} / x : x \in \freenames{P} \}
\end{mathpar}

will replace each (occurrence) of a free name $x$ in $P$ by
$\quotep{\procn{x}|\procn{x}}$.

Also, we will avail ourselves of the notation $x^{L}$ and $x^{R}$ to
denote injections of a name into disjoint copies of the name
space. There are numerous ways to accomplish this. One example can be
found in \cite{MeredithR05}. This notation overloads to vectors of
names: $\vec{x}^{\pi} := (x_{i}^{\pi} \; : \; 0 \leq i < |\vec{x}| )$ where $\pi \in \{L,R\}$.

We also use $P^{\Box} := P|\Box$.

In \cite{MeredithR05} an interpretation of the new operator is
given. It turns out that there are several possible interpretations
all enjoying the requisite algebraic properties of the operator (see
\cite{milner91polyadicpi}). We will therefore make liberal use of
$(\nu\; \vec{x})P$.

% subsection the_syntax_and_semantics_of_the_notation_system (end)   

\input{qm2pi.qmops} 

\input{qm2pi.sterngerlach} 

\input{qm2pi.metric} 

% section concurrent_process_calculi (end)

%\input{qm2pi.proofsketch}

% section proof sketch (end)

%\input{qm2pi.slviaknots} 

% section spatial logic via knots (end)

\input{qm2pi.conclusion}

% section conclusion (end)

%\input{qm2pi.dtcodes} 

% section wiring algorithm (end)

\input{qm2pi.ack} 

% section acknowledgments (end)

\newpage


\bibliographystyle{plain}   
\bibliography{../../biblios/main.bib}

\input{qm2pi.rhodetails}

\end{document}

 

% subsection basic_interpretation (end)

%\input{qm2pi.rho.presentation} 
\subsection{The syntax and semantics of the notation system}\label{sub:the_syntax_and_semantics_of_the_notation_system} % (fold)

We now summarize a technical presentation of the calculus that
embodies our theory of dynamics. The typical presentation of such a
calculus follows the style of giving generators and relations on
them. The grammar, below, describing term constructors, freely
generates the set of processes, $\Proc$. This set is then quotiented
by a relation known as structural congruence and it is over this set
that the notion of dynamics is expressed. This presentation is
essentially that of \cite{MeredithR05} with the addition of
polyadicity and summation. For readability we have relegated some of
the technical subtleties to an appendix.

\subsubsection{Process grammar}\label{subsub:process_grammar}

\begin{mathpar}
  \inferrule* [lab=synchronization] {} {{M} \bc \pzero \;|\; x?F \;|\; x!C }
  \and
  \inferrule* [lab=abstraction] {} {{F} \bc (x)P}
  \and
  \inferrule* [lab=concretion] {} {{C} \bc \langle Q \rangle}
  \and
  \inferrule* [lab=process] {} {{P,Q} \bc M \;| \;P|Q \;|\; @{x}}
  \and
  \inferrule* [lab=name] {} {{x} \bc \quotep{P}}
\end{mathpar} 

Note that $\vec{x}$ (resp. $\vec{P}$) denotes a vector of names
(resp. processes) of length $|\vec{x}|$ (resp. $|\vec{P}|$). We adopt
the following useful abbreviations.

\begin{mathpar}
   x?(\vec{y}).P := x.(\vec{y})P \and  x\clift{\vec{P}} := x.\clift{\vec{P}}
   \and x!(y) := \lift{x}{\dropn{y}}
   \and \Pi_{i=0}^{n-1}P_i := P_0 | \ldots | P_{n-1}
\end{mathpar}

\subsubsection{Structural congruence}

\paragraph{Free and bound names and alpha-equivalence.} At the
core of structural equivalence is alpha-equivalence which identifies
process that are the same up to a change of variable. Formally, we
recognize the distinction between free and bound names. The free names
of a process, $\freenames{P}$, may be calculated recursively as
follows:

\begin{mathpar}
\freenames{\pzero} := \emptyset
  \and \\
  \freenames{x?(y).P} := \{ x \} \cup (\freenames{P} \setminus \{ y \})
  \and 
  \freenames{x!\langle P \rangle} := \{ x \} \cup \{ P \} 
  \and \\
  \freenames{P|Q} := \freenames{P} \cup \freenames{Q}
  \and \\
  \freenames{@{x}} := \{ x \}
\end{mathpar}

$\pi$
$\quotep{\pi}$

$\freenames{-} : \pi \to \mathcal{P}(\quotep{\pi})$

\begin{eqnarray*}
  \freenames{\pzero} & := & \emptyset \\
  \freenames{x?(y).P} & := & \{ x \} \cup (\freenames{P} \setminus \{ y \}) \\
  \freenames{x!\langle P \rangle} & := & \{ x \} \cup \{ P \} \\
  \freenames{P|Q} & := & \freenames{P} \cup \freenames{Q} \\
  \freenames{\dropn{x}} & := & \{ x \}
\end{eqnarray*}

The bound names of a process, $\boundnames{P}$, are those names occurring in $P$
that are not free. For example, in $x?(y).0$, the name $x$ is free, while $y$ is bound.

\begin{mathpar}
  \inferrule* [lab=monoidal-laws] {} { P|Q \equiv Q|P \and P|0 \equiv P \and P|(Q|R) \equiv (P|Q)|R }
\end{mathpar}

\begin{mathpar}
  \inferrule* [lab=alpha-equivalence] {} { (x)P \equiv (y)P\{y/x\} \and y \not\in \freenames{P} }
\end{mathpar}

\begin{definition}
Then two processes, $P,Q$, are alpha-equivalent if $P = Q\{\vec{y}/\vec{x}\}$ for
some $\vec{x} \in \boundnames{Q},\vec{y} \in \boundnames{P}$, where $Q\{\vec{y}/\vec{x}\}$
denotes the capture-avoiding substitution of $\vec{y}$ for $\vec{x}$ in $Q$.
\end{definition}

\begin{definition}
  The {\em structural congruence} \cite{SangiorgiWalker} , $\equiv$,
  between processes is the least congruence containing
  alpha-equivalence, satisfying the abelian monoid laws
  (associativity, commutativity and $\pzero$ as identity) for parallel
  composition $|$ and for summation $+$.
\end{definition}

\subsection{Name equivalence}

We take name equivalence, written $\nameeq$, to be the smallest
equivalence relation generated by the following rules.

\begin{mathpar}
\inferrule*[lab=Quote-drop]
{ }
{ \quotep{@{x}} \nameeq x }

\inferrule*[lab=Struct-equiv]
{ P \scong Q }
{ \quotep{P} \nameeq \quotep{Q} }
\end{mathpar}

The astute reader will have noticed that the mutual recursion of names
and processes imposes a mutual recursion on alpha-equivalence and
structural equivalence via name-equivalence. Fortunately, all of this
works out pleasantly and we may calculate in the natural way, free of
concern. The reader interested in the details is referred to the
appendix \ref{appendix:rho_details}.

\subsection{Substitution}

We use $\Proc$ for the set of processes, $\QProc$ for the set of
names, and $\id{\{}\vec{y} / \vec{x} \id{\}}$ to denote partial maps,
$s : \QProc \rightarrow \QProc$. A map, $s$ lifts, uniquely, to a map
on process terms, $\widehat{s} : \Proc \rightarrow \Proc$ by the
following equations.

\begin{mathpar}
  (0) \psubstp{Q}{P} := 0 \\
  (R \juxtap S) \psubstp{Q}{P}
  :=    
  (R)\psubstp{Q}{P} \juxtap (S) \psubstp{Q}{P} \\
  (x?(y).R) \psubstp{Q}{P}    
  :=    
  (x)\substp{Q}{P} (z)\concat( (R \psubstn{z}{y}) \psubstp{Q}{P} ) \\
  (\lift{x}{R}) \psubstp{Q}{P}  
  :=
  \lift{(x)\substp{Q}{P}}{ R \psubstp{Q}{P} } \\
%   (\dropn{x})  \psubstp{Q}{P}       
%   := 
%   \left\{ 
%     \begin{array}{ccc} 
%       \dropn{\quotep{Q}} & & x \nameeq \quotep{P} \\
%       \dropn{x} & & otherwise \\
%     \end{array}
%   \right. 
  (\dropn{x})  \psubstp{Q}{P}       
  := 
  \left\{ 
    \begin{array}{ccc} 
      Q & & x \nameeq \quotep{P} \\
      \dropn{x} & & otherwise \\
    \end{array}
  \right.
\end{mathpar}
 

where

\begin{eqnarray}
  (x)\id{\{} \lpquote Q \rpquote / \lpquote P \rpquote \id{\}}            = 
  \left\{ 
    \begin{array}{ccc}
      \lpquote Q \rpquote & & x \nameeq \lpquote P \rpquote \\
      x & & otherwise \\
    \end{array}
  \right. \nonumber
\end{eqnarray}

and $z$ is chosen distinct from $\quotep{P}$, $\quotep{Q}$, the free
names in $Q$, and all the names in $R$. Our $\alpha$-equivalence will
be built in the standard way from this substitution.

\begin{remark}\label{rem:no_self_referential_names}
  One consequence of these definitions is that $\forall P. \quotep{P}
  \not\in \freenames{P}$.
\end{remark}

\subsection{ Dynamic quote: an example }

Anticipating something of what's to come, consider applying the
substitution, $\widehat{\id{\{}u / z \id{\}}}$, to the following pair
of processes, $\lift{w}{y!(z)}$ and $w[ \lpquote y!(z) \rpquote ]$.

\begin{eqnarray}
	\lift{w}{y!(z)}\widehat{\id{\{}u / z \id{\}}}
		& = &
		\lift{w}{y!(u)} \nonumber\\
	w[ \lpquote y!(z) \rpquote ] \widehat{ \id{\{}u / z \id{\}} }
		& = &
		w[ \lpquote y!(z) \rpquote ] \nonumber
\end{eqnarray}

Because the body of the process between quotes is impervious to
substitution, we get radically different answers. In fact, by
examining the first process in an input context,
e.g. $x?(z).\lift{w}{y!(z)}$, we see that the process under the lift
operator may be shaped by prefixed inputs binding a name inside it. In
this sense, the lift operator will be seen as a way to dynamically
construct processes before reifying them as names.

Finally equipped with these standard features we can present the
dynamics of the calculus.

\subsubsection{Operational semantics} 

Finally, we introduce the computational dynamics. What marks these
algebras as distinct from other more traditionally studied algebraic
structures, e.g. vector spaces or polynomial rings, is the manner in
which dynamics is captured. In traditional structures, dynamics is typically
expressed through morphisms between such structures, as in linear maps
between vector spaces or morphisms between rings. In algebras
associated with the semantics of computation, the dynamics is
expressed as part of the algebraic structure itself, through a
reduction reduction relation typically denoted by $\red$. Below, we
give a recursive presentation of this relation for the calculus used
in the encoding.

$\red \subseteq \pi \times \pi$
$\red : \pi \to \mathcal{P}(\pi)$

\begin{mathpar}
  \inferrule* [lab=Comm] { \textsf{match}( x_{src}, x_{trgt} ) } { x_{trgt}?(y)P \; | \; x_{src}!\langle {Q} \rangle \red P\{\quotep{Q}/y}\} }
  \and \\
  \inferrule* [lab=Par] {{P} \red {P}'} {{{P} | {Q}} \red {{P}' | {Q}}}
  \and
  \inferrule* [lab=Equiv]{{{P} \scong {P}'} \andalso {{P}' \red {Q}'} \andalso {{Q}' \scong {Q}}}{{P} \red {Q}}
\end{mathpar}

\begin{eqnarray*}
  match_{\equiv} (\quotep{P},\quotep{Q}) & := & P \equiv Q \\
  match_{\dagger}(\quotep{P},\quotep{Q}) & := & \forall R. P|Q \red^{*} R => R \red^{*} 0 \\
  match_{K}(\quotep{P},\quotep{Q}) & := & K \mbox{ for some context } K
\end{eqnarray*}

$u?(x)P | u!\langle Q \rangle \red P\{\quotep{Q}/x\}$

%We write $\wred$ for $\red^*$, and $P\red$ if $\exists Q $ such that $ P \red Q$.
We write $P\red$ if $\exists Q $ such that $ P \red Q$ and $P\not\red$, otherwise.

\section{Replication}

As mentioned before, it is known that replication (and hence
recursion) can be implemented in a higher-order process algebra
\cite{SangiorgiWalker}. As our first example of calculation with the
machinery thus far presented we give the construction explicitly in
the {\rhoc}.

\begin{eqnarray}
	D_{x} & := & \prefix{x}{y}{(\binpar{\outputp{x}{y}}{@{y}})} \nonumber\\
	\bangp_{x}{P} & := & \binpar{{x}!\langle{\binpar{D_{x}}{P}}\rangle}{D_{x}} \nonumber
\end{eqnarray}

\begin{eqnarray}
	\bangp_{x}{P} & & \nonumber\\
	=
	& {x}!\langle{(\prefix{x}{y}{(\outputp{x}{y} | @{y})) | P}}\rangle 
	      | \prefix{x}{y}{(\outputp{x}{y} | @{y})} & \nonumber\\
	\red
	& (\outputp{x}{y} | @{y})\substn{\quotep{(\prefix{x}{y}{(@{y} | \outputp{x}{y})) | P}}}{y} & \nonumber\\
	=
	& \outputp{x}{\quotep{(\prefix{x}{y}{(\outputp{x}{y} | @{y})) | P}}}
	  | {(\prefix{x}{y}{(\outputp{x}{y} | @{y})) | P}} & \nonumber\\
	\red
	& \ldots & \nonumber\\
	\red^*
	& P | P | \ldots & \nonumber
\end{eqnarray}

Of course, this encoding, as an implementation, runs away, unfolding
$\bangp{P}$ eagerly. A lazier and more implementable replication
operator, restricted to input-guarded processes, may be obtained as follows.

\begin{eqnarray}
\bangp{\prefix{u}{v}{P}} 
	:= 
	\binpar{\lift{x}{\prefix{u}{v}{(\binpar{D(x)}{P})}}}{D(x)} \nonumber
\end{eqnarray}

\begin{remark}
  Note that the lazier definition still does not deal with summation
  or mixed summation (i.e. sums over input and output). The reader is
  invited to construct definitions of replication that deal with these
  features. 

  Further, the definitions are parameterized in a name, $x$. Can you,
  gentle reader, make a definition that eliminates this parameter and
  guarantees no accidental interaction between the replication
  machinery and the process being replicated -- i.e. no accidental
  sharing of names used by the process to get its work done and the
  name(s) used by the replication to effect copying. This latter
  revision of the definition of replication is crucial to obtaining
  the expected identity $!!P \sim !P$.
\end{remark}

\begin{remark}\label{rem:paradoxical_combinator}
  The reader familiar with the lambda calculus will have noticed the
  similarity between $D$ and the paradoxical combinator.

  [Ed. note: the existence of this seems to suggest we have to be more
  restrictive on the set of processes and names we admit if we are to
  support no-cloning.]
\end{remark}

\subsubsection{Bisimulation}

The computational dynamics gives rise to another kind of equivalence,
the equivalence of computational behavior. As previously mentioned
this is typically captured \emph{via} some form of bisimulation.

% The notion we use in this paper is weak barbed bisimulation
% \cite{milner91polyadicpi}.

The notion we use in this paper is derived from weak barbed
bisimulation \cite{milner91polyadicpi}. 

\begin{definition}
An \emph{observation relation}, $\downarrow_{\mathcal N}$, over a set
of names, $\mathcal N$, is the smallest relation satisfying the rules
below.

\infrule[Out-barb]{y \in {\mathcal N}, \; x \nameeq y}
		  {\outputp{x}{v} \downarrow_{\mathcal N} x}
\infrule[Par-barb]{\mbox{$P\downarrow_{\mathcal N} x$ or $Q\downarrow_{\mathcal N} x$}}
		  {\binpar{P}{Q} \downarrow_{\mathcal N} x}

We write $P \Downarrow_{\mathcal N} x$ if there is $Q$ such that 
$P \wred Q$ and $Q \downarrow_{\mathcal N} x$.
\end{definition}

\begin{definition}
%\label{def.bbisim}
An  ${\mathcal N}$-\emph{barbed bisimulation} over a set of names, ${\mathcal N}$, is a symmetric binary relation 
${\mathcal S}_{\mathcal N}$ between agents such that $P\rel{S}_{\mathcal N}Q$ implies:
\begin{enumerate}
\item If $P \red P'$ then $Q \wred Q'$ and $P'\rel{S}_{\mathcal N} Q'$.
\item If $P\downarrow_{\mathcal N} x$, then $Q\Downarrow_{\mathcal N} x$.
\end{enumerate}
$P$ is ${\mathcal N}$-barbed bisimilar to $Q$, written
$P \wbbisim_{\mathcal N} Q$, if $P \rel{S}_{\mathcal N} Q$ for some ${\mathcal N}$-barbed bisimulation ${\mathcal S}_{\mathcal N}$.
\end{definition}

$\mathcal{R} \subseteq \pi \times \pi$

$P \mathcal{R} Q => \forall P'. P \red P' \Rightarrow \exists Q'. Q \red Q', P' \mathcal{R} Q'$

$P \vdash x \Rightarrow Q \vdash x$

\begin{mathpar}
  \inferrule*[lab=Out-barb]{x \nameeq y}{{y}!\langle{Q}\rangle \vdash x}
  \and
  \inferrule*[lab=Par-barb]{\mbox{$P\vdash x$ or $Q\vdash x$}}{\binpar{P}{Q} \vdash x}
\end{mathpar}

\subsubsection{Contexts}

One of the principle advantages of computational calculi like the
$\pi$-calculus is a well-defined notion of context,
contextual-equivalence and a correlation between
contextual-equivalence and notions of bisimulation. The notion of
context allows the decomposition of a process into (sub-)process and
its syntactic environment, its context. Thus, a context may be
thought of as a process with a ``hole'' (written $\Box$) in it. The
application of a context $M$ to a process $P$, written $M[P]$, is
tantamount to filling the hole in $M$ with $P$. In this paper we do
not need the full weight of this theory, but do make use of the notion
of context in the proof the main theorem. 

\begin{mathpar}
  \inferrule* [lab=summation] {} {{M_{M},M_{N}} \bc \Box \;|\; x.M_{A} \;|\; M_{M}+M_{N}}
  \and
  \inferrule* [lab=agent] {} {{M_{A}} \bc (\vec{x})M_{P} \;| \; \clift{P_0,\ldots,M_{P},\ldots,P_N}}
  \and \\
  \inferrule* [lab=process] {} {{M_{P}} \bc M_{N} \;| \;P|M_{P} }
\end{mathpar} 

\begin{mathpar}
  \inferrule* [lab=sychronization] {} {M_{N} \bc \Box \;|\; x?M_{F} \;|\; x!M_{C}}
  \and
  \inferrule* [lab=abstraction] {} {{M_{F}} \bc (x)M_{P} }
  \and
  \inferrule* [lab=concretion] {} {{M_{C}} \bc \langle M_{P} \rangle }
  \and \\
  \inferrule* [lab=process] {} {{M_{P}} \bc M_{N} \;| \;P|M_{P} }
\end{mathpar}

\begin{definition}[contextual application] Given a context $M$, and
  process $P$, we define the \emph{contextual application}, $M[P] :=
  M\{P/\Box\}$. That is, the contextual application of M to P is the
  substitution of $P$ for $\Box$ in $M$.
\end{definition}

$\meaningof{-} : L \to \mathcal{P}(\pi)$

\begin{mathpar}
  \inferrule* [lab=collection] {} {\meaningof{true} = \pi, \and \meaningof{~E} = \pi \setminus \meaningof{E}, \and \meaningof{E_{1} \& E_{2}} = \meaningof{E_{1}} \cap \meaningof{E_{2}}}
\end{mathpar}

\begin{mathpar}
  \inferrule* [lab=structure] {} {\meaningof{0} = \{ P \in \pi | P \equiv 0 \}, \and \\ \meaningof{E_1 | E_2} = \{ P \in \pi | P \equiv P_{1} | P_{2}, P_{1} \in \meaningof{E_{1}}, P_{2} \in \meaningof{E_2}\} }
\end{mathpar}

\begin{mathpar}
 \inferrule* [lab=behavior] {} {\meaningof{\langle a?b \rangle E} = \{ P \in \pi | P \equiv Q | u?(y)P', \\ \and \\\\ \and \\ \;\;\; u \in \meaningof{a}, \forall z.P'\{z/y\} \in \meaningof{E\{z/b\}}\}, \and \\ \meaningof{a!E} = \{ P \in \pi | P \equiv Q | x!\langle P' \rangle, x \in \meaningof{a} P' \in \meaningof{E}\} }
\end{mathpar}

\begin{mathpar}
 \inferrule* [lab=nominal] {} {\meaningof{\quotep{E}} = \{ \quotep{P} \in \quotep{\pi} | P \in \meaningof{E} \}, \and \meaningof{\quotep{P}} = \{ \quotep{Q} \in \quotep{\pi} | P \equiv Q \} \and \\ \meaningof{@\quotep{E}} = \{ P \in \pi | P \equiv @x, x \in \meaningof{E} \}}
\end{mathpar}

\begin{eqnarray*}
  \\
  \meaningof{-} : TS \to ST
\end{eqnarray*}

\begin{eqnarray*}
  \\
  L : TS \to ST
\end{eqnarray*}

\begin{eqnarray*}
  \\
  P \models E \iff P \in \meaningof{E}
\end{eqnarray*}

\begin{eqnarray*}
  P \approx_{L} Q \iff \forall E \in L. P \models E \iff Q \models E
\end{eqnarray*}

\begin{eqnarray*}
  P \approx_{K} Q
\end{eqnarray*}

\begin{eqnarray*}
  P \approx Q
\end{eqnarray*}

$\approx_{K} = \approx = \approx_{L}$

\subsubsection{Contextual duality}

Note that contexts extend the quotation operation to a family of
operations from processes to names. Given a context, $M$, we can
define a \emph{nominal context}, $\quotep{M}$ by $\quotep{M}[P] :=
\quotep{M[P]}$. To foreshadow what is to come we observe that these
operations enjoy a duality with processes very much like the duality
between vectors and maps from vectors to scalars.

Further, because the calculus is essentially higher-order, we have a
correspondence between contexts and processes. More specifically,
given a name $x$ and a context $M$ we can construct $M^{*}_{x}$ such
that 

\begin{mathpar}
  M^{*}_{x} | \lift{x}{P} \red M[P]
\end{mathpar}

namely,

\begin{mathpar}
  M^{*}_{x} := x?(u).M[\dropn{u}]
\end{mathpar}

The dependence of $M^{*}_{x}$ on a name makes it an abstraction, 

\begin{mathpar}
  M^{*} := (x)x?(u).M[\dropn{u}]
\end{mathpar}

\subsection{Additional notation}

It will sometimes be convenient to denote the process a name
quotes. We already have the notation $x = \quotep{P}$, but it will be
convenient to introduce an alternate notation, $\procn{x}$, when we
want to emphasize the connection to the use of the name. Note that, by
virtue of name equivalence, $\quotep{\procn{x}} \nameeq x$; so, the
notation is consistent with previous definitions.

Further, because names have structure it is possible to effect
substitutions on the basis of that structure. This means we need to
upgrade our notation for substitutions, which we accomplish by
adapting comprehension notation. Thus,

\begin{mathpar}
  P\{ y / x : x \in S \}
\end{mathpar}

is interpreted to mean the process derived from P by replacing (in a
capture-avoiding manner) each occurrence of $x$ in $S$ by $y$. For example,

\begin{mathpar}
  P\{ \quotep{\procn{x}|\procn{x}} / x : x \in \freenames{P} \}
\end{mathpar}

will replace each (occurrence) of a free name $x$ in $P$ by
$\quotep{\procn{x}|\procn{x}}$.

Also, we will avail ourselves of the notation $x^{L}$ and $x^{R}$ to
denote injections of a name into disjoint copies of the name
space. There are numerous ways to accomplish this. One example can be
found in \cite{MeredithR05}. This notation overloads to vectors of
names: $\vec{x}^{\pi} := (x_{i}^{\pi} \; : \; 0 \leq i < |\vec{x}| )$ where $\pi \in \{L,R\}$.

We also use $P^{\Box} := P|\Box$.

In \cite{MeredithR05} an interpretation of the new operator is
given. It turns out that there are several possible interpretations
all enjoying the requisite algebraic properties of the operator (see
\cite{milner91polyadicpi}). We will therefore make liberal use of
$(\nu\; \vec{x})P$.

% subsection the_syntax_and_semantics_of_the_notation_system (end)   

\section{Interpretation of QM}
\subsection{Supporting definitions}
\subsubsection{Multiplication}
\begin{mathpar}
  \quotep{Q} \cdot \quotep{R} := \quotep{Q|R}
  \and \\
  \quotep{Q} \cdot P := P\{ \quotep{Q|R} / \quotep{R} : \quotep{R} \in \freenames{P} \}
\end{mathpar}

\paragraph{Discussion}
The first line needs little explanation. The second line says that
each free name of the process is replaced with the multiplication of
that name by the scalar. Multiplication of a scalar (name) by a state
(process) results in a process all the names of which have been `moved
over' by parallel composition with the process the scalar
quotes. There is a subtlety that the bound names have to be
manipulated so that multiplied names aren't accidentally
captured. There are many ways to achieve this.

\begin{remark}\label{rem:multiplication_identities}
  The reader is invited to verify that for all $x,y,z \in \QProc$ and $P \in \Proc$
  \begin{mathpar}
    x \cdot \quotep{0} \equiv x 
    \and
    x \cdot y \equiv y \cdot x
    \and
    x \cdot (y \cdot z) \equiv (x \cdot y) \cdot z
    \and \\
    \quotep{0} \cdot P \equiv P
    \and \\
    x \cdot (y \cdot P) \equiv (x \cdot y) \cdot P
    \and \\
    x \cdot (P|Q) \equiv (x \cdot P) | (x \cdot Q)
    \and \\    
  \end{mathpar}
\end{remark}

\subsubsection{Tensor product}

We define a tensor product on processes by structural induction.

\paragraph{Tensor of sums} First note that all summations, including
$\pzero$ and sequence, can be written $\Sigma_{i} x_{i}.A_{i} +
\Sigma_{j} x_{j}.C_{j}$, where we have grouped input-guarded processes
together and output-guarded processes together.

Thus, we can define the tensor product of two summations, $N_{1}\otimes N_{2}$, where

\begin{mathpar}
  N_{1} := \Sigma_{i} x_{i}.A_{i} + \Sigma_{j} x_{j}.C_{j}
  \and
  N_{2} := \Sigma_{i'} y_{i'}.B_{i'} + \Sigma_{j'} y_{j'}.D_{j'} 
\end{mathpar}

as follows.

\begin{mathpar}
  \Sigma_{i} x_{i}.A_{i} + \Sigma_{j} x_{j}.C_{j} \otimes \Sigma_{i'}
  y_{i'}.B_{i'} + \Sigma_{j'} y_{j'}.D_{j'} 
  \and \\
  := \; \Sigma_{i} \Sigma_{i'} \quotep{\stackrel{\vee}{x_{i}}| \stackrel{\vee}{y_{i'}}}.(A_{i}\otimes B_{i'}) \; | \; \Sigma_{i'} \Sigma_{i} \quotep{\stackrel{\vee}{y_{i'}}|\stackrel{\vee}{x_{i}}}.(B_{i'}\otimes A_{i})
  \and
  \;\; | \;\; \Sigma_{j} \Sigma_{j'} \quotep{\stackrel{\vee}{x_{j}}|\stackrel{\vee}{y_{j'}}}.(A_{j}\otimes B_{j'}) \; | \; \Sigma_{j'} \Sigma_{j} \quotep{\stackrel{\vee}{y_{j'}}|\stackrel{\vee}{x_{j}}}.(B_{j'}\otimes A_{j})
\end{mathpar}

\begin{remark}
  Do we need to $x^{L}$ and $y^{R}$ for this construction as well?
\end{remark}

\paragraph{Tensor of parallel compositions} Next, we distribute tensor
over par.

\begin{mathpar}
  P_{1}|P_{2} \otimes Q_{1}|Q_{2} := (P_{1} \otimes Q_{1}) | (P_{1}
  \otimes Q_{2}) | (P_{2} \otimes Q_{1}) | (P_{2} \otimes Q_{2})
\end{mathpar}

\paragraph{Tensor with dropped names} We treat tensor of a
process with a dropped name as parallel composition.

\begin{mathpar}
  P \otimes \dropn{x} := P | \dropn{x}
\end{mathpar}

\paragraph{Tensor of agents}

Finally, we need to define tensor on agents. Note that the definition
of tensor on normal products only tensors inputs with inputs and
outputs with outputs. Thus, we only have to define the operation on
``homogeneous'' pairings.

\begin{mathpar}
  (\vec{x})P \otimes (\vec{y})Q
  \and \\
  := (x_{0}^{L}|y_{0}^{R},\ldots,x_{0}^{L}|y_{n}^{R},\ldots,x_{m}^{L}|y_{0}^{R},\ldots,x_{m}^{L}|y_{n}^R)(P\{ \vec{x}^{L}/\vec{x}\} \otimes Q \{ \vec{y}^{R}/\vec{y}\})
  \and \\
  \clift{\vec{P}} \otimes \clift{\vec{Q}}
  \and \\
  := \clift{P_{0}\otimes Q_{0},\ldots,P_{0}\otimes Q_{n},\ldots,P_{m}\otimes Q_{0},\ldots,P_{m}\otimes Q_{n}}
\end{mathpar}

\begin{remark}
  Observe that arities of tensored abstractions matches arities of
  tensored concretions if the original arities matched. Note also that
  the length of the arities corresponds to the increase in dimension
  we see in ordinary vector space tensor product.
\end{remark}

\begin{remark}
  Operationally, this definition distributes the tensor down to
  components ``linked'' by summation. Tensor over summation is
  intriguing in that it mixes names. Moreover, as a consequence of the
  way it mixes names we have the identities for all $x \in \QProc$ and
  $P,Q \in \Proc$

  \begin{mathpar}
    (x \cdot P) \otimes Q \equiv x \cdot (P \otimes Q) \equiv P \otimes (x \cdot Q)
    \and
    P \otimes \pzero \equiv P
  \end{mathpar}

  that the reader is invited to verify.
\end{remark}

\subsubsection{Annihilation}
\begin{mathpar}
  P^{\perp} := \{ Q | \forall R. P|Q \red^{*} R \Rightarrow R \red^{*} \pzero \}
  \and \\
  P^{\underline{\perp}} := \Sigma_{Q \in P^{\perp}} \quotep{Q}?(y).(\dropn{y}|Q) | \Sigma_{Q \in P^{\perp}} \quotep{Q}\clift{\Box}
\end{mathpar}

\paragraph{Discussion} The reader will note that $P^{\perp}$ is a
\emph{set} of processes, while $P^{\underline{\perp}}$ is a
\emph{context}. We call the set $P^{\perp}$ the \emph{annihilators} of
$P$. The parallel composition of a process in the annihilators of $P$
with $P$ will result in a process, the state space of which has all
paths eventually leading to $\pzero$. Execution may endure loops; but
under reasonable conditions of fairness (naturally guaranteed under
most notions of bisimulation) such a composite process cannot get
stuck in such a loop and will, eventually pop out and terminate.

The context $P^{\underline{\perp}}$ is ready and willing to ``take the
$P$ out of'' the process to which it is applied. It will effectively
transmit the code of the process to which it is applied to one of the
annihilators and run the process against it.

\subsubsection{Evaluation}
We fix $M$ a domain of fully abstract interpretation with an equality
coincident with bisimulation. We take $\meaningof{\cdot} : \Proc \to
M$ to be the map interpreting processes and $\nmeaningof{\cdot} : \M
\to Proc$ to be the map running the other way. Then we define

\begin{mathpar}
  \int P := \nmeaningof{\meaningof{P}}
\end{mathpar}

\paragraph{Discussion}
There are many fully abstract interpretations of Milner's
$\pi$-calculus. Any of them can be used as a basis for interpreting
the reflective calculus here. Equipped with such a domain it is
largely a matter of grinding through to check that the Yoneda
construction for the normalization-by-evaluation program can be
extended to this setting.

\begin{remark}
  The reader is invited to verify that $\int (P^{\underline{\perp}}[P]) = 0$.
\end{remark}

\subsection{Quantum mechanics}

Table \ref{tbl:core_qm_op_defns} gives the core operational definitions

\begin{table}[htp]\label{tbl:core_qm_op_defns}
  \center{
    \fbox{
      \begin{tabular}{c|c}
        quantum mechanics & process calculus \\
        \hline
        scalar & $x := \quotep{P}$ \\
        state vector & $\state{P} := P$ \\
        dual & $\state{P}^{*} := \event{P^{\underline{\perp}}} := \quotep{P^{\underline{\perp}}}[-]$ \\
        matrix & $ \Sigma_{\alpha} \state{P_{\alpha}}x_{\alpha}\event{Q_{\alpha}}$ \\
        vector addition & $\state{P} + \state{Q} := \state{P | Q}$ \\
        tensor product & $\state{P} \otimes \state{Q} := \state{P \otimes Q}$ \\
        inner product & $\innerprod{P}{Q} := \quotep{\int P^{\underline{\perp}}[Q]}$ \\
      \end{tabular}
    }
  }
  \caption{QM - operational definitions}
\end{table}

where

\begin{mathpar}
  \prmatrix{P}{Q} := \fprmatrix{P}{\quotep{\pzero}}{Q}
  \and
  \fprmatrix{P}{x}{Q} := (\state{P},x,\event{Q})
  \and
  (\fprmatrix{P}{x}{Q})(\state{R}) := x \cdot \innerprod{Q}{R} \cdot \state{P}
  \and
  (\fprmatrix{P}{x}{Q})(\event{R}) := x \cdot \innerprod{R}{P} \cdot \event{Q}
\end{mathpar}

\paragraph{Discussion}
As promised: vectors (aka states) are represented as processes; duals
as contextual duals; inner product definition should be compared with
standard inner product definition for ....

\begin{remark}
  Assuming $\int (P^{\underline{\perp}}[P]) = 0$, the reader is
  invited to verify that $(\fprmatrix{P}{x}{P})(\state{P}) = x \cdot \state{P}$.
\end{remark}

\begin{remark}
  The reader is invited to verify that $\innerprod{P}{Q}$ could
  equally well have been written $\quotep{\int \stackrel{\vee}{x}}$
  where $x = \event{P^{\underline{\perp}}}(Q)$.

  One of the motivations for this remark is that there is another way
  to factor these operations. We could package up evaluation in the dual:

  \begin{mathpar}
    \state{P}^{*} := \event{\int P^{\underline{\perp}}} := \quotep{\int P^{\underline{\perp}}}[-]
  \end{mathpar}

  and then have inner product defined by
  
  \begin{mathpar}
    \innerprod{P}{Q} := \event{P}(Q)
  \end{mathpar}

  Hopefully, experience with the calculations will provide guidance on
  the best factoring.
\end{remark}

\begin{remark}
  Assuming $\int (P^{\underline{\perp}}[P]) = 0$, the reader is
  invited to verify that $\forall P,Q. (\prmatrix{0}{Q})(\state{0}) =
  \state{0}$ and dually $(\prmatrix{P}{0})(\event{0}) = \event{0}$.
\end{remark}

\begin{remark}
  i'm a little worried that i don't (yet) have proper support for
  complex conjugacy. But, the observation above may give us a
  clue. According to Abramsky, it must be the case that the scalars
  are iso to the homset of the identity for the tensor -- which the
  observation above characterizes. 

  For now, we will simply bookmark the notion with $\overline{x}$.
\end{remark}

\subsubsection{Adjointness}

We need to give a definition of $(\cdot)^{\dagger}$ for matrices. The
obvious candidate definition is
\begin{mathpar}
(\Sigma_{\alpha}\fprmatrix{P_{\alpha}}{x_{\alpha}}{Q_{\alpha}})^{\dagger}
= \Sigma_{\alpha}\fprmatrix{(Q_{\alpha}^{\underline{\perp}})^{*}}{\overline{x}_{\alpha}}{P_{\alpha}^{\underline{\perp}}} 
\end{mathpar}

But, $(Q_{\alpha}^{\underline{\perp}})^{*}$ requires a name along
which to communicate the process to achieve the context application.

\subsubsection{Basis for a basis}
If processes label states and ``addition'' of states (a.k.a. vector
addition) is interpreted as parallel composition, what corresponds to
notions of linear independence and basis? Here, we recall that Yoshida
has developed a set of \emph{combinators} for an asynchronous verison
of Milner's $\pi$-calculus. These are a finite set of processes such
any process can be expressed as parallel composition of these
combinators together with liberal uses of the new operator and
replication. We can simply give a translation of these into the
present calculus and have reasonable expectation that the property
carries over. That is, that the resultant set allows to express all
processes via parallel composition. Note, however, that there is no
new operator or replication in this calculus. As a result, we expect
that the corresponding set is actually infinite. That is, we expect
that the space is actually infinite dimensional.

\begin{remark}
  The attentive reader may be a bit concerned. Certainly, the
  collection $S$, $K$ and $I$ is a finite set of
  combinators. Shouldn't we expect to see a finite set of combinators
  for an effectively equivalent system? i am very sympathetic to this
  critique and feel it warrants full attention. On the other hand, i
  also have in mind the following analogy. The natural numbers, as a
  monoid under addition, has exactly $1$ generator, while the natural
  numbers, as a monoid under multiplication, has countably many
  generators (the primes). We observe that the application of the
  lambda calculus is much less resource sensitive than the parallel
  composition of the $\pi$-calculus. Could it be the case that we have
  an analogy of the form
  
  \begin{mathpar}
    m + n : MN :: m*n : M|N
  \end{mathpar}

  giving a similar blow up in the set of ``primes''?  This is such a
  wonderful thought that, even if it's not true, i think it's worth
  writing down.
\end{remark}
 

\documentclass[12pt]{llncs}
%\documentclass{jktr}

\usepackage[pdftex]{hyperref}                   
\usepackage {listings}
\usepackage {mathpartir}
\usepackage{bcprules}
%\usepackage{listings}
                       
\usepackage{graphicx} 
%\usepackage[margins=2.5cm,nohead,nofoot]{geometry}
%\usepackage{geometry}
\usepackage{amsfonts}
\usepackage{amstext}
\usepackage{latexsym}
\usepackage{amssymb}
\usepackage{color}


%\include{myPreamble}
\include{qm2pi.local} 

%\ifpdf
%\usepackage[pdftex]{graphicx}
%\else
%\usepackage{graphicx}
%\fi

 % \ifpdf
%  \usepackage{pdfsync}
%  \if


%\title{Brief Article}
%\author{David F. Snyder}
%\author{L.G. Meredith}

%\address{Dept. of Math., Texas State University--San Marcos, San Marcos, TX 78666}
       
\pagestyle{empty}


\begin{document}

\lstset{language=[Objective]Caml,frame=shadowbox}

\input{qm2pi.front}

% section front matter (end)

\input{qm2pi.intro} 
 
% section introduction (end)

% \input{qm2pi.knotations} 

% section notation (end)

\input{qm2pi.process.calculi} 

% section concurrent_process_calculi_and_spatial_logics_ (end)
    
%\input{qm2pi.knots2pi} 

%\input{qm2pi.trefoil} 

%\input{qm2pi.mainthm} 

% subsection basic_interpretation (end)

%\input{qm2pi.rho.presentation} 
\subsection{The syntax and semantics of the notation system}\label{sub:the_syntax_and_semantics_of_the_notation_system} % (fold)

We now summarize a technical presentation of the calculus that
embodies our theory of dynamics. The typical presentation of such a
calculus follows the style of giving generators and relations on
them. The grammar, below, describing term constructors, freely
generates the set of processes, $\Proc$. This set is then quotiented
by a relation known as structural congruence and it is over this set
that the notion of dynamics is expressed. This presentation is
essentially that of \cite{MeredithR05} with the addition of
polyadicity and summation. For readability we have relegated some of
the technical subtleties to an appendix.

\subsubsection{Process grammar}\label{subsub:process_grammar}

\begin{mathpar}
  \inferrule* [lab=synchronization] {} {{M} \bc \pzero \;|\; x?F \;|\; x!C }
  \and
  \inferrule* [lab=abstraction] {} {{F} \bc (x)P}
  \and
  \inferrule* [lab=concretion] {} {{C} \bc \langle Q \rangle}
  \and
  \inferrule* [lab=process] {} {{P,Q} \bc M \;| \;P|Q \;|\; @{x}}
  \and
  \inferrule* [lab=name] {} {{x} \bc \quotep{P}}
\end{mathpar} 

Note that $\vec{x}$ (resp. $\vec{P}$) denotes a vector of names
(resp. processes) of length $|\vec{x}|$ (resp. $|\vec{P}|$). We adopt
the following useful abbreviations.

\begin{mathpar}
   x?(\vec{y}).P := x.(\vec{y})P \and  x\clift{\vec{P}} := x.\clift{\vec{P}}
   \and x!(y) := \lift{x}{\dropn{y}}
   \and \Pi_{i=0}^{n-1}P_i := P_0 | \ldots | P_{n-1}
\end{mathpar}

\subsubsection{Structural congruence}

\paragraph{Free and bound names and alpha-equivalence.} At the
core of structural equivalence is alpha-equivalence which identifies
process that are the same up to a change of variable. Formally, we
recognize the distinction between free and bound names. The free names
of a process, $\freenames{P}$, may be calculated recursively as
follows:

\begin{mathpar}
\freenames{\pzero} := \emptyset
  \and \\
  \freenames{x?(y).P} := \{ x \} \cup (\freenames{P} \setminus \{ y \})
  \and 
  \freenames{x!\langle P \rangle} := \{ x \} \cup \{ P \} 
  \and \\
  \freenames{P|Q} := \freenames{P} \cup \freenames{Q}
  \and \\
  \freenames{@{x}} := \{ x \}
\end{mathpar}

$\pi$
$\quotep{\pi}$

$\freenames{-} : \pi \to \mathcal{P}(\quotep{\pi})$

\begin{eqnarray*}
  \freenames{\pzero} & := & \emptyset \\
  \freenames{x?(y).P} & := & \{ x \} \cup (\freenames{P} \setminus \{ y \}) \\
  \freenames{x!\langle P \rangle} & := & \{ x \} \cup \{ P \} \\
  \freenames{P|Q} & := & \freenames{P} \cup \freenames{Q} \\
  \freenames{\dropn{x}} & := & \{ x \}
\end{eqnarray*}

The bound names of a process, $\boundnames{P}$, are those names occurring in $P$
that are not free. For example, in $x?(y).0$, the name $x$ is free, while $y$ is bound.

\begin{mathpar}
  \inferrule* [lab=monoidal-laws] {} { P|Q \equiv Q|P \and P|0 \equiv P \and P|(Q|R) \equiv (P|Q)|R }
\end{mathpar}

\begin{mathpar}
  \inferrule* [lab=alpha-equivalence] {} { (x)P \equiv (y)P\{y/x\} \and y \not\in \freenames{P} }
\end{mathpar}

\begin{definition}
Then two processes, $P,Q$, are alpha-equivalent if $P = Q\{\vec{y}/\vec{x}\}$ for
some $\vec{x} \in \boundnames{Q},\vec{y} \in \boundnames{P}$, where $Q\{\vec{y}/\vec{x}\}$
denotes the capture-avoiding substitution of $\vec{y}$ for $\vec{x}$ in $Q$.
\end{definition}

\begin{definition}
  The {\em structural congruence} \cite{SangiorgiWalker} , $\equiv$,
  between processes is the least congruence containing
  alpha-equivalence, satisfying the abelian monoid laws
  (associativity, commutativity and $\pzero$ as identity) for parallel
  composition $|$ and for summation $+$.
\end{definition}

\subsection{Name equivalence}

We take name equivalence, written $\nameeq$, to be the smallest
equivalence relation generated by the following rules.

\begin{mathpar}
\inferrule*[lab=Quote-drop]
{ }
{ \quotep{@{x}} \nameeq x }

\inferrule*[lab=Struct-equiv]
{ P \scong Q }
{ \quotep{P} \nameeq \quotep{Q} }
\end{mathpar}

The astute reader will have noticed that the mutual recursion of names
and processes imposes a mutual recursion on alpha-equivalence and
structural equivalence via name-equivalence. Fortunately, all of this
works out pleasantly and we may calculate in the natural way, free of
concern. The reader interested in the details is referred to the
appendix \ref{appendix:rho_details}.

\subsection{Substitution}

We use $\Proc$ for the set of processes, $\QProc$ for the set of
names, and $\id{\{}\vec{y} / \vec{x} \id{\}}$ to denote partial maps,
$s : \QProc \rightarrow \QProc$. A map, $s$ lifts, uniquely, to a map
on process terms, $\widehat{s} : \Proc \rightarrow \Proc$ by the
following equations.

\begin{mathpar}
  (0) \psubstp{Q}{P} := 0 \\
  (R \juxtap S) \psubstp{Q}{P}
  :=    
  (R)\psubstp{Q}{P} \juxtap (S) \psubstp{Q}{P} \\
  (x?(y).R) \psubstp{Q}{P}    
  :=    
  (x)\substp{Q}{P} (z)\concat( (R \psubstn{z}{y}) \psubstp{Q}{P} ) \\
  (\lift{x}{R}) \psubstp{Q}{P}  
  :=
  \lift{(x)\substp{Q}{P}}{ R \psubstp{Q}{P} } \\
%   (\dropn{x})  \psubstp{Q}{P}       
%   := 
%   \left\{ 
%     \begin{array}{ccc} 
%       \dropn{\quotep{Q}} & & x \nameeq \quotep{P} \\
%       \dropn{x} & & otherwise \\
%     \end{array}
%   \right. 
  (\dropn{x})  \psubstp{Q}{P}       
  := 
  \left\{ 
    \begin{array}{ccc} 
      Q & & x \nameeq \quotep{P} \\
      \dropn{x} & & otherwise \\
    \end{array}
  \right.
\end{mathpar}
 

where

\begin{eqnarray}
  (x)\id{\{} \lpquote Q \rpquote / \lpquote P \rpquote \id{\}}            = 
  \left\{ 
    \begin{array}{ccc}
      \lpquote Q \rpquote & & x \nameeq \lpquote P \rpquote \\
      x & & otherwise \\
    \end{array}
  \right. \nonumber
\end{eqnarray}

and $z$ is chosen distinct from $\quotep{P}$, $\quotep{Q}$, the free
names in $Q$, and all the names in $R$. Our $\alpha$-equivalence will
be built in the standard way from this substitution.

\begin{remark}\label{rem:no_self_referential_names}
  One consequence of these definitions is that $\forall P. \quotep{P}
  \not\in \freenames{P}$.
\end{remark}

\subsection{ Dynamic quote: an example }

Anticipating something of what's to come, consider applying the
substitution, $\widehat{\id{\{}u / z \id{\}}}$, to the following pair
of processes, $\lift{w}{y!(z)}$ and $w[ \lpquote y!(z) \rpquote ]$.

\begin{eqnarray}
	\lift{w}{y!(z)}\widehat{\id{\{}u / z \id{\}}}
		& = &
		\lift{w}{y!(u)} \nonumber\\
	w[ \lpquote y!(z) \rpquote ] \widehat{ \id{\{}u / z \id{\}} }
		& = &
		w[ \lpquote y!(z) \rpquote ] \nonumber
\end{eqnarray}

Because the body of the process between quotes is impervious to
substitution, we get radically different answers. In fact, by
examining the first process in an input context,
e.g. $x?(z).\lift{w}{y!(z)}$, we see that the process under the lift
operator may be shaped by prefixed inputs binding a name inside it. In
this sense, the lift operator will be seen as a way to dynamically
construct processes before reifying them as names.

Finally equipped with these standard features we can present the
dynamics of the calculus.

\subsubsection{Operational semantics} 

Finally, we introduce the computational dynamics. What marks these
algebras as distinct from other more traditionally studied algebraic
structures, e.g. vector spaces or polynomial rings, is the manner in
which dynamics is captured. In traditional structures, dynamics is typically
expressed through morphisms between such structures, as in linear maps
between vector spaces or morphisms between rings. In algebras
associated with the semantics of computation, the dynamics is
expressed as part of the algebraic structure itself, through a
reduction reduction relation typically denoted by $\red$. Below, we
give a recursive presentation of this relation for the calculus used
in the encoding.

$\red \subseteq \pi \times \pi$
$\red : \pi \to \mathcal{P}(\pi)$

\begin{mathpar}
  \inferrule* [lab=Comm] { \textsf{match}( x_{src}, x_{trgt} ) } { x_{trgt}?(y)P \; | \; x_{src}!\langle {Q} \rangle \red P\{\quotep{Q}/y}\} }
  \and \\
  \inferrule* [lab=Par] {{P} \red {P}'} {{{P} | {Q}} \red {{P}' | {Q}}}
  \and
  \inferrule* [lab=Equiv]{{{P} \scong {P}'} \andalso {{P}' \red {Q}'} \andalso {{Q}' \scong {Q}}}{{P} \red {Q}}
\end{mathpar}

\begin{eqnarray*}
  match_{\equiv} (\quotep{P},\quotep{Q}) & := & P \equiv Q \\
  match_{\dagger}(\quotep{P},\quotep{Q}) & := & \forall R. P|Q \red^{*} R => R \red^{*} 0 \\
  match_{K}(\quotep{P},\quotep{Q}) & := & K \mbox{ for some context } K
\end{eqnarray*}

$u?(x)P | u!\langle Q \rangle \red P\{\quotep{Q}/x\}$

%We write $\wred$ for $\red^*$, and $P\red$ if $\exists Q $ such that $ P \red Q$.
We write $P\red$ if $\exists Q $ such that $ P \red Q$ and $P\not\red$, otherwise.

\section{Replication}

As mentioned before, it is known that replication (and hence
recursion) can be implemented in a higher-order process algebra
\cite{SangiorgiWalker}. As our first example of calculation with the
machinery thus far presented we give the construction explicitly in
the {\rhoc}.

\begin{eqnarray}
	D_{x} & := & \prefix{x}{y}{(\binpar{\outputp{x}{y}}{@{y}})} \nonumber\\
	\bangp_{x}{P} & := & \binpar{{x}!\langle{\binpar{D_{x}}{P}}\rangle}{D_{x}} \nonumber
\end{eqnarray}

\begin{eqnarray}
	\bangp_{x}{P} & & \nonumber\\
	=
	& {x}!\langle{(\prefix{x}{y}{(\outputp{x}{y} | @{y})) | P}}\rangle 
	      | \prefix{x}{y}{(\outputp{x}{y} | @{y})} & \nonumber\\
	\red
	& (\outputp{x}{y} | @{y})\substn{\quotep{(\prefix{x}{y}{(@{y} | \outputp{x}{y})) | P}}}{y} & \nonumber\\
	=
	& \outputp{x}{\quotep{(\prefix{x}{y}{(\outputp{x}{y} | @{y})) | P}}}
	  | {(\prefix{x}{y}{(\outputp{x}{y} | @{y})) | P}} & \nonumber\\
	\red
	& \ldots & \nonumber\\
	\red^*
	& P | P | \ldots & \nonumber
\end{eqnarray}

Of course, this encoding, as an implementation, runs away, unfolding
$\bangp{P}$ eagerly. A lazier and more implementable replication
operator, restricted to input-guarded processes, may be obtained as follows.

\begin{eqnarray}
\bangp{\prefix{u}{v}{P}} 
	:= 
	\binpar{\lift{x}{\prefix{u}{v}{(\binpar{D(x)}{P})}}}{D(x)} \nonumber
\end{eqnarray}

\begin{remark}
  Note that the lazier definition still does not deal with summation
  or mixed summation (i.e. sums over input and output). The reader is
  invited to construct definitions of replication that deal with these
  features. 

  Further, the definitions are parameterized in a name, $x$. Can you,
  gentle reader, make a definition that eliminates this parameter and
  guarantees no accidental interaction between the replication
  machinery and the process being replicated -- i.e. no accidental
  sharing of names used by the process to get its work done and the
  name(s) used by the replication to effect copying. This latter
  revision of the definition of replication is crucial to obtaining
  the expected identity $!!P \sim !P$.
\end{remark}

\begin{remark}\label{rem:paradoxical_combinator}
  The reader familiar with the lambda calculus will have noticed the
  similarity between $D$ and the paradoxical combinator.

  [Ed. note: the existence of this seems to suggest we have to be more
  restrictive on the set of processes and names we admit if we are to
  support no-cloning.]
\end{remark}

\subsubsection{Bisimulation}

The computational dynamics gives rise to another kind of equivalence,
the equivalence of computational behavior. As previously mentioned
this is typically captured \emph{via} some form of bisimulation.

% The notion we use in this paper is weak barbed bisimulation
% \cite{milner91polyadicpi}.

The notion we use in this paper is derived from weak barbed
bisimulation \cite{milner91polyadicpi}. 

\begin{definition}
An \emph{observation relation}, $\downarrow_{\mathcal N}$, over a set
of names, $\mathcal N$, is the smallest relation satisfying the rules
below.

\infrule[Out-barb]{y \in {\mathcal N}, \; x \nameeq y}
		  {\outputp{x}{v} \downarrow_{\mathcal N} x}
\infrule[Par-barb]{\mbox{$P\downarrow_{\mathcal N} x$ or $Q\downarrow_{\mathcal N} x$}}
		  {\binpar{P}{Q} \downarrow_{\mathcal N} x}

We write $P \Downarrow_{\mathcal N} x$ if there is $Q$ such that 
$P \wred Q$ and $Q \downarrow_{\mathcal N} x$.
\end{definition}

\begin{definition}
%\label{def.bbisim}
An  ${\mathcal N}$-\emph{barbed bisimulation} over a set of names, ${\mathcal N}$, is a symmetric binary relation 
${\mathcal S}_{\mathcal N}$ between agents such that $P\rel{S}_{\mathcal N}Q$ implies:
\begin{enumerate}
\item If $P \red P'$ then $Q \wred Q'$ and $P'\rel{S}_{\mathcal N} Q'$.
\item If $P\downarrow_{\mathcal N} x$, then $Q\Downarrow_{\mathcal N} x$.
\end{enumerate}
$P$ is ${\mathcal N}$-barbed bisimilar to $Q$, written
$P \wbbisim_{\mathcal N} Q$, if $P \rel{S}_{\mathcal N} Q$ for some ${\mathcal N}$-barbed bisimulation ${\mathcal S}_{\mathcal N}$.
\end{definition}

$\mathcal{R} \subseteq \pi \times \pi$

$P \mathcal{R} Q => \forall P'. P \red P' \Rightarrow \exists Q'. Q \red Q', P' \mathcal{R} Q'$

$P \vdash x \Rightarrow Q \vdash x$

\begin{mathpar}
  \inferrule*[lab=Out-barb]{x \nameeq y}{{y}!\langle{Q}\rangle \vdash x}
  \and
  \inferrule*[lab=Par-barb]{\mbox{$P\vdash x$ or $Q\vdash x$}}{\binpar{P}{Q} \vdash x}
\end{mathpar}

\subsubsection{Contexts}

One of the principle advantages of computational calculi like the
$\pi$-calculus is a well-defined notion of context,
contextual-equivalence and a correlation between
contextual-equivalence and notions of bisimulation. The notion of
context allows the decomposition of a process into (sub-)process and
its syntactic environment, its context. Thus, a context may be
thought of as a process with a ``hole'' (written $\Box$) in it. The
application of a context $M$ to a process $P$, written $M[P]$, is
tantamount to filling the hole in $M$ with $P$. In this paper we do
not need the full weight of this theory, but do make use of the notion
of context in the proof the main theorem. 

\begin{mathpar}
  \inferrule* [lab=summation] {} {{M_{M},M_{N}} \bc \Box \;|\; x.M_{A} \;|\; M_{M}+M_{N}}
  \and
  \inferrule* [lab=agent] {} {{M_{A}} \bc (\vec{x})M_{P} \;| \; \clift{P_0,\ldots,M_{P},\ldots,P_N}}
  \and \\
  \inferrule* [lab=process] {} {{M_{P}} \bc M_{N} \;| \;P|M_{P} }
\end{mathpar} 

\begin{mathpar}
  \inferrule* [lab=sychronization] {} {M_{N} \bc \Box \;|\; x?M_{F} \;|\; x!M_{C}}
  \and
  \inferrule* [lab=abstraction] {} {{M_{F}} \bc (x)M_{P} }
  \and
  \inferrule* [lab=concretion] {} {{M_{C}} \bc \langle M_{P} \rangle }
  \and \\
  \inferrule* [lab=process] {} {{M_{P}} \bc M_{N} \;| \;P|M_{P} }
\end{mathpar}

\begin{definition}[contextual application] Given a context $M$, and
  process $P$, we define the \emph{contextual application}, $M[P] :=
  M\{P/\Box\}$. That is, the contextual application of M to P is the
  substitution of $P$ for $\Box$ in $M$.
\end{definition}

$\meaningof{-} : L \to \mathcal{P}(\pi)$

\begin{mathpar}
  \inferrule* [lab=collection] {} {\meaningof{true} = \pi, \and \meaningof{~E} = \pi \setminus \meaningof{E}, \and \meaningof{E_{1} \& E_{2}} = \meaningof{E_{1}} \cap \meaningof{E_{2}}}
\end{mathpar}

\begin{mathpar}
  \inferrule* [lab=structure] {} {\meaningof{0} = \{ P \in \pi | P \equiv 0 \}, \and \\ \meaningof{E_1 | E_2} = \{ P \in \pi | P \equiv P_{1} | P_{2}, P_{1} \in \meaningof{E_{1}}, P_{2} \in \meaningof{E_2}\} }
\end{mathpar}

\begin{mathpar}
 \inferrule* [lab=behavior] {} {\meaningof{\langle a?b \rangle E} = \{ P \in \pi | P \equiv Q | u?(y)P', \\ \and \\\\ \and \\ \;\;\; u \in \meaningof{a}, \forall z.P'\{z/y\} \in \meaningof{E\{z/b\}}\}, \and \\ \meaningof{a!E} = \{ P \in \pi | P \equiv Q | x!\langle P' \rangle, x \in \meaningof{a} P' \in \meaningof{E}\} }
\end{mathpar}

\begin{mathpar}
 \inferrule* [lab=nominal] {} {\meaningof{\quotep{E}} = \{ \quotep{P} \in \quotep{\pi} | P \in \meaningof{E} \}, \and \meaningof{\quotep{P}} = \{ \quotep{Q} \in \quotep{\pi} | P \equiv Q \} \and \\ \meaningof{@\quotep{E}} = \{ P \in \pi | P \equiv @x, x \in \meaningof{E} \}}
\end{mathpar}

\begin{eqnarray*}
  \\
  \meaningof{-} : TS \to ST
\end{eqnarray*}

\begin{eqnarray*}
  \\
  L : TS \to ST
\end{eqnarray*}

\begin{eqnarray*}
  \\
  P \models E \iff P \in \meaningof{E}
\end{eqnarray*}

\begin{eqnarray*}
  P \approx_{L} Q \iff \forall E \in L. P \models E \iff Q \models E
\end{eqnarray*}

\begin{eqnarray*}
  P \approx_{K} Q
\end{eqnarray*}

\begin{eqnarray*}
  P \approx Q
\end{eqnarray*}

$\approx_{K} = \approx = \approx_{L}$

\subsubsection{Contextual duality}

Note that contexts extend the quotation operation to a family of
operations from processes to names. Given a context, $M$, we can
define a \emph{nominal context}, $\quotep{M}$ by $\quotep{M}[P] :=
\quotep{M[P]}$. To foreshadow what is to come we observe that these
operations enjoy a duality with processes very much like the duality
between vectors and maps from vectors to scalars.

Further, because the calculus is essentially higher-order, we have a
correspondence between contexts and processes. More specifically,
given a name $x$ and a context $M$ we can construct $M^{*}_{x}$ such
that 

\begin{mathpar}
  M^{*}_{x} | \lift{x}{P} \red M[P]
\end{mathpar}

namely,

\begin{mathpar}
  M^{*}_{x} := x?(u).M[\dropn{u}]
\end{mathpar}

The dependence of $M^{*}_{x}$ on a name makes it an abstraction, 

\begin{mathpar}
  M^{*} := (x)x?(u).M[\dropn{u}]
\end{mathpar}

\subsection{Additional notation}

It will sometimes be convenient to denote the process a name
quotes. We already have the notation $x = \quotep{P}$, but it will be
convenient to introduce an alternate notation, $\procn{x}$, when we
want to emphasize the connection to the use of the name. Note that, by
virtue of name equivalence, $\quotep{\procn{x}} \nameeq x$; so, the
notation is consistent with previous definitions.

Further, because names have structure it is possible to effect
substitutions on the basis of that structure. This means we need to
upgrade our notation for substitutions, which we accomplish by
adapting comprehension notation. Thus,

\begin{mathpar}
  P\{ y / x : x \in S \}
\end{mathpar}

is interpreted to mean the process derived from P by replacing (in a
capture-avoiding manner) each occurrence of $x$ in $S$ by $y$. For example,

\begin{mathpar}
  P\{ \quotep{\procn{x}|\procn{x}} / x : x \in \freenames{P} \}
\end{mathpar}

will replace each (occurrence) of a free name $x$ in $P$ by
$\quotep{\procn{x}|\procn{x}}$.

Also, we will avail ourselves of the notation $x^{L}$ and $x^{R}$ to
denote injections of a name into disjoint copies of the name
space. There are numerous ways to accomplish this. One example can be
found in \cite{MeredithR05}. This notation overloads to vectors of
names: $\vec{x}^{\pi} := (x_{i}^{\pi} \; : \; 0 \leq i < |\vec{x}| )$ where $\pi \in \{L,R\}$.

We also use $P^{\Box} := P|\Box$.

In \cite{MeredithR05} an interpretation of the new operator is
given. It turns out that there are several possible interpretations
all enjoying the requisite algebraic properties of the operator (see
\cite{milner91polyadicpi}). We will therefore make liberal use of
$(\nu\; \vec{x})P$.

% subsection the_syntax_and_semantics_of_the_notation_system (end)   

\input{qm2pi.qmops} 

\input{qm2pi.sterngerlach} 

\input{qm2pi.metric} 

% section concurrent_process_calculi (end)

%\input{qm2pi.proofsketch}

% section proof sketch (end)

%\input{qm2pi.slviaknots} 

% section spatial logic via knots (end)

\input{qm2pi.conclusion}

% section conclusion (end)

%\input{qm2pi.dtcodes} 

% section wiring algorithm (end)

\input{qm2pi.ack} 

% section acknowledgments (end)

\newpage


\bibliographystyle{plain}   
\bibliography{../../biblios/main.bib}

\input{qm2pi.rhodetails}

\end{document}

 

\documentclass[12pt]{llncs}
%\documentclass{jktr}

\usepackage[pdftex]{hyperref}                   
\usepackage {listings}
\usepackage {mathpartir}
\usepackage{bcprules}
%\usepackage{listings}
                       
\usepackage{graphicx} 
%\usepackage[margins=2.5cm,nohead,nofoot]{geometry}
%\usepackage{geometry}
\usepackage{amsfonts}
\usepackage{amstext}
\usepackage{latexsym}
\usepackage{amssymb}
\usepackage{color}


%\include{myPreamble}
\include{qm2pi.local} 

%\ifpdf
%\usepackage[pdftex]{graphicx}
%\else
%\usepackage{graphicx}
%\fi

 % \ifpdf
%  \usepackage{pdfsync}
%  \if


%\title{Brief Article}
%\author{David F. Snyder}
%\author{L.G. Meredith}

%\address{Dept. of Math., Texas State University--San Marcos, San Marcos, TX 78666}
       
\pagestyle{empty}


\begin{document}

\lstset{language=[Objective]Caml,frame=shadowbox}

\input{qm2pi.front}

% section front matter (end)

\input{qm2pi.intro} 
 
% section introduction (end)

% \input{qm2pi.knotations} 

% section notation (end)

\input{qm2pi.process.calculi} 

% section concurrent_process_calculi_and_spatial_logics_ (end)
    
%\input{qm2pi.knots2pi} 

%\input{qm2pi.trefoil} 

%\input{qm2pi.mainthm} 

% subsection basic_interpretation (end)

%\input{qm2pi.rho.presentation} 
\subsection{The syntax and semantics of the notation system}\label{sub:the_syntax_and_semantics_of_the_notation_system} % (fold)

We now summarize a technical presentation of the calculus that
embodies our theory of dynamics. The typical presentation of such a
calculus follows the style of giving generators and relations on
them. The grammar, below, describing term constructors, freely
generates the set of processes, $\Proc$. This set is then quotiented
by a relation known as structural congruence and it is over this set
that the notion of dynamics is expressed. This presentation is
essentially that of \cite{MeredithR05} with the addition of
polyadicity and summation. For readability we have relegated some of
the technical subtleties to an appendix.

\subsubsection{Process grammar}\label{subsub:process_grammar}

\begin{mathpar}
  \inferrule* [lab=synchronization] {} {{M} \bc \pzero \;|\; x?F \;|\; x!C }
  \and
  \inferrule* [lab=abstraction] {} {{F} \bc (x)P}
  \and
  \inferrule* [lab=concretion] {} {{C} \bc \langle Q \rangle}
  \and
  \inferrule* [lab=process] {} {{P,Q} \bc M \;| \;P|Q \;|\; @{x}}
  \and
  \inferrule* [lab=name] {} {{x} \bc \quotep{P}}
\end{mathpar} 

Note that $\vec{x}$ (resp. $\vec{P}$) denotes a vector of names
(resp. processes) of length $|\vec{x}|$ (resp. $|\vec{P}|$). We adopt
the following useful abbreviations.

\begin{mathpar}
   x?(\vec{y}).P := x.(\vec{y})P \and  x\clift{\vec{P}} := x.\clift{\vec{P}}
   \and x!(y) := \lift{x}{\dropn{y}}
   \and \Pi_{i=0}^{n-1}P_i := P_0 | \ldots | P_{n-1}
\end{mathpar}

\subsubsection{Structural congruence}

\paragraph{Free and bound names and alpha-equivalence.} At the
core of structural equivalence is alpha-equivalence which identifies
process that are the same up to a change of variable. Formally, we
recognize the distinction between free and bound names. The free names
of a process, $\freenames{P}$, may be calculated recursively as
follows:

\begin{mathpar}
\freenames{\pzero} := \emptyset
  \and \\
  \freenames{x?(y).P} := \{ x \} \cup (\freenames{P} \setminus \{ y \})
  \and 
  \freenames{x!\langle P \rangle} := \{ x \} \cup \{ P \} 
  \and \\
  \freenames{P|Q} := \freenames{P} \cup \freenames{Q}
  \and \\
  \freenames{@{x}} := \{ x \}
\end{mathpar}

$\pi$
$\quotep{\pi}$

$\freenames{-} : \pi \to \mathcal{P}(\quotep{\pi})$

\begin{eqnarray*}
  \freenames{\pzero} & := & \emptyset \\
  \freenames{x?(y).P} & := & \{ x \} \cup (\freenames{P} \setminus \{ y \}) \\
  \freenames{x!\langle P \rangle} & := & \{ x \} \cup \{ P \} \\
  \freenames{P|Q} & := & \freenames{P} \cup \freenames{Q} \\
  \freenames{\dropn{x}} & := & \{ x \}
\end{eqnarray*}

The bound names of a process, $\boundnames{P}$, are those names occurring in $P$
that are not free. For example, in $x?(y).0$, the name $x$ is free, while $y$ is bound.

\begin{mathpar}
  \inferrule* [lab=monoidal-laws] {} { P|Q \equiv Q|P \and P|0 \equiv P \and P|(Q|R) \equiv (P|Q)|R }
\end{mathpar}

\begin{mathpar}
  \inferrule* [lab=alpha-equivalence] {} { (x)P \equiv (y)P\{y/x\} \and y \not\in \freenames{P} }
\end{mathpar}

\begin{definition}
Then two processes, $P,Q$, are alpha-equivalent if $P = Q\{\vec{y}/\vec{x}\}$ for
some $\vec{x} \in \boundnames{Q},\vec{y} \in \boundnames{P}$, where $Q\{\vec{y}/\vec{x}\}$
denotes the capture-avoiding substitution of $\vec{y}$ for $\vec{x}$ in $Q$.
\end{definition}

\begin{definition}
  The {\em structural congruence} \cite{SangiorgiWalker} , $\equiv$,
  between processes is the least congruence containing
  alpha-equivalence, satisfying the abelian monoid laws
  (associativity, commutativity and $\pzero$ as identity) for parallel
  composition $|$ and for summation $+$.
\end{definition}

\subsection{Name equivalence}

We take name equivalence, written $\nameeq$, to be the smallest
equivalence relation generated by the following rules.

\begin{mathpar}
\inferrule*[lab=Quote-drop]
{ }
{ \quotep{@{x}} \nameeq x }

\inferrule*[lab=Struct-equiv]
{ P \scong Q }
{ \quotep{P} \nameeq \quotep{Q} }
\end{mathpar}

The astute reader will have noticed that the mutual recursion of names
and processes imposes a mutual recursion on alpha-equivalence and
structural equivalence via name-equivalence. Fortunately, all of this
works out pleasantly and we may calculate in the natural way, free of
concern. The reader interested in the details is referred to the
appendix \ref{appendix:rho_details}.

\subsection{Substitution}

We use $\Proc$ for the set of processes, $\QProc$ for the set of
names, and $\id{\{}\vec{y} / \vec{x} \id{\}}$ to denote partial maps,
$s : \QProc \rightarrow \QProc$. A map, $s$ lifts, uniquely, to a map
on process terms, $\widehat{s} : \Proc \rightarrow \Proc$ by the
following equations.

\begin{mathpar}
  (0) \psubstp{Q}{P} := 0 \\
  (R \juxtap S) \psubstp{Q}{P}
  :=    
  (R)\psubstp{Q}{P} \juxtap (S) \psubstp{Q}{P} \\
  (x?(y).R) \psubstp{Q}{P}    
  :=    
  (x)\substp{Q}{P} (z)\concat( (R \psubstn{z}{y}) \psubstp{Q}{P} ) \\
  (\lift{x}{R}) \psubstp{Q}{P}  
  :=
  \lift{(x)\substp{Q}{P}}{ R \psubstp{Q}{P} } \\
%   (\dropn{x})  \psubstp{Q}{P}       
%   := 
%   \left\{ 
%     \begin{array}{ccc} 
%       \dropn{\quotep{Q}} & & x \nameeq \quotep{P} \\
%       \dropn{x} & & otherwise \\
%     \end{array}
%   \right. 
  (\dropn{x})  \psubstp{Q}{P}       
  := 
  \left\{ 
    \begin{array}{ccc} 
      Q & & x \nameeq \quotep{P} \\
      \dropn{x} & & otherwise \\
    \end{array}
  \right.
\end{mathpar}
 

where

\begin{eqnarray}
  (x)\id{\{} \lpquote Q \rpquote / \lpquote P \rpquote \id{\}}            = 
  \left\{ 
    \begin{array}{ccc}
      \lpquote Q \rpquote & & x \nameeq \lpquote P \rpquote \\
      x & & otherwise \\
    \end{array}
  \right. \nonumber
\end{eqnarray}

and $z$ is chosen distinct from $\quotep{P}$, $\quotep{Q}$, the free
names in $Q$, and all the names in $R$. Our $\alpha$-equivalence will
be built in the standard way from this substitution.

\begin{remark}\label{rem:no_self_referential_names}
  One consequence of these definitions is that $\forall P. \quotep{P}
  \not\in \freenames{P}$.
\end{remark}

\subsection{ Dynamic quote: an example }

Anticipating something of what's to come, consider applying the
substitution, $\widehat{\id{\{}u / z \id{\}}}$, to the following pair
of processes, $\lift{w}{y!(z)}$ and $w[ \lpquote y!(z) \rpquote ]$.

\begin{eqnarray}
	\lift{w}{y!(z)}\widehat{\id{\{}u / z \id{\}}}
		& = &
		\lift{w}{y!(u)} \nonumber\\
	w[ \lpquote y!(z) \rpquote ] \widehat{ \id{\{}u / z \id{\}} }
		& = &
		w[ \lpquote y!(z) \rpquote ] \nonumber
\end{eqnarray}

Because the body of the process between quotes is impervious to
substitution, we get radically different answers. In fact, by
examining the first process in an input context,
e.g. $x?(z).\lift{w}{y!(z)}$, we see that the process under the lift
operator may be shaped by prefixed inputs binding a name inside it. In
this sense, the lift operator will be seen as a way to dynamically
construct processes before reifying them as names.

Finally equipped with these standard features we can present the
dynamics of the calculus.

\subsubsection{Operational semantics} 

Finally, we introduce the computational dynamics. What marks these
algebras as distinct from other more traditionally studied algebraic
structures, e.g. vector spaces or polynomial rings, is the manner in
which dynamics is captured. In traditional structures, dynamics is typically
expressed through morphisms between such structures, as in linear maps
between vector spaces or morphisms between rings. In algebras
associated with the semantics of computation, the dynamics is
expressed as part of the algebraic structure itself, through a
reduction reduction relation typically denoted by $\red$. Below, we
give a recursive presentation of this relation for the calculus used
in the encoding.

$\red \subseteq \pi \times \pi$
$\red : \pi \to \mathcal{P}(\pi)$

\begin{mathpar}
  \inferrule* [lab=Comm] { \textsf{match}( x_{src}, x_{trgt} ) } { x_{trgt}?(y)P \; | \; x_{src}!\langle {Q} \rangle \red P\{\quotep{Q}/y}\} }
  \and \\
  \inferrule* [lab=Par] {{P} \red {P}'} {{{P} | {Q}} \red {{P}' | {Q}}}
  \and
  \inferrule* [lab=Equiv]{{{P} \scong {P}'} \andalso {{P}' \red {Q}'} \andalso {{Q}' \scong {Q}}}{{P} \red {Q}}
\end{mathpar}

\begin{eqnarray*}
  match_{\equiv} (\quotep{P},\quotep{Q}) & := & P \equiv Q \\
  match_{\dagger}(\quotep{P},\quotep{Q}) & := & \forall R. P|Q \red^{*} R => R \red^{*} 0 \\
  match_{K}(\quotep{P},\quotep{Q}) & := & K \mbox{ for some context } K
\end{eqnarray*}

$u?(x)P | u!\langle Q \rangle \red P\{\quotep{Q}/x\}$

%We write $\wred$ for $\red^*$, and $P\red$ if $\exists Q $ such that $ P \red Q$.
We write $P\red$ if $\exists Q $ such that $ P \red Q$ and $P\not\red$, otherwise.

\section{Replication}

As mentioned before, it is known that replication (and hence
recursion) can be implemented in a higher-order process algebra
\cite{SangiorgiWalker}. As our first example of calculation with the
machinery thus far presented we give the construction explicitly in
the {\rhoc}.

\begin{eqnarray}
	D_{x} & := & \prefix{x}{y}{(\binpar{\outputp{x}{y}}{@{y}})} \nonumber\\
	\bangp_{x}{P} & := & \binpar{{x}!\langle{\binpar{D_{x}}{P}}\rangle}{D_{x}} \nonumber
\end{eqnarray}

\begin{eqnarray}
	\bangp_{x}{P} & & \nonumber\\
	=
	& {x}!\langle{(\prefix{x}{y}{(\outputp{x}{y} | @{y})) | P}}\rangle 
	      | \prefix{x}{y}{(\outputp{x}{y} | @{y})} & \nonumber\\
	\red
	& (\outputp{x}{y} | @{y})\substn{\quotep{(\prefix{x}{y}{(@{y} | \outputp{x}{y})) | P}}}{y} & \nonumber\\
	=
	& \outputp{x}{\quotep{(\prefix{x}{y}{(\outputp{x}{y} | @{y})) | P}}}
	  | {(\prefix{x}{y}{(\outputp{x}{y} | @{y})) | P}} & \nonumber\\
	\red
	& \ldots & \nonumber\\
	\red^*
	& P | P | \ldots & \nonumber
\end{eqnarray}

Of course, this encoding, as an implementation, runs away, unfolding
$\bangp{P}$ eagerly. A lazier and more implementable replication
operator, restricted to input-guarded processes, may be obtained as follows.

\begin{eqnarray}
\bangp{\prefix{u}{v}{P}} 
	:= 
	\binpar{\lift{x}{\prefix{u}{v}{(\binpar{D(x)}{P})}}}{D(x)} \nonumber
\end{eqnarray}

\begin{remark}
  Note that the lazier definition still does not deal with summation
  or mixed summation (i.e. sums over input and output). The reader is
  invited to construct definitions of replication that deal with these
  features. 

  Further, the definitions are parameterized in a name, $x$. Can you,
  gentle reader, make a definition that eliminates this parameter and
  guarantees no accidental interaction between the replication
  machinery and the process being replicated -- i.e. no accidental
  sharing of names used by the process to get its work done and the
  name(s) used by the replication to effect copying. This latter
  revision of the definition of replication is crucial to obtaining
  the expected identity $!!P \sim !P$.
\end{remark}

\begin{remark}\label{rem:paradoxical_combinator}
  The reader familiar with the lambda calculus will have noticed the
  similarity between $D$ and the paradoxical combinator.

  [Ed. note: the existence of this seems to suggest we have to be more
  restrictive on the set of processes and names we admit if we are to
  support no-cloning.]
\end{remark}

\subsubsection{Bisimulation}

The computational dynamics gives rise to another kind of equivalence,
the equivalence of computational behavior. As previously mentioned
this is typically captured \emph{via} some form of bisimulation.

% The notion we use in this paper is weak barbed bisimulation
% \cite{milner91polyadicpi}.

The notion we use in this paper is derived from weak barbed
bisimulation \cite{milner91polyadicpi}. 

\begin{definition}
An \emph{observation relation}, $\downarrow_{\mathcal N}$, over a set
of names, $\mathcal N$, is the smallest relation satisfying the rules
below.

\infrule[Out-barb]{y \in {\mathcal N}, \; x \nameeq y}
		  {\outputp{x}{v} \downarrow_{\mathcal N} x}
\infrule[Par-barb]{\mbox{$P\downarrow_{\mathcal N} x$ or $Q\downarrow_{\mathcal N} x$}}
		  {\binpar{P}{Q} \downarrow_{\mathcal N} x}

We write $P \Downarrow_{\mathcal N} x$ if there is $Q$ such that 
$P \wred Q$ and $Q \downarrow_{\mathcal N} x$.
\end{definition}

\begin{definition}
%\label{def.bbisim}
An  ${\mathcal N}$-\emph{barbed bisimulation} over a set of names, ${\mathcal N}$, is a symmetric binary relation 
${\mathcal S}_{\mathcal N}$ between agents such that $P\rel{S}_{\mathcal N}Q$ implies:
\begin{enumerate}
\item If $P \red P'$ then $Q \wred Q'$ and $P'\rel{S}_{\mathcal N} Q'$.
\item If $P\downarrow_{\mathcal N} x$, then $Q\Downarrow_{\mathcal N} x$.
\end{enumerate}
$P$ is ${\mathcal N}$-barbed bisimilar to $Q$, written
$P \wbbisim_{\mathcal N} Q$, if $P \rel{S}_{\mathcal N} Q$ for some ${\mathcal N}$-barbed bisimulation ${\mathcal S}_{\mathcal N}$.
\end{definition}

$\mathcal{R} \subseteq \pi \times \pi$

$P \mathcal{R} Q => \forall P'. P \red P' \Rightarrow \exists Q'. Q \red Q', P' \mathcal{R} Q'$

$P \vdash x \Rightarrow Q \vdash x$

\begin{mathpar}
  \inferrule*[lab=Out-barb]{x \nameeq y}{{y}!\langle{Q}\rangle \vdash x}
  \and
  \inferrule*[lab=Par-barb]{\mbox{$P\vdash x$ or $Q\vdash x$}}{\binpar{P}{Q} \vdash x}
\end{mathpar}

\subsubsection{Contexts}

One of the principle advantages of computational calculi like the
$\pi$-calculus is a well-defined notion of context,
contextual-equivalence and a correlation between
contextual-equivalence and notions of bisimulation. The notion of
context allows the decomposition of a process into (sub-)process and
its syntactic environment, its context. Thus, a context may be
thought of as a process with a ``hole'' (written $\Box$) in it. The
application of a context $M$ to a process $P$, written $M[P]$, is
tantamount to filling the hole in $M$ with $P$. In this paper we do
not need the full weight of this theory, but do make use of the notion
of context in the proof the main theorem. 

\begin{mathpar}
  \inferrule* [lab=summation] {} {{M_{M},M_{N}} \bc \Box \;|\; x.M_{A} \;|\; M_{M}+M_{N}}
  \and
  \inferrule* [lab=agent] {} {{M_{A}} \bc (\vec{x})M_{P} \;| \; \clift{P_0,\ldots,M_{P},\ldots,P_N}}
  \and \\
  \inferrule* [lab=process] {} {{M_{P}} \bc M_{N} \;| \;P|M_{P} }
\end{mathpar} 

\begin{mathpar}
  \inferrule* [lab=sychronization] {} {M_{N} \bc \Box \;|\; x?M_{F} \;|\; x!M_{C}}
  \and
  \inferrule* [lab=abstraction] {} {{M_{F}} \bc (x)M_{P} }
  \and
  \inferrule* [lab=concretion] {} {{M_{C}} \bc \langle M_{P} \rangle }
  \and \\
  \inferrule* [lab=process] {} {{M_{P}} \bc M_{N} \;| \;P|M_{P} }
\end{mathpar}

\begin{definition}[contextual application] Given a context $M$, and
  process $P$, we define the \emph{contextual application}, $M[P] :=
  M\{P/\Box\}$. That is, the contextual application of M to P is the
  substitution of $P$ for $\Box$ in $M$.
\end{definition}

$\meaningof{-} : L \to \mathcal{P}(\pi)$

\begin{mathpar}
  \inferrule* [lab=collection] {} {\meaningof{true} = \pi, \and \meaningof{~E} = \pi \setminus \meaningof{E}, \and \meaningof{E_{1} \& E_{2}} = \meaningof{E_{1}} \cap \meaningof{E_{2}}}
\end{mathpar}

\begin{mathpar}
  \inferrule* [lab=structure] {} {\meaningof{0} = \{ P \in \pi | P \equiv 0 \}, \and \\ \meaningof{E_1 | E_2} = \{ P \in \pi | P \equiv P_{1} | P_{2}, P_{1} \in \meaningof{E_{1}}, P_{2} \in \meaningof{E_2}\} }
\end{mathpar}

\begin{mathpar}
 \inferrule* [lab=behavior] {} {\meaningof{\langle a?b \rangle E} = \{ P \in \pi | P \equiv Q | u?(y)P', \\ \and \\\\ \and \\ \;\;\; u \in \meaningof{a}, \forall z.P'\{z/y\} \in \meaningof{E\{z/b\}}\}, \and \\ \meaningof{a!E} = \{ P \in \pi | P \equiv Q | x!\langle P' \rangle, x \in \meaningof{a} P' \in \meaningof{E}\} }
\end{mathpar}

\begin{mathpar}
 \inferrule* [lab=nominal] {} {\meaningof{\quotep{E}} = \{ \quotep{P} \in \quotep{\pi} | P \in \meaningof{E} \}, \and \meaningof{\quotep{P}} = \{ \quotep{Q} \in \quotep{\pi} | P \equiv Q \} \and \\ \meaningof{@\quotep{E}} = \{ P \in \pi | P \equiv @x, x \in \meaningof{E} \}}
\end{mathpar}

\begin{eqnarray*}
  \\
  \meaningof{-} : TS \to ST
\end{eqnarray*}

\begin{eqnarray*}
  \\
  L : TS \to ST
\end{eqnarray*}

\begin{eqnarray*}
  \\
  P \models E \iff P \in \meaningof{E}
\end{eqnarray*}

\begin{eqnarray*}
  P \approx_{L} Q \iff \forall E \in L. P \models E \iff Q \models E
\end{eqnarray*}

\begin{eqnarray*}
  P \approx_{K} Q
\end{eqnarray*}

\begin{eqnarray*}
  P \approx Q
\end{eqnarray*}

$\approx_{K} = \approx = \approx_{L}$

\subsubsection{Contextual duality}

Note that contexts extend the quotation operation to a family of
operations from processes to names. Given a context, $M$, we can
define a \emph{nominal context}, $\quotep{M}$ by $\quotep{M}[P] :=
\quotep{M[P]}$. To foreshadow what is to come we observe that these
operations enjoy a duality with processes very much like the duality
between vectors and maps from vectors to scalars.

Further, because the calculus is essentially higher-order, we have a
correspondence between contexts and processes. More specifically,
given a name $x$ and a context $M$ we can construct $M^{*}_{x}$ such
that 

\begin{mathpar}
  M^{*}_{x} | \lift{x}{P} \red M[P]
\end{mathpar}

namely,

\begin{mathpar}
  M^{*}_{x} := x?(u).M[\dropn{u}]
\end{mathpar}

The dependence of $M^{*}_{x}$ on a name makes it an abstraction, 

\begin{mathpar}
  M^{*} := (x)x?(u).M[\dropn{u}]
\end{mathpar}

\subsection{Additional notation}

It will sometimes be convenient to denote the process a name
quotes. We already have the notation $x = \quotep{P}$, but it will be
convenient to introduce an alternate notation, $\procn{x}$, when we
want to emphasize the connection to the use of the name. Note that, by
virtue of name equivalence, $\quotep{\procn{x}} \nameeq x$; so, the
notation is consistent with previous definitions.

Further, because names have structure it is possible to effect
substitutions on the basis of that structure. This means we need to
upgrade our notation for substitutions, which we accomplish by
adapting comprehension notation. Thus,

\begin{mathpar}
  P\{ y / x : x \in S \}
\end{mathpar}

is interpreted to mean the process derived from P by replacing (in a
capture-avoiding manner) each occurrence of $x$ in $S$ by $y$. For example,

\begin{mathpar}
  P\{ \quotep{\procn{x}|\procn{x}} / x : x \in \freenames{P} \}
\end{mathpar}

will replace each (occurrence) of a free name $x$ in $P$ by
$\quotep{\procn{x}|\procn{x}}$.

Also, we will avail ourselves of the notation $x^{L}$ and $x^{R}$ to
denote injections of a name into disjoint copies of the name
space. There are numerous ways to accomplish this. One example can be
found in \cite{MeredithR05}. This notation overloads to vectors of
names: $\vec{x}^{\pi} := (x_{i}^{\pi} \; : \; 0 \leq i < |\vec{x}| )$ where $\pi \in \{L,R\}$.

We also use $P^{\Box} := P|\Box$.

In \cite{MeredithR05} an interpretation of the new operator is
given. It turns out that there are several possible interpretations
all enjoying the requisite algebraic properties of the operator (see
\cite{milner91polyadicpi}). We will therefore make liberal use of
$(\nu\; \vec{x})P$.

% subsection the_syntax_and_semantics_of_the_notation_system (end)   

\input{qm2pi.qmops} 

\input{qm2pi.sterngerlach} 

\input{qm2pi.metric} 

% section concurrent_process_calculi (end)

%\input{qm2pi.proofsketch}

% section proof sketch (end)

%\input{qm2pi.slviaknots} 

% section spatial logic via knots (end)

\input{qm2pi.conclusion}

% section conclusion (end)

%\input{qm2pi.dtcodes} 

% section wiring algorithm (end)

\input{qm2pi.ack} 

% section acknowledgments (end)

\newpage


\bibliographystyle{plain}   
\bibliography{../../biblios/main.bib}

\input{qm2pi.rhodetails}

\end{document}

 

% section concurrent_process_calculi (end)

%\documentclass[12pt]{llncs}
%\documentclass{jktr}

\usepackage[pdftex]{hyperref}                   
\usepackage {listings}
\usepackage {mathpartir}
\usepackage{bcprules}
%\usepackage{listings}
                       
\usepackage{graphicx} 
%\usepackage[margins=2.5cm,nohead,nofoot]{geometry}
%\usepackage{geometry}
\usepackage{amsfonts}
\usepackage{amstext}
\usepackage{latexsym}
\usepackage{amssymb}
\usepackage{color}


%\include{myPreamble}
\include{qm2pi.local} 

%\ifpdf
%\usepackage[pdftex]{graphicx}
%\else
%\usepackage{graphicx}
%\fi

 % \ifpdf
%  \usepackage{pdfsync}
%  \if


%\title{Brief Article}
%\author{David F. Snyder}
%\author{L.G. Meredith}

%\address{Dept. of Math., Texas State University--San Marcos, San Marcos, TX 78666}
       
\pagestyle{empty}


\begin{document}

\lstset{language=[Objective]Caml,frame=shadowbox}

\input{qm2pi.front}

% section front matter (end)

\input{qm2pi.intro} 
 
% section introduction (end)

% \input{qm2pi.knotations} 

% section notation (end)

\input{qm2pi.process.calculi} 

% section concurrent_process_calculi_and_spatial_logics_ (end)
    
%\input{qm2pi.knots2pi} 

%\input{qm2pi.trefoil} 

%\input{qm2pi.mainthm} 

% subsection basic_interpretation (end)

%\input{qm2pi.rho.presentation} 
\subsection{The syntax and semantics of the notation system}\label{sub:the_syntax_and_semantics_of_the_notation_system} % (fold)

We now summarize a technical presentation of the calculus that
embodies our theory of dynamics. The typical presentation of such a
calculus follows the style of giving generators and relations on
them. The grammar, below, describing term constructors, freely
generates the set of processes, $\Proc$. This set is then quotiented
by a relation known as structural congruence and it is over this set
that the notion of dynamics is expressed. This presentation is
essentially that of \cite{MeredithR05} with the addition of
polyadicity and summation. For readability we have relegated some of
the technical subtleties to an appendix.

\subsubsection{Process grammar}\label{subsub:process_grammar}

\begin{mathpar}
  \inferrule* [lab=synchronization] {} {{M} \bc \pzero \;|\; x?F \;|\; x!C }
  \and
  \inferrule* [lab=abstraction] {} {{F} \bc (x)P}
  \and
  \inferrule* [lab=concretion] {} {{C} \bc \langle Q \rangle}
  \and
  \inferrule* [lab=process] {} {{P,Q} \bc M \;| \;P|Q \;|\; @{x}}
  \and
  \inferrule* [lab=name] {} {{x} \bc \quotep{P}}
\end{mathpar} 

Note that $\vec{x}$ (resp. $\vec{P}$) denotes a vector of names
(resp. processes) of length $|\vec{x}|$ (resp. $|\vec{P}|$). We adopt
the following useful abbreviations.

\begin{mathpar}
   x?(\vec{y}).P := x.(\vec{y})P \and  x\clift{\vec{P}} := x.\clift{\vec{P}}
   \and x!(y) := \lift{x}{\dropn{y}}
   \and \Pi_{i=0}^{n-1}P_i := P_0 | \ldots | P_{n-1}
\end{mathpar}

\subsubsection{Structural congruence}

\paragraph{Free and bound names and alpha-equivalence.} At the
core of structural equivalence is alpha-equivalence which identifies
process that are the same up to a change of variable. Formally, we
recognize the distinction between free and bound names. The free names
of a process, $\freenames{P}$, may be calculated recursively as
follows:

\begin{mathpar}
\freenames{\pzero} := \emptyset
  \and \\
  \freenames{x?(y).P} := \{ x \} \cup (\freenames{P} \setminus \{ y \})
  \and 
  \freenames{x!\langle P \rangle} := \{ x \} \cup \{ P \} 
  \and \\
  \freenames{P|Q} := \freenames{P} \cup \freenames{Q}
  \and \\
  \freenames{@{x}} := \{ x \}
\end{mathpar}

$\pi$
$\quotep{\pi}$

$\freenames{-} : \pi \to \mathcal{P}(\quotep{\pi})$

\begin{eqnarray*}
  \freenames{\pzero} & := & \emptyset \\
  \freenames{x?(y).P} & := & \{ x \} \cup (\freenames{P} \setminus \{ y \}) \\
  \freenames{x!\langle P \rangle} & := & \{ x \} \cup \{ P \} \\
  \freenames{P|Q} & := & \freenames{P} \cup \freenames{Q} \\
  \freenames{\dropn{x}} & := & \{ x \}
\end{eqnarray*}

The bound names of a process, $\boundnames{P}$, are those names occurring in $P$
that are not free. For example, in $x?(y).0$, the name $x$ is free, while $y$ is bound.

\begin{mathpar}
  \inferrule* [lab=monoidal-laws] {} { P|Q \equiv Q|P \and P|0 \equiv P \and P|(Q|R) \equiv (P|Q)|R }
\end{mathpar}

\begin{mathpar}
  \inferrule* [lab=alpha-equivalence] {} { (x)P \equiv (y)P\{y/x\} \and y \not\in \freenames{P} }
\end{mathpar}

\begin{definition}
Then two processes, $P,Q$, are alpha-equivalent if $P = Q\{\vec{y}/\vec{x}\}$ for
some $\vec{x} \in \boundnames{Q},\vec{y} \in \boundnames{P}$, where $Q\{\vec{y}/\vec{x}\}$
denotes the capture-avoiding substitution of $\vec{y}$ for $\vec{x}$ in $Q$.
\end{definition}

\begin{definition}
  The {\em structural congruence} \cite{SangiorgiWalker} , $\equiv$,
  between processes is the least congruence containing
  alpha-equivalence, satisfying the abelian monoid laws
  (associativity, commutativity and $\pzero$ as identity) for parallel
  composition $|$ and for summation $+$.
\end{definition}

\subsection{Name equivalence}

We take name equivalence, written $\nameeq$, to be the smallest
equivalence relation generated by the following rules.

\begin{mathpar}
\inferrule*[lab=Quote-drop]
{ }
{ \quotep{@{x}} \nameeq x }

\inferrule*[lab=Struct-equiv]
{ P \scong Q }
{ \quotep{P} \nameeq \quotep{Q} }
\end{mathpar}

The astute reader will have noticed that the mutual recursion of names
and processes imposes a mutual recursion on alpha-equivalence and
structural equivalence via name-equivalence. Fortunately, all of this
works out pleasantly and we may calculate in the natural way, free of
concern. The reader interested in the details is referred to the
appendix \ref{appendix:rho_details}.

\subsection{Substitution}

We use $\Proc$ for the set of processes, $\QProc$ for the set of
names, and $\id{\{}\vec{y} / \vec{x} \id{\}}$ to denote partial maps,
$s : \QProc \rightarrow \QProc$. A map, $s$ lifts, uniquely, to a map
on process terms, $\widehat{s} : \Proc \rightarrow \Proc$ by the
following equations.

\begin{mathpar}
  (0) \psubstp{Q}{P} := 0 \\
  (R \juxtap S) \psubstp{Q}{P}
  :=    
  (R)\psubstp{Q}{P} \juxtap (S) \psubstp{Q}{P} \\
  (x?(y).R) \psubstp{Q}{P}    
  :=    
  (x)\substp{Q}{P} (z)\concat( (R \psubstn{z}{y}) \psubstp{Q}{P} ) \\
  (\lift{x}{R}) \psubstp{Q}{P}  
  :=
  \lift{(x)\substp{Q}{P}}{ R \psubstp{Q}{P} } \\
%   (\dropn{x})  \psubstp{Q}{P}       
%   := 
%   \left\{ 
%     \begin{array}{ccc} 
%       \dropn{\quotep{Q}} & & x \nameeq \quotep{P} \\
%       \dropn{x} & & otherwise \\
%     \end{array}
%   \right. 
  (\dropn{x})  \psubstp{Q}{P}       
  := 
  \left\{ 
    \begin{array}{ccc} 
      Q & & x \nameeq \quotep{P} \\
      \dropn{x} & & otherwise \\
    \end{array}
  \right.
\end{mathpar}
 

where

\begin{eqnarray}
  (x)\id{\{} \lpquote Q \rpquote / \lpquote P \rpquote \id{\}}            = 
  \left\{ 
    \begin{array}{ccc}
      \lpquote Q \rpquote & & x \nameeq \lpquote P \rpquote \\
      x & & otherwise \\
    \end{array}
  \right. \nonumber
\end{eqnarray}

and $z$ is chosen distinct from $\quotep{P}$, $\quotep{Q}$, the free
names in $Q$, and all the names in $R$. Our $\alpha$-equivalence will
be built in the standard way from this substitution.

\begin{remark}\label{rem:no_self_referential_names}
  One consequence of these definitions is that $\forall P. \quotep{P}
  \not\in \freenames{P}$.
\end{remark}

\subsection{ Dynamic quote: an example }

Anticipating something of what's to come, consider applying the
substitution, $\widehat{\id{\{}u / z \id{\}}}$, to the following pair
of processes, $\lift{w}{y!(z)}$ and $w[ \lpquote y!(z) \rpquote ]$.

\begin{eqnarray}
	\lift{w}{y!(z)}\widehat{\id{\{}u / z \id{\}}}
		& = &
		\lift{w}{y!(u)} \nonumber\\
	w[ \lpquote y!(z) \rpquote ] \widehat{ \id{\{}u / z \id{\}} }
		& = &
		w[ \lpquote y!(z) \rpquote ] \nonumber
\end{eqnarray}

Because the body of the process between quotes is impervious to
substitution, we get radically different answers. In fact, by
examining the first process in an input context,
e.g. $x?(z).\lift{w}{y!(z)}$, we see that the process under the lift
operator may be shaped by prefixed inputs binding a name inside it. In
this sense, the lift operator will be seen as a way to dynamically
construct processes before reifying them as names.

Finally equipped with these standard features we can present the
dynamics of the calculus.

\subsubsection{Operational semantics} 

Finally, we introduce the computational dynamics. What marks these
algebras as distinct from other more traditionally studied algebraic
structures, e.g. vector spaces or polynomial rings, is the manner in
which dynamics is captured. In traditional structures, dynamics is typically
expressed through morphisms between such structures, as in linear maps
between vector spaces or morphisms between rings. In algebras
associated with the semantics of computation, the dynamics is
expressed as part of the algebraic structure itself, through a
reduction reduction relation typically denoted by $\red$. Below, we
give a recursive presentation of this relation for the calculus used
in the encoding.

$\red \subseteq \pi \times \pi$
$\red : \pi \to \mathcal{P}(\pi)$

\begin{mathpar}
  \inferrule* [lab=Comm] { \textsf{match}( x_{src}, x_{trgt} ) } { x_{trgt}?(y)P \; | \; x_{src}!\langle {Q} \rangle \red P\{\quotep{Q}/y}\} }
  \and \\
  \inferrule* [lab=Par] {{P} \red {P}'} {{{P} | {Q}} \red {{P}' | {Q}}}
  \and
  \inferrule* [lab=Equiv]{{{P} \scong {P}'} \andalso {{P}' \red {Q}'} \andalso {{Q}' \scong {Q}}}{{P} \red {Q}}
\end{mathpar}

\begin{eqnarray*}
  match_{\equiv} (\quotep{P},\quotep{Q}) & := & P \equiv Q \\
  match_{\dagger}(\quotep{P},\quotep{Q}) & := & \forall R. P|Q \red^{*} R => R \red^{*} 0 \\
  match_{K}(\quotep{P},\quotep{Q}) & := & K \mbox{ for some context } K
\end{eqnarray*}

$u?(x)P | u!\langle Q \rangle \red P\{\quotep{Q}/x\}$

%We write $\wred$ for $\red^*$, and $P\red$ if $\exists Q $ such that $ P \red Q$.
We write $P\red$ if $\exists Q $ such that $ P \red Q$ and $P\not\red$, otherwise.

\section{Replication}

As mentioned before, it is known that replication (and hence
recursion) can be implemented in a higher-order process algebra
\cite{SangiorgiWalker}. As our first example of calculation with the
machinery thus far presented we give the construction explicitly in
the {\rhoc}.

\begin{eqnarray}
	D_{x} & := & \prefix{x}{y}{(\binpar{\outputp{x}{y}}{@{y}})} \nonumber\\
	\bangp_{x}{P} & := & \binpar{{x}!\langle{\binpar{D_{x}}{P}}\rangle}{D_{x}} \nonumber
\end{eqnarray}

\begin{eqnarray}
	\bangp_{x}{P} & & \nonumber\\
	=
	& {x}!\langle{(\prefix{x}{y}{(\outputp{x}{y} | @{y})) | P}}\rangle 
	      | \prefix{x}{y}{(\outputp{x}{y} | @{y})} & \nonumber\\
	\red
	& (\outputp{x}{y} | @{y})\substn{\quotep{(\prefix{x}{y}{(@{y} | \outputp{x}{y})) | P}}}{y} & \nonumber\\
	=
	& \outputp{x}{\quotep{(\prefix{x}{y}{(\outputp{x}{y} | @{y})) | P}}}
	  | {(\prefix{x}{y}{(\outputp{x}{y} | @{y})) | P}} & \nonumber\\
	\red
	& \ldots & \nonumber\\
	\red^*
	& P | P | \ldots & \nonumber
\end{eqnarray}

Of course, this encoding, as an implementation, runs away, unfolding
$\bangp{P}$ eagerly. A lazier and more implementable replication
operator, restricted to input-guarded processes, may be obtained as follows.

\begin{eqnarray}
\bangp{\prefix{u}{v}{P}} 
	:= 
	\binpar{\lift{x}{\prefix{u}{v}{(\binpar{D(x)}{P})}}}{D(x)} \nonumber
\end{eqnarray}

\begin{remark}
  Note that the lazier definition still does not deal with summation
  or mixed summation (i.e. sums over input and output). The reader is
  invited to construct definitions of replication that deal with these
  features. 

  Further, the definitions are parameterized in a name, $x$. Can you,
  gentle reader, make a definition that eliminates this parameter and
  guarantees no accidental interaction between the replication
  machinery and the process being replicated -- i.e. no accidental
  sharing of names used by the process to get its work done and the
  name(s) used by the replication to effect copying. This latter
  revision of the definition of replication is crucial to obtaining
  the expected identity $!!P \sim !P$.
\end{remark}

\begin{remark}\label{rem:paradoxical_combinator}
  The reader familiar with the lambda calculus will have noticed the
  similarity between $D$ and the paradoxical combinator.

  [Ed. note: the existence of this seems to suggest we have to be more
  restrictive on the set of processes and names we admit if we are to
  support no-cloning.]
\end{remark}

\subsubsection{Bisimulation}

The computational dynamics gives rise to another kind of equivalence,
the equivalence of computational behavior. As previously mentioned
this is typically captured \emph{via} some form of bisimulation.

% The notion we use in this paper is weak barbed bisimulation
% \cite{milner91polyadicpi}.

The notion we use in this paper is derived from weak barbed
bisimulation \cite{milner91polyadicpi}. 

\begin{definition}
An \emph{observation relation}, $\downarrow_{\mathcal N}$, over a set
of names, $\mathcal N$, is the smallest relation satisfying the rules
below.

\infrule[Out-barb]{y \in {\mathcal N}, \; x \nameeq y}
		  {\outputp{x}{v} \downarrow_{\mathcal N} x}
\infrule[Par-barb]{\mbox{$P\downarrow_{\mathcal N} x$ or $Q\downarrow_{\mathcal N} x$}}
		  {\binpar{P}{Q} \downarrow_{\mathcal N} x}

We write $P \Downarrow_{\mathcal N} x$ if there is $Q$ such that 
$P \wred Q$ and $Q \downarrow_{\mathcal N} x$.
\end{definition}

\begin{definition}
%\label{def.bbisim}
An  ${\mathcal N}$-\emph{barbed bisimulation} over a set of names, ${\mathcal N}$, is a symmetric binary relation 
${\mathcal S}_{\mathcal N}$ between agents such that $P\rel{S}_{\mathcal N}Q$ implies:
\begin{enumerate}
\item If $P \red P'$ then $Q \wred Q'$ and $P'\rel{S}_{\mathcal N} Q'$.
\item If $P\downarrow_{\mathcal N} x$, then $Q\Downarrow_{\mathcal N} x$.
\end{enumerate}
$P$ is ${\mathcal N}$-barbed bisimilar to $Q$, written
$P \wbbisim_{\mathcal N} Q$, if $P \rel{S}_{\mathcal N} Q$ for some ${\mathcal N}$-barbed bisimulation ${\mathcal S}_{\mathcal N}$.
\end{definition}

$\mathcal{R} \subseteq \pi \times \pi$

$P \mathcal{R} Q => \forall P'. P \red P' \Rightarrow \exists Q'. Q \red Q', P' \mathcal{R} Q'$

$P \vdash x \Rightarrow Q \vdash x$

\begin{mathpar}
  \inferrule*[lab=Out-barb]{x \nameeq y}{{y}!\langle{Q}\rangle \vdash x}
  \and
  \inferrule*[lab=Par-barb]{\mbox{$P\vdash x$ or $Q\vdash x$}}{\binpar{P}{Q} \vdash x}
\end{mathpar}

\subsubsection{Contexts}

One of the principle advantages of computational calculi like the
$\pi$-calculus is a well-defined notion of context,
contextual-equivalence and a correlation between
contextual-equivalence and notions of bisimulation. The notion of
context allows the decomposition of a process into (sub-)process and
its syntactic environment, its context. Thus, a context may be
thought of as a process with a ``hole'' (written $\Box$) in it. The
application of a context $M$ to a process $P$, written $M[P]$, is
tantamount to filling the hole in $M$ with $P$. In this paper we do
not need the full weight of this theory, but do make use of the notion
of context in the proof the main theorem. 

\begin{mathpar}
  \inferrule* [lab=summation] {} {{M_{M},M_{N}} \bc \Box \;|\; x.M_{A} \;|\; M_{M}+M_{N}}
  \and
  \inferrule* [lab=agent] {} {{M_{A}} \bc (\vec{x})M_{P} \;| \; \clift{P_0,\ldots,M_{P},\ldots,P_N}}
  \and \\
  \inferrule* [lab=process] {} {{M_{P}} \bc M_{N} \;| \;P|M_{P} }
\end{mathpar} 

\begin{mathpar}
  \inferrule* [lab=sychronization] {} {M_{N} \bc \Box \;|\; x?M_{F} \;|\; x!M_{C}}
  \and
  \inferrule* [lab=abstraction] {} {{M_{F}} \bc (x)M_{P} }
  \and
  \inferrule* [lab=concretion] {} {{M_{C}} \bc \langle M_{P} \rangle }
  \and \\
  \inferrule* [lab=process] {} {{M_{P}} \bc M_{N} \;| \;P|M_{P} }
\end{mathpar}

\begin{definition}[contextual application] Given a context $M$, and
  process $P$, we define the \emph{contextual application}, $M[P] :=
  M\{P/\Box\}$. That is, the contextual application of M to P is the
  substitution of $P$ for $\Box$ in $M$.
\end{definition}

$\meaningof{-} : L \to \mathcal{P}(\pi)$

\begin{mathpar}
  \inferrule* [lab=collection] {} {\meaningof{true} = \pi, \and \meaningof{~E} = \pi \setminus \meaningof{E}, \and \meaningof{E_{1} \& E_{2}} = \meaningof{E_{1}} \cap \meaningof{E_{2}}}
\end{mathpar}

\begin{mathpar}
  \inferrule* [lab=structure] {} {\meaningof{0} = \{ P \in \pi | P \equiv 0 \}, \and \\ \meaningof{E_1 | E_2} = \{ P \in \pi | P \equiv P_{1} | P_{2}, P_{1} \in \meaningof{E_{1}}, P_{2} \in \meaningof{E_2}\} }
\end{mathpar}

\begin{mathpar}
 \inferrule* [lab=behavior] {} {\meaningof{\langle a?b \rangle E} = \{ P \in \pi | P \equiv Q | u?(y)P', \\ \and \\\\ \and \\ \;\;\; u \in \meaningof{a}, \forall z.P'\{z/y\} \in \meaningof{E\{z/b\}}\}, \and \\ \meaningof{a!E} = \{ P \in \pi | P \equiv Q | x!\langle P' \rangle, x \in \meaningof{a} P' \in \meaningof{E}\} }
\end{mathpar}

\begin{mathpar}
 \inferrule* [lab=nominal] {} {\meaningof{\quotep{E}} = \{ \quotep{P} \in \quotep{\pi} | P \in \meaningof{E} \}, \and \meaningof{\quotep{P}} = \{ \quotep{Q} \in \quotep{\pi} | P \equiv Q \} \and \\ \meaningof{@\quotep{E}} = \{ P \in \pi | P \equiv @x, x \in \meaningof{E} \}}
\end{mathpar}

\begin{eqnarray*}
  \\
  \meaningof{-} : TS \to ST
\end{eqnarray*}

\begin{eqnarray*}
  \\
  L : TS \to ST
\end{eqnarray*}

\begin{eqnarray*}
  \\
  P \models E \iff P \in \meaningof{E}
\end{eqnarray*}

\begin{eqnarray*}
  P \approx_{L} Q \iff \forall E \in L. P \models E \iff Q \models E
\end{eqnarray*}

\begin{eqnarray*}
  P \approx_{K} Q
\end{eqnarray*}

\begin{eqnarray*}
  P \approx Q
\end{eqnarray*}

$\approx_{K} = \approx = \approx_{L}$

\subsubsection{Contextual duality}

Note that contexts extend the quotation operation to a family of
operations from processes to names. Given a context, $M$, we can
define a \emph{nominal context}, $\quotep{M}$ by $\quotep{M}[P] :=
\quotep{M[P]}$. To foreshadow what is to come we observe that these
operations enjoy a duality with processes very much like the duality
between vectors and maps from vectors to scalars.

Further, because the calculus is essentially higher-order, we have a
correspondence between contexts and processes. More specifically,
given a name $x$ and a context $M$ we can construct $M^{*}_{x}$ such
that 

\begin{mathpar}
  M^{*}_{x} | \lift{x}{P} \red M[P]
\end{mathpar}

namely,

\begin{mathpar}
  M^{*}_{x} := x?(u).M[\dropn{u}]
\end{mathpar}

The dependence of $M^{*}_{x}$ on a name makes it an abstraction, 

\begin{mathpar}
  M^{*} := (x)x?(u).M[\dropn{u}]
\end{mathpar}

\subsection{Additional notation}

It will sometimes be convenient to denote the process a name
quotes. We already have the notation $x = \quotep{P}$, but it will be
convenient to introduce an alternate notation, $\procn{x}$, when we
want to emphasize the connection to the use of the name. Note that, by
virtue of name equivalence, $\quotep{\procn{x}} \nameeq x$; so, the
notation is consistent with previous definitions.

Further, because names have structure it is possible to effect
substitutions on the basis of that structure. This means we need to
upgrade our notation for substitutions, which we accomplish by
adapting comprehension notation. Thus,

\begin{mathpar}
  P\{ y / x : x \in S \}
\end{mathpar}

is interpreted to mean the process derived from P by replacing (in a
capture-avoiding manner) each occurrence of $x$ in $S$ by $y$. For example,

\begin{mathpar}
  P\{ \quotep{\procn{x}|\procn{x}} / x : x \in \freenames{P} \}
\end{mathpar}

will replace each (occurrence) of a free name $x$ in $P$ by
$\quotep{\procn{x}|\procn{x}}$.

Also, we will avail ourselves of the notation $x^{L}$ and $x^{R}$ to
denote injections of a name into disjoint copies of the name
space. There are numerous ways to accomplish this. One example can be
found in \cite{MeredithR05}. This notation overloads to vectors of
names: $\vec{x}^{\pi} := (x_{i}^{\pi} \; : \; 0 \leq i < |\vec{x}| )$ where $\pi \in \{L,R\}$.

We also use $P^{\Box} := P|\Box$.

In \cite{MeredithR05} an interpretation of the new operator is
given. It turns out that there are several possible interpretations
all enjoying the requisite algebraic properties of the operator (see
\cite{milner91polyadicpi}). We will therefore make liberal use of
$(\nu\; \vec{x})P$.

% subsection the_syntax_and_semantics_of_the_notation_system (end)   

\input{qm2pi.qmops} 

\input{qm2pi.sterngerlach} 

\input{qm2pi.metric} 

% section concurrent_process_calculi (end)

%\input{qm2pi.proofsketch}

% section proof sketch (end)

%\input{qm2pi.slviaknots} 

% section spatial logic via knots (end)

\input{qm2pi.conclusion}

% section conclusion (end)

%\input{qm2pi.dtcodes} 

% section wiring algorithm (end)

\input{qm2pi.ack} 

% section acknowledgments (end)

\newpage


\bibliographystyle{plain}   
\bibliography{../../biblios/main.bib}

\input{qm2pi.rhodetails}

\end{document}



% section proof sketch (end)

%\section{Unlikely characters: spatial logic for
  knots}\label{sub:characteristic_formulae} % (fold)

Associated to the mobile process calculi are a family of logics known
as the Hennessy-Milner logics. These logics typically enjoy a
semantics interpreting formulae as sets of processes that when
factored through the encoding outlined above allows an identification
of classes of knots with logical formulae. In the context of this
encoding the sub-family known as the spatial logics \cite{CairesC03}
\cite{CairesC04} \cite{Caires04} are of particular interest providing
several important features for expressing and reasoning about
properties (i.e. classes) of knots. We hint here at how this may be done.

%\begin{description}
%\item [structural connectives] 
\subsubsection{Structural connectives} The spatial logics enjoy
structural connectives corresponding, at the logical level, to the
parallel composition ($P | Q$) and new name ($(\nu \; x)P$)
connectives for processes. As illustrated in the examples below, these
connectives are extremely expressive given the shape of our encoding.
%\item [decideable satisfaction]

\subsubsection{Decideable satisfaction}
In \cite{Caires04} the satisfaction relation is shown to be decideable
for a rich class of processes. It further turns out that the image of
the our encoding is a proper subset of that class. This result
provides the basis for an algorithm by which to search for knots
enjoying a given property.
%\item [characteristic formulae]

\subsubsection{Characteristic formulae}
In the same paper \cite{Caires04} , Caires presents a means of calculating
characteristic formulae, selecting equivalence classes of processes
up to a pre--specified depth limit on the support set of names. Composed with our
encoding, this characteristic formula can be used to select
characteristic formulae for knots.
%\end{description}

\subsubsection{Spatial logic formulae}

The grammar below (segmented for comprehension) summarizes the syntax
of spatial logic formulae. We employ illustrative examples in the
sequel to provide an intuitive understanding of their meaning
referring the reader to \cite{Caires04} for a more detailed explication
of the semantics.

\begin{mathpar}
  \inferrule* [lab=boolean] {} {{A,B} \bc T \;|\; \neg A \;|\; A \wedge B \;|\; \eta = \eta'}
  \and
  \inferrule* [lab=spatial] {} {|\; \pzero \;|\; A | B \;|\; x \text{\textregistered} A \;|\; \forall x . A \;|\;  H x . A}
  \and
  \inferrule* [lab=behavioral] {} {|\; \alpha . A}
  \and 
  \inferrule* [lab=recursion] {} {|\; X(\vec{u}) \;|\; \mu X(\vec{u}) . A}
  \and
  \inferrule* [lab=action] {} {\alpha \bc \langle x?(\vec{y}) \rangle \;|\; \langle x!(\vec{y}) \rangle \;|\; \langle \tau \rangle}
  \and 
  \inferrule* [lab=name] {} {\eta \bc x \;|\; \tau}
\end{mathpar} 

% subsection characteristic_formulae (end)   	 

\subsection{Example formulae}\label{sub:example_formulae_} % (fold)

\subsubsection{Crossing as formula.}
% 
% \begin{align*}
%   \frac{d}{dx} \sin x &= \cos x 
%   & \frac{d}{dx} e^x &= e^x \\
%   \frac{d}{dx} \cos x &= - \sin x 
%   & \frac{d}{dx} \log x &= \frac{1}{x} \\
% \end{align*} 

\begin{align*}
 \mu C(x_{0},x_{1},y_{0},y_{1},u).&(\langle x_{0}?(z) \rangle(\langle u! \rangle\langle y_{1}!z \rangle C(x_{0},x_{1},y_{0},y_{1},u)) & \\
  & \wedge \langle y_{1}?(z) \rangle (\langle u! \rangle \langle x_{0}!z \rangle C(x_{0},x_{1},y_{0},y_{1},u)) & \\
  & \wedge \langle x_{1}?(z) \rangle (\langle u? \rangle \langle y_{0}!z \rangle C(x_{0},x_{1},y_{0},y_{1},u)) & \\
  & \wedge \langle y_{0}?(z) \rangle (\langle u? \rangle \langle x_{1}!z \rangle C(x_{0},x_{1},y_{0},y_{1},u))) &
\end{align*}

The lexicographical similarity between the shape of this formulae and
the shape of definition of the process representing a crossing reveals
the intuitive meaning of this formulae. It describes the capabilities
of a process that has the right to represent a crossing. For example
it picks out processes that may perform an input on the port $x_0$ in
its initial menu of capabilities. What differentiates the formula
from the process, however, is that the crossing process is the
smallest candidate to satisfy the formula. Infinitely many other
processes -- with internal behavior hidden behind this interface, so
to speak -- also satisfy this formula. Even this simple formula,
then, can be seen to open a new view onto knots, providing a
computational interpretation of \emph{virtual} knots.

Note that this formula is derived by hand. A similar formula can be
derived by employing Caires' calculation of characteristic formula
\cite{Caires04} to the process representing a crossing. In light of
this discussion, we let
$\meaningof{C}_{\phi}(x0,x1,y0,y1,u)$ denote a formula specifying the
dynamics we wish to capture of a crossing. To guarantee we preserve
the shape of the interface and minimal semantics we demand that
$\meaningof{C}_{\phi}(x0,x1,y0,y1,u) \Rightarrow
\textbf{C}(x0,x1,y0,y1,u)$ where $\textbf{C}(x0,x1,y0,y1,u)$ denotes
the formula above.
                            
\subsubsection{Crossing number constraints.}
The moral content of the context lemma (Lemma \ref{context}) is that the notion of
``locality'' in the Reidemeister moves is effectively captured by the
parallel composition operator of the process calculus. This intuition
extends through the logic. Given a formula,
$\meaningof{C}_{\phi}(x0,x1,y0,y1,u)$, we can use the structural
connectives to specify constraints on crossing numbers, such as at
least $n$ crossings, or exactly $n$ crossings.
\begin{mathpar}
  \inferrule* [lab=at-least-n] {} { K^{\geq n}_{\phi}(\vec{xs},\vec{ys}) := \Pi_{i=0}^{n-1} Hu . \meaningof{C}_{\phi}(xs_i,ys_i,u) | T }
  \and 
  \inferrule* [lab=exactly-n] {} { K^{= n}_{\phi}(\vec{xs},\vec{ys}) := \Pi_{i=0}^{n-1} Hu . \meaningof{C}_{\phi}(xs_i,ys_i,u) | \neg (\forall x_0,y_0,x_1,y_1,u . \meaningof{C}_{\phi}(x_0,y_0,x_1,y_1,u) | T) }
\end{mathpar}

To round out this section, recall that the encoding of an $n$-crossing
knot decomposes into a parallel composition of $n$ \emph{copies} of a
crossing process together with a wiring harness. To specify different
knot classes with the same crossing number amounts to specifying
logical constraints on the wiring harness. In the interest of space,
we defer examples to a forthcoming paper. Suffice it to say that both
the conditions ``alternating knot'' and ``contains the tangle
corresponding to 5/3'' are expressible. For example, it is possible to
calculate the characteristic formula of a process corresponding to the
tangle 5/3 and conjoin it into the classifying formula via the
composition connective of the logic.

Finally, we wish to observe that it is entirely within reason to
contemplate a more domain-specific version of spatial logic tailored
to the shape of processes in the image of the encoding. Such a
domain-specific logic would have a better claim to the title formal
language of knot properties.

% subsection example_formulae_ (end)

% section knots_as_processes (end) 

% section spatial logic via knots (end)

\section{Conclusions and future work}

\paragraph{Testing physical space}
You, gentle reader, may wonder why of all the theorems to be proved
given this set up we pick the one above. In some sense it's hardly
central to quantum mechanics. We see it as central in the sense that
it firmly establishes a notion of physical space arising from a notion
of the equivalence of behavior. Relating bisimulation to a metric is a
big step forward, but one is faced with interpreting the relationship
of that metric space to something more physical. Quantum mechanical
notions of ``physical'' space are still far from intuitive, but by
relating this idea of distance as testing to calculations that predict
physical circumstances we are making a not insignificant step forward
toward an understanding of the physical space we inhabit as
essentially dynamic.

\paragraph{Effectivity and simulation}
One of the observations we have yet to make is that the entire program
spelled out here is effective. We have built various interpreters for
the reflective calculus at work in this interpretation. In principle,
then, we can simulate quantum mechanics on a computer. The place where
the simulation may lose fidelity is the infinitely branching summation
for the annihilator.

In this connection i also want to point out that the evaluation style
calculation of the inner product puts the non-determinism of the
summation right at the heart of measurement. This suggests that
Milner's original reduction-based formulation of the dynamics of his
calculi in terms of sums was not just notationally suggestive of a
notion of measure-and-continue but captured some significant part of
the physics.

\paragraph{Quantum continuations}
In light of this last observation i want to point out that the
predominant account of quantum mechanics is missing a key aspect of a
truly compositional story of the physical situation. In a real lab,
when a measurement is made the observation can be made to feed into
another device that then makes another measurement conditioned on the
results of the first. This means that after the superposition was
collapsed the entire experimental set up remained in
superposition. While QM offers a means of writing this down it doesn't
quite line up well with the well-trodden formulation of computation
and continuation that we see so succinctly expressed in Milner's
calculi. This suggests that there might be advantages to this account
of dynamics waiting to be explored.

\paragraph{Quantum logic}
In this connection, we also note that by virtue of having the
Hennessy-Milner construction, we can pull the construction through the
interpretation of QM. This gives us a natural candidate for a quantum
logic that enjoys an extremely tight connection with it's domain of
interpretation, making the construction much less ad hoc (rather it is
the image of functor!).

\paragraph{Quantum probabiity}
i have questions about the basis of the interpretation of inner
product as probability amplitude. In particular, using which
axiomatization of probability theory does the notion of probability
amplitude earn the right to be so dubbed? In other words, where is the
proof that the operation for calculating a probability amplitude (and
then squaring) satisfies the axioms of what it means to calculate a
probability? Even if such a proof exists (i have yet to find it in the
literature), i wonder if it might not be possible to turn things on
their heads. Can we view the calculation of the probability amplitude
as an axiomatization of probability? If so, then the definition we
give for calculating probability amplitude may provide the basis for
an \emph{effective} theory of probability.

\paragraph{Quantum vs ``biological'' information}
Finally, i want to conclude with a more philosophical observation. At
a recent workshop in which QM was a predominant topic i noticed
something about quantum information. The speaker was giving a riveting
discussion of axiomatic QM and showing how properties of ``no
cloning'' and ``no deleting'' emerged as consequences of the
axiomatization. Theorems of this form are necessary to give us a sense
of confidence that our axioms characterize the physical theory. What
struck me, though, was that if quantum information is neither erasable
nor replicable it is markedly different from \emph{life}. Two of the
things we know about life is that

\begin{itemize}
  \item it ends;
  \item to gain some measure of persistence, to transcend it's
    finitude it is imminently copyable.
\end{itemize}

Both of these qualities are summarized succinctly in the aphorism: all
flesh is grass. For me these two kinds of ``information'' -- call them
quantum and biological -- are end points on a spectrum of strategies
for persistence. At one end, we have those curious entities that enjoy
uniqueness and permanence; at the other, we have those who in the face
of a certain end and an uncertain present make a go of passing
something on. To me one of the more remarkable aspects of the latter
strategy is that in the presence of noise (and certain features of
copying) we get a kind of dynamism, a chance for improvement against a
given persistent condition.

% subsection other_calculi_other_bisimulations_and_geometry_as_behavior (end)




% section conclusion (end)

%\documentclass[12pt]{llncs}
%\documentclass{jktr}

\usepackage[pdftex]{hyperref}                   
\usepackage {listings}
\usepackage {mathpartir}
\usepackage{bcprules}
%\usepackage{listings}
                       
\usepackage{graphicx} 
%\usepackage[margins=2.5cm,nohead,nofoot]{geometry}
%\usepackage{geometry}
\usepackage{amsfonts}
\usepackage{amstext}
\usepackage{latexsym}
\usepackage{amssymb}
\usepackage{color}


%\include{myPreamble}
\include{qm2pi.local} 

%\ifpdf
%\usepackage[pdftex]{graphicx}
%\else
%\usepackage{graphicx}
%\fi

 % \ifpdf
%  \usepackage{pdfsync}
%  \if


%\title{Brief Article}
%\author{David F. Snyder}
%\author{L.G. Meredith}

%\address{Dept. of Math., Texas State University--San Marcos, San Marcos, TX 78666}
       
\pagestyle{empty}


\begin{document}

\lstset{language=[Objective]Caml,frame=shadowbox}

\input{qm2pi.front}

% section front matter (end)

\input{qm2pi.intro} 
 
% section introduction (end)

% \input{qm2pi.knotations} 

% section notation (end)

\input{qm2pi.process.calculi} 

% section concurrent_process_calculi_and_spatial_logics_ (end)
    
%\input{qm2pi.knots2pi} 

%\input{qm2pi.trefoil} 

%\input{qm2pi.mainthm} 

% subsection basic_interpretation (end)

%\input{qm2pi.rho.presentation} 
\subsection{The syntax and semantics of the notation system}\label{sub:the_syntax_and_semantics_of_the_notation_system} % (fold)

We now summarize a technical presentation of the calculus that
embodies our theory of dynamics. The typical presentation of such a
calculus follows the style of giving generators and relations on
them. The grammar, below, describing term constructors, freely
generates the set of processes, $\Proc$. This set is then quotiented
by a relation known as structural congruence and it is over this set
that the notion of dynamics is expressed. This presentation is
essentially that of \cite{MeredithR05} with the addition of
polyadicity and summation. For readability we have relegated some of
the technical subtleties to an appendix.

\subsubsection{Process grammar}\label{subsub:process_grammar}

\begin{mathpar}
  \inferrule* [lab=synchronization] {} {{M} \bc \pzero \;|\; x?F \;|\; x!C }
  \and
  \inferrule* [lab=abstraction] {} {{F} \bc (x)P}
  \and
  \inferrule* [lab=concretion] {} {{C} \bc \langle Q \rangle}
  \and
  \inferrule* [lab=process] {} {{P,Q} \bc M \;| \;P|Q \;|\; @{x}}
  \and
  \inferrule* [lab=name] {} {{x} \bc \quotep{P}}
\end{mathpar} 

Note that $\vec{x}$ (resp. $\vec{P}$) denotes a vector of names
(resp. processes) of length $|\vec{x}|$ (resp. $|\vec{P}|$). We adopt
the following useful abbreviations.

\begin{mathpar}
   x?(\vec{y}).P := x.(\vec{y})P \and  x\clift{\vec{P}} := x.\clift{\vec{P}}
   \and x!(y) := \lift{x}{\dropn{y}}
   \and \Pi_{i=0}^{n-1}P_i := P_0 | \ldots | P_{n-1}
\end{mathpar}

\subsubsection{Structural congruence}

\paragraph{Free and bound names and alpha-equivalence.} At the
core of structural equivalence is alpha-equivalence which identifies
process that are the same up to a change of variable. Formally, we
recognize the distinction between free and bound names. The free names
of a process, $\freenames{P}$, may be calculated recursively as
follows:

\begin{mathpar}
\freenames{\pzero} := \emptyset
  \and \\
  \freenames{x?(y).P} := \{ x \} \cup (\freenames{P} \setminus \{ y \})
  \and 
  \freenames{x!\langle P \rangle} := \{ x \} \cup \{ P \} 
  \and \\
  \freenames{P|Q} := \freenames{P} \cup \freenames{Q}
  \and \\
  \freenames{@{x}} := \{ x \}
\end{mathpar}

$\pi$
$\quotep{\pi}$

$\freenames{-} : \pi \to \mathcal{P}(\quotep{\pi})$

\begin{eqnarray*}
  \freenames{\pzero} & := & \emptyset \\
  \freenames{x?(y).P} & := & \{ x \} \cup (\freenames{P} \setminus \{ y \}) \\
  \freenames{x!\langle P \rangle} & := & \{ x \} \cup \{ P \} \\
  \freenames{P|Q} & := & \freenames{P} \cup \freenames{Q} \\
  \freenames{\dropn{x}} & := & \{ x \}
\end{eqnarray*}

The bound names of a process, $\boundnames{P}$, are those names occurring in $P$
that are not free. For example, in $x?(y).0$, the name $x$ is free, while $y$ is bound.

\begin{mathpar}
  \inferrule* [lab=monoidal-laws] {} { P|Q \equiv Q|P \and P|0 \equiv P \and P|(Q|R) \equiv (P|Q)|R }
\end{mathpar}

\begin{mathpar}
  \inferrule* [lab=alpha-equivalence] {} { (x)P \equiv (y)P\{y/x\} \and y \not\in \freenames{P} }
\end{mathpar}

\begin{definition}
Then two processes, $P,Q$, are alpha-equivalent if $P = Q\{\vec{y}/\vec{x}\}$ for
some $\vec{x} \in \boundnames{Q},\vec{y} \in \boundnames{P}$, where $Q\{\vec{y}/\vec{x}\}$
denotes the capture-avoiding substitution of $\vec{y}$ for $\vec{x}$ in $Q$.
\end{definition}

\begin{definition}
  The {\em structural congruence} \cite{SangiorgiWalker} , $\equiv$,
  between processes is the least congruence containing
  alpha-equivalence, satisfying the abelian monoid laws
  (associativity, commutativity and $\pzero$ as identity) for parallel
  composition $|$ and for summation $+$.
\end{definition}

\subsection{Name equivalence}

We take name equivalence, written $\nameeq$, to be the smallest
equivalence relation generated by the following rules.

\begin{mathpar}
\inferrule*[lab=Quote-drop]
{ }
{ \quotep{@{x}} \nameeq x }

\inferrule*[lab=Struct-equiv]
{ P \scong Q }
{ \quotep{P} \nameeq \quotep{Q} }
\end{mathpar}

The astute reader will have noticed that the mutual recursion of names
and processes imposes a mutual recursion on alpha-equivalence and
structural equivalence via name-equivalence. Fortunately, all of this
works out pleasantly and we may calculate in the natural way, free of
concern. The reader interested in the details is referred to the
appendix \ref{appendix:rho_details}.

\subsection{Substitution}

We use $\Proc$ for the set of processes, $\QProc$ for the set of
names, and $\id{\{}\vec{y} / \vec{x} \id{\}}$ to denote partial maps,
$s : \QProc \rightarrow \QProc$. A map, $s$ lifts, uniquely, to a map
on process terms, $\widehat{s} : \Proc \rightarrow \Proc$ by the
following equations.

\begin{mathpar}
  (0) \psubstp{Q}{P} := 0 \\
  (R \juxtap S) \psubstp{Q}{P}
  :=    
  (R)\psubstp{Q}{P} \juxtap (S) \psubstp{Q}{P} \\
  (x?(y).R) \psubstp{Q}{P}    
  :=    
  (x)\substp{Q}{P} (z)\concat( (R \psubstn{z}{y}) \psubstp{Q}{P} ) \\
  (\lift{x}{R}) \psubstp{Q}{P}  
  :=
  \lift{(x)\substp{Q}{P}}{ R \psubstp{Q}{P} } \\
%   (\dropn{x})  \psubstp{Q}{P}       
%   := 
%   \left\{ 
%     \begin{array}{ccc} 
%       \dropn{\quotep{Q}} & & x \nameeq \quotep{P} \\
%       \dropn{x} & & otherwise \\
%     \end{array}
%   \right. 
  (\dropn{x})  \psubstp{Q}{P}       
  := 
  \left\{ 
    \begin{array}{ccc} 
      Q & & x \nameeq \quotep{P} \\
      \dropn{x} & & otherwise \\
    \end{array}
  \right.
\end{mathpar}
 

where

\begin{eqnarray}
  (x)\id{\{} \lpquote Q \rpquote / \lpquote P \rpquote \id{\}}            = 
  \left\{ 
    \begin{array}{ccc}
      \lpquote Q \rpquote & & x \nameeq \lpquote P \rpquote \\
      x & & otherwise \\
    \end{array}
  \right. \nonumber
\end{eqnarray}

and $z$ is chosen distinct from $\quotep{P}$, $\quotep{Q}$, the free
names in $Q$, and all the names in $R$. Our $\alpha$-equivalence will
be built in the standard way from this substitution.

\begin{remark}\label{rem:no_self_referential_names}
  One consequence of these definitions is that $\forall P. \quotep{P}
  \not\in \freenames{P}$.
\end{remark}

\subsection{ Dynamic quote: an example }

Anticipating something of what's to come, consider applying the
substitution, $\widehat{\id{\{}u / z \id{\}}}$, to the following pair
of processes, $\lift{w}{y!(z)}$ and $w[ \lpquote y!(z) \rpquote ]$.

\begin{eqnarray}
	\lift{w}{y!(z)}\widehat{\id{\{}u / z \id{\}}}
		& = &
		\lift{w}{y!(u)} \nonumber\\
	w[ \lpquote y!(z) \rpquote ] \widehat{ \id{\{}u / z \id{\}} }
		& = &
		w[ \lpquote y!(z) \rpquote ] \nonumber
\end{eqnarray}

Because the body of the process between quotes is impervious to
substitution, we get radically different answers. In fact, by
examining the first process in an input context,
e.g. $x?(z).\lift{w}{y!(z)}$, we see that the process under the lift
operator may be shaped by prefixed inputs binding a name inside it. In
this sense, the lift operator will be seen as a way to dynamically
construct processes before reifying them as names.

Finally equipped with these standard features we can present the
dynamics of the calculus.

\subsubsection{Operational semantics} 

Finally, we introduce the computational dynamics. What marks these
algebras as distinct from other more traditionally studied algebraic
structures, e.g. vector spaces or polynomial rings, is the manner in
which dynamics is captured. In traditional structures, dynamics is typically
expressed through morphisms between such structures, as in linear maps
between vector spaces or morphisms between rings. In algebras
associated with the semantics of computation, the dynamics is
expressed as part of the algebraic structure itself, through a
reduction reduction relation typically denoted by $\red$. Below, we
give a recursive presentation of this relation for the calculus used
in the encoding.

$\red \subseteq \pi \times \pi$
$\red : \pi \to \mathcal{P}(\pi)$

\begin{mathpar}
  \inferrule* [lab=Comm] { \textsf{match}( x_{src}, x_{trgt} ) } { x_{trgt}?(y)P \; | \; x_{src}!\langle {Q} \rangle \red P\{\quotep{Q}/y}\} }
  \and \\
  \inferrule* [lab=Par] {{P} \red {P}'} {{{P} | {Q}} \red {{P}' | {Q}}}
  \and
  \inferrule* [lab=Equiv]{{{P} \scong {P}'} \andalso {{P}' \red {Q}'} \andalso {{Q}' \scong {Q}}}{{P} \red {Q}}
\end{mathpar}

\begin{eqnarray*}
  match_{\equiv} (\quotep{P},\quotep{Q}) & := & P \equiv Q \\
  match_{\dagger}(\quotep{P},\quotep{Q}) & := & \forall R. P|Q \red^{*} R => R \red^{*} 0 \\
  match_{K}(\quotep{P},\quotep{Q}) & := & K \mbox{ for some context } K
\end{eqnarray*}

$u?(x)P | u!\langle Q \rangle \red P\{\quotep{Q}/x\}$

%We write $\wred$ for $\red^*$, and $P\red$ if $\exists Q $ such that $ P \red Q$.
We write $P\red$ if $\exists Q $ such that $ P \red Q$ and $P\not\red$, otherwise.

\section{Replication}

As mentioned before, it is known that replication (and hence
recursion) can be implemented in a higher-order process algebra
\cite{SangiorgiWalker}. As our first example of calculation with the
machinery thus far presented we give the construction explicitly in
the {\rhoc}.

\begin{eqnarray}
	D_{x} & := & \prefix{x}{y}{(\binpar{\outputp{x}{y}}{@{y}})} \nonumber\\
	\bangp_{x}{P} & := & \binpar{{x}!\langle{\binpar{D_{x}}{P}}\rangle}{D_{x}} \nonumber
\end{eqnarray}

\begin{eqnarray}
	\bangp_{x}{P} & & \nonumber\\
	=
	& {x}!\langle{(\prefix{x}{y}{(\outputp{x}{y} | @{y})) | P}}\rangle 
	      | \prefix{x}{y}{(\outputp{x}{y} | @{y})} & \nonumber\\
	\red
	& (\outputp{x}{y} | @{y})\substn{\quotep{(\prefix{x}{y}{(@{y} | \outputp{x}{y})) | P}}}{y} & \nonumber\\
	=
	& \outputp{x}{\quotep{(\prefix{x}{y}{(\outputp{x}{y} | @{y})) | P}}}
	  | {(\prefix{x}{y}{(\outputp{x}{y} | @{y})) | P}} & \nonumber\\
	\red
	& \ldots & \nonumber\\
	\red^*
	& P | P | \ldots & \nonumber
\end{eqnarray}

Of course, this encoding, as an implementation, runs away, unfolding
$\bangp{P}$ eagerly. A lazier and more implementable replication
operator, restricted to input-guarded processes, may be obtained as follows.

\begin{eqnarray}
\bangp{\prefix{u}{v}{P}} 
	:= 
	\binpar{\lift{x}{\prefix{u}{v}{(\binpar{D(x)}{P})}}}{D(x)} \nonumber
\end{eqnarray}

\begin{remark}
  Note that the lazier definition still does not deal with summation
  or mixed summation (i.e. sums over input and output). The reader is
  invited to construct definitions of replication that deal with these
  features. 

  Further, the definitions are parameterized in a name, $x$. Can you,
  gentle reader, make a definition that eliminates this parameter and
  guarantees no accidental interaction between the replication
  machinery and the process being replicated -- i.e. no accidental
  sharing of names used by the process to get its work done and the
  name(s) used by the replication to effect copying. This latter
  revision of the definition of replication is crucial to obtaining
  the expected identity $!!P \sim !P$.
\end{remark}

\begin{remark}\label{rem:paradoxical_combinator}
  The reader familiar with the lambda calculus will have noticed the
  similarity between $D$ and the paradoxical combinator.

  [Ed. note: the existence of this seems to suggest we have to be more
  restrictive on the set of processes and names we admit if we are to
  support no-cloning.]
\end{remark}

\subsubsection{Bisimulation}

The computational dynamics gives rise to another kind of equivalence,
the equivalence of computational behavior. As previously mentioned
this is typically captured \emph{via} some form of bisimulation.

% The notion we use in this paper is weak barbed bisimulation
% \cite{milner91polyadicpi}.

The notion we use in this paper is derived from weak barbed
bisimulation \cite{milner91polyadicpi}. 

\begin{definition}
An \emph{observation relation}, $\downarrow_{\mathcal N}$, over a set
of names, $\mathcal N$, is the smallest relation satisfying the rules
below.

\infrule[Out-barb]{y \in {\mathcal N}, \; x \nameeq y}
		  {\outputp{x}{v} \downarrow_{\mathcal N} x}
\infrule[Par-barb]{\mbox{$P\downarrow_{\mathcal N} x$ or $Q\downarrow_{\mathcal N} x$}}
		  {\binpar{P}{Q} \downarrow_{\mathcal N} x}

We write $P \Downarrow_{\mathcal N} x$ if there is $Q$ such that 
$P \wred Q$ and $Q \downarrow_{\mathcal N} x$.
\end{definition}

\begin{definition}
%\label{def.bbisim}
An  ${\mathcal N}$-\emph{barbed bisimulation} over a set of names, ${\mathcal N}$, is a symmetric binary relation 
${\mathcal S}_{\mathcal N}$ between agents such that $P\rel{S}_{\mathcal N}Q$ implies:
\begin{enumerate}
\item If $P \red P'$ then $Q \wred Q'$ and $P'\rel{S}_{\mathcal N} Q'$.
\item If $P\downarrow_{\mathcal N} x$, then $Q\Downarrow_{\mathcal N} x$.
\end{enumerate}
$P$ is ${\mathcal N}$-barbed bisimilar to $Q$, written
$P \wbbisim_{\mathcal N} Q$, if $P \rel{S}_{\mathcal N} Q$ for some ${\mathcal N}$-barbed bisimulation ${\mathcal S}_{\mathcal N}$.
\end{definition}

$\mathcal{R} \subseteq \pi \times \pi$

$P \mathcal{R} Q => \forall P'. P \red P' \Rightarrow \exists Q'. Q \red Q', P' \mathcal{R} Q'$

$P \vdash x \Rightarrow Q \vdash x$

\begin{mathpar}
  \inferrule*[lab=Out-barb]{x \nameeq y}{{y}!\langle{Q}\rangle \vdash x}
  \and
  \inferrule*[lab=Par-barb]{\mbox{$P\vdash x$ or $Q\vdash x$}}{\binpar{P}{Q} \vdash x}
\end{mathpar}

\subsubsection{Contexts}

One of the principle advantages of computational calculi like the
$\pi$-calculus is a well-defined notion of context,
contextual-equivalence and a correlation between
contextual-equivalence and notions of bisimulation. The notion of
context allows the decomposition of a process into (sub-)process and
its syntactic environment, its context. Thus, a context may be
thought of as a process with a ``hole'' (written $\Box$) in it. The
application of a context $M$ to a process $P$, written $M[P]$, is
tantamount to filling the hole in $M$ with $P$. In this paper we do
not need the full weight of this theory, but do make use of the notion
of context in the proof the main theorem. 

\begin{mathpar}
  \inferrule* [lab=summation] {} {{M_{M},M_{N}} \bc \Box \;|\; x.M_{A} \;|\; M_{M}+M_{N}}
  \and
  \inferrule* [lab=agent] {} {{M_{A}} \bc (\vec{x})M_{P} \;| \; \clift{P_0,\ldots,M_{P},\ldots,P_N}}
  \and \\
  \inferrule* [lab=process] {} {{M_{P}} \bc M_{N} \;| \;P|M_{P} }
\end{mathpar} 

\begin{mathpar}
  \inferrule* [lab=sychronization] {} {M_{N} \bc \Box \;|\; x?M_{F} \;|\; x!M_{C}}
  \and
  \inferrule* [lab=abstraction] {} {{M_{F}} \bc (x)M_{P} }
  \and
  \inferrule* [lab=concretion] {} {{M_{C}} \bc \langle M_{P} \rangle }
  \and \\
  \inferrule* [lab=process] {} {{M_{P}} \bc M_{N} \;| \;P|M_{P} }
\end{mathpar}

\begin{definition}[contextual application] Given a context $M$, and
  process $P$, we define the \emph{contextual application}, $M[P] :=
  M\{P/\Box\}$. That is, the contextual application of M to P is the
  substitution of $P$ for $\Box$ in $M$.
\end{definition}

$\meaningof{-} : L \to \mathcal{P}(\pi)$

\begin{mathpar}
  \inferrule* [lab=collection] {} {\meaningof{true} = \pi, \and \meaningof{~E} = \pi \setminus \meaningof{E}, \and \meaningof{E_{1} \& E_{2}} = \meaningof{E_{1}} \cap \meaningof{E_{2}}}
\end{mathpar}

\begin{mathpar}
  \inferrule* [lab=structure] {} {\meaningof{0} = \{ P \in \pi | P \equiv 0 \}, \and \\ \meaningof{E_1 | E_2} = \{ P \in \pi | P \equiv P_{1} | P_{2}, P_{1} \in \meaningof{E_{1}}, P_{2} \in \meaningof{E_2}\} }
\end{mathpar}

\begin{mathpar}
 \inferrule* [lab=behavior] {} {\meaningof{\langle a?b \rangle E} = \{ P \in \pi | P \equiv Q | u?(y)P', \\ \and \\\\ \and \\ \;\;\; u \in \meaningof{a}, \forall z.P'\{z/y\} \in \meaningof{E\{z/b\}}\}, \and \\ \meaningof{a!E} = \{ P \in \pi | P \equiv Q | x!\langle P' \rangle, x \in \meaningof{a} P' \in \meaningof{E}\} }
\end{mathpar}

\begin{mathpar}
 \inferrule* [lab=nominal] {} {\meaningof{\quotep{E}} = \{ \quotep{P} \in \quotep{\pi} | P \in \meaningof{E} \}, \and \meaningof{\quotep{P}} = \{ \quotep{Q} \in \quotep{\pi} | P \equiv Q \} \and \\ \meaningof{@\quotep{E}} = \{ P \in \pi | P \equiv @x, x \in \meaningof{E} \}}
\end{mathpar}

\begin{eqnarray*}
  \\
  \meaningof{-} : TS \to ST
\end{eqnarray*}

\begin{eqnarray*}
  \\
  L : TS \to ST
\end{eqnarray*}

\begin{eqnarray*}
  \\
  P \models E \iff P \in \meaningof{E}
\end{eqnarray*}

\begin{eqnarray*}
  P \approx_{L} Q \iff \forall E \in L. P \models E \iff Q \models E
\end{eqnarray*}

\begin{eqnarray*}
  P \approx_{K} Q
\end{eqnarray*}

\begin{eqnarray*}
  P \approx Q
\end{eqnarray*}

$\approx_{K} = \approx = \approx_{L}$

\subsubsection{Contextual duality}

Note that contexts extend the quotation operation to a family of
operations from processes to names. Given a context, $M$, we can
define a \emph{nominal context}, $\quotep{M}$ by $\quotep{M}[P] :=
\quotep{M[P]}$. To foreshadow what is to come we observe that these
operations enjoy a duality with processes very much like the duality
between vectors and maps from vectors to scalars.

Further, because the calculus is essentially higher-order, we have a
correspondence between contexts and processes. More specifically,
given a name $x$ and a context $M$ we can construct $M^{*}_{x}$ such
that 

\begin{mathpar}
  M^{*}_{x} | \lift{x}{P} \red M[P]
\end{mathpar}

namely,

\begin{mathpar}
  M^{*}_{x} := x?(u).M[\dropn{u}]
\end{mathpar}

The dependence of $M^{*}_{x}$ on a name makes it an abstraction, 

\begin{mathpar}
  M^{*} := (x)x?(u).M[\dropn{u}]
\end{mathpar}

\subsection{Additional notation}

It will sometimes be convenient to denote the process a name
quotes. We already have the notation $x = \quotep{P}$, but it will be
convenient to introduce an alternate notation, $\procn{x}$, when we
want to emphasize the connection to the use of the name. Note that, by
virtue of name equivalence, $\quotep{\procn{x}} \nameeq x$; so, the
notation is consistent with previous definitions.

Further, because names have structure it is possible to effect
substitutions on the basis of that structure. This means we need to
upgrade our notation for substitutions, which we accomplish by
adapting comprehension notation. Thus,

\begin{mathpar}
  P\{ y / x : x \in S \}
\end{mathpar}

is interpreted to mean the process derived from P by replacing (in a
capture-avoiding manner) each occurrence of $x$ in $S$ by $y$. For example,

\begin{mathpar}
  P\{ \quotep{\procn{x}|\procn{x}} / x : x \in \freenames{P} \}
\end{mathpar}

will replace each (occurrence) of a free name $x$ in $P$ by
$\quotep{\procn{x}|\procn{x}}$.

Also, we will avail ourselves of the notation $x^{L}$ and $x^{R}$ to
denote injections of a name into disjoint copies of the name
space. There are numerous ways to accomplish this. One example can be
found in \cite{MeredithR05}. This notation overloads to vectors of
names: $\vec{x}^{\pi} := (x_{i}^{\pi} \; : \; 0 \leq i < |\vec{x}| )$ where $\pi \in \{L,R\}$.

We also use $P^{\Box} := P|\Box$.

In \cite{MeredithR05} an interpretation of the new operator is
given. It turns out that there are several possible interpretations
all enjoying the requisite algebraic properties of the operator (see
\cite{milner91polyadicpi}). We will therefore make liberal use of
$(\nu\; \vec{x})P$.

% subsection the_syntax_and_semantics_of_the_notation_system (end)   

\input{qm2pi.qmops} 

\input{qm2pi.sterngerlach} 

\input{qm2pi.metric} 

% section concurrent_process_calculi (end)

%\input{qm2pi.proofsketch}

% section proof sketch (end)

%\input{qm2pi.slviaknots} 

% section spatial logic via knots (end)

\input{qm2pi.conclusion}

% section conclusion (end)

%\input{qm2pi.dtcodes} 

% section wiring algorithm (end)

\input{qm2pi.ack} 

% section acknowledgments (end)

\newpage


\bibliographystyle{plain}   
\bibliography{../../biblios/main.bib}

\input{qm2pi.rhodetails}

\end{document}

 

% section wiring algorithm (end)

\documentclass[12pt]{llncs}
%\documentclass{jktr}

\usepackage[pdftex]{hyperref}                   
\usepackage {listings}
\usepackage {mathpartir}
\usepackage{bcprules}
%\usepackage{listings}
                       
\usepackage{graphicx} 
%\usepackage[margins=2.5cm,nohead,nofoot]{geometry}
%\usepackage{geometry}
\usepackage{amsfonts}
\usepackage{amstext}
\usepackage{latexsym}
\usepackage{amssymb}
\usepackage{color}


%\include{myPreamble}
\include{qm2pi.local} 

%\ifpdf
%\usepackage[pdftex]{graphicx}
%\else
%\usepackage{graphicx}
%\fi

 % \ifpdf
%  \usepackage{pdfsync}
%  \if


%\title{Brief Article}
%\author{David F. Snyder}
%\author{L.G. Meredith}

%\address{Dept. of Math., Texas State University--San Marcos, San Marcos, TX 78666}
       
\pagestyle{empty}


\begin{document}

\lstset{language=[Objective]Caml,frame=shadowbox}

\input{qm2pi.front}

% section front matter (end)

\input{qm2pi.intro} 
 
% section introduction (end)

% \input{qm2pi.knotations} 

% section notation (end)

\input{qm2pi.process.calculi} 

% section concurrent_process_calculi_and_spatial_logics_ (end)
    
%\input{qm2pi.knots2pi} 

%\input{qm2pi.trefoil} 

%\input{qm2pi.mainthm} 

% subsection basic_interpretation (end)

%\input{qm2pi.rho.presentation} 
\subsection{The syntax and semantics of the notation system}\label{sub:the_syntax_and_semantics_of_the_notation_system} % (fold)

We now summarize a technical presentation of the calculus that
embodies our theory of dynamics. The typical presentation of such a
calculus follows the style of giving generators and relations on
them. The grammar, below, describing term constructors, freely
generates the set of processes, $\Proc$. This set is then quotiented
by a relation known as structural congruence and it is over this set
that the notion of dynamics is expressed. This presentation is
essentially that of \cite{MeredithR05} with the addition of
polyadicity and summation. For readability we have relegated some of
the technical subtleties to an appendix.

\subsubsection{Process grammar}\label{subsub:process_grammar}

\begin{mathpar}
  \inferrule* [lab=synchronization] {} {{M} \bc \pzero \;|\; x?F \;|\; x!C }
  \and
  \inferrule* [lab=abstraction] {} {{F} \bc (x)P}
  \and
  \inferrule* [lab=concretion] {} {{C} \bc \langle Q \rangle}
  \and
  \inferrule* [lab=process] {} {{P,Q} \bc M \;| \;P|Q \;|\; @{x}}
  \and
  \inferrule* [lab=name] {} {{x} \bc \quotep{P}}
\end{mathpar} 

Note that $\vec{x}$ (resp. $\vec{P}$) denotes a vector of names
(resp. processes) of length $|\vec{x}|$ (resp. $|\vec{P}|$). We adopt
the following useful abbreviations.

\begin{mathpar}
   x?(\vec{y}).P := x.(\vec{y})P \and  x\clift{\vec{P}} := x.\clift{\vec{P}}
   \and x!(y) := \lift{x}{\dropn{y}}
   \and \Pi_{i=0}^{n-1}P_i := P_0 | \ldots | P_{n-1}
\end{mathpar}

\subsubsection{Structural congruence}

\paragraph{Free and bound names and alpha-equivalence.} At the
core of structural equivalence is alpha-equivalence which identifies
process that are the same up to a change of variable. Formally, we
recognize the distinction between free and bound names. The free names
of a process, $\freenames{P}$, may be calculated recursively as
follows:

\begin{mathpar}
\freenames{\pzero} := \emptyset
  \and \\
  \freenames{x?(y).P} := \{ x \} \cup (\freenames{P} \setminus \{ y \})
  \and 
  \freenames{x!\langle P \rangle} := \{ x \} \cup \{ P \} 
  \and \\
  \freenames{P|Q} := \freenames{P} \cup \freenames{Q}
  \and \\
  \freenames{@{x}} := \{ x \}
\end{mathpar}

$\pi$
$\quotep{\pi}$

$\freenames{-} : \pi \to \mathcal{P}(\quotep{\pi})$

\begin{eqnarray*}
  \freenames{\pzero} & := & \emptyset \\
  \freenames{x?(y).P} & := & \{ x \} \cup (\freenames{P} \setminus \{ y \}) \\
  \freenames{x!\langle P \rangle} & := & \{ x \} \cup \{ P \} \\
  \freenames{P|Q} & := & \freenames{P} \cup \freenames{Q} \\
  \freenames{\dropn{x}} & := & \{ x \}
\end{eqnarray*}

The bound names of a process, $\boundnames{P}$, are those names occurring in $P$
that are not free. For example, in $x?(y).0$, the name $x$ is free, while $y$ is bound.

\begin{mathpar}
  \inferrule* [lab=monoidal-laws] {} { P|Q \equiv Q|P \and P|0 \equiv P \and P|(Q|R) \equiv (P|Q)|R }
\end{mathpar}

\begin{mathpar}
  \inferrule* [lab=alpha-equivalence] {} { (x)P \equiv (y)P\{y/x\} \and y \not\in \freenames{P} }
\end{mathpar}

\begin{definition}
Then two processes, $P,Q$, are alpha-equivalent if $P = Q\{\vec{y}/\vec{x}\}$ for
some $\vec{x} \in \boundnames{Q},\vec{y} \in \boundnames{P}$, where $Q\{\vec{y}/\vec{x}\}$
denotes the capture-avoiding substitution of $\vec{y}$ for $\vec{x}$ in $Q$.
\end{definition}

\begin{definition}
  The {\em structural congruence} \cite{SangiorgiWalker} , $\equiv$,
  between processes is the least congruence containing
  alpha-equivalence, satisfying the abelian monoid laws
  (associativity, commutativity and $\pzero$ as identity) for parallel
  composition $|$ and for summation $+$.
\end{definition}

\subsection{Name equivalence}

We take name equivalence, written $\nameeq$, to be the smallest
equivalence relation generated by the following rules.

\begin{mathpar}
\inferrule*[lab=Quote-drop]
{ }
{ \quotep{@{x}} \nameeq x }

\inferrule*[lab=Struct-equiv]
{ P \scong Q }
{ \quotep{P} \nameeq \quotep{Q} }
\end{mathpar}

The astute reader will have noticed that the mutual recursion of names
and processes imposes a mutual recursion on alpha-equivalence and
structural equivalence via name-equivalence. Fortunately, all of this
works out pleasantly and we may calculate in the natural way, free of
concern. The reader interested in the details is referred to the
appendix \ref{appendix:rho_details}.

\subsection{Substitution}

We use $\Proc$ for the set of processes, $\QProc$ for the set of
names, and $\id{\{}\vec{y} / \vec{x} \id{\}}$ to denote partial maps,
$s : \QProc \rightarrow \QProc$. A map, $s$ lifts, uniquely, to a map
on process terms, $\widehat{s} : \Proc \rightarrow \Proc$ by the
following equations.

\begin{mathpar}
  (0) \psubstp{Q}{P} := 0 \\
  (R \juxtap S) \psubstp{Q}{P}
  :=    
  (R)\psubstp{Q}{P} \juxtap (S) \psubstp{Q}{P} \\
  (x?(y).R) \psubstp{Q}{P}    
  :=    
  (x)\substp{Q}{P} (z)\concat( (R \psubstn{z}{y}) \psubstp{Q}{P} ) \\
  (\lift{x}{R}) \psubstp{Q}{P}  
  :=
  \lift{(x)\substp{Q}{P}}{ R \psubstp{Q}{P} } \\
%   (\dropn{x})  \psubstp{Q}{P}       
%   := 
%   \left\{ 
%     \begin{array}{ccc} 
%       \dropn{\quotep{Q}} & & x \nameeq \quotep{P} \\
%       \dropn{x} & & otherwise \\
%     \end{array}
%   \right. 
  (\dropn{x})  \psubstp{Q}{P}       
  := 
  \left\{ 
    \begin{array}{ccc} 
      Q & & x \nameeq \quotep{P} \\
      \dropn{x} & & otherwise \\
    \end{array}
  \right.
\end{mathpar}
 

where

\begin{eqnarray}
  (x)\id{\{} \lpquote Q \rpquote / \lpquote P \rpquote \id{\}}            = 
  \left\{ 
    \begin{array}{ccc}
      \lpquote Q \rpquote & & x \nameeq \lpquote P \rpquote \\
      x & & otherwise \\
    \end{array}
  \right. \nonumber
\end{eqnarray}

and $z$ is chosen distinct from $\quotep{P}$, $\quotep{Q}$, the free
names in $Q$, and all the names in $R$. Our $\alpha$-equivalence will
be built in the standard way from this substitution.

\begin{remark}\label{rem:no_self_referential_names}
  One consequence of these definitions is that $\forall P. \quotep{P}
  \not\in \freenames{P}$.
\end{remark}

\subsection{ Dynamic quote: an example }

Anticipating something of what's to come, consider applying the
substitution, $\widehat{\id{\{}u / z \id{\}}}$, to the following pair
of processes, $\lift{w}{y!(z)}$ and $w[ \lpquote y!(z) \rpquote ]$.

\begin{eqnarray}
	\lift{w}{y!(z)}\widehat{\id{\{}u / z \id{\}}}
		& = &
		\lift{w}{y!(u)} \nonumber\\
	w[ \lpquote y!(z) \rpquote ] \widehat{ \id{\{}u / z \id{\}} }
		& = &
		w[ \lpquote y!(z) \rpquote ] \nonumber
\end{eqnarray}

Because the body of the process between quotes is impervious to
substitution, we get radically different answers. In fact, by
examining the first process in an input context,
e.g. $x?(z).\lift{w}{y!(z)}$, we see that the process under the lift
operator may be shaped by prefixed inputs binding a name inside it. In
this sense, the lift operator will be seen as a way to dynamically
construct processes before reifying them as names.

Finally equipped with these standard features we can present the
dynamics of the calculus.

\subsubsection{Operational semantics} 

Finally, we introduce the computational dynamics. What marks these
algebras as distinct from other more traditionally studied algebraic
structures, e.g. vector spaces or polynomial rings, is the manner in
which dynamics is captured. In traditional structures, dynamics is typically
expressed through morphisms between such structures, as in linear maps
between vector spaces or morphisms between rings. In algebras
associated with the semantics of computation, the dynamics is
expressed as part of the algebraic structure itself, through a
reduction reduction relation typically denoted by $\red$. Below, we
give a recursive presentation of this relation for the calculus used
in the encoding.

$\red \subseteq \pi \times \pi$
$\red : \pi \to \mathcal{P}(\pi)$

\begin{mathpar}
  \inferrule* [lab=Comm] { \textsf{match}( x_{src}, x_{trgt} ) } { x_{trgt}?(y)P \; | \; x_{src}!\langle {Q} \rangle \red P\{\quotep{Q}/y}\} }
  \and \\
  \inferrule* [lab=Par] {{P} \red {P}'} {{{P} | {Q}} \red {{P}' | {Q}}}
  \and
  \inferrule* [lab=Equiv]{{{P} \scong {P}'} \andalso {{P}' \red {Q}'} \andalso {{Q}' \scong {Q}}}{{P} \red {Q}}
\end{mathpar}

\begin{eqnarray*}
  match_{\equiv} (\quotep{P},\quotep{Q}) & := & P \equiv Q \\
  match_{\dagger}(\quotep{P},\quotep{Q}) & := & \forall R. P|Q \red^{*} R => R \red^{*} 0 \\
  match_{K}(\quotep{P},\quotep{Q}) & := & K \mbox{ for some context } K
\end{eqnarray*}

$u?(x)P | u!\langle Q \rangle \red P\{\quotep{Q}/x\}$

%We write $\wred$ for $\red^*$, and $P\red$ if $\exists Q $ such that $ P \red Q$.
We write $P\red$ if $\exists Q $ such that $ P \red Q$ and $P\not\red$, otherwise.

\section{Replication}

As mentioned before, it is known that replication (and hence
recursion) can be implemented in a higher-order process algebra
\cite{SangiorgiWalker}. As our first example of calculation with the
machinery thus far presented we give the construction explicitly in
the {\rhoc}.

\begin{eqnarray}
	D_{x} & := & \prefix{x}{y}{(\binpar{\outputp{x}{y}}{@{y}})} \nonumber\\
	\bangp_{x}{P} & := & \binpar{{x}!\langle{\binpar{D_{x}}{P}}\rangle}{D_{x}} \nonumber
\end{eqnarray}

\begin{eqnarray}
	\bangp_{x}{P} & & \nonumber\\
	=
	& {x}!\langle{(\prefix{x}{y}{(\outputp{x}{y} | @{y})) | P}}\rangle 
	      | \prefix{x}{y}{(\outputp{x}{y} | @{y})} & \nonumber\\
	\red
	& (\outputp{x}{y} | @{y})\substn{\quotep{(\prefix{x}{y}{(@{y} | \outputp{x}{y})) | P}}}{y} & \nonumber\\
	=
	& \outputp{x}{\quotep{(\prefix{x}{y}{(\outputp{x}{y} | @{y})) | P}}}
	  | {(\prefix{x}{y}{(\outputp{x}{y} | @{y})) | P}} & \nonumber\\
	\red
	& \ldots & \nonumber\\
	\red^*
	& P | P | \ldots & \nonumber
\end{eqnarray}

Of course, this encoding, as an implementation, runs away, unfolding
$\bangp{P}$ eagerly. A lazier and more implementable replication
operator, restricted to input-guarded processes, may be obtained as follows.

\begin{eqnarray}
\bangp{\prefix{u}{v}{P}} 
	:= 
	\binpar{\lift{x}{\prefix{u}{v}{(\binpar{D(x)}{P})}}}{D(x)} \nonumber
\end{eqnarray}

\begin{remark}
  Note that the lazier definition still does not deal with summation
  or mixed summation (i.e. sums over input and output). The reader is
  invited to construct definitions of replication that deal with these
  features. 

  Further, the definitions are parameterized in a name, $x$. Can you,
  gentle reader, make a definition that eliminates this parameter and
  guarantees no accidental interaction between the replication
  machinery and the process being replicated -- i.e. no accidental
  sharing of names used by the process to get its work done and the
  name(s) used by the replication to effect copying. This latter
  revision of the definition of replication is crucial to obtaining
  the expected identity $!!P \sim !P$.
\end{remark}

\begin{remark}\label{rem:paradoxical_combinator}
  The reader familiar with the lambda calculus will have noticed the
  similarity between $D$ and the paradoxical combinator.

  [Ed. note: the existence of this seems to suggest we have to be more
  restrictive on the set of processes and names we admit if we are to
  support no-cloning.]
\end{remark}

\subsubsection{Bisimulation}

The computational dynamics gives rise to another kind of equivalence,
the equivalence of computational behavior. As previously mentioned
this is typically captured \emph{via} some form of bisimulation.

% The notion we use in this paper is weak barbed bisimulation
% \cite{milner91polyadicpi}.

The notion we use in this paper is derived from weak barbed
bisimulation \cite{milner91polyadicpi}. 

\begin{definition}
An \emph{observation relation}, $\downarrow_{\mathcal N}$, over a set
of names, $\mathcal N$, is the smallest relation satisfying the rules
below.

\infrule[Out-barb]{y \in {\mathcal N}, \; x \nameeq y}
		  {\outputp{x}{v} \downarrow_{\mathcal N} x}
\infrule[Par-barb]{\mbox{$P\downarrow_{\mathcal N} x$ or $Q\downarrow_{\mathcal N} x$}}
		  {\binpar{P}{Q} \downarrow_{\mathcal N} x}

We write $P \Downarrow_{\mathcal N} x$ if there is $Q$ such that 
$P \wred Q$ and $Q \downarrow_{\mathcal N} x$.
\end{definition}

\begin{definition}
%\label{def.bbisim}
An  ${\mathcal N}$-\emph{barbed bisimulation} over a set of names, ${\mathcal N}$, is a symmetric binary relation 
${\mathcal S}_{\mathcal N}$ between agents such that $P\rel{S}_{\mathcal N}Q$ implies:
\begin{enumerate}
\item If $P \red P'$ then $Q \wred Q'$ and $P'\rel{S}_{\mathcal N} Q'$.
\item If $P\downarrow_{\mathcal N} x$, then $Q\Downarrow_{\mathcal N} x$.
\end{enumerate}
$P$ is ${\mathcal N}$-barbed bisimilar to $Q$, written
$P \wbbisim_{\mathcal N} Q$, if $P \rel{S}_{\mathcal N} Q$ for some ${\mathcal N}$-barbed bisimulation ${\mathcal S}_{\mathcal N}$.
\end{definition}

$\mathcal{R} \subseteq \pi \times \pi$

$P \mathcal{R} Q => \forall P'. P \red P' \Rightarrow \exists Q'. Q \red Q', P' \mathcal{R} Q'$

$P \vdash x \Rightarrow Q \vdash x$

\begin{mathpar}
  \inferrule*[lab=Out-barb]{x \nameeq y}{{y}!\langle{Q}\rangle \vdash x}
  \and
  \inferrule*[lab=Par-barb]{\mbox{$P\vdash x$ or $Q\vdash x$}}{\binpar{P}{Q} \vdash x}
\end{mathpar}

\subsubsection{Contexts}

One of the principle advantages of computational calculi like the
$\pi$-calculus is a well-defined notion of context,
contextual-equivalence and a correlation between
contextual-equivalence and notions of bisimulation. The notion of
context allows the decomposition of a process into (sub-)process and
its syntactic environment, its context. Thus, a context may be
thought of as a process with a ``hole'' (written $\Box$) in it. The
application of a context $M$ to a process $P$, written $M[P]$, is
tantamount to filling the hole in $M$ with $P$. In this paper we do
not need the full weight of this theory, but do make use of the notion
of context in the proof the main theorem. 

\begin{mathpar}
  \inferrule* [lab=summation] {} {{M_{M},M_{N}} \bc \Box \;|\; x.M_{A} \;|\; M_{M}+M_{N}}
  \and
  \inferrule* [lab=agent] {} {{M_{A}} \bc (\vec{x})M_{P} \;| \; \clift{P_0,\ldots,M_{P},\ldots,P_N}}
  \and \\
  \inferrule* [lab=process] {} {{M_{P}} \bc M_{N} \;| \;P|M_{P} }
\end{mathpar} 

\begin{mathpar}
  \inferrule* [lab=sychronization] {} {M_{N} \bc \Box \;|\; x?M_{F} \;|\; x!M_{C}}
  \and
  \inferrule* [lab=abstraction] {} {{M_{F}} \bc (x)M_{P} }
  \and
  \inferrule* [lab=concretion] {} {{M_{C}} \bc \langle M_{P} \rangle }
  \and \\
  \inferrule* [lab=process] {} {{M_{P}} \bc M_{N} \;| \;P|M_{P} }
\end{mathpar}

\begin{definition}[contextual application] Given a context $M$, and
  process $P$, we define the \emph{contextual application}, $M[P] :=
  M\{P/\Box\}$. That is, the contextual application of M to P is the
  substitution of $P$ for $\Box$ in $M$.
\end{definition}

$\meaningof{-} : L \to \mathcal{P}(\pi)$

\begin{mathpar}
  \inferrule* [lab=collection] {} {\meaningof{true} = \pi, \and \meaningof{~E} = \pi \setminus \meaningof{E}, \and \meaningof{E_{1} \& E_{2}} = \meaningof{E_{1}} \cap \meaningof{E_{2}}}
\end{mathpar}

\begin{mathpar}
  \inferrule* [lab=structure] {} {\meaningof{0} = \{ P \in \pi | P \equiv 0 \}, \and \\ \meaningof{E_1 | E_2} = \{ P \in \pi | P \equiv P_{1} | P_{2}, P_{1} \in \meaningof{E_{1}}, P_{2} \in \meaningof{E_2}\} }
\end{mathpar}

\begin{mathpar}
 \inferrule* [lab=behavior] {} {\meaningof{\langle a?b \rangle E} = \{ P \in \pi | P \equiv Q | u?(y)P', \\ \and \\\\ \and \\ \;\;\; u \in \meaningof{a}, \forall z.P'\{z/y\} \in \meaningof{E\{z/b\}}\}, \and \\ \meaningof{a!E} = \{ P \in \pi | P \equiv Q | x!\langle P' \rangle, x \in \meaningof{a} P' \in \meaningof{E}\} }
\end{mathpar}

\begin{mathpar}
 \inferrule* [lab=nominal] {} {\meaningof{\quotep{E}} = \{ \quotep{P} \in \quotep{\pi} | P \in \meaningof{E} \}, \and \meaningof{\quotep{P}} = \{ \quotep{Q} \in \quotep{\pi} | P \equiv Q \} \and \\ \meaningof{@\quotep{E}} = \{ P \in \pi | P \equiv @x, x \in \meaningof{E} \}}
\end{mathpar}

\begin{eqnarray*}
  \\
  \meaningof{-} : TS \to ST
\end{eqnarray*}

\begin{eqnarray*}
  \\
  L : TS \to ST
\end{eqnarray*}

\begin{eqnarray*}
  \\
  P \models E \iff P \in \meaningof{E}
\end{eqnarray*}

\begin{eqnarray*}
  P \approx_{L} Q \iff \forall E \in L. P \models E \iff Q \models E
\end{eqnarray*}

\begin{eqnarray*}
  P \approx_{K} Q
\end{eqnarray*}

\begin{eqnarray*}
  P \approx Q
\end{eqnarray*}

$\approx_{K} = \approx = \approx_{L}$

\subsubsection{Contextual duality}

Note that contexts extend the quotation operation to a family of
operations from processes to names. Given a context, $M$, we can
define a \emph{nominal context}, $\quotep{M}$ by $\quotep{M}[P] :=
\quotep{M[P]}$. To foreshadow what is to come we observe that these
operations enjoy a duality with processes very much like the duality
between vectors and maps from vectors to scalars.

Further, because the calculus is essentially higher-order, we have a
correspondence between contexts and processes. More specifically,
given a name $x$ and a context $M$ we can construct $M^{*}_{x}$ such
that 

\begin{mathpar}
  M^{*}_{x} | \lift{x}{P} \red M[P]
\end{mathpar}

namely,

\begin{mathpar}
  M^{*}_{x} := x?(u).M[\dropn{u}]
\end{mathpar}

The dependence of $M^{*}_{x}$ on a name makes it an abstraction, 

\begin{mathpar}
  M^{*} := (x)x?(u).M[\dropn{u}]
\end{mathpar}

\subsection{Additional notation}

It will sometimes be convenient to denote the process a name
quotes. We already have the notation $x = \quotep{P}$, but it will be
convenient to introduce an alternate notation, $\procn{x}$, when we
want to emphasize the connection to the use of the name. Note that, by
virtue of name equivalence, $\quotep{\procn{x}} \nameeq x$; so, the
notation is consistent with previous definitions.

Further, because names have structure it is possible to effect
substitutions on the basis of that structure. This means we need to
upgrade our notation for substitutions, which we accomplish by
adapting comprehension notation. Thus,

\begin{mathpar}
  P\{ y / x : x \in S \}
\end{mathpar}

is interpreted to mean the process derived from P by replacing (in a
capture-avoiding manner) each occurrence of $x$ in $S$ by $y$. For example,

\begin{mathpar}
  P\{ \quotep{\procn{x}|\procn{x}} / x : x \in \freenames{P} \}
\end{mathpar}

will replace each (occurrence) of a free name $x$ in $P$ by
$\quotep{\procn{x}|\procn{x}}$.

Also, we will avail ourselves of the notation $x^{L}$ and $x^{R}$ to
denote injections of a name into disjoint copies of the name
space. There are numerous ways to accomplish this. One example can be
found in \cite{MeredithR05}. This notation overloads to vectors of
names: $\vec{x}^{\pi} := (x_{i}^{\pi} \; : \; 0 \leq i < |\vec{x}| )$ where $\pi \in \{L,R\}$.

We also use $P^{\Box} := P|\Box$.

In \cite{MeredithR05} an interpretation of the new operator is
given. It turns out that there are several possible interpretations
all enjoying the requisite algebraic properties of the operator (see
\cite{milner91polyadicpi}). We will therefore make liberal use of
$(\nu\; \vec{x})P$.

% subsection the_syntax_and_semantics_of_the_notation_system (end)   

\input{qm2pi.qmops} 

\input{qm2pi.sterngerlach} 

\input{qm2pi.metric} 

% section concurrent_process_calculi (end)

%\input{qm2pi.proofsketch}

% section proof sketch (end)

%\input{qm2pi.slviaknots} 

% section spatial logic via knots (end)

\input{qm2pi.conclusion}

% section conclusion (end)

%\input{qm2pi.dtcodes} 

% section wiring algorithm (end)

\input{qm2pi.ack} 

% section acknowledgments (end)

\newpage


\bibliographystyle{plain}   
\bibliography{../../biblios/main.bib}

\input{qm2pi.rhodetails}

\end{document}

 

% section acknowledgments (end)

\newpage


\bibliographystyle{plain}   
\bibliography{../../biblios/main.bib}

\documentclass[12pt]{llncs}
%\documentclass{jktr}

\usepackage[pdftex]{hyperref}                   
\usepackage {listings}
\usepackage {mathpartir}
\usepackage{bcprules}
%\usepackage{listings}
                       
\usepackage{graphicx} 
%\usepackage[margins=2.5cm,nohead,nofoot]{geometry}
%\usepackage{geometry}
\usepackage{amsfonts}
\usepackage{amstext}
\usepackage{latexsym}
\usepackage{amssymb}
\usepackage{color}


%\include{myPreamble}
\include{qm2pi.local} 

%\ifpdf
%\usepackage[pdftex]{graphicx}
%\else
%\usepackage{graphicx}
%\fi

 % \ifpdf
%  \usepackage{pdfsync}
%  \if


%\title{Brief Article}
%\author{David F. Snyder}
%\author{L.G. Meredith}

%\address{Dept. of Math., Texas State University--San Marcos, San Marcos, TX 78666}
       
\pagestyle{empty}


\begin{document}

\lstset{language=[Objective]Caml,frame=shadowbox}

\input{qm2pi.front}

% section front matter (end)

\input{qm2pi.intro} 
 
% section introduction (end)

% \input{qm2pi.knotations} 

% section notation (end)

\input{qm2pi.process.calculi} 

% section concurrent_process_calculi_and_spatial_logics_ (end)
    
%\input{qm2pi.knots2pi} 

%\input{qm2pi.trefoil} 

%\input{qm2pi.mainthm} 

% subsection basic_interpretation (end)

%\input{qm2pi.rho.presentation} 
\subsection{The syntax and semantics of the notation system}\label{sub:the_syntax_and_semantics_of_the_notation_system} % (fold)

We now summarize a technical presentation of the calculus that
embodies our theory of dynamics. The typical presentation of such a
calculus follows the style of giving generators and relations on
them. The grammar, below, describing term constructors, freely
generates the set of processes, $\Proc$. This set is then quotiented
by a relation known as structural congruence and it is over this set
that the notion of dynamics is expressed. This presentation is
essentially that of \cite{MeredithR05} with the addition of
polyadicity and summation. For readability we have relegated some of
the technical subtleties to an appendix.

\subsubsection{Process grammar}\label{subsub:process_grammar}

\begin{mathpar}
  \inferrule* [lab=synchronization] {} {{M} \bc \pzero \;|\; x?F \;|\; x!C }
  \and
  \inferrule* [lab=abstraction] {} {{F} \bc (x)P}
  \and
  \inferrule* [lab=concretion] {} {{C} \bc \langle Q \rangle}
  \and
  \inferrule* [lab=process] {} {{P,Q} \bc M \;| \;P|Q \;|\; @{x}}
  \and
  \inferrule* [lab=name] {} {{x} \bc \quotep{P}}
\end{mathpar} 

Note that $\vec{x}$ (resp. $\vec{P}$) denotes a vector of names
(resp. processes) of length $|\vec{x}|$ (resp. $|\vec{P}|$). We adopt
the following useful abbreviations.

\begin{mathpar}
   x?(\vec{y}).P := x.(\vec{y})P \and  x\clift{\vec{P}} := x.\clift{\vec{P}}
   \and x!(y) := \lift{x}{\dropn{y}}
   \and \Pi_{i=0}^{n-1}P_i := P_0 | \ldots | P_{n-1}
\end{mathpar}

\subsubsection{Structural congruence}

\paragraph{Free and bound names and alpha-equivalence.} At the
core of structural equivalence is alpha-equivalence which identifies
process that are the same up to a change of variable. Formally, we
recognize the distinction between free and bound names. The free names
of a process, $\freenames{P}$, may be calculated recursively as
follows:

\begin{mathpar}
\freenames{\pzero} := \emptyset
  \and \\
  \freenames{x?(y).P} := \{ x \} \cup (\freenames{P} \setminus \{ y \})
  \and 
  \freenames{x!\langle P \rangle} := \{ x \} \cup \{ P \} 
  \and \\
  \freenames{P|Q} := \freenames{P} \cup \freenames{Q}
  \and \\
  \freenames{@{x}} := \{ x \}
\end{mathpar}

$\pi$
$\quotep{\pi}$

$\freenames{-} : \pi \to \mathcal{P}(\quotep{\pi})$

\begin{eqnarray*}
  \freenames{\pzero} & := & \emptyset \\
  \freenames{x?(y).P} & := & \{ x \} \cup (\freenames{P} \setminus \{ y \}) \\
  \freenames{x!\langle P \rangle} & := & \{ x \} \cup \{ P \} \\
  \freenames{P|Q} & := & \freenames{P} \cup \freenames{Q} \\
  \freenames{\dropn{x}} & := & \{ x \}
\end{eqnarray*}

The bound names of a process, $\boundnames{P}$, are those names occurring in $P$
that are not free. For example, in $x?(y).0$, the name $x$ is free, while $y$ is bound.

\begin{mathpar}
  \inferrule* [lab=monoidal-laws] {} { P|Q \equiv Q|P \and P|0 \equiv P \and P|(Q|R) \equiv (P|Q)|R }
\end{mathpar}

\begin{mathpar}
  \inferrule* [lab=alpha-equivalence] {} { (x)P \equiv (y)P\{y/x\} \and y \not\in \freenames{P} }
\end{mathpar}

\begin{definition}
Then two processes, $P,Q$, are alpha-equivalent if $P = Q\{\vec{y}/\vec{x}\}$ for
some $\vec{x} \in \boundnames{Q},\vec{y} \in \boundnames{P}$, where $Q\{\vec{y}/\vec{x}\}$
denotes the capture-avoiding substitution of $\vec{y}$ for $\vec{x}$ in $Q$.
\end{definition}

\begin{definition}
  The {\em structural congruence} \cite{SangiorgiWalker} , $\equiv$,
  between processes is the least congruence containing
  alpha-equivalence, satisfying the abelian monoid laws
  (associativity, commutativity and $\pzero$ as identity) for parallel
  composition $|$ and for summation $+$.
\end{definition}

\subsection{Name equivalence}

We take name equivalence, written $\nameeq$, to be the smallest
equivalence relation generated by the following rules.

\begin{mathpar}
\inferrule*[lab=Quote-drop]
{ }
{ \quotep{@{x}} \nameeq x }

\inferrule*[lab=Struct-equiv]
{ P \scong Q }
{ \quotep{P} \nameeq \quotep{Q} }
\end{mathpar}

The astute reader will have noticed that the mutual recursion of names
and processes imposes a mutual recursion on alpha-equivalence and
structural equivalence via name-equivalence. Fortunately, all of this
works out pleasantly and we may calculate in the natural way, free of
concern. The reader interested in the details is referred to the
appendix \ref{appendix:rho_details}.

\subsection{Substitution}

We use $\Proc$ for the set of processes, $\QProc$ for the set of
names, and $\id{\{}\vec{y} / \vec{x} \id{\}}$ to denote partial maps,
$s : \QProc \rightarrow \QProc$. A map, $s$ lifts, uniquely, to a map
on process terms, $\widehat{s} : \Proc \rightarrow \Proc$ by the
following equations.

\begin{mathpar}
  (0) \psubstp{Q}{P} := 0 \\
  (R \juxtap S) \psubstp{Q}{P}
  :=    
  (R)\psubstp{Q}{P} \juxtap (S) \psubstp{Q}{P} \\
  (x?(y).R) \psubstp{Q}{P}    
  :=    
  (x)\substp{Q}{P} (z)\concat( (R \psubstn{z}{y}) \psubstp{Q}{P} ) \\
  (\lift{x}{R}) \psubstp{Q}{P}  
  :=
  \lift{(x)\substp{Q}{P}}{ R \psubstp{Q}{P} } \\
%   (\dropn{x})  \psubstp{Q}{P}       
%   := 
%   \left\{ 
%     \begin{array}{ccc} 
%       \dropn{\quotep{Q}} & & x \nameeq \quotep{P} \\
%       \dropn{x} & & otherwise \\
%     \end{array}
%   \right. 
  (\dropn{x})  \psubstp{Q}{P}       
  := 
  \left\{ 
    \begin{array}{ccc} 
      Q & & x \nameeq \quotep{P} \\
      \dropn{x} & & otherwise \\
    \end{array}
  \right.
\end{mathpar}
 

where

\begin{eqnarray}
  (x)\id{\{} \lpquote Q \rpquote / \lpquote P \rpquote \id{\}}            = 
  \left\{ 
    \begin{array}{ccc}
      \lpquote Q \rpquote & & x \nameeq \lpquote P \rpquote \\
      x & & otherwise \\
    \end{array}
  \right. \nonumber
\end{eqnarray}

and $z$ is chosen distinct from $\quotep{P}$, $\quotep{Q}$, the free
names in $Q$, and all the names in $R$. Our $\alpha$-equivalence will
be built in the standard way from this substitution.

\begin{remark}\label{rem:no_self_referential_names}
  One consequence of these definitions is that $\forall P. \quotep{P}
  \not\in \freenames{P}$.
\end{remark}

\subsection{ Dynamic quote: an example }

Anticipating something of what's to come, consider applying the
substitution, $\widehat{\id{\{}u / z \id{\}}}$, to the following pair
of processes, $\lift{w}{y!(z)}$ and $w[ \lpquote y!(z) \rpquote ]$.

\begin{eqnarray}
	\lift{w}{y!(z)}\widehat{\id{\{}u / z \id{\}}}
		& = &
		\lift{w}{y!(u)} \nonumber\\
	w[ \lpquote y!(z) \rpquote ] \widehat{ \id{\{}u / z \id{\}} }
		& = &
		w[ \lpquote y!(z) \rpquote ] \nonumber
\end{eqnarray}

Because the body of the process between quotes is impervious to
substitution, we get radically different answers. In fact, by
examining the first process in an input context,
e.g. $x?(z).\lift{w}{y!(z)}$, we see that the process under the lift
operator may be shaped by prefixed inputs binding a name inside it. In
this sense, the lift operator will be seen as a way to dynamically
construct processes before reifying them as names.

Finally equipped with these standard features we can present the
dynamics of the calculus.

\subsubsection{Operational semantics} 

Finally, we introduce the computational dynamics. What marks these
algebras as distinct from other more traditionally studied algebraic
structures, e.g. vector spaces or polynomial rings, is the manner in
which dynamics is captured. In traditional structures, dynamics is typically
expressed through morphisms between such structures, as in linear maps
between vector spaces or morphisms between rings. In algebras
associated with the semantics of computation, the dynamics is
expressed as part of the algebraic structure itself, through a
reduction reduction relation typically denoted by $\red$. Below, we
give a recursive presentation of this relation for the calculus used
in the encoding.

$\red \subseteq \pi \times \pi$
$\red : \pi \to \mathcal{P}(\pi)$

\begin{mathpar}
  \inferrule* [lab=Comm] { \textsf{match}( x_{src}, x_{trgt} ) } { x_{trgt}?(y)P \; | \; x_{src}!\langle {Q} \rangle \red P\{\quotep{Q}/y}\} }
  \and \\
  \inferrule* [lab=Par] {{P} \red {P}'} {{{P} | {Q}} \red {{P}' | {Q}}}
  \and
  \inferrule* [lab=Equiv]{{{P} \scong {P}'} \andalso {{P}' \red {Q}'} \andalso {{Q}' \scong {Q}}}{{P} \red {Q}}
\end{mathpar}

\begin{eqnarray*}
  match_{\equiv} (\quotep{P},\quotep{Q}) & := & P \equiv Q \\
  match_{\dagger}(\quotep{P},\quotep{Q}) & := & \forall R. P|Q \red^{*} R => R \red^{*} 0 \\
  match_{K}(\quotep{P},\quotep{Q}) & := & K \mbox{ for some context } K
\end{eqnarray*}

$u?(x)P | u!\langle Q \rangle \red P\{\quotep{Q}/x\}$

%We write $\wred$ for $\red^*$, and $P\red$ if $\exists Q $ such that $ P \red Q$.
We write $P\red$ if $\exists Q $ such that $ P \red Q$ and $P\not\red$, otherwise.

\section{Replication}

As mentioned before, it is known that replication (and hence
recursion) can be implemented in a higher-order process algebra
\cite{SangiorgiWalker}. As our first example of calculation with the
machinery thus far presented we give the construction explicitly in
the {\rhoc}.

\begin{eqnarray}
	D_{x} & := & \prefix{x}{y}{(\binpar{\outputp{x}{y}}{@{y}})} \nonumber\\
	\bangp_{x}{P} & := & \binpar{{x}!\langle{\binpar{D_{x}}{P}}\rangle}{D_{x}} \nonumber
\end{eqnarray}

\begin{eqnarray}
	\bangp_{x}{P} & & \nonumber\\
	=
	& {x}!\langle{(\prefix{x}{y}{(\outputp{x}{y} | @{y})) | P}}\rangle 
	      | \prefix{x}{y}{(\outputp{x}{y} | @{y})} & \nonumber\\
	\red
	& (\outputp{x}{y} | @{y})\substn{\quotep{(\prefix{x}{y}{(@{y} | \outputp{x}{y})) | P}}}{y} & \nonumber\\
	=
	& \outputp{x}{\quotep{(\prefix{x}{y}{(\outputp{x}{y} | @{y})) | P}}}
	  | {(\prefix{x}{y}{(\outputp{x}{y} | @{y})) | P}} & \nonumber\\
	\red
	& \ldots & \nonumber\\
	\red^*
	& P | P | \ldots & \nonumber
\end{eqnarray}

Of course, this encoding, as an implementation, runs away, unfolding
$\bangp{P}$ eagerly. A lazier and more implementable replication
operator, restricted to input-guarded processes, may be obtained as follows.

\begin{eqnarray}
\bangp{\prefix{u}{v}{P}} 
	:= 
	\binpar{\lift{x}{\prefix{u}{v}{(\binpar{D(x)}{P})}}}{D(x)} \nonumber
\end{eqnarray}

\begin{remark}
  Note that the lazier definition still does not deal with summation
  or mixed summation (i.e. sums over input and output). The reader is
  invited to construct definitions of replication that deal with these
  features. 

  Further, the definitions are parameterized in a name, $x$. Can you,
  gentle reader, make a definition that eliminates this parameter and
  guarantees no accidental interaction between the replication
  machinery and the process being replicated -- i.e. no accidental
  sharing of names used by the process to get its work done and the
  name(s) used by the replication to effect copying. This latter
  revision of the definition of replication is crucial to obtaining
  the expected identity $!!P \sim !P$.
\end{remark}

\begin{remark}\label{rem:paradoxical_combinator}
  The reader familiar with the lambda calculus will have noticed the
  similarity between $D$ and the paradoxical combinator.

  [Ed. note: the existence of this seems to suggest we have to be more
  restrictive on the set of processes and names we admit if we are to
  support no-cloning.]
\end{remark}

\subsubsection{Bisimulation}

The computational dynamics gives rise to another kind of equivalence,
the equivalence of computational behavior. As previously mentioned
this is typically captured \emph{via} some form of bisimulation.

% The notion we use in this paper is weak barbed bisimulation
% \cite{milner91polyadicpi}.

The notion we use in this paper is derived from weak barbed
bisimulation \cite{milner91polyadicpi}. 

\begin{definition}
An \emph{observation relation}, $\downarrow_{\mathcal N}$, over a set
of names, $\mathcal N$, is the smallest relation satisfying the rules
below.

\infrule[Out-barb]{y \in {\mathcal N}, \; x \nameeq y}
		  {\outputp{x}{v} \downarrow_{\mathcal N} x}
\infrule[Par-barb]{\mbox{$P\downarrow_{\mathcal N} x$ or $Q\downarrow_{\mathcal N} x$}}
		  {\binpar{P}{Q} \downarrow_{\mathcal N} x}

We write $P \Downarrow_{\mathcal N} x$ if there is $Q$ such that 
$P \wred Q$ and $Q \downarrow_{\mathcal N} x$.
\end{definition}

\begin{definition}
%\label{def.bbisim}
An  ${\mathcal N}$-\emph{barbed bisimulation} over a set of names, ${\mathcal N}$, is a symmetric binary relation 
${\mathcal S}_{\mathcal N}$ between agents such that $P\rel{S}_{\mathcal N}Q$ implies:
\begin{enumerate}
\item If $P \red P'$ then $Q \wred Q'$ and $P'\rel{S}_{\mathcal N} Q'$.
\item If $P\downarrow_{\mathcal N} x$, then $Q\Downarrow_{\mathcal N} x$.
\end{enumerate}
$P$ is ${\mathcal N}$-barbed bisimilar to $Q$, written
$P \wbbisim_{\mathcal N} Q$, if $P \rel{S}_{\mathcal N} Q$ for some ${\mathcal N}$-barbed bisimulation ${\mathcal S}_{\mathcal N}$.
\end{definition}

$\mathcal{R} \subseteq \pi \times \pi$

$P \mathcal{R} Q => \forall P'. P \red P' \Rightarrow \exists Q'. Q \red Q', P' \mathcal{R} Q'$

$P \vdash x \Rightarrow Q \vdash x$

\begin{mathpar}
  \inferrule*[lab=Out-barb]{x \nameeq y}{{y}!\langle{Q}\rangle \vdash x}
  \and
  \inferrule*[lab=Par-barb]{\mbox{$P\vdash x$ or $Q\vdash x$}}{\binpar{P}{Q} \vdash x}
\end{mathpar}

\subsubsection{Contexts}

One of the principle advantages of computational calculi like the
$\pi$-calculus is a well-defined notion of context,
contextual-equivalence and a correlation between
contextual-equivalence and notions of bisimulation. The notion of
context allows the decomposition of a process into (sub-)process and
its syntactic environment, its context. Thus, a context may be
thought of as a process with a ``hole'' (written $\Box$) in it. The
application of a context $M$ to a process $P$, written $M[P]$, is
tantamount to filling the hole in $M$ with $P$. In this paper we do
not need the full weight of this theory, but do make use of the notion
of context in the proof the main theorem. 

\begin{mathpar}
  \inferrule* [lab=summation] {} {{M_{M},M_{N}} \bc \Box \;|\; x.M_{A} \;|\; M_{M}+M_{N}}
  \and
  \inferrule* [lab=agent] {} {{M_{A}} \bc (\vec{x})M_{P} \;| \; \clift{P_0,\ldots,M_{P},\ldots,P_N}}
  \and \\
  \inferrule* [lab=process] {} {{M_{P}} \bc M_{N} \;| \;P|M_{P} }
\end{mathpar} 

\begin{mathpar}
  \inferrule* [lab=sychronization] {} {M_{N} \bc \Box \;|\; x?M_{F} \;|\; x!M_{C}}
  \and
  \inferrule* [lab=abstraction] {} {{M_{F}} \bc (x)M_{P} }
  \and
  \inferrule* [lab=concretion] {} {{M_{C}} \bc \langle M_{P} \rangle }
  \and \\
  \inferrule* [lab=process] {} {{M_{P}} \bc M_{N} \;| \;P|M_{P} }
\end{mathpar}

\begin{definition}[contextual application] Given a context $M$, and
  process $P$, we define the \emph{contextual application}, $M[P] :=
  M\{P/\Box\}$. That is, the contextual application of M to P is the
  substitution of $P$ for $\Box$ in $M$.
\end{definition}

$\meaningof{-} : L \to \mathcal{P}(\pi)$

\begin{mathpar}
  \inferrule* [lab=collection] {} {\meaningof{true} = \pi, \and \meaningof{~E} = \pi \setminus \meaningof{E}, \and \meaningof{E_{1} \& E_{2}} = \meaningof{E_{1}} \cap \meaningof{E_{2}}}
\end{mathpar}

\begin{mathpar}
  \inferrule* [lab=structure] {} {\meaningof{0} = \{ P \in \pi | P \equiv 0 \}, \and \\ \meaningof{E_1 | E_2} = \{ P \in \pi | P \equiv P_{1} | P_{2}, P_{1} \in \meaningof{E_{1}}, P_{2} \in \meaningof{E_2}\} }
\end{mathpar}

\begin{mathpar}
 \inferrule* [lab=behavior] {} {\meaningof{\langle a?b \rangle E} = \{ P \in \pi | P \equiv Q | u?(y)P', \\ \and \\\\ \and \\ \;\;\; u \in \meaningof{a}, \forall z.P'\{z/y\} \in \meaningof{E\{z/b\}}\}, \and \\ \meaningof{a!E} = \{ P \in \pi | P \equiv Q | x!\langle P' \rangle, x \in \meaningof{a} P' \in \meaningof{E}\} }
\end{mathpar}

\begin{mathpar}
 \inferrule* [lab=nominal] {} {\meaningof{\quotep{E}} = \{ \quotep{P} \in \quotep{\pi} | P \in \meaningof{E} \}, \and \meaningof{\quotep{P}} = \{ \quotep{Q} \in \quotep{\pi} | P \equiv Q \} \and \\ \meaningof{@\quotep{E}} = \{ P \in \pi | P \equiv @x, x \in \meaningof{E} \}}
\end{mathpar}

\begin{eqnarray*}
  \\
  \meaningof{-} : TS \to ST
\end{eqnarray*}

\begin{eqnarray*}
  \\
  L : TS \to ST
\end{eqnarray*}

\begin{eqnarray*}
  \\
  P \models E \iff P \in \meaningof{E}
\end{eqnarray*}

\begin{eqnarray*}
  P \approx_{L} Q \iff \forall E \in L. P \models E \iff Q \models E
\end{eqnarray*}

\begin{eqnarray*}
  P \approx_{K} Q
\end{eqnarray*}

\begin{eqnarray*}
  P \approx Q
\end{eqnarray*}

$\approx_{K} = \approx = \approx_{L}$

\subsubsection{Contextual duality}

Note that contexts extend the quotation operation to a family of
operations from processes to names. Given a context, $M$, we can
define a \emph{nominal context}, $\quotep{M}$ by $\quotep{M}[P] :=
\quotep{M[P]}$. To foreshadow what is to come we observe that these
operations enjoy a duality with processes very much like the duality
between vectors and maps from vectors to scalars.

Further, because the calculus is essentially higher-order, we have a
correspondence between contexts and processes. More specifically,
given a name $x$ and a context $M$ we can construct $M^{*}_{x}$ such
that 

\begin{mathpar}
  M^{*}_{x} | \lift{x}{P} \red M[P]
\end{mathpar}

namely,

\begin{mathpar}
  M^{*}_{x} := x?(u).M[\dropn{u}]
\end{mathpar}

The dependence of $M^{*}_{x}$ on a name makes it an abstraction, 

\begin{mathpar}
  M^{*} := (x)x?(u).M[\dropn{u}]
\end{mathpar}

\subsection{Additional notation}

It will sometimes be convenient to denote the process a name
quotes. We already have the notation $x = \quotep{P}$, but it will be
convenient to introduce an alternate notation, $\procn{x}$, when we
want to emphasize the connection to the use of the name. Note that, by
virtue of name equivalence, $\quotep{\procn{x}} \nameeq x$; so, the
notation is consistent with previous definitions.

Further, because names have structure it is possible to effect
substitutions on the basis of that structure. This means we need to
upgrade our notation for substitutions, which we accomplish by
adapting comprehension notation. Thus,

\begin{mathpar}
  P\{ y / x : x \in S \}
\end{mathpar}

is interpreted to mean the process derived from P by replacing (in a
capture-avoiding manner) each occurrence of $x$ in $S$ by $y$. For example,

\begin{mathpar}
  P\{ \quotep{\procn{x}|\procn{x}} / x : x \in \freenames{P} \}
\end{mathpar}

will replace each (occurrence) of a free name $x$ in $P$ by
$\quotep{\procn{x}|\procn{x}}$.

Also, we will avail ourselves of the notation $x^{L}$ and $x^{R}$ to
denote injections of a name into disjoint copies of the name
space. There are numerous ways to accomplish this. One example can be
found in \cite{MeredithR05}. This notation overloads to vectors of
names: $\vec{x}^{\pi} := (x_{i}^{\pi} \; : \; 0 \leq i < |\vec{x}| )$ where $\pi \in \{L,R\}$.

We also use $P^{\Box} := P|\Box$.

In \cite{MeredithR05} an interpretation of the new operator is
given. It turns out that there are several possible interpretations
all enjoying the requisite algebraic properties of the operator (see
\cite{milner91polyadicpi}). We will therefore make liberal use of
$(\nu\; \vec{x})P$.

% subsection the_syntax_and_semantics_of_the_notation_system (end)   

\input{qm2pi.qmops} 

\input{qm2pi.sterngerlach} 

\input{qm2pi.metric} 

% section concurrent_process_calculi (end)

%\input{qm2pi.proofsketch}

% section proof sketch (end)

%\input{qm2pi.slviaknots} 

% section spatial logic via knots (end)

\input{qm2pi.conclusion}

% section conclusion (end)

%\input{qm2pi.dtcodes} 

% section wiring algorithm (end)

\input{qm2pi.ack} 

% section acknowledgments (end)

\newpage


\bibliographystyle{plain}   
\bibliography{../../biblios/main.bib}

\input{qm2pi.rhodetails}

\end{document}



\end{document}

 

% section notation (end)

\input{qm2pi.process.calculi} 

% section concurrent_process_calculi_and_spatial_logics_ (end)
    
%\documentclass[12pt]{llncs}
%\documentclass{jktr}

\usepackage[pdftex]{hyperref}                   
\usepackage {listings}
\usepackage {mathpartir}
\usepackage{bcprules}
%\usepackage{listings}
                       
\usepackage{graphicx} 
%\usepackage[margins=2.5cm,nohead,nofoot]{geometry}
%\usepackage{geometry}
\usepackage{amsfonts}
\usepackage{amstext}
\usepackage{latexsym}
\usepackage{amssymb}
\usepackage{color}


%\include{myPreamble}
\documentclass[12pt]{llncs}
%\documentclass{jktr}

\usepackage[pdftex]{hyperref}                   
\usepackage {listings}
\usepackage {mathpartir}
\usepackage{bcprules}
%\usepackage{listings}
                       
\usepackage{graphicx} 
%\usepackage[margins=2.5cm,nohead,nofoot]{geometry}
%\usepackage{geometry}
\usepackage{amsfonts}
\usepackage{amstext}
\usepackage{latexsym}
\usepackage{amssymb}
\usepackage{color}


%\include{myPreamble}
\include{qm2pi.local} 

%\ifpdf
%\usepackage[pdftex]{graphicx}
%\else
%\usepackage{graphicx}
%\fi

 % \ifpdf
%  \usepackage{pdfsync}
%  \if


%\title{Brief Article}
%\author{David F. Snyder}
%\author{L.G. Meredith}

%\address{Dept. of Math., Texas State University--San Marcos, San Marcos, TX 78666}
       
\pagestyle{empty}


\begin{document}

\lstset{language=[Objective]Caml,frame=shadowbox}

\input{qm2pi.front}

% section front matter (end)

\input{qm2pi.intro} 
 
% section introduction (end)

% \input{qm2pi.knotations} 

% section notation (end)

\input{qm2pi.process.calculi} 

% section concurrent_process_calculi_and_spatial_logics_ (end)
    
%\input{qm2pi.knots2pi} 

%\input{qm2pi.trefoil} 

%\input{qm2pi.mainthm} 

% subsection basic_interpretation (end)

%\input{qm2pi.rho.presentation} 
\subsection{The syntax and semantics of the notation system}\label{sub:the_syntax_and_semantics_of_the_notation_system} % (fold)

We now summarize a technical presentation of the calculus that
embodies our theory of dynamics. The typical presentation of such a
calculus follows the style of giving generators and relations on
them. The grammar, below, describing term constructors, freely
generates the set of processes, $\Proc$. This set is then quotiented
by a relation known as structural congruence and it is over this set
that the notion of dynamics is expressed. This presentation is
essentially that of \cite{MeredithR05} with the addition of
polyadicity and summation. For readability we have relegated some of
the technical subtleties to an appendix.

\subsubsection{Process grammar}\label{subsub:process_grammar}

\begin{mathpar}
  \inferrule* [lab=synchronization] {} {{M} \bc \pzero \;|\; x?F \;|\; x!C }
  \and
  \inferrule* [lab=abstraction] {} {{F} \bc (x)P}
  \and
  \inferrule* [lab=concretion] {} {{C} \bc \langle Q \rangle}
  \and
  \inferrule* [lab=process] {} {{P,Q} \bc M \;| \;P|Q \;|\; @{x}}
  \and
  \inferrule* [lab=name] {} {{x} \bc \quotep{P}}
\end{mathpar} 

Note that $\vec{x}$ (resp. $\vec{P}$) denotes a vector of names
(resp. processes) of length $|\vec{x}|$ (resp. $|\vec{P}|$). We adopt
the following useful abbreviations.

\begin{mathpar}
   x?(\vec{y}).P := x.(\vec{y})P \and  x\clift{\vec{P}} := x.\clift{\vec{P}}
   \and x!(y) := \lift{x}{\dropn{y}}
   \and \Pi_{i=0}^{n-1}P_i := P_0 | \ldots | P_{n-1}
\end{mathpar}

\subsubsection{Structural congruence}

\paragraph{Free and bound names and alpha-equivalence.} At the
core of structural equivalence is alpha-equivalence which identifies
process that are the same up to a change of variable. Formally, we
recognize the distinction between free and bound names. The free names
of a process, $\freenames{P}$, may be calculated recursively as
follows:

\begin{mathpar}
\freenames{\pzero} := \emptyset
  \and \\
  \freenames{x?(y).P} := \{ x \} \cup (\freenames{P} \setminus \{ y \})
  \and 
  \freenames{x!\langle P \rangle} := \{ x \} \cup \{ P \} 
  \and \\
  \freenames{P|Q} := \freenames{P} \cup \freenames{Q}
  \and \\
  \freenames{@{x}} := \{ x \}
\end{mathpar}

$\pi$
$\quotep{\pi}$

$\freenames{-} : \pi \to \mathcal{P}(\quotep{\pi})$

\begin{eqnarray*}
  \freenames{\pzero} & := & \emptyset \\
  \freenames{x?(y).P} & := & \{ x \} \cup (\freenames{P} \setminus \{ y \}) \\
  \freenames{x!\langle P \rangle} & := & \{ x \} \cup \{ P \} \\
  \freenames{P|Q} & := & \freenames{P} \cup \freenames{Q} \\
  \freenames{\dropn{x}} & := & \{ x \}
\end{eqnarray*}

The bound names of a process, $\boundnames{P}$, are those names occurring in $P$
that are not free. For example, in $x?(y).0$, the name $x$ is free, while $y$ is bound.

\begin{mathpar}
  \inferrule* [lab=monoidal-laws] {} { P|Q \equiv Q|P \and P|0 \equiv P \and P|(Q|R) \equiv (P|Q)|R }
\end{mathpar}

\begin{mathpar}
  \inferrule* [lab=alpha-equivalence] {} { (x)P \equiv (y)P\{y/x\} \and y \not\in \freenames{P} }
\end{mathpar}

\begin{definition}
Then two processes, $P,Q$, are alpha-equivalent if $P = Q\{\vec{y}/\vec{x}\}$ for
some $\vec{x} \in \boundnames{Q},\vec{y} \in \boundnames{P}$, where $Q\{\vec{y}/\vec{x}\}$
denotes the capture-avoiding substitution of $\vec{y}$ for $\vec{x}$ in $Q$.
\end{definition}

\begin{definition}
  The {\em structural congruence} \cite{SangiorgiWalker} , $\equiv$,
  between processes is the least congruence containing
  alpha-equivalence, satisfying the abelian monoid laws
  (associativity, commutativity and $\pzero$ as identity) for parallel
  composition $|$ and for summation $+$.
\end{definition}

\subsection{Name equivalence}

We take name equivalence, written $\nameeq$, to be the smallest
equivalence relation generated by the following rules.

\begin{mathpar}
\inferrule*[lab=Quote-drop]
{ }
{ \quotep{@{x}} \nameeq x }

\inferrule*[lab=Struct-equiv]
{ P \scong Q }
{ \quotep{P} \nameeq \quotep{Q} }
\end{mathpar}

The astute reader will have noticed that the mutual recursion of names
and processes imposes a mutual recursion on alpha-equivalence and
structural equivalence via name-equivalence. Fortunately, all of this
works out pleasantly and we may calculate in the natural way, free of
concern. The reader interested in the details is referred to the
appendix \ref{appendix:rho_details}.

\subsection{Substitution}

We use $\Proc$ for the set of processes, $\QProc$ for the set of
names, and $\id{\{}\vec{y} / \vec{x} \id{\}}$ to denote partial maps,
$s : \QProc \rightarrow \QProc$. A map, $s$ lifts, uniquely, to a map
on process terms, $\widehat{s} : \Proc \rightarrow \Proc$ by the
following equations.

\begin{mathpar}
  (0) \psubstp{Q}{P} := 0 \\
  (R \juxtap S) \psubstp{Q}{P}
  :=    
  (R)\psubstp{Q}{P} \juxtap (S) \psubstp{Q}{P} \\
  (x?(y).R) \psubstp{Q}{P}    
  :=    
  (x)\substp{Q}{P} (z)\concat( (R \psubstn{z}{y}) \psubstp{Q}{P} ) \\
  (\lift{x}{R}) \psubstp{Q}{P}  
  :=
  \lift{(x)\substp{Q}{P}}{ R \psubstp{Q}{P} } \\
%   (\dropn{x})  \psubstp{Q}{P}       
%   := 
%   \left\{ 
%     \begin{array}{ccc} 
%       \dropn{\quotep{Q}} & & x \nameeq \quotep{P} \\
%       \dropn{x} & & otherwise \\
%     \end{array}
%   \right. 
  (\dropn{x})  \psubstp{Q}{P}       
  := 
  \left\{ 
    \begin{array}{ccc} 
      Q & & x \nameeq \quotep{P} \\
      \dropn{x} & & otherwise \\
    \end{array}
  \right.
\end{mathpar}
 

where

\begin{eqnarray}
  (x)\id{\{} \lpquote Q \rpquote / \lpquote P \rpquote \id{\}}            = 
  \left\{ 
    \begin{array}{ccc}
      \lpquote Q \rpquote & & x \nameeq \lpquote P \rpquote \\
      x & & otherwise \\
    \end{array}
  \right. \nonumber
\end{eqnarray}

and $z$ is chosen distinct from $\quotep{P}$, $\quotep{Q}$, the free
names in $Q$, and all the names in $R$. Our $\alpha$-equivalence will
be built in the standard way from this substitution.

\begin{remark}\label{rem:no_self_referential_names}
  One consequence of these definitions is that $\forall P. \quotep{P}
  \not\in \freenames{P}$.
\end{remark}

\subsection{ Dynamic quote: an example }

Anticipating something of what's to come, consider applying the
substitution, $\widehat{\id{\{}u / z \id{\}}}$, to the following pair
of processes, $\lift{w}{y!(z)}$ and $w[ \lpquote y!(z) \rpquote ]$.

\begin{eqnarray}
	\lift{w}{y!(z)}\widehat{\id{\{}u / z \id{\}}}
		& = &
		\lift{w}{y!(u)} \nonumber\\
	w[ \lpquote y!(z) \rpquote ] \widehat{ \id{\{}u / z \id{\}} }
		& = &
		w[ \lpquote y!(z) \rpquote ] \nonumber
\end{eqnarray}

Because the body of the process between quotes is impervious to
substitution, we get radically different answers. In fact, by
examining the first process in an input context,
e.g. $x?(z).\lift{w}{y!(z)}$, we see that the process under the lift
operator may be shaped by prefixed inputs binding a name inside it. In
this sense, the lift operator will be seen as a way to dynamically
construct processes before reifying them as names.

Finally equipped with these standard features we can present the
dynamics of the calculus.

\subsubsection{Operational semantics} 

Finally, we introduce the computational dynamics. What marks these
algebras as distinct from other more traditionally studied algebraic
structures, e.g. vector spaces or polynomial rings, is the manner in
which dynamics is captured. In traditional structures, dynamics is typically
expressed through morphisms between such structures, as in linear maps
between vector spaces or morphisms between rings. In algebras
associated with the semantics of computation, the dynamics is
expressed as part of the algebraic structure itself, through a
reduction reduction relation typically denoted by $\red$. Below, we
give a recursive presentation of this relation for the calculus used
in the encoding.

$\red \subseteq \pi \times \pi$
$\red : \pi \to \mathcal{P}(\pi)$

\begin{mathpar}
  \inferrule* [lab=Comm] { \textsf{match}( x_{src}, x_{trgt} ) } { x_{trgt}?(y)P \; | \; x_{src}!\langle {Q} \rangle \red P\{\quotep{Q}/y}\} }
  \and \\
  \inferrule* [lab=Par] {{P} \red {P}'} {{{P} | {Q}} \red {{P}' | {Q}}}
  \and
  \inferrule* [lab=Equiv]{{{P} \scong {P}'} \andalso {{P}' \red {Q}'} \andalso {{Q}' \scong {Q}}}{{P} \red {Q}}
\end{mathpar}

\begin{eqnarray*}
  match_{\equiv} (\quotep{P},\quotep{Q}) & := & P \equiv Q \\
  match_{\dagger}(\quotep{P},\quotep{Q}) & := & \forall R. P|Q \red^{*} R => R \red^{*} 0 \\
  match_{K}(\quotep{P},\quotep{Q}) & := & K \mbox{ for some context } K
\end{eqnarray*}

$u?(x)P | u!\langle Q \rangle \red P\{\quotep{Q}/x\}$

%We write $\wred$ for $\red^*$, and $P\red$ if $\exists Q $ such that $ P \red Q$.
We write $P\red$ if $\exists Q $ such that $ P \red Q$ and $P\not\red$, otherwise.

\section{Replication}

As mentioned before, it is known that replication (and hence
recursion) can be implemented in a higher-order process algebra
\cite{SangiorgiWalker}. As our first example of calculation with the
machinery thus far presented we give the construction explicitly in
the {\rhoc}.

\begin{eqnarray}
	D_{x} & := & \prefix{x}{y}{(\binpar{\outputp{x}{y}}{@{y}})} \nonumber\\
	\bangp_{x}{P} & := & \binpar{{x}!\langle{\binpar{D_{x}}{P}}\rangle}{D_{x}} \nonumber
\end{eqnarray}

\begin{eqnarray}
	\bangp_{x}{P} & & \nonumber\\
	=
	& {x}!\langle{(\prefix{x}{y}{(\outputp{x}{y} | @{y})) | P}}\rangle 
	      | \prefix{x}{y}{(\outputp{x}{y} | @{y})} & \nonumber\\
	\red
	& (\outputp{x}{y} | @{y})\substn{\quotep{(\prefix{x}{y}{(@{y} | \outputp{x}{y})) | P}}}{y} & \nonumber\\
	=
	& \outputp{x}{\quotep{(\prefix{x}{y}{(\outputp{x}{y} | @{y})) | P}}}
	  | {(\prefix{x}{y}{(\outputp{x}{y} | @{y})) | P}} & \nonumber\\
	\red
	& \ldots & \nonumber\\
	\red^*
	& P | P | \ldots & \nonumber
\end{eqnarray}

Of course, this encoding, as an implementation, runs away, unfolding
$\bangp{P}$ eagerly. A lazier and more implementable replication
operator, restricted to input-guarded processes, may be obtained as follows.

\begin{eqnarray}
\bangp{\prefix{u}{v}{P}} 
	:= 
	\binpar{\lift{x}{\prefix{u}{v}{(\binpar{D(x)}{P})}}}{D(x)} \nonumber
\end{eqnarray}

\begin{remark}
  Note that the lazier definition still does not deal with summation
  or mixed summation (i.e. sums over input and output). The reader is
  invited to construct definitions of replication that deal with these
  features. 

  Further, the definitions are parameterized in a name, $x$. Can you,
  gentle reader, make a definition that eliminates this parameter and
  guarantees no accidental interaction between the replication
  machinery and the process being replicated -- i.e. no accidental
  sharing of names used by the process to get its work done and the
  name(s) used by the replication to effect copying. This latter
  revision of the definition of replication is crucial to obtaining
  the expected identity $!!P \sim !P$.
\end{remark}

\begin{remark}\label{rem:paradoxical_combinator}
  The reader familiar with the lambda calculus will have noticed the
  similarity between $D$ and the paradoxical combinator.

  [Ed. note: the existence of this seems to suggest we have to be more
  restrictive on the set of processes and names we admit if we are to
  support no-cloning.]
\end{remark}

\subsubsection{Bisimulation}

The computational dynamics gives rise to another kind of equivalence,
the equivalence of computational behavior. As previously mentioned
this is typically captured \emph{via} some form of bisimulation.

% The notion we use in this paper is weak barbed bisimulation
% \cite{milner91polyadicpi}.

The notion we use in this paper is derived from weak barbed
bisimulation \cite{milner91polyadicpi}. 

\begin{definition}
An \emph{observation relation}, $\downarrow_{\mathcal N}$, over a set
of names, $\mathcal N$, is the smallest relation satisfying the rules
below.

\infrule[Out-barb]{y \in {\mathcal N}, \; x \nameeq y}
		  {\outputp{x}{v} \downarrow_{\mathcal N} x}
\infrule[Par-barb]{\mbox{$P\downarrow_{\mathcal N} x$ or $Q\downarrow_{\mathcal N} x$}}
		  {\binpar{P}{Q} \downarrow_{\mathcal N} x}

We write $P \Downarrow_{\mathcal N} x$ if there is $Q$ such that 
$P \wred Q$ and $Q \downarrow_{\mathcal N} x$.
\end{definition}

\begin{definition}
%\label{def.bbisim}
An  ${\mathcal N}$-\emph{barbed bisimulation} over a set of names, ${\mathcal N}$, is a symmetric binary relation 
${\mathcal S}_{\mathcal N}$ between agents such that $P\rel{S}_{\mathcal N}Q$ implies:
\begin{enumerate}
\item If $P \red P'$ then $Q \wred Q'$ and $P'\rel{S}_{\mathcal N} Q'$.
\item If $P\downarrow_{\mathcal N} x$, then $Q\Downarrow_{\mathcal N} x$.
\end{enumerate}
$P$ is ${\mathcal N}$-barbed bisimilar to $Q$, written
$P \wbbisim_{\mathcal N} Q$, if $P \rel{S}_{\mathcal N} Q$ for some ${\mathcal N}$-barbed bisimulation ${\mathcal S}_{\mathcal N}$.
\end{definition}

$\mathcal{R} \subseteq \pi \times \pi$

$P \mathcal{R} Q => \forall P'. P \red P' \Rightarrow \exists Q'. Q \red Q', P' \mathcal{R} Q'$

$P \vdash x \Rightarrow Q \vdash x$

\begin{mathpar}
  \inferrule*[lab=Out-barb]{x \nameeq y}{{y}!\langle{Q}\rangle \vdash x}
  \and
  \inferrule*[lab=Par-barb]{\mbox{$P\vdash x$ or $Q\vdash x$}}{\binpar{P}{Q} \vdash x}
\end{mathpar}

\subsubsection{Contexts}

One of the principle advantages of computational calculi like the
$\pi$-calculus is a well-defined notion of context,
contextual-equivalence and a correlation between
contextual-equivalence and notions of bisimulation. The notion of
context allows the decomposition of a process into (sub-)process and
its syntactic environment, its context. Thus, a context may be
thought of as a process with a ``hole'' (written $\Box$) in it. The
application of a context $M$ to a process $P$, written $M[P]$, is
tantamount to filling the hole in $M$ with $P$. In this paper we do
not need the full weight of this theory, but do make use of the notion
of context in the proof the main theorem. 

\begin{mathpar}
  \inferrule* [lab=summation] {} {{M_{M},M_{N}} \bc \Box \;|\; x.M_{A} \;|\; M_{M}+M_{N}}
  \and
  \inferrule* [lab=agent] {} {{M_{A}} \bc (\vec{x})M_{P} \;| \; \clift{P_0,\ldots,M_{P},\ldots,P_N}}
  \and \\
  \inferrule* [lab=process] {} {{M_{P}} \bc M_{N} \;| \;P|M_{P} }
\end{mathpar} 

\begin{mathpar}
  \inferrule* [lab=sychronization] {} {M_{N} \bc \Box \;|\; x?M_{F} \;|\; x!M_{C}}
  \and
  \inferrule* [lab=abstraction] {} {{M_{F}} \bc (x)M_{P} }
  \and
  \inferrule* [lab=concretion] {} {{M_{C}} \bc \langle M_{P} \rangle }
  \and \\
  \inferrule* [lab=process] {} {{M_{P}} \bc M_{N} \;| \;P|M_{P} }
\end{mathpar}

\begin{definition}[contextual application] Given a context $M$, and
  process $P$, we define the \emph{contextual application}, $M[P] :=
  M\{P/\Box\}$. That is, the contextual application of M to P is the
  substitution of $P$ for $\Box$ in $M$.
\end{definition}

$\meaningof{-} : L \to \mathcal{P}(\pi)$

\begin{mathpar}
  \inferrule* [lab=collection] {} {\meaningof{true} = \pi, \and \meaningof{~E} = \pi \setminus \meaningof{E}, \and \meaningof{E_{1} \& E_{2}} = \meaningof{E_{1}} \cap \meaningof{E_{2}}}
\end{mathpar}

\begin{mathpar}
  \inferrule* [lab=structure] {} {\meaningof{0} = \{ P \in \pi | P \equiv 0 \}, \and \\ \meaningof{E_1 | E_2} = \{ P \in \pi | P \equiv P_{1} | P_{2}, P_{1} \in \meaningof{E_{1}}, P_{2} \in \meaningof{E_2}\} }
\end{mathpar}

\begin{mathpar}
 \inferrule* [lab=behavior] {} {\meaningof{\langle a?b \rangle E} = \{ P \in \pi | P \equiv Q | u?(y)P', \\ \and \\\\ \and \\ \;\;\; u \in \meaningof{a}, \forall z.P'\{z/y\} \in \meaningof{E\{z/b\}}\}, \and \\ \meaningof{a!E} = \{ P \in \pi | P \equiv Q | x!\langle P' \rangle, x \in \meaningof{a} P' \in \meaningof{E}\} }
\end{mathpar}

\begin{mathpar}
 \inferrule* [lab=nominal] {} {\meaningof{\quotep{E}} = \{ \quotep{P} \in \quotep{\pi} | P \in \meaningof{E} \}, \and \meaningof{\quotep{P}} = \{ \quotep{Q} \in \quotep{\pi} | P \equiv Q \} \and \\ \meaningof{@\quotep{E}} = \{ P \in \pi | P \equiv @x, x \in \meaningof{E} \}}
\end{mathpar}

\begin{eqnarray*}
  \\
  \meaningof{-} : TS \to ST
\end{eqnarray*}

\begin{eqnarray*}
  \\
  L : TS \to ST
\end{eqnarray*}

\begin{eqnarray*}
  \\
  P \models E \iff P \in \meaningof{E}
\end{eqnarray*}

\begin{eqnarray*}
  P \approx_{L} Q \iff \forall E \in L. P \models E \iff Q \models E
\end{eqnarray*}

\begin{eqnarray*}
  P \approx_{K} Q
\end{eqnarray*}

\begin{eqnarray*}
  P \approx Q
\end{eqnarray*}

$\approx_{K} = \approx = \approx_{L}$

\subsubsection{Contextual duality}

Note that contexts extend the quotation operation to a family of
operations from processes to names. Given a context, $M$, we can
define a \emph{nominal context}, $\quotep{M}$ by $\quotep{M}[P] :=
\quotep{M[P]}$. To foreshadow what is to come we observe that these
operations enjoy a duality with processes very much like the duality
between vectors and maps from vectors to scalars.

Further, because the calculus is essentially higher-order, we have a
correspondence between contexts and processes. More specifically,
given a name $x$ and a context $M$ we can construct $M^{*}_{x}$ such
that 

\begin{mathpar}
  M^{*}_{x} | \lift{x}{P} \red M[P]
\end{mathpar}

namely,

\begin{mathpar}
  M^{*}_{x} := x?(u).M[\dropn{u}]
\end{mathpar}

The dependence of $M^{*}_{x}$ on a name makes it an abstraction, 

\begin{mathpar}
  M^{*} := (x)x?(u).M[\dropn{u}]
\end{mathpar}

\subsection{Additional notation}

It will sometimes be convenient to denote the process a name
quotes. We already have the notation $x = \quotep{P}$, but it will be
convenient to introduce an alternate notation, $\procn{x}$, when we
want to emphasize the connection to the use of the name. Note that, by
virtue of name equivalence, $\quotep{\procn{x}} \nameeq x$; so, the
notation is consistent with previous definitions.

Further, because names have structure it is possible to effect
substitutions on the basis of that structure. This means we need to
upgrade our notation for substitutions, which we accomplish by
adapting comprehension notation. Thus,

\begin{mathpar}
  P\{ y / x : x \in S \}
\end{mathpar}

is interpreted to mean the process derived from P by replacing (in a
capture-avoiding manner) each occurrence of $x$ in $S$ by $y$. For example,

\begin{mathpar}
  P\{ \quotep{\procn{x}|\procn{x}} / x : x \in \freenames{P} \}
\end{mathpar}

will replace each (occurrence) of a free name $x$ in $P$ by
$\quotep{\procn{x}|\procn{x}}$.

Also, we will avail ourselves of the notation $x^{L}$ and $x^{R}$ to
denote injections of a name into disjoint copies of the name
space. There are numerous ways to accomplish this. One example can be
found in \cite{MeredithR05}. This notation overloads to vectors of
names: $\vec{x}^{\pi} := (x_{i}^{\pi} \; : \; 0 \leq i < |\vec{x}| )$ where $\pi \in \{L,R\}$.

We also use $P^{\Box} := P|\Box$.

In \cite{MeredithR05} an interpretation of the new operator is
given. It turns out that there are several possible interpretations
all enjoying the requisite algebraic properties of the operator (see
\cite{milner91polyadicpi}). We will therefore make liberal use of
$(\nu\; \vec{x})P$.

% subsection the_syntax_and_semantics_of_the_notation_system (end)   

\input{qm2pi.qmops} 

\input{qm2pi.sterngerlach} 

\input{qm2pi.metric} 

% section concurrent_process_calculi (end)

%\input{qm2pi.proofsketch}

% section proof sketch (end)

%\input{qm2pi.slviaknots} 

% section spatial logic via knots (end)

\input{qm2pi.conclusion}

% section conclusion (end)

%\input{qm2pi.dtcodes} 

% section wiring algorithm (end)

\input{qm2pi.ack} 

% section acknowledgments (end)

\newpage


\bibliographystyle{plain}   
\bibliography{../../biblios/main.bib}

\input{qm2pi.rhodetails}

\end{document}

 

%\ifpdf
%\usepackage[pdftex]{graphicx}
%\else
%\usepackage{graphicx}
%\fi

 % \ifpdf
%  \usepackage{pdfsync}
%  \if


%\title{Brief Article}
%\author{David F. Snyder}
%\author{L.G. Meredith}

%\address{Dept. of Math., Texas State University--San Marcos, San Marcos, TX 78666}
       
\pagestyle{empty}


\begin{document}

\lstset{language=[Objective]Caml,frame=shadowbox}

\documentclass[12pt]{llncs}
%\documentclass{jktr}

\usepackage[pdftex]{hyperref}                   
\usepackage {listings}
\usepackage {mathpartir}
\usepackage{bcprules}
%\usepackage{listings}
                       
\usepackage{graphicx} 
%\usepackage[margins=2.5cm,nohead,nofoot]{geometry}
%\usepackage{geometry}
\usepackage{amsfonts}
\usepackage{amstext}
\usepackage{latexsym}
\usepackage{amssymb}
\usepackage{color}


%\include{myPreamble}
\include{qm2pi.local} 

%\ifpdf
%\usepackage[pdftex]{graphicx}
%\else
%\usepackage{graphicx}
%\fi

 % \ifpdf
%  \usepackage{pdfsync}
%  \if


%\title{Brief Article}
%\author{David F. Snyder}
%\author{L.G. Meredith}

%\address{Dept. of Math., Texas State University--San Marcos, San Marcos, TX 78666}
       
\pagestyle{empty}


\begin{document}

\lstset{language=[Objective]Caml,frame=shadowbox}

\input{qm2pi.front}

% section front matter (end)

\input{qm2pi.intro} 
 
% section introduction (end)

% \input{qm2pi.knotations} 

% section notation (end)

\input{qm2pi.process.calculi} 

% section concurrent_process_calculi_and_spatial_logics_ (end)
    
%\input{qm2pi.knots2pi} 

%\input{qm2pi.trefoil} 

%\input{qm2pi.mainthm} 

% subsection basic_interpretation (end)

%\input{qm2pi.rho.presentation} 
\subsection{The syntax and semantics of the notation system}\label{sub:the_syntax_and_semantics_of_the_notation_system} % (fold)

We now summarize a technical presentation of the calculus that
embodies our theory of dynamics. The typical presentation of such a
calculus follows the style of giving generators and relations on
them. The grammar, below, describing term constructors, freely
generates the set of processes, $\Proc$. This set is then quotiented
by a relation known as structural congruence and it is over this set
that the notion of dynamics is expressed. This presentation is
essentially that of \cite{MeredithR05} with the addition of
polyadicity and summation. For readability we have relegated some of
the technical subtleties to an appendix.

\subsubsection{Process grammar}\label{subsub:process_grammar}

\begin{mathpar}
  \inferrule* [lab=synchronization] {} {{M} \bc \pzero \;|\; x?F \;|\; x!C }
  \and
  \inferrule* [lab=abstraction] {} {{F} \bc (x)P}
  \and
  \inferrule* [lab=concretion] {} {{C} \bc \langle Q \rangle}
  \and
  \inferrule* [lab=process] {} {{P,Q} \bc M \;| \;P|Q \;|\; @{x}}
  \and
  \inferrule* [lab=name] {} {{x} \bc \quotep{P}}
\end{mathpar} 

Note that $\vec{x}$ (resp. $\vec{P}$) denotes a vector of names
(resp. processes) of length $|\vec{x}|$ (resp. $|\vec{P}|$). We adopt
the following useful abbreviations.

\begin{mathpar}
   x?(\vec{y}).P := x.(\vec{y})P \and  x\clift{\vec{P}} := x.\clift{\vec{P}}
   \and x!(y) := \lift{x}{\dropn{y}}
   \and \Pi_{i=0}^{n-1}P_i := P_0 | \ldots | P_{n-1}
\end{mathpar}

\subsubsection{Structural congruence}

\paragraph{Free and bound names and alpha-equivalence.} At the
core of structural equivalence is alpha-equivalence which identifies
process that are the same up to a change of variable. Formally, we
recognize the distinction between free and bound names. The free names
of a process, $\freenames{P}$, may be calculated recursively as
follows:

\begin{mathpar}
\freenames{\pzero} := \emptyset
  \and \\
  \freenames{x?(y).P} := \{ x \} \cup (\freenames{P} \setminus \{ y \})
  \and 
  \freenames{x!\langle P \rangle} := \{ x \} \cup \{ P \} 
  \and \\
  \freenames{P|Q} := \freenames{P} \cup \freenames{Q}
  \and \\
  \freenames{@{x}} := \{ x \}
\end{mathpar}

$\pi$
$\quotep{\pi}$

$\freenames{-} : \pi \to \mathcal{P}(\quotep{\pi})$

\begin{eqnarray*}
  \freenames{\pzero} & := & \emptyset \\
  \freenames{x?(y).P} & := & \{ x \} \cup (\freenames{P} \setminus \{ y \}) \\
  \freenames{x!\langle P \rangle} & := & \{ x \} \cup \{ P \} \\
  \freenames{P|Q} & := & \freenames{P} \cup \freenames{Q} \\
  \freenames{\dropn{x}} & := & \{ x \}
\end{eqnarray*}

The bound names of a process, $\boundnames{P}$, are those names occurring in $P$
that are not free. For example, in $x?(y).0$, the name $x$ is free, while $y$ is bound.

\begin{mathpar}
  \inferrule* [lab=monoidal-laws] {} { P|Q \equiv Q|P \and P|0 \equiv P \and P|(Q|R) \equiv (P|Q)|R }
\end{mathpar}

\begin{mathpar}
  \inferrule* [lab=alpha-equivalence] {} { (x)P \equiv (y)P\{y/x\} \and y \not\in \freenames{P} }
\end{mathpar}

\begin{definition}
Then two processes, $P,Q$, are alpha-equivalent if $P = Q\{\vec{y}/\vec{x}\}$ for
some $\vec{x} \in \boundnames{Q},\vec{y} \in \boundnames{P}$, where $Q\{\vec{y}/\vec{x}\}$
denotes the capture-avoiding substitution of $\vec{y}$ for $\vec{x}$ in $Q$.
\end{definition}

\begin{definition}
  The {\em structural congruence} \cite{SangiorgiWalker} , $\equiv$,
  between processes is the least congruence containing
  alpha-equivalence, satisfying the abelian monoid laws
  (associativity, commutativity and $\pzero$ as identity) for parallel
  composition $|$ and for summation $+$.
\end{definition}

\subsection{Name equivalence}

We take name equivalence, written $\nameeq$, to be the smallest
equivalence relation generated by the following rules.

\begin{mathpar}
\inferrule*[lab=Quote-drop]
{ }
{ \quotep{@{x}} \nameeq x }

\inferrule*[lab=Struct-equiv]
{ P \scong Q }
{ \quotep{P} \nameeq \quotep{Q} }
\end{mathpar}

The astute reader will have noticed that the mutual recursion of names
and processes imposes a mutual recursion on alpha-equivalence and
structural equivalence via name-equivalence. Fortunately, all of this
works out pleasantly and we may calculate in the natural way, free of
concern. The reader interested in the details is referred to the
appendix \ref{appendix:rho_details}.

\subsection{Substitution}

We use $\Proc$ for the set of processes, $\QProc$ for the set of
names, and $\id{\{}\vec{y} / \vec{x} \id{\}}$ to denote partial maps,
$s : \QProc \rightarrow \QProc$. A map, $s$ lifts, uniquely, to a map
on process terms, $\widehat{s} : \Proc \rightarrow \Proc$ by the
following equations.

\begin{mathpar}
  (0) \psubstp{Q}{P} := 0 \\
  (R \juxtap S) \psubstp{Q}{P}
  :=    
  (R)\psubstp{Q}{P} \juxtap (S) \psubstp{Q}{P} \\
  (x?(y).R) \psubstp{Q}{P}    
  :=    
  (x)\substp{Q}{P} (z)\concat( (R \psubstn{z}{y}) \psubstp{Q}{P} ) \\
  (\lift{x}{R}) \psubstp{Q}{P}  
  :=
  \lift{(x)\substp{Q}{P}}{ R \psubstp{Q}{P} } \\
%   (\dropn{x})  \psubstp{Q}{P}       
%   := 
%   \left\{ 
%     \begin{array}{ccc} 
%       \dropn{\quotep{Q}} & & x \nameeq \quotep{P} \\
%       \dropn{x} & & otherwise \\
%     \end{array}
%   \right. 
  (\dropn{x})  \psubstp{Q}{P}       
  := 
  \left\{ 
    \begin{array}{ccc} 
      Q & & x \nameeq \quotep{P} \\
      \dropn{x} & & otherwise \\
    \end{array}
  \right.
\end{mathpar}
 

where

\begin{eqnarray}
  (x)\id{\{} \lpquote Q \rpquote / \lpquote P \rpquote \id{\}}            = 
  \left\{ 
    \begin{array}{ccc}
      \lpquote Q \rpquote & & x \nameeq \lpquote P \rpquote \\
      x & & otherwise \\
    \end{array}
  \right. \nonumber
\end{eqnarray}

and $z$ is chosen distinct from $\quotep{P}$, $\quotep{Q}$, the free
names in $Q$, and all the names in $R$. Our $\alpha$-equivalence will
be built in the standard way from this substitution.

\begin{remark}\label{rem:no_self_referential_names}
  One consequence of these definitions is that $\forall P. \quotep{P}
  \not\in \freenames{P}$.
\end{remark}

\subsection{ Dynamic quote: an example }

Anticipating something of what's to come, consider applying the
substitution, $\widehat{\id{\{}u / z \id{\}}}$, to the following pair
of processes, $\lift{w}{y!(z)}$ and $w[ \lpquote y!(z) \rpquote ]$.

\begin{eqnarray}
	\lift{w}{y!(z)}\widehat{\id{\{}u / z \id{\}}}
		& = &
		\lift{w}{y!(u)} \nonumber\\
	w[ \lpquote y!(z) \rpquote ] \widehat{ \id{\{}u / z \id{\}} }
		& = &
		w[ \lpquote y!(z) \rpquote ] \nonumber
\end{eqnarray}

Because the body of the process between quotes is impervious to
substitution, we get radically different answers. In fact, by
examining the first process in an input context,
e.g. $x?(z).\lift{w}{y!(z)}$, we see that the process under the lift
operator may be shaped by prefixed inputs binding a name inside it. In
this sense, the lift operator will be seen as a way to dynamically
construct processes before reifying them as names.

Finally equipped with these standard features we can present the
dynamics of the calculus.

\subsubsection{Operational semantics} 

Finally, we introduce the computational dynamics. What marks these
algebras as distinct from other more traditionally studied algebraic
structures, e.g. vector spaces or polynomial rings, is the manner in
which dynamics is captured. In traditional structures, dynamics is typically
expressed through morphisms between such structures, as in linear maps
between vector spaces or morphisms between rings. In algebras
associated with the semantics of computation, the dynamics is
expressed as part of the algebraic structure itself, through a
reduction reduction relation typically denoted by $\red$. Below, we
give a recursive presentation of this relation for the calculus used
in the encoding.

$\red \subseteq \pi \times \pi$
$\red : \pi \to \mathcal{P}(\pi)$

\begin{mathpar}
  \inferrule* [lab=Comm] { \textsf{match}( x_{src}, x_{trgt} ) } { x_{trgt}?(y)P \; | \; x_{src}!\langle {Q} \rangle \red P\{\quotep{Q}/y}\} }
  \and \\
  \inferrule* [lab=Par] {{P} \red {P}'} {{{P} | {Q}} \red {{P}' | {Q}}}
  \and
  \inferrule* [lab=Equiv]{{{P} \scong {P}'} \andalso {{P}' \red {Q}'} \andalso {{Q}' \scong {Q}}}{{P} \red {Q}}
\end{mathpar}

\begin{eqnarray*}
  match_{\equiv} (\quotep{P},\quotep{Q}) & := & P \equiv Q \\
  match_{\dagger}(\quotep{P},\quotep{Q}) & := & \forall R. P|Q \red^{*} R => R \red^{*} 0 \\
  match_{K}(\quotep{P},\quotep{Q}) & := & K \mbox{ for some context } K
\end{eqnarray*}

$u?(x)P | u!\langle Q \rangle \red P\{\quotep{Q}/x\}$

%We write $\wred$ for $\red^*$, and $P\red$ if $\exists Q $ such that $ P \red Q$.
We write $P\red$ if $\exists Q $ such that $ P \red Q$ and $P\not\red$, otherwise.

\section{Replication}

As mentioned before, it is known that replication (and hence
recursion) can be implemented in a higher-order process algebra
\cite{SangiorgiWalker}. As our first example of calculation with the
machinery thus far presented we give the construction explicitly in
the {\rhoc}.

\begin{eqnarray}
	D_{x} & := & \prefix{x}{y}{(\binpar{\outputp{x}{y}}{@{y}})} \nonumber\\
	\bangp_{x}{P} & := & \binpar{{x}!\langle{\binpar{D_{x}}{P}}\rangle}{D_{x}} \nonumber
\end{eqnarray}

\begin{eqnarray}
	\bangp_{x}{P} & & \nonumber\\
	=
	& {x}!\langle{(\prefix{x}{y}{(\outputp{x}{y} | @{y})) | P}}\rangle 
	      | \prefix{x}{y}{(\outputp{x}{y} | @{y})} & \nonumber\\
	\red
	& (\outputp{x}{y} | @{y})\substn{\quotep{(\prefix{x}{y}{(@{y} | \outputp{x}{y})) | P}}}{y} & \nonumber\\
	=
	& \outputp{x}{\quotep{(\prefix{x}{y}{(\outputp{x}{y} | @{y})) | P}}}
	  | {(\prefix{x}{y}{(\outputp{x}{y} | @{y})) | P}} & \nonumber\\
	\red
	& \ldots & \nonumber\\
	\red^*
	& P | P | \ldots & \nonumber
\end{eqnarray}

Of course, this encoding, as an implementation, runs away, unfolding
$\bangp{P}$ eagerly. A lazier and more implementable replication
operator, restricted to input-guarded processes, may be obtained as follows.

\begin{eqnarray}
\bangp{\prefix{u}{v}{P}} 
	:= 
	\binpar{\lift{x}{\prefix{u}{v}{(\binpar{D(x)}{P})}}}{D(x)} \nonumber
\end{eqnarray}

\begin{remark}
  Note that the lazier definition still does not deal with summation
  or mixed summation (i.e. sums over input and output). The reader is
  invited to construct definitions of replication that deal with these
  features. 

  Further, the definitions are parameterized in a name, $x$. Can you,
  gentle reader, make a definition that eliminates this parameter and
  guarantees no accidental interaction between the replication
  machinery and the process being replicated -- i.e. no accidental
  sharing of names used by the process to get its work done and the
  name(s) used by the replication to effect copying. This latter
  revision of the definition of replication is crucial to obtaining
  the expected identity $!!P \sim !P$.
\end{remark}

\begin{remark}\label{rem:paradoxical_combinator}
  The reader familiar with the lambda calculus will have noticed the
  similarity between $D$ and the paradoxical combinator.

  [Ed. note: the existence of this seems to suggest we have to be more
  restrictive on the set of processes and names we admit if we are to
  support no-cloning.]
\end{remark}

\subsubsection{Bisimulation}

The computational dynamics gives rise to another kind of equivalence,
the equivalence of computational behavior. As previously mentioned
this is typically captured \emph{via} some form of bisimulation.

% The notion we use in this paper is weak barbed bisimulation
% \cite{milner91polyadicpi}.

The notion we use in this paper is derived from weak barbed
bisimulation \cite{milner91polyadicpi}. 

\begin{definition}
An \emph{observation relation}, $\downarrow_{\mathcal N}$, over a set
of names, $\mathcal N$, is the smallest relation satisfying the rules
below.

\infrule[Out-barb]{y \in {\mathcal N}, \; x \nameeq y}
		  {\outputp{x}{v} \downarrow_{\mathcal N} x}
\infrule[Par-barb]{\mbox{$P\downarrow_{\mathcal N} x$ or $Q\downarrow_{\mathcal N} x$}}
		  {\binpar{P}{Q} \downarrow_{\mathcal N} x}

We write $P \Downarrow_{\mathcal N} x$ if there is $Q$ such that 
$P \wred Q$ and $Q \downarrow_{\mathcal N} x$.
\end{definition}

\begin{definition}
%\label{def.bbisim}
An  ${\mathcal N}$-\emph{barbed bisimulation} over a set of names, ${\mathcal N}$, is a symmetric binary relation 
${\mathcal S}_{\mathcal N}$ between agents such that $P\rel{S}_{\mathcal N}Q$ implies:
\begin{enumerate}
\item If $P \red P'$ then $Q \wred Q'$ and $P'\rel{S}_{\mathcal N} Q'$.
\item If $P\downarrow_{\mathcal N} x$, then $Q\Downarrow_{\mathcal N} x$.
\end{enumerate}
$P$ is ${\mathcal N}$-barbed bisimilar to $Q$, written
$P \wbbisim_{\mathcal N} Q$, if $P \rel{S}_{\mathcal N} Q$ for some ${\mathcal N}$-barbed bisimulation ${\mathcal S}_{\mathcal N}$.
\end{definition}

$\mathcal{R} \subseteq \pi \times \pi$

$P \mathcal{R} Q => \forall P'. P \red P' \Rightarrow \exists Q'. Q \red Q', P' \mathcal{R} Q'$

$P \vdash x \Rightarrow Q \vdash x$

\begin{mathpar}
  \inferrule*[lab=Out-barb]{x \nameeq y}{{y}!\langle{Q}\rangle \vdash x}
  \and
  \inferrule*[lab=Par-barb]{\mbox{$P\vdash x$ or $Q\vdash x$}}{\binpar{P}{Q} \vdash x}
\end{mathpar}

\subsubsection{Contexts}

One of the principle advantages of computational calculi like the
$\pi$-calculus is a well-defined notion of context,
contextual-equivalence and a correlation between
contextual-equivalence and notions of bisimulation. The notion of
context allows the decomposition of a process into (sub-)process and
its syntactic environment, its context. Thus, a context may be
thought of as a process with a ``hole'' (written $\Box$) in it. The
application of a context $M$ to a process $P$, written $M[P]$, is
tantamount to filling the hole in $M$ with $P$. In this paper we do
not need the full weight of this theory, but do make use of the notion
of context in the proof the main theorem. 

\begin{mathpar}
  \inferrule* [lab=summation] {} {{M_{M},M_{N}} \bc \Box \;|\; x.M_{A} \;|\; M_{M}+M_{N}}
  \and
  \inferrule* [lab=agent] {} {{M_{A}} \bc (\vec{x})M_{P} \;| \; \clift{P_0,\ldots,M_{P},\ldots,P_N}}
  \and \\
  \inferrule* [lab=process] {} {{M_{P}} \bc M_{N} \;| \;P|M_{P} }
\end{mathpar} 

\begin{mathpar}
  \inferrule* [lab=sychronization] {} {M_{N} \bc \Box \;|\; x?M_{F} \;|\; x!M_{C}}
  \and
  \inferrule* [lab=abstraction] {} {{M_{F}} \bc (x)M_{P} }
  \and
  \inferrule* [lab=concretion] {} {{M_{C}} \bc \langle M_{P} \rangle }
  \and \\
  \inferrule* [lab=process] {} {{M_{P}} \bc M_{N} \;| \;P|M_{P} }
\end{mathpar}

\begin{definition}[contextual application] Given a context $M$, and
  process $P$, we define the \emph{contextual application}, $M[P] :=
  M\{P/\Box\}$. That is, the contextual application of M to P is the
  substitution of $P$ for $\Box$ in $M$.
\end{definition}

$\meaningof{-} : L \to \mathcal{P}(\pi)$

\begin{mathpar}
  \inferrule* [lab=collection] {} {\meaningof{true} = \pi, \and \meaningof{~E} = \pi \setminus \meaningof{E}, \and \meaningof{E_{1} \& E_{2}} = \meaningof{E_{1}} \cap \meaningof{E_{2}}}
\end{mathpar}

\begin{mathpar}
  \inferrule* [lab=structure] {} {\meaningof{0} = \{ P \in \pi | P \equiv 0 \}, \and \\ \meaningof{E_1 | E_2} = \{ P \in \pi | P \equiv P_{1} | P_{2}, P_{1} \in \meaningof{E_{1}}, P_{2} \in \meaningof{E_2}\} }
\end{mathpar}

\begin{mathpar}
 \inferrule* [lab=behavior] {} {\meaningof{\langle a?b \rangle E} = \{ P \in \pi | P \equiv Q | u?(y)P', \\ \and \\\\ \and \\ \;\;\; u \in \meaningof{a}, \forall z.P'\{z/y\} \in \meaningof{E\{z/b\}}\}, \and \\ \meaningof{a!E} = \{ P \in \pi | P \equiv Q | x!\langle P' \rangle, x \in \meaningof{a} P' \in \meaningof{E}\} }
\end{mathpar}

\begin{mathpar}
 \inferrule* [lab=nominal] {} {\meaningof{\quotep{E}} = \{ \quotep{P} \in \quotep{\pi} | P \in \meaningof{E} \}, \and \meaningof{\quotep{P}} = \{ \quotep{Q} \in \quotep{\pi} | P \equiv Q \} \and \\ \meaningof{@\quotep{E}} = \{ P \in \pi | P \equiv @x, x \in \meaningof{E} \}}
\end{mathpar}

\begin{eqnarray*}
  \\
  \meaningof{-} : TS \to ST
\end{eqnarray*}

\begin{eqnarray*}
  \\
  L : TS \to ST
\end{eqnarray*}

\begin{eqnarray*}
  \\
  P \models E \iff P \in \meaningof{E}
\end{eqnarray*}

\begin{eqnarray*}
  P \approx_{L} Q \iff \forall E \in L. P \models E \iff Q \models E
\end{eqnarray*}

\begin{eqnarray*}
  P \approx_{K} Q
\end{eqnarray*}

\begin{eqnarray*}
  P \approx Q
\end{eqnarray*}

$\approx_{K} = \approx = \approx_{L}$

\subsubsection{Contextual duality}

Note that contexts extend the quotation operation to a family of
operations from processes to names. Given a context, $M$, we can
define a \emph{nominal context}, $\quotep{M}$ by $\quotep{M}[P] :=
\quotep{M[P]}$. To foreshadow what is to come we observe that these
operations enjoy a duality with processes very much like the duality
between vectors and maps from vectors to scalars.

Further, because the calculus is essentially higher-order, we have a
correspondence between contexts and processes. More specifically,
given a name $x$ and a context $M$ we can construct $M^{*}_{x}$ such
that 

\begin{mathpar}
  M^{*}_{x} | \lift{x}{P} \red M[P]
\end{mathpar}

namely,

\begin{mathpar}
  M^{*}_{x} := x?(u).M[\dropn{u}]
\end{mathpar}

The dependence of $M^{*}_{x}$ on a name makes it an abstraction, 

\begin{mathpar}
  M^{*} := (x)x?(u).M[\dropn{u}]
\end{mathpar}

\subsection{Additional notation}

It will sometimes be convenient to denote the process a name
quotes. We already have the notation $x = \quotep{P}$, but it will be
convenient to introduce an alternate notation, $\procn{x}$, when we
want to emphasize the connection to the use of the name. Note that, by
virtue of name equivalence, $\quotep{\procn{x}} \nameeq x$; so, the
notation is consistent with previous definitions.

Further, because names have structure it is possible to effect
substitutions on the basis of that structure. This means we need to
upgrade our notation for substitutions, which we accomplish by
adapting comprehension notation. Thus,

\begin{mathpar}
  P\{ y / x : x \in S \}
\end{mathpar}

is interpreted to mean the process derived from P by replacing (in a
capture-avoiding manner) each occurrence of $x$ in $S$ by $y$. For example,

\begin{mathpar}
  P\{ \quotep{\procn{x}|\procn{x}} / x : x \in \freenames{P} \}
\end{mathpar}

will replace each (occurrence) of a free name $x$ in $P$ by
$\quotep{\procn{x}|\procn{x}}$.

Also, we will avail ourselves of the notation $x^{L}$ and $x^{R}$ to
denote injections of a name into disjoint copies of the name
space. There are numerous ways to accomplish this. One example can be
found in \cite{MeredithR05}. This notation overloads to vectors of
names: $\vec{x}^{\pi} := (x_{i}^{\pi} \; : \; 0 \leq i < |\vec{x}| )$ where $\pi \in \{L,R\}$.

We also use $P^{\Box} := P|\Box$.

In \cite{MeredithR05} an interpretation of the new operator is
given. It turns out that there are several possible interpretations
all enjoying the requisite algebraic properties of the operator (see
\cite{milner91polyadicpi}). We will therefore make liberal use of
$(\nu\; \vec{x})P$.

% subsection the_syntax_and_semantics_of_the_notation_system (end)   

\input{qm2pi.qmops} 

\input{qm2pi.sterngerlach} 

\input{qm2pi.metric} 

% section concurrent_process_calculi (end)

%\input{qm2pi.proofsketch}

% section proof sketch (end)

%\input{qm2pi.slviaknots} 

% section spatial logic via knots (end)

\input{qm2pi.conclusion}

% section conclusion (end)

%\input{qm2pi.dtcodes} 

% section wiring algorithm (end)

\input{qm2pi.ack} 

% section acknowledgments (end)

\newpage


\bibliographystyle{plain}   
\bibliography{../../biblios/main.bib}

\input{qm2pi.rhodetails}

\end{document}



% section front matter (end)

\section{Introduction}\label{sec:introduction} % (fold)
In this draft of the material i am going to have to dispense with the
usual writing conventions adopted in papers on these topics. i'm going
to have adopt whatever tone i need at the time i'm writing up the
calculations. Sometimes this may be very conversational; others it may
be the barest mathematical grunts; others still it may be that i have
lifted text from one of my other papers because the exposition of some
point was better said there. i hope that my readers are not unduly put
out by this decision. i'm not doing this to flout convention or be
rebellious. i find these calculations very technically challenging. To
keep everything going technically, something has to give; i have to
let go of some cognitive burden. So, the academic writing style --
with all of its trade-offs in terms of facilitating technical
communication -- is what i'm letting go of. Perhaps subsequent drafts
can be tightened and polished, but for now, i'm going to speak as if
we were sitting together in a coffee shop with a laptop, wifi and a
pad of paper and a pencil.

So, here's what i have to say. We -- you and i, comfortably ensconced
in our coffee shop and well-equipped with our tools -- can realize and
carry out the calculations of quantum mechanics over a very different
formal theory of dynamics, a formal theory of dynamics that
corresponds to a theory of concurrent computation with
\emph{reflection}. It has the advantage that the underlying theory is
already `quantized', but supports analogues all of the continuuous
operations. Strikingly, this underlying theory has recently been
connected with a notion of metric that we can show, by calculating
together, coincides with the metric induced by the inner product.

There are a lot of reasons why you might be interested in seeing
calculations of this form. Here's why i'm interested. For the past
several centuries there has been no competitor to the ``Newtonian''
account of dynamics. As a result the predominant share of accounts of
dynamical systems and situations have had to be formulated in terms of
the Newtonian machinery. i view this as an intellectually dangerous
position to occupy. Everything, despite it's intrinsic shape, turns
into a nail to be hit with this hammer. Recently, however, the theory
of computation has matured to the point where we have candidates for
theories of dynamics that offer very different perspective on
reasoning about dynamical systems and situations. Testing these
candidates against very successful accounts of dynamical situations,
like quantum mechanics, is going to give us some sense of how mature
they are and some measure of the quality of these accounts of
dynamics.

\subsection{Summary of contributions and outline of paper}

So, we're going to develop an interpretation of the operations of
quantum mechanics normally interpreted by Hilbert spaces and
operators. We're going to do this over a theory of computation. Note
that this is very different than the usual quantum computation program
which develops notions of computation over quantum mechanics. Rather,
we are developing a story that aligns with Wheeler's slogan: It from
Bit. To do this we will first provide an account of the theory of
computation at play here. Then we will dive into a calculation-driven
interpretation of the operations of quantum mechanics.

The reason we take this approach is that -- until very recently --
there hasn't been an axiomatic account of quantum mechanics. As a
result there has been no sharp delineation of the mathematical theory
supporting interpretation of the physical theory and the physical
theory, itself. So, ambient features of the maths are free to be
exploited (or supressed) without a real accounting of their physical
relevance. There is no sharp statement ``here's the physical theory''
qua \emph{theory} and ``here's the mathematical interpretation''
enabling a judgment of how faithful the interpretation is -- apart
from experimental observation. When there is an axiomatic account we
can judge how well a given mathematical formalism supports an
interpretation of the axioms, independent of
experimentation. Likewise, we can judge how well we have captured our
physical evidence and experience with our axiomatics, independent of
any specific mathematical implementation, with accidental detail that
may or may not have physical significance. 

In lieu of a fully fleshed out and vetted axiomatic account of quantum
mechanics, interpreting the operational notions in service of modeling
physical systems will have to suffice. In other words, we are not in
the business of providing a model of Hilbert spaces and operators. We
are in the business of providing a model of quantum mechanics because
we are motivated by testing our notions of dynamics against physical
theory; and, the predictive calculations of the physical theory must
serve as the best formulation -- shy of a fully fleshed out axiomatic
account -- of the physical theory itself (as they have for scientific
theories since time immemorial). Put another way, despite a
whole-hearted commitment to an It-from-Bit ontology, we are firmly
aligned with the shut-up-and-calculate camp as the best way to obtain
results either from the physical perspective or as a quality assurance
measure of our fledgling theory of dynamics.

In detail, we present a reflective process calculus. Then we develop
intuitive correspondences between the notions available in this
calculus and the usual physical notions supporting quantum mechanical
calculations. Thus, 

\begin{table}[htp]
  \center{
    \fbox{
      \begin{tabular}{c|c}
        quantum mechanics & process calculus \\
        \hline
        scalar & name \\
        state vector & process \\
        dual & contextual duals \\
        matrix & formal sums of process-context-dual pairs \\
        orthogonality & process annihilation \\
        inner product & execution-formula + quoting
      \end{tabular}
    }
  }
  \caption{QM - process calculi correspondences}
\end{table}

Then we tighten up these intuitions to operational definitions. We
employ the Dirac notation as the best proxy we can find for an
abstract syntax of the quantum mechanical notions. The definitions we
develop put us in contact with equational constraints coming from the
theory that we demonstrate the definitions and calculations satisfy.

This puts us in a position to shut up and calculate for the
Stern-Gerlach experimental set up, showing how these predictive
calculations become calculations on processes in our theory of a
reflective process calculus.

Penultimately, we demonstrate that the notion of metric coming from
the inner product coincides with the notion of metric available from
the theory of bisimulation. This demonstration gives us the right to
think of space as arising from behavior. Finally, we consider where we
might go from the new vantage point we have obtained.

% section introduction (end) 
 
% section introduction (end)

% \documentclass[12pt]{llncs}
%\documentclass{jktr}

\usepackage[pdftex]{hyperref}                   
\usepackage {listings}
\usepackage {mathpartir}
\usepackage{bcprules}
%\usepackage{listings}
                       
\usepackage{graphicx} 
%\usepackage[margins=2.5cm,nohead,nofoot]{geometry}
%\usepackage{geometry}
\usepackage{amsfonts}
\usepackage{amstext}
\usepackage{latexsym}
\usepackage{amssymb}
\usepackage{color}


%\include{myPreamble}
\include{qm2pi.local} 

%\ifpdf
%\usepackage[pdftex]{graphicx}
%\else
%\usepackage{graphicx}
%\fi

 % \ifpdf
%  \usepackage{pdfsync}
%  \if


%\title{Brief Article}
%\author{David F. Snyder}
%\author{L.G. Meredith}

%\address{Dept. of Math., Texas State University--San Marcos, San Marcos, TX 78666}
       
\pagestyle{empty}


\begin{document}

\lstset{language=[Objective]Caml,frame=shadowbox}

\input{qm2pi.front}

% section front matter (end)

\input{qm2pi.intro} 
 
% section introduction (end)

% \input{qm2pi.knotations} 

% section notation (end)

\input{qm2pi.process.calculi} 

% section concurrent_process_calculi_and_spatial_logics_ (end)
    
%\input{qm2pi.knots2pi} 

%\input{qm2pi.trefoil} 

%\input{qm2pi.mainthm} 

% subsection basic_interpretation (end)

%\input{qm2pi.rho.presentation} 
\subsection{The syntax and semantics of the notation system}\label{sub:the_syntax_and_semantics_of_the_notation_system} % (fold)

We now summarize a technical presentation of the calculus that
embodies our theory of dynamics. The typical presentation of such a
calculus follows the style of giving generators and relations on
them. The grammar, below, describing term constructors, freely
generates the set of processes, $\Proc$. This set is then quotiented
by a relation known as structural congruence and it is over this set
that the notion of dynamics is expressed. This presentation is
essentially that of \cite{MeredithR05} with the addition of
polyadicity and summation. For readability we have relegated some of
the technical subtleties to an appendix.

\subsubsection{Process grammar}\label{subsub:process_grammar}

\begin{mathpar}
  \inferrule* [lab=synchronization] {} {{M} \bc \pzero \;|\; x?F \;|\; x!C }
  \and
  \inferrule* [lab=abstraction] {} {{F} \bc (x)P}
  \and
  \inferrule* [lab=concretion] {} {{C} \bc \langle Q \rangle}
  \and
  \inferrule* [lab=process] {} {{P,Q} \bc M \;| \;P|Q \;|\; @{x}}
  \and
  \inferrule* [lab=name] {} {{x} \bc \quotep{P}}
\end{mathpar} 

Note that $\vec{x}$ (resp. $\vec{P}$) denotes a vector of names
(resp. processes) of length $|\vec{x}|$ (resp. $|\vec{P}|$). We adopt
the following useful abbreviations.

\begin{mathpar}
   x?(\vec{y}).P := x.(\vec{y})P \and  x\clift{\vec{P}} := x.\clift{\vec{P}}
   \and x!(y) := \lift{x}{\dropn{y}}
   \and \Pi_{i=0}^{n-1}P_i := P_0 | \ldots | P_{n-1}
\end{mathpar}

\subsubsection{Structural congruence}

\paragraph{Free and bound names and alpha-equivalence.} At the
core of structural equivalence is alpha-equivalence which identifies
process that are the same up to a change of variable. Formally, we
recognize the distinction between free and bound names. The free names
of a process, $\freenames{P}$, may be calculated recursively as
follows:

\begin{mathpar}
\freenames{\pzero} := \emptyset
  \and \\
  \freenames{x?(y).P} := \{ x \} \cup (\freenames{P} \setminus \{ y \})
  \and 
  \freenames{x!\langle P \rangle} := \{ x \} \cup \{ P \} 
  \and \\
  \freenames{P|Q} := \freenames{P} \cup \freenames{Q}
  \and \\
  \freenames{@{x}} := \{ x \}
\end{mathpar}

$\pi$
$\quotep{\pi}$

$\freenames{-} : \pi \to \mathcal{P}(\quotep{\pi})$

\begin{eqnarray*}
  \freenames{\pzero} & := & \emptyset \\
  \freenames{x?(y).P} & := & \{ x \} \cup (\freenames{P} \setminus \{ y \}) \\
  \freenames{x!\langle P \rangle} & := & \{ x \} \cup \{ P \} \\
  \freenames{P|Q} & := & \freenames{P} \cup \freenames{Q} \\
  \freenames{\dropn{x}} & := & \{ x \}
\end{eqnarray*}

The bound names of a process, $\boundnames{P}$, are those names occurring in $P$
that are not free. For example, in $x?(y).0$, the name $x$ is free, while $y$ is bound.

\begin{mathpar}
  \inferrule* [lab=monoidal-laws] {} { P|Q \equiv Q|P \and P|0 \equiv P \and P|(Q|R) \equiv (P|Q)|R }
\end{mathpar}

\begin{mathpar}
  \inferrule* [lab=alpha-equivalence] {} { (x)P \equiv (y)P\{y/x\} \and y \not\in \freenames{P} }
\end{mathpar}

\begin{definition}
Then two processes, $P,Q$, are alpha-equivalent if $P = Q\{\vec{y}/\vec{x}\}$ for
some $\vec{x} \in \boundnames{Q},\vec{y} \in \boundnames{P}$, where $Q\{\vec{y}/\vec{x}\}$
denotes the capture-avoiding substitution of $\vec{y}$ for $\vec{x}$ in $Q$.
\end{definition}

\begin{definition}
  The {\em structural congruence} \cite{SangiorgiWalker} , $\equiv$,
  between processes is the least congruence containing
  alpha-equivalence, satisfying the abelian monoid laws
  (associativity, commutativity and $\pzero$ as identity) for parallel
  composition $|$ and for summation $+$.
\end{definition}

\subsection{Name equivalence}

We take name equivalence, written $\nameeq$, to be the smallest
equivalence relation generated by the following rules.

\begin{mathpar}
\inferrule*[lab=Quote-drop]
{ }
{ \quotep{@{x}} \nameeq x }

\inferrule*[lab=Struct-equiv]
{ P \scong Q }
{ \quotep{P} \nameeq \quotep{Q} }
\end{mathpar}

The astute reader will have noticed that the mutual recursion of names
and processes imposes a mutual recursion on alpha-equivalence and
structural equivalence via name-equivalence. Fortunately, all of this
works out pleasantly and we may calculate in the natural way, free of
concern. The reader interested in the details is referred to the
appendix \ref{appendix:rho_details}.

\subsection{Substitution}

We use $\Proc$ for the set of processes, $\QProc$ for the set of
names, and $\id{\{}\vec{y} / \vec{x} \id{\}}$ to denote partial maps,
$s : \QProc \rightarrow \QProc$. A map, $s$ lifts, uniquely, to a map
on process terms, $\widehat{s} : \Proc \rightarrow \Proc$ by the
following equations.

\begin{mathpar}
  (0) \psubstp{Q}{P} := 0 \\
  (R \juxtap S) \psubstp{Q}{P}
  :=    
  (R)\psubstp{Q}{P} \juxtap (S) \psubstp{Q}{P} \\
  (x?(y).R) \psubstp{Q}{P}    
  :=    
  (x)\substp{Q}{P} (z)\concat( (R \psubstn{z}{y}) \psubstp{Q}{P} ) \\
  (\lift{x}{R}) \psubstp{Q}{P}  
  :=
  \lift{(x)\substp{Q}{P}}{ R \psubstp{Q}{P} } \\
%   (\dropn{x})  \psubstp{Q}{P}       
%   := 
%   \left\{ 
%     \begin{array}{ccc} 
%       \dropn{\quotep{Q}} & & x \nameeq \quotep{P} \\
%       \dropn{x} & & otherwise \\
%     \end{array}
%   \right. 
  (\dropn{x})  \psubstp{Q}{P}       
  := 
  \left\{ 
    \begin{array}{ccc} 
      Q & & x \nameeq \quotep{P} \\
      \dropn{x} & & otherwise \\
    \end{array}
  \right.
\end{mathpar}
 

where

\begin{eqnarray}
  (x)\id{\{} \lpquote Q \rpquote / \lpquote P \rpquote \id{\}}            = 
  \left\{ 
    \begin{array}{ccc}
      \lpquote Q \rpquote & & x \nameeq \lpquote P \rpquote \\
      x & & otherwise \\
    \end{array}
  \right. \nonumber
\end{eqnarray}

and $z$ is chosen distinct from $\quotep{P}$, $\quotep{Q}$, the free
names in $Q$, and all the names in $R$. Our $\alpha$-equivalence will
be built in the standard way from this substitution.

\begin{remark}\label{rem:no_self_referential_names}
  One consequence of these definitions is that $\forall P. \quotep{P}
  \not\in \freenames{P}$.
\end{remark}

\subsection{ Dynamic quote: an example }

Anticipating something of what's to come, consider applying the
substitution, $\widehat{\id{\{}u / z \id{\}}}$, to the following pair
of processes, $\lift{w}{y!(z)}$ and $w[ \lpquote y!(z) \rpquote ]$.

\begin{eqnarray}
	\lift{w}{y!(z)}\widehat{\id{\{}u / z \id{\}}}
		& = &
		\lift{w}{y!(u)} \nonumber\\
	w[ \lpquote y!(z) \rpquote ] \widehat{ \id{\{}u / z \id{\}} }
		& = &
		w[ \lpquote y!(z) \rpquote ] \nonumber
\end{eqnarray}

Because the body of the process between quotes is impervious to
substitution, we get radically different answers. In fact, by
examining the first process in an input context,
e.g. $x?(z).\lift{w}{y!(z)}$, we see that the process under the lift
operator may be shaped by prefixed inputs binding a name inside it. In
this sense, the lift operator will be seen as a way to dynamically
construct processes before reifying them as names.

Finally equipped with these standard features we can present the
dynamics of the calculus.

\subsubsection{Operational semantics} 

Finally, we introduce the computational dynamics. What marks these
algebras as distinct from other more traditionally studied algebraic
structures, e.g. vector spaces or polynomial rings, is the manner in
which dynamics is captured. In traditional structures, dynamics is typically
expressed through morphisms between such structures, as in linear maps
between vector spaces or morphisms between rings. In algebras
associated with the semantics of computation, the dynamics is
expressed as part of the algebraic structure itself, through a
reduction reduction relation typically denoted by $\red$. Below, we
give a recursive presentation of this relation for the calculus used
in the encoding.

$\red \subseteq \pi \times \pi$
$\red : \pi \to \mathcal{P}(\pi)$

\begin{mathpar}
  \inferrule* [lab=Comm] { \textsf{match}( x_{src}, x_{trgt} ) } { x_{trgt}?(y)P \; | \; x_{src}!\langle {Q} \rangle \red P\{\quotep{Q}/y}\} }
  \and \\
  \inferrule* [lab=Par] {{P} \red {P}'} {{{P} | {Q}} \red {{P}' | {Q}}}
  \and
  \inferrule* [lab=Equiv]{{{P} \scong {P}'} \andalso {{P}' \red {Q}'} \andalso {{Q}' \scong {Q}}}{{P} \red {Q}}
\end{mathpar}

\begin{eqnarray*}
  match_{\equiv} (\quotep{P},\quotep{Q}) & := & P \equiv Q \\
  match_{\dagger}(\quotep{P},\quotep{Q}) & := & \forall R. P|Q \red^{*} R => R \red^{*} 0 \\
  match_{K}(\quotep{P},\quotep{Q}) & := & K \mbox{ for some context } K
\end{eqnarray*}

$u?(x)P | u!\langle Q \rangle \red P\{\quotep{Q}/x\}$

%We write $\wred$ for $\red^*$, and $P\red$ if $\exists Q $ such that $ P \red Q$.
We write $P\red$ if $\exists Q $ such that $ P \red Q$ and $P\not\red$, otherwise.

\section{Replication}

As mentioned before, it is known that replication (and hence
recursion) can be implemented in a higher-order process algebra
\cite{SangiorgiWalker}. As our first example of calculation with the
machinery thus far presented we give the construction explicitly in
the {\rhoc}.

\begin{eqnarray}
	D_{x} & := & \prefix{x}{y}{(\binpar{\outputp{x}{y}}{@{y}})} \nonumber\\
	\bangp_{x}{P} & := & \binpar{{x}!\langle{\binpar{D_{x}}{P}}\rangle}{D_{x}} \nonumber
\end{eqnarray}

\begin{eqnarray}
	\bangp_{x}{P} & & \nonumber\\
	=
	& {x}!\langle{(\prefix{x}{y}{(\outputp{x}{y} | @{y})) | P}}\rangle 
	      | \prefix{x}{y}{(\outputp{x}{y} | @{y})} & \nonumber\\
	\red
	& (\outputp{x}{y} | @{y})\substn{\quotep{(\prefix{x}{y}{(@{y} | \outputp{x}{y})) | P}}}{y} & \nonumber\\
	=
	& \outputp{x}{\quotep{(\prefix{x}{y}{(\outputp{x}{y} | @{y})) | P}}}
	  | {(\prefix{x}{y}{(\outputp{x}{y} | @{y})) | P}} & \nonumber\\
	\red
	& \ldots & \nonumber\\
	\red^*
	& P | P | \ldots & \nonumber
\end{eqnarray}

Of course, this encoding, as an implementation, runs away, unfolding
$\bangp{P}$ eagerly. A lazier and more implementable replication
operator, restricted to input-guarded processes, may be obtained as follows.

\begin{eqnarray}
\bangp{\prefix{u}{v}{P}} 
	:= 
	\binpar{\lift{x}{\prefix{u}{v}{(\binpar{D(x)}{P})}}}{D(x)} \nonumber
\end{eqnarray}

\begin{remark}
  Note that the lazier definition still does not deal with summation
  or mixed summation (i.e. sums over input and output). The reader is
  invited to construct definitions of replication that deal with these
  features. 

  Further, the definitions are parameterized in a name, $x$. Can you,
  gentle reader, make a definition that eliminates this parameter and
  guarantees no accidental interaction between the replication
  machinery and the process being replicated -- i.e. no accidental
  sharing of names used by the process to get its work done and the
  name(s) used by the replication to effect copying. This latter
  revision of the definition of replication is crucial to obtaining
  the expected identity $!!P \sim !P$.
\end{remark}

\begin{remark}\label{rem:paradoxical_combinator}
  The reader familiar with the lambda calculus will have noticed the
  similarity between $D$ and the paradoxical combinator.

  [Ed. note: the existence of this seems to suggest we have to be more
  restrictive on the set of processes and names we admit if we are to
  support no-cloning.]
\end{remark}

\subsubsection{Bisimulation}

The computational dynamics gives rise to another kind of equivalence,
the equivalence of computational behavior. As previously mentioned
this is typically captured \emph{via} some form of bisimulation.

% The notion we use in this paper is weak barbed bisimulation
% \cite{milner91polyadicpi}.

The notion we use in this paper is derived from weak barbed
bisimulation \cite{milner91polyadicpi}. 

\begin{definition}
An \emph{observation relation}, $\downarrow_{\mathcal N}$, over a set
of names, $\mathcal N$, is the smallest relation satisfying the rules
below.

\infrule[Out-barb]{y \in {\mathcal N}, \; x \nameeq y}
		  {\outputp{x}{v} \downarrow_{\mathcal N} x}
\infrule[Par-barb]{\mbox{$P\downarrow_{\mathcal N} x$ or $Q\downarrow_{\mathcal N} x$}}
		  {\binpar{P}{Q} \downarrow_{\mathcal N} x}

We write $P \Downarrow_{\mathcal N} x$ if there is $Q$ such that 
$P \wred Q$ and $Q \downarrow_{\mathcal N} x$.
\end{definition}

\begin{definition}
%\label{def.bbisim}
An  ${\mathcal N}$-\emph{barbed bisimulation} over a set of names, ${\mathcal N}$, is a symmetric binary relation 
${\mathcal S}_{\mathcal N}$ between agents such that $P\rel{S}_{\mathcal N}Q$ implies:
\begin{enumerate}
\item If $P \red P'$ then $Q \wred Q'$ and $P'\rel{S}_{\mathcal N} Q'$.
\item If $P\downarrow_{\mathcal N} x$, then $Q\Downarrow_{\mathcal N} x$.
\end{enumerate}
$P$ is ${\mathcal N}$-barbed bisimilar to $Q$, written
$P \wbbisim_{\mathcal N} Q$, if $P \rel{S}_{\mathcal N} Q$ for some ${\mathcal N}$-barbed bisimulation ${\mathcal S}_{\mathcal N}$.
\end{definition}

$\mathcal{R} \subseteq \pi \times \pi$

$P \mathcal{R} Q => \forall P'. P \red P' \Rightarrow \exists Q'. Q \red Q', P' \mathcal{R} Q'$

$P \vdash x \Rightarrow Q \vdash x$

\begin{mathpar}
  \inferrule*[lab=Out-barb]{x \nameeq y}{{y}!\langle{Q}\rangle \vdash x}
  \and
  \inferrule*[lab=Par-barb]{\mbox{$P\vdash x$ or $Q\vdash x$}}{\binpar{P}{Q} \vdash x}
\end{mathpar}

\subsubsection{Contexts}

One of the principle advantages of computational calculi like the
$\pi$-calculus is a well-defined notion of context,
contextual-equivalence and a correlation between
contextual-equivalence and notions of bisimulation. The notion of
context allows the decomposition of a process into (sub-)process and
its syntactic environment, its context. Thus, a context may be
thought of as a process with a ``hole'' (written $\Box$) in it. The
application of a context $M$ to a process $P$, written $M[P]$, is
tantamount to filling the hole in $M$ with $P$. In this paper we do
not need the full weight of this theory, but do make use of the notion
of context in the proof the main theorem. 

\begin{mathpar}
  \inferrule* [lab=summation] {} {{M_{M},M_{N}} \bc \Box \;|\; x.M_{A} \;|\; M_{M}+M_{N}}
  \and
  \inferrule* [lab=agent] {} {{M_{A}} \bc (\vec{x})M_{P} \;| \; \clift{P_0,\ldots,M_{P},\ldots,P_N}}
  \and \\
  \inferrule* [lab=process] {} {{M_{P}} \bc M_{N} \;| \;P|M_{P} }
\end{mathpar} 

\begin{mathpar}
  \inferrule* [lab=sychronization] {} {M_{N} \bc \Box \;|\; x?M_{F} \;|\; x!M_{C}}
  \and
  \inferrule* [lab=abstraction] {} {{M_{F}} \bc (x)M_{P} }
  \and
  \inferrule* [lab=concretion] {} {{M_{C}} \bc \langle M_{P} \rangle }
  \and \\
  \inferrule* [lab=process] {} {{M_{P}} \bc M_{N} \;| \;P|M_{P} }
\end{mathpar}

\begin{definition}[contextual application] Given a context $M$, and
  process $P$, we define the \emph{contextual application}, $M[P] :=
  M\{P/\Box\}$. That is, the contextual application of M to P is the
  substitution of $P$ for $\Box$ in $M$.
\end{definition}

$\meaningof{-} : L \to \mathcal{P}(\pi)$

\begin{mathpar}
  \inferrule* [lab=collection] {} {\meaningof{true} = \pi, \and \meaningof{~E} = \pi \setminus \meaningof{E}, \and \meaningof{E_{1} \& E_{2}} = \meaningof{E_{1}} \cap \meaningof{E_{2}}}
\end{mathpar}

\begin{mathpar}
  \inferrule* [lab=structure] {} {\meaningof{0} = \{ P \in \pi | P \equiv 0 \}, \and \\ \meaningof{E_1 | E_2} = \{ P \in \pi | P \equiv P_{1} | P_{2}, P_{1} \in \meaningof{E_{1}}, P_{2} \in \meaningof{E_2}\} }
\end{mathpar}

\begin{mathpar}
 \inferrule* [lab=behavior] {} {\meaningof{\langle a?b \rangle E} = \{ P \in \pi | P \equiv Q | u?(y)P', \\ \and \\\\ \and \\ \;\;\; u \in \meaningof{a}, \forall z.P'\{z/y\} \in \meaningof{E\{z/b\}}\}, \and \\ \meaningof{a!E} = \{ P \in \pi | P \equiv Q | x!\langle P' \rangle, x \in \meaningof{a} P' \in \meaningof{E}\} }
\end{mathpar}

\begin{mathpar}
 \inferrule* [lab=nominal] {} {\meaningof{\quotep{E}} = \{ \quotep{P} \in \quotep{\pi} | P \in \meaningof{E} \}, \and \meaningof{\quotep{P}} = \{ \quotep{Q} \in \quotep{\pi} | P \equiv Q \} \and \\ \meaningof{@\quotep{E}} = \{ P \in \pi | P \equiv @x, x \in \meaningof{E} \}}
\end{mathpar}

\begin{eqnarray*}
  \\
  \meaningof{-} : TS \to ST
\end{eqnarray*}

\begin{eqnarray*}
  \\
  L : TS \to ST
\end{eqnarray*}

\begin{eqnarray*}
  \\
  P \models E \iff P \in \meaningof{E}
\end{eqnarray*}

\begin{eqnarray*}
  P \approx_{L} Q \iff \forall E \in L. P \models E \iff Q \models E
\end{eqnarray*}

\begin{eqnarray*}
  P \approx_{K} Q
\end{eqnarray*}

\begin{eqnarray*}
  P \approx Q
\end{eqnarray*}

$\approx_{K} = \approx = \approx_{L}$

\subsubsection{Contextual duality}

Note that contexts extend the quotation operation to a family of
operations from processes to names. Given a context, $M$, we can
define a \emph{nominal context}, $\quotep{M}$ by $\quotep{M}[P] :=
\quotep{M[P]}$. To foreshadow what is to come we observe that these
operations enjoy a duality with processes very much like the duality
between vectors and maps from vectors to scalars.

Further, because the calculus is essentially higher-order, we have a
correspondence between contexts and processes. More specifically,
given a name $x$ and a context $M$ we can construct $M^{*}_{x}$ such
that 

\begin{mathpar}
  M^{*}_{x} | \lift{x}{P} \red M[P]
\end{mathpar}

namely,

\begin{mathpar}
  M^{*}_{x} := x?(u).M[\dropn{u}]
\end{mathpar}

The dependence of $M^{*}_{x}$ on a name makes it an abstraction, 

\begin{mathpar}
  M^{*} := (x)x?(u).M[\dropn{u}]
\end{mathpar}

\subsection{Additional notation}

It will sometimes be convenient to denote the process a name
quotes. We already have the notation $x = \quotep{P}$, but it will be
convenient to introduce an alternate notation, $\procn{x}$, when we
want to emphasize the connection to the use of the name. Note that, by
virtue of name equivalence, $\quotep{\procn{x}} \nameeq x$; so, the
notation is consistent with previous definitions.

Further, because names have structure it is possible to effect
substitutions on the basis of that structure. This means we need to
upgrade our notation for substitutions, which we accomplish by
adapting comprehension notation. Thus,

\begin{mathpar}
  P\{ y / x : x \in S \}
\end{mathpar}

is interpreted to mean the process derived from P by replacing (in a
capture-avoiding manner) each occurrence of $x$ in $S$ by $y$. For example,

\begin{mathpar}
  P\{ \quotep{\procn{x}|\procn{x}} / x : x \in \freenames{P} \}
\end{mathpar}

will replace each (occurrence) of a free name $x$ in $P$ by
$\quotep{\procn{x}|\procn{x}}$.

Also, we will avail ourselves of the notation $x^{L}$ and $x^{R}$ to
denote injections of a name into disjoint copies of the name
space. There are numerous ways to accomplish this. One example can be
found in \cite{MeredithR05}. This notation overloads to vectors of
names: $\vec{x}^{\pi} := (x_{i}^{\pi} \; : \; 0 \leq i < |\vec{x}| )$ where $\pi \in \{L,R\}$.

We also use $P^{\Box} := P|\Box$.

In \cite{MeredithR05} an interpretation of the new operator is
given. It turns out that there are several possible interpretations
all enjoying the requisite algebraic properties of the operator (see
\cite{milner91polyadicpi}). We will therefore make liberal use of
$(\nu\; \vec{x})P$.

% subsection the_syntax_and_semantics_of_the_notation_system (end)   

\input{qm2pi.qmops} 

\input{qm2pi.sterngerlach} 

\input{qm2pi.metric} 

% section concurrent_process_calculi (end)

%\input{qm2pi.proofsketch}

% section proof sketch (end)

%\input{qm2pi.slviaknots} 

% section spatial logic via knots (end)

\input{qm2pi.conclusion}

% section conclusion (end)

%\input{qm2pi.dtcodes} 

% section wiring algorithm (end)

\input{qm2pi.ack} 

% section acknowledgments (end)

\newpage


\bibliographystyle{plain}   
\bibliography{../../biblios/main.bib}

\input{qm2pi.rhodetails}

\end{document}

 

% section notation (end)

\input{qm2pi.process.calculi} 

% section concurrent_process_calculi_and_spatial_logics_ (end)
    
%\documentclass[12pt]{llncs}
%\documentclass{jktr}

\usepackage[pdftex]{hyperref}                   
\usepackage {listings}
\usepackage {mathpartir}
\usepackage{bcprules}
%\usepackage{listings}
                       
\usepackage{graphicx} 
%\usepackage[margins=2.5cm,nohead,nofoot]{geometry}
%\usepackage{geometry}
\usepackage{amsfonts}
\usepackage{amstext}
\usepackage{latexsym}
\usepackage{amssymb}
\usepackage{color}


%\include{myPreamble}
\include{qm2pi.local} 

%\ifpdf
%\usepackage[pdftex]{graphicx}
%\else
%\usepackage{graphicx}
%\fi

 % \ifpdf
%  \usepackage{pdfsync}
%  \if


%\title{Brief Article}
%\author{David F. Snyder}
%\author{L.G. Meredith}

%\address{Dept. of Math., Texas State University--San Marcos, San Marcos, TX 78666}
       
\pagestyle{empty}


\begin{document}

\lstset{language=[Objective]Caml,frame=shadowbox}

\input{qm2pi.front}

% section front matter (end)

\input{qm2pi.intro} 
 
% section introduction (end)

% \input{qm2pi.knotations} 

% section notation (end)

\input{qm2pi.process.calculi} 

% section concurrent_process_calculi_and_spatial_logics_ (end)
    
%\input{qm2pi.knots2pi} 

%\input{qm2pi.trefoil} 

%\input{qm2pi.mainthm} 

% subsection basic_interpretation (end)

%\input{qm2pi.rho.presentation} 
\subsection{The syntax and semantics of the notation system}\label{sub:the_syntax_and_semantics_of_the_notation_system} % (fold)

We now summarize a technical presentation of the calculus that
embodies our theory of dynamics. The typical presentation of such a
calculus follows the style of giving generators and relations on
them. The grammar, below, describing term constructors, freely
generates the set of processes, $\Proc$. This set is then quotiented
by a relation known as structural congruence and it is over this set
that the notion of dynamics is expressed. This presentation is
essentially that of \cite{MeredithR05} with the addition of
polyadicity and summation. For readability we have relegated some of
the technical subtleties to an appendix.

\subsubsection{Process grammar}\label{subsub:process_grammar}

\begin{mathpar}
  \inferrule* [lab=synchronization] {} {{M} \bc \pzero \;|\; x?F \;|\; x!C }
  \and
  \inferrule* [lab=abstraction] {} {{F} \bc (x)P}
  \and
  \inferrule* [lab=concretion] {} {{C} \bc \langle Q \rangle}
  \and
  \inferrule* [lab=process] {} {{P,Q} \bc M \;| \;P|Q \;|\; @{x}}
  \and
  \inferrule* [lab=name] {} {{x} \bc \quotep{P}}
\end{mathpar} 

Note that $\vec{x}$ (resp. $\vec{P}$) denotes a vector of names
(resp. processes) of length $|\vec{x}|$ (resp. $|\vec{P}|$). We adopt
the following useful abbreviations.

\begin{mathpar}
   x?(\vec{y}).P := x.(\vec{y})P \and  x\clift{\vec{P}} := x.\clift{\vec{P}}
   \and x!(y) := \lift{x}{\dropn{y}}
   \and \Pi_{i=0}^{n-1}P_i := P_0 | \ldots | P_{n-1}
\end{mathpar}

\subsubsection{Structural congruence}

\paragraph{Free and bound names and alpha-equivalence.} At the
core of structural equivalence is alpha-equivalence which identifies
process that are the same up to a change of variable. Formally, we
recognize the distinction between free and bound names. The free names
of a process, $\freenames{P}$, may be calculated recursively as
follows:

\begin{mathpar}
\freenames{\pzero} := \emptyset
  \and \\
  \freenames{x?(y).P} := \{ x \} \cup (\freenames{P} \setminus \{ y \})
  \and 
  \freenames{x!\langle P \rangle} := \{ x \} \cup \{ P \} 
  \and \\
  \freenames{P|Q} := \freenames{P} \cup \freenames{Q}
  \and \\
  \freenames{@{x}} := \{ x \}
\end{mathpar}

$\pi$
$\quotep{\pi}$

$\freenames{-} : \pi \to \mathcal{P}(\quotep{\pi})$

\begin{eqnarray*}
  \freenames{\pzero} & := & \emptyset \\
  \freenames{x?(y).P} & := & \{ x \} \cup (\freenames{P} \setminus \{ y \}) \\
  \freenames{x!\langle P \rangle} & := & \{ x \} \cup \{ P \} \\
  \freenames{P|Q} & := & \freenames{P} \cup \freenames{Q} \\
  \freenames{\dropn{x}} & := & \{ x \}
\end{eqnarray*}

The bound names of a process, $\boundnames{P}$, are those names occurring in $P$
that are not free. For example, in $x?(y).0$, the name $x$ is free, while $y$ is bound.

\begin{mathpar}
  \inferrule* [lab=monoidal-laws] {} { P|Q \equiv Q|P \and P|0 \equiv P \and P|(Q|R) \equiv (P|Q)|R }
\end{mathpar}

\begin{mathpar}
  \inferrule* [lab=alpha-equivalence] {} { (x)P \equiv (y)P\{y/x\} \and y \not\in \freenames{P} }
\end{mathpar}

\begin{definition}
Then two processes, $P,Q$, are alpha-equivalent if $P = Q\{\vec{y}/\vec{x}\}$ for
some $\vec{x} \in \boundnames{Q},\vec{y} \in \boundnames{P}$, where $Q\{\vec{y}/\vec{x}\}$
denotes the capture-avoiding substitution of $\vec{y}$ for $\vec{x}$ in $Q$.
\end{definition}

\begin{definition}
  The {\em structural congruence} \cite{SangiorgiWalker} , $\equiv$,
  between processes is the least congruence containing
  alpha-equivalence, satisfying the abelian monoid laws
  (associativity, commutativity and $\pzero$ as identity) for parallel
  composition $|$ and for summation $+$.
\end{definition}

\subsection{Name equivalence}

We take name equivalence, written $\nameeq$, to be the smallest
equivalence relation generated by the following rules.

\begin{mathpar}
\inferrule*[lab=Quote-drop]
{ }
{ \quotep{@{x}} \nameeq x }

\inferrule*[lab=Struct-equiv]
{ P \scong Q }
{ \quotep{P} \nameeq \quotep{Q} }
\end{mathpar}

The astute reader will have noticed that the mutual recursion of names
and processes imposes a mutual recursion on alpha-equivalence and
structural equivalence via name-equivalence. Fortunately, all of this
works out pleasantly and we may calculate in the natural way, free of
concern. The reader interested in the details is referred to the
appendix \ref{appendix:rho_details}.

\subsection{Substitution}

We use $\Proc$ for the set of processes, $\QProc$ for the set of
names, and $\id{\{}\vec{y} / \vec{x} \id{\}}$ to denote partial maps,
$s : \QProc \rightarrow \QProc$. A map, $s$ lifts, uniquely, to a map
on process terms, $\widehat{s} : \Proc \rightarrow \Proc$ by the
following equations.

\begin{mathpar}
  (0) \psubstp{Q}{P} := 0 \\
  (R \juxtap S) \psubstp{Q}{P}
  :=    
  (R)\psubstp{Q}{P} \juxtap (S) \psubstp{Q}{P} \\
  (x?(y).R) \psubstp{Q}{P}    
  :=    
  (x)\substp{Q}{P} (z)\concat( (R \psubstn{z}{y}) \psubstp{Q}{P} ) \\
  (\lift{x}{R}) \psubstp{Q}{P}  
  :=
  \lift{(x)\substp{Q}{P}}{ R \psubstp{Q}{P} } \\
%   (\dropn{x})  \psubstp{Q}{P}       
%   := 
%   \left\{ 
%     \begin{array}{ccc} 
%       \dropn{\quotep{Q}} & & x \nameeq \quotep{P} \\
%       \dropn{x} & & otherwise \\
%     \end{array}
%   \right. 
  (\dropn{x})  \psubstp{Q}{P}       
  := 
  \left\{ 
    \begin{array}{ccc} 
      Q & & x \nameeq \quotep{P} \\
      \dropn{x} & & otherwise \\
    \end{array}
  \right.
\end{mathpar}
 

where

\begin{eqnarray}
  (x)\id{\{} \lpquote Q \rpquote / \lpquote P \rpquote \id{\}}            = 
  \left\{ 
    \begin{array}{ccc}
      \lpquote Q \rpquote & & x \nameeq \lpquote P \rpquote \\
      x & & otherwise \\
    \end{array}
  \right. \nonumber
\end{eqnarray}

and $z$ is chosen distinct from $\quotep{P}$, $\quotep{Q}$, the free
names in $Q$, and all the names in $R$. Our $\alpha$-equivalence will
be built in the standard way from this substitution.

\begin{remark}\label{rem:no_self_referential_names}
  One consequence of these definitions is that $\forall P. \quotep{P}
  \not\in \freenames{P}$.
\end{remark}

\subsection{ Dynamic quote: an example }

Anticipating something of what's to come, consider applying the
substitution, $\widehat{\id{\{}u / z \id{\}}}$, to the following pair
of processes, $\lift{w}{y!(z)}$ and $w[ \lpquote y!(z) \rpquote ]$.

\begin{eqnarray}
	\lift{w}{y!(z)}\widehat{\id{\{}u / z \id{\}}}
		& = &
		\lift{w}{y!(u)} \nonumber\\
	w[ \lpquote y!(z) \rpquote ] \widehat{ \id{\{}u / z \id{\}} }
		& = &
		w[ \lpquote y!(z) \rpquote ] \nonumber
\end{eqnarray}

Because the body of the process between quotes is impervious to
substitution, we get radically different answers. In fact, by
examining the first process in an input context,
e.g. $x?(z).\lift{w}{y!(z)}$, we see that the process under the lift
operator may be shaped by prefixed inputs binding a name inside it. In
this sense, the lift operator will be seen as a way to dynamically
construct processes before reifying them as names.

Finally equipped with these standard features we can present the
dynamics of the calculus.

\subsubsection{Operational semantics} 

Finally, we introduce the computational dynamics. What marks these
algebras as distinct from other more traditionally studied algebraic
structures, e.g. vector spaces or polynomial rings, is the manner in
which dynamics is captured. In traditional structures, dynamics is typically
expressed through morphisms between such structures, as in linear maps
between vector spaces or morphisms between rings. In algebras
associated with the semantics of computation, the dynamics is
expressed as part of the algebraic structure itself, through a
reduction reduction relation typically denoted by $\red$. Below, we
give a recursive presentation of this relation for the calculus used
in the encoding.

$\red \subseteq \pi \times \pi$
$\red : \pi \to \mathcal{P}(\pi)$

\begin{mathpar}
  \inferrule* [lab=Comm] { \textsf{match}( x_{src}, x_{trgt} ) } { x_{trgt}?(y)P \; | \; x_{src}!\langle {Q} \rangle \red P\{\quotep{Q}/y}\} }
  \and \\
  \inferrule* [lab=Par] {{P} \red {P}'} {{{P} | {Q}} \red {{P}' | {Q}}}
  \and
  \inferrule* [lab=Equiv]{{{P} \scong {P}'} \andalso {{P}' \red {Q}'} \andalso {{Q}' \scong {Q}}}{{P} \red {Q}}
\end{mathpar}

\begin{eqnarray*}
  match_{\equiv} (\quotep{P},\quotep{Q}) & := & P \equiv Q \\
  match_{\dagger}(\quotep{P},\quotep{Q}) & := & \forall R. P|Q \red^{*} R => R \red^{*} 0 \\
  match_{K}(\quotep{P},\quotep{Q}) & := & K \mbox{ for some context } K
\end{eqnarray*}

$u?(x)P | u!\langle Q \rangle \red P\{\quotep{Q}/x\}$

%We write $\wred$ for $\red^*$, and $P\red$ if $\exists Q $ such that $ P \red Q$.
We write $P\red$ if $\exists Q $ such that $ P \red Q$ and $P\not\red$, otherwise.

\section{Replication}

As mentioned before, it is known that replication (and hence
recursion) can be implemented in a higher-order process algebra
\cite{SangiorgiWalker}. As our first example of calculation with the
machinery thus far presented we give the construction explicitly in
the {\rhoc}.

\begin{eqnarray}
	D_{x} & := & \prefix{x}{y}{(\binpar{\outputp{x}{y}}{@{y}})} \nonumber\\
	\bangp_{x}{P} & := & \binpar{{x}!\langle{\binpar{D_{x}}{P}}\rangle}{D_{x}} \nonumber
\end{eqnarray}

\begin{eqnarray}
	\bangp_{x}{P} & & \nonumber\\
	=
	& {x}!\langle{(\prefix{x}{y}{(\outputp{x}{y} | @{y})) | P}}\rangle 
	      | \prefix{x}{y}{(\outputp{x}{y} | @{y})} & \nonumber\\
	\red
	& (\outputp{x}{y} | @{y})\substn{\quotep{(\prefix{x}{y}{(@{y} | \outputp{x}{y})) | P}}}{y} & \nonumber\\
	=
	& \outputp{x}{\quotep{(\prefix{x}{y}{(\outputp{x}{y} | @{y})) | P}}}
	  | {(\prefix{x}{y}{(\outputp{x}{y} | @{y})) | P}} & \nonumber\\
	\red
	& \ldots & \nonumber\\
	\red^*
	& P | P | \ldots & \nonumber
\end{eqnarray}

Of course, this encoding, as an implementation, runs away, unfolding
$\bangp{P}$ eagerly. A lazier and more implementable replication
operator, restricted to input-guarded processes, may be obtained as follows.

\begin{eqnarray}
\bangp{\prefix{u}{v}{P}} 
	:= 
	\binpar{\lift{x}{\prefix{u}{v}{(\binpar{D(x)}{P})}}}{D(x)} \nonumber
\end{eqnarray}

\begin{remark}
  Note that the lazier definition still does not deal with summation
  or mixed summation (i.e. sums over input and output). The reader is
  invited to construct definitions of replication that deal with these
  features. 

  Further, the definitions are parameterized in a name, $x$. Can you,
  gentle reader, make a definition that eliminates this parameter and
  guarantees no accidental interaction between the replication
  machinery and the process being replicated -- i.e. no accidental
  sharing of names used by the process to get its work done and the
  name(s) used by the replication to effect copying. This latter
  revision of the definition of replication is crucial to obtaining
  the expected identity $!!P \sim !P$.
\end{remark}

\begin{remark}\label{rem:paradoxical_combinator}
  The reader familiar with the lambda calculus will have noticed the
  similarity between $D$ and the paradoxical combinator.

  [Ed. note: the existence of this seems to suggest we have to be more
  restrictive on the set of processes and names we admit if we are to
  support no-cloning.]
\end{remark}

\subsubsection{Bisimulation}

The computational dynamics gives rise to another kind of equivalence,
the equivalence of computational behavior. As previously mentioned
this is typically captured \emph{via} some form of bisimulation.

% The notion we use in this paper is weak barbed bisimulation
% \cite{milner91polyadicpi}.

The notion we use in this paper is derived from weak barbed
bisimulation \cite{milner91polyadicpi}. 

\begin{definition}
An \emph{observation relation}, $\downarrow_{\mathcal N}$, over a set
of names, $\mathcal N$, is the smallest relation satisfying the rules
below.

\infrule[Out-barb]{y \in {\mathcal N}, \; x \nameeq y}
		  {\outputp{x}{v} \downarrow_{\mathcal N} x}
\infrule[Par-barb]{\mbox{$P\downarrow_{\mathcal N} x$ or $Q\downarrow_{\mathcal N} x$}}
		  {\binpar{P}{Q} \downarrow_{\mathcal N} x}

We write $P \Downarrow_{\mathcal N} x$ if there is $Q$ such that 
$P \wred Q$ and $Q \downarrow_{\mathcal N} x$.
\end{definition}

\begin{definition}
%\label{def.bbisim}
An  ${\mathcal N}$-\emph{barbed bisimulation} over a set of names, ${\mathcal N}$, is a symmetric binary relation 
${\mathcal S}_{\mathcal N}$ between agents such that $P\rel{S}_{\mathcal N}Q$ implies:
\begin{enumerate}
\item If $P \red P'$ then $Q \wred Q'$ and $P'\rel{S}_{\mathcal N} Q'$.
\item If $P\downarrow_{\mathcal N} x$, then $Q\Downarrow_{\mathcal N} x$.
\end{enumerate}
$P$ is ${\mathcal N}$-barbed bisimilar to $Q$, written
$P \wbbisim_{\mathcal N} Q$, if $P \rel{S}_{\mathcal N} Q$ for some ${\mathcal N}$-barbed bisimulation ${\mathcal S}_{\mathcal N}$.
\end{definition}

$\mathcal{R} \subseteq \pi \times \pi$

$P \mathcal{R} Q => \forall P'. P \red P' \Rightarrow \exists Q'. Q \red Q', P' \mathcal{R} Q'$

$P \vdash x \Rightarrow Q \vdash x$

\begin{mathpar}
  \inferrule*[lab=Out-barb]{x \nameeq y}{{y}!\langle{Q}\rangle \vdash x}
  \and
  \inferrule*[lab=Par-barb]{\mbox{$P\vdash x$ or $Q\vdash x$}}{\binpar{P}{Q} \vdash x}
\end{mathpar}

\subsubsection{Contexts}

One of the principle advantages of computational calculi like the
$\pi$-calculus is a well-defined notion of context,
contextual-equivalence and a correlation between
contextual-equivalence and notions of bisimulation. The notion of
context allows the decomposition of a process into (sub-)process and
its syntactic environment, its context. Thus, a context may be
thought of as a process with a ``hole'' (written $\Box$) in it. The
application of a context $M$ to a process $P$, written $M[P]$, is
tantamount to filling the hole in $M$ with $P$. In this paper we do
not need the full weight of this theory, but do make use of the notion
of context in the proof the main theorem. 

\begin{mathpar}
  \inferrule* [lab=summation] {} {{M_{M},M_{N}} \bc \Box \;|\; x.M_{A} \;|\; M_{M}+M_{N}}
  \and
  \inferrule* [lab=agent] {} {{M_{A}} \bc (\vec{x})M_{P} \;| \; \clift{P_0,\ldots,M_{P},\ldots,P_N}}
  \and \\
  \inferrule* [lab=process] {} {{M_{P}} \bc M_{N} \;| \;P|M_{P} }
\end{mathpar} 

\begin{mathpar}
  \inferrule* [lab=sychronization] {} {M_{N} \bc \Box \;|\; x?M_{F} \;|\; x!M_{C}}
  \and
  \inferrule* [lab=abstraction] {} {{M_{F}} \bc (x)M_{P} }
  \and
  \inferrule* [lab=concretion] {} {{M_{C}} \bc \langle M_{P} \rangle }
  \and \\
  \inferrule* [lab=process] {} {{M_{P}} \bc M_{N} \;| \;P|M_{P} }
\end{mathpar}

\begin{definition}[contextual application] Given a context $M$, and
  process $P$, we define the \emph{contextual application}, $M[P] :=
  M\{P/\Box\}$. That is, the contextual application of M to P is the
  substitution of $P$ for $\Box$ in $M$.
\end{definition}

$\meaningof{-} : L \to \mathcal{P}(\pi)$

\begin{mathpar}
  \inferrule* [lab=collection] {} {\meaningof{true} = \pi, \and \meaningof{~E} = \pi \setminus \meaningof{E}, \and \meaningof{E_{1} \& E_{2}} = \meaningof{E_{1}} \cap \meaningof{E_{2}}}
\end{mathpar}

\begin{mathpar}
  \inferrule* [lab=structure] {} {\meaningof{0} = \{ P \in \pi | P \equiv 0 \}, \and \\ \meaningof{E_1 | E_2} = \{ P \in \pi | P \equiv P_{1} | P_{2}, P_{1} \in \meaningof{E_{1}}, P_{2} \in \meaningof{E_2}\} }
\end{mathpar}

\begin{mathpar}
 \inferrule* [lab=behavior] {} {\meaningof{\langle a?b \rangle E} = \{ P \in \pi | P \equiv Q | u?(y)P', \\ \and \\\\ \and \\ \;\;\; u \in \meaningof{a}, \forall z.P'\{z/y\} \in \meaningof{E\{z/b\}}\}, \and \\ \meaningof{a!E} = \{ P \in \pi | P \equiv Q | x!\langle P' \rangle, x \in \meaningof{a} P' \in \meaningof{E}\} }
\end{mathpar}

\begin{mathpar}
 \inferrule* [lab=nominal] {} {\meaningof{\quotep{E}} = \{ \quotep{P} \in \quotep{\pi} | P \in \meaningof{E} \}, \and \meaningof{\quotep{P}} = \{ \quotep{Q} \in \quotep{\pi} | P \equiv Q \} \and \\ \meaningof{@\quotep{E}} = \{ P \in \pi | P \equiv @x, x \in \meaningof{E} \}}
\end{mathpar}

\begin{eqnarray*}
  \\
  \meaningof{-} : TS \to ST
\end{eqnarray*}

\begin{eqnarray*}
  \\
  L : TS \to ST
\end{eqnarray*}

\begin{eqnarray*}
  \\
  P \models E \iff P \in \meaningof{E}
\end{eqnarray*}

\begin{eqnarray*}
  P \approx_{L} Q \iff \forall E \in L. P \models E \iff Q \models E
\end{eqnarray*}

\begin{eqnarray*}
  P \approx_{K} Q
\end{eqnarray*}

\begin{eqnarray*}
  P \approx Q
\end{eqnarray*}

$\approx_{K} = \approx = \approx_{L}$

\subsubsection{Contextual duality}

Note that contexts extend the quotation operation to a family of
operations from processes to names. Given a context, $M$, we can
define a \emph{nominal context}, $\quotep{M}$ by $\quotep{M}[P] :=
\quotep{M[P]}$. To foreshadow what is to come we observe that these
operations enjoy a duality with processes very much like the duality
between vectors and maps from vectors to scalars.

Further, because the calculus is essentially higher-order, we have a
correspondence between contexts and processes. More specifically,
given a name $x$ and a context $M$ we can construct $M^{*}_{x}$ such
that 

\begin{mathpar}
  M^{*}_{x} | \lift{x}{P} \red M[P]
\end{mathpar}

namely,

\begin{mathpar}
  M^{*}_{x} := x?(u).M[\dropn{u}]
\end{mathpar}

The dependence of $M^{*}_{x}$ on a name makes it an abstraction, 

\begin{mathpar}
  M^{*} := (x)x?(u).M[\dropn{u}]
\end{mathpar}

\subsection{Additional notation}

It will sometimes be convenient to denote the process a name
quotes. We already have the notation $x = \quotep{P}$, but it will be
convenient to introduce an alternate notation, $\procn{x}$, when we
want to emphasize the connection to the use of the name. Note that, by
virtue of name equivalence, $\quotep{\procn{x}} \nameeq x$; so, the
notation is consistent with previous definitions.

Further, because names have structure it is possible to effect
substitutions on the basis of that structure. This means we need to
upgrade our notation for substitutions, which we accomplish by
adapting comprehension notation. Thus,

\begin{mathpar}
  P\{ y / x : x \in S \}
\end{mathpar}

is interpreted to mean the process derived from P by replacing (in a
capture-avoiding manner) each occurrence of $x$ in $S$ by $y$. For example,

\begin{mathpar}
  P\{ \quotep{\procn{x}|\procn{x}} / x : x \in \freenames{P} \}
\end{mathpar}

will replace each (occurrence) of a free name $x$ in $P$ by
$\quotep{\procn{x}|\procn{x}}$.

Also, we will avail ourselves of the notation $x^{L}$ and $x^{R}$ to
denote injections of a name into disjoint copies of the name
space. There are numerous ways to accomplish this. One example can be
found in \cite{MeredithR05}. This notation overloads to vectors of
names: $\vec{x}^{\pi} := (x_{i}^{\pi} \; : \; 0 \leq i < |\vec{x}| )$ where $\pi \in \{L,R\}$.

We also use $P^{\Box} := P|\Box$.

In \cite{MeredithR05} an interpretation of the new operator is
given. It turns out that there are several possible interpretations
all enjoying the requisite algebraic properties of the operator (see
\cite{milner91polyadicpi}). We will therefore make liberal use of
$(\nu\; \vec{x})P$.

% subsection the_syntax_and_semantics_of_the_notation_system (end)   

\input{qm2pi.qmops} 

\input{qm2pi.sterngerlach} 

\input{qm2pi.metric} 

% section concurrent_process_calculi (end)

%\input{qm2pi.proofsketch}

% section proof sketch (end)

%\input{qm2pi.slviaknots} 

% section spatial logic via knots (end)

\input{qm2pi.conclusion}

% section conclusion (end)

%\input{qm2pi.dtcodes} 

% section wiring algorithm (end)

\input{qm2pi.ack} 

% section acknowledgments (end)

\newpage


\bibliographystyle{plain}   
\bibliography{../../biblios/main.bib}

\input{qm2pi.rhodetails}

\end{document}

 

%\documentclass[12pt]{llncs}
%\documentclass{jktr}

\usepackage[pdftex]{hyperref}                   
\usepackage {listings}
\usepackage {mathpartir}
\usepackage{bcprules}
%\usepackage{listings}
                       
\usepackage{graphicx} 
%\usepackage[margins=2.5cm,nohead,nofoot]{geometry}
%\usepackage{geometry}
\usepackage{amsfonts}
\usepackage{amstext}
\usepackage{latexsym}
\usepackage{amssymb}
\usepackage{color}


%\include{myPreamble}
\include{qm2pi.local} 

%\ifpdf
%\usepackage[pdftex]{graphicx}
%\else
%\usepackage{graphicx}
%\fi

 % \ifpdf
%  \usepackage{pdfsync}
%  \if


%\title{Brief Article}
%\author{David F. Snyder}
%\author{L.G. Meredith}

%\address{Dept. of Math., Texas State University--San Marcos, San Marcos, TX 78666}
       
\pagestyle{empty}


\begin{document}

\lstset{language=[Objective]Caml,frame=shadowbox}

\input{qm2pi.front}

% section front matter (end)

\input{qm2pi.intro} 
 
% section introduction (end)

% \input{qm2pi.knotations} 

% section notation (end)

\input{qm2pi.process.calculi} 

% section concurrent_process_calculi_and_spatial_logics_ (end)
    
%\input{qm2pi.knots2pi} 

%\input{qm2pi.trefoil} 

%\input{qm2pi.mainthm} 

% subsection basic_interpretation (end)

%\input{qm2pi.rho.presentation} 
\subsection{The syntax and semantics of the notation system}\label{sub:the_syntax_and_semantics_of_the_notation_system} % (fold)

We now summarize a technical presentation of the calculus that
embodies our theory of dynamics. The typical presentation of such a
calculus follows the style of giving generators and relations on
them. The grammar, below, describing term constructors, freely
generates the set of processes, $\Proc$. This set is then quotiented
by a relation known as structural congruence and it is over this set
that the notion of dynamics is expressed. This presentation is
essentially that of \cite{MeredithR05} with the addition of
polyadicity and summation. For readability we have relegated some of
the technical subtleties to an appendix.

\subsubsection{Process grammar}\label{subsub:process_grammar}

\begin{mathpar}
  \inferrule* [lab=synchronization] {} {{M} \bc \pzero \;|\; x?F \;|\; x!C }
  \and
  \inferrule* [lab=abstraction] {} {{F} \bc (x)P}
  \and
  \inferrule* [lab=concretion] {} {{C} \bc \langle Q \rangle}
  \and
  \inferrule* [lab=process] {} {{P,Q} \bc M \;| \;P|Q \;|\; @{x}}
  \and
  \inferrule* [lab=name] {} {{x} \bc \quotep{P}}
\end{mathpar} 

Note that $\vec{x}$ (resp. $\vec{P}$) denotes a vector of names
(resp. processes) of length $|\vec{x}|$ (resp. $|\vec{P}|$). We adopt
the following useful abbreviations.

\begin{mathpar}
   x?(\vec{y}).P := x.(\vec{y})P \and  x\clift{\vec{P}} := x.\clift{\vec{P}}
   \and x!(y) := \lift{x}{\dropn{y}}
   \and \Pi_{i=0}^{n-1}P_i := P_0 | \ldots | P_{n-1}
\end{mathpar}

\subsubsection{Structural congruence}

\paragraph{Free and bound names and alpha-equivalence.} At the
core of structural equivalence is alpha-equivalence which identifies
process that are the same up to a change of variable. Formally, we
recognize the distinction between free and bound names. The free names
of a process, $\freenames{P}$, may be calculated recursively as
follows:

\begin{mathpar}
\freenames{\pzero} := \emptyset
  \and \\
  \freenames{x?(y).P} := \{ x \} \cup (\freenames{P} \setminus \{ y \})
  \and 
  \freenames{x!\langle P \rangle} := \{ x \} \cup \{ P \} 
  \and \\
  \freenames{P|Q} := \freenames{P} \cup \freenames{Q}
  \and \\
  \freenames{@{x}} := \{ x \}
\end{mathpar}

$\pi$
$\quotep{\pi}$

$\freenames{-} : \pi \to \mathcal{P}(\quotep{\pi})$

\begin{eqnarray*}
  \freenames{\pzero} & := & \emptyset \\
  \freenames{x?(y).P} & := & \{ x \} \cup (\freenames{P} \setminus \{ y \}) \\
  \freenames{x!\langle P \rangle} & := & \{ x \} \cup \{ P \} \\
  \freenames{P|Q} & := & \freenames{P} \cup \freenames{Q} \\
  \freenames{\dropn{x}} & := & \{ x \}
\end{eqnarray*}

The bound names of a process, $\boundnames{P}$, are those names occurring in $P$
that are not free. For example, in $x?(y).0$, the name $x$ is free, while $y$ is bound.

\begin{mathpar}
  \inferrule* [lab=monoidal-laws] {} { P|Q \equiv Q|P \and P|0 \equiv P \and P|(Q|R) \equiv (P|Q)|R }
\end{mathpar}

\begin{mathpar}
  \inferrule* [lab=alpha-equivalence] {} { (x)P \equiv (y)P\{y/x\} \and y \not\in \freenames{P} }
\end{mathpar}

\begin{definition}
Then two processes, $P,Q$, are alpha-equivalent if $P = Q\{\vec{y}/\vec{x}\}$ for
some $\vec{x} \in \boundnames{Q},\vec{y} \in \boundnames{P}$, where $Q\{\vec{y}/\vec{x}\}$
denotes the capture-avoiding substitution of $\vec{y}$ for $\vec{x}$ in $Q$.
\end{definition}

\begin{definition}
  The {\em structural congruence} \cite{SangiorgiWalker} , $\equiv$,
  between processes is the least congruence containing
  alpha-equivalence, satisfying the abelian monoid laws
  (associativity, commutativity and $\pzero$ as identity) for parallel
  composition $|$ and for summation $+$.
\end{definition}

\subsection{Name equivalence}

We take name equivalence, written $\nameeq$, to be the smallest
equivalence relation generated by the following rules.

\begin{mathpar}
\inferrule*[lab=Quote-drop]
{ }
{ \quotep{@{x}} \nameeq x }

\inferrule*[lab=Struct-equiv]
{ P \scong Q }
{ \quotep{P} \nameeq \quotep{Q} }
\end{mathpar}

The astute reader will have noticed that the mutual recursion of names
and processes imposes a mutual recursion on alpha-equivalence and
structural equivalence via name-equivalence. Fortunately, all of this
works out pleasantly and we may calculate in the natural way, free of
concern. The reader interested in the details is referred to the
appendix \ref{appendix:rho_details}.

\subsection{Substitution}

We use $\Proc$ for the set of processes, $\QProc$ for the set of
names, and $\id{\{}\vec{y} / \vec{x} \id{\}}$ to denote partial maps,
$s : \QProc \rightarrow \QProc$. A map, $s$ lifts, uniquely, to a map
on process terms, $\widehat{s} : \Proc \rightarrow \Proc$ by the
following equations.

\begin{mathpar}
  (0) \psubstp{Q}{P} := 0 \\
  (R \juxtap S) \psubstp{Q}{P}
  :=    
  (R)\psubstp{Q}{P} \juxtap (S) \psubstp{Q}{P} \\
  (x?(y).R) \psubstp{Q}{P}    
  :=    
  (x)\substp{Q}{P} (z)\concat( (R \psubstn{z}{y}) \psubstp{Q}{P} ) \\
  (\lift{x}{R}) \psubstp{Q}{P}  
  :=
  \lift{(x)\substp{Q}{P}}{ R \psubstp{Q}{P} } \\
%   (\dropn{x})  \psubstp{Q}{P}       
%   := 
%   \left\{ 
%     \begin{array}{ccc} 
%       \dropn{\quotep{Q}} & & x \nameeq \quotep{P} \\
%       \dropn{x} & & otherwise \\
%     \end{array}
%   \right. 
  (\dropn{x})  \psubstp{Q}{P}       
  := 
  \left\{ 
    \begin{array}{ccc} 
      Q & & x \nameeq \quotep{P} \\
      \dropn{x} & & otherwise \\
    \end{array}
  \right.
\end{mathpar}
 

where

\begin{eqnarray}
  (x)\id{\{} \lpquote Q \rpquote / \lpquote P \rpquote \id{\}}            = 
  \left\{ 
    \begin{array}{ccc}
      \lpquote Q \rpquote & & x \nameeq \lpquote P \rpquote \\
      x & & otherwise \\
    \end{array}
  \right. \nonumber
\end{eqnarray}

and $z$ is chosen distinct from $\quotep{P}$, $\quotep{Q}$, the free
names in $Q$, and all the names in $R$. Our $\alpha$-equivalence will
be built in the standard way from this substitution.

\begin{remark}\label{rem:no_self_referential_names}
  One consequence of these definitions is that $\forall P. \quotep{P}
  \not\in \freenames{P}$.
\end{remark}

\subsection{ Dynamic quote: an example }

Anticipating something of what's to come, consider applying the
substitution, $\widehat{\id{\{}u / z \id{\}}}$, to the following pair
of processes, $\lift{w}{y!(z)}$ and $w[ \lpquote y!(z) \rpquote ]$.

\begin{eqnarray}
	\lift{w}{y!(z)}\widehat{\id{\{}u / z \id{\}}}
		& = &
		\lift{w}{y!(u)} \nonumber\\
	w[ \lpquote y!(z) \rpquote ] \widehat{ \id{\{}u / z \id{\}} }
		& = &
		w[ \lpquote y!(z) \rpquote ] \nonumber
\end{eqnarray}

Because the body of the process between quotes is impervious to
substitution, we get radically different answers. In fact, by
examining the first process in an input context,
e.g. $x?(z).\lift{w}{y!(z)}$, we see that the process under the lift
operator may be shaped by prefixed inputs binding a name inside it. In
this sense, the lift operator will be seen as a way to dynamically
construct processes before reifying them as names.

Finally equipped with these standard features we can present the
dynamics of the calculus.

\subsubsection{Operational semantics} 

Finally, we introduce the computational dynamics. What marks these
algebras as distinct from other more traditionally studied algebraic
structures, e.g. vector spaces or polynomial rings, is the manner in
which dynamics is captured. In traditional structures, dynamics is typically
expressed through morphisms between such structures, as in linear maps
between vector spaces or morphisms between rings. In algebras
associated with the semantics of computation, the dynamics is
expressed as part of the algebraic structure itself, through a
reduction reduction relation typically denoted by $\red$. Below, we
give a recursive presentation of this relation for the calculus used
in the encoding.

$\red \subseteq \pi \times \pi$
$\red : \pi \to \mathcal{P}(\pi)$

\begin{mathpar}
  \inferrule* [lab=Comm] { \textsf{match}( x_{src}, x_{trgt} ) } { x_{trgt}?(y)P \; | \; x_{src}!\langle {Q} \rangle \red P\{\quotep{Q}/y}\} }
  \and \\
  \inferrule* [lab=Par] {{P} \red {P}'} {{{P} | {Q}} \red {{P}' | {Q}}}
  \and
  \inferrule* [lab=Equiv]{{{P} \scong {P}'} \andalso {{P}' \red {Q}'} \andalso {{Q}' \scong {Q}}}{{P} \red {Q}}
\end{mathpar}

\begin{eqnarray*}
  match_{\equiv} (\quotep{P},\quotep{Q}) & := & P \equiv Q \\
  match_{\dagger}(\quotep{P},\quotep{Q}) & := & \forall R. P|Q \red^{*} R => R \red^{*} 0 \\
  match_{K}(\quotep{P},\quotep{Q}) & := & K \mbox{ for some context } K
\end{eqnarray*}

$u?(x)P | u!\langle Q \rangle \red P\{\quotep{Q}/x\}$

%We write $\wred$ for $\red^*$, and $P\red$ if $\exists Q $ such that $ P \red Q$.
We write $P\red$ if $\exists Q $ such that $ P \red Q$ and $P\not\red$, otherwise.

\section{Replication}

As mentioned before, it is known that replication (and hence
recursion) can be implemented in a higher-order process algebra
\cite{SangiorgiWalker}. As our first example of calculation with the
machinery thus far presented we give the construction explicitly in
the {\rhoc}.

\begin{eqnarray}
	D_{x} & := & \prefix{x}{y}{(\binpar{\outputp{x}{y}}{@{y}})} \nonumber\\
	\bangp_{x}{P} & := & \binpar{{x}!\langle{\binpar{D_{x}}{P}}\rangle}{D_{x}} \nonumber
\end{eqnarray}

\begin{eqnarray}
	\bangp_{x}{P} & & \nonumber\\
	=
	& {x}!\langle{(\prefix{x}{y}{(\outputp{x}{y} | @{y})) | P}}\rangle 
	      | \prefix{x}{y}{(\outputp{x}{y} | @{y})} & \nonumber\\
	\red
	& (\outputp{x}{y} | @{y})\substn{\quotep{(\prefix{x}{y}{(@{y} | \outputp{x}{y})) | P}}}{y} & \nonumber\\
	=
	& \outputp{x}{\quotep{(\prefix{x}{y}{(\outputp{x}{y} | @{y})) | P}}}
	  | {(\prefix{x}{y}{(\outputp{x}{y} | @{y})) | P}} & \nonumber\\
	\red
	& \ldots & \nonumber\\
	\red^*
	& P | P | \ldots & \nonumber
\end{eqnarray}

Of course, this encoding, as an implementation, runs away, unfolding
$\bangp{P}$ eagerly. A lazier and more implementable replication
operator, restricted to input-guarded processes, may be obtained as follows.

\begin{eqnarray}
\bangp{\prefix{u}{v}{P}} 
	:= 
	\binpar{\lift{x}{\prefix{u}{v}{(\binpar{D(x)}{P})}}}{D(x)} \nonumber
\end{eqnarray}

\begin{remark}
  Note that the lazier definition still does not deal with summation
  or mixed summation (i.e. sums over input and output). The reader is
  invited to construct definitions of replication that deal with these
  features. 

  Further, the definitions are parameterized in a name, $x$. Can you,
  gentle reader, make a definition that eliminates this parameter and
  guarantees no accidental interaction between the replication
  machinery and the process being replicated -- i.e. no accidental
  sharing of names used by the process to get its work done and the
  name(s) used by the replication to effect copying. This latter
  revision of the definition of replication is crucial to obtaining
  the expected identity $!!P \sim !P$.
\end{remark}

\begin{remark}\label{rem:paradoxical_combinator}
  The reader familiar with the lambda calculus will have noticed the
  similarity between $D$ and the paradoxical combinator.

  [Ed. note: the existence of this seems to suggest we have to be more
  restrictive on the set of processes and names we admit if we are to
  support no-cloning.]
\end{remark}

\subsubsection{Bisimulation}

The computational dynamics gives rise to another kind of equivalence,
the equivalence of computational behavior. As previously mentioned
this is typically captured \emph{via} some form of bisimulation.

% The notion we use in this paper is weak barbed bisimulation
% \cite{milner91polyadicpi}.

The notion we use in this paper is derived from weak barbed
bisimulation \cite{milner91polyadicpi}. 

\begin{definition}
An \emph{observation relation}, $\downarrow_{\mathcal N}$, over a set
of names, $\mathcal N$, is the smallest relation satisfying the rules
below.

\infrule[Out-barb]{y \in {\mathcal N}, \; x \nameeq y}
		  {\outputp{x}{v} \downarrow_{\mathcal N} x}
\infrule[Par-barb]{\mbox{$P\downarrow_{\mathcal N} x$ or $Q\downarrow_{\mathcal N} x$}}
		  {\binpar{P}{Q} \downarrow_{\mathcal N} x}

We write $P \Downarrow_{\mathcal N} x$ if there is $Q$ such that 
$P \wred Q$ and $Q \downarrow_{\mathcal N} x$.
\end{definition}

\begin{definition}
%\label{def.bbisim}
An  ${\mathcal N}$-\emph{barbed bisimulation} over a set of names, ${\mathcal N}$, is a symmetric binary relation 
${\mathcal S}_{\mathcal N}$ between agents such that $P\rel{S}_{\mathcal N}Q$ implies:
\begin{enumerate}
\item If $P \red P'$ then $Q \wred Q'$ and $P'\rel{S}_{\mathcal N} Q'$.
\item If $P\downarrow_{\mathcal N} x$, then $Q\Downarrow_{\mathcal N} x$.
\end{enumerate}
$P$ is ${\mathcal N}$-barbed bisimilar to $Q$, written
$P \wbbisim_{\mathcal N} Q$, if $P \rel{S}_{\mathcal N} Q$ for some ${\mathcal N}$-barbed bisimulation ${\mathcal S}_{\mathcal N}$.
\end{definition}

$\mathcal{R} \subseteq \pi \times \pi$

$P \mathcal{R} Q => \forall P'. P \red P' \Rightarrow \exists Q'. Q \red Q', P' \mathcal{R} Q'$

$P \vdash x \Rightarrow Q \vdash x$

\begin{mathpar}
  \inferrule*[lab=Out-barb]{x \nameeq y}{{y}!\langle{Q}\rangle \vdash x}
  \and
  \inferrule*[lab=Par-barb]{\mbox{$P\vdash x$ or $Q\vdash x$}}{\binpar{P}{Q} \vdash x}
\end{mathpar}

\subsubsection{Contexts}

One of the principle advantages of computational calculi like the
$\pi$-calculus is a well-defined notion of context,
contextual-equivalence and a correlation between
contextual-equivalence and notions of bisimulation. The notion of
context allows the decomposition of a process into (sub-)process and
its syntactic environment, its context. Thus, a context may be
thought of as a process with a ``hole'' (written $\Box$) in it. The
application of a context $M$ to a process $P$, written $M[P]$, is
tantamount to filling the hole in $M$ with $P$. In this paper we do
not need the full weight of this theory, but do make use of the notion
of context in the proof the main theorem. 

\begin{mathpar}
  \inferrule* [lab=summation] {} {{M_{M},M_{N}} \bc \Box \;|\; x.M_{A} \;|\; M_{M}+M_{N}}
  \and
  \inferrule* [lab=agent] {} {{M_{A}} \bc (\vec{x})M_{P} \;| \; \clift{P_0,\ldots,M_{P},\ldots,P_N}}
  \and \\
  \inferrule* [lab=process] {} {{M_{P}} \bc M_{N} \;| \;P|M_{P} }
\end{mathpar} 

\begin{mathpar}
  \inferrule* [lab=sychronization] {} {M_{N} \bc \Box \;|\; x?M_{F} \;|\; x!M_{C}}
  \and
  \inferrule* [lab=abstraction] {} {{M_{F}} \bc (x)M_{P} }
  \and
  \inferrule* [lab=concretion] {} {{M_{C}} \bc \langle M_{P} \rangle }
  \and \\
  \inferrule* [lab=process] {} {{M_{P}} \bc M_{N} \;| \;P|M_{P} }
\end{mathpar}

\begin{definition}[contextual application] Given a context $M$, and
  process $P$, we define the \emph{contextual application}, $M[P] :=
  M\{P/\Box\}$. That is, the contextual application of M to P is the
  substitution of $P$ for $\Box$ in $M$.
\end{definition}

$\meaningof{-} : L \to \mathcal{P}(\pi)$

\begin{mathpar}
  \inferrule* [lab=collection] {} {\meaningof{true} = \pi, \and \meaningof{~E} = \pi \setminus \meaningof{E}, \and \meaningof{E_{1} \& E_{2}} = \meaningof{E_{1}} \cap \meaningof{E_{2}}}
\end{mathpar}

\begin{mathpar}
  \inferrule* [lab=structure] {} {\meaningof{0} = \{ P \in \pi | P \equiv 0 \}, \and \\ \meaningof{E_1 | E_2} = \{ P \in \pi | P \equiv P_{1} | P_{2}, P_{1} \in \meaningof{E_{1}}, P_{2} \in \meaningof{E_2}\} }
\end{mathpar}

\begin{mathpar}
 \inferrule* [lab=behavior] {} {\meaningof{\langle a?b \rangle E} = \{ P \in \pi | P \equiv Q | u?(y)P', \\ \and \\\\ \and \\ \;\;\; u \in \meaningof{a}, \forall z.P'\{z/y\} \in \meaningof{E\{z/b\}}\}, \and \\ \meaningof{a!E} = \{ P \in \pi | P \equiv Q | x!\langle P' \rangle, x \in \meaningof{a} P' \in \meaningof{E}\} }
\end{mathpar}

\begin{mathpar}
 \inferrule* [lab=nominal] {} {\meaningof{\quotep{E}} = \{ \quotep{P} \in \quotep{\pi} | P \in \meaningof{E} \}, \and \meaningof{\quotep{P}} = \{ \quotep{Q} \in \quotep{\pi} | P \equiv Q \} \and \\ \meaningof{@\quotep{E}} = \{ P \in \pi | P \equiv @x, x \in \meaningof{E} \}}
\end{mathpar}

\begin{eqnarray*}
  \\
  \meaningof{-} : TS \to ST
\end{eqnarray*}

\begin{eqnarray*}
  \\
  L : TS \to ST
\end{eqnarray*}

\begin{eqnarray*}
  \\
  P \models E \iff P \in \meaningof{E}
\end{eqnarray*}

\begin{eqnarray*}
  P \approx_{L} Q \iff \forall E \in L. P \models E \iff Q \models E
\end{eqnarray*}

\begin{eqnarray*}
  P \approx_{K} Q
\end{eqnarray*}

\begin{eqnarray*}
  P \approx Q
\end{eqnarray*}

$\approx_{K} = \approx = \approx_{L}$

\subsubsection{Contextual duality}

Note that contexts extend the quotation operation to a family of
operations from processes to names. Given a context, $M$, we can
define a \emph{nominal context}, $\quotep{M}$ by $\quotep{M}[P] :=
\quotep{M[P]}$. To foreshadow what is to come we observe that these
operations enjoy a duality with processes very much like the duality
between vectors and maps from vectors to scalars.

Further, because the calculus is essentially higher-order, we have a
correspondence between contexts and processes. More specifically,
given a name $x$ and a context $M$ we can construct $M^{*}_{x}$ such
that 

\begin{mathpar}
  M^{*}_{x} | \lift{x}{P} \red M[P]
\end{mathpar}

namely,

\begin{mathpar}
  M^{*}_{x} := x?(u).M[\dropn{u}]
\end{mathpar}

The dependence of $M^{*}_{x}$ on a name makes it an abstraction, 

\begin{mathpar}
  M^{*} := (x)x?(u).M[\dropn{u}]
\end{mathpar}

\subsection{Additional notation}

It will sometimes be convenient to denote the process a name
quotes. We already have the notation $x = \quotep{P}$, but it will be
convenient to introduce an alternate notation, $\procn{x}$, when we
want to emphasize the connection to the use of the name. Note that, by
virtue of name equivalence, $\quotep{\procn{x}} \nameeq x$; so, the
notation is consistent with previous definitions.

Further, because names have structure it is possible to effect
substitutions on the basis of that structure. This means we need to
upgrade our notation for substitutions, which we accomplish by
adapting comprehension notation. Thus,

\begin{mathpar}
  P\{ y / x : x \in S \}
\end{mathpar}

is interpreted to mean the process derived from P by replacing (in a
capture-avoiding manner) each occurrence of $x$ in $S$ by $y$. For example,

\begin{mathpar}
  P\{ \quotep{\procn{x}|\procn{x}} / x : x \in \freenames{P} \}
\end{mathpar}

will replace each (occurrence) of a free name $x$ in $P$ by
$\quotep{\procn{x}|\procn{x}}$.

Also, we will avail ourselves of the notation $x^{L}$ and $x^{R}$ to
denote injections of a name into disjoint copies of the name
space. There are numerous ways to accomplish this. One example can be
found in \cite{MeredithR05}. This notation overloads to vectors of
names: $\vec{x}^{\pi} := (x_{i}^{\pi} \; : \; 0 \leq i < |\vec{x}| )$ where $\pi \in \{L,R\}$.

We also use $P^{\Box} := P|\Box$.

In \cite{MeredithR05} an interpretation of the new operator is
given. It turns out that there are several possible interpretations
all enjoying the requisite algebraic properties of the operator (see
\cite{milner91polyadicpi}). We will therefore make liberal use of
$(\nu\; \vec{x})P$.

% subsection the_syntax_and_semantics_of_the_notation_system (end)   

\input{qm2pi.qmops} 

\input{qm2pi.sterngerlach} 

\input{qm2pi.metric} 

% section concurrent_process_calculi (end)

%\input{qm2pi.proofsketch}

% section proof sketch (end)

%\input{qm2pi.slviaknots} 

% section spatial logic via knots (end)

\input{qm2pi.conclusion}

% section conclusion (end)

%\input{qm2pi.dtcodes} 

% section wiring algorithm (end)

\input{qm2pi.ack} 

% section acknowledgments (end)

\newpage


\bibliographystyle{plain}   
\bibliography{../../biblios/main.bib}

\input{qm2pi.rhodetails}

\end{document}

 

%\documentclass[12pt]{llncs}
%\documentclass{jktr}

\usepackage[pdftex]{hyperref}                   
\usepackage {listings}
\usepackage {mathpartir}
\usepackage{bcprules}
%\usepackage{listings}
                       
\usepackage{graphicx} 
%\usepackage[margins=2.5cm,nohead,nofoot]{geometry}
%\usepackage{geometry}
\usepackage{amsfonts}
\usepackage{amstext}
\usepackage{latexsym}
\usepackage{amssymb}
\usepackage{color}


%\include{myPreamble}
\include{qm2pi.local} 

%\ifpdf
%\usepackage[pdftex]{graphicx}
%\else
%\usepackage{graphicx}
%\fi

 % \ifpdf
%  \usepackage{pdfsync}
%  \if


%\title{Brief Article}
%\author{David F. Snyder}
%\author{L.G. Meredith}

%\address{Dept. of Math., Texas State University--San Marcos, San Marcos, TX 78666}
       
\pagestyle{empty}


\begin{document}

\lstset{language=[Objective]Caml,frame=shadowbox}

\input{qm2pi.front}

% section front matter (end)

\input{qm2pi.intro} 
 
% section introduction (end)

% \input{qm2pi.knotations} 

% section notation (end)

\input{qm2pi.process.calculi} 

% section concurrent_process_calculi_and_spatial_logics_ (end)
    
%\input{qm2pi.knots2pi} 

%\input{qm2pi.trefoil} 

%\input{qm2pi.mainthm} 

% subsection basic_interpretation (end)

%\input{qm2pi.rho.presentation} 
\subsection{The syntax and semantics of the notation system}\label{sub:the_syntax_and_semantics_of_the_notation_system} % (fold)

We now summarize a technical presentation of the calculus that
embodies our theory of dynamics. The typical presentation of such a
calculus follows the style of giving generators and relations on
them. The grammar, below, describing term constructors, freely
generates the set of processes, $\Proc$. This set is then quotiented
by a relation known as structural congruence and it is over this set
that the notion of dynamics is expressed. This presentation is
essentially that of \cite{MeredithR05} with the addition of
polyadicity and summation. For readability we have relegated some of
the technical subtleties to an appendix.

\subsubsection{Process grammar}\label{subsub:process_grammar}

\begin{mathpar}
  \inferrule* [lab=synchronization] {} {{M} \bc \pzero \;|\; x?F \;|\; x!C }
  \and
  \inferrule* [lab=abstraction] {} {{F} \bc (x)P}
  \and
  \inferrule* [lab=concretion] {} {{C} \bc \langle Q \rangle}
  \and
  \inferrule* [lab=process] {} {{P,Q} \bc M \;| \;P|Q \;|\; @{x}}
  \and
  \inferrule* [lab=name] {} {{x} \bc \quotep{P}}
\end{mathpar} 

Note that $\vec{x}$ (resp. $\vec{P}$) denotes a vector of names
(resp. processes) of length $|\vec{x}|$ (resp. $|\vec{P}|$). We adopt
the following useful abbreviations.

\begin{mathpar}
   x?(\vec{y}).P := x.(\vec{y})P \and  x\clift{\vec{P}} := x.\clift{\vec{P}}
   \and x!(y) := \lift{x}{\dropn{y}}
   \and \Pi_{i=0}^{n-1}P_i := P_0 | \ldots | P_{n-1}
\end{mathpar}

\subsubsection{Structural congruence}

\paragraph{Free and bound names and alpha-equivalence.} At the
core of structural equivalence is alpha-equivalence which identifies
process that are the same up to a change of variable. Formally, we
recognize the distinction between free and bound names. The free names
of a process, $\freenames{P}$, may be calculated recursively as
follows:

\begin{mathpar}
\freenames{\pzero} := \emptyset
  \and \\
  \freenames{x?(y).P} := \{ x \} \cup (\freenames{P} \setminus \{ y \})
  \and 
  \freenames{x!\langle P \rangle} := \{ x \} \cup \{ P \} 
  \and \\
  \freenames{P|Q} := \freenames{P} \cup \freenames{Q}
  \and \\
  \freenames{@{x}} := \{ x \}
\end{mathpar}

$\pi$
$\quotep{\pi}$

$\freenames{-} : \pi \to \mathcal{P}(\quotep{\pi})$

\begin{eqnarray*}
  \freenames{\pzero} & := & \emptyset \\
  \freenames{x?(y).P} & := & \{ x \} \cup (\freenames{P} \setminus \{ y \}) \\
  \freenames{x!\langle P \rangle} & := & \{ x \} \cup \{ P \} \\
  \freenames{P|Q} & := & \freenames{P} \cup \freenames{Q} \\
  \freenames{\dropn{x}} & := & \{ x \}
\end{eqnarray*}

The bound names of a process, $\boundnames{P}$, are those names occurring in $P$
that are not free. For example, in $x?(y).0$, the name $x$ is free, while $y$ is bound.

\begin{mathpar}
  \inferrule* [lab=monoidal-laws] {} { P|Q \equiv Q|P \and P|0 \equiv P \and P|(Q|R) \equiv (P|Q)|R }
\end{mathpar}

\begin{mathpar}
  \inferrule* [lab=alpha-equivalence] {} { (x)P \equiv (y)P\{y/x\} \and y \not\in \freenames{P} }
\end{mathpar}

\begin{definition}
Then two processes, $P,Q$, are alpha-equivalent if $P = Q\{\vec{y}/\vec{x}\}$ for
some $\vec{x} \in \boundnames{Q},\vec{y} \in \boundnames{P}$, where $Q\{\vec{y}/\vec{x}\}$
denotes the capture-avoiding substitution of $\vec{y}$ for $\vec{x}$ in $Q$.
\end{definition}

\begin{definition}
  The {\em structural congruence} \cite{SangiorgiWalker} , $\equiv$,
  between processes is the least congruence containing
  alpha-equivalence, satisfying the abelian monoid laws
  (associativity, commutativity and $\pzero$ as identity) for parallel
  composition $|$ and for summation $+$.
\end{definition}

\subsection{Name equivalence}

We take name equivalence, written $\nameeq$, to be the smallest
equivalence relation generated by the following rules.

\begin{mathpar}
\inferrule*[lab=Quote-drop]
{ }
{ \quotep{@{x}} \nameeq x }

\inferrule*[lab=Struct-equiv]
{ P \scong Q }
{ \quotep{P} \nameeq \quotep{Q} }
\end{mathpar}

The astute reader will have noticed that the mutual recursion of names
and processes imposes a mutual recursion on alpha-equivalence and
structural equivalence via name-equivalence. Fortunately, all of this
works out pleasantly and we may calculate in the natural way, free of
concern. The reader interested in the details is referred to the
appendix \ref{appendix:rho_details}.

\subsection{Substitution}

We use $\Proc$ for the set of processes, $\QProc$ for the set of
names, and $\id{\{}\vec{y} / \vec{x} \id{\}}$ to denote partial maps,
$s : \QProc \rightarrow \QProc$. A map, $s$ lifts, uniquely, to a map
on process terms, $\widehat{s} : \Proc \rightarrow \Proc$ by the
following equations.

\begin{mathpar}
  (0) \psubstp{Q}{P} := 0 \\
  (R \juxtap S) \psubstp{Q}{P}
  :=    
  (R)\psubstp{Q}{P} \juxtap (S) \psubstp{Q}{P} \\
  (x?(y).R) \psubstp{Q}{P}    
  :=    
  (x)\substp{Q}{P} (z)\concat( (R \psubstn{z}{y}) \psubstp{Q}{P} ) \\
  (\lift{x}{R}) \psubstp{Q}{P}  
  :=
  \lift{(x)\substp{Q}{P}}{ R \psubstp{Q}{P} } \\
%   (\dropn{x})  \psubstp{Q}{P}       
%   := 
%   \left\{ 
%     \begin{array}{ccc} 
%       \dropn{\quotep{Q}} & & x \nameeq \quotep{P} \\
%       \dropn{x} & & otherwise \\
%     \end{array}
%   \right. 
  (\dropn{x})  \psubstp{Q}{P}       
  := 
  \left\{ 
    \begin{array}{ccc} 
      Q & & x \nameeq \quotep{P} \\
      \dropn{x} & & otherwise \\
    \end{array}
  \right.
\end{mathpar}
 

where

\begin{eqnarray}
  (x)\id{\{} \lpquote Q \rpquote / \lpquote P \rpquote \id{\}}            = 
  \left\{ 
    \begin{array}{ccc}
      \lpquote Q \rpquote & & x \nameeq \lpquote P \rpquote \\
      x & & otherwise \\
    \end{array}
  \right. \nonumber
\end{eqnarray}

and $z$ is chosen distinct from $\quotep{P}$, $\quotep{Q}$, the free
names in $Q$, and all the names in $R$. Our $\alpha$-equivalence will
be built in the standard way from this substitution.

\begin{remark}\label{rem:no_self_referential_names}
  One consequence of these definitions is that $\forall P. \quotep{P}
  \not\in \freenames{P}$.
\end{remark}

\subsection{ Dynamic quote: an example }

Anticipating something of what's to come, consider applying the
substitution, $\widehat{\id{\{}u / z \id{\}}}$, to the following pair
of processes, $\lift{w}{y!(z)}$ and $w[ \lpquote y!(z) \rpquote ]$.

\begin{eqnarray}
	\lift{w}{y!(z)}\widehat{\id{\{}u / z \id{\}}}
		& = &
		\lift{w}{y!(u)} \nonumber\\
	w[ \lpquote y!(z) \rpquote ] \widehat{ \id{\{}u / z \id{\}} }
		& = &
		w[ \lpquote y!(z) \rpquote ] \nonumber
\end{eqnarray}

Because the body of the process between quotes is impervious to
substitution, we get radically different answers. In fact, by
examining the first process in an input context,
e.g. $x?(z).\lift{w}{y!(z)}$, we see that the process under the lift
operator may be shaped by prefixed inputs binding a name inside it. In
this sense, the lift operator will be seen as a way to dynamically
construct processes before reifying them as names.

Finally equipped with these standard features we can present the
dynamics of the calculus.

\subsubsection{Operational semantics} 

Finally, we introduce the computational dynamics. What marks these
algebras as distinct from other more traditionally studied algebraic
structures, e.g. vector spaces or polynomial rings, is the manner in
which dynamics is captured. In traditional structures, dynamics is typically
expressed through morphisms between such structures, as in linear maps
between vector spaces or morphisms between rings. In algebras
associated with the semantics of computation, the dynamics is
expressed as part of the algebraic structure itself, through a
reduction reduction relation typically denoted by $\red$. Below, we
give a recursive presentation of this relation for the calculus used
in the encoding.

$\red \subseteq \pi \times \pi$
$\red : \pi \to \mathcal{P}(\pi)$

\begin{mathpar}
  \inferrule* [lab=Comm] { \textsf{match}( x_{src}, x_{trgt} ) } { x_{trgt}?(y)P \; | \; x_{src}!\langle {Q} \rangle \red P\{\quotep{Q}/y}\} }
  \and \\
  \inferrule* [lab=Par] {{P} \red {P}'} {{{P} | {Q}} \red {{P}' | {Q}}}
  \and
  \inferrule* [lab=Equiv]{{{P} \scong {P}'} \andalso {{P}' \red {Q}'} \andalso {{Q}' \scong {Q}}}{{P} \red {Q}}
\end{mathpar}

\begin{eqnarray*}
  match_{\equiv} (\quotep{P},\quotep{Q}) & := & P \equiv Q \\
  match_{\dagger}(\quotep{P},\quotep{Q}) & := & \forall R. P|Q \red^{*} R => R \red^{*} 0 \\
  match_{K}(\quotep{P},\quotep{Q}) & := & K \mbox{ for some context } K
\end{eqnarray*}

$u?(x)P | u!\langle Q \rangle \red P\{\quotep{Q}/x\}$

%We write $\wred$ for $\red^*$, and $P\red$ if $\exists Q $ such that $ P \red Q$.
We write $P\red$ if $\exists Q $ such that $ P \red Q$ and $P\not\red$, otherwise.

\section{Replication}

As mentioned before, it is known that replication (and hence
recursion) can be implemented in a higher-order process algebra
\cite{SangiorgiWalker}. As our first example of calculation with the
machinery thus far presented we give the construction explicitly in
the {\rhoc}.

\begin{eqnarray}
	D_{x} & := & \prefix{x}{y}{(\binpar{\outputp{x}{y}}{@{y}})} \nonumber\\
	\bangp_{x}{P} & := & \binpar{{x}!\langle{\binpar{D_{x}}{P}}\rangle}{D_{x}} \nonumber
\end{eqnarray}

\begin{eqnarray}
	\bangp_{x}{P} & & \nonumber\\
	=
	& {x}!\langle{(\prefix{x}{y}{(\outputp{x}{y} | @{y})) | P}}\rangle 
	      | \prefix{x}{y}{(\outputp{x}{y} | @{y})} & \nonumber\\
	\red
	& (\outputp{x}{y} | @{y})\substn{\quotep{(\prefix{x}{y}{(@{y} | \outputp{x}{y})) | P}}}{y} & \nonumber\\
	=
	& \outputp{x}{\quotep{(\prefix{x}{y}{(\outputp{x}{y} | @{y})) | P}}}
	  | {(\prefix{x}{y}{(\outputp{x}{y} | @{y})) | P}} & \nonumber\\
	\red
	& \ldots & \nonumber\\
	\red^*
	& P | P | \ldots & \nonumber
\end{eqnarray}

Of course, this encoding, as an implementation, runs away, unfolding
$\bangp{P}$ eagerly. A lazier and more implementable replication
operator, restricted to input-guarded processes, may be obtained as follows.

\begin{eqnarray}
\bangp{\prefix{u}{v}{P}} 
	:= 
	\binpar{\lift{x}{\prefix{u}{v}{(\binpar{D(x)}{P})}}}{D(x)} \nonumber
\end{eqnarray}

\begin{remark}
  Note that the lazier definition still does not deal with summation
  or mixed summation (i.e. sums over input and output). The reader is
  invited to construct definitions of replication that deal with these
  features. 

  Further, the definitions are parameterized in a name, $x$. Can you,
  gentle reader, make a definition that eliminates this parameter and
  guarantees no accidental interaction between the replication
  machinery and the process being replicated -- i.e. no accidental
  sharing of names used by the process to get its work done and the
  name(s) used by the replication to effect copying. This latter
  revision of the definition of replication is crucial to obtaining
  the expected identity $!!P \sim !P$.
\end{remark}

\begin{remark}\label{rem:paradoxical_combinator}
  The reader familiar with the lambda calculus will have noticed the
  similarity between $D$ and the paradoxical combinator.

  [Ed. note: the existence of this seems to suggest we have to be more
  restrictive on the set of processes and names we admit if we are to
  support no-cloning.]
\end{remark}

\subsubsection{Bisimulation}

The computational dynamics gives rise to another kind of equivalence,
the equivalence of computational behavior. As previously mentioned
this is typically captured \emph{via} some form of bisimulation.

% The notion we use in this paper is weak barbed bisimulation
% \cite{milner91polyadicpi}.

The notion we use in this paper is derived from weak barbed
bisimulation \cite{milner91polyadicpi}. 

\begin{definition}
An \emph{observation relation}, $\downarrow_{\mathcal N}$, over a set
of names, $\mathcal N$, is the smallest relation satisfying the rules
below.

\infrule[Out-barb]{y \in {\mathcal N}, \; x \nameeq y}
		  {\outputp{x}{v} \downarrow_{\mathcal N} x}
\infrule[Par-barb]{\mbox{$P\downarrow_{\mathcal N} x$ or $Q\downarrow_{\mathcal N} x$}}
		  {\binpar{P}{Q} \downarrow_{\mathcal N} x}

We write $P \Downarrow_{\mathcal N} x$ if there is $Q$ such that 
$P \wred Q$ and $Q \downarrow_{\mathcal N} x$.
\end{definition}

\begin{definition}
%\label{def.bbisim}
An  ${\mathcal N}$-\emph{barbed bisimulation} over a set of names, ${\mathcal N}$, is a symmetric binary relation 
${\mathcal S}_{\mathcal N}$ between agents such that $P\rel{S}_{\mathcal N}Q$ implies:
\begin{enumerate}
\item If $P \red P'$ then $Q \wred Q'$ and $P'\rel{S}_{\mathcal N} Q'$.
\item If $P\downarrow_{\mathcal N} x$, then $Q\Downarrow_{\mathcal N} x$.
\end{enumerate}
$P$ is ${\mathcal N}$-barbed bisimilar to $Q$, written
$P \wbbisim_{\mathcal N} Q$, if $P \rel{S}_{\mathcal N} Q$ for some ${\mathcal N}$-barbed bisimulation ${\mathcal S}_{\mathcal N}$.
\end{definition}

$\mathcal{R} \subseteq \pi \times \pi$

$P \mathcal{R} Q => \forall P'. P \red P' \Rightarrow \exists Q'. Q \red Q', P' \mathcal{R} Q'$

$P \vdash x \Rightarrow Q \vdash x$

\begin{mathpar}
  \inferrule*[lab=Out-barb]{x \nameeq y}{{y}!\langle{Q}\rangle \vdash x}
  \and
  \inferrule*[lab=Par-barb]{\mbox{$P\vdash x$ or $Q\vdash x$}}{\binpar{P}{Q} \vdash x}
\end{mathpar}

\subsubsection{Contexts}

One of the principle advantages of computational calculi like the
$\pi$-calculus is a well-defined notion of context,
contextual-equivalence and a correlation between
contextual-equivalence and notions of bisimulation. The notion of
context allows the decomposition of a process into (sub-)process and
its syntactic environment, its context. Thus, a context may be
thought of as a process with a ``hole'' (written $\Box$) in it. The
application of a context $M$ to a process $P$, written $M[P]$, is
tantamount to filling the hole in $M$ with $P$. In this paper we do
not need the full weight of this theory, but do make use of the notion
of context in the proof the main theorem. 

\begin{mathpar}
  \inferrule* [lab=summation] {} {{M_{M},M_{N}} \bc \Box \;|\; x.M_{A} \;|\; M_{M}+M_{N}}
  \and
  \inferrule* [lab=agent] {} {{M_{A}} \bc (\vec{x})M_{P} \;| \; \clift{P_0,\ldots,M_{P},\ldots,P_N}}
  \and \\
  \inferrule* [lab=process] {} {{M_{P}} \bc M_{N} \;| \;P|M_{P} }
\end{mathpar} 

\begin{mathpar}
  \inferrule* [lab=sychronization] {} {M_{N} \bc \Box \;|\; x?M_{F} \;|\; x!M_{C}}
  \and
  \inferrule* [lab=abstraction] {} {{M_{F}} \bc (x)M_{P} }
  \and
  \inferrule* [lab=concretion] {} {{M_{C}} \bc \langle M_{P} \rangle }
  \and \\
  \inferrule* [lab=process] {} {{M_{P}} \bc M_{N} \;| \;P|M_{P} }
\end{mathpar}

\begin{definition}[contextual application] Given a context $M$, and
  process $P$, we define the \emph{contextual application}, $M[P] :=
  M\{P/\Box\}$. That is, the contextual application of M to P is the
  substitution of $P$ for $\Box$ in $M$.
\end{definition}

$\meaningof{-} : L \to \mathcal{P}(\pi)$

\begin{mathpar}
  \inferrule* [lab=collection] {} {\meaningof{true} = \pi, \and \meaningof{~E} = \pi \setminus \meaningof{E}, \and \meaningof{E_{1} \& E_{2}} = \meaningof{E_{1}} \cap \meaningof{E_{2}}}
\end{mathpar}

\begin{mathpar}
  \inferrule* [lab=structure] {} {\meaningof{0} = \{ P \in \pi | P \equiv 0 \}, \and \\ \meaningof{E_1 | E_2} = \{ P \in \pi | P \equiv P_{1} | P_{2}, P_{1} \in \meaningof{E_{1}}, P_{2} \in \meaningof{E_2}\} }
\end{mathpar}

\begin{mathpar}
 \inferrule* [lab=behavior] {} {\meaningof{\langle a?b \rangle E} = \{ P \in \pi | P \equiv Q | u?(y)P', \\ \and \\\\ \and \\ \;\;\; u \in \meaningof{a}, \forall z.P'\{z/y\} \in \meaningof{E\{z/b\}}\}, \and \\ \meaningof{a!E} = \{ P \in \pi | P \equiv Q | x!\langle P' \rangle, x \in \meaningof{a} P' \in \meaningof{E}\} }
\end{mathpar}

\begin{mathpar}
 \inferrule* [lab=nominal] {} {\meaningof{\quotep{E}} = \{ \quotep{P} \in \quotep{\pi} | P \in \meaningof{E} \}, \and \meaningof{\quotep{P}} = \{ \quotep{Q} \in \quotep{\pi} | P \equiv Q \} \and \\ \meaningof{@\quotep{E}} = \{ P \in \pi | P \equiv @x, x \in \meaningof{E} \}}
\end{mathpar}

\begin{eqnarray*}
  \\
  \meaningof{-} : TS \to ST
\end{eqnarray*}

\begin{eqnarray*}
  \\
  L : TS \to ST
\end{eqnarray*}

\begin{eqnarray*}
  \\
  P \models E \iff P \in \meaningof{E}
\end{eqnarray*}

\begin{eqnarray*}
  P \approx_{L} Q \iff \forall E \in L. P \models E \iff Q \models E
\end{eqnarray*}

\begin{eqnarray*}
  P \approx_{K} Q
\end{eqnarray*}

\begin{eqnarray*}
  P \approx Q
\end{eqnarray*}

$\approx_{K} = \approx = \approx_{L}$

\subsubsection{Contextual duality}

Note that contexts extend the quotation operation to a family of
operations from processes to names. Given a context, $M$, we can
define a \emph{nominal context}, $\quotep{M}$ by $\quotep{M}[P] :=
\quotep{M[P]}$. To foreshadow what is to come we observe that these
operations enjoy a duality with processes very much like the duality
between vectors and maps from vectors to scalars.

Further, because the calculus is essentially higher-order, we have a
correspondence between contexts and processes. More specifically,
given a name $x$ and a context $M$ we can construct $M^{*}_{x}$ such
that 

\begin{mathpar}
  M^{*}_{x} | \lift{x}{P} \red M[P]
\end{mathpar}

namely,

\begin{mathpar}
  M^{*}_{x} := x?(u).M[\dropn{u}]
\end{mathpar}

The dependence of $M^{*}_{x}$ on a name makes it an abstraction, 

\begin{mathpar}
  M^{*} := (x)x?(u).M[\dropn{u}]
\end{mathpar}

\subsection{Additional notation}

It will sometimes be convenient to denote the process a name
quotes. We already have the notation $x = \quotep{P}$, but it will be
convenient to introduce an alternate notation, $\procn{x}$, when we
want to emphasize the connection to the use of the name. Note that, by
virtue of name equivalence, $\quotep{\procn{x}} \nameeq x$; so, the
notation is consistent with previous definitions.

Further, because names have structure it is possible to effect
substitutions on the basis of that structure. This means we need to
upgrade our notation for substitutions, which we accomplish by
adapting comprehension notation. Thus,

\begin{mathpar}
  P\{ y / x : x \in S \}
\end{mathpar}

is interpreted to mean the process derived from P by replacing (in a
capture-avoiding manner) each occurrence of $x$ in $S$ by $y$. For example,

\begin{mathpar}
  P\{ \quotep{\procn{x}|\procn{x}} / x : x \in \freenames{P} \}
\end{mathpar}

will replace each (occurrence) of a free name $x$ in $P$ by
$\quotep{\procn{x}|\procn{x}}$.

Also, we will avail ourselves of the notation $x^{L}$ and $x^{R}$ to
denote injections of a name into disjoint copies of the name
space. There are numerous ways to accomplish this. One example can be
found in \cite{MeredithR05}. This notation overloads to vectors of
names: $\vec{x}^{\pi} := (x_{i}^{\pi} \; : \; 0 \leq i < |\vec{x}| )$ where $\pi \in \{L,R\}$.

We also use $P^{\Box} := P|\Box$.

In \cite{MeredithR05} an interpretation of the new operator is
given. It turns out that there are several possible interpretations
all enjoying the requisite algebraic properties of the operator (see
\cite{milner91polyadicpi}). We will therefore make liberal use of
$(\nu\; \vec{x})P$.

% subsection the_syntax_and_semantics_of_the_notation_system (end)   

\input{qm2pi.qmops} 

\input{qm2pi.sterngerlach} 

\input{qm2pi.metric} 

% section concurrent_process_calculi (end)

%\input{qm2pi.proofsketch}

% section proof sketch (end)

%\input{qm2pi.slviaknots} 

% section spatial logic via knots (end)

\input{qm2pi.conclusion}

% section conclusion (end)

%\input{qm2pi.dtcodes} 

% section wiring algorithm (end)

\input{qm2pi.ack} 

% section acknowledgments (end)

\newpage


\bibliographystyle{plain}   
\bibliography{../../biblios/main.bib}

\input{qm2pi.rhodetails}

\end{document}

 

% subsection basic_interpretation (end)

%\input{qm2pi.rho.presentation} 
\subsection{The syntax and semantics of the notation system}\label{sub:the_syntax_and_semantics_of_the_notation_system} % (fold)

We now summarize a technical presentation of the calculus that
embodies our theory of dynamics. The typical presentation of such a
calculus follows the style of giving generators and relations on
them. The grammar, below, describing term constructors, freely
generates the set of processes, $\Proc$. This set is then quotiented
by a relation known as structural congruence and it is over this set
that the notion of dynamics is expressed. This presentation is
essentially that of \cite{MeredithR05} with the addition of
polyadicity and summation. For readability we have relegated some of
the technical subtleties to an appendix.

\subsubsection{Process grammar}\label{subsub:process_grammar}

\begin{mathpar}
  \inferrule* [lab=synchronization] {} {{M} \bc \pzero \;|\; x?F \;|\; x!C }
  \and
  \inferrule* [lab=abstraction] {} {{F} \bc (x)P}
  \and
  \inferrule* [lab=concretion] {} {{C} \bc \langle Q \rangle}
  \and
  \inferrule* [lab=process] {} {{P,Q} \bc M \;| \;P|Q \;|\; @{x}}
  \and
  \inferrule* [lab=name] {} {{x} \bc \quotep{P}}
\end{mathpar} 

Note that $\vec{x}$ (resp. $\vec{P}$) denotes a vector of names
(resp. processes) of length $|\vec{x}|$ (resp. $|\vec{P}|$). We adopt
the following useful abbreviations.

\begin{mathpar}
   x?(\vec{y}).P := x.(\vec{y})P \and  x\clift{\vec{P}} := x.\clift{\vec{P}}
   \and x!(y) := \lift{x}{\dropn{y}}
   \and \Pi_{i=0}^{n-1}P_i := P_0 | \ldots | P_{n-1}
\end{mathpar}

\subsubsection{Structural congruence}

\paragraph{Free and bound names and alpha-equivalence.} At the
core of structural equivalence is alpha-equivalence which identifies
process that are the same up to a change of variable. Formally, we
recognize the distinction between free and bound names. The free names
of a process, $\freenames{P}$, may be calculated recursively as
follows:

\begin{mathpar}
\freenames{\pzero} := \emptyset
  \and \\
  \freenames{x?(y).P} := \{ x \} \cup (\freenames{P} \setminus \{ y \})
  \and 
  \freenames{x!\langle P \rangle} := \{ x \} \cup \{ P \} 
  \and \\
  \freenames{P|Q} := \freenames{P} \cup \freenames{Q}
  \and \\
  \freenames{@{x}} := \{ x \}
\end{mathpar}

$\pi$
$\quotep{\pi}$

$\freenames{-} : \pi \to \mathcal{P}(\quotep{\pi})$

\begin{eqnarray*}
  \freenames{\pzero} & := & \emptyset \\
  \freenames{x?(y).P} & := & \{ x \} \cup (\freenames{P} \setminus \{ y \}) \\
  \freenames{x!\langle P \rangle} & := & \{ x \} \cup \{ P \} \\
  \freenames{P|Q} & := & \freenames{P} \cup \freenames{Q} \\
  \freenames{\dropn{x}} & := & \{ x \}
\end{eqnarray*}

The bound names of a process, $\boundnames{P}$, are those names occurring in $P$
that are not free. For example, in $x?(y).0$, the name $x$ is free, while $y$ is bound.

\begin{mathpar}
  \inferrule* [lab=monoidal-laws] {} { P|Q \equiv Q|P \and P|0 \equiv P \and P|(Q|R) \equiv (P|Q)|R }
\end{mathpar}

\begin{mathpar}
  \inferrule* [lab=alpha-equivalence] {} { (x)P \equiv (y)P\{y/x\} \and y \not\in \freenames{P} }
\end{mathpar}

\begin{definition}
Then two processes, $P,Q$, are alpha-equivalent if $P = Q\{\vec{y}/\vec{x}\}$ for
some $\vec{x} \in \boundnames{Q},\vec{y} \in \boundnames{P}$, where $Q\{\vec{y}/\vec{x}\}$
denotes the capture-avoiding substitution of $\vec{y}$ for $\vec{x}$ in $Q$.
\end{definition}

\begin{definition}
  The {\em structural congruence} \cite{SangiorgiWalker} , $\equiv$,
  between processes is the least congruence containing
  alpha-equivalence, satisfying the abelian monoid laws
  (associativity, commutativity and $\pzero$ as identity) for parallel
  composition $|$ and for summation $+$.
\end{definition}

\subsection{Name equivalence}

We take name equivalence, written $\nameeq$, to be the smallest
equivalence relation generated by the following rules.

\begin{mathpar}
\inferrule*[lab=Quote-drop]
{ }
{ \quotep{@{x}} \nameeq x }

\inferrule*[lab=Struct-equiv]
{ P \scong Q }
{ \quotep{P} \nameeq \quotep{Q} }
\end{mathpar}

The astute reader will have noticed that the mutual recursion of names
and processes imposes a mutual recursion on alpha-equivalence and
structural equivalence via name-equivalence. Fortunately, all of this
works out pleasantly and we may calculate in the natural way, free of
concern. The reader interested in the details is referred to the
appendix \ref{appendix:rho_details}.

\subsection{Substitution}

We use $\Proc$ for the set of processes, $\QProc$ for the set of
names, and $\id{\{}\vec{y} / \vec{x} \id{\}}$ to denote partial maps,
$s : \QProc \rightarrow \QProc$. A map, $s$ lifts, uniquely, to a map
on process terms, $\widehat{s} : \Proc \rightarrow \Proc$ by the
following equations.

\begin{mathpar}
  (0) \psubstp{Q}{P} := 0 \\
  (R \juxtap S) \psubstp{Q}{P}
  :=    
  (R)\psubstp{Q}{P} \juxtap (S) \psubstp{Q}{P} \\
  (x?(y).R) \psubstp{Q}{P}    
  :=    
  (x)\substp{Q}{P} (z)\concat( (R \psubstn{z}{y}) \psubstp{Q}{P} ) \\
  (\lift{x}{R}) \psubstp{Q}{P}  
  :=
  \lift{(x)\substp{Q}{P}}{ R \psubstp{Q}{P} } \\
%   (\dropn{x})  \psubstp{Q}{P}       
%   := 
%   \left\{ 
%     \begin{array}{ccc} 
%       \dropn{\quotep{Q}} & & x \nameeq \quotep{P} \\
%       \dropn{x} & & otherwise \\
%     \end{array}
%   \right. 
  (\dropn{x})  \psubstp{Q}{P}       
  := 
  \left\{ 
    \begin{array}{ccc} 
      Q & & x \nameeq \quotep{P} \\
      \dropn{x} & & otherwise \\
    \end{array}
  \right.
\end{mathpar}
 

where

\begin{eqnarray}
  (x)\id{\{} \lpquote Q \rpquote / \lpquote P \rpquote \id{\}}            = 
  \left\{ 
    \begin{array}{ccc}
      \lpquote Q \rpquote & & x \nameeq \lpquote P \rpquote \\
      x & & otherwise \\
    \end{array}
  \right. \nonumber
\end{eqnarray}

and $z$ is chosen distinct from $\quotep{P}$, $\quotep{Q}$, the free
names in $Q$, and all the names in $R$. Our $\alpha$-equivalence will
be built in the standard way from this substitution.

\begin{remark}\label{rem:no_self_referential_names}
  One consequence of these definitions is that $\forall P. \quotep{P}
  \not\in \freenames{P}$.
\end{remark}

\subsection{ Dynamic quote: an example }

Anticipating something of what's to come, consider applying the
substitution, $\widehat{\id{\{}u / z \id{\}}}$, to the following pair
of processes, $\lift{w}{y!(z)}$ and $w[ \lpquote y!(z) \rpquote ]$.

\begin{eqnarray}
	\lift{w}{y!(z)}\widehat{\id{\{}u / z \id{\}}}
		& = &
		\lift{w}{y!(u)} \nonumber\\
	w[ \lpquote y!(z) \rpquote ] \widehat{ \id{\{}u / z \id{\}} }
		& = &
		w[ \lpquote y!(z) \rpquote ] \nonumber
\end{eqnarray}

Because the body of the process between quotes is impervious to
substitution, we get radically different answers. In fact, by
examining the first process in an input context,
e.g. $x?(z).\lift{w}{y!(z)}$, we see that the process under the lift
operator may be shaped by prefixed inputs binding a name inside it. In
this sense, the lift operator will be seen as a way to dynamically
construct processes before reifying them as names.

Finally equipped with these standard features we can present the
dynamics of the calculus.

\subsubsection{Operational semantics} 

Finally, we introduce the computational dynamics. What marks these
algebras as distinct from other more traditionally studied algebraic
structures, e.g. vector spaces or polynomial rings, is the manner in
which dynamics is captured. In traditional structures, dynamics is typically
expressed through morphisms between such structures, as in linear maps
between vector spaces or morphisms between rings. In algebras
associated with the semantics of computation, the dynamics is
expressed as part of the algebraic structure itself, through a
reduction reduction relation typically denoted by $\red$. Below, we
give a recursive presentation of this relation for the calculus used
in the encoding.

$\red \subseteq \pi \times \pi$
$\red : \pi \to \mathcal{P}(\pi)$

\begin{mathpar}
  \inferrule* [lab=Comm] { \textsf{match}( x_{src}, x_{trgt} ) } { x_{trgt}?(y)P \; | \; x_{src}!\langle {Q} \rangle \red P\{\quotep{Q}/y}\} }
  \and \\
  \inferrule* [lab=Par] {{P} \red {P}'} {{{P} | {Q}} \red {{P}' | {Q}}}
  \and
  \inferrule* [lab=Equiv]{{{P} \scong {P}'} \andalso {{P}' \red {Q}'} \andalso {{Q}' \scong {Q}}}{{P} \red {Q}}
\end{mathpar}

\begin{eqnarray*}
  match_{\equiv} (\quotep{P},\quotep{Q}) & := & P \equiv Q \\
  match_{\dagger}(\quotep{P},\quotep{Q}) & := & \forall R. P|Q \red^{*} R => R \red^{*} 0 \\
  match_{K}(\quotep{P},\quotep{Q}) & := & K \mbox{ for some context } K
\end{eqnarray*}

$u?(x)P | u!\langle Q \rangle \red P\{\quotep{Q}/x\}$

%We write $\wred$ for $\red^*$, and $P\red$ if $\exists Q $ such that $ P \red Q$.
We write $P\red$ if $\exists Q $ such that $ P \red Q$ and $P\not\red$, otherwise.

\section{Replication}

As mentioned before, it is known that replication (and hence
recursion) can be implemented in a higher-order process algebra
\cite{SangiorgiWalker}. As our first example of calculation with the
machinery thus far presented we give the construction explicitly in
the {\rhoc}.

\begin{eqnarray}
	D_{x} & := & \prefix{x}{y}{(\binpar{\outputp{x}{y}}{@{y}})} \nonumber\\
	\bangp_{x}{P} & := & \binpar{{x}!\langle{\binpar{D_{x}}{P}}\rangle}{D_{x}} \nonumber
\end{eqnarray}

\begin{eqnarray}
	\bangp_{x}{P} & & \nonumber\\
	=
	& {x}!\langle{(\prefix{x}{y}{(\outputp{x}{y} | @{y})) | P}}\rangle 
	      | \prefix{x}{y}{(\outputp{x}{y} | @{y})} & \nonumber\\
	\red
	& (\outputp{x}{y} | @{y})\substn{\quotep{(\prefix{x}{y}{(@{y} | \outputp{x}{y})) | P}}}{y} & \nonumber\\
	=
	& \outputp{x}{\quotep{(\prefix{x}{y}{(\outputp{x}{y} | @{y})) | P}}}
	  | {(\prefix{x}{y}{(\outputp{x}{y} | @{y})) | P}} & \nonumber\\
	\red
	& \ldots & \nonumber\\
	\red^*
	& P | P | \ldots & \nonumber
\end{eqnarray}

Of course, this encoding, as an implementation, runs away, unfolding
$\bangp{P}$ eagerly. A lazier and more implementable replication
operator, restricted to input-guarded processes, may be obtained as follows.

\begin{eqnarray}
\bangp{\prefix{u}{v}{P}} 
	:= 
	\binpar{\lift{x}{\prefix{u}{v}{(\binpar{D(x)}{P})}}}{D(x)} \nonumber
\end{eqnarray}

\begin{remark}
  Note that the lazier definition still does not deal with summation
  or mixed summation (i.e. sums over input and output). The reader is
  invited to construct definitions of replication that deal with these
  features. 

  Further, the definitions are parameterized in a name, $x$. Can you,
  gentle reader, make a definition that eliminates this parameter and
  guarantees no accidental interaction between the replication
  machinery and the process being replicated -- i.e. no accidental
  sharing of names used by the process to get its work done and the
  name(s) used by the replication to effect copying. This latter
  revision of the definition of replication is crucial to obtaining
  the expected identity $!!P \sim !P$.
\end{remark}

\begin{remark}\label{rem:paradoxical_combinator}
  The reader familiar with the lambda calculus will have noticed the
  similarity between $D$ and the paradoxical combinator.

  [Ed. note: the existence of this seems to suggest we have to be more
  restrictive on the set of processes and names we admit if we are to
  support no-cloning.]
\end{remark}

\subsubsection{Bisimulation}

The computational dynamics gives rise to another kind of equivalence,
the equivalence of computational behavior. As previously mentioned
this is typically captured \emph{via} some form of bisimulation.

% The notion we use in this paper is weak barbed bisimulation
% \cite{milner91polyadicpi}.

The notion we use in this paper is derived from weak barbed
bisimulation \cite{milner91polyadicpi}. 

\begin{definition}
An \emph{observation relation}, $\downarrow_{\mathcal N}$, over a set
of names, $\mathcal N$, is the smallest relation satisfying the rules
below.

\infrule[Out-barb]{y \in {\mathcal N}, \; x \nameeq y}
		  {\outputp{x}{v} \downarrow_{\mathcal N} x}
\infrule[Par-barb]{\mbox{$P\downarrow_{\mathcal N} x$ or $Q\downarrow_{\mathcal N} x$}}
		  {\binpar{P}{Q} \downarrow_{\mathcal N} x}

We write $P \Downarrow_{\mathcal N} x$ if there is $Q$ such that 
$P \wred Q$ and $Q \downarrow_{\mathcal N} x$.
\end{definition}

\begin{definition}
%\label{def.bbisim}
An  ${\mathcal N}$-\emph{barbed bisimulation} over a set of names, ${\mathcal N}$, is a symmetric binary relation 
${\mathcal S}_{\mathcal N}$ between agents such that $P\rel{S}_{\mathcal N}Q$ implies:
\begin{enumerate}
\item If $P \red P'$ then $Q \wred Q'$ and $P'\rel{S}_{\mathcal N} Q'$.
\item If $P\downarrow_{\mathcal N} x$, then $Q\Downarrow_{\mathcal N} x$.
\end{enumerate}
$P$ is ${\mathcal N}$-barbed bisimilar to $Q$, written
$P \wbbisim_{\mathcal N} Q$, if $P \rel{S}_{\mathcal N} Q$ for some ${\mathcal N}$-barbed bisimulation ${\mathcal S}_{\mathcal N}$.
\end{definition}

$\mathcal{R} \subseteq \pi \times \pi$

$P \mathcal{R} Q => \forall P'. P \red P' \Rightarrow \exists Q'. Q \red Q', P' \mathcal{R} Q'$

$P \vdash x \Rightarrow Q \vdash x$

\begin{mathpar}
  \inferrule*[lab=Out-barb]{x \nameeq y}{{y}!\langle{Q}\rangle \vdash x}
  \and
  \inferrule*[lab=Par-barb]{\mbox{$P\vdash x$ or $Q\vdash x$}}{\binpar{P}{Q} \vdash x}
\end{mathpar}

\subsubsection{Contexts}

One of the principle advantages of computational calculi like the
$\pi$-calculus is a well-defined notion of context,
contextual-equivalence and a correlation between
contextual-equivalence and notions of bisimulation. The notion of
context allows the decomposition of a process into (sub-)process and
its syntactic environment, its context. Thus, a context may be
thought of as a process with a ``hole'' (written $\Box$) in it. The
application of a context $M$ to a process $P$, written $M[P]$, is
tantamount to filling the hole in $M$ with $P$. In this paper we do
not need the full weight of this theory, but do make use of the notion
of context in the proof the main theorem. 

\begin{mathpar}
  \inferrule* [lab=summation] {} {{M_{M},M_{N}} \bc \Box \;|\; x.M_{A} \;|\; M_{M}+M_{N}}
  \and
  \inferrule* [lab=agent] {} {{M_{A}} \bc (\vec{x})M_{P} \;| \; \clift{P_0,\ldots,M_{P},\ldots,P_N}}
  \and \\
  \inferrule* [lab=process] {} {{M_{P}} \bc M_{N} \;| \;P|M_{P} }
\end{mathpar} 

\begin{mathpar}
  \inferrule* [lab=sychronization] {} {M_{N} \bc \Box \;|\; x?M_{F} \;|\; x!M_{C}}
  \and
  \inferrule* [lab=abstraction] {} {{M_{F}} \bc (x)M_{P} }
  \and
  \inferrule* [lab=concretion] {} {{M_{C}} \bc \langle M_{P} \rangle }
  \and \\
  \inferrule* [lab=process] {} {{M_{P}} \bc M_{N} \;| \;P|M_{P} }
\end{mathpar}

\begin{definition}[contextual application] Given a context $M$, and
  process $P$, we define the \emph{contextual application}, $M[P] :=
  M\{P/\Box\}$. That is, the contextual application of M to P is the
  substitution of $P$ for $\Box$ in $M$.
\end{definition}

$\meaningof{-} : L \to \mathcal{P}(\pi)$

\begin{mathpar}
  \inferrule* [lab=collection] {} {\meaningof{true} = \pi, \and \meaningof{~E} = \pi \setminus \meaningof{E}, \and \meaningof{E_{1} \& E_{2}} = \meaningof{E_{1}} \cap \meaningof{E_{2}}}
\end{mathpar}

\begin{mathpar}
  \inferrule* [lab=structure] {} {\meaningof{0} = \{ P \in \pi | P \equiv 0 \}, \and \\ \meaningof{E_1 | E_2} = \{ P \in \pi | P \equiv P_{1} | P_{2}, P_{1} \in \meaningof{E_{1}}, P_{2} \in \meaningof{E_2}\} }
\end{mathpar}

\begin{mathpar}
 \inferrule* [lab=behavior] {} {\meaningof{\langle a?b \rangle E} = \{ P \in \pi | P \equiv Q | u?(y)P', \\ \and \\\\ \and \\ \;\;\; u \in \meaningof{a}, \forall z.P'\{z/y\} \in \meaningof{E\{z/b\}}\}, \and \\ \meaningof{a!E} = \{ P \in \pi | P \equiv Q | x!\langle P' \rangle, x \in \meaningof{a} P' \in \meaningof{E}\} }
\end{mathpar}

\begin{mathpar}
 \inferrule* [lab=nominal] {} {\meaningof{\quotep{E}} = \{ \quotep{P} \in \quotep{\pi} | P \in \meaningof{E} \}, \and \meaningof{\quotep{P}} = \{ \quotep{Q} \in \quotep{\pi} | P \equiv Q \} \and \\ \meaningof{@\quotep{E}} = \{ P \in \pi | P \equiv @x, x \in \meaningof{E} \}}
\end{mathpar}

\begin{eqnarray*}
  \\
  \meaningof{-} : TS \to ST
\end{eqnarray*}

\begin{eqnarray*}
  \\
  L : TS \to ST
\end{eqnarray*}

\begin{eqnarray*}
  \\
  P \models E \iff P \in \meaningof{E}
\end{eqnarray*}

\begin{eqnarray*}
  P \approx_{L} Q \iff \forall E \in L. P \models E \iff Q \models E
\end{eqnarray*}

\begin{eqnarray*}
  P \approx_{K} Q
\end{eqnarray*}

\begin{eqnarray*}
  P \approx Q
\end{eqnarray*}

$\approx_{K} = \approx = \approx_{L}$

\subsubsection{Contextual duality}

Note that contexts extend the quotation operation to a family of
operations from processes to names. Given a context, $M$, we can
define a \emph{nominal context}, $\quotep{M}$ by $\quotep{M}[P] :=
\quotep{M[P]}$. To foreshadow what is to come we observe that these
operations enjoy a duality with processes very much like the duality
between vectors and maps from vectors to scalars.

Further, because the calculus is essentially higher-order, we have a
correspondence between contexts and processes. More specifically,
given a name $x$ and a context $M$ we can construct $M^{*}_{x}$ such
that 

\begin{mathpar}
  M^{*}_{x} | \lift{x}{P} \red M[P]
\end{mathpar}

namely,

\begin{mathpar}
  M^{*}_{x} := x?(u).M[\dropn{u}]
\end{mathpar}

The dependence of $M^{*}_{x}$ on a name makes it an abstraction, 

\begin{mathpar}
  M^{*} := (x)x?(u).M[\dropn{u}]
\end{mathpar}

\subsection{Additional notation}

It will sometimes be convenient to denote the process a name
quotes. We already have the notation $x = \quotep{P}$, but it will be
convenient to introduce an alternate notation, $\procn{x}$, when we
want to emphasize the connection to the use of the name. Note that, by
virtue of name equivalence, $\quotep{\procn{x}} \nameeq x$; so, the
notation is consistent with previous definitions.

Further, because names have structure it is possible to effect
substitutions on the basis of that structure. This means we need to
upgrade our notation for substitutions, which we accomplish by
adapting comprehension notation. Thus,

\begin{mathpar}
  P\{ y / x : x \in S \}
\end{mathpar}

is interpreted to mean the process derived from P by replacing (in a
capture-avoiding manner) each occurrence of $x$ in $S$ by $y$. For example,

\begin{mathpar}
  P\{ \quotep{\procn{x}|\procn{x}} / x : x \in \freenames{P} \}
\end{mathpar}

will replace each (occurrence) of a free name $x$ in $P$ by
$\quotep{\procn{x}|\procn{x}}$.

Also, we will avail ourselves of the notation $x^{L}$ and $x^{R}$ to
denote injections of a name into disjoint copies of the name
space. There are numerous ways to accomplish this. One example can be
found in \cite{MeredithR05}. This notation overloads to vectors of
names: $\vec{x}^{\pi} := (x_{i}^{\pi} \; : \; 0 \leq i < |\vec{x}| )$ where $\pi \in \{L,R\}$.

We also use $P^{\Box} := P|\Box$.

In \cite{MeredithR05} an interpretation of the new operator is
given. It turns out that there are several possible interpretations
all enjoying the requisite algebraic properties of the operator (see
\cite{milner91polyadicpi}). We will therefore make liberal use of
$(\nu\; \vec{x})P$.

% subsection the_syntax_and_semantics_of_the_notation_system (end)   

\section{Interpretation of QM}
\subsection{Supporting definitions}
\subsubsection{Multiplication}
\begin{mathpar}
  \quotep{Q} \cdot \quotep{R} := \quotep{Q|R}
  \and \\
  \quotep{Q} \cdot P := P\{ \quotep{Q|R} / \quotep{R} : \quotep{R} \in \freenames{P} \}
\end{mathpar}

\paragraph{Discussion}
The first line needs little explanation. The second line says that
each free name of the process is replaced with the multiplication of
that name by the scalar. Multiplication of a scalar (name) by a state
(process) results in a process all the names of which have been `moved
over' by parallel composition with the process the scalar
quotes. There is a subtlety that the bound names have to be
manipulated so that multiplied names aren't accidentally
captured. There are many ways to achieve this.

\begin{remark}\label{rem:multiplication_identities}
  The reader is invited to verify that for all $x,y,z \in \QProc$ and $P \in \Proc$
  \begin{mathpar}
    x \cdot \quotep{0} \equiv x 
    \and
    x \cdot y \equiv y \cdot x
    \and
    x \cdot (y \cdot z) \equiv (x \cdot y) \cdot z
    \and \\
    \quotep{0} \cdot P \equiv P
    \and \\
    x \cdot (y \cdot P) \equiv (x \cdot y) \cdot P
    \and \\
    x \cdot (P|Q) \equiv (x \cdot P) | (x \cdot Q)
    \and \\    
  \end{mathpar}
\end{remark}

\subsubsection{Tensor product}

We define a tensor product on processes by structural induction.

\paragraph{Tensor of sums} First note that all summations, including
$\pzero$ and sequence, can be written $\Sigma_{i} x_{i}.A_{i} +
\Sigma_{j} x_{j}.C_{j}$, where we have grouped input-guarded processes
together and output-guarded processes together.

Thus, we can define the tensor product of two summations, $N_{1}\otimes N_{2}$, where

\begin{mathpar}
  N_{1} := \Sigma_{i} x_{i}.A_{i} + \Sigma_{j} x_{j}.C_{j}
  \and
  N_{2} := \Sigma_{i'} y_{i'}.B_{i'} + \Sigma_{j'} y_{j'}.D_{j'} 
\end{mathpar}

as follows.

\begin{mathpar}
  \Sigma_{i} x_{i}.A_{i} + \Sigma_{j} x_{j}.C_{j} \otimes \Sigma_{i'}
  y_{i'}.B_{i'} + \Sigma_{j'} y_{j'}.D_{j'} 
  \and \\
  := \; \Sigma_{i} \Sigma_{i'} \quotep{\stackrel{\vee}{x_{i}}| \stackrel{\vee}{y_{i'}}}.(A_{i}\otimes B_{i'}) \; | \; \Sigma_{i'} \Sigma_{i} \quotep{\stackrel{\vee}{y_{i'}}|\stackrel{\vee}{x_{i}}}.(B_{i'}\otimes A_{i})
  \and
  \;\; | \;\; \Sigma_{j} \Sigma_{j'} \quotep{\stackrel{\vee}{x_{j}}|\stackrel{\vee}{y_{j'}}}.(A_{j}\otimes B_{j'}) \; | \; \Sigma_{j'} \Sigma_{j} \quotep{\stackrel{\vee}{y_{j'}}|\stackrel{\vee}{x_{j}}}.(B_{j'}\otimes A_{j})
\end{mathpar}

\begin{remark}
  Do we need to $x^{L}$ and $y^{R}$ for this construction as well?
\end{remark}

\paragraph{Tensor of parallel compositions} Next, we distribute tensor
over par.

\begin{mathpar}
  P_{1}|P_{2} \otimes Q_{1}|Q_{2} := (P_{1} \otimes Q_{1}) | (P_{1}
  \otimes Q_{2}) | (P_{2} \otimes Q_{1}) | (P_{2} \otimes Q_{2})
\end{mathpar}

\paragraph{Tensor with dropped names} We treat tensor of a
process with a dropped name as parallel composition.

\begin{mathpar}
  P \otimes \dropn{x} := P | \dropn{x}
\end{mathpar}

\paragraph{Tensor of agents}

Finally, we need to define tensor on agents. Note that the definition
of tensor on normal products only tensors inputs with inputs and
outputs with outputs. Thus, we only have to define the operation on
``homogeneous'' pairings.

\begin{mathpar}
  (\vec{x})P \otimes (\vec{y})Q
  \and \\
  := (x_{0}^{L}|y_{0}^{R},\ldots,x_{0}^{L}|y_{n}^{R},\ldots,x_{m}^{L}|y_{0}^{R},\ldots,x_{m}^{L}|y_{n}^R)(P\{ \vec{x}^{L}/\vec{x}\} \otimes Q \{ \vec{y}^{R}/\vec{y}\})
  \and \\
  \clift{\vec{P}} \otimes \clift{\vec{Q}}
  \and \\
  := \clift{P_{0}\otimes Q_{0},\ldots,P_{0}\otimes Q_{n},\ldots,P_{m}\otimes Q_{0},\ldots,P_{m}\otimes Q_{n}}
\end{mathpar}

\begin{remark}
  Observe that arities of tensored abstractions matches arities of
  tensored concretions if the original arities matched. Note also that
  the length of the arities corresponds to the increase in dimension
  we see in ordinary vector space tensor product.
\end{remark}

\begin{remark}
  Operationally, this definition distributes the tensor down to
  components ``linked'' by summation. Tensor over summation is
  intriguing in that it mixes names. Moreover, as a consequence of the
  way it mixes names we have the identities for all $x \in \QProc$ and
  $P,Q \in \Proc$

  \begin{mathpar}
    (x \cdot P) \otimes Q \equiv x \cdot (P \otimes Q) \equiv P \otimes (x \cdot Q)
    \and
    P \otimes \pzero \equiv P
  \end{mathpar}

  that the reader is invited to verify.
\end{remark}

\subsubsection{Annihilation}
\begin{mathpar}
  P^{\perp} := \{ Q | \forall R. P|Q \red^{*} R \Rightarrow R \red^{*} \pzero \}
  \and \\
  P^{\underline{\perp}} := \Sigma_{Q \in P^{\perp}} \quotep{Q}?(y).(\dropn{y}|Q) | \Sigma_{Q \in P^{\perp}} \quotep{Q}\clift{\Box}
\end{mathpar}

\paragraph{Discussion} The reader will note that $P^{\perp}$ is a
\emph{set} of processes, while $P^{\underline{\perp}}$ is a
\emph{context}. We call the set $P^{\perp}$ the \emph{annihilators} of
$P$. The parallel composition of a process in the annihilators of $P$
with $P$ will result in a process, the state space of which has all
paths eventually leading to $\pzero$. Execution may endure loops; but
under reasonable conditions of fairness (naturally guaranteed under
most notions of bisimulation) such a composite process cannot get
stuck in such a loop and will, eventually pop out and terminate.

The context $P^{\underline{\perp}}$ is ready and willing to ``take the
$P$ out of'' the process to which it is applied. It will effectively
transmit the code of the process to which it is applied to one of the
annihilators and run the process against it.

\subsubsection{Evaluation}
We fix $M$ a domain of fully abstract interpretation with an equality
coincident with bisimulation. We take $\meaningof{\cdot} : \Proc \to
M$ to be the map interpreting processes and $\nmeaningof{\cdot} : \M
\to Proc$ to be the map running the other way. Then we define

\begin{mathpar}
  \int P := \nmeaningof{\meaningof{P}}
\end{mathpar}

\paragraph{Discussion}
There are many fully abstract interpretations of Milner's
$\pi$-calculus. Any of them can be used as a basis for interpreting
the reflective calculus here. Equipped with such a domain it is
largely a matter of grinding through to check that the Yoneda
construction for the normalization-by-evaluation program can be
extended to this setting.

\begin{remark}
  The reader is invited to verify that $\int (P^{\underline{\perp}}[P]) = 0$.
\end{remark}

\subsection{Quantum mechanics}

Table \ref{tbl:core_qm_op_defns} gives the core operational definitions

\begin{table}[htp]\label{tbl:core_qm_op_defns}
  \center{
    \fbox{
      \begin{tabular}{c|c}
        quantum mechanics & process calculus \\
        \hline
        scalar & $x := \quotep{P}$ \\
        state vector & $\state{P} := P$ \\
        dual & $\state{P}^{*} := \event{P^{\underline{\perp}}} := \quotep{P^{\underline{\perp}}}[-]$ \\
        matrix & $ \Sigma_{\alpha} \state{P_{\alpha}}x_{\alpha}\event{Q_{\alpha}}$ \\
        vector addition & $\state{P} + \state{Q} := \state{P | Q}$ \\
        tensor product & $\state{P} \otimes \state{Q} := \state{P \otimes Q}$ \\
        inner product & $\innerprod{P}{Q} := \quotep{\int P^{\underline{\perp}}[Q]}$ \\
      \end{tabular}
    }
  }
  \caption{QM - operational definitions}
\end{table}

where

\begin{mathpar}
  \prmatrix{P}{Q} := \fprmatrix{P}{\quotep{\pzero}}{Q}
  \and
  \fprmatrix{P}{x}{Q} := (\state{P},x,\event{Q})
  \and
  (\fprmatrix{P}{x}{Q})(\state{R}) := x \cdot \innerprod{Q}{R} \cdot \state{P}
  \and
  (\fprmatrix{P}{x}{Q})(\event{R}) := x \cdot \innerprod{R}{P} \cdot \event{Q}
\end{mathpar}

\paragraph{Discussion}
As promised: vectors (aka states) are represented as processes; duals
as contextual duals; inner product definition should be compared with
standard inner product definition for ....

\begin{remark}
  Assuming $\int (P^{\underline{\perp}}[P]) = 0$, the reader is
  invited to verify that $(\fprmatrix{P}{x}{P})(\state{P}) = x \cdot \state{P}$.
\end{remark}

\begin{remark}
  The reader is invited to verify that $\innerprod{P}{Q}$ could
  equally well have been written $\quotep{\int \stackrel{\vee}{x}}$
  where $x = \event{P^{\underline{\perp}}}(Q)$.

  One of the motivations for this remark is that there is another way
  to factor these operations. We could package up evaluation in the dual:

  \begin{mathpar}
    \state{P}^{*} := \event{\int P^{\underline{\perp}}} := \quotep{\int P^{\underline{\perp}}}[-]
  \end{mathpar}

  and then have inner product defined by
  
  \begin{mathpar}
    \innerprod{P}{Q} := \event{P}(Q)
  \end{mathpar}

  Hopefully, experience with the calculations will provide guidance on
  the best factoring.
\end{remark}

\begin{remark}
  Assuming $\int (P^{\underline{\perp}}[P]) = 0$, the reader is
  invited to verify that $\forall P,Q. (\prmatrix{0}{Q})(\state{0}) =
  \state{0}$ and dually $(\prmatrix{P}{0})(\event{0}) = \event{0}$.
\end{remark}

\begin{remark}
  i'm a little worried that i don't (yet) have proper support for
  complex conjugacy. But, the observation above may give us a
  clue. According to Abramsky, it must be the case that the scalars
  are iso to the homset of the identity for the tensor -- which the
  observation above characterizes. 

  For now, we will simply bookmark the notion with $\overline{x}$.
\end{remark}

\subsubsection{Adjointness}

We need to give a definition of $(\cdot)^{\dagger}$ for matrices. The
obvious candidate definition is
\begin{mathpar}
(\Sigma_{\alpha}\fprmatrix{P_{\alpha}}{x_{\alpha}}{Q_{\alpha}})^{\dagger}
= \Sigma_{\alpha}\fprmatrix{(Q_{\alpha}^{\underline{\perp}})^{*}}{\overline{x}_{\alpha}}{P_{\alpha}^{\underline{\perp}}} 
\end{mathpar}

But, $(Q_{\alpha}^{\underline{\perp}})^{*}$ requires a name along
which to communicate the process to achieve the context application.

\subsubsection{Basis for a basis}
If processes label states and ``addition'' of states (a.k.a. vector
addition) is interpreted as parallel composition, what corresponds to
notions of linear independence and basis? Here, we recall that Yoshida
has developed a set of \emph{combinators} for an asynchronous verison
of Milner's $\pi$-calculus. These are a finite set of processes such
any process can be expressed as parallel composition of these
combinators together with liberal uses of the new operator and
replication. We can simply give a translation of these into the
present calculus and have reasonable expectation that the property
carries over. That is, that the resultant set allows to express all
processes via parallel composition. Note, however, that there is no
new operator or replication in this calculus. As a result, we expect
that the corresponding set is actually infinite. That is, we expect
that the space is actually infinite dimensional.

\begin{remark}
  The attentive reader may be a bit concerned. Certainly, the
  collection $S$, $K$ and $I$ is a finite set of
  combinators. Shouldn't we expect to see a finite set of combinators
  for an effectively equivalent system? i am very sympathetic to this
  critique and feel it warrants full attention. On the other hand, i
  also have in mind the following analogy. The natural numbers, as a
  monoid under addition, has exactly $1$ generator, while the natural
  numbers, as a monoid under multiplication, has countably many
  generators (the primes). We observe that the application of the
  lambda calculus is much less resource sensitive than the parallel
  composition of the $\pi$-calculus. Could it be the case that we have
  an analogy of the form
  
  \begin{mathpar}
    m + n : MN :: m*n : M|N
  \end{mathpar}

  giving a similar blow up in the set of ``primes''?  This is such a
  wonderful thought that, even if it's not true, i think it's worth
  writing down.
\end{remark}
 

\documentclass[12pt]{llncs}
%\documentclass{jktr}

\usepackage[pdftex]{hyperref}                   
\usepackage {listings}
\usepackage {mathpartir}
\usepackage{bcprules}
%\usepackage{listings}
                       
\usepackage{graphicx} 
%\usepackage[margins=2.5cm,nohead,nofoot]{geometry}
%\usepackage{geometry}
\usepackage{amsfonts}
\usepackage{amstext}
\usepackage{latexsym}
\usepackage{amssymb}
\usepackage{color}


%\include{myPreamble}
\include{qm2pi.local} 

%\ifpdf
%\usepackage[pdftex]{graphicx}
%\else
%\usepackage{graphicx}
%\fi

 % \ifpdf
%  \usepackage{pdfsync}
%  \if


%\title{Brief Article}
%\author{David F. Snyder}
%\author{L.G. Meredith}

%\address{Dept. of Math., Texas State University--San Marcos, San Marcos, TX 78666}
       
\pagestyle{empty}


\begin{document}

\lstset{language=[Objective]Caml,frame=shadowbox}

\input{qm2pi.front}

% section front matter (end)

\input{qm2pi.intro} 
 
% section introduction (end)

% \input{qm2pi.knotations} 

% section notation (end)

\input{qm2pi.process.calculi} 

% section concurrent_process_calculi_and_spatial_logics_ (end)
    
%\input{qm2pi.knots2pi} 

%\input{qm2pi.trefoil} 

%\input{qm2pi.mainthm} 

% subsection basic_interpretation (end)

%\input{qm2pi.rho.presentation} 
\subsection{The syntax and semantics of the notation system}\label{sub:the_syntax_and_semantics_of_the_notation_system} % (fold)

We now summarize a technical presentation of the calculus that
embodies our theory of dynamics. The typical presentation of such a
calculus follows the style of giving generators and relations on
them. The grammar, below, describing term constructors, freely
generates the set of processes, $\Proc$. This set is then quotiented
by a relation known as structural congruence and it is over this set
that the notion of dynamics is expressed. This presentation is
essentially that of \cite{MeredithR05} with the addition of
polyadicity and summation. For readability we have relegated some of
the technical subtleties to an appendix.

\subsubsection{Process grammar}\label{subsub:process_grammar}

\begin{mathpar}
  \inferrule* [lab=synchronization] {} {{M} \bc \pzero \;|\; x?F \;|\; x!C }
  \and
  \inferrule* [lab=abstraction] {} {{F} \bc (x)P}
  \and
  \inferrule* [lab=concretion] {} {{C} \bc \langle Q \rangle}
  \and
  \inferrule* [lab=process] {} {{P,Q} \bc M \;| \;P|Q \;|\; @{x}}
  \and
  \inferrule* [lab=name] {} {{x} \bc \quotep{P}}
\end{mathpar} 

Note that $\vec{x}$ (resp. $\vec{P}$) denotes a vector of names
(resp. processes) of length $|\vec{x}|$ (resp. $|\vec{P}|$). We adopt
the following useful abbreviations.

\begin{mathpar}
   x?(\vec{y}).P := x.(\vec{y})P \and  x\clift{\vec{P}} := x.\clift{\vec{P}}
   \and x!(y) := \lift{x}{\dropn{y}}
   \and \Pi_{i=0}^{n-1}P_i := P_0 | \ldots | P_{n-1}
\end{mathpar}

\subsubsection{Structural congruence}

\paragraph{Free and bound names and alpha-equivalence.} At the
core of structural equivalence is alpha-equivalence which identifies
process that are the same up to a change of variable. Formally, we
recognize the distinction between free and bound names. The free names
of a process, $\freenames{P}$, may be calculated recursively as
follows:

\begin{mathpar}
\freenames{\pzero} := \emptyset
  \and \\
  \freenames{x?(y).P} := \{ x \} \cup (\freenames{P} \setminus \{ y \})
  \and 
  \freenames{x!\langle P \rangle} := \{ x \} \cup \{ P \} 
  \and \\
  \freenames{P|Q} := \freenames{P} \cup \freenames{Q}
  \and \\
  \freenames{@{x}} := \{ x \}
\end{mathpar}

$\pi$
$\quotep{\pi}$

$\freenames{-} : \pi \to \mathcal{P}(\quotep{\pi})$

\begin{eqnarray*}
  \freenames{\pzero} & := & \emptyset \\
  \freenames{x?(y).P} & := & \{ x \} \cup (\freenames{P} \setminus \{ y \}) \\
  \freenames{x!\langle P \rangle} & := & \{ x \} \cup \{ P \} \\
  \freenames{P|Q} & := & \freenames{P} \cup \freenames{Q} \\
  \freenames{\dropn{x}} & := & \{ x \}
\end{eqnarray*}

The bound names of a process, $\boundnames{P}$, are those names occurring in $P$
that are not free. For example, in $x?(y).0$, the name $x$ is free, while $y$ is bound.

\begin{mathpar}
  \inferrule* [lab=monoidal-laws] {} { P|Q \equiv Q|P \and P|0 \equiv P \and P|(Q|R) \equiv (P|Q)|R }
\end{mathpar}

\begin{mathpar}
  \inferrule* [lab=alpha-equivalence] {} { (x)P \equiv (y)P\{y/x\} \and y \not\in \freenames{P} }
\end{mathpar}

\begin{definition}
Then two processes, $P,Q$, are alpha-equivalent if $P = Q\{\vec{y}/\vec{x}\}$ for
some $\vec{x} \in \boundnames{Q},\vec{y} \in \boundnames{P}$, where $Q\{\vec{y}/\vec{x}\}$
denotes the capture-avoiding substitution of $\vec{y}$ for $\vec{x}$ in $Q$.
\end{definition}

\begin{definition}
  The {\em structural congruence} \cite{SangiorgiWalker} , $\equiv$,
  between processes is the least congruence containing
  alpha-equivalence, satisfying the abelian monoid laws
  (associativity, commutativity and $\pzero$ as identity) for parallel
  composition $|$ and for summation $+$.
\end{definition}

\subsection{Name equivalence}

We take name equivalence, written $\nameeq$, to be the smallest
equivalence relation generated by the following rules.

\begin{mathpar}
\inferrule*[lab=Quote-drop]
{ }
{ \quotep{@{x}} \nameeq x }

\inferrule*[lab=Struct-equiv]
{ P \scong Q }
{ \quotep{P} \nameeq \quotep{Q} }
\end{mathpar}

The astute reader will have noticed that the mutual recursion of names
and processes imposes a mutual recursion on alpha-equivalence and
structural equivalence via name-equivalence. Fortunately, all of this
works out pleasantly and we may calculate in the natural way, free of
concern. The reader interested in the details is referred to the
appendix \ref{appendix:rho_details}.

\subsection{Substitution}

We use $\Proc$ for the set of processes, $\QProc$ for the set of
names, and $\id{\{}\vec{y} / \vec{x} \id{\}}$ to denote partial maps,
$s : \QProc \rightarrow \QProc$. A map, $s$ lifts, uniquely, to a map
on process terms, $\widehat{s} : \Proc \rightarrow \Proc$ by the
following equations.

\begin{mathpar}
  (0) \psubstp{Q}{P} := 0 \\
  (R \juxtap S) \psubstp{Q}{P}
  :=    
  (R)\psubstp{Q}{P} \juxtap (S) \psubstp{Q}{P} \\
  (x?(y).R) \psubstp{Q}{P}    
  :=    
  (x)\substp{Q}{P} (z)\concat( (R \psubstn{z}{y}) \psubstp{Q}{P} ) \\
  (\lift{x}{R}) \psubstp{Q}{P}  
  :=
  \lift{(x)\substp{Q}{P}}{ R \psubstp{Q}{P} } \\
%   (\dropn{x})  \psubstp{Q}{P}       
%   := 
%   \left\{ 
%     \begin{array}{ccc} 
%       \dropn{\quotep{Q}} & & x \nameeq \quotep{P} \\
%       \dropn{x} & & otherwise \\
%     \end{array}
%   \right. 
  (\dropn{x})  \psubstp{Q}{P}       
  := 
  \left\{ 
    \begin{array}{ccc} 
      Q & & x \nameeq \quotep{P} \\
      \dropn{x} & & otherwise \\
    \end{array}
  \right.
\end{mathpar}
 

where

\begin{eqnarray}
  (x)\id{\{} \lpquote Q \rpquote / \lpquote P \rpquote \id{\}}            = 
  \left\{ 
    \begin{array}{ccc}
      \lpquote Q \rpquote & & x \nameeq \lpquote P \rpquote \\
      x & & otherwise \\
    \end{array}
  \right. \nonumber
\end{eqnarray}

and $z$ is chosen distinct from $\quotep{P}$, $\quotep{Q}$, the free
names in $Q$, and all the names in $R$. Our $\alpha$-equivalence will
be built in the standard way from this substitution.

\begin{remark}\label{rem:no_self_referential_names}
  One consequence of these definitions is that $\forall P. \quotep{P}
  \not\in \freenames{P}$.
\end{remark}

\subsection{ Dynamic quote: an example }

Anticipating something of what's to come, consider applying the
substitution, $\widehat{\id{\{}u / z \id{\}}}$, to the following pair
of processes, $\lift{w}{y!(z)}$ and $w[ \lpquote y!(z) \rpquote ]$.

\begin{eqnarray}
	\lift{w}{y!(z)}\widehat{\id{\{}u / z \id{\}}}
		& = &
		\lift{w}{y!(u)} \nonumber\\
	w[ \lpquote y!(z) \rpquote ] \widehat{ \id{\{}u / z \id{\}} }
		& = &
		w[ \lpquote y!(z) \rpquote ] \nonumber
\end{eqnarray}

Because the body of the process between quotes is impervious to
substitution, we get radically different answers. In fact, by
examining the first process in an input context,
e.g. $x?(z).\lift{w}{y!(z)}$, we see that the process under the lift
operator may be shaped by prefixed inputs binding a name inside it. In
this sense, the lift operator will be seen as a way to dynamically
construct processes before reifying them as names.

Finally equipped with these standard features we can present the
dynamics of the calculus.

\subsubsection{Operational semantics} 

Finally, we introduce the computational dynamics. What marks these
algebras as distinct from other more traditionally studied algebraic
structures, e.g. vector spaces or polynomial rings, is the manner in
which dynamics is captured. In traditional structures, dynamics is typically
expressed through morphisms between such structures, as in linear maps
between vector spaces or morphisms between rings. In algebras
associated with the semantics of computation, the dynamics is
expressed as part of the algebraic structure itself, through a
reduction reduction relation typically denoted by $\red$. Below, we
give a recursive presentation of this relation for the calculus used
in the encoding.

$\red \subseteq \pi \times \pi$
$\red : \pi \to \mathcal{P}(\pi)$

\begin{mathpar}
  \inferrule* [lab=Comm] { \textsf{match}( x_{src}, x_{trgt} ) } { x_{trgt}?(y)P \; | \; x_{src}!\langle {Q} \rangle \red P\{\quotep{Q}/y}\} }
  \and \\
  \inferrule* [lab=Par] {{P} \red {P}'} {{{P} | {Q}} \red {{P}' | {Q}}}
  \and
  \inferrule* [lab=Equiv]{{{P} \scong {P}'} \andalso {{P}' \red {Q}'} \andalso {{Q}' \scong {Q}}}{{P} \red {Q}}
\end{mathpar}

\begin{eqnarray*}
  match_{\equiv} (\quotep{P},\quotep{Q}) & := & P \equiv Q \\
  match_{\dagger}(\quotep{P},\quotep{Q}) & := & \forall R. P|Q \red^{*} R => R \red^{*} 0 \\
  match_{K}(\quotep{P},\quotep{Q}) & := & K \mbox{ for some context } K
\end{eqnarray*}

$u?(x)P | u!\langle Q \rangle \red P\{\quotep{Q}/x\}$

%We write $\wred$ for $\red^*$, and $P\red$ if $\exists Q $ such that $ P \red Q$.
We write $P\red$ if $\exists Q $ such that $ P \red Q$ and $P\not\red$, otherwise.

\section{Replication}

As mentioned before, it is known that replication (and hence
recursion) can be implemented in a higher-order process algebra
\cite{SangiorgiWalker}. As our first example of calculation with the
machinery thus far presented we give the construction explicitly in
the {\rhoc}.

\begin{eqnarray}
	D_{x} & := & \prefix{x}{y}{(\binpar{\outputp{x}{y}}{@{y}})} \nonumber\\
	\bangp_{x}{P} & := & \binpar{{x}!\langle{\binpar{D_{x}}{P}}\rangle}{D_{x}} \nonumber
\end{eqnarray}

\begin{eqnarray}
	\bangp_{x}{P} & & \nonumber\\
	=
	& {x}!\langle{(\prefix{x}{y}{(\outputp{x}{y} | @{y})) | P}}\rangle 
	      | \prefix{x}{y}{(\outputp{x}{y} | @{y})} & \nonumber\\
	\red
	& (\outputp{x}{y} | @{y})\substn{\quotep{(\prefix{x}{y}{(@{y} | \outputp{x}{y})) | P}}}{y} & \nonumber\\
	=
	& \outputp{x}{\quotep{(\prefix{x}{y}{(\outputp{x}{y} | @{y})) | P}}}
	  | {(\prefix{x}{y}{(\outputp{x}{y} | @{y})) | P}} & \nonumber\\
	\red
	& \ldots & \nonumber\\
	\red^*
	& P | P | \ldots & \nonumber
\end{eqnarray}

Of course, this encoding, as an implementation, runs away, unfolding
$\bangp{P}$ eagerly. A lazier and more implementable replication
operator, restricted to input-guarded processes, may be obtained as follows.

\begin{eqnarray}
\bangp{\prefix{u}{v}{P}} 
	:= 
	\binpar{\lift{x}{\prefix{u}{v}{(\binpar{D(x)}{P})}}}{D(x)} \nonumber
\end{eqnarray}

\begin{remark}
  Note that the lazier definition still does not deal with summation
  or mixed summation (i.e. sums over input and output). The reader is
  invited to construct definitions of replication that deal with these
  features. 

  Further, the definitions are parameterized in a name, $x$. Can you,
  gentle reader, make a definition that eliminates this parameter and
  guarantees no accidental interaction between the replication
  machinery and the process being replicated -- i.e. no accidental
  sharing of names used by the process to get its work done and the
  name(s) used by the replication to effect copying. This latter
  revision of the definition of replication is crucial to obtaining
  the expected identity $!!P \sim !P$.
\end{remark}

\begin{remark}\label{rem:paradoxical_combinator}
  The reader familiar with the lambda calculus will have noticed the
  similarity between $D$ and the paradoxical combinator.

  [Ed. note: the existence of this seems to suggest we have to be more
  restrictive on the set of processes and names we admit if we are to
  support no-cloning.]
\end{remark}

\subsubsection{Bisimulation}

The computational dynamics gives rise to another kind of equivalence,
the equivalence of computational behavior. As previously mentioned
this is typically captured \emph{via} some form of bisimulation.

% The notion we use in this paper is weak barbed bisimulation
% \cite{milner91polyadicpi}.

The notion we use in this paper is derived from weak barbed
bisimulation \cite{milner91polyadicpi}. 

\begin{definition}
An \emph{observation relation}, $\downarrow_{\mathcal N}$, over a set
of names, $\mathcal N$, is the smallest relation satisfying the rules
below.

\infrule[Out-barb]{y \in {\mathcal N}, \; x \nameeq y}
		  {\outputp{x}{v} \downarrow_{\mathcal N} x}
\infrule[Par-barb]{\mbox{$P\downarrow_{\mathcal N} x$ or $Q\downarrow_{\mathcal N} x$}}
		  {\binpar{P}{Q} \downarrow_{\mathcal N} x}

We write $P \Downarrow_{\mathcal N} x$ if there is $Q$ such that 
$P \wred Q$ and $Q \downarrow_{\mathcal N} x$.
\end{definition}

\begin{definition}
%\label{def.bbisim}
An  ${\mathcal N}$-\emph{barbed bisimulation} over a set of names, ${\mathcal N}$, is a symmetric binary relation 
${\mathcal S}_{\mathcal N}$ between agents such that $P\rel{S}_{\mathcal N}Q$ implies:
\begin{enumerate}
\item If $P \red P'$ then $Q \wred Q'$ and $P'\rel{S}_{\mathcal N} Q'$.
\item If $P\downarrow_{\mathcal N} x$, then $Q\Downarrow_{\mathcal N} x$.
\end{enumerate}
$P$ is ${\mathcal N}$-barbed bisimilar to $Q$, written
$P \wbbisim_{\mathcal N} Q$, if $P \rel{S}_{\mathcal N} Q$ for some ${\mathcal N}$-barbed bisimulation ${\mathcal S}_{\mathcal N}$.
\end{definition}

$\mathcal{R} \subseteq \pi \times \pi$

$P \mathcal{R} Q => \forall P'. P \red P' \Rightarrow \exists Q'. Q \red Q', P' \mathcal{R} Q'$

$P \vdash x \Rightarrow Q \vdash x$

\begin{mathpar}
  \inferrule*[lab=Out-barb]{x \nameeq y}{{y}!\langle{Q}\rangle \vdash x}
  \and
  \inferrule*[lab=Par-barb]{\mbox{$P\vdash x$ or $Q\vdash x$}}{\binpar{P}{Q} \vdash x}
\end{mathpar}

\subsubsection{Contexts}

One of the principle advantages of computational calculi like the
$\pi$-calculus is a well-defined notion of context,
contextual-equivalence and a correlation between
contextual-equivalence and notions of bisimulation. The notion of
context allows the decomposition of a process into (sub-)process and
its syntactic environment, its context. Thus, a context may be
thought of as a process with a ``hole'' (written $\Box$) in it. The
application of a context $M$ to a process $P$, written $M[P]$, is
tantamount to filling the hole in $M$ with $P$. In this paper we do
not need the full weight of this theory, but do make use of the notion
of context in the proof the main theorem. 

\begin{mathpar}
  \inferrule* [lab=summation] {} {{M_{M},M_{N}} \bc \Box \;|\; x.M_{A} \;|\; M_{M}+M_{N}}
  \and
  \inferrule* [lab=agent] {} {{M_{A}} \bc (\vec{x})M_{P} \;| \; \clift{P_0,\ldots,M_{P},\ldots,P_N}}
  \and \\
  \inferrule* [lab=process] {} {{M_{P}} \bc M_{N} \;| \;P|M_{P} }
\end{mathpar} 

\begin{mathpar}
  \inferrule* [lab=sychronization] {} {M_{N} \bc \Box \;|\; x?M_{F} \;|\; x!M_{C}}
  \and
  \inferrule* [lab=abstraction] {} {{M_{F}} \bc (x)M_{P} }
  \and
  \inferrule* [lab=concretion] {} {{M_{C}} \bc \langle M_{P} \rangle }
  \and \\
  \inferrule* [lab=process] {} {{M_{P}} \bc M_{N} \;| \;P|M_{P} }
\end{mathpar}

\begin{definition}[contextual application] Given a context $M$, and
  process $P$, we define the \emph{contextual application}, $M[P] :=
  M\{P/\Box\}$. That is, the contextual application of M to P is the
  substitution of $P$ for $\Box$ in $M$.
\end{definition}

$\meaningof{-} : L \to \mathcal{P}(\pi)$

\begin{mathpar}
  \inferrule* [lab=collection] {} {\meaningof{true} = \pi, \and \meaningof{~E} = \pi \setminus \meaningof{E}, \and \meaningof{E_{1} \& E_{2}} = \meaningof{E_{1}} \cap \meaningof{E_{2}}}
\end{mathpar}

\begin{mathpar}
  \inferrule* [lab=structure] {} {\meaningof{0} = \{ P \in \pi | P \equiv 0 \}, \and \\ \meaningof{E_1 | E_2} = \{ P \in \pi | P \equiv P_{1} | P_{2}, P_{1} \in \meaningof{E_{1}}, P_{2} \in \meaningof{E_2}\} }
\end{mathpar}

\begin{mathpar}
 \inferrule* [lab=behavior] {} {\meaningof{\langle a?b \rangle E} = \{ P \in \pi | P \equiv Q | u?(y)P', \\ \and \\\\ \and \\ \;\;\; u \in \meaningof{a}, \forall z.P'\{z/y\} \in \meaningof{E\{z/b\}}\}, \and \\ \meaningof{a!E} = \{ P \in \pi | P \equiv Q | x!\langle P' \rangle, x \in \meaningof{a} P' \in \meaningof{E}\} }
\end{mathpar}

\begin{mathpar}
 \inferrule* [lab=nominal] {} {\meaningof{\quotep{E}} = \{ \quotep{P} \in \quotep{\pi} | P \in \meaningof{E} \}, \and \meaningof{\quotep{P}} = \{ \quotep{Q} \in \quotep{\pi} | P \equiv Q \} \and \\ \meaningof{@\quotep{E}} = \{ P \in \pi | P \equiv @x, x \in \meaningof{E} \}}
\end{mathpar}

\begin{eqnarray*}
  \\
  \meaningof{-} : TS \to ST
\end{eqnarray*}

\begin{eqnarray*}
  \\
  L : TS \to ST
\end{eqnarray*}

\begin{eqnarray*}
  \\
  P \models E \iff P \in \meaningof{E}
\end{eqnarray*}

\begin{eqnarray*}
  P \approx_{L} Q \iff \forall E \in L. P \models E \iff Q \models E
\end{eqnarray*}

\begin{eqnarray*}
  P \approx_{K} Q
\end{eqnarray*}

\begin{eqnarray*}
  P \approx Q
\end{eqnarray*}

$\approx_{K} = \approx = \approx_{L}$

\subsubsection{Contextual duality}

Note that contexts extend the quotation operation to a family of
operations from processes to names. Given a context, $M$, we can
define a \emph{nominal context}, $\quotep{M}$ by $\quotep{M}[P] :=
\quotep{M[P]}$. To foreshadow what is to come we observe that these
operations enjoy a duality with processes very much like the duality
between vectors and maps from vectors to scalars.

Further, because the calculus is essentially higher-order, we have a
correspondence between contexts and processes. More specifically,
given a name $x$ and a context $M$ we can construct $M^{*}_{x}$ such
that 

\begin{mathpar}
  M^{*}_{x} | \lift{x}{P} \red M[P]
\end{mathpar}

namely,

\begin{mathpar}
  M^{*}_{x} := x?(u).M[\dropn{u}]
\end{mathpar}

The dependence of $M^{*}_{x}$ on a name makes it an abstraction, 

\begin{mathpar}
  M^{*} := (x)x?(u).M[\dropn{u}]
\end{mathpar}

\subsection{Additional notation}

It will sometimes be convenient to denote the process a name
quotes. We already have the notation $x = \quotep{P}$, but it will be
convenient to introduce an alternate notation, $\procn{x}$, when we
want to emphasize the connection to the use of the name. Note that, by
virtue of name equivalence, $\quotep{\procn{x}} \nameeq x$; so, the
notation is consistent with previous definitions.

Further, because names have structure it is possible to effect
substitutions on the basis of that structure. This means we need to
upgrade our notation for substitutions, which we accomplish by
adapting comprehension notation. Thus,

\begin{mathpar}
  P\{ y / x : x \in S \}
\end{mathpar}

is interpreted to mean the process derived from P by replacing (in a
capture-avoiding manner) each occurrence of $x$ in $S$ by $y$. For example,

\begin{mathpar}
  P\{ \quotep{\procn{x}|\procn{x}} / x : x \in \freenames{P} \}
\end{mathpar}

will replace each (occurrence) of a free name $x$ in $P$ by
$\quotep{\procn{x}|\procn{x}}$.

Also, we will avail ourselves of the notation $x^{L}$ and $x^{R}$ to
denote injections of a name into disjoint copies of the name
space. There are numerous ways to accomplish this. One example can be
found in \cite{MeredithR05}. This notation overloads to vectors of
names: $\vec{x}^{\pi} := (x_{i}^{\pi} \; : \; 0 \leq i < |\vec{x}| )$ where $\pi \in \{L,R\}$.

We also use $P^{\Box} := P|\Box$.

In \cite{MeredithR05} an interpretation of the new operator is
given. It turns out that there are several possible interpretations
all enjoying the requisite algebraic properties of the operator (see
\cite{milner91polyadicpi}). We will therefore make liberal use of
$(\nu\; \vec{x})P$.

% subsection the_syntax_and_semantics_of_the_notation_system (end)   

\input{qm2pi.qmops} 

\input{qm2pi.sterngerlach} 

\input{qm2pi.metric} 

% section concurrent_process_calculi (end)

%\input{qm2pi.proofsketch}

% section proof sketch (end)

%\input{qm2pi.slviaknots} 

% section spatial logic via knots (end)

\input{qm2pi.conclusion}

% section conclusion (end)

%\input{qm2pi.dtcodes} 

% section wiring algorithm (end)

\input{qm2pi.ack} 

% section acknowledgments (end)

\newpage


\bibliographystyle{plain}   
\bibliography{../../biblios/main.bib}

\input{qm2pi.rhodetails}

\end{document}

 

\documentclass[12pt]{llncs}
%\documentclass{jktr}

\usepackage[pdftex]{hyperref}                   
\usepackage {listings}
\usepackage {mathpartir}
\usepackage{bcprules}
%\usepackage{listings}
                       
\usepackage{graphicx} 
%\usepackage[margins=2.5cm,nohead,nofoot]{geometry}
%\usepackage{geometry}
\usepackage{amsfonts}
\usepackage{amstext}
\usepackage{latexsym}
\usepackage{amssymb}
\usepackage{color}


%\include{myPreamble}
\include{qm2pi.local} 

%\ifpdf
%\usepackage[pdftex]{graphicx}
%\else
%\usepackage{graphicx}
%\fi

 % \ifpdf
%  \usepackage{pdfsync}
%  \if


%\title{Brief Article}
%\author{David F. Snyder}
%\author{L.G. Meredith}

%\address{Dept. of Math., Texas State University--San Marcos, San Marcos, TX 78666}
       
\pagestyle{empty}


\begin{document}

\lstset{language=[Objective]Caml,frame=shadowbox}

\input{qm2pi.front}

% section front matter (end)

\input{qm2pi.intro} 
 
% section introduction (end)

% \input{qm2pi.knotations} 

% section notation (end)

\input{qm2pi.process.calculi} 

% section concurrent_process_calculi_and_spatial_logics_ (end)
    
%\input{qm2pi.knots2pi} 

%\input{qm2pi.trefoil} 

%\input{qm2pi.mainthm} 

% subsection basic_interpretation (end)

%\input{qm2pi.rho.presentation} 
\subsection{The syntax and semantics of the notation system}\label{sub:the_syntax_and_semantics_of_the_notation_system} % (fold)

We now summarize a technical presentation of the calculus that
embodies our theory of dynamics. The typical presentation of such a
calculus follows the style of giving generators and relations on
them. The grammar, below, describing term constructors, freely
generates the set of processes, $\Proc$. This set is then quotiented
by a relation known as structural congruence and it is over this set
that the notion of dynamics is expressed. This presentation is
essentially that of \cite{MeredithR05} with the addition of
polyadicity and summation. For readability we have relegated some of
the technical subtleties to an appendix.

\subsubsection{Process grammar}\label{subsub:process_grammar}

\begin{mathpar}
  \inferrule* [lab=synchronization] {} {{M} \bc \pzero \;|\; x?F \;|\; x!C }
  \and
  \inferrule* [lab=abstraction] {} {{F} \bc (x)P}
  \and
  \inferrule* [lab=concretion] {} {{C} \bc \langle Q \rangle}
  \and
  \inferrule* [lab=process] {} {{P,Q} \bc M \;| \;P|Q \;|\; @{x}}
  \and
  \inferrule* [lab=name] {} {{x} \bc \quotep{P}}
\end{mathpar} 

Note that $\vec{x}$ (resp. $\vec{P}$) denotes a vector of names
(resp. processes) of length $|\vec{x}|$ (resp. $|\vec{P}|$). We adopt
the following useful abbreviations.

\begin{mathpar}
   x?(\vec{y}).P := x.(\vec{y})P \and  x\clift{\vec{P}} := x.\clift{\vec{P}}
   \and x!(y) := \lift{x}{\dropn{y}}
   \and \Pi_{i=0}^{n-1}P_i := P_0 | \ldots | P_{n-1}
\end{mathpar}

\subsubsection{Structural congruence}

\paragraph{Free and bound names and alpha-equivalence.} At the
core of structural equivalence is alpha-equivalence which identifies
process that are the same up to a change of variable. Formally, we
recognize the distinction between free and bound names. The free names
of a process, $\freenames{P}$, may be calculated recursively as
follows:

\begin{mathpar}
\freenames{\pzero} := \emptyset
  \and \\
  \freenames{x?(y).P} := \{ x \} \cup (\freenames{P} \setminus \{ y \})
  \and 
  \freenames{x!\langle P \rangle} := \{ x \} \cup \{ P \} 
  \and \\
  \freenames{P|Q} := \freenames{P} \cup \freenames{Q}
  \and \\
  \freenames{@{x}} := \{ x \}
\end{mathpar}

$\pi$
$\quotep{\pi}$

$\freenames{-} : \pi \to \mathcal{P}(\quotep{\pi})$

\begin{eqnarray*}
  \freenames{\pzero} & := & \emptyset \\
  \freenames{x?(y).P} & := & \{ x \} \cup (\freenames{P} \setminus \{ y \}) \\
  \freenames{x!\langle P \rangle} & := & \{ x \} \cup \{ P \} \\
  \freenames{P|Q} & := & \freenames{P} \cup \freenames{Q} \\
  \freenames{\dropn{x}} & := & \{ x \}
\end{eqnarray*}

The bound names of a process, $\boundnames{P}$, are those names occurring in $P$
that are not free. For example, in $x?(y).0$, the name $x$ is free, while $y$ is bound.

\begin{mathpar}
  \inferrule* [lab=monoidal-laws] {} { P|Q \equiv Q|P \and P|0 \equiv P \and P|(Q|R) \equiv (P|Q)|R }
\end{mathpar}

\begin{mathpar}
  \inferrule* [lab=alpha-equivalence] {} { (x)P \equiv (y)P\{y/x\} \and y \not\in \freenames{P} }
\end{mathpar}

\begin{definition}
Then two processes, $P,Q$, are alpha-equivalent if $P = Q\{\vec{y}/\vec{x}\}$ for
some $\vec{x} \in \boundnames{Q},\vec{y} \in \boundnames{P}$, where $Q\{\vec{y}/\vec{x}\}$
denotes the capture-avoiding substitution of $\vec{y}$ for $\vec{x}$ in $Q$.
\end{definition}

\begin{definition}
  The {\em structural congruence} \cite{SangiorgiWalker} , $\equiv$,
  between processes is the least congruence containing
  alpha-equivalence, satisfying the abelian monoid laws
  (associativity, commutativity and $\pzero$ as identity) for parallel
  composition $|$ and for summation $+$.
\end{definition}

\subsection{Name equivalence}

We take name equivalence, written $\nameeq$, to be the smallest
equivalence relation generated by the following rules.

\begin{mathpar}
\inferrule*[lab=Quote-drop]
{ }
{ \quotep{@{x}} \nameeq x }

\inferrule*[lab=Struct-equiv]
{ P \scong Q }
{ \quotep{P} \nameeq \quotep{Q} }
\end{mathpar}

The astute reader will have noticed that the mutual recursion of names
and processes imposes a mutual recursion on alpha-equivalence and
structural equivalence via name-equivalence. Fortunately, all of this
works out pleasantly and we may calculate in the natural way, free of
concern. The reader interested in the details is referred to the
appendix \ref{appendix:rho_details}.

\subsection{Substitution}

We use $\Proc$ for the set of processes, $\QProc$ for the set of
names, and $\id{\{}\vec{y} / \vec{x} \id{\}}$ to denote partial maps,
$s : \QProc \rightarrow \QProc$. A map, $s$ lifts, uniquely, to a map
on process terms, $\widehat{s} : \Proc \rightarrow \Proc$ by the
following equations.

\begin{mathpar}
  (0) \psubstp{Q}{P} := 0 \\
  (R \juxtap S) \psubstp{Q}{P}
  :=    
  (R)\psubstp{Q}{P} \juxtap (S) \psubstp{Q}{P} \\
  (x?(y).R) \psubstp{Q}{P}    
  :=    
  (x)\substp{Q}{P} (z)\concat( (R \psubstn{z}{y}) \psubstp{Q}{P} ) \\
  (\lift{x}{R}) \psubstp{Q}{P}  
  :=
  \lift{(x)\substp{Q}{P}}{ R \psubstp{Q}{P} } \\
%   (\dropn{x})  \psubstp{Q}{P}       
%   := 
%   \left\{ 
%     \begin{array}{ccc} 
%       \dropn{\quotep{Q}} & & x \nameeq \quotep{P} \\
%       \dropn{x} & & otherwise \\
%     \end{array}
%   \right. 
  (\dropn{x})  \psubstp{Q}{P}       
  := 
  \left\{ 
    \begin{array}{ccc} 
      Q & & x \nameeq \quotep{P} \\
      \dropn{x} & & otherwise \\
    \end{array}
  \right.
\end{mathpar}
 

where

\begin{eqnarray}
  (x)\id{\{} \lpquote Q \rpquote / \lpquote P \rpquote \id{\}}            = 
  \left\{ 
    \begin{array}{ccc}
      \lpquote Q \rpquote & & x \nameeq \lpquote P \rpquote \\
      x & & otherwise \\
    \end{array}
  \right. \nonumber
\end{eqnarray}

and $z$ is chosen distinct from $\quotep{P}$, $\quotep{Q}$, the free
names in $Q$, and all the names in $R$. Our $\alpha$-equivalence will
be built in the standard way from this substitution.

\begin{remark}\label{rem:no_self_referential_names}
  One consequence of these definitions is that $\forall P. \quotep{P}
  \not\in \freenames{P}$.
\end{remark}

\subsection{ Dynamic quote: an example }

Anticipating something of what's to come, consider applying the
substitution, $\widehat{\id{\{}u / z \id{\}}}$, to the following pair
of processes, $\lift{w}{y!(z)}$ and $w[ \lpquote y!(z) \rpquote ]$.

\begin{eqnarray}
	\lift{w}{y!(z)}\widehat{\id{\{}u / z \id{\}}}
		& = &
		\lift{w}{y!(u)} \nonumber\\
	w[ \lpquote y!(z) \rpquote ] \widehat{ \id{\{}u / z \id{\}} }
		& = &
		w[ \lpquote y!(z) \rpquote ] \nonumber
\end{eqnarray}

Because the body of the process between quotes is impervious to
substitution, we get radically different answers. In fact, by
examining the first process in an input context,
e.g. $x?(z).\lift{w}{y!(z)}$, we see that the process under the lift
operator may be shaped by prefixed inputs binding a name inside it. In
this sense, the lift operator will be seen as a way to dynamically
construct processes before reifying them as names.

Finally equipped with these standard features we can present the
dynamics of the calculus.

\subsubsection{Operational semantics} 

Finally, we introduce the computational dynamics. What marks these
algebras as distinct from other more traditionally studied algebraic
structures, e.g. vector spaces or polynomial rings, is the manner in
which dynamics is captured. In traditional structures, dynamics is typically
expressed through morphisms between such structures, as in linear maps
between vector spaces or morphisms between rings. In algebras
associated with the semantics of computation, the dynamics is
expressed as part of the algebraic structure itself, through a
reduction reduction relation typically denoted by $\red$. Below, we
give a recursive presentation of this relation for the calculus used
in the encoding.

$\red \subseteq \pi \times \pi$
$\red : \pi \to \mathcal{P}(\pi)$

\begin{mathpar}
  \inferrule* [lab=Comm] { \textsf{match}( x_{src}, x_{trgt} ) } { x_{trgt}?(y)P \; | \; x_{src}!\langle {Q} \rangle \red P\{\quotep{Q}/y}\} }
  \and \\
  \inferrule* [lab=Par] {{P} \red {P}'} {{{P} | {Q}} \red {{P}' | {Q}}}
  \and
  \inferrule* [lab=Equiv]{{{P} \scong {P}'} \andalso {{P}' \red {Q}'} \andalso {{Q}' \scong {Q}}}{{P} \red {Q}}
\end{mathpar}

\begin{eqnarray*}
  match_{\equiv} (\quotep{P},\quotep{Q}) & := & P \equiv Q \\
  match_{\dagger}(\quotep{P},\quotep{Q}) & := & \forall R. P|Q \red^{*} R => R \red^{*} 0 \\
  match_{K}(\quotep{P},\quotep{Q}) & := & K \mbox{ for some context } K
\end{eqnarray*}

$u?(x)P | u!\langle Q \rangle \red P\{\quotep{Q}/x\}$

%We write $\wred$ for $\red^*$, and $P\red$ if $\exists Q $ such that $ P \red Q$.
We write $P\red$ if $\exists Q $ such that $ P \red Q$ and $P\not\red$, otherwise.

\section{Replication}

As mentioned before, it is known that replication (and hence
recursion) can be implemented in a higher-order process algebra
\cite{SangiorgiWalker}. As our first example of calculation with the
machinery thus far presented we give the construction explicitly in
the {\rhoc}.

\begin{eqnarray}
	D_{x} & := & \prefix{x}{y}{(\binpar{\outputp{x}{y}}{@{y}})} \nonumber\\
	\bangp_{x}{P} & := & \binpar{{x}!\langle{\binpar{D_{x}}{P}}\rangle}{D_{x}} \nonumber
\end{eqnarray}

\begin{eqnarray}
	\bangp_{x}{P} & & \nonumber\\
	=
	& {x}!\langle{(\prefix{x}{y}{(\outputp{x}{y} | @{y})) | P}}\rangle 
	      | \prefix{x}{y}{(\outputp{x}{y} | @{y})} & \nonumber\\
	\red
	& (\outputp{x}{y} | @{y})\substn{\quotep{(\prefix{x}{y}{(@{y} | \outputp{x}{y})) | P}}}{y} & \nonumber\\
	=
	& \outputp{x}{\quotep{(\prefix{x}{y}{(\outputp{x}{y} | @{y})) | P}}}
	  | {(\prefix{x}{y}{(\outputp{x}{y} | @{y})) | P}} & \nonumber\\
	\red
	& \ldots & \nonumber\\
	\red^*
	& P | P | \ldots & \nonumber
\end{eqnarray}

Of course, this encoding, as an implementation, runs away, unfolding
$\bangp{P}$ eagerly. A lazier and more implementable replication
operator, restricted to input-guarded processes, may be obtained as follows.

\begin{eqnarray}
\bangp{\prefix{u}{v}{P}} 
	:= 
	\binpar{\lift{x}{\prefix{u}{v}{(\binpar{D(x)}{P})}}}{D(x)} \nonumber
\end{eqnarray}

\begin{remark}
  Note that the lazier definition still does not deal with summation
  or mixed summation (i.e. sums over input and output). The reader is
  invited to construct definitions of replication that deal with these
  features. 

  Further, the definitions are parameterized in a name, $x$. Can you,
  gentle reader, make a definition that eliminates this parameter and
  guarantees no accidental interaction between the replication
  machinery and the process being replicated -- i.e. no accidental
  sharing of names used by the process to get its work done and the
  name(s) used by the replication to effect copying. This latter
  revision of the definition of replication is crucial to obtaining
  the expected identity $!!P \sim !P$.
\end{remark}

\begin{remark}\label{rem:paradoxical_combinator}
  The reader familiar with the lambda calculus will have noticed the
  similarity between $D$ and the paradoxical combinator.

  [Ed. note: the existence of this seems to suggest we have to be more
  restrictive on the set of processes and names we admit if we are to
  support no-cloning.]
\end{remark}

\subsubsection{Bisimulation}

The computational dynamics gives rise to another kind of equivalence,
the equivalence of computational behavior. As previously mentioned
this is typically captured \emph{via} some form of bisimulation.

% The notion we use in this paper is weak barbed bisimulation
% \cite{milner91polyadicpi}.

The notion we use in this paper is derived from weak barbed
bisimulation \cite{milner91polyadicpi}. 

\begin{definition}
An \emph{observation relation}, $\downarrow_{\mathcal N}$, over a set
of names, $\mathcal N$, is the smallest relation satisfying the rules
below.

\infrule[Out-barb]{y \in {\mathcal N}, \; x \nameeq y}
		  {\outputp{x}{v} \downarrow_{\mathcal N} x}
\infrule[Par-barb]{\mbox{$P\downarrow_{\mathcal N} x$ or $Q\downarrow_{\mathcal N} x$}}
		  {\binpar{P}{Q} \downarrow_{\mathcal N} x}

We write $P \Downarrow_{\mathcal N} x$ if there is $Q$ such that 
$P \wred Q$ and $Q \downarrow_{\mathcal N} x$.
\end{definition}

\begin{definition}
%\label{def.bbisim}
An  ${\mathcal N}$-\emph{barbed bisimulation} over a set of names, ${\mathcal N}$, is a symmetric binary relation 
${\mathcal S}_{\mathcal N}$ between agents such that $P\rel{S}_{\mathcal N}Q$ implies:
\begin{enumerate}
\item If $P \red P'$ then $Q \wred Q'$ and $P'\rel{S}_{\mathcal N} Q'$.
\item If $P\downarrow_{\mathcal N} x$, then $Q\Downarrow_{\mathcal N} x$.
\end{enumerate}
$P$ is ${\mathcal N}$-barbed bisimilar to $Q$, written
$P \wbbisim_{\mathcal N} Q$, if $P \rel{S}_{\mathcal N} Q$ for some ${\mathcal N}$-barbed bisimulation ${\mathcal S}_{\mathcal N}$.
\end{definition}

$\mathcal{R} \subseteq \pi \times \pi$

$P \mathcal{R} Q => \forall P'. P \red P' \Rightarrow \exists Q'. Q \red Q', P' \mathcal{R} Q'$

$P \vdash x \Rightarrow Q \vdash x$

\begin{mathpar}
  \inferrule*[lab=Out-barb]{x \nameeq y}{{y}!\langle{Q}\rangle \vdash x}
  \and
  \inferrule*[lab=Par-barb]{\mbox{$P\vdash x$ or $Q\vdash x$}}{\binpar{P}{Q} \vdash x}
\end{mathpar}

\subsubsection{Contexts}

One of the principle advantages of computational calculi like the
$\pi$-calculus is a well-defined notion of context,
contextual-equivalence and a correlation between
contextual-equivalence and notions of bisimulation. The notion of
context allows the decomposition of a process into (sub-)process and
its syntactic environment, its context. Thus, a context may be
thought of as a process with a ``hole'' (written $\Box$) in it. The
application of a context $M$ to a process $P$, written $M[P]$, is
tantamount to filling the hole in $M$ with $P$. In this paper we do
not need the full weight of this theory, but do make use of the notion
of context in the proof the main theorem. 

\begin{mathpar}
  \inferrule* [lab=summation] {} {{M_{M},M_{N}} \bc \Box \;|\; x.M_{A} \;|\; M_{M}+M_{N}}
  \and
  \inferrule* [lab=agent] {} {{M_{A}} \bc (\vec{x})M_{P} \;| \; \clift{P_0,\ldots,M_{P},\ldots,P_N}}
  \and \\
  \inferrule* [lab=process] {} {{M_{P}} \bc M_{N} \;| \;P|M_{P} }
\end{mathpar} 

\begin{mathpar}
  \inferrule* [lab=sychronization] {} {M_{N} \bc \Box \;|\; x?M_{F} \;|\; x!M_{C}}
  \and
  \inferrule* [lab=abstraction] {} {{M_{F}} \bc (x)M_{P} }
  \and
  \inferrule* [lab=concretion] {} {{M_{C}} \bc \langle M_{P} \rangle }
  \and \\
  \inferrule* [lab=process] {} {{M_{P}} \bc M_{N} \;| \;P|M_{P} }
\end{mathpar}

\begin{definition}[contextual application] Given a context $M$, and
  process $P$, we define the \emph{contextual application}, $M[P] :=
  M\{P/\Box\}$. That is, the contextual application of M to P is the
  substitution of $P$ for $\Box$ in $M$.
\end{definition}

$\meaningof{-} : L \to \mathcal{P}(\pi)$

\begin{mathpar}
  \inferrule* [lab=collection] {} {\meaningof{true} = \pi, \and \meaningof{~E} = \pi \setminus \meaningof{E}, \and \meaningof{E_{1} \& E_{2}} = \meaningof{E_{1}} \cap \meaningof{E_{2}}}
\end{mathpar}

\begin{mathpar}
  \inferrule* [lab=structure] {} {\meaningof{0} = \{ P \in \pi | P \equiv 0 \}, \and \\ \meaningof{E_1 | E_2} = \{ P \in \pi | P \equiv P_{1} | P_{2}, P_{1} \in \meaningof{E_{1}}, P_{2} \in \meaningof{E_2}\} }
\end{mathpar}

\begin{mathpar}
 \inferrule* [lab=behavior] {} {\meaningof{\langle a?b \rangle E} = \{ P \in \pi | P \equiv Q | u?(y)P', \\ \and \\\\ \and \\ \;\;\; u \in \meaningof{a}, \forall z.P'\{z/y\} \in \meaningof{E\{z/b\}}\}, \and \\ \meaningof{a!E} = \{ P \in \pi | P \equiv Q | x!\langle P' \rangle, x \in \meaningof{a} P' \in \meaningof{E}\} }
\end{mathpar}

\begin{mathpar}
 \inferrule* [lab=nominal] {} {\meaningof{\quotep{E}} = \{ \quotep{P} \in \quotep{\pi} | P \in \meaningof{E} \}, \and \meaningof{\quotep{P}} = \{ \quotep{Q} \in \quotep{\pi} | P \equiv Q \} \and \\ \meaningof{@\quotep{E}} = \{ P \in \pi | P \equiv @x, x \in \meaningof{E} \}}
\end{mathpar}

\begin{eqnarray*}
  \\
  \meaningof{-} : TS \to ST
\end{eqnarray*}

\begin{eqnarray*}
  \\
  L : TS \to ST
\end{eqnarray*}

\begin{eqnarray*}
  \\
  P \models E \iff P \in \meaningof{E}
\end{eqnarray*}

\begin{eqnarray*}
  P \approx_{L} Q \iff \forall E \in L. P \models E \iff Q \models E
\end{eqnarray*}

\begin{eqnarray*}
  P \approx_{K} Q
\end{eqnarray*}

\begin{eqnarray*}
  P \approx Q
\end{eqnarray*}

$\approx_{K} = \approx = \approx_{L}$

\subsubsection{Contextual duality}

Note that contexts extend the quotation operation to a family of
operations from processes to names. Given a context, $M$, we can
define a \emph{nominal context}, $\quotep{M}$ by $\quotep{M}[P] :=
\quotep{M[P]}$. To foreshadow what is to come we observe that these
operations enjoy a duality with processes very much like the duality
between vectors and maps from vectors to scalars.

Further, because the calculus is essentially higher-order, we have a
correspondence between contexts and processes. More specifically,
given a name $x$ and a context $M$ we can construct $M^{*}_{x}$ such
that 

\begin{mathpar}
  M^{*}_{x} | \lift{x}{P} \red M[P]
\end{mathpar}

namely,

\begin{mathpar}
  M^{*}_{x} := x?(u).M[\dropn{u}]
\end{mathpar}

The dependence of $M^{*}_{x}$ on a name makes it an abstraction, 

\begin{mathpar}
  M^{*} := (x)x?(u).M[\dropn{u}]
\end{mathpar}

\subsection{Additional notation}

It will sometimes be convenient to denote the process a name
quotes. We already have the notation $x = \quotep{P}$, but it will be
convenient to introduce an alternate notation, $\procn{x}$, when we
want to emphasize the connection to the use of the name. Note that, by
virtue of name equivalence, $\quotep{\procn{x}} \nameeq x$; so, the
notation is consistent with previous definitions.

Further, because names have structure it is possible to effect
substitutions on the basis of that structure. This means we need to
upgrade our notation for substitutions, which we accomplish by
adapting comprehension notation. Thus,

\begin{mathpar}
  P\{ y / x : x \in S \}
\end{mathpar}

is interpreted to mean the process derived from P by replacing (in a
capture-avoiding manner) each occurrence of $x$ in $S$ by $y$. For example,

\begin{mathpar}
  P\{ \quotep{\procn{x}|\procn{x}} / x : x \in \freenames{P} \}
\end{mathpar}

will replace each (occurrence) of a free name $x$ in $P$ by
$\quotep{\procn{x}|\procn{x}}$.

Also, we will avail ourselves of the notation $x^{L}$ and $x^{R}$ to
denote injections of a name into disjoint copies of the name
space. There are numerous ways to accomplish this. One example can be
found in \cite{MeredithR05}. This notation overloads to vectors of
names: $\vec{x}^{\pi} := (x_{i}^{\pi} \; : \; 0 \leq i < |\vec{x}| )$ where $\pi \in \{L,R\}$.

We also use $P^{\Box} := P|\Box$.

In \cite{MeredithR05} an interpretation of the new operator is
given. It turns out that there are several possible interpretations
all enjoying the requisite algebraic properties of the operator (see
\cite{milner91polyadicpi}). We will therefore make liberal use of
$(\nu\; \vec{x})P$.

% subsection the_syntax_and_semantics_of_the_notation_system (end)   

\input{qm2pi.qmops} 

\input{qm2pi.sterngerlach} 

\input{qm2pi.metric} 

% section concurrent_process_calculi (end)

%\input{qm2pi.proofsketch}

% section proof sketch (end)

%\input{qm2pi.slviaknots} 

% section spatial logic via knots (end)

\input{qm2pi.conclusion}

% section conclusion (end)

%\input{qm2pi.dtcodes} 

% section wiring algorithm (end)

\input{qm2pi.ack} 

% section acknowledgments (end)

\newpage


\bibliographystyle{plain}   
\bibliography{../../biblios/main.bib}

\input{qm2pi.rhodetails}

\end{document}

 

% section concurrent_process_calculi (end)

%\documentclass[12pt]{llncs}
%\documentclass{jktr}

\usepackage[pdftex]{hyperref}                   
\usepackage {listings}
\usepackage {mathpartir}
\usepackage{bcprules}
%\usepackage{listings}
                       
\usepackage{graphicx} 
%\usepackage[margins=2.5cm,nohead,nofoot]{geometry}
%\usepackage{geometry}
\usepackage{amsfonts}
\usepackage{amstext}
\usepackage{latexsym}
\usepackage{amssymb}
\usepackage{color}


%\include{myPreamble}
\include{qm2pi.local} 

%\ifpdf
%\usepackage[pdftex]{graphicx}
%\else
%\usepackage{graphicx}
%\fi

 % \ifpdf
%  \usepackage{pdfsync}
%  \if


%\title{Brief Article}
%\author{David F. Snyder}
%\author{L.G. Meredith}

%\address{Dept. of Math., Texas State University--San Marcos, San Marcos, TX 78666}
       
\pagestyle{empty}


\begin{document}

\lstset{language=[Objective]Caml,frame=shadowbox}

\input{qm2pi.front}

% section front matter (end)

\input{qm2pi.intro} 
 
% section introduction (end)

% \input{qm2pi.knotations} 

% section notation (end)

\input{qm2pi.process.calculi} 

% section concurrent_process_calculi_and_spatial_logics_ (end)
    
%\input{qm2pi.knots2pi} 

%\input{qm2pi.trefoil} 

%\input{qm2pi.mainthm} 

% subsection basic_interpretation (end)

%\input{qm2pi.rho.presentation} 
\subsection{The syntax and semantics of the notation system}\label{sub:the_syntax_and_semantics_of_the_notation_system} % (fold)

We now summarize a technical presentation of the calculus that
embodies our theory of dynamics. The typical presentation of such a
calculus follows the style of giving generators and relations on
them. The grammar, below, describing term constructors, freely
generates the set of processes, $\Proc$. This set is then quotiented
by a relation known as structural congruence and it is over this set
that the notion of dynamics is expressed. This presentation is
essentially that of \cite{MeredithR05} with the addition of
polyadicity and summation. For readability we have relegated some of
the technical subtleties to an appendix.

\subsubsection{Process grammar}\label{subsub:process_grammar}

\begin{mathpar}
  \inferrule* [lab=synchronization] {} {{M} \bc \pzero \;|\; x?F \;|\; x!C }
  \and
  \inferrule* [lab=abstraction] {} {{F} \bc (x)P}
  \and
  \inferrule* [lab=concretion] {} {{C} \bc \langle Q \rangle}
  \and
  \inferrule* [lab=process] {} {{P,Q} \bc M \;| \;P|Q \;|\; @{x}}
  \and
  \inferrule* [lab=name] {} {{x} \bc \quotep{P}}
\end{mathpar} 

Note that $\vec{x}$ (resp. $\vec{P}$) denotes a vector of names
(resp. processes) of length $|\vec{x}|$ (resp. $|\vec{P}|$). We adopt
the following useful abbreviations.

\begin{mathpar}
   x?(\vec{y}).P := x.(\vec{y})P \and  x\clift{\vec{P}} := x.\clift{\vec{P}}
   \and x!(y) := \lift{x}{\dropn{y}}
   \and \Pi_{i=0}^{n-1}P_i := P_0 | \ldots | P_{n-1}
\end{mathpar}

\subsubsection{Structural congruence}

\paragraph{Free and bound names and alpha-equivalence.} At the
core of structural equivalence is alpha-equivalence which identifies
process that are the same up to a change of variable. Formally, we
recognize the distinction between free and bound names. The free names
of a process, $\freenames{P}$, may be calculated recursively as
follows:

\begin{mathpar}
\freenames{\pzero} := \emptyset
  \and \\
  \freenames{x?(y).P} := \{ x \} \cup (\freenames{P} \setminus \{ y \})
  \and 
  \freenames{x!\langle P \rangle} := \{ x \} \cup \{ P \} 
  \and \\
  \freenames{P|Q} := \freenames{P} \cup \freenames{Q}
  \and \\
  \freenames{@{x}} := \{ x \}
\end{mathpar}

$\pi$
$\quotep{\pi}$

$\freenames{-} : \pi \to \mathcal{P}(\quotep{\pi})$

\begin{eqnarray*}
  \freenames{\pzero} & := & \emptyset \\
  \freenames{x?(y).P} & := & \{ x \} \cup (\freenames{P} \setminus \{ y \}) \\
  \freenames{x!\langle P \rangle} & := & \{ x \} \cup \{ P \} \\
  \freenames{P|Q} & := & \freenames{P} \cup \freenames{Q} \\
  \freenames{\dropn{x}} & := & \{ x \}
\end{eqnarray*}

The bound names of a process, $\boundnames{P}$, are those names occurring in $P$
that are not free. For example, in $x?(y).0$, the name $x$ is free, while $y$ is bound.

\begin{mathpar}
  \inferrule* [lab=monoidal-laws] {} { P|Q \equiv Q|P \and P|0 \equiv P \and P|(Q|R) \equiv (P|Q)|R }
\end{mathpar}

\begin{mathpar}
  \inferrule* [lab=alpha-equivalence] {} { (x)P \equiv (y)P\{y/x\} \and y \not\in \freenames{P} }
\end{mathpar}

\begin{definition}
Then two processes, $P,Q$, are alpha-equivalent if $P = Q\{\vec{y}/\vec{x}\}$ for
some $\vec{x} \in \boundnames{Q},\vec{y} \in \boundnames{P}$, where $Q\{\vec{y}/\vec{x}\}$
denotes the capture-avoiding substitution of $\vec{y}$ for $\vec{x}$ in $Q$.
\end{definition}

\begin{definition}
  The {\em structural congruence} \cite{SangiorgiWalker} , $\equiv$,
  between processes is the least congruence containing
  alpha-equivalence, satisfying the abelian monoid laws
  (associativity, commutativity and $\pzero$ as identity) for parallel
  composition $|$ and for summation $+$.
\end{definition}

\subsection{Name equivalence}

We take name equivalence, written $\nameeq$, to be the smallest
equivalence relation generated by the following rules.

\begin{mathpar}
\inferrule*[lab=Quote-drop]
{ }
{ \quotep{@{x}} \nameeq x }

\inferrule*[lab=Struct-equiv]
{ P \scong Q }
{ \quotep{P} \nameeq \quotep{Q} }
\end{mathpar}

The astute reader will have noticed that the mutual recursion of names
and processes imposes a mutual recursion on alpha-equivalence and
structural equivalence via name-equivalence. Fortunately, all of this
works out pleasantly and we may calculate in the natural way, free of
concern. The reader interested in the details is referred to the
appendix \ref{appendix:rho_details}.

\subsection{Substitution}

We use $\Proc$ for the set of processes, $\QProc$ for the set of
names, and $\id{\{}\vec{y} / \vec{x} \id{\}}$ to denote partial maps,
$s : \QProc \rightarrow \QProc$. A map, $s$ lifts, uniquely, to a map
on process terms, $\widehat{s} : \Proc \rightarrow \Proc$ by the
following equations.

\begin{mathpar}
  (0) \psubstp{Q}{P} := 0 \\
  (R \juxtap S) \psubstp{Q}{P}
  :=    
  (R)\psubstp{Q}{P} \juxtap (S) \psubstp{Q}{P} \\
  (x?(y).R) \psubstp{Q}{P}    
  :=    
  (x)\substp{Q}{P} (z)\concat( (R \psubstn{z}{y}) \psubstp{Q}{P} ) \\
  (\lift{x}{R}) \psubstp{Q}{P}  
  :=
  \lift{(x)\substp{Q}{P}}{ R \psubstp{Q}{P} } \\
%   (\dropn{x})  \psubstp{Q}{P}       
%   := 
%   \left\{ 
%     \begin{array}{ccc} 
%       \dropn{\quotep{Q}} & & x \nameeq \quotep{P} \\
%       \dropn{x} & & otherwise \\
%     \end{array}
%   \right. 
  (\dropn{x})  \psubstp{Q}{P}       
  := 
  \left\{ 
    \begin{array}{ccc} 
      Q & & x \nameeq \quotep{P} \\
      \dropn{x} & & otherwise \\
    \end{array}
  \right.
\end{mathpar}
 

where

\begin{eqnarray}
  (x)\id{\{} \lpquote Q \rpquote / \lpquote P \rpquote \id{\}}            = 
  \left\{ 
    \begin{array}{ccc}
      \lpquote Q \rpquote & & x \nameeq \lpquote P \rpquote \\
      x & & otherwise \\
    \end{array}
  \right. \nonumber
\end{eqnarray}

and $z$ is chosen distinct from $\quotep{P}$, $\quotep{Q}$, the free
names in $Q$, and all the names in $R$. Our $\alpha$-equivalence will
be built in the standard way from this substitution.

\begin{remark}\label{rem:no_self_referential_names}
  One consequence of these definitions is that $\forall P. \quotep{P}
  \not\in \freenames{P}$.
\end{remark}

\subsection{ Dynamic quote: an example }

Anticipating something of what's to come, consider applying the
substitution, $\widehat{\id{\{}u / z \id{\}}}$, to the following pair
of processes, $\lift{w}{y!(z)}$ and $w[ \lpquote y!(z) \rpquote ]$.

\begin{eqnarray}
	\lift{w}{y!(z)}\widehat{\id{\{}u / z \id{\}}}
		& = &
		\lift{w}{y!(u)} \nonumber\\
	w[ \lpquote y!(z) \rpquote ] \widehat{ \id{\{}u / z \id{\}} }
		& = &
		w[ \lpquote y!(z) \rpquote ] \nonumber
\end{eqnarray}

Because the body of the process between quotes is impervious to
substitution, we get radically different answers. In fact, by
examining the first process in an input context,
e.g. $x?(z).\lift{w}{y!(z)}$, we see that the process under the lift
operator may be shaped by prefixed inputs binding a name inside it. In
this sense, the lift operator will be seen as a way to dynamically
construct processes before reifying them as names.

Finally equipped with these standard features we can present the
dynamics of the calculus.

\subsubsection{Operational semantics} 

Finally, we introduce the computational dynamics. What marks these
algebras as distinct from other more traditionally studied algebraic
structures, e.g. vector spaces or polynomial rings, is the manner in
which dynamics is captured. In traditional structures, dynamics is typically
expressed through morphisms between such structures, as in linear maps
between vector spaces or morphisms between rings. In algebras
associated with the semantics of computation, the dynamics is
expressed as part of the algebraic structure itself, through a
reduction reduction relation typically denoted by $\red$. Below, we
give a recursive presentation of this relation for the calculus used
in the encoding.

$\red \subseteq \pi \times \pi$
$\red : \pi \to \mathcal{P}(\pi)$

\begin{mathpar}
  \inferrule* [lab=Comm] { \textsf{match}( x_{src}, x_{trgt} ) } { x_{trgt}?(y)P \; | \; x_{src}!\langle {Q} \rangle \red P\{\quotep{Q}/y}\} }
  \and \\
  \inferrule* [lab=Par] {{P} \red {P}'} {{{P} | {Q}} \red {{P}' | {Q}}}
  \and
  \inferrule* [lab=Equiv]{{{P} \scong {P}'} \andalso {{P}' \red {Q}'} \andalso {{Q}' \scong {Q}}}{{P} \red {Q}}
\end{mathpar}

\begin{eqnarray*}
  match_{\equiv} (\quotep{P},\quotep{Q}) & := & P \equiv Q \\
  match_{\dagger}(\quotep{P},\quotep{Q}) & := & \forall R. P|Q \red^{*} R => R \red^{*} 0 \\
  match_{K}(\quotep{P},\quotep{Q}) & := & K \mbox{ for some context } K
\end{eqnarray*}

$u?(x)P | u!\langle Q \rangle \red P\{\quotep{Q}/x\}$

%We write $\wred$ for $\red^*$, and $P\red$ if $\exists Q $ such that $ P \red Q$.
We write $P\red$ if $\exists Q $ such that $ P \red Q$ and $P\not\red$, otherwise.

\section{Replication}

As mentioned before, it is known that replication (and hence
recursion) can be implemented in a higher-order process algebra
\cite{SangiorgiWalker}. As our first example of calculation with the
machinery thus far presented we give the construction explicitly in
the {\rhoc}.

\begin{eqnarray}
	D_{x} & := & \prefix{x}{y}{(\binpar{\outputp{x}{y}}{@{y}})} \nonumber\\
	\bangp_{x}{P} & := & \binpar{{x}!\langle{\binpar{D_{x}}{P}}\rangle}{D_{x}} \nonumber
\end{eqnarray}

\begin{eqnarray}
	\bangp_{x}{P} & & \nonumber\\
	=
	& {x}!\langle{(\prefix{x}{y}{(\outputp{x}{y} | @{y})) | P}}\rangle 
	      | \prefix{x}{y}{(\outputp{x}{y} | @{y})} & \nonumber\\
	\red
	& (\outputp{x}{y} | @{y})\substn{\quotep{(\prefix{x}{y}{(@{y} | \outputp{x}{y})) | P}}}{y} & \nonumber\\
	=
	& \outputp{x}{\quotep{(\prefix{x}{y}{(\outputp{x}{y} | @{y})) | P}}}
	  | {(\prefix{x}{y}{(\outputp{x}{y} | @{y})) | P}} & \nonumber\\
	\red
	& \ldots & \nonumber\\
	\red^*
	& P | P | \ldots & \nonumber
\end{eqnarray}

Of course, this encoding, as an implementation, runs away, unfolding
$\bangp{P}$ eagerly. A lazier and more implementable replication
operator, restricted to input-guarded processes, may be obtained as follows.

\begin{eqnarray}
\bangp{\prefix{u}{v}{P}} 
	:= 
	\binpar{\lift{x}{\prefix{u}{v}{(\binpar{D(x)}{P})}}}{D(x)} \nonumber
\end{eqnarray}

\begin{remark}
  Note that the lazier definition still does not deal with summation
  or mixed summation (i.e. sums over input and output). The reader is
  invited to construct definitions of replication that deal with these
  features. 

  Further, the definitions are parameterized in a name, $x$. Can you,
  gentle reader, make a definition that eliminates this parameter and
  guarantees no accidental interaction between the replication
  machinery and the process being replicated -- i.e. no accidental
  sharing of names used by the process to get its work done and the
  name(s) used by the replication to effect copying. This latter
  revision of the definition of replication is crucial to obtaining
  the expected identity $!!P \sim !P$.
\end{remark}

\begin{remark}\label{rem:paradoxical_combinator}
  The reader familiar with the lambda calculus will have noticed the
  similarity between $D$ and the paradoxical combinator.

  [Ed. note: the existence of this seems to suggest we have to be more
  restrictive on the set of processes and names we admit if we are to
  support no-cloning.]
\end{remark}

\subsubsection{Bisimulation}

The computational dynamics gives rise to another kind of equivalence,
the equivalence of computational behavior. As previously mentioned
this is typically captured \emph{via} some form of bisimulation.

% The notion we use in this paper is weak barbed bisimulation
% \cite{milner91polyadicpi}.

The notion we use in this paper is derived from weak barbed
bisimulation \cite{milner91polyadicpi}. 

\begin{definition}
An \emph{observation relation}, $\downarrow_{\mathcal N}$, over a set
of names, $\mathcal N$, is the smallest relation satisfying the rules
below.

\infrule[Out-barb]{y \in {\mathcal N}, \; x \nameeq y}
		  {\outputp{x}{v} \downarrow_{\mathcal N} x}
\infrule[Par-barb]{\mbox{$P\downarrow_{\mathcal N} x$ or $Q\downarrow_{\mathcal N} x$}}
		  {\binpar{P}{Q} \downarrow_{\mathcal N} x}

We write $P \Downarrow_{\mathcal N} x$ if there is $Q$ such that 
$P \wred Q$ and $Q \downarrow_{\mathcal N} x$.
\end{definition}

\begin{definition}
%\label{def.bbisim}
An  ${\mathcal N}$-\emph{barbed bisimulation} over a set of names, ${\mathcal N}$, is a symmetric binary relation 
${\mathcal S}_{\mathcal N}$ between agents such that $P\rel{S}_{\mathcal N}Q$ implies:
\begin{enumerate}
\item If $P \red P'$ then $Q \wred Q'$ and $P'\rel{S}_{\mathcal N} Q'$.
\item If $P\downarrow_{\mathcal N} x$, then $Q\Downarrow_{\mathcal N} x$.
\end{enumerate}
$P$ is ${\mathcal N}$-barbed bisimilar to $Q$, written
$P \wbbisim_{\mathcal N} Q$, if $P \rel{S}_{\mathcal N} Q$ for some ${\mathcal N}$-barbed bisimulation ${\mathcal S}_{\mathcal N}$.
\end{definition}

$\mathcal{R} \subseteq \pi \times \pi$

$P \mathcal{R} Q => \forall P'. P \red P' \Rightarrow \exists Q'. Q \red Q', P' \mathcal{R} Q'$

$P \vdash x \Rightarrow Q \vdash x$

\begin{mathpar}
  \inferrule*[lab=Out-barb]{x \nameeq y}{{y}!\langle{Q}\rangle \vdash x}
  \and
  \inferrule*[lab=Par-barb]{\mbox{$P\vdash x$ or $Q\vdash x$}}{\binpar{P}{Q} \vdash x}
\end{mathpar}

\subsubsection{Contexts}

One of the principle advantages of computational calculi like the
$\pi$-calculus is a well-defined notion of context,
contextual-equivalence and a correlation between
contextual-equivalence and notions of bisimulation. The notion of
context allows the decomposition of a process into (sub-)process and
its syntactic environment, its context. Thus, a context may be
thought of as a process with a ``hole'' (written $\Box$) in it. The
application of a context $M$ to a process $P$, written $M[P]$, is
tantamount to filling the hole in $M$ with $P$. In this paper we do
not need the full weight of this theory, but do make use of the notion
of context in the proof the main theorem. 

\begin{mathpar}
  \inferrule* [lab=summation] {} {{M_{M},M_{N}} \bc \Box \;|\; x.M_{A} \;|\; M_{M}+M_{N}}
  \and
  \inferrule* [lab=agent] {} {{M_{A}} \bc (\vec{x})M_{P} \;| \; \clift{P_0,\ldots,M_{P},\ldots,P_N}}
  \and \\
  \inferrule* [lab=process] {} {{M_{P}} \bc M_{N} \;| \;P|M_{P} }
\end{mathpar} 

\begin{mathpar}
  \inferrule* [lab=sychronization] {} {M_{N} \bc \Box \;|\; x?M_{F} \;|\; x!M_{C}}
  \and
  \inferrule* [lab=abstraction] {} {{M_{F}} \bc (x)M_{P} }
  \and
  \inferrule* [lab=concretion] {} {{M_{C}} \bc \langle M_{P} \rangle }
  \and \\
  \inferrule* [lab=process] {} {{M_{P}} \bc M_{N} \;| \;P|M_{P} }
\end{mathpar}

\begin{definition}[contextual application] Given a context $M$, and
  process $P$, we define the \emph{contextual application}, $M[P] :=
  M\{P/\Box\}$. That is, the contextual application of M to P is the
  substitution of $P$ for $\Box$ in $M$.
\end{definition}

$\meaningof{-} : L \to \mathcal{P}(\pi)$

\begin{mathpar}
  \inferrule* [lab=collection] {} {\meaningof{true} = \pi, \and \meaningof{~E} = \pi \setminus \meaningof{E}, \and \meaningof{E_{1} \& E_{2}} = \meaningof{E_{1}} \cap \meaningof{E_{2}}}
\end{mathpar}

\begin{mathpar}
  \inferrule* [lab=structure] {} {\meaningof{0} = \{ P \in \pi | P \equiv 0 \}, \and \\ \meaningof{E_1 | E_2} = \{ P \in \pi | P \equiv P_{1} | P_{2}, P_{1} \in \meaningof{E_{1}}, P_{2} \in \meaningof{E_2}\} }
\end{mathpar}

\begin{mathpar}
 \inferrule* [lab=behavior] {} {\meaningof{\langle a?b \rangle E} = \{ P \in \pi | P \equiv Q | u?(y)P', \\ \and \\\\ \and \\ \;\;\; u \in \meaningof{a}, \forall z.P'\{z/y\} \in \meaningof{E\{z/b\}}\}, \and \\ \meaningof{a!E} = \{ P \in \pi | P \equiv Q | x!\langle P' \rangle, x \in \meaningof{a} P' \in \meaningof{E}\} }
\end{mathpar}

\begin{mathpar}
 \inferrule* [lab=nominal] {} {\meaningof{\quotep{E}} = \{ \quotep{P} \in \quotep{\pi} | P \in \meaningof{E} \}, \and \meaningof{\quotep{P}} = \{ \quotep{Q} \in \quotep{\pi} | P \equiv Q \} \and \\ \meaningof{@\quotep{E}} = \{ P \in \pi | P \equiv @x, x \in \meaningof{E} \}}
\end{mathpar}

\begin{eqnarray*}
  \\
  \meaningof{-} : TS \to ST
\end{eqnarray*}

\begin{eqnarray*}
  \\
  L : TS \to ST
\end{eqnarray*}

\begin{eqnarray*}
  \\
  P \models E \iff P \in \meaningof{E}
\end{eqnarray*}

\begin{eqnarray*}
  P \approx_{L} Q \iff \forall E \in L. P \models E \iff Q \models E
\end{eqnarray*}

\begin{eqnarray*}
  P \approx_{K} Q
\end{eqnarray*}

\begin{eqnarray*}
  P \approx Q
\end{eqnarray*}

$\approx_{K} = \approx = \approx_{L}$

\subsubsection{Contextual duality}

Note that contexts extend the quotation operation to a family of
operations from processes to names. Given a context, $M$, we can
define a \emph{nominal context}, $\quotep{M}$ by $\quotep{M}[P] :=
\quotep{M[P]}$. To foreshadow what is to come we observe that these
operations enjoy a duality with processes very much like the duality
between vectors and maps from vectors to scalars.

Further, because the calculus is essentially higher-order, we have a
correspondence between contexts and processes. More specifically,
given a name $x$ and a context $M$ we can construct $M^{*}_{x}$ such
that 

\begin{mathpar}
  M^{*}_{x} | \lift{x}{P} \red M[P]
\end{mathpar}

namely,

\begin{mathpar}
  M^{*}_{x} := x?(u).M[\dropn{u}]
\end{mathpar}

The dependence of $M^{*}_{x}$ on a name makes it an abstraction, 

\begin{mathpar}
  M^{*} := (x)x?(u).M[\dropn{u}]
\end{mathpar}

\subsection{Additional notation}

It will sometimes be convenient to denote the process a name
quotes. We already have the notation $x = \quotep{P}$, but it will be
convenient to introduce an alternate notation, $\procn{x}$, when we
want to emphasize the connection to the use of the name. Note that, by
virtue of name equivalence, $\quotep{\procn{x}} \nameeq x$; so, the
notation is consistent with previous definitions.

Further, because names have structure it is possible to effect
substitutions on the basis of that structure. This means we need to
upgrade our notation for substitutions, which we accomplish by
adapting comprehension notation. Thus,

\begin{mathpar}
  P\{ y / x : x \in S \}
\end{mathpar}

is interpreted to mean the process derived from P by replacing (in a
capture-avoiding manner) each occurrence of $x$ in $S$ by $y$. For example,

\begin{mathpar}
  P\{ \quotep{\procn{x}|\procn{x}} / x : x \in \freenames{P} \}
\end{mathpar}

will replace each (occurrence) of a free name $x$ in $P$ by
$\quotep{\procn{x}|\procn{x}}$.

Also, we will avail ourselves of the notation $x^{L}$ and $x^{R}$ to
denote injections of a name into disjoint copies of the name
space. There are numerous ways to accomplish this. One example can be
found in \cite{MeredithR05}. This notation overloads to vectors of
names: $\vec{x}^{\pi} := (x_{i}^{\pi} \; : \; 0 \leq i < |\vec{x}| )$ where $\pi \in \{L,R\}$.

We also use $P^{\Box} := P|\Box$.

In \cite{MeredithR05} an interpretation of the new operator is
given. It turns out that there are several possible interpretations
all enjoying the requisite algebraic properties of the operator (see
\cite{milner91polyadicpi}). We will therefore make liberal use of
$(\nu\; \vec{x})P$.

% subsection the_syntax_and_semantics_of_the_notation_system (end)   

\input{qm2pi.qmops} 

\input{qm2pi.sterngerlach} 

\input{qm2pi.metric} 

% section concurrent_process_calculi (end)

%\input{qm2pi.proofsketch}

% section proof sketch (end)

%\input{qm2pi.slviaknots} 

% section spatial logic via knots (end)

\input{qm2pi.conclusion}

% section conclusion (end)

%\input{qm2pi.dtcodes} 

% section wiring algorithm (end)

\input{qm2pi.ack} 

% section acknowledgments (end)

\newpage


\bibliographystyle{plain}   
\bibliography{../../biblios/main.bib}

\input{qm2pi.rhodetails}

\end{document}



% section proof sketch (end)

%\section{Unlikely characters: spatial logic for
  knots}\label{sub:characteristic_formulae} % (fold)

Associated to the mobile process calculi are a family of logics known
as the Hennessy-Milner logics. These logics typically enjoy a
semantics interpreting formulae as sets of processes that when
factored through the encoding outlined above allows an identification
of classes of knots with logical formulae. In the context of this
encoding the sub-family known as the spatial logics \cite{CairesC03}
\cite{CairesC04} \cite{Caires04} are of particular interest providing
several important features for expressing and reasoning about
properties (i.e. classes) of knots. We hint here at how this may be done.

%\begin{description}
%\item [structural connectives] 
\subsubsection{Structural connectives} The spatial logics enjoy
structural connectives corresponding, at the logical level, to the
parallel composition ($P | Q$) and new name ($(\nu \; x)P$)
connectives for processes. As illustrated in the examples below, these
connectives are extremely expressive given the shape of our encoding.
%\item [decideable satisfaction]

\subsubsection{Decideable satisfaction}
In \cite{Caires04} the satisfaction relation is shown to be decideable
for a rich class of processes. It further turns out that the image of
the our encoding is a proper subset of that class. This result
provides the basis for an algorithm by which to search for knots
enjoying a given property.
%\item [characteristic formulae]

\subsubsection{Characteristic formulae}
In the same paper \cite{Caires04} , Caires presents a means of calculating
characteristic formulae, selecting equivalence classes of processes
up to a pre--specified depth limit on the support set of names. Composed with our
encoding, this characteristic formula can be used to select
characteristic formulae for knots.
%\end{description}

\subsubsection{Spatial logic formulae}

The grammar below (segmented for comprehension) summarizes the syntax
of spatial logic formulae. We employ illustrative examples in the
sequel to provide an intuitive understanding of their meaning
referring the reader to \cite{Caires04} for a more detailed explication
of the semantics.

\begin{mathpar}
  \inferrule* [lab=boolean] {} {{A,B} \bc T \;|\; \neg A \;|\; A \wedge B \;|\; \eta = \eta'}
  \and
  \inferrule* [lab=spatial] {} {|\; \pzero \;|\; A | B \;|\; x \text{\textregistered} A \;|\; \forall x . A \;|\;  H x . A}
  \and
  \inferrule* [lab=behavioral] {} {|\; \alpha . A}
  \and 
  \inferrule* [lab=recursion] {} {|\; X(\vec{u}) \;|\; \mu X(\vec{u}) . A}
  \and
  \inferrule* [lab=action] {} {\alpha \bc \langle x?(\vec{y}) \rangle \;|\; \langle x!(\vec{y}) \rangle \;|\; \langle \tau \rangle}
  \and 
  \inferrule* [lab=name] {} {\eta \bc x \;|\; \tau}
\end{mathpar} 

% subsection characteristic_formulae (end)   	 

\subsection{Example formulae}\label{sub:example_formulae_} % (fold)

\subsubsection{Crossing as formula.}
% 
% \begin{align*}
%   \frac{d}{dx} \sin x &= \cos x 
%   & \frac{d}{dx} e^x &= e^x \\
%   \frac{d}{dx} \cos x &= - \sin x 
%   & \frac{d}{dx} \log x &= \frac{1}{x} \\
% \end{align*} 

\begin{align*}
 \mu C(x_{0},x_{1},y_{0},y_{1},u).&(\langle x_{0}?(z) \rangle(\langle u! \rangle\langle y_{1}!z \rangle C(x_{0},x_{1},y_{0},y_{1},u)) & \\
  & \wedge \langle y_{1}?(z) \rangle (\langle u! \rangle \langle x_{0}!z \rangle C(x_{0},x_{1},y_{0},y_{1},u)) & \\
  & \wedge \langle x_{1}?(z) \rangle (\langle u? \rangle \langle y_{0}!z \rangle C(x_{0},x_{1},y_{0},y_{1},u)) & \\
  & \wedge \langle y_{0}?(z) \rangle (\langle u? \rangle \langle x_{1}!z \rangle C(x_{0},x_{1},y_{0},y_{1},u))) &
\end{align*}

The lexicographical similarity between the shape of this formulae and
the shape of definition of the process representing a crossing reveals
the intuitive meaning of this formulae. It describes the capabilities
of a process that has the right to represent a crossing. For example
it picks out processes that may perform an input on the port $x_0$ in
its initial menu of capabilities. What differentiates the formula
from the process, however, is that the crossing process is the
smallest candidate to satisfy the formula. Infinitely many other
processes -- with internal behavior hidden behind this interface, so
to speak -- also satisfy this formula. Even this simple formula,
then, can be seen to open a new view onto knots, providing a
computational interpretation of \emph{virtual} knots.

Note that this formula is derived by hand. A similar formula can be
derived by employing Caires' calculation of characteristic formula
\cite{Caires04} to the process representing a crossing. In light of
this discussion, we let
$\meaningof{C}_{\phi}(x0,x1,y0,y1,u)$ denote a formula specifying the
dynamics we wish to capture of a crossing. To guarantee we preserve
the shape of the interface and minimal semantics we demand that
$\meaningof{C}_{\phi}(x0,x1,y0,y1,u) \Rightarrow
\textbf{C}(x0,x1,y0,y1,u)$ where $\textbf{C}(x0,x1,y0,y1,u)$ denotes
the formula above.
                            
\subsubsection{Crossing number constraints.}
The moral content of the context lemma (Lemma \ref{context}) is that the notion of
``locality'' in the Reidemeister moves is effectively captured by the
parallel composition operator of the process calculus. This intuition
extends through the logic. Given a formula,
$\meaningof{C}_{\phi}(x0,x1,y0,y1,u)$, we can use the structural
connectives to specify constraints on crossing numbers, such as at
least $n$ crossings, or exactly $n$ crossings.
\begin{mathpar}
  \inferrule* [lab=at-least-n] {} { K^{\geq n}_{\phi}(\vec{xs},\vec{ys}) := \Pi_{i=0}^{n-1} Hu . \meaningof{C}_{\phi}(xs_i,ys_i,u) | T }
  \and 
  \inferrule* [lab=exactly-n] {} { K^{= n}_{\phi}(\vec{xs},\vec{ys}) := \Pi_{i=0}^{n-1} Hu . \meaningof{C}_{\phi}(xs_i,ys_i,u) | \neg (\forall x_0,y_0,x_1,y_1,u . \meaningof{C}_{\phi}(x_0,y_0,x_1,y_1,u) | T) }
\end{mathpar}

To round out this section, recall that the encoding of an $n$-crossing
knot decomposes into a parallel composition of $n$ \emph{copies} of a
crossing process together with a wiring harness. To specify different
knot classes with the same crossing number amounts to specifying
logical constraints on the wiring harness. In the interest of space,
we defer examples to a forthcoming paper. Suffice it to say that both
the conditions ``alternating knot'' and ``contains the tangle
corresponding to 5/3'' are expressible. For example, it is possible to
calculate the characteristic formula of a process corresponding to the
tangle 5/3 and conjoin it into the classifying formula via the
composition connective of the logic.

Finally, we wish to observe that it is entirely within reason to
contemplate a more domain-specific version of spatial logic tailored
to the shape of processes in the image of the encoding. Such a
domain-specific logic would have a better claim to the title formal
language of knot properties.

% subsection example_formulae_ (end)

% section knots_as_processes (end) 

% section spatial logic via knots (end)

\section{Conclusions and future work}

\paragraph{Testing physical space}
You, gentle reader, may wonder why of all the theorems to be proved
given this set up we pick the one above. In some sense it's hardly
central to quantum mechanics. We see it as central in the sense that
it firmly establishes a notion of physical space arising from a notion
of the equivalence of behavior. Relating bisimulation to a metric is a
big step forward, but one is faced with interpreting the relationship
of that metric space to something more physical. Quantum mechanical
notions of ``physical'' space are still far from intuitive, but by
relating this idea of distance as testing to calculations that predict
physical circumstances we are making a not insignificant step forward
toward an understanding of the physical space we inhabit as
essentially dynamic.

\paragraph{Effectivity and simulation}
One of the observations we have yet to make is that the entire program
spelled out here is effective. We have built various interpreters for
the reflective calculus at work in this interpretation. In principle,
then, we can simulate quantum mechanics on a computer. The place where
the simulation may lose fidelity is the infinitely branching summation
for the annihilator.

In this connection i also want to point out that the evaluation style
calculation of the inner product puts the non-determinism of the
summation right at the heart of measurement. This suggests that
Milner's original reduction-based formulation of the dynamics of his
calculi in terms of sums was not just notationally suggestive of a
notion of measure-and-continue but captured some significant part of
the physics.

\paragraph{Quantum continuations}
In light of this last observation i want to point out that the
predominant account of quantum mechanics is missing a key aspect of a
truly compositional story of the physical situation. In a real lab,
when a measurement is made the observation can be made to feed into
another device that then makes another measurement conditioned on the
results of the first. This means that after the superposition was
collapsed the entire experimental set up remained in
superposition. While QM offers a means of writing this down it doesn't
quite line up well with the well-trodden formulation of computation
and continuation that we see so succinctly expressed in Milner's
calculi. This suggests that there might be advantages to this account
of dynamics waiting to be explored.

\paragraph{Quantum logic}
In this connection, we also note that by virtue of having the
Hennessy-Milner construction, we can pull the construction through the
interpretation of QM. This gives us a natural candidate for a quantum
logic that enjoys an extremely tight connection with it's domain of
interpretation, making the construction much less ad hoc (rather it is
the image of functor!).

\paragraph{Quantum probabiity}
i have questions about the basis of the interpretation of inner
product as probability amplitude. In particular, using which
axiomatization of probability theory does the notion of probability
amplitude earn the right to be so dubbed? In other words, where is the
proof that the operation for calculating a probability amplitude (and
then squaring) satisfies the axioms of what it means to calculate a
probability? Even if such a proof exists (i have yet to find it in the
literature), i wonder if it might not be possible to turn things on
their heads. Can we view the calculation of the probability amplitude
as an axiomatization of probability? If so, then the definition we
give for calculating probability amplitude may provide the basis for
an \emph{effective} theory of probability.

\paragraph{Quantum vs ``biological'' information}
Finally, i want to conclude with a more philosophical observation. At
a recent workshop in which QM was a predominant topic i noticed
something about quantum information. The speaker was giving a riveting
discussion of axiomatic QM and showing how properties of ``no
cloning'' and ``no deleting'' emerged as consequences of the
axiomatization. Theorems of this form are necessary to give us a sense
of confidence that our axioms characterize the physical theory. What
struck me, though, was that if quantum information is neither erasable
nor replicable it is markedly different from \emph{life}. Two of the
things we know about life is that

\begin{itemize}
  \item it ends;
  \item to gain some measure of persistence, to transcend it's
    finitude it is imminently copyable.
\end{itemize}

Both of these qualities are summarized succinctly in the aphorism: all
flesh is grass. For me these two kinds of ``information'' -- call them
quantum and biological -- are end points on a spectrum of strategies
for persistence. At one end, we have those curious entities that enjoy
uniqueness and permanence; at the other, we have those who in the face
of a certain end and an uncertain present make a go of passing
something on. To me one of the more remarkable aspects of the latter
strategy is that in the presence of noise (and certain features of
copying) we get a kind of dynamism, a chance for improvement against a
given persistent condition.

% subsection other_calculi_other_bisimulations_and_geometry_as_behavior (end)




% section conclusion (end)

%\documentclass[12pt]{llncs}
%\documentclass{jktr}

\usepackage[pdftex]{hyperref}                   
\usepackage {listings}
\usepackage {mathpartir}
\usepackage{bcprules}
%\usepackage{listings}
                       
\usepackage{graphicx} 
%\usepackage[margins=2.5cm,nohead,nofoot]{geometry}
%\usepackage{geometry}
\usepackage{amsfonts}
\usepackage{amstext}
\usepackage{latexsym}
\usepackage{amssymb}
\usepackage{color}


%\include{myPreamble}
\include{qm2pi.local} 

%\ifpdf
%\usepackage[pdftex]{graphicx}
%\else
%\usepackage{graphicx}
%\fi

 % \ifpdf
%  \usepackage{pdfsync}
%  \if


%\title{Brief Article}
%\author{David F. Snyder}
%\author{L.G. Meredith}

%\address{Dept. of Math., Texas State University--San Marcos, San Marcos, TX 78666}
       
\pagestyle{empty}


\begin{document}

\lstset{language=[Objective]Caml,frame=shadowbox}

\input{qm2pi.front}

% section front matter (end)

\input{qm2pi.intro} 
 
% section introduction (end)

% \input{qm2pi.knotations} 

% section notation (end)

\input{qm2pi.process.calculi} 

% section concurrent_process_calculi_and_spatial_logics_ (end)
    
%\input{qm2pi.knots2pi} 

%\input{qm2pi.trefoil} 

%\input{qm2pi.mainthm} 

% subsection basic_interpretation (end)

%\input{qm2pi.rho.presentation} 
\subsection{The syntax and semantics of the notation system}\label{sub:the_syntax_and_semantics_of_the_notation_system} % (fold)

We now summarize a technical presentation of the calculus that
embodies our theory of dynamics. The typical presentation of such a
calculus follows the style of giving generators and relations on
them. The grammar, below, describing term constructors, freely
generates the set of processes, $\Proc$. This set is then quotiented
by a relation known as structural congruence and it is over this set
that the notion of dynamics is expressed. This presentation is
essentially that of \cite{MeredithR05} with the addition of
polyadicity and summation. For readability we have relegated some of
the technical subtleties to an appendix.

\subsubsection{Process grammar}\label{subsub:process_grammar}

\begin{mathpar}
  \inferrule* [lab=synchronization] {} {{M} \bc \pzero \;|\; x?F \;|\; x!C }
  \and
  \inferrule* [lab=abstraction] {} {{F} \bc (x)P}
  \and
  \inferrule* [lab=concretion] {} {{C} \bc \langle Q \rangle}
  \and
  \inferrule* [lab=process] {} {{P,Q} \bc M \;| \;P|Q \;|\; @{x}}
  \and
  \inferrule* [lab=name] {} {{x} \bc \quotep{P}}
\end{mathpar} 

Note that $\vec{x}$ (resp. $\vec{P}$) denotes a vector of names
(resp. processes) of length $|\vec{x}|$ (resp. $|\vec{P}|$). We adopt
the following useful abbreviations.

\begin{mathpar}
   x?(\vec{y}).P := x.(\vec{y})P \and  x\clift{\vec{P}} := x.\clift{\vec{P}}
   \and x!(y) := \lift{x}{\dropn{y}}
   \and \Pi_{i=0}^{n-1}P_i := P_0 | \ldots | P_{n-1}
\end{mathpar}

\subsubsection{Structural congruence}

\paragraph{Free and bound names and alpha-equivalence.} At the
core of structural equivalence is alpha-equivalence which identifies
process that are the same up to a change of variable. Formally, we
recognize the distinction between free and bound names. The free names
of a process, $\freenames{P}$, may be calculated recursively as
follows:

\begin{mathpar}
\freenames{\pzero} := \emptyset
  \and \\
  \freenames{x?(y).P} := \{ x \} \cup (\freenames{P} \setminus \{ y \})
  \and 
  \freenames{x!\langle P \rangle} := \{ x \} \cup \{ P \} 
  \and \\
  \freenames{P|Q} := \freenames{P} \cup \freenames{Q}
  \and \\
  \freenames{@{x}} := \{ x \}
\end{mathpar}

$\pi$
$\quotep{\pi}$

$\freenames{-} : \pi \to \mathcal{P}(\quotep{\pi})$

\begin{eqnarray*}
  \freenames{\pzero} & := & \emptyset \\
  \freenames{x?(y).P} & := & \{ x \} \cup (\freenames{P} \setminus \{ y \}) \\
  \freenames{x!\langle P \rangle} & := & \{ x \} \cup \{ P \} \\
  \freenames{P|Q} & := & \freenames{P} \cup \freenames{Q} \\
  \freenames{\dropn{x}} & := & \{ x \}
\end{eqnarray*}

The bound names of a process, $\boundnames{P}$, are those names occurring in $P$
that are not free. For example, in $x?(y).0$, the name $x$ is free, while $y$ is bound.

\begin{mathpar}
  \inferrule* [lab=monoidal-laws] {} { P|Q \equiv Q|P \and P|0 \equiv P \and P|(Q|R) \equiv (P|Q)|R }
\end{mathpar}

\begin{mathpar}
  \inferrule* [lab=alpha-equivalence] {} { (x)P \equiv (y)P\{y/x\} \and y \not\in \freenames{P} }
\end{mathpar}

\begin{definition}
Then two processes, $P,Q$, are alpha-equivalent if $P = Q\{\vec{y}/\vec{x}\}$ for
some $\vec{x} \in \boundnames{Q},\vec{y} \in \boundnames{P}$, where $Q\{\vec{y}/\vec{x}\}$
denotes the capture-avoiding substitution of $\vec{y}$ for $\vec{x}$ in $Q$.
\end{definition}

\begin{definition}
  The {\em structural congruence} \cite{SangiorgiWalker} , $\equiv$,
  between processes is the least congruence containing
  alpha-equivalence, satisfying the abelian monoid laws
  (associativity, commutativity and $\pzero$ as identity) for parallel
  composition $|$ and for summation $+$.
\end{definition}

\subsection{Name equivalence}

We take name equivalence, written $\nameeq$, to be the smallest
equivalence relation generated by the following rules.

\begin{mathpar}
\inferrule*[lab=Quote-drop]
{ }
{ \quotep{@{x}} \nameeq x }

\inferrule*[lab=Struct-equiv]
{ P \scong Q }
{ \quotep{P} \nameeq \quotep{Q} }
\end{mathpar}

The astute reader will have noticed that the mutual recursion of names
and processes imposes a mutual recursion on alpha-equivalence and
structural equivalence via name-equivalence. Fortunately, all of this
works out pleasantly and we may calculate in the natural way, free of
concern. The reader interested in the details is referred to the
appendix \ref{appendix:rho_details}.

\subsection{Substitution}

We use $\Proc$ for the set of processes, $\QProc$ for the set of
names, and $\id{\{}\vec{y} / \vec{x} \id{\}}$ to denote partial maps,
$s : \QProc \rightarrow \QProc$. A map, $s$ lifts, uniquely, to a map
on process terms, $\widehat{s} : \Proc \rightarrow \Proc$ by the
following equations.

\begin{mathpar}
  (0) \psubstp{Q}{P} := 0 \\
  (R \juxtap S) \psubstp{Q}{P}
  :=    
  (R)\psubstp{Q}{P} \juxtap (S) \psubstp{Q}{P} \\
  (x?(y).R) \psubstp{Q}{P}    
  :=    
  (x)\substp{Q}{P} (z)\concat( (R \psubstn{z}{y}) \psubstp{Q}{P} ) \\
  (\lift{x}{R}) \psubstp{Q}{P}  
  :=
  \lift{(x)\substp{Q}{P}}{ R \psubstp{Q}{P} } \\
%   (\dropn{x})  \psubstp{Q}{P}       
%   := 
%   \left\{ 
%     \begin{array}{ccc} 
%       \dropn{\quotep{Q}} & & x \nameeq \quotep{P} \\
%       \dropn{x} & & otherwise \\
%     \end{array}
%   \right. 
  (\dropn{x})  \psubstp{Q}{P}       
  := 
  \left\{ 
    \begin{array}{ccc} 
      Q & & x \nameeq \quotep{P} \\
      \dropn{x} & & otherwise \\
    \end{array}
  \right.
\end{mathpar}
 

where

\begin{eqnarray}
  (x)\id{\{} \lpquote Q \rpquote / \lpquote P \rpquote \id{\}}            = 
  \left\{ 
    \begin{array}{ccc}
      \lpquote Q \rpquote & & x \nameeq \lpquote P \rpquote \\
      x & & otherwise \\
    \end{array}
  \right. \nonumber
\end{eqnarray}

and $z$ is chosen distinct from $\quotep{P}$, $\quotep{Q}$, the free
names in $Q$, and all the names in $R$. Our $\alpha$-equivalence will
be built in the standard way from this substitution.

\begin{remark}\label{rem:no_self_referential_names}
  One consequence of these definitions is that $\forall P. \quotep{P}
  \not\in \freenames{P}$.
\end{remark}

\subsection{ Dynamic quote: an example }

Anticipating something of what's to come, consider applying the
substitution, $\widehat{\id{\{}u / z \id{\}}}$, to the following pair
of processes, $\lift{w}{y!(z)}$ and $w[ \lpquote y!(z) \rpquote ]$.

\begin{eqnarray}
	\lift{w}{y!(z)}\widehat{\id{\{}u / z \id{\}}}
		& = &
		\lift{w}{y!(u)} \nonumber\\
	w[ \lpquote y!(z) \rpquote ] \widehat{ \id{\{}u / z \id{\}} }
		& = &
		w[ \lpquote y!(z) \rpquote ] \nonumber
\end{eqnarray}

Because the body of the process between quotes is impervious to
substitution, we get radically different answers. In fact, by
examining the first process in an input context,
e.g. $x?(z).\lift{w}{y!(z)}$, we see that the process under the lift
operator may be shaped by prefixed inputs binding a name inside it. In
this sense, the lift operator will be seen as a way to dynamically
construct processes before reifying them as names.

Finally equipped with these standard features we can present the
dynamics of the calculus.

\subsubsection{Operational semantics} 

Finally, we introduce the computational dynamics. What marks these
algebras as distinct from other more traditionally studied algebraic
structures, e.g. vector spaces or polynomial rings, is the manner in
which dynamics is captured. In traditional structures, dynamics is typically
expressed through morphisms between such structures, as in linear maps
between vector spaces or morphisms between rings. In algebras
associated with the semantics of computation, the dynamics is
expressed as part of the algebraic structure itself, through a
reduction reduction relation typically denoted by $\red$. Below, we
give a recursive presentation of this relation for the calculus used
in the encoding.

$\red \subseteq \pi \times \pi$
$\red : \pi \to \mathcal{P}(\pi)$

\begin{mathpar}
  \inferrule* [lab=Comm] { \textsf{match}( x_{src}, x_{trgt} ) } { x_{trgt}?(y)P \; | \; x_{src}!\langle {Q} \rangle \red P\{\quotep{Q}/y}\} }
  \and \\
  \inferrule* [lab=Par] {{P} \red {P}'} {{{P} | {Q}} \red {{P}' | {Q}}}
  \and
  \inferrule* [lab=Equiv]{{{P} \scong {P}'} \andalso {{P}' \red {Q}'} \andalso {{Q}' \scong {Q}}}{{P} \red {Q}}
\end{mathpar}

\begin{eqnarray*}
  match_{\equiv} (\quotep{P},\quotep{Q}) & := & P \equiv Q \\
  match_{\dagger}(\quotep{P},\quotep{Q}) & := & \forall R. P|Q \red^{*} R => R \red^{*} 0 \\
  match_{K}(\quotep{P},\quotep{Q}) & := & K \mbox{ for some context } K
\end{eqnarray*}

$u?(x)P | u!\langle Q \rangle \red P\{\quotep{Q}/x\}$

%We write $\wred$ for $\red^*$, and $P\red$ if $\exists Q $ such that $ P \red Q$.
We write $P\red$ if $\exists Q $ such that $ P \red Q$ and $P\not\red$, otherwise.

\section{Replication}

As mentioned before, it is known that replication (and hence
recursion) can be implemented in a higher-order process algebra
\cite{SangiorgiWalker}. As our first example of calculation with the
machinery thus far presented we give the construction explicitly in
the {\rhoc}.

\begin{eqnarray}
	D_{x} & := & \prefix{x}{y}{(\binpar{\outputp{x}{y}}{@{y}})} \nonumber\\
	\bangp_{x}{P} & := & \binpar{{x}!\langle{\binpar{D_{x}}{P}}\rangle}{D_{x}} \nonumber
\end{eqnarray}

\begin{eqnarray}
	\bangp_{x}{P} & & \nonumber\\
	=
	& {x}!\langle{(\prefix{x}{y}{(\outputp{x}{y} | @{y})) | P}}\rangle 
	      | \prefix{x}{y}{(\outputp{x}{y} | @{y})} & \nonumber\\
	\red
	& (\outputp{x}{y} | @{y})\substn{\quotep{(\prefix{x}{y}{(@{y} | \outputp{x}{y})) | P}}}{y} & \nonumber\\
	=
	& \outputp{x}{\quotep{(\prefix{x}{y}{(\outputp{x}{y} | @{y})) | P}}}
	  | {(\prefix{x}{y}{(\outputp{x}{y} | @{y})) | P}} & \nonumber\\
	\red
	& \ldots & \nonumber\\
	\red^*
	& P | P | \ldots & \nonumber
\end{eqnarray}

Of course, this encoding, as an implementation, runs away, unfolding
$\bangp{P}$ eagerly. A lazier and more implementable replication
operator, restricted to input-guarded processes, may be obtained as follows.

\begin{eqnarray}
\bangp{\prefix{u}{v}{P}} 
	:= 
	\binpar{\lift{x}{\prefix{u}{v}{(\binpar{D(x)}{P})}}}{D(x)} \nonumber
\end{eqnarray}

\begin{remark}
  Note that the lazier definition still does not deal with summation
  or mixed summation (i.e. sums over input and output). The reader is
  invited to construct definitions of replication that deal with these
  features. 

  Further, the definitions are parameterized in a name, $x$. Can you,
  gentle reader, make a definition that eliminates this parameter and
  guarantees no accidental interaction between the replication
  machinery and the process being replicated -- i.e. no accidental
  sharing of names used by the process to get its work done and the
  name(s) used by the replication to effect copying. This latter
  revision of the definition of replication is crucial to obtaining
  the expected identity $!!P \sim !P$.
\end{remark}

\begin{remark}\label{rem:paradoxical_combinator}
  The reader familiar with the lambda calculus will have noticed the
  similarity between $D$ and the paradoxical combinator.

  [Ed. note: the existence of this seems to suggest we have to be more
  restrictive on the set of processes and names we admit if we are to
  support no-cloning.]
\end{remark}

\subsubsection{Bisimulation}

The computational dynamics gives rise to another kind of equivalence,
the equivalence of computational behavior. As previously mentioned
this is typically captured \emph{via} some form of bisimulation.

% The notion we use in this paper is weak barbed bisimulation
% \cite{milner91polyadicpi}.

The notion we use in this paper is derived from weak barbed
bisimulation \cite{milner91polyadicpi}. 

\begin{definition}
An \emph{observation relation}, $\downarrow_{\mathcal N}$, over a set
of names, $\mathcal N$, is the smallest relation satisfying the rules
below.

\infrule[Out-barb]{y \in {\mathcal N}, \; x \nameeq y}
		  {\outputp{x}{v} \downarrow_{\mathcal N} x}
\infrule[Par-barb]{\mbox{$P\downarrow_{\mathcal N} x$ or $Q\downarrow_{\mathcal N} x$}}
		  {\binpar{P}{Q} \downarrow_{\mathcal N} x}

We write $P \Downarrow_{\mathcal N} x$ if there is $Q$ such that 
$P \wred Q$ and $Q \downarrow_{\mathcal N} x$.
\end{definition}

\begin{definition}
%\label{def.bbisim}
An  ${\mathcal N}$-\emph{barbed bisimulation} over a set of names, ${\mathcal N}$, is a symmetric binary relation 
${\mathcal S}_{\mathcal N}$ between agents such that $P\rel{S}_{\mathcal N}Q$ implies:
\begin{enumerate}
\item If $P \red P'$ then $Q \wred Q'$ and $P'\rel{S}_{\mathcal N} Q'$.
\item If $P\downarrow_{\mathcal N} x$, then $Q\Downarrow_{\mathcal N} x$.
\end{enumerate}
$P$ is ${\mathcal N}$-barbed bisimilar to $Q$, written
$P \wbbisim_{\mathcal N} Q$, if $P \rel{S}_{\mathcal N} Q$ for some ${\mathcal N}$-barbed bisimulation ${\mathcal S}_{\mathcal N}$.
\end{definition}

$\mathcal{R} \subseteq \pi \times \pi$

$P \mathcal{R} Q => \forall P'. P \red P' \Rightarrow \exists Q'. Q \red Q', P' \mathcal{R} Q'$

$P \vdash x \Rightarrow Q \vdash x$

\begin{mathpar}
  \inferrule*[lab=Out-barb]{x \nameeq y}{{y}!\langle{Q}\rangle \vdash x}
  \and
  \inferrule*[lab=Par-barb]{\mbox{$P\vdash x$ or $Q\vdash x$}}{\binpar{P}{Q} \vdash x}
\end{mathpar}

\subsubsection{Contexts}

One of the principle advantages of computational calculi like the
$\pi$-calculus is a well-defined notion of context,
contextual-equivalence and a correlation between
contextual-equivalence and notions of bisimulation. The notion of
context allows the decomposition of a process into (sub-)process and
its syntactic environment, its context. Thus, a context may be
thought of as a process with a ``hole'' (written $\Box$) in it. The
application of a context $M$ to a process $P$, written $M[P]$, is
tantamount to filling the hole in $M$ with $P$. In this paper we do
not need the full weight of this theory, but do make use of the notion
of context in the proof the main theorem. 

\begin{mathpar}
  \inferrule* [lab=summation] {} {{M_{M},M_{N}} \bc \Box \;|\; x.M_{A} \;|\; M_{M}+M_{N}}
  \and
  \inferrule* [lab=agent] {} {{M_{A}} \bc (\vec{x})M_{P} \;| \; \clift{P_0,\ldots,M_{P},\ldots,P_N}}
  \and \\
  \inferrule* [lab=process] {} {{M_{P}} \bc M_{N} \;| \;P|M_{P} }
\end{mathpar} 

\begin{mathpar}
  \inferrule* [lab=sychronization] {} {M_{N} \bc \Box \;|\; x?M_{F} \;|\; x!M_{C}}
  \and
  \inferrule* [lab=abstraction] {} {{M_{F}} \bc (x)M_{P} }
  \and
  \inferrule* [lab=concretion] {} {{M_{C}} \bc \langle M_{P} \rangle }
  \and \\
  \inferrule* [lab=process] {} {{M_{P}} \bc M_{N} \;| \;P|M_{P} }
\end{mathpar}

\begin{definition}[contextual application] Given a context $M$, and
  process $P$, we define the \emph{contextual application}, $M[P] :=
  M\{P/\Box\}$. That is, the contextual application of M to P is the
  substitution of $P$ for $\Box$ in $M$.
\end{definition}

$\meaningof{-} : L \to \mathcal{P}(\pi)$

\begin{mathpar}
  \inferrule* [lab=collection] {} {\meaningof{true} = \pi, \and \meaningof{~E} = \pi \setminus \meaningof{E}, \and \meaningof{E_{1} \& E_{2}} = \meaningof{E_{1}} \cap \meaningof{E_{2}}}
\end{mathpar}

\begin{mathpar}
  \inferrule* [lab=structure] {} {\meaningof{0} = \{ P \in \pi | P \equiv 0 \}, \and \\ \meaningof{E_1 | E_2} = \{ P \in \pi | P \equiv P_{1} | P_{2}, P_{1} \in \meaningof{E_{1}}, P_{2} \in \meaningof{E_2}\} }
\end{mathpar}

\begin{mathpar}
 \inferrule* [lab=behavior] {} {\meaningof{\langle a?b \rangle E} = \{ P \in \pi | P \equiv Q | u?(y)P', \\ \and \\\\ \and \\ \;\;\; u \in \meaningof{a}, \forall z.P'\{z/y\} \in \meaningof{E\{z/b\}}\}, \and \\ \meaningof{a!E} = \{ P \in \pi | P \equiv Q | x!\langle P' \rangle, x \in \meaningof{a} P' \in \meaningof{E}\} }
\end{mathpar}

\begin{mathpar}
 \inferrule* [lab=nominal] {} {\meaningof{\quotep{E}} = \{ \quotep{P} \in \quotep{\pi} | P \in \meaningof{E} \}, \and \meaningof{\quotep{P}} = \{ \quotep{Q} \in \quotep{\pi} | P \equiv Q \} \and \\ \meaningof{@\quotep{E}} = \{ P \in \pi | P \equiv @x, x \in \meaningof{E} \}}
\end{mathpar}

\begin{eqnarray*}
  \\
  \meaningof{-} : TS \to ST
\end{eqnarray*}

\begin{eqnarray*}
  \\
  L : TS \to ST
\end{eqnarray*}

\begin{eqnarray*}
  \\
  P \models E \iff P \in \meaningof{E}
\end{eqnarray*}

\begin{eqnarray*}
  P \approx_{L} Q \iff \forall E \in L. P \models E \iff Q \models E
\end{eqnarray*}

\begin{eqnarray*}
  P \approx_{K} Q
\end{eqnarray*}

\begin{eqnarray*}
  P \approx Q
\end{eqnarray*}

$\approx_{K} = \approx = \approx_{L}$

\subsubsection{Contextual duality}

Note that contexts extend the quotation operation to a family of
operations from processes to names. Given a context, $M$, we can
define a \emph{nominal context}, $\quotep{M}$ by $\quotep{M}[P] :=
\quotep{M[P]}$. To foreshadow what is to come we observe that these
operations enjoy a duality with processes very much like the duality
between vectors and maps from vectors to scalars.

Further, because the calculus is essentially higher-order, we have a
correspondence between contexts and processes. More specifically,
given a name $x$ and a context $M$ we can construct $M^{*}_{x}$ such
that 

\begin{mathpar}
  M^{*}_{x} | \lift{x}{P} \red M[P]
\end{mathpar}

namely,

\begin{mathpar}
  M^{*}_{x} := x?(u).M[\dropn{u}]
\end{mathpar}

The dependence of $M^{*}_{x}$ on a name makes it an abstraction, 

\begin{mathpar}
  M^{*} := (x)x?(u).M[\dropn{u}]
\end{mathpar}

\subsection{Additional notation}

It will sometimes be convenient to denote the process a name
quotes. We already have the notation $x = \quotep{P}$, but it will be
convenient to introduce an alternate notation, $\procn{x}$, when we
want to emphasize the connection to the use of the name. Note that, by
virtue of name equivalence, $\quotep{\procn{x}} \nameeq x$; so, the
notation is consistent with previous definitions.

Further, because names have structure it is possible to effect
substitutions on the basis of that structure. This means we need to
upgrade our notation for substitutions, which we accomplish by
adapting comprehension notation. Thus,

\begin{mathpar}
  P\{ y / x : x \in S \}
\end{mathpar}

is interpreted to mean the process derived from P by replacing (in a
capture-avoiding manner) each occurrence of $x$ in $S$ by $y$. For example,

\begin{mathpar}
  P\{ \quotep{\procn{x}|\procn{x}} / x : x \in \freenames{P} \}
\end{mathpar}

will replace each (occurrence) of a free name $x$ in $P$ by
$\quotep{\procn{x}|\procn{x}}$.

Also, we will avail ourselves of the notation $x^{L}$ and $x^{R}$ to
denote injections of a name into disjoint copies of the name
space. There are numerous ways to accomplish this. One example can be
found in \cite{MeredithR05}. This notation overloads to vectors of
names: $\vec{x}^{\pi} := (x_{i}^{\pi} \; : \; 0 \leq i < |\vec{x}| )$ where $\pi \in \{L,R\}$.

We also use $P^{\Box} := P|\Box$.

In \cite{MeredithR05} an interpretation of the new operator is
given. It turns out that there are several possible interpretations
all enjoying the requisite algebraic properties of the operator (see
\cite{milner91polyadicpi}). We will therefore make liberal use of
$(\nu\; \vec{x})P$.

% subsection the_syntax_and_semantics_of_the_notation_system (end)   

\input{qm2pi.qmops} 

\input{qm2pi.sterngerlach} 

\input{qm2pi.metric} 

% section concurrent_process_calculi (end)

%\input{qm2pi.proofsketch}

% section proof sketch (end)

%\input{qm2pi.slviaknots} 

% section spatial logic via knots (end)

\input{qm2pi.conclusion}

% section conclusion (end)

%\input{qm2pi.dtcodes} 

% section wiring algorithm (end)

\input{qm2pi.ack} 

% section acknowledgments (end)

\newpage


\bibliographystyle{plain}   
\bibliography{../../biblios/main.bib}

\input{qm2pi.rhodetails}

\end{document}

 

% section wiring algorithm (end)

\documentclass[12pt]{llncs}
%\documentclass{jktr}

\usepackage[pdftex]{hyperref}                   
\usepackage {listings}
\usepackage {mathpartir}
\usepackage{bcprules}
%\usepackage{listings}
                       
\usepackage{graphicx} 
%\usepackage[margins=2.5cm,nohead,nofoot]{geometry}
%\usepackage{geometry}
\usepackage{amsfonts}
\usepackage{amstext}
\usepackage{latexsym}
\usepackage{amssymb}
\usepackage{color}


%\include{myPreamble}
\include{qm2pi.local} 

%\ifpdf
%\usepackage[pdftex]{graphicx}
%\else
%\usepackage{graphicx}
%\fi

 % \ifpdf
%  \usepackage{pdfsync}
%  \if


%\title{Brief Article}
%\author{David F. Snyder}
%\author{L.G. Meredith}

%\address{Dept. of Math., Texas State University--San Marcos, San Marcos, TX 78666}
       
\pagestyle{empty}


\begin{document}

\lstset{language=[Objective]Caml,frame=shadowbox}

\input{qm2pi.front}

% section front matter (end)

\input{qm2pi.intro} 
 
% section introduction (end)

% \input{qm2pi.knotations} 

% section notation (end)

\input{qm2pi.process.calculi} 

% section concurrent_process_calculi_and_spatial_logics_ (end)
    
%\input{qm2pi.knots2pi} 

%\input{qm2pi.trefoil} 

%\input{qm2pi.mainthm} 

% subsection basic_interpretation (end)

%\input{qm2pi.rho.presentation} 
\subsection{The syntax and semantics of the notation system}\label{sub:the_syntax_and_semantics_of_the_notation_system} % (fold)

We now summarize a technical presentation of the calculus that
embodies our theory of dynamics. The typical presentation of such a
calculus follows the style of giving generators and relations on
them. The grammar, below, describing term constructors, freely
generates the set of processes, $\Proc$. This set is then quotiented
by a relation known as structural congruence and it is over this set
that the notion of dynamics is expressed. This presentation is
essentially that of \cite{MeredithR05} with the addition of
polyadicity and summation. For readability we have relegated some of
the technical subtleties to an appendix.

\subsubsection{Process grammar}\label{subsub:process_grammar}

\begin{mathpar}
  \inferrule* [lab=synchronization] {} {{M} \bc \pzero \;|\; x?F \;|\; x!C }
  \and
  \inferrule* [lab=abstraction] {} {{F} \bc (x)P}
  \and
  \inferrule* [lab=concretion] {} {{C} \bc \langle Q \rangle}
  \and
  \inferrule* [lab=process] {} {{P,Q} \bc M \;| \;P|Q \;|\; @{x}}
  \and
  \inferrule* [lab=name] {} {{x} \bc \quotep{P}}
\end{mathpar} 

Note that $\vec{x}$ (resp. $\vec{P}$) denotes a vector of names
(resp. processes) of length $|\vec{x}|$ (resp. $|\vec{P}|$). We adopt
the following useful abbreviations.

\begin{mathpar}
   x?(\vec{y}).P := x.(\vec{y})P \and  x\clift{\vec{P}} := x.\clift{\vec{P}}
   \and x!(y) := \lift{x}{\dropn{y}}
   \and \Pi_{i=0}^{n-1}P_i := P_0 | \ldots | P_{n-1}
\end{mathpar}

\subsubsection{Structural congruence}

\paragraph{Free and bound names and alpha-equivalence.} At the
core of structural equivalence is alpha-equivalence which identifies
process that are the same up to a change of variable. Formally, we
recognize the distinction between free and bound names. The free names
of a process, $\freenames{P}$, may be calculated recursively as
follows:

\begin{mathpar}
\freenames{\pzero} := \emptyset
  \and \\
  \freenames{x?(y).P} := \{ x \} \cup (\freenames{P} \setminus \{ y \})
  \and 
  \freenames{x!\langle P \rangle} := \{ x \} \cup \{ P \} 
  \and \\
  \freenames{P|Q} := \freenames{P} \cup \freenames{Q}
  \and \\
  \freenames{@{x}} := \{ x \}
\end{mathpar}

$\pi$
$\quotep{\pi}$

$\freenames{-} : \pi \to \mathcal{P}(\quotep{\pi})$

\begin{eqnarray*}
  \freenames{\pzero} & := & \emptyset \\
  \freenames{x?(y).P} & := & \{ x \} \cup (\freenames{P} \setminus \{ y \}) \\
  \freenames{x!\langle P \rangle} & := & \{ x \} \cup \{ P \} \\
  \freenames{P|Q} & := & \freenames{P} \cup \freenames{Q} \\
  \freenames{\dropn{x}} & := & \{ x \}
\end{eqnarray*}

The bound names of a process, $\boundnames{P}$, are those names occurring in $P$
that are not free. For example, in $x?(y).0$, the name $x$ is free, while $y$ is bound.

\begin{mathpar}
  \inferrule* [lab=monoidal-laws] {} { P|Q \equiv Q|P \and P|0 \equiv P \and P|(Q|R) \equiv (P|Q)|R }
\end{mathpar}

\begin{mathpar}
  \inferrule* [lab=alpha-equivalence] {} { (x)P \equiv (y)P\{y/x\} \and y \not\in \freenames{P} }
\end{mathpar}

\begin{definition}
Then two processes, $P,Q$, are alpha-equivalent if $P = Q\{\vec{y}/\vec{x}\}$ for
some $\vec{x} \in \boundnames{Q},\vec{y} \in \boundnames{P}$, where $Q\{\vec{y}/\vec{x}\}$
denotes the capture-avoiding substitution of $\vec{y}$ for $\vec{x}$ in $Q$.
\end{definition}

\begin{definition}
  The {\em structural congruence} \cite{SangiorgiWalker} , $\equiv$,
  between processes is the least congruence containing
  alpha-equivalence, satisfying the abelian monoid laws
  (associativity, commutativity and $\pzero$ as identity) for parallel
  composition $|$ and for summation $+$.
\end{definition}

\subsection{Name equivalence}

We take name equivalence, written $\nameeq$, to be the smallest
equivalence relation generated by the following rules.

\begin{mathpar}
\inferrule*[lab=Quote-drop]
{ }
{ \quotep{@{x}} \nameeq x }

\inferrule*[lab=Struct-equiv]
{ P \scong Q }
{ \quotep{P} \nameeq \quotep{Q} }
\end{mathpar}

The astute reader will have noticed that the mutual recursion of names
and processes imposes a mutual recursion on alpha-equivalence and
structural equivalence via name-equivalence. Fortunately, all of this
works out pleasantly and we may calculate in the natural way, free of
concern. The reader interested in the details is referred to the
appendix \ref{appendix:rho_details}.

\subsection{Substitution}

We use $\Proc$ for the set of processes, $\QProc$ for the set of
names, and $\id{\{}\vec{y} / \vec{x} \id{\}}$ to denote partial maps,
$s : \QProc \rightarrow \QProc$. A map, $s$ lifts, uniquely, to a map
on process terms, $\widehat{s} : \Proc \rightarrow \Proc$ by the
following equations.

\begin{mathpar}
  (0) \psubstp{Q}{P} := 0 \\
  (R \juxtap S) \psubstp{Q}{P}
  :=    
  (R)\psubstp{Q}{P} \juxtap (S) \psubstp{Q}{P} \\
  (x?(y).R) \psubstp{Q}{P}    
  :=    
  (x)\substp{Q}{P} (z)\concat( (R \psubstn{z}{y}) \psubstp{Q}{P} ) \\
  (\lift{x}{R}) \psubstp{Q}{P}  
  :=
  \lift{(x)\substp{Q}{P}}{ R \psubstp{Q}{P} } \\
%   (\dropn{x})  \psubstp{Q}{P}       
%   := 
%   \left\{ 
%     \begin{array}{ccc} 
%       \dropn{\quotep{Q}} & & x \nameeq \quotep{P} \\
%       \dropn{x} & & otherwise \\
%     \end{array}
%   \right. 
  (\dropn{x})  \psubstp{Q}{P}       
  := 
  \left\{ 
    \begin{array}{ccc} 
      Q & & x \nameeq \quotep{P} \\
      \dropn{x} & & otherwise \\
    \end{array}
  \right.
\end{mathpar}
 

where

\begin{eqnarray}
  (x)\id{\{} \lpquote Q \rpquote / \lpquote P \rpquote \id{\}}            = 
  \left\{ 
    \begin{array}{ccc}
      \lpquote Q \rpquote & & x \nameeq \lpquote P \rpquote \\
      x & & otherwise \\
    \end{array}
  \right. \nonumber
\end{eqnarray}

and $z$ is chosen distinct from $\quotep{P}$, $\quotep{Q}$, the free
names in $Q$, and all the names in $R$. Our $\alpha$-equivalence will
be built in the standard way from this substitution.

\begin{remark}\label{rem:no_self_referential_names}
  One consequence of these definitions is that $\forall P. \quotep{P}
  \not\in \freenames{P}$.
\end{remark}

\subsection{ Dynamic quote: an example }

Anticipating something of what's to come, consider applying the
substitution, $\widehat{\id{\{}u / z \id{\}}}$, to the following pair
of processes, $\lift{w}{y!(z)}$ and $w[ \lpquote y!(z) \rpquote ]$.

\begin{eqnarray}
	\lift{w}{y!(z)}\widehat{\id{\{}u / z \id{\}}}
		& = &
		\lift{w}{y!(u)} \nonumber\\
	w[ \lpquote y!(z) \rpquote ] \widehat{ \id{\{}u / z \id{\}} }
		& = &
		w[ \lpquote y!(z) \rpquote ] \nonumber
\end{eqnarray}

Because the body of the process between quotes is impervious to
substitution, we get radically different answers. In fact, by
examining the first process in an input context,
e.g. $x?(z).\lift{w}{y!(z)}$, we see that the process under the lift
operator may be shaped by prefixed inputs binding a name inside it. In
this sense, the lift operator will be seen as a way to dynamically
construct processes before reifying them as names.

Finally equipped with these standard features we can present the
dynamics of the calculus.

\subsubsection{Operational semantics} 

Finally, we introduce the computational dynamics. What marks these
algebras as distinct from other more traditionally studied algebraic
structures, e.g. vector spaces or polynomial rings, is the manner in
which dynamics is captured. In traditional structures, dynamics is typically
expressed through morphisms between such structures, as in linear maps
between vector spaces or morphisms between rings. In algebras
associated with the semantics of computation, the dynamics is
expressed as part of the algebraic structure itself, through a
reduction reduction relation typically denoted by $\red$. Below, we
give a recursive presentation of this relation for the calculus used
in the encoding.

$\red \subseteq \pi \times \pi$
$\red : \pi \to \mathcal{P}(\pi)$

\begin{mathpar}
  \inferrule* [lab=Comm] { \textsf{match}( x_{src}, x_{trgt} ) } { x_{trgt}?(y)P \; | \; x_{src}!\langle {Q} \rangle \red P\{\quotep{Q}/y}\} }
  \and \\
  \inferrule* [lab=Par] {{P} \red {P}'} {{{P} | {Q}} \red {{P}' | {Q}}}
  \and
  \inferrule* [lab=Equiv]{{{P} \scong {P}'} \andalso {{P}' \red {Q}'} \andalso {{Q}' \scong {Q}}}{{P} \red {Q}}
\end{mathpar}

\begin{eqnarray*}
  match_{\equiv} (\quotep{P},\quotep{Q}) & := & P \equiv Q \\
  match_{\dagger}(\quotep{P},\quotep{Q}) & := & \forall R. P|Q \red^{*} R => R \red^{*} 0 \\
  match_{K}(\quotep{P},\quotep{Q}) & := & K \mbox{ for some context } K
\end{eqnarray*}

$u?(x)P | u!\langle Q \rangle \red P\{\quotep{Q}/x\}$

%We write $\wred$ for $\red^*$, and $P\red$ if $\exists Q $ such that $ P \red Q$.
We write $P\red$ if $\exists Q $ such that $ P \red Q$ and $P\not\red$, otherwise.

\section{Replication}

As mentioned before, it is known that replication (and hence
recursion) can be implemented in a higher-order process algebra
\cite{SangiorgiWalker}. As our first example of calculation with the
machinery thus far presented we give the construction explicitly in
the {\rhoc}.

\begin{eqnarray}
	D_{x} & := & \prefix{x}{y}{(\binpar{\outputp{x}{y}}{@{y}})} \nonumber\\
	\bangp_{x}{P} & := & \binpar{{x}!\langle{\binpar{D_{x}}{P}}\rangle}{D_{x}} \nonumber
\end{eqnarray}

\begin{eqnarray}
	\bangp_{x}{P} & & \nonumber\\
	=
	& {x}!\langle{(\prefix{x}{y}{(\outputp{x}{y} | @{y})) | P}}\rangle 
	      | \prefix{x}{y}{(\outputp{x}{y} | @{y})} & \nonumber\\
	\red
	& (\outputp{x}{y} | @{y})\substn{\quotep{(\prefix{x}{y}{(@{y} | \outputp{x}{y})) | P}}}{y} & \nonumber\\
	=
	& \outputp{x}{\quotep{(\prefix{x}{y}{(\outputp{x}{y} | @{y})) | P}}}
	  | {(\prefix{x}{y}{(\outputp{x}{y} | @{y})) | P}} & \nonumber\\
	\red
	& \ldots & \nonumber\\
	\red^*
	& P | P | \ldots & \nonumber
\end{eqnarray}

Of course, this encoding, as an implementation, runs away, unfolding
$\bangp{P}$ eagerly. A lazier and more implementable replication
operator, restricted to input-guarded processes, may be obtained as follows.

\begin{eqnarray}
\bangp{\prefix{u}{v}{P}} 
	:= 
	\binpar{\lift{x}{\prefix{u}{v}{(\binpar{D(x)}{P})}}}{D(x)} \nonumber
\end{eqnarray}

\begin{remark}
  Note that the lazier definition still does not deal with summation
  or mixed summation (i.e. sums over input and output). The reader is
  invited to construct definitions of replication that deal with these
  features. 

  Further, the definitions are parameterized in a name, $x$. Can you,
  gentle reader, make a definition that eliminates this parameter and
  guarantees no accidental interaction between the replication
  machinery and the process being replicated -- i.e. no accidental
  sharing of names used by the process to get its work done and the
  name(s) used by the replication to effect copying. This latter
  revision of the definition of replication is crucial to obtaining
  the expected identity $!!P \sim !P$.
\end{remark}

\begin{remark}\label{rem:paradoxical_combinator}
  The reader familiar with the lambda calculus will have noticed the
  similarity between $D$ and the paradoxical combinator.

  [Ed. note: the existence of this seems to suggest we have to be more
  restrictive on the set of processes and names we admit if we are to
  support no-cloning.]
\end{remark}

\subsubsection{Bisimulation}

The computational dynamics gives rise to another kind of equivalence,
the equivalence of computational behavior. As previously mentioned
this is typically captured \emph{via} some form of bisimulation.

% The notion we use in this paper is weak barbed bisimulation
% \cite{milner91polyadicpi}.

The notion we use in this paper is derived from weak barbed
bisimulation \cite{milner91polyadicpi}. 

\begin{definition}
An \emph{observation relation}, $\downarrow_{\mathcal N}$, over a set
of names, $\mathcal N$, is the smallest relation satisfying the rules
below.

\infrule[Out-barb]{y \in {\mathcal N}, \; x \nameeq y}
		  {\outputp{x}{v} \downarrow_{\mathcal N} x}
\infrule[Par-barb]{\mbox{$P\downarrow_{\mathcal N} x$ or $Q\downarrow_{\mathcal N} x$}}
		  {\binpar{P}{Q} \downarrow_{\mathcal N} x}

We write $P \Downarrow_{\mathcal N} x$ if there is $Q$ such that 
$P \wred Q$ and $Q \downarrow_{\mathcal N} x$.
\end{definition}

\begin{definition}
%\label{def.bbisim}
An  ${\mathcal N}$-\emph{barbed bisimulation} over a set of names, ${\mathcal N}$, is a symmetric binary relation 
${\mathcal S}_{\mathcal N}$ between agents such that $P\rel{S}_{\mathcal N}Q$ implies:
\begin{enumerate}
\item If $P \red P'$ then $Q \wred Q'$ and $P'\rel{S}_{\mathcal N} Q'$.
\item If $P\downarrow_{\mathcal N} x$, then $Q\Downarrow_{\mathcal N} x$.
\end{enumerate}
$P$ is ${\mathcal N}$-barbed bisimilar to $Q$, written
$P \wbbisim_{\mathcal N} Q$, if $P \rel{S}_{\mathcal N} Q$ for some ${\mathcal N}$-barbed bisimulation ${\mathcal S}_{\mathcal N}$.
\end{definition}

$\mathcal{R} \subseteq \pi \times \pi$

$P \mathcal{R} Q => \forall P'. P \red P' \Rightarrow \exists Q'. Q \red Q', P' \mathcal{R} Q'$

$P \vdash x \Rightarrow Q \vdash x$

\begin{mathpar}
  \inferrule*[lab=Out-barb]{x \nameeq y}{{y}!\langle{Q}\rangle \vdash x}
  \and
  \inferrule*[lab=Par-barb]{\mbox{$P\vdash x$ or $Q\vdash x$}}{\binpar{P}{Q} \vdash x}
\end{mathpar}

\subsubsection{Contexts}

One of the principle advantages of computational calculi like the
$\pi$-calculus is a well-defined notion of context,
contextual-equivalence and a correlation between
contextual-equivalence and notions of bisimulation. The notion of
context allows the decomposition of a process into (sub-)process and
its syntactic environment, its context. Thus, a context may be
thought of as a process with a ``hole'' (written $\Box$) in it. The
application of a context $M$ to a process $P$, written $M[P]$, is
tantamount to filling the hole in $M$ with $P$. In this paper we do
not need the full weight of this theory, but do make use of the notion
of context in the proof the main theorem. 

\begin{mathpar}
  \inferrule* [lab=summation] {} {{M_{M},M_{N}} \bc \Box \;|\; x.M_{A} \;|\; M_{M}+M_{N}}
  \and
  \inferrule* [lab=agent] {} {{M_{A}} \bc (\vec{x})M_{P} \;| \; \clift{P_0,\ldots,M_{P},\ldots,P_N}}
  \and \\
  \inferrule* [lab=process] {} {{M_{P}} \bc M_{N} \;| \;P|M_{P} }
\end{mathpar} 

\begin{mathpar}
  \inferrule* [lab=sychronization] {} {M_{N} \bc \Box \;|\; x?M_{F} \;|\; x!M_{C}}
  \and
  \inferrule* [lab=abstraction] {} {{M_{F}} \bc (x)M_{P} }
  \and
  \inferrule* [lab=concretion] {} {{M_{C}} \bc \langle M_{P} \rangle }
  \and \\
  \inferrule* [lab=process] {} {{M_{P}} \bc M_{N} \;| \;P|M_{P} }
\end{mathpar}

\begin{definition}[contextual application] Given a context $M$, and
  process $P$, we define the \emph{contextual application}, $M[P] :=
  M\{P/\Box\}$. That is, the contextual application of M to P is the
  substitution of $P$ for $\Box$ in $M$.
\end{definition}

$\meaningof{-} : L \to \mathcal{P}(\pi)$

\begin{mathpar}
  \inferrule* [lab=collection] {} {\meaningof{true} = \pi, \and \meaningof{~E} = \pi \setminus \meaningof{E}, \and \meaningof{E_{1} \& E_{2}} = \meaningof{E_{1}} \cap \meaningof{E_{2}}}
\end{mathpar}

\begin{mathpar}
  \inferrule* [lab=structure] {} {\meaningof{0} = \{ P \in \pi | P \equiv 0 \}, \and \\ \meaningof{E_1 | E_2} = \{ P \in \pi | P \equiv P_{1} | P_{2}, P_{1} \in \meaningof{E_{1}}, P_{2} \in \meaningof{E_2}\} }
\end{mathpar}

\begin{mathpar}
 \inferrule* [lab=behavior] {} {\meaningof{\langle a?b \rangle E} = \{ P \in \pi | P \equiv Q | u?(y)P', \\ \and \\\\ \and \\ \;\;\; u \in \meaningof{a}, \forall z.P'\{z/y\} \in \meaningof{E\{z/b\}}\}, \and \\ \meaningof{a!E} = \{ P \in \pi | P \equiv Q | x!\langle P' \rangle, x \in \meaningof{a} P' \in \meaningof{E}\} }
\end{mathpar}

\begin{mathpar}
 \inferrule* [lab=nominal] {} {\meaningof{\quotep{E}} = \{ \quotep{P} \in \quotep{\pi} | P \in \meaningof{E} \}, \and \meaningof{\quotep{P}} = \{ \quotep{Q} \in \quotep{\pi} | P \equiv Q \} \and \\ \meaningof{@\quotep{E}} = \{ P \in \pi | P \equiv @x, x \in \meaningof{E} \}}
\end{mathpar}

\begin{eqnarray*}
  \\
  \meaningof{-} : TS \to ST
\end{eqnarray*}

\begin{eqnarray*}
  \\
  L : TS \to ST
\end{eqnarray*}

\begin{eqnarray*}
  \\
  P \models E \iff P \in \meaningof{E}
\end{eqnarray*}

\begin{eqnarray*}
  P \approx_{L} Q \iff \forall E \in L. P \models E \iff Q \models E
\end{eqnarray*}

\begin{eqnarray*}
  P \approx_{K} Q
\end{eqnarray*}

\begin{eqnarray*}
  P \approx Q
\end{eqnarray*}

$\approx_{K} = \approx = \approx_{L}$

\subsubsection{Contextual duality}

Note that contexts extend the quotation operation to a family of
operations from processes to names. Given a context, $M$, we can
define a \emph{nominal context}, $\quotep{M}$ by $\quotep{M}[P] :=
\quotep{M[P]}$. To foreshadow what is to come we observe that these
operations enjoy a duality with processes very much like the duality
between vectors and maps from vectors to scalars.

Further, because the calculus is essentially higher-order, we have a
correspondence between contexts and processes. More specifically,
given a name $x$ and a context $M$ we can construct $M^{*}_{x}$ such
that 

\begin{mathpar}
  M^{*}_{x} | \lift{x}{P} \red M[P]
\end{mathpar}

namely,

\begin{mathpar}
  M^{*}_{x} := x?(u).M[\dropn{u}]
\end{mathpar}

The dependence of $M^{*}_{x}$ on a name makes it an abstraction, 

\begin{mathpar}
  M^{*} := (x)x?(u).M[\dropn{u}]
\end{mathpar}

\subsection{Additional notation}

It will sometimes be convenient to denote the process a name
quotes. We already have the notation $x = \quotep{P}$, but it will be
convenient to introduce an alternate notation, $\procn{x}$, when we
want to emphasize the connection to the use of the name. Note that, by
virtue of name equivalence, $\quotep{\procn{x}} \nameeq x$; so, the
notation is consistent with previous definitions.

Further, because names have structure it is possible to effect
substitutions on the basis of that structure. This means we need to
upgrade our notation for substitutions, which we accomplish by
adapting comprehension notation. Thus,

\begin{mathpar}
  P\{ y / x : x \in S \}
\end{mathpar}

is interpreted to mean the process derived from P by replacing (in a
capture-avoiding manner) each occurrence of $x$ in $S$ by $y$. For example,

\begin{mathpar}
  P\{ \quotep{\procn{x}|\procn{x}} / x : x \in \freenames{P} \}
\end{mathpar}

will replace each (occurrence) of a free name $x$ in $P$ by
$\quotep{\procn{x}|\procn{x}}$.

Also, we will avail ourselves of the notation $x^{L}$ and $x^{R}$ to
denote injections of a name into disjoint copies of the name
space. There are numerous ways to accomplish this. One example can be
found in \cite{MeredithR05}. This notation overloads to vectors of
names: $\vec{x}^{\pi} := (x_{i}^{\pi} \; : \; 0 \leq i < |\vec{x}| )$ where $\pi \in \{L,R\}$.

We also use $P^{\Box} := P|\Box$.

In \cite{MeredithR05} an interpretation of the new operator is
given. It turns out that there are several possible interpretations
all enjoying the requisite algebraic properties of the operator (see
\cite{milner91polyadicpi}). We will therefore make liberal use of
$(\nu\; \vec{x})P$.

% subsection the_syntax_and_semantics_of_the_notation_system (end)   

\input{qm2pi.qmops} 

\input{qm2pi.sterngerlach} 

\input{qm2pi.metric} 

% section concurrent_process_calculi (end)

%\input{qm2pi.proofsketch}

% section proof sketch (end)

%\input{qm2pi.slviaknots} 

% section spatial logic via knots (end)

\input{qm2pi.conclusion}

% section conclusion (end)

%\input{qm2pi.dtcodes} 

% section wiring algorithm (end)

\input{qm2pi.ack} 

% section acknowledgments (end)

\newpage


\bibliographystyle{plain}   
\bibliography{../../biblios/main.bib}

\input{qm2pi.rhodetails}

\end{document}

 

% section acknowledgments (end)

\newpage


\bibliographystyle{plain}   
\bibliography{../../biblios/main.bib}

\documentclass[12pt]{llncs}
%\documentclass{jktr}

\usepackage[pdftex]{hyperref}                   
\usepackage {listings}
\usepackage {mathpartir}
\usepackage{bcprules}
%\usepackage{listings}
                       
\usepackage{graphicx} 
%\usepackage[margins=2.5cm,nohead,nofoot]{geometry}
%\usepackage{geometry}
\usepackage{amsfonts}
\usepackage{amstext}
\usepackage{latexsym}
\usepackage{amssymb}
\usepackage{color}


%\include{myPreamble}
\include{qm2pi.local} 

%\ifpdf
%\usepackage[pdftex]{graphicx}
%\else
%\usepackage{graphicx}
%\fi

 % \ifpdf
%  \usepackage{pdfsync}
%  \if


%\title{Brief Article}
%\author{David F. Snyder}
%\author{L.G. Meredith}

%\address{Dept. of Math., Texas State University--San Marcos, San Marcos, TX 78666}
       
\pagestyle{empty}


\begin{document}

\lstset{language=[Objective]Caml,frame=shadowbox}

\input{qm2pi.front}

% section front matter (end)

\input{qm2pi.intro} 
 
% section introduction (end)

% \input{qm2pi.knotations} 

% section notation (end)

\input{qm2pi.process.calculi} 

% section concurrent_process_calculi_and_spatial_logics_ (end)
    
%\input{qm2pi.knots2pi} 

%\input{qm2pi.trefoil} 

%\input{qm2pi.mainthm} 

% subsection basic_interpretation (end)

%\input{qm2pi.rho.presentation} 
\subsection{The syntax and semantics of the notation system}\label{sub:the_syntax_and_semantics_of_the_notation_system} % (fold)

We now summarize a technical presentation of the calculus that
embodies our theory of dynamics. The typical presentation of such a
calculus follows the style of giving generators and relations on
them. The grammar, below, describing term constructors, freely
generates the set of processes, $\Proc$. This set is then quotiented
by a relation known as structural congruence and it is over this set
that the notion of dynamics is expressed. This presentation is
essentially that of \cite{MeredithR05} with the addition of
polyadicity and summation. For readability we have relegated some of
the technical subtleties to an appendix.

\subsubsection{Process grammar}\label{subsub:process_grammar}

\begin{mathpar}
  \inferrule* [lab=synchronization] {} {{M} \bc \pzero \;|\; x?F \;|\; x!C }
  \and
  \inferrule* [lab=abstraction] {} {{F} \bc (x)P}
  \and
  \inferrule* [lab=concretion] {} {{C} \bc \langle Q \rangle}
  \and
  \inferrule* [lab=process] {} {{P,Q} \bc M \;| \;P|Q \;|\; @{x}}
  \and
  \inferrule* [lab=name] {} {{x} \bc \quotep{P}}
\end{mathpar} 

Note that $\vec{x}$ (resp. $\vec{P}$) denotes a vector of names
(resp. processes) of length $|\vec{x}|$ (resp. $|\vec{P}|$). We adopt
the following useful abbreviations.

\begin{mathpar}
   x?(\vec{y}).P := x.(\vec{y})P \and  x\clift{\vec{P}} := x.\clift{\vec{P}}
   \and x!(y) := \lift{x}{\dropn{y}}
   \and \Pi_{i=0}^{n-1}P_i := P_0 | \ldots | P_{n-1}
\end{mathpar}

\subsubsection{Structural congruence}

\paragraph{Free and bound names and alpha-equivalence.} At the
core of structural equivalence is alpha-equivalence which identifies
process that are the same up to a change of variable. Formally, we
recognize the distinction between free and bound names. The free names
of a process, $\freenames{P}$, may be calculated recursively as
follows:

\begin{mathpar}
\freenames{\pzero} := \emptyset
  \and \\
  \freenames{x?(y).P} := \{ x \} \cup (\freenames{P} \setminus \{ y \})
  \and 
  \freenames{x!\langle P \rangle} := \{ x \} \cup \{ P \} 
  \and \\
  \freenames{P|Q} := \freenames{P} \cup \freenames{Q}
  \and \\
  \freenames{@{x}} := \{ x \}
\end{mathpar}

$\pi$
$\quotep{\pi}$

$\freenames{-} : \pi \to \mathcal{P}(\quotep{\pi})$

\begin{eqnarray*}
  \freenames{\pzero} & := & \emptyset \\
  \freenames{x?(y).P} & := & \{ x \} \cup (\freenames{P} \setminus \{ y \}) \\
  \freenames{x!\langle P \rangle} & := & \{ x \} \cup \{ P \} \\
  \freenames{P|Q} & := & \freenames{P} \cup \freenames{Q} \\
  \freenames{\dropn{x}} & := & \{ x \}
\end{eqnarray*}

The bound names of a process, $\boundnames{P}$, are those names occurring in $P$
that are not free. For example, in $x?(y).0$, the name $x$ is free, while $y$ is bound.

\begin{mathpar}
  \inferrule* [lab=monoidal-laws] {} { P|Q \equiv Q|P \and P|0 \equiv P \and P|(Q|R) \equiv (P|Q)|R }
\end{mathpar}

\begin{mathpar}
  \inferrule* [lab=alpha-equivalence] {} { (x)P \equiv (y)P\{y/x\} \and y \not\in \freenames{P} }
\end{mathpar}

\begin{definition}
Then two processes, $P,Q$, are alpha-equivalent if $P = Q\{\vec{y}/\vec{x}\}$ for
some $\vec{x} \in \boundnames{Q},\vec{y} \in \boundnames{P}$, where $Q\{\vec{y}/\vec{x}\}$
denotes the capture-avoiding substitution of $\vec{y}$ for $\vec{x}$ in $Q$.
\end{definition}

\begin{definition}
  The {\em structural congruence} \cite{SangiorgiWalker} , $\equiv$,
  between processes is the least congruence containing
  alpha-equivalence, satisfying the abelian monoid laws
  (associativity, commutativity and $\pzero$ as identity) for parallel
  composition $|$ and for summation $+$.
\end{definition}

\subsection{Name equivalence}

We take name equivalence, written $\nameeq$, to be the smallest
equivalence relation generated by the following rules.

\begin{mathpar}
\inferrule*[lab=Quote-drop]
{ }
{ \quotep{@{x}} \nameeq x }

\inferrule*[lab=Struct-equiv]
{ P \scong Q }
{ \quotep{P} \nameeq \quotep{Q} }
\end{mathpar}

The astute reader will have noticed that the mutual recursion of names
and processes imposes a mutual recursion on alpha-equivalence and
structural equivalence via name-equivalence. Fortunately, all of this
works out pleasantly and we may calculate in the natural way, free of
concern. The reader interested in the details is referred to the
appendix \ref{appendix:rho_details}.

\subsection{Substitution}

We use $\Proc$ for the set of processes, $\QProc$ for the set of
names, and $\id{\{}\vec{y} / \vec{x} \id{\}}$ to denote partial maps,
$s : \QProc \rightarrow \QProc$. A map, $s$ lifts, uniquely, to a map
on process terms, $\widehat{s} : \Proc \rightarrow \Proc$ by the
following equations.

\begin{mathpar}
  (0) \psubstp{Q}{P} := 0 \\
  (R \juxtap S) \psubstp{Q}{P}
  :=    
  (R)\psubstp{Q}{P} \juxtap (S) \psubstp{Q}{P} \\
  (x?(y).R) \psubstp{Q}{P}    
  :=    
  (x)\substp{Q}{P} (z)\concat( (R \psubstn{z}{y}) \psubstp{Q}{P} ) \\
  (\lift{x}{R}) \psubstp{Q}{P}  
  :=
  \lift{(x)\substp{Q}{P}}{ R \psubstp{Q}{P} } \\
%   (\dropn{x})  \psubstp{Q}{P}       
%   := 
%   \left\{ 
%     \begin{array}{ccc} 
%       \dropn{\quotep{Q}} & & x \nameeq \quotep{P} \\
%       \dropn{x} & & otherwise \\
%     \end{array}
%   \right. 
  (\dropn{x})  \psubstp{Q}{P}       
  := 
  \left\{ 
    \begin{array}{ccc} 
      Q & & x \nameeq \quotep{P} \\
      \dropn{x} & & otherwise \\
    \end{array}
  \right.
\end{mathpar}
 

where

\begin{eqnarray}
  (x)\id{\{} \lpquote Q \rpquote / \lpquote P \rpquote \id{\}}            = 
  \left\{ 
    \begin{array}{ccc}
      \lpquote Q \rpquote & & x \nameeq \lpquote P \rpquote \\
      x & & otherwise \\
    \end{array}
  \right. \nonumber
\end{eqnarray}

and $z$ is chosen distinct from $\quotep{P}$, $\quotep{Q}$, the free
names in $Q$, and all the names in $R$. Our $\alpha$-equivalence will
be built in the standard way from this substitution.

\begin{remark}\label{rem:no_self_referential_names}
  One consequence of these definitions is that $\forall P. \quotep{P}
  \not\in \freenames{P}$.
\end{remark}

\subsection{ Dynamic quote: an example }

Anticipating something of what's to come, consider applying the
substitution, $\widehat{\id{\{}u / z \id{\}}}$, to the following pair
of processes, $\lift{w}{y!(z)}$ and $w[ \lpquote y!(z) \rpquote ]$.

\begin{eqnarray}
	\lift{w}{y!(z)}\widehat{\id{\{}u / z \id{\}}}
		& = &
		\lift{w}{y!(u)} \nonumber\\
	w[ \lpquote y!(z) \rpquote ] \widehat{ \id{\{}u / z \id{\}} }
		& = &
		w[ \lpquote y!(z) \rpquote ] \nonumber
\end{eqnarray}

Because the body of the process between quotes is impervious to
substitution, we get radically different answers. In fact, by
examining the first process in an input context,
e.g. $x?(z).\lift{w}{y!(z)}$, we see that the process under the lift
operator may be shaped by prefixed inputs binding a name inside it. In
this sense, the lift operator will be seen as a way to dynamically
construct processes before reifying them as names.

Finally equipped with these standard features we can present the
dynamics of the calculus.

\subsubsection{Operational semantics} 

Finally, we introduce the computational dynamics. What marks these
algebras as distinct from other more traditionally studied algebraic
structures, e.g. vector spaces or polynomial rings, is the manner in
which dynamics is captured. In traditional structures, dynamics is typically
expressed through morphisms between such structures, as in linear maps
between vector spaces or morphisms between rings. In algebras
associated with the semantics of computation, the dynamics is
expressed as part of the algebraic structure itself, through a
reduction reduction relation typically denoted by $\red$. Below, we
give a recursive presentation of this relation for the calculus used
in the encoding.

$\red \subseteq \pi \times \pi$
$\red : \pi \to \mathcal{P}(\pi)$

\begin{mathpar}
  \inferrule* [lab=Comm] { \textsf{match}( x_{src}, x_{trgt} ) } { x_{trgt}?(y)P \; | \; x_{src}!\langle {Q} \rangle \red P\{\quotep{Q}/y}\} }
  \and \\
  \inferrule* [lab=Par] {{P} \red {P}'} {{{P} | {Q}} \red {{P}' | {Q}}}
  \and
  \inferrule* [lab=Equiv]{{{P} \scong {P}'} \andalso {{P}' \red {Q}'} \andalso {{Q}' \scong {Q}}}{{P} \red {Q}}
\end{mathpar}

\begin{eqnarray*}
  match_{\equiv} (\quotep{P},\quotep{Q}) & := & P \equiv Q \\
  match_{\dagger}(\quotep{P},\quotep{Q}) & := & \forall R. P|Q \red^{*} R => R \red^{*} 0 \\
  match_{K}(\quotep{P},\quotep{Q}) & := & K \mbox{ for some context } K
\end{eqnarray*}

$u?(x)P | u!\langle Q \rangle \red P\{\quotep{Q}/x\}$

%We write $\wred$ for $\red^*$, and $P\red$ if $\exists Q $ such that $ P \red Q$.
We write $P\red$ if $\exists Q $ such that $ P \red Q$ and $P\not\red$, otherwise.

\section{Replication}

As mentioned before, it is known that replication (and hence
recursion) can be implemented in a higher-order process algebra
\cite{SangiorgiWalker}. As our first example of calculation with the
machinery thus far presented we give the construction explicitly in
the {\rhoc}.

\begin{eqnarray}
	D_{x} & := & \prefix{x}{y}{(\binpar{\outputp{x}{y}}{@{y}})} \nonumber\\
	\bangp_{x}{P} & := & \binpar{{x}!\langle{\binpar{D_{x}}{P}}\rangle}{D_{x}} \nonumber
\end{eqnarray}

\begin{eqnarray}
	\bangp_{x}{P} & & \nonumber\\
	=
	& {x}!\langle{(\prefix{x}{y}{(\outputp{x}{y} | @{y})) | P}}\rangle 
	      | \prefix{x}{y}{(\outputp{x}{y} | @{y})} & \nonumber\\
	\red
	& (\outputp{x}{y} | @{y})\substn{\quotep{(\prefix{x}{y}{(@{y} | \outputp{x}{y})) | P}}}{y} & \nonumber\\
	=
	& \outputp{x}{\quotep{(\prefix{x}{y}{(\outputp{x}{y} | @{y})) | P}}}
	  | {(\prefix{x}{y}{(\outputp{x}{y} | @{y})) | P}} & \nonumber\\
	\red
	& \ldots & \nonumber\\
	\red^*
	& P | P | \ldots & \nonumber
\end{eqnarray}

Of course, this encoding, as an implementation, runs away, unfolding
$\bangp{P}$ eagerly. A lazier and more implementable replication
operator, restricted to input-guarded processes, may be obtained as follows.

\begin{eqnarray}
\bangp{\prefix{u}{v}{P}} 
	:= 
	\binpar{\lift{x}{\prefix{u}{v}{(\binpar{D(x)}{P})}}}{D(x)} \nonumber
\end{eqnarray}

\begin{remark}
  Note that the lazier definition still does not deal with summation
  or mixed summation (i.e. sums over input and output). The reader is
  invited to construct definitions of replication that deal with these
  features. 

  Further, the definitions are parameterized in a name, $x$. Can you,
  gentle reader, make a definition that eliminates this parameter and
  guarantees no accidental interaction between the replication
  machinery and the process being replicated -- i.e. no accidental
  sharing of names used by the process to get its work done and the
  name(s) used by the replication to effect copying. This latter
  revision of the definition of replication is crucial to obtaining
  the expected identity $!!P \sim !P$.
\end{remark}

\begin{remark}\label{rem:paradoxical_combinator}
  The reader familiar with the lambda calculus will have noticed the
  similarity between $D$ and the paradoxical combinator.

  [Ed. note: the existence of this seems to suggest we have to be more
  restrictive on the set of processes and names we admit if we are to
  support no-cloning.]
\end{remark}

\subsubsection{Bisimulation}

The computational dynamics gives rise to another kind of equivalence,
the equivalence of computational behavior. As previously mentioned
this is typically captured \emph{via} some form of bisimulation.

% The notion we use in this paper is weak barbed bisimulation
% \cite{milner91polyadicpi}.

The notion we use in this paper is derived from weak barbed
bisimulation \cite{milner91polyadicpi}. 

\begin{definition}
An \emph{observation relation}, $\downarrow_{\mathcal N}$, over a set
of names, $\mathcal N$, is the smallest relation satisfying the rules
below.

\infrule[Out-barb]{y \in {\mathcal N}, \; x \nameeq y}
		  {\outputp{x}{v} \downarrow_{\mathcal N} x}
\infrule[Par-barb]{\mbox{$P\downarrow_{\mathcal N} x$ or $Q\downarrow_{\mathcal N} x$}}
		  {\binpar{P}{Q} \downarrow_{\mathcal N} x}

We write $P \Downarrow_{\mathcal N} x$ if there is $Q$ such that 
$P \wred Q$ and $Q \downarrow_{\mathcal N} x$.
\end{definition}

\begin{definition}
%\label{def.bbisim}
An  ${\mathcal N}$-\emph{barbed bisimulation} over a set of names, ${\mathcal N}$, is a symmetric binary relation 
${\mathcal S}_{\mathcal N}$ between agents such that $P\rel{S}_{\mathcal N}Q$ implies:
\begin{enumerate}
\item If $P \red P'$ then $Q \wred Q'$ and $P'\rel{S}_{\mathcal N} Q'$.
\item If $P\downarrow_{\mathcal N} x$, then $Q\Downarrow_{\mathcal N} x$.
\end{enumerate}
$P$ is ${\mathcal N}$-barbed bisimilar to $Q$, written
$P \wbbisim_{\mathcal N} Q$, if $P \rel{S}_{\mathcal N} Q$ for some ${\mathcal N}$-barbed bisimulation ${\mathcal S}_{\mathcal N}$.
\end{definition}

$\mathcal{R} \subseteq \pi \times \pi$

$P \mathcal{R} Q => \forall P'. P \red P' \Rightarrow \exists Q'. Q \red Q', P' \mathcal{R} Q'$

$P \vdash x \Rightarrow Q \vdash x$

\begin{mathpar}
  \inferrule*[lab=Out-barb]{x \nameeq y}{{y}!\langle{Q}\rangle \vdash x}
  \and
  \inferrule*[lab=Par-barb]{\mbox{$P\vdash x$ or $Q\vdash x$}}{\binpar{P}{Q} \vdash x}
\end{mathpar}

\subsubsection{Contexts}

One of the principle advantages of computational calculi like the
$\pi$-calculus is a well-defined notion of context,
contextual-equivalence and a correlation between
contextual-equivalence and notions of bisimulation. The notion of
context allows the decomposition of a process into (sub-)process and
its syntactic environment, its context. Thus, a context may be
thought of as a process with a ``hole'' (written $\Box$) in it. The
application of a context $M$ to a process $P$, written $M[P]$, is
tantamount to filling the hole in $M$ with $P$. In this paper we do
not need the full weight of this theory, but do make use of the notion
of context in the proof the main theorem. 

\begin{mathpar}
  \inferrule* [lab=summation] {} {{M_{M},M_{N}} \bc \Box \;|\; x.M_{A} \;|\; M_{M}+M_{N}}
  \and
  \inferrule* [lab=agent] {} {{M_{A}} \bc (\vec{x})M_{P} \;| \; \clift{P_0,\ldots,M_{P},\ldots,P_N}}
  \and \\
  \inferrule* [lab=process] {} {{M_{P}} \bc M_{N} \;| \;P|M_{P} }
\end{mathpar} 

\begin{mathpar}
  \inferrule* [lab=sychronization] {} {M_{N} \bc \Box \;|\; x?M_{F} \;|\; x!M_{C}}
  \and
  \inferrule* [lab=abstraction] {} {{M_{F}} \bc (x)M_{P} }
  \and
  \inferrule* [lab=concretion] {} {{M_{C}} \bc \langle M_{P} \rangle }
  \and \\
  \inferrule* [lab=process] {} {{M_{P}} \bc M_{N} \;| \;P|M_{P} }
\end{mathpar}

\begin{definition}[contextual application] Given a context $M$, and
  process $P$, we define the \emph{contextual application}, $M[P] :=
  M\{P/\Box\}$. That is, the contextual application of M to P is the
  substitution of $P$ for $\Box$ in $M$.
\end{definition}

$\meaningof{-} : L \to \mathcal{P}(\pi)$

\begin{mathpar}
  \inferrule* [lab=collection] {} {\meaningof{true} = \pi, \and \meaningof{~E} = \pi \setminus \meaningof{E}, \and \meaningof{E_{1} \& E_{2}} = \meaningof{E_{1}} \cap \meaningof{E_{2}}}
\end{mathpar}

\begin{mathpar}
  \inferrule* [lab=structure] {} {\meaningof{0} = \{ P \in \pi | P \equiv 0 \}, \and \\ \meaningof{E_1 | E_2} = \{ P \in \pi | P \equiv P_{1} | P_{2}, P_{1} \in \meaningof{E_{1}}, P_{2} \in \meaningof{E_2}\} }
\end{mathpar}

\begin{mathpar}
 \inferrule* [lab=behavior] {} {\meaningof{\langle a?b \rangle E} = \{ P \in \pi | P \equiv Q | u?(y)P', \\ \and \\\\ \and \\ \;\;\; u \in \meaningof{a}, \forall z.P'\{z/y\} \in \meaningof{E\{z/b\}}\}, \and \\ \meaningof{a!E} = \{ P \in \pi | P \equiv Q | x!\langle P' \rangle, x \in \meaningof{a} P' \in \meaningof{E}\} }
\end{mathpar}

\begin{mathpar}
 \inferrule* [lab=nominal] {} {\meaningof{\quotep{E}} = \{ \quotep{P} \in \quotep{\pi} | P \in \meaningof{E} \}, \and \meaningof{\quotep{P}} = \{ \quotep{Q} \in \quotep{\pi} | P \equiv Q \} \and \\ \meaningof{@\quotep{E}} = \{ P \in \pi | P \equiv @x, x \in \meaningof{E} \}}
\end{mathpar}

\begin{eqnarray*}
  \\
  \meaningof{-} : TS \to ST
\end{eqnarray*}

\begin{eqnarray*}
  \\
  L : TS \to ST
\end{eqnarray*}

\begin{eqnarray*}
  \\
  P \models E \iff P \in \meaningof{E}
\end{eqnarray*}

\begin{eqnarray*}
  P \approx_{L} Q \iff \forall E \in L. P \models E \iff Q \models E
\end{eqnarray*}

\begin{eqnarray*}
  P \approx_{K} Q
\end{eqnarray*}

\begin{eqnarray*}
  P \approx Q
\end{eqnarray*}

$\approx_{K} = \approx = \approx_{L}$

\subsubsection{Contextual duality}

Note that contexts extend the quotation operation to a family of
operations from processes to names. Given a context, $M$, we can
define a \emph{nominal context}, $\quotep{M}$ by $\quotep{M}[P] :=
\quotep{M[P]}$. To foreshadow what is to come we observe that these
operations enjoy a duality with processes very much like the duality
between vectors and maps from vectors to scalars.

Further, because the calculus is essentially higher-order, we have a
correspondence between contexts and processes. More specifically,
given a name $x$ and a context $M$ we can construct $M^{*}_{x}$ such
that 

\begin{mathpar}
  M^{*}_{x} | \lift{x}{P} \red M[P]
\end{mathpar}

namely,

\begin{mathpar}
  M^{*}_{x} := x?(u).M[\dropn{u}]
\end{mathpar}

The dependence of $M^{*}_{x}$ on a name makes it an abstraction, 

\begin{mathpar}
  M^{*} := (x)x?(u).M[\dropn{u}]
\end{mathpar}

\subsection{Additional notation}

It will sometimes be convenient to denote the process a name
quotes. We already have the notation $x = \quotep{P}$, but it will be
convenient to introduce an alternate notation, $\procn{x}$, when we
want to emphasize the connection to the use of the name. Note that, by
virtue of name equivalence, $\quotep{\procn{x}} \nameeq x$; so, the
notation is consistent with previous definitions.

Further, because names have structure it is possible to effect
substitutions on the basis of that structure. This means we need to
upgrade our notation for substitutions, which we accomplish by
adapting comprehension notation. Thus,

\begin{mathpar}
  P\{ y / x : x \in S \}
\end{mathpar}

is interpreted to mean the process derived from P by replacing (in a
capture-avoiding manner) each occurrence of $x$ in $S$ by $y$. For example,

\begin{mathpar}
  P\{ \quotep{\procn{x}|\procn{x}} / x : x \in \freenames{P} \}
\end{mathpar}

will replace each (occurrence) of a free name $x$ in $P$ by
$\quotep{\procn{x}|\procn{x}}$.

Also, we will avail ourselves of the notation $x^{L}$ and $x^{R}$ to
denote injections of a name into disjoint copies of the name
space. There are numerous ways to accomplish this. One example can be
found in \cite{MeredithR05}. This notation overloads to vectors of
names: $\vec{x}^{\pi} := (x_{i}^{\pi} \; : \; 0 \leq i < |\vec{x}| )$ where $\pi \in \{L,R\}$.

We also use $P^{\Box} := P|\Box$.

In \cite{MeredithR05} an interpretation of the new operator is
given. It turns out that there are several possible interpretations
all enjoying the requisite algebraic properties of the operator (see
\cite{milner91polyadicpi}). We will therefore make liberal use of
$(\nu\; \vec{x})P$.

% subsection the_syntax_and_semantics_of_the_notation_system (end)   

\input{qm2pi.qmops} 

\input{qm2pi.sterngerlach} 

\input{qm2pi.metric} 

% section concurrent_process_calculi (end)

%\input{qm2pi.proofsketch}

% section proof sketch (end)

%\input{qm2pi.slviaknots} 

% section spatial logic via knots (end)

\input{qm2pi.conclusion}

% section conclusion (end)

%\input{qm2pi.dtcodes} 

% section wiring algorithm (end)

\input{qm2pi.ack} 

% section acknowledgments (end)

\newpage


\bibliographystyle{plain}   
\bibliography{../../biblios/main.bib}

\input{qm2pi.rhodetails}

\end{document}



\end{document}

 

%\documentclass[12pt]{llncs}
%\documentclass{jktr}

\usepackage[pdftex]{hyperref}                   
\usepackage {listings}
\usepackage {mathpartir}
\usepackage{bcprules}
%\usepackage{listings}
                       
\usepackage{graphicx} 
%\usepackage[margins=2.5cm,nohead,nofoot]{geometry}
%\usepackage{geometry}
\usepackage{amsfonts}
\usepackage{amstext}
\usepackage{latexsym}
\usepackage{amssymb}
\usepackage{color}


%\include{myPreamble}
\documentclass[12pt]{llncs}
%\documentclass{jktr}

\usepackage[pdftex]{hyperref}                   
\usepackage {listings}
\usepackage {mathpartir}
\usepackage{bcprules}
%\usepackage{listings}
                       
\usepackage{graphicx} 
%\usepackage[margins=2.5cm,nohead,nofoot]{geometry}
%\usepackage{geometry}
\usepackage{amsfonts}
\usepackage{amstext}
\usepackage{latexsym}
\usepackage{amssymb}
\usepackage{color}


%\include{myPreamble}
\include{qm2pi.local} 

%\ifpdf
%\usepackage[pdftex]{graphicx}
%\else
%\usepackage{graphicx}
%\fi

 % \ifpdf
%  \usepackage{pdfsync}
%  \if


%\title{Brief Article}
%\author{David F. Snyder}
%\author{L.G. Meredith}

%\address{Dept. of Math., Texas State University--San Marcos, San Marcos, TX 78666}
       
\pagestyle{empty}


\begin{document}

\lstset{language=[Objective]Caml,frame=shadowbox}

\input{qm2pi.front}

% section front matter (end)

\input{qm2pi.intro} 
 
% section introduction (end)

% \input{qm2pi.knotations} 

% section notation (end)

\input{qm2pi.process.calculi} 

% section concurrent_process_calculi_and_spatial_logics_ (end)
    
%\input{qm2pi.knots2pi} 

%\input{qm2pi.trefoil} 

%\input{qm2pi.mainthm} 

% subsection basic_interpretation (end)

%\input{qm2pi.rho.presentation} 
\subsection{The syntax and semantics of the notation system}\label{sub:the_syntax_and_semantics_of_the_notation_system} % (fold)

We now summarize a technical presentation of the calculus that
embodies our theory of dynamics. The typical presentation of such a
calculus follows the style of giving generators and relations on
them. The grammar, below, describing term constructors, freely
generates the set of processes, $\Proc$. This set is then quotiented
by a relation known as structural congruence and it is over this set
that the notion of dynamics is expressed. This presentation is
essentially that of \cite{MeredithR05} with the addition of
polyadicity and summation. For readability we have relegated some of
the technical subtleties to an appendix.

\subsubsection{Process grammar}\label{subsub:process_grammar}

\begin{mathpar}
  \inferrule* [lab=synchronization] {} {{M} \bc \pzero \;|\; x?F \;|\; x!C }
  \and
  \inferrule* [lab=abstraction] {} {{F} \bc (x)P}
  \and
  \inferrule* [lab=concretion] {} {{C} \bc \langle Q \rangle}
  \and
  \inferrule* [lab=process] {} {{P,Q} \bc M \;| \;P|Q \;|\; @{x}}
  \and
  \inferrule* [lab=name] {} {{x} \bc \quotep{P}}
\end{mathpar} 

Note that $\vec{x}$ (resp. $\vec{P}$) denotes a vector of names
(resp. processes) of length $|\vec{x}|$ (resp. $|\vec{P}|$). We adopt
the following useful abbreviations.

\begin{mathpar}
   x?(\vec{y}).P := x.(\vec{y})P \and  x\clift{\vec{P}} := x.\clift{\vec{P}}
   \and x!(y) := \lift{x}{\dropn{y}}
   \and \Pi_{i=0}^{n-1}P_i := P_0 | \ldots | P_{n-1}
\end{mathpar}

\subsubsection{Structural congruence}

\paragraph{Free and bound names and alpha-equivalence.} At the
core of structural equivalence is alpha-equivalence which identifies
process that are the same up to a change of variable. Formally, we
recognize the distinction between free and bound names. The free names
of a process, $\freenames{P}$, may be calculated recursively as
follows:

\begin{mathpar}
\freenames{\pzero} := \emptyset
  \and \\
  \freenames{x?(y).P} := \{ x \} \cup (\freenames{P} \setminus \{ y \})
  \and 
  \freenames{x!\langle P \rangle} := \{ x \} \cup \{ P \} 
  \and \\
  \freenames{P|Q} := \freenames{P} \cup \freenames{Q}
  \and \\
  \freenames{@{x}} := \{ x \}
\end{mathpar}

$\pi$
$\quotep{\pi}$

$\freenames{-} : \pi \to \mathcal{P}(\quotep{\pi})$

\begin{eqnarray*}
  \freenames{\pzero} & := & \emptyset \\
  \freenames{x?(y).P} & := & \{ x \} \cup (\freenames{P} \setminus \{ y \}) \\
  \freenames{x!\langle P \rangle} & := & \{ x \} \cup \{ P \} \\
  \freenames{P|Q} & := & \freenames{P} \cup \freenames{Q} \\
  \freenames{\dropn{x}} & := & \{ x \}
\end{eqnarray*}

The bound names of a process, $\boundnames{P}$, are those names occurring in $P$
that are not free. For example, in $x?(y).0$, the name $x$ is free, while $y$ is bound.

\begin{mathpar}
  \inferrule* [lab=monoidal-laws] {} { P|Q \equiv Q|P \and P|0 \equiv P \and P|(Q|R) \equiv (P|Q)|R }
\end{mathpar}

\begin{mathpar}
  \inferrule* [lab=alpha-equivalence] {} { (x)P \equiv (y)P\{y/x\} \and y \not\in \freenames{P} }
\end{mathpar}

\begin{definition}
Then two processes, $P,Q$, are alpha-equivalent if $P = Q\{\vec{y}/\vec{x}\}$ for
some $\vec{x} \in \boundnames{Q},\vec{y} \in \boundnames{P}$, where $Q\{\vec{y}/\vec{x}\}$
denotes the capture-avoiding substitution of $\vec{y}$ for $\vec{x}$ in $Q$.
\end{definition}

\begin{definition}
  The {\em structural congruence} \cite{SangiorgiWalker} , $\equiv$,
  between processes is the least congruence containing
  alpha-equivalence, satisfying the abelian monoid laws
  (associativity, commutativity and $\pzero$ as identity) for parallel
  composition $|$ and for summation $+$.
\end{definition}

\subsection{Name equivalence}

We take name equivalence, written $\nameeq$, to be the smallest
equivalence relation generated by the following rules.

\begin{mathpar}
\inferrule*[lab=Quote-drop]
{ }
{ \quotep{@{x}} \nameeq x }

\inferrule*[lab=Struct-equiv]
{ P \scong Q }
{ \quotep{P} \nameeq \quotep{Q} }
\end{mathpar}

The astute reader will have noticed that the mutual recursion of names
and processes imposes a mutual recursion on alpha-equivalence and
structural equivalence via name-equivalence. Fortunately, all of this
works out pleasantly and we may calculate in the natural way, free of
concern. The reader interested in the details is referred to the
appendix \ref{appendix:rho_details}.

\subsection{Substitution}

We use $\Proc$ for the set of processes, $\QProc$ for the set of
names, and $\id{\{}\vec{y} / \vec{x} \id{\}}$ to denote partial maps,
$s : \QProc \rightarrow \QProc$. A map, $s$ lifts, uniquely, to a map
on process terms, $\widehat{s} : \Proc \rightarrow \Proc$ by the
following equations.

\begin{mathpar}
  (0) \psubstp{Q}{P} := 0 \\
  (R \juxtap S) \psubstp{Q}{P}
  :=    
  (R)\psubstp{Q}{P} \juxtap (S) \psubstp{Q}{P} \\
  (x?(y).R) \psubstp{Q}{P}    
  :=    
  (x)\substp{Q}{P} (z)\concat( (R \psubstn{z}{y}) \psubstp{Q}{P} ) \\
  (\lift{x}{R}) \psubstp{Q}{P}  
  :=
  \lift{(x)\substp{Q}{P}}{ R \psubstp{Q}{P} } \\
%   (\dropn{x})  \psubstp{Q}{P}       
%   := 
%   \left\{ 
%     \begin{array}{ccc} 
%       \dropn{\quotep{Q}} & & x \nameeq \quotep{P} \\
%       \dropn{x} & & otherwise \\
%     \end{array}
%   \right. 
  (\dropn{x})  \psubstp{Q}{P}       
  := 
  \left\{ 
    \begin{array}{ccc} 
      Q & & x \nameeq \quotep{P} \\
      \dropn{x} & & otherwise \\
    \end{array}
  \right.
\end{mathpar}
 

where

\begin{eqnarray}
  (x)\id{\{} \lpquote Q \rpquote / \lpquote P \rpquote \id{\}}            = 
  \left\{ 
    \begin{array}{ccc}
      \lpquote Q \rpquote & & x \nameeq \lpquote P \rpquote \\
      x & & otherwise \\
    \end{array}
  \right. \nonumber
\end{eqnarray}

and $z$ is chosen distinct from $\quotep{P}$, $\quotep{Q}$, the free
names in $Q$, and all the names in $R$. Our $\alpha$-equivalence will
be built in the standard way from this substitution.

\begin{remark}\label{rem:no_self_referential_names}
  One consequence of these definitions is that $\forall P. \quotep{P}
  \not\in \freenames{P}$.
\end{remark}

\subsection{ Dynamic quote: an example }

Anticipating something of what's to come, consider applying the
substitution, $\widehat{\id{\{}u / z \id{\}}}$, to the following pair
of processes, $\lift{w}{y!(z)}$ and $w[ \lpquote y!(z) \rpquote ]$.

\begin{eqnarray}
	\lift{w}{y!(z)}\widehat{\id{\{}u / z \id{\}}}
		& = &
		\lift{w}{y!(u)} \nonumber\\
	w[ \lpquote y!(z) \rpquote ] \widehat{ \id{\{}u / z \id{\}} }
		& = &
		w[ \lpquote y!(z) \rpquote ] \nonumber
\end{eqnarray}

Because the body of the process between quotes is impervious to
substitution, we get radically different answers. In fact, by
examining the first process in an input context,
e.g. $x?(z).\lift{w}{y!(z)}$, we see that the process under the lift
operator may be shaped by prefixed inputs binding a name inside it. In
this sense, the lift operator will be seen as a way to dynamically
construct processes before reifying them as names.

Finally equipped with these standard features we can present the
dynamics of the calculus.

\subsubsection{Operational semantics} 

Finally, we introduce the computational dynamics. What marks these
algebras as distinct from other more traditionally studied algebraic
structures, e.g. vector spaces or polynomial rings, is the manner in
which dynamics is captured. In traditional structures, dynamics is typically
expressed through morphisms between such structures, as in linear maps
between vector spaces or morphisms between rings. In algebras
associated with the semantics of computation, the dynamics is
expressed as part of the algebraic structure itself, through a
reduction reduction relation typically denoted by $\red$. Below, we
give a recursive presentation of this relation for the calculus used
in the encoding.

$\red \subseteq \pi \times \pi$
$\red : \pi \to \mathcal{P}(\pi)$

\begin{mathpar}
  \inferrule* [lab=Comm] { \textsf{match}( x_{src}, x_{trgt} ) } { x_{trgt}?(y)P \; | \; x_{src}!\langle {Q} \rangle \red P\{\quotep{Q}/y}\} }
  \and \\
  \inferrule* [lab=Par] {{P} \red {P}'} {{{P} | {Q}} \red {{P}' | {Q}}}
  \and
  \inferrule* [lab=Equiv]{{{P} \scong {P}'} \andalso {{P}' \red {Q}'} \andalso {{Q}' \scong {Q}}}{{P} \red {Q}}
\end{mathpar}

\begin{eqnarray*}
  match_{\equiv} (\quotep{P},\quotep{Q}) & := & P \equiv Q \\
  match_{\dagger}(\quotep{P},\quotep{Q}) & := & \forall R. P|Q \red^{*} R => R \red^{*} 0 \\
  match_{K}(\quotep{P},\quotep{Q}) & := & K \mbox{ for some context } K
\end{eqnarray*}

$u?(x)P | u!\langle Q \rangle \red P\{\quotep{Q}/x\}$

%We write $\wred$ for $\red^*$, and $P\red$ if $\exists Q $ such that $ P \red Q$.
We write $P\red$ if $\exists Q $ such that $ P \red Q$ and $P\not\red$, otherwise.

\section{Replication}

As mentioned before, it is known that replication (and hence
recursion) can be implemented in a higher-order process algebra
\cite{SangiorgiWalker}. As our first example of calculation with the
machinery thus far presented we give the construction explicitly in
the {\rhoc}.

\begin{eqnarray}
	D_{x} & := & \prefix{x}{y}{(\binpar{\outputp{x}{y}}{@{y}})} \nonumber\\
	\bangp_{x}{P} & := & \binpar{{x}!\langle{\binpar{D_{x}}{P}}\rangle}{D_{x}} \nonumber
\end{eqnarray}

\begin{eqnarray}
	\bangp_{x}{P} & & \nonumber\\
	=
	& {x}!\langle{(\prefix{x}{y}{(\outputp{x}{y} | @{y})) | P}}\rangle 
	      | \prefix{x}{y}{(\outputp{x}{y} | @{y})} & \nonumber\\
	\red
	& (\outputp{x}{y} | @{y})\substn{\quotep{(\prefix{x}{y}{(@{y} | \outputp{x}{y})) | P}}}{y} & \nonumber\\
	=
	& \outputp{x}{\quotep{(\prefix{x}{y}{(\outputp{x}{y} | @{y})) | P}}}
	  | {(\prefix{x}{y}{(\outputp{x}{y} | @{y})) | P}} & \nonumber\\
	\red
	& \ldots & \nonumber\\
	\red^*
	& P | P | \ldots & \nonumber
\end{eqnarray}

Of course, this encoding, as an implementation, runs away, unfolding
$\bangp{P}$ eagerly. A lazier and more implementable replication
operator, restricted to input-guarded processes, may be obtained as follows.

\begin{eqnarray}
\bangp{\prefix{u}{v}{P}} 
	:= 
	\binpar{\lift{x}{\prefix{u}{v}{(\binpar{D(x)}{P})}}}{D(x)} \nonumber
\end{eqnarray}

\begin{remark}
  Note that the lazier definition still does not deal with summation
  or mixed summation (i.e. sums over input and output). The reader is
  invited to construct definitions of replication that deal with these
  features. 

  Further, the definitions are parameterized in a name, $x$. Can you,
  gentle reader, make a definition that eliminates this parameter and
  guarantees no accidental interaction between the replication
  machinery and the process being replicated -- i.e. no accidental
  sharing of names used by the process to get its work done and the
  name(s) used by the replication to effect copying. This latter
  revision of the definition of replication is crucial to obtaining
  the expected identity $!!P \sim !P$.
\end{remark}

\begin{remark}\label{rem:paradoxical_combinator}
  The reader familiar with the lambda calculus will have noticed the
  similarity between $D$ and the paradoxical combinator.

  [Ed. note: the existence of this seems to suggest we have to be more
  restrictive on the set of processes and names we admit if we are to
  support no-cloning.]
\end{remark}

\subsubsection{Bisimulation}

The computational dynamics gives rise to another kind of equivalence,
the equivalence of computational behavior. As previously mentioned
this is typically captured \emph{via} some form of bisimulation.

% The notion we use in this paper is weak barbed bisimulation
% \cite{milner91polyadicpi}.

The notion we use in this paper is derived from weak barbed
bisimulation \cite{milner91polyadicpi}. 

\begin{definition}
An \emph{observation relation}, $\downarrow_{\mathcal N}$, over a set
of names, $\mathcal N$, is the smallest relation satisfying the rules
below.

\infrule[Out-barb]{y \in {\mathcal N}, \; x \nameeq y}
		  {\outputp{x}{v} \downarrow_{\mathcal N} x}
\infrule[Par-barb]{\mbox{$P\downarrow_{\mathcal N} x$ or $Q\downarrow_{\mathcal N} x$}}
		  {\binpar{P}{Q} \downarrow_{\mathcal N} x}

We write $P \Downarrow_{\mathcal N} x$ if there is $Q$ such that 
$P \wred Q$ and $Q \downarrow_{\mathcal N} x$.
\end{definition}

\begin{definition}
%\label{def.bbisim}
An  ${\mathcal N}$-\emph{barbed bisimulation} over a set of names, ${\mathcal N}$, is a symmetric binary relation 
${\mathcal S}_{\mathcal N}$ between agents such that $P\rel{S}_{\mathcal N}Q$ implies:
\begin{enumerate}
\item If $P \red P'$ then $Q \wred Q'$ and $P'\rel{S}_{\mathcal N} Q'$.
\item If $P\downarrow_{\mathcal N} x$, then $Q\Downarrow_{\mathcal N} x$.
\end{enumerate}
$P$ is ${\mathcal N}$-barbed bisimilar to $Q$, written
$P \wbbisim_{\mathcal N} Q$, if $P \rel{S}_{\mathcal N} Q$ for some ${\mathcal N}$-barbed bisimulation ${\mathcal S}_{\mathcal N}$.
\end{definition}

$\mathcal{R} \subseteq \pi \times \pi$

$P \mathcal{R} Q => \forall P'. P \red P' \Rightarrow \exists Q'. Q \red Q', P' \mathcal{R} Q'$

$P \vdash x \Rightarrow Q \vdash x$

\begin{mathpar}
  \inferrule*[lab=Out-barb]{x \nameeq y}{{y}!\langle{Q}\rangle \vdash x}
  \and
  \inferrule*[lab=Par-barb]{\mbox{$P\vdash x$ or $Q\vdash x$}}{\binpar{P}{Q} \vdash x}
\end{mathpar}

\subsubsection{Contexts}

One of the principle advantages of computational calculi like the
$\pi$-calculus is a well-defined notion of context,
contextual-equivalence and a correlation between
contextual-equivalence and notions of bisimulation. The notion of
context allows the decomposition of a process into (sub-)process and
its syntactic environment, its context. Thus, a context may be
thought of as a process with a ``hole'' (written $\Box$) in it. The
application of a context $M$ to a process $P$, written $M[P]$, is
tantamount to filling the hole in $M$ with $P$. In this paper we do
not need the full weight of this theory, but do make use of the notion
of context in the proof the main theorem. 

\begin{mathpar}
  \inferrule* [lab=summation] {} {{M_{M},M_{N}} \bc \Box \;|\; x.M_{A} \;|\; M_{M}+M_{N}}
  \and
  \inferrule* [lab=agent] {} {{M_{A}} \bc (\vec{x})M_{P} \;| \; \clift{P_0,\ldots,M_{P},\ldots,P_N}}
  \and \\
  \inferrule* [lab=process] {} {{M_{P}} \bc M_{N} \;| \;P|M_{P} }
\end{mathpar} 

\begin{mathpar}
  \inferrule* [lab=sychronization] {} {M_{N} \bc \Box \;|\; x?M_{F} \;|\; x!M_{C}}
  \and
  \inferrule* [lab=abstraction] {} {{M_{F}} \bc (x)M_{P} }
  \and
  \inferrule* [lab=concretion] {} {{M_{C}} \bc \langle M_{P} \rangle }
  \and \\
  \inferrule* [lab=process] {} {{M_{P}} \bc M_{N} \;| \;P|M_{P} }
\end{mathpar}

\begin{definition}[contextual application] Given a context $M$, and
  process $P$, we define the \emph{contextual application}, $M[P] :=
  M\{P/\Box\}$. That is, the contextual application of M to P is the
  substitution of $P$ for $\Box$ in $M$.
\end{definition}

$\meaningof{-} : L \to \mathcal{P}(\pi)$

\begin{mathpar}
  \inferrule* [lab=collection] {} {\meaningof{true} = \pi, \and \meaningof{~E} = \pi \setminus \meaningof{E}, \and \meaningof{E_{1} \& E_{2}} = \meaningof{E_{1}} \cap \meaningof{E_{2}}}
\end{mathpar}

\begin{mathpar}
  \inferrule* [lab=structure] {} {\meaningof{0} = \{ P \in \pi | P \equiv 0 \}, \and \\ \meaningof{E_1 | E_2} = \{ P \in \pi | P \equiv P_{1} | P_{2}, P_{1} \in \meaningof{E_{1}}, P_{2} \in \meaningof{E_2}\} }
\end{mathpar}

\begin{mathpar}
 \inferrule* [lab=behavior] {} {\meaningof{\langle a?b \rangle E} = \{ P \in \pi | P \equiv Q | u?(y)P', \\ \and \\\\ \and \\ \;\;\; u \in \meaningof{a}, \forall z.P'\{z/y\} \in \meaningof{E\{z/b\}}\}, \and \\ \meaningof{a!E} = \{ P \in \pi | P \equiv Q | x!\langle P' \rangle, x \in \meaningof{a} P' \in \meaningof{E}\} }
\end{mathpar}

\begin{mathpar}
 \inferrule* [lab=nominal] {} {\meaningof{\quotep{E}} = \{ \quotep{P} \in \quotep{\pi} | P \in \meaningof{E} \}, \and \meaningof{\quotep{P}} = \{ \quotep{Q} \in \quotep{\pi} | P \equiv Q \} \and \\ \meaningof{@\quotep{E}} = \{ P \in \pi | P \equiv @x, x \in \meaningof{E} \}}
\end{mathpar}

\begin{eqnarray*}
  \\
  \meaningof{-} : TS \to ST
\end{eqnarray*}

\begin{eqnarray*}
  \\
  L : TS \to ST
\end{eqnarray*}

\begin{eqnarray*}
  \\
  P \models E \iff P \in \meaningof{E}
\end{eqnarray*}

\begin{eqnarray*}
  P \approx_{L} Q \iff \forall E \in L. P \models E \iff Q \models E
\end{eqnarray*}

\begin{eqnarray*}
  P \approx_{K} Q
\end{eqnarray*}

\begin{eqnarray*}
  P \approx Q
\end{eqnarray*}

$\approx_{K} = \approx = \approx_{L}$

\subsubsection{Contextual duality}

Note that contexts extend the quotation operation to a family of
operations from processes to names. Given a context, $M$, we can
define a \emph{nominal context}, $\quotep{M}$ by $\quotep{M}[P] :=
\quotep{M[P]}$. To foreshadow what is to come we observe that these
operations enjoy a duality with processes very much like the duality
between vectors and maps from vectors to scalars.

Further, because the calculus is essentially higher-order, we have a
correspondence between contexts and processes. More specifically,
given a name $x$ and a context $M$ we can construct $M^{*}_{x}$ such
that 

\begin{mathpar}
  M^{*}_{x} | \lift{x}{P} \red M[P]
\end{mathpar}

namely,

\begin{mathpar}
  M^{*}_{x} := x?(u).M[\dropn{u}]
\end{mathpar}

The dependence of $M^{*}_{x}$ on a name makes it an abstraction, 

\begin{mathpar}
  M^{*} := (x)x?(u).M[\dropn{u}]
\end{mathpar}

\subsection{Additional notation}

It will sometimes be convenient to denote the process a name
quotes. We already have the notation $x = \quotep{P}$, but it will be
convenient to introduce an alternate notation, $\procn{x}$, when we
want to emphasize the connection to the use of the name. Note that, by
virtue of name equivalence, $\quotep{\procn{x}} \nameeq x$; so, the
notation is consistent with previous definitions.

Further, because names have structure it is possible to effect
substitutions on the basis of that structure. This means we need to
upgrade our notation for substitutions, which we accomplish by
adapting comprehension notation. Thus,

\begin{mathpar}
  P\{ y / x : x \in S \}
\end{mathpar}

is interpreted to mean the process derived from P by replacing (in a
capture-avoiding manner) each occurrence of $x$ in $S$ by $y$. For example,

\begin{mathpar}
  P\{ \quotep{\procn{x}|\procn{x}} / x : x \in \freenames{P} \}
\end{mathpar}

will replace each (occurrence) of a free name $x$ in $P$ by
$\quotep{\procn{x}|\procn{x}}$.

Also, we will avail ourselves of the notation $x^{L}$ and $x^{R}$ to
denote injections of a name into disjoint copies of the name
space. There are numerous ways to accomplish this. One example can be
found in \cite{MeredithR05}. This notation overloads to vectors of
names: $\vec{x}^{\pi} := (x_{i}^{\pi} \; : \; 0 \leq i < |\vec{x}| )$ where $\pi \in \{L,R\}$.

We also use $P^{\Box} := P|\Box$.

In \cite{MeredithR05} an interpretation of the new operator is
given. It turns out that there are several possible interpretations
all enjoying the requisite algebraic properties of the operator (see
\cite{milner91polyadicpi}). We will therefore make liberal use of
$(\nu\; \vec{x})P$.

% subsection the_syntax_and_semantics_of_the_notation_system (end)   

\input{qm2pi.qmops} 

\input{qm2pi.sterngerlach} 

\input{qm2pi.metric} 

% section concurrent_process_calculi (end)

%\input{qm2pi.proofsketch}

% section proof sketch (end)

%\input{qm2pi.slviaknots} 

% section spatial logic via knots (end)

\input{qm2pi.conclusion}

% section conclusion (end)

%\input{qm2pi.dtcodes} 

% section wiring algorithm (end)

\input{qm2pi.ack} 

% section acknowledgments (end)

\newpage


\bibliographystyle{plain}   
\bibliography{../../biblios/main.bib}

\input{qm2pi.rhodetails}

\end{document}

 

%\ifpdf
%\usepackage[pdftex]{graphicx}
%\else
%\usepackage{graphicx}
%\fi

 % \ifpdf
%  \usepackage{pdfsync}
%  \if


%\title{Brief Article}
%\author{David F. Snyder}
%\author{L.G. Meredith}

%\address{Dept. of Math., Texas State University--San Marcos, San Marcos, TX 78666}
       
\pagestyle{empty}


\begin{document}

\lstset{language=[Objective]Caml,frame=shadowbox}

\documentclass[12pt]{llncs}
%\documentclass{jktr}

\usepackage[pdftex]{hyperref}                   
\usepackage {listings}
\usepackage {mathpartir}
\usepackage{bcprules}
%\usepackage{listings}
                       
\usepackage{graphicx} 
%\usepackage[margins=2.5cm,nohead,nofoot]{geometry}
%\usepackage{geometry}
\usepackage{amsfonts}
\usepackage{amstext}
\usepackage{latexsym}
\usepackage{amssymb}
\usepackage{color}


%\include{myPreamble}
\include{qm2pi.local} 

%\ifpdf
%\usepackage[pdftex]{graphicx}
%\else
%\usepackage{graphicx}
%\fi

 % \ifpdf
%  \usepackage{pdfsync}
%  \if


%\title{Brief Article}
%\author{David F. Snyder}
%\author{L.G. Meredith}

%\address{Dept. of Math., Texas State University--San Marcos, San Marcos, TX 78666}
       
\pagestyle{empty}


\begin{document}

\lstset{language=[Objective]Caml,frame=shadowbox}

\input{qm2pi.front}

% section front matter (end)

\input{qm2pi.intro} 
 
% section introduction (end)

% \input{qm2pi.knotations} 

% section notation (end)

\input{qm2pi.process.calculi} 

% section concurrent_process_calculi_and_spatial_logics_ (end)
    
%\input{qm2pi.knots2pi} 

%\input{qm2pi.trefoil} 

%\input{qm2pi.mainthm} 

% subsection basic_interpretation (end)

%\input{qm2pi.rho.presentation} 
\subsection{The syntax and semantics of the notation system}\label{sub:the_syntax_and_semantics_of_the_notation_system} % (fold)

We now summarize a technical presentation of the calculus that
embodies our theory of dynamics. The typical presentation of such a
calculus follows the style of giving generators and relations on
them. The grammar, below, describing term constructors, freely
generates the set of processes, $\Proc$. This set is then quotiented
by a relation known as structural congruence and it is over this set
that the notion of dynamics is expressed. This presentation is
essentially that of \cite{MeredithR05} with the addition of
polyadicity and summation. For readability we have relegated some of
the technical subtleties to an appendix.

\subsubsection{Process grammar}\label{subsub:process_grammar}

\begin{mathpar}
  \inferrule* [lab=synchronization] {} {{M} \bc \pzero \;|\; x?F \;|\; x!C }
  \and
  \inferrule* [lab=abstraction] {} {{F} \bc (x)P}
  \and
  \inferrule* [lab=concretion] {} {{C} \bc \langle Q \rangle}
  \and
  \inferrule* [lab=process] {} {{P,Q} \bc M \;| \;P|Q \;|\; @{x}}
  \and
  \inferrule* [lab=name] {} {{x} \bc \quotep{P}}
\end{mathpar} 

Note that $\vec{x}$ (resp. $\vec{P}$) denotes a vector of names
(resp. processes) of length $|\vec{x}|$ (resp. $|\vec{P}|$). We adopt
the following useful abbreviations.

\begin{mathpar}
   x?(\vec{y}).P := x.(\vec{y})P \and  x\clift{\vec{P}} := x.\clift{\vec{P}}
   \and x!(y) := \lift{x}{\dropn{y}}
   \and \Pi_{i=0}^{n-1}P_i := P_0 | \ldots | P_{n-1}
\end{mathpar}

\subsubsection{Structural congruence}

\paragraph{Free and bound names and alpha-equivalence.} At the
core of structural equivalence is alpha-equivalence which identifies
process that are the same up to a change of variable. Formally, we
recognize the distinction between free and bound names. The free names
of a process, $\freenames{P}$, may be calculated recursively as
follows:

\begin{mathpar}
\freenames{\pzero} := \emptyset
  \and \\
  \freenames{x?(y).P} := \{ x \} \cup (\freenames{P} \setminus \{ y \})
  \and 
  \freenames{x!\langle P \rangle} := \{ x \} \cup \{ P \} 
  \and \\
  \freenames{P|Q} := \freenames{P} \cup \freenames{Q}
  \and \\
  \freenames{@{x}} := \{ x \}
\end{mathpar}

$\pi$
$\quotep{\pi}$

$\freenames{-} : \pi \to \mathcal{P}(\quotep{\pi})$

\begin{eqnarray*}
  \freenames{\pzero} & := & \emptyset \\
  \freenames{x?(y).P} & := & \{ x \} \cup (\freenames{P} \setminus \{ y \}) \\
  \freenames{x!\langle P \rangle} & := & \{ x \} \cup \{ P \} \\
  \freenames{P|Q} & := & \freenames{P} \cup \freenames{Q} \\
  \freenames{\dropn{x}} & := & \{ x \}
\end{eqnarray*}

The bound names of a process, $\boundnames{P}$, are those names occurring in $P$
that are not free. For example, in $x?(y).0$, the name $x$ is free, while $y$ is bound.

\begin{mathpar}
  \inferrule* [lab=monoidal-laws] {} { P|Q \equiv Q|P \and P|0 \equiv P \and P|(Q|R) \equiv (P|Q)|R }
\end{mathpar}

\begin{mathpar}
  \inferrule* [lab=alpha-equivalence] {} { (x)P \equiv (y)P\{y/x\} \and y \not\in \freenames{P} }
\end{mathpar}

\begin{definition}
Then two processes, $P,Q$, are alpha-equivalent if $P = Q\{\vec{y}/\vec{x}\}$ for
some $\vec{x} \in \boundnames{Q},\vec{y} \in \boundnames{P}$, where $Q\{\vec{y}/\vec{x}\}$
denotes the capture-avoiding substitution of $\vec{y}$ for $\vec{x}$ in $Q$.
\end{definition}

\begin{definition}
  The {\em structural congruence} \cite{SangiorgiWalker} , $\equiv$,
  between processes is the least congruence containing
  alpha-equivalence, satisfying the abelian monoid laws
  (associativity, commutativity and $\pzero$ as identity) for parallel
  composition $|$ and for summation $+$.
\end{definition}

\subsection{Name equivalence}

We take name equivalence, written $\nameeq$, to be the smallest
equivalence relation generated by the following rules.

\begin{mathpar}
\inferrule*[lab=Quote-drop]
{ }
{ \quotep{@{x}} \nameeq x }

\inferrule*[lab=Struct-equiv]
{ P \scong Q }
{ \quotep{P} \nameeq \quotep{Q} }
\end{mathpar}

The astute reader will have noticed that the mutual recursion of names
and processes imposes a mutual recursion on alpha-equivalence and
structural equivalence via name-equivalence. Fortunately, all of this
works out pleasantly and we may calculate in the natural way, free of
concern. The reader interested in the details is referred to the
appendix \ref{appendix:rho_details}.

\subsection{Substitution}

We use $\Proc$ for the set of processes, $\QProc$ for the set of
names, and $\id{\{}\vec{y} / \vec{x} \id{\}}$ to denote partial maps,
$s : \QProc \rightarrow \QProc$. A map, $s$ lifts, uniquely, to a map
on process terms, $\widehat{s} : \Proc \rightarrow \Proc$ by the
following equations.

\begin{mathpar}
  (0) \psubstp{Q}{P} := 0 \\
  (R \juxtap S) \psubstp{Q}{P}
  :=    
  (R)\psubstp{Q}{P} \juxtap (S) \psubstp{Q}{P} \\
  (x?(y).R) \psubstp{Q}{P}    
  :=    
  (x)\substp{Q}{P} (z)\concat( (R \psubstn{z}{y}) \psubstp{Q}{P} ) \\
  (\lift{x}{R}) \psubstp{Q}{P}  
  :=
  \lift{(x)\substp{Q}{P}}{ R \psubstp{Q}{P} } \\
%   (\dropn{x})  \psubstp{Q}{P}       
%   := 
%   \left\{ 
%     \begin{array}{ccc} 
%       \dropn{\quotep{Q}} & & x \nameeq \quotep{P} \\
%       \dropn{x} & & otherwise \\
%     \end{array}
%   \right. 
  (\dropn{x})  \psubstp{Q}{P}       
  := 
  \left\{ 
    \begin{array}{ccc} 
      Q & & x \nameeq \quotep{P} \\
      \dropn{x} & & otherwise \\
    \end{array}
  \right.
\end{mathpar}
 

where

\begin{eqnarray}
  (x)\id{\{} \lpquote Q \rpquote / \lpquote P \rpquote \id{\}}            = 
  \left\{ 
    \begin{array}{ccc}
      \lpquote Q \rpquote & & x \nameeq \lpquote P \rpquote \\
      x & & otherwise \\
    \end{array}
  \right. \nonumber
\end{eqnarray}

and $z$ is chosen distinct from $\quotep{P}$, $\quotep{Q}$, the free
names in $Q$, and all the names in $R$. Our $\alpha$-equivalence will
be built in the standard way from this substitution.

\begin{remark}\label{rem:no_self_referential_names}
  One consequence of these definitions is that $\forall P. \quotep{P}
  \not\in \freenames{P}$.
\end{remark}

\subsection{ Dynamic quote: an example }

Anticipating something of what's to come, consider applying the
substitution, $\widehat{\id{\{}u / z \id{\}}}$, to the following pair
of processes, $\lift{w}{y!(z)}$ and $w[ \lpquote y!(z) \rpquote ]$.

\begin{eqnarray}
	\lift{w}{y!(z)}\widehat{\id{\{}u / z \id{\}}}
		& = &
		\lift{w}{y!(u)} \nonumber\\
	w[ \lpquote y!(z) \rpquote ] \widehat{ \id{\{}u / z \id{\}} }
		& = &
		w[ \lpquote y!(z) \rpquote ] \nonumber
\end{eqnarray}

Because the body of the process between quotes is impervious to
substitution, we get radically different answers. In fact, by
examining the first process in an input context,
e.g. $x?(z).\lift{w}{y!(z)}$, we see that the process under the lift
operator may be shaped by prefixed inputs binding a name inside it. In
this sense, the lift operator will be seen as a way to dynamically
construct processes before reifying them as names.

Finally equipped with these standard features we can present the
dynamics of the calculus.

\subsubsection{Operational semantics} 

Finally, we introduce the computational dynamics. What marks these
algebras as distinct from other more traditionally studied algebraic
structures, e.g. vector spaces or polynomial rings, is the manner in
which dynamics is captured. In traditional structures, dynamics is typically
expressed through morphisms between such structures, as in linear maps
between vector spaces or morphisms between rings. In algebras
associated with the semantics of computation, the dynamics is
expressed as part of the algebraic structure itself, through a
reduction reduction relation typically denoted by $\red$. Below, we
give a recursive presentation of this relation for the calculus used
in the encoding.

$\red \subseteq \pi \times \pi$
$\red : \pi \to \mathcal{P}(\pi)$

\begin{mathpar}
  \inferrule* [lab=Comm] { \textsf{match}( x_{src}, x_{trgt} ) } { x_{trgt}?(y)P \; | \; x_{src}!\langle {Q} \rangle \red P\{\quotep{Q}/y}\} }
  \and \\
  \inferrule* [lab=Par] {{P} \red {P}'} {{{P} | {Q}} \red {{P}' | {Q}}}
  \and
  \inferrule* [lab=Equiv]{{{P} \scong {P}'} \andalso {{P}' \red {Q}'} \andalso {{Q}' \scong {Q}}}{{P} \red {Q}}
\end{mathpar}

\begin{eqnarray*}
  match_{\equiv} (\quotep{P},\quotep{Q}) & := & P \equiv Q \\
  match_{\dagger}(\quotep{P},\quotep{Q}) & := & \forall R. P|Q \red^{*} R => R \red^{*} 0 \\
  match_{K}(\quotep{P},\quotep{Q}) & := & K \mbox{ for some context } K
\end{eqnarray*}

$u?(x)P | u!\langle Q \rangle \red P\{\quotep{Q}/x\}$

%We write $\wred$ for $\red^*$, and $P\red$ if $\exists Q $ such that $ P \red Q$.
We write $P\red$ if $\exists Q $ such that $ P \red Q$ and $P\not\red$, otherwise.

\section{Replication}

As mentioned before, it is known that replication (and hence
recursion) can be implemented in a higher-order process algebra
\cite{SangiorgiWalker}. As our first example of calculation with the
machinery thus far presented we give the construction explicitly in
the {\rhoc}.

\begin{eqnarray}
	D_{x} & := & \prefix{x}{y}{(\binpar{\outputp{x}{y}}{@{y}})} \nonumber\\
	\bangp_{x}{P} & := & \binpar{{x}!\langle{\binpar{D_{x}}{P}}\rangle}{D_{x}} \nonumber
\end{eqnarray}

\begin{eqnarray}
	\bangp_{x}{P} & & \nonumber\\
	=
	& {x}!\langle{(\prefix{x}{y}{(\outputp{x}{y} | @{y})) | P}}\rangle 
	      | \prefix{x}{y}{(\outputp{x}{y} | @{y})} & \nonumber\\
	\red
	& (\outputp{x}{y} | @{y})\substn{\quotep{(\prefix{x}{y}{(@{y} | \outputp{x}{y})) | P}}}{y} & \nonumber\\
	=
	& \outputp{x}{\quotep{(\prefix{x}{y}{(\outputp{x}{y} | @{y})) | P}}}
	  | {(\prefix{x}{y}{(\outputp{x}{y} | @{y})) | P}} & \nonumber\\
	\red
	& \ldots & \nonumber\\
	\red^*
	& P | P | \ldots & \nonumber
\end{eqnarray}

Of course, this encoding, as an implementation, runs away, unfolding
$\bangp{P}$ eagerly. A lazier and more implementable replication
operator, restricted to input-guarded processes, may be obtained as follows.

\begin{eqnarray}
\bangp{\prefix{u}{v}{P}} 
	:= 
	\binpar{\lift{x}{\prefix{u}{v}{(\binpar{D(x)}{P})}}}{D(x)} \nonumber
\end{eqnarray}

\begin{remark}
  Note that the lazier definition still does not deal with summation
  or mixed summation (i.e. sums over input and output). The reader is
  invited to construct definitions of replication that deal with these
  features. 

  Further, the definitions are parameterized in a name, $x$. Can you,
  gentle reader, make a definition that eliminates this parameter and
  guarantees no accidental interaction between the replication
  machinery and the process being replicated -- i.e. no accidental
  sharing of names used by the process to get its work done and the
  name(s) used by the replication to effect copying. This latter
  revision of the definition of replication is crucial to obtaining
  the expected identity $!!P \sim !P$.
\end{remark}

\begin{remark}\label{rem:paradoxical_combinator}
  The reader familiar with the lambda calculus will have noticed the
  similarity between $D$ and the paradoxical combinator.

  [Ed. note: the existence of this seems to suggest we have to be more
  restrictive on the set of processes and names we admit if we are to
  support no-cloning.]
\end{remark}

\subsubsection{Bisimulation}

The computational dynamics gives rise to another kind of equivalence,
the equivalence of computational behavior. As previously mentioned
this is typically captured \emph{via} some form of bisimulation.

% The notion we use in this paper is weak barbed bisimulation
% \cite{milner91polyadicpi}.

The notion we use in this paper is derived from weak barbed
bisimulation \cite{milner91polyadicpi}. 

\begin{definition}
An \emph{observation relation}, $\downarrow_{\mathcal N}$, over a set
of names, $\mathcal N$, is the smallest relation satisfying the rules
below.

\infrule[Out-barb]{y \in {\mathcal N}, \; x \nameeq y}
		  {\outputp{x}{v} \downarrow_{\mathcal N} x}
\infrule[Par-barb]{\mbox{$P\downarrow_{\mathcal N} x$ or $Q\downarrow_{\mathcal N} x$}}
		  {\binpar{P}{Q} \downarrow_{\mathcal N} x}

We write $P \Downarrow_{\mathcal N} x$ if there is $Q$ such that 
$P \wred Q$ and $Q \downarrow_{\mathcal N} x$.
\end{definition}

\begin{definition}
%\label{def.bbisim}
An  ${\mathcal N}$-\emph{barbed bisimulation} over a set of names, ${\mathcal N}$, is a symmetric binary relation 
${\mathcal S}_{\mathcal N}$ between agents such that $P\rel{S}_{\mathcal N}Q$ implies:
\begin{enumerate}
\item If $P \red P'$ then $Q \wred Q'$ and $P'\rel{S}_{\mathcal N} Q'$.
\item If $P\downarrow_{\mathcal N} x$, then $Q\Downarrow_{\mathcal N} x$.
\end{enumerate}
$P$ is ${\mathcal N}$-barbed bisimilar to $Q$, written
$P \wbbisim_{\mathcal N} Q$, if $P \rel{S}_{\mathcal N} Q$ for some ${\mathcal N}$-barbed bisimulation ${\mathcal S}_{\mathcal N}$.
\end{definition}

$\mathcal{R} \subseteq \pi \times \pi$

$P \mathcal{R} Q => \forall P'. P \red P' \Rightarrow \exists Q'. Q \red Q', P' \mathcal{R} Q'$

$P \vdash x \Rightarrow Q \vdash x$

\begin{mathpar}
  \inferrule*[lab=Out-barb]{x \nameeq y}{{y}!\langle{Q}\rangle \vdash x}
  \and
  \inferrule*[lab=Par-barb]{\mbox{$P\vdash x$ or $Q\vdash x$}}{\binpar{P}{Q} \vdash x}
\end{mathpar}

\subsubsection{Contexts}

One of the principle advantages of computational calculi like the
$\pi$-calculus is a well-defined notion of context,
contextual-equivalence and a correlation between
contextual-equivalence and notions of bisimulation. The notion of
context allows the decomposition of a process into (sub-)process and
its syntactic environment, its context. Thus, a context may be
thought of as a process with a ``hole'' (written $\Box$) in it. The
application of a context $M$ to a process $P$, written $M[P]$, is
tantamount to filling the hole in $M$ with $P$. In this paper we do
not need the full weight of this theory, but do make use of the notion
of context in the proof the main theorem. 

\begin{mathpar}
  \inferrule* [lab=summation] {} {{M_{M},M_{N}} \bc \Box \;|\; x.M_{A} \;|\; M_{M}+M_{N}}
  \and
  \inferrule* [lab=agent] {} {{M_{A}} \bc (\vec{x})M_{P} \;| \; \clift{P_0,\ldots,M_{P},\ldots,P_N}}
  \and \\
  \inferrule* [lab=process] {} {{M_{P}} \bc M_{N} \;| \;P|M_{P} }
\end{mathpar} 

\begin{mathpar}
  \inferrule* [lab=sychronization] {} {M_{N} \bc \Box \;|\; x?M_{F} \;|\; x!M_{C}}
  \and
  \inferrule* [lab=abstraction] {} {{M_{F}} \bc (x)M_{P} }
  \and
  \inferrule* [lab=concretion] {} {{M_{C}} \bc \langle M_{P} \rangle }
  \and \\
  \inferrule* [lab=process] {} {{M_{P}} \bc M_{N} \;| \;P|M_{P} }
\end{mathpar}

\begin{definition}[contextual application] Given a context $M$, and
  process $P$, we define the \emph{contextual application}, $M[P] :=
  M\{P/\Box\}$. That is, the contextual application of M to P is the
  substitution of $P$ for $\Box$ in $M$.
\end{definition}

$\meaningof{-} : L \to \mathcal{P}(\pi)$

\begin{mathpar}
  \inferrule* [lab=collection] {} {\meaningof{true} = \pi, \and \meaningof{~E} = \pi \setminus \meaningof{E}, \and \meaningof{E_{1} \& E_{2}} = \meaningof{E_{1}} \cap \meaningof{E_{2}}}
\end{mathpar}

\begin{mathpar}
  \inferrule* [lab=structure] {} {\meaningof{0} = \{ P \in \pi | P \equiv 0 \}, \and \\ \meaningof{E_1 | E_2} = \{ P \in \pi | P \equiv P_{1} | P_{2}, P_{1} \in \meaningof{E_{1}}, P_{2} \in \meaningof{E_2}\} }
\end{mathpar}

\begin{mathpar}
 \inferrule* [lab=behavior] {} {\meaningof{\langle a?b \rangle E} = \{ P \in \pi | P \equiv Q | u?(y)P', \\ \and \\\\ \and \\ \;\;\; u \in \meaningof{a}, \forall z.P'\{z/y\} \in \meaningof{E\{z/b\}}\}, \and \\ \meaningof{a!E} = \{ P \in \pi | P \equiv Q | x!\langle P' \rangle, x \in \meaningof{a} P' \in \meaningof{E}\} }
\end{mathpar}

\begin{mathpar}
 \inferrule* [lab=nominal] {} {\meaningof{\quotep{E}} = \{ \quotep{P} \in \quotep{\pi} | P \in \meaningof{E} \}, \and \meaningof{\quotep{P}} = \{ \quotep{Q} \in \quotep{\pi} | P \equiv Q \} \and \\ \meaningof{@\quotep{E}} = \{ P \in \pi | P \equiv @x, x \in \meaningof{E} \}}
\end{mathpar}

\begin{eqnarray*}
  \\
  \meaningof{-} : TS \to ST
\end{eqnarray*}

\begin{eqnarray*}
  \\
  L : TS \to ST
\end{eqnarray*}

\begin{eqnarray*}
  \\
  P \models E \iff P \in \meaningof{E}
\end{eqnarray*}

\begin{eqnarray*}
  P \approx_{L} Q \iff \forall E \in L. P \models E \iff Q \models E
\end{eqnarray*}

\begin{eqnarray*}
  P \approx_{K} Q
\end{eqnarray*}

\begin{eqnarray*}
  P \approx Q
\end{eqnarray*}

$\approx_{K} = \approx = \approx_{L}$

\subsubsection{Contextual duality}

Note that contexts extend the quotation operation to a family of
operations from processes to names. Given a context, $M$, we can
define a \emph{nominal context}, $\quotep{M}$ by $\quotep{M}[P] :=
\quotep{M[P]}$. To foreshadow what is to come we observe that these
operations enjoy a duality with processes very much like the duality
between vectors and maps from vectors to scalars.

Further, because the calculus is essentially higher-order, we have a
correspondence between contexts and processes. More specifically,
given a name $x$ and a context $M$ we can construct $M^{*}_{x}$ such
that 

\begin{mathpar}
  M^{*}_{x} | \lift{x}{P} \red M[P]
\end{mathpar}

namely,

\begin{mathpar}
  M^{*}_{x} := x?(u).M[\dropn{u}]
\end{mathpar}

The dependence of $M^{*}_{x}$ on a name makes it an abstraction, 

\begin{mathpar}
  M^{*} := (x)x?(u).M[\dropn{u}]
\end{mathpar}

\subsection{Additional notation}

It will sometimes be convenient to denote the process a name
quotes. We already have the notation $x = \quotep{P}$, but it will be
convenient to introduce an alternate notation, $\procn{x}$, when we
want to emphasize the connection to the use of the name. Note that, by
virtue of name equivalence, $\quotep{\procn{x}} \nameeq x$; so, the
notation is consistent with previous definitions.

Further, because names have structure it is possible to effect
substitutions on the basis of that structure. This means we need to
upgrade our notation for substitutions, which we accomplish by
adapting comprehension notation. Thus,

\begin{mathpar}
  P\{ y / x : x \in S \}
\end{mathpar}

is interpreted to mean the process derived from P by replacing (in a
capture-avoiding manner) each occurrence of $x$ in $S$ by $y$. For example,

\begin{mathpar}
  P\{ \quotep{\procn{x}|\procn{x}} / x : x \in \freenames{P} \}
\end{mathpar}

will replace each (occurrence) of a free name $x$ in $P$ by
$\quotep{\procn{x}|\procn{x}}$.

Also, we will avail ourselves of the notation $x^{L}$ and $x^{R}$ to
denote injections of a name into disjoint copies of the name
space. There are numerous ways to accomplish this. One example can be
found in \cite{MeredithR05}. This notation overloads to vectors of
names: $\vec{x}^{\pi} := (x_{i}^{\pi} \; : \; 0 \leq i < |\vec{x}| )$ where $\pi \in \{L,R\}$.

We also use $P^{\Box} := P|\Box$.

In \cite{MeredithR05} an interpretation of the new operator is
given. It turns out that there are several possible interpretations
all enjoying the requisite algebraic properties of the operator (see
\cite{milner91polyadicpi}). We will therefore make liberal use of
$(\nu\; \vec{x})P$.

% subsection the_syntax_and_semantics_of_the_notation_system (end)   

\input{qm2pi.qmops} 

\input{qm2pi.sterngerlach} 

\input{qm2pi.metric} 

% section concurrent_process_calculi (end)

%\input{qm2pi.proofsketch}

% section proof sketch (end)

%\input{qm2pi.slviaknots} 

% section spatial logic via knots (end)

\input{qm2pi.conclusion}

% section conclusion (end)

%\input{qm2pi.dtcodes} 

% section wiring algorithm (end)

\input{qm2pi.ack} 

% section acknowledgments (end)

\newpage


\bibliographystyle{plain}   
\bibliography{../../biblios/main.bib}

\input{qm2pi.rhodetails}

\end{document}



% section front matter (end)

\section{Introduction}\label{sec:introduction} % (fold)
In this draft of the material i am going to have to dispense with the
usual writing conventions adopted in papers on these topics. i'm going
to have adopt whatever tone i need at the time i'm writing up the
calculations. Sometimes this may be very conversational; others it may
be the barest mathematical grunts; others still it may be that i have
lifted text from one of my other papers because the exposition of some
point was better said there. i hope that my readers are not unduly put
out by this decision. i'm not doing this to flout convention or be
rebellious. i find these calculations very technically challenging. To
keep everything going technically, something has to give; i have to
let go of some cognitive burden. So, the academic writing style --
with all of its trade-offs in terms of facilitating technical
communication -- is what i'm letting go of. Perhaps subsequent drafts
can be tightened and polished, but for now, i'm going to speak as if
we were sitting together in a coffee shop with a laptop, wifi and a
pad of paper and a pencil.

So, here's what i have to say. We -- you and i, comfortably ensconced
in our coffee shop and well-equipped with our tools -- can realize and
carry out the calculations of quantum mechanics over a very different
formal theory of dynamics, a formal theory of dynamics that
corresponds to a theory of concurrent computation with
\emph{reflection}. It has the advantage that the underlying theory is
already `quantized', but supports analogues all of the continuuous
operations. Strikingly, this underlying theory has recently been
connected with a notion of metric that we can show, by calculating
together, coincides with the metric induced by the inner product.

There are a lot of reasons why you might be interested in seeing
calculations of this form. Here's why i'm interested. For the past
several centuries there has been no competitor to the ``Newtonian''
account of dynamics. As a result the predominant share of accounts of
dynamical systems and situations have had to be formulated in terms of
the Newtonian machinery. i view this as an intellectually dangerous
position to occupy. Everything, despite it's intrinsic shape, turns
into a nail to be hit with this hammer. Recently, however, the theory
of computation has matured to the point where we have candidates for
theories of dynamics that offer very different perspective on
reasoning about dynamical systems and situations. Testing these
candidates against very successful accounts of dynamical situations,
like quantum mechanics, is going to give us some sense of how mature
they are and some measure of the quality of these accounts of
dynamics.

\subsection{Summary of contributions and outline of paper}

So, we're going to develop an interpretation of the operations of
quantum mechanics normally interpreted by Hilbert spaces and
operators. We're going to do this over a theory of computation. Note
that this is very different than the usual quantum computation program
which develops notions of computation over quantum mechanics. Rather,
we are developing a story that aligns with Wheeler's slogan: It from
Bit. To do this we will first provide an account of the theory of
computation at play here. Then we will dive into a calculation-driven
interpretation of the operations of quantum mechanics.

The reason we take this approach is that -- until very recently --
there hasn't been an axiomatic account of quantum mechanics. As a
result there has been no sharp delineation of the mathematical theory
supporting interpretation of the physical theory and the physical
theory, itself. So, ambient features of the maths are free to be
exploited (or supressed) without a real accounting of their physical
relevance. There is no sharp statement ``here's the physical theory''
qua \emph{theory} and ``here's the mathematical interpretation''
enabling a judgment of how faithful the interpretation is -- apart
from experimental observation. When there is an axiomatic account we
can judge how well a given mathematical formalism supports an
interpretation of the axioms, independent of
experimentation. Likewise, we can judge how well we have captured our
physical evidence and experience with our axiomatics, independent of
any specific mathematical implementation, with accidental detail that
may or may not have physical significance. 

In lieu of a fully fleshed out and vetted axiomatic account of quantum
mechanics, interpreting the operational notions in service of modeling
physical systems will have to suffice. In other words, we are not in
the business of providing a model of Hilbert spaces and operators. We
are in the business of providing a model of quantum mechanics because
we are motivated by testing our notions of dynamics against physical
theory; and, the predictive calculations of the physical theory must
serve as the best formulation -- shy of a fully fleshed out axiomatic
account -- of the physical theory itself (as they have for scientific
theories since time immemorial). Put another way, despite a
whole-hearted commitment to an It-from-Bit ontology, we are firmly
aligned with the shut-up-and-calculate camp as the best way to obtain
results either from the physical perspective or as a quality assurance
measure of our fledgling theory of dynamics.

In detail, we present a reflective process calculus. Then we develop
intuitive correspondences between the notions available in this
calculus and the usual physical notions supporting quantum mechanical
calculations. Thus, 

\begin{table}[htp]
  \center{
    \fbox{
      \begin{tabular}{c|c}
        quantum mechanics & process calculus \\
        \hline
        scalar & name \\
        state vector & process \\
        dual & contextual duals \\
        matrix & formal sums of process-context-dual pairs \\
        orthogonality & process annihilation \\
        inner product & execution-formula + quoting
      \end{tabular}
    }
  }
  \caption{QM - process calculi correspondences}
\end{table}

Then we tighten up these intuitions to operational definitions. We
employ the Dirac notation as the best proxy we can find for an
abstract syntax of the quantum mechanical notions. The definitions we
develop put us in contact with equational constraints coming from the
theory that we demonstrate the definitions and calculations satisfy.

This puts us in a position to shut up and calculate for the
Stern-Gerlach experimental set up, showing how these predictive
calculations become calculations on processes in our theory of a
reflective process calculus.

Penultimately, we demonstrate that the notion of metric coming from
the inner product coincides with the notion of metric available from
the theory of bisimulation. This demonstration gives us the right to
think of space as arising from behavior. Finally, we consider where we
might go from the new vantage point we have obtained.

% section introduction (end) 
 
% section introduction (end)

% \documentclass[12pt]{llncs}
%\documentclass{jktr}

\usepackage[pdftex]{hyperref}                   
\usepackage {listings}
\usepackage {mathpartir}
\usepackage{bcprules}
%\usepackage{listings}
                       
\usepackage{graphicx} 
%\usepackage[margins=2.5cm,nohead,nofoot]{geometry}
%\usepackage{geometry}
\usepackage{amsfonts}
\usepackage{amstext}
\usepackage{latexsym}
\usepackage{amssymb}
\usepackage{color}


%\include{myPreamble}
\include{qm2pi.local} 

%\ifpdf
%\usepackage[pdftex]{graphicx}
%\else
%\usepackage{graphicx}
%\fi

 % \ifpdf
%  \usepackage{pdfsync}
%  \if


%\title{Brief Article}
%\author{David F. Snyder}
%\author{L.G. Meredith}

%\address{Dept. of Math., Texas State University--San Marcos, San Marcos, TX 78666}
       
\pagestyle{empty}


\begin{document}

\lstset{language=[Objective]Caml,frame=shadowbox}

\input{qm2pi.front}

% section front matter (end)

\input{qm2pi.intro} 
 
% section introduction (end)

% \input{qm2pi.knotations} 

% section notation (end)

\input{qm2pi.process.calculi} 

% section concurrent_process_calculi_and_spatial_logics_ (end)
    
%\input{qm2pi.knots2pi} 

%\input{qm2pi.trefoil} 

%\input{qm2pi.mainthm} 

% subsection basic_interpretation (end)

%\input{qm2pi.rho.presentation} 
\subsection{The syntax and semantics of the notation system}\label{sub:the_syntax_and_semantics_of_the_notation_system} % (fold)

We now summarize a technical presentation of the calculus that
embodies our theory of dynamics. The typical presentation of such a
calculus follows the style of giving generators and relations on
them. The grammar, below, describing term constructors, freely
generates the set of processes, $\Proc$. This set is then quotiented
by a relation known as structural congruence and it is over this set
that the notion of dynamics is expressed. This presentation is
essentially that of \cite{MeredithR05} with the addition of
polyadicity and summation. For readability we have relegated some of
the technical subtleties to an appendix.

\subsubsection{Process grammar}\label{subsub:process_grammar}

\begin{mathpar}
  \inferrule* [lab=synchronization] {} {{M} \bc \pzero \;|\; x?F \;|\; x!C }
  \and
  \inferrule* [lab=abstraction] {} {{F} \bc (x)P}
  \and
  \inferrule* [lab=concretion] {} {{C} \bc \langle Q \rangle}
  \and
  \inferrule* [lab=process] {} {{P,Q} \bc M \;| \;P|Q \;|\; @{x}}
  \and
  \inferrule* [lab=name] {} {{x} \bc \quotep{P}}
\end{mathpar} 

Note that $\vec{x}$ (resp. $\vec{P}$) denotes a vector of names
(resp. processes) of length $|\vec{x}|$ (resp. $|\vec{P}|$). We adopt
the following useful abbreviations.

\begin{mathpar}
   x?(\vec{y}).P := x.(\vec{y})P \and  x\clift{\vec{P}} := x.\clift{\vec{P}}
   \and x!(y) := \lift{x}{\dropn{y}}
   \and \Pi_{i=0}^{n-1}P_i := P_0 | \ldots | P_{n-1}
\end{mathpar}

\subsubsection{Structural congruence}

\paragraph{Free and bound names and alpha-equivalence.} At the
core of structural equivalence is alpha-equivalence which identifies
process that are the same up to a change of variable. Formally, we
recognize the distinction between free and bound names. The free names
of a process, $\freenames{P}$, may be calculated recursively as
follows:

\begin{mathpar}
\freenames{\pzero} := \emptyset
  \and \\
  \freenames{x?(y).P} := \{ x \} \cup (\freenames{P} \setminus \{ y \})
  \and 
  \freenames{x!\langle P \rangle} := \{ x \} \cup \{ P \} 
  \and \\
  \freenames{P|Q} := \freenames{P} \cup \freenames{Q}
  \and \\
  \freenames{@{x}} := \{ x \}
\end{mathpar}

$\pi$
$\quotep{\pi}$

$\freenames{-} : \pi \to \mathcal{P}(\quotep{\pi})$

\begin{eqnarray*}
  \freenames{\pzero} & := & \emptyset \\
  \freenames{x?(y).P} & := & \{ x \} \cup (\freenames{P} \setminus \{ y \}) \\
  \freenames{x!\langle P \rangle} & := & \{ x \} \cup \{ P \} \\
  \freenames{P|Q} & := & \freenames{P} \cup \freenames{Q} \\
  \freenames{\dropn{x}} & := & \{ x \}
\end{eqnarray*}

The bound names of a process, $\boundnames{P}$, are those names occurring in $P$
that are not free. For example, in $x?(y).0$, the name $x$ is free, while $y$ is bound.

\begin{mathpar}
  \inferrule* [lab=monoidal-laws] {} { P|Q \equiv Q|P \and P|0 \equiv P \and P|(Q|R) \equiv (P|Q)|R }
\end{mathpar}

\begin{mathpar}
  \inferrule* [lab=alpha-equivalence] {} { (x)P \equiv (y)P\{y/x\} \and y \not\in \freenames{P} }
\end{mathpar}

\begin{definition}
Then two processes, $P,Q$, are alpha-equivalent if $P = Q\{\vec{y}/\vec{x}\}$ for
some $\vec{x} \in \boundnames{Q},\vec{y} \in \boundnames{P}$, where $Q\{\vec{y}/\vec{x}\}$
denotes the capture-avoiding substitution of $\vec{y}$ for $\vec{x}$ in $Q$.
\end{definition}

\begin{definition}
  The {\em structural congruence} \cite{SangiorgiWalker} , $\equiv$,
  between processes is the least congruence containing
  alpha-equivalence, satisfying the abelian monoid laws
  (associativity, commutativity and $\pzero$ as identity) for parallel
  composition $|$ and for summation $+$.
\end{definition}

\subsection{Name equivalence}

We take name equivalence, written $\nameeq$, to be the smallest
equivalence relation generated by the following rules.

\begin{mathpar}
\inferrule*[lab=Quote-drop]
{ }
{ \quotep{@{x}} \nameeq x }

\inferrule*[lab=Struct-equiv]
{ P \scong Q }
{ \quotep{P} \nameeq \quotep{Q} }
\end{mathpar}

The astute reader will have noticed that the mutual recursion of names
and processes imposes a mutual recursion on alpha-equivalence and
structural equivalence via name-equivalence. Fortunately, all of this
works out pleasantly and we may calculate in the natural way, free of
concern. The reader interested in the details is referred to the
appendix \ref{appendix:rho_details}.

\subsection{Substitution}

We use $\Proc$ for the set of processes, $\QProc$ for the set of
names, and $\id{\{}\vec{y} / \vec{x} \id{\}}$ to denote partial maps,
$s : \QProc \rightarrow \QProc$. A map, $s$ lifts, uniquely, to a map
on process terms, $\widehat{s} : \Proc \rightarrow \Proc$ by the
following equations.

\begin{mathpar}
  (0) \psubstp{Q}{P} := 0 \\
  (R \juxtap S) \psubstp{Q}{P}
  :=    
  (R)\psubstp{Q}{P} \juxtap (S) \psubstp{Q}{P} \\
  (x?(y).R) \psubstp{Q}{P}    
  :=    
  (x)\substp{Q}{P} (z)\concat( (R \psubstn{z}{y}) \psubstp{Q}{P} ) \\
  (\lift{x}{R}) \psubstp{Q}{P}  
  :=
  \lift{(x)\substp{Q}{P}}{ R \psubstp{Q}{P} } \\
%   (\dropn{x})  \psubstp{Q}{P}       
%   := 
%   \left\{ 
%     \begin{array}{ccc} 
%       \dropn{\quotep{Q}} & & x \nameeq \quotep{P} \\
%       \dropn{x} & & otherwise \\
%     \end{array}
%   \right. 
  (\dropn{x})  \psubstp{Q}{P}       
  := 
  \left\{ 
    \begin{array}{ccc} 
      Q & & x \nameeq \quotep{P} \\
      \dropn{x} & & otherwise \\
    \end{array}
  \right.
\end{mathpar}
 

where

\begin{eqnarray}
  (x)\id{\{} \lpquote Q \rpquote / \lpquote P \rpquote \id{\}}            = 
  \left\{ 
    \begin{array}{ccc}
      \lpquote Q \rpquote & & x \nameeq \lpquote P \rpquote \\
      x & & otherwise \\
    \end{array}
  \right. \nonumber
\end{eqnarray}

and $z$ is chosen distinct from $\quotep{P}$, $\quotep{Q}$, the free
names in $Q$, and all the names in $R$. Our $\alpha$-equivalence will
be built in the standard way from this substitution.

\begin{remark}\label{rem:no_self_referential_names}
  One consequence of these definitions is that $\forall P. \quotep{P}
  \not\in \freenames{P}$.
\end{remark}

\subsection{ Dynamic quote: an example }

Anticipating something of what's to come, consider applying the
substitution, $\widehat{\id{\{}u / z \id{\}}}$, to the following pair
of processes, $\lift{w}{y!(z)}$ and $w[ \lpquote y!(z) \rpquote ]$.

\begin{eqnarray}
	\lift{w}{y!(z)}\widehat{\id{\{}u / z \id{\}}}
		& = &
		\lift{w}{y!(u)} \nonumber\\
	w[ \lpquote y!(z) \rpquote ] \widehat{ \id{\{}u / z \id{\}} }
		& = &
		w[ \lpquote y!(z) \rpquote ] \nonumber
\end{eqnarray}

Because the body of the process between quotes is impervious to
substitution, we get radically different answers. In fact, by
examining the first process in an input context,
e.g. $x?(z).\lift{w}{y!(z)}$, we see that the process under the lift
operator may be shaped by prefixed inputs binding a name inside it. In
this sense, the lift operator will be seen as a way to dynamically
construct processes before reifying them as names.

Finally equipped with these standard features we can present the
dynamics of the calculus.

\subsubsection{Operational semantics} 

Finally, we introduce the computational dynamics. What marks these
algebras as distinct from other more traditionally studied algebraic
structures, e.g. vector spaces or polynomial rings, is the manner in
which dynamics is captured. In traditional structures, dynamics is typically
expressed through morphisms between such structures, as in linear maps
between vector spaces or morphisms between rings. In algebras
associated with the semantics of computation, the dynamics is
expressed as part of the algebraic structure itself, through a
reduction reduction relation typically denoted by $\red$. Below, we
give a recursive presentation of this relation for the calculus used
in the encoding.

$\red \subseteq \pi \times \pi$
$\red : \pi \to \mathcal{P}(\pi)$

\begin{mathpar}
  \inferrule* [lab=Comm] { \textsf{match}( x_{src}, x_{trgt} ) } { x_{trgt}?(y)P \; | \; x_{src}!\langle {Q} \rangle \red P\{\quotep{Q}/y}\} }
  \and \\
  \inferrule* [lab=Par] {{P} \red {P}'} {{{P} | {Q}} \red {{P}' | {Q}}}
  \and
  \inferrule* [lab=Equiv]{{{P} \scong {P}'} \andalso {{P}' \red {Q}'} \andalso {{Q}' \scong {Q}}}{{P} \red {Q}}
\end{mathpar}

\begin{eqnarray*}
  match_{\equiv} (\quotep{P},\quotep{Q}) & := & P \equiv Q \\
  match_{\dagger}(\quotep{P},\quotep{Q}) & := & \forall R. P|Q \red^{*} R => R \red^{*} 0 \\
  match_{K}(\quotep{P},\quotep{Q}) & := & K \mbox{ for some context } K
\end{eqnarray*}

$u?(x)P | u!\langle Q \rangle \red P\{\quotep{Q}/x\}$

%We write $\wred$ for $\red^*$, and $P\red$ if $\exists Q $ such that $ P \red Q$.
We write $P\red$ if $\exists Q $ such that $ P \red Q$ and $P\not\red$, otherwise.

\section{Replication}

As mentioned before, it is known that replication (and hence
recursion) can be implemented in a higher-order process algebra
\cite{SangiorgiWalker}. As our first example of calculation with the
machinery thus far presented we give the construction explicitly in
the {\rhoc}.

\begin{eqnarray}
	D_{x} & := & \prefix{x}{y}{(\binpar{\outputp{x}{y}}{@{y}})} \nonumber\\
	\bangp_{x}{P} & := & \binpar{{x}!\langle{\binpar{D_{x}}{P}}\rangle}{D_{x}} \nonumber
\end{eqnarray}

\begin{eqnarray}
	\bangp_{x}{P} & & \nonumber\\
	=
	& {x}!\langle{(\prefix{x}{y}{(\outputp{x}{y} | @{y})) | P}}\rangle 
	      | \prefix{x}{y}{(\outputp{x}{y} | @{y})} & \nonumber\\
	\red
	& (\outputp{x}{y} | @{y})\substn{\quotep{(\prefix{x}{y}{(@{y} | \outputp{x}{y})) | P}}}{y} & \nonumber\\
	=
	& \outputp{x}{\quotep{(\prefix{x}{y}{(\outputp{x}{y} | @{y})) | P}}}
	  | {(\prefix{x}{y}{(\outputp{x}{y} | @{y})) | P}} & \nonumber\\
	\red
	& \ldots & \nonumber\\
	\red^*
	& P | P | \ldots & \nonumber
\end{eqnarray}

Of course, this encoding, as an implementation, runs away, unfolding
$\bangp{P}$ eagerly. A lazier and more implementable replication
operator, restricted to input-guarded processes, may be obtained as follows.

\begin{eqnarray}
\bangp{\prefix{u}{v}{P}} 
	:= 
	\binpar{\lift{x}{\prefix{u}{v}{(\binpar{D(x)}{P})}}}{D(x)} \nonumber
\end{eqnarray}

\begin{remark}
  Note that the lazier definition still does not deal with summation
  or mixed summation (i.e. sums over input and output). The reader is
  invited to construct definitions of replication that deal with these
  features. 

  Further, the definitions are parameterized in a name, $x$. Can you,
  gentle reader, make a definition that eliminates this parameter and
  guarantees no accidental interaction between the replication
  machinery and the process being replicated -- i.e. no accidental
  sharing of names used by the process to get its work done and the
  name(s) used by the replication to effect copying. This latter
  revision of the definition of replication is crucial to obtaining
  the expected identity $!!P \sim !P$.
\end{remark}

\begin{remark}\label{rem:paradoxical_combinator}
  The reader familiar with the lambda calculus will have noticed the
  similarity between $D$ and the paradoxical combinator.

  [Ed. note: the existence of this seems to suggest we have to be more
  restrictive on the set of processes and names we admit if we are to
  support no-cloning.]
\end{remark}

\subsubsection{Bisimulation}

The computational dynamics gives rise to another kind of equivalence,
the equivalence of computational behavior. As previously mentioned
this is typically captured \emph{via} some form of bisimulation.

% The notion we use in this paper is weak barbed bisimulation
% \cite{milner91polyadicpi}.

The notion we use in this paper is derived from weak barbed
bisimulation \cite{milner91polyadicpi}. 

\begin{definition}
An \emph{observation relation}, $\downarrow_{\mathcal N}$, over a set
of names, $\mathcal N$, is the smallest relation satisfying the rules
below.

\infrule[Out-barb]{y \in {\mathcal N}, \; x \nameeq y}
		  {\outputp{x}{v} \downarrow_{\mathcal N} x}
\infrule[Par-barb]{\mbox{$P\downarrow_{\mathcal N} x$ or $Q\downarrow_{\mathcal N} x$}}
		  {\binpar{P}{Q} \downarrow_{\mathcal N} x}

We write $P \Downarrow_{\mathcal N} x$ if there is $Q$ such that 
$P \wred Q$ and $Q \downarrow_{\mathcal N} x$.
\end{definition}

\begin{definition}
%\label{def.bbisim}
An  ${\mathcal N}$-\emph{barbed bisimulation} over a set of names, ${\mathcal N}$, is a symmetric binary relation 
${\mathcal S}_{\mathcal N}$ between agents such that $P\rel{S}_{\mathcal N}Q$ implies:
\begin{enumerate}
\item If $P \red P'$ then $Q \wred Q'$ and $P'\rel{S}_{\mathcal N} Q'$.
\item If $P\downarrow_{\mathcal N} x$, then $Q\Downarrow_{\mathcal N} x$.
\end{enumerate}
$P$ is ${\mathcal N}$-barbed bisimilar to $Q$, written
$P \wbbisim_{\mathcal N} Q$, if $P \rel{S}_{\mathcal N} Q$ for some ${\mathcal N}$-barbed bisimulation ${\mathcal S}_{\mathcal N}$.
\end{definition}

$\mathcal{R} \subseteq \pi \times \pi$

$P \mathcal{R} Q => \forall P'. P \red P' \Rightarrow \exists Q'. Q \red Q', P' \mathcal{R} Q'$

$P \vdash x \Rightarrow Q \vdash x$

\begin{mathpar}
  \inferrule*[lab=Out-barb]{x \nameeq y}{{y}!\langle{Q}\rangle \vdash x}
  \and
  \inferrule*[lab=Par-barb]{\mbox{$P\vdash x$ or $Q\vdash x$}}{\binpar{P}{Q} \vdash x}
\end{mathpar}

\subsubsection{Contexts}

One of the principle advantages of computational calculi like the
$\pi$-calculus is a well-defined notion of context,
contextual-equivalence and a correlation between
contextual-equivalence and notions of bisimulation. The notion of
context allows the decomposition of a process into (sub-)process and
its syntactic environment, its context. Thus, a context may be
thought of as a process with a ``hole'' (written $\Box$) in it. The
application of a context $M$ to a process $P$, written $M[P]$, is
tantamount to filling the hole in $M$ with $P$. In this paper we do
not need the full weight of this theory, but do make use of the notion
of context in the proof the main theorem. 

\begin{mathpar}
  \inferrule* [lab=summation] {} {{M_{M},M_{N}} \bc \Box \;|\; x.M_{A} \;|\; M_{M}+M_{N}}
  \and
  \inferrule* [lab=agent] {} {{M_{A}} \bc (\vec{x})M_{P} \;| \; \clift{P_0,\ldots,M_{P},\ldots,P_N}}
  \and \\
  \inferrule* [lab=process] {} {{M_{P}} \bc M_{N} \;| \;P|M_{P} }
\end{mathpar} 

\begin{mathpar}
  \inferrule* [lab=sychronization] {} {M_{N} \bc \Box \;|\; x?M_{F} \;|\; x!M_{C}}
  \and
  \inferrule* [lab=abstraction] {} {{M_{F}} \bc (x)M_{P} }
  \and
  \inferrule* [lab=concretion] {} {{M_{C}} \bc \langle M_{P} \rangle }
  \and \\
  \inferrule* [lab=process] {} {{M_{P}} \bc M_{N} \;| \;P|M_{P} }
\end{mathpar}

\begin{definition}[contextual application] Given a context $M$, and
  process $P$, we define the \emph{contextual application}, $M[P] :=
  M\{P/\Box\}$. That is, the contextual application of M to P is the
  substitution of $P$ for $\Box$ in $M$.
\end{definition}

$\meaningof{-} : L \to \mathcal{P}(\pi)$

\begin{mathpar}
  \inferrule* [lab=collection] {} {\meaningof{true} = \pi, \and \meaningof{~E} = \pi \setminus \meaningof{E}, \and \meaningof{E_{1} \& E_{2}} = \meaningof{E_{1}} \cap \meaningof{E_{2}}}
\end{mathpar}

\begin{mathpar}
  \inferrule* [lab=structure] {} {\meaningof{0} = \{ P \in \pi | P \equiv 0 \}, \and \\ \meaningof{E_1 | E_2} = \{ P \in \pi | P \equiv P_{1} | P_{2}, P_{1} \in \meaningof{E_{1}}, P_{2} \in \meaningof{E_2}\} }
\end{mathpar}

\begin{mathpar}
 \inferrule* [lab=behavior] {} {\meaningof{\langle a?b \rangle E} = \{ P \in \pi | P \equiv Q | u?(y)P', \\ \and \\\\ \and \\ \;\;\; u \in \meaningof{a}, \forall z.P'\{z/y\} \in \meaningof{E\{z/b\}}\}, \and \\ \meaningof{a!E} = \{ P \in \pi | P \equiv Q | x!\langle P' \rangle, x \in \meaningof{a} P' \in \meaningof{E}\} }
\end{mathpar}

\begin{mathpar}
 \inferrule* [lab=nominal] {} {\meaningof{\quotep{E}} = \{ \quotep{P} \in \quotep{\pi} | P \in \meaningof{E} \}, \and \meaningof{\quotep{P}} = \{ \quotep{Q} \in \quotep{\pi} | P \equiv Q \} \and \\ \meaningof{@\quotep{E}} = \{ P \in \pi | P \equiv @x, x \in \meaningof{E} \}}
\end{mathpar}

\begin{eqnarray*}
  \\
  \meaningof{-} : TS \to ST
\end{eqnarray*}

\begin{eqnarray*}
  \\
  L : TS \to ST
\end{eqnarray*}

\begin{eqnarray*}
  \\
  P \models E \iff P \in \meaningof{E}
\end{eqnarray*}

\begin{eqnarray*}
  P \approx_{L} Q \iff \forall E \in L. P \models E \iff Q \models E
\end{eqnarray*}

\begin{eqnarray*}
  P \approx_{K} Q
\end{eqnarray*}

\begin{eqnarray*}
  P \approx Q
\end{eqnarray*}

$\approx_{K} = \approx = \approx_{L}$

\subsubsection{Contextual duality}

Note that contexts extend the quotation operation to a family of
operations from processes to names. Given a context, $M$, we can
define a \emph{nominal context}, $\quotep{M}$ by $\quotep{M}[P] :=
\quotep{M[P]}$. To foreshadow what is to come we observe that these
operations enjoy a duality with processes very much like the duality
between vectors and maps from vectors to scalars.

Further, because the calculus is essentially higher-order, we have a
correspondence between contexts and processes. More specifically,
given a name $x$ and a context $M$ we can construct $M^{*}_{x}$ such
that 

\begin{mathpar}
  M^{*}_{x} | \lift{x}{P} \red M[P]
\end{mathpar}

namely,

\begin{mathpar}
  M^{*}_{x} := x?(u).M[\dropn{u}]
\end{mathpar}

The dependence of $M^{*}_{x}$ on a name makes it an abstraction, 

\begin{mathpar}
  M^{*} := (x)x?(u).M[\dropn{u}]
\end{mathpar}

\subsection{Additional notation}

It will sometimes be convenient to denote the process a name
quotes. We already have the notation $x = \quotep{P}$, but it will be
convenient to introduce an alternate notation, $\procn{x}$, when we
want to emphasize the connection to the use of the name. Note that, by
virtue of name equivalence, $\quotep{\procn{x}} \nameeq x$; so, the
notation is consistent with previous definitions.

Further, because names have structure it is possible to effect
substitutions on the basis of that structure. This means we need to
upgrade our notation for substitutions, which we accomplish by
adapting comprehension notation. Thus,

\begin{mathpar}
  P\{ y / x : x \in S \}
\end{mathpar}

is interpreted to mean the process derived from P by replacing (in a
capture-avoiding manner) each occurrence of $x$ in $S$ by $y$. For example,

\begin{mathpar}
  P\{ \quotep{\procn{x}|\procn{x}} / x : x \in \freenames{P} \}
\end{mathpar}

will replace each (occurrence) of a free name $x$ in $P$ by
$\quotep{\procn{x}|\procn{x}}$.

Also, we will avail ourselves of the notation $x^{L}$ and $x^{R}$ to
denote injections of a name into disjoint copies of the name
space. There are numerous ways to accomplish this. One example can be
found in \cite{MeredithR05}. This notation overloads to vectors of
names: $\vec{x}^{\pi} := (x_{i}^{\pi} \; : \; 0 \leq i < |\vec{x}| )$ where $\pi \in \{L,R\}$.

We also use $P^{\Box} := P|\Box$.

In \cite{MeredithR05} an interpretation of the new operator is
given. It turns out that there are several possible interpretations
all enjoying the requisite algebraic properties of the operator (see
\cite{milner91polyadicpi}). We will therefore make liberal use of
$(\nu\; \vec{x})P$.

% subsection the_syntax_and_semantics_of_the_notation_system (end)   

\input{qm2pi.qmops} 

\input{qm2pi.sterngerlach} 

\input{qm2pi.metric} 

% section concurrent_process_calculi (end)

%\input{qm2pi.proofsketch}

% section proof sketch (end)

%\input{qm2pi.slviaknots} 

% section spatial logic via knots (end)

\input{qm2pi.conclusion}

% section conclusion (end)

%\input{qm2pi.dtcodes} 

% section wiring algorithm (end)

\input{qm2pi.ack} 

% section acknowledgments (end)

\newpage


\bibliographystyle{plain}   
\bibliography{../../biblios/main.bib}

\input{qm2pi.rhodetails}

\end{document}

 

% section notation (end)

\input{qm2pi.process.calculi} 

% section concurrent_process_calculi_and_spatial_logics_ (end)
    
%\documentclass[12pt]{llncs}
%\documentclass{jktr}

\usepackage[pdftex]{hyperref}                   
\usepackage {listings}
\usepackage {mathpartir}
\usepackage{bcprules}
%\usepackage{listings}
                       
\usepackage{graphicx} 
%\usepackage[margins=2.5cm,nohead,nofoot]{geometry}
%\usepackage{geometry}
\usepackage{amsfonts}
\usepackage{amstext}
\usepackage{latexsym}
\usepackage{amssymb}
\usepackage{color}


%\include{myPreamble}
\include{qm2pi.local} 

%\ifpdf
%\usepackage[pdftex]{graphicx}
%\else
%\usepackage{graphicx}
%\fi

 % \ifpdf
%  \usepackage{pdfsync}
%  \if


%\title{Brief Article}
%\author{David F. Snyder}
%\author{L.G. Meredith}

%\address{Dept. of Math., Texas State University--San Marcos, San Marcos, TX 78666}
       
\pagestyle{empty}


\begin{document}

\lstset{language=[Objective]Caml,frame=shadowbox}

\input{qm2pi.front}

% section front matter (end)

\input{qm2pi.intro} 
 
% section introduction (end)

% \input{qm2pi.knotations} 

% section notation (end)

\input{qm2pi.process.calculi} 

% section concurrent_process_calculi_and_spatial_logics_ (end)
    
%\input{qm2pi.knots2pi} 

%\input{qm2pi.trefoil} 

%\input{qm2pi.mainthm} 

% subsection basic_interpretation (end)

%\input{qm2pi.rho.presentation} 
\subsection{The syntax and semantics of the notation system}\label{sub:the_syntax_and_semantics_of_the_notation_system} % (fold)

We now summarize a technical presentation of the calculus that
embodies our theory of dynamics. The typical presentation of such a
calculus follows the style of giving generators and relations on
them. The grammar, below, describing term constructors, freely
generates the set of processes, $\Proc$. This set is then quotiented
by a relation known as structural congruence and it is over this set
that the notion of dynamics is expressed. This presentation is
essentially that of \cite{MeredithR05} with the addition of
polyadicity and summation. For readability we have relegated some of
the technical subtleties to an appendix.

\subsubsection{Process grammar}\label{subsub:process_grammar}

\begin{mathpar}
  \inferrule* [lab=synchronization] {} {{M} \bc \pzero \;|\; x?F \;|\; x!C }
  \and
  \inferrule* [lab=abstraction] {} {{F} \bc (x)P}
  \and
  \inferrule* [lab=concretion] {} {{C} \bc \langle Q \rangle}
  \and
  \inferrule* [lab=process] {} {{P,Q} \bc M \;| \;P|Q \;|\; @{x}}
  \and
  \inferrule* [lab=name] {} {{x} \bc \quotep{P}}
\end{mathpar} 

Note that $\vec{x}$ (resp. $\vec{P}$) denotes a vector of names
(resp. processes) of length $|\vec{x}|$ (resp. $|\vec{P}|$). We adopt
the following useful abbreviations.

\begin{mathpar}
   x?(\vec{y}).P := x.(\vec{y})P \and  x\clift{\vec{P}} := x.\clift{\vec{P}}
   \and x!(y) := \lift{x}{\dropn{y}}
   \and \Pi_{i=0}^{n-1}P_i := P_0 | \ldots | P_{n-1}
\end{mathpar}

\subsubsection{Structural congruence}

\paragraph{Free and bound names and alpha-equivalence.} At the
core of structural equivalence is alpha-equivalence which identifies
process that are the same up to a change of variable. Formally, we
recognize the distinction between free and bound names. The free names
of a process, $\freenames{P}$, may be calculated recursively as
follows:

\begin{mathpar}
\freenames{\pzero} := \emptyset
  \and \\
  \freenames{x?(y).P} := \{ x \} \cup (\freenames{P} \setminus \{ y \})
  \and 
  \freenames{x!\langle P \rangle} := \{ x \} \cup \{ P \} 
  \and \\
  \freenames{P|Q} := \freenames{P} \cup \freenames{Q}
  \and \\
  \freenames{@{x}} := \{ x \}
\end{mathpar}

$\pi$
$\quotep{\pi}$

$\freenames{-} : \pi \to \mathcal{P}(\quotep{\pi})$

\begin{eqnarray*}
  \freenames{\pzero} & := & \emptyset \\
  \freenames{x?(y).P} & := & \{ x \} \cup (\freenames{P} \setminus \{ y \}) \\
  \freenames{x!\langle P \rangle} & := & \{ x \} \cup \{ P \} \\
  \freenames{P|Q} & := & \freenames{P} \cup \freenames{Q} \\
  \freenames{\dropn{x}} & := & \{ x \}
\end{eqnarray*}

The bound names of a process, $\boundnames{P}$, are those names occurring in $P$
that are not free. For example, in $x?(y).0$, the name $x$ is free, while $y$ is bound.

\begin{mathpar}
  \inferrule* [lab=monoidal-laws] {} { P|Q \equiv Q|P \and P|0 \equiv P \and P|(Q|R) \equiv (P|Q)|R }
\end{mathpar}

\begin{mathpar}
  \inferrule* [lab=alpha-equivalence] {} { (x)P \equiv (y)P\{y/x\} \and y \not\in \freenames{P} }
\end{mathpar}

\begin{definition}
Then two processes, $P,Q$, are alpha-equivalent if $P = Q\{\vec{y}/\vec{x}\}$ for
some $\vec{x} \in \boundnames{Q},\vec{y} \in \boundnames{P}$, where $Q\{\vec{y}/\vec{x}\}$
denotes the capture-avoiding substitution of $\vec{y}$ for $\vec{x}$ in $Q$.
\end{definition}

\begin{definition}
  The {\em structural congruence} \cite{SangiorgiWalker} , $\equiv$,
  between processes is the least congruence containing
  alpha-equivalence, satisfying the abelian monoid laws
  (associativity, commutativity and $\pzero$ as identity) for parallel
  composition $|$ and for summation $+$.
\end{definition}

\subsection{Name equivalence}

We take name equivalence, written $\nameeq$, to be the smallest
equivalence relation generated by the following rules.

\begin{mathpar}
\inferrule*[lab=Quote-drop]
{ }
{ \quotep{@{x}} \nameeq x }

\inferrule*[lab=Struct-equiv]
{ P \scong Q }
{ \quotep{P} \nameeq \quotep{Q} }
\end{mathpar}

The astute reader will have noticed that the mutual recursion of names
and processes imposes a mutual recursion on alpha-equivalence and
structural equivalence via name-equivalence. Fortunately, all of this
works out pleasantly and we may calculate in the natural way, free of
concern. The reader interested in the details is referred to the
appendix \ref{appendix:rho_details}.

\subsection{Substitution}

We use $\Proc$ for the set of processes, $\QProc$ for the set of
names, and $\id{\{}\vec{y} / \vec{x} \id{\}}$ to denote partial maps,
$s : \QProc \rightarrow \QProc$. A map, $s$ lifts, uniquely, to a map
on process terms, $\widehat{s} : \Proc \rightarrow \Proc$ by the
following equations.

\begin{mathpar}
  (0) \psubstp{Q}{P} := 0 \\
  (R \juxtap S) \psubstp{Q}{P}
  :=    
  (R)\psubstp{Q}{P} \juxtap (S) \psubstp{Q}{P} \\
  (x?(y).R) \psubstp{Q}{P}    
  :=    
  (x)\substp{Q}{P} (z)\concat( (R \psubstn{z}{y}) \psubstp{Q}{P} ) \\
  (\lift{x}{R}) \psubstp{Q}{P}  
  :=
  \lift{(x)\substp{Q}{P}}{ R \psubstp{Q}{P} } \\
%   (\dropn{x})  \psubstp{Q}{P}       
%   := 
%   \left\{ 
%     \begin{array}{ccc} 
%       \dropn{\quotep{Q}} & & x \nameeq \quotep{P} \\
%       \dropn{x} & & otherwise \\
%     \end{array}
%   \right. 
  (\dropn{x})  \psubstp{Q}{P}       
  := 
  \left\{ 
    \begin{array}{ccc} 
      Q & & x \nameeq \quotep{P} \\
      \dropn{x} & & otherwise \\
    \end{array}
  \right.
\end{mathpar}
 

where

\begin{eqnarray}
  (x)\id{\{} \lpquote Q \rpquote / \lpquote P \rpquote \id{\}}            = 
  \left\{ 
    \begin{array}{ccc}
      \lpquote Q \rpquote & & x \nameeq \lpquote P \rpquote \\
      x & & otherwise \\
    \end{array}
  \right. \nonumber
\end{eqnarray}

and $z$ is chosen distinct from $\quotep{P}$, $\quotep{Q}$, the free
names in $Q$, and all the names in $R$. Our $\alpha$-equivalence will
be built in the standard way from this substitution.

\begin{remark}\label{rem:no_self_referential_names}
  One consequence of these definitions is that $\forall P. \quotep{P}
  \not\in \freenames{P}$.
\end{remark}

\subsection{ Dynamic quote: an example }

Anticipating something of what's to come, consider applying the
substitution, $\widehat{\id{\{}u / z \id{\}}}$, to the following pair
of processes, $\lift{w}{y!(z)}$ and $w[ \lpquote y!(z) \rpquote ]$.

\begin{eqnarray}
	\lift{w}{y!(z)}\widehat{\id{\{}u / z \id{\}}}
		& = &
		\lift{w}{y!(u)} \nonumber\\
	w[ \lpquote y!(z) \rpquote ] \widehat{ \id{\{}u / z \id{\}} }
		& = &
		w[ \lpquote y!(z) \rpquote ] \nonumber
\end{eqnarray}

Because the body of the process between quotes is impervious to
substitution, we get radically different answers. In fact, by
examining the first process in an input context,
e.g. $x?(z).\lift{w}{y!(z)}$, we see that the process under the lift
operator may be shaped by prefixed inputs binding a name inside it. In
this sense, the lift operator will be seen as a way to dynamically
construct processes before reifying them as names.

Finally equipped with these standard features we can present the
dynamics of the calculus.

\subsubsection{Operational semantics} 

Finally, we introduce the computational dynamics. What marks these
algebras as distinct from other more traditionally studied algebraic
structures, e.g. vector spaces or polynomial rings, is the manner in
which dynamics is captured. In traditional structures, dynamics is typically
expressed through morphisms between such structures, as in linear maps
between vector spaces or morphisms between rings. In algebras
associated with the semantics of computation, the dynamics is
expressed as part of the algebraic structure itself, through a
reduction reduction relation typically denoted by $\red$. Below, we
give a recursive presentation of this relation for the calculus used
in the encoding.

$\red \subseteq \pi \times \pi$
$\red : \pi \to \mathcal{P}(\pi)$

\begin{mathpar}
  \inferrule* [lab=Comm] { \textsf{match}( x_{src}, x_{trgt} ) } { x_{trgt}?(y)P \; | \; x_{src}!\langle {Q} \rangle \red P\{\quotep{Q}/y}\} }
  \and \\
  \inferrule* [lab=Par] {{P} \red {P}'} {{{P} | {Q}} \red {{P}' | {Q}}}
  \and
  \inferrule* [lab=Equiv]{{{P} \scong {P}'} \andalso {{P}' \red {Q}'} \andalso {{Q}' \scong {Q}}}{{P} \red {Q}}
\end{mathpar}

\begin{eqnarray*}
  match_{\equiv} (\quotep{P},\quotep{Q}) & := & P \equiv Q \\
  match_{\dagger}(\quotep{P},\quotep{Q}) & := & \forall R. P|Q \red^{*} R => R \red^{*} 0 \\
  match_{K}(\quotep{P},\quotep{Q}) & := & K \mbox{ for some context } K
\end{eqnarray*}

$u?(x)P | u!\langle Q \rangle \red P\{\quotep{Q}/x\}$

%We write $\wred$ for $\red^*$, and $P\red$ if $\exists Q $ such that $ P \red Q$.
We write $P\red$ if $\exists Q $ such that $ P \red Q$ and $P\not\red$, otherwise.

\section{Replication}

As mentioned before, it is known that replication (and hence
recursion) can be implemented in a higher-order process algebra
\cite{SangiorgiWalker}. As our first example of calculation with the
machinery thus far presented we give the construction explicitly in
the {\rhoc}.

\begin{eqnarray}
	D_{x} & := & \prefix{x}{y}{(\binpar{\outputp{x}{y}}{@{y}})} \nonumber\\
	\bangp_{x}{P} & := & \binpar{{x}!\langle{\binpar{D_{x}}{P}}\rangle}{D_{x}} \nonumber
\end{eqnarray}

\begin{eqnarray}
	\bangp_{x}{P} & & \nonumber\\
	=
	& {x}!\langle{(\prefix{x}{y}{(\outputp{x}{y} | @{y})) | P}}\rangle 
	      | \prefix{x}{y}{(\outputp{x}{y} | @{y})} & \nonumber\\
	\red
	& (\outputp{x}{y} | @{y})\substn{\quotep{(\prefix{x}{y}{(@{y} | \outputp{x}{y})) | P}}}{y} & \nonumber\\
	=
	& \outputp{x}{\quotep{(\prefix{x}{y}{(\outputp{x}{y} | @{y})) | P}}}
	  | {(\prefix{x}{y}{(\outputp{x}{y} | @{y})) | P}} & \nonumber\\
	\red
	& \ldots & \nonumber\\
	\red^*
	& P | P | \ldots & \nonumber
\end{eqnarray}

Of course, this encoding, as an implementation, runs away, unfolding
$\bangp{P}$ eagerly. A lazier and more implementable replication
operator, restricted to input-guarded processes, may be obtained as follows.

\begin{eqnarray}
\bangp{\prefix{u}{v}{P}} 
	:= 
	\binpar{\lift{x}{\prefix{u}{v}{(\binpar{D(x)}{P})}}}{D(x)} \nonumber
\end{eqnarray}

\begin{remark}
  Note that the lazier definition still does not deal with summation
  or mixed summation (i.e. sums over input and output). The reader is
  invited to construct definitions of replication that deal with these
  features. 

  Further, the definitions are parameterized in a name, $x$. Can you,
  gentle reader, make a definition that eliminates this parameter and
  guarantees no accidental interaction between the replication
  machinery and the process being replicated -- i.e. no accidental
  sharing of names used by the process to get its work done and the
  name(s) used by the replication to effect copying. This latter
  revision of the definition of replication is crucial to obtaining
  the expected identity $!!P \sim !P$.
\end{remark}

\begin{remark}\label{rem:paradoxical_combinator}
  The reader familiar with the lambda calculus will have noticed the
  similarity between $D$ and the paradoxical combinator.

  [Ed. note: the existence of this seems to suggest we have to be more
  restrictive on the set of processes and names we admit if we are to
  support no-cloning.]
\end{remark}

\subsubsection{Bisimulation}

The computational dynamics gives rise to another kind of equivalence,
the equivalence of computational behavior. As previously mentioned
this is typically captured \emph{via} some form of bisimulation.

% The notion we use in this paper is weak barbed bisimulation
% \cite{milner91polyadicpi}.

The notion we use in this paper is derived from weak barbed
bisimulation \cite{milner91polyadicpi}. 

\begin{definition}
An \emph{observation relation}, $\downarrow_{\mathcal N}$, over a set
of names, $\mathcal N$, is the smallest relation satisfying the rules
below.

\infrule[Out-barb]{y \in {\mathcal N}, \; x \nameeq y}
		  {\outputp{x}{v} \downarrow_{\mathcal N} x}
\infrule[Par-barb]{\mbox{$P\downarrow_{\mathcal N} x$ or $Q\downarrow_{\mathcal N} x$}}
		  {\binpar{P}{Q} \downarrow_{\mathcal N} x}

We write $P \Downarrow_{\mathcal N} x$ if there is $Q$ such that 
$P \wred Q$ and $Q \downarrow_{\mathcal N} x$.
\end{definition}

\begin{definition}
%\label{def.bbisim}
An  ${\mathcal N}$-\emph{barbed bisimulation} over a set of names, ${\mathcal N}$, is a symmetric binary relation 
${\mathcal S}_{\mathcal N}$ between agents such that $P\rel{S}_{\mathcal N}Q$ implies:
\begin{enumerate}
\item If $P \red P'$ then $Q \wred Q'$ and $P'\rel{S}_{\mathcal N} Q'$.
\item If $P\downarrow_{\mathcal N} x$, then $Q\Downarrow_{\mathcal N} x$.
\end{enumerate}
$P$ is ${\mathcal N}$-barbed bisimilar to $Q$, written
$P \wbbisim_{\mathcal N} Q$, if $P \rel{S}_{\mathcal N} Q$ for some ${\mathcal N}$-barbed bisimulation ${\mathcal S}_{\mathcal N}$.
\end{definition}

$\mathcal{R} \subseteq \pi \times \pi$

$P \mathcal{R} Q => \forall P'. P \red P' \Rightarrow \exists Q'. Q \red Q', P' \mathcal{R} Q'$

$P \vdash x \Rightarrow Q \vdash x$

\begin{mathpar}
  \inferrule*[lab=Out-barb]{x \nameeq y}{{y}!\langle{Q}\rangle \vdash x}
  \and
  \inferrule*[lab=Par-barb]{\mbox{$P\vdash x$ or $Q\vdash x$}}{\binpar{P}{Q} \vdash x}
\end{mathpar}

\subsubsection{Contexts}

One of the principle advantages of computational calculi like the
$\pi$-calculus is a well-defined notion of context,
contextual-equivalence and a correlation between
contextual-equivalence and notions of bisimulation. The notion of
context allows the decomposition of a process into (sub-)process and
its syntactic environment, its context. Thus, a context may be
thought of as a process with a ``hole'' (written $\Box$) in it. The
application of a context $M$ to a process $P$, written $M[P]$, is
tantamount to filling the hole in $M$ with $P$. In this paper we do
not need the full weight of this theory, but do make use of the notion
of context in the proof the main theorem. 

\begin{mathpar}
  \inferrule* [lab=summation] {} {{M_{M},M_{N}} \bc \Box \;|\; x.M_{A} \;|\; M_{M}+M_{N}}
  \and
  \inferrule* [lab=agent] {} {{M_{A}} \bc (\vec{x})M_{P} \;| \; \clift{P_0,\ldots,M_{P},\ldots,P_N}}
  \and \\
  \inferrule* [lab=process] {} {{M_{P}} \bc M_{N} \;| \;P|M_{P} }
\end{mathpar} 

\begin{mathpar}
  \inferrule* [lab=sychronization] {} {M_{N} \bc \Box \;|\; x?M_{F} \;|\; x!M_{C}}
  \and
  \inferrule* [lab=abstraction] {} {{M_{F}} \bc (x)M_{P} }
  \and
  \inferrule* [lab=concretion] {} {{M_{C}} \bc \langle M_{P} \rangle }
  \and \\
  \inferrule* [lab=process] {} {{M_{P}} \bc M_{N} \;| \;P|M_{P} }
\end{mathpar}

\begin{definition}[contextual application] Given a context $M$, and
  process $P$, we define the \emph{contextual application}, $M[P] :=
  M\{P/\Box\}$. That is, the contextual application of M to P is the
  substitution of $P$ for $\Box$ in $M$.
\end{definition}

$\meaningof{-} : L \to \mathcal{P}(\pi)$

\begin{mathpar}
  \inferrule* [lab=collection] {} {\meaningof{true} = \pi, \and \meaningof{~E} = \pi \setminus \meaningof{E}, \and \meaningof{E_{1} \& E_{2}} = \meaningof{E_{1}} \cap \meaningof{E_{2}}}
\end{mathpar}

\begin{mathpar}
  \inferrule* [lab=structure] {} {\meaningof{0} = \{ P \in \pi | P \equiv 0 \}, \and \\ \meaningof{E_1 | E_2} = \{ P \in \pi | P \equiv P_{1} | P_{2}, P_{1} \in \meaningof{E_{1}}, P_{2} \in \meaningof{E_2}\} }
\end{mathpar}

\begin{mathpar}
 \inferrule* [lab=behavior] {} {\meaningof{\langle a?b \rangle E} = \{ P \in \pi | P \equiv Q | u?(y)P', \\ \and \\\\ \and \\ \;\;\; u \in \meaningof{a}, \forall z.P'\{z/y\} \in \meaningof{E\{z/b\}}\}, \and \\ \meaningof{a!E} = \{ P \in \pi | P \equiv Q | x!\langle P' \rangle, x \in \meaningof{a} P' \in \meaningof{E}\} }
\end{mathpar}

\begin{mathpar}
 \inferrule* [lab=nominal] {} {\meaningof{\quotep{E}} = \{ \quotep{P} \in \quotep{\pi} | P \in \meaningof{E} \}, \and \meaningof{\quotep{P}} = \{ \quotep{Q} \in \quotep{\pi} | P \equiv Q \} \and \\ \meaningof{@\quotep{E}} = \{ P \in \pi | P \equiv @x, x \in \meaningof{E} \}}
\end{mathpar}

\begin{eqnarray*}
  \\
  \meaningof{-} : TS \to ST
\end{eqnarray*}

\begin{eqnarray*}
  \\
  L : TS \to ST
\end{eqnarray*}

\begin{eqnarray*}
  \\
  P \models E \iff P \in \meaningof{E}
\end{eqnarray*}

\begin{eqnarray*}
  P \approx_{L} Q \iff \forall E \in L. P \models E \iff Q \models E
\end{eqnarray*}

\begin{eqnarray*}
  P \approx_{K} Q
\end{eqnarray*}

\begin{eqnarray*}
  P \approx Q
\end{eqnarray*}

$\approx_{K} = \approx = \approx_{L}$

\subsubsection{Contextual duality}

Note that contexts extend the quotation operation to a family of
operations from processes to names. Given a context, $M$, we can
define a \emph{nominal context}, $\quotep{M}$ by $\quotep{M}[P] :=
\quotep{M[P]}$. To foreshadow what is to come we observe that these
operations enjoy a duality with processes very much like the duality
between vectors and maps from vectors to scalars.

Further, because the calculus is essentially higher-order, we have a
correspondence between contexts and processes. More specifically,
given a name $x$ and a context $M$ we can construct $M^{*}_{x}$ such
that 

\begin{mathpar}
  M^{*}_{x} | \lift{x}{P} \red M[P]
\end{mathpar}

namely,

\begin{mathpar}
  M^{*}_{x} := x?(u).M[\dropn{u}]
\end{mathpar}

The dependence of $M^{*}_{x}$ on a name makes it an abstraction, 

\begin{mathpar}
  M^{*} := (x)x?(u).M[\dropn{u}]
\end{mathpar}

\subsection{Additional notation}

It will sometimes be convenient to denote the process a name
quotes. We already have the notation $x = \quotep{P}$, but it will be
convenient to introduce an alternate notation, $\procn{x}$, when we
want to emphasize the connection to the use of the name. Note that, by
virtue of name equivalence, $\quotep{\procn{x}} \nameeq x$; so, the
notation is consistent with previous definitions.

Further, because names have structure it is possible to effect
substitutions on the basis of that structure. This means we need to
upgrade our notation for substitutions, which we accomplish by
adapting comprehension notation. Thus,

\begin{mathpar}
  P\{ y / x : x \in S \}
\end{mathpar}

is interpreted to mean the process derived from P by replacing (in a
capture-avoiding manner) each occurrence of $x$ in $S$ by $y$. For example,

\begin{mathpar}
  P\{ \quotep{\procn{x}|\procn{x}} / x : x \in \freenames{P} \}
\end{mathpar}

will replace each (occurrence) of a free name $x$ in $P$ by
$\quotep{\procn{x}|\procn{x}}$.

Also, we will avail ourselves of the notation $x^{L}$ and $x^{R}$ to
denote injections of a name into disjoint copies of the name
space. There are numerous ways to accomplish this. One example can be
found in \cite{MeredithR05}. This notation overloads to vectors of
names: $\vec{x}^{\pi} := (x_{i}^{\pi} \; : \; 0 \leq i < |\vec{x}| )$ where $\pi \in \{L,R\}$.

We also use $P^{\Box} := P|\Box$.

In \cite{MeredithR05} an interpretation of the new operator is
given. It turns out that there are several possible interpretations
all enjoying the requisite algebraic properties of the operator (see
\cite{milner91polyadicpi}). We will therefore make liberal use of
$(\nu\; \vec{x})P$.

% subsection the_syntax_and_semantics_of_the_notation_system (end)   

\input{qm2pi.qmops} 

\input{qm2pi.sterngerlach} 

\input{qm2pi.metric} 

% section concurrent_process_calculi (end)

%\input{qm2pi.proofsketch}

% section proof sketch (end)

%\input{qm2pi.slviaknots} 

% section spatial logic via knots (end)

\input{qm2pi.conclusion}

% section conclusion (end)

%\input{qm2pi.dtcodes} 

% section wiring algorithm (end)

\input{qm2pi.ack} 

% section acknowledgments (end)

\newpage


\bibliographystyle{plain}   
\bibliography{../../biblios/main.bib}

\input{qm2pi.rhodetails}

\end{document}

 

%\documentclass[12pt]{llncs}
%\documentclass{jktr}

\usepackage[pdftex]{hyperref}                   
\usepackage {listings}
\usepackage {mathpartir}
\usepackage{bcprules}
%\usepackage{listings}
                       
\usepackage{graphicx} 
%\usepackage[margins=2.5cm,nohead,nofoot]{geometry}
%\usepackage{geometry}
\usepackage{amsfonts}
\usepackage{amstext}
\usepackage{latexsym}
\usepackage{amssymb}
\usepackage{color}


%\include{myPreamble}
\include{qm2pi.local} 

%\ifpdf
%\usepackage[pdftex]{graphicx}
%\else
%\usepackage{graphicx}
%\fi

 % \ifpdf
%  \usepackage{pdfsync}
%  \if


%\title{Brief Article}
%\author{David F. Snyder}
%\author{L.G. Meredith}

%\address{Dept. of Math., Texas State University--San Marcos, San Marcos, TX 78666}
       
\pagestyle{empty}


\begin{document}

\lstset{language=[Objective]Caml,frame=shadowbox}

\input{qm2pi.front}

% section front matter (end)

\input{qm2pi.intro} 
 
% section introduction (end)

% \input{qm2pi.knotations} 

% section notation (end)

\input{qm2pi.process.calculi} 

% section concurrent_process_calculi_and_spatial_logics_ (end)
    
%\input{qm2pi.knots2pi} 

%\input{qm2pi.trefoil} 

%\input{qm2pi.mainthm} 

% subsection basic_interpretation (end)

%\input{qm2pi.rho.presentation} 
\subsection{The syntax and semantics of the notation system}\label{sub:the_syntax_and_semantics_of_the_notation_system} % (fold)

We now summarize a technical presentation of the calculus that
embodies our theory of dynamics. The typical presentation of such a
calculus follows the style of giving generators and relations on
them. The grammar, below, describing term constructors, freely
generates the set of processes, $\Proc$. This set is then quotiented
by a relation known as structural congruence and it is over this set
that the notion of dynamics is expressed. This presentation is
essentially that of \cite{MeredithR05} with the addition of
polyadicity and summation. For readability we have relegated some of
the technical subtleties to an appendix.

\subsubsection{Process grammar}\label{subsub:process_grammar}

\begin{mathpar}
  \inferrule* [lab=synchronization] {} {{M} \bc \pzero \;|\; x?F \;|\; x!C }
  \and
  \inferrule* [lab=abstraction] {} {{F} \bc (x)P}
  \and
  \inferrule* [lab=concretion] {} {{C} \bc \langle Q \rangle}
  \and
  \inferrule* [lab=process] {} {{P,Q} \bc M \;| \;P|Q \;|\; @{x}}
  \and
  \inferrule* [lab=name] {} {{x} \bc \quotep{P}}
\end{mathpar} 

Note that $\vec{x}$ (resp. $\vec{P}$) denotes a vector of names
(resp. processes) of length $|\vec{x}|$ (resp. $|\vec{P}|$). We adopt
the following useful abbreviations.

\begin{mathpar}
   x?(\vec{y}).P := x.(\vec{y})P \and  x\clift{\vec{P}} := x.\clift{\vec{P}}
   \and x!(y) := \lift{x}{\dropn{y}}
   \and \Pi_{i=0}^{n-1}P_i := P_0 | \ldots | P_{n-1}
\end{mathpar}

\subsubsection{Structural congruence}

\paragraph{Free and bound names and alpha-equivalence.} At the
core of structural equivalence is alpha-equivalence which identifies
process that are the same up to a change of variable. Formally, we
recognize the distinction between free and bound names. The free names
of a process, $\freenames{P}$, may be calculated recursively as
follows:

\begin{mathpar}
\freenames{\pzero} := \emptyset
  \and \\
  \freenames{x?(y).P} := \{ x \} \cup (\freenames{P} \setminus \{ y \})
  \and 
  \freenames{x!\langle P \rangle} := \{ x \} \cup \{ P \} 
  \and \\
  \freenames{P|Q} := \freenames{P} \cup \freenames{Q}
  \and \\
  \freenames{@{x}} := \{ x \}
\end{mathpar}

$\pi$
$\quotep{\pi}$

$\freenames{-} : \pi \to \mathcal{P}(\quotep{\pi})$

\begin{eqnarray*}
  \freenames{\pzero} & := & \emptyset \\
  \freenames{x?(y).P} & := & \{ x \} \cup (\freenames{P} \setminus \{ y \}) \\
  \freenames{x!\langle P \rangle} & := & \{ x \} \cup \{ P \} \\
  \freenames{P|Q} & := & \freenames{P} \cup \freenames{Q} \\
  \freenames{\dropn{x}} & := & \{ x \}
\end{eqnarray*}

The bound names of a process, $\boundnames{P}$, are those names occurring in $P$
that are not free. For example, in $x?(y).0$, the name $x$ is free, while $y$ is bound.

\begin{mathpar}
  \inferrule* [lab=monoidal-laws] {} { P|Q \equiv Q|P \and P|0 \equiv P \and P|(Q|R) \equiv (P|Q)|R }
\end{mathpar}

\begin{mathpar}
  \inferrule* [lab=alpha-equivalence] {} { (x)P \equiv (y)P\{y/x\} \and y \not\in \freenames{P} }
\end{mathpar}

\begin{definition}
Then two processes, $P,Q$, are alpha-equivalent if $P = Q\{\vec{y}/\vec{x}\}$ for
some $\vec{x} \in \boundnames{Q},\vec{y} \in \boundnames{P}$, where $Q\{\vec{y}/\vec{x}\}$
denotes the capture-avoiding substitution of $\vec{y}$ for $\vec{x}$ in $Q$.
\end{definition}

\begin{definition}
  The {\em structural congruence} \cite{SangiorgiWalker} , $\equiv$,
  between processes is the least congruence containing
  alpha-equivalence, satisfying the abelian monoid laws
  (associativity, commutativity and $\pzero$ as identity) for parallel
  composition $|$ and for summation $+$.
\end{definition}

\subsection{Name equivalence}

We take name equivalence, written $\nameeq$, to be the smallest
equivalence relation generated by the following rules.

\begin{mathpar}
\inferrule*[lab=Quote-drop]
{ }
{ \quotep{@{x}} \nameeq x }

\inferrule*[lab=Struct-equiv]
{ P \scong Q }
{ \quotep{P} \nameeq \quotep{Q} }
\end{mathpar}

The astute reader will have noticed that the mutual recursion of names
and processes imposes a mutual recursion on alpha-equivalence and
structural equivalence via name-equivalence. Fortunately, all of this
works out pleasantly and we may calculate in the natural way, free of
concern. The reader interested in the details is referred to the
appendix \ref{appendix:rho_details}.

\subsection{Substitution}

We use $\Proc$ for the set of processes, $\QProc$ for the set of
names, and $\id{\{}\vec{y} / \vec{x} \id{\}}$ to denote partial maps,
$s : \QProc \rightarrow \QProc$. A map, $s$ lifts, uniquely, to a map
on process terms, $\widehat{s} : \Proc \rightarrow \Proc$ by the
following equations.

\begin{mathpar}
  (0) \psubstp{Q}{P} := 0 \\
  (R \juxtap S) \psubstp{Q}{P}
  :=    
  (R)\psubstp{Q}{P} \juxtap (S) \psubstp{Q}{P} \\
  (x?(y).R) \psubstp{Q}{P}    
  :=    
  (x)\substp{Q}{P} (z)\concat( (R \psubstn{z}{y}) \psubstp{Q}{P} ) \\
  (\lift{x}{R}) \psubstp{Q}{P}  
  :=
  \lift{(x)\substp{Q}{P}}{ R \psubstp{Q}{P} } \\
%   (\dropn{x})  \psubstp{Q}{P}       
%   := 
%   \left\{ 
%     \begin{array}{ccc} 
%       \dropn{\quotep{Q}} & & x \nameeq \quotep{P} \\
%       \dropn{x} & & otherwise \\
%     \end{array}
%   \right. 
  (\dropn{x})  \psubstp{Q}{P}       
  := 
  \left\{ 
    \begin{array}{ccc} 
      Q & & x \nameeq \quotep{P} \\
      \dropn{x} & & otherwise \\
    \end{array}
  \right.
\end{mathpar}
 

where

\begin{eqnarray}
  (x)\id{\{} \lpquote Q \rpquote / \lpquote P \rpquote \id{\}}            = 
  \left\{ 
    \begin{array}{ccc}
      \lpquote Q \rpquote & & x \nameeq \lpquote P \rpquote \\
      x & & otherwise \\
    \end{array}
  \right. \nonumber
\end{eqnarray}

and $z$ is chosen distinct from $\quotep{P}$, $\quotep{Q}$, the free
names in $Q$, and all the names in $R$. Our $\alpha$-equivalence will
be built in the standard way from this substitution.

\begin{remark}\label{rem:no_self_referential_names}
  One consequence of these definitions is that $\forall P. \quotep{P}
  \not\in \freenames{P}$.
\end{remark}

\subsection{ Dynamic quote: an example }

Anticipating something of what's to come, consider applying the
substitution, $\widehat{\id{\{}u / z \id{\}}}$, to the following pair
of processes, $\lift{w}{y!(z)}$ and $w[ \lpquote y!(z) \rpquote ]$.

\begin{eqnarray}
	\lift{w}{y!(z)}\widehat{\id{\{}u / z \id{\}}}
		& = &
		\lift{w}{y!(u)} \nonumber\\
	w[ \lpquote y!(z) \rpquote ] \widehat{ \id{\{}u / z \id{\}} }
		& = &
		w[ \lpquote y!(z) \rpquote ] \nonumber
\end{eqnarray}

Because the body of the process between quotes is impervious to
substitution, we get radically different answers. In fact, by
examining the first process in an input context,
e.g. $x?(z).\lift{w}{y!(z)}$, we see that the process under the lift
operator may be shaped by prefixed inputs binding a name inside it. In
this sense, the lift operator will be seen as a way to dynamically
construct processes before reifying them as names.

Finally equipped with these standard features we can present the
dynamics of the calculus.

\subsubsection{Operational semantics} 

Finally, we introduce the computational dynamics. What marks these
algebras as distinct from other more traditionally studied algebraic
structures, e.g. vector spaces or polynomial rings, is the manner in
which dynamics is captured. In traditional structures, dynamics is typically
expressed through morphisms between such structures, as in linear maps
between vector spaces or morphisms between rings. In algebras
associated with the semantics of computation, the dynamics is
expressed as part of the algebraic structure itself, through a
reduction reduction relation typically denoted by $\red$. Below, we
give a recursive presentation of this relation for the calculus used
in the encoding.

$\red \subseteq \pi \times \pi$
$\red : \pi \to \mathcal{P}(\pi)$

\begin{mathpar}
  \inferrule* [lab=Comm] { \textsf{match}( x_{src}, x_{trgt} ) } { x_{trgt}?(y)P \; | \; x_{src}!\langle {Q} \rangle \red P\{\quotep{Q}/y}\} }
  \and \\
  \inferrule* [lab=Par] {{P} \red {P}'} {{{P} | {Q}} \red {{P}' | {Q}}}
  \and
  \inferrule* [lab=Equiv]{{{P} \scong {P}'} \andalso {{P}' \red {Q}'} \andalso {{Q}' \scong {Q}}}{{P} \red {Q}}
\end{mathpar}

\begin{eqnarray*}
  match_{\equiv} (\quotep{P},\quotep{Q}) & := & P \equiv Q \\
  match_{\dagger}(\quotep{P},\quotep{Q}) & := & \forall R. P|Q \red^{*} R => R \red^{*} 0 \\
  match_{K}(\quotep{P},\quotep{Q}) & := & K \mbox{ for some context } K
\end{eqnarray*}

$u?(x)P | u!\langle Q \rangle \red P\{\quotep{Q}/x\}$

%We write $\wred$ for $\red^*$, and $P\red$ if $\exists Q $ such that $ P \red Q$.
We write $P\red$ if $\exists Q $ such that $ P \red Q$ and $P\not\red$, otherwise.

\section{Replication}

As mentioned before, it is known that replication (and hence
recursion) can be implemented in a higher-order process algebra
\cite{SangiorgiWalker}. As our first example of calculation with the
machinery thus far presented we give the construction explicitly in
the {\rhoc}.

\begin{eqnarray}
	D_{x} & := & \prefix{x}{y}{(\binpar{\outputp{x}{y}}{@{y}})} \nonumber\\
	\bangp_{x}{P} & := & \binpar{{x}!\langle{\binpar{D_{x}}{P}}\rangle}{D_{x}} \nonumber
\end{eqnarray}

\begin{eqnarray}
	\bangp_{x}{P} & & \nonumber\\
	=
	& {x}!\langle{(\prefix{x}{y}{(\outputp{x}{y} | @{y})) | P}}\rangle 
	      | \prefix{x}{y}{(\outputp{x}{y} | @{y})} & \nonumber\\
	\red
	& (\outputp{x}{y} | @{y})\substn{\quotep{(\prefix{x}{y}{(@{y} | \outputp{x}{y})) | P}}}{y} & \nonumber\\
	=
	& \outputp{x}{\quotep{(\prefix{x}{y}{(\outputp{x}{y} | @{y})) | P}}}
	  | {(\prefix{x}{y}{(\outputp{x}{y} | @{y})) | P}} & \nonumber\\
	\red
	& \ldots & \nonumber\\
	\red^*
	& P | P | \ldots & \nonumber
\end{eqnarray}

Of course, this encoding, as an implementation, runs away, unfolding
$\bangp{P}$ eagerly. A lazier and more implementable replication
operator, restricted to input-guarded processes, may be obtained as follows.

\begin{eqnarray}
\bangp{\prefix{u}{v}{P}} 
	:= 
	\binpar{\lift{x}{\prefix{u}{v}{(\binpar{D(x)}{P})}}}{D(x)} \nonumber
\end{eqnarray}

\begin{remark}
  Note that the lazier definition still does not deal with summation
  or mixed summation (i.e. sums over input and output). The reader is
  invited to construct definitions of replication that deal with these
  features. 

  Further, the definitions are parameterized in a name, $x$. Can you,
  gentle reader, make a definition that eliminates this parameter and
  guarantees no accidental interaction between the replication
  machinery and the process being replicated -- i.e. no accidental
  sharing of names used by the process to get its work done and the
  name(s) used by the replication to effect copying. This latter
  revision of the definition of replication is crucial to obtaining
  the expected identity $!!P \sim !P$.
\end{remark}

\begin{remark}\label{rem:paradoxical_combinator}
  The reader familiar with the lambda calculus will have noticed the
  similarity between $D$ and the paradoxical combinator.

  [Ed. note: the existence of this seems to suggest we have to be more
  restrictive on the set of processes and names we admit if we are to
  support no-cloning.]
\end{remark}

\subsubsection{Bisimulation}

The computational dynamics gives rise to another kind of equivalence,
the equivalence of computational behavior. As previously mentioned
this is typically captured \emph{via} some form of bisimulation.

% The notion we use in this paper is weak barbed bisimulation
% \cite{milner91polyadicpi}.

The notion we use in this paper is derived from weak barbed
bisimulation \cite{milner91polyadicpi}. 

\begin{definition}
An \emph{observation relation}, $\downarrow_{\mathcal N}$, over a set
of names, $\mathcal N$, is the smallest relation satisfying the rules
below.

\infrule[Out-barb]{y \in {\mathcal N}, \; x \nameeq y}
		  {\outputp{x}{v} \downarrow_{\mathcal N} x}
\infrule[Par-barb]{\mbox{$P\downarrow_{\mathcal N} x$ or $Q\downarrow_{\mathcal N} x$}}
		  {\binpar{P}{Q} \downarrow_{\mathcal N} x}

We write $P \Downarrow_{\mathcal N} x$ if there is $Q$ such that 
$P \wred Q$ and $Q \downarrow_{\mathcal N} x$.
\end{definition}

\begin{definition}
%\label{def.bbisim}
An  ${\mathcal N}$-\emph{barbed bisimulation} over a set of names, ${\mathcal N}$, is a symmetric binary relation 
${\mathcal S}_{\mathcal N}$ between agents such that $P\rel{S}_{\mathcal N}Q$ implies:
\begin{enumerate}
\item If $P \red P'$ then $Q \wred Q'$ and $P'\rel{S}_{\mathcal N} Q'$.
\item If $P\downarrow_{\mathcal N} x$, then $Q\Downarrow_{\mathcal N} x$.
\end{enumerate}
$P$ is ${\mathcal N}$-barbed bisimilar to $Q$, written
$P \wbbisim_{\mathcal N} Q$, if $P \rel{S}_{\mathcal N} Q$ for some ${\mathcal N}$-barbed bisimulation ${\mathcal S}_{\mathcal N}$.
\end{definition}

$\mathcal{R} \subseteq \pi \times \pi$

$P \mathcal{R} Q => \forall P'. P \red P' \Rightarrow \exists Q'. Q \red Q', P' \mathcal{R} Q'$

$P \vdash x \Rightarrow Q \vdash x$

\begin{mathpar}
  \inferrule*[lab=Out-barb]{x \nameeq y}{{y}!\langle{Q}\rangle \vdash x}
  \and
  \inferrule*[lab=Par-barb]{\mbox{$P\vdash x$ or $Q\vdash x$}}{\binpar{P}{Q} \vdash x}
\end{mathpar}

\subsubsection{Contexts}

One of the principle advantages of computational calculi like the
$\pi$-calculus is a well-defined notion of context,
contextual-equivalence and a correlation between
contextual-equivalence and notions of bisimulation. The notion of
context allows the decomposition of a process into (sub-)process and
its syntactic environment, its context. Thus, a context may be
thought of as a process with a ``hole'' (written $\Box$) in it. The
application of a context $M$ to a process $P$, written $M[P]$, is
tantamount to filling the hole in $M$ with $P$. In this paper we do
not need the full weight of this theory, but do make use of the notion
of context in the proof the main theorem. 

\begin{mathpar}
  \inferrule* [lab=summation] {} {{M_{M},M_{N}} \bc \Box \;|\; x.M_{A} \;|\; M_{M}+M_{N}}
  \and
  \inferrule* [lab=agent] {} {{M_{A}} \bc (\vec{x})M_{P} \;| \; \clift{P_0,\ldots,M_{P},\ldots,P_N}}
  \and \\
  \inferrule* [lab=process] {} {{M_{P}} \bc M_{N} \;| \;P|M_{P} }
\end{mathpar} 

\begin{mathpar}
  \inferrule* [lab=sychronization] {} {M_{N} \bc \Box \;|\; x?M_{F} \;|\; x!M_{C}}
  \and
  \inferrule* [lab=abstraction] {} {{M_{F}} \bc (x)M_{P} }
  \and
  \inferrule* [lab=concretion] {} {{M_{C}} \bc \langle M_{P} \rangle }
  \and \\
  \inferrule* [lab=process] {} {{M_{P}} \bc M_{N} \;| \;P|M_{P} }
\end{mathpar}

\begin{definition}[contextual application] Given a context $M$, and
  process $P$, we define the \emph{contextual application}, $M[P] :=
  M\{P/\Box\}$. That is, the contextual application of M to P is the
  substitution of $P$ for $\Box$ in $M$.
\end{definition}

$\meaningof{-} : L \to \mathcal{P}(\pi)$

\begin{mathpar}
  \inferrule* [lab=collection] {} {\meaningof{true} = \pi, \and \meaningof{~E} = \pi \setminus \meaningof{E}, \and \meaningof{E_{1} \& E_{2}} = \meaningof{E_{1}} \cap \meaningof{E_{2}}}
\end{mathpar}

\begin{mathpar}
  \inferrule* [lab=structure] {} {\meaningof{0} = \{ P \in \pi | P \equiv 0 \}, \and \\ \meaningof{E_1 | E_2} = \{ P \in \pi | P \equiv P_{1} | P_{2}, P_{1} \in \meaningof{E_{1}}, P_{2} \in \meaningof{E_2}\} }
\end{mathpar}

\begin{mathpar}
 \inferrule* [lab=behavior] {} {\meaningof{\langle a?b \rangle E} = \{ P \in \pi | P \equiv Q | u?(y)P', \\ \and \\\\ \and \\ \;\;\; u \in \meaningof{a}, \forall z.P'\{z/y\} \in \meaningof{E\{z/b\}}\}, \and \\ \meaningof{a!E} = \{ P \in \pi | P \equiv Q | x!\langle P' \rangle, x \in \meaningof{a} P' \in \meaningof{E}\} }
\end{mathpar}

\begin{mathpar}
 \inferrule* [lab=nominal] {} {\meaningof{\quotep{E}} = \{ \quotep{P} \in \quotep{\pi} | P \in \meaningof{E} \}, \and \meaningof{\quotep{P}} = \{ \quotep{Q} \in \quotep{\pi} | P \equiv Q \} \and \\ \meaningof{@\quotep{E}} = \{ P \in \pi | P \equiv @x, x \in \meaningof{E} \}}
\end{mathpar}

\begin{eqnarray*}
  \\
  \meaningof{-} : TS \to ST
\end{eqnarray*}

\begin{eqnarray*}
  \\
  L : TS \to ST
\end{eqnarray*}

\begin{eqnarray*}
  \\
  P \models E \iff P \in \meaningof{E}
\end{eqnarray*}

\begin{eqnarray*}
  P \approx_{L} Q \iff \forall E \in L. P \models E \iff Q \models E
\end{eqnarray*}

\begin{eqnarray*}
  P \approx_{K} Q
\end{eqnarray*}

\begin{eqnarray*}
  P \approx Q
\end{eqnarray*}

$\approx_{K} = \approx = \approx_{L}$

\subsubsection{Contextual duality}

Note that contexts extend the quotation operation to a family of
operations from processes to names. Given a context, $M$, we can
define a \emph{nominal context}, $\quotep{M}$ by $\quotep{M}[P] :=
\quotep{M[P]}$. To foreshadow what is to come we observe that these
operations enjoy a duality with processes very much like the duality
between vectors and maps from vectors to scalars.

Further, because the calculus is essentially higher-order, we have a
correspondence between contexts and processes. More specifically,
given a name $x$ and a context $M$ we can construct $M^{*}_{x}$ such
that 

\begin{mathpar}
  M^{*}_{x} | \lift{x}{P} \red M[P]
\end{mathpar}

namely,

\begin{mathpar}
  M^{*}_{x} := x?(u).M[\dropn{u}]
\end{mathpar}

The dependence of $M^{*}_{x}$ on a name makes it an abstraction, 

\begin{mathpar}
  M^{*} := (x)x?(u).M[\dropn{u}]
\end{mathpar}

\subsection{Additional notation}

It will sometimes be convenient to denote the process a name
quotes. We already have the notation $x = \quotep{P}$, but it will be
convenient to introduce an alternate notation, $\procn{x}$, when we
want to emphasize the connection to the use of the name. Note that, by
virtue of name equivalence, $\quotep{\procn{x}} \nameeq x$; so, the
notation is consistent with previous definitions.

Further, because names have structure it is possible to effect
substitutions on the basis of that structure. This means we need to
upgrade our notation for substitutions, which we accomplish by
adapting comprehension notation. Thus,

\begin{mathpar}
  P\{ y / x : x \in S \}
\end{mathpar}

is interpreted to mean the process derived from P by replacing (in a
capture-avoiding manner) each occurrence of $x$ in $S$ by $y$. For example,

\begin{mathpar}
  P\{ \quotep{\procn{x}|\procn{x}} / x : x \in \freenames{P} \}
\end{mathpar}

will replace each (occurrence) of a free name $x$ in $P$ by
$\quotep{\procn{x}|\procn{x}}$.

Also, we will avail ourselves of the notation $x^{L}$ and $x^{R}$ to
denote injections of a name into disjoint copies of the name
space. There are numerous ways to accomplish this. One example can be
found in \cite{MeredithR05}. This notation overloads to vectors of
names: $\vec{x}^{\pi} := (x_{i}^{\pi} \; : \; 0 \leq i < |\vec{x}| )$ where $\pi \in \{L,R\}$.

We also use $P^{\Box} := P|\Box$.

In \cite{MeredithR05} an interpretation of the new operator is
given. It turns out that there are several possible interpretations
all enjoying the requisite algebraic properties of the operator (see
\cite{milner91polyadicpi}). We will therefore make liberal use of
$(\nu\; \vec{x})P$.

% subsection the_syntax_and_semantics_of_the_notation_system (end)   

\input{qm2pi.qmops} 

\input{qm2pi.sterngerlach} 

\input{qm2pi.metric} 

% section concurrent_process_calculi (end)

%\input{qm2pi.proofsketch}

% section proof sketch (end)

%\input{qm2pi.slviaknots} 

% section spatial logic via knots (end)

\input{qm2pi.conclusion}

% section conclusion (end)

%\input{qm2pi.dtcodes} 

% section wiring algorithm (end)

\input{qm2pi.ack} 

% section acknowledgments (end)

\newpage


\bibliographystyle{plain}   
\bibliography{../../biblios/main.bib}

\input{qm2pi.rhodetails}

\end{document}

 

%\documentclass[12pt]{llncs}
%\documentclass{jktr}

\usepackage[pdftex]{hyperref}                   
\usepackage {listings}
\usepackage {mathpartir}
\usepackage{bcprules}
%\usepackage{listings}
                       
\usepackage{graphicx} 
%\usepackage[margins=2.5cm,nohead,nofoot]{geometry}
%\usepackage{geometry}
\usepackage{amsfonts}
\usepackage{amstext}
\usepackage{latexsym}
\usepackage{amssymb}
\usepackage{color}


%\include{myPreamble}
\include{qm2pi.local} 

%\ifpdf
%\usepackage[pdftex]{graphicx}
%\else
%\usepackage{graphicx}
%\fi

 % \ifpdf
%  \usepackage{pdfsync}
%  \if


%\title{Brief Article}
%\author{David F. Snyder}
%\author{L.G. Meredith}

%\address{Dept. of Math., Texas State University--San Marcos, San Marcos, TX 78666}
       
\pagestyle{empty}


\begin{document}

\lstset{language=[Objective]Caml,frame=shadowbox}

\input{qm2pi.front}

% section front matter (end)

\input{qm2pi.intro} 
 
% section introduction (end)

% \input{qm2pi.knotations} 

% section notation (end)

\input{qm2pi.process.calculi} 

% section concurrent_process_calculi_and_spatial_logics_ (end)
    
%\input{qm2pi.knots2pi} 

%\input{qm2pi.trefoil} 

%\input{qm2pi.mainthm} 

% subsection basic_interpretation (end)

%\input{qm2pi.rho.presentation} 
\subsection{The syntax and semantics of the notation system}\label{sub:the_syntax_and_semantics_of_the_notation_system} % (fold)

We now summarize a technical presentation of the calculus that
embodies our theory of dynamics. The typical presentation of such a
calculus follows the style of giving generators and relations on
them. The grammar, below, describing term constructors, freely
generates the set of processes, $\Proc$. This set is then quotiented
by a relation known as structural congruence and it is over this set
that the notion of dynamics is expressed. This presentation is
essentially that of \cite{MeredithR05} with the addition of
polyadicity and summation. For readability we have relegated some of
the technical subtleties to an appendix.

\subsubsection{Process grammar}\label{subsub:process_grammar}

\begin{mathpar}
  \inferrule* [lab=synchronization] {} {{M} \bc \pzero \;|\; x?F \;|\; x!C }
  \and
  \inferrule* [lab=abstraction] {} {{F} \bc (x)P}
  \and
  \inferrule* [lab=concretion] {} {{C} \bc \langle Q \rangle}
  \and
  \inferrule* [lab=process] {} {{P,Q} \bc M \;| \;P|Q \;|\; @{x}}
  \and
  \inferrule* [lab=name] {} {{x} \bc \quotep{P}}
\end{mathpar} 

Note that $\vec{x}$ (resp. $\vec{P}$) denotes a vector of names
(resp. processes) of length $|\vec{x}|$ (resp. $|\vec{P}|$). We adopt
the following useful abbreviations.

\begin{mathpar}
   x?(\vec{y}).P := x.(\vec{y})P \and  x\clift{\vec{P}} := x.\clift{\vec{P}}
   \and x!(y) := \lift{x}{\dropn{y}}
   \and \Pi_{i=0}^{n-1}P_i := P_0 | \ldots | P_{n-1}
\end{mathpar}

\subsubsection{Structural congruence}

\paragraph{Free and bound names and alpha-equivalence.} At the
core of structural equivalence is alpha-equivalence which identifies
process that are the same up to a change of variable. Formally, we
recognize the distinction between free and bound names. The free names
of a process, $\freenames{P}$, may be calculated recursively as
follows:

\begin{mathpar}
\freenames{\pzero} := \emptyset
  \and \\
  \freenames{x?(y).P} := \{ x \} \cup (\freenames{P} \setminus \{ y \})
  \and 
  \freenames{x!\langle P \rangle} := \{ x \} \cup \{ P \} 
  \and \\
  \freenames{P|Q} := \freenames{P} \cup \freenames{Q}
  \and \\
  \freenames{@{x}} := \{ x \}
\end{mathpar}

$\pi$
$\quotep{\pi}$

$\freenames{-} : \pi \to \mathcal{P}(\quotep{\pi})$

\begin{eqnarray*}
  \freenames{\pzero} & := & \emptyset \\
  \freenames{x?(y).P} & := & \{ x \} \cup (\freenames{P} \setminus \{ y \}) \\
  \freenames{x!\langle P \rangle} & := & \{ x \} \cup \{ P \} \\
  \freenames{P|Q} & := & \freenames{P} \cup \freenames{Q} \\
  \freenames{\dropn{x}} & := & \{ x \}
\end{eqnarray*}

The bound names of a process, $\boundnames{P}$, are those names occurring in $P$
that are not free. For example, in $x?(y).0$, the name $x$ is free, while $y$ is bound.

\begin{mathpar}
  \inferrule* [lab=monoidal-laws] {} { P|Q \equiv Q|P \and P|0 \equiv P \and P|(Q|R) \equiv (P|Q)|R }
\end{mathpar}

\begin{mathpar}
  \inferrule* [lab=alpha-equivalence] {} { (x)P \equiv (y)P\{y/x\} \and y \not\in \freenames{P} }
\end{mathpar}

\begin{definition}
Then two processes, $P,Q$, are alpha-equivalent if $P = Q\{\vec{y}/\vec{x}\}$ for
some $\vec{x} \in \boundnames{Q},\vec{y} \in \boundnames{P}$, where $Q\{\vec{y}/\vec{x}\}$
denotes the capture-avoiding substitution of $\vec{y}$ for $\vec{x}$ in $Q$.
\end{definition}

\begin{definition}
  The {\em structural congruence} \cite{SangiorgiWalker} , $\equiv$,
  between processes is the least congruence containing
  alpha-equivalence, satisfying the abelian monoid laws
  (associativity, commutativity and $\pzero$ as identity) for parallel
  composition $|$ and for summation $+$.
\end{definition}

\subsection{Name equivalence}

We take name equivalence, written $\nameeq$, to be the smallest
equivalence relation generated by the following rules.

\begin{mathpar}
\inferrule*[lab=Quote-drop]
{ }
{ \quotep{@{x}} \nameeq x }

\inferrule*[lab=Struct-equiv]
{ P \scong Q }
{ \quotep{P} \nameeq \quotep{Q} }
\end{mathpar}

The astute reader will have noticed that the mutual recursion of names
and processes imposes a mutual recursion on alpha-equivalence and
structural equivalence via name-equivalence. Fortunately, all of this
works out pleasantly and we may calculate in the natural way, free of
concern. The reader interested in the details is referred to the
appendix \ref{appendix:rho_details}.

\subsection{Substitution}

We use $\Proc$ for the set of processes, $\QProc$ for the set of
names, and $\id{\{}\vec{y} / \vec{x} \id{\}}$ to denote partial maps,
$s : \QProc \rightarrow \QProc$. A map, $s$ lifts, uniquely, to a map
on process terms, $\widehat{s} : \Proc \rightarrow \Proc$ by the
following equations.

\begin{mathpar}
  (0) \psubstp{Q}{P} := 0 \\
  (R \juxtap S) \psubstp{Q}{P}
  :=    
  (R)\psubstp{Q}{P} \juxtap (S) \psubstp{Q}{P} \\
  (x?(y).R) \psubstp{Q}{P}    
  :=    
  (x)\substp{Q}{P} (z)\concat( (R \psubstn{z}{y}) \psubstp{Q}{P} ) \\
  (\lift{x}{R}) \psubstp{Q}{P}  
  :=
  \lift{(x)\substp{Q}{P}}{ R \psubstp{Q}{P} } \\
%   (\dropn{x})  \psubstp{Q}{P}       
%   := 
%   \left\{ 
%     \begin{array}{ccc} 
%       \dropn{\quotep{Q}} & & x \nameeq \quotep{P} \\
%       \dropn{x} & & otherwise \\
%     \end{array}
%   \right. 
  (\dropn{x})  \psubstp{Q}{P}       
  := 
  \left\{ 
    \begin{array}{ccc} 
      Q & & x \nameeq \quotep{P} \\
      \dropn{x} & & otherwise \\
    \end{array}
  \right.
\end{mathpar}
 

where

\begin{eqnarray}
  (x)\id{\{} \lpquote Q \rpquote / \lpquote P \rpquote \id{\}}            = 
  \left\{ 
    \begin{array}{ccc}
      \lpquote Q \rpquote & & x \nameeq \lpquote P \rpquote \\
      x & & otherwise \\
    \end{array}
  \right. \nonumber
\end{eqnarray}

and $z$ is chosen distinct from $\quotep{P}$, $\quotep{Q}$, the free
names in $Q$, and all the names in $R$. Our $\alpha$-equivalence will
be built in the standard way from this substitution.

\begin{remark}\label{rem:no_self_referential_names}
  One consequence of these definitions is that $\forall P. \quotep{P}
  \not\in \freenames{P}$.
\end{remark}

\subsection{ Dynamic quote: an example }

Anticipating something of what's to come, consider applying the
substitution, $\widehat{\id{\{}u / z \id{\}}}$, to the following pair
of processes, $\lift{w}{y!(z)}$ and $w[ \lpquote y!(z) \rpquote ]$.

\begin{eqnarray}
	\lift{w}{y!(z)}\widehat{\id{\{}u / z \id{\}}}
		& = &
		\lift{w}{y!(u)} \nonumber\\
	w[ \lpquote y!(z) \rpquote ] \widehat{ \id{\{}u / z \id{\}} }
		& = &
		w[ \lpquote y!(z) \rpquote ] \nonumber
\end{eqnarray}

Because the body of the process between quotes is impervious to
substitution, we get radically different answers. In fact, by
examining the first process in an input context,
e.g. $x?(z).\lift{w}{y!(z)}$, we see that the process under the lift
operator may be shaped by prefixed inputs binding a name inside it. In
this sense, the lift operator will be seen as a way to dynamically
construct processes before reifying them as names.

Finally equipped with these standard features we can present the
dynamics of the calculus.

\subsubsection{Operational semantics} 

Finally, we introduce the computational dynamics. What marks these
algebras as distinct from other more traditionally studied algebraic
structures, e.g. vector spaces or polynomial rings, is the manner in
which dynamics is captured. In traditional structures, dynamics is typically
expressed through morphisms between such structures, as in linear maps
between vector spaces or morphisms between rings. In algebras
associated with the semantics of computation, the dynamics is
expressed as part of the algebraic structure itself, through a
reduction reduction relation typically denoted by $\red$. Below, we
give a recursive presentation of this relation for the calculus used
in the encoding.

$\red \subseteq \pi \times \pi$
$\red : \pi \to \mathcal{P}(\pi)$

\begin{mathpar}
  \inferrule* [lab=Comm] { \textsf{match}( x_{src}, x_{trgt} ) } { x_{trgt}?(y)P \; | \; x_{src}!\langle {Q} \rangle \red P\{\quotep{Q}/y}\} }
  \and \\
  \inferrule* [lab=Par] {{P} \red {P}'} {{{P} | {Q}} \red {{P}' | {Q}}}
  \and
  \inferrule* [lab=Equiv]{{{P} \scong {P}'} \andalso {{P}' \red {Q}'} \andalso {{Q}' \scong {Q}}}{{P} \red {Q}}
\end{mathpar}

\begin{eqnarray*}
  match_{\equiv} (\quotep{P},\quotep{Q}) & := & P \equiv Q \\
  match_{\dagger}(\quotep{P},\quotep{Q}) & := & \forall R. P|Q \red^{*} R => R \red^{*} 0 \\
  match_{K}(\quotep{P},\quotep{Q}) & := & K \mbox{ for some context } K
\end{eqnarray*}

$u?(x)P | u!\langle Q \rangle \red P\{\quotep{Q}/x\}$

%We write $\wred$ for $\red^*$, and $P\red$ if $\exists Q $ such that $ P \red Q$.
We write $P\red$ if $\exists Q $ such that $ P \red Q$ and $P\not\red$, otherwise.

\section{Replication}

As mentioned before, it is known that replication (and hence
recursion) can be implemented in a higher-order process algebra
\cite{SangiorgiWalker}. As our first example of calculation with the
machinery thus far presented we give the construction explicitly in
the {\rhoc}.

\begin{eqnarray}
	D_{x} & := & \prefix{x}{y}{(\binpar{\outputp{x}{y}}{@{y}})} \nonumber\\
	\bangp_{x}{P} & := & \binpar{{x}!\langle{\binpar{D_{x}}{P}}\rangle}{D_{x}} \nonumber
\end{eqnarray}

\begin{eqnarray}
	\bangp_{x}{P} & & \nonumber\\
	=
	& {x}!\langle{(\prefix{x}{y}{(\outputp{x}{y} | @{y})) | P}}\rangle 
	      | \prefix{x}{y}{(\outputp{x}{y} | @{y})} & \nonumber\\
	\red
	& (\outputp{x}{y} | @{y})\substn{\quotep{(\prefix{x}{y}{(@{y} | \outputp{x}{y})) | P}}}{y} & \nonumber\\
	=
	& \outputp{x}{\quotep{(\prefix{x}{y}{(\outputp{x}{y} | @{y})) | P}}}
	  | {(\prefix{x}{y}{(\outputp{x}{y} | @{y})) | P}} & \nonumber\\
	\red
	& \ldots & \nonumber\\
	\red^*
	& P | P | \ldots & \nonumber
\end{eqnarray}

Of course, this encoding, as an implementation, runs away, unfolding
$\bangp{P}$ eagerly. A lazier and more implementable replication
operator, restricted to input-guarded processes, may be obtained as follows.

\begin{eqnarray}
\bangp{\prefix{u}{v}{P}} 
	:= 
	\binpar{\lift{x}{\prefix{u}{v}{(\binpar{D(x)}{P})}}}{D(x)} \nonumber
\end{eqnarray}

\begin{remark}
  Note that the lazier definition still does not deal with summation
  or mixed summation (i.e. sums over input and output). The reader is
  invited to construct definitions of replication that deal with these
  features. 

  Further, the definitions are parameterized in a name, $x$. Can you,
  gentle reader, make a definition that eliminates this parameter and
  guarantees no accidental interaction between the replication
  machinery and the process being replicated -- i.e. no accidental
  sharing of names used by the process to get its work done and the
  name(s) used by the replication to effect copying. This latter
  revision of the definition of replication is crucial to obtaining
  the expected identity $!!P \sim !P$.
\end{remark}

\begin{remark}\label{rem:paradoxical_combinator}
  The reader familiar with the lambda calculus will have noticed the
  similarity between $D$ and the paradoxical combinator.

  [Ed. note: the existence of this seems to suggest we have to be more
  restrictive on the set of processes and names we admit if we are to
  support no-cloning.]
\end{remark}

\subsubsection{Bisimulation}

The computational dynamics gives rise to another kind of equivalence,
the equivalence of computational behavior. As previously mentioned
this is typically captured \emph{via} some form of bisimulation.

% The notion we use in this paper is weak barbed bisimulation
% \cite{milner91polyadicpi}.

The notion we use in this paper is derived from weak barbed
bisimulation \cite{milner91polyadicpi}. 

\begin{definition}
An \emph{observation relation}, $\downarrow_{\mathcal N}$, over a set
of names, $\mathcal N$, is the smallest relation satisfying the rules
below.

\infrule[Out-barb]{y \in {\mathcal N}, \; x \nameeq y}
		  {\outputp{x}{v} \downarrow_{\mathcal N} x}
\infrule[Par-barb]{\mbox{$P\downarrow_{\mathcal N} x$ or $Q\downarrow_{\mathcal N} x$}}
		  {\binpar{P}{Q} \downarrow_{\mathcal N} x}

We write $P \Downarrow_{\mathcal N} x$ if there is $Q$ such that 
$P \wred Q$ and $Q \downarrow_{\mathcal N} x$.
\end{definition}

\begin{definition}
%\label{def.bbisim}
An  ${\mathcal N}$-\emph{barbed bisimulation} over a set of names, ${\mathcal N}$, is a symmetric binary relation 
${\mathcal S}_{\mathcal N}$ between agents such that $P\rel{S}_{\mathcal N}Q$ implies:
\begin{enumerate}
\item If $P \red P'$ then $Q \wred Q'$ and $P'\rel{S}_{\mathcal N} Q'$.
\item If $P\downarrow_{\mathcal N} x$, then $Q\Downarrow_{\mathcal N} x$.
\end{enumerate}
$P$ is ${\mathcal N}$-barbed bisimilar to $Q$, written
$P \wbbisim_{\mathcal N} Q$, if $P \rel{S}_{\mathcal N} Q$ for some ${\mathcal N}$-barbed bisimulation ${\mathcal S}_{\mathcal N}$.
\end{definition}

$\mathcal{R} \subseteq \pi \times \pi$

$P \mathcal{R} Q => \forall P'. P \red P' \Rightarrow \exists Q'. Q \red Q', P' \mathcal{R} Q'$

$P \vdash x \Rightarrow Q \vdash x$

\begin{mathpar}
  \inferrule*[lab=Out-barb]{x \nameeq y}{{y}!\langle{Q}\rangle \vdash x}
  \and
  \inferrule*[lab=Par-barb]{\mbox{$P\vdash x$ or $Q\vdash x$}}{\binpar{P}{Q} \vdash x}
\end{mathpar}

\subsubsection{Contexts}

One of the principle advantages of computational calculi like the
$\pi$-calculus is a well-defined notion of context,
contextual-equivalence and a correlation between
contextual-equivalence and notions of bisimulation. The notion of
context allows the decomposition of a process into (sub-)process and
its syntactic environment, its context. Thus, a context may be
thought of as a process with a ``hole'' (written $\Box$) in it. The
application of a context $M$ to a process $P$, written $M[P]$, is
tantamount to filling the hole in $M$ with $P$. In this paper we do
not need the full weight of this theory, but do make use of the notion
of context in the proof the main theorem. 

\begin{mathpar}
  \inferrule* [lab=summation] {} {{M_{M},M_{N}} \bc \Box \;|\; x.M_{A} \;|\; M_{M}+M_{N}}
  \and
  \inferrule* [lab=agent] {} {{M_{A}} \bc (\vec{x})M_{P} \;| \; \clift{P_0,\ldots,M_{P},\ldots,P_N}}
  \and \\
  \inferrule* [lab=process] {} {{M_{P}} \bc M_{N} \;| \;P|M_{P} }
\end{mathpar} 

\begin{mathpar}
  \inferrule* [lab=sychronization] {} {M_{N} \bc \Box \;|\; x?M_{F} \;|\; x!M_{C}}
  \and
  \inferrule* [lab=abstraction] {} {{M_{F}} \bc (x)M_{P} }
  \and
  \inferrule* [lab=concretion] {} {{M_{C}} \bc \langle M_{P} \rangle }
  \and \\
  \inferrule* [lab=process] {} {{M_{P}} \bc M_{N} \;| \;P|M_{P} }
\end{mathpar}

\begin{definition}[contextual application] Given a context $M$, and
  process $P$, we define the \emph{contextual application}, $M[P] :=
  M\{P/\Box\}$. That is, the contextual application of M to P is the
  substitution of $P$ for $\Box$ in $M$.
\end{definition}

$\meaningof{-} : L \to \mathcal{P}(\pi)$

\begin{mathpar}
  \inferrule* [lab=collection] {} {\meaningof{true} = \pi, \and \meaningof{~E} = \pi \setminus \meaningof{E}, \and \meaningof{E_{1} \& E_{2}} = \meaningof{E_{1}} \cap \meaningof{E_{2}}}
\end{mathpar}

\begin{mathpar}
  \inferrule* [lab=structure] {} {\meaningof{0} = \{ P \in \pi | P \equiv 0 \}, \and \\ \meaningof{E_1 | E_2} = \{ P \in \pi | P \equiv P_{1} | P_{2}, P_{1} \in \meaningof{E_{1}}, P_{2} \in \meaningof{E_2}\} }
\end{mathpar}

\begin{mathpar}
 \inferrule* [lab=behavior] {} {\meaningof{\langle a?b \rangle E} = \{ P \in \pi | P \equiv Q | u?(y)P', \\ \and \\\\ \and \\ \;\;\; u \in \meaningof{a}, \forall z.P'\{z/y\} \in \meaningof{E\{z/b\}}\}, \and \\ \meaningof{a!E} = \{ P \in \pi | P \equiv Q | x!\langle P' \rangle, x \in \meaningof{a} P' \in \meaningof{E}\} }
\end{mathpar}

\begin{mathpar}
 \inferrule* [lab=nominal] {} {\meaningof{\quotep{E}} = \{ \quotep{P} \in \quotep{\pi} | P \in \meaningof{E} \}, \and \meaningof{\quotep{P}} = \{ \quotep{Q} \in \quotep{\pi} | P \equiv Q \} \and \\ \meaningof{@\quotep{E}} = \{ P \in \pi | P \equiv @x, x \in \meaningof{E} \}}
\end{mathpar}

\begin{eqnarray*}
  \\
  \meaningof{-} : TS \to ST
\end{eqnarray*}

\begin{eqnarray*}
  \\
  L : TS \to ST
\end{eqnarray*}

\begin{eqnarray*}
  \\
  P \models E \iff P \in \meaningof{E}
\end{eqnarray*}

\begin{eqnarray*}
  P \approx_{L} Q \iff \forall E \in L. P \models E \iff Q \models E
\end{eqnarray*}

\begin{eqnarray*}
  P \approx_{K} Q
\end{eqnarray*}

\begin{eqnarray*}
  P \approx Q
\end{eqnarray*}

$\approx_{K} = \approx = \approx_{L}$

\subsubsection{Contextual duality}

Note that contexts extend the quotation operation to a family of
operations from processes to names. Given a context, $M$, we can
define a \emph{nominal context}, $\quotep{M}$ by $\quotep{M}[P] :=
\quotep{M[P]}$. To foreshadow what is to come we observe that these
operations enjoy a duality with processes very much like the duality
between vectors and maps from vectors to scalars.

Further, because the calculus is essentially higher-order, we have a
correspondence between contexts and processes. More specifically,
given a name $x$ and a context $M$ we can construct $M^{*}_{x}$ such
that 

\begin{mathpar}
  M^{*}_{x} | \lift{x}{P} \red M[P]
\end{mathpar}

namely,

\begin{mathpar}
  M^{*}_{x} := x?(u).M[\dropn{u}]
\end{mathpar}

The dependence of $M^{*}_{x}$ on a name makes it an abstraction, 

\begin{mathpar}
  M^{*} := (x)x?(u).M[\dropn{u}]
\end{mathpar}

\subsection{Additional notation}

It will sometimes be convenient to denote the process a name
quotes. We already have the notation $x = \quotep{P}$, but it will be
convenient to introduce an alternate notation, $\procn{x}$, when we
want to emphasize the connection to the use of the name. Note that, by
virtue of name equivalence, $\quotep{\procn{x}} \nameeq x$; so, the
notation is consistent with previous definitions.

Further, because names have structure it is possible to effect
substitutions on the basis of that structure. This means we need to
upgrade our notation for substitutions, which we accomplish by
adapting comprehension notation. Thus,

\begin{mathpar}
  P\{ y / x : x \in S \}
\end{mathpar}

is interpreted to mean the process derived from P by replacing (in a
capture-avoiding manner) each occurrence of $x$ in $S$ by $y$. For example,

\begin{mathpar}
  P\{ \quotep{\procn{x}|\procn{x}} / x : x \in \freenames{P} \}
\end{mathpar}

will replace each (occurrence) of a free name $x$ in $P$ by
$\quotep{\procn{x}|\procn{x}}$.

Also, we will avail ourselves of the notation $x^{L}$ and $x^{R}$ to
denote injections of a name into disjoint copies of the name
space. There are numerous ways to accomplish this. One example can be
found in \cite{MeredithR05}. This notation overloads to vectors of
names: $\vec{x}^{\pi} := (x_{i}^{\pi} \; : \; 0 \leq i < |\vec{x}| )$ where $\pi \in \{L,R\}$.

We also use $P^{\Box} := P|\Box$.

In \cite{MeredithR05} an interpretation of the new operator is
given. It turns out that there are several possible interpretations
all enjoying the requisite algebraic properties of the operator (see
\cite{milner91polyadicpi}). We will therefore make liberal use of
$(\nu\; \vec{x})P$.

% subsection the_syntax_and_semantics_of_the_notation_system (end)   

\input{qm2pi.qmops} 

\input{qm2pi.sterngerlach} 

\input{qm2pi.metric} 

% section concurrent_process_calculi (end)

%\input{qm2pi.proofsketch}

% section proof sketch (end)

%\input{qm2pi.slviaknots} 

% section spatial logic via knots (end)

\input{qm2pi.conclusion}

% section conclusion (end)

%\input{qm2pi.dtcodes} 

% section wiring algorithm (end)

\input{qm2pi.ack} 

% section acknowledgments (end)

\newpage


\bibliographystyle{plain}   
\bibliography{../../biblios/main.bib}

\input{qm2pi.rhodetails}

\end{document}

 

% subsection basic_interpretation (end)

%\input{qm2pi.rho.presentation} 
\subsection{The syntax and semantics of the notation system}\label{sub:the_syntax_and_semantics_of_the_notation_system} % (fold)

We now summarize a technical presentation of the calculus that
embodies our theory of dynamics. The typical presentation of such a
calculus follows the style of giving generators and relations on
them. The grammar, below, describing term constructors, freely
generates the set of processes, $\Proc$. This set is then quotiented
by a relation known as structural congruence and it is over this set
that the notion of dynamics is expressed. This presentation is
essentially that of \cite{MeredithR05} with the addition of
polyadicity and summation. For readability we have relegated some of
the technical subtleties to an appendix.

\subsubsection{Process grammar}\label{subsub:process_grammar}

\begin{mathpar}
  \inferrule* [lab=synchronization] {} {{M} \bc \pzero \;|\; x?F \;|\; x!C }
  \and
  \inferrule* [lab=abstraction] {} {{F} \bc (x)P}
  \and
  \inferrule* [lab=concretion] {} {{C} \bc \langle Q \rangle}
  \and
  \inferrule* [lab=process] {} {{P,Q} \bc M \;| \;P|Q \;|\; @{x}}
  \and
  \inferrule* [lab=name] {} {{x} \bc \quotep{P}}
\end{mathpar} 

Note that $\vec{x}$ (resp. $\vec{P}$) denotes a vector of names
(resp. processes) of length $|\vec{x}|$ (resp. $|\vec{P}|$). We adopt
the following useful abbreviations.

\begin{mathpar}
   x?(\vec{y}).P := x.(\vec{y})P \and  x\clift{\vec{P}} := x.\clift{\vec{P}}
   \and x!(y) := \lift{x}{\dropn{y}}
   \and \Pi_{i=0}^{n-1}P_i := P_0 | \ldots | P_{n-1}
\end{mathpar}

\subsubsection{Structural congruence}

\paragraph{Free and bound names and alpha-equivalence.} At the
core of structural equivalence is alpha-equivalence which identifies
process that are the same up to a change of variable. Formally, we
recognize the distinction between free and bound names. The free names
of a process, $\freenames{P}$, may be calculated recursively as
follows:

\begin{mathpar}
\freenames{\pzero} := \emptyset
  \and \\
  \freenames{x?(y).P} := \{ x \} \cup (\freenames{P} \setminus \{ y \})
  \and 
  \freenames{x!\langle P \rangle} := \{ x \} \cup \{ P \} 
  \and \\
  \freenames{P|Q} := \freenames{P} \cup \freenames{Q}
  \and \\
  \freenames{@{x}} := \{ x \}
\end{mathpar}

$\pi$
$\quotep{\pi}$

$\freenames{-} : \pi \to \mathcal{P}(\quotep{\pi})$

\begin{eqnarray*}
  \freenames{\pzero} & := & \emptyset \\
  \freenames{x?(y).P} & := & \{ x \} \cup (\freenames{P} \setminus \{ y \}) \\
  \freenames{x!\langle P \rangle} & := & \{ x \} \cup \{ P \} \\
  \freenames{P|Q} & := & \freenames{P} \cup \freenames{Q} \\
  \freenames{\dropn{x}} & := & \{ x \}
\end{eqnarray*}

The bound names of a process, $\boundnames{P}$, are those names occurring in $P$
that are not free. For example, in $x?(y).0$, the name $x$ is free, while $y$ is bound.

\begin{mathpar}
  \inferrule* [lab=monoidal-laws] {} { P|Q \equiv Q|P \and P|0 \equiv P \and P|(Q|R) \equiv (P|Q)|R }
\end{mathpar}

\begin{mathpar}
  \inferrule* [lab=alpha-equivalence] {} { (x)P \equiv (y)P\{y/x\} \and y \not\in \freenames{P} }
\end{mathpar}

\begin{definition}
Then two processes, $P,Q$, are alpha-equivalent if $P = Q\{\vec{y}/\vec{x}\}$ for
some $\vec{x} \in \boundnames{Q},\vec{y} \in \boundnames{P}$, where $Q\{\vec{y}/\vec{x}\}$
denotes the capture-avoiding substitution of $\vec{y}$ for $\vec{x}$ in $Q$.
\end{definition}

\begin{definition}
  The {\em structural congruence} \cite{SangiorgiWalker} , $\equiv$,
  between processes is the least congruence containing
  alpha-equivalence, satisfying the abelian monoid laws
  (associativity, commutativity and $\pzero$ as identity) for parallel
  composition $|$ and for summation $+$.
\end{definition}

\subsection{Name equivalence}

We take name equivalence, written $\nameeq$, to be the smallest
equivalence relation generated by the following rules.

\begin{mathpar}
\inferrule*[lab=Quote-drop]
{ }
{ \quotep{@{x}} \nameeq x }

\inferrule*[lab=Struct-equiv]
{ P \scong Q }
{ \quotep{P} \nameeq \quotep{Q} }
\end{mathpar}

The astute reader will have noticed that the mutual recursion of names
and processes imposes a mutual recursion on alpha-equivalence and
structural equivalence via name-equivalence. Fortunately, all of this
works out pleasantly and we may calculate in the natural way, free of
concern. The reader interested in the details is referred to the
appendix \ref{appendix:rho_details}.

\subsection{Substitution}

We use $\Proc$ for the set of processes, $\QProc$ for the set of
names, and $\id{\{}\vec{y} / \vec{x} \id{\}}$ to denote partial maps,
$s : \QProc \rightarrow \QProc$. A map, $s$ lifts, uniquely, to a map
on process terms, $\widehat{s} : \Proc \rightarrow \Proc$ by the
following equations.

\begin{mathpar}
  (0) \psubstp{Q}{P} := 0 \\
  (R \juxtap S) \psubstp{Q}{P}
  :=    
  (R)\psubstp{Q}{P} \juxtap (S) \psubstp{Q}{P} \\
  (x?(y).R) \psubstp{Q}{P}    
  :=    
  (x)\substp{Q}{P} (z)\concat( (R \psubstn{z}{y}) \psubstp{Q}{P} ) \\
  (\lift{x}{R}) \psubstp{Q}{P}  
  :=
  \lift{(x)\substp{Q}{P}}{ R \psubstp{Q}{P} } \\
%   (\dropn{x})  \psubstp{Q}{P}       
%   := 
%   \left\{ 
%     \begin{array}{ccc} 
%       \dropn{\quotep{Q}} & & x \nameeq \quotep{P} \\
%       \dropn{x} & & otherwise \\
%     \end{array}
%   \right. 
  (\dropn{x})  \psubstp{Q}{P}       
  := 
  \left\{ 
    \begin{array}{ccc} 
      Q & & x \nameeq \quotep{P} \\
      \dropn{x} & & otherwise \\
    \end{array}
  \right.
\end{mathpar}
 

where

\begin{eqnarray}
  (x)\id{\{} \lpquote Q \rpquote / \lpquote P \rpquote \id{\}}            = 
  \left\{ 
    \begin{array}{ccc}
      \lpquote Q \rpquote & & x \nameeq \lpquote P \rpquote \\
      x & & otherwise \\
    \end{array}
  \right. \nonumber
\end{eqnarray}

and $z$ is chosen distinct from $\quotep{P}$, $\quotep{Q}$, the free
names in $Q$, and all the names in $R$. Our $\alpha$-equivalence will
be built in the standard way from this substitution.

\begin{remark}\label{rem:no_self_referential_names}
  One consequence of these definitions is that $\forall P. \quotep{P}
  \not\in \freenames{P}$.
\end{remark}

\subsection{ Dynamic quote: an example }

Anticipating something of what's to come, consider applying the
substitution, $\widehat{\id{\{}u / z \id{\}}}$, to the following pair
of processes, $\lift{w}{y!(z)}$ and $w[ \lpquote y!(z) \rpquote ]$.

\begin{eqnarray}
	\lift{w}{y!(z)}\widehat{\id{\{}u / z \id{\}}}
		& = &
		\lift{w}{y!(u)} \nonumber\\
	w[ \lpquote y!(z) \rpquote ] \widehat{ \id{\{}u / z \id{\}} }
		& = &
		w[ \lpquote y!(z) \rpquote ] \nonumber
\end{eqnarray}

Because the body of the process between quotes is impervious to
substitution, we get radically different answers. In fact, by
examining the first process in an input context,
e.g. $x?(z).\lift{w}{y!(z)}$, we see that the process under the lift
operator may be shaped by prefixed inputs binding a name inside it. In
this sense, the lift operator will be seen as a way to dynamically
construct processes before reifying them as names.

Finally equipped with these standard features we can present the
dynamics of the calculus.

\subsubsection{Operational semantics} 

Finally, we introduce the computational dynamics. What marks these
algebras as distinct from other more traditionally studied algebraic
structures, e.g. vector spaces or polynomial rings, is the manner in
which dynamics is captured. In traditional structures, dynamics is typically
expressed through morphisms between such structures, as in linear maps
between vector spaces or morphisms between rings. In algebras
associated with the semantics of computation, the dynamics is
expressed as part of the algebraic structure itself, through a
reduction reduction relation typically denoted by $\red$. Below, we
give a recursive presentation of this relation for the calculus used
in the encoding.

$\red \subseteq \pi \times \pi$
$\red : \pi \to \mathcal{P}(\pi)$

\begin{mathpar}
  \inferrule* [lab=Comm] { \textsf{match}( x_{src}, x_{trgt} ) } { x_{trgt}?(y)P \; | \; x_{src}!\langle {Q} \rangle \red P\{\quotep{Q}/y}\} }
  \and \\
  \inferrule* [lab=Par] {{P} \red {P}'} {{{P} | {Q}} \red {{P}' | {Q}}}
  \and
  \inferrule* [lab=Equiv]{{{P} \scong {P}'} \andalso {{P}' \red {Q}'} \andalso {{Q}' \scong {Q}}}{{P} \red {Q}}
\end{mathpar}

\begin{eqnarray*}
  match_{\equiv} (\quotep{P},\quotep{Q}) & := & P \equiv Q \\
  match_{\dagger}(\quotep{P},\quotep{Q}) & := & \forall R. P|Q \red^{*} R => R \red^{*} 0 \\
  match_{K}(\quotep{P},\quotep{Q}) & := & K \mbox{ for some context } K
\end{eqnarray*}

$u?(x)P | u!\langle Q \rangle \red P\{\quotep{Q}/x\}$

%We write $\wred$ for $\red^*$, and $P\red$ if $\exists Q $ such that $ P \red Q$.
We write $P\red$ if $\exists Q $ such that $ P \red Q$ and $P\not\red$, otherwise.

\section{Replication}

As mentioned before, it is known that replication (and hence
recursion) can be implemented in a higher-order process algebra
\cite{SangiorgiWalker}. As our first example of calculation with the
machinery thus far presented we give the construction explicitly in
the {\rhoc}.

\begin{eqnarray}
	D_{x} & := & \prefix{x}{y}{(\binpar{\outputp{x}{y}}{@{y}})} \nonumber\\
	\bangp_{x}{P} & := & \binpar{{x}!\langle{\binpar{D_{x}}{P}}\rangle}{D_{x}} \nonumber
\end{eqnarray}

\begin{eqnarray}
	\bangp_{x}{P} & & \nonumber\\
	=
	& {x}!\langle{(\prefix{x}{y}{(\outputp{x}{y} | @{y})) | P}}\rangle 
	      | \prefix{x}{y}{(\outputp{x}{y} | @{y})} & \nonumber\\
	\red
	& (\outputp{x}{y} | @{y})\substn{\quotep{(\prefix{x}{y}{(@{y} | \outputp{x}{y})) | P}}}{y} & \nonumber\\
	=
	& \outputp{x}{\quotep{(\prefix{x}{y}{(\outputp{x}{y} | @{y})) | P}}}
	  | {(\prefix{x}{y}{(\outputp{x}{y} | @{y})) | P}} & \nonumber\\
	\red
	& \ldots & \nonumber\\
	\red^*
	& P | P | \ldots & \nonumber
\end{eqnarray}

Of course, this encoding, as an implementation, runs away, unfolding
$\bangp{P}$ eagerly. A lazier and more implementable replication
operator, restricted to input-guarded processes, may be obtained as follows.

\begin{eqnarray}
\bangp{\prefix{u}{v}{P}} 
	:= 
	\binpar{\lift{x}{\prefix{u}{v}{(\binpar{D(x)}{P})}}}{D(x)} \nonumber
\end{eqnarray}

\begin{remark}
  Note that the lazier definition still does not deal with summation
  or mixed summation (i.e. sums over input and output). The reader is
  invited to construct definitions of replication that deal with these
  features. 

  Further, the definitions are parameterized in a name, $x$. Can you,
  gentle reader, make a definition that eliminates this parameter and
  guarantees no accidental interaction between the replication
  machinery and the process being replicated -- i.e. no accidental
  sharing of names used by the process to get its work done and the
  name(s) used by the replication to effect copying. This latter
  revision of the definition of replication is crucial to obtaining
  the expected identity $!!P \sim !P$.
\end{remark}

\begin{remark}\label{rem:paradoxical_combinator}
  The reader familiar with the lambda calculus will have noticed the
  similarity between $D$ and the paradoxical combinator.

  [Ed. note: the existence of this seems to suggest we have to be more
  restrictive on the set of processes and names we admit if we are to
  support no-cloning.]
\end{remark}

\subsubsection{Bisimulation}

The computational dynamics gives rise to another kind of equivalence,
the equivalence of computational behavior. As previously mentioned
this is typically captured \emph{via} some form of bisimulation.

% The notion we use in this paper is weak barbed bisimulation
% \cite{milner91polyadicpi}.

The notion we use in this paper is derived from weak barbed
bisimulation \cite{milner91polyadicpi}. 

\begin{definition}
An \emph{observation relation}, $\downarrow_{\mathcal N}$, over a set
of names, $\mathcal N$, is the smallest relation satisfying the rules
below.

\infrule[Out-barb]{y \in {\mathcal N}, \; x \nameeq y}
		  {\outputp{x}{v} \downarrow_{\mathcal N} x}
\infrule[Par-barb]{\mbox{$P\downarrow_{\mathcal N} x$ or $Q\downarrow_{\mathcal N} x$}}
		  {\binpar{P}{Q} \downarrow_{\mathcal N} x}

We write $P \Downarrow_{\mathcal N} x$ if there is $Q$ such that 
$P \wred Q$ and $Q \downarrow_{\mathcal N} x$.
\end{definition}

\begin{definition}
%\label{def.bbisim}
An  ${\mathcal N}$-\emph{barbed bisimulation} over a set of names, ${\mathcal N}$, is a symmetric binary relation 
${\mathcal S}_{\mathcal N}$ between agents such that $P\rel{S}_{\mathcal N}Q$ implies:
\begin{enumerate}
\item If $P \red P'$ then $Q \wred Q'$ and $P'\rel{S}_{\mathcal N} Q'$.
\item If $P\downarrow_{\mathcal N} x$, then $Q\Downarrow_{\mathcal N} x$.
\end{enumerate}
$P$ is ${\mathcal N}$-barbed bisimilar to $Q$, written
$P \wbbisim_{\mathcal N} Q$, if $P \rel{S}_{\mathcal N} Q$ for some ${\mathcal N}$-barbed bisimulation ${\mathcal S}_{\mathcal N}$.
\end{definition}

$\mathcal{R} \subseteq \pi \times \pi$

$P \mathcal{R} Q => \forall P'. P \red P' \Rightarrow \exists Q'. Q \red Q', P' \mathcal{R} Q'$

$P \vdash x \Rightarrow Q \vdash x$

\begin{mathpar}
  \inferrule*[lab=Out-barb]{x \nameeq y}{{y}!\langle{Q}\rangle \vdash x}
  \and
  \inferrule*[lab=Par-barb]{\mbox{$P\vdash x$ or $Q\vdash x$}}{\binpar{P}{Q} \vdash x}
\end{mathpar}

\subsubsection{Contexts}

One of the principle advantages of computational calculi like the
$\pi$-calculus is a well-defined notion of context,
contextual-equivalence and a correlation between
contextual-equivalence and notions of bisimulation. The notion of
context allows the decomposition of a process into (sub-)process and
its syntactic environment, its context. Thus, a context may be
thought of as a process with a ``hole'' (written $\Box$) in it. The
application of a context $M$ to a process $P$, written $M[P]$, is
tantamount to filling the hole in $M$ with $P$. In this paper we do
not need the full weight of this theory, but do make use of the notion
of context in the proof the main theorem. 

\begin{mathpar}
  \inferrule* [lab=summation] {} {{M_{M},M_{N}} \bc \Box \;|\; x.M_{A} \;|\; M_{M}+M_{N}}
  \and
  \inferrule* [lab=agent] {} {{M_{A}} \bc (\vec{x})M_{P} \;| \; \clift{P_0,\ldots,M_{P},\ldots,P_N}}
  \and \\
  \inferrule* [lab=process] {} {{M_{P}} \bc M_{N} \;| \;P|M_{P} }
\end{mathpar} 

\begin{mathpar}
  \inferrule* [lab=sychronization] {} {M_{N} \bc \Box \;|\; x?M_{F} \;|\; x!M_{C}}
  \and
  \inferrule* [lab=abstraction] {} {{M_{F}} \bc (x)M_{P} }
  \and
  \inferrule* [lab=concretion] {} {{M_{C}} \bc \langle M_{P} \rangle }
  \and \\
  \inferrule* [lab=process] {} {{M_{P}} \bc M_{N} \;| \;P|M_{P} }
\end{mathpar}

\begin{definition}[contextual application] Given a context $M$, and
  process $P$, we define the \emph{contextual application}, $M[P] :=
  M\{P/\Box\}$. That is, the contextual application of M to P is the
  substitution of $P$ for $\Box$ in $M$.
\end{definition}

$\meaningof{-} : L \to \mathcal{P}(\pi)$

\begin{mathpar}
  \inferrule* [lab=collection] {} {\meaningof{true} = \pi, \and \meaningof{~E} = \pi \setminus \meaningof{E}, \and \meaningof{E_{1} \& E_{2}} = \meaningof{E_{1}} \cap \meaningof{E_{2}}}
\end{mathpar}

\begin{mathpar}
  \inferrule* [lab=structure] {} {\meaningof{0} = \{ P \in \pi | P \equiv 0 \}, \and \\ \meaningof{E_1 | E_2} = \{ P \in \pi | P \equiv P_{1} | P_{2}, P_{1} \in \meaningof{E_{1}}, P_{2} \in \meaningof{E_2}\} }
\end{mathpar}

\begin{mathpar}
 \inferrule* [lab=behavior] {} {\meaningof{\langle a?b \rangle E} = \{ P \in \pi | P \equiv Q | u?(y)P', \\ \and \\\\ \and \\ \;\;\; u \in \meaningof{a}, \forall z.P'\{z/y\} \in \meaningof{E\{z/b\}}\}, \and \\ \meaningof{a!E} = \{ P \in \pi | P \equiv Q | x!\langle P' \rangle, x \in \meaningof{a} P' \in \meaningof{E}\} }
\end{mathpar}

\begin{mathpar}
 \inferrule* [lab=nominal] {} {\meaningof{\quotep{E}} = \{ \quotep{P} \in \quotep{\pi} | P \in \meaningof{E} \}, \and \meaningof{\quotep{P}} = \{ \quotep{Q} \in \quotep{\pi} | P \equiv Q \} \and \\ \meaningof{@\quotep{E}} = \{ P \in \pi | P \equiv @x, x \in \meaningof{E} \}}
\end{mathpar}

\begin{eqnarray*}
  \\
  \meaningof{-} : TS \to ST
\end{eqnarray*}

\begin{eqnarray*}
  \\
  L : TS \to ST
\end{eqnarray*}

\begin{eqnarray*}
  \\
  P \models E \iff P \in \meaningof{E}
\end{eqnarray*}

\begin{eqnarray*}
  P \approx_{L} Q \iff \forall E \in L. P \models E \iff Q \models E
\end{eqnarray*}

\begin{eqnarray*}
  P \approx_{K} Q
\end{eqnarray*}

\begin{eqnarray*}
  P \approx Q
\end{eqnarray*}

$\approx_{K} = \approx = \approx_{L}$

\subsubsection{Contextual duality}

Note that contexts extend the quotation operation to a family of
operations from processes to names. Given a context, $M$, we can
define a \emph{nominal context}, $\quotep{M}$ by $\quotep{M}[P] :=
\quotep{M[P]}$. To foreshadow what is to come we observe that these
operations enjoy a duality with processes very much like the duality
between vectors and maps from vectors to scalars.

Further, because the calculus is essentially higher-order, we have a
correspondence between contexts and processes. More specifically,
given a name $x$ and a context $M$ we can construct $M^{*}_{x}$ such
that 

\begin{mathpar}
  M^{*}_{x} | \lift{x}{P} \red M[P]
\end{mathpar}

namely,

\begin{mathpar}
  M^{*}_{x} := x?(u).M[\dropn{u}]
\end{mathpar}

The dependence of $M^{*}_{x}$ on a name makes it an abstraction, 

\begin{mathpar}
  M^{*} := (x)x?(u).M[\dropn{u}]
\end{mathpar}

\subsection{Additional notation}

It will sometimes be convenient to denote the process a name
quotes. We already have the notation $x = \quotep{P}$, but it will be
convenient to introduce an alternate notation, $\procn{x}$, when we
want to emphasize the connection to the use of the name. Note that, by
virtue of name equivalence, $\quotep{\procn{x}} \nameeq x$; so, the
notation is consistent with previous definitions.

Further, because names have structure it is possible to effect
substitutions on the basis of that structure. This means we need to
upgrade our notation for substitutions, which we accomplish by
adapting comprehension notation. Thus,

\begin{mathpar}
  P\{ y / x : x \in S \}
\end{mathpar}

is interpreted to mean the process derived from P by replacing (in a
capture-avoiding manner) each occurrence of $x$ in $S$ by $y$. For example,

\begin{mathpar}
  P\{ \quotep{\procn{x}|\procn{x}} / x : x \in \freenames{P} \}
\end{mathpar}

will replace each (occurrence) of a free name $x$ in $P$ by
$\quotep{\procn{x}|\procn{x}}$.

Also, we will avail ourselves of the notation $x^{L}$ and $x^{R}$ to
denote injections of a name into disjoint copies of the name
space. There are numerous ways to accomplish this. One example can be
found in \cite{MeredithR05}. This notation overloads to vectors of
names: $\vec{x}^{\pi} := (x_{i}^{\pi} \; : \; 0 \leq i < |\vec{x}| )$ where $\pi \in \{L,R\}$.

We also use $P^{\Box} := P|\Box$.

In \cite{MeredithR05} an interpretation of the new operator is
given. It turns out that there are several possible interpretations
all enjoying the requisite algebraic properties of the operator (see
\cite{milner91polyadicpi}). We will therefore make liberal use of
$(\nu\; \vec{x})P$.

% subsection the_syntax_and_semantics_of_the_notation_system (end)   

\section{Interpretation of QM}
\subsection{Supporting definitions}
\subsubsection{Multiplication}
\begin{mathpar}
  \quotep{Q} \cdot \quotep{R} := \quotep{Q|R}
  \and \\
  \quotep{Q} \cdot P := P\{ \quotep{Q|R} / \quotep{R} : \quotep{R} \in \freenames{P} \}
\end{mathpar}

\paragraph{Discussion}
The first line needs little explanation. The second line says that
each free name of the process is replaced with the multiplication of
that name by the scalar. Multiplication of a scalar (name) by a state
(process) results in a process all the names of which have been `moved
over' by parallel composition with the process the scalar
quotes. There is a subtlety that the bound names have to be
manipulated so that multiplied names aren't accidentally
captured. There are many ways to achieve this.

\begin{remark}\label{rem:multiplication_identities}
  The reader is invited to verify that for all $x,y,z \in \QProc$ and $P \in \Proc$
  \begin{mathpar}
    x \cdot \quotep{0} \equiv x 
    \and
    x \cdot y \equiv y \cdot x
    \and
    x \cdot (y \cdot z) \equiv (x \cdot y) \cdot z
    \and \\
    \quotep{0} \cdot P \equiv P
    \and \\
    x \cdot (y \cdot P) \equiv (x \cdot y) \cdot P
    \and \\
    x \cdot (P|Q) \equiv (x \cdot P) | (x \cdot Q)
    \and \\    
  \end{mathpar}
\end{remark}

\subsubsection{Tensor product}

We define a tensor product on processes by structural induction.

\paragraph{Tensor of sums} First note that all summations, including
$\pzero$ and sequence, can be written $\Sigma_{i} x_{i}.A_{i} +
\Sigma_{j} x_{j}.C_{j}$, where we have grouped input-guarded processes
together and output-guarded processes together.

Thus, we can define the tensor product of two summations, $N_{1}\otimes N_{2}$, where

\begin{mathpar}
  N_{1} := \Sigma_{i} x_{i}.A_{i} + \Sigma_{j} x_{j}.C_{j}
  \and
  N_{2} := \Sigma_{i'} y_{i'}.B_{i'} + \Sigma_{j'} y_{j'}.D_{j'} 
\end{mathpar}

as follows.

\begin{mathpar}
  \Sigma_{i} x_{i}.A_{i} + \Sigma_{j} x_{j}.C_{j} \otimes \Sigma_{i'}
  y_{i'}.B_{i'} + \Sigma_{j'} y_{j'}.D_{j'} 
  \and \\
  := \; \Sigma_{i} \Sigma_{i'} \quotep{\stackrel{\vee}{x_{i}}| \stackrel{\vee}{y_{i'}}}.(A_{i}\otimes B_{i'}) \; | \; \Sigma_{i'} \Sigma_{i} \quotep{\stackrel{\vee}{y_{i'}}|\stackrel{\vee}{x_{i}}}.(B_{i'}\otimes A_{i})
  \and
  \;\; | \;\; \Sigma_{j} \Sigma_{j'} \quotep{\stackrel{\vee}{x_{j}}|\stackrel{\vee}{y_{j'}}}.(A_{j}\otimes B_{j'}) \; | \; \Sigma_{j'} \Sigma_{j} \quotep{\stackrel{\vee}{y_{j'}}|\stackrel{\vee}{x_{j}}}.(B_{j'}\otimes A_{j})
\end{mathpar}

\begin{remark}
  Do we need to $x^{L}$ and $y^{R}$ for this construction as well?
\end{remark}

\paragraph{Tensor of parallel compositions} Next, we distribute tensor
over par.

\begin{mathpar}
  P_{1}|P_{2} \otimes Q_{1}|Q_{2} := (P_{1} \otimes Q_{1}) | (P_{1}
  \otimes Q_{2}) | (P_{2} \otimes Q_{1}) | (P_{2} \otimes Q_{2})
\end{mathpar}

\paragraph{Tensor with dropped names} We treat tensor of a
process with a dropped name as parallel composition.

\begin{mathpar}
  P \otimes \dropn{x} := P | \dropn{x}
\end{mathpar}

\paragraph{Tensor of agents}

Finally, we need to define tensor on agents. Note that the definition
of tensor on normal products only tensors inputs with inputs and
outputs with outputs. Thus, we only have to define the operation on
``homogeneous'' pairings.

\begin{mathpar}
  (\vec{x})P \otimes (\vec{y})Q
  \and \\
  := (x_{0}^{L}|y_{0}^{R},\ldots,x_{0}^{L}|y_{n}^{R},\ldots,x_{m}^{L}|y_{0}^{R},\ldots,x_{m}^{L}|y_{n}^R)(P\{ \vec{x}^{L}/\vec{x}\} \otimes Q \{ \vec{y}^{R}/\vec{y}\})
  \and \\
  \clift{\vec{P}} \otimes \clift{\vec{Q}}
  \and \\
  := \clift{P_{0}\otimes Q_{0},\ldots,P_{0}\otimes Q_{n},\ldots,P_{m}\otimes Q_{0},\ldots,P_{m}\otimes Q_{n}}
\end{mathpar}

\begin{remark}
  Observe that arities of tensored abstractions matches arities of
  tensored concretions if the original arities matched. Note also that
  the length of the arities corresponds to the increase in dimension
  we see in ordinary vector space tensor product.
\end{remark}

\begin{remark}
  Operationally, this definition distributes the tensor down to
  components ``linked'' by summation. Tensor over summation is
  intriguing in that it mixes names. Moreover, as a consequence of the
  way it mixes names we have the identities for all $x \in \QProc$ and
  $P,Q \in \Proc$

  \begin{mathpar}
    (x \cdot P) \otimes Q \equiv x \cdot (P \otimes Q) \equiv P \otimes (x \cdot Q)
    \and
    P \otimes \pzero \equiv P
  \end{mathpar}

  that the reader is invited to verify.
\end{remark}

\subsubsection{Annihilation}
\begin{mathpar}
  P^{\perp} := \{ Q | \forall R. P|Q \red^{*} R \Rightarrow R \red^{*} \pzero \}
  \and \\
  P^{\underline{\perp}} := \Sigma_{Q \in P^{\perp}} \quotep{Q}?(y).(\dropn{y}|Q) | \Sigma_{Q \in P^{\perp}} \quotep{Q}\clift{\Box}
\end{mathpar}

\paragraph{Discussion} The reader will note that $P^{\perp}$ is a
\emph{set} of processes, while $P^{\underline{\perp}}$ is a
\emph{context}. We call the set $P^{\perp}$ the \emph{annihilators} of
$P$. The parallel composition of a process in the annihilators of $P$
with $P$ will result in a process, the state space of which has all
paths eventually leading to $\pzero$. Execution may endure loops; but
under reasonable conditions of fairness (naturally guaranteed under
most notions of bisimulation) such a composite process cannot get
stuck in such a loop and will, eventually pop out and terminate.

The context $P^{\underline{\perp}}$ is ready and willing to ``take the
$P$ out of'' the process to which it is applied. It will effectively
transmit the code of the process to which it is applied to one of the
annihilators and run the process against it.

\subsubsection{Evaluation}
We fix $M$ a domain of fully abstract interpretation with an equality
coincident with bisimulation. We take $\meaningof{\cdot} : \Proc \to
M$ to be the map interpreting processes and $\nmeaningof{\cdot} : \M
\to Proc$ to be the map running the other way. Then we define

\begin{mathpar}
  \int P := \nmeaningof{\meaningof{P}}
\end{mathpar}

\paragraph{Discussion}
There are many fully abstract interpretations of Milner's
$\pi$-calculus. Any of them can be used as a basis for interpreting
the reflective calculus here. Equipped with such a domain it is
largely a matter of grinding through to check that the Yoneda
construction for the normalization-by-evaluation program can be
extended to this setting.

\begin{remark}
  The reader is invited to verify that $\int (P^{\underline{\perp}}[P]) = 0$.
\end{remark}

\subsection{Quantum mechanics}

Table \ref{tbl:core_qm_op_defns} gives the core operational definitions

\begin{table}[htp]\label{tbl:core_qm_op_defns}
  \center{
    \fbox{
      \begin{tabular}{c|c}
        quantum mechanics & process calculus \\
        \hline
        scalar & $x := \quotep{P}$ \\
        state vector & $\state{P} := P$ \\
        dual & $\state{P}^{*} := \event{P^{\underline{\perp}}} := \quotep{P^{\underline{\perp}}}[-]$ \\
        matrix & $ \Sigma_{\alpha} \state{P_{\alpha}}x_{\alpha}\event{Q_{\alpha}}$ \\
        vector addition & $\state{P} + \state{Q} := \state{P | Q}$ \\
        tensor product & $\state{P} \otimes \state{Q} := \state{P \otimes Q}$ \\
        inner product & $\innerprod{P}{Q} := \quotep{\int P^{\underline{\perp}}[Q]}$ \\
      \end{tabular}
    }
  }
  \caption{QM - operational definitions}
\end{table}

where

\begin{mathpar}
  \prmatrix{P}{Q} := \fprmatrix{P}{\quotep{\pzero}}{Q}
  \and
  \fprmatrix{P}{x}{Q} := (\state{P},x,\event{Q})
  \and
  (\fprmatrix{P}{x}{Q})(\state{R}) := x \cdot \innerprod{Q}{R} \cdot \state{P}
  \and
  (\fprmatrix{P}{x}{Q})(\event{R}) := x \cdot \innerprod{R}{P} \cdot \event{Q}
\end{mathpar}

\paragraph{Discussion}
As promised: vectors (aka states) are represented as processes; duals
as contextual duals; inner product definition should be compared with
standard inner product definition for ....

\begin{remark}
  Assuming $\int (P^{\underline{\perp}}[P]) = 0$, the reader is
  invited to verify that $(\fprmatrix{P}{x}{P})(\state{P}) = x \cdot \state{P}$.
\end{remark}

\begin{remark}
  The reader is invited to verify that $\innerprod{P}{Q}$ could
  equally well have been written $\quotep{\int \stackrel{\vee}{x}}$
  where $x = \event{P^{\underline{\perp}}}(Q)$.

  One of the motivations for this remark is that there is another way
  to factor these operations. We could package up evaluation in the dual:

  \begin{mathpar}
    \state{P}^{*} := \event{\int P^{\underline{\perp}}} := \quotep{\int P^{\underline{\perp}}}[-]
  \end{mathpar}

  and then have inner product defined by
  
  \begin{mathpar}
    \innerprod{P}{Q} := \event{P}(Q)
  \end{mathpar}

  Hopefully, experience with the calculations will provide guidance on
  the best factoring.
\end{remark}

\begin{remark}
  Assuming $\int (P^{\underline{\perp}}[P]) = 0$, the reader is
  invited to verify that $\forall P,Q. (\prmatrix{0}{Q})(\state{0}) =
  \state{0}$ and dually $(\prmatrix{P}{0})(\event{0}) = \event{0}$.
\end{remark}

\begin{remark}
  i'm a little worried that i don't (yet) have proper support for
  complex conjugacy. But, the observation above may give us a
  clue. According to Abramsky, it must be the case that the scalars
  are iso to the homset of the identity for the tensor -- which the
  observation above characterizes. 

  For now, we will simply bookmark the notion with $\overline{x}$.
\end{remark}

\subsubsection{Adjointness}

We need to give a definition of $(\cdot)^{\dagger}$ for matrices. The
obvious candidate definition is
\begin{mathpar}
(\Sigma_{\alpha}\fprmatrix{P_{\alpha}}{x_{\alpha}}{Q_{\alpha}})^{\dagger}
= \Sigma_{\alpha}\fprmatrix{(Q_{\alpha}^{\underline{\perp}})^{*}}{\overline{x}_{\alpha}}{P_{\alpha}^{\underline{\perp}}} 
\end{mathpar}

But, $(Q_{\alpha}^{\underline{\perp}})^{*}$ requires a name along
which to communicate the process to achieve the context application.

\subsubsection{Basis for a basis}
If processes label states and ``addition'' of states (a.k.a. vector
addition) is interpreted as parallel composition, what corresponds to
notions of linear independence and basis? Here, we recall that Yoshida
has developed a set of \emph{combinators} for an asynchronous verison
of Milner's $\pi$-calculus. These are a finite set of processes such
any process can be expressed as parallel composition of these
combinators together with liberal uses of the new operator and
replication. We can simply give a translation of these into the
present calculus and have reasonable expectation that the property
carries over. That is, that the resultant set allows to express all
processes via parallel composition. Note, however, that there is no
new operator or replication in this calculus. As a result, we expect
that the corresponding set is actually infinite. That is, we expect
that the space is actually infinite dimensional.

\begin{remark}
  The attentive reader may be a bit concerned. Certainly, the
  collection $S$, $K$ and $I$ is a finite set of
  combinators. Shouldn't we expect to see a finite set of combinators
  for an effectively equivalent system? i am very sympathetic to this
  critique and feel it warrants full attention. On the other hand, i
  also have in mind the following analogy. The natural numbers, as a
  monoid under addition, has exactly $1$ generator, while the natural
  numbers, as a monoid under multiplication, has countably many
  generators (the primes). We observe that the application of the
  lambda calculus is much less resource sensitive than the parallel
  composition of the $\pi$-calculus. Could it be the case that we have
  an analogy of the form
  
  \begin{mathpar}
    m + n : MN :: m*n : M|N
  \end{mathpar}

  giving a similar blow up in the set of ``primes''?  This is such a
  wonderful thought that, even if it's not true, i think it's worth
  writing down.
\end{remark}
 

\documentclass[12pt]{llncs}
%\documentclass{jktr}

\usepackage[pdftex]{hyperref}                   
\usepackage {listings}
\usepackage {mathpartir}
\usepackage{bcprules}
%\usepackage{listings}
                       
\usepackage{graphicx} 
%\usepackage[margins=2.5cm,nohead,nofoot]{geometry}
%\usepackage{geometry}
\usepackage{amsfonts}
\usepackage{amstext}
\usepackage{latexsym}
\usepackage{amssymb}
\usepackage{color}


%\include{myPreamble}
\include{qm2pi.local} 

%\ifpdf
%\usepackage[pdftex]{graphicx}
%\else
%\usepackage{graphicx}
%\fi

 % \ifpdf
%  \usepackage{pdfsync}
%  \if


%\title{Brief Article}
%\author{David F. Snyder}
%\author{L.G. Meredith}

%\address{Dept. of Math., Texas State University--San Marcos, San Marcos, TX 78666}
       
\pagestyle{empty}


\begin{document}

\lstset{language=[Objective]Caml,frame=shadowbox}

\input{qm2pi.front}

% section front matter (end)

\input{qm2pi.intro} 
 
% section introduction (end)

% \input{qm2pi.knotations} 

% section notation (end)

\input{qm2pi.process.calculi} 

% section concurrent_process_calculi_and_spatial_logics_ (end)
    
%\input{qm2pi.knots2pi} 

%\input{qm2pi.trefoil} 

%\input{qm2pi.mainthm} 

% subsection basic_interpretation (end)

%\input{qm2pi.rho.presentation} 
\subsection{The syntax and semantics of the notation system}\label{sub:the_syntax_and_semantics_of_the_notation_system} % (fold)

We now summarize a technical presentation of the calculus that
embodies our theory of dynamics. The typical presentation of such a
calculus follows the style of giving generators and relations on
them. The grammar, below, describing term constructors, freely
generates the set of processes, $\Proc$. This set is then quotiented
by a relation known as structural congruence and it is over this set
that the notion of dynamics is expressed. This presentation is
essentially that of \cite{MeredithR05} with the addition of
polyadicity and summation. For readability we have relegated some of
the technical subtleties to an appendix.

\subsubsection{Process grammar}\label{subsub:process_grammar}

\begin{mathpar}
  \inferrule* [lab=synchronization] {} {{M} \bc \pzero \;|\; x?F \;|\; x!C }
  \and
  \inferrule* [lab=abstraction] {} {{F} \bc (x)P}
  \and
  \inferrule* [lab=concretion] {} {{C} \bc \langle Q \rangle}
  \and
  \inferrule* [lab=process] {} {{P,Q} \bc M \;| \;P|Q \;|\; @{x}}
  \and
  \inferrule* [lab=name] {} {{x} \bc \quotep{P}}
\end{mathpar} 

Note that $\vec{x}$ (resp. $\vec{P}$) denotes a vector of names
(resp. processes) of length $|\vec{x}|$ (resp. $|\vec{P}|$). We adopt
the following useful abbreviations.

\begin{mathpar}
   x?(\vec{y}).P := x.(\vec{y})P \and  x\clift{\vec{P}} := x.\clift{\vec{P}}
   \and x!(y) := \lift{x}{\dropn{y}}
   \and \Pi_{i=0}^{n-1}P_i := P_0 | \ldots | P_{n-1}
\end{mathpar}

\subsubsection{Structural congruence}

\paragraph{Free and bound names and alpha-equivalence.} At the
core of structural equivalence is alpha-equivalence which identifies
process that are the same up to a change of variable. Formally, we
recognize the distinction between free and bound names. The free names
of a process, $\freenames{P}$, may be calculated recursively as
follows:

\begin{mathpar}
\freenames{\pzero} := \emptyset
  \and \\
  \freenames{x?(y).P} := \{ x \} \cup (\freenames{P} \setminus \{ y \})
  \and 
  \freenames{x!\langle P \rangle} := \{ x \} \cup \{ P \} 
  \and \\
  \freenames{P|Q} := \freenames{P} \cup \freenames{Q}
  \and \\
  \freenames{@{x}} := \{ x \}
\end{mathpar}

$\pi$
$\quotep{\pi}$

$\freenames{-} : \pi \to \mathcal{P}(\quotep{\pi})$

\begin{eqnarray*}
  \freenames{\pzero} & := & \emptyset \\
  \freenames{x?(y).P} & := & \{ x \} \cup (\freenames{P} \setminus \{ y \}) \\
  \freenames{x!\langle P \rangle} & := & \{ x \} \cup \{ P \} \\
  \freenames{P|Q} & := & \freenames{P} \cup \freenames{Q} \\
  \freenames{\dropn{x}} & := & \{ x \}
\end{eqnarray*}

The bound names of a process, $\boundnames{P}$, are those names occurring in $P$
that are not free. For example, in $x?(y).0$, the name $x$ is free, while $y$ is bound.

\begin{mathpar}
  \inferrule* [lab=monoidal-laws] {} { P|Q \equiv Q|P \and P|0 \equiv P \and P|(Q|R) \equiv (P|Q)|R }
\end{mathpar}

\begin{mathpar}
  \inferrule* [lab=alpha-equivalence] {} { (x)P \equiv (y)P\{y/x\} \and y \not\in \freenames{P} }
\end{mathpar}

\begin{definition}
Then two processes, $P,Q$, are alpha-equivalent if $P = Q\{\vec{y}/\vec{x}\}$ for
some $\vec{x} \in \boundnames{Q},\vec{y} \in \boundnames{P}$, where $Q\{\vec{y}/\vec{x}\}$
denotes the capture-avoiding substitution of $\vec{y}$ for $\vec{x}$ in $Q$.
\end{definition}

\begin{definition}
  The {\em structural congruence} \cite{SangiorgiWalker} , $\equiv$,
  between processes is the least congruence containing
  alpha-equivalence, satisfying the abelian monoid laws
  (associativity, commutativity and $\pzero$ as identity) for parallel
  composition $|$ and for summation $+$.
\end{definition}

\subsection{Name equivalence}

We take name equivalence, written $\nameeq$, to be the smallest
equivalence relation generated by the following rules.

\begin{mathpar}
\inferrule*[lab=Quote-drop]
{ }
{ \quotep{@{x}} \nameeq x }

\inferrule*[lab=Struct-equiv]
{ P \scong Q }
{ \quotep{P} \nameeq \quotep{Q} }
\end{mathpar}

The astute reader will have noticed that the mutual recursion of names
and processes imposes a mutual recursion on alpha-equivalence and
structural equivalence via name-equivalence. Fortunately, all of this
works out pleasantly and we may calculate in the natural way, free of
concern. The reader interested in the details is referred to the
appendix \ref{appendix:rho_details}.

\subsection{Substitution}

We use $\Proc$ for the set of processes, $\QProc$ for the set of
names, and $\id{\{}\vec{y} / \vec{x} \id{\}}$ to denote partial maps,
$s : \QProc \rightarrow \QProc$. A map, $s$ lifts, uniquely, to a map
on process terms, $\widehat{s} : \Proc \rightarrow \Proc$ by the
following equations.

\begin{mathpar}
  (0) \psubstp{Q}{P} := 0 \\
  (R \juxtap S) \psubstp{Q}{P}
  :=    
  (R)\psubstp{Q}{P} \juxtap (S) \psubstp{Q}{P} \\
  (x?(y).R) \psubstp{Q}{P}    
  :=    
  (x)\substp{Q}{P} (z)\concat( (R \psubstn{z}{y}) \psubstp{Q}{P} ) \\
  (\lift{x}{R}) \psubstp{Q}{P}  
  :=
  \lift{(x)\substp{Q}{P}}{ R \psubstp{Q}{P} } \\
%   (\dropn{x})  \psubstp{Q}{P}       
%   := 
%   \left\{ 
%     \begin{array}{ccc} 
%       \dropn{\quotep{Q}} & & x \nameeq \quotep{P} \\
%       \dropn{x} & & otherwise \\
%     \end{array}
%   \right. 
  (\dropn{x})  \psubstp{Q}{P}       
  := 
  \left\{ 
    \begin{array}{ccc} 
      Q & & x \nameeq \quotep{P} \\
      \dropn{x} & & otherwise \\
    \end{array}
  \right.
\end{mathpar}
 

where

\begin{eqnarray}
  (x)\id{\{} \lpquote Q \rpquote / \lpquote P \rpquote \id{\}}            = 
  \left\{ 
    \begin{array}{ccc}
      \lpquote Q \rpquote & & x \nameeq \lpquote P \rpquote \\
      x & & otherwise \\
    \end{array}
  \right. \nonumber
\end{eqnarray}

and $z$ is chosen distinct from $\quotep{P}$, $\quotep{Q}$, the free
names in $Q$, and all the names in $R$. Our $\alpha$-equivalence will
be built in the standard way from this substitution.

\begin{remark}\label{rem:no_self_referential_names}
  One consequence of these definitions is that $\forall P. \quotep{P}
  \not\in \freenames{P}$.
\end{remark}

\subsection{ Dynamic quote: an example }

Anticipating something of what's to come, consider applying the
substitution, $\widehat{\id{\{}u / z \id{\}}}$, to the following pair
of processes, $\lift{w}{y!(z)}$ and $w[ \lpquote y!(z) \rpquote ]$.

\begin{eqnarray}
	\lift{w}{y!(z)}\widehat{\id{\{}u / z \id{\}}}
		& = &
		\lift{w}{y!(u)} \nonumber\\
	w[ \lpquote y!(z) \rpquote ] \widehat{ \id{\{}u / z \id{\}} }
		& = &
		w[ \lpquote y!(z) \rpquote ] \nonumber
\end{eqnarray}

Because the body of the process between quotes is impervious to
substitution, we get radically different answers. In fact, by
examining the first process in an input context,
e.g. $x?(z).\lift{w}{y!(z)}$, we see that the process under the lift
operator may be shaped by prefixed inputs binding a name inside it. In
this sense, the lift operator will be seen as a way to dynamically
construct processes before reifying them as names.

Finally equipped with these standard features we can present the
dynamics of the calculus.

\subsubsection{Operational semantics} 

Finally, we introduce the computational dynamics. What marks these
algebras as distinct from other more traditionally studied algebraic
structures, e.g. vector spaces or polynomial rings, is the manner in
which dynamics is captured. In traditional structures, dynamics is typically
expressed through morphisms between such structures, as in linear maps
between vector spaces or morphisms between rings. In algebras
associated with the semantics of computation, the dynamics is
expressed as part of the algebraic structure itself, through a
reduction reduction relation typically denoted by $\red$. Below, we
give a recursive presentation of this relation for the calculus used
in the encoding.

$\red \subseteq \pi \times \pi$
$\red : \pi \to \mathcal{P}(\pi)$

\begin{mathpar}
  \inferrule* [lab=Comm] { \textsf{match}( x_{src}, x_{trgt} ) } { x_{trgt}?(y)P \; | \; x_{src}!\langle {Q} \rangle \red P\{\quotep{Q}/y}\} }
  \and \\
  \inferrule* [lab=Par] {{P} \red {P}'} {{{P} | {Q}} \red {{P}' | {Q}}}
  \and
  \inferrule* [lab=Equiv]{{{P} \scong {P}'} \andalso {{P}' \red {Q}'} \andalso {{Q}' \scong {Q}}}{{P} \red {Q}}
\end{mathpar}

\begin{eqnarray*}
  match_{\equiv} (\quotep{P},\quotep{Q}) & := & P \equiv Q \\
  match_{\dagger}(\quotep{P},\quotep{Q}) & := & \forall R. P|Q \red^{*} R => R \red^{*} 0 \\
  match_{K}(\quotep{P},\quotep{Q}) & := & K \mbox{ for some context } K
\end{eqnarray*}

$u?(x)P | u!\langle Q \rangle \red P\{\quotep{Q}/x\}$

%We write $\wred$ for $\red^*$, and $P\red$ if $\exists Q $ such that $ P \red Q$.
We write $P\red$ if $\exists Q $ such that $ P \red Q$ and $P\not\red$, otherwise.

\section{Replication}

As mentioned before, it is known that replication (and hence
recursion) can be implemented in a higher-order process algebra
\cite{SangiorgiWalker}. As our first example of calculation with the
machinery thus far presented we give the construction explicitly in
the {\rhoc}.

\begin{eqnarray}
	D_{x} & := & \prefix{x}{y}{(\binpar{\outputp{x}{y}}{@{y}})} \nonumber\\
	\bangp_{x}{P} & := & \binpar{{x}!\langle{\binpar{D_{x}}{P}}\rangle}{D_{x}} \nonumber
\end{eqnarray}

\begin{eqnarray}
	\bangp_{x}{P} & & \nonumber\\
	=
	& {x}!\langle{(\prefix{x}{y}{(\outputp{x}{y} | @{y})) | P}}\rangle 
	      | \prefix{x}{y}{(\outputp{x}{y} | @{y})} & \nonumber\\
	\red
	& (\outputp{x}{y} | @{y})\substn{\quotep{(\prefix{x}{y}{(@{y} | \outputp{x}{y})) | P}}}{y} & \nonumber\\
	=
	& \outputp{x}{\quotep{(\prefix{x}{y}{(\outputp{x}{y} | @{y})) | P}}}
	  | {(\prefix{x}{y}{(\outputp{x}{y} | @{y})) | P}} & \nonumber\\
	\red
	& \ldots & \nonumber\\
	\red^*
	& P | P | \ldots & \nonumber
\end{eqnarray}

Of course, this encoding, as an implementation, runs away, unfolding
$\bangp{P}$ eagerly. A lazier and more implementable replication
operator, restricted to input-guarded processes, may be obtained as follows.

\begin{eqnarray}
\bangp{\prefix{u}{v}{P}} 
	:= 
	\binpar{\lift{x}{\prefix{u}{v}{(\binpar{D(x)}{P})}}}{D(x)} \nonumber
\end{eqnarray}

\begin{remark}
  Note that the lazier definition still does not deal with summation
  or mixed summation (i.e. sums over input and output). The reader is
  invited to construct definitions of replication that deal with these
  features. 

  Further, the definitions are parameterized in a name, $x$. Can you,
  gentle reader, make a definition that eliminates this parameter and
  guarantees no accidental interaction between the replication
  machinery and the process being replicated -- i.e. no accidental
  sharing of names used by the process to get its work done and the
  name(s) used by the replication to effect copying. This latter
  revision of the definition of replication is crucial to obtaining
  the expected identity $!!P \sim !P$.
\end{remark}

\begin{remark}\label{rem:paradoxical_combinator}
  The reader familiar with the lambda calculus will have noticed the
  similarity between $D$ and the paradoxical combinator.

  [Ed. note: the existence of this seems to suggest we have to be more
  restrictive on the set of processes and names we admit if we are to
  support no-cloning.]
\end{remark}

\subsubsection{Bisimulation}

The computational dynamics gives rise to another kind of equivalence,
the equivalence of computational behavior. As previously mentioned
this is typically captured \emph{via} some form of bisimulation.

% The notion we use in this paper is weak barbed bisimulation
% \cite{milner91polyadicpi}.

The notion we use in this paper is derived from weak barbed
bisimulation \cite{milner91polyadicpi}. 

\begin{definition}
An \emph{observation relation}, $\downarrow_{\mathcal N}$, over a set
of names, $\mathcal N$, is the smallest relation satisfying the rules
below.

\infrule[Out-barb]{y \in {\mathcal N}, \; x \nameeq y}
		  {\outputp{x}{v} \downarrow_{\mathcal N} x}
\infrule[Par-barb]{\mbox{$P\downarrow_{\mathcal N} x$ or $Q\downarrow_{\mathcal N} x$}}
		  {\binpar{P}{Q} \downarrow_{\mathcal N} x}

We write $P \Downarrow_{\mathcal N} x$ if there is $Q$ such that 
$P \wred Q$ and $Q \downarrow_{\mathcal N} x$.
\end{definition}

\begin{definition}
%\label{def.bbisim}
An  ${\mathcal N}$-\emph{barbed bisimulation} over a set of names, ${\mathcal N}$, is a symmetric binary relation 
${\mathcal S}_{\mathcal N}$ between agents such that $P\rel{S}_{\mathcal N}Q$ implies:
\begin{enumerate}
\item If $P \red P'$ then $Q \wred Q'$ and $P'\rel{S}_{\mathcal N} Q'$.
\item If $P\downarrow_{\mathcal N} x$, then $Q\Downarrow_{\mathcal N} x$.
\end{enumerate}
$P$ is ${\mathcal N}$-barbed bisimilar to $Q$, written
$P \wbbisim_{\mathcal N} Q$, if $P \rel{S}_{\mathcal N} Q$ for some ${\mathcal N}$-barbed bisimulation ${\mathcal S}_{\mathcal N}$.
\end{definition}

$\mathcal{R} \subseteq \pi \times \pi$

$P \mathcal{R} Q => \forall P'. P \red P' \Rightarrow \exists Q'. Q \red Q', P' \mathcal{R} Q'$

$P \vdash x \Rightarrow Q \vdash x$

\begin{mathpar}
  \inferrule*[lab=Out-barb]{x \nameeq y}{{y}!\langle{Q}\rangle \vdash x}
  \and
  \inferrule*[lab=Par-barb]{\mbox{$P\vdash x$ or $Q\vdash x$}}{\binpar{P}{Q} \vdash x}
\end{mathpar}

\subsubsection{Contexts}

One of the principle advantages of computational calculi like the
$\pi$-calculus is a well-defined notion of context,
contextual-equivalence and a correlation between
contextual-equivalence and notions of bisimulation. The notion of
context allows the decomposition of a process into (sub-)process and
its syntactic environment, its context. Thus, a context may be
thought of as a process with a ``hole'' (written $\Box$) in it. The
application of a context $M$ to a process $P$, written $M[P]$, is
tantamount to filling the hole in $M$ with $P$. In this paper we do
not need the full weight of this theory, but do make use of the notion
of context in the proof the main theorem. 

\begin{mathpar}
  \inferrule* [lab=summation] {} {{M_{M},M_{N}} \bc \Box \;|\; x.M_{A} \;|\; M_{M}+M_{N}}
  \and
  \inferrule* [lab=agent] {} {{M_{A}} \bc (\vec{x})M_{P} \;| \; \clift{P_0,\ldots,M_{P},\ldots,P_N}}
  \and \\
  \inferrule* [lab=process] {} {{M_{P}} \bc M_{N} \;| \;P|M_{P} }
\end{mathpar} 

\begin{mathpar}
  \inferrule* [lab=sychronization] {} {M_{N} \bc \Box \;|\; x?M_{F} \;|\; x!M_{C}}
  \and
  \inferrule* [lab=abstraction] {} {{M_{F}} \bc (x)M_{P} }
  \and
  \inferrule* [lab=concretion] {} {{M_{C}} \bc \langle M_{P} \rangle }
  \and \\
  \inferrule* [lab=process] {} {{M_{P}} \bc M_{N} \;| \;P|M_{P} }
\end{mathpar}

\begin{definition}[contextual application] Given a context $M$, and
  process $P$, we define the \emph{contextual application}, $M[P] :=
  M\{P/\Box\}$. That is, the contextual application of M to P is the
  substitution of $P$ for $\Box$ in $M$.
\end{definition}

$\meaningof{-} : L \to \mathcal{P}(\pi)$

\begin{mathpar}
  \inferrule* [lab=collection] {} {\meaningof{true} = \pi, \and \meaningof{~E} = \pi \setminus \meaningof{E}, \and \meaningof{E_{1} \& E_{2}} = \meaningof{E_{1}} \cap \meaningof{E_{2}}}
\end{mathpar}

\begin{mathpar}
  \inferrule* [lab=structure] {} {\meaningof{0} = \{ P \in \pi | P \equiv 0 \}, \and \\ \meaningof{E_1 | E_2} = \{ P \in \pi | P \equiv P_{1} | P_{2}, P_{1} \in \meaningof{E_{1}}, P_{2} \in \meaningof{E_2}\} }
\end{mathpar}

\begin{mathpar}
 \inferrule* [lab=behavior] {} {\meaningof{\langle a?b \rangle E} = \{ P \in \pi | P \equiv Q | u?(y)P', \\ \and \\\\ \and \\ \;\;\; u \in \meaningof{a}, \forall z.P'\{z/y\} \in \meaningof{E\{z/b\}}\}, \and \\ \meaningof{a!E} = \{ P \in \pi | P \equiv Q | x!\langle P' \rangle, x \in \meaningof{a} P' \in \meaningof{E}\} }
\end{mathpar}

\begin{mathpar}
 \inferrule* [lab=nominal] {} {\meaningof{\quotep{E}} = \{ \quotep{P} \in \quotep{\pi} | P \in \meaningof{E} \}, \and \meaningof{\quotep{P}} = \{ \quotep{Q} \in \quotep{\pi} | P \equiv Q \} \and \\ \meaningof{@\quotep{E}} = \{ P \in \pi | P \equiv @x, x \in \meaningof{E} \}}
\end{mathpar}

\begin{eqnarray*}
  \\
  \meaningof{-} : TS \to ST
\end{eqnarray*}

\begin{eqnarray*}
  \\
  L : TS \to ST
\end{eqnarray*}

\begin{eqnarray*}
  \\
  P \models E \iff P \in \meaningof{E}
\end{eqnarray*}

\begin{eqnarray*}
  P \approx_{L} Q \iff \forall E \in L. P \models E \iff Q \models E
\end{eqnarray*}

\begin{eqnarray*}
  P \approx_{K} Q
\end{eqnarray*}

\begin{eqnarray*}
  P \approx Q
\end{eqnarray*}

$\approx_{K} = \approx = \approx_{L}$

\subsubsection{Contextual duality}

Note that contexts extend the quotation operation to a family of
operations from processes to names. Given a context, $M$, we can
define a \emph{nominal context}, $\quotep{M}$ by $\quotep{M}[P] :=
\quotep{M[P]}$. To foreshadow what is to come we observe that these
operations enjoy a duality with processes very much like the duality
between vectors and maps from vectors to scalars.

Further, because the calculus is essentially higher-order, we have a
correspondence between contexts and processes. More specifically,
given a name $x$ and a context $M$ we can construct $M^{*}_{x}$ such
that 

\begin{mathpar}
  M^{*}_{x} | \lift{x}{P} \red M[P]
\end{mathpar}

namely,

\begin{mathpar}
  M^{*}_{x} := x?(u).M[\dropn{u}]
\end{mathpar}

The dependence of $M^{*}_{x}$ on a name makes it an abstraction, 

\begin{mathpar}
  M^{*} := (x)x?(u).M[\dropn{u}]
\end{mathpar}

\subsection{Additional notation}

It will sometimes be convenient to denote the process a name
quotes. We already have the notation $x = \quotep{P}$, but it will be
convenient to introduce an alternate notation, $\procn{x}$, when we
want to emphasize the connection to the use of the name. Note that, by
virtue of name equivalence, $\quotep{\procn{x}} \nameeq x$; so, the
notation is consistent with previous definitions.

Further, because names have structure it is possible to effect
substitutions on the basis of that structure. This means we need to
upgrade our notation for substitutions, which we accomplish by
adapting comprehension notation. Thus,

\begin{mathpar}
  P\{ y / x : x \in S \}
\end{mathpar}

is interpreted to mean the process derived from P by replacing (in a
capture-avoiding manner) each occurrence of $x$ in $S$ by $y$. For example,

\begin{mathpar}
  P\{ \quotep{\procn{x}|\procn{x}} / x : x \in \freenames{P} \}
\end{mathpar}

will replace each (occurrence) of a free name $x$ in $P$ by
$\quotep{\procn{x}|\procn{x}}$.

Also, we will avail ourselves of the notation $x^{L}$ and $x^{R}$ to
denote injections of a name into disjoint copies of the name
space. There are numerous ways to accomplish this. One example can be
found in \cite{MeredithR05}. This notation overloads to vectors of
names: $\vec{x}^{\pi} := (x_{i}^{\pi} \; : \; 0 \leq i < |\vec{x}| )$ where $\pi \in \{L,R\}$.

We also use $P^{\Box} := P|\Box$.

In \cite{MeredithR05} an interpretation of the new operator is
given. It turns out that there are several possible interpretations
all enjoying the requisite algebraic properties of the operator (see
\cite{milner91polyadicpi}). We will therefore make liberal use of
$(\nu\; \vec{x})P$.

% subsection the_syntax_and_semantics_of_the_notation_system (end)   

\input{qm2pi.qmops} 

\input{qm2pi.sterngerlach} 

\input{qm2pi.metric} 

% section concurrent_process_calculi (end)

%\input{qm2pi.proofsketch}

% section proof sketch (end)

%\input{qm2pi.slviaknots} 

% section spatial logic via knots (end)

\input{qm2pi.conclusion}

% section conclusion (end)

%\input{qm2pi.dtcodes} 

% section wiring algorithm (end)

\input{qm2pi.ack} 

% section acknowledgments (end)

\newpage


\bibliographystyle{plain}   
\bibliography{../../biblios/main.bib}

\input{qm2pi.rhodetails}

\end{document}

 

\documentclass[12pt]{llncs}
%\documentclass{jktr}

\usepackage[pdftex]{hyperref}                   
\usepackage {listings}
\usepackage {mathpartir}
\usepackage{bcprules}
%\usepackage{listings}
                       
\usepackage{graphicx} 
%\usepackage[margins=2.5cm,nohead,nofoot]{geometry}
%\usepackage{geometry}
\usepackage{amsfonts}
\usepackage{amstext}
\usepackage{latexsym}
\usepackage{amssymb}
\usepackage{color}


%\include{myPreamble}
\include{qm2pi.local} 

%\ifpdf
%\usepackage[pdftex]{graphicx}
%\else
%\usepackage{graphicx}
%\fi

 % \ifpdf
%  \usepackage{pdfsync}
%  \if


%\title{Brief Article}
%\author{David F. Snyder}
%\author{L.G. Meredith}

%\address{Dept. of Math., Texas State University--San Marcos, San Marcos, TX 78666}
       
\pagestyle{empty}


\begin{document}

\lstset{language=[Objective]Caml,frame=shadowbox}

\input{qm2pi.front}

% section front matter (end)

\input{qm2pi.intro} 
 
% section introduction (end)

% \input{qm2pi.knotations} 

% section notation (end)

\input{qm2pi.process.calculi} 

% section concurrent_process_calculi_and_spatial_logics_ (end)
    
%\input{qm2pi.knots2pi} 

%\input{qm2pi.trefoil} 

%\input{qm2pi.mainthm} 

% subsection basic_interpretation (end)

%\input{qm2pi.rho.presentation} 
\subsection{The syntax and semantics of the notation system}\label{sub:the_syntax_and_semantics_of_the_notation_system} % (fold)

We now summarize a technical presentation of the calculus that
embodies our theory of dynamics. The typical presentation of such a
calculus follows the style of giving generators and relations on
them. The grammar, below, describing term constructors, freely
generates the set of processes, $\Proc$. This set is then quotiented
by a relation known as structural congruence and it is over this set
that the notion of dynamics is expressed. This presentation is
essentially that of \cite{MeredithR05} with the addition of
polyadicity and summation. For readability we have relegated some of
the technical subtleties to an appendix.

\subsubsection{Process grammar}\label{subsub:process_grammar}

\begin{mathpar}
  \inferrule* [lab=synchronization] {} {{M} \bc \pzero \;|\; x?F \;|\; x!C }
  \and
  \inferrule* [lab=abstraction] {} {{F} \bc (x)P}
  \and
  \inferrule* [lab=concretion] {} {{C} \bc \langle Q \rangle}
  \and
  \inferrule* [lab=process] {} {{P,Q} \bc M \;| \;P|Q \;|\; @{x}}
  \and
  \inferrule* [lab=name] {} {{x} \bc \quotep{P}}
\end{mathpar} 

Note that $\vec{x}$ (resp. $\vec{P}$) denotes a vector of names
(resp. processes) of length $|\vec{x}|$ (resp. $|\vec{P}|$). We adopt
the following useful abbreviations.

\begin{mathpar}
   x?(\vec{y}).P := x.(\vec{y})P \and  x\clift{\vec{P}} := x.\clift{\vec{P}}
   \and x!(y) := \lift{x}{\dropn{y}}
   \and \Pi_{i=0}^{n-1}P_i := P_0 | \ldots | P_{n-1}
\end{mathpar}

\subsubsection{Structural congruence}

\paragraph{Free and bound names and alpha-equivalence.} At the
core of structural equivalence is alpha-equivalence which identifies
process that are the same up to a change of variable. Formally, we
recognize the distinction between free and bound names. The free names
of a process, $\freenames{P}$, may be calculated recursively as
follows:

\begin{mathpar}
\freenames{\pzero} := \emptyset
  \and \\
  \freenames{x?(y).P} := \{ x \} \cup (\freenames{P} \setminus \{ y \})
  \and 
  \freenames{x!\langle P \rangle} := \{ x \} \cup \{ P \} 
  \and \\
  \freenames{P|Q} := \freenames{P} \cup \freenames{Q}
  \and \\
  \freenames{@{x}} := \{ x \}
\end{mathpar}

$\pi$
$\quotep{\pi}$

$\freenames{-} : \pi \to \mathcal{P}(\quotep{\pi})$

\begin{eqnarray*}
  \freenames{\pzero} & := & \emptyset \\
  \freenames{x?(y).P} & := & \{ x \} \cup (\freenames{P} \setminus \{ y \}) \\
  \freenames{x!\langle P \rangle} & := & \{ x \} \cup \{ P \} \\
  \freenames{P|Q} & := & \freenames{P} \cup \freenames{Q} \\
  \freenames{\dropn{x}} & := & \{ x \}
\end{eqnarray*}

The bound names of a process, $\boundnames{P}$, are those names occurring in $P$
that are not free. For example, in $x?(y).0$, the name $x$ is free, while $y$ is bound.

\begin{mathpar}
  \inferrule* [lab=monoidal-laws] {} { P|Q \equiv Q|P \and P|0 \equiv P \and P|(Q|R) \equiv (P|Q)|R }
\end{mathpar}

\begin{mathpar}
  \inferrule* [lab=alpha-equivalence] {} { (x)P \equiv (y)P\{y/x\} \and y \not\in \freenames{P} }
\end{mathpar}

\begin{definition}
Then two processes, $P,Q$, are alpha-equivalent if $P = Q\{\vec{y}/\vec{x}\}$ for
some $\vec{x} \in \boundnames{Q},\vec{y} \in \boundnames{P}$, where $Q\{\vec{y}/\vec{x}\}$
denotes the capture-avoiding substitution of $\vec{y}$ for $\vec{x}$ in $Q$.
\end{definition}

\begin{definition}
  The {\em structural congruence} \cite{SangiorgiWalker} , $\equiv$,
  between processes is the least congruence containing
  alpha-equivalence, satisfying the abelian monoid laws
  (associativity, commutativity and $\pzero$ as identity) for parallel
  composition $|$ and for summation $+$.
\end{definition}

\subsection{Name equivalence}

We take name equivalence, written $\nameeq$, to be the smallest
equivalence relation generated by the following rules.

\begin{mathpar}
\inferrule*[lab=Quote-drop]
{ }
{ \quotep{@{x}} \nameeq x }

\inferrule*[lab=Struct-equiv]
{ P \scong Q }
{ \quotep{P} \nameeq \quotep{Q} }
\end{mathpar}

The astute reader will have noticed that the mutual recursion of names
and processes imposes a mutual recursion on alpha-equivalence and
structural equivalence via name-equivalence. Fortunately, all of this
works out pleasantly and we may calculate in the natural way, free of
concern. The reader interested in the details is referred to the
appendix \ref{appendix:rho_details}.

\subsection{Substitution}

We use $\Proc$ for the set of processes, $\QProc$ for the set of
names, and $\id{\{}\vec{y} / \vec{x} \id{\}}$ to denote partial maps,
$s : \QProc \rightarrow \QProc$. A map, $s$ lifts, uniquely, to a map
on process terms, $\widehat{s} : \Proc \rightarrow \Proc$ by the
following equations.

\begin{mathpar}
  (0) \psubstp{Q}{P} := 0 \\
  (R \juxtap S) \psubstp{Q}{P}
  :=    
  (R)\psubstp{Q}{P} \juxtap (S) \psubstp{Q}{P} \\
  (x?(y).R) \psubstp{Q}{P}    
  :=    
  (x)\substp{Q}{P} (z)\concat( (R \psubstn{z}{y}) \psubstp{Q}{P} ) \\
  (\lift{x}{R}) \psubstp{Q}{P}  
  :=
  \lift{(x)\substp{Q}{P}}{ R \psubstp{Q}{P} } \\
%   (\dropn{x})  \psubstp{Q}{P}       
%   := 
%   \left\{ 
%     \begin{array}{ccc} 
%       \dropn{\quotep{Q}} & & x \nameeq \quotep{P} \\
%       \dropn{x} & & otherwise \\
%     \end{array}
%   \right. 
  (\dropn{x})  \psubstp{Q}{P}       
  := 
  \left\{ 
    \begin{array}{ccc} 
      Q & & x \nameeq \quotep{P} \\
      \dropn{x} & & otherwise \\
    \end{array}
  \right.
\end{mathpar}
 

where

\begin{eqnarray}
  (x)\id{\{} \lpquote Q \rpquote / \lpquote P \rpquote \id{\}}            = 
  \left\{ 
    \begin{array}{ccc}
      \lpquote Q \rpquote & & x \nameeq \lpquote P \rpquote \\
      x & & otherwise \\
    \end{array}
  \right. \nonumber
\end{eqnarray}

and $z$ is chosen distinct from $\quotep{P}$, $\quotep{Q}$, the free
names in $Q$, and all the names in $R$. Our $\alpha$-equivalence will
be built in the standard way from this substitution.

\begin{remark}\label{rem:no_self_referential_names}
  One consequence of these definitions is that $\forall P. \quotep{P}
  \not\in \freenames{P}$.
\end{remark}

\subsection{ Dynamic quote: an example }

Anticipating something of what's to come, consider applying the
substitution, $\widehat{\id{\{}u / z \id{\}}}$, to the following pair
of processes, $\lift{w}{y!(z)}$ and $w[ \lpquote y!(z) \rpquote ]$.

\begin{eqnarray}
	\lift{w}{y!(z)}\widehat{\id{\{}u / z \id{\}}}
		& = &
		\lift{w}{y!(u)} \nonumber\\
	w[ \lpquote y!(z) \rpquote ] \widehat{ \id{\{}u / z \id{\}} }
		& = &
		w[ \lpquote y!(z) \rpquote ] \nonumber
\end{eqnarray}

Because the body of the process between quotes is impervious to
substitution, we get radically different answers. In fact, by
examining the first process in an input context,
e.g. $x?(z).\lift{w}{y!(z)}$, we see that the process under the lift
operator may be shaped by prefixed inputs binding a name inside it. In
this sense, the lift operator will be seen as a way to dynamically
construct processes before reifying them as names.

Finally equipped with these standard features we can present the
dynamics of the calculus.

\subsubsection{Operational semantics} 

Finally, we introduce the computational dynamics. What marks these
algebras as distinct from other more traditionally studied algebraic
structures, e.g. vector spaces or polynomial rings, is the manner in
which dynamics is captured. In traditional structures, dynamics is typically
expressed through morphisms between such structures, as in linear maps
between vector spaces or morphisms between rings. In algebras
associated with the semantics of computation, the dynamics is
expressed as part of the algebraic structure itself, through a
reduction reduction relation typically denoted by $\red$. Below, we
give a recursive presentation of this relation for the calculus used
in the encoding.

$\red \subseteq \pi \times \pi$
$\red : \pi \to \mathcal{P}(\pi)$

\begin{mathpar}
  \inferrule* [lab=Comm] { \textsf{match}( x_{src}, x_{trgt} ) } { x_{trgt}?(y)P \; | \; x_{src}!\langle {Q} \rangle \red P\{\quotep{Q}/y}\} }
  \and \\
  \inferrule* [lab=Par] {{P} \red {P}'} {{{P} | {Q}} \red {{P}' | {Q}}}
  \and
  \inferrule* [lab=Equiv]{{{P} \scong {P}'} \andalso {{P}' \red {Q}'} \andalso {{Q}' \scong {Q}}}{{P} \red {Q}}
\end{mathpar}

\begin{eqnarray*}
  match_{\equiv} (\quotep{P},\quotep{Q}) & := & P \equiv Q \\
  match_{\dagger}(\quotep{P},\quotep{Q}) & := & \forall R. P|Q \red^{*} R => R \red^{*} 0 \\
  match_{K}(\quotep{P},\quotep{Q}) & := & K \mbox{ for some context } K
\end{eqnarray*}

$u?(x)P | u!\langle Q \rangle \red P\{\quotep{Q}/x\}$

%We write $\wred$ for $\red^*$, and $P\red$ if $\exists Q $ such that $ P \red Q$.
We write $P\red$ if $\exists Q $ such that $ P \red Q$ and $P\not\red$, otherwise.

\section{Replication}

As mentioned before, it is known that replication (and hence
recursion) can be implemented in a higher-order process algebra
\cite{SangiorgiWalker}. As our first example of calculation with the
machinery thus far presented we give the construction explicitly in
the {\rhoc}.

\begin{eqnarray}
	D_{x} & := & \prefix{x}{y}{(\binpar{\outputp{x}{y}}{@{y}})} \nonumber\\
	\bangp_{x}{P} & := & \binpar{{x}!\langle{\binpar{D_{x}}{P}}\rangle}{D_{x}} \nonumber
\end{eqnarray}

\begin{eqnarray}
	\bangp_{x}{P} & & \nonumber\\
	=
	& {x}!\langle{(\prefix{x}{y}{(\outputp{x}{y} | @{y})) | P}}\rangle 
	      | \prefix{x}{y}{(\outputp{x}{y} | @{y})} & \nonumber\\
	\red
	& (\outputp{x}{y} | @{y})\substn{\quotep{(\prefix{x}{y}{(@{y} | \outputp{x}{y})) | P}}}{y} & \nonumber\\
	=
	& \outputp{x}{\quotep{(\prefix{x}{y}{(\outputp{x}{y} | @{y})) | P}}}
	  | {(\prefix{x}{y}{(\outputp{x}{y} | @{y})) | P}} & \nonumber\\
	\red
	& \ldots & \nonumber\\
	\red^*
	& P | P | \ldots & \nonumber
\end{eqnarray}

Of course, this encoding, as an implementation, runs away, unfolding
$\bangp{P}$ eagerly. A lazier and more implementable replication
operator, restricted to input-guarded processes, may be obtained as follows.

\begin{eqnarray}
\bangp{\prefix{u}{v}{P}} 
	:= 
	\binpar{\lift{x}{\prefix{u}{v}{(\binpar{D(x)}{P})}}}{D(x)} \nonumber
\end{eqnarray}

\begin{remark}
  Note that the lazier definition still does not deal with summation
  or mixed summation (i.e. sums over input and output). The reader is
  invited to construct definitions of replication that deal with these
  features. 

  Further, the definitions are parameterized in a name, $x$. Can you,
  gentle reader, make a definition that eliminates this parameter and
  guarantees no accidental interaction between the replication
  machinery and the process being replicated -- i.e. no accidental
  sharing of names used by the process to get its work done and the
  name(s) used by the replication to effect copying. This latter
  revision of the definition of replication is crucial to obtaining
  the expected identity $!!P \sim !P$.
\end{remark}

\begin{remark}\label{rem:paradoxical_combinator}
  The reader familiar with the lambda calculus will have noticed the
  similarity between $D$ and the paradoxical combinator.

  [Ed. note: the existence of this seems to suggest we have to be more
  restrictive on the set of processes and names we admit if we are to
  support no-cloning.]
\end{remark}

\subsubsection{Bisimulation}

The computational dynamics gives rise to another kind of equivalence,
the equivalence of computational behavior. As previously mentioned
this is typically captured \emph{via} some form of bisimulation.

% The notion we use in this paper is weak barbed bisimulation
% \cite{milner91polyadicpi}.

The notion we use in this paper is derived from weak barbed
bisimulation \cite{milner91polyadicpi}. 

\begin{definition}
An \emph{observation relation}, $\downarrow_{\mathcal N}$, over a set
of names, $\mathcal N$, is the smallest relation satisfying the rules
below.

\infrule[Out-barb]{y \in {\mathcal N}, \; x \nameeq y}
		  {\outputp{x}{v} \downarrow_{\mathcal N} x}
\infrule[Par-barb]{\mbox{$P\downarrow_{\mathcal N} x$ or $Q\downarrow_{\mathcal N} x$}}
		  {\binpar{P}{Q} \downarrow_{\mathcal N} x}

We write $P \Downarrow_{\mathcal N} x$ if there is $Q$ such that 
$P \wred Q$ and $Q \downarrow_{\mathcal N} x$.
\end{definition}

\begin{definition}
%\label{def.bbisim}
An  ${\mathcal N}$-\emph{barbed bisimulation} over a set of names, ${\mathcal N}$, is a symmetric binary relation 
${\mathcal S}_{\mathcal N}$ between agents such that $P\rel{S}_{\mathcal N}Q$ implies:
\begin{enumerate}
\item If $P \red P'$ then $Q \wred Q'$ and $P'\rel{S}_{\mathcal N} Q'$.
\item If $P\downarrow_{\mathcal N} x$, then $Q\Downarrow_{\mathcal N} x$.
\end{enumerate}
$P$ is ${\mathcal N}$-barbed bisimilar to $Q$, written
$P \wbbisim_{\mathcal N} Q$, if $P \rel{S}_{\mathcal N} Q$ for some ${\mathcal N}$-barbed bisimulation ${\mathcal S}_{\mathcal N}$.
\end{definition}

$\mathcal{R} \subseteq \pi \times \pi$

$P \mathcal{R} Q => \forall P'. P \red P' \Rightarrow \exists Q'. Q \red Q', P' \mathcal{R} Q'$

$P \vdash x \Rightarrow Q \vdash x$

\begin{mathpar}
  \inferrule*[lab=Out-barb]{x \nameeq y}{{y}!\langle{Q}\rangle \vdash x}
  \and
  \inferrule*[lab=Par-barb]{\mbox{$P\vdash x$ or $Q\vdash x$}}{\binpar{P}{Q} \vdash x}
\end{mathpar}

\subsubsection{Contexts}

One of the principle advantages of computational calculi like the
$\pi$-calculus is a well-defined notion of context,
contextual-equivalence and a correlation between
contextual-equivalence and notions of bisimulation. The notion of
context allows the decomposition of a process into (sub-)process and
its syntactic environment, its context. Thus, a context may be
thought of as a process with a ``hole'' (written $\Box$) in it. The
application of a context $M$ to a process $P$, written $M[P]$, is
tantamount to filling the hole in $M$ with $P$. In this paper we do
not need the full weight of this theory, but do make use of the notion
of context in the proof the main theorem. 

\begin{mathpar}
  \inferrule* [lab=summation] {} {{M_{M},M_{N}} \bc \Box \;|\; x.M_{A} \;|\; M_{M}+M_{N}}
  \and
  \inferrule* [lab=agent] {} {{M_{A}} \bc (\vec{x})M_{P} \;| \; \clift{P_0,\ldots,M_{P},\ldots,P_N}}
  \and \\
  \inferrule* [lab=process] {} {{M_{P}} \bc M_{N} \;| \;P|M_{P} }
\end{mathpar} 

\begin{mathpar}
  \inferrule* [lab=sychronization] {} {M_{N} \bc \Box \;|\; x?M_{F} \;|\; x!M_{C}}
  \and
  \inferrule* [lab=abstraction] {} {{M_{F}} \bc (x)M_{P} }
  \and
  \inferrule* [lab=concretion] {} {{M_{C}} \bc \langle M_{P} \rangle }
  \and \\
  \inferrule* [lab=process] {} {{M_{P}} \bc M_{N} \;| \;P|M_{P} }
\end{mathpar}

\begin{definition}[contextual application] Given a context $M$, and
  process $P$, we define the \emph{contextual application}, $M[P] :=
  M\{P/\Box\}$. That is, the contextual application of M to P is the
  substitution of $P$ for $\Box$ in $M$.
\end{definition}

$\meaningof{-} : L \to \mathcal{P}(\pi)$

\begin{mathpar}
  \inferrule* [lab=collection] {} {\meaningof{true} = \pi, \and \meaningof{~E} = \pi \setminus \meaningof{E}, \and \meaningof{E_{1} \& E_{2}} = \meaningof{E_{1}} \cap \meaningof{E_{2}}}
\end{mathpar}

\begin{mathpar}
  \inferrule* [lab=structure] {} {\meaningof{0} = \{ P \in \pi | P \equiv 0 \}, \and \\ \meaningof{E_1 | E_2} = \{ P \in \pi | P \equiv P_{1} | P_{2}, P_{1} \in \meaningof{E_{1}}, P_{2} \in \meaningof{E_2}\} }
\end{mathpar}

\begin{mathpar}
 \inferrule* [lab=behavior] {} {\meaningof{\langle a?b \rangle E} = \{ P \in \pi | P \equiv Q | u?(y)P', \\ \and \\\\ \and \\ \;\;\; u \in \meaningof{a}, \forall z.P'\{z/y\} \in \meaningof{E\{z/b\}}\}, \and \\ \meaningof{a!E} = \{ P \in \pi | P \equiv Q | x!\langle P' \rangle, x \in \meaningof{a} P' \in \meaningof{E}\} }
\end{mathpar}

\begin{mathpar}
 \inferrule* [lab=nominal] {} {\meaningof{\quotep{E}} = \{ \quotep{P} \in \quotep{\pi} | P \in \meaningof{E} \}, \and \meaningof{\quotep{P}} = \{ \quotep{Q} \in \quotep{\pi} | P \equiv Q \} \and \\ \meaningof{@\quotep{E}} = \{ P \in \pi | P \equiv @x, x \in \meaningof{E} \}}
\end{mathpar}

\begin{eqnarray*}
  \\
  \meaningof{-} : TS \to ST
\end{eqnarray*}

\begin{eqnarray*}
  \\
  L : TS \to ST
\end{eqnarray*}

\begin{eqnarray*}
  \\
  P \models E \iff P \in \meaningof{E}
\end{eqnarray*}

\begin{eqnarray*}
  P \approx_{L} Q \iff \forall E \in L. P \models E \iff Q \models E
\end{eqnarray*}

\begin{eqnarray*}
  P \approx_{K} Q
\end{eqnarray*}

\begin{eqnarray*}
  P \approx Q
\end{eqnarray*}

$\approx_{K} = \approx = \approx_{L}$

\subsubsection{Contextual duality}

Note that contexts extend the quotation operation to a family of
operations from processes to names. Given a context, $M$, we can
define a \emph{nominal context}, $\quotep{M}$ by $\quotep{M}[P] :=
\quotep{M[P]}$. To foreshadow what is to come we observe that these
operations enjoy a duality with processes very much like the duality
between vectors and maps from vectors to scalars.

Further, because the calculus is essentially higher-order, we have a
correspondence between contexts and processes. More specifically,
given a name $x$ and a context $M$ we can construct $M^{*}_{x}$ such
that 

\begin{mathpar}
  M^{*}_{x} | \lift{x}{P} \red M[P]
\end{mathpar}

namely,

\begin{mathpar}
  M^{*}_{x} := x?(u).M[\dropn{u}]
\end{mathpar}

The dependence of $M^{*}_{x}$ on a name makes it an abstraction, 

\begin{mathpar}
  M^{*} := (x)x?(u).M[\dropn{u}]
\end{mathpar}

\subsection{Additional notation}

It will sometimes be convenient to denote the process a name
quotes. We already have the notation $x = \quotep{P}$, but it will be
convenient to introduce an alternate notation, $\procn{x}$, when we
want to emphasize the connection to the use of the name. Note that, by
virtue of name equivalence, $\quotep{\procn{x}} \nameeq x$; so, the
notation is consistent with previous definitions.

Further, because names have structure it is possible to effect
substitutions on the basis of that structure. This means we need to
upgrade our notation for substitutions, which we accomplish by
adapting comprehension notation. Thus,

\begin{mathpar}
  P\{ y / x : x \in S \}
\end{mathpar}

is interpreted to mean the process derived from P by replacing (in a
capture-avoiding manner) each occurrence of $x$ in $S$ by $y$. For example,

\begin{mathpar}
  P\{ \quotep{\procn{x}|\procn{x}} / x : x \in \freenames{P} \}
\end{mathpar}

will replace each (occurrence) of a free name $x$ in $P$ by
$\quotep{\procn{x}|\procn{x}}$.

Also, we will avail ourselves of the notation $x^{L}$ and $x^{R}$ to
denote injections of a name into disjoint copies of the name
space. There are numerous ways to accomplish this. One example can be
found in \cite{MeredithR05}. This notation overloads to vectors of
names: $\vec{x}^{\pi} := (x_{i}^{\pi} \; : \; 0 \leq i < |\vec{x}| )$ where $\pi \in \{L,R\}$.

We also use $P^{\Box} := P|\Box$.

In \cite{MeredithR05} an interpretation of the new operator is
given. It turns out that there are several possible interpretations
all enjoying the requisite algebraic properties of the operator (see
\cite{milner91polyadicpi}). We will therefore make liberal use of
$(\nu\; \vec{x})P$.

% subsection the_syntax_and_semantics_of_the_notation_system (end)   

\input{qm2pi.qmops} 

\input{qm2pi.sterngerlach} 

\input{qm2pi.metric} 

% section concurrent_process_calculi (end)

%\input{qm2pi.proofsketch}

% section proof sketch (end)

%\input{qm2pi.slviaknots} 

% section spatial logic via knots (end)

\input{qm2pi.conclusion}

% section conclusion (end)

%\input{qm2pi.dtcodes} 

% section wiring algorithm (end)

\input{qm2pi.ack} 

% section acknowledgments (end)

\newpage


\bibliographystyle{plain}   
\bibliography{../../biblios/main.bib}

\input{qm2pi.rhodetails}

\end{document}

 

% section concurrent_process_calculi (end)

%\documentclass[12pt]{llncs}
%\documentclass{jktr}

\usepackage[pdftex]{hyperref}                   
\usepackage {listings}
\usepackage {mathpartir}
\usepackage{bcprules}
%\usepackage{listings}
                       
\usepackage{graphicx} 
%\usepackage[margins=2.5cm,nohead,nofoot]{geometry}
%\usepackage{geometry}
\usepackage{amsfonts}
\usepackage{amstext}
\usepackage{latexsym}
\usepackage{amssymb}
\usepackage{color}


%\include{myPreamble}
\include{qm2pi.local} 

%\ifpdf
%\usepackage[pdftex]{graphicx}
%\else
%\usepackage{graphicx}
%\fi

 % \ifpdf
%  \usepackage{pdfsync}
%  \if


%\title{Brief Article}
%\author{David F. Snyder}
%\author{L.G. Meredith}

%\address{Dept. of Math., Texas State University--San Marcos, San Marcos, TX 78666}
       
\pagestyle{empty}


\begin{document}

\lstset{language=[Objective]Caml,frame=shadowbox}

\input{qm2pi.front}

% section front matter (end)

\input{qm2pi.intro} 
 
% section introduction (end)

% \input{qm2pi.knotations} 

% section notation (end)

\input{qm2pi.process.calculi} 

% section concurrent_process_calculi_and_spatial_logics_ (end)
    
%\input{qm2pi.knots2pi} 

%\input{qm2pi.trefoil} 

%\input{qm2pi.mainthm} 

% subsection basic_interpretation (end)

%\input{qm2pi.rho.presentation} 
\subsection{The syntax and semantics of the notation system}\label{sub:the_syntax_and_semantics_of_the_notation_system} % (fold)

We now summarize a technical presentation of the calculus that
embodies our theory of dynamics. The typical presentation of such a
calculus follows the style of giving generators and relations on
them. The grammar, below, describing term constructors, freely
generates the set of processes, $\Proc$. This set is then quotiented
by a relation known as structural congruence and it is over this set
that the notion of dynamics is expressed. This presentation is
essentially that of \cite{MeredithR05} with the addition of
polyadicity and summation. For readability we have relegated some of
the technical subtleties to an appendix.

\subsubsection{Process grammar}\label{subsub:process_grammar}

\begin{mathpar}
  \inferrule* [lab=synchronization] {} {{M} \bc \pzero \;|\; x?F \;|\; x!C }
  \and
  \inferrule* [lab=abstraction] {} {{F} \bc (x)P}
  \and
  \inferrule* [lab=concretion] {} {{C} \bc \langle Q \rangle}
  \and
  \inferrule* [lab=process] {} {{P,Q} \bc M \;| \;P|Q \;|\; @{x}}
  \and
  \inferrule* [lab=name] {} {{x} \bc \quotep{P}}
\end{mathpar} 

Note that $\vec{x}$ (resp. $\vec{P}$) denotes a vector of names
(resp. processes) of length $|\vec{x}|$ (resp. $|\vec{P}|$). We adopt
the following useful abbreviations.

\begin{mathpar}
   x?(\vec{y}).P := x.(\vec{y})P \and  x\clift{\vec{P}} := x.\clift{\vec{P}}
   \and x!(y) := \lift{x}{\dropn{y}}
   \and \Pi_{i=0}^{n-1}P_i := P_0 | \ldots | P_{n-1}
\end{mathpar}

\subsubsection{Structural congruence}

\paragraph{Free and bound names and alpha-equivalence.} At the
core of structural equivalence is alpha-equivalence which identifies
process that are the same up to a change of variable. Formally, we
recognize the distinction between free and bound names. The free names
of a process, $\freenames{P}$, may be calculated recursively as
follows:

\begin{mathpar}
\freenames{\pzero} := \emptyset
  \and \\
  \freenames{x?(y).P} := \{ x \} \cup (\freenames{P} \setminus \{ y \})
  \and 
  \freenames{x!\langle P \rangle} := \{ x \} \cup \{ P \} 
  \and \\
  \freenames{P|Q} := \freenames{P} \cup \freenames{Q}
  \and \\
  \freenames{@{x}} := \{ x \}
\end{mathpar}

$\pi$
$\quotep{\pi}$

$\freenames{-} : \pi \to \mathcal{P}(\quotep{\pi})$

\begin{eqnarray*}
  \freenames{\pzero} & := & \emptyset \\
  \freenames{x?(y).P} & := & \{ x \} \cup (\freenames{P} \setminus \{ y \}) \\
  \freenames{x!\langle P \rangle} & := & \{ x \} \cup \{ P \} \\
  \freenames{P|Q} & := & \freenames{P} \cup \freenames{Q} \\
  \freenames{\dropn{x}} & := & \{ x \}
\end{eqnarray*}

The bound names of a process, $\boundnames{P}$, are those names occurring in $P$
that are not free. For example, in $x?(y).0$, the name $x$ is free, while $y$ is bound.

\begin{mathpar}
  \inferrule* [lab=monoidal-laws] {} { P|Q \equiv Q|P \and P|0 \equiv P \and P|(Q|R) \equiv (P|Q)|R }
\end{mathpar}

\begin{mathpar}
  \inferrule* [lab=alpha-equivalence] {} { (x)P \equiv (y)P\{y/x\} \and y \not\in \freenames{P} }
\end{mathpar}

\begin{definition}
Then two processes, $P,Q$, are alpha-equivalent if $P = Q\{\vec{y}/\vec{x}\}$ for
some $\vec{x} \in \boundnames{Q},\vec{y} \in \boundnames{P}$, where $Q\{\vec{y}/\vec{x}\}$
denotes the capture-avoiding substitution of $\vec{y}$ for $\vec{x}$ in $Q$.
\end{definition}

\begin{definition}
  The {\em structural congruence} \cite{SangiorgiWalker} , $\equiv$,
  between processes is the least congruence containing
  alpha-equivalence, satisfying the abelian monoid laws
  (associativity, commutativity and $\pzero$ as identity) for parallel
  composition $|$ and for summation $+$.
\end{definition}

\subsection{Name equivalence}

We take name equivalence, written $\nameeq$, to be the smallest
equivalence relation generated by the following rules.

\begin{mathpar}
\inferrule*[lab=Quote-drop]
{ }
{ \quotep{@{x}} \nameeq x }

\inferrule*[lab=Struct-equiv]
{ P \scong Q }
{ \quotep{P} \nameeq \quotep{Q} }
\end{mathpar}

The astute reader will have noticed that the mutual recursion of names
and processes imposes a mutual recursion on alpha-equivalence and
structural equivalence via name-equivalence. Fortunately, all of this
works out pleasantly and we may calculate in the natural way, free of
concern. The reader interested in the details is referred to the
appendix \ref{appendix:rho_details}.

\subsection{Substitution}

We use $\Proc$ for the set of processes, $\QProc$ for the set of
names, and $\id{\{}\vec{y} / \vec{x} \id{\}}$ to denote partial maps,
$s : \QProc \rightarrow \QProc$. A map, $s$ lifts, uniquely, to a map
on process terms, $\widehat{s} : \Proc \rightarrow \Proc$ by the
following equations.

\begin{mathpar}
  (0) \psubstp{Q}{P} := 0 \\
  (R \juxtap S) \psubstp{Q}{P}
  :=    
  (R)\psubstp{Q}{P} \juxtap (S) \psubstp{Q}{P} \\
  (x?(y).R) \psubstp{Q}{P}    
  :=    
  (x)\substp{Q}{P} (z)\concat( (R \psubstn{z}{y}) \psubstp{Q}{P} ) \\
  (\lift{x}{R}) \psubstp{Q}{P}  
  :=
  \lift{(x)\substp{Q}{P}}{ R \psubstp{Q}{P} } \\
%   (\dropn{x})  \psubstp{Q}{P}       
%   := 
%   \left\{ 
%     \begin{array}{ccc} 
%       \dropn{\quotep{Q}} & & x \nameeq \quotep{P} \\
%       \dropn{x} & & otherwise \\
%     \end{array}
%   \right. 
  (\dropn{x})  \psubstp{Q}{P}       
  := 
  \left\{ 
    \begin{array}{ccc} 
      Q & & x \nameeq \quotep{P} \\
      \dropn{x} & & otherwise \\
    \end{array}
  \right.
\end{mathpar}
 

where

\begin{eqnarray}
  (x)\id{\{} \lpquote Q \rpquote / \lpquote P \rpquote \id{\}}            = 
  \left\{ 
    \begin{array}{ccc}
      \lpquote Q \rpquote & & x \nameeq \lpquote P \rpquote \\
      x & & otherwise \\
    \end{array}
  \right. \nonumber
\end{eqnarray}

and $z$ is chosen distinct from $\quotep{P}$, $\quotep{Q}$, the free
names in $Q$, and all the names in $R$. Our $\alpha$-equivalence will
be built in the standard way from this substitution.

\begin{remark}\label{rem:no_self_referential_names}
  One consequence of these definitions is that $\forall P. \quotep{P}
  \not\in \freenames{P}$.
\end{remark}

\subsection{ Dynamic quote: an example }

Anticipating something of what's to come, consider applying the
substitution, $\widehat{\id{\{}u / z \id{\}}}$, to the following pair
of processes, $\lift{w}{y!(z)}$ and $w[ \lpquote y!(z) \rpquote ]$.

\begin{eqnarray}
	\lift{w}{y!(z)}\widehat{\id{\{}u / z \id{\}}}
		& = &
		\lift{w}{y!(u)} \nonumber\\
	w[ \lpquote y!(z) \rpquote ] \widehat{ \id{\{}u / z \id{\}} }
		& = &
		w[ \lpquote y!(z) \rpquote ] \nonumber
\end{eqnarray}

Because the body of the process between quotes is impervious to
substitution, we get radically different answers. In fact, by
examining the first process in an input context,
e.g. $x?(z).\lift{w}{y!(z)}$, we see that the process under the lift
operator may be shaped by prefixed inputs binding a name inside it. In
this sense, the lift operator will be seen as a way to dynamically
construct processes before reifying them as names.

Finally equipped with these standard features we can present the
dynamics of the calculus.

\subsubsection{Operational semantics} 

Finally, we introduce the computational dynamics. What marks these
algebras as distinct from other more traditionally studied algebraic
structures, e.g. vector spaces or polynomial rings, is the manner in
which dynamics is captured. In traditional structures, dynamics is typically
expressed through morphisms between such structures, as in linear maps
between vector spaces or morphisms between rings. In algebras
associated with the semantics of computation, the dynamics is
expressed as part of the algebraic structure itself, through a
reduction reduction relation typically denoted by $\red$. Below, we
give a recursive presentation of this relation for the calculus used
in the encoding.

$\red \subseteq \pi \times \pi$
$\red : \pi \to \mathcal{P}(\pi)$

\begin{mathpar}
  \inferrule* [lab=Comm] { \textsf{match}( x_{src}, x_{trgt} ) } { x_{trgt}?(y)P \; | \; x_{src}!\langle {Q} \rangle \red P\{\quotep{Q}/y}\} }
  \and \\
  \inferrule* [lab=Par] {{P} \red {P}'} {{{P} | {Q}} \red {{P}' | {Q}}}
  \and
  \inferrule* [lab=Equiv]{{{P} \scong {P}'} \andalso {{P}' \red {Q}'} \andalso {{Q}' \scong {Q}}}{{P} \red {Q}}
\end{mathpar}

\begin{eqnarray*}
  match_{\equiv} (\quotep{P},\quotep{Q}) & := & P \equiv Q \\
  match_{\dagger}(\quotep{P},\quotep{Q}) & := & \forall R. P|Q \red^{*} R => R \red^{*} 0 \\
  match_{K}(\quotep{P},\quotep{Q}) & := & K \mbox{ for some context } K
\end{eqnarray*}

$u?(x)P | u!\langle Q \rangle \red P\{\quotep{Q}/x\}$

%We write $\wred$ for $\red^*$, and $P\red$ if $\exists Q $ such that $ P \red Q$.
We write $P\red$ if $\exists Q $ such that $ P \red Q$ and $P\not\red$, otherwise.

\section{Replication}

As mentioned before, it is known that replication (and hence
recursion) can be implemented in a higher-order process algebra
\cite{SangiorgiWalker}. As our first example of calculation with the
machinery thus far presented we give the construction explicitly in
the {\rhoc}.

\begin{eqnarray}
	D_{x} & := & \prefix{x}{y}{(\binpar{\outputp{x}{y}}{@{y}})} \nonumber\\
	\bangp_{x}{P} & := & \binpar{{x}!\langle{\binpar{D_{x}}{P}}\rangle}{D_{x}} \nonumber
\end{eqnarray}

\begin{eqnarray}
	\bangp_{x}{P} & & \nonumber\\
	=
	& {x}!\langle{(\prefix{x}{y}{(\outputp{x}{y} | @{y})) | P}}\rangle 
	      | \prefix{x}{y}{(\outputp{x}{y} | @{y})} & \nonumber\\
	\red
	& (\outputp{x}{y} | @{y})\substn{\quotep{(\prefix{x}{y}{(@{y} | \outputp{x}{y})) | P}}}{y} & \nonumber\\
	=
	& \outputp{x}{\quotep{(\prefix{x}{y}{(\outputp{x}{y} | @{y})) | P}}}
	  | {(\prefix{x}{y}{(\outputp{x}{y} | @{y})) | P}} & \nonumber\\
	\red
	& \ldots & \nonumber\\
	\red^*
	& P | P | \ldots & \nonumber
\end{eqnarray}

Of course, this encoding, as an implementation, runs away, unfolding
$\bangp{P}$ eagerly. A lazier and more implementable replication
operator, restricted to input-guarded processes, may be obtained as follows.

\begin{eqnarray}
\bangp{\prefix{u}{v}{P}} 
	:= 
	\binpar{\lift{x}{\prefix{u}{v}{(\binpar{D(x)}{P})}}}{D(x)} \nonumber
\end{eqnarray}

\begin{remark}
  Note that the lazier definition still does not deal with summation
  or mixed summation (i.e. sums over input and output). The reader is
  invited to construct definitions of replication that deal with these
  features. 

  Further, the definitions are parameterized in a name, $x$. Can you,
  gentle reader, make a definition that eliminates this parameter and
  guarantees no accidental interaction between the replication
  machinery and the process being replicated -- i.e. no accidental
  sharing of names used by the process to get its work done and the
  name(s) used by the replication to effect copying. This latter
  revision of the definition of replication is crucial to obtaining
  the expected identity $!!P \sim !P$.
\end{remark}

\begin{remark}\label{rem:paradoxical_combinator}
  The reader familiar with the lambda calculus will have noticed the
  similarity between $D$ and the paradoxical combinator.

  [Ed. note: the existence of this seems to suggest we have to be more
  restrictive on the set of processes and names we admit if we are to
  support no-cloning.]
\end{remark}

\subsubsection{Bisimulation}

The computational dynamics gives rise to another kind of equivalence,
the equivalence of computational behavior. As previously mentioned
this is typically captured \emph{via} some form of bisimulation.

% The notion we use in this paper is weak barbed bisimulation
% \cite{milner91polyadicpi}.

The notion we use in this paper is derived from weak barbed
bisimulation \cite{milner91polyadicpi}. 

\begin{definition}
An \emph{observation relation}, $\downarrow_{\mathcal N}$, over a set
of names, $\mathcal N$, is the smallest relation satisfying the rules
below.

\infrule[Out-barb]{y \in {\mathcal N}, \; x \nameeq y}
		  {\outputp{x}{v} \downarrow_{\mathcal N} x}
\infrule[Par-barb]{\mbox{$P\downarrow_{\mathcal N} x$ or $Q\downarrow_{\mathcal N} x$}}
		  {\binpar{P}{Q} \downarrow_{\mathcal N} x}

We write $P \Downarrow_{\mathcal N} x$ if there is $Q$ such that 
$P \wred Q$ and $Q \downarrow_{\mathcal N} x$.
\end{definition}

\begin{definition}
%\label{def.bbisim}
An  ${\mathcal N}$-\emph{barbed bisimulation} over a set of names, ${\mathcal N}$, is a symmetric binary relation 
${\mathcal S}_{\mathcal N}$ between agents such that $P\rel{S}_{\mathcal N}Q$ implies:
\begin{enumerate}
\item If $P \red P'$ then $Q \wred Q'$ and $P'\rel{S}_{\mathcal N} Q'$.
\item If $P\downarrow_{\mathcal N} x$, then $Q\Downarrow_{\mathcal N} x$.
\end{enumerate}
$P$ is ${\mathcal N}$-barbed bisimilar to $Q$, written
$P \wbbisim_{\mathcal N} Q$, if $P \rel{S}_{\mathcal N} Q$ for some ${\mathcal N}$-barbed bisimulation ${\mathcal S}_{\mathcal N}$.
\end{definition}

$\mathcal{R} \subseteq \pi \times \pi$

$P \mathcal{R} Q => \forall P'. P \red P' \Rightarrow \exists Q'. Q \red Q', P' \mathcal{R} Q'$

$P \vdash x \Rightarrow Q \vdash x$

\begin{mathpar}
  \inferrule*[lab=Out-barb]{x \nameeq y}{{y}!\langle{Q}\rangle \vdash x}
  \and
  \inferrule*[lab=Par-barb]{\mbox{$P\vdash x$ or $Q\vdash x$}}{\binpar{P}{Q} \vdash x}
\end{mathpar}

\subsubsection{Contexts}

One of the principle advantages of computational calculi like the
$\pi$-calculus is a well-defined notion of context,
contextual-equivalence and a correlation between
contextual-equivalence and notions of bisimulation. The notion of
context allows the decomposition of a process into (sub-)process and
its syntactic environment, its context. Thus, a context may be
thought of as a process with a ``hole'' (written $\Box$) in it. The
application of a context $M$ to a process $P$, written $M[P]$, is
tantamount to filling the hole in $M$ with $P$. In this paper we do
not need the full weight of this theory, but do make use of the notion
of context in the proof the main theorem. 

\begin{mathpar}
  \inferrule* [lab=summation] {} {{M_{M},M_{N}} \bc \Box \;|\; x.M_{A} \;|\; M_{M}+M_{N}}
  \and
  \inferrule* [lab=agent] {} {{M_{A}} \bc (\vec{x})M_{P} \;| \; \clift{P_0,\ldots,M_{P},\ldots,P_N}}
  \and \\
  \inferrule* [lab=process] {} {{M_{P}} \bc M_{N} \;| \;P|M_{P} }
\end{mathpar} 

\begin{mathpar}
  \inferrule* [lab=sychronization] {} {M_{N} \bc \Box \;|\; x?M_{F} \;|\; x!M_{C}}
  \and
  \inferrule* [lab=abstraction] {} {{M_{F}} \bc (x)M_{P} }
  \and
  \inferrule* [lab=concretion] {} {{M_{C}} \bc \langle M_{P} \rangle }
  \and \\
  \inferrule* [lab=process] {} {{M_{P}} \bc M_{N} \;| \;P|M_{P} }
\end{mathpar}

\begin{definition}[contextual application] Given a context $M$, and
  process $P$, we define the \emph{contextual application}, $M[P] :=
  M\{P/\Box\}$. That is, the contextual application of M to P is the
  substitution of $P$ for $\Box$ in $M$.
\end{definition}

$\meaningof{-} : L \to \mathcal{P}(\pi)$

\begin{mathpar}
  \inferrule* [lab=collection] {} {\meaningof{true} = \pi, \and \meaningof{~E} = \pi \setminus \meaningof{E}, \and \meaningof{E_{1} \& E_{2}} = \meaningof{E_{1}} \cap \meaningof{E_{2}}}
\end{mathpar}

\begin{mathpar}
  \inferrule* [lab=structure] {} {\meaningof{0} = \{ P \in \pi | P \equiv 0 \}, \and \\ \meaningof{E_1 | E_2} = \{ P \in \pi | P \equiv P_{1} | P_{2}, P_{1} \in \meaningof{E_{1}}, P_{2} \in \meaningof{E_2}\} }
\end{mathpar}

\begin{mathpar}
 \inferrule* [lab=behavior] {} {\meaningof{\langle a?b \rangle E} = \{ P \in \pi | P \equiv Q | u?(y)P', \\ \and \\\\ \and \\ \;\;\; u \in \meaningof{a}, \forall z.P'\{z/y\} \in \meaningof{E\{z/b\}}\}, \and \\ \meaningof{a!E} = \{ P \in \pi | P \equiv Q | x!\langle P' \rangle, x \in \meaningof{a} P' \in \meaningof{E}\} }
\end{mathpar}

\begin{mathpar}
 \inferrule* [lab=nominal] {} {\meaningof{\quotep{E}} = \{ \quotep{P} \in \quotep{\pi} | P \in \meaningof{E} \}, \and \meaningof{\quotep{P}} = \{ \quotep{Q} \in \quotep{\pi} | P \equiv Q \} \and \\ \meaningof{@\quotep{E}} = \{ P \in \pi | P \equiv @x, x \in \meaningof{E} \}}
\end{mathpar}

\begin{eqnarray*}
  \\
  \meaningof{-} : TS \to ST
\end{eqnarray*}

\begin{eqnarray*}
  \\
  L : TS \to ST
\end{eqnarray*}

\begin{eqnarray*}
  \\
  P \models E \iff P \in \meaningof{E}
\end{eqnarray*}

\begin{eqnarray*}
  P \approx_{L} Q \iff \forall E \in L. P \models E \iff Q \models E
\end{eqnarray*}

\begin{eqnarray*}
  P \approx_{K} Q
\end{eqnarray*}

\begin{eqnarray*}
  P \approx Q
\end{eqnarray*}

$\approx_{K} = \approx = \approx_{L}$

\subsubsection{Contextual duality}

Note that contexts extend the quotation operation to a family of
operations from processes to names. Given a context, $M$, we can
define a \emph{nominal context}, $\quotep{M}$ by $\quotep{M}[P] :=
\quotep{M[P]}$. To foreshadow what is to come we observe that these
operations enjoy a duality with processes very much like the duality
between vectors and maps from vectors to scalars.

Further, because the calculus is essentially higher-order, we have a
correspondence between contexts and processes. More specifically,
given a name $x$ and a context $M$ we can construct $M^{*}_{x}$ such
that 

\begin{mathpar}
  M^{*}_{x} | \lift{x}{P} \red M[P]
\end{mathpar}

namely,

\begin{mathpar}
  M^{*}_{x} := x?(u).M[\dropn{u}]
\end{mathpar}

The dependence of $M^{*}_{x}$ on a name makes it an abstraction, 

\begin{mathpar}
  M^{*} := (x)x?(u).M[\dropn{u}]
\end{mathpar}

\subsection{Additional notation}

It will sometimes be convenient to denote the process a name
quotes. We already have the notation $x = \quotep{P}$, but it will be
convenient to introduce an alternate notation, $\procn{x}$, when we
want to emphasize the connection to the use of the name. Note that, by
virtue of name equivalence, $\quotep{\procn{x}} \nameeq x$; so, the
notation is consistent with previous definitions.

Further, because names have structure it is possible to effect
substitutions on the basis of that structure. This means we need to
upgrade our notation for substitutions, which we accomplish by
adapting comprehension notation. Thus,

\begin{mathpar}
  P\{ y / x : x \in S \}
\end{mathpar}

is interpreted to mean the process derived from P by replacing (in a
capture-avoiding manner) each occurrence of $x$ in $S$ by $y$. For example,

\begin{mathpar}
  P\{ \quotep{\procn{x}|\procn{x}} / x : x \in \freenames{P} \}
\end{mathpar}

will replace each (occurrence) of a free name $x$ in $P$ by
$\quotep{\procn{x}|\procn{x}}$.

Also, we will avail ourselves of the notation $x^{L}$ and $x^{R}$ to
denote injections of a name into disjoint copies of the name
space. There are numerous ways to accomplish this. One example can be
found in \cite{MeredithR05}. This notation overloads to vectors of
names: $\vec{x}^{\pi} := (x_{i}^{\pi} \; : \; 0 \leq i < |\vec{x}| )$ where $\pi \in \{L,R\}$.

We also use $P^{\Box} := P|\Box$.

In \cite{MeredithR05} an interpretation of the new operator is
given. It turns out that there are several possible interpretations
all enjoying the requisite algebraic properties of the operator (see
\cite{milner91polyadicpi}). We will therefore make liberal use of
$(\nu\; \vec{x})P$.

% subsection the_syntax_and_semantics_of_the_notation_system (end)   

\input{qm2pi.qmops} 

\input{qm2pi.sterngerlach} 

\input{qm2pi.metric} 

% section concurrent_process_calculi (end)

%\input{qm2pi.proofsketch}

% section proof sketch (end)

%\input{qm2pi.slviaknots} 

% section spatial logic via knots (end)

\input{qm2pi.conclusion}

% section conclusion (end)

%\input{qm2pi.dtcodes} 

% section wiring algorithm (end)

\input{qm2pi.ack} 

% section acknowledgments (end)

\newpage


\bibliographystyle{plain}   
\bibliography{../../biblios/main.bib}

\input{qm2pi.rhodetails}

\end{document}



% section proof sketch (end)

%\section{Unlikely characters: spatial logic for
  knots}\label{sub:characteristic_formulae} % (fold)

Associated to the mobile process calculi are a family of logics known
as the Hennessy-Milner logics. These logics typically enjoy a
semantics interpreting formulae as sets of processes that when
factored through the encoding outlined above allows an identification
of classes of knots with logical formulae. In the context of this
encoding the sub-family known as the spatial logics \cite{CairesC03}
\cite{CairesC04} \cite{Caires04} are of particular interest providing
several important features for expressing and reasoning about
properties (i.e. classes) of knots. We hint here at how this may be done.

%\begin{description}
%\item [structural connectives] 
\subsubsection{Structural connectives} The spatial logics enjoy
structural connectives corresponding, at the logical level, to the
parallel composition ($P | Q$) and new name ($(\nu \; x)P$)
connectives for processes. As illustrated in the examples below, these
connectives are extremely expressive given the shape of our encoding.
%\item [decideable satisfaction]

\subsubsection{Decideable satisfaction}
In \cite{Caires04} the satisfaction relation is shown to be decideable
for a rich class of processes. It further turns out that the image of
the our encoding is a proper subset of that class. This result
provides the basis for an algorithm by which to search for knots
enjoying a given property.
%\item [characteristic formulae]

\subsubsection{Characteristic formulae}
In the same paper \cite{Caires04} , Caires presents a means of calculating
characteristic formulae, selecting equivalence classes of processes
up to a pre--specified depth limit on the support set of names. Composed with our
encoding, this characteristic formula can be used to select
characteristic formulae for knots.
%\end{description}

\subsubsection{Spatial logic formulae}

The grammar below (segmented for comprehension) summarizes the syntax
of spatial logic formulae. We employ illustrative examples in the
sequel to provide an intuitive understanding of their meaning
referring the reader to \cite{Caires04} for a more detailed explication
of the semantics.

\begin{mathpar}
  \inferrule* [lab=boolean] {} {{A,B} \bc T \;|\; \neg A \;|\; A \wedge B \;|\; \eta = \eta'}
  \and
  \inferrule* [lab=spatial] {} {|\; \pzero \;|\; A | B \;|\; x \text{\textregistered} A \;|\; \forall x . A \;|\;  H x . A}
  \and
  \inferrule* [lab=behavioral] {} {|\; \alpha . A}
  \and 
  \inferrule* [lab=recursion] {} {|\; X(\vec{u}) \;|\; \mu X(\vec{u}) . A}
  \and
  \inferrule* [lab=action] {} {\alpha \bc \langle x?(\vec{y}) \rangle \;|\; \langle x!(\vec{y}) \rangle \;|\; \langle \tau \rangle}
  \and 
  \inferrule* [lab=name] {} {\eta \bc x \;|\; \tau}
\end{mathpar} 

% subsection characteristic_formulae (end)   	 

\subsection{Example formulae}\label{sub:example_formulae_} % (fold)

\subsubsection{Crossing as formula.}
% 
% \begin{align*}
%   \frac{d}{dx} \sin x &= \cos x 
%   & \frac{d}{dx} e^x &= e^x \\
%   \frac{d}{dx} \cos x &= - \sin x 
%   & \frac{d}{dx} \log x &= \frac{1}{x} \\
% \end{align*} 

\begin{align*}
 \mu C(x_{0},x_{1},y_{0},y_{1},u).&(\langle x_{0}?(z) \rangle(\langle u! \rangle\langle y_{1}!z \rangle C(x_{0},x_{1},y_{0},y_{1},u)) & \\
  & \wedge \langle y_{1}?(z) \rangle (\langle u! \rangle \langle x_{0}!z \rangle C(x_{0},x_{1},y_{0},y_{1},u)) & \\
  & \wedge \langle x_{1}?(z) \rangle (\langle u? \rangle \langle y_{0}!z \rangle C(x_{0},x_{1},y_{0},y_{1},u)) & \\
  & \wedge \langle y_{0}?(z) \rangle (\langle u? \rangle \langle x_{1}!z \rangle C(x_{0},x_{1},y_{0},y_{1},u))) &
\end{align*}

The lexicographical similarity between the shape of this formulae and
the shape of definition of the process representing a crossing reveals
the intuitive meaning of this formulae. It describes the capabilities
of a process that has the right to represent a crossing. For example
it picks out processes that may perform an input on the port $x_0$ in
its initial menu of capabilities. What differentiates the formula
from the process, however, is that the crossing process is the
smallest candidate to satisfy the formula. Infinitely many other
processes -- with internal behavior hidden behind this interface, so
to speak -- also satisfy this formula. Even this simple formula,
then, can be seen to open a new view onto knots, providing a
computational interpretation of \emph{virtual} knots.

Note that this formula is derived by hand. A similar formula can be
derived by employing Caires' calculation of characteristic formula
\cite{Caires04} to the process representing a crossing. In light of
this discussion, we let
$\meaningof{C}_{\phi}(x0,x1,y0,y1,u)$ denote a formula specifying the
dynamics we wish to capture of a crossing. To guarantee we preserve
the shape of the interface and minimal semantics we demand that
$\meaningof{C}_{\phi}(x0,x1,y0,y1,u) \Rightarrow
\textbf{C}(x0,x1,y0,y1,u)$ where $\textbf{C}(x0,x1,y0,y1,u)$ denotes
the formula above.
                            
\subsubsection{Crossing number constraints.}
The moral content of the context lemma (Lemma \ref{context}) is that the notion of
``locality'' in the Reidemeister moves is effectively captured by the
parallel composition operator of the process calculus. This intuition
extends through the logic. Given a formula,
$\meaningof{C}_{\phi}(x0,x1,y0,y1,u)$, we can use the structural
connectives to specify constraints on crossing numbers, such as at
least $n$ crossings, or exactly $n$ crossings.
\begin{mathpar}
  \inferrule* [lab=at-least-n] {} { K^{\geq n}_{\phi}(\vec{xs},\vec{ys}) := \Pi_{i=0}^{n-1} Hu . \meaningof{C}_{\phi}(xs_i,ys_i,u) | T }
  \and 
  \inferrule* [lab=exactly-n] {} { K^{= n}_{\phi}(\vec{xs},\vec{ys}) := \Pi_{i=0}^{n-1} Hu . \meaningof{C}_{\phi}(xs_i,ys_i,u) | \neg (\forall x_0,y_0,x_1,y_1,u . \meaningof{C}_{\phi}(x_0,y_0,x_1,y_1,u) | T) }
\end{mathpar}

To round out this section, recall that the encoding of an $n$-crossing
knot decomposes into a parallel composition of $n$ \emph{copies} of a
crossing process together with a wiring harness. To specify different
knot classes with the same crossing number amounts to specifying
logical constraints on the wiring harness. In the interest of space,
we defer examples to a forthcoming paper. Suffice it to say that both
the conditions ``alternating knot'' and ``contains the tangle
corresponding to 5/3'' are expressible. For example, it is possible to
calculate the characteristic formula of a process corresponding to the
tangle 5/3 and conjoin it into the classifying formula via the
composition connective of the logic.

Finally, we wish to observe that it is entirely within reason to
contemplate a more domain-specific version of spatial logic tailored
to the shape of processes in the image of the encoding. Such a
domain-specific logic would have a better claim to the title formal
language of knot properties.

% subsection example_formulae_ (end)

% section knots_as_processes (end) 

% section spatial logic via knots (end)

\section{Conclusions and future work}

\paragraph{Testing physical space}
You, gentle reader, may wonder why of all the theorems to be proved
given this set up we pick the one above. In some sense it's hardly
central to quantum mechanics. We see it as central in the sense that
it firmly establishes a notion of physical space arising from a notion
of the equivalence of behavior. Relating bisimulation to a metric is a
big step forward, but one is faced with interpreting the relationship
of that metric space to something more physical. Quantum mechanical
notions of ``physical'' space are still far from intuitive, but by
relating this idea of distance as testing to calculations that predict
physical circumstances we are making a not insignificant step forward
toward an understanding of the physical space we inhabit as
essentially dynamic.

\paragraph{Effectivity and simulation}
One of the observations we have yet to make is that the entire program
spelled out here is effective. We have built various interpreters for
the reflective calculus at work in this interpretation. In principle,
then, we can simulate quantum mechanics on a computer. The place where
the simulation may lose fidelity is the infinitely branching summation
for the annihilator.

In this connection i also want to point out that the evaluation style
calculation of the inner product puts the non-determinism of the
summation right at the heart of measurement. This suggests that
Milner's original reduction-based formulation of the dynamics of his
calculi in terms of sums was not just notationally suggestive of a
notion of measure-and-continue but captured some significant part of
the physics.

\paragraph{Quantum continuations}
In light of this last observation i want to point out that the
predominant account of quantum mechanics is missing a key aspect of a
truly compositional story of the physical situation. In a real lab,
when a measurement is made the observation can be made to feed into
another device that then makes another measurement conditioned on the
results of the first. This means that after the superposition was
collapsed the entire experimental set up remained in
superposition. While QM offers a means of writing this down it doesn't
quite line up well with the well-trodden formulation of computation
and continuation that we see so succinctly expressed in Milner's
calculi. This suggests that there might be advantages to this account
of dynamics waiting to be explored.

\paragraph{Quantum logic}
In this connection, we also note that by virtue of having the
Hennessy-Milner construction, we can pull the construction through the
interpretation of QM. This gives us a natural candidate for a quantum
logic that enjoys an extremely tight connection with it's domain of
interpretation, making the construction much less ad hoc (rather it is
the image of functor!).

\paragraph{Quantum probabiity}
i have questions about the basis of the interpretation of inner
product as probability amplitude. In particular, using which
axiomatization of probability theory does the notion of probability
amplitude earn the right to be so dubbed? In other words, where is the
proof that the operation for calculating a probability amplitude (and
then squaring) satisfies the axioms of what it means to calculate a
probability? Even if such a proof exists (i have yet to find it in the
literature), i wonder if it might not be possible to turn things on
their heads. Can we view the calculation of the probability amplitude
as an axiomatization of probability? If so, then the definition we
give for calculating probability amplitude may provide the basis for
an \emph{effective} theory of probability.

\paragraph{Quantum vs ``biological'' information}
Finally, i want to conclude with a more philosophical observation. At
a recent workshop in which QM was a predominant topic i noticed
something about quantum information. The speaker was giving a riveting
discussion of axiomatic QM and showing how properties of ``no
cloning'' and ``no deleting'' emerged as consequences of the
axiomatization. Theorems of this form are necessary to give us a sense
of confidence that our axioms characterize the physical theory. What
struck me, though, was that if quantum information is neither erasable
nor replicable it is markedly different from \emph{life}. Two of the
things we know about life is that

\begin{itemize}
  \item it ends;
  \item to gain some measure of persistence, to transcend it's
    finitude it is imminently copyable.
\end{itemize}

Both of these qualities are summarized succinctly in the aphorism: all
flesh is grass. For me these two kinds of ``information'' -- call them
quantum and biological -- are end points on a spectrum of strategies
for persistence. At one end, we have those curious entities that enjoy
uniqueness and permanence; at the other, we have those who in the face
of a certain end and an uncertain present make a go of passing
something on. To me one of the more remarkable aspects of the latter
strategy is that in the presence of noise (and certain features of
copying) we get a kind of dynamism, a chance for improvement against a
given persistent condition.

% subsection other_calculi_other_bisimulations_and_geometry_as_behavior (end)




% section conclusion (end)

%\documentclass[12pt]{llncs}
%\documentclass{jktr}

\usepackage[pdftex]{hyperref}                   
\usepackage {listings}
\usepackage {mathpartir}
\usepackage{bcprules}
%\usepackage{listings}
                       
\usepackage{graphicx} 
%\usepackage[margins=2.5cm,nohead,nofoot]{geometry}
%\usepackage{geometry}
\usepackage{amsfonts}
\usepackage{amstext}
\usepackage{latexsym}
\usepackage{amssymb}
\usepackage{color}


%\include{myPreamble}
\include{qm2pi.local} 

%\ifpdf
%\usepackage[pdftex]{graphicx}
%\else
%\usepackage{graphicx}
%\fi

 % \ifpdf
%  \usepackage{pdfsync}
%  \if


%\title{Brief Article}
%\author{David F. Snyder}
%\author{L.G. Meredith}

%\address{Dept. of Math., Texas State University--San Marcos, San Marcos, TX 78666}
       
\pagestyle{empty}


\begin{document}

\lstset{language=[Objective]Caml,frame=shadowbox}

\input{qm2pi.front}

% section front matter (end)

\input{qm2pi.intro} 
 
% section introduction (end)

% \input{qm2pi.knotations} 

% section notation (end)

\input{qm2pi.process.calculi} 

% section concurrent_process_calculi_and_spatial_logics_ (end)
    
%\input{qm2pi.knots2pi} 

%\input{qm2pi.trefoil} 

%\input{qm2pi.mainthm} 

% subsection basic_interpretation (end)

%\input{qm2pi.rho.presentation} 
\subsection{The syntax and semantics of the notation system}\label{sub:the_syntax_and_semantics_of_the_notation_system} % (fold)

We now summarize a technical presentation of the calculus that
embodies our theory of dynamics. The typical presentation of such a
calculus follows the style of giving generators and relations on
them. The grammar, below, describing term constructors, freely
generates the set of processes, $\Proc$. This set is then quotiented
by a relation known as structural congruence and it is over this set
that the notion of dynamics is expressed. This presentation is
essentially that of \cite{MeredithR05} with the addition of
polyadicity and summation. For readability we have relegated some of
the technical subtleties to an appendix.

\subsubsection{Process grammar}\label{subsub:process_grammar}

\begin{mathpar}
  \inferrule* [lab=synchronization] {} {{M} \bc \pzero \;|\; x?F \;|\; x!C }
  \and
  \inferrule* [lab=abstraction] {} {{F} \bc (x)P}
  \and
  \inferrule* [lab=concretion] {} {{C} \bc \langle Q \rangle}
  \and
  \inferrule* [lab=process] {} {{P,Q} \bc M \;| \;P|Q \;|\; @{x}}
  \and
  \inferrule* [lab=name] {} {{x} \bc \quotep{P}}
\end{mathpar} 

Note that $\vec{x}$ (resp. $\vec{P}$) denotes a vector of names
(resp. processes) of length $|\vec{x}|$ (resp. $|\vec{P}|$). We adopt
the following useful abbreviations.

\begin{mathpar}
   x?(\vec{y}).P := x.(\vec{y})P \and  x\clift{\vec{P}} := x.\clift{\vec{P}}
   \and x!(y) := \lift{x}{\dropn{y}}
   \and \Pi_{i=0}^{n-1}P_i := P_0 | \ldots | P_{n-1}
\end{mathpar}

\subsubsection{Structural congruence}

\paragraph{Free and bound names and alpha-equivalence.} At the
core of structural equivalence is alpha-equivalence which identifies
process that are the same up to a change of variable. Formally, we
recognize the distinction between free and bound names. The free names
of a process, $\freenames{P}$, may be calculated recursively as
follows:

\begin{mathpar}
\freenames{\pzero} := \emptyset
  \and \\
  \freenames{x?(y).P} := \{ x \} \cup (\freenames{P} \setminus \{ y \})
  \and 
  \freenames{x!\langle P \rangle} := \{ x \} \cup \{ P \} 
  \and \\
  \freenames{P|Q} := \freenames{P} \cup \freenames{Q}
  \and \\
  \freenames{@{x}} := \{ x \}
\end{mathpar}

$\pi$
$\quotep{\pi}$

$\freenames{-} : \pi \to \mathcal{P}(\quotep{\pi})$

\begin{eqnarray*}
  \freenames{\pzero} & := & \emptyset \\
  \freenames{x?(y).P} & := & \{ x \} \cup (\freenames{P} \setminus \{ y \}) \\
  \freenames{x!\langle P \rangle} & := & \{ x \} \cup \{ P \} \\
  \freenames{P|Q} & := & \freenames{P} \cup \freenames{Q} \\
  \freenames{\dropn{x}} & := & \{ x \}
\end{eqnarray*}

The bound names of a process, $\boundnames{P}$, are those names occurring in $P$
that are not free. For example, in $x?(y).0$, the name $x$ is free, while $y$ is bound.

\begin{mathpar}
  \inferrule* [lab=monoidal-laws] {} { P|Q \equiv Q|P \and P|0 \equiv P \and P|(Q|R) \equiv (P|Q)|R }
\end{mathpar}

\begin{mathpar}
  \inferrule* [lab=alpha-equivalence] {} { (x)P \equiv (y)P\{y/x\} \and y \not\in \freenames{P} }
\end{mathpar}

\begin{definition}
Then two processes, $P,Q$, are alpha-equivalent if $P = Q\{\vec{y}/\vec{x}\}$ for
some $\vec{x} \in \boundnames{Q},\vec{y} \in \boundnames{P}$, where $Q\{\vec{y}/\vec{x}\}$
denotes the capture-avoiding substitution of $\vec{y}$ for $\vec{x}$ in $Q$.
\end{definition}

\begin{definition}
  The {\em structural congruence} \cite{SangiorgiWalker} , $\equiv$,
  between processes is the least congruence containing
  alpha-equivalence, satisfying the abelian monoid laws
  (associativity, commutativity and $\pzero$ as identity) for parallel
  composition $|$ and for summation $+$.
\end{definition}

\subsection{Name equivalence}

We take name equivalence, written $\nameeq$, to be the smallest
equivalence relation generated by the following rules.

\begin{mathpar}
\inferrule*[lab=Quote-drop]
{ }
{ \quotep{@{x}} \nameeq x }

\inferrule*[lab=Struct-equiv]
{ P \scong Q }
{ \quotep{P} \nameeq \quotep{Q} }
\end{mathpar}

The astute reader will have noticed that the mutual recursion of names
and processes imposes a mutual recursion on alpha-equivalence and
structural equivalence via name-equivalence. Fortunately, all of this
works out pleasantly and we may calculate in the natural way, free of
concern. The reader interested in the details is referred to the
appendix \ref{appendix:rho_details}.

\subsection{Substitution}

We use $\Proc$ for the set of processes, $\QProc$ for the set of
names, and $\id{\{}\vec{y} / \vec{x} \id{\}}$ to denote partial maps,
$s : \QProc \rightarrow \QProc$. A map, $s$ lifts, uniquely, to a map
on process terms, $\widehat{s} : \Proc \rightarrow \Proc$ by the
following equations.

\begin{mathpar}
  (0) \psubstp{Q}{P} := 0 \\
  (R \juxtap S) \psubstp{Q}{P}
  :=    
  (R)\psubstp{Q}{P} \juxtap (S) \psubstp{Q}{P} \\
  (x?(y).R) \psubstp{Q}{P}    
  :=    
  (x)\substp{Q}{P} (z)\concat( (R \psubstn{z}{y}) \psubstp{Q}{P} ) \\
  (\lift{x}{R}) \psubstp{Q}{P}  
  :=
  \lift{(x)\substp{Q}{P}}{ R \psubstp{Q}{P} } \\
%   (\dropn{x})  \psubstp{Q}{P}       
%   := 
%   \left\{ 
%     \begin{array}{ccc} 
%       \dropn{\quotep{Q}} & & x \nameeq \quotep{P} \\
%       \dropn{x} & & otherwise \\
%     \end{array}
%   \right. 
  (\dropn{x})  \psubstp{Q}{P}       
  := 
  \left\{ 
    \begin{array}{ccc} 
      Q & & x \nameeq \quotep{P} \\
      \dropn{x} & & otherwise \\
    \end{array}
  \right.
\end{mathpar}
 

where

\begin{eqnarray}
  (x)\id{\{} \lpquote Q \rpquote / \lpquote P \rpquote \id{\}}            = 
  \left\{ 
    \begin{array}{ccc}
      \lpquote Q \rpquote & & x \nameeq \lpquote P \rpquote \\
      x & & otherwise \\
    \end{array}
  \right. \nonumber
\end{eqnarray}

and $z$ is chosen distinct from $\quotep{P}$, $\quotep{Q}$, the free
names in $Q$, and all the names in $R$. Our $\alpha$-equivalence will
be built in the standard way from this substitution.

\begin{remark}\label{rem:no_self_referential_names}
  One consequence of these definitions is that $\forall P. \quotep{P}
  \not\in \freenames{P}$.
\end{remark}

\subsection{ Dynamic quote: an example }

Anticipating something of what's to come, consider applying the
substitution, $\widehat{\id{\{}u / z \id{\}}}$, to the following pair
of processes, $\lift{w}{y!(z)}$ and $w[ \lpquote y!(z) \rpquote ]$.

\begin{eqnarray}
	\lift{w}{y!(z)}\widehat{\id{\{}u / z \id{\}}}
		& = &
		\lift{w}{y!(u)} \nonumber\\
	w[ \lpquote y!(z) \rpquote ] \widehat{ \id{\{}u / z \id{\}} }
		& = &
		w[ \lpquote y!(z) \rpquote ] \nonumber
\end{eqnarray}

Because the body of the process between quotes is impervious to
substitution, we get radically different answers. In fact, by
examining the first process in an input context,
e.g. $x?(z).\lift{w}{y!(z)}$, we see that the process under the lift
operator may be shaped by prefixed inputs binding a name inside it. In
this sense, the lift operator will be seen as a way to dynamically
construct processes before reifying them as names.

Finally equipped with these standard features we can present the
dynamics of the calculus.

\subsubsection{Operational semantics} 

Finally, we introduce the computational dynamics. What marks these
algebras as distinct from other more traditionally studied algebraic
structures, e.g. vector spaces or polynomial rings, is the manner in
which dynamics is captured. In traditional structures, dynamics is typically
expressed through morphisms between such structures, as in linear maps
between vector spaces or morphisms between rings. In algebras
associated with the semantics of computation, the dynamics is
expressed as part of the algebraic structure itself, through a
reduction reduction relation typically denoted by $\red$. Below, we
give a recursive presentation of this relation for the calculus used
in the encoding.

$\red \subseteq \pi \times \pi$
$\red : \pi \to \mathcal{P}(\pi)$

\begin{mathpar}
  \inferrule* [lab=Comm] { \textsf{match}( x_{src}, x_{trgt} ) } { x_{trgt}?(y)P \; | \; x_{src}!\langle {Q} \rangle \red P\{\quotep{Q}/y}\} }
  \and \\
  \inferrule* [lab=Par] {{P} \red {P}'} {{{P} | {Q}} \red {{P}' | {Q}}}
  \and
  \inferrule* [lab=Equiv]{{{P} \scong {P}'} \andalso {{P}' \red {Q}'} \andalso {{Q}' \scong {Q}}}{{P} \red {Q}}
\end{mathpar}

\begin{eqnarray*}
  match_{\equiv} (\quotep{P},\quotep{Q}) & := & P \equiv Q \\
  match_{\dagger}(\quotep{P},\quotep{Q}) & := & \forall R. P|Q \red^{*} R => R \red^{*} 0 \\
  match_{K}(\quotep{P},\quotep{Q}) & := & K \mbox{ for some context } K
\end{eqnarray*}

$u?(x)P | u!\langle Q \rangle \red P\{\quotep{Q}/x\}$

%We write $\wred$ for $\red^*$, and $P\red$ if $\exists Q $ such that $ P \red Q$.
We write $P\red$ if $\exists Q $ such that $ P \red Q$ and $P\not\red$, otherwise.

\section{Replication}

As mentioned before, it is known that replication (and hence
recursion) can be implemented in a higher-order process algebra
\cite{SangiorgiWalker}. As our first example of calculation with the
machinery thus far presented we give the construction explicitly in
the {\rhoc}.

\begin{eqnarray}
	D_{x} & := & \prefix{x}{y}{(\binpar{\outputp{x}{y}}{@{y}})} \nonumber\\
	\bangp_{x}{P} & := & \binpar{{x}!\langle{\binpar{D_{x}}{P}}\rangle}{D_{x}} \nonumber
\end{eqnarray}

\begin{eqnarray}
	\bangp_{x}{P} & & \nonumber\\
	=
	& {x}!\langle{(\prefix{x}{y}{(\outputp{x}{y} | @{y})) | P}}\rangle 
	      | \prefix{x}{y}{(\outputp{x}{y} | @{y})} & \nonumber\\
	\red
	& (\outputp{x}{y} | @{y})\substn{\quotep{(\prefix{x}{y}{(@{y} | \outputp{x}{y})) | P}}}{y} & \nonumber\\
	=
	& \outputp{x}{\quotep{(\prefix{x}{y}{(\outputp{x}{y} | @{y})) | P}}}
	  | {(\prefix{x}{y}{(\outputp{x}{y} | @{y})) | P}} & \nonumber\\
	\red
	& \ldots & \nonumber\\
	\red^*
	& P | P | \ldots & \nonumber
\end{eqnarray}

Of course, this encoding, as an implementation, runs away, unfolding
$\bangp{P}$ eagerly. A lazier and more implementable replication
operator, restricted to input-guarded processes, may be obtained as follows.

\begin{eqnarray}
\bangp{\prefix{u}{v}{P}} 
	:= 
	\binpar{\lift{x}{\prefix{u}{v}{(\binpar{D(x)}{P})}}}{D(x)} \nonumber
\end{eqnarray}

\begin{remark}
  Note that the lazier definition still does not deal with summation
  or mixed summation (i.e. sums over input and output). The reader is
  invited to construct definitions of replication that deal with these
  features. 

  Further, the definitions are parameterized in a name, $x$. Can you,
  gentle reader, make a definition that eliminates this parameter and
  guarantees no accidental interaction between the replication
  machinery and the process being replicated -- i.e. no accidental
  sharing of names used by the process to get its work done and the
  name(s) used by the replication to effect copying. This latter
  revision of the definition of replication is crucial to obtaining
  the expected identity $!!P \sim !P$.
\end{remark}

\begin{remark}\label{rem:paradoxical_combinator}
  The reader familiar with the lambda calculus will have noticed the
  similarity between $D$ and the paradoxical combinator.

  [Ed. note: the existence of this seems to suggest we have to be more
  restrictive on the set of processes and names we admit if we are to
  support no-cloning.]
\end{remark}

\subsubsection{Bisimulation}

The computational dynamics gives rise to another kind of equivalence,
the equivalence of computational behavior. As previously mentioned
this is typically captured \emph{via} some form of bisimulation.

% The notion we use in this paper is weak barbed bisimulation
% \cite{milner91polyadicpi}.

The notion we use in this paper is derived from weak barbed
bisimulation \cite{milner91polyadicpi}. 

\begin{definition}
An \emph{observation relation}, $\downarrow_{\mathcal N}$, over a set
of names, $\mathcal N$, is the smallest relation satisfying the rules
below.

\infrule[Out-barb]{y \in {\mathcal N}, \; x \nameeq y}
		  {\outputp{x}{v} \downarrow_{\mathcal N} x}
\infrule[Par-barb]{\mbox{$P\downarrow_{\mathcal N} x$ or $Q\downarrow_{\mathcal N} x$}}
		  {\binpar{P}{Q} \downarrow_{\mathcal N} x}

We write $P \Downarrow_{\mathcal N} x$ if there is $Q$ such that 
$P \wred Q$ and $Q \downarrow_{\mathcal N} x$.
\end{definition}

\begin{definition}
%\label{def.bbisim}
An  ${\mathcal N}$-\emph{barbed bisimulation} over a set of names, ${\mathcal N}$, is a symmetric binary relation 
${\mathcal S}_{\mathcal N}$ between agents such that $P\rel{S}_{\mathcal N}Q$ implies:
\begin{enumerate}
\item If $P \red P'$ then $Q \wred Q'$ and $P'\rel{S}_{\mathcal N} Q'$.
\item If $P\downarrow_{\mathcal N} x$, then $Q\Downarrow_{\mathcal N} x$.
\end{enumerate}
$P$ is ${\mathcal N}$-barbed bisimilar to $Q$, written
$P \wbbisim_{\mathcal N} Q$, if $P \rel{S}_{\mathcal N} Q$ for some ${\mathcal N}$-barbed bisimulation ${\mathcal S}_{\mathcal N}$.
\end{definition}

$\mathcal{R} \subseteq \pi \times \pi$

$P \mathcal{R} Q => \forall P'. P \red P' \Rightarrow \exists Q'. Q \red Q', P' \mathcal{R} Q'$

$P \vdash x \Rightarrow Q \vdash x$

\begin{mathpar}
  \inferrule*[lab=Out-barb]{x \nameeq y}{{y}!\langle{Q}\rangle \vdash x}
  \and
  \inferrule*[lab=Par-barb]{\mbox{$P\vdash x$ or $Q\vdash x$}}{\binpar{P}{Q} \vdash x}
\end{mathpar}

\subsubsection{Contexts}

One of the principle advantages of computational calculi like the
$\pi$-calculus is a well-defined notion of context,
contextual-equivalence and a correlation between
contextual-equivalence and notions of bisimulation. The notion of
context allows the decomposition of a process into (sub-)process and
its syntactic environment, its context. Thus, a context may be
thought of as a process with a ``hole'' (written $\Box$) in it. The
application of a context $M$ to a process $P$, written $M[P]$, is
tantamount to filling the hole in $M$ with $P$. In this paper we do
not need the full weight of this theory, but do make use of the notion
of context in the proof the main theorem. 

\begin{mathpar}
  \inferrule* [lab=summation] {} {{M_{M},M_{N}} \bc \Box \;|\; x.M_{A} \;|\; M_{M}+M_{N}}
  \and
  \inferrule* [lab=agent] {} {{M_{A}} \bc (\vec{x})M_{P} \;| \; \clift{P_0,\ldots,M_{P},\ldots,P_N}}
  \and \\
  \inferrule* [lab=process] {} {{M_{P}} \bc M_{N} \;| \;P|M_{P} }
\end{mathpar} 

\begin{mathpar}
  \inferrule* [lab=sychronization] {} {M_{N} \bc \Box \;|\; x?M_{F} \;|\; x!M_{C}}
  \and
  \inferrule* [lab=abstraction] {} {{M_{F}} \bc (x)M_{P} }
  \and
  \inferrule* [lab=concretion] {} {{M_{C}} \bc \langle M_{P} \rangle }
  \and \\
  \inferrule* [lab=process] {} {{M_{P}} \bc M_{N} \;| \;P|M_{P} }
\end{mathpar}

\begin{definition}[contextual application] Given a context $M$, and
  process $P$, we define the \emph{contextual application}, $M[P] :=
  M\{P/\Box\}$. That is, the contextual application of M to P is the
  substitution of $P$ for $\Box$ in $M$.
\end{definition}

$\meaningof{-} : L \to \mathcal{P}(\pi)$

\begin{mathpar}
  \inferrule* [lab=collection] {} {\meaningof{true} = \pi, \and \meaningof{~E} = \pi \setminus \meaningof{E}, \and \meaningof{E_{1} \& E_{2}} = \meaningof{E_{1}} \cap \meaningof{E_{2}}}
\end{mathpar}

\begin{mathpar}
  \inferrule* [lab=structure] {} {\meaningof{0} = \{ P \in \pi | P \equiv 0 \}, \and \\ \meaningof{E_1 | E_2} = \{ P \in \pi | P \equiv P_{1} | P_{2}, P_{1} \in \meaningof{E_{1}}, P_{2} \in \meaningof{E_2}\} }
\end{mathpar}

\begin{mathpar}
 \inferrule* [lab=behavior] {} {\meaningof{\langle a?b \rangle E} = \{ P \in \pi | P \equiv Q | u?(y)P', \\ \and \\\\ \and \\ \;\;\; u \in \meaningof{a}, \forall z.P'\{z/y\} \in \meaningof{E\{z/b\}}\}, \and \\ \meaningof{a!E} = \{ P \in \pi | P \equiv Q | x!\langle P' \rangle, x \in \meaningof{a} P' \in \meaningof{E}\} }
\end{mathpar}

\begin{mathpar}
 \inferrule* [lab=nominal] {} {\meaningof{\quotep{E}} = \{ \quotep{P} \in \quotep{\pi} | P \in \meaningof{E} \}, \and \meaningof{\quotep{P}} = \{ \quotep{Q} \in \quotep{\pi} | P \equiv Q \} \and \\ \meaningof{@\quotep{E}} = \{ P \in \pi | P \equiv @x, x \in \meaningof{E} \}}
\end{mathpar}

\begin{eqnarray*}
  \\
  \meaningof{-} : TS \to ST
\end{eqnarray*}

\begin{eqnarray*}
  \\
  L : TS \to ST
\end{eqnarray*}

\begin{eqnarray*}
  \\
  P \models E \iff P \in \meaningof{E}
\end{eqnarray*}

\begin{eqnarray*}
  P \approx_{L} Q \iff \forall E \in L. P \models E \iff Q \models E
\end{eqnarray*}

\begin{eqnarray*}
  P \approx_{K} Q
\end{eqnarray*}

\begin{eqnarray*}
  P \approx Q
\end{eqnarray*}

$\approx_{K} = \approx = \approx_{L}$

\subsubsection{Contextual duality}

Note that contexts extend the quotation operation to a family of
operations from processes to names. Given a context, $M$, we can
define a \emph{nominal context}, $\quotep{M}$ by $\quotep{M}[P] :=
\quotep{M[P]}$. To foreshadow what is to come we observe that these
operations enjoy a duality with processes very much like the duality
between vectors and maps from vectors to scalars.

Further, because the calculus is essentially higher-order, we have a
correspondence between contexts and processes. More specifically,
given a name $x$ and a context $M$ we can construct $M^{*}_{x}$ such
that 

\begin{mathpar}
  M^{*}_{x} | \lift{x}{P} \red M[P]
\end{mathpar}

namely,

\begin{mathpar}
  M^{*}_{x} := x?(u).M[\dropn{u}]
\end{mathpar}

The dependence of $M^{*}_{x}$ on a name makes it an abstraction, 

\begin{mathpar}
  M^{*} := (x)x?(u).M[\dropn{u}]
\end{mathpar}

\subsection{Additional notation}

It will sometimes be convenient to denote the process a name
quotes. We already have the notation $x = \quotep{P}$, but it will be
convenient to introduce an alternate notation, $\procn{x}$, when we
want to emphasize the connection to the use of the name. Note that, by
virtue of name equivalence, $\quotep{\procn{x}} \nameeq x$; so, the
notation is consistent with previous definitions.

Further, because names have structure it is possible to effect
substitutions on the basis of that structure. This means we need to
upgrade our notation for substitutions, which we accomplish by
adapting comprehension notation. Thus,

\begin{mathpar}
  P\{ y / x : x \in S \}
\end{mathpar}

is interpreted to mean the process derived from P by replacing (in a
capture-avoiding manner) each occurrence of $x$ in $S$ by $y$. For example,

\begin{mathpar}
  P\{ \quotep{\procn{x}|\procn{x}} / x : x \in \freenames{P} \}
\end{mathpar}

will replace each (occurrence) of a free name $x$ in $P$ by
$\quotep{\procn{x}|\procn{x}}$.

Also, we will avail ourselves of the notation $x^{L}$ and $x^{R}$ to
denote injections of a name into disjoint copies of the name
space. There are numerous ways to accomplish this. One example can be
found in \cite{MeredithR05}. This notation overloads to vectors of
names: $\vec{x}^{\pi} := (x_{i}^{\pi} \; : \; 0 \leq i < |\vec{x}| )$ where $\pi \in \{L,R\}$.

We also use $P^{\Box} := P|\Box$.

In \cite{MeredithR05} an interpretation of the new operator is
given. It turns out that there are several possible interpretations
all enjoying the requisite algebraic properties of the operator (see
\cite{milner91polyadicpi}). We will therefore make liberal use of
$(\nu\; \vec{x})P$.

% subsection the_syntax_and_semantics_of_the_notation_system (end)   

\input{qm2pi.qmops} 

\input{qm2pi.sterngerlach} 

\input{qm2pi.metric} 

% section concurrent_process_calculi (end)

%\input{qm2pi.proofsketch}

% section proof sketch (end)

%\input{qm2pi.slviaknots} 

% section spatial logic via knots (end)

\input{qm2pi.conclusion}

% section conclusion (end)

%\input{qm2pi.dtcodes} 

% section wiring algorithm (end)

\input{qm2pi.ack} 

% section acknowledgments (end)

\newpage


\bibliographystyle{plain}   
\bibliography{../../biblios/main.bib}

\input{qm2pi.rhodetails}

\end{document}

 

% section wiring algorithm (end)

\documentclass[12pt]{llncs}
%\documentclass{jktr}

\usepackage[pdftex]{hyperref}                   
\usepackage {listings}
\usepackage {mathpartir}
\usepackage{bcprules}
%\usepackage{listings}
                       
\usepackage{graphicx} 
%\usepackage[margins=2.5cm,nohead,nofoot]{geometry}
%\usepackage{geometry}
\usepackage{amsfonts}
\usepackage{amstext}
\usepackage{latexsym}
\usepackage{amssymb}
\usepackage{color}


%\include{myPreamble}
\include{qm2pi.local} 

%\ifpdf
%\usepackage[pdftex]{graphicx}
%\else
%\usepackage{graphicx}
%\fi

 % \ifpdf
%  \usepackage{pdfsync}
%  \if


%\title{Brief Article}
%\author{David F. Snyder}
%\author{L.G. Meredith}

%\address{Dept. of Math., Texas State University--San Marcos, San Marcos, TX 78666}
       
\pagestyle{empty}


\begin{document}

\lstset{language=[Objective]Caml,frame=shadowbox}

\input{qm2pi.front}

% section front matter (end)

\input{qm2pi.intro} 
 
% section introduction (end)

% \input{qm2pi.knotations} 

% section notation (end)

\input{qm2pi.process.calculi} 

% section concurrent_process_calculi_and_spatial_logics_ (end)
    
%\input{qm2pi.knots2pi} 

%\input{qm2pi.trefoil} 

%\input{qm2pi.mainthm} 

% subsection basic_interpretation (end)

%\input{qm2pi.rho.presentation} 
\subsection{The syntax and semantics of the notation system}\label{sub:the_syntax_and_semantics_of_the_notation_system} % (fold)

We now summarize a technical presentation of the calculus that
embodies our theory of dynamics. The typical presentation of such a
calculus follows the style of giving generators and relations on
them. The grammar, below, describing term constructors, freely
generates the set of processes, $\Proc$. This set is then quotiented
by a relation known as structural congruence and it is over this set
that the notion of dynamics is expressed. This presentation is
essentially that of \cite{MeredithR05} with the addition of
polyadicity and summation. For readability we have relegated some of
the technical subtleties to an appendix.

\subsubsection{Process grammar}\label{subsub:process_grammar}

\begin{mathpar}
  \inferrule* [lab=synchronization] {} {{M} \bc \pzero \;|\; x?F \;|\; x!C }
  \and
  \inferrule* [lab=abstraction] {} {{F} \bc (x)P}
  \and
  \inferrule* [lab=concretion] {} {{C} \bc \langle Q \rangle}
  \and
  \inferrule* [lab=process] {} {{P,Q} \bc M \;| \;P|Q \;|\; @{x}}
  \and
  \inferrule* [lab=name] {} {{x} \bc \quotep{P}}
\end{mathpar} 

Note that $\vec{x}$ (resp. $\vec{P}$) denotes a vector of names
(resp. processes) of length $|\vec{x}|$ (resp. $|\vec{P}|$). We adopt
the following useful abbreviations.

\begin{mathpar}
   x?(\vec{y}).P := x.(\vec{y})P \and  x\clift{\vec{P}} := x.\clift{\vec{P}}
   \and x!(y) := \lift{x}{\dropn{y}}
   \and \Pi_{i=0}^{n-1}P_i := P_0 | \ldots | P_{n-1}
\end{mathpar}

\subsubsection{Structural congruence}

\paragraph{Free and bound names and alpha-equivalence.} At the
core of structural equivalence is alpha-equivalence which identifies
process that are the same up to a change of variable. Formally, we
recognize the distinction between free and bound names. The free names
of a process, $\freenames{P}$, may be calculated recursively as
follows:

\begin{mathpar}
\freenames{\pzero} := \emptyset
  \and \\
  \freenames{x?(y).P} := \{ x \} \cup (\freenames{P} \setminus \{ y \})
  \and 
  \freenames{x!\langle P \rangle} := \{ x \} \cup \{ P \} 
  \and \\
  \freenames{P|Q} := \freenames{P} \cup \freenames{Q}
  \and \\
  \freenames{@{x}} := \{ x \}
\end{mathpar}

$\pi$
$\quotep{\pi}$

$\freenames{-} : \pi \to \mathcal{P}(\quotep{\pi})$

\begin{eqnarray*}
  \freenames{\pzero} & := & \emptyset \\
  \freenames{x?(y).P} & := & \{ x \} \cup (\freenames{P} \setminus \{ y \}) \\
  \freenames{x!\langle P \rangle} & := & \{ x \} \cup \{ P \} \\
  \freenames{P|Q} & := & \freenames{P} \cup \freenames{Q} \\
  \freenames{\dropn{x}} & := & \{ x \}
\end{eqnarray*}

The bound names of a process, $\boundnames{P}$, are those names occurring in $P$
that are not free. For example, in $x?(y).0$, the name $x$ is free, while $y$ is bound.

\begin{mathpar}
  \inferrule* [lab=monoidal-laws] {} { P|Q \equiv Q|P \and P|0 \equiv P \and P|(Q|R) \equiv (P|Q)|R }
\end{mathpar}

\begin{mathpar}
  \inferrule* [lab=alpha-equivalence] {} { (x)P \equiv (y)P\{y/x\} \and y \not\in \freenames{P} }
\end{mathpar}

\begin{definition}
Then two processes, $P,Q$, are alpha-equivalent if $P = Q\{\vec{y}/\vec{x}\}$ for
some $\vec{x} \in \boundnames{Q},\vec{y} \in \boundnames{P}$, where $Q\{\vec{y}/\vec{x}\}$
denotes the capture-avoiding substitution of $\vec{y}$ for $\vec{x}$ in $Q$.
\end{definition}

\begin{definition}
  The {\em structural congruence} \cite{SangiorgiWalker} , $\equiv$,
  between processes is the least congruence containing
  alpha-equivalence, satisfying the abelian monoid laws
  (associativity, commutativity and $\pzero$ as identity) for parallel
  composition $|$ and for summation $+$.
\end{definition}

\subsection{Name equivalence}

We take name equivalence, written $\nameeq$, to be the smallest
equivalence relation generated by the following rules.

\begin{mathpar}
\inferrule*[lab=Quote-drop]
{ }
{ \quotep{@{x}} \nameeq x }

\inferrule*[lab=Struct-equiv]
{ P \scong Q }
{ \quotep{P} \nameeq \quotep{Q} }
\end{mathpar}

The astute reader will have noticed that the mutual recursion of names
and processes imposes a mutual recursion on alpha-equivalence and
structural equivalence via name-equivalence. Fortunately, all of this
works out pleasantly and we may calculate in the natural way, free of
concern. The reader interested in the details is referred to the
appendix \ref{appendix:rho_details}.

\subsection{Substitution}

We use $\Proc$ for the set of processes, $\QProc$ for the set of
names, and $\id{\{}\vec{y} / \vec{x} \id{\}}$ to denote partial maps,
$s : \QProc \rightarrow \QProc$. A map, $s$ lifts, uniquely, to a map
on process terms, $\widehat{s} : \Proc \rightarrow \Proc$ by the
following equations.

\begin{mathpar}
  (0) \psubstp{Q}{P} := 0 \\
  (R \juxtap S) \psubstp{Q}{P}
  :=    
  (R)\psubstp{Q}{P} \juxtap (S) \psubstp{Q}{P} \\
  (x?(y).R) \psubstp{Q}{P}    
  :=    
  (x)\substp{Q}{P} (z)\concat( (R \psubstn{z}{y}) \psubstp{Q}{P} ) \\
  (\lift{x}{R}) \psubstp{Q}{P}  
  :=
  \lift{(x)\substp{Q}{P}}{ R \psubstp{Q}{P} } \\
%   (\dropn{x})  \psubstp{Q}{P}       
%   := 
%   \left\{ 
%     \begin{array}{ccc} 
%       \dropn{\quotep{Q}} & & x \nameeq \quotep{P} \\
%       \dropn{x} & & otherwise \\
%     \end{array}
%   \right. 
  (\dropn{x})  \psubstp{Q}{P}       
  := 
  \left\{ 
    \begin{array}{ccc} 
      Q & & x \nameeq \quotep{P} \\
      \dropn{x} & & otherwise \\
    \end{array}
  \right.
\end{mathpar}
 

where

\begin{eqnarray}
  (x)\id{\{} \lpquote Q \rpquote / \lpquote P \rpquote \id{\}}            = 
  \left\{ 
    \begin{array}{ccc}
      \lpquote Q \rpquote & & x \nameeq \lpquote P \rpquote \\
      x & & otherwise \\
    \end{array}
  \right. \nonumber
\end{eqnarray}

and $z$ is chosen distinct from $\quotep{P}$, $\quotep{Q}$, the free
names in $Q$, and all the names in $R$. Our $\alpha$-equivalence will
be built in the standard way from this substitution.

\begin{remark}\label{rem:no_self_referential_names}
  One consequence of these definitions is that $\forall P. \quotep{P}
  \not\in \freenames{P}$.
\end{remark}

\subsection{ Dynamic quote: an example }

Anticipating something of what's to come, consider applying the
substitution, $\widehat{\id{\{}u / z \id{\}}}$, to the following pair
of processes, $\lift{w}{y!(z)}$ and $w[ \lpquote y!(z) \rpquote ]$.

\begin{eqnarray}
	\lift{w}{y!(z)}\widehat{\id{\{}u / z \id{\}}}
		& = &
		\lift{w}{y!(u)} \nonumber\\
	w[ \lpquote y!(z) \rpquote ] \widehat{ \id{\{}u / z \id{\}} }
		& = &
		w[ \lpquote y!(z) \rpquote ] \nonumber
\end{eqnarray}

Because the body of the process between quotes is impervious to
substitution, we get radically different answers. In fact, by
examining the first process in an input context,
e.g. $x?(z).\lift{w}{y!(z)}$, we see that the process under the lift
operator may be shaped by prefixed inputs binding a name inside it. In
this sense, the lift operator will be seen as a way to dynamically
construct processes before reifying them as names.

Finally equipped with these standard features we can present the
dynamics of the calculus.

\subsubsection{Operational semantics} 

Finally, we introduce the computational dynamics. What marks these
algebras as distinct from other more traditionally studied algebraic
structures, e.g. vector spaces or polynomial rings, is the manner in
which dynamics is captured. In traditional structures, dynamics is typically
expressed through morphisms between such structures, as in linear maps
between vector spaces or morphisms between rings. In algebras
associated with the semantics of computation, the dynamics is
expressed as part of the algebraic structure itself, through a
reduction reduction relation typically denoted by $\red$. Below, we
give a recursive presentation of this relation for the calculus used
in the encoding.

$\red \subseteq \pi \times \pi$
$\red : \pi \to \mathcal{P}(\pi)$

\begin{mathpar}
  \inferrule* [lab=Comm] { \textsf{match}( x_{src}, x_{trgt} ) } { x_{trgt}?(y)P \; | \; x_{src}!\langle {Q} \rangle \red P\{\quotep{Q}/y}\} }
  \and \\
  \inferrule* [lab=Par] {{P} \red {P}'} {{{P} | {Q}} \red {{P}' | {Q}}}
  \and
  \inferrule* [lab=Equiv]{{{P} \scong {P}'} \andalso {{P}' \red {Q}'} \andalso {{Q}' \scong {Q}}}{{P} \red {Q}}
\end{mathpar}

\begin{eqnarray*}
  match_{\equiv} (\quotep{P},\quotep{Q}) & := & P \equiv Q \\
  match_{\dagger}(\quotep{P},\quotep{Q}) & := & \forall R. P|Q \red^{*} R => R \red^{*} 0 \\
  match_{K}(\quotep{P},\quotep{Q}) & := & K \mbox{ for some context } K
\end{eqnarray*}

$u?(x)P | u!\langle Q \rangle \red P\{\quotep{Q}/x\}$

%We write $\wred$ for $\red^*$, and $P\red$ if $\exists Q $ such that $ P \red Q$.
We write $P\red$ if $\exists Q $ such that $ P \red Q$ and $P\not\red$, otherwise.

\section{Replication}

As mentioned before, it is known that replication (and hence
recursion) can be implemented in a higher-order process algebra
\cite{SangiorgiWalker}. As our first example of calculation with the
machinery thus far presented we give the construction explicitly in
the {\rhoc}.

\begin{eqnarray}
	D_{x} & := & \prefix{x}{y}{(\binpar{\outputp{x}{y}}{@{y}})} \nonumber\\
	\bangp_{x}{P} & := & \binpar{{x}!\langle{\binpar{D_{x}}{P}}\rangle}{D_{x}} \nonumber
\end{eqnarray}

\begin{eqnarray}
	\bangp_{x}{P} & & \nonumber\\
	=
	& {x}!\langle{(\prefix{x}{y}{(\outputp{x}{y} | @{y})) | P}}\rangle 
	      | \prefix{x}{y}{(\outputp{x}{y} | @{y})} & \nonumber\\
	\red
	& (\outputp{x}{y} | @{y})\substn{\quotep{(\prefix{x}{y}{(@{y} | \outputp{x}{y})) | P}}}{y} & \nonumber\\
	=
	& \outputp{x}{\quotep{(\prefix{x}{y}{(\outputp{x}{y} | @{y})) | P}}}
	  | {(\prefix{x}{y}{(\outputp{x}{y} | @{y})) | P}} & \nonumber\\
	\red
	& \ldots & \nonumber\\
	\red^*
	& P | P | \ldots & \nonumber
\end{eqnarray}

Of course, this encoding, as an implementation, runs away, unfolding
$\bangp{P}$ eagerly. A lazier and more implementable replication
operator, restricted to input-guarded processes, may be obtained as follows.

\begin{eqnarray}
\bangp{\prefix{u}{v}{P}} 
	:= 
	\binpar{\lift{x}{\prefix{u}{v}{(\binpar{D(x)}{P})}}}{D(x)} \nonumber
\end{eqnarray}

\begin{remark}
  Note that the lazier definition still does not deal with summation
  or mixed summation (i.e. sums over input and output). The reader is
  invited to construct definitions of replication that deal with these
  features. 

  Further, the definitions are parameterized in a name, $x$. Can you,
  gentle reader, make a definition that eliminates this parameter and
  guarantees no accidental interaction between the replication
  machinery and the process being replicated -- i.e. no accidental
  sharing of names used by the process to get its work done and the
  name(s) used by the replication to effect copying. This latter
  revision of the definition of replication is crucial to obtaining
  the expected identity $!!P \sim !P$.
\end{remark}

\begin{remark}\label{rem:paradoxical_combinator}
  The reader familiar with the lambda calculus will have noticed the
  similarity between $D$ and the paradoxical combinator.

  [Ed. note: the existence of this seems to suggest we have to be more
  restrictive on the set of processes and names we admit if we are to
  support no-cloning.]
\end{remark}

\subsubsection{Bisimulation}

The computational dynamics gives rise to another kind of equivalence,
the equivalence of computational behavior. As previously mentioned
this is typically captured \emph{via} some form of bisimulation.

% The notion we use in this paper is weak barbed bisimulation
% \cite{milner91polyadicpi}.

The notion we use in this paper is derived from weak barbed
bisimulation \cite{milner91polyadicpi}. 

\begin{definition}
An \emph{observation relation}, $\downarrow_{\mathcal N}$, over a set
of names, $\mathcal N$, is the smallest relation satisfying the rules
below.

\infrule[Out-barb]{y \in {\mathcal N}, \; x \nameeq y}
		  {\outputp{x}{v} \downarrow_{\mathcal N} x}
\infrule[Par-barb]{\mbox{$P\downarrow_{\mathcal N} x$ or $Q\downarrow_{\mathcal N} x$}}
		  {\binpar{P}{Q} \downarrow_{\mathcal N} x}

We write $P \Downarrow_{\mathcal N} x$ if there is $Q$ such that 
$P \wred Q$ and $Q \downarrow_{\mathcal N} x$.
\end{definition}

\begin{definition}
%\label{def.bbisim}
An  ${\mathcal N}$-\emph{barbed bisimulation} over a set of names, ${\mathcal N}$, is a symmetric binary relation 
${\mathcal S}_{\mathcal N}$ between agents such that $P\rel{S}_{\mathcal N}Q$ implies:
\begin{enumerate}
\item If $P \red P'$ then $Q \wred Q'$ and $P'\rel{S}_{\mathcal N} Q'$.
\item If $P\downarrow_{\mathcal N} x$, then $Q\Downarrow_{\mathcal N} x$.
\end{enumerate}
$P$ is ${\mathcal N}$-barbed bisimilar to $Q$, written
$P \wbbisim_{\mathcal N} Q$, if $P \rel{S}_{\mathcal N} Q$ for some ${\mathcal N}$-barbed bisimulation ${\mathcal S}_{\mathcal N}$.
\end{definition}

$\mathcal{R} \subseteq \pi \times \pi$

$P \mathcal{R} Q => \forall P'. P \red P' \Rightarrow \exists Q'. Q \red Q', P' \mathcal{R} Q'$

$P \vdash x \Rightarrow Q \vdash x$

\begin{mathpar}
  \inferrule*[lab=Out-barb]{x \nameeq y}{{y}!\langle{Q}\rangle \vdash x}
  \and
  \inferrule*[lab=Par-barb]{\mbox{$P\vdash x$ or $Q\vdash x$}}{\binpar{P}{Q} \vdash x}
\end{mathpar}

\subsubsection{Contexts}

One of the principle advantages of computational calculi like the
$\pi$-calculus is a well-defined notion of context,
contextual-equivalence and a correlation between
contextual-equivalence and notions of bisimulation. The notion of
context allows the decomposition of a process into (sub-)process and
its syntactic environment, its context. Thus, a context may be
thought of as a process with a ``hole'' (written $\Box$) in it. The
application of a context $M$ to a process $P$, written $M[P]$, is
tantamount to filling the hole in $M$ with $P$. In this paper we do
not need the full weight of this theory, but do make use of the notion
of context in the proof the main theorem. 

\begin{mathpar}
  \inferrule* [lab=summation] {} {{M_{M},M_{N}} \bc \Box \;|\; x.M_{A} \;|\; M_{M}+M_{N}}
  \and
  \inferrule* [lab=agent] {} {{M_{A}} \bc (\vec{x})M_{P} \;| \; \clift{P_0,\ldots,M_{P},\ldots,P_N}}
  \and \\
  \inferrule* [lab=process] {} {{M_{P}} \bc M_{N} \;| \;P|M_{P} }
\end{mathpar} 

\begin{mathpar}
  \inferrule* [lab=sychronization] {} {M_{N} \bc \Box \;|\; x?M_{F} \;|\; x!M_{C}}
  \and
  \inferrule* [lab=abstraction] {} {{M_{F}} \bc (x)M_{P} }
  \and
  \inferrule* [lab=concretion] {} {{M_{C}} \bc \langle M_{P} \rangle }
  \and \\
  \inferrule* [lab=process] {} {{M_{P}} \bc M_{N} \;| \;P|M_{P} }
\end{mathpar}

\begin{definition}[contextual application] Given a context $M$, and
  process $P$, we define the \emph{contextual application}, $M[P] :=
  M\{P/\Box\}$. That is, the contextual application of M to P is the
  substitution of $P$ for $\Box$ in $M$.
\end{definition}

$\meaningof{-} : L \to \mathcal{P}(\pi)$

\begin{mathpar}
  \inferrule* [lab=collection] {} {\meaningof{true} = \pi, \and \meaningof{~E} = \pi \setminus \meaningof{E}, \and \meaningof{E_{1} \& E_{2}} = \meaningof{E_{1}} \cap \meaningof{E_{2}}}
\end{mathpar}

\begin{mathpar}
  \inferrule* [lab=structure] {} {\meaningof{0} = \{ P \in \pi | P \equiv 0 \}, \and \\ \meaningof{E_1 | E_2} = \{ P \in \pi | P \equiv P_{1} | P_{2}, P_{1} \in \meaningof{E_{1}}, P_{2} \in \meaningof{E_2}\} }
\end{mathpar}

\begin{mathpar}
 \inferrule* [lab=behavior] {} {\meaningof{\langle a?b \rangle E} = \{ P \in \pi | P \equiv Q | u?(y)P', \\ \and \\\\ \and \\ \;\;\; u \in \meaningof{a}, \forall z.P'\{z/y\} \in \meaningof{E\{z/b\}}\}, \and \\ \meaningof{a!E} = \{ P \in \pi | P \equiv Q | x!\langle P' \rangle, x \in \meaningof{a} P' \in \meaningof{E}\} }
\end{mathpar}

\begin{mathpar}
 \inferrule* [lab=nominal] {} {\meaningof{\quotep{E}} = \{ \quotep{P} \in \quotep{\pi} | P \in \meaningof{E} \}, \and \meaningof{\quotep{P}} = \{ \quotep{Q} \in \quotep{\pi} | P \equiv Q \} \and \\ \meaningof{@\quotep{E}} = \{ P \in \pi | P \equiv @x, x \in \meaningof{E} \}}
\end{mathpar}

\begin{eqnarray*}
  \\
  \meaningof{-} : TS \to ST
\end{eqnarray*}

\begin{eqnarray*}
  \\
  L : TS \to ST
\end{eqnarray*}

\begin{eqnarray*}
  \\
  P \models E \iff P \in \meaningof{E}
\end{eqnarray*}

\begin{eqnarray*}
  P \approx_{L} Q \iff \forall E \in L. P \models E \iff Q \models E
\end{eqnarray*}

\begin{eqnarray*}
  P \approx_{K} Q
\end{eqnarray*}

\begin{eqnarray*}
  P \approx Q
\end{eqnarray*}

$\approx_{K} = \approx = \approx_{L}$

\subsubsection{Contextual duality}

Note that contexts extend the quotation operation to a family of
operations from processes to names. Given a context, $M$, we can
define a \emph{nominal context}, $\quotep{M}$ by $\quotep{M}[P] :=
\quotep{M[P]}$. To foreshadow what is to come we observe that these
operations enjoy a duality with processes very much like the duality
between vectors and maps from vectors to scalars.

Further, because the calculus is essentially higher-order, we have a
correspondence between contexts and processes. More specifically,
given a name $x$ and a context $M$ we can construct $M^{*}_{x}$ such
that 

\begin{mathpar}
  M^{*}_{x} | \lift{x}{P} \red M[P]
\end{mathpar}

namely,

\begin{mathpar}
  M^{*}_{x} := x?(u).M[\dropn{u}]
\end{mathpar}

The dependence of $M^{*}_{x}$ on a name makes it an abstraction, 

\begin{mathpar}
  M^{*} := (x)x?(u).M[\dropn{u}]
\end{mathpar}

\subsection{Additional notation}

It will sometimes be convenient to denote the process a name
quotes. We already have the notation $x = \quotep{P}$, but it will be
convenient to introduce an alternate notation, $\procn{x}$, when we
want to emphasize the connection to the use of the name. Note that, by
virtue of name equivalence, $\quotep{\procn{x}} \nameeq x$; so, the
notation is consistent with previous definitions.

Further, because names have structure it is possible to effect
substitutions on the basis of that structure. This means we need to
upgrade our notation for substitutions, which we accomplish by
adapting comprehension notation. Thus,

\begin{mathpar}
  P\{ y / x : x \in S \}
\end{mathpar}

is interpreted to mean the process derived from P by replacing (in a
capture-avoiding manner) each occurrence of $x$ in $S$ by $y$. For example,

\begin{mathpar}
  P\{ \quotep{\procn{x}|\procn{x}} / x : x \in \freenames{P} \}
\end{mathpar}

will replace each (occurrence) of a free name $x$ in $P$ by
$\quotep{\procn{x}|\procn{x}}$.

Also, we will avail ourselves of the notation $x^{L}$ and $x^{R}$ to
denote injections of a name into disjoint copies of the name
space. There are numerous ways to accomplish this. One example can be
found in \cite{MeredithR05}. This notation overloads to vectors of
names: $\vec{x}^{\pi} := (x_{i}^{\pi} \; : \; 0 \leq i < |\vec{x}| )$ where $\pi \in \{L,R\}$.

We also use $P^{\Box} := P|\Box$.

In \cite{MeredithR05} an interpretation of the new operator is
given. It turns out that there are several possible interpretations
all enjoying the requisite algebraic properties of the operator (see
\cite{milner91polyadicpi}). We will therefore make liberal use of
$(\nu\; \vec{x})P$.

% subsection the_syntax_and_semantics_of_the_notation_system (end)   

\input{qm2pi.qmops} 

\input{qm2pi.sterngerlach} 

\input{qm2pi.metric} 

% section concurrent_process_calculi (end)

%\input{qm2pi.proofsketch}

% section proof sketch (end)

%\input{qm2pi.slviaknots} 

% section spatial logic via knots (end)

\input{qm2pi.conclusion}

% section conclusion (end)

%\input{qm2pi.dtcodes} 

% section wiring algorithm (end)

\input{qm2pi.ack} 

% section acknowledgments (end)

\newpage


\bibliographystyle{plain}   
\bibliography{../../biblios/main.bib}

\input{qm2pi.rhodetails}

\end{document}

 

% section acknowledgments (end)

\newpage


\bibliographystyle{plain}   
\bibliography{../../biblios/main.bib}

\documentclass[12pt]{llncs}
%\documentclass{jktr}

\usepackage[pdftex]{hyperref}                   
\usepackage {listings}
\usepackage {mathpartir}
\usepackage{bcprules}
%\usepackage{listings}
                       
\usepackage{graphicx} 
%\usepackage[margins=2.5cm,nohead,nofoot]{geometry}
%\usepackage{geometry}
\usepackage{amsfonts}
\usepackage{amstext}
\usepackage{latexsym}
\usepackage{amssymb}
\usepackage{color}


%\include{myPreamble}
\include{qm2pi.local} 

%\ifpdf
%\usepackage[pdftex]{graphicx}
%\else
%\usepackage{graphicx}
%\fi

 % \ifpdf
%  \usepackage{pdfsync}
%  \if


%\title{Brief Article}
%\author{David F. Snyder}
%\author{L.G. Meredith}

%\address{Dept. of Math., Texas State University--San Marcos, San Marcos, TX 78666}
       
\pagestyle{empty}


\begin{document}

\lstset{language=[Objective]Caml,frame=shadowbox}

\input{qm2pi.front}

% section front matter (end)

\input{qm2pi.intro} 
 
% section introduction (end)

% \input{qm2pi.knotations} 

% section notation (end)

\input{qm2pi.process.calculi} 

% section concurrent_process_calculi_and_spatial_logics_ (end)
    
%\input{qm2pi.knots2pi} 

%\input{qm2pi.trefoil} 

%\input{qm2pi.mainthm} 

% subsection basic_interpretation (end)

%\input{qm2pi.rho.presentation} 
\subsection{The syntax and semantics of the notation system}\label{sub:the_syntax_and_semantics_of_the_notation_system} % (fold)

We now summarize a technical presentation of the calculus that
embodies our theory of dynamics. The typical presentation of such a
calculus follows the style of giving generators and relations on
them. The grammar, below, describing term constructors, freely
generates the set of processes, $\Proc$. This set is then quotiented
by a relation known as structural congruence and it is over this set
that the notion of dynamics is expressed. This presentation is
essentially that of \cite{MeredithR05} with the addition of
polyadicity and summation. For readability we have relegated some of
the technical subtleties to an appendix.

\subsubsection{Process grammar}\label{subsub:process_grammar}

\begin{mathpar}
  \inferrule* [lab=synchronization] {} {{M} \bc \pzero \;|\; x?F \;|\; x!C }
  \and
  \inferrule* [lab=abstraction] {} {{F} \bc (x)P}
  \and
  \inferrule* [lab=concretion] {} {{C} \bc \langle Q \rangle}
  \and
  \inferrule* [lab=process] {} {{P,Q} \bc M \;| \;P|Q \;|\; @{x}}
  \and
  \inferrule* [lab=name] {} {{x} \bc \quotep{P}}
\end{mathpar} 

Note that $\vec{x}$ (resp. $\vec{P}$) denotes a vector of names
(resp. processes) of length $|\vec{x}|$ (resp. $|\vec{P}|$). We adopt
the following useful abbreviations.

\begin{mathpar}
   x?(\vec{y}).P := x.(\vec{y})P \and  x\clift{\vec{P}} := x.\clift{\vec{P}}
   \and x!(y) := \lift{x}{\dropn{y}}
   \and \Pi_{i=0}^{n-1}P_i := P_0 | \ldots | P_{n-1}
\end{mathpar}

\subsubsection{Structural congruence}

\paragraph{Free and bound names and alpha-equivalence.} At the
core of structural equivalence is alpha-equivalence which identifies
process that are the same up to a change of variable. Formally, we
recognize the distinction between free and bound names. The free names
of a process, $\freenames{P}$, may be calculated recursively as
follows:

\begin{mathpar}
\freenames{\pzero} := \emptyset
  \and \\
  \freenames{x?(y).P} := \{ x \} \cup (\freenames{P} \setminus \{ y \})
  \and 
  \freenames{x!\langle P \rangle} := \{ x \} \cup \{ P \} 
  \and \\
  \freenames{P|Q} := \freenames{P} \cup \freenames{Q}
  \and \\
  \freenames{@{x}} := \{ x \}
\end{mathpar}

$\pi$
$\quotep{\pi}$

$\freenames{-} : \pi \to \mathcal{P}(\quotep{\pi})$

\begin{eqnarray*}
  \freenames{\pzero} & := & \emptyset \\
  \freenames{x?(y).P} & := & \{ x \} \cup (\freenames{P} \setminus \{ y \}) \\
  \freenames{x!\langle P \rangle} & := & \{ x \} \cup \{ P \} \\
  \freenames{P|Q} & := & \freenames{P} \cup \freenames{Q} \\
  \freenames{\dropn{x}} & := & \{ x \}
\end{eqnarray*}

The bound names of a process, $\boundnames{P}$, are those names occurring in $P$
that are not free. For example, in $x?(y).0$, the name $x$ is free, while $y$ is bound.

\begin{mathpar}
  \inferrule* [lab=monoidal-laws] {} { P|Q \equiv Q|P \and P|0 \equiv P \and P|(Q|R) \equiv (P|Q)|R }
\end{mathpar}

\begin{mathpar}
  \inferrule* [lab=alpha-equivalence] {} { (x)P \equiv (y)P\{y/x\} \and y \not\in \freenames{P} }
\end{mathpar}

\begin{definition}
Then two processes, $P,Q$, are alpha-equivalent if $P = Q\{\vec{y}/\vec{x}\}$ for
some $\vec{x} \in \boundnames{Q},\vec{y} \in \boundnames{P}$, where $Q\{\vec{y}/\vec{x}\}$
denotes the capture-avoiding substitution of $\vec{y}$ for $\vec{x}$ in $Q$.
\end{definition}

\begin{definition}
  The {\em structural congruence} \cite{SangiorgiWalker} , $\equiv$,
  between processes is the least congruence containing
  alpha-equivalence, satisfying the abelian monoid laws
  (associativity, commutativity and $\pzero$ as identity) for parallel
  composition $|$ and for summation $+$.
\end{definition}

\subsection{Name equivalence}

We take name equivalence, written $\nameeq$, to be the smallest
equivalence relation generated by the following rules.

\begin{mathpar}
\inferrule*[lab=Quote-drop]
{ }
{ \quotep{@{x}} \nameeq x }

\inferrule*[lab=Struct-equiv]
{ P \scong Q }
{ \quotep{P} \nameeq \quotep{Q} }
\end{mathpar}

The astute reader will have noticed that the mutual recursion of names
and processes imposes a mutual recursion on alpha-equivalence and
structural equivalence via name-equivalence. Fortunately, all of this
works out pleasantly and we may calculate in the natural way, free of
concern. The reader interested in the details is referred to the
appendix \ref{appendix:rho_details}.

\subsection{Substitution}

We use $\Proc$ for the set of processes, $\QProc$ for the set of
names, and $\id{\{}\vec{y} / \vec{x} \id{\}}$ to denote partial maps,
$s : \QProc \rightarrow \QProc$. A map, $s$ lifts, uniquely, to a map
on process terms, $\widehat{s} : \Proc \rightarrow \Proc$ by the
following equations.

\begin{mathpar}
  (0) \psubstp{Q}{P} := 0 \\
  (R \juxtap S) \psubstp{Q}{P}
  :=    
  (R)\psubstp{Q}{P} \juxtap (S) \psubstp{Q}{P} \\
  (x?(y).R) \psubstp{Q}{P}    
  :=    
  (x)\substp{Q}{P} (z)\concat( (R \psubstn{z}{y}) \psubstp{Q}{P} ) \\
  (\lift{x}{R}) \psubstp{Q}{P}  
  :=
  \lift{(x)\substp{Q}{P}}{ R \psubstp{Q}{P} } \\
%   (\dropn{x})  \psubstp{Q}{P}       
%   := 
%   \left\{ 
%     \begin{array}{ccc} 
%       \dropn{\quotep{Q}} & & x \nameeq \quotep{P} \\
%       \dropn{x} & & otherwise \\
%     \end{array}
%   \right. 
  (\dropn{x})  \psubstp{Q}{P}       
  := 
  \left\{ 
    \begin{array}{ccc} 
      Q & & x \nameeq \quotep{P} \\
      \dropn{x} & & otherwise \\
    \end{array}
  \right.
\end{mathpar}
 

where

\begin{eqnarray}
  (x)\id{\{} \lpquote Q \rpquote / \lpquote P \rpquote \id{\}}            = 
  \left\{ 
    \begin{array}{ccc}
      \lpquote Q \rpquote & & x \nameeq \lpquote P \rpquote \\
      x & & otherwise \\
    \end{array}
  \right. \nonumber
\end{eqnarray}

and $z$ is chosen distinct from $\quotep{P}$, $\quotep{Q}$, the free
names in $Q$, and all the names in $R$. Our $\alpha$-equivalence will
be built in the standard way from this substitution.

\begin{remark}\label{rem:no_self_referential_names}
  One consequence of these definitions is that $\forall P. \quotep{P}
  \not\in \freenames{P}$.
\end{remark}

\subsection{ Dynamic quote: an example }

Anticipating something of what's to come, consider applying the
substitution, $\widehat{\id{\{}u / z \id{\}}}$, to the following pair
of processes, $\lift{w}{y!(z)}$ and $w[ \lpquote y!(z) \rpquote ]$.

\begin{eqnarray}
	\lift{w}{y!(z)}\widehat{\id{\{}u / z \id{\}}}
		& = &
		\lift{w}{y!(u)} \nonumber\\
	w[ \lpquote y!(z) \rpquote ] \widehat{ \id{\{}u / z \id{\}} }
		& = &
		w[ \lpquote y!(z) \rpquote ] \nonumber
\end{eqnarray}

Because the body of the process between quotes is impervious to
substitution, we get radically different answers. In fact, by
examining the first process in an input context,
e.g. $x?(z).\lift{w}{y!(z)}$, we see that the process under the lift
operator may be shaped by prefixed inputs binding a name inside it. In
this sense, the lift operator will be seen as a way to dynamically
construct processes before reifying them as names.

Finally equipped with these standard features we can present the
dynamics of the calculus.

\subsubsection{Operational semantics} 

Finally, we introduce the computational dynamics. What marks these
algebras as distinct from other more traditionally studied algebraic
structures, e.g. vector spaces or polynomial rings, is the manner in
which dynamics is captured. In traditional structures, dynamics is typically
expressed through morphisms between such structures, as in linear maps
between vector spaces or morphisms between rings. In algebras
associated with the semantics of computation, the dynamics is
expressed as part of the algebraic structure itself, through a
reduction reduction relation typically denoted by $\red$. Below, we
give a recursive presentation of this relation for the calculus used
in the encoding.

$\red \subseteq \pi \times \pi$
$\red : \pi \to \mathcal{P}(\pi)$

\begin{mathpar}
  \inferrule* [lab=Comm] { \textsf{match}( x_{src}, x_{trgt} ) } { x_{trgt}?(y)P \; | \; x_{src}!\langle {Q} \rangle \red P\{\quotep{Q}/y}\} }
  \and \\
  \inferrule* [lab=Par] {{P} \red {P}'} {{{P} | {Q}} \red {{P}' | {Q}}}
  \and
  \inferrule* [lab=Equiv]{{{P} \scong {P}'} \andalso {{P}' \red {Q}'} \andalso {{Q}' \scong {Q}}}{{P} \red {Q}}
\end{mathpar}

\begin{eqnarray*}
  match_{\equiv} (\quotep{P},\quotep{Q}) & := & P \equiv Q \\
  match_{\dagger}(\quotep{P},\quotep{Q}) & := & \forall R. P|Q \red^{*} R => R \red^{*} 0 \\
  match_{K}(\quotep{P},\quotep{Q}) & := & K \mbox{ for some context } K
\end{eqnarray*}

$u?(x)P | u!\langle Q \rangle \red P\{\quotep{Q}/x\}$

%We write $\wred$ for $\red^*$, and $P\red$ if $\exists Q $ such that $ P \red Q$.
We write $P\red$ if $\exists Q $ such that $ P \red Q$ and $P\not\red$, otherwise.

\section{Replication}

As mentioned before, it is known that replication (and hence
recursion) can be implemented in a higher-order process algebra
\cite{SangiorgiWalker}. As our first example of calculation with the
machinery thus far presented we give the construction explicitly in
the {\rhoc}.

\begin{eqnarray}
	D_{x} & := & \prefix{x}{y}{(\binpar{\outputp{x}{y}}{@{y}})} \nonumber\\
	\bangp_{x}{P} & := & \binpar{{x}!\langle{\binpar{D_{x}}{P}}\rangle}{D_{x}} \nonumber
\end{eqnarray}

\begin{eqnarray}
	\bangp_{x}{P} & & \nonumber\\
	=
	& {x}!\langle{(\prefix{x}{y}{(\outputp{x}{y} | @{y})) | P}}\rangle 
	      | \prefix{x}{y}{(\outputp{x}{y} | @{y})} & \nonumber\\
	\red
	& (\outputp{x}{y} | @{y})\substn{\quotep{(\prefix{x}{y}{(@{y} | \outputp{x}{y})) | P}}}{y} & \nonumber\\
	=
	& \outputp{x}{\quotep{(\prefix{x}{y}{(\outputp{x}{y} | @{y})) | P}}}
	  | {(\prefix{x}{y}{(\outputp{x}{y} | @{y})) | P}} & \nonumber\\
	\red
	& \ldots & \nonumber\\
	\red^*
	& P | P | \ldots & \nonumber
\end{eqnarray}

Of course, this encoding, as an implementation, runs away, unfolding
$\bangp{P}$ eagerly. A lazier and more implementable replication
operator, restricted to input-guarded processes, may be obtained as follows.

\begin{eqnarray}
\bangp{\prefix{u}{v}{P}} 
	:= 
	\binpar{\lift{x}{\prefix{u}{v}{(\binpar{D(x)}{P})}}}{D(x)} \nonumber
\end{eqnarray}

\begin{remark}
  Note that the lazier definition still does not deal with summation
  or mixed summation (i.e. sums over input and output). The reader is
  invited to construct definitions of replication that deal with these
  features. 

  Further, the definitions are parameterized in a name, $x$. Can you,
  gentle reader, make a definition that eliminates this parameter and
  guarantees no accidental interaction between the replication
  machinery and the process being replicated -- i.e. no accidental
  sharing of names used by the process to get its work done and the
  name(s) used by the replication to effect copying. This latter
  revision of the definition of replication is crucial to obtaining
  the expected identity $!!P \sim !P$.
\end{remark}

\begin{remark}\label{rem:paradoxical_combinator}
  The reader familiar with the lambda calculus will have noticed the
  similarity between $D$ and the paradoxical combinator.

  [Ed. note: the existence of this seems to suggest we have to be more
  restrictive on the set of processes and names we admit if we are to
  support no-cloning.]
\end{remark}

\subsubsection{Bisimulation}

The computational dynamics gives rise to another kind of equivalence,
the equivalence of computational behavior. As previously mentioned
this is typically captured \emph{via} some form of bisimulation.

% The notion we use in this paper is weak barbed bisimulation
% \cite{milner91polyadicpi}.

The notion we use in this paper is derived from weak barbed
bisimulation \cite{milner91polyadicpi}. 

\begin{definition}
An \emph{observation relation}, $\downarrow_{\mathcal N}$, over a set
of names, $\mathcal N$, is the smallest relation satisfying the rules
below.

\infrule[Out-barb]{y \in {\mathcal N}, \; x \nameeq y}
		  {\outputp{x}{v} \downarrow_{\mathcal N} x}
\infrule[Par-barb]{\mbox{$P\downarrow_{\mathcal N} x$ or $Q\downarrow_{\mathcal N} x$}}
		  {\binpar{P}{Q} \downarrow_{\mathcal N} x}

We write $P \Downarrow_{\mathcal N} x$ if there is $Q$ such that 
$P \wred Q$ and $Q \downarrow_{\mathcal N} x$.
\end{definition}

\begin{definition}
%\label{def.bbisim}
An  ${\mathcal N}$-\emph{barbed bisimulation} over a set of names, ${\mathcal N}$, is a symmetric binary relation 
${\mathcal S}_{\mathcal N}$ between agents such that $P\rel{S}_{\mathcal N}Q$ implies:
\begin{enumerate}
\item If $P \red P'$ then $Q \wred Q'$ and $P'\rel{S}_{\mathcal N} Q'$.
\item If $P\downarrow_{\mathcal N} x$, then $Q\Downarrow_{\mathcal N} x$.
\end{enumerate}
$P$ is ${\mathcal N}$-barbed bisimilar to $Q$, written
$P \wbbisim_{\mathcal N} Q$, if $P \rel{S}_{\mathcal N} Q$ for some ${\mathcal N}$-barbed bisimulation ${\mathcal S}_{\mathcal N}$.
\end{definition}

$\mathcal{R} \subseteq \pi \times \pi$

$P \mathcal{R} Q => \forall P'. P \red P' \Rightarrow \exists Q'. Q \red Q', P' \mathcal{R} Q'$

$P \vdash x \Rightarrow Q \vdash x$

\begin{mathpar}
  \inferrule*[lab=Out-barb]{x \nameeq y}{{y}!\langle{Q}\rangle \vdash x}
  \and
  \inferrule*[lab=Par-barb]{\mbox{$P\vdash x$ or $Q\vdash x$}}{\binpar{P}{Q} \vdash x}
\end{mathpar}

\subsubsection{Contexts}

One of the principle advantages of computational calculi like the
$\pi$-calculus is a well-defined notion of context,
contextual-equivalence and a correlation between
contextual-equivalence and notions of bisimulation. The notion of
context allows the decomposition of a process into (sub-)process and
its syntactic environment, its context. Thus, a context may be
thought of as a process with a ``hole'' (written $\Box$) in it. The
application of a context $M$ to a process $P$, written $M[P]$, is
tantamount to filling the hole in $M$ with $P$. In this paper we do
not need the full weight of this theory, but do make use of the notion
of context in the proof the main theorem. 

\begin{mathpar}
  \inferrule* [lab=summation] {} {{M_{M},M_{N}} \bc \Box \;|\; x.M_{A} \;|\; M_{M}+M_{N}}
  \and
  \inferrule* [lab=agent] {} {{M_{A}} \bc (\vec{x})M_{P} \;| \; \clift{P_0,\ldots,M_{P},\ldots,P_N}}
  \and \\
  \inferrule* [lab=process] {} {{M_{P}} \bc M_{N} \;| \;P|M_{P} }
\end{mathpar} 

\begin{mathpar}
  \inferrule* [lab=sychronization] {} {M_{N} \bc \Box \;|\; x?M_{F} \;|\; x!M_{C}}
  \and
  \inferrule* [lab=abstraction] {} {{M_{F}} \bc (x)M_{P} }
  \and
  \inferrule* [lab=concretion] {} {{M_{C}} \bc \langle M_{P} \rangle }
  \and \\
  \inferrule* [lab=process] {} {{M_{P}} \bc M_{N} \;| \;P|M_{P} }
\end{mathpar}

\begin{definition}[contextual application] Given a context $M$, and
  process $P$, we define the \emph{contextual application}, $M[P] :=
  M\{P/\Box\}$. That is, the contextual application of M to P is the
  substitution of $P$ for $\Box$ in $M$.
\end{definition}

$\meaningof{-} : L \to \mathcal{P}(\pi)$

\begin{mathpar}
  \inferrule* [lab=collection] {} {\meaningof{true} = \pi, \and \meaningof{~E} = \pi \setminus \meaningof{E}, \and \meaningof{E_{1} \& E_{2}} = \meaningof{E_{1}} \cap \meaningof{E_{2}}}
\end{mathpar}

\begin{mathpar}
  \inferrule* [lab=structure] {} {\meaningof{0} = \{ P \in \pi | P \equiv 0 \}, \and \\ \meaningof{E_1 | E_2} = \{ P \in \pi | P \equiv P_{1} | P_{2}, P_{1} \in \meaningof{E_{1}}, P_{2} \in \meaningof{E_2}\} }
\end{mathpar}

\begin{mathpar}
 \inferrule* [lab=behavior] {} {\meaningof{\langle a?b \rangle E} = \{ P \in \pi | P \equiv Q | u?(y)P', \\ \and \\\\ \and \\ \;\;\; u \in \meaningof{a}, \forall z.P'\{z/y\} \in \meaningof{E\{z/b\}}\}, \and \\ \meaningof{a!E} = \{ P \in \pi | P \equiv Q | x!\langle P' \rangle, x \in \meaningof{a} P' \in \meaningof{E}\} }
\end{mathpar}

\begin{mathpar}
 \inferrule* [lab=nominal] {} {\meaningof{\quotep{E}} = \{ \quotep{P} \in \quotep{\pi} | P \in \meaningof{E} \}, \and \meaningof{\quotep{P}} = \{ \quotep{Q} \in \quotep{\pi} | P \equiv Q \} \and \\ \meaningof{@\quotep{E}} = \{ P \in \pi | P \equiv @x, x \in \meaningof{E} \}}
\end{mathpar}

\begin{eqnarray*}
  \\
  \meaningof{-} : TS \to ST
\end{eqnarray*}

\begin{eqnarray*}
  \\
  L : TS \to ST
\end{eqnarray*}

\begin{eqnarray*}
  \\
  P \models E \iff P \in \meaningof{E}
\end{eqnarray*}

\begin{eqnarray*}
  P \approx_{L} Q \iff \forall E \in L. P \models E \iff Q \models E
\end{eqnarray*}

\begin{eqnarray*}
  P \approx_{K} Q
\end{eqnarray*}

\begin{eqnarray*}
  P \approx Q
\end{eqnarray*}

$\approx_{K} = \approx = \approx_{L}$

\subsubsection{Contextual duality}

Note that contexts extend the quotation operation to a family of
operations from processes to names. Given a context, $M$, we can
define a \emph{nominal context}, $\quotep{M}$ by $\quotep{M}[P] :=
\quotep{M[P]}$. To foreshadow what is to come we observe that these
operations enjoy a duality with processes very much like the duality
between vectors and maps from vectors to scalars.

Further, because the calculus is essentially higher-order, we have a
correspondence between contexts and processes. More specifically,
given a name $x$ and a context $M$ we can construct $M^{*}_{x}$ such
that 

\begin{mathpar}
  M^{*}_{x} | \lift{x}{P} \red M[P]
\end{mathpar}

namely,

\begin{mathpar}
  M^{*}_{x} := x?(u).M[\dropn{u}]
\end{mathpar}

The dependence of $M^{*}_{x}$ on a name makes it an abstraction, 

\begin{mathpar}
  M^{*} := (x)x?(u).M[\dropn{u}]
\end{mathpar}

\subsection{Additional notation}

It will sometimes be convenient to denote the process a name
quotes. We already have the notation $x = \quotep{P}$, but it will be
convenient to introduce an alternate notation, $\procn{x}$, when we
want to emphasize the connection to the use of the name. Note that, by
virtue of name equivalence, $\quotep{\procn{x}} \nameeq x$; so, the
notation is consistent with previous definitions.

Further, because names have structure it is possible to effect
substitutions on the basis of that structure. This means we need to
upgrade our notation for substitutions, which we accomplish by
adapting comprehension notation. Thus,

\begin{mathpar}
  P\{ y / x : x \in S \}
\end{mathpar}

is interpreted to mean the process derived from P by replacing (in a
capture-avoiding manner) each occurrence of $x$ in $S$ by $y$. For example,

\begin{mathpar}
  P\{ \quotep{\procn{x}|\procn{x}} / x : x \in \freenames{P} \}
\end{mathpar}

will replace each (occurrence) of a free name $x$ in $P$ by
$\quotep{\procn{x}|\procn{x}}$.

Also, we will avail ourselves of the notation $x^{L}$ and $x^{R}$ to
denote injections of a name into disjoint copies of the name
space. There are numerous ways to accomplish this. One example can be
found in \cite{MeredithR05}. This notation overloads to vectors of
names: $\vec{x}^{\pi} := (x_{i}^{\pi} \; : \; 0 \leq i < |\vec{x}| )$ where $\pi \in \{L,R\}$.

We also use $P^{\Box} := P|\Box$.

In \cite{MeredithR05} an interpretation of the new operator is
given. It turns out that there are several possible interpretations
all enjoying the requisite algebraic properties of the operator (see
\cite{milner91polyadicpi}). We will therefore make liberal use of
$(\nu\; \vec{x})P$.

% subsection the_syntax_and_semantics_of_the_notation_system (end)   

\input{qm2pi.qmops} 

\input{qm2pi.sterngerlach} 

\input{qm2pi.metric} 

% section concurrent_process_calculi (end)

%\input{qm2pi.proofsketch}

% section proof sketch (end)

%\input{qm2pi.slviaknots} 

% section spatial logic via knots (end)

\input{qm2pi.conclusion}

% section conclusion (end)

%\input{qm2pi.dtcodes} 

% section wiring algorithm (end)

\input{qm2pi.ack} 

% section acknowledgments (end)

\newpage


\bibliographystyle{plain}   
\bibliography{../../biblios/main.bib}

\input{qm2pi.rhodetails}

\end{document}



\end{document}

 

%\documentclass[12pt]{llncs}
%\documentclass{jktr}

\usepackage[pdftex]{hyperref}                   
\usepackage {listings}
\usepackage {mathpartir}
\usepackage{bcprules}
%\usepackage{listings}
                       
\usepackage{graphicx} 
%\usepackage[margins=2.5cm,nohead,nofoot]{geometry}
%\usepackage{geometry}
\usepackage{amsfonts}
\usepackage{amstext}
\usepackage{latexsym}
\usepackage{amssymb}
\usepackage{color}


%\include{myPreamble}
\documentclass[12pt]{llncs}
%\documentclass{jktr}

\usepackage[pdftex]{hyperref}                   
\usepackage {listings}
\usepackage {mathpartir}
\usepackage{bcprules}
%\usepackage{listings}
                       
\usepackage{graphicx} 
%\usepackage[margins=2.5cm,nohead,nofoot]{geometry}
%\usepackage{geometry}
\usepackage{amsfonts}
\usepackage{amstext}
\usepackage{latexsym}
\usepackage{amssymb}
\usepackage{color}


%\include{myPreamble}
\include{qm2pi.local} 

%\ifpdf
%\usepackage[pdftex]{graphicx}
%\else
%\usepackage{graphicx}
%\fi

 % \ifpdf
%  \usepackage{pdfsync}
%  \if


%\title{Brief Article}
%\author{David F. Snyder}
%\author{L.G. Meredith}

%\address{Dept. of Math., Texas State University--San Marcos, San Marcos, TX 78666}
       
\pagestyle{empty}


\begin{document}

\lstset{language=[Objective]Caml,frame=shadowbox}

\input{qm2pi.front}

% section front matter (end)

\input{qm2pi.intro} 
 
% section introduction (end)

% \input{qm2pi.knotations} 

% section notation (end)

\input{qm2pi.process.calculi} 

% section concurrent_process_calculi_and_spatial_logics_ (end)
    
%\input{qm2pi.knots2pi} 

%\input{qm2pi.trefoil} 

%\input{qm2pi.mainthm} 

% subsection basic_interpretation (end)

%\input{qm2pi.rho.presentation} 
\subsection{The syntax and semantics of the notation system}\label{sub:the_syntax_and_semantics_of_the_notation_system} % (fold)

We now summarize a technical presentation of the calculus that
embodies our theory of dynamics. The typical presentation of such a
calculus follows the style of giving generators and relations on
them. The grammar, below, describing term constructors, freely
generates the set of processes, $\Proc$. This set is then quotiented
by a relation known as structural congruence and it is over this set
that the notion of dynamics is expressed. This presentation is
essentially that of \cite{MeredithR05} with the addition of
polyadicity and summation. For readability we have relegated some of
the technical subtleties to an appendix.

\subsubsection{Process grammar}\label{subsub:process_grammar}

\begin{mathpar}
  \inferrule* [lab=synchronization] {} {{M} \bc \pzero \;|\; x?F \;|\; x!C }
  \and
  \inferrule* [lab=abstraction] {} {{F} \bc (x)P}
  \and
  \inferrule* [lab=concretion] {} {{C} \bc \langle Q \rangle}
  \and
  \inferrule* [lab=process] {} {{P,Q} \bc M \;| \;P|Q \;|\; @{x}}
  \and
  \inferrule* [lab=name] {} {{x} \bc \quotep{P}}
\end{mathpar} 

Note that $\vec{x}$ (resp. $\vec{P}$) denotes a vector of names
(resp. processes) of length $|\vec{x}|$ (resp. $|\vec{P}|$). We adopt
the following useful abbreviations.

\begin{mathpar}
   x?(\vec{y}).P := x.(\vec{y})P \and  x\clift{\vec{P}} := x.\clift{\vec{P}}
   \and x!(y) := \lift{x}{\dropn{y}}
   \and \Pi_{i=0}^{n-1}P_i := P_0 | \ldots | P_{n-1}
\end{mathpar}

\subsubsection{Structural congruence}

\paragraph{Free and bound names and alpha-equivalence.} At the
core of structural equivalence is alpha-equivalence which identifies
process that are the same up to a change of variable. Formally, we
recognize the distinction between free and bound names. The free names
of a process, $\freenames{P}$, may be calculated recursively as
follows:

\begin{mathpar}
\freenames{\pzero} := \emptyset
  \and \\
  \freenames{x?(y).P} := \{ x \} \cup (\freenames{P} \setminus \{ y \})
  \and 
  \freenames{x!\langle P \rangle} := \{ x \} \cup \{ P \} 
  \and \\
  \freenames{P|Q} := \freenames{P} \cup \freenames{Q}
  \and \\
  \freenames{@{x}} := \{ x \}
\end{mathpar}

$\pi$
$\quotep{\pi}$

$\freenames{-} : \pi \to \mathcal{P}(\quotep{\pi})$

\begin{eqnarray*}
  \freenames{\pzero} & := & \emptyset \\
  \freenames{x?(y).P} & := & \{ x \} \cup (\freenames{P} \setminus \{ y \}) \\
  \freenames{x!\langle P \rangle} & := & \{ x \} \cup \{ P \} \\
  \freenames{P|Q} & := & \freenames{P} \cup \freenames{Q} \\
  \freenames{\dropn{x}} & := & \{ x \}
\end{eqnarray*}

The bound names of a process, $\boundnames{P}$, are those names occurring in $P$
that are not free. For example, in $x?(y).0$, the name $x$ is free, while $y$ is bound.

\begin{mathpar}
  \inferrule* [lab=monoidal-laws] {} { P|Q \equiv Q|P \and P|0 \equiv P \and P|(Q|R) \equiv (P|Q)|R }
\end{mathpar}

\begin{mathpar}
  \inferrule* [lab=alpha-equivalence] {} { (x)P \equiv (y)P\{y/x\} \and y \not\in \freenames{P} }
\end{mathpar}

\begin{definition}
Then two processes, $P,Q$, are alpha-equivalent if $P = Q\{\vec{y}/\vec{x}\}$ for
some $\vec{x} \in \boundnames{Q},\vec{y} \in \boundnames{P}$, where $Q\{\vec{y}/\vec{x}\}$
denotes the capture-avoiding substitution of $\vec{y}$ for $\vec{x}$ in $Q$.
\end{definition}

\begin{definition}
  The {\em structural congruence} \cite{SangiorgiWalker} , $\equiv$,
  between processes is the least congruence containing
  alpha-equivalence, satisfying the abelian monoid laws
  (associativity, commutativity and $\pzero$ as identity) for parallel
  composition $|$ and for summation $+$.
\end{definition}

\subsection{Name equivalence}

We take name equivalence, written $\nameeq$, to be the smallest
equivalence relation generated by the following rules.

\begin{mathpar}
\inferrule*[lab=Quote-drop]
{ }
{ \quotep{@{x}} \nameeq x }

\inferrule*[lab=Struct-equiv]
{ P \scong Q }
{ \quotep{P} \nameeq \quotep{Q} }
\end{mathpar}

The astute reader will have noticed that the mutual recursion of names
and processes imposes a mutual recursion on alpha-equivalence and
structural equivalence via name-equivalence. Fortunately, all of this
works out pleasantly and we may calculate in the natural way, free of
concern. The reader interested in the details is referred to the
appendix \ref{appendix:rho_details}.

\subsection{Substitution}

We use $\Proc$ for the set of processes, $\QProc$ for the set of
names, and $\id{\{}\vec{y} / \vec{x} \id{\}}$ to denote partial maps,
$s : \QProc \rightarrow \QProc$. A map, $s$ lifts, uniquely, to a map
on process terms, $\widehat{s} : \Proc \rightarrow \Proc$ by the
following equations.

\begin{mathpar}
  (0) \psubstp{Q}{P} := 0 \\
  (R \juxtap S) \psubstp{Q}{P}
  :=    
  (R)\psubstp{Q}{P} \juxtap (S) \psubstp{Q}{P} \\
  (x?(y).R) \psubstp{Q}{P}    
  :=    
  (x)\substp{Q}{P} (z)\concat( (R \psubstn{z}{y}) \psubstp{Q}{P} ) \\
  (\lift{x}{R}) \psubstp{Q}{P}  
  :=
  \lift{(x)\substp{Q}{P}}{ R \psubstp{Q}{P} } \\
%   (\dropn{x})  \psubstp{Q}{P}       
%   := 
%   \left\{ 
%     \begin{array}{ccc} 
%       \dropn{\quotep{Q}} & & x \nameeq \quotep{P} \\
%       \dropn{x} & & otherwise \\
%     \end{array}
%   \right. 
  (\dropn{x})  \psubstp{Q}{P}       
  := 
  \left\{ 
    \begin{array}{ccc} 
      Q & & x \nameeq \quotep{P} \\
      \dropn{x} & & otherwise \\
    \end{array}
  \right.
\end{mathpar}
 

where

\begin{eqnarray}
  (x)\id{\{} \lpquote Q \rpquote / \lpquote P \rpquote \id{\}}            = 
  \left\{ 
    \begin{array}{ccc}
      \lpquote Q \rpquote & & x \nameeq \lpquote P \rpquote \\
      x & & otherwise \\
    \end{array}
  \right. \nonumber
\end{eqnarray}

and $z$ is chosen distinct from $\quotep{P}$, $\quotep{Q}$, the free
names in $Q$, and all the names in $R$. Our $\alpha$-equivalence will
be built in the standard way from this substitution.

\begin{remark}\label{rem:no_self_referential_names}
  One consequence of these definitions is that $\forall P. \quotep{P}
  \not\in \freenames{P}$.
\end{remark}

\subsection{ Dynamic quote: an example }

Anticipating something of what's to come, consider applying the
substitution, $\widehat{\id{\{}u / z \id{\}}}$, to the following pair
of processes, $\lift{w}{y!(z)}$ and $w[ \lpquote y!(z) \rpquote ]$.

\begin{eqnarray}
	\lift{w}{y!(z)}\widehat{\id{\{}u / z \id{\}}}
		& = &
		\lift{w}{y!(u)} \nonumber\\
	w[ \lpquote y!(z) \rpquote ] \widehat{ \id{\{}u / z \id{\}} }
		& = &
		w[ \lpquote y!(z) \rpquote ] \nonumber
\end{eqnarray}

Because the body of the process between quotes is impervious to
substitution, we get radically different answers. In fact, by
examining the first process in an input context,
e.g. $x?(z).\lift{w}{y!(z)}$, we see that the process under the lift
operator may be shaped by prefixed inputs binding a name inside it. In
this sense, the lift operator will be seen as a way to dynamically
construct processes before reifying them as names.

Finally equipped with these standard features we can present the
dynamics of the calculus.

\subsubsection{Operational semantics} 

Finally, we introduce the computational dynamics. What marks these
algebras as distinct from other more traditionally studied algebraic
structures, e.g. vector spaces or polynomial rings, is the manner in
which dynamics is captured. In traditional structures, dynamics is typically
expressed through morphisms between such structures, as in linear maps
between vector spaces or morphisms between rings. In algebras
associated with the semantics of computation, the dynamics is
expressed as part of the algebraic structure itself, through a
reduction reduction relation typically denoted by $\red$. Below, we
give a recursive presentation of this relation for the calculus used
in the encoding.

$\red \subseteq \pi \times \pi$
$\red : \pi \to \mathcal{P}(\pi)$

\begin{mathpar}
  \inferrule* [lab=Comm] { \textsf{match}( x_{src}, x_{trgt} ) } { x_{trgt}?(y)P \; | \; x_{src}!\langle {Q} \rangle \red P\{\quotep{Q}/y}\} }
  \and \\
  \inferrule* [lab=Par] {{P} \red {P}'} {{{P} | {Q}} \red {{P}' | {Q}}}
  \and
  \inferrule* [lab=Equiv]{{{P} \scong {P}'} \andalso {{P}' \red {Q}'} \andalso {{Q}' \scong {Q}}}{{P} \red {Q}}
\end{mathpar}

\begin{eqnarray*}
  match_{\equiv} (\quotep{P},\quotep{Q}) & := & P \equiv Q \\
  match_{\dagger}(\quotep{P},\quotep{Q}) & := & \forall R. P|Q \red^{*} R => R \red^{*} 0 \\
  match_{K}(\quotep{P},\quotep{Q}) & := & K \mbox{ for some context } K
\end{eqnarray*}

$u?(x)P | u!\langle Q \rangle \red P\{\quotep{Q}/x\}$

%We write $\wred$ for $\red^*$, and $P\red$ if $\exists Q $ such that $ P \red Q$.
We write $P\red$ if $\exists Q $ such that $ P \red Q$ and $P\not\red$, otherwise.

\section{Replication}

As mentioned before, it is known that replication (and hence
recursion) can be implemented in a higher-order process algebra
\cite{SangiorgiWalker}. As our first example of calculation with the
machinery thus far presented we give the construction explicitly in
the {\rhoc}.

\begin{eqnarray}
	D_{x} & := & \prefix{x}{y}{(\binpar{\outputp{x}{y}}{@{y}})} \nonumber\\
	\bangp_{x}{P} & := & \binpar{{x}!\langle{\binpar{D_{x}}{P}}\rangle}{D_{x}} \nonumber
\end{eqnarray}

\begin{eqnarray}
	\bangp_{x}{P} & & \nonumber\\
	=
	& {x}!\langle{(\prefix{x}{y}{(\outputp{x}{y} | @{y})) | P}}\rangle 
	      | \prefix{x}{y}{(\outputp{x}{y} | @{y})} & \nonumber\\
	\red
	& (\outputp{x}{y} | @{y})\substn{\quotep{(\prefix{x}{y}{(@{y} | \outputp{x}{y})) | P}}}{y} & \nonumber\\
	=
	& \outputp{x}{\quotep{(\prefix{x}{y}{(\outputp{x}{y} | @{y})) | P}}}
	  | {(\prefix{x}{y}{(\outputp{x}{y} | @{y})) | P}} & \nonumber\\
	\red
	& \ldots & \nonumber\\
	\red^*
	& P | P | \ldots & \nonumber
\end{eqnarray}

Of course, this encoding, as an implementation, runs away, unfolding
$\bangp{P}$ eagerly. A lazier and more implementable replication
operator, restricted to input-guarded processes, may be obtained as follows.

\begin{eqnarray}
\bangp{\prefix{u}{v}{P}} 
	:= 
	\binpar{\lift{x}{\prefix{u}{v}{(\binpar{D(x)}{P})}}}{D(x)} \nonumber
\end{eqnarray}

\begin{remark}
  Note that the lazier definition still does not deal with summation
  or mixed summation (i.e. sums over input and output). The reader is
  invited to construct definitions of replication that deal with these
  features. 

  Further, the definitions are parameterized in a name, $x$. Can you,
  gentle reader, make a definition that eliminates this parameter and
  guarantees no accidental interaction between the replication
  machinery and the process being replicated -- i.e. no accidental
  sharing of names used by the process to get its work done and the
  name(s) used by the replication to effect copying. This latter
  revision of the definition of replication is crucial to obtaining
  the expected identity $!!P \sim !P$.
\end{remark}

\begin{remark}\label{rem:paradoxical_combinator}
  The reader familiar with the lambda calculus will have noticed the
  similarity between $D$ and the paradoxical combinator.

  [Ed. note: the existence of this seems to suggest we have to be more
  restrictive on the set of processes and names we admit if we are to
  support no-cloning.]
\end{remark}

\subsubsection{Bisimulation}

The computational dynamics gives rise to another kind of equivalence,
the equivalence of computational behavior. As previously mentioned
this is typically captured \emph{via} some form of bisimulation.

% The notion we use in this paper is weak barbed bisimulation
% \cite{milner91polyadicpi}.

The notion we use in this paper is derived from weak barbed
bisimulation \cite{milner91polyadicpi}. 

\begin{definition}
An \emph{observation relation}, $\downarrow_{\mathcal N}$, over a set
of names, $\mathcal N$, is the smallest relation satisfying the rules
below.

\infrule[Out-barb]{y \in {\mathcal N}, \; x \nameeq y}
		  {\outputp{x}{v} \downarrow_{\mathcal N} x}
\infrule[Par-barb]{\mbox{$P\downarrow_{\mathcal N} x$ or $Q\downarrow_{\mathcal N} x$}}
		  {\binpar{P}{Q} \downarrow_{\mathcal N} x}

We write $P \Downarrow_{\mathcal N} x$ if there is $Q$ such that 
$P \wred Q$ and $Q \downarrow_{\mathcal N} x$.
\end{definition}

\begin{definition}
%\label{def.bbisim}
An  ${\mathcal N}$-\emph{barbed bisimulation} over a set of names, ${\mathcal N}$, is a symmetric binary relation 
${\mathcal S}_{\mathcal N}$ between agents such that $P\rel{S}_{\mathcal N}Q$ implies:
\begin{enumerate}
\item If $P \red P'$ then $Q \wred Q'$ and $P'\rel{S}_{\mathcal N} Q'$.
\item If $P\downarrow_{\mathcal N} x$, then $Q\Downarrow_{\mathcal N} x$.
\end{enumerate}
$P$ is ${\mathcal N}$-barbed bisimilar to $Q$, written
$P \wbbisim_{\mathcal N} Q$, if $P \rel{S}_{\mathcal N} Q$ for some ${\mathcal N}$-barbed bisimulation ${\mathcal S}_{\mathcal N}$.
\end{definition}

$\mathcal{R} \subseteq \pi \times \pi$

$P \mathcal{R} Q => \forall P'. P \red P' \Rightarrow \exists Q'. Q \red Q', P' \mathcal{R} Q'$

$P \vdash x \Rightarrow Q \vdash x$

\begin{mathpar}
  \inferrule*[lab=Out-barb]{x \nameeq y}{{y}!\langle{Q}\rangle \vdash x}
  \and
  \inferrule*[lab=Par-barb]{\mbox{$P\vdash x$ or $Q\vdash x$}}{\binpar{P}{Q} \vdash x}
\end{mathpar}

\subsubsection{Contexts}

One of the principle advantages of computational calculi like the
$\pi$-calculus is a well-defined notion of context,
contextual-equivalence and a correlation between
contextual-equivalence and notions of bisimulation. The notion of
context allows the decomposition of a process into (sub-)process and
its syntactic environment, its context. Thus, a context may be
thought of as a process with a ``hole'' (written $\Box$) in it. The
application of a context $M$ to a process $P$, written $M[P]$, is
tantamount to filling the hole in $M$ with $P$. In this paper we do
not need the full weight of this theory, but do make use of the notion
of context in the proof the main theorem. 

\begin{mathpar}
  \inferrule* [lab=summation] {} {{M_{M},M_{N}} \bc \Box \;|\; x.M_{A} \;|\; M_{M}+M_{N}}
  \and
  \inferrule* [lab=agent] {} {{M_{A}} \bc (\vec{x})M_{P} \;| \; \clift{P_0,\ldots,M_{P},\ldots,P_N}}
  \and \\
  \inferrule* [lab=process] {} {{M_{P}} \bc M_{N} \;| \;P|M_{P} }
\end{mathpar} 

\begin{mathpar}
  \inferrule* [lab=sychronization] {} {M_{N} \bc \Box \;|\; x?M_{F} \;|\; x!M_{C}}
  \and
  \inferrule* [lab=abstraction] {} {{M_{F}} \bc (x)M_{P} }
  \and
  \inferrule* [lab=concretion] {} {{M_{C}} \bc \langle M_{P} \rangle }
  \and \\
  \inferrule* [lab=process] {} {{M_{P}} \bc M_{N} \;| \;P|M_{P} }
\end{mathpar}

\begin{definition}[contextual application] Given a context $M$, and
  process $P$, we define the \emph{contextual application}, $M[P] :=
  M\{P/\Box\}$. That is, the contextual application of M to P is the
  substitution of $P$ for $\Box$ in $M$.
\end{definition}

$\meaningof{-} : L \to \mathcal{P}(\pi)$

\begin{mathpar}
  \inferrule* [lab=collection] {} {\meaningof{true} = \pi, \and \meaningof{~E} = \pi \setminus \meaningof{E}, \and \meaningof{E_{1} \& E_{2}} = \meaningof{E_{1}} \cap \meaningof{E_{2}}}
\end{mathpar}

\begin{mathpar}
  \inferrule* [lab=structure] {} {\meaningof{0} = \{ P \in \pi | P \equiv 0 \}, \and \\ \meaningof{E_1 | E_2} = \{ P \in \pi | P \equiv P_{1} | P_{2}, P_{1} \in \meaningof{E_{1}}, P_{2} \in \meaningof{E_2}\} }
\end{mathpar}

\begin{mathpar}
 \inferrule* [lab=behavior] {} {\meaningof{\langle a?b \rangle E} = \{ P \in \pi | P \equiv Q | u?(y)P', \\ \and \\\\ \and \\ \;\;\; u \in \meaningof{a}, \forall z.P'\{z/y\} \in \meaningof{E\{z/b\}}\}, \and \\ \meaningof{a!E} = \{ P \in \pi | P \equiv Q | x!\langle P' \rangle, x \in \meaningof{a} P' \in \meaningof{E}\} }
\end{mathpar}

\begin{mathpar}
 \inferrule* [lab=nominal] {} {\meaningof{\quotep{E}} = \{ \quotep{P} \in \quotep{\pi} | P \in \meaningof{E} \}, \and \meaningof{\quotep{P}} = \{ \quotep{Q} \in \quotep{\pi} | P \equiv Q \} \and \\ \meaningof{@\quotep{E}} = \{ P \in \pi | P \equiv @x, x \in \meaningof{E} \}}
\end{mathpar}

\begin{eqnarray*}
  \\
  \meaningof{-} : TS \to ST
\end{eqnarray*}

\begin{eqnarray*}
  \\
  L : TS \to ST
\end{eqnarray*}

\begin{eqnarray*}
  \\
  P \models E \iff P \in \meaningof{E}
\end{eqnarray*}

\begin{eqnarray*}
  P \approx_{L} Q \iff \forall E \in L. P \models E \iff Q \models E
\end{eqnarray*}

\begin{eqnarray*}
  P \approx_{K} Q
\end{eqnarray*}

\begin{eqnarray*}
  P \approx Q
\end{eqnarray*}

$\approx_{K} = \approx = \approx_{L}$

\subsubsection{Contextual duality}

Note that contexts extend the quotation operation to a family of
operations from processes to names. Given a context, $M$, we can
define a \emph{nominal context}, $\quotep{M}$ by $\quotep{M}[P] :=
\quotep{M[P]}$. To foreshadow what is to come we observe that these
operations enjoy a duality with processes very much like the duality
between vectors and maps from vectors to scalars.

Further, because the calculus is essentially higher-order, we have a
correspondence between contexts and processes. More specifically,
given a name $x$ and a context $M$ we can construct $M^{*}_{x}$ such
that 

\begin{mathpar}
  M^{*}_{x} | \lift{x}{P} \red M[P]
\end{mathpar}

namely,

\begin{mathpar}
  M^{*}_{x} := x?(u).M[\dropn{u}]
\end{mathpar}

The dependence of $M^{*}_{x}$ on a name makes it an abstraction, 

\begin{mathpar}
  M^{*} := (x)x?(u).M[\dropn{u}]
\end{mathpar}

\subsection{Additional notation}

It will sometimes be convenient to denote the process a name
quotes. We already have the notation $x = \quotep{P}$, but it will be
convenient to introduce an alternate notation, $\procn{x}$, when we
want to emphasize the connection to the use of the name. Note that, by
virtue of name equivalence, $\quotep{\procn{x}} \nameeq x$; so, the
notation is consistent with previous definitions.

Further, because names have structure it is possible to effect
substitutions on the basis of that structure. This means we need to
upgrade our notation for substitutions, which we accomplish by
adapting comprehension notation. Thus,

\begin{mathpar}
  P\{ y / x : x \in S \}
\end{mathpar}

is interpreted to mean the process derived from P by replacing (in a
capture-avoiding manner) each occurrence of $x$ in $S$ by $y$. For example,

\begin{mathpar}
  P\{ \quotep{\procn{x}|\procn{x}} / x : x \in \freenames{P} \}
\end{mathpar}

will replace each (occurrence) of a free name $x$ in $P$ by
$\quotep{\procn{x}|\procn{x}}$.

Also, we will avail ourselves of the notation $x^{L}$ and $x^{R}$ to
denote injections of a name into disjoint copies of the name
space. There are numerous ways to accomplish this. One example can be
found in \cite{MeredithR05}. This notation overloads to vectors of
names: $\vec{x}^{\pi} := (x_{i}^{\pi} \; : \; 0 \leq i < |\vec{x}| )$ where $\pi \in \{L,R\}$.

We also use $P^{\Box} := P|\Box$.

In \cite{MeredithR05} an interpretation of the new operator is
given. It turns out that there are several possible interpretations
all enjoying the requisite algebraic properties of the operator (see
\cite{milner91polyadicpi}). We will therefore make liberal use of
$(\nu\; \vec{x})P$.

% subsection the_syntax_and_semantics_of_the_notation_system (end)   

\input{qm2pi.qmops} 

\input{qm2pi.sterngerlach} 

\input{qm2pi.metric} 

% section concurrent_process_calculi (end)

%\input{qm2pi.proofsketch}

% section proof sketch (end)

%\input{qm2pi.slviaknots} 

% section spatial logic via knots (end)

\input{qm2pi.conclusion}

% section conclusion (end)

%\input{qm2pi.dtcodes} 

% section wiring algorithm (end)

\input{qm2pi.ack} 

% section acknowledgments (end)

\newpage


\bibliographystyle{plain}   
\bibliography{../../biblios/main.bib}

\input{qm2pi.rhodetails}

\end{document}

 

%\ifpdf
%\usepackage[pdftex]{graphicx}
%\else
%\usepackage{graphicx}
%\fi

 % \ifpdf
%  \usepackage{pdfsync}
%  \if


%\title{Brief Article}
%\author{David F. Snyder}
%\author{L.G. Meredith}

%\address{Dept. of Math., Texas State University--San Marcos, San Marcos, TX 78666}
       
\pagestyle{empty}


\begin{document}

\lstset{language=[Objective]Caml,frame=shadowbox}

\documentclass[12pt]{llncs}
%\documentclass{jktr}

\usepackage[pdftex]{hyperref}                   
\usepackage {listings}
\usepackage {mathpartir}
\usepackage{bcprules}
%\usepackage{listings}
                       
\usepackage{graphicx} 
%\usepackage[margins=2.5cm,nohead,nofoot]{geometry}
%\usepackage{geometry}
\usepackage{amsfonts}
\usepackage{amstext}
\usepackage{latexsym}
\usepackage{amssymb}
\usepackage{color}


%\include{myPreamble}
\include{qm2pi.local} 

%\ifpdf
%\usepackage[pdftex]{graphicx}
%\else
%\usepackage{graphicx}
%\fi

 % \ifpdf
%  \usepackage{pdfsync}
%  \if


%\title{Brief Article}
%\author{David F. Snyder}
%\author{L.G. Meredith}

%\address{Dept. of Math., Texas State University--San Marcos, San Marcos, TX 78666}
       
\pagestyle{empty}


\begin{document}

\lstset{language=[Objective]Caml,frame=shadowbox}

\input{qm2pi.front}

% section front matter (end)

\input{qm2pi.intro} 
 
% section introduction (end)

% \input{qm2pi.knotations} 

% section notation (end)

\input{qm2pi.process.calculi} 

% section concurrent_process_calculi_and_spatial_logics_ (end)
    
%\input{qm2pi.knots2pi} 

%\input{qm2pi.trefoil} 

%\input{qm2pi.mainthm} 

% subsection basic_interpretation (end)

%\input{qm2pi.rho.presentation} 
\subsection{The syntax and semantics of the notation system}\label{sub:the_syntax_and_semantics_of_the_notation_system} % (fold)

We now summarize a technical presentation of the calculus that
embodies our theory of dynamics. The typical presentation of such a
calculus follows the style of giving generators and relations on
them. The grammar, below, describing term constructors, freely
generates the set of processes, $\Proc$. This set is then quotiented
by a relation known as structural congruence and it is over this set
that the notion of dynamics is expressed. This presentation is
essentially that of \cite{MeredithR05} with the addition of
polyadicity and summation. For readability we have relegated some of
the technical subtleties to an appendix.

\subsubsection{Process grammar}\label{subsub:process_grammar}

\begin{mathpar}
  \inferrule* [lab=synchronization] {} {{M} \bc \pzero \;|\; x?F \;|\; x!C }
  \and
  \inferrule* [lab=abstraction] {} {{F} \bc (x)P}
  \and
  \inferrule* [lab=concretion] {} {{C} \bc \langle Q \rangle}
  \and
  \inferrule* [lab=process] {} {{P,Q} \bc M \;| \;P|Q \;|\; @{x}}
  \and
  \inferrule* [lab=name] {} {{x} \bc \quotep{P}}
\end{mathpar} 

Note that $\vec{x}$ (resp. $\vec{P}$) denotes a vector of names
(resp. processes) of length $|\vec{x}|$ (resp. $|\vec{P}|$). We adopt
the following useful abbreviations.

\begin{mathpar}
   x?(\vec{y}).P := x.(\vec{y})P \and  x\clift{\vec{P}} := x.\clift{\vec{P}}
   \and x!(y) := \lift{x}{\dropn{y}}
   \and \Pi_{i=0}^{n-1}P_i := P_0 | \ldots | P_{n-1}
\end{mathpar}

\subsubsection{Structural congruence}

\paragraph{Free and bound names and alpha-equivalence.} At the
core of structural equivalence is alpha-equivalence which identifies
process that are the same up to a change of variable. Formally, we
recognize the distinction between free and bound names. The free names
of a process, $\freenames{P}$, may be calculated recursively as
follows:

\begin{mathpar}
\freenames{\pzero} := \emptyset
  \and \\
  \freenames{x?(y).P} := \{ x \} \cup (\freenames{P} \setminus \{ y \})
  \and 
  \freenames{x!\langle P \rangle} := \{ x \} \cup \{ P \} 
  \and \\
  \freenames{P|Q} := \freenames{P} \cup \freenames{Q}
  \and \\
  \freenames{@{x}} := \{ x \}
\end{mathpar}

$\pi$
$\quotep{\pi}$

$\freenames{-} : \pi \to \mathcal{P}(\quotep{\pi})$

\begin{eqnarray*}
  \freenames{\pzero} & := & \emptyset \\
  \freenames{x?(y).P} & := & \{ x \} \cup (\freenames{P} \setminus \{ y \}) \\
  \freenames{x!\langle P \rangle} & := & \{ x \} \cup \{ P \} \\
  \freenames{P|Q} & := & \freenames{P} \cup \freenames{Q} \\
  \freenames{\dropn{x}} & := & \{ x \}
\end{eqnarray*}

The bound names of a process, $\boundnames{P}$, are those names occurring in $P$
that are not free. For example, in $x?(y).0$, the name $x$ is free, while $y$ is bound.

\begin{mathpar}
  \inferrule* [lab=monoidal-laws] {} { P|Q \equiv Q|P \and P|0 \equiv P \and P|(Q|R) \equiv (P|Q)|R }
\end{mathpar}

\begin{mathpar}
  \inferrule* [lab=alpha-equivalence] {} { (x)P \equiv (y)P\{y/x\} \and y \not\in \freenames{P} }
\end{mathpar}

\begin{definition}
Then two processes, $P,Q$, are alpha-equivalent if $P = Q\{\vec{y}/\vec{x}\}$ for
some $\vec{x} \in \boundnames{Q},\vec{y} \in \boundnames{P}$, where $Q\{\vec{y}/\vec{x}\}$
denotes the capture-avoiding substitution of $\vec{y}$ for $\vec{x}$ in $Q$.
\end{definition}

\begin{definition}
  The {\em structural congruence} \cite{SangiorgiWalker} , $\equiv$,
  between processes is the least congruence containing
  alpha-equivalence, satisfying the abelian monoid laws
  (associativity, commutativity and $\pzero$ as identity) for parallel
  composition $|$ and for summation $+$.
\end{definition}

\subsection{Name equivalence}

We take name equivalence, written $\nameeq$, to be the smallest
equivalence relation generated by the following rules.

\begin{mathpar}
\inferrule*[lab=Quote-drop]
{ }
{ \quotep{@{x}} \nameeq x }

\inferrule*[lab=Struct-equiv]
{ P \scong Q }
{ \quotep{P} \nameeq \quotep{Q} }
\end{mathpar}

The astute reader will have noticed that the mutual recursion of names
and processes imposes a mutual recursion on alpha-equivalence and
structural equivalence via name-equivalence. Fortunately, all of this
works out pleasantly and we may calculate in the natural way, free of
concern. The reader interested in the details is referred to the
appendix \ref{appendix:rho_details}.

\subsection{Substitution}

We use $\Proc$ for the set of processes, $\QProc$ for the set of
names, and $\id{\{}\vec{y} / \vec{x} \id{\}}$ to denote partial maps,
$s : \QProc \rightarrow \QProc$. A map, $s$ lifts, uniquely, to a map
on process terms, $\widehat{s} : \Proc \rightarrow \Proc$ by the
following equations.

\begin{mathpar}
  (0) \psubstp{Q}{P} := 0 \\
  (R \juxtap S) \psubstp{Q}{P}
  :=    
  (R)\psubstp{Q}{P} \juxtap (S) \psubstp{Q}{P} \\
  (x?(y).R) \psubstp{Q}{P}    
  :=    
  (x)\substp{Q}{P} (z)\concat( (R \psubstn{z}{y}) \psubstp{Q}{P} ) \\
  (\lift{x}{R}) \psubstp{Q}{P}  
  :=
  \lift{(x)\substp{Q}{P}}{ R \psubstp{Q}{P} } \\
%   (\dropn{x})  \psubstp{Q}{P}       
%   := 
%   \left\{ 
%     \begin{array}{ccc} 
%       \dropn{\quotep{Q}} & & x \nameeq \quotep{P} \\
%       \dropn{x} & & otherwise \\
%     \end{array}
%   \right. 
  (\dropn{x})  \psubstp{Q}{P}       
  := 
  \left\{ 
    \begin{array}{ccc} 
      Q & & x \nameeq \quotep{P} \\
      \dropn{x} & & otherwise \\
    \end{array}
  \right.
\end{mathpar}
 

where

\begin{eqnarray}
  (x)\id{\{} \lpquote Q \rpquote / \lpquote P \rpquote \id{\}}            = 
  \left\{ 
    \begin{array}{ccc}
      \lpquote Q \rpquote & & x \nameeq \lpquote P \rpquote \\
      x & & otherwise \\
    \end{array}
  \right. \nonumber
\end{eqnarray}

and $z$ is chosen distinct from $\quotep{P}$, $\quotep{Q}$, the free
names in $Q$, and all the names in $R$. Our $\alpha$-equivalence will
be built in the standard way from this substitution.

\begin{remark}\label{rem:no_self_referential_names}
  One consequence of these definitions is that $\forall P. \quotep{P}
  \not\in \freenames{P}$.
\end{remark}

\subsection{ Dynamic quote: an example }

Anticipating something of what's to come, consider applying the
substitution, $\widehat{\id{\{}u / z \id{\}}}$, to the following pair
of processes, $\lift{w}{y!(z)}$ and $w[ \lpquote y!(z) \rpquote ]$.

\begin{eqnarray}
	\lift{w}{y!(z)}\widehat{\id{\{}u / z \id{\}}}
		& = &
		\lift{w}{y!(u)} \nonumber\\
	w[ \lpquote y!(z) \rpquote ] \widehat{ \id{\{}u / z \id{\}} }
		& = &
		w[ \lpquote y!(z) \rpquote ] \nonumber
\end{eqnarray}

Because the body of the process between quotes is impervious to
substitution, we get radically different answers. In fact, by
examining the first process in an input context,
e.g. $x?(z).\lift{w}{y!(z)}$, we see that the process under the lift
operator may be shaped by prefixed inputs binding a name inside it. In
this sense, the lift operator will be seen as a way to dynamically
construct processes before reifying them as names.

Finally equipped with these standard features we can present the
dynamics of the calculus.

\subsubsection{Operational semantics} 

Finally, we introduce the computational dynamics. What marks these
algebras as distinct from other more traditionally studied algebraic
structures, e.g. vector spaces or polynomial rings, is the manner in
which dynamics is captured. In traditional structures, dynamics is typically
expressed through morphisms between such structures, as in linear maps
between vector spaces or morphisms between rings. In algebras
associated with the semantics of computation, the dynamics is
expressed as part of the algebraic structure itself, through a
reduction reduction relation typically denoted by $\red$. Below, we
give a recursive presentation of this relation for the calculus used
in the encoding.

$\red \subseteq \pi \times \pi$
$\red : \pi \to \mathcal{P}(\pi)$

\begin{mathpar}
  \inferrule* [lab=Comm] { \textsf{match}( x_{src}, x_{trgt} ) } { x_{trgt}?(y)P \; | \; x_{src}!\langle {Q} \rangle \red P\{\quotep{Q}/y}\} }
  \and \\
  \inferrule* [lab=Par] {{P} \red {P}'} {{{P} | {Q}} \red {{P}' | {Q}}}
  \and
  \inferrule* [lab=Equiv]{{{P} \scong {P}'} \andalso {{P}' \red {Q}'} \andalso {{Q}' \scong {Q}}}{{P} \red {Q}}
\end{mathpar}

\begin{eqnarray*}
  match_{\equiv} (\quotep{P},\quotep{Q}) & := & P \equiv Q \\
  match_{\dagger}(\quotep{P},\quotep{Q}) & := & \forall R. P|Q \red^{*} R => R \red^{*} 0 \\
  match_{K}(\quotep{P},\quotep{Q}) & := & K \mbox{ for some context } K
\end{eqnarray*}

$u?(x)P | u!\langle Q \rangle \red P\{\quotep{Q}/x\}$

%We write $\wred$ for $\red^*$, and $P\red$ if $\exists Q $ such that $ P \red Q$.
We write $P\red$ if $\exists Q $ such that $ P \red Q$ and $P\not\red$, otherwise.

\section{Replication}

As mentioned before, it is known that replication (and hence
recursion) can be implemented in a higher-order process algebra
\cite{SangiorgiWalker}. As our first example of calculation with the
machinery thus far presented we give the construction explicitly in
the {\rhoc}.

\begin{eqnarray}
	D_{x} & := & \prefix{x}{y}{(\binpar{\outputp{x}{y}}{@{y}})} \nonumber\\
	\bangp_{x}{P} & := & \binpar{{x}!\langle{\binpar{D_{x}}{P}}\rangle}{D_{x}} \nonumber
\end{eqnarray}

\begin{eqnarray}
	\bangp_{x}{P} & & \nonumber\\
	=
	& {x}!\langle{(\prefix{x}{y}{(\outputp{x}{y} | @{y})) | P}}\rangle 
	      | \prefix{x}{y}{(\outputp{x}{y} | @{y})} & \nonumber\\
	\red
	& (\outputp{x}{y} | @{y})\substn{\quotep{(\prefix{x}{y}{(@{y} | \outputp{x}{y})) | P}}}{y} & \nonumber\\
	=
	& \outputp{x}{\quotep{(\prefix{x}{y}{(\outputp{x}{y} | @{y})) | P}}}
	  | {(\prefix{x}{y}{(\outputp{x}{y} | @{y})) | P}} & \nonumber\\
	\red
	& \ldots & \nonumber\\
	\red^*
	& P | P | \ldots & \nonumber
\end{eqnarray}

Of course, this encoding, as an implementation, runs away, unfolding
$\bangp{P}$ eagerly. A lazier and more implementable replication
operator, restricted to input-guarded processes, may be obtained as follows.

\begin{eqnarray}
\bangp{\prefix{u}{v}{P}} 
	:= 
	\binpar{\lift{x}{\prefix{u}{v}{(\binpar{D(x)}{P})}}}{D(x)} \nonumber
\end{eqnarray}

\begin{remark}
  Note that the lazier definition still does not deal with summation
  or mixed summation (i.e. sums over input and output). The reader is
  invited to construct definitions of replication that deal with these
  features. 

  Further, the definitions are parameterized in a name, $x$. Can you,
  gentle reader, make a definition that eliminates this parameter and
  guarantees no accidental interaction between the replication
  machinery and the process being replicated -- i.e. no accidental
  sharing of names used by the process to get its work done and the
  name(s) used by the replication to effect copying. This latter
  revision of the definition of replication is crucial to obtaining
  the expected identity $!!P \sim !P$.
\end{remark}

\begin{remark}\label{rem:paradoxical_combinator}
  The reader familiar with the lambda calculus will have noticed the
  similarity between $D$ and the paradoxical combinator.

  [Ed. note: the existence of this seems to suggest we have to be more
  restrictive on the set of processes and names we admit if we are to
  support no-cloning.]
\end{remark}

\subsubsection{Bisimulation}

The computational dynamics gives rise to another kind of equivalence,
the equivalence of computational behavior. As previously mentioned
this is typically captured \emph{via} some form of bisimulation.

% The notion we use in this paper is weak barbed bisimulation
% \cite{milner91polyadicpi}.

The notion we use in this paper is derived from weak barbed
bisimulation \cite{milner91polyadicpi}. 

\begin{definition}
An \emph{observation relation}, $\downarrow_{\mathcal N}$, over a set
of names, $\mathcal N$, is the smallest relation satisfying the rules
below.

\infrule[Out-barb]{y \in {\mathcal N}, \; x \nameeq y}
		  {\outputp{x}{v} \downarrow_{\mathcal N} x}
\infrule[Par-barb]{\mbox{$P\downarrow_{\mathcal N} x$ or $Q\downarrow_{\mathcal N} x$}}
		  {\binpar{P}{Q} \downarrow_{\mathcal N} x}

We write $P \Downarrow_{\mathcal N} x$ if there is $Q$ such that 
$P \wred Q$ and $Q \downarrow_{\mathcal N} x$.
\end{definition}

\begin{definition}
%\label{def.bbisim}
An  ${\mathcal N}$-\emph{barbed bisimulation} over a set of names, ${\mathcal N}$, is a symmetric binary relation 
${\mathcal S}_{\mathcal N}$ between agents such that $P\rel{S}_{\mathcal N}Q$ implies:
\begin{enumerate}
\item If $P \red P'$ then $Q \wred Q'$ and $P'\rel{S}_{\mathcal N} Q'$.
\item If $P\downarrow_{\mathcal N} x$, then $Q\Downarrow_{\mathcal N} x$.
\end{enumerate}
$P$ is ${\mathcal N}$-barbed bisimilar to $Q$, written
$P \wbbisim_{\mathcal N} Q$, if $P \rel{S}_{\mathcal N} Q$ for some ${\mathcal N}$-barbed bisimulation ${\mathcal S}_{\mathcal N}$.
\end{definition}

$\mathcal{R} \subseteq \pi \times \pi$

$P \mathcal{R} Q => \forall P'. P \red P' \Rightarrow \exists Q'. Q \red Q', P' \mathcal{R} Q'$

$P \vdash x \Rightarrow Q \vdash x$

\begin{mathpar}
  \inferrule*[lab=Out-barb]{x \nameeq y}{{y}!\langle{Q}\rangle \vdash x}
  \and
  \inferrule*[lab=Par-barb]{\mbox{$P\vdash x$ or $Q\vdash x$}}{\binpar{P}{Q} \vdash x}
\end{mathpar}

\subsubsection{Contexts}

One of the principle advantages of computational calculi like the
$\pi$-calculus is a well-defined notion of context,
contextual-equivalence and a correlation between
contextual-equivalence and notions of bisimulation. The notion of
context allows the decomposition of a process into (sub-)process and
its syntactic environment, its context. Thus, a context may be
thought of as a process with a ``hole'' (written $\Box$) in it. The
application of a context $M$ to a process $P$, written $M[P]$, is
tantamount to filling the hole in $M$ with $P$. In this paper we do
not need the full weight of this theory, but do make use of the notion
of context in the proof the main theorem. 

\begin{mathpar}
  \inferrule* [lab=summation] {} {{M_{M},M_{N}} \bc \Box \;|\; x.M_{A} \;|\; M_{M}+M_{N}}
  \and
  \inferrule* [lab=agent] {} {{M_{A}} \bc (\vec{x})M_{P} \;| \; \clift{P_0,\ldots,M_{P},\ldots,P_N}}
  \and \\
  \inferrule* [lab=process] {} {{M_{P}} \bc M_{N} \;| \;P|M_{P} }
\end{mathpar} 

\begin{mathpar}
  \inferrule* [lab=sychronization] {} {M_{N} \bc \Box \;|\; x?M_{F} \;|\; x!M_{C}}
  \and
  \inferrule* [lab=abstraction] {} {{M_{F}} \bc (x)M_{P} }
  \and
  \inferrule* [lab=concretion] {} {{M_{C}} \bc \langle M_{P} \rangle }
  \and \\
  \inferrule* [lab=process] {} {{M_{P}} \bc M_{N} \;| \;P|M_{P} }
\end{mathpar}

\begin{definition}[contextual application] Given a context $M$, and
  process $P$, we define the \emph{contextual application}, $M[P] :=
  M\{P/\Box\}$. That is, the contextual application of M to P is the
  substitution of $P$ for $\Box$ in $M$.
\end{definition}

$\meaningof{-} : L \to \mathcal{P}(\pi)$

\begin{mathpar}
  \inferrule* [lab=collection] {} {\meaningof{true} = \pi, \and \meaningof{~E} = \pi \setminus \meaningof{E}, \and \meaningof{E_{1} \& E_{2}} = \meaningof{E_{1}} \cap \meaningof{E_{2}}}
\end{mathpar}

\begin{mathpar}
  \inferrule* [lab=structure] {} {\meaningof{0} = \{ P \in \pi | P \equiv 0 \}, \and \\ \meaningof{E_1 | E_2} = \{ P \in \pi | P \equiv P_{1} | P_{2}, P_{1} \in \meaningof{E_{1}}, P_{2} \in \meaningof{E_2}\} }
\end{mathpar}

\begin{mathpar}
 \inferrule* [lab=behavior] {} {\meaningof{\langle a?b \rangle E} = \{ P \in \pi | P \equiv Q | u?(y)P', \\ \and \\\\ \and \\ \;\;\; u \in \meaningof{a}, \forall z.P'\{z/y\} \in \meaningof{E\{z/b\}}\}, \and \\ \meaningof{a!E} = \{ P \in \pi | P \equiv Q | x!\langle P' \rangle, x \in \meaningof{a} P' \in \meaningof{E}\} }
\end{mathpar}

\begin{mathpar}
 \inferrule* [lab=nominal] {} {\meaningof{\quotep{E}} = \{ \quotep{P} \in \quotep{\pi} | P \in \meaningof{E} \}, \and \meaningof{\quotep{P}} = \{ \quotep{Q} \in \quotep{\pi} | P \equiv Q \} \and \\ \meaningof{@\quotep{E}} = \{ P \in \pi | P \equiv @x, x \in \meaningof{E} \}}
\end{mathpar}

\begin{eqnarray*}
  \\
  \meaningof{-} : TS \to ST
\end{eqnarray*}

\begin{eqnarray*}
  \\
  L : TS \to ST
\end{eqnarray*}

\begin{eqnarray*}
  \\
  P \models E \iff P \in \meaningof{E}
\end{eqnarray*}

\begin{eqnarray*}
  P \approx_{L} Q \iff \forall E \in L. P \models E \iff Q \models E
\end{eqnarray*}

\begin{eqnarray*}
  P \approx_{K} Q
\end{eqnarray*}

\begin{eqnarray*}
  P \approx Q
\end{eqnarray*}

$\approx_{K} = \approx = \approx_{L}$

\subsubsection{Contextual duality}

Note that contexts extend the quotation operation to a family of
operations from processes to names. Given a context, $M$, we can
define a \emph{nominal context}, $\quotep{M}$ by $\quotep{M}[P] :=
\quotep{M[P]}$. To foreshadow what is to come we observe that these
operations enjoy a duality with processes very much like the duality
between vectors and maps from vectors to scalars.

Further, because the calculus is essentially higher-order, we have a
correspondence between contexts and processes. More specifically,
given a name $x$ and a context $M$ we can construct $M^{*}_{x}$ such
that 

\begin{mathpar}
  M^{*}_{x} | \lift{x}{P} \red M[P]
\end{mathpar}

namely,

\begin{mathpar}
  M^{*}_{x} := x?(u).M[\dropn{u}]
\end{mathpar}

The dependence of $M^{*}_{x}$ on a name makes it an abstraction, 

\begin{mathpar}
  M^{*} := (x)x?(u).M[\dropn{u}]
\end{mathpar}

\subsection{Additional notation}

It will sometimes be convenient to denote the process a name
quotes. We already have the notation $x = \quotep{P}$, but it will be
convenient to introduce an alternate notation, $\procn{x}$, when we
want to emphasize the connection to the use of the name. Note that, by
virtue of name equivalence, $\quotep{\procn{x}} \nameeq x$; so, the
notation is consistent with previous definitions.

Further, because names have structure it is possible to effect
substitutions on the basis of that structure. This means we need to
upgrade our notation for substitutions, which we accomplish by
adapting comprehension notation. Thus,

\begin{mathpar}
  P\{ y / x : x \in S \}
\end{mathpar}

is interpreted to mean the process derived from P by replacing (in a
capture-avoiding manner) each occurrence of $x$ in $S$ by $y$. For example,

\begin{mathpar}
  P\{ \quotep{\procn{x}|\procn{x}} / x : x \in \freenames{P} \}
\end{mathpar}

will replace each (occurrence) of a free name $x$ in $P$ by
$\quotep{\procn{x}|\procn{x}}$.

Also, we will avail ourselves of the notation $x^{L}$ and $x^{R}$ to
denote injections of a name into disjoint copies of the name
space. There are numerous ways to accomplish this. One example can be
found in \cite{MeredithR05}. This notation overloads to vectors of
names: $\vec{x}^{\pi} := (x_{i}^{\pi} \; : \; 0 \leq i < |\vec{x}| )$ where $\pi \in \{L,R\}$.

We also use $P^{\Box} := P|\Box$.

In \cite{MeredithR05} an interpretation of the new operator is
given. It turns out that there are several possible interpretations
all enjoying the requisite algebraic properties of the operator (see
\cite{milner91polyadicpi}). We will therefore make liberal use of
$(\nu\; \vec{x})P$.

% subsection the_syntax_and_semantics_of_the_notation_system (end)   

\input{qm2pi.qmops} 

\input{qm2pi.sterngerlach} 

\input{qm2pi.metric} 

% section concurrent_process_calculi (end)

%\input{qm2pi.proofsketch}

% section proof sketch (end)

%\input{qm2pi.slviaknots} 

% section spatial logic via knots (end)

\input{qm2pi.conclusion}

% section conclusion (end)

%\input{qm2pi.dtcodes} 

% section wiring algorithm (end)

\input{qm2pi.ack} 

% section acknowledgments (end)

\newpage


\bibliographystyle{plain}   
\bibliography{../../biblios/main.bib}

\input{qm2pi.rhodetails}

\end{document}



% section front matter (end)

\section{Introduction}\label{sec:introduction} % (fold)
In this draft of the material i am going to have to dispense with the
usual writing conventions adopted in papers on these topics. i'm going
to have adopt whatever tone i need at the time i'm writing up the
calculations. Sometimes this may be very conversational; others it may
be the barest mathematical grunts; others still it may be that i have
lifted text from one of my other papers because the exposition of some
point was better said there. i hope that my readers are not unduly put
out by this decision. i'm not doing this to flout convention or be
rebellious. i find these calculations very technically challenging. To
keep everything going technically, something has to give; i have to
let go of some cognitive burden. So, the academic writing style --
with all of its trade-offs in terms of facilitating technical
communication -- is what i'm letting go of. Perhaps subsequent drafts
can be tightened and polished, but for now, i'm going to speak as if
we were sitting together in a coffee shop with a laptop, wifi and a
pad of paper and a pencil.

So, here's what i have to say. We -- you and i, comfortably ensconced
in our coffee shop and well-equipped with our tools -- can realize and
carry out the calculations of quantum mechanics over a very different
formal theory of dynamics, a formal theory of dynamics that
corresponds to a theory of concurrent computation with
\emph{reflection}. It has the advantage that the underlying theory is
already `quantized', but supports analogues all of the continuuous
operations. Strikingly, this underlying theory has recently been
connected with a notion of metric that we can show, by calculating
together, coincides with the metric induced by the inner product.

There are a lot of reasons why you might be interested in seeing
calculations of this form. Here's why i'm interested. For the past
several centuries there has been no competitor to the ``Newtonian''
account of dynamics. As a result the predominant share of accounts of
dynamical systems and situations have had to be formulated in terms of
the Newtonian machinery. i view this as an intellectually dangerous
position to occupy. Everything, despite it's intrinsic shape, turns
into a nail to be hit with this hammer. Recently, however, the theory
of computation has matured to the point where we have candidates for
theories of dynamics that offer very different perspective on
reasoning about dynamical systems and situations. Testing these
candidates against very successful accounts of dynamical situations,
like quantum mechanics, is going to give us some sense of how mature
they are and some measure of the quality of these accounts of
dynamics.

\subsection{Summary of contributions and outline of paper}

So, we're going to develop an interpretation of the operations of
quantum mechanics normally interpreted by Hilbert spaces and
operators. We're going to do this over a theory of computation. Note
that this is very different than the usual quantum computation program
which develops notions of computation over quantum mechanics. Rather,
we are developing a story that aligns with Wheeler's slogan: It from
Bit. To do this we will first provide an account of the theory of
computation at play here. Then we will dive into a calculation-driven
interpretation of the operations of quantum mechanics.

The reason we take this approach is that -- until very recently --
there hasn't been an axiomatic account of quantum mechanics. As a
result there has been no sharp delineation of the mathematical theory
supporting interpretation of the physical theory and the physical
theory, itself. So, ambient features of the maths are free to be
exploited (or supressed) without a real accounting of their physical
relevance. There is no sharp statement ``here's the physical theory''
qua \emph{theory} and ``here's the mathematical interpretation''
enabling a judgment of how faithful the interpretation is -- apart
from experimental observation. When there is an axiomatic account we
can judge how well a given mathematical formalism supports an
interpretation of the axioms, independent of
experimentation. Likewise, we can judge how well we have captured our
physical evidence and experience with our axiomatics, independent of
any specific mathematical implementation, with accidental detail that
may or may not have physical significance. 

In lieu of a fully fleshed out and vetted axiomatic account of quantum
mechanics, interpreting the operational notions in service of modeling
physical systems will have to suffice. In other words, we are not in
the business of providing a model of Hilbert spaces and operators. We
are in the business of providing a model of quantum mechanics because
we are motivated by testing our notions of dynamics against physical
theory; and, the predictive calculations of the physical theory must
serve as the best formulation -- shy of a fully fleshed out axiomatic
account -- of the physical theory itself (as they have for scientific
theories since time immemorial). Put another way, despite a
whole-hearted commitment to an It-from-Bit ontology, we are firmly
aligned with the shut-up-and-calculate camp as the best way to obtain
results either from the physical perspective or as a quality assurance
measure of our fledgling theory of dynamics.

In detail, we present a reflective process calculus. Then we develop
intuitive correspondences between the notions available in this
calculus and the usual physical notions supporting quantum mechanical
calculations. Thus, 

\begin{table}[htp]
  \center{
    \fbox{
      \begin{tabular}{c|c}
        quantum mechanics & process calculus \\
        \hline
        scalar & name \\
        state vector & process \\
        dual & contextual duals \\
        matrix & formal sums of process-context-dual pairs \\
        orthogonality & process annihilation \\
        inner product & execution-formula + quoting
      \end{tabular}
    }
  }
  \caption{QM - process calculi correspondences}
\end{table}

Then we tighten up these intuitions to operational definitions. We
employ the Dirac notation as the best proxy we can find for an
abstract syntax of the quantum mechanical notions. The definitions we
develop put us in contact with equational constraints coming from the
theory that we demonstrate the definitions and calculations satisfy.

This puts us in a position to shut up and calculate for the
Stern-Gerlach experimental set up, showing how these predictive
calculations become calculations on processes in our theory of a
reflective process calculus.

Penultimately, we demonstrate that the notion of metric coming from
the inner product coincides with the notion of metric available from
the theory of bisimulation. This demonstration gives us the right to
think of space as arising from behavior. Finally, we consider where we
might go from the new vantage point we have obtained.

% section introduction (end) 
 
% section introduction (end)

% \documentclass[12pt]{llncs}
%\documentclass{jktr}

\usepackage[pdftex]{hyperref}                   
\usepackage {listings}
\usepackage {mathpartir}
\usepackage{bcprules}
%\usepackage{listings}
                       
\usepackage{graphicx} 
%\usepackage[margins=2.5cm,nohead,nofoot]{geometry}
%\usepackage{geometry}
\usepackage{amsfonts}
\usepackage{amstext}
\usepackage{latexsym}
\usepackage{amssymb}
\usepackage{color}


%\include{myPreamble}
\include{qm2pi.local} 

%\ifpdf
%\usepackage[pdftex]{graphicx}
%\else
%\usepackage{graphicx}
%\fi

 % \ifpdf
%  \usepackage{pdfsync}
%  \if


%\title{Brief Article}
%\author{David F. Snyder}
%\author{L.G. Meredith}

%\address{Dept. of Math., Texas State University--San Marcos, San Marcos, TX 78666}
       
\pagestyle{empty}


\begin{document}

\lstset{language=[Objective]Caml,frame=shadowbox}

\input{qm2pi.front}

% section front matter (end)

\input{qm2pi.intro} 
 
% section introduction (end)

% \input{qm2pi.knotations} 

% section notation (end)

\input{qm2pi.process.calculi} 

% section concurrent_process_calculi_and_spatial_logics_ (end)
    
%\input{qm2pi.knots2pi} 

%\input{qm2pi.trefoil} 

%\input{qm2pi.mainthm} 

% subsection basic_interpretation (end)

%\input{qm2pi.rho.presentation} 
\subsection{The syntax and semantics of the notation system}\label{sub:the_syntax_and_semantics_of_the_notation_system} % (fold)

We now summarize a technical presentation of the calculus that
embodies our theory of dynamics. The typical presentation of such a
calculus follows the style of giving generators and relations on
them. The grammar, below, describing term constructors, freely
generates the set of processes, $\Proc$. This set is then quotiented
by a relation known as structural congruence and it is over this set
that the notion of dynamics is expressed. This presentation is
essentially that of \cite{MeredithR05} with the addition of
polyadicity and summation. For readability we have relegated some of
the technical subtleties to an appendix.

\subsubsection{Process grammar}\label{subsub:process_grammar}

\begin{mathpar}
  \inferrule* [lab=synchronization] {} {{M} \bc \pzero \;|\; x?F \;|\; x!C }
  \and
  \inferrule* [lab=abstraction] {} {{F} \bc (x)P}
  \and
  \inferrule* [lab=concretion] {} {{C} \bc \langle Q \rangle}
  \and
  \inferrule* [lab=process] {} {{P,Q} \bc M \;| \;P|Q \;|\; @{x}}
  \and
  \inferrule* [lab=name] {} {{x} \bc \quotep{P}}
\end{mathpar} 

Note that $\vec{x}$ (resp. $\vec{P}$) denotes a vector of names
(resp. processes) of length $|\vec{x}|$ (resp. $|\vec{P}|$). We adopt
the following useful abbreviations.

\begin{mathpar}
   x?(\vec{y}).P := x.(\vec{y})P \and  x\clift{\vec{P}} := x.\clift{\vec{P}}
   \and x!(y) := \lift{x}{\dropn{y}}
   \and \Pi_{i=0}^{n-1}P_i := P_0 | \ldots | P_{n-1}
\end{mathpar}

\subsubsection{Structural congruence}

\paragraph{Free and bound names and alpha-equivalence.} At the
core of structural equivalence is alpha-equivalence which identifies
process that are the same up to a change of variable. Formally, we
recognize the distinction between free and bound names. The free names
of a process, $\freenames{P}$, may be calculated recursively as
follows:

\begin{mathpar}
\freenames{\pzero} := \emptyset
  \and \\
  \freenames{x?(y).P} := \{ x \} \cup (\freenames{P} \setminus \{ y \})
  \and 
  \freenames{x!\langle P \rangle} := \{ x \} \cup \{ P \} 
  \and \\
  \freenames{P|Q} := \freenames{P} \cup \freenames{Q}
  \and \\
  \freenames{@{x}} := \{ x \}
\end{mathpar}

$\pi$
$\quotep{\pi}$

$\freenames{-} : \pi \to \mathcal{P}(\quotep{\pi})$

\begin{eqnarray*}
  \freenames{\pzero} & := & \emptyset \\
  \freenames{x?(y).P} & := & \{ x \} \cup (\freenames{P} \setminus \{ y \}) \\
  \freenames{x!\langle P \rangle} & := & \{ x \} \cup \{ P \} \\
  \freenames{P|Q} & := & \freenames{P} \cup \freenames{Q} \\
  \freenames{\dropn{x}} & := & \{ x \}
\end{eqnarray*}

The bound names of a process, $\boundnames{P}$, are those names occurring in $P$
that are not free. For example, in $x?(y).0$, the name $x$ is free, while $y$ is bound.

\begin{mathpar}
  \inferrule* [lab=monoidal-laws] {} { P|Q \equiv Q|P \and P|0 \equiv P \and P|(Q|R) \equiv (P|Q)|R }
\end{mathpar}

\begin{mathpar}
  \inferrule* [lab=alpha-equivalence] {} { (x)P \equiv (y)P\{y/x\} \and y \not\in \freenames{P} }
\end{mathpar}

\begin{definition}
Then two processes, $P,Q$, are alpha-equivalent if $P = Q\{\vec{y}/\vec{x}\}$ for
some $\vec{x} \in \boundnames{Q},\vec{y} \in \boundnames{P}$, where $Q\{\vec{y}/\vec{x}\}$
denotes the capture-avoiding substitution of $\vec{y}$ for $\vec{x}$ in $Q$.
\end{definition}

\begin{definition}
  The {\em structural congruence} \cite{SangiorgiWalker} , $\equiv$,
  between processes is the least congruence containing
  alpha-equivalence, satisfying the abelian monoid laws
  (associativity, commutativity and $\pzero$ as identity) for parallel
  composition $|$ and for summation $+$.
\end{definition}

\subsection{Name equivalence}

We take name equivalence, written $\nameeq$, to be the smallest
equivalence relation generated by the following rules.

\begin{mathpar}
\inferrule*[lab=Quote-drop]
{ }
{ \quotep{@{x}} \nameeq x }

\inferrule*[lab=Struct-equiv]
{ P \scong Q }
{ \quotep{P} \nameeq \quotep{Q} }
\end{mathpar}

The astute reader will have noticed that the mutual recursion of names
and processes imposes a mutual recursion on alpha-equivalence and
structural equivalence via name-equivalence. Fortunately, all of this
works out pleasantly and we may calculate in the natural way, free of
concern. The reader interested in the details is referred to the
appendix \ref{appendix:rho_details}.

\subsection{Substitution}

We use $\Proc$ for the set of processes, $\QProc$ for the set of
names, and $\id{\{}\vec{y} / \vec{x} \id{\}}$ to denote partial maps,
$s : \QProc \rightarrow \QProc$. A map, $s$ lifts, uniquely, to a map
on process terms, $\widehat{s} : \Proc \rightarrow \Proc$ by the
following equations.

\begin{mathpar}
  (0) \psubstp{Q}{P} := 0 \\
  (R \juxtap S) \psubstp{Q}{P}
  :=    
  (R)\psubstp{Q}{P} \juxtap (S) \psubstp{Q}{P} \\
  (x?(y).R) \psubstp{Q}{P}    
  :=    
  (x)\substp{Q}{P} (z)\concat( (R \psubstn{z}{y}) \psubstp{Q}{P} ) \\
  (\lift{x}{R}) \psubstp{Q}{P}  
  :=
  \lift{(x)\substp{Q}{P}}{ R \psubstp{Q}{P} } \\
%   (\dropn{x})  \psubstp{Q}{P}       
%   := 
%   \left\{ 
%     \begin{array}{ccc} 
%       \dropn{\quotep{Q}} & & x \nameeq \quotep{P} \\
%       \dropn{x} & & otherwise \\
%     \end{array}
%   \right. 
  (\dropn{x})  \psubstp{Q}{P}       
  := 
  \left\{ 
    \begin{array}{ccc} 
      Q & & x \nameeq \quotep{P} \\
      \dropn{x} & & otherwise \\
    \end{array}
  \right.
\end{mathpar}
 

where

\begin{eqnarray}
  (x)\id{\{} \lpquote Q \rpquote / \lpquote P \rpquote \id{\}}            = 
  \left\{ 
    \begin{array}{ccc}
      \lpquote Q \rpquote & & x \nameeq \lpquote P \rpquote \\
      x & & otherwise \\
    \end{array}
  \right. \nonumber
\end{eqnarray}

and $z$ is chosen distinct from $\quotep{P}$, $\quotep{Q}$, the free
names in $Q$, and all the names in $R$. Our $\alpha$-equivalence will
be built in the standard way from this substitution.

\begin{remark}\label{rem:no_self_referential_names}
  One consequence of these definitions is that $\forall P. \quotep{P}
  \not\in \freenames{P}$.
\end{remark}

\subsection{ Dynamic quote: an example }

Anticipating something of what's to come, consider applying the
substitution, $\widehat{\id{\{}u / z \id{\}}}$, to the following pair
of processes, $\lift{w}{y!(z)}$ and $w[ \lpquote y!(z) \rpquote ]$.

\begin{eqnarray}
	\lift{w}{y!(z)}\widehat{\id{\{}u / z \id{\}}}
		& = &
		\lift{w}{y!(u)} \nonumber\\
	w[ \lpquote y!(z) \rpquote ] \widehat{ \id{\{}u / z \id{\}} }
		& = &
		w[ \lpquote y!(z) \rpquote ] \nonumber
\end{eqnarray}

Because the body of the process between quotes is impervious to
substitution, we get radically different answers. In fact, by
examining the first process in an input context,
e.g. $x?(z).\lift{w}{y!(z)}$, we see that the process under the lift
operator may be shaped by prefixed inputs binding a name inside it. In
this sense, the lift operator will be seen as a way to dynamically
construct processes before reifying them as names.

Finally equipped with these standard features we can present the
dynamics of the calculus.

\subsubsection{Operational semantics} 

Finally, we introduce the computational dynamics. What marks these
algebras as distinct from other more traditionally studied algebraic
structures, e.g. vector spaces or polynomial rings, is the manner in
which dynamics is captured. In traditional structures, dynamics is typically
expressed through morphisms between such structures, as in linear maps
between vector spaces or morphisms between rings. In algebras
associated with the semantics of computation, the dynamics is
expressed as part of the algebraic structure itself, through a
reduction reduction relation typically denoted by $\red$. Below, we
give a recursive presentation of this relation for the calculus used
in the encoding.

$\red \subseteq \pi \times \pi$
$\red : \pi \to \mathcal{P}(\pi)$

\begin{mathpar}
  \inferrule* [lab=Comm] { \textsf{match}( x_{src}, x_{trgt} ) } { x_{trgt}?(y)P \; | \; x_{src}!\langle {Q} \rangle \red P\{\quotep{Q}/y}\} }
  \and \\
  \inferrule* [lab=Par] {{P} \red {P}'} {{{P} | {Q}} \red {{P}' | {Q}}}
  \and
  \inferrule* [lab=Equiv]{{{P} \scong {P}'} \andalso {{P}' \red {Q}'} \andalso {{Q}' \scong {Q}}}{{P} \red {Q}}
\end{mathpar}

\begin{eqnarray*}
  match_{\equiv} (\quotep{P},\quotep{Q}) & := & P \equiv Q \\
  match_{\dagger}(\quotep{P},\quotep{Q}) & := & \forall R. P|Q \red^{*} R => R \red^{*} 0 \\
  match_{K}(\quotep{P},\quotep{Q}) & := & K \mbox{ for some context } K
\end{eqnarray*}

$u?(x)P | u!\langle Q \rangle \red P\{\quotep{Q}/x\}$

%We write $\wred$ for $\red^*$, and $P\red$ if $\exists Q $ such that $ P \red Q$.
We write $P\red$ if $\exists Q $ such that $ P \red Q$ and $P\not\red$, otherwise.

\section{Replication}

As mentioned before, it is known that replication (and hence
recursion) can be implemented in a higher-order process algebra
\cite{SangiorgiWalker}. As our first example of calculation with the
machinery thus far presented we give the construction explicitly in
the {\rhoc}.

\begin{eqnarray}
	D_{x} & := & \prefix{x}{y}{(\binpar{\outputp{x}{y}}{@{y}})} \nonumber\\
	\bangp_{x}{P} & := & \binpar{{x}!\langle{\binpar{D_{x}}{P}}\rangle}{D_{x}} \nonumber
\end{eqnarray}

\begin{eqnarray}
	\bangp_{x}{P} & & \nonumber\\
	=
	& {x}!\langle{(\prefix{x}{y}{(\outputp{x}{y} | @{y})) | P}}\rangle 
	      | \prefix{x}{y}{(\outputp{x}{y} | @{y})} & \nonumber\\
	\red
	& (\outputp{x}{y} | @{y})\substn{\quotep{(\prefix{x}{y}{(@{y} | \outputp{x}{y})) | P}}}{y} & \nonumber\\
	=
	& \outputp{x}{\quotep{(\prefix{x}{y}{(\outputp{x}{y} | @{y})) | P}}}
	  | {(\prefix{x}{y}{(\outputp{x}{y} | @{y})) | P}} & \nonumber\\
	\red
	& \ldots & \nonumber\\
	\red^*
	& P | P | \ldots & \nonumber
\end{eqnarray}

Of course, this encoding, as an implementation, runs away, unfolding
$\bangp{P}$ eagerly. A lazier and more implementable replication
operator, restricted to input-guarded processes, may be obtained as follows.

\begin{eqnarray}
\bangp{\prefix{u}{v}{P}} 
	:= 
	\binpar{\lift{x}{\prefix{u}{v}{(\binpar{D(x)}{P})}}}{D(x)} \nonumber
\end{eqnarray}

\begin{remark}
  Note that the lazier definition still does not deal with summation
  or mixed summation (i.e. sums over input and output). The reader is
  invited to construct definitions of replication that deal with these
  features. 

  Further, the definitions are parameterized in a name, $x$. Can you,
  gentle reader, make a definition that eliminates this parameter and
  guarantees no accidental interaction between the replication
  machinery and the process being replicated -- i.e. no accidental
  sharing of names used by the process to get its work done and the
  name(s) used by the replication to effect copying. This latter
  revision of the definition of replication is crucial to obtaining
  the expected identity $!!P \sim !P$.
\end{remark}

\begin{remark}\label{rem:paradoxical_combinator}
  The reader familiar with the lambda calculus will have noticed the
  similarity between $D$ and the paradoxical combinator.

  [Ed. note: the existence of this seems to suggest we have to be more
  restrictive on the set of processes and names we admit if we are to
  support no-cloning.]
\end{remark}

\subsubsection{Bisimulation}

The computational dynamics gives rise to another kind of equivalence,
the equivalence of computational behavior. As previously mentioned
this is typically captured \emph{via} some form of bisimulation.

% The notion we use in this paper is weak barbed bisimulation
% \cite{milner91polyadicpi}.

The notion we use in this paper is derived from weak barbed
bisimulation \cite{milner91polyadicpi}. 

\begin{definition}
An \emph{observation relation}, $\downarrow_{\mathcal N}$, over a set
of names, $\mathcal N$, is the smallest relation satisfying the rules
below.

\infrule[Out-barb]{y \in {\mathcal N}, \; x \nameeq y}
		  {\outputp{x}{v} \downarrow_{\mathcal N} x}
\infrule[Par-barb]{\mbox{$P\downarrow_{\mathcal N} x$ or $Q\downarrow_{\mathcal N} x$}}
		  {\binpar{P}{Q} \downarrow_{\mathcal N} x}

We write $P \Downarrow_{\mathcal N} x$ if there is $Q$ such that 
$P \wred Q$ and $Q \downarrow_{\mathcal N} x$.
\end{definition}

\begin{definition}
%\label{def.bbisim}
An  ${\mathcal N}$-\emph{barbed bisimulation} over a set of names, ${\mathcal N}$, is a symmetric binary relation 
${\mathcal S}_{\mathcal N}$ between agents such that $P\rel{S}_{\mathcal N}Q$ implies:
\begin{enumerate}
\item If $P \red P'$ then $Q \wred Q'$ and $P'\rel{S}_{\mathcal N} Q'$.
\item If $P\downarrow_{\mathcal N} x$, then $Q\Downarrow_{\mathcal N} x$.
\end{enumerate}
$P$ is ${\mathcal N}$-barbed bisimilar to $Q$, written
$P \wbbisim_{\mathcal N} Q$, if $P \rel{S}_{\mathcal N} Q$ for some ${\mathcal N}$-barbed bisimulation ${\mathcal S}_{\mathcal N}$.
\end{definition}

$\mathcal{R} \subseteq \pi \times \pi$

$P \mathcal{R} Q => \forall P'. P \red P' \Rightarrow \exists Q'. Q \red Q', P' \mathcal{R} Q'$

$P \vdash x \Rightarrow Q \vdash x$

\begin{mathpar}
  \inferrule*[lab=Out-barb]{x \nameeq y}{{y}!\langle{Q}\rangle \vdash x}
  \and
  \inferrule*[lab=Par-barb]{\mbox{$P\vdash x$ or $Q\vdash x$}}{\binpar{P}{Q} \vdash x}
\end{mathpar}

\subsubsection{Contexts}

One of the principle advantages of computational calculi like the
$\pi$-calculus is a well-defined notion of context,
contextual-equivalence and a correlation between
contextual-equivalence and notions of bisimulation. The notion of
context allows the decomposition of a process into (sub-)process and
its syntactic environment, its context. Thus, a context may be
thought of as a process with a ``hole'' (written $\Box$) in it. The
application of a context $M$ to a process $P$, written $M[P]$, is
tantamount to filling the hole in $M$ with $P$. In this paper we do
not need the full weight of this theory, but do make use of the notion
of context in the proof the main theorem. 

\begin{mathpar}
  \inferrule* [lab=summation] {} {{M_{M},M_{N}} \bc \Box \;|\; x.M_{A} \;|\; M_{M}+M_{N}}
  \and
  \inferrule* [lab=agent] {} {{M_{A}} \bc (\vec{x})M_{P} \;| \; \clift{P_0,\ldots,M_{P},\ldots,P_N}}
  \and \\
  \inferrule* [lab=process] {} {{M_{P}} \bc M_{N} \;| \;P|M_{P} }
\end{mathpar} 

\begin{mathpar}
  \inferrule* [lab=sychronization] {} {M_{N} \bc \Box \;|\; x?M_{F} \;|\; x!M_{C}}
  \and
  \inferrule* [lab=abstraction] {} {{M_{F}} \bc (x)M_{P} }
  \and
  \inferrule* [lab=concretion] {} {{M_{C}} \bc \langle M_{P} \rangle }
  \and \\
  \inferrule* [lab=process] {} {{M_{P}} \bc M_{N} \;| \;P|M_{P} }
\end{mathpar}

\begin{definition}[contextual application] Given a context $M$, and
  process $P$, we define the \emph{contextual application}, $M[P] :=
  M\{P/\Box\}$. That is, the contextual application of M to P is the
  substitution of $P$ for $\Box$ in $M$.
\end{definition}

$\meaningof{-} : L \to \mathcal{P}(\pi)$

\begin{mathpar}
  \inferrule* [lab=collection] {} {\meaningof{true} = \pi, \and \meaningof{~E} = \pi \setminus \meaningof{E}, \and \meaningof{E_{1} \& E_{2}} = \meaningof{E_{1}} \cap \meaningof{E_{2}}}
\end{mathpar}

\begin{mathpar}
  \inferrule* [lab=structure] {} {\meaningof{0} = \{ P \in \pi | P \equiv 0 \}, \and \\ \meaningof{E_1 | E_2} = \{ P \in \pi | P \equiv P_{1} | P_{2}, P_{1} \in \meaningof{E_{1}}, P_{2} \in \meaningof{E_2}\} }
\end{mathpar}

\begin{mathpar}
 \inferrule* [lab=behavior] {} {\meaningof{\langle a?b \rangle E} = \{ P \in \pi | P \equiv Q | u?(y)P', \\ \and \\\\ \and \\ \;\;\; u \in \meaningof{a}, \forall z.P'\{z/y\} \in \meaningof{E\{z/b\}}\}, \and \\ \meaningof{a!E} = \{ P \in \pi | P \equiv Q | x!\langle P' \rangle, x \in \meaningof{a} P' \in \meaningof{E}\} }
\end{mathpar}

\begin{mathpar}
 \inferrule* [lab=nominal] {} {\meaningof{\quotep{E}} = \{ \quotep{P} \in \quotep{\pi} | P \in \meaningof{E} \}, \and \meaningof{\quotep{P}} = \{ \quotep{Q} \in \quotep{\pi} | P \equiv Q \} \and \\ \meaningof{@\quotep{E}} = \{ P \in \pi | P \equiv @x, x \in \meaningof{E} \}}
\end{mathpar}

\begin{eqnarray*}
  \\
  \meaningof{-} : TS \to ST
\end{eqnarray*}

\begin{eqnarray*}
  \\
  L : TS \to ST
\end{eqnarray*}

\begin{eqnarray*}
  \\
  P \models E \iff P \in \meaningof{E}
\end{eqnarray*}

\begin{eqnarray*}
  P \approx_{L} Q \iff \forall E \in L. P \models E \iff Q \models E
\end{eqnarray*}

\begin{eqnarray*}
  P \approx_{K} Q
\end{eqnarray*}

\begin{eqnarray*}
  P \approx Q
\end{eqnarray*}

$\approx_{K} = \approx = \approx_{L}$

\subsubsection{Contextual duality}

Note that contexts extend the quotation operation to a family of
operations from processes to names. Given a context, $M$, we can
define a \emph{nominal context}, $\quotep{M}$ by $\quotep{M}[P] :=
\quotep{M[P]}$. To foreshadow what is to come we observe that these
operations enjoy a duality with processes very much like the duality
between vectors and maps from vectors to scalars.

Further, because the calculus is essentially higher-order, we have a
correspondence between contexts and processes. More specifically,
given a name $x$ and a context $M$ we can construct $M^{*}_{x}$ such
that 

\begin{mathpar}
  M^{*}_{x} | \lift{x}{P} \red M[P]
\end{mathpar}

namely,

\begin{mathpar}
  M^{*}_{x} := x?(u).M[\dropn{u}]
\end{mathpar}

The dependence of $M^{*}_{x}$ on a name makes it an abstraction, 

\begin{mathpar}
  M^{*} := (x)x?(u).M[\dropn{u}]
\end{mathpar}

\subsection{Additional notation}

It will sometimes be convenient to denote the process a name
quotes. We already have the notation $x = \quotep{P}$, but it will be
convenient to introduce an alternate notation, $\procn{x}$, when we
want to emphasize the connection to the use of the name. Note that, by
virtue of name equivalence, $\quotep{\procn{x}} \nameeq x$; so, the
notation is consistent with previous definitions.

Further, because names have structure it is possible to effect
substitutions on the basis of that structure. This means we need to
upgrade our notation for substitutions, which we accomplish by
adapting comprehension notation. Thus,

\begin{mathpar}
  P\{ y / x : x \in S \}
\end{mathpar}

is interpreted to mean the process derived from P by replacing (in a
capture-avoiding manner) each occurrence of $x$ in $S$ by $y$. For example,

\begin{mathpar}
  P\{ \quotep{\procn{x}|\procn{x}} / x : x \in \freenames{P} \}
\end{mathpar}

will replace each (occurrence) of a free name $x$ in $P$ by
$\quotep{\procn{x}|\procn{x}}$.

Also, we will avail ourselves of the notation $x^{L}$ and $x^{R}$ to
denote injections of a name into disjoint copies of the name
space. There are numerous ways to accomplish this. One example can be
found in \cite{MeredithR05}. This notation overloads to vectors of
names: $\vec{x}^{\pi} := (x_{i}^{\pi} \; : \; 0 \leq i < |\vec{x}| )$ where $\pi \in \{L,R\}$.

We also use $P^{\Box} := P|\Box$.

In \cite{MeredithR05} an interpretation of the new operator is
given. It turns out that there are several possible interpretations
all enjoying the requisite algebraic properties of the operator (see
\cite{milner91polyadicpi}). We will therefore make liberal use of
$(\nu\; \vec{x})P$.

% subsection the_syntax_and_semantics_of_the_notation_system (end)   

\input{qm2pi.qmops} 

\input{qm2pi.sterngerlach} 

\input{qm2pi.metric} 

% section concurrent_process_calculi (end)

%\input{qm2pi.proofsketch}

% section proof sketch (end)

%\input{qm2pi.slviaknots} 

% section spatial logic via knots (end)

\input{qm2pi.conclusion}

% section conclusion (end)

%\input{qm2pi.dtcodes} 

% section wiring algorithm (end)

\input{qm2pi.ack} 

% section acknowledgments (end)

\newpage


\bibliographystyle{plain}   
\bibliography{../../biblios/main.bib}

\input{qm2pi.rhodetails}

\end{document}

 

% section notation (end)

\input{qm2pi.process.calculi} 

% section concurrent_process_calculi_and_spatial_logics_ (end)
    
%\documentclass[12pt]{llncs}
%\documentclass{jktr}

\usepackage[pdftex]{hyperref}                   
\usepackage {listings}
\usepackage {mathpartir}
\usepackage{bcprules}
%\usepackage{listings}
                       
\usepackage{graphicx} 
%\usepackage[margins=2.5cm,nohead,nofoot]{geometry}
%\usepackage{geometry}
\usepackage{amsfonts}
\usepackage{amstext}
\usepackage{latexsym}
\usepackage{amssymb}
\usepackage{color}


%\include{myPreamble}
\include{qm2pi.local} 

%\ifpdf
%\usepackage[pdftex]{graphicx}
%\else
%\usepackage{graphicx}
%\fi

 % \ifpdf
%  \usepackage{pdfsync}
%  \if


%\title{Brief Article}
%\author{David F. Snyder}
%\author{L.G. Meredith}

%\address{Dept. of Math., Texas State University--San Marcos, San Marcos, TX 78666}
       
\pagestyle{empty}


\begin{document}

\lstset{language=[Objective]Caml,frame=shadowbox}

\input{qm2pi.front}

% section front matter (end)

\input{qm2pi.intro} 
 
% section introduction (end)

% \input{qm2pi.knotations} 

% section notation (end)

\input{qm2pi.process.calculi} 

% section concurrent_process_calculi_and_spatial_logics_ (end)
    
%\input{qm2pi.knots2pi} 

%\input{qm2pi.trefoil} 

%\input{qm2pi.mainthm} 

% subsection basic_interpretation (end)

%\input{qm2pi.rho.presentation} 
\subsection{The syntax and semantics of the notation system}\label{sub:the_syntax_and_semantics_of_the_notation_system} % (fold)

We now summarize a technical presentation of the calculus that
embodies our theory of dynamics. The typical presentation of such a
calculus follows the style of giving generators and relations on
them. The grammar, below, describing term constructors, freely
generates the set of processes, $\Proc$. This set is then quotiented
by a relation known as structural congruence and it is over this set
that the notion of dynamics is expressed. This presentation is
essentially that of \cite{MeredithR05} with the addition of
polyadicity and summation. For readability we have relegated some of
the technical subtleties to an appendix.

\subsubsection{Process grammar}\label{subsub:process_grammar}

\begin{mathpar}
  \inferrule* [lab=synchronization] {} {{M} \bc \pzero \;|\; x?F \;|\; x!C }
  \and
  \inferrule* [lab=abstraction] {} {{F} \bc (x)P}
  \and
  \inferrule* [lab=concretion] {} {{C} \bc \langle Q \rangle}
  \and
  \inferrule* [lab=process] {} {{P,Q} \bc M \;| \;P|Q \;|\; @{x}}
  \and
  \inferrule* [lab=name] {} {{x} \bc \quotep{P}}
\end{mathpar} 

Note that $\vec{x}$ (resp. $\vec{P}$) denotes a vector of names
(resp. processes) of length $|\vec{x}|$ (resp. $|\vec{P}|$). We adopt
the following useful abbreviations.

\begin{mathpar}
   x?(\vec{y}).P := x.(\vec{y})P \and  x\clift{\vec{P}} := x.\clift{\vec{P}}
   \and x!(y) := \lift{x}{\dropn{y}}
   \and \Pi_{i=0}^{n-1}P_i := P_0 | \ldots | P_{n-1}
\end{mathpar}

\subsubsection{Structural congruence}

\paragraph{Free and bound names and alpha-equivalence.} At the
core of structural equivalence is alpha-equivalence which identifies
process that are the same up to a change of variable. Formally, we
recognize the distinction between free and bound names. The free names
of a process, $\freenames{P}$, may be calculated recursively as
follows:

\begin{mathpar}
\freenames{\pzero} := \emptyset
  \and \\
  \freenames{x?(y).P} := \{ x \} \cup (\freenames{P} \setminus \{ y \})
  \and 
  \freenames{x!\langle P \rangle} := \{ x \} \cup \{ P \} 
  \and \\
  \freenames{P|Q} := \freenames{P} \cup \freenames{Q}
  \and \\
  \freenames{@{x}} := \{ x \}
\end{mathpar}

$\pi$
$\quotep{\pi}$

$\freenames{-} : \pi \to \mathcal{P}(\quotep{\pi})$

\begin{eqnarray*}
  \freenames{\pzero} & := & \emptyset \\
  \freenames{x?(y).P} & := & \{ x \} \cup (\freenames{P} \setminus \{ y \}) \\
  \freenames{x!\langle P \rangle} & := & \{ x \} \cup \{ P \} \\
  \freenames{P|Q} & := & \freenames{P} \cup \freenames{Q} \\
  \freenames{\dropn{x}} & := & \{ x \}
\end{eqnarray*}

The bound names of a process, $\boundnames{P}$, are those names occurring in $P$
that are not free. For example, in $x?(y).0$, the name $x$ is free, while $y$ is bound.

\begin{mathpar}
  \inferrule* [lab=monoidal-laws] {} { P|Q \equiv Q|P \and P|0 \equiv P \and P|(Q|R) \equiv (P|Q)|R }
\end{mathpar}

\begin{mathpar}
  \inferrule* [lab=alpha-equivalence] {} { (x)P \equiv (y)P\{y/x\} \and y \not\in \freenames{P} }
\end{mathpar}

\begin{definition}
Then two processes, $P,Q$, are alpha-equivalent if $P = Q\{\vec{y}/\vec{x}\}$ for
some $\vec{x} \in \boundnames{Q},\vec{y} \in \boundnames{P}$, where $Q\{\vec{y}/\vec{x}\}$
denotes the capture-avoiding substitution of $\vec{y}$ for $\vec{x}$ in $Q$.
\end{definition}

\begin{definition}
  The {\em structural congruence} \cite{SangiorgiWalker} , $\equiv$,
  between processes is the least congruence containing
  alpha-equivalence, satisfying the abelian monoid laws
  (associativity, commutativity and $\pzero$ as identity) for parallel
  composition $|$ and for summation $+$.
\end{definition}

\subsection{Name equivalence}

We take name equivalence, written $\nameeq$, to be the smallest
equivalence relation generated by the following rules.

\begin{mathpar}
\inferrule*[lab=Quote-drop]
{ }
{ \quotep{@{x}} \nameeq x }

\inferrule*[lab=Struct-equiv]
{ P \scong Q }
{ \quotep{P} \nameeq \quotep{Q} }
\end{mathpar}

The astute reader will have noticed that the mutual recursion of names
and processes imposes a mutual recursion on alpha-equivalence and
structural equivalence via name-equivalence. Fortunately, all of this
works out pleasantly and we may calculate in the natural way, free of
concern. The reader interested in the details is referred to the
appendix \ref{appendix:rho_details}.

\subsection{Substitution}

We use $\Proc$ for the set of processes, $\QProc$ for the set of
names, and $\id{\{}\vec{y} / \vec{x} \id{\}}$ to denote partial maps,
$s : \QProc \rightarrow \QProc$. A map, $s$ lifts, uniquely, to a map
on process terms, $\widehat{s} : \Proc \rightarrow \Proc$ by the
following equations.

\begin{mathpar}
  (0) \psubstp{Q}{P} := 0 \\
  (R \juxtap S) \psubstp{Q}{P}
  :=    
  (R)\psubstp{Q}{P} \juxtap (S) \psubstp{Q}{P} \\
  (x?(y).R) \psubstp{Q}{P}    
  :=    
  (x)\substp{Q}{P} (z)\concat( (R \psubstn{z}{y}) \psubstp{Q}{P} ) \\
  (\lift{x}{R}) \psubstp{Q}{P}  
  :=
  \lift{(x)\substp{Q}{P}}{ R \psubstp{Q}{P} } \\
%   (\dropn{x})  \psubstp{Q}{P}       
%   := 
%   \left\{ 
%     \begin{array}{ccc} 
%       \dropn{\quotep{Q}} & & x \nameeq \quotep{P} \\
%       \dropn{x} & & otherwise \\
%     \end{array}
%   \right. 
  (\dropn{x})  \psubstp{Q}{P}       
  := 
  \left\{ 
    \begin{array}{ccc} 
      Q & & x \nameeq \quotep{P} \\
      \dropn{x} & & otherwise \\
    \end{array}
  \right.
\end{mathpar}
 

where

\begin{eqnarray}
  (x)\id{\{} \lpquote Q \rpquote / \lpquote P \rpquote \id{\}}            = 
  \left\{ 
    \begin{array}{ccc}
      \lpquote Q \rpquote & & x \nameeq \lpquote P \rpquote \\
      x & & otherwise \\
    \end{array}
  \right. \nonumber
\end{eqnarray}

and $z$ is chosen distinct from $\quotep{P}$, $\quotep{Q}$, the free
names in $Q$, and all the names in $R$. Our $\alpha$-equivalence will
be built in the standard way from this substitution.

\begin{remark}\label{rem:no_self_referential_names}
  One consequence of these definitions is that $\forall P. \quotep{P}
  \not\in \freenames{P}$.
\end{remark}

\subsection{ Dynamic quote: an example }

Anticipating something of what's to come, consider applying the
substitution, $\widehat{\id{\{}u / z \id{\}}}$, to the following pair
of processes, $\lift{w}{y!(z)}$ and $w[ \lpquote y!(z) \rpquote ]$.

\begin{eqnarray}
	\lift{w}{y!(z)}\widehat{\id{\{}u / z \id{\}}}
		& = &
		\lift{w}{y!(u)} \nonumber\\
	w[ \lpquote y!(z) \rpquote ] \widehat{ \id{\{}u / z \id{\}} }
		& = &
		w[ \lpquote y!(z) \rpquote ] \nonumber
\end{eqnarray}

Because the body of the process between quotes is impervious to
substitution, we get radically different answers. In fact, by
examining the first process in an input context,
e.g. $x?(z).\lift{w}{y!(z)}$, we see that the process under the lift
operator may be shaped by prefixed inputs binding a name inside it. In
this sense, the lift operator will be seen as a way to dynamically
construct processes before reifying them as names.

Finally equipped with these standard features we can present the
dynamics of the calculus.

\subsubsection{Operational semantics} 

Finally, we introduce the computational dynamics. What marks these
algebras as distinct from other more traditionally studied algebraic
structures, e.g. vector spaces or polynomial rings, is the manner in
which dynamics is captured. In traditional structures, dynamics is typically
expressed through morphisms between such structures, as in linear maps
between vector spaces or morphisms between rings. In algebras
associated with the semantics of computation, the dynamics is
expressed as part of the algebraic structure itself, through a
reduction reduction relation typically denoted by $\red$. Below, we
give a recursive presentation of this relation for the calculus used
in the encoding.

$\red \subseteq \pi \times \pi$
$\red : \pi \to \mathcal{P}(\pi)$

\begin{mathpar}
  \inferrule* [lab=Comm] { \textsf{match}( x_{src}, x_{trgt} ) } { x_{trgt}?(y)P \; | \; x_{src}!\langle {Q} \rangle \red P\{\quotep{Q}/y}\} }
  \and \\
  \inferrule* [lab=Par] {{P} \red {P}'} {{{P} | {Q}} \red {{P}' | {Q}}}
  \and
  \inferrule* [lab=Equiv]{{{P} \scong {P}'} \andalso {{P}' \red {Q}'} \andalso {{Q}' \scong {Q}}}{{P} \red {Q}}
\end{mathpar}

\begin{eqnarray*}
  match_{\equiv} (\quotep{P},\quotep{Q}) & := & P \equiv Q \\
  match_{\dagger}(\quotep{P},\quotep{Q}) & := & \forall R. P|Q \red^{*} R => R \red^{*} 0 \\
  match_{K}(\quotep{P},\quotep{Q}) & := & K \mbox{ for some context } K
\end{eqnarray*}

$u?(x)P | u!\langle Q \rangle \red P\{\quotep{Q}/x\}$

%We write $\wred$ for $\red^*$, and $P\red$ if $\exists Q $ such that $ P \red Q$.
We write $P\red$ if $\exists Q $ such that $ P \red Q$ and $P\not\red$, otherwise.

\section{Replication}

As mentioned before, it is known that replication (and hence
recursion) can be implemented in a higher-order process algebra
\cite{SangiorgiWalker}. As our first example of calculation with the
machinery thus far presented we give the construction explicitly in
the {\rhoc}.

\begin{eqnarray}
	D_{x} & := & \prefix{x}{y}{(\binpar{\outputp{x}{y}}{@{y}})} \nonumber\\
	\bangp_{x}{P} & := & \binpar{{x}!\langle{\binpar{D_{x}}{P}}\rangle}{D_{x}} \nonumber
\end{eqnarray}

\begin{eqnarray}
	\bangp_{x}{P} & & \nonumber\\
	=
	& {x}!\langle{(\prefix{x}{y}{(\outputp{x}{y} | @{y})) | P}}\rangle 
	      | \prefix{x}{y}{(\outputp{x}{y} | @{y})} & \nonumber\\
	\red
	& (\outputp{x}{y} | @{y})\substn{\quotep{(\prefix{x}{y}{(@{y} | \outputp{x}{y})) | P}}}{y} & \nonumber\\
	=
	& \outputp{x}{\quotep{(\prefix{x}{y}{(\outputp{x}{y} | @{y})) | P}}}
	  | {(\prefix{x}{y}{(\outputp{x}{y} | @{y})) | P}} & \nonumber\\
	\red
	& \ldots & \nonumber\\
	\red^*
	& P | P | \ldots & \nonumber
\end{eqnarray}

Of course, this encoding, as an implementation, runs away, unfolding
$\bangp{P}$ eagerly. A lazier and more implementable replication
operator, restricted to input-guarded processes, may be obtained as follows.

\begin{eqnarray}
\bangp{\prefix{u}{v}{P}} 
	:= 
	\binpar{\lift{x}{\prefix{u}{v}{(\binpar{D(x)}{P})}}}{D(x)} \nonumber
\end{eqnarray}

\begin{remark}
  Note that the lazier definition still does not deal with summation
  or mixed summation (i.e. sums over input and output). The reader is
  invited to construct definitions of replication that deal with these
  features. 

  Further, the definitions are parameterized in a name, $x$. Can you,
  gentle reader, make a definition that eliminates this parameter and
  guarantees no accidental interaction between the replication
  machinery and the process being replicated -- i.e. no accidental
  sharing of names used by the process to get its work done and the
  name(s) used by the replication to effect copying. This latter
  revision of the definition of replication is crucial to obtaining
  the expected identity $!!P \sim !P$.
\end{remark}

\begin{remark}\label{rem:paradoxical_combinator}
  The reader familiar with the lambda calculus will have noticed the
  similarity between $D$ and the paradoxical combinator.

  [Ed. note: the existence of this seems to suggest we have to be more
  restrictive on the set of processes and names we admit if we are to
  support no-cloning.]
\end{remark}

\subsubsection{Bisimulation}

The computational dynamics gives rise to another kind of equivalence,
the equivalence of computational behavior. As previously mentioned
this is typically captured \emph{via} some form of bisimulation.

% The notion we use in this paper is weak barbed bisimulation
% \cite{milner91polyadicpi}.

The notion we use in this paper is derived from weak barbed
bisimulation \cite{milner91polyadicpi}. 

\begin{definition}
An \emph{observation relation}, $\downarrow_{\mathcal N}$, over a set
of names, $\mathcal N$, is the smallest relation satisfying the rules
below.

\infrule[Out-barb]{y \in {\mathcal N}, \; x \nameeq y}
		  {\outputp{x}{v} \downarrow_{\mathcal N} x}
\infrule[Par-barb]{\mbox{$P\downarrow_{\mathcal N} x$ or $Q\downarrow_{\mathcal N} x$}}
		  {\binpar{P}{Q} \downarrow_{\mathcal N} x}

We write $P \Downarrow_{\mathcal N} x$ if there is $Q$ such that 
$P \wred Q$ and $Q \downarrow_{\mathcal N} x$.
\end{definition}

\begin{definition}
%\label{def.bbisim}
An  ${\mathcal N}$-\emph{barbed bisimulation} over a set of names, ${\mathcal N}$, is a symmetric binary relation 
${\mathcal S}_{\mathcal N}$ between agents such that $P\rel{S}_{\mathcal N}Q$ implies:
\begin{enumerate}
\item If $P \red P'$ then $Q \wred Q'$ and $P'\rel{S}_{\mathcal N} Q'$.
\item If $P\downarrow_{\mathcal N} x$, then $Q\Downarrow_{\mathcal N} x$.
\end{enumerate}
$P$ is ${\mathcal N}$-barbed bisimilar to $Q$, written
$P \wbbisim_{\mathcal N} Q$, if $P \rel{S}_{\mathcal N} Q$ for some ${\mathcal N}$-barbed bisimulation ${\mathcal S}_{\mathcal N}$.
\end{definition}

$\mathcal{R} \subseteq \pi \times \pi$

$P \mathcal{R} Q => \forall P'. P \red P' \Rightarrow \exists Q'. Q \red Q', P' \mathcal{R} Q'$

$P \vdash x \Rightarrow Q \vdash x$

\begin{mathpar}
  \inferrule*[lab=Out-barb]{x \nameeq y}{{y}!\langle{Q}\rangle \vdash x}
  \and
  \inferrule*[lab=Par-barb]{\mbox{$P\vdash x$ or $Q\vdash x$}}{\binpar{P}{Q} \vdash x}
\end{mathpar}

\subsubsection{Contexts}

One of the principle advantages of computational calculi like the
$\pi$-calculus is a well-defined notion of context,
contextual-equivalence and a correlation between
contextual-equivalence and notions of bisimulation. The notion of
context allows the decomposition of a process into (sub-)process and
its syntactic environment, its context. Thus, a context may be
thought of as a process with a ``hole'' (written $\Box$) in it. The
application of a context $M$ to a process $P$, written $M[P]$, is
tantamount to filling the hole in $M$ with $P$. In this paper we do
not need the full weight of this theory, but do make use of the notion
of context in the proof the main theorem. 

\begin{mathpar}
  \inferrule* [lab=summation] {} {{M_{M},M_{N}} \bc \Box \;|\; x.M_{A} \;|\; M_{M}+M_{N}}
  \and
  \inferrule* [lab=agent] {} {{M_{A}} \bc (\vec{x})M_{P} \;| \; \clift{P_0,\ldots,M_{P},\ldots,P_N}}
  \and \\
  \inferrule* [lab=process] {} {{M_{P}} \bc M_{N} \;| \;P|M_{P} }
\end{mathpar} 

\begin{mathpar}
  \inferrule* [lab=sychronization] {} {M_{N} \bc \Box \;|\; x?M_{F} \;|\; x!M_{C}}
  \and
  \inferrule* [lab=abstraction] {} {{M_{F}} \bc (x)M_{P} }
  \and
  \inferrule* [lab=concretion] {} {{M_{C}} \bc \langle M_{P} \rangle }
  \and \\
  \inferrule* [lab=process] {} {{M_{P}} \bc M_{N} \;| \;P|M_{P} }
\end{mathpar}

\begin{definition}[contextual application] Given a context $M$, and
  process $P$, we define the \emph{contextual application}, $M[P] :=
  M\{P/\Box\}$. That is, the contextual application of M to P is the
  substitution of $P$ for $\Box$ in $M$.
\end{definition}

$\meaningof{-} : L \to \mathcal{P}(\pi)$

\begin{mathpar}
  \inferrule* [lab=collection] {} {\meaningof{true} = \pi, \and \meaningof{~E} = \pi \setminus \meaningof{E}, \and \meaningof{E_{1} \& E_{2}} = \meaningof{E_{1}} \cap \meaningof{E_{2}}}
\end{mathpar}

\begin{mathpar}
  \inferrule* [lab=structure] {} {\meaningof{0} = \{ P \in \pi | P \equiv 0 \}, \and \\ \meaningof{E_1 | E_2} = \{ P \in \pi | P \equiv P_{1} | P_{2}, P_{1} \in \meaningof{E_{1}}, P_{2} \in \meaningof{E_2}\} }
\end{mathpar}

\begin{mathpar}
 \inferrule* [lab=behavior] {} {\meaningof{\langle a?b \rangle E} = \{ P \in \pi | P \equiv Q | u?(y)P', \\ \and \\\\ \and \\ \;\;\; u \in \meaningof{a}, \forall z.P'\{z/y\} \in \meaningof{E\{z/b\}}\}, \and \\ \meaningof{a!E} = \{ P \in \pi | P \equiv Q | x!\langle P' \rangle, x \in \meaningof{a} P' \in \meaningof{E}\} }
\end{mathpar}

\begin{mathpar}
 \inferrule* [lab=nominal] {} {\meaningof{\quotep{E}} = \{ \quotep{P} \in \quotep{\pi} | P \in \meaningof{E} \}, \and \meaningof{\quotep{P}} = \{ \quotep{Q} \in \quotep{\pi} | P \equiv Q \} \and \\ \meaningof{@\quotep{E}} = \{ P \in \pi | P \equiv @x, x \in \meaningof{E} \}}
\end{mathpar}

\begin{eqnarray*}
  \\
  \meaningof{-} : TS \to ST
\end{eqnarray*}

\begin{eqnarray*}
  \\
  L : TS \to ST
\end{eqnarray*}

\begin{eqnarray*}
  \\
  P \models E \iff P \in \meaningof{E}
\end{eqnarray*}

\begin{eqnarray*}
  P \approx_{L} Q \iff \forall E \in L. P \models E \iff Q \models E
\end{eqnarray*}

\begin{eqnarray*}
  P \approx_{K} Q
\end{eqnarray*}

\begin{eqnarray*}
  P \approx Q
\end{eqnarray*}

$\approx_{K} = \approx = \approx_{L}$

\subsubsection{Contextual duality}

Note that contexts extend the quotation operation to a family of
operations from processes to names. Given a context, $M$, we can
define a \emph{nominal context}, $\quotep{M}$ by $\quotep{M}[P] :=
\quotep{M[P]}$. To foreshadow what is to come we observe that these
operations enjoy a duality with processes very much like the duality
between vectors and maps from vectors to scalars.

Further, because the calculus is essentially higher-order, we have a
correspondence between contexts and processes. More specifically,
given a name $x$ and a context $M$ we can construct $M^{*}_{x}$ such
that 

\begin{mathpar}
  M^{*}_{x} | \lift{x}{P} \red M[P]
\end{mathpar}

namely,

\begin{mathpar}
  M^{*}_{x} := x?(u).M[\dropn{u}]
\end{mathpar}

The dependence of $M^{*}_{x}$ on a name makes it an abstraction, 

\begin{mathpar}
  M^{*} := (x)x?(u).M[\dropn{u}]
\end{mathpar}

\subsection{Additional notation}

It will sometimes be convenient to denote the process a name
quotes. We already have the notation $x = \quotep{P}$, but it will be
convenient to introduce an alternate notation, $\procn{x}$, when we
want to emphasize the connection to the use of the name. Note that, by
virtue of name equivalence, $\quotep{\procn{x}} \nameeq x$; so, the
notation is consistent with previous definitions.

Further, because names have structure it is possible to effect
substitutions on the basis of that structure. This means we need to
upgrade our notation for substitutions, which we accomplish by
adapting comprehension notation. Thus,

\begin{mathpar}
  P\{ y / x : x \in S \}
\end{mathpar}

is interpreted to mean the process derived from P by replacing (in a
capture-avoiding manner) each occurrence of $x$ in $S$ by $y$. For example,

\begin{mathpar}
  P\{ \quotep{\procn{x}|\procn{x}} / x : x \in \freenames{P} \}
\end{mathpar}

will replace each (occurrence) of a free name $x$ in $P$ by
$\quotep{\procn{x}|\procn{x}}$.

Also, we will avail ourselves of the notation $x^{L}$ and $x^{R}$ to
denote injections of a name into disjoint copies of the name
space. There are numerous ways to accomplish this. One example can be
found in \cite{MeredithR05}. This notation overloads to vectors of
names: $\vec{x}^{\pi} := (x_{i}^{\pi} \; : \; 0 \leq i < |\vec{x}| )$ where $\pi \in \{L,R\}$.

We also use $P^{\Box} := P|\Box$.

In \cite{MeredithR05} an interpretation of the new operator is
given. It turns out that there are several possible interpretations
all enjoying the requisite algebraic properties of the operator (see
\cite{milner91polyadicpi}). We will therefore make liberal use of
$(\nu\; \vec{x})P$.

% subsection the_syntax_and_semantics_of_the_notation_system (end)   

\input{qm2pi.qmops} 

\input{qm2pi.sterngerlach} 

\input{qm2pi.metric} 

% section concurrent_process_calculi (end)

%\input{qm2pi.proofsketch}

% section proof sketch (end)

%\input{qm2pi.slviaknots} 

% section spatial logic via knots (end)

\input{qm2pi.conclusion}

% section conclusion (end)

%\input{qm2pi.dtcodes} 

% section wiring algorithm (end)

\input{qm2pi.ack} 

% section acknowledgments (end)

\newpage


\bibliographystyle{plain}   
\bibliography{../../biblios/main.bib}

\input{qm2pi.rhodetails}

\end{document}

 

%\documentclass[12pt]{llncs}
%\documentclass{jktr}

\usepackage[pdftex]{hyperref}                   
\usepackage {listings}
\usepackage {mathpartir}
\usepackage{bcprules}
%\usepackage{listings}
                       
\usepackage{graphicx} 
%\usepackage[margins=2.5cm,nohead,nofoot]{geometry}
%\usepackage{geometry}
\usepackage{amsfonts}
\usepackage{amstext}
\usepackage{latexsym}
\usepackage{amssymb}
\usepackage{color}


%\include{myPreamble}
\include{qm2pi.local} 

%\ifpdf
%\usepackage[pdftex]{graphicx}
%\else
%\usepackage{graphicx}
%\fi

 % \ifpdf
%  \usepackage{pdfsync}
%  \if


%\title{Brief Article}
%\author{David F. Snyder}
%\author{L.G. Meredith}

%\address{Dept. of Math., Texas State University--San Marcos, San Marcos, TX 78666}
       
\pagestyle{empty}


\begin{document}

\lstset{language=[Objective]Caml,frame=shadowbox}

\input{qm2pi.front}

% section front matter (end)

\input{qm2pi.intro} 
 
% section introduction (end)

% \input{qm2pi.knotations} 

% section notation (end)

\input{qm2pi.process.calculi} 

% section concurrent_process_calculi_and_spatial_logics_ (end)
    
%\input{qm2pi.knots2pi} 

%\input{qm2pi.trefoil} 

%\input{qm2pi.mainthm} 

% subsection basic_interpretation (end)

%\input{qm2pi.rho.presentation} 
\subsection{The syntax and semantics of the notation system}\label{sub:the_syntax_and_semantics_of_the_notation_system} % (fold)

We now summarize a technical presentation of the calculus that
embodies our theory of dynamics. The typical presentation of such a
calculus follows the style of giving generators and relations on
them. The grammar, below, describing term constructors, freely
generates the set of processes, $\Proc$. This set is then quotiented
by a relation known as structural congruence and it is over this set
that the notion of dynamics is expressed. This presentation is
essentially that of \cite{MeredithR05} with the addition of
polyadicity and summation. For readability we have relegated some of
the technical subtleties to an appendix.

\subsubsection{Process grammar}\label{subsub:process_grammar}

\begin{mathpar}
  \inferrule* [lab=synchronization] {} {{M} \bc \pzero \;|\; x?F \;|\; x!C }
  \and
  \inferrule* [lab=abstraction] {} {{F} \bc (x)P}
  \and
  \inferrule* [lab=concretion] {} {{C} \bc \langle Q \rangle}
  \and
  \inferrule* [lab=process] {} {{P,Q} \bc M \;| \;P|Q \;|\; @{x}}
  \and
  \inferrule* [lab=name] {} {{x} \bc \quotep{P}}
\end{mathpar} 

Note that $\vec{x}$ (resp. $\vec{P}$) denotes a vector of names
(resp. processes) of length $|\vec{x}|$ (resp. $|\vec{P}|$). We adopt
the following useful abbreviations.

\begin{mathpar}
   x?(\vec{y}).P := x.(\vec{y})P \and  x\clift{\vec{P}} := x.\clift{\vec{P}}
   \and x!(y) := \lift{x}{\dropn{y}}
   \and \Pi_{i=0}^{n-1}P_i := P_0 | \ldots | P_{n-1}
\end{mathpar}

\subsubsection{Structural congruence}

\paragraph{Free and bound names and alpha-equivalence.} At the
core of structural equivalence is alpha-equivalence which identifies
process that are the same up to a change of variable. Formally, we
recognize the distinction between free and bound names. The free names
of a process, $\freenames{P}$, may be calculated recursively as
follows:

\begin{mathpar}
\freenames{\pzero} := \emptyset
  \and \\
  \freenames{x?(y).P} := \{ x \} \cup (\freenames{P} \setminus \{ y \})
  \and 
  \freenames{x!\langle P \rangle} := \{ x \} \cup \{ P \} 
  \and \\
  \freenames{P|Q} := \freenames{P} \cup \freenames{Q}
  \and \\
  \freenames{@{x}} := \{ x \}
\end{mathpar}

$\pi$
$\quotep{\pi}$

$\freenames{-} : \pi \to \mathcal{P}(\quotep{\pi})$

\begin{eqnarray*}
  \freenames{\pzero} & := & \emptyset \\
  \freenames{x?(y).P} & := & \{ x \} \cup (\freenames{P} \setminus \{ y \}) \\
  \freenames{x!\langle P \rangle} & := & \{ x \} \cup \{ P \} \\
  \freenames{P|Q} & := & \freenames{P} \cup \freenames{Q} \\
  \freenames{\dropn{x}} & := & \{ x \}
\end{eqnarray*}

The bound names of a process, $\boundnames{P}$, are those names occurring in $P$
that are not free. For example, in $x?(y).0$, the name $x$ is free, while $y$ is bound.

\begin{mathpar}
  \inferrule* [lab=monoidal-laws] {} { P|Q \equiv Q|P \and P|0 \equiv P \and P|(Q|R) \equiv (P|Q)|R }
\end{mathpar}

\begin{mathpar}
  \inferrule* [lab=alpha-equivalence] {} { (x)P \equiv (y)P\{y/x\} \and y \not\in \freenames{P} }
\end{mathpar}

\begin{definition}
Then two processes, $P,Q$, are alpha-equivalent if $P = Q\{\vec{y}/\vec{x}\}$ for
some $\vec{x} \in \boundnames{Q},\vec{y} \in \boundnames{P}$, where $Q\{\vec{y}/\vec{x}\}$
denotes the capture-avoiding substitution of $\vec{y}$ for $\vec{x}$ in $Q$.
\end{definition}

\begin{definition}
  The {\em structural congruence} \cite{SangiorgiWalker} , $\equiv$,
  between processes is the least congruence containing
  alpha-equivalence, satisfying the abelian monoid laws
  (associativity, commutativity and $\pzero$ as identity) for parallel
  composition $|$ and for summation $+$.
\end{definition}

\subsection{Name equivalence}

We take name equivalence, written $\nameeq$, to be the smallest
equivalence relation generated by the following rules.

\begin{mathpar}
\inferrule*[lab=Quote-drop]
{ }
{ \quotep{@{x}} \nameeq x }

\inferrule*[lab=Struct-equiv]
{ P \scong Q }
{ \quotep{P} \nameeq \quotep{Q} }
\end{mathpar}

The astute reader will have noticed that the mutual recursion of names
and processes imposes a mutual recursion on alpha-equivalence and
structural equivalence via name-equivalence. Fortunately, all of this
works out pleasantly and we may calculate in the natural way, free of
concern. The reader interested in the details is referred to the
appendix \ref{appendix:rho_details}.

\subsection{Substitution}

We use $\Proc$ for the set of processes, $\QProc$ for the set of
names, and $\id{\{}\vec{y} / \vec{x} \id{\}}$ to denote partial maps,
$s : \QProc \rightarrow \QProc$. A map, $s$ lifts, uniquely, to a map
on process terms, $\widehat{s} : \Proc \rightarrow \Proc$ by the
following equations.

\begin{mathpar}
  (0) \psubstp{Q}{P} := 0 \\
  (R \juxtap S) \psubstp{Q}{P}
  :=    
  (R)\psubstp{Q}{P} \juxtap (S) \psubstp{Q}{P} \\
  (x?(y).R) \psubstp{Q}{P}    
  :=    
  (x)\substp{Q}{P} (z)\concat( (R \psubstn{z}{y}) \psubstp{Q}{P} ) \\
  (\lift{x}{R}) \psubstp{Q}{P}  
  :=
  \lift{(x)\substp{Q}{P}}{ R \psubstp{Q}{P} } \\
%   (\dropn{x})  \psubstp{Q}{P}       
%   := 
%   \left\{ 
%     \begin{array}{ccc} 
%       \dropn{\quotep{Q}} & & x \nameeq \quotep{P} \\
%       \dropn{x} & & otherwise \\
%     \end{array}
%   \right. 
  (\dropn{x})  \psubstp{Q}{P}       
  := 
  \left\{ 
    \begin{array}{ccc} 
      Q & & x \nameeq \quotep{P} \\
      \dropn{x} & & otherwise \\
    \end{array}
  \right.
\end{mathpar}
 

where

\begin{eqnarray}
  (x)\id{\{} \lpquote Q \rpquote / \lpquote P \rpquote \id{\}}            = 
  \left\{ 
    \begin{array}{ccc}
      \lpquote Q \rpquote & & x \nameeq \lpquote P \rpquote \\
      x & & otherwise \\
    \end{array}
  \right. \nonumber
\end{eqnarray}

and $z$ is chosen distinct from $\quotep{P}$, $\quotep{Q}$, the free
names in $Q$, and all the names in $R$. Our $\alpha$-equivalence will
be built in the standard way from this substitution.

\begin{remark}\label{rem:no_self_referential_names}
  One consequence of these definitions is that $\forall P. \quotep{P}
  \not\in \freenames{P}$.
\end{remark}

\subsection{ Dynamic quote: an example }

Anticipating something of what's to come, consider applying the
substitution, $\widehat{\id{\{}u / z \id{\}}}$, to the following pair
of processes, $\lift{w}{y!(z)}$ and $w[ \lpquote y!(z) \rpquote ]$.

\begin{eqnarray}
	\lift{w}{y!(z)}\widehat{\id{\{}u / z \id{\}}}
		& = &
		\lift{w}{y!(u)} \nonumber\\
	w[ \lpquote y!(z) \rpquote ] \widehat{ \id{\{}u / z \id{\}} }
		& = &
		w[ \lpquote y!(z) \rpquote ] \nonumber
\end{eqnarray}

Because the body of the process between quotes is impervious to
substitution, we get radically different answers. In fact, by
examining the first process in an input context,
e.g. $x?(z).\lift{w}{y!(z)}$, we see that the process under the lift
operator may be shaped by prefixed inputs binding a name inside it. In
this sense, the lift operator will be seen as a way to dynamically
construct processes before reifying them as names.

Finally equipped with these standard features we can present the
dynamics of the calculus.

\subsubsection{Operational semantics} 

Finally, we introduce the computational dynamics. What marks these
algebras as distinct from other more traditionally studied algebraic
structures, e.g. vector spaces or polynomial rings, is the manner in
which dynamics is captured. In traditional structures, dynamics is typically
expressed through morphisms between such structures, as in linear maps
between vector spaces or morphisms between rings. In algebras
associated with the semantics of computation, the dynamics is
expressed as part of the algebraic structure itself, through a
reduction reduction relation typically denoted by $\red$. Below, we
give a recursive presentation of this relation for the calculus used
in the encoding.

$\red \subseteq \pi \times \pi$
$\red : \pi \to \mathcal{P}(\pi)$

\begin{mathpar}
  \inferrule* [lab=Comm] { \textsf{match}( x_{src}, x_{trgt} ) } { x_{trgt}?(y)P \; | \; x_{src}!\langle {Q} \rangle \red P\{\quotep{Q}/y}\} }
  \and \\
  \inferrule* [lab=Par] {{P} \red {P}'} {{{P} | {Q}} \red {{P}' | {Q}}}
  \and
  \inferrule* [lab=Equiv]{{{P} \scong {P}'} \andalso {{P}' \red {Q}'} \andalso {{Q}' \scong {Q}}}{{P} \red {Q}}
\end{mathpar}

\begin{eqnarray*}
  match_{\equiv} (\quotep{P},\quotep{Q}) & := & P \equiv Q \\
  match_{\dagger}(\quotep{P},\quotep{Q}) & := & \forall R. P|Q \red^{*} R => R \red^{*} 0 \\
  match_{K}(\quotep{P},\quotep{Q}) & := & K \mbox{ for some context } K
\end{eqnarray*}

$u?(x)P | u!\langle Q \rangle \red P\{\quotep{Q}/x\}$

%We write $\wred$ for $\red^*$, and $P\red$ if $\exists Q $ such that $ P \red Q$.
We write $P\red$ if $\exists Q $ such that $ P \red Q$ and $P\not\red$, otherwise.

\section{Replication}

As mentioned before, it is known that replication (and hence
recursion) can be implemented in a higher-order process algebra
\cite{SangiorgiWalker}. As our first example of calculation with the
machinery thus far presented we give the construction explicitly in
the {\rhoc}.

\begin{eqnarray}
	D_{x} & := & \prefix{x}{y}{(\binpar{\outputp{x}{y}}{@{y}})} \nonumber\\
	\bangp_{x}{P} & := & \binpar{{x}!\langle{\binpar{D_{x}}{P}}\rangle}{D_{x}} \nonumber
\end{eqnarray}

\begin{eqnarray}
	\bangp_{x}{P} & & \nonumber\\
	=
	& {x}!\langle{(\prefix{x}{y}{(\outputp{x}{y} | @{y})) | P}}\rangle 
	      | \prefix{x}{y}{(\outputp{x}{y} | @{y})} & \nonumber\\
	\red
	& (\outputp{x}{y} | @{y})\substn{\quotep{(\prefix{x}{y}{(@{y} | \outputp{x}{y})) | P}}}{y} & \nonumber\\
	=
	& \outputp{x}{\quotep{(\prefix{x}{y}{(\outputp{x}{y} | @{y})) | P}}}
	  | {(\prefix{x}{y}{(\outputp{x}{y} | @{y})) | P}} & \nonumber\\
	\red
	& \ldots & \nonumber\\
	\red^*
	& P | P | \ldots & \nonumber
\end{eqnarray}

Of course, this encoding, as an implementation, runs away, unfolding
$\bangp{P}$ eagerly. A lazier and more implementable replication
operator, restricted to input-guarded processes, may be obtained as follows.

\begin{eqnarray}
\bangp{\prefix{u}{v}{P}} 
	:= 
	\binpar{\lift{x}{\prefix{u}{v}{(\binpar{D(x)}{P})}}}{D(x)} \nonumber
\end{eqnarray}

\begin{remark}
  Note that the lazier definition still does not deal with summation
  or mixed summation (i.e. sums over input and output). The reader is
  invited to construct definitions of replication that deal with these
  features. 

  Further, the definitions are parameterized in a name, $x$. Can you,
  gentle reader, make a definition that eliminates this parameter and
  guarantees no accidental interaction between the replication
  machinery and the process being replicated -- i.e. no accidental
  sharing of names used by the process to get its work done and the
  name(s) used by the replication to effect copying. This latter
  revision of the definition of replication is crucial to obtaining
  the expected identity $!!P \sim !P$.
\end{remark}

\begin{remark}\label{rem:paradoxical_combinator}
  The reader familiar with the lambda calculus will have noticed the
  similarity between $D$ and the paradoxical combinator.

  [Ed. note: the existence of this seems to suggest we have to be more
  restrictive on the set of processes and names we admit if we are to
  support no-cloning.]
\end{remark}

\subsubsection{Bisimulation}

The computational dynamics gives rise to another kind of equivalence,
the equivalence of computational behavior. As previously mentioned
this is typically captured \emph{via} some form of bisimulation.

% The notion we use in this paper is weak barbed bisimulation
% \cite{milner91polyadicpi}.

The notion we use in this paper is derived from weak barbed
bisimulation \cite{milner91polyadicpi}. 

\begin{definition}
An \emph{observation relation}, $\downarrow_{\mathcal N}$, over a set
of names, $\mathcal N$, is the smallest relation satisfying the rules
below.

\infrule[Out-barb]{y \in {\mathcal N}, \; x \nameeq y}
		  {\outputp{x}{v} \downarrow_{\mathcal N} x}
\infrule[Par-barb]{\mbox{$P\downarrow_{\mathcal N} x$ or $Q\downarrow_{\mathcal N} x$}}
		  {\binpar{P}{Q} \downarrow_{\mathcal N} x}

We write $P \Downarrow_{\mathcal N} x$ if there is $Q$ such that 
$P \wred Q$ and $Q \downarrow_{\mathcal N} x$.
\end{definition}

\begin{definition}
%\label{def.bbisim}
An  ${\mathcal N}$-\emph{barbed bisimulation} over a set of names, ${\mathcal N}$, is a symmetric binary relation 
${\mathcal S}_{\mathcal N}$ between agents such that $P\rel{S}_{\mathcal N}Q$ implies:
\begin{enumerate}
\item If $P \red P'$ then $Q \wred Q'$ and $P'\rel{S}_{\mathcal N} Q'$.
\item If $P\downarrow_{\mathcal N} x$, then $Q\Downarrow_{\mathcal N} x$.
\end{enumerate}
$P$ is ${\mathcal N}$-barbed bisimilar to $Q$, written
$P \wbbisim_{\mathcal N} Q$, if $P \rel{S}_{\mathcal N} Q$ for some ${\mathcal N}$-barbed bisimulation ${\mathcal S}_{\mathcal N}$.
\end{definition}

$\mathcal{R} \subseteq \pi \times \pi$

$P \mathcal{R} Q => \forall P'. P \red P' \Rightarrow \exists Q'. Q \red Q', P' \mathcal{R} Q'$

$P \vdash x \Rightarrow Q \vdash x$

\begin{mathpar}
  \inferrule*[lab=Out-barb]{x \nameeq y}{{y}!\langle{Q}\rangle \vdash x}
  \and
  \inferrule*[lab=Par-barb]{\mbox{$P\vdash x$ or $Q\vdash x$}}{\binpar{P}{Q} \vdash x}
\end{mathpar}

\subsubsection{Contexts}

One of the principle advantages of computational calculi like the
$\pi$-calculus is a well-defined notion of context,
contextual-equivalence and a correlation between
contextual-equivalence and notions of bisimulation. The notion of
context allows the decomposition of a process into (sub-)process and
its syntactic environment, its context. Thus, a context may be
thought of as a process with a ``hole'' (written $\Box$) in it. The
application of a context $M$ to a process $P$, written $M[P]$, is
tantamount to filling the hole in $M$ with $P$. In this paper we do
not need the full weight of this theory, but do make use of the notion
of context in the proof the main theorem. 

\begin{mathpar}
  \inferrule* [lab=summation] {} {{M_{M},M_{N}} \bc \Box \;|\; x.M_{A} \;|\; M_{M}+M_{N}}
  \and
  \inferrule* [lab=agent] {} {{M_{A}} \bc (\vec{x})M_{P} \;| \; \clift{P_0,\ldots,M_{P},\ldots,P_N}}
  \and \\
  \inferrule* [lab=process] {} {{M_{P}} \bc M_{N} \;| \;P|M_{P} }
\end{mathpar} 

\begin{mathpar}
  \inferrule* [lab=sychronization] {} {M_{N} \bc \Box \;|\; x?M_{F} \;|\; x!M_{C}}
  \and
  \inferrule* [lab=abstraction] {} {{M_{F}} \bc (x)M_{P} }
  \and
  \inferrule* [lab=concretion] {} {{M_{C}} \bc \langle M_{P} \rangle }
  \and \\
  \inferrule* [lab=process] {} {{M_{P}} \bc M_{N} \;| \;P|M_{P} }
\end{mathpar}

\begin{definition}[contextual application] Given a context $M$, and
  process $P$, we define the \emph{contextual application}, $M[P] :=
  M\{P/\Box\}$. That is, the contextual application of M to P is the
  substitution of $P$ for $\Box$ in $M$.
\end{definition}

$\meaningof{-} : L \to \mathcal{P}(\pi)$

\begin{mathpar}
  \inferrule* [lab=collection] {} {\meaningof{true} = \pi, \and \meaningof{~E} = \pi \setminus \meaningof{E}, \and \meaningof{E_{1} \& E_{2}} = \meaningof{E_{1}} \cap \meaningof{E_{2}}}
\end{mathpar}

\begin{mathpar}
  \inferrule* [lab=structure] {} {\meaningof{0} = \{ P \in \pi | P \equiv 0 \}, \and \\ \meaningof{E_1 | E_2} = \{ P \in \pi | P \equiv P_{1} | P_{2}, P_{1} \in \meaningof{E_{1}}, P_{2} \in \meaningof{E_2}\} }
\end{mathpar}

\begin{mathpar}
 \inferrule* [lab=behavior] {} {\meaningof{\langle a?b \rangle E} = \{ P \in \pi | P \equiv Q | u?(y)P', \\ \and \\\\ \and \\ \;\;\; u \in \meaningof{a}, \forall z.P'\{z/y\} \in \meaningof{E\{z/b\}}\}, \and \\ \meaningof{a!E} = \{ P \in \pi | P \equiv Q | x!\langle P' \rangle, x \in \meaningof{a} P' \in \meaningof{E}\} }
\end{mathpar}

\begin{mathpar}
 \inferrule* [lab=nominal] {} {\meaningof{\quotep{E}} = \{ \quotep{P} \in \quotep{\pi} | P \in \meaningof{E} \}, \and \meaningof{\quotep{P}} = \{ \quotep{Q} \in \quotep{\pi} | P \equiv Q \} \and \\ \meaningof{@\quotep{E}} = \{ P \in \pi | P \equiv @x, x \in \meaningof{E} \}}
\end{mathpar}

\begin{eqnarray*}
  \\
  \meaningof{-} : TS \to ST
\end{eqnarray*}

\begin{eqnarray*}
  \\
  L : TS \to ST
\end{eqnarray*}

\begin{eqnarray*}
  \\
  P \models E \iff P \in \meaningof{E}
\end{eqnarray*}

\begin{eqnarray*}
  P \approx_{L} Q \iff \forall E \in L. P \models E \iff Q \models E
\end{eqnarray*}

\begin{eqnarray*}
  P \approx_{K} Q
\end{eqnarray*}

\begin{eqnarray*}
  P \approx Q
\end{eqnarray*}

$\approx_{K} = \approx = \approx_{L}$

\subsubsection{Contextual duality}

Note that contexts extend the quotation operation to a family of
operations from processes to names. Given a context, $M$, we can
define a \emph{nominal context}, $\quotep{M}$ by $\quotep{M}[P] :=
\quotep{M[P]}$. To foreshadow what is to come we observe that these
operations enjoy a duality with processes very much like the duality
between vectors and maps from vectors to scalars.

Further, because the calculus is essentially higher-order, we have a
correspondence between contexts and processes. More specifically,
given a name $x$ and a context $M$ we can construct $M^{*}_{x}$ such
that 

\begin{mathpar}
  M^{*}_{x} | \lift{x}{P} \red M[P]
\end{mathpar}

namely,

\begin{mathpar}
  M^{*}_{x} := x?(u).M[\dropn{u}]
\end{mathpar}

The dependence of $M^{*}_{x}$ on a name makes it an abstraction, 

\begin{mathpar}
  M^{*} := (x)x?(u).M[\dropn{u}]
\end{mathpar}

\subsection{Additional notation}

It will sometimes be convenient to denote the process a name
quotes. We already have the notation $x = \quotep{P}$, but it will be
convenient to introduce an alternate notation, $\procn{x}$, when we
want to emphasize the connection to the use of the name. Note that, by
virtue of name equivalence, $\quotep{\procn{x}} \nameeq x$; so, the
notation is consistent with previous definitions.

Further, because names have structure it is possible to effect
substitutions on the basis of that structure. This means we need to
upgrade our notation for substitutions, which we accomplish by
adapting comprehension notation. Thus,

\begin{mathpar}
  P\{ y / x : x \in S \}
\end{mathpar}

is interpreted to mean the process derived from P by replacing (in a
capture-avoiding manner) each occurrence of $x$ in $S$ by $y$. For example,

\begin{mathpar}
  P\{ \quotep{\procn{x}|\procn{x}} / x : x \in \freenames{P} \}
\end{mathpar}

will replace each (occurrence) of a free name $x$ in $P$ by
$\quotep{\procn{x}|\procn{x}}$.

Also, we will avail ourselves of the notation $x^{L}$ and $x^{R}$ to
denote injections of a name into disjoint copies of the name
space. There are numerous ways to accomplish this. One example can be
found in \cite{MeredithR05}. This notation overloads to vectors of
names: $\vec{x}^{\pi} := (x_{i}^{\pi} \; : \; 0 \leq i < |\vec{x}| )$ where $\pi \in \{L,R\}$.

We also use $P^{\Box} := P|\Box$.

In \cite{MeredithR05} an interpretation of the new operator is
given. It turns out that there are several possible interpretations
all enjoying the requisite algebraic properties of the operator (see
\cite{milner91polyadicpi}). We will therefore make liberal use of
$(\nu\; \vec{x})P$.

% subsection the_syntax_and_semantics_of_the_notation_system (end)   

\input{qm2pi.qmops} 

\input{qm2pi.sterngerlach} 

\input{qm2pi.metric} 

% section concurrent_process_calculi (end)

%\input{qm2pi.proofsketch}

% section proof sketch (end)

%\input{qm2pi.slviaknots} 

% section spatial logic via knots (end)

\input{qm2pi.conclusion}

% section conclusion (end)

%\input{qm2pi.dtcodes} 

% section wiring algorithm (end)

\input{qm2pi.ack} 

% section acknowledgments (end)

\newpage


\bibliographystyle{plain}   
\bibliography{../../biblios/main.bib}

\input{qm2pi.rhodetails}

\end{document}

 

%\documentclass[12pt]{llncs}
%\documentclass{jktr}

\usepackage[pdftex]{hyperref}                   
\usepackage {listings}
\usepackage {mathpartir}
\usepackage{bcprules}
%\usepackage{listings}
                       
\usepackage{graphicx} 
%\usepackage[margins=2.5cm,nohead,nofoot]{geometry}
%\usepackage{geometry}
\usepackage{amsfonts}
\usepackage{amstext}
\usepackage{latexsym}
\usepackage{amssymb}
\usepackage{color}


%\include{myPreamble}
\include{qm2pi.local} 

%\ifpdf
%\usepackage[pdftex]{graphicx}
%\else
%\usepackage{graphicx}
%\fi

 % \ifpdf
%  \usepackage{pdfsync}
%  \if


%\title{Brief Article}
%\author{David F. Snyder}
%\author{L.G. Meredith}

%\address{Dept. of Math., Texas State University--San Marcos, San Marcos, TX 78666}
       
\pagestyle{empty}


\begin{document}

\lstset{language=[Objective]Caml,frame=shadowbox}

\input{qm2pi.front}

% section front matter (end)

\input{qm2pi.intro} 
 
% section introduction (end)

% \input{qm2pi.knotations} 

% section notation (end)

\input{qm2pi.process.calculi} 

% section concurrent_process_calculi_and_spatial_logics_ (end)
    
%\input{qm2pi.knots2pi} 

%\input{qm2pi.trefoil} 

%\input{qm2pi.mainthm} 

% subsection basic_interpretation (end)

%\input{qm2pi.rho.presentation} 
\subsection{The syntax and semantics of the notation system}\label{sub:the_syntax_and_semantics_of_the_notation_system} % (fold)

We now summarize a technical presentation of the calculus that
embodies our theory of dynamics. The typical presentation of such a
calculus follows the style of giving generators and relations on
them. The grammar, below, describing term constructors, freely
generates the set of processes, $\Proc$. This set is then quotiented
by a relation known as structural congruence and it is over this set
that the notion of dynamics is expressed. This presentation is
essentially that of \cite{MeredithR05} with the addition of
polyadicity and summation. For readability we have relegated some of
the technical subtleties to an appendix.

\subsubsection{Process grammar}\label{subsub:process_grammar}

\begin{mathpar}
  \inferrule* [lab=synchronization] {} {{M} \bc \pzero \;|\; x?F \;|\; x!C }
  \and
  \inferrule* [lab=abstraction] {} {{F} \bc (x)P}
  \and
  \inferrule* [lab=concretion] {} {{C} \bc \langle Q \rangle}
  \and
  \inferrule* [lab=process] {} {{P,Q} \bc M \;| \;P|Q \;|\; @{x}}
  \and
  \inferrule* [lab=name] {} {{x} \bc \quotep{P}}
\end{mathpar} 

Note that $\vec{x}$ (resp. $\vec{P}$) denotes a vector of names
(resp. processes) of length $|\vec{x}|$ (resp. $|\vec{P}|$). We adopt
the following useful abbreviations.

\begin{mathpar}
   x?(\vec{y}).P := x.(\vec{y})P \and  x\clift{\vec{P}} := x.\clift{\vec{P}}
   \and x!(y) := \lift{x}{\dropn{y}}
   \and \Pi_{i=0}^{n-1}P_i := P_0 | \ldots | P_{n-1}
\end{mathpar}

\subsubsection{Structural congruence}

\paragraph{Free and bound names and alpha-equivalence.} At the
core of structural equivalence is alpha-equivalence which identifies
process that are the same up to a change of variable. Formally, we
recognize the distinction between free and bound names. The free names
of a process, $\freenames{P}$, may be calculated recursively as
follows:

\begin{mathpar}
\freenames{\pzero} := \emptyset
  \and \\
  \freenames{x?(y).P} := \{ x \} \cup (\freenames{P} \setminus \{ y \})
  \and 
  \freenames{x!\langle P \rangle} := \{ x \} \cup \{ P \} 
  \and \\
  \freenames{P|Q} := \freenames{P} \cup \freenames{Q}
  \and \\
  \freenames{@{x}} := \{ x \}
\end{mathpar}

$\pi$
$\quotep{\pi}$

$\freenames{-} : \pi \to \mathcal{P}(\quotep{\pi})$

\begin{eqnarray*}
  \freenames{\pzero} & := & \emptyset \\
  \freenames{x?(y).P} & := & \{ x \} \cup (\freenames{P} \setminus \{ y \}) \\
  \freenames{x!\langle P \rangle} & := & \{ x \} \cup \{ P \} \\
  \freenames{P|Q} & := & \freenames{P} \cup \freenames{Q} \\
  \freenames{\dropn{x}} & := & \{ x \}
\end{eqnarray*}

The bound names of a process, $\boundnames{P}$, are those names occurring in $P$
that are not free. For example, in $x?(y).0$, the name $x$ is free, while $y$ is bound.

\begin{mathpar}
  \inferrule* [lab=monoidal-laws] {} { P|Q \equiv Q|P \and P|0 \equiv P \and P|(Q|R) \equiv (P|Q)|R }
\end{mathpar}

\begin{mathpar}
  \inferrule* [lab=alpha-equivalence] {} { (x)P \equiv (y)P\{y/x\} \and y \not\in \freenames{P} }
\end{mathpar}

\begin{definition}
Then two processes, $P,Q$, are alpha-equivalent if $P = Q\{\vec{y}/\vec{x}\}$ for
some $\vec{x} \in \boundnames{Q},\vec{y} \in \boundnames{P}$, where $Q\{\vec{y}/\vec{x}\}$
denotes the capture-avoiding substitution of $\vec{y}$ for $\vec{x}$ in $Q$.
\end{definition}

\begin{definition}
  The {\em structural congruence} \cite{SangiorgiWalker} , $\equiv$,
  between processes is the least congruence containing
  alpha-equivalence, satisfying the abelian monoid laws
  (associativity, commutativity and $\pzero$ as identity) for parallel
  composition $|$ and for summation $+$.
\end{definition}

\subsection{Name equivalence}

We take name equivalence, written $\nameeq$, to be the smallest
equivalence relation generated by the following rules.

\begin{mathpar}
\inferrule*[lab=Quote-drop]
{ }
{ \quotep{@{x}} \nameeq x }

\inferrule*[lab=Struct-equiv]
{ P \scong Q }
{ \quotep{P} \nameeq \quotep{Q} }
\end{mathpar}

The astute reader will have noticed that the mutual recursion of names
and processes imposes a mutual recursion on alpha-equivalence and
structural equivalence via name-equivalence. Fortunately, all of this
works out pleasantly and we may calculate in the natural way, free of
concern. The reader interested in the details is referred to the
appendix \ref{appendix:rho_details}.

\subsection{Substitution}

We use $\Proc$ for the set of processes, $\QProc$ for the set of
names, and $\id{\{}\vec{y} / \vec{x} \id{\}}$ to denote partial maps,
$s : \QProc \rightarrow \QProc$. A map, $s$ lifts, uniquely, to a map
on process terms, $\widehat{s} : \Proc \rightarrow \Proc$ by the
following equations.

\begin{mathpar}
  (0) \psubstp{Q}{P} := 0 \\
  (R \juxtap S) \psubstp{Q}{P}
  :=    
  (R)\psubstp{Q}{P} \juxtap (S) \psubstp{Q}{P} \\
  (x?(y).R) \psubstp{Q}{P}    
  :=    
  (x)\substp{Q}{P} (z)\concat( (R \psubstn{z}{y}) \psubstp{Q}{P} ) \\
  (\lift{x}{R}) \psubstp{Q}{P}  
  :=
  \lift{(x)\substp{Q}{P}}{ R \psubstp{Q}{P} } \\
%   (\dropn{x})  \psubstp{Q}{P}       
%   := 
%   \left\{ 
%     \begin{array}{ccc} 
%       \dropn{\quotep{Q}} & & x \nameeq \quotep{P} \\
%       \dropn{x} & & otherwise \\
%     \end{array}
%   \right. 
  (\dropn{x})  \psubstp{Q}{P}       
  := 
  \left\{ 
    \begin{array}{ccc} 
      Q & & x \nameeq \quotep{P} \\
      \dropn{x} & & otherwise \\
    \end{array}
  \right.
\end{mathpar}
 

where

\begin{eqnarray}
  (x)\id{\{} \lpquote Q \rpquote / \lpquote P \rpquote \id{\}}            = 
  \left\{ 
    \begin{array}{ccc}
      \lpquote Q \rpquote & & x \nameeq \lpquote P \rpquote \\
      x & & otherwise \\
    \end{array}
  \right. \nonumber
\end{eqnarray}

and $z$ is chosen distinct from $\quotep{P}$, $\quotep{Q}$, the free
names in $Q$, and all the names in $R$. Our $\alpha$-equivalence will
be built in the standard way from this substitution.

\begin{remark}\label{rem:no_self_referential_names}
  One consequence of these definitions is that $\forall P. \quotep{P}
  \not\in \freenames{P}$.
\end{remark}

\subsection{ Dynamic quote: an example }

Anticipating something of what's to come, consider applying the
substitution, $\widehat{\id{\{}u / z \id{\}}}$, to the following pair
of processes, $\lift{w}{y!(z)}$ and $w[ \lpquote y!(z) \rpquote ]$.

\begin{eqnarray}
	\lift{w}{y!(z)}\widehat{\id{\{}u / z \id{\}}}
		& = &
		\lift{w}{y!(u)} \nonumber\\
	w[ \lpquote y!(z) \rpquote ] \widehat{ \id{\{}u / z \id{\}} }
		& = &
		w[ \lpquote y!(z) \rpquote ] \nonumber
\end{eqnarray}

Because the body of the process between quotes is impervious to
substitution, we get radically different answers. In fact, by
examining the first process in an input context,
e.g. $x?(z).\lift{w}{y!(z)}$, we see that the process under the lift
operator may be shaped by prefixed inputs binding a name inside it. In
this sense, the lift operator will be seen as a way to dynamically
construct processes before reifying them as names.

Finally equipped with these standard features we can present the
dynamics of the calculus.

\subsubsection{Operational semantics} 

Finally, we introduce the computational dynamics. What marks these
algebras as distinct from other more traditionally studied algebraic
structures, e.g. vector spaces or polynomial rings, is the manner in
which dynamics is captured. In traditional structures, dynamics is typically
expressed through morphisms between such structures, as in linear maps
between vector spaces or morphisms between rings. In algebras
associated with the semantics of computation, the dynamics is
expressed as part of the algebraic structure itself, through a
reduction reduction relation typically denoted by $\red$. Below, we
give a recursive presentation of this relation for the calculus used
in the encoding.

$\red \subseteq \pi \times \pi$
$\red : \pi \to \mathcal{P}(\pi)$

\begin{mathpar}
  \inferrule* [lab=Comm] { \textsf{match}( x_{src}, x_{trgt} ) } { x_{trgt}?(y)P \; | \; x_{src}!\langle {Q} \rangle \red P\{\quotep{Q}/y}\} }
  \and \\
  \inferrule* [lab=Par] {{P} \red {P}'} {{{P} | {Q}} \red {{P}' | {Q}}}
  \and
  \inferrule* [lab=Equiv]{{{P} \scong {P}'} \andalso {{P}' \red {Q}'} \andalso {{Q}' \scong {Q}}}{{P} \red {Q}}
\end{mathpar}

\begin{eqnarray*}
  match_{\equiv} (\quotep{P},\quotep{Q}) & := & P \equiv Q \\
  match_{\dagger}(\quotep{P},\quotep{Q}) & := & \forall R. P|Q \red^{*} R => R \red^{*} 0 \\
  match_{K}(\quotep{P},\quotep{Q}) & := & K \mbox{ for some context } K
\end{eqnarray*}

$u?(x)P | u!\langle Q \rangle \red P\{\quotep{Q}/x\}$

%We write $\wred$ for $\red^*$, and $P\red$ if $\exists Q $ such that $ P \red Q$.
We write $P\red$ if $\exists Q $ such that $ P \red Q$ and $P\not\red$, otherwise.

\section{Replication}

As mentioned before, it is known that replication (and hence
recursion) can be implemented in a higher-order process algebra
\cite{SangiorgiWalker}. As our first example of calculation with the
machinery thus far presented we give the construction explicitly in
the {\rhoc}.

\begin{eqnarray}
	D_{x} & := & \prefix{x}{y}{(\binpar{\outputp{x}{y}}{@{y}})} \nonumber\\
	\bangp_{x}{P} & := & \binpar{{x}!\langle{\binpar{D_{x}}{P}}\rangle}{D_{x}} \nonumber
\end{eqnarray}

\begin{eqnarray}
	\bangp_{x}{P} & & \nonumber\\
	=
	& {x}!\langle{(\prefix{x}{y}{(\outputp{x}{y} | @{y})) | P}}\rangle 
	      | \prefix{x}{y}{(\outputp{x}{y} | @{y})} & \nonumber\\
	\red
	& (\outputp{x}{y} | @{y})\substn{\quotep{(\prefix{x}{y}{(@{y} | \outputp{x}{y})) | P}}}{y} & \nonumber\\
	=
	& \outputp{x}{\quotep{(\prefix{x}{y}{(\outputp{x}{y} | @{y})) | P}}}
	  | {(\prefix{x}{y}{(\outputp{x}{y} | @{y})) | P}} & \nonumber\\
	\red
	& \ldots & \nonumber\\
	\red^*
	& P | P | \ldots & \nonumber
\end{eqnarray}

Of course, this encoding, as an implementation, runs away, unfolding
$\bangp{P}$ eagerly. A lazier and more implementable replication
operator, restricted to input-guarded processes, may be obtained as follows.

\begin{eqnarray}
\bangp{\prefix{u}{v}{P}} 
	:= 
	\binpar{\lift{x}{\prefix{u}{v}{(\binpar{D(x)}{P})}}}{D(x)} \nonumber
\end{eqnarray}

\begin{remark}
  Note that the lazier definition still does not deal with summation
  or mixed summation (i.e. sums over input and output). The reader is
  invited to construct definitions of replication that deal with these
  features. 

  Further, the definitions are parameterized in a name, $x$. Can you,
  gentle reader, make a definition that eliminates this parameter and
  guarantees no accidental interaction between the replication
  machinery and the process being replicated -- i.e. no accidental
  sharing of names used by the process to get its work done and the
  name(s) used by the replication to effect copying. This latter
  revision of the definition of replication is crucial to obtaining
  the expected identity $!!P \sim !P$.
\end{remark}

\begin{remark}\label{rem:paradoxical_combinator}
  The reader familiar with the lambda calculus will have noticed the
  similarity between $D$ and the paradoxical combinator.

  [Ed. note: the existence of this seems to suggest we have to be more
  restrictive on the set of processes and names we admit if we are to
  support no-cloning.]
\end{remark}

\subsubsection{Bisimulation}

The computational dynamics gives rise to another kind of equivalence,
the equivalence of computational behavior. As previously mentioned
this is typically captured \emph{via} some form of bisimulation.

% The notion we use in this paper is weak barbed bisimulation
% \cite{milner91polyadicpi}.

The notion we use in this paper is derived from weak barbed
bisimulation \cite{milner91polyadicpi}. 

\begin{definition}
An \emph{observation relation}, $\downarrow_{\mathcal N}$, over a set
of names, $\mathcal N$, is the smallest relation satisfying the rules
below.

\infrule[Out-barb]{y \in {\mathcal N}, \; x \nameeq y}
		  {\outputp{x}{v} \downarrow_{\mathcal N} x}
\infrule[Par-barb]{\mbox{$P\downarrow_{\mathcal N} x$ or $Q\downarrow_{\mathcal N} x$}}
		  {\binpar{P}{Q} \downarrow_{\mathcal N} x}

We write $P \Downarrow_{\mathcal N} x$ if there is $Q$ such that 
$P \wred Q$ and $Q \downarrow_{\mathcal N} x$.
\end{definition}

\begin{definition}
%\label{def.bbisim}
An  ${\mathcal N}$-\emph{barbed bisimulation} over a set of names, ${\mathcal N}$, is a symmetric binary relation 
${\mathcal S}_{\mathcal N}$ between agents such that $P\rel{S}_{\mathcal N}Q$ implies:
\begin{enumerate}
\item If $P \red P'$ then $Q \wred Q'$ and $P'\rel{S}_{\mathcal N} Q'$.
\item If $P\downarrow_{\mathcal N} x$, then $Q\Downarrow_{\mathcal N} x$.
\end{enumerate}
$P$ is ${\mathcal N}$-barbed bisimilar to $Q$, written
$P \wbbisim_{\mathcal N} Q$, if $P \rel{S}_{\mathcal N} Q$ for some ${\mathcal N}$-barbed bisimulation ${\mathcal S}_{\mathcal N}$.
\end{definition}

$\mathcal{R} \subseteq \pi \times \pi$

$P \mathcal{R} Q => \forall P'. P \red P' \Rightarrow \exists Q'. Q \red Q', P' \mathcal{R} Q'$

$P \vdash x \Rightarrow Q \vdash x$

\begin{mathpar}
  \inferrule*[lab=Out-barb]{x \nameeq y}{{y}!\langle{Q}\rangle \vdash x}
  \and
  \inferrule*[lab=Par-barb]{\mbox{$P\vdash x$ or $Q\vdash x$}}{\binpar{P}{Q} \vdash x}
\end{mathpar}

\subsubsection{Contexts}

One of the principle advantages of computational calculi like the
$\pi$-calculus is a well-defined notion of context,
contextual-equivalence and a correlation between
contextual-equivalence and notions of bisimulation. The notion of
context allows the decomposition of a process into (sub-)process and
its syntactic environment, its context. Thus, a context may be
thought of as a process with a ``hole'' (written $\Box$) in it. The
application of a context $M$ to a process $P$, written $M[P]$, is
tantamount to filling the hole in $M$ with $P$. In this paper we do
not need the full weight of this theory, but do make use of the notion
of context in the proof the main theorem. 

\begin{mathpar}
  \inferrule* [lab=summation] {} {{M_{M},M_{N}} \bc \Box \;|\; x.M_{A} \;|\; M_{M}+M_{N}}
  \and
  \inferrule* [lab=agent] {} {{M_{A}} \bc (\vec{x})M_{P} \;| \; \clift{P_0,\ldots,M_{P},\ldots,P_N}}
  \and \\
  \inferrule* [lab=process] {} {{M_{P}} \bc M_{N} \;| \;P|M_{P} }
\end{mathpar} 

\begin{mathpar}
  \inferrule* [lab=sychronization] {} {M_{N} \bc \Box \;|\; x?M_{F} \;|\; x!M_{C}}
  \and
  \inferrule* [lab=abstraction] {} {{M_{F}} \bc (x)M_{P} }
  \and
  \inferrule* [lab=concretion] {} {{M_{C}} \bc \langle M_{P} \rangle }
  \and \\
  \inferrule* [lab=process] {} {{M_{P}} \bc M_{N} \;| \;P|M_{P} }
\end{mathpar}

\begin{definition}[contextual application] Given a context $M$, and
  process $P$, we define the \emph{contextual application}, $M[P] :=
  M\{P/\Box\}$. That is, the contextual application of M to P is the
  substitution of $P$ for $\Box$ in $M$.
\end{definition}

$\meaningof{-} : L \to \mathcal{P}(\pi)$

\begin{mathpar}
  \inferrule* [lab=collection] {} {\meaningof{true} = \pi, \and \meaningof{~E} = \pi \setminus \meaningof{E}, \and \meaningof{E_{1} \& E_{2}} = \meaningof{E_{1}} \cap \meaningof{E_{2}}}
\end{mathpar}

\begin{mathpar}
  \inferrule* [lab=structure] {} {\meaningof{0} = \{ P \in \pi | P \equiv 0 \}, \and \\ \meaningof{E_1 | E_2} = \{ P \in \pi | P \equiv P_{1} | P_{2}, P_{1} \in \meaningof{E_{1}}, P_{2} \in \meaningof{E_2}\} }
\end{mathpar}

\begin{mathpar}
 \inferrule* [lab=behavior] {} {\meaningof{\langle a?b \rangle E} = \{ P \in \pi | P \equiv Q | u?(y)P', \\ \and \\\\ \and \\ \;\;\; u \in \meaningof{a}, \forall z.P'\{z/y\} \in \meaningof{E\{z/b\}}\}, \and \\ \meaningof{a!E} = \{ P \in \pi | P \equiv Q | x!\langle P' \rangle, x \in \meaningof{a} P' \in \meaningof{E}\} }
\end{mathpar}

\begin{mathpar}
 \inferrule* [lab=nominal] {} {\meaningof{\quotep{E}} = \{ \quotep{P} \in \quotep{\pi} | P \in \meaningof{E} \}, \and \meaningof{\quotep{P}} = \{ \quotep{Q} \in \quotep{\pi} | P \equiv Q \} \and \\ \meaningof{@\quotep{E}} = \{ P \in \pi | P \equiv @x, x \in \meaningof{E} \}}
\end{mathpar}

\begin{eqnarray*}
  \\
  \meaningof{-} : TS \to ST
\end{eqnarray*}

\begin{eqnarray*}
  \\
  L : TS \to ST
\end{eqnarray*}

\begin{eqnarray*}
  \\
  P \models E \iff P \in \meaningof{E}
\end{eqnarray*}

\begin{eqnarray*}
  P \approx_{L} Q \iff \forall E \in L. P \models E \iff Q \models E
\end{eqnarray*}

\begin{eqnarray*}
  P \approx_{K} Q
\end{eqnarray*}

\begin{eqnarray*}
  P \approx Q
\end{eqnarray*}

$\approx_{K} = \approx = \approx_{L}$

\subsubsection{Contextual duality}

Note that contexts extend the quotation operation to a family of
operations from processes to names. Given a context, $M$, we can
define a \emph{nominal context}, $\quotep{M}$ by $\quotep{M}[P] :=
\quotep{M[P]}$. To foreshadow what is to come we observe that these
operations enjoy a duality with processes very much like the duality
between vectors and maps from vectors to scalars.

Further, because the calculus is essentially higher-order, we have a
correspondence between contexts and processes. More specifically,
given a name $x$ and a context $M$ we can construct $M^{*}_{x}$ such
that 

\begin{mathpar}
  M^{*}_{x} | \lift{x}{P} \red M[P]
\end{mathpar}

namely,

\begin{mathpar}
  M^{*}_{x} := x?(u).M[\dropn{u}]
\end{mathpar}

The dependence of $M^{*}_{x}$ on a name makes it an abstraction, 

\begin{mathpar}
  M^{*} := (x)x?(u).M[\dropn{u}]
\end{mathpar}

\subsection{Additional notation}

It will sometimes be convenient to denote the process a name
quotes. We already have the notation $x = \quotep{P}$, but it will be
convenient to introduce an alternate notation, $\procn{x}$, when we
want to emphasize the connection to the use of the name. Note that, by
virtue of name equivalence, $\quotep{\procn{x}} \nameeq x$; so, the
notation is consistent with previous definitions.

Further, because names have structure it is possible to effect
substitutions on the basis of that structure. This means we need to
upgrade our notation for substitutions, which we accomplish by
adapting comprehension notation. Thus,

\begin{mathpar}
  P\{ y / x : x \in S \}
\end{mathpar}

is interpreted to mean the process derived from P by replacing (in a
capture-avoiding manner) each occurrence of $x$ in $S$ by $y$. For example,

\begin{mathpar}
  P\{ \quotep{\procn{x}|\procn{x}} / x : x \in \freenames{P} \}
\end{mathpar}

will replace each (occurrence) of a free name $x$ in $P$ by
$\quotep{\procn{x}|\procn{x}}$.

Also, we will avail ourselves of the notation $x^{L}$ and $x^{R}$ to
denote injections of a name into disjoint copies of the name
space. There are numerous ways to accomplish this. One example can be
found in \cite{MeredithR05}. This notation overloads to vectors of
names: $\vec{x}^{\pi} := (x_{i}^{\pi} \; : \; 0 \leq i < |\vec{x}| )$ where $\pi \in \{L,R\}$.

We also use $P^{\Box} := P|\Box$.

In \cite{MeredithR05} an interpretation of the new operator is
given. It turns out that there are several possible interpretations
all enjoying the requisite algebraic properties of the operator (see
\cite{milner91polyadicpi}). We will therefore make liberal use of
$(\nu\; \vec{x})P$.

% subsection the_syntax_and_semantics_of_the_notation_system (end)   

\input{qm2pi.qmops} 

\input{qm2pi.sterngerlach} 

\input{qm2pi.metric} 

% section concurrent_process_calculi (end)

%\input{qm2pi.proofsketch}

% section proof sketch (end)

%\input{qm2pi.slviaknots} 

% section spatial logic via knots (end)

\input{qm2pi.conclusion}

% section conclusion (end)

%\input{qm2pi.dtcodes} 

% section wiring algorithm (end)

\input{qm2pi.ack} 

% section acknowledgments (end)

\newpage


\bibliographystyle{plain}   
\bibliography{../../biblios/main.bib}

\input{qm2pi.rhodetails}

\end{document}

 

% subsection basic_interpretation (end)

%\input{qm2pi.rho.presentation} 
\subsection{The syntax and semantics of the notation system}\label{sub:the_syntax_and_semantics_of_the_notation_system} % (fold)

We now summarize a technical presentation of the calculus that
embodies our theory of dynamics. The typical presentation of such a
calculus follows the style of giving generators and relations on
them. The grammar, below, describing term constructors, freely
generates the set of processes, $\Proc$. This set is then quotiented
by a relation known as structural congruence and it is over this set
that the notion of dynamics is expressed. This presentation is
essentially that of \cite{MeredithR05} with the addition of
polyadicity and summation. For readability we have relegated some of
the technical subtleties to an appendix.

\subsubsection{Process grammar}\label{subsub:process_grammar}

\begin{mathpar}
  \inferrule* [lab=synchronization] {} {{M} \bc \pzero \;|\; x?F \;|\; x!C }
  \and
  \inferrule* [lab=abstraction] {} {{F} \bc (x)P}
  \and
  \inferrule* [lab=concretion] {} {{C} \bc \langle Q \rangle}
  \and
  \inferrule* [lab=process] {} {{P,Q} \bc M \;| \;P|Q \;|\; @{x}}
  \and
  \inferrule* [lab=name] {} {{x} \bc \quotep{P}}
\end{mathpar} 

Note that $\vec{x}$ (resp. $\vec{P}$) denotes a vector of names
(resp. processes) of length $|\vec{x}|$ (resp. $|\vec{P}|$). We adopt
the following useful abbreviations.

\begin{mathpar}
   x?(\vec{y}).P := x.(\vec{y})P \and  x\clift{\vec{P}} := x.\clift{\vec{P}}
   \and x!(y) := \lift{x}{\dropn{y}}
   \and \Pi_{i=0}^{n-1}P_i := P_0 | \ldots | P_{n-1}
\end{mathpar}

\subsubsection{Structural congruence}

\paragraph{Free and bound names and alpha-equivalence.} At the
core of structural equivalence is alpha-equivalence which identifies
process that are the same up to a change of variable. Formally, we
recognize the distinction between free and bound names. The free names
of a process, $\freenames{P}$, may be calculated recursively as
follows:

\begin{mathpar}
\freenames{\pzero} := \emptyset
  \and \\
  \freenames{x?(y).P} := \{ x \} \cup (\freenames{P} \setminus \{ y \})
  \and 
  \freenames{x!\langle P \rangle} := \{ x \} \cup \{ P \} 
  \and \\
  \freenames{P|Q} := \freenames{P} \cup \freenames{Q}
  \and \\
  \freenames{@{x}} := \{ x \}
\end{mathpar}

$\pi$
$\quotep{\pi}$

$\freenames{-} : \pi \to \mathcal{P}(\quotep{\pi})$

\begin{eqnarray*}
  \freenames{\pzero} & := & \emptyset \\
  \freenames{x?(y).P} & := & \{ x \} \cup (\freenames{P} \setminus \{ y \}) \\
  \freenames{x!\langle P \rangle} & := & \{ x \} \cup \{ P \} \\
  \freenames{P|Q} & := & \freenames{P} \cup \freenames{Q} \\
  \freenames{\dropn{x}} & := & \{ x \}
\end{eqnarray*}

The bound names of a process, $\boundnames{P}$, are those names occurring in $P$
that are not free. For example, in $x?(y).0$, the name $x$ is free, while $y$ is bound.

\begin{mathpar}
  \inferrule* [lab=monoidal-laws] {} { P|Q \equiv Q|P \and P|0 \equiv P \and P|(Q|R) \equiv (P|Q)|R }
\end{mathpar}

\begin{mathpar}
  \inferrule* [lab=alpha-equivalence] {} { (x)P \equiv (y)P\{y/x\} \and y \not\in \freenames{P} }
\end{mathpar}

\begin{definition}
Then two processes, $P,Q$, are alpha-equivalent if $P = Q\{\vec{y}/\vec{x}\}$ for
some $\vec{x} \in \boundnames{Q},\vec{y} \in \boundnames{P}$, where $Q\{\vec{y}/\vec{x}\}$
denotes the capture-avoiding substitution of $\vec{y}$ for $\vec{x}$ in $Q$.
\end{definition}

\begin{definition}
  The {\em structural congruence} \cite{SangiorgiWalker} , $\equiv$,
  between processes is the least congruence containing
  alpha-equivalence, satisfying the abelian monoid laws
  (associativity, commutativity and $\pzero$ as identity) for parallel
  composition $|$ and for summation $+$.
\end{definition}

\subsection{Name equivalence}

We take name equivalence, written $\nameeq$, to be the smallest
equivalence relation generated by the following rules.

\begin{mathpar}
\inferrule*[lab=Quote-drop]
{ }
{ \quotep{@{x}} \nameeq x }

\inferrule*[lab=Struct-equiv]
{ P \scong Q }
{ \quotep{P} \nameeq \quotep{Q} }
\end{mathpar}

The astute reader will have noticed that the mutual recursion of names
and processes imposes a mutual recursion on alpha-equivalence and
structural equivalence via name-equivalence. Fortunately, all of this
works out pleasantly and we may calculate in the natural way, free of
concern. The reader interested in the details is referred to the
appendix \ref{appendix:rho_details}.

\subsection{Substitution}

We use $\Proc$ for the set of processes, $\QProc$ for the set of
names, and $\id{\{}\vec{y} / \vec{x} \id{\}}$ to denote partial maps,
$s : \QProc \rightarrow \QProc$. A map, $s$ lifts, uniquely, to a map
on process terms, $\widehat{s} : \Proc \rightarrow \Proc$ by the
following equations.

\begin{mathpar}
  (0) \psubstp{Q}{P} := 0 \\
  (R \juxtap S) \psubstp{Q}{P}
  :=    
  (R)\psubstp{Q}{P} \juxtap (S) \psubstp{Q}{P} \\
  (x?(y).R) \psubstp{Q}{P}    
  :=    
  (x)\substp{Q}{P} (z)\concat( (R \psubstn{z}{y}) \psubstp{Q}{P} ) \\
  (\lift{x}{R}) \psubstp{Q}{P}  
  :=
  \lift{(x)\substp{Q}{P}}{ R \psubstp{Q}{P} } \\
%   (\dropn{x})  \psubstp{Q}{P}       
%   := 
%   \left\{ 
%     \begin{array}{ccc} 
%       \dropn{\quotep{Q}} & & x \nameeq \quotep{P} \\
%       \dropn{x} & & otherwise \\
%     \end{array}
%   \right. 
  (\dropn{x})  \psubstp{Q}{P}       
  := 
  \left\{ 
    \begin{array}{ccc} 
      Q & & x \nameeq \quotep{P} \\
      \dropn{x} & & otherwise \\
    \end{array}
  \right.
\end{mathpar}
 

where

\begin{eqnarray}
  (x)\id{\{} \lpquote Q \rpquote / \lpquote P \rpquote \id{\}}            = 
  \left\{ 
    \begin{array}{ccc}
      \lpquote Q \rpquote & & x \nameeq \lpquote P \rpquote \\
      x & & otherwise \\
    \end{array}
  \right. \nonumber
\end{eqnarray}

and $z$ is chosen distinct from $\quotep{P}$, $\quotep{Q}$, the free
names in $Q$, and all the names in $R$. Our $\alpha$-equivalence will
be built in the standard way from this substitution.

\begin{remark}\label{rem:no_self_referential_names}
  One consequence of these definitions is that $\forall P. \quotep{P}
  \not\in \freenames{P}$.
\end{remark}

\subsection{ Dynamic quote: an example }

Anticipating something of what's to come, consider applying the
substitution, $\widehat{\id{\{}u / z \id{\}}}$, to the following pair
of processes, $\lift{w}{y!(z)}$ and $w[ \lpquote y!(z) \rpquote ]$.

\begin{eqnarray}
	\lift{w}{y!(z)}\widehat{\id{\{}u / z \id{\}}}
		& = &
		\lift{w}{y!(u)} \nonumber\\
	w[ \lpquote y!(z) \rpquote ] \widehat{ \id{\{}u / z \id{\}} }
		& = &
		w[ \lpquote y!(z) \rpquote ] \nonumber
\end{eqnarray}

Because the body of the process between quotes is impervious to
substitution, we get radically different answers. In fact, by
examining the first process in an input context,
e.g. $x?(z).\lift{w}{y!(z)}$, we see that the process under the lift
operator may be shaped by prefixed inputs binding a name inside it. In
this sense, the lift operator will be seen as a way to dynamically
construct processes before reifying them as names.

Finally equipped with these standard features we can present the
dynamics of the calculus.

\subsubsection{Operational semantics} 

Finally, we introduce the computational dynamics. What marks these
algebras as distinct from other more traditionally studied algebraic
structures, e.g. vector spaces or polynomial rings, is the manner in
which dynamics is captured. In traditional structures, dynamics is typically
expressed through morphisms between such structures, as in linear maps
between vector spaces or morphisms between rings. In algebras
associated with the semantics of computation, the dynamics is
expressed as part of the algebraic structure itself, through a
reduction reduction relation typically denoted by $\red$. Below, we
give a recursive presentation of this relation for the calculus used
in the encoding.

$\red \subseteq \pi \times \pi$
$\red : \pi \to \mathcal{P}(\pi)$

\begin{mathpar}
  \inferrule* [lab=Comm] { \textsf{match}( x_{src}, x_{trgt} ) } { x_{trgt}?(y)P \; | \; x_{src}!\langle {Q} \rangle \red P\{\quotep{Q}/y}\} }
  \and \\
  \inferrule* [lab=Par] {{P} \red {P}'} {{{P} | {Q}} \red {{P}' | {Q}}}
  \and
  \inferrule* [lab=Equiv]{{{P} \scong {P}'} \andalso {{P}' \red {Q}'} \andalso {{Q}' \scong {Q}}}{{P} \red {Q}}
\end{mathpar}

\begin{eqnarray*}
  match_{\equiv} (\quotep{P},\quotep{Q}) & := & P \equiv Q \\
  match_{\dagger}(\quotep{P},\quotep{Q}) & := & \forall R. P|Q \red^{*} R => R \red^{*} 0 \\
  match_{K}(\quotep{P},\quotep{Q}) & := & K \mbox{ for some context } K
\end{eqnarray*}

$u?(x)P | u!\langle Q \rangle \red P\{\quotep{Q}/x\}$

%We write $\wred$ for $\red^*$, and $P\red$ if $\exists Q $ such that $ P \red Q$.
We write $P\red$ if $\exists Q $ such that $ P \red Q$ and $P\not\red$, otherwise.

\section{Replication}

As mentioned before, it is known that replication (and hence
recursion) can be implemented in a higher-order process algebra
\cite{SangiorgiWalker}. As our first example of calculation with the
machinery thus far presented we give the construction explicitly in
the {\rhoc}.

\begin{eqnarray}
	D_{x} & := & \prefix{x}{y}{(\binpar{\outputp{x}{y}}{@{y}})} \nonumber\\
	\bangp_{x}{P} & := & \binpar{{x}!\langle{\binpar{D_{x}}{P}}\rangle}{D_{x}} \nonumber
\end{eqnarray}

\begin{eqnarray}
	\bangp_{x}{P} & & \nonumber\\
	=
	& {x}!\langle{(\prefix{x}{y}{(\outputp{x}{y} | @{y})) | P}}\rangle 
	      | \prefix{x}{y}{(\outputp{x}{y} | @{y})} & \nonumber\\
	\red
	& (\outputp{x}{y} | @{y})\substn{\quotep{(\prefix{x}{y}{(@{y} | \outputp{x}{y})) | P}}}{y} & \nonumber\\
	=
	& \outputp{x}{\quotep{(\prefix{x}{y}{(\outputp{x}{y} | @{y})) | P}}}
	  | {(\prefix{x}{y}{(\outputp{x}{y} | @{y})) | P}} & \nonumber\\
	\red
	& \ldots & \nonumber\\
	\red^*
	& P | P | \ldots & \nonumber
\end{eqnarray}

Of course, this encoding, as an implementation, runs away, unfolding
$\bangp{P}$ eagerly. A lazier and more implementable replication
operator, restricted to input-guarded processes, may be obtained as follows.

\begin{eqnarray}
\bangp{\prefix{u}{v}{P}} 
	:= 
	\binpar{\lift{x}{\prefix{u}{v}{(\binpar{D(x)}{P})}}}{D(x)} \nonumber
\end{eqnarray}

\begin{remark}
  Note that the lazier definition still does not deal with summation
  or mixed summation (i.e. sums over input and output). The reader is
  invited to construct definitions of replication that deal with these
  features. 

  Further, the definitions are parameterized in a name, $x$. Can you,
  gentle reader, make a definition that eliminates this parameter and
  guarantees no accidental interaction between the replication
  machinery and the process being replicated -- i.e. no accidental
  sharing of names used by the process to get its work done and the
  name(s) used by the replication to effect copying. This latter
  revision of the definition of replication is crucial to obtaining
  the expected identity $!!P \sim !P$.
\end{remark}

\begin{remark}\label{rem:paradoxical_combinator}
  The reader familiar with the lambda calculus will have noticed the
  similarity between $D$ and the paradoxical combinator.

  [Ed. note: the existence of this seems to suggest we have to be more
  restrictive on the set of processes and names we admit if we are to
  support no-cloning.]
\end{remark}

\subsubsection{Bisimulation}

The computational dynamics gives rise to another kind of equivalence,
the equivalence of computational behavior. As previously mentioned
this is typically captured \emph{via} some form of bisimulation.

% The notion we use in this paper is weak barbed bisimulation
% \cite{milner91polyadicpi}.

The notion we use in this paper is derived from weak barbed
bisimulation \cite{milner91polyadicpi}. 

\begin{definition}
An \emph{observation relation}, $\downarrow_{\mathcal N}$, over a set
of names, $\mathcal N$, is the smallest relation satisfying the rules
below.

\infrule[Out-barb]{y \in {\mathcal N}, \; x \nameeq y}
		  {\outputp{x}{v} \downarrow_{\mathcal N} x}
\infrule[Par-barb]{\mbox{$P\downarrow_{\mathcal N} x$ or $Q\downarrow_{\mathcal N} x$}}
		  {\binpar{P}{Q} \downarrow_{\mathcal N} x}

We write $P \Downarrow_{\mathcal N} x$ if there is $Q$ such that 
$P \wred Q$ and $Q \downarrow_{\mathcal N} x$.
\end{definition}

\begin{definition}
%\label{def.bbisim}
An  ${\mathcal N}$-\emph{barbed bisimulation} over a set of names, ${\mathcal N}$, is a symmetric binary relation 
${\mathcal S}_{\mathcal N}$ between agents such that $P\rel{S}_{\mathcal N}Q$ implies:
\begin{enumerate}
\item If $P \red P'$ then $Q \wred Q'$ and $P'\rel{S}_{\mathcal N} Q'$.
\item If $P\downarrow_{\mathcal N} x$, then $Q\Downarrow_{\mathcal N} x$.
\end{enumerate}
$P$ is ${\mathcal N}$-barbed bisimilar to $Q$, written
$P \wbbisim_{\mathcal N} Q$, if $P \rel{S}_{\mathcal N} Q$ for some ${\mathcal N}$-barbed bisimulation ${\mathcal S}_{\mathcal N}$.
\end{definition}

$\mathcal{R} \subseteq \pi \times \pi$

$P \mathcal{R} Q => \forall P'. P \red P' \Rightarrow \exists Q'. Q \red Q', P' \mathcal{R} Q'$

$P \vdash x \Rightarrow Q \vdash x$

\begin{mathpar}
  \inferrule*[lab=Out-barb]{x \nameeq y}{{y}!\langle{Q}\rangle \vdash x}
  \and
  \inferrule*[lab=Par-barb]{\mbox{$P\vdash x$ or $Q\vdash x$}}{\binpar{P}{Q} \vdash x}
\end{mathpar}

\subsubsection{Contexts}

One of the principle advantages of computational calculi like the
$\pi$-calculus is a well-defined notion of context,
contextual-equivalence and a correlation between
contextual-equivalence and notions of bisimulation. The notion of
context allows the decomposition of a process into (sub-)process and
its syntactic environment, its context. Thus, a context may be
thought of as a process with a ``hole'' (written $\Box$) in it. The
application of a context $M$ to a process $P$, written $M[P]$, is
tantamount to filling the hole in $M$ with $P$. In this paper we do
not need the full weight of this theory, but do make use of the notion
of context in the proof the main theorem. 

\begin{mathpar}
  \inferrule* [lab=summation] {} {{M_{M},M_{N}} \bc \Box \;|\; x.M_{A} \;|\; M_{M}+M_{N}}
  \and
  \inferrule* [lab=agent] {} {{M_{A}} \bc (\vec{x})M_{P} \;| \; \clift{P_0,\ldots,M_{P},\ldots,P_N}}
  \and \\
  \inferrule* [lab=process] {} {{M_{P}} \bc M_{N} \;| \;P|M_{P} }
\end{mathpar} 

\begin{mathpar}
  \inferrule* [lab=sychronization] {} {M_{N} \bc \Box \;|\; x?M_{F} \;|\; x!M_{C}}
  \and
  \inferrule* [lab=abstraction] {} {{M_{F}} \bc (x)M_{P} }
  \and
  \inferrule* [lab=concretion] {} {{M_{C}} \bc \langle M_{P} \rangle }
  \and \\
  \inferrule* [lab=process] {} {{M_{P}} \bc M_{N} \;| \;P|M_{P} }
\end{mathpar}

\begin{definition}[contextual application] Given a context $M$, and
  process $P$, we define the \emph{contextual application}, $M[P] :=
  M\{P/\Box\}$. That is, the contextual application of M to P is the
  substitution of $P$ for $\Box$ in $M$.
\end{definition}

$\meaningof{-} : L \to \mathcal{P}(\pi)$

\begin{mathpar}
  \inferrule* [lab=collection] {} {\meaningof{true} = \pi, \and \meaningof{~E} = \pi \setminus \meaningof{E}, \and \meaningof{E_{1} \& E_{2}} = \meaningof{E_{1}} \cap \meaningof{E_{2}}}
\end{mathpar}

\begin{mathpar}
  \inferrule* [lab=structure] {} {\meaningof{0} = \{ P \in \pi | P \equiv 0 \}, \and \\ \meaningof{E_1 | E_2} = \{ P \in \pi | P \equiv P_{1} | P_{2}, P_{1} \in \meaningof{E_{1}}, P_{2} \in \meaningof{E_2}\} }
\end{mathpar}

\begin{mathpar}
 \inferrule* [lab=behavior] {} {\meaningof{\langle a?b \rangle E} = \{ P \in \pi | P \equiv Q | u?(y)P', \\ \and \\\\ \and \\ \;\;\; u \in \meaningof{a}, \forall z.P'\{z/y\} \in \meaningof{E\{z/b\}}\}, \and \\ \meaningof{a!E} = \{ P \in \pi | P \equiv Q | x!\langle P' \rangle, x \in \meaningof{a} P' \in \meaningof{E}\} }
\end{mathpar}

\begin{mathpar}
 \inferrule* [lab=nominal] {} {\meaningof{\quotep{E}} = \{ \quotep{P} \in \quotep{\pi} | P \in \meaningof{E} \}, \and \meaningof{\quotep{P}} = \{ \quotep{Q} \in \quotep{\pi} | P \equiv Q \} \and \\ \meaningof{@\quotep{E}} = \{ P \in \pi | P \equiv @x, x \in \meaningof{E} \}}
\end{mathpar}

\begin{eqnarray*}
  \\
  \meaningof{-} : TS \to ST
\end{eqnarray*}

\begin{eqnarray*}
  \\
  L : TS \to ST
\end{eqnarray*}

\begin{eqnarray*}
  \\
  P \models E \iff P \in \meaningof{E}
\end{eqnarray*}

\begin{eqnarray*}
  P \approx_{L} Q \iff \forall E \in L. P \models E \iff Q \models E
\end{eqnarray*}

\begin{eqnarray*}
  P \approx_{K} Q
\end{eqnarray*}

\begin{eqnarray*}
  P \approx Q
\end{eqnarray*}

$\approx_{K} = \approx = \approx_{L}$

\subsubsection{Contextual duality}

Note that contexts extend the quotation operation to a family of
operations from processes to names. Given a context, $M$, we can
define a \emph{nominal context}, $\quotep{M}$ by $\quotep{M}[P] :=
\quotep{M[P]}$. To foreshadow what is to come we observe that these
operations enjoy a duality with processes very much like the duality
between vectors and maps from vectors to scalars.

Further, because the calculus is essentially higher-order, we have a
correspondence between contexts and processes. More specifically,
given a name $x$ and a context $M$ we can construct $M^{*}_{x}$ such
that 

\begin{mathpar}
  M^{*}_{x} | \lift{x}{P} \red M[P]
\end{mathpar}

namely,

\begin{mathpar}
  M^{*}_{x} := x?(u).M[\dropn{u}]
\end{mathpar}

The dependence of $M^{*}_{x}$ on a name makes it an abstraction, 

\begin{mathpar}
  M^{*} := (x)x?(u).M[\dropn{u}]
\end{mathpar}

\subsection{Additional notation}

It will sometimes be convenient to denote the process a name
quotes. We already have the notation $x = \quotep{P}$, but it will be
convenient to introduce an alternate notation, $\procn{x}$, when we
want to emphasize the connection to the use of the name. Note that, by
virtue of name equivalence, $\quotep{\procn{x}} \nameeq x$; so, the
notation is consistent with previous definitions.

Further, because names have structure it is possible to effect
substitutions on the basis of that structure. This means we need to
upgrade our notation for substitutions, which we accomplish by
adapting comprehension notation. Thus,

\begin{mathpar}
  P\{ y / x : x \in S \}
\end{mathpar}

is interpreted to mean the process derived from P by replacing (in a
capture-avoiding manner) each occurrence of $x$ in $S$ by $y$. For example,

\begin{mathpar}
  P\{ \quotep{\procn{x}|\procn{x}} / x : x \in \freenames{P} \}
\end{mathpar}

will replace each (occurrence) of a free name $x$ in $P$ by
$\quotep{\procn{x}|\procn{x}}$.

Also, we will avail ourselves of the notation $x^{L}$ and $x^{R}$ to
denote injections of a name into disjoint copies of the name
space. There are numerous ways to accomplish this. One example can be
found in \cite{MeredithR05}. This notation overloads to vectors of
names: $\vec{x}^{\pi} := (x_{i}^{\pi} \; : \; 0 \leq i < |\vec{x}| )$ where $\pi \in \{L,R\}$.

We also use $P^{\Box} := P|\Box$.

In \cite{MeredithR05} an interpretation of the new operator is
given. It turns out that there are several possible interpretations
all enjoying the requisite algebraic properties of the operator (see
\cite{milner91polyadicpi}). We will therefore make liberal use of
$(\nu\; \vec{x})P$.

% subsection the_syntax_and_semantics_of_the_notation_system (end)   

\section{Interpretation of QM}
\subsection{Supporting definitions}
\subsubsection{Multiplication}
\begin{mathpar}
  \quotep{Q} \cdot \quotep{R} := \quotep{Q|R}
  \and \\
  \quotep{Q} \cdot P := P\{ \quotep{Q|R} / \quotep{R} : \quotep{R} \in \freenames{P} \}
\end{mathpar}

\paragraph{Discussion}
The first line needs little explanation. The second line says that
each free name of the process is replaced with the multiplication of
that name by the scalar. Multiplication of a scalar (name) by a state
(process) results in a process all the names of which have been `moved
over' by parallel composition with the process the scalar
quotes. There is a subtlety that the bound names have to be
manipulated so that multiplied names aren't accidentally
captured. There are many ways to achieve this.

\begin{remark}\label{rem:multiplication_identities}
  The reader is invited to verify that for all $x,y,z \in \QProc$ and $P \in \Proc$
  \begin{mathpar}
    x \cdot \quotep{0} \equiv x 
    \and
    x \cdot y \equiv y \cdot x
    \and
    x \cdot (y \cdot z) \equiv (x \cdot y) \cdot z
    \and \\
    \quotep{0} \cdot P \equiv P
    \and \\
    x \cdot (y \cdot P) \equiv (x \cdot y) \cdot P
    \and \\
    x \cdot (P|Q) \equiv (x \cdot P) | (x \cdot Q)
    \and \\    
  \end{mathpar}
\end{remark}

\subsubsection{Tensor product}

We define a tensor product on processes by structural induction.

\paragraph{Tensor of sums} First note that all summations, including
$\pzero$ and sequence, can be written $\Sigma_{i} x_{i}.A_{i} +
\Sigma_{j} x_{j}.C_{j}$, where we have grouped input-guarded processes
together and output-guarded processes together.

Thus, we can define the tensor product of two summations, $N_{1}\otimes N_{2}$, where

\begin{mathpar}
  N_{1} := \Sigma_{i} x_{i}.A_{i} + \Sigma_{j} x_{j}.C_{j}
  \and
  N_{2} := \Sigma_{i'} y_{i'}.B_{i'} + \Sigma_{j'} y_{j'}.D_{j'} 
\end{mathpar}

as follows.

\begin{mathpar}
  \Sigma_{i} x_{i}.A_{i} + \Sigma_{j} x_{j}.C_{j} \otimes \Sigma_{i'}
  y_{i'}.B_{i'} + \Sigma_{j'} y_{j'}.D_{j'} 
  \and \\
  := \; \Sigma_{i} \Sigma_{i'} \quotep{\stackrel{\vee}{x_{i}}| \stackrel{\vee}{y_{i'}}}.(A_{i}\otimes B_{i'}) \; | \; \Sigma_{i'} \Sigma_{i} \quotep{\stackrel{\vee}{y_{i'}}|\stackrel{\vee}{x_{i}}}.(B_{i'}\otimes A_{i})
  \and
  \;\; | \;\; \Sigma_{j} \Sigma_{j'} \quotep{\stackrel{\vee}{x_{j}}|\stackrel{\vee}{y_{j'}}}.(A_{j}\otimes B_{j'}) \; | \; \Sigma_{j'} \Sigma_{j} \quotep{\stackrel{\vee}{y_{j'}}|\stackrel{\vee}{x_{j}}}.(B_{j'}\otimes A_{j})
\end{mathpar}

\begin{remark}
  Do we need to $x^{L}$ and $y^{R}$ for this construction as well?
\end{remark}

\paragraph{Tensor of parallel compositions} Next, we distribute tensor
over par.

\begin{mathpar}
  P_{1}|P_{2} \otimes Q_{1}|Q_{2} := (P_{1} \otimes Q_{1}) | (P_{1}
  \otimes Q_{2}) | (P_{2} \otimes Q_{1}) | (P_{2} \otimes Q_{2})
\end{mathpar}

\paragraph{Tensor with dropped names} We treat tensor of a
process with a dropped name as parallel composition.

\begin{mathpar}
  P \otimes \dropn{x} := P | \dropn{x}
\end{mathpar}

\paragraph{Tensor of agents}

Finally, we need to define tensor on agents. Note that the definition
of tensor on normal products only tensors inputs with inputs and
outputs with outputs. Thus, we only have to define the operation on
``homogeneous'' pairings.

\begin{mathpar}
  (\vec{x})P \otimes (\vec{y})Q
  \and \\
  := (x_{0}^{L}|y_{0}^{R},\ldots,x_{0}^{L}|y_{n}^{R},\ldots,x_{m}^{L}|y_{0}^{R},\ldots,x_{m}^{L}|y_{n}^R)(P\{ \vec{x}^{L}/\vec{x}\} \otimes Q \{ \vec{y}^{R}/\vec{y}\})
  \and \\
  \clift{\vec{P}} \otimes \clift{\vec{Q}}
  \and \\
  := \clift{P_{0}\otimes Q_{0},\ldots,P_{0}\otimes Q_{n},\ldots,P_{m}\otimes Q_{0},\ldots,P_{m}\otimes Q_{n}}
\end{mathpar}

\begin{remark}
  Observe that arities of tensored abstractions matches arities of
  tensored concretions if the original arities matched. Note also that
  the length of the arities corresponds to the increase in dimension
  we see in ordinary vector space tensor product.
\end{remark}

\begin{remark}
  Operationally, this definition distributes the tensor down to
  components ``linked'' by summation. Tensor over summation is
  intriguing in that it mixes names. Moreover, as a consequence of the
  way it mixes names we have the identities for all $x \in \QProc$ and
  $P,Q \in \Proc$

  \begin{mathpar}
    (x \cdot P) \otimes Q \equiv x \cdot (P \otimes Q) \equiv P \otimes (x \cdot Q)
    \and
    P \otimes \pzero \equiv P
  \end{mathpar}

  that the reader is invited to verify.
\end{remark}

\subsubsection{Annihilation}
\begin{mathpar}
  P^{\perp} := \{ Q | \forall R. P|Q \red^{*} R \Rightarrow R \red^{*} \pzero \}
  \and \\
  P^{\underline{\perp}} := \Sigma_{Q \in P^{\perp}} \quotep{Q}?(y).(\dropn{y}|Q) | \Sigma_{Q \in P^{\perp}} \quotep{Q}\clift{\Box}
\end{mathpar}

\paragraph{Discussion} The reader will note that $P^{\perp}$ is a
\emph{set} of processes, while $P^{\underline{\perp}}$ is a
\emph{context}. We call the set $P^{\perp}$ the \emph{annihilators} of
$P$. The parallel composition of a process in the annihilators of $P$
with $P$ will result in a process, the state space of which has all
paths eventually leading to $\pzero$. Execution may endure loops; but
under reasonable conditions of fairness (naturally guaranteed under
most notions of bisimulation) such a composite process cannot get
stuck in such a loop and will, eventually pop out and terminate.

The context $P^{\underline{\perp}}$ is ready and willing to ``take the
$P$ out of'' the process to which it is applied. It will effectively
transmit the code of the process to which it is applied to one of the
annihilators and run the process against it.

\subsubsection{Evaluation}
We fix $M$ a domain of fully abstract interpretation with an equality
coincident with bisimulation. We take $\meaningof{\cdot} : \Proc \to
M$ to be the map interpreting processes and $\nmeaningof{\cdot} : \M
\to Proc$ to be the map running the other way. Then we define

\begin{mathpar}
  \int P := \nmeaningof{\meaningof{P}}
\end{mathpar}

\paragraph{Discussion}
There are many fully abstract interpretations of Milner's
$\pi$-calculus. Any of them can be used as a basis for interpreting
the reflective calculus here. Equipped with such a domain it is
largely a matter of grinding through to check that the Yoneda
construction for the normalization-by-evaluation program can be
extended to this setting.

\begin{remark}
  The reader is invited to verify that $\int (P^{\underline{\perp}}[P]) = 0$.
\end{remark}

\subsection{Quantum mechanics}

Table \ref{tbl:core_qm_op_defns} gives the core operational definitions

\begin{table}[htp]\label{tbl:core_qm_op_defns}
  \center{
    \fbox{
      \begin{tabular}{c|c}
        quantum mechanics & process calculus \\
        \hline
        scalar & $x := \quotep{P}$ \\
        state vector & $\state{P} := P$ \\
        dual & $\state{P}^{*} := \event{P^{\underline{\perp}}} := \quotep{P^{\underline{\perp}}}[-]$ \\
        matrix & $ \Sigma_{\alpha} \state{P_{\alpha}}x_{\alpha}\event{Q_{\alpha}}$ \\
        vector addition & $\state{P} + \state{Q} := \state{P | Q}$ \\
        tensor product & $\state{P} \otimes \state{Q} := \state{P \otimes Q}$ \\
        inner product & $\innerprod{P}{Q} := \quotep{\int P^{\underline{\perp}}[Q]}$ \\
      \end{tabular}
    }
  }
  \caption{QM - operational definitions}
\end{table}

where

\begin{mathpar}
  \prmatrix{P}{Q} := \fprmatrix{P}{\quotep{\pzero}}{Q}
  \and
  \fprmatrix{P}{x}{Q} := (\state{P},x,\event{Q})
  \and
  (\fprmatrix{P}{x}{Q})(\state{R}) := x \cdot \innerprod{Q}{R} \cdot \state{P}
  \and
  (\fprmatrix{P}{x}{Q})(\event{R}) := x \cdot \innerprod{R}{P} \cdot \event{Q}
\end{mathpar}

\paragraph{Discussion}
As promised: vectors (aka states) are represented as processes; duals
as contextual duals; inner product definition should be compared with
standard inner product definition for ....

\begin{remark}
  Assuming $\int (P^{\underline{\perp}}[P]) = 0$, the reader is
  invited to verify that $(\fprmatrix{P}{x}{P})(\state{P}) = x \cdot \state{P}$.
\end{remark}

\begin{remark}
  The reader is invited to verify that $\innerprod{P}{Q}$ could
  equally well have been written $\quotep{\int \stackrel{\vee}{x}}$
  where $x = \event{P^{\underline{\perp}}}(Q)$.

  One of the motivations for this remark is that there is another way
  to factor these operations. We could package up evaluation in the dual:

  \begin{mathpar}
    \state{P}^{*} := \event{\int P^{\underline{\perp}}} := \quotep{\int P^{\underline{\perp}}}[-]
  \end{mathpar}

  and then have inner product defined by
  
  \begin{mathpar}
    \innerprod{P}{Q} := \event{P}(Q)
  \end{mathpar}

  Hopefully, experience with the calculations will provide guidance on
  the best factoring.
\end{remark}

\begin{remark}
  Assuming $\int (P^{\underline{\perp}}[P]) = 0$, the reader is
  invited to verify that $\forall P,Q. (\prmatrix{0}{Q})(\state{0}) =
  \state{0}$ and dually $(\prmatrix{P}{0})(\event{0}) = \event{0}$.
\end{remark}

\begin{remark}
  i'm a little worried that i don't (yet) have proper support for
  complex conjugacy. But, the observation above may give us a
  clue. According to Abramsky, it must be the case that the scalars
  are iso to the homset of the identity for the tensor -- which the
  observation above characterizes. 

  For now, we will simply bookmark the notion with $\overline{x}$.
\end{remark}

\subsubsection{Adjointness}

We need to give a definition of $(\cdot)^{\dagger}$ for matrices. The
obvious candidate definition is
\begin{mathpar}
(\Sigma_{\alpha}\fprmatrix{P_{\alpha}}{x_{\alpha}}{Q_{\alpha}})^{\dagger}
= \Sigma_{\alpha}\fprmatrix{(Q_{\alpha}^{\underline{\perp}})^{*}}{\overline{x}_{\alpha}}{P_{\alpha}^{\underline{\perp}}} 
\end{mathpar}

But, $(Q_{\alpha}^{\underline{\perp}})^{*}$ requires a name along
which to communicate the process to achieve the context application.

\subsubsection{Basis for a basis}
If processes label states and ``addition'' of states (a.k.a. vector
addition) is interpreted as parallel composition, what corresponds to
notions of linear independence and basis? Here, we recall that Yoshida
has developed a set of \emph{combinators} for an asynchronous verison
of Milner's $\pi$-calculus. These are a finite set of processes such
any process can be expressed as parallel composition of these
combinators together with liberal uses of the new operator and
replication. We can simply give a translation of these into the
present calculus and have reasonable expectation that the property
carries over. That is, that the resultant set allows to express all
processes via parallel composition. Note, however, that there is no
new operator or replication in this calculus. As a result, we expect
that the corresponding set is actually infinite. That is, we expect
that the space is actually infinite dimensional.

\begin{remark}
  The attentive reader may be a bit concerned. Certainly, the
  collection $S$, $K$ and $I$ is a finite set of
  combinators. Shouldn't we expect to see a finite set of combinators
  for an effectively equivalent system? i am very sympathetic to this
  critique and feel it warrants full attention. On the other hand, i
  also have in mind the following analogy. The natural numbers, as a
  monoid under addition, has exactly $1$ generator, while the natural
  numbers, as a monoid under multiplication, has countably many
  generators (the primes). We observe that the application of the
  lambda calculus is much less resource sensitive than the parallel
  composition of the $\pi$-calculus. Could it be the case that we have
  an analogy of the form
  
  \begin{mathpar}
    m + n : MN :: m*n : M|N
  \end{mathpar}

  giving a similar blow up in the set of ``primes''?  This is such a
  wonderful thought that, even if it's not true, i think it's worth
  writing down.
\end{remark}
 

\documentclass[12pt]{llncs}
%\documentclass{jktr}

\usepackage[pdftex]{hyperref}                   
\usepackage {listings}
\usepackage {mathpartir}
\usepackage{bcprules}
%\usepackage{listings}
                       
\usepackage{graphicx} 
%\usepackage[margins=2.5cm,nohead,nofoot]{geometry}
%\usepackage{geometry}
\usepackage{amsfonts}
\usepackage{amstext}
\usepackage{latexsym}
\usepackage{amssymb}
\usepackage{color}


%\include{myPreamble}
\include{qm2pi.local} 

%\ifpdf
%\usepackage[pdftex]{graphicx}
%\else
%\usepackage{graphicx}
%\fi

 % \ifpdf
%  \usepackage{pdfsync}
%  \if


%\title{Brief Article}
%\author{David F. Snyder}
%\author{L.G. Meredith}

%\address{Dept. of Math., Texas State University--San Marcos, San Marcos, TX 78666}
       
\pagestyle{empty}


\begin{document}

\lstset{language=[Objective]Caml,frame=shadowbox}

\input{qm2pi.front}

% section front matter (end)

\input{qm2pi.intro} 
 
% section introduction (end)

% \input{qm2pi.knotations} 

% section notation (end)

\input{qm2pi.process.calculi} 

% section concurrent_process_calculi_and_spatial_logics_ (end)
    
%\input{qm2pi.knots2pi} 

%\input{qm2pi.trefoil} 

%\input{qm2pi.mainthm} 

% subsection basic_interpretation (end)

%\input{qm2pi.rho.presentation} 
\subsection{The syntax and semantics of the notation system}\label{sub:the_syntax_and_semantics_of_the_notation_system} % (fold)

We now summarize a technical presentation of the calculus that
embodies our theory of dynamics. The typical presentation of such a
calculus follows the style of giving generators and relations on
them. The grammar, below, describing term constructors, freely
generates the set of processes, $\Proc$. This set is then quotiented
by a relation known as structural congruence and it is over this set
that the notion of dynamics is expressed. This presentation is
essentially that of \cite{MeredithR05} with the addition of
polyadicity and summation. For readability we have relegated some of
the technical subtleties to an appendix.

\subsubsection{Process grammar}\label{subsub:process_grammar}

\begin{mathpar}
  \inferrule* [lab=synchronization] {} {{M} \bc \pzero \;|\; x?F \;|\; x!C }
  \and
  \inferrule* [lab=abstraction] {} {{F} \bc (x)P}
  \and
  \inferrule* [lab=concretion] {} {{C} \bc \langle Q \rangle}
  \and
  \inferrule* [lab=process] {} {{P,Q} \bc M \;| \;P|Q \;|\; @{x}}
  \and
  \inferrule* [lab=name] {} {{x} \bc \quotep{P}}
\end{mathpar} 

Note that $\vec{x}$ (resp. $\vec{P}$) denotes a vector of names
(resp. processes) of length $|\vec{x}|$ (resp. $|\vec{P}|$). We adopt
the following useful abbreviations.

\begin{mathpar}
   x?(\vec{y}).P := x.(\vec{y})P \and  x\clift{\vec{P}} := x.\clift{\vec{P}}
   \and x!(y) := \lift{x}{\dropn{y}}
   \and \Pi_{i=0}^{n-1}P_i := P_0 | \ldots | P_{n-1}
\end{mathpar}

\subsubsection{Structural congruence}

\paragraph{Free and bound names and alpha-equivalence.} At the
core of structural equivalence is alpha-equivalence which identifies
process that are the same up to a change of variable. Formally, we
recognize the distinction between free and bound names. The free names
of a process, $\freenames{P}$, may be calculated recursively as
follows:

\begin{mathpar}
\freenames{\pzero} := \emptyset
  \and \\
  \freenames{x?(y).P} := \{ x \} \cup (\freenames{P} \setminus \{ y \})
  \and 
  \freenames{x!\langle P \rangle} := \{ x \} \cup \{ P \} 
  \and \\
  \freenames{P|Q} := \freenames{P} \cup \freenames{Q}
  \and \\
  \freenames{@{x}} := \{ x \}
\end{mathpar}

$\pi$
$\quotep{\pi}$

$\freenames{-} : \pi \to \mathcal{P}(\quotep{\pi})$

\begin{eqnarray*}
  \freenames{\pzero} & := & \emptyset \\
  \freenames{x?(y).P} & := & \{ x \} \cup (\freenames{P} \setminus \{ y \}) \\
  \freenames{x!\langle P \rangle} & := & \{ x \} \cup \{ P \} \\
  \freenames{P|Q} & := & \freenames{P} \cup \freenames{Q} \\
  \freenames{\dropn{x}} & := & \{ x \}
\end{eqnarray*}

The bound names of a process, $\boundnames{P}$, are those names occurring in $P$
that are not free. For example, in $x?(y).0$, the name $x$ is free, while $y$ is bound.

\begin{mathpar}
  \inferrule* [lab=monoidal-laws] {} { P|Q \equiv Q|P \and P|0 \equiv P \and P|(Q|R) \equiv (P|Q)|R }
\end{mathpar}

\begin{mathpar}
  \inferrule* [lab=alpha-equivalence] {} { (x)P \equiv (y)P\{y/x\} \and y \not\in \freenames{P} }
\end{mathpar}

\begin{definition}
Then two processes, $P,Q$, are alpha-equivalent if $P = Q\{\vec{y}/\vec{x}\}$ for
some $\vec{x} \in \boundnames{Q},\vec{y} \in \boundnames{P}$, where $Q\{\vec{y}/\vec{x}\}$
denotes the capture-avoiding substitution of $\vec{y}$ for $\vec{x}$ in $Q$.
\end{definition}

\begin{definition}
  The {\em structural congruence} \cite{SangiorgiWalker} , $\equiv$,
  between processes is the least congruence containing
  alpha-equivalence, satisfying the abelian monoid laws
  (associativity, commutativity and $\pzero$ as identity) for parallel
  composition $|$ and for summation $+$.
\end{definition}

\subsection{Name equivalence}

We take name equivalence, written $\nameeq$, to be the smallest
equivalence relation generated by the following rules.

\begin{mathpar}
\inferrule*[lab=Quote-drop]
{ }
{ \quotep{@{x}} \nameeq x }

\inferrule*[lab=Struct-equiv]
{ P \scong Q }
{ \quotep{P} \nameeq \quotep{Q} }
\end{mathpar}

The astute reader will have noticed that the mutual recursion of names
and processes imposes a mutual recursion on alpha-equivalence and
structural equivalence via name-equivalence. Fortunately, all of this
works out pleasantly and we may calculate in the natural way, free of
concern. The reader interested in the details is referred to the
appendix \ref{appendix:rho_details}.

\subsection{Substitution}

We use $\Proc$ for the set of processes, $\QProc$ for the set of
names, and $\id{\{}\vec{y} / \vec{x} \id{\}}$ to denote partial maps,
$s : \QProc \rightarrow \QProc$. A map, $s$ lifts, uniquely, to a map
on process terms, $\widehat{s} : \Proc \rightarrow \Proc$ by the
following equations.

\begin{mathpar}
  (0) \psubstp{Q}{P} := 0 \\
  (R \juxtap S) \psubstp{Q}{P}
  :=    
  (R)\psubstp{Q}{P} \juxtap (S) \psubstp{Q}{P} \\
  (x?(y).R) \psubstp{Q}{P}    
  :=    
  (x)\substp{Q}{P} (z)\concat( (R \psubstn{z}{y}) \psubstp{Q}{P} ) \\
  (\lift{x}{R}) \psubstp{Q}{P}  
  :=
  \lift{(x)\substp{Q}{P}}{ R \psubstp{Q}{P} } \\
%   (\dropn{x})  \psubstp{Q}{P}       
%   := 
%   \left\{ 
%     \begin{array}{ccc} 
%       \dropn{\quotep{Q}} & & x \nameeq \quotep{P} \\
%       \dropn{x} & & otherwise \\
%     \end{array}
%   \right. 
  (\dropn{x})  \psubstp{Q}{P}       
  := 
  \left\{ 
    \begin{array}{ccc} 
      Q & & x \nameeq \quotep{P} \\
      \dropn{x} & & otherwise \\
    \end{array}
  \right.
\end{mathpar}
 

where

\begin{eqnarray}
  (x)\id{\{} \lpquote Q \rpquote / \lpquote P \rpquote \id{\}}            = 
  \left\{ 
    \begin{array}{ccc}
      \lpquote Q \rpquote & & x \nameeq \lpquote P \rpquote \\
      x & & otherwise \\
    \end{array}
  \right. \nonumber
\end{eqnarray}

and $z$ is chosen distinct from $\quotep{P}$, $\quotep{Q}$, the free
names in $Q$, and all the names in $R$. Our $\alpha$-equivalence will
be built in the standard way from this substitution.

\begin{remark}\label{rem:no_self_referential_names}
  One consequence of these definitions is that $\forall P. \quotep{P}
  \not\in \freenames{P}$.
\end{remark}

\subsection{ Dynamic quote: an example }

Anticipating something of what's to come, consider applying the
substitution, $\widehat{\id{\{}u / z \id{\}}}$, to the following pair
of processes, $\lift{w}{y!(z)}$ and $w[ \lpquote y!(z) \rpquote ]$.

\begin{eqnarray}
	\lift{w}{y!(z)}\widehat{\id{\{}u / z \id{\}}}
		& = &
		\lift{w}{y!(u)} \nonumber\\
	w[ \lpquote y!(z) \rpquote ] \widehat{ \id{\{}u / z \id{\}} }
		& = &
		w[ \lpquote y!(z) \rpquote ] \nonumber
\end{eqnarray}

Because the body of the process between quotes is impervious to
substitution, we get radically different answers. In fact, by
examining the first process in an input context,
e.g. $x?(z).\lift{w}{y!(z)}$, we see that the process under the lift
operator may be shaped by prefixed inputs binding a name inside it. In
this sense, the lift operator will be seen as a way to dynamically
construct processes before reifying them as names.

Finally equipped with these standard features we can present the
dynamics of the calculus.

\subsubsection{Operational semantics} 

Finally, we introduce the computational dynamics. What marks these
algebras as distinct from other more traditionally studied algebraic
structures, e.g. vector spaces or polynomial rings, is the manner in
which dynamics is captured. In traditional structures, dynamics is typically
expressed through morphisms between such structures, as in linear maps
between vector spaces or morphisms between rings. In algebras
associated with the semantics of computation, the dynamics is
expressed as part of the algebraic structure itself, through a
reduction reduction relation typically denoted by $\red$. Below, we
give a recursive presentation of this relation for the calculus used
in the encoding.

$\red \subseteq \pi \times \pi$
$\red : \pi \to \mathcal{P}(\pi)$

\begin{mathpar}
  \inferrule* [lab=Comm] { \textsf{match}( x_{src}, x_{trgt} ) } { x_{trgt}?(y)P \; | \; x_{src}!\langle {Q} \rangle \red P\{\quotep{Q}/y}\} }
  \and \\
  \inferrule* [lab=Par] {{P} \red {P}'} {{{P} | {Q}} \red {{P}' | {Q}}}
  \and
  \inferrule* [lab=Equiv]{{{P} \scong {P}'} \andalso {{P}' \red {Q}'} \andalso {{Q}' \scong {Q}}}{{P} \red {Q}}
\end{mathpar}

\begin{eqnarray*}
  match_{\equiv} (\quotep{P},\quotep{Q}) & := & P \equiv Q \\
  match_{\dagger}(\quotep{P},\quotep{Q}) & := & \forall R. P|Q \red^{*} R => R \red^{*} 0 \\
  match_{K}(\quotep{P},\quotep{Q}) & := & K \mbox{ for some context } K
\end{eqnarray*}

$u?(x)P | u!\langle Q \rangle \red P\{\quotep{Q}/x\}$

%We write $\wred$ for $\red^*$, and $P\red$ if $\exists Q $ such that $ P \red Q$.
We write $P\red$ if $\exists Q $ such that $ P \red Q$ and $P\not\red$, otherwise.

\section{Replication}

As mentioned before, it is known that replication (and hence
recursion) can be implemented in a higher-order process algebra
\cite{SangiorgiWalker}. As our first example of calculation with the
machinery thus far presented we give the construction explicitly in
the {\rhoc}.

\begin{eqnarray}
	D_{x} & := & \prefix{x}{y}{(\binpar{\outputp{x}{y}}{@{y}})} \nonumber\\
	\bangp_{x}{P} & := & \binpar{{x}!\langle{\binpar{D_{x}}{P}}\rangle}{D_{x}} \nonumber
\end{eqnarray}

\begin{eqnarray}
	\bangp_{x}{P} & & \nonumber\\
	=
	& {x}!\langle{(\prefix{x}{y}{(\outputp{x}{y} | @{y})) | P}}\rangle 
	      | \prefix{x}{y}{(\outputp{x}{y} | @{y})} & \nonumber\\
	\red
	& (\outputp{x}{y} | @{y})\substn{\quotep{(\prefix{x}{y}{(@{y} | \outputp{x}{y})) | P}}}{y} & \nonumber\\
	=
	& \outputp{x}{\quotep{(\prefix{x}{y}{(\outputp{x}{y} | @{y})) | P}}}
	  | {(\prefix{x}{y}{(\outputp{x}{y} | @{y})) | P}} & \nonumber\\
	\red
	& \ldots & \nonumber\\
	\red^*
	& P | P | \ldots & \nonumber
\end{eqnarray}

Of course, this encoding, as an implementation, runs away, unfolding
$\bangp{P}$ eagerly. A lazier and more implementable replication
operator, restricted to input-guarded processes, may be obtained as follows.

\begin{eqnarray}
\bangp{\prefix{u}{v}{P}} 
	:= 
	\binpar{\lift{x}{\prefix{u}{v}{(\binpar{D(x)}{P})}}}{D(x)} \nonumber
\end{eqnarray}

\begin{remark}
  Note that the lazier definition still does not deal with summation
  or mixed summation (i.e. sums over input and output). The reader is
  invited to construct definitions of replication that deal with these
  features. 

  Further, the definitions are parameterized in a name, $x$. Can you,
  gentle reader, make a definition that eliminates this parameter and
  guarantees no accidental interaction between the replication
  machinery and the process being replicated -- i.e. no accidental
  sharing of names used by the process to get its work done and the
  name(s) used by the replication to effect copying. This latter
  revision of the definition of replication is crucial to obtaining
  the expected identity $!!P \sim !P$.
\end{remark}

\begin{remark}\label{rem:paradoxical_combinator}
  The reader familiar with the lambda calculus will have noticed the
  similarity between $D$ and the paradoxical combinator.

  [Ed. note: the existence of this seems to suggest we have to be more
  restrictive on the set of processes and names we admit if we are to
  support no-cloning.]
\end{remark}

\subsubsection{Bisimulation}

The computational dynamics gives rise to another kind of equivalence,
the equivalence of computational behavior. As previously mentioned
this is typically captured \emph{via} some form of bisimulation.

% The notion we use in this paper is weak barbed bisimulation
% \cite{milner91polyadicpi}.

The notion we use in this paper is derived from weak barbed
bisimulation \cite{milner91polyadicpi}. 

\begin{definition}
An \emph{observation relation}, $\downarrow_{\mathcal N}$, over a set
of names, $\mathcal N$, is the smallest relation satisfying the rules
below.

\infrule[Out-barb]{y \in {\mathcal N}, \; x \nameeq y}
		  {\outputp{x}{v} \downarrow_{\mathcal N} x}
\infrule[Par-barb]{\mbox{$P\downarrow_{\mathcal N} x$ or $Q\downarrow_{\mathcal N} x$}}
		  {\binpar{P}{Q} \downarrow_{\mathcal N} x}

We write $P \Downarrow_{\mathcal N} x$ if there is $Q$ such that 
$P \wred Q$ and $Q \downarrow_{\mathcal N} x$.
\end{definition}

\begin{definition}
%\label{def.bbisim}
An  ${\mathcal N}$-\emph{barbed bisimulation} over a set of names, ${\mathcal N}$, is a symmetric binary relation 
${\mathcal S}_{\mathcal N}$ between agents such that $P\rel{S}_{\mathcal N}Q$ implies:
\begin{enumerate}
\item If $P \red P'$ then $Q \wred Q'$ and $P'\rel{S}_{\mathcal N} Q'$.
\item If $P\downarrow_{\mathcal N} x$, then $Q\Downarrow_{\mathcal N} x$.
\end{enumerate}
$P$ is ${\mathcal N}$-barbed bisimilar to $Q$, written
$P \wbbisim_{\mathcal N} Q$, if $P \rel{S}_{\mathcal N} Q$ for some ${\mathcal N}$-barbed bisimulation ${\mathcal S}_{\mathcal N}$.
\end{definition}

$\mathcal{R} \subseteq \pi \times \pi$

$P \mathcal{R} Q => \forall P'. P \red P' \Rightarrow \exists Q'. Q \red Q', P' \mathcal{R} Q'$

$P \vdash x \Rightarrow Q \vdash x$

\begin{mathpar}
  \inferrule*[lab=Out-barb]{x \nameeq y}{{y}!\langle{Q}\rangle \vdash x}
  \and
  \inferrule*[lab=Par-barb]{\mbox{$P\vdash x$ or $Q\vdash x$}}{\binpar{P}{Q} \vdash x}
\end{mathpar}

\subsubsection{Contexts}

One of the principle advantages of computational calculi like the
$\pi$-calculus is a well-defined notion of context,
contextual-equivalence and a correlation between
contextual-equivalence and notions of bisimulation. The notion of
context allows the decomposition of a process into (sub-)process and
its syntactic environment, its context. Thus, a context may be
thought of as a process with a ``hole'' (written $\Box$) in it. The
application of a context $M$ to a process $P$, written $M[P]$, is
tantamount to filling the hole in $M$ with $P$. In this paper we do
not need the full weight of this theory, but do make use of the notion
of context in the proof the main theorem. 

\begin{mathpar}
  \inferrule* [lab=summation] {} {{M_{M},M_{N}} \bc \Box \;|\; x.M_{A} \;|\; M_{M}+M_{N}}
  \and
  \inferrule* [lab=agent] {} {{M_{A}} \bc (\vec{x})M_{P} \;| \; \clift{P_0,\ldots,M_{P},\ldots,P_N}}
  \and \\
  \inferrule* [lab=process] {} {{M_{P}} \bc M_{N} \;| \;P|M_{P} }
\end{mathpar} 

\begin{mathpar}
  \inferrule* [lab=sychronization] {} {M_{N} \bc \Box \;|\; x?M_{F} \;|\; x!M_{C}}
  \and
  \inferrule* [lab=abstraction] {} {{M_{F}} \bc (x)M_{P} }
  \and
  \inferrule* [lab=concretion] {} {{M_{C}} \bc \langle M_{P} \rangle }
  \and \\
  \inferrule* [lab=process] {} {{M_{P}} \bc M_{N} \;| \;P|M_{P} }
\end{mathpar}

\begin{definition}[contextual application] Given a context $M$, and
  process $P$, we define the \emph{contextual application}, $M[P] :=
  M\{P/\Box\}$. That is, the contextual application of M to P is the
  substitution of $P$ for $\Box$ in $M$.
\end{definition}

$\meaningof{-} : L \to \mathcal{P}(\pi)$

\begin{mathpar}
  \inferrule* [lab=collection] {} {\meaningof{true} = \pi, \and \meaningof{~E} = \pi \setminus \meaningof{E}, \and \meaningof{E_{1} \& E_{2}} = \meaningof{E_{1}} \cap \meaningof{E_{2}}}
\end{mathpar}

\begin{mathpar}
  \inferrule* [lab=structure] {} {\meaningof{0} = \{ P \in \pi | P \equiv 0 \}, \and \\ \meaningof{E_1 | E_2} = \{ P \in \pi | P \equiv P_{1} | P_{2}, P_{1} \in \meaningof{E_{1}}, P_{2} \in \meaningof{E_2}\} }
\end{mathpar}

\begin{mathpar}
 \inferrule* [lab=behavior] {} {\meaningof{\langle a?b \rangle E} = \{ P \in \pi | P \equiv Q | u?(y)P', \\ \and \\\\ \and \\ \;\;\; u \in \meaningof{a}, \forall z.P'\{z/y\} \in \meaningof{E\{z/b\}}\}, \and \\ \meaningof{a!E} = \{ P \in \pi | P \equiv Q | x!\langle P' \rangle, x \in \meaningof{a} P' \in \meaningof{E}\} }
\end{mathpar}

\begin{mathpar}
 \inferrule* [lab=nominal] {} {\meaningof{\quotep{E}} = \{ \quotep{P} \in \quotep{\pi} | P \in \meaningof{E} \}, \and \meaningof{\quotep{P}} = \{ \quotep{Q} \in \quotep{\pi} | P \equiv Q \} \and \\ \meaningof{@\quotep{E}} = \{ P \in \pi | P \equiv @x, x \in \meaningof{E} \}}
\end{mathpar}

\begin{eqnarray*}
  \\
  \meaningof{-} : TS \to ST
\end{eqnarray*}

\begin{eqnarray*}
  \\
  L : TS \to ST
\end{eqnarray*}

\begin{eqnarray*}
  \\
  P \models E \iff P \in \meaningof{E}
\end{eqnarray*}

\begin{eqnarray*}
  P \approx_{L} Q \iff \forall E \in L. P \models E \iff Q \models E
\end{eqnarray*}

\begin{eqnarray*}
  P \approx_{K} Q
\end{eqnarray*}

\begin{eqnarray*}
  P \approx Q
\end{eqnarray*}

$\approx_{K} = \approx = \approx_{L}$

\subsubsection{Contextual duality}

Note that contexts extend the quotation operation to a family of
operations from processes to names. Given a context, $M$, we can
define a \emph{nominal context}, $\quotep{M}$ by $\quotep{M}[P] :=
\quotep{M[P]}$. To foreshadow what is to come we observe that these
operations enjoy a duality with processes very much like the duality
between vectors and maps from vectors to scalars.

Further, because the calculus is essentially higher-order, we have a
correspondence between contexts and processes. More specifically,
given a name $x$ and a context $M$ we can construct $M^{*}_{x}$ such
that 

\begin{mathpar}
  M^{*}_{x} | \lift{x}{P} \red M[P]
\end{mathpar}

namely,

\begin{mathpar}
  M^{*}_{x} := x?(u).M[\dropn{u}]
\end{mathpar}

The dependence of $M^{*}_{x}$ on a name makes it an abstraction, 

\begin{mathpar}
  M^{*} := (x)x?(u).M[\dropn{u}]
\end{mathpar}

\subsection{Additional notation}

It will sometimes be convenient to denote the process a name
quotes. We already have the notation $x = \quotep{P}$, but it will be
convenient to introduce an alternate notation, $\procn{x}$, when we
want to emphasize the connection to the use of the name. Note that, by
virtue of name equivalence, $\quotep{\procn{x}} \nameeq x$; so, the
notation is consistent with previous definitions.

Further, because names have structure it is possible to effect
substitutions on the basis of that structure. This means we need to
upgrade our notation for substitutions, which we accomplish by
adapting comprehension notation. Thus,

\begin{mathpar}
  P\{ y / x : x \in S \}
\end{mathpar}

is interpreted to mean the process derived from P by replacing (in a
capture-avoiding manner) each occurrence of $x$ in $S$ by $y$. For example,

\begin{mathpar}
  P\{ \quotep{\procn{x}|\procn{x}} / x : x \in \freenames{P} \}
\end{mathpar}

will replace each (occurrence) of a free name $x$ in $P$ by
$\quotep{\procn{x}|\procn{x}}$.

Also, we will avail ourselves of the notation $x^{L}$ and $x^{R}$ to
denote injections of a name into disjoint copies of the name
space. There are numerous ways to accomplish this. One example can be
found in \cite{MeredithR05}. This notation overloads to vectors of
names: $\vec{x}^{\pi} := (x_{i}^{\pi} \; : \; 0 \leq i < |\vec{x}| )$ where $\pi \in \{L,R\}$.

We also use $P^{\Box} := P|\Box$.

In \cite{MeredithR05} an interpretation of the new operator is
given. It turns out that there are several possible interpretations
all enjoying the requisite algebraic properties of the operator (see
\cite{milner91polyadicpi}). We will therefore make liberal use of
$(\nu\; \vec{x})P$.

% subsection the_syntax_and_semantics_of_the_notation_system (end)   

\input{qm2pi.qmops} 

\input{qm2pi.sterngerlach} 

\input{qm2pi.metric} 

% section concurrent_process_calculi (end)

%\input{qm2pi.proofsketch}

% section proof sketch (end)

%\input{qm2pi.slviaknots} 

% section spatial logic via knots (end)

\input{qm2pi.conclusion}

% section conclusion (end)

%\input{qm2pi.dtcodes} 

% section wiring algorithm (end)

\input{qm2pi.ack} 

% section acknowledgments (end)

\newpage


\bibliographystyle{plain}   
\bibliography{../../biblios/main.bib}

\input{qm2pi.rhodetails}

\end{document}

 

\documentclass[12pt]{llncs}
%\documentclass{jktr}

\usepackage[pdftex]{hyperref}                   
\usepackage {listings}
\usepackage {mathpartir}
\usepackage{bcprules}
%\usepackage{listings}
                       
\usepackage{graphicx} 
%\usepackage[margins=2.5cm,nohead,nofoot]{geometry}
%\usepackage{geometry}
\usepackage{amsfonts}
\usepackage{amstext}
\usepackage{latexsym}
\usepackage{amssymb}
\usepackage{color}


%\include{myPreamble}
\include{qm2pi.local} 

%\ifpdf
%\usepackage[pdftex]{graphicx}
%\else
%\usepackage{graphicx}
%\fi

 % \ifpdf
%  \usepackage{pdfsync}
%  \if


%\title{Brief Article}
%\author{David F. Snyder}
%\author{L.G. Meredith}

%\address{Dept. of Math., Texas State University--San Marcos, San Marcos, TX 78666}
       
\pagestyle{empty}


\begin{document}

\lstset{language=[Objective]Caml,frame=shadowbox}

\input{qm2pi.front}

% section front matter (end)

\input{qm2pi.intro} 
 
% section introduction (end)

% \input{qm2pi.knotations} 

% section notation (end)

\input{qm2pi.process.calculi} 

% section concurrent_process_calculi_and_spatial_logics_ (end)
    
%\input{qm2pi.knots2pi} 

%\input{qm2pi.trefoil} 

%\input{qm2pi.mainthm} 

% subsection basic_interpretation (end)

%\input{qm2pi.rho.presentation} 
\subsection{The syntax and semantics of the notation system}\label{sub:the_syntax_and_semantics_of_the_notation_system} % (fold)

We now summarize a technical presentation of the calculus that
embodies our theory of dynamics. The typical presentation of such a
calculus follows the style of giving generators and relations on
them. The grammar, below, describing term constructors, freely
generates the set of processes, $\Proc$. This set is then quotiented
by a relation known as structural congruence and it is over this set
that the notion of dynamics is expressed. This presentation is
essentially that of \cite{MeredithR05} with the addition of
polyadicity and summation. For readability we have relegated some of
the technical subtleties to an appendix.

\subsubsection{Process grammar}\label{subsub:process_grammar}

\begin{mathpar}
  \inferrule* [lab=synchronization] {} {{M} \bc \pzero \;|\; x?F \;|\; x!C }
  \and
  \inferrule* [lab=abstraction] {} {{F} \bc (x)P}
  \and
  \inferrule* [lab=concretion] {} {{C} \bc \langle Q \rangle}
  \and
  \inferrule* [lab=process] {} {{P,Q} \bc M \;| \;P|Q \;|\; @{x}}
  \and
  \inferrule* [lab=name] {} {{x} \bc \quotep{P}}
\end{mathpar} 

Note that $\vec{x}$ (resp. $\vec{P}$) denotes a vector of names
(resp. processes) of length $|\vec{x}|$ (resp. $|\vec{P}|$). We adopt
the following useful abbreviations.

\begin{mathpar}
   x?(\vec{y}).P := x.(\vec{y})P \and  x\clift{\vec{P}} := x.\clift{\vec{P}}
   \and x!(y) := \lift{x}{\dropn{y}}
   \and \Pi_{i=0}^{n-1}P_i := P_0 | \ldots | P_{n-1}
\end{mathpar}

\subsubsection{Structural congruence}

\paragraph{Free and bound names and alpha-equivalence.} At the
core of structural equivalence is alpha-equivalence which identifies
process that are the same up to a change of variable. Formally, we
recognize the distinction between free and bound names. The free names
of a process, $\freenames{P}$, may be calculated recursively as
follows:

\begin{mathpar}
\freenames{\pzero} := \emptyset
  \and \\
  \freenames{x?(y).P} := \{ x \} \cup (\freenames{P} \setminus \{ y \})
  \and 
  \freenames{x!\langle P \rangle} := \{ x \} \cup \{ P \} 
  \and \\
  \freenames{P|Q} := \freenames{P} \cup \freenames{Q}
  \and \\
  \freenames{@{x}} := \{ x \}
\end{mathpar}

$\pi$
$\quotep{\pi}$

$\freenames{-} : \pi \to \mathcal{P}(\quotep{\pi})$

\begin{eqnarray*}
  \freenames{\pzero} & := & \emptyset \\
  \freenames{x?(y).P} & := & \{ x \} \cup (\freenames{P} \setminus \{ y \}) \\
  \freenames{x!\langle P \rangle} & := & \{ x \} \cup \{ P \} \\
  \freenames{P|Q} & := & \freenames{P} \cup \freenames{Q} \\
  \freenames{\dropn{x}} & := & \{ x \}
\end{eqnarray*}

The bound names of a process, $\boundnames{P}$, are those names occurring in $P$
that are not free. For example, in $x?(y).0$, the name $x$ is free, while $y$ is bound.

\begin{mathpar}
  \inferrule* [lab=monoidal-laws] {} { P|Q \equiv Q|P \and P|0 \equiv P \and P|(Q|R) \equiv (P|Q)|R }
\end{mathpar}

\begin{mathpar}
  \inferrule* [lab=alpha-equivalence] {} { (x)P \equiv (y)P\{y/x\} \and y \not\in \freenames{P} }
\end{mathpar}

\begin{definition}
Then two processes, $P,Q$, are alpha-equivalent if $P = Q\{\vec{y}/\vec{x}\}$ for
some $\vec{x} \in \boundnames{Q},\vec{y} \in \boundnames{P}$, where $Q\{\vec{y}/\vec{x}\}$
denotes the capture-avoiding substitution of $\vec{y}$ for $\vec{x}$ in $Q$.
\end{definition}

\begin{definition}
  The {\em structural congruence} \cite{SangiorgiWalker} , $\equiv$,
  between processes is the least congruence containing
  alpha-equivalence, satisfying the abelian monoid laws
  (associativity, commutativity and $\pzero$ as identity) for parallel
  composition $|$ and for summation $+$.
\end{definition}

\subsection{Name equivalence}

We take name equivalence, written $\nameeq$, to be the smallest
equivalence relation generated by the following rules.

\begin{mathpar}
\inferrule*[lab=Quote-drop]
{ }
{ \quotep{@{x}} \nameeq x }

\inferrule*[lab=Struct-equiv]
{ P \scong Q }
{ \quotep{P} \nameeq \quotep{Q} }
\end{mathpar}

The astute reader will have noticed that the mutual recursion of names
and processes imposes a mutual recursion on alpha-equivalence and
structural equivalence via name-equivalence. Fortunately, all of this
works out pleasantly and we may calculate in the natural way, free of
concern. The reader interested in the details is referred to the
appendix \ref{appendix:rho_details}.

\subsection{Substitution}

We use $\Proc$ for the set of processes, $\QProc$ for the set of
names, and $\id{\{}\vec{y} / \vec{x} \id{\}}$ to denote partial maps,
$s : \QProc \rightarrow \QProc$. A map, $s$ lifts, uniquely, to a map
on process terms, $\widehat{s} : \Proc \rightarrow \Proc$ by the
following equations.

\begin{mathpar}
  (0) \psubstp{Q}{P} := 0 \\
  (R \juxtap S) \psubstp{Q}{P}
  :=    
  (R)\psubstp{Q}{P} \juxtap (S) \psubstp{Q}{P} \\
  (x?(y).R) \psubstp{Q}{P}    
  :=    
  (x)\substp{Q}{P} (z)\concat( (R \psubstn{z}{y}) \psubstp{Q}{P} ) \\
  (\lift{x}{R}) \psubstp{Q}{P}  
  :=
  \lift{(x)\substp{Q}{P}}{ R \psubstp{Q}{P} } \\
%   (\dropn{x})  \psubstp{Q}{P}       
%   := 
%   \left\{ 
%     \begin{array}{ccc} 
%       \dropn{\quotep{Q}} & & x \nameeq \quotep{P} \\
%       \dropn{x} & & otherwise \\
%     \end{array}
%   \right. 
  (\dropn{x})  \psubstp{Q}{P}       
  := 
  \left\{ 
    \begin{array}{ccc} 
      Q & & x \nameeq \quotep{P} \\
      \dropn{x} & & otherwise \\
    \end{array}
  \right.
\end{mathpar}
 

where

\begin{eqnarray}
  (x)\id{\{} \lpquote Q \rpquote / \lpquote P \rpquote \id{\}}            = 
  \left\{ 
    \begin{array}{ccc}
      \lpquote Q \rpquote & & x \nameeq \lpquote P \rpquote \\
      x & & otherwise \\
    \end{array}
  \right. \nonumber
\end{eqnarray}

and $z$ is chosen distinct from $\quotep{P}$, $\quotep{Q}$, the free
names in $Q$, and all the names in $R$. Our $\alpha$-equivalence will
be built in the standard way from this substitution.

\begin{remark}\label{rem:no_self_referential_names}
  One consequence of these definitions is that $\forall P. \quotep{P}
  \not\in \freenames{P}$.
\end{remark}

\subsection{ Dynamic quote: an example }

Anticipating something of what's to come, consider applying the
substitution, $\widehat{\id{\{}u / z \id{\}}}$, to the following pair
of processes, $\lift{w}{y!(z)}$ and $w[ \lpquote y!(z) \rpquote ]$.

\begin{eqnarray}
	\lift{w}{y!(z)}\widehat{\id{\{}u / z \id{\}}}
		& = &
		\lift{w}{y!(u)} \nonumber\\
	w[ \lpquote y!(z) \rpquote ] \widehat{ \id{\{}u / z \id{\}} }
		& = &
		w[ \lpquote y!(z) \rpquote ] \nonumber
\end{eqnarray}

Because the body of the process between quotes is impervious to
substitution, we get radically different answers. In fact, by
examining the first process in an input context,
e.g. $x?(z).\lift{w}{y!(z)}$, we see that the process under the lift
operator may be shaped by prefixed inputs binding a name inside it. In
this sense, the lift operator will be seen as a way to dynamically
construct processes before reifying them as names.

Finally equipped with these standard features we can present the
dynamics of the calculus.

\subsubsection{Operational semantics} 

Finally, we introduce the computational dynamics. What marks these
algebras as distinct from other more traditionally studied algebraic
structures, e.g. vector spaces or polynomial rings, is the manner in
which dynamics is captured. In traditional structures, dynamics is typically
expressed through morphisms between such structures, as in linear maps
between vector spaces or morphisms between rings. In algebras
associated with the semantics of computation, the dynamics is
expressed as part of the algebraic structure itself, through a
reduction reduction relation typically denoted by $\red$. Below, we
give a recursive presentation of this relation for the calculus used
in the encoding.

$\red \subseteq \pi \times \pi$
$\red : \pi \to \mathcal{P}(\pi)$

\begin{mathpar}
  \inferrule* [lab=Comm] { \textsf{match}( x_{src}, x_{trgt} ) } { x_{trgt}?(y)P \; | \; x_{src}!\langle {Q} \rangle \red P\{\quotep{Q}/y}\} }
  \and \\
  \inferrule* [lab=Par] {{P} \red {P}'} {{{P} | {Q}} \red {{P}' | {Q}}}
  \and
  \inferrule* [lab=Equiv]{{{P} \scong {P}'} \andalso {{P}' \red {Q}'} \andalso {{Q}' \scong {Q}}}{{P} \red {Q}}
\end{mathpar}

\begin{eqnarray*}
  match_{\equiv} (\quotep{P},\quotep{Q}) & := & P \equiv Q \\
  match_{\dagger}(\quotep{P},\quotep{Q}) & := & \forall R. P|Q \red^{*} R => R \red^{*} 0 \\
  match_{K}(\quotep{P},\quotep{Q}) & := & K \mbox{ for some context } K
\end{eqnarray*}

$u?(x)P | u!\langle Q \rangle \red P\{\quotep{Q}/x\}$

%We write $\wred$ for $\red^*$, and $P\red$ if $\exists Q $ such that $ P \red Q$.
We write $P\red$ if $\exists Q $ such that $ P \red Q$ and $P\not\red$, otherwise.

\section{Replication}

As mentioned before, it is known that replication (and hence
recursion) can be implemented in a higher-order process algebra
\cite{SangiorgiWalker}. As our first example of calculation with the
machinery thus far presented we give the construction explicitly in
the {\rhoc}.

\begin{eqnarray}
	D_{x} & := & \prefix{x}{y}{(\binpar{\outputp{x}{y}}{@{y}})} \nonumber\\
	\bangp_{x}{P} & := & \binpar{{x}!\langle{\binpar{D_{x}}{P}}\rangle}{D_{x}} \nonumber
\end{eqnarray}

\begin{eqnarray}
	\bangp_{x}{P} & & \nonumber\\
	=
	& {x}!\langle{(\prefix{x}{y}{(\outputp{x}{y} | @{y})) | P}}\rangle 
	      | \prefix{x}{y}{(\outputp{x}{y} | @{y})} & \nonumber\\
	\red
	& (\outputp{x}{y} | @{y})\substn{\quotep{(\prefix{x}{y}{(@{y} | \outputp{x}{y})) | P}}}{y} & \nonumber\\
	=
	& \outputp{x}{\quotep{(\prefix{x}{y}{(\outputp{x}{y} | @{y})) | P}}}
	  | {(\prefix{x}{y}{(\outputp{x}{y} | @{y})) | P}} & \nonumber\\
	\red
	& \ldots & \nonumber\\
	\red^*
	& P | P | \ldots & \nonumber
\end{eqnarray}

Of course, this encoding, as an implementation, runs away, unfolding
$\bangp{P}$ eagerly. A lazier and more implementable replication
operator, restricted to input-guarded processes, may be obtained as follows.

\begin{eqnarray}
\bangp{\prefix{u}{v}{P}} 
	:= 
	\binpar{\lift{x}{\prefix{u}{v}{(\binpar{D(x)}{P})}}}{D(x)} \nonumber
\end{eqnarray}

\begin{remark}
  Note that the lazier definition still does not deal with summation
  or mixed summation (i.e. sums over input and output). The reader is
  invited to construct definitions of replication that deal with these
  features. 

  Further, the definitions are parameterized in a name, $x$. Can you,
  gentle reader, make a definition that eliminates this parameter and
  guarantees no accidental interaction between the replication
  machinery and the process being replicated -- i.e. no accidental
  sharing of names used by the process to get its work done and the
  name(s) used by the replication to effect copying. This latter
  revision of the definition of replication is crucial to obtaining
  the expected identity $!!P \sim !P$.
\end{remark}

\begin{remark}\label{rem:paradoxical_combinator}
  The reader familiar with the lambda calculus will have noticed the
  similarity between $D$ and the paradoxical combinator.

  [Ed. note: the existence of this seems to suggest we have to be more
  restrictive on the set of processes and names we admit if we are to
  support no-cloning.]
\end{remark}

\subsubsection{Bisimulation}

The computational dynamics gives rise to another kind of equivalence,
the equivalence of computational behavior. As previously mentioned
this is typically captured \emph{via} some form of bisimulation.

% The notion we use in this paper is weak barbed bisimulation
% \cite{milner91polyadicpi}.

The notion we use in this paper is derived from weak barbed
bisimulation \cite{milner91polyadicpi}. 

\begin{definition}
An \emph{observation relation}, $\downarrow_{\mathcal N}$, over a set
of names, $\mathcal N$, is the smallest relation satisfying the rules
below.

\infrule[Out-barb]{y \in {\mathcal N}, \; x \nameeq y}
		  {\outputp{x}{v} \downarrow_{\mathcal N} x}
\infrule[Par-barb]{\mbox{$P\downarrow_{\mathcal N} x$ or $Q\downarrow_{\mathcal N} x$}}
		  {\binpar{P}{Q} \downarrow_{\mathcal N} x}

We write $P \Downarrow_{\mathcal N} x$ if there is $Q$ such that 
$P \wred Q$ and $Q \downarrow_{\mathcal N} x$.
\end{definition}

\begin{definition}
%\label{def.bbisim}
An  ${\mathcal N}$-\emph{barbed bisimulation} over a set of names, ${\mathcal N}$, is a symmetric binary relation 
${\mathcal S}_{\mathcal N}$ between agents such that $P\rel{S}_{\mathcal N}Q$ implies:
\begin{enumerate}
\item If $P \red P'$ then $Q \wred Q'$ and $P'\rel{S}_{\mathcal N} Q'$.
\item If $P\downarrow_{\mathcal N} x$, then $Q\Downarrow_{\mathcal N} x$.
\end{enumerate}
$P$ is ${\mathcal N}$-barbed bisimilar to $Q$, written
$P \wbbisim_{\mathcal N} Q$, if $P \rel{S}_{\mathcal N} Q$ for some ${\mathcal N}$-barbed bisimulation ${\mathcal S}_{\mathcal N}$.
\end{definition}

$\mathcal{R} \subseteq \pi \times \pi$

$P \mathcal{R} Q => \forall P'. P \red P' \Rightarrow \exists Q'. Q \red Q', P' \mathcal{R} Q'$

$P \vdash x \Rightarrow Q \vdash x$

\begin{mathpar}
  \inferrule*[lab=Out-barb]{x \nameeq y}{{y}!\langle{Q}\rangle \vdash x}
  \and
  \inferrule*[lab=Par-barb]{\mbox{$P\vdash x$ or $Q\vdash x$}}{\binpar{P}{Q} \vdash x}
\end{mathpar}

\subsubsection{Contexts}

One of the principle advantages of computational calculi like the
$\pi$-calculus is a well-defined notion of context,
contextual-equivalence and a correlation between
contextual-equivalence and notions of bisimulation. The notion of
context allows the decomposition of a process into (sub-)process and
its syntactic environment, its context. Thus, a context may be
thought of as a process with a ``hole'' (written $\Box$) in it. The
application of a context $M$ to a process $P$, written $M[P]$, is
tantamount to filling the hole in $M$ with $P$. In this paper we do
not need the full weight of this theory, but do make use of the notion
of context in the proof the main theorem. 

\begin{mathpar}
  \inferrule* [lab=summation] {} {{M_{M},M_{N}} \bc \Box \;|\; x.M_{A} \;|\; M_{M}+M_{N}}
  \and
  \inferrule* [lab=agent] {} {{M_{A}} \bc (\vec{x})M_{P} \;| \; \clift{P_0,\ldots,M_{P},\ldots,P_N}}
  \and \\
  \inferrule* [lab=process] {} {{M_{P}} \bc M_{N} \;| \;P|M_{P} }
\end{mathpar} 

\begin{mathpar}
  \inferrule* [lab=sychronization] {} {M_{N} \bc \Box \;|\; x?M_{F} \;|\; x!M_{C}}
  \and
  \inferrule* [lab=abstraction] {} {{M_{F}} \bc (x)M_{P} }
  \and
  \inferrule* [lab=concretion] {} {{M_{C}} \bc \langle M_{P} \rangle }
  \and \\
  \inferrule* [lab=process] {} {{M_{P}} \bc M_{N} \;| \;P|M_{P} }
\end{mathpar}

\begin{definition}[contextual application] Given a context $M$, and
  process $P$, we define the \emph{contextual application}, $M[P] :=
  M\{P/\Box\}$. That is, the contextual application of M to P is the
  substitution of $P$ for $\Box$ in $M$.
\end{definition}

$\meaningof{-} : L \to \mathcal{P}(\pi)$

\begin{mathpar}
  \inferrule* [lab=collection] {} {\meaningof{true} = \pi, \and \meaningof{~E} = \pi \setminus \meaningof{E}, \and \meaningof{E_{1} \& E_{2}} = \meaningof{E_{1}} \cap \meaningof{E_{2}}}
\end{mathpar}

\begin{mathpar}
  \inferrule* [lab=structure] {} {\meaningof{0} = \{ P \in \pi | P \equiv 0 \}, \and \\ \meaningof{E_1 | E_2} = \{ P \in \pi | P \equiv P_{1} | P_{2}, P_{1} \in \meaningof{E_{1}}, P_{2} \in \meaningof{E_2}\} }
\end{mathpar}

\begin{mathpar}
 \inferrule* [lab=behavior] {} {\meaningof{\langle a?b \rangle E} = \{ P \in \pi | P \equiv Q | u?(y)P', \\ \and \\\\ \and \\ \;\;\; u \in \meaningof{a}, \forall z.P'\{z/y\} \in \meaningof{E\{z/b\}}\}, \and \\ \meaningof{a!E} = \{ P \in \pi | P \equiv Q | x!\langle P' \rangle, x \in \meaningof{a} P' \in \meaningof{E}\} }
\end{mathpar}

\begin{mathpar}
 \inferrule* [lab=nominal] {} {\meaningof{\quotep{E}} = \{ \quotep{P} \in \quotep{\pi} | P \in \meaningof{E} \}, \and \meaningof{\quotep{P}} = \{ \quotep{Q} \in \quotep{\pi} | P \equiv Q \} \and \\ \meaningof{@\quotep{E}} = \{ P \in \pi | P \equiv @x, x \in \meaningof{E} \}}
\end{mathpar}

\begin{eqnarray*}
  \\
  \meaningof{-} : TS \to ST
\end{eqnarray*}

\begin{eqnarray*}
  \\
  L : TS \to ST
\end{eqnarray*}

\begin{eqnarray*}
  \\
  P \models E \iff P \in \meaningof{E}
\end{eqnarray*}

\begin{eqnarray*}
  P \approx_{L} Q \iff \forall E \in L. P \models E \iff Q \models E
\end{eqnarray*}

\begin{eqnarray*}
  P \approx_{K} Q
\end{eqnarray*}

\begin{eqnarray*}
  P \approx Q
\end{eqnarray*}

$\approx_{K} = \approx = \approx_{L}$

\subsubsection{Contextual duality}

Note that contexts extend the quotation operation to a family of
operations from processes to names. Given a context, $M$, we can
define a \emph{nominal context}, $\quotep{M}$ by $\quotep{M}[P] :=
\quotep{M[P]}$. To foreshadow what is to come we observe that these
operations enjoy a duality with processes very much like the duality
between vectors and maps from vectors to scalars.

Further, because the calculus is essentially higher-order, we have a
correspondence between contexts and processes. More specifically,
given a name $x$ and a context $M$ we can construct $M^{*}_{x}$ such
that 

\begin{mathpar}
  M^{*}_{x} | \lift{x}{P} \red M[P]
\end{mathpar}

namely,

\begin{mathpar}
  M^{*}_{x} := x?(u).M[\dropn{u}]
\end{mathpar}

The dependence of $M^{*}_{x}$ on a name makes it an abstraction, 

\begin{mathpar}
  M^{*} := (x)x?(u).M[\dropn{u}]
\end{mathpar}

\subsection{Additional notation}

It will sometimes be convenient to denote the process a name
quotes. We already have the notation $x = \quotep{P}$, but it will be
convenient to introduce an alternate notation, $\procn{x}$, when we
want to emphasize the connection to the use of the name. Note that, by
virtue of name equivalence, $\quotep{\procn{x}} \nameeq x$; so, the
notation is consistent with previous definitions.

Further, because names have structure it is possible to effect
substitutions on the basis of that structure. This means we need to
upgrade our notation for substitutions, which we accomplish by
adapting comprehension notation. Thus,

\begin{mathpar}
  P\{ y / x : x \in S \}
\end{mathpar}

is interpreted to mean the process derived from P by replacing (in a
capture-avoiding manner) each occurrence of $x$ in $S$ by $y$. For example,

\begin{mathpar}
  P\{ \quotep{\procn{x}|\procn{x}} / x : x \in \freenames{P} \}
\end{mathpar}

will replace each (occurrence) of a free name $x$ in $P$ by
$\quotep{\procn{x}|\procn{x}}$.

Also, we will avail ourselves of the notation $x^{L}$ and $x^{R}$ to
denote injections of a name into disjoint copies of the name
space. There are numerous ways to accomplish this. One example can be
found in \cite{MeredithR05}. This notation overloads to vectors of
names: $\vec{x}^{\pi} := (x_{i}^{\pi} \; : \; 0 \leq i < |\vec{x}| )$ where $\pi \in \{L,R\}$.

We also use $P^{\Box} := P|\Box$.

In \cite{MeredithR05} an interpretation of the new operator is
given. It turns out that there are several possible interpretations
all enjoying the requisite algebraic properties of the operator (see
\cite{milner91polyadicpi}). We will therefore make liberal use of
$(\nu\; \vec{x})P$.

% subsection the_syntax_and_semantics_of_the_notation_system (end)   

\input{qm2pi.qmops} 

\input{qm2pi.sterngerlach} 

\input{qm2pi.metric} 

% section concurrent_process_calculi (end)

%\input{qm2pi.proofsketch}

% section proof sketch (end)

%\input{qm2pi.slviaknots} 

% section spatial logic via knots (end)

\input{qm2pi.conclusion}

% section conclusion (end)

%\input{qm2pi.dtcodes} 

% section wiring algorithm (end)

\input{qm2pi.ack} 

% section acknowledgments (end)

\newpage


\bibliographystyle{plain}   
\bibliography{../../biblios/main.bib}

\input{qm2pi.rhodetails}

\end{document}

 

% section concurrent_process_calculi (end)

%\documentclass[12pt]{llncs}
%\documentclass{jktr}

\usepackage[pdftex]{hyperref}                   
\usepackage {listings}
\usepackage {mathpartir}
\usepackage{bcprules}
%\usepackage{listings}
                       
\usepackage{graphicx} 
%\usepackage[margins=2.5cm,nohead,nofoot]{geometry}
%\usepackage{geometry}
\usepackage{amsfonts}
\usepackage{amstext}
\usepackage{latexsym}
\usepackage{amssymb}
\usepackage{color}


%\include{myPreamble}
\include{qm2pi.local} 

%\ifpdf
%\usepackage[pdftex]{graphicx}
%\else
%\usepackage{graphicx}
%\fi

 % \ifpdf
%  \usepackage{pdfsync}
%  \if


%\title{Brief Article}
%\author{David F. Snyder}
%\author{L.G. Meredith}

%\address{Dept. of Math., Texas State University--San Marcos, San Marcos, TX 78666}
       
\pagestyle{empty}


\begin{document}

\lstset{language=[Objective]Caml,frame=shadowbox}

\input{qm2pi.front}

% section front matter (end)

\input{qm2pi.intro} 
 
% section introduction (end)

% \input{qm2pi.knotations} 

% section notation (end)

\input{qm2pi.process.calculi} 

% section concurrent_process_calculi_and_spatial_logics_ (end)
    
%\input{qm2pi.knots2pi} 

%\input{qm2pi.trefoil} 

%\input{qm2pi.mainthm} 

% subsection basic_interpretation (end)

%\input{qm2pi.rho.presentation} 
\subsection{The syntax and semantics of the notation system}\label{sub:the_syntax_and_semantics_of_the_notation_system} % (fold)

We now summarize a technical presentation of the calculus that
embodies our theory of dynamics. The typical presentation of such a
calculus follows the style of giving generators and relations on
them. The grammar, below, describing term constructors, freely
generates the set of processes, $\Proc$. This set is then quotiented
by a relation known as structural congruence and it is over this set
that the notion of dynamics is expressed. This presentation is
essentially that of \cite{MeredithR05} with the addition of
polyadicity and summation. For readability we have relegated some of
the technical subtleties to an appendix.

\subsubsection{Process grammar}\label{subsub:process_grammar}

\begin{mathpar}
  \inferrule* [lab=synchronization] {} {{M} \bc \pzero \;|\; x?F \;|\; x!C }
  \and
  \inferrule* [lab=abstraction] {} {{F} \bc (x)P}
  \and
  \inferrule* [lab=concretion] {} {{C} \bc \langle Q \rangle}
  \and
  \inferrule* [lab=process] {} {{P,Q} \bc M \;| \;P|Q \;|\; @{x}}
  \and
  \inferrule* [lab=name] {} {{x} \bc \quotep{P}}
\end{mathpar} 

Note that $\vec{x}$ (resp. $\vec{P}$) denotes a vector of names
(resp. processes) of length $|\vec{x}|$ (resp. $|\vec{P}|$). We adopt
the following useful abbreviations.

\begin{mathpar}
   x?(\vec{y}).P := x.(\vec{y})P \and  x\clift{\vec{P}} := x.\clift{\vec{P}}
   \and x!(y) := \lift{x}{\dropn{y}}
   \and \Pi_{i=0}^{n-1}P_i := P_0 | \ldots | P_{n-1}
\end{mathpar}

\subsubsection{Structural congruence}

\paragraph{Free and bound names and alpha-equivalence.} At the
core of structural equivalence is alpha-equivalence which identifies
process that are the same up to a change of variable. Formally, we
recognize the distinction between free and bound names. The free names
of a process, $\freenames{P}$, may be calculated recursively as
follows:

\begin{mathpar}
\freenames{\pzero} := \emptyset
  \and \\
  \freenames{x?(y).P} := \{ x \} \cup (\freenames{P} \setminus \{ y \})
  \and 
  \freenames{x!\langle P \rangle} := \{ x \} \cup \{ P \} 
  \and \\
  \freenames{P|Q} := \freenames{P} \cup \freenames{Q}
  \and \\
  \freenames{@{x}} := \{ x \}
\end{mathpar}

$\pi$
$\quotep{\pi}$

$\freenames{-} : \pi \to \mathcal{P}(\quotep{\pi})$

\begin{eqnarray*}
  \freenames{\pzero} & := & \emptyset \\
  \freenames{x?(y).P} & := & \{ x \} \cup (\freenames{P} \setminus \{ y \}) \\
  \freenames{x!\langle P \rangle} & := & \{ x \} \cup \{ P \} \\
  \freenames{P|Q} & := & \freenames{P} \cup \freenames{Q} \\
  \freenames{\dropn{x}} & := & \{ x \}
\end{eqnarray*}

The bound names of a process, $\boundnames{P}$, are those names occurring in $P$
that are not free. For example, in $x?(y).0$, the name $x$ is free, while $y$ is bound.

\begin{mathpar}
  \inferrule* [lab=monoidal-laws] {} { P|Q \equiv Q|P \and P|0 \equiv P \and P|(Q|R) \equiv (P|Q)|R }
\end{mathpar}

\begin{mathpar}
  \inferrule* [lab=alpha-equivalence] {} { (x)P \equiv (y)P\{y/x\} \and y \not\in \freenames{P} }
\end{mathpar}

\begin{definition}
Then two processes, $P,Q$, are alpha-equivalent if $P = Q\{\vec{y}/\vec{x}\}$ for
some $\vec{x} \in \boundnames{Q},\vec{y} \in \boundnames{P}$, where $Q\{\vec{y}/\vec{x}\}$
denotes the capture-avoiding substitution of $\vec{y}$ for $\vec{x}$ in $Q$.
\end{definition}

\begin{definition}
  The {\em structural congruence} \cite{SangiorgiWalker} , $\equiv$,
  between processes is the least congruence containing
  alpha-equivalence, satisfying the abelian monoid laws
  (associativity, commutativity and $\pzero$ as identity) for parallel
  composition $|$ and for summation $+$.
\end{definition}

\subsection{Name equivalence}

We take name equivalence, written $\nameeq$, to be the smallest
equivalence relation generated by the following rules.

\begin{mathpar}
\inferrule*[lab=Quote-drop]
{ }
{ \quotep{@{x}} \nameeq x }

\inferrule*[lab=Struct-equiv]
{ P \scong Q }
{ \quotep{P} \nameeq \quotep{Q} }
\end{mathpar}

The astute reader will have noticed that the mutual recursion of names
and processes imposes a mutual recursion on alpha-equivalence and
structural equivalence via name-equivalence. Fortunately, all of this
works out pleasantly and we may calculate in the natural way, free of
concern. The reader interested in the details is referred to the
appendix \ref{appendix:rho_details}.

\subsection{Substitution}

We use $\Proc$ for the set of processes, $\QProc$ for the set of
names, and $\id{\{}\vec{y} / \vec{x} \id{\}}$ to denote partial maps,
$s : \QProc \rightarrow \QProc$. A map, $s$ lifts, uniquely, to a map
on process terms, $\widehat{s} : \Proc \rightarrow \Proc$ by the
following equations.

\begin{mathpar}
  (0) \psubstp{Q}{P} := 0 \\
  (R \juxtap S) \psubstp{Q}{P}
  :=    
  (R)\psubstp{Q}{P} \juxtap (S) \psubstp{Q}{P} \\
  (x?(y).R) \psubstp{Q}{P}    
  :=    
  (x)\substp{Q}{P} (z)\concat( (R \psubstn{z}{y}) \psubstp{Q}{P} ) \\
  (\lift{x}{R}) \psubstp{Q}{P}  
  :=
  \lift{(x)\substp{Q}{P}}{ R \psubstp{Q}{P} } \\
%   (\dropn{x})  \psubstp{Q}{P}       
%   := 
%   \left\{ 
%     \begin{array}{ccc} 
%       \dropn{\quotep{Q}} & & x \nameeq \quotep{P} \\
%       \dropn{x} & & otherwise \\
%     \end{array}
%   \right. 
  (\dropn{x})  \psubstp{Q}{P}       
  := 
  \left\{ 
    \begin{array}{ccc} 
      Q & & x \nameeq \quotep{P} \\
      \dropn{x} & & otherwise \\
    \end{array}
  \right.
\end{mathpar}
 

where

\begin{eqnarray}
  (x)\id{\{} \lpquote Q \rpquote / \lpquote P \rpquote \id{\}}            = 
  \left\{ 
    \begin{array}{ccc}
      \lpquote Q \rpquote & & x \nameeq \lpquote P \rpquote \\
      x & & otherwise \\
    \end{array}
  \right. \nonumber
\end{eqnarray}

and $z$ is chosen distinct from $\quotep{P}$, $\quotep{Q}$, the free
names in $Q$, and all the names in $R$. Our $\alpha$-equivalence will
be built in the standard way from this substitution.

\begin{remark}\label{rem:no_self_referential_names}
  One consequence of these definitions is that $\forall P. \quotep{P}
  \not\in \freenames{P}$.
\end{remark}

\subsection{ Dynamic quote: an example }

Anticipating something of what's to come, consider applying the
substitution, $\widehat{\id{\{}u / z \id{\}}}$, to the following pair
of processes, $\lift{w}{y!(z)}$ and $w[ \lpquote y!(z) \rpquote ]$.

\begin{eqnarray}
	\lift{w}{y!(z)}\widehat{\id{\{}u / z \id{\}}}
		& = &
		\lift{w}{y!(u)} \nonumber\\
	w[ \lpquote y!(z) \rpquote ] \widehat{ \id{\{}u / z \id{\}} }
		& = &
		w[ \lpquote y!(z) \rpquote ] \nonumber
\end{eqnarray}

Because the body of the process between quotes is impervious to
substitution, we get radically different answers. In fact, by
examining the first process in an input context,
e.g. $x?(z).\lift{w}{y!(z)}$, we see that the process under the lift
operator may be shaped by prefixed inputs binding a name inside it. In
this sense, the lift operator will be seen as a way to dynamically
construct processes before reifying them as names.

Finally equipped with these standard features we can present the
dynamics of the calculus.

\subsubsection{Operational semantics} 

Finally, we introduce the computational dynamics. What marks these
algebras as distinct from other more traditionally studied algebraic
structures, e.g. vector spaces or polynomial rings, is the manner in
which dynamics is captured. In traditional structures, dynamics is typically
expressed through morphisms between such structures, as in linear maps
between vector spaces or morphisms between rings. In algebras
associated with the semantics of computation, the dynamics is
expressed as part of the algebraic structure itself, through a
reduction reduction relation typically denoted by $\red$. Below, we
give a recursive presentation of this relation for the calculus used
in the encoding.

$\red \subseteq \pi \times \pi$
$\red : \pi \to \mathcal{P}(\pi)$

\begin{mathpar}
  \inferrule* [lab=Comm] { \textsf{match}( x_{src}, x_{trgt} ) } { x_{trgt}?(y)P \; | \; x_{src}!\langle {Q} \rangle \red P\{\quotep{Q}/y}\} }
  \and \\
  \inferrule* [lab=Par] {{P} \red {P}'} {{{P} | {Q}} \red {{P}' | {Q}}}
  \and
  \inferrule* [lab=Equiv]{{{P} \scong {P}'} \andalso {{P}' \red {Q}'} \andalso {{Q}' \scong {Q}}}{{P} \red {Q}}
\end{mathpar}

\begin{eqnarray*}
  match_{\equiv} (\quotep{P},\quotep{Q}) & := & P \equiv Q \\
  match_{\dagger}(\quotep{P},\quotep{Q}) & := & \forall R. P|Q \red^{*} R => R \red^{*} 0 \\
  match_{K}(\quotep{P},\quotep{Q}) & := & K \mbox{ for some context } K
\end{eqnarray*}

$u?(x)P | u!\langle Q \rangle \red P\{\quotep{Q}/x\}$

%We write $\wred$ for $\red^*$, and $P\red$ if $\exists Q $ such that $ P \red Q$.
We write $P\red$ if $\exists Q $ such that $ P \red Q$ and $P\not\red$, otherwise.

\section{Replication}

As mentioned before, it is known that replication (and hence
recursion) can be implemented in a higher-order process algebra
\cite{SangiorgiWalker}. As our first example of calculation with the
machinery thus far presented we give the construction explicitly in
the {\rhoc}.

\begin{eqnarray}
	D_{x} & := & \prefix{x}{y}{(\binpar{\outputp{x}{y}}{@{y}})} \nonumber\\
	\bangp_{x}{P} & := & \binpar{{x}!\langle{\binpar{D_{x}}{P}}\rangle}{D_{x}} \nonumber
\end{eqnarray}

\begin{eqnarray}
	\bangp_{x}{P} & & \nonumber\\
	=
	& {x}!\langle{(\prefix{x}{y}{(\outputp{x}{y} | @{y})) | P}}\rangle 
	      | \prefix{x}{y}{(\outputp{x}{y} | @{y})} & \nonumber\\
	\red
	& (\outputp{x}{y} | @{y})\substn{\quotep{(\prefix{x}{y}{(@{y} | \outputp{x}{y})) | P}}}{y} & \nonumber\\
	=
	& \outputp{x}{\quotep{(\prefix{x}{y}{(\outputp{x}{y} | @{y})) | P}}}
	  | {(\prefix{x}{y}{(\outputp{x}{y} | @{y})) | P}} & \nonumber\\
	\red
	& \ldots & \nonumber\\
	\red^*
	& P | P | \ldots & \nonumber
\end{eqnarray}

Of course, this encoding, as an implementation, runs away, unfolding
$\bangp{P}$ eagerly. A lazier and more implementable replication
operator, restricted to input-guarded processes, may be obtained as follows.

\begin{eqnarray}
\bangp{\prefix{u}{v}{P}} 
	:= 
	\binpar{\lift{x}{\prefix{u}{v}{(\binpar{D(x)}{P})}}}{D(x)} \nonumber
\end{eqnarray}

\begin{remark}
  Note that the lazier definition still does not deal with summation
  or mixed summation (i.e. sums over input and output). The reader is
  invited to construct definitions of replication that deal with these
  features. 

  Further, the definitions are parameterized in a name, $x$. Can you,
  gentle reader, make a definition that eliminates this parameter and
  guarantees no accidental interaction between the replication
  machinery and the process being replicated -- i.e. no accidental
  sharing of names used by the process to get its work done and the
  name(s) used by the replication to effect copying. This latter
  revision of the definition of replication is crucial to obtaining
  the expected identity $!!P \sim !P$.
\end{remark}

\begin{remark}\label{rem:paradoxical_combinator}
  The reader familiar with the lambda calculus will have noticed the
  similarity between $D$ and the paradoxical combinator.

  [Ed. note: the existence of this seems to suggest we have to be more
  restrictive on the set of processes and names we admit if we are to
  support no-cloning.]
\end{remark}

\subsubsection{Bisimulation}

The computational dynamics gives rise to another kind of equivalence,
the equivalence of computational behavior. As previously mentioned
this is typically captured \emph{via} some form of bisimulation.

% The notion we use in this paper is weak barbed bisimulation
% \cite{milner91polyadicpi}.

The notion we use in this paper is derived from weak barbed
bisimulation \cite{milner91polyadicpi}. 

\begin{definition}
An \emph{observation relation}, $\downarrow_{\mathcal N}$, over a set
of names, $\mathcal N$, is the smallest relation satisfying the rules
below.

\infrule[Out-barb]{y \in {\mathcal N}, \; x \nameeq y}
		  {\outputp{x}{v} \downarrow_{\mathcal N} x}
\infrule[Par-barb]{\mbox{$P\downarrow_{\mathcal N} x$ or $Q\downarrow_{\mathcal N} x$}}
		  {\binpar{P}{Q} \downarrow_{\mathcal N} x}

We write $P \Downarrow_{\mathcal N} x$ if there is $Q$ such that 
$P \wred Q$ and $Q \downarrow_{\mathcal N} x$.
\end{definition}

\begin{definition}
%\label{def.bbisim}
An  ${\mathcal N}$-\emph{barbed bisimulation} over a set of names, ${\mathcal N}$, is a symmetric binary relation 
${\mathcal S}_{\mathcal N}$ between agents such that $P\rel{S}_{\mathcal N}Q$ implies:
\begin{enumerate}
\item If $P \red P'$ then $Q \wred Q'$ and $P'\rel{S}_{\mathcal N} Q'$.
\item If $P\downarrow_{\mathcal N} x$, then $Q\Downarrow_{\mathcal N} x$.
\end{enumerate}
$P$ is ${\mathcal N}$-barbed bisimilar to $Q$, written
$P \wbbisim_{\mathcal N} Q$, if $P \rel{S}_{\mathcal N} Q$ for some ${\mathcal N}$-barbed bisimulation ${\mathcal S}_{\mathcal N}$.
\end{definition}

$\mathcal{R} \subseteq \pi \times \pi$

$P \mathcal{R} Q => \forall P'. P \red P' \Rightarrow \exists Q'. Q \red Q', P' \mathcal{R} Q'$

$P \vdash x \Rightarrow Q \vdash x$

\begin{mathpar}
  \inferrule*[lab=Out-barb]{x \nameeq y}{{y}!\langle{Q}\rangle \vdash x}
  \and
  \inferrule*[lab=Par-barb]{\mbox{$P\vdash x$ or $Q\vdash x$}}{\binpar{P}{Q} \vdash x}
\end{mathpar}

\subsubsection{Contexts}

One of the principle advantages of computational calculi like the
$\pi$-calculus is a well-defined notion of context,
contextual-equivalence and a correlation between
contextual-equivalence and notions of bisimulation. The notion of
context allows the decomposition of a process into (sub-)process and
its syntactic environment, its context. Thus, a context may be
thought of as a process with a ``hole'' (written $\Box$) in it. The
application of a context $M$ to a process $P$, written $M[P]$, is
tantamount to filling the hole in $M$ with $P$. In this paper we do
not need the full weight of this theory, but do make use of the notion
of context in the proof the main theorem. 

\begin{mathpar}
  \inferrule* [lab=summation] {} {{M_{M},M_{N}} \bc \Box \;|\; x.M_{A} \;|\; M_{M}+M_{N}}
  \and
  \inferrule* [lab=agent] {} {{M_{A}} \bc (\vec{x})M_{P} \;| \; \clift{P_0,\ldots,M_{P},\ldots,P_N}}
  \and \\
  \inferrule* [lab=process] {} {{M_{P}} \bc M_{N} \;| \;P|M_{P} }
\end{mathpar} 

\begin{mathpar}
  \inferrule* [lab=sychronization] {} {M_{N} \bc \Box \;|\; x?M_{F} \;|\; x!M_{C}}
  \and
  \inferrule* [lab=abstraction] {} {{M_{F}} \bc (x)M_{P} }
  \and
  \inferrule* [lab=concretion] {} {{M_{C}} \bc \langle M_{P} \rangle }
  \and \\
  \inferrule* [lab=process] {} {{M_{P}} \bc M_{N} \;| \;P|M_{P} }
\end{mathpar}

\begin{definition}[contextual application] Given a context $M$, and
  process $P$, we define the \emph{contextual application}, $M[P] :=
  M\{P/\Box\}$. That is, the contextual application of M to P is the
  substitution of $P$ for $\Box$ in $M$.
\end{definition}

$\meaningof{-} : L \to \mathcal{P}(\pi)$

\begin{mathpar}
  \inferrule* [lab=collection] {} {\meaningof{true} = \pi, \and \meaningof{~E} = \pi \setminus \meaningof{E}, \and \meaningof{E_{1} \& E_{2}} = \meaningof{E_{1}} \cap \meaningof{E_{2}}}
\end{mathpar}

\begin{mathpar}
  \inferrule* [lab=structure] {} {\meaningof{0} = \{ P \in \pi | P \equiv 0 \}, \and \\ \meaningof{E_1 | E_2} = \{ P \in \pi | P \equiv P_{1} | P_{2}, P_{1} \in \meaningof{E_{1}}, P_{2} \in \meaningof{E_2}\} }
\end{mathpar}

\begin{mathpar}
 \inferrule* [lab=behavior] {} {\meaningof{\langle a?b \rangle E} = \{ P \in \pi | P \equiv Q | u?(y)P', \\ \and \\\\ \and \\ \;\;\; u \in \meaningof{a}, \forall z.P'\{z/y\} \in \meaningof{E\{z/b\}}\}, \and \\ \meaningof{a!E} = \{ P \in \pi | P \equiv Q | x!\langle P' \rangle, x \in \meaningof{a} P' \in \meaningof{E}\} }
\end{mathpar}

\begin{mathpar}
 \inferrule* [lab=nominal] {} {\meaningof{\quotep{E}} = \{ \quotep{P} \in \quotep{\pi} | P \in \meaningof{E} \}, \and \meaningof{\quotep{P}} = \{ \quotep{Q} \in \quotep{\pi} | P \equiv Q \} \and \\ \meaningof{@\quotep{E}} = \{ P \in \pi | P \equiv @x, x \in \meaningof{E} \}}
\end{mathpar}

\begin{eqnarray*}
  \\
  \meaningof{-} : TS \to ST
\end{eqnarray*}

\begin{eqnarray*}
  \\
  L : TS \to ST
\end{eqnarray*}

\begin{eqnarray*}
  \\
  P \models E \iff P \in \meaningof{E}
\end{eqnarray*}

\begin{eqnarray*}
  P \approx_{L} Q \iff \forall E \in L. P \models E \iff Q \models E
\end{eqnarray*}

\begin{eqnarray*}
  P \approx_{K} Q
\end{eqnarray*}

\begin{eqnarray*}
  P \approx Q
\end{eqnarray*}

$\approx_{K} = \approx = \approx_{L}$

\subsubsection{Contextual duality}

Note that contexts extend the quotation operation to a family of
operations from processes to names. Given a context, $M$, we can
define a \emph{nominal context}, $\quotep{M}$ by $\quotep{M}[P] :=
\quotep{M[P]}$. To foreshadow what is to come we observe that these
operations enjoy a duality with processes very much like the duality
between vectors and maps from vectors to scalars.

Further, because the calculus is essentially higher-order, we have a
correspondence between contexts and processes. More specifically,
given a name $x$ and a context $M$ we can construct $M^{*}_{x}$ such
that 

\begin{mathpar}
  M^{*}_{x} | \lift{x}{P} \red M[P]
\end{mathpar}

namely,

\begin{mathpar}
  M^{*}_{x} := x?(u).M[\dropn{u}]
\end{mathpar}

The dependence of $M^{*}_{x}$ on a name makes it an abstraction, 

\begin{mathpar}
  M^{*} := (x)x?(u).M[\dropn{u}]
\end{mathpar}

\subsection{Additional notation}

It will sometimes be convenient to denote the process a name
quotes. We already have the notation $x = \quotep{P}$, but it will be
convenient to introduce an alternate notation, $\procn{x}$, when we
want to emphasize the connection to the use of the name. Note that, by
virtue of name equivalence, $\quotep{\procn{x}} \nameeq x$; so, the
notation is consistent with previous definitions.

Further, because names have structure it is possible to effect
substitutions on the basis of that structure. This means we need to
upgrade our notation for substitutions, which we accomplish by
adapting comprehension notation. Thus,

\begin{mathpar}
  P\{ y / x : x \in S \}
\end{mathpar}

is interpreted to mean the process derived from P by replacing (in a
capture-avoiding manner) each occurrence of $x$ in $S$ by $y$. For example,

\begin{mathpar}
  P\{ \quotep{\procn{x}|\procn{x}} / x : x \in \freenames{P} \}
\end{mathpar}

will replace each (occurrence) of a free name $x$ in $P$ by
$\quotep{\procn{x}|\procn{x}}$.

Also, we will avail ourselves of the notation $x^{L}$ and $x^{R}$ to
denote injections of a name into disjoint copies of the name
space. There are numerous ways to accomplish this. One example can be
found in \cite{MeredithR05}. This notation overloads to vectors of
names: $\vec{x}^{\pi} := (x_{i}^{\pi} \; : \; 0 \leq i < |\vec{x}| )$ where $\pi \in \{L,R\}$.

We also use $P^{\Box} := P|\Box$.

In \cite{MeredithR05} an interpretation of the new operator is
given. It turns out that there are several possible interpretations
all enjoying the requisite algebraic properties of the operator (see
\cite{milner91polyadicpi}). We will therefore make liberal use of
$(\nu\; \vec{x})P$.

% subsection the_syntax_and_semantics_of_the_notation_system (end)   

\input{qm2pi.qmops} 

\input{qm2pi.sterngerlach} 

\input{qm2pi.metric} 

% section concurrent_process_calculi (end)

%\input{qm2pi.proofsketch}

% section proof sketch (end)

%\input{qm2pi.slviaknots} 

% section spatial logic via knots (end)

\input{qm2pi.conclusion}

% section conclusion (end)

%\input{qm2pi.dtcodes} 

% section wiring algorithm (end)

\input{qm2pi.ack} 

% section acknowledgments (end)

\newpage


\bibliographystyle{plain}   
\bibliography{../../biblios/main.bib}

\input{qm2pi.rhodetails}

\end{document}



% section proof sketch (end)

%\section{Unlikely characters: spatial logic for
  knots}\label{sub:characteristic_formulae} % (fold)

Associated to the mobile process calculi are a family of logics known
as the Hennessy-Milner logics. These logics typically enjoy a
semantics interpreting formulae as sets of processes that when
factored through the encoding outlined above allows an identification
of classes of knots with logical formulae. In the context of this
encoding the sub-family known as the spatial logics \cite{CairesC03}
\cite{CairesC04} \cite{Caires04} are of particular interest providing
several important features for expressing and reasoning about
properties (i.e. classes) of knots. We hint here at how this may be done.

%\begin{description}
%\item [structural connectives] 
\subsubsection{Structural connectives} The spatial logics enjoy
structural connectives corresponding, at the logical level, to the
parallel composition ($P | Q$) and new name ($(\nu \; x)P$)
connectives for processes. As illustrated in the examples below, these
connectives are extremely expressive given the shape of our encoding.
%\item [decideable satisfaction]

\subsubsection{Decideable satisfaction}
In \cite{Caires04} the satisfaction relation is shown to be decideable
for a rich class of processes. It further turns out that the image of
the our encoding is a proper subset of that class. This result
provides the basis for an algorithm by which to search for knots
enjoying a given property.
%\item [characteristic formulae]

\subsubsection{Characteristic formulae}
In the same paper \cite{Caires04} , Caires presents a means of calculating
characteristic formulae, selecting equivalence classes of processes
up to a pre--specified depth limit on the support set of names. Composed with our
encoding, this characteristic formula can be used to select
characteristic formulae for knots.
%\end{description}

\subsubsection{Spatial logic formulae}

The grammar below (segmented for comprehension) summarizes the syntax
of spatial logic formulae. We employ illustrative examples in the
sequel to provide an intuitive understanding of their meaning
referring the reader to \cite{Caires04} for a more detailed explication
of the semantics.

\begin{mathpar}
  \inferrule* [lab=boolean] {} {{A,B} \bc T \;|\; \neg A \;|\; A \wedge B \;|\; \eta = \eta'}
  \and
  \inferrule* [lab=spatial] {} {|\; \pzero \;|\; A | B \;|\; x \text{\textregistered} A \;|\; \forall x . A \;|\;  H x . A}
  \and
  \inferrule* [lab=behavioral] {} {|\; \alpha . A}
  \and 
  \inferrule* [lab=recursion] {} {|\; X(\vec{u}) \;|\; \mu X(\vec{u}) . A}
  \and
  \inferrule* [lab=action] {} {\alpha \bc \langle x?(\vec{y}) \rangle \;|\; \langle x!(\vec{y}) \rangle \;|\; \langle \tau \rangle}
  \and 
  \inferrule* [lab=name] {} {\eta \bc x \;|\; \tau}
\end{mathpar} 

% subsection characteristic_formulae (end)   	 

\subsection{Example formulae}\label{sub:example_formulae_} % (fold)

\subsubsection{Crossing as formula.}
% 
% \begin{align*}
%   \frac{d}{dx} \sin x &= \cos x 
%   & \frac{d}{dx} e^x &= e^x \\
%   \frac{d}{dx} \cos x &= - \sin x 
%   & \frac{d}{dx} \log x &= \frac{1}{x} \\
% \end{align*} 

\begin{align*}
 \mu C(x_{0},x_{1},y_{0},y_{1},u).&(\langle x_{0}?(z) \rangle(\langle u! \rangle\langle y_{1}!z \rangle C(x_{0},x_{1},y_{0},y_{1},u)) & \\
  & \wedge \langle y_{1}?(z) \rangle (\langle u! \rangle \langle x_{0}!z \rangle C(x_{0},x_{1},y_{0},y_{1},u)) & \\
  & \wedge \langle x_{1}?(z) \rangle (\langle u? \rangle \langle y_{0}!z \rangle C(x_{0},x_{1},y_{0},y_{1},u)) & \\
  & \wedge \langle y_{0}?(z) \rangle (\langle u? \rangle \langle x_{1}!z \rangle C(x_{0},x_{1},y_{0},y_{1},u))) &
\end{align*}

The lexicographical similarity between the shape of this formulae and
the shape of definition of the process representing a crossing reveals
the intuitive meaning of this formulae. It describes the capabilities
of a process that has the right to represent a crossing. For example
it picks out processes that may perform an input on the port $x_0$ in
its initial menu of capabilities. What differentiates the formula
from the process, however, is that the crossing process is the
smallest candidate to satisfy the formula. Infinitely many other
processes -- with internal behavior hidden behind this interface, so
to speak -- also satisfy this formula. Even this simple formula,
then, can be seen to open a new view onto knots, providing a
computational interpretation of \emph{virtual} knots.

Note that this formula is derived by hand. A similar formula can be
derived by employing Caires' calculation of characteristic formula
\cite{Caires04} to the process representing a crossing. In light of
this discussion, we let
$\meaningof{C}_{\phi}(x0,x1,y0,y1,u)$ denote a formula specifying the
dynamics we wish to capture of a crossing. To guarantee we preserve
the shape of the interface and minimal semantics we demand that
$\meaningof{C}_{\phi}(x0,x1,y0,y1,u) \Rightarrow
\textbf{C}(x0,x1,y0,y1,u)$ where $\textbf{C}(x0,x1,y0,y1,u)$ denotes
the formula above.
                            
\subsubsection{Crossing number constraints.}
The moral content of the context lemma (Lemma \ref{context}) is that the notion of
``locality'' in the Reidemeister moves is effectively captured by the
parallel composition operator of the process calculus. This intuition
extends through the logic. Given a formula,
$\meaningof{C}_{\phi}(x0,x1,y0,y1,u)$, we can use the structural
connectives to specify constraints on crossing numbers, such as at
least $n$ crossings, or exactly $n$ crossings.
\begin{mathpar}
  \inferrule* [lab=at-least-n] {} { K^{\geq n}_{\phi}(\vec{xs},\vec{ys}) := \Pi_{i=0}^{n-1} Hu . \meaningof{C}_{\phi}(xs_i,ys_i,u) | T }
  \and 
  \inferrule* [lab=exactly-n] {} { K^{= n}_{\phi}(\vec{xs},\vec{ys}) := \Pi_{i=0}^{n-1} Hu . \meaningof{C}_{\phi}(xs_i,ys_i,u) | \neg (\forall x_0,y_0,x_1,y_1,u . \meaningof{C}_{\phi}(x_0,y_0,x_1,y_1,u) | T) }
\end{mathpar}

To round out this section, recall that the encoding of an $n$-crossing
knot decomposes into a parallel composition of $n$ \emph{copies} of a
crossing process together with a wiring harness. To specify different
knot classes with the same crossing number amounts to specifying
logical constraints on the wiring harness. In the interest of space,
we defer examples to a forthcoming paper. Suffice it to say that both
the conditions ``alternating knot'' and ``contains the tangle
corresponding to 5/3'' are expressible. For example, it is possible to
calculate the characteristic formula of a process corresponding to the
tangle 5/3 and conjoin it into the classifying formula via the
composition connective of the logic.

Finally, we wish to observe that it is entirely within reason to
contemplate a more domain-specific version of spatial logic tailored
to the shape of processes in the image of the encoding. Such a
domain-specific logic would have a better claim to the title formal
language of knot properties.

% subsection example_formulae_ (end)

% section knots_as_processes (end) 

% section spatial logic via knots (end)

\section{Conclusions and future work}

\paragraph{Testing physical space}
You, gentle reader, may wonder why of all the theorems to be proved
given this set up we pick the one above. In some sense it's hardly
central to quantum mechanics. We see it as central in the sense that
it firmly establishes a notion of physical space arising from a notion
of the equivalence of behavior. Relating bisimulation to a metric is a
big step forward, but one is faced with interpreting the relationship
of that metric space to something more physical. Quantum mechanical
notions of ``physical'' space are still far from intuitive, but by
relating this idea of distance as testing to calculations that predict
physical circumstances we are making a not insignificant step forward
toward an understanding of the physical space we inhabit as
essentially dynamic.

\paragraph{Effectivity and simulation}
One of the observations we have yet to make is that the entire program
spelled out here is effective. We have built various interpreters for
the reflective calculus at work in this interpretation. In principle,
then, we can simulate quantum mechanics on a computer. The place where
the simulation may lose fidelity is the infinitely branching summation
for the annihilator.

In this connection i also want to point out that the evaluation style
calculation of the inner product puts the non-determinism of the
summation right at the heart of measurement. This suggests that
Milner's original reduction-based formulation of the dynamics of his
calculi in terms of sums was not just notationally suggestive of a
notion of measure-and-continue but captured some significant part of
the physics.

\paragraph{Quantum continuations}
In light of this last observation i want to point out that the
predominant account of quantum mechanics is missing a key aspect of a
truly compositional story of the physical situation. In a real lab,
when a measurement is made the observation can be made to feed into
another device that then makes another measurement conditioned on the
results of the first. This means that after the superposition was
collapsed the entire experimental set up remained in
superposition. While QM offers a means of writing this down it doesn't
quite line up well with the well-trodden formulation of computation
and continuation that we see so succinctly expressed in Milner's
calculi. This suggests that there might be advantages to this account
of dynamics waiting to be explored.

\paragraph{Quantum logic}
In this connection, we also note that by virtue of having the
Hennessy-Milner construction, we can pull the construction through the
interpretation of QM. This gives us a natural candidate for a quantum
logic that enjoys an extremely tight connection with it's domain of
interpretation, making the construction much less ad hoc (rather it is
the image of functor!).

\paragraph{Quantum probabiity}
i have questions about the basis of the interpretation of inner
product as probability amplitude. In particular, using which
axiomatization of probability theory does the notion of probability
amplitude earn the right to be so dubbed? In other words, where is the
proof that the operation for calculating a probability amplitude (and
then squaring) satisfies the axioms of what it means to calculate a
probability? Even if such a proof exists (i have yet to find it in the
literature), i wonder if it might not be possible to turn things on
their heads. Can we view the calculation of the probability amplitude
as an axiomatization of probability? If so, then the definition we
give for calculating probability amplitude may provide the basis for
an \emph{effective} theory of probability.

\paragraph{Quantum vs ``biological'' information}
Finally, i want to conclude with a more philosophical observation. At
a recent workshop in which QM was a predominant topic i noticed
something about quantum information. The speaker was giving a riveting
discussion of axiomatic QM and showing how properties of ``no
cloning'' and ``no deleting'' emerged as consequences of the
axiomatization. Theorems of this form are necessary to give us a sense
of confidence that our axioms characterize the physical theory. What
struck me, though, was that if quantum information is neither erasable
nor replicable it is markedly different from \emph{life}. Two of the
things we know about life is that

\begin{itemize}
  \item it ends;
  \item to gain some measure of persistence, to transcend it's
    finitude it is imminently copyable.
\end{itemize}

Both of these qualities are summarized succinctly in the aphorism: all
flesh is grass. For me these two kinds of ``information'' -- call them
quantum and biological -- are end points on a spectrum of strategies
for persistence. At one end, we have those curious entities that enjoy
uniqueness and permanence; at the other, we have those who in the face
of a certain end and an uncertain present make a go of passing
something on. To me one of the more remarkable aspects of the latter
strategy is that in the presence of noise (and certain features of
copying) we get a kind of dynamism, a chance for improvement against a
given persistent condition.

% subsection other_calculi_other_bisimulations_and_geometry_as_behavior (end)




% section conclusion (end)

%\documentclass[12pt]{llncs}
%\documentclass{jktr}

\usepackage[pdftex]{hyperref}                   
\usepackage {listings}
\usepackage {mathpartir}
\usepackage{bcprules}
%\usepackage{listings}
                       
\usepackage{graphicx} 
%\usepackage[margins=2.5cm,nohead,nofoot]{geometry}
%\usepackage{geometry}
\usepackage{amsfonts}
\usepackage{amstext}
\usepackage{latexsym}
\usepackage{amssymb}
\usepackage{color}


%\include{myPreamble}
\include{qm2pi.local} 

%\ifpdf
%\usepackage[pdftex]{graphicx}
%\else
%\usepackage{graphicx}
%\fi

 % \ifpdf
%  \usepackage{pdfsync}
%  \if


%\title{Brief Article}
%\author{David F. Snyder}
%\author{L.G. Meredith}

%\address{Dept. of Math., Texas State University--San Marcos, San Marcos, TX 78666}
       
\pagestyle{empty}


\begin{document}

\lstset{language=[Objective]Caml,frame=shadowbox}

\input{qm2pi.front}

% section front matter (end)

\input{qm2pi.intro} 
 
% section introduction (end)

% \input{qm2pi.knotations} 

% section notation (end)

\input{qm2pi.process.calculi} 

% section concurrent_process_calculi_and_spatial_logics_ (end)
    
%\input{qm2pi.knots2pi} 

%\input{qm2pi.trefoil} 

%\input{qm2pi.mainthm} 

% subsection basic_interpretation (end)

%\input{qm2pi.rho.presentation} 
\subsection{The syntax and semantics of the notation system}\label{sub:the_syntax_and_semantics_of_the_notation_system} % (fold)

We now summarize a technical presentation of the calculus that
embodies our theory of dynamics. The typical presentation of such a
calculus follows the style of giving generators and relations on
them. The grammar, below, describing term constructors, freely
generates the set of processes, $\Proc$. This set is then quotiented
by a relation known as structural congruence and it is over this set
that the notion of dynamics is expressed. This presentation is
essentially that of \cite{MeredithR05} with the addition of
polyadicity and summation. For readability we have relegated some of
the technical subtleties to an appendix.

\subsubsection{Process grammar}\label{subsub:process_grammar}

\begin{mathpar}
  \inferrule* [lab=synchronization] {} {{M} \bc \pzero \;|\; x?F \;|\; x!C }
  \and
  \inferrule* [lab=abstraction] {} {{F} \bc (x)P}
  \and
  \inferrule* [lab=concretion] {} {{C} \bc \langle Q \rangle}
  \and
  \inferrule* [lab=process] {} {{P,Q} \bc M \;| \;P|Q \;|\; @{x}}
  \and
  \inferrule* [lab=name] {} {{x} \bc \quotep{P}}
\end{mathpar} 

Note that $\vec{x}$ (resp. $\vec{P}$) denotes a vector of names
(resp. processes) of length $|\vec{x}|$ (resp. $|\vec{P}|$). We adopt
the following useful abbreviations.

\begin{mathpar}
   x?(\vec{y}).P := x.(\vec{y})P \and  x\clift{\vec{P}} := x.\clift{\vec{P}}
   \and x!(y) := \lift{x}{\dropn{y}}
   \and \Pi_{i=0}^{n-1}P_i := P_0 | \ldots | P_{n-1}
\end{mathpar}

\subsubsection{Structural congruence}

\paragraph{Free and bound names and alpha-equivalence.} At the
core of structural equivalence is alpha-equivalence which identifies
process that are the same up to a change of variable. Formally, we
recognize the distinction between free and bound names. The free names
of a process, $\freenames{P}$, may be calculated recursively as
follows:

\begin{mathpar}
\freenames{\pzero} := \emptyset
  \and \\
  \freenames{x?(y).P} := \{ x \} \cup (\freenames{P} \setminus \{ y \})
  \and 
  \freenames{x!\langle P \rangle} := \{ x \} \cup \{ P \} 
  \and \\
  \freenames{P|Q} := \freenames{P} \cup \freenames{Q}
  \and \\
  \freenames{@{x}} := \{ x \}
\end{mathpar}

$\pi$
$\quotep{\pi}$

$\freenames{-} : \pi \to \mathcal{P}(\quotep{\pi})$

\begin{eqnarray*}
  \freenames{\pzero} & := & \emptyset \\
  \freenames{x?(y).P} & := & \{ x \} \cup (\freenames{P} \setminus \{ y \}) \\
  \freenames{x!\langle P \rangle} & := & \{ x \} \cup \{ P \} \\
  \freenames{P|Q} & := & \freenames{P} \cup \freenames{Q} \\
  \freenames{\dropn{x}} & := & \{ x \}
\end{eqnarray*}

The bound names of a process, $\boundnames{P}$, are those names occurring in $P$
that are not free. For example, in $x?(y).0$, the name $x$ is free, while $y$ is bound.

\begin{mathpar}
  \inferrule* [lab=monoidal-laws] {} { P|Q \equiv Q|P \and P|0 \equiv P \and P|(Q|R) \equiv (P|Q)|R }
\end{mathpar}

\begin{mathpar}
  \inferrule* [lab=alpha-equivalence] {} { (x)P \equiv (y)P\{y/x\} \and y \not\in \freenames{P} }
\end{mathpar}

\begin{definition}
Then two processes, $P,Q$, are alpha-equivalent if $P = Q\{\vec{y}/\vec{x}\}$ for
some $\vec{x} \in \boundnames{Q},\vec{y} \in \boundnames{P}$, where $Q\{\vec{y}/\vec{x}\}$
denotes the capture-avoiding substitution of $\vec{y}$ for $\vec{x}$ in $Q$.
\end{definition}

\begin{definition}
  The {\em structural congruence} \cite{SangiorgiWalker} , $\equiv$,
  between processes is the least congruence containing
  alpha-equivalence, satisfying the abelian monoid laws
  (associativity, commutativity and $\pzero$ as identity) for parallel
  composition $|$ and for summation $+$.
\end{definition}

\subsection{Name equivalence}

We take name equivalence, written $\nameeq$, to be the smallest
equivalence relation generated by the following rules.

\begin{mathpar}
\inferrule*[lab=Quote-drop]
{ }
{ \quotep{@{x}} \nameeq x }

\inferrule*[lab=Struct-equiv]
{ P \scong Q }
{ \quotep{P} \nameeq \quotep{Q} }
\end{mathpar}

The astute reader will have noticed that the mutual recursion of names
and processes imposes a mutual recursion on alpha-equivalence and
structural equivalence via name-equivalence. Fortunately, all of this
works out pleasantly and we may calculate in the natural way, free of
concern. The reader interested in the details is referred to the
appendix \ref{appendix:rho_details}.

\subsection{Substitution}

We use $\Proc$ for the set of processes, $\QProc$ for the set of
names, and $\id{\{}\vec{y} / \vec{x} \id{\}}$ to denote partial maps,
$s : \QProc \rightarrow \QProc$. A map, $s$ lifts, uniquely, to a map
on process terms, $\widehat{s} : \Proc \rightarrow \Proc$ by the
following equations.

\begin{mathpar}
  (0) \psubstp{Q}{P} := 0 \\
  (R \juxtap S) \psubstp{Q}{P}
  :=    
  (R)\psubstp{Q}{P} \juxtap (S) \psubstp{Q}{P} \\
  (x?(y).R) \psubstp{Q}{P}    
  :=    
  (x)\substp{Q}{P} (z)\concat( (R \psubstn{z}{y}) \psubstp{Q}{P} ) \\
  (\lift{x}{R}) \psubstp{Q}{P}  
  :=
  \lift{(x)\substp{Q}{P}}{ R \psubstp{Q}{P} } \\
%   (\dropn{x})  \psubstp{Q}{P}       
%   := 
%   \left\{ 
%     \begin{array}{ccc} 
%       \dropn{\quotep{Q}} & & x \nameeq \quotep{P} \\
%       \dropn{x} & & otherwise \\
%     \end{array}
%   \right. 
  (\dropn{x})  \psubstp{Q}{P}       
  := 
  \left\{ 
    \begin{array}{ccc} 
      Q & & x \nameeq \quotep{P} \\
      \dropn{x} & & otherwise \\
    \end{array}
  \right.
\end{mathpar}
 

where

\begin{eqnarray}
  (x)\id{\{} \lpquote Q \rpquote / \lpquote P \rpquote \id{\}}            = 
  \left\{ 
    \begin{array}{ccc}
      \lpquote Q \rpquote & & x \nameeq \lpquote P \rpquote \\
      x & & otherwise \\
    \end{array}
  \right. \nonumber
\end{eqnarray}

and $z$ is chosen distinct from $\quotep{P}$, $\quotep{Q}$, the free
names in $Q$, and all the names in $R$. Our $\alpha$-equivalence will
be built in the standard way from this substitution.

\begin{remark}\label{rem:no_self_referential_names}
  One consequence of these definitions is that $\forall P. \quotep{P}
  \not\in \freenames{P}$.
\end{remark}

\subsection{ Dynamic quote: an example }

Anticipating something of what's to come, consider applying the
substitution, $\widehat{\id{\{}u / z \id{\}}}$, to the following pair
of processes, $\lift{w}{y!(z)}$ and $w[ \lpquote y!(z) \rpquote ]$.

\begin{eqnarray}
	\lift{w}{y!(z)}\widehat{\id{\{}u / z \id{\}}}
		& = &
		\lift{w}{y!(u)} \nonumber\\
	w[ \lpquote y!(z) \rpquote ] \widehat{ \id{\{}u / z \id{\}} }
		& = &
		w[ \lpquote y!(z) \rpquote ] \nonumber
\end{eqnarray}

Because the body of the process between quotes is impervious to
substitution, we get radically different answers. In fact, by
examining the first process in an input context,
e.g. $x?(z).\lift{w}{y!(z)}$, we see that the process under the lift
operator may be shaped by prefixed inputs binding a name inside it. In
this sense, the lift operator will be seen as a way to dynamically
construct processes before reifying them as names.

Finally equipped with these standard features we can present the
dynamics of the calculus.

\subsubsection{Operational semantics} 

Finally, we introduce the computational dynamics. What marks these
algebras as distinct from other more traditionally studied algebraic
structures, e.g. vector spaces or polynomial rings, is the manner in
which dynamics is captured. In traditional structures, dynamics is typically
expressed through morphisms between such structures, as in linear maps
between vector spaces or morphisms between rings. In algebras
associated with the semantics of computation, the dynamics is
expressed as part of the algebraic structure itself, through a
reduction reduction relation typically denoted by $\red$. Below, we
give a recursive presentation of this relation for the calculus used
in the encoding.

$\red \subseteq \pi \times \pi$
$\red : \pi \to \mathcal{P}(\pi)$

\begin{mathpar}
  \inferrule* [lab=Comm] { \textsf{match}( x_{src}, x_{trgt} ) } { x_{trgt}?(y)P \; | \; x_{src}!\langle {Q} \rangle \red P\{\quotep{Q}/y}\} }
  \and \\
  \inferrule* [lab=Par] {{P} \red {P}'} {{{P} | {Q}} \red {{P}' | {Q}}}
  \and
  \inferrule* [lab=Equiv]{{{P} \scong {P}'} \andalso {{P}' \red {Q}'} \andalso {{Q}' \scong {Q}}}{{P} \red {Q}}
\end{mathpar}

\begin{eqnarray*}
  match_{\equiv} (\quotep{P},\quotep{Q}) & := & P \equiv Q \\
  match_{\dagger}(\quotep{P},\quotep{Q}) & := & \forall R. P|Q \red^{*} R => R \red^{*} 0 \\
  match_{K}(\quotep{P},\quotep{Q}) & := & K \mbox{ for some context } K
\end{eqnarray*}

$u?(x)P | u!\langle Q \rangle \red P\{\quotep{Q}/x\}$

%We write $\wred$ for $\red^*$, and $P\red$ if $\exists Q $ such that $ P \red Q$.
We write $P\red$ if $\exists Q $ such that $ P \red Q$ and $P\not\red$, otherwise.

\section{Replication}

As mentioned before, it is known that replication (and hence
recursion) can be implemented in a higher-order process algebra
\cite{SangiorgiWalker}. As our first example of calculation with the
machinery thus far presented we give the construction explicitly in
the {\rhoc}.

\begin{eqnarray}
	D_{x} & := & \prefix{x}{y}{(\binpar{\outputp{x}{y}}{@{y}})} \nonumber\\
	\bangp_{x}{P} & := & \binpar{{x}!\langle{\binpar{D_{x}}{P}}\rangle}{D_{x}} \nonumber
\end{eqnarray}

\begin{eqnarray}
	\bangp_{x}{P} & & \nonumber\\
	=
	& {x}!\langle{(\prefix{x}{y}{(\outputp{x}{y} | @{y})) | P}}\rangle 
	      | \prefix{x}{y}{(\outputp{x}{y} | @{y})} & \nonumber\\
	\red
	& (\outputp{x}{y} | @{y})\substn{\quotep{(\prefix{x}{y}{(@{y} | \outputp{x}{y})) | P}}}{y} & \nonumber\\
	=
	& \outputp{x}{\quotep{(\prefix{x}{y}{(\outputp{x}{y} | @{y})) | P}}}
	  | {(\prefix{x}{y}{(\outputp{x}{y} | @{y})) | P}} & \nonumber\\
	\red
	& \ldots & \nonumber\\
	\red^*
	& P | P | \ldots & \nonumber
\end{eqnarray}

Of course, this encoding, as an implementation, runs away, unfolding
$\bangp{P}$ eagerly. A lazier and more implementable replication
operator, restricted to input-guarded processes, may be obtained as follows.

\begin{eqnarray}
\bangp{\prefix{u}{v}{P}} 
	:= 
	\binpar{\lift{x}{\prefix{u}{v}{(\binpar{D(x)}{P})}}}{D(x)} \nonumber
\end{eqnarray}

\begin{remark}
  Note that the lazier definition still does not deal with summation
  or mixed summation (i.e. sums over input and output). The reader is
  invited to construct definitions of replication that deal with these
  features. 

  Further, the definitions are parameterized in a name, $x$. Can you,
  gentle reader, make a definition that eliminates this parameter and
  guarantees no accidental interaction between the replication
  machinery and the process being replicated -- i.e. no accidental
  sharing of names used by the process to get its work done and the
  name(s) used by the replication to effect copying. This latter
  revision of the definition of replication is crucial to obtaining
  the expected identity $!!P \sim !P$.
\end{remark}

\begin{remark}\label{rem:paradoxical_combinator}
  The reader familiar with the lambda calculus will have noticed the
  similarity between $D$ and the paradoxical combinator.

  [Ed. note: the existence of this seems to suggest we have to be more
  restrictive on the set of processes and names we admit if we are to
  support no-cloning.]
\end{remark}

\subsubsection{Bisimulation}

The computational dynamics gives rise to another kind of equivalence,
the equivalence of computational behavior. As previously mentioned
this is typically captured \emph{via} some form of bisimulation.

% The notion we use in this paper is weak barbed bisimulation
% \cite{milner91polyadicpi}.

The notion we use in this paper is derived from weak barbed
bisimulation \cite{milner91polyadicpi}. 

\begin{definition}
An \emph{observation relation}, $\downarrow_{\mathcal N}$, over a set
of names, $\mathcal N$, is the smallest relation satisfying the rules
below.

\infrule[Out-barb]{y \in {\mathcal N}, \; x \nameeq y}
		  {\outputp{x}{v} \downarrow_{\mathcal N} x}
\infrule[Par-barb]{\mbox{$P\downarrow_{\mathcal N} x$ or $Q\downarrow_{\mathcal N} x$}}
		  {\binpar{P}{Q} \downarrow_{\mathcal N} x}

We write $P \Downarrow_{\mathcal N} x$ if there is $Q$ such that 
$P \wred Q$ and $Q \downarrow_{\mathcal N} x$.
\end{definition}

\begin{definition}
%\label{def.bbisim}
An  ${\mathcal N}$-\emph{barbed bisimulation} over a set of names, ${\mathcal N}$, is a symmetric binary relation 
${\mathcal S}_{\mathcal N}$ between agents such that $P\rel{S}_{\mathcal N}Q$ implies:
\begin{enumerate}
\item If $P \red P'$ then $Q \wred Q'$ and $P'\rel{S}_{\mathcal N} Q'$.
\item If $P\downarrow_{\mathcal N} x$, then $Q\Downarrow_{\mathcal N} x$.
\end{enumerate}
$P$ is ${\mathcal N}$-barbed bisimilar to $Q$, written
$P \wbbisim_{\mathcal N} Q$, if $P \rel{S}_{\mathcal N} Q$ for some ${\mathcal N}$-barbed bisimulation ${\mathcal S}_{\mathcal N}$.
\end{definition}

$\mathcal{R} \subseteq \pi \times \pi$

$P \mathcal{R} Q => \forall P'. P \red P' \Rightarrow \exists Q'. Q \red Q', P' \mathcal{R} Q'$

$P \vdash x \Rightarrow Q \vdash x$

\begin{mathpar}
  \inferrule*[lab=Out-barb]{x \nameeq y}{{y}!\langle{Q}\rangle \vdash x}
  \and
  \inferrule*[lab=Par-barb]{\mbox{$P\vdash x$ or $Q\vdash x$}}{\binpar{P}{Q} \vdash x}
\end{mathpar}

\subsubsection{Contexts}

One of the principle advantages of computational calculi like the
$\pi$-calculus is a well-defined notion of context,
contextual-equivalence and a correlation between
contextual-equivalence and notions of bisimulation. The notion of
context allows the decomposition of a process into (sub-)process and
its syntactic environment, its context. Thus, a context may be
thought of as a process with a ``hole'' (written $\Box$) in it. The
application of a context $M$ to a process $P$, written $M[P]$, is
tantamount to filling the hole in $M$ with $P$. In this paper we do
not need the full weight of this theory, but do make use of the notion
of context in the proof the main theorem. 

\begin{mathpar}
  \inferrule* [lab=summation] {} {{M_{M},M_{N}} \bc \Box \;|\; x.M_{A} \;|\; M_{M}+M_{N}}
  \and
  \inferrule* [lab=agent] {} {{M_{A}} \bc (\vec{x})M_{P} \;| \; \clift{P_0,\ldots,M_{P},\ldots,P_N}}
  \and \\
  \inferrule* [lab=process] {} {{M_{P}} \bc M_{N} \;| \;P|M_{P} }
\end{mathpar} 

\begin{mathpar}
  \inferrule* [lab=sychronization] {} {M_{N} \bc \Box \;|\; x?M_{F} \;|\; x!M_{C}}
  \and
  \inferrule* [lab=abstraction] {} {{M_{F}} \bc (x)M_{P} }
  \and
  \inferrule* [lab=concretion] {} {{M_{C}} \bc \langle M_{P} \rangle }
  \and \\
  \inferrule* [lab=process] {} {{M_{P}} \bc M_{N} \;| \;P|M_{P} }
\end{mathpar}

\begin{definition}[contextual application] Given a context $M$, and
  process $P$, we define the \emph{contextual application}, $M[P] :=
  M\{P/\Box\}$. That is, the contextual application of M to P is the
  substitution of $P$ for $\Box$ in $M$.
\end{definition}

$\meaningof{-} : L \to \mathcal{P}(\pi)$

\begin{mathpar}
  \inferrule* [lab=collection] {} {\meaningof{true} = \pi, \and \meaningof{~E} = \pi \setminus \meaningof{E}, \and \meaningof{E_{1} \& E_{2}} = \meaningof{E_{1}} \cap \meaningof{E_{2}}}
\end{mathpar}

\begin{mathpar}
  \inferrule* [lab=structure] {} {\meaningof{0} = \{ P \in \pi | P \equiv 0 \}, \and \\ \meaningof{E_1 | E_2} = \{ P \in \pi | P \equiv P_{1} | P_{2}, P_{1} \in \meaningof{E_{1}}, P_{2} \in \meaningof{E_2}\} }
\end{mathpar}

\begin{mathpar}
 \inferrule* [lab=behavior] {} {\meaningof{\langle a?b \rangle E} = \{ P \in \pi | P \equiv Q | u?(y)P', \\ \and \\\\ \and \\ \;\;\; u \in \meaningof{a}, \forall z.P'\{z/y\} \in \meaningof{E\{z/b\}}\}, \and \\ \meaningof{a!E} = \{ P \in \pi | P \equiv Q | x!\langle P' \rangle, x \in \meaningof{a} P' \in \meaningof{E}\} }
\end{mathpar}

\begin{mathpar}
 \inferrule* [lab=nominal] {} {\meaningof{\quotep{E}} = \{ \quotep{P} \in \quotep{\pi} | P \in \meaningof{E} \}, \and \meaningof{\quotep{P}} = \{ \quotep{Q} \in \quotep{\pi} | P \equiv Q \} \and \\ \meaningof{@\quotep{E}} = \{ P \in \pi | P \equiv @x, x \in \meaningof{E} \}}
\end{mathpar}

\begin{eqnarray*}
  \\
  \meaningof{-} : TS \to ST
\end{eqnarray*}

\begin{eqnarray*}
  \\
  L : TS \to ST
\end{eqnarray*}

\begin{eqnarray*}
  \\
  P \models E \iff P \in \meaningof{E}
\end{eqnarray*}

\begin{eqnarray*}
  P \approx_{L} Q \iff \forall E \in L. P \models E \iff Q \models E
\end{eqnarray*}

\begin{eqnarray*}
  P \approx_{K} Q
\end{eqnarray*}

\begin{eqnarray*}
  P \approx Q
\end{eqnarray*}

$\approx_{K} = \approx = \approx_{L}$

\subsubsection{Contextual duality}

Note that contexts extend the quotation operation to a family of
operations from processes to names. Given a context, $M$, we can
define a \emph{nominal context}, $\quotep{M}$ by $\quotep{M}[P] :=
\quotep{M[P]}$. To foreshadow what is to come we observe that these
operations enjoy a duality with processes very much like the duality
between vectors and maps from vectors to scalars.

Further, because the calculus is essentially higher-order, we have a
correspondence between contexts and processes. More specifically,
given a name $x$ and a context $M$ we can construct $M^{*}_{x}$ such
that 

\begin{mathpar}
  M^{*}_{x} | \lift{x}{P} \red M[P]
\end{mathpar}

namely,

\begin{mathpar}
  M^{*}_{x} := x?(u).M[\dropn{u}]
\end{mathpar}

The dependence of $M^{*}_{x}$ on a name makes it an abstraction, 

\begin{mathpar}
  M^{*} := (x)x?(u).M[\dropn{u}]
\end{mathpar}

\subsection{Additional notation}

It will sometimes be convenient to denote the process a name
quotes. We already have the notation $x = \quotep{P}$, but it will be
convenient to introduce an alternate notation, $\procn{x}$, when we
want to emphasize the connection to the use of the name. Note that, by
virtue of name equivalence, $\quotep{\procn{x}} \nameeq x$; so, the
notation is consistent with previous definitions.

Further, because names have structure it is possible to effect
substitutions on the basis of that structure. This means we need to
upgrade our notation for substitutions, which we accomplish by
adapting comprehension notation. Thus,

\begin{mathpar}
  P\{ y / x : x \in S \}
\end{mathpar}

is interpreted to mean the process derived from P by replacing (in a
capture-avoiding manner) each occurrence of $x$ in $S$ by $y$. For example,

\begin{mathpar}
  P\{ \quotep{\procn{x}|\procn{x}} / x : x \in \freenames{P} \}
\end{mathpar}

will replace each (occurrence) of a free name $x$ in $P$ by
$\quotep{\procn{x}|\procn{x}}$.

Also, we will avail ourselves of the notation $x^{L}$ and $x^{R}$ to
denote injections of a name into disjoint copies of the name
space. There are numerous ways to accomplish this. One example can be
found in \cite{MeredithR05}. This notation overloads to vectors of
names: $\vec{x}^{\pi} := (x_{i}^{\pi} \; : \; 0 \leq i < |\vec{x}| )$ where $\pi \in \{L,R\}$.

We also use $P^{\Box} := P|\Box$.

In \cite{MeredithR05} an interpretation of the new operator is
given. It turns out that there are several possible interpretations
all enjoying the requisite algebraic properties of the operator (see
\cite{milner91polyadicpi}). We will therefore make liberal use of
$(\nu\; \vec{x})P$.

% subsection the_syntax_and_semantics_of_the_notation_system (end)   

\input{qm2pi.qmops} 

\input{qm2pi.sterngerlach} 

\input{qm2pi.metric} 

% section concurrent_process_calculi (end)

%\input{qm2pi.proofsketch}

% section proof sketch (end)

%\input{qm2pi.slviaknots} 

% section spatial logic via knots (end)

\input{qm2pi.conclusion}

% section conclusion (end)

%\input{qm2pi.dtcodes} 

% section wiring algorithm (end)

\input{qm2pi.ack} 

% section acknowledgments (end)

\newpage


\bibliographystyle{plain}   
\bibliography{../../biblios/main.bib}

\input{qm2pi.rhodetails}

\end{document}

 

% section wiring algorithm (end)

\documentclass[12pt]{llncs}
%\documentclass{jktr}

\usepackage[pdftex]{hyperref}                   
\usepackage {listings}
\usepackage {mathpartir}
\usepackage{bcprules}
%\usepackage{listings}
                       
\usepackage{graphicx} 
%\usepackage[margins=2.5cm,nohead,nofoot]{geometry}
%\usepackage{geometry}
\usepackage{amsfonts}
\usepackage{amstext}
\usepackage{latexsym}
\usepackage{amssymb}
\usepackage{color}


%\include{myPreamble}
\include{qm2pi.local} 

%\ifpdf
%\usepackage[pdftex]{graphicx}
%\else
%\usepackage{graphicx}
%\fi

 % \ifpdf
%  \usepackage{pdfsync}
%  \if


%\title{Brief Article}
%\author{David F. Snyder}
%\author{L.G. Meredith}

%\address{Dept. of Math., Texas State University--San Marcos, San Marcos, TX 78666}
       
\pagestyle{empty}


\begin{document}

\lstset{language=[Objective]Caml,frame=shadowbox}

\input{qm2pi.front}

% section front matter (end)

\input{qm2pi.intro} 
 
% section introduction (end)

% \input{qm2pi.knotations} 

% section notation (end)

\input{qm2pi.process.calculi} 

% section concurrent_process_calculi_and_spatial_logics_ (end)
    
%\input{qm2pi.knots2pi} 

%\input{qm2pi.trefoil} 

%\input{qm2pi.mainthm} 

% subsection basic_interpretation (end)

%\input{qm2pi.rho.presentation} 
\subsection{The syntax and semantics of the notation system}\label{sub:the_syntax_and_semantics_of_the_notation_system} % (fold)

We now summarize a technical presentation of the calculus that
embodies our theory of dynamics. The typical presentation of such a
calculus follows the style of giving generators and relations on
them. The grammar, below, describing term constructors, freely
generates the set of processes, $\Proc$. This set is then quotiented
by a relation known as structural congruence and it is over this set
that the notion of dynamics is expressed. This presentation is
essentially that of \cite{MeredithR05} with the addition of
polyadicity and summation. For readability we have relegated some of
the technical subtleties to an appendix.

\subsubsection{Process grammar}\label{subsub:process_grammar}

\begin{mathpar}
  \inferrule* [lab=synchronization] {} {{M} \bc \pzero \;|\; x?F \;|\; x!C }
  \and
  \inferrule* [lab=abstraction] {} {{F} \bc (x)P}
  \and
  \inferrule* [lab=concretion] {} {{C} \bc \langle Q \rangle}
  \and
  \inferrule* [lab=process] {} {{P,Q} \bc M \;| \;P|Q \;|\; @{x}}
  \and
  \inferrule* [lab=name] {} {{x} \bc \quotep{P}}
\end{mathpar} 

Note that $\vec{x}$ (resp. $\vec{P}$) denotes a vector of names
(resp. processes) of length $|\vec{x}|$ (resp. $|\vec{P}|$). We adopt
the following useful abbreviations.

\begin{mathpar}
   x?(\vec{y}).P := x.(\vec{y})P \and  x\clift{\vec{P}} := x.\clift{\vec{P}}
   \and x!(y) := \lift{x}{\dropn{y}}
   \and \Pi_{i=0}^{n-1}P_i := P_0 | \ldots | P_{n-1}
\end{mathpar}

\subsubsection{Structural congruence}

\paragraph{Free and bound names and alpha-equivalence.} At the
core of structural equivalence is alpha-equivalence which identifies
process that are the same up to a change of variable. Formally, we
recognize the distinction between free and bound names. The free names
of a process, $\freenames{P}$, may be calculated recursively as
follows:

\begin{mathpar}
\freenames{\pzero} := \emptyset
  \and \\
  \freenames{x?(y).P} := \{ x \} \cup (\freenames{P} \setminus \{ y \})
  \and 
  \freenames{x!\langle P \rangle} := \{ x \} \cup \{ P \} 
  \and \\
  \freenames{P|Q} := \freenames{P} \cup \freenames{Q}
  \and \\
  \freenames{@{x}} := \{ x \}
\end{mathpar}

$\pi$
$\quotep{\pi}$

$\freenames{-} : \pi \to \mathcal{P}(\quotep{\pi})$

\begin{eqnarray*}
  \freenames{\pzero} & := & \emptyset \\
  \freenames{x?(y).P} & := & \{ x \} \cup (\freenames{P} \setminus \{ y \}) \\
  \freenames{x!\langle P \rangle} & := & \{ x \} \cup \{ P \} \\
  \freenames{P|Q} & := & \freenames{P} \cup \freenames{Q} \\
  \freenames{\dropn{x}} & := & \{ x \}
\end{eqnarray*}

The bound names of a process, $\boundnames{P}$, are those names occurring in $P$
that are not free. For example, in $x?(y).0$, the name $x$ is free, while $y$ is bound.

\begin{mathpar}
  \inferrule* [lab=monoidal-laws] {} { P|Q \equiv Q|P \and P|0 \equiv P \and P|(Q|R) \equiv (P|Q)|R }
\end{mathpar}

\begin{mathpar}
  \inferrule* [lab=alpha-equivalence] {} { (x)P \equiv (y)P\{y/x\} \and y \not\in \freenames{P} }
\end{mathpar}

\begin{definition}
Then two processes, $P,Q$, are alpha-equivalent if $P = Q\{\vec{y}/\vec{x}\}$ for
some $\vec{x} \in \boundnames{Q},\vec{y} \in \boundnames{P}$, where $Q\{\vec{y}/\vec{x}\}$
denotes the capture-avoiding substitution of $\vec{y}$ for $\vec{x}$ in $Q$.
\end{definition}

\begin{definition}
  The {\em structural congruence} \cite{SangiorgiWalker} , $\equiv$,
  between processes is the least congruence containing
  alpha-equivalence, satisfying the abelian monoid laws
  (associativity, commutativity and $\pzero$ as identity) for parallel
  composition $|$ and for summation $+$.
\end{definition}

\subsection{Name equivalence}

We take name equivalence, written $\nameeq$, to be the smallest
equivalence relation generated by the following rules.

\begin{mathpar}
\inferrule*[lab=Quote-drop]
{ }
{ \quotep{@{x}} \nameeq x }

\inferrule*[lab=Struct-equiv]
{ P \scong Q }
{ \quotep{P} \nameeq \quotep{Q} }
\end{mathpar}

The astute reader will have noticed that the mutual recursion of names
and processes imposes a mutual recursion on alpha-equivalence and
structural equivalence via name-equivalence. Fortunately, all of this
works out pleasantly and we may calculate in the natural way, free of
concern. The reader interested in the details is referred to the
appendix \ref{appendix:rho_details}.

\subsection{Substitution}

We use $\Proc$ for the set of processes, $\QProc$ for the set of
names, and $\id{\{}\vec{y} / \vec{x} \id{\}}$ to denote partial maps,
$s : \QProc \rightarrow \QProc$. A map, $s$ lifts, uniquely, to a map
on process terms, $\widehat{s} : \Proc \rightarrow \Proc$ by the
following equations.

\begin{mathpar}
  (0) \psubstp{Q}{P} := 0 \\
  (R \juxtap S) \psubstp{Q}{P}
  :=    
  (R)\psubstp{Q}{P} \juxtap (S) \psubstp{Q}{P} \\
  (x?(y).R) \psubstp{Q}{P}    
  :=    
  (x)\substp{Q}{P} (z)\concat( (R \psubstn{z}{y}) \psubstp{Q}{P} ) \\
  (\lift{x}{R}) \psubstp{Q}{P}  
  :=
  \lift{(x)\substp{Q}{P}}{ R \psubstp{Q}{P} } \\
%   (\dropn{x})  \psubstp{Q}{P}       
%   := 
%   \left\{ 
%     \begin{array}{ccc} 
%       \dropn{\quotep{Q}} & & x \nameeq \quotep{P} \\
%       \dropn{x} & & otherwise \\
%     \end{array}
%   \right. 
  (\dropn{x})  \psubstp{Q}{P}       
  := 
  \left\{ 
    \begin{array}{ccc} 
      Q & & x \nameeq \quotep{P} \\
      \dropn{x} & & otherwise \\
    \end{array}
  \right.
\end{mathpar}
 

where

\begin{eqnarray}
  (x)\id{\{} \lpquote Q \rpquote / \lpquote P \rpquote \id{\}}            = 
  \left\{ 
    \begin{array}{ccc}
      \lpquote Q \rpquote & & x \nameeq \lpquote P \rpquote \\
      x & & otherwise \\
    \end{array}
  \right. \nonumber
\end{eqnarray}

and $z$ is chosen distinct from $\quotep{P}$, $\quotep{Q}$, the free
names in $Q$, and all the names in $R$. Our $\alpha$-equivalence will
be built in the standard way from this substitution.

\begin{remark}\label{rem:no_self_referential_names}
  One consequence of these definitions is that $\forall P. \quotep{P}
  \not\in \freenames{P}$.
\end{remark}

\subsection{ Dynamic quote: an example }

Anticipating something of what's to come, consider applying the
substitution, $\widehat{\id{\{}u / z \id{\}}}$, to the following pair
of processes, $\lift{w}{y!(z)}$ and $w[ \lpquote y!(z) \rpquote ]$.

\begin{eqnarray}
	\lift{w}{y!(z)}\widehat{\id{\{}u / z \id{\}}}
		& = &
		\lift{w}{y!(u)} \nonumber\\
	w[ \lpquote y!(z) \rpquote ] \widehat{ \id{\{}u / z \id{\}} }
		& = &
		w[ \lpquote y!(z) \rpquote ] \nonumber
\end{eqnarray}

Because the body of the process between quotes is impervious to
substitution, we get radically different answers. In fact, by
examining the first process in an input context,
e.g. $x?(z).\lift{w}{y!(z)}$, we see that the process under the lift
operator may be shaped by prefixed inputs binding a name inside it. In
this sense, the lift operator will be seen as a way to dynamically
construct processes before reifying them as names.

Finally equipped with these standard features we can present the
dynamics of the calculus.

\subsubsection{Operational semantics} 

Finally, we introduce the computational dynamics. What marks these
algebras as distinct from other more traditionally studied algebraic
structures, e.g. vector spaces or polynomial rings, is the manner in
which dynamics is captured. In traditional structures, dynamics is typically
expressed through morphisms between such structures, as in linear maps
between vector spaces or morphisms between rings. In algebras
associated with the semantics of computation, the dynamics is
expressed as part of the algebraic structure itself, through a
reduction reduction relation typically denoted by $\red$. Below, we
give a recursive presentation of this relation for the calculus used
in the encoding.

$\red \subseteq \pi \times \pi$
$\red : \pi \to \mathcal{P}(\pi)$

\begin{mathpar}
  \inferrule* [lab=Comm] { \textsf{match}( x_{src}, x_{trgt} ) } { x_{trgt}?(y)P \; | \; x_{src}!\langle {Q} \rangle \red P\{\quotep{Q}/y}\} }
  \and \\
  \inferrule* [lab=Par] {{P} \red {P}'} {{{P} | {Q}} \red {{P}' | {Q}}}
  \and
  \inferrule* [lab=Equiv]{{{P} \scong {P}'} \andalso {{P}' \red {Q}'} \andalso {{Q}' \scong {Q}}}{{P} \red {Q}}
\end{mathpar}

\begin{eqnarray*}
  match_{\equiv} (\quotep{P},\quotep{Q}) & := & P \equiv Q \\
  match_{\dagger}(\quotep{P},\quotep{Q}) & := & \forall R. P|Q \red^{*} R => R \red^{*} 0 \\
  match_{K}(\quotep{P},\quotep{Q}) & := & K \mbox{ for some context } K
\end{eqnarray*}

$u?(x)P | u!\langle Q \rangle \red P\{\quotep{Q}/x\}$

%We write $\wred$ for $\red^*$, and $P\red$ if $\exists Q $ such that $ P \red Q$.
We write $P\red$ if $\exists Q $ such that $ P \red Q$ and $P\not\red$, otherwise.

\section{Replication}

As mentioned before, it is known that replication (and hence
recursion) can be implemented in a higher-order process algebra
\cite{SangiorgiWalker}. As our first example of calculation with the
machinery thus far presented we give the construction explicitly in
the {\rhoc}.

\begin{eqnarray}
	D_{x} & := & \prefix{x}{y}{(\binpar{\outputp{x}{y}}{@{y}})} \nonumber\\
	\bangp_{x}{P} & := & \binpar{{x}!\langle{\binpar{D_{x}}{P}}\rangle}{D_{x}} \nonumber
\end{eqnarray}

\begin{eqnarray}
	\bangp_{x}{P} & & \nonumber\\
	=
	& {x}!\langle{(\prefix{x}{y}{(\outputp{x}{y} | @{y})) | P}}\rangle 
	      | \prefix{x}{y}{(\outputp{x}{y} | @{y})} & \nonumber\\
	\red
	& (\outputp{x}{y} | @{y})\substn{\quotep{(\prefix{x}{y}{(@{y} | \outputp{x}{y})) | P}}}{y} & \nonumber\\
	=
	& \outputp{x}{\quotep{(\prefix{x}{y}{(\outputp{x}{y} | @{y})) | P}}}
	  | {(\prefix{x}{y}{(\outputp{x}{y} | @{y})) | P}} & \nonumber\\
	\red
	& \ldots & \nonumber\\
	\red^*
	& P | P | \ldots & \nonumber
\end{eqnarray}

Of course, this encoding, as an implementation, runs away, unfolding
$\bangp{P}$ eagerly. A lazier and more implementable replication
operator, restricted to input-guarded processes, may be obtained as follows.

\begin{eqnarray}
\bangp{\prefix{u}{v}{P}} 
	:= 
	\binpar{\lift{x}{\prefix{u}{v}{(\binpar{D(x)}{P})}}}{D(x)} \nonumber
\end{eqnarray}

\begin{remark}
  Note that the lazier definition still does not deal with summation
  or mixed summation (i.e. sums over input and output). The reader is
  invited to construct definitions of replication that deal with these
  features. 

  Further, the definitions are parameterized in a name, $x$. Can you,
  gentle reader, make a definition that eliminates this parameter and
  guarantees no accidental interaction between the replication
  machinery and the process being replicated -- i.e. no accidental
  sharing of names used by the process to get its work done and the
  name(s) used by the replication to effect copying. This latter
  revision of the definition of replication is crucial to obtaining
  the expected identity $!!P \sim !P$.
\end{remark}

\begin{remark}\label{rem:paradoxical_combinator}
  The reader familiar with the lambda calculus will have noticed the
  similarity between $D$ and the paradoxical combinator.

  [Ed. note: the existence of this seems to suggest we have to be more
  restrictive on the set of processes and names we admit if we are to
  support no-cloning.]
\end{remark}

\subsubsection{Bisimulation}

The computational dynamics gives rise to another kind of equivalence,
the equivalence of computational behavior. As previously mentioned
this is typically captured \emph{via} some form of bisimulation.

% The notion we use in this paper is weak barbed bisimulation
% \cite{milner91polyadicpi}.

The notion we use in this paper is derived from weak barbed
bisimulation \cite{milner91polyadicpi}. 

\begin{definition}
An \emph{observation relation}, $\downarrow_{\mathcal N}$, over a set
of names, $\mathcal N$, is the smallest relation satisfying the rules
below.

\infrule[Out-barb]{y \in {\mathcal N}, \; x \nameeq y}
		  {\outputp{x}{v} \downarrow_{\mathcal N} x}
\infrule[Par-barb]{\mbox{$P\downarrow_{\mathcal N} x$ or $Q\downarrow_{\mathcal N} x$}}
		  {\binpar{P}{Q} \downarrow_{\mathcal N} x}

We write $P \Downarrow_{\mathcal N} x$ if there is $Q$ such that 
$P \wred Q$ and $Q \downarrow_{\mathcal N} x$.
\end{definition}

\begin{definition}
%\label{def.bbisim}
An  ${\mathcal N}$-\emph{barbed bisimulation} over a set of names, ${\mathcal N}$, is a symmetric binary relation 
${\mathcal S}_{\mathcal N}$ between agents such that $P\rel{S}_{\mathcal N}Q$ implies:
\begin{enumerate}
\item If $P \red P'$ then $Q \wred Q'$ and $P'\rel{S}_{\mathcal N} Q'$.
\item If $P\downarrow_{\mathcal N} x$, then $Q\Downarrow_{\mathcal N} x$.
\end{enumerate}
$P$ is ${\mathcal N}$-barbed bisimilar to $Q$, written
$P \wbbisim_{\mathcal N} Q$, if $P \rel{S}_{\mathcal N} Q$ for some ${\mathcal N}$-barbed bisimulation ${\mathcal S}_{\mathcal N}$.
\end{definition}

$\mathcal{R} \subseteq \pi \times \pi$

$P \mathcal{R} Q => \forall P'. P \red P' \Rightarrow \exists Q'. Q \red Q', P' \mathcal{R} Q'$

$P \vdash x \Rightarrow Q \vdash x$

\begin{mathpar}
  \inferrule*[lab=Out-barb]{x \nameeq y}{{y}!\langle{Q}\rangle \vdash x}
  \and
  \inferrule*[lab=Par-barb]{\mbox{$P\vdash x$ or $Q\vdash x$}}{\binpar{P}{Q} \vdash x}
\end{mathpar}

\subsubsection{Contexts}

One of the principle advantages of computational calculi like the
$\pi$-calculus is a well-defined notion of context,
contextual-equivalence and a correlation between
contextual-equivalence and notions of bisimulation. The notion of
context allows the decomposition of a process into (sub-)process and
its syntactic environment, its context. Thus, a context may be
thought of as a process with a ``hole'' (written $\Box$) in it. The
application of a context $M$ to a process $P$, written $M[P]$, is
tantamount to filling the hole in $M$ with $P$. In this paper we do
not need the full weight of this theory, but do make use of the notion
of context in the proof the main theorem. 

\begin{mathpar}
  \inferrule* [lab=summation] {} {{M_{M},M_{N}} \bc \Box \;|\; x.M_{A} \;|\; M_{M}+M_{N}}
  \and
  \inferrule* [lab=agent] {} {{M_{A}} \bc (\vec{x})M_{P} \;| \; \clift{P_0,\ldots,M_{P},\ldots,P_N}}
  \and \\
  \inferrule* [lab=process] {} {{M_{P}} \bc M_{N} \;| \;P|M_{P} }
\end{mathpar} 

\begin{mathpar}
  \inferrule* [lab=sychronization] {} {M_{N} \bc \Box \;|\; x?M_{F} \;|\; x!M_{C}}
  \and
  \inferrule* [lab=abstraction] {} {{M_{F}} \bc (x)M_{P} }
  \and
  \inferrule* [lab=concretion] {} {{M_{C}} \bc \langle M_{P} \rangle }
  \and \\
  \inferrule* [lab=process] {} {{M_{P}} \bc M_{N} \;| \;P|M_{P} }
\end{mathpar}

\begin{definition}[contextual application] Given a context $M$, and
  process $P$, we define the \emph{contextual application}, $M[P] :=
  M\{P/\Box\}$. That is, the contextual application of M to P is the
  substitution of $P$ for $\Box$ in $M$.
\end{definition}

$\meaningof{-} : L \to \mathcal{P}(\pi)$

\begin{mathpar}
  \inferrule* [lab=collection] {} {\meaningof{true} = \pi, \and \meaningof{~E} = \pi \setminus \meaningof{E}, \and \meaningof{E_{1} \& E_{2}} = \meaningof{E_{1}} \cap \meaningof{E_{2}}}
\end{mathpar}

\begin{mathpar}
  \inferrule* [lab=structure] {} {\meaningof{0} = \{ P \in \pi | P \equiv 0 \}, \and \\ \meaningof{E_1 | E_2} = \{ P \in \pi | P \equiv P_{1} | P_{2}, P_{1} \in \meaningof{E_{1}}, P_{2} \in \meaningof{E_2}\} }
\end{mathpar}

\begin{mathpar}
 \inferrule* [lab=behavior] {} {\meaningof{\langle a?b \rangle E} = \{ P \in \pi | P \equiv Q | u?(y)P', \\ \and \\\\ \and \\ \;\;\; u \in \meaningof{a}, \forall z.P'\{z/y\} \in \meaningof{E\{z/b\}}\}, \and \\ \meaningof{a!E} = \{ P \in \pi | P \equiv Q | x!\langle P' \rangle, x \in \meaningof{a} P' \in \meaningof{E}\} }
\end{mathpar}

\begin{mathpar}
 \inferrule* [lab=nominal] {} {\meaningof{\quotep{E}} = \{ \quotep{P} \in \quotep{\pi} | P \in \meaningof{E} \}, \and \meaningof{\quotep{P}} = \{ \quotep{Q} \in \quotep{\pi} | P \equiv Q \} \and \\ \meaningof{@\quotep{E}} = \{ P \in \pi | P \equiv @x, x \in \meaningof{E} \}}
\end{mathpar}

\begin{eqnarray*}
  \\
  \meaningof{-} : TS \to ST
\end{eqnarray*}

\begin{eqnarray*}
  \\
  L : TS \to ST
\end{eqnarray*}

\begin{eqnarray*}
  \\
  P \models E \iff P \in \meaningof{E}
\end{eqnarray*}

\begin{eqnarray*}
  P \approx_{L} Q \iff \forall E \in L. P \models E \iff Q \models E
\end{eqnarray*}

\begin{eqnarray*}
  P \approx_{K} Q
\end{eqnarray*}

\begin{eqnarray*}
  P \approx Q
\end{eqnarray*}

$\approx_{K} = \approx = \approx_{L}$

\subsubsection{Contextual duality}

Note that contexts extend the quotation operation to a family of
operations from processes to names. Given a context, $M$, we can
define a \emph{nominal context}, $\quotep{M}$ by $\quotep{M}[P] :=
\quotep{M[P]}$. To foreshadow what is to come we observe that these
operations enjoy a duality with processes very much like the duality
between vectors and maps from vectors to scalars.

Further, because the calculus is essentially higher-order, we have a
correspondence between contexts and processes. More specifically,
given a name $x$ and a context $M$ we can construct $M^{*}_{x}$ such
that 

\begin{mathpar}
  M^{*}_{x} | \lift{x}{P} \red M[P]
\end{mathpar}

namely,

\begin{mathpar}
  M^{*}_{x} := x?(u).M[\dropn{u}]
\end{mathpar}

The dependence of $M^{*}_{x}$ on a name makes it an abstraction, 

\begin{mathpar}
  M^{*} := (x)x?(u).M[\dropn{u}]
\end{mathpar}

\subsection{Additional notation}

It will sometimes be convenient to denote the process a name
quotes. We already have the notation $x = \quotep{P}$, but it will be
convenient to introduce an alternate notation, $\procn{x}$, when we
want to emphasize the connection to the use of the name. Note that, by
virtue of name equivalence, $\quotep{\procn{x}} \nameeq x$; so, the
notation is consistent with previous definitions.

Further, because names have structure it is possible to effect
substitutions on the basis of that structure. This means we need to
upgrade our notation for substitutions, which we accomplish by
adapting comprehension notation. Thus,

\begin{mathpar}
  P\{ y / x : x \in S \}
\end{mathpar}

is interpreted to mean the process derived from P by replacing (in a
capture-avoiding manner) each occurrence of $x$ in $S$ by $y$. For example,

\begin{mathpar}
  P\{ \quotep{\procn{x}|\procn{x}} / x : x \in \freenames{P} \}
\end{mathpar}

will replace each (occurrence) of a free name $x$ in $P$ by
$\quotep{\procn{x}|\procn{x}}$.

Also, we will avail ourselves of the notation $x^{L}$ and $x^{R}$ to
denote injections of a name into disjoint copies of the name
space. There are numerous ways to accomplish this. One example can be
found in \cite{MeredithR05}. This notation overloads to vectors of
names: $\vec{x}^{\pi} := (x_{i}^{\pi} \; : \; 0 \leq i < |\vec{x}| )$ where $\pi \in \{L,R\}$.

We also use $P^{\Box} := P|\Box$.

In \cite{MeredithR05} an interpretation of the new operator is
given. It turns out that there are several possible interpretations
all enjoying the requisite algebraic properties of the operator (see
\cite{milner91polyadicpi}). We will therefore make liberal use of
$(\nu\; \vec{x})P$.

% subsection the_syntax_and_semantics_of_the_notation_system (end)   

\input{qm2pi.qmops} 

\input{qm2pi.sterngerlach} 

\input{qm2pi.metric} 

% section concurrent_process_calculi (end)

%\input{qm2pi.proofsketch}

% section proof sketch (end)

%\input{qm2pi.slviaknots} 

% section spatial logic via knots (end)

\input{qm2pi.conclusion}

% section conclusion (end)

%\input{qm2pi.dtcodes} 

% section wiring algorithm (end)

\input{qm2pi.ack} 

% section acknowledgments (end)

\newpage


\bibliographystyle{plain}   
\bibliography{../../biblios/main.bib}

\input{qm2pi.rhodetails}

\end{document}

 

% section acknowledgments (end)

\newpage


\bibliographystyle{plain}   
\bibliography{../../biblios/main.bib}

\documentclass[12pt]{llncs}
%\documentclass{jktr}

\usepackage[pdftex]{hyperref}                   
\usepackage {listings}
\usepackage {mathpartir}
\usepackage{bcprules}
%\usepackage{listings}
                       
\usepackage{graphicx} 
%\usepackage[margins=2.5cm,nohead,nofoot]{geometry}
%\usepackage{geometry}
\usepackage{amsfonts}
\usepackage{amstext}
\usepackage{latexsym}
\usepackage{amssymb}
\usepackage{color}


%\include{myPreamble}
\include{qm2pi.local} 

%\ifpdf
%\usepackage[pdftex]{graphicx}
%\else
%\usepackage{graphicx}
%\fi

 % \ifpdf
%  \usepackage{pdfsync}
%  \if


%\title{Brief Article}
%\author{David F. Snyder}
%\author{L.G. Meredith}

%\address{Dept. of Math., Texas State University--San Marcos, San Marcos, TX 78666}
       
\pagestyle{empty}


\begin{document}

\lstset{language=[Objective]Caml,frame=shadowbox}

\input{qm2pi.front}

% section front matter (end)

\input{qm2pi.intro} 
 
% section introduction (end)

% \input{qm2pi.knotations} 

% section notation (end)

\input{qm2pi.process.calculi} 

% section concurrent_process_calculi_and_spatial_logics_ (end)
    
%\input{qm2pi.knots2pi} 

%\input{qm2pi.trefoil} 

%\input{qm2pi.mainthm} 

% subsection basic_interpretation (end)

%\input{qm2pi.rho.presentation} 
\subsection{The syntax and semantics of the notation system}\label{sub:the_syntax_and_semantics_of_the_notation_system} % (fold)

We now summarize a technical presentation of the calculus that
embodies our theory of dynamics. The typical presentation of such a
calculus follows the style of giving generators and relations on
them. The grammar, below, describing term constructors, freely
generates the set of processes, $\Proc$. This set is then quotiented
by a relation known as structural congruence and it is over this set
that the notion of dynamics is expressed. This presentation is
essentially that of \cite{MeredithR05} with the addition of
polyadicity and summation. For readability we have relegated some of
the technical subtleties to an appendix.

\subsubsection{Process grammar}\label{subsub:process_grammar}

\begin{mathpar}
  \inferrule* [lab=synchronization] {} {{M} \bc \pzero \;|\; x?F \;|\; x!C }
  \and
  \inferrule* [lab=abstraction] {} {{F} \bc (x)P}
  \and
  \inferrule* [lab=concretion] {} {{C} \bc \langle Q \rangle}
  \and
  \inferrule* [lab=process] {} {{P,Q} \bc M \;| \;P|Q \;|\; @{x}}
  \and
  \inferrule* [lab=name] {} {{x} \bc \quotep{P}}
\end{mathpar} 

Note that $\vec{x}$ (resp. $\vec{P}$) denotes a vector of names
(resp. processes) of length $|\vec{x}|$ (resp. $|\vec{P}|$). We adopt
the following useful abbreviations.

\begin{mathpar}
   x?(\vec{y}).P := x.(\vec{y})P \and  x\clift{\vec{P}} := x.\clift{\vec{P}}
   \and x!(y) := \lift{x}{\dropn{y}}
   \and \Pi_{i=0}^{n-1}P_i := P_0 | \ldots | P_{n-1}
\end{mathpar}

\subsubsection{Structural congruence}

\paragraph{Free and bound names and alpha-equivalence.} At the
core of structural equivalence is alpha-equivalence which identifies
process that are the same up to a change of variable. Formally, we
recognize the distinction between free and bound names. The free names
of a process, $\freenames{P}$, may be calculated recursively as
follows:

\begin{mathpar}
\freenames{\pzero} := \emptyset
  \and \\
  \freenames{x?(y).P} := \{ x \} \cup (\freenames{P} \setminus \{ y \})
  \and 
  \freenames{x!\langle P \rangle} := \{ x \} \cup \{ P \} 
  \and \\
  \freenames{P|Q} := \freenames{P} \cup \freenames{Q}
  \and \\
  \freenames{@{x}} := \{ x \}
\end{mathpar}

$\pi$
$\quotep{\pi}$

$\freenames{-} : \pi \to \mathcal{P}(\quotep{\pi})$

\begin{eqnarray*}
  \freenames{\pzero} & := & \emptyset \\
  \freenames{x?(y).P} & := & \{ x \} \cup (\freenames{P} \setminus \{ y \}) \\
  \freenames{x!\langle P \rangle} & := & \{ x \} \cup \{ P \} \\
  \freenames{P|Q} & := & \freenames{P} \cup \freenames{Q} \\
  \freenames{\dropn{x}} & := & \{ x \}
\end{eqnarray*}

The bound names of a process, $\boundnames{P}$, are those names occurring in $P$
that are not free. For example, in $x?(y).0$, the name $x$ is free, while $y$ is bound.

\begin{mathpar}
  \inferrule* [lab=monoidal-laws] {} { P|Q \equiv Q|P \and P|0 \equiv P \and P|(Q|R) \equiv (P|Q)|R }
\end{mathpar}

\begin{mathpar}
  \inferrule* [lab=alpha-equivalence] {} { (x)P \equiv (y)P\{y/x\} \and y \not\in \freenames{P} }
\end{mathpar}

\begin{definition}
Then two processes, $P,Q$, are alpha-equivalent if $P = Q\{\vec{y}/\vec{x}\}$ for
some $\vec{x} \in \boundnames{Q},\vec{y} \in \boundnames{P}$, where $Q\{\vec{y}/\vec{x}\}$
denotes the capture-avoiding substitution of $\vec{y}$ for $\vec{x}$ in $Q$.
\end{definition}

\begin{definition}
  The {\em structural congruence} \cite{SangiorgiWalker} , $\equiv$,
  between processes is the least congruence containing
  alpha-equivalence, satisfying the abelian monoid laws
  (associativity, commutativity and $\pzero$ as identity) for parallel
  composition $|$ and for summation $+$.
\end{definition}

\subsection{Name equivalence}

We take name equivalence, written $\nameeq$, to be the smallest
equivalence relation generated by the following rules.

\begin{mathpar}
\inferrule*[lab=Quote-drop]
{ }
{ \quotep{@{x}} \nameeq x }

\inferrule*[lab=Struct-equiv]
{ P \scong Q }
{ \quotep{P} \nameeq \quotep{Q} }
\end{mathpar}

The astute reader will have noticed that the mutual recursion of names
and processes imposes a mutual recursion on alpha-equivalence and
structural equivalence via name-equivalence. Fortunately, all of this
works out pleasantly and we may calculate in the natural way, free of
concern. The reader interested in the details is referred to the
appendix \ref{appendix:rho_details}.

\subsection{Substitution}

We use $\Proc$ for the set of processes, $\QProc$ for the set of
names, and $\id{\{}\vec{y} / \vec{x} \id{\}}$ to denote partial maps,
$s : \QProc \rightarrow \QProc$. A map, $s$ lifts, uniquely, to a map
on process terms, $\widehat{s} : \Proc \rightarrow \Proc$ by the
following equations.

\begin{mathpar}
  (0) \psubstp{Q}{P} := 0 \\
  (R \juxtap S) \psubstp{Q}{P}
  :=    
  (R)\psubstp{Q}{P} \juxtap (S) \psubstp{Q}{P} \\
  (x?(y).R) \psubstp{Q}{P}    
  :=    
  (x)\substp{Q}{P} (z)\concat( (R \psubstn{z}{y}) \psubstp{Q}{P} ) \\
  (\lift{x}{R}) \psubstp{Q}{P}  
  :=
  \lift{(x)\substp{Q}{P}}{ R \psubstp{Q}{P} } \\
%   (\dropn{x})  \psubstp{Q}{P}       
%   := 
%   \left\{ 
%     \begin{array}{ccc} 
%       \dropn{\quotep{Q}} & & x \nameeq \quotep{P} \\
%       \dropn{x} & & otherwise \\
%     \end{array}
%   \right. 
  (\dropn{x})  \psubstp{Q}{P}       
  := 
  \left\{ 
    \begin{array}{ccc} 
      Q & & x \nameeq \quotep{P} \\
      \dropn{x} & & otherwise \\
    \end{array}
  \right.
\end{mathpar}
 

where

\begin{eqnarray}
  (x)\id{\{} \lpquote Q \rpquote / \lpquote P \rpquote \id{\}}            = 
  \left\{ 
    \begin{array}{ccc}
      \lpquote Q \rpquote & & x \nameeq \lpquote P \rpquote \\
      x & & otherwise \\
    \end{array}
  \right. \nonumber
\end{eqnarray}

and $z$ is chosen distinct from $\quotep{P}$, $\quotep{Q}$, the free
names in $Q$, and all the names in $R$. Our $\alpha$-equivalence will
be built in the standard way from this substitution.

\begin{remark}\label{rem:no_self_referential_names}
  One consequence of these definitions is that $\forall P. \quotep{P}
  \not\in \freenames{P}$.
\end{remark}

\subsection{ Dynamic quote: an example }

Anticipating something of what's to come, consider applying the
substitution, $\widehat{\id{\{}u / z \id{\}}}$, to the following pair
of processes, $\lift{w}{y!(z)}$ and $w[ \lpquote y!(z) \rpquote ]$.

\begin{eqnarray}
	\lift{w}{y!(z)}\widehat{\id{\{}u / z \id{\}}}
		& = &
		\lift{w}{y!(u)} \nonumber\\
	w[ \lpquote y!(z) \rpquote ] \widehat{ \id{\{}u / z \id{\}} }
		& = &
		w[ \lpquote y!(z) \rpquote ] \nonumber
\end{eqnarray}

Because the body of the process between quotes is impervious to
substitution, we get radically different answers. In fact, by
examining the first process in an input context,
e.g. $x?(z).\lift{w}{y!(z)}$, we see that the process under the lift
operator may be shaped by prefixed inputs binding a name inside it. In
this sense, the lift operator will be seen as a way to dynamically
construct processes before reifying them as names.

Finally equipped with these standard features we can present the
dynamics of the calculus.

\subsubsection{Operational semantics} 

Finally, we introduce the computational dynamics. What marks these
algebras as distinct from other more traditionally studied algebraic
structures, e.g. vector spaces or polynomial rings, is the manner in
which dynamics is captured. In traditional structures, dynamics is typically
expressed through morphisms between such structures, as in linear maps
between vector spaces or morphisms between rings. In algebras
associated with the semantics of computation, the dynamics is
expressed as part of the algebraic structure itself, through a
reduction reduction relation typically denoted by $\red$. Below, we
give a recursive presentation of this relation for the calculus used
in the encoding.

$\red \subseteq \pi \times \pi$
$\red : \pi \to \mathcal{P}(\pi)$

\begin{mathpar}
  \inferrule* [lab=Comm] { \textsf{match}( x_{src}, x_{trgt} ) } { x_{trgt}?(y)P \; | \; x_{src}!\langle {Q} \rangle \red P\{\quotep{Q}/y}\} }
  \and \\
  \inferrule* [lab=Par] {{P} \red {P}'} {{{P} | {Q}} \red {{P}' | {Q}}}
  \and
  \inferrule* [lab=Equiv]{{{P} \scong {P}'} \andalso {{P}' \red {Q}'} \andalso {{Q}' \scong {Q}}}{{P} \red {Q}}
\end{mathpar}

\begin{eqnarray*}
  match_{\equiv} (\quotep{P},\quotep{Q}) & := & P \equiv Q \\
  match_{\dagger}(\quotep{P},\quotep{Q}) & := & \forall R. P|Q \red^{*} R => R \red^{*} 0 \\
  match_{K}(\quotep{P},\quotep{Q}) & := & K \mbox{ for some context } K
\end{eqnarray*}

$u?(x)P | u!\langle Q \rangle \red P\{\quotep{Q}/x\}$

%We write $\wred$ for $\red^*$, and $P\red$ if $\exists Q $ such that $ P \red Q$.
We write $P\red$ if $\exists Q $ such that $ P \red Q$ and $P\not\red$, otherwise.

\section{Replication}

As mentioned before, it is known that replication (and hence
recursion) can be implemented in a higher-order process algebra
\cite{SangiorgiWalker}. As our first example of calculation with the
machinery thus far presented we give the construction explicitly in
the {\rhoc}.

\begin{eqnarray}
	D_{x} & := & \prefix{x}{y}{(\binpar{\outputp{x}{y}}{@{y}})} \nonumber\\
	\bangp_{x}{P} & := & \binpar{{x}!\langle{\binpar{D_{x}}{P}}\rangle}{D_{x}} \nonumber
\end{eqnarray}

\begin{eqnarray}
	\bangp_{x}{P} & & \nonumber\\
	=
	& {x}!\langle{(\prefix{x}{y}{(\outputp{x}{y} | @{y})) | P}}\rangle 
	      | \prefix{x}{y}{(\outputp{x}{y} | @{y})} & \nonumber\\
	\red
	& (\outputp{x}{y} | @{y})\substn{\quotep{(\prefix{x}{y}{(@{y} | \outputp{x}{y})) | P}}}{y} & \nonumber\\
	=
	& \outputp{x}{\quotep{(\prefix{x}{y}{(\outputp{x}{y} | @{y})) | P}}}
	  | {(\prefix{x}{y}{(\outputp{x}{y} | @{y})) | P}} & \nonumber\\
	\red
	& \ldots & \nonumber\\
	\red^*
	& P | P | \ldots & \nonumber
\end{eqnarray}

Of course, this encoding, as an implementation, runs away, unfolding
$\bangp{P}$ eagerly. A lazier and more implementable replication
operator, restricted to input-guarded processes, may be obtained as follows.

\begin{eqnarray}
\bangp{\prefix{u}{v}{P}} 
	:= 
	\binpar{\lift{x}{\prefix{u}{v}{(\binpar{D(x)}{P})}}}{D(x)} \nonumber
\end{eqnarray}

\begin{remark}
  Note that the lazier definition still does not deal with summation
  or mixed summation (i.e. sums over input and output). The reader is
  invited to construct definitions of replication that deal with these
  features. 

  Further, the definitions are parameterized in a name, $x$. Can you,
  gentle reader, make a definition that eliminates this parameter and
  guarantees no accidental interaction between the replication
  machinery and the process being replicated -- i.e. no accidental
  sharing of names used by the process to get its work done and the
  name(s) used by the replication to effect copying. This latter
  revision of the definition of replication is crucial to obtaining
  the expected identity $!!P \sim !P$.
\end{remark}

\begin{remark}\label{rem:paradoxical_combinator}
  The reader familiar with the lambda calculus will have noticed the
  similarity between $D$ and the paradoxical combinator.

  [Ed. note: the existence of this seems to suggest we have to be more
  restrictive on the set of processes and names we admit if we are to
  support no-cloning.]
\end{remark}

\subsubsection{Bisimulation}

The computational dynamics gives rise to another kind of equivalence,
the equivalence of computational behavior. As previously mentioned
this is typically captured \emph{via} some form of bisimulation.

% The notion we use in this paper is weak barbed bisimulation
% \cite{milner91polyadicpi}.

The notion we use in this paper is derived from weak barbed
bisimulation \cite{milner91polyadicpi}. 

\begin{definition}
An \emph{observation relation}, $\downarrow_{\mathcal N}$, over a set
of names, $\mathcal N$, is the smallest relation satisfying the rules
below.

\infrule[Out-barb]{y \in {\mathcal N}, \; x \nameeq y}
		  {\outputp{x}{v} \downarrow_{\mathcal N} x}
\infrule[Par-barb]{\mbox{$P\downarrow_{\mathcal N} x$ or $Q\downarrow_{\mathcal N} x$}}
		  {\binpar{P}{Q} \downarrow_{\mathcal N} x}

We write $P \Downarrow_{\mathcal N} x$ if there is $Q$ such that 
$P \wred Q$ and $Q \downarrow_{\mathcal N} x$.
\end{definition}

\begin{definition}
%\label{def.bbisim}
An  ${\mathcal N}$-\emph{barbed bisimulation} over a set of names, ${\mathcal N}$, is a symmetric binary relation 
${\mathcal S}_{\mathcal N}$ between agents such that $P\rel{S}_{\mathcal N}Q$ implies:
\begin{enumerate}
\item If $P \red P'$ then $Q \wred Q'$ and $P'\rel{S}_{\mathcal N} Q'$.
\item If $P\downarrow_{\mathcal N} x$, then $Q\Downarrow_{\mathcal N} x$.
\end{enumerate}
$P$ is ${\mathcal N}$-barbed bisimilar to $Q$, written
$P \wbbisim_{\mathcal N} Q$, if $P \rel{S}_{\mathcal N} Q$ for some ${\mathcal N}$-barbed bisimulation ${\mathcal S}_{\mathcal N}$.
\end{definition}

$\mathcal{R} \subseteq \pi \times \pi$

$P \mathcal{R} Q => \forall P'. P \red P' \Rightarrow \exists Q'. Q \red Q', P' \mathcal{R} Q'$

$P \vdash x \Rightarrow Q \vdash x$

\begin{mathpar}
  \inferrule*[lab=Out-barb]{x \nameeq y}{{y}!\langle{Q}\rangle \vdash x}
  \and
  \inferrule*[lab=Par-barb]{\mbox{$P\vdash x$ or $Q\vdash x$}}{\binpar{P}{Q} \vdash x}
\end{mathpar}

\subsubsection{Contexts}

One of the principle advantages of computational calculi like the
$\pi$-calculus is a well-defined notion of context,
contextual-equivalence and a correlation between
contextual-equivalence and notions of bisimulation. The notion of
context allows the decomposition of a process into (sub-)process and
its syntactic environment, its context. Thus, a context may be
thought of as a process with a ``hole'' (written $\Box$) in it. The
application of a context $M$ to a process $P$, written $M[P]$, is
tantamount to filling the hole in $M$ with $P$. In this paper we do
not need the full weight of this theory, but do make use of the notion
of context in the proof the main theorem. 

\begin{mathpar}
  \inferrule* [lab=summation] {} {{M_{M},M_{N}} \bc \Box \;|\; x.M_{A} \;|\; M_{M}+M_{N}}
  \and
  \inferrule* [lab=agent] {} {{M_{A}} \bc (\vec{x})M_{P} \;| \; \clift{P_0,\ldots,M_{P},\ldots,P_N}}
  \and \\
  \inferrule* [lab=process] {} {{M_{P}} \bc M_{N} \;| \;P|M_{P} }
\end{mathpar} 

\begin{mathpar}
  \inferrule* [lab=sychronization] {} {M_{N} \bc \Box \;|\; x?M_{F} \;|\; x!M_{C}}
  \and
  \inferrule* [lab=abstraction] {} {{M_{F}} \bc (x)M_{P} }
  \and
  \inferrule* [lab=concretion] {} {{M_{C}} \bc \langle M_{P} \rangle }
  \and \\
  \inferrule* [lab=process] {} {{M_{P}} \bc M_{N} \;| \;P|M_{P} }
\end{mathpar}

\begin{definition}[contextual application] Given a context $M$, and
  process $P$, we define the \emph{contextual application}, $M[P] :=
  M\{P/\Box\}$. That is, the contextual application of M to P is the
  substitution of $P$ for $\Box$ in $M$.
\end{definition}

$\meaningof{-} : L \to \mathcal{P}(\pi)$

\begin{mathpar}
  \inferrule* [lab=collection] {} {\meaningof{true} = \pi, \and \meaningof{~E} = \pi \setminus \meaningof{E}, \and \meaningof{E_{1} \& E_{2}} = \meaningof{E_{1}} \cap \meaningof{E_{2}}}
\end{mathpar}

\begin{mathpar}
  \inferrule* [lab=structure] {} {\meaningof{0} = \{ P \in \pi | P \equiv 0 \}, \and \\ \meaningof{E_1 | E_2} = \{ P \in \pi | P \equiv P_{1} | P_{2}, P_{1} \in \meaningof{E_{1}}, P_{2} \in \meaningof{E_2}\} }
\end{mathpar}

\begin{mathpar}
 \inferrule* [lab=behavior] {} {\meaningof{\langle a?b \rangle E} = \{ P \in \pi | P \equiv Q | u?(y)P', \\ \and \\\\ \and \\ \;\;\; u \in \meaningof{a}, \forall z.P'\{z/y\} \in \meaningof{E\{z/b\}}\}, \and \\ \meaningof{a!E} = \{ P \in \pi | P \equiv Q | x!\langle P' \rangle, x \in \meaningof{a} P' \in \meaningof{E}\} }
\end{mathpar}

\begin{mathpar}
 \inferrule* [lab=nominal] {} {\meaningof{\quotep{E}} = \{ \quotep{P} \in \quotep{\pi} | P \in \meaningof{E} \}, \and \meaningof{\quotep{P}} = \{ \quotep{Q} \in \quotep{\pi} | P \equiv Q \} \and \\ \meaningof{@\quotep{E}} = \{ P \in \pi | P \equiv @x, x \in \meaningof{E} \}}
\end{mathpar}

\begin{eqnarray*}
  \\
  \meaningof{-} : TS \to ST
\end{eqnarray*}

\begin{eqnarray*}
  \\
  L : TS \to ST
\end{eqnarray*}

\begin{eqnarray*}
  \\
  P \models E \iff P \in \meaningof{E}
\end{eqnarray*}

\begin{eqnarray*}
  P \approx_{L} Q \iff \forall E \in L. P \models E \iff Q \models E
\end{eqnarray*}

\begin{eqnarray*}
  P \approx_{K} Q
\end{eqnarray*}

\begin{eqnarray*}
  P \approx Q
\end{eqnarray*}

$\approx_{K} = \approx = \approx_{L}$

\subsubsection{Contextual duality}

Note that contexts extend the quotation operation to a family of
operations from processes to names. Given a context, $M$, we can
define a \emph{nominal context}, $\quotep{M}$ by $\quotep{M}[P] :=
\quotep{M[P]}$. To foreshadow what is to come we observe that these
operations enjoy a duality with processes very much like the duality
between vectors and maps from vectors to scalars.

Further, because the calculus is essentially higher-order, we have a
correspondence between contexts and processes. More specifically,
given a name $x$ and a context $M$ we can construct $M^{*}_{x}$ such
that 

\begin{mathpar}
  M^{*}_{x} | \lift{x}{P} \red M[P]
\end{mathpar}

namely,

\begin{mathpar}
  M^{*}_{x} := x?(u).M[\dropn{u}]
\end{mathpar}

The dependence of $M^{*}_{x}$ on a name makes it an abstraction, 

\begin{mathpar}
  M^{*} := (x)x?(u).M[\dropn{u}]
\end{mathpar}

\subsection{Additional notation}

It will sometimes be convenient to denote the process a name
quotes. We already have the notation $x = \quotep{P}$, but it will be
convenient to introduce an alternate notation, $\procn{x}$, when we
want to emphasize the connection to the use of the name. Note that, by
virtue of name equivalence, $\quotep{\procn{x}} \nameeq x$; so, the
notation is consistent with previous definitions.

Further, because names have structure it is possible to effect
substitutions on the basis of that structure. This means we need to
upgrade our notation for substitutions, which we accomplish by
adapting comprehension notation. Thus,

\begin{mathpar}
  P\{ y / x : x \in S \}
\end{mathpar}

is interpreted to mean the process derived from P by replacing (in a
capture-avoiding manner) each occurrence of $x$ in $S$ by $y$. For example,

\begin{mathpar}
  P\{ \quotep{\procn{x}|\procn{x}} / x : x \in \freenames{P} \}
\end{mathpar}

will replace each (occurrence) of a free name $x$ in $P$ by
$\quotep{\procn{x}|\procn{x}}$.

Also, we will avail ourselves of the notation $x^{L}$ and $x^{R}$ to
denote injections of a name into disjoint copies of the name
space. There are numerous ways to accomplish this. One example can be
found in \cite{MeredithR05}. This notation overloads to vectors of
names: $\vec{x}^{\pi} := (x_{i}^{\pi} \; : \; 0 \leq i < |\vec{x}| )$ where $\pi \in \{L,R\}$.

We also use $P^{\Box} := P|\Box$.

In \cite{MeredithR05} an interpretation of the new operator is
given. It turns out that there are several possible interpretations
all enjoying the requisite algebraic properties of the operator (see
\cite{milner91polyadicpi}). We will therefore make liberal use of
$(\nu\; \vec{x})P$.

% subsection the_syntax_and_semantics_of_the_notation_system (end)   

\input{qm2pi.qmops} 

\input{qm2pi.sterngerlach} 

\input{qm2pi.metric} 

% section concurrent_process_calculi (end)

%\input{qm2pi.proofsketch}

% section proof sketch (end)

%\input{qm2pi.slviaknots} 

% section spatial logic via knots (end)

\input{qm2pi.conclusion}

% section conclusion (end)

%\input{qm2pi.dtcodes} 

% section wiring algorithm (end)

\input{qm2pi.ack} 

% section acknowledgments (end)

\newpage


\bibliographystyle{plain}   
\bibliography{../../biblios/main.bib}

\input{qm2pi.rhodetails}

\end{document}



\end{document}

 

% subsection basic_interpretation (end)

%\input{qm2pi.rho.presentation} 
\subsection{The syntax and semantics of the notation system}\label{sub:the_syntax_and_semantics_of_the_notation_system} % (fold)

We now summarize a technical presentation of the calculus that
embodies our theory of dynamics. The typical presentation of such a
calculus follows the style of giving generators and relations on
them. The grammar, below, describing term constructors, freely
generates the set of processes, $\Proc$. This set is then quotiented
by a relation known as structural congruence and it is over this set
that the notion of dynamics is expressed. This presentation is
essentially that of \cite{MeredithR05} with the addition of
polyadicity and summation. For readability we have relegated some of
the technical subtleties to an appendix.

\subsubsection{Process grammar}\label{subsub:process_grammar}

\begin{mathpar}
  \inferrule* [lab=synchronization] {} {{M} \bc \pzero \;|\; x?F \;|\; x!C }
  \and
  \inferrule* [lab=abstraction] {} {{F} \bc (x)P}
  \and
  \inferrule* [lab=concretion] {} {{C} \bc \langle Q \rangle}
  \and
  \inferrule* [lab=process] {} {{P,Q} \bc M \;| \;P|Q \;|\; @{x}}
  \and
  \inferrule* [lab=name] {} {{x} \bc \quotep{P}}
\end{mathpar} 

Note that $\vec{x}$ (resp. $\vec{P}$) denotes a vector of names
(resp. processes) of length $|\vec{x}|$ (resp. $|\vec{P}|$). We adopt
the following useful abbreviations.

\begin{mathpar}
   x?(\vec{y}).P := x.(\vec{y})P \and  x\clift{\vec{P}} := x.\clift{\vec{P}}
   \and x!(y) := \lift{x}{\dropn{y}}
   \and \Pi_{i=0}^{n-1}P_i := P_0 | \ldots | P_{n-1}
\end{mathpar}

\subsubsection{Structural congruence}

\paragraph{Free and bound names and alpha-equivalence.} At the
core of structural equivalence is alpha-equivalence which identifies
process that are the same up to a change of variable. Formally, we
recognize the distinction between free and bound names. The free names
of a process, $\freenames{P}$, may be calculated recursively as
follows:

\begin{mathpar}
\freenames{\pzero} := \emptyset
  \and \\
  \freenames{x?(y).P} := \{ x \} \cup (\freenames{P} \setminus \{ y \})
  \and 
  \freenames{x!\langle P \rangle} := \{ x \} \cup \{ P \} 
  \and \\
  \freenames{P|Q} := \freenames{P} \cup \freenames{Q}
  \and \\
  \freenames{@{x}} := \{ x \}
\end{mathpar}

$\pi$
$\quotep{\pi}$

$\freenames{-} : \pi \to \mathcal{P}(\quotep{\pi})$

\begin{eqnarray*}
  \freenames{\pzero} & := & \emptyset \\
  \freenames{x?(y).P} & := & \{ x \} \cup (\freenames{P} \setminus \{ y \}) \\
  \freenames{x!\langle P \rangle} & := & \{ x \} \cup \{ P \} \\
  \freenames{P|Q} & := & \freenames{P} \cup \freenames{Q} \\
  \freenames{\dropn{x}} & := & \{ x \}
\end{eqnarray*}

The bound names of a process, $\boundnames{P}$, are those names occurring in $P$
that are not free. For example, in $x?(y).0$, the name $x$ is free, while $y$ is bound.

\begin{mathpar}
  \inferrule* [lab=monoidal-laws] {} { P|Q \equiv Q|P \and P|0 \equiv P \and P|(Q|R) \equiv (P|Q)|R }
\end{mathpar}

\begin{mathpar}
  \inferrule* [lab=alpha-equivalence] {} { (x)P \equiv (y)P\{y/x\} \and y \not\in \freenames{P} }
\end{mathpar}

\begin{definition}
Then two processes, $P,Q$, are alpha-equivalent if $P = Q\{\vec{y}/\vec{x}\}$ for
some $\vec{x} \in \boundnames{Q},\vec{y} \in \boundnames{P}$, where $Q\{\vec{y}/\vec{x}\}$
denotes the capture-avoiding substitution of $\vec{y}$ for $\vec{x}$ in $Q$.
\end{definition}

\begin{definition}
  The {\em structural congruence} \cite{SangiorgiWalker} , $\equiv$,
  between processes is the least congruence containing
  alpha-equivalence, satisfying the abelian monoid laws
  (associativity, commutativity and $\pzero$ as identity) for parallel
  composition $|$ and for summation $+$.
\end{definition}

\subsection{Name equivalence}

We take name equivalence, written $\nameeq$, to be the smallest
equivalence relation generated by the following rules.

\begin{mathpar}
\inferrule*[lab=Quote-drop]
{ }
{ \quotep{@{x}} \nameeq x }

\inferrule*[lab=Struct-equiv]
{ P \scong Q }
{ \quotep{P} \nameeq \quotep{Q} }
\end{mathpar}

The astute reader will have noticed that the mutual recursion of names
and processes imposes a mutual recursion on alpha-equivalence and
structural equivalence via name-equivalence. Fortunately, all of this
works out pleasantly and we may calculate in the natural way, free of
concern. The reader interested in the details is referred to the
appendix \ref{appendix:rho_details}.

\subsection{Substitution}

We use $\Proc$ for the set of processes, $\QProc$ for the set of
names, and $\id{\{}\vec{y} / \vec{x} \id{\}}$ to denote partial maps,
$s : \QProc \rightarrow \QProc$. A map, $s$ lifts, uniquely, to a map
on process terms, $\widehat{s} : \Proc \rightarrow \Proc$ by the
following equations.

\begin{mathpar}
  (0) \psubstp{Q}{P} := 0 \\
  (R \juxtap S) \psubstp{Q}{P}
  :=    
  (R)\psubstp{Q}{P} \juxtap (S) \psubstp{Q}{P} \\
  (x?(y).R) \psubstp{Q}{P}    
  :=    
  (x)\substp{Q}{P} (z)\concat( (R \psubstn{z}{y}) \psubstp{Q}{P} ) \\
  (\lift{x}{R}) \psubstp{Q}{P}  
  :=
  \lift{(x)\substp{Q}{P}}{ R \psubstp{Q}{P} } \\
%   (\dropn{x})  \psubstp{Q}{P}       
%   := 
%   \left\{ 
%     \begin{array}{ccc} 
%       \dropn{\quotep{Q}} & & x \nameeq \quotep{P} \\
%       \dropn{x} & & otherwise \\
%     \end{array}
%   \right. 
  (\dropn{x})  \psubstp{Q}{P}       
  := 
  \left\{ 
    \begin{array}{ccc} 
      Q & & x \nameeq \quotep{P} \\
      \dropn{x} & & otherwise \\
    \end{array}
  \right.
\end{mathpar}
 

where

\begin{eqnarray}
  (x)\id{\{} \lpquote Q \rpquote / \lpquote P \rpquote \id{\}}            = 
  \left\{ 
    \begin{array}{ccc}
      \lpquote Q \rpquote & & x \nameeq \lpquote P \rpquote \\
      x & & otherwise \\
    \end{array}
  \right. \nonumber
\end{eqnarray}

and $z$ is chosen distinct from $\quotep{P}$, $\quotep{Q}$, the free
names in $Q$, and all the names in $R$. Our $\alpha$-equivalence will
be built in the standard way from this substitution.

\begin{remark}\label{rem:no_self_referential_names}
  One consequence of these definitions is that $\forall P. \quotep{P}
  \not\in \freenames{P}$.
\end{remark}

\subsection{ Dynamic quote: an example }

Anticipating something of what's to come, consider applying the
substitution, $\widehat{\id{\{}u / z \id{\}}}$, to the following pair
of processes, $\lift{w}{y!(z)}$ and $w[ \lpquote y!(z) \rpquote ]$.

\begin{eqnarray}
	\lift{w}{y!(z)}\widehat{\id{\{}u / z \id{\}}}
		& = &
		\lift{w}{y!(u)} \nonumber\\
	w[ \lpquote y!(z) \rpquote ] \widehat{ \id{\{}u / z \id{\}} }
		& = &
		w[ \lpquote y!(z) \rpquote ] \nonumber
\end{eqnarray}

Because the body of the process between quotes is impervious to
substitution, we get radically different answers. In fact, by
examining the first process in an input context,
e.g. $x?(z).\lift{w}{y!(z)}$, we see that the process under the lift
operator may be shaped by prefixed inputs binding a name inside it. In
this sense, the lift operator will be seen as a way to dynamically
construct processes before reifying them as names.

Finally equipped with these standard features we can present the
dynamics of the calculus.

\subsubsection{Operational semantics} 

Finally, we introduce the computational dynamics. What marks these
algebras as distinct from other more traditionally studied algebraic
structures, e.g. vector spaces or polynomial rings, is the manner in
which dynamics is captured. In traditional structures, dynamics is typically
expressed through morphisms between such structures, as in linear maps
between vector spaces or morphisms between rings. In algebras
associated with the semantics of computation, the dynamics is
expressed as part of the algebraic structure itself, through a
reduction reduction relation typically denoted by $\red$. Below, we
give a recursive presentation of this relation for the calculus used
in the encoding.

$\red \subseteq \pi \times \pi$
$\red : \pi \to \mathcal{P}(\pi)$

\begin{mathpar}
  \inferrule* [lab=Comm] { \textsf{match}( x_{src}, x_{trgt} ) } { x_{trgt}?(y)P \; | \; x_{src}!\langle {Q} \rangle \red P\{\quotep{Q}/y}\} }
  \and \\
  \inferrule* [lab=Par] {{P} \red {P}'} {{{P} | {Q}} \red {{P}' | {Q}}}
  \and
  \inferrule* [lab=Equiv]{{{P} \scong {P}'} \andalso {{P}' \red {Q}'} \andalso {{Q}' \scong {Q}}}{{P} \red {Q}}
\end{mathpar}

\begin{eqnarray*}
  match_{\equiv} (\quotep{P},\quotep{Q}) & := & P \equiv Q \\
  match_{\dagger}(\quotep{P},\quotep{Q}) & := & \forall R. P|Q \red^{*} R => R \red^{*} 0 \\
  match_{K}(\quotep{P},\quotep{Q}) & := & K \mbox{ for some context } K
\end{eqnarray*}

$u?(x)P | u!\langle Q \rangle \red P\{\quotep{Q}/x\}$

%We write $\wred$ for $\red^*$, and $P\red$ if $\exists Q $ such that $ P \red Q$.
We write $P\red$ if $\exists Q $ such that $ P \red Q$ and $P\not\red$, otherwise.

\section{Replication}

As mentioned before, it is known that replication (and hence
recursion) can be implemented in a higher-order process algebra
\cite{SangiorgiWalker}. As our first example of calculation with the
machinery thus far presented we give the construction explicitly in
the {\rhoc}.

\begin{eqnarray}
	D_{x} & := & \prefix{x}{y}{(\binpar{\outputp{x}{y}}{@{y}})} \nonumber\\
	\bangp_{x}{P} & := & \binpar{{x}!\langle{\binpar{D_{x}}{P}}\rangle}{D_{x}} \nonumber
\end{eqnarray}

\begin{eqnarray}
	\bangp_{x}{P} & & \nonumber\\
	=
	& {x}!\langle{(\prefix{x}{y}{(\outputp{x}{y} | @{y})) | P}}\rangle 
	      | \prefix{x}{y}{(\outputp{x}{y} | @{y})} & \nonumber\\
	\red
	& (\outputp{x}{y} | @{y})\substn{\quotep{(\prefix{x}{y}{(@{y} | \outputp{x}{y})) | P}}}{y} & \nonumber\\
	=
	& \outputp{x}{\quotep{(\prefix{x}{y}{(\outputp{x}{y} | @{y})) | P}}}
	  | {(\prefix{x}{y}{(\outputp{x}{y} | @{y})) | P}} & \nonumber\\
	\red
	& \ldots & \nonumber\\
	\red^*
	& P | P | \ldots & \nonumber
\end{eqnarray}

Of course, this encoding, as an implementation, runs away, unfolding
$\bangp{P}$ eagerly. A lazier and more implementable replication
operator, restricted to input-guarded processes, may be obtained as follows.

\begin{eqnarray}
\bangp{\prefix{u}{v}{P}} 
	:= 
	\binpar{\lift{x}{\prefix{u}{v}{(\binpar{D(x)}{P})}}}{D(x)} \nonumber
\end{eqnarray}

\begin{remark}
  Note that the lazier definition still does not deal with summation
  or mixed summation (i.e. sums over input and output). The reader is
  invited to construct definitions of replication that deal with these
  features. 

  Further, the definitions are parameterized in a name, $x$. Can you,
  gentle reader, make a definition that eliminates this parameter and
  guarantees no accidental interaction between the replication
  machinery and the process being replicated -- i.e. no accidental
  sharing of names used by the process to get its work done and the
  name(s) used by the replication to effect copying. This latter
  revision of the definition of replication is crucial to obtaining
  the expected identity $!!P \sim !P$.
\end{remark}

\begin{remark}\label{rem:paradoxical_combinator}
  The reader familiar with the lambda calculus will have noticed the
  similarity between $D$ and the paradoxical combinator.

  [Ed. note: the existence of this seems to suggest we have to be more
  restrictive on the set of processes and names we admit if we are to
  support no-cloning.]
\end{remark}

\subsubsection{Bisimulation}

The computational dynamics gives rise to another kind of equivalence,
the equivalence of computational behavior. As previously mentioned
this is typically captured \emph{via} some form of bisimulation.

% The notion we use in this paper is weak barbed bisimulation
% \cite{milner91polyadicpi}.

The notion we use in this paper is derived from weak barbed
bisimulation \cite{milner91polyadicpi}. 

\begin{definition}
An \emph{observation relation}, $\downarrow_{\mathcal N}$, over a set
of names, $\mathcal N$, is the smallest relation satisfying the rules
below.

\infrule[Out-barb]{y \in {\mathcal N}, \; x \nameeq y}
		  {\outputp{x}{v} \downarrow_{\mathcal N} x}
\infrule[Par-barb]{\mbox{$P\downarrow_{\mathcal N} x$ or $Q\downarrow_{\mathcal N} x$}}
		  {\binpar{P}{Q} \downarrow_{\mathcal N} x}

We write $P \Downarrow_{\mathcal N} x$ if there is $Q$ such that 
$P \wred Q$ and $Q \downarrow_{\mathcal N} x$.
\end{definition}

\begin{definition}
%\label{def.bbisim}
An  ${\mathcal N}$-\emph{barbed bisimulation} over a set of names, ${\mathcal N}$, is a symmetric binary relation 
${\mathcal S}_{\mathcal N}$ between agents such that $P\rel{S}_{\mathcal N}Q$ implies:
\begin{enumerate}
\item If $P \red P'$ then $Q \wred Q'$ and $P'\rel{S}_{\mathcal N} Q'$.
\item If $P\downarrow_{\mathcal N} x$, then $Q\Downarrow_{\mathcal N} x$.
\end{enumerate}
$P$ is ${\mathcal N}$-barbed bisimilar to $Q$, written
$P \wbbisim_{\mathcal N} Q$, if $P \rel{S}_{\mathcal N} Q$ for some ${\mathcal N}$-barbed bisimulation ${\mathcal S}_{\mathcal N}$.
\end{definition}

$\mathcal{R} \subseteq \pi \times \pi$

$P \mathcal{R} Q => \forall P'. P \red P' \Rightarrow \exists Q'. Q \red Q', P' \mathcal{R} Q'$

$P \vdash x \Rightarrow Q \vdash x$

\begin{mathpar}
  \inferrule*[lab=Out-barb]{x \nameeq y}{{y}!\langle{Q}\rangle \vdash x}
  \and
  \inferrule*[lab=Par-barb]{\mbox{$P\vdash x$ or $Q\vdash x$}}{\binpar{P}{Q} \vdash x}
\end{mathpar}

\subsubsection{Contexts}

One of the principle advantages of computational calculi like the
$\pi$-calculus is a well-defined notion of context,
contextual-equivalence and a correlation between
contextual-equivalence and notions of bisimulation. The notion of
context allows the decomposition of a process into (sub-)process and
its syntactic environment, its context. Thus, a context may be
thought of as a process with a ``hole'' (written $\Box$) in it. The
application of a context $M$ to a process $P$, written $M[P]$, is
tantamount to filling the hole in $M$ with $P$. In this paper we do
not need the full weight of this theory, but do make use of the notion
of context in the proof the main theorem. 

\begin{mathpar}
  \inferrule* [lab=summation] {} {{M_{M},M_{N}} \bc \Box \;|\; x.M_{A} \;|\; M_{M}+M_{N}}
  \and
  \inferrule* [lab=agent] {} {{M_{A}} \bc (\vec{x})M_{P} \;| \; \clift{P_0,\ldots,M_{P},\ldots,P_N}}
  \and \\
  \inferrule* [lab=process] {} {{M_{P}} \bc M_{N} \;| \;P|M_{P} }
\end{mathpar} 

\begin{mathpar}
  \inferrule* [lab=sychronization] {} {M_{N} \bc \Box \;|\; x?M_{F} \;|\; x!M_{C}}
  \and
  \inferrule* [lab=abstraction] {} {{M_{F}} \bc (x)M_{P} }
  \and
  \inferrule* [lab=concretion] {} {{M_{C}} \bc \langle M_{P} \rangle }
  \and \\
  \inferrule* [lab=process] {} {{M_{P}} \bc M_{N} \;| \;P|M_{P} }
\end{mathpar}

\begin{definition}[contextual application] Given a context $M$, and
  process $P$, we define the \emph{contextual application}, $M[P] :=
  M\{P/\Box\}$. That is, the contextual application of M to P is the
  substitution of $P$ for $\Box$ in $M$.
\end{definition}

$\meaningof{-} : L \to \mathcal{P}(\pi)$

\begin{mathpar}
  \inferrule* [lab=collection] {} {\meaningof{true} = \pi, \and \meaningof{~E} = \pi \setminus \meaningof{E}, \and \meaningof{E_{1} \& E_{2}} = \meaningof{E_{1}} \cap \meaningof{E_{2}}}
\end{mathpar}

\begin{mathpar}
  \inferrule* [lab=structure] {} {\meaningof{0} = \{ P \in \pi | P \equiv 0 \}, \and \\ \meaningof{E_1 | E_2} = \{ P \in \pi | P \equiv P_{1} | P_{2}, P_{1} \in \meaningof{E_{1}}, P_{2} \in \meaningof{E_2}\} }
\end{mathpar}

\begin{mathpar}
 \inferrule* [lab=behavior] {} {\meaningof{\langle a?b \rangle E} = \{ P \in \pi | P \equiv Q | u?(y)P', \\ \and \\\\ \and \\ \;\;\; u \in \meaningof{a}, \forall z.P'\{z/y\} \in \meaningof{E\{z/b\}}\}, \and \\ \meaningof{a!E} = \{ P \in \pi | P \equiv Q | x!\langle P' \rangle, x \in \meaningof{a} P' \in \meaningof{E}\} }
\end{mathpar}

\begin{mathpar}
 \inferrule* [lab=nominal] {} {\meaningof{\quotep{E}} = \{ \quotep{P} \in \quotep{\pi} | P \in \meaningof{E} \}, \and \meaningof{\quotep{P}} = \{ \quotep{Q} \in \quotep{\pi} | P \equiv Q \} \and \\ \meaningof{@\quotep{E}} = \{ P \in \pi | P \equiv @x, x \in \meaningof{E} \}}
\end{mathpar}

\begin{eqnarray*}
  \\
  \meaningof{-} : TS \to ST
\end{eqnarray*}

\begin{eqnarray*}
  \\
  L : TS \to ST
\end{eqnarray*}

\begin{eqnarray*}
  \\
  P \models E \iff P \in \meaningof{E}
\end{eqnarray*}

\begin{eqnarray*}
  P \approx_{L} Q \iff \forall E \in L. P \models E \iff Q \models E
\end{eqnarray*}

\begin{eqnarray*}
  P \approx_{K} Q
\end{eqnarray*}

\begin{eqnarray*}
  P \approx Q
\end{eqnarray*}

$\approx_{K} = \approx = \approx_{L}$

\subsubsection{Contextual duality}

Note that contexts extend the quotation operation to a family of
operations from processes to names. Given a context, $M$, we can
define a \emph{nominal context}, $\quotep{M}$ by $\quotep{M}[P] :=
\quotep{M[P]}$. To foreshadow what is to come we observe that these
operations enjoy a duality with processes very much like the duality
between vectors and maps from vectors to scalars.

Further, because the calculus is essentially higher-order, we have a
correspondence between contexts and processes. More specifically,
given a name $x$ and a context $M$ we can construct $M^{*}_{x}$ such
that 

\begin{mathpar}
  M^{*}_{x} | \lift{x}{P} \red M[P]
\end{mathpar}

namely,

\begin{mathpar}
  M^{*}_{x} := x?(u).M[\dropn{u}]
\end{mathpar}

The dependence of $M^{*}_{x}$ on a name makes it an abstraction, 

\begin{mathpar}
  M^{*} := (x)x?(u).M[\dropn{u}]
\end{mathpar}

\subsection{Additional notation}

It will sometimes be convenient to denote the process a name
quotes. We already have the notation $x = \quotep{P}$, but it will be
convenient to introduce an alternate notation, $\procn{x}$, when we
want to emphasize the connection to the use of the name. Note that, by
virtue of name equivalence, $\quotep{\procn{x}} \nameeq x$; so, the
notation is consistent with previous definitions.

Further, because names have structure it is possible to effect
substitutions on the basis of that structure. This means we need to
upgrade our notation for substitutions, which we accomplish by
adapting comprehension notation. Thus,

\begin{mathpar}
  P\{ y / x : x \in S \}
\end{mathpar}

is interpreted to mean the process derived from P by replacing (in a
capture-avoiding manner) each occurrence of $x$ in $S$ by $y$. For example,

\begin{mathpar}
  P\{ \quotep{\procn{x}|\procn{x}} / x : x \in \freenames{P} \}
\end{mathpar}

will replace each (occurrence) of a free name $x$ in $P$ by
$\quotep{\procn{x}|\procn{x}}$.

Also, we will avail ourselves of the notation $x^{L}$ and $x^{R}$ to
denote injections of a name into disjoint copies of the name
space. There are numerous ways to accomplish this. One example can be
found in \cite{MeredithR05}. This notation overloads to vectors of
names: $\vec{x}^{\pi} := (x_{i}^{\pi} \; : \; 0 \leq i < |\vec{x}| )$ where $\pi \in \{L,R\}$.

We also use $P^{\Box} := P|\Box$.

In \cite{MeredithR05} an interpretation of the new operator is
given. It turns out that there are several possible interpretations
all enjoying the requisite algebraic properties of the operator (see
\cite{milner91polyadicpi}). We will therefore make liberal use of
$(\nu\; \vec{x})P$.

% subsection the_syntax_and_semantics_of_the_notation_system (end)   

\section{Interpretation of QM}
\subsection{Supporting definitions}
\subsubsection{Multiplication}
\begin{mathpar}
  \quotep{Q} \cdot \quotep{R} := \quotep{Q|R}
  \and \\
  \quotep{Q} \cdot P := P\{ \quotep{Q|R} / \quotep{R} : \quotep{R} \in \freenames{P} \}
\end{mathpar}

\paragraph{Discussion}
The first line needs little explanation. The second line says that
each free name of the process is replaced with the multiplication of
that name by the scalar. Multiplication of a scalar (name) by a state
(process) results in a process all the names of which have been `moved
over' by parallel composition with the process the scalar
quotes. There is a subtlety that the bound names have to be
manipulated so that multiplied names aren't accidentally
captured. There are many ways to achieve this.

\begin{remark}\label{rem:multiplication_identities}
  The reader is invited to verify that for all $x,y,z \in \QProc$ and $P \in \Proc$
  \begin{mathpar}
    x \cdot \quotep{0} \equiv x 
    \and
    x \cdot y \equiv y \cdot x
    \and
    x \cdot (y \cdot z) \equiv (x \cdot y) \cdot z
    \and \\
    \quotep{0} \cdot P \equiv P
    \and \\
    x \cdot (y \cdot P) \equiv (x \cdot y) \cdot P
    \and \\
    x \cdot (P|Q) \equiv (x \cdot P) | (x \cdot Q)
    \and \\    
  \end{mathpar}
\end{remark}

\subsubsection{Tensor product}

We define a tensor product on processes by structural induction.

\paragraph{Tensor of sums} First note that all summations, including
$\pzero$ and sequence, can be written $\Sigma_{i} x_{i}.A_{i} +
\Sigma_{j} x_{j}.C_{j}$, where we have grouped input-guarded processes
together and output-guarded processes together.

Thus, we can define the tensor product of two summations, $N_{1}\otimes N_{2}$, where

\begin{mathpar}
  N_{1} := \Sigma_{i} x_{i}.A_{i} + \Sigma_{j} x_{j}.C_{j}
  \and
  N_{2} := \Sigma_{i'} y_{i'}.B_{i'} + \Sigma_{j'} y_{j'}.D_{j'} 
\end{mathpar}

as follows.

\begin{mathpar}
  \Sigma_{i} x_{i}.A_{i} + \Sigma_{j} x_{j}.C_{j} \otimes \Sigma_{i'}
  y_{i'}.B_{i'} + \Sigma_{j'} y_{j'}.D_{j'} 
  \and \\
  := \; \Sigma_{i} \Sigma_{i'} \quotep{\stackrel{\vee}{x_{i}}| \stackrel{\vee}{y_{i'}}}.(A_{i}\otimes B_{i'}) \; | \; \Sigma_{i'} \Sigma_{i} \quotep{\stackrel{\vee}{y_{i'}}|\stackrel{\vee}{x_{i}}}.(B_{i'}\otimes A_{i})
  \and
  \;\; | \;\; \Sigma_{j} \Sigma_{j'} \quotep{\stackrel{\vee}{x_{j}}|\stackrel{\vee}{y_{j'}}}.(A_{j}\otimes B_{j'}) \; | \; \Sigma_{j'} \Sigma_{j} \quotep{\stackrel{\vee}{y_{j'}}|\stackrel{\vee}{x_{j}}}.(B_{j'}\otimes A_{j})
\end{mathpar}

\begin{remark}
  Do we need to $x^{L}$ and $y^{R}$ for this construction as well?
\end{remark}

\paragraph{Tensor of parallel compositions} Next, we distribute tensor
over par.

\begin{mathpar}
  P_{1}|P_{2} \otimes Q_{1}|Q_{2} := (P_{1} \otimes Q_{1}) | (P_{1}
  \otimes Q_{2}) | (P_{2} \otimes Q_{1}) | (P_{2} \otimes Q_{2})
\end{mathpar}

\paragraph{Tensor with dropped names} We treat tensor of a
process with a dropped name as parallel composition.

\begin{mathpar}
  P \otimes \dropn{x} := P | \dropn{x}
\end{mathpar}

\paragraph{Tensor of agents}

Finally, we need to define tensor on agents. Note that the definition
of tensor on normal products only tensors inputs with inputs and
outputs with outputs. Thus, we only have to define the operation on
``homogeneous'' pairings.

\begin{mathpar}
  (\vec{x})P \otimes (\vec{y})Q
  \and \\
  := (x_{0}^{L}|y_{0}^{R},\ldots,x_{0}^{L}|y_{n}^{R},\ldots,x_{m}^{L}|y_{0}^{R},\ldots,x_{m}^{L}|y_{n}^R)(P\{ \vec{x}^{L}/\vec{x}\} \otimes Q \{ \vec{y}^{R}/\vec{y}\})
  \and \\
  \clift{\vec{P}} \otimes \clift{\vec{Q}}
  \and \\
  := \clift{P_{0}\otimes Q_{0},\ldots,P_{0}\otimes Q_{n},\ldots,P_{m}\otimes Q_{0},\ldots,P_{m}\otimes Q_{n}}
\end{mathpar}

\begin{remark}
  Observe that arities of tensored abstractions matches arities of
  tensored concretions if the original arities matched. Note also that
  the length of the arities corresponds to the increase in dimension
  we see in ordinary vector space tensor product.
\end{remark}

\begin{remark}
  Operationally, this definition distributes the tensor down to
  components ``linked'' by summation. Tensor over summation is
  intriguing in that it mixes names. Moreover, as a consequence of the
  way it mixes names we have the identities for all $x \in \QProc$ and
  $P,Q \in \Proc$

  \begin{mathpar}
    (x \cdot P) \otimes Q \equiv x \cdot (P \otimes Q) \equiv P \otimes (x \cdot Q)
    \and
    P \otimes \pzero \equiv P
  \end{mathpar}

  that the reader is invited to verify.
\end{remark}

\subsubsection{Annihilation}
\begin{mathpar}
  P^{\perp} := \{ Q | \forall R. P|Q \red^{*} R \Rightarrow R \red^{*} \pzero \}
  \and \\
  P^{\underline{\perp}} := \Sigma_{Q \in P^{\perp}} \quotep{Q}?(y).(\dropn{y}|Q) | \Sigma_{Q \in P^{\perp}} \quotep{Q}\clift{\Box}
\end{mathpar}

\paragraph{Discussion} The reader will note that $P^{\perp}$ is a
\emph{set} of processes, while $P^{\underline{\perp}}$ is a
\emph{context}. We call the set $P^{\perp}$ the \emph{annihilators} of
$P$. The parallel composition of a process in the annihilators of $P$
with $P$ will result in a process, the state space of which has all
paths eventually leading to $\pzero$. Execution may endure loops; but
under reasonable conditions of fairness (naturally guaranteed under
most notions of bisimulation) such a composite process cannot get
stuck in such a loop and will, eventually pop out and terminate.

The context $P^{\underline{\perp}}$ is ready and willing to ``take the
$P$ out of'' the process to which it is applied. It will effectively
transmit the code of the process to which it is applied to one of the
annihilators and run the process against it.

\subsubsection{Evaluation}
We fix $M$ a domain of fully abstract interpretation with an equality
coincident with bisimulation. We take $\meaningof{\cdot} : \Proc \to
M$ to be the map interpreting processes and $\nmeaningof{\cdot} : \M
\to Proc$ to be the map running the other way. Then we define

\begin{mathpar}
  \int P := \nmeaningof{\meaningof{P}}
\end{mathpar}

\paragraph{Discussion}
There are many fully abstract interpretations of Milner's
$\pi$-calculus. Any of them can be used as a basis for interpreting
the reflective calculus here. Equipped with such a domain it is
largely a matter of grinding through to check that the Yoneda
construction for the normalization-by-evaluation program can be
extended to this setting.

\begin{remark}
  The reader is invited to verify that $\int (P^{\underline{\perp}}[P]) = 0$.
\end{remark}

\subsection{Quantum mechanics}

Table \ref{tbl:core_qm_op_defns} gives the core operational definitions

\begin{table}[htp]\label{tbl:core_qm_op_defns}
  \center{
    \fbox{
      \begin{tabular}{c|c}
        quantum mechanics & process calculus \\
        \hline
        scalar & $x := \quotep{P}$ \\
        state vector & $\state{P} := P$ \\
        dual & $\state{P}^{*} := \event{P^{\underline{\perp}}} := \quotep{P^{\underline{\perp}}}[-]$ \\
        matrix & $ \Sigma_{\alpha} \state{P_{\alpha}}x_{\alpha}\event{Q_{\alpha}}$ \\
        vector addition & $\state{P} + \state{Q} := \state{P | Q}$ \\
        tensor product & $\state{P} \otimes \state{Q} := \state{P \otimes Q}$ \\
        inner product & $\innerprod{P}{Q} := \quotep{\int P^{\underline{\perp}}[Q]}$ \\
      \end{tabular}
    }
  }
  \caption{QM - operational definitions}
\end{table}

where

\begin{mathpar}
  \prmatrix{P}{Q} := \fprmatrix{P}{\quotep{\pzero}}{Q}
  \and
  \fprmatrix{P}{x}{Q} := (\state{P},x,\event{Q})
  \and
  (\fprmatrix{P}{x}{Q})(\state{R}) := x \cdot \innerprod{Q}{R} \cdot \state{P}
  \and
  (\fprmatrix{P}{x}{Q})(\event{R}) := x \cdot \innerprod{R}{P} \cdot \event{Q}
\end{mathpar}

\paragraph{Discussion}
As promised: vectors (aka states) are represented as processes; duals
as contextual duals; inner product definition should be compared with
standard inner product definition for ....

\begin{remark}
  Assuming $\int (P^{\underline{\perp}}[P]) = 0$, the reader is
  invited to verify that $(\fprmatrix{P}{x}{P})(\state{P}) = x \cdot \state{P}$.
\end{remark}

\begin{remark}
  The reader is invited to verify that $\innerprod{P}{Q}$ could
  equally well have been written $\quotep{\int \stackrel{\vee}{x}}$
  where $x = \event{P^{\underline{\perp}}}(Q)$.

  One of the motivations for this remark is that there is another way
  to factor these operations. We could package up evaluation in the dual:

  \begin{mathpar}
    \state{P}^{*} := \event{\int P^{\underline{\perp}}} := \quotep{\int P^{\underline{\perp}}}[-]
  \end{mathpar}

  and then have inner product defined by
  
  \begin{mathpar}
    \innerprod{P}{Q} := \event{P}(Q)
  \end{mathpar}

  Hopefully, experience with the calculations will provide guidance on
  the best factoring.
\end{remark}

\begin{remark}
  Assuming $\int (P^{\underline{\perp}}[P]) = 0$, the reader is
  invited to verify that $\forall P,Q. (\prmatrix{0}{Q})(\state{0}) =
  \state{0}$ and dually $(\prmatrix{P}{0})(\event{0}) = \event{0}$.
\end{remark}

\begin{remark}
  i'm a little worried that i don't (yet) have proper support for
  complex conjugacy. But, the observation above may give us a
  clue. According to Abramsky, it must be the case that the scalars
  are iso to the homset of the identity for the tensor -- which the
  observation above characterizes. 

  For now, we will simply bookmark the notion with $\overline{x}$.
\end{remark}

\subsubsection{Adjointness}

We need to give a definition of $(\cdot)^{\dagger}$ for matrices. The
obvious candidate definition is
\begin{mathpar}
(\Sigma_{\alpha}\fprmatrix{P_{\alpha}}{x_{\alpha}}{Q_{\alpha}})^{\dagger}
= \Sigma_{\alpha}\fprmatrix{(Q_{\alpha}^{\underline{\perp}})^{*}}{\overline{x}_{\alpha}}{P_{\alpha}^{\underline{\perp}}} 
\end{mathpar}

But, $(Q_{\alpha}^{\underline{\perp}})^{*}$ requires a name along
which to communicate the process to achieve the context application.

\subsubsection{Basis for a basis}
If processes label states and ``addition'' of states (a.k.a. vector
addition) is interpreted as parallel composition, what corresponds to
notions of linear independence and basis? Here, we recall that Yoshida
has developed a set of \emph{combinators} for an asynchronous verison
of Milner's $\pi$-calculus. These are a finite set of processes such
any process can be expressed as parallel composition of these
combinators together with liberal uses of the new operator and
replication. We can simply give a translation of these into the
present calculus and have reasonable expectation that the property
carries over. That is, that the resultant set allows to express all
processes via parallel composition. Note, however, that there is no
new operator or replication in this calculus. As a result, we expect
that the corresponding set is actually infinite. That is, we expect
that the space is actually infinite dimensional.

\begin{remark}
  The attentive reader may be a bit concerned. Certainly, the
  collection $S$, $K$ and $I$ is a finite set of
  combinators. Shouldn't we expect to see a finite set of combinators
  for an effectively equivalent system? i am very sympathetic to this
  critique and feel it warrants full attention. On the other hand, i
  also have in mind the following analogy. The natural numbers, as a
  monoid under addition, has exactly $1$ generator, while the natural
  numbers, as a monoid under multiplication, has countably many
  generators (the primes). We observe that the application of the
  lambda calculus is much less resource sensitive than the parallel
  composition of the $\pi$-calculus. Could it be the case that we have
  an analogy of the form
  
  \begin{mathpar}
    m + n : MN :: m*n : M|N
  \end{mathpar}

  giving a similar blow up in the set of ``primes''?  This is such a
  wonderful thought that, even if it's not true, i think it's worth
  writing down.
\end{remark}
 

\documentclass[12pt]{llncs}
%\documentclass{jktr}

\usepackage[pdftex]{hyperref}                   
\usepackage {listings}
\usepackage {mathpartir}
\usepackage{bcprules}
%\usepackage{listings}
                       
\usepackage{graphicx} 
%\usepackage[margins=2.5cm,nohead,nofoot]{geometry}
%\usepackage{geometry}
\usepackage{amsfonts}
\usepackage{amstext}
\usepackage{latexsym}
\usepackage{amssymb}
\usepackage{color}


%\include{myPreamble}
\documentclass[12pt]{llncs}
%\documentclass{jktr}

\usepackage[pdftex]{hyperref}                   
\usepackage {listings}
\usepackage {mathpartir}
\usepackage{bcprules}
%\usepackage{listings}
                       
\usepackage{graphicx} 
%\usepackage[margins=2.5cm,nohead,nofoot]{geometry}
%\usepackage{geometry}
\usepackage{amsfonts}
\usepackage{amstext}
\usepackage{latexsym}
\usepackage{amssymb}
\usepackage{color}


%\include{myPreamble}
\include{qm2pi.local} 

%\ifpdf
%\usepackage[pdftex]{graphicx}
%\else
%\usepackage{graphicx}
%\fi

 % \ifpdf
%  \usepackage{pdfsync}
%  \if


%\title{Brief Article}
%\author{David F. Snyder}
%\author{L.G. Meredith}

%\address{Dept. of Math., Texas State University--San Marcos, San Marcos, TX 78666}
       
\pagestyle{empty}


\begin{document}

\lstset{language=[Objective]Caml,frame=shadowbox}

\input{qm2pi.front}

% section front matter (end)

\input{qm2pi.intro} 
 
% section introduction (end)

% \input{qm2pi.knotations} 

% section notation (end)

\input{qm2pi.process.calculi} 

% section concurrent_process_calculi_and_spatial_logics_ (end)
    
%\input{qm2pi.knots2pi} 

%\input{qm2pi.trefoil} 

%\input{qm2pi.mainthm} 

% subsection basic_interpretation (end)

%\input{qm2pi.rho.presentation} 
\subsection{The syntax and semantics of the notation system}\label{sub:the_syntax_and_semantics_of_the_notation_system} % (fold)

We now summarize a technical presentation of the calculus that
embodies our theory of dynamics. The typical presentation of such a
calculus follows the style of giving generators and relations on
them. The grammar, below, describing term constructors, freely
generates the set of processes, $\Proc$. This set is then quotiented
by a relation known as structural congruence and it is over this set
that the notion of dynamics is expressed. This presentation is
essentially that of \cite{MeredithR05} with the addition of
polyadicity and summation. For readability we have relegated some of
the technical subtleties to an appendix.

\subsubsection{Process grammar}\label{subsub:process_grammar}

\begin{mathpar}
  \inferrule* [lab=synchronization] {} {{M} \bc \pzero \;|\; x?F \;|\; x!C }
  \and
  \inferrule* [lab=abstraction] {} {{F} \bc (x)P}
  \and
  \inferrule* [lab=concretion] {} {{C} \bc \langle Q \rangle}
  \and
  \inferrule* [lab=process] {} {{P,Q} \bc M \;| \;P|Q \;|\; @{x}}
  \and
  \inferrule* [lab=name] {} {{x} \bc \quotep{P}}
\end{mathpar} 

Note that $\vec{x}$ (resp. $\vec{P}$) denotes a vector of names
(resp. processes) of length $|\vec{x}|$ (resp. $|\vec{P}|$). We adopt
the following useful abbreviations.

\begin{mathpar}
   x?(\vec{y}).P := x.(\vec{y})P \and  x\clift{\vec{P}} := x.\clift{\vec{P}}
   \and x!(y) := \lift{x}{\dropn{y}}
   \and \Pi_{i=0}^{n-1}P_i := P_0 | \ldots | P_{n-1}
\end{mathpar}

\subsubsection{Structural congruence}

\paragraph{Free and bound names and alpha-equivalence.} At the
core of structural equivalence is alpha-equivalence which identifies
process that are the same up to a change of variable. Formally, we
recognize the distinction between free and bound names. The free names
of a process, $\freenames{P}$, may be calculated recursively as
follows:

\begin{mathpar}
\freenames{\pzero} := \emptyset
  \and \\
  \freenames{x?(y).P} := \{ x \} \cup (\freenames{P} \setminus \{ y \})
  \and 
  \freenames{x!\langle P \rangle} := \{ x \} \cup \{ P \} 
  \and \\
  \freenames{P|Q} := \freenames{P} \cup \freenames{Q}
  \and \\
  \freenames{@{x}} := \{ x \}
\end{mathpar}

$\pi$
$\quotep{\pi}$

$\freenames{-} : \pi \to \mathcal{P}(\quotep{\pi})$

\begin{eqnarray*}
  \freenames{\pzero} & := & \emptyset \\
  \freenames{x?(y).P} & := & \{ x \} \cup (\freenames{P} \setminus \{ y \}) \\
  \freenames{x!\langle P \rangle} & := & \{ x \} \cup \{ P \} \\
  \freenames{P|Q} & := & \freenames{P} \cup \freenames{Q} \\
  \freenames{\dropn{x}} & := & \{ x \}
\end{eqnarray*}

The bound names of a process, $\boundnames{P}$, are those names occurring in $P$
that are not free. For example, in $x?(y).0$, the name $x$ is free, while $y$ is bound.

\begin{mathpar}
  \inferrule* [lab=monoidal-laws] {} { P|Q \equiv Q|P \and P|0 \equiv P \and P|(Q|R) \equiv (P|Q)|R }
\end{mathpar}

\begin{mathpar}
  \inferrule* [lab=alpha-equivalence] {} { (x)P \equiv (y)P\{y/x\} \and y \not\in \freenames{P} }
\end{mathpar}

\begin{definition}
Then two processes, $P,Q$, are alpha-equivalent if $P = Q\{\vec{y}/\vec{x}\}$ for
some $\vec{x} \in \boundnames{Q},\vec{y} \in \boundnames{P}$, where $Q\{\vec{y}/\vec{x}\}$
denotes the capture-avoiding substitution of $\vec{y}$ for $\vec{x}$ in $Q$.
\end{definition}

\begin{definition}
  The {\em structural congruence} \cite{SangiorgiWalker} , $\equiv$,
  between processes is the least congruence containing
  alpha-equivalence, satisfying the abelian monoid laws
  (associativity, commutativity and $\pzero$ as identity) for parallel
  composition $|$ and for summation $+$.
\end{definition}

\subsection{Name equivalence}

We take name equivalence, written $\nameeq$, to be the smallest
equivalence relation generated by the following rules.

\begin{mathpar}
\inferrule*[lab=Quote-drop]
{ }
{ \quotep{@{x}} \nameeq x }

\inferrule*[lab=Struct-equiv]
{ P \scong Q }
{ \quotep{P} \nameeq \quotep{Q} }
\end{mathpar}

The astute reader will have noticed that the mutual recursion of names
and processes imposes a mutual recursion on alpha-equivalence and
structural equivalence via name-equivalence. Fortunately, all of this
works out pleasantly and we may calculate in the natural way, free of
concern. The reader interested in the details is referred to the
appendix \ref{appendix:rho_details}.

\subsection{Substitution}

We use $\Proc$ for the set of processes, $\QProc$ for the set of
names, and $\id{\{}\vec{y} / \vec{x} \id{\}}$ to denote partial maps,
$s : \QProc \rightarrow \QProc$. A map, $s$ lifts, uniquely, to a map
on process terms, $\widehat{s} : \Proc \rightarrow \Proc$ by the
following equations.

\begin{mathpar}
  (0) \psubstp{Q}{P} := 0 \\
  (R \juxtap S) \psubstp{Q}{P}
  :=    
  (R)\psubstp{Q}{P} \juxtap (S) \psubstp{Q}{P} \\
  (x?(y).R) \psubstp{Q}{P}    
  :=    
  (x)\substp{Q}{P} (z)\concat( (R \psubstn{z}{y}) \psubstp{Q}{P} ) \\
  (\lift{x}{R}) \psubstp{Q}{P}  
  :=
  \lift{(x)\substp{Q}{P}}{ R \psubstp{Q}{P} } \\
%   (\dropn{x})  \psubstp{Q}{P}       
%   := 
%   \left\{ 
%     \begin{array}{ccc} 
%       \dropn{\quotep{Q}} & & x \nameeq \quotep{P} \\
%       \dropn{x} & & otherwise \\
%     \end{array}
%   \right. 
  (\dropn{x})  \psubstp{Q}{P}       
  := 
  \left\{ 
    \begin{array}{ccc} 
      Q & & x \nameeq \quotep{P} \\
      \dropn{x} & & otherwise \\
    \end{array}
  \right.
\end{mathpar}
 

where

\begin{eqnarray}
  (x)\id{\{} \lpquote Q \rpquote / \lpquote P \rpquote \id{\}}            = 
  \left\{ 
    \begin{array}{ccc}
      \lpquote Q \rpquote & & x \nameeq \lpquote P \rpquote \\
      x & & otherwise \\
    \end{array}
  \right. \nonumber
\end{eqnarray}

and $z$ is chosen distinct from $\quotep{P}$, $\quotep{Q}$, the free
names in $Q$, and all the names in $R$. Our $\alpha$-equivalence will
be built in the standard way from this substitution.

\begin{remark}\label{rem:no_self_referential_names}
  One consequence of these definitions is that $\forall P. \quotep{P}
  \not\in \freenames{P}$.
\end{remark}

\subsection{ Dynamic quote: an example }

Anticipating something of what's to come, consider applying the
substitution, $\widehat{\id{\{}u / z \id{\}}}$, to the following pair
of processes, $\lift{w}{y!(z)}$ and $w[ \lpquote y!(z) \rpquote ]$.

\begin{eqnarray}
	\lift{w}{y!(z)}\widehat{\id{\{}u / z \id{\}}}
		& = &
		\lift{w}{y!(u)} \nonumber\\
	w[ \lpquote y!(z) \rpquote ] \widehat{ \id{\{}u / z \id{\}} }
		& = &
		w[ \lpquote y!(z) \rpquote ] \nonumber
\end{eqnarray}

Because the body of the process between quotes is impervious to
substitution, we get radically different answers. In fact, by
examining the first process in an input context,
e.g. $x?(z).\lift{w}{y!(z)}$, we see that the process under the lift
operator may be shaped by prefixed inputs binding a name inside it. In
this sense, the lift operator will be seen as a way to dynamically
construct processes before reifying them as names.

Finally equipped with these standard features we can present the
dynamics of the calculus.

\subsubsection{Operational semantics} 

Finally, we introduce the computational dynamics. What marks these
algebras as distinct from other more traditionally studied algebraic
structures, e.g. vector spaces or polynomial rings, is the manner in
which dynamics is captured. In traditional structures, dynamics is typically
expressed through morphisms between such structures, as in linear maps
between vector spaces or morphisms between rings. In algebras
associated with the semantics of computation, the dynamics is
expressed as part of the algebraic structure itself, through a
reduction reduction relation typically denoted by $\red$. Below, we
give a recursive presentation of this relation for the calculus used
in the encoding.

$\red \subseteq \pi \times \pi$
$\red : \pi \to \mathcal{P}(\pi)$

\begin{mathpar}
  \inferrule* [lab=Comm] { \textsf{match}( x_{src}, x_{trgt} ) } { x_{trgt}?(y)P \; | \; x_{src}!\langle {Q} \rangle \red P\{\quotep{Q}/y}\} }
  \and \\
  \inferrule* [lab=Par] {{P} \red {P}'} {{{P} | {Q}} \red {{P}' | {Q}}}
  \and
  \inferrule* [lab=Equiv]{{{P} \scong {P}'} \andalso {{P}' \red {Q}'} \andalso {{Q}' \scong {Q}}}{{P} \red {Q}}
\end{mathpar}

\begin{eqnarray*}
  match_{\equiv} (\quotep{P},\quotep{Q}) & := & P \equiv Q \\
  match_{\dagger}(\quotep{P},\quotep{Q}) & := & \forall R. P|Q \red^{*} R => R \red^{*} 0 \\
  match_{K}(\quotep{P},\quotep{Q}) & := & K \mbox{ for some context } K
\end{eqnarray*}

$u?(x)P | u!\langle Q \rangle \red P\{\quotep{Q}/x\}$

%We write $\wred$ for $\red^*$, and $P\red$ if $\exists Q $ such that $ P \red Q$.
We write $P\red$ if $\exists Q $ such that $ P \red Q$ and $P\not\red$, otherwise.

\section{Replication}

As mentioned before, it is known that replication (and hence
recursion) can be implemented in a higher-order process algebra
\cite{SangiorgiWalker}. As our first example of calculation with the
machinery thus far presented we give the construction explicitly in
the {\rhoc}.

\begin{eqnarray}
	D_{x} & := & \prefix{x}{y}{(\binpar{\outputp{x}{y}}{@{y}})} \nonumber\\
	\bangp_{x}{P} & := & \binpar{{x}!\langle{\binpar{D_{x}}{P}}\rangle}{D_{x}} \nonumber
\end{eqnarray}

\begin{eqnarray}
	\bangp_{x}{P} & & \nonumber\\
	=
	& {x}!\langle{(\prefix{x}{y}{(\outputp{x}{y} | @{y})) | P}}\rangle 
	      | \prefix{x}{y}{(\outputp{x}{y} | @{y})} & \nonumber\\
	\red
	& (\outputp{x}{y} | @{y})\substn{\quotep{(\prefix{x}{y}{(@{y} | \outputp{x}{y})) | P}}}{y} & \nonumber\\
	=
	& \outputp{x}{\quotep{(\prefix{x}{y}{(\outputp{x}{y} | @{y})) | P}}}
	  | {(\prefix{x}{y}{(\outputp{x}{y} | @{y})) | P}} & \nonumber\\
	\red
	& \ldots & \nonumber\\
	\red^*
	& P | P | \ldots & \nonumber
\end{eqnarray}

Of course, this encoding, as an implementation, runs away, unfolding
$\bangp{P}$ eagerly. A lazier and more implementable replication
operator, restricted to input-guarded processes, may be obtained as follows.

\begin{eqnarray}
\bangp{\prefix{u}{v}{P}} 
	:= 
	\binpar{\lift{x}{\prefix{u}{v}{(\binpar{D(x)}{P})}}}{D(x)} \nonumber
\end{eqnarray}

\begin{remark}
  Note that the lazier definition still does not deal with summation
  or mixed summation (i.e. sums over input and output). The reader is
  invited to construct definitions of replication that deal with these
  features. 

  Further, the definitions are parameterized in a name, $x$. Can you,
  gentle reader, make a definition that eliminates this parameter and
  guarantees no accidental interaction between the replication
  machinery and the process being replicated -- i.e. no accidental
  sharing of names used by the process to get its work done and the
  name(s) used by the replication to effect copying. This latter
  revision of the definition of replication is crucial to obtaining
  the expected identity $!!P \sim !P$.
\end{remark}

\begin{remark}\label{rem:paradoxical_combinator}
  The reader familiar with the lambda calculus will have noticed the
  similarity between $D$ and the paradoxical combinator.

  [Ed. note: the existence of this seems to suggest we have to be more
  restrictive on the set of processes and names we admit if we are to
  support no-cloning.]
\end{remark}

\subsubsection{Bisimulation}

The computational dynamics gives rise to another kind of equivalence,
the equivalence of computational behavior. As previously mentioned
this is typically captured \emph{via} some form of bisimulation.

% The notion we use in this paper is weak barbed bisimulation
% \cite{milner91polyadicpi}.

The notion we use in this paper is derived from weak barbed
bisimulation \cite{milner91polyadicpi}. 

\begin{definition}
An \emph{observation relation}, $\downarrow_{\mathcal N}$, over a set
of names, $\mathcal N$, is the smallest relation satisfying the rules
below.

\infrule[Out-barb]{y \in {\mathcal N}, \; x \nameeq y}
		  {\outputp{x}{v} \downarrow_{\mathcal N} x}
\infrule[Par-barb]{\mbox{$P\downarrow_{\mathcal N} x$ or $Q\downarrow_{\mathcal N} x$}}
		  {\binpar{P}{Q} \downarrow_{\mathcal N} x}

We write $P \Downarrow_{\mathcal N} x$ if there is $Q$ such that 
$P \wred Q$ and $Q \downarrow_{\mathcal N} x$.
\end{definition}

\begin{definition}
%\label{def.bbisim}
An  ${\mathcal N}$-\emph{barbed bisimulation} over a set of names, ${\mathcal N}$, is a symmetric binary relation 
${\mathcal S}_{\mathcal N}$ between agents such that $P\rel{S}_{\mathcal N}Q$ implies:
\begin{enumerate}
\item If $P \red P'$ then $Q \wred Q'$ and $P'\rel{S}_{\mathcal N} Q'$.
\item If $P\downarrow_{\mathcal N} x$, then $Q\Downarrow_{\mathcal N} x$.
\end{enumerate}
$P$ is ${\mathcal N}$-barbed bisimilar to $Q$, written
$P \wbbisim_{\mathcal N} Q$, if $P \rel{S}_{\mathcal N} Q$ for some ${\mathcal N}$-barbed bisimulation ${\mathcal S}_{\mathcal N}$.
\end{definition}

$\mathcal{R} \subseteq \pi \times \pi$

$P \mathcal{R} Q => \forall P'. P \red P' \Rightarrow \exists Q'. Q \red Q', P' \mathcal{R} Q'$

$P \vdash x \Rightarrow Q \vdash x$

\begin{mathpar}
  \inferrule*[lab=Out-barb]{x \nameeq y}{{y}!\langle{Q}\rangle \vdash x}
  \and
  \inferrule*[lab=Par-barb]{\mbox{$P\vdash x$ or $Q\vdash x$}}{\binpar{P}{Q} \vdash x}
\end{mathpar}

\subsubsection{Contexts}

One of the principle advantages of computational calculi like the
$\pi$-calculus is a well-defined notion of context,
contextual-equivalence and a correlation between
contextual-equivalence and notions of bisimulation. The notion of
context allows the decomposition of a process into (sub-)process and
its syntactic environment, its context. Thus, a context may be
thought of as a process with a ``hole'' (written $\Box$) in it. The
application of a context $M$ to a process $P$, written $M[P]$, is
tantamount to filling the hole in $M$ with $P$. In this paper we do
not need the full weight of this theory, but do make use of the notion
of context in the proof the main theorem. 

\begin{mathpar}
  \inferrule* [lab=summation] {} {{M_{M},M_{N}} \bc \Box \;|\; x.M_{A} \;|\; M_{M}+M_{N}}
  \and
  \inferrule* [lab=agent] {} {{M_{A}} \bc (\vec{x})M_{P} \;| \; \clift{P_0,\ldots,M_{P},\ldots,P_N}}
  \and \\
  \inferrule* [lab=process] {} {{M_{P}} \bc M_{N} \;| \;P|M_{P} }
\end{mathpar} 

\begin{mathpar}
  \inferrule* [lab=sychronization] {} {M_{N} \bc \Box \;|\; x?M_{F} \;|\; x!M_{C}}
  \and
  \inferrule* [lab=abstraction] {} {{M_{F}} \bc (x)M_{P} }
  \and
  \inferrule* [lab=concretion] {} {{M_{C}} \bc \langle M_{P} \rangle }
  \and \\
  \inferrule* [lab=process] {} {{M_{P}} \bc M_{N} \;| \;P|M_{P} }
\end{mathpar}

\begin{definition}[contextual application] Given a context $M$, and
  process $P$, we define the \emph{contextual application}, $M[P] :=
  M\{P/\Box\}$. That is, the contextual application of M to P is the
  substitution of $P$ for $\Box$ in $M$.
\end{definition}

$\meaningof{-} : L \to \mathcal{P}(\pi)$

\begin{mathpar}
  \inferrule* [lab=collection] {} {\meaningof{true} = \pi, \and \meaningof{~E} = \pi \setminus \meaningof{E}, \and \meaningof{E_{1} \& E_{2}} = \meaningof{E_{1}} \cap \meaningof{E_{2}}}
\end{mathpar}

\begin{mathpar}
  \inferrule* [lab=structure] {} {\meaningof{0} = \{ P \in \pi | P \equiv 0 \}, \and \\ \meaningof{E_1 | E_2} = \{ P \in \pi | P \equiv P_{1} | P_{2}, P_{1} \in \meaningof{E_{1}}, P_{2} \in \meaningof{E_2}\} }
\end{mathpar}

\begin{mathpar}
 \inferrule* [lab=behavior] {} {\meaningof{\langle a?b \rangle E} = \{ P \in \pi | P \equiv Q | u?(y)P', \\ \and \\\\ \and \\ \;\;\; u \in \meaningof{a}, \forall z.P'\{z/y\} \in \meaningof{E\{z/b\}}\}, \and \\ \meaningof{a!E} = \{ P \in \pi | P \equiv Q | x!\langle P' \rangle, x \in \meaningof{a} P' \in \meaningof{E}\} }
\end{mathpar}

\begin{mathpar}
 \inferrule* [lab=nominal] {} {\meaningof{\quotep{E}} = \{ \quotep{P} \in \quotep{\pi} | P \in \meaningof{E} \}, \and \meaningof{\quotep{P}} = \{ \quotep{Q} \in \quotep{\pi} | P \equiv Q \} \and \\ \meaningof{@\quotep{E}} = \{ P \in \pi | P \equiv @x, x \in \meaningof{E} \}}
\end{mathpar}

\begin{eqnarray*}
  \\
  \meaningof{-} : TS \to ST
\end{eqnarray*}

\begin{eqnarray*}
  \\
  L : TS \to ST
\end{eqnarray*}

\begin{eqnarray*}
  \\
  P \models E \iff P \in \meaningof{E}
\end{eqnarray*}

\begin{eqnarray*}
  P \approx_{L} Q \iff \forall E \in L. P \models E \iff Q \models E
\end{eqnarray*}

\begin{eqnarray*}
  P \approx_{K} Q
\end{eqnarray*}

\begin{eqnarray*}
  P \approx Q
\end{eqnarray*}

$\approx_{K} = \approx = \approx_{L}$

\subsubsection{Contextual duality}

Note that contexts extend the quotation operation to a family of
operations from processes to names. Given a context, $M$, we can
define a \emph{nominal context}, $\quotep{M}$ by $\quotep{M}[P] :=
\quotep{M[P]}$. To foreshadow what is to come we observe that these
operations enjoy a duality with processes very much like the duality
between vectors and maps from vectors to scalars.

Further, because the calculus is essentially higher-order, we have a
correspondence between contexts and processes. More specifically,
given a name $x$ and a context $M$ we can construct $M^{*}_{x}$ such
that 

\begin{mathpar}
  M^{*}_{x} | \lift{x}{P} \red M[P]
\end{mathpar}

namely,

\begin{mathpar}
  M^{*}_{x} := x?(u).M[\dropn{u}]
\end{mathpar}

The dependence of $M^{*}_{x}$ on a name makes it an abstraction, 

\begin{mathpar}
  M^{*} := (x)x?(u).M[\dropn{u}]
\end{mathpar}

\subsection{Additional notation}

It will sometimes be convenient to denote the process a name
quotes. We already have the notation $x = \quotep{P}$, but it will be
convenient to introduce an alternate notation, $\procn{x}$, when we
want to emphasize the connection to the use of the name. Note that, by
virtue of name equivalence, $\quotep{\procn{x}} \nameeq x$; so, the
notation is consistent with previous definitions.

Further, because names have structure it is possible to effect
substitutions on the basis of that structure. This means we need to
upgrade our notation for substitutions, which we accomplish by
adapting comprehension notation. Thus,

\begin{mathpar}
  P\{ y / x : x \in S \}
\end{mathpar}

is interpreted to mean the process derived from P by replacing (in a
capture-avoiding manner) each occurrence of $x$ in $S$ by $y$. For example,

\begin{mathpar}
  P\{ \quotep{\procn{x}|\procn{x}} / x : x \in \freenames{P} \}
\end{mathpar}

will replace each (occurrence) of a free name $x$ in $P$ by
$\quotep{\procn{x}|\procn{x}}$.

Also, we will avail ourselves of the notation $x^{L}$ and $x^{R}$ to
denote injections of a name into disjoint copies of the name
space. There are numerous ways to accomplish this. One example can be
found in \cite{MeredithR05}. This notation overloads to vectors of
names: $\vec{x}^{\pi} := (x_{i}^{\pi} \; : \; 0 \leq i < |\vec{x}| )$ where $\pi \in \{L,R\}$.

We also use $P^{\Box} := P|\Box$.

In \cite{MeredithR05} an interpretation of the new operator is
given. It turns out that there are several possible interpretations
all enjoying the requisite algebraic properties of the operator (see
\cite{milner91polyadicpi}). We will therefore make liberal use of
$(\nu\; \vec{x})P$.

% subsection the_syntax_and_semantics_of_the_notation_system (end)   

\input{qm2pi.qmops} 

\input{qm2pi.sterngerlach} 

\input{qm2pi.metric} 

% section concurrent_process_calculi (end)

%\input{qm2pi.proofsketch}

% section proof sketch (end)

%\input{qm2pi.slviaknots} 

% section spatial logic via knots (end)

\input{qm2pi.conclusion}

% section conclusion (end)

%\input{qm2pi.dtcodes} 

% section wiring algorithm (end)

\input{qm2pi.ack} 

% section acknowledgments (end)

\newpage


\bibliographystyle{plain}   
\bibliography{../../biblios/main.bib}

\input{qm2pi.rhodetails}

\end{document}

 

%\ifpdf
%\usepackage[pdftex]{graphicx}
%\else
%\usepackage{graphicx}
%\fi

 % \ifpdf
%  \usepackage{pdfsync}
%  \if


%\title{Brief Article}
%\author{David F. Snyder}
%\author{L.G. Meredith}

%\address{Dept. of Math., Texas State University--San Marcos, San Marcos, TX 78666}
       
\pagestyle{empty}


\begin{document}

\lstset{language=[Objective]Caml,frame=shadowbox}

\documentclass[12pt]{llncs}
%\documentclass{jktr}

\usepackage[pdftex]{hyperref}                   
\usepackage {listings}
\usepackage {mathpartir}
\usepackage{bcprules}
%\usepackage{listings}
                       
\usepackage{graphicx} 
%\usepackage[margins=2.5cm,nohead,nofoot]{geometry}
%\usepackage{geometry}
\usepackage{amsfonts}
\usepackage{amstext}
\usepackage{latexsym}
\usepackage{amssymb}
\usepackage{color}


%\include{myPreamble}
\include{qm2pi.local} 

%\ifpdf
%\usepackage[pdftex]{graphicx}
%\else
%\usepackage{graphicx}
%\fi

 % \ifpdf
%  \usepackage{pdfsync}
%  \if


%\title{Brief Article}
%\author{David F. Snyder}
%\author{L.G. Meredith}

%\address{Dept. of Math., Texas State University--San Marcos, San Marcos, TX 78666}
       
\pagestyle{empty}


\begin{document}

\lstset{language=[Objective]Caml,frame=shadowbox}

\input{qm2pi.front}

% section front matter (end)

\input{qm2pi.intro} 
 
% section introduction (end)

% \input{qm2pi.knotations} 

% section notation (end)

\input{qm2pi.process.calculi} 

% section concurrent_process_calculi_and_spatial_logics_ (end)
    
%\input{qm2pi.knots2pi} 

%\input{qm2pi.trefoil} 

%\input{qm2pi.mainthm} 

% subsection basic_interpretation (end)

%\input{qm2pi.rho.presentation} 
\subsection{The syntax and semantics of the notation system}\label{sub:the_syntax_and_semantics_of_the_notation_system} % (fold)

We now summarize a technical presentation of the calculus that
embodies our theory of dynamics. The typical presentation of such a
calculus follows the style of giving generators and relations on
them. The grammar, below, describing term constructors, freely
generates the set of processes, $\Proc$. This set is then quotiented
by a relation known as structural congruence and it is over this set
that the notion of dynamics is expressed. This presentation is
essentially that of \cite{MeredithR05} with the addition of
polyadicity and summation. For readability we have relegated some of
the technical subtleties to an appendix.

\subsubsection{Process grammar}\label{subsub:process_grammar}

\begin{mathpar}
  \inferrule* [lab=synchronization] {} {{M} \bc \pzero \;|\; x?F \;|\; x!C }
  \and
  \inferrule* [lab=abstraction] {} {{F} \bc (x)P}
  \and
  \inferrule* [lab=concretion] {} {{C} \bc \langle Q \rangle}
  \and
  \inferrule* [lab=process] {} {{P,Q} \bc M \;| \;P|Q \;|\; @{x}}
  \and
  \inferrule* [lab=name] {} {{x} \bc \quotep{P}}
\end{mathpar} 

Note that $\vec{x}$ (resp. $\vec{P}$) denotes a vector of names
(resp. processes) of length $|\vec{x}|$ (resp. $|\vec{P}|$). We adopt
the following useful abbreviations.

\begin{mathpar}
   x?(\vec{y}).P := x.(\vec{y})P \and  x\clift{\vec{P}} := x.\clift{\vec{P}}
   \and x!(y) := \lift{x}{\dropn{y}}
   \and \Pi_{i=0}^{n-1}P_i := P_0 | \ldots | P_{n-1}
\end{mathpar}

\subsubsection{Structural congruence}

\paragraph{Free and bound names and alpha-equivalence.} At the
core of structural equivalence is alpha-equivalence which identifies
process that are the same up to a change of variable. Formally, we
recognize the distinction between free and bound names. The free names
of a process, $\freenames{P}$, may be calculated recursively as
follows:

\begin{mathpar}
\freenames{\pzero} := \emptyset
  \and \\
  \freenames{x?(y).P} := \{ x \} \cup (\freenames{P} \setminus \{ y \})
  \and 
  \freenames{x!\langle P \rangle} := \{ x \} \cup \{ P \} 
  \and \\
  \freenames{P|Q} := \freenames{P} \cup \freenames{Q}
  \and \\
  \freenames{@{x}} := \{ x \}
\end{mathpar}

$\pi$
$\quotep{\pi}$

$\freenames{-} : \pi \to \mathcal{P}(\quotep{\pi})$

\begin{eqnarray*}
  \freenames{\pzero} & := & \emptyset \\
  \freenames{x?(y).P} & := & \{ x \} \cup (\freenames{P} \setminus \{ y \}) \\
  \freenames{x!\langle P \rangle} & := & \{ x \} \cup \{ P \} \\
  \freenames{P|Q} & := & \freenames{P} \cup \freenames{Q} \\
  \freenames{\dropn{x}} & := & \{ x \}
\end{eqnarray*}

The bound names of a process, $\boundnames{P}$, are those names occurring in $P$
that are not free. For example, in $x?(y).0$, the name $x$ is free, while $y$ is bound.

\begin{mathpar}
  \inferrule* [lab=monoidal-laws] {} { P|Q \equiv Q|P \and P|0 \equiv P \and P|(Q|R) \equiv (P|Q)|R }
\end{mathpar}

\begin{mathpar}
  \inferrule* [lab=alpha-equivalence] {} { (x)P \equiv (y)P\{y/x\} \and y \not\in \freenames{P} }
\end{mathpar}

\begin{definition}
Then two processes, $P,Q$, are alpha-equivalent if $P = Q\{\vec{y}/\vec{x}\}$ for
some $\vec{x} \in \boundnames{Q},\vec{y} \in \boundnames{P}$, where $Q\{\vec{y}/\vec{x}\}$
denotes the capture-avoiding substitution of $\vec{y}$ for $\vec{x}$ in $Q$.
\end{definition}

\begin{definition}
  The {\em structural congruence} \cite{SangiorgiWalker} , $\equiv$,
  between processes is the least congruence containing
  alpha-equivalence, satisfying the abelian monoid laws
  (associativity, commutativity and $\pzero$ as identity) for parallel
  composition $|$ and for summation $+$.
\end{definition}

\subsection{Name equivalence}

We take name equivalence, written $\nameeq$, to be the smallest
equivalence relation generated by the following rules.

\begin{mathpar}
\inferrule*[lab=Quote-drop]
{ }
{ \quotep{@{x}} \nameeq x }

\inferrule*[lab=Struct-equiv]
{ P \scong Q }
{ \quotep{P} \nameeq \quotep{Q} }
\end{mathpar}

The astute reader will have noticed that the mutual recursion of names
and processes imposes a mutual recursion on alpha-equivalence and
structural equivalence via name-equivalence. Fortunately, all of this
works out pleasantly and we may calculate in the natural way, free of
concern. The reader interested in the details is referred to the
appendix \ref{appendix:rho_details}.

\subsection{Substitution}

We use $\Proc$ for the set of processes, $\QProc$ for the set of
names, and $\id{\{}\vec{y} / \vec{x} \id{\}}$ to denote partial maps,
$s : \QProc \rightarrow \QProc$. A map, $s$ lifts, uniquely, to a map
on process terms, $\widehat{s} : \Proc \rightarrow \Proc$ by the
following equations.

\begin{mathpar}
  (0) \psubstp{Q}{P} := 0 \\
  (R \juxtap S) \psubstp{Q}{P}
  :=    
  (R)\psubstp{Q}{P} \juxtap (S) \psubstp{Q}{P} \\
  (x?(y).R) \psubstp{Q}{P}    
  :=    
  (x)\substp{Q}{P} (z)\concat( (R \psubstn{z}{y}) \psubstp{Q}{P} ) \\
  (\lift{x}{R}) \psubstp{Q}{P}  
  :=
  \lift{(x)\substp{Q}{P}}{ R \psubstp{Q}{P} } \\
%   (\dropn{x})  \psubstp{Q}{P}       
%   := 
%   \left\{ 
%     \begin{array}{ccc} 
%       \dropn{\quotep{Q}} & & x \nameeq \quotep{P} \\
%       \dropn{x} & & otherwise \\
%     \end{array}
%   \right. 
  (\dropn{x})  \psubstp{Q}{P}       
  := 
  \left\{ 
    \begin{array}{ccc} 
      Q & & x \nameeq \quotep{P} \\
      \dropn{x} & & otherwise \\
    \end{array}
  \right.
\end{mathpar}
 

where

\begin{eqnarray}
  (x)\id{\{} \lpquote Q \rpquote / \lpquote P \rpquote \id{\}}            = 
  \left\{ 
    \begin{array}{ccc}
      \lpquote Q \rpquote & & x \nameeq \lpquote P \rpquote \\
      x & & otherwise \\
    \end{array}
  \right. \nonumber
\end{eqnarray}

and $z$ is chosen distinct from $\quotep{P}$, $\quotep{Q}$, the free
names in $Q$, and all the names in $R$. Our $\alpha$-equivalence will
be built in the standard way from this substitution.

\begin{remark}\label{rem:no_self_referential_names}
  One consequence of these definitions is that $\forall P. \quotep{P}
  \not\in \freenames{P}$.
\end{remark}

\subsection{ Dynamic quote: an example }

Anticipating something of what's to come, consider applying the
substitution, $\widehat{\id{\{}u / z \id{\}}}$, to the following pair
of processes, $\lift{w}{y!(z)}$ and $w[ \lpquote y!(z) \rpquote ]$.

\begin{eqnarray}
	\lift{w}{y!(z)}\widehat{\id{\{}u / z \id{\}}}
		& = &
		\lift{w}{y!(u)} \nonumber\\
	w[ \lpquote y!(z) \rpquote ] \widehat{ \id{\{}u / z \id{\}} }
		& = &
		w[ \lpquote y!(z) \rpquote ] \nonumber
\end{eqnarray}

Because the body of the process between quotes is impervious to
substitution, we get radically different answers. In fact, by
examining the first process in an input context,
e.g. $x?(z).\lift{w}{y!(z)}$, we see that the process under the lift
operator may be shaped by prefixed inputs binding a name inside it. In
this sense, the lift operator will be seen as a way to dynamically
construct processes before reifying them as names.

Finally equipped with these standard features we can present the
dynamics of the calculus.

\subsubsection{Operational semantics} 

Finally, we introduce the computational dynamics. What marks these
algebras as distinct from other more traditionally studied algebraic
structures, e.g. vector spaces or polynomial rings, is the manner in
which dynamics is captured. In traditional structures, dynamics is typically
expressed through morphisms between such structures, as in linear maps
between vector spaces or morphisms between rings. In algebras
associated with the semantics of computation, the dynamics is
expressed as part of the algebraic structure itself, through a
reduction reduction relation typically denoted by $\red$. Below, we
give a recursive presentation of this relation for the calculus used
in the encoding.

$\red \subseteq \pi \times \pi$
$\red : \pi \to \mathcal{P}(\pi)$

\begin{mathpar}
  \inferrule* [lab=Comm] { \textsf{match}( x_{src}, x_{trgt} ) } { x_{trgt}?(y)P \; | \; x_{src}!\langle {Q} \rangle \red P\{\quotep{Q}/y}\} }
  \and \\
  \inferrule* [lab=Par] {{P} \red {P}'} {{{P} | {Q}} \red {{P}' | {Q}}}
  \and
  \inferrule* [lab=Equiv]{{{P} \scong {P}'} \andalso {{P}' \red {Q}'} \andalso {{Q}' \scong {Q}}}{{P} \red {Q}}
\end{mathpar}

\begin{eqnarray*}
  match_{\equiv} (\quotep{P},\quotep{Q}) & := & P \equiv Q \\
  match_{\dagger}(\quotep{P},\quotep{Q}) & := & \forall R. P|Q \red^{*} R => R \red^{*} 0 \\
  match_{K}(\quotep{P},\quotep{Q}) & := & K \mbox{ for some context } K
\end{eqnarray*}

$u?(x)P | u!\langle Q \rangle \red P\{\quotep{Q}/x\}$

%We write $\wred$ for $\red^*$, and $P\red$ if $\exists Q $ such that $ P \red Q$.
We write $P\red$ if $\exists Q $ such that $ P \red Q$ and $P\not\red$, otherwise.

\section{Replication}

As mentioned before, it is known that replication (and hence
recursion) can be implemented in a higher-order process algebra
\cite{SangiorgiWalker}. As our first example of calculation with the
machinery thus far presented we give the construction explicitly in
the {\rhoc}.

\begin{eqnarray}
	D_{x} & := & \prefix{x}{y}{(\binpar{\outputp{x}{y}}{@{y}})} \nonumber\\
	\bangp_{x}{P} & := & \binpar{{x}!\langle{\binpar{D_{x}}{P}}\rangle}{D_{x}} \nonumber
\end{eqnarray}

\begin{eqnarray}
	\bangp_{x}{P} & & \nonumber\\
	=
	& {x}!\langle{(\prefix{x}{y}{(\outputp{x}{y} | @{y})) | P}}\rangle 
	      | \prefix{x}{y}{(\outputp{x}{y} | @{y})} & \nonumber\\
	\red
	& (\outputp{x}{y} | @{y})\substn{\quotep{(\prefix{x}{y}{(@{y} | \outputp{x}{y})) | P}}}{y} & \nonumber\\
	=
	& \outputp{x}{\quotep{(\prefix{x}{y}{(\outputp{x}{y} | @{y})) | P}}}
	  | {(\prefix{x}{y}{(\outputp{x}{y} | @{y})) | P}} & \nonumber\\
	\red
	& \ldots & \nonumber\\
	\red^*
	& P | P | \ldots & \nonumber
\end{eqnarray}

Of course, this encoding, as an implementation, runs away, unfolding
$\bangp{P}$ eagerly. A lazier and more implementable replication
operator, restricted to input-guarded processes, may be obtained as follows.

\begin{eqnarray}
\bangp{\prefix{u}{v}{P}} 
	:= 
	\binpar{\lift{x}{\prefix{u}{v}{(\binpar{D(x)}{P})}}}{D(x)} \nonumber
\end{eqnarray}

\begin{remark}
  Note that the lazier definition still does not deal with summation
  or mixed summation (i.e. sums over input and output). The reader is
  invited to construct definitions of replication that deal with these
  features. 

  Further, the definitions are parameterized in a name, $x$. Can you,
  gentle reader, make a definition that eliminates this parameter and
  guarantees no accidental interaction between the replication
  machinery and the process being replicated -- i.e. no accidental
  sharing of names used by the process to get its work done and the
  name(s) used by the replication to effect copying. This latter
  revision of the definition of replication is crucial to obtaining
  the expected identity $!!P \sim !P$.
\end{remark}

\begin{remark}\label{rem:paradoxical_combinator}
  The reader familiar with the lambda calculus will have noticed the
  similarity between $D$ and the paradoxical combinator.

  [Ed. note: the existence of this seems to suggest we have to be more
  restrictive on the set of processes and names we admit if we are to
  support no-cloning.]
\end{remark}

\subsubsection{Bisimulation}

The computational dynamics gives rise to another kind of equivalence,
the equivalence of computational behavior. As previously mentioned
this is typically captured \emph{via} some form of bisimulation.

% The notion we use in this paper is weak barbed bisimulation
% \cite{milner91polyadicpi}.

The notion we use in this paper is derived from weak barbed
bisimulation \cite{milner91polyadicpi}. 

\begin{definition}
An \emph{observation relation}, $\downarrow_{\mathcal N}$, over a set
of names, $\mathcal N$, is the smallest relation satisfying the rules
below.

\infrule[Out-barb]{y \in {\mathcal N}, \; x \nameeq y}
		  {\outputp{x}{v} \downarrow_{\mathcal N} x}
\infrule[Par-barb]{\mbox{$P\downarrow_{\mathcal N} x$ or $Q\downarrow_{\mathcal N} x$}}
		  {\binpar{P}{Q} \downarrow_{\mathcal N} x}

We write $P \Downarrow_{\mathcal N} x$ if there is $Q$ such that 
$P \wred Q$ and $Q \downarrow_{\mathcal N} x$.
\end{definition}

\begin{definition}
%\label{def.bbisim}
An  ${\mathcal N}$-\emph{barbed bisimulation} over a set of names, ${\mathcal N}$, is a symmetric binary relation 
${\mathcal S}_{\mathcal N}$ between agents such that $P\rel{S}_{\mathcal N}Q$ implies:
\begin{enumerate}
\item If $P \red P'$ then $Q \wred Q'$ and $P'\rel{S}_{\mathcal N} Q'$.
\item If $P\downarrow_{\mathcal N} x$, then $Q\Downarrow_{\mathcal N} x$.
\end{enumerate}
$P$ is ${\mathcal N}$-barbed bisimilar to $Q$, written
$P \wbbisim_{\mathcal N} Q$, if $P \rel{S}_{\mathcal N} Q$ for some ${\mathcal N}$-barbed bisimulation ${\mathcal S}_{\mathcal N}$.
\end{definition}

$\mathcal{R} \subseteq \pi \times \pi$

$P \mathcal{R} Q => \forall P'. P \red P' \Rightarrow \exists Q'. Q \red Q', P' \mathcal{R} Q'$

$P \vdash x \Rightarrow Q \vdash x$

\begin{mathpar}
  \inferrule*[lab=Out-barb]{x \nameeq y}{{y}!\langle{Q}\rangle \vdash x}
  \and
  \inferrule*[lab=Par-barb]{\mbox{$P\vdash x$ or $Q\vdash x$}}{\binpar{P}{Q} \vdash x}
\end{mathpar}

\subsubsection{Contexts}

One of the principle advantages of computational calculi like the
$\pi$-calculus is a well-defined notion of context,
contextual-equivalence and a correlation between
contextual-equivalence and notions of bisimulation. The notion of
context allows the decomposition of a process into (sub-)process and
its syntactic environment, its context. Thus, a context may be
thought of as a process with a ``hole'' (written $\Box$) in it. The
application of a context $M$ to a process $P$, written $M[P]$, is
tantamount to filling the hole in $M$ with $P$. In this paper we do
not need the full weight of this theory, but do make use of the notion
of context in the proof the main theorem. 

\begin{mathpar}
  \inferrule* [lab=summation] {} {{M_{M},M_{N}} \bc \Box \;|\; x.M_{A} \;|\; M_{M}+M_{N}}
  \and
  \inferrule* [lab=agent] {} {{M_{A}} \bc (\vec{x})M_{P} \;| \; \clift{P_0,\ldots,M_{P},\ldots,P_N}}
  \and \\
  \inferrule* [lab=process] {} {{M_{P}} \bc M_{N} \;| \;P|M_{P} }
\end{mathpar} 

\begin{mathpar}
  \inferrule* [lab=sychronization] {} {M_{N} \bc \Box \;|\; x?M_{F} \;|\; x!M_{C}}
  \and
  \inferrule* [lab=abstraction] {} {{M_{F}} \bc (x)M_{P} }
  \and
  \inferrule* [lab=concretion] {} {{M_{C}} \bc \langle M_{P} \rangle }
  \and \\
  \inferrule* [lab=process] {} {{M_{P}} \bc M_{N} \;| \;P|M_{P} }
\end{mathpar}

\begin{definition}[contextual application] Given a context $M$, and
  process $P$, we define the \emph{contextual application}, $M[P] :=
  M\{P/\Box\}$. That is, the contextual application of M to P is the
  substitution of $P$ for $\Box$ in $M$.
\end{definition}

$\meaningof{-} : L \to \mathcal{P}(\pi)$

\begin{mathpar}
  \inferrule* [lab=collection] {} {\meaningof{true} = \pi, \and \meaningof{~E} = \pi \setminus \meaningof{E}, \and \meaningof{E_{1} \& E_{2}} = \meaningof{E_{1}} \cap \meaningof{E_{2}}}
\end{mathpar}

\begin{mathpar}
  \inferrule* [lab=structure] {} {\meaningof{0} = \{ P \in \pi | P \equiv 0 \}, \and \\ \meaningof{E_1 | E_2} = \{ P \in \pi | P \equiv P_{1} | P_{2}, P_{1} \in \meaningof{E_{1}}, P_{2} \in \meaningof{E_2}\} }
\end{mathpar}

\begin{mathpar}
 \inferrule* [lab=behavior] {} {\meaningof{\langle a?b \rangle E} = \{ P \in \pi | P \equiv Q | u?(y)P', \\ \and \\\\ \and \\ \;\;\; u \in \meaningof{a}, \forall z.P'\{z/y\} \in \meaningof{E\{z/b\}}\}, \and \\ \meaningof{a!E} = \{ P \in \pi | P \equiv Q | x!\langle P' \rangle, x \in \meaningof{a} P' \in \meaningof{E}\} }
\end{mathpar}

\begin{mathpar}
 \inferrule* [lab=nominal] {} {\meaningof{\quotep{E}} = \{ \quotep{P} \in \quotep{\pi} | P \in \meaningof{E} \}, \and \meaningof{\quotep{P}} = \{ \quotep{Q} \in \quotep{\pi} | P \equiv Q \} \and \\ \meaningof{@\quotep{E}} = \{ P \in \pi | P \equiv @x, x \in \meaningof{E} \}}
\end{mathpar}

\begin{eqnarray*}
  \\
  \meaningof{-} : TS \to ST
\end{eqnarray*}

\begin{eqnarray*}
  \\
  L : TS \to ST
\end{eqnarray*}

\begin{eqnarray*}
  \\
  P \models E \iff P \in \meaningof{E}
\end{eqnarray*}

\begin{eqnarray*}
  P \approx_{L} Q \iff \forall E \in L. P \models E \iff Q \models E
\end{eqnarray*}

\begin{eqnarray*}
  P \approx_{K} Q
\end{eqnarray*}

\begin{eqnarray*}
  P \approx Q
\end{eqnarray*}

$\approx_{K} = \approx = \approx_{L}$

\subsubsection{Contextual duality}

Note that contexts extend the quotation operation to a family of
operations from processes to names. Given a context, $M$, we can
define a \emph{nominal context}, $\quotep{M}$ by $\quotep{M}[P] :=
\quotep{M[P]}$. To foreshadow what is to come we observe that these
operations enjoy a duality with processes very much like the duality
between vectors and maps from vectors to scalars.

Further, because the calculus is essentially higher-order, we have a
correspondence between contexts and processes. More specifically,
given a name $x$ and a context $M$ we can construct $M^{*}_{x}$ such
that 

\begin{mathpar}
  M^{*}_{x} | \lift{x}{P} \red M[P]
\end{mathpar}

namely,

\begin{mathpar}
  M^{*}_{x} := x?(u).M[\dropn{u}]
\end{mathpar}

The dependence of $M^{*}_{x}$ on a name makes it an abstraction, 

\begin{mathpar}
  M^{*} := (x)x?(u).M[\dropn{u}]
\end{mathpar}

\subsection{Additional notation}

It will sometimes be convenient to denote the process a name
quotes. We already have the notation $x = \quotep{P}$, but it will be
convenient to introduce an alternate notation, $\procn{x}$, when we
want to emphasize the connection to the use of the name. Note that, by
virtue of name equivalence, $\quotep{\procn{x}} \nameeq x$; so, the
notation is consistent with previous definitions.

Further, because names have structure it is possible to effect
substitutions on the basis of that structure. This means we need to
upgrade our notation for substitutions, which we accomplish by
adapting comprehension notation. Thus,

\begin{mathpar}
  P\{ y / x : x \in S \}
\end{mathpar}

is interpreted to mean the process derived from P by replacing (in a
capture-avoiding manner) each occurrence of $x$ in $S$ by $y$. For example,

\begin{mathpar}
  P\{ \quotep{\procn{x}|\procn{x}} / x : x \in \freenames{P} \}
\end{mathpar}

will replace each (occurrence) of a free name $x$ in $P$ by
$\quotep{\procn{x}|\procn{x}}$.

Also, we will avail ourselves of the notation $x^{L}$ and $x^{R}$ to
denote injections of a name into disjoint copies of the name
space. There are numerous ways to accomplish this. One example can be
found in \cite{MeredithR05}. This notation overloads to vectors of
names: $\vec{x}^{\pi} := (x_{i}^{\pi} \; : \; 0 \leq i < |\vec{x}| )$ where $\pi \in \{L,R\}$.

We also use $P^{\Box} := P|\Box$.

In \cite{MeredithR05} an interpretation of the new operator is
given. It turns out that there are several possible interpretations
all enjoying the requisite algebraic properties of the operator (see
\cite{milner91polyadicpi}). We will therefore make liberal use of
$(\nu\; \vec{x})P$.

% subsection the_syntax_and_semantics_of_the_notation_system (end)   

\input{qm2pi.qmops} 

\input{qm2pi.sterngerlach} 

\input{qm2pi.metric} 

% section concurrent_process_calculi (end)

%\input{qm2pi.proofsketch}

% section proof sketch (end)

%\input{qm2pi.slviaknots} 

% section spatial logic via knots (end)

\input{qm2pi.conclusion}

% section conclusion (end)

%\input{qm2pi.dtcodes} 

% section wiring algorithm (end)

\input{qm2pi.ack} 

% section acknowledgments (end)

\newpage


\bibliographystyle{plain}   
\bibliography{../../biblios/main.bib}

\input{qm2pi.rhodetails}

\end{document}



% section front matter (end)

\section{Introduction}\label{sec:introduction} % (fold)
In this draft of the material i am going to have to dispense with the
usual writing conventions adopted in papers on these topics. i'm going
to have adopt whatever tone i need at the time i'm writing up the
calculations. Sometimes this may be very conversational; others it may
be the barest mathematical grunts; others still it may be that i have
lifted text from one of my other papers because the exposition of some
point was better said there. i hope that my readers are not unduly put
out by this decision. i'm not doing this to flout convention or be
rebellious. i find these calculations very technically challenging. To
keep everything going technically, something has to give; i have to
let go of some cognitive burden. So, the academic writing style --
with all of its trade-offs in terms of facilitating technical
communication -- is what i'm letting go of. Perhaps subsequent drafts
can be tightened and polished, but for now, i'm going to speak as if
we were sitting together in a coffee shop with a laptop, wifi and a
pad of paper and a pencil.

So, here's what i have to say. We -- you and i, comfortably ensconced
in our coffee shop and well-equipped with our tools -- can realize and
carry out the calculations of quantum mechanics over a very different
formal theory of dynamics, a formal theory of dynamics that
corresponds to a theory of concurrent computation with
\emph{reflection}. It has the advantage that the underlying theory is
already `quantized', but supports analogues all of the continuuous
operations. Strikingly, this underlying theory has recently been
connected with a notion of metric that we can show, by calculating
together, coincides with the metric induced by the inner product.

There are a lot of reasons why you might be interested in seeing
calculations of this form. Here's why i'm interested. For the past
several centuries there has been no competitor to the ``Newtonian''
account of dynamics. As a result the predominant share of accounts of
dynamical systems and situations have had to be formulated in terms of
the Newtonian machinery. i view this as an intellectually dangerous
position to occupy. Everything, despite it's intrinsic shape, turns
into a nail to be hit with this hammer. Recently, however, the theory
of computation has matured to the point where we have candidates for
theories of dynamics that offer very different perspective on
reasoning about dynamical systems and situations. Testing these
candidates against very successful accounts of dynamical situations,
like quantum mechanics, is going to give us some sense of how mature
they are and some measure of the quality of these accounts of
dynamics.

\subsection{Summary of contributions and outline of paper}

So, we're going to develop an interpretation of the operations of
quantum mechanics normally interpreted by Hilbert spaces and
operators. We're going to do this over a theory of computation. Note
that this is very different than the usual quantum computation program
which develops notions of computation over quantum mechanics. Rather,
we are developing a story that aligns with Wheeler's slogan: It from
Bit. To do this we will first provide an account of the theory of
computation at play here. Then we will dive into a calculation-driven
interpretation of the operations of quantum mechanics.

The reason we take this approach is that -- until very recently --
there hasn't been an axiomatic account of quantum mechanics. As a
result there has been no sharp delineation of the mathematical theory
supporting interpretation of the physical theory and the physical
theory, itself. So, ambient features of the maths are free to be
exploited (or supressed) without a real accounting of their physical
relevance. There is no sharp statement ``here's the physical theory''
qua \emph{theory} and ``here's the mathematical interpretation''
enabling a judgment of how faithful the interpretation is -- apart
from experimental observation. When there is an axiomatic account we
can judge how well a given mathematical formalism supports an
interpretation of the axioms, independent of
experimentation. Likewise, we can judge how well we have captured our
physical evidence and experience with our axiomatics, independent of
any specific mathematical implementation, with accidental detail that
may or may not have physical significance. 

In lieu of a fully fleshed out and vetted axiomatic account of quantum
mechanics, interpreting the operational notions in service of modeling
physical systems will have to suffice. In other words, we are not in
the business of providing a model of Hilbert spaces and operators. We
are in the business of providing a model of quantum mechanics because
we are motivated by testing our notions of dynamics against physical
theory; and, the predictive calculations of the physical theory must
serve as the best formulation -- shy of a fully fleshed out axiomatic
account -- of the physical theory itself (as they have for scientific
theories since time immemorial). Put another way, despite a
whole-hearted commitment to an It-from-Bit ontology, we are firmly
aligned with the shut-up-and-calculate camp as the best way to obtain
results either from the physical perspective or as a quality assurance
measure of our fledgling theory of dynamics.

In detail, we present a reflective process calculus. Then we develop
intuitive correspondences between the notions available in this
calculus and the usual physical notions supporting quantum mechanical
calculations. Thus, 

\begin{table}[htp]
  \center{
    \fbox{
      \begin{tabular}{c|c}
        quantum mechanics & process calculus \\
        \hline
        scalar & name \\
        state vector & process \\
        dual & contextual duals \\
        matrix & formal sums of process-context-dual pairs \\
        orthogonality & process annihilation \\
        inner product & execution-formula + quoting
      \end{tabular}
    }
  }
  \caption{QM - process calculi correspondences}
\end{table}

Then we tighten up these intuitions to operational definitions. We
employ the Dirac notation as the best proxy we can find for an
abstract syntax of the quantum mechanical notions. The definitions we
develop put us in contact with equational constraints coming from the
theory that we demonstrate the definitions and calculations satisfy.

This puts us in a position to shut up and calculate for the
Stern-Gerlach experimental set up, showing how these predictive
calculations become calculations on processes in our theory of a
reflective process calculus.

Penultimately, we demonstrate that the notion of metric coming from
the inner product coincides with the notion of metric available from
the theory of bisimulation. This demonstration gives us the right to
think of space as arising from behavior. Finally, we consider where we
might go from the new vantage point we have obtained.

% section introduction (end) 
 
% section introduction (end)

% \documentclass[12pt]{llncs}
%\documentclass{jktr}

\usepackage[pdftex]{hyperref}                   
\usepackage {listings}
\usepackage {mathpartir}
\usepackage{bcprules}
%\usepackage{listings}
                       
\usepackage{graphicx} 
%\usepackage[margins=2.5cm,nohead,nofoot]{geometry}
%\usepackage{geometry}
\usepackage{amsfonts}
\usepackage{amstext}
\usepackage{latexsym}
\usepackage{amssymb}
\usepackage{color}


%\include{myPreamble}
\include{qm2pi.local} 

%\ifpdf
%\usepackage[pdftex]{graphicx}
%\else
%\usepackage{graphicx}
%\fi

 % \ifpdf
%  \usepackage{pdfsync}
%  \if


%\title{Brief Article}
%\author{David F. Snyder}
%\author{L.G. Meredith}

%\address{Dept. of Math., Texas State University--San Marcos, San Marcos, TX 78666}
       
\pagestyle{empty}


\begin{document}

\lstset{language=[Objective]Caml,frame=shadowbox}

\input{qm2pi.front}

% section front matter (end)

\input{qm2pi.intro} 
 
% section introduction (end)

% \input{qm2pi.knotations} 

% section notation (end)

\input{qm2pi.process.calculi} 

% section concurrent_process_calculi_and_spatial_logics_ (end)
    
%\input{qm2pi.knots2pi} 

%\input{qm2pi.trefoil} 

%\input{qm2pi.mainthm} 

% subsection basic_interpretation (end)

%\input{qm2pi.rho.presentation} 
\subsection{The syntax and semantics of the notation system}\label{sub:the_syntax_and_semantics_of_the_notation_system} % (fold)

We now summarize a technical presentation of the calculus that
embodies our theory of dynamics. The typical presentation of such a
calculus follows the style of giving generators and relations on
them. The grammar, below, describing term constructors, freely
generates the set of processes, $\Proc$. This set is then quotiented
by a relation known as structural congruence and it is over this set
that the notion of dynamics is expressed. This presentation is
essentially that of \cite{MeredithR05} with the addition of
polyadicity and summation. For readability we have relegated some of
the technical subtleties to an appendix.

\subsubsection{Process grammar}\label{subsub:process_grammar}

\begin{mathpar}
  \inferrule* [lab=synchronization] {} {{M} \bc \pzero \;|\; x?F \;|\; x!C }
  \and
  \inferrule* [lab=abstraction] {} {{F} \bc (x)P}
  \and
  \inferrule* [lab=concretion] {} {{C} \bc \langle Q \rangle}
  \and
  \inferrule* [lab=process] {} {{P,Q} \bc M \;| \;P|Q \;|\; @{x}}
  \and
  \inferrule* [lab=name] {} {{x} \bc \quotep{P}}
\end{mathpar} 

Note that $\vec{x}$ (resp. $\vec{P}$) denotes a vector of names
(resp. processes) of length $|\vec{x}|$ (resp. $|\vec{P}|$). We adopt
the following useful abbreviations.

\begin{mathpar}
   x?(\vec{y}).P := x.(\vec{y})P \and  x\clift{\vec{P}} := x.\clift{\vec{P}}
   \and x!(y) := \lift{x}{\dropn{y}}
   \and \Pi_{i=0}^{n-1}P_i := P_0 | \ldots | P_{n-1}
\end{mathpar}

\subsubsection{Structural congruence}

\paragraph{Free and bound names and alpha-equivalence.} At the
core of structural equivalence is alpha-equivalence which identifies
process that are the same up to a change of variable. Formally, we
recognize the distinction between free and bound names. The free names
of a process, $\freenames{P}$, may be calculated recursively as
follows:

\begin{mathpar}
\freenames{\pzero} := \emptyset
  \and \\
  \freenames{x?(y).P} := \{ x \} \cup (\freenames{P} \setminus \{ y \})
  \and 
  \freenames{x!\langle P \rangle} := \{ x \} \cup \{ P \} 
  \and \\
  \freenames{P|Q} := \freenames{P} \cup \freenames{Q}
  \and \\
  \freenames{@{x}} := \{ x \}
\end{mathpar}

$\pi$
$\quotep{\pi}$

$\freenames{-} : \pi \to \mathcal{P}(\quotep{\pi})$

\begin{eqnarray*}
  \freenames{\pzero} & := & \emptyset \\
  \freenames{x?(y).P} & := & \{ x \} \cup (\freenames{P} \setminus \{ y \}) \\
  \freenames{x!\langle P \rangle} & := & \{ x \} \cup \{ P \} \\
  \freenames{P|Q} & := & \freenames{P} \cup \freenames{Q} \\
  \freenames{\dropn{x}} & := & \{ x \}
\end{eqnarray*}

The bound names of a process, $\boundnames{P}$, are those names occurring in $P$
that are not free. For example, in $x?(y).0$, the name $x$ is free, while $y$ is bound.

\begin{mathpar}
  \inferrule* [lab=monoidal-laws] {} { P|Q \equiv Q|P \and P|0 \equiv P \and P|(Q|R) \equiv (P|Q)|R }
\end{mathpar}

\begin{mathpar}
  \inferrule* [lab=alpha-equivalence] {} { (x)P \equiv (y)P\{y/x\} \and y \not\in \freenames{P} }
\end{mathpar}

\begin{definition}
Then two processes, $P,Q$, are alpha-equivalent if $P = Q\{\vec{y}/\vec{x}\}$ for
some $\vec{x} \in \boundnames{Q},\vec{y} \in \boundnames{P}$, where $Q\{\vec{y}/\vec{x}\}$
denotes the capture-avoiding substitution of $\vec{y}$ for $\vec{x}$ in $Q$.
\end{definition}

\begin{definition}
  The {\em structural congruence} \cite{SangiorgiWalker} , $\equiv$,
  between processes is the least congruence containing
  alpha-equivalence, satisfying the abelian monoid laws
  (associativity, commutativity and $\pzero$ as identity) for parallel
  composition $|$ and for summation $+$.
\end{definition}

\subsection{Name equivalence}

We take name equivalence, written $\nameeq$, to be the smallest
equivalence relation generated by the following rules.

\begin{mathpar}
\inferrule*[lab=Quote-drop]
{ }
{ \quotep{@{x}} \nameeq x }

\inferrule*[lab=Struct-equiv]
{ P \scong Q }
{ \quotep{P} \nameeq \quotep{Q} }
\end{mathpar}

The astute reader will have noticed that the mutual recursion of names
and processes imposes a mutual recursion on alpha-equivalence and
structural equivalence via name-equivalence. Fortunately, all of this
works out pleasantly and we may calculate in the natural way, free of
concern. The reader interested in the details is referred to the
appendix \ref{appendix:rho_details}.

\subsection{Substitution}

We use $\Proc$ for the set of processes, $\QProc$ for the set of
names, and $\id{\{}\vec{y} / \vec{x} \id{\}}$ to denote partial maps,
$s : \QProc \rightarrow \QProc$. A map, $s$ lifts, uniquely, to a map
on process terms, $\widehat{s} : \Proc \rightarrow \Proc$ by the
following equations.

\begin{mathpar}
  (0) \psubstp{Q}{P} := 0 \\
  (R \juxtap S) \psubstp{Q}{P}
  :=    
  (R)\psubstp{Q}{P} \juxtap (S) \psubstp{Q}{P} \\
  (x?(y).R) \psubstp{Q}{P}    
  :=    
  (x)\substp{Q}{P} (z)\concat( (R \psubstn{z}{y}) \psubstp{Q}{P} ) \\
  (\lift{x}{R}) \psubstp{Q}{P}  
  :=
  \lift{(x)\substp{Q}{P}}{ R \psubstp{Q}{P} } \\
%   (\dropn{x})  \psubstp{Q}{P}       
%   := 
%   \left\{ 
%     \begin{array}{ccc} 
%       \dropn{\quotep{Q}} & & x \nameeq \quotep{P} \\
%       \dropn{x} & & otherwise \\
%     \end{array}
%   \right. 
  (\dropn{x})  \psubstp{Q}{P}       
  := 
  \left\{ 
    \begin{array}{ccc} 
      Q & & x \nameeq \quotep{P} \\
      \dropn{x} & & otherwise \\
    \end{array}
  \right.
\end{mathpar}
 

where

\begin{eqnarray}
  (x)\id{\{} \lpquote Q \rpquote / \lpquote P \rpquote \id{\}}            = 
  \left\{ 
    \begin{array}{ccc}
      \lpquote Q \rpquote & & x \nameeq \lpquote P \rpquote \\
      x & & otherwise \\
    \end{array}
  \right. \nonumber
\end{eqnarray}

and $z$ is chosen distinct from $\quotep{P}$, $\quotep{Q}$, the free
names in $Q$, and all the names in $R$. Our $\alpha$-equivalence will
be built in the standard way from this substitution.

\begin{remark}\label{rem:no_self_referential_names}
  One consequence of these definitions is that $\forall P. \quotep{P}
  \not\in \freenames{P}$.
\end{remark}

\subsection{ Dynamic quote: an example }

Anticipating something of what's to come, consider applying the
substitution, $\widehat{\id{\{}u / z \id{\}}}$, to the following pair
of processes, $\lift{w}{y!(z)}$ and $w[ \lpquote y!(z) \rpquote ]$.

\begin{eqnarray}
	\lift{w}{y!(z)}\widehat{\id{\{}u / z \id{\}}}
		& = &
		\lift{w}{y!(u)} \nonumber\\
	w[ \lpquote y!(z) \rpquote ] \widehat{ \id{\{}u / z \id{\}} }
		& = &
		w[ \lpquote y!(z) \rpquote ] \nonumber
\end{eqnarray}

Because the body of the process between quotes is impervious to
substitution, we get radically different answers. In fact, by
examining the first process in an input context,
e.g. $x?(z).\lift{w}{y!(z)}$, we see that the process under the lift
operator may be shaped by prefixed inputs binding a name inside it. In
this sense, the lift operator will be seen as a way to dynamically
construct processes before reifying them as names.

Finally equipped with these standard features we can present the
dynamics of the calculus.

\subsubsection{Operational semantics} 

Finally, we introduce the computational dynamics. What marks these
algebras as distinct from other more traditionally studied algebraic
structures, e.g. vector spaces or polynomial rings, is the manner in
which dynamics is captured. In traditional structures, dynamics is typically
expressed through morphisms between such structures, as in linear maps
between vector spaces or morphisms between rings. In algebras
associated with the semantics of computation, the dynamics is
expressed as part of the algebraic structure itself, through a
reduction reduction relation typically denoted by $\red$. Below, we
give a recursive presentation of this relation for the calculus used
in the encoding.

$\red \subseteq \pi \times \pi$
$\red : \pi \to \mathcal{P}(\pi)$

\begin{mathpar}
  \inferrule* [lab=Comm] { \textsf{match}( x_{src}, x_{trgt} ) } { x_{trgt}?(y)P \; | \; x_{src}!\langle {Q} \rangle \red P\{\quotep{Q}/y}\} }
  \and \\
  \inferrule* [lab=Par] {{P} \red {P}'} {{{P} | {Q}} \red {{P}' | {Q}}}
  \and
  \inferrule* [lab=Equiv]{{{P} \scong {P}'} \andalso {{P}' \red {Q}'} \andalso {{Q}' \scong {Q}}}{{P} \red {Q}}
\end{mathpar}

\begin{eqnarray*}
  match_{\equiv} (\quotep{P},\quotep{Q}) & := & P \equiv Q \\
  match_{\dagger}(\quotep{P},\quotep{Q}) & := & \forall R. P|Q \red^{*} R => R \red^{*} 0 \\
  match_{K}(\quotep{P},\quotep{Q}) & := & K \mbox{ for some context } K
\end{eqnarray*}

$u?(x)P | u!\langle Q \rangle \red P\{\quotep{Q}/x\}$

%We write $\wred$ for $\red^*$, and $P\red$ if $\exists Q $ such that $ P \red Q$.
We write $P\red$ if $\exists Q $ such that $ P \red Q$ and $P\not\red$, otherwise.

\section{Replication}

As mentioned before, it is known that replication (and hence
recursion) can be implemented in a higher-order process algebra
\cite{SangiorgiWalker}. As our first example of calculation with the
machinery thus far presented we give the construction explicitly in
the {\rhoc}.

\begin{eqnarray}
	D_{x} & := & \prefix{x}{y}{(\binpar{\outputp{x}{y}}{@{y}})} \nonumber\\
	\bangp_{x}{P} & := & \binpar{{x}!\langle{\binpar{D_{x}}{P}}\rangle}{D_{x}} \nonumber
\end{eqnarray}

\begin{eqnarray}
	\bangp_{x}{P} & & \nonumber\\
	=
	& {x}!\langle{(\prefix{x}{y}{(\outputp{x}{y} | @{y})) | P}}\rangle 
	      | \prefix{x}{y}{(\outputp{x}{y} | @{y})} & \nonumber\\
	\red
	& (\outputp{x}{y} | @{y})\substn{\quotep{(\prefix{x}{y}{(@{y} | \outputp{x}{y})) | P}}}{y} & \nonumber\\
	=
	& \outputp{x}{\quotep{(\prefix{x}{y}{(\outputp{x}{y} | @{y})) | P}}}
	  | {(\prefix{x}{y}{(\outputp{x}{y} | @{y})) | P}} & \nonumber\\
	\red
	& \ldots & \nonumber\\
	\red^*
	& P | P | \ldots & \nonumber
\end{eqnarray}

Of course, this encoding, as an implementation, runs away, unfolding
$\bangp{P}$ eagerly. A lazier and more implementable replication
operator, restricted to input-guarded processes, may be obtained as follows.

\begin{eqnarray}
\bangp{\prefix{u}{v}{P}} 
	:= 
	\binpar{\lift{x}{\prefix{u}{v}{(\binpar{D(x)}{P})}}}{D(x)} \nonumber
\end{eqnarray}

\begin{remark}
  Note that the lazier definition still does not deal with summation
  or mixed summation (i.e. sums over input and output). The reader is
  invited to construct definitions of replication that deal with these
  features. 

  Further, the definitions are parameterized in a name, $x$. Can you,
  gentle reader, make a definition that eliminates this parameter and
  guarantees no accidental interaction between the replication
  machinery and the process being replicated -- i.e. no accidental
  sharing of names used by the process to get its work done and the
  name(s) used by the replication to effect copying. This latter
  revision of the definition of replication is crucial to obtaining
  the expected identity $!!P \sim !P$.
\end{remark}

\begin{remark}\label{rem:paradoxical_combinator}
  The reader familiar with the lambda calculus will have noticed the
  similarity between $D$ and the paradoxical combinator.

  [Ed. note: the existence of this seems to suggest we have to be more
  restrictive on the set of processes and names we admit if we are to
  support no-cloning.]
\end{remark}

\subsubsection{Bisimulation}

The computational dynamics gives rise to another kind of equivalence,
the equivalence of computational behavior. As previously mentioned
this is typically captured \emph{via} some form of bisimulation.

% The notion we use in this paper is weak barbed bisimulation
% \cite{milner91polyadicpi}.

The notion we use in this paper is derived from weak barbed
bisimulation \cite{milner91polyadicpi}. 

\begin{definition}
An \emph{observation relation}, $\downarrow_{\mathcal N}$, over a set
of names, $\mathcal N$, is the smallest relation satisfying the rules
below.

\infrule[Out-barb]{y \in {\mathcal N}, \; x \nameeq y}
		  {\outputp{x}{v} \downarrow_{\mathcal N} x}
\infrule[Par-barb]{\mbox{$P\downarrow_{\mathcal N} x$ or $Q\downarrow_{\mathcal N} x$}}
		  {\binpar{P}{Q} \downarrow_{\mathcal N} x}

We write $P \Downarrow_{\mathcal N} x$ if there is $Q$ such that 
$P \wred Q$ and $Q \downarrow_{\mathcal N} x$.
\end{definition}

\begin{definition}
%\label{def.bbisim}
An  ${\mathcal N}$-\emph{barbed bisimulation} over a set of names, ${\mathcal N}$, is a symmetric binary relation 
${\mathcal S}_{\mathcal N}$ between agents such that $P\rel{S}_{\mathcal N}Q$ implies:
\begin{enumerate}
\item If $P \red P'$ then $Q \wred Q'$ and $P'\rel{S}_{\mathcal N} Q'$.
\item If $P\downarrow_{\mathcal N} x$, then $Q\Downarrow_{\mathcal N} x$.
\end{enumerate}
$P$ is ${\mathcal N}$-barbed bisimilar to $Q$, written
$P \wbbisim_{\mathcal N} Q$, if $P \rel{S}_{\mathcal N} Q$ for some ${\mathcal N}$-barbed bisimulation ${\mathcal S}_{\mathcal N}$.
\end{definition}

$\mathcal{R} \subseteq \pi \times \pi$

$P \mathcal{R} Q => \forall P'. P \red P' \Rightarrow \exists Q'. Q \red Q', P' \mathcal{R} Q'$

$P \vdash x \Rightarrow Q \vdash x$

\begin{mathpar}
  \inferrule*[lab=Out-barb]{x \nameeq y}{{y}!\langle{Q}\rangle \vdash x}
  \and
  \inferrule*[lab=Par-barb]{\mbox{$P\vdash x$ or $Q\vdash x$}}{\binpar{P}{Q} \vdash x}
\end{mathpar}

\subsubsection{Contexts}

One of the principle advantages of computational calculi like the
$\pi$-calculus is a well-defined notion of context,
contextual-equivalence and a correlation between
contextual-equivalence and notions of bisimulation. The notion of
context allows the decomposition of a process into (sub-)process and
its syntactic environment, its context. Thus, a context may be
thought of as a process with a ``hole'' (written $\Box$) in it. The
application of a context $M$ to a process $P$, written $M[P]$, is
tantamount to filling the hole in $M$ with $P$. In this paper we do
not need the full weight of this theory, but do make use of the notion
of context in the proof the main theorem. 

\begin{mathpar}
  \inferrule* [lab=summation] {} {{M_{M},M_{N}} \bc \Box \;|\; x.M_{A} \;|\; M_{M}+M_{N}}
  \and
  \inferrule* [lab=agent] {} {{M_{A}} \bc (\vec{x})M_{P} \;| \; \clift{P_0,\ldots,M_{P},\ldots,P_N}}
  \and \\
  \inferrule* [lab=process] {} {{M_{P}} \bc M_{N} \;| \;P|M_{P} }
\end{mathpar} 

\begin{mathpar}
  \inferrule* [lab=sychronization] {} {M_{N} \bc \Box \;|\; x?M_{F} \;|\; x!M_{C}}
  \and
  \inferrule* [lab=abstraction] {} {{M_{F}} \bc (x)M_{P} }
  \and
  \inferrule* [lab=concretion] {} {{M_{C}} \bc \langle M_{P} \rangle }
  \and \\
  \inferrule* [lab=process] {} {{M_{P}} \bc M_{N} \;| \;P|M_{P} }
\end{mathpar}

\begin{definition}[contextual application] Given a context $M$, and
  process $P$, we define the \emph{contextual application}, $M[P] :=
  M\{P/\Box\}$. That is, the contextual application of M to P is the
  substitution of $P$ for $\Box$ in $M$.
\end{definition}

$\meaningof{-} : L \to \mathcal{P}(\pi)$

\begin{mathpar}
  \inferrule* [lab=collection] {} {\meaningof{true} = \pi, \and \meaningof{~E} = \pi \setminus \meaningof{E}, \and \meaningof{E_{1} \& E_{2}} = \meaningof{E_{1}} \cap \meaningof{E_{2}}}
\end{mathpar}

\begin{mathpar}
  \inferrule* [lab=structure] {} {\meaningof{0} = \{ P \in \pi | P \equiv 0 \}, \and \\ \meaningof{E_1 | E_2} = \{ P \in \pi | P \equiv P_{1} | P_{2}, P_{1} \in \meaningof{E_{1}}, P_{2} \in \meaningof{E_2}\} }
\end{mathpar}

\begin{mathpar}
 \inferrule* [lab=behavior] {} {\meaningof{\langle a?b \rangle E} = \{ P \in \pi | P \equiv Q | u?(y)P', \\ \and \\\\ \and \\ \;\;\; u \in \meaningof{a}, \forall z.P'\{z/y\} \in \meaningof{E\{z/b\}}\}, \and \\ \meaningof{a!E} = \{ P \in \pi | P \equiv Q | x!\langle P' \rangle, x \in \meaningof{a} P' \in \meaningof{E}\} }
\end{mathpar}

\begin{mathpar}
 \inferrule* [lab=nominal] {} {\meaningof{\quotep{E}} = \{ \quotep{P} \in \quotep{\pi} | P \in \meaningof{E} \}, \and \meaningof{\quotep{P}} = \{ \quotep{Q} \in \quotep{\pi} | P \equiv Q \} \and \\ \meaningof{@\quotep{E}} = \{ P \in \pi | P \equiv @x, x \in \meaningof{E} \}}
\end{mathpar}

\begin{eqnarray*}
  \\
  \meaningof{-} : TS \to ST
\end{eqnarray*}

\begin{eqnarray*}
  \\
  L : TS \to ST
\end{eqnarray*}

\begin{eqnarray*}
  \\
  P \models E \iff P \in \meaningof{E}
\end{eqnarray*}

\begin{eqnarray*}
  P \approx_{L} Q \iff \forall E \in L. P \models E \iff Q \models E
\end{eqnarray*}

\begin{eqnarray*}
  P \approx_{K} Q
\end{eqnarray*}

\begin{eqnarray*}
  P \approx Q
\end{eqnarray*}

$\approx_{K} = \approx = \approx_{L}$

\subsubsection{Contextual duality}

Note that contexts extend the quotation operation to a family of
operations from processes to names. Given a context, $M$, we can
define a \emph{nominal context}, $\quotep{M}$ by $\quotep{M}[P] :=
\quotep{M[P]}$. To foreshadow what is to come we observe that these
operations enjoy a duality with processes very much like the duality
between vectors and maps from vectors to scalars.

Further, because the calculus is essentially higher-order, we have a
correspondence between contexts and processes. More specifically,
given a name $x$ and a context $M$ we can construct $M^{*}_{x}$ such
that 

\begin{mathpar}
  M^{*}_{x} | \lift{x}{P} \red M[P]
\end{mathpar}

namely,

\begin{mathpar}
  M^{*}_{x} := x?(u).M[\dropn{u}]
\end{mathpar}

The dependence of $M^{*}_{x}$ on a name makes it an abstraction, 

\begin{mathpar}
  M^{*} := (x)x?(u).M[\dropn{u}]
\end{mathpar}

\subsection{Additional notation}

It will sometimes be convenient to denote the process a name
quotes. We already have the notation $x = \quotep{P}$, but it will be
convenient to introduce an alternate notation, $\procn{x}$, when we
want to emphasize the connection to the use of the name. Note that, by
virtue of name equivalence, $\quotep{\procn{x}} \nameeq x$; so, the
notation is consistent with previous definitions.

Further, because names have structure it is possible to effect
substitutions on the basis of that structure. This means we need to
upgrade our notation for substitutions, which we accomplish by
adapting comprehension notation. Thus,

\begin{mathpar}
  P\{ y / x : x \in S \}
\end{mathpar}

is interpreted to mean the process derived from P by replacing (in a
capture-avoiding manner) each occurrence of $x$ in $S$ by $y$. For example,

\begin{mathpar}
  P\{ \quotep{\procn{x}|\procn{x}} / x : x \in \freenames{P} \}
\end{mathpar}

will replace each (occurrence) of a free name $x$ in $P$ by
$\quotep{\procn{x}|\procn{x}}$.

Also, we will avail ourselves of the notation $x^{L}$ and $x^{R}$ to
denote injections of a name into disjoint copies of the name
space. There are numerous ways to accomplish this. One example can be
found in \cite{MeredithR05}. This notation overloads to vectors of
names: $\vec{x}^{\pi} := (x_{i}^{\pi} \; : \; 0 \leq i < |\vec{x}| )$ where $\pi \in \{L,R\}$.

We also use $P^{\Box} := P|\Box$.

In \cite{MeredithR05} an interpretation of the new operator is
given. It turns out that there are several possible interpretations
all enjoying the requisite algebraic properties of the operator (see
\cite{milner91polyadicpi}). We will therefore make liberal use of
$(\nu\; \vec{x})P$.

% subsection the_syntax_and_semantics_of_the_notation_system (end)   

\input{qm2pi.qmops} 

\input{qm2pi.sterngerlach} 

\input{qm2pi.metric} 

% section concurrent_process_calculi (end)

%\input{qm2pi.proofsketch}

% section proof sketch (end)

%\input{qm2pi.slviaknots} 

% section spatial logic via knots (end)

\input{qm2pi.conclusion}

% section conclusion (end)

%\input{qm2pi.dtcodes} 

% section wiring algorithm (end)

\input{qm2pi.ack} 

% section acknowledgments (end)

\newpage


\bibliographystyle{plain}   
\bibliography{../../biblios/main.bib}

\input{qm2pi.rhodetails}

\end{document}

 

% section notation (end)

\input{qm2pi.process.calculi} 

% section concurrent_process_calculi_and_spatial_logics_ (end)
    
%\documentclass[12pt]{llncs}
%\documentclass{jktr}

\usepackage[pdftex]{hyperref}                   
\usepackage {listings}
\usepackage {mathpartir}
\usepackage{bcprules}
%\usepackage{listings}
                       
\usepackage{graphicx} 
%\usepackage[margins=2.5cm,nohead,nofoot]{geometry}
%\usepackage{geometry}
\usepackage{amsfonts}
\usepackage{amstext}
\usepackage{latexsym}
\usepackage{amssymb}
\usepackage{color}


%\include{myPreamble}
\include{qm2pi.local} 

%\ifpdf
%\usepackage[pdftex]{graphicx}
%\else
%\usepackage{graphicx}
%\fi

 % \ifpdf
%  \usepackage{pdfsync}
%  \if


%\title{Brief Article}
%\author{David F. Snyder}
%\author{L.G. Meredith}

%\address{Dept. of Math., Texas State University--San Marcos, San Marcos, TX 78666}
       
\pagestyle{empty}


\begin{document}

\lstset{language=[Objective]Caml,frame=shadowbox}

\input{qm2pi.front}

% section front matter (end)

\input{qm2pi.intro} 
 
% section introduction (end)

% \input{qm2pi.knotations} 

% section notation (end)

\input{qm2pi.process.calculi} 

% section concurrent_process_calculi_and_spatial_logics_ (end)
    
%\input{qm2pi.knots2pi} 

%\input{qm2pi.trefoil} 

%\input{qm2pi.mainthm} 

% subsection basic_interpretation (end)

%\input{qm2pi.rho.presentation} 
\subsection{The syntax and semantics of the notation system}\label{sub:the_syntax_and_semantics_of_the_notation_system} % (fold)

We now summarize a technical presentation of the calculus that
embodies our theory of dynamics. The typical presentation of such a
calculus follows the style of giving generators and relations on
them. The grammar, below, describing term constructors, freely
generates the set of processes, $\Proc$. This set is then quotiented
by a relation known as structural congruence and it is over this set
that the notion of dynamics is expressed. This presentation is
essentially that of \cite{MeredithR05} with the addition of
polyadicity and summation. For readability we have relegated some of
the technical subtleties to an appendix.

\subsubsection{Process grammar}\label{subsub:process_grammar}

\begin{mathpar}
  \inferrule* [lab=synchronization] {} {{M} \bc \pzero \;|\; x?F \;|\; x!C }
  \and
  \inferrule* [lab=abstraction] {} {{F} \bc (x)P}
  \and
  \inferrule* [lab=concretion] {} {{C} \bc \langle Q \rangle}
  \and
  \inferrule* [lab=process] {} {{P,Q} \bc M \;| \;P|Q \;|\; @{x}}
  \and
  \inferrule* [lab=name] {} {{x} \bc \quotep{P}}
\end{mathpar} 

Note that $\vec{x}$ (resp. $\vec{P}$) denotes a vector of names
(resp. processes) of length $|\vec{x}|$ (resp. $|\vec{P}|$). We adopt
the following useful abbreviations.

\begin{mathpar}
   x?(\vec{y}).P := x.(\vec{y})P \and  x\clift{\vec{P}} := x.\clift{\vec{P}}
   \and x!(y) := \lift{x}{\dropn{y}}
   \and \Pi_{i=0}^{n-1}P_i := P_0 | \ldots | P_{n-1}
\end{mathpar}

\subsubsection{Structural congruence}

\paragraph{Free and bound names and alpha-equivalence.} At the
core of structural equivalence is alpha-equivalence which identifies
process that are the same up to a change of variable. Formally, we
recognize the distinction between free and bound names. The free names
of a process, $\freenames{P}$, may be calculated recursively as
follows:

\begin{mathpar}
\freenames{\pzero} := \emptyset
  \and \\
  \freenames{x?(y).P} := \{ x \} \cup (\freenames{P} \setminus \{ y \})
  \and 
  \freenames{x!\langle P \rangle} := \{ x \} \cup \{ P \} 
  \and \\
  \freenames{P|Q} := \freenames{P} \cup \freenames{Q}
  \and \\
  \freenames{@{x}} := \{ x \}
\end{mathpar}

$\pi$
$\quotep{\pi}$

$\freenames{-} : \pi \to \mathcal{P}(\quotep{\pi})$

\begin{eqnarray*}
  \freenames{\pzero} & := & \emptyset \\
  \freenames{x?(y).P} & := & \{ x \} \cup (\freenames{P} \setminus \{ y \}) \\
  \freenames{x!\langle P \rangle} & := & \{ x \} \cup \{ P \} \\
  \freenames{P|Q} & := & \freenames{P} \cup \freenames{Q} \\
  \freenames{\dropn{x}} & := & \{ x \}
\end{eqnarray*}

The bound names of a process, $\boundnames{P}$, are those names occurring in $P$
that are not free. For example, in $x?(y).0$, the name $x$ is free, while $y$ is bound.

\begin{mathpar}
  \inferrule* [lab=monoidal-laws] {} { P|Q \equiv Q|P \and P|0 \equiv P \and P|(Q|R) \equiv (P|Q)|R }
\end{mathpar}

\begin{mathpar}
  \inferrule* [lab=alpha-equivalence] {} { (x)P \equiv (y)P\{y/x\} \and y \not\in \freenames{P} }
\end{mathpar}

\begin{definition}
Then two processes, $P,Q$, are alpha-equivalent if $P = Q\{\vec{y}/\vec{x}\}$ for
some $\vec{x} \in \boundnames{Q},\vec{y} \in \boundnames{P}$, where $Q\{\vec{y}/\vec{x}\}$
denotes the capture-avoiding substitution of $\vec{y}$ for $\vec{x}$ in $Q$.
\end{definition}

\begin{definition}
  The {\em structural congruence} \cite{SangiorgiWalker} , $\equiv$,
  between processes is the least congruence containing
  alpha-equivalence, satisfying the abelian monoid laws
  (associativity, commutativity and $\pzero$ as identity) for parallel
  composition $|$ and for summation $+$.
\end{definition}

\subsection{Name equivalence}

We take name equivalence, written $\nameeq$, to be the smallest
equivalence relation generated by the following rules.

\begin{mathpar}
\inferrule*[lab=Quote-drop]
{ }
{ \quotep{@{x}} \nameeq x }

\inferrule*[lab=Struct-equiv]
{ P \scong Q }
{ \quotep{P} \nameeq \quotep{Q} }
\end{mathpar}

The astute reader will have noticed that the mutual recursion of names
and processes imposes a mutual recursion on alpha-equivalence and
structural equivalence via name-equivalence. Fortunately, all of this
works out pleasantly and we may calculate in the natural way, free of
concern. The reader interested in the details is referred to the
appendix \ref{appendix:rho_details}.

\subsection{Substitution}

We use $\Proc$ for the set of processes, $\QProc$ for the set of
names, and $\id{\{}\vec{y} / \vec{x} \id{\}}$ to denote partial maps,
$s : \QProc \rightarrow \QProc$. A map, $s$ lifts, uniquely, to a map
on process terms, $\widehat{s} : \Proc \rightarrow \Proc$ by the
following equations.

\begin{mathpar}
  (0) \psubstp{Q}{P} := 0 \\
  (R \juxtap S) \psubstp{Q}{P}
  :=    
  (R)\psubstp{Q}{P} \juxtap (S) \psubstp{Q}{P} \\
  (x?(y).R) \psubstp{Q}{P}    
  :=    
  (x)\substp{Q}{P} (z)\concat( (R \psubstn{z}{y}) \psubstp{Q}{P} ) \\
  (\lift{x}{R}) \psubstp{Q}{P}  
  :=
  \lift{(x)\substp{Q}{P}}{ R \psubstp{Q}{P} } \\
%   (\dropn{x})  \psubstp{Q}{P}       
%   := 
%   \left\{ 
%     \begin{array}{ccc} 
%       \dropn{\quotep{Q}} & & x \nameeq \quotep{P} \\
%       \dropn{x} & & otherwise \\
%     \end{array}
%   \right. 
  (\dropn{x})  \psubstp{Q}{P}       
  := 
  \left\{ 
    \begin{array}{ccc} 
      Q & & x \nameeq \quotep{P} \\
      \dropn{x} & & otherwise \\
    \end{array}
  \right.
\end{mathpar}
 

where

\begin{eqnarray}
  (x)\id{\{} \lpquote Q \rpquote / \lpquote P \rpquote \id{\}}            = 
  \left\{ 
    \begin{array}{ccc}
      \lpquote Q \rpquote & & x \nameeq \lpquote P \rpquote \\
      x & & otherwise \\
    \end{array}
  \right. \nonumber
\end{eqnarray}

and $z$ is chosen distinct from $\quotep{P}$, $\quotep{Q}$, the free
names in $Q$, and all the names in $R$. Our $\alpha$-equivalence will
be built in the standard way from this substitution.

\begin{remark}\label{rem:no_self_referential_names}
  One consequence of these definitions is that $\forall P. \quotep{P}
  \not\in \freenames{P}$.
\end{remark}

\subsection{ Dynamic quote: an example }

Anticipating something of what's to come, consider applying the
substitution, $\widehat{\id{\{}u / z \id{\}}}$, to the following pair
of processes, $\lift{w}{y!(z)}$ and $w[ \lpquote y!(z) \rpquote ]$.

\begin{eqnarray}
	\lift{w}{y!(z)}\widehat{\id{\{}u / z \id{\}}}
		& = &
		\lift{w}{y!(u)} \nonumber\\
	w[ \lpquote y!(z) \rpquote ] \widehat{ \id{\{}u / z \id{\}} }
		& = &
		w[ \lpquote y!(z) \rpquote ] \nonumber
\end{eqnarray}

Because the body of the process between quotes is impervious to
substitution, we get radically different answers. In fact, by
examining the first process in an input context,
e.g. $x?(z).\lift{w}{y!(z)}$, we see that the process under the lift
operator may be shaped by prefixed inputs binding a name inside it. In
this sense, the lift operator will be seen as a way to dynamically
construct processes before reifying them as names.

Finally equipped with these standard features we can present the
dynamics of the calculus.

\subsubsection{Operational semantics} 

Finally, we introduce the computational dynamics. What marks these
algebras as distinct from other more traditionally studied algebraic
structures, e.g. vector spaces or polynomial rings, is the manner in
which dynamics is captured. In traditional structures, dynamics is typically
expressed through morphisms between such structures, as in linear maps
between vector spaces or morphisms between rings. In algebras
associated with the semantics of computation, the dynamics is
expressed as part of the algebraic structure itself, through a
reduction reduction relation typically denoted by $\red$. Below, we
give a recursive presentation of this relation for the calculus used
in the encoding.

$\red \subseteq \pi \times \pi$
$\red : \pi \to \mathcal{P}(\pi)$

\begin{mathpar}
  \inferrule* [lab=Comm] { \textsf{match}( x_{src}, x_{trgt} ) } { x_{trgt}?(y)P \; | \; x_{src}!\langle {Q} \rangle \red P\{\quotep{Q}/y}\} }
  \and \\
  \inferrule* [lab=Par] {{P} \red {P}'} {{{P} | {Q}} \red {{P}' | {Q}}}
  \and
  \inferrule* [lab=Equiv]{{{P} \scong {P}'} \andalso {{P}' \red {Q}'} \andalso {{Q}' \scong {Q}}}{{P} \red {Q}}
\end{mathpar}

\begin{eqnarray*}
  match_{\equiv} (\quotep{P},\quotep{Q}) & := & P \equiv Q \\
  match_{\dagger}(\quotep{P},\quotep{Q}) & := & \forall R. P|Q \red^{*} R => R \red^{*} 0 \\
  match_{K}(\quotep{P},\quotep{Q}) & := & K \mbox{ for some context } K
\end{eqnarray*}

$u?(x)P | u!\langle Q \rangle \red P\{\quotep{Q}/x\}$

%We write $\wred$ for $\red^*$, and $P\red$ if $\exists Q $ such that $ P \red Q$.
We write $P\red$ if $\exists Q $ such that $ P \red Q$ and $P\not\red$, otherwise.

\section{Replication}

As mentioned before, it is known that replication (and hence
recursion) can be implemented in a higher-order process algebra
\cite{SangiorgiWalker}. As our first example of calculation with the
machinery thus far presented we give the construction explicitly in
the {\rhoc}.

\begin{eqnarray}
	D_{x} & := & \prefix{x}{y}{(\binpar{\outputp{x}{y}}{@{y}})} \nonumber\\
	\bangp_{x}{P} & := & \binpar{{x}!\langle{\binpar{D_{x}}{P}}\rangle}{D_{x}} \nonumber
\end{eqnarray}

\begin{eqnarray}
	\bangp_{x}{P} & & \nonumber\\
	=
	& {x}!\langle{(\prefix{x}{y}{(\outputp{x}{y} | @{y})) | P}}\rangle 
	      | \prefix{x}{y}{(\outputp{x}{y} | @{y})} & \nonumber\\
	\red
	& (\outputp{x}{y} | @{y})\substn{\quotep{(\prefix{x}{y}{(@{y} | \outputp{x}{y})) | P}}}{y} & \nonumber\\
	=
	& \outputp{x}{\quotep{(\prefix{x}{y}{(\outputp{x}{y} | @{y})) | P}}}
	  | {(\prefix{x}{y}{(\outputp{x}{y} | @{y})) | P}} & \nonumber\\
	\red
	& \ldots & \nonumber\\
	\red^*
	& P | P | \ldots & \nonumber
\end{eqnarray}

Of course, this encoding, as an implementation, runs away, unfolding
$\bangp{P}$ eagerly. A lazier and more implementable replication
operator, restricted to input-guarded processes, may be obtained as follows.

\begin{eqnarray}
\bangp{\prefix{u}{v}{P}} 
	:= 
	\binpar{\lift{x}{\prefix{u}{v}{(\binpar{D(x)}{P})}}}{D(x)} \nonumber
\end{eqnarray}

\begin{remark}
  Note that the lazier definition still does not deal with summation
  or mixed summation (i.e. sums over input and output). The reader is
  invited to construct definitions of replication that deal with these
  features. 

  Further, the definitions are parameterized in a name, $x$. Can you,
  gentle reader, make a definition that eliminates this parameter and
  guarantees no accidental interaction between the replication
  machinery and the process being replicated -- i.e. no accidental
  sharing of names used by the process to get its work done and the
  name(s) used by the replication to effect copying. This latter
  revision of the definition of replication is crucial to obtaining
  the expected identity $!!P \sim !P$.
\end{remark}

\begin{remark}\label{rem:paradoxical_combinator}
  The reader familiar with the lambda calculus will have noticed the
  similarity between $D$ and the paradoxical combinator.

  [Ed. note: the existence of this seems to suggest we have to be more
  restrictive on the set of processes and names we admit if we are to
  support no-cloning.]
\end{remark}

\subsubsection{Bisimulation}

The computational dynamics gives rise to another kind of equivalence,
the equivalence of computational behavior. As previously mentioned
this is typically captured \emph{via} some form of bisimulation.

% The notion we use in this paper is weak barbed bisimulation
% \cite{milner91polyadicpi}.

The notion we use in this paper is derived from weak barbed
bisimulation \cite{milner91polyadicpi}. 

\begin{definition}
An \emph{observation relation}, $\downarrow_{\mathcal N}$, over a set
of names, $\mathcal N$, is the smallest relation satisfying the rules
below.

\infrule[Out-barb]{y \in {\mathcal N}, \; x \nameeq y}
		  {\outputp{x}{v} \downarrow_{\mathcal N} x}
\infrule[Par-barb]{\mbox{$P\downarrow_{\mathcal N} x$ or $Q\downarrow_{\mathcal N} x$}}
		  {\binpar{P}{Q} \downarrow_{\mathcal N} x}

We write $P \Downarrow_{\mathcal N} x$ if there is $Q$ such that 
$P \wred Q$ and $Q \downarrow_{\mathcal N} x$.
\end{definition}

\begin{definition}
%\label{def.bbisim}
An  ${\mathcal N}$-\emph{barbed bisimulation} over a set of names, ${\mathcal N}$, is a symmetric binary relation 
${\mathcal S}_{\mathcal N}$ between agents such that $P\rel{S}_{\mathcal N}Q$ implies:
\begin{enumerate}
\item If $P \red P'$ then $Q \wred Q'$ and $P'\rel{S}_{\mathcal N} Q'$.
\item If $P\downarrow_{\mathcal N} x$, then $Q\Downarrow_{\mathcal N} x$.
\end{enumerate}
$P$ is ${\mathcal N}$-barbed bisimilar to $Q$, written
$P \wbbisim_{\mathcal N} Q$, if $P \rel{S}_{\mathcal N} Q$ for some ${\mathcal N}$-barbed bisimulation ${\mathcal S}_{\mathcal N}$.
\end{definition}

$\mathcal{R} \subseteq \pi \times \pi$

$P \mathcal{R} Q => \forall P'. P \red P' \Rightarrow \exists Q'. Q \red Q', P' \mathcal{R} Q'$

$P \vdash x \Rightarrow Q \vdash x$

\begin{mathpar}
  \inferrule*[lab=Out-barb]{x \nameeq y}{{y}!\langle{Q}\rangle \vdash x}
  \and
  \inferrule*[lab=Par-barb]{\mbox{$P\vdash x$ or $Q\vdash x$}}{\binpar{P}{Q} \vdash x}
\end{mathpar}

\subsubsection{Contexts}

One of the principle advantages of computational calculi like the
$\pi$-calculus is a well-defined notion of context,
contextual-equivalence and a correlation between
contextual-equivalence and notions of bisimulation. The notion of
context allows the decomposition of a process into (sub-)process and
its syntactic environment, its context. Thus, a context may be
thought of as a process with a ``hole'' (written $\Box$) in it. The
application of a context $M$ to a process $P$, written $M[P]$, is
tantamount to filling the hole in $M$ with $P$. In this paper we do
not need the full weight of this theory, but do make use of the notion
of context in the proof the main theorem. 

\begin{mathpar}
  \inferrule* [lab=summation] {} {{M_{M},M_{N}} \bc \Box \;|\; x.M_{A} \;|\; M_{M}+M_{N}}
  \and
  \inferrule* [lab=agent] {} {{M_{A}} \bc (\vec{x})M_{P} \;| \; \clift{P_0,\ldots,M_{P},\ldots,P_N}}
  \and \\
  \inferrule* [lab=process] {} {{M_{P}} \bc M_{N} \;| \;P|M_{P} }
\end{mathpar} 

\begin{mathpar}
  \inferrule* [lab=sychronization] {} {M_{N} \bc \Box \;|\; x?M_{F} \;|\; x!M_{C}}
  \and
  \inferrule* [lab=abstraction] {} {{M_{F}} \bc (x)M_{P} }
  \and
  \inferrule* [lab=concretion] {} {{M_{C}} \bc \langle M_{P} \rangle }
  \and \\
  \inferrule* [lab=process] {} {{M_{P}} \bc M_{N} \;| \;P|M_{P} }
\end{mathpar}

\begin{definition}[contextual application] Given a context $M$, and
  process $P$, we define the \emph{contextual application}, $M[P] :=
  M\{P/\Box\}$. That is, the contextual application of M to P is the
  substitution of $P$ for $\Box$ in $M$.
\end{definition}

$\meaningof{-} : L \to \mathcal{P}(\pi)$

\begin{mathpar}
  \inferrule* [lab=collection] {} {\meaningof{true} = \pi, \and \meaningof{~E} = \pi \setminus \meaningof{E}, \and \meaningof{E_{1} \& E_{2}} = \meaningof{E_{1}} \cap \meaningof{E_{2}}}
\end{mathpar}

\begin{mathpar}
  \inferrule* [lab=structure] {} {\meaningof{0} = \{ P \in \pi | P \equiv 0 \}, \and \\ \meaningof{E_1 | E_2} = \{ P \in \pi | P \equiv P_{1} | P_{2}, P_{1} \in \meaningof{E_{1}}, P_{2} \in \meaningof{E_2}\} }
\end{mathpar}

\begin{mathpar}
 \inferrule* [lab=behavior] {} {\meaningof{\langle a?b \rangle E} = \{ P \in \pi | P \equiv Q | u?(y)P', \\ \and \\\\ \and \\ \;\;\; u \in \meaningof{a}, \forall z.P'\{z/y\} \in \meaningof{E\{z/b\}}\}, \and \\ \meaningof{a!E} = \{ P \in \pi | P \equiv Q | x!\langle P' \rangle, x \in \meaningof{a} P' \in \meaningof{E}\} }
\end{mathpar}

\begin{mathpar}
 \inferrule* [lab=nominal] {} {\meaningof{\quotep{E}} = \{ \quotep{P} \in \quotep{\pi} | P \in \meaningof{E} \}, \and \meaningof{\quotep{P}} = \{ \quotep{Q} \in \quotep{\pi} | P \equiv Q \} \and \\ \meaningof{@\quotep{E}} = \{ P \in \pi | P \equiv @x, x \in \meaningof{E} \}}
\end{mathpar}

\begin{eqnarray*}
  \\
  \meaningof{-} : TS \to ST
\end{eqnarray*}

\begin{eqnarray*}
  \\
  L : TS \to ST
\end{eqnarray*}

\begin{eqnarray*}
  \\
  P \models E \iff P \in \meaningof{E}
\end{eqnarray*}

\begin{eqnarray*}
  P \approx_{L} Q \iff \forall E \in L. P \models E \iff Q \models E
\end{eqnarray*}

\begin{eqnarray*}
  P \approx_{K} Q
\end{eqnarray*}

\begin{eqnarray*}
  P \approx Q
\end{eqnarray*}

$\approx_{K} = \approx = \approx_{L}$

\subsubsection{Contextual duality}

Note that contexts extend the quotation operation to a family of
operations from processes to names. Given a context, $M$, we can
define a \emph{nominal context}, $\quotep{M}$ by $\quotep{M}[P] :=
\quotep{M[P]}$. To foreshadow what is to come we observe that these
operations enjoy a duality with processes very much like the duality
between vectors and maps from vectors to scalars.

Further, because the calculus is essentially higher-order, we have a
correspondence between contexts and processes. More specifically,
given a name $x$ and a context $M$ we can construct $M^{*}_{x}$ such
that 

\begin{mathpar}
  M^{*}_{x} | \lift{x}{P} \red M[P]
\end{mathpar}

namely,

\begin{mathpar}
  M^{*}_{x} := x?(u).M[\dropn{u}]
\end{mathpar}

The dependence of $M^{*}_{x}$ on a name makes it an abstraction, 

\begin{mathpar}
  M^{*} := (x)x?(u).M[\dropn{u}]
\end{mathpar}

\subsection{Additional notation}

It will sometimes be convenient to denote the process a name
quotes. We already have the notation $x = \quotep{P}$, but it will be
convenient to introduce an alternate notation, $\procn{x}$, when we
want to emphasize the connection to the use of the name. Note that, by
virtue of name equivalence, $\quotep{\procn{x}} \nameeq x$; so, the
notation is consistent with previous definitions.

Further, because names have structure it is possible to effect
substitutions on the basis of that structure. This means we need to
upgrade our notation for substitutions, which we accomplish by
adapting comprehension notation. Thus,

\begin{mathpar}
  P\{ y / x : x \in S \}
\end{mathpar}

is interpreted to mean the process derived from P by replacing (in a
capture-avoiding manner) each occurrence of $x$ in $S$ by $y$. For example,

\begin{mathpar}
  P\{ \quotep{\procn{x}|\procn{x}} / x : x \in \freenames{P} \}
\end{mathpar}

will replace each (occurrence) of a free name $x$ in $P$ by
$\quotep{\procn{x}|\procn{x}}$.

Also, we will avail ourselves of the notation $x^{L}$ and $x^{R}$ to
denote injections of a name into disjoint copies of the name
space. There are numerous ways to accomplish this. One example can be
found in \cite{MeredithR05}. This notation overloads to vectors of
names: $\vec{x}^{\pi} := (x_{i}^{\pi} \; : \; 0 \leq i < |\vec{x}| )$ where $\pi \in \{L,R\}$.

We also use $P^{\Box} := P|\Box$.

In \cite{MeredithR05} an interpretation of the new operator is
given. It turns out that there are several possible interpretations
all enjoying the requisite algebraic properties of the operator (see
\cite{milner91polyadicpi}). We will therefore make liberal use of
$(\nu\; \vec{x})P$.

% subsection the_syntax_and_semantics_of_the_notation_system (end)   

\input{qm2pi.qmops} 

\input{qm2pi.sterngerlach} 

\input{qm2pi.metric} 

% section concurrent_process_calculi (end)

%\input{qm2pi.proofsketch}

% section proof sketch (end)

%\input{qm2pi.slviaknots} 

% section spatial logic via knots (end)

\input{qm2pi.conclusion}

% section conclusion (end)

%\input{qm2pi.dtcodes} 

% section wiring algorithm (end)

\input{qm2pi.ack} 

% section acknowledgments (end)

\newpage


\bibliographystyle{plain}   
\bibliography{../../biblios/main.bib}

\input{qm2pi.rhodetails}

\end{document}

 

%\documentclass[12pt]{llncs}
%\documentclass{jktr}

\usepackage[pdftex]{hyperref}                   
\usepackage {listings}
\usepackage {mathpartir}
\usepackage{bcprules}
%\usepackage{listings}
                       
\usepackage{graphicx} 
%\usepackage[margins=2.5cm,nohead,nofoot]{geometry}
%\usepackage{geometry}
\usepackage{amsfonts}
\usepackage{amstext}
\usepackage{latexsym}
\usepackage{amssymb}
\usepackage{color}


%\include{myPreamble}
\include{qm2pi.local} 

%\ifpdf
%\usepackage[pdftex]{graphicx}
%\else
%\usepackage{graphicx}
%\fi

 % \ifpdf
%  \usepackage{pdfsync}
%  \if


%\title{Brief Article}
%\author{David F. Snyder}
%\author{L.G. Meredith}

%\address{Dept. of Math., Texas State University--San Marcos, San Marcos, TX 78666}
       
\pagestyle{empty}


\begin{document}

\lstset{language=[Objective]Caml,frame=shadowbox}

\input{qm2pi.front}

% section front matter (end)

\input{qm2pi.intro} 
 
% section introduction (end)

% \input{qm2pi.knotations} 

% section notation (end)

\input{qm2pi.process.calculi} 

% section concurrent_process_calculi_and_spatial_logics_ (end)
    
%\input{qm2pi.knots2pi} 

%\input{qm2pi.trefoil} 

%\input{qm2pi.mainthm} 

% subsection basic_interpretation (end)

%\input{qm2pi.rho.presentation} 
\subsection{The syntax and semantics of the notation system}\label{sub:the_syntax_and_semantics_of_the_notation_system} % (fold)

We now summarize a technical presentation of the calculus that
embodies our theory of dynamics. The typical presentation of such a
calculus follows the style of giving generators and relations on
them. The grammar, below, describing term constructors, freely
generates the set of processes, $\Proc$. This set is then quotiented
by a relation known as structural congruence and it is over this set
that the notion of dynamics is expressed. This presentation is
essentially that of \cite{MeredithR05} with the addition of
polyadicity and summation. For readability we have relegated some of
the technical subtleties to an appendix.

\subsubsection{Process grammar}\label{subsub:process_grammar}

\begin{mathpar}
  \inferrule* [lab=synchronization] {} {{M} \bc \pzero \;|\; x?F \;|\; x!C }
  \and
  \inferrule* [lab=abstraction] {} {{F} \bc (x)P}
  \and
  \inferrule* [lab=concretion] {} {{C} \bc \langle Q \rangle}
  \and
  \inferrule* [lab=process] {} {{P,Q} \bc M \;| \;P|Q \;|\; @{x}}
  \and
  \inferrule* [lab=name] {} {{x} \bc \quotep{P}}
\end{mathpar} 

Note that $\vec{x}$ (resp. $\vec{P}$) denotes a vector of names
(resp. processes) of length $|\vec{x}|$ (resp. $|\vec{P}|$). We adopt
the following useful abbreviations.

\begin{mathpar}
   x?(\vec{y}).P := x.(\vec{y})P \and  x\clift{\vec{P}} := x.\clift{\vec{P}}
   \and x!(y) := \lift{x}{\dropn{y}}
   \and \Pi_{i=0}^{n-1}P_i := P_0 | \ldots | P_{n-1}
\end{mathpar}

\subsubsection{Structural congruence}

\paragraph{Free and bound names and alpha-equivalence.} At the
core of structural equivalence is alpha-equivalence which identifies
process that are the same up to a change of variable. Formally, we
recognize the distinction between free and bound names. The free names
of a process, $\freenames{P}$, may be calculated recursively as
follows:

\begin{mathpar}
\freenames{\pzero} := \emptyset
  \and \\
  \freenames{x?(y).P} := \{ x \} \cup (\freenames{P} \setminus \{ y \})
  \and 
  \freenames{x!\langle P \rangle} := \{ x \} \cup \{ P \} 
  \and \\
  \freenames{P|Q} := \freenames{P} \cup \freenames{Q}
  \and \\
  \freenames{@{x}} := \{ x \}
\end{mathpar}

$\pi$
$\quotep{\pi}$

$\freenames{-} : \pi \to \mathcal{P}(\quotep{\pi})$

\begin{eqnarray*}
  \freenames{\pzero} & := & \emptyset \\
  \freenames{x?(y).P} & := & \{ x \} \cup (\freenames{P} \setminus \{ y \}) \\
  \freenames{x!\langle P \rangle} & := & \{ x \} \cup \{ P \} \\
  \freenames{P|Q} & := & \freenames{P} \cup \freenames{Q} \\
  \freenames{\dropn{x}} & := & \{ x \}
\end{eqnarray*}

The bound names of a process, $\boundnames{P}$, are those names occurring in $P$
that are not free. For example, in $x?(y).0$, the name $x$ is free, while $y$ is bound.

\begin{mathpar}
  \inferrule* [lab=monoidal-laws] {} { P|Q \equiv Q|P \and P|0 \equiv P \and P|(Q|R) \equiv (P|Q)|R }
\end{mathpar}

\begin{mathpar}
  \inferrule* [lab=alpha-equivalence] {} { (x)P \equiv (y)P\{y/x\} \and y \not\in \freenames{P} }
\end{mathpar}

\begin{definition}
Then two processes, $P,Q$, are alpha-equivalent if $P = Q\{\vec{y}/\vec{x}\}$ for
some $\vec{x} \in \boundnames{Q},\vec{y} \in \boundnames{P}$, where $Q\{\vec{y}/\vec{x}\}$
denotes the capture-avoiding substitution of $\vec{y}$ for $\vec{x}$ in $Q$.
\end{definition}

\begin{definition}
  The {\em structural congruence} \cite{SangiorgiWalker} , $\equiv$,
  between processes is the least congruence containing
  alpha-equivalence, satisfying the abelian monoid laws
  (associativity, commutativity and $\pzero$ as identity) for parallel
  composition $|$ and for summation $+$.
\end{definition}

\subsection{Name equivalence}

We take name equivalence, written $\nameeq$, to be the smallest
equivalence relation generated by the following rules.

\begin{mathpar}
\inferrule*[lab=Quote-drop]
{ }
{ \quotep{@{x}} \nameeq x }

\inferrule*[lab=Struct-equiv]
{ P \scong Q }
{ \quotep{P} \nameeq \quotep{Q} }
\end{mathpar}

The astute reader will have noticed that the mutual recursion of names
and processes imposes a mutual recursion on alpha-equivalence and
structural equivalence via name-equivalence. Fortunately, all of this
works out pleasantly and we may calculate in the natural way, free of
concern. The reader interested in the details is referred to the
appendix \ref{appendix:rho_details}.

\subsection{Substitution}

We use $\Proc$ for the set of processes, $\QProc$ for the set of
names, and $\id{\{}\vec{y} / \vec{x} \id{\}}$ to denote partial maps,
$s : \QProc \rightarrow \QProc$. A map, $s$ lifts, uniquely, to a map
on process terms, $\widehat{s} : \Proc \rightarrow \Proc$ by the
following equations.

\begin{mathpar}
  (0) \psubstp{Q}{P} := 0 \\
  (R \juxtap S) \psubstp{Q}{P}
  :=    
  (R)\psubstp{Q}{P} \juxtap (S) \psubstp{Q}{P} \\
  (x?(y).R) \psubstp{Q}{P}    
  :=    
  (x)\substp{Q}{P} (z)\concat( (R \psubstn{z}{y}) \psubstp{Q}{P} ) \\
  (\lift{x}{R}) \psubstp{Q}{P}  
  :=
  \lift{(x)\substp{Q}{P}}{ R \psubstp{Q}{P} } \\
%   (\dropn{x})  \psubstp{Q}{P}       
%   := 
%   \left\{ 
%     \begin{array}{ccc} 
%       \dropn{\quotep{Q}} & & x \nameeq \quotep{P} \\
%       \dropn{x} & & otherwise \\
%     \end{array}
%   \right. 
  (\dropn{x})  \psubstp{Q}{P}       
  := 
  \left\{ 
    \begin{array}{ccc} 
      Q & & x \nameeq \quotep{P} \\
      \dropn{x} & & otherwise \\
    \end{array}
  \right.
\end{mathpar}
 

where

\begin{eqnarray}
  (x)\id{\{} \lpquote Q \rpquote / \lpquote P \rpquote \id{\}}            = 
  \left\{ 
    \begin{array}{ccc}
      \lpquote Q \rpquote & & x \nameeq \lpquote P \rpquote \\
      x & & otherwise \\
    \end{array}
  \right. \nonumber
\end{eqnarray}

and $z$ is chosen distinct from $\quotep{P}$, $\quotep{Q}$, the free
names in $Q$, and all the names in $R$. Our $\alpha$-equivalence will
be built in the standard way from this substitution.

\begin{remark}\label{rem:no_self_referential_names}
  One consequence of these definitions is that $\forall P. \quotep{P}
  \not\in \freenames{P}$.
\end{remark}

\subsection{ Dynamic quote: an example }

Anticipating something of what's to come, consider applying the
substitution, $\widehat{\id{\{}u / z \id{\}}}$, to the following pair
of processes, $\lift{w}{y!(z)}$ and $w[ \lpquote y!(z) \rpquote ]$.

\begin{eqnarray}
	\lift{w}{y!(z)}\widehat{\id{\{}u / z \id{\}}}
		& = &
		\lift{w}{y!(u)} \nonumber\\
	w[ \lpquote y!(z) \rpquote ] \widehat{ \id{\{}u / z \id{\}} }
		& = &
		w[ \lpquote y!(z) \rpquote ] \nonumber
\end{eqnarray}

Because the body of the process between quotes is impervious to
substitution, we get radically different answers. In fact, by
examining the first process in an input context,
e.g. $x?(z).\lift{w}{y!(z)}$, we see that the process under the lift
operator may be shaped by prefixed inputs binding a name inside it. In
this sense, the lift operator will be seen as a way to dynamically
construct processes before reifying them as names.

Finally equipped with these standard features we can present the
dynamics of the calculus.

\subsubsection{Operational semantics} 

Finally, we introduce the computational dynamics. What marks these
algebras as distinct from other more traditionally studied algebraic
structures, e.g. vector spaces or polynomial rings, is the manner in
which dynamics is captured. In traditional structures, dynamics is typically
expressed through morphisms between such structures, as in linear maps
between vector spaces or morphisms between rings. In algebras
associated with the semantics of computation, the dynamics is
expressed as part of the algebraic structure itself, through a
reduction reduction relation typically denoted by $\red$. Below, we
give a recursive presentation of this relation for the calculus used
in the encoding.

$\red \subseteq \pi \times \pi$
$\red : \pi \to \mathcal{P}(\pi)$

\begin{mathpar}
  \inferrule* [lab=Comm] { \textsf{match}( x_{src}, x_{trgt} ) } { x_{trgt}?(y)P \; | \; x_{src}!\langle {Q} \rangle \red P\{\quotep{Q}/y}\} }
  \and \\
  \inferrule* [lab=Par] {{P} \red {P}'} {{{P} | {Q}} \red {{P}' | {Q}}}
  \and
  \inferrule* [lab=Equiv]{{{P} \scong {P}'} \andalso {{P}' \red {Q}'} \andalso {{Q}' \scong {Q}}}{{P} \red {Q}}
\end{mathpar}

\begin{eqnarray*}
  match_{\equiv} (\quotep{P},\quotep{Q}) & := & P \equiv Q \\
  match_{\dagger}(\quotep{P},\quotep{Q}) & := & \forall R. P|Q \red^{*} R => R \red^{*} 0 \\
  match_{K}(\quotep{P},\quotep{Q}) & := & K \mbox{ for some context } K
\end{eqnarray*}

$u?(x)P | u!\langle Q \rangle \red P\{\quotep{Q}/x\}$

%We write $\wred$ for $\red^*$, and $P\red$ if $\exists Q $ such that $ P \red Q$.
We write $P\red$ if $\exists Q $ such that $ P \red Q$ and $P\not\red$, otherwise.

\section{Replication}

As mentioned before, it is known that replication (and hence
recursion) can be implemented in a higher-order process algebra
\cite{SangiorgiWalker}. As our first example of calculation with the
machinery thus far presented we give the construction explicitly in
the {\rhoc}.

\begin{eqnarray}
	D_{x} & := & \prefix{x}{y}{(\binpar{\outputp{x}{y}}{@{y}})} \nonumber\\
	\bangp_{x}{P} & := & \binpar{{x}!\langle{\binpar{D_{x}}{P}}\rangle}{D_{x}} \nonumber
\end{eqnarray}

\begin{eqnarray}
	\bangp_{x}{P} & & \nonumber\\
	=
	& {x}!\langle{(\prefix{x}{y}{(\outputp{x}{y} | @{y})) | P}}\rangle 
	      | \prefix{x}{y}{(\outputp{x}{y} | @{y})} & \nonumber\\
	\red
	& (\outputp{x}{y} | @{y})\substn{\quotep{(\prefix{x}{y}{(@{y} | \outputp{x}{y})) | P}}}{y} & \nonumber\\
	=
	& \outputp{x}{\quotep{(\prefix{x}{y}{(\outputp{x}{y} | @{y})) | P}}}
	  | {(\prefix{x}{y}{(\outputp{x}{y} | @{y})) | P}} & \nonumber\\
	\red
	& \ldots & \nonumber\\
	\red^*
	& P | P | \ldots & \nonumber
\end{eqnarray}

Of course, this encoding, as an implementation, runs away, unfolding
$\bangp{P}$ eagerly. A lazier and more implementable replication
operator, restricted to input-guarded processes, may be obtained as follows.

\begin{eqnarray}
\bangp{\prefix{u}{v}{P}} 
	:= 
	\binpar{\lift{x}{\prefix{u}{v}{(\binpar{D(x)}{P})}}}{D(x)} \nonumber
\end{eqnarray}

\begin{remark}
  Note that the lazier definition still does not deal with summation
  or mixed summation (i.e. sums over input and output). The reader is
  invited to construct definitions of replication that deal with these
  features. 

  Further, the definitions are parameterized in a name, $x$. Can you,
  gentle reader, make a definition that eliminates this parameter and
  guarantees no accidental interaction between the replication
  machinery and the process being replicated -- i.e. no accidental
  sharing of names used by the process to get its work done and the
  name(s) used by the replication to effect copying. This latter
  revision of the definition of replication is crucial to obtaining
  the expected identity $!!P \sim !P$.
\end{remark}

\begin{remark}\label{rem:paradoxical_combinator}
  The reader familiar with the lambda calculus will have noticed the
  similarity between $D$ and the paradoxical combinator.

  [Ed. note: the existence of this seems to suggest we have to be more
  restrictive on the set of processes and names we admit if we are to
  support no-cloning.]
\end{remark}

\subsubsection{Bisimulation}

The computational dynamics gives rise to another kind of equivalence,
the equivalence of computational behavior. As previously mentioned
this is typically captured \emph{via} some form of bisimulation.

% The notion we use in this paper is weak barbed bisimulation
% \cite{milner91polyadicpi}.

The notion we use in this paper is derived from weak barbed
bisimulation \cite{milner91polyadicpi}. 

\begin{definition}
An \emph{observation relation}, $\downarrow_{\mathcal N}$, over a set
of names, $\mathcal N$, is the smallest relation satisfying the rules
below.

\infrule[Out-barb]{y \in {\mathcal N}, \; x \nameeq y}
		  {\outputp{x}{v} \downarrow_{\mathcal N} x}
\infrule[Par-barb]{\mbox{$P\downarrow_{\mathcal N} x$ or $Q\downarrow_{\mathcal N} x$}}
		  {\binpar{P}{Q} \downarrow_{\mathcal N} x}

We write $P \Downarrow_{\mathcal N} x$ if there is $Q$ such that 
$P \wred Q$ and $Q \downarrow_{\mathcal N} x$.
\end{definition}

\begin{definition}
%\label{def.bbisim}
An  ${\mathcal N}$-\emph{barbed bisimulation} over a set of names, ${\mathcal N}$, is a symmetric binary relation 
${\mathcal S}_{\mathcal N}$ between agents such that $P\rel{S}_{\mathcal N}Q$ implies:
\begin{enumerate}
\item If $P \red P'$ then $Q \wred Q'$ and $P'\rel{S}_{\mathcal N} Q'$.
\item If $P\downarrow_{\mathcal N} x$, then $Q\Downarrow_{\mathcal N} x$.
\end{enumerate}
$P$ is ${\mathcal N}$-barbed bisimilar to $Q$, written
$P \wbbisim_{\mathcal N} Q$, if $P \rel{S}_{\mathcal N} Q$ for some ${\mathcal N}$-barbed bisimulation ${\mathcal S}_{\mathcal N}$.
\end{definition}

$\mathcal{R} \subseteq \pi \times \pi$

$P \mathcal{R} Q => \forall P'. P \red P' \Rightarrow \exists Q'. Q \red Q', P' \mathcal{R} Q'$

$P \vdash x \Rightarrow Q \vdash x$

\begin{mathpar}
  \inferrule*[lab=Out-barb]{x \nameeq y}{{y}!\langle{Q}\rangle \vdash x}
  \and
  \inferrule*[lab=Par-barb]{\mbox{$P\vdash x$ or $Q\vdash x$}}{\binpar{P}{Q} \vdash x}
\end{mathpar}

\subsubsection{Contexts}

One of the principle advantages of computational calculi like the
$\pi$-calculus is a well-defined notion of context,
contextual-equivalence and a correlation between
contextual-equivalence and notions of bisimulation. The notion of
context allows the decomposition of a process into (sub-)process and
its syntactic environment, its context. Thus, a context may be
thought of as a process with a ``hole'' (written $\Box$) in it. The
application of a context $M$ to a process $P$, written $M[P]$, is
tantamount to filling the hole in $M$ with $P$. In this paper we do
not need the full weight of this theory, but do make use of the notion
of context in the proof the main theorem. 

\begin{mathpar}
  \inferrule* [lab=summation] {} {{M_{M},M_{N}} \bc \Box \;|\; x.M_{A} \;|\; M_{M}+M_{N}}
  \and
  \inferrule* [lab=agent] {} {{M_{A}} \bc (\vec{x})M_{P} \;| \; \clift{P_0,\ldots,M_{P},\ldots,P_N}}
  \and \\
  \inferrule* [lab=process] {} {{M_{P}} \bc M_{N} \;| \;P|M_{P} }
\end{mathpar} 

\begin{mathpar}
  \inferrule* [lab=sychronization] {} {M_{N} \bc \Box \;|\; x?M_{F} \;|\; x!M_{C}}
  \and
  \inferrule* [lab=abstraction] {} {{M_{F}} \bc (x)M_{P} }
  \and
  \inferrule* [lab=concretion] {} {{M_{C}} \bc \langle M_{P} \rangle }
  \and \\
  \inferrule* [lab=process] {} {{M_{P}} \bc M_{N} \;| \;P|M_{P} }
\end{mathpar}

\begin{definition}[contextual application] Given a context $M$, and
  process $P$, we define the \emph{contextual application}, $M[P] :=
  M\{P/\Box\}$. That is, the contextual application of M to P is the
  substitution of $P$ for $\Box$ in $M$.
\end{definition}

$\meaningof{-} : L \to \mathcal{P}(\pi)$

\begin{mathpar}
  \inferrule* [lab=collection] {} {\meaningof{true} = \pi, \and \meaningof{~E} = \pi \setminus \meaningof{E}, \and \meaningof{E_{1} \& E_{2}} = \meaningof{E_{1}} \cap \meaningof{E_{2}}}
\end{mathpar}

\begin{mathpar}
  \inferrule* [lab=structure] {} {\meaningof{0} = \{ P \in \pi | P \equiv 0 \}, \and \\ \meaningof{E_1 | E_2} = \{ P \in \pi | P \equiv P_{1} | P_{2}, P_{1} \in \meaningof{E_{1}}, P_{2} \in \meaningof{E_2}\} }
\end{mathpar}

\begin{mathpar}
 \inferrule* [lab=behavior] {} {\meaningof{\langle a?b \rangle E} = \{ P \in \pi | P \equiv Q | u?(y)P', \\ \and \\\\ \and \\ \;\;\; u \in \meaningof{a}, \forall z.P'\{z/y\} \in \meaningof{E\{z/b\}}\}, \and \\ \meaningof{a!E} = \{ P \in \pi | P \equiv Q | x!\langle P' \rangle, x \in \meaningof{a} P' \in \meaningof{E}\} }
\end{mathpar}

\begin{mathpar}
 \inferrule* [lab=nominal] {} {\meaningof{\quotep{E}} = \{ \quotep{P} \in \quotep{\pi} | P \in \meaningof{E} \}, \and \meaningof{\quotep{P}} = \{ \quotep{Q} \in \quotep{\pi} | P \equiv Q \} \and \\ \meaningof{@\quotep{E}} = \{ P \in \pi | P \equiv @x, x \in \meaningof{E} \}}
\end{mathpar}

\begin{eqnarray*}
  \\
  \meaningof{-} : TS \to ST
\end{eqnarray*}

\begin{eqnarray*}
  \\
  L : TS \to ST
\end{eqnarray*}

\begin{eqnarray*}
  \\
  P \models E \iff P \in \meaningof{E}
\end{eqnarray*}

\begin{eqnarray*}
  P \approx_{L} Q \iff \forall E \in L. P \models E \iff Q \models E
\end{eqnarray*}

\begin{eqnarray*}
  P \approx_{K} Q
\end{eqnarray*}

\begin{eqnarray*}
  P \approx Q
\end{eqnarray*}

$\approx_{K} = \approx = \approx_{L}$

\subsubsection{Contextual duality}

Note that contexts extend the quotation operation to a family of
operations from processes to names. Given a context, $M$, we can
define a \emph{nominal context}, $\quotep{M}$ by $\quotep{M}[P] :=
\quotep{M[P]}$. To foreshadow what is to come we observe that these
operations enjoy a duality with processes very much like the duality
between vectors and maps from vectors to scalars.

Further, because the calculus is essentially higher-order, we have a
correspondence between contexts and processes. More specifically,
given a name $x$ and a context $M$ we can construct $M^{*}_{x}$ such
that 

\begin{mathpar}
  M^{*}_{x} | \lift{x}{P} \red M[P]
\end{mathpar}

namely,

\begin{mathpar}
  M^{*}_{x} := x?(u).M[\dropn{u}]
\end{mathpar}

The dependence of $M^{*}_{x}$ on a name makes it an abstraction, 

\begin{mathpar}
  M^{*} := (x)x?(u).M[\dropn{u}]
\end{mathpar}

\subsection{Additional notation}

It will sometimes be convenient to denote the process a name
quotes. We already have the notation $x = \quotep{P}$, but it will be
convenient to introduce an alternate notation, $\procn{x}$, when we
want to emphasize the connection to the use of the name. Note that, by
virtue of name equivalence, $\quotep{\procn{x}} \nameeq x$; so, the
notation is consistent with previous definitions.

Further, because names have structure it is possible to effect
substitutions on the basis of that structure. This means we need to
upgrade our notation for substitutions, which we accomplish by
adapting comprehension notation. Thus,

\begin{mathpar}
  P\{ y / x : x \in S \}
\end{mathpar}

is interpreted to mean the process derived from P by replacing (in a
capture-avoiding manner) each occurrence of $x$ in $S$ by $y$. For example,

\begin{mathpar}
  P\{ \quotep{\procn{x}|\procn{x}} / x : x \in \freenames{P} \}
\end{mathpar}

will replace each (occurrence) of a free name $x$ in $P$ by
$\quotep{\procn{x}|\procn{x}}$.

Also, we will avail ourselves of the notation $x^{L}$ and $x^{R}$ to
denote injections of a name into disjoint copies of the name
space. There are numerous ways to accomplish this. One example can be
found in \cite{MeredithR05}. This notation overloads to vectors of
names: $\vec{x}^{\pi} := (x_{i}^{\pi} \; : \; 0 \leq i < |\vec{x}| )$ where $\pi \in \{L,R\}$.

We also use $P^{\Box} := P|\Box$.

In \cite{MeredithR05} an interpretation of the new operator is
given. It turns out that there are several possible interpretations
all enjoying the requisite algebraic properties of the operator (see
\cite{milner91polyadicpi}). We will therefore make liberal use of
$(\nu\; \vec{x})P$.

% subsection the_syntax_and_semantics_of_the_notation_system (end)   

\input{qm2pi.qmops} 

\input{qm2pi.sterngerlach} 

\input{qm2pi.metric} 

% section concurrent_process_calculi (end)

%\input{qm2pi.proofsketch}

% section proof sketch (end)

%\input{qm2pi.slviaknots} 

% section spatial logic via knots (end)

\input{qm2pi.conclusion}

% section conclusion (end)

%\input{qm2pi.dtcodes} 

% section wiring algorithm (end)

\input{qm2pi.ack} 

% section acknowledgments (end)

\newpage


\bibliographystyle{plain}   
\bibliography{../../biblios/main.bib}

\input{qm2pi.rhodetails}

\end{document}

 

%\documentclass[12pt]{llncs}
%\documentclass{jktr}

\usepackage[pdftex]{hyperref}                   
\usepackage {listings}
\usepackage {mathpartir}
\usepackage{bcprules}
%\usepackage{listings}
                       
\usepackage{graphicx} 
%\usepackage[margins=2.5cm,nohead,nofoot]{geometry}
%\usepackage{geometry}
\usepackage{amsfonts}
\usepackage{amstext}
\usepackage{latexsym}
\usepackage{amssymb}
\usepackage{color}


%\include{myPreamble}
\include{qm2pi.local} 

%\ifpdf
%\usepackage[pdftex]{graphicx}
%\else
%\usepackage{graphicx}
%\fi

 % \ifpdf
%  \usepackage{pdfsync}
%  \if


%\title{Brief Article}
%\author{David F. Snyder}
%\author{L.G. Meredith}

%\address{Dept. of Math., Texas State University--San Marcos, San Marcos, TX 78666}
       
\pagestyle{empty}


\begin{document}

\lstset{language=[Objective]Caml,frame=shadowbox}

\input{qm2pi.front}

% section front matter (end)

\input{qm2pi.intro} 
 
% section introduction (end)

% \input{qm2pi.knotations} 

% section notation (end)

\input{qm2pi.process.calculi} 

% section concurrent_process_calculi_and_spatial_logics_ (end)
    
%\input{qm2pi.knots2pi} 

%\input{qm2pi.trefoil} 

%\input{qm2pi.mainthm} 

% subsection basic_interpretation (end)

%\input{qm2pi.rho.presentation} 
\subsection{The syntax and semantics of the notation system}\label{sub:the_syntax_and_semantics_of_the_notation_system} % (fold)

We now summarize a technical presentation of the calculus that
embodies our theory of dynamics. The typical presentation of such a
calculus follows the style of giving generators and relations on
them. The grammar, below, describing term constructors, freely
generates the set of processes, $\Proc$. This set is then quotiented
by a relation known as structural congruence and it is over this set
that the notion of dynamics is expressed. This presentation is
essentially that of \cite{MeredithR05} with the addition of
polyadicity and summation. For readability we have relegated some of
the technical subtleties to an appendix.

\subsubsection{Process grammar}\label{subsub:process_grammar}

\begin{mathpar}
  \inferrule* [lab=synchronization] {} {{M} \bc \pzero \;|\; x?F \;|\; x!C }
  \and
  \inferrule* [lab=abstraction] {} {{F} \bc (x)P}
  \and
  \inferrule* [lab=concretion] {} {{C} \bc \langle Q \rangle}
  \and
  \inferrule* [lab=process] {} {{P,Q} \bc M \;| \;P|Q \;|\; @{x}}
  \and
  \inferrule* [lab=name] {} {{x} \bc \quotep{P}}
\end{mathpar} 

Note that $\vec{x}$ (resp. $\vec{P}$) denotes a vector of names
(resp. processes) of length $|\vec{x}|$ (resp. $|\vec{P}|$). We adopt
the following useful abbreviations.

\begin{mathpar}
   x?(\vec{y}).P := x.(\vec{y})P \and  x\clift{\vec{P}} := x.\clift{\vec{P}}
   \and x!(y) := \lift{x}{\dropn{y}}
   \and \Pi_{i=0}^{n-1}P_i := P_0 | \ldots | P_{n-1}
\end{mathpar}

\subsubsection{Structural congruence}

\paragraph{Free and bound names and alpha-equivalence.} At the
core of structural equivalence is alpha-equivalence which identifies
process that are the same up to a change of variable. Formally, we
recognize the distinction between free and bound names. The free names
of a process, $\freenames{P}$, may be calculated recursively as
follows:

\begin{mathpar}
\freenames{\pzero} := \emptyset
  \and \\
  \freenames{x?(y).P} := \{ x \} \cup (\freenames{P} \setminus \{ y \})
  \and 
  \freenames{x!\langle P \rangle} := \{ x \} \cup \{ P \} 
  \and \\
  \freenames{P|Q} := \freenames{P} \cup \freenames{Q}
  \and \\
  \freenames{@{x}} := \{ x \}
\end{mathpar}

$\pi$
$\quotep{\pi}$

$\freenames{-} : \pi \to \mathcal{P}(\quotep{\pi})$

\begin{eqnarray*}
  \freenames{\pzero} & := & \emptyset \\
  \freenames{x?(y).P} & := & \{ x \} \cup (\freenames{P} \setminus \{ y \}) \\
  \freenames{x!\langle P \rangle} & := & \{ x \} \cup \{ P \} \\
  \freenames{P|Q} & := & \freenames{P} \cup \freenames{Q} \\
  \freenames{\dropn{x}} & := & \{ x \}
\end{eqnarray*}

The bound names of a process, $\boundnames{P}$, are those names occurring in $P$
that are not free. For example, in $x?(y).0$, the name $x$ is free, while $y$ is bound.

\begin{mathpar}
  \inferrule* [lab=monoidal-laws] {} { P|Q \equiv Q|P \and P|0 \equiv P \and P|(Q|R) \equiv (P|Q)|R }
\end{mathpar}

\begin{mathpar}
  \inferrule* [lab=alpha-equivalence] {} { (x)P \equiv (y)P\{y/x\} \and y \not\in \freenames{P} }
\end{mathpar}

\begin{definition}
Then two processes, $P,Q$, are alpha-equivalent if $P = Q\{\vec{y}/\vec{x}\}$ for
some $\vec{x} \in \boundnames{Q},\vec{y} \in \boundnames{P}$, where $Q\{\vec{y}/\vec{x}\}$
denotes the capture-avoiding substitution of $\vec{y}$ for $\vec{x}$ in $Q$.
\end{definition}

\begin{definition}
  The {\em structural congruence} \cite{SangiorgiWalker} , $\equiv$,
  between processes is the least congruence containing
  alpha-equivalence, satisfying the abelian monoid laws
  (associativity, commutativity and $\pzero$ as identity) for parallel
  composition $|$ and for summation $+$.
\end{definition}

\subsection{Name equivalence}

We take name equivalence, written $\nameeq$, to be the smallest
equivalence relation generated by the following rules.

\begin{mathpar}
\inferrule*[lab=Quote-drop]
{ }
{ \quotep{@{x}} \nameeq x }

\inferrule*[lab=Struct-equiv]
{ P \scong Q }
{ \quotep{P} \nameeq \quotep{Q} }
\end{mathpar}

The astute reader will have noticed that the mutual recursion of names
and processes imposes a mutual recursion on alpha-equivalence and
structural equivalence via name-equivalence. Fortunately, all of this
works out pleasantly and we may calculate in the natural way, free of
concern. The reader interested in the details is referred to the
appendix \ref{appendix:rho_details}.

\subsection{Substitution}

We use $\Proc$ for the set of processes, $\QProc$ for the set of
names, and $\id{\{}\vec{y} / \vec{x} \id{\}}$ to denote partial maps,
$s : \QProc \rightarrow \QProc$. A map, $s$ lifts, uniquely, to a map
on process terms, $\widehat{s} : \Proc \rightarrow \Proc$ by the
following equations.

\begin{mathpar}
  (0) \psubstp{Q}{P} := 0 \\
  (R \juxtap S) \psubstp{Q}{P}
  :=    
  (R)\psubstp{Q}{P} \juxtap (S) \psubstp{Q}{P} \\
  (x?(y).R) \psubstp{Q}{P}    
  :=    
  (x)\substp{Q}{P} (z)\concat( (R \psubstn{z}{y}) \psubstp{Q}{P} ) \\
  (\lift{x}{R}) \psubstp{Q}{P}  
  :=
  \lift{(x)\substp{Q}{P}}{ R \psubstp{Q}{P} } \\
%   (\dropn{x})  \psubstp{Q}{P}       
%   := 
%   \left\{ 
%     \begin{array}{ccc} 
%       \dropn{\quotep{Q}} & & x \nameeq \quotep{P} \\
%       \dropn{x} & & otherwise \\
%     \end{array}
%   \right. 
  (\dropn{x})  \psubstp{Q}{P}       
  := 
  \left\{ 
    \begin{array}{ccc} 
      Q & & x \nameeq \quotep{P} \\
      \dropn{x} & & otherwise \\
    \end{array}
  \right.
\end{mathpar}
 

where

\begin{eqnarray}
  (x)\id{\{} \lpquote Q \rpquote / \lpquote P \rpquote \id{\}}            = 
  \left\{ 
    \begin{array}{ccc}
      \lpquote Q \rpquote & & x \nameeq \lpquote P \rpquote \\
      x & & otherwise \\
    \end{array}
  \right. \nonumber
\end{eqnarray}

and $z$ is chosen distinct from $\quotep{P}$, $\quotep{Q}$, the free
names in $Q$, and all the names in $R$. Our $\alpha$-equivalence will
be built in the standard way from this substitution.

\begin{remark}\label{rem:no_self_referential_names}
  One consequence of these definitions is that $\forall P. \quotep{P}
  \not\in \freenames{P}$.
\end{remark}

\subsection{ Dynamic quote: an example }

Anticipating something of what's to come, consider applying the
substitution, $\widehat{\id{\{}u / z \id{\}}}$, to the following pair
of processes, $\lift{w}{y!(z)}$ and $w[ \lpquote y!(z) \rpquote ]$.

\begin{eqnarray}
	\lift{w}{y!(z)}\widehat{\id{\{}u / z \id{\}}}
		& = &
		\lift{w}{y!(u)} \nonumber\\
	w[ \lpquote y!(z) \rpquote ] \widehat{ \id{\{}u / z \id{\}} }
		& = &
		w[ \lpquote y!(z) \rpquote ] \nonumber
\end{eqnarray}

Because the body of the process between quotes is impervious to
substitution, we get radically different answers. In fact, by
examining the first process in an input context,
e.g. $x?(z).\lift{w}{y!(z)}$, we see that the process under the lift
operator may be shaped by prefixed inputs binding a name inside it. In
this sense, the lift operator will be seen as a way to dynamically
construct processes before reifying them as names.

Finally equipped with these standard features we can present the
dynamics of the calculus.

\subsubsection{Operational semantics} 

Finally, we introduce the computational dynamics. What marks these
algebras as distinct from other more traditionally studied algebraic
structures, e.g. vector spaces or polynomial rings, is the manner in
which dynamics is captured. In traditional structures, dynamics is typically
expressed through morphisms between such structures, as in linear maps
between vector spaces or morphisms between rings. In algebras
associated with the semantics of computation, the dynamics is
expressed as part of the algebraic structure itself, through a
reduction reduction relation typically denoted by $\red$. Below, we
give a recursive presentation of this relation for the calculus used
in the encoding.

$\red \subseteq \pi \times \pi$
$\red : \pi \to \mathcal{P}(\pi)$

\begin{mathpar}
  \inferrule* [lab=Comm] { \textsf{match}( x_{src}, x_{trgt} ) } { x_{trgt}?(y)P \; | \; x_{src}!\langle {Q} \rangle \red P\{\quotep{Q}/y}\} }
  \and \\
  \inferrule* [lab=Par] {{P} \red {P}'} {{{P} | {Q}} \red {{P}' | {Q}}}
  \and
  \inferrule* [lab=Equiv]{{{P} \scong {P}'} \andalso {{P}' \red {Q}'} \andalso {{Q}' \scong {Q}}}{{P} \red {Q}}
\end{mathpar}

\begin{eqnarray*}
  match_{\equiv} (\quotep{P},\quotep{Q}) & := & P \equiv Q \\
  match_{\dagger}(\quotep{P},\quotep{Q}) & := & \forall R. P|Q \red^{*} R => R \red^{*} 0 \\
  match_{K}(\quotep{P},\quotep{Q}) & := & K \mbox{ for some context } K
\end{eqnarray*}

$u?(x)P | u!\langle Q \rangle \red P\{\quotep{Q}/x\}$

%We write $\wred$ for $\red^*$, and $P\red$ if $\exists Q $ such that $ P \red Q$.
We write $P\red$ if $\exists Q $ such that $ P \red Q$ and $P\not\red$, otherwise.

\section{Replication}

As mentioned before, it is known that replication (and hence
recursion) can be implemented in a higher-order process algebra
\cite{SangiorgiWalker}. As our first example of calculation with the
machinery thus far presented we give the construction explicitly in
the {\rhoc}.

\begin{eqnarray}
	D_{x} & := & \prefix{x}{y}{(\binpar{\outputp{x}{y}}{@{y}})} \nonumber\\
	\bangp_{x}{P} & := & \binpar{{x}!\langle{\binpar{D_{x}}{P}}\rangle}{D_{x}} \nonumber
\end{eqnarray}

\begin{eqnarray}
	\bangp_{x}{P} & & \nonumber\\
	=
	& {x}!\langle{(\prefix{x}{y}{(\outputp{x}{y} | @{y})) | P}}\rangle 
	      | \prefix{x}{y}{(\outputp{x}{y} | @{y})} & \nonumber\\
	\red
	& (\outputp{x}{y} | @{y})\substn{\quotep{(\prefix{x}{y}{(@{y} | \outputp{x}{y})) | P}}}{y} & \nonumber\\
	=
	& \outputp{x}{\quotep{(\prefix{x}{y}{(\outputp{x}{y} | @{y})) | P}}}
	  | {(\prefix{x}{y}{(\outputp{x}{y} | @{y})) | P}} & \nonumber\\
	\red
	& \ldots & \nonumber\\
	\red^*
	& P | P | \ldots & \nonumber
\end{eqnarray}

Of course, this encoding, as an implementation, runs away, unfolding
$\bangp{P}$ eagerly. A lazier and more implementable replication
operator, restricted to input-guarded processes, may be obtained as follows.

\begin{eqnarray}
\bangp{\prefix{u}{v}{P}} 
	:= 
	\binpar{\lift{x}{\prefix{u}{v}{(\binpar{D(x)}{P})}}}{D(x)} \nonumber
\end{eqnarray}

\begin{remark}
  Note that the lazier definition still does not deal with summation
  or mixed summation (i.e. sums over input and output). The reader is
  invited to construct definitions of replication that deal with these
  features. 

  Further, the definitions are parameterized in a name, $x$. Can you,
  gentle reader, make a definition that eliminates this parameter and
  guarantees no accidental interaction between the replication
  machinery and the process being replicated -- i.e. no accidental
  sharing of names used by the process to get its work done and the
  name(s) used by the replication to effect copying. This latter
  revision of the definition of replication is crucial to obtaining
  the expected identity $!!P \sim !P$.
\end{remark}

\begin{remark}\label{rem:paradoxical_combinator}
  The reader familiar with the lambda calculus will have noticed the
  similarity between $D$ and the paradoxical combinator.

  [Ed. note: the existence of this seems to suggest we have to be more
  restrictive on the set of processes and names we admit if we are to
  support no-cloning.]
\end{remark}

\subsubsection{Bisimulation}

The computational dynamics gives rise to another kind of equivalence,
the equivalence of computational behavior. As previously mentioned
this is typically captured \emph{via} some form of bisimulation.

% The notion we use in this paper is weak barbed bisimulation
% \cite{milner91polyadicpi}.

The notion we use in this paper is derived from weak barbed
bisimulation \cite{milner91polyadicpi}. 

\begin{definition}
An \emph{observation relation}, $\downarrow_{\mathcal N}$, over a set
of names, $\mathcal N$, is the smallest relation satisfying the rules
below.

\infrule[Out-barb]{y \in {\mathcal N}, \; x \nameeq y}
		  {\outputp{x}{v} \downarrow_{\mathcal N} x}
\infrule[Par-barb]{\mbox{$P\downarrow_{\mathcal N} x$ or $Q\downarrow_{\mathcal N} x$}}
		  {\binpar{P}{Q} \downarrow_{\mathcal N} x}

We write $P \Downarrow_{\mathcal N} x$ if there is $Q$ such that 
$P \wred Q$ and $Q \downarrow_{\mathcal N} x$.
\end{definition}

\begin{definition}
%\label{def.bbisim}
An  ${\mathcal N}$-\emph{barbed bisimulation} over a set of names, ${\mathcal N}$, is a symmetric binary relation 
${\mathcal S}_{\mathcal N}$ between agents such that $P\rel{S}_{\mathcal N}Q$ implies:
\begin{enumerate}
\item If $P \red P'$ then $Q \wred Q'$ and $P'\rel{S}_{\mathcal N} Q'$.
\item If $P\downarrow_{\mathcal N} x$, then $Q\Downarrow_{\mathcal N} x$.
\end{enumerate}
$P$ is ${\mathcal N}$-barbed bisimilar to $Q$, written
$P \wbbisim_{\mathcal N} Q$, if $P \rel{S}_{\mathcal N} Q$ for some ${\mathcal N}$-barbed bisimulation ${\mathcal S}_{\mathcal N}$.
\end{definition}

$\mathcal{R} \subseteq \pi \times \pi$

$P \mathcal{R} Q => \forall P'. P \red P' \Rightarrow \exists Q'. Q \red Q', P' \mathcal{R} Q'$

$P \vdash x \Rightarrow Q \vdash x$

\begin{mathpar}
  \inferrule*[lab=Out-barb]{x \nameeq y}{{y}!\langle{Q}\rangle \vdash x}
  \and
  \inferrule*[lab=Par-barb]{\mbox{$P\vdash x$ or $Q\vdash x$}}{\binpar{P}{Q} \vdash x}
\end{mathpar}

\subsubsection{Contexts}

One of the principle advantages of computational calculi like the
$\pi$-calculus is a well-defined notion of context,
contextual-equivalence and a correlation between
contextual-equivalence and notions of bisimulation. The notion of
context allows the decomposition of a process into (sub-)process and
its syntactic environment, its context. Thus, a context may be
thought of as a process with a ``hole'' (written $\Box$) in it. The
application of a context $M$ to a process $P$, written $M[P]$, is
tantamount to filling the hole in $M$ with $P$. In this paper we do
not need the full weight of this theory, but do make use of the notion
of context in the proof the main theorem. 

\begin{mathpar}
  \inferrule* [lab=summation] {} {{M_{M},M_{N}} \bc \Box \;|\; x.M_{A} \;|\; M_{M}+M_{N}}
  \and
  \inferrule* [lab=agent] {} {{M_{A}} \bc (\vec{x})M_{P} \;| \; \clift{P_0,\ldots,M_{P},\ldots,P_N}}
  \and \\
  \inferrule* [lab=process] {} {{M_{P}} \bc M_{N} \;| \;P|M_{P} }
\end{mathpar} 

\begin{mathpar}
  \inferrule* [lab=sychronization] {} {M_{N} \bc \Box \;|\; x?M_{F} \;|\; x!M_{C}}
  \and
  \inferrule* [lab=abstraction] {} {{M_{F}} \bc (x)M_{P} }
  \and
  \inferrule* [lab=concretion] {} {{M_{C}} \bc \langle M_{P} \rangle }
  \and \\
  \inferrule* [lab=process] {} {{M_{P}} \bc M_{N} \;| \;P|M_{P} }
\end{mathpar}

\begin{definition}[contextual application] Given a context $M$, and
  process $P$, we define the \emph{contextual application}, $M[P] :=
  M\{P/\Box\}$. That is, the contextual application of M to P is the
  substitution of $P$ for $\Box$ in $M$.
\end{definition}

$\meaningof{-} : L \to \mathcal{P}(\pi)$

\begin{mathpar}
  \inferrule* [lab=collection] {} {\meaningof{true} = \pi, \and \meaningof{~E} = \pi \setminus \meaningof{E}, \and \meaningof{E_{1} \& E_{2}} = \meaningof{E_{1}} \cap \meaningof{E_{2}}}
\end{mathpar}

\begin{mathpar}
  \inferrule* [lab=structure] {} {\meaningof{0} = \{ P \in \pi | P \equiv 0 \}, \and \\ \meaningof{E_1 | E_2} = \{ P \in \pi | P \equiv P_{1} | P_{2}, P_{1} \in \meaningof{E_{1}}, P_{2} \in \meaningof{E_2}\} }
\end{mathpar}

\begin{mathpar}
 \inferrule* [lab=behavior] {} {\meaningof{\langle a?b \rangle E} = \{ P \in \pi | P \equiv Q | u?(y)P', \\ \and \\\\ \and \\ \;\;\; u \in \meaningof{a}, \forall z.P'\{z/y\} \in \meaningof{E\{z/b\}}\}, \and \\ \meaningof{a!E} = \{ P \in \pi | P \equiv Q | x!\langle P' \rangle, x \in \meaningof{a} P' \in \meaningof{E}\} }
\end{mathpar}

\begin{mathpar}
 \inferrule* [lab=nominal] {} {\meaningof{\quotep{E}} = \{ \quotep{P} \in \quotep{\pi} | P \in \meaningof{E} \}, \and \meaningof{\quotep{P}} = \{ \quotep{Q} \in \quotep{\pi} | P \equiv Q \} \and \\ \meaningof{@\quotep{E}} = \{ P \in \pi | P \equiv @x, x \in \meaningof{E} \}}
\end{mathpar}

\begin{eqnarray*}
  \\
  \meaningof{-} : TS \to ST
\end{eqnarray*}

\begin{eqnarray*}
  \\
  L : TS \to ST
\end{eqnarray*}

\begin{eqnarray*}
  \\
  P \models E \iff P \in \meaningof{E}
\end{eqnarray*}

\begin{eqnarray*}
  P \approx_{L} Q \iff \forall E \in L. P \models E \iff Q \models E
\end{eqnarray*}

\begin{eqnarray*}
  P \approx_{K} Q
\end{eqnarray*}

\begin{eqnarray*}
  P \approx Q
\end{eqnarray*}

$\approx_{K} = \approx = \approx_{L}$

\subsubsection{Contextual duality}

Note that contexts extend the quotation operation to a family of
operations from processes to names. Given a context, $M$, we can
define a \emph{nominal context}, $\quotep{M}$ by $\quotep{M}[P] :=
\quotep{M[P]}$. To foreshadow what is to come we observe that these
operations enjoy a duality with processes very much like the duality
between vectors and maps from vectors to scalars.

Further, because the calculus is essentially higher-order, we have a
correspondence between contexts and processes. More specifically,
given a name $x$ and a context $M$ we can construct $M^{*}_{x}$ such
that 

\begin{mathpar}
  M^{*}_{x} | \lift{x}{P} \red M[P]
\end{mathpar}

namely,

\begin{mathpar}
  M^{*}_{x} := x?(u).M[\dropn{u}]
\end{mathpar}

The dependence of $M^{*}_{x}$ on a name makes it an abstraction, 

\begin{mathpar}
  M^{*} := (x)x?(u).M[\dropn{u}]
\end{mathpar}

\subsection{Additional notation}

It will sometimes be convenient to denote the process a name
quotes. We already have the notation $x = \quotep{P}$, but it will be
convenient to introduce an alternate notation, $\procn{x}$, when we
want to emphasize the connection to the use of the name. Note that, by
virtue of name equivalence, $\quotep{\procn{x}} \nameeq x$; so, the
notation is consistent with previous definitions.

Further, because names have structure it is possible to effect
substitutions on the basis of that structure. This means we need to
upgrade our notation for substitutions, which we accomplish by
adapting comprehension notation. Thus,

\begin{mathpar}
  P\{ y / x : x \in S \}
\end{mathpar}

is interpreted to mean the process derived from P by replacing (in a
capture-avoiding manner) each occurrence of $x$ in $S$ by $y$. For example,

\begin{mathpar}
  P\{ \quotep{\procn{x}|\procn{x}} / x : x \in \freenames{P} \}
\end{mathpar}

will replace each (occurrence) of a free name $x$ in $P$ by
$\quotep{\procn{x}|\procn{x}}$.

Also, we will avail ourselves of the notation $x^{L}$ and $x^{R}$ to
denote injections of a name into disjoint copies of the name
space. There are numerous ways to accomplish this. One example can be
found in \cite{MeredithR05}. This notation overloads to vectors of
names: $\vec{x}^{\pi} := (x_{i}^{\pi} \; : \; 0 \leq i < |\vec{x}| )$ where $\pi \in \{L,R\}$.

We also use $P^{\Box} := P|\Box$.

In \cite{MeredithR05} an interpretation of the new operator is
given. It turns out that there are several possible interpretations
all enjoying the requisite algebraic properties of the operator (see
\cite{milner91polyadicpi}). We will therefore make liberal use of
$(\nu\; \vec{x})P$.

% subsection the_syntax_and_semantics_of_the_notation_system (end)   

\input{qm2pi.qmops} 

\input{qm2pi.sterngerlach} 

\input{qm2pi.metric} 

% section concurrent_process_calculi (end)

%\input{qm2pi.proofsketch}

% section proof sketch (end)

%\input{qm2pi.slviaknots} 

% section spatial logic via knots (end)

\input{qm2pi.conclusion}

% section conclusion (end)

%\input{qm2pi.dtcodes} 

% section wiring algorithm (end)

\input{qm2pi.ack} 

% section acknowledgments (end)

\newpage


\bibliographystyle{plain}   
\bibliography{../../biblios/main.bib}

\input{qm2pi.rhodetails}

\end{document}

 

% subsection basic_interpretation (end)

%\input{qm2pi.rho.presentation} 
\subsection{The syntax and semantics of the notation system}\label{sub:the_syntax_and_semantics_of_the_notation_system} % (fold)

We now summarize a technical presentation of the calculus that
embodies our theory of dynamics. The typical presentation of such a
calculus follows the style of giving generators and relations on
them. The grammar, below, describing term constructors, freely
generates the set of processes, $\Proc$. This set is then quotiented
by a relation known as structural congruence and it is over this set
that the notion of dynamics is expressed. This presentation is
essentially that of \cite{MeredithR05} with the addition of
polyadicity and summation. For readability we have relegated some of
the technical subtleties to an appendix.

\subsubsection{Process grammar}\label{subsub:process_grammar}

\begin{mathpar}
  \inferrule* [lab=synchronization] {} {{M} \bc \pzero \;|\; x?F \;|\; x!C }
  \and
  \inferrule* [lab=abstraction] {} {{F} \bc (x)P}
  \and
  \inferrule* [lab=concretion] {} {{C} \bc \langle Q \rangle}
  \and
  \inferrule* [lab=process] {} {{P,Q} \bc M \;| \;P|Q \;|\; @{x}}
  \and
  \inferrule* [lab=name] {} {{x} \bc \quotep{P}}
\end{mathpar} 

Note that $\vec{x}$ (resp. $\vec{P}$) denotes a vector of names
(resp. processes) of length $|\vec{x}|$ (resp. $|\vec{P}|$). We adopt
the following useful abbreviations.

\begin{mathpar}
   x?(\vec{y}).P := x.(\vec{y})P \and  x\clift{\vec{P}} := x.\clift{\vec{P}}
   \and x!(y) := \lift{x}{\dropn{y}}
   \and \Pi_{i=0}^{n-1}P_i := P_0 | \ldots | P_{n-1}
\end{mathpar}

\subsubsection{Structural congruence}

\paragraph{Free and bound names and alpha-equivalence.} At the
core of structural equivalence is alpha-equivalence which identifies
process that are the same up to a change of variable. Formally, we
recognize the distinction between free and bound names. The free names
of a process, $\freenames{P}$, may be calculated recursively as
follows:

\begin{mathpar}
\freenames{\pzero} := \emptyset
  \and \\
  \freenames{x?(y).P} := \{ x \} \cup (\freenames{P} \setminus \{ y \})
  \and 
  \freenames{x!\langle P \rangle} := \{ x \} \cup \{ P \} 
  \and \\
  \freenames{P|Q} := \freenames{P} \cup \freenames{Q}
  \and \\
  \freenames{@{x}} := \{ x \}
\end{mathpar}

$\pi$
$\quotep{\pi}$

$\freenames{-} : \pi \to \mathcal{P}(\quotep{\pi})$

\begin{eqnarray*}
  \freenames{\pzero} & := & \emptyset \\
  \freenames{x?(y).P} & := & \{ x \} \cup (\freenames{P} \setminus \{ y \}) \\
  \freenames{x!\langle P \rangle} & := & \{ x \} \cup \{ P \} \\
  \freenames{P|Q} & := & \freenames{P} \cup \freenames{Q} \\
  \freenames{\dropn{x}} & := & \{ x \}
\end{eqnarray*}

The bound names of a process, $\boundnames{P}$, are those names occurring in $P$
that are not free. For example, in $x?(y).0$, the name $x$ is free, while $y$ is bound.

\begin{mathpar}
  \inferrule* [lab=monoidal-laws] {} { P|Q \equiv Q|P \and P|0 \equiv P \and P|(Q|R) \equiv (P|Q)|R }
\end{mathpar}

\begin{mathpar}
  \inferrule* [lab=alpha-equivalence] {} { (x)P \equiv (y)P\{y/x\} \and y \not\in \freenames{P} }
\end{mathpar}

\begin{definition}
Then two processes, $P,Q$, are alpha-equivalent if $P = Q\{\vec{y}/\vec{x}\}$ for
some $\vec{x} \in \boundnames{Q},\vec{y} \in \boundnames{P}$, where $Q\{\vec{y}/\vec{x}\}$
denotes the capture-avoiding substitution of $\vec{y}$ for $\vec{x}$ in $Q$.
\end{definition}

\begin{definition}
  The {\em structural congruence} \cite{SangiorgiWalker} , $\equiv$,
  between processes is the least congruence containing
  alpha-equivalence, satisfying the abelian monoid laws
  (associativity, commutativity and $\pzero$ as identity) for parallel
  composition $|$ and for summation $+$.
\end{definition}

\subsection{Name equivalence}

We take name equivalence, written $\nameeq$, to be the smallest
equivalence relation generated by the following rules.

\begin{mathpar}
\inferrule*[lab=Quote-drop]
{ }
{ \quotep{@{x}} \nameeq x }

\inferrule*[lab=Struct-equiv]
{ P \scong Q }
{ \quotep{P} \nameeq \quotep{Q} }
\end{mathpar}

The astute reader will have noticed that the mutual recursion of names
and processes imposes a mutual recursion on alpha-equivalence and
structural equivalence via name-equivalence. Fortunately, all of this
works out pleasantly and we may calculate in the natural way, free of
concern. The reader interested in the details is referred to the
appendix \ref{appendix:rho_details}.

\subsection{Substitution}

We use $\Proc$ for the set of processes, $\QProc$ for the set of
names, and $\id{\{}\vec{y} / \vec{x} \id{\}}$ to denote partial maps,
$s : \QProc \rightarrow \QProc$. A map, $s$ lifts, uniquely, to a map
on process terms, $\widehat{s} : \Proc \rightarrow \Proc$ by the
following equations.

\begin{mathpar}
  (0) \psubstp{Q}{P} := 0 \\
  (R \juxtap S) \psubstp{Q}{P}
  :=    
  (R)\psubstp{Q}{P} \juxtap (S) \psubstp{Q}{P} \\
  (x?(y).R) \psubstp{Q}{P}    
  :=    
  (x)\substp{Q}{P} (z)\concat( (R \psubstn{z}{y}) \psubstp{Q}{P} ) \\
  (\lift{x}{R}) \psubstp{Q}{P}  
  :=
  \lift{(x)\substp{Q}{P}}{ R \psubstp{Q}{P} } \\
%   (\dropn{x})  \psubstp{Q}{P}       
%   := 
%   \left\{ 
%     \begin{array}{ccc} 
%       \dropn{\quotep{Q}} & & x \nameeq \quotep{P} \\
%       \dropn{x} & & otherwise \\
%     \end{array}
%   \right. 
  (\dropn{x})  \psubstp{Q}{P}       
  := 
  \left\{ 
    \begin{array}{ccc} 
      Q & & x \nameeq \quotep{P} \\
      \dropn{x} & & otherwise \\
    \end{array}
  \right.
\end{mathpar}
 

where

\begin{eqnarray}
  (x)\id{\{} \lpquote Q \rpquote / \lpquote P \rpquote \id{\}}            = 
  \left\{ 
    \begin{array}{ccc}
      \lpquote Q \rpquote & & x \nameeq \lpquote P \rpquote \\
      x & & otherwise \\
    \end{array}
  \right. \nonumber
\end{eqnarray}

and $z$ is chosen distinct from $\quotep{P}$, $\quotep{Q}$, the free
names in $Q$, and all the names in $R$. Our $\alpha$-equivalence will
be built in the standard way from this substitution.

\begin{remark}\label{rem:no_self_referential_names}
  One consequence of these definitions is that $\forall P. \quotep{P}
  \not\in \freenames{P}$.
\end{remark}

\subsection{ Dynamic quote: an example }

Anticipating something of what's to come, consider applying the
substitution, $\widehat{\id{\{}u / z \id{\}}}$, to the following pair
of processes, $\lift{w}{y!(z)}$ and $w[ \lpquote y!(z) \rpquote ]$.

\begin{eqnarray}
	\lift{w}{y!(z)}\widehat{\id{\{}u / z \id{\}}}
		& = &
		\lift{w}{y!(u)} \nonumber\\
	w[ \lpquote y!(z) \rpquote ] \widehat{ \id{\{}u / z \id{\}} }
		& = &
		w[ \lpquote y!(z) \rpquote ] \nonumber
\end{eqnarray}

Because the body of the process between quotes is impervious to
substitution, we get radically different answers. In fact, by
examining the first process in an input context,
e.g. $x?(z).\lift{w}{y!(z)}$, we see that the process under the lift
operator may be shaped by prefixed inputs binding a name inside it. In
this sense, the lift operator will be seen as a way to dynamically
construct processes before reifying them as names.

Finally equipped with these standard features we can present the
dynamics of the calculus.

\subsubsection{Operational semantics} 

Finally, we introduce the computational dynamics. What marks these
algebras as distinct from other more traditionally studied algebraic
structures, e.g. vector spaces or polynomial rings, is the manner in
which dynamics is captured. In traditional structures, dynamics is typically
expressed through morphisms between such structures, as in linear maps
between vector spaces or morphisms between rings. In algebras
associated with the semantics of computation, the dynamics is
expressed as part of the algebraic structure itself, through a
reduction reduction relation typically denoted by $\red$. Below, we
give a recursive presentation of this relation for the calculus used
in the encoding.

$\red \subseteq \pi \times \pi$
$\red : \pi \to \mathcal{P}(\pi)$

\begin{mathpar}
  \inferrule* [lab=Comm] { \textsf{match}( x_{src}, x_{trgt} ) } { x_{trgt}?(y)P \; | \; x_{src}!\langle {Q} \rangle \red P\{\quotep{Q}/y}\} }
  \and \\
  \inferrule* [lab=Par] {{P} \red {P}'} {{{P} | {Q}} \red {{P}' | {Q}}}
  \and
  \inferrule* [lab=Equiv]{{{P} \scong {P}'} \andalso {{P}' \red {Q}'} \andalso {{Q}' \scong {Q}}}{{P} \red {Q}}
\end{mathpar}

\begin{eqnarray*}
  match_{\equiv} (\quotep{P},\quotep{Q}) & := & P \equiv Q \\
  match_{\dagger}(\quotep{P},\quotep{Q}) & := & \forall R. P|Q \red^{*} R => R \red^{*} 0 \\
  match_{K}(\quotep{P},\quotep{Q}) & := & K \mbox{ for some context } K
\end{eqnarray*}

$u?(x)P | u!\langle Q \rangle \red P\{\quotep{Q}/x\}$

%We write $\wred$ for $\red^*$, and $P\red$ if $\exists Q $ such that $ P \red Q$.
We write $P\red$ if $\exists Q $ such that $ P \red Q$ and $P\not\red$, otherwise.

\section{Replication}

As mentioned before, it is known that replication (and hence
recursion) can be implemented in a higher-order process algebra
\cite{SangiorgiWalker}. As our first example of calculation with the
machinery thus far presented we give the construction explicitly in
the {\rhoc}.

\begin{eqnarray}
	D_{x} & := & \prefix{x}{y}{(\binpar{\outputp{x}{y}}{@{y}})} \nonumber\\
	\bangp_{x}{P} & := & \binpar{{x}!\langle{\binpar{D_{x}}{P}}\rangle}{D_{x}} \nonumber
\end{eqnarray}

\begin{eqnarray}
	\bangp_{x}{P} & & \nonumber\\
	=
	& {x}!\langle{(\prefix{x}{y}{(\outputp{x}{y} | @{y})) | P}}\rangle 
	      | \prefix{x}{y}{(\outputp{x}{y} | @{y})} & \nonumber\\
	\red
	& (\outputp{x}{y} | @{y})\substn{\quotep{(\prefix{x}{y}{(@{y} | \outputp{x}{y})) | P}}}{y} & \nonumber\\
	=
	& \outputp{x}{\quotep{(\prefix{x}{y}{(\outputp{x}{y} | @{y})) | P}}}
	  | {(\prefix{x}{y}{(\outputp{x}{y} | @{y})) | P}} & \nonumber\\
	\red
	& \ldots & \nonumber\\
	\red^*
	& P | P | \ldots & \nonumber
\end{eqnarray}

Of course, this encoding, as an implementation, runs away, unfolding
$\bangp{P}$ eagerly. A lazier and more implementable replication
operator, restricted to input-guarded processes, may be obtained as follows.

\begin{eqnarray}
\bangp{\prefix{u}{v}{P}} 
	:= 
	\binpar{\lift{x}{\prefix{u}{v}{(\binpar{D(x)}{P})}}}{D(x)} \nonumber
\end{eqnarray}

\begin{remark}
  Note that the lazier definition still does not deal with summation
  or mixed summation (i.e. sums over input and output). The reader is
  invited to construct definitions of replication that deal with these
  features. 

  Further, the definitions are parameterized in a name, $x$. Can you,
  gentle reader, make a definition that eliminates this parameter and
  guarantees no accidental interaction between the replication
  machinery and the process being replicated -- i.e. no accidental
  sharing of names used by the process to get its work done and the
  name(s) used by the replication to effect copying. This latter
  revision of the definition of replication is crucial to obtaining
  the expected identity $!!P \sim !P$.
\end{remark}

\begin{remark}\label{rem:paradoxical_combinator}
  The reader familiar with the lambda calculus will have noticed the
  similarity between $D$ and the paradoxical combinator.

  [Ed. note: the existence of this seems to suggest we have to be more
  restrictive on the set of processes and names we admit if we are to
  support no-cloning.]
\end{remark}

\subsubsection{Bisimulation}

The computational dynamics gives rise to another kind of equivalence,
the equivalence of computational behavior. As previously mentioned
this is typically captured \emph{via} some form of bisimulation.

% The notion we use in this paper is weak barbed bisimulation
% \cite{milner91polyadicpi}.

The notion we use in this paper is derived from weak barbed
bisimulation \cite{milner91polyadicpi}. 

\begin{definition}
An \emph{observation relation}, $\downarrow_{\mathcal N}$, over a set
of names, $\mathcal N$, is the smallest relation satisfying the rules
below.

\infrule[Out-barb]{y \in {\mathcal N}, \; x \nameeq y}
		  {\outputp{x}{v} \downarrow_{\mathcal N} x}
\infrule[Par-barb]{\mbox{$P\downarrow_{\mathcal N} x$ or $Q\downarrow_{\mathcal N} x$}}
		  {\binpar{P}{Q} \downarrow_{\mathcal N} x}

We write $P \Downarrow_{\mathcal N} x$ if there is $Q$ such that 
$P \wred Q$ and $Q \downarrow_{\mathcal N} x$.
\end{definition}

\begin{definition}
%\label{def.bbisim}
An  ${\mathcal N}$-\emph{barbed bisimulation} over a set of names, ${\mathcal N}$, is a symmetric binary relation 
${\mathcal S}_{\mathcal N}$ between agents such that $P\rel{S}_{\mathcal N}Q$ implies:
\begin{enumerate}
\item If $P \red P'$ then $Q \wred Q'$ and $P'\rel{S}_{\mathcal N} Q'$.
\item If $P\downarrow_{\mathcal N} x$, then $Q\Downarrow_{\mathcal N} x$.
\end{enumerate}
$P$ is ${\mathcal N}$-barbed bisimilar to $Q$, written
$P \wbbisim_{\mathcal N} Q$, if $P \rel{S}_{\mathcal N} Q$ for some ${\mathcal N}$-barbed bisimulation ${\mathcal S}_{\mathcal N}$.
\end{definition}

$\mathcal{R} \subseteq \pi \times \pi$

$P \mathcal{R} Q => \forall P'. P \red P' \Rightarrow \exists Q'. Q \red Q', P' \mathcal{R} Q'$

$P \vdash x \Rightarrow Q \vdash x$

\begin{mathpar}
  \inferrule*[lab=Out-barb]{x \nameeq y}{{y}!\langle{Q}\rangle \vdash x}
  \and
  \inferrule*[lab=Par-barb]{\mbox{$P\vdash x$ or $Q\vdash x$}}{\binpar{P}{Q} \vdash x}
\end{mathpar}

\subsubsection{Contexts}

One of the principle advantages of computational calculi like the
$\pi$-calculus is a well-defined notion of context,
contextual-equivalence and a correlation between
contextual-equivalence and notions of bisimulation. The notion of
context allows the decomposition of a process into (sub-)process and
its syntactic environment, its context. Thus, a context may be
thought of as a process with a ``hole'' (written $\Box$) in it. The
application of a context $M$ to a process $P$, written $M[P]$, is
tantamount to filling the hole in $M$ with $P$. In this paper we do
not need the full weight of this theory, but do make use of the notion
of context in the proof the main theorem. 

\begin{mathpar}
  \inferrule* [lab=summation] {} {{M_{M},M_{N}} \bc \Box \;|\; x.M_{A} \;|\; M_{M}+M_{N}}
  \and
  \inferrule* [lab=agent] {} {{M_{A}} \bc (\vec{x})M_{P} \;| \; \clift{P_0,\ldots,M_{P},\ldots,P_N}}
  \and \\
  \inferrule* [lab=process] {} {{M_{P}} \bc M_{N} \;| \;P|M_{P} }
\end{mathpar} 

\begin{mathpar}
  \inferrule* [lab=sychronization] {} {M_{N} \bc \Box \;|\; x?M_{F} \;|\; x!M_{C}}
  \and
  \inferrule* [lab=abstraction] {} {{M_{F}} \bc (x)M_{P} }
  \and
  \inferrule* [lab=concretion] {} {{M_{C}} \bc \langle M_{P} \rangle }
  \and \\
  \inferrule* [lab=process] {} {{M_{P}} \bc M_{N} \;| \;P|M_{P} }
\end{mathpar}

\begin{definition}[contextual application] Given a context $M$, and
  process $P$, we define the \emph{contextual application}, $M[P] :=
  M\{P/\Box\}$. That is, the contextual application of M to P is the
  substitution of $P$ for $\Box$ in $M$.
\end{definition}

$\meaningof{-} : L \to \mathcal{P}(\pi)$

\begin{mathpar}
  \inferrule* [lab=collection] {} {\meaningof{true} = \pi, \and \meaningof{~E} = \pi \setminus \meaningof{E}, \and \meaningof{E_{1} \& E_{2}} = \meaningof{E_{1}} \cap \meaningof{E_{2}}}
\end{mathpar}

\begin{mathpar}
  \inferrule* [lab=structure] {} {\meaningof{0} = \{ P \in \pi | P \equiv 0 \}, \and \\ \meaningof{E_1 | E_2} = \{ P \in \pi | P \equiv P_{1} | P_{2}, P_{1} \in \meaningof{E_{1}}, P_{2} \in \meaningof{E_2}\} }
\end{mathpar}

\begin{mathpar}
 \inferrule* [lab=behavior] {} {\meaningof{\langle a?b \rangle E} = \{ P \in \pi | P \equiv Q | u?(y)P', \\ \and \\\\ \and \\ \;\;\; u \in \meaningof{a}, \forall z.P'\{z/y\} \in \meaningof{E\{z/b\}}\}, \and \\ \meaningof{a!E} = \{ P \in \pi | P \equiv Q | x!\langle P' \rangle, x \in \meaningof{a} P' \in \meaningof{E}\} }
\end{mathpar}

\begin{mathpar}
 \inferrule* [lab=nominal] {} {\meaningof{\quotep{E}} = \{ \quotep{P} \in \quotep{\pi} | P \in \meaningof{E} \}, \and \meaningof{\quotep{P}} = \{ \quotep{Q} \in \quotep{\pi} | P \equiv Q \} \and \\ \meaningof{@\quotep{E}} = \{ P \in \pi | P \equiv @x, x \in \meaningof{E} \}}
\end{mathpar}

\begin{eqnarray*}
  \\
  \meaningof{-} : TS \to ST
\end{eqnarray*}

\begin{eqnarray*}
  \\
  L : TS \to ST
\end{eqnarray*}

\begin{eqnarray*}
  \\
  P \models E \iff P \in \meaningof{E}
\end{eqnarray*}

\begin{eqnarray*}
  P \approx_{L} Q \iff \forall E \in L. P \models E \iff Q \models E
\end{eqnarray*}

\begin{eqnarray*}
  P \approx_{K} Q
\end{eqnarray*}

\begin{eqnarray*}
  P \approx Q
\end{eqnarray*}

$\approx_{K} = \approx = \approx_{L}$

\subsubsection{Contextual duality}

Note that contexts extend the quotation operation to a family of
operations from processes to names. Given a context, $M$, we can
define a \emph{nominal context}, $\quotep{M}$ by $\quotep{M}[P] :=
\quotep{M[P]}$. To foreshadow what is to come we observe that these
operations enjoy a duality with processes very much like the duality
between vectors and maps from vectors to scalars.

Further, because the calculus is essentially higher-order, we have a
correspondence between contexts and processes. More specifically,
given a name $x$ and a context $M$ we can construct $M^{*}_{x}$ such
that 

\begin{mathpar}
  M^{*}_{x} | \lift{x}{P} \red M[P]
\end{mathpar}

namely,

\begin{mathpar}
  M^{*}_{x} := x?(u).M[\dropn{u}]
\end{mathpar}

The dependence of $M^{*}_{x}$ on a name makes it an abstraction, 

\begin{mathpar}
  M^{*} := (x)x?(u).M[\dropn{u}]
\end{mathpar}

\subsection{Additional notation}

It will sometimes be convenient to denote the process a name
quotes. We already have the notation $x = \quotep{P}$, but it will be
convenient to introduce an alternate notation, $\procn{x}$, when we
want to emphasize the connection to the use of the name. Note that, by
virtue of name equivalence, $\quotep{\procn{x}} \nameeq x$; so, the
notation is consistent with previous definitions.

Further, because names have structure it is possible to effect
substitutions on the basis of that structure. This means we need to
upgrade our notation for substitutions, which we accomplish by
adapting comprehension notation. Thus,

\begin{mathpar}
  P\{ y / x : x \in S \}
\end{mathpar}

is interpreted to mean the process derived from P by replacing (in a
capture-avoiding manner) each occurrence of $x$ in $S$ by $y$. For example,

\begin{mathpar}
  P\{ \quotep{\procn{x}|\procn{x}} / x : x \in \freenames{P} \}
\end{mathpar}

will replace each (occurrence) of a free name $x$ in $P$ by
$\quotep{\procn{x}|\procn{x}}$.

Also, we will avail ourselves of the notation $x^{L}$ and $x^{R}$ to
denote injections of a name into disjoint copies of the name
space. There are numerous ways to accomplish this. One example can be
found in \cite{MeredithR05}. This notation overloads to vectors of
names: $\vec{x}^{\pi} := (x_{i}^{\pi} \; : \; 0 \leq i < |\vec{x}| )$ where $\pi \in \{L,R\}$.

We also use $P^{\Box} := P|\Box$.

In \cite{MeredithR05} an interpretation of the new operator is
given. It turns out that there are several possible interpretations
all enjoying the requisite algebraic properties of the operator (see
\cite{milner91polyadicpi}). We will therefore make liberal use of
$(\nu\; \vec{x})P$.

% subsection the_syntax_and_semantics_of_the_notation_system (end)   

\section{Interpretation of QM}
\subsection{Supporting definitions}
\subsubsection{Multiplication}
\begin{mathpar}
  \quotep{Q} \cdot \quotep{R} := \quotep{Q|R}
  \and \\
  \quotep{Q} \cdot P := P\{ \quotep{Q|R} / \quotep{R} : \quotep{R} \in \freenames{P} \}
\end{mathpar}

\paragraph{Discussion}
The first line needs little explanation. The second line says that
each free name of the process is replaced with the multiplication of
that name by the scalar. Multiplication of a scalar (name) by a state
(process) results in a process all the names of which have been `moved
over' by parallel composition with the process the scalar
quotes. There is a subtlety that the bound names have to be
manipulated so that multiplied names aren't accidentally
captured. There are many ways to achieve this.

\begin{remark}\label{rem:multiplication_identities}
  The reader is invited to verify that for all $x,y,z \in \QProc$ and $P \in \Proc$
  \begin{mathpar}
    x \cdot \quotep{0} \equiv x 
    \and
    x \cdot y \equiv y \cdot x
    \and
    x \cdot (y \cdot z) \equiv (x \cdot y) \cdot z
    \and \\
    \quotep{0} \cdot P \equiv P
    \and \\
    x \cdot (y \cdot P) \equiv (x \cdot y) \cdot P
    \and \\
    x \cdot (P|Q) \equiv (x \cdot P) | (x \cdot Q)
    \and \\    
  \end{mathpar}
\end{remark}

\subsubsection{Tensor product}

We define a tensor product on processes by structural induction.

\paragraph{Tensor of sums} First note that all summations, including
$\pzero$ and sequence, can be written $\Sigma_{i} x_{i}.A_{i} +
\Sigma_{j} x_{j}.C_{j}$, where we have grouped input-guarded processes
together and output-guarded processes together.

Thus, we can define the tensor product of two summations, $N_{1}\otimes N_{2}$, where

\begin{mathpar}
  N_{1} := \Sigma_{i} x_{i}.A_{i} + \Sigma_{j} x_{j}.C_{j}
  \and
  N_{2} := \Sigma_{i'} y_{i'}.B_{i'} + \Sigma_{j'} y_{j'}.D_{j'} 
\end{mathpar}

as follows.

\begin{mathpar}
  \Sigma_{i} x_{i}.A_{i} + \Sigma_{j} x_{j}.C_{j} \otimes \Sigma_{i'}
  y_{i'}.B_{i'} + \Sigma_{j'} y_{j'}.D_{j'} 
  \and \\
  := \; \Sigma_{i} \Sigma_{i'} \quotep{\stackrel{\vee}{x_{i}}| \stackrel{\vee}{y_{i'}}}.(A_{i}\otimes B_{i'}) \; | \; \Sigma_{i'} \Sigma_{i} \quotep{\stackrel{\vee}{y_{i'}}|\stackrel{\vee}{x_{i}}}.(B_{i'}\otimes A_{i})
  \and
  \;\; | \;\; \Sigma_{j} \Sigma_{j'} \quotep{\stackrel{\vee}{x_{j}}|\stackrel{\vee}{y_{j'}}}.(A_{j}\otimes B_{j'}) \; | \; \Sigma_{j'} \Sigma_{j} \quotep{\stackrel{\vee}{y_{j'}}|\stackrel{\vee}{x_{j}}}.(B_{j'}\otimes A_{j})
\end{mathpar}

\begin{remark}
  Do we need to $x^{L}$ and $y^{R}$ for this construction as well?
\end{remark}

\paragraph{Tensor of parallel compositions} Next, we distribute tensor
over par.

\begin{mathpar}
  P_{1}|P_{2} \otimes Q_{1}|Q_{2} := (P_{1} \otimes Q_{1}) | (P_{1}
  \otimes Q_{2}) | (P_{2} \otimes Q_{1}) | (P_{2} \otimes Q_{2})
\end{mathpar}

\paragraph{Tensor with dropped names} We treat tensor of a
process with a dropped name as parallel composition.

\begin{mathpar}
  P \otimes \dropn{x} := P | \dropn{x}
\end{mathpar}

\paragraph{Tensor of agents}

Finally, we need to define tensor on agents. Note that the definition
of tensor on normal products only tensors inputs with inputs and
outputs with outputs. Thus, we only have to define the operation on
``homogeneous'' pairings.

\begin{mathpar}
  (\vec{x})P \otimes (\vec{y})Q
  \and \\
  := (x_{0}^{L}|y_{0}^{R},\ldots,x_{0}^{L}|y_{n}^{R},\ldots,x_{m}^{L}|y_{0}^{R},\ldots,x_{m}^{L}|y_{n}^R)(P\{ \vec{x}^{L}/\vec{x}\} \otimes Q \{ \vec{y}^{R}/\vec{y}\})
  \and \\
  \clift{\vec{P}} \otimes \clift{\vec{Q}}
  \and \\
  := \clift{P_{0}\otimes Q_{0},\ldots,P_{0}\otimes Q_{n},\ldots,P_{m}\otimes Q_{0},\ldots,P_{m}\otimes Q_{n}}
\end{mathpar}

\begin{remark}
  Observe that arities of tensored abstractions matches arities of
  tensored concretions if the original arities matched. Note also that
  the length of the arities corresponds to the increase in dimension
  we see in ordinary vector space tensor product.
\end{remark}

\begin{remark}
  Operationally, this definition distributes the tensor down to
  components ``linked'' by summation. Tensor over summation is
  intriguing in that it mixes names. Moreover, as a consequence of the
  way it mixes names we have the identities for all $x \in \QProc$ and
  $P,Q \in \Proc$

  \begin{mathpar}
    (x \cdot P) \otimes Q \equiv x \cdot (P \otimes Q) \equiv P \otimes (x \cdot Q)
    \and
    P \otimes \pzero \equiv P
  \end{mathpar}

  that the reader is invited to verify.
\end{remark}

\subsubsection{Annihilation}
\begin{mathpar}
  P^{\perp} := \{ Q | \forall R. P|Q \red^{*} R \Rightarrow R \red^{*} \pzero \}
  \and \\
  P^{\underline{\perp}} := \Sigma_{Q \in P^{\perp}} \quotep{Q}?(y).(\dropn{y}|Q) | \Sigma_{Q \in P^{\perp}} \quotep{Q}\clift{\Box}
\end{mathpar}

\paragraph{Discussion} The reader will note that $P^{\perp}$ is a
\emph{set} of processes, while $P^{\underline{\perp}}$ is a
\emph{context}. We call the set $P^{\perp}$ the \emph{annihilators} of
$P$. The parallel composition of a process in the annihilators of $P$
with $P$ will result in a process, the state space of which has all
paths eventually leading to $\pzero$. Execution may endure loops; but
under reasonable conditions of fairness (naturally guaranteed under
most notions of bisimulation) such a composite process cannot get
stuck in such a loop and will, eventually pop out and terminate.

The context $P^{\underline{\perp}}$ is ready and willing to ``take the
$P$ out of'' the process to which it is applied. It will effectively
transmit the code of the process to which it is applied to one of the
annihilators and run the process against it.

\subsubsection{Evaluation}
We fix $M$ a domain of fully abstract interpretation with an equality
coincident with bisimulation. We take $\meaningof{\cdot} : \Proc \to
M$ to be the map interpreting processes and $\nmeaningof{\cdot} : \M
\to Proc$ to be the map running the other way. Then we define

\begin{mathpar}
  \int P := \nmeaningof{\meaningof{P}}
\end{mathpar}

\paragraph{Discussion}
There are many fully abstract interpretations of Milner's
$\pi$-calculus. Any of them can be used as a basis for interpreting
the reflective calculus here. Equipped with such a domain it is
largely a matter of grinding through to check that the Yoneda
construction for the normalization-by-evaluation program can be
extended to this setting.

\begin{remark}
  The reader is invited to verify that $\int (P^{\underline{\perp}}[P]) = 0$.
\end{remark}

\subsection{Quantum mechanics}

Table \ref{tbl:core_qm_op_defns} gives the core operational definitions

\begin{table}[htp]\label{tbl:core_qm_op_defns}
  \center{
    \fbox{
      \begin{tabular}{c|c}
        quantum mechanics & process calculus \\
        \hline
        scalar & $x := \quotep{P}$ \\
        state vector & $\state{P} := P$ \\
        dual & $\state{P}^{*} := \event{P^{\underline{\perp}}} := \quotep{P^{\underline{\perp}}}[-]$ \\
        matrix & $ \Sigma_{\alpha} \state{P_{\alpha}}x_{\alpha}\event{Q_{\alpha}}$ \\
        vector addition & $\state{P} + \state{Q} := \state{P | Q}$ \\
        tensor product & $\state{P} \otimes \state{Q} := \state{P \otimes Q}$ \\
        inner product & $\innerprod{P}{Q} := \quotep{\int P^{\underline{\perp}}[Q]}$ \\
      \end{tabular}
    }
  }
  \caption{QM - operational definitions}
\end{table}

where

\begin{mathpar}
  \prmatrix{P}{Q} := \fprmatrix{P}{\quotep{\pzero}}{Q}
  \and
  \fprmatrix{P}{x}{Q} := (\state{P},x,\event{Q})
  \and
  (\fprmatrix{P}{x}{Q})(\state{R}) := x \cdot \innerprod{Q}{R} \cdot \state{P}
  \and
  (\fprmatrix{P}{x}{Q})(\event{R}) := x \cdot \innerprod{R}{P} \cdot \event{Q}
\end{mathpar}

\paragraph{Discussion}
As promised: vectors (aka states) are represented as processes; duals
as contextual duals; inner product definition should be compared with
standard inner product definition for ....

\begin{remark}
  Assuming $\int (P^{\underline{\perp}}[P]) = 0$, the reader is
  invited to verify that $(\fprmatrix{P}{x}{P})(\state{P}) = x \cdot \state{P}$.
\end{remark}

\begin{remark}
  The reader is invited to verify that $\innerprod{P}{Q}$ could
  equally well have been written $\quotep{\int \stackrel{\vee}{x}}$
  where $x = \event{P^{\underline{\perp}}}(Q)$.

  One of the motivations for this remark is that there is another way
  to factor these operations. We could package up evaluation in the dual:

  \begin{mathpar}
    \state{P}^{*} := \event{\int P^{\underline{\perp}}} := \quotep{\int P^{\underline{\perp}}}[-]
  \end{mathpar}

  and then have inner product defined by
  
  \begin{mathpar}
    \innerprod{P}{Q} := \event{P}(Q)
  \end{mathpar}

  Hopefully, experience with the calculations will provide guidance on
  the best factoring.
\end{remark}

\begin{remark}
  Assuming $\int (P^{\underline{\perp}}[P]) = 0$, the reader is
  invited to verify that $\forall P,Q. (\prmatrix{0}{Q})(\state{0}) =
  \state{0}$ and dually $(\prmatrix{P}{0})(\event{0}) = \event{0}$.
\end{remark}

\begin{remark}
  i'm a little worried that i don't (yet) have proper support for
  complex conjugacy. But, the observation above may give us a
  clue. According to Abramsky, it must be the case that the scalars
  are iso to the homset of the identity for the tensor -- which the
  observation above characterizes. 

  For now, we will simply bookmark the notion with $\overline{x}$.
\end{remark}

\subsubsection{Adjointness}

We need to give a definition of $(\cdot)^{\dagger}$ for matrices. The
obvious candidate definition is
\begin{mathpar}
(\Sigma_{\alpha}\fprmatrix{P_{\alpha}}{x_{\alpha}}{Q_{\alpha}})^{\dagger}
= \Sigma_{\alpha}\fprmatrix{(Q_{\alpha}^{\underline{\perp}})^{*}}{\overline{x}_{\alpha}}{P_{\alpha}^{\underline{\perp}}} 
\end{mathpar}

But, $(Q_{\alpha}^{\underline{\perp}})^{*}$ requires a name along
which to communicate the process to achieve the context application.

\subsubsection{Basis for a basis}
If processes label states and ``addition'' of states (a.k.a. vector
addition) is interpreted as parallel composition, what corresponds to
notions of linear independence and basis? Here, we recall that Yoshida
has developed a set of \emph{combinators} for an asynchronous verison
of Milner's $\pi$-calculus. These are a finite set of processes such
any process can be expressed as parallel composition of these
combinators together with liberal uses of the new operator and
replication. We can simply give a translation of these into the
present calculus and have reasonable expectation that the property
carries over. That is, that the resultant set allows to express all
processes via parallel composition. Note, however, that there is no
new operator or replication in this calculus. As a result, we expect
that the corresponding set is actually infinite. That is, we expect
that the space is actually infinite dimensional.

\begin{remark}
  The attentive reader may be a bit concerned. Certainly, the
  collection $S$, $K$ and $I$ is a finite set of
  combinators. Shouldn't we expect to see a finite set of combinators
  for an effectively equivalent system? i am very sympathetic to this
  critique and feel it warrants full attention. On the other hand, i
  also have in mind the following analogy. The natural numbers, as a
  monoid under addition, has exactly $1$ generator, while the natural
  numbers, as a monoid under multiplication, has countably many
  generators (the primes). We observe that the application of the
  lambda calculus is much less resource sensitive than the parallel
  composition of the $\pi$-calculus. Could it be the case that we have
  an analogy of the form
  
  \begin{mathpar}
    m + n : MN :: m*n : M|N
  \end{mathpar}

  giving a similar blow up in the set of ``primes''?  This is such a
  wonderful thought that, even if it's not true, i think it's worth
  writing down.
\end{remark}
 

\documentclass[12pt]{llncs}
%\documentclass{jktr}

\usepackage[pdftex]{hyperref}                   
\usepackage {listings}
\usepackage {mathpartir}
\usepackage{bcprules}
%\usepackage{listings}
                       
\usepackage{graphicx} 
%\usepackage[margins=2.5cm,nohead,nofoot]{geometry}
%\usepackage{geometry}
\usepackage{amsfonts}
\usepackage{amstext}
\usepackage{latexsym}
\usepackage{amssymb}
\usepackage{color}


%\include{myPreamble}
\include{qm2pi.local} 

%\ifpdf
%\usepackage[pdftex]{graphicx}
%\else
%\usepackage{graphicx}
%\fi

 % \ifpdf
%  \usepackage{pdfsync}
%  \if


%\title{Brief Article}
%\author{David F. Snyder}
%\author{L.G. Meredith}

%\address{Dept. of Math., Texas State University--San Marcos, San Marcos, TX 78666}
       
\pagestyle{empty}


\begin{document}

\lstset{language=[Objective]Caml,frame=shadowbox}

\input{qm2pi.front}

% section front matter (end)

\input{qm2pi.intro} 
 
% section introduction (end)

% \input{qm2pi.knotations} 

% section notation (end)

\input{qm2pi.process.calculi} 

% section concurrent_process_calculi_and_spatial_logics_ (end)
    
%\input{qm2pi.knots2pi} 

%\input{qm2pi.trefoil} 

%\input{qm2pi.mainthm} 

% subsection basic_interpretation (end)

%\input{qm2pi.rho.presentation} 
\subsection{The syntax and semantics of the notation system}\label{sub:the_syntax_and_semantics_of_the_notation_system} % (fold)

We now summarize a technical presentation of the calculus that
embodies our theory of dynamics. The typical presentation of such a
calculus follows the style of giving generators and relations on
them. The grammar, below, describing term constructors, freely
generates the set of processes, $\Proc$. This set is then quotiented
by a relation known as structural congruence and it is over this set
that the notion of dynamics is expressed. This presentation is
essentially that of \cite{MeredithR05} with the addition of
polyadicity and summation. For readability we have relegated some of
the technical subtleties to an appendix.

\subsubsection{Process grammar}\label{subsub:process_grammar}

\begin{mathpar}
  \inferrule* [lab=synchronization] {} {{M} \bc \pzero \;|\; x?F \;|\; x!C }
  \and
  \inferrule* [lab=abstraction] {} {{F} \bc (x)P}
  \and
  \inferrule* [lab=concretion] {} {{C} \bc \langle Q \rangle}
  \and
  \inferrule* [lab=process] {} {{P,Q} \bc M \;| \;P|Q \;|\; @{x}}
  \and
  \inferrule* [lab=name] {} {{x} \bc \quotep{P}}
\end{mathpar} 

Note that $\vec{x}$ (resp. $\vec{P}$) denotes a vector of names
(resp. processes) of length $|\vec{x}|$ (resp. $|\vec{P}|$). We adopt
the following useful abbreviations.

\begin{mathpar}
   x?(\vec{y}).P := x.(\vec{y})P \and  x\clift{\vec{P}} := x.\clift{\vec{P}}
   \and x!(y) := \lift{x}{\dropn{y}}
   \and \Pi_{i=0}^{n-1}P_i := P_0 | \ldots | P_{n-1}
\end{mathpar}

\subsubsection{Structural congruence}

\paragraph{Free and bound names and alpha-equivalence.} At the
core of structural equivalence is alpha-equivalence which identifies
process that are the same up to a change of variable. Formally, we
recognize the distinction between free and bound names. The free names
of a process, $\freenames{P}$, may be calculated recursively as
follows:

\begin{mathpar}
\freenames{\pzero} := \emptyset
  \and \\
  \freenames{x?(y).P} := \{ x \} \cup (\freenames{P} \setminus \{ y \})
  \and 
  \freenames{x!\langle P \rangle} := \{ x \} \cup \{ P \} 
  \and \\
  \freenames{P|Q} := \freenames{P} \cup \freenames{Q}
  \and \\
  \freenames{@{x}} := \{ x \}
\end{mathpar}

$\pi$
$\quotep{\pi}$

$\freenames{-} : \pi \to \mathcal{P}(\quotep{\pi})$

\begin{eqnarray*}
  \freenames{\pzero} & := & \emptyset \\
  \freenames{x?(y).P} & := & \{ x \} \cup (\freenames{P} \setminus \{ y \}) \\
  \freenames{x!\langle P \rangle} & := & \{ x \} \cup \{ P \} \\
  \freenames{P|Q} & := & \freenames{P} \cup \freenames{Q} \\
  \freenames{\dropn{x}} & := & \{ x \}
\end{eqnarray*}

The bound names of a process, $\boundnames{P}$, are those names occurring in $P$
that are not free. For example, in $x?(y).0$, the name $x$ is free, while $y$ is bound.

\begin{mathpar}
  \inferrule* [lab=monoidal-laws] {} { P|Q \equiv Q|P \and P|0 \equiv P \and P|(Q|R) \equiv (P|Q)|R }
\end{mathpar}

\begin{mathpar}
  \inferrule* [lab=alpha-equivalence] {} { (x)P \equiv (y)P\{y/x\} \and y \not\in \freenames{P} }
\end{mathpar}

\begin{definition}
Then two processes, $P,Q$, are alpha-equivalent if $P = Q\{\vec{y}/\vec{x}\}$ for
some $\vec{x} \in \boundnames{Q},\vec{y} \in \boundnames{P}$, where $Q\{\vec{y}/\vec{x}\}$
denotes the capture-avoiding substitution of $\vec{y}$ for $\vec{x}$ in $Q$.
\end{definition}

\begin{definition}
  The {\em structural congruence} \cite{SangiorgiWalker} , $\equiv$,
  between processes is the least congruence containing
  alpha-equivalence, satisfying the abelian monoid laws
  (associativity, commutativity and $\pzero$ as identity) for parallel
  composition $|$ and for summation $+$.
\end{definition}

\subsection{Name equivalence}

We take name equivalence, written $\nameeq$, to be the smallest
equivalence relation generated by the following rules.

\begin{mathpar}
\inferrule*[lab=Quote-drop]
{ }
{ \quotep{@{x}} \nameeq x }

\inferrule*[lab=Struct-equiv]
{ P \scong Q }
{ \quotep{P} \nameeq \quotep{Q} }
\end{mathpar}

The astute reader will have noticed that the mutual recursion of names
and processes imposes a mutual recursion on alpha-equivalence and
structural equivalence via name-equivalence. Fortunately, all of this
works out pleasantly and we may calculate in the natural way, free of
concern. The reader interested in the details is referred to the
appendix \ref{appendix:rho_details}.

\subsection{Substitution}

We use $\Proc$ for the set of processes, $\QProc$ for the set of
names, and $\id{\{}\vec{y} / \vec{x} \id{\}}$ to denote partial maps,
$s : \QProc \rightarrow \QProc$. A map, $s$ lifts, uniquely, to a map
on process terms, $\widehat{s} : \Proc \rightarrow \Proc$ by the
following equations.

\begin{mathpar}
  (0) \psubstp{Q}{P} := 0 \\
  (R \juxtap S) \psubstp{Q}{P}
  :=    
  (R)\psubstp{Q}{P} \juxtap (S) \psubstp{Q}{P} \\
  (x?(y).R) \psubstp{Q}{P}    
  :=    
  (x)\substp{Q}{P} (z)\concat( (R \psubstn{z}{y}) \psubstp{Q}{P} ) \\
  (\lift{x}{R}) \psubstp{Q}{P}  
  :=
  \lift{(x)\substp{Q}{P}}{ R \psubstp{Q}{P} } \\
%   (\dropn{x})  \psubstp{Q}{P}       
%   := 
%   \left\{ 
%     \begin{array}{ccc} 
%       \dropn{\quotep{Q}} & & x \nameeq \quotep{P} \\
%       \dropn{x} & & otherwise \\
%     \end{array}
%   \right. 
  (\dropn{x})  \psubstp{Q}{P}       
  := 
  \left\{ 
    \begin{array}{ccc} 
      Q & & x \nameeq \quotep{P} \\
      \dropn{x} & & otherwise \\
    \end{array}
  \right.
\end{mathpar}
 

where

\begin{eqnarray}
  (x)\id{\{} \lpquote Q \rpquote / \lpquote P \rpquote \id{\}}            = 
  \left\{ 
    \begin{array}{ccc}
      \lpquote Q \rpquote & & x \nameeq \lpquote P \rpquote \\
      x & & otherwise \\
    \end{array}
  \right. \nonumber
\end{eqnarray}

and $z$ is chosen distinct from $\quotep{P}$, $\quotep{Q}$, the free
names in $Q$, and all the names in $R$. Our $\alpha$-equivalence will
be built in the standard way from this substitution.

\begin{remark}\label{rem:no_self_referential_names}
  One consequence of these definitions is that $\forall P. \quotep{P}
  \not\in \freenames{P}$.
\end{remark}

\subsection{ Dynamic quote: an example }

Anticipating something of what's to come, consider applying the
substitution, $\widehat{\id{\{}u / z \id{\}}}$, to the following pair
of processes, $\lift{w}{y!(z)}$ and $w[ \lpquote y!(z) \rpquote ]$.

\begin{eqnarray}
	\lift{w}{y!(z)}\widehat{\id{\{}u / z \id{\}}}
		& = &
		\lift{w}{y!(u)} \nonumber\\
	w[ \lpquote y!(z) \rpquote ] \widehat{ \id{\{}u / z \id{\}} }
		& = &
		w[ \lpquote y!(z) \rpquote ] \nonumber
\end{eqnarray}

Because the body of the process between quotes is impervious to
substitution, we get radically different answers. In fact, by
examining the first process in an input context,
e.g. $x?(z).\lift{w}{y!(z)}$, we see that the process under the lift
operator may be shaped by prefixed inputs binding a name inside it. In
this sense, the lift operator will be seen as a way to dynamically
construct processes before reifying them as names.

Finally equipped with these standard features we can present the
dynamics of the calculus.

\subsubsection{Operational semantics} 

Finally, we introduce the computational dynamics. What marks these
algebras as distinct from other more traditionally studied algebraic
structures, e.g. vector spaces or polynomial rings, is the manner in
which dynamics is captured. In traditional structures, dynamics is typically
expressed through morphisms between such structures, as in linear maps
between vector spaces or morphisms between rings. In algebras
associated with the semantics of computation, the dynamics is
expressed as part of the algebraic structure itself, through a
reduction reduction relation typically denoted by $\red$. Below, we
give a recursive presentation of this relation for the calculus used
in the encoding.

$\red \subseteq \pi \times \pi$
$\red : \pi \to \mathcal{P}(\pi)$

\begin{mathpar}
  \inferrule* [lab=Comm] { \textsf{match}( x_{src}, x_{trgt} ) } { x_{trgt}?(y)P \; | \; x_{src}!\langle {Q} \rangle \red P\{\quotep{Q}/y}\} }
  \and \\
  \inferrule* [lab=Par] {{P} \red {P}'} {{{P} | {Q}} \red {{P}' | {Q}}}
  \and
  \inferrule* [lab=Equiv]{{{P} \scong {P}'} \andalso {{P}' \red {Q}'} \andalso {{Q}' \scong {Q}}}{{P} \red {Q}}
\end{mathpar}

\begin{eqnarray*}
  match_{\equiv} (\quotep{P},\quotep{Q}) & := & P \equiv Q \\
  match_{\dagger}(\quotep{P},\quotep{Q}) & := & \forall R. P|Q \red^{*} R => R \red^{*} 0 \\
  match_{K}(\quotep{P},\quotep{Q}) & := & K \mbox{ for some context } K
\end{eqnarray*}

$u?(x)P | u!\langle Q \rangle \red P\{\quotep{Q}/x\}$

%We write $\wred$ for $\red^*$, and $P\red$ if $\exists Q $ such that $ P \red Q$.
We write $P\red$ if $\exists Q $ such that $ P \red Q$ and $P\not\red$, otherwise.

\section{Replication}

As mentioned before, it is known that replication (and hence
recursion) can be implemented in a higher-order process algebra
\cite{SangiorgiWalker}. As our first example of calculation with the
machinery thus far presented we give the construction explicitly in
the {\rhoc}.

\begin{eqnarray}
	D_{x} & := & \prefix{x}{y}{(\binpar{\outputp{x}{y}}{@{y}})} \nonumber\\
	\bangp_{x}{P} & := & \binpar{{x}!\langle{\binpar{D_{x}}{P}}\rangle}{D_{x}} \nonumber
\end{eqnarray}

\begin{eqnarray}
	\bangp_{x}{P} & & \nonumber\\
	=
	& {x}!\langle{(\prefix{x}{y}{(\outputp{x}{y} | @{y})) | P}}\rangle 
	      | \prefix{x}{y}{(\outputp{x}{y} | @{y})} & \nonumber\\
	\red
	& (\outputp{x}{y} | @{y})\substn{\quotep{(\prefix{x}{y}{(@{y} | \outputp{x}{y})) | P}}}{y} & \nonumber\\
	=
	& \outputp{x}{\quotep{(\prefix{x}{y}{(\outputp{x}{y} | @{y})) | P}}}
	  | {(\prefix{x}{y}{(\outputp{x}{y} | @{y})) | P}} & \nonumber\\
	\red
	& \ldots & \nonumber\\
	\red^*
	& P | P | \ldots & \nonumber
\end{eqnarray}

Of course, this encoding, as an implementation, runs away, unfolding
$\bangp{P}$ eagerly. A lazier and more implementable replication
operator, restricted to input-guarded processes, may be obtained as follows.

\begin{eqnarray}
\bangp{\prefix{u}{v}{P}} 
	:= 
	\binpar{\lift{x}{\prefix{u}{v}{(\binpar{D(x)}{P})}}}{D(x)} \nonumber
\end{eqnarray}

\begin{remark}
  Note that the lazier definition still does not deal with summation
  or mixed summation (i.e. sums over input and output). The reader is
  invited to construct definitions of replication that deal with these
  features. 

  Further, the definitions are parameterized in a name, $x$. Can you,
  gentle reader, make a definition that eliminates this parameter and
  guarantees no accidental interaction between the replication
  machinery and the process being replicated -- i.e. no accidental
  sharing of names used by the process to get its work done and the
  name(s) used by the replication to effect copying. This latter
  revision of the definition of replication is crucial to obtaining
  the expected identity $!!P \sim !P$.
\end{remark}

\begin{remark}\label{rem:paradoxical_combinator}
  The reader familiar with the lambda calculus will have noticed the
  similarity between $D$ and the paradoxical combinator.

  [Ed. note: the existence of this seems to suggest we have to be more
  restrictive on the set of processes and names we admit if we are to
  support no-cloning.]
\end{remark}

\subsubsection{Bisimulation}

The computational dynamics gives rise to another kind of equivalence,
the equivalence of computational behavior. As previously mentioned
this is typically captured \emph{via} some form of bisimulation.

% The notion we use in this paper is weak barbed bisimulation
% \cite{milner91polyadicpi}.

The notion we use in this paper is derived from weak barbed
bisimulation \cite{milner91polyadicpi}. 

\begin{definition}
An \emph{observation relation}, $\downarrow_{\mathcal N}$, over a set
of names, $\mathcal N$, is the smallest relation satisfying the rules
below.

\infrule[Out-barb]{y \in {\mathcal N}, \; x \nameeq y}
		  {\outputp{x}{v} \downarrow_{\mathcal N} x}
\infrule[Par-barb]{\mbox{$P\downarrow_{\mathcal N} x$ or $Q\downarrow_{\mathcal N} x$}}
		  {\binpar{P}{Q} \downarrow_{\mathcal N} x}

We write $P \Downarrow_{\mathcal N} x$ if there is $Q$ such that 
$P \wred Q$ and $Q \downarrow_{\mathcal N} x$.
\end{definition}

\begin{definition}
%\label{def.bbisim}
An  ${\mathcal N}$-\emph{barbed bisimulation} over a set of names, ${\mathcal N}$, is a symmetric binary relation 
${\mathcal S}_{\mathcal N}$ between agents such that $P\rel{S}_{\mathcal N}Q$ implies:
\begin{enumerate}
\item If $P \red P'$ then $Q \wred Q'$ and $P'\rel{S}_{\mathcal N} Q'$.
\item If $P\downarrow_{\mathcal N} x$, then $Q\Downarrow_{\mathcal N} x$.
\end{enumerate}
$P$ is ${\mathcal N}$-barbed bisimilar to $Q$, written
$P \wbbisim_{\mathcal N} Q$, if $P \rel{S}_{\mathcal N} Q$ for some ${\mathcal N}$-barbed bisimulation ${\mathcal S}_{\mathcal N}$.
\end{definition}

$\mathcal{R} \subseteq \pi \times \pi$

$P \mathcal{R} Q => \forall P'. P \red P' \Rightarrow \exists Q'. Q \red Q', P' \mathcal{R} Q'$

$P \vdash x \Rightarrow Q \vdash x$

\begin{mathpar}
  \inferrule*[lab=Out-barb]{x \nameeq y}{{y}!\langle{Q}\rangle \vdash x}
  \and
  \inferrule*[lab=Par-barb]{\mbox{$P\vdash x$ or $Q\vdash x$}}{\binpar{P}{Q} \vdash x}
\end{mathpar}

\subsubsection{Contexts}

One of the principle advantages of computational calculi like the
$\pi$-calculus is a well-defined notion of context,
contextual-equivalence and a correlation between
contextual-equivalence and notions of bisimulation. The notion of
context allows the decomposition of a process into (sub-)process and
its syntactic environment, its context. Thus, a context may be
thought of as a process with a ``hole'' (written $\Box$) in it. The
application of a context $M$ to a process $P$, written $M[P]$, is
tantamount to filling the hole in $M$ with $P$. In this paper we do
not need the full weight of this theory, but do make use of the notion
of context in the proof the main theorem. 

\begin{mathpar}
  \inferrule* [lab=summation] {} {{M_{M},M_{N}} \bc \Box \;|\; x.M_{A} \;|\; M_{M}+M_{N}}
  \and
  \inferrule* [lab=agent] {} {{M_{A}} \bc (\vec{x})M_{P} \;| \; \clift{P_0,\ldots,M_{P},\ldots,P_N}}
  \and \\
  \inferrule* [lab=process] {} {{M_{P}} \bc M_{N} \;| \;P|M_{P} }
\end{mathpar} 

\begin{mathpar}
  \inferrule* [lab=sychronization] {} {M_{N} \bc \Box \;|\; x?M_{F} \;|\; x!M_{C}}
  \and
  \inferrule* [lab=abstraction] {} {{M_{F}} \bc (x)M_{P} }
  \and
  \inferrule* [lab=concretion] {} {{M_{C}} \bc \langle M_{P} \rangle }
  \and \\
  \inferrule* [lab=process] {} {{M_{P}} \bc M_{N} \;| \;P|M_{P} }
\end{mathpar}

\begin{definition}[contextual application] Given a context $M$, and
  process $P$, we define the \emph{contextual application}, $M[P] :=
  M\{P/\Box\}$. That is, the contextual application of M to P is the
  substitution of $P$ for $\Box$ in $M$.
\end{definition}

$\meaningof{-} : L \to \mathcal{P}(\pi)$

\begin{mathpar}
  \inferrule* [lab=collection] {} {\meaningof{true} = \pi, \and \meaningof{~E} = \pi \setminus \meaningof{E}, \and \meaningof{E_{1} \& E_{2}} = \meaningof{E_{1}} \cap \meaningof{E_{2}}}
\end{mathpar}

\begin{mathpar}
  \inferrule* [lab=structure] {} {\meaningof{0} = \{ P \in \pi | P \equiv 0 \}, \and \\ \meaningof{E_1 | E_2} = \{ P \in \pi | P \equiv P_{1} | P_{2}, P_{1} \in \meaningof{E_{1}}, P_{2} \in \meaningof{E_2}\} }
\end{mathpar}

\begin{mathpar}
 \inferrule* [lab=behavior] {} {\meaningof{\langle a?b \rangle E} = \{ P \in \pi | P \equiv Q | u?(y)P', \\ \and \\\\ \and \\ \;\;\; u \in \meaningof{a}, \forall z.P'\{z/y\} \in \meaningof{E\{z/b\}}\}, \and \\ \meaningof{a!E} = \{ P \in \pi | P \equiv Q | x!\langle P' \rangle, x \in \meaningof{a} P' \in \meaningof{E}\} }
\end{mathpar}

\begin{mathpar}
 \inferrule* [lab=nominal] {} {\meaningof{\quotep{E}} = \{ \quotep{P} \in \quotep{\pi} | P \in \meaningof{E} \}, \and \meaningof{\quotep{P}} = \{ \quotep{Q} \in \quotep{\pi} | P \equiv Q \} \and \\ \meaningof{@\quotep{E}} = \{ P \in \pi | P \equiv @x, x \in \meaningof{E} \}}
\end{mathpar}

\begin{eqnarray*}
  \\
  \meaningof{-} : TS \to ST
\end{eqnarray*}

\begin{eqnarray*}
  \\
  L : TS \to ST
\end{eqnarray*}

\begin{eqnarray*}
  \\
  P \models E \iff P \in \meaningof{E}
\end{eqnarray*}

\begin{eqnarray*}
  P \approx_{L} Q \iff \forall E \in L. P \models E \iff Q \models E
\end{eqnarray*}

\begin{eqnarray*}
  P \approx_{K} Q
\end{eqnarray*}

\begin{eqnarray*}
  P \approx Q
\end{eqnarray*}

$\approx_{K} = \approx = \approx_{L}$

\subsubsection{Contextual duality}

Note that contexts extend the quotation operation to a family of
operations from processes to names. Given a context, $M$, we can
define a \emph{nominal context}, $\quotep{M}$ by $\quotep{M}[P] :=
\quotep{M[P]}$. To foreshadow what is to come we observe that these
operations enjoy a duality with processes very much like the duality
between vectors and maps from vectors to scalars.

Further, because the calculus is essentially higher-order, we have a
correspondence between contexts and processes. More specifically,
given a name $x$ and a context $M$ we can construct $M^{*}_{x}$ such
that 

\begin{mathpar}
  M^{*}_{x} | \lift{x}{P} \red M[P]
\end{mathpar}

namely,

\begin{mathpar}
  M^{*}_{x} := x?(u).M[\dropn{u}]
\end{mathpar}

The dependence of $M^{*}_{x}$ on a name makes it an abstraction, 

\begin{mathpar}
  M^{*} := (x)x?(u).M[\dropn{u}]
\end{mathpar}

\subsection{Additional notation}

It will sometimes be convenient to denote the process a name
quotes. We already have the notation $x = \quotep{P}$, but it will be
convenient to introduce an alternate notation, $\procn{x}$, when we
want to emphasize the connection to the use of the name. Note that, by
virtue of name equivalence, $\quotep{\procn{x}} \nameeq x$; so, the
notation is consistent with previous definitions.

Further, because names have structure it is possible to effect
substitutions on the basis of that structure. This means we need to
upgrade our notation for substitutions, which we accomplish by
adapting comprehension notation. Thus,

\begin{mathpar}
  P\{ y / x : x \in S \}
\end{mathpar}

is interpreted to mean the process derived from P by replacing (in a
capture-avoiding manner) each occurrence of $x$ in $S$ by $y$. For example,

\begin{mathpar}
  P\{ \quotep{\procn{x}|\procn{x}} / x : x \in \freenames{P} \}
\end{mathpar}

will replace each (occurrence) of a free name $x$ in $P$ by
$\quotep{\procn{x}|\procn{x}}$.

Also, we will avail ourselves of the notation $x^{L}$ and $x^{R}$ to
denote injections of a name into disjoint copies of the name
space. There are numerous ways to accomplish this. One example can be
found in \cite{MeredithR05}. This notation overloads to vectors of
names: $\vec{x}^{\pi} := (x_{i}^{\pi} \; : \; 0 \leq i < |\vec{x}| )$ where $\pi \in \{L,R\}$.

We also use $P^{\Box} := P|\Box$.

In \cite{MeredithR05} an interpretation of the new operator is
given. It turns out that there are several possible interpretations
all enjoying the requisite algebraic properties of the operator (see
\cite{milner91polyadicpi}). We will therefore make liberal use of
$(\nu\; \vec{x})P$.

% subsection the_syntax_and_semantics_of_the_notation_system (end)   

\input{qm2pi.qmops} 

\input{qm2pi.sterngerlach} 

\input{qm2pi.metric} 

% section concurrent_process_calculi (end)

%\input{qm2pi.proofsketch}

% section proof sketch (end)

%\input{qm2pi.slviaknots} 

% section spatial logic via knots (end)

\input{qm2pi.conclusion}

% section conclusion (end)

%\input{qm2pi.dtcodes} 

% section wiring algorithm (end)

\input{qm2pi.ack} 

% section acknowledgments (end)

\newpage


\bibliographystyle{plain}   
\bibliography{../../biblios/main.bib}

\input{qm2pi.rhodetails}

\end{document}

 

\documentclass[12pt]{llncs}
%\documentclass{jktr}

\usepackage[pdftex]{hyperref}                   
\usepackage {listings}
\usepackage {mathpartir}
\usepackage{bcprules}
%\usepackage{listings}
                       
\usepackage{graphicx} 
%\usepackage[margins=2.5cm,nohead,nofoot]{geometry}
%\usepackage{geometry}
\usepackage{amsfonts}
\usepackage{amstext}
\usepackage{latexsym}
\usepackage{amssymb}
\usepackage{color}


%\include{myPreamble}
\include{qm2pi.local} 

%\ifpdf
%\usepackage[pdftex]{graphicx}
%\else
%\usepackage{graphicx}
%\fi

 % \ifpdf
%  \usepackage{pdfsync}
%  \if


%\title{Brief Article}
%\author{David F. Snyder}
%\author{L.G. Meredith}

%\address{Dept. of Math., Texas State University--San Marcos, San Marcos, TX 78666}
       
\pagestyle{empty}


\begin{document}

\lstset{language=[Objective]Caml,frame=shadowbox}

\input{qm2pi.front}

% section front matter (end)

\input{qm2pi.intro} 
 
% section introduction (end)

% \input{qm2pi.knotations} 

% section notation (end)

\input{qm2pi.process.calculi} 

% section concurrent_process_calculi_and_spatial_logics_ (end)
    
%\input{qm2pi.knots2pi} 

%\input{qm2pi.trefoil} 

%\input{qm2pi.mainthm} 

% subsection basic_interpretation (end)

%\input{qm2pi.rho.presentation} 
\subsection{The syntax and semantics of the notation system}\label{sub:the_syntax_and_semantics_of_the_notation_system} % (fold)

We now summarize a technical presentation of the calculus that
embodies our theory of dynamics. The typical presentation of such a
calculus follows the style of giving generators and relations on
them. The grammar, below, describing term constructors, freely
generates the set of processes, $\Proc$. This set is then quotiented
by a relation known as structural congruence and it is over this set
that the notion of dynamics is expressed. This presentation is
essentially that of \cite{MeredithR05} with the addition of
polyadicity and summation. For readability we have relegated some of
the technical subtleties to an appendix.

\subsubsection{Process grammar}\label{subsub:process_grammar}

\begin{mathpar}
  \inferrule* [lab=synchronization] {} {{M} \bc \pzero \;|\; x?F \;|\; x!C }
  \and
  \inferrule* [lab=abstraction] {} {{F} \bc (x)P}
  \and
  \inferrule* [lab=concretion] {} {{C} \bc \langle Q \rangle}
  \and
  \inferrule* [lab=process] {} {{P,Q} \bc M \;| \;P|Q \;|\; @{x}}
  \and
  \inferrule* [lab=name] {} {{x} \bc \quotep{P}}
\end{mathpar} 

Note that $\vec{x}$ (resp. $\vec{P}$) denotes a vector of names
(resp. processes) of length $|\vec{x}|$ (resp. $|\vec{P}|$). We adopt
the following useful abbreviations.

\begin{mathpar}
   x?(\vec{y}).P := x.(\vec{y})P \and  x\clift{\vec{P}} := x.\clift{\vec{P}}
   \and x!(y) := \lift{x}{\dropn{y}}
   \and \Pi_{i=0}^{n-1}P_i := P_0 | \ldots | P_{n-1}
\end{mathpar}

\subsubsection{Structural congruence}

\paragraph{Free and bound names and alpha-equivalence.} At the
core of structural equivalence is alpha-equivalence which identifies
process that are the same up to a change of variable. Formally, we
recognize the distinction between free and bound names. The free names
of a process, $\freenames{P}$, may be calculated recursively as
follows:

\begin{mathpar}
\freenames{\pzero} := \emptyset
  \and \\
  \freenames{x?(y).P} := \{ x \} \cup (\freenames{P} \setminus \{ y \})
  \and 
  \freenames{x!\langle P \rangle} := \{ x \} \cup \{ P \} 
  \and \\
  \freenames{P|Q} := \freenames{P} \cup \freenames{Q}
  \and \\
  \freenames{@{x}} := \{ x \}
\end{mathpar}

$\pi$
$\quotep{\pi}$

$\freenames{-} : \pi \to \mathcal{P}(\quotep{\pi})$

\begin{eqnarray*}
  \freenames{\pzero} & := & \emptyset \\
  \freenames{x?(y).P} & := & \{ x \} \cup (\freenames{P} \setminus \{ y \}) \\
  \freenames{x!\langle P \rangle} & := & \{ x \} \cup \{ P \} \\
  \freenames{P|Q} & := & \freenames{P} \cup \freenames{Q} \\
  \freenames{\dropn{x}} & := & \{ x \}
\end{eqnarray*}

The bound names of a process, $\boundnames{P}$, are those names occurring in $P$
that are not free. For example, in $x?(y).0$, the name $x$ is free, while $y$ is bound.

\begin{mathpar}
  \inferrule* [lab=monoidal-laws] {} { P|Q \equiv Q|P \and P|0 \equiv P \and P|(Q|R) \equiv (P|Q)|R }
\end{mathpar}

\begin{mathpar}
  \inferrule* [lab=alpha-equivalence] {} { (x)P \equiv (y)P\{y/x\} \and y \not\in \freenames{P} }
\end{mathpar}

\begin{definition}
Then two processes, $P,Q$, are alpha-equivalent if $P = Q\{\vec{y}/\vec{x}\}$ for
some $\vec{x} \in \boundnames{Q},\vec{y} \in \boundnames{P}$, where $Q\{\vec{y}/\vec{x}\}$
denotes the capture-avoiding substitution of $\vec{y}$ for $\vec{x}$ in $Q$.
\end{definition}

\begin{definition}
  The {\em structural congruence} \cite{SangiorgiWalker} , $\equiv$,
  between processes is the least congruence containing
  alpha-equivalence, satisfying the abelian monoid laws
  (associativity, commutativity and $\pzero$ as identity) for parallel
  composition $|$ and for summation $+$.
\end{definition}

\subsection{Name equivalence}

We take name equivalence, written $\nameeq$, to be the smallest
equivalence relation generated by the following rules.

\begin{mathpar}
\inferrule*[lab=Quote-drop]
{ }
{ \quotep{@{x}} \nameeq x }

\inferrule*[lab=Struct-equiv]
{ P \scong Q }
{ \quotep{P} \nameeq \quotep{Q} }
\end{mathpar}

The astute reader will have noticed that the mutual recursion of names
and processes imposes a mutual recursion on alpha-equivalence and
structural equivalence via name-equivalence. Fortunately, all of this
works out pleasantly and we may calculate in the natural way, free of
concern. The reader interested in the details is referred to the
appendix \ref{appendix:rho_details}.

\subsection{Substitution}

We use $\Proc$ for the set of processes, $\QProc$ for the set of
names, and $\id{\{}\vec{y} / \vec{x} \id{\}}$ to denote partial maps,
$s : \QProc \rightarrow \QProc$. A map, $s$ lifts, uniquely, to a map
on process terms, $\widehat{s} : \Proc \rightarrow \Proc$ by the
following equations.

\begin{mathpar}
  (0) \psubstp{Q}{P} := 0 \\
  (R \juxtap S) \psubstp{Q}{P}
  :=    
  (R)\psubstp{Q}{P} \juxtap (S) \psubstp{Q}{P} \\
  (x?(y).R) \psubstp{Q}{P}    
  :=    
  (x)\substp{Q}{P} (z)\concat( (R \psubstn{z}{y}) \psubstp{Q}{P} ) \\
  (\lift{x}{R}) \psubstp{Q}{P}  
  :=
  \lift{(x)\substp{Q}{P}}{ R \psubstp{Q}{P} } \\
%   (\dropn{x})  \psubstp{Q}{P}       
%   := 
%   \left\{ 
%     \begin{array}{ccc} 
%       \dropn{\quotep{Q}} & & x \nameeq \quotep{P} \\
%       \dropn{x} & & otherwise \\
%     \end{array}
%   \right. 
  (\dropn{x})  \psubstp{Q}{P}       
  := 
  \left\{ 
    \begin{array}{ccc} 
      Q & & x \nameeq \quotep{P} \\
      \dropn{x} & & otherwise \\
    \end{array}
  \right.
\end{mathpar}
 

where

\begin{eqnarray}
  (x)\id{\{} \lpquote Q \rpquote / \lpquote P \rpquote \id{\}}            = 
  \left\{ 
    \begin{array}{ccc}
      \lpquote Q \rpquote & & x \nameeq \lpquote P \rpquote \\
      x & & otherwise \\
    \end{array}
  \right. \nonumber
\end{eqnarray}

and $z$ is chosen distinct from $\quotep{P}$, $\quotep{Q}$, the free
names in $Q$, and all the names in $R$. Our $\alpha$-equivalence will
be built in the standard way from this substitution.

\begin{remark}\label{rem:no_self_referential_names}
  One consequence of these definitions is that $\forall P. \quotep{P}
  \not\in \freenames{P}$.
\end{remark}

\subsection{ Dynamic quote: an example }

Anticipating something of what's to come, consider applying the
substitution, $\widehat{\id{\{}u / z \id{\}}}$, to the following pair
of processes, $\lift{w}{y!(z)}$ and $w[ \lpquote y!(z) \rpquote ]$.

\begin{eqnarray}
	\lift{w}{y!(z)}\widehat{\id{\{}u / z \id{\}}}
		& = &
		\lift{w}{y!(u)} \nonumber\\
	w[ \lpquote y!(z) \rpquote ] \widehat{ \id{\{}u / z \id{\}} }
		& = &
		w[ \lpquote y!(z) \rpquote ] \nonumber
\end{eqnarray}

Because the body of the process between quotes is impervious to
substitution, we get radically different answers. In fact, by
examining the first process in an input context,
e.g. $x?(z).\lift{w}{y!(z)}$, we see that the process under the lift
operator may be shaped by prefixed inputs binding a name inside it. In
this sense, the lift operator will be seen as a way to dynamically
construct processes before reifying them as names.

Finally equipped with these standard features we can present the
dynamics of the calculus.

\subsubsection{Operational semantics} 

Finally, we introduce the computational dynamics. What marks these
algebras as distinct from other more traditionally studied algebraic
structures, e.g. vector spaces or polynomial rings, is the manner in
which dynamics is captured. In traditional structures, dynamics is typically
expressed through morphisms between such structures, as in linear maps
between vector spaces or morphisms between rings. In algebras
associated with the semantics of computation, the dynamics is
expressed as part of the algebraic structure itself, through a
reduction reduction relation typically denoted by $\red$. Below, we
give a recursive presentation of this relation for the calculus used
in the encoding.

$\red \subseteq \pi \times \pi$
$\red : \pi \to \mathcal{P}(\pi)$

\begin{mathpar}
  \inferrule* [lab=Comm] { \textsf{match}( x_{src}, x_{trgt} ) } { x_{trgt}?(y)P \; | \; x_{src}!\langle {Q} \rangle \red P\{\quotep{Q}/y}\} }
  \and \\
  \inferrule* [lab=Par] {{P} \red {P}'} {{{P} | {Q}} \red {{P}' | {Q}}}
  \and
  \inferrule* [lab=Equiv]{{{P} \scong {P}'} \andalso {{P}' \red {Q}'} \andalso {{Q}' \scong {Q}}}{{P} \red {Q}}
\end{mathpar}

\begin{eqnarray*}
  match_{\equiv} (\quotep{P},\quotep{Q}) & := & P \equiv Q \\
  match_{\dagger}(\quotep{P},\quotep{Q}) & := & \forall R. P|Q \red^{*} R => R \red^{*} 0 \\
  match_{K}(\quotep{P},\quotep{Q}) & := & K \mbox{ for some context } K
\end{eqnarray*}

$u?(x)P | u!\langle Q \rangle \red P\{\quotep{Q}/x\}$

%We write $\wred$ for $\red^*$, and $P\red$ if $\exists Q $ such that $ P \red Q$.
We write $P\red$ if $\exists Q $ such that $ P \red Q$ and $P\not\red$, otherwise.

\section{Replication}

As mentioned before, it is known that replication (and hence
recursion) can be implemented in a higher-order process algebra
\cite{SangiorgiWalker}. As our first example of calculation with the
machinery thus far presented we give the construction explicitly in
the {\rhoc}.

\begin{eqnarray}
	D_{x} & := & \prefix{x}{y}{(\binpar{\outputp{x}{y}}{@{y}})} \nonumber\\
	\bangp_{x}{P} & := & \binpar{{x}!\langle{\binpar{D_{x}}{P}}\rangle}{D_{x}} \nonumber
\end{eqnarray}

\begin{eqnarray}
	\bangp_{x}{P} & & \nonumber\\
	=
	& {x}!\langle{(\prefix{x}{y}{(\outputp{x}{y} | @{y})) | P}}\rangle 
	      | \prefix{x}{y}{(\outputp{x}{y} | @{y})} & \nonumber\\
	\red
	& (\outputp{x}{y} | @{y})\substn{\quotep{(\prefix{x}{y}{(@{y} | \outputp{x}{y})) | P}}}{y} & \nonumber\\
	=
	& \outputp{x}{\quotep{(\prefix{x}{y}{(\outputp{x}{y} | @{y})) | P}}}
	  | {(\prefix{x}{y}{(\outputp{x}{y} | @{y})) | P}} & \nonumber\\
	\red
	& \ldots & \nonumber\\
	\red^*
	& P | P | \ldots & \nonumber
\end{eqnarray}

Of course, this encoding, as an implementation, runs away, unfolding
$\bangp{P}$ eagerly. A lazier and more implementable replication
operator, restricted to input-guarded processes, may be obtained as follows.

\begin{eqnarray}
\bangp{\prefix{u}{v}{P}} 
	:= 
	\binpar{\lift{x}{\prefix{u}{v}{(\binpar{D(x)}{P})}}}{D(x)} \nonumber
\end{eqnarray}

\begin{remark}
  Note that the lazier definition still does not deal with summation
  or mixed summation (i.e. sums over input and output). The reader is
  invited to construct definitions of replication that deal with these
  features. 

  Further, the definitions are parameterized in a name, $x$. Can you,
  gentle reader, make a definition that eliminates this parameter and
  guarantees no accidental interaction between the replication
  machinery and the process being replicated -- i.e. no accidental
  sharing of names used by the process to get its work done and the
  name(s) used by the replication to effect copying. This latter
  revision of the definition of replication is crucial to obtaining
  the expected identity $!!P \sim !P$.
\end{remark}

\begin{remark}\label{rem:paradoxical_combinator}
  The reader familiar with the lambda calculus will have noticed the
  similarity between $D$ and the paradoxical combinator.

  [Ed. note: the existence of this seems to suggest we have to be more
  restrictive on the set of processes and names we admit if we are to
  support no-cloning.]
\end{remark}

\subsubsection{Bisimulation}

The computational dynamics gives rise to another kind of equivalence,
the equivalence of computational behavior. As previously mentioned
this is typically captured \emph{via} some form of bisimulation.

% The notion we use in this paper is weak barbed bisimulation
% \cite{milner91polyadicpi}.

The notion we use in this paper is derived from weak barbed
bisimulation \cite{milner91polyadicpi}. 

\begin{definition}
An \emph{observation relation}, $\downarrow_{\mathcal N}$, over a set
of names, $\mathcal N$, is the smallest relation satisfying the rules
below.

\infrule[Out-barb]{y \in {\mathcal N}, \; x \nameeq y}
		  {\outputp{x}{v} \downarrow_{\mathcal N} x}
\infrule[Par-barb]{\mbox{$P\downarrow_{\mathcal N} x$ or $Q\downarrow_{\mathcal N} x$}}
		  {\binpar{P}{Q} \downarrow_{\mathcal N} x}

We write $P \Downarrow_{\mathcal N} x$ if there is $Q$ such that 
$P \wred Q$ and $Q \downarrow_{\mathcal N} x$.
\end{definition}

\begin{definition}
%\label{def.bbisim}
An  ${\mathcal N}$-\emph{barbed bisimulation} over a set of names, ${\mathcal N}$, is a symmetric binary relation 
${\mathcal S}_{\mathcal N}$ between agents such that $P\rel{S}_{\mathcal N}Q$ implies:
\begin{enumerate}
\item If $P \red P'$ then $Q \wred Q'$ and $P'\rel{S}_{\mathcal N} Q'$.
\item If $P\downarrow_{\mathcal N} x$, then $Q\Downarrow_{\mathcal N} x$.
\end{enumerate}
$P$ is ${\mathcal N}$-barbed bisimilar to $Q$, written
$P \wbbisim_{\mathcal N} Q$, if $P \rel{S}_{\mathcal N} Q$ for some ${\mathcal N}$-barbed bisimulation ${\mathcal S}_{\mathcal N}$.
\end{definition}

$\mathcal{R} \subseteq \pi \times \pi$

$P \mathcal{R} Q => \forall P'. P \red P' \Rightarrow \exists Q'. Q \red Q', P' \mathcal{R} Q'$

$P \vdash x \Rightarrow Q \vdash x$

\begin{mathpar}
  \inferrule*[lab=Out-barb]{x \nameeq y}{{y}!\langle{Q}\rangle \vdash x}
  \and
  \inferrule*[lab=Par-barb]{\mbox{$P\vdash x$ or $Q\vdash x$}}{\binpar{P}{Q} \vdash x}
\end{mathpar}

\subsubsection{Contexts}

One of the principle advantages of computational calculi like the
$\pi$-calculus is a well-defined notion of context,
contextual-equivalence and a correlation between
contextual-equivalence and notions of bisimulation. The notion of
context allows the decomposition of a process into (sub-)process and
its syntactic environment, its context. Thus, a context may be
thought of as a process with a ``hole'' (written $\Box$) in it. The
application of a context $M$ to a process $P$, written $M[P]$, is
tantamount to filling the hole in $M$ with $P$. In this paper we do
not need the full weight of this theory, but do make use of the notion
of context in the proof the main theorem. 

\begin{mathpar}
  \inferrule* [lab=summation] {} {{M_{M},M_{N}} \bc \Box \;|\; x.M_{A} \;|\; M_{M}+M_{N}}
  \and
  \inferrule* [lab=agent] {} {{M_{A}} \bc (\vec{x})M_{P} \;| \; \clift{P_0,\ldots,M_{P},\ldots,P_N}}
  \and \\
  \inferrule* [lab=process] {} {{M_{P}} \bc M_{N} \;| \;P|M_{P} }
\end{mathpar} 

\begin{mathpar}
  \inferrule* [lab=sychronization] {} {M_{N} \bc \Box \;|\; x?M_{F} \;|\; x!M_{C}}
  \and
  \inferrule* [lab=abstraction] {} {{M_{F}} \bc (x)M_{P} }
  \and
  \inferrule* [lab=concretion] {} {{M_{C}} \bc \langle M_{P} \rangle }
  \and \\
  \inferrule* [lab=process] {} {{M_{P}} \bc M_{N} \;| \;P|M_{P} }
\end{mathpar}

\begin{definition}[contextual application] Given a context $M$, and
  process $P$, we define the \emph{contextual application}, $M[P] :=
  M\{P/\Box\}$. That is, the contextual application of M to P is the
  substitution of $P$ for $\Box$ in $M$.
\end{definition}

$\meaningof{-} : L \to \mathcal{P}(\pi)$

\begin{mathpar}
  \inferrule* [lab=collection] {} {\meaningof{true} = \pi, \and \meaningof{~E} = \pi \setminus \meaningof{E}, \and \meaningof{E_{1} \& E_{2}} = \meaningof{E_{1}} \cap \meaningof{E_{2}}}
\end{mathpar}

\begin{mathpar}
  \inferrule* [lab=structure] {} {\meaningof{0} = \{ P \in \pi | P \equiv 0 \}, \and \\ \meaningof{E_1 | E_2} = \{ P \in \pi | P \equiv P_{1} | P_{2}, P_{1} \in \meaningof{E_{1}}, P_{2} \in \meaningof{E_2}\} }
\end{mathpar}

\begin{mathpar}
 \inferrule* [lab=behavior] {} {\meaningof{\langle a?b \rangle E} = \{ P \in \pi | P \equiv Q | u?(y)P', \\ \and \\\\ \and \\ \;\;\; u \in \meaningof{a}, \forall z.P'\{z/y\} \in \meaningof{E\{z/b\}}\}, \and \\ \meaningof{a!E} = \{ P \in \pi | P \equiv Q | x!\langle P' \rangle, x \in \meaningof{a} P' \in \meaningof{E}\} }
\end{mathpar}

\begin{mathpar}
 \inferrule* [lab=nominal] {} {\meaningof{\quotep{E}} = \{ \quotep{P} \in \quotep{\pi} | P \in \meaningof{E} \}, \and \meaningof{\quotep{P}} = \{ \quotep{Q} \in \quotep{\pi} | P \equiv Q \} \and \\ \meaningof{@\quotep{E}} = \{ P \in \pi | P \equiv @x, x \in \meaningof{E} \}}
\end{mathpar}

\begin{eqnarray*}
  \\
  \meaningof{-} : TS \to ST
\end{eqnarray*}

\begin{eqnarray*}
  \\
  L : TS \to ST
\end{eqnarray*}

\begin{eqnarray*}
  \\
  P \models E \iff P \in \meaningof{E}
\end{eqnarray*}

\begin{eqnarray*}
  P \approx_{L} Q \iff \forall E \in L. P \models E \iff Q \models E
\end{eqnarray*}

\begin{eqnarray*}
  P \approx_{K} Q
\end{eqnarray*}

\begin{eqnarray*}
  P \approx Q
\end{eqnarray*}

$\approx_{K} = \approx = \approx_{L}$

\subsubsection{Contextual duality}

Note that contexts extend the quotation operation to a family of
operations from processes to names. Given a context, $M$, we can
define a \emph{nominal context}, $\quotep{M}$ by $\quotep{M}[P] :=
\quotep{M[P]}$. To foreshadow what is to come we observe that these
operations enjoy a duality with processes very much like the duality
between vectors and maps from vectors to scalars.

Further, because the calculus is essentially higher-order, we have a
correspondence between contexts and processes. More specifically,
given a name $x$ and a context $M$ we can construct $M^{*}_{x}$ such
that 

\begin{mathpar}
  M^{*}_{x} | \lift{x}{P} \red M[P]
\end{mathpar}

namely,

\begin{mathpar}
  M^{*}_{x} := x?(u).M[\dropn{u}]
\end{mathpar}

The dependence of $M^{*}_{x}$ on a name makes it an abstraction, 

\begin{mathpar}
  M^{*} := (x)x?(u).M[\dropn{u}]
\end{mathpar}

\subsection{Additional notation}

It will sometimes be convenient to denote the process a name
quotes. We already have the notation $x = \quotep{P}$, but it will be
convenient to introduce an alternate notation, $\procn{x}$, when we
want to emphasize the connection to the use of the name. Note that, by
virtue of name equivalence, $\quotep{\procn{x}} \nameeq x$; so, the
notation is consistent with previous definitions.

Further, because names have structure it is possible to effect
substitutions on the basis of that structure. This means we need to
upgrade our notation for substitutions, which we accomplish by
adapting comprehension notation. Thus,

\begin{mathpar}
  P\{ y / x : x \in S \}
\end{mathpar}

is interpreted to mean the process derived from P by replacing (in a
capture-avoiding manner) each occurrence of $x$ in $S$ by $y$. For example,

\begin{mathpar}
  P\{ \quotep{\procn{x}|\procn{x}} / x : x \in \freenames{P} \}
\end{mathpar}

will replace each (occurrence) of a free name $x$ in $P$ by
$\quotep{\procn{x}|\procn{x}}$.

Also, we will avail ourselves of the notation $x^{L}$ and $x^{R}$ to
denote injections of a name into disjoint copies of the name
space. There are numerous ways to accomplish this. One example can be
found in \cite{MeredithR05}. This notation overloads to vectors of
names: $\vec{x}^{\pi} := (x_{i}^{\pi} \; : \; 0 \leq i < |\vec{x}| )$ where $\pi \in \{L,R\}$.

We also use $P^{\Box} := P|\Box$.

In \cite{MeredithR05} an interpretation of the new operator is
given. It turns out that there are several possible interpretations
all enjoying the requisite algebraic properties of the operator (see
\cite{milner91polyadicpi}). We will therefore make liberal use of
$(\nu\; \vec{x})P$.

% subsection the_syntax_and_semantics_of_the_notation_system (end)   

\input{qm2pi.qmops} 

\input{qm2pi.sterngerlach} 

\input{qm2pi.metric} 

% section concurrent_process_calculi (end)

%\input{qm2pi.proofsketch}

% section proof sketch (end)

%\input{qm2pi.slviaknots} 

% section spatial logic via knots (end)

\input{qm2pi.conclusion}

% section conclusion (end)

%\input{qm2pi.dtcodes} 

% section wiring algorithm (end)

\input{qm2pi.ack} 

% section acknowledgments (end)

\newpage


\bibliographystyle{plain}   
\bibliography{../../biblios/main.bib}

\input{qm2pi.rhodetails}

\end{document}

 

% section concurrent_process_calculi (end)

%\documentclass[12pt]{llncs}
%\documentclass{jktr}

\usepackage[pdftex]{hyperref}                   
\usepackage {listings}
\usepackage {mathpartir}
\usepackage{bcprules}
%\usepackage{listings}
                       
\usepackage{graphicx} 
%\usepackage[margins=2.5cm,nohead,nofoot]{geometry}
%\usepackage{geometry}
\usepackage{amsfonts}
\usepackage{amstext}
\usepackage{latexsym}
\usepackage{amssymb}
\usepackage{color}


%\include{myPreamble}
\include{qm2pi.local} 

%\ifpdf
%\usepackage[pdftex]{graphicx}
%\else
%\usepackage{graphicx}
%\fi

 % \ifpdf
%  \usepackage{pdfsync}
%  \if


%\title{Brief Article}
%\author{David F. Snyder}
%\author{L.G. Meredith}

%\address{Dept. of Math., Texas State University--San Marcos, San Marcos, TX 78666}
       
\pagestyle{empty}


\begin{document}

\lstset{language=[Objective]Caml,frame=shadowbox}

\input{qm2pi.front}

% section front matter (end)

\input{qm2pi.intro} 
 
% section introduction (end)

% \input{qm2pi.knotations} 

% section notation (end)

\input{qm2pi.process.calculi} 

% section concurrent_process_calculi_and_spatial_logics_ (end)
    
%\input{qm2pi.knots2pi} 

%\input{qm2pi.trefoil} 

%\input{qm2pi.mainthm} 

% subsection basic_interpretation (end)

%\input{qm2pi.rho.presentation} 
\subsection{The syntax and semantics of the notation system}\label{sub:the_syntax_and_semantics_of_the_notation_system} % (fold)

We now summarize a technical presentation of the calculus that
embodies our theory of dynamics. The typical presentation of such a
calculus follows the style of giving generators and relations on
them. The grammar, below, describing term constructors, freely
generates the set of processes, $\Proc$. This set is then quotiented
by a relation known as structural congruence and it is over this set
that the notion of dynamics is expressed. This presentation is
essentially that of \cite{MeredithR05} with the addition of
polyadicity and summation. For readability we have relegated some of
the technical subtleties to an appendix.

\subsubsection{Process grammar}\label{subsub:process_grammar}

\begin{mathpar}
  \inferrule* [lab=synchronization] {} {{M} \bc \pzero \;|\; x?F \;|\; x!C }
  \and
  \inferrule* [lab=abstraction] {} {{F} \bc (x)P}
  \and
  \inferrule* [lab=concretion] {} {{C} \bc \langle Q \rangle}
  \and
  \inferrule* [lab=process] {} {{P,Q} \bc M \;| \;P|Q \;|\; @{x}}
  \and
  \inferrule* [lab=name] {} {{x} \bc \quotep{P}}
\end{mathpar} 

Note that $\vec{x}$ (resp. $\vec{P}$) denotes a vector of names
(resp. processes) of length $|\vec{x}|$ (resp. $|\vec{P}|$). We adopt
the following useful abbreviations.

\begin{mathpar}
   x?(\vec{y}).P := x.(\vec{y})P \and  x\clift{\vec{P}} := x.\clift{\vec{P}}
   \and x!(y) := \lift{x}{\dropn{y}}
   \and \Pi_{i=0}^{n-1}P_i := P_0 | \ldots | P_{n-1}
\end{mathpar}

\subsubsection{Structural congruence}

\paragraph{Free and bound names and alpha-equivalence.} At the
core of structural equivalence is alpha-equivalence which identifies
process that are the same up to a change of variable. Formally, we
recognize the distinction between free and bound names. The free names
of a process, $\freenames{P}$, may be calculated recursively as
follows:

\begin{mathpar}
\freenames{\pzero} := \emptyset
  \and \\
  \freenames{x?(y).P} := \{ x \} \cup (\freenames{P} \setminus \{ y \})
  \and 
  \freenames{x!\langle P \rangle} := \{ x \} \cup \{ P \} 
  \and \\
  \freenames{P|Q} := \freenames{P} \cup \freenames{Q}
  \and \\
  \freenames{@{x}} := \{ x \}
\end{mathpar}

$\pi$
$\quotep{\pi}$

$\freenames{-} : \pi \to \mathcal{P}(\quotep{\pi})$

\begin{eqnarray*}
  \freenames{\pzero} & := & \emptyset \\
  \freenames{x?(y).P} & := & \{ x \} \cup (\freenames{P} \setminus \{ y \}) \\
  \freenames{x!\langle P \rangle} & := & \{ x \} \cup \{ P \} \\
  \freenames{P|Q} & := & \freenames{P} \cup \freenames{Q} \\
  \freenames{\dropn{x}} & := & \{ x \}
\end{eqnarray*}

The bound names of a process, $\boundnames{P}$, are those names occurring in $P$
that are not free. For example, in $x?(y).0$, the name $x$ is free, while $y$ is bound.

\begin{mathpar}
  \inferrule* [lab=monoidal-laws] {} { P|Q \equiv Q|P \and P|0 \equiv P \and P|(Q|R) \equiv (P|Q)|R }
\end{mathpar}

\begin{mathpar}
  \inferrule* [lab=alpha-equivalence] {} { (x)P \equiv (y)P\{y/x\} \and y \not\in \freenames{P} }
\end{mathpar}

\begin{definition}
Then two processes, $P,Q$, are alpha-equivalent if $P = Q\{\vec{y}/\vec{x}\}$ for
some $\vec{x} \in \boundnames{Q},\vec{y} \in \boundnames{P}$, where $Q\{\vec{y}/\vec{x}\}$
denotes the capture-avoiding substitution of $\vec{y}$ for $\vec{x}$ in $Q$.
\end{definition}

\begin{definition}
  The {\em structural congruence} \cite{SangiorgiWalker} , $\equiv$,
  between processes is the least congruence containing
  alpha-equivalence, satisfying the abelian monoid laws
  (associativity, commutativity and $\pzero$ as identity) for parallel
  composition $|$ and for summation $+$.
\end{definition}

\subsection{Name equivalence}

We take name equivalence, written $\nameeq$, to be the smallest
equivalence relation generated by the following rules.

\begin{mathpar}
\inferrule*[lab=Quote-drop]
{ }
{ \quotep{@{x}} \nameeq x }

\inferrule*[lab=Struct-equiv]
{ P \scong Q }
{ \quotep{P} \nameeq \quotep{Q} }
\end{mathpar}

The astute reader will have noticed that the mutual recursion of names
and processes imposes a mutual recursion on alpha-equivalence and
structural equivalence via name-equivalence. Fortunately, all of this
works out pleasantly and we may calculate in the natural way, free of
concern. The reader interested in the details is referred to the
appendix \ref{appendix:rho_details}.

\subsection{Substitution}

We use $\Proc$ for the set of processes, $\QProc$ for the set of
names, and $\id{\{}\vec{y} / \vec{x} \id{\}}$ to denote partial maps,
$s : \QProc \rightarrow \QProc$. A map, $s$ lifts, uniquely, to a map
on process terms, $\widehat{s} : \Proc \rightarrow \Proc$ by the
following equations.

\begin{mathpar}
  (0) \psubstp{Q}{P} := 0 \\
  (R \juxtap S) \psubstp{Q}{P}
  :=    
  (R)\psubstp{Q}{P} \juxtap (S) \psubstp{Q}{P} \\
  (x?(y).R) \psubstp{Q}{P}    
  :=    
  (x)\substp{Q}{P} (z)\concat( (R \psubstn{z}{y}) \psubstp{Q}{P} ) \\
  (\lift{x}{R}) \psubstp{Q}{P}  
  :=
  \lift{(x)\substp{Q}{P}}{ R \psubstp{Q}{P} } \\
%   (\dropn{x})  \psubstp{Q}{P}       
%   := 
%   \left\{ 
%     \begin{array}{ccc} 
%       \dropn{\quotep{Q}} & & x \nameeq \quotep{P} \\
%       \dropn{x} & & otherwise \\
%     \end{array}
%   \right. 
  (\dropn{x})  \psubstp{Q}{P}       
  := 
  \left\{ 
    \begin{array}{ccc} 
      Q & & x \nameeq \quotep{P} \\
      \dropn{x} & & otherwise \\
    \end{array}
  \right.
\end{mathpar}
 

where

\begin{eqnarray}
  (x)\id{\{} \lpquote Q \rpquote / \lpquote P \rpquote \id{\}}            = 
  \left\{ 
    \begin{array}{ccc}
      \lpquote Q \rpquote & & x \nameeq \lpquote P \rpquote \\
      x & & otherwise \\
    \end{array}
  \right. \nonumber
\end{eqnarray}

and $z$ is chosen distinct from $\quotep{P}$, $\quotep{Q}$, the free
names in $Q$, and all the names in $R$. Our $\alpha$-equivalence will
be built in the standard way from this substitution.

\begin{remark}\label{rem:no_self_referential_names}
  One consequence of these definitions is that $\forall P. \quotep{P}
  \not\in \freenames{P}$.
\end{remark}

\subsection{ Dynamic quote: an example }

Anticipating something of what's to come, consider applying the
substitution, $\widehat{\id{\{}u / z \id{\}}}$, to the following pair
of processes, $\lift{w}{y!(z)}$ and $w[ \lpquote y!(z) \rpquote ]$.

\begin{eqnarray}
	\lift{w}{y!(z)}\widehat{\id{\{}u / z \id{\}}}
		& = &
		\lift{w}{y!(u)} \nonumber\\
	w[ \lpquote y!(z) \rpquote ] \widehat{ \id{\{}u / z \id{\}} }
		& = &
		w[ \lpquote y!(z) \rpquote ] \nonumber
\end{eqnarray}

Because the body of the process between quotes is impervious to
substitution, we get radically different answers. In fact, by
examining the first process in an input context,
e.g. $x?(z).\lift{w}{y!(z)}$, we see that the process under the lift
operator may be shaped by prefixed inputs binding a name inside it. In
this sense, the lift operator will be seen as a way to dynamically
construct processes before reifying them as names.

Finally equipped with these standard features we can present the
dynamics of the calculus.

\subsubsection{Operational semantics} 

Finally, we introduce the computational dynamics. What marks these
algebras as distinct from other more traditionally studied algebraic
structures, e.g. vector spaces or polynomial rings, is the manner in
which dynamics is captured. In traditional structures, dynamics is typically
expressed through morphisms between such structures, as in linear maps
between vector spaces or morphisms between rings. In algebras
associated with the semantics of computation, the dynamics is
expressed as part of the algebraic structure itself, through a
reduction reduction relation typically denoted by $\red$. Below, we
give a recursive presentation of this relation for the calculus used
in the encoding.

$\red \subseteq \pi \times \pi$
$\red : \pi \to \mathcal{P}(\pi)$

\begin{mathpar}
  \inferrule* [lab=Comm] { \textsf{match}( x_{src}, x_{trgt} ) } { x_{trgt}?(y)P \; | \; x_{src}!\langle {Q} \rangle \red P\{\quotep{Q}/y}\} }
  \and \\
  \inferrule* [lab=Par] {{P} \red {P}'} {{{P} | {Q}} \red {{P}' | {Q}}}
  \and
  \inferrule* [lab=Equiv]{{{P} \scong {P}'} \andalso {{P}' \red {Q}'} \andalso {{Q}' \scong {Q}}}{{P} \red {Q}}
\end{mathpar}

\begin{eqnarray*}
  match_{\equiv} (\quotep{P},\quotep{Q}) & := & P \equiv Q \\
  match_{\dagger}(\quotep{P},\quotep{Q}) & := & \forall R. P|Q \red^{*} R => R \red^{*} 0 \\
  match_{K}(\quotep{P},\quotep{Q}) & := & K \mbox{ for some context } K
\end{eqnarray*}

$u?(x)P | u!\langle Q \rangle \red P\{\quotep{Q}/x\}$

%We write $\wred$ for $\red^*$, and $P\red$ if $\exists Q $ such that $ P \red Q$.
We write $P\red$ if $\exists Q $ such that $ P \red Q$ and $P\not\red$, otherwise.

\section{Replication}

As mentioned before, it is known that replication (and hence
recursion) can be implemented in a higher-order process algebra
\cite{SangiorgiWalker}. As our first example of calculation with the
machinery thus far presented we give the construction explicitly in
the {\rhoc}.

\begin{eqnarray}
	D_{x} & := & \prefix{x}{y}{(\binpar{\outputp{x}{y}}{@{y}})} \nonumber\\
	\bangp_{x}{P} & := & \binpar{{x}!\langle{\binpar{D_{x}}{P}}\rangle}{D_{x}} \nonumber
\end{eqnarray}

\begin{eqnarray}
	\bangp_{x}{P} & & \nonumber\\
	=
	& {x}!\langle{(\prefix{x}{y}{(\outputp{x}{y} | @{y})) | P}}\rangle 
	      | \prefix{x}{y}{(\outputp{x}{y} | @{y})} & \nonumber\\
	\red
	& (\outputp{x}{y} | @{y})\substn{\quotep{(\prefix{x}{y}{(@{y} | \outputp{x}{y})) | P}}}{y} & \nonumber\\
	=
	& \outputp{x}{\quotep{(\prefix{x}{y}{(\outputp{x}{y} | @{y})) | P}}}
	  | {(\prefix{x}{y}{(\outputp{x}{y} | @{y})) | P}} & \nonumber\\
	\red
	& \ldots & \nonumber\\
	\red^*
	& P | P | \ldots & \nonumber
\end{eqnarray}

Of course, this encoding, as an implementation, runs away, unfolding
$\bangp{P}$ eagerly. A lazier and more implementable replication
operator, restricted to input-guarded processes, may be obtained as follows.

\begin{eqnarray}
\bangp{\prefix{u}{v}{P}} 
	:= 
	\binpar{\lift{x}{\prefix{u}{v}{(\binpar{D(x)}{P})}}}{D(x)} \nonumber
\end{eqnarray}

\begin{remark}
  Note that the lazier definition still does not deal with summation
  or mixed summation (i.e. sums over input and output). The reader is
  invited to construct definitions of replication that deal with these
  features. 

  Further, the definitions are parameterized in a name, $x$. Can you,
  gentle reader, make a definition that eliminates this parameter and
  guarantees no accidental interaction between the replication
  machinery and the process being replicated -- i.e. no accidental
  sharing of names used by the process to get its work done and the
  name(s) used by the replication to effect copying. This latter
  revision of the definition of replication is crucial to obtaining
  the expected identity $!!P \sim !P$.
\end{remark}

\begin{remark}\label{rem:paradoxical_combinator}
  The reader familiar with the lambda calculus will have noticed the
  similarity between $D$ and the paradoxical combinator.

  [Ed. note: the existence of this seems to suggest we have to be more
  restrictive on the set of processes and names we admit if we are to
  support no-cloning.]
\end{remark}

\subsubsection{Bisimulation}

The computational dynamics gives rise to another kind of equivalence,
the equivalence of computational behavior. As previously mentioned
this is typically captured \emph{via} some form of bisimulation.

% The notion we use in this paper is weak barbed bisimulation
% \cite{milner91polyadicpi}.

The notion we use in this paper is derived from weak barbed
bisimulation \cite{milner91polyadicpi}. 

\begin{definition}
An \emph{observation relation}, $\downarrow_{\mathcal N}$, over a set
of names, $\mathcal N$, is the smallest relation satisfying the rules
below.

\infrule[Out-barb]{y \in {\mathcal N}, \; x \nameeq y}
		  {\outputp{x}{v} \downarrow_{\mathcal N} x}
\infrule[Par-barb]{\mbox{$P\downarrow_{\mathcal N} x$ or $Q\downarrow_{\mathcal N} x$}}
		  {\binpar{P}{Q} \downarrow_{\mathcal N} x}

We write $P \Downarrow_{\mathcal N} x$ if there is $Q$ such that 
$P \wred Q$ and $Q \downarrow_{\mathcal N} x$.
\end{definition}

\begin{definition}
%\label{def.bbisim}
An  ${\mathcal N}$-\emph{barbed bisimulation} over a set of names, ${\mathcal N}$, is a symmetric binary relation 
${\mathcal S}_{\mathcal N}$ between agents such that $P\rel{S}_{\mathcal N}Q$ implies:
\begin{enumerate}
\item If $P \red P'$ then $Q \wred Q'$ and $P'\rel{S}_{\mathcal N} Q'$.
\item If $P\downarrow_{\mathcal N} x$, then $Q\Downarrow_{\mathcal N} x$.
\end{enumerate}
$P$ is ${\mathcal N}$-barbed bisimilar to $Q$, written
$P \wbbisim_{\mathcal N} Q$, if $P \rel{S}_{\mathcal N} Q$ for some ${\mathcal N}$-barbed bisimulation ${\mathcal S}_{\mathcal N}$.
\end{definition}

$\mathcal{R} \subseteq \pi \times \pi$

$P \mathcal{R} Q => \forall P'. P \red P' \Rightarrow \exists Q'. Q \red Q', P' \mathcal{R} Q'$

$P \vdash x \Rightarrow Q \vdash x$

\begin{mathpar}
  \inferrule*[lab=Out-barb]{x \nameeq y}{{y}!\langle{Q}\rangle \vdash x}
  \and
  \inferrule*[lab=Par-barb]{\mbox{$P\vdash x$ or $Q\vdash x$}}{\binpar{P}{Q} \vdash x}
\end{mathpar}

\subsubsection{Contexts}

One of the principle advantages of computational calculi like the
$\pi$-calculus is a well-defined notion of context,
contextual-equivalence and a correlation between
contextual-equivalence and notions of bisimulation. The notion of
context allows the decomposition of a process into (sub-)process and
its syntactic environment, its context. Thus, a context may be
thought of as a process with a ``hole'' (written $\Box$) in it. The
application of a context $M$ to a process $P$, written $M[P]$, is
tantamount to filling the hole in $M$ with $P$. In this paper we do
not need the full weight of this theory, but do make use of the notion
of context in the proof the main theorem. 

\begin{mathpar}
  \inferrule* [lab=summation] {} {{M_{M},M_{N}} \bc \Box \;|\; x.M_{A} \;|\; M_{M}+M_{N}}
  \and
  \inferrule* [lab=agent] {} {{M_{A}} \bc (\vec{x})M_{P} \;| \; \clift{P_0,\ldots,M_{P},\ldots,P_N}}
  \and \\
  \inferrule* [lab=process] {} {{M_{P}} \bc M_{N} \;| \;P|M_{P} }
\end{mathpar} 

\begin{mathpar}
  \inferrule* [lab=sychronization] {} {M_{N} \bc \Box \;|\; x?M_{F} \;|\; x!M_{C}}
  \and
  \inferrule* [lab=abstraction] {} {{M_{F}} \bc (x)M_{P} }
  \and
  \inferrule* [lab=concretion] {} {{M_{C}} \bc \langle M_{P} \rangle }
  \and \\
  \inferrule* [lab=process] {} {{M_{P}} \bc M_{N} \;| \;P|M_{P} }
\end{mathpar}

\begin{definition}[contextual application] Given a context $M$, and
  process $P$, we define the \emph{contextual application}, $M[P] :=
  M\{P/\Box\}$. That is, the contextual application of M to P is the
  substitution of $P$ for $\Box$ in $M$.
\end{definition}

$\meaningof{-} : L \to \mathcal{P}(\pi)$

\begin{mathpar}
  \inferrule* [lab=collection] {} {\meaningof{true} = \pi, \and \meaningof{~E} = \pi \setminus \meaningof{E}, \and \meaningof{E_{1} \& E_{2}} = \meaningof{E_{1}} \cap \meaningof{E_{2}}}
\end{mathpar}

\begin{mathpar}
  \inferrule* [lab=structure] {} {\meaningof{0} = \{ P \in \pi | P \equiv 0 \}, \and \\ \meaningof{E_1 | E_2} = \{ P \in \pi | P \equiv P_{1} | P_{2}, P_{1} \in \meaningof{E_{1}}, P_{2} \in \meaningof{E_2}\} }
\end{mathpar}

\begin{mathpar}
 \inferrule* [lab=behavior] {} {\meaningof{\langle a?b \rangle E} = \{ P \in \pi | P \equiv Q | u?(y)P', \\ \and \\\\ \and \\ \;\;\; u \in \meaningof{a}, \forall z.P'\{z/y\} \in \meaningof{E\{z/b\}}\}, \and \\ \meaningof{a!E} = \{ P \in \pi | P \equiv Q | x!\langle P' \rangle, x \in \meaningof{a} P' \in \meaningof{E}\} }
\end{mathpar}

\begin{mathpar}
 \inferrule* [lab=nominal] {} {\meaningof{\quotep{E}} = \{ \quotep{P} \in \quotep{\pi} | P \in \meaningof{E} \}, \and \meaningof{\quotep{P}} = \{ \quotep{Q} \in \quotep{\pi} | P \equiv Q \} \and \\ \meaningof{@\quotep{E}} = \{ P \in \pi | P \equiv @x, x \in \meaningof{E} \}}
\end{mathpar}

\begin{eqnarray*}
  \\
  \meaningof{-} : TS \to ST
\end{eqnarray*}

\begin{eqnarray*}
  \\
  L : TS \to ST
\end{eqnarray*}

\begin{eqnarray*}
  \\
  P \models E \iff P \in \meaningof{E}
\end{eqnarray*}

\begin{eqnarray*}
  P \approx_{L} Q \iff \forall E \in L. P \models E \iff Q \models E
\end{eqnarray*}

\begin{eqnarray*}
  P \approx_{K} Q
\end{eqnarray*}

\begin{eqnarray*}
  P \approx Q
\end{eqnarray*}

$\approx_{K} = \approx = \approx_{L}$

\subsubsection{Contextual duality}

Note that contexts extend the quotation operation to a family of
operations from processes to names. Given a context, $M$, we can
define a \emph{nominal context}, $\quotep{M}$ by $\quotep{M}[P] :=
\quotep{M[P]}$. To foreshadow what is to come we observe that these
operations enjoy a duality with processes very much like the duality
between vectors and maps from vectors to scalars.

Further, because the calculus is essentially higher-order, we have a
correspondence between contexts and processes. More specifically,
given a name $x$ and a context $M$ we can construct $M^{*}_{x}$ such
that 

\begin{mathpar}
  M^{*}_{x} | \lift{x}{P} \red M[P]
\end{mathpar}

namely,

\begin{mathpar}
  M^{*}_{x} := x?(u).M[\dropn{u}]
\end{mathpar}

The dependence of $M^{*}_{x}$ on a name makes it an abstraction, 

\begin{mathpar}
  M^{*} := (x)x?(u).M[\dropn{u}]
\end{mathpar}

\subsection{Additional notation}

It will sometimes be convenient to denote the process a name
quotes. We already have the notation $x = \quotep{P}$, but it will be
convenient to introduce an alternate notation, $\procn{x}$, when we
want to emphasize the connection to the use of the name. Note that, by
virtue of name equivalence, $\quotep{\procn{x}} \nameeq x$; so, the
notation is consistent with previous definitions.

Further, because names have structure it is possible to effect
substitutions on the basis of that structure. This means we need to
upgrade our notation for substitutions, which we accomplish by
adapting comprehension notation. Thus,

\begin{mathpar}
  P\{ y / x : x \in S \}
\end{mathpar}

is interpreted to mean the process derived from P by replacing (in a
capture-avoiding manner) each occurrence of $x$ in $S$ by $y$. For example,

\begin{mathpar}
  P\{ \quotep{\procn{x}|\procn{x}} / x : x \in \freenames{P} \}
\end{mathpar}

will replace each (occurrence) of a free name $x$ in $P$ by
$\quotep{\procn{x}|\procn{x}}$.

Also, we will avail ourselves of the notation $x^{L}$ and $x^{R}$ to
denote injections of a name into disjoint copies of the name
space. There are numerous ways to accomplish this. One example can be
found in \cite{MeredithR05}. This notation overloads to vectors of
names: $\vec{x}^{\pi} := (x_{i}^{\pi} \; : \; 0 \leq i < |\vec{x}| )$ where $\pi \in \{L,R\}$.

We also use $P^{\Box} := P|\Box$.

In \cite{MeredithR05} an interpretation of the new operator is
given. It turns out that there are several possible interpretations
all enjoying the requisite algebraic properties of the operator (see
\cite{milner91polyadicpi}). We will therefore make liberal use of
$(\nu\; \vec{x})P$.

% subsection the_syntax_and_semantics_of_the_notation_system (end)   

\input{qm2pi.qmops} 

\input{qm2pi.sterngerlach} 

\input{qm2pi.metric} 

% section concurrent_process_calculi (end)

%\input{qm2pi.proofsketch}

% section proof sketch (end)

%\input{qm2pi.slviaknots} 

% section spatial logic via knots (end)

\input{qm2pi.conclusion}

% section conclusion (end)

%\input{qm2pi.dtcodes} 

% section wiring algorithm (end)

\input{qm2pi.ack} 

% section acknowledgments (end)

\newpage


\bibliographystyle{plain}   
\bibliography{../../biblios/main.bib}

\input{qm2pi.rhodetails}

\end{document}



% section proof sketch (end)

%\section{Unlikely characters: spatial logic for
  knots}\label{sub:characteristic_formulae} % (fold)

Associated to the mobile process calculi are a family of logics known
as the Hennessy-Milner logics. These logics typically enjoy a
semantics interpreting formulae as sets of processes that when
factored through the encoding outlined above allows an identification
of classes of knots with logical formulae. In the context of this
encoding the sub-family known as the spatial logics \cite{CairesC03}
\cite{CairesC04} \cite{Caires04} are of particular interest providing
several important features for expressing and reasoning about
properties (i.e. classes) of knots. We hint here at how this may be done.

%\begin{description}
%\item [structural connectives] 
\subsubsection{Structural connectives} The spatial logics enjoy
structural connectives corresponding, at the logical level, to the
parallel composition ($P | Q$) and new name ($(\nu \; x)P$)
connectives for processes. As illustrated in the examples below, these
connectives are extremely expressive given the shape of our encoding.
%\item [decideable satisfaction]

\subsubsection{Decideable satisfaction}
In \cite{Caires04} the satisfaction relation is shown to be decideable
for a rich class of processes. It further turns out that the image of
the our encoding is a proper subset of that class. This result
provides the basis for an algorithm by which to search for knots
enjoying a given property.
%\item [characteristic formulae]

\subsubsection{Characteristic formulae}
In the same paper \cite{Caires04} , Caires presents a means of calculating
characteristic formulae, selecting equivalence classes of processes
up to a pre--specified depth limit on the support set of names. Composed with our
encoding, this characteristic formula can be used to select
characteristic formulae for knots.
%\end{description}

\subsubsection{Spatial logic formulae}

The grammar below (segmented for comprehension) summarizes the syntax
of spatial logic formulae. We employ illustrative examples in the
sequel to provide an intuitive understanding of their meaning
referring the reader to \cite{Caires04} for a more detailed explication
of the semantics.

\begin{mathpar}
  \inferrule* [lab=boolean] {} {{A,B} \bc T \;|\; \neg A \;|\; A \wedge B \;|\; \eta = \eta'}
  \and
  \inferrule* [lab=spatial] {} {|\; \pzero \;|\; A | B \;|\; x \text{\textregistered} A \;|\; \forall x . A \;|\;  H x . A}
  \and
  \inferrule* [lab=behavioral] {} {|\; \alpha . A}
  \and 
  \inferrule* [lab=recursion] {} {|\; X(\vec{u}) \;|\; \mu X(\vec{u}) . A}
  \and
  \inferrule* [lab=action] {} {\alpha \bc \langle x?(\vec{y}) \rangle \;|\; \langle x!(\vec{y}) \rangle \;|\; \langle \tau \rangle}
  \and 
  \inferrule* [lab=name] {} {\eta \bc x \;|\; \tau}
\end{mathpar} 

% subsection characteristic_formulae (end)   	 

\subsection{Example formulae}\label{sub:example_formulae_} % (fold)

\subsubsection{Crossing as formula.}
% 
% \begin{align*}
%   \frac{d}{dx} \sin x &= \cos x 
%   & \frac{d}{dx} e^x &= e^x \\
%   \frac{d}{dx} \cos x &= - \sin x 
%   & \frac{d}{dx} \log x &= \frac{1}{x} \\
% \end{align*} 

\begin{align*}
 \mu C(x_{0},x_{1},y_{0},y_{1},u).&(\langle x_{0}?(z) \rangle(\langle u! \rangle\langle y_{1}!z \rangle C(x_{0},x_{1},y_{0},y_{1},u)) & \\
  & \wedge \langle y_{1}?(z) \rangle (\langle u! \rangle \langle x_{0}!z \rangle C(x_{0},x_{1},y_{0},y_{1},u)) & \\
  & \wedge \langle x_{1}?(z) \rangle (\langle u? \rangle \langle y_{0}!z \rangle C(x_{0},x_{1},y_{0},y_{1},u)) & \\
  & \wedge \langle y_{0}?(z) \rangle (\langle u? \rangle \langle x_{1}!z \rangle C(x_{0},x_{1},y_{0},y_{1},u))) &
\end{align*}

The lexicographical similarity between the shape of this formulae and
the shape of definition of the process representing a crossing reveals
the intuitive meaning of this formulae. It describes the capabilities
of a process that has the right to represent a crossing. For example
it picks out processes that may perform an input on the port $x_0$ in
its initial menu of capabilities. What differentiates the formula
from the process, however, is that the crossing process is the
smallest candidate to satisfy the formula. Infinitely many other
processes -- with internal behavior hidden behind this interface, so
to speak -- also satisfy this formula. Even this simple formula,
then, can be seen to open a new view onto knots, providing a
computational interpretation of \emph{virtual} knots.

Note that this formula is derived by hand. A similar formula can be
derived by employing Caires' calculation of characteristic formula
\cite{Caires04} to the process representing a crossing. In light of
this discussion, we let
$\meaningof{C}_{\phi}(x0,x1,y0,y1,u)$ denote a formula specifying the
dynamics we wish to capture of a crossing. To guarantee we preserve
the shape of the interface and minimal semantics we demand that
$\meaningof{C}_{\phi}(x0,x1,y0,y1,u) \Rightarrow
\textbf{C}(x0,x1,y0,y1,u)$ where $\textbf{C}(x0,x1,y0,y1,u)$ denotes
the formula above.
                            
\subsubsection{Crossing number constraints.}
The moral content of the context lemma (Lemma \ref{context}) is that the notion of
``locality'' in the Reidemeister moves is effectively captured by the
parallel composition operator of the process calculus. This intuition
extends through the logic. Given a formula,
$\meaningof{C}_{\phi}(x0,x1,y0,y1,u)$, we can use the structural
connectives to specify constraints on crossing numbers, such as at
least $n$ crossings, or exactly $n$ crossings.
\begin{mathpar}
  \inferrule* [lab=at-least-n] {} { K^{\geq n}_{\phi}(\vec{xs},\vec{ys}) := \Pi_{i=0}^{n-1} Hu . \meaningof{C}_{\phi}(xs_i,ys_i,u) | T }
  \and 
  \inferrule* [lab=exactly-n] {} { K^{= n}_{\phi}(\vec{xs},\vec{ys}) := \Pi_{i=0}^{n-1} Hu . \meaningof{C}_{\phi}(xs_i,ys_i,u) | \neg (\forall x_0,y_0,x_1,y_1,u . \meaningof{C}_{\phi}(x_0,y_0,x_1,y_1,u) | T) }
\end{mathpar}

To round out this section, recall that the encoding of an $n$-crossing
knot decomposes into a parallel composition of $n$ \emph{copies} of a
crossing process together with a wiring harness. To specify different
knot classes with the same crossing number amounts to specifying
logical constraints on the wiring harness. In the interest of space,
we defer examples to a forthcoming paper. Suffice it to say that both
the conditions ``alternating knot'' and ``contains the tangle
corresponding to 5/3'' are expressible. For example, it is possible to
calculate the characteristic formula of a process corresponding to the
tangle 5/3 and conjoin it into the classifying formula via the
composition connective of the logic.

Finally, we wish to observe that it is entirely within reason to
contemplate a more domain-specific version of spatial logic tailored
to the shape of processes in the image of the encoding. Such a
domain-specific logic would have a better claim to the title formal
language of knot properties.

% subsection example_formulae_ (end)

% section knots_as_processes (end) 

% section spatial logic via knots (end)

\section{Conclusions and future work}

\paragraph{Testing physical space}
You, gentle reader, may wonder why of all the theorems to be proved
given this set up we pick the one above. In some sense it's hardly
central to quantum mechanics. We see it as central in the sense that
it firmly establishes a notion of physical space arising from a notion
of the equivalence of behavior. Relating bisimulation to a metric is a
big step forward, but one is faced with interpreting the relationship
of that metric space to something more physical. Quantum mechanical
notions of ``physical'' space are still far from intuitive, but by
relating this idea of distance as testing to calculations that predict
physical circumstances we are making a not insignificant step forward
toward an understanding of the physical space we inhabit as
essentially dynamic.

\paragraph{Effectivity and simulation}
One of the observations we have yet to make is that the entire program
spelled out here is effective. We have built various interpreters for
the reflective calculus at work in this interpretation. In principle,
then, we can simulate quantum mechanics on a computer. The place where
the simulation may lose fidelity is the infinitely branching summation
for the annihilator.

In this connection i also want to point out that the evaluation style
calculation of the inner product puts the non-determinism of the
summation right at the heart of measurement. This suggests that
Milner's original reduction-based formulation of the dynamics of his
calculi in terms of sums was not just notationally suggestive of a
notion of measure-and-continue but captured some significant part of
the physics.

\paragraph{Quantum continuations}
In light of this last observation i want to point out that the
predominant account of quantum mechanics is missing a key aspect of a
truly compositional story of the physical situation. In a real lab,
when a measurement is made the observation can be made to feed into
another device that then makes another measurement conditioned on the
results of the first. This means that after the superposition was
collapsed the entire experimental set up remained in
superposition. While QM offers a means of writing this down it doesn't
quite line up well with the well-trodden formulation of computation
and continuation that we see so succinctly expressed in Milner's
calculi. This suggests that there might be advantages to this account
of dynamics waiting to be explored.

\paragraph{Quantum logic}
In this connection, we also note that by virtue of having the
Hennessy-Milner construction, we can pull the construction through the
interpretation of QM. This gives us a natural candidate for a quantum
logic that enjoys an extremely tight connection with it's domain of
interpretation, making the construction much less ad hoc (rather it is
the image of functor!).

\paragraph{Quantum probabiity}
i have questions about the basis of the interpretation of inner
product as probability amplitude. In particular, using which
axiomatization of probability theory does the notion of probability
amplitude earn the right to be so dubbed? In other words, where is the
proof that the operation for calculating a probability amplitude (and
then squaring) satisfies the axioms of what it means to calculate a
probability? Even if such a proof exists (i have yet to find it in the
literature), i wonder if it might not be possible to turn things on
their heads. Can we view the calculation of the probability amplitude
as an axiomatization of probability? If so, then the definition we
give for calculating probability amplitude may provide the basis for
an \emph{effective} theory of probability.

\paragraph{Quantum vs ``biological'' information}
Finally, i want to conclude with a more philosophical observation. At
a recent workshop in which QM was a predominant topic i noticed
something about quantum information. The speaker was giving a riveting
discussion of axiomatic QM and showing how properties of ``no
cloning'' and ``no deleting'' emerged as consequences of the
axiomatization. Theorems of this form are necessary to give us a sense
of confidence that our axioms characterize the physical theory. What
struck me, though, was that if quantum information is neither erasable
nor replicable it is markedly different from \emph{life}. Two of the
things we know about life is that

\begin{itemize}
  \item it ends;
  \item to gain some measure of persistence, to transcend it's
    finitude it is imminently copyable.
\end{itemize}

Both of these qualities are summarized succinctly in the aphorism: all
flesh is grass. For me these two kinds of ``information'' -- call them
quantum and biological -- are end points on a spectrum of strategies
for persistence. At one end, we have those curious entities that enjoy
uniqueness and permanence; at the other, we have those who in the face
of a certain end and an uncertain present make a go of passing
something on. To me one of the more remarkable aspects of the latter
strategy is that in the presence of noise (and certain features of
copying) we get a kind of dynamism, a chance for improvement against a
given persistent condition.

% subsection other_calculi_other_bisimulations_and_geometry_as_behavior (end)




% section conclusion (end)

%\documentclass[12pt]{llncs}
%\documentclass{jktr}

\usepackage[pdftex]{hyperref}                   
\usepackage {listings}
\usepackage {mathpartir}
\usepackage{bcprules}
%\usepackage{listings}
                       
\usepackage{graphicx} 
%\usepackage[margins=2.5cm,nohead,nofoot]{geometry}
%\usepackage{geometry}
\usepackage{amsfonts}
\usepackage{amstext}
\usepackage{latexsym}
\usepackage{amssymb}
\usepackage{color}


%\include{myPreamble}
\include{qm2pi.local} 

%\ifpdf
%\usepackage[pdftex]{graphicx}
%\else
%\usepackage{graphicx}
%\fi

 % \ifpdf
%  \usepackage{pdfsync}
%  \if


%\title{Brief Article}
%\author{David F. Snyder}
%\author{L.G. Meredith}

%\address{Dept. of Math., Texas State University--San Marcos, San Marcos, TX 78666}
       
\pagestyle{empty}


\begin{document}

\lstset{language=[Objective]Caml,frame=shadowbox}

\input{qm2pi.front}

% section front matter (end)

\input{qm2pi.intro} 
 
% section introduction (end)

% \input{qm2pi.knotations} 

% section notation (end)

\input{qm2pi.process.calculi} 

% section concurrent_process_calculi_and_spatial_logics_ (end)
    
%\input{qm2pi.knots2pi} 

%\input{qm2pi.trefoil} 

%\input{qm2pi.mainthm} 

% subsection basic_interpretation (end)

%\input{qm2pi.rho.presentation} 
\subsection{The syntax and semantics of the notation system}\label{sub:the_syntax_and_semantics_of_the_notation_system} % (fold)

We now summarize a technical presentation of the calculus that
embodies our theory of dynamics. The typical presentation of such a
calculus follows the style of giving generators and relations on
them. The grammar, below, describing term constructors, freely
generates the set of processes, $\Proc$. This set is then quotiented
by a relation known as structural congruence and it is over this set
that the notion of dynamics is expressed. This presentation is
essentially that of \cite{MeredithR05} with the addition of
polyadicity and summation. For readability we have relegated some of
the technical subtleties to an appendix.

\subsubsection{Process grammar}\label{subsub:process_grammar}

\begin{mathpar}
  \inferrule* [lab=synchronization] {} {{M} \bc \pzero \;|\; x?F \;|\; x!C }
  \and
  \inferrule* [lab=abstraction] {} {{F} \bc (x)P}
  \and
  \inferrule* [lab=concretion] {} {{C} \bc \langle Q \rangle}
  \and
  \inferrule* [lab=process] {} {{P,Q} \bc M \;| \;P|Q \;|\; @{x}}
  \and
  \inferrule* [lab=name] {} {{x} \bc \quotep{P}}
\end{mathpar} 

Note that $\vec{x}$ (resp. $\vec{P}$) denotes a vector of names
(resp. processes) of length $|\vec{x}|$ (resp. $|\vec{P}|$). We adopt
the following useful abbreviations.

\begin{mathpar}
   x?(\vec{y}).P := x.(\vec{y})P \and  x\clift{\vec{P}} := x.\clift{\vec{P}}
   \and x!(y) := \lift{x}{\dropn{y}}
   \and \Pi_{i=0}^{n-1}P_i := P_0 | \ldots | P_{n-1}
\end{mathpar}

\subsubsection{Structural congruence}

\paragraph{Free and bound names and alpha-equivalence.} At the
core of structural equivalence is alpha-equivalence which identifies
process that are the same up to a change of variable. Formally, we
recognize the distinction between free and bound names. The free names
of a process, $\freenames{P}$, may be calculated recursively as
follows:

\begin{mathpar}
\freenames{\pzero} := \emptyset
  \and \\
  \freenames{x?(y).P} := \{ x \} \cup (\freenames{P} \setminus \{ y \})
  \and 
  \freenames{x!\langle P \rangle} := \{ x \} \cup \{ P \} 
  \and \\
  \freenames{P|Q} := \freenames{P} \cup \freenames{Q}
  \and \\
  \freenames{@{x}} := \{ x \}
\end{mathpar}

$\pi$
$\quotep{\pi}$

$\freenames{-} : \pi \to \mathcal{P}(\quotep{\pi})$

\begin{eqnarray*}
  \freenames{\pzero} & := & \emptyset \\
  \freenames{x?(y).P} & := & \{ x \} \cup (\freenames{P} \setminus \{ y \}) \\
  \freenames{x!\langle P \rangle} & := & \{ x \} \cup \{ P \} \\
  \freenames{P|Q} & := & \freenames{P} \cup \freenames{Q} \\
  \freenames{\dropn{x}} & := & \{ x \}
\end{eqnarray*}

The bound names of a process, $\boundnames{P}$, are those names occurring in $P$
that are not free. For example, in $x?(y).0$, the name $x$ is free, while $y$ is bound.

\begin{mathpar}
  \inferrule* [lab=monoidal-laws] {} { P|Q \equiv Q|P \and P|0 \equiv P \and P|(Q|R) \equiv (P|Q)|R }
\end{mathpar}

\begin{mathpar}
  \inferrule* [lab=alpha-equivalence] {} { (x)P \equiv (y)P\{y/x\} \and y \not\in \freenames{P} }
\end{mathpar}

\begin{definition}
Then two processes, $P,Q$, are alpha-equivalent if $P = Q\{\vec{y}/\vec{x}\}$ for
some $\vec{x} \in \boundnames{Q},\vec{y} \in \boundnames{P}$, where $Q\{\vec{y}/\vec{x}\}$
denotes the capture-avoiding substitution of $\vec{y}$ for $\vec{x}$ in $Q$.
\end{definition}

\begin{definition}
  The {\em structural congruence} \cite{SangiorgiWalker} , $\equiv$,
  between processes is the least congruence containing
  alpha-equivalence, satisfying the abelian monoid laws
  (associativity, commutativity and $\pzero$ as identity) for parallel
  composition $|$ and for summation $+$.
\end{definition}

\subsection{Name equivalence}

We take name equivalence, written $\nameeq$, to be the smallest
equivalence relation generated by the following rules.

\begin{mathpar}
\inferrule*[lab=Quote-drop]
{ }
{ \quotep{@{x}} \nameeq x }

\inferrule*[lab=Struct-equiv]
{ P \scong Q }
{ \quotep{P} \nameeq \quotep{Q} }
\end{mathpar}

The astute reader will have noticed that the mutual recursion of names
and processes imposes a mutual recursion on alpha-equivalence and
structural equivalence via name-equivalence. Fortunately, all of this
works out pleasantly and we may calculate in the natural way, free of
concern. The reader interested in the details is referred to the
appendix \ref{appendix:rho_details}.

\subsection{Substitution}

We use $\Proc$ for the set of processes, $\QProc$ for the set of
names, and $\id{\{}\vec{y} / \vec{x} \id{\}}$ to denote partial maps,
$s : \QProc \rightarrow \QProc$. A map, $s$ lifts, uniquely, to a map
on process terms, $\widehat{s} : \Proc \rightarrow \Proc$ by the
following equations.

\begin{mathpar}
  (0) \psubstp{Q}{P} := 0 \\
  (R \juxtap S) \psubstp{Q}{P}
  :=    
  (R)\psubstp{Q}{P} \juxtap (S) \psubstp{Q}{P} \\
  (x?(y).R) \psubstp{Q}{P}    
  :=    
  (x)\substp{Q}{P} (z)\concat( (R \psubstn{z}{y}) \psubstp{Q}{P} ) \\
  (\lift{x}{R}) \psubstp{Q}{P}  
  :=
  \lift{(x)\substp{Q}{P}}{ R \psubstp{Q}{P} } \\
%   (\dropn{x})  \psubstp{Q}{P}       
%   := 
%   \left\{ 
%     \begin{array}{ccc} 
%       \dropn{\quotep{Q}} & & x \nameeq \quotep{P} \\
%       \dropn{x} & & otherwise \\
%     \end{array}
%   \right. 
  (\dropn{x})  \psubstp{Q}{P}       
  := 
  \left\{ 
    \begin{array}{ccc} 
      Q & & x \nameeq \quotep{P} \\
      \dropn{x} & & otherwise \\
    \end{array}
  \right.
\end{mathpar}
 

where

\begin{eqnarray}
  (x)\id{\{} \lpquote Q \rpquote / \lpquote P \rpquote \id{\}}            = 
  \left\{ 
    \begin{array}{ccc}
      \lpquote Q \rpquote & & x \nameeq \lpquote P \rpquote \\
      x & & otherwise \\
    \end{array}
  \right. \nonumber
\end{eqnarray}

and $z$ is chosen distinct from $\quotep{P}$, $\quotep{Q}$, the free
names in $Q$, and all the names in $R$. Our $\alpha$-equivalence will
be built in the standard way from this substitution.

\begin{remark}\label{rem:no_self_referential_names}
  One consequence of these definitions is that $\forall P. \quotep{P}
  \not\in \freenames{P}$.
\end{remark}

\subsection{ Dynamic quote: an example }

Anticipating something of what's to come, consider applying the
substitution, $\widehat{\id{\{}u / z \id{\}}}$, to the following pair
of processes, $\lift{w}{y!(z)}$ and $w[ \lpquote y!(z) \rpquote ]$.

\begin{eqnarray}
	\lift{w}{y!(z)}\widehat{\id{\{}u / z \id{\}}}
		& = &
		\lift{w}{y!(u)} \nonumber\\
	w[ \lpquote y!(z) \rpquote ] \widehat{ \id{\{}u / z \id{\}} }
		& = &
		w[ \lpquote y!(z) \rpquote ] \nonumber
\end{eqnarray}

Because the body of the process between quotes is impervious to
substitution, we get radically different answers. In fact, by
examining the first process in an input context,
e.g. $x?(z).\lift{w}{y!(z)}$, we see that the process under the lift
operator may be shaped by prefixed inputs binding a name inside it. In
this sense, the lift operator will be seen as a way to dynamically
construct processes before reifying them as names.

Finally equipped with these standard features we can present the
dynamics of the calculus.

\subsubsection{Operational semantics} 

Finally, we introduce the computational dynamics. What marks these
algebras as distinct from other more traditionally studied algebraic
structures, e.g. vector spaces or polynomial rings, is the manner in
which dynamics is captured. In traditional structures, dynamics is typically
expressed through morphisms between such structures, as in linear maps
between vector spaces or morphisms between rings. In algebras
associated with the semantics of computation, the dynamics is
expressed as part of the algebraic structure itself, through a
reduction reduction relation typically denoted by $\red$. Below, we
give a recursive presentation of this relation for the calculus used
in the encoding.

$\red \subseteq \pi \times \pi$
$\red : \pi \to \mathcal{P}(\pi)$

\begin{mathpar}
  \inferrule* [lab=Comm] { \textsf{match}( x_{src}, x_{trgt} ) } { x_{trgt}?(y)P \; | \; x_{src}!\langle {Q} \rangle \red P\{\quotep{Q}/y}\} }
  \and \\
  \inferrule* [lab=Par] {{P} \red {P}'} {{{P} | {Q}} \red {{P}' | {Q}}}
  \and
  \inferrule* [lab=Equiv]{{{P} \scong {P}'} \andalso {{P}' \red {Q}'} \andalso {{Q}' \scong {Q}}}{{P} \red {Q}}
\end{mathpar}

\begin{eqnarray*}
  match_{\equiv} (\quotep{P},\quotep{Q}) & := & P \equiv Q \\
  match_{\dagger}(\quotep{P},\quotep{Q}) & := & \forall R. P|Q \red^{*} R => R \red^{*} 0 \\
  match_{K}(\quotep{P},\quotep{Q}) & := & K \mbox{ for some context } K
\end{eqnarray*}

$u?(x)P | u!\langle Q \rangle \red P\{\quotep{Q}/x\}$

%We write $\wred$ for $\red^*$, and $P\red$ if $\exists Q $ such that $ P \red Q$.
We write $P\red$ if $\exists Q $ such that $ P \red Q$ and $P\not\red$, otherwise.

\section{Replication}

As mentioned before, it is known that replication (and hence
recursion) can be implemented in a higher-order process algebra
\cite{SangiorgiWalker}. As our first example of calculation with the
machinery thus far presented we give the construction explicitly in
the {\rhoc}.

\begin{eqnarray}
	D_{x} & := & \prefix{x}{y}{(\binpar{\outputp{x}{y}}{@{y}})} \nonumber\\
	\bangp_{x}{P} & := & \binpar{{x}!\langle{\binpar{D_{x}}{P}}\rangle}{D_{x}} \nonumber
\end{eqnarray}

\begin{eqnarray}
	\bangp_{x}{P} & & \nonumber\\
	=
	& {x}!\langle{(\prefix{x}{y}{(\outputp{x}{y} | @{y})) | P}}\rangle 
	      | \prefix{x}{y}{(\outputp{x}{y} | @{y})} & \nonumber\\
	\red
	& (\outputp{x}{y} | @{y})\substn{\quotep{(\prefix{x}{y}{(@{y} | \outputp{x}{y})) | P}}}{y} & \nonumber\\
	=
	& \outputp{x}{\quotep{(\prefix{x}{y}{(\outputp{x}{y} | @{y})) | P}}}
	  | {(\prefix{x}{y}{(\outputp{x}{y} | @{y})) | P}} & \nonumber\\
	\red
	& \ldots & \nonumber\\
	\red^*
	& P | P | \ldots & \nonumber
\end{eqnarray}

Of course, this encoding, as an implementation, runs away, unfolding
$\bangp{P}$ eagerly. A lazier and more implementable replication
operator, restricted to input-guarded processes, may be obtained as follows.

\begin{eqnarray}
\bangp{\prefix{u}{v}{P}} 
	:= 
	\binpar{\lift{x}{\prefix{u}{v}{(\binpar{D(x)}{P})}}}{D(x)} \nonumber
\end{eqnarray}

\begin{remark}
  Note that the lazier definition still does not deal with summation
  or mixed summation (i.e. sums over input and output). The reader is
  invited to construct definitions of replication that deal with these
  features. 

  Further, the definitions are parameterized in a name, $x$. Can you,
  gentle reader, make a definition that eliminates this parameter and
  guarantees no accidental interaction between the replication
  machinery and the process being replicated -- i.e. no accidental
  sharing of names used by the process to get its work done and the
  name(s) used by the replication to effect copying. This latter
  revision of the definition of replication is crucial to obtaining
  the expected identity $!!P \sim !P$.
\end{remark}

\begin{remark}\label{rem:paradoxical_combinator}
  The reader familiar with the lambda calculus will have noticed the
  similarity between $D$ and the paradoxical combinator.

  [Ed. note: the existence of this seems to suggest we have to be more
  restrictive on the set of processes and names we admit if we are to
  support no-cloning.]
\end{remark}

\subsubsection{Bisimulation}

The computational dynamics gives rise to another kind of equivalence,
the equivalence of computational behavior. As previously mentioned
this is typically captured \emph{via} some form of bisimulation.

% The notion we use in this paper is weak barbed bisimulation
% \cite{milner91polyadicpi}.

The notion we use in this paper is derived from weak barbed
bisimulation \cite{milner91polyadicpi}. 

\begin{definition}
An \emph{observation relation}, $\downarrow_{\mathcal N}$, over a set
of names, $\mathcal N$, is the smallest relation satisfying the rules
below.

\infrule[Out-barb]{y \in {\mathcal N}, \; x \nameeq y}
		  {\outputp{x}{v} \downarrow_{\mathcal N} x}
\infrule[Par-barb]{\mbox{$P\downarrow_{\mathcal N} x$ or $Q\downarrow_{\mathcal N} x$}}
		  {\binpar{P}{Q} \downarrow_{\mathcal N} x}

We write $P \Downarrow_{\mathcal N} x$ if there is $Q$ such that 
$P \wred Q$ and $Q \downarrow_{\mathcal N} x$.
\end{definition}

\begin{definition}
%\label{def.bbisim}
An  ${\mathcal N}$-\emph{barbed bisimulation} over a set of names, ${\mathcal N}$, is a symmetric binary relation 
${\mathcal S}_{\mathcal N}$ between agents such that $P\rel{S}_{\mathcal N}Q$ implies:
\begin{enumerate}
\item If $P \red P'$ then $Q \wred Q'$ and $P'\rel{S}_{\mathcal N} Q'$.
\item If $P\downarrow_{\mathcal N} x$, then $Q\Downarrow_{\mathcal N} x$.
\end{enumerate}
$P$ is ${\mathcal N}$-barbed bisimilar to $Q$, written
$P \wbbisim_{\mathcal N} Q$, if $P \rel{S}_{\mathcal N} Q$ for some ${\mathcal N}$-barbed bisimulation ${\mathcal S}_{\mathcal N}$.
\end{definition}

$\mathcal{R} \subseteq \pi \times \pi$

$P \mathcal{R} Q => \forall P'. P \red P' \Rightarrow \exists Q'. Q \red Q', P' \mathcal{R} Q'$

$P \vdash x \Rightarrow Q \vdash x$

\begin{mathpar}
  \inferrule*[lab=Out-barb]{x \nameeq y}{{y}!\langle{Q}\rangle \vdash x}
  \and
  \inferrule*[lab=Par-barb]{\mbox{$P\vdash x$ or $Q\vdash x$}}{\binpar{P}{Q} \vdash x}
\end{mathpar}

\subsubsection{Contexts}

One of the principle advantages of computational calculi like the
$\pi$-calculus is a well-defined notion of context,
contextual-equivalence and a correlation between
contextual-equivalence and notions of bisimulation. The notion of
context allows the decomposition of a process into (sub-)process and
its syntactic environment, its context. Thus, a context may be
thought of as a process with a ``hole'' (written $\Box$) in it. The
application of a context $M$ to a process $P$, written $M[P]$, is
tantamount to filling the hole in $M$ with $P$. In this paper we do
not need the full weight of this theory, but do make use of the notion
of context in the proof the main theorem. 

\begin{mathpar}
  \inferrule* [lab=summation] {} {{M_{M},M_{N}} \bc \Box \;|\; x.M_{A} \;|\; M_{M}+M_{N}}
  \and
  \inferrule* [lab=agent] {} {{M_{A}} \bc (\vec{x})M_{P} \;| \; \clift{P_0,\ldots,M_{P},\ldots,P_N}}
  \and \\
  \inferrule* [lab=process] {} {{M_{P}} \bc M_{N} \;| \;P|M_{P} }
\end{mathpar} 

\begin{mathpar}
  \inferrule* [lab=sychronization] {} {M_{N} \bc \Box \;|\; x?M_{F} \;|\; x!M_{C}}
  \and
  \inferrule* [lab=abstraction] {} {{M_{F}} \bc (x)M_{P} }
  \and
  \inferrule* [lab=concretion] {} {{M_{C}} \bc \langle M_{P} \rangle }
  \and \\
  \inferrule* [lab=process] {} {{M_{P}} \bc M_{N} \;| \;P|M_{P} }
\end{mathpar}

\begin{definition}[contextual application] Given a context $M$, and
  process $P$, we define the \emph{contextual application}, $M[P] :=
  M\{P/\Box\}$. That is, the contextual application of M to P is the
  substitution of $P$ for $\Box$ in $M$.
\end{definition}

$\meaningof{-} : L \to \mathcal{P}(\pi)$

\begin{mathpar}
  \inferrule* [lab=collection] {} {\meaningof{true} = \pi, \and \meaningof{~E} = \pi \setminus \meaningof{E}, \and \meaningof{E_{1} \& E_{2}} = \meaningof{E_{1}} \cap \meaningof{E_{2}}}
\end{mathpar}

\begin{mathpar}
  \inferrule* [lab=structure] {} {\meaningof{0} = \{ P \in \pi | P \equiv 0 \}, \and \\ \meaningof{E_1 | E_2} = \{ P \in \pi | P \equiv P_{1} | P_{2}, P_{1} \in \meaningof{E_{1}}, P_{2} \in \meaningof{E_2}\} }
\end{mathpar}

\begin{mathpar}
 \inferrule* [lab=behavior] {} {\meaningof{\langle a?b \rangle E} = \{ P \in \pi | P \equiv Q | u?(y)P', \\ \and \\\\ \and \\ \;\;\; u \in \meaningof{a}, \forall z.P'\{z/y\} \in \meaningof{E\{z/b\}}\}, \and \\ \meaningof{a!E} = \{ P \in \pi | P \equiv Q | x!\langle P' \rangle, x \in \meaningof{a} P' \in \meaningof{E}\} }
\end{mathpar}

\begin{mathpar}
 \inferrule* [lab=nominal] {} {\meaningof{\quotep{E}} = \{ \quotep{P} \in \quotep{\pi} | P \in \meaningof{E} \}, \and \meaningof{\quotep{P}} = \{ \quotep{Q} \in \quotep{\pi} | P \equiv Q \} \and \\ \meaningof{@\quotep{E}} = \{ P \in \pi | P \equiv @x, x \in \meaningof{E} \}}
\end{mathpar}

\begin{eqnarray*}
  \\
  \meaningof{-} : TS \to ST
\end{eqnarray*}

\begin{eqnarray*}
  \\
  L : TS \to ST
\end{eqnarray*}

\begin{eqnarray*}
  \\
  P \models E \iff P \in \meaningof{E}
\end{eqnarray*}

\begin{eqnarray*}
  P \approx_{L} Q \iff \forall E \in L. P \models E \iff Q \models E
\end{eqnarray*}

\begin{eqnarray*}
  P \approx_{K} Q
\end{eqnarray*}

\begin{eqnarray*}
  P \approx Q
\end{eqnarray*}

$\approx_{K} = \approx = \approx_{L}$

\subsubsection{Contextual duality}

Note that contexts extend the quotation operation to a family of
operations from processes to names. Given a context, $M$, we can
define a \emph{nominal context}, $\quotep{M}$ by $\quotep{M}[P] :=
\quotep{M[P]}$. To foreshadow what is to come we observe that these
operations enjoy a duality with processes very much like the duality
between vectors and maps from vectors to scalars.

Further, because the calculus is essentially higher-order, we have a
correspondence between contexts and processes. More specifically,
given a name $x$ and a context $M$ we can construct $M^{*}_{x}$ such
that 

\begin{mathpar}
  M^{*}_{x} | \lift{x}{P} \red M[P]
\end{mathpar}

namely,

\begin{mathpar}
  M^{*}_{x} := x?(u).M[\dropn{u}]
\end{mathpar}

The dependence of $M^{*}_{x}$ on a name makes it an abstraction, 

\begin{mathpar}
  M^{*} := (x)x?(u).M[\dropn{u}]
\end{mathpar}

\subsection{Additional notation}

It will sometimes be convenient to denote the process a name
quotes. We already have the notation $x = \quotep{P}$, but it will be
convenient to introduce an alternate notation, $\procn{x}$, when we
want to emphasize the connection to the use of the name. Note that, by
virtue of name equivalence, $\quotep{\procn{x}} \nameeq x$; so, the
notation is consistent with previous definitions.

Further, because names have structure it is possible to effect
substitutions on the basis of that structure. This means we need to
upgrade our notation for substitutions, which we accomplish by
adapting comprehension notation. Thus,

\begin{mathpar}
  P\{ y / x : x \in S \}
\end{mathpar}

is interpreted to mean the process derived from P by replacing (in a
capture-avoiding manner) each occurrence of $x$ in $S$ by $y$. For example,

\begin{mathpar}
  P\{ \quotep{\procn{x}|\procn{x}} / x : x \in \freenames{P} \}
\end{mathpar}

will replace each (occurrence) of a free name $x$ in $P$ by
$\quotep{\procn{x}|\procn{x}}$.

Also, we will avail ourselves of the notation $x^{L}$ and $x^{R}$ to
denote injections of a name into disjoint copies of the name
space. There are numerous ways to accomplish this. One example can be
found in \cite{MeredithR05}. This notation overloads to vectors of
names: $\vec{x}^{\pi} := (x_{i}^{\pi} \; : \; 0 \leq i < |\vec{x}| )$ where $\pi \in \{L,R\}$.

We also use $P^{\Box} := P|\Box$.

In \cite{MeredithR05} an interpretation of the new operator is
given. It turns out that there are several possible interpretations
all enjoying the requisite algebraic properties of the operator (see
\cite{milner91polyadicpi}). We will therefore make liberal use of
$(\nu\; \vec{x})P$.

% subsection the_syntax_and_semantics_of_the_notation_system (end)   

\input{qm2pi.qmops} 

\input{qm2pi.sterngerlach} 

\input{qm2pi.metric} 

% section concurrent_process_calculi (end)

%\input{qm2pi.proofsketch}

% section proof sketch (end)

%\input{qm2pi.slviaknots} 

% section spatial logic via knots (end)

\input{qm2pi.conclusion}

% section conclusion (end)

%\input{qm2pi.dtcodes} 

% section wiring algorithm (end)

\input{qm2pi.ack} 

% section acknowledgments (end)

\newpage


\bibliographystyle{plain}   
\bibliography{../../biblios/main.bib}

\input{qm2pi.rhodetails}

\end{document}

 

% section wiring algorithm (end)

\documentclass[12pt]{llncs}
%\documentclass{jktr}

\usepackage[pdftex]{hyperref}                   
\usepackage {listings}
\usepackage {mathpartir}
\usepackage{bcprules}
%\usepackage{listings}
                       
\usepackage{graphicx} 
%\usepackage[margins=2.5cm,nohead,nofoot]{geometry}
%\usepackage{geometry}
\usepackage{amsfonts}
\usepackage{amstext}
\usepackage{latexsym}
\usepackage{amssymb}
\usepackage{color}


%\include{myPreamble}
\include{qm2pi.local} 

%\ifpdf
%\usepackage[pdftex]{graphicx}
%\else
%\usepackage{graphicx}
%\fi

 % \ifpdf
%  \usepackage{pdfsync}
%  \if


%\title{Brief Article}
%\author{David F. Snyder}
%\author{L.G. Meredith}

%\address{Dept. of Math., Texas State University--San Marcos, San Marcos, TX 78666}
       
\pagestyle{empty}


\begin{document}

\lstset{language=[Objective]Caml,frame=shadowbox}

\input{qm2pi.front}

% section front matter (end)

\input{qm2pi.intro} 
 
% section introduction (end)

% \input{qm2pi.knotations} 

% section notation (end)

\input{qm2pi.process.calculi} 

% section concurrent_process_calculi_and_spatial_logics_ (end)
    
%\input{qm2pi.knots2pi} 

%\input{qm2pi.trefoil} 

%\input{qm2pi.mainthm} 

% subsection basic_interpretation (end)

%\input{qm2pi.rho.presentation} 
\subsection{The syntax and semantics of the notation system}\label{sub:the_syntax_and_semantics_of_the_notation_system} % (fold)

We now summarize a technical presentation of the calculus that
embodies our theory of dynamics. The typical presentation of such a
calculus follows the style of giving generators and relations on
them. The grammar, below, describing term constructors, freely
generates the set of processes, $\Proc$. This set is then quotiented
by a relation known as structural congruence and it is over this set
that the notion of dynamics is expressed. This presentation is
essentially that of \cite{MeredithR05} with the addition of
polyadicity and summation. For readability we have relegated some of
the technical subtleties to an appendix.

\subsubsection{Process grammar}\label{subsub:process_grammar}

\begin{mathpar}
  \inferrule* [lab=synchronization] {} {{M} \bc \pzero \;|\; x?F \;|\; x!C }
  \and
  \inferrule* [lab=abstraction] {} {{F} \bc (x)P}
  \and
  \inferrule* [lab=concretion] {} {{C} \bc \langle Q \rangle}
  \and
  \inferrule* [lab=process] {} {{P,Q} \bc M \;| \;P|Q \;|\; @{x}}
  \and
  \inferrule* [lab=name] {} {{x} \bc \quotep{P}}
\end{mathpar} 

Note that $\vec{x}$ (resp. $\vec{P}$) denotes a vector of names
(resp. processes) of length $|\vec{x}|$ (resp. $|\vec{P}|$). We adopt
the following useful abbreviations.

\begin{mathpar}
   x?(\vec{y}).P := x.(\vec{y})P \and  x\clift{\vec{P}} := x.\clift{\vec{P}}
   \and x!(y) := \lift{x}{\dropn{y}}
   \and \Pi_{i=0}^{n-1}P_i := P_0 | \ldots | P_{n-1}
\end{mathpar}

\subsubsection{Structural congruence}

\paragraph{Free and bound names and alpha-equivalence.} At the
core of structural equivalence is alpha-equivalence which identifies
process that are the same up to a change of variable. Formally, we
recognize the distinction between free and bound names. The free names
of a process, $\freenames{P}$, may be calculated recursively as
follows:

\begin{mathpar}
\freenames{\pzero} := \emptyset
  \and \\
  \freenames{x?(y).P} := \{ x \} \cup (\freenames{P} \setminus \{ y \})
  \and 
  \freenames{x!\langle P \rangle} := \{ x \} \cup \{ P \} 
  \and \\
  \freenames{P|Q} := \freenames{P} \cup \freenames{Q}
  \and \\
  \freenames{@{x}} := \{ x \}
\end{mathpar}

$\pi$
$\quotep{\pi}$

$\freenames{-} : \pi \to \mathcal{P}(\quotep{\pi})$

\begin{eqnarray*}
  \freenames{\pzero} & := & \emptyset \\
  \freenames{x?(y).P} & := & \{ x \} \cup (\freenames{P} \setminus \{ y \}) \\
  \freenames{x!\langle P \rangle} & := & \{ x \} \cup \{ P \} \\
  \freenames{P|Q} & := & \freenames{P} \cup \freenames{Q} \\
  \freenames{\dropn{x}} & := & \{ x \}
\end{eqnarray*}

The bound names of a process, $\boundnames{P}$, are those names occurring in $P$
that are not free. For example, in $x?(y).0$, the name $x$ is free, while $y$ is bound.

\begin{mathpar}
  \inferrule* [lab=monoidal-laws] {} { P|Q \equiv Q|P \and P|0 \equiv P \and P|(Q|R) \equiv (P|Q)|R }
\end{mathpar}

\begin{mathpar}
  \inferrule* [lab=alpha-equivalence] {} { (x)P \equiv (y)P\{y/x\} \and y \not\in \freenames{P} }
\end{mathpar}

\begin{definition}
Then two processes, $P,Q$, are alpha-equivalent if $P = Q\{\vec{y}/\vec{x}\}$ for
some $\vec{x} \in \boundnames{Q},\vec{y} \in \boundnames{P}$, where $Q\{\vec{y}/\vec{x}\}$
denotes the capture-avoiding substitution of $\vec{y}$ for $\vec{x}$ in $Q$.
\end{definition}

\begin{definition}
  The {\em structural congruence} \cite{SangiorgiWalker} , $\equiv$,
  between processes is the least congruence containing
  alpha-equivalence, satisfying the abelian monoid laws
  (associativity, commutativity and $\pzero$ as identity) for parallel
  composition $|$ and for summation $+$.
\end{definition}

\subsection{Name equivalence}

We take name equivalence, written $\nameeq$, to be the smallest
equivalence relation generated by the following rules.

\begin{mathpar}
\inferrule*[lab=Quote-drop]
{ }
{ \quotep{@{x}} \nameeq x }

\inferrule*[lab=Struct-equiv]
{ P \scong Q }
{ \quotep{P} \nameeq \quotep{Q} }
\end{mathpar}

The astute reader will have noticed that the mutual recursion of names
and processes imposes a mutual recursion on alpha-equivalence and
structural equivalence via name-equivalence. Fortunately, all of this
works out pleasantly and we may calculate in the natural way, free of
concern. The reader interested in the details is referred to the
appendix \ref{appendix:rho_details}.

\subsection{Substitution}

We use $\Proc$ for the set of processes, $\QProc$ for the set of
names, and $\id{\{}\vec{y} / \vec{x} \id{\}}$ to denote partial maps,
$s : \QProc \rightarrow \QProc$. A map, $s$ lifts, uniquely, to a map
on process terms, $\widehat{s} : \Proc \rightarrow \Proc$ by the
following equations.

\begin{mathpar}
  (0) \psubstp{Q}{P} := 0 \\
  (R \juxtap S) \psubstp{Q}{P}
  :=    
  (R)\psubstp{Q}{P} \juxtap (S) \psubstp{Q}{P} \\
  (x?(y).R) \psubstp{Q}{P}    
  :=    
  (x)\substp{Q}{P} (z)\concat( (R \psubstn{z}{y}) \psubstp{Q}{P} ) \\
  (\lift{x}{R}) \psubstp{Q}{P}  
  :=
  \lift{(x)\substp{Q}{P}}{ R \psubstp{Q}{P} } \\
%   (\dropn{x})  \psubstp{Q}{P}       
%   := 
%   \left\{ 
%     \begin{array}{ccc} 
%       \dropn{\quotep{Q}} & & x \nameeq \quotep{P} \\
%       \dropn{x} & & otherwise \\
%     \end{array}
%   \right. 
  (\dropn{x})  \psubstp{Q}{P}       
  := 
  \left\{ 
    \begin{array}{ccc} 
      Q & & x \nameeq \quotep{P} \\
      \dropn{x} & & otherwise \\
    \end{array}
  \right.
\end{mathpar}
 

where

\begin{eqnarray}
  (x)\id{\{} \lpquote Q \rpquote / \lpquote P \rpquote \id{\}}            = 
  \left\{ 
    \begin{array}{ccc}
      \lpquote Q \rpquote & & x \nameeq \lpquote P \rpquote \\
      x & & otherwise \\
    \end{array}
  \right. \nonumber
\end{eqnarray}

and $z$ is chosen distinct from $\quotep{P}$, $\quotep{Q}$, the free
names in $Q$, and all the names in $R$. Our $\alpha$-equivalence will
be built in the standard way from this substitution.

\begin{remark}\label{rem:no_self_referential_names}
  One consequence of these definitions is that $\forall P. \quotep{P}
  \not\in \freenames{P}$.
\end{remark}

\subsection{ Dynamic quote: an example }

Anticipating something of what's to come, consider applying the
substitution, $\widehat{\id{\{}u / z \id{\}}}$, to the following pair
of processes, $\lift{w}{y!(z)}$ and $w[ \lpquote y!(z) \rpquote ]$.

\begin{eqnarray}
	\lift{w}{y!(z)}\widehat{\id{\{}u / z \id{\}}}
		& = &
		\lift{w}{y!(u)} \nonumber\\
	w[ \lpquote y!(z) \rpquote ] \widehat{ \id{\{}u / z \id{\}} }
		& = &
		w[ \lpquote y!(z) \rpquote ] \nonumber
\end{eqnarray}

Because the body of the process between quotes is impervious to
substitution, we get radically different answers. In fact, by
examining the first process in an input context,
e.g. $x?(z).\lift{w}{y!(z)}$, we see that the process under the lift
operator may be shaped by prefixed inputs binding a name inside it. In
this sense, the lift operator will be seen as a way to dynamically
construct processes before reifying them as names.

Finally equipped with these standard features we can present the
dynamics of the calculus.

\subsubsection{Operational semantics} 

Finally, we introduce the computational dynamics. What marks these
algebras as distinct from other more traditionally studied algebraic
structures, e.g. vector spaces or polynomial rings, is the manner in
which dynamics is captured. In traditional structures, dynamics is typically
expressed through morphisms between such structures, as in linear maps
between vector spaces or morphisms between rings. In algebras
associated with the semantics of computation, the dynamics is
expressed as part of the algebraic structure itself, through a
reduction reduction relation typically denoted by $\red$. Below, we
give a recursive presentation of this relation for the calculus used
in the encoding.

$\red \subseteq \pi \times \pi$
$\red : \pi \to \mathcal{P}(\pi)$

\begin{mathpar}
  \inferrule* [lab=Comm] { \textsf{match}( x_{src}, x_{trgt} ) } { x_{trgt}?(y)P \; | \; x_{src}!\langle {Q} \rangle \red P\{\quotep{Q}/y}\} }
  \and \\
  \inferrule* [lab=Par] {{P} \red {P}'} {{{P} | {Q}} \red {{P}' | {Q}}}
  \and
  \inferrule* [lab=Equiv]{{{P} \scong {P}'} \andalso {{P}' \red {Q}'} \andalso {{Q}' \scong {Q}}}{{P} \red {Q}}
\end{mathpar}

\begin{eqnarray*}
  match_{\equiv} (\quotep{P},\quotep{Q}) & := & P \equiv Q \\
  match_{\dagger}(\quotep{P},\quotep{Q}) & := & \forall R. P|Q \red^{*} R => R \red^{*} 0 \\
  match_{K}(\quotep{P},\quotep{Q}) & := & K \mbox{ for some context } K
\end{eqnarray*}

$u?(x)P | u!\langle Q \rangle \red P\{\quotep{Q}/x\}$

%We write $\wred$ for $\red^*$, and $P\red$ if $\exists Q $ such that $ P \red Q$.
We write $P\red$ if $\exists Q $ such that $ P \red Q$ and $P\not\red$, otherwise.

\section{Replication}

As mentioned before, it is known that replication (and hence
recursion) can be implemented in a higher-order process algebra
\cite{SangiorgiWalker}. As our first example of calculation with the
machinery thus far presented we give the construction explicitly in
the {\rhoc}.

\begin{eqnarray}
	D_{x} & := & \prefix{x}{y}{(\binpar{\outputp{x}{y}}{@{y}})} \nonumber\\
	\bangp_{x}{P} & := & \binpar{{x}!\langle{\binpar{D_{x}}{P}}\rangle}{D_{x}} \nonumber
\end{eqnarray}

\begin{eqnarray}
	\bangp_{x}{P} & & \nonumber\\
	=
	& {x}!\langle{(\prefix{x}{y}{(\outputp{x}{y} | @{y})) | P}}\rangle 
	      | \prefix{x}{y}{(\outputp{x}{y} | @{y})} & \nonumber\\
	\red
	& (\outputp{x}{y} | @{y})\substn{\quotep{(\prefix{x}{y}{(@{y} | \outputp{x}{y})) | P}}}{y} & \nonumber\\
	=
	& \outputp{x}{\quotep{(\prefix{x}{y}{(\outputp{x}{y} | @{y})) | P}}}
	  | {(\prefix{x}{y}{(\outputp{x}{y} | @{y})) | P}} & \nonumber\\
	\red
	& \ldots & \nonumber\\
	\red^*
	& P | P | \ldots & \nonumber
\end{eqnarray}

Of course, this encoding, as an implementation, runs away, unfolding
$\bangp{P}$ eagerly. A lazier and more implementable replication
operator, restricted to input-guarded processes, may be obtained as follows.

\begin{eqnarray}
\bangp{\prefix{u}{v}{P}} 
	:= 
	\binpar{\lift{x}{\prefix{u}{v}{(\binpar{D(x)}{P})}}}{D(x)} \nonumber
\end{eqnarray}

\begin{remark}
  Note that the lazier definition still does not deal with summation
  or mixed summation (i.e. sums over input and output). The reader is
  invited to construct definitions of replication that deal with these
  features. 

  Further, the definitions are parameterized in a name, $x$. Can you,
  gentle reader, make a definition that eliminates this parameter and
  guarantees no accidental interaction between the replication
  machinery and the process being replicated -- i.e. no accidental
  sharing of names used by the process to get its work done and the
  name(s) used by the replication to effect copying. This latter
  revision of the definition of replication is crucial to obtaining
  the expected identity $!!P \sim !P$.
\end{remark}

\begin{remark}\label{rem:paradoxical_combinator}
  The reader familiar with the lambda calculus will have noticed the
  similarity between $D$ and the paradoxical combinator.

  [Ed. note: the existence of this seems to suggest we have to be more
  restrictive on the set of processes and names we admit if we are to
  support no-cloning.]
\end{remark}

\subsubsection{Bisimulation}

The computational dynamics gives rise to another kind of equivalence,
the equivalence of computational behavior. As previously mentioned
this is typically captured \emph{via} some form of bisimulation.

% The notion we use in this paper is weak barbed bisimulation
% \cite{milner91polyadicpi}.

The notion we use in this paper is derived from weak barbed
bisimulation \cite{milner91polyadicpi}. 

\begin{definition}
An \emph{observation relation}, $\downarrow_{\mathcal N}$, over a set
of names, $\mathcal N$, is the smallest relation satisfying the rules
below.

\infrule[Out-barb]{y \in {\mathcal N}, \; x \nameeq y}
		  {\outputp{x}{v} \downarrow_{\mathcal N} x}
\infrule[Par-barb]{\mbox{$P\downarrow_{\mathcal N} x$ or $Q\downarrow_{\mathcal N} x$}}
		  {\binpar{P}{Q} \downarrow_{\mathcal N} x}

We write $P \Downarrow_{\mathcal N} x$ if there is $Q$ such that 
$P \wred Q$ and $Q \downarrow_{\mathcal N} x$.
\end{definition}

\begin{definition}
%\label{def.bbisim}
An  ${\mathcal N}$-\emph{barbed bisimulation} over a set of names, ${\mathcal N}$, is a symmetric binary relation 
${\mathcal S}_{\mathcal N}$ between agents such that $P\rel{S}_{\mathcal N}Q$ implies:
\begin{enumerate}
\item If $P \red P'$ then $Q \wred Q'$ and $P'\rel{S}_{\mathcal N} Q'$.
\item If $P\downarrow_{\mathcal N} x$, then $Q\Downarrow_{\mathcal N} x$.
\end{enumerate}
$P$ is ${\mathcal N}$-barbed bisimilar to $Q$, written
$P \wbbisim_{\mathcal N} Q$, if $P \rel{S}_{\mathcal N} Q$ for some ${\mathcal N}$-barbed bisimulation ${\mathcal S}_{\mathcal N}$.
\end{definition}

$\mathcal{R} \subseteq \pi \times \pi$

$P \mathcal{R} Q => \forall P'. P \red P' \Rightarrow \exists Q'. Q \red Q', P' \mathcal{R} Q'$

$P \vdash x \Rightarrow Q \vdash x$

\begin{mathpar}
  \inferrule*[lab=Out-barb]{x \nameeq y}{{y}!\langle{Q}\rangle \vdash x}
  \and
  \inferrule*[lab=Par-barb]{\mbox{$P\vdash x$ or $Q\vdash x$}}{\binpar{P}{Q} \vdash x}
\end{mathpar}

\subsubsection{Contexts}

One of the principle advantages of computational calculi like the
$\pi$-calculus is a well-defined notion of context,
contextual-equivalence and a correlation between
contextual-equivalence and notions of bisimulation. The notion of
context allows the decomposition of a process into (sub-)process and
its syntactic environment, its context. Thus, a context may be
thought of as a process with a ``hole'' (written $\Box$) in it. The
application of a context $M$ to a process $P$, written $M[P]$, is
tantamount to filling the hole in $M$ with $P$. In this paper we do
not need the full weight of this theory, but do make use of the notion
of context in the proof the main theorem. 

\begin{mathpar}
  \inferrule* [lab=summation] {} {{M_{M},M_{N}} \bc \Box \;|\; x.M_{A} \;|\; M_{M}+M_{N}}
  \and
  \inferrule* [lab=agent] {} {{M_{A}} \bc (\vec{x})M_{P} \;| \; \clift{P_0,\ldots,M_{P},\ldots,P_N}}
  \and \\
  \inferrule* [lab=process] {} {{M_{P}} \bc M_{N} \;| \;P|M_{P} }
\end{mathpar} 

\begin{mathpar}
  \inferrule* [lab=sychronization] {} {M_{N} \bc \Box \;|\; x?M_{F} \;|\; x!M_{C}}
  \and
  \inferrule* [lab=abstraction] {} {{M_{F}} \bc (x)M_{P} }
  \and
  \inferrule* [lab=concretion] {} {{M_{C}} \bc \langle M_{P} \rangle }
  \and \\
  \inferrule* [lab=process] {} {{M_{P}} \bc M_{N} \;| \;P|M_{P} }
\end{mathpar}

\begin{definition}[contextual application] Given a context $M$, and
  process $P$, we define the \emph{contextual application}, $M[P] :=
  M\{P/\Box\}$. That is, the contextual application of M to P is the
  substitution of $P$ for $\Box$ in $M$.
\end{definition}

$\meaningof{-} : L \to \mathcal{P}(\pi)$

\begin{mathpar}
  \inferrule* [lab=collection] {} {\meaningof{true} = \pi, \and \meaningof{~E} = \pi \setminus \meaningof{E}, \and \meaningof{E_{1} \& E_{2}} = \meaningof{E_{1}} \cap \meaningof{E_{2}}}
\end{mathpar}

\begin{mathpar}
  \inferrule* [lab=structure] {} {\meaningof{0} = \{ P \in \pi | P \equiv 0 \}, \and \\ \meaningof{E_1 | E_2} = \{ P \in \pi | P \equiv P_{1} | P_{2}, P_{1} \in \meaningof{E_{1}}, P_{2} \in \meaningof{E_2}\} }
\end{mathpar}

\begin{mathpar}
 \inferrule* [lab=behavior] {} {\meaningof{\langle a?b \rangle E} = \{ P \in \pi | P \equiv Q | u?(y)P', \\ \and \\\\ \and \\ \;\;\; u \in \meaningof{a}, \forall z.P'\{z/y\} \in \meaningof{E\{z/b\}}\}, \and \\ \meaningof{a!E} = \{ P \in \pi | P \equiv Q | x!\langle P' \rangle, x \in \meaningof{a} P' \in \meaningof{E}\} }
\end{mathpar}

\begin{mathpar}
 \inferrule* [lab=nominal] {} {\meaningof{\quotep{E}} = \{ \quotep{P} \in \quotep{\pi} | P \in \meaningof{E} \}, \and \meaningof{\quotep{P}} = \{ \quotep{Q} \in \quotep{\pi} | P \equiv Q \} \and \\ \meaningof{@\quotep{E}} = \{ P \in \pi | P \equiv @x, x \in \meaningof{E} \}}
\end{mathpar}

\begin{eqnarray*}
  \\
  \meaningof{-} : TS \to ST
\end{eqnarray*}

\begin{eqnarray*}
  \\
  L : TS \to ST
\end{eqnarray*}

\begin{eqnarray*}
  \\
  P \models E \iff P \in \meaningof{E}
\end{eqnarray*}

\begin{eqnarray*}
  P \approx_{L} Q \iff \forall E \in L. P \models E \iff Q \models E
\end{eqnarray*}

\begin{eqnarray*}
  P \approx_{K} Q
\end{eqnarray*}

\begin{eqnarray*}
  P \approx Q
\end{eqnarray*}

$\approx_{K} = \approx = \approx_{L}$

\subsubsection{Contextual duality}

Note that contexts extend the quotation operation to a family of
operations from processes to names. Given a context, $M$, we can
define a \emph{nominal context}, $\quotep{M}$ by $\quotep{M}[P] :=
\quotep{M[P]}$. To foreshadow what is to come we observe that these
operations enjoy a duality with processes very much like the duality
between vectors and maps from vectors to scalars.

Further, because the calculus is essentially higher-order, we have a
correspondence between contexts and processes. More specifically,
given a name $x$ and a context $M$ we can construct $M^{*}_{x}$ such
that 

\begin{mathpar}
  M^{*}_{x} | \lift{x}{P} \red M[P]
\end{mathpar}

namely,

\begin{mathpar}
  M^{*}_{x} := x?(u).M[\dropn{u}]
\end{mathpar}

The dependence of $M^{*}_{x}$ on a name makes it an abstraction, 

\begin{mathpar}
  M^{*} := (x)x?(u).M[\dropn{u}]
\end{mathpar}

\subsection{Additional notation}

It will sometimes be convenient to denote the process a name
quotes. We already have the notation $x = \quotep{P}$, but it will be
convenient to introduce an alternate notation, $\procn{x}$, when we
want to emphasize the connection to the use of the name. Note that, by
virtue of name equivalence, $\quotep{\procn{x}} \nameeq x$; so, the
notation is consistent with previous definitions.

Further, because names have structure it is possible to effect
substitutions on the basis of that structure. This means we need to
upgrade our notation for substitutions, which we accomplish by
adapting comprehension notation. Thus,

\begin{mathpar}
  P\{ y / x : x \in S \}
\end{mathpar}

is interpreted to mean the process derived from P by replacing (in a
capture-avoiding manner) each occurrence of $x$ in $S$ by $y$. For example,

\begin{mathpar}
  P\{ \quotep{\procn{x}|\procn{x}} / x : x \in \freenames{P} \}
\end{mathpar}

will replace each (occurrence) of a free name $x$ in $P$ by
$\quotep{\procn{x}|\procn{x}}$.

Also, we will avail ourselves of the notation $x^{L}$ and $x^{R}$ to
denote injections of a name into disjoint copies of the name
space. There are numerous ways to accomplish this. One example can be
found in \cite{MeredithR05}. This notation overloads to vectors of
names: $\vec{x}^{\pi} := (x_{i}^{\pi} \; : \; 0 \leq i < |\vec{x}| )$ where $\pi \in \{L,R\}$.

We also use $P^{\Box} := P|\Box$.

In \cite{MeredithR05} an interpretation of the new operator is
given. It turns out that there are several possible interpretations
all enjoying the requisite algebraic properties of the operator (see
\cite{milner91polyadicpi}). We will therefore make liberal use of
$(\nu\; \vec{x})P$.

% subsection the_syntax_and_semantics_of_the_notation_system (end)   

\input{qm2pi.qmops} 

\input{qm2pi.sterngerlach} 

\input{qm2pi.metric} 

% section concurrent_process_calculi (end)

%\input{qm2pi.proofsketch}

% section proof sketch (end)

%\input{qm2pi.slviaknots} 

% section spatial logic via knots (end)

\input{qm2pi.conclusion}

% section conclusion (end)

%\input{qm2pi.dtcodes} 

% section wiring algorithm (end)

\input{qm2pi.ack} 

% section acknowledgments (end)

\newpage


\bibliographystyle{plain}   
\bibliography{../../biblios/main.bib}

\input{qm2pi.rhodetails}

\end{document}

 

% section acknowledgments (end)

\newpage


\bibliographystyle{plain}   
\bibliography{../../biblios/main.bib}

\documentclass[12pt]{llncs}
%\documentclass{jktr}

\usepackage[pdftex]{hyperref}                   
\usepackage {listings}
\usepackage {mathpartir}
\usepackage{bcprules}
%\usepackage{listings}
                       
\usepackage{graphicx} 
%\usepackage[margins=2.5cm,nohead,nofoot]{geometry}
%\usepackage{geometry}
\usepackage{amsfonts}
\usepackage{amstext}
\usepackage{latexsym}
\usepackage{amssymb}
\usepackage{color}


%\include{myPreamble}
\include{qm2pi.local} 

%\ifpdf
%\usepackage[pdftex]{graphicx}
%\else
%\usepackage{graphicx}
%\fi

 % \ifpdf
%  \usepackage{pdfsync}
%  \if


%\title{Brief Article}
%\author{David F. Snyder}
%\author{L.G. Meredith}

%\address{Dept. of Math., Texas State University--San Marcos, San Marcos, TX 78666}
       
\pagestyle{empty}


\begin{document}

\lstset{language=[Objective]Caml,frame=shadowbox}

\input{qm2pi.front}

% section front matter (end)

\input{qm2pi.intro} 
 
% section introduction (end)

% \input{qm2pi.knotations} 

% section notation (end)

\input{qm2pi.process.calculi} 

% section concurrent_process_calculi_and_spatial_logics_ (end)
    
%\input{qm2pi.knots2pi} 

%\input{qm2pi.trefoil} 

%\input{qm2pi.mainthm} 

% subsection basic_interpretation (end)

%\input{qm2pi.rho.presentation} 
\subsection{The syntax and semantics of the notation system}\label{sub:the_syntax_and_semantics_of_the_notation_system} % (fold)

We now summarize a technical presentation of the calculus that
embodies our theory of dynamics. The typical presentation of such a
calculus follows the style of giving generators and relations on
them. The grammar, below, describing term constructors, freely
generates the set of processes, $\Proc$. This set is then quotiented
by a relation known as structural congruence and it is over this set
that the notion of dynamics is expressed. This presentation is
essentially that of \cite{MeredithR05} with the addition of
polyadicity and summation. For readability we have relegated some of
the technical subtleties to an appendix.

\subsubsection{Process grammar}\label{subsub:process_grammar}

\begin{mathpar}
  \inferrule* [lab=synchronization] {} {{M} \bc \pzero \;|\; x?F \;|\; x!C }
  \and
  \inferrule* [lab=abstraction] {} {{F} \bc (x)P}
  \and
  \inferrule* [lab=concretion] {} {{C} \bc \langle Q \rangle}
  \and
  \inferrule* [lab=process] {} {{P,Q} \bc M \;| \;P|Q \;|\; @{x}}
  \and
  \inferrule* [lab=name] {} {{x} \bc \quotep{P}}
\end{mathpar} 

Note that $\vec{x}$ (resp. $\vec{P}$) denotes a vector of names
(resp. processes) of length $|\vec{x}|$ (resp. $|\vec{P}|$). We adopt
the following useful abbreviations.

\begin{mathpar}
   x?(\vec{y}).P := x.(\vec{y})P \and  x\clift{\vec{P}} := x.\clift{\vec{P}}
   \and x!(y) := \lift{x}{\dropn{y}}
   \and \Pi_{i=0}^{n-1}P_i := P_0 | \ldots | P_{n-1}
\end{mathpar}

\subsubsection{Structural congruence}

\paragraph{Free and bound names and alpha-equivalence.} At the
core of structural equivalence is alpha-equivalence which identifies
process that are the same up to a change of variable. Formally, we
recognize the distinction between free and bound names. The free names
of a process, $\freenames{P}$, may be calculated recursively as
follows:

\begin{mathpar}
\freenames{\pzero} := \emptyset
  \and \\
  \freenames{x?(y).P} := \{ x \} \cup (\freenames{P} \setminus \{ y \})
  \and 
  \freenames{x!\langle P \rangle} := \{ x \} \cup \{ P \} 
  \and \\
  \freenames{P|Q} := \freenames{P} \cup \freenames{Q}
  \and \\
  \freenames{@{x}} := \{ x \}
\end{mathpar}

$\pi$
$\quotep{\pi}$

$\freenames{-} : \pi \to \mathcal{P}(\quotep{\pi})$

\begin{eqnarray*}
  \freenames{\pzero} & := & \emptyset \\
  \freenames{x?(y).P} & := & \{ x \} \cup (\freenames{P} \setminus \{ y \}) \\
  \freenames{x!\langle P \rangle} & := & \{ x \} \cup \{ P \} \\
  \freenames{P|Q} & := & \freenames{P} \cup \freenames{Q} \\
  \freenames{\dropn{x}} & := & \{ x \}
\end{eqnarray*}

The bound names of a process, $\boundnames{P}$, are those names occurring in $P$
that are not free. For example, in $x?(y).0$, the name $x$ is free, while $y$ is bound.

\begin{mathpar}
  \inferrule* [lab=monoidal-laws] {} { P|Q \equiv Q|P \and P|0 \equiv P \and P|(Q|R) \equiv (P|Q)|R }
\end{mathpar}

\begin{mathpar}
  \inferrule* [lab=alpha-equivalence] {} { (x)P \equiv (y)P\{y/x\} \and y \not\in \freenames{P} }
\end{mathpar}

\begin{definition}
Then two processes, $P,Q$, are alpha-equivalent if $P = Q\{\vec{y}/\vec{x}\}$ for
some $\vec{x} \in \boundnames{Q},\vec{y} \in \boundnames{P}$, where $Q\{\vec{y}/\vec{x}\}$
denotes the capture-avoiding substitution of $\vec{y}$ for $\vec{x}$ in $Q$.
\end{definition}

\begin{definition}
  The {\em structural congruence} \cite{SangiorgiWalker} , $\equiv$,
  between processes is the least congruence containing
  alpha-equivalence, satisfying the abelian monoid laws
  (associativity, commutativity and $\pzero$ as identity) for parallel
  composition $|$ and for summation $+$.
\end{definition}

\subsection{Name equivalence}

We take name equivalence, written $\nameeq$, to be the smallest
equivalence relation generated by the following rules.

\begin{mathpar}
\inferrule*[lab=Quote-drop]
{ }
{ \quotep{@{x}} \nameeq x }

\inferrule*[lab=Struct-equiv]
{ P \scong Q }
{ \quotep{P} \nameeq \quotep{Q} }
\end{mathpar}

The astute reader will have noticed that the mutual recursion of names
and processes imposes a mutual recursion on alpha-equivalence and
structural equivalence via name-equivalence. Fortunately, all of this
works out pleasantly and we may calculate in the natural way, free of
concern. The reader interested in the details is referred to the
appendix \ref{appendix:rho_details}.

\subsection{Substitution}

We use $\Proc$ for the set of processes, $\QProc$ for the set of
names, and $\id{\{}\vec{y} / \vec{x} \id{\}}$ to denote partial maps,
$s : \QProc \rightarrow \QProc$. A map, $s$ lifts, uniquely, to a map
on process terms, $\widehat{s} : \Proc \rightarrow \Proc$ by the
following equations.

\begin{mathpar}
  (0) \psubstp{Q}{P} := 0 \\
  (R \juxtap S) \psubstp{Q}{P}
  :=    
  (R)\psubstp{Q}{P} \juxtap (S) \psubstp{Q}{P} \\
  (x?(y).R) \psubstp{Q}{P}    
  :=    
  (x)\substp{Q}{P} (z)\concat( (R \psubstn{z}{y}) \psubstp{Q}{P} ) \\
  (\lift{x}{R}) \psubstp{Q}{P}  
  :=
  \lift{(x)\substp{Q}{P}}{ R \psubstp{Q}{P} } \\
%   (\dropn{x})  \psubstp{Q}{P}       
%   := 
%   \left\{ 
%     \begin{array}{ccc} 
%       \dropn{\quotep{Q}} & & x \nameeq \quotep{P} \\
%       \dropn{x} & & otherwise \\
%     \end{array}
%   \right. 
  (\dropn{x})  \psubstp{Q}{P}       
  := 
  \left\{ 
    \begin{array}{ccc} 
      Q & & x \nameeq \quotep{P} \\
      \dropn{x} & & otherwise \\
    \end{array}
  \right.
\end{mathpar}
 

where

\begin{eqnarray}
  (x)\id{\{} \lpquote Q \rpquote / \lpquote P \rpquote \id{\}}            = 
  \left\{ 
    \begin{array}{ccc}
      \lpquote Q \rpquote & & x \nameeq \lpquote P \rpquote \\
      x & & otherwise \\
    \end{array}
  \right. \nonumber
\end{eqnarray}

and $z$ is chosen distinct from $\quotep{P}$, $\quotep{Q}$, the free
names in $Q$, and all the names in $R$. Our $\alpha$-equivalence will
be built in the standard way from this substitution.

\begin{remark}\label{rem:no_self_referential_names}
  One consequence of these definitions is that $\forall P. \quotep{P}
  \not\in \freenames{P}$.
\end{remark}

\subsection{ Dynamic quote: an example }

Anticipating something of what's to come, consider applying the
substitution, $\widehat{\id{\{}u / z \id{\}}}$, to the following pair
of processes, $\lift{w}{y!(z)}$ and $w[ \lpquote y!(z) \rpquote ]$.

\begin{eqnarray}
	\lift{w}{y!(z)}\widehat{\id{\{}u / z \id{\}}}
		& = &
		\lift{w}{y!(u)} \nonumber\\
	w[ \lpquote y!(z) \rpquote ] \widehat{ \id{\{}u / z \id{\}} }
		& = &
		w[ \lpquote y!(z) \rpquote ] \nonumber
\end{eqnarray}

Because the body of the process between quotes is impervious to
substitution, we get radically different answers. In fact, by
examining the first process in an input context,
e.g. $x?(z).\lift{w}{y!(z)}$, we see that the process under the lift
operator may be shaped by prefixed inputs binding a name inside it. In
this sense, the lift operator will be seen as a way to dynamically
construct processes before reifying them as names.

Finally equipped with these standard features we can present the
dynamics of the calculus.

\subsubsection{Operational semantics} 

Finally, we introduce the computational dynamics. What marks these
algebras as distinct from other more traditionally studied algebraic
structures, e.g. vector spaces or polynomial rings, is the manner in
which dynamics is captured. In traditional structures, dynamics is typically
expressed through morphisms between such structures, as in linear maps
between vector spaces or morphisms between rings. In algebras
associated with the semantics of computation, the dynamics is
expressed as part of the algebraic structure itself, through a
reduction reduction relation typically denoted by $\red$. Below, we
give a recursive presentation of this relation for the calculus used
in the encoding.

$\red \subseteq \pi \times \pi$
$\red : \pi \to \mathcal{P}(\pi)$

\begin{mathpar}
  \inferrule* [lab=Comm] { \textsf{match}( x_{src}, x_{trgt} ) } { x_{trgt}?(y)P \; | \; x_{src}!\langle {Q} \rangle \red P\{\quotep{Q}/y}\} }
  \and \\
  \inferrule* [lab=Par] {{P} \red {P}'} {{{P} | {Q}} \red {{P}' | {Q}}}
  \and
  \inferrule* [lab=Equiv]{{{P} \scong {P}'} \andalso {{P}' \red {Q}'} \andalso {{Q}' \scong {Q}}}{{P} \red {Q}}
\end{mathpar}

\begin{eqnarray*}
  match_{\equiv} (\quotep{P},\quotep{Q}) & := & P \equiv Q \\
  match_{\dagger}(\quotep{P},\quotep{Q}) & := & \forall R. P|Q \red^{*} R => R \red^{*} 0 \\
  match_{K}(\quotep{P},\quotep{Q}) & := & K \mbox{ for some context } K
\end{eqnarray*}

$u?(x)P | u!\langle Q \rangle \red P\{\quotep{Q}/x\}$

%We write $\wred$ for $\red^*$, and $P\red$ if $\exists Q $ such that $ P \red Q$.
We write $P\red$ if $\exists Q $ such that $ P \red Q$ and $P\not\red$, otherwise.

\section{Replication}

As mentioned before, it is known that replication (and hence
recursion) can be implemented in a higher-order process algebra
\cite{SangiorgiWalker}. As our first example of calculation with the
machinery thus far presented we give the construction explicitly in
the {\rhoc}.

\begin{eqnarray}
	D_{x} & := & \prefix{x}{y}{(\binpar{\outputp{x}{y}}{@{y}})} \nonumber\\
	\bangp_{x}{P} & := & \binpar{{x}!\langle{\binpar{D_{x}}{P}}\rangle}{D_{x}} \nonumber
\end{eqnarray}

\begin{eqnarray}
	\bangp_{x}{P} & & \nonumber\\
	=
	& {x}!\langle{(\prefix{x}{y}{(\outputp{x}{y} | @{y})) | P}}\rangle 
	      | \prefix{x}{y}{(\outputp{x}{y} | @{y})} & \nonumber\\
	\red
	& (\outputp{x}{y} | @{y})\substn{\quotep{(\prefix{x}{y}{(@{y} | \outputp{x}{y})) | P}}}{y} & \nonumber\\
	=
	& \outputp{x}{\quotep{(\prefix{x}{y}{(\outputp{x}{y} | @{y})) | P}}}
	  | {(\prefix{x}{y}{(\outputp{x}{y} | @{y})) | P}} & \nonumber\\
	\red
	& \ldots & \nonumber\\
	\red^*
	& P | P | \ldots & \nonumber
\end{eqnarray}

Of course, this encoding, as an implementation, runs away, unfolding
$\bangp{P}$ eagerly. A lazier and more implementable replication
operator, restricted to input-guarded processes, may be obtained as follows.

\begin{eqnarray}
\bangp{\prefix{u}{v}{P}} 
	:= 
	\binpar{\lift{x}{\prefix{u}{v}{(\binpar{D(x)}{P})}}}{D(x)} \nonumber
\end{eqnarray}

\begin{remark}
  Note that the lazier definition still does not deal with summation
  or mixed summation (i.e. sums over input and output). The reader is
  invited to construct definitions of replication that deal with these
  features. 

  Further, the definitions are parameterized in a name, $x$. Can you,
  gentle reader, make a definition that eliminates this parameter and
  guarantees no accidental interaction between the replication
  machinery and the process being replicated -- i.e. no accidental
  sharing of names used by the process to get its work done and the
  name(s) used by the replication to effect copying. This latter
  revision of the definition of replication is crucial to obtaining
  the expected identity $!!P \sim !P$.
\end{remark}

\begin{remark}\label{rem:paradoxical_combinator}
  The reader familiar with the lambda calculus will have noticed the
  similarity between $D$ and the paradoxical combinator.

  [Ed. note: the existence of this seems to suggest we have to be more
  restrictive on the set of processes and names we admit if we are to
  support no-cloning.]
\end{remark}

\subsubsection{Bisimulation}

The computational dynamics gives rise to another kind of equivalence,
the equivalence of computational behavior. As previously mentioned
this is typically captured \emph{via} some form of bisimulation.

% The notion we use in this paper is weak barbed bisimulation
% \cite{milner91polyadicpi}.

The notion we use in this paper is derived from weak barbed
bisimulation \cite{milner91polyadicpi}. 

\begin{definition}
An \emph{observation relation}, $\downarrow_{\mathcal N}$, over a set
of names, $\mathcal N$, is the smallest relation satisfying the rules
below.

\infrule[Out-barb]{y \in {\mathcal N}, \; x \nameeq y}
		  {\outputp{x}{v} \downarrow_{\mathcal N} x}
\infrule[Par-barb]{\mbox{$P\downarrow_{\mathcal N} x$ or $Q\downarrow_{\mathcal N} x$}}
		  {\binpar{P}{Q} \downarrow_{\mathcal N} x}

We write $P \Downarrow_{\mathcal N} x$ if there is $Q$ such that 
$P \wred Q$ and $Q \downarrow_{\mathcal N} x$.
\end{definition}

\begin{definition}
%\label{def.bbisim}
An  ${\mathcal N}$-\emph{barbed bisimulation} over a set of names, ${\mathcal N}$, is a symmetric binary relation 
${\mathcal S}_{\mathcal N}$ between agents such that $P\rel{S}_{\mathcal N}Q$ implies:
\begin{enumerate}
\item If $P \red P'$ then $Q \wred Q'$ and $P'\rel{S}_{\mathcal N} Q'$.
\item If $P\downarrow_{\mathcal N} x$, then $Q\Downarrow_{\mathcal N} x$.
\end{enumerate}
$P$ is ${\mathcal N}$-barbed bisimilar to $Q$, written
$P \wbbisim_{\mathcal N} Q$, if $P \rel{S}_{\mathcal N} Q$ for some ${\mathcal N}$-barbed bisimulation ${\mathcal S}_{\mathcal N}$.
\end{definition}

$\mathcal{R} \subseteq \pi \times \pi$

$P \mathcal{R} Q => \forall P'. P \red P' \Rightarrow \exists Q'. Q \red Q', P' \mathcal{R} Q'$

$P \vdash x \Rightarrow Q \vdash x$

\begin{mathpar}
  \inferrule*[lab=Out-barb]{x \nameeq y}{{y}!\langle{Q}\rangle \vdash x}
  \and
  \inferrule*[lab=Par-barb]{\mbox{$P\vdash x$ or $Q\vdash x$}}{\binpar{P}{Q} \vdash x}
\end{mathpar}

\subsubsection{Contexts}

One of the principle advantages of computational calculi like the
$\pi$-calculus is a well-defined notion of context,
contextual-equivalence and a correlation between
contextual-equivalence and notions of bisimulation. The notion of
context allows the decomposition of a process into (sub-)process and
its syntactic environment, its context. Thus, a context may be
thought of as a process with a ``hole'' (written $\Box$) in it. The
application of a context $M$ to a process $P$, written $M[P]$, is
tantamount to filling the hole in $M$ with $P$. In this paper we do
not need the full weight of this theory, but do make use of the notion
of context in the proof the main theorem. 

\begin{mathpar}
  \inferrule* [lab=summation] {} {{M_{M},M_{N}} \bc \Box \;|\; x.M_{A} \;|\; M_{M}+M_{N}}
  \and
  \inferrule* [lab=agent] {} {{M_{A}} \bc (\vec{x})M_{P} \;| \; \clift{P_0,\ldots,M_{P},\ldots,P_N}}
  \and \\
  \inferrule* [lab=process] {} {{M_{P}} \bc M_{N} \;| \;P|M_{P} }
\end{mathpar} 

\begin{mathpar}
  \inferrule* [lab=sychronization] {} {M_{N} \bc \Box \;|\; x?M_{F} \;|\; x!M_{C}}
  \and
  \inferrule* [lab=abstraction] {} {{M_{F}} \bc (x)M_{P} }
  \and
  \inferrule* [lab=concretion] {} {{M_{C}} \bc \langle M_{P} \rangle }
  \and \\
  \inferrule* [lab=process] {} {{M_{P}} \bc M_{N} \;| \;P|M_{P} }
\end{mathpar}

\begin{definition}[contextual application] Given a context $M$, and
  process $P$, we define the \emph{contextual application}, $M[P] :=
  M\{P/\Box\}$. That is, the contextual application of M to P is the
  substitution of $P$ for $\Box$ in $M$.
\end{definition}

$\meaningof{-} : L \to \mathcal{P}(\pi)$

\begin{mathpar}
  \inferrule* [lab=collection] {} {\meaningof{true} = \pi, \and \meaningof{~E} = \pi \setminus \meaningof{E}, \and \meaningof{E_{1} \& E_{2}} = \meaningof{E_{1}} \cap \meaningof{E_{2}}}
\end{mathpar}

\begin{mathpar}
  \inferrule* [lab=structure] {} {\meaningof{0} = \{ P \in \pi | P \equiv 0 \}, \and \\ \meaningof{E_1 | E_2} = \{ P \in \pi | P \equiv P_{1} | P_{2}, P_{1} \in \meaningof{E_{1}}, P_{2} \in \meaningof{E_2}\} }
\end{mathpar}

\begin{mathpar}
 \inferrule* [lab=behavior] {} {\meaningof{\langle a?b \rangle E} = \{ P \in \pi | P \equiv Q | u?(y)P', \\ \and \\\\ \and \\ \;\;\; u \in \meaningof{a}, \forall z.P'\{z/y\} \in \meaningof{E\{z/b\}}\}, \and \\ \meaningof{a!E} = \{ P \in \pi | P \equiv Q | x!\langle P' \rangle, x \in \meaningof{a} P' \in \meaningof{E}\} }
\end{mathpar}

\begin{mathpar}
 \inferrule* [lab=nominal] {} {\meaningof{\quotep{E}} = \{ \quotep{P} \in \quotep{\pi} | P \in \meaningof{E} \}, \and \meaningof{\quotep{P}} = \{ \quotep{Q} \in \quotep{\pi} | P \equiv Q \} \and \\ \meaningof{@\quotep{E}} = \{ P \in \pi | P \equiv @x, x \in \meaningof{E} \}}
\end{mathpar}

\begin{eqnarray*}
  \\
  \meaningof{-} : TS \to ST
\end{eqnarray*}

\begin{eqnarray*}
  \\
  L : TS \to ST
\end{eqnarray*}

\begin{eqnarray*}
  \\
  P \models E \iff P \in \meaningof{E}
\end{eqnarray*}

\begin{eqnarray*}
  P \approx_{L} Q \iff \forall E \in L. P \models E \iff Q \models E
\end{eqnarray*}

\begin{eqnarray*}
  P \approx_{K} Q
\end{eqnarray*}

\begin{eqnarray*}
  P \approx Q
\end{eqnarray*}

$\approx_{K} = \approx = \approx_{L}$

\subsubsection{Contextual duality}

Note that contexts extend the quotation operation to a family of
operations from processes to names. Given a context, $M$, we can
define a \emph{nominal context}, $\quotep{M}$ by $\quotep{M}[P] :=
\quotep{M[P]}$. To foreshadow what is to come we observe that these
operations enjoy a duality with processes very much like the duality
between vectors and maps from vectors to scalars.

Further, because the calculus is essentially higher-order, we have a
correspondence between contexts and processes. More specifically,
given a name $x$ and a context $M$ we can construct $M^{*}_{x}$ such
that 

\begin{mathpar}
  M^{*}_{x} | \lift{x}{P} \red M[P]
\end{mathpar}

namely,

\begin{mathpar}
  M^{*}_{x} := x?(u).M[\dropn{u}]
\end{mathpar}

The dependence of $M^{*}_{x}$ on a name makes it an abstraction, 

\begin{mathpar}
  M^{*} := (x)x?(u).M[\dropn{u}]
\end{mathpar}

\subsection{Additional notation}

It will sometimes be convenient to denote the process a name
quotes. We already have the notation $x = \quotep{P}$, but it will be
convenient to introduce an alternate notation, $\procn{x}$, when we
want to emphasize the connection to the use of the name. Note that, by
virtue of name equivalence, $\quotep{\procn{x}} \nameeq x$; so, the
notation is consistent with previous definitions.

Further, because names have structure it is possible to effect
substitutions on the basis of that structure. This means we need to
upgrade our notation for substitutions, which we accomplish by
adapting comprehension notation. Thus,

\begin{mathpar}
  P\{ y / x : x \in S \}
\end{mathpar}

is interpreted to mean the process derived from P by replacing (in a
capture-avoiding manner) each occurrence of $x$ in $S$ by $y$. For example,

\begin{mathpar}
  P\{ \quotep{\procn{x}|\procn{x}} / x : x \in \freenames{P} \}
\end{mathpar}

will replace each (occurrence) of a free name $x$ in $P$ by
$\quotep{\procn{x}|\procn{x}}$.

Also, we will avail ourselves of the notation $x^{L}$ and $x^{R}$ to
denote injections of a name into disjoint copies of the name
space. There are numerous ways to accomplish this. One example can be
found in \cite{MeredithR05}. This notation overloads to vectors of
names: $\vec{x}^{\pi} := (x_{i}^{\pi} \; : \; 0 \leq i < |\vec{x}| )$ where $\pi \in \{L,R\}$.

We also use $P^{\Box} := P|\Box$.

In \cite{MeredithR05} an interpretation of the new operator is
given. It turns out that there are several possible interpretations
all enjoying the requisite algebraic properties of the operator (see
\cite{milner91polyadicpi}). We will therefore make liberal use of
$(\nu\; \vec{x})P$.

% subsection the_syntax_and_semantics_of_the_notation_system (end)   

\input{qm2pi.qmops} 

\input{qm2pi.sterngerlach} 

\input{qm2pi.metric} 

% section concurrent_process_calculi (end)

%\input{qm2pi.proofsketch}

% section proof sketch (end)

%\input{qm2pi.slviaknots} 

% section spatial logic via knots (end)

\input{qm2pi.conclusion}

% section conclusion (end)

%\input{qm2pi.dtcodes} 

% section wiring algorithm (end)

\input{qm2pi.ack} 

% section acknowledgments (end)

\newpage


\bibliographystyle{plain}   
\bibliography{../../biblios/main.bib}

\input{qm2pi.rhodetails}

\end{document}



\end{document}

 

\documentclass[12pt]{llncs}
%\documentclass{jktr}

\usepackage[pdftex]{hyperref}                   
\usepackage {listings}
\usepackage {mathpartir}
\usepackage{bcprules}
%\usepackage{listings}
                       
\usepackage{graphicx} 
%\usepackage[margins=2.5cm,nohead,nofoot]{geometry}
%\usepackage{geometry}
\usepackage{amsfonts}
\usepackage{amstext}
\usepackage{latexsym}
\usepackage{amssymb}
\usepackage{color}


%\include{myPreamble}
\documentclass[12pt]{llncs}
%\documentclass{jktr}

\usepackage[pdftex]{hyperref}                   
\usepackage {listings}
\usepackage {mathpartir}
\usepackage{bcprules}
%\usepackage{listings}
                       
\usepackage{graphicx} 
%\usepackage[margins=2.5cm,nohead,nofoot]{geometry}
%\usepackage{geometry}
\usepackage{amsfonts}
\usepackage{amstext}
\usepackage{latexsym}
\usepackage{amssymb}
\usepackage{color}


%\include{myPreamble}
\include{qm2pi.local} 

%\ifpdf
%\usepackage[pdftex]{graphicx}
%\else
%\usepackage{graphicx}
%\fi

 % \ifpdf
%  \usepackage{pdfsync}
%  \if


%\title{Brief Article}
%\author{David F. Snyder}
%\author{L.G. Meredith}

%\address{Dept. of Math., Texas State University--San Marcos, San Marcos, TX 78666}
       
\pagestyle{empty}


\begin{document}

\lstset{language=[Objective]Caml,frame=shadowbox}

\input{qm2pi.front}

% section front matter (end)

\input{qm2pi.intro} 
 
% section introduction (end)

% \input{qm2pi.knotations} 

% section notation (end)

\input{qm2pi.process.calculi} 

% section concurrent_process_calculi_and_spatial_logics_ (end)
    
%\input{qm2pi.knots2pi} 

%\input{qm2pi.trefoil} 

%\input{qm2pi.mainthm} 

% subsection basic_interpretation (end)

%\input{qm2pi.rho.presentation} 
\subsection{The syntax and semantics of the notation system}\label{sub:the_syntax_and_semantics_of_the_notation_system} % (fold)

We now summarize a technical presentation of the calculus that
embodies our theory of dynamics. The typical presentation of such a
calculus follows the style of giving generators and relations on
them. The grammar, below, describing term constructors, freely
generates the set of processes, $\Proc$. This set is then quotiented
by a relation known as structural congruence and it is over this set
that the notion of dynamics is expressed. This presentation is
essentially that of \cite{MeredithR05} with the addition of
polyadicity and summation. For readability we have relegated some of
the technical subtleties to an appendix.

\subsubsection{Process grammar}\label{subsub:process_grammar}

\begin{mathpar}
  \inferrule* [lab=synchronization] {} {{M} \bc \pzero \;|\; x?F \;|\; x!C }
  \and
  \inferrule* [lab=abstraction] {} {{F} \bc (x)P}
  \and
  \inferrule* [lab=concretion] {} {{C} \bc \langle Q \rangle}
  \and
  \inferrule* [lab=process] {} {{P,Q} \bc M \;| \;P|Q \;|\; @{x}}
  \and
  \inferrule* [lab=name] {} {{x} \bc \quotep{P}}
\end{mathpar} 

Note that $\vec{x}$ (resp. $\vec{P}$) denotes a vector of names
(resp. processes) of length $|\vec{x}|$ (resp. $|\vec{P}|$). We adopt
the following useful abbreviations.

\begin{mathpar}
   x?(\vec{y}).P := x.(\vec{y})P \and  x\clift{\vec{P}} := x.\clift{\vec{P}}
   \and x!(y) := \lift{x}{\dropn{y}}
   \and \Pi_{i=0}^{n-1}P_i := P_0 | \ldots | P_{n-1}
\end{mathpar}

\subsubsection{Structural congruence}

\paragraph{Free and bound names and alpha-equivalence.} At the
core of structural equivalence is alpha-equivalence which identifies
process that are the same up to a change of variable. Formally, we
recognize the distinction between free and bound names. The free names
of a process, $\freenames{P}$, may be calculated recursively as
follows:

\begin{mathpar}
\freenames{\pzero} := \emptyset
  \and \\
  \freenames{x?(y).P} := \{ x \} \cup (\freenames{P} \setminus \{ y \})
  \and 
  \freenames{x!\langle P \rangle} := \{ x \} \cup \{ P \} 
  \and \\
  \freenames{P|Q} := \freenames{P} \cup \freenames{Q}
  \and \\
  \freenames{@{x}} := \{ x \}
\end{mathpar}

$\pi$
$\quotep{\pi}$

$\freenames{-} : \pi \to \mathcal{P}(\quotep{\pi})$

\begin{eqnarray*}
  \freenames{\pzero} & := & \emptyset \\
  \freenames{x?(y).P} & := & \{ x \} \cup (\freenames{P} \setminus \{ y \}) \\
  \freenames{x!\langle P \rangle} & := & \{ x \} \cup \{ P \} \\
  \freenames{P|Q} & := & \freenames{P} \cup \freenames{Q} \\
  \freenames{\dropn{x}} & := & \{ x \}
\end{eqnarray*}

The bound names of a process, $\boundnames{P}$, are those names occurring in $P$
that are not free. For example, in $x?(y).0$, the name $x$ is free, while $y$ is bound.

\begin{mathpar}
  \inferrule* [lab=monoidal-laws] {} { P|Q \equiv Q|P \and P|0 \equiv P \and P|(Q|R) \equiv (P|Q)|R }
\end{mathpar}

\begin{mathpar}
  \inferrule* [lab=alpha-equivalence] {} { (x)P \equiv (y)P\{y/x\} \and y \not\in \freenames{P} }
\end{mathpar}

\begin{definition}
Then two processes, $P,Q$, are alpha-equivalent if $P = Q\{\vec{y}/\vec{x}\}$ for
some $\vec{x} \in \boundnames{Q},\vec{y} \in \boundnames{P}$, where $Q\{\vec{y}/\vec{x}\}$
denotes the capture-avoiding substitution of $\vec{y}$ for $\vec{x}$ in $Q$.
\end{definition}

\begin{definition}
  The {\em structural congruence} \cite{SangiorgiWalker} , $\equiv$,
  between processes is the least congruence containing
  alpha-equivalence, satisfying the abelian monoid laws
  (associativity, commutativity and $\pzero$ as identity) for parallel
  composition $|$ and for summation $+$.
\end{definition}

\subsection{Name equivalence}

We take name equivalence, written $\nameeq$, to be the smallest
equivalence relation generated by the following rules.

\begin{mathpar}
\inferrule*[lab=Quote-drop]
{ }
{ \quotep{@{x}} \nameeq x }

\inferrule*[lab=Struct-equiv]
{ P \scong Q }
{ \quotep{P} \nameeq \quotep{Q} }
\end{mathpar}

The astute reader will have noticed that the mutual recursion of names
and processes imposes a mutual recursion on alpha-equivalence and
structural equivalence via name-equivalence. Fortunately, all of this
works out pleasantly and we may calculate in the natural way, free of
concern. The reader interested in the details is referred to the
appendix \ref{appendix:rho_details}.

\subsection{Substitution}

We use $\Proc$ for the set of processes, $\QProc$ for the set of
names, and $\id{\{}\vec{y} / \vec{x} \id{\}}$ to denote partial maps,
$s : \QProc \rightarrow \QProc$. A map, $s$ lifts, uniquely, to a map
on process terms, $\widehat{s} : \Proc \rightarrow \Proc$ by the
following equations.

\begin{mathpar}
  (0) \psubstp{Q}{P} := 0 \\
  (R \juxtap S) \psubstp{Q}{P}
  :=    
  (R)\psubstp{Q}{P} \juxtap (S) \psubstp{Q}{P} \\
  (x?(y).R) \psubstp{Q}{P}    
  :=    
  (x)\substp{Q}{P} (z)\concat( (R \psubstn{z}{y}) \psubstp{Q}{P} ) \\
  (\lift{x}{R}) \psubstp{Q}{P}  
  :=
  \lift{(x)\substp{Q}{P}}{ R \psubstp{Q}{P} } \\
%   (\dropn{x})  \psubstp{Q}{P}       
%   := 
%   \left\{ 
%     \begin{array}{ccc} 
%       \dropn{\quotep{Q}} & & x \nameeq \quotep{P} \\
%       \dropn{x} & & otherwise \\
%     \end{array}
%   \right. 
  (\dropn{x})  \psubstp{Q}{P}       
  := 
  \left\{ 
    \begin{array}{ccc} 
      Q & & x \nameeq \quotep{P} \\
      \dropn{x} & & otherwise \\
    \end{array}
  \right.
\end{mathpar}
 

where

\begin{eqnarray}
  (x)\id{\{} \lpquote Q \rpquote / \lpquote P \rpquote \id{\}}            = 
  \left\{ 
    \begin{array}{ccc}
      \lpquote Q \rpquote & & x \nameeq \lpquote P \rpquote \\
      x & & otherwise \\
    \end{array}
  \right. \nonumber
\end{eqnarray}

and $z$ is chosen distinct from $\quotep{P}$, $\quotep{Q}$, the free
names in $Q$, and all the names in $R$. Our $\alpha$-equivalence will
be built in the standard way from this substitution.

\begin{remark}\label{rem:no_self_referential_names}
  One consequence of these definitions is that $\forall P. \quotep{P}
  \not\in \freenames{P}$.
\end{remark}

\subsection{ Dynamic quote: an example }

Anticipating something of what's to come, consider applying the
substitution, $\widehat{\id{\{}u / z \id{\}}}$, to the following pair
of processes, $\lift{w}{y!(z)}$ and $w[ \lpquote y!(z) \rpquote ]$.

\begin{eqnarray}
	\lift{w}{y!(z)}\widehat{\id{\{}u / z \id{\}}}
		& = &
		\lift{w}{y!(u)} \nonumber\\
	w[ \lpquote y!(z) \rpquote ] \widehat{ \id{\{}u / z \id{\}} }
		& = &
		w[ \lpquote y!(z) \rpquote ] \nonumber
\end{eqnarray}

Because the body of the process between quotes is impervious to
substitution, we get radically different answers. In fact, by
examining the first process in an input context,
e.g. $x?(z).\lift{w}{y!(z)}$, we see that the process under the lift
operator may be shaped by prefixed inputs binding a name inside it. In
this sense, the lift operator will be seen as a way to dynamically
construct processes before reifying them as names.

Finally equipped with these standard features we can present the
dynamics of the calculus.

\subsubsection{Operational semantics} 

Finally, we introduce the computational dynamics. What marks these
algebras as distinct from other more traditionally studied algebraic
structures, e.g. vector spaces or polynomial rings, is the manner in
which dynamics is captured. In traditional structures, dynamics is typically
expressed through morphisms between such structures, as in linear maps
between vector spaces or morphisms between rings. In algebras
associated with the semantics of computation, the dynamics is
expressed as part of the algebraic structure itself, through a
reduction reduction relation typically denoted by $\red$. Below, we
give a recursive presentation of this relation for the calculus used
in the encoding.

$\red \subseteq \pi \times \pi$
$\red : \pi \to \mathcal{P}(\pi)$

\begin{mathpar}
  \inferrule* [lab=Comm] { \textsf{match}( x_{src}, x_{trgt} ) } { x_{trgt}?(y)P \; | \; x_{src}!\langle {Q} \rangle \red P\{\quotep{Q}/y}\} }
  \and \\
  \inferrule* [lab=Par] {{P} \red {P}'} {{{P} | {Q}} \red {{P}' | {Q}}}
  \and
  \inferrule* [lab=Equiv]{{{P} \scong {P}'} \andalso {{P}' \red {Q}'} \andalso {{Q}' \scong {Q}}}{{P} \red {Q}}
\end{mathpar}

\begin{eqnarray*}
  match_{\equiv} (\quotep{P},\quotep{Q}) & := & P \equiv Q \\
  match_{\dagger}(\quotep{P},\quotep{Q}) & := & \forall R. P|Q \red^{*} R => R \red^{*} 0 \\
  match_{K}(\quotep{P},\quotep{Q}) & := & K \mbox{ for some context } K
\end{eqnarray*}

$u?(x)P | u!\langle Q \rangle \red P\{\quotep{Q}/x\}$

%We write $\wred$ for $\red^*$, and $P\red$ if $\exists Q $ such that $ P \red Q$.
We write $P\red$ if $\exists Q $ such that $ P \red Q$ and $P\not\red$, otherwise.

\section{Replication}

As mentioned before, it is known that replication (and hence
recursion) can be implemented in a higher-order process algebra
\cite{SangiorgiWalker}. As our first example of calculation with the
machinery thus far presented we give the construction explicitly in
the {\rhoc}.

\begin{eqnarray}
	D_{x} & := & \prefix{x}{y}{(\binpar{\outputp{x}{y}}{@{y}})} \nonumber\\
	\bangp_{x}{P} & := & \binpar{{x}!\langle{\binpar{D_{x}}{P}}\rangle}{D_{x}} \nonumber
\end{eqnarray}

\begin{eqnarray}
	\bangp_{x}{P} & & \nonumber\\
	=
	& {x}!\langle{(\prefix{x}{y}{(\outputp{x}{y} | @{y})) | P}}\rangle 
	      | \prefix{x}{y}{(\outputp{x}{y} | @{y})} & \nonumber\\
	\red
	& (\outputp{x}{y} | @{y})\substn{\quotep{(\prefix{x}{y}{(@{y} | \outputp{x}{y})) | P}}}{y} & \nonumber\\
	=
	& \outputp{x}{\quotep{(\prefix{x}{y}{(\outputp{x}{y} | @{y})) | P}}}
	  | {(\prefix{x}{y}{(\outputp{x}{y} | @{y})) | P}} & \nonumber\\
	\red
	& \ldots & \nonumber\\
	\red^*
	& P | P | \ldots & \nonumber
\end{eqnarray}

Of course, this encoding, as an implementation, runs away, unfolding
$\bangp{P}$ eagerly. A lazier and more implementable replication
operator, restricted to input-guarded processes, may be obtained as follows.

\begin{eqnarray}
\bangp{\prefix{u}{v}{P}} 
	:= 
	\binpar{\lift{x}{\prefix{u}{v}{(\binpar{D(x)}{P})}}}{D(x)} \nonumber
\end{eqnarray}

\begin{remark}
  Note that the lazier definition still does not deal with summation
  or mixed summation (i.e. sums over input and output). The reader is
  invited to construct definitions of replication that deal with these
  features. 

  Further, the definitions are parameterized in a name, $x$. Can you,
  gentle reader, make a definition that eliminates this parameter and
  guarantees no accidental interaction between the replication
  machinery and the process being replicated -- i.e. no accidental
  sharing of names used by the process to get its work done and the
  name(s) used by the replication to effect copying. This latter
  revision of the definition of replication is crucial to obtaining
  the expected identity $!!P \sim !P$.
\end{remark}

\begin{remark}\label{rem:paradoxical_combinator}
  The reader familiar with the lambda calculus will have noticed the
  similarity between $D$ and the paradoxical combinator.

  [Ed. note: the existence of this seems to suggest we have to be more
  restrictive on the set of processes and names we admit if we are to
  support no-cloning.]
\end{remark}

\subsubsection{Bisimulation}

The computational dynamics gives rise to another kind of equivalence,
the equivalence of computational behavior. As previously mentioned
this is typically captured \emph{via} some form of bisimulation.

% The notion we use in this paper is weak barbed bisimulation
% \cite{milner91polyadicpi}.

The notion we use in this paper is derived from weak barbed
bisimulation \cite{milner91polyadicpi}. 

\begin{definition}
An \emph{observation relation}, $\downarrow_{\mathcal N}$, over a set
of names, $\mathcal N$, is the smallest relation satisfying the rules
below.

\infrule[Out-barb]{y \in {\mathcal N}, \; x \nameeq y}
		  {\outputp{x}{v} \downarrow_{\mathcal N} x}
\infrule[Par-barb]{\mbox{$P\downarrow_{\mathcal N} x$ or $Q\downarrow_{\mathcal N} x$}}
		  {\binpar{P}{Q} \downarrow_{\mathcal N} x}

We write $P \Downarrow_{\mathcal N} x$ if there is $Q$ such that 
$P \wred Q$ and $Q \downarrow_{\mathcal N} x$.
\end{definition}

\begin{definition}
%\label{def.bbisim}
An  ${\mathcal N}$-\emph{barbed bisimulation} over a set of names, ${\mathcal N}$, is a symmetric binary relation 
${\mathcal S}_{\mathcal N}$ between agents such that $P\rel{S}_{\mathcal N}Q$ implies:
\begin{enumerate}
\item If $P \red P'$ then $Q \wred Q'$ and $P'\rel{S}_{\mathcal N} Q'$.
\item If $P\downarrow_{\mathcal N} x$, then $Q\Downarrow_{\mathcal N} x$.
\end{enumerate}
$P$ is ${\mathcal N}$-barbed bisimilar to $Q$, written
$P \wbbisim_{\mathcal N} Q$, if $P \rel{S}_{\mathcal N} Q$ for some ${\mathcal N}$-barbed bisimulation ${\mathcal S}_{\mathcal N}$.
\end{definition}

$\mathcal{R} \subseteq \pi \times \pi$

$P \mathcal{R} Q => \forall P'. P \red P' \Rightarrow \exists Q'. Q \red Q', P' \mathcal{R} Q'$

$P \vdash x \Rightarrow Q \vdash x$

\begin{mathpar}
  \inferrule*[lab=Out-barb]{x \nameeq y}{{y}!\langle{Q}\rangle \vdash x}
  \and
  \inferrule*[lab=Par-barb]{\mbox{$P\vdash x$ or $Q\vdash x$}}{\binpar{P}{Q} \vdash x}
\end{mathpar}

\subsubsection{Contexts}

One of the principle advantages of computational calculi like the
$\pi$-calculus is a well-defined notion of context,
contextual-equivalence and a correlation between
contextual-equivalence and notions of bisimulation. The notion of
context allows the decomposition of a process into (sub-)process and
its syntactic environment, its context. Thus, a context may be
thought of as a process with a ``hole'' (written $\Box$) in it. The
application of a context $M$ to a process $P$, written $M[P]$, is
tantamount to filling the hole in $M$ with $P$. In this paper we do
not need the full weight of this theory, but do make use of the notion
of context in the proof the main theorem. 

\begin{mathpar}
  \inferrule* [lab=summation] {} {{M_{M},M_{N}} \bc \Box \;|\; x.M_{A} \;|\; M_{M}+M_{N}}
  \and
  \inferrule* [lab=agent] {} {{M_{A}} \bc (\vec{x})M_{P} \;| \; \clift{P_0,\ldots,M_{P},\ldots,P_N}}
  \and \\
  \inferrule* [lab=process] {} {{M_{P}} \bc M_{N} \;| \;P|M_{P} }
\end{mathpar} 

\begin{mathpar}
  \inferrule* [lab=sychronization] {} {M_{N} \bc \Box \;|\; x?M_{F} \;|\; x!M_{C}}
  \and
  \inferrule* [lab=abstraction] {} {{M_{F}} \bc (x)M_{P} }
  \and
  \inferrule* [lab=concretion] {} {{M_{C}} \bc \langle M_{P} \rangle }
  \and \\
  \inferrule* [lab=process] {} {{M_{P}} \bc M_{N} \;| \;P|M_{P} }
\end{mathpar}

\begin{definition}[contextual application] Given a context $M$, and
  process $P$, we define the \emph{contextual application}, $M[P] :=
  M\{P/\Box\}$. That is, the contextual application of M to P is the
  substitution of $P$ for $\Box$ in $M$.
\end{definition}

$\meaningof{-} : L \to \mathcal{P}(\pi)$

\begin{mathpar}
  \inferrule* [lab=collection] {} {\meaningof{true} = \pi, \and \meaningof{~E} = \pi \setminus \meaningof{E}, \and \meaningof{E_{1} \& E_{2}} = \meaningof{E_{1}} \cap \meaningof{E_{2}}}
\end{mathpar}

\begin{mathpar}
  \inferrule* [lab=structure] {} {\meaningof{0} = \{ P \in \pi | P \equiv 0 \}, \and \\ \meaningof{E_1 | E_2} = \{ P \in \pi | P \equiv P_{1} | P_{2}, P_{1} \in \meaningof{E_{1}}, P_{2} \in \meaningof{E_2}\} }
\end{mathpar}

\begin{mathpar}
 \inferrule* [lab=behavior] {} {\meaningof{\langle a?b \rangle E} = \{ P \in \pi | P \equiv Q | u?(y)P', \\ \and \\\\ \and \\ \;\;\; u \in \meaningof{a}, \forall z.P'\{z/y\} \in \meaningof{E\{z/b\}}\}, \and \\ \meaningof{a!E} = \{ P \in \pi | P \equiv Q | x!\langle P' \rangle, x \in \meaningof{a} P' \in \meaningof{E}\} }
\end{mathpar}

\begin{mathpar}
 \inferrule* [lab=nominal] {} {\meaningof{\quotep{E}} = \{ \quotep{P} \in \quotep{\pi} | P \in \meaningof{E} \}, \and \meaningof{\quotep{P}} = \{ \quotep{Q} \in \quotep{\pi} | P \equiv Q \} \and \\ \meaningof{@\quotep{E}} = \{ P \in \pi | P \equiv @x, x \in \meaningof{E} \}}
\end{mathpar}

\begin{eqnarray*}
  \\
  \meaningof{-} : TS \to ST
\end{eqnarray*}

\begin{eqnarray*}
  \\
  L : TS \to ST
\end{eqnarray*}

\begin{eqnarray*}
  \\
  P \models E \iff P \in \meaningof{E}
\end{eqnarray*}

\begin{eqnarray*}
  P \approx_{L} Q \iff \forall E \in L. P \models E \iff Q \models E
\end{eqnarray*}

\begin{eqnarray*}
  P \approx_{K} Q
\end{eqnarray*}

\begin{eqnarray*}
  P \approx Q
\end{eqnarray*}

$\approx_{K} = \approx = \approx_{L}$

\subsubsection{Contextual duality}

Note that contexts extend the quotation operation to a family of
operations from processes to names. Given a context, $M$, we can
define a \emph{nominal context}, $\quotep{M}$ by $\quotep{M}[P] :=
\quotep{M[P]}$. To foreshadow what is to come we observe that these
operations enjoy a duality with processes very much like the duality
between vectors and maps from vectors to scalars.

Further, because the calculus is essentially higher-order, we have a
correspondence between contexts and processes. More specifically,
given a name $x$ and a context $M$ we can construct $M^{*}_{x}$ such
that 

\begin{mathpar}
  M^{*}_{x} | \lift{x}{P} \red M[P]
\end{mathpar}

namely,

\begin{mathpar}
  M^{*}_{x} := x?(u).M[\dropn{u}]
\end{mathpar}

The dependence of $M^{*}_{x}$ on a name makes it an abstraction, 

\begin{mathpar}
  M^{*} := (x)x?(u).M[\dropn{u}]
\end{mathpar}

\subsection{Additional notation}

It will sometimes be convenient to denote the process a name
quotes. We already have the notation $x = \quotep{P}$, but it will be
convenient to introduce an alternate notation, $\procn{x}$, when we
want to emphasize the connection to the use of the name. Note that, by
virtue of name equivalence, $\quotep{\procn{x}} \nameeq x$; so, the
notation is consistent with previous definitions.

Further, because names have structure it is possible to effect
substitutions on the basis of that structure. This means we need to
upgrade our notation for substitutions, which we accomplish by
adapting comprehension notation. Thus,

\begin{mathpar}
  P\{ y / x : x \in S \}
\end{mathpar}

is interpreted to mean the process derived from P by replacing (in a
capture-avoiding manner) each occurrence of $x$ in $S$ by $y$. For example,

\begin{mathpar}
  P\{ \quotep{\procn{x}|\procn{x}} / x : x \in \freenames{P} \}
\end{mathpar}

will replace each (occurrence) of a free name $x$ in $P$ by
$\quotep{\procn{x}|\procn{x}}$.

Also, we will avail ourselves of the notation $x^{L}$ and $x^{R}$ to
denote injections of a name into disjoint copies of the name
space. There are numerous ways to accomplish this. One example can be
found in \cite{MeredithR05}. This notation overloads to vectors of
names: $\vec{x}^{\pi} := (x_{i}^{\pi} \; : \; 0 \leq i < |\vec{x}| )$ where $\pi \in \{L,R\}$.

We also use $P^{\Box} := P|\Box$.

In \cite{MeredithR05} an interpretation of the new operator is
given. It turns out that there are several possible interpretations
all enjoying the requisite algebraic properties of the operator (see
\cite{milner91polyadicpi}). We will therefore make liberal use of
$(\nu\; \vec{x})P$.

% subsection the_syntax_and_semantics_of_the_notation_system (end)   

\input{qm2pi.qmops} 

\input{qm2pi.sterngerlach} 

\input{qm2pi.metric} 

% section concurrent_process_calculi (end)

%\input{qm2pi.proofsketch}

% section proof sketch (end)

%\input{qm2pi.slviaknots} 

% section spatial logic via knots (end)

\input{qm2pi.conclusion}

% section conclusion (end)

%\input{qm2pi.dtcodes} 

% section wiring algorithm (end)

\input{qm2pi.ack} 

% section acknowledgments (end)

\newpage


\bibliographystyle{plain}   
\bibliography{../../biblios/main.bib}

\input{qm2pi.rhodetails}

\end{document}

 

%\ifpdf
%\usepackage[pdftex]{graphicx}
%\else
%\usepackage{graphicx}
%\fi

 % \ifpdf
%  \usepackage{pdfsync}
%  \if


%\title{Brief Article}
%\author{David F. Snyder}
%\author{L.G. Meredith}

%\address{Dept. of Math., Texas State University--San Marcos, San Marcos, TX 78666}
       
\pagestyle{empty}


\begin{document}

\lstset{language=[Objective]Caml,frame=shadowbox}

\documentclass[12pt]{llncs}
%\documentclass{jktr}

\usepackage[pdftex]{hyperref}                   
\usepackage {listings}
\usepackage {mathpartir}
\usepackage{bcprules}
%\usepackage{listings}
                       
\usepackage{graphicx} 
%\usepackage[margins=2.5cm,nohead,nofoot]{geometry}
%\usepackage{geometry}
\usepackage{amsfonts}
\usepackage{amstext}
\usepackage{latexsym}
\usepackage{amssymb}
\usepackage{color}


%\include{myPreamble}
\include{qm2pi.local} 

%\ifpdf
%\usepackage[pdftex]{graphicx}
%\else
%\usepackage{graphicx}
%\fi

 % \ifpdf
%  \usepackage{pdfsync}
%  \if


%\title{Brief Article}
%\author{David F. Snyder}
%\author{L.G. Meredith}

%\address{Dept. of Math., Texas State University--San Marcos, San Marcos, TX 78666}
       
\pagestyle{empty}


\begin{document}

\lstset{language=[Objective]Caml,frame=shadowbox}

\input{qm2pi.front}

% section front matter (end)

\input{qm2pi.intro} 
 
% section introduction (end)

% \input{qm2pi.knotations} 

% section notation (end)

\input{qm2pi.process.calculi} 

% section concurrent_process_calculi_and_spatial_logics_ (end)
    
%\input{qm2pi.knots2pi} 

%\input{qm2pi.trefoil} 

%\input{qm2pi.mainthm} 

% subsection basic_interpretation (end)

%\input{qm2pi.rho.presentation} 
\subsection{The syntax and semantics of the notation system}\label{sub:the_syntax_and_semantics_of_the_notation_system} % (fold)

We now summarize a technical presentation of the calculus that
embodies our theory of dynamics. The typical presentation of such a
calculus follows the style of giving generators and relations on
them. The grammar, below, describing term constructors, freely
generates the set of processes, $\Proc$. This set is then quotiented
by a relation known as structural congruence and it is over this set
that the notion of dynamics is expressed. This presentation is
essentially that of \cite{MeredithR05} with the addition of
polyadicity and summation. For readability we have relegated some of
the technical subtleties to an appendix.

\subsubsection{Process grammar}\label{subsub:process_grammar}

\begin{mathpar}
  \inferrule* [lab=synchronization] {} {{M} \bc \pzero \;|\; x?F \;|\; x!C }
  \and
  \inferrule* [lab=abstraction] {} {{F} \bc (x)P}
  \and
  \inferrule* [lab=concretion] {} {{C} \bc \langle Q \rangle}
  \and
  \inferrule* [lab=process] {} {{P,Q} \bc M \;| \;P|Q \;|\; @{x}}
  \and
  \inferrule* [lab=name] {} {{x} \bc \quotep{P}}
\end{mathpar} 

Note that $\vec{x}$ (resp. $\vec{P}$) denotes a vector of names
(resp. processes) of length $|\vec{x}|$ (resp. $|\vec{P}|$). We adopt
the following useful abbreviations.

\begin{mathpar}
   x?(\vec{y}).P := x.(\vec{y})P \and  x\clift{\vec{P}} := x.\clift{\vec{P}}
   \and x!(y) := \lift{x}{\dropn{y}}
   \and \Pi_{i=0}^{n-1}P_i := P_0 | \ldots | P_{n-1}
\end{mathpar}

\subsubsection{Structural congruence}

\paragraph{Free and bound names and alpha-equivalence.} At the
core of structural equivalence is alpha-equivalence which identifies
process that are the same up to a change of variable. Formally, we
recognize the distinction between free and bound names. The free names
of a process, $\freenames{P}$, may be calculated recursively as
follows:

\begin{mathpar}
\freenames{\pzero} := \emptyset
  \and \\
  \freenames{x?(y).P} := \{ x \} \cup (\freenames{P} \setminus \{ y \})
  \and 
  \freenames{x!\langle P \rangle} := \{ x \} \cup \{ P \} 
  \and \\
  \freenames{P|Q} := \freenames{P} \cup \freenames{Q}
  \and \\
  \freenames{@{x}} := \{ x \}
\end{mathpar}

$\pi$
$\quotep{\pi}$

$\freenames{-} : \pi \to \mathcal{P}(\quotep{\pi})$

\begin{eqnarray*}
  \freenames{\pzero} & := & \emptyset \\
  \freenames{x?(y).P} & := & \{ x \} \cup (\freenames{P} \setminus \{ y \}) \\
  \freenames{x!\langle P \rangle} & := & \{ x \} \cup \{ P \} \\
  \freenames{P|Q} & := & \freenames{P} \cup \freenames{Q} \\
  \freenames{\dropn{x}} & := & \{ x \}
\end{eqnarray*}

The bound names of a process, $\boundnames{P}$, are those names occurring in $P$
that are not free. For example, in $x?(y).0$, the name $x$ is free, while $y$ is bound.

\begin{mathpar}
  \inferrule* [lab=monoidal-laws] {} { P|Q \equiv Q|P \and P|0 \equiv P \and P|(Q|R) \equiv (P|Q)|R }
\end{mathpar}

\begin{mathpar}
  \inferrule* [lab=alpha-equivalence] {} { (x)P \equiv (y)P\{y/x\} \and y \not\in \freenames{P} }
\end{mathpar}

\begin{definition}
Then two processes, $P,Q$, are alpha-equivalent if $P = Q\{\vec{y}/\vec{x}\}$ for
some $\vec{x} \in \boundnames{Q},\vec{y} \in \boundnames{P}$, where $Q\{\vec{y}/\vec{x}\}$
denotes the capture-avoiding substitution of $\vec{y}$ for $\vec{x}$ in $Q$.
\end{definition}

\begin{definition}
  The {\em structural congruence} \cite{SangiorgiWalker} , $\equiv$,
  between processes is the least congruence containing
  alpha-equivalence, satisfying the abelian monoid laws
  (associativity, commutativity and $\pzero$ as identity) for parallel
  composition $|$ and for summation $+$.
\end{definition}

\subsection{Name equivalence}

We take name equivalence, written $\nameeq$, to be the smallest
equivalence relation generated by the following rules.

\begin{mathpar}
\inferrule*[lab=Quote-drop]
{ }
{ \quotep{@{x}} \nameeq x }

\inferrule*[lab=Struct-equiv]
{ P \scong Q }
{ \quotep{P} \nameeq \quotep{Q} }
\end{mathpar}

The astute reader will have noticed that the mutual recursion of names
and processes imposes a mutual recursion on alpha-equivalence and
structural equivalence via name-equivalence. Fortunately, all of this
works out pleasantly and we may calculate in the natural way, free of
concern. The reader interested in the details is referred to the
appendix \ref{appendix:rho_details}.

\subsection{Substitution}

We use $\Proc$ for the set of processes, $\QProc$ for the set of
names, and $\id{\{}\vec{y} / \vec{x} \id{\}}$ to denote partial maps,
$s : \QProc \rightarrow \QProc$. A map, $s$ lifts, uniquely, to a map
on process terms, $\widehat{s} : \Proc \rightarrow \Proc$ by the
following equations.

\begin{mathpar}
  (0) \psubstp{Q}{P} := 0 \\
  (R \juxtap S) \psubstp{Q}{P}
  :=    
  (R)\psubstp{Q}{P} \juxtap (S) \psubstp{Q}{P} \\
  (x?(y).R) \psubstp{Q}{P}    
  :=    
  (x)\substp{Q}{P} (z)\concat( (R \psubstn{z}{y}) \psubstp{Q}{P} ) \\
  (\lift{x}{R}) \psubstp{Q}{P}  
  :=
  \lift{(x)\substp{Q}{P}}{ R \psubstp{Q}{P} } \\
%   (\dropn{x})  \psubstp{Q}{P}       
%   := 
%   \left\{ 
%     \begin{array}{ccc} 
%       \dropn{\quotep{Q}} & & x \nameeq \quotep{P} \\
%       \dropn{x} & & otherwise \\
%     \end{array}
%   \right. 
  (\dropn{x})  \psubstp{Q}{P}       
  := 
  \left\{ 
    \begin{array}{ccc} 
      Q & & x \nameeq \quotep{P} \\
      \dropn{x} & & otherwise \\
    \end{array}
  \right.
\end{mathpar}
 

where

\begin{eqnarray}
  (x)\id{\{} \lpquote Q \rpquote / \lpquote P \rpquote \id{\}}            = 
  \left\{ 
    \begin{array}{ccc}
      \lpquote Q \rpquote & & x \nameeq \lpquote P \rpquote \\
      x & & otherwise \\
    \end{array}
  \right. \nonumber
\end{eqnarray}

and $z$ is chosen distinct from $\quotep{P}$, $\quotep{Q}$, the free
names in $Q$, and all the names in $R$. Our $\alpha$-equivalence will
be built in the standard way from this substitution.

\begin{remark}\label{rem:no_self_referential_names}
  One consequence of these definitions is that $\forall P. \quotep{P}
  \not\in \freenames{P}$.
\end{remark}

\subsection{ Dynamic quote: an example }

Anticipating something of what's to come, consider applying the
substitution, $\widehat{\id{\{}u / z \id{\}}}$, to the following pair
of processes, $\lift{w}{y!(z)}$ and $w[ \lpquote y!(z) \rpquote ]$.

\begin{eqnarray}
	\lift{w}{y!(z)}\widehat{\id{\{}u / z \id{\}}}
		& = &
		\lift{w}{y!(u)} \nonumber\\
	w[ \lpquote y!(z) \rpquote ] \widehat{ \id{\{}u / z \id{\}} }
		& = &
		w[ \lpquote y!(z) \rpquote ] \nonumber
\end{eqnarray}

Because the body of the process between quotes is impervious to
substitution, we get radically different answers. In fact, by
examining the first process in an input context,
e.g. $x?(z).\lift{w}{y!(z)}$, we see that the process under the lift
operator may be shaped by prefixed inputs binding a name inside it. In
this sense, the lift operator will be seen as a way to dynamically
construct processes before reifying them as names.

Finally equipped with these standard features we can present the
dynamics of the calculus.

\subsubsection{Operational semantics} 

Finally, we introduce the computational dynamics. What marks these
algebras as distinct from other more traditionally studied algebraic
structures, e.g. vector spaces or polynomial rings, is the manner in
which dynamics is captured. In traditional structures, dynamics is typically
expressed through morphisms between such structures, as in linear maps
between vector spaces or morphisms between rings. In algebras
associated with the semantics of computation, the dynamics is
expressed as part of the algebraic structure itself, through a
reduction reduction relation typically denoted by $\red$. Below, we
give a recursive presentation of this relation for the calculus used
in the encoding.

$\red \subseteq \pi \times \pi$
$\red : \pi \to \mathcal{P}(\pi)$

\begin{mathpar}
  \inferrule* [lab=Comm] { \textsf{match}( x_{src}, x_{trgt} ) } { x_{trgt}?(y)P \; | \; x_{src}!\langle {Q} \rangle \red P\{\quotep{Q}/y}\} }
  \and \\
  \inferrule* [lab=Par] {{P} \red {P}'} {{{P} | {Q}} \red {{P}' | {Q}}}
  \and
  \inferrule* [lab=Equiv]{{{P} \scong {P}'} \andalso {{P}' \red {Q}'} \andalso {{Q}' \scong {Q}}}{{P} \red {Q}}
\end{mathpar}

\begin{eqnarray*}
  match_{\equiv} (\quotep{P},\quotep{Q}) & := & P \equiv Q \\
  match_{\dagger}(\quotep{P},\quotep{Q}) & := & \forall R. P|Q \red^{*} R => R \red^{*} 0 \\
  match_{K}(\quotep{P},\quotep{Q}) & := & K \mbox{ for some context } K
\end{eqnarray*}

$u?(x)P | u!\langle Q \rangle \red P\{\quotep{Q}/x\}$

%We write $\wred$ for $\red^*$, and $P\red$ if $\exists Q $ such that $ P \red Q$.
We write $P\red$ if $\exists Q $ such that $ P \red Q$ and $P\not\red$, otherwise.

\section{Replication}

As mentioned before, it is known that replication (and hence
recursion) can be implemented in a higher-order process algebra
\cite{SangiorgiWalker}. As our first example of calculation with the
machinery thus far presented we give the construction explicitly in
the {\rhoc}.

\begin{eqnarray}
	D_{x} & := & \prefix{x}{y}{(\binpar{\outputp{x}{y}}{@{y}})} \nonumber\\
	\bangp_{x}{P} & := & \binpar{{x}!\langle{\binpar{D_{x}}{P}}\rangle}{D_{x}} \nonumber
\end{eqnarray}

\begin{eqnarray}
	\bangp_{x}{P} & & \nonumber\\
	=
	& {x}!\langle{(\prefix{x}{y}{(\outputp{x}{y} | @{y})) | P}}\rangle 
	      | \prefix{x}{y}{(\outputp{x}{y} | @{y})} & \nonumber\\
	\red
	& (\outputp{x}{y} | @{y})\substn{\quotep{(\prefix{x}{y}{(@{y} | \outputp{x}{y})) | P}}}{y} & \nonumber\\
	=
	& \outputp{x}{\quotep{(\prefix{x}{y}{(\outputp{x}{y} | @{y})) | P}}}
	  | {(\prefix{x}{y}{(\outputp{x}{y} | @{y})) | P}} & \nonumber\\
	\red
	& \ldots & \nonumber\\
	\red^*
	& P | P | \ldots & \nonumber
\end{eqnarray}

Of course, this encoding, as an implementation, runs away, unfolding
$\bangp{P}$ eagerly. A lazier and more implementable replication
operator, restricted to input-guarded processes, may be obtained as follows.

\begin{eqnarray}
\bangp{\prefix{u}{v}{P}} 
	:= 
	\binpar{\lift{x}{\prefix{u}{v}{(\binpar{D(x)}{P})}}}{D(x)} \nonumber
\end{eqnarray}

\begin{remark}
  Note that the lazier definition still does not deal with summation
  or mixed summation (i.e. sums over input and output). The reader is
  invited to construct definitions of replication that deal with these
  features. 

  Further, the definitions are parameterized in a name, $x$. Can you,
  gentle reader, make a definition that eliminates this parameter and
  guarantees no accidental interaction between the replication
  machinery and the process being replicated -- i.e. no accidental
  sharing of names used by the process to get its work done and the
  name(s) used by the replication to effect copying. This latter
  revision of the definition of replication is crucial to obtaining
  the expected identity $!!P \sim !P$.
\end{remark}

\begin{remark}\label{rem:paradoxical_combinator}
  The reader familiar with the lambda calculus will have noticed the
  similarity between $D$ and the paradoxical combinator.

  [Ed. note: the existence of this seems to suggest we have to be more
  restrictive on the set of processes and names we admit if we are to
  support no-cloning.]
\end{remark}

\subsubsection{Bisimulation}

The computational dynamics gives rise to another kind of equivalence,
the equivalence of computational behavior. As previously mentioned
this is typically captured \emph{via} some form of bisimulation.

% The notion we use in this paper is weak barbed bisimulation
% \cite{milner91polyadicpi}.

The notion we use in this paper is derived from weak barbed
bisimulation \cite{milner91polyadicpi}. 

\begin{definition}
An \emph{observation relation}, $\downarrow_{\mathcal N}$, over a set
of names, $\mathcal N$, is the smallest relation satisfying the rules
below.

\infrule[Out-barb]{y \in {\mathcal N}, \; x \nameeq y}
		  {\outputp{x}{v} \downarrow_{\mathcal N} x}
\infrule[Par-barb]{\mbox{$P\downarrow_{\mathcal N} x$ or $Q\downarrow_{\mathcal N} x$}}
		  {\binpar{P}{Q} \downarrow_{\mathcal N} x}

We write $P \Downarrow_{\mathcal N} x$ if there is $Q$ such that 
$P \wred Q$ and $Q \downarrow_{\mathcal N} x$.
\end{definition}

\begin{definition}
%\label{def.bbisim}
An  ${\mathcal N}$-\emph{barbed bisimulation} over a set of names, ${\mathcal N}$, is a symmetric binary relation 
${\mathcal S}_{\mathcal N}$ between agents such that $P\rel{S}_{\mathcal N}Q$ implies:
\begin{enumerate}
\item If $P \red P'$ then $Q \wred Q'$ and $P'\rel{S}_{\mathcal N} Q'$.
\item If $P\downarrow_{\mathcal N} x$, then $Q\Downarrow_{\mathcal N} x$.
\end{enumerate}
$P$ is ${\mathcal N}$-barbed bisimilar to $Q$, written
$P \wbbisim_{\mathcal N} Q$, if $P \rel{S}_{\mathcal N} Q$ for some ${\mathcal N}$-barbed bisimulation ${\mathcal S}_{\mathcal N}$.
\end{definition}

$\mathcal{R} \subseteq \pi \times \pi$

$P \mathcal{R} Q => \forall P'. P \red P' \Rightarrow \exists Q'. Q \red Q', P' \mathcal{R} Q'$

$P \vdash x \Rightarrow Q \vdash x$

\begin{mathpar}
  \inferrule*[lab=Out-barb]{x \nameeq y}{{y}!\langle{Q}\rangle \vdash x}
  \and
  \inferrule*[lab=Par-barb]{\mbox{$P\vdash x$ or $Q\vdash x$}}{\binpar{P}{Q} \vdash x}
\end{mathpar}

\subsubsection{Contexts}

One of the principle advantages of computational calculi like the
$\pi$-calculus is a well-defined notion of context,
contextual-equivalence and a correlation between
contextual-equivalence and notions of bisimulation. The notion of
context allows the decomposition of a process into (sub-)process and
its syntactic environment, its context. Thus, a context may be
thought of as a process with a ``hole'' (written $\Box$) in it. The
application of a context $M$ to a process $P$, written $M[P]$, is
tantamount to filling the hole in $M$ with $P$. In this paper we do
not need the full weight of this theory, but do make use of the notion
of context in the proof the main theorem. 

\begin{mathpar}
  \inferrule* [lab=summation] {} {{M_{M},M_{N}} \bc \Box \;|\; x.M_{A} \;|\; M_{M}+M_{N}}
  \and
  \inferrule* [lab=agent] {} {{M_{A}} \bc (\vec{x})M_{P} \;| \; \clift{P_0,\ldots,M_{P},\ldots,P_N}}
  \and \\
  \inferrule* [lab=process] {} {{M_{P}} \bc M_{N} \;| \;P|M_{P} }
\end{mathpar} 

\begin{mathpar}
  \inferrule* [lab=sychronization] {} {M_{N} \bc \Box \;|\; x?M_{F} \;|\; x!M_{C}}
  \and
  \inferrule* [lab=abstraction] {} {{M_{F}} \bc (x)M_{P} }
  \and
  \inferrule* [lab=concretion] {} {{M_{C}} \bc \langle M_{P} \rangle }
  \and \\
  \inferrule* [lab=process] {} {{M_{P}} \bc M_{N} \;| \;P|M_{P} }
\end{mathpar}

\begin{definition}[contextual application] Given a context $M$, and
  process $P$, we define the \emph{contextual application}, $M[P] :=
  M\{P/\Box\}$. That is, the contextual application of M to P is the
  substitution of $P$ for $\Box$ in $M$.
\end{definition}

$\meaningof{-} : L \to \mathcal{P}(\pi)$

\begin{mathpar}
  \inferrule* [lab=collection] {} {\meaningof{true} = \pi, \and \meaningof{~E} = \pi \setminus \meaningof{E}, \and \meaningof{E_{1} \& E_{2}} = \meaningof{E_{1}} \cap \meaningof{E_{2}}}
\end{mathpar}

\begin{mathpar}
  \inferrule* [lab=structure] {} {\meaningof{0} = \{ P \in \pi | P \equiv 0 \}, \and \\ \meaningof{E_1 | E_2} = \{ P \in \pi | P \equiv P_{1} | P_{2}, P_{1} \in \meaningof{E_{1}}, P_{2} \in \meaningof{E_2}\} }
\end{mathpar}

\begin{mathpar}
 \inferrule* [lab=behavior] {} {\meaningof{\langle a?b \rangle E} = \{ P \in \pi | P \equiv Q | u?(y)P', \\ \and \\\\ \and \\ \;\;\; u \in \meaningof{a}, \forall z.P'\{z/y\} \in \meaningof{E\{z/b\}}\}, \and \\ \meaningof{a!E} = \{ P \in \pi | P \equiv Q | x!\langle P' \rangle, x \in \meaningof{a} P' \in \meaningof{E}\} }
\end{mathpar}

\begin{mathpar}
 \inferrule* [lab=nominal] {} {\meaningof{\quotep{E}} = \{ \quotep{P} \in \quotep{\pi} | P \in \meaningof{E} \}, \and \meaningof{\quotep{P}} = \{ \quotep{Q} \in \quotep{\pi} | P \equiv Q \} \and \\ \meaningof{@\quotep{E}} = \{ P \in \pi | P \equiv @x, x \in \meaningof{E} \}}
\end{mathpar}

\begin{eqnarray*}
  \\
  \meaningof{-} : TS \to ST
\end{eqnarray*}

\begin{eqnarray*}
  \\
  L : TS \to ST
\end{eqnarray*}

\begin{eqnarray*}
  \\
  P \models E \iff P \in \meaningof{E}
\end{eqnarray*}

\begin{eqnarray*}
  P \approx_{L} Q \iff \forall E \in L. P \models E \iff Q \models E
\end{eqnarray*}

\begin{eqnarray*}
  P \approx_{K} Q
\end{eqnarray*}

\begin{eqnarray*}
  P \approx Q
\end{eqnarray*}

$\approx_{K} = \approx = \approx_{L}$

\subsubsection{Contextual duality}

Note that contexts extend the quotation operation to a family of
operations from processes to names. Given a context, $M$, we can
define a \emph{nominal context}, $\quotep{M}$ by $\quotep{M}[P] :=
\quotep{M[P]}$. To foreshadow what is to come we observe that these
operations enjoy a duality with processes very much like the duality
between vectors and maps from vectors to scalars.

Further, because the calculus is essentially higher-order, we have a
correspondence between contexts and processes. More specifically,
given a name $x$ and a context $M$ we can construct $M^{*}_{x}$ such
that 

\begin{mathpar}
  M^{*}_{x} | \lift{x}{P} \red M[P]
\end{mathpar}

namely,

\begin{mathpar}
  M^{*}_{x} := x?(u).M[\dropn{u}]
\end{mathpar}

The dependence of $M^{*}_{x}$ on a name makes it an abstraction, 

\begin{mathpar}
  M^{*} := (x)x?(u).M[\dropn{u}]
\end{mathpar}

\subsection{Additional notation}

It will sometimes be convenient to denote the process a name
quotes. We already have the notation $x = \quotep{P}$, but it will be
convenient to introduce an alternate notation, $\procn{x}$, when we
want to emphasize the connection to the use of the name. Note that, by
virtue of name equivalence, $\quotep{\procn{x}} \nameeq x$; so, the
notation is consistent with previous definitions.

Further, because names have structure it is possible to effect
substitutions on the basis of that structure. This means we need to
upgrade our notation for substitutions, which we accomplish by
adapting comprehension notation. Thus,

\begin{mathpar}
  P\{ y / x : x \in S \}
\end{mathpar}

is interpreted to mean the process derived from P by replacing (in a
capture-avoiding manner) each occurrence of $x$ in $S$ by $y$. For example,

\begin{mathpar}
  P\{ \quotep{\procn{x}|\procn{x}} / x : x \in \freenames{P} \}
\end{mathpar}

will replace each (occurrence) of a free name $x$ in $P$ by
$\quotep{\procn{x}|\procn{x}}$.

Also, we will avail ourselves of the notation $x^{L}$ and $x^{R}$ to
denote injections of a name into disjoint copies of the name
space. There are numerous ways to accomplish this. One example can be
found in \cite{MeredithR05}. This notation overloads to vectors of
names: $\vec{x}^{\pi} := (x_{i}^{\pi} \; : \; 0 \leq i < |\vec{x}| )$ where $\pi \in \{L,R\}$.

We also use $P^{\Box} := P|\Box$.

In \cite{MeredithR05} an interpretation of the new operator is
given. It turns out that there are several possible interpretations
all enjoying the requisite algebraic properties of the operator (see
\cite{milner91polyadicpi}). We will therefore make liberal use of
$(\nu\; \vec{x})P$.

% subsection the_syntax_and_semantics_of_the_notation_system (end)   

\input{qm2pi.qmops} 

\input{qm2pi.sterngerlach} 

\input{qm2pi.metric} 

% section concurrent_process_calculi (end)

%\input{qm2pi.proofsketch}

% section proof sketch (end)

%\input{qm2pi.slviaknots} 

% section spatial logic via knots (end)

\input{qm2pi.conclusion}

% section conclusion (end)

%\input{qm2pi.dtcodes} 

% section wiring algorithm (end)

\input{qm2pi.ack} 

% section acknowledgments (end)

\newpage


\bibliographystyle{plain}   
\bibliography{../../biblios/main.bib}

\input{qm2pi.rhodetails}

\end{document}



% section front matter (end)

\section{Introduction}\label{sec:introduction} % (fold)
In this draft of the material i am going to have to dispense with the
usual writing conventions adopted in papers on these topics. i'm going
to have adopt whatever tone i need at the time i'm writing up the
calculations. Sometimes this may be very conversational; others it may
be the barest mathematical grunts; others still it may be that i have
lifted text from one of my other papers because the exposition of some
point was better said there. i hope that my readers are not unduly put
out by this decision. i'm not doing this to flout convention or be
rebellious. i find these calculations very technically challenging. To
keep everything going technically, something has to give; i have to
let go of some cognitive burden. So, the academic writing style --
with all of its trade-offs in terms of facilitating technical
communication -- is what i'm letting go of. Perhaps subsequent drafts
can be tightened and polished, but for now, i'm going to speak as if
we were sitting together in a coffee shop with a laptop, wifi and a
pad of paper and a pencil.

So, here's what i have to say. We -- you and i, comfortably ensconced
in our coffee shop and well-equipped with our tools -- can realize and
carry out the calculations of quantum mechanics over a very different
formal theory of dynamics, a formal theory of dynamics that
corresponds to a theory of concurrent computation with
\emph{reflection}. It has the advantage that the underlying theory is
already `quantized', but supports analogues all of the continuuous
operations. Strikingly, this underlying theory has recently been
connected with a notion of metric that we can show, by calculating
together, coincides with the metric induced by the inner product.

There are a lot of reasons why you might be interested in seeing
calculations of this form. Here's why i'm interested. For the past
several centuries there has been no competitor to the ``Newtonian''
account of dynamics. As a result the predominant share of accounts of
dynamical systems and situations have had to be formulated in terms of
the Newtonian machinery. i view this as an intellectually dangerous
position to occupy. Everything, despite it's intrinsic shape, turns
into a nail to be hit with this hammer. Recently, however, the theory
of computation has matured to the point where we have candidates for
theories of dynamics that offer very different perspective on
reasoning about dynamical systems and situations. Testing these
candidates against very successful accounts of dynamical situations,
like quantum mechanics, is going to give us some sense of how mature
they are and some measure of the quality of these accounts of
dynamics.

\subsection{Summary of contributions and outline of paper}

So, we're going to develop an interpretation of the operations of
quantum mechanics normally interpreted by Hilbert spaces and
operators. We're going to do this over a theory of computation. Note
that this is very different than the usual quantum computation program
which develops notions of computation over quantum mechanics. Rather,
we are developing a story that aligns with Wheeler's slogan: It from
Bit. To do this we will first provide an account of the theory of
computation at play here. Then we will dive into a calculation-driven
interpretation of the operations of quantum mechanics.

The reason we take this approach is that -- until very recently --
there hasn't been an axiomatic account of quantum mechanics. As a
result there has been no sharp delineation of the mathematical theory
supporting interpretation of the physical theory and the physical
theory, itself. So, ambient features of the maths are free to be
exploited (or supressed) without a real accounting of their physical
relevance. There is no sharp statement ``here's the physical theory''
qua \emph{theory} and ``here's the mathematical interpretation''
enabling a judgment of how faithful the interpretation is -- apart
from experimental observation. When there is an axiomatic account we
can judge how well a given mathematical formalism supports an
interpretation of the axioms, independent of
experimentation. Likewise, we can judge how well we have captured our
physical evidence and experience with our axiomatics, independent of
any specific mathematical implementation, with accidental detail that
may or may not have physical significance. 

In lieu of a fully fleshed out and vetted axiomatic account of quantum
mechanics, interpreting the operational notions in service of modeling
physical systems will have to suffice. In other words, we are not in
the business of providing a model of Hilbert spaces and operators. We
are in the business of providing a model of quantum mechanics because
we are motivated by testing our notions of dynamics against physical
theory; and, the predictive calculations of the physical theory must
serve as the best formulation -- shy of a fully fleshed out axiomatic
account -- of the physical theory itself (as they have for scientific
theories since time immemorial). Put another way, despite a
whole-hearted commitment to an It-from-Bit ontology, we are firmly
aligned with the shut-up-and-calculate camp as the best way to obtain
results either from the physical perspective or as a quality assurance
measure of our fledgling theory of dynamics.

In detail, we present a reflective process calculus. Then we develop
intuitive correspondences between the notions available in this
calculus and the usual physical notions supporting quantum mechanical
calculations. Thus, 

\begin{table}[htp]
  \center{
    \fbox{
      \begin{tabular}{c|c}
        quantum mechanics & process calculus \\
        \hline
        scalar & name \\
        state vector & process \\
        dual & contextual duals \\
        matrix & formal sums of process-context-dual pairs \\
        orthogonality & process annihilation \\
        inner product & execution-formula + quoting
      \end{tabular}
    }
  }
  \caption{QM - process calculi correspondences}
\end{table}

Then we tighten up these intuitions to operational definitions. We
employ the Dirac notation as the best proxy we can find for an
abstract syntax of the quantum mechanical notions. The definitions we
develop put us in contact with equational constraints coming from the
theory that we demonstrate the definitions and calculations satisfy.

This puts us in a position to shut up and calculate for the
Stern-Gerlach experimental set up, showing how these predictive
calculations become calculations on processes in our theory of a
reflective process calculus.

Penultimately, we demonstrate that the notion of metric coming from
the inner product coincides with the notion of metric available from
the theory of bisimulation. This demonstration gives us the right to
think of space as arising from behavior. Finally, we consider where we
might go from the new vantage point we have obtained.

% section introduction (end) 
 
% section introduction (end)

% \documentclass[12pt]{llncs}
%\documentclass{jktr}

\usepackage[pdftex]{hyperref}                   
\usepackage {listings}
\usepackage {mathpartir}
\usepackage{bcprules}
%\usepackage{listings}
                       
\usepackage{graphicx} 
%\usepackage[margins=2.5cm,nohead,nofoot]{geometry}
%\usepackage{geometry}
\usepackage{amsfonts}
\usepackage{amstext}
\usepackage{latexsym}
\usepackage{amssymb}
\usepackage{color}


%\include{myPreamble}
\include{qm2pi.local} 

%\ifpdf
%\usepackage[pdftex]{graphicx}
%\else
%\usepackage{graphicx}
%\fi

 % \ifpdf
%  \usepackage{pdfsync}
%  \if


%\title{Brief Article}
%\author{David F. Snyder}
%\author{L.G. Meredith}

%\address{Dept. of Math., Texas State University--San Marcos, San Marcos, TX 78666}
       
\pagestyle{empty}


\begin{document}

\lstset{language=[Objective]Caml,frame=shadowbox}

\input{qm2pi.front}

% section front matter (end)

\input{qm2pi.intro} 
 
% section introduction (end)

% \input{qm2pi.knotations} 

% section notation (end)

\input{qm2pi.process.calculi} 

% section concurrent_process_calculi_and_spatial_logics_ (end)
    
%\input{qm2pi.knots2pi} 

%\input{qm2pi.trefoil} 

%\input{qm2pi.mainthm} 

% subsection basic_interpretation (end)

%\input{qm2pi.rho.presentation} 
\subsection{The syntax and semantics of the notation system}\label{sub:the_syntax_and_semantics_of_the_notation_system} % (fold)

We now summarize a technical presentation of the calculus that
embodies our theory of dynamics. The typical presentation of such a
calculus follows the style of giving generators and relations on
them. The grammar, below, describing term constructors, freely
generates the set of processes, $\Proc$. This set is then quotiented
by a relation known as structural congruence and it is over this set
that the notion of dynamics is expressed. This presentation is
essentially that of \cite{MeredithR05} with the addition of
polyadicity and summation. For readability we have relegated some of
the technical subtleties to an appendix.

\subsubsection{Process grammar}\label{subsub:process_grammar}

\begin{mathpar}
  \inferrule* [lab=synchronization] {} {{M} \bc \pzero \;|\; x?F \;|\; x!C }
  \and
  \inferrule* [lab=abstraction] {} {{F} \bc (x)P}
  \and
  \inferrule* [lab=concretion] {} {{C} \bc \langle Q \rangle}
  \and
  \inferrule* [lab=process] {} {{P,Q} \bc M \;| \;P|Q \;|\; @{x}}
  \and
  \inferrule* [lab=name] {} {{x} \bc \quotep{P}}
\end{mathpar} 

Note that $\vec{x}$ (resp. $\vec{P}$) denotes a vector of names
(resp. processes) of length $|\vec{x}|$ (resp. $|\vec{P}|$). We adopt
the following useful abbreviations.

\begin{mathpar}
   x?(\vec{y}).P := x.(\vec{y})P \and  x\clift{\vec{P}} := x.\clift{\vec{P}}
   \and x!(y) := \lift{x}{\dropn{y}}
   \and \Pi_{i=0}^{n-1}P_i := P_0 | \ldots | P_{n-1}
\end{mathpar}

\subsubsection{Structural congruence}

\paragraph{Free and bound names and alpha-equivalence.} At the
core of structural equivalence is alpha-equivalence which identifies
process that are the same up to a change of variable. Formally, we
recognize the distinction between free and bound names. The free names
of a process, $\freenames{P}$, may be calculated recursively as
follows:

\begin{mathpar}
\freenames{\pzero} := \emptyset
  \and \\
  \freenames{x?(y).P} := \{ x \} \cup (\freenames{P} \setminus \{ y \})
  \and 
  \freenames{x!\langle P \rangle} := \{ x \} \cup \{ P \} 
  \and \\
  \freenames{P|Q} := \freenames{P} \cup \freenames{Q}
  \and \\
  \freenames{@{x}} := \{ x \}
\end{mathpar}

$\pi$
$\quotep{\pi}$

$\freenames{-} : \pi \to \mathcal{P}(\quotep{\pi})$

\begin{eqnarray*}
  \freenames{\pzero} & := & \emptyset \\
  \freenames{x?(y).P} & := & \{ x \} \cup (\freenames{P} \setminus \{ y \}) \\
  \freenames{x!\langle P \rangle} & := & \{ x \} \cup \{ P \} \\
  \freenames{P|Q} & := & \freenames{P} \cup \freenames{Q} \\
  \freenames{\dropn{x}} & := & \{ x \}
\end{eqnarray*}

The bound names of a process, $\boundnames{P}$, are those names occurring in $P$
that are not free. For example, in $x?(y).0$, the name $x$ is free, while $y$ is bound.

\begin{mathpar}
  \inferrule* [lab=monoidal-laws] {} { P|Q \equiv Q|P \and P|0 \equiv P \and P|(Q|R) \equiv (P|Q)|R }
\end{mathpar}

\begin{mathpar}
  \inferrule* [lab=alpha-equivalence] {} { (x)P \equiv (y)P\{y/x\} \and y \not\in \freenames{P} }
\end{mathpar}

\begin{definition}
Then two processes, $P,Q$, are alpha-equivalent if $P = Q\{\vec{y}/\vec{x}\}$ for
some $\vec{x} \in \boundnames{Q},\vec{y} \in \boundnames{P}$, where $Q\{\vec{y}/\vec{x}\}$
denotes the capture-avoiding substitution of $\vec{y}$ for $\vec{x}$ in $Q$.
\end{definition}

\begin{definition}
  The {\em structural congruence} \cite{SangiorgiWalker} , $\equiv$,
  between processes is the least congruence containing
  alpha-equivalence, satisfying the abelian monoid laws
  (associativity, commutativity and $\pzero$ as identity) for parallel
  composition $|$ and for summation $+$.
\end{definition}

\subsection{Name equivalence}

We take name equivalence, written $\nameeq$, to be the smallest
equivalence relation generated by the following rules.

\begin{mathpar}
\inferrule*[lab=Quote-drop]
{ }
{ \quotep{@{x}} \nameeq x }

\inferrule*[lab=Struct-equiv]
{ P \scong Q }
{ \quotep{P} \nameeq \quotep{Q} }
\end{mathpar}

The astute reader will have noticed that the mutual recursion of names
and processes imposes a mutual recursion on alpha-equivalence and
structural equivalence via name-equivalence. Fortunately, all of this
works out pleasantly and we may calculate in the natural way, free of
concern. The reader interested in the details is referred to the
appendix \ref{appendix:rho_details}.

\subsection{Substitution}

We use $\Proc$ for the set of processes, $\QProc$ for the set of
names, and $\id{\{}\vec{y} / \vec{x} \id{\}}$ to denote partial maps,
$s : \QProc \rightarrow \QProc$. A map, $s$ lifts, uniquely, to a map
on process terms, $\widehat{s} : \Proc \rightarrow \Proc$ by the
following equations.

\begin{mathpar}
  (0) \psubstp{Q}{P} := 0 \\
  (R \juxtap S) \psubstp{Q}{P}
  :=    
  (R)\psubstp{Q}{P} \juxtap (S) \psubstp{Q}{P} \\
  (x?(y).R) \psubstp{Q}{P}    
  :=    
  (x)\substp{Q}{P} (z)\concat( (R \psubstn{z}{y}) \psubstp{Q}{P} ) \\
  (\lift{x}{R}) \psubstp{Q}{P}  
  :=
  \lift{(x)\substp{Q}{P}}{ R \psubstp{Q}{P} } \\
%   (\dropn{x})  \psubstp{Q}{P}       
%   := 
%   \left\{ 
%     \begin{array}{ccc} 
%       \dropn{\quotep{Q}} & & x \nameeq \quotep{P} \\
%       \dropn{x} & & otherwise \\
%     \end{array}
%   \right. 
  (\dropn{x})  \psubstp{Q}{P}       
  := 
  \left\{ 
    \begin{array}{ccc} 
      Q & & x \nameeq \quotep{P} \\
      \dropn{x} & & otherwise \\
    \end{array}
  \right.
\end{mathpar}
 

where

\begin{eqnarray}
  (x)\id{\{} \lpquote Q \rpquote / \lpquote P \rpquote \id{\}}            = 
  \left\{ 
    \begin{array}{ccc}
      \lpquote Q \rpquote & & x \nameeq \lpquote P \rpquote \\
      x & & otherwise \\
    \end{array}
  \right. \nonumber
\end{eqnarray}

and $z$ is chosen distinct from $\quotep{P}$, $\quotep{Q}$, the free
names in $Q$, and all the names in $R$. Our $\alpha$-equivalence will
be built in the standard way from this substitution.

\begin{remark}\label{rem:no_self_referential_names}
  One consequence of these definitions is that $\forall P. \quotep{P}
  \not\in \freenames{P}$.
\end{remark}

\subsection{ Dynamic quote: an example }

Anticipating something of what's to come, consider applying the
substitution, $\widehat{\id{\{}u / z \id{\}}}$, to the following pair
of processes, $\lift{w}{y!(z)}$ and $w[ \lpquote y!(z) \rpquote ]$.

\begin{eqnarray}
	\lift{w}{y!(z)}\widehat{\id{\{}u / z \id{\}}}
		& = &
		\lift{w}{y!(u)} \nonumber\\
	w[ \lpquote y!(z) \rpquote ] \widehat{ \id{\{}u / z \id{\}} }
		& = &
		w[ \lpquote y!(z) \rpquote ] \nonumber
\end{eqnarray}

Because the body of the process between quotes is impervious to
substitution, we get radically different answers. In fact, by
examining the first process in an input context,
e.g. $x?(z).\lift{w}{y!(z)}$, we see that the process under the lift
operator may be shaped by prefixed inputs binding a name inside it. In
this sense, the lift operator will be seen as a way to dynamically
construct processes before reifying them as names.

Finally equipped with these standard features we can present the
dynamics of the calculus.

\subsubsection{Operational semantics} 

Finally, we introduce the computational dynamics. What marks these
algebras as distinct from other more traditionally studied algebraic
structures, e.g. vector spaces or polynomial rings, is the manner in
which dynamics is captured. In traditional structures, dynamics is typically
expressed through morphisms between such structures, as in linear maps
between vector spaces or morphisms between rings. In algebras
associated with the semantics of computation, the dynamics is
expressed as part of the algebraic structure itself, through a
reduction reduction relation typically denoted by $\red$. Below, we
give a recursive presentation of this relation for the calculus used
in the encoding.

$\red \subseteq \pi \times \pi$
$\red : \pi \to \mathcal{P}(\pi)$

\begin{mathpar}
  \inferrule* [lab=Comm] { \textsf{match}( x_{src}, x_{trgt} ) } { x_{trgt}?(y)P \; | \; x_{src}!\langle {Q} \rangle \red P\{\quotep{Q}/y}\} }
  \and \\
  \inferrule* [lab=Par] {{P} \red {P}'} {{{P} | {Q}} \red {{P}' | {Q}}}
  \and
  \inferrule* [lab=Equiv]{{{P} \scong {P}'} \andalso {{P}' \red {Q}'} \andalso {{Q}' \scong {Q}}}{{P} \red {Q}}
\end{mathpar}

\begin{eqnarray*}
  match_{\equiv} (\quotep{P},\quotep{Q}) & := & P \equiv Q \\
  match_{\dagger}(\quotep{P},\quotep{Q}) & := & \forall R. P|Q \red^{*} R => R \red^{*} 0 \\
  match_{K}(\quotep{P},\quotep{Q}) & := & K \mbox{ for some context } K
\end{eqnarray*}

$u?(x)P | u!\langle Q \rangle \red P\{\quotep{Q}/x\}$

%We write $\wred$ for $\red^*$, and $P\red$ if $\exists Q $ such that $ P \red Q$.
We write $P\red$ if $\exists Q $ such that $ P \red Q$ and $P\not\red$, otherwise.

\section{Replication}

As mentioned before, it is known that replication (and hence
recursion) can be implemented in a higher-order process algebra
\cite{SangiorgiWalker}. As our first example of calculation with the
machinery thus far presented we give the construction explicitly in
the {\rhoc}.

\begin{eqnarray}
	D_{x} & := & \prefix{x}{y}{(\binpar{\outputp{x}{y}}{@{y}})} \nonumber\\
	\bangp_{x}{P} & := & \binpar{{x}!\langle{\binpar{D_{x}}{P}}\rangle}{D_{x}} \nonumber
\end{eqnarray}

\begin{eqnarray}
	\bangp_{x}{P} & & \nonumber\\
	=
	& {x}!\langle{(\prefix{x}{y}{(\outputp{x}{y} | @{y})) | P}}\rangle 
	      | \prefix{x}{y}{(\outputp{x}{y} | @{y})} & \nonumber\\
	\red
	& (\outputp{x}{y} | @{y})\substn{\quotep{(\prefix{x}{y}{(@{y} | \outputp{x}{y})) | P}}}{y} & \nonumber\\
	=
	& \outputp{x}{\quotep{(\prefix{x}{y}{(\outputp{x}{y} | @{y})) | P}}}
	  | {(\prefix{x}{y}{(\outputp{x}{y} | @{y})) | P}} & \nonumber\\
	\red
	& \ldots & \nonumber\\
	\red^*
	& P | P | \ldots & \nonumber
\end{eqnarray}

Of course, this encoding, as an implementation, runs away, unfolding
$\bangp{P}$ eagerly. A lazier and more implementable replication
operator, restricted to input-guarded processes, may be obtained as follows.

\begin{eqnarray}
\bangp{\prefix{u}{v}{P}} 
	:= 
	\binpar{\lift{x}{\prefix{u}{v}{(\binpar{D(x)}{P})}}}{D(x)} \nonumber
\end{eqnarray}

\begin{remark}
  Note that the lazier definition still does not deal with summation
  or mixed summation (i.e. sums over input and output). The reader is
  invited to construct definitions of replication that deal with these
  features. 

  Further, the definitions are parameterized in a name, $x$. Can you,
  gentle reader, make a definition that eliminates this parameter and
  guarantees no accidental interaction between the replication
  machinery and the process being replicated -- i.e. no accidental
  sharing of names used by the process to get its work done and the
  name(s) used by the replication to effect copying. This latter
  revision of the definition of replication is crucial to obtaining
  the expected identity $!!P \sim !P$.
\end{remark}

\begin{remark}\label{rem:paradoxical_combinator}
  The reader familiar with the lambda calculus will have noticed the
  similarity between $D$ and the paradoxical combinator.

  [Ed. note: the existence of this seems to suggest we have to be more
  restrictive on the set of processes and names we admit if we are to
  support no-cloning.]
\end{remark}

\subsubsection{Bisimulation}

The computational dynamics gives rise to another kind of equivalence,
the equivalence of computational behavior. As previously mentioned
this is typically captured \emph{via} some form of bisimulation.

% The notion we use in this paper is weak barbed bisimulation
% \cite{milner91polyadicpi}.

The notion we use in this paper is derived from weak barbed
bisimulation \cite{milner91polyadicpi}. 

\begin{definition}
An \emph{observation relation}, $\downarrow_{\mathcal N}$, over a set
of names, $\mathcal N$, is the smallest relation satisfying the rules
below.

\infrule[Out-barb]{y \in {\mathcal N}, \; x \nameeq y}
		  {\outputp{x}{v} \downarrow_{\mathcal N} x}
\infrule[Par-barb]{\mbox{$P\downarrow_{\mathcal N} x$ or $Q\downarrow_{\mathcal N} x$}}
		  {\binpar{P}{Q} \downarrow_{\mathcal N} x}

We write $P \Downarrow_{\mathcal N} x$ if there is $Q$ such that 
$P \wred Q$ and $Q \downarrow_{\mathcal N} x$.
\end{definition}

\begin{definition}
%\label{def.bbisim}
An  ${\mathcal N}$-\emph{barbed bisimulation} over a set of names, ${\mathcal N}$, is a symmetric binary relation 
${\mathcal S}_{\mathcal N}$ between agents such that $P\rel{S}_{\mathcal N}Q$ implies:
\begin{enumerate}
\item If $P \red P'$ then $Q \wred Q'$ and $P'\rel{S}_{\mathcal N} Q'$.
\item If $P\downarrow_{\mathcal N} x$, then $Q\Downarrow_{\mathcal N} x$.
\end{enumerate}
$P$ is ${\mathcal N}$-barbed bisimilar to $Q$, written
$P \wbbisim_{\mathcal N} Q$, if $P \rel{S}_{\mathcal N} Q$ for some ${\mathcal N}$-barbed bisimulation ${\mathcal S}_{\mathcal N}$.
\end{definition}

$\mathcal{R} \subseteq \pi \times \pi$

$P \mathcal{R} Q => \forall P'. P \red P' \Rightarrow \exists Q'. Q \red Q', P' \mathcal{R} Q'$

$P \vdash x \Rightarrow Q \vdash x$

\begin{mathpar}
  \inferrule*[lab=Out-barb]{x \nameeq y}{{y}!\langle{Q}\rangle \vdash x}
  \and
  \inferrule*[lab=Par-barb]{\mbox{$P\vdash x$ or $Q\vdash x$}}{\binpar{P}{Q} \vdash x}
\end{mathpar}

\subsubsection{Contexts}

One of the principle advantages of computational calculi like the
$\pi$-calculus is a well-defined notion of context,
contextual-equivalence and a correlation between
contextual-equivalence and notions of bisimulation. The notion of
context allows the decomposition of a process into (sub-)process and
its syntactic environment, its context. Thus, a context may be
thought of as a process with a ``hole'' (written $\Box$) in it. The
application of a context $M$ to a process $P$, written $M[P]$, is
tantamount to filling the hole in $M$ with $P$. In this paper we do
not need the full weight of this theory, but do make use of the notion
of context in the proof the main theorem. 

\begin{mathpar}
  \inferrule* [lab=summation] {} {{M_{M},M_{N}} \bc \Box \;|\; x.M_{A} \;|\; M_{M}+M_{N}}
  \and
  \inferrule* [lab=agent] {} {{M_{A}} \bc (\vec{x})M_{P} \;| \; \clift{P_0,\ldots,M_{P},\ldots,P_N}}
  \and \\
  \inferrule* [lab=process] {} {{M_{P}} \bc M_{N} \;| \;P|M_{P} }
\end{mathpar} 

\begin{mathpar}
  \inferrule* [lab=sychronization] {} {M_{N} \bc \Box \;|\; x?M_{F} \;|\; x!M_{C}}
  \and
  \inferrule* [lab=abstraction] {} {{M_{F}} \bc (x)M_{P} }
  \and
  \inferrule* [lab=concretion] {} {{M_{C}} \bc \langle M_{P} \rangle }
  \and \\
  \inferrule* [lab=process] {} {{M_{P}} \bc M_{N} \;| \;P|M_{P} }
\end{mathpar}

\begin{definition}[contextual application] Given a context $M$, and
  process $P$, we define the \emph{contextual application}, $M[P] :=
  M\{P/\Box\}$. That is, the contextual application of M to P is the
  substitution of $P$ for $\Box$ in $M$.
\end{definition}

$\meaningof{-} : L \to \mathcal{P}(\pi)$

\begin{mathpar}
  \inferrule* [lab=collection] {} {\meaningof{true} = \pi, \and \meaningof{~E} = \pi \setminus \meaningof{E}, \and \meaningof{E_{1} \& E_{2}} = \meaningof{E_{1}} \cap \meaningof{E_{2}}}
\end{mathpar}

\begin{mathpar}
  \inferrule* [lab=structure] {} {\meaningof{0} = \{ P \in \pi | P \equiv 0 \}, \and \\ \meaningof{E_1 | E_2} = \{ P \in \pi | P \equiv P_{1} | P_{2}, P_{1} \in \meaningof{E_{1}}, P_{2} \in \meaningof{E_2}\} }
\end{mathpar}

\begin{mathpar}
 \inferrule* [lab=behavior] {} {\meaningof{\langle a?b \rangle E} = \{ P \in \pi | P \equiv Q | u?(y)P', \\ \and \\\\ \and \\ \;\;\; u \in \meaningof{a}, \forall z.P'\{z/y\} \in \meaningof{E\{z/b\}}\}, \and \\ \meaningof{a!E} = \{ P \in \pi | P \equiv Q | x!\langle P' \rangle, x \in \meaningof{a} P' \in \meaningof{E}\} }
\end{mathpar}

\begin{mathpar}
 \inferrule* [lab=nominal] {} {\meaningof{\quotep{E}} = \{ \quotep{P} \in \quotep{\pi} | P \in \meaningof{E} \}, \and \meaningof{\quotep{P}} = \{ \quotep{Q} \in \quotep{\pi} | P \equiv Q \} \and \\ \meaningof{@\quotep{E}} = \{ P \in \pi | P \equiv @x, x \in \meaningof{E} \}}
\end{mathpar}

\begin{eqnarray*}
  \\
  \meaningof{-} : TS \to ST
\end{eqnarray*}

\begin{eqnarray*}
  \\
  L : TS \to ST
\end{eqnarray*}

\begin{eqnarray*}
  \\
  P \models E \iff P \in \meaningof{E}
\end{eqnarray*}

\begin{eqnarray*}
  P \approx_{L} Q \iff \forall E \in L. P \models E \iff Q \models E
\end{eqnarray*}

\begin{eqnarray*}
  P \approx_{K} Q
\end{eqnarray*}

\begin{eqnarray*}
  P \approx Q
\end{eqnarray*}

$\approx_{K} = \approx = \approx_{L}$

\subsubsection{Contextual duality}

Note that contexts extend the quotation operation to a family of
operations from processes to names. Given a context, $M$, we can
define a \emph{nominal context}, $\quotep{M}$ by $\quotep{M}[P] :=
\quotep{M[P]}$. To foreshadow what is to come we observe that these
operations enjoy a duality with processes very much like the duality
between vectors and maps from vectors to scalars.

Further, because the calculus is essentially higher-order, we have a
correspondence between contexts and processes. More specifically,
given a name $x$ and a context $M$ we can construct $M^{*}_{x}$ such
that 

\begin{mathpar}
  M^{*}_{x} | \lift{x}{P} \red M[P]
\end{mathpar}

namely,

\begin{mathpar}
  M^{*}_{x} := x?(u).M[\dropn{u}]
\end{mathpar}

The dependence of $M^{*}_{x}$ on a name makes it an abstraction, 

\begin{mathpar}
  M^{*} := (x)x?(u).M[\dropn{u}]
\end{mathpar}

\subsection{Additional notation}

It will sometimes be convenient to denote the process a name
quotes. We already have the notation $x = \quotep{P}$, but it will be
convenient to introduce an alternate notation, $\procn{x}$, when we
want to emphasize the connection to the use of the name. Note that, by
virtue of name equivalence, $\quotep{\procn{x}} \nameeq x$; so, the
notation is consistent with previous definitions.

Further, because names have structure it is possible to effect
substitutions on the basis of that structure. This means we need to
upgrade our notation for substitutions, which we accomplish by
adapting comprehension notation. Thus,

\begin{mathpar}
  P\{ y / x : x \in S \}
\end{mathpar}

is interpreted to mean the process derived from P by replacing (in a
capture-avoiding manner) each occurrence of $x$ in $S$ by $y$. For example,

\begin{mathpar}
  P\{ \quotep{\procn{x}|\procn{x}} / x : x \in \freenames{P} \}
\end{mathpar}

will replace each (occurrence) of a free name $x$ in $P$ by
$\quotep{\procn{x}|\procn{x}}$.

Also, we will avail ourselves of the notation $x^{L}$ and $x^{R}$ to
denote injections of a name into disjoint copies of the name
space. There are numerous ways to accomplish this. One example can be
found in \cite{MeredithR05}. This notation overloads to vectors of
names: $\vec{x}^{\pi} := (x_{i}^{\pi} \; : \; 0 \leq i < |\vec{x}| )$ where $\pi \in \{L,R\}$.

We also use $P^{\Box} := P|\Box$.

In \cite{MeredithR05} an interpretation of the new operator is
given. It turns out that there are several possible interpretations
all enjoying the requisite algebraic properties of the operator (see
\cite{milner91polyadicpi}). We will therefore make liberal use of
$(\nu\; \vec{x})P$.

% subsection the_syntax_and_semantics_of_the_notation_system (end)   

\input{qm2pi.qmops} 

\input{qm2pi.sterngerlach} 

\input{qm2pi.metric} 

% section concurrent_process_calculi (end)

%\input{qm2pi.proofsketch}

% section proof sketch (end)

%\input{qm2pi.slviaknots} 

% section spatial logic via knots (end)

\input{qm2pi.conclusion}

% section conclusion (end)

%\input{qm2pi.dtcodes} 

% section wiring algorithm (end)

\input{qm2pi.ack} 

% section acknowledgments (end)

\newpage


\bibliographystyle{plain}   
\bibliography{../../biblios/main.bib}

\input{qm2pi.rhodetails}

\end{document}

 

% section notation (end)

\input{qm2pi.process.calculi} 

% section concurrent_process_calculi_and_spatial_logics_ (end)
    
%\documentclass[12pt]{llncs}
%\documentclass{jktr}

\usepackage[pdftex]{hyperref}                   
\usepackage {listings}
\usepackage {mathpartir}
\usepackage{bcprules}
%\usepackage{listings}
                       
\usepackage{graphicx} 
%\usepackage[margins=2.5cm,nohead,nofoot]{geometry}
%\usepackage{geometry}
\usepackage{amsfonts}
\usepackage{amstext}
\usepackage{latexsym}
\usepackage{amssymb}
\usepackage{color}


%\include{myPreamble}
\include{qm2pi.local} 

%\ifpdf
%\usepackage[pdftex]{graphicx}
%\else
%\usepackage{graphicx}
%\fi

 % \ifpdf
%  \usepackage{pdfsync}
%  \if


%\title{Brief Article}
%\author{David F. Snyder}
%\author{L.G. Meredith}

%\address{Dept. of Math., Texas State University--San Marcos, San Marcos, TX 78666}
       
\pagestyle{empty}


\begin{document}

\lstset{language=[Objective]Caml,frame=shadowbox}

\input{qm2pi.front}

% section front matter (end)

\input{qm2pi.intro} 
 
% section introduction (end)

% \input{qm2pi.knotations} 

% section notation (end)

\input{qm2pi.process.calculi} 

% section concurrent_process_calculi_and_spatial_logics_ (end)
    
%\input{qm2pi.knots2pi} 

%\input{qm2pi.trefoil} 

%\input{qm2pi.mainthm} 

% subsection basic_interpretation (end)

%\input{qm2pi.rho.presentation} 
\subsection{The syntax and semantics of the notation system}\label{sub:the_syntax_and_semantics_of_the_notation_system} % (fold)

We now summarize a technical presentation of the calculus that
embodies our theory of dynamics. The typical presentation of such a
calculus follows the style of giving generators and relations on
them. The grammar, below, describing term constructors, freely
generates the set of processes, $\Proc$. This set is then quotiented
by a relation known as structural congruence and it is over this set
that the notion of dynamics is expressed. This presentation is
essentially that of \cite{MeredithR05} with the addition of
polyadicity and summation. For readability we have relegated some of
the technical subtleties to an appendix.

\subsubsection{Process grammar}\label{subsub:process_grammar}

\begin{mathpar}
  \inferrule* [lab=synchronization] {} {{M} \bc \pzero \;|\; x?F \;|\; x!C }
  \and
  \inferrule* [lab=abstraction] {} {{F} \bc (x)P}
  \and
  \inferrule* [lab=concretion] {} {{C} \bc \langle Q \rangle}
  \and
  \inferrule* [lab=process] {} {{P,Q} \bc M \;| \;P|Q \;|\; @{x}}
  \and
  \inferrule* [lab=name] {} {{x} \bc \quotep{P}}
\end{mathpar} 

Note that $\vec{x}$ (resp. $\vec{P}$) denotes a vector of names
(resp. processes) of length $|\vec{x}|$ (resp. $|\vec{P}|$). We adopt
the following useful abbreviations.

\begin{mathpar}
   x?(\vec{y}).P := x.(\vec{y})P \and  x\clift{\vec{P}} := x.\clift{\vec{P}}
   \and x!(y) := \lift{x}{\dropn{y}}
   \and \Pi_{i=0}^{n-1}P_i := P_0 | \ldots | P_{n-1}
\end{mathpar}

\subsubsection{Structural congruence}

\paragraph{Free and bound names and alpha-equivalence.} At the
core of structural equivalence is alpha-equivalence which identifies
process that are the same up to a change of variable. Formally, we
recognize the distinction between free and bound names. The free names
of a process, $\freenames{P}$, may be calculated recursively as
follows:

\begin{mathpar}
\freenames{\pzero} := \emptyset
  \and \\
  \freenames{x?(y).P} := \{ x \} \cup (\freenames{P} \setminus \{ y \})
  \and 
  \freenames{x!\langle P \rangle} := \{ x \} \cup \{ P \} 
  \and \\
  \freenames{P|Q} := \freenames{P} \cup \freenames{Q}
  \and \\
  \freenames{@{x}} := \{ x \}
\end{mathpar}

$\pi$
$\quotep{\pi}$

$\freenames{-} : \pi \to \mathcal{P}(\quotep{\pi})$

\begin{eqnarray*}
  \freenames{\pzero} & := & \emptyset \\
  \freenames{x?(y).P} & := & \{ x \} \cup (\freenames{P} \setminus \{ y \}) \\
  \freenames{x!\langle P \rangle} & := & \{ x \} \cup \{ P \} \\
  \freenames{P|Q} & := & \freenames{P} \cup \freenames{Q} \\
  \freenames{\dropn{x}} & := & \{ x \}
\end{eqnarray*}

The bound names of a process, $\boundnames{P}$, are those names occurring in $P$
that are not free. For example, in $x?(y).0$, the name $x$ is free, while $y$ is bound.

\begin{mathpar}
  \inferrule* [lab=monoidal-laws] {} { P|Q \equiv Q|P \and P|0 \equiv P \and P|(Q|R) \equiv (P|Q)|R }
\end{mathpar}

\begin{mathpar}
  \inferrule* [lab=alpha-equivalence] {} { (x)P \equiv (y)P\{y/x\} \and y \not\in \freenames{P} }
\end{mathpar}

\begin{definition}
Then two processes, $P,Q$, are alpha-equivalent if $P = Q\{\vec{y}/\vec{x}\}$ for
some $\vec{x} \in \boundnames{Q},\vec{y} \in \boundnames{P}$, where $Q\{\vec{y}/\vec{x}\}$
denotes the capture-avoiding substitution of $\vec{y}$ for $\vec{x}$ in $Q$.
\end{definition}

\begin{definition}
  The {\em structural congruence} \cite{SangiorgiWalker} , $\equiv$,
  between processes is the least congruence containing
  alpha-equivalence, satisfying the abelian monoid laws
  (associativity, commutativity and $\pzero$ as identity) for parallel
  composition $|$ and for summation $+$.
\end{definition}

\subsection{Name equivalence}

We take name equivalence, written $\nameeq$, to be the smallest
equivalence relation generated by the following rules.

\begin{mathpar}
\inferrule*[lab=Quote-drop]
{ }
{ \quotep{@{x}} \nameeq x }

\inferrule*[lab=Struct-equiv]
{ P \scong Q }
{ \quotep{P} \nameeq \quotep{Q} }
\end{mathpar}

The astute reader will have noticed that the mutual recursion of names
and processes imposes a mutual recursion on alpha-equivalence and
structural equivalence via name-equivalence. Fortunately, all of this
works out pleasantly and we may calculate in the natural way, free of
concern. The reader interested in the details is referred to the
appendix \ref{appendix:rho_details}.

\subsection{Substitution}

We use $\Proc$ for the set of processes, $\QProc$ for the set of
names, and $\id{\{}\vec{y} / \vec{x} \id{\}}$ to denote partial maps,
$s : \QProc \rightarrow \QProc$. A map, $s$ lifts, uniquely, to a map
on process terms, $\widehat{s} : \Proc \rightarrow \Proc$ by the
following equations.

\begin{mathpar}
  (0) \psubstp{Q}{P} := 0 \\
  (R \juxtap S) \psubstp{Q}{P}
  :=    
  (R)\psubstp{Q}{P} \juxtap (S) \psubstp{Q}{P} \\
  (x?(y).R) \psubstp{Q}{P}    
  :=    
  (x)\substp{Q}{P} (z)\concat( (R \psubstn{z}{y}) \psubstp{Q}{P} ) \\
  (\lift{x}{R}) \psubstp{Q}{P}  
  :=
  \lift{(x)\substp{Q}{P}}{ R \psubstp{Q}{P} } \\
%   (\dropn{x})  \psubstp{Q}{P}       
%   := 
%   \left\{ 
%     \begin{array}{ccc} 
%       \dropn{\quotep{Q}} & & x \nameeq \quotep{P} \\
%       \dropn{x} & & otherwise \\
%     \end{array}
%   \right. 
  (\dropn{x})  \psubstp{Q}{P}       
  := 
  \left\{ 
    \begin{array}{ccc} 
      Q & & x \nameeq \quotep{P} \\
      \dropn{x} & & otherwise \\
    \end{array}
  \right.
\end{mathpar}
 

where

\begin{eqnarray}
  (x)\id{\{} \lpquote Q \rpquote / \lpquote P \rpquote \id{\}}            = 
  \left\{ 
    \begin{array}{ccc}
      \lpquote Q \rpquote & & x \nameeq \lpquote P \rpquote \\
      x & & otherwise \\
    \end{array}
  \right. \nonumber
\end{eqnarray}

and $z$ is chosen distinct from $\quotep{P}$, $\quotep{Q}$, the free
names in $Q$, and all the names in $R$. Our $\alpha$-equivalence will
be built in the standard way from this substitution.

\begin{remark}\label{rem:no_self_referential_names}
  One consequence of these definitions is that $\forall P. \quotep{P}
  \not\in \freenames{P}$.
\end{remark}

\subsection{ Dynamic quote: an example }

Anticipating something of what's to come, consider applying the
substitution, $\widehat{\id{\{}u / z \id{\}}}$, to the following pair
of processes, $\lift{w}{y!(z)}$ and $w[ \lpquote y!(z) \rpquote ]$.

\begin{eqnarray}
	\lift{w}{y!(z)}\widehat{\id{\{}u / z \id{\}}}
		& = &
		\lift{w}{y!(u)} \nonumber\\
	w[ \lpquote y!(z) \rpquote ] \widehat{ \id{\{}u / z \id{\}} }
		& = &
		w[ \lpquote y!(z) \rpquote ] \nonumber
\end{eqnarray}

Because the body of the process between quotes is impervious to
substitution, we get radically different answers. In fact, by
examining the first process in an input context,
e.g. $x?(z).\lift{w}{y!(z)}$, we see that the process under the lift
operator may be shaped by prefixed inputs binding a name inside it. In
this sense, the lift operator will be seen as a way to dynamically
construct processes before reifying them as names.

Finally equipped with these standard features we can present the
dynamics of the calculus.

\subsubsection{Operational semantics} 

Finally, we introduce the computational dynamics. What marks these
algebras as distinct from other more traditionally studied algebraic
structures, e.g. vector spaces or polynomial rings, is the manner in
which dynamics is captured. In traditional structures, dynamics is typically
expressed through morphisms between such structures, as in linear maps
between vector spaces or morphisms between rings. In algebras
associated with the semantics of computation, the dynamics is
expressed as part of the algebraic structure itself, through a
reduction reduction relation typically denoted by $\red$. Below, we
give a recursive presentation of this relation for the calculus used
in the encoding.

$\red \subseteq \pi \times \pi$
$\red : \pi \to \mathcal{P}(\pi)$

\begin{mathpar}
  \inferrule* [lab=Comm] { \textsf{match}( x_{src}, x_{trgt} ) } { x_{trgt}?(y)P \; | \; x_{src}!\langle {Q} \rangle \red P\{\quotep{Q}/y}\} }
  \and \\
  \inferrule* [lab=Par] {{P} \red {P}'} {{{P} | {Q}} \red {{P}' | {Q}}}
  \and
  \inferrule* [lab=Equiv]{{{P} \scong {P}'} \andalso {{P}' \red {Q}'} \andalso {{Q}' \scong {Q}}}{{P} \red {Q}}
\end{mathpar}

\begin{eqnarray*}
  match_{\equiv} (\quotep{P},\quotep{Q}) & := & P \equiv Q \\
  match_{\dagger}(\quotep{P},\quotep{Q}) & := & \forall R. P|Q \red^{*} R => R \red^{*} 0 \\
  match_{K}(\quotep{P},\quotep{Q}) & := & K \mbox{ for some context } K
\end{eqnarray*}

$u?(x)P | u!\langle Q \rangle \red P\{\quotep{Q}/x\}$

%We write $\wred$ for $\red^*$, and $P\red$ if $\exists Q $ such that $ P \red Q$.
We write $P\red$ if $\exists Q $ such that $ P \red Q$ and $P\not\red$, otherwise.

\section{Replication}

As mentioned before, it is known that replication (and hence
recursion) can be implemented in a higher-order process algebra
\cite{SangiorgiWalker}. As our first example of calculation with the
machinery thus far presented we give the construction explicitly in
the {\rhoc}.

\begin{eqnarray}
	D_{x} & := & \prefix{x}{y}{(\binpar{\outputp{x}{y}}{@{y}})} \nonumber\\
	\bangp_{x}{P} & := & \binpar{{x}!\langle{\binpar{D_{x}}{P}}\rangle}{D_{x}} \nonumber
\end{eqnarray}

\begin{eqnarray}
	\bangp_{x}{P} & & \nonumber\\
	=
	& {x}!\langle{(\prefix{x}{y}{(\outputp{x}{y} | @{y})) | P}}\rangle 
	      | \prefix{x}{y}{(\outputp{x}{y} | @{y})} & \nonumber\\
	\red
	& (\outputp{x}{y} | @{y})\substn{\quotep{(\prefix{x}{y}{(@{y} | \outputp{x}{y})) | P}}}{y} & \nonumber\\
	=
	& \outputp{x}{\quotep{(\prefix{x}{y}{(\outputp{x}{y} | @{y})) | P}}}
	  | {(\prefix{x}{y}{(\outputp{x}{y} | @{y})) | P}} & \nonumber\\
	\red
	& \ldots & \nonumber\\
	\red^*
	& P | P | \ldots & \nonumber
\end{eqnarray}

Of course, this encoding, as an implementation, runs away, unfolding
$\bangp{P}$ eagerly. A lazier and more implementable replication
operator, restricted to input-guarded processes, may be obtained as follows.

\begin{eqnarray}
\bangp{\prefix{u}{v}{P}} 
	:= 
	\binpar{\lift{x}{\prefix{u}{v}{(\binpar{D(x)}{P})}}}{D(x)} \nonumber
\end{eqnarray}

\begin{remark}
  Note that the lazier definition still does not deal with summation
  or mixed summation (i.e. sums over input and output). The reader is
  invited to construct definitions of replication that deal with these
  features. 

  Further, the definitions are parameterized in a name, $x$. Can you,
  gentle reader, make a definition that eliminates this parameter and
  guarantees no accidental interaction between the replication
  machinery and the process being replicated -- i.e. no accidental
  sharing of names used by the process to get its work done and the
  name(s) used by the replication to effect copying. This latter
  revision of the definition of replication is crucial to obtaining
  the expected identity $!!P \sim !P$.
\end{remark}

\begin{remark}\label{rem:paradoxical_combinator}
  The reader familiar with the lambda calculus will have noticed the
  similarity between $D$ and the paradoxical combinator.

  [Ed. note: the existence of this seems to suggest we have to be more
  restrictive on the set of processes and names we admit if we are to
  support no-cloning.]
\end{remark}

\subsubsection{Bisimulation}

The computational dynamics gives rise to another kind of equivalence,
the equivalence of computational behavior. As previously mentioned
this is typically captured \emph{via} some form of bisimulation.

% The notion we use in this paper is weak barbed bisimulation
% \cite{milner91polyadicpi}.

The notion we use in this paper is derived from weak barbed
bisimulation \cite{milner91polyadicpi}. 

\begin{definition}
An \emph{observation relation}, $\downarrow_{\mathcal N}$, over a set
of names, $\mathcal N$, is the smallest relation satisfying the rules
below.

\infrule[Out-barb]{y \in {\mathcal N}, \; x \nameeq y}
		  {\outputp{x}{v} \downarrow_{\mathcal N} x}
\infrule[Par-barb]{\mbox{$P\downarrow_{\mathcal N} x$ or $Q\downarrow_{\mathcal N} x$}}
		  {\binpar{P}{Q} \downarrow_{\mathcal N} x}

We write $P \Downarrow_{\mathcal N} x$ if there is $Q$ such that 
$P \wred Q$ and $Q \downarrow_{\mathcal N} x$.
\end{definition}

\begin{definition}
%\label{def.bbisim}
An  ${\mathcal N}$-\emph{barbed bisimulation} over a set of names, ${\mathcal N}$, is a symmetric binary relation 
${\mathcal S}_{\mathcal N}$ between agents such that $P\rel{S}_{\mathcal N}Q$ implies:
\begin{enumerate}
\item If $P \red P'$ then $Q \wred Q'$ and $P'\rel{S}_{\mathcal N} Q'$.
\item If $P\downarrow_{\mathcal N} x$, then $Q\Downarrow_{\mathcal N} x$.
\end{enumerate}
$P$ is ${\mathcal N}$-barbed bisimilar to $Q$, written
$P \wbbisim_{\mathcal N} Q$, if $P \rel{S}_{\mathcal N} Q$ for some ${\mathcal N}$-barbed bisimulation ${\mathcal S}_{\mathcal N}$.
\end{definition}

$\mathcal{R} \subseteq \pi \times \pi$

$P \mathcal{R} Q => \forall P'. P \red P' \Rightarrow \exists Q'. Q \red Q', P' \mathcal{R} Q'$

$P \vdash x \Rightarrow Q \vdash x$

\begin{mathpar}
  \inferrule*[lab=Out-barb]{x \nameeq y}{{y}!\langle{Q}\rangle \vdash x}
  \and
  \inferrule*[lab=Par-barb]{\mbox{$P\vdash x$ or $Q\vdash x$}}{\binpar{P}{Q} \vdash x}
\end{mathpar}

\subsubsection{Contexts}

One of the principle advantages of computational calculi like the
$\pi$-calculus is a well-defined notion of context,
contextual-equivalence and a correlation between
contextual-equivalence and notions of bisimulation. The notion of
context allows the decomposition of a process into (sub-)process and
its syntactic environment, its context. Thus, a context may be
thought of as a process with a ``hole'' (written $\Box$) in it. The
application of a context $M$ to a process $P$, written $M[P]$, is
tantamount to filling the hole in $M$ with $P$. In this paper we do
not need the full weight of this theory, but do make use of the notion
of context in the proof the main theorem. 

\begin{mathpar}
  \inferrule* [lab=summation] {} {{M_{M},M_{N}} \bc \Box \;|\; x.M_{A} \;|\; M_{M}+M_{N}}
  \and
  \inferrule* [lab=agent] {} {{M_{A}} \bc (\vec{x})M_{P} \;| \; \clift{P_0,\ldots,M_{P},\ldots,P_N}}
  \and \\
  \inferrule* [lab=process] {} {{M_{P}} \bc M_{N} \;| \;P|M_{P} }
\end{mathpar} 

\begin{mathpar}
  \inferrule* [lab=sychronization] {} {M_{N} \bc \Box \;|\; x?M_{F} \;|\; x!M_{C}}
  \and
  \inferrule* [lab=abstraction] {} {{M_{F}} \bc (x)M_{P} }
  \and
  \inferrule* [lab=concretion] {} {{M_{C}} \bc \langle M_{P} \rangle }
  \and \\
  \inferrule* [lab=process] {} {{M_{P}} \bc M_{N} \;| \;P|M_{P} }
\end{mathpar}

\begin{definition}[contextual application] Given a context $M$, and
  process $P$, we define the \emph{contextual application}, $M[P] :=
  M\{P/\Box\}$. That is, the contextual application of M to P is the
  substitution of $P$ for $\Box$ in $M$.
\end{definition}

$\meaningof{-} : L \to \mathcal{P}(\pi)$

\begin{mathpar}
  \inferrule* [lab=collection] {} {\meaningof{true} = \pi, \and \meaningof{~E} = \pi \setminus \meaningof{E}, \and \meaningof{E_{1} \& E_{2}} = \meaningof{E_{1}} \cap \meaningof{E_{2}}}
\end{mathpar}

\begin{mathpar}
  \inferrule* [lab=structure] {} {\meaningof{0} = \{ P \in \pi | P \equiv 0 \}, \and \\ \meaningof{E_1 | E_2} = \{ P \in \pi | P \equiv P_{1} | P_{2}, P_{1} \in \meaningof{E_{1}}, P_{2} \in \meaningof{E_2}\} }
\end{mathpar}

\begin{mathpar}
 \inferrule* [lab=behavior] {} {\meaningof{\langle a?b \rangle E} = \{ P \in \pi | P \equiv Q | u?(y)P', \\ \and \\\\ \and \\ \;\;\; u \in \meaningof{a}, \forall z.P'\{z/y\} \in \meaningof{E\{z/b\}}\}, \and \\ \meaningof{a!E} = \{ P \in \pi | P \equiv Q | x!\langle P' \rangle, x \in \meaningof{a} P' \in \meaningof{E}\} }
\end{mathpar}

\begin{mathpar}
 \inferrule* [lab=nominal] {} {\meaningof{\quotep{E}} = \{ \quotep{P} \in \quotep{\pi} | P \in \meaningof{E} \}, \and \meaningof{\quotep{P}} = \{ \quotep{Q} \in \quotep{\pi} | P \equiv Q \} \and \\ \meaningof{@\quotep{E}} = \{ P \in \pi | P \equiv @x, x \in \meaningof{E} \}}
\end{mathpar}

\begin{eqnarray*}
  \\
  \meaningof{-} : TS \to ST
\end{eqnarray*}

\begin{eqnarray*}
  \\
  L : TS \to ST
\end{eqnarray*}

\begin{eqnarray*}
  \\
  P \models E \iff P \in \meaningof{E}
\end{eqnarray*}

\begin{eqnarray*}
  P \approx_{L} Q \iff \forall E \in L. P \models E \iff Q \models E
\end{eqnarray*}

\begin{eqnarray*}
  P \approx_{K} Q
\end{eqnarray*}

\begin{eqnarray*}
  P \approx Q
\end{eqnarray*}

$\approx_{K} = \approx = \approx_{L}$

\subsubsection{Contextual duality}

Note that contexts extend the quotation operation to a family of
operations from processes to names. Given a context, $M$, we can
define a \emph{nominal context}, $\quotep{M}$ by $\quotep{M}[P] :=
\quotep{M[P]}$. To foreshadow what is to come we observe that these
operations enjoy a duality with processes very much like the duality
between vectors and maps from vectors to scalars.

Further, because the calculus is essentially higher-order, we have a
correspondence between contexts and processes. More specifically,
given a name $x$ and a context $M$ we can construct $M^{*}_{x}$ such
that 

\begin{mathpar}
  M^{*}_{x} | \lift{x}{P} \red M[P]
\end{mathpar}

namely,

\begin{mathpar}
  M^{*}_{x} := x?(u).M[\dropn{u}]
\end{mathpar}

The dependence of $M^{*}_{x}$ on a name makes it an abstraction, 

\begin{mathpar}
  M^{*} := (x)x?(u).M[\dropn{u}]
\end{mathpar}

\subsection{Additional notation}

It will sometimes be convenient to denote the process a name
quotes. We already have the notation $x = \quotep{P}$, but it will be
convenient to introduce an alternate notation, $\procn{x}$, when we
want to emphasize the connection to the use of the name. Note that, by
virtue of name equivalence, $\quotep{\procn{x}} \nameeq x$; so, the
notation is consistent with previous definitions.

Further, because names have structure it is possible to effect
substitutions on the basis of that structure. This means we need to
upgrade our notation for substitutions, which we accomplish by
adapting comprehension notation. Thus,

\begin{mathpar}
  P\{ y / x : x \in S \}
\end{mathpar}

is interpreted to mean the process derived from P by replacing (in a
capture-avoiding manner) each occurrence of $x$ in $S$ by $y$. For example,

\begin{mathpar}
  P\{ \quotep{\procn{x}|\procn{x}} / x : x \in \freenames{P} \}
\end{mathpar}

will replace each (occurrence) of a free name $x$ in $P$ by
$\quotep{\procn{x}|\procn{x}}$.

Also, we will avail ourselves of the notation $x^{L}$ and $x^{R}$ to
denote injections of a name into disjoint copies of the name
space. There are numerous ways to accomplish this. One example can be
found in \cite{MeredithR05}. This notation overloads to vectors of
names: $\vec{x}^{\pi} := (x_{i}^{\pi} \; : \; 0 \leq i < |\vec{x}| )$ where $\pi \in \{L,R\}$.

We also use $P^{\Box} := P|\Box$.

In \cite{MeredithR05} an interpretation of the new operator is
given. It turns out that there are several possible interpretations
all enjoying the requisite algebraic properties of the operator (see
\cite{milner91polyadicpi}). We will therefore make liberal use of
$(\nu\; \vec{x})P$.

% subsection the_syntax_and_semantics_of_the_notation_system (end)   

\input{qm2pi.qmops} 

\input{qm2pi.sterngerlach} 

\input{qm2pi.metric} 

% section concurrent_process_calculi (end)

%\input{qm2pi.proofsketch}

% section proof sketch (end)

%\input{qm2pi.slviaknots} 

% section spatial logic via knots (end)

\input{qm2pi.conclusion}

% section conclusion (end)

%\input{qm2pi.dtcodes} 

% section wiring algorithm (end)

\input{qm2pi.ack} 

% section acknowledgments (end)

\newpage


\bibliographystyle{plain}   
\bibliography{../../biblios/main.bib}

\input{qm2pi.rhodetails}

\end{document}

 

%\documentclass[12pt]{llncs}
%\documentclass{jktr}

\usepackage[pdftex]{hyperref}                   
\usepackage {listings}
\usepackage {mathpartir}
\usepackage{bcprules}
%\usepackage{listings}
                       
\usepackage{graphicx} 
%\usepackage[margins=2.5cm,nohead,nofoot]{geometry}
%\usepackage{geometry}
\usepackage{amsfonts}
\usepackage{amstext}
\usepackage{latexsym}
\usepackage{amssymb}
\usepackage{color}


%\include{myPreamble}
\include{qm2pi.local} 

%\ifpdf
%\usepackage[pdftex]{graphicx}
%\else
%\usepackage{graphicx}
%\fi

 % \ifpdf
%  \usepackage{pdfsync}
%  \if


%\title{Brief Article}
%\author{David F. Snyder}
%\author{L.G. Meredith}

%\address{Dept. of Math., Texas State University--San Marcos, San Marcos, TX 78666}
       
\pagestyle{empty}


\begin{document}

\lstset{language=[Objective]Caml,frame=shadowbox}

\input{qm2pi.front}

% section front matter (end)

\input{qm2pi.intro} 
 
% section introduction (end)

% \input{qm2pi.knotations} 

% section notation (end)

\input{qm2pi.process.calculi} 

% section concurrent_process_calculi_and_spatial_logics_ (end)
    
%\input{qm2pi.knots2pi} 

%\input{qm2pi.trefoil} 

%\input{qm2pi.mainthm} 

% subsection basic_interpretation (end)

%\input{qm2pi.rho.presentation} 
\subsection{The syntax and semantics of the notation system}\label{sub:the_syntax_and_semantics_of_the_notation_system} % (fold)

We now summarize a technical presentation of the calculus that
embodies our theory of dynamics. The typical presentation of such a
calculus follows the style of giving generators and relations on
them. The grammar, below, describing term constructors, freely
generates the set of processes, $\Proc$. This set is then quotiented
by a relation known as structural congruence and it is over this set
that the notion of dynamics is expressed. This presentation is
essentially that of \cite{MeredithR05} with the addition of
polyadicity and summation. For readability we have relegated some of
the technical subtleties to an appendix.

\subsubsection{Process grammar}\label{subsub:process_grammar}

\begin{mathpar}
  \inferrule* [lab=synchronization] {} {{M} \bc \pzero \;|\; x?F \;|\; x!C }
  \and
  \inferrule* [lab=abstraction] {} {{F} \bc (x)P}
  \and
  \inferrule* [lab=concretion] {} {{C} \bc \langle Q \rangle}
  \and
  \inferrule* [lab=process] {} {{P,Q} \bc M \;| \;P|Q \;|\; @{x}}
  \and
  \inferrule* [lab=name] {} {{x} \bc \quotep{P}}
\end{mathpar} 

Note that $\vec{x}$ (resp. $\vec{P}$) denotes a vector of names
(resp. processes) of length $|\vec{x}|$ (resp. $|\vec{P}|$). We adopt
the following useful abbreviations.

\begin{mathpar}
   x?(\vec{y}).P := x.(\vec{y})P \and  x\clift{\vec{P}} := x.\clift{\vec{P}}
   \and x!(y) := \lift{x}{\dropn{y}}
   \and \Pi_{i=0}^{n-1}P_i := P_0 | \ldots | P_{n-1}
\end{mathpar}

\subsubsection{Structural congruence}

\paragraph{Free and bound names and alpha-equivalence.} At the
core of structural equivalence is alpha-equivalence which identifies
process that are the same up to a change of variable. Formally, we
recognize the distinction between free and bound names. The free names
of a process, $\freenames{P}$, may be calculated recursively as
follows:

\begin{mathpar}
\freenames{\pzero} := \emptyset
  \and \\
  \freenames{x?(y).P} := \{ x \} \cup (\freenames{P} \setminus \{ y \})
  \and 
  \freenames{x!\langle P \rangle} := \{ x \} \cup \{ P \} 
  \and \\
  \freenames{P|Q} := \freenames{P} \cup \freenames{Q}
  \and \\
  \freenames{@{x}} := \{ x \}
\end{mathpar}

$\pi$
$\quotep{\pi}$

$\freenames{-} : \pi \to \mathcal{P}(\quotep{\pi})$

\begin{eqnarray*}
  \freenames{\pzero} & := & \emptyset \\
  \freenames{x?(y).P} & := & \{ x \} \cup (\freenames{P} \setminus \{ y \}) \\
  \freenames{x!\langle P \rangle} & := & \{ x \} \cup \{ P \} \\
  \freenames{P|Q} & := & \freenames{P} \cup \freenames{Q} \\
  \freenames{\dropn{x}} & := & \{ x \}
\end{eqnarray*}

The bound names of a process, $\boundnames{P}$, are those names occurring in $P$
that are not free. For example, in $x?(y).0$, the name $x$ is free, while $y$ is bound.

\begin{mathpar}
  \inferrule* [lab=monoidal-laws] {} { P|Q \equiv Q|P \and P|0 \equiv P \and P|(Q|R) \equiv (P|Q)|R }
\end{mathpar}

\begin{mathpar}
  \inferrule* [lab=alpha-equivalence] {} { (x)P \equiv (y)P\{y/x\} \and y \not\in \freenames{P} }
\end{mathpar}

\begin{definition}
Then two processes, $P,Q$, are alpha-equivalent if $P = Q\{\vec{y}/\vec{x}\}$ for
some $\vec{x} \in \boundnames{Q},\vec{y} \in \boundnames{P}$, where $Q\{\vec{y}/\vec{x}\}$
denotes the capture-avoiding substitution of $\vec{y}$ for $\vec{x}$ in $Q$.
\end{definition}

\begin{definition}
  The {\em structural congruence} \cite{SangiorgiWalker} , $\equiv$,
  between processes is the least congruence containing
  alpha-equivalence, satisfying the abelian monoid laws
  (associativity, commutativity and $\pzero$ as identity) for parallel
  composition $|$ and for summation $+$.
\end{definition}

\subsection{Name equivalence}

We take name equivalence, written $\nameeq$, to be the smallest
equivalence relation generated by the following rules.

\begin{mathpar}
\inferrule*[lab=Quote-drop]
{ }
{ \quotep{@{x}} \nameeq x }

\inferrule*[lab=Struct-equiv]
{ P \scong Q }
{ \quotep{P} \nameeq \quotep{Q} }
\end{mathpar}

The astute reader will have noticed that the mutual recursion of names
and processes imposes a mutual recursion on alpha-equivalence and
structural equivalence via name-equivalence. Fortunately, all of this
works out pleasantly and we may calculate in the natural way, free of
concern. The reader interested in the details is referred to the
appendix \ref{appendix:rho_details}.

\subsection{Substitution}

We use $\Proc$ for the set of processes, $\QProc$ for the set of
names, and $\id{\{}\vec{y} / \vec{x} \id{\}}$ to denote partial maps,
$s : \QProc \rightarrow \QProc$. A map, $s$ lifts, uniquely, to a map
on process terms, $\widehat{s} : \Proc \rightarrow \Proc$ by the
following equations.

\begin{mathpar}
  (0) \psubstp{Q}{P} := 0 \\
  (R \juxtap S) \psubstp{Q}{P}
  :=    
  (R)\psubstp{Q}{P} \juxtap (S) \psubstp{Q}{P} \\
  (x?(y).R) \psubstp{Q}{P}    
  :=    
  (x)\substp{Q}{P} (z)\concat( (R \psubstn{z}{y}) \psubstp{Q}{P} ) \\
  (\lift{x}{R}) \psubstp{Q}{P}  
  :=
  \lift{(x)\substp{Q}{P}}{ R \psubstp{Q}{P} } \\
%   (\dropn{x})  \psubstp{Q}{P}       
%   := 
%   \left\{ 
%     \begin{array}{ccc} 
%       \dropn{\quotep{Q}} & & x \nameeq \quotep{P} \\
%       \dropn{x} & & otherwise \\
%     \end{array}
%   \right. 
  (\dropn{x})  \psubstp{Q}{P}       
  := 
  \left\{ 
    \begin{array}{ccc} 
      Q & & x \nameeq \quotep{P} \\
      \dropn{x} & & otherwise \\
    \end{array}
  \right.
\end{mathpar}
 

where

\begin{eqnarray}
  (x)\id{\{} \lpquote Q \rpquote / \lpquote P \rpquote \id{\}}            = 
  \left\{ 
    \begin{array}{ccc}
      \lpquote Q \rpquote & & x \nameeq \lpquote P \rpquote \\
      x & & otherwise \\
    \end{array}
  \right. \nonumber
\end{eqnarray}

and $z$ is chosen distinct from $\quotep{P}$, $\quotep{Q}$, the free
names in $Q$, and all the names in $R$. Our $\alpha$-equivalence will
be built in the standard way from this substitution.

\begin{remark}\label{rem:no_self_referential_names}
  One consequence of these definitions is that $\forall P. \quotep{P}
  \not\in \freenames{P}$.
\end{remark}

\subsection{ Dynamic quote: an example }

Anticipating something of what's to come, consider applying the
substitution, $\widehat{\id{\{}u / z \id{\}}}$, to the following pair
of processes, $\lift{w}{y!(z)}$ and $w[ \lpquote y!(z) \rpquote ]$.

\begin{eqnarray}
	\lift{w}{y!(z)}\widehat{\id{\{}u / z \id{\}}}
		& = &
		\lift{w}{y!(u)} \nonumber\\
	w[ \lpquote y!(z) \rpquote ] \widehat{ \id{\{}u / z \id{\}} }
		& = &
		w[ \lpquote y!(z) \rpquote ] \nonumber
\end{eqnarray}

Because the body of the process between quotes is impervious to
substitution, we get radically different answers. In fact, by
examining the first process in an input context,
e.g. $x?(z).\lift{w}{y!(z)}$, we see that the process under the lift
operator may be shaped by prefixed inputs binding a name inside it. In
this sense, the lift operator will be seen as a way to dynamically
construct processes before reifying them as names.

Finally equipped with these standard features we can present the
dynamics of the calculus.

\subsubsection{Operational semantics} 

Finally, we introduce the computational dynamics. What marks these
algebras as distinct from other more traditionally studied algebraic
structures, e.g. vector spaces or polynomial rings, is the manner in
which dynamics is captured. In traditional structures, dynamics is typically
expressed through morphisms between such structures, as in linear maps
between vector spaces or morphisms between rings. In algebras
associated with the semantics of computation, the dynamics is
expressed as part of the algebraic structure itself, through a
reduction reduction relation typically denoted by $\red$. Below, we
give a recursive presentation of this relation for the calculus used
in the encoding.

$\red \subseteq \pi \times \pi$
$\red : \pi \to \mathcal{P}(\pi)$

\begin{mathpar}
  \inferrule* [lab=Comm] { \textsf{match}( x_{src}, x_{trgt} ) } { x_{trgt}?(y)P \; | \; x_{src}!\langle {Q} \rangle \red P\{\quotep{Q}/y}\} }
  \and \\
  \inferrule* [lab=Par] {{P} \red {P}'} {{{P} | {Q}} \red {{P}' | {Q}}}
  \and
  \inferrule* [lab=Equiv]{{{P} \scong {P}'} \andalso {{P}' \red {Q}'} \andalso {{Q}' \scong {Q}}}{{P} \red {Q}}
\end{mathpar}

\begin{eqnarray*}
  match_{\equiv} (\quotep{P},\quotep{Q}) & := & P \equiv Q \\
  match_{\dagger}(\quotep{P},\quotep{Q}) & := & \forall R. P|Q \red^{*} R => R \red^{*} 0 \\
  match_{K}(\quotep{P},\quotep{Q}) & := & K \mbox{ for some context } K
\end{eqnarray*}

$u?(x)P | u!\langle Q \rangle \red P\{\quotep{Q}/x\}$

%We write $\wred$ for $\red^*$, and $P\red$ if $\exists Q $ such that $ P \red Q$.
We write $P\red$ if $\exists Q $ such that $ P \red Q$ and $P\not\red$, otherwise.

\section{Replication}

As mentioned before, it is known that replication (and hence
recursion) can be implemented in a higher-order process algebra
\cite{SangiorgiWalker}. As our first example of calculation with the
machinery thus far presented we give the construction explicitly in
the {\rhoc}.

\begin{eqnarray}
	D_{x} & := & \prefix{x}{y}{(\binpar{\outputp{x}{y}}{@{y}})} \nonumber\\
	\bangp_{x}{P} & := & \binpar{{x}!\langle{\binpar{D_{x}}{P}}\rangle}{D_{x}} \nonumber
\end{eqnarray}

\begin{eqnarray}
	\bangp_{x}{P} & & \nonumber\\
	=
	& {x}!\langle{(\prefix{x}{y}{(\outputp{x}{y} | @{y})) | P}}\rangle 
	      | \prefix{x}{y}{(\outputp{x}{y} | @{y})} & \nonumber\\
	\red
	& (\outputp{x}{y} | @{y})\substn{\quotep{(\prefix{x}{y}{(@{y} | \outputp{x}{y})) | P}}}{y} & \nonumber\\
	=
	& \outputp{x}{\quotep{(\prefix{x}{y}{(\outputp{x}{y} | @{y})) | P}}}
	  | {(\prefix{x}{y}{(\outputp{x}{y} | @{y})) | P}} & \nonumber\\
	\red
	& \ldots & \nonumber\\
	\red^*
	& P | P | \ldots & \nonumber
\end{eqnarray}

Of course, this encoding, as an implementation, runs away, unfolding
$\bangp{P}$ eagerly. A lazier and more implementable replication
operator, restricted to input-guarded processes, may be obtained as follows.

\begin{eqnarray}
\bangp{\prefix{u}{v}{P}} 
	:= 
	\binpar{\lift{x}{\prefix{u}{v}{(\binpar{D(x)}{P})}}}{D(x)} \nonumber
\end{eqnarray}

\begin{remark}
  Note that the lazier definition still does not deal with summation
  or mixed summation (i.e. sums over input and output). The reader is
  invited to construct definitions of replication that deal with these
  features. 

  Further, the definitions are parameterized in a name, $x$. Can you,
  gentle reader, make a definition that eliminates this parameter and
  guarantees no accidental interaction between the replication
  machinery and the process being replicated -- i.e. no accidental
  sharing of names used by the process to get its work done and the
  name(s) used by the replication to effect copying. This latter
  revision of the definition of replication is crucial to obtaining
  the expected identity $!!P \sim !P$.
\end{remark}

\begin{remark}\label{rem:paradoxical_combinator}
  The reader familiar with the lambda calculus will have noticed the
  similarity between $D$ and the paradoxical combinator.

  [Ed. note: the existence of this seems to suggest we have to be more
  restrictive on the set of processes and names we admit if we are to
  support no-cloning.]
\end{remark}

\subsubsection{Bisimulation}

The computational dynamics gives rise to another kind of equivalence,
the equivalence of computational behavior. As previously mentioned
this is typically captured \emph{via} some form of bisimulation.

% The notion we use in this paper is weak barbed bisimulation
% \cite{milner91polyadicpi}.

The notion we use in this paper is derived from weak barbed
bisimulation \cite{milner91polyadicpi}. 

\begin{definition}
An \emph{observation relation}, $\downarrow_{\mathcal N}$, over a set
of names, $\mathcal N$, is the smallest relation satisfying the rules
below.

\infrule[Out-barb]{y \in {\mathcal N}, \; x \nameeq y}
		  {\outputp{x}{v} \downarrow_{\mathcal N} x}
\infrule[Par-barb]{\mbox{$P\downarrow_{\mathcal N} x$ or $Q\downarrow_{\mathcal N} x$}}
		  {\binpar{P}{Q} \downarrow_{\mathcal N} x}

We write $P \Downarrow_{\mathcal N} x$ if there is $Q$ such that 
$P \wred Q$ and $Q \downarrow_{\mathcal N} x$.
\end{definition}

\begin{definition}
%\label{def.bbisim}
An  ${\mathcal N}$-\emph{barbed bisimulation} over a set of names, ${\mathcal N}$, is a symmetric binary relation 
${\mathcal S}_{\mathcal N}$ between agents such that $P\rel{S}_{\mathcal N}Q$ implies:
\begin{enumerate}
\item If $P \red P'$ then $Q \wred Q'$ and $P'\rel{S}_{\mathcal N} Q'$.
\item If $P\downarrow_{\mathcal N} x$, then $Q\Downarrow_{\mathcal N} x$.
\end{enumerate}
$P$ is ${\mathcal N}$-barbed bisimilar to $Q$, written
$P \wbbisim_{\mathcal N} Q$, if $P \rel{S}_{\mathcal N} Q$ for some ${\mathcal N}$-barbed bisimulation ${\mathcal S}_{\mathcal N}$.
\end{definition}

$\mathcal{R} \subseteq \pi \times \pi$

$P \mathcal{R} Q => \forall P'. P \red P' \Rightarrow \exists Q'. Q \red Q', P' \mathcal{R} Q'$

$P \vdash x \Rightarrow Q \vdash x$

\begin{mathpar}
  \inferrule*[lab=Out-barb]{x \nameeq y}{{y}!\langle{Q}\rangle \vdash x}
  \and
  \inferrule*[lab=Par-barb]{\mbox{$P\vdash x$ or $Q\vdash x$}}{\binpar{P}{Q} \vdash x}
\end{mathpar}

\subsubsection{Contexts}

One of the principle advantages of computational calculi like the
$\pi$-calculus is a well-defined notion of context,
contextual-equivalence and a correlation between
contextual-equivalence and notions of bisimulation. The notion of
context allows the decomposition of a process into (sub-)process and
its syntactic environment, its context. Thus, a context may be
thought of as a process with a ``hole'' (written $\Box$) in it. The
application of a context $M$ to a process $P$, written $M[P]$, is
tantamount to filling the hole in $M$ with $P$. In this paper we do
not need the full weight of this theory, but do make use of the notion
of context in the proof the main theorem. 

\begin{mathpar}
  \inferrule* [lab=summation] {} {{M_{M},M_{N}} \bc \Box \;|\; x.M_{A} \;|\; M_{M}+M_{N}}
  \and
  \inferrule* [lab=agent] {} {{M_{A}} \bc (\vec{x})M_{P} \;| \; \clift{P_0,\ldots,M_{P},\ldots,P_N}}
  \and \\
  \inferrule* [lab=process] {} {{M_{P}} \bc M_{N} \;| \;P|M_{P} }
\end{mathpar} 

\begin{mathpar}
  \inferrule* [lab=sychronization] {} {M_{N} \bc \Box \;|\; x?M_{F} \;|\; x!M_{C}}
  \and
  \inferrule* [lab=abstraction] {} {{M_{F}} \bc (x)M_{P} }
  \and
  \inferrule* [lab=concretion] {} {{M_{C}} \bc \langle M_{P} \rangle }
  \and \\
  \inferrule* [lab=process] {} {{M_{P}} \bc M_{N} \;| \;P|M_{P} }
\end{mathpar}

\begin{definition}[contextual application] Given a context $M$, and
  process $P$, we define the \emph{contextual application}, $M[P] :=
  M\{P/\Box\}$. That is, the contextual application of M to P is the
  substitution of $P$ for $\Box$ in $M$.
\end{definition}

$\meaningof{-} : L \to \mathcal{P}(\pi)$

\begin{mathpar}
  \inferrule* [lab=collection] {} {\meaningof{true} = \pi, \and \meaningof{~E} = \pi \setminus \meaningof{E}, \and \meaningof{E_{1} \& E_{2}} = \meaningof{E_{1}} \cap \meaningof{E_{2}}}
\end{mathpar}

\begin{mathpar}
  \inferrule* [lab=structure] {} {\meaningof{0} = \{ P \in \pi | P \equiv 0 \}, \and \\ \meaningof{E_1 | E_2} = \{ P \in \pi | P \equiv P_{1} | P_{2}, P_{1} \in \meaningof{E_{1}}, P_{2} \in \meaningof{E_2}\} }
\end{mathpar}

\begin{mathpar}
 \inferrule* [lab=behavior] {} {\meaningof{\langle a?b \rangle E} = \{ P \in \pi | P \equiv Q | u?(y)P', \\ \and \\\\ \and \\ \;\;\; u \in \meaningof{a}, \forall z.P'\{z/y\} \in \meaningof{E\{z/b\}}\}, \and \\ \meaningof{a!E} = \{ P \in \pi | P \equiv Q | x!\langle P' \rangle, x \in \meaningof{a} P' \in \meaningof{E}\} }
\end{mathpar}

\begin{mathpar}
 \inferrule* [lab=nominal] {} {\meaningof{\quotep{E}} = \{ \quotep{P} \in \quotep{\pi} | P \in \meaningof{E} \}, \and \meaningof{\quotep{P}} = \{ \quotep{Q} \in \quotep{\pi} | P \equiv Q \} \and \\ \meaningof{@\quotep{E}} = \{ P \in \pi | P \equiv @x, x \in \meaningof{E} \}}
\end{mathpar}

\begin{eqnarray*}
  \\
  \meaningof{-} : TS \to ST
\end{eqnarray*}

\begin{eqnarray*}
  \\
  L : TS \to ST
\end{eqnarray*}

\begin{eqnarray*}
  \\
  P \models E \iff P \in \meaningof{E}
\end{eqnarray*}

\begin{eqnarray*}
  P \approx_{L} Q \iff \forall E \in L. P \models E \iff Q \models E
\end{eqnarray*}

\begin{eqnarray*}
  P \approx_{K} Q
\end{eqnarray*}

\begin{eqnarray*}
  P \approx Q
\end{eqnarray*}

$\approx_{K} = \approx = \approx_{L}$

\subsubsection{Contextual duality}

Note that contexts extend the quotation operation to a family of
operations from processes to names. Given a context, $M$, we can
define a \emph{nominal context}, $\quotep{M}$ by $\quotep{M}[P] :=
\quotep{M[P]}$. To foreshadow what is to come we observe that these
operations enjoy a duality with processes very much like the duality
between vectors and maps from vectors to scalars.

Further, because the calculus is essentially higher-order, we have a
correspondence between contexts and processes. More specifically,
given a name $x$ and a context $M$ we can construct $M^{*}_{x}$ such
that 

\begin{mathpar}
  M^{*}_{x} | \lift{x}{P} \red M[P]
\end{mathpar}

namely,

\begin{mathpar}
  M^{*}_{x} := x?(u).M[\dropn{u}]
\end{mathpar}

The dependence of $M^{*}_{x}$ on a name makes it an abstraction, 

\begin{mathpar}
  M^{*} := (x)x?(u).M[\dropn{u}]
\end{mathpar}

\subsection{Additional notation}

It will sometimes be convenient to denote the process a name
quotes. We already have the notation $x = \quotep{P}$, but it will be
convenient to introduce an alternate notation, $\procn{x}$, when we
want to emphasize the connection to the use of the name. Note that, by
virtue of name equivalence, $\quotep{\procn{x}} \nameeq x$; so, the
notation is consistent with previous definitions.

Further, because names have structure it is possible to effect
substitutions on the basis of that structure. This means we need to
upgrade our notation for substitutions, which we accomplish by
adapting comprehension notation. Thus,

\begin{mathpar}
  P\{ y / x : x \in S \}
\end{mathpar}

is interpreted to mean the process derived from P by replacing (in a
capture-avoiding manner) each occurrence of $x$ in $S$ by $y$. For example,

\begin{mathpar}
  P\{ \quotep{\procn{x}|\procn{x}} / x : x \in \freenames{P} \}
\end{mathpar}

will replace each (occurrence) of a free name $x$ in $P$ by
$\quotep{\procn{x}|\procn{x}}$.

Also, we will avail ourselves of the notation $x^{L}$ and $x^{R}$ to
denote injections of a name into disjoint copies of the name
space. There are numerous ways to accomplish this. One example can be
found in \cite{MeredithR05}. This notation overloads to vectors of
names: $\vec{x}^{\pi} := (x_{i}^{\pi} \; : \; 0 \leq i < |\vec{x}| )$ where $\pi \in \{L,R\}$.

We also use $P^{\Box} := P|\Box$.

In \cite{MeredithR05} an interpretation of the new operator is
given. It turns out that there are several possible interpretations
all enjoying the requisite algebraic properties of the operator (see
\cite{milner91polyadicpi}). We will therefore make liberal use of
$(\nu\; \vec{x})P$.

% subsection the_syntax_and_semantics_of_the_notation_system (end)   

\input{qm2pi.qmops} 

\input{qm2pi.sterngerlach} 

\input{qm2pi.metric} 

% section concurrent_process_calculi (end)

%\input{qm2pi.proofsketch}

% section proof sketch (end)

%\input{qm2pi.slviaknots} 

% section spatial logic via knots (end)

\input{qm2pi.conclusion}

% section conclusion (end)

%\input{qm2pi.dtcodes} 

% section wiring algorithm (end)

\input{qm2pi.ack} 

% section acknowledgments (end)

\newpage


\bibliographystyle{plain}   
\bibliography{../../biblios/main.bib}

\input{qm2pi.rhodetails}

\end{document}

 

%\documentclass[12pt]{llncs}
%\documentclass{jktr}

\usepackage[pdftex]{hyperref}                   
\usepackage {listings}
\usepackage {mathpartir}
\usepackage{bcprules}
%\usepackage{listings}
                       
\usepackage{graphicx} 
%\usepackage[margins=2.5cm,nohead,nofoot]{geometry}
%\usepackage{geometry}
\usepackage{amsfonts}
\usepackage{amstext}
\usepackage{latexsym}
\usepackage{amssymb}
\usepackage{color}


%\include{myPreamble}
\include{qm2pi.local} 

%\ifpdf
%\usepackage[pdftex]{graphicx}
%\else
%\usepackage{graphicx}
%\fi

 % \ifpdf
%  \usepackage{pdfsync}
%  \if


%\title{Brief Article}
%\author{David F. Snyder}
%\author{L.G. Meredith}

%\address{Dept. of Math., Texas State University--San Marcos, San Marcos, TX 78666}
       
\pagestyle{empty}


\begin{document}

\lstset{language=[Objective]Caml,frame=shadowbox}

\input{qm2pi.front}

% section front matter (end)

\input{qm2pi.intro} 
 
% section introduction (end)

% \input{qm2pi.knotations} 

% section notation (end)

\input{qm2pi.process.calculi} 

% section concurrent_process_calculi_and_spatial_logics_ (end)
    
%\input{qm2pi.knots2pi} 

%\input{qm2pi.trefoil} 

%\input{qm2pi.mainthm} 

% subsection basic_interpretation (end)

%\input{qm2pi.rho.presentation} 
\subsection{The syntax and semantics of the notation system}\label{sub:the_syntax_and_semantics_of_the_notation_system} % (fold)

We now summarize a technical presentation of the calculus that
embodies our theory of dynamics. The typical presentation of such a
calculus follows the style of giving generators and relations on
them. The grammar, below, describing term constructors, freely
generates the set of processes, $\Proc$. This set is then quotiented
by a relation known as structural congruence and it is over this set
that the notion of dynamics is expressed. This presentation is
essentially that of \cite{MeredithR05} with the addition of
polyadicity and summation. For readability we have relegated some of
the technical subtleties to an appendix.

\subsubsection{Process grammar}\label{subsub:process_grammar}

\begin{mathpar}
  \inferrule* [lab=synchronization] {} {{M} \bc \pzero \;|\; x?F \;|\; x!C }
  \and
  \inferrule* [lab=abstraction] {} {{F} \bc (x)P}
  \and
  \inferrule* [lab=concretion] {} {{C} \bc \langle Q \rangle}
  \and
  \inferrule* [lab=process] {} {{P,Q} \bc M \;| \;P|Q \;|\; @{x}}
  \and
  \inferrule* [lab=name] {} {{x} \bc \quotep{P}}
\end{mathpar} 

Note that $\vec{x}$ (resp. $\vec{P}$) denotes a vector of names
(resp. processes) of length $|\vec{x}|$ (resp. $|\vec{P}|$). We adopt
the following useful abbreviations.

\begin{mathpar}
   x?(\vec{y}).P := x.(\vec{y})P \and  x\clift{\vec{P}} := x.\clift{\vec{P}}
   \and x!(y) := \lift{x}{\dropn{y}}
   \and \Pi_{i=0}^{n-1}P_i := P_0 | \ldots | P_{n-1}
\end{mathpar}

\subsubsection{Structural congruence}

\paragraph{Free and bound names and alpha-equivalence.} At the
core of structural equivalence is alpha-equivalence which identifies
process that are the same up to a change of variable. Formally, we
recognize the distinction between free and bound names. The free names
of a process, $\freenames{P}$, may be calculated recursively as
follows:

\begin{mathpar}
\freenames{\pzero} := \emptyset
  \and \\
  \freenames{x?(y).P} := \{ x \} \cup (\freenames{P} \setminus \{ y \})
  \and 
  \freenames{x!\langle P \rangle} := \{ x \} \cup \{ P \} 
  \and \\
  \freenames{P|Q} := \freenames{P} \cup \freenames{Q}
  \and \\
  \freenames{@{x}} := \{ x \}
\end{mathpar}

$\pi$
$\quotep{\pi}$

$\freenames{-} : \pi \to \mathcal{P}(\quotep{\pi})$

\begin{eqnarray*}
  \freenames{\pzero} & := & \emptyset \\
  \freenames{x?(y).P} & := & \{ x \} \cup (\freenames{P} \setminus \{ y \}) \\
  \freenames{x!\langle P \rangle} & := & \{ x \} \cup \{ P \} \\
  \freenames{P|Q} & := & \freenames{P} \cup \freenames{Q} \\
  \freenames{\dropn{x}} & := & \{ x \}
\end{eqnarray*}

The bound names of a process, $\boundnames{P}$, are those names occurring in $P$
that are not free. For example, in $x?(y).0$, the name $x$ is free, while $y$ is bound.

\begin{mathpar}
  \inferrule* [lab=monoidal-laws] {} { P|Q \equiv Q|P \and P|0 \equiv P \and P|(Q|R) \equiv (P|Q)|R }
\end{mathpar}

\begin{mathpar}
  \inferrule* [lab=alpha-equivalence] {} { (x)P \equiv (y)P\{y/x\} \and y \not\in \freenames{P} }
\end{mathpar}

\begin{definition}
Then two processes, $P,Q$, are alpha-equivalent if $P = Q\{\vec{y}/\vec{x}\}$ for
some $\vec{x} \in \boundnames{Q},\vec{y} \in \boundnames{P}$, where $Q\{\vec{y}/\vec{x}\}$
denotes the capture-avoiding substitution of $\vec{y}$ for $\vec{x}$ in $Q$.
\end{definition}

\begin{definition}
  The {\em structural congruence} \cite{SangiorgiWalker} , $\equiv$,
  between processes is the least congruence containing
  alpha-equivalence, satisfying the abelian monoid laws
  (associativity, commutativity and $\pzero$ as identity) for parallel
  composition $|$ and for summation $+$.
\end{definition}

\subsection{Name equivalence}

We take name equivalence, written $\nameeq$, to be the smallest
equivalence relation generated by the following rules.

\begin{mathpar}
\inferrule*[lab=Quote-drop]
{ }
{ \quotep{@{x}} \nameeq x }

\inferrule*[lab=Struct-equiv]
{ P \scong Q }
{ \quotep{P} \nameeq \quotep{Q} }
\end{mathpar}

The astute reader will have noticed that the mutual recursion of names
and processes imposes a mutual recursion on alpha-equivalence and
structural equivalence via name-equivalence. Fortunately, all of this
works out pleasantly and we may calculate in the natural way, free of
concern. The reader interested in the details is referred to the
appendix \ref{appendix:rho_details}.

\subsection{Substitution}

We use $\Proc$ for the set of processes, $\QProc$ for the set of
names, and $\id{\{}\vec{y} / \vec{x} \id{\}}$ to denote partial maps,
$s : \QProc \rightarrow \QProc$. A map, $s$ lifts, uniquely, to a map
on process terms, $\widehat{s} : \Proc \rightarrow \Proc$ by the
following equations.

\begin{mathpar}
  (0) \psubstp{Q}{P} := 0 \\
  (R \juxtap S) \psubstp{Q}{P}
  :=    
  (R)\psubstp{Q}{P} \juxtap (S) \psubstp{Q}{P} \\
  (x?(y).R) \psubstp{Q}{P}    
  :=    
  (x)\substp{Q}{P} (z)\concat( (R \psubstn{z}{y}) \psubstp{Q}{P} ) \\
  (\lift{x}{R}) \psubstp{Q}{P}  
  :=
  \lift{(x)\substp{Q}{P}}{ R \psubstp{Q}{P} } \\
%   (\dropn{x})  \psubstp{Q}{P}       
%   := 
%   \left\{ 
%     \begin{array}{ccc} 
%       \dropn{\quotep{Q}} & & x \nameeq \quotep{P} \\
%       \dropn{x} & & otherwise \\
%     \end{array}
%   \right. 
  (\dropn{x})  \psubstp{Q}{P}       
  := 
  \left\{ 
    \begin{array}{ccc} 
      Q & & x \nameeq \quotep{P} \\
      \dropn{x} & & otherwise \\
    \end{array}
  \right.
\end{mathpar}
 

where

\begin{eqnarray}
  (x)\id{\{} \lpquote Q \rpquote / \lpquote P \rpquote \id{\}}            = 
  \left\{ 
    \begin{array}{ccc}
      \lpquote Q \rpquote & & x \nameeq \lpquote P \rpquote \\
      x & & otherwise \\
    \end{array}
  \right. \nonumber
\end{eqnarray}

and $z$ is chosen distinct from $\quotep{P}$, $\quotep{Q}$, the free
names in $Q$, and all the names in $R$. Our $\alpha$-equivalence will
be built in the standard way from this substitution.

\begin{remark}\label{rem:no_self_referential_names}
  One consequence of these definitions is that $\forall P. \quotep{P}
  \not\in \freenames{P}$.
\end{remark}

\subsection{ Dynamic quote: an example }

Anticipating something of what's to come, consider applying the
substitution, $\widehat{\id{\{}u / z \id{\}}}$, to the following pair
of processes, $\lift{w}{y!(z)}$ and $w[ \lpquote y!(z) \rpquote ]$.

\begin{eqnarray}
	\lift{w}{y!(z)}\widehat{\id{\{}u / z \id{\}}}
		& = &
		\lift{w}{y!(u)} \nonumber\\
	w[ \lpquote y!(z) \rpquote ] \widehat{ \id{\{}u / z \id{\}} }
		& = &
		w[ \lpquote y!(z) \rpquote ] \nonumber
\end{eqnarray}

Because the body of the process between quotes is impervious to
substitution, we get radically different answers. In fact, by
examining the first process in an input context,
e.g. $x?(z).\lift{w}{y!(z)}$, we see that the process under the lift
operator may be shaped by prefixed inputs binding a name inside it. In
this sense, the lift operator will be seen as a way to dynamically
construct processes before reifying them as names.

Finally equipped with these standard features we can present the
dynamics of the calculus.

\subsubsection{Operational semantics} 

Finally, we introduce the computational dynamics. What marks these
algebras as distinct from other more traditionally studied algebraic
structures, e.g. vector spaces or polynomial rings, is the manner in
which dynamics is captured. In traditional structures, dynamics is typically
expressed through morphisms between such structures, as in linear maps
between vector spaces or morphisms between rings. In algebras
associated with the semantics of computation, the dynamics is
expressed as part of the algebraic structure itself, through a
reduction reduction relation typically denoted by $\red$. Below, we
give a recursive presentation of this relation for the calculus used
in the encoding.

$\red \subseteq \pi \times \pi$
$\red : \pi \to \mathcal{P}(\pi)$

\begin{mathpar}
  \inferrule* [lab=Comm] { \textsf{match}( x_{src}, x_{trgt} ) } { x_{trgt}?(y)P \; | \; x_{src}!\langle {Q} \rangle \red P\{\quotep{Q}/y}\} }
  \and \\
  \inferrule* [lab=Par] {{P} \red {P}'} {{{P} | {Q}} \red {{P}' | {Q}}}
  \and
  \inferrule* [lab=Equiv]{{{P} \scong {P}'} \andalso {{P}' \red {Q}'} \andalso {{Q}' \scong {Q}}}{{P} \red {Q}}
\end{mathpar}

\begin{eqnarray*}
  match_{\equiv} (\quotep{P},\quotep{Q}) & := & P \equiv Q \\
  match_{\dagger}(\quotep{P},\quotep{Q}) & := & \forall R. P|Q \red^{*} R => R \red^{*} 0 \\
  match_{K}(\quotep{P},\quotep{Q}) & := & K \mbox{ for some context } K
\end{eqnarray*}

$u?(x)P | u!\langle Q \rangle \red P\{\quotep{Q}/x\}$

%We write $\wred$ for $\red^*$, and $P\red$ if $\exists Q $ such that $ P \red Q$.
We write $P\red$ if $\exists Q $ such that $ P \red Q$ and $P\not\red$, otherwise.

\section{Replication}

As mentioned before, it is known that replication (and hence
recursion) can be implemented in a higher-order process algebra
\cite{SangiorgiWalker}. As our first example of calculation with the
machinery thus far presented we give the construction explicitly in
the {\rhoc}.

\begin{eqnarray}
	D_{x} & := & \prefix{x}{y}{(\binpar{\outputp{x}{y}}{@{y}})} \nonumber\\
	\bangp_{x}{P} & := & \binpar{{x}!\langle{\binpar{D_{x}}{P}}\rangle}{D_{x}} \nonumber
\end{eqnarray}

\begin{eqnarray}
	\bangp_{x}{P} & & \nonumber\\
	=
	& {x}!\langle{(\prefix{x}{y}{(\outputp{x}{y} | @{y})) | P}}\rangle 
	      | \prefix{x}{y}{(\outputp{x}{y} | @{y})} & \nonumber\\
	\red
	& (\outputp{x}{y} | @{y})\substn{\quotep{(\prefix{x}{y}{(@{y} | \outputp{x}{y})) | P}}}{y} & \nonumber\\
	=
	& \outputp{x}{\quotep{(\prefix{x}{y}{(\outputp{x}{y} | @{y})) | P}}}
	  | {(\prefix{x}{y}{(\outputp{x}{y} | @{y})) | P}} & \nonumber\\
	\red
	& \ldots & \nonumber\\
	\red^*
	& P | P | \ldots & \nonumber
\end{eqnarray}

Of course, this encoding, as an implementation, runs away, unfolding
$\bangp{P}$ eagerly. A lazier and more implementable replication
operator, restricted to input-guarded processes, may be obtained as follows.

\begin{eqnarray}
\bangp{\prefix{u}{v}{P}} 
	:= 
	\binpar{\lift{x}{\prefix{u}{v}{(\binpar{D(x)}{P})}}}{D(x)} \nonumber
\end{eqnarray}

\begin{remark}
  Note that the lazier definition still does not deal with summation
  or mixed summation (i.e. sums over input and output). The reader is
  invited to construct definitions of replication that deal with these
  features. 

  Further, the definitions are parameterized in a name, $x$. Can you,
  gentle reader, make a definition that eliminates this parameter and
  guarantees no accidental interaction between the replication
  machinery and the process being replicated -- i.e. no accidental
  sharing of names used by the process to get its work done and the
  name(s) used by the replication to effect copying. This latter
  revision of the definition of replication is crucial to obtaining
  the expected identity $!!P \sim !P$.
\end{remark}

\begin{remark}\label{rem:paradoxical_combinator}
  The reader familiar with the lambda calculus will have noticed the
  similarity between $D$ and the paradoxical combinator.

  [Ed. note: the existence of this seems to suggest we have to be more
  restrictive on the set of processes and names we admit if we are to
  support no-cloning.]
\end{remark}

\subsubsection{Bisimulation}

The computational dynamics gives rise to another kind of equivalence,
the equivalence of computational behavior. As previously mentioned
this is typically captured \emph{via} some form of bisimulation.

% The notion we use in this paper is weak barbed bisimulation
% \cite{milner91polyadicpi}.

The notion we use in this paper is derived from weak barbed
bisimulation \cite{milner91polyadicpi}. 

\begin{definition}
An \emph{observation relation}, $\downarrow_{\mathcal N}$, over a set
of names, $\mathcal N$, is the smallest relation satisfying the rules
below.

\infrule[Out-barb]{y \in {\mathcal N}, \; x \nameeq y}
		  {\outputp{x}{v} \downarrow_{\mathcal N} x}
\infrule[Par-barb]{\mbox{$P\downarrow_{\mathcal N} x$ or $Q\downarrow_{\mathcal N} x$}}
		  {\binpar{P}{Q} \downarrow_{\mathcal N} x}

We write $P \Downarrow_{\mathcal N} x$ if there is $Q$ such that 
$P \wred Q$ and $Q \downarrow_{\mathcal N} x$.
\end{definition}

\begin{definition}
%\label{def.bbisim}
An  ${\mathcal N}$-\emph{barbed bisimulation} over a set of names, ${\mathcal N}$, is a symmetric binary relation 
${\mathcal S}_{\mathcal N}$ between agents such that $P\rel{S}_{\mathcal N}Q$ implies:
\begin{enumerate}
\item If $P \red P'$ then $Q \wred Q'$ and $P'\rel{S}_{\mathcal N} Q'$.
\item If $P\downarrow_{\mathcal N} x$, then $Q\Downarrow_{\mathcal N} x$.
\end{enumerate}
$P$ is ${\mathcal N}$-barbed bisimilar to $Q$, written
$P \wbbisim_{\mathcal N} Q$, if $P \rel{S}_{\mathcal N} Q$ for some ${\mathcal N}$-barbed bisimulation ${\mathcal S}_{\mathcal N}$.
\end{definition}

$\mathcal{R} \subseteq \pi \times \pi$

$P \mathcal{R} Q => \forall P'. P \red P' \Rightarrow \exists Q'. Q \red Q', P' \mathcal{R} Q'$

$P \vdash x \Rightarrow Q \vdash x$

\begin{mathpar}
  \inferrule*[lab=Out-barb]{x \nameeq y}{{y}!\langle{Q}\rangle \vdash x}
  \and
  \inferrule*[lab=Par-barb]{\mbox{$P\vdash x$ or $Q\vdash x$}}{\binpar{P}{Q} \vdash x}
\end{mathpar}

\subsubsection{Contexts}

One of the principle advantages of computational calculi like the
$\pi$-calculus is a well-defined notion of context,
contextual-equivalence and a correlation between
contextual-equivalence and notions of bisimulation. The notion of
context allows the decomposition of a process into (sub-)process and
its syntactic environment, its context. Thus, a context may be
thought of as a process with a ``hole'' (written $\Box$) in it. The
application of a context $M$ to a process $P$, written $M[P]$, is
tantamount to filling the hole in $M$ with $P$. In this paper we do
not need the full weight of this theory, but do make use of the notion
of context in the proof the main theorem. 

\begin{mathpar}
  \inferrule* [lab=summation] {} {{M_{M},M_{N}} \bc \Box \;|\; x.M_{A} \;|\; M_{M}+M_{N}}
  \and
  \inferrule* [lab=agent] {} {{M_{A}} \bc (\vec{x})M_{P} \;| \; \clift{P_0,\ldots,M_{P},\ldots,P_N}}
  \and \\
  \inferrule* [lab=process] {} {{M_{P}} \bc M_{N} \;| \;P|M_{P} }
\end{mathpar} 

\begin{mathpar}
  \inferrule* [lab=sychronization] {} {M_{N} \bc \Box \;|\; x?M_{F} \;|\; x!M_{C}}
  \and
  \inferrule* [lab=abstraction] {} {{M_{F}} \bc (x)M_{P} }
  \and
  \inferrule* [lab=concretion] {} {{M_{C}} \bc \langle M_{P} \rangle }
  \and \\
  \inferrule* [lab=process] {} {{M_{P}} \bc M_{N} \;| \;P|M_{P} }
\end{mathpar}

\begin{definition}[contextual application] Given a context $M$, and
  process $P$, we define the \emph{contextual application}, $M[P] :=
  M\{P/\Box\}$. That is, the contextual application of M to P is the
  substitution of $P$ for $\Box$ in $M$.
\end{definition}

$\meaningof{-} : L \to \mathcal{P}(\pi)$

\begin{mathpar}
  \inferrule* [lab=collection] {} {\meaningof{true} = \pi, \and \meaningof{~E} = \pi \setminus \meaningof{E}, \and \meaningof{E_{1} \& E_{2}} = \meaningof{E_{1}} \cap \meaningof{E_{2}}}
\end{mathpar}

\begin{mathpar}
  \inferrule* [lab=structure] {} {\meaningof{0} = \{ P \in \pi | P \equiv 0 \}, \and \\ \meaningof{E_1 | E_2} = \{ P \in \pi | P \equiv P_{1} | P_{2}, P_{1} \in \meaningof{E_{1}}, P_{2} \in \meaningof{E_2}\} }
\end{mathpar}

\begin{mathpar}
 \inferrule* [lab=behavior] {} {\meaningof{\langle a?b \rangle E} = \{ P \in \pi | P \equiv Q | u?(y)P', \\ \and \\\\ \and \\ \;\;\; u \in \meaningof{a}, \forall z.P'\{z/y\} \in \meaningof{E\{z/b\}}\}, \and \\ \meaningof{a!E} = \{ P \in \pi | P \equiv Q | x!\langle P' \rangle, x \in \meaningof{a} P' \in \meaningof{E}\} }
\end{mathpar}

\begin{mathpar}
 \inferrule* [lab=nominal] {} {\meaningof{\quotep{E}} = \{ \quotep{P} \in \quotep{\pi} | P \in \meaningof{E} \}, \and \meaningof{\quotep{P}} = \{ \quotep{Q} \in \quotep{\pi} | P \equiv Q \} \and \\ \meaningof{@\quotep{E}} = \{ P \in \pi | P \equiv @x, x \in \meaningof{E} \}}
\end{mathpar}

\begin{eqnarray*}
  \\
  \meaningof{-} : TS \to ST
\end{eqnarray*}

\begin{eqnarray*}
  \\
  L : TS \to ST
\end{eqnarray*}

\begin{eqnarray*}
  \\
  P \models E \iff P \in \meaningof{E}
\end{eqnarray*}

\begin{eqnarray*}
  P \approx_{L} Q \iff \forall E \in L. P \models E \iff Q \models E
\end{eqnarray*}

\begin{eqnarray*}
  P \approx_{K} Q
\end{eqnarray*}

\begin{eqnarray*}
  P \approx Q
\end{eqnarray*}

$\approx_{K} = \approx = \approx_{L}$

\subsubsection{Contextual duality}

Note that contexts extend the quotation operation to a family of
operations from processes to names. Given a context, $M$, we can
define a \emph{nominal context}, $\quotep{M}$ by $\quotep{M}[P] :=
\quotep{M[P]}$. To foreshadow what is to come we observe that these
operations enjoy a duality with processes very much like the duality
between vectors and maps from vectors to scalars.

Further, because the calculus is essentially higher-order, we have a
correspondence between contexts and processes. More specifically,
given a name $x$ and a context $M$ we can construct $M^{*}_{x}$ such
that 

\begin{mathpar}
  M^{*}_{x} | \lift{x}{P} \red M[P]
\end{mathpar}

namely,

\begin{mathpar}
  M^{*}_{x} := x?(u).M[\dropn{u}]
\end{mathpar}

The dependence of $M^{*}_{x}$ on a name makes it an abstraction, 

\begin{mathpar}
  M^{*} := (x)x?(u).M[\dropn{u}]
\end{mathpar}

\subsection{Additional notation}

It will sometimes be convenient to denote the process a name
quotes. We already have the notation $x = \quotep{P}$, but it will be
convenient to introduce an alternate notation, $\procn{x}$, when we
want to emphasize the connection to the use of the name. Note that, by
virtue of name equivalence, $\quotep{\procn{x}} \nameeq x$; so, the
notation is consistent with previous definitions.

Further, because names have structure it is possible to effect
substitutions on the basis of that structure. This means we need to
upgrade our notation for substitutions, which we accomplish by
adapting comprehension notation. Thus,

\begin{mathpar}
  P\{ y / x : x \in S \}
\end{mathpar}

is interpreted to mean the process derived from P by replacing (in a
capture-avoiding manner) each occurrence of $x$ in $S$ by $y$. For example,

\begin{mathpar}
  P\{ \quotep{\procn{x}|\procn{x}} / x : x \in \freenames{P} \}
\end{mathpar}

will replace each (occurrence) of a free name $x$ in $P$ by
$\quotep{\procn{x}|\procn{x}}$.

Also, we will avail ourselves of the notation $x^{L}$ and $x^{R}$ to
denote injections of a name into disjoint copies of the name
space. There are numerous ways to accomplish this. One example can be
found in \cite{MeredithR05}. This notation overloads to vectors of
names: $\vec{x}^{\pi} := (x_{i}^{\pi} \; : \; 0 \leq i < |\vec{x}| )$ where $\pi \in \{L,R\}$.

We also use $P^{\Box} := P|\Box$.

In \cite{MeredithR05} an interpretation of the new operator is
given. It turns out that there are several possible interpretations
all enjoying the requisite algebraic properties of the operator (see
\cite{milner91polyadicpi}). We will therefore make liberal use of
$(\nu\; \vec{x})P$.

% subsection the_syntax_and_semantics_of_the_notation_system (end)   

\input{qm2pi.qmops} 

\input{qm2pi.sterngerlach} 

\input{qm2pi.metric} 

% section concurrent_process_calculi (end)

%\input{qm2pi.proofsketch}

% section proof sketch (end)

%\input{qm2pi.slviaknots} 

% section spatial logic via knots (end)

\input{qm2pi.conclusion}

% section conclusion (end)

%\input{qm2pi.dtcodes} 

% section wiring algorithm (end)

\input{qm2pi.ack} 

% section acknowledgments (end)

\newpage


\bibliographystyle{plain}   
\bibliography{../../biblios/main.bib}

\input{qm2pi.rhodetails}

\end{document}

 

% subsection basic_interpretation (end)

%\input{qm2pi.rho.presentation} 
\subsection{The syntax and semantics of the notation system}\label{sub:the_syntax_and_semantics_of_the_notation_system} % (fold)

We now summarize a technical presentation of the calculus that
embodies our theory of dynamics. The typical presentation of such a
calculus follows the style of giving generators and relations on
them. The grammar, below, describing term constructors, freely
generates the set of processes, $\Proc$. This set is then quotiented
by a relation known as structural congruence and it is over this set
that the notion of dynamics is expressed. This presentation is
essentially that of \cite{MeredithR05} with the addition of
polyadicity and summation. For readability we have relegated some of
the technical subtleties to an appendix.

\subsubsection{Process grammar}\label{subsub:process_grammar}

\begin{mathpar}
  \inferrule* [lab=synchronization] {} {{M} \bc \pzero \;|\; x?F \;|\; x!C }
  \and
  \inferrule* [lab=abstraction] {} {{F} \bc (x)P}
  \and
  \inferrule* [lab=concretion] {} {{C} \bc \langle Q \rangle}
  \and
  \inferrule* [lab=process] {} {{P,Q} \bc M \;| \;P|Q \;|\; @{x}}
  \and
  \inferrule* [lab=name] {} {{x} \bc \quotep{P}}
\end{mathpar} 

Note that $\vec{x}$ (resp. $\vec{P}$) denotes a vector of names
(resp. processes) of length $|\vec{x}|$ (resp. $|\vec{P}|$). We adopt
the following useful abbreviations.

\begin{mathpar}
   x?(\vec{y}).P := x.(\vec{y})P \and  x\clift{\vec{P}} := x.\clift{\vec{P}}
   \and x!(y) := \lift{x}{\dropn{y}}
   \and \Pi_{i=0}^{n-1}P_i := P_0 | \ldots | P_{n-1}
\end{mathpar}

\subsubsection{Structural congruence}

\paragraph{Free and bound names and alpha-equivalence.} At the
core of structural equivalence is alpha-equivalence which identifies
process that are the same up to a change of variable. Formally, we
recognize the distinction between free and bound names. The free names
of a process, $\freenames{P}$, may be calculated recursively as
follows:

\begin{mathpar}
\freenames{\pzero} := \emptyset
  \and \\
  \freenames{x?(y).P} := \{ x \} \cup (\freenames{P} \setminus \{ y \})
  \and 
  \freenames{x!\langle P \rangle} := \{ x \} \cup \{ P \} 
  \and \\
  \freenames{P|Q} := \freenames{P} \cup \freenames{Q}
  \and \\
  \freenames{@{x}} := \{ x \}
\end{mathpar}

$\pi$
$\quotep{\pi}$

$\freenames{-} : \pi \to \mathcal{P}(\quotep{\pi})$

\begin{eqnarray*}
  \freenames{\pzero} & := & \emptyset \\
  \freenames{x?(y).P} & := & \{ x \} \cup (\freenames{P} \setminus \{ y \}) \\
  \freenames{x!\langle P \rangle} & := & \{ x \} \cup \{ P \} \\
  \freenames{P|Q} & := & \freenames{P} \cup \freenames{Q} \\
  \freenames{\dropn{x}} & := & \{ x \}
\end{eqnarray*}

The bound names of a process, $\boundnames{P}$, are those names occurring in $P$
that are not free. For example, in $x?(y).0$, the name $x$ is free, while $y$ is bound.

\begin{mathpar}
  \inferrule* [lab=monoidal-laws] {} { P|Q \equiv Q|P \and P|0 \equiv P \and P|(Q|R) \equiv (P|Q)|R }
\end{mathpar}

\begin{mathpar}
  \inferrule* [lab=alpha-equivalence] {} { (x)P \equiv (y)P\{y/x\} \and y \not\in \freenames{P} }
\end{mathpar}

\begin{definition}
Then two processes, $P,Q$, are alpha-equivalent if $P = Q\{\vec{y}/\vec{x}\}$ for
some $\vec{x} \in \boundnames{Q},\vec{y} \in \boundnames{P}$, where $Q\{\vec{y}/\vec{x}\}$
denotes the capture-avoiding substitution of $\vec{y}$ for $\vec{x}$ in $Q$.
\end{definition}

\begin{definition}
  The {\em structural congruence} \cite{SangiorgiWalker} , $\equiv$,
  between processes is the least congruence containing
  alpha-equivalence, satisfying the abelian monoid laws
  (associativity, commutativity and $\pzero$ as identity) for parallel
  composition $|$ and for summation $+$.
\end{definition}

\subsection{Name equivalence}

We take name equivalence, written $\nameeq$, to be the smallest
equivalence relation generated by the following rules.

\begin{mathpar}
\inferrule*[lab=Quote-drop]
{ }
{ \quotep{@{x}} \nameeq x }

\inferrule*[lab=Struct-equiv]
{ P \scong Q }
{ \quotep{P} \nameeq \quotep{Q} }
\end{mathpar}

The astute reader will have noticed that the mutual recursion of names
and processes imposes a mutual recursion on alpha-equivalence and
structural equivalence via name-equivalence. Fortunately, all of this
works out pleasantly and we may calculate in the natural way, free of
concern. The reader interested in the details is referred to the
appendix \ref{appendix:rho_details}.

\subsection{Substitution}

We use $\Proc$ for the set of processes, $\QProc$ for the set of
names, and $\id{\{}\vec{y} / \vec{x} \id{\}}$ to denote partial maps,
$s : \QProc \rightarrow \QProc$. A map, $s$ lifts, uniquely, to a map
on process terms, $\widehat{s} : \Proc \rightarrow \Proc$ by the
following equations.

\begin{mathpar}
  (0) \psubstp{Q}{P} := 0 \\
  (R \juxtap S) \psubstp{Q}{P}
  :=    
  (R)\psubstp{Q}{P} \juxtap (S) \psubstp{Q}{P} \\
  (x?(y).R) \psubstp{Q}{P}    
  :=    
  (x)\substp{Q}{P} (z)\concat( (R \psubstn{z}{y}) \psubstp{Q}{P} ) \\
  (\lift{x}{R}) \psubstp{Q}{P}  
  :=
  \lift{(x)\substp{Q}{P}}{ R \psubstp{Q}{P} } \\
%   (\dropn{x})  \psubstp{Q}{P}       
%   := 
%   \left\{ 
%     \begin{array}{ccc} 
%       \dropn{\quotep{Q}} & & x \nameeq \quotep{P} \\
%       \dropn{x} & & otherwise \\
%     \end{array}
%   \right. 
  (\dropn{x})  \psubstp{Q}{P}       
  := 
  \left\{ 
    \begin{array}{ccc} 
      Q & & x \nameeq \quotep{P} \\
      \dropn{x} & & otherwise \\
    \end{array}
  \right.
\end{mathpar}
 

where

\begin{eqnarray}
  (x)\id{\{} \lpquote Q \rpquote / \lpquote P \rpquote \id{\}}            = 
  \left\{ 
    \begin{array}{ccc}
      \lpquote Q \rpquote & & x \nameeq \lpquote P \rpquote \\
      x & & otherwise \\
    \end{array}
  \right. \nonumber
\end{eqnarray}

and $z$ is chosen distinct from $\quotep{P}$, $\quotep{Q}$, the free
names in $Q$, and all the names in $R$. Our $\alpha$-equivalence will
be built in the standard way from this substitution.

\begin{remark}\label{rem:no_self_referential_names}
  One consequence of these definitions is that $\forall P. \quotep{P}
  \not\in \freenames{P}$.
\end{remark}

\subsection{ Dynamic quote: an example }

Anticipating something of what's to come, consider applying the
substitution, $\widehat{\id{\{}u / z \id{\}}}$, to the following pair
of processes, $\lift{w}{y!(z)}$ and $w[ \lpquote y!(z) \rpquote ]$.

\begin{eqnarray}
	\lift{w}{y!(z)}\widehat{\id{\{}u / z \id{\}}}
		& = &
		\lift{w}{y!(u)} \nonumber\\
	w[ \lpquote y!(z) \rpquote ] \widehat{ \id{\{}u / z \id{\}} }
		& = &
		w[ \lpquote y!(z) \rpquote ] \nonumber
\end{eqnarray}

Because the body of the process between quotes is impervious to
substitution, we get radically different answers. In fact, by
examining the first process in an input context,
e.g. $x?(z).\lift{w}{y!(z)}$, we see that the process under the lift
operator may be shaped by prefixed inputs binding a name inside it. In
this sense, the lift operator will be seen as a way to dynamically
construct processes before reifying them as names.

Finally equipped with these standard features we can present the
dynamics of the calculus.

\subsubsection{Operational semantics} 

Finally, we introduce the computational dynamics. What marks these
algebras as distinct from other more traditionally studied algebraic
structures, e.g. vector spaces or polynomial rings, is the manner in
which dynamics is captured. In traditional structures, dynamics is typically
expressed through morphisms between such structures, as in linear maps
between vector spaces or morphisms between rings. In algebras
associated with the semantics of computation, the dynamics is
expressed as part of the algebraic structure itself, through a
reduction reduction relation typically denoted by $\red$. Below, we
give a recursive presentation of this relation for the calculus used
in the encoding.

$\red \subseteq \pi \times \pi$
$\red : \pi \to \mathcal{P}(\pi)$

\begin{mathpar}
  \inferrule* [lab=Comm] { \textsf{match}( x_{src}, x_{trgt} ) } { x_{trgt}?(y)P \; | \; x_{src}!\langle {Q} \rangle \red P\{\quotep{Q}/y}\} }
  \and \\
  \inferrule* [lab=Par] {{P} \red {P}'} {{{P} | {Q}} \red {{P}' | {Q}}}
  \and
  \inferrule* [lab=Equiv]{{{P} \scong {P}'} \andalso {{P}' \red {Q}'} \andalso {{Q}' \scong {Q}}}{{P} \red {Q}}
\end{mathpar}

\begin{eqnarray*}
  match_{\equiv} (\quotep{P},\quotep{Q}) & := & P \equiv Q \\
  match_{\dagger}(\quotep{P},\quotep{Q}) & := & \forall R. P|Q \red^{*} R => R \red^{*} 0 \\
  match_{K}(\quotep{P},\quotep{Q}) & := & K \mbox{ for some context } K
\end{eqnarray*}

$u?(x)P | u!\langle Q \rangle \red P\{\quotep{Q}/x\}$

%We write $\wred$ for $\red^*$, and $P\red$ if $\exists Q $ such that $ P \red Q$.
We write $P\red$ if $\exists Q $ such that $ P \red Q$ and $P\not\red$, otherwise.

\section{Replication}

As mentioned before, it is known that replication (and hence
recursion) can be implemented in a higher-order process algebra
\cite{SangiorgiWalker}. As our first example of calculation with the
machinery thus far presented we give the construction explicitly in
the {\rhoc}.

\begin{eqnarray}
	D_{x} & := & \prefix{x}{y}{(\binpar{\outputp{x}{y}}{@{y}})} \nonumber\\
	\bangp_{x}{P} & := & \binpar{{x}!\langle{\binpar{D_{x}}{P}}\rangle}{D_{x}} \nonumber
\end{eqnarray}

\begin{eqnarray}
	\bangp_{x}{P} & & \nonumber\\
	=
	& {x}!\langle{(\prefix{x}{y}{(\outputp{x}{y} | @{y})) | P}}\rangle 
	      | \prefix{x}{y}{(\outputp{x}{y} | @{y})} & \nonumber\\
	\red
	& (\outputp{x}{y} | @{y})\substn{\quotep{(\prefix{x}{y}{(@{y} | \outputp{x}{y})) | P}}}{y} & \nonumber\\
	=
	& \outputp{x}{\quotep{(\prefix{x}{y}{(\outputp{x}{y} | @{y})) | P}}}
	  | {(\prefix{x}{y}{(\outputp{x}{y} | @{y})) | P}} & \nonumber\\
	\red
	& \ldots & \nonumber\\
	\red^*
	& P | P | \ldots & \nonumber
\end{eqnarray}

Of course, this encoding, as an implementation, runs away, unfolding
$\bangp{P}$ eagerly. A lazier and more implementable replication
operator, restricted to input-guarded processes, may be obtained as follows.

\begin{eqnarray}
\bangp{\prefix{u}{v}{P}} 
	:= 
	\binpar{\lift{x}{\prefix{u}{v}{(\binpar{D(x)}{P})}}}{D(x)} \nonumber
\end{eqnarray}

\begin{remark}
  Note that the lazier definition still does not deal with summation
  or mixed summation (i.e. sums over input and output). The reader is
  invited to construct definitions of replication that deal with these
  features. 

  Further, the definitions are parameterized in a name, $x$. Can you,
  gentle reader, make a definition that eliminates this parameter and
  guarantees no accidental interaction between the replication
  machinery and the process being replicated -- i.e. no accidental
  sharing of names used by the process to get its work done and the
  name(s) used by the replication to effect copying. This latter
  revision of the definition of replication is crucial to obtaining
  the expected identity $!!P \sim !P$.
\end{remark}

\begin{remark}\label{rem:paradoxical_combinator}
  The reader familiar with the lambda calculus will have noticed the
  similarity between $D$ and the paradoxical combinator.

  [Ed. note: the existence of this seems to suggest we have to be more
  restrictive on the set of processes and names we admit if we are to
  support no-cloning.]
\end{remark}

\subsubsection{Bisimulation}

The computational dynamics gives rise to another kind of equivalence,
the equivalence of computational behavior. As previously mentioned
this is typically captured \emph{via} some form of bisimulation.

% The notion we use in this paper is weak barbed bisimulation
% \cite{milner91polyadicpi}.

The notion we use in this paper is derived from weak barbed
bisimulation \cite{milner91polyadicpi}. 

\begin{definition}
An \emph{observation relation}, $\downarrow_{\mathcal N}$, over a set
of names, $\mathcal N$, is the smallest relation satisfying the rules
below.

\infrule[Out-barb]{y \in {\mathcal N}, \; x \nameeq y}
		  {\outputp{x}{v} \downarrow_{\mathcal N} x}
\infrule[Par-barb]{\mbox{$P\downarrow_{\mathcal N} x$ or $Q\downarrow_{\mathcal N} x$}}
		  {\binpar{P}{Q} \downarrow_{\mathcal N} x}

We write $P \Downarrow_{\mathcal N} x$ if there is $Q$ such that 
$P \wred Q$ and $Q \downarrow_{\mathcal N} x$.
\end{definition}

\begin{definition}
%\label{def.bbisim}
An  ${\mathcal N}$-\emph{barbed bisimulation} over a set of names, ${\mathcal N}$, is a symmetric binary relation 
${\mathcal S}_{\mathcal N}$ between agents such that $P\rel{S}_{\mathcal N}Q$ implies:
\begin{enumerate}
\item If $P \red P'$ then $Q \wred Q'$ and $P'\rel{S}_{\mathcal N} Q'$.
\item If $P\downarrow_{\mathcal N} x$, then $Q\Downarrow_{\mathcal N} x$.
\end{enumerate}
$P$ is ${\mathcal N}$-barbed bisimilar to $Q$, written
$P \wbbisim_{\mathcal N} Q$, if $P \rel{S}_{\mathcal N} Q$ for some ${\mathcal N}$-barbed bisimulation ${\mathcal S}_{\mathcal N}$.
\end{definition}

$\mathcal{R} \subseteq \pi \times \pi$

$P \mathcal{R} Q => \forall P'. P \red P' \Rightarrow \exists Q'. Q \red Q', P' \mathcal{R} Q'$

$P \vdash x \Rightarrow Q \vdash x$

\begin{mathpar}
  \inferrule*[lab=Out-barb]{x \nameeq y}{{y}!\langle{Q}\rangle \vdash x}
  \and
  \inferrule*[lab=Par-barb]{\mbox{$P\vdash x$ or $Q\vdash x$}}{\binpar{P}{Q} \vdash x}
\end{mathpar}

\subsubsection{Contexts}

One of the principle advantages of computational calculi like the
$\pi$-calculus is a well-defined notion of context,
contextual-equivalence and a correlation between
contextual-equivalence and notions of bisimulation. The notion of
context allows the decomposition of a process into (sub-)process and
its syntactic environment, its context. Thus, a context may be
thought of as a process with a ``hole'' (written $\Box$) in it. The
application of a context $M$ to a process $P$, written $M[P]$, is
tantamount to filling the hole in $M$ with $P$. In this paper we do
not need the full weight of this theory, but do make use of the notion
of context in the proof the main theorem. 

\begin{mathpar}
  \inferrule* [lab=summation] {} {{M_{M},M_{N}} \bc \Box \;|\; x.M_{A} \;|\; M_{M}+M_{N}}
  \and
  \inferrule* [lab=agent] {} {{M_{A}} \bc (\vec{x})M_{P} \;| \; \clift{P_0,\ldots,M_{P},\ldots,P_N}}
  \and \\
  \inferrule* [lab=process] {} {{M_{P}} \bc M_{N} \;| \;P|M_{P} }
\end{mathpar} 

\begin{mathpar}
  \inferrule* [lab=sychronization] {} {M_{N} \bc \Box \;|\; x?M_{F} \;|\; x!M_{C}}
  \and
  \inferrule* [lab=abstraction] {} {{M_{F}} \bc (x)M_{P} }
  \and
  \inferrule* [lab=concretion] {} {{M_{C}} \bc \langle M_{P} \rangle }
  \and \\
  \inferrule* [lab=process] {} {{M_{P}} \bc M_{N} \;| \;P|M_{P} }
\end{mathpar}

\begin{definition}[contextual application] Given a context $M$, and
  process $P$, we define the \emph{contextual application}, $M[P] :=
  M\{P/\Box\}$. That is, the contextual application of M to P is the
  substitution of $P$ for $\Box$ in $M$.
\end{definition}

$\meaningof{-} : L \to \mathcal{P}(\pi)$

\begin{mathpar}
  \inferrule* [lab=collection] {} {\meaningof{true} = \pi, \and \meaningof{~E} = \pi \setminus \meaningof{E}, \and \meaningof{E_{1} \& E_{2}} = \meaningof{E_{1}} \cap \meaningof{E_{2}}}
\end{mathpar}

\begin{mathpar}
  \inferrule* [lab=structure] {} {\meaningof{0} = \{ P \in \pi | P \equiv 0 \}, \and \\ \meaningof{E_1 | E_2} = \{ P \in \pi | P \equiv P_{1} | P_{2}, P_{1} \in \meaningof{E_{1}}, P_{2} \in \meaningof{E_2}\} }
\end{mathpar}

\begin{mathpar}
 \inferrule* [lab=behavior] {} {\meaningof{\langle a?b \rangle E} = \{ P \in \pi | P \equiv Q | u?(y)P', \\ \and \\\\ \and \\ \;\;\; u \in \meaningof{a}, \forall z.P'\{z/y\} \in \meaningof{E\{z/b\}}\}, \and \\ \meaningof{a!E} = \{ P \in \pi | P \equiv Q | x!\langle P' \rangle, x \in \meaningof{a} P' \in \meaningof{E}\} }
\end{mathpar}

\begin{mathpar}
 \inferrule* [lab=nominal] {} {\meaningof{\quotep{E}} = \{ \quotep{P} \in \quotep{\pi} | P \in \meaningof{E} \}, \and \meaningof{\quotep{P}} = \{ \quotep{Q} \in \quotep{\pi} | P \equiv Q \} \and \\ \meaningof{@\quotep{E}} = \{ P \in \pi | P \equiv @x, x \in \meaningof{E} \}}
\end{mathpar}

\begin{eqnarray*}
  \\
  \meaningof{-} : TS \to ST
\end{eqnarray*}

\begin{eqnarray*}
  \\
  L : TS \to ST
\end{eqnarray*}

\begin{eqnarray*}
  \\
  P \models E \iff P \in \meaningof{E}
\end{eqnarray*}

\begin{eqnarray*}
  P \approx_{L} Q \iff \forall E \in L. P \models E \iff Q \models E
\end{eqnarray*}

\begin{eqnarray*}
  P \approx_{K} Q
\end{eqnarray*}

\begin{eqnarray*}
  P \approx Q
\end{eqnarray*}

$\approx_{K} = \approx = \approx_{L}$

\subsubsection{Contextual duality}

Note that contexts extend the quotation operation to a family of
operations from processes to names. Given a context, $M$, we can
define a \emph{nominal context}, $\quotep{M}$ by $\quotep{M}[P] :=
\quotep{M[P]}$. To foreshadow what is to come we observe that these
operations enjoy a duality with processes very much like the duality
between vectors and maps from vectors to scalars.

Further, because the calculus is essentially higher-order, we have a
correspondence between contexts and processes. More specifically,
given a name $x$ and a context $M$ we can construct $M^{*}_{x}$ such
that 

\begin{mathpar}
  M^{*}_{x} | \lift{x}{P} \red M[P]
\end{mathpar}

namely,

\begin{mathpar}
  M^{*}_{x} := x?(u).M[\dropn{u}]
\end{mathpar}

The dependence of $M^{*}_{x}$ on a name makes it an abstraction, 

\begin{mathpar}
  M^{*} := (x)x?(u).M[\dropn{u}]
\end{mathpar}

\subsection{Additional notation}

It will sometimes be convenient to denote the process a name
quotes. We already have the notation $x = \quotep{P}$, but it will be
convenient to introduce an alternate notation, $\procn{x}$, when we
want to emphasize the connection to the use of the name. Note that, by
virtue of name equivalence, $\quotep{\procn{x}} \nameeq x$; so, the
notation is consistent with previous definitions.

Further, because names have structure it is possible to effect
substitutions on the basis of that structure. This means we need to
upgrade our notation for substitutions, which we accomplish by
adapting comprehension notation. Thus,

\begin{mathpar}
  P\{ y / x : x \in S \}
\end{mathpar}

is interpreted to mean the process derived from P by replacing (in a
capture-avoiding manner) each occurrence of $x$ in $S$ by $y$. For example,

\begin{mathpar}
  P\{ \quotep{\procn{x}|\procn{x}} / x : x \in \freenames{P} \}
\end{mathpar}

will replace each (occurrence) of a free name $x$ in $P$ by
$\quotep{\procn{x}|\procn{x}}$.

Also, we will avail ourselves of the notation $x^{L}$ and $x^{R}$ to
denote injections of a name into disjoint copies of the name
space. There are numerous ways to accomplish this. One example can be
found in \cite{MeredithR05}. This notation overloads to vectors of
names: $\vec{x}^{\pi} := (x_{i}^{\pi} \; : \; 0 \leq i < |\vec{x}| )$ where $\pi \in \{L,R\}$.

We also use $P^{\Box} := P|\Box$.

In \cite{MeredithR05} an interpretation of the new operator is
given. It turns out that there are several possible interpretations
all enjoying the requisite algebraic properties of the operator (see
\cite{milner91polyadicpi}). We will therefore make liberal use of
$(\nu\; \vec{x})P$.

% subsection the_syntax_and_semantics_of_the_notation_system (end)   

\section{Interpretation of QM}
\subsection{Supporting definitions}
\subsubsection{Multiplication}
\begin{mathpar}
  \quotep{Q} \cdot \quotep{R} := \quotep{Q|R}
  \and \\
  \quotep{Q} \cdot P := P\{ \quotep{Q|R} / \quotep{R} : \quotep{R} \in \freenames{P} \}
\end{mathpar}

\paragraph{Discussion}
The first line needs little explanation. The second line says that
each free name of the process is replaced with the multiplication of
that name by the scalar. Multiplication of a scalar (name) by a state
(process) results in a process all the names of which have been `moved
over' by parallel composition with the process the scalar
quotes. There is a subtlety that the bound names have to be
manipulated so that multiplied names aren't accidentally
captured. There are many ways to achieve this.

\begin{remark}\label{rem:multiplication_identities}
  The reader is invited to verify that for all $x,y,z \in \QProc$ and $P \in \Proc$
  \begin{mathpar}
    x \cdot \quotep{0} \equiv x 
    \and
    x \cdot y \equiv y \cdot x
    \and
    x \cdot (y \cdot z) \equiv (x \cdot y) \cdot z
    \and \\
    \quotep{0} \cdot P \equiv P
    \and \\
    x \cdot (y \cdot P) \equiv (x \cdot y) \cdot P
    \and \\
    x \cdot (P|Q) \equiv (x \cdot P) | (x \cdot Q)
    \and \\    
  \end{mathpar}
\end{remark}

\subsubsection{Tensor product}

We define a tensor product on processes by structural induction.

\paragraph{Tensor of sums} First note that all summations, including
$\pzero$ and sequence, can be written $\Sigma_{i} x_{i}.A_{i} +
\Sigma_{j} x_{j}.C_{j}$, where we have grouped input-guarded processes
together and output-guarded processes together.

Thus, we can define the tensor product of two summations, $N_{1}\otimes N_{2}$, where

\begin{mathpar}
  N_{1} := \Sigma_{i} x_{i}.A_{i} + \Sigma_{j} x_{j}.C_{j}
  \and
  N_{2} := \Sigma_{i'} y_{i'}.B_{i'} + \Sigma_{j'} y_{j'}.D_{j'} 
\end{mathpar}

as follows.

\begin{mathpar}
  \Sigma_{i} x_{i}.A_{i} + \Sigma_{j} x_{j}.C_{j} \otimes \Sigma_{i'}
  y_{i'}.B_{i'} + \Sigma_{j'} y_{j'}.D_{j'} 
  \and \\
  := \; \Sigma_{i} \Sigma_{i'} \quotep{\stackrel{\vee}{x_{i}}| \stackrel{\vee}{y_{i'}}}.(A_{i}\otimes B_{i'}) \; | \; \Sigma_{i'} \Sigma_{i} \quotep{\stackrel{\vee}{y_{i'}}|\stackrel{\vee}{x_{i}}}.(B_{i'}\otimes A_{i})
  \and
  \;\; | \;\; \Sigma_{j} \Sigma_{j'} \quotep{\stackrel{\vee}{x_{j}}|\stackrel{\vee}{y_{j'}}}.(A_{j}\otimes B_{j'}) \; | \; \Sigma_{j'} \Sigma_{j} \quotep{\stackrel{\vee}{y_{j'}}|\stackrel{\vee}{x_{j}}}.(B_{j'}\otimes A_{j})
\end{mathpar}

\begin{remark}
  Do we need to $x^{L}$ and $y^{R}$ for this construction as well?
\end{remark}

\paragraph{Tensor of parallel compositions} Next, we distribute tensor
over par.

\begin{mathpar}
  P_{1}|P_{2} \otimes Q_{1}|Q_{2} := (P_{1} \otimes Q_{1}) | (P_{1}
  \otimes Q_{2}) | (P_{2} \otimes Q_{1}) | (P_{2} \otimes Q_{2})
\end{mathpar}

\paragraph{Tensor with dropped names} We treat tensor of a
process with a dropped name as parallel composition.

\begin{mathpar}
  P \otimes \dropn{x} := P | \dropn{x}
\end{mathpar}

\paragraph{Tensor of agents}

Finally, we need to define tensor on agents. Note that the definition
of tensor on normal products only tensors inputs with inputs and
outputs with outputs. Thus, we only have to define the operation on
``homogeneous'' pairings.

\begin{mathpar}
  (\vec{x})P \otimes (\vec{y})Q
  \and \\
  := (x_{0}^{L}|y_{0}^{R},\ldots,x_{0}^{L}|y_{n}^{R},\ldots,x_{m}^{L}|y_{0}^{R},\ldots,x_{m}^{L}|y_{n}^R)(P\{ \vec{x}^{L}/\vec{x}\} \otimes Q \{ \vec{y}^{R}/\vec{y}\})
  \and \\
  \clift{\vec{P}} \otimes \clift{\vec{Q}}
  \and \\
  := \clift{P_{0}\otimes Q_{0},\ldots,P_{0}\otimes Q_{n},\ldots,P_{m}\otimes Q_{0},\ldots,P_{m}\otimes Q_{n}}
\end{mathpar}

\begin{remark}
  Observe that arities of tensored abstractions matches arities of
  tensored concretions if the original arities matched. Note also that
  the length of the arities corresponds to the increase in dimension
  we see in ordinary vector space tensor product.
\end{remark}

\begin{remark}
  Operationally, this definition distributes the tensor down to
  components ``linked'' by summation. Tensor over summation is
  intriguing in that it mixes names. Moreover, as a consequence of the
  way it mixes names we have the identities for all $x \in \QProc$ and
  $P,Q \in \Proc$

  \begin{mathpar}
    (x \cdot P) \otimes Q \equiv x \cdot (P \otimes Q) \equiv P \otimes (x \cdot Q)
    \and
    P \otimes \pzero \equiv P
  \end{mathpar}

  that the reader is invited to verify.
\end{remark}

\subsubsection{Annihilation}
\begin{mathpar}
  P^{\perp} := \{ Q | \forall R. P|Q \red^{*} R \Rightarrow R \red^{*} \pzero \}
  \and \\
  P^{\underline{\perp}} := \Sigma_{Q \in P^{\perp}} \quotep{Q}?(y).(\dropn{y}|Q) | \Sigma_{Q \in P^{\perp}} \quotep{Q}\clift{\Box}
\end{mathpar}

\paragraph{Discussion} The reader will note that $P^{\perp}$ is a
\emph{set} of processes, while $P^{\underline{\perp}}$ is a
\emph{context}. We call the set $P^{\perp}$ the \emph{annihilators} of
$P$. The parallel composition of a process in the annihilators of $P$
with $P$ will result in a process, the state space of which has all
paths eventually leading to $\pzero$. Execution may endure loops; but
under reasonable conditions of fairness (naturally guaranteed under
most notions of bisimulation) such a composite process cannot get
stuck in such a loop and will, eventually pop out and terminate.

The context $P^{\underline{\perp}}$ is ready and willing to ``take the
$P$ out of'' the process to which it is applied. It will effectively
transmit the code of the process to which it is applied to one of the
annihilators and run the process against it.

\subsubsection{Evaluation}
We fix $M$ a domain of fully abstract interpretation with an equality
coincident with bisimulation. We take $\meaningof{\cdot} : \Proc \to
M$ to be the map interpreting processes and $\nmeaningof{\cdot} : \M
\to Proc$ to be the map running the other way. Then we define

\begin{mathpar}
  \int P := \nmeaningof{\meaningof{P}}
\end{mathpar}

\paragraph{Discussion}
There are many fully abstract interpretations of Milner's
$\pi$-calculus. Any of them can be used as a basis for interpreting
the reflective calculus here. Equipped with such a domain it is
largely a matter of grinding through to check that the Yoneda
construction for the normalization-by-evaluation program can be
extended to this setting.

\begin{remark}
  The reader is invited to verify that $\int (P^{\underline{\perp}}[P]) = 0$.
\end{remark}

\subsection{Quantum mechanics}

Table \ref{tbl:core_qm_op_defns} gives the core operational definitions

\begin{table}[htp]\label{tbl:core_qm_op_defns}
  \center{
    \fbox{
      \begin{tabular}{c|c}
        quantum mechanics & process calculus \\
        \hline
        scalar & $x := \quotep{P}$ \\
        state vector & $\state{P} := P$ \\
        dual & $\state{P}^{*} := \event{P^{\underline{\perp}}} := \quotep{P^{\underline{\perp}}}[-]$ \\
        matrix & $ \Sigma_{\alpha} \state{P_{\alpha}}x_{\alpha}\event{Q_{\alpha}}$ \\
        vector addition & $\state{P} + \state{Q} := \state{P | Q}$ \\
        tensor product & $\state{P} \otimes \state{Q} := \state{P \otimes Q}$ \\
        inner product & $\innerprod{P}{Q} := \quotep{\int P^{\underline{\perp}}[Q]}$ \\
      \end{tabular}
    }
  }
  \caption{QM - operational definitions}
\end{table}

where

\begin{mathpar}
  \prmatrix{P}{Q} := \fprmatrix{P}{\quotep{\pzero}}{Q}
  \and
  \fprmatrix{P}{x}{Q} := (\state{P},x,\event{Q})
  \and
  (\fprmatrix{P}{x}{Q})(\state{R}) := x \cdot \innerprod{Q}{R} \cdot \state{P}
  \and
  (\fprmatrix{P}{x}{Q})(\event{R}) := x \cdot \innerprod{R}{P} \cdot \event{Q}
\end{mathpar}

\paragraph{Discussion}
As promised: vectors (aka states) are represented as processes; duals
as contextual duals; inner product definition should be compared with
standard inner product definition for ....

\begin{remark}
  Assuming $\int (P^{\underline{\perp}}[P]) = 0$, the reader is
  invited to verify that $(\fprmatrix{P}{x}{P})(\state{P}) = x \cdot \state{P}$.
\end{remark}

\begin{remark}
  The reader is invited to verify that $\innerprod{P}{Q}$ could
  equally well have been written $\quotep{\int \stackrel{\vee}{x}}$
  where $x = \event{P^{\underline{\perp}}}(Q)$.

  One of the motivations for this remark is that there is another way
  to factor these operations. We could package up evaluation in the dual:

  \begin{mathpar}
    \state{P}^{*} := \event{\int P^{\underline{\perp}}} := \quotep{\int P^{\underline{\perp}}}[-]
  \end{mathpar}

  and then have inner product defined by
  
  \begin{mathpar}
    \innerprod{P}{Q} := \event{P}(Q)
  \end{mathpar}

  Hopefully, experience with the calculations will provide guidance on
  the best factoring.
\end{remark}

\begin{remark}
  Assuming $\int (P^{\underline{\perp}}[P]) = 0$, the reader is
  invited to verify that $\forall P,Q. (\prmatrix{0}{Q})(\state{0}) =
  \state{0}$ and dually $(\prmatrix{P}{0})(\event{0}) = \event{0}$.
\end{remark}

\begin{remark}
  i'm a little worried that i don't (yet) have proper support for
  complex conjugacy. But, the observation above may give us a
  clue. According to Abramsky, it must be the case that the scalars
  are iso to the homset of the identity for the tensor -- which the
  observation above characterizes. 

  For now, we will simply bookmark the notion with $\overline{x}$.
\end{remark}

\subsubsection{Adjointness}

We need to give a definition of $(\cdot)^{\dagger}$ for matrices. The
obvious candidate definition is
\begin{mathpar}
(\Sigma_{\alpha}\fprmatrix{P_{\alpha}}{x_{\alpha}}{Q_{\alpha}})^{\dagger}
= \Sigma_{\alpha}\fprmatrix{(Q_{\alpha}^{\underline{\perp}})^{*}}{\overline{x}_{\alpha}}{P_{\alpha}^{\underline{\perp}}} 
\end{mathpar}

But, $(Q_{\alpha}^{\underline{\perp}})^{*}$ requires a name along
which to communicate the process to achieve the context application.

\subsubsection{Basis for a basis}
If processes label states and ``addition'' of states (a.k.a. vector
addition) is interpreted as parallel composition, what corresponds to
notions of linear independence and basis? Here, we recall that Yoshida
has developed a set of \emph{combinators} for an asynchronous verison
of Milner's $\pi$-calculus. These are a finite set of processes such
any process can be expressed as parallel composition of these
combinators together with liberal uses of the new operator and
replication. We can simply give a translation of these into the
present calculus and have reasonable expectation that the property
carries over. That is, that the resultant set allows to express all
processes via parallel composition. Note, however, that there is no
new operator or replication in this calculus. As a result, we expect
that the corresponding set is actually infinite. That is, we expect
that the space is actually infinite dimensional.

\begin{remark}
  The attentive reader may be a bit concerned. Certainly, the
  collection $S$, $K$ and $I$ is a finite set of
  combinators. Shouldn't we expect to see a finite set of combinators
  for an effectively equivalent system? i am very sympathetic to this
  critique and feel it warrants full attention. On the other hand, i
  also have in mind the following analogy. The natural numbers, as a
  monoid under addition, has exactly $1$ generator, while the natural
  numbers, as a monoid under multiplication, has countably many
  generators (the primes). We observe that the application of the
  lambda calculus is much less resource sensitive than the parallel
  composition of the $\pi$-calculus. Could it be the case that we have
  an analogy of the form
  
  \begin{mathpar}
    m + n : MN :: m*n : M|N
  \end{mathpar}

  giving a similar blow up in the set of ``primes''?  This is such a
  wonderful thought that, even if it's not true, i think it's worth
  writing down.
\end{remark}
 

\documentclass[12pt]{llncs}
%\documentclass{jktr}

\usepackage[pdftex]{hyperref}                   
\usepackage {listings}
\usepackage {mathpartir}
\usepackage{bcprules}
%\usepackage{listings}
                       
\usepackage{graphicx} 
%\usepackage[margins=2.5cm,nohead,nofoot]{geometry}
%\usepackage{geometry}
\usepackage{amsfonts}
\usepackage{amstext}
\usepackage{latexsym}
\usepackage{amssymb}
\usepackage{color}


%\include{myPreamble}
\include{qm2pi.local} 

%\ifpdf
%\usepackage[pdftex]{graphicx}
%\else
%\usepackage{graphicx}
%\fi

 % \ifpdf
%  \usepackage{pdfsync}
%  \if


%\title{Brief Article}
%\author{David F. Snyder}
%\author{L.G. Meredith}

%\address{Dept. of Math., Texas State University--San Marcos, San Marcos, TX 78666}
       
\pagestyle{empty}


\begin{document}

\lstset{language=[Objective]Caml,frame=shadowbox}

\input{qm2pi.front}

% section front matter (end)

\input{qm2pi.intro} 
 
% section introduction (end)

% \input{qm2pi.knotations} 

% section notation (end)

\input{qm2pi.process.calculi} 

% section concurrent_process_calculi_and_spatial_logics_ (end)
    
%\input{qm2pi.knots2pi} 

%\input{qm2pi.trefoil} 

%\input{qm2pi.mainthm} 

% subsection basic_interpretation (end)

%\input{qm2pi.rho.presentation} 
\subsection{The syntax and semantics of the notation system}\label{sub:the_syntax_and_semantics_of_the_notation_system} % (fold)

We now summarize a technical presentation of the calculus that
embodies our theory of dynamics. The typical presentation of such a
calculus follows the style of giving generators and relations on
them. The grammar, below, describing term constructors, freely
generates the set of processes, $\Proc$. This set is then quotiented
by a relation known as structural congruence and it is over this set
that the notion of dynamics is expressed. This presentation is
essentially that of \cite{MeredithR05} with the addition of
polyadicity and summation. For readability we have relegated some of
the technical subtleties to an appendix.

\subsubsection{Process grammar}\label{subsub:process_grammar}

\begin{mathpar}
  \inferrule* [lab=synchronization] {} {{M} \bc \pzero \;|\; x?F \;|\; x!C }
  \and
  \inferrule* [lab=abstraction] {} {{F} \bc (x)P}
  \and
  \inferrule* [lab=concretion] {} {{C} \bc \langle Q \rangle}
  \and
  \inferrule* [lab=process] {} {{P,Q} \bc M \;| \;P|Q \;|\; @{x}}
  \and
  \inferrule* [lab=name] {} {{x} \bc \quotep{P}}
\end{mathpar} 

Note that $\vec{x}$ (resp. $\vec{P}$) denotes a vector of names
(resp. processes) of length $|\vec{x}|$ (resp. $|\vec{P}|$). We adopt
the following useful abbreviations.

\begin{mathpar}
   x?(\vec{y}).P := x.(\vec{y})P \and  x\clift{\vec{P}} := x.\clift{\vec{P}}
   \and x!(y) := \lift{x}{\dropn{y}}
   \and \Pi_{i=0}^{n-1}P_i := P_0 | \ldots | P_{n-1}
\end{mathpar}

\subsubsection{Structural congruence}

\paragraph{Free and bound names and alpha-equivalence.} At the
core of structural equivalence is alpha-equivalence which identifies
process that are the same up to a change of variable. Formally, we
recognize the distinction between free and bound names. The free names
of a process, $\freenames{P}$, may be calculated recursively as
follows:

\begin{mathpar}
\freenames{\pzero} := \emptyset
  \and \\
  \freenames{x?(y).P} := \{ x \} \cup (\freenames{P} \setminus \{ y \})
  \and 
  \freenames{x!\langle P \rangle} := \{ x \} \cup \{ P \} 
  \and \\
  \freenames{P|Q} := \freenames{P} \cup \freenames{Q}
  \and \\
  \freenames{@{x}} := \{ x \}
\end{mathpar}

$\pi$
$\quotep{\pi}$

$\freenames{-} : \pi \to \mathcal{P}(\quotep{\pi})$

\begin{eqnarray*}
  \freenames{\pzero} & := & \emptyset \\
  \freenames{x?(y).P} & := & \{ x \} \cup (\freenames{P} \setminus \{ y \}) \\
  \freenames{x!\langle P \rangle} & := & \{ x \} \cup \{ P \} \\
  \freenames{P|Q} & := & \freenames{P} \cup \freenames{Q} \\
  \freenames{\dropn{x}} & := & \{ x \}
\end{eqnarray*}

The bound names of a process, $\boundnames{P}$, are those names occurring in $P$
that are not free. For example, in $x?(y).0$, the name $x$ is free, while $y$ is bound.

\begin{mathpar}
  \inferrule* [lab=monoidal-laws] {} { P|Q \equiv Q|P \and P|0 \equiv P \and P|(Q|R) \equiv (P|Q)|R }
\end{mathpar}

\begin{mathpar}
  \inferrule* [lab=alpha-equivalence] {} { (x)P \equiv (y)P\{y/x\} \and y \not\in \freenames{P} }
\end{mathpar}

\begin{definition}
Then two processes, $P,Q$, are alpha-equivalent if $P = Q\{\vec{y}/\vec{x}\}$ for
some $\vec{x} \in \boundnames{Q},\vec{y} \in \boundnames{P}$, where $Q\{\vec{y}/\vec{x}\}$
denotes the capture-avoiding substitution of $\vec{y}$ for $\vec{x}$ in $Q$.
\end{definition}

\begin{definition}
  The {\em structural congruence} \cite{SangiorgiWalker} , $\equiv$,
  between processes is the least congruence containing
  alpha-equivalence, satisfying the abelian monoid laws
  (associativity, commutativity and $\pzero$ as identity) for parallel
  composition $|$ and for summation $+$.
\end{definition}

\subsection{Name equivalence}

We take name equivalence, written $\nameeq$, to be the smallest
equivalence relation generated by the following rules.

\begin{mathpar}
\inferrule*[lab=Quote-drop]
{ }
{ \quotep{@{x}} \nameeq x }

\inferrule*[lab=Struct-equiv]
{ P \scong Q }
{ \quotep{P} \nameeq \quotep{Q} }
\end{mathpar}

The astute reader will have noticed that the mutual recursion of names
and processes imposes a mutual recursion on alpha-equivalence and
structural equivalence via name-equivalence. Fortunately, all of this
works out pleasantly and we may calculate in the natural way, free of
concern. The reader interested in the details is referred to the
appendix \ref{appendix:rho_details}.

\subsection{Substitution}

We use $\Proc$ for the set of processes, $\QProc$ for the set of
names, and $\id{\{}\vec{y} / \vec{x} \id{\}}$ to denote partial maps,
$s : \QProc \rightarrow \QProc$. A map, $s$ lifts, uniquely, to a map
on process terms, $\widehat{s} : \Proc \rightarrow \Proc$ by the
following equations.

\begin{mathpar}
  (0) \psubstp{Q}{P} := 0 \\
  (R \juxtap S) \psubstp{Q}{P}
  :=    
  (R)\psubstp{Q}{P} \juxtap (S) \psubstp{Q}{P} \\
  (x?(y).R) \psubstp{Q}{P}    
  :=    
  (x)\substp{Q}{P} (z)\concat( (R \psubstn{z}{y}) \psubstp{Q}{P} ) \\
  (\lift{x}{R}) \psubstp{Q}{P}  
  :=
  \lift{(x)\substp{Q}{P}}{ R \psubstp{Q}{P} } \\
%   (\dropn{x})  \psubstp{Q}{P}       
%   := 
%   \left\{ 
%     \begin{array}{ccc} 
%       \dropn{\quotep{Q}} & & x \nameeq \quotep{P} \\
%       \dropn{x} & & otherwise \\
%     \end{array}
%   \right. 
  (\dropn{x})  \psubstp{Q}{P}       
  := 
  \left\{ 
    \begin{array}{ccc} 
      Q & & x \nameeq \quotep{P} \\
      \dropn{x} & & otherwise \\
    \end{array}
  \right.
\end{mathpar}
 

where

\begin{eqnarray}
  (x)\id{\{} \lpquote Q \rpquote / \lpquote P \rpquote \id{\}}            = 
  \left\{ 
    \begin{array}{ccc}
      \lpquote Q \rpquote & & x \nameeq \lpquote P \rpquote \\
      x & & otherwise \\
    \end{array}
  \right. \nonumber
\end{eqnarray}

and $z$ is chosen distinct from $\quotep{P}$, $\quotep{Q}$, the free
names in $Q$, and all the names in $R$. Our $\alpha$-equivalence will
be built in the standard way from this substitution.

\begin{remark}\label{rem:no_self_referential_names}
  One consequence of these definitions is that $\forall P. \quotep{P}
  \not\in \freenames{P}$.
\end{remark}

\subsection{ Dynamic quote: an example }

Anticipating something of what's to come, consider applying the
substitution, $\widehat{\id{\{}u / z \id{\}}}$, to the following pair
of processes, $\lift{w}{y!(z)}$ and $w[ \lpquote y!(z) \rpquote ]$.

\begin{eqnarray}
	\lift{w}{y!(z)}\widehat{\id{\{}u / z \id{\}}}
		& = &
		\lift{w}{y!(u)} \nonumber\\
	w[ \lpquote y!(z) \rpquote ] \widehat{ \id{\{}u / z \id{\}} }
		& = &
		w[ \lpquote y!(z) \rpquote ] \nonumber
\end{eqnarray}

Because the body of the process between quotes is impervious to
substitution, we get radically different answers. In fact, by
examining the first process in an input context,
e.g. $x?(z).\lift{w}{y!(z)}$, we see that the process under the lift
operator may be shaped by prefixed inputs binding a name inside it. In
this sense, the lift operator will be seen as a way to dynamically
construct processes before reifying them as names.

Finally equipped with these standard features we can present the
dynamics of the calculus.

\subsubsection{Operational semantics} 

Finally, we introduce the computational dynamics. What marks these
algebras as distinct from other more traditionally studied algebraic
structures, e.g. vector spaces or polynomial rings, is the manner in
which dynamics is captured. In traditional structures, dynamics is typically
expressed through morphisms between such structures, as in linear maps
between vector spaces or morphisms between rings. In algebras
associated with the semantics of computation, the dynamics is
expressed as part of the algebraic structure itself, through a
reduction reduction relation typically denoted by $\red$. Below, we
give a recursive presentation of this relation for the calculus used
in the encoding.

$\red \subseteq \pi \times \pi$
$\red : \pi \to \mathcal{P}(\pi)$

\begin{mathpar}
  \inferrule* [lab=Comm] { \textsf{match}( x_{src}, x_{trgt} ) } { x_{trgt}?(y)P \; | \; x_{src}!\langle {Q} \rangle \red P\{\quotep{Q}/y}\} }
  \and \\
  \inferrule* [lab=Par] {{P} \red {P}'} {{{P} | {Q}} \red {{P}' | {Q}}}
  \and
  \inferrule* [lab=Equiv]{{{P} \scong {P}'} \andalso {{P}' \red {Q}'} \andalso {{Q}' \scong {Q}}}{{P} \red {Q}}
\end{mathpar}

\begin{eqnarray*}
  match_{\equiv} (\quotep{P},\quotep{Q}) & := & P \equiv Q \\
  match_{\dagger}(\quotep{P},\quotep{Q}) & := & \forall R. P|Q \red^{*} R => R \red^{*} 0 \\
  match_{K}(\quotep{P},\quotep{Q}) & := & K \mbox{ for some context } K
\end{eqnarray*}

$u?(x)P | u!\langle Q \rangle \red P\{\quotep{Q}/x\}$

%We write $\wred$ for $\red^*$, and $P\red$ if $\exists Q $ such that $ P \red Q$.
We write $P\red$ if $\exists Q $ such that $ P \red Q$ and $P\not\red$, otherwise.

\section{Replication}

As mentioned before, it is known that replication (and hence
recursion) can be implemented in a higher-order process algebra
\cite{SangiorgiWalker}. As our first example of calculation with the
machinery thus far presented we give the construction explicitly in
the {\rhoc}.

\begin{eqnarray}
	D_{x} & := & \prefix{x}{y}{(\binpar{\outputp{x}{y}}{@{y}})} \nonumber\\
	\bangp_{x}{P} & := & \binpar{{x}!\langle{\binpar{D_{x}}{P}}\rangle}{D_{x}} \nonumber
\end{eqnarray}

\begin{eqnarray}
	\bangp_{x}{P} & & \nonumber\\
	=
	& {x}!\langle{(\prefix{x}{y}{(\outputp{x}{y} | @{y})) | P}}\rangle 
	      | \prefix{x}{y}{(\outputp{x}{y} | @{y})} & \nonumber\\
	\red
	& (\outputp{x}{y} | @{y})\substn{\quotep{(\prefix{x}{y}{(@{y} | \outputp{x}{y})) | P}}}{y} & \nonumber\\
	=
	& \outputp{x}{\quotep{(\prefix{x}{y}{(\outputp{x}{y} | @{y})) | P}}}
	  | {(\prefix{x}{y}{(\outputp{x}{y} | @{y})) | P}} & \nonumber\\
	\red
	& \ldots & \nonumber\\
	\red^*
	& P | P | \ldots & \nonumber
\end{eqnarray}

Of course, this encoding, as an implementation, runs away, unfolding
$\bangp{P}$ eagerly. A lazier and more implementable replication
operator, restricted to input-guarded processes, may be obtained as follows.

\begin{eqnarray}
\bangp{\prefix{u}{v}{P}} 
	:= 
	\binpar{\lift{x}{\prefix{u}{v}{(\binpar{D(x)}{P})}}}{D(x)} \nonumber
\end{eqnarray}

\begin{remark}
  Note that the lazier definition still does not deal with summation
  or mixed summation (i.e. sums over input and output). The reader is
  invited to construct definitions of replication that deal with these
  features. 

  Further, the definitions are parameterized in a name, $x$. Can you,
  gentle reader, make a definition that eliminates this parameter and
  guarantees no accidental interaction between the replication
  machinery and the process being replicated -- i.e. no accidental
  sharing of names used by the process to get its work done and the
  name(s) used by the replication to effect copying. This latter
  revision of the definition of replication is crucial to obtaining
  the expected identity $!!P \sim !P$.
\end{remark}

\begin{remark}\label{rem:paradoxical_combinator}
  The reader familiar with the lambda calculus will have noticed the
  similarity between $D$ and the paradoxical combinator.

  [Ed. note: the existence of this seems to suggest we have to be more
  restrictive on the set of processes and names we admit if we are to
  support no-cloning.]
\end{remark}

\subsubsection{Bisimulation}

The computational dynamics gives rise to another kind of equivalence,
the equivalence of computational behavior. As previously mentioned
this is typically captured \emph{via} some form of bisimulation.

% The notion we use in this paper is weak barbed bisimulation
% \cite{milner91polyadicpi}.

The notion we use in this paper is derived from weak barbed
bisimulation \cite{milner91polyadicpi}. 

\begin{definition}
An \emph{observation relation}, $\downarrow_{\mathcal N}$, over a set
of names, $\mathcal N$, is the smallest relation satisfying the rules
below.

\infrule[Out-barb]{y \in {\mathcal N}, \; x \nameeq y}
		  {\outputp{x}{v} \downarrow_{\mathcal N} x}
\infrule[Par-barb]{\mbox{$P\downarrow_{\mathcal N} x$ or $Q\downarrow_{\mathcal N} x$}}
		  {\binpar{P}{Q} \downarrow_{\mathcal N} x}

We write $P \Downarrow_{\mathcal N} x$ if there is $Q$ such that 
$P \wred Q$ and $Q \downarrow_{\mathcal N} x$.
\end{definition}

\begin{definition}
%\label{def.bbisim}
An  ${\mathcal N}$-\emph{barbed bisimulation} over a set of names, ${\mathcal N}$, is a symmetric binary relation 
${\mathcal S}_{\mathcal N}$ between agents such that $P\rel{S}_{\mathcal N}Q$ implies:
\begin{enumerate}
\item If $P \red P'$ then $Q \wred Q'$ and $P'\rel{S}_{\mathcal N} Q'$.
\item If $P\downarrow_{\mathcal N} x$, then $Q\Downarrow_{\mathcal N} x$.
\end{enumerate}
$P$ is ${\mathcal N}$-barbed bisimilar to $Q$, written
$P \wbbisim_{\mathcal N} Q$, if $P \rel{S}_{\mathcal N} Q$ for some ${\mathcal N}$-barbed bisimulation ${\mathcal S}_{\mathcal N}$.
\end{definition}

$\mathcal{R} \subseteq \pi \times \pi$

$P \mathcal{R} Q => \forall P'. P \red P' \Rightarrow \exists Q'. Q \red Q', P' \mathcal{R} Q'$

$P \vdash x \Rightarrow Q \vdash x$

\begin{mathpar}
  \inferrule*[lab=Out-barb]{x \nameeq y}{{y}!\langle{Q}\rangle \vdash x}
  \and
  \inferrule*[lab=Par-barb]{\mbox{$P\vdash x$ or $Q\vdash x$}}{\binpar{P}{Q} \vdash x}
\end{mathpar}

\subsubsection{Contexts}

One of the principle advantages of computational calculi like the
$\pi$-calculus is a well-defined notion of context,
contextual-equivalence and a correlation between
contextual-equivalence and notions of bisimulation. The notion of
context allows the decomposition of a process into (sub-)process and
its syntactic environment, its context. Thus, a context may be
thought of as a process with a ``hole'' (written $\Box$) in it. The
application of a context $M$ to a process $P$, written $M[P]$, is
tantamount to filling the hole in $M$ with $P$. In this paper we do
not need the full weight of this theory, but do make use of the notion
of context in the proof the main theorem. 

\begin{mathpar}
  \inferrule* [lab=summation] {} {{M_{M},M_{N}} \bc \Box \;|\; x.M_{A} \;|\; M_{M}+M_{N}}
  \and
  \inferrule* [lab=agent] {} {{M_{A}} \bc (\vec{x})M_{P} \;| \; \clift{P_0,\ldots,M_{P},\ldots,P_N}}
  \and \\
  \inferrule* [lab=process] {} {{M_{P}} \bc M_{N} \;| \;P|M_{P} }
\end{mathpar} 

\begin{mathpar}
  \inferrule* [lab=sychronization] {} {M_{N} \bc \Box \;|\; x?M_{F} \;|\; x!M_{C}}
  \and
  \inferrule* [lab=abstraction] {} {{M_{F}} \bc (x)M_{P} }
  \and
  \inferrule* [lab=concretion] {} {{M_{C}} \bc \langle M_{P} \rangle }
  \and \\
  \inferrule* [lab=process] {} {{M_{P}} \bc M_{N} \;| \;P|M_{P} }
\end{mathpar}

\begin{definition}[contextual application] Given a context $M$, and
  process $P$, we define the \emph{contextual application}, $M[P] :=
  M\{P/\Box\}$. That is, the contextual application of M to P is the
  substitution of $P$ for $\Box$ in $M$.
\end{definition}

$\meaningof{-} : L \to \mathcal{P}(\pi)$

\begin{mathpar}
  \inferrule* [lab=collection] {} {\meaningof{true} = \pi, \and \meaningof{~E} = \pi \setminus \meaningof{E}, \and \meaningof{E_{1} \& E_{2}} = \meaningof{E_{1}} \cap \meaningof{E_{2}}}
\end{mathpar}

\begin{mathpar}
  \inferrule* [lab=structure] {} {\meaningof{0} = \{ P \in \pi | P \equiv 0 \}, \and \\ \meaningof{E_1 | E_2} = \{ P \in \pi | P \equiv P_{1} | P_{2}, P_{1} \in \meaningof{E_{1}}, P_{2} \in \meaningof{E_2}\} }
\end{mathpar}

\begin{mathpar}
 \inferrule* [lab=behavior] {} {\meaningof{\langle a?b \rangle E} = \{ P \in \pi | P \equiv Q | u?(y)P', \\ \and \\\\ \and \\ \;\;\; u \in \meaningof{a}, \forall z.P'\{z/y\} \in \meaningof{E\{z/b\}}\}, \and \\ \meaningof{a!E} = \{ P \in \pi | P \equiv Q | x!\langle P' \rangle, x \in \meaningof{a} P' \in \meaningof{E}\} }
\end{mathpar}

\begin{mathpar}
 \inferrule* [lab=nominal] {} {\meaningof{\quotep{E}} = \{ \quotep{P} \in \quotep{\pi} | P \in \meaningof{E} \}, \and \meaningof{\quotep{P}} = \{ \quotep{Q} \in \quotep{\pi} | P \equiv Q \} \and \\ \meaningof{@\quotep{E}} = \{ P \in \pi | P \equiv @x, x \in \meaningof{E} \}}
\end{mathpar}

\begin{eqnarray*}
  \\
  \meaningof{-} : TS \to ST
\end{eqnarray*}

\begin{eqnarray*}
  \\
  L : TS \to ST
\end{eqnarray*}

\begin{eqnarray*}
  \\
  P \models E \iff P \in \meaningof{E}
\end{eqnarray*}

\begin{eqnarray*}
  P \approx_{L} Q \iff \forall E \in L. P \models E \iff Q \models E
\end{eqnarray*}

\begin{eqnarray*}
  P \approx_{K} Q
\end{eqnarray*}

\begin{eqnarray*}
  P \approx Q
\end{eqnarray*}

$\approx_{K} = \approx = \approx_{L}$

\subsubsection{Contextual duality}

Note that contexts extend the quotation operation to a family of
operations from processes to names. Given a context, $M$, we can
define a \emph{nominal context}, $\quotep{M}$ by $\quotep{M}[P] :=
\quotep{M[P]}$. To foreshadow what is to come we observe that these
operations enjoy a duality with processes very much like the duality
between vectors and maps from vectors to scalars.

Further, because the calculus is essentially higher-order, we have a
correspondence between contexts and processes. More specifically,
given a name $x$ and a context $M$ we can construct $M^{*}_{x}$ such
that 

\begin{mathpar}
  M^{*}_{x} | \lift{x}{P} \red M[P]
\end{mathpar}

namely,

\begin{mathpar}
  M^{*}_{x} := x?(u).M[\dropn{u}]
\end{mathpar}

The dependence of $M^{*}_{x}$ on a name makes it an abstraction, 

\begin{mathpar}
  M^{*} := (x)x?(u).M[\dropn{u}]
\end{mathpar}

\subsection{Additional notation}

It will sometimes be convenient to denote the process a name
quotes. We already have the notation $x = \quotep{P}$, but it will be
convenient to introduce an alternate notation, $\procn{x}$, when we
want to emphasize the connection to the use of the name. Note that, by
virtue of name equivalence, $\quotep{\procn{x}} \nameeq x$; so, the
notation is consistent with previous definitions.

Further, because names have structure it is possible to effect
substitutions on the basis of that structure. This means we need to
upgrade our notation for substitutions, which we accomplish by
adapting comprehension notation. Thus,

\begin{mathpar}
  P\{ y / x : x \in S \}
\end{mathpar}

is interpreted to mean the process derived from P by replacing (in a
capture-avoiding manner) each occurrence of $x$ in $S$ by $y$. For example,

\begin{mathpar}
  P\{ \quotep{\procn{x}|\procn{x}} / x : x \in \freenames{P} \}
\end{mathpar}

will replace each (occurrence) of a free name $x$ in $P$ by
$\quotep{\procn{x}|\procn{x}}$.

Also, we will avail ourselves of the notation $x^{L}$ and $x^{R}$ to
denote injections of a name into disjoint copies of the name
space. There are numerous ways to accomplish this. One example can be
found in \cite{MeredithR05}. This notation overloads to vectors of
names: $\vec{x}^{\pi} := (x_{i}^{\pi} \; : \; 0 \leq i < |\vec{x}| )$ where $\pi \in \{L,R\}$.

We also use $P^{\Box} := P|\Box$.

In \cite{MeredithR05} an interpretation of the new operator is
given. It turns out that there are several possible interpretations
all enjoying the requisite algebraic properties of the operator (see
\cite{milner91polyadicpi}). We will therefore make liberal use of
$(\nu\; \vec{x})P$.

% subsection the_syntax_and_semantics_of_the_notation_system (end)   

\input{qm2pi.qmops} 

\input{qm2pi.sterngerlach} 

\input{qm2pi.metric} 

% section concurrent_process_calculi (end)

%\input{qm2pi.proofsketch}

% section proof sketch (end)

%\input{qm2pi.slviaknots} 

% section spatial logic via knots (end)

\input{qm2pi.conclusion}

% section conclusion (end)

%\input{qm2pi.dtcodes} 

% section wiring algorithm (end)

\input{qm2pi.ack} 

% section acknowledgments (end)

\newpage


\bibliographystyle{plain}   
\bibliography{../../biblios/main.bib}

\input{qm2pi.rhodetails}

\end{document}

 

\documentclass[12pt]{llncs}
%\documentclass{jktr}

\usepackage[pdftex]{hyperref}                   
\usepackage {listings}
\usepackage {mathpartir}
\usepackage{bcprules}
%\usepackage{listings}
                       
\usepackage{graphicx} 
%\usepackage[margins=2.5cm,nohead,nofoot]{geometry}
%\usepackage{geometry}
\usepackage{amsfonts}
\usepackage{amstext}
\usepackage{latexsym}
\usepackage{amssymb}
\usepackage{color}


%\include{myPreamble}
\include{qm2pi.local} 

%\ifpdf
%\usepackage[pdftex]{graphicx}
%\else
%\usepackage{graphicx}
%\fi

 % \ifpdf
%  \usepackage{pdfsync}
%  \if


%\title{Brief Article}
%\author{David F. Snyder}
%\author{L.G. Meredith}

%\address{Dept. of Math., Texas State University--San Marcos, San Marcos, TX 78666}
       
\pagestyle{empty}


\begin{document}

\lstset{language=[Objective]Caml,frame=shadowbox}

\input{qm2pi.front}

% section front matter (end)

\input{qm2pi.intro} 
 
% section introduction (end)

% \input{qm2pi.knotations} 

% section notation (end)

\input{qm2pi.process.calculi} 

% section concurrent_process_calculi_and_spatial_logics_ (end)
    
%\input{qm2pi.knots2pi} 

%\input{qm2pi.trefoil} 

%\input{qm2pi.mainthm} 

% subsection basic_interpretation (end)

%\input{qm2pi.rho.presentation} 
\subsection{The syntax and semantics of the notation system}\label{sub:the_syntax_and_semantics_of_the_notation_system} % (fold)

We now summarize a technical presentation of the calculus that
embodies our theory of dynamics. The typical presentation of such a
calculus follows the style of giving generators and relations on
them. The grammar, below, describing term constructors, freely
generates the set of processes, $\Proc$. This set is then quotiented
by a relation known as structural congruence and it is over this set
that the notion of dynamics is expressed. This presentation is
essentially that of \cite{MeredithR05} with the addition of
polyadicity and summation. For readability we have relegated some of
the technical subtleties to an appendix.

\subsubsection{Process grammar}\label{subsub:process_grammar}

\begin{mathpar}
  \inferrule* [lab=synchronization] {} {{M} \bc \pzero \;|\; x?F \;|\; x!C }
  \and
  \inferrule* [lab=abstraction] {} {{F} \bc (x)P}
  \and
  \inferrule* [lab=concretion] {} {{C} \bc \langle Q \rangle}
  \and
  \inferrule* [lab=process] {} {{P,Q} \bc M \;| \;P|Q \;|\; @{x}}
  \and
  \inferrule* [lab=name] {} {{x} \bc \quotep{P}}
\end{mathpar} 

Note that $\vec{x}$ (resp. $\vec{P}$) denotes a vector of names
(resp. processes) of length $|\vec{x}|$ (resp. $|\vec{P}|$). We adopt
the following useful abbreviations.

\begin{mathpar}
   x?(\vec{y}).P := x.(\vec{y})P \and  x\clift{\vec{P}} := x.\clift{\vec{P}}
   \and x!(y) := \lift{x}{\dropn{y}}
   \and \Pi_{i=0}^{n-1}P_i := P_0 | \ldots | P_{n-1}
\end{mathpar}

\subsubsection{Structural congruence}

\paragraph{Free and bound names and alpha-equivalence.} At the
core of structural equivalence is alpha-equivalence which identifies
process that are the same up to a change of variable. Formally, we
recognize the distinction between free and bound names. The free names
of a process, $\freenames{P}$, may be calculated recursively as
follows:

\begin{mathpar}
\freenames{\pzero} := \emptyset
  \and \\
  \freenames{x?(y).P} := \{ x \} \cup (\freenames{P} \setminus \{ y \})
  \and 
  \freenames{x!\langle P \rangle} := \{ x \} \cup \{ P \} 
  \and \\
  \freenames{P|Q} := \freenames{P} \cup \freenames{Q}
  \and \\
  \freenames{@{x}} := \{ x \}
\end{mathpar}

$\pi$
$\quotep{\pi}$

$\freenames{-} : \pi \to \mathcal{P}(\quotep{\pi})$

\begin{eqnarray*}
  \freenames{\pzero} & := & \emptyset \\
  \freenames{x?(y).P} & := & \{ x \} \cup (\freenames{P} \setminus \{ y \}) \\
  \freenames{x!\langle P \rangle} & := & \{ x \} \cup \{ P \} \\
  \freenames{P|Q} & := & \freenames{P} \cup \freenames{Q} \\
  \freenames{\dropn{x}} & := & \{ x \}
\end{eqnarray*}

The bound names of a process, $\boundnames{P}$, are those names occurring in $P$
that are not free. For example, in $x?(y).0$, the name $x$ is free, while $y$ is bound.

\begin{mathpar}
  \inferrule* [lab=monoidal-laws] {} { P|Q \equiv Q|P \and P|0 \equiv P \and P|(Q|R) \equiv (P|Q)|R }
\end{mathpar}

\begin{mathpar}
  \inferrule* [lab=alpha-equivalence] {} { (x)P \equiv (y)P\{y/x\} \and y \not\in \freenames{P} }
\end{mathpar}

\begin{definition}
Then two processes, $P,Q$, are alpha-equivalent if $P = Q\{\vec{y}/\vec{x}\}$ for
some $\vec{x} \in \boundnames{Q},\vec{y} \in \boundnames{P}$, where $Q\{\vec{y}/\vec{x}\}$
denotes the capture-avoiding substitution of $\vec{y}$ for $\vec{x}$ in $Q$.
\end{definition}

\begin{definition}
  The {\em structural congruence} \cite{SangiorgiWalker} , $\equiv$,
  between processes is the least congruence containing
  alpha-equivalence, satisfying the abelian monoid laws
  (associativity, commutativity and $\pzero$ as identity) for parallel
  composition $|$ and for summation $+$.
\end{definition}

\subsection{Name equivalence}

We take name equivalence, written $\nameeq$, to be the smallest
equivalence relation generated by the following rules.

\begin{mathpar}
\inferrule*[lab=Quote-drop]
{ }
{ \quotep{@{x}} \nameeq x }

\inferrule*[lab=Struct-equiv]
{ P \scong Q }
{ \quotep{P} \nameeq \quotep{Q} }
\end{mathpar}

The astute reader will have noticed that the mutual recursion of names
and processes imposes a mutual recursion on alpha-equivalence and
structural equivalence via name-equivalence. Fortunately, all of this
works out pleasantly and we may calculate in the natural way, free of
concern. The reader interested in the details is referred to the
appendix \ref{appendix:rho_details}.

\subsection{Substitution}

We use $\Proc$ for the set of processes, $\QProc$ for the set of
names, and $\id{\{}\vec{y} / \vec{x} \id{\}}$ to denote partial maps,
$s : \QProc \rightarrow \QProc$. A map, $s$ lifts, uniquely, to a map
on process terms, $\widehat{s} : \Proc \rightarrow \Proc$ by the
following equations.

\begin{mathpar}
  (0) \psubstp{Q}{P} := 0 \\
  (R \juxtap S) \psubstp{Q}{P}
  :=    
  (R)\psubstp{Q}{P} \juxtap (S) \psubstp{Q}{P} \\
  (x?(y).R) \psubstp{Q}{P}    
  :=    
  (x)\substp{Q}{P} (z)\concat( (R \psubstn{z}{y}) \psubstp{Q}{P} ) \\
  (\lift{x}{R}) \psubstp{Q}{P}  
  :=
  \lift{(x)\substp{Q}{P}}{ R \psubstp{Q}{P} } \\
%   (\dropn{x})  \psubstp{Q}{P}       
%   := 
%   \left\{ 
%     \begin{array}{ccc} 
%       \dropn{\quotep{Q}} & & x \nameeq \quotep{P} \\
%       \dropn{x} & & otherwise \\
%     \end{array}
%   \right. 
  (\dropn{x})  \psubstp{Q}{P}       
  := 
  \left\{ 
    \begin{array}{ccc} 
      Q & & x \nameeq \quotep{P} \\
      \dropn{x} & & otherwise \\
    \end{array}
  \right.
\end{mathpar}
 

where

\begin{eqnarray}
  (x)\id{\{} \lpquote Q \rpquote / \lpquote P \rpquote \id{\}}            = 
  \left\{ 
    \begin{array}{ccc}
      \lpquote Q \rpquote & & x \nameeq \lpquote P \rpquote \\
      x & & otherwise \\
    \end{array}
  \right. \nonumber
\end{eqnarray}

and $z$ is chosen distinct from $\quotep{P}$, $\quotep{Q}$, the free
names in $Q$, and all the names in $R$. Our $\alpha$-equivalence will
be built in the standard way from this substitution.

\begin{remark}\label{rem:no_self_referential_names}
  One consequence of these definitions is that $\forall P. \quotep{P}
  \not\in \freenames{P}$.
\end{remark}

\subsection{ Dynamic quote: an example }

Anticipating something of what's to come, consider applying the
substitution, $\widehat{\id{\{}u / z \id{\}}}$, to the following pair
of processes, $\lift{w}{y!(z)}$ and $w[ \lpquote y!(z) \rpquote ]$.

\begin{eqnarray}
	\lift{w}{y!(z)}\widehat{\id{\{}u / z \id{\}}}
		& = &
		\lift{w}{y!(u)} \nonumber\\
	w[ \lpquote y!(z) \rpquote ] \widehat{ \id{\{}u / z \id{\}} }
		& = &
		w[ \lpquote y!(z) \rpquote ] \nonumber
\end{eqnarray}

Because the body of the process between quotes is impervious to
substitution, we get radically different answers. In fact, by
examining the first process in an input context,
e.g. $x?(z).\lift{w}{y!(z)}$, we see that the process under the lift
operator may be shaped by prefixed inputs binding a name inside it. In
this sense, the lift operator will be seen as a way to dynamically
construct processes before reifying them as names.

Finally equipped with these standard features we can present the
dynamics of the calculus.

\subsubsection{Operational semantics} 

Finally, we introduce the computational dynamics. What marks these
algebras as distinct from other more traditionally studied algebraic
structures, e.g. vector spaces or polynomial rings, is the manner in
which dynamics is captured. In traditional structures, dynamics is typically
expressed through morphisms between such structures, as in linear maps
between vector spaces or morphisms between rings. In algebras
associated with the semantics of computation, the dynamics is
expressed as part of the algebraic structure itself, through a
reduction reduction relation typically denoted by $\red$. Below, we
give a recursive presentation of this relation for the calculus used
in the encoding.

$\red \subseteq \pi \times \pi$
$\red : \pi \to \mathcal{P}(\pi)$

\begin{mathpar}
  \inferrule* [lab=Comm] { \textsf{match}( x_{src}, x_{trgt} ) } { x_{trgt}?(y)P \; | \; x_{src}!\langle {Q} \rangle \red P\{\quotep{Q}/y}\} }
  \and \\
  \inferrule* [lab=Par] {{P} \red {P}'} {{{P} | {Q}} \red {{P}' | {Q}}}
  \and
  \inferrule* [lab=Equiv]{{{P} \scong {P}'} \andalso {{P}' \red {Q}'} \andalso {{Q}' \scong {Q}}}{{P} \red {Q}}
\end{mathpar}

\begin{eqnarray*}
  match_{\equiv} (\quotep{P},\quotep{Q}) & := & P \equiv Q \\
  match_{\dagger}(\quotep{P},\quotep{Q}) & := & \forall R. P|Q \red^{*} R => R \red^{*} 0 \\
  match_{K}(\quotep{P},\quotep{Q}) & := & K \mbox{ for some context } K
\end{eqnarray*}

$u?(x)P | u!\langle Q \rangle \red P\{\quotep{Q}/x\}$

%We write $\wred$ for $\red^*$, and $P\red$ if $\exists Q $ such that $ P \red Q$.
We write $P\red$ if $\exists Q $ such that $ P \red Q$ and $P\not\red$, otherwise.

\section{Replication}

As mentioned before, it is known that replication (and hence
recursion) can be implemented in a higher-order process algebra
\cite{SangiorgiWalker}. As our first example of calculation with the
machinery thus far presented we give the construction explicitly in
the {\rhoc}.

\begin{eqnarray}
	D_{x} & := & \prefix{x}{y}{(\binpar{\outputp{x}{y}}{@{y}})} \nonumber\\
	\bangp_{x}{P} & := & \binpar{{x}!\langle{\binpar{D_{x}}{P}}\rangle}{D_{x}} \nonumber
\end{eqnarray}

\begin{eqnarray}
	\bangp_{x}{P} & & \nonumber\\
	=
	& {x}!\langle{(\prefix{x}{y}{(\outputp{x}{y} | @{y})) | P}}\rangle 
	      | \prefix{x}{y}{(\outputp{x}{y} | @{y})} & \nonumber\\
	\red
	& (\outputp{x}{y} | @{y})\substn{\quotep{(\prefix{x}{y}{(@{y} | \outputp{x}{y})) | P}}}{y} & \nonumber\\
	=
	& \outputp{x}{\quotep{(\prefix{x}{y}{(\outputp{x}{y} | @{y})) | P}}}
	  | {(\prefix{x}{y}{(\outputp{x}{y} | @{y})) | P}} & \nonumber\\
	\red
	& \ldots & \nonumber\\
	\red^*
	& P | P | \ldots & \nonumber
\end{eqnarray}

Of course, this encoding, as an implementation, runs away, unfolding
$\bangp{P}$ eagerly. A lazier and more implementable replication
operator, restricted to input-guarded processes, may be obtained as follows.

\begin{eqnarray}
\bangp{\prefix{u}{v}{P}} 
	:= 
	\binpar{\lift{x}{\prefix{u}{v}{(\binpar{D(x)}{P})}}}{D(x)} \nonumber
\end{eqnarray}

\begin{remark}
  Note that the lazier definition still does not deal with summation
  or mixed summation (i.e. sums over input and output). The reader is
  invited to construct definitions of replication that deal with these
  features. 

  Further, the definitions are parameterized in a name, $x$. Can you,
  gentle reader, make a definition that eliminates this parameter and
  guarantees no accidental interaction between the replication
  machinery and the process being replicated -- i.e. no accidental
  sharing of names used by the process to get its work done and the
  name(s) used by the replication to effect copying. This latter
  revision of the definition of replication is crucial to obtaining
  the expected identity $!!P \sim !P$.
\end{remark}

\begin{remark}\label{rem:paradoxical_combinator}
  The reader familiar with the lambda calculus will have noticed the
  similarity between $D$ and the paradoxical combinator.

  [Ed. note: the existence of this seems to suggest we have to be more
  restrictive on the set of processes and names we admit if we are to
  support no-cloning.]
\end{remark}

\subsubsection{Bisimulation}

The computational dynamics gives rise to another kind of equivalence,
the equivalence of computational behavior. As previously mentioned
this is typically captured \emph{via} some form of bisimulation.

% The notion we use in this paper is weak barbed bisimulation
% \cite{milner91polyadicpi}.

The notion we use in this paper is derived from weak barbed
bisimulation \cite{milner91polyadicpi}. 

\begin{definition}
An \emph{observation relation}, $\downarrow_{\mathcal N}$, over a set
of names, $\mathcal N$, is the smallest relation satisfying the rules
below.

\infrule[Out-barb]{y \in {\mathcal N}, \; x \nameeq y}
		  {\outputp{x}{v} \downarrow_{\mathcal N} x}
\infrule[Par-barb]{\mbox{$P\downarrow_{\mathcal N} x$ or $Q\downarrow_{\mathcal N} x$}}
		  {\binpar{P}{Q} \downarrow_{\mathcal N} x}

We write $P \Downarrow_{\mathcal N} x$ if there is $Q$ such that 
$P \wred Q$ and $Q \downarrow_{\mathcal N} x$.
\end{definition}

\begin{definition}
%\label{def.bbisim}
An  ${\mathcal N}$-\emph{barbed bisimulation} over a set of names, ${\mathcal N}$, is a symmetric binary relation 
${\mathcal S}_{\mathcal N}$ between agents such that $P\rel{S}_{\mathcal N}Q$ implies:
\begin{enumerate}
\item If $P \red P'$ then $Q \wred Q'$ and $P'\rel{S}_{\mathcal N} Q'$.
\item If $P\downarrow_{\mathcal N} x$, then $Q\Downarrow_{\mathcal N} x$.
\end{enumerate}
$P$ is ${\mathcal N}$-barbed bisimilar to $Q$, written
$P \wbbisim_{\mathcal N} Q$, if $P \rel{S}_{\mathcal N} Q$ for some ${\mathcal N}$-barbed bisimulation ${\mathcal S}_{\mathcal N}$.
\end{definition}

$\mathcal{R} \subseteq \pi \times \pi$

$P \mathcal{R} Q => \forall P'. P \red P' \Rightarrow \exists Q'. Q \red Q', P' \mathcal{R} Q'$

$P \vdash x \Rightarrow Q \vdash x$

\begin{mathpar}
  \inferrule*[lab=Out-barb]{x \nameeq y}{{y}!\langle{Q}\rangle \vdash x}
  \and
  \inferrule*[lab=Par-barb]{\mbox{$P\vdash x$ or $Q\vdash x$}}{\binpar{P}{Q} \vdash x}
\end{mathpar}

\subsubsection{Contexts}

One of the principle advantages of computational calculi like the
$\pi$-calculus is a well-defined notion of context,
contextual-equivalence and a correlation between
contextual-equivalence and notions of bisimulation. The notion of
context allows the decomposition of a process into (sub-)process and
its syntactic environment, its context. Thus, a context may be
thought of as a process with a ``hole'' (written $\Box$) in it. The
application of a context $M$ to a process $P$, written $M[P]$, is
tantamount to filling the hole in $M$ with $P$. In this paper we do
not need the full weight of this theory, but do make use of the notion
of context in the proof the main theorem. 

\begin{mathpar}
  \inferrule* [lab=summation] {} {{M_{M},M_{N}} \bc \Box \;|\; x.M_{A} \;|\; M_{M}+M_{N}}
  \and
  \inferrule* [lab=agent] {} {{M_{A}} \bc (\vec{x})M_{P} \;| \; \clift{P_0,\ldots,M_{P},\ldots,P_N}}
  \and \\
  \inferrule* [lab=process] {} {{M_{P}} \bc M_{N} \;| \;P|M_{P} }
\end{mathpar} 

\begin{mathpar}
  \inferrule* [lab=sychronization] {} {M_{N} \bc \Box \;|\; x?M_{F} \;|\; x!M_{C}}
  \and
  \inferrule* [lab=abstraction] {} {{M_{F}} \bc (x)M_{P} }
  \and
  \inferrule* [lab=concretion] {} {{M_{C}} \bc \langle M_{P} \rangle }
  \and \\
  \inferrule* [lab=process] {} {{M_{P}} \bc M_{N} \;| \;P|M_{P} }
\end{mathpar}

\begin{definition}[contextual application] Given a context $M$, and
  process $P$, we define the \emph{contextual application}, $M[P] :=
  M\{P/\Box\}$. That is, the contextual application of M to P is the
  substitution of $P$ for $\Box$ in $M$.
\end{definition}

$\meaningof{-} : L \to \mathcal{P}(\pi)$

\begin{mathpar}
  \inferrule* [lab=collection] {} {\meaningof{true} = \pi, \and \meaningof{~E} = \pi \setminus \meaningof{E}, \and \meaningof{E_{1} \& E_{2}} = \meaningof{E_{1}} \cap \meaningof{E_{2}}}
\end{mathpar}

\begin{mathpar}
  \inferrule* [lab=structure] {} {\meaningof{0} = \{ P \in \pi | P \equiv 0 \}, \and \\ \meaningof{E_1 | E_2} = \{ P \in \pi | P \equiv P_{1} | P_{2}, P_{1} \in \meaningof{E_{1}}, P_{2} \in \meaningof{E_2}\} }
\end{mathpar}

\begin{mathpar}
 \inferrule* [lab=behavior] {} {\meaningof{\langle a?b \rangle E} = \{ P \in \pi | P \equiv Q | u?(y)P', \\ \and \\\\ \and \\ \;\;\; u \in \meaningof{a}, \forall z.P'\{z/y\} \in \meaningof{E\{z/b\}}\}, \and \\ \meaningof{a!E} = \{ P \in \pi | P \equiv Q | x!\langle P' \rangle, x \in \meaningof{a} P' \in \meaningof{E}\} }
\end{mathpar}

\begin{mathpar}
 \inferrule* [lab=nominal] {} {\meaningof{\quotep{E}} = \{ \quotep{P} \in \quotep{\pi} | P \in \meaningof{E} \}, \and \meaningof{\quotep{P}} = \{ \quotep{Q} \in \quotep{\pi} | P \equiv Q \} \and \\ \meaningof{@\quotep{E}} = \{ P \in \pi | P \equiv @x, x \in \meaningof{E} \}}
\end{mathpar}

\begin{eqnarray*}
  \\
  \meaningof{-} : TS \to ST
\end{eqnarray*}

\begin{eqnarray*}
  \\
  L : TS \to ST
\end{eqnarray*}

\begin{eqnarray*}
  \\
  P \models E \iff P \in \meaningof{E}
\end{eqnarray*}

\begin{eqnarray*}
  P \approx_{L} Q \iff \forall E \in L. P \models E \iff Q \models E
\end{eqnarray*}

\begin{eqnarray*}
  P \approx_{K} Q
\end{eqnarray*}

\begin{eqnarray*}
  P \approx Q
\end{eqnarray*}

$\approx_{K} = \approx = \approx_{L}$

\subsubsection{Contextual duality}

Note that contexts extend the quotation operation to a family of
operations from processes to names. Given a context, $M$, we can
define a \emph{nominal context}, $\quotep{M}$ by $\quotep{M}[P] :=
\quotep{M[P]}$. To foreshadow what is to come we observe that these
operations enjoy a duality with processes very much like the duality
between vectors and maps from vectors to scalars.

Further, because the calculus is essentially higher-order, we have a
correspondence between contexts and processes. More specifically,
given a name $x$ and a context $M$ we can construct $M^{*}_{x}$ such
that 

\begin{mathpar}
  M^{*}_{x} | \lift{x}{P} \red M[P]
\end{mathpar}

namely,

\begin{mathpar}
  M^{*}_{x} := x?(u).M[\dropn{u}]
\end{mathpar}

The dependence of $M^{*}_{x}$ on a name makes it an abstraction, 

\begin{mathpar}
  M^{*} := (x)x?(u).M[\dropn{u}]
\end{mathpar}

\subsection{Additional notation}

It will sometimes be convenient to denote the process a name
quotes. We already have the notation $x = \quotep{P}$, but it will be
convenient to introduce an alternate notation, $\procn{x}$, when we
want to emphasize the connection to the use of the name. Note that, by
virtue of name equivalence, $\quotep{\procn{x}} \nameeq x$; so, the
notation is consistent with previous definitions.

Further, because names have structure it is possible to effect
substitutions on the basis of that structure. This means we need to
upgrade our notation for substitutions, which we accomplish by
adapting comprehension notation. Thus,

\begin{mathpar}
  P\{ y / x : x \in S \}
\end{mathpar}

is interpreted to mean the process derived from P by replacing (in a
capture-avoiding manner) each occurrence of $x$ in $S$ by $y$. For example,

\begin{mathpar}
  P\{ \quotep{\procn{x}|\procn{x}} / x : x \in \freenames{P} \}
\end{mathpar}

will replace each (occurrence) of a free name $x$ in $P$ by
$\quotep{\procn{x}|\procn{x}}$.

Also, we will avail ourselves of the notation $x^{L}$ and $x^{R}$ to
denote injections of a name into disjoint copies of the name
space. There are numerous ways to accomplish this. One example can be
found in \cite{MeredithR05}. This notation overloads to vectors of
names: $\vec{x}^{\pi} := (x_{i}^{\pi} \; : \; 0 \leq i < |\vec{x}| )$ where $\pi \in \{L,R\}$.

We also use $P^{\Box} := P|\Box$.

In \cite{MeredithR05} an interpretation of the new operator is
given. It turns out that there are several possible interpretations
all enjoying the requisite algebraic properties of the operator (see
\cite{milner91polyadicpi}). We will therefore make liberal use of
$(\nu\; \vec{x})P$.

% subsection the_syntax_and_semantics_of_the_notation_system (end)   

\input{qm2pi.qmops} 

\input{qm2pi.sterngerlach} 

\input{qm2pi.metric} 

% section concurrent_process_calculi (end)

%\input{qm2pi.proofsketch}

% section proof sketch (end)

%\input{qm2pi.slviaknots} 

% section spatial logic via knots (end)

\input{qm2pi.conclusion}

% section conclusion (end)

%\input{qm2pi.dtcodes} 

% section wiring algorithm (end)

\input{qm2pi.ack} 

% section acknowledgments (end)

\newpage


\bibliographystyle{plain}   
\bibliography{../../biblios/main.bib}

\input{qm2pi.rhodetails}

\end{document}

 

% section concurrent_process_calculi (end)

%\documentclass[12pt]{llncs}
%\documentclass{jktr}

\usepackage[pdftex]{hyperref}                   
\usepackage {listings}
\usepackage {mathpartir}
\usepackage{bcprules}
%\usepackage{listings}
                       
\usepackage{graphicx} 
%\usepackage[margins=2.5cm,nohead,nofoot]{geometry}
%\usepackage{geometry}
\usepackage{amsfonts}
\usepackage{amstext}
\usepackage{latexsym}
\usepackage{amssymb}
\usepackage{color}


%\include{myPreamble}
\include{qm2pi.local} 

%\ifpdf
%\usepackage[pdftex]{graphicx}
%\else
%\usepackage{graphicx}
%\fi

 % \ifpdf
%  \usepackage{pdfsync}
%  \if


%\title{Brief Article}
%\author{David F. Snyder}
%\author{L.G. Meredith}

%\address{Dept. of Math., Texas State University--San Marcos, San Marcos, TX 78666}
       
\pagestyle{empty}


\begin{document}

\lstset{language=[Objective]Caml,frame=shadowbox}

\input{qm2pi.front}

% section front matter (end)

\input{qm2pi.intro} 
 
% section introduction (end)

% \input{qm2pi.knotations} 

% section notation (end)

\input{qm2pi.process.calculi} 

% section concurrent_process_calculi_and_spatial_logics_ (end)
    
%\input{qm2pi.knots2pi} 

%\input{qm2pi.trefoil} 

%\input{qm2pi.mainthm} 

% subsection basic_interpretation (end)

%\input{qm2pi.rho.presentation} 
\subsection{The syntax and semantics of the notation system}\label{sub:the_syntax_and_semantics_of_the_notation_system} % (fold)

We now summarize a technical presentation of the calculus that
embodies our theory of dynamics. The typical presentation of such a
calculus follows the style of giving generators and relations on
them. The grammar, below, describing term constructors, freely
generates the set of processes, $\Proc$. This set is then quotiented
by a relation known as structural congruence and it is over this set
that the notion of dynamics is expressed. This presentation is
essentially that of \cite{MeredithR05} with the addition of
polyadicity and summation. For readability we have relegated some of
the technical subtleties to an appendix.

\subsubsection{Process grammar}\label{subsub:process_grammar}

\begin{mathpar}
  \inferrule* [lab=synchronization] {} {{M} \bc \pzero \;|\; x?F \;|\; x!C }
  \and
  \inferrule* [lab=abstraction] {} {{F} \bc (x)P}
  \and
  \inferrule* [lab=concretion] {} {{C} \bc \langle Q \rangle}
  \and
  \inferrule* [lab=process] {} {{P,Q} \bc M \;| \;P|Q \;|\; @{x}}
  \and
  \inferrule* [lab=name] {} {{x} \bc \quotep{P}}
\end{mathpar} 

Note that $\vec{x}$ (resp. $\vec{P}$) denotes a vector of names
(resp. processes) of length $|\vec{x}|$ (resp. $|\vec{P}|$). We adopt
the following useful abbreviations.

\begin{mathpar}
   x?(\vec{y}).P := x.(\vec{y})P \and  x\clift{\vec{P}} := x.\clift{\vec{P}}
   \and x!(y) := \lift{x}{\dropn{y}}
   \and \Pi_{i=0}^{n-1}P_i := P_0 | \ldots | P_{n-1}
\end{mathpar}

\subsubsection{Structural congruence}

\paragraph{Free and bound names and alpha-equivalence.} At the
core of structural equivalence is alpha-equivalence which identifies
process that are the same up to a change of variable. Formally, we
recognize the distinction between free and bound names. The free names
of a process, $\freenames{P}$, may be calculated recursively as
follows:

\begin{mathpar}
\freenames{\pzero} := \emptyset
  \and \\
  \freenames{x?(y).P} := \{ x \} \cup (\freenames{P} \setminus \{ y \})
  \and 
  \freenames{x!\langle P \rangle} := \{ x \} \cup \{ P \} 
  \and \\
  \freenames{P|Q} := \freenames{P} \cup \freenames{Q}
  \and \\
  \freenames{@{x}} := \{ x \}
\end{mathpar}

$\pi$
$\quotep{\pi}$

$\freenames{-} : \pi \to \mathcal{P}(\quotep{\pi})$

\begin{eqnarray*}
  \freenames{\pzero} & := & \emptyset \\
  \freenames{x?(y).P} & := & \{ x \} \cup (\freenames{P} \setminus \{ y \}) \\
  \freenames{x!\langle P \rangle} & := & \{ x \} \cup \{ P \} \\
  \freenames{P|Q} & := & \freenames{P} \cup \freenames{Q} \\
  \freenames{\dropn{x}} & := & \{ x \}
\end{eqnarray*}

The bound names of a process, $\boundnames{P}$, are those names occurring in $P$
that are not free. For example, in $x?(y).0$, the name $x$ is free, while $y$ is bound.

\begin{mathpar}
  \inferrule* [lab=monoidal-laws] {} { P|Q \equiv Q|P \and P|0 \equiv P \and P|(Q|R) \equiv (P|Q)|R }
\end{mathpar}

\begin{mathpar}
  \inferrule* [lab=alpha-equivalence] {} { (x)P \equiv (y)P\{y/x\} \and y \not\in \freenames{P} }
\end{mathpar}

\begin{definition}
Then two processes, $P,Q$, are alpha-equivalent if $P = Q\{\vec{y}/\vec{x}\}$ for
some $\vec{x} \in \boundnames{Q},\vec{y} \in \boundnames{P}$, where $Q\{\vec{y}/\vec{x}\}$
denotes the capture-avoiding substitution of $\vec{y}$ for $\vec{x}$ in $Q$.
\end{definition}

\begin{definition}
  The {\em structural congruence} \cite{SangiorgiWalker} , $\equiv$,
  between processes is the least congruence containing
  alpha-equivalence, satisfying the abelian monoid laws
  (associativity, commutativity and $\pzero$ as identity) for parallel
  composition $|$ and for summation $+$.
\end{definition}

\subsection{Name equivalence}

We take name equivalence, written $\nameeq$, to be the smallest
equivalence relation generated by the following rules.

\begin{mathpar}
\inferrule*[lab=Quote-drop]
{ }
{ \quotep{@{x}} \nameeq x }

\inferrule*[lab=Struct-equiv]
{ P \scong Q }
{ \quotep{P} \nameeq \quotep{Q} }
\end{mathpar}

The astute reader will have noticed that the mutual recursion of names
and processes imposes a mutual recursion on alpha-equivalence and
structural equivalence via name-equivalence. Fortunately, all of this
works out pleasantly and we may calculate in the natural way, free of
concern. The reader interested in the details is referred to the
appendix \ref{appendix:rho_details}.

\subsection{Substitution}

We use $\Proc$ for the set of processes, $\QProc$ for the set of
names, and $\id{\{}\vec{y} / \vec{x} \id{\}}$ to denote partial maps,
$s : \QProc \rightarrow \QProc$. A map, $s$ lifts, uniquely, to a map
on process terms, $\widehat{s} : \Proc \rightarrow \Proc$ by the
following equations.

\begin{mathpar}
  (0) \psubstp{Q}{P} := 0 \\
  (R \juxtap S) \psubstp{Q}{P}
  :=    
  (R)\psubstp{Q}{P} \juxtap (S) \psubstp{Q}{P} \\
  (x?(y).R) \psubstp{Q}{P}    
  :=    
  (x)\substp{Q}{P} (z)\concat( (R \psubstn{z}{y}) \psubstp{Q}{P} ) \\
  (\lift{x}{R}) \psubstp{Q}{P}  
  :=
  \lift{(x)\substp{Q}{P}}{ R \psubstp{Q}{P} } \\
%   (\dropn{x})  \psubstp{Q}{P}       
%   := 
%   \left\{ 
%     \begin{array}{ccc} 
%       \dropn{\quotep{Q}} & & x \nameeq \quotep{P} \\
%       \dropn{x} & & otherwise \\
%     \end{array}
%   \right. 
  (\dropn{x})  \psubstp{Q}{P}       
  := 
  \left\{ 
    \begin{array}{ccc} 
      Q & & x \nameeq \quotep{P} \\
      \dropn{x} & & otherwise \\
    \end{array}
  \right.
\end{mathpar}
 

where

\begin{eqnarray}
  (x)\id{\{} \lpquote Q \rpquote / \lpquote P \rpquote \id{\}}            = 
  \left\{ 
    \begin{array}{ccc}
      \lpquote Q \rpquote & & x \nameeq \lpquote P \rpquote \\
      x & & otherwise \\
    \end{array}
  \right. \nonumber
\end{eqnarray}

and $z$ is chosen distinct from $\quotep{P}$, $\quotep{Q}$, the free
names in $Q$, and all the names in $R$. Our $\alpha$-equivalence will
be built in the standard way from this substitution.

\begin{remark}\label{rem:no_self_referential_names}
  One consequence of these definitions is that $\forall P. \quotep{P}
  \not\in \freenames{P}$.
\end{remark}

\subsection{ Dynamic quote: an example }

Anticipating something of what's to come, consider applying the
substitution, $\widehat{\id{\{}u / z \id{\}}}$, to the following pair
of processes, $\lift{w}{y!(z)}$ and $w[ \lpquote y!(z) \rpquote ]$.

\begin{eqnarray}
	\lift{w}{y!(z)}\widehat{\id{\{}u / z \id{\}}}
		& = &
		\lift{w}{y!(u)} \nonumber\\
	w[ \lpquote y!(z) \rpquote ] \widehat{ \id{\{}u / z \id{\}} }
		& = &
		w[ \lpquote y!(z) \rpquote ] \nonumber
\end{eqnarray}

Because the body of the process between quotes is impervious to
substitution, we get radically different answers. In fact, by
examining the first process in an input context,
e.g. $x?(z).\lift{w}{y!(z)}$, we see that the process under the lift
operator may be shaped by prefixed inputs binding a name inside it. In
this sense, the lift operator will be seen as a way to dynamically
construct processes before reifying them as names.

Finally equipped with these standard features we can present the
dynamics of the calculus.

\subsubsection{Operational semantics} 

Finally, we introduce the computational dynamics. What marks these
algebras as distinct from other more traditionally studied algebraic
structures, e.g. vector spaces or polynomial rings, is the manner in
which dynamics is captured. In traditional structures, dynamics is typically
expressed through morphisms between such structures, as in linear maps
between vector spaces or morphisms between rings. In algebras
associated with the semantics of computation, the dynamics is
expressed as part of the algebraic structure itself, through a
reduction reduction relation typically denoted by $\red$. Below, we
give a recursive presentation of this relation for the calculus used
in the encoding.

$\red \subseteq \pi \times \pi$
$\red : \pi \to \mathcal{P}(\pi)$

\begin{mathpar}
  \inferrule* [lab=Comm] { \textsf{match}( x_{src}, x_{trgt} ) } { x_{trgt}?(y)P \; | \; x_{src}!\langle {Q} \rangle \red P\{\quotep{Q}/y}\} }
  \and \\
  \inferrule* [lab=Par] {{P} \red {P}'} {{{P} | {Q}} \red {{P}' | {Q}}}
  \and
  \inferrule* [lab=Equiv]{{{P} \scong {P}'} \andalso {{P}' \red {Q}'} \andalso {{Q}' \scong {Q}}}{{P} \red {Q}}
\end{mathpar}

\begin{eqnarray*}
  match_{\equiv} (\quotep{P},\quotep{Q}) & := & P \equiv Q \\
  match_{\dagger}(\quotep{P},\quotep{Q}) & := & \forall R. P|Q \red^{*} R => R \red^{*} 0 \\
  match_{K}(\quotep{P},\quotep{Q}) & := & K \mbox{ for some context } K
\end{eqnarray*}

$u?(x)P | u!\langle Q \rangle \red P\{\quotep{Q}/x\}$

%We write $\wred$ for $\red^*$, and $P\red$ if $\exists Q $ such that $ P \red Q$.
We write $P\red$ if $\exists Q $ such that $ P \red Q$ and $P\not\red$, otherwise.

\section{Replication}

As mentioned before, it is known that replication (and hence
recursion) can be implemented in a higher-order process algebra
\cite{SangiorgiWalker}. As our first example of calculation with the
machinery thus far presented we give the construction explicitly in
the {\rhoc}.

\begin{eqnarray}
	D_{x} & := & \prefix{x}{y}{(\binpar{\outputp{x}{y}}{@{y}})} \nonumber\\
	\bangp_{x}{P} & := & \binpar{{x}!\langle{\binpar{D_{x}}{P}}\rangle}{D_{x}} \nonumber
\end{eqnarray}

\begin{eqnarray}
	\bangp_{x}{P} & & \nonumber\\
	=
	& {x}!\langle{(\prefix{x}{y}{(\outputp{x}{y} | @{y})) | P}}\rangle 
	      | \prefix{x}{y}{(\outputp{x}{y} | @{y})} & \nonumber\\
	\red
	& (\outputp{x}{y} | @{y})\substn{\quotep{(\prefix{x}{y}{(@{y} | \outputp{x}{y})) | P}}}{y} & \nonumber\\
	=
	& \outputp{x}{\quotep{(\prefix{x}{y}{(\outputp{x}{y} | @{y})) | P}}}
	  | {(\prefix{x}{y}{(\outputp{x}{y} | @{y})) | P}} & \nonumber\\
	\red
	& \ldots & \nonumber\\
	\red^*
	& P | P | \ldots & \nonumber
\end{eqnarray}

Of course, this encoding, as an implementation, runs away, unfolding
$\bangp{P}$ eagerly. A lazier and more implementable replication
operator, restricted to input-guarded processes, may be obtained as follows.

\begin{eqnarray}
\bangp{\prefix{u}{v}{P}} 
	:= 
	\binpar{\lift{x}{\prefix{u}{v}{(\binpar{D(x)}{P})}}}{D(x)} \nonumber
\end{eqnarray}

\begin{remark}
  Note that the lazier definition still does not deal with summation
  or mixed summation (i.e. sums over input and output). The reader is
  invited to construct definitions of replication that deal with these
  features. 

  Further, the definitions are parameterized in a name, $x$. Can you,
  gentle reader, make a definition that eliminates this parameter and
  guarantees no accidental interaction between the replication
  machinery and the process being replicated -- i.e. no accidental
  sharing of names used by the process to get its work done and the
  name(s) used by the replication to effect copying. This latter
  revision of the definition of replication is crucial to obtaining
  the expected identity $!!P \sim !P$.
\end{remark}

\begin{remark}\label{rem:paradoxical_combinator}
  The reader familiar with the lambda calculus will have noticed the
  similarity between $D$ and the paradoxical combinator.

  [Ed. note: the existence of this seems to suggest we have to be more
  restrictive on the set of processes and names we admit if we are to
  support no-cloning.]
\end{remark}

\subsubsection{Bisimulation}

The computational dynamics gives rise to another kind of equivalence,
the equivalence of computational behavior. As previously mentioned
this is typically captured \emph{via} some form of bisimulation.

% The notion we use in this paper is weak barbed bisimulation
% \cite{milner91polyadicpi}.

The notion we use in this paper is derived from weak barbed
bisimulation \cite{milner91polyadicpi}. 

\begin{definition}
An \emph{observation relation}, $\downarrow_{\mathcal N}$, over a set
of names, $\mathcal N$, is the smallest relation satisfying the rules
below.

\infrule[Out-barb]{y \in {\mathcal N}, \; x \nameeq y}
		  {\outputp{x}{v} \downarrow_{\mathcal N} x}
\infrule[Par-barb]{\mbox{$P\downarrow_{\mathcal N} x$ or $Q\downarrow_{\mathcal N} x$}}
		  {\binpar{P}{Q} \downarrow_{\mathcal N} x}

We write $P \Downarrow_{\mathcal N} x$ if there is $Q$ such that 
$P \wred Q$ and $Q \downarrow_{\mathcal N} x$.
\end{definition}

\begin{definition}
%\label{def.bbisim}
An  ${\mathcal N}$-\emph{barbed bisimulation} over a set of names, ${\mathcal N}$, is a symmetric binary relation 
${\mathcal S}_{\mathcal N}$ between agents such that $P\rel{S}_{\mathcal N}Q$ implies:
\begin{enumerate}
\item If $P \red P'$ then $Q \wred Q'$ and $P'\rel{S}_{\mathcal N} Q'$.
\item If $P\downarrow_{\mathcal N} x$, then $Q\Downarrow_{\mathcal N} x$.
\end{enumerate}
$P$ is ${\mathcal N}$-barbed bisimilar to $Q$, written
$P \wbbisim_{\mathcal N} Q$, if $P \rel{S}_{\mathcal N} Q$ for some ${\mathcal N}$-barbed bisimulation ${\mathcal S}_{\mathcal N}$.
\end{definition}

$\mathcal{R} \subseteq \pi \times \pi$

$P \mathcal{R} Q => \forall P'. P \red P' \Rightarrow \exists Q'. Q \red Q', P' \mathcal{R} Q'$

$P \vdash x \Rightarrow Q \vdash x$

\begin{mathpar}
  \inferrule*[lab=Out-barb]{x \nameeq y}{{y}!\langle{Q}\rangle \vdash x}
  \and
  \inferrule*[lab=Par-barb]{\mbox{$P\vdash x$ or $Q\vdash x$}}{\binpar{P}{Q} \vdash x}
\end{mathpar}

\subsubsection{Contexts}

One of the principle advantages of computational calculi like the
$\pi$-calculus is a well-defined notion of context,
contextual-equivalence and a correlation between
contextual-equivalence and notions of bisimulation. The notion of
context allows the decomposition of a process into (sub-)process and
its syntactic environment, its context. Thus, a context may be
thought of as a process with a ``hole'' (written $\Box$) in it. The
application of a context $M$ to a process $P$, written $M[P]$, is
tantamount to filling the hole in $M$ with $P$. In this paper we do
not need the full weight of this theory, but do make use of the notion
of context in the proof the main theorem. 

\begin{mathpar}
  \inferrule* [lab=summation] {} {{M_{M},M_{N}} \bc \Box \;|\; x.M_{A} \;|\; M_{M}+M_{N}}
  \and
  \inferrule* [lab=agent] {} {{M_{A}} \bc (\vec{x})M_{P} \;| \; \clift{P_0,\ldots,M_{P},\ldots,P_N}}
  \and \\
  \inferrule* [lab=process] {} {{M_{P}} \bc M_{N} \;| \;P|M_{P} }
\end{mathpar} 

\begin{mathpar}
  \inferrule* [lab=sychronization] {} {M_{N} \bc \Box \;|\; x?M_{F} \;|\; x!M_{C}}
  \and
  \inferrule* [lab=abstraction] {} {{M_{F}} \bc (x)M_{P} }
  \and
  \inferrule* [lab=concretion] {} {{M_{C}} \bc \langle M_{P} \rangle }
  \and \\
  \inferrule* [lab=process] {} {{M_{P}} \bc M_{N} \;| \;P|M_{P} }
\end{mathpar}

\begin{definition}[contextual application] Given a context $M$, and
  process $P$, we define the \emph{contextual application}, $M[P] :=
  M\{P/\Box\}$. That is, the contextual application of M to P is the
  substitution of $P$ for $\Box$ in $M$.
\end{definition}

$\meaningof{-} : L \to \mathcal{P}(\pi)$

\begin{mathpar}
  \inferrule* [lab=collection] {} {\meaningof{true} = \pi, \and \meaningof{~E} = \pi \setminus \meaningof{E}, \and \meaningof{E_{1} \& E_{2}} = \meaningof{E_{1}} \cap \meaningof{E_{2}}}
\end{mathpar}

\begin{mathpar}
  \inferrule* [lab=structure] {} {\meaningof{0} = \{ P \in \pi | P \equiv 0 \}, \and \\ \meaningof{E_1 | E_2} = \{ P \in \pi | P \equiv P_{1} | P_{2}, P_{1} \in \meaningof{E_{1}}, P_{2} \in \meaningof{E_2}\} }
\end{mathpar}

\begin{mathpar}
 \inferrule* [lab=behavior] {} {\meaningof{\langle a?b \rangle E} = \{ P \in \pi | P \equiv Q | u?(y)P', \\ \and \\\\ \and \\ \;\;\; u \in \meaningof{a}, \forall z.P'\{z/y\} \in \meaningof{E\{z/b\}}\}, \and \\ \meaningof{a!E} = \{ P \in \pi | P \equiv Q | x!\langle P' \rangle, x \in \meaningof{a} P' \in \meaningof{E}\} }
\end{mathpar}

\begin{mathpar}
 \inferrule* [lab=nominal] {} {\meaningof{\quotep{E}} = \{ \quotep{P} \in \quotep{\pi} | P \in \meaningof{E} \}, \and \meaningof{\quotep{P}} = \{ \quotep{Q} \in \quotep{\pi} | P \equiv Q \} \and \\ \meaningof{@\quotep{E}} = \{ P \in \pi | P \equiv @x, x \in \meaningof{E} \}}
\end{mathpar}

\begin{eqnarray*}
  \\
  \meaningof{-} : TS \to ST
\end{eqnarray*}

\begin{eqnarray*}
  \\
  L : TS \to ST
\end{eqnarray*}

\begin{eqnarray*}
  \\
  P \models E \iff P \in \meaningof{E}
\end{eqnarray*}

\begin{eqnarray*}
  P \approx_{L} Q \iff \forall E \in L. P \models E \iff Q \models E
\end{eqnarray*}

\begin{eqnarray*}
  P \approx_{K} Q
\end{eqnarray*}

\begin{eqnarray*}
  P \approx Q
\end{eqnarray*}

$\approx_{K} = \approx = \approx_{L}$

\subsubsection{Contextual duality}

Note that contexts extend the quotation operation to a family of
operations from processes to names. Given a context, $M$, we can
define a \emph{nominal context}, $\quotep{M}$ by $\quotep{M}[P] :=
\quotep{M[P]}$. To foreshadow what is to come we observe that these
operations enjoy a duality with processes very much like the duality
between vectors and maps from vectors to scalars.

Further, because the calculus is essentially higher-order, we have a
correspondence between contexts and processes. More specifically,
given a name $x$ and a context $M$ we can construct $M^{*}_{x}$ such
that 

\begin{mathpar}
  M^{*}_{x} | \lift{x}{P} \red M[P]
\end{mathpar}

namely,

\begin{mathpar}
  M^{*}_{x} := x?(u).M[\dropn{u}]
\end{mathpar}

The dependence of $M^{*}_{x}$ on a name makes it an abstraction, 

\begin{mathpar}
  M^{*} := (x)x?(u).M[\dropn{u}]
\end{mathpar}

\subsection{Additional notation}

It will sometimes be convenient to denote the process a name
quotes. We already have the notation $x = \quotep{P}$, but it will be
convenient to introduce an alternate notation, $\procn{x}$, when we
want to emphasize the connection to the use of the name. Note that, by
virtue of name equivalence, $\quotep{\procn{x}} \nameeq x$; so, the
notation is consistent with previous definitions.

Further, because names have structure it is possible to effect
substitutions on the basis of that structure. This means we need to
upgrade our notation for substitutions, which we accomplish by
adapting comprehension notation. Thus,

\begin{mathpar}
  P\{ y / x : x \in S \}
\end{mathpar}

is interpreted to mean the process derived from P by replacing (in a
capture-avoiding manner) each occurrence of $x$ in $S$ by $y$. For example,

\begin{mathpar}
  P\{ \quotep{\procn{x}|\procn{x}} / x : x \in \freenames{P} \}
\end{mathpar}

will replace each (occurrence) of a free name $x$ in $P$ by
$\quotep{\procn{x}|\procn{x}}$.

Also, we will avail ourselves of the notation $x^{L}$ and $x^{R}$ to
denote injections of a name into disjoint copies of the name
space. There are numerous ways to accomplish this. One example can be
found in \cite{MeredithR05}. This notation overloads to vectors of
names: $\vec{x}^{\pi} := (x_{i}^{\pi} \; : \; 0 \leq i < |\vec{x}| )$ where $\pi \in \{L,R\}$.

We also use $P^{\Box} := P|\Box$.

In \cite{MeredithR05} an interpretation of the new operator is
given. It turns out that there are several possible interpretations
all enjoying the requisite algebraic properties of the operator (see
\cite{milner91polyadicpi}). We will therefore make liberal use of
$(\nu\; \vec{x})P$.

% subsection the_syntax_and_semantics_of_the_notation_system (end)   

\input{qm2pi.qmops} 

\input{qm2pi.sterngerlach} 

\input{qm2pi.metric} 

% section concurrent_process_calculi (end)

%\input{qm2pi.proofsketch}

% section proof sketch (end)

%\input{qm2pi.slviaknots} 

% section spatial logic via knots (end)

\input{qm2pi.conclusion}

% section conclusion (end)

%\input{qm2pi.dtcodes} 

% section wiring algorithm (end)

\input{qm2pi.ack} 

% section acknowledgments (end)

\newpage


\bibliographystyle{plain}   
\bibliography{../../biblios/main.bib}

\input{qm2pi.rhodetails}

\end{document}



% section proof sketch (end)

%\section{Unlikely characters: spatial logic for
  knots}\label{sub:characteristic_formulae} % (fold)

Associated to the mobile process calculi are a family of logics known
as the Hennessy-Milner logics. These logics typically enjoy a
semantics interpreting formulae as sets of processes that when
factored through the encoding outlined above allows an identification
of classes of knots with logical formulae. In the context of this
encoding the sub-family known as the spatial logics \cite{CairesC03}
\cite{CairesC04} \cite{Caires04} are of particular interest providing
several important features for expressing and reasoning about
properties (i.e. classes) of knots. We hint here at how this may be done.

%\begin{description}
%\item [structural connectives] 
\subsubsection{Structural connectives} The spatial logics enjoy
structural connectives corresponding, at the logical level, to the
parallel composition ($P | Q$) and new name ($(\nu \; x)P$)
connectives for processes. As illustrated in the examples below, these
connectives are extremely expressive given the shape of our encoding.
%\item [decideable satisfaction]

\subsubsection{Decideable satisfaction}
In \cite{Caires04} the satisfaction relation is shown to be decideable
for a rich class of processes. It further turns out that the image of
the our encoding is a proper subset of that class. This result
provides the basis for an algorithm by which to search for knots
enjoying a given property.
%\item [characteristic formulae]

\subsubsection{Characteristic formulae}
In the same paper \cite{Caires04} , Caires presents a means of calculating
characteristic formulae, selecting equivalence classes of processes
up to a pre--specified depth limit on the support set of names. Composed with our
encoding, this characteristic formula can be used to select
characteristic formulae for knots.
%\end{description}

\subsubsection{Spatial logic formulae}

The grammar below (segmented for comprehension) summarizes the syntax
of spatial logic formulae. We employ illustrative examples in the
sequel to provide an intuitive understanding of their meaning
referring the reader to \cite{Caires04} for a more detailed explication
of the semantics.

\begin{mathpar}
  \inferrule* [lab=boolean] {} {{A,B} \bc T \;|\; \neg A \;|\; A \wedge B \;|\; \eta = \eta'}
  \and
  \inferrule* [lab=spatial] {} {|\; \pzero \;|\; A | B \;|\; x \text{\textregistered} A \;|\; \forall x . A \;|\;  H x . A}
  \and
  \inferrule* [lab=behavioral] {} {|\; \alpha . A}
  \and 
  \inferrule* [lab=recursion] {} {|\; X(\vec{u}) \;|\; \mu X(\vec{u}) . A}
  \and
  \inferrule* [lab=action] {} {\alpha \bc \langle x?(\vec{y}) \rangle \;|\; \langle x!(\vec{y}) \rangle \;|\; \langle \tau \rangle}
  \and 
  \inferrule* [lab=name] {} {\eta \bc x \;|\; \tau}
\end{mathpar} 

% subsection characteristic_formulae (end)   	 

\subsection{Example formulae}\label{sub:example_formulae_} % (fold)

\subsubsection{Crossing as formula.}
% 
% \begin{align*}
%   \frac{d}{dx} \sin x &= \cos x 
%   & \frac{d}{dx} e^x &= e^x \\
%   \frac{d}{dx} \cos x &= - \sin x 
%   & \frac{d}{dx} \log x &= \frac{1}{x} \\
% \end{align*} 

\begin{align*}
 \mu C(x_{0},x_{1},y_{0},y_{1},u).&(\langle x_{0}?(z) \rangle(\langle u! \rangle\langle y_{1}!z \rangle C(x_{0},x_{1},y_{0},y_{1},u)) & \\
  & \wedge \langle y_{1}?(z) \rangle (\langle u! \rangle \langle x_{0}!z \rangle C(x_{0},x_{1},y_{0},y_{1},u)) & \\
  & \wedge \langle x_{1}?(z) \rangle (\langle u? \rangle \langle y_{0}!z \rangle C(x_{0},x_{1},y_{0},y_{1},u)) & \\
  & \wedge \langle y_{0}?(z) \rangle (\langle u? \rangle \langle x_{1}!z \rangle C(x_{0},x_{1},y_{0},y_{1},u))) &
\end{align*}

The lexicographical similarity between the shape of this formulae and
the shape of definition of the process representing a crossing reveals
the intuitive meaning of this formulae. It describes the capabilities
of a process that has the right to represent a crossing. For example
it picks out processes that may perform an input on the port $x_0$ in
its initial menu of capabilities. What differentiates the formula
from the process, however, is that the crossing process is the
smallest candidate to satisfy the formula. Infinitely many other
processes -- with internal behavior hidden behind this interface, so
to speak -- also satisfy this formula. Even this simple formula,
then, can be seen to open a new view onto knots, providing a
computational interpretation of \emph{virtual} knots.

Note that this formula is derived by hand. A similar formula can be
derived by employing Caires' calculation of characteristic formula
\cite{Caires04} to the process representing a crossing. In light of
this discussion, we let
$\meaningof{C}_{\phi}(x0,x1,y0,y1,u)$ denote a formula specifying the
dynamics we wish to capture of a crossing. To guarantee we preserve
the shape of the interface and minimal semantics we demand that
$\meaningof{C}_{\phi}(x0,x1,y0,y1,u) \Rightarrow
\textbf{C}(x0,x1,y0,y1,u)$ where $\textbf{C}(x0,x1,y0,y1,u)$ denotes
the formula above.
                            
\subsubsection{Crossing number constraints.}
The moral content of the context lemma (Lemma \ref{context}) is that the notion of
``locality'' in the Reidemeister moves is effectively captured by the
parallel composition operator of the process calculus. This intuition
extends through the logic. Given a formula,
$\meaningof{C}_{\phi}(x0,x1,y0,y1,u)$, we can use the structural
connectives to specify constraints on crossing numbers, such as at
least $n$ crossings, or exactly $n$ crossings.
\begin{mathpar}
  \inferrule* [lab=at-least-n] {} { K^{\geq n}_{\phi}(\vec{xs},\vec{ys}) := \Pi_{i=0}^{n-1} Hu . \meaningof{C}_{\phi}(xs_i,ys_i,u) | T }
  \and 
  \inferrule* [lab=exactly-n] {} { K^{= n}_{\phi}(\vec{xs},\vec{ys}) := \Pi_{i=0}^{n-1} Hu . \meaningof{C}_{\phi}(xs_i,ys_i,u) | \neg (\forall x_0,y_0,x_1,y_1,u . \meaningof{C}_{\phi}(x_0,y_0,x_1,y_1,u) | T) }
\end{mathpar}

To round out this section, recall that the encoding of an $n$-crossing
knot decomposes into a parallel composition of $n$ \emph{copies} of a
crossing process together with a wiring harness. To specify different
knot classes with the same crossing number amounts to specifying
logical constraints on the wiring harness. In the interest of space,
we defer examples to a forthcoming paper. Suffice it to say that both
the conditions ``alternating knot'' and ``contains the tangle
corresponding to 5/3'' are expressible. For example, it is possible to
calculate the characteristic formula of a process corresponding to the
tangle 5/3 and conjoin it into the classifying formula via the
composition connective of the logic.

Finally, we wish to observe that it is entirely within reason to
contemplate a more domain-specific version of spatial logic tailored
to the shape of processes in the image of the encoding. Such a
domain-specific logic would have a better claim to the title formal
language of knot properties.

% subsection example_formulae_ (end)

% section knots_as_processes (end) 

% section spatial logic via knots (end)

\section{Conclusions and future work}

\paragraph{Testing physical space}
You, gentle reader, may wonder why of all the theorems to be proved
given this set up we pick the one above. In some sense it's hardly
central to quantum mechanics. We see it as central in the sense that
it firmly establishes a notion of physical space arising from a notion
of the equivalence of behavior. Relating bisimulation to a metric is a
big step forward, but one is faced with interpreting the relationship
of that metric space to something more physical. Quantum mechanical
notions of ``physical'' space are still far from intuitive, but by
relating this idea of distance as testing to calculations that predict
physical circumstances we are making a not insignificant step forward
toward an understanding of the physical space we inhabit as
essentially dynamic.

\paragraph{Effectivity and simulation}
One of the observations we have yet to make is that the entire program
spelled out here is effective. We have built various interpreters for
the reflective calculus at work in this interpretation. In principle,
then, we can simulate quantum mechanics on a computer. The place where
the simulation may lose fidelity is the infinitely branching summation
for the annihilator.

In this connection i also want to point out that the evaluation style
calculation of the inner product puts the non-determinism of the
summation right at the heart of measurement. This suggests that
Milner's original reduction-based formulation of the dynamics of his
calculi in terms of sums was not just notationally suggestive of a
notion of measure-and-continue but captured some significant part of
the physics.

\paragraph{Quantum continuations}
In light of this last observation i want to point out that the
predominant account of quantum mechanics is missing a key aspect of a
truly compositional story of the physical situation. In a real lab,
when a measurement is made the observation can be made to feed into
another device that then makes another measurement conditioned on the
results of the first. This means that after the superposition was
collapsed the entire experimental set up remained in
superposition. While QM offers a means of writing this down it doesn't
quite line up well with the well-trodden formulation of computation
and continuation that we see so succinctly expressed in Milner's
calculi. This suggests that there might be advantages to this account
of dynamics waiting to be explored.

\paragraph{Quantum logic}
In this connection, we also note that by virtue of having the
Hennessy-Milner construction, we can pull the construction through the
interpretation of QM. This gives us a natural candidate for a quantum
logic that enjoys an extremely tight connection with it's domain of
interpretation, making the construction much less ad hoc (rather it is
the image of functor!).

\paragraph{Quantum probabiity}
i have questions about the basis of the interpretation of inner
product as probability amplitude. In particular, using which
axiomatization of probability theory does the notion of probability
amplitude earn the right to be so dubbed? In other words, where is the
proof that the operation for calculating a probability amplitude (and
then squaring) satisfies the axioms of what it means to calculate a
probability? Even if such a proof exists (i have yet to find it in the
literature), i wonder if it might not be possible to turn things on
their heads. Can we view the calculation of the probability amplitude
as an axiomatization of probability? If so, then the definition we
give for calculating probability amplitude may provide the basis for
an \emph{effective} theory of probability.

\paragraph{Quantum vs ``biological'' information}
Finally, i want to conclude with a more philosophical observation. At
a recent workshop in which QM was a predominant topic i noticed
something about quantum information. The speaker was giving a riveting
discussion of axiomatic QM and showing how properties of ``no
cloning'' and ``no deleting'' emerged as consequences of the
axiomatization. Theorems of this form are necessary to give us a sense
of confidence that our axioms characterize the physical theory. What
struck me, though, was that if quantum information is neither erasable
nor replicable it is markedly different from \emph{life}. Two of the
things we know about life is that

\begin{itemize}
  \item it ends;
  \item to gain some measure of persistence, to transcend it's
    finitude it is imminently copyable.
\end{itemize}

Both of these qualities are summarized succinctly in the aphorism: all
flesh is grass. For me these two kinds of ``information'' -- call them
quantum and biological -- are end points on a spectrum of strategies
for persistence. At one end, we have those curious entities that enjoy
uniqueness and permanence; at the other, we have those who in the face
of a certain end and an uncertain present make a go of passing
something on. To me one of the more remarkable aspects of the latter
strategy is that in the presence of noise (and certain features of
copying) we get a kind of dynamism, a chance for improvement against a
given persistent condition.

% subsection other_calculi_other_bisimulations_and_geometry_as_behavior (end)




% section conclusion (end)

%\documentclass[12pt]{llncs}
%\documentclass{jktr}

\usepackage[pdftex]{hyperref}                   
\usepackage {listings}
\usepackage {mathpartir}
\usepackage{bcprules}
%\usepackage{listings}
                       
\usepackage{graphicx} 
%\usepackage[margins=2.5cm,nohead,nofoot]{geometry}
%\usepackage{geometry}
\usepackage{amsfonts}
\usepackage{amstext}
\usepackage{latexsym}
\usepackage{amssymb}
\usepackage{color}


%\include{myPreamble}
\include{qm2pi.local} 

%\ifpdf
%\usepackage[pdftex]{graphicx}
%\else
%\usepackage{graphicx}
%\fi

 % \ifpdf
%  \usepackage{pdfsync}
%  \if


%\title{Brief Article}
%\author{David F. Snyder}
%\author{L.G. Meredith}

%\address{Dept. of Math., Texas State University--San Marcos, San Marcos, TX 78666}
       
\pagestyle{empty}


\begin{document}

\lstset{language=[Objective]Caml,frame=shadowbox}

\input{qm2pi.front}

% section front matter (end)

\input{qm2pi.intro} 
 
% section introduction (end)

% \input{qm2pi.knotations} 

% section notation (end)

\input{qm2pi.process.calculi} 

% section concurrent_process_calculi_and_spatial_logics_ (end)
    
%\input{qm2pi.knots2pi} 

%\input{qm2pi.trefoil} 

%\input{qm2pi.mainthm} 

% subsection basic_interpretation (end)

%\input{qm2pi.rho.presentation} 
\subsection{The syntax and semantics of the notation system}\label{sub:the_syntax_and_semantics_of_the_notation_system} % (fold)

We now summarize a technical presentation of the calculus that
embodies our theory of dynamics. The typical presentation of such a
calculus follows the style of giving generators and relations on
them. The grammar, below, describing term constructors, freely
generates the set of processes, $\Proc$. This set is then quotiented
by a relation known as structural congruence and it is over this set
that the notion of dynamics is expressed. This presentation is
essentially that of \cite{MeredithR05} with the addition of
polyadicity and summation. For readability we have relegated some of
the technical subtleties to an appendix.

\subsubsection{Process grammar}\label{subsub:process_grammar}

\begin{mathpar}
  \inferrule* [lab=synchronization] {} {{M} \bc \pzero \;|\; x?F \;|\; x!C }
  \and
  \inferrule* [lab=abstraction] {} {{F} \bc (x)P}
  \and
  \inferrule* [lab=concretion] {} {{C} \bc \langle Q \rangle}
  \and
  \inferrule* [lab=process] {} {{P,Q} \bc M \;| \;P|Q \;|\; @{x}}
  \and
  \inferrule* [lab=name] {} {{x} \bc \quotep{P}}
\end{mathpar} 

Note that $\vec{x}$ (resp. $\vec{P}$) denotes a vector of names
(resp. processes) of length $|\vec{x}|$ (resp. $|\vec{P}|$). We adopt
the following useful abbreviations.

\begin{mathpar}
   x?(\vec{y}).P := x.(\vec{y})P \and  x\clift{\vec{P}} := x.\clift{\vec{P}}
   \and x!(y) := \lift{x}{\dropn{y}}
   \and \Pi_{i=0}^{n-1}P_i := P_0 | \ldots | P_{n-1}
\end{mathpar}

\subsubsection{Structural congruence}

\paragraph{Free and bound names and alpha-equivalence.} At the
core of structural equivalence is alpha-equivalence which identifies
process that are the same up to a change of variable. Formally, we
recognize the distinction between free and bound names. The free names
of a process, $\freenames{P}$, may be calculated recursively as
follows:

\begin{mathpar}
\freenames{\pzero} := \emptyset
  \and \\
  \freenames{x?(y).P} := \{ x \} \cup (\freenames{P} \setminus \{ y \})
  \and 
  \freenames{x!\langle P \rangle} := \{ x \} \cup \{ P \} 
  \and \\
  \freenames{P|Q} := \freenames{P} \cup \freenames{Q}
  \and \\
  \freenames{@{x}} := \{ x \}
\end{mathpar}

$\pi$
$\quotep{\pi}$

$\freenames{-} : \pi \to \mathcal{P}(\quotep{\pi})$

\begin{eqnarray*}
  \freenames{\pzero} & := & \emptyset \\
  \freenames{x?(y).P} & := & \{ x \} \cup (\freenames{P} \setminus \{ y \}) \\
  \freenames{x!\langle P \rangle} & := & \{ x \} \cup \{ P \} \\
  \freenames{P|Q} & := & \freenames{P} \cup \freenames{Q} \\
  \freenames{\dropn{x}} & := & \{ x \}
\end{eqnarray*}

The bound names of a process, $\boundnames{P}$, are those names occurring in $P$
that are not free. For example, in $x?(y).0$, the name $x$ is free, while $y$ is bound.

\begin{mathpar}
  \inferrule* [lab=monoidal-laws] {} { P|Q \equiv Q|P \and P|0 \equiv P \and P|(Q|R) \equiv (P|Q)|R }
\end{mathpar}

\begin{mathpar}
  \inferrule* [lab=alpha-equivalence] {} { (x)P \equiv (y)P\{y/x\} \and y \not\in \freenames{P} }
\end{mathpar}

\begin{definition}
Then two processes, $P,Q$, are alpha-equivalent if $P = Q\{\vec{y}/\vec{x}\}$ for
some $\vec{x} \in \boundnames{Q},\vec{y} \in \boundnames{P}$, where $Q\{\vec{y}/\vec{x}\}$
denotes the capture-avoiding substitution of $\vec{y}$ for $\vec{x}$ in $Q$.
\end{definition}

\begin{definition}
  The {\em structural congruence} \cite{SangiorgiWalker} , $\equiv$,
  between processes is the least congruence containing
  alpha-equivalence, satisfying the abelian monoid laws
  (associativity, commutativity and $\pzero$ as identity) for parallel
  composition $|$ and for summation $+$.
\end{definition}

\subsection{Name equivalence}

We take name equivalence, written $\nameeq$, to be the smallest
equivalence relation generated by the following rules.

\begin{mathpar}
\inferrule*[lab=Quote-drop]
{ }
{ \quotep{@{x}} \nameeq x }

\inferrule*[lab=Struct-equiv]
{ P \scong Q }
{ \quotep{P} \nameeq \quotep{Q} }
\end{mathpar}

The astute reader will have noticed that the mutual recursion of names
and processes imposes a mutual recursion on alpha-equivalence and
structural equivalence via name-equivalence. Fortunately, all of this
works out pleasantly and we may calculate in the natural way, free of
concern. The reader interested in the details is referred to the
appendix \ref{appendix:rho_details}.

\subsection{Substitution}

We use $\Proc$ for the set of processes, $\QProc$ for the set of
names, and $\id{\{}\vec{y} / \vec{x} \id{\}}$ to denote partial maps,
$s : \QProc \rightarrow \QProc$. A map, $s$ lifts, uniquely, to a map
on process terms, $\widehat{s} : \Proc \rightarrow \Proc$ by the
following equations.

\begin{mathpar}
  (0) \psubstp{Q}{P} := 0 \\
  (R \juxtap S) \psubstp{Q}{P}
  :=    
  (R)\psubstp{Q}{P} \juxtap (S) \psubstp{Q}{P} \\
  (x?(y).R) \psubstp{Q}{P}    
  :=    
  (x)\substp{Q}{P} (z)\concat( (R \psubstn{z}{y}) \psubstp{Q}{P} ) \\
  (\lift{x}{R}) \psubstp{Q}{P}  
  :=
  \lift{(x)\substp{Q}{P}}{ R \psubstp{Q}{P} } \\
%   (\dropn{x})  \psubstp{Q}{P}       
%   := 
%   \left\{ 
%     \begin{array}{ccc} 
%       \dropn{\quotep{Q}} & & x \nameeq \quotep{P} \\
%       \dropn{x} & & otherwise \\
%     \end{array}
%   \right. 
  (\dropn{x})  \psubstp{Q}{P}       
  := 
  \left\{ 
    \begin{array}{ccc} 
      Q & & x \nameeq \quotep{P} \\
      \dropn{x} & & otherwise \\
    \end{array}
  \right.
\end{mathpar}
 

where

\begin{eqnarray}
  (x)\id{\{} \lpquote Q \rpquote / \lpquote P \rpquote \id{\}}            = 
  \left\{ 
    \begin{array}{ccc}
      \lpquote Q \rpquote & & x \nameeq \lpquote P \rpquote \\
      x & & otherwise \\
    \end{array}
  \right. \nonumber
\end{eqnarray}

and $z$ is chosen distinct from $\quotep{P}$, $\quotep{Q}$, the free
names in $Q$, and all the names in $R$. Our $\alpha$-equivalence will
be built in the standard way from this substitution.

\begin{remark}\label{rem:no_self_referential_names}
  One consequence of these definitions is that $\forall P. \quotep{P}
  \not\in \freenames{P}$.
\end{remark}

\subsection{ Dynamic quote: an example }

Anticipating something of what's to come, consider applying the
substitution, $\widehat{\id{\{}u / z \id{\}}}$, to the following pair
of processes, $\lift{w}{y!(z)}$ and $w[ \lpquote y!(z) \rpquote ]$.

\begin{eqnarray}
	\lift{w}{y!(z)}\widehat{\id{\{}u / z \id{\}}}
		& = &
		\lift{w}{y!(u)} \nonumber\\
	w[ \lpquote y!(z) \rpquote ] \widehat{ \id{\{}u / z \id{\}} }
		& = &
		w[ \lpquote y!(z) \rpquote ] \nonumber
\end{eqnarray}

Because the body of the process between quotes is impervious to
substitution, we get radically different answers. In fact, by
examining the first process in an input context,
e.g. $x?(z).\lift{w}{y!(z)}$, we see that the process under the lift
operator may be shaped by prefixed inputs binding a name inside it. In
this sense, the lift operator will be seen as a way to dynamically
construct processes before reifying them as names.

Finally equipped with these standard features we can present the
dynamics of the calculus.

\subsubsection{Operational semantics} 

Finally, we introduce the computational dynamics. What marks these
algebras as distinct from other more traditionally studied algebraic
structures, e.g. vector spaces or polynomial rings, is the manner in
which dynamics is captured. In traditional structures, dynamics is typically
expressed through morphisms between such structures, as in linear maps
between vector spaces or morphisms between rings. In algebras
associated with the semantics of computation, the dynamics is
expressed as part of the algebraic structure itself, through a
reduction reduction relation typically denoted by $\red$. Below, we
give a recursive presentation of this relation for the calculus used
in the encoding.

$\red \subseteq \pi \times \pi$
$\red : \pi \to \mathcal{P}(\pi)$

\begin{mathpar}
  \inferrule* [lab=Comm] { \textsf{match}( x_{src}, x_{trgt} ) } { x_{trgt}?(y)P \; | \; x_{src}!\langle {Q} \rangle \red P\{\quotep{Q}/y}\} }
  \and \\
  \inferrule* [lab=Par] {{P} \red {P}'} {{{P} | {Q}} \red {{P}' | {Q}}}
  \and
  \inferrule* [lab=Equiv]{{{P} \scong {P}'} \andalso {{P}' \red {Q}'} \andalso {{Q}' \scong {Q}}}{{P} \red {Q}}
\end{mathpar}

\begin{eqnarray*}
  match_{\equiv} (\quotep{P},\quotep{Q}) & := & P \equiv Q \\
  match_{\dagger}(\quotep{P},\quotep{Q}) & := & \forall R. P|Q \red^{*} R => R \red^{*} 0 \\
  match_{K}(\quotep{P},\quotep{Q}) & := & K \mbox{ for some context } K
\end{eqnarray*}

$u?(x)P | u!\langle Q \rangle \red P\{\quotep{Q}/x\}$

%We write $\wred$ for $\red^*$, and $P\red$ if $\exists Q $ such that $ P \red Q$.
We write $P\red$ if $\exists Q $ such that $ P \red Q$ and $P\not\red$, otherwise.

\section{Replication}

As mentioned before, it is known that replication (and hence
recursion) can be implemented in a higher-order process algebra
\cite{SangiorgiWalker}. As our first example of calculation with the
machinery thus far presented we give the construction explicitly in
the {\rhoc}.

\begin{eqnarray}
	D_{x} & := & \prefix{x}{y}{(\binpar{\outputp{x}{y}}{@{y}})} \nonumber\\
	\bangp_{x}{P} & := & \binpar{{x}!\langle{\binpar{D_{x}}{P}}\rangle}{D_{x}} \nonumber
\end{eqnarray}

\begin{eqnarray}
	\bangp_{x}{P} & & \nonumber\\
	=
	& {x}!\langle{(\prefix{x}{y}{(\outputp{x}{y} | @{y})) | P}}\rangle 
	      | \prefix{x}{y}{(\outputp{x}{y} | @{y})} & \nonumber\\
	\red
	& (\outputp{x}{y} | @{y})\substn{\quotep{(\prefix{x}{y}{(@{y} | \outputp{x}{y})) | P}}}{y} & \nonumber\\
	=
	& \outputp{x}{\quotep{(\prefix{x}{y}{(\outputp{x}{y} | @{y})) | P}}}
	  | {(\prefix{x}{y}{(\outputp{x}{y} | @{y})) | P}} & \nonumber\\
	\red
	& \ldots & \nonumber\\
	\red^*
	& P | P | \ldots & \nonumber
\end{eqnarray}

Of course, this encoding, as an implementation, runs away, unfolding
$\bangp{P}$ eagerly. A lazier and more implementable replication
operator, restricted to input-guarded processes, may be obtained as follows.

\begin{eqnarray}
\bangp{\prefix{u}{v}{P}} 
	:= 
	\binpar{\lift{x}{\prefix{u}{v}{(\binpar{D(x)}{P})}}}{D(x)} \nonumber
\end{eqnarray}

\begin{remark}
  Note that the lazier definition still does not deal with summation
  or mixed summation (i.e. sums over input and output). The reader is
  invited to construct definitions of replication that deal with these
  features. 

  Further, the definitions are parameterized in a name, $x$. Can you,
  gentle reader, make a definition that eliminates this parameter and
  guarantees no accidental interaction between the replication
  machinery and the process being replicated -- i.e. no accidental
  sharing of names used by the process to get its work done and the
  name(s) used by the replication to effect copying. This latter
  revision of the definition of replication is crucial to obtaining
  the expected identity $!!P \sim !P$.
\end{remark}

\begin{remark}\label{rem:paradoxical_combinator}
  The reader familiar with the lambda calculus will have noticed the
  similarity between $D$ and the paradoxical combinator.

  [Ed. note: the existence of this seems to suggest we have to be more
  restrictive on the set of processes and names we admit if we are to
  support no-cloning.]
\end{remark}

\subsubsection{Bisimulation}

The computational dynamics gives rise to another kind of equivalence,
the equivalence of computational behavior. As previously mentioned
this is typically captured \emph{via} some form of bisimulation.

% The notion we use in this paper is weak barbed bisimulation
% \cite{milner91polyadicpi}.

The notion we use in this paper is derived from weak barbed
bisimulation \cite{milner91polyadicpi}. 

\begin{definition}
An \emph{observation relation}, $\downarrow_{\mathcal N}$, over a set
of names, $\mathcal N$, is the smallest relation satisfying the rules
below.

\infrule[Out-barb]{y \in {\mathcal N}, \; x \nameeq y}
		  {\outputp{x}{v} \downarrow_{\mathcal N} x}
\infrule[Par-barb]{\mbox{$P\downarrow_{\mathcal N} x$ or $Q\downarrow_{\mathcal N} x$}}
		  {\binpar{P}{Q} \downarrow_{\mathcal N} x}

We write $P \Downarrow_{\mathcal N} x$ if there is $Q$ such that 
$P \wred Q$ and $Q \downarrow_{\mathcal N} x$.
\end{definition}

\begin{definition}
%\label{def.bbisim}
An  ${\mathcal N}$-\emph{barbed bisimulation} over a set of names, ${\mathcal N}$, is a symmetric binary relation 
${\mathcal S}_{\mathcal N}$ between agents such that $P\rel{S}_{\mathcal N}Q$ implies:
\begin{enumerate}
\item If $P \red P'$ then $Q \wred Q'$ and $P'\rel{S}_{\mathcal N} Q'$.
\item If $P\downarrow_{\mathcal N} x$, then $Q\Downarrow_{\mathcal N} x$.
\end{enumerate}
$P$ is ${\mathcal N}$-barbed bisimilar to $Q$, written
$P \wbbisim_{\mathcal N} Q$, if $P \rel{S}_{\mathcal N} Q$ for some ${\mathcal N}$-barbed bisimulation ${\mathcal S}_{\mathcal N}$.
\end{definition}

$\mathcal{R} \subseteq \pi \times \pi$

$P \mathcal{R} Q => \forall P'. P \red P' \Rightarrow \exists Q'. Q \red Q', P' \mathcal{R} Q'$

$P \vdash x \Rightarrow Q \vdash x$

\begin{mathpar}
  \inferrule*[lab=Out-barb]{x \nameeq y}{{y}!\langle{Q}\rangle \vdash x}
  \and
  \inferrule*[lab=Par-barb]{\mbox{$P\vdash x$ or $Q\vdash x$}}{\binpar{P}{Q} \vdash x}
\end{mathpar}

\subsubsection{Contexts}

One of the principle advantages of computational calculi like the
$\pi$-calculus is a well-defined notion of context,
contextual-equivalence and a correlation between
contextual-equivalence and notions of bisimulation. The notion of
context allows the decomposition of a process into (sub-)process and
its syntactic environment, its context. Thus, a context may be
thought of as a process with a ``hole'' (written $\Box$) in it. The
application of a context $M$ to a process $P$, written $M[P]$, is
tantamount to filling the hole in $M$ with $P$. In this paper we do
not need the full weight of this theory, but do make use of the notion
of context in the proof the main theorem. 

\begin{mathpar}
  \inferrule* [lab=summation] {} {{M_{M},M_{N}} \bc \Box \;|\; x.M_{A} \;|\; M_{M}+M_{N}}
  \and
  \inferrule* [lab=agent] {} {{M_{A}} \bc (\vec{x})M_{P} \;| \; \clift{P_0,\ldots,M_{P},\ldots,P_N}}
  \and \\
  \inferrule* [lab=process] {} {{M_{P}} \bc M_{N} \;| \;P|M_{P} }
\end{mathpar} 

\begin{mathpar}
  \inferrule* [lab=sychronization] {} {M_{N} \bc \Box \;|\; x?M_{F} \;|\; x!M_{C}}
  \and
  \inferrule* [lab=abstraction] {} {{M_{F}} \bc (x)M_{P} }
  \and
  \inferrule* [lab=concretion] {} {{M_{C}} \bc \langle M_{P} \rangle }
  \and \\
  \inferrule* [lab=process] {} {{M_{P}} \bc M_{N} \;| \;P|M_{P} }
\end{mathpar}

\begin{definition}[contextual application] Given a context $M$, and
  process $P$, we define the \emph{contextual application}, $M[P] :=
  M\{P/\Box\}$. That is, the contextual application of M to P is the
  substitution of $P$ for $\Box$ in $M$.
\end{definition}

$\meaningof{-} : L \to \mathcal{P}(\pi)$

\begin{mathpar}
  \inferrule* [lab=collection] {} {\meaningof{true} = \pi, \and \meaningof{~E} = \pi \setminus \meaningof{E}, \and \meaningof{E_{1} \& E_{2}} = \meaningof{E_{1}} \cap \meaningof{E_{2}}}
\end{mathpar}

\begin{mathpar}
  \inferrule* [lab=structure] {} {\meaningof{0} = \{ P \in \pi | P \equiv 0 \}, \and \\ \meaningof{E_1 | E_2} = \{ P \in \pi | P \equiv P_{1} | P_{2}, P_{1} \in \meaningof{E_{1}}, P_{2} \in \meaningof{E_2}\} }
\end{mathpar}

\begin{mathpar}
 \inferrule* [lab=behavior] {} {\meaningof{\langle a?b \rangle E} = \{ P \in \pi | P \equiv Q | u?(y)P', \\ \and \\\\ \and \\ \;\;\; u \in \meaningof{a}, \forall z.P'\{z/y\} \in \meaningof{E\{z/b\}}\}, \and \\ \meaningof{a!E} = \{ P \in \pi | P \equiv Q | x!\langle P' \rangle, x \in \meaningof{a} P' \in \meaningof{E}\} }
\end{mathpar}

\begin{mathpar}
 \inferrule* [lab=nominal] {} {\meaningof{\quotep{E}} = \{ \quotep{P} \in \quotep{\pi} | P \in \meaningof{E} \}, \and \meaningof{\quotep{P}} = \{ \quotep{Q} \in \quotep{\pi} | P \equiv Q \} \and \\ \meaningof{@\quotep{E}} = \{ P \in \pi | P \equiv @x, x \in \meaningof{E} \}}
\end{mathpar}

\begin{eqnarray*}
  \\
  \meaningof{-} : TS \to ST
\end{eqnarray*}

\begin{eqnarray*}
  \\
  L : TS \to ST
\end{eqnarray*}

\begin{eqnarray*}
  \\
  P \models E \iff P \in \meaningof{E}
\end{eqnarray*}

\begin{eqnarray*}
  P \approx_{L} Q \iff \forall E \in L. P \models E \iff Q \models E
\end{eqnarray*}

\begin{eqnarray*}
  P \approx_{K} Q
\end{eqnarray*}

\begin{eqnarray*}
  P \approx Q
\end{eqnarray*}

$\approx_{K} = \approx = \approx_{L}$

\subsubsection{Contextual duality}

Note that contexts extend the quotation operation to a family of
operations from processes to names. Given a context, $M$, we can
define a \emph{nominal context}, $\quotep{M}$ by $\quotep{M}[P] :=
\quotep{M[P]}$. To foreshadow what is to come we observe that these
operations enjoy a duality with processes very much like the duality
between vectors and maps from vectors to scalars.

Further, because the calculus is essentially higher-order, we have a
correspondence between contexts and processes. More specifically,
given a name $x$ and a context $M$ we can construct $M^{*}_{x}$ such
that 

\begin{mathpar}
  M^{*}_{x} | \lift{x}{P} \red M[P]
\end{mathpar}

namely,

\begin{mathpar}
  M^{*}_{x} := x?(u).M[\dropn{u}]
\end{mathpar}

The dependence of $M^{*}_{x}$ on a name makes it an abstraction, 

\begin{mathpar}
  M^{*} := (x)x?(u).M[\dropn{u}]
\end{mathpar}

\subsection{Additional notation}

It will sometimes be convenient to denote the process a name
quotes. We already have the notation $x = \quotep{P}$, but it will be
convenient to introduce an alternate notation, $\procn{x}$, when we
want to emphasize the connection to the use of the name. Note that, by
virtue of name equivalence, $\quotep{\procn{x}} \nameeq x$; so, the
notation is consistent with previous definitions.

Further, because names have structure it is possible to effect
substitutions on the basis of that structure. This means we need to
upgrade our notation for substitutions, which we accomplish by
adapting comprehension notation. Thus,

\begin{mathpar}
  P\{ y / x : x \in S \}
\end{mathpar}

is interpreted to mean the process derived from P by replacing (in a
capture-avoiding manner) each occurrence of $x$ in $S$ by $y$. For example,

\begin{mathpar}
  P\{ \quotep{\procn{x}|\procn{x}} / x : x \in \freenames{P} \}
\end{mathpar}

will replace each (occurrence) of a free name $x$ in $P$ by
$\quotep{\procn{x}|\procn{x}}$.

Also, we will avail ourselves of the notation $x^{L}$ and $x^{R}$ to
denote injections of a name into disjoint copies of the name
space. There are numerous ways to accomplish this. One example can be
found in \cite{MeredithR05}. This notation overloads to vectors of
names: $\vec{x}^{\pi} := (x_{i}^{\pi} \; : \; 0 \leq i < |\vec{x}| )$ where $\pi \in \{L,R\}$.

We also use $P^{\Box} := P|\Box$.

In \cite{MeredithR05} an interpretation of the new operator is
given. It turns out that there are several possible interpretations
all enjoying the requisite algebraic properties of the operator (see
\cite{milner91polyadicpi}). We will therefore make liberal use of
$(\nu\; \vec{x})P$.

% subsection the_syntax_and_semantics_of_the_notation_system (end)   

\input{qm2pi.qmops} 

\input{qm2pi.sterngerlach} 

\input{qm2pi.metric} 

% section concurrent_process_calculi (end)

%\input{qm2pi.proofsketch}

% section proof sketch (end)

%\input{qm2pi.slviaknots} 

% section spatial logic via knots (end)

\input{qm2pi.conclusion}

% section conclusion (end)

%\input{qm2pi.dtcodes} 

% section wiring algorithm (end)

\input{qm2pi.ack} 

% section acknowledgments (end)

\newpage


\bibliographystyle{plain}   
\bibliography{../../biblios/main.bib}

\input{qm2pi.rhodetails}

\end{document}

 

% section wiring algorithm (end)

\documentclass[12pt]{llncs}
%\documentclass{jktr}

\usepackage[pdftex]{hyperref}                   
\usepackage {listings}
\usepackage {mathpartir}
\usepackage{bcprules}
%\usepackage{listings}
                       
\usepackage{graphicx} 
%\usepackage[margins=2.5cm,nohead,nofoot]{geometry}
%\usepackage{geometry}
\usepackage{amsfonts}
\usepackage{amstext}
\usepackage{latexsym}
\usepackage{amssymb}
\usepackage{color}


%\include{myPreamble}
\include{qm2pi.local} 

%\ifpdf
%\usepackage[pdftex]{graphicx}
%\else
%\usepackage{graphicx}
%\fi

 % \ifpdf
%  \usepackage{pdfsync}
%  \if


%\title{Brief Article}
%\author{David F. Snyder}
%\author{L.G. Meredith}

%\address{Dept. of Math., Texas State University--San Marcos, San Marcos, TX 78666}
       
\pagestyle{empty}


\begin{document}

\lstset{language=[Objective]Caml,frame=shadowbox}

\input{qm2pi.front}

% section front matter (end)

\input{qm2pi.intro} 
 
% section introduction (end)

% \input{qm2pi.knotations} 

% section notation (end)

\input{qm2pi.process.calculi} 

% section concurrent_process_calculi_and_spatial_logics_ (end)
    
%\input{qm2pi.knots2pi} 

%\input{qm2pi.trefoil} 

%\input{qm2pi.mainthm} 

% subsection basic_interpretation (end)

%\input{qm2pi.rho.presentation} 
\subsection{The syntax and semantics of the notation system}\label{sub:the_syntax_and_semantics_of_the_notation_system} % (fold)

We now summarize a technical presentation of the calculus that
embodies our theory of dynamics. The typical presentation of such a
calculus follows the style of giving generators and relations on
them. The grammar, below, describing term constructors, freely
generates the set of processes, $\Proc$. This set is then quotiented
by a relation known as structural congruence and it is over this set
that the notion of dynamics is expressed. This presentation is
essentially that of \cite{MeredithR05} with the addition of
polyadicity and summation. For readability we have relegated some of
the technical subtleties to an appendix.

\subsubsection{Process grammar}\label{subsub:process_grammar}

\begin{mathpar}
  \inferrule* [lab=synchronization] {} {{M} \bc \pzero \;|\; x?F \;|\; x!C }
  \and
  \inferrule* [lab=abstraction] {} {{F} \bc (x)P}
  \and
  \inferrule* [lab=concretion] {} {{C} \bc \langle Q \rangle}
  \and
  \inferrule* [lab=process] {} {{P,Q} \bc M \;| \;P|Q \;|\; @{x}}
  \and
  \inferrule* [lab=name] {} {{x} \bc \quotep{P}}
\end{mathpar} 

Note that $\vec{x}$ (resp. $\vec{P}$) denotes a vector of names
(resp. processes) of length $|\vec{x}|$ (resp. $|\vec{P}|$). We adopt
the following useful abbreviations.

\begin{mathpar}
   x?(\vec{y}).P := x.(\vec{y})P \and  x\clift{\vec{P}} := x.\clift{\vec{P}}
   \and x!(y) := \lift{x}{\dropn{y}}
   \and \Pi_{i=0}^{n-1}P_i := P_0 | \ldots | P_{n-1}
\end{mathpar}

\subsubsection{Structural congruence}

\paragraph{Free and bound names and alpha-equivalence.} At the
core of structural equivalence is alpha-equivalence which identifies
process that are the same up to a change of variable. Formally, we
recognize the distinction between free and bound names. The free names
of a process, $\freenames{P}$, may be calculated recursively as
follows:

\begin{mathpar}
\freenames{\pzero} := \emptyset
  \and \\
  \freenames{x?(y).P} := \{ x \} \cup (\freenames{P} \setminus \{ y \})
  \and 
  \freenames{x!\langle P \rangle} := \{ x \} \cup \{ P \} 
  \and \\
  \freenames{P|Q} := \freenames{P} \cup \freenames{Q}
  \and \\
  \freenames{@{x}} := \{ x \}
\end{mathpar}

$\pi$
$\quotep{\pi}$

$\freenames{-} : \pi \to \mathcal{P}(\quotep{\pi})$

\begin{eqnarray*}
  \freenames{\pzero} & := & \emptyset \\
  \freenames{x?(y).P} & := & \{ x \} \cup (\freenames{P} \setminus \{ y \}) \\
  \freenames{x!\langle P \rangle} & := & \{ x \} \cup \{ P \} \\
  \freenames{P|Q} & := & \freenames{P} \cup \freenames{Q} \\
  \freenames{\dropn{x}} & := & \{ x \}
\end{eqnarray*}

The bound names of a process, $\boundnames{P}$, are those names occurring in $P$
that are not free. For example, in $x?(y).0$, the name $x$ is free, while $y$ is bound.

\begin{mathpar}
  \inferrule* [lab=monoidal-laws] {} { P|Q \equiv Q|P \and P|0 \equiv P \and P|(Q|R) \equiv (P|Q)|R }
\end{mathpar}

\begin{mathpar}
  \inferrule* [lab=alpha-equivalence] {} { (x)P \equiv (y)P\{y/x\} \and y \not\in \freenames{P} }
\end{mathpar}

\begin{definition}
Then two processes, $P,Q$, are alpha-equivalent if $P = Q\{\vec{y}/\vec{x}\}$ for
some $\vec{x} \in \boundnames{Q},\vec{y} \in \boundnames{P}$, where $Q\{\vec{y}/\vec{x}\}$
denotes the capture-avoiding substitution of $\vec{y}$ for $\vec{x}$ in $Q$.
\end{definition}

\begin{definition}
  The {\em structural congruence} \cite{SangiorgiWalker} , $\equiv$,
  between processes is the least congruence containing
  alpha-equivalence, satisfying the abelian monoid laws
  (associativity, commutativity and $\pzero$ as identity) for parallel
  composition $|$ and for summation $+$.
\end{definition}

\subsection{Name equivalence}

We take name equivalence, written $\nameeq$, to be the smallest
equivalence relation generated by the following rules.

\begin{mathpar}
\inferrule*[lab=Quote-drop]
{ }
{ \quotep{@{x}} \nameeq x }

\inferrule*[lab=Struct-equiv]
{ P \scong Q }
{ \quotep{P} \nameeq \quotep{Q} }
\end{mathpar}

The astute reader will have noticed that the mutual recursion of names
and processes imposes a mutual recursion on alpha-equivalence and
structural equivalence via name-equivalence. Fortunately, all of this
works out pleasantly and we may calculate in the natural way, free of
concern. The reader interested in the details is referred to the
appendix \ref{appendix:rho_details}.

\subsection{Substitution}

We use $\Proc$ for the set of processes, $\QProc$ for the set of
names, and $\id{\{}\vec{y} / \vec{x} \id{\}}$ to denote partial maps,
$s : \QProc \rightarrow \QProc$. A map, $s$ lifts, uniquely, to a map
on process terms, $\widehat{s} : \Proc \rightarrow \Proc$ by the
following equations.

\begin{mathpar}
  (0) \psubstp{Q}{P} := 0 \\
  (R \juxtap S) \psubstp{Q}{P}
  :=    
  (R)\psubstp{Q}{P} \juxtap (S) \psubstp{Q}{P} \\
  (x?(y).R) \psubstp{Q}{P}    
  :=    
  (x)\substp{Q}{P} (z)\concat( (R \psubstn{z}{y}) \psubstp{Q}{P} ) \\
  (\lift{x}{R}) \psubstp{Q}{P}  
  :=
  \lift{(x)\substp{Q}{P}}{ R \psubstp{Q}{P} } \\
%   (\dropn{x})  \psubstp{Q}{P}       
%   := 
%   \left\{ 
%     \begin{array}{ccc} 
%       \dropn{\quotep{Q}} & & x \nameeq \quotep{P} \\
%       \dropn{x} & & otherwise \\
%     \end{array}
%   \right. 
  (\dropn{x})  \psubstp{Q}{P}       
  := 
  \left\{ 
    \begin{array}{ccc} 
      Q & & x \nameeq \quotep{P} \\
      \dropn{x} & & otherwise \\
    \end{array}
  \right.
\end{mathpar}
 

where

\begin{eqnarray}
  (x)\id{\{} \lpquote Q \rpquote / \lpquote P \rpquote \id{\}}            = 
  \left\{ 
    \begin{array}{ccc}
      \lpquote Q \rpquote & & x \nameeq \lpquote P \rpquote \\
      x & & otherwise \\
    \end{array}
  \right. \nonumber
\end{eqnarray}

and $z$ is chosen distinct from $\quotep{P}$, $\quotep{Q}$, the free
names in $Q$, and all the names in $R$. Our $\alpha$-equivalence will
be built in the standard way from this substitution.

\begin{remark}\label{rem:no_self_referential_names}
  One consequence of these definitions is that $\forall P. \quotep{P}
  \not\in \freenames{P}$.
\end{remark}

\subsection{ Dynamic quote: an example }

Anticipating something of what's to come, consider applying the
substitution, $\widehat{\id{\{}u / z \id{\}}}$, to the following pair
of processes, $\lift{w}{y!(z)}$ and $w[ \lpquote y!(z) \rpquote ]$.

\begin{eqnarray}
	\lift{w}{y!(z)}\widehat{\id{\{}u / z \id{\}}}
		& = &
		\lift{w}{y!(u)} \nonumber\\
	w[ \lpquote y!(z) \rpquote ] \widehat{ \id{\{}u / z \id{\}} }
		& = &
		w[ \lpquote y!(z) \rpquote ] \nonumber
\end{eqnarray}

Because the body of the process between quotes is impervious to
substitution, we get radically different answers. In fact, by
examining the first process in an input context,
e.g. $x?(z).\lift{w}{y!(z)}$, we see that the process under the lift
operator may be shaped by prefixed inputs binding a name inside it. In
this sense, the lift operator will be seen as a way to dynamically
construct processes before reifying them as names.

Finally equipped with these standard features we can present the
dynamics of the calculus.

\subsubsection{Operational semantics} 

Finally, we introduce the computational dynamics. What marks these
algebras as distinct from other more traditionally studied algebraic
structures, e.g. vector spaces or polynomial rings, is the manner in
which dynamics is captured. In traditional structures, dynamics is typically
expressed through morphisms between such structures, as in linear maps
between vector spaces or morphisms between rings. In algebras
associated with the semantics of computation, the dynamics is
expressed as part of the algebraic structure itself, through a
reduction reduction relation typically denoted by $\red$. Below, we
give a recursive presentation of this relation for the calculus used
in the encoding.

$\red \subseteq \pi \times \pi$
$\red : \pi \to \mathcal{P}(\pi)$

\begin{mathpar}
  \inferrule* [lab=Comm] { \textsf{match}( x_{src}, x_{trgt} ) } { x_{trgt}?(y)P \; | \; x_{src}!\langle {Q} \rangle \red P\{\quotep{Q}/y}\} }
  \and \\
  \inferrule* [lab=Par] {{P} \red {P}'} {{{P} | {Q}} \red {{P}' | {Q}}}
  \and
  \inferrule* [lab=Equiv]{{{P} \scong {P}'} \andalso {{P}' \red {Q}'} \andalso {{Q}' \scong {Q}}}{{P} \red {Q}}
\end{mathpar}

\begin{eqnarray*}
  match_{\equiv} (\quotep{P},\quotep{Q}) & := & P \equiv Q \\
  match_{\dagger}(\quotep{P},\quotep{Q}) & := & \forall R. P|Q \red^{*} R => R \red^{*} 0 \\
  match_{K}(\quotep{P},\quotep{Q}) & := & K \mbox{ for some context } K
\end{eqnarray*}

$u?(x)P | u!\langle Q \rangle \red P\{\quotep{Q}/x\}$

%We write $\wred$ for $\red^*$, and $P\red$ if $\exists Q $ such that $ P \red Q$.
We write $P\red$ if $\exists Q $ such that $ P \red Q$ and $P\not\red$, otherwise.

\section{Replication}

As mentioned before, it is known that replication (and hence
recursion) can be implemented in a higher-order process algebra
\cite{SangiorgiWalker}. As our first example of calculation with the
machinery thus far presented we give the construction explicitly in
the {\rhoc}.

\begin{eqnarray}
	D_{x} & := & \prefix{x}{y}{(\binpar{\outputp{x}{y}}{@{y}})} \nonumber\\
	\bangp_{x}{P} & := & \binpar{{x}!\langle{\binpar{D_{x}}{P}}\rangle}{D_{x}} \nonumber
\end{eqnarray}

\begin{eqnarray}
	\bangp_{x}{P} & & \nonumber\\
	=
	& {x}!\langle{(\prefix{x}{y}{(\outputp{x}{y} | @{y})) | P}}\rangle 
	      | \prefix{x}{y}{(\outputp{x}{y} | @{y})} & \nonumber\\
	\red
	& (\outputp{x}{y} | @{y})\substn{\quotep{(\prefix{x}{y}{(@{y} | \outputp{x}{y})) | P}}}{y} & \nonumber\\
	=
	& \outputp{x}{\quotep{(\prefix{x}{y}{(\outputp{x}{y} | @{y})) | P}}}
	  | {(\prefix{x}{y}{(\outputp{x}{y} | @{y})) | P}} & \nonumber\\
	\red
	& \ldots & \nonumber\\
	\red^*
	& P | P | \ldots & \nonumber
\end{eqnarray}

Of course, this encoding, as an implementation, runs away, unfolding
$\bangp{P}$ eagerly. A lazier and more implementable replication
operator, restricted to input-guarded processes, may be obtained as follows.

\begin{eqnarray}
\bangp{\prefix{u}{v}{P}} 
	:= 
	\binpar{\lift{x}{\prefix{u}{v}{(\binpar{D(x)}{P})}}}{D(x)} \nonumber
\end{eqnarray}

\begin{remark}
  Note that the lazier definition still does not deal with summation
  or mixed summation (i.e. sums over input and output). The reader is
  invited to construct definitions of replication that deal with these
  features. 

  Further, the definitions are parameterized in a name, $x$. Can you,
  gentle reader, make a definition that eliminates this parameter and
  guarantees no accidental interaction between the replication
  machinery and the process being replicated -- i.e. no accidental
  sharing of names used by the process to get its work done and the
  name(s) used by the replication to effect copying. This latter
  revision of the definition of replication is crucial to obtaining
  the expected identity $!!P \sim !P$.
\end{remark}

\begin{remark}\label{rem:paradoxical_combinator}
  The reader familiar with the lambda calculus will have noticed the
  similarity between $D$ and the paradoxical combinator.

  [Ed. note: the existence of this seems to suggest we have to be more
  restrictive on the set of processes and names we admit if we are to
  support no-cloning.]
\end{remark}

\subsubsection{Bisimulation}

The computational dynamics gives rise to another kind of equivalence,
the equivalence of computational behavior. As previously mentioned
this is typically captured \emph{via} some form of bisimulation.

% The notion we use in this paper is weak barbed bisimulation
% \cite{milner91polyadicpi}.

The notion we use in this paper is derived from weak barbed
bisimulation \cite{milner91polyadicpi}. 

\begin{definition}
An \emph{observation relation}, $\downarrow_{\mathcal N}$, over a set
of names, $\mathcal N$, is the smallest relation satisfying the rules
below.

\infrule[Out-barb]{y \in {\mathcal N}, \; x \nameeq y}
		  {\outputp{x}{v} \downarrow_{\mathcal N} x}
\infrule[Par-barb]{\mbox{$P\downarrow_{\mathcal N} x$ or $Q\downarrow_{\mathcal N} x$}}
		  {\binpar{P}{Q} \downarrow_{\mathcal N} x}

We write $P \Downarrow_{\mathcal N} x$ if there is $Q$ such that 
$P \wred Q$ and $Q \downarrow_{\mathcal N} x$.
\end{definition}

\begin{definition}
%\label{def.bbisim}
An  ${\mathcal N}$-\emph{barbed bisimulation} over a set of names, ${\mathcal N}$, is a symmetric binary relation 
${\mathcal S}_{\mathcal N}$ between agents such that $P\rel{S}_{\mathcal N}Q$ implies:
\begin{enumerate}
\item If $P \red P'$ then $Q \wred Q'$ and $P'\rel{S}_{\mathcal N} Q'$.
\item If $P\downarrow_{\mathcal N} x$, then $Q\Downarrow_{\mathcal N} x$.
\end{enumerate}
$P$ is ${\mathcal N}$-barbed bisimilar to $Q$, written
$P \wbbisim_{\mathcal N} Q$, if $P \rel{S}_{\mathcal N} Q$ for some ${\mathcal N}$-barbed bisimulation ${\mathcal S}_{\mathcal N}$.
\end{definition}

$\mathcal{R} \subseteq \pi \times \pi$

$P \mathcal{R} Q => \forall P'. P \red P' \Rightarrow \exists Q'. Q \red Q', P' \mathcal{R} Q'$

$P \vdash x \Rightarrow Q \vdash x$

\begin{mathpar}
  \inferrule*[lab=Out-barb]{x \nameeq y}{{y}!\langle{Q}\rangle \vdash x}
  \and
  \inferrule*[lab=Par-barb]{\mbox{$P\vdash x$ or $Q\vdash x$}}{\binpar{P}{Q} \vdash x}
\end{mathpar}

\subsubsection{Contexts}

One of the principle advantages of computational calculi like the
$\pi$-calculus is a well-defined notion of context,
contextual-equivalence and a correlation between
contextual-equivalence and notions of bisimulation. The notion of
context allows the decomposition of a process into (sub-)process and
its syntactic environment, its context. Thus, a context may be
thought of as a process with a ``hole'' (written $\Box$) in it. The
application of a context $M$ to a process $P$, written $M[P]$, is
tantamount to filling the hole in $M$ with $P$. In this paper we do
not need the full weight of this theory, but do make use of the notion
of context in the proof the main theorem. 

\begin{mathpar}
  \inferrule* [lab=summation] {} {{M_{M},M_{N}} \bc \Box \;|\; x.M_{A} \;|\; M_{M}+M_{N}}
  \and
  \inferrule* [lab=agent] {} {{M_{A}} \bc (\vec{x})M_{P} \;| \; \clift{P_0,\ldots,M_{P},\ldots,P_N}}
  \and \\
  \inferrule* [lab=process] {} {{M_{P}} \bc M_{N} \;| \;P|M_{P} }
\end{mathpar} 

\begin{mathpar}
  \inferrule* [lab=sychronization] {} {M_{N} \bc \Box \;|\; x?M_{F} \;|\; x!M_{C}}
  \and
  \inferrule* [lab=abstraction] {} {{M_{F}} \bc (x)M_{P} }
  \and
  \inferrule* [lab=concretion] {} {{M_{C}} \bc \langle M_{P} \rangle }
  \and \\
  \inferrule* [lab=process] {} {{M_{P}} \bc M_{N} \;| \;P|M_{P} }
\end{mathpar}

\begin{definition}[contextual application] Given a context $M$, and
  process $P$, we define the \emph{contextual application}, $M[P] :=
  M\{P/\Box\}$. That is, the contextual application of M to P is the
  substitution of $P$ for $\Box$ in $M$.
\end{definition}

$\meaningof{-} : L \to \mathcal{P}(\pi)$

\begin{mathpar}
  \inferrule* [lab=collection] {} {\meaningof{true} = \pi, \and \meaningof{~E} = \pi \setminus \meaningof{E}, \and \meaningof{E_{1} \& E_{2}} = \meaningof{E_{1}} \cap \meaningof{E_{2}}}
\end{mathpar}

\begin{mathpar}
  \inferrule* [lab=structure] {} {\meaningof{0} = \{ P \in \pi | P \equiv 0 \}, \and \\ \meaningof{E_1 | E_2} = \{ P \in \pi | P \equiv P_{1} | P_{2}, P_{1} \in \meaningof{E_{1}}, P_{2} \in \meaningof{E_2}\} }
\end{mathpar}

\begin{mathpar}
 \inferrule* [lab=behavior] {} {\meaningof{\langle a?b \rangle E} = \{ P \in \pi | P \equiv Q | u?(y)P', \\ \and \\\\ \and \\ \;\;\; u \in \meaningof{a}, \forall z.P'\{z/y\} \in \meaningof{E\{z/b\}}\}, \and \\ \meaningof{a!E} = \{ P \in \pi | P \equiv Q | x!\langle P' \rangle, x \in \meaningof{a} P' \in \meaningof{E}\} }
\end{mathpar}

\begin{mathpar}
 \inferrule* [lab=nominal] {} {\meaningof{\quotep{E}} = \{ \quotep{P} \in \quotep{\pi} | P \in \meaningof{E} \}, \and \meaningof{\quotep{P}} = \{ \quotep{Q} \in \quotep{\pi} | P \equiv Q \} \and \\ \meaningof{@\quotep{E}} = \{ P \in \pi | P \equiv @x, x \in \meaningof{E} \}}
\end{mathpar}

\begin{eqnarray*}
  \\
  \meaningof{-} : TS \to ST
\end{eqnarray*}

\begin{eqnarray*}
  \\
  L : TS \to ST
\end{eqnarray*}

\begin{eqnarray*}
  \\
  P \models E \iff P \in \meaningof{E}
\end{eqnarray*}

\begin{eqnarray*}
  P \approx_{L} Q \iff \forall E \in L. P \models E \iff Q \models E
\end{eqnarray*}

\begin{eqnarray*}
  P \approx_{K} Q
\end{eqnarray*}

\begin{eqnarray*}
  P \approx Q
\end{eqnarray*}

$\approx_{K} = \approx = \approx_{L}$

\subsubsection{Contextual duality}

Note that contexts extend the quotation operation to a family of
operations from processes to names. Given a context, $M$, we can
define a \emph{nominal context}, $\quotep{M}$ by $\quotep{M}[P] :=
\quotep{M[P]}$. To foreshadow what is to come we observe that these
operations enjoy a duality with processes very much like the duality
between vectors and maps from vectors to scalars.

Further, because the calculus is essentially higher-order, we have a
correspondence between contexts and processes. More specifically,
given a name $x$ and a context $M$ we can construct $M^{*}_{x}$ such
that 

\begin{mathpar}
  M^{*}_{x} | \lift{x}{P} \red M[P]
\end{mathpar}

namely,

\begin{mathpar}
  M^{*}_{x} := x?(u).M[\dropn{u}]
\end{mathpar}

The dependence of $M^{*}_{x}$ on a name makes it an abstraction, 

\begin{mathpar}
  M^{*} := (x)x?(u).M[\dropn{u}]
\end{mathpar}

\subsection{Additional notation}

It will sometimes be convenient to denote the process a name
quotes. We already have the notation $x = \quotep{P}$, but it will be
convenient to introduce an alternate notation, $\procn{x}$, when we
want to emphasize the connection to the use of the name. Note that, by
virtue of name equivalence, $\quotep{\procn{x}} \nameeq x$; so, the
notation is consistent with previous definitions.

Further, because names have structure it is possible to effect
substitutions on the basis of that structure. This means we need to
upgrade our notation for substitutions, which we accomplish by
adapting comprehension notation. Thus,

\begin{mathpar}
  P\{ y / x : x \in S \}
\end{mathpar}

is interpreted to mean the process derived from P by replacing (in a
capture-avoiding manner) each occurrence of $x$ in $S$ by $y$. For example,

\begin{mathpar}
  P\{ \quotep{\procn{x}|\procn{x}} / x : x \in \freenames{P} \}
\end{mathpar}

will replace each (occurrence) of a free name $x$ in $P$ by
$\quotep{\procn{x}|\procn{x}}$.

Also, we will avail ourselves of the notation $x^{L}$ and $x^{R}$ to
denote injections of a name into disjoint copies of the name
space. There are numerous ways to accomplish this. One example can be
found in \cite{MeredithR05}. This notation overloads to vectors of
names: $\vec{x}^{\pi} := (x_{i}^{\pi} \; : \; 0 \leq i < |\vec{x}| )$ where $\pi \in \{L,R\}$.

We also use $P^{\Box} := P|\Box$.

In \cite{MeredithR05} an interpretation of the new operator is
given. It turns out that there are several possible interpretations
all enjoying the requisite algebraic properties of the operator (see
\cite{milner91polyadicpi}). We will therefore make liberal use of
$(\nu\; \vec{x})P$.

% subsection the_syntax_and_semantics_of_the_notation_system (end)   

\input{qm2pi.qmops} 

\input{qm2pi.sterngerlach} 

\input{qm2pi.metric} 

% section concurrent_process_calculi (end)

%\input{qm2pi.proofsketch}

% section proof sketch (end)

%\input{qm2pi.slviaknots} 

% section spatial logic via knots (end)

\input{qm2pi.conclusion}

% section conclusion (end)

%\input{qm2pi.dtcodes} 

% section wiring algorithm (end)

\input{qm2pi.ack} 

% section acknowledgments (end)

\newpage


\bibliographystyle{plain}   
\bibliography{../../biblios/main.bib}

\input{qm2pi.rhodetails}

\end{document}

 

% section acknowledgments (end)

\newpage


\bibliographystyle{plain}   
\bibliography{../../biblios/main.bib}

\documentclass[12pt]{llncs}
%\documentclass{jktr}

\usepackage[pdftex]{hyperref}                   
\usepackage {listings}
\usepackage {mathpartir}
\usepackage{bcprules}
%\usepackage{listings}
                       
\usepackage{graphicx} 
%\usepackage[margins=2.5cm,nohead,nofoot]{geometry}
%\usepackage{geometry}
\usepackage{amsfonts}
\usepackage{amstext}
\usepackage{latexsym}
\usepackage{amssymb}
\usepackage{color}


%\include{myPreamble}
\include{qm2pi.local} 

%\ifpdf
%\usepackage[pdftex]{graphicx}
%\else
%\usepackage{graphicx}
%\fi

 % \ifpdf
%  \usepackage{pdfsync}
%  \if


%\title{Brief Article}
%\author{David F. Snyder}
%\author{L.G. Meredith}

%\address{Dept. of Math., Texas State University--San Marcos, San Marcos, TX 78666}
       
\pagestyle{empty}


\begin{document}

\lstset{language=[Objective]Caml,frame=shadowbox}

\input{qm2pi.front}

% section front matter (end)

\input{qm2pi.intro} 
 
% section introduction (end)

% \input{qm2pi.knotations} 

% section notation (end)

\input{qm2pi.process.calculi} 

% section concurrent_process_calculi_and_spatial_logics_ (end)
    
%\input{qm2pi.knots2pi} 

%\input{qm2pi.trefoil} 

%\input{qm2pi.mainthm} 

% subsection basic_interpretation (end)

%\input{qm2pi.rho.presentation} 
\subsection{The syntax and semantics of the notation system}\label{sub:the_syntax_and_semantics_of_the_notation_system} % (fold)

We now summarize a technical presentation of the calculus that
embodies our theory of dynamics. The typical presentation of such a
calculus follows the style of giving generators and relations on
them. The grammar, below, describing term constructors, freely
generates the set of processes, $\Proc$. This set is then quotiented
by a relation known as structural congruence and it is over this set
that the notion of dynamics is expressed. This presentation is
essentially that of \cite{MeredithR05} with the addition of
polyadicity and summation. For readability we have relegated some of
the technical subtleties to an appendix.

\subsubsection{Process grammar}\label{subsub:process_grammar}

\begin{mathpar}
  \inferrule* [lab=synchronization] {} {{M} \bc \pzero \;|\; x?F \;|\; x!C }
  \and
  \inferrule* [lab=abstraction] {} {{F} \bc (x)P}
  \and
  \inferrule* [lab=concretion] {} {{C} \bc \langle Q \rangle}
  \and
  \inferrule* [lab=process] {} {{P,Q} \bc M \;| \;P|Q \;|\; @{x}}
  \and
  \inferrule* [lab=name] {} {{x} \bc \quotep{P}}
\end{mathpar} 

Note that $\vec{x}$ (resp. $\vec{P}$) denotes a vector of names
(resp. processes) of length $|\vec{x}|$ (resp. $|\vec{P}|$). We adopt
the following useful abbreviations.

\begin{mathpar}
   x?(\vec{y}).P := x.(\vec{y})P \and  x\clift{\vec{P}} := x.\clift{\vec{P}}
   \and x!(y) := \lift{x}{\dropn{y}}
   \and \Pi_{i=0}^{n-1}P_i := P_0 | \ldots | P_{n-1}
\end{mathpar}

\subsubsection{Structural congruence}

\paragraph{Free and bound names and alpha-equivalence.} At the
core of structural equivalence is alpha-equivalence which identifies
process that are the same up to a change of variable. Formally, we
recognize the distinction between free and bound names. The free names
of a process, $\freenames{P}$, may be calculated recursively as
follows:

\begin{mathpar}
\freenames{\pzero} := \emptyset
  \and \\
  \freenames{x?(y).P} := \{ x \} \cup (\freenames{P} \setminus \{ y \})
  \and 
  \freenames{x!\langle P \rangle} := \{ x \} \cup \{ P \} 
  \and \\
  \freenames{P|Q} := \freenames{P} \cup \freenames{Q}
  \and \\
  \freenames{@{x}} := \{ x \}
\end{mathpar}

$\pi$
$\quotep{\pi}$

$\freenames{-} : \pi \to \mathcal{P}(\quotep{\pi})$

\begin{eqnarray*}
  \freenames{\pzero} & := & \emptyset \\
  \freenames{x?(y).P} & := & \{ x \} \cup (\freenames{P} \setminus \{ y \}) \\
  \freenames{x!\langle P \rangle} & := & \{ x \} \cup \{ P \} \\
  \freenames{P|Q} & := & \freenames{P} \cup \freenames{Q} \\
  \freenames{\dropn{x}} & := & \{ x \}
\end{eqnarray*}

The bound names of a process, $\boundnames{P}$, are those names occurring in $P$
that are not free. For example, in $x?(y).0$, the name $x$ is free, while $y$ is bound.

\begin{mathpar}
  \inferrule* [lab=monoidal-laws] {} { P|Q \equiv Q|P \and P|0 \equiv P \and P|(Q|R) \equiv (P|Q)|R }
\end{mathpar}

\begin{mathpar}
  \inferrule* [lab=alpha-equivalence] {} { (x)P \equiv (y)P\{y/x\} \and y \not\in \freenames{P} }
\end{mathpar}

\begin{definition}
Then two processes, $P,Q$, are alpha-equivalent if $P = Q\{\vec{y}/\vec{x}\}$ for
some $\vec{x} \in \boundnames{Q},\vec{y} \in \boundnames{P}$, where $Q\{\vec{y}/\vec{x}\}$
denotes the capture-avoiding substitution of $\vec{y}$ for $\vec{x}$ in $Q$.
\end{definition}

\begin{definition}
  The {\em structural congruence} \cite{SangiorgiWalker} , $\equiv$,
  between processes is the least congruence containing
  alpha-equivalence, satisfying the abelian monoid laws
  (associativity, commutativity and $\pzero$ as identity) for parallel
  composition $|$ and for summation $+$.
\end{definition}

\subsection{Name equivalence}

We take name equivalence, written $\nameeq$, to be the smallest
equivalence relation generated by the following rules.

\begin{mathpar}
\inferrule*[lab=Quote-drop]
{ }
{ \quotep{@{x}} \nameeq x }

\inferrule*[lab=Struct-equiv]
{ P \scong Q }
{ \quotep{P} \nameeq \quotep{Q} }
\end{mathpar}

The astute reader will have noticed that the mutual recursion of names
and processes imposes a mutual recursion on alpha-equivalence and
structural equivalence via name-equivalence. Fortunately, all of this
works out pleasantly and we may calculate in the natural way, free of
concern. The reader interested in the details is referred to the
appendix \ref{appendix:rho_details}.

\subsection{Substitution}

We use $\Proc$ for the set of processes, $\QProc$ for the set of
names, and $\id{\{}\vec{y} / \vec{x} \id{\}}$ to denote partial maps,
$s : \QProc \rightarrow \QProc$. A map, $s$ lifts, uniquely, to a map
on process terms, $\widehat{s} : \Proc \rightarrow \Proc$ by the
following equations.

\begin{mathpar}
  (0) \psubstp{Q}{P} := 0 \\
  (R \juxtap S) \psubstp{Q}{P}
  :=    
  (R)\psubstp{Q}{P} \juxtap (S) \psubstp{Q}{P} \\
  (x?(y).R) \psubstp{Q}{P}    
  :=    
  (x)\substp{Q}{P} (z)\concat( (R \psubstn{z}{y}) \psubstp{Q}{P} ) \\
  (\lift{x}{R}) \psubstp{Q}{P}  
  :=
  \lift{(x)\substp{Q}{P}}{ R \psubstp{Q}{P} } \\
%   (\dropn{x})  \psubstp{Q}{P}       
%   := 
%   \left\{ 
%     \begin{array}{ccc} 
%       \dropn{\quotep{Q}} & & x \nameeq \quotep{P} \\
%       \dropn{x} & & otherwise \\
%     \end{array}
%   \right. 
  (\dropn{x})  \psubstp{Q}{P}       
  := 
  \left\{ 
    \begin{array}{ccc} 
      Q & & x \nameeq \quotep{P} \\
      \dropn{x} & & otherwise \\
    \end{array}
  \right.
\end{mathpar}
 

where

\begin{eqnarray}
  (x)\id{\{} \lpquote Q \rpquote / \lpquote P \rpquote \id{\}}            = 
  \left\{ 
    \begin{array}{ccc}
      \lpquote Q \rpquote & & x \nameeq \lpquote P \rpquote \\
      x & & otherwise \\
    \end{array}
  \right. \nonumber
\end{eqnarray}

and $z$ is chosen distinct from $\quotep{P}$, $\quotep{Q}$, the free
names in $Q$, and all the names in $R$. Our $\alpha$-equivalence will
be built in the standard way from this substitution.

\begin{remark}\label{rem:no_self_referential_names}
  One consequence of these definitions is that $\forall P. \quotep{P}
  \not\in \freenames{P}$.
\end{remark}

\subsection{ Dynamic quote: an example }

Anticipating something of what's to come, consider applying the
substitution, $\widehat{\id{\{}u / z \id{\}}}$, to the following pair
of processes, $\lift{w}{y!(z)}$ and $w[ \lpquote y!(z) \rpquote ]$.

\begin{eqnarray}
	\lift{w}{y!(z)}\widehat{\id{\{}u / z \id{\}}}
		& = &
		\lift{w}{y!(u)} \nonumber\\
	w[ \lpquote y!(z) \rpquote ] \widehat{ \id{\{}u / z \id{\}} }
		& = &
		w[ \lpquote y!(z) \rpquote ] \nonumber
\end{eqnarray}

Because the body of the process between quotes is impervious to
substitution, we get radically different answers. In fact, by
examining the first process in an input context,
e.g. $x?(z).\lift{w}{y!(z)}$, we see that the process under the lift
operator may be shaped by prefixed inputs binding a name inside it. In
this sense, the lift operator will be seen as a way to dynamically
construct processes before reifying them as names.

Finally equipped with these standard features we can present the
dynamics of the calculus.

\subsubsection{Operational semantics} 

Finally, we introduce the computational dynamics. What marks these
algebras as distinct from other more traditionally studied algebraic
structures, e.g. vector spaces or polynomial rings, is the manner in
which dynamics is captured. In traditional structures, dynamics is typically
expressed through morphisms between such structures, as in linear maps
between vector spaces or morphisms between rings. In algebras
associated with the semantics of computation, the dynamics is
expressed as part of the algebraic structure itself, through a
reduction reduction relation typically denoted by $\red$. Below, we
give a recursive presentation of this relation for the calculus used
in the encoding.

$\red \subseteq \pi \times \pi$
$\red : \pi \to \mathcal{P}(\pi)$

\begin{mathpar}
  \inferrule* [lab=Comm] { \textsf{match}( x_{src}, x_{trgt} ) } { x_{trgt}?(y)P \; | \; x_{src}!\langle {Q} \rangle \red P\{\quotep{Q}/y}\} }
  \and \\
  \inferrule* [lab=Par] {{P} \red {P}'} {{{P} | {Q}} \red {{P}' | {Q}}}
  \and
  \inferrule* [lab=Equiv]{{{P} \scong {P}'} \andalso {{P}' \red {Q}'} \andalso {{Q}' \scong {Q}}}{{P} \red {Q}}
\end{mathpar}

\begin{eqnarray*}
  match_{\equiv} (\quotep{P},\quotep{Q}) & := & P \equiv Q \\
  match_{\dagger}(\quotep{P},\quotep{Q}) & := & \forall R. P|Q \red^{*} R => R \red^{*} 0 \\
  match_{K}(\quotep{P},\quotep{Q}) & := & K \mbox{ for some context } K
\end{eqnarray*}

$u?(x)P | u!\langle Q \rangle \red P\{\quotep{Q}/x\}$

%We write $\wred$ for $\red^*$, and $P\red$ if $\exists Q $ such that $ P \red Q$.
We write $P\red$ if $\exists Q $ such that $ P \red Q$ and $P\not\red$, otherwise.

\section{Replication}

As mentioned before, it is known that replication (and hence
recursion) can be implemented in a higher-order process algebra
\cite{SangiorgiWalker}. As our first example of calculation with the
machinery thus far presented we give the construction explicitly in
the {\rhoc}.

\begin{eqnarray}
	D_{x} & := & \prefix{x}{y}{(\binpar{\outputp{x}{y}}{@{y}})} \nonumber\\
	\bangp_{x}{P} & := & \binpar{{x}!\langle{\binpar{D_{x}}{P}}\rangle}{D_{x}} \nonumber
\end{eqnarray}

\begin{eqnarray}
	\bangp_{x}{P} & & \nonumber\\
	=
	& {x}!\langle{(\prefix{x}{y}{(\outputp{x}{y} | @{y})) | P}}\rangle 
	      | \prefix{x}{y}{(\outputp{x}{y} | @{y})} & \nonumber\\
	\red
	& (\outputp{x}{y} | @{y})\substn{\quotep{(\prefix{x}{y}{(@{y} | \outputp{x}{y})) | P}}}{y} & \nonumber\\
	=
	& \outputp{x}{\quotep{(\prefix{x}{y}{(\outputp{x}{y} | @{y})) | P}}}
	  | {(\prefix{x}{y}{(\outputp{x}{y} | @{y})) | P}} & \nonumber\\
	\red
	& \ldots & \nonumber\\
	\red^*
	& P | P | \ldots & \nonumber
\end{eqnarray}

Of course, this encoding, as an implementation, runs away, unfolding
$\bangp{P}$ eagerly. A lazier and more implementable replication
operator, restricted to input-guarded processes, may be obtained as follows.

\begin{eqnarray}
\bangp{\prefix{u}{v}{P}} 
	:= 
	\binpar{\lift{x}{\prefix{u}{v}{(\binpar{D(x)}{P})}}}{D(x)} \nonumber
\end{eqnarray}

\begin{remark}
  Note that the lazier definition still does not deal with summation
  or mixed summation (i.e. sums over input and output). The reader is
  invited to construct definitions of replication that deal with these
  features. 

  Further, the definitions are parameterized in a name, $x$. Can you,
  gentle reader, make a definition that eliminates this parameter and
  guarantees no accidental interaction between the replication
  machinery and the process being replicated -- i.e. no accidental
  sharing of names used by the process to get its work done and the
  name(s) used by the replication to effect copying. This latter
  revision of the definition of replication is crucial to obtaining
  the expected identity $!!P \sim !P$.
\end{remark}

\begin{remark}\label{rem:paradoxical_combinator}
  The reader familiar with the lambda calculus will have noticed the
  similarity between $D$ and the paradoxical combinator.

  [Ed. note: the existence of this seems to suggest we have to be more
  restrictive on the set of processes and names we admit if we are to
  support no-cloning.]
\end{remark}

\subsubsection{Bisimulation}

The computational dynamics gives rise to another kind of equivalence,
the equivalence of computational behavior. As previously mentioned
this is typically captured \emph{via} some form of bisimulation.

% The notion we use in this paper is weak barbed bisimulation
% \cite{milner91polyadicpi}.

The notion we use in this paper is derived from weak barbed
bisimulation \cite{milner91polyadicpi}. 

\begin{definition}
An \emph{observation relation}, $\downarrow_{\mathcal N}$, over a set
of names, $\mathcal N$, is the smallest relation satisfying the rules
below.

\infrule[Out-barb]{y \in {\mathcal N}, \; x \nameeq y}
		  {\outputp{x}{v} \downarrow_{\mathcal N} x}
\infrule[Par-barb]{\mbox{$P\downarrow_{\mathcal N} x$ or $Q\downarrow_{\mathcal N} x$}}
		  {\binpar{P}{Q} \downarrow_{\mathcal N} x}

We write $P \Downarrow_{\mathcal N} x$ if there is $Q$ such that 
$P \wred Q$ and $Q \downarrow_{\mathcal N} x$.
\end{definition}

\begin{definition}
%\label{def.bbisim}
An  ${\mathcal N}$-\emph{barbed bisimulation} over a set of names, ${\mathcal N}$, is a symmetric binary relation 
${\mathcal S}_{\mathcal N}$ between agents such that $P\rel{S}_{\mathcal N}Q$ implies:
\begin{enumerate}
\item If $P \red P'$ then $Q \wred Q'$ and $P'\rel{S}_{\mathcal N} Q'$.
\item If $P\downarrow_{\mathcal N} x$, then $Q\Downarrow_{\mathcal N} x$.
\end{enumerate}
$P$ is ${\mathcal N}$-barbed bisimilar to $Q$, written
$P \wbbisim_{\mathcal N} Q$, if $P \rel{S}_{\mathcal N} Q$ for some ${\mathcal N}$-barbed bisimulation ${\mathcal S}_{\mathcal N}$.
\end{definition}

$\mathcal{R} \subseteq \pi \times \pi$

$P \mathcal{R} Q => \forall P'. P \red P' \Rightarrow \exists Q'. Q \red Q', P' \mathcal{R} Q'$

$P \vdash x \Rightarrow Q \vdash x$

\begin{mathpar}
  \inferrule*[lab=Out-barb]{x \nameeq y}{{y}!\langle{Q}\rangle \vdash x}
  \and
  \inferrule*[lab=Par-barb]{\mbox{$P\vdash x$ or $Q\vdash x$}}{\binpar{P}{Q} \vdash x}
\end{mathpar}

\subsubsection{Contexts}

One of the principle advantages of computational calculi like the
$\pi$-calculus is a well-defined notion of context,
contextual-equivalence and a correlation between
contextual-equivalence and notions of bisimulation. The notion of
context allows the decomposition of a process into (sub-)process and
its syntactic environment, its context. Thus, a context may be
thought of as a process with a ``hole'' (written $\Box$) in it. The
application of a context $M$ to a process $P$, written $M[P]$, is
tantamount to filling the hole in $M$ with $P$. In this paper we do
not need the full weight of this theory, but do make use of the notion
of context in the proof the main theorem. 

\begin{mathpar}
  \inferrule* [lab=summation] {} {{M_{M},M_{N}} \bc \Box \;|\; x.M_{A} \;|\; M_{M}+M_{N}}
  \and
  \inferrule* [lab=agent] {} {{M_{A}} \bc (\vec{x})M_{P} \;| \; \clift{P_0,\ldots,M_{P},\ldots,P_N}}
  \and \\
  \inferrule* [lab=process] {} {{M_{P}} \bc M_{N} \;| \;P|M_{P} }
\end{mathpar} 

\begin{mathpar}
  \inferrule* [lab=sychronization] {} {M_{N} \bc \Box \;|\; x?M_{F} \;|\; x!M_{C}}
  \and
  \inferrule* [lab=abstraction] {} {{M_{F}} \bc (x)M_{P} }
  \and
  \inferrule* [lab=concretion] {} {{M_{C}} \bc \langle M_{P} \rangle }
  \and \\
  \inferrule* [lab=process] {} {{M_{P}} \bc M_{N} \;| \;P|M_{P} }
\end{mathpar}

\begin{definition}[contextual application] Given a context $M$, and
  process $P$, we define the \emph{contextual application}, $M[P] :=
  M\{P/\Box\}$. That is, the contextual application of M to P is the
  substitution of $P$ for $\Box$ in $M$.
\end{definition}

$\meaningof{-} : L \to \mathcal{P}(\pi)$

\begin{mathpar}
  \inferrule* [lab=collection] {} {\meaningof{true} = \pi, \and \meaningof{~E} = \pi \setminus \meaningof{E}, \and \meaningof{E_{1} \& E_{2}} = \meaningof{E_{1}} \cap \meaningof{E_{2}}}
\end{mathpar}

\begin{mathpar}
  \inferrule* [lab=structure] {} {\meaningof{0} = \{ P \in \pi | P \equiv 0 \}, \and \\ \meaningof{E_1 | E_2} = \{ P \in \pi | P \equiv P_{1} | P_{2}, P_{1} \in \meaningof{E_{1}}, P_{2} \in \meaningof{E_2}\} }
\end{mathpar}

\begin{mathpar}
 \inferrule* [lab=behavior] {} {\meaningof{\langle a?b \rangle E} = \{ P \in \pi | P \equiv Q | u?(y)P', \\ \and \\\\ \and \\ \;\;\; u \in \meaningof{a}, \forall z.P'\{z/y\} \in \meaningof{E\{z/b\}}\}, \and \\ \meaningof{a!E} = \{ P \in \pi | P \equiv Q | x!\langle P' \rangle, x \in \meaningof{a} P' \in \meaningof{E}\} }
\end{mathpar}

\begin{mathpar}
 \inferrule* [lab=nominal] {} {\meaningof{\quotep{E}} = \{ \quotep{P} \in \quotep{\pi} | P \in \meaningof{E} \}, \and \meaningof{\quotep{P}} = \{ \quotep{Q} \in \quotep{\pi} | P \equiv Q \} \and \\ \meaningof{@\quotep{E}} = \{ P \in \pi | P \equiv @x, x \in \meaningof{E} \}}
\end{mathpar}

\begin{eqnarray*}
  \\
  \meaningof{-} : TS \to ST
\end{eqnarray*}

\begin{eqnarray*}
  \\
  L : TS \to ST
\end{eqnarray*}

\begin{eqnarray*}
  \\
  P \models E \iff P \in \meaningof{E}
\end{eqnarray*}

\begin{eqnarray*}
  P \approx_{L} Q \iff \forall E \in L. P \models E \iff Q \models E
\end{eqnarray*}

\begin{eqnarray*}
  P \approx_{K} Q
\end{eqnarray*}

\begin{eqnarray*}
  P \approx Q
\end{eqnarray*}

$\approx_{K} = \approx = \approx_{L}$

\subsubsection{Contextual duality}

Note that contexts extend the quotation operation to a family of
operations from processes to names. Given a context, $M$, we can
define a \emph{nominal context}, $\quotep{M}$ by $\quotep{M}[P] :=
\quotep{M[P]}$. To foreshadow what is to come we observe that these
operations enjoy a duality with processes very much like the duality
between vectors and maps from vectors to scalars.

Further, because the calculus is essentially higher-order, we have a
correspondence between contexts and processes. More specifically,
given a name $x$ and a context $M$ we can construct $M^{*}_{x}$ such
that 

\begin{mathpar}
  M^{*}_{x} | \lift{x}{P} \red M[P]
\end{mathpar}

namely,

\begin{mathpar}
  M^{*}_{x} := x?(u).M[\dropn{u}]
\end{mathpar}

The dependence of $M^{*}_{x}$ on a name makes it an abstraction, 

\begin{mathpar}
  M^{*} := (x)x?(u).M[\dropn{u}]
\end{mathpar}

\subsection{Additional notation}

It will sometimes be convenient to denote the process a name
quotes. We already have the notation $x = \quotep{P}$, but it will be
convenient to introduce an alternate notation, $\procn{x}$, when we
want to emphasize the connection to the use of the name. Note that, by
virtue of name equivalence, $\quotep{\procn{x}} \nameeq x$; so, the
notation is consistent with previous definitions.

Further, because names have structure it is possible to effect
substitutions on the basis of that structure. This means we need to
upgrade our notation for substitutions, which we accomplish by
adapting comprehension notation. Thus,

\begin{mathpar}
  P\{ y / x : x \in S \}
\end{mathpar}

is interpreted to mean the process derived from P by replacing (in a
capture-avoiding manner) each occurrence of $x$ in $S$ by $y$. For example,

\begin{mathpar}
  P\{ \quotep{\procn{x}|\procn{x}} / x : x \in \freenames{P} \}
\end{mathpar}

will replace each (occurrence) of a free name $x$ in $P$ by
$\quotep{\procn{x}|\procn{x}}$.

Also, we will avail ourselves of the notation $x^{L}$ and $x^{R}$ to
denote injections of a name into disjoint copies of the name
space. There are numerous ways to accomplish this. One example can be
found in \cite{MeredithR05}. This notation overloads to vectors of
names: $\vec{x}^{\pi} := (x_{i}^{\pi} \; : \; 0 \leq i < |\vec{x}| )$ where $\pi \in \{L,R\}$.

We also use $P^{\Box} := P|\Box$.

In \cite{MeredithR05} an interpretation of the new operator is
given. It turns out that there are several possible interpretations
all enjoying the requisite algebraic properties of the operator (see
\cite{milner91polyadicpi}). We will therefore make liberal use of
$(\nu\; \vec{x})P$.

% subsection the_syntax_and_semantics_of_the_notation_system (end)   

\input{qm2pi.qmops} 

\input{qm2pi.sterngerlach} 

\input{qm2pi.metric} 

% section concurrent_process_calculi (end)

%\input{qm2pi.proofsketch}

% section proof sketch (end)

%\input{qm2pi.slviaknots} 

% section spatial logic via knots (end)

\input{qm2pi.conclusion}

% section conclusion (end)

%\input{qm2pi.dtcodes} 

% section wiring algorithm (end)

\input{qm2pi.ack} 

% section acknowledgments (end)

\newpage


\bibliographystyle{plain}   
\bibliography{../../biblios/main.bib}

\input{qm2pi.rhodetails}

\end{document}



\end{document}

 

% section concurrent_process_calculi (end)

%\documentclass[12pt]{llncs}
%\documentclass{jktr}

\usepackage[pdftex]{hyperref}                   
\usepackage {listings}
\usepackage {mathpartir}
\usepackage{bcprules}
%\usepackage{listings}
                       
\usepackage{graphicx} 
%\usepackage[margins=2.5cm,nohead,nofoot]{geometry}
%\usepackage{geometry}
\usepackage{amsfonts}
\usepackage{amstext}
\usepackage{latexsym}
\usepackage{amssymb}
\usepackage{color}


%\include{myPreamble}
\documentclass[12pt]{llncs}
%\documentclass{jktr}

\usepackage[pdftex]{hyperref}                   
\usepackage {listings}
\usepackage {mathpartir}
\usepackage{bcprules}
%\usepackage{listings}
                       
\usepackage{graphicx} 
%\usepackage[margins=2.5cm,nohead,nofoot]{geometry}
%\usepackage{geometry}
\usepackage{amsfonts}
\usepackage{amstext}
\usepackage{latexsym}
\usepackage{amssymb}
\usepackage{color}


%\include{myPreamble}
\include{qm2pi.local} 

%\ifpdf
%\usepackage[pdftex]{graphicx}
%\else
%\usepackage{graphicx}
%\fi

 % \ifpdf
%  \usepackage{pdfsync}
%  \if


%\title{Brief Article}
%\author{David F. Snyder}
%\author{L.G. Meredith}

%\address{Dept. of Math., Texas State University--San Marcos, San Marcos, TX 78666}
       
\pagestyle{empty}


\begin{document}

\lstset{language=[Objective]Caml,frame=shadowbox}

\input{qm2pi.front}

% section front matter (end)

\input{qm2pi.intro} 
 
% section introduction (end)

% \input{qm2pi.knotations} 

% section notation (end)

\input{qm2pi.process.calculi} 

% section concurrent_process_calculi_and_spatial_logics_ (end)
    
%\input{qm2pi.knots2pi} 

%\input{qm2pi.trefoil} 

%\input{qm2pi.mainthm} 

% subsection basic_interpretation (end)

%\input{qm2pi.rho.presentation} 
\subsection{The syntax and semantics of the notation system}\label{sub:the_syntax_and_semantics_of_the_notation_system} % (fold)

We now summarize a technical presentation of the calculus that
embodies our theory of dynamics. The typical presentation of such a
calculus follows the style of giving generators and relations on
them. The grammar, below, describing term constructors, freely
generates the set of processes, $\Proc$. This set is then quotiented
by a relation known as structural congruence and it is over this set
that the notion of dynamics is expressed. This presentation is
essentially that of \cite{MeredithR05} with the addition of
polyadicity and summation. For readability we have relegated some of
the technical subtleties to an appendix.

\subsubsection{Process grammar}\label{subsub:process_grammar}

\begin{mathpar}
  \inferrule* [lab=synchronization] {} {{M} \bc \pzero \;|\; x?F \;|\; x!C }
  \and
  \inferrule* [lab=abstraction] {} {{F} \bc (x)P}
  \and
  \inferrule* [lab=concretion] {} {{C} \bc \langle Q \rangle}
  \and
  \inferrule* [lab=process] {} {{P,Q} \bc M \;| \;P|Q \;|\; @{x}}
  \and
  \inferrule* [lab=name] {} {{x} \bc \quotep{P}}
\end{mathpar} 

Note that $\vec{x}$ (resp. $\vec{P}$) denotes a vector of names
(resp. processes) of length $|\vec{x}|$ (resp. $|\vec{P}|$). We adopt
the following useful abbreviations.

\begin{mathpar}
   x?(\vec{y}).P := x.(\vec{y})P \and  x\clift{\vec{P}} := x.\clift{\vec{P}}
   \and x!(y) := \lift{x}{\dropn{y}}
   \and \Pi_{i=0}^{n-1}P_i := P_0 | \ldots | P_{n-1}
\end{mathpar}

\subsubsection{Structural congruence}

\paragraph{Free and bound names and alpha-equivalence.} At the
core of structural equivalence is alpha-equivalence which identifies
process that are the same up to a change of variable. Formally, we
recognize the distinction between free and bound names. The free names
of a process, $\freenames{P}$, may be calculated recursively as
follows:

\begin{mathpar}
\freenames{\pzero} := \emptyset
  \and \\
  \freenames{x?(y).P} := \{ x \} \cup (\freenames{P} \setminus \{ y \})
  \and 
  \freenames{x!\langle P \rangle} := \{ x \} \cup \{ P \} 
  \and \\
  \freenames{P|Q} := \freenames{P} \cup \freenames{Q}
  \and \\
  \freenames{@{x}} := \{ x \}
\end{mathpar}

$\pi$
$\quotep{\pi}$

$\freenames{-} : \pi \to \mathcal{P}(\quotep{\pi})$

\begin{eqnarray*}
  \freenames{\pzero} & := & \emptyset \\
  \freenames{x?(y).P} & := & \{ x \} \cup (\freenames{P} \setminus \{ y \}) \\
  \freenames{x!\langle P \rangle} & := & \{ x \} \cup \{ P \} \\
  \freenames{P|Q} & := & \freenames{P} \cup \freenames{Q} \\
  \freenames{\dropn{x}} & := & \{ x \}
\end{eqnarray*}

The bound names of a process, $\boundnames{P}$, are those names occurring in $P$
that are not free. For example, in $x?(y).0$, the name $x$ is free, while $y$ is bound.

\begin{mathpar}
  \inferrule* [lab=monoidal-laws] {} { P|Q \equiv Q|P \and P|0 \equiv P \and P|(Q|R) \equiv (P|Q)|R }
\end{mathpar}

\begin{mathpar}
  \inferrule* [lab=alpha-equivalence] {} { (x)P \equiv (y)P\{y/x\} \and y \not\in \freenames{P} }
\end{mathpar}

\begin{definition}
Then two processes, $P,Q$, are alpha-equivalent if $P = Q\{\vec{y}/\vec{x}\}$ for
some $\vec{x} \in \boundnames{Q},\vec{y} \in \boundnames{P}$, where $Q\{\vec{y}/\vec{x}\}$
denotes the capture-avoiding substitution of $\vec{y}$ for $\vec{x}$ in $Q$.
\end{definition}

\begin{definition}
  The {\em structural congruence} \cite{SangiorgiWalker} , $\equiv$,
  between processes is the least congruence containing
  alpha-equivalence, satisfying the abelian monoid laws
  (associativity, commutativity and $\pzero$ as identity) for parallel
  composition $|$ and for summation $+$.
\end{definition}

\subsection{Name equivalence}

We take name equivalence, written $\nameeq$, to be the smallest
equivalence relation generated by the following rules.

\begin{mathpar}
\inferrule*[lab=Quote-drop]
{ }
{ \quotep{@{x}} \nameeq x }

\inferrule*[lab=Struct-equiv]
{ P \scong Q }
{ \quotep{P} \nameeq \quotep{Q} }
\end{mathpar}

The astute reader will have noticed that the mutual recursion of names
and processes imposes a mutual recursion on alpha-equivalence and
structural equivalence via name-equivalence. Fortunately, all of this
works out pleasantly and we may calculate in the natural way, free of
concern. The reader interested in the details is referred to the
appendix \ref{appendix:rho_details}.

\subsection{Substitution}

We use $\Proc$ for the set of processes, $\QProc$ for the set of
names, and $\id{\{}\vec{y} / \vec{x} \id{\}}$ to denote partial maps,
$s : \QProc \rightarrow \QProc$. A map, $s$ lifts, uniquely, to a map
on process terms, $\widehat{s} : \Proc \rightarrow \Proc$ by the
following equations.

\begin{mathpar}
  (0) \psubstp{Q}{P} := 0 \\
  (R \juxtap S) \psubstp{Q}{P}
  :=    
  (R)\psubstp{Q}{P} \juxtap (S) \psubstp{Q}{P} \\
  (x?(y).R) \psubstp{Q}{P}    
  :=    
  (x)\substp{Q}{P} (z)\concat( (R \psubstn{z}{y}) \psubstp{Q}{P} ) \\
  (\lift{x}{R}) \psubstp{Q}{P}  
  :=
  \lift{(x)\substp{Q}{P}}{ R \psubstp{Q}{P} } \\
%   (\dropn{x})  \psubstp{Q}{P}       
%   := 
%   \left\{ 
%     \begin{array}{ccc} 
%       \dropn{\quotep{Q}} & & x \nameeq \quotep{P} \\
%       \dropn{x} & & otherwise \\
%     \end{array}
%   \right. 
  (\dropn{x})  \psubstp{Q}{P}       
  := 
  \left\{ 
    \begin{array}{ccc} 
      Q & & x \nameeq \quotep{P} \\
      \dropn{x} & & otherwise \\
    \end{array}
  \right.
\end{mathpar}
 

where

\begin{eqnarray}
  (x)\id{\{} \lpquote Q \rpquote / \lpquote P \rpquote \id{\}}            = 
  \left\{ 
    \begin{array}{ccc}
      \lpquote Q \rpquote & & x \nameeq \lpquote P \rpquote \\
      x & & otherwise \\
    \end{array}
  \right. \nonumber
\end{eqnarray}

and $z$ is chosen distinct from $\quotep{P}$, $\quotep{Q}$, the free
names in $Q$, and all the names in $R$. Our $\alpha$-equivalence will
be built in the standard way from this substitution.

\begin{remark}\label{rem:no_self_referential_names}
  One consequence of these definitions is that $\forall P. \quotep{P}
  \not\in \freenames{P}$.
\end{remark}

\subsection{ Dynamic quote: an example }

Anticipating something of what's to come, consider applying the
substitution, $\widehat{\id{\{}u / z \id{\}}}$, to the following pair
of processes, $\lift{w}{y!(z)}$ and $w[ \lpquote y!(z) \rpquote ]$.

\begin{eqnarray}
	\lift{w}{y!(z)}\widehat{\id{\{}u / z \id{\}}}
		& = &
		\lift{w}{y!(u)} \nonumber\\
	w[ \lpquote y!(z) \rpquote ] \widehat{ \id{\{}u / z \id{\}} }
		& = &
		w[ \lpquote y!(z) \rpquote ] \nonumber
\end{eqnarray}

Because the body of the process between quotes is impervious to
substitution, we get radically different answers. In fact, by
examining the first process in an input context,
e.g. $x?(z).\lift{w}{y!(z)}$, we see that the process under the lift
operator may be shaped by prefixed inputs binding a name inside it. In
this sense, the lift operator will be seen as a way to dynamically
construct processes before reifying them as names.

Finally equipped with these standard features we can present the
dynamics of the calculus.

\subsubsection{Operational semantics} 

Finally, we introduce the computational dynamics. What marks these
algebras as distinct from other more traditionally studied algebraic
structures, e.g. vector spaces or polynomial rings, is the manner in
which dynamics is captured. In traditional structures, dynamics is typically
expressed through morphisms between such structures, as in linear maps
between vector spaces or morphisms between rings. In algebras
associated with the semantics of computation, the dynamics is
expressed as part of the algebraic structure itself, through a
reduction reduction relation typically denoted by $\red$. Below, we
give a recursive presentation of this relation for the calculus used
in the encoding.

$\red \subseteq \pi \times \pi$
$\red : \pi \to \mathcal{P}(\pi)$

\begin{mathpar}
  \inferrule* [lab=Comm] { \textsf{match}( x_{src}, x_{trgt} ) } { x_{trgt}?(y)P \; | \; x_{src}!\langle {Q} \rangle \red P\{\quotep{Q}/y}\} }
  \and \\
  \inferrule* [lab=Par] {{P} \red {P}'} {{{P} | {Q}} \red {{P}' | {Q}}}
  \and
  \inferrule* [lab=Equiv]{{{P} \scong {P}'} \andalso {{P}' \red {Q}'} \andalso {{Q}' \scong {Q}}}{{P} \red {Q}}
\end{mathpar}

\begin{eqnarray*}
  match_{\equiv} (\quotep{P},\quotep{Q}) & := & P \equiv Q \\
  match_{\dagger}(\quotep{P},\quotep{Q}) & := & \forall R. P|Q \red^{*} R => R \red^{*} 0 \\
  match_{K}(\quotep{P},\quotep{Q}) & := & K \mbox{ for some context } K
\end{eqnarray*}

$u?(x)P | u!\langle Q \rangle \red P\{\quotep{Q}/x\}$

%We write $\wred$ for $\red^*$, and $P\red$ if $\exists Q $ such that $ P \red Q$.
We write $P\red$ if $\exists Q $ such that $ P \red Q$ and $P\not\red$, otherwise.

\section{Replication}

As mentioned before, it is known that replication (and hence
recursion) can be implemented in a higher-order process algebra
\cite{SangiorgiWalker}. As our first example of calculation with the
machinery thus far presented we give the construction explicitly in
the {\rhoc}.

\begin{eqnarray}
	D_{x} & := & \prefix{x}{y}{(\binpar{\outputp{x}{y}}{@{y}})} \nonumber\\
	\bangp_{x}{P} & := & \binpar{{x}!\langle{\binpar{D_{x}}{P}}\rangle}{D_{x}} \nonumber
\end{eqnarray}

\begin{eqnarray}
	\bangp_{x}{P} & & \nonumber\\
	=
	& {x}!\langle{(\prefix{x}{y}{(\outputp{x}{y} | @{y})) | P}}\rangle 
	      | \prefix{x}{y}{(\outputp{x}{y} | @{y})} & \nonumber\\
	\red
	& (\outputp{x}{y} | @{y})\substn{\quotep{(\prefix{x}{y}{(@{y} | \outputp{x}{y})) | P}}}{y} & \nonumber\\
	=
	& \outputp{x}{\quotep{(\prefix{x}{y}{(\outputp{x}{y} | @{y})) | P}}}
	  | {(\prefix{x}{y}{(\outputp{x}{y} | @{y})) | P}} & \nonumber\\
	\red
	& \ldots & \nonumber\\
	\red^*
	& P | P | \ldots & \nonumber
\end{eqnarray}

Of course, this encoding, as an implementation, runs away, unfolding
$\bangp{P}$ eagerly. A lazier and more implementable replication
operator, restricted to input-guarded processes, may be obtained as follows.

\begin{eqnarray}
\bangp{\prefix{u}{v}{P}} 
	:= 
	\binpar{\lift{x}{\prefix{u}{v}{(\binpar{D(x)}{P})}}}{D(x)} \nonumber
\end{eqnarray}

\begin{remark}
  Note that the lazier definition still does not deal with summation
  or mixed summation (i.e. sums over input and output). The reader is
  invited to construct definitions of replication that deal with these
  features. 

  Further, the definitions are parameterized in a name, $x$. Can you,
  gentle reader, make a definition that eliminates this parameter and
  guarantees no accidental interaction between the replication
  machinery and the process being replicated -- i.e. no accidental
  sharing of names used by the process to get its work done and the
  name(s) used by the replication to effect copying. This latter
  revision of the definition of replication is crucial to obtaining
  the expected identity $!!P \sim !P$.
\end{remark}

\begin{remark}\label{rem:paradoxical_combinator}
  The reader familiar with the lambda calculus will have noticed the
  similarity between $D$ and the paradoxical combinator.

  [Ed. note: the existence of this seems to suggest we have to be more
  restrictive on the set of processes and names we admit if we are to
  support no-cloning.]
\end{remark}

\subsubsection{Bisimulation}

The computational dynamics gives rise to another kind of equivalence,
the equivalence of computational behavior. As previously mentioned
this is typically captured \emph{via} some form of bisimulation.

% The notion we use in this paper is weak barbed bisimulation
% \cite{milner91polyadicpi}.

The notion we use in this paper is derived from weak barbed
bisimulation \cite{milner91polyadicpi}. 

\begin{definition}
An \emph{observation relation}, $\downarrow_{\mathcal N}$, over a set
of names, $\mathcal N$, is the smallest relation satisfying the rules
below.

\infrule[Out-barb]{y \in {\mathcal N}, \; x \nameeq y}
		  {\outputp{x}{v} \downarrow_{\mathcal N} x}
\infrule[Par-barb]{\mbox{$P\downarrow_{\mathcal N} x$ or $Q\downarrow_{\mathcal N} x$}}
		  {\binpar{P}{Q} \downarrow_{\mathcal N} x}

We write $P \Downarrow_{\mathcal N} x$ if there is $Q$ such that 
$P \wred Q$ and $Q \downarrow_{\mathcal N} x$.
\end{definition}

\begin{definition}
%\label{def.bbisim}
An  ${\mathcal N}$-\emph{barbed bisimulation} over a set of names, ${\mathcal N}$, is a symmetric binary relation 
${\mathcal S}_{\mathcal N}$ between agents such that $P\rel{S}_{\mathcal N}Q$ implies:
\begin{enumerate}
\item If $P \red P'$ then $Q \wred Q'$ and $P'\rel{S}_{\mathcal N} Q'$.
\item If $P\downarrow_{\mathcal N} x$, then $Q\Downarrow_{\mathcal N} x$.
\end{enumerate}
$P$ is ${\mathcal N}$-barbed bisimilar to $Q$, written
$P \wbbisim_{\mathcal N} Q$, if $P \rel{S}_{\mathcal N} Q$ for some ${\mathcal N}$-barbed bisimulation ${\mathcal S}_{\mathcal N}$.
\end{definition}

$\mathcal{R} \subseteq \pi \times \pi$

$P \mathcal{R} Q => \forall P'. P \red P' \Rightarrow \exists Q'. Q \red Q', P' \mathcal{R} Q'$

$P \vdash x \Rightarrow Q \vdash x$

\begin{mathpar}
  \inferrule*[lab=Out-barb]{x \nameeq y}{{y}!\langle{Q}\rangle \vdash x}
  \and
  \inferrule*[lab=Par-barb]{\mbox{$P\vdash x$ or $Q\vdash x$}}{\binpar{P}{Q} \vdash x}
\end{mathpar}

\subsubsection{Contexts}

One of the principle advantages of computational calculi like the
$\pi$-calculus is a well-defined notion of context,
contextual-equivalence and a correlation between
contextual-equivalence and notions of bisimulation. The notion of
context allows the decomposition of a process into (sub-)process and
its syntactic environment, its context. Thus, a context may be
thought of as a process with a ``hole'' (written $\Box$) in it. The
application of a context $M$ to a process $P$, written $M[P]$, is
tantamount to filling the hole in $M$ with $P$. In this paper we do
not need the full weight of this theory, but do make use of the notion
of context in the proof the main theorem. 

\begin{mathpar}
  \inferrule* [lab=summation] {} {{M_{M},M_{N}} \bc \Box \;|\; x.M_{A} \;|\; M_{M}+M_{N}}
  \and
  \inferrule* [lab=agent] {} {{M_{A}} \bc (\vec{x})M_{P} \;| \; \clift{P_0,\ldots,M_{P},\ldots,P_N}}
  \and \\
  \inferrule* [lab=process] {} {{M_{P}} \bc M_{N} \;| \;P|M_{P} }
\end{mathpar} 

\begin{mathpar}
  \inferrule* [lab=sychronization] {} {M_{N} \bc \Box \;|\; x?M_{F} \;|\; x!M_{C}}
  \and
  \inferrule* [lab=abstraction] {} {{M_{F}} \bc (x)M_{P} }
  \and
  \inferrule* [lab=concretion] {} {{M_{C}} \bc \langle M_{P} \rangle }
  \and \\
  \inferrule* [lab=process] {} {{M_{P}} \bc M_{N} \;| \;P|M_{P} }
\end{mathpar}

\begin{definition}[contextual application] Given a context $M$, and
  process $P$, we define the \emph{contextual application}, $M[P] :=
  M\{P/\Box\}$. That is, the contextual application of M to P is the
  substitution of $P$ for $\Box$ in $M$.
\end{definition}

$\meaningof{-} : L \to \mathcal{P}(\pi)$

\begin{mathpar}
  \inferrule* [lab=collection] {} {\meaningof{true} = \pi, \and \meaningof{~E} = \pi \setminus \meaningof{E}, \and \meaningof{E_{1} \& E_{2}} = \meaningof{E_{1}} \cap \meaningof{E_{2}}}
\end{mathpar}

\begin{mathpar}
  \inferrule* [lab=structure] {} {\meaningof{0} = \{ P \in \pi | P \equiv 0 \}, \and \\ \meaningof{E_1 | E_2} = \{ P \in \pi | P \equiv P_{1} | P_{2}, P_{1} \in \meaningof{E_{1}}, P_{2} \in \meaningof{E_2}\} }
\end{mathpar}

\begin{mathpar}
 \inferrule* [lab=behavior] {} {\meaningof{\langle a?b \rangle E} = \{ P \in \pi | P \equiv Q | u?(y)P', \\ \and \\\\ \and \\ \;\;\; u \in \meaningof{a}, \forall z.P'\{z/y\} \in \meaningof{E\{z/b\}}\}, \and \\ \meaningof{a!E} = \{ P \in \pi | P \equiv Q | x!\langle P' \rangle, x \in \meaningof{a} P' \in \meaningof{E}\} }
\end{mathpar}

\begin{mathpar}
 \inferrule* [lab=nominal] {} {\meaningof{\quotep{E}} = \{ \quotep{P} \in \quotep{\pi} | P \in \meaningof{E} \}, \and \meaningof{\quotep{P}} = \{ \quotep{Q} \in \quotep{\pi} | P \equiv Q \} \and \\ \meaningof{@\quotep{E}} = \{ P \in \pi | P \equiv @x, x \in \meaningof{E} \}}
\end{mathpar}

\begin{eqnarray*}
  \\
  \meaningof{-} : TS \to ST
\end{eqnarray*}

\begin{eqnarray*}
  \\
  L : TS \to ST
\end{eqnarray*}

\begin{eqnarray*}
  \\
  P \models E \iff P \in \meaningof{E}
\end{eqnarray*}

\begin{eqnarray*}
  P \approx_{L} Q \iff \forall E \in L. P \models E \iff Q \models E
\end{eqnarray*}

\begin{eqnarray*}
  P \approx_{K} Q
\end{eqnarray*}

\begin{eqnarray*}
  P \approx Q
\end{eqnarray*}

$\approx_{K} = \approx = \approx_{L}$

\subsubsection{Contextual duality}

Note that contexts extend the quotation operation to a family of
operations from processes to names. Given a context, $M$, we can
define a \emph{nominal context}, $\quotep{M}$ by $\quotep{M}[P] :=
\quotep{M[P]}$. To foreshadow what is to come we observe that these
operations enjoy a duality with processes very much like the duality
between vectors and maps from vectors to scalars.

Further, because the calculus is essentially higher-order, we have a
correspondence between contexts and processes. More specifically,
given a name $x$ and a context $M$ we can construct $M^{*}_{x}$ such
that 

\begin{mathpar}
  M^{*}_{x} | \lift{x}{P} \red M[P]
\end{mathpar}

namely,

\begin{mathpar}
  M^{*}_{x} := x?(u).M[\dropn{u}]
\end{mathpar}

The dependence of $M^{*}_{x}$ on a name makes it an abstraction, 

\begin{mathpar}
  M^{*} := (x)x?(u).M[\dropn{u}]
\end{mathpar}

\subsection{Additional notation}

It will sometimes be convenient to denote the process a name
quotes. We already have the notation $x = \quotep{P}$, but it will be
convenient to introduce an alternate notation, $\procn{x}$, when we
want to emphasize the connection to the use of the name. Note that, by
virtue of name equivalence, $\quotep{\procn{x}} \nameeq x$; so, the
notation is consistent with previous definitions.

Further, because names have structure it is possible to effect
substitutions on the basis of that structure. This means we need to
upgrade our notation for substitutions, which we accomplish by
adapting comprehension notation. Thus,

\begin{mathpar}
  P\{ y / x : x \in S \}
\end{mathpar}

is interpreted to mean the process derived from P by replacing (in a
capture-avoiding manner) each occurrence of $x$ in $S$ by $y$. For example,

\begin{mathpar}
  P\{ \quotep{\procn{x}|\procn{x}} / x : x \in \freenames{P} \}
\end{mathpar}

will replace each (occurrence) of a free name $x$ in $P$ by
$\quotep{\procn{x}|\procn{x}}$.

Also, we will avail ourselves of the notation $x^{L}$ and $x^{R}$ to
denote injections of a name into disjoint copies of the name
space. There are numerous ways to accomplish this. One example can be
found in \cite{MeredithR05}. This notation overloads to vectors of
names: $\vec{x}^{\pi} := (x_{i}^{\pi} \; : \; 0 \leq i < |\vec{x}| )$ where $\pi \in \{L,R\}$.

We also use $P^{\Box} := P|\Box$.

In \cite{MeredithR05} an interpretation of the new operator is
given. It turns out that there are several possible interpretations
all enjoying the requisite algebraic properties of the operator (see
\cite{milner91polyadicpi}). We will therefore make liberal use of
$(\nu\; \vec{x})P$.

% subsection the_syntax_and_semantics_of_the_notation_system (end)   

\input{qm2pi.qmops} 

\input{qm2pi.sterngerlach} 

\input{qm2pi.metric} 

% section concurrent_process_calculi (end)

%\input{qm2pi.proofsketch}

% section proof sketch (end)

%\input{qm2pi.slviaknots} 

% section spatial logic via knots (end)

\input{qm2pi.conclusion}

% section conclusion (end)

%\input{qm2pi.dtcodes} 

% section wiring algorithm (end)

\input{qm2pi.ack} 

% section acknowledgments (end)

\newpage


\bibliographystyle{plain}   
\bibliography{../../biblios/main.bib}

\input{qm2pi.rhodetails}

\end{document}

 

%\ifpdf
%\usepackage[pdftex]{graphicx}
%\else
%\usepackage{graphicx}
%\fi

 % \ifpdf
%  \usepackage{pdfsync}
%  \if


%\title{Brief Article}
%\author{David F. Snyder}
%\author{L.G. Meredith}

%\address{Dept. of Math., Texas State University--San Marcos, San Marcos, TX 78666}
       
\pagestyle{empty}


\begin{document}

\lstset{language=[Objective]Caml,frame=shadowbox}

\documentclass[12pt]{llncs}
%\documentclass{jktr}

\usepackage[pdftex]{hyperref}                   
\usepackage {listings}
\usepackage {mathpartir}
\usepackage{bcprules}
%\usepackage{listings}
                       
\usepackage{graphicx} 
%\usepackage[margins=2.5cm,nohead,nofoot]{geometry}
%\usepackage{geometry}
\usepackage{amsfonts}
\usepackage{amstext}
\usepackage{latexsym}
\usepackage{amssymb}
\usepackage{color}


%\include{myPreamble}
\include{qm2pi.local} 

%\ifpdf
%\usepackage[pdftex]{graphicx}
%\else
%\usepackage{graphicx}
%\fi

 % \ifpdf
%  \usepackage{pdfsync}
%  \if


%\title{Brief Article}
%\author{David F. Snyder}
%\author{L.G. Meredith}

%\address{Dept. of Math., Texas State University--San Marcos, San Marcos, TX 78666}
       
\pagestyle{empty}


\begin{document}

\lstset{language=[Objective]Caml,frame=shadowbox}

\input{qm2pi.front}

% section front matter (end)

\input{qm2pi.intro} 
 
% section introduction (end)

% \input{qm2pi.knotations} 

% section notation (end)

\input{qm2pi.process.calculi} 

% section concurrent_process_calculi_and_spatial_logics_ (end)
    
%\input{qm2pi.knots2pi} 

%\input{qm2pi.trefoil} 

%\input{qm2pi.mainthm} 

% subsection basic_interpretation (end)

%\input{qm2pi.rho.presentation} 
\subsection{The syntax and semantics of the notation system}\label{sub:the_syntax_and_semantics_of_the_notation_system} % (fold)

We now summarize a technical presentation of the calculus that
embodies our theory of dynamics. The typical presentation of such a
calculus follows the style of giving generators and relations on
them. The grammar, below, describing term constructors, freely
generates the set of processes, $\Proc$. This set is then quotiented
by a relation known as structural congruence and it is over this set
that the notion of dynamics is expressed. This presentation is
essentially that of \cite{MeredithR05} with the addition of
polyadicity and summation. For readability we have relegated some of
the technical subtleties to an appendix.

\subsubsection{Process grammar}\label{subsub:process_grammar}

\begin{mathpar}
  \inferrule* [lab=synchronization] {} {{M} \bc \pzero \;|\; x?F \;|\; x!C }
  \and
  \inferrule* [lab=abstraction] {} {{F} \bc (x)P}
  \and
  \inferrule* [lab=concretion] {} {{C} \bc \langle Q \rangle}
  \and
  \inferrule* [lab=process] {} {{P,Q} \bc M \;| \;P|Q \;|\; @{x}}
  \and
  \inferrule* [lab=name] {} {{x} \bc \quotep{P}}
\end{mathpar} 

Note that $\vec{x}$ (resp. $\vec{P}$) denotes a vector of names
(resp. processes) of length $|\vec{x}|$ (resp. $|\vec{P}|$). We adopt
the following useful abbreviations.

\begin{mathpar}
   x?(\vec{y}).P := x.(\vec{y})P \and  x\clift{\vec{P}} := x.\clift{\vec{P}}
   \and x!(y) := \lift{x}{\dropn{y}}
   \and \Pi_{i=0}^{n-1}P_i := P_0 | \ldots | P_{n-1}
\end{mathpar}

\subsubsection{Structural congruence}

\paragraph{Free and bound names and alpha-equivalence.} At the
core of structural equivalence is alpha-equivalence which identifies
process that are the same up to a change of variable. Formally, we
recognize the distinction between free and bound names. The free names
of a process, $\freenames{P}$, may be calculated recursively as
follows:

\begin{mathpar}
\freenames{\pzero} := \emptyset
  \and \\
  \freenames{x?(y).P} := \{ x \} \cup (\freenames{P} \setminus \{ y \})
  \and 
  \freenames{x!\langle P \rangle} := \{ x \} \cup \{ P \} 
  \and \\
  \freenames{P|Q} := \freenames{P} \cup \freenames{Q}
  \and \\
  \freenames{@{x}} := \{ x \}
\end{mathpar}

$\pi$
$\quotep{\pi}$

$\freenames{-} : \pi \to \mathcal{P}(\quotep{\pi})$

\begin{eqnarray*}
  \freenames{\pzero} & := & \emptyset \\
  \freenames{x?(y).P} & := & \{ x \} \cup (\freenames{P} \setminus \{ y \}) \\
  \freenames{x!\langle P \rangle} & := & \{ x \} \cup \{ P \} \\
  \freenames{P|Q} & := & \freenames{P} \cup \freenames{Q} \\
  \freenames{\dropn{x}} & := & \{ x \}
\end{eqnarray*}

The bound names of a process, $\boundnames{P}$, are those names occurring in $P$
that are not free. For example, in $x?(y).0$, the name $x$ is free, while $y$ is bound.

\begin{mathpar}
  \inferrule* [lab=monoidal-laws] {} { P|Q \equiv Q|P \and P|0 \equiv P \and P|(Q|R) \equiv (P|Q)|R }
\end{mathpar}

\begin{mathpar}
  \inferrule* [lab=alpha-equivalence] {} { (x)P \equiv (y)P\{y/x\} \and y \not\in \freenames{P} }
\end{mathpar}

\begin{definition}
Then two processes, $P,Q$, are alpha-equivalent if $P = Q\{\vec{y}/\vec{x}\}$ for
some $\vec{x} \in \boundnames{Q},\vec{y} \in \boundnames{P}$, where $Q\{\vec{y}/\vec{x}\}$
denotes the capture-avoiding substitution of $\vec{y}$ for $\vec{x}$ in $Q$.
\end{definition}

\begin{definition}
  The {\em structural congruence} \cite{SangiorgiWalker} , $\equiv$,
  between processes is the least congruence containing
  alpha-equivalence, satisfying the abelian monoid laws
  (associativity, commutativity and $\pzero$ as identity) for parallel
  composition $|$ and for summation $+$.
\end{definition}

\subsection{Name equivalence}

We take name equivalence, written $\nameeq$, to be the smallest
equivalence relation generated by the following rules.

\begin{mathpar}
\inferrule*[lab=Quote-drop]
{ }
{ \quotep{@{x}} \nameeq x }

\inferrule*[lab=Struct-equiv]
{ P \scong Q }
{ \quotep{P} \nameeq \quotep{Q} }
\end{mathpar}

The astute reader will have noticed that the mutual recursion of names
and processes imposes a mutual recursion on alpha-equivalence and
structural equivalence via name-equivalence. Fortunately, all of this
works out pleasantly and we may calculate in the natural way, free of
concern. The reader interested in the details is referred to the
appendix \ref{appendix:rho_details}.

\subsection{Substitution}

We use $\Proc$ for the set of processes, $\QProc$ for the set of
names, and $\id{\{}\vec{y} / \vec{x} \id{\}}$ to denote partial maps,
$s : \QProc \rightarrow \QProc$. A map, $s$ lifts, uniquely, to a map
on process terms, $\widehat{s} : \Proc \rightarrow \Proc$ by the
following equations.

\begin{mathpar}
  (0) \psubstp{Q}{P} := 0 \\
  (R \juxtap S) \psubstp{Q}{P}
  :=    
  (R)\psubstp{Q}{P} \juxtap (S) \psubstp{Q}{P} \\
  (x?(y).R) \psubstp{Q}{P}    
  :=    
  (x)\substp{Q}{P} (z)\concat( (R \psubstn{z}{y}) \psubstp{Q}{P} ) \\
  (\lift{x}{R}) \psubstp{Q}{P}  
  :=
  \lift{(x)\substp{Q}{P}}{ R \psubstp{Q}{P} } \\
%   (\dropn{x})  \psubstp{Q}{P}       
%   := 
%   \left\{ 
%     \begin{array}{ccc} 
%       \dropn{\quotep{Q}} & & x \nameeq \quotep{P} \\
%       \dropn{x} & & otherwise \\
%     \end{array}
%   \right. 
  (\dropn{x})  \psubstp{Q}{P}       
  := 
  \left\{ 
    \begin{array}{ccc} 
      Q & & x \nameeq \quotep{P} \\
      \dropn{x} & & otherwise \\
    \end{array}
  \right.
\end{mathpar}
 

where

\begin{eqnarray}
  (x)\id{\{} \lpquote Q \rpquote / \lpquote P \rpquote \id{\}}            = 
  \left\{ 
    \begin{array}{ccc}
      \lpquote Q \rpquote & & x \nameeq \lpquote P \rpquote \\
      x & & otherwise \\
    \end{array}
  \right. \nonumber
\end{eqnarray}

and $z$ is chosen distinct from $\quotep{P}$, $\quotep{Q}$, the free
names in $Q$, and all the names in $R$. Our $\alpha$-equivalence will
be built in the standard way from this substitution.

\begin{remark}\label{rem:no_self_referential_names}
  One consequence of these definitions is that $\forall P. \quotep{P}
  \not\in \freenames{P}$.
\end{remark}

\subsection{ Dynamic quote: an example }

Anticipating something of what's to come, consider applying the
substitution, $\widehat{\id{\{}u / z \id{\}}}$, to the following pair
of processes, $\lift{w}{y!(z)}$ and $w[ \lpquote y!(z) \rpquote ]$.

\begin{eqnarray}
	\lift{w}{y!(z)}\widehat{\id{\{}u / z \id{\}}}
		& = &
		\lift{w}{y!(u)} \nonumber\\
	w[ \lpquote y!(z) \rpquote ] \widehat{ \id{\{}u / z \id{\}} }
		& = &
		w[ \lpquote y!(z) \rpquote ] \nonumber
\end{eqnarray}

Because the body of the process between quotes is impervious to
substitution, we get radically different answers. In fact, by
examining the first process in an input context,
e.g. $x?(z).\lift{w}{y!(z)}$, we see that the process under the lift
operator may be shaped by prefixed inputs binding a name inside it. In
this sense, the lift operator will be seen as a way to dynamically
construct processes before reifying them as names.

Finally equipped with these standard features we can present the
dynamics of the calculus.

\subsubsection{Operational semantics} 

Finally, we introduce the computational dynamics. What marks these
algebras as distinct from other more traditionally studied algebraic
structures, e.g. vector spaces or polynomial rings, is the manner in
which dynamics is captured. In traditional structures, dynamics is typically
expressed through morphisms between such structures, as in linear maps
between vector spaces or morphisms between rings. In algebras
associated with the semantics of computation, the dynamics is
expressed as part of the algebraic structure itself, through a
reduction reduction relation typically denoted by $\red$. Below, we
give a recursive presentation of this relation for the calculus used
in the encoding.

$\red \subseteq \pi \times \pi$
$\red : \pi \to \mathcal{P}(\pi)$

\begin{mathpar}
  \inferrule* [lab=Comm] { \textsf{match}( x_{src}, x_{trgt} ) } { x_{trgt}?(y)P \; | \; x_{src}!\langle {Q} \rangle \red P\{\quotep{Q}/y}\} }
  \and \\
  \inferrule* [lab=Par] {{P} \red {P}'} {{{P} | {Q}} \red {{P}' | {Q}}}
  \and
  \inferrule* [lab=Equiv]{{{P} \scong {P}'} \andalso {{P}' \red {Q}'} \andalso {{Q}' \scong {Q}}}{{P} \red {Q}}
\end{mathpar}

\begin{eqnarray*}
  match_{\equiv} (\quotep{P},\quotep{Q}) & := & P \equiv Q \\
  match_{\dagger}(\quotep{P},\quotep{Q}) & := & \forall R. P|Q \red^{*} R => R \red^{*} 0 \\
  match_{K}(\quotep{P},\quotep{Q}) & := & K \mbox{ for some context } K
\end{eqnarray*}

$u?(x)P | u!\langle Q \rangle \red P\{\quotep{Q}/x\}$

%We write $\wred$ for $\red^*$, and $P\red$ if $\exists Q $ such that $ P \red Q$.
We write $P\red$ if $\exists Q $ such that $ P \red Q$ and $P\not\red$, otherwise.

\section{Replication}

As mentioned before, it is known that replication (and hence
recursion) can be implemented in a higher-order process algebra
\cite{SangiorgiWalker}. As our first example of calculation with the
machinery thus far presented we give the construction explicitly in
the {\rhoc}.

\begin{eqnarray}
	D_{x} & := & \prefix{x}{y}{(\binpar{\outputp{x}{y}}{@{y}})} \nonumber\\
	\bangp_{x}{P} & := & \binpar{{x}!\langle{\binpar{D_{x}}{P}}\rangle}{D_{x}} \nonumber
\end{eqnarray}

\begin{eqnarray}
	\bangp_{x}{P} & & \nonumber\\
	=
	& {x}!\langle{(\prefix{x}{y}{(\outputp{x}{y} | @{y})) | P}}\rangle 
	      | \prefix{x}{y}{(\outputp{x}{y} | @{y})} & \nonumber\\
	\red
	& (\outputp{x}{y} | @{y})\substn{\quotep{(\prefix{x}{y}{(@{y} | \outputp{x}{y})) | P}}}{y} & \nonumber\\
	=
	& \outputp{x}{\quotep{(\prefix{x}{y}{(\outputp{x}{y} | @{y})) | P}}}
	  | {(\prefix{x}{y}{(\outputp{x}{y} | @{y})) | P}} & \nonumber\\
	\red
	& \ldots & \nonumber\\
	\red^*
	& P | P | \ldots & \nonumber
\end{eqnarray}

Of course, this encoding, as an implementation, runs away, unfolding
$\bangp{P}$ eagerly. A lazier and more implementable replication
operator, restricted to input-guarded processes, may be obtained as follows.

\begin{eqnarray}
\bangp{\prefix{u}{v}{P}} 
	:= 
	\binpar{\lift{x}{\prefix{u}{v}{(\binpar{D(x)}{P})}}}{D(x)} \nonumber
\end{eqnarray}

\begin{remark}
  Note that the lazier definition still does not deal with summation
  or mixed summation (i.e. sums over input and output). The reader is
  invited to construct definitions of replication that deal with these
  features. 

  Further, the definitions are parameterized in a name, $x$. Can you,
  gentle reader, make a definition that eliminates this parameter and
  guarantees no accidental interaction between the replication
  machinery and the process being replicated -- i.e. no accidental
  sharing of names used by the process to get its work done and the
  name(s) used by the replication to effect copying. This latter
  revision of the definition of replication is crucial to obtaining
  the expected identity $!!P \sim !P$.
\end{remark}

\begin{remark}\label{rem:paradoxical_combinator}
  The reader familiar with the lambda calculus will have noticed the
  similarity between $D$ and the paradoxical combinator.

  [Ed. note: the existence of this seems to suggest we have to be more
  restrictive on the set of processes and names we admit if we are to
  support no-cloning.]
\end{remark}

\subsubsection{Bisimulation}

The computational dynamics gives rise to another kind of equivalence,
the equivalence of computational behavior. As previously mentioned
this is typically captured \emph{via} some form of bisimulation.

% The notion we use in this paper is weak barbed bisimulation
% \cite{milner91polyadicpi}.

The notion we use in this paper is derived from weak barbed
bisimulation \cite{milner91polyadicpi}. 

\begin{definition}
An \emph{observation relation}, $\downarrow_{\mathcal N}$, over a set
of names, $\mathcal N$, is the smallest relation satisfying the rules
below.

\infrule[Out-barb]{y \in {\mathcal N}, \; x \nameeq y}
		  {\outputp{x}{v} \downarrow_{\mathcal N} x}
\infrule[Par-barb]{\mbox{$P\downarrow_{\mathcal N} x$ or $Q\downarrow_{\mathcal N} x$}}
		  {\binpar{P}{Q} \downarrow_{\mathcal N} x}

We write $P \Downarrow_{\mathcal N} x$ if there is $Q$ such that 
$P \wred Q$ and $Q \downarrow_{\mathcal N} x$.
\end{definition}

\begin{definition}
%\label{def.bbisim}
An  ${\mathcal N}$-\emph{barbed bisimulation} over a set of names, ${\mathcal N}$, is a symmetric binary relation 
${\mathcal S}_{\mathcal N}$ between agents such that $P\rel{S}_{\mathcal N}Q$ implies:
\begin{enumerate}
\item If $P \red P'$ then $Q \wred Q'$ and $P'\rel{S}_{\mathcal N} Q'$.
\item If $P\downarrow_{\mathcal N} x$, then $Q\Downarrow_{\mathcal N} x$.
\end{enumerate}
$P$ is ${\mathcal N}$-barbed bisimilar to $Q$, written
$P \wbbisim_{\mathcal N} Q$, if $P \rel{S}_{\mathcal N} Q$ for some ${\mathcal N}$-barbed bisimulation ${\mathcal S}_{\mathcal N}$.
\end{definition}

$\mathcal{R} \subseteq \pi \times \pi$

$P \mathcal{R} Q => \forall P'. P \red P' \Rightarrow \exists Q'. Q \red Q', P' \mathcal{R} Q'$

$P \vdash x \Rightarrow Q \vdash x$

\begin{mathpar}
  \inferrule*[lab=Out-barb]{x \nameeq y}{{y}!\langle{Q}\rangle \vdash x}
  \and
  \inferrule*[lab=Par-barb]{\mbox{$P\vdash x$ or $Q\vdash x$}}{\binpar{P}{Q} \vdash x}
\end{mathpar}

\subsubsection{Contexts}

One of the principle advantages of computational calculi like the
$\pi$-calculus is a well-defined notion of context,
contextual-equivalence and a correlation between
contextual-equivalence and notions of bisimulation. The notion of
context allows the decomposition of a process into (sub-)process and
its syntactic environment, its context. Thus, a context may be
thought of as a process with a ``hole'' (written $\Box$) in it. The
application of a context $M$ to a process $P$, written $M[P]$, is
tantamount to filling the hole in $M$ with $P$. In this paper we do
not need the full weight of this theory, but do make use of the notion
of context in the proof the main theorem. 

\begin{mathpar}
  \inferrule* [lab=summation] {} {{M_{M},M_{N}} \bc \Box \;|\; x.M_{A} \;|\; M_{M}+M_{N}}
  \and
  \inferrule* [lab=agent] {} {{M_{A}} \bc (\vec{x})M_{P} \;| \; \clift{P_0,\ldots,M_{P},\ldots,P_N}}
  \and \\
  \inferrule* [lab=process] {} {{M_{P}} \bc M_{N} \;| \;P|M_{P} }
\end{mathpar} 

\begin{mathpar}
  \inferrule* [lab=sychronization] {} {M_{N} \bc \Box \;|\; x?M_{F} \;|\; x!M_{C}}
  \and
  \inferrule* [lab=abstraction] {} {{M_{F}} \bc (x)M_{P} }
  \and
  \inferrule* [lab=concretion] {} {{M_{C}} \bc \langle M_{P} \rangle }
  \and \\
  \inferrule* [lab=process] {} {{M_{P}} \bc M_{N} \;| \;P|M_{P} }
\end{mathpar}

\begin{definition}[contextual application] Given a context $M$, and
  process $P$, we define the \emph{contextual application}, $M[P] :=
  M\{P/\Box\}$. That is, the contextual application of M to P is the
  substitution of $P$ for $\Box$ in $M$.
\end{definition}

$\meaningof{-} : L \to \mathcal{P}(\pi)$

\begin{mathpar}
  \inferrule* [lab=collection] {} {\meaningof{true} = \pi, \and \meaningof{~E} = \pi \setminus \meaningof{E}, \and \meaningof{E_{1} \& E_{2}} = \meaningof{E_{1}} \cap \meaningof{E_{2}}}
\end{mathpar}

\begin{mathpar}
  \inferrule* [lab=structure] {} {\meaningof{0} = \{ P \in \pi | P \equiv 0 \}, \and \\ \meaningof{E_1 | E_2} = \{ P \in \pi | P \equiv P_{1} | P_{2}, P_{1} \in \meaningof{E_{1}}, P_{2} \in \meaningof{E_2}\} }
\end{mathpar}

\begin{mathpar}
 \inferrule* [lab=behavior] {} {\meaningof{\langle a?b \rangle E} = \{ P \in \pi | P \equiv Q | u?(y)P', \\ \and \\\\ \and \\ \;\;\; u \in \meaningof{a}, \forall z.P'\{z/y\} \in \meaningof{E\{z/b\}}\}, \and \\ \meaningof{a!E} = \{ P \in \pi | P \equiv Q | x!\langle P' \rangle, x \in \meaningof{a} P' \in \meaningof{E}\} }
\end{mathpar}

\begin{mathpar}
 \inferrule* [lab=nominal] {} {\meaningof{\quotep{E}} = \{ \quotep{P} \in \quotep{\pi} | P \in \meaningof{E} \}, \and \meaningof{\quotep{P}} = \{ \quotep{Q} \in \quotep{\pi} | P \equiv Q \} \and \\ \meaningof{@\quotep{E}} = \{ P \in \pi | P \equiv @x, x \in \meaningof{E} \}}
\end{mathpar}

\begin{eqnarray*}
  \\
  \meaningof{-} : TS \to ST
\end{eqnarray*}

\begin{eqnarray*}
  \\
  L : TS \to ST
\end{eqnarray*}

\begin{eqnarray*}
  \\
  P \models E \iff P \in \meaningof{E}
\end{eqnarray*}

\begin{eqnarray*}
  P \approx_{L} Q \iff \forall E \in L. P \models E \iff Q \models E
\end{eqnarray*}

\begin{eqnarray*}
  P \approx_{K} Q
\end{eqnarray*}

\begin{eqnarray*}
  P \approx Q
\end{eqnarray*}

$\approx_{K} = \approx = \approx_{L}$

\subsubsection{Contextual duality}

Note that contexts extend the quotation operation to a family of
operations from processes to names. Given a context, $M$, we can
define a \emph{nominal context}, $\quotep{M}$ by $\quotep{M}[P] :=
\quotep{M[P]}$. To foreshadow what is to come we observe that these
operations enjoy a duality with processes very much like the duality
between vectors and maps from vectors to scalars.

Further, because the calculus is essentially higher-order, we have a
correspondence between contexts and processes. More specifically,
given a name $x$ and a context $M$ we can construct $M^{*}_{x}$ such
that 

\begin{mathpar}
  M^{*}_{x} | \lift{x}{P} \red M[P]
\end{mathpar}

namely,

\begin{mathpar}
  M^{*}_{x} := x?(u).M[\dropn{u}]
\end{mathpar}

The dependence of $M^{*}_{x}$ on a name makes it an abstraction, 

\begin{mathpar}
  M^{*} := (x)x?(u).M[\dropn{u}]
\end{mathpar}

\subsection{Additional notation}

It will sometimes be convenient to denote the process a name
quotes. We already have the notation $x = \quotep{P}$, but it will be
convenient to introduce an alternate notation, $\procn{x}$, when we
want to emphasize the connection to the use of the name. Note that, by
virtue of name equivalence, $\quotep{\procn{x}} \nameeq x$; so, the
notation is consistent with previous definitions.

Further, because names have structure it is possible to effect
substitutions on the basis of that structure. This means we need to
upgrade our notation for substitutions, which we accomplish by
adapting comprehension notation. Thus,

\begin{mathpar}
  P\{ y / x : x \in S \}
\end{mathpar}

is interpreted to mean the process derived from P by replacing (in a
capture-avoiding manner) each occurrence of $x$ in $S$ by $y$. For example,

\begin{mathpar}
  P\{ \quotep{\procn{x}|\procn{x}} / x : x \in \freenames{P} \}
\end{mathpar}

will replace each (occurrence) of a free name $x$ in $P$ by
$\quotep{\procn{x}|\procn{x}}$.

Also, we will avail ourselves of the notation $x^{L}$ and $x^{R}$ to
denote injections of a name into disjoint copies of the name
space. There are numerous ways to accomplish this. One example can be
found in \cite{MeredithR05}. This notation overloads to vectors of
names: $\vec{x}^{\pi} := (x_{i}^{\pi} \; : \; 0 \leq i < |\vec{x}| )$ where $\pi \in \{L,R\}$.

We also use $P^{\Box} := P|\Box$.

In \cite{MeredithR05} an interpretation of the new operator is
given. It turns out that there are several possible interpretations
all enjoying the requisite algebraic properties of the operator (see
\cite{milner91polyadicpi}). We will therefore make liberal use of
$(\nu\; \vec{x})P$.

% subsection the_syntax_and_semantics_of_the_notation_system (end)   

\input{qm2pi.qmops} 

\input{qm2pi.sterngerlach} 

\input{qm2pi.metric} 

% section concurrent_process_calculi (end)

%\input{qm2pi.proofsketch}

% section proof sketch (end)

%\input{qm2pi.slviaknots} 

% section spatial logic via knots (end)

\input{qm2pi.conclusion}

% section conclusion (end)

%\input{qm2pi.dtcodes} 

% section wiring algorithm (end)

\input{qm2pi.ack} 

% section acknowledgments (end)

\newpage


\bibliographystyle{plain}   
\bibliography{../../biblios/main.bib}

\input{qm2pi.rhodetails}

\end{document}



% section front matter (end)

\section{Introduction}\label{sec:introduction} % (fold)
In this draft of the material i am going to have to dispense with the
usual writing conventions adopted in papers on these topics. i'm going
to have adopt whatever tone i need at the time i'm writing up the
calculations. Sometimes this may be very conversational; others it may
be the barest mathematical grunts; others still it may be that i have
lifted text from one of my other papers because the exposition of some
point was better said there. i hope that my readers are not unduly put
out by this decision. i'm not doing this to flout convention or be
rebellious. i find these calculations very technically challenging. To
keep everything going technically, something has to give; i have to
let go of some cognitive burden. So, the academic writing style --
with all of its trade-offs in terms of facilitating technical
communication -- is what i'm letting go of. Perhaps subsequent drafts
can be tightened and polished, but for now, i'm going to speak as if
we were sitting together in a coffee shop with a laptop, wifi and a
pad of paper and a pencil.

So, here's what i have to say. We -- you and i, comfortably ensconced
in our coffee shop and well-equipped with our tools -- can realize and
carry out the calculations of quantum mechanics over a very different
formal theory of dynamics, a formal theory of dynamics that
corresponds to a theory of concurrent computation with
\emph{reflection}. It has the advantage that the underlying theory is
already `quantized', but supports analogues all of the continuuous
operations. Strikingly, this underlying theory has recently been
connected with a notion of metric that we can show, by calculating
together, coincides with the metric induced by the inner product.

There are a lot of reasons why you might be interested in seeing
calculations of this form. Here's why i'm interested. For the past
several centuries there has been no competitor to the ``Newtonian''
account of dynamics. As a result the predominant share of accounts of
dynamical systems and situations have had to be formulated in terms of
the Newtonian machinery. i view this as an intellectually dangerous
position to occupy. Everything, despite it's intrinsic shape, turns
into a nail to be hit with this hammer. Recently, however, the theory
of computation has matured to the point where we have candidates for
theories of dynamics that offer very different perspective on
reasoning about dynamical systems and situations. Testing these
candidates against very successful accounts of dynamical situations,
like quantum mechanics, is going to give us some sense of how mature
they are and some measure of the quality of these accounts of
dynamics.

\subsection{Summary of contributions and outline of paper}

So, we're going to develop an interpretation of the operations of
quantum mechanics normally interpreted by Hilbert spaces and
operators. We're going to do this over a theory of computation. Note
that this is very different than the usual quantum computation program
which develops notions of computation over quantum mechanics. Rather,
we are developing a story that aligns with Wheeler's slogan: It from
Bit. To do this we will first provide an account of the theory of
computation at play here. Then we will dive into a calculation-driven
interpretation of the operations of quantum mechanics.

The reason we take this approach is that -- until very recently --
there hasn't been an axiomatic account of quantum mechanics. As a
result there has been no sharp delineation of the mathematical theory
supporting interpretation of the physical theory and the physical
theory, itself. So, ambient features of the maths are free to be
exploited (or supressed) without a real accounting of their physical
relevance. There is no sharp statement ``here's the physical theory''
qua \emph{theory} and ``here's the mathematical interpretation''
enabling a judgment of how faithful the interpretation is -- apart
from experimental observation. When there is an axiomatic account we
can judge how well a given mathematical formalism supports an
interpretation of the axioms, independent of
experimentation. Likewise, we can judge how well we have captured our
physical evidence and experience with our axiomatics, independent of
any specific mathematical implementation, with accidental detail that
may or may not have physical significance. 

In lieu of a fully fleshed out and vetted axiomatic account of quantum
mechanics, interpreting the operational notions in service of modeling
physical systems will have to suffice. In other words, we are not in
the business of providing a model of Hilbert spaces and operators. We
are in the business of providing a model of quantum mechanics because
we are motivated by testing our notions of dynamics against physical
theory; and, the predictive calculations of the physical theory must
serve as the best formulation -- shy of a fully fleshed out axiomatic
account -- of the physical theory itself (as they have for scientific
theories since time immemorial). Put another way, despite a
whole-hearted commitment to an It-from-Bit ontology, we are firmly
aligned with the shut-up-and-calculate camp as the best way to obtain
results either from the physical perspective or as a quality assurance
measure of our fledgling theory of dynamics.

In detail, we present a reflective process calculus. Then we develop
intuitive correspondences between the notions available in this
calculus and the usual physical notions supporting quantum mechanical
calculations. Thus, 

\begin{table}[htp]
  \center{
    \fbox{
      \begin{tabular}{c|c}
        quantum mechanics & process calculus \\
        \hline
        scalar & name \\
        state vector & process \\
        dual & contextual duals \\
        matrix & formal sums of process-context-dual pairs \\
        orthogonality & process annihilation \\
        inner product & execution-formula + quoting
      \end{tabular}
    }
  }
  \caption{QM - process calculi correspondences}
\end{table}

Then we tighten up these intuitions to operational definitions. We
employ the Dirac notation as the best proxy we can find for an
abstract syntax of the quantum mechanical notions. The definitions we
develop put us in contact with equational constraints coming from the
theory that we demonstrate the definitions and calculations satisfy.

This puts us in a position to shut up and calculate for the
Stern-Gerlach experimental set up, showing how these predictive
calculations become calculations on processes in our theory of a
reflective process calculus.

Penultimately, we demonstrate that the notion of metric coming from
the inner product coincides with the notion of metric available from
the theory of bisimulation. This demonstration gives us the right to
think of space as arising from behavior. Finally, we consider where we
might go from the new vantage point we have obtained.

% section introduction (end) 
 
% section introduction (end)

% \documentclass[12pt]{llncs}
%\documentclass{jktr}

\usepackage[pdftex]{hyperref}                   
\usepackage {listings}
\usepackage {mathpartir}
\usepackage{bcprules}
%\usepackage{listings}
                       
\usepackage{graphicx} 
%\usepackage[margins=2.5cm,nohead,nofoot]{geometry}
%\usepackage{geometry}
\usepackage{amsfonts}
\usepackage{amstext}
\usepackage{latexsym}
\usepackage{amssymb}
\usepackage{color}


%\include{myPreamble}
\include{qm2pi.local} 

%\ifpdf
%\usepackage[pdftex]{graphicx}
%\else
%\usepackage{graphicx}
%\fi

 % \ifpdf
%  \usepackage{pdfsync}
%  \if


%\title{Brief Article}
%\author{David F. Snyder}
%\author{L.G. Meredith}

%\address{Dept. of Math., Texas State University--San Marcos, San Marcos, TX 78666}
       
\pagestyle{empty}


\begin{document}

\lstset{language=[Objective]Caml,frame=shadowbox}

\input{qm2pi.front}

% section front matter (end)

\input{qm2pi.intro} 
 
% section introduction (end)

% \input{qm2pi.knotations} 

% section notation (end)

\input{qm2pi.process.calculi} 

% section concurrent_process_calculi_and_spatial_logics_ (end)
    
%\input{qm2pi.knots2pi} 

%\input{qm2pi.trefoil} 

%\input{qm2pi.mainthm} 

% subsection basic_interpretation (end)

%\input{qm2pi.rho.presentation} 
\subsection{The syntax and semantics of the notation system}\label{sub:the_syntax_and_semantics_of_the_notation_system} % (fold)

We now summarize a technical presentation of the calculus that
embodies our theory of dynamics. The typical presentation of such a
calculus follows the style of giving generators and relations on
them. The grammar, below, describing term constructors, freely
generates the set of processes, $\Proc$. This set is then quotiented
by a relation known as structural congruence and it is over this set
that the notion of dynamics is expressed. This presentation is
essentially that of \cite{MeredithR05} with the addition of
polyadicity and summation. For readability we have relegated some of
the technical subtleties to an appendix.

\subsubsection{Process grammar}\label{subsub:process_grammar}

\begin{mathpar}
  \inferrule* [lab=synchronization] {} {{M} \bc \pzero \;|\; x?F \;|\; x!C }
  \and
  \inferrule* [lab=abstraction] {} {{F} \bc (x)P}
  \and
  \inferrule* [lab=concretion] {} {{C} \bc \langle Q \rangle}
  \and
  \inferrule* [lab=process] {} {{P,Q} \bc M \;| \;P|Q \;|\; @{x}}
  \and
  \inferrule* [lab=name] {} {{x} \bc \quotep{P}}
\end{mathpar} 

Note that $\vec{x}$ (resp. $\vec{P}$) denotes a vector of names
(resp. processes) of length $|\vec{x}|$ (resp. $|\vec{P}|$). We adopt
the following useful abbreviations.

\begin{mathpar}
   x?(\vec{y}).P := x.(\vec{y})P \and  x\clift{\vec{P}} := x.\clift{\vec{P}}
   \and x!(y) := \lift{x}{\dropn{y}}
   \and \Pi_{i=0}^{n-1}P_i := P_0 | \ldots | P_{n-1}
\end{mathpar}

\subsubsection{Structural congruence}

\paragraph{Free and bound names and alpha-equivalence.} At the
core of structural equivalence is alpha-equivalence which identifies
process that are the same up to a change of variable. Formally, we
recognize the distinction between free and bound names. The free names
of a process, $\freenames{P}$, may be calculated recursively as
follows:

\begin{mathpar}
\freenames{\pzero} := \emptyset
  \and \\
  \freenames{x?(y).P} := \{ x \} \cup (\freenames{P} \setminus \{ y \})
  \and 
  \freenames{x!\langle P \rangle} := \{ x \} \cup \{ P \} 
  \and \\
  \freenames{P|Q} := \freenames{P} \cup \freenames{Q}
  \and \\
  \freenames{@{x}} := \{ x \}
\end{mathpar}

$\pi$
$\quotep{\pi}$

$\freenames{-} : \pi \to \mathcal{P}(\quotep{\pi})$

\begin{eqnarray*}
  \freenames{\pzero} & := & \emptyset \\
  \freenames{x?(y).P} & := & \{ x \} \cup (\freenames{P} \setminus \{ y \}) \\
  \freenames{x!\langle P \rangle} & := & \{ x \} \cup \{ P \} \\
  \freenames{P|Q} & := & \freenames{P} \cup \freenames{Q} \\
  \freenames{\dropn{x}} & := & \{ x \}
\end{eqnarray*}

The bound names of a process, $\boundnames{P}$, are those names occurring in $P$
that are not free. For example, in $x?(y).0$, the name $x$ is free, while $y$ is bound.

\begin{mathpar}
  \inferrule* [lab=monoidal-laws] {} { P|Q \equiv Q|P \and P|0 \equiv P \and P|(Q|R) \equiv (P|Q)|R }
\end{mathpar}

\begin{mathpar}
  \inferrule* [lab=alpha-equivalence] {} { (x)P \equiv (y)P\{y/x\} \and y \not\in \freenames{P} }
\end{mathpar}

\begin{definition}
Then two processes, $P,Q$, are alpha-equivalent if $P = Q\{\vec{y}/\vec{x}\}$ for
some $\vec{x} \in \boundnames{Q},\vec{y} \in \boundnames{P}$, where $Q\{\vec{y}/\vec{x}\}$
denotes the capture-avoiding substitution of $\vec{y}$ for $\vec{x}$ in $Q$.
\end{definition}

\begin{definition}
  The {\em structural congruence} \cite{SangiorgiWalker} , $\equiv$,
  between processes is the least congruence containing
  alpha-equivalence, satisfying the abelian monoid laws
  (associativity, commutativity and $\pzero$ as identity) for parallel
  composition $|$ and for summation $+$.
\end{definition}

\subsection{Name equivalence}

We take name equivalence, written $\nameeq$, to be the smallest
equivalence relation generated by the following rules.

\begin{mathpar}
\inferrule*[lab=Quote-drop]
{ }
{ \quotep{@{x}} \nameeq x }

\inferrule*[lab=Struct-equiv]
{ P \scong Q }
{ \quotep{P} \nameeq \quotep{Q} }
\end{mathpar}

The astute reader will have noticed that the mutual recursion of names
and processes imposes a mutual recursion on alpha-equivalence and
structural equivalence via name-equivalence. Fortunately, all of this
works out pleasantly and we may calculate in the natural way, free of
concern. The reader interested in the details is referred to the
appendix \ref{appendix:rho_details}.

\subsection{Substitution}

We use $\Proc$ for the set of processes, $\QProc$ for the set of
names, and $\id{\{}\vec{y} / \vec{x} \id{\}}$ to denote partial maps,
$s : \QProc \rightarrow \QProc$. A map, $s$ lifts, uniquely, to a map
on process terms, $\widehat{s} : \Proc \rightarrow \Proc$ by the
following equations.

\begin{mathpar}
  (0) \psubstp{Q}{P} := 0 \\
  (R \juxtap S) \psubstp{Q}{P}
  :=    
  (R)\psubstp{Q}{P} \juxtap (S) \psubstp{Q}{P} \\
  (x?(y).R) \psubstp{Q}{P}    
  :=    
  (x)\substp{Q}{P} (z)\concat( (R \psubstn{z}{y}) \psubstp{Q}{P} ) \\
  (\lift{x}{R}) \psubstp{Q}{P}  
  :=
  \lift{(x)\substp{Q}{P}}{ R \psubstp{Q}{P} } \\
%   (\dropn{x})  \psubstp{Q}{P}       
%   := 
%   \left\{ 
%     \begin{array}{ccc} 
%       \dropn{\quotep{Q}} & & x \nameeq \quotep{P} \\
%       \dropn{x} & & otherwise \\
%     \end{array}
%   \right. 
  (\dropn{x})  \psubstp{Q}{P}       
  := 
  \left\{ 
    \begin{array}{ccc} 
      Q & & x \nameeq \quotep{P} \\
      \dropn{x} & & otherwise \\
    \end{array}
  \right.
\end{mathpar}
 

where

\begin{eqnarray}
  (x)\id{\{} \lpquote Q \rpquote / \lpquote P \rpquote \id{\}}            = 
  \left\{ 
    \begin{array}{ccc}
      \lpquote Q \rpquote & & x \nameeq \lpquote P \rpquote \\
      x & & otherwise \\
    \end{array}
  \right. \nonumber
\end{eqnarray}

and $z$ is chosen distinct from $\quotep{P}$, $\quotep{Q}$, the free
names in $Q$, and all the names in $R$. Our $\alpha$-equivalence will
be built in the standard way from this substitution.

\begin{remark}\label{rem:no_self_referential_names}
  One consequence of these definitions is that $\forall P. \quotep{P}
  \not\in \freenames{P}$.
\end{remark}

\subsection{ Dynamic quote: an example }

Anticipating something of what's to come, consider applying the
substitution, $\widehat{\id{\{}u / z \id{\}}}$, to the following pair
of processes, $\lift{w}{y!(z)}$ and $w[ \lpquote y!(z) \rpquote ]$.

\begin{eqnarray}
	\lift{w}{y!(z)}\widehat{\id{\{}u / z \id{\}}}
		& = &
		\lift{w}{y!(u)} \nonumber\\
	w[ \lpquote y!(z) \rpquote ] \widehat{ \id{\{}u / z \id{\}} }
		& = &
		w[ \lpquote y!(z) \rpquote ] \nonumber
\end{eqnarray}

Because the body of the process between quotes is impervious to
substitution, we get radically different answers. In fact, by
examining the first process in an input context,
e.g. $x?(z).\lift{w}{y!(z)}$, we see that the process under the lift
operator may be shaped by prefixed inputs binding a name inside it. In
this sense, the lift operator will be seen as a way to dynamically
construct processes before reifying them as names.

Finally equipped with these standard features we can present the
dynamics of the calculus.

\subsubsection{Operational semantics} 

Finally, we introduce the computational dynamics. What marks these
algebras as distinct from other more traditionally studied algebraic
structures, e.g. vector spaces or polynomial rings, is the manner in
which dynamics is captured. In traditional structures, dynamics is typically
expressed through morphisms between such structures, as in linear maps
between vector spaces or morphisms between rings. In algebras
associated with the semantics of computation, the dynamics is
expressed as part of the algebraic structure itself, through a
reduction reduction relation typically denoted by $\red$. Below, we
give a recursive presentation of this relation for the calculus used
in the encoding.

$\red \subseteq \pi \times \pi$
$\red : \pi \to \mathcal{P}(\pi)$

\begin{mathpar}
  \inferrule* [lab=Comm] { \textsf{match}( x_{src}, x_{trgt} ) } { x_{trgt}?(y)P \; | \; x_{src}!\langle {Q} \rangle \red P\{\quotep{Q}/y}\} }
  \and \\
  \inferrule* [lab=Par] {{P} \red {P}'} {{{P} | {Q}} \red {{P}' | {Q}}}
  \and
  \inferrule* [lab=Equiv]{{{P} \scong {P}'} \andalso {{P}' \red {Q}'} \andalso {{Q}' \scong {Q}}}{{P} \red {Q}}
\end{mathpar}

\begin{eqnarray*}
  match_{\equiv} (\quotep{P},\quotep{Q}) & := & P \equiv Q \\
  match_{\dagger}(\quotep{P},\quotep{Q}) & := & \forall R. P|Q \red^{*} R => R \red^{*} 0 \\
  match_{K}(\quotep{P},\quotep{Q}) & := & K \mbox{ for some context } K
\end{eqnarray*}

$u?(x)P | u!\langle Q \rangle \red P\{\quotep{Q}/x\}$

%We write $\wred$ for $\red^*$, and $P\red$ if $\exists Q $ such that $ P \red Q$.
We write $P\red$ if $\exists Q $ such that $ P \red Q$ and $P\not\red$, otherwise.

\section{Replication}

As mentioned before, it is known that replication (and hence
recursion) can be implemented in a higher-order process algebra
\cite{SangiorgiWalker}. As our first example of calculation with the
machinery thus far presented we give the construction explicitly in
the {\rhoc}.

\begin{eqnarray}
	D_{x} & := & \prefix{x}{y}{(\binpar{\outputp{x}{y}}{@{y}})} \nonumber\\
	\bangp_{x}{P} & := & \binpar{{x}!\langle{\binpar{D_{x}}{P}}\rangle}{D_{x}} \nonumber
\end{eqnarray}

\begin{eqnarray}
	\bangp_{x}{P} & & \nonumber\\
	=
	& {x}!\langle{(\prefix{x}{y}{(\outputp{x}{y} | @{y})) | P}}\rangle 
	      | \prefix{x}{y}{(\outputp{x}{y} | @{y})} & \nonumber\\
	\red
	& (\outputp{x}{y} | @{y})\substn{\quotep{(\prefix{x}{y}{(@{y} | \outputp{x}{y})) | P}}}{y} & \nonumber\\
	=
	& \outputp{x}{\quotep{(\prefix{x}{y}{(\outputp{x}{y} | @{y})) | P}}}
	  | {(\prefix{x}{y}{(\outputp{x}{y} | @{y})) | P}} & \nonumber\\
	\red
	& \ldots & \nonumber\\
	\red^*
	& P | P | \ldots & \nonumber
\end{eqnarray}

Of course, this encoding, as an implementation, runs away, unfolding
$\bangp{P}$ eagerly. A lazier and more implementable replication
operator, restricted to input-guarded processes, may be obtained as follows.

\begin{eqnarray}
\bangp{\prefix{u}{v}{P}} 
	:= 
	\binpar{\lift{x}{\prefix{u}{v}{(\binpar{D(x)}{P})}}}{D(x)} \nonumber
\end{eqnarray}

\begin{remark}
  Note that the lazier definition still does not deal with summation
  or mixed summation (i.e. sums over input and output). The reader is
  invited to construct definitions of replication that deal with these
  features. 

  Further, the definitions are parameterized in a name, $x$. Can you,
  gentle reader, make a definition that eliminates this parameter and
  guarantees no accidental interaction between the replication
  machinery and the process being replicated -- i.e. no accidental
  sharing of names used by the process to get its work done and the
  name(s) used by the replication to effect copying. This latter
  revision of the definition of replication is crucial to obtaining
  the expected identity $!!P \sim !P$.
\end{remark}

\begin{remark}\label{rem:paradoxical_combinator}
  The reader familiar with the lambda calculus will have noticed the
  similarity between $D$ and the paradoxical combinator.

  [Ed. note: the existence of this seems to suggest we have to be more
  restrictive on the set of processes and names we admit if we are to
  support no-cloning.]
\end{remark}

\subsubsection{Bisimulation}

The computational dynamics gives rise to another kind of equivalence,
the equivalence of computational behavior. As previously mentioned
this is typically captured \emph{via} some form of bisimulation.

% The notion we use in this paper is weak barbed bisimulation
% \cite{milner91polyadicpi}.

The notion we use in this paper is derived from weak barbed
bisimulation \cite{milner91polyadicpi}. 

\begin{definition}
An \emph{observation relation}, $\downarrow_{\mathcal N}$, over a set
of names, $\mathcal N$, is the smallest relation satisfying the rules
below.

\infrule[Out-barb]{y \in {\mathcal N}, \; x \nameeq y}
		  {\outputp{x}{v} \downarrow_{\mathcal N} x}
\infrule[Par-barb]{\mbox{$P\downarrow_{\mathcal N} x$ or $Q\downarrow_{\mathcal N} x$}}
		  {\binpar{P}{Q} \downarrow_{\mathcal N} x}

We write $P \Downarrow_{\mathcal N} x$ if there is $Q$ such that 
$P \wred Q$ and $Q \downarrow_{\mathcal N} x$.
\end{definition}

\begin{definition}
%\label{def.bbisim}
An  ${\mathcal N}$-\emph{barbed bisimulation} over a set of names, ${\mathcal N}$, is a symmetric binary relation 
${\mathcal S}_{\mathcal N}$ between agents such that $P\rel{S}_{\mathcal N}Q$ implies:
\begin{enumerate}
\item If $P \red P'$ then $Q \wred Q'$ and $P'\rel{S}_{\mathcal N} Q'$.
\item If $P\downarrow_{\mathcal N} x$, then $Q\Downarrow_{\mathcal N} x$.
\end{enumerate}
$P$ is ${\mathcal N}$-barbed bisimilar to $Q$, written
$P \wbbisim_{\mathcal N} Q$, if $P \rel{S}_{\mathcal N} Q$ for some ${\mathcal N}$-barbed bisimulation ${\mathcal S}_{\mathcal N}$.
\end{definition}

$\mathcal{R} \subseteq \pi \times \pi$

$P \mathcal{R} Q => \forall P'. P \red P' \Rightarrow \exists Q'. Q \red Q', P' \mathcal{R} Q'$

$P \vdash x \Rightarrow Q \vdash x$

\begin{mathpar}
  \inferrule*[lab=Out-barb]{x \nameeq y}{{y}!\langle{Q}\rangle \vdash x}
  \and
  \inferrule*[lab=Par-barb]{\mbox{$P\vdash x$ or $Q\vdash x$}}{\binpar{P}{Q} \vdash x}
\end{mathpar}

\subsubsection{Contexts}

One of the principle advantages of computational calculi like the
$\pi$-calculus is a well-defined notion of context,
contextual-equivalence and a correlation between
contextual-equivalence and notions of bisimulation. The notion of
context allows the decomposition of a process into (sub-)process and
its syntactic environment, its context. Thus, a context may be
thought of as a process with a ``hole'' (written $\Box$) in it. The
application of a context $M$ to a process $P$, written $M[P]$, is
tantamount to filling the hole in $M$ with $P$. In this paper we do
not need the full weight of this theory, but do make use of the notion
of context in the proof the main theorem. 

\begin{mathpar}
  \inferrule* [lab=summation] {} {{M_{M},M_{N}} \bc \Box \;|\; x.M_{A} \;|\; M_{M}+M_{N}}
  \and
  \inferrule* [lab=agent] {} {{M_{A}} \bc (\vec{x})M_{P} \;| \; \clift{P_0,\ldots,M_{P},\ldots,P_N}}
  \and \\
  \inferrule* [lab=process] {} {{M_{P}} \bc M_{N} \;| \;P|M_{P} }
\end{mathpar} 

\begin{mathpar}
  \inferrule* [lab=sychronization] {} {M_{N} \bc \Box \;|\; x?M_{F} \;|\; x!M_{C}}
  \and
  \inferrule* [lab=abstraction] {} {{M_{F}} \bc (x)M_{P} }
  \and
  \inferrule* [lab=concretion] {} {{M_{C}} \bc \langle M_{P} \rangle }
  \and \\
  \inferrule* [lab=process] {} {{M_{P}} \bc M_{N} \;| \;P|M_{P} }
\end{mathpar}

\begin{definition}[contextual application] Given a context $M$, and
  process $P$, we define the \emph{contextual application}, $M[P] :=
  M\{P/\Box\}$. That is, the contextual application of M to P is the
  substitution of $P$ for $\Box$ in $M$.
\end{definition}

$\meaningof{-} : L \to \mathcal{P}(\pi)$

\begin{mathpar}
  \inferrule* [lab=collection] {} {\meaningof{true} = \pi, \and \meaningof{~E} = \pi \setminus \meaningof{E}, \and \meaningof{E_{1} \& E_{2}} = \meaningof{E_{1}} \cap \meaningof{E_{2}}}
\end{mathpar}

\begin{mathpar}
  \inferrule* [lab=structure] {} {\meaningof{0} = \{ P \in \pi | P \equiv 0 \}, \and \\ \meaningof{E_1 | E_2} = \{ P \in \pi | P \equiv P_{1} | P_{2}, P_{1} \in \meaningof{E_{1}}, P_{2} \in \meaningof{E_2}\} }
\end{mathpar}

\begin{mathpar}
 \inferrule* [lab=behavior] {} {\meaningof{\langle a?b \rangle E} = \{ P \in \pi | P \equiv Q | u?(y)P', \\ \and \\\\ \and \\ \;\;\; u \in \meaningof{a}, \forall z.P'\{z/y\} \in \meaningof{E\{z/b\}}\}, \and \\ \meaningof{a!E} = \{ P \in \pi | P \equiv Q | x!\langle P' \rangle, x \in \meaningof{a} P' \in \meaningof{E}\} }
\end{mathpar}

\begin{mathpar}
 \inferrule* [lab=nominal] {} {\meaningof{\quotep{E}} = \{ \quotep{P} \in \quotep{\pi} | P \in \meaningof{E} \}, \and \meaningof{\quotep{P}} = \{ \quotep{Q} \in \quotep{\pi} | P \equiv Q \} \and \\ \meaningof{@\quotep{E}} = \{ P \in \pi | P \equiv @x, x \in \meaningof{E} \}}
\end{mathpar}

\begin{eqnarray*}
  \\
  \meaningof{-} : TS \to ST
\end{eqnarray*}

\begin{eqnarray*}
  \\
  L : TS \to ST
\end{eqnarray*}

\begin{eqnarray*}
  \\
  P \models E \iff P \in \meaningof{E}
\end{eqnarray*}

\begin{eqnarray*}
  P \approx_{L} Q \iff \forall E \in L. P \models E \iff Q \models E
\end{eqnarray*}

\begin{eqnarray*}
  P \approx_{K} Q
\end{eqnarray*}

\begin{eqnarray*}
  P \approx Q
\end{eqnarray*}

$\approx_{K} = \approx = \approx_{L}$

\subsubsection{Contextual duality}

Note that contexts extend the quotation operation to a family of
operations from processes to names. Given a context, $M$, we can
define a \emph{nominal context}, $\quotep{M}$ by $\quotep{M}[P] :=
\quotep{M[P]}$. To foreshadow what is to come we observe that these
operations enjoy a duality with processes very much like the duality
between vectors and maps from vectors to scalars.

Further, because the calculus is essentially higher-order, we have a
correspondence between contexts and processes. More specifically,
given a name $x$ and a context $M$ we can construct $M^{*}_{x}$ such
that 

\begin{mathpar}
  M^{*}_{x} | \lift{x}{P} \red M[P]
\end{mathpar}

namely,

\begin{mathpar}
  M^{*}_{x} := x?(u).M[\dropn{u}]
\end{mathpar}

The dependence of $M^{*}_{x}$ on a name makes it an abstraction, 

\begin{mathpar}
  M^{*} := (x)x?(u).M[\dropn{u}]
\end{mathpar}

\subsection{Additional notation}

It will sometimes be convenient to denote the process a name
quotes. We already have the notation $x = \quotep{P}$, but it will be
convenient to introduce an alternate notation, $\procn{x}$, when we
want to emphasize the connection to the use of the name. Note that, by
virtue of name equivalence, $\quotep{\procn{x}} \nameeq x$; so, the
notation is consistent with previous definitions.

Further, because names have structure it is possible to effect
substitutions on the basis of that structure. This means we need to
upgrade our notation for substitutions, which we accomplish by
adapting comprehension notation. Thus,

\begin{mathpar}
  P\{ y / x : x \in S \}
\end{mathpar}

is interpreted to mean the process derived from P by replacing (in a
capture-avoiding manner) each occurrence of $x$ in $S$ by $y$. For example,

\begin{mathpar}
  P\{ \quotep{\procn{x}|\procn{x}} / x : x \in \freenames{P} \}
\end{mathpar}

will replace each (occurrence) of a free name $x$ in $P$ by
$\quotep{\procn{x}|\procn{x}}$.

Also, we will avail ourselves of the notation $x^{L}$ and $x^{R}$ to
denote injections of a name into disjoint copies of the name
space. There are numerous ways to accomplish this. One example can be
found in \cite{MeredithR05}. This notation overloads to vectors of
names: $\vec{x}^{\pi} := (x_{i}^{\pi} \; : \; 0 \leq i < |\vec{x}| )$ where $\pi \in \{L,R\}$.

We also use $P^{\Box} := P|\Box$.

In \cite{MeredithR05} an interpretation of the new operator is
given. It turns out that there are several possible interpretations
all enjoying the requisite algebraic properties of the operator (see
\cite{milner91polyadicpi}). We will therefore make liberal use of
$(\nu\; \vec{x})P$.

% subsection the_syntax_and_semantics_of_the_notation_system (end)   

\input{qm2pi.qmops} 

\input{qm2pi.sterngerlach} 

\input{qm2pi.metric} 

% section concurrent_process_calculi (end)

%\input{qm2pi.proofsketch}

% section proof sketch (end)

%\input{qm2pi.slviaknots} 

% section spatial logic via knots (end)

\input{qm2pi.conclusion}

% section conclusion (end)

%\input{qm2pi.dtcodes} 

% section wiring algorithm (end)

\input{qm2pi.ack} 

% section acknowledgments (end)

\newpage


\bibliographystyle{plain}   
\bibliography{../../biblios/main.bib}

\input{qm2pi.rhodetails}

\end{document}

 

% section notation (end)

\input{qm2pi.process.calculi} 

% section concurrent_process_calculi_and_spatial_logics_ (end)
    
%\documentclass[12pt]{llncs}
%\documentclass{jktr}

\usepackage[pdftex]{hyperref}                   
\usepackage {listings}
\usepackage {mathpartir}
\usepackage{bcprules}
%\usepackage{listings}
                       
\usepackage{graphicx} 
%\usepackage[margins=2.5cm,nohead,nofoot]{geometry}
%\usepackage{geometry}
\usepackage{amsfonts}
\usepackage{amstext}
\usepackage{latexsym}
\usepackage{amssymb}
\usepackage{color}


%\include{myPreamble}
\include{qm2pi.local} 

%\ifpdf
%\usepackage[pdftex]{graphicx}
%\else
%\usepackage{graphicx}
%\fi

 % \ifpdf
%  \usepackage{pdfsync}
%  \if


%\title{Brief Article}
%\author{David F. Snyder}
%\author{L.G. Meredith}

%\address{Dept. of Math., Texas State University--San Marcos, San Marcos, TX 78666}
       
\pagestyle{empty}


\begin{document}

\lstset{language=[Objective]Caml,frame=shadowbox}

\input{qm2pi.front}

% section front matter (end)

\input{qm2pi.intro} 
 
% section introduction (end)

% \input{qm2pi.knotations} 

% section notation (end)

\input{qm2pi.process.calculi} 

% section concurrent_process_calculi_and_spatial_logics_ (end)
    
%\input{qm2pi.knots2pi} 

%\input{qm2pi.trefoil} 

%\input{qm2pi.mainthm} 

% subsection basic_interpretation (end)

%\input{qm2pi.rho.presentation} 
\subsection{The syntax and semantics of the notation system}\label{sub:the_syntax_and_semantics_of_the_notation_system} % (fold)

We now summarize a technical presentation of the calculus that
embodies our theory of dynamics. The typical presentation of such a
calculus follows the style of giving generators and relations on
them. The grammar, below, describing term constructors, freely
generates the set of processes, $\Proc$. This set is then quotiented
by a relation known as structural congruence and it is over this set
that the notion of dynamics is expressed. This presentation is
essentially that of \cite{MeredithR05} with the addition of
polyadicity and summation. For readability we have relegated some of
the technical subtleties to an appendix.

\subsubsection{Process grammar}\label{subsub:process_grammar}

\begin{mathpar}
  \inferrule* [lab=synchronization] {} {{M} \bc \pzero \;|\; x?F \;|\; x!C }
  \and
  \inferrule* [lab=abstraction] {} {{F} \bc (x)P}
  \and
  \inferrule* [lab=concretion] {} {{C} \bc \langle Q \rangle}
  \and
  \inferrule* [lab=process] {} {{P,Q} \bc M \;| \;P|Q \;|\; @{x}}
  \and
  \inferrule* [lab=name] {} {{x} \bc \quotep{P}}
\end{mathpar} 

Note that $\vec{x}$ (resp. $\vec{P}$) denotes a vector of names
(resp. processes) of length $|\vec{x}|$ (resp. $|\vec{P}|$). We adopt
the following useful abbreviations.

\begin{mathpar}
   x?(\vec{y}).P := x.(\vec{y})P \and  x\clift{\vec{P}} := x.\clift{\vec{P}}
   \and x!(y) := \lift{x}{\dropn{y}}
   \and \Pi_{i=0}^{n-1}P_i := P_0 | \ldots | P_{n-1}
\end{mathpar}

\subsubsection{Structural congruence}

\paragraph{Free and bound names and alpha-equivalence.} At the
core of structural equivalence is alpha-equivalence which identifies
process that are the same up to a change of variable. Formally, we
recognize the distinction between free and bound names. The free names
of a process, $\freenames{P}$, may be calculated recursively as
follows:

\begin{mathpar}
\freenames{\pzero} := \emptyset
  \and \\
  \freenames{x?(y).P} := \{ x \} \cup (\freenames{P} \setminus \{ y \})
  \and 
  \freenames{x!\langle P \rangle} := \{ x \} \cup \{ P \} 
  \and \\
  \freenames{P|Q} := \freenames{P} \cup \freenames{Q}
  \and \\
  \freenames{@{x}} := \{ x \}
\end{mathpar}

$\pi$
$\quotep{\pi}$

$\freenames{-} : \pi \to \mathcal{P}(\quotep{\pi})$

\begin{eqnarray*}
  \freenames{\pzero} & := & \emptyset \\
  \freenames{x?(y).P} & := & \{ x \} \cup (\freenames{P} \setminus \{ y \}) \\
  \freenames{x!\langle P \rangle} & := & \{ x \} \cup \{ P \} \\
  \freenames{P|Q} & := & \freenames{P} \cup \freenames{Q} \\
  \freenames{\dropn{x}} & := & \{ x \}
\end{eqnarray*}

The bound names of a process, $\boundnames{P}$, are those names occurring in $P$
that are not free. For example, in $x?(y).0$, the name $x$ is free, while $y$ is bound.

\begin{mathpar}
  \inferrule* [lab=monoidal-laws] {} { P|Q \equiv Q|P \and P|0 \equiv P \and P|(Q|R) \equiv (P|Q)|R }
\end{mathpar}

\begin{mathpar}
  \inferrule* [lab=alpha-equivalence] {} { (x)P \equiv (y)P\{y/x\} \and y \not\in \freenames{P} }
\end{mathpar}

\begin{definition}
Then two processes, $P,Q$, are alpha-equivalent if $P = Q\{\vec{y}/\vec{x}\}$ for
some $\vec{x} \in \boundnames{Q},\vec{y} \in \boundnames{P}$, where $Q\{\vec{y}/\vec{x}\}$
denotes the capture-avoiding substitution of $\vec{y}$ for $\vec{x}$ in $Q$.
\end{definition}

\begin{definition}
  The {\em structural congruence} \cite{SangiorgiWalker} , $\equiv$,
  between processes is the least congruence containing
  alpha-equivalence, satisfying the abelian monoid laws
  (associativity, commutativity and $\pzero$ as identity) for parallel
  composition $|$ and for summation $+$.
\end{definition}

\subsection{Name equivalence}

We take name equivalence, written $\nameeq$, to be the smallest
equivalence relation generated by the following rules.

\begin{mathpar}
\inferrule*[lab=Quote-drop]
{ }
{ \quotep{@{x}} \nameeq x }

\inferrule*[lab=Struct-equiv]
{ P \scong Q }
{ \quotep{P} \nameeq \quotep{Q} }
\end{mathpar}

The astute reader will have noticed that the mutual recursion of names
and processes imposes a mutual recursion on alpha-equivalence and
structural equivalence via name-equivalence. Fortunately, all of this
works out pleasantly and we may calculate in the natural way, free of
concern. The reader interested in the details is referred to the
appendix \ref{appendix:rho_details}.

\subsection{Substitution}

We use $\Proc$ for the set of processes, $\QProc$ for the set of
names, and $\id{\{}\vec{y} / \vec{x} \id{\}}$ to denote partial maps,
$s : \QProc \rightarrow \QProc$. A map, $s$ lifts, uniquely, to a map
on process terms, $\widehat{s} : \Proc \rightarrow \Proc$ by the
following equations.

\begin{mathpar}
  (0) \psubstp{Q}{P} := 0 \\
  (R \juxtap S) \psubstp{Q}{P}
  :=    
  (R)\psubstp{Q}{P} \juxtap (S) \psubstp{Q}{P} \\
  (x?(y).R) \psubstp{Q}{P}    
  :=    
  (x)\substp{Q}{P} (z)\concat( (R \psubstn{z}{y}) \psubstp{Q}{P} ) \\
  (\lift{x}{R}) \psubstp{Q}{P}  
  :=
  \lift{(x)\substp{Q}{P}}{ R \psubstp{Q}{P} } \\
%   (\dropn{x})  \psubstp{Q}{P}       
%   := 
%   \left\{ 
%     \begin{array}{ccc} 
%       \dropn{\quotep{Q}} & & x \nameeq \quotep{P} \\
%       \dropn{x} & & otherwise \\
%     \end{array}
%   \right. 
  (\dropn{x})  \psubstp{Q}{P}       
  := 
  \left\{ 
    \begin{array}{ccc} 
      Q & & x \nameeq \quotep{P} \\
      \dropn{x} & & otherwise \\
    \end{array}
  \right.
\end{mathpar}
 

where

\begin{eqnarray}
  (x)\id{\{} \lpquote Q \rpquote / \lpquote P \rpquote \id{\}}            = 
  \left\{ 
    \begin{array}{ccc}
      \lpquote Q \rpquote & & x \nameeq \lpquote P \rpquote \\
      x & & otherwise \\
    \end{array}
  \right. \nonumber
\end{eqnarray}

and $z$ is chosen distinct from $\quotep{P}$, $\quotep{Q}$, the free
names in $Q$, and all the names in $R$. Our $\alpha$-equivalence will
be built in the standard way from this substitution.

\begin{remark}\label{rem:no_self_referential_names}
  One consequence of these definitions is that $\forall P. \quotep{P}
  \not\in \freenames{P}$.
\end{remark}

\subsection{ Dynamic quote: an example }

Anticipating something of what's to come, consider applying the
substitution, $\widehat{\id{\{}u / z \id{\}}}$, to the following pair
of processes, $\lift{w}{y!(z)}$ and $w[ \lpquote y!(z) \rpquote ]$.

\begin{eqnarray}
	\lift{w}{y!(z)}\widehat{\id{\{}u / z \id{\}}}
		& = &
		\lift{w}{y!(u)} \nonumber\\
	w[ \lpquote y!(z) \rpquote ] \widehat{ \id{\{}u / z \id{\}} }
		& = &
		w[ \lpquote y!(z) \rpquote ] \nonumber
\end{eqnarray}

Because the body of the process between quotes is impervious to
substitution, we get radically different answers. In fact, by
examining the first process in an input context,
e.g. $x?(z).\lift{w}{y!(z)}$, we see that the process under the lift
operator may be shaped by prefixed inputs binding a name inside it. In
this sense, the lift operator will be seen as a way to dynamically
construct processes before reifying them as names.

Finally equipped with these standard features we can present the
dynamics of the calculus.

\subsubsection{Operational semantics} 

Finally, we introduce the computational dynamics. What marks these
algebras as distinct from other more traditionally studied algebraic
structures, e.g. vector spaces or polynomial rings, is the manner in
which dynamics is captured. In traditional structures, dynamics is typically
expressed through morphisms between such structures, as in linear maps
between vector spaces or morphisms between rings. In algebras
associated with the semantics of computation, the dynamics is
expressed as part of the algebraic structure itself, through a
reduction reduction relation typically denoted by $\red$. Below, we
give a recursive presentation of this relation for the calculus used
in the encoding.

$\red \subseteq \pi \times \pi$
$\red : \pi \to \mathcal{P}(\pi)$

\begin{mathpar}
  \inferrule* [lab=Comm] { \textsf{match}( x_{src}, x_{trgt} ) } { x_{trgt}?(y)P \; | \; x_{src}!\langle {Q} \rangle \red P\{\quotep{Q}/y}\} }
  \and \\
  \inferrule* [lab=Par] {{P} \red {P}'} {{{P} | {Q}} \red {{P}' | {Q}}}
  \and
  \inferrule* [lab=Equiv]{{{P} \scong {P}'} \andalso {{P}' \red {Q}'} \andalso {{Q}' \scong {Q}}}{{P} \red {Q}}
\end{mathpar}

\begin{eqnarray*}
  match_{\equiv} (\quotep{P},\quotep{Q}) & := & P \equiv Q \\
  match_{\dagger}(\quotep{P},\quotep{Q}) & := & \forall R. P|Q \red^{*} R => R \red^{*} 0 \\
  match_{K}(\quotep{P},\quotep{Q}) & := & K \mbox{ for some context } K
\end{eqnarray*}

$u?(x)P | u!\langle Q \rangle \red P\{\quotep{Q}/x\}$

%We write $\wred$ for $\red^*$, and $P\red$ if $\exists Q $ such that $ P \red Q$.
We write $P\red$ if $\exists Q $ such that $ P \red Q$ and $P\not\red$, otherwise.

\section{Replication}

As mentioned before, it is known that replication (and hence
recursion) can be implemented in a higher-order process algebra
\cite{SangiorgiWalker}. As our first example of calculation with the
machinery thus far presented we give the construction explicitly in
the {\rhoc}.

\begin{eqnarray}
	D_{x} & := & \prefix{x}{y}{(\binpar{\outputp{x}{y}}{@{y}})} \nonumber\\
	\bangp_{x}{P} & := & \binpar{{x}!\langle{\binpar{D_{x}}{P}}\rangle}{D_{x}} \nonumber
\end{eqnarray}

\begin{eqnarray}
	\bangp_{x}{P} & & \nonumber\\
	=
	& {x}!\langle{(\prefix{x}{y}{(\outputp{x}{y} | @{y})) | P}}\rangle 
	      | \prefix{x}{y}{(\outputp{x}{y} | @{y})} & \nonumber\\
	\red
	& (\outputp{x}{y} | @{y})\substn{\quotep{(\prefix{x}{y}{(@{y} | \outputp{x}{y})) | P}}}{y} & \nonumber\\
	=
	& \outputp{x}{\quotep{(\prefix{x}{y}{(\outputp{x}{y} | @{y})) | P}}}
	  | {(\prefix{x}{y}{(\outputp{x}{y} | @{y})) | P}} & \nonumber\\
	\red
	& \ldots & \nonumber\\
	\red^*
	& P | P | \ldots & \nonumber
\end{eqnarray}

Of course, this encoding, as an implementation, runs away, unfolding
$\bangp{P}$ eagerly. A lazier and more implementable replication
operator, restricted to input-guarded processes, may be obtained as follows.

\begin{eqnarray}
\bangp{\prefix{u}{v}{P}} 
	:= 
	\binpar{\lift{x}{\prefix{u}{v}{(\binpar{D(x)}{P})}}}{D(x)} \nonumber
\end{eqnarray}

\begin{remark}
  Note that the lazier definition still does not deal with summation
  or mixed summation (i.e. sums over input and output). The reader is
  invited to construct definitions of replication that deal with these
  features. 

  Further, the definitions are parameterized in a name, $x$. Can you,
  gentle reader, make a definition that eliminates this parameter and
  guarantees no accidental interaction between the replication
  machinery and the process being replicated -- i.e. no accidental
  sharing of names used by the process to get its work done and the
  name(s) used by the replication to effect copying. This latter
  revision of the definition of replication is crucial to obtaining
  the expected identity $!!P \sim !P$.
\end{remark}

\begin{remark}\label{rem:paradoxical_combinator}
  The reader familiar with the lambda calculus will have noticed the
  similarity between $D$ and the paradoxical combinator.

  [Ed. note: the existence of this seems to suggest we have to be more
  restrictive on the set of processes and names we admit if we are to
  support no-cloning.]
\end{remark}

\subsubsection{Bisimulation}

The computational dynamics gives rise to another kind of equivalence,
the equivalence of computational behavior. As previously mentioned
this is typically captured \emph{via} some form of bisimulation.

% The notion we use in this paper is weak barbed bisimulation
% \cite{milner91polyadicpi}.

The notion we use in this paper is derived from weak barbed
bisimulation \cite{milner91polyadicpi}. 

\begin{definition}
An \emph{observation relation}, $\downarrow_{\mathcal N}$, over a set
of names, $\mathcal N$, is the smallest relation satisfying the rules
below.

\infrule[Out-barb]{y \in {\mathcal N}, \; x \nameeq y}
		  {\outputp{x}{v} \downarrow_{\mathcal N} x}
\infrule[Par-barb]{\mbox{$P\downarrow_{\mathcal N} x$ or $Q\downarrow_{\mathcal N} x$}}
		  {\binpar{P}{Q} \downarrow_{\mathcal N} x}

We write $P \Downarrow_{\mathcal N} x$ if there is $Q$ such that 
$P \wred Q$ and $Q \downarrow_{\mathcal N} x$.
\end{definition}

\begin{definition}
%\label{def.bbisim}
An  ${\mathcal N}$-\emph{barbed bisimulation} over a set of names, ${\mathcal N}$, is a symmetric binary relation 
${\mathcal S}_{\mathcal N}$ between agents such that $P\rel{S}_{\mathcal N}Q$ implies:
\begin{enumerate}
\item If $P \red P'$ then $Q \wred Q'$ and $P'\rel{S}_{\mathcal N} Q'$.
\item If $P\downarrow_{\mathcal N} x$, then $Q\Downarrow_{\mathcal N} x$.
\end{enumerate}
$P$ is ${\mathcal N}$-barbed bisimilar to $Q$, written
$P \wbbisim_{\mathcal N} Q$, if $P \rel{S}_{\mathcal N} Q$ for some ${\mathcal N}$-barbed bisimulation ${\mathcal S}_{\mathcal N}$.
\end{definition}

$\mathcal{R} \subseteq \pi \times \pi$

$P \mathcal{R} Q => \forall P'. P \red P' \Rightarrow \exists Q'. Q \red Q', P' \mathcal{R} Q'$

$P \vdash x \Rightarrow Q \vdash x$

\begin{mathpar}
  \inferrule*[lab=Out-barb]{x \nameeq y}{{y}!\langle{Q}\rangle \vdash x}
  \and
  \inferrule*[lab=Par-barb]{\mbox{$P\vdash x$ or $Q\vdash x$}}{\binpar{P}{Q} \vdash x}
\end{mathpar}

\subsubsection{Contexts}

One of the principle advantages of computational calculi like the
$\pi$-calculus is a well-defined notion of context,
contextual-equivalence and a correlation between
contextual-equivalence and notions of bisimulation. The notion of
context allows the decomposition of a process into (sub-)process and
its syntactic environment, its context. Thus, a context may be
thought of as a process with a ``hole'' (written $\Box$) in it. The
application of a context $M$ to a process $P$, written $M[P]$, is
tantamount to filling the hole in $M$ with $P$. In this paper we do
not need the full weight of this theory, but do make use of the notion
of context in the proof the main theorem. 

\begin{mathpar}
  \inferrule* [lab=summation] {} {{M_{M},M_{N}} \bc \Box \;|\; x.M_{A} \;|\; M_{M}+M_{N}}
  \and
  \inferrule* [lab=agent] {} {{M_{A}} \bc (\vec{x})M_{P} \;| \; \clift{P_0,\ldots,M_{P},\ldots,P_N}}
  \and \\
  \inferrule* [lab=process] {} {{M_{P}} \bc M_{N} \;| \;P|M_{P} }
\end{mathpar} 

\begin{mathpar}
  \inferrule* [lab=sychronization] {} {M_{N} \bc \Box \;|\; x?M_{F} \;|\; x!M_{C}}
  \and
  \inferrule* [lab=abstraction] {} {{M_{F}} \bc (x)M_{P} }
  \and
  \inferrule* [lab=concretion] {} {{M_{C}} \bc \langle M_{P} \rangle }
  \and \\
  \inferrule* [lab=process] {} {{M_{P}} \bc M_{N} \;| \;P|M_{P} }
\end{mathpar}

\begin{definition}[contextual application] Given a context $M$, and
  process $P$, we define the \emph{contextual application}, $M[P] :=
  M\{P/\Box\}$. That is, the contextual application of M to P is the
  substitution of $P$ for $\Box$ in $M$.
\end{definition}

$\meaningof{-} : L \to \mathcal{P}(\pi)$

\begin{mathpar}
  \inferrule* [lab=collection] {} {\meaningof{true} = \pi, \and \meaningof{~E} = \pi \setminus \meaningof{E}, \and \meaningof{E_{1} \& E_{2}} = \meaningof{E_{1}} \cap \meaningof{E_{2}}}
\end{mathpar}

\begin{mathpar}
  \inferrule* [lab=structure] {} {\meaningof{0} = \{ P \in \pi | P \equiv 0 \}, \and \\ \meaningof{E_1 | E_2} = \{ P \in \pi | P \equiv P_{1} | P_{2}, P_{1} \in \meaningof{E_{1}}, P_{2} \in \meaningof{E_2}\} }
\end{mathpar}

\begin{mathpar}
 \inferrule* [lab=behavior] {} {\meaningof{\langle a?b \rangle E} = \{ P \in \pi | P \equiv Q | u?(y)P', \\ \and \\\\ \and \\ \;\;\; u \in \meaningof{a}, \forall z.P'\{z/y\} \in \meaningof{E\{z/b\}}\}, \and \\ \meaningof{a!E} = \{ P \in \pi | P \equiv Q | x!\langle P' \rangle, x \in \meaningof{a} P' \in \meaningof{E}\} }
\end{mathpar}

\begin{mathpar}
 \inferrule* [lab=nominal] {} {\meaningof{\quotep{E}} = \{ \quotep{P} \in \quotep{\pi} | P \in \meaningof{E} \}, \and \meaningof{\quotep{P}} = \{ \quotep{Q} \in \quotep{\pi} | P \equiv Q \} \and \\ \meaningof{@\quotep{E}} = \{ P \in \pi | P \equiv @x, x \in \meaningof{E} \}}
\end{mathpar}

\begin{eqnarray*}
  \\
  \meaningof{-} : TS \to ST
\end{eqnarray*}

\begin{eqnarray*}
  \\
  L : TS \to ST
\end{eqnarray*}

\begin{eqnarray*}
  \\
  P \models E \iff P \in \meaningof{E}
\end{eqnarray*}

\begin{eqnarray*}
  P \approx_{L} Q \iff \forall E \in L. P \models E \iff Q \models E
\end{eqnarray*}

\begin{eqnarray*}
  P \approx_{K} Q
\end{eqnarray*}

\begin{eqnarray*}
  P \approx Q
\end{eqnarray*}

$\approx_{K} = \approx = \approx_{L}$

\subsubsection{Contextual duality}

Note that contexts extend the quotation operation to a family of
operations from processes to names. Given a context, $M$, we can
define a \emph{nominal context}, $\quotep{M}$ by $\quotep{M}[P] :=
\quotep{M[P]}$. To foreshadow what is to come we observe that these
operations enjoy a duality with processes very much like the duality
between vectors and maps from vectors to scalars.

Further, because the calculus is essentially higher-order, we have a
correspondence between contexts and processes. More specifically,
given a name $x$ and a context $M$ we can construct $M^{*}_{x}$ such
that 

\begin{mathpar}
  M^{*}_{x} | \lift{x}{P} \red M[P]
\end{mathpar}

namely,

\begin{mathpar}
  M^{*}_{x} := x?(u).M[\dropn{u}]
\end{mathpar}

The dependence of $M^{*}_{x}$ on a name makes it an abstraction, 

\begin{mathpar}
  M^{*} := (x)x?(u).M[\dropn{u}]
\end{mathpar}

\subsection{Additional notation}

It will sometimes be convenient to denote the process a name
quotes. We already have the notation $x = \quotep{P}$, but it will be
convenient to introduce an alternate notation, $\procn{x}$, when we
want to emphasize the connection to the use of the name. Note that, by
virtue of name equivalence, $\quotep{\procn{x}} \nameeq x$; so, the
notation is consistent with previous definitions.

Further, because names have structure it is possible to effect
substitutions on the basis of that structure. This means we need to
upgrade our notation for substitutions, which we accomplish by
adapting comprehension notation. Thus,

\begin{mathpar}
  P\{ y / x : x \in S \}
\end{mathpar}

is interpreted to mean the process derived from P by replacing (in a
capture-avoiding manner) each occurrence of $x$ in $S$ by $y$. For example,

\begin{mathpar}
  P\{ \quotep{\procn{x}|\procn{x}} / x : x \in \freenames{P} \}
\end{mathpar}

will replace each (occurrence) of a free name $x$ in $P$ by
$\quotep{\procn{x}|\procn{x}}$.

Also, we will avail ourselves of the notation $x^{L}$ and $x^{R}$ to
denote injections of a name into disjoint copies of the name
space. There are numerous ways to accomplish this. One example can be
found in \cite{MeredithR05}. This notation overloads to vectors of
names: $\vec{x}^{\pi} := (x_{i}^{\pi} \; : \; 0 \leq i < |\vec{x}| )$ where $\pi \in \{L,R\}$.

We also use $P^{\Box} := P|\Box$.

In \cite{MeredithR05} an interpretation of the new operator is
given. It turns out that there are several possible interpretations
all enjoying the requisite algebraic properties of the operator (see
\cite{milner91polyadicpi}). We will therefore make liberal use of
$(\nu\; \vec{x})P$.

% subsection the_syntax_and_semantics_of_the_notation_system (end)   

\input{qm2pi.qmops} 

\input{qm2pi.sterngerlach} 

\input{qm2pi.metric} 

% section concurrent_process_calculi (end)

%\input{qm2pi.proofsketch}

% section proof sketch (end)

%\input{qm2pi.slviaknots} 

% section spatial logic via knots (end)

\input{qm2pi.conclusion}

% section conclusion (end)

%\input{qm2pi.dtcodes} 

% section wiring algorithm (end)

\input{qm2pi.ack} 

% section acknowledgments (end)

\newpage


\bibliographystyle{plain}   
\bibliography{../../biblios/main.bib}

\input{qm2pi.rhodetails}

\end{document}

 

%\documentclass[12pt]{llncs}
%\documentclass{jktr}

\usepackage[pdftex]{hyperref}                   
\usepackage {listings}
\usepackage {mathpartir}
\usepackage{bcprules}
%\usepackage{listings}
                       
\usepackage{graphicx} 
%\usepackage[margins=2.5cm,nohead,nofoot]{geometry}
%\usepackage{geometry}
\usepackage{amsfonts}
\usepackage{amstext}
\usepackage{latexsym}
\usepackage{amssymb}
\usepackage{color}


%\include{myPreamble}
\include{qm2pi.local} 

%\ifpdf
%\usepackage[pdftex]{graphicx}
%\else
%\usepackage{graphicx}
%\fi

 % \ifpdf
%  \usepackage{pdfsync}
%  \if


%\title{Brief Article}
%\author{David F. Snyder}
%\author{L.G. Meredith}

%\address{Dept. of Math., Texas State University--San Marcos, San Marcos, TX 78666}
       
\pagestyle{empty}


\begin{document}

\lstset{language=[Objective]Caml,frame=shadowbox}

\input{qm2pi.front}

% section front matter (end)

\input{qm2pi.intro} 
 
% section introduction (end)

% \input{qm2pi.knotations} 

% section notation (end)

\input{qm2pi.process.calculi} 

% section concurrent_process_calculi_and_spatial_logics_ (end)
    
%\input{qm2pi.knots2pi} 

%\input{qm2pi.trefoil} 

%\input{qm2pi.mainthm} 

% subsection basic_interpretation (end)

%\input{qm2pi.rho.presentation} 
\subsection{The syntax and semantics of the notation system}\label{sub:the_syntax_and_semantics_of_the_notation_system} % (fold)

We now summarize a technical presentation of the calculus that
embodies our theory of dynamics. The typical presentation of such a
calculus follows the style of giving generators and relations on
them. The grammar, below, describing term constructors, freely
generates the set of processes, $\Proc$. This set is then quotiented
by a relation known as structural congruence and it is over this set
that the notion of dynamics is expressed. This presentation is
essentially that of \cite{MeredithR05} with the addition of
polyadicity and summation. For readability we have relegated some of
the technical subtleties to an appendix.

\subsubsection{Process grammar}\label{subsub:process_grammar}

\begin{mathpar}
  \inferrule* [lab=synchronization] {} {{M} \bc \pzero \;|\; x?F \;|\; x!C }
  \and
  \inferrule* [lab=abstraction] {} {{F} \bc (x)P}
  \and
  \inferrule* [lab=concretion] {} {{C} \bc \langle Q \rangle}
  \and
  \inferrule* [lab=process] {} {{P,Q} \bc M \;| \;P|Q \;|\; @{x}}
  \and
  \inferrule* [lab=name] {} {{x} \bc \quotep{P}}
\end{mathpar} 

Note that $\vec{x}$ (resp. $\vec{P}$) denotes a vector of names
(resp. processes) of length $|\vec{x}|$ (resp. $|\vec{P}|$). We adopt
the following useful abbreviations.

\begin{mathpar}
   x?(\vec{y}).P := x.(\vec{y})P \and  x\clift{\vec{P}} := x.\clift{\vec{P}}
   \and x!(y) := \lift{x}{\dropn{y}}
   \and \Pi_{i=0}^{n-1}P_i := P_0 | \ldots | P_{n-1}
\end{mathpar}

\subsubsection{Structural congruence}

\paragraph{Free and bound names and alpha-equivalence.} At the
core of structural equivalence is alpha-equivalence which identifies
process that are the same up to a change of variable. Formally, we
recognize the distinction between free and bound names. The free names
of a process, $\freenames{P}$, may be calculated recursively as
follows:

\begin{mathpar}
\freenames{\pzero} := \emptyset
  \and \\
  \freenames{x?(y).P} := \{ x \} \cup (\freenames{P} \setminus \{ y \})
  \and 
  \freenames{x!\langle P \rangle} := \{ x \} \cup \{ P \} 
  \and \\
  \freenames{P|Q} := \freenames{P} \cup \freenames{Q}
  \and \\
  \freenames{@{x}} := \{ x \}
\end{mathpar}

$\pi$
$\quotep{\pi}$

$\freenames{-} : \pi \to \mathcal{P}(\quotep{\pi})$

\begin{eqnarray*}
  \freenames{\pzero} & := & \emptyset \\
  \freenames{x?(y).P} & := & \{ x \} \cup (\freenames{P} \setminus \{ y \}) \\
  \freenames{x!\langle P \rangle} & := & \{ x \} \cup \{ P \} \\
  \freenames{P|Q} & := & \freenames{P} \cup \freenames{Q} \\
  \freenames{\dropn{x}} & := & \{ x \}
\end{eqnarray*}

The bound names of a process, $\boundnames{P}$, are those names occurring in $P$
that are not free. For example, in $x?(y).0$, the name $x$ is free, while $y$ is bound.

\begin{mathpar}
  \inferrule* [lab=monoidal-laws] {} { P|Q \equiv Q|P \and P|0 \equiv P \and P|(Q|R) \equiv (P|Q)|R }
\end{mathpar}

\begin{mathpar}
  \inferrule* [lab=alpha-equivalence] {} { (x)P \equiv (y)P\{y/x\} \and y \not\in \freenames{P} }
\end{mathpar}

\begin{definition}
Then two processes, $P,Q$, are alpha-equivalent if $P = Q\{\vec{y}/\vec{x}\}$ for
some $\vec{x} \in \boundnames{Q},\vec{y} \in \boundnames{P}$, where $Q\{\vec{y}/\vec{x}\}$
denotes the capture-avoiding substitution of $\vec{y}$ for $\vec{x}$ in $Q$.
\end{definition}

\begin{definition}
  The {\em structural congruence} \cite{SangiorgiWalker} , $\equiv$,
  between processes is the least congruence containing
  alpha-equivalence, satisfying the abelian monoid laws
  (associativity, commutativity and $\pzero$ as identity) for parallel
  composition $|$ and for summation $+$.
\end{definition}

\subsection{Name equivalence}

We take name equivalence, written $\nameeq$, to be the smallest
equivalence relation generated by the following rules.

\begin{mathpar}
\inferrule*[lab=Quote-drop]
{ }
{ \quotep{@{x}} \nameeq x }

\inferrule*[lab=Struct-equiv]
{ P \scong Q }
{ \quotep{P} \nameeq \quotep{Q} }
\end{mathpar}

The astute reader will have noticed that the mutual recursion of names
and processes imposes a mutual recursion on alpha-equivalence and
structural equivalence via name-equivalence. Fortunately, all of this
works out pleasantly and we may calculate in the natural way, free of
concern. The reader interested in the details is referred to the
appendix \ref{appendix:rho_details}.

\subsection{Substitution}

We use $\Proc$ for the set of processes, $\QProc$ for the set of
names, and $\id{\{}\vec{y} / \vec{x} \id{\}}$ to denote partial maps,
$s : \QProc \rightarrow \QProc$. A map, $s$ lifts, uniquely, to a map
on process terms, $\widehat{s} : \Proc \rightarrow \Proc$ by the
following equations.

\begin{mathpar}
  (0) \psubstp{Q}{P} := 0 \\
  (R \juxtap S) \psubstp{Q}{P}
  :=    
  (R)\psubstp{Q}{P} \juxtap (S) \psubstp{Q}{P} \\
  (x?(y).R) \psubstp{Q}{P}    
  :=    
  (x)\substp{Q}{P} (z)\concat( (R \psubstn{z}{y}) \psubstp{Q}{P} ) \\
  (\lift{x}{R}) \psubstp{Q}{P}  
  :=
  \lift{(x)\substp{Q}{P}}{ R \psubstp{Q}{P} } \\
%   (\dropn{x})  \psubstp{Q}{P}       
%   := 
%   \left\{ 
%     \begin{array}{ccc} 
%       \dropn{\quotep{Q}} & & x \nameeq \quotep{P} \\
%       \dropn{x} & & otherwise \\
%     \end{array}
%   \right. 
  (\dropn{x})  \psubstp{Q}{P}       
  := 
  \left\{ 
    \begin{array}{ccc} 
      Q & & x \nameeq \quotep{P} \\
      \dropn{x} & & otherwise \\
    \end{array}
  \right.
\end{mathpar}
 

where

\begin{eqnarray}
  (x)\id{\{} \lpquote Q \rpquote / \lpquote P \rpquote \id{\}}            = 
  \left\{ 
    \begin{array}{ccc}
      \lpquote Q \rpquote & & x \nameeq \lpquote P \rpquote \\
      x & & otherwise \\
    \end{array}
  \right. \nonumber
\end{eqnarray}

and $z$ is chosen distinct from $\quotep{P}$, $\quotep{Q}$, the free
names in $Q$, and all the names in $R$. Our $\alpha$-equivalence will
be built in the standard way from this substitution.

\begin{remark}\label{rem:no_self_referential_names}
  One consequence of these definitions is that $\forall P. \quotep{P}
  \not\in \freenames{P}$.
\end{remark}

\subsection{ Dynamic quote: an example }

Anticipating something of what's to come, consider applying the
substitution, $\widehat{\id{\{}u / z \id{\}}}$, to the following pair
of processes, $\lift{w}{y!(z)}$ and $w[ \lpquote y!(z) \rpquote ]$.

\begin{eqnarray}
	\lift{w}{y!(z)}\widehat{\id{\{}u / z \id{\}}}
		& = &
		\lift{w}{y!(u)} \nonumber\\
	w[ \lpquote y!(z) \rpquote ] \widehat{ \id{\{}u / z \id{\}} }
		& = &
		w[ \lpquote y!(z) \rpquote ] \nonumber
\end{eqnarray}

Because the body of the process between quotes is impervious to
substitution, we get radically different answers. In fact, by
examining the first process in an input context,
e.g. $x?(z).\lift{w}{y!(z)}$, we see that the process under the lift
operator may be shaped by prefixed inputs binding a name inside it. In
this sense, the lift operator will be seen as a way to dynamically
construct processes before reifying them as names.

Finally equipped with these standard features we can present the
dynamics of the calculus.

\subsubsection{Operational semantics} 

Finally, we introduce the computational dynamics. What marks these
algebras as distinct from other more traditionally studied algebraic
structures, e.g. vector spaces or polynomial rings, is the manner in
which dynamics is captured. In traditional structures, dynamics is typically
expressed through morphisms between such structures, as in linear maps
between vector spaces or morphisms between rings. In algebras
associated with the semantics of computation, the dynamics is
expressed as part of the algebraic structure itself, through a
reduction reduction relation typically denoted by $\red$. Below, we
give a recursive presentation of this relation for the calculus used
in the encoding.

$\red \subseteq \pi \times \pi$
$\red : \pi \to \mathcal{P}(\pi)$

\begin{mathpar}
  \inferrule* [lab=Comm] { \textsf{match}( x_{src}, x_{trgt} ) } { x_{trgt}?(y)P \; | \; x_{src}!\langle {Q} \rangle \red P\{\quotep{Q}/y}\} }
  \and \\
  \inferrule* [lab=Par] {{P} \red {P}'} {{{P} | {Q}} \red {{P}' | {Q}}}
  \and
  \inferrule* [lab=Equiv]{{{P} \scong {P}'} \andalso {{P}' \red {Q}'} \andalso {{Q}' \scong {Q}}}{{P} \red {Q}}
\end{mathpar}

\begin{eqnarray*}
  match_{\equiv} (\quotep{P},\quotep{Q}) & := & P \equiv Q \\
  match_{\dagger}(\quotep{P},\quotep{Q}) & := & \forall R. P|Q \red^{*} R => R \red^{*} 0 \\
  match_{K}(\quotep{P},\quotep{Q}) & := & K \mbox{ for some context } K
\end{eqnarray*}

$u?(x)P | u!\langle Q \rangle \red P\{\quotep{Q}/x\}$

%We write $\wred$ for $\red^*$, and $P\red$ if $\exists Q $ such that $ P \red Q$.
We write $P\red$ if $\exists Q $ such that $ P \red Q$ and $P\not\red$, otherwise.

\section{Replication}

As mentioned before, it is known that replication (and hence
recursion) can be implemented in a higher-order process algebra
\cite{SangiorgiWalker}. As our first example of calculation with the
machinery thus far presented we give the construction explicitly in
the {\rhoc}.

\begin{eqnarray}
	D_{x} & := & \prefix{x}{y}{(\binpar{\outputp{x}{y}}{@{y}})} \nonumber\\
	\bangp_{x}{P} & := & \binpar{{x}!\langle{\binpar{D_{x}}{P}}\rangle}{D_{x}} \nonumber
\end{eqnarray}

\begin{eqnarray}
	\bangp_{x}{P} & & \nonumber\\
	=
	& {x}!\langle{(\prefix{x}{y}{(\outputp{x}{y} | @{y})) | P}}\rangle 
	      | \prefix{x}{y}{(\outputp{x}{y} | @{y})} & \nonumber\\
	\red
	& (\outputp{x}{y} | @{y})\substn{\quotep{(\prefix{x}{y}{(@{y} | \outputp{x}{y})) | P}}}{y} & \nonumber\\
	=
	& \outputp{x}{\quotep{(\prefix{x}{y}{(\outputp{x}{y} | @{y})) | P}}}
	  | {(\prefix{x}{y}{(\outputp{x}{y} | @{y})) | P}} & \nonumber\\
	\red
	& \ldots & \nonumber\\
	\red^*
	& P | P | \ldots & \nonumber
\end{eqnarray}

Of course, this encoding, as an implementation, runs away, unfolding
$\bangp{P}$ eagerly. A lazier and more implementable replication
operator, restricted to input-guarded processes, may be obtained as follows.

\begin{eqnarray}
\bangp{\prefix{u}{v}{P}} 
	:= 
	\binpar{\lift{x}{\prefix{u}{v}{(\binpar{D(x)}{P})}}}{D(x)} \nonumber
\end{eqnarray}

\begin{remark}
  Note that the lazier definition still does not deal with summation
  or mixed summation (i.e. sums over input and output). The reader is
  invited to construct definitions of replication that deal with these
  features. 

  Further, the definitions are parameterized in a name, $x$. Can you,
  gentle reader, make a definition that eliminates this parameter and
  guarantees no accidental interaction between the replication
  machinery and the process being replicated -- i.e. no accidental
  sharing of names used by the process to get its work done and the
  name(s) used by the replication to effect copying. This latter
  revision of the definition of replication is crucial to obtaining
  the expected identity $!!P \sim !P$.
\end{remark}

\begin{remark}\label{rem:paradoxical_combinator}
  The reader familiar with the lambda calculus will have noticed the
  similarity between $D$ and the paradoxical combinator.

  [Ed. note: the existence of this seems to suggest we have to be more
  restrictive on the set of processes and names we admit if we are to
  support no-cloning.]
\end{remark}

\subsubsection{Bisimulation}

The computational dynamics gives rise to another kind of equivalence,
the equivalence of computational behavior. As previously mentioned
this is typically captured \emph{via} some form of bisimulation.

% The notion we use in this paper is weak barbed bisimulation
% \cite{milner91polyadicpi}.

The notion we use in this paper is derived from weak barbed
bisimulation \cite{milner91polyadicpi}. 

\begin{definition}
An \emph{observation relation}, $\downarrow_{\mathcal N}$, over a set
of names, $\mathcal N$, is the smallest relation satisfying the rules
below.

\infrule[Out-barb]{y \in {\mathcal N}, \; x \nameeq y}
		  {\outputp{x}{v} \downarrow_{\mathcal N} x}
\infrule[Par-barb]{\mbox{$P\downarrow_{\mathcal N} x$ or $Q\downarrow_{\mathcal N} x$}}
		  {\binpar{P}{Q} \downarrow_{\mathcal N} x}

We write $P \Downarrow_{\mathcal N} x$ if there is $Q$ such that 
$P \wred Q$ and $Q \downarrow_{\mathcal N} x$.
\end{definition}

\begin{definition}
%\label{def.bbisim}
An  ${\mathcal N}$-\emph{barbed bisimulation} over a set of names, ${\mathcal N}$, is a symmetric binary relation 
${\mathcal S}_{\mathcal N}$ between agents such that $P\rel{S}_{\mathcal N}Q$ implies:
\begin{enumerate}
\item If $P \red P'$ then $Q \wred Q'$ and $P'\rel{S}_{\mathcal N} Q'$.
\item If $P\downarrow_{\mathcal N} x$, then $Q\Downarrow_{\mathcal N} x$.
\end{enumerate}
$P$ is ${\mathcal N}$-barbed bisimilar to $Q$, written
$P \wbbisim_{\mathcal N} Q$, if $P \rel{S}_{\mathcal N} Q$ for some ${\mathcal N}$-barbed bisimulation ${\mathcal S}_{\mathcal N}$.
\end{definition}

$\mathcal{R} \subseteq \pi \times \pi$

$P \mathcal{R} Q => \forall P'. P \red P' \Rightarrow \exists Q'. Q \red Q', P' \mathcal{R} Q'$

$P \vdash x \Rightarrow Q \vdash x$

\begin{mathpar}
  \inferrule*[lab=Out-barb]{x \nameeq y}{{y}!\langle{Q}\rangle \vdash x}
  \and
  \inferrule*[lab=Par-barb]{\mbox{$P\vdash x$ or $Q\vdash x$}}{\binpar{P}{Q} \vdash x}
\end{mathpar}

\subsubsection{Contexts}

One of the principle advantages of computational calculi like the
$\pi$-calculus is a well-defined notion of context,
contextual-equivalence and a correlation between
contextual-equivalence and notions of bisimulation. The notion of
context allows the decomposition of a process into (sub-)process and
its syntactic environment, its context. Thus, a context may be
thought of as a process with a ``hole'' (written $\Box$) in it. The
application of a context $M$ to a process $P$, written $M[P]$, is
tantamount to filling the hole in $M$ with $P$. In this paper we do
not need the full weight of this theory, but do make use of the notion
of context in the proof the main theorem. 

\begin{mathpar}
  \inferrule* [lab=summation] {} {{M_{M},M_{N}} \bc \Box \;|\; x.M_{A} \;|\; M_{M}+M_{N}}
  \and
  \inferrule* [lab=agent] {} {{M_{A}} \bc (\vec{x})M_{P} \;| \; \clift{P_0,\ldots,M_{P},\ldots,P_N}}
  \and \\
  \inferrule* [lab=process] {} {{M_{P}} \bc M_{N} \;| \;P|M_{P} }
\end{mathpar} 

\begin{mathpar}
  \inferrule* [lab=sychronization] {} {M_{N} \bc \Box \;|\; x?M_{F} \;|\; x!M_{C}}
  \and
  \inferrule* [lab=abstraction] {} {{M_{F}} \bc (x)M_{P} }
  \and
  \inferrule* [lab=concretion] {} {{M_{C}} \bc \langle M_{P} \rangle }
  \and \\
  \inferrule* [lab=process] {} {{M_{P}} \bc M_{N} \;| \;P|M_{P} }
\end{mathpar}

\begin{definition}[contextual application] Given a context $M$, and
  process $P$, we define the \emph{contextual application}, $M[P] :=
  M\{P/\Box\}$. That is, the contextual application of M to P is the
  substitution of $P$ for $\Box$ in $M$.
\end{definition}

$\meaningof{-} : L \to \mathcal{P}(\pi)$

\begin{mathpar}
  \inferrule* [lab=collection] {} {\meaningof{true} = \pi, \and \meaningof{~E} = \pi \setminus \meaningof{E}, \and \meaningof{E_{1} \& E_{2}} = \meaningof{E_{1}} \cap \meaningof{E_{2}}}
\end{mathpar}

\begin{mathpar}
  \inferrule* [lab=structure] {} {\meaningof{0} = \{ P \in \pi | P \equiv 0 \}, \and \\ \meaningof{E_1 | E_2} = \{ P \in \pi | P \equiv P_{1} | P_{2}, P_{1} \in \meaningof{E_{1}}, P_{2} \in \meaningof{E_2}\} }
\end{mathpar}

\begin{mathpar}
 \inferrule* [lab=behavior] {} {\meaningof{\langle a?b \rangle E} = \{ P \in \pi | P \equiv Q | u?(y)P', \\ \and \\\\ \and \\ \;\;\; u \in \meaningof{a}, \forall z.P'\{z/y\} \in \meaningof{E\{z/b\}}\}, \and \\ \meaningof{a!E} = \{ P \in \pi | P \equiv Q | x!\langle P' \rangle, x \in \meaningof{a} P' \in \meaningof{E}\} }
\end{mathpar}

\begin{mathpar}
 \inferrule* [lab=nominal] {} {\meaningof{\quotep{E}} = \{ \quotep{P} \in \quotep{\pi} | P \in \meaningof{E} \}, \and \meaningof{\quotep{P}} = \{ \quotep{Q} \in \quotep{\pi} | P \equiv Q \} \and \\ \meaningof{@\quotep{E}} = \{ P \in \pi | P \equiv @x, x \in \meaningof{E} \}}
\end{mathpar}

\begin{eqnarray*}
  \\
  \meaningof{-} : TS \to ST
\end{eqnarray*}

\begin{eqnarray*}
  \\
  L : TS \to ST
\end{eqnarray*}

\begin{eqnarray*}
  \\
  P \models E \iff P \in \meaningof{E}
\end{eqnarray*}

\begin{eqnarray*}
  P \approx_{L} Q \iff \forall E \in L. P \models E \iff Q \models E
\end{eqnarray*}

\begin{eqnarray*}
  P \approx_{K} Q
\end{eqnarray*}

\begin{eqnarray*}
  P \approx Q
\end{eqnarray*}

$\approx_{K} = \approx = \approx_{L}$

\subsubsection{Contextual duality}

Note that contexts extend the quotation operation to a family of
operations from processes to names. Given a context, $M$, we can
define a \emph{nominal context}, $\quotep{M}$ by $\quotep{M}[P] :=
\quotep{M[P]}$. To foreshadow what is to come we observe that these
operations enjoy a duality with processes very much like the duality
between vectors and maps from vectors to scalars.

Further, because the calculus is essentially higher-order, we have a
correspondence between contexts and processes. More specifically,
given a name $x$ and a context $M$ we can construct $M^{*}_{x}$ such
that 

\begin{mathpar}
  M^{*}_{x} | \lift{x}{P} \red M[P]
\end{mathpar}

namely,

\begin{mathpar}
  M^{*}_{x} := x?(u).M[\dropn{u}]
\end{mathpar}

The dependence of $M^{*}_{x}$ on a name makes it an abstraction, 

\begin{mathpar}
  M^{*} := (x)x?(u).M[\dropn{u}]
\end{mathpar}

\subsection{Additional notation}

It will sometimes be convenient to denote the process a name
quotes. We already have the notation $x = \quotep{P}$, but it will be
convenient to introduce an alternate notation, $\procn{x}$, when we
want to emphasize the connection to the use of the name. Note that, by
virtue of name equivalence, $\quotep{\procn{x}} \nameeq x$; so, the
notation is consistent with previous definitions.

Further, because names have structure it is possible to effect
substitutions on the basis of that structure. This means we need to
upgrade our notation for substitutions, which we accomplish by
adapting comprehension notation. Thus,

\begin{mathpar}
  P\{ y / x : x \in S \}
\end{mathpar}

is interpreted to mean the process derived from P by replacing (in a
capture-avoiding manner) each occurrence of $x$ in $S$ by $y$. For example,

\begin{mathpar}
  P\{ \quotep{\procn{x}|\procn{x}} / x : x \in \freenames{P} \}
\end{mathpar}

will replace each (occurrence) of a free name $x$ in $P$ by
$\quotep{\procn{x}|\procn{x}}$.

Also, we will avail ourselves of the notation $x^{L}$ and $x^{R}$ to
denote injections of a name into disjoint copies of the name
space. There are numerous ways to accomplish this. One example can be
found in \cite{MeredithR05}. This notation overloads to vectors of
names: $\vec{x}^{\pi} := (x_{i}^{\pi} \; : \; 0 \leq i < |\vec{x}| )$ where $\pi \in \{L,R\}$.

We also use $P^{\Box} := P|\Box$.

In \cite{MeredithR05} an interpretation of the new operator is
given. It turns out that there are several possible interpretations
all enjoying the requisite algebraic properties of the operator (see
\cite{milner91polyadicpi}). We will therefore make liberal use of
$(\nu\; \vec{x})P$.

% subsection the_syntax_and_semantics_of_the_notation_system (end)   

\input{qm2pi.qmops} 

\input{qm2pi.sterngerlach} 

\input{qm2pi.metric} 

% section concurrent_process_calculi (end)

%\input{qm2pi.proofsketch}

% section proof sketch (end)

%\input{qm2pi.slviaknots} 

% section spatial logic via knots (end)

\input{qm2pi.conclusion}

% section conclusion (end)

%\input{qm2pi.dtcodes} 

% section wiring algorithm (end)

\input{qm2pi.ack} 

% section acknowledgments (end)

\newpage


\bibliographystyle{plain}   
\bibliography{../../biblios/main.bib}

\input{qm2pi.rhodetails}

\end{document}

 

%\documentclass[12pt]{llncs}
%\documentclass{jktr}

\usepackage[pdftex]{hyperref}                   
\usepackage {listings}
\usepackage {mathpartir}
\usepackage{bcprules}
%\usepackage{listings}
                       
\usepackage{graphicx} 
%\usepackage[margins=2.5cm,nohead,nofoot]{geometry}
%\usepackage{geometry}
\usepackage{amsfonts}
\usepackage{amstext}
\usepackage{latexsym}
\usepackage{amssymb}
\usepackage{color}


%\include{myPreamble}
\include{qm2pi.local} 

%\ifpdf
%\usepackage[pdftex]{graphicx}
%\else
%\usepackage{graphicx}
%\fi

 % \ifpdf
%  \usepackage{pdfsync}
%  \if


%\title{Brief Article}
%\author{David F. Snyder}
%\author{L.G. Meredith}

%\address{Dept. of Math., Texas State University--San Marcos, San Marcos, TX 78666}
       
\pagestyle{empty}


\begin{document}

\lstset{language=[Objective]Caml,frame=shadowbox}

\input{qm2pi.front}

% section front matter (end)

\input{qm2pi.intro} 
 
% section introduction (end)

% \input{qm2pi.knotations} 

% section notation (end)

\input{qm2pi.process.calculi} 

% section concurrent_process_calculi_and_spatial_logics_ (end)
    
%\input{qm2pi.knots2pi} 

%\input{qm2pi.trefoil} 

%\input{qm2pi.mainthm} 

% subsection basic_interpretation (end)

%\input{qm2pi.rho.presentation} 
\subsection{The syntax and semantics of the notation system}\label{sub:the_syntax_and_semantics_of_the_notation_system} % (fold)

We now summarize a technical presentation of the calculus that
embodies our theory of dynamics. The typical presentation of such a
calculus follows the style of giving generators and relations on
them. The grammar, below, describing term constructors, freely
generates the set of processes, $\Proc$. This set is then quotiented
by a relation known as structural congruence and it is over this set
that the notion of dynamics is expressed. This presentation is
essentially that of \cite{MeredithR05} with the addition of
polyadicity and summation. For readability we have relegated some of
the technical subtleties to an appendix.

\subsubsection{Process grammar}\label{subsub:process_grammar}

\begin{mathpar}
  \inferrule* [lab=synchronization] {} {{M} \bc \pzero \;|\; x?F \;|\; x!C }
  \and
  \inferrule* [lab=abstraction] {} {{F} \bc (x)P}
  \and
  \inferrule* [lab=concretion] {} {{C} \bc \langle Q \rangle}
  \and
  \inferrule* [lab=process] {} {{P,Q} \bc M \;| \;P|Q \;|\; @{x}}
  \and
  \inferrule* [lab=name] {} {{x} \bc \quotep{P}}
\end{mathpar} 

Note that $\vec{x}$ (resp. $\vec{P}$) denotes a vector of names
(resp. processes) of length $|\vec{x}|$ (resp. $|\vec{P}|$). We adopt
the following useful abbreviations.

\begin{mathpar}
   x?(\vec{y}).P := x.(\vec{y})P \and  x\clift{\vec{P}} := x.\clift{\vec{P}}
   \and x!(y) := \lift{x}{\dropn{y}}
   \and \Pi_{i=0}^{n-1}P_i := P_0 | \ldots | P_{n-1}
\end{mathpar}

\subsubsection{Structural congruence}

\paragraph{Free and bound names and alpha-equivalence.} At the
core of structural equivalence is alpha-equivalence which identifies
process that are the same up to a change of variable. Formally, we
recognize the distinction between free and bound names. The free names
of a process, $\freenames{P}$, may be calculated recursively as
follows:

\begin{mathpar}
\freenames{\pzero} := \emptyset
  \and \\
  \freenames{x?(y).P} := \{ x \} \cup (\freenames{P} \setminus \{ y \})
  \and 
  \freenames{x!\langle P \rangle} := \{ x \} \cup \{ P \} 
  \and \\
  \freenames{P|Q} := \freenames{P} \cup \freenames{Q}
  \and \\
  \freenames{@{x}} := \{ x \}
\end{mathpar}

$\pi$
$\quotep{\pi}$

$\freenames{-} : \pi \to \mathcal{P}(\quotep{\pi})$

\begin{eqnarray*}
  \freenames{\pzero} & := & \emptyset \\
  \freenames{x?(y).P} & := & \{ x \} \cup (\freenames{P} \setminus \{ y \}) \\
  \freenames{x!\langle P \rangle} & := & \{ x \} \cup \{ P \} \\
  \freenames{P|Q} & := & \freenames{P} \cup \freenames{Q} \\
  \freenames{\dropn{x}} & := & \{ x \}
\end{eqnarray*}

The bound names of a process, $\boundnames{P}$, are those names occurring in $P$
that are not free. For example, in $x?(y).0$, the name $x$ is free, while $y$ is bound.

\begin{mathpar}
  \inferrule* [lab=monoidal-laws] {} { P|Q \equiv Q|P \and P|0 \equiv P \and P|(Q|R) \equiv (P|Q)|R }
\end{mathpar}

\begin{mathpar}
  \inferrule* [lab=alpha-equivalence] {} { (x)P \equiv (y)P\{y/x\} \and y \not\in \freenames{P} }
\end{mathpar}

\begin{definition}
Then two processes, $P,Q$, are alpha-equivalent if $P = Q\{\vec{y}/\vec{x}\}$ for
some $\vec{x} \in \boundnames{Q},\vec{y} \in \boundnames{P}$, where $Q\{\vec{y}/\vec{x}\}$
denotes the capture-avoiding substitution of $\vec{y}$ for $\vec{x}$ in $Q$.
\end{definition}

\begin{definition}
  The {\em structural congruence} \cite{SangiorgiWalker} , $\equiv$,
  between processes is the least congruence containing
  alpha-equivalence, satisfying the abelian monoid laws
  (associativity, commutativity and $\pzero$ as identity) for parallel
  composition $|$ and for summation $+$.
\end{definition}

\subsection{Name equivalence}

We take name equivalence, written $\nameeq$, to be the smallest
equivalence relation generated by the following rules.

\begin{mathpar}
\inferrule*[lab=Quote-drop]
{ }
{ \quotep{@{x}} \nameeq x }

\inferrule*[lab=Struct-equiv]
{ P \scong Q }
{ \quotep{P} \nameeq \quotep{Q} }
\end{mathpar}

The astute reader will have noticed that the mutual recursion of names
and processes imposes a mutual recursion on alpha-equivalence and
structural equivalence via name-equivalence. Fortunately, all of this
works out pleasantly and we may calculate in the natural way, free of
concern. The reader interested in the details is referred to the
appendix \ref{appendix:rho_details}.

\subsection{Substitution}

We use $\Proc$ for the set of processes, $\QProc$ for the set of
names, and $\id{\{}\vec{y} / \vec{x} \id{\}}$ to denote partial maps,
$s : \QProc \rightarrow \QProc$. A map, $s$ lifts, uniquely, to a map
on process terms, $\widehat{s} : \Proc \rightarrow \Proc$ by the
following equations.

\begin{mathpar}
  (0) \psubstp{Q}{P} := 0 \\
  (R \juxtap S) \psubstp{Q}{P}
  :=    
  (R)\psubstp{Q}{P} \juxtap (S) \psubstp{Q}{P} \\
  (x?(y).R) \psubstp{Q}{P}    
  :=    
  (x)\substp{Q}{P} (z)\concat( (R \psubstn{z}{y}) \psubstp{Q}{P} ) \\
  (\lift{x}{R}) \psubstp{Q}{P}  
  :=
  \lift{(x)\substp{Q}{P}}{ R \psubstp{Q}{P} } \\
%   (\dropn{x})  \psubstp{Q}{P}       
%   := 
%   \left\{ 
%     \begin{array}{ccc} 
%       \dropn{\quotep{Q}} & & x \nameeq \quotep{P} \\
%       \dropn{x} & & otherwise \\
%     \end{array}
%   \right. 
  (\dropn{x})  \psubstp{Q}{P}       
  := 
  \left\{ 
    \begin{array}{ccc} 
      Q & & x \nameeq \quotep{P} \\
      \dropn{x} & & otherwise \\
    \end{array}
  \right.
\end{mathpar}
 

where

\begin{eqnarray}
  (x)\id{\{} \lpquote Q \rpquote / \lpquote P \rpquote \id{\}}            = 
  \left\{ 
    \begin{array}{ccc}
      \lpquote Q \rpquote & & x \nameeq \lpquote P \rpquote \\
      x & & otherwise \\
    \end{array}
  \right. \nonumber
\end{eqnarray}

and $z$ is chosen distinct from $\quotep{P}$, $\quotep{Q}$, the free
names in $Q$, and all the names in $R$. Our $\alpha$-equivalence will
be built in the standard way from this substitution.

\begin{remark}\label{rem:no_self_referential_names}
  One consequence of these definitions is that $\forall P. \quotep{P}
  \not\in \freenames{P}$.
\end{remark}

\subsection{ Dynamic quote: an example }

Anticipating something of what's to come, consider applying the
substitution, $\widehat{\id{\{}u / z \id{\}}}$, to the following pair
of processes, $\lift{w}{y!(z)}$ and $w[ \lpquote y!(z) \rpquote ]$.

\begin{eqnarray}
	\lift{w}{y!(z)}\widehat{\id{\{}u / z \id{\}}}
		& = &
		\lift{w}{y!(u)} \nonumber\\
	w[ \lpquote y!(z) \rpquote ] \widehat{ \id{\{}u / z \id{\}} }
		& = &
		w[ \lpquote y!(z) \rpquote ] \nonumber
\end{eqnarray}

Because the body of the process between quotes is impervious to
substitution, we get radically different answers. In fact, by
examining the first process in an input context,
e.g. $x?(z).\lift{w}{y!(z)}$, we see that the process under the lift
operator may be shaped by prefixed inputs binding a name inside it. In
this sense, the lift operator will be seen as a way to dynamically
construct processes before reifying them as names.

Finally equipped with these standard features we can present the
dynamics of the calculus.

\subsubsection{Operational semantics} 

Finally, we introduce the computational dynamics. What marks these
algebras as distinct from other more traditionally studied algebraic
structures, e.g. vector spaces or polynomial rings, is the manner in
which dynamics is captured. In traditional structures, dynamics is typically
expressed through morphisms between such structures, as in linear maps
between vector spaces or morphisms between rings. In algebras
associated with the semantics of computation, the dynamics is
expressed as part of the algebraic structure itself, through a
reduction reduction relation typically denoted by $\red$. Below, we
give a recursive presentation of this relation for the calculus used
in the encoding.

$\red \subseteq \pi \times \pi$
$\red : \pi \to \mathcal{P}(\pi)$

\begin{mathpar}
  \inferrule* [lab=Comm] { \textsf{match}( x_{src}, x_{trgt} ) } { x_{trgt}?(y)P \; | \; x_{src}!\langle {Q} \rangle \red P\{\quotep{Q}/y}\} }
  \and \\
  \inferrule* [lab=Par] {{P} \red {P}'} {{{P} | {Q}} \red {{P}' | {Q}}}
  \and
  \inferrule* [lab=Equiv]{{{P} \scong {P}'} \andalso {{P}' \red {Q}'} \andalso {{Q}' \scong {Q}}}{{P} \red {Q}}
\end{mathpar}

\begin{eqnarray*}
  match_{\equiv} (\quotep{P},\quotep{Q}) & := & P \equiv Q \\
  match_{\dagger}(\quotep{P},\quotep{Q}) & := & \forall R. P|Q \red^{*} R => R \red^{*} 0 \\
  match_{K}(\quotep{P},\quotep{Q}) & := & K \mbox{ for some context } K
\end{eqnarray*}

$u?(x)P | u!\langle Q \rangle \red P\{\quotep{Q}/x\}$

%We write $\wred$ for $\red^*$, and $P\red$ if $\exists Q $ such that $ P \red Q$.
We write $P\red$ if $\exists Q $ such that $ P \red Q$ and $P\not\red$, otherwise.

\section{Replication}

As mentioned before, it is known that replication (and hence
recursion) can be implemented in a higher-order process algebra
\cite{SangiorgiWalker}. As our first example of calculation with the
machinery thus far presented we give the construction explicitly in
the {\rhoc}.

\begin{eqnarray}
	D_{x} & := & \prefix{x}{y}{(\binpar{\outputp{x}{y}}{@{y}})} \nonumber\\
	\bangp_{x}{P} & := & \binpar{{x}!\langle{\binpar{D_{x}}{P}}\rangle}{D_{x}} \nonumber
\end{eqnarray}

\begin{eqnarray}
	\bangp_{x}{P} & & \nonumber\\
	=
	& {x}!\langle{(\prefix{x}{y}{(\outputp{x}{y} | @{y})) | P}}\rangle 
	      | \prefix{x}{y}{(\outputp{x}{y} | @{y})} & \nonumber\\
	\red
	& (\outputp{x}{y} | @{y})\substn{\quotep{(\prefix{x}{y}{(@{y} | \outputp{x}{y})) | P}}}{y} & \nonumber\\
	=
	& \outputp{x}{\quotep{(\prefix{x}{y}{(\outputp{x}{y} | @{y})) | P}}}
	  | {(\prefix{x}{y}{(\outputp{x}{y} | @{y})) | P}} & \nonumber\\
	\red
	& \ldots & \nonumber\\
	\red^*
	& P | P | \ldots & \nonumber
\end{eqnarray}

Of course, this encoding, as an implementation, runs away, unfolding
$\bangp{P}$ eagerly. A lazier and more implementable replication
operator, restricted to input-guarded processes, may be obtained as follows.

\begin{eqnarray}
\bangp{\prefix{u}{v}{P}} 
	:= 
	\binpar{\lift{x}{\prefix{u}{v}{(\binpar{D(x)}{P})}}}{D(x)} \nonumber
\end{eqnarray}

\begin{remark}
  Note that the lazier definition still does not deal with summation
  or mixed summation (i.e. sums over input and output). The reader is
  invited to construct definitions of replication that deal with these
  features. 

  Further, the definitions are parameterized in a name, $x$. Can you,
  gentle reader, make a definition that eliminates this parameter and
  guarantees no accidental interaction between the replication
  machinery and the process being replicated -- i.e. no accidental
  sharing of names used by the process to get its work done and the
  name(s) used by the replication to effect copying. This latter
  revision of the definition of replication is crucial to obtaining
  the expected identity $!!P \sim !P$.
\end{remark}

\begin{remark}\label{rem:paradoxical_combinator}
  The reader familiar with the lambda calculus will have noticed the
  similarity between $D$ and the paradoxical combinator.

  [Ed. note: the existence of this seems to suggest we have to be more
  restrictive on the set of processes and names we admit if we are to
  support no-cloning.]
\end{remark}

\subsubsection{Bisimulation}

The computational dynamics gives rise to another kind of equivalence,
the equivalence of computational behavior. As previously mentioned
this is typically captured \emph{via} some form of bisimulation.

% The notion we use in this paper is weak barbed bisimulation
% \cite{milner91polyadicpi}.

The notion we use in this paper is derived from weak barbed
bisimulation \cite{milner91polyadicpi}. 

\begin{definition}
An \emph{observation relation}, $\downarrow_{\mathcal N}$, over a set
of names, $\mathcal N$, is the smallest relation satisfying the rules
below.

\infrule[Out-barb]{y \in {\mathcal N}, \; x \nameeq y}
		  {\outputp{x}{v} \downarrow_{\mathcal N} x}
\infrule[Par-barb]{\mbox{$P\downarrow_{\mathcal N} x$ or $Q\downarrow_{\mathcal N} x$}}
		  {\binpar{P}{Q} \downarrow_{\mathcal N} x}

We write $P \Downarrow_{\mathcal N} x$ if there is $Q$ such that 
$P \wred Q$ and $Q \downarrow_{\mathcal N} x$.
\end{definition}

\begin{definition}
%\label{def.bbisim}
An  ${\mathcal N}$-\emph{barbed bisimulation} over a set of names, ${\mathcal N}$, is a symmetric binary relation 
${\mathcal S}_{\mathcal N}$ between agents such that $P\rel{S}_{\mathcal N}Q$ implies:
\begin{enumerate}
\item If $P \red P'$ then $Q \wred Q'$ and $P'\rel{S}_{\mathcal N} Q'$.
\item If $P\downarrow_{\mathcal N} x$, then $Q\Downarrow_{\mathcal N} x$.
\end{enumerate}
$P$ is ${\mathcal N}$-barbed bisimilar to $Q$, written
$P \wbbisim_{\mathcal N} Q$, if $P \rel{S}_{\mathcal N} Q$ for some ${\mathcal N}$-barbed bisimulation ${\mathcal S}_{\mathcal N}$.
\end{definition}

$\mathcal{R} \subseteq \pi \times \pi$

$P \mathcal{R} Q => \forall P'. P \red P' \Rightarrow \exists Q'. Q \red Q', P' \mathcal{R} Q'$

$P \vdash x \Rightarrow Q \vdash x$

\begin{mathpar}
  \inferrule*[lab=Out-barb]{x \nameeq y}{{y}!\langle{Q}\rangle \vdash x}
  \and
  \inferrule*[lab=Par-barb]{\mbox{$P\vdash x$ or $Q\vdash x$}}{\binpar{P}{Q} \vdash x}
\end{mathpar}

\subsubsection{Contexts}

One of the principle advantages of computational calculi like the
$\pi$-calculus is a well-defined notion of context,
contextual-equivalence and a correlation between
contextual-equivalence and notions of bisimulation. The notion of
context allows the decomposition of a process into (sub-)process and
its syntactic environment, its context. Thus, a context may be
thought of as a process with a ``hole'' (written $\Box$) in it. The
application of a context $M$ to a process $P$, written $M[P]$, is
tantamount to filling the hole in $M$ with $P$. In this paper we do
not need the full weight of this theory, but do make use of the notion
of context in the proof the main theorem. 

\begin{mathpar}
  \inferrule* [lab=summation] {} {{M_{M},M_{N}} \bc \Box \;|\; x.M_{A} \;|\; M_{M}+M_{N}}
  \and
  \inferrule* [lab=agent] {} {{M_{A}} \bc (\vec{x})M_{P} \;| \; \clift{P_0,\ldots,M_{P},\ldots,P_N}}
  \and \\
  \inferrule* [lab=process] {} {{M_{P}} \bc M_{N} \;| \;P|M_{P} }
\end{mathpar} 

\begin{mathpar}
  \inferrule* [lab=sychronization] {} {M_{N} \bc \Box \;|\; x?M_{F} \;|\; x!M_{C}}
  \and
  \inferrule* [lab=abstraction] {} {{M_{F}} \bc (x)M_{P} }
  \and
  \inferrule* [lab=concretion] {} {{M_{C}} \bc \langle M_{P} \rangle }
  \and \\
  \inferrule* [lab=process] {} {{M_{P}} \bc M_{N} \;| \;P|M_{P} }
\end{mathpar}

\begin{definition}[contextual application] Given a context $M$, and
  process $P$, we define the \emph{contextual application}, $M[P] :=
  M\{P/\Box\}$. That is, the contextual application of M to P is the
  substitution of $P$ for $\Box$ in $M$.
\end{definition}

$\meaningof{-} : L \to \mathcal{P}(\pi)$

\begin{mathpar}
  \inferrule* [lab=collection] {} {\meaningof{true} = \pi, \and \meaningof{~E} = \pi \setminus \meaningof{E}, \and \meaningof{E_{1} \& E_{2}} = \meaningof{E_{1}} \cap \meaningof{E_{2}}}
\end{mathpar}

\begin{mathpar}
  \inferrule* [lab=structure] {} {\meaningof{0} = \{ P \in \pi | P \equiv 0 \}, \and \\ \meaningof{E_1 | E_2} = \{ P \in \pi | P \equiv P_{1} | P_{2}, P_{1} \in \meaningof{E_{1}}, P_{2} \in \meaningof{E_2}\} }
\end{mathpar}

\begin{mathpar}
 \inferrule* [lab=behavior] {} {\meaningof{\langle a?b \rangle E} = \{ P \in \pi | P \equiv Q | u?(y)P', \\ \and \\\\ \and \\ \;\;\; u \in \meaningof{a}, \forall z.P'\{z/y\} \in \meaningof{E\{z/b\}}\}, \and \\ \meaningof{a!E} = \{ P \in \pi | P \equiv Q | x!\langle P' \rangle, x \in \meaningof{a} P' \in \meaningof{E}\} }
\end{mathpar}

\begin{mathpar}
 \inferrule* [lab=nominal] {} {\meaningof{\quotep{E}} = \{ \quotep{P} \in \quotep{\pi} | P \in \meaningof{E} \}, \and \meaningof{\quotep{P}} = \{ \quotep{Q} \in \quotep{\pi} | P \equiv Q \} \and \\ \meaningof{@\quotep{E}} = \{ P \in \pi | P \equiv @x, x \in \meaningof{E} \}}
\end{mathpar}

\begin{eqnarray*}
  \\
  \meaningof{-} : TS \to ST
\end{eqnarray*}

\begin{eqnarray*}
  \\
  L : TS \to ST
\end{eqnarray*}

\begin{eqnarray*}
  \\
  P \models E \iff P \in \meaningof{E}
\end{eqnarray*}

\begin{eqnarray*}
  P \approx_{L} Q \iff \forall E \in L. P \models E \iff Q \models E
\end{eqnarray*}

\begin{eqnarray*}
  P \approx_{K} Q
\end{eqnarray*}

\begin{eqnarray*}
  P \approx Q
\end{eqnarray*}

$\approx_{K} = \approx = \approx_{L}$

\subsubsection{Contextual duality}

Note that contexts extend the quotation operation to a family of
operations from processes to names. Given a context, $M$, we can
define a \emph{nominal context}, $\quotep{M}$ by $\quotep{M}[P] :=
\quotep{M[P]}$. To foreshadow what is to come we observe that these
operations enjoy a duality with processes very much like the duality
between vectors and maps from vectors to scalars.

Further, because the calculus is essentially higher-order, we have a
correspondence between contexts and processes. More specifically,
given a name $x$ and a context $M$ we can construct $M^{*}_{x}$ such
that 

\begin{mathpar}
  M^{*}_{x} | \lift{x}{P} \red M[P]
\end{mathpar}

namely,

\begin{mathpar}
  M^{*}_{x} := x?(u).M[\dropn{u}]
\end{mathpar}

The dependence of $M^{*}_{x}$ on a name makes it an abstraction, 

\begin{mathpar}
  M^{*} := (x)x?(u).M[\dropn{u}]
\end{mathpar}

\subsection{Additional notation}

It will sometimes be convenient to denote the process a name
quotes. We already have the notation $x = \quotep{P}$, but it will be
convenient to introduce an alternate notation, $\procn{x}$, when we
want to emphasize the connection to the use of the name. Note that, by
virtue of name equivalence, $\quotep{\procn{x}} \nameeq x$; so, the
notation is consistent with previous definitions.

Further, because names have structure it is possible to effect
substitutions on the basis of that structure. This means we need to
upgrade our notation for substitutions, which we accomplish by
adapting comprehension notation. Thus,

\begin{mathpar}
  P\{ y / x : x \in S \}
\end{mathpar}

is interpreted to mean the process derived from P by replacing (in a
capture-avoiding manner) each occurrence of $x$ in $S$ by $y$. For example,

\begin{mathpar}
  P\{ \quotep{\procn{x}|\procn{x}} / x : x \in \freenames{P} \}
\end{mathpar}

will replace each (occurrence) of a free name $x$ in $P$ by
$\quotep{\procn{x}|\procn{x}}$.

Also, we will avail ourselves of the notation $x^{L}$ and $x^{R}$ to
denote injections of a name into disjoint copies of the name
space. There are numerous ways to accomplish this. One example can be
found in \cite{MeredithR05}. This notation overloads to vectors of
names: $\vec{x}^{\pi} := (x_{i}^{\pi} \; : \; 0 \leq i < |\vec{x}| )$ where $\pi \in \{L,R\}$.

We also use $P^{\Box} := P|\Box$.

In \cite{MeredithR05} an interpretation of the new operator is
given. It turns out that there are several possible interpretations
all enjoying the requisite algebraic properties of the operator (see
\cite{milner91polyadicpi}). We will therefore make liberal use of
$(\nu\; \vec{x})P$.

% subsection the_syntax_and_semantics_of_the_notation_system (end)   

\input{qm2pi.qmops} 

\input{qm2pi.sterngerlach} 

\input{qm2pi.metric} 

% section concurrent_process_calculi (end)

%\input{qm2pi.proofsketch}

% section proof sketch (end)

%\input{qm2pi.slviaknots} 

% section spatial logic via knots (end)

\input{qm2pi.conclusion}

% section conclusion (end)

%\input{qm2pi.dtcodes} 

% section wiring algorithm (end)

\input{qm2pi.ack} 

% section acknowledgments (end)

\newpage


\bibliographystyle{plain}   
\bibliography{../../biblios/main.bib}

\input{qm2pi.rhodetails}

\end{document}

 

% subsection basic_interpretation (end)

%\input{qm2pi.rho.presentation} 
\subsection{The syntax and semantics of the notation system}\label{sub:the_syntax_and_semantics_of_the_notation_system} % (fold)

We now summarize a technical presentation of the calculus that
embodies our theory of dynamics. The typical presentation of such a
calculus follows the style of giving generators and relations on
them. The grammar, below, describing term constructors, freely
generates the set of processes, $\Proc$. This set is then quotiented
by a relation known as structural congruence and it is over this set
that the notion of dynamics is expressed. This presentation is
essentially that of \cite{MeredithR05} with the addition of
polyadicity and summation. For readability we have relegated some of
the technical subtleties to an appendix.

\subsubsection{Process grammar}\label{subsub:process_grammar}

\begin{mathpar}
  \inferrule* [lab=synchronization] {} {{M} \bc \pzero \;|\; x?F \;|\; x!C }
  \and
  \inferrule* [lab=abstraction] {} {{F} \bc (x)P}
  \and
  \inferrule* [lab=concretion] {} {{C} \bc \langle Q \rangle}
  \and
  \inferrule* [lab=process] {} {{P,Q} \bc M \;| \;P|Q \;|\; @{x}}
  \and
  \inferrule* [lab=name] {} {{x} \bc \quotep{P}}
\end{mathpar} 

Note that $\vec{x}$ (resp. $\vec{P}$) denotes a vector of names
(resp. processes) of length $|\vec{x}|$ (resp. $|\vec{P}|$). We adopt
the following useful abbreviations.

\begin{mathpar}
   x?(\vec{y}).P := x.(\vec{y})P \and  x\clift{\vec{P}} := x.\clift{\vec{P}}
   \and x!(y) := \lift{x}{\dropn{y}}
   \and \Pi_{i=0}^{n-1}P_i := P_0 | \ldots | P_{n-1}
\end{mathpar}

\subsubsection{Structural congruence}

\paragraph{Free and bound names and alpha-equivalence.} At the
core of structural equivalence is alpha-equivalence which identifies
process that are the same up to a change of variable. Formally, we
recognize the distinction between free and bound names. The free names
of a process, $\freenames{P}$, may be calculated recursively as
follows:

\begin{mathpar}
\freenames{\pzero} := \emptyset
  \and \\
  \freenames{x?(y).P} := \{ x \} \cup (\freenames{P} \setminus \{ y \})
  \and 
  \freenames{x!\langle P \rangle} := \{ x \} \cup \{ P \} 
  \and \\
  \freenames{P|Q} := \freenames{P} \cup \freenames{Q}
  \and \\
  \freenames{@{x}} := \{ x \}
\end{mathpar}

$\pi$
$\quotep{\pi}$

$\freenames{-} : \pi \to \mathcal{P}(\quotep{\pi})$

\begin{eqnarray*}
  \freenames{\pzero} & := & \emptyset \\
  \freenames{x?(y).P} & := & \{ x \} \cup (\freenames{P} \setminus \{ y \}) \\
  \freenames{x!\langle P \rangle} & := & \{ x \} \cup \{ P \} \\
  \freenames{P|Q} & := & \freenames{P} \cup \freenames{Q} \\
  \freenames{\dropn{x}} & := & \{ x \}
\end{eqnarray*}

The bound names of a process, $\boundnames{P}$, are those names occurring in $P$
that are not free. For example, in $x?(y).0$, the name $x$ is free, while $y$ is bound.

\begin{mathpar}
  \inferrule* [lab=monoidal-laws] {} { P|Q \equiv Q|P \and P|0 \equiv P \and P|(Q|R) \equiv (P|Q)|R }
\end{mathpar}

\begin{mathpar}
  \inferrule* [lab=alpha-equivalence] {} { (x)P \equiv (y)P\{y/x\} \and y \not\in \freenames{P} }
\end{mathpar}

\begin{definition}
Then two processes, $P,Q$, are alpha-equivalent if $P = Q\{\vec{y}/\vec{x}\}$ for
some $\vec{x} \in \boundnames{Q},\vec{y} \in \boundnames{P}$, where $Q\{\vec{y}/\vec{x}\}$
denotes the capture-avoiding substitution of $\vec{y}$ for $\vec{x}$ in $Q$.
\end{definition}

\begin{definition}
  The {\em structural congruence} \cite{SangiorgiWalker} , $\equiv$,
  between processes is the least congruence containing
  alpha-equivalence, satisfying the abelian monoid laws
  (associativity, commutativity and $\pzero$ as identity) for parallel
  composition $|$ and for summation $+$.
\end{definition}

\subsection{Name equivalence}

We take name equivalence, written $\nameeq$, to be the smallest
equivalence relation generated by the following rules.

\begin{mathpar}
\inferrule*[lab=Quote-drop]
{ }
{ \quotep{@{x}} \nameeq x }

\inferrule*[lab=Struct-equiv]
{ P \scong Q }
{ \quotep{P} \nameeq \quotep{Q} }
\end{mathpar}

The astute reader will have noticed that the mutual recursion of names
and processes imposes a mutual recursion on alpha-equivalence and
structural equivalence via name-equivalence. Fortunately, all of this
works out pleasantly and we may calculate in the natural way, free of
concern. The reader interested in the details is referred to the
appendix \ref{appendix:rho_details}.

\subsection{Substitution}

We use $\Proc$ for the set of processes, $\QProc$ for the set of
names, and $\id{\{}\vec{y} / \vec{x} \id{\}}$ to denote partial maps,
$s : \QProc \rightarrow \QProc$. A map, $s$ lifts, uniquely, to a map
on process terms, $\widehat{s} : \Proc \rightarrow \Proc$ by the
following equations.

\begin{mathpar}
  (0) \psubstp{Q}{P} := 0 \\
  (R \juxtap S) \psubstp{Q}{P}
  :=    
  (R)\psubstp{Q}{P} \juxtap (S) \psubstp{Q}{P} \\
  (x?(y).R) \psubstp{Q}{P}    
  :=    
  (x)\substp{Q}{P} (z)\concat( (R \psubstn{z}{y}) \psubstp{Q}{P} ) \\
  (\lift{x}{R}) \psubstp{Q}{P}  
  :=
  \lift{(x)\substp{Q}{P}}{ R \psubstp{Q}{P} } \\
%   (\dropn{x})  \psubstp{Q}{P}       
%   := 
%   \left\{ 
%     \begin{array}{ccc} 
%       \dropn{\quotep{Q}} & & x \nameeq \quotep{P} \\
%       \dropn{x} & & otherwise \\
%     \end{array}
%   \right. 
  (\dropn{x})  \psubstp{Q}{P}       
  := 
  \left\{ 
    \begin{array}{ccc} 
      Q & & x \nameeq \quotep{P} \\
      \dropn{x} & & otherwise \\
    \end{array}
  \right.
\end{mathpar}
 

where

\begin{eqnarray}
  (x)\id{\{} \lpquote Q \rpquote / \lpquote P \rpquote \id{\}}            = 
  \left\{ 
    \begin{array}{ccc}
      \lpquote Q \rpquote & & x \nameeq \lpquote P \rpquote \\
      x & & otherwise \\
    \end{array}
  \right. \nonumber
\end{eqnarray}

and $z$ is chosen distinct from $\quotep{P}$, $\quotep{Q}$, the free
names in $Q$, and all the names in $R$. Our $\alpha$-equivalence will
be built in the standard way from this substitution.

\begin{remark}\label{rem:no_self_referential_names}
  One consequence of these definitions is that $\forall P. \quotep{P}
  \not\in \freenames{P}$.
\end{remark}

\subsection{ Dynamic quote: an example }

Anticipating something of what's to come, consider applying the
substitution, $\widehat{\id{\{}u / z \id{\}}}$, to the following pair
of processes, $\lift{w}{y!(z)}$ and $w[ \lpquote y!(z) \rpquote ]$.

\begin{eqnarray}
	\lift{w}{y!(z)}\widehat{\id{\{}u / z \id{\}}}
		& = &
		\lift{w}{y!(u)} \nonumber\\
	w[ \lpquote y!(z) \rpquote ] \widehat{ \id{\{}u / z \id{\}} }
		& = &
		w[ \lpquote y!(z) \rpquote ] \nonumber
\end{eqnarray}

Because the body of the process between quotes is impervious to
substitution, we get radically different answers. In fact, by
examining the first process in an input context,
e.g. $x?(z).\lift{w}{y!(z)}$, we see that the process under the lift
operator may be shaped by prefixed inputs binding a name inside it. In
this sense, the lift operator will be seen as a way to dynamically
construct processes before reifying them as names.

Finally equipped with these standard features we can present the
dynamics of the calculus.

\subsubsection{Operational semantics} 

Finally, we introduce the computational dynamics. What marks these
algebras as distinct from other more traditionally studied algebraic
structures, e.g. vector spaces or polynomial rings, is the manner in
which dynamics is captured. In traditional structures, dynamics is typically
expressed through morphisms between such structures, as in linear maps
between vector spaces or morphisms between rings. In algebras
associated with the semantics of computation, the dynamics is
expressed as part of the algebraic structure itself, through a
reduction reduction relation typically denoted by $\red$. Below, we
give a recursive presentation of this relation for the calculus used
in the encoding.

$\red \subseteq \pi \times \pi$
$\red : \pi \to \mathcal{P}(\pi)$

\begin{mathpar}
  \inferrule* [lab=Comm] { \textsf{match}( x_{src}, x_{trgt} ) } { x_{trgt}?(y)P \; | \; x_{src}!\langle {Q} \rangle \red P\{\quotep{Q}/y}\} }
  \and \\
  \inferrule* [lab=Par] {{P} \red {P}'} {{{P} | {Q}} \red {{P}' | {Q}}}
  \and
  \inferrule* [lab=Equiv]{{{P} \scong {P}'} \andalso {{P}' \red {Q}'} \andalso {{Q}' \scong {Q}}}{{P} \red {Q}}
\end{mathpar}

\begin{eqnarray*}
  match_{\equiv} (\quotep{P},\quotep{Q}) & := & P \equiv Q \\
  match_{\dagger}(\quotep{P},\quotep{Q}) & := & \forall R. P|Q \red^{*} R => R \red^{*} 0 \\
  match_{K}(\quotep{P},\quotep{Q}) & := & K \mbox{ for some context } K
\end{eqnarray*}

$u?(x)P | u!\langle Q \rangle \red P\{\quotep{Q}/x\}$

%We write $\wred$ for $\red^*$, and $P\red$ if $\exists Q $ such that $ P \red Q$.
We write $P\red$ if $\exists Q $ such that $ P \red Q$ and $P\not\red$, otherwise.

\section{Replication}

As mentioned before, it is known that replication (and hence
recursion) can be implemented in a higher-order process algebra
\cite{SangiorgiWalker}. As our first example of calculation with the
machinery thus far presented we give the construction explicitly in
the {\rhoc}.

\begin{eqnarray}
	D_{x} & := & \prefix{x}{y}{(\binpar{\outputp{x}{y}}{@{y}})} \nonumber\\
	\bangp_{x}{P} & := & \binpar{{x}!\langle{\binpar{D_{x}}{P}}\rangle}{D_{x}} \nonumber
\end{eqnarray}

\begin{eqnarray}
	\bangp_{x}{P} & & \nonumber\\
	=
	& {x}!\langle{(\prefix{x}{y}{(\outputp{x}{y} | @{y})) | P}}\rangle 
	      | \prefix{x}{y}{(\outputp{x}{y} | @{y})} & \nonumber\\
	\red
	& (\outputp{x}{y} | @{y})\substn{\quotep{(\prefix{x}{y}{(@{y} | \outputp{x}{y})) | P}}}{y} & \nonumber\\
	=
	& \outputp{x}{\quotep{(\prefix{x}{y}{(\outputp{x}{y} | @{y})) | P}}}
	  | {(\prefix{x}{y}{(\outputp{x}{y} | @{y})) | P}} & \nonumber\\
	\red
	& \ldots & \nonumber\\
	\red^*
	& P | P | \ldots & \nonumber
\end{eqnarray}

Of course, this encoding, as an implementation, runs away, unfolding
$\bangp{P}$ eagerly. A lazier and more implementable replication
operator, restricted to input-guarded processes, may be obtained as follows.

\begin{eqnarray}
\bangp{\prefix{u}{v}{P}} 
	:= 
	\binpar{\lift{x}{\prefix{u}{v}{(\binpar{D(x)}{P})}}}{D(x)} \nonumber
\end{eqnarray}

\begin{remark}
  Note that the lazier definition still does not deal with summation
  or mixed summation (i.e. sums over input and output). The reader is
  invited to construct definitions of replication that deal with these
  features. 

  Further, the definitions are parameterized in a name, $x$. Can you,
  gentle reader, make a definition that eliminates this parameter and
  guarantees no accidental interaction between the replication
  machinery and the process being replicated -- i.e. no accidental
  sharing of names used by the process to get its work done and the
  name(s) used by the replication to effect copying. This latter
  revision of the definition of replication is crucial to obtaining
  the expected identity $!!P \sim !P$.
\end{remark}

\begin{remark}\label{rem:paradoxical_combinator}
  The reader familiar with the lambda calculus will have noticed the
  similarity between $D$ and the paradoxical combinator.

  [Ed. note: the existence of this seems to suggest we have to be more
  restrictive on the set of processes and names we admit if we are to
  support no-cloning.]
\end{remark}

\subsubsection{Bisimulation}

The computational dynamics gives rise to another kind of equivalence,
the equivalence of computational behavior. As previously mentioned
this is typically captured \emph{via} some form of bisimulation.

% The notion we use in this paper is weak barbed bisimulation
% \cite{milner91polyadicpi}.

The notion we use in this paper is derived from weak barbed
bisimulation \cite{milner91polyadicpi}. 

\begin{definition}
An \emph{observation relation}, $\downarrow_{\mathcal N}$, over a set
of names, $\mathcal N$, is the smallest relation satisfying the rules
below.

\infrule[Out-barb]{y \in {\mathcal N}, \; x \nameeq y}
		  {\outputp{x}{v} \downarrow_{\mathcal N} x}
\infrule[Par-barb]{\mbox{$P\downarrow_{\mathcal N} x$ or $Q\downarrow_{\mathcal N} x$}}
		  {\binpar{P}{Q} \downarrow_{\mathcal N} x}

We write $P \Downarrow_{\mathcal N} x$ if there is $Q$ such that 
$P \wred Q$ and $Q \downarrow_{\mathcal N} x$.
\end{definition}

\begin{definition}
%\label{def.bbisim}
An  ${\mathcal N}$-\emph{barbed bisimulation} over a set of names, ${\mathcal N}$, is a symmetric binary relation 
${\mathcal S}_{\mathcal N}$ between agents such that $P\rel{S}_{\mathcal N}Q$ implies:
\begin{enumerate}
\item If $P \red P'$ then $Q \wred Q'$ and $P'\rel{S}_{\mathcal N} Q'$.
\item If $P\downarrow_{\mathcal N} x$, then $Q\Downarrow_{\mathcal N} x$.
\end{enumerate}
$P$ is ${\mathcal N}$-barbed bisimilar to $Q$, written
$P \wbbisim_{\mathcal N} Q$, if $P \rel{S}_{\mathcal N} Q$ for some ${\mathcal N}$-barbed bisimulation ${\mathcal S}_{\mathcal N}$.
\end{definition}

$\mathcal{R} \subseteq \pi \times \pi$

$P \mathcal{R} Q => \forall P'. P \red P' \Rightarrow \exists Q'. Q \red Q', P' \mathcal{R} Q'$

$P \vdash x \Rightarrow Q \vdash x$

\begin{mathpar}
  \inferrule*[lab=Out-barb]{x \nameeq y}{{y}!\langle{Q}\rangle \vdash x}
  \and
  \inferrule*[lab=Par-barb]{\mbox{$P\vdash x$ or $Q\vdash x$}}{\binpar{P}{Q} \vdash x}
\end{mathpar}

\subsubsection{Contexts}

One of the principle advantages of computational calculi like the
$\pi$-calculus is a well-defined notion of context,
contextual-equivalence and a correlation between
contextual-equivalence and notions of bisimulation. The notion of
context allows the decomposition of a process into (sub-)process and
its syntactic environment, its context. Thus, a context may be
thought of as a process with a ``hole'' (written $\Box$) in it. The
application of a context $M$ to a process $P$, written $M[P]$, is
tantamount to filling the hole in $M$ with $P$. In this paper we do
not need the full weight of this theory, but do make use of the notion
of context in the proof the main theorem. 

\begin{mathpar}
  \inferrule* [lab=summation] {} {{M_{M},M_{N}} \bc \Box \;|\; x.M_{A} \;|\; M_{M}+M_{N}}
  \and
  \inferrule* [lab=agent] {} {{M_{A}} \bc (\vec{x})M_{P} \;| \; \clift{P_0,\ldots,M_{P},\ldots,P_N}}
  \and \\
  \inferrule* [lab=process] {} {{M_{P}} \bc M_{N} \;| \;P|M_{P} }
\end{mathpar} 

\begin{mathpar}
  \inferrule* [lab=sychronization] {} {M_{N} \bc \Box \;|\; x?M_{F} \;|\; x!M_{C}}
  \and
  \inferrule* [lab=abstraction] {} {{M_{F}} \bc (x)M_{P} }
  \and
  \inferrule* [lab=concretion] {} {{M_{C}} \bc \langle M_{P} \rangle }
  \and \\
  \inferrule* [lab=process] {} {{M_{P}} \bc M_{N} \;| \;P|M_{P} }
\end{mathpar}

\begin{definition}[contextual application] Given a context $M$, and
  process $P$, we define the \emph{contextual application}, $M[P] :=
  M\{P/\Box\}$. That is, the contextual application of M to P is the
  substitution of $P$ for $\Box$ in $M$.
\end{definition}

$\meaningof{-} : L \to \mathcal{P}(\pi)$

\begin{mathpar}
  \inferrule* [lab=collection] {} {\meaningof{true} = \pi, \and \meaningof{~E} = \pi \setminus \meaningof{E}, \and \meaningof{E_{1} \& E_{2}} = \meaningof{E_{1}} \cap \meaningof{E_{2}}}
\end{mathpar}

\begin{mathpar}
  \inferrule* [lab=structure] {} {\meaningof{0} = \{ P \in \pi | P \equiv 0 \}, \and \\ \meaningof{E_1 | E_2} = \{ P \in \pi | P \equiv P_{1} | P_{2}, P_{1} \in \meaningof{E_{1}}, P_{2} \in \meaningof{E_2}\} }
\end{mathpar}

\begin{mathpar}
 \inferrule* [lab=behavior] {} {\meaningof{\langle a?b \rangle E} = \{ P \in \pi | P \equiv Q | u?(y)P', \\ \and \\\\ \and \\ \;\;\; u \in \meaningof{a}, \forall z.P'\{z/y\} \in \meaningof{E\{z/b\}}\}, \and \\ \meaningof{a!E} = \{ P \in \pi | P \equiv Q | x!\langle P' \rangle, x \in \meaningof{a} P' \in \meaningof{E}\} }
\end{mathpar}

\begin{mathpar}
 \inferrule* [lab=nominal] {} {\meaningof{\quotep{E}} = \{ \quotep{P} \in \quotep{\pi} | P \in \meaningof{E} \}, \and \meaningof{\quotep{P}} = \{ \quotep{Q} \in \quotep{\pi} | P \equiv Q \} \and \\ \meaningof{@\quotep{E}} = \{ P \in \pi | P \equiv @x, x \in \meaningof{E} \}}
\end{mathpar}

\begin{eqnarray*}
  \\
  \meaningof{-} : TS \to ST
\end{eqnarray*}

\begin{eqnarray*}
  \\
  L : TS \to ST
\end{eqnarray*}

\begin{eqnarray*}
  \\
  P \models E \iff P \in \meaningof{E}
\end{eqnarray*}

\begin{eqnarray*}
  P \approx_{L} Q \iff \forall E \in L. P \models E \iff Q \models E
\end{eqnarray*}

\begin{eqnarray*}
  P \approx_{K} Q
\end{eqnarray*}

\begin{eqnarray*}
  P \approx Q
\end{eqnarray*}

$\approx_{K} = \approx = \approx_{L}$

\subsubsection{Contextual duality}

Note that contexts extend the quotation operation to a family of
operations from processes to names. Given a context, $M$, we can
define a \emph{nominal context}, $\quotep{M}$ by $\quotep{M}[P] :=
\quotep{M[P]}$. To foreshadow what is to come we observe that these
operations enjoy a duality with processes very much like the duality
between vectors and maps from vectors to scalars.

Further, because the calculus is essentially higher-order, we have a
correspondence between contexts and processes. More specifically,
given a name $x$ and a context $M$ we can construct $M^{*}_{x}$ such
that 

\begin{mathpar}
  M^{*}_{x} | \lift{x}{P} \red M[P]
\end{mathpar}

namely,

\begin{mathpar}
  M^{*}_{x} := x?(u).M[\dropn{u}]
\end{mathpar}

The dependence of $M^{*}_{x}$ on a name makes it an abstraction, 

\begin{mathpar}
  M^{*} := (x)x?(u).M[\dropn{u}]
\end{mathpar}

\subsection{Additional notation}

It will sometimes be convenient to denote the process a name
quotes. We already have the notation $x = \quotep{P}$, but it will be
convenient to introduce an alternate notation, $\procn{x}$, when we
want to emphasize the connection to the use of the name. Note that, by
virtue of name equivalence, $\quotep{\procn{x}} \nameeq x$; so, the
notation is consistent with previous definitions.

Further, because names have structure it is possible to effect
substitutions on the basis of that structure. This means we need to
upgrade our notation for substitutions, which we accomplish by
adapting comprehension notation. Thus,

\begin{mathpar}
  P\{ y / x : x \in S \}
\end{mathpar}

is interpreted to mean the process derived from P by replacing (in a
capture-avoiding manner) each occurrence of $x$ in $S$ by $y$. For example,

\begin{mathpar}
  P\{ \quotep{\procn{x}|\procn{x}} / x : x \in \freenames{P} \}
\end{mathpar}

will replace each (occurrence) of a free name $x$ in $P$ by
$\quotep{\procn{x}|\procn{x}}$.

Also, we will avail ourselves of the notation $x^{L}$ and $x^{R}$ to
denote injections of a name into disjoint copies of the name
space. There are numerous ways to accomplish this. One example can be
found in \cite{MeredithR05}. This notation overloads to vectors of
names: $\vec{x}^{\pi} := (x_{i}^{\pi} \; : \; 0 \leq i < |\vec{x}| )$ where $\pi \in \{L,R\}$.

We also use $P^{\Box} := P|\Box$.

In \cite{MeredithR05} an interpretation of the new operator is
given. It turns out that there are several possible interpretations
all enjoying the requisite algebraic properties of the operator (see
\cite{milner91polyadicpi}). We will therefore make liberal use of
$(\nu\; \vec{x})P$.

% subsection the_syntax_and_semantics_of_the_notation_system (end)   

\section{Interpretation of QM}
\subsection{Supporting definitions}
\subsubsection{Multiplication}
\begin{mathpar}
  \quotep{Q} \cdot \quotep{R} := \quotep{Q|R}
  \and \\
  \quotep{Q} \cdot P := P\{ \quotep{Q|R} / \quotep{R} : \quotep{R} \in \freenames{P} \}
\end{mathpar}

\paragraph{Discussion}
The first line needs little explanation. The second line says that
each free name of the process is replaced with the multiplication of
that name by the scalar. Multiplication of a scalar (name) by a state
(process) results in a process all the names of which have been `moved
over' by parallel composition with the process the scalar
quotes. There is a subtlety that the bound names have to be
manipulated so that multiplied names aren't accidentally
captured. There are many ways to achieve this.

\begin{remark}\label{rem:multiplication_identities}
  The reader is invited to verify that for all $x,y,z \in \QProc$ and $P \in \Proc$
  \begin{mathpar}
    x \cdot \quotep{0} \equiv x 
    \and
    x \cdot y \equiv y \cdot x
    \and
    x \cdot (y \cdot z) \equiv (x \cdot y) \cdot z
    \and \\
    \quotep{0} \cdot P \equiv P
    \and \\
    x \cdot (y \cdot P) \equiv (x \cdot y) \cdot P
    \and \\
    x \cdot (P|Q) \equiv (x \cdot P) | (x \cdot Q)
    \and \\    
  \end{mathpar}
\end{remark}

\subsubsection{Tensor product}

We define a tensor product on processes by structural induction.

\paragraph{Tensor of sums} First note that all summations, including
$\pzero$ and sequence, can be written $\Sigma_{i} x_{i}.A_{i} +
\Sigma_{j} x_{j}.C_{j}$, where we have grouped input-guarded processes
together and output-guarded processes together.

Thus, we can define the tensor product of two summations, $N_{1}\otimes N_{2}$, where

\begin{mathpar}
  N_{1} := \Sigma_{i} x_{i}.A_{i} + \Sigma_{j} x_{j}.C_{j}
  \and
  N_{2} := \Sigma_{i'} y_{i'}.B_{i'} + \Sigma_{j'} y_{j'}.D_{j'} 
\end{mathpar}

as follows.

\begin{mathpar}
  \Sigma_{i} x_{i}.A_{i} + \Sigma_{j} x_{j}.C_{j} \otimes \Sigma_{i'}
  y_{i'}.B_{i'} + \Sigma_{j'} y_{j'}.D_{j'} 
  \and \\
  := \; \Sigma_{i} \Sigma_{i'} \quotep{\stackrel{\vee}{x_{i}}| \stackrel{\vee}{y_{i'}}}.(A_{i}\otimes B_{i'}) \; | \; \Sigma_{i'} \Sigma_{i} \quotep{\stackrel{\vee}{y_{i'}}|\stackrel{\vee}{x_{i}}}.(B_{i'}\otimes A_{i})
  \and
  \;\; | \;\; \Sigma_{j} \Sigma_{j'} \quotep{\stackrel{\vee}{x_{j}}|\stackrel{\vee}{y_{j'}}}.(A_{j}\otimes B_{j'}) \; | \; \Sigma_{j'} \Sigma_{j} \quotep{\stackrel{\vee}{y_{j'}}|\stackrel{\vee}{x_{j}}}.(B_{j'}\otimes A_{j})
\end{mathpar}

\begin{remark}
  Do we need to $x^{L}$ and $y^{R}$ for this construction as well?
\end{remark}

\paragraph{Tensor of parallel compositions} Next, we distribute tensor
over par.

\begin{mathpar}
  P_{1}|P_{2} \otimes Q_{1}|Q_{2} := (P_{1} \otimes Q_{1}) | (P_{1}
  \otimes Q_{2}) | (P_{2} \otimes Q_{1}) | (P_{2} \otimes Q_{2})
\end{mathpar}

\paragraph{Tensor with dropped names} We treat tensor of a
process with a dropped name as parallel composition.

\begin{mathpar}
  P \otimes \dropn{x} := P | \dropn{x}
\end{mathpar}

\paragraph{Tensor of agents}

Finally, we need to define tensor on agents. Note that the definition
of tensor on normal products only tensors inputs with inputs and
outputs with outputs. Thus, we only have to define the operation on
``homogeneous'' pairings.

\begin{mathpar}
  (\vec{x})P \otimes (\vec{y})Q
  \and \\
  := (x_{0}^{L}|y_{0}^{R},\ldots,x_{0}^{L}|y_{n}^{R},\ldots,x_{m}^{L}|y_{0}^{R},\ldots,x_{m}^{L}|y_{n}^R)(P\{ \vec{x}^{L}/\vec{x}\} \otimes Q \{ \vec{y}^{R}/\vec{y}\})
  \and \\
  \clift{\vec{P}} \otimes \clift{\vec{Q}}
  \and \\
  := \clift{P_{0}\otimes Q_{0},\ldots,P_{0}\otimes Q_{n},\ldots,P_{m}\otimes Q_{0},\ldots,P_{m}\otimes Q_{n}}
\end{mathpar}

\begin{remark}
  Observe that arities of tensored abstractions matches arities of
  tensored concretions if the original arities matched. Note also that
  the length of the arities corresponds to the increase in dimension
  we see in ordinary vector space tensor product.
\end{remark}

\begin{remark}
  Operationally, this definition distributes the tensor down to
  components ``linked'' by summation. Tensor over summation is
  intriguing in that it mixes names. Moreover, as a consequence of the
  way it mixes names we have the identities for all $x \in \QProc$ and
  $P,Q \in \Proc$

  \begin{mathpar}
    (x \cdot P) \otimes Q \equiv x \cdot (P \otimes Q) \equiv P \otimes (x \cdot Q)
    \and
    P \otimes \pzero \equiv P
  \end{mathpar}

  that the reader is invited to verify.
\end{remark}

\subsubsection{Annihilation}
\begin{mathpar}
  P^{\perp} := \{ Q | \forall R. P|Q \red^{*} R \Rightarrow R \red^{*} \pzero \}
  \and \\
  P^{\underline{\perp}} := \Sigma_{Q \in P^{\perp}} \quotep{Q}?(y).(\dropn{y}|Q) | \Sigma_{Q \in P^{\perp}} \quotep{Q}\clift{\Box}
\end{mathpar}

\paragraph{Discussion} The reader will note that $P^{\perp}$ is a
\emph{set} of processes, while $P^{\underline{\perp}}$ is a
\emph{context}. We call the set $P^{\perp}$ the \emph{annihilators} of
$P$. The parallel composition of a process in the annihilators of $P$
with $P$ will result in a process, the state space of which has all
paths eventually leading to $\pzero$. Execution may endure loops; but
under reasonable conditions of fairness (naturally guaranteed under
most notions of bisimulation) such a composite process cannot get
stuck in such a loop and will, eventually pop out and terminate.

The context $P^{\underline{\perp}}$ is ready and willing to ``take the
$P$ out of'' the process to which it is applied. It will effectively
transmit the code of the process to which it is applied to one of the
annihilators and run the process against it.

\subsubsection{Evaluation}
We fix $M$ a domain of fully abstract interpretation with an equality
coincident with bisimulation. We take $\meaningof{\cdot} : \Proc \to
M$ to be the map interpreting processes and $\nmeaningof{\cdot} : \M
\to Proc$ to be the map running the other way. Then we define

\begin{mathpar}
  \int P := \nmeaningof{\meaningof{P}}
\end{mathpar}

\paragraph{Discussion}
There are many fully abstract interpretations of Milner's
$\pi$-calculus. Any of them can be used as a basis for interpreting
the reflective calculus here. Equipped with such a domain it is
largely a matter of grinding through to check that the Yoneda
construction for the normalization-by-evaluation program can be
extended to this setting.

\begin{remark}
  The reader is invited to verify that $\int (P^{\underline{\perp}}[P]) = 0$.
\end{remark}

\subsection{Quantum mechanics}

Table \ref{tbl:core_qm_op_defns} gives the core operational definitions

\begin{table}[htp]\label{tbl:core_qm_op_defns}
  \center{
    \fbox{
      \begin{tabular}{c|c}
        quantum mechanics & process calculus \\
        \hline
        scalar & $x := \quotep{P}$ \\
        state vector & $\state{P} := P$ \\
        dual & $\state{P}^{*} := \event{P^{\underline{\perp}}} := \quotep{P^{\underline{\perp}}}[-]$ \\
        matrix & $ \Sigma_{\alpha} \state{P_{\alpha}}x_{\alpha}\event{Q_{\alpha}}$ \\
        vector addition & $\state{P} + \state{Q} := \state{P | Q}$ \\
        tensor product & $\state{P} \otimes \state{Q} := \state{P \otimes Q}$ \\
        inner product & $\innerprod{P}{Q} := \quotep{\int P^{\underline{\perp}}[Q]}$ \\
      \end{tabular}
    }
  }
  \caption{QM - operational definitions}
\end{table}

where

\begin{mathpar}
  \prmatrix{P}{Q} := \fprmatrix{P}{\quotep{\pzero}}{Q}
  \and
  \fprmatrix{P}{x}{Q} := (\state{P},x,\event{Q})
  \and
  (\fprmatrix{P}{x}{Q})(\state{R}) := x \cdot \innerprod{Q}{R} \cdot \state{P}
  \and
  (\fprmatrix{P}{x}{Q})(\event{R}) := x \cdot \innerprod{R}{P} \cdot \event{Q}
\end{mathpar}

\paragraph{Discussion}
As promised: vectors (aka states) are represented as processes; duals
as contextual duals; inner product definition should be compared with
standard inner product definition for ....

\begin{remark}
  Assuming $\int (P^{\underline{\perp}}[P]) = 0$, the reader is
  invited to verify that $(\fprmatrix{P}{x}{P})(\state{P}) = x \cdot \state{P}$.
\end{remark}

\begin{remark}
  The reader is invited to verify that $\innerprod{P}{Q}$ could
  equally well have been written $\quotep{\int \stackrel{\vee}{x}}$
  where $x = \event{P^{\underline{\perp}}}(Q)$.

  One of the motivations for this remark is that there is another way
  to factor these operations. We could package up evaluation in the dual:

  \begin{mathpar}
    \state{P}^{*} := \event{\int P^{\underline{\perp}}} := \quotep{\int P^{\underline{\perp}}}[-]
  \end{mathpar}

  and then have inner product defined by
  
  \begin{mathpar}
    \innerprod{P}{Q} := \event{P}(Q)
  \end{mathpar}

  Hopefully, experience with the calculations will provide guidance on
  the best factoring.
\end{remark}

\begin{remark}
  Assuming $\int (P^{\underline{\perp}}[P]) = 0$, the reader is
  invited to verify that $\forall P,Q. (\prmatrix{0}{Q})(\state{0}) =
  \state{0}$ and dually $(\prmatrix{P}{0})(\event{0}) = \event{0}$.
\end{remark}

\begin{remark}
  i'm a little worried that i don't (yet) have proper support for
  complex conjugacy. But, the observation above may give us a
  clue. According to Abramsky, it must be the case that the scalars
  are iso to the homset of the identity for the tensor -- which the
  observation above characterizes. 

  For now, we will simply bookmark the notion with $\overline{x}$.
\end{remark}

\subsubsection{Adjointness}

We need to give a definition of $(\cdot)^{\dagger}$ for matrices. The
obvious candidate definition is
\begin{mathpar}
(\Sigma_{\alpha}\fprmatrix{P_{\alpha}}{x_{\alpha}}{Q_{\alpha}})^{\dagger}
= \Sigma_{\alpha}\fprmatrix{(Q_{\alpha}^{\underline{\perp}})^{*}}{\overline{x}_{\alpha}}{P_{\alpha}^{\underline{\perp}}} 
\end{mathpar}

But, $(Q_{\alpha}^{\underline{\perp}})^{*}$ requires a name along
which to communicate the process to achieve the context application.

\subsubsection{Basis for a basis}
If processes label states and ``addition'' of states (a.k.a. vector
addition) is interpreted as parallel composition, what corresponds to
notions of linear independence and basis? Here, we recall that Yoshida
has developed a set of \emph{combinators} for an asynchronous verison
of Milner's $\pi$-calculus. These are a finite set of processes such
any process can be expressed as parallel composition of these
combinators together with liberal uses of the new operator and
replication. We can simply give a translation of these into the
present calculus and have reasonable expectation that the property
carries over. That is, that the resultant set allows to express all
processes via parallel composition. Note, however, that there is no
new operator or replication in this calculus. As a result, we expect
that the corresponding set is actually infinite. That is, we expect
that the space is actually infinite dimensional.

\begin{remark}
  The attentive reader may be a bit concerned. Certainly, the
  collection $S$, $K$ and $I$ is a finite set of
  combinators. Shouldn't we expect to see a finite set of combinators
  for an effectively equivalent system? i am very sympathetic to this
  critique and feel it warrants full attention. On the other hand, i
  also have in mind the following analogy. The natural numbers, as a
  monoid under addition, has exactly $1$ generator, while the natural
  numbers, as a monoid under multiplication, has countably many
  generators (the primes). We observe that the application of the
  lambda calculus is much less resource sensitive than the parallel
  composition of the $\pi$-calculus. Could it be the case that we have
  an analogy of the form
  
  \begin{mathpar}
    m + n : MN :: m*n : M|N
  \end{mathpar}

  giving a similar blow up in the set of ``primes''?  This is such a
  wonderful thought that, even if it's not true, i think it's worth
  writing down.
\end{remark}
 

\documentclass[12pt]{llncs}
%\documentclass{jktr}

\usepackage[pdftex]{hyperref}                   
\usepackage {listings}
\usepackage {mathpartir}
\usepackage{bcprules}
%\usepackage{listings}
                       
\usepackage{graphicx} 
%\usepackage[margins=2.5cm,nohead,nofoot]{geometry}
%\usepackage{geometry}
\usepackage{amsfonts}
\usepackage{amstext}
\usepackage{latexsym}
\usepackage{amssymb}
\usepackage{color}


%\include{myPreamble}
\include{qm2pi.local} 

%\ifpdf
%\usepackage[pdftex]{graphicx}
%\else
%\usepackage{graphicx}
%\fi

 % \ifpdf
%  \usepackage{pdfsync}
%  \if


%\title{Brief Article}
%\author{David F. Snyder}
%\author{L.G. Meredith}

%\address{Dept. of Math., Texas State University--San Marcos, San Marcos, TX 78666}
       
\pagestyle{empty}


\begin{document}

\lstset{language=[Objective]Caml,frame=shadowbox}

\input{qm2pi.front}

% section front matter (end)

\input{qm2pi.intro} 
 
% section introduction (end)

% \input{qm2pi.knotations} 

% section notation (end)

\input{qm2pi.process.calculi} 

% section concurrent_process_calculi_and_spatial_logics_ (end)
    
%\input{qm2pi.knots2pi} 

%\input{qm2pi.trefoil} 

%\input{qm2pi.mainthm} 

% subsection basic_interpretation (end)

%\input{qm2pi.rho.presentation} 
\subsection{The syntax and semantics of the notation system}\label{sub:the_syntax_and_semantics_of_the_notation_system} % (fold)

We now summarize a technical presentation of the calculus that
embodies our theory of dynamics. The typical presentation of such a
calculus follows the style of giving generators and relations on
them. The grammar, below, describing term constructors, freely
generates the set of processes, $\Proc$. This set is then quotiented
by a relation known as structural congruence and it is over this set
that the notion of dynamics is expressed. This presentation is
essentially that of \cite{MeredithR05} with the addition of
polyadicity and summation. For readability we have relegated some of
the technical subtleties to an appendix.

\subsubsection{Process grammar}\label{subsub:process_grammar}

\begin{mathpar}
  \inferrule* [lab=synchronization] {} {{M} \bc \pzero \;|\; x?F \;|\; x!C }
  \and
  \inferrule* [lab=abstraction] {} {{F} \bc (x)P}
  \and
  \inferrule* [lab=concretion] {} {{C} \bc \langle Q \rangle}
  \and
  \inferrule* [lab=process] {} {{P,Q} \bc M \;| \;P|Q \;|\; @{x}}
  \and
  \inferrule* [lab=name] {} {{x} \bc \quotep{P}}
\end{mathpar} 

Note that $\vec{x}$ (resp. $\vec{P}$) denotes a vector of names
(resp. processes) of length $|\vec{x}|$ (resp. $|\vec{P}|$). We adopt
the following useful abbreviations.

\begin{mathpar}
   x?(\vec{y}).P := x.(\vec{y})P \and  x\clift{\vec{P}} := x.\clift{\vec{P}}
   \and x!(y) := \lift{x}{\dropn{y}}
   \and \Pi_{i=0}^{n-1}P_i := P_0 | \ldots | P_{n-1}
\end{mathpar}

\subsubsection{Structural congruence}

\paragraph{Free and bound names and alpha-equivalence.} At the
core of structural equivalence is alpha-equivalence which identifies
process that are the same up to a change of variable. Formally, we
recognize the distinction between free and bound names. The free names
of a process, $\freenames{P}$, may be calculated recursively as
follows:

\begin{mathpar}
\freenames{\pzero} := \emptyset
  \and \\
  \freenames{x?(y).P} := \{ x \} \cup (\freenames{P} \setminus \{ y \})
  \and 
  \freenames{x!\langle P \rangle} := \{ x \} \cup \{ P \} 
  \and \\
  \freenames{P|Q} := \freenames{P} \cup \freenames{Q}
  \and \\
  \freenames{@{x}} := \{ x \}
\end{mathpar}

$\pi$
$\quotep{\pi}$

$\freenames{-} : \pi \to \mathcal{P}(\quotep{\pi})$

\begin{eqnarray*}
  \freenames{\pzero} & := & \emptyset \\
  \freenames{x?(y).P} & := & \{ x \} \cup (\freenames{P} \setminus \{ y \}) \\
  \freenames{x!\langle P \rangle} & := & \{ x \} \cup \{ P \} \\
  \freenames{P|Q} & := & \freenames{P} \cup \freenames{Q} \\
  \freenames{\dropn{x}} & := & \{ x \}
\end{eqnarray*}

The bound names of a process, $\boundnames{P}$, are those names occurring in $P$
that are not free. For example, in $x?(y).0$, the name $x$ is free, while $y$ is bound.

\begin{mathpar}
  \inferrule* [lab=monoidal-laws] {} { P|Q \equiv Q|P \and P|0 \equiv P \and P|(Q|R) \equiv (P|Q)|R }
\end{mathpar}

\begin{mathpar}
  \inferrule* [lab=alpha-equivalence] {} { (x)P \equiv (y)P\{y/x\} \and y \not\in \freenames{P} }
\end{mathpar}

\begin{definition}
Then two processes, $P,Q$, are alpha-equivalent if $P = Q\{\vec{y}/\vec{x}\}$ for
some $\vec{x} \in \boundnames{Q},\vec{y} \in \boundnames{P}$, where $Q\{\vec{y}/\vec{x}\}$
denotes the capture-avoiding substitution of $\vec{y}$ for $\vec{x}$ in $Q$.
\end{definition}

\begin{definition}
  The {\em structural congruence} \cite{SangiorgiWalker} , $\equiv$,
  between processes is the least congruence containing
  alpha-equivalence, satisfying the abelian monoid laws
  (associativity, commutativity and $\pzero$ as identity) for parallel
  composition $|$ and for summation $+$.
\end{definition}

\subsection{Name equivalence}

We take name equivalence, written $\nameeq$, to be the smallest
equivalence relation generated by the following rules.

\begin{mathpar}
\inferrule*[lab=Quote-drop]
{ }
{ \quotep{@{x}} \nameeq x }

\inferrule*[lab=Struct-equiv]
{ P \scong Q }
{ \quotep{P} \nameeq \quotep{Q} }
\end{mathpar}

The astute reader will have noticed that the mutual recursion of names
and processes imposes a mutual recursion on alpha-equivalence and
structural equivalence via name-equivalence. Fortunately, all of this
works out pleasantly and we may calculate in the natural way, free of
concern. The reader interested in the details is referred to the
appendix \ref{appendix:rho_details}.

\subsection{Substitution}

We use $\Proc$ for the set of processes, $\QProc$ for the set of
names, and $\id{\{}\vec{y} / \vec{x} \id{\}}$ to denote partial maps,
$s : \QProc \rightarrow \QProc$. A map, $s$ lifts, uniquely, to a map
on process terms, $\widehat{s} : \Proc \rightarrow \Proc$ by the
following equations.

\begin{mathpar}
  (0) \psubstp{Q}{P} := 0 \\
  (R \juxtap S) \psubstp{Q}{P}
  :=    
  (R)\psubstp{Q}{P} \juxtap (S) \psubstp{Q}{P} \\
  (x?(y).R) \psubstp{Q}{P}    
  :=    
  (x)\substp{Q}{P} (z)\concat( (R \psubstn{z}{y}) \psubstp{Q}{P} ) \\
  (\lift{x}{R}) \psubstp{Q}{P}  
  :=
  \lift{(x)\substp{Q}{P}}{ R \psubstp{Q}{P} } \\
%   (\dropn{x})  \psubstp{Q}{P}       
%   := 
%   \left\{ 
%     \begin{array}{ccc} 
%       \dropn{\quotep{Q}} & & x \nameeq \quotep{P} \\
%       \dropn{x} & & otherwise \\
%     \end{array}
%   \right. 
  (\dropn{x})  \psubstp{Q}{P}       
  := 
  \left\{ 
    \begin{array}{ccc} 
      Q & & x \nameeq \quotep{P} \\
      \dropn{x} & & otherwise \\
    \end{array}
  \right.
\end{mathpar}
 

where

\begin{eqnarray}
  (x)\id{\{} \lpquote Q \rpquote / \lpquote P \rpquote \id{\}}            = 
  \left\{ 
    \begin{array}{ccc}
      \lpquote Q \rpquote & & x \nameeq \lpquote P \rpquote \\
      x & & otherwise \\
    \end{array}
  \right. \nonumber
\end{eqnarray}

and $z$ is chosen distinct from $\quotep{P}$, $\quotep{Q}$, the free
names in $Q$, and all the names in $R$. Our $\alpha$-equivalence will
be built in the standard way from this substitution.

\begin{remark}\label{rem:no_self_referential_names}
  One consequence of these definitions is that $\forall P. \quotep{P}
  \not\in \freenames{P}$.
\end{remark}

\subsection{ Dynamic quote: an example }

Anticipating something of what's to come, consider applying the
substitution, $\widehat{\id{\{}u / z \id{\}}}$, to the following pair
of processes, $\lift{w}{y!(z)}$ and $w[ \lpquote y!(z) \rpquote ]$.

\begin{eqnarray}
	\lift{w}{y!(z)}\widehat{\id{\{}u / z \id{\}}}
		& = &
		\lift{w}{y!(u)} \nonumber\\
	w[ \lpquote y!(z) \rpquote ] \widehat{ \id{\{}u / z \id{\}} }
		& = &
		w[ \lpquote y!(z) \rpquote ] \nonumber
\end{eqnarray}

Because the body of the process between quotes is impervious to
substitution, we get radically different answers. In fact, by
examining the first process in an input context,
e.g. $x?(z).\lift{w}{y!(z)}$, we see that the process under the lift
operator may be shaped by prefixed inputs binding a name inside it. In
this sense, the lift operator will be seen as a way to dynamically
construct processes before reifying them as names.

Finally equipped with these standard features we can present the
dynamics of the calculus.

\subsubsection{Operational semantics} 

Finally, we introduce the computational dynamics. What marks these
algebras as distinct from other more traditionally studied algebraic
structures, e.g. vector spaces or polynomial rings, is the manner in
which dynamics is captured. In traditional structures, dynamics is typically
expressed through morphisms between such structures, as in linear maps
between vector spaces or morphisms between rings. In algebras
associated with the semantics of computation, the dynamics is
expressed as part of the algebraic structure itself, through a
reduction reduction relation typically denoted by $\red$. Below, we
give a recursive presentation of this relation for the calculus used
in the encoding.

$\red \subseteq \pi \times \pi$
$\red : \pi \to \mathcal{P}(\pi)$

\begin{mathpar}
  \inferrule* [lab=Comm] { \textsf{match}( x_{src}, x_{trgt} ) } { x_{trgt}?(y)P \; | \; x_{src}!\langle {Q} \rangle \red P\{\quotep{Q}/y}\} }
  \and \\
  \inferrule* [lab=Par] {{P} \red {P}'} {{{P} | {Q}} \red {{P}' | {Q}}}
  \and
  \inferrule* [lab=Equiv]{{{P} \scong {P}'} \andalso {{P}' \red {Q}'} \andalso {{Q}' \scong {Q}}}{{P} \red {Q}}
\end{mathpar}

\begin{eqnarray*}
  match_{\equiv} (\quotep{P},\quotep{Q}) & := & P \equiv Q \\
  match_{\dagger}(\quotep{P},\quotep{Q}) & := & \forall R. P|Q \red^{*} R => R \red^{*} 0 \\
  match_{K}(\quotep{P},\quotep{Q}) & := & K \mbox{ for some context } K
\end{eqnarray*}

$u?(x)P | u!\langle Q \rangle \red P\{\quotep{Q}/x\}$

%We write $\wred$ for $\red^*$, and $P\red$ if $\exists Q $ such that $ P \red Q$.
We write $P\red$ if $\exists Q $ such that $ P \red Q$ and $P\not\red$, otherwise.

\section{Replication}

As mentioned before, it is known that replication (and hence
recursion) can be implemented in a higher-order process algebra
\cite{SangiorgiWalker}. As our first example of calculation with the
machinery thus far presented we give the construction explicitly in
the {\rhoc}.

\begin{eqnarray}
	D_{x} & := & \prefix{x}{y}{(\binpar{\outputp{x}{y}}{@{y}})} \nonumber\\
	\bangp_{x}{P} & := & \binpar{{x}!\langle{\binpar{D_{x}}{P}}\rangle}{D_{x}} \nonumber
\end{eqnarray}

\begin{eqnarray}
	\bangp_{x}{P} & & \nonumber\\
	=
	& {x}!\langle{(\prefix{x}{y}{(\outputp{x}{y} | @{y})) | P}}\rangle 
	      | \prefix{x}{y}{(\outputp{x}{y} | @{y})} & \nonumber\\
	\red
	& (\outputp{x}{y} | @{y})\substn{\quotep{(\prefix{x}{y}{(@{y} | \outputp{x}{y})) | P}}}{y} & \nonumber\\
	=
	& \outputp{x}{\quotep{(\prefix{x}{y}{(\outputp{x}{y} | @{y})) | P}}}
	  | {(\prefix{x}{y}{(\outputp{x}{y} | @{y})) | P}} & \nonumber\\
	\red
	& \ldots & \nonumber\\
	\red^*
	& P | P | \ldots & \nonumber
\end{eqnarray}

Of course, this encoding, as an implementation, runs away, unfolding
$\bangp{P}$ eagerly. A lazier and more implementable replication
operator, restricted to input-guarded processes, may be obtained as follows.

\begin{eqnarray}
\bangp{\prefix{u}{v}{P}} 
	:= 
	\binpar{\lift{x}{\prefix{u}{v}{(\binpar{D(x)}{P})}}}{D(x)} \nonumber
\end{eqnarray}

\begin{remark}
  Note that the lazier definition still does not deal with summation
  or mixed summation (i.e. sums over input and output). The reader is
  invited to construct definitions of replication that deal with these
  features. 

  Further, the definitions are parameterized in a name, $x$. Can you,
  gentle reader, make a definition that eliminates this parameter and
  guarantees no accidental interaction between the replication
  machinery and the process being replicated -- i.e. no accidental
  sharing of names used by the process to get its work done and the
  name(s) used by the replication to effect copying. This latter
  revision of the definition of replication is crucial to obtaining
  the expected identity $!!P \sim !P$.
\end{remark}

\begin{remark}\label{rem:paradoxical_combinator}
  The reader familiar with the lambda calculus will have noticed the
  similarity between $D$ and the paradoxical combinator.

  [Ed. note: the existence of this seems to suggest we have to be more
  restrictive on the set of processes and names we admit if we are to
  support no-cloning.]
\end{remark}

\subsubsection{Bisimulation}

The computational dynamics gives rise to another kind of equivalence,
the equivalence of computational behavior. As previously mentioned
this is typically captured \emph{via} some form of bisimulation.

% The notion we use in this paper is weak barbed bisimulation
% \cite{milner91polyadicpi}.

The notion we use in this paper is derived from weak barbed
bisimulation \cite{milner91polyadicpi}. 

\begin{definition}
An \emph{observation relation}, $\downarrow_{\mathcal N}$, over a set
of names, $\mathcal N$, is the smallest relation satisfying the rules
below.

\infrule[Out-barb]{y \in {\mathcal N}, \; x \nameeq y}
		  {\outputp{x}{v} \downarrow_{\mathcal N} x}
\infrule[Par-barb]{\mbox{$P\downarrow_{\mathcal N} x$ or $Q\downarrow_{\mathcal N} x$}}
		  {\binpar{P}{Q} \downarrow_{\mathcal N} x}

We write $P \Downarrow_{\mathcal N} x$ if there is $Q$ such that 
$P \wred Q$ and $Q \downarrow_{\mathcal N} x$.
\end{definition}

\begin{definition}
%\label{def.bbisim}
An  ${\mathcal N}$-\emph{barbed bisimulation} over a set of names, ${\mathcal N}$, is a symmetric binary relation 
${\mathcal S}_{\mathcal N}$ between agents such that $P\rel{S}_{\mathcal N}Q$ implies:
\begin{enumerate}
\item If $P \red P'$ then $Q \wred Q'$ and $P'\rel{S}_{\mathcal N} Q'$.
\item If $P\downarrow_{\mathcal N} x$, then $Q\Downarrow_{\mathcal N} x$.
\end{enumerate}
$P$ is ${\mathcal N}$-barbed bisimilar to $Q$, written
$P \wbbisim_{\mathcal N} Q$, if $P \rel{S}_{\mathcal N} Q$ for some ${\mathcal N}$-barbed bisimulation ${\mathcal S}_{\mathcal N}$.
\end{definition}

$\mathcal{R} \subseteq \pi \times \pi$

$P \mathcal{R} Q => \forall P'. P \red P' \Rightarrow \exists Q'. Q \red Q', P' \mathcal{R} Q'$

$P \vdash x \Rightarrow Q \vdash x$

\begin{mathpar}
  \inferrule*[lab=Out-barb]{x \nameeq y}{{y}!\langle{Q}\rangle \vdash x}
  \and
  \inferrule*[lab=Par-barb]{\mbox{$P\vdash x$ or $Q\vdash x$}}{\binpar{P}{Q} \vdash x}
\end{mathpar}

\subsubsection{Contexts}

One of the principle advantages of computational calculi like the
$\pi$-calculus is a well-defined notion of context,
contextual-equivalence and a correlation between
contextual-equivalence and notions of bisimulation. The notion of
context allows the decomposition of a process into (sub-)process and
its syntactic environment, its context. Thus, a context may be
thought of as a process with a ``hole'' (written $\Box$) in it. The
application of a context $M$ to a process $P$, written $M[P]$, is
tantamount to filling the hole in $M$ with $P$. In this paper we do
not need the full weight of this theory, but do make use of the notion
of context in the proof the main theorem. 

\begin{mathpar}
  \inferrule* [lab=summation] {} {{M_{M},M_{N}} \bc \Box \;|\; x.M_{A} \;|\; M_{M}+M_{N}}
  \and
  \inferrule* [lab=agent] {} {{M_{A}} \bc (\vec{x})M_{P} \;| \; \clift{P_0,\ldots,M_{P},\ldots,P_N}}
  \and \\
  \inferrule* [lab=process] {} {{M_{P}} \bc M_{N} \;| \;P|M_{P} }
\end{mathpar} 

\begin{mathpar}
  \inferrule* [lab=sychronization] {} {M_{N} \bc \Box \;|\; x?M_{F} \;|\; x!M_{C}}
  \and
  \inferrule* [lab=abstraction] {} {{M_{F}} \bc (x)M_{P} }
  \and
  \inferrule* [lab=concretion] {} {{M_{C}} \bc \langle M_{P} \rangle }
  \and \\
  \inferrule* [lab=process] {} {{M_{P}} \bc M_{N} \;| \;P|M_{P} }
\end{mathpar}

\begin{definition}[contextual application] Given a context $M$, and
  process $P$, we define the \emph{contextual application}, $M[P] :=
  M\{P/\Box\}$. That is, the contextual application of M to P is the
  substitution of $P$ for $\Box$ in $M$.
\end{definition}

$\meaningof{-} : L \to \mathcal{P}(\pi)$

\begin{mathpar}
  \inferrule* [lab=collection] {} {\meaningof{true} = \pi, \and \meaningof{~E} = \pi \setminus \meaningof{E}, \and \meaningof{E_{1} \& E_{2}} = \meaningof{E_{1}} \cap \meaningof{E_{2}}}
\end{mathpar}

\begin{mathpar}
  \inferrule* [lab=structure] {} {\meaningof{0} = \{ P \in \pi | P \equiv 0 \}, \and \\ \meaningof{E_1 | E_2} = \{ P \in \pi | P \equiv P_{1} | P_{2}, P_{1} \in \meaningof{E_{1}}, P_{2} \in \meaningof{E_2}\} }
\end{mathpar}

\begin{mathpar}
 \inferrule* [lab=behavior] {} {\meaningof{\langle a?b \rangle E} = \{ P \in \pi | P \equiv Q | u?(y)P', \\ \and \\\\ \and \\ \;\;\; u \in \meaningof{a}, \forall z.P'\{z/y\} \in \meaningof{E\{z/b\}}\}, \and \\ \meaningof{a!E} = \{ P \in \pi | P \equiv Q | x!\langle P' \rangle, x \in \meaningof{a} P' \in \meaningof{E}\} }
\end{mathpar}

\begin{mathpar}
 \inferrule* [lab=nominal] {} {\meaningof{\quotep{E}} = \{ \quotep{P} \in \quotep{\pi} | P \in \meaningof{E} \}, \and \meaningof{\quotep{P}} = \{ \quotep{Q} \in \quotep{\pi} | P \equiv Q \} \and \\ \meaningof{@\quotep{E}} = \{ P \in \pi | P \equiv @x, x \in \meaningof{E} \}}
\end{mathpar}

\begin{eqnarray*}
  \\
  \meaningof{-} : TS \to ST
\end{eqnarray*}

\begin{eqnarray*}
  \\
  L : TS \to ST
\end{eqnarray*}

\begin{eqnarray*}
  \\
  P \models E \iff P \in \meaningof{E}
\end{eqnarray*}

\begin{eqnarray*}
  P \approx_{L} Q \iff \forall E \in L. P \models E \iff Q \models E
\end{eqnarray*}

\begin{eqnarray*}
  P \approx_{K} Q
\end{eqnarray*}

\begin{eqnarray*}
  P \approx Q
\end{eqnarray*}

$\approx_{K} = \approx = \approx_{L}$

\subsubsection{Contextual duality}

Note that contexts extend the quotation operation to a family of
operations from processes to names. Given a context, $M$, we can
define a \emph{nominal context}, $\quotep{M}$ by $\quotep{M}[P] :=
\quotep{M[P]}$. To foreshadow what is to come we observe that these
operations enjoy a duality with processes very much like the duality
between vectors and maps from vectors to scalars.

Further, because the calculus is essentially higher-order, we have a
correspondence between contexts and processes. More specifically,
given a name $x$ and a context $M$ we can construct $M^{*}_{x}$ such
that 

\begin{mathpar}
  M^{*}_{x} | \lift{x}{P} \red M[P]
\end{mathpar}

namely,

\begin{mathpar}
  M^{*}_{x} := x?(u).M[\dropn{u}]
\end{mathpar}

The dependence of $M^{*}_{x}$ on a name makes it an abstraction, 

\begin{mathpar}
  M^{*} := (x)x?(u).M[\dropn{u}]
\end{mathpar}

\subsection{Additional notation}

It will sometimes be convenient to denote the process a name
quotes. We already have the notation $x = \quotep{P}$, but it will be
convenient to introduce an alternate notation, $\procn{x}$, when we
want to emphasize the connection to the use of the name. Note that, by
virtue of name equivalence, $\quotep{\procn{x}} \nameeq x$; so, the
notation is consistent with previous definitions.

Further, because names have structure it is possible to effect
substitutions on the basis of that structure. This means we need to
upgrade our notation for substitutions, which we accomplish by
adapting comprehension notation. Thus,

\begin{mathpar}
  P\{ y / x : x \in S \}
\end{mathpar}

is interpreted to mean the process derived from P by replacing (in a
capture-avoiding manner) each occurrence of $x$ in $S$ by $y$. For example,

\begin{mathpar}
  P\{ \quotep{\procn{x}|\procn{x}} / x : x \in \freenames{P} \}
\end{mathpar}

will replace each (occurrence) of a free name $x$ in $P$ by
$\quotep{\procn{x}|\procn{x}}$.

Also, we will avail ourselves of the notation $x^{L}$ and $x^{R}$ to
denote injections of a name into disjoint copies of the name
space. There are numerous ways to accomplish this. One example can be
found in \cite{MeredithR05}. This notation overloads to vectors of
names: $\vec{x}^{\pi} := (x_{i}^{\pi} \; : \; 0 \leq i < |\vec{x}| )$ where $\pi \in \{L,R\}$.

We also use $P^{\Box} := P|\Box$.

In \cite{MeredithR05} an interpretation of the new operator is
given. It turns out that there are several possible interpretations
all enjoying the requisite algebraic properties of the operator (see
\cite{milner91polyadicpi}). We will therefore make liberal use of
$(\nu\; \vec{x})P$.

% subsection the_syntax_and_semantics_of_the_notation_system (end)   

\input{qm2pi.qmops} 

\input{qm2pi.sterngerlach} 

\input{qm2pi.metric} 

% section concurrent_process_calculi (end)

%\input{qm2pi.proofsketch}

% section proof sketch (end)

%\input{qm2pi.slviaknots} 

% section spatial logic via knots (end)

\input{qm2pi.conclusion}

% section conclusion (end)

%\input{qm2pi.dtcodes} 

% section wiring algorithm (end)

\input{qm2pi.ack} 

% section acknowledgments (end)

\newpage


\bibliographystyle{plain}   
\bibliography{../../biblios/main.bib}

\input{qm2pi.rhodetails}

\end{document}

 

\documentclass[12pt]{llncs}
%\documentclass{jktr}

\usepackage[pdftex]{hyperref}                   
\usepackage {listings}
\usepackage {mathpartir}
\usepackage{bcprules}
%\usepackage{listings}
                       
\usepackage{graphicx} 
%\usepackage[margins=2.5cm,nohead,nofoot]{geometry}
%\usepackage{geometry}
\usepackage{amsfonts}
\usepackage{amstext}
\usepackage{latexsym}
\usepackage{amssymb}
\usepackage{color}


%\include{myPreamble}
\include{qm2pi.local} 

%\ifpdf
%\usepackage[pdftex]{graphicx}
%\else
%\usepackage{graphicx}
%\fi

 % \ifpdf
%  \usepackage{pdfsync}
%  \if


%\title{Brief Article}
%\author{David F. Snyder}
%\author{L.G. Meredith}

%\address{Dept. of Math., Texas State University--San Marcos, San Marcos, TX 78666}
       
\pagestyle{empty}


\begin{document}

\lstset{language=[Objective]Caml,frame=shadowbox}

\input{qm2pi.front}

% section front matter (end)

\input{qm2pi.intro} 
 
% section introduction (end)

% \input{qm2pi.knotations} 

% section notation (end)

\input{qm2pi.process.calculi} 

% section concurrent_process_calculi_and_spatial_logics_ (end)
    
%\input{qm2pi.knots2pi} 

%\input{qm2pi.trefoil} 

%\input{qm2pi.mainthm} 

% subsection basic_interpretation (end)

%\input{qm2pi.rho.presentation} 
\subsection{The syntax and semantics of the notation system}\label{sub:the_syntax_and_semantics_of_the_notation_system} % (fold)

We now summarize a technical presentation of the calculus that
embodies our theory of dynamics. The typical presentation of such a
calculus follows the style of giving generators and relations on
them. The grammar, below, describing term constructors, freely
generates the set of processes, $\Proc$. This set is then quotiented
by a relation known as structural congruence and it is over this set
that the notion of dynamics is expressed. This presentation is
essentially that of \cite{MeredithR05} with the addition of
polyadicity and summation. For readability we have relegated some of
the technical subtleties to an appendix.

\subsubsection{Process grammar}\label{subsub:process_grammar}

\begin{mathpar}
  \inferrule* [lab=synchronization] {} {{M} \bc \pzero \;|\; x?F \;|\; x!C }
  \and
  \inferrule* [lab=abstraction] {} {{F} \bc (x)P}
  \and
  \inferrule* [lab=concretion] {} {{C} \bc \langle Q \rangle}
  \and
  \inferrule* [lab=process] {} {{P,Q} \bc M \;| \;P|Q \;|\; @{x}}
  \and
  \inferrule* [lab=name] {} {{x} \bc \quotep{P}}
\end{mathpar} 

Note that $\vec{x}$ (resp. $\vec{P}$) denotes a vector of names
(resp. processes) of length $|\vec{x}|$ (resp. $|\vec{P}|$). We adopt
the following useful abbreviations.

\begin{mathpar}
   x?(\vec{y}).P := x.(\vec{y})P \and  x\clift{\vec{P}} := x.\clift{\vec{P}}
   \and x!(y) := \lift{x}{\dropn{y}}
   \and \Pi_{i=0}^{n-1}P_i := P_0 | \ldots | P_{n-1}
\end{mathpar}

\subsubsection{Structural congruence}

\paragraph{Free and bound names and alpha-equivalence.} At the
core of structural equivalence is alpha-equivalence which identifies
process that are the same up to a change of variable. Formally, we
recognize the distinction between free and bound names. The free names
of a process, $\freenames{P}$, may be calculated recursively as
follows:

\begin{mathpar}
\freenames{\pzero} := \emptyset
  \and \\
  \freenames{x?(y).P} := \{ x \} \cup (\freenames{P} \setminus \{ y \})
  \and 
  \freenames{x!\langle P \rangle} := \{ x \} \cup \{ P \} 
  \and \\
  \freenames{P|Q} := \freenames{P} \cup \freenames{Q}
  \and \\
  \freenames{@{x}} := \{ x \}
\end{mathpar}

$\pi$
$\quotep{\pi}$

$\freenames{-} : \pi \to \mathcal{P}(\quotep{\pi})$

\begin{eqnarray*}
  \freenames{\pzero} & := & \emptyset \\
  \freenames{x?(y).P} & := & \{ x \} \cup (\freenames{P} \setminus \{ y \}) \\
  \freenames{x!\langle P \rangle} & := & \{ x \} \cup \{ P \} \\
  \freenames{P|Q} & := & \freenames{P} \cup \freenames{Q} \\
  \freenames{\dropn{x}} & := & \{ x \}
\end{eqnarray*}

The bound names of a process, $\boundnames{P}$, are those names occurring in $P$
that are not free. For example, in $x?(y).0$, the name $x$ is free, while $y$ is bound.

\begin{mathpar}
  \inferrule* [lab=monoidal-laws] {} { P|Q \equiv Q|P \and P|0 \equiv P \and P|(Q|R) \equiv (P|Q)|R }
\end{mathpar}

\begin{mathpar}
  \inferrule* [lab=alpha-equivalence] {} { (x)P \equiv (y)P\{y/x\} \and y \not\in \freenames{P} }
\end{mathpar}

\begin{definition}
Then two processes, $P,Q$, are alpha-equivalent if $P = Q\{\vec{y}/\vec{x}\}$ for
some $\vec{x} \in \boundnames{Q},\vec{y} \in \boundnames{P}$, where $Q\{\vec{y}/\vec{x}\}$
denotes the capture-avoiding substitution of $\vec{y}$ for $\vec{x}$ in $Q$.
\end{definition}

\begin{definition}
  The {\em structural congruence} \cite{SangiorgiWalker} , $\equiv$,
  between processes is the least congruence containing
  alpha-equivalence, satisfying the abelian monoid laws
  (associativity, commutativity and $\pzero$ as identity) for parallel
  composition $|$ and for summation $+$.
\end{definition}

\subsection{Name equivalence}

We take name equivalence, written $\nameeq$, to be the smallest
equivalence relation generated by the following rules.

\begin{mathpar}
\inferrule*[lab=Quote-drop]
{ }
{ \quotep{@{x}} \nameeq x }

\inferrule*[lab=Struct-equiv]
{ P \scong Q }
{ \quotep{P} \nameeq \quotep{Q} }
\end{mathpar}

The astute reader will have noticed that the mutual recursion of names
and processes imposes a mutual recursion on alpha-equivalence and
structural equivalence via name-equivalence. Fortunately, all of this
works out pleasantly and we may calculate in the natural way, free of
concern. The reader interested in the details is referred to the
appendix \ref{appendix:rho_details}.

\subsection{Substitution}

We use $\Proc$ for the set of processes, $\QProc$ for the set of
names, and $\id{\{}\vec{y} / \vec{x} \id{\}}$ to denote partial maps,
$s : \QProc \rightarrow \QProc$. A map, $s$ lifts, uniquely, to a map
on process terms, $\widehat{s} : \Proc \rightarrow \Proc$ by the
following equations.

\begin{mathpar}
  (0) \psubstp{Q}{P} := 0 \\
  (R \juxtap S) \psubstp{Q}{P}
  :=    
  (R)\psubstp{Q}{P} \juxtap (S) \psubstp{Q}{P} \\
  (x?(y).R) \psubstp{Q}{P}    
  :=    
  (x)\substp{Q}{P} (z)\concat( (R \psubstn{z}{y}) \psubstp{Q}{P} ) \\
  (\lift{x}{R}) \psubstp{Q}{P}  
  :=
  \lift{(x)\substp{Q}{P}}{ R \psubstp{Q}{P} } \\
%   (\dropn{x})  \psubstp{Q}{P}       
%   := 
%   \left\{ 
%     \begin{array}{ccc} 
%       \dropn{\quotep{Q}} & & x \nameeq \quotep{P} \\
%       \dropn{x} & & otherwise \\
%     \end{array}
%   \right. 
  (\dropn{x})  \psubstp{Q}{P}       
  := 
  \left\{ 
    \begin{array}{ccc} 
      Q & & x \nameeq \quotep{P} \\
      \dropn{x} & & otherwise \\
    \end{array}
  \right.
\end{mathpar}
 

where

\begin{eqnarray}
  (x)\id{\{} \lpquote Q \rpquote / \lpquote P \rpquote \id{\}}            = 
  \left\{ 
    \begin{array}{ccc}
      \lpquote Q \rpquote & & x \nameeq \lpquote P \rpquote \\
      x & & otherwise \\
    \end{array}
  \right. \nonumber
\end{eqnarray}

and $z$ is chosen distinct from $\quotep{P}$, $\quotep{Q}$, the free
names in $Q$, and all the names in $R$. Our $\alpha$-equivalence will
be built in the standard way from this substitution.

\begin{remark}\label{rem:no_self_referential_names}
  One consequence of these definitions is that $\forall P. \quotep{P}
  \not\in \freenames{P}$.
\end{remark}

\subsection{ Dynamic quote: an example }

Anticipating something of what's to come, consider applying the
substitution, $\widehat{\id{\{}u / z \id{\}}}$, to the following pair
of processes, $\lift{w}{y!(z)}$ and $w[ \lpquote y!(z) \rpquote ]$.

\begin{eqnarray}
	\lift{w}{y!(z)}\widehat{\id{\{}u / z \id{\}}}
		& = &
		\lift{w}{y!(u)} \nonumber\\
	w[ \lpquote y!(z) \rpquote ] \widehat{ \id{\{}u / z \id{\}} }
		& = &
		w[ \lpquote y!(z) \rpquote ] \nonumber
\end{eqnarray}

Because the body of the process between quotes is impervious to
substitution, we get radically different answers. In fact, by
examining the first process in an input context,
e.g. $x?(z).\lift{w}{y!(z)}$, we see that the process under the lift
operator may be shaped by prefixed inputs binding a name inside it. In
this sense, the lift operator will be seen as a way to dynamically
construct processes before reifying them as names.

Finally equipped with these standard features we can present the
dynamics of the calculus.

\subsubsection{Operational semantics} 

Finally, we introduce the computational dynamics. What marks these
algebras as distinct from other more traditionally studied algebraic
structures, e.g. vector spaces or polynomial rings, is the manner in
which dynamics is captured. In traditional structures, dynamics is typically
expressed through morphisms between such structures, as in linear maps
between vector spaces or morphisms between rings. In algebras
associated with the semantics of computation, the dynamics is
expressed as part of the algebraic structure itself, through a
reduction reduction relation typically denoted by $\red$. Below, we
give a recursive presentation of this relation for the calculus used
in the encoding.

$\red \subseteq \pi \times \pi$
$\red : \pi \to \mathcal{P}(\pi)$

\begin{mathpar}
  \inferrule* [lab=Comm] { \textsf{match}( x_{src}, x_{trgt} ) } { x_{trgt}?(y)P \; | \; x_{src}!\langle {Q} \rangle \red P\{\quotep{Q}/y}\} }
  \and \\
  \inferrule* [lab=Par] {{P} \red {P}'} {{{P} | {Q}} \red {{P}' | {Q}}}
  \and
  \inferrule* [lab=Equiv]{{{P} \scong {P}'} \andalso {{P}' \red {Q}'} \andalso {{Q}' \scong {Q}}}{{P} \red {Q}}
\end{mathpar}

\begin{eqnarray*}
  match_{\equiv} (\quotep{P},\quotep{Q}) & := & P \equiv Q \\
  match_{\dagger}(\quotep{P},\quotep{Q}) & := & \forall R. P|Q \red^{*} R => R \red^{*} 0 \\
  match_{K}(\quotep{P},\quotep{Q}) & := & K \mbox{ for some context } K
\end{eqnarray*}

$u?(x)P | u!\langle Q \rangle \red P\{\quotep{Q}/x\}$

%We write $\wred$ for $\red^*$, and $P\red$ if $\exists Q $ such that $ P \red Q$.
We write $P\red$ if $\exists Q $ such that $ P \red Q$ and $P\not\red$, otherwise.

\section{Replication}

As mentioned before, it is known that replication (and hence
recursion) can be implemented in a higher-order process algebra
\cite{SangiorgiWalker}. As our first example of calculation with the
machinery thus far presented we give the construction explicitly in
the {\rhoc}.

\begin{eqnarray}
	D_{x} & := & \prefix{x}{y}{(\binpar{\outputp{x}{y}}{@{y}})} \nonumber\\
	\bangp_{x}{P} & := & \binpar{{x}!\langle{\binpar{D_{x}}{P}}\rangle}{D_{x}} \nonumber
\end{eqnarray}

\begin{eqnarray}
	\bangp_{x}{P} & & \nonumber\\
	=
	& {x}!\langle{(\prefix{x}{y}{(\outputp{x}{y} | @{y})) | P}}\rangle 
	      | \prefix{x}{y}{(\outputp{x}{y} | @{y})} & \nonumber\\
	\red
	& (\outputp{x}{y} | @{y})\substn{\quotep{(\prefix{x}{y}{(@{y} | \outputp{x}{y})) | P}}}{y} & \nonumber\\
	=
	& \outputp{x}{\quotep{(\prefix{x}{y}{(\outputp{x}{y} | @{y})) | P}}}
	  | {(\prefix{x}{y}{(\outputp{x}{y} | @{y})) | P}} & \nonumber\\
	\red
	& \ldots & \nonumber\\
	\red^*
	& P | P | \ldots & \nonumber
\end{eqnarray}

Of course, this encoding, as an implementation, runs away, unfolding
$\bangp{P}$ eagerly. A lazier and more implementable replication
operator, restricted to input-guarded processes, may be obtained as follows.

\begin{eqnarray}
\bangp{\prefix{u}{v}{P}} 
	:= 
	\binpar{\lift{x}{\prefix{u}{v}{(\binpar{D(x)}{P})}}}{D(x)} \nonumber
\end{eqnarray}

\begin{remark}
  Note that the lazier definition still does not deal with summation
  or mixed summation (i.e. sums over input and output). The reader is
  invited to construct definitions of replication that deal with these
  features. 

  Further, the definitions are parameterized in a name, $x$. Can you,
  gentle reader, make a definition that eliminates this parameter and
  guarantees no accidental interaction between the replication
  machinery and the process being replicated -- i.e. no accidental
  sharing of names used by the process to get its work done and the
  name(s) used by the replication to effect copying. This latter
  revision of the definition of replication is crucial to obtaining
  the expected identity $!!P \sim !P$.
\end{remark}

\begin{remark}\label{rem:paradoxical_combinator}
  The reader familiar with the lambda calculus will have noticed the
  similarity between $D$ and the paradoxical combinator.

  [Ed. note: the existence of this seems to suggest we have to be more
  restrictive on the set of processes and names we admit if we are to
  support no-cloning.]
\end{remark}

\subsubsection{Bisimulation}

The computational dynamics gives rise to another kind of equivalence,
the equivalence of computational behavior. As previously mentioned
this is typically captured \emph{via} some form of bisimulation.

% The notion we use in this paper is weak barbed bisimulation
% \cite{milner91polyadicpi}.

The notion we use in this paper is derived from weak barbed
bisimulation \cite{milner91polyadicpi}. 

\begin{definition}
An \emph{observation relation}, $\downarrow_{\mathcal N}$, over a set
of names, $\mathcal N$, is the smallest relation satisfying the rules
below.

\infrule[Out-barb]{y \in {\mathcal N}, \; x \nameeq y}
		  {\outputp{x}{v} \downarrow_{\mathcal N} x}
\infrule[Par-barb]{\mbox{$P\downarrow_{\mathcal N} x$ or $Q\downarrow_{\mathcal N} x$}}
		  {\binpar{P}{Q} \downarrow_{\mathcal N} x}

We write $P \Downarrow_{\mathcal N} x$ if there is $Q$ such that 
$P \wred Q$ and $Q \downarrow_{\mathcal N} x$.
\end{definition}

\begin{definition}
%\label{def.bbisim}
An  ${\mathcal N}$-\emph{barbed bisimulation} over a set of names, ${\mathcal N}$, is a symmetric binary relation 
${\mathcal S}_{\mathcal N}$ between agents such that $P\rel{S}_{\mathcal N}Q$ implies:
\begin{enumerate}
\item If $P \red P'$ then $Q \wred Q'$ and $P'\rel{S}_{\mathcal N} Q'$.
\item If $P\downarrow_{\mathcal N} x$, then $Q\Downarrow_{\mathcal N} x$.
\end{enumerate}
$P$ is ${\mathcal N}$-barbed bisimilar to $Q$, written
$P \wbbisim_{\mathcal N} Q$, if $P \rel{S}_{\mathcal N} Q$ for some ${\mathcal N}$-barbed bisimulation ${\mathcal S}_{\mathcal N}$.
\end{definition}

$\mathcal{R} \subseteq \pi \times \pi$

$P \mathcal{R} Q => \forall P'. P \red P' \Rightarrow \exists Q'. Q \red Q', P' \mathcal{R} Q'$

$P \vdash x \Rightarrow Q \vdash x$

\begin{mathpar}
  \inferrule*[lab=Out-barb]{x \nameeq y}{{y}!\langle{Q}\rangle \vdash x}
  \and
  \inferrule*[lab=Par-barb]{\mbox{$P\vdash x$ or $Q\vdash x$}}{\binpar{P}{Q} \vdash x}
\end{mathpar}

\subsubsection{Contexts}

One of the principle advantages of computational calculi like the
$\pi$-calculus is a well-defined notion of context,
contextual-equivalence and a correlation between
contextual-equivalence and notions of bisimulation. The notion of
context allows the decomposition of a process into (sub-)process and
its syntactic environment, its context. Thus, a context may be
thought of as a process with a ``hole'' (written $\Box$) in it. The
application of a context $M$ to a process $P$, written $M[P]$, is
tantamount to filling the hole in $M$ with $P$. In this paper we do
not need the full weight of this theory, but do make use of the notion
of context in the proof the main theorem. 

\begin{mathpar}
  \inferrule* [lab=summation] {} {{M_{M},M_{N}} \bc \Box \;|\; x.M_{A} \;|\; M_{M}+M_{N}}
  \and
  \inferrule* [lab=agent] {} {{M_{A}} \bc (\vec{x})M_{P} \;| \; \clift{P_0,\ldots,M_{P},\ldots,P_N}}
  \and \\
  \inferrule* [lab=process] {} {{M_{P}} \bc M_{N} \;| \;P|M_{P} }
\end{mathpar} 

\begin{mathpar}
  \inferrule* [lab=sychronization] {} {M_{N} \bc \Box \;|\; x?M_{F} \;|\; x!M_{C}}
  \and
  \inferrule* [lab=abstraction] {} {{M_{F}} \bc (x)M_{P} }
  \and
  \inferrule* [lab=concretion] {} {{M_{C}} \bc \langle M_{P} \rangle }
  \and \\
  \inferrule* [lab=process] {} {{M_{P}} \bc M_{N} \;| \;P|M_{P} }
\end{mathpar}

\begin{definition}[contextual application] Given a context $M$, and
  process $P$, we define the \emph{contextual application}, $M[P] :=
  M\{P/\Box\}$. That is, the contextual application of M to P is the
  substitution of $P$ for $\Box$ in $M$.
\end{definition}

$\meaningof{-} : L \to \mathcal{P}(\pi)$

\begin{mathpar}
  \inferrule* [lab=collection] {} {\meaningof{true} = \pi, \and \meaningof{~E} = \pi \setminus \meaningof{E}, \and \meaningof{E_{1} \& E_{2}} = \meaningof{E_{1}} \cap \meaningof{E_{2}}}
\end{mathpar}

\begin{mathpar}
  \inferrule* [lab=structure] {} {\meaningof{0} = \{ P \in \pi | P \equiv 0 \}, \and \\ \meaningof{E_1 | E_2} = \{ P \in \pi | P \equiv P_{1} | P_{2}, P_{1} \in \meaningof{E_{1}}, P_{2} \in \meaningof{E_2}\} }
\end{mathpar}

\begin{mathpar}
 \inferrule* [lab=behavior] {} {\meaningof{\langle a?b \rangle E} = \{ P \in \pi | P \equiv Q | u?(y)P', \\ \and \\\\ \and \\ \;\;\; u \in \meaningof{a}, \forall z.P'\{z/y\} \in \meaningof{E\{z/b\}}\}, \and \\ \meaningof{a!E} = \{ P \in \pi | P \equiv Q | x!\langle P' \rangle, x \in \meaningof{a} P' \in \meaningof{E}\} }
\end{mathpar}

\begin{mathpar}
 \inferrule* [lab=nominal] {} {\meaningof{\quotep{E}} = \{ \quotep{P} \in \quotep{\pi} | P \in \meaningof{E} \}, \and \meaningof{\quotep{P}} = \{ \quotep{Q} \in \quotep{\pi} | P \equiv Q \} \and \\ \meaningof{@\quotep{E}} = \{ P \in \pi | P \equiv @x, x \in \meaningof{E} \}}
\end{mathpar}

\begin{eqnarray*}
  \\
  \meaningof{-} : TS \to ST
\end{eqnarray*}

\begin{eqnarray*}
  \\
  L : TS \to ST
\end{eqnarray*}

\begin{eqnarray*}
  \\
  P \models E \iff P \in \meaningof{E}
\end{eqnarray*}

\begin{eqnarray*}
  P \approx_{L} Q \iff \forall E \in L. P \models E \iff Q \models E
\end{eqnarray*}

\begin{eqnarray*}
  P \approx_{K} Q
\end{eqnarray*}

\begin{eqnarray*}
  P \approx Q
\end{eqnarray*}

$\approx_{K} = \approx = \approx_{L}$

\subsubsection{Contextual duality}

Note that contexts extend the quotation operation to a family of
operations from processes to names. Given a context, $M$, we can
define a \emph{nominal context}, $\quotep{M}$ by $\quotep{M}[P] :=
\quotep{M[P]}$. To foreshadow what is to come we observe that these
operations enjoy a duality with processes very much like the duality
between vectors and maps from vectors to scalars.

Further, because the calculus is essentially higher-order, we have a
correspondence between contexts and processes. More specifically,
given a name $x$ and a context $M$ we can construct $M^{*}_{x}$ such
that 

\begin{mathpar}
  M^{*}_{x} | \lift{x}{P} \red M[P]
\end{mathpar}

namely,

\begin{mathpar}
  M^{*}_{x} := x?(u).M[\dropn{u}]
\end{mathpar}

The dependence of $M^{*}_{x}$ on a name makes it an abstraction, 

\begin{mathpar}
  M^{*} := (x)x?(u).M[\dropn{u}]
\end{mathpar}

\subsection{Additional notation}

It will sometimes be convenient to denote the process a name
quotes. We already have the notation $x = \quotep{P}$, but it will be
convenient to introduce an alternate notation, $\procn{x}$, when we
want to emphasize the connection to the use of the name. Note that, by
virtue of name equivalence, $\quotep{\procn{x}} \nameeq x$; so, the
notation is consistent with previous definitions.

Further, because names have structure it is possible to effect
substitutions on the basis of that structure. This means we need to
upgrade our notation for substitutions, which we accomplish by
adapting comprehension notation. Thus,

\begin{mathpar}
  P\{ y / x : x \in S \}
\end{mathpar}

is interpreted to mean the process derived from P by replacing (in a
capture-avoiding manner) each occurrence of $x$ in $S$ by $y$. For example,

\begin{mathpar}
  P\{ \quotep{\procn{x}|\procn{x}} / x : x \in \freenames{P} \}
\end{mathpar}

will replace each (occurrence) of a free name $x$ in $P$ by
$\quotep{\procn{x}|\procn{x}}$.

Also, we will avail ourselves of the notation $x^{L}$ and $x^{R}$ to
denote injections of a name into disjoint copies of the name
space. There are numerous ways to accomplish this. One example can be
found in \cite{MeredithR05}. This notation overloads to vectors of
names: $\vec{x}^{\pi} := (x_{i}^{\pi} \; : \; 0 \leq i < |\vec{x}| )$ where $\pi \in \{L,R\}$.

We also use $P^{\Box} := P|\Box$.

In \cite{MeredithR05} an interpretation of the new operator is
given. It turns out that there are several possible interpretations
all enjoying the requisite algebraic properties of the operator (see
\cite{milner91polyadicpi}). We will therefore make liberal use of
$(\nu\; \vec{x})P$.

% subsection the_syntax_and_semantics_of_the_notation_system (end)   

\input{qm2pi.qmops} 

\input{qm2pi.sterngerlach} 

\input{qm2pi.metric} 

% section concurrent_process_calculi (end)

%\input{qm2pi.proofsketch}

% section proof sketch (end)

%\input{qm2pi.slviaknots} 

% section spatial logic via knots (end)

\input{qm2pi.conclusion}

% section conclusion (end)

%\input{qm2pi.dtcodes} 

% section wiring algorithm (end)

\input{qm2pi.ack} 

% section acknowledgments (end)

\newpage


\bibliographystyle{plain}   
\bibliography{../../biblios/main.bib}

\input{qm2pi.rhodetails}

\end{document}

 

% section concurrent_process_calculi (end)

%\documentclass[12pt]{llncs}
%\documentclass{jktr}

\usepackage[pdftex]{hyperref}                   
\usepackage {listings}
\usepackage {mathpartir}
\usepackage{bcprules}
%\usepackage{listings}
                       
\usepackage{graphicx} 
%\usepackage[margins=2.5cm,nohead,nofoot]{geometry}
%\usepackage{geometry}
\usepackage{amsfonts}
\usepackage{amstext}
\usepackage{latexsym}
\usepackage{amssymb}
\usepackage{color}


%\include{myPreamble}
\include{qm2pi.local} 

%\ifpdf
%\usepackage[pdftex]{graphicx}
%\else
%\usepackage{graphicx}
%\fi

 % \ifpdf
%  \usepackage{pdfsync}
%  \if


%\title{Brief Article}
%\author{David F. Snyder}
%\author{L.G. Meredith}

%\address{Dept. of Math., Texas State University--San Marcos, San Marcos, TX 78666}
       
\pagestyle{empty}


\begin{document}

\lstset{language=[Objective]Caml,frame=shadowbox}

\input{qm2pi.front}

% section front matter (end)

\input{qm2pi.intro} 
 
% section introduction (end)

% \input{qm2pi.knotations} 

% section notation (end)

\input{qm2pi.process.calculi} 

% section concurrent_process_calculi_and_spatial_logics_ (end)
    
%\input{qm2pi.knots2pi} 

%\input{qm2pi.trefoil} 

%\input{qm2pi.mainthm} 

% subsection basic_interpretation (end)

%\input{qm2pi.rho.presentation} 
\subsection{The syntax and semantics of the notation system}\label{sub:the_syntax_and_semantics_of_the_notation_system} % (fold)

We now summarize a technical presentation of the calculus that
embodies our theory of dynamics. The typical presentation of such a
calculus follows the style of giving generators and relations on
them. The grammar, below, describing term constructors, freely
generates the set of processes, $\Proc$. This set is then quotiented
by a relation known as structural congruence and it is over this set
that the notion of dynamics is expressed. This presentation is
essentially that of \cite{MeredithR05} with the addition of
polyadicity and summation. For readability we have relegated some of
the technical subtleties to an appendix.

\subsubsection{Process grammar}\label{subsub:process_grammar}

\begin{mathpar}
  \inferrule* [lab=synchronization] {} {{M} \bc \pzero \;|\; x?F \;|\; x!C }
  \and
  \inferrule* [lab=abstraction] {} {{F} \bc (x)P}
  \and
  \inferrule* [lab=concretion] {} {{C} \bc \langle Q \rangle}
  \and
  \inferrule* [lab=process] {} {{P,Q} \bc M \;| \;P|Q \;|\; @{x}}
  \and
  \inferrule* [lab=name] {} {{x} \bc \quotep{P}}
\end{mathpar} 

Note that $\vec{x}$ (resp. $\vec{P}$) denotes a vector of names
(resp. processes) of length $|\vec{x}|$ (resp. $|\vec{P}|$). We adopt
the following useful abbreviations.

\begin{mathpar}
   x?(\vec{y}).P := x.(\vec{y})P \and  x\clift{\vec{P}} := x.\clift{\vec{P}}
   \and x!(y) := \lift{x}{\dropn{y}}
   \and \Pi_{i=0}^{n-1}P_i := P_0 | \ldots | P_{n-1}
\end{mathpar}

\subsubsection{Structural congruence}

\paragraph{Free and bound names and alpha-equivalence.} At the
core of structural equivalence is alpha-equivalence which identifies
process that are the same up to a change of variable. Formally, we
recognize the distinction between free and bound names. The free names
of a process, $\freenames{P}$, may be calculated recursively as
follows:

\begin{mathpar}
\freenames{\pzero} := \emptyset
  \and \\
  \freenames{x?(y).P} := \{ x \} \cup (\freenames{P} \setminus \{ y \})
  \and 
  \freenames{x!\langle P \rangle} := \{ x \} \cup \{ P \} 
  \and \\
  \freenames{P|Q} := \freenames{P} \cup \freenames{Q}
  \and \\
  \freenames{@{x}} := \{ x \}
\end{mathpar}

$\pi$
$\quotep{\pi}$

$\freenames{-} : \pi \to \mathcal{P}(\quotep{\pi})$

\begin{eqnarray*}
  \freenames{\pzero} & := & \emptyset \\
  \freenames{x?(y).P} & := & \{ x \} \cup (\freenames{P} \setminus \{ y \}) \\
  \freenames{x!\langle P \rangle} & := & \{ x \} \cup \{ P \} \\
  \freenames{P|Q} & := & \freenames{P} \cup \freenames{Q} \\
  \freenames{\dropn{x}} & := & \{ x \}
\end{eqnarray*}

The bound names of a process, $\boundnames{P}$, are those names occurring in $P$
that are not free. For example, in $x?(y).0$, the name $x$ is free, while $y$ is bound.

\begin{mathpar}
  \inferrule* [lab=monoidal-laws] {} { P|Q \equiv Q|P \and P|0 \equiv P \and P|(Q|R) \equiv (P|Q)|R }
\end{mathpar}

\begin{mathpar}
  \inferrule* [lab=alpha-equivalence] {} { (x)P \equiv (y)P\{y/x\} \and y \not\in \freenames{P} }
\end{mathpar}

\begin{definition}
Then two processes, $P,Q$, are alpha-equivalent if $P = Q\{\vec{y}/\vec{x}\}$ for
some $\vec{x} \in \boundnames{Q},\vec{y} \in \boundnames{P}$, where $Q\{\vec{y}/\vec{x}\}$
denotes the capture-avoiding substitution of $\vec{y}$ for $\vec{x}$ in $Q$.
\end{definition}

\begin{definition}
  The {\em structural congruence} \cite{SangiorgiWalker} , $\equiv$,
  between processes is the least congruence containing
  alpha-equivalence, satisfying the abelian monoid laws
  (associativity, commutativity and $\pzero$ as identity) for parallel
  composition $|$ and for summation $+$.
\end{definition}

\subsection{Name equivalence}

We take name equivalence, written $\nameeq$, to be the smallest
equivalence relation generated by the following rules.

\begin{mathpar}
\inferrule*[lab=Quote-drop]
{ }
{ \quotep{@{x}} \nameeq x }

\inferrule*[lab=Struct-equiv]
{ P \scong Q }
{ \quotep{P} \nameeq \quotep{Q} }
\end{mathpar}

The astute reader will have noticed that the mutual recursion of names
and processes imposes a mutual recursion on alpha-equivalence and
structural equivalence via name-equivalence. Fortunately, all of this
works out pleasantly and we may calculate in the natural way, free of
concern. The reader interested in the details is referred to the
appendix \ref{appendix:rho_details}.

\subsection{Substitution}

We use $\Proc$ for the set of processes, $\QProc$ for the set of
names, and $\id{\{}\vec{y} / \vec{x} \id{\}}$ to denote partial maps,
$s : \QProc \rightarrow \QProc$. A map, $s$ lifts, uniquely, to a map
on process terms, $\widehat{s} : \Proc \rightarrow \Proc$ by the
following equations.

\begin{mathpar}
  (0) \psubstp{Q}{P} := 0 \\
  (R \juxtap S) \psubstp{Q}{P}
  :=    
  (R)\psubstp{Q}{P} \juxtap (S) \psubstp{Q}{P} \\
  (x?(y).R) \psubstp{Q}{P}    
  :=    
  (x)\substp{Q}{P} (z)\concat( (R \psubstn{z}{y}) \psubstp{Q}{P} ) \\
  (\lift{x}{R}) \psubstp{Q}{P}  
  :=
  \lift{(x)\substp{Q}{P}}{ R \psubstp{Q}{P} } \\
%   (\dropn{x})  \psubstp{Q}{P}       
%   := 
%   \left\{ 
%     \begin{array}{ccc} 
%       \dropn{\quotep{Q}} & & x \nameeq \quotep{P} \\
%       \dropn{x} & & otherwise \\
%     \end{array}
%   \right. 
  (\dropn{x})  \psubstp{Q}{P}       
  := 
  \left\{ 
    \begin{array}{ccc} 
      Q & & x \nameeq \quotep{P} \\
      \dropn{x} & & otherwise \\
    \end{array}
  \right.
\end{mathpar}
 

where

\begin{eqnarray}
  (x)\id{\{} \lpquote Q \rpquote / \lpquote P \rpquote \id{\}}            = 
  \left\{ 
    \begin{array}{ccc}
      \lpquote Q \rpquote & & x \nameeq \lpquote P \rpquote \\
      x & & otherwise \\
    \end{array}
  \right. \nonumber
\end{eqnarray}

and $z$ is chosen distinct from $\quotep{P}$, $\quotep{Q}$, the free
names in $Q$, and all the names in $R$. Our $\alpha$-equivalence will
be built in the standard way from this substitution.

\begin{remark}\label{rem:no_self_referential_names}
  One consequence of these definitions is that $\forall P. \quotep{P}
  \not\in \freenames{P}$.
\end{remark}

\subsection{ Dynamic quote: an example }

Anticipating something of what's to come, consider applying the
substitution, $\widehat{\id{\{}u / z \id{\}}}$, to the following pair
of processes, $\lift{w}{y!(z)}$ and $w[ \lpquote y!(z) \rpquote ]$.

\begin{eqnarray}
	\lift{w}{y!(z)}\widehat{\id{\{}u / z \id{\}}}
		& = &
		\lift{w}{y!(u)} \nonumber\\
	w[ \lpquote y!(z) \rpquote ] \widehat{ \id{\{}u / z \id{\}} }
		& = &
		w[ \lpquote y!(z) \rpquote ] \nonumber
\end{eqnarray}

Because the body of the process between quotes is impervious to
substitution, we get radically different answers. In fact, by
examining the first process in an input context,
e.g. $x?(z).\lift{w}{y!(z)}$, we see that the process under the lift
operator may be shaped by prefixed inputs binding a name inside it. In
this sense, the lift operator will be seen as a way to dynamically
construct processes before reifying them as names.

Finally equipped with these standard features we can present the
dynamics of the calculus.

\subsubsection{Operational semantics} 

Finally, we introduce the computational dynamics. What marks these
algebras as distinct from other more traditionally studied algebraic
structures, e.g. vector spaces or polynomial rings, is the manner in
which dynamics is captured. In traditional structures, dynamics is typically
expressed through morphisms between such structures, as in linear maps
between vector spaces or morphisms between rings. In algebras
associated with the semantics of computation, the dynamics is
expressed as part of the algebraic structure itself, through a
reduction reduction relation typically denoted by $\red$. Below, we
give a recursive presentation of this relation for the calculus used
in the encoding.

$\red \subseteq \pi \times \pi$
$\red : \pi \to \mathcal{P}(\pi)$

\begin{mathpar}
  \inferrule* [lab=Comm] { \textsf{match}( x_{src}, x_{trgt} ) } { x_{trgt}?(y)P \; | \; x_{src}!\langle {Q} \rangle \red P\{\quotep{Q}/y}\} }
  \and \\
  \inferrule* [lab=Par] {{P} \red {P}'} {{{P} | {Q}} \red {{P}' | {Q}}}
  \and
  \inferrule* [lab=Equiv]{{{P} \scong {P}'} \andalso {{P}' \red {Q}'} \andalso {{Q}' \scong {Q}}}{{P} \red {Q}}
\end{mathpar}

\begin{eqnarray*}
  match_{\equiv} (\quotep{P},\quotep{Q}) & := & P \equiv Q \\
  match_{\dagger}(\quotep{P},\quotep{Q}) & := & \forall R. P|Q \red^{*} R => R \red^{*} 0 \\
  match_{K}(\quotep{P},\quotep{Q}) & := & K \mbox{ for some context } K
\end{eqnarray*}

$u?(x)P | u!\langle Q \rangle \red P\{\quotep{Q}/x\}$

%We write $\wred$ for $\red^*$, and $P\red$ if $\exists Q $ such that $ P \red Q$.
We write $P\red$ if $\exists Q $ such that $ P \red Q$ and $P\not\red$, otherwise.

\section{Replication}

As mentioned before, it is known that replication (and hence
recursion) can be implemented in a higher-order process algebra
\cite{SangiorgiWalker}. As our first example of calculation with the
machinery thus far presented we give the construction explicitly in
the {\rhoc}.

\begin{eqnarray}
	D_{x} & := & \prefix{x}{y}{(\binpar{\outputp{x}{y}}{@{y}})} \nonumber\\
	\bangp_{x}{P} & := & \binpar{{x}!\langle{\binpar{D_{x}}{P}}\rangle}{D_{x}} \nonumber
\end{eqnarray}

\begin{eqnarray}
	\bangp_{x}{P} & & \nonumber\\
	=
	& {x}!\langle{(\prefix{x}{y}{(\outputp{x}{y} | @{y})) | P}}\rangle 
	      | \prefix{x}{y}{(\outputp{x}{y} | @{y})} & \nonumber\\
	\red
	& (\outputp{x}{y} | @{y})\substn{\quotep{(\prefix{x}{y}{(@{y} | \outputp{x}{y})) | P}}}{y} & \nonumber\\
	=
	& \outputp{x}{\quotep{(\prefix{x}{y}{(\outputp{x}{y} | @{y})) | P}}}
	  | {(\prefix{x}{y}{(\outputp{x}{y} | @{y})) | P}} & \nonumber\\
	\red
	& \ldots & \nonumber\\
	\red^*
	& P | P | \ldots & \nonumber
\end{eqnarray}

Of course, this encoding, as an implementation, runs away, unfolding
$\bangp{P}$ eagerly. A lazier and more implementable replication
operator, restricted to input-guarded processes, may be obtained as follows.

\begin{eqnarray}
\bangp{\prefix{u}{v}{P}} 
	:= 
	\binpar{\lift{x}{\prefix{u}{v}{(\binpar{D(x)}{P})}}}{D(x)} \nonumber
\end{eqnarray}

\begin{remark}
  Note that the lazier definition still does not deal with summation
  or mixed summation (i.e. sums over input and output). The reader is
  invited to construct definitions of replication that deal with these
  features. 

  Further, the definitions are parameterized in a name, $x$. Can you,
  gentle reader, make a definition that eliminates this parameter and
  guarantees no accidental interaction between the replication
  machinery and the process being replicated -- i.e. no accidental
  sharing of names used by the process to get its work done and the
  name(s) used by the replication to effect copying. This latter
  revision of the definition of replication is crucial to obtaining
  the expected identity $!!P \sim !P$.
\end{remark}

\begin{remark}\label{rem:paradoxical_combinator}
  The reader familiar with the lambda calculus will have noticed the
  similarity between $D$ and the paradoxical combinator.

  [Ed. note: the existence of this seems to suggest we have to be more
  restrictive on the set of processes and names we admit if we are to
  support no-cloning.]
\end{remark}

\subsubsection{Bisimulation}

The computational dynamics gives rise to another kind of equivalence,
the equivalence of computational behavior. As previously mentioned
this is typically captured \emph{via} some form of bisimulation.

% The notion we use in this paper is weak barbed bisimulation
% \cite{milner91polyadicpi}.

The notion we use in this paper is derived from weak barbed
bisimulation \cite{milner91polyadicpi}. 

\begin{definition}
An \emph{observation relation}, $\downarrow_{\mathcal N}$, over a set
of names, $\mathcal N$, is the smallest relation satisfying the rules
below.

\infrule[Out-barb]{y \in {\mathcal N}, \; x \nameeq y}
		  {\outputp{x}{v} \downarrow_{\mathcal N} x}
\infrule[Par-barb]{\mbox{$P\downarrow_{\mathcal N} x$ or $Q\downarrow_{\mathcal N} x$}}
		  {\binpar{P}{Q} \downarrow_{\mathcal N} x}

We write $P \Downarrow_{\mathcal N} x$ if there is $Q$ such that 
$P \wred Q$ and $Q \downarrow_{\mathcal N} x$.
\end{definition}

\begin{definition}
%\label{def.bbisim}
An  ${\mathcal N}$-\emph{barbed bisimulation} over a set of names, ${\mathcal N}$, is a symmetric binary relation 
${\mathcal S}_{\mathcal N}$ between agents such that $P\rel{S}_{\mathcal N}Q$ implies:
\begin{enumerate}
\item If $P \red P'$ then $Q \wred Q'$ and $P'\rel{S}_{\mathcal N} Q'$.
\item If $P\downarrow_{\mathcal N} x$, then $Q\Downarrow_{\mathcal N} x$.
\end{enumerate}
$P$ is ${\mathcal N}$-barbed bisimilar to $Q$, written
$P \wbbisim_{\mathcal N} Q$, if $P \rel{S}_{\mathcal N} Q$ for some ${\mathcal N}$-barbed bisimulation ${\mathcal S}_{\mathcal N}$.
\end{definition}

$\mathcal{R} \subseteq \pi \times \pi$

$P \mathcal{R} Q => \forall P'. P \red P' \Rightarrow \exists Q'. Q \red Q', P' \mathcal{R} Q'$

$P \vdash x \Rightarrow Q \vdash x$

\begin{mathpar}
  \inferrule*[lab=Out-barb]{x \nameeq y}{{y}!\langle{Q}\rangle \vdash x}
  \and
  \inferrule*[lab=Par-barb]{\mbox{$P\vdash x$ or $Q\vdash x$}}{\binpar{P}{Q} \vdash x}
\end{mathpar}

\subsubsection{Contexts}

One of the principle advantages of computational calculi like the
$\pi$-calculus is a well-defined notion of context,
contextual-equivalence and a correlation between
contextual-equivalence and notions of bisimulation. The notion of
context allows the decomposition of a process into (sub-)process and
its syntactic environment, its context. Thus, a context may be
thought of as a process with a ``hole'' (written $\Box$) in it. The
application of a context $M$ to a process $P$, written $M[P]$, is
tantamount to filling the hole in $M$ with $P$. In this paper we do
not need the full weight of this theory, but do make use of the notion
of context in the proof the main theorem. 

\begin{mathpar}
  \inferrule* [lab=summation] {} {{M_{M},M_{N}} \bc \Box \;|\; x.M_{A} \;|\; M_{M}+M_{N}}
  \and
  \inferrule* [lab=agent] {} {{M_{A}} \bc (\vec{x})M_{P} \;| \; \clift{P_0,\ldots,M_{P},\ldots,P_N}}
  \and \\
  \inferrule* [lab=process] {} {{M_{P}} \bc M_{N} \;| \;P|M_{P} }
\end{mathpar} 

\begin{mathpar}
  \inferrule* [lab=sychronization] {} {M_{N} \bc \Box \;|\; x?M_{F} \;|\; x!M_{C}}
  \and
  \inferrule* [lab=abstraction] {} {{M_{F}} \bc (x)M_{P} }
  \and
  \inferrule* [lab=concretion] {} {{M_{C}} \bc \langle M_{P} \rangle }
  \and \\
  \inferrule* [lab=process] {} {{M_{P}} \bc M_{N} \;| \;P|M_{P} }
\end{mathpar}

\begin{definition}[contextual application] Given a context $M$, and
  process $P$, we define the \emph{contextual application}, $M[P] :=
  M\{P/\Box\}$. That is, the contextual application of M to P is the
  substitution of $P$ for $\Box$ in $M$.
\end{definition}

$\meaningof{-} : L \to \mathcal{P}(\pi)$

\begin{mathpar}
  \inferrule* [lab=collection] {} {\meaningof{true} = \pi, \and \meaningof{~E} = \pi \setminus \meaningof{E}, \and \meaningof{E_{1} \& E_{2}} = \meaningof{E_{1}} \cap \meaningof{E_{2}}}
\end{mathpar}

\begin{mathpar}
  \inferrule* [lab=structure] {} {\meaningof{0} = \{ P \in \pi | P \equiv 0 \}, \and \\ \meaningof{E_1 | E_2} = \{ P \in \pi | P \equiv P_{1} | P_{2}, P_{1} \in \meaningof{E_{1}}, P_{2} \in \meaningof{E_2}\} }
\end{mathpar}

\begin{mathpar}
 \inferrule* [lab=behavior] {} {\meaningof{\langle a?b \rangle E} = \{ P \in \pi | P \equiv Q | u?(y)P', \\ \and \\\\ \and \\ \;\;\; u \in \meaningof{a}, \forall z.P'\{z/y\} \in \meaningof{E\{z/b\}}\}, \and \\ \meaningof{a!E} = \{ P \in \pi | P \equiv Q | x!\langle P' \rangle, x \in \meaningof{a} P' \in \meaningof{E}\} }
\end{mathpar}

\begin{mathpar}
 \inferrule* [lab=nominal] {} {\meaningof{\quotep{E}} = \{ \quotep{P} \in \quotep{\pi} | P \in \meaningof{E} \}, \and \meaningof{\quotep{P}} = \{ \quotep{Q} \in \quotep{\pi} | P \equiv Q \} \and \\ \meaningof{@\quotep{E}} = \{ P \in \pi | P \equiv @x, x \in \meaningof{E} \}}
\end{mathpar}

\begin{eqnarray*}
  \\
  \meaningof{-} : TS \to ST
\end{eqnarray*}

\begin{eqnarray*}
  \\
  L : TS \to ST
\end{eqnarray*}

\begin{eqnarray*}
  \\
  P \models E \iff P \in \meaningof{E}
\end{eqnarray*}

\begin{eqnarray*}
  P \approx_{L} Q \iff \forall E \in L. P \models E \iff Q \models E
\end{eqnarray*}

\begin{eqnarray*}
  P \approx_{K} Q
\end{eqnarray*}

\begin{eqnarray*}
  P \approx Q
\end{eqnarray*}

$\approx_{K} = \approx = \approx_{L}$

\subsubsection{Contextual duality}

Note that contexts extend the quotation operation to a family of
operations from processes to names. Given a context, $M$, we can
define a \emph{nominal context}, $\quotep{M}$ by $\quotep{M}[P] :=
\quotep{M[P]}$. To foreshadow what is to come we observe that these
operations enjoy a duality with processes very much like the duality
between vectors and maps from vectors to scalars.

Further, because the calculus is essentially higher-order, we have a
correspondence between contexts and processes. More specifically,
given a name $x$ and a context $M$ we can construct $M^{*}_{x}$ such
that 

\begin{mathpar}
  M^{*}_{x} | \lift{x}{P} \red M[P]
\end{mathpar}

namely,

\begin{mathpar}
  M^{*}_{x} := x?(u).M[\dropn{u}]
\end{mathpar}

The dependence of $M^{*}_{x}$ on a name makes it an abstraction, 

\begin{mathpar}
  M^{*} := (x)x?(u).M[\dropn{u}]
\end{mathpar}

\subsection{Additional notation}

It will sometimes be convenient to denote the process a name
quotes. We already have the notation $x = \quotep{P}$, but it will be
convenient to introduce an alternate notation, $\procn{x}$, when we
want to emphasize the connection to the use of the name. Note that, by
virtue of name equivalence, $\quotep{\procn{x}} \nameeq x$; so, the
notation is consistent with previous definitions.

Further, because names have structure it is possible to effect
substitutions on the basis of that structure. This means we need to
upgrade our notation for substitutions, which we accomplish by
adapting comprehension notation. Thus,

\begin{mathpar}
  P\{ y / x : x \in S \}
\end{mathpar}

is interpreted to mean the process derived from P by replacing (in a
capture-avoiding manner) each occurrence of $x$ in $S$ by $y$. For example,

\begin{mathpar}
  P\{ \quotep{\procn{x}|\procn{x}} / x : x \in \freenames{P} \}
\end{mathpar}

will replace each (occurrence) of a free name $x$ in $P$ by
$\quotep{\procn{x}|\procn{x}}$.

Also, we will avail ourselves of the notation $x^{L}$ and $x^{R}$ to
denote injections of a name into disjoint copies of the name
space. There are numerous ways to accomplish this. One example can be
found in \cite{MeredithR05}. This notation overloads to vectors of
names: $\vec{x}^{\pi} := (x_{i}^{\pi} \; : \; 0 \leq i < |\vec{x}| )$ where $\pi \in \{L,R\}$.

We also use $P^{\Box} := P|\Box$.

In \cite{MeredithR05} an interpretation of the new operator is
given. It turns out that there are several possible interpretations
all enjoying the requisite algebraic properties of the operator (see
\cite{milner91polyadicpi}). We will therefore make liberal use of
$(\nu\; \vec{x})P$.

% subsection the_syntax_and_semantics_of_the_notation_system (end)   

\input{qm2pi.qmops} 

\input{qm2pi.sterngerlach} 

\input{qm2pi.metric} 

% section concurrent_process_calculi (end)

%\input{qm2pi.proofsketch}

% section proof sketch (end)

%\input{qm2pi.slviaknots} 

% section spatial logic via knots (end)

\input{qm2pi.conclusion}

% section conclusion (end)

%\input{qm2pi.dtcodes} 

% section wiring algorithm (end)

\input{qm2pi.ack} 

% section acknowledgments (end)

\newpage


\bibliographystyle{plain}   
\bibliography{../../biblios/main.bib}

\input{qm2pi.rhodetails}

\end{document}



% section proof sketch (end)

%\section{Unlikely characters: spatial logic for
  knots}\label{sub:characteristic_formulae} % (fold)

Associated to the mobile process calculi are a family of logics known
as the Hennessy-Milner logics. These logics typically enjoy a
semantics interpreting formulae as sets of processes that when
factored through the encoding outlined above allows an identification
of classes of knots with logical formulae. In the context of this
encoding the sub-family known as the spatial logics \cite{CairesC03}
\cite{CairesC04} \cite{Caires04} are of particular interest providing
several important features for expressing and reasoning about
properties (i.e. classes) of knots. We hint here at how this may be done.

%\begin{description}
%\item [structural connectives] 
\subsubsection{Structural connectives} The spatial logics enjoy
structural connectives corresponding, at the logical level, to the
parallel composition ($P | Q$) and new name ($(\nu \; x)P$)
connectives for processes. As illustrated in the examples below, these
connectives are extremely expressive given the shape of our encoding.
%\item [decideable satisfaction]

\subsubsection{Decideable satisfaction}
In \cite{Caires04} the satisfaction relation is shown to be decideable
for a rich class of processes. It further turns out that the image of
the our encoding is a proper subset of that class. This result
provides the basis for an algorithm by which to search for knots
enjoying a given property.
%\item [characteristic formulae]

\subsubsection{Characteristic formulae}
In the same paper \cite{Caires04} , Caires presents a means of calculating
characteristic formulae, selecting equivalence classes of processes
up to a pre--specified depth limit on the support set of names. Composed with our
encoding, this characteristic formula can be used to select
characteristic formulae for knots.
%\end{description}

\subsubsection{Spatial logic formulae}

The grammar below (segmented for comprehension) summarizes the syntax
of spatial logic formulae. We employ illustrative examples in the
sequel to provide an intuitive understanding of their meaning
referring the reader to \cite{Caires04} for a more detailed explication
of the semantics.

\begin{mathpar}
  \inferrule* [lab=boolean] {} {{A,B} \bc T \;|\; \neg A \;|\; A \wedge B \;|\; \eta = \eta'}
  \and
  \inferrule* [lab=spatial] {} {|\; \pzero \;|\; A | B \;|\; x \text{\textregistered} A \;|\; \forall x . A \;|\;  H x . A}
  \and
  \inferrule* [lab=behavioral] {} {|\; \alpha . A}
  \and 
  \inferrule* [lab=recursion] {} {|\; X(\vec{u}) \;|\; \mu X(\vec{u}) . A}
  \and
  \inferrule* [lab=action] {} {\alpha \bc \langle x?(\vec{y}) \rangle \;|\; \langle x!(\vec{y}) \rangle \;|\; \langle \tau \rangle}
  \and 
  \inferrule* [lab=name] {} {\eta \bc x \;|\; \tau}
\end{mathpar} 

% subsection characteristic_formulae (end)   	 

\subsection{Example formulae}\label{sub:example_formulae_} % (fold)

\subsubsection{Crossing as formula.}
% 
% \begin{align*}
%   \frac{d}{dx} \sin x &= \cos x 
%   & \frac{d}{dx} e^x &= e^x \\
%   \frac{d}{dx} \cos x &= - \sin x 
%   & \frac{d}{dx} \log x &= \frac{1}{x} \\
% \end{align*} 

\begin{align*}
 \mu C(x_{0},x_{1},y_{0},y_{1},u).&(\langle x_{0}?(z) \rangle(\langle u! \rangle\langle y_{1}!z \rangle C(x_{0},x_{1},y_{0},y_{1},u)) & \\
  & \wedge \langle y_{1}?(z) \rangle (\langle u! \rangle \langle x_{0}!z \rangle C(x_{0},x_{1},y_{0},y_{1},u)) & \\
  & \wedge \langle x_{1}?(z) \rangle (\langle u? \rangle \langle y_{0}!z \rangle C(x_{0},x_{1},y_{0},y_{1},u)) & \\
  & \wedge \langle y_{0}?(z) \rangle (\langle u? \rangle \langle x_{1}!z \rangle C(x_{0},x_{1},y_{0},y_{1},u))) &
\end{align*}

The lexicographical similarity between the shape of this formulae and
the shape of definition of the process representing a crossing reveals
the intuitive meaning of this formulae. It describes the capabilities
of a process that has the right to represent a crossing. For example
it picks out processes that may perform an input on the port $x_0$ in
its initial menu of capabilities. What differentiates the formula
from the process, however, is that the crossing process is the
smallest candidate to satisfy the formula. Infinitely many other
processes -- with internal behavior hidden behind this interface, so
to speak -- also satisfy this formula. Even this simple formula,
then, can be seen to open a new view onto knots, providing a
computational interpretation of \emph{virtual} knots.

Note that this formula is derived by hand. A similar formula can be
derived by employing Caires' calculation of characteristic formula
\cite{Caires04} to the process representing a crossing. In light of
this discussion, we let
$\meaningof{C}_{\phi}(x0,x1,y0,y1,u)$ denote a formula specifying the
dynamics we wish to capture of a crossing. To guarantee we preserve
the shape of the interface and minimal semantics we demand that
$\meaningof{C}_{\phi}(x0,x1,y0,y1,u) \Rightarrow
\textbf{C}(x0,x1,y0,y1,u)$ where $\textbf{C}(x0,x1,y0,y1,u)$ denotes
the formula above.
                            
\subsubsection{Crossing number constraints.}
The moral content of the context lemma (Lemma \ref{context}) is that the notion of
``locality'' in the Reidemeister moves is effectively captured by the
parallel composition operator of the process calculus. This intuition
extends through the logic. Given a formula,
$\meaningof{C}_{\phi}(x0,x1,y0,y1,u)$, we can use the structural
connectives to specify constraints on crossing numbers, such as at
least $n$ crossings, or exactly $n$ crossings.
\begin{mathpar}
  \inferrule* [lab=at-least-n] {} { K^{\geq n}_{\phi}(\vec{xs},\vec{ys}) := \Pi_{i=0}^{n-1} Hu . \meaningof{C}_{\phi}(xs_i,ys_i,u) | T }
  \and 
  \inferrule* [lab=exactly-n] {} { K^{= n}_{\phi}(\vec{xs},\vec{ys}) := \Pi_{i=0}^{n-1} Hu . \meaningof{C}_{\phi}(xs_i,ys_i,u) | \neg (\forall x_0,y_0,x_1,y_1,u . \meaningof{C}_{\phi}(x_0,y_0,x_1,y_1,u) | T) }
\end{mathpar}

To round out this section, recall that the encoding of an $n$-crossing
knot decomposes into a parallel composition of $n$ \emph{copies} of a
crossing process together with a wiring harness. To specify different
knot classes with the same crossing number amounts to specifying
logical constraints on the wiring harness. In the interest of space,
we defer examples to a forthcoming paper. Suffice it to say that both
the conditions ``alternating knot'' and ``contains the tangle
corresponding to 5/3'' are expressible. For example, it is possible to
calculate the characteristic formula of a process corresponding to the
tangle 5/3 and conjoin it into the classifying formula via the
composition connective of the logic.

Finally, we wish to observe that it is entirely within reason to
contemplate a more domain-specific version of spatial logic tailored
to the shape of processes in the image of the encoding. Such a
domain-specific logic would have a better claim to the title formal
language of knot properties.

% subsection example_formulae_ (end)

% section knots_as_processes (end) 

% section spatial logic via knots (end)

\section{Conclusions and future work}

\paragraph{Testing physical space}
You, gentle reader, may wonder why of all the theorems to be proved
given this set up we pick the one above. In some sense it's hardly
central to quantum mechanics. We see it as central in the sense that
it firmly establishes a notion of physical space arising from a notion
of the equivalence of behavior. Relating bisimulation to a metric is a
big step forward, but one is faced with interpreting the relationship
of that metric space to something more physical. Quantum mechanical
notions of ``physical'' space are still far from intuitive, but by
relating this idea of distance as testing to calculations that predict
physical circumstances we are making a not insignificant step forward
toward an understanding of the physical space we inhabit as
essentially dynamic.

\paragraph{Effectivity and simulation}
One of the observations we have yet to make is that the entire program
spelled out here is effective. We have built various interpreters for
the reflective calculus at work in this interpretation. In principle,
then, we can simulate quantum mechanics on a computer. The place where
the simulation may lose fidelity is the infinitely branching summation
for the annihilator.

In this connection i also want to point out that the evaluation style
calculation of the inner product puts the non-determinism of the
summation right at the heart of measurement. This suggests that
Milner's original reduction-based formulation of the dynamics of his
calculi in terms of sums was not just notationally suggestive of a
notion of measure-and-continue but captured some significant part of
the physics.

\paragraph{Quantum continuations}
In light of this last observation i want to point out that the
predominant account of quantum mechanics is missing a key aspect of a
truly compositional story of the physical situation. In a real lab,
when a measurement is made the observation can be made to feed into
another device that then makes another measurement conditioned on the
results of the first. This means that after the superposition was
collapsed the entire experimental set up remained in
superposition. While QM offers a means of writing this down it doesn't
quite line up well with the well-trodden formulation of computation
and continuation that we see so succinctly expressed in Milner's
calculi. This suggests that there might be advantages to this account
of dynamics waiting to be explored.

\paragraph{Quantum logic}
In this connection, we also note that by virtue of having the
Hennessy-Milner construction, we can pull the construction through the
interpretation of QM. This gives us a natural candidate for a quantum
logic that enjoys an extremely tight connection with it's domain of
interpretation, making the construction much less ad hoc (rather it is
the image of functor!).

\paragraph{Quantum probabiity}
i have questions about the basis of the interpretation of inner
product as probability amplitude. In particular, using which
axiomatization of probability theory does the notion of probability
amplitude earn the right to be so dubbed? In other words, where is the
proof that the operation for calculating a probability amplitude (and
then squaring) satisfies the axioms of what it means to calculate a
probability? Even if such a proof exists (i have yet to find it in the
literature), i wonder if it might not be possible to turn things on
their heads. Can we view the calculation of the probability amplitude
as an axiomatization of probability? If so, then the definition we
give for calculating probability amplitude may provide the basis for
an \emph{effective} theory of probability.

\paragraph{Quantum vs ``biological'' information}
Finally, i want to conclude with a more philosophical observation. At
a recent workshop in which QM was a predominant topic i noticed
something about quantum information. The speaker was giving a riveting
discussion of axiomatic QM and showing how properties of ``no
cloning'' and ``no deleting'' emerged as consequences of the
axiomatization. Theorems of this form are necessary to give us a sense
of confidence that our axioms characterize the physical theory. What
struck me, though, was that if quantum information is neither erasable
nor replicable it is markedly different from \emph{life}. Two of the
things we know about life is that

\begin{itemize}
  \item it ends;
  \item to gain some measure of persistence, to transcend it's
    finitude it is imminently copyable.
\end{itemize}

Both of these qualities are summarized succinctly in the aphorism: all
flesh is grass. For me these two kinds of ``information'' -- call them
quantum and biological -- are end points on a spectrum of strategies
for persistence. At one end, we have those curious entities that enjoy
uniqueness and permanence; at the other, we have those who in the face
of a certain end and an uncertain present make a go of passing
something on. To me one of the more remarkable aspects of the latter
strategy is that in the presence of noise (and certain features of
copying) we get a kind of dynamism, a chance for improvement against a
given persistent condition.

% subsection other_calculi_other_bisimulations_and_geometry_as_behavior (end)




% section conclusion (end)

%\documentclass[12pt]{llncs}
%\documentclass{jktr}

\usepackage[pdftex]{hyperref}                   
\usepackage {listings}
\usepackage {mathpartir}
\usepackage{bcprules}
%\usepackage{listings}
                       
\usepackage{graphicx} 
%\usepackage[margins=2.5cm,nohead,nofoot]{geometry}
%\usepackage{geometry}
\usepackage{amsfonts}
\usepackage{amstext}
\usepackage{latexsym}
\usepackage{amssymb}
\usepackage{color}


%\include{myPreamble}
\include{qm2pi.local} 

%\ifpdf
%\usepackage[pdftex]{graphicx}
%\else
%\usepackage{graphicx}
%\fi

 % \ifpdf
%  \usepackage{pdfsync}
%  \if


%\title{Brief Article}
%\author{David F. Snyder}
%\author{L.G. Meredith}

%\address{Dept. of Math., Texas State University--San Marcos, San Marcos, TX 78666}
       
\pagestyle{empty}


\begin{document}

\lstset{language=[Objective]Caml,frame=shadowbox}

\input{qm2pi.front}

% section front matter (end)

\input{qm2pi.intro} 
 
% section introduction (end)

% \input{qm2pi.knotations} 

% section notation (end)

\input{qm2pi.process.calculi} 

% section concurrent_process_calculi_and_spatial_logics_ (end)
    
%\input{qm2pi.knots2pi} 

%\input{qm2pi.trefoil} 

%\input{qm2pi.mainthm} 

% subsection basic_interpretation (end)

%\input{qm2pi.rho.presentation} 
\subsection{The syntax and semantics of the notation system}\label{sub:the_syntax_and_semantics_of_the_notation_system} % (fold)

We now summarize a technical presentation of the calculus that
embodies our theory of dynamics. The typical presentation of such a
calculus follows the style of giving generators and relations on
them. The grammar, below, describing term constructors, freely
generates the set of processes, $\Proc$. This set is then quotiented
by a relation known as structural congruence and it is over this set
that the notion of dynamics is expressed. This presentation is
essentially that of \cite{MeredithR05} with the addition of
polyadicity and summation. For readability we have relegated some of
the technical subtleties to an appendix.

\subsubsection{Process grammar}\label{subsub:process_grammar}

\begin{mathpar}
  \inferrule* [lab=synchronization] {} {{M} \bc \pzero \;|\; x?F \;|\; x!C }
  \and
  \inferrule* [lab=abstraction] {} {{F} \bc (x)P}
  \and
  \inferrule* [lab=concretion] {} {{C} \bc \langle Q \rangle}
  \and
  \inferrule* [lab=process] {} {{P,Q} \bc M \;| \;P|Q \;|\; @{x}}
  \and
  \inferrule* [lab=name] {} {{x} \bc \quotep{P}}
\end{mathpar} 

Note that $\vec{x}$ (resp. $\vec{P}$) denotes a vector of names
(resp. processes) of length $|\vec{x}|$ (resp. $|\vec{P}|$). We adopt
the following useful abbreviations.

\begin{mathpar}
   x?(\vec{y}).P := x.(\vec{y})P \and  x\clift{\vec{P}} := x.\clift{\vec{P}}
   \and x!(y) := \lift{x}{\dropn{y}}
   \and \Pi_{i=0}^{n-1}P_i := P_0 | \ldots | P_{n-1}
\end{mathpar}

\subsubsection{Structural congruence}

\paragraph{Free and bound names and alpha-equivalence.} At the
core of structural equivalence is alpha-equivalence which identifies
process that are the same up to a change of variable. Formally, we
recognize the distinction between free and bound names. The free names
of a process, $\freenames{P}$, may be calculated recursively as
follows:

\begin{mathpar}
\freenames{\pzero} := \emptyset
  \and \\
  \freenames{x?(y).P} := \{ x \} \cup (\freenames{P} \setminus \{ y \})
  \and 
  \freenames{x!\langle P \rangle} := \{ x \} \cup \{ P \} 
  \and \\
  \freenames{P|Q} := \freenames{P} \cup \freenames{Q}
  \and \\
  \freenames{@{x}} := \{ x \}
\end{mathpar}

$\pi$
$\quotep{\pi}$

$\freenames{-} : \pi \to \mathcal{P}(\quotep{\pi})$

\begin{eqnarray*}
  \freenames{\pzero} & := & \emptyset \\
  \freenames{x?(y).P} & := & \{ x \} \cup (\freenames{P} \setminus \{ y \}) \\
  \freenames{x!\langle P \rangle} & := & \{ x \} \cup \{ P \} \\
  \freenames{P|Q} & := & \freenames{P} \cup \freenames{Q} \\
  \freenames{\dropn{x}} & := & \{ x \}
\end{eqnarray*}

The bound names of a process, $\boundnames{P}$, are those names occurring in $P$
that are not free. For example, in $x?(y).0$, the name $x$ is free, while $y$ is bound.

\begin{mathpar}
  \inferrule* [lab=monoidal-laws] {} { P|Q \equiv Q|P \and P|0 \equiv P \and P|(Q|R) \equiv (P|Q)|R }
\end{mathpar}

\begin{mathpar}
  \inferrule* [lab=alpha-equivalence] {} { (x)P \equiv (y)P\{y/x\} \and y \not\in \freenames{P} }
\end{mathpar}

\begin{definition}
Then two processes, $P,Q$, are alpha-equivalent if $P = Q\{\vec{y}/\vec{x}\}$ for
some $\vec{x} \in \boundnames{Q},\vec{y} \in \boundnames{P}$, where $Q\{\vec{y}/\vec{x}\}$
denotes the capture-avoiding substitution of $\vec{y}$ for $\vec{x}$ in $Q$.
\end{definition}

\begin{definition}
  The {\em structural congruence} \cite{SangiorgiWalker} , $\equiv$,
  between processes is the least congruence containing
  alpha-equivalence, satisfying the abelian monoid laws
  (associativity, commutativity and $\pzero$ as identity) for parallel
  composition $|$ and for summation $+$.
\end{definition}

\subsection{Name equivalence}

We take name equivalence, written $\nameeq$, to be the smallest
equivalence relation generated by the following rules.

\begin{mathpar}
\inferrule*[lab=Quote-drop]
{ }
{ \quotep{@{x}} \nameeq x }

\inferrule*[lab=Struct-equiv]
{ P \scong Q }
{ \quotep{P} \nameeq \quotep{Q} }
\end{mathpar}

The astute reader will have noticed that the mutual recursion of names
and processes imposes a mutual recursion on alpha-equivalence and
structural equivalence via name-equivalence. Fortunately, all of this
works out pleasantly and we may calculate in the natural way, free of
concern. The reader interested in the details is referred to the
appendix \ref{appendix:rho_details}.

\subsection{Substitution}

We use $\Proc$ for the set of processes, $\QProc$ for the set of
names, and $\id{\{}\vec{y} / \vec{x} \id{\}}$ to denote partial maps,
$s : \QProc \rightarrow \QProc$. A map, $s$ lifts, uniquely, to a map
on process terms, $\widehat{s} : \Proc \rightarrow \Proc$ by the
following equations.

\begin{mathpar}
  (0) \psubstp{Q}{P} := 0 \\
  (R \juxtap S) \psubstp{Q}{P}
  :=    
  (R)\psubstp{Q}{P} \juxtap (S) \psubstp{Q}{P} \\
  (x?(y).R) \psubstp{Q}{P}    
  :=    
  (x)\substp{Q}{P} (z)\concat( (R \psubstn{z}{y}) \psubstp{Q}{P} ) \\
  (\lift{x}{R}) \psubstp{Q}{P}  
  :=
  \lift{(x)\substp{Q}{P}}{ R \psubstp{Q}{P} } \\
%   (\dropn{x})  \psubstp{Q}{P}       
%   := 
%   \left\{ 
%     \begin{array}{ccc} 
%       \dropn{\quotep{Q}} & & x \nameeq \quotep{P} \\
%       \dropn{x} & & otherwise \\
%     \end{array}
%   \right. 
  (\dropn{x})  \psubstp{Q}{P}       
  := 
  \left\{ 
    \begin{array}{ccc} 
      Q & & x \nameeq \quotep{P} \\
      \dropn{x} & & otherwise \\
    \end{array}
  \right.
\end{mathpar}
 

where

\begin{eqnarray}
  (x)\id{\{} \lpquote Q \rpquote / \lpquote P \rpquote \id{\}}            = 
  \left\{ 
    \begin{array}{ccc}
      \lpquote Q \rpquote & & x \nameeq \lpquote P \rpquote \\
      x & & otherwise \\
    \end{array}
  \right. \nonumber
\end{eqnarray}

and $z$ is chosen distinct from $\quotep{P}$, $\quotep{Q}$, the free
names in $Q$, and all the names in $R$. Our $\alpha$-equivalence will
be built in the standard way from this substitution.

\begin{remark}\label{rem:no_self_referential_names}
  One consequence of these definitions is that $\forall P. \quotep{P}
  \not\in \freenames{P}$.
\end{remark}

\subsection{ Dynamic quote: an example }

Anticipating something of what's to come, consider applying the
substitution, $\widehat{\id{\{}u / z \id{\}}}$, to the following pair
of processes, $\lift{w}{y!(z)}$ and $w[ \lpquote y!(z) \rpquote ]$.

\begin{eqnarray}
	\lift{w}{y!(z)}\widehat{\id{\{}u / z \id{\}}}
		& = &
		\lift{w}{y!(u)} \nonumber\\
	w[ \lpquote y!(z) \rpquote ] \widehat{ \id{\{}u / z \id{\}} }
		& = &
		w[ \lpquote y!(z) \rpquote ] \nonumber
\end{eqnarray}

Because the body of the process between quotes is impervious to
substitution, we get radically different answers. In fact, by
examining the first process in an input context,
e.g. $x?(z).\lift{w}{y!(z)}$, we see that the process under the lift
operator may be shaped by prefixed inputs binding a name inside it. In
this sense, the lift operator will be seen as a way to dynamically
construct processes before reifying them as names.

Finally equipped with these standard features we can present the
dynamics of the calculus.

\subsubsection{Operational semantics} 

Finally, we introduce the computational dynamics. What marks these
algebras as distinct from other more traditionally studied algebraic
structures, e.g. vector spaces or polynomial rings, is the manner in
which dynamics is captured. In traditional structures, dynamics is typically
expressed through morphisms between such structures, as in linear maps
between vector spaces or morphisms between rings. In algebras
associated with the semantics of computation, the dynamics is
expressed as part of the algebraic structure itself, through a
reduction reduction relation typically denoted by $\red$. Below, we
give a recursive presentation of this relation for the calculus used
in the encoding.

$\red \subseteq \pi \times \pi$
$\red : \pi \to \mathcal{P}(\pi)$

\begin{mathpar}
  \inferrule* [lab=Comm] { \textsf{match}( x_{src}, x_{trgt} ) } { x_{trgt}?(y)P \; | \; x_{src}!\langle {Q} \rangle \red P\{\quotep{Q}/y}\} }
  \and \\
  \inferrule* [lab=Par] {{P} \red {P}'} {{{P} | {Q}} \red {{P}' | {Q}}}
  \and
  \inferrule* [lab=Equiv]{{{P} \scong {P}'} \andalso {{P}' \red {Q}'} \andalso {{Q}' \scong {Q}}}{{P} \red {Q}}
\end{mathpar}

\begin{eqnarray*}
  match_{\equiv} (\quotep{P},\quotep{Q}) & := & P \equiv Q \\
  match_{\dagger}(\quotep{P},\quotep{Q}) & := & \forall R. P|Q \red^{*} R => R \red^{*} 0 \\
  match_{K}(\quotep{P},\quotep{Q}) & := & K \mbox{ for some context } K
\end{eqnarray*}

$u?(x)P | u!\langle Q \rangle \red P\{\quotep{Q}/x\}$

%We write $\wred$ for $\red^*$, and $P\red$ if $\exists Q $ such that $ P \red Q$.
We write $P\red$ if $\exists Q $ such that $ P \red Q$ and $P\not\red$, otherwise.

\section{Replication}

As mentioned before, it is known that replication (and hence
recursion) can be implemented in a higher-order process algebra
\cite{SangiorgiWalker}. As our first example of calculation with the
machinery thus far presented we give the construction explicitly in
the {\rhoc}.

\begin{eqnarray}
	D_{x} & := & \prefix{x}{y}{(\binpar{\outputp{x}{y}}{@{y}})} \nonumber\\
	\bangp_{x}{P} & := & \binpar{{x}!\langle{\binpar{D_{x}}{P}}\rangle}{D_{x}} \nonumber
\end{eqnarray}

\begin{eqnarray}
	\bangp_{x}{P} & & \nonumber\\
	=
	& {x}!\langle{(\prefix{x}{y}{(\outputp{x}{y} | @{y})) | P}}\rangle 
	      | \prefix{x}{y}{(\outputp{x}{y} | @{y})} & \nonumber\\
	\red
	& (\outputp{x}{y} | @{y})\substn{\quotep{(\prefix{x}{y}{(@{y} | \outputp{x}{y})) | P}}}{y} & \nonumber\\
	=
	& \outputp{x}{\quotep{(\prefix{x}{y}{(\outputp{x}{y} | @{y})) | P}}}
	  | {(\prefix{x}{y}{(\outputp{x}{y} | @{y})) | P}} & \nonumber\\
	\red
	& \ldots & \nonumber\\
	\red^*
	& P | P | \ldots & \nonumber
\end{eqnarray}

Of course, this encoding, as an implementation, runs away, unfolding
$\bangp{P}$ eagerly. A lazier and more implementable replication
operator, restricted to input-guarded processes, may be obtained as follows.

\begin{eqnarray}
\bangp{\prefix{u}{v}{P}} 
	:= 
	\binpar{\lift{x}{\prefix{u}{v}{(\binpar{D(x)}{P})}}}{D(x)} \nonumber
\end{eqnarray}

\begin{remark}
  Note that the lazier definition still does not deal with summation
  or mixed summation (i.e. sums over input and output). The reader is
  invited to construct definitions of replication that deal with these
  features. 

  Further, the definitions are parameterized in a name, $x$. Can you,
  gentle reader, make a definition that eliminates this parameter and
  guarantees no accidental interaction between the replication
  machinery and the process being replicated -- i.e. no accidental
  sharing of names used by the process to get its work done and the
  name(s) used by the replication to effect copying. This latter
  revision of the definition of replication is crucial to obtaining
  the expected identity $!!P \sim !P$.
\end{remark}

\begin{remark}\label{rem:paradoxical_combinator}
  The reader familiar with the lambda calculus will have noticed the
  similarity between $D$ and the paradoxical combinator.

  [Ed. note: the existence of this seems to suggest we have to be more
  restrictive on the set of processes and names we admit if we are to
  support no-cloning.]
\end{remark}

\subsubsection{Bisimulation}

The computational dynamics gives rise to another kind of equivalence,
the equivalence of computational behavior. As previously mentioned
this is typically captured \emph{via} some form of bisimulation.

% The notion we use in this paper is weak barbed bisimulation
% \cite{milner91polyadicpi}.

The notion we use in this paper is derived from weak barbed
bisimulation \cite{milner91polyadicpi}. 

\begin{definition}
An \emph{observation relation}, $\downarrow_{\mathcal N}$, over a set
of names, $\mathcal N$, is the smallest relation satisfying the rules
below.

\infrule[Out-barb]{y \in {\mathcal N}, \; x \nameeq y}
		  {\outputp{x}{v} \downarrow_{\mathcal N} x}
\infrule[Par-barb]{\mbox{$P\downarrow_{\mathcal N} x$ or $Q\downarrow_{\mathcal N} x$}}
		  {\binpar{P}{Q} \downarrow_{\mathcal N} x}

We write $P \Downarrow_{\mathcal N} x$ if there is $Q$ such that 
$P \wred Q$ and $Q \downarrow_{\mathcal N} x$.
\end{definition}

\begin{definition}
%\label{def.bbisim}
An  ${\mathcal N}$-\emph{barbed bisimulation} over a set of names, ${\mathcal N}$, is a symmetric binary relation 
${\mathcal S}_{\mathcal N}$ between agents such that $P\rel{S}_{\mathcal N}Q$ implies:
\begin{enumerate}
\item If $P \red P'$ then $Q \wred Q'$ and $P'\rel{S}_{\mathcal N} Q'$.
\item If $P\downarrow_{\mathcal N} x$, then $Q\Downarrow_{\mathcal N} x$.
\end{enumerate}
$P$ is ${\mathcal N}$-barbed bisimilar to $Q$, written
$P \wbbisim_{\mathcal N} Q$, if $P \rel{S}_{\mathcal N} Q$ for some ${\mathcal N}$-barbed bisimulation ${\mathcal S}_{\mathcal N}$.
\end{definition}

$\mathcal{R} \subseteq \pi \times \pi$

$P \mathcal{R} Q => \forall P'. P \red P' \Rightarrow \exists Q'. Q \red Q', P' \mathcal{R} Q'$

$P \vdash x \Rightarrow Q \vdash x$

\begin{mathpar}
  \inferrule*[lab=Out-barb]{x \nameeq y}{{y}!\langle{Q}\rangle \vdash x}
  \and
  \inferrule*[lab=Par-barb]{\mbox{$P\vdash x$ or $Q\vdash x$}}{\binpar{P}{Q} \vdash x}
\end{mathpar}

\subsubsection{Contexts}

One of the principle advantages of computational calculi like the
$\pi$-calculus is a well-defined notion of context,
contextual-equivalence and a correlation between
contextual-equivalence and notions of bisimulation. The notion of
context allows the decomposition of a process into (sub-)process and
its syntactic environment, its context. Thus, a context may be
thought of as a process with a ``hole'' (written $\Box$) in it. The
application of a context $M$ to a process $P$, written $M[P]$, is
tantamount to filling the hole in $M$ with $P$. In this paper we do
not need the full weight of this theory, but do make use of the notion
of context in the proof the main theorem. 

\begin{mathpar}
  \inferrule* [lab=summation] {} {{M_{M},M_{N}} \bc \Box \;|\; x.M_{A} \;|\; M_{M}+M_{N}}
  \and
  \inferrule* [lab=agent] {} {{M_{A}} \bc (\vec{x})M_{P} \;| \; \clift{P_0,\ldots,M_{P},\ldots,P_N}}
  \and \\
  \inferrule* [lab=process] {} {{M_{P}} \bc M_{N} \;| \;P|M_{P} }
\end{mathpar} 

\begin{mathpar}
  \inferrule* [lab=sychronization] {} {M_{N} \bc \Box \;|\; x?M_{F} \;|\; x!M_{C}}
  \and
  \inferrule* [lab=abstraction] {} {{M_{F}} \bc (x)M_{P} }
  \and
  \inferrule* [lab=concretion] {} {{M_{C}} \bc \langle M_{P} \rangle }
  \and \\
  \inferrule* [lab=process] {} {{M_{P}} \bc M_{N} \;| \;P|M_{P} }
\end{mathpar}

\begin{definition}[contextual application] Given a context $M$, and
  process $P$, we define the \emph{contextual application}, $M[P] :=
  M\{P/\Box\}$. That is, the contextual application of M to P is the
  substitution of $P$ for $\Box$ in $M$.
\end{definition}

$\meaningof{-} : L \to \mathcal{P}(\pi)$

\begin{mathpar}
  \inferrule* [lab=collection] {} {\meaningof{true} = \pi, \and \meaningof{~E} = \pi \setminus \meaningof{E}, \and \meaningof{E_{1} \& E_{2}} = \meaningof{E_{1}} \cap \meaningof{E_{2}}}
\end{mathpar}

\begin{mathpar}
  \inferrule* [lab=structure] {} {\meaningof{0} = \{ P \in \pi | P \equiv 0 \}, \and \\ \meaningof{E_1 | E_2} = \{ P \in \pi | P \equiv P_{1} | P_{2}, P_{1} \in \meaningof{E_{1}}, P_{2} \in \meaningof{E_2}\} }
\end{mathpar}

\begin{mathpar}
 \inferrule* [lab=behavior] {} {\meaningof{\langle a?b \rangle E} = \{ P \in \pi | P \equiv Q | u?(y)P', \\ \and \\\\ \and \\ \;\;\; u \in \meaningof{a}, \forall z.P'\{z/y\} \in \meaningof{E\{z/b\}}\}, \and \\ \meaningof{a!E} = \{ P \in \pi | P \equiv Q | x!\langle P' \rangle, x \in \meaningof{a} P' \in \meaningof{E}\} }
\end{mathpar}

\begin{mathpar}
 \inferrule* [lab=nominal] {} {\meaningof{\quotep{E}} = \{ \quotep{P} \in \quotep{\pi} | P \in \meaningof{E} \}, \and \meaningof{\quotep{P}} = \{ \quotep{Q} \in \quotep{\pi} | P \equiv Q \} \and \\ \meaningof{@\quotep{E}} = \{ P \in \pi | P \equiv @x, x \in \meaningof{E} \}}
\end{mathpar}

\begin{eqnarray*}
  \\
  \meaningof{-} : TS \to ST
\end{eqnarray*}

\begin{eqnarray*}
  \\
  L : TS \to ST
\end{eqnarray*}

\begin{eqnarray*}
  \\
  P \models E \iff P \in \meaningof{E}
\end{eqnarray*}

\begin{eqnarray*}
  P \approx_{L} Q \iff \forall E \in L. P \models E \iff Q \models E
\end{eqnarray*}

\begin{eqnarray*}
  P \approx_{K} Q
\end{eqnarray*}

\begin{eqnarray*}
  P \approx Q
\end{eqnarray*}

$\approx_{K} = \approx = \approx_{L}$

\subsubsection{Contextual duality}

Note that contexts extend the quotation operation to a family of
operations from processes to names. Given a context, $M$, we can
define a \emph{nominal context}, $\quotep{M}$ by $\quotep{M}[P] :=
\quotep{M[P]}$. To foreshadow what is to come we observe that these
operations enjoy a duality with processes very much like the duality
between vectors and maps from vectors to scalars.

Further, because the calculus is essentially higher-order, we have a
correspondence between contexts and processes. More specifically,
given a name $x$ and a context $M$ we can construct $M^{*}_{x}$ such
that 

\begin{mathpar}
  M^{*}_{x} | \lift{x}{P} \red M[P]
\end{mathpar}

namely,

\begin{mathpar}
  M^{*}_{x} := x?(u).M[\dropn{u}]
\end{mathpar}

The dependence of $M^{*}_{x}$ on a name makes it an abstraction, 

\begin{mathpar}
  M^{*} := (x)x?(u).M[\dropn{u}]
\end{mathpar}

\subsection{Additional notation}

It will sometimes be convenient to denote the process a name
quotes. We already have the notation $x = \quotep{P}$, but it will be
convenient to introduce an alternate notation, $\procn{x}$, when we
want to emphasize the connection to the use of the name. Note that, by
virtue of name equivalence, $\quotep{\procn{x}} \nameeq x$; so, the
notation is consistent with previous definitions.

Further, because names have structure it is possible to effect
substitutions on the basis of that structure. This means we need to
upgrade our notation for substitutions, which we accomplish by
adapting comprehension notation. Thus,

\begin{mathpar}
  P\{ y / x : x \in S \}
\end{mathpar}

is interpreted to mean the process derived from P by replacing (in a
capture-avoiding manner) each occurrence of $x$ in $S$ by $y$. For example,

\begin{mathpar}
  P\{ \quotep{\procn{x}|\procn{x}} / x : x \in \freenames{P} \}
\end{mathpar}

will replace each (occurrence) of a free name $x$ in $P$ by
$\quotep{\procn{x}|\procn{x}}$.

Also, we will avail ourselves of the notation $x^{L}$ and $x^{R}$ to
denote injections of a name into disjoint copies of the name
space. There are numerous ways to accomplish this. One example can be
found in \cite{MeredithR05}. This notation overloads to vectors of
names: $\vec{x}^{\pi} := (x_{i}^{\pi} \; : \; 0 \leq i < |\vec{x}| )$ where $\pi \in \{L,R\}$.

We also use $P^{\Box} := P|\Box$.

In \cite{MeredithR05} an interpretation of the new operator is
given. It turns out that there are several possible interpretations
all enjoying the requisite algebraic properties of the operator (see
\cite{milner91polyadicpi}). We will therefore make liberal use of
$(\nu\; \vec{x})P$.

% subsection the_syntax_and_semantics_of_the_notation_system (end)   

\input{qm2pi.qmops} 

\input{qm2pi.sterngerlach} 

\input{qm2pi.metric} 

% section concurrent_process_calculi (end)

%\input{qm2pi.proofsketch}

% section proof sketch (end)

%\input{qm2pi.slviaknots} 

% section spatial logic via knots (end)

\input{qm2pi.conclusion}

% section conclusion (end)

%\input{qm2pi.dtcodes} 

% section wiring algorithm (end)

\input{qm2pi.ack} 

% section acknowledgments (end)

\newpage


\bibliographystyle{plain}   
\bibliography{../../biblios/main.bib}

\input{qm2pi.rhodetails}

\end{document}

 

% section wiring algorithm (end)

\documentclass[12pt]{llncs}
%\documentclass{jktr}

\usepackage[pdftex]{hyperref}                   
\usepackage {listings}
\usepackage {mathpartir}
\usepackage{bcprules}
%\usepackage{listings}
                       
\usepackage{graphicx} 
%\usepackage[margins=2.5cm,nohead,nofoot]{geometry}
%\usepackage{geometry}
\usepackage{amsfonts}
\usepackage{amstext}
\usepackage{latexsym}
\usepackage{amssymb}
\usepackage{color}


%\include{myPreamble}
\include{qm2pi.local} 

%\ifpdf
%\usepackage[pdftex]{graphicx}
%\else
%\usepackage{graphicx}
%\fi

 % \ifpdf
%  \usepackage{pdfsync}
%  \if


%\title{Brief Article}
%\author{David F. Snyder}
%\author{L.G. Meredith}

%\address{Dept. of Math., Texas State University--San Marcos, San Marcos, TX 78666}
       
\pagestyle{empty}


\begin{document}

\lstset{language=[Objective]Caml,frame=shadowbox}

\input{qm2pi.front}

% section front matter (end)

\input{qm2pi.intro} 
 
% section introduction (end)

% \input{qm2pi.knotations} 

% section notation (end)

\input{qm2pi.process.calculi} 

% section concurrent_process_calculi_and_spatial_logics_ (end)
    
%\input{qm2pi.knots2pi} 

%\input{qm2pi.trefoil} 

%\input{qm2pi.mainthm} 

% subsection basic_interpretation (end)

%\input{qm2pi.rho.presentation} 
\subsection{The syntax and semantics of the notation system}\label{sub:the_syntax_and_semantics_of_the_notation_system} % (fold)

We now summarize a technical presentation of the calculus that
embodies our theory of dynamics. The typical presentation of such a
calculus follows the style of giving generators and relations on
them. The grammar, below, describing term constructors, freely
generates the set of processes, $\Proc$. This set is then quotiented
by a relation known as structural congruence and it is over this set
that the notion of dynamics is expressed. This presentation is
essentially that of \cite{MeredithR05} with the addition of
polyadicity and summation. For readability we have relegated some of
the technical subtleties to an appendix.

\subsubsection{Process grammar}\label{subsub:process_grammar}

\begin{mathpar}
  \inferrule* [lab=synchronization] {} {{M} \bc \pzero \;|\; x?F \;|\; x!C }
  \and
  \inferrule* [lab=abstraction] {} {{F} \bc (x)P}
  \and
  \inferrule* [lab=concretion] {} {{C} \bc \langle Q \rangle}
  \and
  \inferrule* [lab=process] {} {{P,Q} \bc M \;| \;P|Q \;|\; @{x}}
  \and
  \inferrule* [lab=name] {} {{x} \bc \quotep{P}}
\end{mathpar} 

Note that $\vec{x}$ (resp. $\vec{P}$) denotes a vector of names
(resp. processes) of length $|\vec{x}|$ (resp. $|\vec{P}|$). We adopt
the following useful abbreviations.

\begin{mathpar}
   x?(\vec{y}).P := x.(\vec{y})P \and  x\clift{\vec{P}} := x.\clift{\vec{P}}
   \and x!(y) := \lift{x}{\dropn{y}}
   \and \Pi_{i=0}^{n-1}P_i := P_0 | \ldots | P_{n-1}
\end{mathpar}

\subsubsection{Structural congruence}

\paragraph{Free and bound names and alpha-equivalence.} At the
core of structural equivalence is alpha-equivalence which identifies
process that are the same up to a change of variable. Formally, we
recognize the distinction between free and bound names. The free names
of a process, $\freenames{P}$, may be calculated recursively as
follows:

\begin{mathpar}
\freenames{\pzero} := \emptyset
  \and \\
  \freenames{x?(y).P} := \{ x \} \cup (\freenames{P} \setminus \{ y \})
  \and 
  \freenames{x!\langle P \rangle} := \{ x \} \cup \{ P \} 
  \and \\
  \freenames{P|Q} := \freenames{P} \cup \freenames{Q}
  \and \\
  \freenames{@{x}} := \{ x \}
\end{mathpar}

$\pi$
$\quotep{\pi}$

$\freenames{-} : \pi \to \mathcal{P}(\quotep{\pi})$

\begin{eqnarray*}
  \freenames{\pzero} & := & \emptyset \\
  \freenames{x?(y).P} & := & \{ x \} \cup (\freenames{P} \setminus \{ y \}) \\
  \freenames{x!\langle P \rangle} & := & \{ x \} \cup \{ P \} \\
  \freenames{P|Q} & := & \freenames{P} \cup \freenames{Q} \\
  \freenames{\dropn{x}} & := & \{ x \}
\end{eqnarray*}

The bound names of a process, $\boundnames{P}$, are those names occurring in $P$
that are not free. For example, in $x?(y).0$, the name $x$ is free, while $y$ is bound.

\begin{mathpar}
  \inferrule* [lab=monoidal-laws] {} { P|Q \equiv Q|P \and P|0 \equiv P \and P|(Q|R) \equiv (P|Q)|R }
\end{mathpar}

\begin{mathpar}
  \inferrule* [lab=alpha-equivalence] {} { (x)P \equiv (y)P\{y/x\} \and y \not\in \freenames{P} }
\end{mathpar}

\begin{definition}
Then two processes, $P,Q$, are alpha-equivalent if $P = Q\{\vec{y}/\vec{x}\}$ for
some $\vec{x} \in \boundnames{Q},\vec{y} \in \boundnames{P}$, where $Q\{\vec{y}/\vec{x}\}$
denotes the capture-avoiding substitution of $\vec{y}$ for $\vec{x}$ in $Q$.
\end{definition}

\begin{definition}
  The {\em structural congruence} \cite{SangiorgiWalker} , $\equiv$,
  between processes is the least congruence containing
  alpha-equivalence, satisfying the abelian monoid laws
  (associativity, commutativity and $\pzero$ as identity) for parallel
  composition $|$ and for summation $+$.
\end{definition}

\subsection{Name equivalence}

We take name equivalence, written $\nameeq$, to be the smallest
equivalence relation generated by the following rules.

\begin{mathpar}
\inferrule*[lab=Quote-drop]
{ }
{ \quotep{@{x}} \nameeq x }

\inferrule*[lab=Struct-equiv]
{ P \scong Q }
{ \quotep{P} \nameeq \quotep{Q} }
\end{mathpar}

The astute reader will have noticed that the mutual recursion of names
and processes imposes a mutual recursion on alpha-equivalence and
structural equivalence via name-equivalence. Fortunately, all of this
works out pleasantly and we may calculate in the natural way, free of
concern. The reader interested in the details is referred to the
appendix \ref{appendix:rho_details}.

\subsection{Substitution}

We use $\Proc$ for the set of processes, $\QProc$ for the set of
names, and $\id{\{}\vec{y} / \vec{x} \id{\}}$ to denote partial maps,
$s : \QProc \rightarrow \QProc$. A map, $s$ lifts, uniquely, to a map
on process terms, $\widehat{s} : \Proc \rightarrow \Proc$ by the
following equations.

\begin{mathpar}
  (0) \psubstp{Q}{P} := 0 \\
  (R \juxtap S) \psubstp{Q}{P}
  :=    
  (R)\psubstp{Q}{P} \juxtap (S) \psubstp{Q}{P} \\
  (x?(y).R) \psubstp{Q}{P}    
  :=    
  (x)\substp{Q}{P} (z)\concat( (R \psubstn{z}{y}) \psubstp{Q}{P} ) \\
  (\lift{x}{R}) \psubstp{Q}{P}  
  :=
  \lift{(x)\substp{Q}{P}}{ R \psubstp{Q}{P} } \\
%   (\dropn{x})  \psubstp{Q}{P}       
%   := 
%   \left\{ 
%     \begin{array}{ccc} 
%       \dropn{\quotep{Q}} & & x \nameeq \quotep{P} \\
%       \dropn{x} & & otherwise \\
%     \end{array}
%   \right. 
  (\dropn{x})  \psubstp{Q}{P}       
  := 
  \left\{ 
    \begin{array}{ccc} 
      Q & & x \nameeq \quotep{P} \\
      \dropn{x} & & otherwise \\
    \end{array}
  \right.
\end{mathpar}
 

where

\begin{eqnarray}
  (x)\id{\{} \lpquote Q \rpquote / \lpquote P \rpquote \id{\}}            = 
  \left\{ 
    \begin{array}{ccc}
      \lpquote Q \rpquote & & x \nameeq \lpquote P \rpquote \\
      x & & otherwise \\
    \end{array}
  \right. \nonumber
\end{eqnarray}

and $z$ is chosen distinct from $\quotep{P}$, $\quotep{Q}$, the free
names in $Q$, and all the names in $R$. Our $\alpha$-equivalence will
be built in the standard way from this substitution.

\begin{remark}\label{rem:no_self_referential_names}
  One consequence of these definitions is that $\forall P. \quotep{P}
  \not\in \freenames{P}$.
\end{remark}

\subsection{ Dynamic quote: an example }

Anticipating something of what's to come, consider applying the
substitution, $\widehat{\id{\{}u / z \id{\}}}$, to the following pair
of processes, $\lift{w}{y!(z)}$ and $w[ \lpquote y!(z) \rpquote ]$.

\begin{eqnarray}
	\lift{w}{y!(z)}\widehat{\id{\{}u / z \id{\}}}
		& = &
		\lift{w}{y!(u)} \nonumber\\
	w[ \lpquote y!(z) \rpquote ] \widehat{ \id{\{}u / z \id{\}} }
		& = &
		w[ \lpquote y!(z) \rpquote ] \nonumber
\end{eqnarray}

Because the body of the process between quotes is impervious to
substitution, we get radically different answers. In fact, by
examining the first process in an input context,
e.g. $x?(z).\lift{w}{y!(z)}$, we see that the process under the lift
operator may be shaped by prefixed inputs binding a name inside it. In
this sense, the lift operator will be seen as a way to dynamically
construct processes before reifying them as names.

Finally equipped with these standard features we can present the
dynamics of the calculus.

\subsubsection{Operational semantics} 

Finally, we introduce the computational dynamics. What marks these
algebras as distinct from other more traditionally studied algebraic
structures, e.g. vector spaces or polynomial rings, is the manner in
which dynamics is captured. In traditional structures, dynamics is typically
expressed through morphisms between such structures, as in linear maps
between vector spaces or morphisms between rings. In algebras
associated with the semantics of computation, the dynamics is
expressed as part of the algebraic structure itself, through a
reduction reduction relation typically denoted by $\red$. Below, we
give a recursive presentation of this relation for the calculus used
in the encoding.

$\red \subseteq \pi \times \pi$
$\red : \pi \to \mathcal{P}(\pi)$

\begin{mathpar}
  \inferrule* [lab=Comm] { \textsf{match}( x_{src}, x_{trgt} ) } { x_{trgt}?(y)P \; | \; x_{src}!\langle {Q} \rangle \red P\{\quotep{Q}/y}\} }
  \and \\
  \inferrule* [lab=Par] {{P} \red {P}'} {{{P} | {Q}} \red {{P}' | {Q}}}
  \and
  \inferrule* [lab=Equiv]{{{P} \scong {P}'} \andalso {{P}' \red {Q}'} \andalso {{Q}' \scong {Q}}}{{P} \red {Q}}
\end{mathpar}

\begin{eqnarray*}
  match_{\equiv} (\quotep{P},\quotep{Q}) & := & P \equiv Q \\
  match_{\dagger}(\quotep{P},\quotep{Q}) & := & \forall R. P|Q \red^{*} R => R \red^{*} 0 \\
  match_{K}(\quotep{P},\quotep{Q}) & := & K \mbox{ for some context } K
\end{eqnarray*}

$u?(x)P | u!\langle Q \rangle \red P\{\quotep{Q}/x\}$

%We write $\wred$ for $\red^*$, and $P\red$ if $\exists Q $ such that $ P \red Q$.
We write $P\red$ if $\exists Q $ such that $ P \red Q$ and $P\not\red$, otherwise.

\section{Replication}

As mentioned before, it is known that replication (and hence
recursion) can be implemented in a higher-order process algebra
\cite{SangiorgiWalker}. As our first example of calculation with the
machinery thus far presented we give the construction explicitly in
the {\rhoc}.

\begin{eqnarray}
	D_{x} & := & \prefix{x}{y}{(\binpar{\outputp{x}{y}}{@{y}})} \nonumber\\
	\bangp_{x}{P} & := & \binpar{{x}!\langle{\binpar{D_{x}}{P}}\rangle}{D_{x}} \nonumber
\end{eqnarray}

\begin{eqnarray}
	\bangp_{x}{P} & & \nonumber\\
	=
	& {x}!\langle{(\prefix{x}{y}{(\outputp{x}{y} | @{y})) | P}}\rangle 
	      | \prefix{x}{y}{(\outputp{x}{y} | @{y})} & \nonumber\\
	\red
	& (\outputp{x}{y} | @{y})\substn{\quotep{(\prefix{x}{y}{(@{y} | \outputp{x}{y})) | P}}}{y} & \nonumber\\
	=
	& \outputp{x}{\quotep{(\prefix{x}{y}{(\outputp{x}{y} | @{y})) | P}}}
	  | {(\prefix{x}{y}{(\outputp{x}{y} | @{y})) | P}} & \nonumber\\
	\red
	& \ldots & \nonumber\\
	\red^*
	& P | P | \ldots & \nonumber
\end{eqnarray}

Of course, this encoding, as an implementation, runs away, unfolding
$\bangp{P}$ eagerly. A lazier and more implementable replication
operator, restricted to input-guarded processes, may be obtained as follows.

\begin{eqnarray}
\bangp{\prefix{u}{v}{P}} 
	:= 
	\binpar{\lift{x}{\prefix{u}{v}{(\binpar{D(x)}{P})}}}{D(x)} \nonumber
\end{eqnarray}

\begin{remark}
  Note that the lazier definition still does not deal with summation
  or mixed summation (i.e. sums over input and output). The reader is
  invited to construct definitions of replication that deal with these
  features. 

  Further, the definitions are parameterized in a name, $x$. Can you,
  gentle reader, make a definition that eliminates this parameter and
  guarantees no accidental interaction between the replication
  machinery and the process being replicated -- i.e. no accidental
  sharing of names used by the process to get its work done and the
  name(s) used by the replication to effect copying. This latter
  revision of the definition of replication is crucial to obtaining
  the expected identity $!!P \sim !P$.
\end{remark}

\begin{remark}\label{rem:paradoxical_combinator}
  The reader familiar with the lambda calculus will have noticed the
  similarity between $D$ and the paradoxical combinator.

  [Ed. note: the existence of this seems to suggest we have to be more
  restrictive on the set of processes and names we admit if we are to
  support no-cloning.]
\end{remark}

\subsubsection{Bisimulation}

The computational dynamics gives rise to another kind of equivalence,
the equivalence of computational behavior. As previously mentioned
this is typically captured \emph{via} some form of bisimulation.

% The notion we use in this paper is weak barbed bisimulation
% \cite{milner91polyadicpi}.

The notion we use in this paper is derived from weak barbed
bisimulation \cite{milner91polyadicpi}. 

\begin{definition}
An \emph{observation relation}, $\downarrow_{\mathcal N}$, over a set
of names, $\mathcal N$, is the smallest relation satisfying the rules
below.

\infrule[Out-barb]{y \in {\mathcal N}, \; x \nameeq y}
		  {\outputp{x}{v} \downarrow_{\mathcal N} x}
\infrule[Par-barb]{\mbox{$P\downarrow_{\mathcal N} x$ or $Q\downarrow_{\mathcal N} x$}}
		  {\binpar{P}{Q} \downarrow_{\mathcal N} x}

We write $P \Downarrow_{\mathcal N} x$ if there is $Q$ such that 
$P \wred Q$ and $Q \downarrow_{\mathcal N} x$.
\end{definition}

\begin{definition}
%\label{def.bbisim}
An  ${\mathcal N}$-\emph{barbed bisimulation} over a set of names, ${\mathcal N}$, is a symmetric binary relation 
${\mathcal S}_{\mathcal N}$ between agents such that $P\rel{S}_{\mathcal N}Q$ implies:
\begin{enumerate}
\item If $P \red P'$ then $Q \wred Q'$ and $P'\rel{S}_{\mathcal N} Q'$.
\item If $P\downarrow_{\mathcal N} x$, then $Q\Downarrow_{\mathcal N} x$.
\end{enumerate}
$P$ is ${\mathcal N}$-barbed bisimilar to $Q$, written
$P \wbbisim_{\mathcal N} Q$, if $P \rel{S}_{\mathcal N} Q$ for some ${\mathcal N}$-barbed bisimulation ${\mathcal S}_{\mathcal N}$.
\end{definition}

$\mathcal{R} \subseteq \pi \times \pi$

$P \mathcal{R} Q => \forall P'. P \red P' \Rightarrow \exists Q'. Q \red Q', P' \mathcal{R} Q'$

$P \vdash x \Rightarrow Q \vdash x$

\begin{mathpar}
  \inferrule*[lab=Out-barb]{x \nameeq y}{{y}!\langle{Q}\rangle \vdash x}
  \and
  \inferrule*[lab=Par-barb]{\mbox{$P\vdash x$ or $Q\vdash x$}}{\binpar{P}{Q} \vdash x}
\end{mathpar}

\subsubsection{Contexts}

One of the principle advantages of computational calculi like the
$\pi$-calculus is a well-defined notion of context,
contextual-equivalence and a correlation between
contextual-equivalence and notions of bisimulation. The notion of
context allows the decomposition of a process into (sub-)process and
its syntactic environment, its context. Thus, a context may be
thought of as a process with a ``hole'' (written $\Box$) in it. The
application of a context $M$ to a process $P$, written $M[P]$, is
tantamount to filling the hole in $M$ with $P$. In this paper we do
not need the full weight of this theory, but do make use of the notion
of context in the proof the main theorem. 

\begin{mathpar}
  \inferrule* [lab=summation] {} {{M_{M},M_{N}} \bc \Box \;|\; x.M_{A} \;|\; M_{M}+M_{N}}
  \and
  \inferrule* [lab=agent] {} {{M_{A}} \bc (\vec{x})M_{P} \;| \; \clift{P_0,\ldots,M_{P},\ldots,P_N}}
  \and \\
  \inferrule* [lab=process] {} {{M_{P}} \bc M_{N} \;| \;P|M_{P} }
\end{mathpar} 

\begin{mathpar}
  \inferrule* [lab=sychronization] {} {M_{N} \bc \Box \;|\; x?M_{F} \;|\; x!M_{C}}
  \and
  \inferrule* [lab=abstraction] {} {{M_{F}} \bc (x)M_{P} }
  \and
  \inferrule* [lab=concretion] {} {{M_{C}} \bc \langle M_{P} \rangle }
  \and \\
  \inferrule* [lab=process] {} {{M_{P}} \bc M_{N} \;| \;P|M_{P} }
\end{mathpar}

\begin{definition}[contextual application] Given a context $M$, and
  process $P$, we define the \emph{contextual application}, $M[P] :=
  M\{P/\Box\}$. That is, the contextual application of M to P is the
  substitution of $P$ for $\Box$ in $M$.
\end{definition}

$\meaningof{-} : L \to \mathcal{P}(\pi)$

\begin{mathpar}
  \inferrule* [lab=collection] {} {\meaningof{true} = \pi, \and \meaningof{~E} = \pi \setminus \meaningof{E}, \and \meaningof{E_{1} \& E_{2}} = \meaningof{E_{1}} \cap \meaningof{E_{2}}}
\end{mathpar}

\begin{mathpar}
  \inferrule* [lab=structure] {} {\meaningof{0} = \{ P \in \pi | P \equiv 0 \}, \and \\ \meaningof{E_1 | E_2} = \{ P \in \pi | P \equiv P_{1} | P_{2}, P_{1} \in \meaningof{E_{1}}, P_{2} \in \meaningof{E_2}\} }
\end{mathpar}

\begin{mathpar}
 \inferrule* [lab=behavior] {} {\meaningof{\langle a?b \rangle E} = \{ P \in \pi | P \equiv Q | u?(y)P', \\ \and \\\\ \and \\ \;\;\; u \in \meaningof{a}, \forall z.P'\{z/y\} \in \meaningof{E\{z/b\}}\}, \and \\ \meaningof{a!E} = \{ P \in \pi | P \equiv Q | x!\langle P' \rangle, x \in \meaningof{a} P' \in \meaningof{E}\} }
\end{mathpar}

\begin{mathpar}
 \inferrule* [lab=nominal] {} {\meaningof{\quotep{E}} = \{ \quotep{P} \in \quotep{\pi} | P \in \meaningof{E} \}, \and \meaningof{\quotep{P}} = \{ \quotep{Q} \in \quotep{\pi} | P \equiv Q \} \and \\ \meaningof{@\quotep{E}} = \{ P \in \pi | P \equiv @x, x \in \meaningof{E} \}}
\end{mathpar}

\begin{eqnarray*}
  \\
  \meaningof{-} : TS \to ST
\end{eqnarray*}

\begin{eqnarray*}
  \\
  L : TS \to ST
\end{eqnarray*}

\begin{eqnarray*}
  \\
  P \models E \iff P \in \meaningof{E}
\end{eqnarray*}

\begin{eqnarray*}
  P \approx_{L} Q \iff \forall E \in L. P \models E \iff Q \models E
\end{eqnarray*}

\begin{eqnarray*}
  P \approx_{K} Q
\end{eqnarray*}

\begin{eqnarray*}
  P \approx Q
\end{eqnarray*}

$\approx_{K} = \approx = \approx_{L}$

\subsubsection{Contextual duality}

Note that contexts extend the quotation operation to a family of
operations from processes to names. Given a context, $M$, we can
define a \emph{nominal context}, $\quotep{M}$ by $\quotep{M}[P] :=
\quotep{M[P]}$. To foreshadow what is to come we observe that these
operations enjoy a duality with processes very much like the duality
between vectors and maps from vectors to scalars.

Further, because the calculus is essentially higher-order, we have a
correspondence between contexts and processes. More specifically,
given a name $x$ and a context $M$ we can construct $M^{*}_{x}$ such
that 

\begin{mathpar}
  M^{*}_{x} | \lift{x}{P} \red M[P]
\end{mathpar}

namely,

\begin{mathpar}
  M^{*}_{x} := x?(u).M[\dropn{u}]
\end{mathpar}

The dependence of $M^{*}_{x}$ on a name makes it an abstraction, 

\begin{mathpar}
  M^{*} := (x)x?(u).M[\dropn{u}]
\end{mathpar}

\subsection{Additional notation}

It will sometimes be convenient to denote the process a name
quotes. We already have the notation $x = \quotep{P}$, but it will be
convenient to introduce an alternate notation, $\procn{x}$, when we
want to emphasize the connection to the use of the name. Note that, by
virtue of name equivalence, $\quotep{\procn{x}} \nameeq x$; so, the
notation is consistent with previous definitions.

Further, because names have structure it is possible to effect
substitutions on the basis of that structure. This means we need to
upgrade our notation for substitutions, which we accomplish by
adapting comprehension notation. Thus,

\begin{mathpar}
  P\{ y / x : x \in S \}
\end{mathpar}

is interpreted to mean the process derived from P by replacing (in a
capture-avoiding manner) each occurrence of $x$ in $S$ by $y$. For example,

\begin{mathpar}
  P\{ \quotep{\procn{x}|\procn{x}} / x : x \in \freenames{P} \}
\end{mathpar}

will replace each (occurrence) of a free name $x$ in $P$ by
$\quotep{\procn{x}|\procn{x}}$.

Also, we will avail ourselves of the notation $x^{L}$ and $x^{R}$ to
denote injections of a name into disjoint copies of the name
space. There are numerous ways to accomplish this. One example can be
found in \cite{MeredithR05}. This notation overloads to vectors of
names: $\vec{x}^{\pi} := (x_{i}^{\pi} \; : \; 0 \leq i < |\vec{x}| )$ where $\pi \in \{L,R\}$.

We also use $P^{\Box} := P|\Box$.

In \cite{MeredithR05} an interpretation of the new operator is
given. It turns out that there are several possible interpretations
all enjoying the requisite algebraic properties of the operator (see
\cite{milner91polyadicpi}). We will therefore make liberal use of
$(\nu\; \vec{x})P$.

% subsection the_syntax_and_semantics_of_the_notation_system (end)   

\input{qm2pi.qmops} 

\input{qm2pi.sterngerlach} 

\input{qm2pi.metric} 

% section concurrent_process_calculi (end)

%\input{qm2pi.proofsketch}

% section proof sketch (end)

%\input{qm2pi.slviaknots} 

% section spatial logic via knots (end)

\input{qm2pi.conclusion}

% section conclusion (end)

%\input{qm2pi.dtcodes} 

% section wiring algorithm (end)

\input{qm2pi.ack} 

% section acknowledgments (end)

\newpage


\bibliographystyle{plain}   
\bibliography{../../biblios/main.bib}

\input{qm2pi.rhodetails}

\end{document}

 

% section acknowledgments (end)

\newpage


\bibliographystyle{plain}   
\bibliography{../../biblios/main.bib}

\documentclass[12pt]{llncs}
%\documentclass{jktr}

\usepackage[pdftex]{hyperref}                   
\usepackage {listings}
\usepackage {mathpartir}
\usepackage{bcprules}
%\usepackage{listings}
                       
\usepackage{graphicx} 
%\usepackage[margins=2.5cm,nohead,nofoot]{geometry}
%\usepackage{geometry}
\usepackage{amsfonts}
\usepackage{amstext}
\usepackage{latexsym}
\usepackage{amssymb}
\usepackage{color}


%\include{myPreamble}
\include{qm2pi.local} 

%\ifpdf
%\usepackage[pdftex]{graphicx}
%\else
%\usepackage{graphicx}
%\fi

 % \ifpdf
%  \usepackage{pdfsync}
%  \if


%\title{Brief Article}
%\author{David F. Snyder}
%\author{L.G. Meredith}

%\address{Dept. of Math., Texas State University--San Marcos, San Marcos, TX 78666}
       
\pagestyle{empty}


\begin{document}

\lstset{language=[Objective]Caml,frame=shadowbox}

\input{qm2pi.front}

% section front matter (end)

\input{qm2pi.intro} 
 
% section introduction (end)

% \input{qm2pi.knotations} 

% section notation (end)

\input{qm2pi.process.calculi} 

% section concurrent_process_calculi_and_spatial_logics_ (end)
    
%\input{qm2pi.knots2pi} 

%\input{qm2pi.trefoil} 

%\input{qm2pi.mainthm} 

% subsection basic_interpretation (end)

%\input{qm2pi.rho.presentation} 
\subsection{The syntax and semantics of the notation system}\label{sub:the_syntax_and_semantics_of_the_notation_system} % (fold)

We now summarize a technical presentation of the calculus that
embodies our theory of dynamics. The typical presentation of such a
calculus follows the style of giving generators and relations on
them. The grammar, below, describing term constructors, freely
generates the set of processes, $\Proc$. This set is then quotiented
by a relation known as structural congruence and it is over this set
that the notion of dynamics is expressed. This presentation is
essentially that of \cite{MeredithR05} with the addition of
polyadicity and summation. For readability we have relegated some of
the technical subtleties to an appendix.

\subsubsection{Process grammar}\label{subsub:process_grammar}

\begin{mathpar}
  \inferrule* [lab=synchronization] {} {{M} \bc \pzero \;|\; x?F \;|\; x!C }
  \and
  \inferrule* [lab=abstraction] {} {{F} \bc (x)P}
  \and
  \inferrule* [lab=concretion] {} {{C} \bc \langle Q \rangle}
  \and
  \inferrule* [lab=process] {} {{P,Q} \bc M \;| \;P|Q \;|\; @{x}}
  \and
  \inferrule* [lab=name] {} {{x} \bc \quotep{P}}
\end{mathpar} 

Note that $\vec{x}$ (resp. $\vec{P}$) denotes a vector of names
(resp. processes) of length $|\vec{x}|$ (resp. $|\vec{P}|$). We adopt
the following useful abbreviations.

\begin{mathpar}
   x?(\vec{y}).P := x.(\vec{y})P \and  x\clift{\vec{P}} := x.\clift{\vec{P}}
   \and x!(y) := \lift{x}{\dropn{y}}
   \and \Pi_{i=0}^{n-1}P_i := P_0 | \ldots | P_{n-1}
\end{mathpar}

\subsubsection{Structural congruence}

\paragraph{Free and bound names and alpha-equivalence.} At the
core of structural equivalence is alpha-equivalence which identifies
process that are the same up to a change of variable. Formally, we
recognize the distinction between free and bound names. The free names
of a process, $\freenames{P}$, may be calculated recursively as
follows:

\begin{mathpar}
\freenames{\pzero} := \emptyset
  \and \\
  \freenames{x?(y).P} := \{ x \} \cup (\freenames{P} \setminus \{ y \})
  \and 
  \freenames{x!\langle P \rangle} := \{ x \} \cup \{ P \} 
  \and \\
  \freenames{P|Q} := \freenames{P} \cup \freenames{Q}
  \and \\
  \freenames{@{x}} := \{ x \}
\end{mathpar}

$\pi$
$\quotep{\pi}$

$\freenames{-} : \pi \to \mathcal{P}(\quotep{\pi})$

\begin{eqnarray*}
  \freenames{\pzero} & := & \emptyset \\
  \freenames{x?(y).P} & := & \{ x \} \cup (\freenames{P} \setminus \{ y \}) \\
  \freenames{x!\langle P \rangle} & := & \{ x \} \cup \{ P \} \\
  \freenames{P|Q} & := & \freenames{P} \cup \freenames{Q} \\
  \freenames{\dropn{x}} & := & \{ x \}
\end{eqnarray*}

The bound names of a process, $\boundnames{P}$, are those names occurring in $P$
that are not free. For example, in $x?(y).0$, the name $x$ is free, while $y$ is bound.

\begin{mathpar}
  \inferrule* [lab=monoidal-laws] {} { P|Q \equiv Q|P \and P|0 \equiv P \and P|(Q|R) \equiv (P|Q)|R }
\end{mathpar}

\begin{mathpar}
  \inferrule* [lab=alpha-equivalence] {} { (x)P \equiv (y)P\{y/x\} \and y \not\in \freenames{P} }
\end{mathpar}

\begin{definition}
Then two processes, $P,Q$, are alpha-equivalent if $P = Q\{\vec{y}/\vec{x}\}$ for
some $\vec{x} \in \boundnames{Q},\vec{y} \in \boundnames{P}$, where $Q\{\vec{y}/\vec{x}\}$
denotes the capture-avoiding substitution of $\vec{y}$ for $\vec{x}$ in $Q$.
\end{definition}

\begin{definition}
  The {\em structural congruence} \cite{SangiorgiWalker} , $\equiv$,
  between processes is the least congruence containing
  alpha-equivalence, satisfying the abelian monoid laws
  (associativity, commutativity and $\pzero$ as identity) for parallel
  composition $|$ and for summation $+$.
\end{definition}

\subsection{Name equivalence}

We take name equivalence, written $\nameeq$, to be the smallest
equivalence relation generated by the following rules.

\begin{mathpar}
\inferrule*[lab=Quote-drop]
{ }
{ \quotep{@{x}} \nameeq x }

\inferrule*[lab=Struct-equiv]
{ P \scong Q }
{ \quotep{P} \nameeq \quotep{Q} }
\end{mathpar}

The astute reader will have noticed that the mutual recursion of names
and processes imposes a mutual recursion on alpha-equivalence and
structural equivalence via name-equivalence. Fortunately, all of this
works out pleasantly and we may calculate in the natural way, free of
concern. The reader interested in the details is referred to the
appendix \ref{appendix:rho_details}.

\subsection{Substitution}

We use $\Proc$ for the set of processes, $\QProc$ for the set of
names, and $\id{\{}\vec{y} / \vec{x} \id{\}}$ to denote partial maps,
$s : \QProc \rightarrow \QProc$. A map, $s$ lifts, uniquely, to a map
on process terms, $\widehat{s} : \Proc \rightarrow \Proc$ by the
following equations.

\begin{mathpar}
  (0) \psubstp{Q}{P} := 0 \\
  (R \juxtap S) \psubstp{Q}{P}
  :=    
  (R)\psubstp{Q}{P} \juxtap (S) \psubstp{Q}{P} \\
  (x?(y).R) \psubstp{Q}{P}    
  :=    
  (x)\substp{Q}{P} (z)\concat( (R \psubstn{z}{y}) \psubstp{Q}{P} ) \\
  (\lift{x}{R}) \psubstp{Q}{P}  
  :=
  \lift{(x)\substp{Q}{P}}{ R \psubstp{Q}{P} } \\
%   (\dropn{x})  \psubstp{Q}{P}       
%   := 
%   \left\{ 
%     \begin{array}{ccc} 
%       \dropn{\quotep{Q}} & & x \nameeq \quotep{P} \\
%       \dropn{x} & & otherwise \\
%     \end{array}
%   \right. 
  (\dropn{x})  \psubstp{Q}{P}       
  := 
  \left\{ 
    \begin{array}{ccc} 
      Q & & x \nameeq \quotep{P} \\
      \dropn{x} & & otherwise \\
    \end{array}
  \right.
\end{mathpar}
 

where

\begin{eqnarray}
  (x)\id{\{} \lpquote Q \rpquote / \lpquote P \rpquote \id{\}}            = 
  \left\{ 
    \begin{array}{ccc}
      \lpquote Q \rpquote & & x \nameeq \lpquote P \rpquote \\
      x & & otherwise \\
    \end{array}
  \right. \nonumber
\end{eqnarray}

and $z$ is chosen distinct from $\quotep{P}$, $\quotep{Q}$, the free
names in $Q$, and all the names in $R$. Our $\alpha$-equivalence will
be built in the standard way from this substitution.

\begin{remark}\label{rem:no_self_referential_names}
  One consequence of these definitions is that $\forall P. \quotep{P}
  \not\in \freenames{P}$.
\end{remark}

\subsection{ Dynamic quote: an example }

Anticipating something of what's to come, consider applying the
substitution, $\widehat{\id{\{}u / z \id{\}}}$, to the following pair
of processes, $\lift{w}{y!(z)}$ and $w[ \lpquote y!(z) \rpquote ]$.

\begin{eqnarray}
	\lift{w}{y!(z)}\widehat{\id{\{}u / z \id{\}}}
		& = &
		\lift{w}{y!(u)} \nonumber\\
	w[ \lpquote y!(z) \rpquote ] \widehat{ \id{\{}u / z \id{\}} }
		& = &
		w[ \lpquote y!(z) \rpquote ] \nonumber
\end{eqnarray}

Because the body of the process between quotes is impervious to
substitution, we get radically different answers. In fact, by
examining the first process in an input context,
e.g. $x?(z).\lift{w}{y!(z)}$, we see that the process under the lift
operator may be shaped by prefixed inputs binding a name inside it. In
this sense, the lift operator will be seen as a way to dynamically
construct processes before reifying them as names.

Finally equipped with these standard features we can present the
dynamics of the calculus.

\subsubsection{Operational semantics} 

Finally, we introduce the computational dynamics. What marks these
algebras as distinct from other more traditionally studied algebraic
structures, e.g. vector spaces or polynomial rings, is the manner in
which dynamics is captured. In traditional structures, dynamics is typically
expressed through morphisms between such structures, as in linear maps
between vector spaces or morphisms between rings. In algebras
associated with the semantics of computation, the dynamics is
expressed as part of the algebraic structure itself, through a
reduction reduction relation typically denoted by $\red$. Below, we
give a recursive presentation of this relation for the calculus used
in the encoding.

$\red \subseteq \pi \times \pi$
$\red : \pi \to \mathcal{P}(\pi)$

\begin{mathpar}
  \inferrule* [lab=Comm] { \textsf{match}( x_{src}, x_{trgt} ) } { x_{trgt}?(y)P \; | \; x_{src}!\langle {Q} \rangle \red P\{\quotep{Q}/y}\} }
  \and \\
  \inferrule* [lab=Par] {{P} \red {P}'} {{{P} | {Q}} \red {{P}' | {Q}}}
  \and
  \inferrule* [lab=Equiv]{{{P} \scong {P}'} \andalso {{P}' \red {Q}'} \andalso {{Q}' \scong {Q}}}{{P} \red {Q}}
\end{mathpar}

\begin{eqnarray*}
  match_{\equiv} (\quotep{P},\quotep{Q}) & := & P \equiv Q \\
  match_{\dagger}(\quotep{P},\quotep{Q}) & := & \forall R. P|Q \red^{*} R => R \red^{*} 0 \\
  match_{K}(\quotep{P},\quotep{Q}) & := & K \mbox{ for some context } K
\end{eqnarray*}

$u?(x)P | u!\langle Q \rangle \red P\{\quotep{Q}/x\}$

%We write $\wred$ for $\red^*$, and $P\red$ if $\exists Q $ such that $ P \red Q$.
We write $P\red$ if $\exists Q $ such that $ P \red Q$ and $P\not\red$, otherwise.

\section{Replication}

As mentioned before, it is known that replication (and hence
recursion) can be implemented in a higher-order process algebra
\cite{SangiorgiWalker}. As our first example of calculation with the
machinery thus far presented we give the construction explicitly in
the {\rhoc}.

\begin{eqnarray}
	D_{x} & := & \prefix{x}{y}{(\binpar{\outputp{x}{y}}{@{y}})} \nonumber\\
	\bangp_{x}{P} & := & \binpar{{x}!\langle{\binpar{D_{x}}{P}}\rangle}{D_{x}} \nonumber
\end{eqnarray}

\begin{eqnarray}
	\bangp_{x}{P} & & \nonumber\\
	=
	& {x}!\langle{(\prefix{x}{y}{(\outputp{x}{y} | @{y})) | P}}\rangle 
	      | \prefix{x}{y}{(\outputp{x}{y} | @{y})} & \nonumber\\
	\red
	& (\outputp{x}{y} | @{y})\substn{\quotep{(\prefix{x}{y}{(@{y} | \outputp{x}{y})) | P}}}{y} & \nonumber\\
	=
	& \outputp{x}{\quotep{(\prefix{x}{y}{(\outputp{x}{y} | @{y})) | P}}}
	  | {(\prefix{x}{y}{(\outputp{x}{y} | @{y})) | P}} & \nonumber\\
	\red
	& \ldots & \nonumber\\
	\red^*
	& P | P | \ldots & \nonumber
\end{eqnarray}

Of course, this encoding, as an implementation, runs away, unfolding
$\bangp{P}$ eagerly. A lazier and more implementable replication
operator, restricted to input-guarded processes, may be obtained as follows.

\begin{eqnarray}
\bangp{\prefix{u}{v}{P}} 
	:= 
	\binpar{\lift{x}{\prefix{u}{v}{(\binpar{D(x)}{P})}}}{D(x)} \nonumber
\end{eqnarray}

\begin{remark}
  Note that the lazier definition still does not deal with summation
  or mixed summation (i.e. sums over input and output). The reader is
  invited to construct definitions of replication that deal with these
  features. 

  Further, the definitions are parameterized in a name, $x$. Can you,
  gentle reader, make a definition that eliminates this parameter and
  guarantees no accidental interaction between the replication
  machinery and the process being replicated -- i.e. no accidental
  sharing of names used by the process to get its work done and the
  name(s) used by the replication to effect copying. This latter
  revision of the definition of replication is crucial to obtaining
  the expected identity $!!P \sim !P$.
\end{remark}

\begin{remark}\label{rem:paradoxical_combinator}
  The reader familiar with the lambda calculus will have noticed the
  similarity between $D$ and the paradoxical combinator.

  [Ed. note: the existence of this seems to suggest we have to be more
  restrictive on the set of processes and names we admit if we are to
  support no-cloning.]
\end{remark}

\subsubsection{Bisimulation}

The computational dynamics gives rise to another kind of equivalence,
the equivalence of computational behavior. As previously mentioned
this is typically captured \emph{via} some form of bisimulation.

% The notion we use in this paper is weak barbed bisimulation
% \cite{milner91polyadicpi}.

The notion we use in this paper is derived from weak barbed
bisimulation \cite{milner91polyadicpi}. 

\begin{definition}
An \emph{observation relation}, $\downarrow_{\mathcal N}$, over a set
of names, $\mathcal N$, is the smallest relation satisfying the rules
below.

\infrule[Out-barb]{y \in {\mathcal N}, \; x \nameeq y}
		  {\outputp{x}{v} \downarrow_{\mathcal N} x}
\infrule[Par-barb]{\mbox{$P\downarrow_{\mathcal N} x$ or $Q\downarrow_{\mathcal N} x$}}
		  {\binpar{P}{Q} \downarrow_{\mathcal N} x}

We write $P \Downarrow_{\mathcal N} x$ if there is $Q$ such that 
$P \wred Q$ and $Q \downarrow_{\mathcal N} x$.
\end{definition}

\begin{definition}
%\label{def.bbisim}
An  ${\mathcal N}$-\emph{barbed bisimulation} over a set of names, ${\mathcal N}$, is a symmetric binary relation 
${\mathcal S}_{\mathcal N}$ between agents such that $P\rel{S}_{\mathcal N}Q$ implies:
\begin{enumerate}
\item If $P \red P'$ then $Q \wred Q'$ and $P'\rel{S}_{\mathcal N} Q'$.
\item If $P\downarrow_{\mathcal N} x$, then $Q\Downarrow_{\mathcal N} x$.
\end{enumerate}
$P$ is ${\mathcal N}$-barbed bisimilar to $Q$, written
$P \wbbisim_{\mathcal N} Q$, if $P \rel{S}_{\mathcal N} Q$ for some ${\mathcal N}$-barbed bisimulation ${\mathcal S}_{\mathcal N}$.
\end{definition}

$\mathcal{R} \subseteq \pi \times \pi$

$P \mathcal{R} Q => \forall P'. P \red P' \Rightarrow \exists Q'. Q \red Q', P' \mathcal{R} Q'$

$P \vdash x \Rightarrow Q \vdash x$

\begin{mathpar}
  \inferrule*[lab=Out-barb]{x \nameeq y}{{y}!\langle{Q}\rangle \vdash x}
  \and
  \inferrule*[lab=Par-barb]{\mbox{$P\vdash x$ or $Q\vdash x$}}{\binpar{P}{Q} \vdash x}
\end{mathpar}

\subsubsection{Contexts}

One of the principle advantages of computational calculi like the
$\pi$-calculus is a well-defined notion of context,
contextual-equivalence and a correlation between
contextual-equivalence and notions of bisimulation. The notion of
context allows the decomposition of a process into (sub-)process and
its syntactic environment, its context. Thus, a context may be
thought of as a process with a ``hole'' (written $\Box$) in it. The
application of a context $M$ to a process $P$, written $M[P]$, is
tantamount to filling the hole in $M$ with $P$. In this paper we do
not need the full weight of this theory, but do make use of the notion
of context in the proof the main theorem. 

\begin{mathpar}
  \inferrule* [lab=summation] {} {{M_{M},M_{N}} \bc \Box \;|\; x.M_{A} \;|\; M_{M}+M_{N}}
  \and
  \inferrule* [lab=agent] {} {{M_{A}} \bc (\vec{x})M_{P} \;| \; \clift{P_0,\ldots,M_{P},\ldots,P_N}}
  \and \\
  \inferrule* [lab=process] {} {{M_{P}} \bc M_{N} \;| \;P|M_{P} }
\end{mathpar} 

\begin{mathpar}
  \inferrule* [lab=sychronization] {} {M_{N} \bc \Box \;|\; x?M_{F} \;|\; x!M_{C}}
  \and
  \inferrule* [lab=abstraction] {} {{M_{F}} \bc (x)M_{P} }
  \and
  \inferrule* [lab=concretion] {} {{M_{C}} \bc \langle M_{P} \rangle }
  \and \\
  \inferrule* [lab=process] {} {{M_{P}} \bc M_{N} \;| \;P|M_{P} }
\end{mathpar}

\begin{definition}[contextual application] Given a context $M$, and
  process $P$, we define the \emph{contextual application}, $M[P] :=
  M\{P/\Box\}$. That is, the contextual application of M to P is the
  substitution of $P$ for $\Box$ in $M$.
\end{definition}

$\meaningof{-} : L \to \mathcal{P}(\pi)$

\begin{mathpar}
  \inferrule* [lab=collection] {} {\meaningof{true} = \pi, \and \meaningof{~E} = \pi \setminus \meaningof{E}, \and \meaningof{E_{1} \& E_{2}} = \meaningof{E_{1}} \cap \meaningof{E_{2}}}
\end{mathpar}

\begin{mathpar}
  \inferrule* [lab=structure] {} {\meaningof{0} = \{ P \in \pi | P \equiv 0 \}, \and \\ \meaningof{E_1 | E_2} = \{ P \in \pi | P \equiv P_{1} | P_{2}, P_{1} \in \meaningof{E_{1}}, P_{2} \in \meaningof{E_2}\} }
\end{mathpar}

\begin{mathpar}
 \inferrule* [lab=behavior] {} {\meaningof{\langle a?b \rangle E} = \{ P \in \pi | P \equiv Q | u?(y)P', \\ \and \\\\ \and \\ \;\;\; u \in \meaningof{a}, \forall z.P'\{z/y\} \in \meaningof{E\{z/b\}}\}, \and \\ \meaningof{a!E} = \{ P \in \pi | P \equiv Q | x!\langle P' \rangle, x \in \meaningof{a} P' \in \meaningof{E}\} }
\end{mathpar}

\begin{mathpar}
 \inferrule* [lab=nominal] {} {\meaningof{\quotep{E}} = \{ \quotep{P} \in \quotep{\pi} | P \in \meaningof{E} \}, \and \meaningof{\quotep{P}} = \{ \quotep{Q} \in \quotep{\pi} | P \equiv Q \} \and \\ \meaningof{@\quotep{E}} = \{ P \in \pi | P \equiv @x, x \in \meaningof{E} \}}
\end{mathpar}

\begin{eqnarray*}
  \\
  \meaningof{-} : TS \to ST
\end{eqnarray*}

\begin{eqnarray*}
  \\
  L : TS \to ST
\end{eqnarray*}

\begin{eqnarray*}
  \\
  P \models E \iff P \in \meaningof{E}
\end{eqnarray*}

\begin{eqnarray*}
  P \approx_{L} Q \iff \forall E \in L. P \models E \iff Q \models E
\end{eqnarray*}

\begin{eqnarray*}
  P \approx_{K} Q
\end{eqnarray*}

\begin{eqnarray*}
  P \approx Q
\end{eqnarray*}

$\approx_{K} = \approx = \approx_{L}$

\subsubsection{Contextual duality}

Note that contexts extend the quotation operation to a family of
operations from processes to names. Given a context, $M$, we can
define a \emph{nominal context}, $\quotep{M}$ by $\quotep{M}[P] :=
\quotep{M[P]}$. To foreshadow what is to come we observe that these
operations enjoy a duality with processes very much like the duality
between vectors and maps from vectors to scalars.

Further, because the calculus is essentially higher-order, we have a
correspondence between contexts and processes. More specifically,
given a name $x$ and a context $M$ we can construct $M^{*}_{x}$ such
that 

\begin{mathpar}
  M^{*}_{x} | \lift{x}{P} \red M[P]
\end{mathpar}

namely,

\begin{mathpar}
  M^{*}_{x} := x?(u).M[\dropn{u}]
\end{mathpar}

The dependence of $M^{*}_{x}$ on a name makes it an abstraction, 

\begin{mathpar}
  M^{*} := (x)x?(u).M[\dropn{u}]
\end{mathpar}

\subsection{Additional notation}

It will sometimes be convenient to denote the process a name
quotes. We already have the notation $x = \quotep{P}$, but it will be
convenient to introduce an alternate notation, $\procn{x}$, when we
want to emphasize the connection to the use of the name. Note that, by
virtue of name equivalence, $\quotep{\procn{x}} \nameeq x$; so, the
notation is consistent with previous definitions.

Further, because names have structure it is possible to effect
substitutions on the basis of that structure. This means we need to
upgrade our notation for substitutions, which we accomplish by
adapting comprehension notation. Thus,

\begin{mathpar}
  P\{ y / x : x \in S \}
\end{mathpar}

is interpreted to mean the process derived from P by replacing (in a
capture-avoiding manner) each occurrence of $x$ in $S$ by $y$. For example,

\begin{mathpar}
  P\{ \quotep{\procn{x}|\procn{x}} / x : x \in \freenames{P} \}
\end{mathpar}

will replace each (occurrence) of a free name $x$ in $P$ by
$\quotep{\procn{x}|\procn{x}}$.

Also, we will avail ourselves of the notation $x^{L}$ and $x^{R}$ to
denote injections of a name into disjoint copies of the name
space. There are numerous ways to accomplish this. One example can be
found in \cite{MeredithR05}. This notation overloads to vectors of
names: $\vec{x}^{\pi} := (x_{i}^{\pi} \; : \; 0 \leq i < |\vec{x}| )$ where $\pi \in \{L,R\}$.

We also use $P^{\Box} := P|\Box$.

In \cite{MeredithR05} an interpretation of the new operator is
given. It turns out that there are several possible interpretations
all enjoying the requisite algebraic properties of the operator (see
\cite{milner91polyadicpi}). We will therefore make liberal use of
$(\nu\; \vec{x})P$.

% subsection the_syntax_and_semantics_of_the_notation_system (end)   

\input{qm2pi.qmops} 

\input{qm2pi.sterngerlach} 

\input{qm2pi.metric} 

% section concurrent_process_calculi (end)

%\input{qm2pi.proofsketch}

% section proof sketch (end)

%\input{qm2pi.slviaknots} 

% section spatial logic via knots (end)

\input{qm2pi.conclusion}

% section conclusion (end)

%\input{qm2pi.dtcodes} 

% section wiring algorithm (end)

\input{qm2pi.ack} 

% section acknowledgments (end)

\newpage


\bibliographystyle{plain}   
\bibliography{../../biblios/main.bib}

\input{qm2pi.rhodetails}

\end{document}



\end{document}



% section proof sketch (end)

%\section{Unlikely characters: spatial logic for
  knots}\label{sub:characteristic_formulae} % (fold)

Associated to the mobile process calculi are a family of logics known
as the Hennessy-Milner logics. These logics typically enjoy a
semantics interpreting formulae as sets of processes that when
factored through the encoding outlined above allows an identification
of classes of knots with logical formulae. In the context of this
encoding the sub-family known as the spatial logics \cite{CairesC03}
\cite{CairesC04} \cite{Caires04} are of particular interest providing
several important features for expressing and reasoning about
properties (i.e. classes) of knots. We hint here at how this may be done.

%\begin{description}
%\item [structural connectives] 
\subsubsection{Structural connectives} The spatial logics enjoy
structural connectives corresponding, at the logical level, to the
parallel composition ($P | Q$) and new name ($(\nu \; x)P$)
connectives for processes. As illustrated in the examples below, these
connectives are extremely expressive given the shape of our encoding.
%\item [decideable satisfaction]

\subsubsection{Decideable satisfaction}
In \cite{Caires04} the satisfaction relation is shown to be decideable
for a rich class of processes. It further turns out that the image of
the our encoding is a proper subset of that class. This result
provides the basis for an algorithm by which to search for knots
enjoying a given property.
%\item [characteristic formulae]

\subsubsection{Characteristic formulae}
In the same paper \cite{Caires04} , Caires presents a means of calculating
characteristic formulae, selecting equivalence classes of processes
up to a pre--specified depth limit on the support set of names. Composed with our
encoding, this characteristic formula can be used to select
characteristic formulae for knots.
%\end{description}

\subsubsection{Spatial logic formulae}

The grammar below (segmented for comprehension) summarizes the syntax
of spatial logic formulae. We employ illustrative examples in the
sequel to provide an intuitive understanding of their meaning
referring the reader to \cite{Caires04} for a more detailed explication
of the semantics.

\begin{mathpar}
  \inferrule* [lab=boolean] {} {{A,B} \bc T \;|\; \neg A \;|\; A \wedge B \;|\; \eta = \eta'}
  \and
  \inferrule* [lab=spatial] {} {|\; \pzero \;|\; A | B \;|\; x \text{\textregistered} A \;|\; \forall x . A \;|\;  H x . A}
  \and
  \inferrule* [lab=behavioral] {} {|\; \alpha . A}
  \and 
  \inferrule* [lab=recursion] {} {|\; X(\vec{u}) \;|\; \mu X(\vec{u}) . A}
  \and
  \inferrule* [lab=action] {} {\alpha \bc \langle x?(\vec{y}) \rangle \;|\; \langle x!(\vec{y}) \rangle \;|\; \langle \tau \rangle}
  \and 
  \inferrule* [lab=name] {} {\eta \bc x \;|\; \tau}
\end{mathpar} 

% subsection characteristic_formulae (end)   	 

\subsection{Example formulae}\label{sub:example_formulae_} % (fold)

\subsubsection{Crossing as formula.}
% 
% \begin{align*}
%   \frac{d}{dx} \sin x &= \cos x 
%   & \frac{d}{dx} e^x &= e^x \\
%   \frac{d}{dx} \cos x &= - \sin x 
%   & \frac{d}{dx} \log x &= \frac{1}{x} \\
% \end{align*} 

\begin{align*}
 \mu C(x_{0},x_{1},y_{0},y_{1},u).&(\langle x_{0}?(z) \rangle(\langle u! \rangle\langle y_{1}!z \rangle C(x_{0},x_{1},y_{0},y_{1},u)) & \\
  & \wedge \langle y_{1}?(z) \rangle (\langle u! \rangle \langle x_{0}!z \rangle C(x_{0},x_{1},y_{0},y_{1},u)) & \\
  & \wedge \langle x_{1}?(z) \rangle (\langle u? \rangle \langle y_{0}!z \rangle C(x_{0},x_{1},y_{0},y_{1},u)) & \\
  & \wedge \langle y_{0}?(z) \rangle (\langle u? \rangle \langle x_{1}!z \rangle C(x_{0},x_{1},y_{0},y_{1},u))) &
\end{align*}

The lexicographical similarity between the shape of this formulae and
the shape of definition of the process representing a crossing reveals
the intuitive meaning of this formulae. It describes the capabilities
of a process that has the right to represent a crossing. For example
it picks out processes that may perform an input on the port $x_0$ in
its initial menu of capabilities. What differentiates the formula
from the process, however, is that the crossing process is the
smallest candidate to satisfy the formula. Infinitely many other
processes -- with internal behavior hidden behind this interface, so
to speak -- also satisfy this formula. Even this simple formula,
then, can be seen to open a new view onto knots, providing a
computational interpretation of \emph{virtual} knots.

Note that this formula is derived by hand. A similar formula can be
derived by employing Caires' calculation of characteristic formula
\cite{Caires04} to the process representing a crossing. In light of
this discussion, we let
$\meaningof{C}_{\phi}(x0,x1,y0,y1,u)$ denote a formula specifying the
dynamics we wish to capture of a crossing. To guarantee we preserve
the shape of the interface and minimal semantics we demand that
$\meaningof{C}_{\phi}(x0,x1,y0,y1,u) \Rightarrow
\textbf{C}(x0,x1,y0,y1,u)$ where $\textbf{C}(x0,x1,y0,y1,u)$ denotes
the formula above.
                            
\subsubsection{Crossing number constraints.}
The moral content of the context lemma (Lemma \ref{context}) is that the notion of
``locality'' in the Reidemeister moves is effectively captured by the
parallel composition operator of the process calculus. This intuition
extends through the logic. Given a formula,
$\meaningof{C}_{\phi}(x0,x1,y0,y1,u)$, we can use the structural
connectives to specify constraints on crossing numbers, such as at
least $n$ crossings, or exactly $n$ crossings.
\begin{mathpar}
  \inferrule* [lab=at-least-n] {} { K^{\geq n}_{\phi}(\vec{xs},\vec{ys}) := \Pi_{i=0}^{n-1} Hu . \meaningof{C}_{\phi}(xs_i,ys_i,u) | T }
  \and 
  \inferrule* [lab=exactly-n] {} { K^{= n}_{\phi}(\vec{xs},\vec{ys}) := \Pi_{i=0}^{n-1} Hu . \meaningof{C}_{\phi}(xs_i,ys_i,u) | \neg (\forall x_0,y_0,x_1,y_1,u . \meaningof{C}_{\phi}(x_0,y_0,x_1,y_1,u) | T) }
\end{mathpar}

To round out this section, recall that the encoding of an $n$-crossing
knot decomposes into a parallel composition of $n$ \emph{copies} of a
crossing process together with a wiring harness. To specify different
knot classes with the same crossing number amounts to specifying
logical constraints on the wiring harness. In the interest of space,
we defer examples to a forthcoming paper. Suffice it to say that both
the conditions ``alternating knot'' and ``contains the tangle
corresponding to 5/3'' are expressible. For example, it is possible to
calculate the characteristic formula of a process corresponding to the
tangle 5/3 and conjoin it into the classifying formula via the
composition connective of the logic.

Finally, we wish to observe that it is entirely within reason to
contemplate a more domain-specific version of spatial logic tailored
to the shape of processes in the image of the encoding. Such a
domain-specific logic would have a better claim to the title formal
language of knot properties.

% subsection example_formulae_ (end)

% section knots_as_processes (end) 

% section spatial logic via knots (end)

\section{Conclusions and future work}

\paragraph{Testing physical space}
You, gentle reader, may wonder why of all the theorems to be proved
given this set up we pick the one above. In some sense it's hardly
central to quantum mechanics. We see it as central in the sense that
it firmly establishes a notion of physical space arising from a notion
of the equivalence of behavior. Relating bisimulation to a metric is a
big step forward, but one is faced with interpreting the relationship
of that metric space to something more physical. Quantum mechanical
notions of ``physical'' space are still far from intuitive, but by
relating this idea of distance as testing to calculations that predict
physical circumstances we are making a not insignificant step forward
toward an understanding of the physical space we inhabit as
essentially dynamic.

\paragraph{Effectivity and simulation}
One of the observations we have yet to make is that the entire program
spelled out here is effective. We have built various interpreters for
the reflective calculus at work in this interpretation. In principle,
then, we can simulate quantum mechanics on a computer. The place where
the simulation may lose fidelity is the infinitely branching summation
for the annihilator.

In this connection i also want to point out that the evaluation style
calculation of the inner product puts the non-determinism of the
summation right at the heart of measurement. This suggests that
Milner's original reduction-based formulation of the dynamics of his
calculi in terms of sums was not just notationally suggestive of a
notion of measure-and-continue but captured some significant part of
the physics.

\paragraph{Quantum continuations}
In light of this last observation i want to point out that the
predominant account of quantum mechanics is missing a key aspect of a
truly compositional story of the physical situation. In a real lab,
when a measurement is made the observation can be made to feed into
another device that then makes another measurement conditioned on the
results of the first. This means that after the superposition was
collapsed the entire experimental set up remained in
superposition. While QM offers a means of writing this down it doesn't
quite line up well with the well-trodden formulation of computation
and continuation that we see so succinctly expressed in Milner's
calculi. This suggests that there might be advantages to this account
of dynamics waiting to be explored.

\paragraph{Quantum logic}
In this connection, we also note that by virtue of having the
Hennessy-Milner construction, we can pull the construction through the
interpretation of QM. This gives us a natural candidate for a quantum
logic that enjoys an extremely tight connection with it's domain of
interpretation, making the construction much less ad hoc (rather it is
the image of functor!).

\paragraph{Quantum probabiity}
i have questions about the basis of the interpretation of inner
product as probability amplitude. In particular, using which
axiomatization of probability theory does the notion of probability
amplitude earn the right to be so dubbed? In other words, where is the
proof that the operation for calculating a probability amplitude (and
then squaring) satisfies the axioms of what it means to calculate a
probability? Even if such a proof exists (i have yet to find it in the
literature), i wonder if it might not be possible to turn things on
their heads. Can we view the calculation of the probability amplitude
as an axiomatization of probability? If so, then the definition we
give for calculating probability amplitude may provide the basis for
an \emph{effective} theory of probability.

\paragraph{Quantum vs ``biological'' information}
Finally, i want to conclude with a more philosophical observation. At
a recent workshop in which QM was a predominant topic i noticed
something about quantum information. The speaker was giving a riveting
discussion of axiomatic QM and showing how properties of ``no
cloning'' and ``no deleting'' emerged as consequences of the
axiomatization. Theorems of this form are necessary to give us a sense
of confidence that our axioms characterize the physical theory. What
struck me, though, was that if quantum information is neither erasable
nor replicable it is markedly different from \emph{life}. Two of the
things we know about life is that

\begin{itemize}
  \item it ends;
  \item to gain some measure of persistence, to transcend it's
    finitude it is imminently copyable.
\end{itemize}

Both of these qualities are summarized succinctly in the aphorism: all
flesh is grass. For me these two kinds of ``information'' -- call them
quantum and biological -- are end points on a spectrum of strategies
for persistence. At one end, we have those curious entities that enjoy
uniqueness and permanence; at the other, we have those who in the face
of a certain end and an uncertain present make a go of passing
something on. To me one of the more remarkable aspects of the latter
strategy is that in the presence of noise (and certain features of
copying) we get a kind of dynamism, a chance for improvement against a
given persistent condition.

% subsection other_calculi_other_bisimulations_and_geometry_as_behavior (end)




% section conclusion (end)

%\documentclass[12pt]{llncs}
%\documentclass{jktr}

\usepackage[pdftex]{hyperref}                   
\usepackage {listings}
\usepackage {mathpartir}
\usepackage{bcprules}
%\usepackage{listings}
                       
\usepackage{graphicx} 
%\usepackage[margins=2.5cm,nohead,nofoot]{geometry}
%\usepackage{geometry}
\usepackage{amsfonts}
\usepackage{amstext}
\usepackage{latexsym}
\usepackage{amssymb}
\usepackage{color}


%\include{myPreamble}
\documentclass[12pt]{llncs}
%\documentclass{jktr}

\usepackage[pdftex]{hyperref}                   
\usepackage {listings}
\usepackage {mathpartir}
\usepackage{bcprules}
%\usepackage{listings}
                       
\usepackage{graphicx} 
%\usepackage[margins=2.5cm,nohead,nofoot]{geometry}
%\usepackage{geometry}
\usepackage{amsfonts}
\usepackage{amstext}
\usepackage{latexsym}
\usepackage{amssymb}
\usepackage{color}


%\include{myPreamble}
\include{qm2pi.local} 

%\ifpdf
%\usepackage[pdftex]{graphicx}
%\else
%\usepackage{graphicx}
%\fi

 % \ifpdf
%  \usepackage{pdfsync}
%  \if


%\title{Brief Article}
%\author{David F. Snyder}
%\author{L.G. Meredith}

%\address{Dept. of Math., Texas State University--San Marcos, San Marcos, TX 78666}
       
\pagestyle{empty}


\begin{document}

\lstset{language=[Objective]Caml,frame=shadowbox}

\input{qm2pi.front}

% section front matter (end)

\input{qm2pi.intro} 
 
% section introduction (end)

% \input{qm2pi.knotations} 

% section notation (end)

\input{qm2pi.process.calculi} 

% section concurrent_process_calculi_and_spatial_logics_ (end)
    
%\input{qm2pi.knots2pi} 

%\input{qm2pi.trefoil} 

%\input{qm2pi.mainthm} 

% subsection basic_interpretation (end)

%\input{qm2pi.rho.presentation} 
\subsection{The syntax and semantics of the notation system}\label{sub:the_syntax_and_semantics_of_the_notation_system} % (fold)

We now summarize a technical presentation of the calculus that
embodies our theory of dynamics. The typical presentation of such a
calculus follows the style of giving generators and relations on
them. The grammar, below, describing term constructors, freely
generates the set of processes, $\Proc$. This set is then quotiented
by a relation known as structural congruence and it is over this set
that the notion of dynamics is expressed. This presentation is
essentially that of \cite{MeredithR05} with the addition of
polyadicity and summation. For readability we have relegated some of
the technical subtleties to an appendix.

\subsubsection{Process grammar}\label{subsub:process_grammar}

\begin{mathpar}
  \inferrule* [lab=synchronization] {} {{M} \bc \pzero \;|\; x?F \;|\; x!C }
  \and
  \inferrule* [lab=abstraction] {} {{F} \bc (x)P}
  \and
  \inferrule* [lab=concretion] {} {{C} \bc \langle Q \rangle}
  \and
  \inferrule* [lab=process] {} {{P,Q} \bc M \;| \;P|Q \;|\; @{x}}
  \and
  \inferrule* [lab=name] {} {{x} \bc \quotep{P}}
\end{mathpar} 

Note that $\vec{x}$ (resp. $\vec{P}$) denotes a vector of names
(resp. processes) of length $|\vec{x}|$ (resp. $|\vec{P}|$). We adopt
the following useful abbreviations.

\begin{mathpar}
   x?(\vec{y}).P := x.(\vec{y})P \and  x\clift{\vec{P}} := x.\clift{\vec{P}}
   \and x!(y) := \lift{x}{\dropn{y}}
   \and \Pi_{i=0}^{n-1}P_i := P_0 | \ldots | P_{n-1}
\end{mathpar}

\subsubsection{Structural congruence}

\paragraph{Free and bound names and alpha-equivalence.} At the
core of structural equivalence is alpha-equivalence which identifies
process that are the same up to a change of variable. Formally, we
recognize the distinction between free and bound names. The free names
of a process, $\freenames{P}$, may be calculated recursively as
follows:

\begin{mathpar}
\freenames{\pzero} := \emptyset
  \and \\
  \freenames{x?(y).P} := \{ x \} \cup (\freenames{P} \setminus \{ y \})
  \and 
  \freenames{x!\langle P \rangle} := \{ x \} \cup \{ P \} 
  \and \\
  \freenames{P|Q} := \freenames{P} \cup \freenames{Q}
  \and \\
  \freenames{@{x}} := \{ x \}
\end{mathpar}

$\pi$
$\quotep{\pi}$

$\freenames{-} : \pi \to \mathcal{P}(\quotep{\pi})$

\begin{eqnarray*}
  \freenames{\pzero} & := & \emptyset \\
  \freenames{x?(y).P} & := & \{ x \} \cup (\freenames{P} \setminus \{ y \}) \\
  \freenames{x!\langle P \rangle} & := & \{ x \} \cup \{ P \} \\
  \freenames{P|Q} & := & \freenames{P} \cup \freenames{Q} \\
  \freenames{\dropn{x}} & := & \{ x \}
\end{eqnarray*}

The bound names of a process, $\boundnames{P}$, are those names occurring in $P$
that are not free. For example, in $x?(y).0$, the name $x$ is free, while $y$ is bound.

\begin{mathpar}
  \inferrule* [lab=monoidal-laws] {} { P|Q \equiv Q|P \and P|0 \equiv P \and P|(Q|R) \equiv (P|Q)|R }
\end{mathpar}

\begin{mathpar}
  \inferrule* [lab=alpha-equivalence] {} { (x)P \equiv (y)P\{y/x\} \and y \not\in \freenames{P} }
\end{mathpar}

\begin{definition}
Then two processes, $P,Q$, are alpha-equivalent if $P = Q\{\vec{y}/\vec{x}\}$ for
some $\vec{x} \in \boundnames{Q},\vec{y} \in \boundnames{P}$, where $Q\{\vec{y}/\vec{x}\}$
denotes the capture-avoiding substitution of $\vec{y}$ for $\vec{x}$ in $Q$.
\end{definition}

\begin{definition}
  The {\em structural congruence} \cite{SangiorgiWalker} , $\equiv$,
  between processes is the least congruence containing
  alpha-equivalence, satisfying the abelian monoid laws
  (associativity, commutativity and $\pzero$ as identity) for parallel
  composition $|$ and for summation $+$.
\end{definition}

\subsection{Name equivalence}

We take name equivalence, written $\nameeq$, to be the smallest
equivalence relation generated by the following rules.

\begin{mathpar}
\inferrule*[lab=Quote-drop]
{ }
{ \quotep{@{x}} \nameeq x }

\inferrule*[lab=Struct-equiv]
{ P \scong Q }
{ \quotep{P} \nameeq \quotep{Q} }
\end{mathpar}

The astute reader will have noticed that the mutual recursion of names
and processes imposes a mutual recursion on alpha-equivalence and
structural equivalence via name-equivalence. Fortunately, all of this
works out pleasantly and we may calculate in the natural way, free of
concern. The reader interested in the details is referred to the
appendix \ref{appendix:rho_details}.

\subsection{Substitution}

We use $\Proc$ for the set of processes, $\QProc$ for the set of
names, and $\id{\{}\vec{y} / \vec{x} \id{\}}$ to denote partial maps,
$s : \QProc \rightarrow \QProc$. A map, $s$ lifts, uniquely, to a map
on process terms, $\widehat{s} : \Proc \rightarrow \Proc$ by the
following equations.

\begin{mathpar}
  (0) \psubstp{Q}{P} := 0 \\
  (R \juxtap S) \psubstp{Q}{P}
  :=    
  (R)\psubstp{Q}{P} \juxtap (S) \psubstp{Q}{P} \\
  (x?(y).R) \psubstp{Q}{P}    
  :=    
  (x)\substp{Q}{P} (z)\concat( (R \psubstn{z}{y}) \psubstp{Q}{P} ) \\
  (\lift{x}{R}) \psubstp{Q}{P}  
  :=
  \lift{(x)\substp{Q}{P}}{ R \psubstp{Q}{P} } \\
%   (\dropn{x})  \psubstp{Q}{P}       
%   := 
%   \left\{ 
%     \begin{array}{ccc} 
%       \dropn{\quotep{Q}} & & x \nameeq \quotep{P} \\
%       \dropn{x} & & otherwise \\
%     \end{array}
%   \right. 
  (\dropn{x})  \psubstp{Q}{P}       
  := 
  \left\{ 
    \begin{array}{ccc} 
      Q & & x \nameeq \quotep{P} \\
      \dropn{x} & & otherwise \\
    \end{array}
  \right.
\end{mathpar}
 

where

\begin{eqnarray}
  (x)\id{\{} \lpquote Q \rpquote / \lpquote P \rpquote \id{\}}            = 
  \left\{ 
    \begin{array}{ccc}
      \lpquote Q \rpquote & & x \nameeq \lpquote P \rpquote \\
      x & & otherwise \\
    \end{array}
  \right. \nonumber
\end{eqnarray}

and $z$ is chosen distinct from $\quotep{P}$, $\quotep{Q}$, the free
names in $Q$, and all the names in $R$. Our $\alpha$-equivalence will
be built in the standard way from this substitution.

\begin{remark}\label{rem:no_self_referential_names}
  One consequence of these definitions is that $\forall P. \quotep{P}
  \not\in \freenames{P}$.
\end{remark}

\subsection{ Dynamic quote: an example }

Anticipating something of what's to come, consider applying the
substitution, $\widehat{\id{\{}u / z \id{\}}}$, to the following pair
of processes, $\lift{w}{y!(z)}$ and $w[ \lpquote y!(z) \rpquote ]$.

\begin{eqnarray}
	\lift{w}{y!(z)}\widehat{\id{\{}u / z \id{\}}}
		& = &
		\lift{w}{y!(u)} \nonumber\\
	w[ \lpquote y!(z) \rpquote ] \widehat{ \id{\{}u / z \id{\}} }
		& = &
		w[ \lpquote y!(z) \rpquote ] \nonumber
\end{eqnarray}

Because the body of the process between quotes is impervious to
substitution, we get radically different answers. In fact, by
examining the first process in an input context,
e.g. $x?(z).\lift{w}{y!(z)}$, we see that the process under the lift
operator may be shaped by prefixed inputs binding a name inside it. In
this sense, the lift operator will be seen as a way to dynamically
construct processes before reifying them as names.

Finally equipped with these standard features we can present the
dynamics of the calculus.

\subsubsection{Operational semantics} 

Finally, we introduce the computational dynamics. What marks these
algebras as distinct from other more traditionally studied algebraic
structures, e.g. vector spaces or polynomial rings, is the manner in
which dynamics is captured. In traditional structures, dynamics is typically
expressed through morphisms between such structures, as in linear maps
between vector spaces or morphisms between rings. In algebras
associated with the semantics of computation, the dynamics is
expressed as part of the algebraic structure itself, through a
reduction reduction relation typically denoted by $\red$. Below, we
give a recursive presentation of this relation for the calculus used
in the encoding.

$\red \subseteq \pi \times \pi$
$\red : \pi \to \mathcal{P}(\pi)$

\begin{mathpar}
  \inferrule* [lab=Comm] { \textsf{match}( x_{src}, x_{trgt} ) } { x_{trgt}?(y)P \; | \; x_{src}!\langle {Q} \rangle \red P\{\quotep{Q}/y}\} }
  \and \\
  \inferrule* [lab=Par] {{P} \red {P}'} {{{P} | {Q}} \red {{P}' | {Q}}}
  \and
  \inferrule* [lab=Equiv]{{{P} \scong {P}'} \andalso {{P}' \red {Q}'} \andalso {{Q}' \scong {Q}}}{{P} \red {Q}}
\end{mathpar}

\begin{eqnarray*}
  match_{\equiv} (\quotep{P},\quotep{Q}) & := & P \equiv Q \\
  match_{\dagger}(\quotep{P},\quotep{Q}) & := & \forall R. P|Q \red^{*} R => R \red^{*} 0 \\
  match_{K}(\quotep{P},\quotep{Q}) & := & K \mbox{ for some context } K
\end{eqnarray*}

$u?(x)P | u!\langle Q \rangle \red P\{\quotep{Q}/x\}$

%We write $\wred$ for $\red^*$, and $P\red$ if $\exists Q $ such that $ P \red Q$.
We write $P\red$ if $\exists Q $ such that $ P \red Q$ and $P\not\red$, otherwise.

\section{Replication}

As mentioned before, it is known that replication (and hence
recursion) can be implemented in a higher-order process algebra
\cite{SangiorgiWalker}. As our first example of calculation with the
machinery thus far presented we give the construction explicitly in
the {\rhoc}.

\begin{eqnarray}
	D_{x} & := & \prefix{x}{y}{(\binpar{\outputp{x}{y}}{@{y}})} \nonumber\\
	\bangp_{x}{P} & := & \binpar{{x}!\langle{\binpar{D_{x}}{P}}\rangle}{D_{x}} \nonumber
\end{eqnarray}

\begin{eqnarray}
	\bangp_{x}{P} & & \nonumber\\
	=
	& {x}!\langle{(\prefix{x}{y}{(\outputp{x}{y} | @{y})) | P}}\rangle 
	      | \prefix{x}{y}{(\outputp{x}{y} | @{y})} & \nonumber\\
	\red
	& (\outputp{x}{y} | @{y})\substn{\quotep{(\prefix{x}{y}{(@{y} | \outputp{x}{y})) | P}}}{y} & \nonumber\\
	=
	& \outputp{x}{\quotep{(\prefix{x}{y}{(\outputp{x}{y} | @{y})) | P}}}
	  | {(\prefix{x}{y}{(\outputp{x}{y} | @{y})) | P}} & \nonumber\\
	\red
	& \ldots & \nonumber\\
	\red^*
	& P | P | \ldots & \nonumber
\end{eqnarray}

Of course, this encoding, as an implementation, runs away, unfolding
$\bangp{P}$ eagerly. A lazier and more implementable replication
operator, restricted to input-guarded processes, may be obtained as follows.

\begin{eqnarray}
\bangp{\prefix{u}{v}{P}} 
	:= 
	\binpar{\lift{x}{\prefix{u}{v}{(\binpar{D(x)}{P})}}}{D(x)} \nonumber
\end{eqnarray}

\begin{remark}
  Note that the lazier definition still does not deal with summation
  or mixed summation (i.e. sums over input and output). The reader is
  invited to construct definitions of replication that deal with these
  features. 

  Further, the definitions are parameterized in a name, $x$. Can you,
  gentle reader, make a definition that eliminates this parameter and
  guarantees no accidental interaction between the replication
  machinery and the process being replicated -- i.e. no accidental
  sharing of names used by the process to get its work done and the
  name(s) used by the replication to effect copying. This latter
  revision of the definition of replication is crucial to obtaining
  the expected identity $!!P \sim !P$.
\end{remark}

\begin{remark}\label{rem:paradoxical_combinator}
  The reader familiar with the lambda calculus will have noticed the
  similarity between $D$ and the paradoxical combinator.

  [Ed. note: the existence of this seems to suggest we have to be more
  restrictive on the set of processes and names we admit if we are to
  support no-cloning.]
\end{remark}

\subsubsection{Bisimulation}

The computational dynamics gives rise to another kind of equivalence,
the equivalence of computational behavior. As previously mentioned
this is typically captured \emph{via} some form of bisimulation.

% The notion we use in this paper is weak barbed bisimulation
% \cite{milner91polyadicpi}.

The notion we use in this paper is derived from weak barbed
bisimulation \cite{milner91polyadicpi}. 

\begin{definition}
An \emph{observation relation}, $\downarrow_{\mathcal N}$, over a set
of names, $\mathcal N$, is the smallest relation satisfying the rules
below.

\infrule[Out-barb]{y \in {\mathcal N}, \; x \nameeq y}
		  {\outputp{x}{v} \downarrow_{\mathcal N} x}
\infrule[Par-barb]{\mbox{$P\downarrow_{\mathcal N} x$ or $Q\downarrow_{\mathcal N} x$}}
		  {\binpar{P}{Q} \downarrow_{\mathcal N} x}

We write $P \Downarrow_{\mathcal N} x$ if there is $Q$ such that 
$P \wred Q$ and $Q \downarrow_{\mathcal N} x$.
\end{definition}

\begin{definition}
%\label{def.bbisim}
An  ${\mathcal N}$-\emph{barbed bisimulation} over a set of names, ${\mathcal N}$, is a symmetric binary relation 
${\mathcal S}_{\mathcal N}$ between agents such that $P\rel{S}_{\mathcal N}Q$ implies:
\begin{enumerate}
\item If $P \red P'$ then $Q \wred Q'$ and $P'\rel{S}_{\mathcal N} Q'$.
\item If $P\downarrow_{\mathcal N} x$, then $Q\Downarrow_{\mathcal N} x$.
\end{enumerate}
$P$ is ${\mathcal N}$-barbed bisimilar to $Q$, written
$P \wbbisim_{\mathcal N} Q$, if $P \rel{S}_{\mathcal N} Q$ for some ${\mathcal N}$-barbed bisimulation ${\mathcal S}_{\mathcal N}$.
\end{definition}

$\mathcal{R} \subseteq \pi \times \pi$

$P \mathcal{R} Q => \forall P'. P \red P' \Rightarrow \exists Q'. Q \red Q', P' \mathcal{R} Q'$

$P \vdash x \Rightarrow Q \vdash x$

\begin{mathpar}
  \inferrule*[lab=Out-barb]{x \nameeq y}{{y}!\langle{Q}\rangle \vdash x}
  \and
  \inferrule*[lab=Par-barb]{\mbox{$P\vdash x$ or $Q\vdash x$}}{\binpar{P}{Q} \vdash x}
\end{mathpar}

\subsubsection{Contexts}

One of the principle advantages of computational calculi like the
$\pi$-calculus is a well-defined notion of context,
contextual-equivalence and a correlation between
contextual-equivalence and notions of bisimulation. The notion of
context allows the decomposition of a process into (sub-)process and
its syntactic environment, its context. Thus, a context may be
thought of as a process with a ``hole'' (written $\Box$) in it. The
application of a context $M$ to a process $P$, written $M[P]$, is
tantamount to filling the hole in $M$ with $P$. In this paper we do
not need the full weight of this theory, but do make use of the notion
of context in the proof the main theorem. 

\begin{mathpar}
  \inferrule* [lab=summation] {} {{M_{M},M_{N}} \bc \Box \;|\; x.M_{A} \;|\; M_{M}+M_{N}}
  \and
  \inferrule* [lab=agent] {} {{M_{A}} \bc (\vec{x})M_{P} \;| \; \clift{P_0,\ldots,M_{P},\ldots,P_N}}
  \and \\
  \inferrule* [lab=process] {} {{M_{P}} \bc M_{N} \;| \;P|M_{P} }
\end{mathpar} 

\begin{mathpar}
  \inferrule* [lab=sychronization] {} {M_{N} \bc \Box \;|\; x?M_{F} \;|\; x!M_{C}}
  \and
  \inferrule* [lab=abstraction] {} {{M_{F}} \bc (x)M_{P} }
  \and
  \inferrule* [lab=concretion] {} {{M_{C}} \bc \langle M_{P} \rangle }
  \and \\
  \inferrule* [lab=process] {} {{M_{P}} \bc M_{N} \;| \;P|M_{P} }
\end{mathpar}

\begin{definition}[contextual application] Given a context $M$, and
  process $P$, we define the \emph{contextual application}, $M[P] :=
  M\{P/\Box\}$. That is, the contextual application of M to P is the
  substitution of $P$ for $\Box$ in $M$.
\end{definition}

$\meaningof{-} : L \to \mathcal{P}(\pi)$

\begin{mathpar}
  \inferrule* [lab=collection] {} {\meaningof{true} = \pi, \and \meaningof{~E} = \pi \setminus \meaningof{E}, \and \meaningof{E_{1} \& E_{2}} = \meaningof{E_{1}} \cap \meaningof{E_{2}}}
\end{mathpar}

\begin{mathpar}
  \inferrule* [lab=structure] {} {\meaningof{0} = \{ P \in \pi | P \equiv 0 \}, \and \\ \meaningof{E_1 | E_2} = \{ P \in \pi | P \equiv P_{1} | P_{2}, P_{1} \in \meaningof{E_{1}}, P_{2} \in \meaningof{E_2}\} }
\end{mathpar}

\begin{mathpar}
 \inferrule* [lab=behavior] {} {\meaningof{\langle a?b \rangle E} = \{ P \in \pi | P \equiv Q | u?(y)P', \\ \and \\\\ \and \\ \;\;\; u \in \meaningof{a}, \forall z.P'\{z/y\} \in \meaningof{E\{z/b\}}\}, \and \\ \meaningof{a!E} = \{ P \in \pi | P \equiv Q | x!\langle P' \rangle, x \in \meaningof{a} P' \in \meaningof{E}\} }
\end{mathpar}

\begin{mathpar}
 \inferrule* [lab=nominal] {} {\meaningof{\quotep{E}} = \{ \quotep{P} \in \quotep{\pi} | P \in \meaningof{E} \}, \and \meaningof{\quotep{P}} = \{ \quotep{Q} \in \quotep{\pi} | P \equiv Q \} \and \\ \meaningof{@\quotep{E}} = \{ P \in \pi | P \equiv @x, x \in \meaningof{E} \}}
\end{mathpar}

\begin{eqnarray*}
  \\
  \meaningof{-} : TS \to ST
\end{eqnarray*}

\begin{eqnarray*}
  \\
  L : TS \to ST
\end{eqnarray*}

\begin{eqnarray*}
  \\
  P \models E \iff P \in \meaningof{E}
\end{eqnarray*}

\begin{eqnarray*}
  P \approx_{L} Q \iff \forall E \in L. P \models E \iff Q \models E
\end{eqnarray*}

\begin{eqnarray*}
  P \approx_{K} Q
\end{eqnarray*}

\begin{eqnarray*}
  P \approx Q
\end{eqnarray*}

$\approx_{K} = \approx = \approx_{L}$

\subsubsection{Contextual duality}

Note that contexts extend the quotation operation to a family of
operations from processes to names. Given a context, $M$, we can
define a \emph{nominal context}, $\quotep{M}$ by $\quotep{M}[P] :=
\quotep{M[P]}$. To foreshadow what is to come we observe that these
operations enjoy a duality with processes very much like the duality
between vectors and maps from vectors to scalars.

Further, because the calculus is essentially higher-order, we have a
correspondence between contexts and processes. More specifically,
given a name $x$ and a context $M$ we can construct $M^{*}_{x}$ such
that 

\begin{mathpar}
  M^{*}_{x} | \lift{x}{P} \red M[P]
\end{mathpar}

namely,

\begin{mathpar}
  M^{*}_{x} := x?(u).M[\dropn{u}]
\end{mathpar}

The dependence of $M^{*}_{x}$ on a name makes it an abstraction, 

\begin{mathpar}
  M^{*} := (x)x?(u).M[\dropn{u}]
\end{mathpar}

\subsection{Additional notation}

It will sometimes be convenient to denote the process a name
quotes. We already have the notation $x = \quotep{P}$, but it will be
convenient to introduce an alternate notation, $\procn{x}$, when we
want to emphasize the connection to the use of the name. Note that, by
virtue of name equivalence, $\quotep{\procn{x}} \nameeq x$; so, the
notation is consistent with previous definitions.

Further, because names have structure it is possible to effect
substitutions on the basis of that structure. This means we need to
upgrade our notation for substitutions, which we accomplish by
adapting comprehension notation. Thus,

\begin{mathpar}
  P\{ y / x : x \in S \}
\end{mathpar}

is interpreted to mean the process derived from P by replacing (in a
capture-avoiding manner) each occurrence of $x$ in $S$ by $y$. For example,

\begin{mathpar}
  P\{ \quotep{\procn{x}|\procn{x}} / x : x \in \freenames{P} \}
\end{mathpar}

will replace each (occurrence) of a free name $x$ in $P$ by
$\quotep{\procn{x}|\procn{x}}$.

Also, we will avail ourselves of the notation $x^{L}$ and $x^{R}$ to
denote injections of a name into disjoint copies of the name
space. There are numerous ways to accomplish this. One example can be
found in \cite{MeredithR05}. This notation overloads to vectors of
names: $\vec{x}^{\pi} := (x_{i}^{\pi} \; : \; 0 \leq i < |\vec{x}| )$ where $\pi \in \{L,R\}$.

We also use $P^{\Box} := P|\Box$.

In \cite{MeredithR05} an interpretation of the new operator is
given. It turns out that there are several possible interpretations
all enjoying the requisite algebraic properties of the operator (see
\cite{milner91polyadicpi}). We will therefore make liberal use of
$(\nu\; \vec{x})P$.

% subsection the_syntax_and_semantics_of_the_notation_system (end)   

\input{qm2pi.qmops} 

\input{qm2pi.sterngerlach} 

\input{qm2pi.metric} 

% section concurrent_process_calculi (end)

%\input{qm2pi.proofsketch}

% section proof sketch (end)

%\input{qm2pi.slviaknots} 

% section spatial logic via knots (end)

\input{qm2pi.conclusion}

% section conclusion (end)

%\input{qm2pi.dtcodes} 

% section wiring algorithm (end)

\input{qm2pi.ack} 

% section acknowledgments (end)

\newpage


\bibliographystyle{plain}   
\bibliography{../../biblios/main.bib}

\input{qm2pi.rhodetails}

\end{document}

 

%\ifpdf
%\usepackage[pdftex]{graphicx}
%\else
%\usepackage{graphicx}
%\fi

 % \ifpdf
%  \usepackage{pdfsync}
%  \if


%\title{Brief Article}
%\author{David F. Snyder}
%\author{L.G. Meredith}

%\address{Dept. of Math., Texas State University--San Marcos, San Marcos, TX 78666}
       
\pagestyle{empty}


\begin{document}

\lstset{language=[Objective]Caml,frame=shadowbox}

\documentclass[12pt]{llncs}
%\documentclass{jktr}

\usepackage[pdftex]{hyperref}                   
\usepackage {listings}
\usepackage {mathpartir}
\usepackage{bcprules}
%\usepackage{listings}
                       
\usepackage{graphicx} 
%\usepackage[margins=2.5cm,nohead,nofoot]{geometry}
%\usepackage{geometry}
\usepackage{amsfonts}
\usepackage{amstext}
\usepackage{latexsym}
\usepackage{amssymb}
\usepackage{color}


%\include{myPreamble}
\include{qm2pi.local} 

%\ifpdf
%\usepackage[pdftex]{graphicx}
%\else
%\usepackage{graphicx}
%\fi

 % \ifpdf
%  \usepackage{pdfsync}
%  \if


%\title{Brief Article}
%\author{David F. Snyder}
%\author{L.G. Meredith}

%\address{Dept. of Math., Texas State University--San Marcos, San Marcos, TX 78666}
       
\pagestyle{empty}


\begin{document}

\lstset{language=[Objective]Caml,frame=shadowbox}

\input{qm2pi.front}

% section front matter (end)

\input{qm2pi.intro} 
 
% section introduction (end)

% \input{qm2pi.knotations} 

% section notation (end)

\input{qm2pi.process.calculi} 

% section concurrent_process_calculi_and_spatial_logics_ (end)
    
%\input{qm2pi.knots2pi} 

%\input{qm2pi.trefoil} 

%\input{qm2pi.mainthm} 

% subsection basic_interpretation (end)

%\input{qm2pi.rho.presentation} 
\subsection{The syntax and semantics of the notation system}\label{sub:the_syntax_and_semantics_of_the_notation_system} % (fold)

We now summarize a technical presentation of the calculus that
embodies our theory of dynamics. The typical presentation of such a
calculus follows the style of giving generators and relations on
them. The grammar, below, describing term constructors, freely
generates the set of processes, $\Proc$. This set is then quotiented
by a relation known as structural congruence and it is over this set
that the notion of dynamics is expressed. This presentation is
essentially that of \cite{MeredithR05} with the addition of
polyadicity and summation. For readability we have relegated some of
the technical subtleties to an appendix.

\subsubsection{Process grammar}\label{subsub:process_grammar}

\begin{mathpar}
  \inferrule* [lab=synchronization] {} {{M} \bc \pzero \;|\; x?F \;|\; x!C }
  \and
  \inferrule* [lab=abstraction] {} {{F} \bc (x)P}
  \and
  \inferrule* [lab=concretion] {} {{C} \bc \langle Q \rangle}
  \and
  \inferrule* [lab=process] {} {{P,Q} \bc M \;| \;P|Q \;|\; @{x}}
  \and
  \inferrule* [lab=name] {} {{x} \bc \quotep{P}}
\end{mathpar} 

Note that $\vec{x}$ (resp. $\vec{P}$) denotes a vector of names
(resp. processes) of length $|\vec{x}|$ (resp. $|\vec{P}|$). We adopt
the following useful abbreviations.

\begin{mathpar}
   x?(\vec{y}).P := x.(\vec{y})P \and  x\clift{\vec{P}} := x.\clift{\vec{P}}
   \and x!(y) := \lift{x}{\dropn{y}}
   \and \Pi_{i=0}^{n-1}P_i := P_0 | \ldots | P_{n-1}
\end{mathpar}

\subsubsection{Structural congruence}

\paragraph{Free and bound names and alpha-equivalence.} At the
core of structural equivalence is alpha-equivalence which identifies
process that are the same up to a change of variable. Formally, we
recognize the distinction between free and bound names. The free names
of a process, $\freenames{P}$, may be calculated recursively as
follows:

\begin{mathpar}
\freenames{\pzero} := \emptyset
  \and \\
  \freenames{x?(y).P} := \{ x \} \cup (\freenames{P} \setminus \{ y \})
  \and 
  \freenames{x!\langle P \rangle} := \{ x \} \cup \{ P \} 
  \and \\
  \freenames{P|Q} := \freenames{P} \cup \freenames{Q}
  \and \\
  \freenames{@{x}} := \{ x \}
\end{mathpar}

$\pi$
$\quotep{\pi}$

$\freenames{-} : \pi \to \mathcal{P}(\quotep{\pi})$

\begin{eqnarray*}
  \freenames{\pzero} & := & \emptyset \\
  \freenames{x?(y).P} & := & \{ x \} \cup (\freenames{P} \setminus \{ y \}) \\
  \freenames{x!\langle P \rangle} & := & \{ x \} \cup \{ P \} \\
  \freenames{P|Q} & := & \freenames{P} \cup \freenames{Q} \\
  \freenames{\dropn{x}} & := & \{ x \}
\end{eqnarray*}

The bound names of a process, $\boundnames{P}$, are those names occurring in $P$
that are not free. For example, in $x?(y).0$, the name $x$ is free, while $y$ is bound.

\begin{mathpar}
  \inferrule* [lab=monoidal-laws] {} { P|Q \equiv Q|P \and P|0 \equiv P \and P|(Q|R) \equiv (P|Q)|R }
\end{mathpar}

\begin{mathpar}
  \inferrule* [lab=alpha-equivalence] {} { (x)P \equiv (y)P\{y/x\} \and y \not\in \freenames{P} }
\end{mathpar}

\begin{definition}
Then two processes, $P,Q$, are alpha-equivalent if $P = Q\{\vec{y}/\vec{x}\}$ for
some $\vec{x} \in \boundnames{Q},\vec{y} \in \boundnames{P}$, where $Q\{\vec{y}/\vec{x}\}$
denotes the capture-avoiding substitution of $\vec{y}$ for $\vec{x}$ in $Q$.
\end{definition}

\begin{definition}
  The {\em structural congruence} \cite{SangiorgiWalker} , $\equiv$,
  between processes is the least congruence containing
  alpha-equivalence, satisfying the abelian monoid laws
  (associativity, commutativity and $\pzero$ as identity) for parallel
  composition $|$ and for summation $+$.
\end{definition}

\subsection{Name equivalence}

We take name equivalence, written $\nameeq$, to be the smallest
equivalence relation generated by the following rules.

\begin{mathpar}
\inferrule*[lab=Quote-drop]
{ }
{ \quotep{@{x}} \nameeq x }

\inferrule*[lab=Struct-equiv]
{ P \scong Q }
{ \quotep{P} \nameeq \quotep{Q} }
\end{mathpar}

The astute reader will have noticed that the mutual recursion of names
and processes imposes a mutual recursion on alpha-equivalence and
structural equivalence via name-equivalence. Fortunately, all of this
works out pleasantly and we may calculate in the natural way, free of
concern. The reader interested in the details is referred to the
appendix \ref{appendix:rho_details}.

\subsection{Substitution}

We use $\Proc$ for the set of processes, $\QProc$ for the set of
names, and $\id{\{}\vec{y} / \vec{x} \id{\}}$ to denote partial maps,
$s : \QProc \rightarrow \QProc$. A map, $s$ lifts, uniquely, to a map
on process terms, $\widehat{s} : \Proc \rightarrow \Proc$ by the
following equations.

\begin{mathpar}
  (0) \psubstp{Q}{P} := 0 \\
  (R \juxtap S) \psubstp{Q}{P}
  :=    
  (R)\psubstp{Q}{P} \juxtap (S) \psubstp{Q}{P} \\
  (x?(y).R) \psubstp{Q}{P}    
  :=    
  (x)\substp{Q}{P} (z)\concat( (R \psubstn{z}{y}) \psubstp{Q}{P} ) \\
  (\lift{x}{R}) \psubstp{Q}{P}  
  :=
  \lift{(x)\substp{Q}{P}}{ R \psubstp{Q}{P} } \\
%   (\dropn{x})  \psubstp{Q}{P}       
%   := 
%   \left\{ 
%     \begin{array}{ccc} 
%       \dropn{\quotep{Q}} & & x \nameeq \quotep{P} \\
%       \dropn{x} & & otherwise \\
%     \end{array}
%   \right. 
  (\dropn{x})  \psubstp{Q}{P}       
  := 
  \left\{ 
    \begin{array}{ccc} 
      Q & & x \nameeq \quotep{P} \\
      \dropn{x} & & otherwise \\
    \end{array}
  \right.
\end{mathpar}
 

where

\begin{eqnarray}
  (x)\id{\{} \lpquote Q \rpquote / \lpquote P \rpquote \id{\}}            = 
  \left\{ 
    \begin{array}{ccc}
      \lpquote Q \rpquote & & x \nameeq \lpquote P \rpquote \\
      x & & otherwise \\
    \end{array}
  \right. \nonumber
\end{eqnarray}

and $z$ is chosen distinct from $\quotep{P}$, $\quotep{Q}$, the free
names in $Q$, and all the names in $R$. Our $\alpha$-equivalence will
be built in the standard way from this substitution.

\begin{remark}\label{rem:no_self_referential_names}
  One consequence of these definitions is that $\forall P. \quotep{P}
  \not\in \freenames{P}$.
\end{remark}

\subsection{ Dynamic quote: an example }

Anticipating something of what's to come, consider applying the
substitution, $\widehat{\id{\{}u / z \id{\}}}$, to the following pair
of processes, $\lift{w}{y!(z)}$ and $w[ \lpquote y!(z) \rpquote ]$.

\begin{eqnarray}
	\lift{w}{y!(z)}\widehat{\id{\{}u / z \id{\}}}
		& = &
		\lift{w}{y!(u)} \nonumber\\
	w[ \lpquote y!(z) \rpquote ] \widehat{ \id{\{}u / z \id{\}} }
		& = &
		w[ \lpquote y!(z) \rpquote ] \nonumber
\end{eqnarray}

Because the body of the process between quotes is impervious to
substitution, we get radically different answers. In fact, by
examining the first process in an input context,
e.g. $x?(z).\lift{w}{y!(z)}$, we see that the process under the lift
operator may be shaped by prefixed inputs binding a name inside it. In
this sense, the lift operator will be seen as a way to dynamically
construct processes before reifying them as names.

Finally equipped with these standard features we can present the
dynamics of the calculus.

\subsubsection{Operational semantics} 

Finally, we introduce the computational dynamics. What marks these
algebras as distinct from other more traditionally studied algebraic
structures, e.g. vector spaces or polynomial rings, is the manner in
which dynamics is captured. In traditional structures, dynamics is typically
expressed through morphisms between such structures, as in linear maps
between vector spaces or morphisms between rings. In algebras
associated with the semantics of computation, the dynamics is
expressed as part of the algebraic structure itself, through a
reduction reduction relation typically denoted by $\red$. Below, we
give a recursive presentation of this relation for the calculus used
in the encoding.

$\red \subseteq \pi \times \pi$
$\red : \pi \to \mathcal{P}(\pi)$

\begin{mathpar}
  \inferrule* [lab=Comm] { \textsf{match}( x_{src}, x_{trgt} ) } { x_{trgt}?(y)P \; | \; x_{src}!\langle {Q} \rangle \red P\{\quotep{Q}/y}\} }
  \and \\
  \inferrule* [lab=Par] {{P} \red {P}'} {{{P} | {Q}} \red {{P}' | {Q}}}
  \and
  \inferrule* [lab=Equiv]{{{P} \scong {P}'} \andalso {{P}' \red {Q}'} \andalso {{Q}' \scong {Q}}}{{P} \red {Q}}
\end{mathpar}

\begin{eqnarray*}
  match_{\equiv} (\quotep{P},\quotep{Q}) & := & P \equiv Q \\
  match_{\dagger}(\quotep{P},\quotep{Q}) & := & \forall R. P|Q \red^{*} R => R \red^{*} 0 \\
  match_{K}(\quotep{P},\quotep{Q}) & := & K \mbox{ for some context } K
\end{eqnarray*}

$u?(x)P | u!\langle Q \rangle \red P\{\quotep{Q}/x\}$

%We write $\wred$ for $\red^*$, and $P\red$ if $\exists Q $ such that $ P \red Q$.
We write $P\red$ if $\exists Q $ such that $ P \red Q$ and $P\not\red$, otherwise.

\section{Replication}

As mentioned before, it is known that replication (and hence
recursion) can be implemented in a higher-order process algebra
\cite{SangiorgiWalker}. As our first example of calculation with the
machinery thus far presented we give the construction explicitly in
the {\rhoc}.

\begin{eqnarray}
	D_{x} & := & \prefix{x}{y}{(\binpar{\outputp{x}{y}}{@{y}})} \nonumber\\
	\bangp_{x}{P} & := & \binpar{{x}!\langle{\binpar{D_{x}}{P}}\rangle}{D_{x}} \nonumber
\end{eqnarray}

\begin{eqnarray}
	\bangp_{x}{P} & & \nonumber\\
	=
	& {x}!\langle{(\prefix{x}{y}{(\outputp{x}{y} | @{y})) | P}}\rangle 
	      | \prefix{x}{y}{(\outputp{x}{y} | @{y})} & \nonumber\\
	\red
	& (\outputp{x}{y} | @{y})\substn{\quotep{(\prefix{x}{y}{(@{y} | \outputp{x}{y})) | P}}}{y} & \nonumber\\
	=
	& \outputp{x}{\quotep{(\prefix{x}{y}{(\outputp{x}{y} | @{y})) | P}}}
	  | {(\prefix{x}{y}{(\outputp{x}{y} | @{y})) | P}} & \nonumber\\
	\red
	& \ldots & \nonumber\\
	\red^*
	& P | P | \ldots & \nonumber
\end{eqnarray}

Of course, this encoding, as an implementation, runs away, unfolding
$\bangp{P}$ eagerly. A lazier and more implementable replication
operator, restricted to input-guarded processes, may be obtained as follows.

\begin{eqnarray}
\bangp{\prefix{u}{v}{P}} 
	:= 
	\binpar{\lift{x}{\prefix{u}{v}{(\binpar{D(x)}{P})}}}{D(x)} \nonumber
\end{eqnarray}

\begin{remark}
  Note that the lazier definition still does not deal with summation
  or mixed summation (i.e. sums over input and output). The reader is
  invited to construct definitions of replication that deal with these
  features. 

  Further, the definitions are parameterized in a name, $x$. Can you,
  gentle reader, make a definition that eliminates this parameter and
  guarantees no accidental interaction between the replication
  machinery and the process being replicated -- i.e. no accidental
  sharing of names used by the process to get its work done and the
  name(s) used by the replication to effect copying. This latter
  revision of the definition of replication is crucial to obtaining
  the expected identity $!!P \sim !P$.
\end{remark}

\begin{remark}\label{rem:paradoxical_combinator}
  The reader familiar with the lambda calculus will have noticed the
  similarity between $D$ and the paradoxical combinator.

  [Ed. note: the existence of this seems to suggest we have to be more
  restrictive on the set of processes and names we admit if we are to
  support no-cloning.]
\end{remark}

\subsubsection{Bisimulation}

The computational dynamics gives rise to another kind of equivalence,
the equivalence of computational behavior. As previously mentioned
this is typically captured \emph{via} some form of bisimulation.

% The notion we use in this paper is weak barbed bisimulation
% \cite{milner91polyadicpi}.

The notion we use in this paper is derived from weak barbed
bisimulation \cite{milner91polyadicpi}. 

\begin{definition}
An \emph{observation relation}, $\downarrow_{\mathcal N}$, over a set
of names, $\mathcal N$, is the smallest relation satisfying the rules
below.

\infrule[Out-barb]{y \in {\mathcal N}, \; x \nameeq y}
		  {\outputp{x}{v} \downarrow_{\mathcal N} x}
\infrule[Par-barb]{\mbox{$P\downarrow_{\mathcal N} x$ or $Q\downarrow_{\mathcal N} x$}}
		  {\binpar{P}{Q} \downarrow_{\mathcal N} x}

We write $P \Downarrow_{\mathcal N} x$ if there is $Q$ such that 
$P \wred Q$ and $Q \downarrow_{\mathcal N} x$.
\end{definition}

\begin{definition}
%\label{def.bbisim}
An  ${\mathcal N}$-\emph{barbed bisimulation} over a set of names, ${\mathcal N}$, is a symmetric binary relation 
${\mathcal S}_{\mathcal N}$ between agents such that $P\rel{S}_{\mathcal N}Q$ implies:
\begin{enumerate}
\item If $P \red P'$ then $Q \wred Q'$ and $P'\rel{S}_{\mathcal N} Q'$.
\item If $P\downarrow_{\mathcal N} x$, then $Q\Downarrow_{\mathcal N} x$.
\end{enumerate}
$P$ is ${\mathcal N}$-barbed bisimilar to $Q$, written
$P \wbbisim_{\mathcal N} Q$, if $P \rel{S}_{\mathcal N} Q$ for some ${\mathcal N}$-barbed bisimulation ${\mathcal S}_{\mathcal N}$.
\end{definition}

$\mathcal{R} \subseteq \pi \times \pi$

$P \mathcal{R} Q => \forall P'. P \red P' \Rightarrow \exists Q'. Q \red Q', P' \mathcal{R} Q'$

$P \vdash x \Rightarrow Q \vdash x$

\begin{mathpar}
  \inferrule*[lab=Out-barb]{x \nameeq y}{{y}!\langle{Q}\rangle \vdash x}
  \and
  \inferrule*[lab=Par-barb]{\mbox{$P\vdash x$ or $Q\vdash x$}}{\binpar{P}{Q} \vdash x}
\end{mathpar}

\subsubsection{Contexts}

One of the principle advantages of computational calculi like the
$\pi$-calculus is a well-defined notion of context,
contextual-equivalence and a correlation between
contextual-equivalence and notions of bisimulation. The notion of
context allows the decomposition of a process into (sub-)process and
its syntactic environment, its context. Thus, a context may be
thought of as a process with a ``hole'' (written $\Box$) in it. The
application of a context $M$ to a process $P$, written $M[P]$, is
tantamount to filling the hole in $M$ with $P$. In this paper we do
not need the full weight of this theory, but do make use of the notion
of context in the proof the main theorem. 

\begin{mathpar}
  \inferrule* [lab=summation] {} {{M_{M},M_{N}} \bc \Box \;|\; x.M_{A} \;|\; M_{M}+M_{N}}
  \and
  \inferrule* [lab=agent] {} {{M_{A}} \bc (\vec{x})M_{P} \;| \; \clift{P_0,\ldots,M_{P},\ldots,P_N}}
  \and \\
  \inferrule* [lab=process] {} {{M_{P}} \bc M_{N} \;| \;P|M_{P} }
\end{mathpar} 

\begin{mathpar}
  \inferrule* [lab=sychronization] {} {M_{N} \bc \Box \;|\; x?M_{F} \;|\; x!M_{C}}
  \and
  \inferrule* [lab=abstraction] {} {{M_{F}} \bc (x)M_{P} }
  \and
  \inferrule* [lab=concretion] {} {{M_{C}} \bc \langle M_{P} \rangle }
  \and \\
  \inferrule* [lab=process] {} {{M_{P}} \bc M_{N} \;| \;P|M_{P} }
\end{mathpar}

\begin{definition}[contextual application] Given a context $M$, and
  process $P$, we define the \emph{contextual application}, $M[P] :=
  M\{P/\Box\}$. That is, the contextual application of M to P is the
  substitution of $P$ for $\Box$ in $M$.
\end{definition}

$\meaningof{-} : L \to \mathcal{P}(\pi)$

\begin{mathpar}
  \inferrule* [lab=collection] {} {\meaningof{true} = \pi, \and \meaningof{~E} = \pi \setminus \meaningof{E}, \and \meaningof{E_{1} \& E_{2}} = \meaningof{E_{1}} \cap \meaningof{E_{2}}}
\end{mathpar}

\begin{mathpar}
  \inferrule* [lab=structure] {} {\meaningof{0} = \{ P \in \pi | P \equiv 0 \}, \and \\ \meaningof{E_1 | E_2} = \{ P \in \pi | P \equiv P_{1} | P_{2}, P_{1} \in \meaningof{E_{1}}, P_{2} \in \meaningof{E_2}\} }
\end{mathpar}

\begin{mathpar}
 \inferrule* [lab=behavior] {} {\meaningof{\langle a?b \rangle E} = \{ P \in \pi | P \equiv Q | u?(y)P', \\ \and \\\\ \and \\ \;\;\; u \in \meaningof{a}, \forall z.P'\{z/y\} \in \meaningof{E\{z/b\}}\}, \and \\ \meaningof{a!E} = \{ P \in \pi | P \equiv Q | x!\langle P' \rangle, x \in \meaningof{a} P' \in \meaningof{E}\} }
\end{mathpar}

\begin{mathpar}
 \inferrule* [lab=nominal] {} {\meaningof{\quotep{E}} = \{ \quotep{P} \in \quotep{\pi} | P \in \meaningof{E} \}, \and \meaningof{\quotep{P}} = \{ \quotep{Q} \in \quotep{\pi} | P \equiv Q \} \and \\ \meaningof{@\quotep{E}} = \{ P \in \pi | P \equiv @x, x \in \meaningof{E} \}}
\end{mathpar}

\begin{eqnarray*}
  \\
  \meaningof{-} : TS \to ST
\end{eqnarray*}

\begin{eqnarray*}
  \\
  L : TS \to ST
\end{eqnarray*}

\begin{eqnarray*}
  \\
  P \models E \iff P \in \meaningof{E}
\end{eqnarray*}

\begin{eqnarray*}
  P \approx_{L} Q \iff \forall E \in L. P \models E \iff Q \models E
\end{eqnarray*}

\begin{eqnarray*}
  P \approx_{K} Q
\end{eqnarray*}

\begin{eqnarray*}
  P \approx Q
\end{eqnarray*}

$\approx_{K} = \approx = \approx_{L}$

\subsubsection{Contextual duality}

Note that contexts extend the quotation operation to a family of
operations from processes to names. Given a context, $M$, we can
define a \emph{nominal context}, $\quotep{M}$ by $\quotep{M}[P] :=
\quotep{M[P]}$. To foreshadow what is to come we observe that these
operations enjoy a duality with processes very much like the duality
between vectors and maps from vectors to scalars.

Further, because the calculus is essentially higher-order, we have a
correspondence between contexts and processes. More specifically,
given a name $x$ and a context $M$ we can construct $M^{*}_{x}$ such
that 

\begin{mathpar}
  M^{*}_{x} | \lift{x}{P} \red M[P]
\end{mathpar}

namely,

\begin{mathpar}
  M^{*}_{x} := x?(u).M[\dropn{u}]
\end{mathpar}

The dependence of $M^{*}_{x}$ on a name makes it an abstraction, 

\begin{mathpar}
  M^{*} := (x)x?(u).M[\dropn{u}]
\end{mathpar}

\subsection{Additional notation}

It will sometimes be convenient to denote the process a name
quotes. We already have the notation $x = \quotep{P}$, but it will be
convenient to introduce an alternate notation, $\procn{x}$, when we
want to emphasize the connection to the use of the name. Note that, by
virtue of name equivalence, $\quotep{\procn{x}} \nameeq x$; so, the
notation is consistent with previous definitions.

Further, because names have structure it is possible to effect
substitutions on the basis of that structure. This means we need to
upgrade our notation for substitutions, which we accomplish by
adapting comprehension notation. Thus,

\begin{mathpar}
  P\{ y / x : x \in S \}
\end{mathpar}

is interpreted to mean the process derived from P by replacing (in a
capture-avoiding manner) each occurrence of $x$ in $S$ by $y$. For example,

\begin{mathpar}
  P\{ \quotep{\procn{x}|\procn{x}} / x : x \in \freenames{P} \}
\end{mathpar}

will replace each (occurrence) of a free name $x$ in $P$ by
$\quotep{\procn{x}|\procn{x}}$.

Also, we will avail ourselves of the notation $x^{L}$ and $x^{R}$ to
denote injections of a name into disjoint copies of the name
space. There are numerous ways to accomplish this. One example can be
found in \cite{MeredithR05}. This notation overloads to vectors of
names: $\vec{x}^{\pi} := (x_{i}^{\pi} \; : \; 0 \leq i < |\vec{x}| )$ where $\pi \in \{L,R\}$.

We also use $P^{\Box} := P|\Box$.

In \cite{MeredithR05} an interpretation of the new operator is
given. It turns out that there are several possible interpretations
all enjoying the requisite algebraic properties of the operator (see
\cite{milner91polyadicpi}). We will therefore make liberal use of
$(\nu\; \vec{x})P$.

% subsection the_syntax_and_semantics_of_the_notation_system (end)   

\input{qm2pi.qmops} 

\input{qm2pi.sterngerlach} 

\input{qm2pi.metric} 

% section concurrent_process_calculi (end)

%\input{qm2pi.proofsketch}

% section proof sketch (end)

%\input{qm2pi.slviaknots} 

% section spatial logic via knots (end)

\input{qm2pi.conclusion}

% section conclusion (end)

%\input{qm2pi.dtcodes} 

% section wiring algorithm (end)

\input{qm2pi.ack} 

% section acknowledgments (end)

\newpage


\bibliographystyle{plain}   
\bibliography{../../biblios/main.bib}

\input{qm2pi.rhodetails}

\end{document}



% section front matter (end)

\section{Introduction}\label{sec:introduction} % (fold)
In this draft of the material i am going to have to dispense with the
usual writing conventions adopted in papers on these topics. i'm going
to have adopt whatever tone i need at the time i'm writing up the
calculations. Sometimes this may be very conversational; others it may
be the barest mathematical grunts; others still it may be that i have
lifted text from one of my other papers because the exposition of some
point was better said there. i hope that my readers are not unduly put
out by this decision. i'm not doing this to flout convention or be
rebellious. i find these calculations very technically challenging. To
keep everything going technically, something has to give; i have to
let go of some cognitive burden. So, the academic writing style --
with all of its trade-offs in terms of facilitating technical
communication -- is what i'm letting go of. Perhaps subsequent drafts
can be tightened and polished, but for now, i'm going to speak as if
we were sitting together in a coffee shop with a laptop, wifi and a
pad of paper and a pencil.

So, here's what i have to say. We -- you and i, comfortably ensconced
in our coffee shop and well-equipped with our tools -- can realize and
carry out the calculations of quantum mechanics over a very different
formal theory of dynamics, a formal theory of dynamics that
corresponds to a theory of concurrent computation with
\emph{reflection}. It has the advantage that the underlying theory is
already `quantized', but supports analogues all of the continuuous
operations. Strikingly, this underlying theory has recently been
connected with a notion of metric that we can show, by calculating
together, coincides with the metric induced by the inner product.

There are a lot of reasons why you might be interested in seeing
calculations of this form. Here's why i'm interested. For the past
several centuries there has been no competitor to the ``Newtonian''
account of dynamics. As a result the predominant share of accounts of
dynamical systems and situations have had to be formulated in terms of
the Newtonian machinery. i view this as an intellectually dangerous
position to occupy. Everything, despite it's intrinsic shape, turns
into a nail to be hit with this hammer. Recently, however, the theory
of computation has matured to the point where we have candidates for
theories of dynamics that offer very different perspective on
reasoning about dynamical systems and situations. Testing these
candidates against very successful accounts of dynamical situations,
like quantum mechanics, is going to give us some sense of how mature
they are and some measure of the quality of these accounts of
dynamics.

\subsection{Summary of contributions and outline of paper}

So, we're going to develop an interpretation of the operations of
quantum mechanics normally interpreted by Hilbert spaces and
operators. We're going to do this over a theory of computation. Note
that this is very different than the usual quantum computation program
which develops notions of computation over quantum mechanics. Rather,
we are developing a story that aligns with Wheeler's slogan: It from
Bit. To do this we will first provide an account of the theory of
computation at play here. Then we will dive into a calculation-driven
interpretation of the operations of quantum mechanics.

The reason we take this approach is that -- until very recently --
there hasn't been an axiomatic account of quantum mechanics. As a
result there has been no sharp delineation of the mathematical theory
supporting interpretation of the physical theory and the physical
theory, itself. So, ambient features of the maths are free to be
exploited (or supressed) without a real accounting of their physical
relevance. There is no sharp statement ``here's the physical theory''
qua \emph{theory} and ``here's the mathematical interpretation''
enabling a judgment of how faithful the interpretation is -- apart
from experimental observation. When there is an axiomatic account we
can judge how well a given mathematical formalism supports an
interpretation of the axioms, independent of
experimentation. Likewise, we can judge how well we have captured our
physical evidence and experience with our axiomatics, independent of
any specific mathematical implementation, with accidental detail that
may or may not have physical significance. 

In lieu of a fully fleshed out and vetted axiomatic account of quantum
mechanics, interpreting the operational notions in service of modeling
physical systems will have to suffice. In other words, we are not in
the business of providing a model of Hilbert spaces and operators. We
are in the business of providing a model of quantum mechanics because
we are motivated by testing our notions of dynamics against physical
theory; and, the predictive calculations of the physical theory must
serve as the best formulation -- shy of a fully fleshed out axiomatic
account -- of the physical theory itself (as they have for scientific
theories since time immemorial). Put another way, despite a
whole-hearted commitment to an It-from-Bit ontology, we are firmly
aligned with the shut-up-and-calculate camp as the best way to obtain
results either from the physical perspective or as a quality assurance
measure of our fledgling theory of dynamics.

In detail, we present a reflective process calculus. Then we develop
intuitive correspondences between the notions available in this
calculus and the usual physical notions supporting quantum mechanical
calculations. Thus, 

\begin{table}[htp]
  \center{
    \fbox{
      \begin{tabular}{c|c}
        quantum mechanics & process calculus \\
        \hline
        scalar & name \\
        state vector & process \\
        dual & contextual duals \\
        matrix & formal sums of process-context-dual pairs \\
        orthogonality & process annihilation \\
        inner product & execution-formula + quoting
      \end{tabular}
    }
  }
  \caption{QM - process calculi correspondences}
\end{table}

Then we tighten up these intuitions to operational definitions. We
employ the Dirac notation as the best proxy we can find for an
abstract syntax of the quantum mechanical notions. The definitions we
develop put us in contact with equational constraints coming from the
theory that we demonstrate the definitions and calculations satisfy.

This puts us in a position to shut up and calculate for the
Stern-Gerlach experimental set up, showing how these predictive
calculations become calculations on processes in our theory of a
reflective process calculus.

Penultimately, we demonstrate that the notion of metric coming from
the inner product coincides with the notion of metric available from
the theory of bisimulation. This demonstration gives us the right to
think of space as arising from behavior. Finally, we consider where we
might go from the new vantage point we have obtained.

% section introduction (end) 
 
% section introduction (end)

% \documentclass[12pt]{llncs}
%\documentclass{jktr}

\usepackage[pdftex]{hyperref}                   
\usepackage {listings}
\usepackage {mathpartir}
\usepackage{bcprules}
%\usepackage{listings}
                       
\usepackage{graphicx} 
%\usepackage[margins=2.5cm,nohead,nofoot]{geometry}
%\usepackage{geometry}
\usepackage{amsfonts}
\usepackage{amstext}
\usepackage{latexsym}
\usepackage{amssymb}
\usepackage{color}


%\include{myPreamble}
\include{qm2pi.local} 

%\ifpdf
%\usepackage[pdftex]{graphicx}
%\else
%\usepackage{graphicx}
%\fi

 % \ifpdf
%  \usepackage{pdfsync}
%  \if


%\title{Brief Article}
%\author{David F. Snyder}
%\author{L.G. Meredith}

%\address{Dept. of Math., Texas State University--San Marcos, San Marcos, TX 78666}
       
\pagestyle{empty}


\begin{document}

\lstset{language=[Objective]Caml,frame=shadowbox}

\input{qm2pi.front}

% section front matter (end)

\input{qm2pi.intro} 
 
% section introduction (end)

% \input{qm2pi.knotations} 

% section notation (end)

\input{qm2pi.process.calculi} 

% section concurrent_process_calculi_and_spatial_logics_ (end)
    
%\input{qm2pi.knots2pi} 

%\input{qm2pi.trefoil} 

%\input{qm2pi.mainthm} 

% subsection basic_interpretation (end)

%\input{qm2pi.rho.presentation} 
\subsection{The syntax and semantics of the notation system}\label{sub:the_syntax_and_semantics_of_the_notation_system} % (fold)

We now summarize a technical presentation of the calculus that
embodies our theory of dynamics. The typical presentation of such a
calculus follows the style of giving generators and relations on
them. The grammar, below, describing term constructors, freely
generates the set of processes, $\Proc$. This set is then quotiented
by a relation known as structural congruence and it is over this set
that the notion of dynamics is expressed. This presentation is
essentially that of \cite{MeredithR05} with the addition of
polyadicity and summation. For readability we have relegated some of
the technical subtleties to an appendix.

\subsubsection{Process grammar}\label{subsub:process_grammar}

\begin{mathpar}
  \inferrule* [lab=synchronization] {} {{M} \bc \pzero \;|\; x?F \;|\; x!C }
  \and
  \inferrule* [lab=abstraction] {} {{F} \bc (x)P}
  \and
  \inferrule* [lab=concretion] {} {{C} \bc \langle Q \rangle}
  \and
  \inferrule* [lab=process] {} {{P,Q} \bc M \;| \;P|Q \;|\; @{x}}
  \and
  \inferrule* [lab=name] {} {{x} \bc \quotep{P}}
\end{mathpar} 

Note that $\vec{x}$ (resp. $\vec{P}$) denotes a vector of names
(resp. processes) of length $|\vec{x}|$ (resp. $|\vec{P}|$). We adopt
the following useful abbreviations.

\begin{mathpar}
   x?(\vec{y}).P := x.(\vec{y})P \and  x\clift{\vec{P}} := x.\clift{\vec{P}}
   \and x!(y) := \lift{x}{\dropn{y}}
   \and \Pi_{i=0}^{n-1}P_i := P_0 | \ldots | P_{n-1}
\end{mathpar}

\subsubsection{Structural congruence}

\paragraph{Free and bound names and alpha-equivalence.} At the
core of structural equivalence is alpha-equivalence which identifies
process that are the same up to a change of variable. Formally, we
recognize the distinction between free and bound names. The free names
of a process, $\freenames{P}$, may be calculated recursively as
follows:

\begin{mathpar}
\freenames{\pzero} := \emptyset
  \and \\
  \freenames{x?(y).P} := \{ x \} \cup (\freenames{P} \setminus \{ y \})
  \and 
  \freenames{x!\langle P \rangle} := \{ x \} \cup \{ P \} 
  \and \\
  \freenames{P|Q} := \freenames{P} \cup \freenames{Q}
  \and \\
  \freenames{@{x}} := \{ x \}
\end{mathpar}

$\pi$
$\quotep{\pi}$

$\freenames{-} : \pi \to \mathcal{P}(\quotep{\pi})$

\begin{eqnarray*}
  \freenames{\pzero} & := & \emptyset \\
  \freenames{x?(y).P} & := & \{ x \} \cup (\freenames{P} \setminus \{ y \}) \\
  \freenames{x!\langle P \rangle} & := & \{ x \} \cup \{ P \} \\
  \freenames{P|Q} & := & \freenames{P} \cup \freenames{Q} \\
  \freenames{\dropn{x}} & := & \{ x \}
\end{eqnarray*}

The bound names of a process, $\boundnames{P}$, are those names occurring in $P$
that are not free. For example, in $x?(y).0$, the name $x$ is free, while $y$ is bound.

\begin{mathpar}
  \inferrule* [lab=monoidal-laws] {} { P|Q \equiv Q|P \and P|0 \equiv P \and P|(Q|R) \equiv (P|Q)|R }
\end{mathpar}

\begin{mathpar}
  \inferrule* [lab=alpha-equivalence] {} { (x)P \equiv (y)P\{y/x\} \and y \not\in \freenames{P} }
\end{mathpar}

\begin{definition}
Then two processes, $P,Q$, are alpha-equivalent if $P = Q\{\vec{y}/\vec{x}\}$ for
some $\vec{x} \in \boundnames{Q},\vec{y} \in \boundnames{P}$, where $Q\{\vec{y}/\vec{x}\}$
denotes the capture-avoiding substitution of $\vec{y}$ for $\vec{x}$ in $Q$.
\end{definition}

\begin{definition}
  The {\em structural congruence} \cite{SangiorgiWalker} , $\equiv$,
  between processes is the least congruence containing
  alpha-equivalence, satisfying the abelian monoid laws
  (associativity, commutativity and $\pzero$ as identity) for parallel
  composition $|$ and for summation $+$.
\end{definition}

\subsection{Name equivalence}

We take name equivalence, written $\nameeq$, to be the smallest
equivalence relation generated by the following rules.

\begin{mathpar}
\inferrule*[lab=Quote-drop]
{ }
{ \quotep{@{x}} \nameeq x }

\inferrule*[lab=Struct-equiv]
{ P \scong Q }
{ \quotep{P} \nameeq \quotep{Q} }
\end{mathpar}

The astute reader will have noticed that the mutual recursion of names
and processes imposes a mutual recursion on alpha-equivalence and
structural equivalence via name-equivalence. Fortunately, all of this
works out pleasantly and we may calculate in the natural way, free of
concern. The reader interested in the details is referred to the
appendix \ref{appendix:rho_details}.

\subsection{Substitution}

We use $\Proc$ for the set of processes, $\QProc$ for the set of
names, and $\id{\{}\vec{y} / \vec{x} \id{\}}$ to denote partial maps,
$s : \QProc \rightarrow \QProc$. A map, $s$ lifts, uniquely, to a map
on process terms, $\widehat{s} : \Proc \rightarrow \Proc$ by the
following equations.

\begin{mathpar}
  (0) \psubstp{Q}{P} := 0 \\
  (R \juxtap S) \psubstp{Q}{P}
  :=    
  (R)\psubstp{Q}{P} \juxtap (S) \psubstp{Q}{P} \\
  (x?(y).R) \psubstp{Q}{P}    
  :=    
  (x)\substp{Q}{P} (z)\concat( (R \psubstn{z}{y}) \psubstp{Q}{P} ) \\
  (\lift{x}{R}) \psubstp{Q}{P}  
  :=
  \lift{(x)\substp{Q}{P}}{ R \psubstp{Q}{P} } \\
%   (\dropn{x})  \psubstp{Q}{P}       
%   := 
%   \left\{ 
%     \begin{array}{ccc} 
%       \dropn{\quotep{Q}} & & x \nameeq \quotep{P} \\
%       \dropn{x} & & otherwise \\
%     \end{array}
%   \right. 
  (\dropn{x})  \psubstp{Q}{P}       
  := 
  \left\{ 
    \begin{array}{ccc} 
      Q & & x \nameeq \quotep{P} \\
      \dropn{x} & & otherwise \\
    \end{array}
  \right.
\end{mathpar}
 

where

\begin{eqnarray}
  (x)\id{\{} \lpquote Q \rpquote / \lpquote P \rpquote \id{\}}            = 
  \left\{ 
    \begin{array}{ccc}
      \lpquote Q \rpquote & & x \nameeq \lpquote P \rpquote \\
      x & & otherwise \\
    \end{array}
  \right. \nonumber
\end{eqnarray}

and $z$ is chosen distinct from $\quotep{P}$, $\quotep{Q}$, the free
names in $Q$, and all the names in $R$. Our $\alpha$-equivalence will
be built in the standard way from this substitution.

\begin{remark}\label{rem:no_self_referential_names}
  One consequence of these definitions is that $\forall P. \quotep{P}
  \not\in \freenames{P}$.
\end{remark}

\subsection{ Dynamic quote: an example }

Anticipating something of what's to come, consider applying the
substitution, $\widehat{\id{\{}u / z \id{\}}}$, to the following pair
of processes, $\lift{w}{y!(z)}$ and $w[ \lpquote y!(z) \rpquote ]$.

\begin{eqnarray}
	\lift{w}{y!(z)}\widehat{\id{\{}u / z \id{\}}}
		& = &
		\lift{w}{y!(u)} \nonumber\\
	w[ \lpquote y!(z) \rpquote ] \widehat{ \id{\{}u / z \id{\}} }
		& = &
		w[ \lpquote y!(z) \rpquote ] \nonumber
\end{eqnarray}

Because the body of the process between quotes is impervious to
substitution, we get radically different answers. In fact, by
examining the first process in an input context,
e.g. $x?(z).\lift{w}{y!(z)}$, we see that the process under the lift
operator may be shaped by prefixed inputs binding a name inside it. In
this sense, the lift operator will be seen as a way to dynamically
construct processes before reifying them as names.

Finally equipped with these standard features we can present the
dynamics of the calculus.

\subsubsection{Operational semantics} 

Finally, we introduce the computational dynamics. What marks these
algebras as distinct from other more traditionally studied algebraic
structures, e.g. vector spaces or polynomial rings, is the manner in
which dynamics is captured. In traditional structures, dynamics is typically
expressed through morphisms between such structures, as in linear maps
between vector spaces or morphisms between rings. In algebras
associated with the semantics of computation, the dynamics is
expressed as part of the algebraic structure itself, through a
reduction reduction relation typically denoted by $\red$. Below, we
give a recursive presentation of this relation for the calculus used
in the encoding.

$\red \subseteq \pi \times \pi$
$\red : \pi \to \mathcal{P}(\pi)$

\begin{mathpar}
  \inferrule* [lab=Comm] { \textsf{match}( x_{src}, x_{trgt} ) } { x_{trgt}?(y)P \; | \; x_{src}!\langle {Q} \rangle \red P\{\quotep{Q}/y}\} }
  \and \\
  \inferrule* [lab=Par] {{P} \red {P}'} {{{P} | {Q}} \red {{P}' | {Q}}}
  \and
  \inferrule* [lab=Equiv]{{{P} \scong {P}'} \andalso {{P}' \red {Q}'} \andalso {{Q}' \scong {Q}}}{{P} \red {Q}}
\end{mathpar}

\begin{eqnarray*}
  match_{\equiv} (\quotep{P},\quotep{Q}) & := & P \equiv Q \\
  match_{\dagger}(\quotep{P},\quotep{Q}) & := & \forall R. P|Q \red^{*} R => R \red^{*} 0 \\
  match_{K}(\quotep{P},\quotep{Q}) & := & K \mbox{ for some context } K
\end{eqnarray*}

$u?(x)P | u!\langle Q \rangle \red P\{\quotep{Q}/x\}$

%We write $\wred$ for $\red^*$, and $P\red$ if $\exists Q $ such that $ P \red Q$.
We write $P\red$ if $\exists Q $ such that $ P \red Q$ and $P\not\red$, otherwise.

\section{Replication}

As mentioned before, it is known that replication (and hence
recursion) can be implemented in a higher-order process algebra
\cite{SangiorgiWalker}. As our first example of calculation with the
machinery thus far presented we give the construction explicitly in
the {\rhoc}.

\begin{eqnarray}
	D_{x} & := & \prefix{x}{y}{(\binpar{\outputp{x}{y}}{@{y}})} \nonumber\\
	\bangp_{x}{P} & := & \binpar{{x}!\langle{\binpar{D_{x}}{P}}\rangle}{D_{x}} \nonumber
\end{eqnarray}

\begin{eqnarray}
	\bangp_{x}{P} & & \nonumber\\
	=
	& {x}!\langle{(\prefix{x}{y}{(\outputp{x}{y} | @{y})) | P}}\rangle 
	      | \prefix{x}{y}{(\outputp{x}{y} | @{y})} & \nonumber\\
	\red
	& (\outputp{x}{y} | @{y})\substn{\quotep{(\prefix{x}{y}{(@{y} | \outputp{x}{y})) | P}}}{y} & \nonumber\\
	=
	& \outputp{x}{\quotep{(\prefix{x}{y}{(\outputp{x}{y} | @{y})) | P}}}
	  | {(\prefix{x}{y}{(\outputp{x}{y} | @{y})) | P}} & \nonumber\\
	\red
	& \ldots & \nonumber\\
	\red^*
	& P | P | \ldots & \nonumber
\end{eqnarray}

Of course, this encoding, as an implementation, runs away, unfolding
$\bangp{P}$ eagerly. A lazier and more implementable replication
operator, restricted to input-guarded processes, may be obtained as follows.

\begin{eqnarray}
\bangp{\prefix{u}{v}{P}} 
	:= 
	\binpar{\lift{x}{\prefix{u}{v}{(\binpar{D(x)}{P})}}}{D(x)} \nonumber
\end{eqnarray}

\begin{remark}
  Note that the lazier definition still does not deal with summation
  or mixed summation (i.e. sums over input and output). The reader is
  invited to construct definitions of replication that deal with these
  features. 

  Further, the definitions are parameterized in a name, $x$. Can you,
  gentle reader, make a definition that eliminates this parameter and
  guarantees no accidental interaction between the replication
  machinery and the process being replicated -- i.e. no accidental
  sharing of names used by the process to get its work done and the
  name(s) used by the replication to effect copying. This latter
  revision of the definition of replication is crucial to obtaining
  the expected identity $!!P \sim !P$.
\end{remark}

\begin{remark}\label{rem:paradoxical_combinator}
  The reader familiar with the lambda calculus will have noticed the
  similarity between $D$ and the paradoxical combinator.

  [Ed. note: the existence of this seems to suggest we have to be more
  restrictive on the set of processes and names we admit if we are to
  support no-cloning.]
\end{remark}

\subsubsection{Bisimulation}

The computational dynamics gives rise to another kind of equivalence,
the equivalence of computational behavior. As previously mentioned
this is typically captured \emph{via} some form of bisimulation.

% The notion we use in this paper is weak barbed bisimulation
% \cite{milner91polyadicpi}.

The notion we use in this paper is derived from weak barbed
bisimulation \cite{milner91polyadicpi}. 

\begin{definition}
An \emph{observation relation}, $\downarrow_{\mathcal N}$, over a set
of names, $\mathcal N$, is the smallest relation satisfying the rules
below.

\infrule[Out-barb]{y \in {\mathcal N}, \; x \nameeq y}
		  {\outputp{x}{v} \downarrow_{\mathcal N} x}
\infrule[Par-barb]{\mbox{$P\downarrow_{\mathcal N} x$ or $Q\downarrow_{\mathcal N} x$}}
		  {\binpar{P}{Q} \downarrow_{\mathcal N} x}

We write $P \Downarrow_{\mathcal N} x$ if there is $Q$ such that 
$P \wred Q$ and $Q \downarrow_{\mathcal N} x$.
\end{definition}

\begin{definition}
%\label{def.bbisim}
An  ${\mathcal N}$-\emph{barbed bisimulation} over a set of names, ${\mathcal N}$, is a symmetric binary relation 
${\mathcal S}_{\mathcal N}$ between agents such that $P\rel{S}_{\mathcal N}Q$ implies:
\begin{enumerate}
\item If $P \red P'$ then $Q \wred Q'$ and $P'\rel{S}_{\mathcal N} Q'$.
\item If $P\downarrow_{\mathcal N} x$, then $Q\Downarrow_{\mathcal N} x$.
\end{enumerate}
$P$ is ${\mathcal N}$-barbed bisimilar to $Q$, written
$P \wbbisim_{\mathcal N} Q$, if $P \rel{S}_{\mathcal N} Q$ for some ${\mathcal N}$-barbed bisimulation ${\mathcal S}_{\mathcal N}$.
\end{definition}

$\mathcal{R} \subseteq \pi \times \pi$

$P \mathcal{R} Q => \forall P'. P \red P' \Rightarrow \exists Q'. Q \red Q', P' \mathcal{R} Q'$

$P \vdash x \Rightarrow Q \vdash x$

\begin{mathpar}
  \inferrule*[lab=Out-barb]{x \nameeq y}{{y}!\langle{Q}\rangle \vdash x}
  \and
  \inferrule*[lab=Par-barb]{\mbox{$P\vdash x$ or $Q\vdash x$}}{\binpar{P}{Q} \vdash x}
\end{mathpar}

\subsubsection{Contexts}

One of the principle advantages of computational calculi like the
$\pi$-calculus is a well-defined notion of context,
contextual-equivalence and a correlation between
contextual-equivalence and notions of bisimulation. The notion of
context allows the decomposition of a process into (sub-)process and
its syntactic environment, its context. Thus, a context may be
thought of as a process with a ``hole'' (written $\Box$) in it. The
application of a context $M$ to a process $P$, written $M[P]$, is
tantamount to filling the hole in $M$ with $P$. In this paper we do
not need the full weight of this theory, but do make use of the notion
of context in the proof the main theorem. 

\begin{mathpar}
  \inferrule* [lab=summation] {} {{M_{M},M_{N}} \bc \Box \;|\; x.M_{A} \;|\; M_{M}+M_{N}}
  \and
  \inferrule* [lab=agent] {} {{M_{A}} \bc (\vec{x})M_{P} \;| \; \clift{P_0,\ldots,M_{P},\ldots,P_N}}
  \and \\
  \inferrule* [lab=process] {} {{M_{P}} \bc M_{N} \;| \;P|M_{P} }
\end{mathpar} 

\begin{mathpar}
  \inferrule* [lab=sychronization] {} {M_{N} \bc \Box \;|\; x?M_{F} \;|\; x!M_{C}}
  \and
  \inferrule* [lab=abstraction] {} {{M_{F}} \bc (x)M_{P} }
  \and
  \inferrule* [lab=concretion] {} {{M_{C}} \bc \langle M_{P} \rangle }
  \and \\
  \inferrule* [lab=process] {} {{M_{P}} \bc M_{N} \;| \;P|M_{P} }
\end{mathpar}

\begin{definition}[contextual application] Given a context $M$, and
  process $P$, we define the \emph{contextual application}, $M[P] :=
  M\{P/\Box\}$. That is, the contextual application of M to P is the
  substitution of $P$ for $\Box$ in $M$.
\end{definition}

$\meaningof{-} : L \to \mathcal{P}(\pi)$

\begin{mathpar}
  \inferrule* [lab=collection] {} {\meaningof{true} = \pi, \and \meaningof{~E} = \pi \setminus \meaningof{E}, \and \meaningof{E_{1} \& E_{2}} = \meaningof{E_{1}} \cap \meaningof{E_{2}}}
\end{mathpar}

\begin{mathpar}
  \inferrule* [lab=structure] {} {\meaningof{0} = \{ P \in \pi | P \equiv 0 \}, \and \\ \meaningof{E_1 | E_2} = \{ P \in \pi | P \equiv P_{1} | P_{2}, P_{1} \in \meaningof{E_{1}}, P_{2} \in \meaningof{E_2}\} }
\end{mathpar}

\begin{mathpar}
 \inferrule* [lab=behavior] {} {\meaningof{\langle a?b \rangle E} = \{ P \in \pi | P \equiv Q | u?(y)P', \\ \and \\\\ \and \\ \;\;\; u \in \meaningof{a}, \forall z.P'\{z/y\} \in \meaningof{E\{z/b\}}\}, \and \\ \meaningof{a!E} = \{ P \in \pi | P \equiv Q | x!\langle P' \rangle, x \in \meaningof{a} P' \in \meaningof{E}\} }
\end{mathpar}

\begin{mathpar}
 \inferrule* [lab=nominal] {} {\meaningof{\quotep{E}} = \{ \quotep{P} \in \quotep{\pi} | P \in \meaningof{E} \}, \and \meaningof{\quotep{P}} = \{ \quotep{Q} \in \quotep{\pi} | P \equiv Q \} \and \\ \meaningof{@\quotep{E}} = \{ P \in \pi | P \equiv @x, x \in \meaningof{E} \}}
\end{mathpar}

\begin{eqnarray*}
  \\
  \meaningof{-} : TS \to ST
\end{eqnarray*}

\begin{eqnarray*}
  \\
  L : TS \to ST
\end{eqnarray*}

\begin{eqnarray*}
  \\
  P \models E \iff P \in \meaningof{E}
\end{eqnarray*}

\begin{eqnarray*}
  P \approx_{L} Q \iff \forall E \in L. P \models E \iff Q \models E
\end{eqnarray*}

\begin{eqnarray*}
  P \approx_{K} Q
\end{eqnarray*}

\begin{eqnarray*}
  P \approx Q
\end{eqnarray*}

$\approx_{K} = \approx = \approx_{L}$

\subsubsection{Contextual duality}

Note that contexts extend the quotation operation to a family of
operations from processes to names. Given a context, $M$, we can
define a \emph{nominal context}, $\quotep{M}$ by $\quotep{M}[P] :=
\quotep{M[P]}$. To foreshadow what is to come we observe that these
operations enjoy a duality with processes very much like the duality
between vectors and maps from vectors to scalars.

Further, because the calculus is essentially higher-order, we have a
correspondence between contexts and processes. More specifically,
given a name $x$ and a context $M$ we can construct $M^{*}_{x}$ such
that 

\begin{mathpar}
  M^{*}_{x} | \lift{x}{P} \red M[P]
\end{mathpar}

namely,

\begin{mathpar}
  M^{*}_{x} := x?(u).M[\dropn{u}]
\end{mathpar}

The dependence of $M^{*}_{x}$ on a name makes it an abstraction, 

\begin{mathpar}
  M^{*} := (x)x?(u).M[\dropn{u}]
\end{mathpar}

\subsection{Additional notation}

It will sometimes be convenient to denote the process a name
quotes. We already have the notation $x = \quotep{P}$, but it will be
convenient to introduce an alternate notation, $\procn{x}$, when we
want to emphasize the connection to the use of the name. Note that, by
virtue of name equivalence, $\quotep{\procn{x}} \nameeq x$; so, the
notation is consistent with previous definitions.

Further, because names have structure it is possible to effect
substitutions on the basis of that structure. This means we need to
upgrade our notation for substitutions, which we accomplish by
adapting comprehension notation. Thus,

\begin{mathpar}
  P\{ y / x : x \in S \}
\end{mathpar}

is interpreted to mean the process derived from P by replacing (in a
capture-avoiding manner) each occurrence of $x$ in $S$ by $y$. For example,

\begin{mathpar}
  P\{ \quotep{\procn{x}|\procn{x}} / x : x \in \freenames{P} \}
\end{mathpar}

will replace each (occurrence) of a free name $x$ in $P$ by
$\quotep{\procn{x}|\procn{x}}$.

Also, we will avail ourselves of the notation $x^{L}$ and $x^{R}$ to
denote injections of a name into disjoint copies of the name
space. There are numerous ways to accomplish this. One example can be
found in \cite{MeredithR05}. This notation overloads to vectors of
names: $\vec{x}^{\pi} := (x_{i}^{\pi} \; : \; 0 \leq i < |\vec{x}| )$ where $\pi \in \{L,R\}$.

We also use $P^{\Box} := P|\Box$.

In \cite{MeredithR05} an interpretation of the new operator is
given. It turns out that there are several possible interpretations
all enjoying the requisite algebraic properties of the operator (see
\cite{milner91polyadicpi}). We will therefore make liberal use of
$(\nu\; \vec{x})P$.

% subsection the_syntax_and_semantics_of_the_notation_system (end)   

\input{qm2pi.qmops} 

\input{qm2pi.sterngerlach} 

\input{qm2pi.metric} 

% section concurrent_process_calculi (end)

%\input{qm2pi.proofsketch}

% section proof sketch (end)

%\input{qm2pi.slviaknots} 

% section spatial logic via knots (end)

\input{qm2pi.conclusion}

% section conclusion (end)

%\input{qm2pi.dtcodes} 

% section wiring algorithm (end)

\input{qm2pi.ack} 

% section acknowledgments (end)

\newpage


\bibliographystyle{plain}   
\bibliography{../../biblios/main.bib}

\input{qm2pi.rhodetails}

\end{document}

 

% section notation (end)

\input{qm2pi.process.calculi} 

% section concurrent_process_calculi_and_spatial_logics_ (end)
    
%\documentclass[12pt]{llncs}
%\documentclass{jktr}

\usepackage[pdftex]{hyperref}                   
\usepackage {listings}
\usepackage {mathpartir}
\usepackage{bcprules}
%\usepackage{listings}
                       
\usepackage{graphicx} 
%\usepackage[margins=2.5cm,nohead,nofoot]{geometry}
%\usepackage{geometry}
\usepackage{amsfonts}
\usepackage{amstext}
\usepackage{latexsym}
\usepackage{amssymb}
\usepackage{color}


%\include{myPreamble}
\include{qm2pi.local} 

%\ifpdf
%\usepackage[pdftex]{graphicx}
%\else
%\usepackage{graphicx}
%\fi

 % \ifpdf
%  \usepackage{pdfsync}
%  \if


%\title{Brief Article}
%\author{David F. Snyder}
%\author{L.G. Meredith}

%\address{Dept. of Math., Texas State University--San Marcos, San Marcos, TX 78666}
       
\pagestyle{empty}


\begin{document}

\lstset{language=[Objective]Caml,frame=shadowbox}

\input{qm2pi.front}

% section front matter (end)

\input{qm2pi.intro} 
 
% section introduction (end)

% \input{qm2pi.knotations} 

% section notation (end)

\input{qm2pi.process.calculi} 

% section concurrent_process_calculi_and_spatial_logics_ (end)
    
%\input{qm2pi.knots2pi} 

%\input{qm2pi.trefoil} 

%\input{qm2pi.mainthm} 

% subsection basic_interpretation (end)

%\input{qm2pi.rho.presentation} 
\subsection{The syntax and semantics of the notation system}\label{sub:the_syntax_and_semantics_of_the_notation_system} % (fold)

We now summarize a technical presentation of the calculus that
embodies our theory of dynamics. The typical presentation of such a
calculus follows the style of giving generators and relations on
them. The grammar, below, describing term constructors, freely
generates the set of processes, $\Proc$. This set is then quotiented
by a relation known as structural congruence and it is over this set
that the notion of dynamics is expressed. This presentation is
essentially that of \cite{MeredithR05} with the addition of
polyadicity and summation. For readability we have relegated some of
the technical subtleties to an appendix.

\subsubsection{Process grammar}\label{subsub:process_grammar}

\begin{mathpar}
  \inferrule* [lab=synchronization] {} {{M} \bc \pzero \;|\; x?F \;|\; x!C }
  \and
  \inferrule* [lab=abstraction] {} {{F} \bc (x)P}
  \and
  \inferrule* [lab=concretion] {} {{C} \bc \langle Q \rangle}
  \and
  \inferrule* [lab=process] {} {{P,Q} \bc M \;| \;P|Q \;|\; @{x}}
  \and
  \inferrule* [lab=name] {} {{x} \bc \quotep{P}}
\end{mathpar} 

Note that $\vec{x}$ (resp. $\vec{P}$) denotes a vector of names
(resp. processes) of length $|\vec{x}|$ (resp. $|\vec{P}|$). We adopt
the following useful abbreviations.

\begin{mathpar}
   x?(\vec{y}).P := x.(\vec{y})P \and  x\clift{\vec{P}} := x.\clift{\vec{P}}
   \and x!(y) := \lift{x}{\dropn{y}}
   \and \Pi_{i=0}^{n-1}P_i := P_0 | \ldots | P_{n-1}
\end{mathpar}

\subsubsection{Structural congruence}

\paragraph{Free and bound names and alpha-equivalence.} At the
core of structural equivalence is alpha-equivalence which identifies
process that are the same up to a change of variable. Formally, we
recognize the distinction between free and bound names. The free names
of a process, $\freenames{P}$, may be calculated recursively as
follows:

\begin{mathpar}
\freenames{\pzero} := \emptyset
  \and \\
  \freenames{x?(y).P} := \{ x \} \cup (\freenames{P} \setminus \{ y \})
  \and 
  \freenames{x!\langle P \rangle} := \{ x \} \cup \{ P \} 
  \and \\
  \freenames{P|Q} := \freenames{P} \cup \freenames{Q}
  \and \\
  \freenames{@{x}} := \{ x \}
\end{mathpar}

$\pi$
$\quotep{\pi}$

$\freenames{-} : \pi \to \mathcal{P}(\quotep{\pi})$

\begin{eqnarray*}
  \freenames{\pzero} & := & \emptyset \\
  \freenames{x?(y).P} & := & \{ x \} \cup (\freenames{P} \setminus \{ y \}) \\
  \freenames{x!\langle P \rangle} & := & \{ x \} \cup \{ P \} \\
  \freenames{P|Q} & := & \freenames{P} \cup \freenames{Q} \\
  \freenames{\dropn{x}} & := & \{ x \}
\end{eqnarray*}

The bound names of a process, $\boundnames{P}$, are those names occurring in $P$
that are not free. For example, in $x?(y).0$, the name $x$ is free, while $y$ is bound.

\begin{mathpar}
  \inferrule* [lab=monoidal-laws] {} { P|Q \equiv Q|P \and P|0 \equiv P \and P|(Q|R) \equiv (P|Q)|R }
\end{mathpar}

\begin{mathpar}
  \inferrule* [lab=alpha-equivalence] {} { (x)P \equiv (y)P\{y/x\} \and y \not\in \freenames{P} }
\end{mathpar}

\begin{definition}
Then two processes, $P,Q$, are alpha-equivalent if $P = Q\{\vec{y}/\vec{x}\}$ for
some $\vec{x} \in \boundnames{Q},\vec{y} \in \boundnames{P}$, where $Q\{\vec{y}/\vec{x}\}$
denotes the capture-avoiding substitution of $\vec{y}$ for $\vec{x}$ in $Q$.
\end{definition}

\begin{definition}
  The {\em structural congruence} \cite{SangiorgiWalker} , $\equiv$,
  between processes is the least congruence containing
  alpha-equivalence, satisfying the abelian monoid laws
  (associativity, commutativity and $\pzero$ as identity) for parallel
  composition $|$ and for summation $+$.
\end{definition}

\subsection{Name equivalence}

We take name equivalence, written $\nameeq$, to be the smallest
equivalence relation generated by the following rules.

\begin{mathpar}
\inferrule*[lab=Quote-drop]
{ }
{ \quotep{@{x}} \nameeq x }

\inferrule*[lab=Struct-equiv]
{ P \scong Q }
{ \quotep{P} \nameeq \quotep{Q} }
\end{mathpar}

The astute reader will have noticed that the mutual recursion of names
and processes imposes a mutual recursion on alpha-equivalence and
structural equivalence via name-equivalence. Fortunately, all of this
works out pleasantly and we may calculate in the natural way, free of
concern. The reader interested in the details is referred to the
appendix \ref{appendix:rho_details}.

\subsection{Substitution}

We use $\Proc$ for the set of processes, $\QProc$ for the set of
names, and $\id{\{}\vec{y} / \vec{x} \id{\}}$ to denote partial maps,
$s : \QProc \rightarrow \QProc$. A map, $s$ lifts, uniquely, to a map
on process terms, $\widehat{s} : \Proc \rightarrow \Proc$ by the
following equations.

\begin{mathpar}
  (0) \psubstp{Q}{P} := 0 \\
  (R \juxtap S) \psubstp{Q}{P}
  :=    
  (R)\psubstp{Q}{P} \juxtap (S) \psubstp{Q}{P} \\
  (x?(y).R) \psubstp{Q}{P}    
  :=    
  (x)\substp{Q}{P} (z)\concat( (R \psubstn{z}{y}) \psubstp{Q}{P} ) \\
  (\lift{x}{R}) \psubstp{Q}{P}  
  :=
  \lift{(x)\substp{Q}{P}}{ R \psubstp{Q}{P} } \\
%   (\dropn{x})  \psubstp{Q}{P}       
%   := 
%   \left\{ 
%     \begin{array}{ccc} 
%       \dropn{\quotep{Q}} & & x \nameeq \quotep{P} \\
%       \dropn{x} & & otherwise \\
%     \end{array}
%   \right. 
  (\dropn{x})  \psubstp{Q}{P}       
  := 
  \left\{ 
    \begin{array}{ccc} 
      Q & & x \nameeq \quotep{P} \\
      \dropn{x} & & otherwise \\
    \end{array}
  \right.
\end{mathpar}
 

where

\begin{eqnarray}
  (x)\id{\{} \lpquote Q \rpquote / \lpquote P \rpquote \id{\}}            = 
  \left\{ 
    \begin{array}{ccc}
      \lpquote Q \rpquote & & x \nameeq \lpquote P \rpquote \\
      x & & otherwise \\
    \end{array}
  \right. \nonumber
\end{eqnarray}

and $z$ is chosen distinct from $\quotep{P}$, $\quotep{Q}$, the free
names in $Q$, and all the names in $R$. Our $\alpha$-equivalence will
be built in the standard way from this substitution.

\begin{remark}\label{rem:no_self_referential_names}
  One consequence of these definitions is that $\forall P. \quotep{P}
  \not\in \freenames{P}$.
\end{remark}

\subsection{ Dynamic quote: an example }

Anticipating something of what's to come, consider applying the
substitution, $\widehat{\id{\{}u / z \id{\}}}$, to the following pair
of processes, $\lift{w}{y!(z)}$ and $w[ \lpquote y!(z) \rpquote ]$.

\begin{eqnarray}
	\lift{w}{y!(z)}\widehat{\id{\{}u / z \id{\}}}
		& = &
		\lift{w}{y!(u)} \nonumber\\
	w[ \lpquote y!(z) \rpquote ] \widehat{ \id{\{}u / z \id{\}} }
		& = &
		w[ \lpquote y!(z) \rpquote ] \nonumber
\end{eqnarray}

Because the body of the process between quotes is impervious to
substitution, we get radically different answers. In fact, by
examining the first process in an input context,
e.g. $x?(z).\lift{w}{y!(z)}$, we see that the process under the lift
operator may be shaped by prefixed inputs binding a name inside it. In
this sense, the lift operator will be seen as a way to dynamically
construct processes before reifying them as names.

Finally equipped with these standard features we can present the
dynamics of the calculus.

\subsubsection{Operational semantics} 

Finally, we introduce the computational dynamics. What marks these
algebras as distinct from other more traditionally studied algebraic
structures, e.g. vector spaces or polynomial rings, is the manner in
which dynamics is captured. In traditional structures, dynamics is typically
expressed through morphisms between such structures, as in linear maps
between vector spaces or morphisms between rings. In algebras
associated with the semantics of computation, the dynamics is
expressed as part of the algebraic structure itself, through a
reduction reduction relation typically denoted by $\red$. Below, we
give a recursive presentation of this relation for the calculus used
in the encoding.

$\red \subseteq \pi \times \pi$
$\red : \pi \to \mathcal{P}(\pi)$

\begin{mathpar}
  \inferrule* [lab=Comm] { \textsf{match}( x_{src}, x_{trgt} ) } { x_{trgt}?(y)P \; | \; x_{src}!\langle {Q} \rangle \red P\{\quotep{Q}/y}\} }
  \and \\
  \inferrule* [lab=Par] {{P} \red {P}'} {{{P} | {Q}} \red {{P}' | {Q}}}
  \and
  \inferrule* [lab=Equiv]{{{P} \scong {P}'} \andalso {{P}' \red {Q}'} \andalso {{Q}' \scong {Q}}}{{P} \red {Q}}
\end{mathpar}

\begin{eqnarray*}
  match_{\equiv} (\quotep{P},\quotep{Q}) & := & P \equiv Q \\
  match_{\dagger}(\quotep{P},\quotep{Q}) & := & \forall R. P|Q \red^{*} R => R \red^{*} 0 \\
  match_{K}(\quotep{P},\quotep{Q}) & := & K \mbox{ for some context } K
\end{eqnarray*}

$u?(x)P | u!\langle Q \rangle \red P\{\quotep{Q}/x\}$

%We write $\wred$ for $\red^*$, and $P\red$ if $\exists Q $ such that $ P \red Q$.
We write $P\red$ if $\exists Q $ such that $ P \red Q$ and $P\not\red$, otherwise.

\section{Replication}

As mentioned before, it is known that replication (and hence
recursion) can be implemented in a higher-order process algebra
\cite{SangiorgiWalker}. As our first example of calculation with the
machinery thus far presented we give the construction explicitly in
the {\rhoc}.

\begin{eqnarray}
	D_{x} & := & \prefix{x}{y}{(\binpar{\outputp{x}{y}}{@{y}})} \nonumber\\
	\bangp_{x}{P} & := & \binpar{{x}!\langle{\binpar{D_{x}}{P}}\rangle}{D_{x}} \nonumber
\end{eqnarray}

\begin{eqnarray}
	\bangp_{x}{P} & & \nonumber\\
	=
	& {x}!\langle{(\prefix{x}{y}{(\outputp{x}{y} | @{y})) | P}}\rangle 
	      | \prefix{x}{y}{(\outputp{x}{y} | @{y})} & \nonumber\\
	\red
	& (\outputp{x}{y} | @{y})\substn{\quotep{(\prefix{x}{y}{(@{y} | \outputp{x}{y})) | P}}}{y} & \nonumber\\
	=
	& \outputp{x}{\quotep{(\prefix{x}{y}{(\outputp{x}{y} | @{y})) | P}}}
	  | {(\prefix{x}{y}{(\outputp{x}{y} | @{y})) | P}} & \nonumber\\
	\red
	& \ldots & \nonumber\\
	\red^*
	& P | P | \ldots & \nonumber
\end{eqnarray}

Of course, this encoding, as an implementation, runs away, unfolding
$\bangp{P}$ eagerly. A lazier and more implementable replication
operator, restricted to input-guarded processes, may be obtained as follows.

\begin{eqnarray}
\bangp{\prefix{u}{v}{P}} 
	:= 
	\binpar{\lift{x}{\prefix{u}{v}{(\binpar{D(x)}{P})}}}{D(x)} \nonumber
\end{eqnarray}

\begin{remark}
  Note that the lazier definition still does not deal with summation
  or mixed summation (i.e. sums over input and output). The reader is
  invited to construct definitions of replication that deal with these
  features. 

  Further, the definitions are parameterized in a name, $x$. Can you,
  gentle reader, make a definition that eliminates this parameter and
  guarantees no accidental interaction between the replication
  machinery and the process being replicated -- i.e. no accidental
  sharing of names used by the process to get its work done and the
  name(s) used by the replication to effect copying. This latter
  revision of the definition of replication is crucial to obtaining
  the expected identity $!!P \sim !P$.
\end{remark}

\begin{remark}\label{rem:paradoxical_combinator}
  The reader familiar with the lambda calculus will have noticed the
  similarity between $D$ and the paradoxical combinator.

  [Ed. note: the existence of this seems to suggest we have to be more
  restrictive on the set of processes and names we admit if we are to
  support no-cloning.]
\end{remark}

\subsubsection{Bisimulation}

The computational dynamics gives rise to another kind of equivalence,
the equivalence of computational behavior. As previously mentioned
this is typically captured \emph{via} some form of bisimulation.

% The notion we use in this paper is weak barbed bisimulation
% \cite{milner91polyadicpi}.

The notion we use in this paper is derived from weak barbed
bisimulation \cite{milner91polyadicpi}. 

\begin{definition}
An \emph{observation relation}, $\downarrow_{\mathcal N}$, over a set
of names, $\mathcal N$, is the smallest relation satisfying the rules
below.

\infrule[Out-barb]{y \in {\mathcal N}, \; x \nameeq y}
		  {\outputp{x}{v} \downarrow_{\mathcal N} x}
\infrule[Par-barb]{\mbox{$P\downarrow_{\mathcal N} x$ or $Q\downarrow_{\mathcal N} x$}}
		  {\binpar{P}{Q} \downarrow_{\mathcal N} x}

We write $P \Downarrow_{\mathcal N} x$ if there is $Q$ such that 
$P \wred Q$ and $Q \downarrow_{\mathcal N} x$.
\end{definition}

\begin{definition}
%\label{def.bbisim}
An  ${\mathcal N}$-\emph{barbed bisimulation} over a set of names, ${\mathcal N}$, is a symmetric binary relation 
${\mathcal S}_{\mathcal N}$ between agents such that $P\rel{S}_{\mathcal N}Q$ implies:
\begin{enumerate}
\item If $P \red P'$ then $Q \wred Q'$ and $P'\rel{S}_{\mathcal N} Q'$.
\item If $P\downarrow_{\mathcal N} x$, then $Q\Downarrow_{\mathcal N} x$.
\end{enumerate}
$P$ is ${\mathcal N}$-barbed bisimilar to $Q$, written
$P \wbbisim_{\mathcal N} Q$, if $P \rel{S}_{\mathcal N} Q$ for some ${\mathcal N}$-barbed bisimulation ${\mathcal S}_{\mathcal N}$.
\end{definition}

$\mathcal{R} \subseteq \pi \times \pi$

$P \mathcal{R} Q => \forall P'. P \red P' \Rightarrow \exists Q'. Q \red Q', P' \mathcal{R} Q'$

$P \vdash x \Rightarrow Q \vdash x$

\begin{mathpar}
  \inferrule*[lab=Out-barb]{x \nameeq y}{{y}!\langle{Q}\rangle \vdash x}
  \and
  \inferrule*[lab=Par-barb]{\mbox{$P\vdash x$ or $Q\vdash x$}}{\binpar{P}{Q} \vdash x}
\end{mathpar}

\subsubsection{Contexts}

One of the principle advantages of computational calculi like the
$\pi$-calculus is a well-defined notion of context,
contextual-equivalence and a correlation between
contextual-equivalence and notions of bisimulation. The notion of
context allows the decomposition of a process into (sub-)process and
its syntactic environment, its context. Thus, a context may be
thought of as a process with a ``hole'' (written $\Box$) in it. The
application of a context $M$ to a process $P$, written $M[P]$, is
tantamount to filling the hole in $M$ with $P$. In this paper we do
not need the full weight of this theory, but do make use of the notion
of context in the proof the main theorem. 

\begin{mathpar}
  \inferrule* [lab=summation] {} {{M_{M},M_{N}} \bc \Box \;|\; x.M_{A} \;|\; M_{M}+M_{N}}
  \and
  \inferrule* [lab=agent] {} {{M_{A}} \bc (\vec{x})M_{P} \;| \; \clift{P_0,\ldots,M_{P},\ldots,P_N}}
  \and \\
  \inferrule* [lab=process] {} {{M_{P}} \bc M_{N} \;| \;P|M_{P} }
\end{mathpar} 

\begin{mathpar}
  \inferrule* [lab=sychronization] {} {M_{N} \bc \Box \;|\; x?M_{F} \;|\; x!M_{C}}
  \and
  \inferrule* [lab=abstraction] {} {{M_{F}} \bc (x)M_{P} }
  \and
  \inferrule* [lab=concretion] {} {{M_{C}} \bc \langle M_{P} \rangle }
  \and \\
  \inferrule* [lab=process] {} {{M_{P}} \bc M_{N} \;| \;P|M_{P} }
\end{mathpar}

\begin{definition}[contextual application] Given a context $M$, and
  process $P$, we define the \emph{contextual application}, $M[P] :=
  M\{P/\Box\}$. That is, the contextual application of M to P is the
  substitution of $P$ for $\Box$ in $M$.
\end{definition}

$\meaningof{-} : L \to \mathcal{P}(\pi)$

\begin{mathpar}
  \inferrule* [lab=collection] {} {\meaningof{true} = \pi, \and \meaningof{~E} = \pi \setminus \meaningof{E}, \and \meaningof{E_{1} \& E_{2}} = \meaningof{E_{1}} \cap \meaningof{E_{2}}}
\end{mathpar}

\begin{mathpar}
  \inferrule* [lab=structure] {} {\meaningof{0} = \{ P \in \pi | P \equiv 0 \}, \and \\ \meaningof{E_1 | E_2} = \{ P \in \pi | P \equiv P_{1} | P_{2}, P_{1} \in \meaningof{E_{1}}, P_{2} \in \meaningof{E_2}\} }
\end{mathpar}

\begin{mathpar}
 \inferrule* [lab=behavior] {} {\meaningof{\langle a?b \rangle E} = \{ P \in \pi | P \equiv Q | u?(y)P', \\ \and \\\\ \and \\ \;\;\; u \in \meaningof{a}, \forall z.P'\{z/y\} \in \meaningof{E\{z/b\}}\}, \and \\ \meaningof{a!E} = \{ P \in \pi | P \equiv Q | x!\langle P' \rangle, x \in \meaningof{a} P' \in \meaningof{E}\} }
\end{mathpar}

\begin{mathpar}
 \inferrule* [lab=nominal] {} {\meaningof{\quotep{E}} = \{ \quotep{P} \in \quotep{\pi} | P \in \meaningof{E} \}, \and \meaningof{\quotep{P}} = \{ \quotep{Q} \in \quotep{\pi} | P \equiv Q \} \and \\ \meaningof{@\quotep{E}} = \{ P \in \pi | P \equiv @x, x \in \meaningof{E} \}}
\end{mathpar}

\begin{eqnarray*}
  \\
  \meaningof{-} : TS \to ST
\end{eqnarray*}

\begin{eqnarray*}
  \\
  L : TS \to ST
\end{eqnarray*}

\begin{eqnarray*}
  \\
  P \models E \iff P \in \meaningof{E}
\end{eqnarray*}

\begin{eqnarray*}
  P \approx_{L} Q \iff \forall E \in L. P \models E \iff Q \models E
\end{eqnarray*}

\begin{eqnarray*}
  P \approx_{K} Q
\end{eqnarray*}

\begin{eqnarray*}
  P \approx Q
\end{eqnarray*}

$\approx_{K} = \approx = \approx_{L}$

\subsubsection{Contextual duality}

Note that contexts extend the quotation operation to a family of
operations from processes to names. Given a context, $M$, we can
define a \emph{nominal context}, $\quotep{M}$ by $\quotep{M}[P] :=
\quotep{M[P]}$. To foreshadow what is to come we observe that these
operations enjoy a duality with processes very much like the duality
between vectors and maps from vectors to scalars.

Further, because the calculus is essentially higher-order, we have a
correspondence between contexts and processes. More specifically,
given a name $x$ and a context $M$ we can construct $M^{*}_{x}$ such
that 

\begin{mathpar}
  M^{*}_{x} | \lift{x}{P} \red M[P]
\end{mathpar}

namely,

\begin{mathpar}
  M^{*}_{x} := x?(u).M[\dropn{u}]
\end{mathpar}

The dependence of $M^{*}_{x}$ on a name makes it an abstraction, 

\begin{mathpar}
  M^{*} := (x)x?(u).M[\dropn{u}]
\end{mathpar}

\subsection{Additional notation}

It will sometimes be convenient to denote the process a name
quotes. We already have the notation $x = \quotep{P}$, but it will be
convenient to introduce an alternate notation, $\procn{x}$, when we
want to emphasize the connection to the use of the name. Note that, by
virtue of name equivalence, $\quotep{\procn{x}} \nameeq x$; so, the
notation is consistent with previous definitions.

Further, because names have structure it is possible to effect
substitutions on the basis of that structure. This means we need to
upgrade our notation for substitutions, which we accomplish by
adapting comprehension notation. Thus,

\begin{mathpar}
  P\{ y / x : x \in S \}
\end{mathpar}

is interpreted to mean the process derived from P by replacing (in a
capture-avoiding manner) each occurrence of $x$ in $S$ by $y$. For example,

\begin{mathpar}
  P\{ \quotep{\procn{x}|\procn{x}} / x : x \in \freenames{P} \}
\end{mathpar}

will replace each (occurrence) of a free name $x$ in $P$ by
$\quotep{\procn{x}|\procn{x}}$.

Also, we will avail ourselves of the notation $x^{L}$ and $x^{R}$ to
denote injections of a name into disjoint copies of the name
space. There are numerous ways to accomplish this. One example can be
found in \cite{MeredithR05}. This notation overloads to vectors of
names: $\vec{x}^{\pi} := (x_{i}^{\pi} \; : \; 0 \leq i < |\vec{x}| )$ where $\pi \in \{L,R\}$.

We also use $P^{\Box} := P|\Box$.

In \cite{MeredithR05} an interpretation of the new operator is
given. It turns out that there are several possible interpretations
all enjoying the requisite algebraic properties of the operator (see
\cite{milner91polyadicpi}). We will therefore make liberal use of
$(\nu\; \vec{x})P$.

% subsection the_syntax_and_semantics_of_the_notation_system (end)   

\input{qm2pi.qmops} 

\input{qm2pi.sterngerlach} 

\input{qm2pi.metric} 

% section concurrent_process_calculi (end)

%\input{qm2pi.proofsketch}

% section proof sketch (end)

%\input{qm2pi.slviaknots} 

% section spatial logic via knots (end)

\input{qm2pi.conclusion}

% section conclusion (end)

%\input{qm2pi.dtcodes} 

% section wiring algorithm (end)

\input{qm2pi.ack} 

% section acknowledgments (end)

\newpage


\bibliographystyle{plain}   
\bibliography{../../biblios/main.bib}

\input{qm2pi.rhodetails}

\end{document}

 

%\documentclass[12pt]{llncs}
%\documentclass{jktr}

\usepackage[pdftex]{hyperref}                   
\usepackage {listings}
\usepackage {mathpartir}
\usepackage{bcprules}
%\usepackage{listings}
                       
\usepackage{graphicx} 
%\usepackage[margins=2.5cm,nohead,nofoot]{geometry}
%\usepackage{geometry}
\usepackage{amsfonts}
\usepackage{amstext}
\usepackage{latexsym}
\usepackage{amssymb}
\usepackage{color}


%\include{myPreamble}
\include{qm2pi.local} 

%\ifpdf
%\usepackage[pdftex]{graphicx}
%\else
%\usepackage{graphicx}
%\fi

 % \ifpdf
%  \usepackage{pdfsync}
%  \if


%\title{Brief Article}
%\author{David F. Snyder}
%\author{L.G. Meredith}

%\address{Dept. of Math., Texas State University--San Marcos, San Marcos, TX 78666}
       
\pagestyle{empty}


\begin{document}

\lstset{language=[Objective]Caml,frame=shadowbox}

\input{qm2pi.front}

% section front matter (end)

\input{qm2pi.intro} 
 
% section introduction (end)

% \input{qm2pi.knotations} 

% section notation (end)

\input{qm2pi.process.calculi} 

% section concurrent_process_calculi_and_spatial_logics_ (end)
    
%\input{qm2pi.knots2pi} 

%\input{qm2pi.trefoil} 

%\input{qm2pi.mainthm} 

% subsection basic_interpretation (end)

%\input{qm2pi.rho.presentation} 
\subsection{The syntax and semantics of the notation system}\label{sub:the_syntax_and_semantics_of_the_notation_system} % (fold)

We now summarize a technical presentation of the calculus that
embodies our theory of dynamics. The typical presentation of such a
calculus follows the style of giving generators and relations on
them. The grammar, below, describing term constructors, freely
generates the set of processes, $\Proc$. This set is then quotiented
by a relation known as structural congruence and it is over this set
that the notion of dynamics is expressed. This presentation is
essentially that of \cite{MeredithR05} with the addition of
polyadicity and summation. For readability we have relegated some of
the technical subtleties to an appendix.

\subsubsection{Process grammar}\label{subsub:process_grammar}

\begin{mathpar}
  \inferrule* [lab=synchronization] {} {{M} \bc \pzero \;|\; x?F \;|\; x!C }
  \and
  \inferrule* [lab=abstraction] {} {{F} \bc (x)P}
  \and
  \inferrule* [lab=concretion] {} {{C} \bc \langle Q \rangle}
  \and
  \inferrule* [lab=process] {} {{P,Q} \bc M \;| \;P|Q \;|\; @{x}}
  \and
  \inferrule* [lab=name] {} {{x} \bc \quotep{P}}
\end{mathpar} 

Note that $\vec{x}$ (resp. $\vec{P}$) denotes a vector of names
(resp. processes) of length $|\vec{x}|$ (resp. $|\vec{P}|$). We adopt
the following useful abbreviations.

\begin{mathpar}
   x?(\vec{y}).P := x.(\vec{y})P \and  x\clift{\vec{P}} := x.\clift{\vec{P}}
   \and x!(y) := \lift{x}{\dropn{y}}
   \and \Pi_{i=0}^{n-1}P_i := P_0 | \ldots | P_{n-1}
\end{mathpar}

\subsubsection{Structural congruence}

\paragraph{Free and bound names and alpha-equivalence.} At the
core of structural equivalence is alpha-equivalence which identifies
process that are the same up to a change of variable. Formally, we
recognize the distinction between free and bound names. The free names
of a process, $\freenames{P}$, may be calculated recursively as
follows:

\begin{mathpar}
\freenames{\pzero} := \emptyset
  \and \\
  \freenames{x?(y).P} := \{ x \} \cup (\freenames{P} \setminus \{ y \})
  \and 
  \freenames{x!\langle P \rangle} := \{ x \} \cup \{ P \} 
  \and \\
  \freenames{P|Q} := \freenames{P} \cup \freenames{Q}
  \and \\
  \freenames{@{x}} := \{ x \}
\end{mathpar}

$\pi$
$\quotep{\pi}$

$\freenames{-} : \pi \to \mathcal{P}(\quotep{\pi})$

\begin{eqnarray*}
  \freenames{\pzero} & := & \emptyset \\
  \freenames{x?(y).P} & := & \{ x \} \cup (\freenames{P} \setminus \{ y \}) \\
  \freenames{x!\langle P \rangle} & := & \{ x \} \cup \{ P \} \\
  \freenames{P|Q} & := & \freenames{P} \cup \freenames{Q} \\
  \freenames{\dropn{x}} & := & \{ x \}
\end{eqnarray*}

The bound names of a process, $\boundnames{P}$, are those names occurring in $P$
that are not free. For example, in $x?(y).0$, the name $x$ is free, while $y$ is bound.

\begin{mathpar}
  \inferrule* [lab=monoidal-laws] {} { P|Q \equiv Q|P \and P|0 \equiv P \and P|(Q|R) \equiv (P|Q)|R }
\end{mathpar}

\begin{mathpar}
  \inferrule* [lab=alpha-equivalence] {} { (x)P \equiv (y)P\{y/x\} \and y \not\in \freenames{P} }
\end{mathpar}

\begin{definition}
Then two processes, $P,Q$, are alpha-equivalent if $P = Q\{\vec{y}/\vec{x}\}$ for
some $\vec{x} \in \boundnames{Q},\vec{y} \in \boundnames{P}$, where $Q\{\vec{y}/\vec{x}\}$
denotes the capture-avoiding substitution of $\vec{y}$ for $\vec{x}$ in $Q$.
\end{definition}

\begin{definition}
  The {\em structural congruence} \cite{SangiorgiWalker} , $\equiv$,
  between processes is the least congruence containing
  alpha-equivalence, satisfying the abelian monoid laws
  (associativity, commutativity and $\pzero$ as identity) for parallel
  composition $|$ and for summation $+$.
\end{definition}

\subsection{Name equivalence}

We take name equivalence, written $\nameeq$, to be the smallest
equivalence relation generated by the following rules.

\begin{mathpar}
\inferrule*[lab=Quote-drop]
{ }
{ \quotep{@{x}} \nameeq x }

\inferrule*[lab=Struct-equiv]
{ P \scong Q }
{ \quotep{P} \nameeq \quotep{Q} }
\end{mathpar}

The astute reader will have noticed that the mutual recursion of names
and processes imposes a mutual recursion on alpha-equivalence and
structural equivalence via name-equivalence. Fortunately, all of this
works out pleasantly and we may calculate in the natural way, free of
concern. The reader interested in the details is referred to the
appendix \ref{appendix:rho_details}.

\subsection{Substitution}

We use $\Proc$ for the set of processes, $\QProc$ for the set of
names, and $\id{\{}\vec{y} / \vec{x} \id{\}}$ to denote partial maps,
$s : \QProc \rightarrow \QProc$. A map, $s$ lifts, uniquely, to a map
on process terms, $\widehat{s} : \Proc \rightarrow \Proc$ by the
following equations.

\begin{mathpar}
  (0) \psubstp{Q}{P} := 0 \\
  (R \juxtap S) \psubstp{Q}{P}
  :=    
  (R)\psubstp{Q}{P} \juxtap (S) \psubstp{Q}{P} \\
  (x?(y).R) \psubstp{Q}{P}    
  :=    
  (x)\substp{Q}{P} (z)\concat( (R \psubstn{z}{y}) \psubstp{Q}{P} ) \\
  (\lift{x}{R}) \psubstp{Q}{P}  
  :=
  \lift{(x)\substp{Q}{P}}{ R \psubstp{Q}{P} } \\
%   (\dropn{x})  \psubstp{Q}{P}       
%   := 
%   \left\{ 
%     \begin{array}{ccc} 
%       \dropn{\quotep{Q}} & & x \nameeq \quotep{P} \\
%       \dropn{x} & & otherwise \\
%     \end{array}
%   \right. 
  (\dropn{x})  \psubstp{Q}{P}       
  := 
  \left\{ 
    \begin{array}{ccc} 
      Q & & x \nameeq \quotep{P} \\
      \dropn{x} & & otherwise \\
    \end{array}
  \right.
\end{mathpar}
 

where

\begin{eqnarray}
  (x)\id{\{} \lpquote Q \rpquote / \lpquote P \rpquote \id{\}}            = 
  \left\{ 
    \begin{array}{ccc}
      \lpquote Q \rpquote & & x \nameeq \lpquote P \rpquote \\
      x & & otherwise \\
    \end{array}
  \right. \nonumber
\end{eqnarray}

and $z$ is chosen distinct from $\quotep{P}$, $\quotep{Q}$, the free
names in $Q$, and all the names in $R$. Our $\alpha$-equivalence will
be built in the standard way from this substitution.

\begin{remark}\label{rem:no_self_referential_names}
  One consequence of these definitions is that $\forall P. \quotep{P}
  \not\in \freenames{P}$.
\end{remark}

\subsection{ Dynamic quote: an example }

Anticipating something of what's to come, consider applying the
substitution, $\widehat{\id{\{}u / z \id{\}}}$, to the following pair
of processes, $\lift{w}{y!(z)}$ and $w[ \lpquote y!(z) \rpquote ]$.

\begin{eqnarray}
	\lift{w}{y!(z)}\widehat{\id{\{}u / z \id{\}}}
		& = &
		\lift{w}{y!(u)} \nonumber\\
	w[ \lpquote y!(z) \rpquote ] \widehat{ \id{\{}u / z \id{\}} }
		& = &
		w[ \lpquote y!(z) \rpquote ] \nonumber
\end{eqnarray}

Because the body of the process between quotes is impervious to
substitution, we get radically different answers. In fact, by
examining the first process in an input context,
e.g. $x?(z).\lift{w}{y!(z)}$, we see that the process under the lift
operator may be shaped by prefixed inputs binding a name inside it. In
this sense, the lift operator will be seen as a way to dynamically
construct processes before reifying them as names.

Finally equipped with these standard features we can present the
dynamics of the calculus.

\subsubsection{Operational semantics} 

Finally, we introduce the computational dynamics. What marks these
algebras as distinct from other more traditionally studied algebraic
structures, e.g. vector spaces or polynomial rings, is the manner in
which dynamics is captured. In traditional structures, dynamics is typically
expressed through morphisms between such structures, as in linear maps
between vector spaces or morphisms between rings. In algebras
associated with the semantics of computation, the dynamics is
expressed as part of the algebraic structure itself, through a
reduction reduction relation typically denoted by $\red$. Below, we
give a recursive presentation of this relation for the calculus used
in the encoding.

$\red \subseteq \pi \times \pi$
$\red : \pi \to \mathcal{P}(\pi)$

\begin{mathpar}
  \inferrule* [lab=Comm] { \textsf{match}( x_{src}, x_{trgt} ) } { x_{trgt}?(y)P \; | \; x_{src}!\langle {Q} \rangle \red P\{\quotep{Q}/y}\} }
  \and \\
  \inferrule* [lab=Par] {{P} \red {P}'} {{{P} | {Q}} \red {{P}' | {Q}}}
  \and
  \inferrule* [lab=Equiv]{{{P} \scong {P}'} \andalso {{P}' \red {Q}'} \andalso {{Q}' \scong {Q}}}{{P} \red {Q}}
\end{mathpar}

\begin{eqnarray*}
  match_{\equiv} (\quotep{P},\quotep{Q}) & := & P \equiv Q \\
  match_{\dagger}(\quotep{P},\quotep{Q}) & := & \forall R. P|Q \red^{*} R => R \red^{*} 0 \\
  match_{K}(\quotep{P},\quotep{Q}) & := & K \mbox{ for some context } K
\end{eqnarray*}

$u?(x)P | u!\langle Q \rangle \red P\{\quotep{Q}/x\}$

%We write $\wred$ for $\red^*$, and $P\red$ if $\exists Q $ such that $ P \red Q$.
We write $P\red$ if $\exists Q $ such that $ P \red Q$ and $P\not\red$, otherwise.

\section{Replication}

As mentioned before, it is known that replication (and hence
recursion) can be implemented in a higher-order process algebra
\cite{SangiorgiWalker}. As our first example of calculation with the
machinery thus far presented we give the construction explicitly in
the {\rhoc}.

\begin{eqnarray}
	D_{x} & := & \prefix{x}{y}{(\binpar{\outputp{x}{y}}{@{y}})} \nonumber\\
	\bangp_{x}{P} & := & \binpar{{x}!\langle{\binpar{D_{x}}{P}}\rangle}{D_{x}} \nonumber
\end{eqnarray}

\begin{eqnarray}
	\bangp_{x}{P} & & \nonumber\\
	=
	& {x}!\langle{(\prefix{x}{y}{(\outputp{x}{y} | @{y})) | P}}\rangle 
	      | \prefix{x}{y}{(\outputp{x}{y} | @{y})} & \nonumber\\
	\red
	& (\outputp{x}{y} | @{y})\substn{\quotep{(\prefix{x}{y}{(@{y} | \outputp{x}{y})) | P}}}{y} & \nonumber\\
	=
	& \outputp{x}{\quotep{(\prefix{x}{y}{(\outputp{x}{y} | @{y})) | P}}}
	  | {(\prefix{x}{y}{(\outputp{x}{y} | @{y})) | P}} & \nonumber\\
	\red
	& \ldots & \nonumber\\
	\red^*
	& P | P | \ldots & \nonumber
\end{eqnarray}

Of course, this encoding, as an implementation, runs away, unfolding
$\bangp{P}$ eagerly. A lazier and more implementable replication
operator, restricted to input-guarded processes, may be obtained as follows.

\begin{eqnarray}
\bangp{\prefix{u}{v}{P}} 
	:= 
	\binpar{\lift{x}{\prefix{u}{v}{(\binpar{D(x)}{P})}}}{D(x)} \nonumber
\end{eqnarray}

\begin{remark}
  Note that the lazier definition still does not deal with summation
  or mixed summation (i.e. sums over input and output). The reader is
  invited to construct definitions of replication that deal with these
  features. 

  Further, the definitions are parameterized in a name, $x$. Can you,
  gentle reader, make a definition that eliminates this parameter and
  guarantees no accidental interaction between the replication
  machinery and the process being replicated -- i.e. no accidental
  sharing of names used by the process to get its work done and the
  name(s) used by the replication to effect copying. This latter
  revision of the definition of replication is crucial to obtaining
  the expected identity $!!P \sim !P$.
\end{remark}

\begin{remark}\label{rem:paradoxical_combinator}
  The reader familiar with the lambda calculus will have noticed the
  similarity between $D$ and the paradoxical combinator.

  [Ed. note: the existence of this seems to suggest we have to be more
  restrictive on the set of processes and names we admit if we are to
  support no-cloning.]
\end{remark}

\subsubsection{Bisimulation}

The computational dynamics gives rise to another kind of equivalence,
the equivalence of computational behavior. As previously mentioned
this is typically captured \emph{via} some form of bisimulation.

% The notion we use in this paper is weak barbed bisimulation
% \cite{milner91polyadicpi}.

The notion we use in this paper is derived from weak barbed
bisimulation \cite{milner91polyadicpi}. 

\begin{definition}
An \emph{observation relation}, $\downarrow_{\mathcal N}$, over a set
of names, $\mathcal N$, is the smallest relation satisfying the rules
below.

\infrule[Out-barb]{y \in {\mathcal N}, \; x \nameeq y}
		  {\outputp{x}{v} \downarrow_{\mathcal N} x}
\infrule[Par-barb]{\mbox{$P\downarrow_{\mathcal N} x$ or $Q\downarrow_{\mathcal N} x$}}
		  {\binpar{P}{Q} \downarrow_{\mathcal N} x}

We write $P \Downarrow_{\mathcal N} x$ if there is $Q$ such that 
$P \wred Q$ and $Q \downarrow_{\mathcal N} x$.
\end{definition}

\begin{definition}
%\label{def.bbisim}
An  ${\mathcal N}$-\emph{barbed bisimulation} over a set of names, ${\mathcal N}$, is a symmetric binary relation 
${\mathcal S}_{\mathcal N}$ between agents such that $P\rel{S}_{\mathcal N}Q$ implies:
\begin{enumerate}
\item If $P \red P'$ then $Q \wred Q'$ and $P'\rel{S}_{\mathcal N} Q'$.
\item If $P\downarrow_{\mathcal N} x$, then $Q\Downarrow_{\mathcal N} x$.
\end{enumerate}
$P$ is ${\mathcal N}$-barbed bisimilar to $Q$, written
$P \wbbisim_{\mathcal N} Q$, if $P \rel{S}_{\mathcal N} Q$ for some ${\mathcal N}$-barbed bisimulation ${\mathcal S}_{\mathcal N}$.
\end{definition}

$\mathcal{R} \subseteq \pi \times \pi$

$P \mathcal{R} Q => \forall P'. P \red P' \Rightarrow \exists Q'. Q \red Q', P' \mathcal{R} Q'$

$P \vdash x \Rightarrow Q \vdash x$

\begin{mathpar}
  \inferrule*[lab=Out-barb]{x \nameeq y}{{y}!\langle{Q}\rangle \vdash x}
  \and
  \inferrule*[lab=Par-barb]{\mbox{$P\vdash x$ or $Q\vdash x$}}{\binpar{P}{Q} \vdash x}
\end{mathpar}

\subsubsection{Contexts}

One of the principle advantages of computational calculi like the
$\pi$-calculus is a well-defined notion of context,
contextual-equivalence and a correlation between
contextual-equivalence and notions of bisimulation. The notion of
context allows the decomposition of a process into (sub-)process and
its syntactic environment, its context. Thus, a context may be
thought of as a process with a ``hole'' (written $\Box$) in it. The
application of a context $M$ to a process $P$, written $M[P]$, is
tantamount to filling the hole in $M$ with $P$. In this paper we do
not need the full weight of this theory, but do make use of the notion
of context in the proof the main theorem. 

\begin{mathpar}
  \inferrule* [lab=summation] {} {{M_{M},M_{N}} \bc \Box \;|\; x.M_{A} \;|\; M_{M}+M_{N}}
  \and
  \inferrule* [lab=agent] {} {{M_{A}} \bc (\vec{x})M_{P} \;| \; \clift{P_0,\ldots,M_{P},\ldots,P_N}}
  \and \\
  \inferrule* [lab=process] {} {{M_{P}} \bc M_{N} \;| \;P|M_{P} }
\end{mathpar} 

\begin{mathpar}
  \inferrule* [lab=sychronization] {} {M_{N} \bc \Box \;|\; x?M_{F} \;|\; x!M_{C}}
  \and
  \inferrule* [lab=abstraction] {} {{M_{F}} \bc (x)M_{P} }
  \and
  \inferrule* [lab=concretion] {} {{M_{C}} \bc \langle M_{P} \rangle }
  \and \\
  \inferrule* [lab=process] {} {{M_{P}} \bc M_{N} \;| \;P|M_{P} }
\end{mathpar}

\begin{definition}[contextual application] Given a context $M$, and
  process $P$, we define the \emph{contextual application}, $M[P] :=
  M\{P/\Box\}$. That is, the contextual application of M to P is the
  substitution of $P$ for $\Box$ in $M$.
\end{definition}

$\meaningof{-} : L \to \mathcal{P}(\pi)$

\begin{mathpar}
  \inferrule* [lab=collection] {} {\meaningof{true} = \pi, \and \meaningof{~E} = \pi \setminus \meaningof{E}, \and \meaningof{E_{1} \& E_{2}} = \meaningof{E_{1}} \cap \meaningof{E_{2}}}
\end{mathpar}

\begin{mathpar}
  \inferrule* [lab=structure] {} {\meaningof{0} = \{ P \in \pi | P \equiv 0 \}, \and \\ \meaningof{E_1 | E_2} = \{ P \in \pi | P \equiv P_{1} | P_{2}, P_{1} \in \meaningof{E_{1}}, P_{2} \in \meaningof{E_2}\} }
\end{mathpar}

\begin{mathpar}
 \inferrule* [lab=behavior] {} {\meaningof{\langle a?b \rangle E} = \{ P \in \pi | P \equiv Q | u?(y)P', \\ \and \\\\ \and \\ \;\;\; u \in \meaningof{a}, \forall z.P'\{z/y\} \in \meaningof{E\{z/b\}}\}, \and \\ \meaningof{a!E} = \{ P \in \pi | P \equiv Q | x!\langle P' \rangle, x \in \meaningof{a} P' \in \meaningof{E}\} }
\end{mathpar}

\begin{mathpar}
 \inferrule* [lab=nominal] {} {\meaningof{\quotep{E}} = \{ \quotep{P} \in \quotep{\pi} | P \in \meaningof{E} \}, \and \meaningof{\quotep{P}} = \{ \quotep{Q} \in \quotep{\pi} | P \equiv Q \} \and \\ \meaningof{@\quotep{E}} = \{ P \in \pi | P \equiv @x, x \in \meaningof{E} \}}
\end{mathpar}

\begin{eqnarray*}
  \\
  \meaningof{-} : TS \to ST
\end{eqnarray*}

\begin{eqnarray*}
  \\
  L : TS \to ST
\end{eqnarray*}

\begin{eqnarray*}
  \\
  P \models E \iff P \in \meaningof{E}
\end{eqnarray*}

\begin{eqnarray*}
  P \approx_{L} Q \iff \forall E \in L. P \models E \iff Q \models E
\end{eqnarray*}

\begin{eqnarray*}
  P \approx_{K} Q
\end{eqnarray*}

\begin{eqnarray*}
  P \approx Q
\end{eqnarray*}

$\approx_{K} = \approx = \approx_{L}$

\subsubsection{Contextual duality}

Note that contexts extend the quotation operation to a family of
operations from processes to names. Given a context, $M$, we can
define a \emph{nominal context}, $\quotep{M}$ by $\quotep{M}[P] :=
\quotep{M[P]}$. To foreshadow what is to come we observe that these
operations enjoy a duality with processes very much like the duality
between vectors and maps from vectors to scalars.

Further, because the calculus is essentially higher-order, we have a
correspondence between contexts and processes. More specifically,
given a name $x$ and a context $M$ we can construct $M^{*}_{x}$ such
that 

\begin{mathpar}
  M^{*}_{x} | \lift{x}{P} \red M[P]
\end{mathpar}

namely,

\begin{mathpar}
  M^{*}_{x} := x?(u).M[\dropn{u}]
\end{mathpar}

The dependence of $M^{*}_{x}$ on a name makes it an abstraction, 

\begin{mathpar}
  M^{*} := (x)x?(u).M[\dropn{u}]
\end{mathpar}

\subsection{Additional notation}

It will sometimes be convenient to denote the process a name
quotes. We already have the notation $x = \quotep{P}$, but it will be
convenient to introduce an alternate notation, $\procn{x}$, when we
want to emphasize the connection to the use of the name. Note that, by
virtue of name equivalence, $\quotep{\procn{x}} \nameeq x$; so, the
notation is consistent with previous definitions.

Further, because names have structure it is possible to effect
substitutions on the basis of that structure. This means we need to
upgrade our notation for substitutions, which we accomplish by
adapting comprehension notation. Thus,

\begin{mathpar}
  P\{ y / x : x \in S \}
\end{mathpar}

is interpreted to mean the process derived from P by replacing (in a
capture-avoiding manner) each occurrence of $x$ in $S$ by $y$. For example,

\begin{mathpar}
  P\{ \quotep{\procn{x}|\procn{x}} / x : x \in \freenames{P} \}
\end{mathpar}

will replace each (occurrence) of a free name $x$ in $P$ by
$\quotep{\procn{x}|\procn{x}}$.

Also, we will avail ourselves of the notation $x^{L}$ and $x^{R}$ to
denote injections of a name into disjoint copies of the name
space. There are numerous ways to accomplish this. One example can be
found in \cite{MeredithR05}. This notation overloads to vectors of
names: $\vec{x}^{\pi} := (x_{i}^{\pi} \; : \; 0 \leq i < |\vec{x}| )$ where $\pi \in \{L,R\}$.

We also use $P^{\Box} := P|\Box$.

In \cite{MeredithR05} an interpretation of the new operator is
given. It turns out that there are several possible interpretations
all enjoying the requisite algebraic properties of the operator (see
\cite{milner91polyadicpi}). We will therefore make liberal use of
$(\nu\; \vec{x})P$.

% subsection the_syntax_and_semantics_of_the_notation_system (end)   

\input{qm2pi.qmops} 

\input{qm2pi.sterngerlach} 

\input{qm2pi.metric} 

% section concurrent_process_calculi (end)

%\input{qm2pi.proofsketch}

% section proof sketch (end)

%\input{qm2pi.slviaknots} 

% section spatial logic via knots (end)

\input{qm2pi.conclusion}

% section conclusion (end)

%\input{qm2pi.dtcodes} 

% section wiring algorithm (end)

\input{qm2pi.ack} 

% section acknowledgments (end)

\newpage


\bibliographystyle{plain}   
\bibliography{../../biblios/main.bib}

\input{qm2pi.rhodetails}

\end{document}

 

%\documentclass[12pt]{llncs}
%\documentclass{jktr}

\usepackage[pdftex]{hyperref}                   
\usepackage {listings}
\usepackage {mathpartir}
\usepackage{bcprules}
%\usepackage{listings}
                       
\usepackage{graphicx} 
%\usepackage[margins=2.5cm,nohead,nofoot]{geometry}
%\usepackage{geometry}
\usepackage{amsfonts}
\usepackage{amstext}
\usepackage{latexsym}
\usepackage{amssymb}
\usepackage{color}


%\include{myPreamble}
\include{qm2pi.local} 

%\ifpdf
%\usepackage[pdftex]{graphicx}
%\else
%\usepackage{graphicx}
%\fi

 % \ifpdf
%  \usepackage{pdfsync}
%  \if


%\title{Brief Article}
%\author{David F. Snyder}
%\author{L.G. Meredith}

%\address{Dept. of Math., Texas State University--San Marcos, San Marcos, TX 78666}
       
\pagestyle{empty}


\begin{document}

\lstset{language=[Objective]Caml,frame=shadowbox}

\input{qm2pi.front}

% section front matter (end)

\input{qm2pi.intro} 
 
% section introduction (end)

% \input{qm2pi.knotations} 

% section notation (end)

\input{qm2pi.process.calculi} 

% section concurrent_process_calculi_and_spatial_logics_ (end)
    
%\input{qm2pi.knots2pi} 

%\input{qm2pi.trefoil} 

%\input{qm2pi.mainthm} 

% subsection basic_interpretation (end)

%\input{qm2pi.rho.presentation} 
\subsection{The syntax and semantics of the notation system}\label{sub:the_syntax_and_semantics_of_the_notation_system} % (fold)

We now summarize a technical presentation of the calculus that
embodies our theory of dynamics. The typical presentation of such a
calculus follows the style of giving generators and relations on
them. The grammar, below, describing term constructors, freely
generates the set of processes, $\Proc$. This set is then quotiented
by a relation known as structural congruence and it is over this set
that the notion of dynamics is expressed. This presentation is
essentially that of \cite{MeredithR05} with the addition of
polyadicity and summation. For readability we have relegated some of
the technical subtleties to an appendix.

\subsubsection{Process grammar}\label{subsub:process_grammar}

\begin{mathpar}
  \inferrule* [lab=synchronization] {} {{M} \bc \pzero \;|\; x?F \;|\; x!C }
  \and
  \inferrule* [lab=abstraction] {} {{F} \bc (x)P}
  \and
  \inferrule* [lab=concretion] {} {{C} \bc \langle Q \rangle}
  \and
  \inferrule* [lab=process] {} {{P,Q} \bc M \;| \;P|Q \;|\; @{x}}
  \and
  \inferrule* [lab=name] {} {{x} \bc \quotep{P}}
\end{mathpar} 

Note that $\vec{x}$ (resp. $\vec{P}$) denotes a vector of names
(resp. processes) of length $|\vec{x}|$ (resp. $|\vec{P}|$). We adopt
the following useful abbreviations.

\begin{mathpar}
   x?(\vec{y}).P := x.(\vec{y})P \and  x\clift{\vec{P}} := x.\clift{\vec{P}}
   \and x!(y) := \lift{x}{\dropn{y}}
   \and \Pi_{i=0}^{n-1}P_i := P_0 | \ldots | P_{n-1}
\end{mathpar}

\subsubsection{Structural congruence}

\paragraph{Free and bound names and alpha-equivalence.} At the
core of structural equivalence is alpha-equivalence which identifies
process that are the same up to a change of variable. Formally, we
recognize the distinction between free and bound names. The free names
of a process, $\freenames{P}$, may be calculated recursively as
follows:

\begin{mathpar}
\freenames{\pzero} := \emptyset
  \and \\
  \freenames{x?(y).P} := \{ x \} \cup (\freenames{P} \setminus \{ y \})
  \and 
  \freenames{x!\langle P \rangle} := \{ x \} \cup \{ P \} 
  \and \\
  \freenames{P|Q} := \freenames{P} \cup \freenames{Q}
  \and \\
  \freenames{@{x}} := \{ x \}
\end{mathpar}

$\pi$
$\quotep{\pi}$

$\freenames{-} : \pi \to \mathcal{P}(\quotep{\pi})$

\begin{eqnarray*}
  \freenames{\pzero} & := & \emptyset \\
  \freenames{x?(y).P} & := & \{ x \} \cup (\freenames{P} \setminus \{ y \}) \\
  \freenames{x!\langle P \rangle} & := & \{ x \} \cup \{ P \} \\
  \freenames{P|Q} & := & \freenames{P} \cup \freenames{Q} \\
  \freenames{\dropn{x}} & := & \{ x \}
\end{eqnarray*}

The bound names of a process, $\boundnames{P}$, are those names occurring in $P$
that are not free. For example, in $x?(y).0$, the name $x$ is free, while $y$ is bound.

\begin{mathpar}
  \inferrule* [lab=monoidal-laws] {} { P|Q \equiv Q|P \and P|0 \equiv P \and P|(Q|R) \equiv (P|Q)|R }
\end{mathpar}

\begin{mathpar}
  \inferrule* [lab=alpha-equivalence] {} { (x)P \equiv (y)P\{y/x\} \and y \not\in \freenames{P} }
\end{mathpar}

\begin{definition}
Then two processes, $P,Q$, are alpha-equivalent if $P = Q\{\vec{y}/\vec{x}\}$ for
some $\vec{x} \in \boundnames{Q},\vec{y} \in \boundnames{P}$, where $Q\{\vec{y}/\vec{x}\}$
denotes the capture-avoiding substitution of $\vec{y}$ for $\vec{x}$ in $Q$.
\end{definition}

\begin{definition}
  The {\em structural congruence} \cite{SangiorgiWalker} , $\equiv$,
  between processes is the least congruence containing
  alpha-equivalence, satisfying the abelian monoid laws
  (associativity, commutativity and $\pzero$ as identity) for parallel
  composition $|$ and for summation $+$.
\end{definition}

\subsection{Name equivalence}

We take name equivalence, written $\nameeq$, to be the smallest
equivalence relation generated by the following rules.

\begin{mathpar}
\inferrule*[lab=Quote-drop]
{ }
{ \quotep{@{x}} \nameeq x }

\inferrule*[lab=Struct-equiv]
{ P \scong Q }
{ \quotep{P} \nameeq \quotep{Q} }
\end{mathpar}

The astute reader will have noticed that the mutual recursion of names
and processes imposes a mutual recursion on alpha-equivalence and
structural equivalence via name-equivalence. Fortunately, all of this
works out pleasantly and we may calculate in the natural way, free of
concern. The reader interested in the details is referred to the
appendix \ref{appendix:rho_details}.

\subsection{Substitution}

We use $\Proc$ for the set of processes, $\QProc$ for the set of
names, and $\id{\{}\vec{y} / \vec{x} \id{\}}$ to denote partial maps,
$s : \QProc \rightarrow \QProc$. A map, $s$ lifts, uniquely, to a map
on process terms, $\widehat{s} : \Proc \rightarrow \Proc$ by the
following equations.

\begin{mathpar}
  (0) \psubstp{Q}{P} := 0 \\
  (R \juxtap S) \psubstp{Q}{P}
  :=    
  (R)\psubstp{Q}{P} \juxtap (S) \psubstp{Q}{P} \\
  (x?(y).R) \psubstp{Q}{P}    
  :=    
  (x)\substp{Q}{P} (z)\concat( (R \psubstn{z}{y}) \psubstp{Q}{P} ) \\
  (\lift{x}{R}) \psubstp{Q}{P}  
  :=
  \lift{(x)\substp{Q}{P}}{ R \psubstp{Q}{P} } \\
%   (\dropn{x})  \psubstp{Q}{P}       
%   := 
%   \left\{ 
%     \begin{array}{ccc} 
%       \dropn{\quotep{Q}} & & x \nameeq \quotep{P} \\
%       \dropn{x} & & otherwise \\
%     \end{array}
%   \right. 
  (\dropn{x})  \psubstp{Q}{P}       
  := 
  \left\{ 
    \begin{array}{ccc} 
      Q & & x \nameeq \quotep{P} \\
      \dropn{x} & & otherwise \\
    \end{array}
  \right.
\end{mathpar}
 

where

\begin{eqnarray}
  (x)\id{\{} \lpquote Q \rpquote / \lpquote P \rpquote \id{\}}            = 
  \left\{ 
    \begin{array}{ccc}
      \lpquote Q \rpquote & & x \nameeq \lpquote P \rpquote \\
      x & & otherwise \\
    \end{array}
  \right. \nonumber
\end{eqnarray}

and $z$ is chosen distinct from $\quotep{P}$, $\quotep{Q}$, the free
names in $Q$, and all the names in $R$. Our $\alpha$-equivalence will
be built in the standard way from this substitution.

\begin{remark}\label{rem:no_self_referential_names}
  One consequence of these definitions is that $\forall P. \quotep{P}
  \not\in \freenames{P}$.
\end{remark}

\subsection{ Dynamic quote: an example }

Anticipating something of what's to come, consider applying the
substitution, $\widehat{\id{\{}u / z \id{\}}}$, to the following pair
of processes, $\lift{w}{y!(z)}$ and $w[ \lpquote y!(z) \rpquote ]$.

\begin{eqnarray}
	\lift{w}{y!(z)}\widehat{\id{\{}u / z \id{\}}}
		& = &
		\lift{w}{y!(u)} \nonumber\\
	w[ \lpquote y!(z) \rpquote ] \widehat{ \id{\{}u / z \id{\}} }
		& = &
		w[ \lpquote y!(z) \rpquote ] \nonumber
\end{eqnarray}

Because the body of the process between quotes is impervious to
substitution, we get radically different answers. In fact, by
examining the first process in an input context,
e.g. $x?(z).\lift{w}{y!(z)}$, we see that the process under the lift
operator may be shaped by prefixed inputs binding a name inside it. In
this sense, the lift operator will be seen as a way to dynamically
construct processes before reifying them as names.

Finally equipped with these standard features we can present the
dynamics of the calculus.

\subsubsection{Operational semantics} 

Finally, we introduce the computational dynamics. What marks these
algebras as distinct from other more traditionally studied algebraic
structures, e.g. vector spaces or polynomial rings, is the manner in
which dynamics is captured. In traditional structures, dynamics is typically
expressed through morphisms between such structures, as in linear maps
between vector spaces or morphisms between rings. In algebras
associated with the semantics of computation, the dynamics is
expressed as part of the algebraic structure itself, through a
reduction reduction relation typically denoted by $\red$. Below, we
give a recursive presentation of this relation for the calculus used
in the encoding.

$\red \subseteq \pi \times \pi$
$\red : \pi \to \mathcal{P}(\pi)$

\begin{mathpar}
  \inferrule* [lab=Comm] { \textsf{match}( x_{src}, x_{trgt} ) } { x_{trgt}?(y)P \; | \; x_{src}!\langle {Q} \rangle \red P\{\quotep{Q}/y}\} }
  \and \\
  \inferrule* [lab=Par] {{P} \red {P}'} {{{P} | {Q}} \red {{P}' | {Q}}}
  \and
  \inferrule* [lab=Equiv]{{{P} \scong {P}'} \andalso {{P}' \red {Q}'} \andalso {{Q}' \scong {Q}}}{{P} \red {Q}}
\end{mathpar}

\begin{eqnarray*}
  match_{\equiv} (\quotep{P},\quotep{Q}) & := & P \equiv Q \\
  match_{\dagger}(\quotep{P},\quotep{Q}) & := & \forall R. P|Q \red^{*} R => R \red^{*} 0 \\
  match_{K}(\quotep{P},\quotep{Q}) & := & K \mbox{ for some context } K
\end{eqnarray*}

$u?(x)P | u!\langle Q \rangle \red P\{\quotep{Q}/x\}$

%We write $\wred$ for $\red^*$, and $P\red$ if $\exists Q $ such that $ P \red Q$.
We write $P\red$ if $\exists Q $ such that $ P \red Q$ and $P\not\red$, otherwise.

\section{Replication}

As mentioned before, it is known that replication (and hence
recursion) can be implemented in a higher-order process algebra
\cite{SangiorgiWalker}. As our first example of calculation with the
machinery thus far presented we give the construction explicitly in
the {\rhoc}.

\begin{eqnarray}
	D_{x} & := & \prefix{x}{y}{(\binpar{\outputp{x}{y}}{@{y}})} \nonumber\\
	\bangp_{x}{P} & := & \binpar{{x}!\langle{\binpar{D_{x}}{P}}\rangle}{D_{x}} \nonumber
\end{eqnarray}

\begin{eqnarray}
	\bangp_{x}{P} & & \nonumber\\
	=
	& {x}!\langle{(\prefix{x}{y}{(\outputp{x}{y} | @{y})) | P}}\rangle 
	      | \prefix{x}{y}{(\outputp{x}{y} | @{y})} & \nonumber\\
	\red
	& (\outputp{x}{y} | @{y})\substn{\quotep{(\prefix{x}{y}{(@{y} | \outputp{x}{y})) | P}}}{y} & \nonumber\\
	=
	& \outputp{x}{\quotep{(\prefix{x}{y}{(\outputp{x}{y} | @{y})) | P}}}
	  | {(\prefix{x}{y}{(\outputp{x}{y} | @{y})) | P}} & \nonumber\\
	\red
	& \ldots & \nonumber\\
	\red^*
	& P | P | \ldots & \nonumber
\end{eqnarray}

Of course, this encoding, as an implementation, runs away, unfolding
$\bangp{P}$ eagerly. A lazier and more implementable replication
operator, restricted to input-guarded processes, may be obtained as follows.

\begin{eqnarray}
\bangp{\prefix{u}{v}{P}} 
	:= 
	\binpar{\lift{x}{\prefix{u}{v}{(\binpar{D(x)}{P})}}}{D(x)} \nonumber
\end{eqnarray}

\begin{remark}
  Note that the lazier definition still does not deal with summation
  or mixed summation (i.e. sums over input and output). The reader is
  invited to construct definitions of replication that deal with these
  features. 

  Further, the definitions are parameterized in a name, $x$. Can you,
  gentle reader, make a definition that eliminates this parameter and
  guarantees no accidental interaction between the replication
  machinery and the process being replicated -- i.e. no accidental
  sharing of names used by the process to get its work done and the
  name(s) used by the replication to effect copying. This latter
  revision of the definition of replication is crucial to obtaining
  the expected identity $!!P \sim !P$.
\end{remark}

\begin{remark}\label{rem:paradoxical_combinator}
  The reader familiar with the lambda calculus will have noticed the
  similarity between $D$ and the paradoxical combinator.

  [Ed. note: the existence of this seems to suggest we have to be more
  restrictive on the set of processes and names we admit if we are to
  support no-cloning.]
\end{remark}

\subsubsection{Bisimulation}

The computational dynamics gives rise to another kind of equivalence,
the equivalence of computational behavior. As previously mentioned
this is typically captured \emph{via} some form of bisimulation.

% The notion we use in this paper is weak barbed bisimulation
% \cite{milner91polyadicpi}.

The notion we use in this paper is derived from weak barbed
bisimulation \cite{milner91polyadicpi}. 

\begin{definition}
An \emph{observation relation}, $\downarrow_{\mathcal N}$, over a set
of names, $\mathcal N$, is the smallest relation satisfying the rules
below.

\infrule[Out-barb]{y \in {\mathcal N}, \; x \nameeq y}
		  {\outputp{x}{v} \downarrow_{\mathcal N} x}
\infrule[Par-barb]{\mbox{$P\downarrow_{\mathcal N} x$ or $Q\downarrow_{\mathcal N} x$}}
		  {\binpar{P}{Q} \downarrow_{\mathcal N} x}

We write $P \Downarrow_{\mathcal N} x$ if there is $Q$ such that 
$P \wred Q$ and $Q \downarrow_{\mathcal N} x$.
\end{definition}

\begin{definition}
%\label{def.bbisim}
An  ${\mathcal N}$-\emph{barbed bisimulation} over a set of names, ${\mathcal N}$, is a symmetric binary relation 
${\mathcal S}_{\mathcal N}$ between agents such that $P\rel{S}_{\mathcal N}Q$ implies:
\begin{enumerate}
\item If $P \red P'$ then $Q \wred Q'$ and $P'\rel{S}_{\mathcal N} Q'$.
\item If $P\downarrow_{\mathcal N} x$, then $Q\Downarrow_{\mathcal N} x$.
\end{enumerate}
$P$ is ${\mathcal N}$-barbed bisimilar to $Q$, written
$P \wbbisim_{\mathcal N} Q$, if $P \rel{S}_{\mathcal N} Q$ for some ${\mathcal N}$-barbed bisimulation ${\mathcal S}_{\mathcal N}$.
\end{definition}

$\mathcal{R} \subseteq \pi \times \pi$

$P \mathcal{R} Q => \forall P'. P \red P' \Rightarrow \exists Q'. Q \red Q', P' \mathcal{R} Q'$

$P \vdash x \Rightarrow Q \vdash x$

\begin{mathpar}
  \inferrule*[lab=Out-barb]{x \nameeq y}{{y}!\langle{Q}\rangle \vdash x}
  \and
  \inferrule*[lab=Par-barb]{\mbox{$P\vdash x$ or $Q\vdash x$}}{\binpar{P}{Q} \vdash x}
\end{mathpar}

\subsubsection{Contexts}

One of the principle advantages of computational calculi like the
$\pi$-calculus is a well-defined notion of context,
contextual-equivalence and a correlation between
contextual-equivalence and notions of bisimulation. The notion of
context allows the decomposition of a process into (sub-)process and
its syntactic environment, its context. Thus, a context may be
thought of as a process with a ``hole'' (written $\Box$) in it. The
application of a context $M$ to a process $P$, written $M[P]$, is
tantamount to filling the hole in $M$ with $P$. In this paper we do
not need the full weight of this theory, but do make use of the notion
of context in the proof the main theorem. 

\begin{mathpar}
  \inferrule* [lab=summation] {} {{M_{M},M_{N}} \bc \Box \;|\; x.M_{A} \;|\; M_{M}+M_{N}}
  \and
  \inferrule* [lab=agent] {} {{M_{A}} \bc (\vec{x})M_{P} \;| \; \clift{P_0,\ldots,M_{P},\ldots,P_N}}
  \and \\
  \inferrule* [lab=process] {} {{M_{P}} \bc M_{N} \;| \;P|M_{P} }
\end{mathpar} 

\begin{mathpar}
  \inferrule* [lab=sychronization] {} {M_{N} \bc \Box \;|\; x?M_{F} \;|\; x!M_{C}}
  \and
  \inferrule* [lab=abstraction] {} {{M_{F}} \bc (x)M_{P} }
  \and
  \inferrule* [lab=concretion] {} {{M_{C}} \bc \langle M_{P} \rangle }
  \and \\
  \inferrule* [lab=process] {} {{M_{P}} \bc M_{N} \;| \;P|M_{P} }
\end{mathpar}

\begin{definition}[contextual application] Given a context $M$, and
  process $P$, we define the \emph{contextual application}, $M[P] :=
  M\{P/\Box\}$. That is, the contextual application of M to P is the
  substitution of $P$ for $\Box$ in $M$.
\end{definition}

$\meaningof{-} : L \to \mathcal{P}(\pi)$

\begin{mathpar}
  \inferrule* [lab=collection] {} {\meaningof{true} = \pi, \and \meaningof{~E} = \pi \setminus \meaningof{E}, \and \meaningof{E_{1} \& E_{2}} = \meaningof{E_{1}} \cap \meaningof{E_{2}}}
\end{mathpar}

\begin{mathpar}
  \inferrule* [lab=structure] {} {\meaningof{0} = \{ P \in \pi | P \equiv 0 \}, \and \\ \meaningof{E_1 | E_2} = \{ P \in \pi | P \equiv P_{1} | P_{2}, P_{1} \in \meaningof{E_{1}}, P_{2} \in \meaningof{E_2}\} }
\end{mathpar}

\begin{mathpar}
 \inferrule* [lab=behavior] {} {\meaningof{\langle a?b \rangle E} = \{ P \in \pi | P \equiv Q | u?(y)P', \\ \and \\\\ \and \\ \;\;\; u \in \meaningof{a}, \forall z.P'\{z/y\} \in \meaningof{E\{z/b\}}\}, \and \\ \meaningof{a!E} = \{ P \in \pi | P \equiv Q | x!\langle P' \rangle, x \in \meaningof{a} P' \in \meaningof{E}\} }
\end{mathpar}

\begin{mathpar}
 \inferrule* [lab=nominal] {} {\meaningof{\quotep{E}} = \{ \quotep{P} \in \quotep{\pi} | P \in \meaningof{E} \}, \and \meaningof{\quotep{P}} = \{ \quotep{Q} \in \quotep{\pi} | P \equiv Q \} \and \\ \meaningof{@\quotep{E}} = \{ P \in \pi | P \equiv @x, x \in \meaningof{E} \}}
\end{mathpar}

\begin{eqnarray*}
  \\
  \meaningof{-} : TS \to ST
\end{eqnarray*}

\begin{eqnarray*}
  \\
  L : TS \to ST
\end{eqnarray*}

\begin{eqnarray*}
  \\
  P \models E \iff P \in \meaningof{E}
\end{eqnarray*}

\begin{eqnarray*}
  P \approx_{L} Q \iff \forall E \in L. P \models E \iff Q \models E
\end{eqnarray*}

\begin{eqnarray*}
  P \approx_{K} Q
\end{eqnarray*}

\begin{eqnarray*}
  P \approx Q
\end{eqnarray*}

$\approx_{K} = \approx = \approx_{L}$

\subsubsection{Contextual duality}

Note that contexts extend the quotation operation to a family of
operations from processes to names. Given a context, $M$, we can
define a \emph{nominal context}, $\quotep{M}$ by $\quotep{M}[P] :=
\quotep{M[P]}$. To foreshadow what is to come we observe that these
operations enjoy a duality with processes very much like the duality
between vectors and maps from vectors to scalars.

Further, because the calculus is essentially higher-order, we have a
correspondence between contexts and processes. More specifically,
given a name $x$ and a context $M$ we can construct $M^{*}_{x}$ such
that 

\begin{mathpar}
  M^{*}_{x} | \lift{x}{P} \red M[P]
\end{mathpar}

namely,

\begin{mathpar}
  M^{*}_{x} := x?(u).M[\dropn{u}]
\end{mathpar}

The dependence of $M^{*}_{x}$ on a name makes it an abstraction, 

\begin{mathpar}
  M^{*} := (x)x?(u).M[\dropn{u}]
\end{mathpar}

\subsection{Additional notation}

It will sometimes be convenient to denote the process a name
quotes. We already have the notation $x = \quotep{P}$, but it will be
convenient to introduce an alternate notation, $\procn{x}$, when we
want to emphasize the connection to the use of the name. Note that, by
virtue of name equivalence, $\quotep{\procn{x}} \nameeq x$; so, the
notation is consistent with previous definitions.

Further, because names have structure it is possible to effect
substitutions on the basis of that structure. This means we need to
upgrade our notation for substitutions, which we accomplish by
adapting comprehension notation. Thus,

\begin{mathpar}
  P\{ y / x : x \in S \}
\end{mathpar}

is interpreted to mean the process derived from P by replacing (in a
capture-avoiding manner) each occurrence of $x$ in $S$ by $y$. For example,

\begin{mathpar}
  P\{ \quotep{\procn{x}|\procn{x}} / x : x \in \freenames{P} \}
\end{mathpar}

will replace each (occurrence) of a free name $x$ in $P$ by
$\quotep{\procn{x}|\procn{x}}$.

Also, we will avail ourselves of the notation $x^{L}$ and $x^{R}$ to
denote injections of a name into disjoint copies of the name
space. There are numerous ways to accomplish this. One example can be
found in \cite{MeredithR05}. This notation overloads to vectors of
names: $\vec{x}^{\pi} := (x_{i}^{\pi} \; : \; 0 \leq i < |\vec{x}| )$ where $\pi \in \{L,R\}$.

We also use $P^{\Box} := P|\Box$.

In \cite{MeredithR05} an interpretation of the new operator is
given. It turns out that there are several possible interpretations
all enjoying the requisite algebraic properties of the operator (see
\cite{milner91polyadicpi}). We will therefore make liberal use of
$(\nu\; \vec{x})P$.

% subsection the_syntax_and_semantics_of_the_notation_system (end)   

\input{qm2pi.qmops} 

\input{qm2pi.sterngerlach} 

\input{qm2pi.metric} 

% section concurrent_process_calculi (end)

%\input{qm2pi.proofsketch}

% section proof sketch (end)

%\input{qm2pi.slviaknots} 

% section spatial logic via knots (end)

\input{qm2pi.conclusion}

% section conclusion (end)

%\input{qm2pi.dtcodes} 

% section wiring algorithm (end)

\input{qm2pi.ack} 

% section acknowledgments (end)

\newpage


\bibliographystyle{plain}   
\bibliography{../../biblios/main.bib}

\input{qm2pi.rhodetails}

\end{document}

 

% subsection basic_interpretation (end)

%\input{qm2pi.rho.presentation} 
\subsection{The syntax and semantics of the notation system}\label{sub:the_syntax_and_semantics_of_the_notation_system} % (fold)

We now summarize a technical presentation of the calculus that
embodies our theory of dynamics. The typical presentation of such a
calculus follows the style of giving generators and relations on
them. The grammar, below, describing term constructors, freely
generates the set of processes, $\Proc$. This set is then quotiented
by a relation known as structural congruence and it is over this set
that the notion of dynamics is expressed. This presentation is
essentially that of \cite{MeredithR05} with the addition of
polyadicity and summation. For readability we have relegated some of
the technical subtleties to an appendix.

\subsubsection{Process grammar}\label{subsub:process_grammar}

\begin{mathpar}
  \inferrule* [lab=synchronization] {} {{M} \bc \pzero \;|\; x?F \;|\; x!C }
  \and
  \inferrule* [lab=abstraction] {} {{F} \bc (x)P}
  \and
  \inferrule* [lab=concretion] {} {{C} \bc \langle Q \rangle}
  \and
  \inferrule* [lab=process] {} {{P,Q} \bc M \;| \;P|Q \;|\; @{x}}
  \and
  \inferrule* [lab=name] {} {{x} \bc \quotep{P}}
\end{mathpar} 

Note that $\vec{x}$ (resp. $\vec{P}$) denotes a vector of names
(resp. processes) of length $|\vec{x}|$ (resp. $|\vec{P}|$). We adopt
the following useful abbreviations.

\begin{mathpar}
   x?(\vec{y}).P := x.(\vec{y})P \and  x\clift{\vec{P}} := x.\clift{\vec{P}}
   \and x!(y) := \lift{x}{\dropn{y}}
   \and \Pi_{i=0}^{n-1}P_i := P_0 | \ldots | P_{n-1}
\end{mathpar}

\subsubsection{Structural congruence}

\paragraph{Free and bound names and alpha-equivalence.} At the
core of structural equivalence is alpha-equivalence which identifies
process that are the same up to a change of variable. Formally, we
recognize the distinction between free and bound names. The free names
of a process, $\freenames{P}$, may be calculated recursively as
follows:

\begin{mathpar}
\freenames{\pzero} := \emptyset
  \and \\
  \freenames{x?(y).P} := \{ x \} \cup (\freenames{P} \setminus \{ y \})
  \and 
  \freenames{x!\langle P \rangle} := \{ x \} \cup \{ P \} 
  \and \\
  \freenames{P|Q} := \freenames{P} \cup \freenames{Q}
  \and \\
  \freenames{@{x}} := \{ x \}
\end{mathpar}

$\pi$
$\quotep{\pi}$

$\freenames{-} : \pi \to \mathcal{P}(\quotep{\pi})$

\begin{eqnarray*}
  \freenames{\pzero} & := & \emptyset \\
  \freenames{x?(y).P} & := & \{ x \} \cup (\freenames{P} \setminus \{ y \}) \\
  \freenames{x!\langle P \rangle} & := & \{ x \} \cup \{ P \} \\
  \freenames{P|Q} & := & \freenames{P} \cup \freenames{Q} \\
  \freenames{\dropn{x}} & := & \{ x \}
\end{eqnarray*}

The bound names of a process, $\boundnames{P}$, are those names occurring in $P$
that are not free. For example, in $x?(y).0$, the name $x$ is free, while $y$ is bound.

\begin{mathpar}
  \inferrule* [lab=monoidal-laws] {} { P|Q \equiv Q|P \and P|0 \equiv P \and P|(Q|R) \equiv (P|Q)|R }
\end{mathpar}

\begin{mathpar}
  \inferrule* [lab=alpha-equivalence] {} { (x)P \equiv (y)P\{y/x\} \and y \not\in \freenames{P} }
\end{mathpar}

\begin{definition}
Then two processes, $P,Q$, are alpha-equivalent if $P = Q\{\vec{y}/\vec{x}\}$ for
some $\vec{x} \in \boundnames{Q},\vec{y} \in \boundnames{P}$, where $Q\{\vec{y}/\vec{x}\}$
denotes the capture-avoiding substitution of $\vec{y}$ for $\vec{x}$ in $Q$.
\end{definition}

\begin{definition}
  The {\em structural congruence} \cite{SangiorgiWalker} , $\equiv$,
  between processes is the least congruence containing
  alpha-equivalence, satisfying the abelian monoid laws
  (associativity, commutativity and $\pzero$ as identity) for parallel
  composition $|$ and for summation $+$.
\end{definition}

\subsection{Name equivalence}

We take name equivalence, written $\nameeq$, to be the smallest
equivalence relation generated by the following rules.

\begin{mathpar}
\inferrule*[lab=Quote-drop]
{ }
{ \quotep{@{x}} \nameeq x }

\inferrule*[lab=Struct-equiv]
{ P \scong Q }
{ \quotep{P} \nameeq \quotep{Q} }
\end{mathpar}

The astute reader will have noticed that the mutual recursion of names
and processes imposes a mutual recursion on alpha-equivalence and
structural equivalence via name-equivalence. Fortunately, all of this
works out pleasantly and we may calculate in the natural way, free of
concern. The reader interested in the details is referred to the
appendix \ref{appendix:rho_details}.

\subsection{Substitution}

We use $\Proc$ for the set of processes, $\QProc$ for the set of
names, and $\id{\{}\vec{y} / \vec{x} \id{\}}$ to denote partial maps,
$s : \QProc \rightarrow \QProc$. A map, $s$ lifts, uniquely, to a map
on process terms, $\widehat{s} : \Proc \rightarrow \Proc$ by the
following equations.

\begin{mathpar}
  (0) \psubstp{Q}{P} := 0 \\
  (R \juxtap S) \psubstp{Q}{P}
  :=    
  (R)\psubstp{Q}{P} \juxtap (S) \psubstp{Q}{P} \\
  (x?(y).R) \psubstp{Q}{P}    
  :=    
  (x)\substp{Q}{P} (z)\concat( (R \psubstn{z}{y}) \psubstp{Q}{P} ) \\
  (\lift{x}{R}) \psubstp{Q}{P}  
  :=
  \lift{(x)\substp{Q}{P}}{ R \psubstp{Q}{P} } \\
%   (\dropn{x})  \psubstp{Q}{P}       
%   := 
%   \left\{ 
%     \begin{array}{ccc} 
%       \dropn{\quotep{Q}} & & x \nameeq \quotep{P} \\
%       \dropn{x} & & otherwise \\
%     \end{array}
%   \right. 
  (\dropn{x})  \psubstp{Q}{P}       
  := 
  \left\{ 
    \begin{array}{ccc} 
      Q & & x \nameeq \quotep{P} \\
      \dropn{x} & & otherwise \\
    \end{array}
  \right.
\end{mathpar}
 

where

\begin{eqnarray}
  (x)\id{\{} \lpquote Q \rpquote / \lpquote P \rpquote \id{\}}            = 
  \left\{ 
    \begin{array}{ccc}
      \lpquote Q \rpquote & & x \nameeq \lpquote P \rpquote \\
      x & & otherwise \\
    \end{array}
  \right. \nonumber
\end{eqnarray}

and $z$ is chosen distinct from $\quotep{P}$, $\quotep{Q}$, the free
names in $Q$, and all the names in $R$. Our $\alpha$-equivalence will
be built in the standard way from this substitution.

\begin{remark}\label{rem:no_self_referential_names}
  One consequence of these definitions is that $\forall P. \quotep{P}
  \not\in \freenames{P}$.
\end{remark}

\subsection{ Dynamic quote: an example }

Anticipating something of what's to come, consider applying the
substitution, $\widehat{\id{\{}u / z \id{\}}}$, to the following pair
of processes, $\lift{w}{y!(z)}$ and $w[ \lpquote y!(z) \rpquote ]$.

\begin{eqnarray}
	\lift{w}{y!(z)}\widehat{\id{\{}u / z \id{\}}}
		& = &
		\lift{w}{y!(u)} \nonumber\\
	w[ \lpquote y!(z) \rpquote ] \widehat{ \id{\{}u / z \id{\}} }
		& = &
		w[ \lpquote y!(z) \rpquote ] \nonumber
\end{eqnarray}

Because the body of the process between quotes is impervious to
substitution, we get radically different answers. In fact, by
examining the first process in an input context,
e.g. $x?(z).\lift{w}{y!(z)}$, we see that the process under the lift
operator may be shaped by prefixed inputs binding a name inside it. In
this sense, the lift operator will be seen as a way to dynamically
construct processes before reifying them as names.

Finally equipped with these standard features we can present the
dynamics of the calculus.

\subsubsection{Operational semantics} 

Finally, we introduce the computational dynamics. What marks these
algebras as distinct from other more traditionally studied algebraic
structures, e.g. vector spaces or polynomial rings, is the manner in
which dynamics is captured. In traditional structures, dynamics is typically
expressed through morphisms between such structures, as in linear maps
between vector spaces or morphisms between rings. In algebras
associated with the semantics of computation, the dynamics is
expressed as part of the algebraic structure itself, through a
reduction reduction relation typically denoted by $\red$. Below, we
give a recursive presentation of this relation for the calculus used
in the encoding.

$\red \subseteq \pi \times \pi$
$\red : \pi \to \mathcal{P}(\pi)$

\begin{mathpar}
  \inferrule* [lab=Comm] { \textsf{match}( x_{src}, x_{trgt} ) } { x_{trgt}?(y)P \; | \; x_{src}!\langle {Q} \rangle \red P\{\quotep{Q}/y}\} }
  \and \\
  \inferrule* [lab=Par] {{P} \red {P}'} {{{P} | {Q}} \red {{P}' | {Q}}}
  \and
  \inferrule* [lab=Equiv]{{{P} \scong {P}'} \andalso {{P}' \red {Q}'} \andalso {{Q}' \scong {Q}}}{{P} \red {Q}}
\end{mathpar}

\begin{eqnarray*}
  match_{\equiv} (\quotep{P},\quotep{Q}) & := & P \equiv Q \\
  match_{\dagger}(\quotep{P},\quotep{Q}) & := & \forall R. P|Q \red^{*} R => R \red^{*} 0 \\
  match_{K}(\quotep{P},\quotep{Q}) & := & K \mbox{ for some context } K
\end{eqnarray*}

$u?(x)P | u!\langle Q \rangle \red P\{\quotep{Q}/x\}$

%We write $\wred$ for $\red^*$, and $P\red$ if $\exists Q $ such that $ P \red Q$.
We write $P\red$ if $\exists Q $ such that $ P \red Q$ and $P\not\red$, otherwise.

\section{Replication}

As mentioned before, it is known that replication (and hence
recursion) can be implemented in a higher-order process algebra
\cite{SangiorgiWalker}. As our first example of calculation with the
machinery thus far presented we give the construction explicitly in
the {\rhoc}.

\begin{eqnarray}
	D_{x} & := & \prefix{x}{y}{(\binpar{\outputp{x}{y}}{@{y}})} \nonumber\\
	\bangp_{x}{P} & := & \binpar{{x}!\langle{\binpar{D_{x}}{P}}\rangle}{D_{x}} \nonumber
\end{eqnarray}

\begin{eqnarray}
	\bangp_{x}{P} & & \nonumber\\
	=
	& {x}!\langle{(\prefix{x}{y}{(\outputp{x}{y} | @{y})) | P}}\rangle 
	      | \prefix{x}{y}{(\outputp{x}{y} | @{y})} & \nonumber\\
	\red
	& (\outputp{x}{y} | @{y})\substn{\quotep{(\prefix{x}{y}{(@{y} | \outputp{x}{y})) | P}}}{y} & \nonumber\\
	=
	& \outputp{x}{\quotep{(\prefix{x}{y}{(\outputp{x}{y} | @{y})) | P}}}
	  | {(\prefix{x}{y}{(\outputp{x}{y} | @{y})) | P}} & \nonumber\\
	\red
	& \ldots & \nonumber\\
	\red^*
	& P | P | \ldots & \nonumber
\end{eqnarray}

Of course, this encoding, as an implementation, runs away, unfolding
$\bangp{P}$ eagerly. A lazier and more implementable replication
operator, restricted to input-guarded processes, may be obtained as follows.

\begin{eqnarray}
\bangp{\prefix{u}{v}{P}} 
	:= 
	\binpar{\lift{x}{\prefix{u}{v}{(\binpar{D(x)}{P})}}}{D(x)} \nonumber
\end{eqnarray}

\begin{remark}
  Note that the lazier definition still does not deal with summation
  or mixed summation (i.e. sums over input and output). The reader is
  invited to construct definitions of replication that deal with these
  features. 

  Further, the definitions are parameterized in a name, $x$. Can you,
  gentle reader, make a definition that eliminates this parameter and
  guarantees no accidental interaction between the replication
  machinery and the process being replicated -- i.e. no accidental
  sharing of names used by the process to get its work done and the
  name(s) used by the replication to effect copying. This latter
  revision of the definition of replication is crucial to obtaining
  the expected identity $!!P \sim !P$.
\end{remark}

\begin{remark}\label{rem:paradoxical_combinator}
  The reader familiar with the lambda calculus will have noticed the
  similarity between $D$ and the paradoxical combinator.

  [Ed. note: the existence of this seems to suggest we have to be more
  restrictive on the set of processes and names we admit if we are to
  support no-cloning.]
\end{remark}

\subsubsection{Bisimulation}

The computational dynamics gives rise to another kind of equivalence,
the equivalence of computational behavior. As previously mentioned
this is typically captured \emph{via} some form of bisimulation.

% The notion we use in this paper is weak barbed bisimulation
% \cite{milner91polyadicpi}.

The notion we use in this paper is derived from weak barbed
bisimulation \cite{milner91polyadicpi}. 

\begin{definition}
An \emph{observation relation}, $\downarrow_{\mathcal N}$, over a set
of names, $\mathcal N$, is the smallest relation satisfying the rules
below.

\infrule[Out-barb]{y \in {\mathcal N}, \; x \nameeq y}
		  {\outputp{x}{v} \downarrow_{\mathcal N} x}
\infrule[Par-barb]{\mbox{$P\downarrow_{\mathcal N} x$ or $Q\downarrow_{\mathcal N} x$}}
		  {\binpar{P}{Q} \downarrow_{\mathcal N} x}

We write $P \Downarrow_{\mathcal N} x$ if there is $Q$ such that 
$P \wred Q$ and $Q \downarrow_{\mathcal N} x$.
\end{definition}

\begin{definition}
%\label{def.bbisim}
An  ${\mathcal N}$-\emph{barbed bisimulation} over a set of names, ${\mathcal N}$, is a symmetric binary relation 
${\mathcal S}_{\mathcal N}$ between agents such that $P\rel{S}_{\mathcal N}Q$ implies:
\begin{enumerate}
\item If $P \red P'$ then $Q \wred Q'$ and $P'\rel{S}_{\mathcal N} Q'$.
\item If $P\downarrow_{\mathcal N} x$, then $Q\Downarrow_{\mathcal N} x$.
\end{enumerate}
$P$ is ${\mathcal N}$-barbed bisimilar to $Q$, written
$P \wbbisim_{\mathcal N} Q$, if $P \rel{S}_{\mathcal N} Q$ for some ${\mathcal N}$-barbed bisimulation ${\mathcal S}_{\mathcal N}$.
\end{definition}

$\mathcal{R} \subseteq \pi \times \pi$

$P \mathcal{R} Q => \forall P'. P \red P' \Rightarrow \exists Q'. Q \red Q', P' \mathcal{R} Q'$

$P \vdash x \Rightarrow Q \vdash x$

\begin{mathpar}
  \inferrule*[lab=Out-barb]{x \nameeq y}{{y}!\langle{Q}\rangle \vdash x}
  \and
  \inferrule*[lab=Par-barb]{\mbox{$P\vdash x$ or $Q\vdash x$}}{\binpar{P}{Q} \vdash x}
\end{mathpar}

\subsubsection{Contexts}

One of the principle advantages of computational calculi like the
$\pi$-calculus is a well-defined notion of context,
contextual-equivalence and a correlation between
contextual-equivalence and notions of bisimulation. The notion of
context allows the decomposition of a process into (sub-)process and
its syntactic environment, its context. Thus, a context may be
thought of as a process with a ``hole'' (written $\Box$) in it. The
application of a context $M$ to a process $P$, written $M[P]$, is
tantamount to filling the hole in $M$ with $P$. In this paper we do
not need the full weight of this theory, but do make use of the notion
of context in the proof the main theorem. 

\begin{mathpar}
  \inferrule* [lab=summation] {} {{M_{M},M_{N}} \bc \Box \;|\; x.M_{A} \;|\; M_{M}+M_{N}}
  \and
  \inferrule* [lab=agent] {} {{M_{A}} \bc (\vec{x})M_{P} \;| \; \clift{P_0,\ldots,M_{P},\ldots,P_N}}
  \and \\
  \inferrule* [lab=process] {} {{M_{P}} \bc M_{N} \;| \;P|M_{P} }
\end{mathpar} 

\begin{mathpar}
  \inferrule* [lab=sychronization] {} {M_{N} \bc \Box \;|\; x?M_{F} \;|\; x!M_{C}}
  \and
  \inferrule* [lab=abstraction] {} {{M_{F}} \bc (x)M_{P} }
  \and
  \inferrule* [lab=concretion] {} {{M_{C}} \bc \langle M_{P} \rangle }
  \and \\
  \inferrule* [lab=process] {} {{M_{P}} \bc M_{N} \;| \;P|M_{P} }
\end{mathpar}

\begin{definition}[contextual application] Given a context $M$, and
  process $P$, we define the \emph{contextual application}, $M[P] :=
  M\{P/\Box\}$. That is, the contextual application of M to P is the
  substitution of $P$ for $\Box$ in $M$.
\end{definition}

$\meaningof{-} : L \to \mathcal{P}(\pi)$

\begin{mathpar}
  \inferrule* [lab=collection] {} {\meaningof{true} = \pi, \and \meaningof{~E} = \pi \setminus \meaningof{E}, \and \meaningof{E_{1} \& E_{2}} = \meaningof{E_{1}} \cap \meaningof{E_{2}}}
\end{mathpar}

\begin{mathpar}
  \inferrule* [lab=structure] {} {\meaningof{0} = \{ P \in \pi | P \equiv 0 \}, \and \\ \meaningof{E_1 | E_2} = \{ P \in \pi | P \equiv P_{1} | P_{2}, P_{1} \in \meaningof{E_{1}}, P_{2} \in \meaningof{E_2}\} }
\end{mathpar}

\begin{mathpar}
 \inferrule* [lab=behavior] {} {\meaningof{\langle a?b \rangle E} = \{ P \in \pi | P \equiv Q | u?(y)P', \\ \and \\\\ \and \\ \;\;\; u \in \meaningof{a}, \forall z.P'\{z/y\} \in \meaningof{E\{z/b\}}\}, \and \\ \meaningof{a!E} = \{ P \in \pi | P \equiv Q | x!\langle P' \rangle, x \in \meaningof{a} P' \in \meaningof{E}\} }
\end{mathpar}

\begin{mathpar}
 \inferrule* [lab=nominal] {} {\meaningof{\quotep{E}} = \{ \quotep{P} \in \quotep{\pi} | P \in \meaningof{E} \}, \and \meaningof{\quotep{P}} = \{ \quotep{Q} \in \quotep{\pi} | P \equiv Q \} \and \\ \meaningof{@\quotep{E}} = \{ P \in \pi | P \equiv @x, x \in \meaningof{E} \}}
\end{mathpar}

\begin{eqnarray*}
  \\
  \meaningof{-} : TS \to ST
\end{eqnarray*}

\begin{eqnarray*}
  \\
  L : TS \to ST
\end{eqnarray*}

\begin{eqnarray*}
  \\
  P \models E \iff P \in \meaningof{E}
\end{eqnarray*}

\begin{eqnarray*}
  P \approx_{L} Q \iff \forall E \in L. P \models E \iff Q \models E
\end{eqnarray*}

\begin{eqnarray*}
  P \approx_{K} Q
\end{eqnarray*}

\begin{eqnarray*}
  P \approx Q
\end{eqnarray*}

$\approx_{K} = \approx = \approx_{L}$

\subsubsection{Contextual duality}

Note that contexts extend the quotation operation to a family of
operations from processes to names. Given a context, $M$, we can
define a \emph{nominal context}, $\quotep{M}$ by $\quotep{M}[P] :=
\quotep{M[P]}$. To foreshadow what is to come we observe that these
operations enjoy a duality with processes very much like the duality
between vectors and maps from vectors to scalars.

Further, because the calculus is essentially higher-order, we have a
correspondence between contexts and processes. More specifically,
given a name $x$ and a context $M$ we can construct $M^{*}_{x}$ such
that 

\begin{mathpar}
  M^{*}_{x} | \lift{x}{P} \red M[P]
\end{mathpar}

namely,

\begin{mathpar}
  M^{*}_{x} := x?(u).M[\dropn{u}]
\end{mathpar}

The dependence of $M^{*}_{x}$ on a name makes it an abstraction, 

\begin{mathpar}
  M^{*} := (x)x?(u).M[\dropn{u}]
\end{mathpar}

\subsection{Additional notation}

It will sometimes be convenient to denote the process a name
quotes. We already have the notation $x = \quotep{P}$, but it will be
convenient to introduce an alternate notation, $\procn{x}$, when we
want to emphasize the connection to the use of the name. Note that, by
virtue of name equivalence, $\quotep{\procn{x}} \nameeq x$; so, the
notation is consistent with previous definitions.

Further, because names have structure it is possible to effect
substitutions on the basis of that structure. This means we need to
upgrade our notation for substitutions, which we accomplish by
adapting comprehension notation. Thus,

\begin{mathpar}
  P\{ y / x : x \in S \}
\end{mathpar}

is interpreted to mean the process derived from P by replacing (in a
capture-avoiding manner) each occurrence of $x$ in $S$ by $y$. For example,

\begin{mathpar}
  P\{ \quotep{\procn{x}|\procn{x}} / x : x \in \freenames{P} \}
\end{mathpar}

will replace each (occurrence) of a free name $x$ in $P$ by
$\quotep{\procn{x}|\procn{x}}$.

Also, we will avail ourselves of the notation $x^{L}$ and $x^{R}$ to
denote injections of a name into disjoint copies of the name
space. There are numerous ways to accomplish this. One example can be
found in \cite{MeredithR05}. This notation overloads to vectors of
names: $\vec{x}^{\pi} := (x_{i}^{\pi} \; : \; 0 \leq i < |\vec{x}| )$ where $\pi \in \{L,R\}$.

We also use $P^{\Box} := P|\Box$.

In \cite{MeredithR05} an interpretation of the new operator is
given. It turns out that there are several possible interpretations
all enjoying the requisite algebraic properties of the operator (see
\cite{milner91polyadicpi}). We will therefore make liberal use of
$(\nu\; \vec{x})P$.

% subsection the_syntax_and_semantics_of_the_notation_system (end)   

\section{Interpretation of QM}
\subsection{Supporting definitions}
\subsubsection{Multiplication}
\begin{mathpar}
  \quotep{Q} \cdot \quotep{R} := \quotep{Q|R}
  \and \\
  \quotep{Q} \cdot P := P\{ \quotep{Q|R} / \quotep{R} : \quotep{R} \in \freenames{P} \}
\end{mathpar}

\paragraph{Discussion}
The first line needs little explanation. The second line says that
each free name of the process is replaced with the multiplication of
that name by the scalar. Multiplication of a scalar (name) by a state
(process) results in a process all the names of which have been `moved
over' by parallel composition with the process the scalar
quotes. There is a subtlety that the bound names have to be
manipulated so that multiplied names aren't accidentally
captured. There are many ways to achieve this.

\begin{remark}\label{rem:multiplication_identities}
  The reader is invited to verify that for all $x,y,z \in \QProc$ and $P \in \Proc$
  \begin{mathpar}
    x \cdot \quotep{0} \equiv x 
    \and
    x \cdot y \equiv y \cdot x
    \and
    x \cdot (y \cdot z) \equiv (x \cdot y) \cdot z
    \and \\
    \quotep{0} \cdot P \equiv P
    \and \\
    x \cdot (y \cdot P) \equiv (x \cdot y) \cdot P
    \and \\
    x \cdot (P|Q) \equiv (x \cdot P) | (x \cdot Q)
    \and \\    
  \end{mathpar}
\end{remark}

\subsubsection{Tensor product}

We define a tensor product on processes by structural induction.

\paragraph{Tensor of sums} First note that all summations, including
$\pzero$ and sequence, can be written $\Sigma_{i} x_{i}.A_{i} +
\Sigma_{j} x_{j}.C_{j}$, where we have grouped input-guarded processes
together and output-guarded processes together.

Thus, we can define the tensor product of two summations, $N_{1}\otimes N_{2}$, where

\begin{mathpar}
  N_{1} := \Sigma_{i} x_{i}.A_{i} + \Sigma_{j} x_{j}.C_{j}
  \and
  N_{2} := \Sigma_{i'} y_{i'}.B_{i'} + \Sigma_{j'} y_{j'}.D_{j'} 
\end{mathpar}

as follows.

\begin{mathpar}
  \Sigma_{i} x_{i}.A_{i} + \Sigma_{j} x_{j}.C_{j} \otimes \Sigma_{i'}
  y_{i'}.B_{i'} + \Sigma_{j'} y_{j'}.D_{j'} 
  \and \\
  := \; \Sigma_{i} \Sigma_{i'} \quotep{\stackrel{\vee}{x_{i}}| \stackrel{\vee}{y_{i'}}}.(A_{i}\otimes B_{i'}) \; | \; \Sigma_{i'} \Sigma_{i} \quotep{\stackrel{\vee}{y_{i'}}|\stackrel{\vee}{x_{i}}}.(B_{i'}\otimes A_{i})
  \and
  \;\; | \;\; \Sigma_{j} \Sigma_{j'} \quotep{\stackrel{\vee}{x_{j}}|\stackrel{\vee}{y_{j'}}}.(A_{j}\otimes B_{j'}) \; | \; \Sigma_{j'} \Sigma_{j} \quotep{\stackrel{\vee}{y_{j'}}|\stackrel{\vee}{x_{j}}}.(B_{j'}\otimes A_{j})
\end{mathpar}

\begin{remark}
  Do we need to $x^{L}$ and $y^{R}$ for this construction as well?
\end{remark}

\paragraph{Tensor of parallel compositions} Next, we distribute tensor
over par.

\begin{mathpar}
  P_{1}|P_{2} \otimes Q_{1}|Q_{2} := (P_{1} \otimes Q_{1}) | (P_{1}
  \otimes Q_{2}) | (P_{2} \otimes Q_{1}) | (P_{2} \otimes Q_{2})
\end{mathpar}

\paragraph{Tensor with dropped names} We treat tensor of a
process with a dropped name as parallel composition.

\begin{mathpar}
  P \otimes \dropn{x} := P | \dropn{x}
\end{mathpar}

\paragraph{Tensor of agents}

Finally, we need to define tensor on agents. Note that the definition
of tensor on normal products only tensors inputs with inputs and
outputs with outputs. Thus, we only have to define the operation on
``homogeneous'' pairings.

\begin{mathpar}
  (\vec{x})P \otimes (\vec{y})Q
  \and \\
  := (x_{0}^{L}|y_{0}^{R},\ldots,x_{0}^{L}|y_{n}^{R},\ldots,x_{m}^{L}|y_{0}^{R},\ldots,x_{m}^{L}|y_{n}^R)(P\{ \vec{x}^{L}/\vec{x}\} \otimes Q \{ \vec{y}^{R}/\vec{y}\})
  \and \\
  \clift{\vec{P}} \otimes \clift{\vec{Q}}
  \and \\
  := \clift{P_{0}\otimes Q_{0},\ldots,P_{0}\otimes Q_{n},\ldots,P_{m}\otimes Q_{0},\ldots,P_{m}\otimes Q_{n}}
\end{mathpar}

\begin{remark}
  Observe that arities of tensored abstractions matches arities of
  tensored concretions if the original arities matched. Note also that
  the length of the arities corresponds to the increase in dimension
  we see in ordinary vector space tensor product.
\end{remark}

\begin{remark}
  Operationally, this definition distributes the tensor down to
  components ``linked'' by summation. Tensor over summation is
  intriguing in that it mixes names. Moreover, as a consequence of the
  way it mixes names we have the identities for all $x \in \QProc$ and
  $P,Q \in \Proc$

  \begin{mathpar}
    (x \cdot P) \otimes Q \equiv x \cdot (P \otimes Q) \equiv P \otimes (x \cdot Q)
    \and
    P \otimes \pzero \equiv P
  \end{mathpar}

  that the reader is invited to verify.
\end{remark}

\subsubsection{Annihilation}
\begin{mathpar}
  P^{\perp} := \{ Q | \forall R. P|Q \red^{*} R \Rightarrow R \red^{*} \pzero \}
  \and \\
  P^{\underline{\perp}} := \Sigma_{Q \in P^{\perp}} \quotep{Q}?(y).(\dropn{y}|Q) | \Sigma_{Q \in P^{\perp}} \quotep{Q}\clift{\Box}
\end{mathpar}

\paragraph{Discussion} The reader will note that $P^{\perp}$ is a
\emph{set} of processes, while $P^{\underline{\perp}}$ is a
\emph{context}. We call the set $P^{\perp}$ the \emph{annihilators} of
$P$. The parallel composition of a process in the annihilators of $P$
with $P$ will result in a process, the state space of which has all
paths eventually leading to $\pzero$. Execution may endure loops; but
under reasonable conditions of fairness (naturally guaranteed under
most notions of bisimulation) such a composite process cannot get
stuck in such a loop and will, eventually pop out and terminate.

The context $P^{\underline{\perp}}$ is ready and willing to ``take the
$P$ out of'' the process to which it is applied. It will effectively
transmit the code of the process to which it is applied to one of the
annihilators and run the process against it.

\subsubsection{Evaluation}
We fix $M$ a domain of fully abstract interpretation with an equality
coincident with bisimulation. We take $\meaningof{\cdot} : \Proc \to
M$ to be the map interpreting processes and $\nmeaningof{\cdot} : \M
\to Proc$ to be the map running the other way. Then we define

\begin{mathpar}
  \int P := \nmeaningof{\meaningof{P}}
\end{mathpar}

\paragraph{Discussion}
There are many fully abstract interpretations of Milner's
$\pi$-calculus. Any of them can be used as a basis for interpreting
the reflective calculus here. Equipped with such a domain it is
largely a matter of grinding through to check that the Yoneda
construction for the normalization-by-evaluation program can be
extended to this setting.

\begin{remark}
  The reader is invited to verify that $\int (P^{\underline{\perp}}[P]) = 0$.
\end{remark}

\subsection{Quantum mechanics}

Table \ref{tbl:core_qm_op_defns} gives the core operational definitions

\begin{table}[htp]\label{tbl:core_qm_op_defns}
  \center{
    \fbox{
      \begin{tabular}{c|c}
        quantum mechanics & process calculus \\
        \hline
        scalar & $x := \quotep{P}$ \\
        state vector & $\state{P} := P$ \\
        dual & $\state{P}^{*} := \event{P^{\underline{\perp}}} := \quotep{P^{\underline{\perp}}}[-]$ \\
        matrix & $ \Sigma_{\alpha} \state{P_{\alpha}}x_{\alpha}\event{Q_{\alpha}}$ \\
        vector addition & $\state{P} + \state{Q} := \state{P | Q}$ \\
        tensor product & $\state{P} \otimes \state{Q} := \state{P \otimes Q}$ \\
        inner product & $\innerprod{P}{Q} := \quotep{\int P^{\underline{\perp}}[Q]}$ \\
      \end{tabular}
    }
  }
  \caption{QM - operational definitions}
\end{table}

where

\begin{mathpar}
  \prmatrix{P}{Q} := \fprmatrix{P}{\quotep{\pzero}}{Q}
  \and
  \fprmatrix{P}{x}{Q} := (\state{P},x,\event{Q})
  \and
  (\fprmatrix{P}{x}{Q})(\state{R}) := x \cdot \innerprod{Q}{R} \cdot \state{P}
  \and
  (\fprmatrix{P}{x}{Q})(\event{R}) := x \cdot \innerprod{R}{P} \cdot \event{Q}
\end{mathpar}

\paragraph{Discussion}
As promised: vectors (aka states) are represented as processes; duals
as contextual duals; inner product definition should be compared with
standard inner product definition for ....

\begin{remark}
  Assuming $\int (P^{\underline{\perp}}[P]) = 0$, the reader is
  invited to verify that $(\fprmatrix{P}{x}{P})(\state{P}) = x \cdot \state{P}$.
\end{remark}

\begin{remark}
  The reader is invited to verify that $\innerprod{P}{Q}$ could
  equally well have been written $\quotep{\int \stackrel{\vee}{x}}$
  where $x = \event{P^{\underline{\perp}}}(Q)$.

  One of the motivations for this remark is that there is another way
  to factor these operations. We could package up evaluation in the dual:

  \begin{mathpar}
    \state{P}^{*} := \event{\int P^{\underline{\perp}}} := \quotep{\int P^{\underline{\perp}}}[-]
  \end{mathpar}

  and then have inner product defined by
  
  \begin{mathpar}
    \innerprod{P}{Q} := \event{P}(Q)
  \end{mathpar}

  Hopefully, experience with the calculations will provide guidance on
  the best factoring.
\end{remark}

\begin{remark}
  Assuming $\int (P^{\underline{\perp}}[P]) = 0$, the reader is
  invited to verify that $\forall P,Q. (\prmatrix{0}{Q})(\state{0}) =
  \state{0}$ and dually $(\prmatrix{P}{0})(\event{0}) = \event{0}$.
\end{remark}

\begin{remark}
  i'm a little worried that i don't (yet) have proper support for
  complex conjugacy. But, the observation above may give us a
  clue. According to Abramsky, it must be the case that the scalars
  are iso to the homset of the identity for the tensor -- which the
  observation above characterizes. 

  For now, we will simply bookmark the notion with $\overline{x}$.
\end{remark}

\subsubsection{Adjointness}

We need to give a definition of $(\cdot)^{\dagger}$ for matrices. The
obvious candidate definition is
\begin{mathpar}
(\Sigma_{\alpha}\fprmatrix{P_{\alpha}}{x_{\alpha}}{Q_{\alpha}})^{\dagger}
= \Sigma_{\alpha}\fprmatrix{(Q_{\alpha}^{\underline{\perp}})^{*}}{\overline{x}_{\alpha}}{P_{\alpha}^{\underline{\perp}}} 
\end{mathpar}

But, $(Q_{\alpha}^{\underline{\perp}})^{*}$ requires a name along
which to communicate the process to achieve the context application.

\subsubsection{Basis for a basis}
If processes label states and ``addition'' of states (a.k.a. vector
addition) is interpreted as parallel composition, what corresponds to
notions of linear independence and basis? Here, we recall that Yoshida
has developed a set of \emph{combinators} for an asynchronous verison
of Milner's $\pi$-calculus. These are a finite set of processes such
any process can be expressed as parallel composition of these
combinators together with liberal uses of the new operator and
replication. We can simply give a translation of these into the
present calculus and have reasonable expectation that the property
carries over. That is, that the resultant set allows to express all
processes via parallel composition. Note, however, that there is no
new operator or replication in this calculus. As a result, we expect
that the corresponding set is actually infinite. That is, we expect
that the space is actually infinite dimensional.

\begin{remark}
  The attentive reader may be a bit concerned. Certainly, the
  collection $S$, $K$ and $I$ is a finite set of
  combinators. Shouldn't we expect to see a finite set of combinators
  for an effectively equivalent system? i am very sympathetic to this
  critique and feel it warrants full attention. On the other hand, i
  also have in mind the following analogy. The natural numbers, as a
  monoid under addition, has exactly $1$ generator, while the natural
  numbers, as a monoid under multiplication, has countably many
  generators (the primes). We observe that the application of the
  lambda calculus is much less resource sensitive than the parallel
  composition of the $\pi$-calculus. Could it be the case that we have
  an analogy of the form
  
  \begin{mathpar}
    m + n : MN :: m*n : M|N
  \end{mathpar}

  giving a similar blow up in the set of ``primes''?  This is such a
  wonderful thought that, even if it's not true, i think it's worth
  writing down.
\end{remark}
 

\documentclass[12pt]{llncs}
%\documentclass{jktr}

\usepackage[pdftex]{hyperref}                   
\usepackage {listings}
\usepackage {mathpartir}
\usepackage{bcprules}
%\usepackage{listings}
                       
\usepackage{graphicx} 
%\usepackage[margins=2.5cm,nohead,nofoot]{geometry}
%\usepackage{geometry}
\usepackage{amsfonts}
\usepackage{amstext}
\usepackage{latexsym}
\usepackage{amssymb}
\usepackage{color}


%\include{myPreamble}
\include{qm2pi.local} 

%\ifpdf
%\usepackage[pdftex]{graphicx}
%\else
%\usepackage{graphicx}
%\fi

 % \ifpdf
%  \usepackage{pdfsync}
%  \if


%\title{Brief Article}
%\author{David F. Snyder}
%\author{L.G. Meredith}

%\address{Dept. of Math., Texas State University--San Marcos, San Marcos, TX 78666}
       
\pagestyle{empty}


\begin{document}

\lstset{language=[Objective]Caml,frame=shadowbox}

\input{qm2pi.front}

% section front matter (end)

\input{qm2pi.intro} 
 
% section introduction (end)

% \input{qm2pi.knotations} 

% section notation (end)

\input{qm2pi.process.calculi} 

% section concurrent_process_calculi_and_spatial_logics_ (end)
    
%\input{qm2pi.knots2pi} 

%\input{qm2pi.trefoil} 

%\input{qm2pi.mainthm} 

% subsection basic_interpretation (end)

%\input{qm2pi.rho.presentation} 
\subsection{The syntax and semantics of the notation system}\label{sub:the_syntax_and_semantics_of_the_notation_system} % (fold)

We now summarize a technical presentation of the calculus that
embodies our theory of dynamics. The typical presentation of such a
calculus follows the style of giving generators and relations on
them. The grammar, below, describing term constructors, freely
generates the set of processes, $\Proc$. This set is then quotiented
by a relation known as structural congruence and it is over this set
that the notion of dynamics is expressed. This presentation is
essentially that of \cite{MeredithR05} with the addition of
polyadicity and summation. For readability we have relegated some of
the technical subtleties to an appendix.

\subsubsection{Process grammar}\label{subsub:process_grammar}

\begin{mathpar}
  \inferrule* [lab=synchronization] {} {{M} \bc \pzero \;|\; x?F \;|\; x!C }
  \and
  \inferrule* [lab=abstraction] {} {{F} \bc (x)P}
  \and
  \inferrule* [lab=concretion] {} {{C} \bc \langle Q \rangle}
  \and
  \inferrule* [lab=process] {} {{P,Q} \bc M \;| \;P|Q \;|\; @{x}}
  \and
  \inferrule* [lab=name] {} {{x} \bc \quotep{P}}
\end{mathpar} 

Note that $\vec{x}$ (resp. $\vec{P}$) denotes a vector of names
(resp. processes) of length $|\vec{x}|$ (resp. $|\vec{P}|$). We adopt
the following useful abbreviations.

\begin{mathpar}
   x?(\vec{y}).P := x.(\vec{y})P \and  x\clift{\vec{P}} := x.\clift{\vec{P}}
   \and x!(y) := \lift{x}{\dropn{y}}
   \and \Pi_{i=0}^{n-1}P_i := P_0 | \ldots | P_{n-1}
\end{mathpar}

\subsubsection{Structural congruence}

\paragraph{Free and bound names and alpha-equivalence.} At the
core of structural equivalence is alpha-equivalence which identifies
process that are the same up to a change of variable. Formally, we
recognize the distinction between free and bound names. The free names
of a process, $\freenames{P}$, may be calculated recursively as
follows:

\begin{mathpar}
\freenames{\pzero} := \emptyset
  \and \\
  \freenames{x?(y).P} := \{ x \} \cup (\freenames{P} \setminus \{ y \})
  \and 
  \freenames{x!\langle P \rangle} := \{ x \} \cup \{ P \} 
  \and \\
  \freenames{P|Q} := \freenames{P} \cup \freenames{Q}
  \and \\
  \freenames{@{x}} := \{ x \}
\end{mathpar}

$\pi$
$\quotep{\pi}$

$\freenames{-} : \pi \to \mathcal{P}(\quotep{\pi})$

\begin{eqnarray*}
  \freenames{\pzero} & := & \emptyset \\
  \freenames{x?(y).P} & := & \{ x \} \cup (\freenames{P} \setminus \{ y \}) \\
  \freenames{x!\langle P \rangle} & := & \{ x \} \cup \{ P \} \\
  \freenames{P|Q} & := & \freenames{P} \cup \freenames{Q} \\
  \freenames{\dropn{x}} & := & \{ x \}
\end{eqnarray*}

The bound names of a process, $\boundnames{P}$, are those names occurring in $P$
that are not free. For example, in $x?(y).0$, the name $x$ is free, while $y$ is bound.

\begin{mathpar}
  \inferrule* [lab=monoidal-laws] {} { P|Q \equiv Q|P \and P|0 \equiv P \and P|(Q|R) \equiv (P|Q)|R }
\end{mathpar}

\begin{mathpar}
  \inferrule* [lab=alpha-equivalence] {} { (x)P \equiv (y)P\{y/x\} \and y \not\in \freenames{P} }
\end{mathpar}

\begin{definition}
Then two processes, $P,Q$, are alpha-equivalent if $P = Q\{\vec{y}/\vec{x}\}$ for
some $\vec{x} \in \boundnames{Q},\vec{y} \in \boundnames{P}$, where $Q\{\vec{y}/\vec{x}\}$
denotes the capture-avoiding substitution of $\vec{y}$ for $\vec{x}$ in $Q$.
\end{definition}

\begin{definition}
  The {\em structural congruence} \cite{SangiorgiWalker} , $\equiv$,
  between processes is the least congruence containing
  alpha-equivalence, satisfying the abelian monoid laws
  (associativity, commutativity and $\pzero$ as identity) for parallel
  composition $|$ and for summation $+$.
\end{definition}

\subsection{Name equivalence}

We take name equivalence, written $\nameeq$, to be the smallest
equivalence relation generated by the following rules.

\begin{mathpar}
\inferrule*[lab=Quote-drop]
{ }
{ \quotep{@{x}} \nameeq x }

\inferrule*[lab=Struct-equiv]
{ P \scong Q }
{ \quotep{P} \nameeq \quotep{Q} }
\end{mathpar}

The astute reader will have noticed that the mutual recursion of names
and processes imposes a mutual recursion on alpha-equivalence and
structural equivalence via name-equivalence. Fortunately, all of this
works out pleasantly and we may calculate in the natural way, free of
concern. The reader interested in the details is referred to the
appendix \ref{appendix:rho_details}.

\subsection{Substitution}

We use $\Proc$ for the set of processes, $\QProc$ for the set of
names, and $\id{\{}\vec{y} / \vec{x} \id{\}}$ to denote partial maps,
$s : \QProc \rightarrow \QProc$. A map, $s$ lifts, uniquely, to a map
on process terms, $\widehat{s} : \Proc \rightarrow \Proc$ by the
following equations.

\begin{mathpar}
  (0) \psubstp{Q}{P} := 0 \\
  (R \juxtap S) \psubstp{Q}{P}
  :=    
  (R)\psubstp{Q}{P} \juxtap (S) \psubstp{Q}{P} \\
  (x?(y).R) \psubstp{Q}{P}    
  :=    
  (x)\substp{Q}{P} (z)\concat( (R \psubstn{z}{y}) \psubstp{Q}{P} ) \\
  (\lift{x}{R}) \psubstp{Q}{P}  
  :=
  \lift{(x)\substp{Q}{P}}{ R \psubstp{Q}{P} } \\
%   (\dropn{x})  \psubstp{Q}{P}       
%   := 
%   \left\{ 
%     \begin{array}{ccc} 
%       \dropn{\quotep{Q}} & & x \nameeq \quotep{P} \\
%       \dropn{x} & & otherwise \\
%     \end{array}
%   \right. 
  (\dropn{x})  \psubstp{Q}{P}       
  := 
  \left\{ 
    \begin{array}{ccc} 
      Q & & x \nameeq \quotep{P} \\
      \dropn{x} & & otherwise \\
    \end{array}
  \right.
\end{mathpar}
 

where

\begin{eqnarray}
  (x)\id{\{} \lpquote Q \rpquote / \lpquote P \rpquote \id{\}}            = 
  \left\{ 
    \begin{array}{ccc}
      \lpquote Q \rpquote & & x \nameeq \lpquote P \rpquote \\
      x & & otherwise \\
    \end{array}
  \right. \nonumber
\end{eqnarray}

and $z$ is chosen distinct from $\quotep{P}$, $\quotep{Q}$, the free
names in $Q$, and all the names in $R$. Our $\alpha$-equivalence will
be built in the standard way from this substitution.

\begin{remark}\label{rem:no_self_referential_names}
  One consequence of these definitions is that $\forall P. \quotep{P}
  \not\in \freenames{P}$.
\end{remark}

\subsection{ Dynamic quote: an example }

Anticipating something of what's to come, consider applying the
substitution, $\widehat{\id{\{}u / z \id{\}}}$, to the following pair
of processes, $\lift{w}{y!(z)}$ and $w[ \lpquote y!(z) \rpquote ]$.

\begin{eqnarray}
	\lift{w}{y!(z)}\widehat{\id{\{}u / z \id{\}}}
		& = &
		\lift{w}{y!(u)} \nonumber\\
	w[ \lpquote y!(z) \rpquote ] \widehat{ \id{\{}u / z \id{\}} }
		& = &
		w[ \lpquote y!(z) \rpquote ] \nonumber
\end{eqnarray}

Because the body of the process between quotes is impervious to
substitution, we get radically different answers. In fact, by
examining the first process in an input context,
e.g. $x?(z).\lift{w}{y!(z)}$, we see that the process under the lift
operator may be shaped by prefixed inputs binding a name inside it. In
this sense, the lift operator will be seen as a way to dynamically
construct processes before reifying them as names.

Finally equipped with these standard features we can present the
dynamics of the calculus.

\subsubsection{Operational semantics} 

Finally, we introduce the computational dynamics. What marks these
algebras as distinct from other more traditionally studied algebraic
structures, e.g. vector spaces or polynomial rings, is the manner in
which dynamics is captured. In traditional structures, dynamics is typically
expressed through morphisms between such structures, as in linear maps
between vector spaces or morphisms between rings. In algebras
associated with the semantics of computation, the dynamics is
expressed as part of the algebraic structure itself, through a
reduction reduction relation typically denoted by $\red$. Below, we
give a recursive presentation of this relation for the calculus used
in the encoding.

$\red \subseteq \pi \times \pi$
$\red : \pi \to \mathcal{P}(\pi)$

\begin{mathpar}
  \inferrule* [lab=Comm] { \textsf{match}( x_{src}, x_{trgt} ) } { x_{trgt}?(y)P \; | \; x_{src}!\langle {Q} \rangle \red P\{\quotep{Q}/y}\} }
  \and \\
  \inferrule* [lab=Par] {{P} \red {P}'} {{{P} | {Q}} \red {{P}' | {Q}}}
  \and
  \inferrule* [lab=Equiv]{{{P} \scong {P}'} \andalso {{P}' \red {Q}'} \andalso {{Q}' \scong {Q}}}{{P} \red {Q}}
\end{mathpar}

\begin{eqnarray*}
  match_{\equiv} (\quotep{P},\quotep{Q}) & := & P \equiv Q \\
  match_{\dagger}(\quotep{P},\quotep{Q}) & := & \forall R. P|Q \red^{*} R => R \red^{*} 0 \\
  match_{K}(\quotep{P},\quotep{Q}) & := & K \mbox{ for some context } K
\end{eqnarray*}

$u?(x)P | u!\langle Q \rangle \red P\{\quotep{Q}/x\}$

%We write $\wred$ for $\red^*$, and $P\red$ if $\exists Q $ such that $ P \red Q$.
We write $P\red$ if $\exists Q $ such that $ P \red Q$ and $P\not\red$, otherwise.

\section{Replication}

As mentioned before, it is known that replication (and hence
recursion) can be implemented in a higher-order process algebra
\cite{SangiorgiWalker}. As our first example of calculation with the
machinery thus far presented we give the construction explicitly in
the {\rhoc}.

\begin{eqnarray}
	D_{x} & := & \prefix{x}{y}{(\binpar{\outputp{x}{y}}{@{y}})} \nonumber\\
	\bangp_{x}{P} & := & \binpar{{x}!\langle{\binpar{D_{x}}{P}}\rangle}{D_{x}} \nonumber
\end{eqnarray}

\begin{eqnarray}
	\bangp_{x}{P} & & \nonumber\\
	=
	& {x}!\langle{(\prefix{x}{y}{(\outputp{x}{y} | @{y})) | P}}\rangle 
	      | \prefix{x}{y}{(\outputp{x}{y} | @{y})} & \nonumber\\
	\red
	& (\outputp{x}{y} | @{y})\substn{\quotep{(\prefix{x}{y}{(@{y} | \outputp{x}{y})) | P}}}{y} & \nonumber\\
	=
	& \outputp{x}{\quotep{(\prefix{x}{y}{(\outputp{x}{y} | @{y})) | P}}}
	  | {(\prefix{x}{y}{(\outputp{x}{y} | @{y})) | P}} & \nonumber\\
	\red
	& \ldots & \nonumber\\
	\red^*
	& P | P | \ldots & \nonumber
\end{eqnarray}

Of course, this encoding, as an implementation, runs away, unfolding
$\bangp{P}$ eagerly. A lazier and more implementable replication
operator, restricted to input-guarded processes, may be obtained as follows.

\begin{eqnarray}
\bangp{\prefix{u}{v}{P}} 
	:= 
	\binpar{\lift{x}{\prefix{u}{v}{(\binpar{D(x)}{P})}}}{D(x)} \nonumber
\end{eqnarray}

\begin{remark}
  Note that the lazier definition still does not deal with summation
  or mixed summation (i.e. sums over input and output). The reader is
  invited to construct definitions of replication that deal with these
  features. 

  Further, the definitions are parameterized in a name, $x$. Can you,
  gentle reader, make a definition that eliminates this parameter and
  guarantees no accidental interaction between the replication
  machinery and the process being replicated -- i.e. no accidental
  sharing of names used by the process to get its work done and the
  name(s) used by the replication to effect copying. This latter
  revision of the definition of replication is crucial to obtaining
  the expected identity $!!P \sim !P$.
\end{remark}

\begin{remark}\label{rem:paradoxical_combinator}
  The reader familiar with the lambda calculus will have noticed the
  similarity between $D$ and the paradoxical combinator.

  [Ed. note: the existence of this seems to suggest we have to be more
  restrictive on the set of processes and names we admit if we are to
  support no-cloning.]
\end{remark}

\subsubsection{Bisimulation}

The computational dynamics gives rise to another kind of equivalence,
the equivalence of computational behavior. As previously mentioned
this is typically captured \emph{via} some form of bisimulation.

% The notion we use in this paper is weak barbed bisimulation
% \cite{milner91polyadicpi}.

The notion we use in this paper is derived from weak barbed
bisimulation \cite{milner91polyadicpi}. 

\begin{definition}
An \emph{observation relation}, $\downarrow_{\mathcal N}$, over a set
of names, $\mathcal N$, is the smallest relation satisfying the rules
below.

\infrule[Out-barb]{y \in {\mathcal N}, \; x \nameeq y}
		  {\outputp{x}{v} \downarrow_{\mathcal N} x}
\infrule[Par-barb]{\mbox{$P\downarrow_{\mathcal N} x$ or $Q\downarrow_{\mathcal N} x$}}
		  {\binpar{P}{Q} \downarrow_{\mathcal N} x}

We write $P \Downarrow_{\mathcal N} x$ if there is $Q$ such that 
$P \wred Q$ and $Q \downarrow_{\mathcal N} x$.
\end{definition}

\begin{definition}
%\label{def.bbisim}
An  ${\mathcal N}$-\emph{barbed bisimulation} over a set of names, ${\mathcal N}$, is a symmetric binary relation 
${\mathcal S}_{\mathcal N}$ between agents such that $P\rel{S}_{\mathcal N}Q$ implies:
\begin{enumerate}
\item If $P \red P'$ then $Q \wred Q'$ and $P'\rel{S}_{\mathcal N} Q'$.
\item If $P\downarrow_{\mathcal N} x$, then $Q\Downarrow_{\mathcal N} x$.
\end{enumerate}
$P$ is ${\mathcal N}$-barbed bisimilar to $Q$, written
$P \wbbisim_{\mathcal N} Q$, if $P \rel{S}_{\mathcal N} Q$ for some ${\mathcal N}$-barbed bisimulation ${\mathcal S}_{\mathcal N}$.
\end{definition}

$\mathcal{R} \subseteq \pi \times \pi$

$P \mathcal{R} Q => \forall P'. P \red P' \Rightarrow \exists Q'. Q \red Q', P' \mathcal{R} Q'$

$P \vdash x \Rightarrow Q \vdash x$

\begin{mathpar}
  \inferrule*[lab=Out-barb]{x \nameeq y}{{y}!\langle{Q}\rangle \vdash x}
  \and
  \inferrule*[lab=Par-barb]{\mbox{$P\vdash x$ or $Q\vdash x$}}{\binpar{P}{Q} \vdash x}
\end{mathpar}

\subsubsection{Contexts}

One of the principle advantages of computational calculi like the
$\pi$-calculus is a well-defined notion of context,
contextual-equivalence and a correlation between
contextual-equivalence and notions of bisimulation. The notion of
context allows the decomposition of a process into (sub-)process and
its syntactic environment, its context. Thus, a context may be
thought of as a process with a ``hole'' (written $\Box$) in it. The
application of a context $M$ to a process $P$, written $M[P]$, is
tantamount to filling the hole in $M$ with $P$. In this paper we do
not need the full weight of this theory, but do make use of the notion
of context in the proof the main theorem. 

\begin{mathpar}
  \inferrule* [lab=summation] {} {{M_{M},M_{N}} \bc \Box \;|\; x.M_{A} \;|\; M_{M}+M_{N}}
  \and
  \inferrule* [lab=agent] {} {{M_{A}} \bc (\vec{x})M_{P} \;| \; \clift{P_0,\ldots,M_{P},\ldots,P_N}}
  \and \\
  \inferrule* [lab=process] {} {{M_{P}} \bc M_{N} \;| \;P|M_{P} }
\end{mathpar} 

\begin{mathpar}
  \inferrule* [lab=sychronization] {} {M_{N} \bc \Box \;|\; x?M_{F} \;|\; x!M_{C}}
  \and
  \inferrule* [lab=abstraction] {} {{M_{F}} \bc (x)M_{P} }
  \and
  \inferrule* [lab=concretion] {} {{M_{C}} \bc \langle M_{P} \rangle }
  \and \\
  \inferrule* [lab=process] {} {{M_{P}} \bc M_{N} \;| \;P|M_{P} }
\end{mathpar}

\begin{definition}[contextual application] Given a context $M$, and
  process $P$, we define the \emph{contextual application}, $M[P] :=
  M\{P/\Box\}$. That is, the contextual application of M to P is the
  substitution of $P$ for $\Box$ in $M$.
\end{definition}

$\meaningof{-} : L \to \mathcal{P}(\pi)$

\begin{mathpar}
  \inferrule* [lab=collection] {} {\meaningof{true} = \pi, \and \meaningof{~E} = \pi \setminus \meaningof{E}, \and \meaningof{E_{1} \& E_{2}} = \meaningof{E_{1}} \cap \meaningof{E_{2}}}
\end{mathpar}

\begin{mathpar}
  \inferrule* [lab=structure] {} {\meaningof{0} = \{ P \in \pi | P \equiv 0 \}, \and \\ \meaningof{E_1 | E_2} = \{ P \in \pi | P \equiv P_{1} | P_{2}, P_{1} \in \meaningof{E_{1}}, P_{2} \in \meaningof{E_2}\} }
\end{mathpar}

\begin{mathpar}
 \inferrule* [lab=behavior] {} {\meaningof{\langle a?b \rangle E} = \{ P \in \pi | P \equiv Q | u?(y)P', \\ \and \\\\ \and \\ \;\;\; u \in \meaningof{a}, \forall z.P'\{z/y\} \in \meaningof{E\{z/b\}}\}, \and \\ \meaningof{a!E} = \{ P \in \pi | P \equiv Q | x!\langle P' \rangle, x \in \meaningof{a} P' \in \meaningof{E}\} }
\end{mathpar}

\begin{mathpar}
 \inferrule* [lab=nominal] {} {\meaningof{\quotep{E}} = \{ \quotep{P} \in \quotep{\pi} | P \in \meaningof{E} \}, \and \meaningof{\quotep{P}} = \{ \quotep{Q} \in \quotep{\pi} | P \equiv Q \} \and \\ \meaningof{@\quotep{E}} = \{ P \in \pi | P \equiv @x, x \in \meaningof{E} \}}
\end{mathpar}

\begin{eqnarray*}
  \\
  \meaningof{-} : TS \to ST
\end{eqnarray*}

\begin{eqnarray*}
  \\
  L : TS \to ST
\end{eqnarray*}

\begin{eqnarray*}
  \\
  P \models E \iff P \in \meaningof{E}
\end{eqnarray*}

\begin{eqnarray*}
  P \approx_{L} Q \iff \forall E \in L. P \models E \iff Q \models E
\end{eqnarray*}

\begin{eqnarray*}
  P \approx_{K} Q
\end{eqnarray*}

\begin{eqnarray*}
  P \approx Q
\end{eqnarray*}

$\approx_{K} = \approx = \approx_{L}$

\subsubsection{Contextual duality}

Note that contexts extend the quotation operation to a family of
operations from processes to names. Given a context, $M$, we can
define a \emph{nominal context}, $\quotep{M}$ by $\quotep{M}[P] :=
\quotep{M[P]}$. To foreshadow what is to come we observe that these
operations enjoy a duality with processes very much like the duality
between vectors and maps from vectors to scalars.

Further, because the calculus is essentially higher-order, we have a
correspondence between contexts and processes. More specifically,
given a name $x$ and a context $M$ we can construct $M^{*}_{x}$ such
that 

\begin{mathpar}
  M^{*}_{x} | \lift{x}{P} \red M[P]
\end{mathpar}

namely,

\begin{mathpar}
  M^{*}_{x} := x?(u).M[\dropn{u}]
\end{mathpar}

The dependence of $M^{*}_{x}$ on a name makes it an abstraction, 

\begin{mathpar}
  M^{*} := (x)x?(u).M[\dropn{u}]
\end{mathpar}

\subsection{Additional notation}

It will sometimes be convenient to denote the process a name
quotes. We already have the notation $x = \quotep{P}$, but it will be
convenient to introduce an alternate notation, $\procn{x}$, when we
want to emphasize the connection to the use of the name. Note that, by
virtue of name equivalence, $\quotep{\procn{x}} \nameeq x$; so, the
notation is consistent with previous definitions.

Further, because names have structure it is possible to effect
substitutions on the basis of that structure. This means we need to
upgrade our notation for substitutions, which we accomplish by
adapting comprehension notation. Thus,

\begin{mathpar}
  P\{ y / x : x \in S \}
\end{mathpar}

is interpreted to mean the process derived from P by replacing (in a
capture-avoiding manner) each occurrence of $x$ in $S$ by $y$. For example,

\begin{mathpar}
  P\{ \quotep{\procn{x}|\procn{x}} / x : x \in \freenames{P} \}
\end{mathpar}

will replace each (occurrence) of a free name $x$ in $P$ by
$\quotep{\procn{x}|\procn{x}}$.

Also, we will avail ourselves of the notation $x^{L}$ and $x^{R}$ to
denote injections of a name into disjoint copies of the name
space. There are numerous ways to accomplish this. One example can be
found in \cite{MeredithR05}. This notation overloads to vectors of
names: $\vec{x}^{\pi} := (x_{i}^{\pi} \; : \; 0 \leq i < |\vec{x}| )$ where $\pi \in \{L,R\}$.

We also use $P^{\Box} := P|\Box$.

In \cite{MeredithR05} an interpretation of the new operator is
given. It turns out that there are several possible interpretations
all enjoying the requisite algebraic properties of the operator (see
\cite{milner91polyadicpi}). We will therefore make liberal use of
$(\nu\; \vec{x})P$.

% subsection the_syntax_and_semantics_of_the_notation_system (end)   

\input{qm2pi.qmops} 

\input{qm2pi.sterngerlach} 

\input{qm2pi.metric} 

% section concurrent_process_calculi (end)

%\input{qm2pi.proofsketch}

% section proof sketch (end)

%\input{qm2pi.slviaknots} 

% section spatial logic via knots (end)

\input{qm2pi.conclusion}

% section conclusion (end)

%\input{qm2pi.dtcodes} 

% section wiring algorithm (end)

\input{qm2pi.ack} 

% section acknowledgments (end)

\newpage


\bibliographystyle{plain}   
\bibliography{../../biblios/main.bib}

\input{qm2pi.rhodetails}

\end{document}

 

\documentclass[12pt]{llncs}
%\documentclass{jktr}

\usepackage[pdftex]{hyperref}                   
\usepackage {listings}
\usepackage {mathpartir}
\usepackage{bcprules}
%\usepackage{listings}
                       
\usepackage{graphicx} 
%\usepackage[margins=2.5cm,nohead,nofoot]{geometry}
%\usepackage{geometry}
\usepackage{amsfonts}
\usepackage{amstext}
\usepackage{latexsym}
\usepackage{amssymb}
\usepackage{color}


%\include{myPreamble}
\include{qm2pi.local} 

%\ifpdf
%\usepackage[pdftex]{graphicx}
%\else
%\usepackage{graphicx}
%\fi

 % \ifpdf
%  \usepackage{pdfsync}
%  \if


%\title{Brief Article}
%\author{David F. Snyder}
%\author{L.G. Meredith}

%\address{Dept. of Math., Texas State University--San Marcos, San Marcos, TX 78666}
       
\pagestyle{empty}


\begin{document}

\lstset{language=[Objective]Caml,frame=shadowbox}

\input{qm2pi.front}

% section front matter (end)

\input{qm2pi.intro} 
 
% section introduction (end)

% \input{qm2pi.knotations} 

% section notation (end)

\input{qm2pi.process.calculi} 

% section concurrent_process_calculi_and_spatial_logics_ (end)
    
%\input{qm2pi.knots2pi} 

%\input{qm2pi.trefoil} 

%\input{qm2pi.mainthm} 

% subsection basic_interpretation (end)

%\input{qm2pi.rho.presentation} 
\subsection{The syntax and semantics of the notation system}\label{sub:the_syntax_and_semantics_of_the_notation_system} % (fold)

We now summarize a technical presentation of the calculus that
embodies our theory of dynamics. The typical presentation of such a
calculus follows the style of giving generators and relations on
them. The grammar, below, describing term constructors, freely
generates the set of processes, $\Proc$. This set is then quotiented
by a relation known as structural congruence and it is over this set
that the notion of dynamics is expressed. This presentation is
essentially that of \cite{MeredithR05} with the addition of
polyadicity and summation. For readability we have relegated some of
the technical subtleties to an appendix.

\subsubsection{Process grammar}\label{subsub:process_grammar}

\begin{mathpar}
  \inferrule* [lab=synchronization] {} {{M} \bc \pzero \;|\; x?F \;|\; x!C }
  \and
  \inferrule* [lab=abstraction] {} {{F} \bc (x)P}
  \and
  \inferrule* [lab=concretion] {} {{C} \bc \langle Q \rangle}
  \and
  \inferrule* [lab=process] {} {{P,Q} \bc M \;| \;P|Q \;|\; @{x}}
  \and
  \inferrule* [lab=name] {} {{x} \bc \quotep{P}}
\end{mathpar} 

Note that $\vec{x}$ (resp. $\vec{P}$) denotes a vector of names
(resp. processes) of length $|\vec{x}|$ (resp. $|\vec{P}|$). We adopt
the following useful abbreviations.

\begin{mathpar}
   x?(\vec{y}).P := x.(\vec{y})P \and  x\clift{\vec{P}} := x.\clift{\vec{P}}
   \and x!(y) := \lift{x}{\dropn{y}}
   \and \Pi_{i=0}^{n-1}P_i := P_0 | \ldots | P_{n-1}
\end{mathpar}

\subsubsection{Structural congruence}

\paragraph{Free and bound names and alpha-equivalence.} At the
core of structural equivalence is alpha-equivalence which identifies
process that are the same up to a change of variable. Formally, we
recognize the distinction between free and bound names. The free names
of a process, $\freenames{P}$, may be calculated recursively as
follows:

\begin{mathpar}
\freenames{\pzero} := \emptyset
  \and \\
  \freenames{x?(y).P} := \{ x \} \cup (\freenames{P} \setminus \{ y \})
  \and 
  \freenames{x!\langle P \rangle} := \{ x \} \cup \{ P \} 
  \and \\
  \freenames{P|Q} := \freenames{P} \cup \freenames{Q}
  \and \\
  \freenames{@{x}} := \{ x \}
\end{mathpar}

$\pi$
$\quotep{\pi}$

$\freenames{-} : \pi \to \mathcal{P}(\quotep{\pi})$

\begin{eqnarray*}
  \freenames{\pzero} & := & \emptyset \\
  \freenames{x?(y).P} & := & \{ x \} \cup (\freenames{P} \setminus \{ y \}) \\
  \freenames{x!\langle P \rangle} & := & \{ x \} \cup \{ P \} \\
  \freenames{P|Q} & := & \freenames{P} \cup \freenames{Q} \\
  \freenames{\dropn{x}} & := & \{ x \}
\end{eqnarray*}

The bound names of a process, $\boundnames{P}$, are those names occurring in $P$
that are not free. For example, in $x?(y).0$, the name $x$ is free, while $y$ is bound.

\begin{mathpar}
  \inferrule* [lab=monoidal-laws] {} { P|Q \equiv Q|P \and P|0 \equiv P \and P|(Q|R) \equiv (P|Q)|R }
\end{mathpar}

\begin{mathpar}
  \inferrule* [lab=alpha-equivalence] {} { (x)P \equiv (y)P\{y/x\} \and y \not\in \freenames{P} }
\end{mathpar}

\begin{definition}
Then two processes, $P,Q$, are alpha-equivalent if $P = Q\{\vec{y}/\vec{x}\}$ for
some $\vec{x} \in \boundnames{Q},\vec{y} \in \boundnames{P}$, where $Q\{\vec{y}/\vec{x}\}$
denotes the capture-avoiding substitution of $\vec{y}$ for $\vec{x}$ in $Q$.
\end{definition}

\begin{definition}
  The {\em structural congruence} \cite{SangiorgiWalker} , $\equiv$,
  between processes is the least congruence containing
  alpha-equivalence, satisfying the abelian monoid laws
  (associativity, commutativity and $\pzero$ as identity) for parallel
  composition $|$ and for summation $+$.
\end{definition}

\subsection{Name equivalence}

We take name equivalence, written $\nameeq$, to be the smallest
equivalence relation generated by the following rules.

\begin{mathpar}
\inferrule*[lab=Quote-drop]
{ }
{ \quotep{@{x}} \nameeq x }

\inferrule*[lab=Struct-equiv]
{ P \scong Q }
{ \quotep{P} \nameeq \quotep{Q} }
\end{mathpar}

The astute reader will have noticed that the mutual recursion of names
and processes imposes a mutual recursion on alpha-equivalence and
structural equivalence via name-equivalence. Fortunately, all of this
works out pleasantly and we may calculate in the natural way, free of
concern. The reader interested in the details is referred to the
appendix \ref{appendix:rho_details}.

\subsection{Substitution}

We use $\Proc$ for the set of processes, $\QProc$ for the set of
names, and $\id{\{}\vec{y} / \vec{x} \id{\}}$ to denote partial maps,
$s : \QProc \rightarrow \QProc$. A map, $s$ lifts, uniquely, to a map
on process terms, $\widehat{s} : \Proc \rightarrow \Proc$ by the
following equations.

\begin{mathpar}
  (0) \psubstp{Q}{P} := 0 \\
  (R \juxtap S) \psubstp{Q}{P}
  :=    
  (R)\psubstp{Q}{P} \juxtap (S) \psubstp{Q}{P} \\
  (x?(y).R) \psubstp{Q}{P}    
  :=    
  (x)\substp{Q}{P} (z)\concat( (R \psubstn{z}{y}) \psubstp{Q}{P} ) \\
  (\lift{x}{R}) \psubstp{Q}{P}  
  :=
  \lift{(x)\substp{Q}{P}}{ R \psubstp{Q}{P} } \\
%   (\dropn{x})  \psubstp{Q}{P}       
%   := 
%   \left\{ 
%     \begin{array}{ccc} 
%       \dropn{\quotep{Q}} & & x \nameeq \quotep{P} \\
%       \dropn{x} & & otherwise \\
%     \end{array}
%   \right. 
  (\dropn{x})  \psubstp{Q}{P}       
  := 
  \left\{ 
    \begin{array}{ccc} 
      Q & & x \nameeq \quotep{P} \\
      \dropn{x} & & otherwise \\
    \end{array}
  \right.
\end{mathpar}
 

where

\begin{eqnarray}
  (x)\id{\{} \lpquote Q \rpquote / \lpquote P \rpquote \id{\}}            = 
  \left\{ 
    \begin{array}{ccc}
      \lpquote Q \rpquote & & x \nameeq \lpquote P \rpquote \\
      x & & otherwise \\
    \end{array}
  \right. \nonumber
\end{eqnarray}

and $z$ is chosen distinct from $\quotep{P}$, $\quotep{Q}$, the free
names in $Q$, and all the names in $R$. Our $\alpha$-equivalence will
be built in the standard way from this substitution.

\begin{remark}\label{rem:no_self_referential_names}
  One consequence of these definitions is that $\forall P. \quotep{P}
  \not\in \freenames{P}$.
\end{remark}

\subsection{ Dynamic quote: an example }

Anticipating something of what's to come, consider applying the
substitution, $\widehat{\id{\{}u / z \id{\}}}$, to the following pair
of processes, $\lift{w}{y!(z)}$ and $w[ \lpquote y!(z) \rpquote ]$.

\begin{eqnarray}
	\lift{w}{y!(z)}\widehat{\id{\{}u / z \id{\}}}
		& = &
		\lift{w}{y!(u)} \nonumber\\
	w[ \lpquote y!(z) \rpquote ] \widehat{ \id{\{}u / z \id{\}} }
		& = &
		w[ \lpquote y!(z) \rpquote ] \nonumber
\end{eqnarray}

Because the body of the process between quotes is impervious to
substitution, we get radically different answers. In fact, by
examining the first process in an input context,
e.g. $x?(z).\lift{w}{y!(z)}$, we see that the process under the lift
operator may be shaped by prefixed inputs binding a name inside it. In
this sense, the lift operator will be seen as a way to dynamically
construct processes before reifying them as names.

Finally equipped with these standard features we can present the
dynamics of the calculus.

\subsubsection{Operational semantics} 

Finally, we introduce the computational dynamics. What marks these
algebras as distinct from other more traditionally studied algebraic
structures, e.g. vector spaces or polynomial rings, is the manner in
which dynamics is captured. In traditional structures, dynamics is typically
expressed through morphisms between such structures, as in linear maps
between vector spaces or morphisms between rings. In algebras
associated with the semantics of computation, the dynamics is
expressed as part of the algebraic structure itself, through a
reduction reduction relation typically denoted by $\red$. Below, we
give a recursive presentation of this relation for the calculus used
in the encoding.

$\red \subseteq \pi \times \pi$
$\red : \pi \to \mathcal{P}(\pi)$

\begin{mathpar}
  \inferrule* [lab=Comm] { \textsf{match}( x_{src}, x_{trgt} ) } { x_{trgt}?(y)P \; | \; x_{src}!\langle {Q} \rangle \red P\{\quotep{Q}/y}\} }
  \and \\
  \inferrule* [lab=Par] {{P} \red {P}'} {{{P} | {Q}} \red {{P}' | {Q}}}
  \and
  \inferrule* [lab=Equiv]{{{P} \scong {P}'} \andalso {{P}' \red {Q}'} \andalso {{Q}' \scong {Q}}}{{P} \red {Q}}
\end{mathpar}

\begin{eqnarray*}
  match_{\equiv} (\quotep{P},\quotep{Q}) & := & P \equiv Q \\
  match_{\dagger}(\quotep{P},\quotep{Q}) & := & \forall R. P|Q \red^{*} R => R \red^{*} 0 \\
  match_{K}(\quotep{P},\quotep{Q}) & := & K \mbox{ for some context } K
\end{eqnarray*}

$u?(x)P | u!\langle Q \rangle \red P\{\quotep{Q}/x\}$

%We write $\wred$ for $\red^*$, and $P\red$ if $\exists Q $ such that $ P \red Q$.
We write $P\red$ if $\exists Q $ such that $ P \red Q$ and $P\not\red$, otherwise.

\section{Replication}

As mentioned before, it is known that replication (and hence
recursion) can be implemented in a higher-order process algebra
\cite{SangiorgiWalker}. As our first example of calculation with the
machinery thus far presented we give the construction explicitly in
the {\rhoc}.

\begin{eqnarray}
	D_{x} & := & \prefix{x}{y}{(\binpar{\outputp{x}{y}}{@{y}})} \nonumber\\
	\bangp_{x}{P} & := & \binpar{{x}!\langle{\binpar{D_{x}}{P}}\rangle}{D_{x}} \nonumber
\end{eqnarray}

\begin{eqnarray}
	\bangp_{x}{P} & & \nonumber\\
	=
	& {x}!\langle{(\prefix{x}{y}{(\outputp{x}{y} | @{y})) | P}}\rangle 
	      | \prefix{x}{y}{(\outputp{x}{y} | @{y})} & \nonumber\\
	\red
	& (\outputp{x}{y} | @{y})\substn{\quotep{(\prefix{x}{y}{(@{y} | \outputp{x}{y})) | P}}}{y} & \nonumber\\
	=
	& \outputp{x}{\quotep{(\prefix{x}{y}{(\outputp{x}{y} | @{y})) | P}}}
	  | {(\prefix{x}{y}{(\outputp{x}{y} | @{y})) | P}} & \nonumber\\
	\red
	& \ldots & \nonumber\\
	\red^*
	& P | P | \ldots & \nonumber
\end{eqnarray}

Of course, this encoding, as an implementation, runs away, unfolding
$\bangp{P}$ eagerly. A lazier and more implementable replication
operator, restricted to input-guarded processes, may be obtained as follows.

\begin{eqnarray}
\bangp{\prefix{u}{v}{P}} 
	:= 
	\binpar{\lift{x}{\prefix{u}{v}{(\binpar{D(x)}{P})}}}{D(x)} \nonumber
\end{eqnarray}

\begin{remark}
  Note that the lazier definition still does not deal with summation
  or mixed summation (i.e. sums over input and output). The reader is
  invited to construct definitions of replication that deal with these
  features. 

  Further, the definitions are parameterized in a name, $x$. Can you,
  gentle reader, make a definition that eliminates this parameter and
  guarantees no accidental interaction between the replication
  machinery and the process being replicated -- i.e. no accidental
  sharing of names used by the process to get its work done and the
  name(s) used by the replication to effect copying. This latter
  revision of the definition of replication is crucial to obtaining
  the expected identity $!!P \sim !P$.
\end{remark}

\begin{remark}\label{rem:paradoxical_combinator}
  The reader familiar with the lambda calculus will have noticed the
  similarity between $D$ and the paradoxical combinator.

  [Ed. note: the existence of this seems to suggest we have to be more
  restrictive on the set of processes and names we admit if we are to
  support no-cloning.]
\end{remark}

\subsubsection{Bisimulation}

The computational dynamics gives rise to another kind of equivalence,
the equivalence of computational behavior. As previously mentioned
this is typically captured \emph{via} some form of bisimulation.

% The notion we use in this paper is weak barbed bisimulation
% \cite{milner91polyadicpi}.

The notion we use in this paper is derived from weak barbed
bisimulation \cite{milner91polyadicpi}. 

\begin{definition}
An \emph{observation relation}, $\downarrow_{\mathcal N}$, over a set
of names, $\mathcal N$, is the smallest relation satisfying the rules
below.

\infrule[Out-barb]{y \in {\mathcal N}, \; x \nameeq y}
		  {\outputp{x}{v} \downarrow_{\mathcal N} x}
\infrule[Par-barb]{\mbox{$P\downarrow_{\mathcal N} x$ or $Q\downarrow_{\mathcal N} x$}}
		  {\binpar{P}{Q} \downarrow_{\mathcal N} x}

We write $P \Downarrow_{\mathcal N} x$ if there is $Q$ such that 
$P \wred Q$ and $Q \downarrow_{\mathcal N} x$.
\end{definition}

\begin{definition}
%\label{def.bbisim}
An  ${\mathcal N}$-\emph{barbed bisimulation} over a set of names, ${\mathcal N}$, is a symmetric binary relation 
${\mathcal S}_{\mathcal N}$ between agents such that $P\rel{S}_{\mathcal N}Q$ implies:
\begin{enumerate}
\item If $P \red P'$ then $Q \wred Q'$ and $P'\rel{S}_{\mathcal N} Q'$.
\item If $P\downarrow_{\mathcal N} x$, then $Q\Downarrow_{\mathcal N} x$.
\end{enumerate}
$P$ is ${\mathcal N}$-barbed bisimilar to $Q$, written
$P \wbbisim_{\mathcal N} Q$, if $P \rel{S}_{\mathcal N} Q$ for some ${\mathcal N}$-barbed bisimulation ${\mathcal S}_{\mathcal N}$.
\end{definition}

$\mathcal{R} \subseteq \pi \times \pi$

$P \mathcal{R} Q => \forall P'. P \red P' \Rightarrow \exists Q'. Q \red Q', P' \mathcal{R} Q'$

$P \vdash x \Rightarrow Q \vdash x$

\begin{mathpar}
  \inferrule*[lab=Out-barb]{x \nameeq y}{{y}!\langle{Q}\rangle \vdash x}
  \and
  \inferrule*[lab=Par-barb]{\mbox{$P\vdash x$ or $Q\vdash x$}}{\binpar{P}{Q} \vdash x}
\end{mathpar}

\subsubsection{Contexts}

One of the principle advantages of computational calculi like the
$\pi$-calculus is a well-defined notion of context,
contextual-equivalence and a correlation between
contextual-equivalence and notions of bisimulation. The notion of
context allows the decomposition of a process into (sub-)process and
its syntactic environment, its context. Thus, a context may be
thought of as a process with a ``hole'' (written $\Box$) in it. The
application of a context $M$ to a process $P$, written $M[P]$, is
tantamount to filling the hole in $M$ with $P$. In this paper we do
not need the full weight of this theory, but do make use of the notion
of context in the proof the main theorem. 

\begin{mathpar}
  \inferrule* [lab=summation] {} {{M_{M},M_{N}} \bc \Box \;|\; x.M_{A} \;|\; M_{M}+M_{N}}
  \and
  \inferrule* [lab=agent] {} {{M_{A}} \bc (\vec{x})M_{P} \;| \; \clift{P_0,\ldots,M_{P},\ldots,P_N}}
  \and \\
  \inferrule* [lab=process] {} {{M_{P}} \bc M_{N} \;| \;P|M_{P} }
\end{mathpar} 

\begin{mathpar}
  \inferrule* [lab=sychronization] {} {M_{N} \bc \Box \;|\; x?M_{F} \;|\; x!M_{C}}
  \and
  \inferrule* [lab=abstraction] {} {{M_{F}} \bc (x)M_{P} }
  \and
  \inferrule* [lab=concretion] {} {{M_{C}} \bc \langle M_{P} \rangle }
  \and \\
  \inferrule* [lab=process] {} {{M_{P}} \bc M_{N} \;| \;P|M_{P} }
\end{mathpar}

\begin{definition}[contextual application] Given a context $M$, and
  process $P$, we define the \emph{contextual application}, $M[P] :=
  M\{P/\Box\}$. That is, the contextual application of M to P is the
  substitution of $P$ for $\Box$ in $M$.
\end{definition}

$\meaningof{-} : L \to \mathcal{P}(\pi)$

\begin{mathpar}
  \inferrule* [lab=collection] {} {\meaningof{true} = \pi, \and \meaningof{~E} = \pi \setminus \meaningof{E}, \and \meaningof{E_{1} \& E_{2}} = \meaningof{E_{1}} \cap \meaningof{E_{2}}}
\end{mathpar}

\begin{mathpar}
  \inferrule* [lab=structure] {} {\meaningof{0} = \{ P \in \pi | P \equiv 0 \}, \and \\ \meaningof{E_1 | E_2} = \{ P \in \pi | P \equiv P_{1} | P_{2}, P_{1} \in \meaningof{E_{1}}, P_{2} \in \meaningof{E_2}\} }
\end{mathpar}

\begin{mathpar}
 \inferrule* [lab=behavior] {} {\meaningof{\langle a?b \rangle E} = \{ P \in \pi | P \equiv Q | u?(y)P', \\ \and \\\\ \and \\ \;\;\; u \in \meaningof{a}, \forall z.P'\{z/y\} \in \meaningof{E\{z/b\}}\}, \and \\ \meaningof{a!E} = \{ P \in \pi | P \equiv Q | x!\langle P' \rangle, x \in \meaningof{a} P' \in \meaningof{E}\} }
\end{mathpar}

\begin{mathpar}
 \inferrule* [lab=nominal] {} {\meaningof{\quotep{E}} = \{ \quotep{P} \in \quotep{\pi} | P \in \meaningof{E} \}, \and \meaningof{\quotep{P}} = \{ \quotep{Q} \in \quotep{\pi} | P \equiv Q \} \and \\ \meaningof{@\quotep{E}} = \{ P \in \pi | P \equiv @x, x \in \meaningof{E} \}}
\end{mathpar}

\begin{eqnarray*}
  \\
  \meaningof{-} : TS \to ST
\end{eqnarray*}

\begin{eqnarray*}
  \\
  L : TS \to ST
\end{eqnarray*}

\begin{eqnarray*}
  \\
  P \models E \iff P \in \meaningof{E}
\end{eqnarray*}

\begin{eqnarray*}
  P \approx_{L} Q \iff \forall E \in L. P \models E \iff Q \models E
\end{eqnarray*}

\begin{eqnarray*}
  P \approx_{K} Q
\end{eqnarray*}

\begin{eqnarray*}
  P \approx Q
\end{eqnarray*}

$\approx_{K} = \approx = \approx_{L}$

\subsubsection{Contextual duality}

Note that contexts extend the quotation operation to a family of
operations from processes to names. Given a context, $M$, we can
define a \emph{nominal context}, $\quotep{M}$ by $\quotep{M}[P] :=
\quotep{M[P]}$. To foreshadow what is to come we observe that these
operations enjoy a duality with processes very much like the duality
between vectors and maps from vectors to scalars.

Further, because the calculus is essentially higher-order, we have a
correspondence between contexts and processes. More specifically,
given a name $x$ and a context $M$ we can construct $M^{*}_{x}$ such
that 

\begin{mathpar}
  M^{*}_{x} | \lift{x}{P} \red M[P]
\end{mathpar}

namely,

\begin{mathpar}
  M^{*}_{x} := x?(u).M[\dropn{u}]
\end{mathpar}

The dependence of $M^{*}_{x}$ on a name makes it an abstraction, 

\begin{mathpar}
  M^{*} := (x)x?(u).M[\dropn{u}]
\end{mathpar}

\subsection{Additional notation}

It will sometimes be convenient to denote the process a name
quotes. We already have the notation $x = \quotep{P}$, but it will be
convenient to introduce an alternate notation, $\procn{x}$, when we
want to emphasize the connection to the use of the name. Note that, by
virtue of name equivalence, $\quotep{\procn{x}} \nameeq x$; so, the
notation is consistent with previous definitions.

Further, because names have structure it is possible to effect
substitutions on the basis of that structure. This means we need to
upgrade our notation for substitutions, which we accomplish by
adapting comprehension notation. Thus,

\begin{mathpar}
  P\{ y / x : x \in S \}
\end{mathpar}

is interpreted to mean the process derived from P by replacing (in a
capture-avoiding manner) each occurrence of $x$ in $S$ by $y$. For example,

\begin{mathpar}
  P\{ \quotep{\procn{x}|\procn{x}} / x : x \in \freenames{P} \}
\end{mathpar}

will replace each (occurrence) of a free name $x$ in $P$ by
$\quotep{\procn{x}|\procn{x}}$.

Also, we will avail ourselves of the notation $x^{L}$ and $x^{R}$ to
denote injections of a name into disjoint copies of the name
space. There are numerous ways to accomplish this. One example can be
found in \cite{MeredithR05}. This notation overloads to vectors of
names: $\vec{x}^{\pi} := (x_{i}^{\pi} \; : \; 0 \leq i < |\vec{x}| )$ where $\pi \in \{L,R\}$.

We also use $P^{\Box} := P|\Box$.

In \cite{MeredithR05} an interpretation of the new operator is
given. It turns out that there are several possible interpretations
all enjoying the requisite algebraic properties of the operator (see
\cite{milner91polyadicpi}). We will therefore make liberal use of
$(\nu\; \vec{x})P$.

% subsection the_syntax_and_semantics_of_the_notation_system (end)   

\input{qm2pi.qmops} 

\input{qm2pi.sterngerlach} 

\input{qm2pi.metric} 

% section concurrent_process_calculi (end)

%\input{qm2pi.proofsketch}

% section proof sketch (end)

%\input{qm2pi.slviaknots} 

% section spatial logic via knots (end)

\input{qm2pi.conclusion}

% section conclusion (end)

%\input{qm2pi.dtcodes} 

% section wiring algorithm (end)

\input{qm2pi.ack} 

% section acknowledgments (end)

\newpage


\bibliographystyle{plain}   
\bibliography{../../biblios/main.bib}

\input{qm2pi.rhodetails}

\end{document}

 

% section concurrent_process_calculi (end)

%\documentclass[12pt]{llncs}
%\documentclass{jktr}

\usepackage[pdftex]{hyperref}                   
\usepackage {listings}
\usepackage {mathpartir}
\usepackage{bcprules}
%\usepackage{listings}
                       
\usepackage{graphicx} 
%\usepackage[margins=2.5cm,nohead,nofoot]{geometry}
%\usepackage{geometry}
\usepackage{amsfonts}
\usepackage{amstext}
\usepackage{latexsym}
\usepackage{amssymb}
\usepackage{color}


%\include{myPreamble}
\include{qm2pi.local} 

%\ifpdf
%\usepackage[pdftex]{graphicx}
%\else
%\usepackage{graphicx}
%\fi

 % \ifpdf
%  \usepackage{pdfsync}
%  \if


%\title{Brief Article}
%\author{David F. Snyder}
%\author{L.G. Meredith}

%\address{Dept. of Math., Texas State University--San Marcos, San Marcos, TX 78666}
       
\pagestyle{empty}


\begin{document}

\lstset{language=[Objective]Caml,frame=shadowbox}

\input{qm2pi.front}

% section front matter (end)

\input{qm2pi.intro} 
 
% section introduction (end)

% \input{qm2pi.knotations} 

% section notation (end)

\input{qm2pi.process.calculi} 

% section concurrent_process_calculi_and_spatial_logics_ (end)
    
%\input{qm2pi.knots2pi} 

%\input{qm2pi.trefoil} 

%\input{qm2pi.mainthm} 

% subsection basic_interpretation (end)

%\input{qm2pi.rho.presentation} 
\subsection{The syntax and semantics of the notation system}\label{sub:the_syntax_and_semantics_of_the_notation_system} % (fold)

We now summarize a technical presentation of the calculus that
embodies our theory of dynamics. The typical presentation of such a
calculus follows the style of giving generators and relations on
them. The grammar, below, describing term constructors, freely
generates the set of processes, $\Proc$. This set is then quotiented
by a relation known as structural congruence and it is over this set
that the notion of dynamics is expressed. This presentation is
essentially that of \cite{MeredithR05} with the addition of
polyadicity and summation. For readability we have relegated some of
the technical subtleties to an appendix.

\subsubsection{Process grammar}\label{subsub:process_grammar}

\begin{mathpar}
  \inferrule* [lab=synchronization] {} {{M} \bc \pzero \;|\; x?F \;|\; x!C }
  \and
  \inferrule* [lab=abstraction] {} {{F} \bc (x)P}
  \and
  \inferrule* [lab=concretion] {} {{C} \bc \langle Q \rangle}
  \and
  \inferrule* [lab=process] {} {{P,Q} \bc M \;| \;P|Q \;|\; @{x}}
  \and
  \inferrule* [lab=name] {} {{x} \bc \quotep{P}}
\end{mathpar} 

Note that $\vec{x}$ (resp. $\vec{P}$) denotes a vector of names
(resp. processes) of length $|\vec{x}|$ (resp. $|\vec{P}|$). We adopt
the following useful abbreviations.

\begin{mathpar}
   x?(\vec{y}).P := x.(\vec{y})P \and  x\clift{\vec{P}} := x.\clift{\vec{P}}
   \and x!(y) := \lift{x}{\dropn{y}}
   \and \Pi_{i=0}^{n-1}P_i := P_0 | \ldots | P_{n-1}
\end{mathpar}

\subsubsection{Structural congruence}

\paragraph{Free and bound names and alpha-equivalence.} At the
core of structural equivalence is alpha-equivalence which identifies
process that are the same up to a change of variable. Formally, we
recognize the distinction between free and bound names. The free names
of a process, $\freenames{P}$, may be calculated recursively as
follows:

\begin{mathpar}
\freenames{\pzero} := \emptyset
  \and \\
  \freenames{x?(y).P} := \{ x \} \cup (\freenames{P} \setminus \{ y \})
  \and 
  \freenames{x!\langle P \rangle} := \{ x \} \cup \{ P \} 
  \and \\
  \freenames{P|Q} := \freenames{P} \cup \freenames{Q}
  \and \\
  \freenames{@{x}} := \{ x \}
\end{mathpar}

$\pi$
$\quotep{\pi}$

$\freenames{-} : \pi \to \mathcal{P}(\quotep{\pi})$

\begin{eqnarray*}
  \freenames{\pzero} & := & \emptyset \\
  \freenames{x?(y).P} & := & \{ x \} \cup (\freenames{P} \setminus \{ y \}) \\
  \freenames{x!\langle P \rangle} & := & \{ x \} \cup \{ P \} \\
  \freenames{P|Q} & := & \freenames{P} \cup \freenames{Q} \\
  \freenames{\dropn{x}} & := & \{ x \}
\end{eqnarray*}

The bound names of a process, $\boundnames{P}$, are those names occurring in $P$
that are not free. For example, in $x?(y).0$, the name $x$ is free, while $y$ is bound.

\begin{mathpar}
  \inferrule* [lab=monoidal-laws] {} { P|Q \equiv Q|P \and P|0 \equiv P \and P|(Q|R) \equiv (P|Q)|R }
\end{mathpar}

\begin{mathpar}
  \inferrule* [lab=alpha-equivalence] {} { (x)P \equiv (y)P\{y/x\} \and y \not\in \freenames{P} }
\end{mathpar}

\begin{definition}
Then two processes, $P,Q$, are alpha-equivalent if $P = Q\{\vec{y}/\vec{x}\}$ for
some $\vec{x} \in \boundnames{Q},\vec{y} \in \boundnames{P}$, where $Q\{\vec{y}/\vec{x}\}$
denotes the capture-avoiding substitution of $\vec{y}$ for $\vec{x}$ in $Q$.
\end{definition}

\begin{definition}
  The {\em structural congruence} \cite{SangiorgiWalker} , $\equiv$,
  between processes is the least congruence containing
  alpha-equivalence, satisfying the abelian monoid laws
  (associativity, commutativity and $\pzero$ as identity) for parallel
  composition $|$ and for summation $+$.
\end{definition}

\subsection{Name equivalence}

We take name equivalence, written $\nameeq$, to be the smallest
equivalence relation generated by the following rules.

\begin{mathpar}
\inferrule*[lab=Quote-drop]
{ }
{ \quotep{@{x}} \nameeq x }

\inferrule*[lab=Struct-equiv]
{ P \scong Q }
{ \quotep{P} \nameeq \quotep{Q} }
\end{mathpar}

The astute reader will have noticed that the mutual recursion of names
and processes imposes a mutual recursion on alpha-equivalence and
structural equivalence via name-equivalence. Fortunately, all of this
works out pleasantly and we may calculate in the natural way, free of
concern. The reader interested in the details is referred to the
appendix \ref{appendix:rho_details}.

\subsection{Substitution}

We use $\Proc$ for the set of processes, $\QProc$ for the set of
names, and $\id{\{}\vec{y} / \vec{x} \id{\}}$ to denote partial maps,
$s : \QProc \rightarrow \QProc$. A map, $s$ lifts, uniquely, to a map
on process terms, $\widehat{s} : \Proc \rightarrow \Proc$ by the
following equations.

\begin{mathpar}
  (0) \psubstp{Q}{P} := 0 \\
  (R \juxtap S) \psubstp{Q}{P}
  :=    
  (R)\psubstp{Q}{P} \juxtap (S) \psubstp{Q}{P} \\
  (x?(y).R) \psubstp{Q}{P}    
  :=    
  (x)\substp{Q}{P} (z)\concat( (R \psubstn{z}{y}) \psubstp{Q}{P} ) \\
  (\lift{x}{R}) \psubstp{Q}{P}  
  :=
  \lift{(x)\substp{Q}{P}}{ R \psubstp{Q}{P} } \\
%   (\dropn{x})  \psubstp{Q}{P}       
%   := 
%   \left\{ 
%     \begin{array}{ccc} 
%       \dropn{\quotep{Q}} & & x \nameeq \quotep{P} \\
%       \dropn{x} & & otherwise \\
%     \end{array}
%   \right. 
  (\dropn{x})  \psubstp{Q}{P}       
  := 
  \left\{ 
    \begin{array}{ccc} 
      Q & & x \nameeq \quotep{P} \\
      \dropn{x} & & otherwise \\
    \end{array}
  \right.
\end{mathpar}
 

where

\begin{eqnarray}
  (x)\id{\{} \lpquote Q \rpquote / \lpquote P \rpquote \id{\}}            = 
  \left\{ 
    \begin{array}{ccc}
      \lpquote Q \rpquote & & x \nameeq \lpquote P \rpquote \\
      x & & otherwise \\
    \end{array}
  \right. \nonumber
\end{eqnarray}

and $z$ is chosen distinct from $\quotep{P}$, $\quotep{Q}$, the free
names in $Q$, and all the names in $R$. Our $\alpha$-equivalence will
be built in the standard way from this substitution.

\begin{remark}\label{rem:no_self_referential_names}
  One consequence of these definitions is that $\forall P. \quotep{P}
  \not\in \freenames{P}$.
\end{remark}

\subsection{ Dynamic quote: an example }

Anticipating something of what's to come, consider applying the
substitution, $\widehat{\id{\{}u / z \id{\}}}$, to the following pair
of processes, $\lift{w}{y!(z)}$ and $w[ \lpquote y!(z) \rpquote ]$.

\begin{eqnarray}
	\lift{w}{y!(z)}\widehat{\id{\{}u / z \id{\}}}
		& = &
		\lift{w}{y!(u)} \nonumber\\
	w[ \lpquote y!(z) \rpquote ] \widehat{ \id{\{}u / z \id{\}} }
		& = &
		w[ \lpquote y!(z) \rpquote ] \nonumber
\end{eqnarray}

Because the body of the process between quotes is impervious to
substitution, we get radically different answers. In fact, by
examining the first process in an input context,
e.g. $x?(z).\lift{w}{y!(z)}$, we see that the process under the lift
operator may be shaped by prefixed inputs binding a name inside it. In
this sense, the lift operator will be seen as a way to dynamically
construct processes before reifying them as names.

Finally equipped with these standard features we can present the
dynamics of the calculus.

\subsubsection{Operational semantics} 

Finally, we introduce the computational dynamics. What marks these
algebras as distinct from other more traditionally studied algebraic
structures, e.g. vector spaces or polynomial rings, is the manner in
which dynamics is captured. In traditional structures, dynamics is typically
expressed through morphisms between such structures, as in linear maps
between vector spaces or morphisms between rings. In algebras
associated with the semantics of computation, the dynamics is
expressed as part of the algebraic structure itself, through a
reduction reduction relation typically denoted by $\red$. Below, we
give a recursive presentation of this relation for the calculus used
in the encoding.

$\red \subseteq \pi \times \pi$
$\red : \pi \to \mathcal{P}(\pi)$

\begin{mathpar}
  \inferrule* [lab=Comm] { \textsf{match}( x_{src}, x_{trgt} ) } { x_{trgt}?(y)P \; | \; x_{src}!\langle {Q} \rangle \red P\{\quotep{Q}/y}\} }
  \and \\
  \inferrule* [lab=Par] {{P} \red {P}'} {{{P} | {Q}} \red {{P}' | {Q}}}
  \and
  \inferrule* [lab=Equiv]{{{P} \scong {P}'} \andalso {{P}' \red {Q}'} \andalso {{Q}' \scong {Q}}}{{P} \red {Q}}
\end{mathpar}

\begin{eqnarray*}
  match_{\equiv} (\quotep{P},\quotep{Q}) & := & P \equiv Q \\
  match_{\dagger}(\quotep{P},\quotep{Q}) & := & \forall R. P|Q \red^{*} R => R \red^{*} 0 \\
  match_{K}(\quotep{P},\quotep{Q}) & := & K \mbox{ for some context } K
\end{eqnarray*}

$u?(x)P | u!\langle Q \rangle \red P\{\quotep{Q}/x\}$

%We write $\wred$ for $\red^*$, and $P\red$ if $\exists Q $ such that $ P \red Q$.
We write $P\red$ if $\exists Q $ such that $ P \red Q$ and $P\not\red$, otherwise.

\section{Replication}

As mentioned before, it is known that replication (and hence
recursion) can be implemented in a higher-order process algebra
\cite{SangiorgiWalker}. As our first example of calculation with the
machinery thus far presented we give the construction explicitly in
the {\rhoc}.

\begin{eqnarray}
	D_{x} & := & \prefix{x}{y}{(\binpar{\outputp{x}{y}}{@{y}})} \nonumber\\
	\bangp_{x}{P} & := & \binpar{{x}!\langle{\binpar{D_{x}}{P}}\rangle}{D_{x}} \nonumber
\end{eqnarray}

\begin{eqnarray}
	\bangp_{x}{P} & & \nonumber\\
	=
	& {x}!\langle{(\prefix{x}{y}{(\outputp{x}{y} | @{y})) | P}}\rangle 
	      | \prefix{x}{y}{(\outputp{x}{y} | @{y})} & \nonumber\\
	\red
	& (\outputp{x}{y} | @{y})\substn{\quotep{(\prefix{x}{y}{(@{y} | \outputp{x}{y})) | P}}}{y} & \nonumber\\
	=
	& \outputp{x}{\quotep{(\prefix{x}{y}{(\outputp{x}{y} | @{y})) | P}}}
	  | {(\prefix{x}{y}{(\outputp{x}{y} | @{y})) | P}} & \nonumber\\
	\red
	& \ldots & \nonumber\\
	\red^*
	& P | P | \ldots & \nonumber
\end{eqnarray}

Of course, this encoding, as an implementation, runs away, unfolding
$\bangp{P}$ eagerly. A lazier and more implementable replication
operator, restricted to input-guarded processes, may be obtained as follows.

\begin{eqnarray}
\bangp{\prefix{u}{v}{P}} 
	:= 
	\binpar{\lift{x}{\prefix{u}{v}{(\binpar{D(x)}{P})}}}{D(x)} \nonumber
\end{eqnarray}

\begin{remark}
  Note that the lazier definition still does not deal with summation
  or mixed summation (i.e. sums over input and output). The reader is
  invited to construct definitions of replication that deal with these
  features. 

  Further, the definitions are parameterized in a name, $x$. Can you,
  gentle reader, make a definition that eliminates this parameter and
  guarantees no accidental interaction between the replication
  machinery and the process being replicated -- i.e. no accidental
  sharing of names used by the process to get its work done and the
  name(s) used by the replication to effect copying. This latter
  revision of the definition of replication is crucial to obtaining
  the expected identity $!!P \sim !P$.
\end{remark}

\begin{remark}\label{rem:paradoxical_combinator}
  The reader familiar with the lambda calculus will have noticed the
  similarity between $D$ and the paradoxical combinator.

  [Ed. note: the existence of this seems to suggest we have to be more
  restrictive on the set of processes and names we admit if we are to
  support no-cloning.]
\end{remark}

\subsubsection{Bisimulation}

The computational dynamics gives rise to another kind of equivalence,
the equivalence of computational behavior. As previously mentioned
this is typically captured \emph{via} some form of bisimulation.

% The notion we use in this paper is weak barbed bisimulation
% \cite{milner91polyadicpi}.

The notion we use in this paper is derived from weak barbed
bisimulation \cite{milner91polyadicpi}. 

\begin{definition}
An \emph{observation relation}, $\downarrow_{\mathcal N}$, over a set
of names, $\mathcal N$, is the smallest relation satisfying the rules
below.

\infrule[Out-barb]{y \in {\mathcal N}, \; x \nameeq y}
		  {\outputp{x}{v} \downarrow_{\mathcal N} x}
\infrule[Par-barb]{\mbox{$P\downarrow_{\mathcal N} x$ or $Q\downarrow_{\mathcal N} x$}}
		  {\binpar{P}{Q} \downarrow_{\mathcal N} x}

We write $P \Downarrow_{\mathcal N} x$ if there is $Q$ such that 
$P \wred Q$ and $Q \downarrow_{\mathcal N} x$.
\end{definition}

\begin{definition}
%\label{def.bbisim}
An  ${\mathcal N}$-\emph{barbed bisimulation} over a set of names, ${\mathcal N}$, is a symmetric binary relation 
${\mathcal S}_{\mathcal N}$ between agents such that $P\rel{S}_{\mathcal N}Q$ implies:
\begin{enumerate}
\item If $P \red P'$ then $Q \wred Q'$ and $P'\rel{S}_{\mathcal N} Q'$.
\item If $P\downarrow_{\mathcal N} x$, then $Q\Downarrow_{\mathcal N} x$.
\end{enumerate}
$P$ is ${\mathcal N}$-barbed bisimilar to $Q$, written
$P \wbbisim_{\mathcal N} Q$, if $P \rel{S}_{\mathcal N} Q$ for some ${\mathcal N}$-barbed bisimulation ${\mathcal S}_{\mathcal N}$.
\end{definition}

$\mathcal{R} \subseteq \pi \times \pi$

$P \mathcal{R} Q => \forall P'. P \red P' \Rightarrow \exists Q'. Q \red Q', P' \mathcal{R} Q'$

$P \vdash x \Rightarrow Q \vdash x$

\begin{mathpar}
  \inferrule*[lab=Out-barb]{x \nameeq y}{{y}!\langle{Q}\rangle \vdash x}
  \and
  \inferrule*[lab=Par-barb]{\mbox{$P\vdash x$ or $Q\vdash x$}}{\binpar{P}{Q} \vdash x}
\end{mathpar}

\subsubsection{Contexts}

One of the principle advantages of computational calculi like the
$\pi$-calculus is a well-defined notion of context,
contextual-equivalence and a correlation between
contextual-equivalence and notions of bisimulation. The notion of
context allows the decomposition of a process into (sub-)process and
its syntactic environment, its context. Thus, a context may be
thought of as a process with a ``hole'' (written $\Box$) in it. The
application of a context $M$ to a process $P$, written $M[P]$, is
tantamount to filling the hole in $M$ with $P$. In this paper we do
not need the full weight of this theory, but do make use of the notion
of context in the proof the main theorem. 

\begin{mathpar}
  \inferrule* [lab=summation] {} {{M_{M},M_{N}} \bc \Box \;|\; x.M_{A} \;|\; M_{M}+M_{N}}
  \and
  \inferrule* [lab=agent] {} {{M_{A}} \bc (\vec{x})M_{P} \;| \; \clift{P_0,\ldots,M_{P},\ldots,P_N}}
  \and \\
  \inferrule* [lab=process] {} {{M_{P}} \bc M_{N} \;| \;P|M_{P} }
\end{mathpar} 

\begin{mathpar}
  \inferrule* [lab=sychronization] {} {M_{N} \bc \Box \;|\; x?M_{F} \;|\; x!M_{C}}
  \and
  \inferrule* [lab=abstraction] {} {{M_{F}} \bc (x)M_{P} }
  \and
  \inferrule* [lab=concretion] {} {{M_{C}} \bc \langle M_{P} \rangle }
  \and \\
  \inferrule* [lab=process] {} {{M_{P}} \bc M_{N} \;| \;P|M_{P} }
\end{mathpar}

\begin{definition}[contextual application] Given a context $M$, and
  process $P$, we define the \emph{contextual application}, $M[P] :=
  M\{P/\Box\}$. That is, the contextual application of M to P is the
  substitution of $P$ for $\Box$ in $M$.
\end{definition}

$\meaningof{-} : L \to \mathcal{P}(\pi)$

\begin{mathpar}
  \inferrule* [lab=collection] {} {\meaningof{true} = \pi, \and \meaningof{~E} = \pi \setminus \meaningof{E}, \and \meaningof{E_{1} \& E_{2}} = \meaningof{E_{1}} \cap \meaningof{E_{2}}}
\end{mathpar}

\begin{mathpar}
  \inferrule* [lab=structure] {} {\meaningof{0} = \{ P \in \pi | P \equiv 0 \}, \and \\ \meaningof{E_1 | E_2} = \{ P \in \pi | P \equiv P_{1} | P_{2}, P_{1} \in \meaningof{E_{1}}, P_{2} \in \meaningof{E_2}\} }
\end{mathpar}

\begin{mathpar}
 \inferrule* [lab=behavior] {} {\meaningof{\langle a?b \rangle E} = \{ P \in \pi | P \equiv Q | u?(y)P', \\ \and \\\\ \and \\ \;\;\; u \in \meaningof{a}, \forall z.P'\{z/y\} \in \meaningof{E\{z/b\}}\}, \and \\ \meaningof{a!E} = \{ P \in \pi | P \equiv Q | x!\langle P' \rangle, x \in \meaningof{a} P' \in \meaningof{E}\} }
\end{mathpar}

\begin{mathpar}
 \inferrule* [lab=nominal] {} {\meaningof{\quotep{E}} = \{ \quotep{P} \in \quotep{\pi} | P \in \meaningof{E} \}, \and \meaningof{\quotep{P}} = \{ \quotep{Q} \in \quotep{\pi} | P \equiv Q \} \and \\ \meaningof{@\quotep{E}} = \{ P \in \pi | P \equiv @x, x \in \meaningof{E} \}}
\end{mathpar}

\begin{eqnarray*}
  \\
  \meaningof{-} : TS \to ST
\end{eqnarray*}

\begin{eqnarray*}
  \\
  L : TS \to ST
\end{eqnarray*}

\begin{eqnarray*}
  \\
  P \models E \iff P \in \meaningof{E}
\end{eqnarray*}

\begin{eqnarray*}
  P \approx_{L} Q \iff \forall E \in L. P \models E \iff Q \models E
\end{eqnarray*}

\begin{eqnarray*}
  P \approx_{K} Q
\end{eqnarray*}

\begin{eqnarray*}
  P \approx Q
\end{eqnarray*}

$\approx_{K} = \approx = \approx_{L}$

\subsubsection{Contextual duality}

Note that contexts extend the quotation operation to a family of
operations from processes to names. Given a context, $M$, we can
define a \emph{nominal context}, $\quotep{M}$ by $\quotep{M}[P] :=
\quotep{M[P]}$. To foreshadow what is to come we observe that these
operations enjoy a duality with processes very much like the duality
between vectors and maps from vectors to scalars.

Further, because the calculus is essentially higher-order, we have a
correspondence between contexts and processes. More specifically,
given a name $x$ and a context $M$ we can construct $M^{*}_{x}$ such
that 

\begin{mathpar}
  M^{*}_{x} | \lift{x}{P} \red M[P]
\end{mathpar}

namely,

\begin{mathpar}
  M^{*}_{x} := x?(u).M[\dropn{u}]
\end{mathpar}

The dependence of $M^{*}_{x}$ on a name makes it an abstraction, 

\begin{mathpar}
  M^{*} := (x)x?(u).M[\dropn{u}]
\end{mathpar}

\subsection{Additional notation}

It will sometimes be convenient to denote the process a name
quotes. We already have the notation $x = \quotep{P}$, but it will be
convenient to introduce an alternate notation, $\procn{x}$, when we
want to emphasize the connection to the use of the name. Note that, by
virtue of name equivalence, $\quotep{\procn{x}} \nameeq x$; so, the
notation is consistent with previous definitions.

Further, because names have structure it is possible to effect
substitutions on the basis of that structure. This means we need to
upgrade our notation for substitutions, which we accomplish by
adapting comprehension notation. Thus,

\begin{mathpar}
  P\{ y / x : x \in S \}
\end{mathpar}

is interpreted to mean the process derived from P by replacing (in a
capture-avoiding manner) each occurrence of $x$ in $S$ by $y$. For example,

\begin{mathpar}
  P\{ \quotep{\procn{x}|\procn{x}} / x : x \in \freenames{P} \}
\end{mathpar}

will replace each (occurrence) of a free name $x$ in $P$ by
$\quotep{\procn{x}|\procn{x}}$.

Also, we will avail ourselves of the notation $x^{L}$ and $x^{R}$ to
denote injections of a name into disjoint copies of the name
space. There are numerous ways to accomplish this. One example can be
found in \cite{MeredithR05}. This notation overloads to vectors of
names: $\vec{x}^{\pi} := (x_{i}^{\pi} \; : \; 0 \leq i < |\vec{x}| )$ where $\pi \in \{L,R\}$.

We also use $P^{\Box} := P|\Box$.

In \cite{MeredithR05} an interpretation of the new operator is
given. It turns out that there are several possible interpretations
all enjoying the requisite algebraic properties of the operator (see
\cite{milner91polyadicpi}). We will therefore make liberal use of
$(\nu\; \vec{x})P$.

% subsection the_syntax_and_semantics_of_the_notation_system (end)   

\input{qm2pi.qmops} 

\input{qm2pi.sterngerlach} 

\input{qm2pi.metric} 

% section concurrent_process_calculi (end)

%\input{qm2pi.proofsketch}

% section proof sketch (end)

%\input{qm2pi.slviaknots} 

% section spatial logic via knots (end)

\input{qm2pi.conclusion}

% section conclusion (end)

%\input{qm2pi.dtcodes} 

% section wiring algorithm (end)

\input{qm2pi.ack} 

% section acknowledgments (end)

\newpage


\bibliographystyle{plain}   
\bibliography{../../biblios/main.bib}

\input{qm2pi.rhodetails}

\end{document}



% section proof sketch (end)

%\section{Unlikely characters: spatial logic for
  knots}\label{sub:characteristic_formulae} % (fold)

Associated to the mobile process calculi are a family of logics known
as the Hennessy-Milner logics. These logics typically enjoy a
semantics interpreting formulae as sets of processes that when
factored through the encoding outlined above allows an identification
of classes of knots with logical formulae. In the context of this
encoding the sub-family known as the spatial logics \cite{CairesC03}
\cite{CairesC04} \cite{Caires04} are of particular interest providing
several important features for expressing and reasoning about
properties (i.e. classes) of knots. We hint here at how this may be done.

%\begin{description}
%\item [structural connectives] 
\subsubsection{Structural connectives} The spatial logics enjoy
structural connectives corresponding, at the logical level, to the
parallel composition ($P | Q$) and new name ($(\nu \; x)P$)
connectives for processes. As illustrated in the examples below, these
connectives are extremely expressive given the shape of our encoding.
%\item [decideable satisfaction]

\subsubsection{Decideable satisfaction}
In \cite{Caires04} the satisfaction relation is shown to be decideable
for a rich class of processes. It further turns out that the image of
the our encoding is a proper subset of that class. This result
provides the basis for an algorithm by which to search for knots
enjoying a given property.
%\item [characteristic formulae]

\subsubsection{Characteristic formulae}
In the same paper \cite{Caires04} , Caires presents a means of calculating
characteristic formulae, selecting equivalence classes of processes
up to a pre--specified depth limit on the support set of names. Composed with our
encoding, this characteristic formula can be used to select
characteristic formulae for knots.
%\end{description}

\subsubsection{Spatial logic formulae}

The grammar below (segmented for comprehension) summarizes the syntax
of spatial logic formulae. We employ illustrative examples in the
sequel to provide an intuitive understanding of their meaning
referring the reader to \cite{Caires04} for a more detailed explication
of the semantics.

\begin{mathpar}
  \inferrule* [lab=boolean] {} {{A,B} \bc T \;|\; \neg A \;|\; A \wedge B \;|\; \eta = \eta'}
  \and
  \inferrule* [lab=spatial] {} {|\; \pzero \;|\; A | B \;|\; x \text{\textregistered} A \;|\; \forall x . A \;|\;  H x . A}
  \and
  \inferrule* [lab=behavioral] {} {|\; \alpha . A}
  \and 
  \inferrule* [lab=recursion] {} {|\; X(\vec{u}) \;|\; \mu X(\vec{u}) . A}
  \and
  \inferrule* [lab=action] {} {\alpha \bc \langle x?(\vec{y}) \rangle \;|\; \langle x!(\vec{y}) \rangle \;|\; \langle \tau \rangle}
  \and 
  \inferrule* [lab=name] {} {\eta \bc x \;|\; \tau}
\end{mathpar} 

% subsection characteristic_formulae (end)   	 

\subsection{Example formulae}\label{sub:example_formulae_} % (fold)

\subsubsection{Crossing as formula.}
% 
% \begin{align*}
%   \frac{d}{dx} \sin x &= \cos x 
%   & \frac{d}{dx} e^x &= e^x \\
%   \frac{d}{dx} \cos x &= - \sin x 
%   & \frac{d}{dx} \log x &= \frac{1}{x} \\
% \end{align*} 

\begin{align*}
 \mu C(x_{0},x_{1},y_{0},y_{1},u).&(\langle x_{0}?(z) \rangle(\langle u! \rangle\langle y_{1}!z \rangle C(x_{0},x_{1},y_{0},y_{1},u)) & \\
  & \wedge \langle y_{1}?(z) \rangle (\langle u! \rangle \langle x_{0}!z \rangle C(x_{0},x_{1},y_{0},y_{1},u)) & \\
  & \wedge \langle x_{1}?(z) \rangle (\langle u? \rangle \langle y_{0}!z \rangle C(x_{0},x_{1},y_{0},y_{1},u)) & \\
  & \wedge \langle y_{0}?(z) \rangle (\langle u? \rangle \langle x_{1}!z \rangle C(x_{0},x_{1},y_{0},y_{1},u))) &
\end{align*}

The lexicographical similarity between the shape of this formulae and
the shape of definition of the process representing a crossing reveals
the intuitive meaning of this formulae. It describes the capabilities
of a process that has the right to represent a crossing. For example
it picks out processes that may perform an input on the port $x_0$ in
its initial menu of capabilities. What differentiates the formula
from the process, however, is that the crossing process is the
smallest candidate to satisfy the formula. Infinitely many other
processes -- with internal behavior hidden behind this interface, so
to speak -- also satisfy this formula. Even this simple formula,
then, can be seen to open a new view onto knots, providing a
computational interpretation of \emph{virtual} knots.

Note that this formula is derived by hand. A similar formula can be
derived by employing Caires' calculation of characteristic formula
\cite{Caires04} to the process representing a crossing. In light of
this discussion, we let
$\meaningof{C}_{\phi}(x0,x1,y0,y1,u)$ denote a formula specifying the
dynamics we wish to capture of a crossing. To guarantee we preserve
the shape of the interface and minimal semantics we demand that
$\meaningof{C}_{\phi}(x0,x1,y0,y1,u) \Rightarrow
\textbf{C}(x0,x1,y0,y1,u)$ where $\textbf{C}(x0,x1,y0,y1,u)$ denotes
the formula above.
                            
\subsubsection{Crossing number constraints.}
The moral content of the context lemma (Lemma \ref{context}) is that the notion of
``locality'' in the Reidemeister moves is effectively captured by the
parallel composition operator of the process calculus. This intuition
extends through the logic. Given a formula,
$\meaningof{C}_{\phi}(x0,x1,y0,y1,u)$, we can use the structural
connectives to specify constraints on crossing numbers, such as at
least $n$ crossings, or exactly $n$ crossings.
\begin{mathpar}
  \inferrule* [lab=at-least-n] {} { K^{\geq n}_{\phi}(\vec{xs},\vec{ys}) := \Pi_{i=0}^{n-1} Hu . \meaningof{C}_{\phi}(xs_i,ys_i,u) | T }
  \and 
  \inferrule* [lab=exactly-n] {} { K^{= n}_{\phi}(\vec{xs},\vec{ys}) := \Pi_{i=0}^{n-1} Hu . \meaningof{C}_{\phi}(xs_i,ys_i,u) | \neg (\forall x_0,y_0,x_1,y_1,u . \meaningof{C}_{\phi}(x_0,y_0,x_1,y_1,u) | T) }
\end{mathpar}

To round out this section, recall that the encoding of an $n$-crossing
knot decomposes into a parallel composition of $n$ \emph{copies} of a
crossing process together with a wiring harness. To specify different
knot classes with the same crossing number amounts to specifying
logical constraints on the wiring harness. In the interest of space,
we defer examples to a forthcoming paper. Suffice it to say that both
the conditions ``alternating knot'' and ``contains the tangle
corresponding to 5/3'' are expressible. For example, it is possible to
calculate the characteristic formula of a process corresponding to the
tangle 5/3 and conjoin it into the classifying formula via the
composition connective of the logic.

Finally, we wish to observe that it is entirely within reason to
contemplate a more domain-specific version of spatial logic tailored
to the shape of processes in the image of the encoding. Such a
domain-specific logic would have a better claim to the title formal
language of knot properties.

% subsection example_formulae_ (end)

% section knots_as_processes (end) 

% section spatial logic via knots (end)

\section{Conclusions and future work}

\paragraph{Testing physical space}
You, gentle reader, may wonder why of all the theorems to be proved
given this set up we pick the one above. In some sense it's hardly
central to quantum mechanics. We see it as central in the sense that
it firmly establishes a notion of physical space arising from a notion
of the equivalence of behavior. Relating bisimulation to a metric is a
big step forward, but one is faced with interpreting the relationship
of that metric space to something more physical. Quantum mechanical
notions of ``physical'' space are still far from intuitive, but by
relating this idea of distance as testing to calculations that predict
physical circumstances we are making a not insignificant step forward
toward an understanding of the physical space we inhabit as
essentially dynamic.

\paragraph{Effectivity and simulation}
One of the observations we have yet to make is that the entire program
spelled out here is effective. We have built various interpreters for
the reflective calculus at work in this interpretation. In principle,
then, we can simulate quantum mechanics on a computer. The place where
the simulation may lose fidelity is the infinitely branching summation
for the annihilator.

In this connection i also want to point out that the evaluation style
calculation of the inner product puts the non-determinism of the
summation right at the heart of measurement. This suggests that
Milner's original reduction-based formulation of the dynamics of his
calculi in terms of sums was not just notationally suggestive of a
notion of measure-and-continue but captured some significant part of
the physics.

\paragraph{Quantum continuations}
In light of this last observation i want to point out that the
predominant account of quantum mechanics is missing a key aspect of a
truly compositional story of the physical situation. In a real lab,
when a measurement is made the observation can be made to feed into
another device that then makes another measurement conditioned on the
results of the first. This means that after the superposition was
collapsed the entire experimental set up remained in
superposition. While QM offers a means of writing this down it doesn't
quite line up well with the well-trodden formulation of computation
and continuation that we see so succinctly expressed in Milner's
calculi. This suggests that there might be advantages to this account
of dynamics waiting to be explored.

\paragraph{Quantum logic}
In this connection, we also note that by virtue of having the
Hennessy-Milner construction, we can pull the construction through the
interpretation of QM. This gives us a natural candidate for a quantum
logic that enjoys an extremely tight connection with it's domain of
interpretation, making the construction much less ad hoc (rather it is
the image of functor!).

\paragraph{Quantum probabiity}
i have questions about the basis of the interpretation of inner
product as probability amplitude. In particular, using which
axiomatization of probability theory does the notion of probability
amplitude earn the right to be so dubbed? In other words, where is the
proof that the operation for calculating a probability amplitude (and
then squaring) satisfies the axioms of what it means to calculate a
probability? Even if such a proof exists (i have yet to find it in the
literature), i wonder if it might not be possible to turn things on
their heads. Can we view the calculation of the probability amplitude
as an axiomatization of probability? If so, then the definition we
give for calculating probability amplitude may provide the basis for
an \emph{effective} theory of probability.

\paragraph{Quantum vs ``biological'' information}
Finally, i want to conclude with a more philosophical observation. At
a recent workshop in which QM was a predominant topic i noticed
something about quantum information. The speaker was giving a riveting
discussion of axiomatic QM and showing how properties of ``no
cloning'' and ``no deleting'' emerged as consequences of the
axiomatization. Theorems of this form are necessary to give us a sense
of confidence that our axioms characterize the physical theory. What
struck me, though, was that if quantum information is neither erasable
nor replicable it is markedly different from \emph{life}. Two of the
things we know about life is that

\begin{itemize}
  \item it ends;
  \item to gain some measure of persistence, to transcend it's
    finitude it is imminently copyable.
\end{itemize}

Both of these qualities are summarized succinctly in the aphorism: all
flesh is grass. For me these two kinds of ``information'' -- call them
quantum and biological -- are end points on a spectrum of strategies
for persistence. At one end, we have those curious entities that enjoy
uniqueness and permanence; at the other, we have those who in the face
of a certain end and an uncertain present make a go of passing
something on. To me one of the more remarkable aspects of the latter
strategy is that in the presence of noise (and certain features of
copying) we get a kind of dynamism, a chance for improvement against a
given persistent condition.

% subsection other_calculi_other_bisimulations_and_geometry_as_behavior (end)




% section conclusion (end)

%\documentclass[12pt]{llncs}
%\documentclass{jktr}

\usepackage[pdftex]{hyperref}                   
\usepackage {listings}
\usepackage {mathpartir}
\usepackage{bcprules}
%\usepackage{listings}
                       
\usepackage{graphicx} 
%\usepackage[margins=2.5cm,nohead,nofoot]{geometry}
%\usepackage{geometry}
\usepackage{amsfonts}
\usepackage{amstext}
\usepackage{latexsym}
\usepackage{amssymb}
\usepackage{color}


%\include{myPreamble}
\include{qm2pi.local} 

%\ifpdf
%\usepackage[pdftex]{graphicx}
%\else
%\usepackage{graphicx}
%\fi

 % \ifpdf
%  \usepackage{pdfsync}
%  \if


%\title{Brief Article}
%\author{David F. Snyder}
%\author{L.G. Meredith}

%\address{Dept. of Math., Texas State University--San Marcos, San Marcos, TX 78666}
       
\pagestyle{empty}


\begin{document}

\lstset{language=[Objective]Caml,frame=shadowbox}

\input{qm2pi.front}

% section front matter (end)

\input{qm2pi.intro} 
 
% section introduction (end)

% \input{qm2pi.knotations} 

% section notation (end)

\input{qm2pi.process.calculi} 

% section concurrent_process_calculi_and_spatial_logics_ (end)
    
%\input{qm2pi.knots2pi} 

%\input{qm2pi.trefoil} 

%\input{qm2pi.mainthm} 

% subsection basic_interpretation (end)

%\input{qm2pi.rho.presentation} 
\subsection{The syntax and semantics of the notation system}\label{sub:the_syntax_and_semantics_of_the_notation_system} % (fold)

We now summarize a technical presentation of the calculus that
embodies our theory of dynamics. The typical presentation of such a
calculus follows the style of giving generators and relations on
them. The grammar, below, describing term constructors, freely
generates the set of processes, $\Proc$. This set is then quotiented
by a relation known as structural congruence and it is over this set
that the notion of dynamics is expressed. This presentation is
essentially that of \cite{MeredithR05} with the addition of
polyadicity and summation. For readability we have relegated some of
the technical subtleties to an appendix.

\subsubsection{Process grammar}\label{subsub:process_grammar}

\begin{mathpar}
  \inferrule* [lab=synchronization] {} {{M} \bc \pzero \;|\; x?F \;|\; x!C }
  \and
  \inferrule* [lab=abstraction] {} {{F} \bc (x)P}
  \and
  \inferrule* [lab=concretion] {} {{C} \bc \langle Q \rangle}
  \and
  \inferrule* [lab=process] {} {{P,Q} \bc M \;| \;P|Q \;|\; @{x}}
  \and
  \inferrule* [lab=name] {} {{x} \bc \quotep{P}}
\end{mathpar} 

Note that $\vec{x}$ (resp. $\vec{P}$) denotes a vector of names
(resp. processes) of length $|\vec{x}|$ (resp. $|\vec{P}|$). We adopt
the following useful abbreviations.

\begin{mathpar}
   x?(\vec{y}).P := x.(\vec{y})P \and  x\clift{\vec{P}} := x.\clift{\vec{P}}
   \and x!(y) := \lift{x}{\dropn{y}}
   \and \Pi_{i=0}^{n-1}P_i := P_0 | \ldots | P_{n-1}
\end{mathpar}

\subsubsection{Structural congruence}

\paragraph{Free and bound names and alpha-equivalence.} At the
core of structural equivalence is alpha-equivalence which identifies
process that are the same up to a change of variable. Formally, we
recognize the distinction between free and bound names. The free names
of a process, $\freenames{P}$, may be calculated recursively as
follows:

\begin{mathpar}
\freenames{\pzero} := \emptyset
  \and \\
  \freenames{x?(y).P} := \{ x \} \cup (\freenames{P} \setminus \{ y \})
  \and 
  \freenames{x!\langle P \rangle} := \{ x \} \cup \{ P \} 
  \and \\
  \freenames{P|Q} := \freenames{P} \cup \freenames{Q}
  \and \\
  \freenames{@{x}} := \{ x \}
\end{mathpar}

$\pi$
$\quotep{\pi}$

$\freenames{-} : \pi \to \mathcal{P}(\quotep{\pi})$

\begin{eqnarray*}
  \freenames{\pzero} & := & \emptyset \\
  \freenames{x?(y).P} & := & \{ x \} \cup (\freenames{P} \setminus \{ y \}) \\
  \freenames{x!\langle P \rangle} & := & \{ x \} \cup \{ P \} \\
  \freenames{P|Q} & := & \freenames{P} \cup \freenames{Q} \\
  \freenames{\dropn{x}} & := & \{ x \}
\end{eqnarray*}

The bound names of a process, $\boundnames{P}$, are those names occurring in $P$
that are not free. For example, in $x?(y).0$, the name $x$ is free, while $y$ is bound.

\begin{mathpar}
  \inferrule* [lab=monoidal-laws] {} { P|Q \equiv Q|P \and P|0 \equiv P \and P|(Q|R) \equiv (P|Q)|R }
\end{mathpar}

\begin{mathpar}
  \inferrule* [lab=alpha-equivalence] {} { (x)P \equiv (y)P\{y/x\} \and y \not\in \freenames{P} }
\end{mathpar}

\begin{definition}
Then two processes, $P,Q$, are alpha-equivalent if $P = Q\{\vec{y}/\vec{x}\}$ for
some $\vec{x} \in \boundnames{Q},\vec{y} \in \boundnames{P}$, where $Q\{\vec{y}/\vec{x}\}$
denotes the capture-avoiding substitution of $\vec{y}$ for $\vec{x}$ in $Q$.
\end{definition}

\begin{definition}
  The {\em structural congruence} \cite{SangiorgiWalker} , $\equiv$,
  between processes is the least congruence containing
  alpha-equivalence, satisfying the abelian monoid laws
  (associativity, commutativity and $\pzero$ as identity) for parallel
  composition $|$ and for summation $+$.
\end{definition}

\subsection{Name equivalence}

We take name equivalence, written $\nameeq$, to be the smallest
equivalence relation generated by the following rules.

\begin{mathpar}
\inferrule*[lab=Quote-drop]
{ }
{ \quotep{@{x}} \nameeq x }

\inferrule*[lab=Struct-equiv]
{ P \scong Q }
{ \quotep{P} \nameeq \quotep{Q} }
\end{mathpar}

The astute reader will have noticed that the mutual recursion of names
and processes imposes a mutual recursion on alpha-equivalence and
structural equivalence via name-equivalence. Fortunately, all of this
works out pleasantly and we may calculate in the natural way, free of
concern. The reader interested in the details is referred to the
appendix \ref{appendix:rho_details}.

\subsection{Substitution}

We use $\Proc$ for the set of processes, $\QProc$ for the set of
names, and $\id{\{}\vec{y} / \vec{x} \id{\}}$ to denote partial maps,
$s : \QProc \rightarrow \QProc$. A map, $s$ lifts, uniquely, to a map
on process terms, $\widehat{s} : \Proc \rightarrow \Proc$ by the
following equations.

\begin{mathpar}
  (0) \psubstp{Q}{P} := 0 \\
  (R \juxtap S) \psubstp{Q}{P}
  :=    
  (R)\psubstp{Q}{P} \juxtap (S) \psubstp{Q}{P} \\
  (x?(y).R) \psubstp{Q}{P}    
  :=    
  (x)\substp{Q}{P} (z)\concat( (R \psubstn{z}{y}) \psubstp{Q}{P} ) \\
  (\lift{x}{R}) \psubstp{Q}{P}  
  :=
  \lift{(x)\substp{Q}{P}}{ R \psubstp{Q}{P} } \\
%   (\dropn{x})  \psubstp{Q}{P}       
%   := 
%   \left\{ 
%     \begin{array}{ccc} 
%       \dropn{\quotep{Q}} & & x \nameeq \quotep{P} \\
%       \dropn{x} & & otherwise \\
%     \end{array}
%   \right. 
  (\dropn{x})  \psubstp{Q}{P}       
  := 
  \left\{ 
    \begin{array}{ccc} 
      Q & & x \nameeq \quotep{P} \\
      \dropn{x} & & otherwise \\
    \end{array}
  \right.
\end{mathpar}
 

where

\begin{eqnarray}
  (x)\id{\{} \lpquote Q \rpquote / \lpquote P \rpquote \id{\}}            = 
  \left\{ 
    \begin{array}{ccc}
      \lpquote Q \rpquote & & x \nameeq \lpquote P \rpquote \\
      x & & otherwise \\
    \end{array}
  \right. \nonumber
\end{eqnarray}

and $z$ is chosen distinct from $\quotep{P}$, $\quotep{Q}$, the free
names in $Q$, and all the names in $R$. Our $\alpha$-equivalence will
be built in the standard way from this substitution.

\begin{remark}\label{rem:no_self_referential_names}
  One consequence of these definitions is that $\forall P. \quotep{P}
  \not\in \freenames{P}$.
\end{remark}

\subsection{ Dynamic quote: an example }

Anticipating something of what's to come, consider applying the
substitution, $\widehat{\id{\{}u / z \id{\}}}$, to the following pair
of processes, $\lift{w}{y!(z)}$ and $w[ \lpquote y!(z) \rpquote ]$.

\begin{eqnarray}
	\lift{w}{y!(z)}\widehat{\id{\{}u / z \id{\}}}
		& = &
		\lift{w}{y!(u)} \nonumber\\
	w[ \lpquote y!(z) \rpquote ] \widehat{ \id{\{}u / z \id{\}} }
		& = &
		w[ \lpquote y!(z) \rpquote ] \nonumber
\end{eqnarray}

Because the body of the process between quotes is impervious to
substitution, we get radically different answers. In fact, by
examining the first process in an input context,
e.g. $x?(z).\lift{w}{y!(z)}$, we see that the process under the lift
operator may be shaped by prefixed inputs binding a name inside it. In
this sense, the lift operator will be seen as a way to dynamically
construct processes before reifying them as names.

Finally equipped with these standard features we can present the
dynamics of the calculus.

\subsubsection{Operational semantics} 

Finally, we introduce the computational dynamics. What marks these
algebras as distinct from other more traditionally studied algebraic
structures, e.g. vector spaces or polynomial rings, is the manner in
which dynamics is captured. In traditional structures, dynamics is typically
expressed through morphisms between such structures, as in linear maps
between vector spaces or morphisms between rings. In algebras
associated with the semantics of computation, the dynamics is
expressed as part of the algebraic structure itself, through a
reduction reduction relation typically denoted by $\red$. Below, we
give a recursive presentation of this relation for the calculus used
in the encoding.

$\red \subseteq \pi \times \pi$
$\red : \pi \to \mathcal{P}(\pi)$

\begin{mathpar}
  \inferrule* [lab=Comm] { \textsf{match}( x_{src}, x_{trgt} ) } { x_{trgt}?(y)P \; | \; x_{src}!\langle {Q} \rangle \red P\{\quotep{Q}/y}\} }
  \and \\
  \inferrule* [lab=Par] {{P} \red {P}'} {{{P} | {Q}} \red {{P}' | {Q}}}
  \and
  \inferrule* [lab=Equiv]{{{P} \scong {P}'} \andalso {{P}' \red {Q}'} \andalso {{Q}' \scong {Q}}}{{P} \red {Q}}
\end{mathpar}

\begin{eqnarray*}
  match_{\equiv} (\quotep{P},\quotep{Q}) & := & P \equiv Q \\
  match_{\dagger}(\quotep{P},\quotep{Q}) & := & \forall R. P|Q \red^{*} R => R \red^{*} 0 \\
  match_{K}(\quotep{P},\quotep{Q}) & := & K \mbox{ for some context } K
\end{eqnarray*}

$u?(x)P | u!\langle Q \rangle \red P\{\quotep{Q}/x\}$

%We write $\wred$ for $\red^*$, and $P\red$ if $\exists Q $ such that $ P \red Q$.
We write $P\red$ if $\exists Q $ such that $ P \red Q$ and $P\not\red$, otherwise.

\section{Replication}

As mentioned before, it is known that replication (and hence
recursion) can be implemented in a higher-order process algebra
\cite{SangiorgiWalker}. As our first example of calculation with the
machinery thus far presented we give the construction explicitly in
the {\rhoc}.

\begin{eqnarray}
	D_{x} & := & \prefix{x}{y}{(\binpar{\outputp{x}{y}}{@{y}})} \nonumber\\
	\bangp_{x}{P} & := & \binpar{{x}!\langle{\binpar{D_{x}}{P}}\rangle}{D_{x}} \nonumber
\end{eqnarray}

\begin{eqnarray}
	\bangp_{x}{P} & & \nonumber\\
	=
	& {x}!\langle{(\prefix{x}{y}{(\outputp{x}{y} | @{y})) | P}}\rangle 
	      | \prefix{x}{y}{(\outputp{x}{y} | @{y})} & \nonumber\\
	\red
	& (\outputp{x}{y} | @{y})\substn{\quotep{(\prefix{x}{y}{(@{y} | \outputp{x}{y})) | P}}}{y} & \nonumber\\
	=
	& \outputp{x}{\quotep{(\prefix{x}{y}{(\outputp{x}{y} | @{y})) | P}}}
	  | {(\prefix{x}{y}{(\outputp{x}{y} | @{y})) | P}} & \nonumber\\
	\red
	& \ldots & \nonumber\\
	\red^*
	& P | P | \ldots & \nonumber
\end{eqnarray}

Of course, this encoding, as an implementation, runs away, unfolding
$\bangp{P}$ eagerly. A lazier and more implementable replication
operator, restricted to input-guarded processes, may be obtained as follows.

\begin{eqnarray}
\bangp{\prefix{u}{v}{P}} 
	:= 
	\binpar{\lift{x}{\prefix{u}{v}{(\binpar{D(x)}{P})}}}{D(x)} \nonumber
\end{eqnarray}

\begin{remark}
  Note that the lazier definition still does not deal with summation
  or mixed summation (i.e. sums over input and output). The reader is
  invited to construct definitions of replication that deal with these
  features. 

  Further, the definitions are parameterized in a name, $x$. Can you,
  gentle reader, make a definition that eliminates this parameter and
  guarantees no accidental interaction between the replication
  machinery and the process being replicated -- i.e. no accidental
  sharing of names used by the process to get its work done and the
  name(s) used by the replication to effect copying. This latter
  revision of the definition of replication is crucial to obtaining
  the expected identity $!!P \sim !P$.
\end{remark}

\begin{remark}\label{rem:paradoxical_combinator}
  The reader familiar with the lambda calculus will have noticed the
  similarity between $D$ and the paradoxical combinator.

  [Ed. note: the existence of this seems to suggest we have to be more
  restrictive on the set of processes and names we admit if we are to
  support no-cloning.]
\end{remark}

\subsubsection{Bisimulation}

The computational dynamics gives rise to another kind of equivalence,
the equivalence of computational behavior. As previously mentioned
this is typically captured \emph{via} some form of bisimulation.

% The notion we use in this paper is weak barbed bisimulation
% \cite{milner91polyadicpi}.

The notion we use in this paper is derived from weak barbed
bisimulation \cite{milner91polyadicpi}. 

\begin{definition}
An \emph{observation relation}, $\downarrow_{\mathcal N}$, over a set
of names, $\mathcal N$, is the smallest relation satisfying the rules
below.

\infrule[Out-barb]{y \in {\mathcal N}, \; x \nameeq y}
		  {\outputp{x}{v} \downarrow_{\mathcal N} x}
\infrule[Par-barb]{\mbox{$P\downarrow_{\mathcal N} x$ or $Q\downarrow_{\mathcal N} x$}}
		  {\binpar{P}{Q} \downarrow_{\mathcal N} x}

We write $P \Downarrow_{\mathcal N} x$ if there is $Q$ such that 
$P \wred Q$ and $Q \downarrow_{\mathcal N} x$.
\end{definition}

\begin{definition}
%\label{def.bbisim}
An  ${\mathcal N}$-\emph{barbed bisimulation} over a set of names, ${\mathcal N}$, is a symmetric binary relation 
${\mathcal S}_{\mathcal N}$ between agents such that $P\rel{S}_{\mathcal N}Q$ implies:
\begin{enumerate}
\item If $P \red P'$ then $Q \wred Q'$ and $P'\rel{S}_{\mathcal N} Q'$.
\item If $P\downarrow_{\mathcal N} x$, then $Q\Downarrow_{\mathcal N} x$.
\end{enumerate}
$P$ is ${\mathcal N}$-barbed bisimilar to $Q$, written
$P \wbbisim_{\mathcal N} Q$, if $P \rel{S}_{\mathcal N} Q$ for some ${\mathcal N}$-barbed bisimulation ${\mathcal S}_{\mathcal N}$.
\end{definition}

$\mathcal{R} \subseteq \pi \times \pi$

$P \mathcal{R} Q => \forall P'. P \red P' \Rightarrow \exists Q'. Q \red Q', P' \mathcal{R} Q'$

$P \vdash x \Rightarrow Q \vdash x$

\begin{mathpar}
  \inferrule*[lab=Out-barb]{x \nameeq y}{{y}!\langle{Q}\rangle \vdash x}
  \and
  \inferrule*[lab=Par-barb]{\mbox{$P\vdash x$ or $Q\vdash x$}}{\binpar{P}{Q} \vdash x}
\end{mathpar}

\subsubsection{Contexts}

One of the principle advantages of computational calculi like the
$\pi$-calculus is a well-defined notion of context,
contextual-equivalence and a correlation between
contextual-equivalence and notions of bisimulation. The notion of
context allows the decomposition of a process into (sub-)process and
its syntactic environment, its context. Thus, a context may be
thought of as a process with a ``hole'' (written $\Box$) in it. The
application of a context $M$ to a process $P$, written $M[P]$, is
tantamount to filling the hole in $M$ with $P$. In this paper we do
not need the full weight of this theory, but do make use of the notion
of context in the proof the main theorem. 

\begin{mathpar}
  \inferrule* [lab=summation] {} {{M_{M},M_{N}} \bc \Box \;|\; x.M_{A} \;|\; M_{M}+M_{N}}
  \and
  \inferrule* [lab=agent] {} {{M_{A}} \bc (\vec{x})M_{P} \;| \; \clift{P_0,\ldots,M_{P},\ldots,P_N}}
  \and \\
  \inferrule* [lab=process] {} {{M_{P}} \bc M_{N} \;| \;P|M_{P} }
\end{mathpar} 

\begin{mathpar}
  \inferrule* [lab=sychronization] {} {M_{N} \bc \Box \;|\; x?M_{F} \;|\; x!M_{C}}
  \and
  \inferrule* [lab=abstraction] {} {{M_{F}} \bc (x)M_{P} }
  \and
  \inferrule* [lab=concretion] {} {{M_{C}} \bc \langle M_{P} \rangle }
  \and \\
  \inferrule* [lab=process] {} {{M_{P}} \bc M_{N} \;| \;P|M_{P} }
\end{mathpar}

\begin{definition}[contextual application] Given a context $M$, and
  process $P$, we define the \emph{contextual application}, $M[P] :=
  M\{P/\Box\}$. That is, the contextual application of M to P is the
  substitution of $P$ for $\Box$ in $M$.
\end{definition}

$\meaningof{-} : L \to \mathcal{P}(\pi)$

\begin{mathpar}
  \inferrule* [lab=collection] {} {\meaningof{true} = \pi, \and \meaningof{~E} = \pi \setminus \meaningof{E}, \and \meaningof{E_{1} \& E_{2}} = \meaningof{E_{1}} \cap \meaningof{E_{2}}}
\end{mathpar}

\begin{mathpar}
  \inferrule* [lab=structure] {} {\meaningof{0} = \{ P \in \pi | P \equiv 0 \}, \and \\ \meaningof{E_1 | E_2} = \{ P \in \pi | P \equiv P_{1} | P_{2}, P_{1} \in \meaningof{E_{1}}, P_{2} \in \meaningof{E_2}\} }
\end{mathpar}

\begin{mathpar}
 \inferrule* [lab=behavior] {} {\meaningof{\langle a?b \rangle E} = \{ P \in \pi | P \equiv Q | u?(y)P', \\ \and \\\\ \and \\ \;\;\; u \in \meaningof{a}, \forall z.P'\{z/y\} \in \meaningof{E\{z/b\}}\}, \and \\ \meaningof{a!E} = \{ P \in \pi | P \equiv Q | x!\langle P' \rangle, x \in \meaningof{a} P' \in \meaningof{E}\} }
\end{mathpar}

\begin{mathpar}
 \inferrule* [lab=nominal] {} {\meaningof{\quotep{E}} = \{ \quotep{P} \in \quotep{\pi} | P \in \meaningof{E} \}, \and \meaningof{\quotep{P}} = \{ \quotep{Q} \in \quotep{\pi} | P \equiv Q \} \and \\ \meaningof{@\quotep{E}} = \{ P \in \pi | P \equiv @x, x \in \meaningof{E} \}}
\end{mathpar}

\begin{eqnarray*}
  \\
  \meaningof{-} : TS \to ST
\end{eqnarray*}

\begin{eqnarray*}
  \\
  L : TS \to ST
\end{eqnarray*}

\begin{eqnarray*}
  \\
  P \models E \iff P \in \meaningof{E}
\end{eqnarray*}

\begin{eqnarray*}
  P \approx_{L} Q \iff \forall E \in L. P \models E \iff Q \models E
\end{eqnarray*}

\begin{eqnarray*}
  P \approx_{K} Q
\end{eqnarray*}

\begin{eqnarray*}
  P \approx Q
\end{eqnarray*}

$\approx_{K} = \approx = \approx_{L}$

\subsubsection{Contextual duality}

Note that contexts extend the quotation operation to a family of
operations from processes to names. Given a context, $M$, we can
define a \emph{nominal context}, $\quotep{M}$ by $\quotep{M}[P] :=
\quotep{M[P]}$. To foreshadow what is to come we observe that these
operations enjoy a duality with processes very much like the duality
between vectors and maps from vectors to scalars.

Further, because the calculus is essentially higher-order, we have a
correspondence between contexts and processes. More specifically,
given a name $x$ and a context $M$ we can construct $M^{*}_{x}$ such
that 

\begin{mathpar}
  M^{*}_{x} | \lift{x}{P} \red M[P]
\end{mathpar}

namely,

\begin{mathpar}
  M^{*}_{x} := x?(u).M[\dropn{u}]
\end{mathpar}

The dependence of $M^{*}_{x}$ on a name makes it an abstraction, 

\begin{mathpar}
  M^{*} := (x)x?(u).M[\dropn{u}]
\end{mathpar}

\subsection{Additional notation}

It will sometimes be convenient to denote the process a name
quotes. We already have the notation $x = \quotep{P}$, but it will be
convenient to introduce an alternate notation, $\procn{x}$, when we
want to emphasize the connection to the use of the name. Note that, by
virtue of name equivalence, $\quotep{\procn{x}} \nameeq x$; so, the
notation is consistent with previous definitions.

Further, because names have structure it is possible to effect
substitutions on the basis of that structure. This means we need to
upgrade our notation for substitutions, which we accomplish by
adapting comprehension notation. Thus,

\begin{mathpar}
  P\{ y / x : x \in S \}
\end{mathpar}

is interpreted to mean the process derived from P by replacing (in a
capture-avoiding manner) each occurrence of $x$ in $S$ by $y$. For example,

\begin{mathpar}
  P\{ \quotep{\procn{x}|\procn{x}} / x : x \in \freenames{P} \}
\end{mathpar}

will replace each (occurrence) of a free name $x$ in $P$ by
$\quotep{\procn{x}|\procn{x}}$.

Also, we will avail ourselves of the notation $x^{L}$ and $x^{R}$ to
denote injections of a name into disjoint copies of the name
space. There are numerous ways to accomplish this. One example can be
found in \cite{MeredithR05}. This notation overloads to vectors of
names: $\vec{x}^{\pi} := (x_{i}^{\pi} \; : \; 0 \leq i < |\vec{x}| )$ where $\pi \in \{L,R\}$.

We also use $P^{\Box} := P|\Box$.

In \cite{MeredithR05} an interpretation of the new operator is
given. It turns out that there are several possible interpretations
all enjoying the requisite algebraic properties of the operator (see
\cite{milner91polyadicpi}). We will therefore make liberal use of
$(\nu\; \vec{x})P$.

% subsection the_syntax_and_semantics_of_the_notation_system (end)   

\input{qm2pi.qmops} 

\input{qm2pi.sterngerlach} 

\input{qm2pi.metric} 

% section concurrent_process_calculi (end)

%\input{qm2pi.proofsketch}

% section proof sketch (end)

%\input{qm2pi.slviaknots} 

% section spatial logic via knots (end)

\input{qm2pi.conclusion}

% section conclusion (end)

%\input{qm2pi.dtcodes} 

% section wiring algorithm (end)

\input{qm2pi.ack} 

% section acknowledgments (end)

\newpage


\bibliographystyle{plain}   
\bibliography{../../biblios/main.bib}

\input{qm2pi.rhodetails}

\end{document}

 

% section wiring algorithm (end)

\documentclass[12pt]{llncs}
%\documentclass{jktr}

\usepackage[pdftex]{hyperref}                   
\usepackage {listings}
\usepackage {mathpartir}
\usepackage{bcprules}
%\usepackage{listings}
                       
\usepackage{graphicx} 
%\usepackage[margins=2.5cm,nohead,nofoot]{geometry}
%\usepackage{geometry}
\usepackage{amsfonts}
\usepackage{amstext}
\usepackage{latexsym}
\usepackage{amssymb}
\usepackage{color}


%\include{myPreamble}
\include{qm2pi.local} 

%\ifpdf
%\usepackage[pdftex]{graphicx}
%\else
%\usepackage{graphicx}
%\fi

 % \ifpdf
%  \usepackage{pdfsync}
%  \if


%\title{Brief Article}
%\author{David F. Snyder}
%\author{L.G. Meredith}

%\address{Dept. of Math., Texas State University--San Marcos, San Marcos, TX 78666}
       
\pagestyle{empty}


\begin{document}

\lstset{language=[Objective]Caml,frame=shadowbox}

\input{qm2pi.front}

% section front matter (end)

\input{qm2pi.intro} 
 
% section introduction (end)

% \input{qm2pi.knotations} 

% section notation (end)

\input{qm2pi.process.calculi} 

% section concurrent_process_calculi_and_spatial_logics_ (end)
    
%\input{qm2pi.knots2pi} 

%\input{qm2pi.trefoil} 

%\input{qm2pi.mainthm} 

% subsection basic_interpretation (end)

%\input{qm2pi.rho.presentation} 
\subsection{The syntax and semantics of the notation system}\label{sub:the_syntax_and_semantics_of_the_notation_system} % (fold)

We now summarize a technical presentation of the calculus that
embodies our theory of dynamics. The typical presentation of such a
calculus follows the style of giving generators and relations on
them. The grammar, below, describing term constructors, freely
generates the set of processes, $\Proc$. This set is then quotiented
by a relation known as structural congruence and it is over this set
that the notion of dynamics is expressed. This presentation is
essentially that of \cite{MeredithR05} with the addition of
polyadicity and summation. For readability we have relegated some of
the technical subtleties to an appendix.

\subsubsection{Process grammar}\label{subsub:process_grammar}

\begin{mathpar}
  \inferrule* [lab=synchronization] {} {{M} \bc \pzero \;|\; x?F \;|\; x!C }
  \and
  \inferrule* [lab=abstraction] {} {{F} \bc (x)P}
  \and
  \inferrule* [lab=concretion] {} {{C} \bc \langle Q \rangle}
  \and
  \inferrule* [lab=process] {} {{P,Q} \bc M \;| \;P|Q \;|\; @{x}}
  \and
  \inferrule* [lab=name] {} {{x} \bc \quotep{P}}
\end{mathpar} 

Note that $\vec{x}$ (resp. $\vec{P}$) denotes a vector of names
(resp. processes) of length $|\vec{x}|$ (resp. $|\vec{P}|$). We adopt
the following useful abbreviations.

\begin{mathpar}
   x?(\vec{y}).P := x.(\vec{y})P \and  x\clift{\vec{P}} := x.\clift{\vec{P}}
   \and x!(y) := \lift{x}{\dropn{y}}
   \and \Pi_{i=0}^{n-1}P_i := P_0 | \ldots | P_{n-1}
\end{mathpar}

\subsubsection{Structural congruence}

\paragraph{Free and bound names and alpha-equivalence.} At the
core of structural equivalence is alpha-equivalence which identifies
process that are the same up to a change of variable. Formally, we
recognize the distinction between free and bound names. The free names
of a process, $\freenames{P}$, may be calculated recursively as
follows:

\begin{mathpar}
\freenames{\pzero} := \emptyset
  \and \\
  \freenames{x?(y).P} := \{ x \} \cup (\freenames{P} \setminus \{ y \})
  \and 
  \freenames{x!\langle P \rangle} := \{ x \} \cup \{ P \} 
  \and \\
  \freenames{P|Q} := \freenames{P} \cup \freenames{Q}
  \and \\
  \freenames{@{x}} := \{ x \}
\end{mathpar}

$\pi$
$\quotep{\pi}$

$\freenames{-} : \pi \to \mathcal{P}(\quotep{\pi})$

\begin{eqnarray*}
  \freenames{\pzero} & := & \emptyset \\
  \freenames{x?(y).P} & := & \{ x \} \cup (\freenames{P} \setminus \{ y \}) \\
  \freenames{x!\langle P \rangle} & := & \{ x \} \cup \{ P \} \\
  \freenames{P|Q} & := & \freenames{P} \cup \freenames{Q} \\
  \freenames{\dropn{x}} & := & \{ x \}
\end{eqnarray*}

The bound names of a process, $\boundnames{P}$, are those names occurring in $P$
that are not free. For example, in $x?(y).0$, the name $x$ is free, while $y$ is bound.

\begin{mathpar}
  \inferrule* [lab=monoidal-laws] {} { P|Q \equiv Q|P \and P|0 \equiv P \and P|(Q|R) \equiv (P|Q)|R }
\end{mathpar}

\begin{mathpar}
  \inferrule* [lab=alpha-equivalence] {} { (x)P \equiv (y)P\{y/x\} \and y \not\in \freenames{P} }
\end{mathpar}

\begin{definition}
Then two processes, $P,Q$, are alpha-equivalent if $P = Q\{\vec{y}/\vec{x}\}$ for
some $\vec{x} \in \boundnames{Q},\vec{y} \in \boundnames{P}$, where $Q\{\vec{y}/\vec{x}\}$
denotes the capture-avoiding substitution of $\vec{y}$ for $\vec{x}$ in $Q$.
\end{definition}

\begin{definition}
  The {\em structural congruence} \cite{SangiorgiWalker} , $\equiv$,
  between processes is the least congruence containing
  alpha-equivalence, satisfying the abelian monoid laws
  (associativity, commutativity and $\pzero$ as identity) for parallel
  composition $|$ and for summation $+$.
\end{definition}

\subsection{Name equivalence}

We take name equivalence, written $\nameeq$, to be the smallest
equivalence relation generated by the following rules.

\begin{mathpar}
\inferrule*[lab=Quote-drop]
{ }
{ \quotep{@{x}} \nameeq x }

\inferrule*[lab=Struct-equiv]
{ P \scong Q }
{ \quotep{P} \nameeq \quotep{Q} }
\end{mathpar}

The astute reader will have noticed that the mutual recursion of names
and processes imposes a mutual recursion on alpha-equivalence and
structural equivalence via name-equivalence. Fortunately, all of this
works out pleasantly and we may calculate in the natural way, free of
concern. The reader interested in the details is referred to the
appendix \ref{appendix:rho_details}.

\subsection{Substitution}

We use $\Proc$ for the set of processes, $\QProc$ for the set of
names, and $\id{\{}\vec{y} / \vec{x} \id{\}}$ to denote partial maps,
$s : \QProc \rightarrow \QProc$. A map, $s$ lifts, uniquely, to a map
on process terms, $\widehat{s} : \Proc \rightarrow \Proc$ by the
following equations.

\begin{mathpar}
  (0) \psubstp{Q}{P} := 0 \\
  (R \juxtap S) \psubstp{Q}{P}
  :=    
  (R)\psubstp{Q}{P} \juxtap (S) \psubstp{Q}{P} \\
  (x?(y).R) \psubstp{Q}{P}    
  :=    
  (x)\substp{Q}{P} (z)\concat( (R \psubstn{z}{y}) \psubstp{Q}{P} ) \\
  (\lift{x}{R}) \psubstp{Q}{P}  
  :=
  \lift{(x)\substp{Q}{P}}{ R \psubstp{Q}{P} } \\
%   (\dropn{x})  \psubstp{Q}{P}       
%   := 
%   \left\{ 
%     \begin{array}{ccc} 
%       \dropn{\quotep{Q}} & & x \nameeq \quotep{P} \\
%       \dropn{x} & & otherwise \\
%     \end{array}
%   \right. 
  (\dropn{x})  \psubstp{Q}{P}       
  := 
  \left\{ 
    \begin{array}{ccc} 
      Q & & x \nameeq \quotep{P} \\
      \dropn{x} & & otherwise \\
    \end{array}
  \right.
\end{mathpar}
 

where

\begin{eqnarray}
  (x)\id{\{} \lpquote Q \rpquote / \lpquote P \rpquote \id{\}}            = 
  \left\{ 
    \begin{array}{ccc}
      \lpquote Q \rpquote & & x \nameeq \lpquote P \rpquote \\
      x & & otherwise \\
    \end{array}
  \right. \nonumber
\end{eqnarray}

and $z$ is chosen distinct from $\quotep{P}$, $\quotep{Q}$, the free
names in $Q$, and all the names in $R$. Our $\alpha$-equivalence will
be built in the standard way from this substitution.

\begin{remark}\label{rem:no_self_referential_names}
  One consequence of these definitions is that $\forall P. \quotep{P}
  \not\in \freenames{P}$.
\end{remark}

\subsection{ Dynamic quote: an example }

Anticipating something of what's to come, consider applying the
substitution, $\widehat{\id{\{}u / z \id{\}}}$, to the following pair
of processes, $\lift{w}{y!(z)}$ and $w[ \lpquote y!(z) \rpquote ]$.

\begin{eqnarray}
	\lift{w}{y!(z)}\widehat{\id{\{}u / z \id{\}}}
		& = &
		\lift{w}{y!(u)} \nonumber\\
	w[ \lpquote y!(z) \rpquote ] \widehat{ \id{\{}u / z \id{\}} }
		& = &
		w[ \lpquote y!(z) \rpquote ] \nonumber
\end{eqnarray}

Because the body of the process between quotes is impervious to
substitution, we get radically different answers. In fact, by
examining the first process in an input context,
e.g. $x?(z).\lift{w}{y!(z)}$, we see that the process under the lift
operator may be shaped by prefixed inputs binding a name inside it. In
this sense, the lift operator will be seen as a way to dynamically
construct processes before reifying them as names.

Finally equipped with these standard features we can present the
dynamics of the calculus.

\subsubsection{Operational semantics} 

Finally, we introduce the computational dynamics. What marks these
algebras as distinct from other more traditionally studied algebraic
structures, e.g. vector spaces or polynomial rings, is the manner in
which dynamics is captured. In traditional structures, dynamics is typically
expressed through morphisms between such structures, as in linear maps
between vector spaces or morphisms between rings. In algebras
associated with the semantics of computation, the dynamics is
expressed as part of the algebraic structure itself, through a
reduction reduction relation typically denoted by $\red$. Below, we
give a recursive presentation of this relation for the calculus used
in the encoding.

$\red \subseteq \pi \times \pi$
$\red : \pi \to \mathcal{P}(\pi)$

\begin{mathpar}
  \inferrule* [lab=Comm] { \textsf{match}( x_{src}, x_{trgt} ) } { x_{trgt}?(y)P \; | \; x_{src}!\langle {Q} \rangle \red P\{\quotep{Q}/y}\} }
  \and \\
  \inferrule* [lab=Par] {{P} \red {P}'} {{{P} | {Q}} \red {{P}' | {Q}}}
  \and
  \inferrule* [lab=Equiv]{{{P} \scong {P}'} \andalso {{P}' \red {Q}'} \andalso {{Q}' \scong {Q}}}{{P} \red {Q}}
\end{mathpar}

\begin{eqnarray*}
  match_{\equiv} (\quotep{P},\quotep{Q}) & := & P \equiv Q \\
  match_{\dagger}(\quotep{P},\quotep{Q}) & := & \forall R. P|Q \red^{*} R => R \red^{*} 0 \\
  match_{K}(\quotep{P},\quotep{Q}) & := & K \mbox{ for some context } K
\end{eqnarray*}

$u?(x)P | u!\langle Q \rangle \red P\{\quotep{Q}/x\}$

%We write $\wred$ for $\red^*$, and $P\red$ if $\exists Q $ such that $ P \red Q$.
We write $P\red$ if $\exists Q $ such that $ P \red Q$ and $P\not\red$, otherwise.

\section{Replication}

As mentioned before, it is known that replication (and hence
recursion) can be implemented in a higher-order process algebra
\cite{SangiorgiWalker}. As our first example of calculation with the
machinery thus far presented we give the construction explicitly in
the {\rhoc}.

\begin{eqnarray}
	D_{x} & := & \prefix{x}{y}{(\binpar{\outputp{x}{y}}{@{y}})} \nonumber\\
	\bangp_{x}{P} & := & \binpar{{x}!\langle{\binpar{D_{x}}{P}}\rangle}{D_{x}} \nonumber
\end{eqnarray}

\begin{eqnarray}
	\bangp_{x}{P} & & \nonumber\\
	=
	& {x}!\langle{(\prefix{x}{y}{(\outputp{x}{y} | @{y})) | P}}\rangle 
	      | \prefix{x}{y}{(\outputp{x}{y} | @{y})} & \nonumber\\
	\red
	& (\outputp{x}{y} | @{y})\substn{\quotep{(\prefix{x}{y}{(@{y} | \outputp{x}{y})) | P}}}{y} & \nonumber\\
	=
	& \outputp{x}{\quotep{(\prefix{x}{y}{(\outputp{x}{y} | @{y})) | P}}}
	  | {(\prefix{x}{y}{(\outputp{x}{y} | @{y})) | P}} & \nonumber\\
	\red
	& \ldots & \nonumber\\
	\red^*
	& P | P | \ldots & \nonumber
\end{eqnarray}

Of course, this encoding, as an implementation, runs away, unfolding
$\bangp{P}$ eagerly. A lazier and more implementable replication
operator, restricted to input-guarded processes, may be obtained as follows.

\begin{eqnarray}
\bangp{\prefix{u}{v}{P}} 
	:= 
	\binpar{\lift{x}{\prefix{u}{v}{(\binpar{D(x)}{P})}}}{D(x)} \nonumber
\end{eqnarray}

\begin{remark}
  Note that the lazier definition still does not deal with summation
  or mixed summation (i.e. sums over input and output). The reader is
  invited to construct definitions of replication that deal with these
  features. 

  Further, the definitions are parameterized in a name, $x$. Can you,
  gentle reader, make a definition that eliminates this parameter and
  guarantees no accidental interaction between the replication
  machinery and the process being replicated -- i.e. no accidental
  sharing of names used by the process to get its work done and the
  name(s) used by the replication to effect copying. This latter
  revision of the definition of replication is crucial to obtaining
  the expected identity $!!P \sim !P$.
\end{remark}

\begin{remark}\label{rem:paradoxical_combinator}
  The reader familiar with the lambda calculus will have noticed the
  similarity between $D$ and the paradoxical combinator.

  [Ed. note: the existence of this seems to suggest we have to be more
  restrictive on the set of processes and names we admit if we are to
  support no-cloning.]
\end{remark}

\subsubsection{Bisimulation}

The computational dynamics gives rise to another kind of equivalence,
the equivalence of computational behavior. As previously mentioned
this is typically captured \emph{via} some form of bisimulation.

% The notion we use in this paper is weak barbed bisimulation
% \cite{milner91polyadicpi}.

The notion we use in this paper is derived from weak barbed
bisimulation \cite{milner91polyadicpi}. 

\begin{definition}
An \emph{observation relation}, $\downarrow_{\mathcal N}$, over a set
of names, $\mathcal N$, is the smallest relation satisfying the rules
below.

\infrule[Out-barb]{y \in {\mathcal N}, \; x \nameeq y}
		  {\outputp{x}{v} \downarrow_{\mathcal N} x}
\infrule[Par-barb]{\mbox{$P\downarrow_{\mathcal N} x$ or $Q\downarrow_{\mathcal N} x$}}
		  {\binpar{P}{Q} \downarrow_{\mathcal N} x}

We write $P \Downarrow_{\mathcal N} x$ if there is $Q$ such that 
$P \wred Q$ and $Q \downarrow_{\mathcal N} x$.
\end{definition}

\begin{definition}
%\label{def.bbisim}
An  ${\mathcal N}$-\emph{barbed bisimulation} over a set of names, ${\mathcal N}$, is a symmetric binary relation 
${\mathcal S}_{\mathcal N}$ between agents such that $P\rel{S}_{\mathcal N}Q$ implies:
\begin{enumerate}
\item If $P \red P'$ then $Q \wred Q'$ and $P'\rel{S}_{\mathcal N} Q'$.
\item If $P\downarrow_{\mathcal N} x$, then $Q\Downarrow_{\mathcal N} x$.
\end{enumerate}
$P$ is ${\mathcal N}$-barbed bisimilar to $Q$, written
$P \wbbisim_{\mathcal N} Q$, if $P \rel{S}_{\mathcal N} Q$ for some ${\mathcal N}$-barbed bisimulation ${\mathcal S}_{\mathcal N}$.
\end{definition}

$\mathcal{R} \subseteq \pi \times \pi$

$P \mathcal{R} Q => \forall P'. P \red P' \Rightarrow \exists Q'. Q \red Q', P' \mathcal{R} Q'$

$P \vdash x \Rightarrow Q \vdash x$

\begin{mathpar}
  \inferrule*[lab=Out-barb]{x \nameeq y}{{y}!\langle{Q}\rangle \vdash x}
  \and
  \inferrule*[lab=Par-barb]{\mbox{$P\vdash x$ or $Q\vdash x$}}{\binpar{P}{Q} \vdash x}
\end{mathpar}

\subsubsection{Contexts}

One of the principle advantages of computational calculi like the
$\pi$-calculus is a well-defined notion of context,
contextual-equivalence and a correlation between
contextual-equivalence and notions of bisimulation. The notion of
context allows the decomposition of a process into (sub-)process and
its syntactic environment, its context. Thus, a context may be
thought of as a process with a ``hole'' (written $\Box$) in it. The
application of a context $M$ to a process $P$, written $M[P]$, is
tantamount to filling the hole in $M$ with $P$. In this paper we do
not need the full weight of this theory, but do make use of the notion
of context in the proof the main theorem. 

\begin{mathpar}
  \inferrule* [lab=summation] {} {{M_{M},M_{N}} \bc \Box \;|\; x.M_{A} \;|\; M_{M}+M_{N}}
  \and
  \inferrule* [lab=agent] {} {{M_{A}} \bc (\vec{x})M_{P} \;| \; \clift{P_0,\ldots,M_{P},\ldots,P_N}}
  \and \\
  \inferrule* [lab=process] {} {{M_{P}} \bc M_{N} \;| \;P|M_{P} }
\end{mathpar} 

\begin{mathpar}
  \inferrule* [lab=sychronization] {} {M_{N} \bc \Box \;|\; x?M_{F} \;|\; x!M_{C}}
  \and
  \inferrule* [lab=abstraction] {} {{M_{F}} \bc (x)M_{P} }
  \and
  \inferrule* [lab=concretion] {} {{M_{C}} \bc \langle M_{P} \rangle }
  \and \\
  \inferrule* [lab=process] {} {{M_{P}} \bc M_{N} \;| \;P|M_{P} }
\end{mathpar}

\begin{definition}[contextual application] Given a context $M$, and
  process $P$, we define the \emph{contextual application}, $M[P] :=
  M\{P/\Box\}$. That is, the contextual application of M to P is the
  substitution of $P$ for $\Box$ in $M$.
\end{definition}

$\meaningof{-} : L \to \mathcal{P}(\pi)$

\begin{mathpar}
  \inferrule* [lab=collection] {} {\meaningof{true} = \pi, \and \meaningof{~E} = \pi \setminus \meaningof{E}, \and \meaningof{E_{1} \& E_{2}} = \meaningof{E_{1}} \cap \meaningof{E_{2}}}
\end{mathpar}

\begin{mathpar}
  \inferrule* [lab=structure] {} {\meaningof{0} = \{ P \in \pi | P \equiv 0 \}, \and \\ \meaningof{E_1 | E_2} = \{ P \in \pi | P \equiv P_{1} | P_{2}, P_{1} \in \meaningof{E_{1}}, P_{2} \in \meaningof{E_2}\} }
\end{mathpar}

\begin{mathpar}
 \inferrule* [lab=behavior] {} {\meaningof{\langle a?b \rangle E} = \{ P \in \pi | P \equiv Q | u?(y)P', \\ \and \\\\ \and \\ \;\;\; u \in \meaningof{a}, \forall z.P'\{z/y\} \in \meaningof{E\{z/b\}}\}, \and \\ \meaningof{a!E} = \{ P \in \pi | P \equiv Q | x!\langle P' \rangle, x \in \meaningof{a} P' \in \meaningof{E}\} }
\end{mathpar}

\begin{mathpar}
 \inferrule* [lab=nominal] {} {\meaningof{\quotep{E}} = \{ \quotep{P} \in \quotep{\pi} | P \in \meaningof{E} \}, \and \meaningof{\quotep{P}} = \{ \quotep{Q} \in \quotep{\pi} | P \equiv Q \} \and \\ \meaningof{@\quotep{E}} = \{ P \in \pi | P \equiv @x, x \in \meaningof{E} \}}
\end{mathpar}

\begin{eqnarray*}
  \\
  \meaningof{-} : TS \to ST
\end{eqnarray*}

\begin{eqnarray*}
  \\
  L : TS \to ST
\end{eqnarray*}

\begin{eqnarray*}
  \\
  P \models E \iff P \in \meaningof{E}
\end{eqnarray*}

\begin{eqnarray*}
  P \approx_{L} Q \iff \forall E \in L. P \models E \iff Q \models E
\end{eqnarray*}

\begin{eqnarray*}
  P \approx_{K} Q
\end{eqnarray*}

\begin{eqnarray*}
  P \approx Q
\end{eqnarray*}

$\approx_{K} = \approx = \approx_{L}$

\subsubsection{Contextual duality}

Note that contexts extend the quotation operation to a family of
operations from processes to names. Given a context, $M$, we can
define a \emph{nominal context}, $\quotep{M}$ by $\quotep{M}[P] :=
\quotep{M[P]}$. To foreshadow what is to come we observe that these
operations enjoy a duality with processes very much like the duality
between vectors and maps from vectors to scalars.

Further, because the calculus is essentially higher-order, we have a
correspondence between contexts and processes. More specifically,
given a name $x$ and a context $M$ we can construct $M^{*}_{x}$ such
that 

\begin{mathpar}
  M^{*}_{x} | \lift{x}{P} \red M[P]
\end{mathpar}

namely,

\begin{mathpar}
  M^{*}_{x} := x?(u).M[\dropn{u}]
\end{mathpar}

The dependence of $M^{*}_{x}$ on a name makes it an abstraction, 

\begin{mathpar}
  M^{*} := (x)x?(u).M[\dropn{u}]
\end{mathpar}

\subsection{Additional notation}

It will sometimes be convenient to denote the process a name
quotes. We already have the notation $x = \quotep{P}$, but it will be
convenient to introduce an alternate notation, $\procn{x}$, when we
want to emphasize the connection to the use of the name. Note that, by
virtue of name equivalence, $\quotep{\procn{x}} \nameeq x$; so, the
notation is consistent with previous definitions.

Further, because names have structure it is possible to effect
substitutions on the basis of that structure. This means we need to
upgrade our notation for substitutions, which we accomplish by
adapting comprehension notation. Thus,

\begin{mathpar}
  P\{ y / x : x \in S \}
\end{mathpar}

is interpreted to mean the process derived from P by replacing (in a
capture-avoiding manner) each occurrence of $x$ in $S$ by $y$. For example,

\begin{mathpar}
  P\{ \quotep{\procn{x}|\procn{x}} / x : x \in \freenames{P} \}
\end{mathpar}

will replace each (occurrence) of a free name $x$ in $P$ by
$\quotep{\procn{x}|\procn{x}}$.

Also, we will avail ourselves of the notation $x^{L}$ and $x^{R}$ to
denote injections of a name into disjoint copies of the name
space. There are numerous ways to accomplish this. One example can be
found in \cite{MeredithR05}. This notation overloads to vectors of
names: $\vec{x}^{\pi} := (x_{i}^{\pi} \; : \; 0 \leq i < |\vec{x}| )$ where $\pi \in \{L,R\}$.

We also use $P^{\Box} := P|\Box$.

In \cite{MeredithR05} an interpretation of the new operator is
given. It turns out that there are several possible interpretations
all enjoying the requisite algebraic properties of the operator (see
\cite{milner91polyadicpi}). We will therefore make liberal use of
$(\nu\; \vec{x})P$.

% subsection the_syntax_and_semantics_of_the_notation_system (end)   

\input{qm2pi.qmops} 

\input{qm2pi.sterngerlach} 

\input{qm2pi.metric} 

% section concurrent_process_calculi (end)

%\input{qm2pi.proofsketch}

% section proof sketch (end)

%\input{qm2pi.slviaknots} 

% section spatial logic via knots (end)

\input{qm2pi.conclusion}

% section conclusion (end)

%\input{qm2pi.dtcodes} 

% section wiring algorithm (end)

\input{qm2pi.ack} 

% section acknowledgments (end)

\newpage


\bibliographystyle{plain}   
\bibliography{../../biblios/main.bib}

\input{qm2pi.rhodetails}

\end{document}

 

% section acknowledgments (end)

\newpage


\bibliographystyle{plain}   
\bibliography{../../biblios/main.bib}

\documentclass[12pt]{llncs}
%\documentclass{jktr}

\usepackage[pdftex]{hyperref}                   
\usepackage {listings}
\usepackage {mathpartir}
\usepackage{bcprules}
%\usepackage{listings}
                       
\usepackage{graphicx} 
%\usepackage[margins=2.5cm,nohead,nofoot]{geometry}
%\usepackage{geometry}
\usepackage{amsfonts}
\usepackage{amstext}
\usepackage{latexsym}
\usepackage{amssymb}
\usepackage{color}


%\include{myPreamble}
\include{qm2pi.local} 

%\ifpdf
%\usepackage[pdftex]{graphicx}
%\else
%\usepackage{graphicx}
%\fi

 % \ifpdf
%  \usepackage{pdfsync}
%  \if


%\title{Brief Article}
%\author{David F. Snyder}
%\author{L.G. Meredith}

%\address{Dept. of Math., Texas State University--San Marcos, San Marcos, TX 78666}
       
\pagestyle{empty}


\begin{document}

\lstset{language=[Objective]Caml,frame=shadowbox}

\input{qm2pi.front}

% section front matter (end)

\input{qm2pi.intro} 
 
% section introduction (end)

% \input{qm2pi.knotations} 

% section notation (end)

\input{qm2pi.process.calculi} 

% section concurrent_process_calculi_and_spatial_logics_ (end)
    
%\input{qm2pi.knots2pi} 

%\input{qm2pi.trefoil} 

%\input{qm2pi.mainthm} 

% subsection basic_interpretation (end)

%\input{qm2pi.rho.presentation} 
\subsection{The syntax and semantics of the notation system}\label{sub:the_syntax_and_semantics_of_the_notation_system} % (fold)

We now summarize a technical presentation of the calculus that
embodies our theory of dynamics. The typical presentation of such a
calculus follows the style of giving generators and relations on
them. The grammar, below, describing term constructors, freely
generates the set of processes, $\Proc$. This set is then quotiented
by a relation known as structural congruence and it is over this set
that the notion of dynamics is expressed. This presentation is
essentially that of \cite{MeredithR05} with the addition of
polyadicity and summation. For readability we have relegated some of
the technical subtleties to an appendix.

\subsubsection{Process grammar}\label{subsub:process_grammar}

\begin{mathpar}
  \inferrule* [lab=synchronization] {} {{M} \bc \pzero \;|\; x?F \;|\; x!C }
  \and
  \inferrule* [lab=abstraction] {} {{F} \bc (x)P}
  \and
  \inferrule* [lab=concretion] {} {{C} \bc \langle Q \rangle}
  \and
  \inferrule* [lab=process] {} {{P,Q} \bc M \;| \;P|Q \;|\; @{x}}
  \and
  \inferrule* [lab=name] {} {{x} \bc \quotep{P}}
\end{mathpar} 

Note that $\vec{x}$ (resp. $\vec{P}$) denotes a vector of names
(resp. processes) of length $|\vec{x}|$ (resp. $|\vec{P}|$). We adopt
the following useful abbreviations.

\begin{mathpar}
   x?(\vec{y}).P := x.(\vec{y})P \and  x\clift{\vec{P}} := x.\clift{\vec{P}}
   \and x!(y) := \lift{x}{\dropn{y}}
   \and \Pi_{i=0}^{n-1}P_i := P_0 | \ldots | P_{n-1}
\end{mathpar}

\subsubsection{Structural congruence}

\paragraph{Free and bound names and alpha-equivalence.} At the
core of structural equivalence is alpha-equivalence which identifies
process that are the same up to a change of variable. Formally, we
recognize the distinction between free and bound names. The free names
of a process, $\freenames{P}$, may be calculated recursively as
follows:

\begin{mathpar}
\freenames{\pzero} := \emptyset
  \and \\
  \freenames{x?(y).P} := \{ x \} \cup (\freenames{P} \setminus \{ y \})
  \and 
  \freenames{x!\langle P \rangle} := \{ x \} \cup \{ P \} 
  \and \\
  \freenames{P|Q} := \freenames{P} \cup \freenames{Q}
  \and \\
  \freenames{@{x}} := \{ x \}
\end{mathpar}

$\pi$
$\quotep{\pi}$

$\freenames{-} : \pi \to \mathcal{P}(\quotep{\pi})$

\begin{eqnarray*}
  \freenames{\pzero} & := & \emptyset \\
  \freenames{x?(y).P} & := & \{ x \} \cup (\freenames{P} \setminus \{ y \}) \\
  \freenames{x!\langle P \rangle} & := & \{ x \} \cup \{ P \} \\
  \freenames{P|Q} & := & \freenames{P} \cup \freenames{Q} \\
  \freenames{\dropn{x}} & := & \{ x \}
\end{eqnarray*}

The bound names of a process, $\boundnames{P}$, are those names occurring in $P$
that are not free. For example, in $x?(y).0$, the name $x$ is free, while $y$ is bound.

\begin{mathpar}
  \inferrule* [lab=monoidal-laws] {} { P|Q \equiv Q|P \and P|0 \equiv P \and P|(Q|R) \equiv (P|Q)|R }
\end{mathpar}

\begin{mathpar}
  \inferrule* [lab=alpha-equivalence] {} { (x)P \equiv (y)P\{y/x\} \and y \not\in \freenames{P} }
\end{mathpar}

\begin{definition}
Then two processes, $P,Q$, are alpha-equivalent if $P = Q\{\vec{y}/\vec{x}\}$ for
some $\vec{x} \in \boundnames{Q},\vec{y} \in \boundnames{P}$, where $Q\{\vec{y}/\vec{x}\}$
denotes the capture-avoiding substitution of $\vec{y}$ for $\vec{x}$ in $Q$.
\end{definition}

\begin{definition}
  The {\em structural congruence} \cite{SangiorgiWalker} , $\equiv$,
  between processes is the least congruence containing
  alpha-equivalence, satisfying the abelian monoid laws
  (associativity, commutativity and $\pzero$ as identity) for parallel
  composition $|$ and for summation $+$.
\end{definition}

\subsection{Name equivalence}

We take name equivalence, written $\nameeq$, to be the smallest
equivalence relation generated by the following rules.

\begin{mathpar}
\inferrule*[lab=Quote-drop]
{ }
{ \quotep{@{x}} \nameeq x }

\inferrule*[lab=Struct-equiv]
{ P \scong Q }
{ \quotep{P} \nameeq \quotep{Q} }
\end{mathpar}

The astute reader will have noticed that the mutual recursion of names
and processes imposes a mutual recursion on alpha-equivalence and
structural equivalence via name-equivalence. Fortunately, all of this
works out pleasantly and we may calculate in the natural way, free of
concern. The reader interested in the details is referred to the
appendix \ref{appendix:rho_details}.

\subsection{Substitution}

We use $\Proc$ for the set of processes, $\QProc$ for the set of
names, and $\id{\{}\vec{y} / \vec{x} \id{\}}$ to denote partial maps,
$s : \QProc \rightarrow \QProc$. A map, $s$ lifts, uniquely, to a map
on process terms, $\widehat{s} : \Proc \rightarrow \Proc$ by the
following equations.

\begin{mathpar}
  (0) \psubstp{Q}{P} := 0 \\
  (R \juxtap S) \psubstp{Q}{P}
  :=    
  (R)\psubstp{Q}{P} \juxtap (S) \psubstp{Q}{P} \\
  (x?(y).R) \psubstp{Q}{P}    
  :=    
  (x)\substp{Q}{P} (z)\concat( (R \psubstn{z}{y}) \psubstp{Q}{P} ) \\
  (\lift{x}{R}) \psubstp{Q}{P}  
  :=
  \lift{(x)\substp{Q}{P}}{ R \psubstp{Q}{P} } \\
%   (\dropn{x})  \psubstp{Q}{P}       
%   := 
%   \left\{ 
%     \begin{array}{ccc} 
%       \dropn{\quotep{Q}} & & x \nameeq \quotep{P} \\
%       \dropn{x} & & otherwise \\
%     \end{array}
%   \right. 
  (\dropn{x})  \psubstp{Q}{P}       
  := 
  \left\{ 
    \begin{array}{ccc} 
      Q & & x \nameeq \quotep{P} \\
      \dropn{x} & & otherwise \\
    \end{array}
  \right.
\end{mathpar}
 

where

\begin{eqnarray}
  (x)\id{\{} \lpquote Q \rpquote / \lpquote P \rpquote \id{\}}            = 
  \left\{ 
    \begin{array}{ccc}
      \lpquote Q \rpquote & & x \nameeq \lpquote P \rpquote \\
      x & & otherwise \\
    \end{array}
  \right. \nonumber
\end{eqnarray}

and $z$ is chosen distinct from $\quotep{P}$, $\quotep{Q}$, the free
names in $Q$, and all the names in $R$. Our $\alpha$-equivalence will
be built in the standard way from this substitution.

\begin{remark}\label{rem:no_self_referential_names}
  One consequence of these definitions is that $\forall P. \quotep{P}
  \not\in \freenames{P}$.
\end{remark}

\subsection{ Dynamic quote: an example }

Anticipating something of what's to come, consider applying the
substitution, $\widehat{\id{\{}u / z \id{\}}}$, to the following pair
of processes, $\lift{w}{y!(z)}$ and $w[ \lpquote y!(z) \rpquote ]$.

\begin{eqnarray}
	\lift{w}{y!(z)}\widehat{\id{\{}u / z \id{\}}}
		& = &
		\lift{w}{y!(u)} \nonumber\\
	w[ \lpquote y!(z) \rpquote ] \widehat{ \id{\{}u / z \id{\}} }
		& = &
		w[ \lpquote y!(z) \rpquote ] \nonumber
\end{eqnarray}

Because the body of the process between quotes is impervious to
substitution, we get radically different answers. In fact, by
examining the first process in an input context,
e.g. $x?(z).\lift{w}{y!(z)}$, we see that the process under the lift
operator may be shaped by prefixed inputs binding a name inside it. In
this sense, the lift operator will be seen as a way to dynamically
construct processes before reifying them as names.

Finally equipped with these standard features we can present the
dynamics of the calculus.

\subsubsection{Operational semantics} 

Finally, we introduce the computational dynamics. What marks these
algebras as distinct from other more traditionally studied algebraic
structures, e.g. vector spaces or polynomial rings, is the manner in
which dynamics is captured. In traditional structures, dynamics is typically
expressed through morphisms between such structures, as in linear maps
between vector spaces or morphisms between rings. In algebras
associated with the semantics of computation, the dynamics is
expressed as part of the algebraic structure itself, through a
reduction reduction relation typically denoted by $\red$. Below, we
give a recursive presentation of this relation for the calculus used
in the encoding.

$\red \subseteq \pi \times \pi$
$\red : \pi \to \mathcal{P}(\pi)$

\begin{mathpar}
  \inferrule* [lab=Comm] { \textsf{match}( x_{src}, x_{trgt} ) } { x_{trgt}?(y)P \; | \; x_{src}!\langle {Q} \rangle \red P\{\quotep{Q}/y}\} }
  \and \\
  \inferrule* [lab=Par] {{P} \red {P}'} {{{P} | {Q}} \red {{P}' | {Q}}}
  \and
  \inferrule* [lab=Equiv]{{{P} \scong {P}'} \andalso {{P}' \red {Q}'} \andalso {{Q}' \scong {Q}}}{{P} \red {Q}}
\end{mathpar}

\begin{eqnarray*}
  match_{\equiv} (\quotep{P},\quotep{Q}) & := & P \equiv Q \\
  match_{\dagger}(\quotep{P},\quotep{Q}) & := & \forall R. P|Q \red^{*} R => R \red^{*} 0 \\
  match_{K}(\quotep{P},\quotep{Q}) & := & K \mbox{ for some context } K
\end{eqnarray*}

$u?(x)P | u!\langle Q \rangle \red P\{\quotep{Q}/x\}$

%We write $\wred$ for $\red^*$, and $P\red$ if $\exists Q $ such that $ P \red Q$.
We write $P\red$ if $\exists Q $ such that $ P \red Q$ and $P\not\red$, otherwise.

\section{Replication}

As mentioned before, it is known that replication (and hence
recursion) can be implemented in a higher-order process algebra
\cite{SangiorgiWalker}. As our first example of calculation with the
machinery thus far presented we give the construction explicitly in
the {\rhoc}.

\begin{eqnarray}
	D_{x} & := & \prefix{x}{y}{(\binpar{\outputp{x}{y}}{@{y}})} \nonumber\\
	\bangp_{x}{P} & := & \binpar{{x}!\langle{\binpar{D_{x}}{P}}\rangle}{D_{x}} \nonumber
\end{eqnarray}

\begin{eqnarray}
	\bangp_{x}{P} & & \nonumber\\
	=
	& {x}!\langle{(\prefix{x}{y}{(\outputp{x}{y} | @{y})) | P}}\rangle 
	      | \prefix{x}{y}{(\outputp{x}{y} | @{y})} & \nonumber\\
	\red
	& (\outputp{x}{y} | @{y})\substn{\quotep{(\prefix{x}{y}{(@{y} | \outputp{x}{y})) | P}}}{y} & \nonumber\\
	=
	& \outputp{x}{\quotep{(\prefix{x}{y}{(\outputp{x}{y} | @{y})) | P}}}
	  | {(\prefix{x}{y}{(\outputp{x}{y} | @{y})) | P}} & \nonumber\\
	\red
	& \ldots & \nonumber\\
	\red^*
	& P | P | \ldots & \nonumber
\end{eqnarray}

Of course, this encoding, as an implementation, runs away, unfolding
$\bangp{P}$ eagerly. A lazier and more implementable replication
operator, restricted to input-guarded processes, may be obtained as follows.

\begin{eqnarray}
\bangp{\prefix{u}{v}{P}} 
	:= 
	\binpar{\lift{x}{\prefix{u}{v}{(\binpar{D(x)}{P})}}}{D(x)} \nonumber
\end{eqnarray}

\begin{remark}
  Note that the lazier definition still does not deal with summation
  or mixed summation (i.e. sums over input and output). The reader is
  invited to construct definitions of replication that deal with these
  features. 

  Further, the definitions are parameterized in a name, $x$. Can you,
  gentle reader, make a definition that eliminates this parameter and
  guarantees no accidental interaction between the replication
  machinery and the process being replicated -- i.e. no accidental
  sharing of names used by the process to get its work done and the
  name(s) used by the replication to effect copying. This latter
  revision of the definition of replication is crucial to obtaining
  the expected identity $!!P \sim !P$.
\end{remark}

\begin{remark}\label{rem:paradoxical_combinator}
  The reader familiar with the lambda calculus will have noticed the
  similarity between $D$ and the paradoxical combinator.

  [Ed. note: the existence of this seems to suggest we have to be more
  restrictive on the set of processes and names we admit if we are to
  support no-cloning.]
\end{remark}

\subsubsection{Bisimulation}

The computational dynamics gives rise to another kind of equivalence,
the equivalence of computational behavior. As previously mentioned
this is typically captured \emph{via} some form of bisimulation.

% The notion we use in this paper is weak barbed bisimulation
% \cite{milner91polyadicpi}.

The notion we use in this paper is derived from weak barbed
bisimulation \cite{milner91polyadicpi}. 

\begin{definition}
An \emph{observation relation}, $\downarrow_{\mathcal N}$, over a set
of names, $\mathcal N$, is the smallest relation satisfying the rules
below.

\infrule[Out-barb]{y \in {\mathcal N}, \; x \nameeq y}
		  {\outputp{x}{v} \downarrow_{\mathcal N} x}
\infrule[Par-barb]{\mbox{$P\downarrow_{\mathcal N} x$ or $Q\downarrow_{\mathcal N} x$}}
		  {\binpar{P}{Q} \downarrow_{\mathcal N} x}

We write $P \Downarrow_{\mathcal N} x$ if there is $Q$ such that 
$P \wred Q$ and $Q \downarrow_{\mathcal N} x$.
\end{definition}

\begin{definition}
%\label{def.bbisim}
An  ${\mathcal N}$-\emph{barbed bisimulation} over a set of names, ${\mathcal N}$, is a symmetric binary relation 
${\mathcal S}_{\mathcal N}$ between agents such that $P\rel{S}_{\mathcal N}Q$ implies:
\begin{enumerate}
\item If $P \red P'$ then $Q \wred Q'$ and $P'\rel{S}_{\mathcal N} Q'$.
\item If $P\downarrow_{\mathcal N} x$, then $Q\Downarrow_{\mathcal N} x$.
\end{enumerate}
$P$ is ${\mathcal N}$-barbed bisimilar to $Q$, written
$P \wbbisim_{\mathcal N} Q$, if $P \rel{S}_{\mathcal N} Q$ for some ${\mathcal N}$-barbed bisimulation ${\mathcal S}_{\mathcal N}$.
\end{definition}

$\mathcal{R} \subseteq \pi \times \pi$

$P \mathcal{R} Q => \forall P'. P \red P' \Rightarrow \exists Q'. Q \red Q', P' \mathcal{R} Q'$

$P \vdash x \Rightarrow Q \vdash x$

\begin{mathpar}
  \inferrule*[lab=Out-barb]{x \nameeq y}{{y}!\langle{Q}\rangle \vdash x}
  \and
  \inferrule*[lab=Par-barb]{\mbox{$P\vdash x$ or $Q\vdash x$}}{\binpar{P}{Q} \vdash x}
\end{mathpar}

\subsubsection{Contexts}

One of the principle advantages of computational calculi like the
$\pi$-calculus is a well-defined notion of context,
contextual-equivalence and a correlation between
contextual-equivalence and notions of bisimulation. The notion of
context allows the decomposition of a process into (sub-)process and
its syntactic environment, its context. Thus, a context may be
thought of as a process with a ``hole'' (written $\Box$) in it. The
application of a context $M$ to a process $P$, written $M[P]$, is
tantamount to filling the hole in $M$ with $P$. In this paper we do
not need the full weight of this theory, but do make use of the notion
of context in the proof the main theorem. 

\begin{mathpar}
  \inferrule* [lab=summation] {} {{M_{M},M_{N}} \bc \Box \;|\; x.M_{A} \;|\; M_{M}+M_{N}}
  \and
  \inferrule* [lab=agent] {} {{M_{A}} \bc (\vec{x})M_{P} \;| \; \clift{P_0,\ldots,M_{P},\ldots,P_N}}
  \and \\
  \inferrule* [lab=process] {} {{M_{P}} \bc M_{N} \;| \;P|M_{P} }
\end{mathpar} 

\begin{mathpar}
  \inferrule* [lab=sychronization] {} {M_{N} \bc \Box \;|\; x?M_{F} \;|\; x!M_{C}}
  \and
  \inferrule* [lab=abstraction] {} {{M_{F}} \bc (x)M_{P} }
  \and
  \inferrule* [lab=concretion] {} {{M_{C}} \bc \langle M_{P} \rangle }
  \and \\
  \inferrule* [lab=process] {} {{M_{P}} \bc M_{N} \;| \;P|M_{P} }
\end{mathpar}

\begin{definition}[contextual application] Given a context $M$, and
  process $P$, we define the \emph{contextual application}, $M[P] :=
  M\{P/\Box\}$. That is, the contextual application of M to P is the
  substitution of $P$ for $\Box$ in $M$.
\end{definition}

$\meaningof{-} : L \to \mathcal{P}(\pi)$

\begin{mathpar}
  \inferrule* [lab=collection] {} {\meaningof{true} = \pi, \and \meaningof{~E} = \pi \setminus \meaningof{E}, \and \meaningof{E_{1} \& E_{2}} = \meaningof{E_{1}} \cap \meaningof{E_{2}}}
\end{mathpar}

\begin{mathpar}
  \inferrule* [lab=structure] {} {\meaningof{0} = \{ P \in \pi | P \equiv 0 \}, \and \\ \meaningof{E_1 | E_2} = \{ P \in \pi | P \equiv P_{1} | P_{2}, P_{1} \in \meaningof{E_{1}}, P_{2} \in \meaningof{E_2}\} }
\end{mathpar}

\begin{mathpar}
 \inferrule* [lab=behavior] {} {\meaningof{\langle a?b \rangle E} = \{ P \in \pi | P \equiv Q | u?(y)P', \\ \and \\\\ \and \\ \;\;\; u \in \meaningof{a}, \forall z.P'\{z/y\} \in \meaningof{E\{z/b\}}\}, \and \\ \meaningof{a!E} = \{ P \in \pi | P \equiv Q | x!\langle P' \rangle, x \in \meaningof{a} P' \in \meaningof{E}\} }
\end{mathpar}

\begin{mathpar}
 \inferrule* [lab=nominal] {} {\meaningof{\quotep{E}} = \{ \quotep{P} \in \quotep{\pi} | P \in \meaningof{E} \}, \and \meaningof{\quotep{P}} = \{ \quotep{Q} \in \quotep{\pi} | P \equiv Q \} \and \\ \meaningof{@\quotep{E}} = \{ P \in \pi | P \equiv @x, x \in \meaningof{E} \}}
\end{mathpar}

\begin{eqnarray*}
  \\
  \meaningof{-} : TS \to ST
\end{eqnarray*}

\begin{eqnarray*}
  \\
  L : TS \to ST
\end{eqnarray*}

\begin{eqnarray*}
  \\
  P \models E \iff P \in \meaningof{E}
\end{eqnarray*}

\begin{eqnarray*}
  P \approx_{L} Q \iff \forall E \in L. P \models E \iff Q \models E
\end{eqnarray*}

\begin{eqnarray*}
  P \approx_{K} Q
\end{eqnarray*}

\begin{eqnarray*}
  P \approx Q
\end{eqnarray*}

$\approx_{K} = \approx = \approx_{L}$

\subsubsection{Contextual duality}

Note that contexts extend the quotation operation to a family of
operations from processes to names. Given a context, $M$, we can
define a \emph{nominal context}, $\quotep{M}$ by $\quotep{M}[P] :=
\quotep{M[P]}$. To foreshadow what is to come we observe that these
operations enjoy a duality with processes very much like the duality
between vectors and maps from vectors to scalars.

Further, because the calculus is essentially higher-order, we have a
correspondence between contexts and processes. More specifically,
given a name $x$ and a context $M$ we can construct $M^{*}_{x}$ such
that 

\begin{mathpar}
  M^{*}_{x} | \lift{x}{P} \red M[P]
\end{mathpar}

namely,

\begin{mathpar}
  M^{*}_{x} := x?(u).M[\dropn{u}]
\end{mathpar}

The dependence of $M^{*}_{x}$ on a name makes it an abstraction, 

\begin{mathpar}
  M^{*} := (x)x?(u).M[\dropn{u}]
\end{mathpar}

\subsection{Additional notation}

It will sometimes be convenient to denote the process a name
quotes. We already have the notation $x = \quotep{P}$, but it will be
convenient to introduce an alternate notation, $\procn{x}$, when we
want to emphasize the connection to the use of the name. Note that, by
virtue of name equivalence, $\quotep{\procn{x}} \nameeq x$; so, the
notation is consistent with previous definitions.

Further, because names have structure it is possible to effect
substitutions on the basis of that structure. This means we need to
upgrade our notation for substitutions, which we accomplish by
adapting comprehension notation. Thus,

\begin{mathpar}
  P\{ y / x : x \in S \}
\end{mathpar}

is interpreted to mean the process derived from P by replacing (in a
capture-avoiding manner) each occurrence of $x$ in $S$ by $y$. For example,

\begin{mathpar}
  P\{ \quotep{\procn{x}|\procn{x}} / x : x \in \freenames{P} \}
\end{mathpar}

will replace each (occurrence) of a free name $x$ in $P$ by
$\quotep{\procn{x}|\procn{x}}$.

Also, we will avail ourselves of the notation $x^{L}$ and $x^{R}$ to
denote injections of a name into disjoint copies of the name
space. There are numerous ways to accomplish this. One example can be
found in \cite{MeredithR05}. This notation overloads to vectors of
names: $\vec{x}^{\pi} := (x_{i}^{\pi} \; : \; 0 \leq i < |\vec{x}| )$ where $\pi \in \{L,R\}$.

We also use $P^{\Box} := P|\Box$.

In \cite{MeredithR05} an interpretation of the new operator is
given. It turns out that there are several possible interpretations
all enjoying the requisite algebraic properties of the operator (see
\cite{milner91polyadicpi}). We will therefore make liberal use of
$(\nu\; \vec{x})P$.

% subsection the_syntax_and_semantics_of_the_notation_system (end)   

\input{qm2pi.qmops} 

\input{qm2pi.sterngerlach} 

\input{qm2pi.metric} 

% section concurrent_process_calculi (end)

%\input{qm2pi.proofsketch}

% section proof sketch (end)

%\input{qm2pi.slviaknots} 

% section spatial logic via knots (end)

\input{qm2pi.conclusion}

% section conclusion (end)

%\input{qm2pi.dtcodes} 

% section wiring algorithm (end)

\input{qm2pi.ack} 

% section acknowledgments (end)

\newpage


\bibliographystyle{plain}   
\bibliography{../../biblios/main.bib}

\input{qm2pi.rhodetails}

\end{document}



\end{document}

 

% section wiring algorithm (end)

\documentclass[12pt]{llncs}
%\documentclass{jktr}

\usepackage[pdftex]{hyperref}                   
\usepackage {listings}
\usepackage {mathpartir}
\usepackage{bcprules}
%\usepackage{listings}
                       
\usepackage{graphicx} 
%\usepackage[margins=2.5cm,nohead,nofoot]{geometry}
%\usepackage{geometry}
\usepackage{amsfonts}
\usepackage{amstext}
\usepackage{latexsym}
\usepackage{amssymb}
\usepackage{color}


%\include{myPreamble}
\documentclass[12pt]{llncs}
%\documentclass{jktr}

\usepackage[pdftex]{hyperref}                   
\usepackage {listings}
\usepackage {mathpartir}
\usepackage{bcprules}
%\usepackage{listings}
                       
\usepackage{graphicx} 
%\usepackage[margins=2.5cm,nohead,nofoot]{geometry}
%\usepackage{geometry}
\usepackage{amsfonts}
\usepackage{amstext}
\usepackage{latexsym}
\usepackage{amssymb}
\usepackage{color}


%\include{myPreamble}
\include{qm2pi.local} 

%\ifpdf
%\usepackage[pdftex]{graphicx}
%\else
%\usepackage{graphicx}
%\fi

 % \ifpdf
%  \usepackage{pdfsync}
%  \if


%\title{Brief Article}
%\author{David F. Snyder}
%\author{L.G. Meredith}

%\address{Dept. of Math., Texas State University--San Marcos, San Marcos, TX 78666}
       
\pagestyle{empty}


\begin{document}

\lstset{language=[Objective]Caml,frame=shadowbox}

\input{qm2pi.front}

% section front matter (end)

\input{qm2pi.intro} 
 
% section introduction (end)

% \input{qm2pi.knotations} 

% section notation (end)

\input{qm2pi.process.calculi} 

% section concurrent_process_calculi_and_spatial_logics_ (end)
    
%\input{qm2pi.knots2pi} 

%\input{qm2pi.trefoil} 

%\input{qm2pi.mainthm} 

% subsection basic_interpretation (end)

%\input{qm2pi.rho.presentation} 
\subsection{The syntax and semantics of the notation system}\label{sub:the_syntax_and_semantics_of_the_notation_system} % (fold)

We now summarize a technical presentation of the calculus that
embodies our theory of dynamics. The typical presentation of such a
calculus follows the style of giving generators and relations on
them. The grammar, below, describing term constructors, freely
generates the set of processes, $\Proc$. This set is then quotiented
by a relation known as structural congruence and it is over this set
that the notion of dynamics is expressed. This presentation is
essentially that of \cite{MeredithR05} with the addition of
polyadicity and summation. For readability we have relegated some of
the technical subtleties to an appendix.

\subsubsection{Process grammar}\label{subsub:process_grammar}

\begin{mathpar}
  \inferrule* [lab=synchronization] {} {{M} \bc \pzero \;|\; x?F \;|\; x!C }
  \and
  \inferrule* [lab=abstraction] {} {{F} \bc (x)P}
  \and
  \inferrule* [lab=concretion] {} {{C} \bc \langle Q \rangle}
  \and
  \inferrule* [lab=process] {} {{P,Q} \bc M \;| \;P|Q \;|\; @{x}}
  \and
  \inferrule* [lab=name] {} {{x} \bc \quotep{P}}
\end{mathpar} 

Note that $\vec{x}$ (resp. $\vec{P}$) denotes a vector of names
(resp. processes) of length $|\vec{x}|$ (resp. $|\vec{P}|$). We adopt
the following useful abbreviations.

\begin{mathpar}
   x?(\vec{y}).P := x.(\vec{y})P \and  x\clift{\vec{P}} := x.\clift{\vec{P}}
   \and x!(y) := \lift{x}{\dropn{y}}
   \and \Pi_{i=0}^{n-1}P_i := P_0 | \ldots | P_{n-1}
\end{mathpar}

\subsubsection{Structural congruence}

\paragraph{Free and bound names and alpha-equivalence.} At the
core of structural equivalence is alpha-equivalence which identifies
process that are the same up to a change of variable. Formally, we
recognize the distinction between free and bound names. The free names
of a process, $\freenames{P}$, may be calculated recursively as
follows:

\begin{mathpar}
\freenames{\pzero} := \emptyset
  \and \\
  \freenames{x?(y).P} := \{ x \} \cup (\freenames{P} \setminus \{ y \})
  \and 
  \freenames{x!\langle P \rangle} := \{ x \} \cup \{ P \} 
  \and \\
  \freenames{P|Q} := \freenames{P} \cup \freenames{Q}
  \and \\
  \freenames{@{x}} := \{ x \}
\end{mathpar}

$\pi$
$\quotep{\pi}$

$\freenames{-} : \pi \to \mathcal{P}(\quotep{\pi})$

\begin{eqnarray*}
  \freenames{\pzero} & := & \emptyset \\
  \freenames{x?(y).P} & := & \{ x \} \cup (\freenames{P} \setminus \{ y \}) \\
  \freenames{x!\langle P \rangle} & := & \{ x \} \cup \{ P \} \\
  \freenames{P|Q} & := & \freenames{P} \cup \freenames{Q} \\
  \freenames{\dropn{x}} & := & \{ x \}
\end{eqnarray*}

The bound names of a process, $\boundnames{P}$, are those names occurring in $P$
that are not free. For example, in $x?(y).0$, the name $x$ is free, while $y$ is bound.

\begin{mathpar}
  \inferrule* [lab=monoidal-laws] {} { P|Q \equiv Q|P \and P|0 \equiv P \and P|(Q|R) \equiv (P|Q)|R }
\end{mathpar}

\begin{mathpar}
  \inferrule* [lab=alpha-equivalence] {} { (x)P \equiv (y)P\{y/x\} \and y \not\in \freenames{P} }
\end{mathpar}

\begin{definition}
Then two processes, $P,Q$, are alpha-equivalent if $P = Q\{\vec{y}/\vec{x}\}$ for
some $\vec{x} \in \boundnames{Q},\vec{y} \in \boundnames{P}$, where $Q\{\vec{y}/\vec{x}\}$
denotes the capture-avoiding substitution of $\vec{y}$ for $\vec{x}$ in $Q$.
\end{definition}

\begin{definition}
  The {\em structural congruence} \cite{SangiorgiWalker} , $\equiv$,
  between processes is the least congruence containing
  alpha-equivalence, satisfying the abelian monoid laws
  (associativity, commutativity and $\pzero$ as identity) for parallel
  composition $|$ and for summation $+$.
\end{definition}

\subsection{Name equivalence}

We take name equivalence, written $\nameeq$, to be the smallest
equivalence relation generated by the following rules.

\begin{mathpar}
\inferrule*[lab=Quote-drop]
{ }
{ \quotep{@{x}} \nameeq x }

\inferrule*[lab=Struct-equiv]
{ P \scong Q }
{ \quotep{P} \nameeq \quotep{Q} }
\end{mathpar}

The astute reader will have noticed that the mutual recursion of names
and processes imposes a mutual recursion on alpha-equivalence and
structural equivalence via name-equivalence. Fortunately, all of this
works out pleasantly and we may calculate in the natural way, free of
concern. The reader interested in the details is referred to the
appendix \ref{appendix:rho_details}.

\subsection{Substitution}

We use $\Proc$ for the set of processes, $\QProc$ for the set of
names, and $\id{\{}\vec{y} / \vec{x} \id{\}}$ to denote partial maps,
$s : \QProc \rightarrow \QProc$. A map, $s$ lifts, uniquely, to a map
on process terms, $\widehat{s} : \Proc \rightarrow \Proc$ by the
following equations.

\begin{mathpar}
  (0) \psubstp{Q}{P} := 0 \\
  (R \juxtap S) \psubstp{Q}{P}
  :=    
  (R)\psubstp{Q}{P} \juxtap (S) \psubstp{Q}{P} \\
  (x?(y).R) \psubstp{Q}{P}    
  :=    
  (x)\substp{Q}{P} (z)\concat( (R \psubstn{z}{y}) \psubstp{Q}{P} ) \\
  (\lift{x}{R}) \psubstp{Q}{P}  
  :=
  \lift{(x)\substp{Q}{P}}{ R \psubstp{Q}{P} } \\
%   (\dropn{x})  \psubstp{Q}{P}       
%   := 
%   \left\{ 
%     \begin{array}{ccc} 
%       \dropn{\quotep{Q}} & & x \nameeq \quotep{P} \\
%       \dropn{x} & & otherwise \\
%     \end{array}
%   \right. 
  (\dropn{x})  \psubstp{Q}{P}       
  := 
  \left\{ 
    \begin{array}{ccc} 
      Q & & x \nameeq \quotep{P} \\
      \dropn{x} & & otherwise \\
    \end{array}
  \right.
\end{mathpar}
 

where

\begin{eqnarray}
  (x)\id{\{} \lpquote Q \rpquote / \lpquote P \rpquote \id{\}}            = 
  \left\{ 
    \begin{array}{ccc}
      \lpquote Q \rpquote & & x \nameeq \lpquote P \rpquote \\
      x & & otherwise \\
    \end{array}
  \right. \nonumber
\end{eqnarray}

and $z$ is chosen distinct from $\quotep{P}$, $\quotep{Q}$, the free
names in $Q$, and all the names in $R$. Our $\alpha$-equivalence will
be built in the standard way from this substitution.

\begin{remark}\label{rem:no_self_referential_names}
  One consequence of these definitions is that $\forall P. \quotep{P}
  \not\in \freenames{P}$.
\end{remark}

\subsection{ Dynamic quote: an example }

Anticipating something of what's to come, consider applying the
substitution, $\widehat{\id{\{}u / z \id{\}}}$, to the following pair
of processes, $\lift{w}{y!(z)}$ and $w[ \lpquote y!(z) \rpquote ]$.

\begin{eqnarray}
	\lift{w}{y!(z)}\widehat{\id{\{}u / z \id{\}}}
		& = &
		\lift{w}{y!(u)} \nonumber\\
	w[ \lpquote y!(z) \rpquote ] \widehat{ \id{\{}u / z \id{\}} }
		& = &
		w[ \lpquote y!(z) \rpquote ] \nonumber
\end{eqnarray}

Because the body of the process between quotes is impervious to
substitution, we get radically different answers. In fact, by
examining the first process in an input context,
e.g. $x?(z).\lift{w}{y!(z)}$, we see that the process under the lift
operator may be shaped by prefixed inputs binding a name inside it. In
this sense, the lift operator will be seen as a way to dynamically
construct processes before reifying them as names.

Finally equipped with these standard features we can present the
dynamics of the calculus.

\subsubsection{Operational semantics} 

Finally, we introduce the computational dynamics. What marks these
algebras as distinct from other more traditionally studied algebraic
structures, e.g. vector spaces or polynomial rings, is the manner in
which dynamics is captured. In traditional structures, dynamics is typically
expressed through morphisms between such structures, as in linear maps
between vector spaces or morphisms between rings. In algebras
associated with the semantics of computation, the dynamics is
expressed as part of the algebraic structure itself, through a
reduction reduction relation typically denoted by $\red$. Below, we
give a recursive presentation of this relation for the calculus used
in the encoding.

$\red \subseteq \pi \times \pi$
$\red : \pi \to \mathcal{P}(\pi)$

\begin{mathpar}
  \inferrule* [lab=Comm] { \textsf{match}( x_{src}, x_{trgt} ) } { x_{trgt}?(y)P \; | \; x_{src}!\langle {Q} \rangle \red P\{\quotep{Q}/y}\} }
  \and \\
  \inferrule* [lab=Par] {{P} \red {P}'} {{{P} | {Q}} \red {{P}' | {Q}}}
  \and
  \inferrule* [lab=Equiv]{{{P} \scong {P}'} \andalso {{P}' \red {Q}'} \andalso {{Q}' \scong {Q}}}{{P} \red {Q}}
\end{mathpar}

\begin{eqnarray*}
  match_{\equiv} (\quotep{P},\quotep{Q}) & := & P \equiv Q \\
  match_{\dagger}(\quotep{P},\quotep{Q}) & := & \forall R. P|Q \red^{*} R => R \red^{*} 0 \\
  match_{K}(\quotep{P},\quotep{Q}) & := & K \mbox{ for some context } K
\end{eqnarray*}

$u?(x)P | u!\langle Q \rangle \red P\{\quotep{Q}/x\}$

%We write $\wred$ for $\red^*$, and $P\red$ if $\exists Q $ such that $ P \red Q$.
We write $P\red$ if $\exists Q $ such that $ P \red Q$ and $P\not\red$, otherwise.

\section{Replication}

As mentioned before, it is known that replication (and hence
recursion) can be implemented in a higher-order process algebra
\cite{SangiorgiWalker}. As our first example of calculation with the
machinery thus far presented we give the construction explicitly in
the {\rhoc}.

\begin{eqnarray}
	D_{x} & := & \prefix{x}{y}{(\binpar{\outputp{x}{y}}{@{y}})} \nonumber\\
	\bangp_{x}{P} & := & \binpar{{x}!\langle{\binpar{D_{x}}{P}}\rangle}{D_{x}} \nonumber
\end{eqnarray}

\begin{eqnarray}
	\bangp_{x}{P} & & \nonumber\\
	=
	& {x}!\langle{(\prefix{x}{y}{(\outputp{x}{y} | @{y})) | P}}\rangle 
	      | \prefix{x}{y}{(\outputp{x}{y} | @{y})} & \nonumber\\
	\red
	& (\outputp{x}{y} | @{y})\substn{\quotep{(\prefix{x}{y}{(@{y} | \outputp{x}{y})) | P}}}{y} & \nonumber\\
	=
	& \outputp{x}{\quotep{(\prefix{x}{y}{(\outputp{x}{y} | @{y})) | P}}}
	  | {(\prefix{x}{y}{(\outputp{x}{y} | @{y})) | P}} & \nonumber\\
	\red
	& \ldots & \nonumber\\
	\red^*
	& P | P | \ldots & \nonumber
\end{eqnarray}

Of course, this encoding, as an implementation, runs away, unfolding
$\bangp{P}$ eagerly. A lazier and more implementable replication
operator, restricted to input-guarded processes, may be obtained as follows.

\begin{eqnarray}
\bangp{\prefix{u}{v}{P}} 
	:= 
	\binpar{\lift{x}{\prefix{u}{v}{(\binpar{D(x)}{P})}}}{D(x)} \nonumber
\end{eqnarray}

\begin{remark}
  Note that the lazier definition still does not deal with summation
  or mixed summation (i.e. sums over input and output). The reader is
  invited to construct definitions of replication that deal with these
  features. 

  Further, the definitions are parameterized in a name, $x$. Can you,
  gentle reader, make a definition that eliminates this parameter and
  guarantees no accidental interaction between the replication
  machinery and the process being replicated -- i.e. no accidental
  sharing of names used by the process to get its work done and the
  name(s) used by the replication to effect copying. This latter
  revision of the definition of replication is crucial to obtaining
  the expected identity $!!P \sim !P$.
\end{remark}

\begin{remark}\label{rem:paradoxical_combinator}
  The reader familiar with the lambda calculus will have noticed the
  similarity between $D$ and the paradoxical combinator.

  [Ed. note: the existence of this seems to suggest we have to be more
  restrictive on the set of processes and names we admit if we are to
  support no-cloning.]
\end{remark}

\subsubsection{Bisimulation}

The computational dynamics gives rise to another kind of equivalence,
the equivalence of computational behavior. As previously mentioned
this is typically captured \emph{via} some form of bisimulation.

% The notion we use in this paper is weak barbed bisimulation
% \cite{milner91polyadicpi}.

The notion we use in this paper is derived from weak barbed
bisimulation \cite{milner91polyadicpi}. 

\begin{definition}
An \emph{observation relation}, $\downarrow_{\mathcal N}$, over a set
of names, $\mathcal N$, is the smallest relation satisfying the rules
below.

\infrule[Out-barb]{y \in {\mathcal N}, \; x \nameeq y}
		  {\outputp{x}{v} \downarrow_{\mathcal N} x}
\infrule[Par-barb]{\mbox{$P\downarrow_{\mathcal N} x$ or $Q\downarrow_{\mathcal N} x$}}
		  {\binpar{P}{Q} \downarrow_{\mathcal N} x}

We write $P \Downarrow_{\mathcal N} x$ if there is $Q$ such that 
$P \wred Q$ and $Q \downarrow_{\mathcal N} x$.
\end{definition}

\begin{definition}
%\label{def.bbisim}
An  ${\mathcal N}$-\emph{barbed bisimulation} over a set of names, ${\mathcal N}$, is a symmetric binary relation 
${\mathcal S}_{\mathcal N}$ between agents such that $P\rel{S}_{\mathcal N}Q$ implies:
\begin{enumerate}
\item If $P \red P'$ then $Q \wred Q'$ and $P'\rel{S}_{\mathcal N} Q'$.
\item If $P\downarrow_{\mathcal N} x$, then $Q\Downarrow_{\mathcal N} x$.
\end{enumerate}
$P$ is ${\mathcal N}$-barbed bisimilar to $Q$, written
$P \wbbisim_{\mathcal N} Q$, if $P \rel{S}_{\mathcal N} Q$ for some ${\mathcal N}$-barbed bisimulation ${\mathcal S}_{\mathcal N}$.
\end{definition}

$\mathcal{R} \subseteq \pi \times \pi$

$P \mathcal{R} Q => \forall P'. P \red P' \Rightarrow \exists Q'. Q \red Q', P' \mathcal{R} Q'$

$P \vdash x \Rightarrow Q \vdash x$

\begin{mathpar}
  \inferrule*[lab=Out-barb]{x \nameeq y}{{y}!\langle{Q}\rangle \vdash x}
  \and
  \inferrule*[lab=Par-barb]{\mbox{$P\vdash x$ or $Q\vdash x$}}{\binpar{P}{Q} \vdash x}
\end{mathpar}

\subsubsection{Contexts}

One of the principle advantages of computational calculi like the
$\pi$-calculus is a well-defined notion of context,
contextual-equivalence and a correlation between
contextual-equivalence and notions of bisimulation. The notion of
context allows the decomposition of a process into (sub-)process and
its syntactic environment, its context. Thus, a context may be
thought of as a process with a ``hole'' (written $\Box$) in it. The
application of a context $M$ to a process $P$, written $M[P]$, is
tantamount to filling the hole in $M$ with $P$. In this paper we do
not need the full weight of this theory, but do make use of the notion
of context in the proof the main theorem. 

\begin{mathpar}
  \inferrule* [lab=summation] {} {{M_{M},M_{N}} \bc \Box \;|\; x.M_{A} \;|\; M_{M}+M_{N}}
  \and
  \inferrule* [lab=agent] {} {{M_{A}} \bc (\vec{x})M_{P} \;| \; \clift{P_0,\ldots,M_{P},\ldots,P_N}}
  \and \\
  \inferrule* [lab=process] {} {{M_{P}} \bc M_{N} \;| \;P|M_{P} }
\end{mathpar} 

\begin{mathpar}
  \inferrule* [lab=sychronization] {} {M_{N} \bc \Box \;|\; x?M_{F} \;|\; x!M_{C}}
  \and
  \inferrule* [lab=abstraction] {} {{M_{F}} \bc (x)M_{P} }
  \and
  \inferrule* [lab=concretion] {} {{M_{C}} \bc \langle M_{P} \rangle }
  \and \\
  \inferrule* [lab=process] {} {{M_{P}} \bc M_{N} \;| \;P|M_{P} }
\end{mathpar}

\begin{definition}[contextual application] Given a context $M$, and
  process $P$, we define the \emph{contextual application}, $M[P] :=
  M\{P/\Box\}$. That is, the contextual application of M to P is the
  substitution of $P$ for $\Box$ in $M$.
\end{definition}

$\meaningof{-} : L \to \mathcal{P}(\pi)$

\begin{mathpar}
  \inferrule* [lab=collection] {} {\meaningof{true} = \pi, \and \meaningof{~E} = \pi \setminus \meaningof{E}, \and \meaningof{E_{1} \& E_{2}} = \meaningof{E_{1}} \cap \meaningof{E_{2}}}
\end{mathpar}

\begin{mathpar}
  \inferrule* [lab=structure] {} {\meaningof{0} = \{ P \in \pi | P \equiv 0 \}, \and \\ \meaningof{E_1 | E_2} = \{ P \in \pi | P \equiv P_{1} | P_{2}, P_{1} \in \meaningof{E_{1}}, P_{2} \in \meaningof{E_2}\} }
\end{mathpar}

\begin{mathpar}
 \inferrule* [lab=behavior] {} {\meaningof{\langle a?b \rangle E} = \{ P \in \pi | P \equiv Q | u?(y)P', \\ \and \\\\ \and \\ \;\;\; u \in \meaningof{a}, \forall z.P'\{z/y\} \in \meaningof{E\{z/b\}}\}, \and \\ \meaningof{a!E} = \{ P \in \pi | P \equiv Q | x!\langle P' \rangle, x \in \meaningof{a} P' \in \meaningof{E}\} }
\end{mathpar}

\begin{mathpar}
 \inferrule* [lab=nominal] {} {\meaningof{\quotep{E}} = \{ \quotep{P} \in \quotep{\pi} | P \in \meaningof{E} \}, \and \meaningof{\quotep{P}} = \{ \quotep{Q} \in \quotep{\pi} | P \equiv Q \} \and \\ \meaningof{@\quotep{E}} = \{ P \in \pi | P \equiv @x, x \in \meaningof{E} \}}
\end{mathpar}

\begin{eqnarray*}
  \\
  \meaningof{-} : TS \to ST
\end{eqnarray*}

\begin{eqnarray*}
  \\
  L : TS \to ST
\end{eqnarray*}

\begin{eqnarray*}
  \\
  P \models E \iff P \in \meaningof{E}
\end{eqnarray*}

\begin{eqnarray*}
  P \approx_{L} Q \iff \forall E \in L. P \models E \iff Q \models E
\end{eqnarray*}

\begin{eqnarray*}
  P \approx_{K} Q
\end{eqnarray*}

\begin{eqnarray*}
  P \approx Q
\end{eqnarray*}

$\approx_{K} = \approx = \approx_{L}$

\subsubsection{Contextual duality}

Note that contexts extend the quotation operation to a family of
operations from processes to names. Given a context, $M$, we can
define a \emph{nominal context}, $\quotep{M}$ by $\quotep{M}[P] :=
\quotep{M[P]}$. To foreshadow what is to come we observe that these
operations enjoy a duality with processes very much like the duality
between vectors and maps from vectors to scalars.

Further, because the calculus is essentially higher-order, we have a
correspondence between contexts and processes. More specifically,
given a name $x$ and a context $M$ we can construct $M^{*}_{x}$ such
that 

\begin{mathpar}
  M^{*}_{x} | \lift{x}{P} \red M[P]
\end{mathpar}

namely,

\begin{mathpar}
  M^{*}_{x} := x?(u).M[\dropn{u}]
\end{mathpar}

The dependence of $M^{*}_{x}$ on a name makes it an abstraction, 

\begin{mathpar}
  M^{*} := (x)x?(u).M[\dropn{u}]
\end{mathpar}

\subsection{Additional notation}

It will sometimes be convenient to denote the process a name
quotes. We already have the notation $x = \quotep{P}$, but it will be
convenient to introduce an alternate notation, $\procn{x}$, when we
want to emphasize the connection to the use of the name. Note that, by
virtue of name equivalence, $\quotep{\procn{x}} \nameeq x$; so, the
notation is consistent with previous definitions.

Further, because names have structure it is possible to effect
substitutions on the basis of that structure. This means we need to
upgrade our notation for substitutions, which we accomplish by
adapting comprehension notation. Thus,

\begin{mathpar}
  P\{ y / x : x \in S \}
\end{mathpar}

is interpreted to mean the process derived from P by replacing (in a
capture-avoiding manner) each occurrence of $x$ in $S$ by $y$. For example,

\begin{mathpar}
  P\{ \quotep{\procn{x}|\procn{x}} / x : x \in \freenames{P} \}
\end{mathpar}

will replace each (occurrence) of a free name $x$ in $P$ by
$\quotep{\procn{x}|\procn{x}}$.

Also, we will avail ourselves of the notation $x^{L}$ and $x^{R}$ to
denote injections of a name into disjoint copies of the name
space. There are numerous ways to accomplish this. One example can be
found in \cite{MeredithR05}. This notation overloads to vectors of
names: $\vec{x}^{\pi} := (x_{i}^{\pi} \; : \; 0 \leq i < |\vec{x}| )$ where $\pi \in \{L,R\}$.

We also use $P^{\Box} := P|\Box$.

In \cite{MeredithR05} an interpretation of the new operator is
given. It turns out that there are several possible interpretations
all enjoying the requisite algebraic properties of the operator (see
\cite{milner91polyadicpi}). We will therefore make liberal use of
$(\nu\; \vec{x})P$.

% subsection the_syntax_and_semantics_of_the_notation_system (end)   

\input{qm2pi.qmops} 

\input{qm2pi.sterngerlach} 

\input{qm2pi.metric} 

% section concurrent_process_calculi (end)

%\input{qm2pi.proofsketch}

% section proof sketch (end)

%\input{qm2pi.slviaknots} 

% section spatial logic via knots (end)

\input{qm2pi.conclusion}

% section conclusion (end)

%\input{qm2pi.dtcodes} 

% section wiring algorithm (end)

\input{qm2pi.ack} 

% section acknowledgments (end)

\newpage


\bibliographystyle{plain}   
\bibliography{../../biblios/main.bib}

\input{qm2pi.rhodetails}

\end{document}

 

%\ifpdf
%\usepackage[pdftex]{graphicx}
%\else
%\usepackage{graphicx}
%\fi

 % \ifpdf
%  \usepackage{pdfsync}
%  \if


%\title{Brief Article}
%\author{David F. Snyder}
%\author{L.G. Meredith}

%\address{Dept. of Math., Texas State University--San Marcos, San Marcos, TX 78666}
       
\pagestyle{empty}


\begin{document}

\lstset{language=[Objective]Caml,frame=shadowbox}

\documentclass[12pt]{llncs}
%\documentclass{jktr}

\usepackage[pdftex]{hyperref}                   
\usepackage {listings}
\usepackage {mathpartir}
\usepackage{bcprules}
%\usepackage{listings}
                       
\usepackage{graphicx} 
%\usepackage[margins=2.5cm,nohead,nofoot]{geometry}
%\usepackage{geometry}
\usepackage{amsfonts}
\usepackage{amstext}
\usepackage{latexsym}
\usepackage{amssymb}
\usepackage{color}


%\include{myPreamble}
\include{qm2pi.local} 

%\ifpdf
%\usepackage[pdftex]{graphicx}
%\else
%\usepackage{graphicx}
%\fi

 % \ifpdf
%  \usepackage{pdfsync}
%  \if


%\title{Brief Article}
%\author{David F. Snyder}
%\author{L.G. Meredith}

%\address{Dept. of Math., Texas State University--San Marcos, San Marcos, TX 78666}
       
\pagestyle{empty}


\begin{document}

\lstset{language=[Objective]Caml,frame=shadowbox}

\input{qm2pi.front}

% section front matter (end)

\input{qm2pi.intro} 
 
% section introduction (end)

% \input{qm2pi.knotations} 

% section notation (end)

\input{qm2pi.process.calculi} 

% section concurrent_process_calculi_and_spatial_logics_ (end)
    
%\input{qm2pi.knots2pi} 

%\input{qm2pi.trefoil} 

%\input{qm2pi.mainthm} 

% subsection basic_interpretation (end)

%\input{qm2pi.rho.presentation} 
\subsection{The syntax and semantics of the notation system}\label{sub:the_syntax_and_semantics_of_the_notation_system} % (fold)

We now summarize a technical presentation of the calculus that
embodies our theory of dynamics. The typical presentation of such a
calculus follows the style of giving generators and relations on
them. The grammar, below, describing term constructors, freely
generates the set of processes, $\Proc$. This set is then quotiented
by a relation known as structural congruence and it is over this set
that the notion of dynamics is expressed. This presentation is
essentially that of \cite{MeredithR05} with the addition of
polyadicity and summation. For readability we have relegated some of
the technical subtleties to an appendix.

\subsubsection{Process grammar}\label{subsub:process_grammar}

\begin{mathpar}
  \inferrule* [lab=synchronization] {} {{M} \bc \pzero \;|\; x?F \;|\; x!C }
  \and
  \inferrule* [lab=abstraction] {} {{F} \bc (x)P}
  \and
  \inferrule* [lab=concretion] {} {{C} \bc \langle Q \rangle}
  \and
  \inferrule* [lab=process] {} {{P,Q} \bc M \;| \;P|Q \;|\; @{x}}
  \and
  \inferrule* [lab=name] {} {{x} \bc \quotep{P}}
\end{mathpar} 

Note that $\vec{x}$ (resp. $\vec{P}$) denotes a vector of names
(resp. processes) of length $|\vec{x}|$ (resp. $|\vec{P}|$). We adopt
the following useful abbreviations.

\begin{mathpar}
   x?(\vec{y}).P := x.(\vec{y})P \and  x\clift{\vec{P}} := x.\clift{\vec{P}}
   \and x!(y) := \lift{x}{\dropn{y}}
   \and \Pi_{i=0}^{n-1}P_i := P_0 | \ldots | P_{n-1}
\end{mathpar}

\subsubsection{Structural congruence}

\paragraph{Free and bound names and alpha-equivalence.} At the
core of structural equivalence is alpha-equivalence which identifies
process that are the same up to a change of variable. Formally, we
recognize the distinction between free and bound names. The free names
of a process, $\freenames{P}$, may be calculated recursively as
follows:

\begin{mathpar}
\freenames{\pzero} := \emptyset
  \and \\
  \freenames{x?(y).P} := \{ x \} \cup (\freenames{P} \setminus \{ y \})
  \and 
  \freenames{x!\langle P \rangle} := \{ x \} \cup \{ P \} 
  \and \\
  \freenames{P|Q} := \freenames{P} \cup \freenames{Q}
  \and \\
  \freenames{@{x}} := \{ x \}
\end{mathpar}

$\pi$
$\quotep{\pi}$

$\freenames{-} : \pi \to \mathcal{P}(\quotep{\pi})$

\begin{eqnarray*}
  \freenames{\pzero} & := & \emptyset \\
  \freenames{x?(y).P} & := & \{ x \} \cup (\freenames{P} \setminus \{ y \}) \\
  \freenames{x!\langle P \rangle} & := & \{ x \} \cup \{ P \} \\
  \freenames{P|Q} & := & \freenames{P} \cup \freenames{Q} \\
  \freenames{\dropn{x}} & := & \{ x \}
\end{eqnarray*}

The bound names of a process, $\boundnames{P}$, are those names occurring in $P$
that are not free. For example, in $x?(y).0$, the name $x$ is free, while $y$ is bound.

\begin{mathpar}
  \inferrule* [lab=monoidal-laws] {} { P|Q \equiv Q|P \and P|0 \equiv P \and P|(Q|R) \equiv (P|Q)|R }
\end{mathpar}

\begin{mathpar}
  \inferrule* [lab=alpha-equivalence] {} { (x)P \equiv (y)P\{y/x\} \and y \not\in \freenames{P} }
\end{mathpar}

\begin{definition}
Then two processes, $P,Q$, are alpha-equivalent if $P = Q\{\vec{y}/\vec{x}\}$ for
some $\vec{x} \in \boundnames{Q},\vec{y} \in \boundnames{P}$, where $Q\{\vec{y}/\vec{x}\}$
denotes the capture-avoiding substitution of $\vec{y}$ for $\vec{x}$ in $Q$.
\end{definition}

\begin{definition}
  The {\em structural congruence} \cite{SangiorgiWalker} , $\equiv$,
  between processes is the least congruence containing
  alpha-equivalence, satisfying the abelian monoid laws
  (associativity, commutativity and $\pzero$ as identity) for parallel
  composition $|$ and for summation $+$.
\end{definition}

\subsection{Name equivalence}

We take name equivalence, written $\nameeq$, to be the smallest
equivalence relation generated by the following rules.

\begin{mathpar}
\inferrule*[lab=Quote-drop]
{ }
{ \quotep{@{x}} \nameeq x }

\inferrule*[lab=Struct-equiv]
{ P \scong Q }
{ \quotep{P} \nameeq \quotep{Q} }
\end{mathpar}

The astute reader will have noticed that the mutual recursion of names
and processes imposes a mutual recursion on alpha-equivalence and
structural equivalence via name-equivalence. Fortunately, all of this
works out pleasantly and we may calculate in the natural way, free of
concern. The reader interested in the details is referred to the
appendix \ref{appendix:rho_details}.

\subsection{Substitution}

We use $\Proc$ for the set of processes, $\QProc$ for the set of
names, and $\id{\{}\vec{y} / \vec{x} \id{\}}$ to denote partial maps,
$s : \QProc \rightarrow \QProc$. A map, $s$ lifts, uniquely, to a map
on process terms, $\widehat{s} : \Proc \rightarrow \Proc$ by the
following equations.

\begin{mathpar}
  (0) \psubstp{Q}{P} := 0 \\
  (R \juxtap S) \psubstp{Q}{P}
  :=    
  (R)\psubstp{Q}{P} \juxtap (S) \psubstp{Q}{P} \\
  (x?(y).R) \psubstp{Q}{P}    
  :=    
  (x)\substp{Q}{P} (z)\concat( (R \psubstn{z}{y}) \psubstp{Q}{P} ) \\
  (\lift{x}{R}) \psubstp{Q}{P}  
  :=
  \lift{(x)\substp{Q}{P}}{ R \psubstp{Q}{P} } \\
%   (\dropn{x})  \psubstp{Q}{P}       
%   := 
%   \left\{ 
%     \begin{array}{ccc} 
%       \dropn{\quotep{Q}} & & x \nameeq \quotep{P} \\
%       \dropn{x} & & otherwise \\
%     \end{array}
%   \right. 
  (\dropn{x})  \psubstp{Q}{P}       
  := 
  \left\{ 
    \begin{array}{ccc} 
      Q & & x \nameeq \quotep{P} \\
      \dropn{x} & & otherwise \\
    \end{array}
  \right.
\end{mathpar}
 

where

\begin{eqnarray}
  (x)\id{\{} \lpquote Q \rpquote / \lpquote P \rpquote \id{\}}            = 
  \left\{ 
    \begin{array}{ccc}
      \lpquote Q \rpquote & & x \nameeq \lpquote P \rpquote \\
      x & & otherwise \\
    \end{array}
  \right. \nonumber
\end{eqnarray}

and $z$ is chosen distinct from $\quotep{P}$, $\quotep{Q}$, the free
names in $Q$, and all the names in $R$. Our $\alpha$-equivalence will
be built in the standard way from this substitution.

\begin{remark}\label{rem:no_self_referential_names}
  One consequence of these definitions is that $\forall P. \quotep{P}
  \not\in \freenames{P}$.
\end{remark}

\subsection{ Dynamic quote: an example }

Anticipating something of what's to come, consider applying the
substitution, $\widehat{\id{\{}u / z \id{\}}}$, to the following pair
of processes, $\lift{w}{y!(z)}$ and $w[ \lpquote y!(z) \rpquote ]$.

\begin{eqnarray}
	\lift{w}{y!(z)}\widehat{\id{\{}u / z \id{\}}}
		& = &
		\lift{w}{y!(u)} \nonumber\\
	w[ \lpquote y!(z) \rpquote ] \widehat{ \id{\{}u / z \id{\}} }
		& = &
		w[ \lpquote y!(z) \rpquote ] \nonumber
\end{eqnarray}

Because the body of the process between quotes is impervious to
substitution, we get radically different answers. In fact, by
examining the first process in an input context,
e.g. $x?(z).\lift{w}{y!(z)}$, we see that the process under the lift
operator may be shaped by prefixed inputs binding a name inside it. In
this sense, the lift operator will be seen as a way to dynamically
construct processes before reifying them as names.

Finally equipped with these standard features we can present the
dynamics of the calculus.

\subsubsection{Operational semantics} 

Finally, we introduce the computational dynamics. What marks these
algebras as distinct from other more traditionally studied algebraic
structures, e.g. vector spaces or polynomial rings, is the manner in
which dynamics is captured. In traditional structures, dynamics is typically
expressed through morphisms between such structures, as in linear maps
between vector spaces or morphisms between rings. In algebras
associated with the semantics of computation, the dynamics is
expressed as part of the algebraic structure itself, through a
reduction reduction relation typically denoted by $\red$. Below, we
give a recursive presentation of this relation for the calculus used
in the encoding.

$\red \subseteq \pi \times \pi$
$\red : \pi \to \mathcal{P}(\pi)$

\begin{mathpar}
  \inferrule* [lab=Comm] { \textsf{match}( x_{src}, x_{trgt} ) } { x_{trgt}?(y)P \; | \; x_{src}!\langle {Q} \rangle \red P\{\quotep{Q}/y}\} }
  \and \\
  \inferrule* [lab=Par] {{P} \red {P}'} {{{P} | {Q}} \red {{P}' | {Q}}}
  \and
  \inferrule* [lab=Equiv]{{{P} \scong {P}'} \andalso {{P}' \red {Q}'} \andalso {{Q}' \scong {Q}}}{{P} \red {Q}}
\end{mathpar}

\begin{eqnarray*}
  match_{\equiv} (\quotep{P},\quotep{Q}) & := & P \equiv Q \\
  match_{\dagger}(\quotep{P},\quotep{Q}) & := & \forall R. P|Q \red^{*} R => R \red^{*} 0 \\
  match_{K}(\quotep{P},\quotep{Q}) & := & K \mbox{ for some context } K
\end{eqnarray*}

$u?(x)P | u!\langle Q \rangle \red P\{\quotep{Q}/x\}$

%We write $\wred$ for $\red^*$, and $P\red$ if $\exists Q $ such that $ P \red Q$.
We write $P\red$ if $\exists Q $ such that $ P \red Q$ and $P\not\red$, otherwise.

\section{Replication}

As mentioned before, it is known that replication (and hence
recursion) can be implemented in a higher-order process algebra
\cite{SangiorgiWalker}. As our first example of calculation with the
machinery thus far presented we give the construction explicitly in
the {\rhoc}.

\begin{eqnarray}
	D_{x} & := & \prefix{x}{y}{(\binpar{\outputp{x}{y}}{@{y}})} \nonumber\\
	\bangp_{x}{P} & := & \binpar{{x}!\langle{\binpar{D_{x}}{P}}\rangle}{D_{x}} \nonumber
\end{eqnarray}

\begin{eqnarray}
	\bangp_{x}{P} & & \nonumber\\
	=
	& {x}!\langle{(\prefix{x}{y}{(\outputp{x}{y} | @{y})) | P}}\rangle 
	      | \prefix{x}{y}{(\outputp{x}{y} | @{y})} & \nonumber\\
	\red
	& (\outputp{x}{y} | @{y})\substn{\quotep{(\prefix{x}{y}{(@{y} | \outputp{x}{y})) | P}}}{y} & \nonumber\\
	=
	& \outputp{x}{\quotep{(\prefix{x}{y}{(\outputp{x}{y} | @{y})) | P}}}
	  | {(\prefix{x}{y}{(\outputp{x}{y} | @{y})) | P}} & \nonumber\\
	\red
	& \ldots & \nonumber\\
	\red^*
	& P | P | \ldots & \nonumber
\end{eqnarray}

Of course, this encoding, as an implementation, runs away, unfolding
$\bangp{P}$ eagerly. A lazier and more implementable replication
operator, restricted to input-guarded processes, may be obtained as follows.

\begin{eqnarray}
\bangp{\prefix{u}{v}{P}} 
	:= 
	\binpar{\lift{x}{\prefix{u}{v}{(\binpar{D(x)}{P})}}}{D(x)} \nonumber
\end{eqnarray}

\begin{remark}
  Note that the lazier definition still does not deal with summation
  or mixed summation (i.e. sums over input and output). The reader is
  invited to construct definitions of replication that deal with these
  features. 

  Further, the definitions are parameterized in a name, $x$. Can you,
  gentle reader, make a definition that eliminates this parameter and
  guarantees no accidental interaction between the replication
  machinery and the process being replicated -- i.e. no accidental
  sharing of names used by the process to get its work done and the
  name(s) used by the replication to effect copying. This latter
  revision of the definition of replication is crucial to obtaining
  the expected identity $!!P \sim !P$.
\end{remark}

\begin{remark}\label{rem:paradoxical_combinator}
  The reader familiar with the lambda calculus will have noticed the
  similarity between $D$ and the paradoxical combinator.

  [Ed. note: the existence of this seems to suggest we have to be more
  restrictive on the set of processes and names we admit if we are to
  support no-cloning.]
\end{remark}

\subsubsection{Bisimulation}

The computational dynamics gives rise to another kind of equivalence,
the equivalence of computational behavior. As previously mentioned
this is typically captured \emph{via} some form of bisimulation.

% The notion we use in this paper is weak barbed bisimulation
% \cite{milner91polyadicpi}.

The notion we use in this paper is derived from weak barbed
bisimulation \cite{milner91polyadicpi}. 

\begin{definition}
An \emph{observation relation}, $\downarrow_{\mathcal N}$, over a set
of names, $\mathcal N$, is the smallest relation satisfying the rules
below.

\infrule[Out-barb]{y \in {\mathcal N}, \; x \nameeq y}
		  {\outputp{x}{v} \downarrow_{\mathcal N} x}
\infrule[Par-barb]{\mbox{$P\downarrow_{\mathcal N} x$ or $Q\downarrow_{\mathcal N} x$}}
		  {\binpar{P}{Q} \downarrow_{\mathcal N} x}

We write $P \Downarrow_{\mathcal N} x$ if there is $Q$ such that 
$P \wred Q$ and $Q \downarrow_{\mathcal N} x$.
\end{definition}

\begin{definition}
%\label{def.bbisim}
An  ${\mathcal N}$-\emph{barbed bisimulation} over a set of names, ${\mathcal N}$, is a symmetric binary relation 
${\mathcal S}_{\mathcal N}$ between agents such that $P\rel{S}_{\mathcal N}Q$ implies:
\begin{enumerate}
\item If $P \red P'$ then $Q \wred Q'$ and $P'\rel{S}_{\mathcal N} Q'$.
\item If $P\downarrow_{\mathcal N} x$, then $Q\Downarrow_{\mathcal N} x$.
\end{enumerate}
$P$ is ${\mathcal N}$-barbed bisimilar to $Q$, written
$P \wbbisim_{\mathcal N} Q$, if $P \rel{S}_{\mathcal N} Q$ for some ${\mathcal N}$-barbed bisimulation ${\mathcal S}_{\mathcal N}$.
\end{definition}

$\mathcal{R} \subseteq \pi \times \pi$

$P \mathcal{R} Q => \forall P'. P \red P' \Rightarrow \exists Q'. Q \red Q', P' \mathcal{R} Q'$

$P \vdash x \Rightarrow Q \vdash x$

\begin{mathpar}
  \inferrule*[lab=Out-barb]{x \nameeq y}{{y}!\langle{Q}\rangle \vdash x}
  \and
  \inferrule*[lab=Par-barb]{\mbox{$P\vdash x$ or $Q\vdash x$}}{\binpar{P}{Q} \vdash x}
\end{mathpar}

\subsubsection{Contexts}

One of the principle advantages of computational calculi like the
$\pi$-calculus is a well-defined notion of context,
contextual-equivalence and a correlation between
contextual-equivalence and notions of bisimulation. The notion of
context allows the decomposition of a process into (sub-)process and
its syntactic environment, its context. Thus, a context may be
thought of as a process with a ``hole'' (written $\Box$) in it. The
application of a context $M$ to a process $P$, written $M[P]$, is
tantamount to filling the hole in $M$ with $P$. In this paper we do
not need the full weight of this theory, but do make use of the notion
of context in the proof the main theorem. 

\begin{mathpar}
  \inferrule* [lab=summation] {} {{M_{M},M_{N}} \bc \Box \;|\; x.M_{A} \;|\; M_{M}+M_{N}}
  \and
  \inferrule* [lab=agent] {} {{M_{A}} \bc (\vec{x})M_{P} \;| \; \clift{P_0,\ldots,M_{P},\ldots,P_N}}
  \and \\
  \inferrule* [lab=process] {} {{M_{P}} \bc M_{N} \;| \;P|M_{P} }
\end{mathpar} 

\begin{mathpar}
  \inferrule* [lab=sychronization] {} {M_{N} \bc \Box \;|\; x?M_{F} \;|\; x!M_{C}}
  \and
  \inferrule* [lab=abstraction] {} {{M_{F}} \bc (x)M_{P} }
  \and
  \inferrule* [lab=concretion] {} {{M_{C}} \bc \langle M_{P} \rangle }
  \and \\
  \inferrule* [lab=process] {} {{M_{P}} \bc M_{N} \;| \;P|M_{P} }
\end{mathpar}

\begin{definition}[contextual application] Given a context $M$, and
  process $P$, we define the \emph{contextual application}, $M[P] :=
  M\{P/\Box\}$. That is, the contextual application of M to P is the
  substitution of $P$ for $\Box$ in $M$.
\end{definition}

$\meaningof{-} : L \to \mathcal{P}(\pi)$

\begin{mathpar}
  \inferrule* [lab=collection] {} {\meaningof{true} = \pi, \and \meaningof{~E} = \pi \setminus \meaningof{E}, \and \meaningof{E_{1} \& E_{2}} = \meaningof{E_{1}} \cap \meaningof{E_{2}}}
\end{mathpar}

\begin{mathpar}
  \inferrule* [lab=structure] {} {\meaningof{0} = \{ P \in \pi | P \equiv 0 \}, \and \\ \meaningof{E_1 | E_2} = \{ P \in \pi | P \equiv P_{1} | P_{2}, P_{1} \in \meaningof{E_{1}}, P_{2} \in \meaningof{E_2}\} }
\end{mathpar}

\begin{mathpar}
 \inferrule* [lab=behavior] {} {\meaningof{\langle a?b \rangle E} = \{ P \in \pi | P \equiv Q | u?(y)P', \\ \and \\\\ \and \\ \;\;\; u \in \meaningof{a}, \forall z.P'\{z/y\} \in \meaningof{E\{z/b\}}\}, \and \\ \meaningof{a!E} = \{ P \in \pi | P \equiv Q | x!\langle P' \rangle, x \in \meaningof{a} P' \in \meaningof{E}\} }
\end{mathpar}

\begin{mathpar}
 \inferrule* [lab=nominal] {} {\meaningof{\quotep{E}} = \{ \quotep{P} \in \quotep{\pi} | P \in \meaningof{E} \}, \and \meaningof{\quotep{P}} = \{ \quotep{Q} \in \quotep{\pi} | P \equiv Q \} \and \\ \meaningof{@\quotep{E}} = \{ P \in \pi | P \equiv @x, x \in \meaningof{E} \}}
\end{mathpar}

\begin{eqnarray*}
  \\
  \meaningof{-} : TS \to ST
\end{eqnarray*}

\begin{eqnarray*}
  \\
  L : TS \to ST
\end{eqnarray*}

\begin{eqnarray*}
  \\
  P \models E \iff P \in \meaningof{E}
\end{eqnarray*}

\begin{eqnarray*}
  P \approx_{L} Q \iff \forall E \in L. P \models E \iff Q \models E
\end{eqnarray*}

\begin{eqnarray*}
  P \approx_{K} Q
\end{eqnarray*}

\begin{eqnarray*}
  P \approx Q
\end{eqnarray*}

$\approx_{K} = \approx = \approx_{L}$

\subsubsection{Contextual duality}

Note that contexts extend the quotation operation to a family of
operations from processes to names. Given a context, $M$, we can
define a \emph{nominal context}, $\quotep{M}$ by $\quotep{M}[P] :=
\quotep{M[P]}$. To foreshadow what is to come we observe that these
operations enjoy a duality with processes very much like the duality
between vectors and maps from vectors to scalars.

Further, because the calculus is essentially higher-order, we have a
correspondence between contexts and processes. More specifically,
given a name $x$ and a context $M$ we can construct $M^{*}_{x}$ such
that 

\begin{mathpar}
  M^{*}_{x} | \lift{x}{P} \red M[P]
\end{mathpar}

namely,

\begin{mathpar}
  M^{*}_{x} := x?(u).M[\dropn{u}]
\end{mathpar}

The dependence of $M^{*}_{x}$ on a name makes it an abstraction, 

\begin{mathpar}
  M^{*} := (x)x?(u).M[\dropn{u}]
\end{mathpar}

\subsection{Additional notation}

It will sometimes be convenient to denote the process a name
quotes. We already have the notation $x = \quotep{P}$, but it will be
convenient to introduce an alternate notation, $\procn{x}$, when we
want to emphasize the connection to the use of the name. Note that, by
virtue of name equivalence, $\quotep{\procn{x}} \nameeq x$; so, the
notation is consistent with previous definitions.

Further, because names have structure it is possible to effect
substitutions on the basis of that structure. This means we need to
upgrade our notation for substitutions, which we accomplish by
adapting comprehension notation. Thus,

\begin{mathpar}
  P\{ y / x : x \in S \}
\end{mathpar}

is interpreted to mean the process derived from P by replacing (in a
capture-avoiding manner) each occurrence of $x$ in $S$ by $y$. For example,

\begin{mathpar}
  P\{ \quotep{\procn{x}|\procn{x}} / x : x \in \freenames{P} \}
\end{mathpar}

will replace each (occurrence) of a free name $x$ in $P$ by
$\quotep{\procn{x}|\procn{x}}$.

Also, we will avail ourselves of the notation $x^{L}$ and $x^{R}$ to
denote injections of a name into disjoint copies of the name
space. There are numerous ways to accomplish this. One example can be
found in \cite{MeredithR05}. This notation overloads to vectors of
names: $\vec{x}^{\pi} := (x_{i}^{\pi} \; : \; 0 \leq i < |\vec{x}| )$ where $\pi \in \{L,R\}$.

We also use $P^{\Box} := P|\Box$.

In \cite{MeredithR05} an interpretation of the new operator is
given. It turns out that there are several possible interpretations
all enjoying the requisite algebraic properties of the operator (see
\cite{milner91polyadicpi}). We will therefore make liberal use of
$(\nu\; \vec{x})P$.

% subsection the_syntax_and_semantics_of_the_notation_system (end)   

\input{qm2pi.qmops} 

\input{qm2pi.sterngerlach} 

\input{qm2pi.metric} 

% section concurrent_process_calculi (end)

%\input{qm2pi.proofsketch}

% section proof sketch (end)

%\input{qm2pi.slviaknots} 

% section spatial logic via knots (end)

\input{qm2pi.conclusion}

% section conclusion (end)

%\input{qm2pi.dtcodes} 

% section wiring algorithm (end)

\input{qm2pi.ack} 

% section acknowledgments (end)

\newpage


\bibliographystyle{plain}   
\bibliography{../../biblios/main.bib}

\input{qm2pi.rhodetails}

\end{document}



% section front matter (end)

\section{Introduction}\label{sec:introduction} % (fold)
In this draft of the material i am going to have to dispense with the
usual writing conventions adopted in papers on these topics. i'm going
to have adopt whatever tone i need at the time i'm writing up the
calculations. Sometimes this may be very conversational; others it may
be the barest mathematical grunts; others still it may be that i have
lifted text from one of my other papers because the exposition of some
point was better said there. i hope that my readers are not unduly put
out by this decision. i'm not doing this to flout convention or be
rebellious. i find these calculations very technically challenging. To
keep everything going technically, something has to give; i have to
let go of some cognitive burden. So, the academic writing style --
with all of its trade-offs in terms of facilitating technical
communication -- is what i'm letting go of. Perhaps subsequent drafts
can be tightened and polished, but for now, i'm going to speak as if
we were sitting together in a coffee shop with a laptop, wifi and a
pad of paper and a pencil.

So, here's what i have to say. We -- you and i, comfortably ensconced
in our coffee shop and well-equipped with our tools -- can realize and
carry out the calculations of quantum mechanics over a very different
formal theory of dynamics, a formal theory of dynamics that
corresponds to a theory of concurrent computation with
\emph{reflection}. It has the advantage that the underlying theory is
already `quantized', but supports analogues all of the continuuous
operations. Strikingly, this underlying theory has recently been
connected with a notion of metric that we can show, by calculating
together, coincides with the metric induced by the inner product.

There are a lot of reasons why you might be interested in seeing
calculations of this form. Here's why i'm interested. For the past
several centuries there has been no competitor to the ``Newtonian''
account of dynamics. As a result the predominant share of accounts of
dynamical systems and situations have had to be formulated in terms of
the Newtonian machinery. i view this as an intellectually dangerous
position to occupy. Everything, despite it's intrinsic shape, turns
into a nail to be hit with this hammer. Recently, however, the theory
of computation has matured to the point where we have candidates for
theories of dynamics that offer very different perspective on
reasoning about dynamical systems and situations. Testing these
candidates against very successful accounts of dynamical situations,
like quantum mechanics, is going to give us some sense of how mature
they are and some measure of the quality of these accounts of
dynamics.

\subsection{Summary of contributions and outline of paper}

So, we're going to develop an interpretation of the operations of
quantum mechanics normally interpreted by Hilbert spaces and
operators. We're going to do this over a theory of computation. Note
that this is very different than the usual quantum computation program
which develops notions of computation over quantum mechanics. Rather,
we are developing a story that aligns with Wheeler's slogan: It from
Bit. To do this we will first provide an account of the theory of
computation at play here. Then we will dive into a calculation-driven
interpretation of the operations of quantum mechanics.

The reason we take this approach is that -- until very recently --
there hasn't been an axiomatic account of quantum mechanics. As a
result there has been no sharp delineation of the mathematical theory
supporting interpretation of the physical theory and the physical
theory, itself. So, ambient features of the maths are free to be
exploited (or supressed) without a real accounting of their physical
relevance. There is no sharp statement ``here's the physical theory''
qua \emph{theory} and ``here's the mathematical interpretation''
enabling a judgment of how faithful the interpretation is -- apart
from experimental observation. When there is an axiomatic account we
can judge how well a given mathematical formalism supports an
interpretation of the axioms, independent of
experimentation. Likewise, we can judge how well we have captured our
physical evidence and experience with our axiomatics, independent of
any specific mathematical implementation, with accidental detail that
may or may not have physical significance. 

In lieu of a fully fleshed out and vetted axiomatic account of quantum
mechanics, interpreting the operational notions in service of modeling
physical systems will have to suffice. In other words, we are not in
the business of providing a model of Hilbert spaces and operators. We
are in the business of providing a model of quantum mechanics because
we are motivated by testing our notions of dynamics against physical
theory; and, the predictive calculations of the physical theory must
serve as the best formulation -- shy of a fully fleshed out axiomatic
account -- of the physical theory itself (as they have for scientific
theories since time immemorial). Put another way, despite a
whole-hearted commitment to an It-from-Bit ontology, we are firmly
aligned with the shut-up-and-calculate camp as the best way to obtain
results either from the physical perspective or as a quality assurance
measure of our fledgling theory of dynamics.

In detail, we present a reflective process calculus. Then we develop
intuitive correspondences between the notions available in this
calculus and the usual physical notions supporting quantum mechanical
calculations. Thus, 

\begin{table}[htp]
  \center{
    \fbox{
      \begin{tabular}{c|c}
        quantum mechanics & process calculus \\
        \hline
        scalar & name \\
        state vector & process \\
        dual & contextual duals \\
        matrix & formal sums of process-context-dual pairs \\
        orthogonality & process annihilation \\
        inner product & execution-formula + quoting
      \end{tabular}
    }
  }
  \caption{QM - process calculi correspondences}
\end{table}

Then we tighten up these intuitions to operational definitions. We
employ the Dirac notation as the best proxy we can find for an
abstract syntax of the quantum mechanical notions. The definitions we
develop put us in contact with equational constraints coming from the
theory that we demonstrate the definitions and calculations satisfy.

This puts us in a position to shut up and calculate for the
Stern-Gerlach experimental set up, showing how these predictive
calculations become calculations on processes in our theory of a
reflective process calculus.

Penultimately, we demonstrate that the notion of metric coming from
the inner product coincides with the notion of metric available from
the theory of bisimulation. This demonstration gives us the right to
think of space as arising from behavior. Finally, we consider where we
might go from the new vantage point we have obtained.

% section introduction (end) 
 
% section introduction (end)

% \documentclass[12pt]{llncs}
%\documentclass{jktr}

\usepackage[pdftex]{hyperref}                   
\usepackage {listings}
\usepackage {mathpartir}
\usepackage{bcprules}
%\usepackage{listings}
                       
\usepackage{graphicx} 
%\usepackage[margins=2.5cm,nohead,nofoot]{geometry}
%\usepackage{geometry}
\usepackage{amsfonts}
\usepackage{amstext}
\usepackage{latexsym}
\usepackage{amssymb}
\usepackage{color}


%\include{myPreamble}
\include{qm2pi.local} 

%\ifpdf
%\usepackage[pdftex]{graphicx}
%\else
%\usepackage{graphicx}
%\fi

 % \ifpdf
%  \usepackage{pdfsync}
%  \if


%\title{Brief Article}
%\author{David F. Snyder}
%\author{L.G. Meredith}

%\address{Dept. of Math., Texas State University--San Marcos, San Marcos, TX 78666}
       
\pagestyle{empty}


\begin{document}

\lstset{language=[Objective]Caml,frame=shadowbox}

\input{qm2pi.front}

% section front matter (end)

\input{qm2pi.intro} 
 
% section introduction (end)

% \input{qm2pi.knotations} 

% section notation (end)

\input{qm2pi.process.calculi} 

% section concurrent_process_calculi_and_spatial_logics_ (end)
    
%\input{qm2pi.knots2pi} 

%\input{qm2pi.trefoil} 

%\input{qm2pi.mainthm} 

% subsection basic_interpretation (end)

%\input{qm2pi.rho.presentation} 
\subsection{The syntax and semantics of the notation system}\label{sub:the_syntax_and_semantics_of_the_notation_system} % (fold)

We now summarize a technical presentation of the calculus that
embodies our theory of dynamics. The typical presentation of such a
calculus follows the style of giving generators and relations on
them. The grammar, below, describing term constructors, freely
generates the set of processes, $\Proc$. This set is then quotiented
by a relation known as structural congruence and it is over this set
that the notion of dynamics is expressed. This presentation is
essentially that of \cite{MeredithR05} with the addition of
polyadicity and summation. For readability we have relegated some of
the technical subtleties to an appendix.

\subsubsection{Process grammar}\label{subsub:process_grammar}

\begin{mathpar}
  \inferrule* [lab=synchronization] {} {{M} \bc \pzero \;|\; x?F \;|\; x!C }
  \and
  \inferrule* [lab=abstraction] {} {{F} \bc (x)P}
  \and
  \inferrule* [lab=concretion] {} {{C} \bc \langle Q \rangle}
  \and
  \inferrule* [lab=process] {} {{P,Q} \bc M \;| \;P|Q \;|\; @{x}}
  \and
  \inferrule* [lab=name] {} {{x} \bc \quotep{P}}
\end{mathpar} 

Note that $\vec{x}$ (resp. $\vec{P}$) denotes a vector of names
(resp. processes) of length $|\vec{x}|$ (resp. $|\vec{P}|$). We adopt
the following useful abbreviations.

\begin{mathpar}
   x?(\vec{y}).P := x.(\vec{y})P \and  x\clift{\vec{P}} := x.\clift{\vec{P}}
   \and x!(y) := \lift{x}{\dropn{y}}
   \and \Pi_{i=0}^{n-1}P_i := P_0 | \ldots | P_{n-1}
\end{mathpar}

\subsubsection{Structural congruence}

\paragraph{Free and bound names and alpha-equivalence.} At the
core of structural equivalence is alpha-equivalence which identifies
process that are the same up to a change of variable. Formally, we
recognize the distinction between free and bound names. The free names
of a process, $\freenames{P}$, may be calculated recursively as
follows:

\begin{mathpar}
\freenames{\pzero} := \emptyset
  \and \\
  \freenames{x?(y).P} := \{ x \} \cup (\freenames{P} \setminus \{ y \})
  \and 
  \freenames{x!\langle P \rangle} := \{ x \} \cup \{ P \} 
  \and \\
  \freenames{P|Q} := \freenames{P} \cup \freenames{Q}
  \and \\
  \freenames{@{x}} := \{ x \}
\end{mathpar}

$\pi$
$\quotep{\pi}$

$\freenames{-} : \pi \to \mathcal{P}(\quotep{\pi})$

\begin{eqnarray*}
  \freenames{\pzero} & := & \emptyset \\
  \freenames{x?(y).P} & := & \{ x \} \cup (\freenames{P} \setminus \{ y \}) \\
  \freenames{x!\langle P \rangle} & := & \{ x \} \cup \{ P \} \\
  \freenames{P|Q} & := & \freenames{P} \cup \freenames{Q} \\
  \freenames{\dropn{x}} & := & \{ x \}
\end{eqnarray*}

The bound names of a process, $\boundnames{P}$, are those names occurring in $P$
that are not free. For example, in $x?(y).0$, the name $x$ is free, while $y$ is bound.

\begin{mathpar}
  \inferrule* [lab=monoidal-laws] {} { P|Q \equiv Q|P \and P|0 \equiv P \and P|(Q|R) \equiv (P|Q)|R }
\end{mathpar}

\begin{mathpar}
  \inferrule* [lab=alpha-equivalence] {} { (x)P \equiv (y)P\{y/x\} \and y \not\in \freenames{P} }
\end{mathpar}

\begin{definition}
Then two processes, $P,Q$, are alpha-equivalent if $P = Q\{\vec{y}/\vec{x}\}$ for
some $\vec{x} \in \boundnames{Q},\vec{y} \in \boundnames{P}$, where $Q\{\vec{y}/\vec{x}\}$
denotes the capture-avoiding substitution of $\vec{y}$ for $\vec{x}$ in $Q$.
\end{definition}

\begin{definition}
  The {\em structural congruence} \cite{SangiorgiWalker} , $\equiv$,
  between processes is the least congruence containing
  alpha-equivalence, satisfying the abelian monoid laws
  (associativity, commutativity and $\pzero$ as identity) for parallel
  composition $|$ and for summation $+$.
\end{definition}

\subsection{Name equivalence}

We take name equivalence, written $\nameeq$, to be the smallest
equivalence relation generated by the following rules.

\begin{mathpar}
\inferrule*[lab=Quote-drop]
{ }
{ \quotep{@{x}} \nameeq x }

\inferrule*[lab=Struct-equiv]
{ P \scong Q }
{ \quotep{P} \nameeq \quotep{Q} }
\end{mathpar}

The astute reader will have noticed that the mutual recursion of names
and processes imposes a mutual recursion on alpha-equivalence and
structural equivalence via name-equivalence. Fortunately, all of this
works out pleasantly and we may calculate in the natural way, free of
concern. The reader interested in the details is referred to the
appendix \ref{appendix:rho_details}.

\subsection{Substitution}

We use $\Proc$ for the set of processes, $\QProc$ for the set of
names, and $\id{\{}\vec{y} / \vec{x} \id{\}}$ to denote partial maps,
$s : \QProc \rightarrow \QProc$. A map, $s$ lifts, uniquely, to a map
on process terms, $\widehat{s} : \Proc \rightarrow \Proc$ by the
following equations.

\begin{mathpar}
  (0) \psubstp{Q}{P} := 0 \\
  (R \juxtap S) \psubstp{Q}{P}
  :=    
  (R)\psubstp{Q}{P} \juxtap (S) \psubstp{Q}{P} \\
  (x?(y).R) \psubstp{Q}{P}    
  :=    
  (x)\substp{Q}{P} (z)\concat( (R \psubstn{z}{y}) \psubstp{Q}{P} ) \\
  (\lift{x}{R}) \psubstp{Q}{P}  
  :=
  \lift{(x)\substp{Q}{P}}{ R \psubstp{Q}{P} } \\
%   (\dropn{x})  \psubstp{Q}{P}       
%   := 
%   \left\{ 
%     \begin{array}{ccc} 
%       \dropn{\quotep{Q}} & & x \nameeq \quotep{P} \\
%       \dropn{x} & & otherwise \\
%     \end{array}
%   \right. 
  (\dropn{x})  \psubstp{Q}{P}       
  := 
  \left\{ 
    \begin{array}{ccc} 
      Q & & x \nameeq \quotep{P} \\
      \dropn{x} & & otherwise \\
    \end{array}
  \right.
\end{mathpar}
 

where

\begin{eqnarray}
  (x)\id{\{} \lpquote Q \rpquote / \lpquote P \rpquote \id{\}}            = 
  \left\{ 
    \begin{array}{ccc}
      \lpquote Q \rpquote & & x \nameeq \lpquote P \rpquote \\
      x & & otherwise \\
    \end{array}
  \right. \nonumber
\end{eqnarray}

and $z$ is chosen distinct from $\quotep{P}$, $\quotep{Q}$, the free
names in $Q$, and all the names in $R$. Our $\alpha$-equivalence will
be built in the standard way from this substitution.

\begin{remark}\label{rem:no_self_referential_names}
  One consequence of these definitions is that $\forall P. \quotep{P}
  \not\in \freenames{P}$.
\end{remark}

\subsection{ Dynamic quote: an example }

Anticipating something of what's to come, consider applying the
substitution, $\widehat{\id{\{}u / z \id{\}}}$, to the following pair
of processes, $\lift{w}{y!(z)}$ and $w[ \lpquote y!(z) \rpquote ]$.

\begin{eqnarray}
	\lift{w}{y!(z)}\widehat{\id{\{}u / z \id{\}}}
		& = &
		\lift{w}{y!(u)} \nonumber\\
	w[ \lpquote y!(z) \rpquote ] \widehat{ \id{\{}u / z \id{\}} }
		& = &
		w[ \lpquote y!(z) \rpquote ] \nonumber
\end{eqnarray}

Because the body of the process between quotes is impervious to
substitution, we get radically different answers. In fact, by
examining the first process in an input context,
e.g. $x?(z).\lift{w}{y!(z)}$, we see that the process under the lift
operator may be shaped by prefixed inputs binding a name inside it. In
this sense, the lift operator will be seen as a way to dynamically
construct processes before reifying them as names.

Finally equipped with these standard features we can present the
dynamics of the calculus.

\subsubsection{Operational semantics} 

Finally, we introduce the computational dynamics. What marks these
algebras as distinct from other more traditionally studied algebraic
structures, e.g. vector spaces or polynomial rings, is the manner in
which dynamics is captured. In traditional structures, dynamics is typically
expressed through morphisms between such structures, as in linear maps
between vector spaces or morphisms between rings. In algebras
associated with the semantics of computation, the dynamics is
expressed as part of the algebraic structure itself, through a
reduction reduction relation typically denoted by $\red$. Below, we
give a recursive presentation of this relation for the calculus used
in the encoding.

$\red \subseteq \pi \times \pi$
$\red : \pi \to \mathcal{P}(\pi)$

\begin{mathpar}
  \inferrule* [lab=Comm] { \textsf{match}( x_{src}, x_{trgt} ) } { x_{trgt}?(y)P \; | \; x_{src}!\langle {Q} \rangle \red P\{\quotep{Q}/y}\} }
  \and \\
  \inferrule* [lab=Par] {{P} \red {P}'} {{{P} | {Q}} \red {{P}' | {Q}}}
  \and
  \inferrule* [lab=Equiv]{{{P} \scong {P}'} \andalso {{P}' \red {Q}'} \andalso {{Q}' \scong {Q}}}{{P} \red {Q}}
\end{mathpar}

\begin{eqnarray*}
  match_{\equiv} (\quotep{P},\quotep{Q}) & := & P \equiv Q \\
  match_{\dagger}(\quotep{P},\quotep{Q}) & := & \forall R. P|Q \red^{*} R => R \red^{*} 0 \\
  match_{K}(\quotep{P},\quotep{Q}) & := & K \mbox{ for some context } K
\end{eqnarray*}

$u?(x)P | u!\langle Q \rangle \red P\{\quotep{Q}/x\}$

%We write $\wred$ for $\red^*$, and $P\red$ if $\exists Q $ such that $ P \red Q$.
We write $P\red$ if $\exists Q $ such that $ P \red Q$ and $P\not\red$, otherwise.

\section{Replication}

As mentioned before, it is known that replication (and hence
recursion) can be implemented in a higher-order process algebra
\cite{SangiorgiWalker}. As our first example of calculation with the
machinery thus far presented we give the construction explicitly in
the {\rhoc}.

\begin{eqnarray}
	D_{x} & := & \prefix{x}{y}{(\binpar{\outputp{x}{y}}{@{y}})} \nonumber\\
	\bangp_{x}{P} & := & \binpar{{x}!\langle{\binpar{D_{x}}{P}}\rangle}{D_{x}} \nonumber
\end{eqnarray}

\begin{eqnarray}
	\bangp_{x}{P} & & \nonumber\\
	=
	& {x}!\langle{(\prefix{x}{y}{(\outputp{x}{y} | @{y})) | P}}\rangle 
	      | \prefix{x}{y}{(\outputp{x}{y} | @{y})} & \nonumber\\
	\red
	& (\outputp{x}{y} | @{y})\substn{\quotep{(\prefix{x}{y}{(@{y} | \outputp{x}{y})) | P}}}{y} & \nonumber\\
	=
	& \outputp{x}{\quotep{(\prefix{x}{y}{(\outputp{x}{y} | @{y})) | P}}}
	  | {(\prefix{x}{y}{(\outputp{x}{y} | @{y})) | P}} & \nonumber\\
	\red
	& \ldots & \nonumber\\
	\red^*
	& P | P | \ldots & \nonumber
\end{eqnarray}

Of course, this encoding, as an implementation, runs away, unfolding
$\bangp{P}$ eagerly. A lazier and more implementable replication
operator, restricted to input-guarded processes, may be obtained as follows.

\begin{eqnarray}
\bangp{\prefix{u}{v}{P}} 
	:= 
	\binpar{\lift{x}{\prefix{u}{v}{(\binpar{D(x)}{P})}}}{D(x)} \nonumber
\end{eqnarray}

\begin{remark}
  Note that the lazier definition still does not deal with summation
  or mixed summation (i.e. sums over input and output). The reader is
  invited to construct definitions of replication that deal with these
  features. 

  Further, the definitions are parameterized in a name, $x$. Can you,
  gentle reader, make a definition that eliminates this parameter and
  guarantees no accidental interaction between the replication
  machinery and the process being replicated -- i.e. no accidental
  sharing of names used by the process to get its work done and the
  name(s) used by the replication to effect copying. This latter
  revision of the definition of replication is crucial to obtaining
  the expected identity $!!P \sim !P$.
\end{remark}

\begin{remark}\label{rem:paradoxical_combinator}
  The reader familiar with the lambda calculus will have noticed the
  similarity between $D$ and the paradoxical combinator.

  [Ed. note: the existence of this seems to suggest we have to be more
  restrictive on the set of processes and names we admit if we are to
  support no-cloning.]
\end{remark}

\subsubsection{Bisimulation}

The computational dynamics gives rise to another kind of equivalence,
the equivalence of computational behavior. As previously mentioned
this is typically captured \emph{via} some form of bisimulation.

% The notion we use in this paper is weak barbed bisimulation
% \cite{milner91polyadicpi}.

The notion we use in this paper is derived from weak barbed
bisimulation \cite{milner91polyadicpi}. 

\begin{definition}
An \emph{observation relation}, $\downarrow_{\mathcal N}$, over a set
of names, $\mathcal N$, is the smallest relation satisfying the rules
below.

\infrule[Out-barb]{y \in {\mathcal N}, \; x \nameeq y}
		  {\outputp{x}{v} \downarrow_{\mathcal N} x}
\infrule[Par-barb]{\mbox{$P\downarrow_{\mathcal N} x$ or $Q\downarrow_{\mathcal N} x$}}
		  {\binpar{P}{Q} \downarrow_{\mathcal N} x}

We write $P \Downarrow_{\mathcal N} x$ if there is $Q$ such that 
$P \wred Q$ and $Q \downarrow_{\mathcal N} x$.
\end{definition}

\begin{definition}
%\label{def.bbisim}
An  ${\mathcal N}$-\emph{barbed bisimulation} over a set of names, ${\mathcal N}$, is a symmetric binary relation 
${\mathcal S}_{\mathcal N}$ between agents such that $P\rel{S}_{\mathcal N}Q$ implies:
\begin{enumerate}
\item If $P \red P'$ then $Q \wred Q'$ and $P'\rel{S}_{\mathcal N} Q'$.
\item If $P\downarrow_{\mathcal N} x$, then $Q\Downarrow_{\mathcal N} x$.
\end{enumerate}
$P$ is ${\mathcal N}$-barbed bisimilar to $Q$, written
$P \wbbisim_{\mathcal N} Q$, if $P \rel{S}_{\mathcal N} Q$ for some ${\mathcal N}$-barbed bisimulation ${\mathcal S}_{\mathcal N}$.
\end{definition}

$\mathcal{R} \subseteq \pi \times \pi$

$P \mathcal{R} Q => \forall P'. P \red P' \Rightarrow \exists Q'. Q \red Q', P' \mathcal{R} Q'$

$P \vdash x \Rightarrow Q \vdash x$

\begin{mathpar}
  \inferrule*[lab=Out-barb]{x \nameeq y}{{y}!\langle{Q}\rangle \vdash x}
  \and
  \inferrule*[lab=Par-barb]{\mbox{$P\vdash x$ or $Q\vdash x$}}{\binpar{P}{Q} \vdash x}
\end{mathpar}

\subsubsection{Contexts}

One of the principle advantages of computational calculi like the
$\pi$-calculus is a well-defined notion of context,
contextual-equivalence and a correlation between
contextual-equivalence and notions of bisimulation. The notion of
context allows the decomposition of a process into (sub-)process and
its syntactic environment, its context. Thus, a context may be
thought of as a process with a ``hole'' (written $\Box$) in it. The
application of a context $M$ to a process $P$, written $M[P]$, is
tantamount to filling the hole in $M$ with $P$. In this paper we do
not need the full weight of this theory, but do make use of the notion
of context in the proof the main theorem. 

\begin{mathpar}
  \inferrule* [lab=summation] {} {{M_{M},M_{N}} \bc \Box \;|\; x.M_{A} \;|\; M_{M}+M_{N}}
  \and
  \inferrule* [lab=agent] {} {{M_{A}} \bc (\vec{x})M_{P} \;| \; \clift{P_0,\ldots,M_{P},\ldots,P_N}}
  \and \\
  \inferrule* [lab=process] {} {{M_{P}} \bc M_{N} \;| \;P|M_{P} }
\end{mathpar} 

\begin{mathpar}
  \inferrule* [lab=sychronization] {} {M_{N} \bc \Box \;|\; x?M_{F} \;|\; x!M_{C}}
  \and
  \inferrule* [lab=abstraction] {} {{M_{F}} \bc (x)M_{P} }
  \and
  \inferrule* [lab=concretion] {} {{M_{C}} \bc \langle M_{P} \rangle }
  \and \\
  \inferrule* [lab=process] {} {{M_{P}} \bc M_{N} \;| \;P|M_{P} }
\end{mathpar}

\begin{definition}[contextual application] Given a context $M$, and
  process $P$, we define the \emph{contextual application}, $M[P] :=
  M\{P/\Box\}$. That is, the contextual application of M to P is the
  substitution of $P$ for $\Box$ in $M$.
\end{definition}

$\meaningof{-} : L \to \mathcal{P}(\pi)$

\begin{mathpar}
  \inferrule* [lab=collection] {} {\meaningof{true} = \pi, \and \meaningof{~E} = \pi \setminus \meaningof{E}, \and \meaningof{E_{1} \& E_{2}} = \meaningof{E_{1}} \cap \meaningof{E_{2}}}
\end{mathpar}

\begin{mathpar}
  \inferrule* [lab=structure] {} {\meaningof{0} = \{ P \in \pi | P \equiv 0 \}, \and \\ \meaningof{E_1 | E_2} = \{ P \in \pi | P \equiv P_{1} | P_{2}, P_{1} \in \meaningof{E_{1}}, P_{2} \in \meaningof{E_2}\} }
\end{mathpar}

\begin{mathpar}
 \inferrule* [lab=behavior] {} {\meaningof{\langle a?b \rangle E} = \{ P \in \pi | P \equiv Q | u?(y)P', \\ \and \\\\ \and \\ \;\;\; u \in \meaningof{a}, \forall z.P'\{z/y\} \in \meaningof{E\{z/b\}}\}, \and \\ \meaningof{a!E} = \{ P \in \pi | P \equiv Q | x!\langle P' \rangle, x \in \meaningof{a} P' \in \meaningof{E}\} }
\end{mathpar}

\begin{mathpar}
 \inferrule* [lab=nominal] {} {\meaningof{\quotep{E}} = \{ \quotep{P} \in \quotep{\pi} | P \in \meaningof{E} \}, \and \meaningof{\quotep{P}} = \{ \quotep{Q} \in \quotep{\pi} | P \equiv Q \} \and \\ \meaningof{@\quotep{E}} = \{ P \in \pi | P \equiv @x, x \in \meaningof{E} \}}
\end{mathpar}

\begin{eqnarray*}
  \\
  \meaningof{-} : TS \to ST
\end{eqnarray*}

\begin{eqnarray*}
  \\
  L : TS \to ST
\end{eqnarray*}

\begin{eqnarray*}
  \\
  P \models E \iff P \in \meaningof{E}
\end{eqnarray*}

\begin{eqnarray*}
  P \approx_{L} Q \iff \forall E \in L. P \models E \iff Q \models E
\end{eqnarray*}

\begin{eqnarray*}
  P \approx_{K} Q
\end{eqnarray*}

\begin{eqnarray*}
  P \approx Q
\end{eqnarray*}

$\approx_{K} = \approx = \approx_{L}$

\subsubsection{Contextual duality}

Note that contexts extend the quotation operation to a family of
operations from processes to names. Given a context, $M$, we can
define a \emph{nominal context}, $\quotep{M}$ by $\quotep{M}[P] :=
\quotep{M[P]}$. To foreshadow what is to come we observe that these
operations enjoy a duality with processes very much like the duality
between vectors and maps from vectors to scalars.

Further, because the calculus is essentially higher-order, we have a
correspondence between contexts and processes. More specifically,
given a name $x$ and a context $M$ we can construct $M^{*}_{x}$ such
that 

\begin{mathpar}
  M^{*}_{x} | \lift{x}{P} \red M[P]
\end{mathpar}

namely,

\begin{mathpar}
  M^{*}_{x} := x?(u).M[\dropn{u}]
\end{mathpar}

The dependence of $M^{*}_{x}$ on a name makes it an abstraction, 

\begin{mathpar}
  M^{*} := (x)x?(u).M[\dropn{u}]
\end{mathpar}

\subsection{Additional notation}

It will sometimes be convenient to denote the process a name
quotes. We already have the notation $x = \quotep{P}$, but it will be
convenient to introduce an alternate notation, $\procn{x}$, when we
want to emphasize the connection to the use of the name. Note that, by
virtue of name equivalence, $\quotep{\procn{x}} \nameeq x$; so, the
notation is consistent with previous definitions.

Further, because names have structure it is possible to effect
substitutions on the basis of that structure. This means we need to
upgrade our notation for substitutions, which we accomplish by
adapting comprehension notation. Thus,

\begin{mathpar}
  P\{ y / x : x \in S \}
\end{mathpar}

is interpreted to mean the process derived from P by replacing (in a
capture-avoiding manner) each occurrence of $x$ in $S$ by $y$. For example,

\begin{mathpar}
  P\{ \quotep{\procn{x}|\procn{x}} / x : x \in \freenames{P} \}
\end{mathpar}

will replace each (occurrence) of a free name $x$ in $P$ by
$\quotep{\procn{x}|\procn{x}}$.

Also, we will avail ourselves of the notation $x^{L}$ and $x^{R}$ to
denote injections of a name into disjoint copies of the name
space. There are numerous ways to accomplish this. One example can be
found in \cite{MeredithR05}. This notation overloads to vectors of
names: $\vec{x}^{\pi} := (x_{i}^{\pi} \; : \; 0 \leq i < |\vec{x}| )$ where $\pi \in \{L,R\}$.

We also use $P^{\Box} := P|\Box$.

In \cite{MeredithR05} an interpretation of the new operator is
given. It turns out that there are several possible interpretations
all enjoying the requisite algebraic properties of the operator (see
\cite{milner91polyadicpi}). We will therefore make liberal use of
$(\nu\; \vec{x})P$.

% subsection the_syntax_and_semantics_of_the_notation_system (end)   

\input{qm2pi.qmops} 

\input{qm2pi.sterngerlach} 

\input{qm2pi.metric} 

% section concurrent_process_calculi (end)

%\input{qm2pi.proofsketch}

% section proof sketch (end)

%\input{qm2pi.slviaknots} 

% section spatial logic via knots (end)

\input{qm2pi.conclusion}

% section conclusion (end)

%\input{qm2pi.dtcodes} 

% section wiring algorithm (end)

\input{qm2pi.ack} 

% section acknowledgments (end)

\newpage


\bibliographystyle{plain}   
\bibliography{../../biblios/main.bib}

\input{qm2pi.rhodetails}

\end{document}

 

% section notation (end)

\input{qm2pi.process.calculi} 

% section concurrent_process_calculi_and_spatial_logics_ (end)
    
%\documentclass[12pt]{llncs}
%\documentclass{jktr}

\usepackage[pdftex]{hyperref}                   
\usepackage {listings}
\usepackage {mathpartir}
\usepackage{bcprules}
%\usepackage{listings}
                       
\usepackage{graphicx} 
%\usepackage[margins=2.5cm,nohead,nofoot]{geometry}
%\usepackage{geometry}
\usepackage{amsfonts}
\usepackage{amstext}
\usepackage{latexsym}
\usepackage{amssymb}
\usepackage{color}


%\include{myPreamble}
\include{qm2pi.local} 

%\ifpdf
%\usepackage[pdftex]{graphicx}
%\else
%\usepackage{graphicx}
%\fi

 % \ifpdf
%  \usepackage{pdfsync}
%  \if


%\title{Brief Article}
%\author{David F. Snyder}
%\author{L.G. Meredith}

%\address{Dept. of Math., Texas State University--San Marcos, San Marcos, TX 78666}
       
\pagestyle{empty}


\begin{document}

\lstset{language=[Objective]Caml,frame=shadowbox}

\input{qm2pi.front}

% section front matter (end)

\input{qm2pi.intro} 
 
% section introduction (end)

% \input{qm2pi.knotations} 

% section notation (end)

\input{qm2pi.process.calculi} 

% section concurrent_process_calculi_and_spatial_logics_ (end)
    
%\input{qm2pi.knots2pi} 

%\input{qm2pi.trefoil} 

%\input{qm2pi.mainthm} 

% subsection basic_interpretation (end)

%\input{qm2pi.rho.presentation} 
\subsection{The syntax and semantics of the notation system}\label{sub:the_syntax_and_semantics_of_the_notation_system} % (fold)

We now summarize a technical presentation of the calculus that
embodies our theory of dynamics. The typical presentation of such a
calculus follows the style of giving generators and relations on
them. The grammar, below, describing term constructors, freely
generates the set of processes, $\Proc$. This set is then quotiented
by a relation known as structural congruence and it is over this set
that the notion of dynamics is expressed. This presentation is
essentially that of \cite{MeredithR05} with the addition of
polyadicity and summation. For readability we have relegated some of
the technical subtleties to an appendix.

\subsubsection{Process grammar}\label{subsub:process_grammar}

\begin{mathpar}
  \inferrule* [lab=synchronization] {} {{M} \bc \pzero \;|\; x?F \;|\; x!C }
  \and
  \inferrule* [lab=abstraction] {} {{F} \bc (x)P}
  \and
  \inferrule* [lab=concretion] {} {{C} \bc \langle Q \rangle}
  \and
  \inferrule* [lab=process] {} {{P,Q} \bc M \;| \;P|Q \;|\; @{x}}
  \and
  \inferrule* [lab=name] {} {{x} \bc \quotep{P}}
\end{mathpar} 

Note that $\vec{x}$ (resp. $\vec{P}$) denotes a vector of names
(resp. processes) of length $|\vec{x}|$ (resp. $|\vec{P}|$). We adopt
the following useful abbreviations.

\begin{mathpar}
   x?(\vec{y}).P := x.(\vec{y})P \and  x\clift{\vec{P}} := x.\clift{\vec{P}}
   \and x!(y) := \lift{x}{\dropn{y}}
   \and \Pi_{i=0}^{n-1}P_i := P_0 | \ldots | P_{n-1}
\end{mathpar}

\subsubsection{Structural congruence}

\paragraph{Free and bound names and alpha-equivalence.} At the
core of structural equivalence is alpha-equivalence which identifies
process that are the same up to a change of variable. Formally, we
recognize the distinction between free and bound names. The free names
of a process, $\freenames{P}$, may be calculated recursively as
follows:

\begin{mathpar}
\freenames{\pzero} := \emptyset
  \and \\
  \freenames{x?(y).P} := \{ x \} \cup (\freenames{P} \setminus \{ y \})
  \and 
  \freenames{x!\langle P \rangle} := \{ x \} \cup \{ P \} 
  \and \\
  \freenames{P|Q} := \freenames{P} \cup \freenames{Q}
  \and \\
  \freenames{@{x}} := \{ x \}
\end{mathpar}

$\pi$
$\quotep{\pi}$

$\freenames{-} : \pi \to \mathcal{P}(\quotep{\pi})$

\begin{eqnarray*}
  \freenames{\pzero} & := & \emptyset \\
  \freenames{x?(y).P} & := & \{ x \} \cup (\freenames{P} \setminus \{ y \}) \\
  \freenames{x!\langle P \rangle} & := & \{ x \} \cup \{ P \} \\
  \freenames{P|Q} & := & \freenames{P} \cup \freenames{Q} \\
  \freenames{\dropn{x}} & := & \{ x \}
\end{eqnarray*}

The bound names of a process, $\boundnames{P}$, are those names occurring in $P$
that are not free. For example, in $x?(y).0$, the name $x$ is free, while $y$ is bound.

\begin{mathpar}
  \inferrule* [lab=monoidal-laws] {} { P|Q \equiv Q|P \and P|0 \equiv P \and P|(Q|R) \equiv (P|Q)|R }
\end{mathpar}

\begin{mathpar}
  \inferrule* [lab=alpha-equivalence] {} { (x)P \equiv (y)P\{y/x\} \and y \not\in \freenames{P} }
\end{mathpar}

\begin{definition}
Then two processes, $P,Q$, are alpha-equivalent if $P = Q\{\vec{y}/\vec{x}\}$ for
some $\vec{x} \in \boundnames{Q},\vec{y} \in \boundnames{P}$, where $Q\{\vec{y}/\vec{x}\}$
denotes the capture-avoiding substitution of $\vec{y}$ for $\vec{x}$ in $Q$.
\end{definition}

\begin{definition}
  The {\em structural congruence} \cite{SangiorgiWalker} , $\equiv$,
  between processes is the least congruence containing
  alpha-equivalence, satisfying the abelian monoid laws
  (associativity, commutativity and $\pzero$ as identity) for parallel
  composition $|$ and for summation $+$.
\end{definition}

\subsection{Name equivalence}

We take name equivalence, written $\nameeq$, to be the smallest
equivalence relation generated by the following rules.

\begin{mathpar}
\inferrule*[lab=Quote-drop]
{ }
{ \quotep{@{x}} \nameeq x }

\inferrule*[lab=Struct-equiv]
{ P \scong Q }
{ \quotep{P} \nameeq \quotep{Q} }
\end{mathpar}

The astute reader will have noticed that the mutual recursion of names
and processes imposes a mutual recursion on alpha-equivalence and
structural equivalence via name-equivalence. Fortunately, all of this
works out pleasantly and we may calculate in the natural way, free of
concern. The reader interested in the details is referred to the
appendix \ref{appendix:rho_details}.

\subsection{Substitution}

We use $\Proc$ for the set of processes, $\QProc$ for the set of
names, and $\id{\{}\vec{y} / \vec{x} \id{\}}$ to denote partial maps,
$s : \QProc \rightarrow \QProc$. A map, $s$ lifts, uniquely, to a map
on process terms, $\widehat{s} : \Proc \rightarrow \Proc$ by the
following equations.

\begin{mathpar}
  (0) \psubstp{Q}{P} := 0 \\
  (R \juxtap S) \psubstp{Q}{P}
  :=    
  (R)\psubstp{Q}{P} \juxtap (S) \psubstp{Q}{P} \\
  (x?(y).R) \psubstp{Q}{P}    
  :=    
  (x)\substp{Q}{P} (z)\concat( (R \psubstn{z}{y}) \psubstp{Q}{P} ) \\
  (\lift{x}{R}) \psubstp{Q}{P}  
  :=
  \lift{(x)\substp{Q}{P}}{ R \psubstp{Q}{P} } \\
%   (\dropn{x})  \psubstp{Q}{P}       
%   := 
%   \left\{ 
%     \begin{array}{ccc} 
%       \dropn{\quotep{Q}} & & x \nameeq \quotep{P} \\
%       \dropn{x} & & otherwise \\
%     \end{array}
%   \right. 
  (\dropn{x})  \psubstp{Q}{P}       
  := 
  \left\{ 
    \begin{array}{ccc} 
      Q & & x \nameeq \quotep{P} \\
      \dropn{x} & & otherwise \\
    \end{array}
  \right.
\end{mathpar}
 

where

\begin{eqnarray}
  (x)\id{\{} \lpquote Q \rpquote / \lpquote P \rpquote \id{\}}            = 
  \left\{ 
    \begin{array}{ccc}
      \lpquote Q \rpquote & & x \nameeq \lpquote P \rpquote \\
      x & & otherwise \\
    \end{array}
  \right. \nonumber
\end{eqnarray}

and $z$ is chosen distinct from $\quotep{P}$, $\quotep{Q}$, the free
names in $Q$, and all the names in $R$. Our $\alpha$-equivalence will
be built in the standard way from this substitution.

\begin{remark}\label{rem:no_self_referential_names}
  One consequence of these definitions is that $\forall P. \quotep{P}
  \not\in \freenames{P}$.
\end{remark}

\subsection{ Dynamic quote: an example }

Anticipating something of what's to come, consider applying the
substitution, $\widehat{\id{\{}u / z \id{\}}}$, to the following pair
of processes, $\lift{w}{y!(z)}$ and $w[ \lpquote y!(z) \rpquote ]$.

\begin{eqnarray}
	\lift{w}{y!(z)}\widehat{\id{\{}u / z \id{\}}}
		& = &
		\lift{w}{y!(u)} \nonumber\\
	w[ \lpquote y!(z) \rpquote ] \widehat{ \id{\{}u / z \id{\}} }
		& = &
		w[ \lpquote y!(z) \rpquote ] \nonumber
\end{eqnarray}

Because the body of the process between quotes is impervious to
substitution, we get radically different answers. In fact, by
examining the first process in an input context,
e.g. $x?(z).\lift{w}{y!(z)}$, we see that the process under the lift
operator may be shaped by prefixed inputs binding a name inside it. In
this sense, the lift operator will be seen as a way to dynamically
construct processes before reifying them as names.

Finally equipped with these standard features we can present the
dynamics of the calculus.

\subsubsection{Operational semantics} 

Finally, we introduce the computational dynamics. What marks these
algebras as distinct from other more traditionally studied algebraic
structures, e.g. vector spaces or polynomial rings, is the manner in
which dynamics is captured. In traditional structures, dynamics is typically
expressed through morphisms between such structures, as in linear maps
between vector spaces or morphisms between rings. In algebras
associated with the semantics of computation, the dynamics is
expressed as part of the algebraic structure itself, through a
reduction reduction relation typically denoted by $\red$. Below, we
give a recursive presentation of this relation for the calculus used
in the encoding.

$\red \subseteq \pi \times \pi$
$\red : \pi \to \mathcal{P}(\pi)$

\begin{mathpar}
  \inferrule* [lab=Comm] { \textsf{match}( x_{src}, x_{trgt} ) } { x_{trgt}?(y)P \; | \; x_{src}!\langle {Q} \rangle \red P\{\quotep{Q}/y}\} }
  \and \\
  \inferrule* [lab=Par] {{P} \red {P}'} {{{P} | {Q}} \red {{P}' | {Q}}}
  \and
  \inferrule* [lab=Equiv]{{{P} \scong {P}'} \andalso {{P}' \red {Q}'} \andalso {{Q}' \scong {Q}}}{{P} \red {Q}}
\end{mathpar}

\begin{eqnarray*}
  match_{\equiv} (\quotep{P},\quotep{Q}) & := & P \equiv Q \\
  match_{\dagger}(\quotep{P},\quotep{Q}) & := & \forall R. P|Q \red^{*} R => R \red^{*} 0 \\
  match_{K}(\quotep{P},\quotep{Q}) & := & K \mbox{ for some context } K
\end{eqnarray*}

$u?(x)P | u!\langle Q \rangle \red P\{\quotep{Q}/x\}$

%We write $\wred$ for $\red^*$, and $P\red$ if $\exists Q $ such that $ P \red Q$.
We write $P\red$ if $\exists Q $ such that $ P \red Q$ and $P\not\red$, otherwise.

\section{Replication}

As mentioned before, it is known that replication (and hence
recursion) can be implemented in a higher-order process algebra
\cite{SangiorgiWalker}. As our first example of calculation with the
machinery thus far presented we give the construction explicitly in
the {\rhoc}.

\begin{eqnarray}
	D_{x} & := & \prefix{x}{y}{(\binpar{\outputp{x}{y}}{@{y}})} \nonumber\\
	\bangp_{x}{P} & := & \binpar{{x}!\langle{\binpar{D_{x}}{P}}\rangle}{D_{x}} \nonumber
\end{eqnarray}

\begin{eqnarray}
	\bangp_{x}{P} & & \nonumber\\
	=
	& {x}!\langle{(\prefix{x}{y}{(\outputp{x}{y} | @{y})) | P}}\rangle 
	      | \prefix{x}{y}{(\outputp{x}{y} | @{y})} & \nonumber\\
	\red
	& (\outputp{x}{y} | @{y})\substn{\quotep{(\prefix{x}{y}{(@{y} | \outputp{x}{y})) | P}}}{y} & \nonumber\\
	=
	& \outputp{x}{\quotep{(\prefix{x}{y}{(\outputp{x}{y} | @{y})) | P}}}
	  | {(\prefix{x}{y}{(\outputp{x}{y} | @{y})) | P}} & \nonumber\\
	\red
	& \ldots & \nonumber\\
	\red^*
	& P | P | \ldots & \nonumber
\end{eqnarray}

Of course, this encoding, as an implementation, runs away, unfolding
$\bangp{P}$ eagerly. A lazier and more implementable replication
operator, restricted to input-guarded processes, may be obtained as follows.

\begin{eqnarray}
\bangp{\prefix{u}{v}{P}} 
	:= 
	\binpar{\lift{x}{\prefix{u}{v}{(\binpar{D(x)}{P})}}}{D(x)} \nonumber
\end{eqnarray}

\begin{remark}
  Note that the lazier definition still does not deal with summation
  or mixed summation (i.e. sums over input and output). The reader is
  invited to construct definitions of replication that deal with these
  features. 

  Further, the definitions are parameterized in a name, $x$. Can you,
  gentle reader, make a definition that eliminates this parameter and
  guarantees no accidental interaction between the replication
  machinery and the process being replicated -- i.e. no accidental
  sharing of names used by the process to get its work done and the
  name(s) used by the replication to effect copying. This latter
  revision of the definition of replication is crucial to obtaining
  the expected identity $!!P \sim !P$.
\end{remark}

\begin{remark}\label{rem:paradoxical_combinator}
  The reader familiar with the lambda calculus will have noticed the
  similarity between $D$ and the paradoxical combinator.

  [Ed. note: the existence of this seems to suggest we have to be more
  restrictive on the set of processes and names we admit if we are to
  support no-cloning.]
\end{remark}

\subsubsection{Bisimulation}

The computational dynamics gives rise to another kind of equivalence,
the equivalence of computational behavior. As previously mentioned
this is typically captured \emph{via} some form of bisimulation.

% The notion we use in this paper is weak barbed bisimulation
% \cite{milner91polyadicpi}.

The notion we use in this paper is derived from weak barbed
bisimulation \cite{milner91polyadicpi}. 

\begin{definition}
An \emph{observation relation}, $\downarrow_{\mathcal N}$, over a set
of names, $\mathcal N$, is the smallest relation satisfying the rules
below.

\infrule[Out-barb]{y \in {\mathcal N}, \; x \nameeq y}
		  {\outputp{x}{v} \downarrow_{\mathcal N} x}
\infrule[Par-barb]{\mbox{$P\downarrow_{\mathcal N} x$ or $Q\downarrow_{\mathcal N} x$}}
		  {\binpar{P}{Q} \downarrow_{\mathcal N} x}

We write $P \Downarrow_{\mathcal N} x$ if there is $Q$ such that 
$P \wred Q$ and $Q \downarrow_{\mathcal N} x$.
\end{definition}

\begin{definition}
%\label{def.bbisim}
An  ${\mathcal N}$-\emph{barbed bisimulation} over a set of names, ${\mathcal N}$, is a symmetric binary relation 
${\mathcal S}_{\mathcal N}$ between agents such that $P\rel{S}_{\mathcal N}Q$ implies:
\begin{enumerate}
\item If $P \red P'$ then $Q \wred Q'$ and $P'\rel{S}_{\mathcal N} Q'$.
\item If $P\downarrow_{\mathcal N} x$, then $Q\Downarrow_{\mathcal N} x$.
\end{enumerate}
$P$ is ${\mathcal N}$-barbed bisimilar to $Q$, written
$P \wbbisim_{\mathcal N} Q$, if $P \rel{S}_{\mathcal N} Q$ for some ${\mathcal N}$-barbed bisimulation ${\mathcal S}_{\mathcal N}$.
\end{definition}

$\mathcal{R} \subseteq \pi \times \pi$

$P \mathcal{R} Q => \forall P'. P \red P' \Rightarrow \exists Q'. Q \red Q', P' \mathcal{R} Q'$

$P \vdash x \Rightarrow Q \vdash x$

\begin{mathpar}
  \inferrule*[lab=Out-barb]{x \nameeq y}{{y}!\langle{Q}\rangle \vdash x}
  \and
  \inferrule*[lab=Par-barb]{\mbox{$P\vdash x$ or $Q\vdash x$}}{\binpar{P}{Q} \vdash x}
\end{mathpar}

\subsubsection{Contexts}

One of the principle advantages of computational calculi like the
$\pi$-calculus is a well-defined notion of context,
contextual-equivalence and a correlation between
contextual-equivalence and notions of bisimulation. The notion of
context allows the decomposition of a process into (sub-)process and
its syntactic environment, its context. Thus, a context may be
thought of as a process with a ``hole'' (written $\Box$) in it. The
application of a context $M$ to a process $P$, written $M[P]$, is
tantamount to filling the hole in $M$ with $P$. In this paper we do
not need the full weight of this theory, but do make use of the notion
of context in the proof the main theorem. 

\begin{mathpar}
  \inferrule* [lab=summation] {} {{M_{M},M_{N}} \bc \Box \;|\; x.M_{A} \;|\; M_{M}+M_{N}}
  \and
  \inferrule* [lab=agent] {} {{M_{A}} \bc (\vec{x})M_{P} \;| \; \clift{P_0,\ldots,M_{P},\ldots,P_N}}
  \and \\
  \inferrule* [lab=process] {} {{M_{P}} \bc M_{N} \;| \;P|M_{P} }
\end{mathpar} 

\begin{mathpar}
  \inferrule* [lab=sychronization] {} {M_{N} \bc \Box \;|\; x?M_{F} \;|\; x!M_{C}}
  \and
  \inferrule* [lab=abstraction] {} {{M_{F}} \bc (x)M_{P} }
  \and
  \inferrule* [lab=concretion] {} {{M_{C}} \bc \langle M_{P} \rangle }
  \and \\
  \inferrule* [lab=process] {} {{M_{P}} \bc M_{N} \;| \;P|M_{P} }
\end{mathpar}

\begin{definition}[contextual application] Given a context $M$, and
  process $P$, we define the \emph{contextual application}, $M[P] :=
  M\{P/\Box\}$. That is, the contextual application of M to P is the
  substitution of $P$ for $\Box$ in $M$.
\end{definition}

$\meaningof{-} : L \to \mathcal{P}(\pi)$

\begin{mathpar}
  \inferrule* [lab=collection] {} {\meaningof{true} = \pi, \and \meaningof{~E} = \pi \setminus \meaningof{E}, \and \meaningof{E_{1} \& E_{2}} = \meaningof{E_{1}} \cap \meaningof{E_{2}}}
\end{mathpar}

\begin{mathpar}
  \inferrule* [lab=structure] {} {\meaningof{0} = \{ P \in \pi | P \equiv 0 \}, \and \\ \meaningof{E_1 | E_2} = \{ P \in \pi | P \equiv P_{1} | P_{2}, P_{1} \in \meaningof{E_{1}}, P_{2} \in \meaningof{E_2}\} }
\end{mathpar}

\begin{mathpar}
 \inferrule* [lab=behavior] {} {\meaningof{\langle a?b \rangle E} = \{ P \in \pi | P \equiv Q | u?(y)P', \\ \and \\\\ \and \\ \;\;\; u \in \meaningof{a}, \forall z.P'\{z/y\} \in \meaningof{E\{z/b\}}\}, \and \\ \meaningof{a!E} = \{ P \in \pi | P \equiv Q | x!\langle P' \rangle, x \in \meaningof{a} P' \in \meaningof{E}\} }
\end{mathpar}

\begin{mathpar}
 \inferrule* [lab=nominal] {} {\meaningof{\quotep{E}} = \{ \quotep{P} \in \quotep{\pi} | P \in \meaningof{E} \}, \and \meaningof{\quotep{P}} = \{ \quotep{Q} \in \quotep{\pi} | P \equiv Q \} \and \\ \meaningof{@\quotep{E}} = \{ P \in \pi | P \equiv @x, x \in \meaningof{E} \}}
\end{mathpar}

\begin{eqnarray*}
  \\
  \meaningof{-} : TS \to ST
\end{eqnarray*}

\begin{eqnarray*}
  \\
  L : TS \to ST
\end{eqnarray*}

\begin{eqnarray*}
  \\
  P \models E \iff P \in \meaningof{E}
\end{eqnarray*}

\begin{eqnarray*}
  P \approx_{L} Q \iff \forall E \in L. P \models E \iff Q \models E
\end{eqnarray*}

\begin{eqnarray*}
  P \approx_{K} Q
\end{eqnarray*}

\begin{eqnarray*}
  P \approx Q
\end{eqnarray*}

$\approx_{K} = \approx = \approx_{L}$

\subsubsection{Contextual duality}

Note that contexts extend the quotation operation to a family of
operations from processes to names. Given a context, $M$, we can
define a \emph{nominal context}, $\quotep{M}$ by $\quotep{M}[P] :=
\quotep{M[P]}$. To foreshadow what is to come we observe that these
operations enjoy a duality with processes very much like the duality
between vectors and maps from vectors to scalars.

Further, because the calculus is essentially higher-order, we have a
correspondence between contexts and processes. More specifically,
given a name $x$ and a context $M$ we can construct $M^{*}_{x}$ such
that 

\begin{mathpar}
  M^{*}_{x} | \lift{x}{P} \red M[P]
\end{mathpar}

namely,

\begin{mathpar}
  M^{*}_{x} := x?(u).M[\dropn{u}]
\end{mathpar}

The dependence of $M^{*}_{x}$ on a name makes it an abstraction, 

\begin{mathpar}
  M^{*} := (x)x?(u).M[\dropn{u}]
\end{mathpar}

\subsection{Additional notation}

It will sometimes be convenient to denote the process a name
quotes. We already have the notation $x = \quotep{P}$, but it will be
convenient to introduce an alternate notation, $\procn{x}$, when we
want to emphasize the connection to the use of the name. Note that, by
virtue of name equivalence, $\quotep{\procn{x}} \nameeq x$; so, the
notation is consistent with previous definitions.

Further, because names have structure it is possible to effect
substitutions on the basis of that structure. This means we need to
upgrade our notation for substitutions, which we accomplish by
adapting comprehension notation. Thus,

\begin{mathpar}
  P\{ y / x : x \in S \}
\end{mathpar}

is interpreted to mean the process derived from P by replacing (in a
capture-avoiding manner) each occurrence of $x$ in $S$ by $y$. For example,

\begin{mathpar}
  P\{ \quotep{\procn{x}|\procn{x}} / x : x \in \freenames{P} \}
\end{mathpar}

will replace each (occurrence) of a free name $x$ in $P$ by
$\quotep{\procn{x}|\procn{x}}$.

Also, we will avail ourselves of the notation $x^{L}$ and $x^{R}$ to
denote injections of a name into disjoint copies of the name
space. There are numerous ways to accomplish this. One example can be
found in \cite{MeredithR05}. This notation overloads to vectors of
names: $\vec{x}^{\pi} := (x_{i}^{\pi} \; : \; 0 \leq i < |\vec{x}| )$ where $\pi \in \{L,R\}$.

We also use $P^{\Box} := P|\Box$.

In \cite{MeredithR05} an interpretation of the new operator is
given. It turns out that there are several possible interpretations
all enjoying the requisite algebraic properties of the operator (see
\cite{milner91polyadicpi}). We will therefore make liberal use of
$(\nu\; \vec{x})P$.

% subsection the_syntax_and_semantics_of_the_notation_system (end)   

\input{qm2pi.qmops} 

\input{qm2pi.sterngerlach} 

\input{qm2pi.metric} 

% section concurrent_process_calculi (end)

%\input{qm2pi.proofsketch}

% section proof sketch (end)

%\input{qm2pi.slviaknots} 

% section spatial logic via knots (end)

\input{qm2pi.conclusion}

% section conclusion (end)

%\input{qm2pi.dtcodes} 

% section wiring algorithm (end)

\input{qm2pi.ack} 

% section acknowledgments (end)

\newpage


\bibliographystyle{plain}   
\bibliography{../../biblios/main.bib}

\input{qm2pi.rhodetails}

\end{document}

 

%\documentclass[12pt]{llncs}
%\documentclass{jktr}

\usepackage[pdftex]{hyperref}                   
\usepackage {listings}
\usepackage {mathpartir}
\usepackage{bcprules}
%\usepackage{listings}
                       
\usepackage{graphicx} 
%\usepackage[margins=2.5cm,nohead,nofoot]{geometry}
%\usepackage{geometry}
\usepackage{amsfonts}
\usepackage{amstext}
\usepackage{latexsym}
\usepackage{amssymb}
\usepackage{color}


%\include{myPreamble}
\include{qm2pi.local} 

%\ifpdf
%\usepackage[pdftex]{graphicx}
%\else
%\usepackage{graphicx}
%\fi

 % \ifpdf
%  \usepackage{pdfsync}
%  \if


%\title{Brief Article}
%\author{David F. Snyder}
%\author{L.G. Meredith}

%\address{Dept. of Math., Texas State University--San Marcos, San Marcos, TX 78666}
       
\pagestyle{empty}


\begin{document}

\lstset{language=[Objective]Caml,frame=shadowbox}

\input{qm2pi.front}

% section front matter (end)

\input{qm2pi.intro} 
 
% section introduction (end)

% \input{qm2pi.knotations} 

% section notation (end)

\input{qm2pi.process.calculi} 

% section concurrent_process_calculi_and_spatial_logics_ (end)
    
%\input{qm2pi.knots2pi} 

%\input{qm2pi.trefoil} 

%\input{qm2pi.mainthm} 

% subsection basic_interpretation (end)

%\input{qm2pi.rho.presentation} 
\subsection{The syntax and semantics of the notation system}\label{sub:the_syntax_and_semantics_of_the_notation_system} % (fold)

We now summarize a technical presentation of the calculus that
embodies our theory of dynamics. The typical presentation of such a
calculus follows the style of giving generators and relations on
them. The grammar, below, describing term constructors, freely
generates the set of processes, $\Proc$. This set is then quotiented
by a relation known as structural congruence and it is over this set
that the notion of dynamics is expressed. This presentation is
essentially that of \cite{MeredithR05} with the addition of
polyadicity and summation. For readability we have relegated some of
the technical subtleties to an appendix.

\subsubsection{Process grammar}\label{subsub:process_grammar}

\begin{mathpar}
  \inferrule* [lab=synchronization] {} {{M} \bc \pzero \;|\; x?F \;|\; x!C }
  \and
  \inferrule* [lab=abstraction] {} {{F} \bc (x)P}
  \and
  \inferrule* [lab=concretion] {} {{C} \bc \langle Q \rangle}
  \and
  \inferrule* [lab=process] {} {{P,Q} \bc M \;| \;P|Q \;|\; @{x}}
  \and
  \inferrule* [lab=name] {} {{x} \bc \quotep{P}}
\end{mathpar} 

Note that $\vec{x}$ (resp. $\vec{P}$) denotes a vector of names
(resp. processes) of length $|\vec{x}|$ (resp. $|\vec{P}|$). We adopt
the following useful abbreviations.

\begin{mathpar}
   x?(\vec{y}).P := x.(\vec{y})P \and  x\clift{\vec{P}} := x.\clift{\vec{P}}
   \and x!(y) := \lift{x}{\dropn{y}}
   \and \Pi_{i=0}^{n-1}P_i := P_0 | \ldots | P_{n-1}
\end{mathpar}

\subsubsection{Structural congruence}

\paragraph{Free and bound names and alpha-equivalence.} At the
core of structural equivalence is alpha-equivalence which identifies
process that are the same up to a change of variable. Formally, we
recognize the distinction between free and bound names. The free names
of a process, $\freenames{P}$, may be calculated recursively as
follows:

\begin{mathpar}
\freenames{\pzero} := \emptyset
  \and \\
  \freenames{x?(y).P} := \{ x \} \cup (\freenames{P} \setminus \{ y \})
  \and 
  \freenames{x!\langle P \rangle} := \{ x \} \cup \{ P \} 
  \and \\
  \freenames{P|Q} := \freenames{P} \cup \freenames{Q}
  \and \\
  \freenames{@{x}} := \{ x \}
\end{mathpar}

$\pi$
$\quotep{\pi}$

$\freenames{-} : \pi \to \mathcal{P}(\quotep{\pi})$

\begin{eqnarray*}
  \freenames{\pzero} & := & \emptyset \\
  \freenames{x?(y).P} & := & \{ x \} \cup (\freenames{P} \setminus \{ y \}) \\
  \freenames{x!\langle P \rangle} & := & \{ x \} \cup \{ P \} \\
  \freenames{P|Q} & := & \freenames{P} \cup \freenames{Q} \\
  \freenames{\dropn{x}} & := & \{ x \}
\end{eqnarray*}

The bound names of a process, $\boundnames{P}$, are those names occurring in $P$
that are not free. For example, in $x?(y).0$, the name $x$ is free, while $y$ is bound.

\begin{mathpar}
  \inferrule* [lab=monoidal-laws] {} { P|Q \equiv Q|P \and P|0 \equiv P \and P|(Q|R) \equiv (P|Q)|R }
\end{mathpar}

\begin{mathpar}
  \inferrule* [lab=alpha-equivalence] {} { (x)P \equiv (y)P\{y/x\} \and y \not\in \freenames{P} }
\end{mathpar}

\begin{definition}
Then two processes, $P,Q$, are alpha-equivalent if $P = Q\{\vec{y}/\vec{x}\}$ for
some $\vec{x} \in \boundnames{Q},\vec{y} \in \boundnames{P}$, where $Q\{\vec{y}/\vec{x}\}$
denotes the capture-avoiding substitution of $\vec{y}$ for $\vec{x}$ in $Q$.
\end{definition}

\begin{definition}
  The {\em structural congruence} \cite{SangiorgiWalker} , $\equiv$,
  between processes is the least congruence containing
  alpha-equivalence, satisfying the abelian monoid laws
  (associativity, commutativity and $\pzero$ as identity) for parallel
  composition $|$ and for summation $+$.
\end{definition}

\subsection{Name equivalence}

We take name equivalence, written $\nameeq$, to be the smallest
equivalence relation generated by the following rules.

\begin{mathpar}
\inferrule*[lab=Quote-drop]
{ }
{ \quotep{@{x}} \nameeq x }

\inferrule*[lab=Struct-equiv]
{ P \scong Q }
{ \quotep{P} \nameeq \quotep{Q} }
\end{mathpar}

The astute reader will have noticed that the mutual recursion of names
and processes imposes a mutual recursion on alpha-equivalence and
structural equivalence via name-equivalence. Fortunately, all of this
works out pleasantly and we may calculate in the natural way, free of
concern. The reader interested in the details is referred to the
appendix \ref{appendix:rho_details}.

\subsection{Substitution}

We use $\Proc$ for the set of processes, $\QProc$ for the set of
names, and $\id{\{}\vec{y} / \vec{x} \id{\}}$ to denote partial maps,
$s : \QProc \rightarrow \QProc$. A map, $s$ lifts, uniquely, to a map
on process terms, $\widehat{s} : \Proc \rightarrow \Proc$ by the
following equations.

\begin{mathpar}
  (0) \psubstp{Q}{P} := 0 \\
  (R \juxtap S) \psubstp{Q}{P}
  :=    
  (R)\psubstp{Q}{P} \juxtap (S) \psubstp{Q}{P} \\
  (x?(y).R) \psubstp{Q}{P}    
  :=    
  (x)\substp{Q}{P} (z)\concat( (R \psubstn{z}{y}) \psubstp{Q}{P} ) \\
  (\lift{x}{R}) \psubstp{Q}{P}  
  :=
  \lift{(x)\substp{Q}{P}}{ R \psubstp{Q}{P} } \\
%   (\dropn{x})  \psubstp{Q}{P}       
%   := 
%   \left\{ 
%     \begin{array}{ccc} 
%       \dropn{\quotep{Q}} & & x \nameeq \quotep{P} \\
%       \dropn{x} & & otherwise \\
%     \end{array}
%   \right. 
  (\dropn{x})  \psubstp{Q}{P}       
  := 
  \left\{ 
    \begin{array}{ccc} 
      Q & & x \nameeq \quotep{P} \\
      \dropn{x} & & otherwise \\
    \end{array}
  \right.
\end{mathpar}
 

where

\begin{eqnarray}
  (x)\id{\{} \lpquote Q \rpquote / \lpquote P \rpquote \id{\}}            = 
  \left\{ 
    \begin{array}{ccc}
      \lpquote Q \rpquote & & x \nameeq \lpquote P \rpquote \\
      x & & otherwise \\
    \end{array}
  \right. \nonumber
\end{eqnarray}

and $z$ is chosen distinct from $\quotep{P}$, $\quotep{Q}$, the free
names in $Q$, and all the names in $R$. Our $\alpha$-equivalence will
be built in the standard way from this substitution.

\begin{remark}\label{rem:no_self_referential_names}
  One consequence of these definitions is that $\forall P. \quotep{P}
  \not\in \freenames{P}$.
\end{remark}

\subsection{ Dynamic quote: an example }

Anticipating something of what's to come, consider applying the
substitution, $\widehat{\id{\{}u / z \id{\}}}$, to the following pair
of processes, $\lift{w}{y!(z)}$ and $w[ \lpquote y!(z) \rpquote ]$.

\begin{eqnarray}
	\lift{w}{y!(z)}\widehat{\id{\{}u / z \id{\}}}
		& = &
		\lift{w}{y!(u)} \nonumber\\
	w[ \lpquote y!(z) \rpquote ] \widehat{ \id{\{}u / z \id{\}} }
		& = &
		w[ \lpquote y!(z) \rpquote ] \nonumber
\end{eqnarray}

Because the body of the process between quotes is impervious to
substitution, we get radically different answers. In fact, by
examining the first process in an input context,
e.g. $x?(z).\lift{w}{y!(z)}$, we see that the process under the lift
operator may be shaped by prefixed inputs binding a name inside it. In
this sense, the lift operator will be seen as a way to dynamically
construct processes before reifying them as names.

Finally equipped with these standard features we can present the
dynamics of the calculus.

\subsubsection{Operational semantics} 

Finally, we introduce the computational dynamics. What marks these
algebras as distinct from other more traditionally studied algebraic
structures, e.g. vector spaces or polynomial rings, is the manner in
which dynamics is captured. In traditional structures, dynamics is typically
expressed through morphisms between such structures, as in linear maps
between vector spaces or morphisms between rings. In algebras
associated with the semantics of computation, the dynamics is
expressed as part of the algebraic structure itself, through a
reduction reduction relation typically denoted by $\red$. Below, we
give a recursive presentation of this relation for the calculus used
in the encoding.

$\red \subseteq \pi \times \pi$
$\red : \pi \to \mathcal{P}(\pi)$

\begin{mathpar}
  \inferrule* [lab=Comm] { \textsf{match}( x_{src}, x_{trgt} ) } { x_{trgt}?(y)P \; | \; x_{src}!\langle {Q} \rangle \red P\{\quotep{Q}/y}\} }
  \and \\
  \inferrule* [lab=Par] {{P} \red {P}'} {{{P} | {Q}} \red {{P}' | {Q}}}
  \and
  \inferrule* [lab=Equiv]{{{P} \scong {P}'} \andalso {{P}' \red {Q}'} \andalso {{Q}' \scong {Q}}}{{P} \red {Q}}
\end{mathpar}

\begin{eqnarray*}
  match_{\equiv} (\quotep{P},\quotep{Q}) & := & P \equiv Q \\
  match_{\dagger}(\quotep{P},\quotep{Q}) & := & \forall R. P|Q \red^{*} R => R \red^{*} 0 \\
  match_{K}(\quotep{P},\quotep{Q}) & := & K \mbox{ for some context } K
\end{eqnarray*}

$u?(x)P | u!\langle Q \rangle \red P\{\quotep{Q}/x\}$

%We write $\wred$ for $\red^*$, and $P\red$ if $\exists Q $ such that $ P \red Q$.
We write $P\red$ if $\exists Q $ such that $ P \red Q$ and $P\not\red$, otherwise.

\section{Replication}

As mentioned before, it is known that replication (and hence
recursion) can be implemented in a higher-order process algebra
\cite{SangiorgiWalker}. As our first example of calculation with the
machinery thus far presented we give the construction explicitly in
the {\rhoc}.

\begin{eqnarray}
	D_{x} & := & \prefix{x}{y}{(\binpar{\outputp{x}{y}}{@{y}})} \nonumber\\
	\bangp_{x}{P} & := & \binpar{{x}!\langle{\binpar{D_{x}}{P}}\rangle}{D_{x}} \nonumber
\end{eqnarray}

\begin{eqnarray}
	\bangp_{x}{P} & & \nonumber\\
	=
	& {x}!\langle{(\prefix{x}{y}{(\outputp{x}{y} | @{y})) | P}}\rangle 
	      | \prefix{x}{y}{(\outputp{x}{y} | @{y})} & \nonumber\\
	\red
	& (\outputp{x}{y} | @{y})\substn{\quotep{(\prefix{x}{y}{(@{y} | \outputp{x}{y})) | P}}}{y} & \nonumber\\
	=
	& \outputp{x}{\quotep{(\prefix{x}{y}{(\outputp{x}{y} | @{y})) | P}}}
	  | {(\prefix{x}{y}{(\outputp{x}{y} | @{y})) | P}} & \nonumber\\
	\red
	& \ldots & \nonumber\\
	\red^*
	& P | P | \ldots & \nonumber
\end{eqnarray}

Of course, this encoding, as an implementation, runs away, unfolding
$\bangp{P}$ eagerly. A lazier and more implementable replication
operator, restricted to input-guarded processes, may be obtained as follows.

\begin{eqnarray}
\bangp{\prefix{u}{v}{P}} 
	:= 
	\binpar{\lift{x}{\prefix{u}{v}{(\binpar{D(x)}{P})}}}{D(x)} \nonumber
\end{eqnarray}

\begin{remark}
  Note that the lazier definition still does not deal with summation
  or mixed summation (i.e. sums over input and output). The reader is
  invited to construct definitions of replication that deal with these
  features. 

  Further, the definitions are parameterized in a name, $x$. Can you,
  gentle reader, make a definition that eliminates this parameter and
  guarantees no accidental interaction between the replication
  machinery and the process being replicated -- i.e. no accidental
  sharing of names used by the process to get its work done and the
  name(s) used by the replication to effect copying. This latter
  revision of the definition of replication is crucial to obtaining
  the expected identity $!!P \sim !P$.
\end{remark}

\begin{remark}\label{rem:paradoxical_combinator}
  The reader familiar with the lambda calculus will have noticed the
  similarity between $D$ and the paradoxical combinator.

  [Ed. note: the existence of this seems to suggest we have to be more
  restrictive on the set of processes and names we admit if we are to
  support no-cloning.]
\end{remark}

\subsubsection{Bisimulation}

The computational dynamics gives rise to another kind of equivalence,
the equivalence of computational behavior. As previously mentioned
this is typically captured \emph{via} some form of bisimulation.

% The notion we use in this paper is weak barbed bisimulation
% \cite{milner91polyadicpi}.

The notion we use in this paper is derived from weak barbed
bisimulation \cite{milner91polyadicpi}. 

\begin{definition}
An \emph{observation relation}, $\downarrow_{\mathcal N}$, over a set
of names, $\mathcal N$, is the smallest relation satisfying the rules
below.

\infrule[Out-barb]{y \in {\mathcal N}, \; x \nameeq y}
		  {\outputp{x}{v} \downarrow_{\mathcal N} x}
\infrule[Par-barb]{\mbox{$P\downarrow_{\mathcal N} x$ or $Q\downarrow_{\mathcal N} x$}}
		  {\binpar{P}{Q} \downarrow_{\mathcal N} x}

We write $P \Downarrow_{\mathcal N} x$ if there is $Q$ such that 
$P \wred Q$ and $Q \downarrow_{\mathcal N} x$.
\end{definition}

\begin{definition}
%\label{def.bbisim}
An  ${\mathcal N}$-\emph{barbed bisimulation} over a set of names, ${\mathcal N}$, is a symmetric binary relation 
${\mathcal S}_{\mathcal N}$ between agents such that $P\rel{S}_{\mathcal N}Q$ implies:
\begin{enumerate}
\item If $P \red P'$ then $Q \wred Q'$ and $P'\rel{S}_{\mathcal N} Q'$.
\item If $P\downarrow_{\mathcal N} x$, then $Q\Downarrow_{\mathcal N} x$.
\end{enumerate}
$P$ is ${\mathcal N}$-barbed bisimilar to $Q$, written
$P \wbbisim_{\mathcal N} Q$, if $P \rel{S}_{\mathcal N} Q$ for some ${\mathcal N}$-barbed bisimulation ${\mathcal S}_{\mathcal N}$.
\end{definition}

$\mathcal{R} \subseteq \pi \times \pi$

$P \mathcal{R} Q => \forall P'. P \red P' \Rightarrow \exists Q'. Q \red Q', P' \mathcal{R} Q'$

$P \vdash x \Rightarrow Q \vdash x$

\begin{mathpar}
  \inferrule*[lab=Out-barb]{x \nameeq y}{{y}!\langle{Q}\rangle \vdash x}
  \and
  \inferrule*[lab=Par-barb]{\mbox{$P\vdash x$ or $Q\vdash x$}}{\binpar{P}{Q} \vdash x}
\end{mathpar}

\subsubsection{Contexts}

One of the principle advantages of computational calculi like the
$\pi$-calculus is a well-defined notion of context,
contextual-equivalence and a correlation between
contextual-equivalence and notions of bisimulation. The notion of
context allows the decomposition of a process into (sub-)process and
its syntactic environment, its context. Thus, a context may be
thought of as a process with a ``hole'' (written $\Box$) in it. The
application of a context $M$ to a process $P$, written $M[P]$, is
tantamount to filling the hole in $M$ with $P$. In this paper we do
not need the full weight of this theory, but do make use of the notion
of context in the proof the main theorem. 

\begin{mathpar}
  \inferrule* [lab=summation] {} {{M_{M},M_{N}} \bc \Box \;|\; x.M_{A} \;|\; M_{M}+M_{N}}
  \and
  \inferrule* [lab=agent] {} {{M_{A}} \bc (\vec{x})M_{P} \;| \; \clift{P_0,\ldots,M_{P},\ldots,P_N}}
  \and \\
  \inferrule* [lab=process] {} {{M_{P}} \bc M_{N} \;| \;P|M_{P} }
\end{mathpar} 

\begin{mathpar}
  \inferrule* [lab=sychronization] {} {M_{N} \bc \Box \;|\; x?M_{F} \;|\; x!M_{C}}
  \and
  \inferrule* [lab=abstraction] {} {{M_{F}} \bc (x)M_{P} }
  \and
  \inferrule* [lab=concretion] {} {{M_{C}} \bc \langle M_{P} \rangle }
  \and \\
  \inferrule* [lab=process] {} {{M_{P}} \bc M_{N} \;| \;P|M_{P} }
\end{mathpar}

\begin{definition}[contextual application] Given a context $M$, and
  process $P$, we define the \emph{contextual application}, $M[P] :=
  M\{P/\Box\}$. That is, the contextual application of M to P is the
  substitution of $P$ for $\Box$ in $M$.
\end{definition}

$\meaningof{-} : L \to \mathcal{P}(\pi)$

\begin{mathpar}
  \inferrule* [lab=collection] {} {\meaningof{true} = \pi, \and \meaningof{~E} = \pi \setminus \meaningof{E}, \and \meaningof{E_{1} \& E_{2}} = \meaningof{E_{1}} \cap \meaningof{E_{2}}}
\end{mathpar}

\begin{mathpar}
  \inferrule* [lab=structure] {} {\meaningof{0} = \{ P \in \pi | P \equiv 0 \}, \and \\ \meaningof{E_1 | E_2} = \{ P \in \pi | P \equiv P_{1} | P_{2}, P_{1} \in \meaningof{E_{1}}, P_{2} \in \meaningof{E_2}\} }
\end{mathpar}

\begin{mathpar}
 \inferrule* [lab=behavior] {} {\meaningof{\langle a?b \rangle E} = \{ P \in \pi | P \equiv Q | u?(y)P', \\ \and \\\\ \and \\ \;\;\; u \in \meaningof{a}, \forall z.P'\{z/y\} \in \meaningof{E\{z/b\}}\}, \and \\ \meaningof{a!E} = \{ P \in \pi | P \equiv Q | x!\langle P' \rangle, x \in \meaningof{a} P' \in \meaningof{E}\} }
\end{mathpar}

\begin{mathpar}
 \inferrule* [lab=nominal] {} {\meaningof{\quotep{E}} = \{ \quotep{P} \in \quotep{\pi} | P \in \meaningof{E} \}, \and \meaningof{\quotep{P}} = \{ \quotep{Q} \in \quotep{\pi} | P \equiv Q \} \and \\ \meaningof{@\quotep{E}} = \{ P \in \pi | P \equiv @x, x \in \meaningof{E} \}}
\end{mathpar}

\begin{eqnarray*}
  \\
  \meaningof{-} : TS \to ST
\end{eqnarray*}

\begin{eqnarray*}
  \\
  L : TS \to ST
\end{eqnarray*}

\begin{eqnarray*}
  \\
  P \models E \iff P \in \meaningof{E}
\end{eqnarray*}

\begin{eqnarray*}
  P \approx_{L} Q \iff \forall E \in L. P \models E \iff Q \models E
\end{eqnarray*}

\begin{eqnarray*}
  P \approx_{K} Q
\end{eqnarray*}

\begin{eqnarray*}
  P \approx Q
\end{eqnarray*}

$\approx_{K} = \approx = \approx_{L}$

\subsubsection{Contextual duality}

Note that contexts extend the quotation operation to a family of
operations from processes to names. Given a context, $M$, we can
define a \emph{nominal context}, $\quotep{M}$ by $\quotep{M}[P] :=
\quotep{M[P]}$. To foreshadow what is to come we observe that these
operations enjoy a duality with processes very much like the duality
between vectors and maps from vectors to scalars.

Further, because the calculus is essentially higher-order, we have a
correspondence between contexts and processes. More specifically,
given a name $x$ and a context $M$ we can construct $M^{*}_{x}$ such
that 

\begin{mathpar}
  M^{*}_{x} | \lift{x}{P} \red M[P]
\end{mathpar}

namely,

\begin{mathpar}
  M^{*}_{x} := x?(u).M[\dropn{u}]
\end{mathpar}

The dependence of $M^{*}_{x}$ on a name makes it an abstraction, 

\begin{mathpar}
  M^{*} := (x)x?(u).M[\dropn{u}]
\end{mathpar}

\subsection{Additional notation}

It will sometimes be convenient to denote the process a name
quotes. We already have the notation $x = \quotep{P}$, but it will be
convenient to introduce an alternate notation, $\procn{x}$, when we
want to emphasize the connection to the use of the name. Note that, by
virtue of name equivalence, $\quotep{\procn{x}} \nameeq x$; so, the
notation is consistent with previous definitions.

Further, because names have structure it is possible to effect
substitutions on the basis of that structure. This means we need to
upgrade our notation for substitutions, which we accomplish by
adapting comprehension notation. Thus,

\begin{mathpar}
  P\{ y / x : x \in S \}
\end{mathpar}

is interpreted to mean the process derived from P by replacing (in a
capture-avoiding manner) each occurrence of $x$ in $S$ by $y$. For example,

\begin{mathpar}
  P\{ \quotep{\procn{x}|\procn{x}} / x : x \in \freenames{P} \}
\end{mathpar}

will replace each (occurrence) of a free name $x$ in $P$ by
$\quotep{\procn{x}|\procn{x}}$.

Also, we will avail ourselves of the notation $x^{L}$ and $x^{R}$ to
denote injections of a name into disjoint copies of the name
space. There are numerous ways to accomplish this. One example can be
found in \cite{MeredithR05}. This notation overloads to vectors of
names: $\vec{x}^{\pi} := (x_{i}^{\pi} \; : \; 0 \leq i < |\vec{x}| )$ where $\pi \in \{L,R\}$.

We also use $P^{\Box} := P|\Box$.

In \cite{MeredithR05} an interpretation of the new operator is
given. It turns out that there are several possible interpretations
all enjoying the requisite algebraic properties of the operator (see
\cite{milner91polyadicpi}). We will therefore make liberal use of
$(\nu\; \vec{x})P$.

% subsection the_syntax_and_semantics_of_the_notation_system (end)   

\input{qm2pi.qmops} 

\input{qm2pi.sterngerlach} 

\input{qm2pi.metric} 

% section concurrent_process_calculi (end)

%\input{qm2pi.proofsketch}

% section proof sketch (end)

%\input{qm2pi.slviaknots} 

% section spatial logic via knots (end)

\input{qm2pi.conclusion}

% section conclusion (end)

%\input{qm2pi.dtcodes} 

% section wiring algorithm (end)

\input{qm2pi.ack} 

% section acknowledgments (end)

\newpage


\bibliographystyle{plain}   
\bibliography{../../biblios/main.bib}

\input{qm2pi.rhodetails}

\end{document}

 

%\documentclass[12pt]{llncs}
%\documentclass{jktr}

\usepackage[pdftex]{hyperref}                   
\usepackage {listings}
\usepackage {mathpartir}
\usepackage{bcprules}
%\usepackage{listings}
                       
\usepackage{graphicx} 
%\usepackage[margins=2.5cm,nohead,nofoot]{geometry}
%\usepackage{geometry}
\usepackage{amsfonts}
\usepackage{amstext}
\usepackage{latexsym}
\usepackage{amssymb}
\usepackage{color}


%\include{myPreamble}
\include{qm2pi.local} 

%\ifpdf
%\usepackage[pdftex]{graphicx}
%\else
%\usepackage{graphicx}
%\fi

 % \ifpdf
%  \usepackage{pdfsync}
%  \if


%\title{Brief Article}
%\author{David F. Snyder}
%\author{L.G. Meredith}

%\address{Dept. of Math., Texas State University--San Marcos, San Marcos, TX 78666}
       
\pagestyle{empty}


\begin{document}

\lstset{language=[Objective]Caml,frame=shadowbox}

\input{qm2pi.front}

% section front matter (end)

\input{qm2pi.intro} 
 
% section introduction (end)

% \input{qm2pi.knotations} 

% section notation (end)

\input{qm2pi.process.calculi} 

% section concurrent_process_calculi_and_spatial_logics_ (end)
    
%\input{qm2pi.knots2pi} 

%\input{qm2pi.trefoil} 

%\input{qm2pi.mainthm} 

% subsection basic_interpretation (end)

%\input{qm2pi.rho.presentation} 
\subsection{The syntax and semantics of the notation system}\label{sub:the_syntax_and_semantics_of_the_notation_system} % (fold)

We now summarize a technical presentation of the calculus that
embodies our theory of dynamics. The typical presentation of such a
calculus follows the style of giving generators and relations on
them. The grammar, below, describing term constructors, freely
generates the set of processes, $\Proc$. This set is then quotiented
by a relation known as structural congruence and it is over this set
that the notion of dynamics is expressed. This presentation is
essentially that of \cite{MeredithR05} with the addition of
polyadicity and summation. For readability we have relegated some of
the technical subtleties to an appendix.

\subsubsection{Process grammar}\label{subsub:process_grammar}

\begin{mathpar}
  \inferrule* [lab=synchronization] {} {{M} \bc \pzero \;|\; x?F \;|\; x!C }
  \and
  \inferrule* [lab=abstraction] {} {{F} \bc (x)P}
  \and
  \inferrule* [lab=concretion] {} {{C} \bc \langle Q \rangle}
  \and
  \inferrule* [lab=process] {} {{P,Q} \bc M \;| \;P|Q \;|\; @{x}}
  \and
  \inferrule* [lab=name] {} {{x} \bc \quotep{P}}
\end{mathpar} 

Note that $\vec{x}$ (resp. $\vec{P}$) denotes a vector of names
(resp. processes) of length $|\vec{x}|$ (resp. $|\vec{P}|$). We adopt
the following useful abbreviations.

\begin{mathpar}
   x?(\vec{y}).P := x.(\vec{y})P \and  x\clift{\vec{P}} := x.\clift{\vec{P}}
   \and x!(y) := \lift{x}{\dropn{y}}
   \and \Pi_{i=0}^{n-1}P_i := P_0 | \ldots | P_{n-1}
\end{mathpar}

\subsubsection{Structural congruence}

\paragraph{Free and bound names and alpha-equivalence.} At the
core of structural equivalence is alpha-equivalence which identifies
process that are the same up to a change of variable. Formally, we
recognize the distinction between free and bound names. The free names
of a process, $\freenames{P}$, may be calculated recursively as
follows:

\begin{mathpar}
\freenames{\pzero} := \emptyset
  \and \\
  \freenames{x?(y).P} := \{ x \} \cup (\freenames{P} \setminus \{ y \})
  \and 
  \freenames{x!\langle P \rangle} := \{ x \} \cup \{ P \} 
  \and \\
  \freenames{P|Q} := \freenames{P} \cup \freenames{Q}
  \and \\
  \freenames{@{x}} := \{ x \}
\end{mathpar}

$\pi$
$\quotep{\pi}$

$\freenames{-} : \pi \to \mathcal{P}(\quotep{\pi})$

\begin{eqnarray*}
  \freenames{\pzero} & := & \emptyset \\
  \freenames{x?(y).P} & := & \{ x \} \cup (\freenames{P} \setminus \{ y \}) \\
  \freenames{x!\langle P \rangle} & := & \{ x \} \cup \{ P \} \\
  \freenames{P|Q} & := & \freenames{P} \cup \freenames{Q} \\
  \freenames{\dropn{x}} & := & \{ x \}
\end{eqnarray*}

The bound names of a process, $\boundnames{P}$, are those names occurring in $P$
that are not free. For example, in $x?(y).0$, the name $x$ is free, while $y$ is bound.

\begin{mathpar}
  \inferrule* [lab=monoidal-laws] {} { P|Q \equiv Q|P \and P|0 \equiv P \and P|(Q|R) \equiv (P|Q)|R }
\end{mathpar}

\begin{mathpar}
  \inferrule* [lab=alpha-equivalence] {} { (x)P \equiv (y)P\{y/x\} \and y \not\in \freenames{P} }
\end{mathpar}

\begin{definition}
Then two processes, $P,Q$, are alpha-equivalent if $P = Q\{\vec{y}/\vec{x}\}$ for
some $\vec{x} \in \boundnames{Q},\vec{y} \in \boundnames{P}$, where $Q\{\vec{y}/\vec{x}\}$
denotes the capture-avoiding substitution of $\vec{y}$ for $\vec{x}$ in $Q$.
\end{definition}

\begin{definition}
  The {\em structural congruence} \cite{SangiorgiWalker} , $\equiv$,
  between processes is the least congruence containing
  alpha-equivalence, satisfying the abelian monoid laws
  (associativity, commutativity and $\pzero$ as identity) for parallel
  composition $|$ and for summation $+$.
\end{definition}

\subsection{Name equivalence}

We take name equivalence, written $\nameeq$, to be the smallest
equivalence relation generated by the following rules.

\begin{mathpar}
\inferrule*[lab=Quote-drop]
{ }
{ \quotep{@{x}} \nameeq x }

\inferrule*[lab=Struct-equiv]
{ P \scong Q }
{ \quotep{P} \nameeq \quotep{Q} }
\end{mathpar}

The astute reader will have noticed that the mutual recursion of names
and processes imposes a mutual recursion on alpha-equivalence and
structural equivalence via name-equivalence. Fortunately, all of this
works out pleasantly and we may calculate in the natural way, free of
concern. The reader interested in the details is referred to the
appendix \ref{appendix:rho_details}.

\subsection{Substitution}

We use $\Proc$ for the set of processes, $\QProc$ for the set of
names, and $\id{\{}\vec{y} / \vec{x} \id{\}}$ to denote partial maps,
$s : \QProc \rightarrow \QProc$. A map, $s$ lifts, uniquely, to a map
on process terms, $\widehat{s} : \Proc \rightarrow \Proc$ by the
following equations.

\begin{mathpar}
  (0) \psubstp{Q}{P} := 0 \\
  (R \juxtap S) \psubstp{Q}{P}
  :=    
  (R)\psubstp{Q}{P} \juxtap (S) \psubstp{Q}{P} \\
  (x?(y).R) \psubstp{Q}{P}    
  :=    
  (x)\substp{Q}{P} (z)\concat( (R \psubstn{z}{y}) \psubstp{Q}{P} ) \\
  (\lift{x}{R}) \psubstp{Q}{P}  
  :=
  \lift{(x)\substp{Q}{P}}{ R \psubstp{Q}{P} } \\
%   (\dropn{x})  \psubstp{Q}{P}       
%   := 
%   \left\{ 
%     \begin{array}{ccc} 
%       \dropn{\quotep{Q}} & & x \nameeq \quotep{P} \\
%       \dropn{x} & & otherwise \\
%     \end{array}
%   \right. 
  (\dropn{x})  \psubstp{Q}{P}       
  := 
  \left\{ 
    \begin{array}{ccc} 
      Q & & x \nameeq \quotep{P} \\
      \dropn{x} & & otherwise \\
    \end{array}
  \right.
\end{mathpar}
 

where

\begin{eqnarray}
  (x)\id{\{} \lpquote Q \rpquote / \lpquote P \rpquote \id{\}}            = 
  \left\{ 
    \begin{array}{ccc}
      \lpquote Q \rpquote & & x \nameeq \lpquote P \rpquote \\
      x & & otherwise \\
    \end{array}
  \right. \nonumber
\end{eqnarray}

and $z$ is chosen distinct from $\quotep{P}$, $\quotep{Q}$, the free
names in $Q$, and all the names in $R$. Our $\alpha$-equivalence will
be built in the standard way from this substitution.

\begin{remark}\label{rem:no_self_referential_names}
  One consequence of these definitions is that $\forall P. \quotep{P}
  \not\in \freenames{P}$.
\end{remark}

\subsection{ Dynamic quote: an example }

Anticipating something of what's to come, consider applying the
substitution, $\widehat{\id{\{}u / z \id{\}}}$, to the following pair
of processes, $\lift{w}{y!(z)}$ and $w[ \lpquote y!(z) \rpquote ]$.

\begin{eqnarray}
	\lift{w}{y!(z)}\widehat{\id{\{}u / z \id{\}}}
		& = &
		\lift{w}{y!(u)} \nonumber\\
	w[ \lpquote y!(z) \rpquote ] \widehat{ \id{\{}u / z \id{\}} }
		& = &
		w[ \lpquote y!(z) \rpquote ] \nonumber
\end{eqnarray}

Because the body of the process between quotes is impervious to
substitution, we get radically different answers. In fact, by
examining the first process in an input context,
e.g. $x?(z).\lift{w}{y!(z)}$, we see that the process under the lift
operator may be shaped by prefixed inputs binding a name inside it. In
this sense, the lift operator will be seen as a way to dynamically
construct processes before reifying them as names.

Finally equipped with these standard features we can present the
dynamics of the calculus.

\subsubsection{Operational semantics} 

Finally, we introduce the computational dynamics. What marks these
algebras as distinct from other more traditionally studied algebraic
structures, e.g. vector spaces or polynomial rings, is the manner in
which dynamics is captured. In traditional structures, dynamics is typically
expressed through morphisms between such structures, as in linear maps
between vector spaces or morphisms between rings. In algebras
associated with the semantics of computation, the dynamics is
expressed as part of the algebraic structure itself, through a
reduction reduction relation typically denoted by $\red$. Below, we
give a recursive presentation of this relation for the calculus used
in the encoding.

$\red \subseteq \pi \times \pi$
$\red : \pi \to \mathcal{P}(\pi)$

\begin{mathpar}
  \inferrule* [lab=Comm] { \textsf{match}( x_{src}, x_{trgt} ) } { x_{trgt}?(y)P \; | \; x_{src}!\langle {Q} \rangle \red P\{\quotep{Q}/y}\} }
  \and \\
  \inferrule* [lab=Par] {{P} \red {P}'} {{{P} | {Q}} \red {{P}' | {Q}}}
  \and
  \inferrule* [lab=Equiv]{{{P} \scong {P}'} \andalso {{P}' \red {Q}'} \andalso {{Q}' \scong {Q}}}{{P} \red {Q}}
\end{mathpar}

\begin{eqnarray*}
  match_{\equiv} (\quotep{P},\quotep{Q}) & := & P \equiv Q \\
  match_{\dagger}(\quotep{P},\quotep{Q}) & := & \forall R. P|Q \red^{*} R => R \red^{*} 0 \\
  match_{K}(\quotep{P},\quotep{Q}) & := & K \mbox{ for some context } K
\end{eqnarray*}

$u?(x)P | u!\langle Q \rangle \red P\{\quotep{Q}/x\}$

%We write $\wred$ for $\red^*$, and $P\red$ if $\exists Q $ such that $ P \red Q$.
We write $P\red$ if $\exists Q $ such that $ P \red Q$ and $P\not\red$, otherwise.

\section{Replication}

As mentioned before, it is known that replication (and hence
recursion) can be implemented in a higher-order process algebra
\cite{SangiorgiWalker}. As our first example of calculation with the
machinery thus far presented we give the construction explicitly in
the {\rhoc}.

\begin{eqnarray}
	D_{x} & := & \prefix{x}{y}{(\binpar{\outputp{x}{y}}{@{y}})} \nonumber\\
	\bangp_{x}{P} & := & \binpar{{x}!\langle{\binpar{D_{x}}{P}}\rangle}{D_{x}} \nonumber
\end{eqnarray}

\begin{eqnarray}
	\bangp_{x}{P} & & \nonumber\\
	=
	& {x}!\langle{(\prefix{x}{y}{(\outputp{x}{y} | @{y})) | P}}\rangle 
	      | \prefix{x}{y}{(\outputp{x}{y} | @{y})} & \nonumber\\
	\red
	& (\outputp{x}{y} | @{y})\substn{\quotep{(\prefix{x}{y}{(@{y} | \outputp{x}{y})) | P}}}{y} & \nonumber\\
	=
	& \outputp{x}{\quotep{(\prefix{x}{y}{(\outputp{x}{y} | @{y})) | P}}}
	  | {(\prefix{x}{y}{(\outputp{x}{y} | @{y})) | P}} & \nonumber\\
	\red
	& \ldots & \nonumber\\
	\red^*
	& P | P | \ldots & \nonumber
\end{eqnarray}

Of course, this encoding, as an implementation, runs away, unfolding
$\bangp{P}$ eagerly. A lazier and more implementable replication
operator, restricted to input-guarded processes, may be obtained as follows.

\begin{eqnarray}
\bangp{\prefix{u}{v}{P}} 
	:= 
	\binpar{\lift{x}{\prefix{u}{v}{(\binpar{D(x)}{P})}}}{D(x)} \nonumber
\end{eqnarray}

\begin{remark}
  Note that the lazier definition still does not deal with summation
  or mixed summation (i.e. sums over input and output). The reader is
  invited to construct definitions of replication that deal with these
  features. 

  Further, the definitions are parameterized in a name, $x$. Can you,
  gentle reader, make a definition that eliminates this parameter and
  guarantees no accidental interaction between the replication
  machinery and the process being replicated -- i.e. no accidental
  sharing of names used by the process to get its work done and the
  name(s) used by the replication to effect copying. This latter
  revision of the definition of replication is crucial to obtaining
  the expected identity $!!P \sim !P$.
\end{remark}

\begin{remark}\label{rem:paradoxical_combinator}
  The reader familiar with the lambda calculus will have noticed the
  similarity between $D$ and the paradoxical combinator.

  [Ed. note: the existence of this seems to suggest we have to be more
  restrictive on the set of processes and names we admit if we are to
  support no-cloning.]
\end{remark}

\subsubsection{Bisimulation}

The computational dynamics gives rise to another kind of equivalence,
the equivalence of computational behavior. As previously mentioned
this is typically captured \emph{via} some form of bisimulation.

% The notion we use in this paper is weak barbed bisimulation
% \cite{milner91polyadicpi}.

The notion we use in this paper is derived from weak barbed
bisimulation \cite{milner91polyadicpi}. 

\begin{definition}
An \emph{observation relation}, $\downarrow_{\mathcal N}$, over a set
of names, $\mathcal N$, is the smallest relation satisfying the rules
below.

\infrule[Out-barb]{y \in {\mathcal N}, \; x \nameeq y}
		  {\outputp{x}{v} \downarrow_{\mathcal N} x}
\infrule[Par-barb]{\mbox{$P\downarrow_{\mathcal N} x$ or $Q\downarrow_{\mathcal N} x$}}
		  {\binpar{P}{Q} \downarrow_{\mathcal N} x}

We write $P \Downarrow_{\mathcal N} x$ if there is $Q$ such that 
$P \wred Q$ and $Q \downarrow_{\mathcal N} x$.
\end{definition}

\begin{definition}
%\label{def.bbisim}
An  ${\mathcal N}$-\emph{barbed bisimulation} over a set of names, ${\mathcal N}$, is a symmetric binary relation 
${\mathcal S}_{\mathcal N}$ between agents such that $P\rel{S}_{\mathcal N}Q$ implies:
\begin{enumerate}
\item If $P \red P'$ then $Q \wred Q'$ and $P'\rel{S}_{\mathcal N} Q'$.
\item If $P\downarrow_{\mathcal N} x$, then $Q\Downarrow_{\mathcal N} x$.
\end{enumerate}
$P$ is ${\mathcal N}$-barbed bisimilar to $Q$, written
$P \wbbisim_{\mathcal N} Q$, if $P \rel{S}_{\mathcal N} Q$ for some ${\mathcal N}$-barbed bisimulation ${\mathcal S}_{\mathcal N}$.
\end{definition}

$\mathcal{R} \subseteq \pi \times \pi$

$P \mathcal{R} Q => \forall P'. P \red P' \Rightarrow \exists Q'. Q \red Q', P' \mathcal{R} Q'$

$P \vdash x \Rightarrow Q \vdash x$

\begin{mathpar}
  \inferrule*[lab=Out-barb]{x \nameeq y}{{y}!\langle{Q}\rangle \vdash x}
  \and
  \inferrule*[lab=Par-barb]{\mbox{$P\vdash x$ or $Q\vdash x$}}{\binpar{P}{Q} \vdash x}
\end{mathpar}

\subsubsection{Contexts}

One of the principle advantages of computational calculi like the
$\pi$-calculus is a well-defined notion of context,
contextual-equivalence and a correlation between
contextual-equivalence and notions of bisimulation. The notion of
context allows the decomposition of a process into (sub-)process and
its syntactic environment, its context. Thus, a context may be
thought of as a process with a ``hole'' (written $\Box$) in it. The
application of a context $M$ to a process $P$, written $M[P]$, is
tantamount to filling the hole in $M$ with $P$. In this paper we do
not need the full weight of this theory, but do make use of the notion
of context in the proof the main theorem. 

\begin{mathpar}
  \inferrule* [lab=summation] {} {{M_{M},M_{N}} \bc \Box \;|\; x.M_{A} \;|\; M_{M}+M_{N}}
  \and
  \inferrule* [lab=agent] {} {{M_{A}} \bc (\vec{x})M_{P} \;| \; \clift{P_0,\ldots,M_{P},\ldots,P_N}}
  \and \\
  \inferrule* [lab=process] {} {{M_{P}} \bc M_{N} \;| \;P|M_{P} }
\end{mathpar} 

\begin{mathpar}
  \inferrule* [lab=sychronization] {} {M_{N} \bc \Box \;|\; x?M_{F} \;|\; x!M_{C}}
  \and
  \inferrule* [lab=abstraction] {} {{M_{F}} \bc (x)M_{P} }
  \and
  \inferrule* [lab=concretion] {} {{M_{C}} \bc \langle M_{P} \rangle }
  \and \\
  \inferrule* [lab=process] {} {{M_{P}} \bc M_{N} \;| \;P|M_{P} }
\end{mathpar}

\begin{definition}[contextual application] Given a context $M$, and
  process $P$, we define the \emph{contextual application}, $M[P] :=
  M\{P/\Box\}$. That is, the contextual application of M to P is the
  substitution of $P$ for $\Box$ in $M$.
\end{definition}

$\meaningof{-} : L \to \mathcal{P}(\pi)$

\begin{mathpar}
  \inferrule* [lab=collection] {} {\meaningof{true} = \pi, \and \meaningof{~E} = \pi \setminus \meaningof{E}, \and \meaningof{E_{1} \& E_{2}} = \meaningof{E_{1}} \cap \meaningof{E_{2}}}
\end{mathpar}

\begin{mathpar}
  \inferrule* [lab=structure] {} {\meaningof{0} = \{ P \in \pi | P \equiv 0 \}, \and \\ \meaningof{E_1 | E_2} = \{ P \in \pi | P \equiv P_{1} | P_{2}, P_{1} \in \meaningof{E_{1}}, P_{2} \in \meaningof{E_2}\} }
\end{mathpar}

\begin{mathpar}
 \inferrule* [lab=behavior] {} {\meaningof{\langle a?b \rangle E} = \{ P \in \pi | P \equiv Q | u?(y)P', \\ \and \\\\ \and \\ \;\;\; u \in \meaningof{a}, \forall z.P'\{z/y\} \in \meaningof{E\{z/b\}}\}, \and \\ \meaningof{a!E} = \{ P \in \pi | P \equiv Q | x!\langle P' \rangle, x \in \meaningof{a} P' \in \meaningof{E}\} }
\end{mathpar}

\begin{mathpar}
 \inferrule* [lab=nominal] {} {\meaningof{\quotep{E}} = \{ \quotep{P} \in \quotep{\pi} | P \in \meaningof{E} \}, \and \meaningof{\quotep{P}} = \{ \quotep{Q} \in \quotep{\pi} | P \equiv Q \} \and \\ \meaningof{@\quotep{E}} = \{ P \in \pi | P \equiv @x, x \in \meaningof{E} \}}
\end{mathpar}

\begin{eqnarray*}
  \\
  \meaningof{-} : TS \to ST
\end{eqnarray*}

\begin{eqnarray*}
  \\
  L : TS \to ST
\end{eqnarray*}

\begin{eqnarray*}
  \\
  P \models E \iff P \in \meaningof{E}
\end{eqnarray*}

\begin{eqnarray*}
  P \approx_{L} Q \iff \forall E \in L. P \models E \iff Q \models E
\end{eqnarray*}

\begin{eqnarray*}
  P \approx_{K} Q
\end{eqnarray*}

\begin{eqnarray*}
  P \approx Q
\end{eqnarray*}

$\approx_{K} = \approx = \approx_{L}$

\subsubsection{Contextual duality}

Note that contexts extend the quotation operation to a family of
operations from processes to names. Given a context, $M$, we can
define a \emph{nominal context}, $\quotep{M}$ by $\quotep{M}[P] :=
\quotep{M[P]}$. To foreshadow what is to come we observe that these
operations enjoy a duality with processes very much like the duality
between vectors and maps from vectors to scalars.

Further, because the calculus is essentially higher-order, we have a
correspondence between contexts and processes. More specifically,
given a name $x$ and a context $M$ we can construct $M^{*}_{x}$ such
that 

\begin{mathpar}
  M^{*}_{x} | \lift{x}{P} \red M[P]
\end{mathpar}

namely,

\begin{mathpar}
  M^{*}_{x} := x?(u).M[\dropn{u}]
\end{mathpar}

The dependence of $M^{*}_{x}$ on a name makes it an abstraction, 

\begin{mathpar}
  M^{*} := (x)x?(u).M[\dropn{u}]
\end{mathpar}

\subsection{Additional notation}

It will sometimes be convenient to denote the process a name
quotes. We already have the notation $x = \quotep{P}$, but it will be
convenient to introduce an alternate notation, $\procn{x}$, when we
want to emphasize the connection to the use of the name. Note that, by
virtue of name equivalence, $\quotep{\procn{x}} \nameeq x$; so, the
notation is consistent with previous definitions.

Further, because names have structure it is possible to effect
substitutions on the basis of that structure. This means we need to
upgrade our notation for substitutions, which we accomplish by
adapting comprehension notation. Thus,

\begin{mathpar}
  P\{ y / x : x \in S \}
\end{mathpar}

is interpreted to mean the process derived from P by replacing (in a
capture-avoiding manner) each occurrence of $x$ in $S$ by $y$. For example,

\begin{mathpar}
  P\{ \quotep{\procn{x}|\procn{x}} / x : x \in \freenames{P} \}
\end{mathpar}

will replace each (occurrence) of a free name $x$ in $P$ by
$\quotep{\procn{x}|\procn{x}}$.

Also, we will avail ourselves of the notation $x^{L}$ and $x^{R}$ to
denote injections of a name into disjoint copies of the name
space. There are numerous ways to accomplish this. One example can be
found in \cite{MeredithR05}. This notation overloads to vectors of
names: $\vec{x}^{\pi} := (x_{i}^{\pi} \; : \; 0 \leq i < |\vec{x}| )$ where $\pi \in \{L,R\}$.

We also use $P^{\Box} := P|\Box$.

In \cite{MeredithR05} an interpretation of the new operator is
given. It turns out that there are several possible interpretations
all enjoying the requisite algebraic properties of the operator (see
\cite{milner91polyadicpi}). We will therefore make liberal use of
$(\nu\; \vec{x})P$.

% subsection the_syntax_and_semantics_of_the_notation_system (end)   

\input{qm2pi.qmops} 

\input{qm2pi.sterngerlach} 

\input{qm2pi.metric} 

% section concurrent_process_calculi (end)

%\input{qm2pi.proofsketch}

% section proof sketch (end)

%\input{qm2pi.slviaknots} 

% section spatial logic via knots (end)

\input{qm2pi.conclusion}

% section conclusion (end)

%\input{qm2pi.dtcodes} 

% section wiring algorithm (end)

\input{qm2pi.ack} 

% section acknowledgments (end)

\newpage


\bibliographystyle{plain}   
\bibliography{../../biblios/main.bib}

\input{qm2pi.rhodetails}

\end{document}

 

% subsection basic_interpretation (end)

%\input{qm2pi.rho.presentation} 
\subsection{The syntax and semantics of the notation system}\label{sub:the_syntax_and_semantics_of_the_notation_system} % (fold)

We now summarize a technical presentation of the calculus that
embodies our theory of dynamics. The typical presentation of such a
calculus follows the style of giving generators and relations on
them. The grammar, below, describing term constructors, freely
generates the set of processes, $\Proc$. This set is then quotiented
by a relation known as structural congruence and it is over this set
that the notion of dynamics is expressed. This presentation is
essentially that of \cite{MeredithR05} with the addition of
polyadicity and summation. For readability we have relegated some of
the technical subtleties to an appendix.

\subsubsection{Process grammar}\label{subsub:process_grammar}

\begin{mathpar}
  \inferrule* [lab=synchronization] {} {{M} \bc \pzero \;|\; x?F \;|\; x!C }
  \and
  \inferrule* [lab=abstraction] {} {{F} \bc (x)P}
  \and
  \inferrule* [lab=concretion] {} {{C} \bc \langle Q \rangle}
  \and
  \inferrule* [lab=process] {} {{P,Q} \bc M \;| \;P|Q \;|\; @{x}}
  \and
  \inferrule* [lab=name] {} {{x} \bc \quotep{P}}
\end{mathpar} 

Note that $\vec{x}$ (resp. $\vec{P}$) denotes a vector of names
(resp. processes) of length $|\vec{x}|$ (resp. $|\vec{P}|$). We adopt
the following useful abbreviations.

\begin{mathpar}
   x?(\vec{y}).P := x.(\vec{y})P \and  x\clift{\vec{P}} := x.\clift{\vec{P}}
   \and x!(y) := \lift{x}{\dropn{y}}
   \and \Pi_{i=0}^{n-1}P_i := P_0 | \ldots | P_{n-1}
\end{mathpar}

\subsubsection{Structural congruence}

\paragraph{Free and bound names and alpha-equivalence.} At the
core of structural equivalence is alpha-equivalence which identifies
process that are the same up to a change of variable. Formally, we
recognize the distinction between free and bound names. The free names
of a process, $\freenames{P}$, may be calculated recursively as
follows:

\begin{mathpar}
\freenames{\pzero} := \emptyset
  \and \\
  \freenames{x?(y).P} := \{ x \} \cup (\freenames{P} \setminus \{ y \})
  \and 
  \freenames{x!\langle P \rangle} := \{ x \} \cup \{ P \} 
  \and \\
  \freenames{P|Q} := \freenames{P} \cup \freenames{Q}
  \and \\
  \freenames{@{x}} := \{ x \}
\end{mathpar}

$\pi$
$\quotep{\pi}$

$\freenames{-} : \pi \to \mathcal{P}(\quotep{\pi})$

\begin{eqnarray*}
  \freenames{\pzero} & := & \emptyset \\
  \freenames{x?(y).P} & := & \{ x \} \cup (\freenames{P} \setminus \{ y \}) \\
  \freenames{x!\langle P \rangle} & := & \{ x \} \cup \{ P \} \\
  \freenames{P|Q} & := & \freenames{P} \cup \freenames{Q} \\
  \freenames{\dropn{x}} & := & \{ x \}
\end{eqnarray*}

The bound names of a process, $\boundnames{P}$, are those names occurring in $P$
that are not free. For example, in $x?(y).0$, the name $x$ is free, while $y$ is bound.

\begin{mathpar}
  \inferrule* [lab=monoidal-laws] {} { P|Q \equiv Q|P \and P|0 \equiv P \and P|(Q|R) \equiv (P|Q)|R }
\end{mathpar}

\begin{mathpar}
  \inferrule* [lab=alpha-equivalence] {} { (x)P \equiv (y)P\{y/x\} \and y \not\in \freenames{P} }
\end{mathpar}

\begin{definition}
Then two processes, $P,Q$, are alpha-equivalent if $P = Q\{\vec{y}/\vec{x}\}$ for
some $\vec{x} \in \boundnames{Q},\vec{y} \in \boundnames{P}$, where $Q\{\vec{y}/\vec{x}\}$
denotes the capture-avoiding substitution of $\vec{y}$ for $\vec{x}$ in $Q$.
\end{definition}

\begin{definition}
  The {\em structural congruence} \cite{SangiorgiWalker} , $\equiv$,
  between processes is the least congruence containing
  alpha-equivalence, satisfying the abelian monoid laws
  (associativity, commutativity and $\pzero$ as identity) for parallel
  composition $|$ and for summation $+$.
\end{definition}

\subsection{Name equivalence}

We take name equivalence, written $\nameeq$, to be the smallest
equivalence relation generated by the following rules.

\begin{mathpar}
\inferrule*[lab=Quote-drop]
{ }
{ \quotep{@{x}} \nameeq x }

\inferrule*[lab=Struct-equiv]
{ P \scong Q }
{ \quotep{P} \nameeq \quotep{Q} }
\end{mathpar}

The astute reader will have noticed that the mutual recursion of names
and processes imposes a mutual recursion on alpha-equivalence and
structural equivalence via name-equivalence. Fortunately, all of this
works out pleasantly and we may calculate in the natural way, free of
concern. The reader interested in the details is referred to the
appendix \ref{appendix:rho_details}.

\subsection{Substitution}

We use $\Proc$ for the set of processes, $\QProc$ for the set of
names, and $\id{\{}\vec{y} / \vec{x} \id{\}}$ to denote partial maps,
$s : \QProc \rightarrow \QProc$. A map, $s$ lifts, uniquely, to a map
on process terms, $\widehat{s} : \Proc \rightarrow \Proc$ by the
following equations.

\begin{mathpar}
  (0) \psubstp{Q}{P} := 0 \\
  (R \juxtap S) \psubstp{Q}{P}
  :=    
  (R)\psubstp{Q}{P} \juxtap (S) \psubstp{Q}{P} \\
  (x?(y).R) \psubstp{Q}{P}    
  :=    
  (x)\substp{Q}{P} (z)\concat( (R \psubstn{z}{y}) \psubstp{Q}{P} ) \\
  (\lift{x}{R}) \psubstp{Q}{P}  
  :=
  \lift{(x)\substp{Q}{P}}{ R \psubstp{Q}{P} } \\
%   (\dropn{x})  \psubstp{Q}{P}       
%   := 
%   \left\{ 
%     \begin{array}{ccc} 
%       \dropn{\quotep{Q}} & & x \nameeq \quotep{P} \\
%       \dropn{x} & & otherwise \\
%     \end{array}
%   \right. 
  (\dropn{x})  \psubstp{Q}{P}       
  := 
  \left\{ 
    \begin{array}{ccc} 
      Q & & x \nameeq \quotep{P} \\
      \dropn{x} & & otherwise \\
    \end{array}
  \right.
\end{mathpar}
 

where

\begin{eqnarray}
  (x)\id{\{} \lpquote Q \rpquote / \lpquote P \rpquote \id{\}}            = 
  \left\{ 
    \begin{array}{ccc}
      \lpquote Q \rpquote & & x \nameeq \lpquote P \rpquote \\
      x & & otherwise \\
    \end{array}
  \right. \nonumber
\end{eqnarray}

and $z$ is chosen distinct from $\quotep{P}$, $\quotep{Q}$, the free
names in $Q$, and all the names in $R$. Our $\alpha$-equivalence will
be built in the standard way from this substitution.

\begin{remark}\label{rem:no_self_referential_names}
  One consequence of these definitions is that $\forall P. \quotep{P}
  \not\in \freenames{P}$.
\end{remark}

\subsection{ Dynamic quote: an example }

Anticipating something of what's to come, consider applying the
substitution, $\widehat{\id{\{}u / z \id{\}}}$, to the following pair
of processes, $\lift{w}{y!(z)}$ and $w[ \lpquote y!(z) \rpquote ]$.

\begin{eqnarray}
	\lift{w}{y!(z)}\widehat{\id{\{}u / z \id{\}}}
		& = &
		\lift{w}{y!(u)} \nonumber\\
	w[ \lpquote y!(z) \rpquote ] \widehat{ \id{\{}u / z \id{\}} }
		& = &
		w[ \lpquote y!(z) \rpquote ] \nonumber
\end{eqnarray}

Because the body of the process between quotes is impervious to
substitution, we get radically different answers. In fact, by
examining the first process in an input context,
e.g. $x?(z).\lift{w}{y!(z)}$, we see that the process under the lift
operator may be shaped by prefixed inputs binding a name inside it. In
this sense, the lift operator will be seen as a way to dynamically
construct processes before reifying them as names.

Finally equipped with these standard features we can present the
dynamics of the calculus.

\subsubsection{Operational semantics} 

Finally, we introduce the computational dynamics. What marks these
algebras as distinct from other more traditionally studied algebraic
structures, e.g. vector spaces or polynomial rings, is the manner in
which dynamics is captured. In traditional structures, dynamics is typically
expressed through morphisms between such structures, as in linear maps
between vector spaces or morphisms between rings. In algebras
associated with the semantics of computation, the dynamics is
expressed as part of the algebraic structure itself, through a
reduction reduction relation typically denoted by $\red$. Below, we
give a recursive presentation of this relation for the calculus used
in the encoding.

$\red \subseteq \pi \times \pi$
$\red : \pi \to \mathcal{P}(\pi)$

\begin{mathpar}
  \inferrule* [lab=Comm] { \textsf{match}( x_{src}, x_{trgt} ) } { x_{trgt}?(y)P \; | \; x_{src}!\langle {Q} \rangle \red P\{\quotep{Q}/y}\} }
  \and \\
  \inferrule* [lab=Par] {{P} \red {P}'} {{{P} | {Q}} \red {{P}' | {Q}}}
  \and
  \inferrule* [lab=Equiv]{{{P} \scong {P}'} \andalso {{P}' \red {Q}'} \andalso {{Q}' \scong {Q}}}{{P} \red {Q}}
\end{mathpar}

\begin{eqnarray*}
  match_{\equiv} (\quotep{P},\quotep{Q}) & := & P \equiv Q \\
  match_{\dagger}(\quotep{P},\quotep{Q}) & := & \forall R. P|Q \red^{*} R => R \red^{*} 0 \\
  match_{K}(\quotep{P},\quotep{Q}) & := & K \mbox{ for some context } K
\end{eqnarray*}

$u?(x)P | u!\langle Q \rangle \red P\{\quotep{Q}/x\}$

%We write $\wred$ for $\red^*$, and $P\red$ if $\exists Q $ such that $ P \red Q$.
We write $P\red$ if $\exists Q $ such that $ P \red Q$ and $P\not\red$, otherwise.

\section{Replication}

As mentioned before, it is known that replication (and hence
recursion) can be implemented in a higher-order process algebra
\cite{SangiorgiWalker}. As our first example of calculation with the
machinery thus far presented we give the construction explicitly in
the {\rhoc}.

\begin{eqnarray}
	D_{x} & := & \prefix{x}{y}{(\binpar{\outputp{x}{y}}{@{y}})} \nonumber\\
	\bangp_{x}{P} & := & \binpar{{x}!\langle{\binpar{D_{x}}{P}}\rangle}{D_{x}} \nonumber
\end{eqnarray}

\begin{eqnarray}
	\bangp_{x}{P} & & \nonumber\\
	=
	& {x}!\langle{(\prefix{x}{y}{(\outputp{x}{y} | @{y})) | P}}\rangle 
	      | \prefix{x}{y}{(\outputp{x}{y} | @{y})} & \nonumber\\
	\red
	& (\outputp{x}{y} | @{y})\substn{\quotep{(\prefix{x}{y}{(@{y} | \outputp{x}{y})) | P}}}{y} & \nonumber\\
	=
	& \outputp{x}{\quotep{(\prefix{x}{y}{(\outputp{x}{y} | @{y})) | P}}}
	  | {(\prefix{x}{y}{(\outputp{x}{y} | @{y})) | P}} & \nonumber\\
	\red
	& \ldots & \nonumber\\
	\red^*
	& P | P | \ldots & \nonumber
\end{eqnarray}

Of course, this encoding, as an implementation, runs away, unfolding
$\bangp{P}$ eagerly. A lazier and more implementable replication
operator, restricted to input-guarded processes, may be obtained as follows.

\begin{eqnarray}
\bangp{\prefix{u}{v}{P}} 
	:= 
	\binpar{\lift{x}{\prefix{u}{v}{(\binpar{D(x)}{P})}}}{D(x)} \nonumber
\end{eqnarray}

\begin{remark}
  Note that the lazier definition still does not deal with summation
  or mixed summation (i.e. sums over input and output). The reader is
  invited to construct definitions of replication that deal with these
  features. 

  Further, the definitions are parameterized in a name, $x$. Can you,
  gentle reader, make a definition that eliminates this parameter and
  guarantees no accidental interaction between the replication
  machinery and the process being replicated -- i.e. no accidental
  sharing of names used by the process to get its work done and the
  name(s) used by the replication to effect copying. This latter
  revision of the definition of replication is crucial to obtaining
  the expected identity $!!P \sim !P$.
\end{remark}

\begin{remark}\label{rem:paradoxical_combinator}
  The reader familiar with the lambda calculus will have noticed the
  similarity between $D$ and the paradoxical combinator.

  [Ed. note: the existence of this seems to suggest we have to be more
  restrictive on the set of processes and names we admit if we are to
  support no-cloning.]
\end{remark}

\subsubsection{Bisimulation}

The computational dynamics gives rise to another kind of equivalence,
the equivalence of computational behavior. As previously mentioned
this is typically captured \emph{via} some form of bisimulation.

% The notion we use in this paper is weak barbed bisimulation
% \cite{milner91polyadicpi}.

The notion we use in this paper is derived from weak barbed
bisimulation \cite{milner91polyadicpi}. 

\begin{definition}
An \emph{observation relation}, $\downarrow_{\mathcal N}$, over a set
of names, $\mathcal N$, is the smallest relation satisfying the rules
below.

\infrule[Out-barb]{y \in {\mathcal N}, \; x \nameeq y}
		  {\outputp{x}{v} \downarrow_{\mathcal N} x}
\infrule[Par-barb]{\mbox{$P\downarrow_{\mathcal N} x$ or $Q\downarrow_{\mathcal N} x$}}
		  {\binpar{P}{Q} \downarrow_{\mathcal N} x}

We write $P \Downarrow_{\mathcal N} x$ if there is $Q$ such that 
$P \wred Q$ and $Q \downarrow_{\mathcal N} x$.
\end{definition}

\begin{definition}
%\label{def.bbisim}
An  ${\mathcal N}$-\emph{barbed bisimulation} over a set of names, ${\mathcal N}$, is a symmetric binary relation 
${\mathcal S}_{\mathcal N}$ between agents such that $P\rel{S}_{\mathcal N}Q$ implies:
\begin{enumerate}
\item If $P \red P'$ then $Q \wred Q'$ and $P'\rel{S}_{\mathcal N} Q'$.
\item If $P\downarrow_{\mathcal N} x$, then $Q\Downarrow_{\mathcal N} x$.
\end{enumerate}
$P$ is ${\mathcal N}$-barbed bisimilar to $Q$, written
$P \wbbisim_{\mathcal N} Q$, if $P \rel{S}_{\mathcal N} Q$ for some ${\mathcal N}$-barbed bisimulation ${\mathcal S}_{\mathcal N}$.
\end{definition}

$\mathcal{R} \subseteq \pi \times \pi$

$P \mathcal{R} Q => \forall P'. P \red P' \Rightarrow \exists Q'. Q \red Q', P' \mathcal{R} Q'$

$P \vdash x \Rightarrow Q \vdash x$

\begin{mathpar}
  \inferrule*[lab=Out-barb]{x \nameeq y}{{y}!\langle{Q}\rangle \vdash x}
  \and
  \inferrule*[lab=Par-barb]{\mbox{$P\vdash x$ or $Q\vdash x$}}{\binpar{P}{Q} \vdash x}
\end{mathpar}

\subsubsection{Contexts}

One of the principle advantages of computational calculi like the
$\pi$-calculus is a well-defined notion of context,
contextual-equivalence and a correlation between
contextual-equivalence and notions of bisimulation. The notion of
context allows the decomposition of a process into (sub-)process and
its syntactic environment, its context. Thus, a context may be
thought of as a process with a ``hole'' (written $\Box$) in it. The
application of a context $M$ to a process $P$, written $M[P]$, is
tantamount to filling the hole in $M$ with $P$. In this paper we do
not need the full weight of this theory, but do make use of the notion
of context in the proof the main theorem. 

\begin{mathpar}
  \inferrule* [lab=summation] {} {{M_{M},M_{N}} \bc \Box \;|\; x.M_{A} \;|\; M_{M}+M_{N}}
  \and
  \inferrule* [lab=agent] {} {{M_{A}} \bc (\vec{x})M_{P} \;| \; \clift{P_0,\ldots,M_{P},\ldots,P_N}}
  \and \\
  \inferrule* [lab=process] {} {{M_{P}} \bc M_{N} \;| \;P|M_{P} }
\end{mathpar} 

\begin{mathpar}
  \inferrule* [lab=sychronization] {} {M_{N} \bc \Box \;|\; x?M_{F} \;|\; x!M_{C}}
  \and
  \inferrule* [lab=abstraction] {} {{M_{F}} \bc (x)M_{P} }
  \and
  \inferrule* [lab=concretion] {} {{M_{C}} \bc \langle M_{P} \rangle }
  \and \\
  \inferrule* [lab=process] {} {{M_{P}} \bc M_{N} \;| \;P|M_{P} }
\end{mathpar}

\begin{definition}[contextual application] Given a context $M$, and
  process $P$, we define the \emph{contextual application}, $M[P] :=
  M\{P/\Box\}$. That is, the contextual application of M to P is the
  substitution of $P$ for $\Box$ in $M$.
\end{definition}

$\meaningof{-} : L \to \mathcal{P}(\pi)$

\begin{mathpar}
  \inferrule* [lab=collection] {} {\meaningof{true} = \pi, \and \meaningof{~E} = \pi \setminus \meaningof{E}, \and \meaningof{E_{1} \& E_{2}} = \meaningof{E_{1}} \cap \meaningof{E_{2}}}
\end{mathpar}

\begin{mathpar}
  \inferrule* [lab=structure] {} {\meaningof{0} = \{ P \in \pi | P \equiv 0 \}, \and \\ \meaningof{E_1 | E_2} = \{ P \in \pi | P \equiv P_{1} | P_{2}, P_{1} \in \meaningof{E_{1}}, P_{2} \in \meaningof{E_2}\} }
\end{mathpar}

\begin{mathpar}
 \inferrule* [lab=behavior] {} {\meaningof{\langle a?b \rangle E} = \{ P \in \pi | P \equiv Q | u?(y)P', \\ \and \\\\ \and \\ \;\;\; u \in \meaningof{a}, \forall z.P'\{z/y\} \in \meaningof{E\{z/b\}}\}, \and \\ \meaningof{a!E} = \{ P \in \pi | P \equiv Q | x!\langle P' \rangle, x \in \meaningof{a} P' \in \meaningof{E}\} }
\end{mathpar}

\begin{mathpar}
 \inferrule* [lab=nominal] {} {\meaningof{\quotep{E}} = \{ \quotep{P} \in \quotep{\pi} | P \in \meaningof{E} \}, \and \meaningof{\quotep{P}} = \{ \quotep{Q} \in \quotep{\pi} | P \equiv Q \} \and \\ \meaningof{@\quotep{E}} = \{ P \in \pi | P \equiv @x, x \in \meaningof{E} \}}
\end{mathpar}

\begin{eqnarray*}
  \\
  \meaningof{-} : TS \to ST
\end{eqnarray*}

\begin{eqnarray*}
  \\
  L : TS \to ST
\end{eqnarray*}

\begin{eqnarray*}
  \\
  P \models E \iff P \in \meaningof{E}
\end{eqnarray*}

\begin{eqnarray*}
  P \approx_{L} Q \iff \forall E \in L. P \models E \iff Q \models E
\end{eqnarray*}

\begin{eqnarray*}
  P \approx_{K} Q
\end{eqnarray*}

\begin{eqnarray*}
  P \approx Q
\end{eqnarray*}

$\approx_{K} = \approx = \approx_{L}$

\subsubsection{Contextual duality}

Note that contexts extend the quotation operation to a family of
operations from processes to names. Given a context, $M$, we can
define a \emph{nominal context}, $\quotep{M}$ by $\quotep{M}[P] :=
\quotep{M[P]}$. To foreshadow what is to come we observe that these
operations enjoy a duality with processes very much like the duality
between vectors and maps from vectors to scalars.

Further, because the calculus is essentially higher-order, we have a
correspondence between contexts and processes. More specifically,
given a name $x$ and a context $M$ we can construct $M^{*}_{x}$ such
that 

\begin{mathpar}
  M^{*}_{x} | \lift{x}{P} \red M[P]
\end{mathpar}

namely,

\begin{mathpar}
  M^{*}_{x} := x?(u).M[\dropn{u}]
\end{mathpar}

The dependence of $M^{*}_{x}$ on a name makes it an abstraction, 

\begin{mathpar}
  M^{*} := (x)x?(u).M[\dropn{u}]
\end{mathpar}

\subsection{Additional notation}

It will sometimes be convenient to denote the process a name
quotes. We already have the notation $x = \quotep{P}$, but it will be
convenient to introduce an alternate notation, $\procn{x}$, when we
want to emphasize the connection to the use of the name. Note that, by
virtue of name equivalence, $\quotep{\procn{x}} \nameeq x$; so, the
notation is consistent with previous definitions.

Further, because names have structure it is possible to effect
substitutions on the basis of that structure. This means we need to
upgrade our notation for substitutions, which we accomplish by
adapting comprehension notation. Thus,

\begin{mathpar}
  P\{ y / x : x \in S \}
\end{mathpar}

is interpreted to mean the process derived from P by replacing (in a
capture-avoiding manner) each occurrence of $x$ in $S$ by $y$. For example,

\begin{mathpar}
  P\{ \quotep{\procn{x}|\procn{x}} / x : x \in \freenames{P} \}
\end{mathpar}

will replace each (occurrence) of a free name $x$ in $P$ by
$\quotep{\procn{x}|\procn{x}}$.

Also, we will avail ourselves of the notation $x^{L}$ and $x^{R}$ to
denote injections of a name into disjoint copies of the name
space. There are numerous ways to accomplish this. One example can be
found in \cite{MeredithR05}. This notation overloads to vectors of
names: $\vec{x}^{\pi} := (x_{i}^{\pi} \; : \; 0 \leq i < |\vec{x}| )$ where $\pi \in \{L,R\}$.

We also use $P^{\Box} := P|\Box$.

In \cite{MeredithR05} an interpretation of the new operator is
given. It turns out that there are several possible interpretations
all enjoying the requisite algebraic properties of the operator (see
\cite{milner91polyadicpi}). We will therefore make liberal use of
$(\nu\; \vec{x})P$.

% subsection the_syntax_and_semantics_of_the_notation_system (end)   

\section{Interpretation of QM}
\subsection{Supporting definitions}
\subsubsection{Multiplication}
\begin{mathpar}
  \quotep{Q} \cdot \quotep{R} := \quotep{Q|R}
  \and \\
  \quotep{Q} \cdot P := P\{ \quotep{Q|R} / \quotep{R} : \quotep{R} \in \freenames{P} \}
\end{mathpar}

\paragraph{Discussion}
The first line needs little explanation. The second line says that
each free name of the process is replaced with the multiplication of
that name by the scalar. Multiplication of a scalar (name) by a state
(process) results in a process all the names of which have been `moved
over' by parallel composition with the process the scalar
quotes. There is a subtlety that the bound names have to be
manipulated so that multiplied names aren't accidentally
captured. There are many ways to achieve this.

\begin{remark}\label{rem:multiplication_identities}
  The reader is invited to verify that for all $x,y,z \in \QProc$ and $P \in \Proc$
  \begin{mathpar}
    x \cdot \quotep{0} \equiv x 
    \and
    x \cdot y \equiv y \cdot x
    \and
    x \cdot (y \cdot z) \equiv (x \cdot y) \cdot z
    \and \\
    \quotep{0} \cdot P \equiv P
    \and \\
    x \cdot (y \cdot P) \equiv (x \cdot y) \cdot P
    \and \\
    x \cdot (P|Q) \equiv (x \cdot P) | (x \cdot Q)
    \and \\    
  \end{mathpar}
\end{remark}

\subsubsection{Tensor product}

We define a tensor product on processes by structural induction.

\paragraph{Tensor of sums} First note that all summations, including
$\pzero$ and sequence, can be written $\Sigma_{i} x_{i}.A_{i} +
\Sigma_{j} x_{j}.C_{j}$, where we have grouped input-guarded processes
together and output-guarded processes together.

Thus, we can define the tensor product of two summations, $N_{1}\otimes N_{2}$, where

\begin{mathpar}
  N_{1} := \Sigma_{i} x_{i}.A_{i} + \Sigma_{j} x_{j}.C_{j}
  \and
  N_{2} := \Sigma_{i'} y_{i'}.B_{i'} + \Sigma_{j'} y_{j'}.D_{j'} 
\end{mathpar}

as follows.

\begin{mathpar}
  \Sigma_{i} x_{i}.A_{i} + \Sigma_{j} x_{j}.C_{j} \otimes \Sigma_{i'}
  y_{i'}.B_{i'} + \Sigma_{j'} y_{j'}.D_{j'} 
  \and \\
  := \; \Sigma_{i} \Sigma_{i'} \quotep{\stackrel{\vee}{x_{i}}| \stackrel{\vee}{y_{i'}}}.(A_{i}\otimes B_{i'}) \; | \; \Sigma_{i'} \Sigma_{i} \quotep{\stackrel{\vee}{y_{i'}}|\stackrel{\vee}{x_{i}}}.(B_{i'}\otimes A_{i})
  \and
  \;\; | \;\; \Sigma_{j} \Sigma_{j'} \quotep{\stackrel{\vee}{x_{j}}|\stackrel{\vee}{y_{j'}}}.(A_{j}\otimes B_{j'}) \; | \; \Sigma_{j'} \Sigma_{j} \quotep{\stackrel{\vee}{y_{j'}}|\stackrel{\vee}{x_{j}}}.(B_{j'}\otimes A_{j})
\end{mathpar}

\begin{remark}
  Do we need to $x^{L}$ and $y^{R}$ for this construction as well?
\end{remark}

\paragraph{Tensor of parallel compositions} Next, we distribute tensor
over par.

\begin{mathpar}
  P_{1}|P_{2} \otimes Q_{1}|Q_{2} := (P_{1} \otimes Q_{1}) | (P_{1}
  \otimes Q_{2}) | (P_{2} \otimes Q_{1}) | (P_{2} \otimes Q_{2})
\end{mathpar}

\paragraph{Tensor with dropped names} We treat tensor of a
process with a dropped name as parallel composition.

\begin{mathpar}
  P \otimes \dropn{x} := P | \dropn{x}
\end{mathpar}

\paragraph{Tensor of agents}

Finally, we need to define tensor on agents. Note that the definition
of tensor on normal products only tensors inputs with inputs and
outputs with outputs. Thus, we only have to define the operation on
``homogeneous'' pairings.

\begin{mathpar}
  (\vec{x})P \otimes (\vec{y})Q
  \and \\
  := (x_{0}^{L}|y_{0}^{R},\ldots,x_{0}^{L}|y_{n}^{R},\ldots,x_{m}^{L}|y_{0}^{R},\ldots,x_{m}^{L}|y_{n}^R)(P\{ \vec{x}^{L}/\vec{x}\} \otimes Q \{ \vec{y}^{R}/\vec{y}\})
  \and \\
  \clift{\vec{P}} \otimes \clift{\vec{Q}}
  \and \\
  := \clift{P_{0}\otimes Q_{0},\ldots,P_{0}\otimes Q_{n},\ldots,P_{m}\otimes Q_{0},\ldots,P_{m}\otimes Q_{n}}
\end{mathpar}

\begin{remark}
  Observe that arities of tensored abstractions matches arities of
  tensored concretions if the original arities matched. Note also that
  the length of the arities corresponds to the increase in dimension
  we see in ordinary vector space tensor product.
\end{remark}

\begin{remark}
  Operationally, this definition distributes the tensor down to
  components ``linked'' by summation. Tensor over summation is
  intriguing in that it mixes names. Moreover, as a consequence of the
  way it mixes names we have the identities for all $x \in \QProc$ and
  $P,Q \in \Proc$

  \begin{mathpar}
    (x \cdot P) \otimes Q \equiv x \cdot (P \otimes Q) \equiv P \otimes (x \cdot Q)
    \and
    P \otimes \pzero \equiv P
  \end{mathpar}

  that the reader is invited to verify.
\end{remark}

\subsubsection{Annihilation}
\begin{mathpar}
  P^{\perp} := \{ Q | \forall R. P|Q \red^{*} R \Rightarrow R \red^{*} \pzero \}
  \and \\
  P^{\underline{\perp}} := \Sigma_{Q \in P^{\perp}} \quotep{Q}?(y).(\dropn{y}|Q) | \Sigma_{Q \in P^{\perp}} \quotep{Q}\clift{\Box}
\end{mathpar}

\paragraph{Discussion} The reader will note that $P^{\perp}$ is a
\emph{set} of processes, while $P^{\underline{\perp}}$ is a
\emph{context}. We call the set $P^{\perp}$ the \emph{annihilators} of
$P$. The parallel composition of a process in the annihilators of $P$
with $P$ will result in a process, the state space of which has all
paths eventually leading to $\pzero$. Execution may endure loops; but
under reasonable conditions of fairness (naturally guaranteed under
most notions of bisimulation) such a composite process cannot get
stuck in such a loop and will, eventually pop out and terminate.

The context $P^{\underline{\perp}}$ is ready and willing to ``take the
$P$ out of'' the process to which it is applied. It will effectively
transmit the code of the process to which it is applied to one of the
annihilators and run the process against it.

\subsubsection{Evaluation}
We fix $M$ a domain of fully abstract interpretation with an equality
coincident with bisimulation. We take $\meaningof{\cdot} : \Proc \to
M$ to be the map interpreting processes and $\nmeaningof{\cdot} : \M
\to Proc$ to be the map running the other way. Then we define

\begin{mathpar}
  \int P := \nmeaningof{\meaningof{P}}
\end{mathpar}

\paragraph{Discussion}
There are many fully abstract interpretations of Milner's
$\pi$-calculus. Any of them can be used as a basis for interpreting
the reflective calculus here. Equipped with such a domain it is
largely a matter of grinding through to check that the Yoneda
construction for the normalization-by-evaluation program can be
extended to this setting.

\begin{remark}
  The reader is invited to verify that $\int (P^{\underline{\perp}}[P]) = 0$.
\end{remark}

\subsection{Quantum mechanics}

Table \ref{tbl:core_qm_op_defns} gives the core operational definitions

\begin{table}[htp]\label{tbl:core_qm_op_defns}
  \center{
    \fbox{
      \begin{tabular}{c|c}
        quantum mechanics & process calculus \\
        \hline
        scalar & $x := \quotep{P}$ \\
        state vector & $\state{P} := P$ \\
        dual & $\state{P}^{*} := \event{P^{\underline{\perp}}} := \quotep{P^{\underline{\perp}}}[-]$ \\
        matrix & $ \Sigma_{\alpha} \state{P_{\alpha}}x_{\alpha}\event{Q_{\alpha}}$ \\
        vector addition & $\state{P} + \state{Q} := \state{P | Q}$ \\
        tensor product & $\state{P} \otimes \state{Q} := \state{P \otimes Q}$ \\
        inner product & $\innerprod{P}{Q} := \quotep{\int P^{\underline{\perp}}[Q]}$ \\
      \end{tabular}
    }
  }
  \caption{QM - operational definitions}
\end{table}

where

\begin{mathpar}
  \prmatrix{P}{Q} := \fprmatrix{P}{\quotep{\pzero}}{Q}
  \and
  \fprmatrix{P}{x}{Q} := (\state{P},x,\event{Q})
  \and
  (\fprmatrix{P}{x}{Q})(\state{R}) := x \cdot \innerprod{Q}{R} \cdot \state{P}
  \and
  (\fprmatrix{P}{x}{Q})(\event{R}) := x \cdot \innerprod{R}{P} \cdot \event{Q}
\end{mathpar}

\paragraph{Discussion}
As promised: vectors (aka states) are represented as processes; duals
as contextual duals; inner product definition should be compared with
standard inner product definition for ....

\begin{remark}
  Assuming $\int (P^{\underline{\perp}}[P]) = 0$, the reader is
  invited to verify that $(\fprmatrix{P}{x}{P})(\state{P}) = x \cdot \state{P}$.
\end{remark}

\begin{remark}
  The reader is invited to verify that $\innerprod{P}{Q}$ could
  equally well have been written $\quotep{\int \stackrel{\vee}{x}}$
  where $x = \event{P^{\underline{\perp}}}(Q)$.

  One of the motivations for this remark is that there is another way
  to factor these operations. We could package up evaluation in the dual:

  \begin{mathpar}
    \state{P}^{*} := \event{\int P^{\underline{\perp}}} := \quotep{\int P^{\underline{\perp}}}[-]
  \end{mathpar}

  and then have inner product defined by
  
  \begin{mathpar}
    \innerprod{P}{Q} := \event{P}(Q)
  \end{mathpar}

  Hopefully, experience with the calculations will provide guidance on
  the best factoring.
\end{remark}

\begin{remark}
  Assuming $\int (P^{\underline{\perp}}[P]) = 0$, the reader is
  invited to verify that $\forall P,Q. (\prmatrix{0}{Q})(\state{0}) =
  \state{0}$ and dually $(\prmatrix{P}{0})(\event{0}) = \event{0}$.
\end{remark}

\begin{remark}
  i'm a little worried that i don't (yet) have proper support for
  complex conjugacy. But, the observation above may give us a
  clue. According to Abramsky, it must be the case that the scalars
  are iso to the homset of the identity for the tensor -- which the
  observation above characterizes. 

  For now, we will simply bookmark the notion with $\overline{x}$.
\end{remark}

\subsubsection{Adjointness}

We need to give a definition of $(\cdot)^{\dagger}$ for matrices. The
obvious candidate definition is
\begin{mathpar}
(\Sigma_{\alpha}\fprmatrix{P_{\alpha}}{x_{\alpha}}{Q_{\alpha}})^{\dagger}
= \Sigma_{\alpha}\fprmatrix{(Q_{\alpha}^{\underline{\perp}})^{*}}{\overline{x}_{\alpha}}{P_{\alpha}^{\underline{\perp}}} 
\end{mathpar}

But, $(Q_{\alpha}^{\underline{\perp}})^{*}$ requires a name along
which to communicate the process to achieve the context application.

\subsubsection{Basis for a basis}
If processes label states and ``addition'' of states (a.k.a. vector
addition) is interpreted as parallel composition, what corresponds to
notions of linear independence and basis? Here, we recall that Yoshida
has developed a set of \emph{combinators} for an asynchronous verison
of Milner's $\pi$-calculus. These are a finite set of processes such
any process can be expressed as parallel composition of these
combinators together with liberal uses of the new operator and
replication. We can simply give a translation of these into the
present calculus and have reasonable expectation that the property
carries over. That is, that the resultant set allows to express all
processes via parallel composition. Note, however, that there is no
new operator or replication in this calculus. As a result, we expect
that the corresponding set is actually infinite. That is, we expect
that the space is actually infinite dimensional.

\begin{remark}
  The attentive reader may be a bit concerned. Certainly, the
  collection $S$, $K$ and $I$ is a finite set of
  combinators. Shouldn't we expect to see a finite set of combinators
  for an effectively equivalent system? i am very sympathetic to this
  critique and feel it warrants full attention. On the other hand, i
  also have in mind the following analogy. The natural numbers, as a
  monoid under addition, has exactly $1$ generator, while the natural
  numbers, as a monoid under multiplication, has countably many
  generators (the primes). We observe that the application of the
  lambda calculus is much less resource sensitive than the parallel
  composition of the $\pi$-calculus. Could it be the case that we have
  an analogy of the form
  
  \begin{mathpar}
    m + n : MN :: m*n : M|N
  \end{mathpar}

  giving a similar blow up in the set of ``primes''?  This is such a
  wonderful thought that, even if it's not true, i think it's worth
  writing down.
\end{remark}
 

\documentclass[12pt]{llncs}
%\documentclass{jktr}

\usepackage[pdftex]{hyperref}                   
\usepackage {listings}
\usepackage {mathpartir}
\usepackage{bcprules}
%\usepackage{listings}
                       
\usepackage{graphicx} 
%\usepackage[margins=2.5cm,nohead,nofoot]{geometry}
%\usepackage{geometry}
\usepackage{amsfonts}
\usepackage{amstext}
\usepackage{latexsym}
\usepackage{amssymb}
\usepackage{color}


%\include{myPreamble}
\include{qm2pi.local} 

%\ifpdf
%\usepackage[pdftex]{graphicx}
%\else
%\usepackage{graphicx}
%\fi

 % \ifpdf
%  \usepackage{pdfsync}
%  \if


%\title{Brief Article}
%\author{David F. Snyder}
%\author{L.G. Meredith}

%\address{Dept. of Math., Texas State University--San Marcos, San Marcos, TX 78666}
       
\pagestyle{empty}


\begin{document}

\lstset{language=[Objective]Caml,frame=shadowbox}

\input{qm2pi.front}

% section front matter (end)

\input{qm2pi.intro} 
 
% section introduction (end)

% \input{qm2pi.knotations} 

% section notation (end)

\input{qm2pi.process.calculi} 

% section concurrent_process_calculi_and_spatial_logics_ (end)
    
%\input{qm2pi.knots2pi} 

%\input{qm2pi.trefoil} 

%\input{qm2pi.mainthm} 

% subsection basic_interpretation (end)

%\input{qm2pi.rho.presentation} 
\subsection{The syntax and semantics of the notation system}\label{sub:the_syntax_and_semantics_of_the_notation_system} % (fold)

We now summarize a technical presentation of the calculus that
embodies our theory of dynamics. The typical presentation of such a
calculus follows the style of giving generators and relations on
them. The grammar, below, describing term constructors, freely
generates the set of processes, $\Proc$. This set is then quotiented
by a relation known as structural congruence and it is over this set
that the notion of dynamics is expressed. This presentation is
essentially that of \cite{MeredithR05} with the addition of
polyadicity and summation. For readability we have relegated some of
the technical subtleties to an appendix.

\subsubsection{Process grammar}\label{subsub:process_grammar}

\begin{mathpar}
  \inferrule* [lab=synchronization] {} {{M} \bc \pzero \;|\; x?F \;|\; x!C }
  \and
  \inferrule* [lab=abstraction] {} {{F} \bc (x)P}
  \and
  \inferrule* [lab=concretion] {} {{C} \bc \langle Q \rangle}
  \and
  \inferrule* [lab=process] {} {{P,Q} \bc M \;| \;P|Q \;|\; @{x}}
  \and
  \inferrule* [lab=name] {} {{x} \bc \quotep{P}}
\end{mathpar} 

Note that $\vec{x}$ (resp. $\vec{P}$) denotes a vector of names
(resp. processes) of length $|\vec{x}|$ (resp. $|\vec{P}|$). We adopt
the following useful abbreviations.

\begin{mathpar}
   x?(\vec{y}).P := x.(\vec{y})P \and  x\clift{\vec{P}} := x.\clift{\vec{P}}
   \and x!(y) := \lift{x}{\dropn{y}}
   \and \Pi_{i=0}^{n-1}P_i := P_0 | \ldots | P_{n-1}
\end{mathpar}

\subsubsection{Structural congruence}

\paragraph{Free and bound names and alpha-equivalence.} At the
core of structural equivalence is alpha-equivalence which identifies
process that are the same up to a change of variable. Formally, we
recognize the distinction between free and bound names. The free names
of a process, $\freenames{P}$, may be calculated recursively as
follows:

\begin{mathpar}
\freenames{\pzero} := \emptyset
  \and \\
  \freenames{x?(y).P} := \{ x \} \cup (\freenames{P} \setminus \{ y \})
  \and 
  \freenames{x!\langle P \rangle} := \{ x \} \cup \{ P \} 
  \and \\
  \freenames{P|Q} := \freenames{P} \cup \freenames{Q}
  \and \\
  \freenames{@{x}} := \{ x \}
\end{mathpar}

$\pi$
$\quotep{\pi}$

$\freenames{-} : \pi \to \mathcal{P}(\quotep{\pi})$

\begin{eqnarray*}
  \freenames{\pzero} & := & \emptyset \\
  \freenames{x?(y).P} & := & \{ x \} \cup (\freenames{P} \setminus \{ y \}) \\
  \freenames{x!\langle P \rangle} & := & \{ x \} \cup \{ P \} \\
  \freenames{P|Q} & := & \freenames{P} \cup \freenames{Q} \\
  \freenames{\dropn{x}} & := & \{ x \}
\end{eqnarray*}

The bound names of a process, $\boundnames{P}$, are those names occurring in $P$
that are not free. For example, in $x?(y).0$, the name $x$ is free, while $y$ is bound.

\begin{mathpar}
  \inferrule* [lab=monoidal-laws] {} { P|Q \equiv Q|P \and P|0 \equiv P \and P|(Q|R) \equiv (P|Q)|R }
\end{mathpar}

\begin{mathpar}
  \inferrule* [lab=alpha-equivalence] {} { (x)P \equiv (y)P\{y/x\} \and y \not\in \freenames{P} }
\end{mathpar}

\begin{definition}
Then two processes, $P,Q$, are alpha-equivalent if $P = Q\{\vec{y}/\vec{x}\}$ for
some $\vec{x} \in \boundnames{Q},\vec{y} \in \boundnames{P}$, where $Q\{\vec{y}/\vec{x}\}$
denotes the capture-avoiding substitution of $\vec{y}$ for $\vec{x}$ in $Q$.
\end{definition}

\begin{definition}
  The {\em structural congruence} \cite{SangiorgiWalker} , $\equiv$,
  between processes is the least congruence containing
  alpha-equivalence, satisfying the abelian monoid laws
  (associativity, commutativity and $\pzero$ as identity) for parallel
  composition $|$ and for summation $+$.
\end{definition}

\subsection{Name equivalence}

We take name equivalence, written $\nameeq$, to be the smallest
equivalence relation generated by the following rules.

\begin{mathpar}
\inferrule*[lab=Quote-drop]
{ }
{ \quotep{@{x}} \nameeq x }

\inferrule*[lab=Struct-equiv]
{ P \scong Q }
{ \quotep{P} \nameeq \quotep{Q} }
\end{mathpar}

The astute reader will have noticed that the mutual recursion of names
and processes imposes a mutual recursion on alpha-equivalence and
structural equivalence via name-equivalence. Fortunately, all of this
works out pleasantly and we may calculate in the natural way, free of
concern. The reader interested in the details is referred to the
appendix \ref{appendix:rho_details}.

\subsection{Substitution}

We use $\Proc$ for the set of processes, $\QProc$ for the set of
names, and $\id{\{}\vec{y} / \vec{x} \id{\}}$ to denote partial maps,
$s : \QProc \rightarrow \QProc$. A map, $s$ lifts, uniquely, to a map
on process terms, $\widehat{s} : \Proc \rightarrow \Proc$ by the
following equations.

\begin{mathpar}
  (0) \psubstp{Q}{P} := 0 \\
  (R \juxtap S) \psubstp{Q}{P}
  :=    
  (R)\psubstp{Q}{P} \juxtap (S) \psubstp{Q}{P} \\
  (x?(y).R) \psubstp{Q}{P}    
  :=    
  (x)\substp{Q}{P} (z)\concat( (R \psubstn{z}{y}) \psubstp{Q}{P} ) \\
  (\lift{x}{R}) \psubstp{Q}{P}  
  :=
  \lift{(x)\substp{Q}{P}}{ R \psubstp{Q}{P} } \\
%   (\dropn{x})  \psubstp{Q}{P}       
%   := 
%   \left\{ 
%     \begin{array}{ccc} 
%       \dropn{\quotep{Q}} & & x \nameeq \quotep{P} \\
%       \dropn{x} & & otherwise \\
%     \end{array}
%   \right. 
  (\dropn{x})  \psubstp{Q}{P}       
  := 
  \left\{ 
    \begin{array}{ccc} 
      Q & & x \nameeq \quotep{P} \\
      \dropn{x} & & otherwise \\
    \end{array}
  \right.
\end{mathpar}
 

where

\begin{eqnarray}
  (x)\id{\{} \lpquote Q \rpquote / \lpquote P \rpquote \id{\}}            = 
  \left\{ 
    \begin{array}{ccc}
      \lpquote Q \rpquote & & x \nameeq \lpquote P \rpquote \\
      x & & otherwise \\
    \end{array}
  \right. \nonumber
\end{eqnarray}

and $z$ is chosen distinct from $\quotep{P}$, $\quotep{Q}$, the free
names in $Q$, and all the names in $R$. Our $\alpha$-equivalence will
be built in the standard way from this substitution.

\begin{remark}\label{rem:no_self_referential_names}
  One consequence of these definitions is that $\forall P. \quotep{P}
  \not\in \freenames{P}$.
\end{remark}

\subsection{ Dynamic quote: an example }

Anticipating something of what's to come, consider applying the
substitution, $\widehat{\id{\{}u / z \id{\}}}$, to the following pair
of processes, $\lift{w}{y!(z)}$ and $w[ \lpquote y!(z) \rpquote ]$.

\begin{eqnarray}
	\lift{w}{y!(z)}\widehat{\id{\{}u / z \id{\}}}
		& = &
		\lift{w}{y!(u)} \nonumber\\
	w[ \lpquote y!(z) \rpquote ] \widehat{ \id{\{}u / z \id{\}} }
		& = &
		w[ \lpquote y!(z) \rpquote ] \nonumber
\end{eqnarray}

Because the body of the process between quotes is impervious to
substitution, we get radically different answers. In fact, by
examining the first process in an input context,
e.g. $x?(z).\lift{w}{y!(z)}$, we see that the process under the lift
operator may be shaped by prefixed inputs binding a name inside it. In
this sense, the lift operator will be seen as a way to dynamically
construct processes before reifying them as names.

Finally equipped with these standard features we can present the
dynamics of the calculus.

\subsubsection{Operational semantics} 

Finally, we introduce the computational dynamics. What marks these
algebras as distinct from other more traditionally studied algebraic
structures, e.g. vector spaces or polynomial rings, is the manner in
which dynamics is captured. In traditional structures, dynamics is typically
expressed through morphisms between such structures, as in linear maps
between vector spaces or morphisms between rings. In algebras
associated with the semantics of computation, the dynamics is
expressed as part of the algebraic structure itself, through a
reduction reduction relation typically denoted by $\red$. Below, we
give a recursive presentation of this relation for the calculus used
in the encoding.

$\red \subseteq \pi \times \pi$
$\red : \pi \to \mathcal{P}(\pi)$

\begin{mathpar}
  \inferrule* [lab=Comm] { \textsf{match}( x_{src}, x_{trgt} ) } { x_{trgt}?(y)P \; | \; x_{src}!\langle {Q} \rangle \red P\{\quotep{Q}/y}\} }
  \and \\
  \inferrule* [lab=Par] {{P} \red {P}'} {{{P} | {Q}} \red {{P}' | {Q}}}
  \and
  \inferrule* [lab=Equiv]{{{P} \scong {P}'} \andalso {{P}' \red {Q}'} \andalso {{Q}' \scong {Q}}}{{P} \red {Q}}
\end{mathpar}

\begin{eqnarray*}
  match_{\equiv} (\quotep{P},\quotep{Q}) & := & P \equiv Q \\
  match_{\dagger}(\quotep{P},\quotep{Q}) & := & \forall R. P|Q \red^{*} R => R \red^{*} 0 \\
  match_{K}(\quotep{P},\quotep{Q}) & := & K \mbox{ for some context } K
\end{eqnarray*}

$u?(x)P | u!\langle Q \rangle \red P\{\quotep{Q}/x\}$

%We write $\wred$ for $\red^*$, and $P\red$ if $\exists Q $ such that $ P \red Q$.
We write $P\red$ if $\exists Q $ such that $ P \red Q$ and $P\not\red$, otherwise.

\section{Replication}

As mentioned before, it is known that replication (and hence
recursion) can be implemented in a higher-order process algebra
\cite{SangiorgiWalker}. As our first example of calculation with the
machinery thus far presented we give the construction explicitly in
the {\rhoc}.

\begin{eqnarray}
	D_{x} & := & \prefix{x}{y}{(\binpar{\outputp{x}{y}}{@{y}})} \nonumber\\
	\bangp_{x}{P} & := & \binpar{{x}!\langle{\binpar{D_{x}}{P}}\rangle}{D_{x}} \nonumber
\end{eqnarray}

\begin{eqnarray}
	\bangp_{x}{P} & & \nonumber\\
	=
	& {x}!\langle{(\prefix{x}{y}{(\outputp{x}{y} | @{y})) | P}}\rangle 
	      | \prefix{x}{y}{(\outputp{x}{y} | @{y})} & \nonumber\\
	\red
	& (\outputp{x}{y} | @{y})\substn{\quotep{(\prefix{x}{y}{(@{y} | \outputp{x}{y})) | P}}}{y} & \nonumber\\
	=
	& \outputp{x}{\quotep{(\prefix{x}{y}{(\outputp{x}{y} | @{y})) | P}}}
	  | {(\prefix{x}{y}{(\outputp{x}{y} | @{y})) | P}} & \nonumber\\
	\red
	& \ldots & \nonumber\\
	\red^*
	& P | P | \ldots & \nonumber
\end{eqnarray}

Of course, this encoding, as an implementation, runs away, unfolding
$\bangp{P}$ eagerly. A lazier and more implementable replication
operator, restricted to input-guarded processes, may be obtained as follows.

\begin{eqnarray}
\bangp{\prefix{u}{v}{P}} 
	:= 
	\binpar{\lift{x}{\prefix{u}{v}{(\binpar{D(x)}{P})}}}{D(x)} \nonumber
\end{eqnarray}

\begin{remark}
  Note that the lazier definition still does not deal with summation
  or mixed summation (i.e. sums over input and output). The reader is
  invited to construct definitions of replication that deal with these
  features. 

  Further, the definitions are parameterized in a name, $x$. Can you,
  gentle reader, make a definition that eliminates this parameter and
  guarantees no accidental interaction between the replication
  machinery and the process being replicated -- i.e. no accidental
  sharing of names used by the process to get its work done and the
  name(s) used by the replication to effect copying. This latter
  revision of the definition of replication is crucial to obtaining
  the expected identity $!!P \sim !P$.
\end{remark}

\begin{remark}\label{rem:paradoxical_combinator}
  The reader familiar with the lambda calculus will have noticed the
  similarity between $D$ and the paradoxical combinator.

  [Ed. note: the existence of this seems to suggest we have to be more
  restrictive on the set of processes and names we admit if we are to
  support no-cloning.]
\end{remark}

\subsubsection{Bisimulation}

The computational dynamics gives rise to another kind of equivalence,
the equivalence of computational behavior. As previously mentioned
this is typically captured \emph{via} some form of bisimulation.

% The notion we use in this paper is weak barbed bisimulation
% \cite{milner91polyadicpi}.

The notion we use in this paper is derived from weak barbed
bisimulation \cite{milner91polyadicpi}. 

\begin{definition}
An \emph{observation relation}, $\downarrow_{\mathcal N}$, over a set
of names, $\mathcal N$, is the smallest relation satisfying the rules
below.

\infrule[Out-barb]{y \in {\mathcal N}, \; x \nameeq y}
		  {\outputp{x}{v} \downarrow_{\mathcal N} x}
\infrule[Par-barb]{\mbox{$P\downarrow_{\mathcal N} x$ or $Q\downarrow_{\mathcal N} x$}}
		  {\binpar{P}{Q} \downarrow_{\mathcal N} x}

We write $P \Downarrow_{\mathcal N} x$ if there is $Q$ such that 
$P \wred Q$ and $Q \downarrow_{\mathcal N} x$.
\end{definition}

\begin{definition}
%\label{def.bbisim}
An  ${\mathcal N}$-\emph{barbed bisimulation} over a set of names, ${\mathcal N}$, is a symmetric binary relation 
${\mathcal S}_{\mathcal N}$ between agents such that $P\rel{S}_{\mathcal N}Q$ implies:
\begin{enumerate}
\item If $P \red P'$ then $Q \wred Q'$ and $P'\rel{S}_{\mathcal N} Q'$.
\item If $P\downarrow_{\mathcal N} x$, then $Q\Downarrow_{\mathcal N} x$.
\end{enumerate}
$P$ is ${\mathcal N}$-barbed bisimilar to $Q$, written
$P \wbbisim_{\mathcal N} Q$, if $P \rel{S}_{\mathcal N} Q$ for some ${\mathcal N}$-barbed bisimulation ${\mathcal S}_{\mathcal N}$.
\end{definition}

$\mathcal{R} \subseteq \pi \times \pi$

$P \mathcal{R} Q => \forall P'. P \red P' \Rightarrow \exists Q'. Q \red Q', P' \mathcal{R} Q'$

$P \vdash x \Rightarrow Q \vdash x$

\begin{mathpar}
  \inferrule*[lab=Out-barb]{x \nameeq y}{{y}!\langle{Q}\rangle \vdash x}
  \and
  \inferrule*[lab=Par-barb]{\mbox{$P\vdash x$ or $Q\vdash x$}}{\binpar{P}{Q} \vdash x}
\end{mathpar}

\subsubsection{Contexts}

One of the principle advantages of computational calculi like the
$\pi$-calculus is a well-defined notion of context,
contextual-equivalence and a correlation between
contextual-equivalence and notions of bisimulation. The notion of
context allows the decomposition of a process into (sub-)process and
its syntactic environment, its context. Thus, a context may be
thought of as a process with a ``hole'' (written $\Box$) in it. The
application of a context $M$ to a process $P$, written $M[P]$, is
tantamount to filling the hole in $M$ with $P$. In this paper we do
not need the full weight of this theory, but do make use of the notion
of context in the proof the main theorem. 

\begin{mathpar}
  \inferrule* [lab=summation] {} {{M_{M},M_{N}} \bc \Box \;|\; x.M_{A} \;|\; M_{M}+M_{N}}
  \and
  \inferrule* [lab=agent] {} {{M_{A}} \bc (\vec{x})M_{P} \;| \; \clift{P_0,\ldots,M_{P},\ldots,P_N}}
  \and \\
  \inferrule* [lab=process] {} {{M_{P}} \bc M_{N} \;| \;P|M_{P} }
\end{mathpar} 

\begin{mathpar}
  \inferrule* [lab=sychronization] {} {M_{N} \bc \Box \;|\; x?M_{F} \;|\; x!M_{C}}
  \and
  \inferrule* [lab=abstraction] {} {{M_{F}} \bc (x)M_{P} }
  \and
  \inferrule* [lab=concretion] {} {{M_{C}} \bc \langle M_{P} \rangle }
  \and \\
  \inferrule* [lab=process] {} {{M_{P}} \bc M_{N} \;| \;P|M_{P} }
\end{mathpar}

\begin{definition}[contextual application] Given a context $M$, and
  process $P$, we define the \emph{contextual application}, $M[P] :=
  M\{P/\Box\}$. That is, the contextual application of M to P is the
  substitution of $P$ for $\Box$ in $M$.
\end{definition}

$\meaningof{-} : L \to \mathcal{P}(\pi)$

\begin{mathpar}
  \inferrule* [lab=collection] {} {\meaningof{true} = \pi, \and \meaningof{~E} = \pi \setminus \meaningof{E}, \and \meaningof{E_{1} \& E_{2}} = \meaningof{E_{1}} \cap \meaningof{E_{2}}}
\end{mathpar}

\begin{mathpar}
  \inferrule* [lab=structure] {} {\meaningof{0} = \{ P \in \pi | P \equiv 0 \}, \and \\ \meaningof{E_1 | E_2} = \{ P \in \pi | P \equiv P_{1} | P_{2}, P_{1} \in \meaningof{E_{1}}, P_{2} \in \meaningof{E_2}\} }
\end{mathpar}

\begin{mathpar}
 \inferrule* [lab=behavior] {} {\meaningof{\langle a?b \rangle E} = \{ P \in \pi | P \equiv Q | u?(y)P', \\ \and \\\\ \and \\ \;\;\; u \in \meaningof{a}, \forall z.P'\{z/y\} \in \meaningof{E\{z/b\}}\}, \and \\ \meaningof{a!E} = \{ P \in \pi | P \equiv Q | x!\langle P' \rangle, x \in \meaningof{a} P' \in \meaningof{E}\} }
\end{mathpar}

\begin{mathpar}
 \inferrule* [lab=nominal] {} {\meaningof{\quotep{E}} = \{ \quotep{P} \in \quotep{\pi} | P \in \meaningof{E} \}, \and \meaningof{\quotep{P}} = \{ \quotep{Q} \in \quotep{\pi} | P \equiv Q \} \and \\ \meaningof{@\quotep{E}} = \{ P \in \pi | P \equiv @x, x \in \meaningof{E} \}}
\end{mathpar}

\begin{eqnarray*}
  \\
  \meaningof{-} : TS \to ST
\end{eqnarray*}

\begin{eqnarray*}
  \\
  L : TS \to ST
\end{eqnarray*}

\begin{eqnarray*}
  \\
  P \models E \iff P \in \meaningof{E}
\end{eqnarray*}

\begin{eqnarray*}
  P \approx_{L} Q \iff \forall E \in L. P \models E \iff Q \models E
\end{eqnarray*}

\begin{eqnarray*}
  P \approx_{K} Q
\end{eqnarray*}

\begin{eqnarray*}
  P \approx Q
\end{eqnarray*}

$\approx_{K} = \approx = \approx_{L}$

\subsubsection{Contextual duality}

Note that contexts extend the quotation operation to a family of
operations from processes to names. Given a context, $M$, we can
define a \emph{nominal context}, $\quotep{M}$ by $\quotep{M}[P] :=
\quotep{M[P]}$. To foreshadow what is to come we observe that these
operations enjoy a duality with processes very much like the duality
between vectors and maps from vectors to scalars.

Further, because the calculus is essentially higher-order, we have a
correspondence between contexts and processes. More specifically,
given a name $x$ and a context $M$ we can construct $M^{*}_{x}$ such
that 

\begin{mathpar}
  M^{*}_{x} | \lift{x}{P} \red M[P]
\end{mathpar}

namely,

\begin{mathpar}
  M^{*}_{x} := x?(u).M[\dropn{u}]
\end{mathpar}

The dependence of $M^{*}_{x}$ on a name makes it an abstraction, 

\begin{mathpar}
  M^{*} := (x)x?(u).M[\dropn{u}]
\end{mathpar}

\subsection{Additional notation}

It will sometimes be convenient to denote the process a name
quotes. We already have the notation $x = \quotep{P}$, but it will be
convenient to introduce an alternate notation, $\procn{x}$, when we
want to emphasize the connection to the use of the name. Note that, by
virtue of name equivalence, $\quotep{\procn{x}} \nameeq x$; so, the
notation is consistent with previous definitions.

Further, because names have structure it is possible to effect
substitutions on the basis of that structure. This means we need to
upgrade our notation for substitutions, which we accomplish by
adapting comprehension notation. Thus,

\begin{mathpar}
  P\{ y / x : x \in S \}
\end{mathpar}

is interpreted to mean the process derived from P by replacing (in a
capture-avoiding manner) each occurrence of $x$ in $S$ by $y$. For example,

\begin{mathpar}
  P\{ \quotep{\procn{x}|\procn{x}} / x : x \in \freenames{P} \}
\end{mathpar}

will replace each (occurrence) of a free name $x$ in $P$ by
$\quotep{\procn{x}|\procn{x}}$.

Also, we will avail ourselves of the notation $x^{L}$ and $x^{R}$ to
denote injections of a name into disjoint copies of the name
space. There are numerous ways to accomplish this. One example can be
found in \cite{MeredithR05}. This notation overloads to vectors of
names: $\vec{x}^{\pi} := (x_{i}^{\pi} \; : \; 0 \leq i < |\vec{x}| )$ where $\pi \in \{L,R\}$.

We also use $P^{\Box} := P|\Box$.

In \cite{MeredithR05} an interpretation of the new operator is
given. It turns out that there are several possible interpretations
all enjoying the requisite algebraic properties of the operator (see
\cite{milner91polyadicpi}). We will therefore make liberal use of
$(\nu\; \vec{x})P$.

% subsection the_syntax_and_semantics_of_the_notation_system (end)   

\input{qm2pi.qmops} 

\input{qm2pi.sterngerlach} 

\input{qm2pi.metric} 

% section concurrent_process_calculi (end)

%\input{qm2pi.proofsketch}

% section proof sketch (end)

%\input{qm2pi.slviaknots} 

% section spatial logic via knots (end)

\input{qm2pi.conclusion}

% section conclusion (end)

%\input{qm2pi.dtcodes} 

% section wiring algorithm (end)

\input{qm2pi.ack} 

% section acknowledgments (end)

\newpage


\bibliographystyle{plain}   
\bibliography{../../biblios/main.bib}

\input{qm2pi.rhodetails}

\end{document}

 

\documentclass[12pt]{llncs}
%\documentclass{jktr}

\usepackage[pdftex]{hyperref}                   
\usepackage {listings}
\usepackage {mathpartir}
\usepackage{bcprules}
%\usepackage{listings}
                       
\usepackage{graphicx} 
%\usepackage[margins=2.5cm,nohead,nofoot]{geometry}
%\usepackage{geometry}
\usepackage{amsfonts}
\usepackage{amstext}
\usepackage{latexsym}
\usepackage{amssymb}
\usepackage{color}


%\include{myPreamble}
\include{qm2pi.local} 

%\ifpdf
%\usepackage[pdftex]{graphicx}
%\else
%\usepackage{graphicx}
%\fi

 % \ifpdf
%  \usepackage{pdfsync}
%  \if


%\title{Brief Article}
%\author{David F. Snyder}
%\author{L.G. Meredith}

%\address{Dept. of Math., Texas State University--San Marcos, San Marcos, TX 78666}
       
\pagestyle{empty}


\begin{document}

\lstset{language=[Objective]Caml,frame=shadowbox}

\input{qm2pi.front}

% section front matter (end)

\input{qm2pi.intro} 
 
% section introduction (end)

% \input{qm2pi.knotations} 

% section notation (end)

\input{qm2pi.process.calculi} 

% section concurrent_process_calculi_and_spatial_logics_ (end)
    
%\input{qm2pi.knots2pi} 

%\input{qm2pi.trefoil} 

%\input{qm2pi.mainthm} 

% subsection basic_interpretation (end)

%\input{qm2pi.rho.presentation} 
\subsection{The syntax and semantics of the notation system}\label{sub:the_syntax_and_semantics_of_the_notation_system} % (fold)

We now summarize a technical presentation of the calculus that
embodies our theory of dynamics. The typical presentation of such a
calculus follows the style of giving generators and relations on
them. The grammar, below, describing term constructors, freely
generates the set of processes, $\Proc$. This set is then quotiented
by a relation known as structural congruence and it is over this set
that the notion of dynamics is expressed. This presentation is
essentially that of \cite{MeredithR05} with the addition of
polyadicity and summation. For readability we have relegated some of
the technical subtleties to an appendix.

\subsubsection{Process grammar}\label{subsub:process_grammar}

\begin{mathpar}
  \inferrule* [lab=synchronization] {} {{M} \bc \pzero \;|\; x?F \;|\; x!C }
  \and
  \inferrule* [lab=abstraction] {} {{F} \bc (x)P}
  \and
  \inferrule* [lab=concretion] {} {{C} \bc \langle Q \rangle}
  \and
  \inferrule* [lab=process] {} {{P,Q} \bc M \;| \;P|Q \;|\; @{x}}
  \and
  \inferrule* [lab=name] {} {{x} \bc \quotep{P}}
\end{mathpar} 

Note that $\vec{x}$ (resp. $\vec{P}$) denotes a vector of names
(resp. processes) of length $|\vec{x}|$ (resp. $|\vec{P}|$). We adopt
the following useful abbreviations.

\begin{mathpar}
   x?(\vec{y}).P := x.(\vec{y})P \and  x\clift{\vec{P}} := x.\clift{\vec{P}}
   \and x!(y) := \lift{x}{\dropn{y}}
   \and \Pi_{i=0}^{n-1}P_i := P_0 | \ldots | P_{n-1}
\end{mathpar}

\subsubsection{Structural congruence}

\paragraph{Free and bound names and alpha-equivalence.} At the
core of structural equivalence is alpha-equivalence which identifies
process that are the same up to a change of variable. Formally, we
recognize the distinction between free and bound names. The free names
of a process, $\freenames{P}$, may be calculated recursively as
follows:

\begin{mathpar}
\freenames{\pzero} := \emptyset
  \and \\
  \freenames{x?(y).P} := \{ x \} \cup (\freenames{P} \setminus \{ y \})
  \and 
  \freenames{x!\langle P \rangle} := \{ x \} \cup \{ P \} 
  \and \\
  \freenames{P|Q} := \freenames{P} \cup \freenames{Q}
  \and \\
  \freenames{@{x}} := \{ x \}
\end{mathpar}

$\pi$
$\quotep{\pi}$

$\freenames{-} : \pi \to \mathcal{P}(\quotep{\pi})$

\begin{eqnarray*}
  \freenames{\pzero} & := & \emptyset \\
  \freenames{x?(y).P} & := & \{ x \} \cup (\freenames{P} \setminus \{ y \}) \\
  \freenames{x!\langle P \rangle} & := & \{ x \} \cup \{ P \} \\
  \freenames{P|Q} & := & \freenames{P} \cup \freenames{Q} \\
  \freenames{\dropn{x}} & := & \{ x \}
\end{eqnarray*}

The bound names of a process, $\boundnames{P}$, are those names occurring in $P$
that are not free. For example, in $x?(y).0$, the name $x$ is free, while $y$ is bound.

\begin{mathpar}
  \inferrule* [lab=monoidal-laws] {} { P|Q \equiv Q|P \and P|0 \equiv P \and P|(Q|R) \equiv (P|Q)|R }
\end{mathpar}

\begin{mathpar}
  \inferrule* [lab=alpha-equivalence] {} { (x)P \equiv (y)P\{y/x\} \and y \not\in \freenames{P} }
\end{mathpar}

\begin{definition}
Then two processes, $P,Q$, are alpha-equivalent if $P = Q\{\vec{y}/\vec{x}\}$ for
some $\vec{x} \in \boundnames{Q},\vec{y} \in \boundnames{P}$, where $Q\{\vec{y}/\vec{x}\}$
denotes the capture-avoiding substitution of $\vec{y}$ for $\vec{x}$ in $Q$.
\end{definition}

\begin{definition}
  The {\em structural congruence} \cite{SangiorgiWalker} , $\equiv$,
  between processes is the least congruence containing
  alpha-equivalence, satisfying the abelian monoid laws
  (associativity, commutativity and $\pzero$ as identity) for parallel
  composition $|$ and for summation $+$.
\end{definition}

\subsection{Name equivalence}

We take name equivalence, written $\nameeq$, to be the smallest
equivalence relation generated by the following rules.

\begin{mathpar}
\inferrule*[lab=Quote-drop]
{ }
{ \quotep{@{x}} \nameeq x }

\inferrule*[lab=Struct-equiv]
{ P \scong Q }
{ \quotep{P} \nameeq \quotep{Q} }
\end{mathpar}

The astute reader will have noticed that the mutual recursion of names
and processes imposes a mutual recursion on alpha-equivalence and
structural equivalence via name-equivalence. Fortunately, all of this
works out pleasantly and we may calculate in the natural way, free of
concern. The reader interested in the details is referred to the
appendix \ref{appendix:rho_details}.

\subsection{Substitution}

We use $\Proc$ for the set of processes, $\QProc$ for the set of
names, and $\id{\{}\vec{y} / \vec{x} \id{\}}$ to denote partial maps,
$s : \QProc \rightarrow \QProc$. A map, $s$ lifts, uniquely, to a map
on process terms, $\widehat{s} : \Proc \rightarrow \Proc$ by the
following equations.

\begin{mathpar}
  (0) \psubstp{Q}{P} := 0 \\
  (R \juxtap S) \psubstp{Q}{P}
  :=    
  (R)\psubstp{Q}{P} \juxtap (S) \psubstp{Q}{P} \\
  (x?(y).R) \psubstp{Q}{P}    
  :=    
  (x)\substp{Q}{P} (z)\concat( (R \psubstn{z}{y}) \psubstp{Q}{P} ) \\
  (\lift{x}{R}) \psubstp{Q}{P}  
  :=
  \lift{(x)\substp{Q}{P}}{ R \psubstp{Q}{P} } \\
%   (\dropn{x})  \psubstp{Q}{P}       
%   := 
%   \left\{ 
%     \begin{array}{ccc} 
%       \dropn{\quotep{Q}} & & x \nameeq \quotep{P} \\
%       \dropn{x} & & otherwise \\
%     \end{array}
%   \right. 
  (\dropn{x})  \psubstp{Q}{P}       
  := 
  \left\{ 
    \begin{array}{ccc} 
      Q & & x \nameeq \quotep{P} \\
      \dropn{x} & & otherwise \\
    \end{array}
  \right.
\end{mathpar}
 

where

\begin{eqnarray}
  (x)\id{\{} \lpquote Q \rpquote / \lpquote P \rpquote \id{\}}            = 
  \left\{ 
    \begin{array}{ccc}
      \lpquote Q \rpquote & & x \nameeq \lpquote P \rpquote \\
      x & & otherwise \\
    \end{array}
  \right. \nonumber
\end{eqnarray}

and $z$ is chosen distinct from $\quotep{P}$, $\quotep{Q}$, the free
names in $Q$, and all the names in $R$. Our $\alpha$-equivalence will
be built in the standard way from this substitution.

\begin{remark}\label{rem:no_self_referential_names}
  One consequence of these definitions is that $\forall P. \quotep{P}
  \not\in \freenames{P}$.
\end{remark}

\subsection{ Dynamic quote: an example }

Anticipating something of what's to come, consider applying the
substitution, $\widehat{\id{\{}u / z \id{\}}}$, to the following pair
of processes, $\lift{w}{y!(z)}$ and $w[ \lpquote y!(z) \rpquote ]$.

\begin{eqnarray}
	\lift{w}{y!(z)}\widehat{\id{\{}u / z \id{\}}}
		& = &
		\lift{w}{y!(u)} \nonumber\\
	w[ \lpquote y!(z) \rpquote ] \widehat{ \id{\{}u / z \id{\}} }
		& = &
		w[ \lpquote y!(z) \rpquote ] \nonumber
\end{eqnarray}

Because the body of the process between quotes is impervious to
substitution, we get radically different answers. In fact, by
examining the first process in an input context,
e.g. $x?(z).\lift{w}{y!(z)}$, we see that the process under the lift
operator may be shaped by prefixed inputs binding a name inside it. In
this sense, the lift operator will be seen as a way to dynamically
construct processes before reifying them as names.

Finally equipped with these standard features we can present the
dynamics of the calculus.

\subsubsection{Operational semantics} 

Finally, we introduce the computational dynamics. What marks these
algebras as distinct from other more traditionally studied algebraic
structures, e.g. vector spaces or polynomial rings, is the manner in
which dynamics is captured. In traditional structures, dynamics is typically
expressed through morphisms between such structures, as in linear maps
between vector spaces or morphisms between rings. In algebras
associated with the semantics of computation, the dynamics is
expressed as part of the algebraic structure itself, through a
reduction reduction relation typically denoted by $\red$. Below, we
give a recursive presentation of this relation for the calculus used
in the encoding.

$\red \subseteq \pi \times \pi$
$\red : \pi \to \mathcal{P}(\pi)$

\begin{mathpar}
  \inferrule* [lab=Comm] { \textsf{match}( x_{src}, x_{trgt} ) } { x_{trgt}?(y)P \; | \; x_{src}!\langle {Q} \rangle \red P\{\quotep{Q}/y}\} }
  \and \\
  \inferrule* [lab=Par] {{P} \red {P}'} {{{P} | {Q}} \red {{P}' | {Q}}}
  \and
  \inferrule* [lab=Equiv]{{{P} \scong {P}'} \andalso {{P}' \red {Q}'} \andalso {{Q}' \scong {Q}}}{{P} \red {Q}}
\end{mathpar}

\begin{eqnarray*}
  match_{\equiv} (\quotep{P},\quotep{Q}) & := & P \equiv Q \\
  match_{\dagger}(\quotep{P},\quotep{Q}) & := & \forall R. P|Q \red^{*} R => R \red^{*} 0 \\
  match_{K}(\quotep{P},\quotep{Q}) & := & K \mbox{ for some context } K
\end{eqnarray*}

$u?(x)P | u!\langle Q \rangle \red P\{\quotep{Q}/x\}$

%We write $\wred$ for $\red^*$, and $P\red$ if $\exists Q $ such that $ P \red Q$.
We write $P\red$ if $\exists Q $ such that $ P \red Q$ and $P\not\red$, otherwise.

\section{Replication}

As mentioned before, it is known that replication (and hence
recursion) can be implemented in a higher-order process algebra
\cite{SangiorgiWalker}. As our first example of calculation with the
machinery thus far presented we give the construction explicitly in
the {\rhoc}.

\begin{eqnarray}
	D_{x} & := & \prefix{x}{y}{(\binpar{\outputp{x}{y}}{@{y}})} \nonumber\\
	\bangp_{x}{P} & := & \binpar{{x}!\langle{\binpar{D_{x}}{P}}\rangle}{D_{x}} \nonumber
\end{eqnarray}

\begin{eqnarray}
	\bangp_{x}{P} & & \nonumber\\
	=
	& {x}!\langle{(\prefix{x}{y}{(\outputp{x}{y} | @{y})) | P}}\rangle 
	      | \prefix{x}{y}{(\outputp{x}{y} | @{y})} & \nonumber\\
	\red
	& (\outputp{x}{y} | @{y})\substn{\quotep{(\prefix{x}{y}{(@{y} | \outputp{x}{y})) | P}}}{y} & \nonumber\\
	=
	& \outputp{x}{\quotep{(\prefix{x}{y}{(\outputp{x}{y} | @{y})) | P}}}
	  | {(\prefix{x}{y}{(\outputp{x}{y} | @{y})) | P}} & \nonumber\\
	\red
	& \ldots & \nonumber\\
	\red^*
	& P | P | \ldots & \nonumber
\end{eqnarray}

Of course, this encoding, as an implementation, runs away, unfolding
$\bangp{P}$ eagerly. A lazier and more implementable replication
operator, restricted to input-guarded processes, may be obtained as follows.

\begin{eqnarray}
\bangp{\prefix{u}{v}{P}} 
	:= 
	\binpar{\lift{x}{\prefix{u}{v}{(\binpar{D(x)}{P})}}}{D(x)} \nonumber
\end{eqnarray}

\begin{remark}
  Note that the lazier definition still does not deal with summation
  or mixed summation (i.e. sums over input and output). The reader is
  invited to construct definitions of replication that deal with these
  features. 

  Further, the definitions are parameterized in a name, $x$. Can you,
  gentle reader, make a definition that eliminates this parameter and
  guarantees no accidental interaction between the replication
  machinery and the process being replicated -- i.e. no accidental
  sharing of names used by the process to get its work done and the
  name(s) used by the replication to effect copying. This latter
  revision of the definition of replication is crucial to obtaining
  the expected identity $!!P \sim !P$.
\end{remark}

\begin{remark}\label{rem:paradoxical_combinator}
  The reader familiar with the lambda calculus will have noticed the
  similarity between $D$ and the paradoxical combinator.

  [Ed. note: the existence of this seems to suggest we have to be more
  restrictive on the set of processes and names we admit if we are to
  support no-cloning.]
\end{remark}

\subsubsection{Bisimulation}

The computational dynamics gives rise to another kind of equivalence,
the equivalence of computational behavior. As previously mentioned
this is typically captured \emph{via} some form of bisimulation.

% The notion we use in this paper is weak barbed bisimulation
% \cite{milner91polyadicpi}.

The notion we use in this paper is derived from weak barbed
bisimulation \cite{milner91polyadicpi}. 

\begin{definition}
An \emph{observation relation}, $\downarrow_{\mathcal N}$, over a set
of names, $\mathcal N$, is the smallest relation satisfying the rules
below.

\infrule[Out-barb]{y \in {\mathcal N}, \; x \nameeq y}
		  {\outputp{x}{v} \downarrow_{\mathcal N} x}
\infrule[Par-barb]{\mbox{$P\downarrow_{\mathcal N} x$ or $Q\downarrow_{\mathcal N} x$}}
		  {\binpar{P}{Q} \downarrow_{\mathcal N} x}

We write $P \Downarrow_{\mathcal N} x$ if there is $Q$ such that 
$P \wred Q$ and $Q \downarrow_{\mathcal N} x$.
\end{definition}

\begin{definition}
%\label{def.bbisim}
An  ${\mathcal N}$-\emph{barbed bisimulation} over a set of names, ${\mathcal N}$, is a symmetric binary relation 
${\mathcal S}_{\mathcal N}$ between agents such that $P\rel{S}_{\mathcal N}Q$ implies:
\begin{enumerate}
\item If $P \red P'$ then $Q \wred Q'$ and $P'\rel{S}_{\mathcal N} Q'$.
\item If $P\downarrow_{\mathcal N} x$, then $Q\Downarrow_{\mathcal N} x$.
\end{enumerate}
$P$ is ${\mathcal N}$-barbed bisimilar to $Q$, written
$P \wbbisim_{\mathcal N} Q$, if $P \rel{S}_{\mathcal N} Q$ for some ${\mathcal N}$-barbed bisimulation ${\mathcal S}_{\mathcal N}$.
\end{definition}

$\mathcal{R} \subseteq \pi \times \pi$

$P \mathcal{R} Q => \forall P'. P \red P' \Rightarrow \exists Q'. Q \red Q', P' \mathcal{R} Q'$

$P \vdash x \Rightarrow Q \vdash x$

\begin{mathpar}
  \inferrule*[lab=Out-barb]{x \nameeq y}{{y}!\langle{Q}\rangle \vdash x}
  \and
  \inferrule*[lab=Par-barb]{\mbox{$P\vdash x$ or $Q\vdash x$}}{\binpar{P}{Q} \vdash x}
\end{mathpar}

\subsubsection{Contexts}

One of the principle advantages of computational calculi like the
$\pi$-calculus is a well-defined notion of context,
contextual-equivalence and a correlation between
contextual-equivalence and notions of bisimulation. The notion of
context allows the decomposition of a process into (sub-)process and
its syntactic environment, its context. Thus, a context may be
thought of as a process with a ``hole'' (written $\Box$) in it. The
application of a context $M$ to a process $P$, written $M[P]$, is
tantamount to filling the hole in $M$ with $P$. In this paper we do
not need the full weight of this theory, but do make use of the notion
of context in the proof the main theorem. 

\begin{mathpar}
  \inferrule* [lab=summation] {} {{M_{M},M_{N}} \bc \Box \;|\; x.M_{A} \;|\; M_{M}+M_{N}}
  \and
  \inferrule* [lab=agent] {} {{M_{A}} \bc (\vec{x})M_{P} \;| \; \clift{P_0,\ldots,M_{P},\ldots,P_N}}
  \and \\
  \inferrule* [lab=process] {} {{M_{P}} \bc M_{N} \;| \;P|M_{P} }
\end{mathpar} 

\begin{mathpar}
  \inferrule* [lab=sychronization] {} {M_{N} \bc \Box \;|\; x?M_{F} \;|\; x!M_{C}}
  \and
  \inferrule* [lab=abstraction] {} {{M_{F}} \bc (x)M_{P} }
  \and
  \inferrule* [lab=concretion] {} {{M_{C}} \bc \langle M_{P} \rangle }
  \and \\
  \inferrule* [lab=process] {} {{M_{P}} \bc M_{N} \;| \;P|M_{P} }
\end{mathpar}

\begin{definition}[contextual application] Given a context $M$, and
  process $P$, we define the \emph{contextual application}, $M[P] :=
  M\{P/\Box\}$. That is, the contextual application of M to P is the
  substitution of $P$ for $\Box$ in $M$.
\end{definition}

$\meaningof{-} : L \to \mathcal{P}(\pi)$

\begin{mathpar}
  \inferrule* [lab=collection] {} {\meaningof{true} = \pi, \and \meaningof{~E} = \pi \setminus \meaningof{E}, \and \meaningof{E_{1} \& E_{2}} = \meaningof{E_{1}} \cap \meaningof{E_{2}}}
\end{mathpar}

\begin{mathpar}
  \inferrule* [lab=structure] {} {\meaningof{0} = \{ P \in \pi | P \equiv 0 \}, \and \\ \meaningof{E_1 | E_2} = \{ P \in \pi | P \equiv P_{1} | P_{2}, P_{1} \in \meaningof{E_{1}}, P_{2} \in \meaningof{E_2}\} }
\end{mathpar}

\begin{mathpar}
 \inferrule* [lab=behavior] {} {\meaningof{\langle a?b \rangle E} = \{ P \in \pi | P \equiv Q | u?(y)P', \\ \and \\\\ \and \\ \;\;\; u \in \meaningof{a}, \forall z.P'\{z/y\} \in \meaningof{E\{z/b\}}\}, \and \\ \meaningof{a!E} = \{ P \in \pi | P \equiv Q | x!\langle P' \rangle, x \in \meaningof{a} P' \in \meaningof{E}\} }
\end{mathpar}

\begin{mathpar}
 \inferrule* [lab=nominal] {} {\meaningof{\quotep{E}} = \{ \quotep{P} \in \quotep{\pi} | P \in \meaningof{E} \}, \and \meaningof{\quotep{P}} = \{ \quotep{Q} \in \quotep{\pi} | P \equiv Q \} \and \\ \meaningof{@\quotep{E}} = \{ P \in \pi | P \equiv @x, x \in \meaningof{E} \}}
\end{mathpar}

\begin{eqnarray*}
  \\
  \meaningof{-} : TS \to ST
\end{eqnarray*}

\begin{eqnarray*}
  \\
  L : TS \to ST
\end{eqnarray*}

\begin{eqnarray*}
  \\
  P \models E \iff P \in \meaningof{E}
\end{eqnarray*}

\begin{eqnarray*}
  P \approx_{L} Q \iff \forall E \in L. P \models E \iff Q \models E
\end{eqnarray*}

\begin{eqnarray*}
  P \approx_{K} Q
\end{eqnarray*}

\begin{eqnarray*}
  P \approx Q
\end{eqnarray*}

$\approx_{K} = \approx = \approx_{L}$

\subsubsection{Contextual duality}

Note that contexts extend the quotation operation to a family of
operations from processes to names. Given a context, $M$, we can
define a \emph{nominal context}, $\quotep{M}$ by $\quotep{M}[P] :=
\quotep{M[P]}$. To foreshadow what is to come we observe that these
operations enjoy a duality with processes very much like the duality
between vectors and maps from vectors to scalars.

Further, because the calculus is essentially higher-order, we have a
correspondence between contexts and processes. More specifically,
given a name $x$ and a context $M$ we can construct $M^{*}_{x}$ such
that 

\begin{mathpar}
  M^{*}_{x} | \lift{x}{P} \red M[P]
\end{mathpar}

namely,

\begin{mathpar}
  M^{*}_{x} := x?(u).M[\dropn{u}]
\end{mathpar}

The dependence of $M^{*}_{x}$ on a name makes it an abstraction, 

\begin{mathpar}
  M^{*} := (x)x?(u).M[\dropn{u}]
\end{mathpar}

\subsection{Additional notation}

It will sometimes be convenient to denote the process a name
quotes. We already have the notation $x = \quotep{P}$, but it will be
convenient to introduce an alternate notation, $\procn{x}$, when we
want to emphasize the connection to the use of the name. Note that, by
virtue of name equivalence, $\quotep{\procn{x}} \nameeq x$; so, the
notation is consistent with previous definitions.

Further, because names have structure it is possible to effect
substitutions on the basis of that structure. This means we need to
upgrade our notation for substitutions, which we accomplish by
adapting comprehension notation. Thus,

\begin{mathpar}
  P\{ y / x : x \in S \}
\end{mathpar}

is interpreted to mean the process derived from P by replacing (in a
capture-avoiding manner) each occurrence of $x$ in $S$ by $y$. For example,

\begin{mathpar}
  P\{ \quotep{\procn{x}|\procn{x}} / x : x \in \freenames{P} \}
\end{mathpar}

will replace each (occurrence) of a free name $x$ in $P$ by
$\quotep{\procn{x}|\procn{x}}$.

Also, we will avail ourselves of the notation $x^{L}$ and $x^{R}$ to
denote injections of a name into disjoint copies of the name
space. There are numerous ways to accomplish this. One example can be
found in \cite{MeredithR05}. This notation overloads to vectors of
names: $\vec{x}^{\pi} := (x_{i}^{\pi} \; : \; 0 \leq i < |\vec{x}| )$ where $\pi \in \{L,R\}$.

We also use $P^{\Box} := P|\Box$.

In \cite{MeredithR05} an interpretation of the new operator is
given. It turns out that there are several possible interpretations
all enjoying the requisite algebraic properties of the operator (see
\cite{milner91polyadicpi}). We will therefore make liberal use of
$(\nu\; \vec{x})P$.

% subsection the_syntax_and_semantics_of_the_notation_system (end)   

\input{qm2pi.qmops} 

\input{qm2pi.sterngerlach} 

\input{qm2pi.metric} 

% section concurrent_process_calculi (end)

%\input{qm2pi.proofsketch}

% section proof sketch (end)

%\input{qm2pi.slviaknots} 

% section spatial logic via knots (end)

\input{qm2pi.conclusion}

% section conclusion (end)

%\input{qm2pi.dtcodes} 

% section wiring algorithm (end)

\input{qm2pi.ack} 

% section acknowledgments (end)

\newpage


\bibliographystyle{plain}   
\bibliography{../../biblios/main.bib}

\input{qm2pi.rhodetails}

\end{document}

 

% section concurrent_process_calculi (end)

%\documentclass[12pt]{llncs}
%\documentclass{jktr}

\usepackage[pdftex]{hyperref}                   
\usepackage {listings}
\usepackage {mathpartir}
\usepackage{bcprules}
%\usepackage{listings}
                       
\usepackage{graphicx} 
%\usepackage[margins=2.5cm,nohead,nofoot]{geometry}
%\usepackage{geometry}
\usepackage{amsfonts}
\usepackage{amstext}
\usepackage{latexsym}
\usepackage{amssymb}
\usepackage{color}


%\include{myPreamble}
\include{qm2pi.local} 

%\ifpdf
%\usepackage[pdftex]{graphicx}
%\else
%\usepackage{graphicx}
%\fi

 % \ifpdf
%  \usepackage{pdfsync}
%  \if


%\title{Brief Article}
%\author{David F. Snyder}
%\author{L.G. Meredith}

%\address{Dept. of Math., Texas State University--San Marcos, San Marcos, TX 78666}
       
\pagestyle{empty}


\begin{document}

\lstset{language=[Objective]Caml,frame=shadowbox}

\input{qm2pi.front}

% section front matter (end)

\input{qm2pi.intro} 
 
% section introduction (end)

% \input{qm2pi.knotations} 

% section notation (end)

\input{qm2pi.process.calculi} 

% section concurrent_process_calculi_and_spatial_logics_ (end)
    
%\input{qm2pi.knots2pi} 

%\input{qm2pi.trefoil} 

%\input{qm2pi.mainthm} 

% subsection basic_interpretation (end)

%\input{qm2pi.rho.presentation} 
\subsection{The syntax and semantics of the notation system}\label{sub:the_syntax_and_semantics_of_the_notation_system} % (fold)

We now summarize a technical presentation of the calculus that
embodies our theory of dynamics. The typical presentation of such a
calculus follows the style of giving generators and relations on
them. The grammar, below, describing term constructors, freely
generates the set of processes, $\Proc$. This set is then quotiented
by a relation known as structural congruence and it is over this set
that the notion of dynamics is expressed. This presentation is
essentially that of \cite{MeredithR05} with the addition of
polyadicity and summation. For readability we have relegated some of
the technical subtleties to an appendix.

\subsubsection{Process grammar}\label{subsub:process_grammar}

\begin{mathpar}
  \inferrule* [lab=synchronization] {} {{M} \bc \pzero \;|\; x?F \;|\; x!C }
  \and
  \inferrule* [lab=abstraction] {} {{F} \bc (x)P}
  \and
  \inferrule* [lab=concretion] {} {{C} \bc \langle Q \rangle}
  \and
  \inferrule* [lab=process] {} {{P,Q} \bc M \;| \;P|Q \;|\; @{x}}
  \and
  \inferrule* [lab=name] {} {{x} \bc \quotep{P}}
\end{mathpar} 

Note that $\vec{x}$ (resp. $\vec{P}$) denotes a vector of names
(resp. processes) of length $|\vec{x}|$ (resp. $|\vec{P}|$). We adopt
the following useful abbreviations.

\begin{mathpar}
   x?(\vec{y}).P := x.(\vec{y})P \and  x\clift{\vec{P}} := x.\clift{\vec{P}}
   \and x!(y) := \lift{x}{\dropn{y}}
   \and \Pi_{i=0}^{n-1}P_i := P_0 | \ldots | P_{n-1}
\end{mathpar}

\subsubsection{Structural congruence}

\paragraph{Free and bound names and alpha-equivalence.} At the
core of structural equivalence is alpha-equivalence which identifies
process that are the same up to a change of variable. Formally, we
recognize the distinction between free and bound names. The free names
of a process, $\freenames{P}$, may be calculated recursively as
follows:

\begin{mathpar}
\freenames{\pzero} := \emptyset
  \and \\
  \freenames{x?(y).P} := \{ x \} \cup (\freenames{P} \setminus \{ y \})
  \and 
  \freenames{x!\langle P \rangle} := \{ x \} \cup \{ P \} 
  \and \\
  \freenames{P|Q} := \freenames{P} \cup \freenames{Q}
  \and \\
  \freenames{@{x}} := \{ x \}
\end{mathpar}

$\pi$
$\quotep{\pi}$

$\freenames{-} : \pi \to \mathcal{P}(\quotep{\pi})$

\begin{eqnarray*}
  \freenames{\pzero} & := & \emptyset \\
  \freenames{x?(y).P} & := & \{ x \} \cup (\freenames{P} \setminus \{ y \}) \\
  \freenames{x!\langle P \rangle} & := & \{ x \} \cup \{ P \} \\
  \freenames{P|Q} & := & \freenames{P} \cup \freenames{Q} \\
  \freenames{\dropn{x}} & := & \{ x \}
\end{eqnarray*}

The bound names of a process, $\boundnames{P}$, are those names occurring in $P$
that are not free. For example, in $x?(y).0$, the name $x$ is free, while $y$ is bound.

\begin{mathpar}
  \inferrule* [lab=monoidal-laws] {} { P|Q \equiv Q|P \and P|0 \equiv P \and P|(Q|R) \equiv (P|Q)|R }
\end{mathpar}

\begin{mathpar}
  \inferrule* [lab=alpha-equivalence] {} { (x)P \equiv (y)P\{y/x\} \and y \not\in \freenames{P} }
\end{mathpar}

\begin{definition}
Then two processes, $P,Q$, are alpha-equivalent if $P = Q\{\vec{y}/\vec{x}\}$ for
some $\vec{x} \in \boundnames{Q},\vec{y} \in \boundnames{P}$, where $Q\{\vec{y}/\vec{x}\}$
denotes the capture-avoiding substitution of $\vec{y}$ for $\vec{x}$ in $Q$.
\end{definition}

\begin{definition}
  The {\em structural congruence} \cite{SangiorgiWalker} , $\equiv$,
  between processes is the least congruence containing
  alpha-equivalence, satisfying the abelian monoid laws
  (associativity, commutativity and $\pzero$ as identity) for parallel
  composition $|$ and for summation $+$.
\end{definition}

\subsection{Name equivalence}

We take name equivalence, written $\nameeq$, to be the smallest
equivalence relation generated by the following rules.

\begin{mathpar}
\inferrule*[lab=Quote-drop]
{ }
{ \quotep{@{x}} \nameeq x }

\inferrule*[lab=Struct-equiv]
{ P \scong Q }
{ \quotep{P} \nameeq \quotep{Q} }
\end{mathpar}

The astute reader will have noticed that the mutual recursion of names
and processes imposes a mutual recursion on alpha-equivalence and
structural equivalence via name-equivalence. Fortunately, all of this
works out pleasantly and we may calculate in the natural way, free of
concern. The reader interested in the details is referred to the
appendix \ref{appendix:rho_details}.

\subsection{Substitution}

We use $\Proc$ for the set of processes, $\QProc$ for the set of
names, and $\id{\{}\vec{y} / \vec{x} \id{\}}$ to denote partial maps,
$s : \QProc \rightarrow \QProc$. A map, $s$ lifts, uniquely, to a map
on process terms, $\widehat{s} : \Proc \rightarrow \Proc$ by the
following equations.

\begin{mathpar}
  (0) \psubstp{Q}{P} := 0 \\
  (R \juxtap S) \psubstp{Q}{P}
  :=    
  (R)\psubstp{Q}{P} \juxtap (S) \psubstp{Q}{P} \\
  (x?(y).R) \psubstp{Q}{P}    
  :=    
  (x)\substp{Q}{P} (z)\concat( (R \psubstn{z}{y}) \psubstp{Q}{P} ) \\
  (\lift{x}{R}) \psubstp{Q}{P}  
  :=
  \lift{(x)\substp{Q}{P}}{ R \psubstp{Q}{P} } \\
%   (\dropn{x})  \psubstp{Q}{P}       
%   := 
%   \left\{ 
%     \begin{array}{ccc} 
%       \dropn{\quotep{Q}} & & x \nameeq \quotep{P} \\
%       \dropn{x} & & otherwise \\
%     \end{array}
%   \right. 
  (\dropn{x})  \psubstp{Q}{P}       
  := 
  \left\{ 
    \begin{array}{ccc} 
      Q & & x \nameeq \quotep{P} \\
      \dropn{x} & & otherwise \\
    \end{array}
  \right.
\end{mathpar}
 

where

\begin{eqnarray}
  (x)\id{\{} \lpquote Q \rpquote / \lpquote P \rpquote \id{\}}            = 
  \left\{ 
    \begin{array}{ccc}
      \lpquote Q \rpquote & & x \nameeq \lpquote P \rpquote \\
      x & & otherwise \\
    \end{array}
  \right. \nonumber
\end{eqnarray}

and $z$ is chosen distinct from $\quotep{P}$, $\quotep{Q}$, the free
names in $Q$, and all the names in $R$. Our $\alpha$-equivalence will
be built in the standard way from this substitution.

\begin{remark}\label{rem:no_self_referential_names}
  One consequence of these definitions is that $\forall P. \quotep{P}
  \not\in \freenames{P}$.
\end{remark}

\subsection{ Dynamic quote: an example }

Anticipating something of what's to come, consider applying the
substitution, $\widehat{\id{\{}u / z \id{\}}}$, to the following pair
of processes, $\lift{w}{y!(z)}$ and $w[ \lpquote y!(z) \rpquote ]$.

\begin{eqnarray}
	\lift{w}{y!(z)}\widehat{\id{\{}u / z \id{\}}}
		& = &
		\lift{w}{y!(u)} \nonumber\\
	w[ \lpquote y!(z) \rpquote ] \widehat{ \id{\{}u / z \id{\}} }
		& = &
		w[ \lpquote y!(z) \rpquote ] \nonumber
\end{eqnarray}

Because the body of the process between quotes is impervious to
substitution, we get radically different answers. In fact, by
examining the first process in an input context,
e.g. $x?(z).\lift{w}{y!(z)}$, we see that the process under the lift
operator may be shaped by prefixed inputs binding a name inside it. In
this sense, the lift operator will be seen as a way to dynamically
construct processes before reifying them as names.

Finally equipped with these standard features we can present the
dynamics of the calculus.

\subsubsection{Operational semantics} 

Finally, we introduce the computational dynamics. What marks these
algebras as distinct from other more traditionally studied algebraic
structures, e.g. vector spaces or polynomial rings, is the manner in
which dynamics is captured. In traditional structures, dynamics is typically
expressed through morphisms between such structures, as in linear maps
between vector spaces or morphisms between rings. In algebras
associated with the semantics of computation, the dynamics is
expressed as part of the algebraic structure itself, through a
reduction reduction relation typically denoted by $\red$. Below, we
give a recursive presentation of this relation for the calculus used
in the encoding.

$\red \subseteq \pi \times \pi$
$\red : \pi \to \mathcal{P}(\pi)$

\begin{mathpar}
  \inferrule* [lab=Comm] { \textsf{match}( x_{src}, x_{trgt} ) } { x_{trgt}?(y)P \; | \; x_{src}!\langle {Q} \rangle \red P\{\quotep{Q}/y}\} }
  \and \\
  \inferrule* [lab=Par] {{P} \red {P}'} {{{P} | {Q}} \red {{P}' | {Q}}}
  \and
  \inferrule* [lab=Equiv]{{{P} \scong {P}'} \andalso {{P}' \red {Q}'} \andalso {{Q}' \scong {Q}}}{{P} \red {Q}}
\end{mathpar}

\begin{eqnarray*}
  match_{\equiv} (\quotep{P},\quotep{Q}) & := & P \equiv Q \\
  match_{\dagger}(\quotep{P},\quotep{Q}) & := & \forall R. P|Q \red^{*} R => R \red^{*} 0 \\
  match_{K}(\quotep{P},\quotep{Q}) & := & K \mbox{ for some context } K
\end{eqnarray*}

$u?(x)P | u!\langle Q \rangle \red P\{\quotep{Q}/x\}$

%We write $\wred$ for $\red^*$, and $P\red$ if $\exists Q $ such that $ P \red Q$.
We write $P\red$ if $\exists Q $ such that $ P \red Q$ and $P\not\red$, otherwise.

\section{Replication}

As mentioned before, it is known that replication (and hence
recursion) can be implemented in a higher-order process algebra
\cite{SangiorgiWalker}. As our first example of calculation with the
machinery thus far presented we give the construction explicitly in
the {\rhoc}.

\begin{eqnarray}
	D_{x} & := & \prefix{x}{y}{(\binpar{\outputp{x}{y}}{@{y}})} \nonumber\\
	\bangp_{x}{P} & := & \binpar{{x}!\langle{\binpar{D_{x}}{P}}\rangle}{D_{x}} \nonumber
\end{eqnarray}

\begin{eqnarray}
	\bangp_{x}{P} & & \nonumber\\
	=
	& {x}!\langle{(\prefix{x}{y}{(\outputp{x}{y} | @{y})) | P}}\rangle 
	      | \prefix{x}{y}{(\outputp{x}{y} | @{y})} & \nonumber\\
	\red
	& (\outputp{x}{y} | @{y})\substn{\quotep{(\prefix{x}{y}{(@{y} | \outputp{x}{y})) | P}}}{y} & \nonumber\\
	=
	& \outputp{x}{\quotep{(\prefix{x}{y}{(\outputp{x}{y} | @{y})) | P}}}
	  | {(\prefix{x}{y}{(\outputp{x}{y} | @{y})) | P}} & \nonumber\\
	\red
	& \ldots & \nonumber\\
	\red^*
	& P | P | \ldots & \nonumber
\end{eqnarray}

Of course, this encoding, as an implementation, runs away, unfolding
$\bangp{P}$ eagerly. A lazier and more implementable replication
operator, restricted to input-guarded processes, may be obtained as follows.

\begin{eqnarray}
\bangp{\prefix{u}{v}{P}} 
	:= 
	\binpar{\lift{x}{\prefix{u}{v}{(\binpar{D(x)}{P})}}}{D(x)} \nonumber
\end{eqnarray}

\begin{remark}
  Note that the lazier definition still does not deal with summation
  or mixed summation (i.e. sums over input and output). The reader is
  invited to construct definitions of replication that deal with these
  features. 

  Further, the definitions are parameterized in a name, $x$. Can you,
  gentle reader, make a definition that eliminates this parameter and
  guarantees no accidental interaction between the replication
  machinery and the process being replicated -- i.e. no accidental
  sharing of names used by the process to get its work done and the
  name(s) used by the replication to effect copying. This latter
  revision of the definition of replication is crucial to obtaining
  the expected identity $!!P \sim !P$.
\end{remark}

\begin{remark}\label{rem:paradoxical_combinator}
  The reader familiar with the lambda calculus will have noticed the
  similarity between $D$ and the paradoxical combinator.

  [Ed. note: the existence of this seems to suggest we have to be more
  restrictive on the set of processes and names we admit if we are to
  support no-cloning.]
\end{remark}

\subsubsection{Bisimulation}

The computational dynamics gives rise to another kind of equivalence,
the equivalence of computational behavior. As previously mentioned
this is typically captured \emph{via} some form of bisimulation.

% The notion we use in this paper is weak barbed bisimulation
% \cite{milner91polyadicpi}.

The notion we use in this paper is derived from weak barbed
bisimulation \cite{milner91polyadicpi}. 

\begin{definition}
An \emph{observation relation}, $\downarrow_{\mathcal N}$, over a set
of names, $\mathcal N$, is the smallest relation satisfying the rules
below.

\infrule[Out-barb]{y \in {\mathcal N}, \; x \nameeq y}
		  {\outputp{x}{v} \downarrow_{\mathcal N} x}
\infrule[Par-barb]{\mbox{$P\downarrow_{\mathcal N} x$ or $Q\downarrow_{\mathcal N} x$}}
		  {\binpar{P}{Q} \downarrow_{\mathcal N} x}

We write $P \Downarrow_{\mathcal N} x$ if there is $Q$ such that 
$P \wred Q$ and $Q \downarrow_{\mathcal N} x$.
\end{definition}

\begin{definition}
%\label{def.bbisim}
An  ${\mathcal N}$-\emph{barbed bisimulation} over a set of names, ${\mathcal N}$, is a symmetric binary relation 
${\mathcal S}_{\mathcal N}$ between agents such that $P\rel{S}_{\mathcal N}Q$ implies:
\begin{enumerate}
\item If $P \red P'$ then $Q \wred Q'$ and $P'\rel{S}_{\mathcal N} Q'$.
\item If $P\downarrow_{\mathcal N} x$, then $Q\Downarrow_{\mathcal N} x$.
\end{enumerate}
$P$ is ${\mathcal N}$-barbed bisimilar to $Q$, written
$P \wbbisim_{\mathcal N} Q$, if $P \rel{S}_{\mathcal N} Q$ for some ${\mathcal N}$-barbed bisimulation ${\mathcal S}_{\mathcal N}$.
\end{definition}

$\mathcal{R} \subseteq \pi \times \pi$

$P \mathcal{R} Q => \forall P'. P \red P' \Rightarrow \exists Q'. Q \red Q', P' \mathcal{R} Q'$

$P \vdash x \Rightarrow Q \vdash x$

\begin{mathpar}
  \inferrule*[lab=Out-barb]{x \nameeq y}{{y}!\langle{Q}\rangle \vdash x}
  \and
  \inferrule*[lab=Par-barb]{\mbox{$P\vdash x$ or $Q\vdash x$}}{\binpar{P}{Q} \vdash x}
\end{mathpar}

\subsubsection{Contexts}

One of the principle advantages of computational calculi like the
$\pi$-calculus is a well-defined notion of context,
contextual-equivalence and a correlation between
contextual-equivalence and notions of bisimulation. The notion of
context allows the decomposition of a process into (sub-)process and
its syntactic environment, its context. Thus, a context may be
thought of as a process with a ``hole'' (written $\Box$) in it. The
application of a context $M$ to a process $P$, written $M[P]$, is
tantamount to filling the hole in $M$ with $P$. In this paper we do
not need the full weight of this theory, but do make use of the notion
of context in the proof the main theorem. 

\begin{mathpar}
  \inferrule* [lab=summation] {} {{M_{M},M_{N}} \bc \Box \;|\; x.M_{A} \;|\; M_{M}+M_{N}}
  \and
  \inferrule* [lab=agent] {} {{M_{A}} \bc (\vec{x})M_{P} \;| \; \clift{P_0,\ldots,M_{P},\ldots,P_N}}
  \and \\
  \inferrule* [lab=process] {} {{M_{P}} \bc M_{N} \;| \;P|M_{P} }
\end{mathpar} 

\begin{mathpar}
  \inferrule* [lab=sychronization] {} {M_{N} \bc \Box \;|\; x?M_{F} \;|\; x!M_{C}}
  \and
  \inferrule* [lab=abstraction] {} {{M_{F}} \bc (x)M_{P} }
  \and
  \inferrule* [lab=concretion] {} {{M_{C}} \bc \langle M_{P} \rangle }
  \and \\
  \inferrule* [lab=process] {} {{M_{P}} \bc M_{N} \;| \;P|M_{P} }
\end{mathpar}

\begin{definition}[contextual application] Given a context $M$, and
  process $P$, we define the \emph{contextual application}, $M[P] :=
  M\{P/\Box\}$. That is, the contextual application of M to P is the
  substitution of $P$ for $\Box$ in $M$.
\end{definition}

$\meaningof{-} : L \to \mathcal{P}(\pi)$

\begin{mathpar}
  \inferrule* [lab=collection] {} {\meaningof{true} = \pi, \and \meaningof{~E} = \pi \setminus \meaningof{E}, \and \meaningof{E_{1} \& E_{2}} = \meaningof{E_{1}} \cap \meaningof{E_{2}}}
\end{mathpar}

\begin{mathpar}
  \inferrule* [lab=structure] {} {\meaningof{0} = \{ P \in \pi | P \equiv 0 \}, \and \\ \meaningof{E_1 | E_2} = \{ P \in \pi | P \equiv P_{1} | P_{2}, P_{1} \in \meaningof{E_{1}}, P_{2} \in \meaningof{E_2}\} }
\end{mathpar}

\begin{mathpar}
 \inferrule* [lab=behavior] {} {\meaningof{\langle a?b \rangle E} = \{ P \in \pi | P \equiv Q | u?(y)P', \\ \and \\\\ \and \\ \;\;\; u \in \meaningof{a}, \forall z.P'\{z/y\} \in \meaningof{E\{z/b\}}\}, \and \\ \meaningof{a!E} = \{ P \in \pi | P \equiv Q | x!\langle P' \rangle, x \in \meaningof{a} P' \in \meaningof{E}\} }
\end{mathpar}

\begin{mathpar}
 \inferrule* [lab=nominal] {} {\meaningof{\quotep{E}} = \{ \quotep{P} \in \quotep{\pi} | P \in \meaningof{E} \}, \and \meaningof{\quotep{P}} = \{ \quotep{Q} \in \quotep{\pi} | P \equiv Q \} \and \\ \meaningof{@\quotep{E}} = \{ P \in \pi | P \equiv @x, x \in \meaningof{E} \}}
\end{mathpar}

\begin{eqnarray*}
  \\
  \meaningof{-} : TS \to ST
\end{eqnarray*}

\begin{eqnarray*}
  \\
  L : TS \to ST
\end{eqnarray*}

\begin{eqnarray*}
  \\
  P \models E \iff P \in \meaningof{E}
\end{eqnarray*}

\begin{eqnarray*}
  P \approx_{L} Q \iff \forall E \in L. P \models E \iff Q \models E
\end{eqnarray*}

\begin{eqnarray*}
  P \approx_{K} Q
\end{eqnarray*}

\begin{eqnarray*}
  P \approx Q
\end{eqnarray*}

$\approx_{K} = \approx = \approx_{L}$

\subsubsection{Contextual duality}

Note that contexts extend the quotation operation to a family of
operations from processes to names. Given a context, $M$, we can
define a \emph{nominal context}, $\quotep{M}$ by $\quotep{M}[P] :=
\quotep{M[P]}$. To foreshadow what is to come we observe that these
operations enjoy a duality with processes very much like the duality
between vectors and maps from vectors to scalars.

Further, because the calculus is essentially higher-order, we have a
correspondence between contexts and processes. More specifically,
given a name $x$ and a context $M$ we can construct $M^{*}_{x}$ such
that 

\begin{mathpar}
  M^{*}_{x} | \lift{x}{P} \red M[P]
\end{mathpar}

namely,

\begin{mathpar}
  M^{*}_{x} := x?(u).M[\dropn{u}]
\end{mathpar}

The dependence of $M^{*}_{x}$ on a name makes it an abstraction, 

\begin{mathpar}
  M^{*} := (x)x?(u).M[\dropn{u}]
\end{mathpar}

\subsection{Additional notation}

It will sometimes be convenient to denote the process a name
quotes. We already have the notation $x = \quotep{P}$, but it will be
convenient to introduce an alternate notation, $\procn{x}$, when we
want to emphasize the connection to the use of the name. Note that, by
virtue of name equivalence, $\quotep{\procn{x}} \nameeq x$; so, the
notation is consistent with previous definitions.

Further, because names have structure it is possible to effect
substitutions on the basis of that structure. This means we need to
upgrade our notation for substitutions, which we accomplish by
adapting comprehension notation. Thus,

\begin{mathpar}
  P\{ y / x : x \in S \}
\end{mathpar}

is interpreted to mean the process derived from P by replacing (in a
capture-avoiding manner) each occurrence of $x$ in $S$ by $y$. For example,

\begin{mathpar}
  P\{ \quotep{\procn{x}|\procn{x}} / x : x \in \freenames{P} \}
\end{mathpar}

will replace each (occurrence) of a free name $x$ in $P$ by
$\quotep{\procn{x}|\procn{x}}$.

Also, we will avail ourselves of the notation $x^{L}$ and $x^{R}$ to
denote injections of a name into disjoint copies of the name
space. There are numerous ways to accomplish this. One example can be
found in \cite{MeredithR05}. This notation overloads to vectors of
names: $\vec{x}^{\pi} := (x_{i}^{\pi} \; : \; 0 \leq i < |\vec{x}| )$ where $\pi \in \{L,R\}$.

We also use $P^{\Box} := P|\Box$.

In \cite{MeredithR05} an interpretation of the new operator is
given. It turns out that there are several possible interpretations
all enjoying the requisite algebraic properties of the operator (see
\cite{milner91polyadicpi}). We will therefore make liberal use of
$(\nu\; \vec{x})P$.

% subsection the_syntax_and_semantics_of_the_notation_system (end)   

\input{qm2pi.qmops} 

\input{qm2pi.sterngerlach} 

\input{qm2pi.metric} 

% section concurrent_process_calculi (end)

%\input{qm2pi.proofsketch}

% section proof sketch (end)

%\input{qm2pi.slviaknots} 

% section spatial logic via knots (end)

\input{qm2pi.conclusion}

% section conclusion (end)

%\input{qm2pi.dtcodes} 

% section wiring algorithm (end)

\input{qm2pi.ack} 

% section acknowledgments (end)

\newpage


\bibliographystyle{plain}   
\bibliography{../../biblios/main.bib}

\input{qm2pi.rhodetails}

\end{document}



% section proof sketch (end)

%\section{Unlikely characters: spatial logic for
  knots}\label{sub:characteristic_formulae} % (fold)

Associated to the mobile process calculi are a family of logics known
as the Hennessy-Milner logics. These logics typically enjoy a
semantics interpreting formulae as sets of processes that when
factored through the encoding outlined above allows an identification
of classes of knots with logical formulae. In the context of this
encoding the sub-family known as the spatial logics \cite{CairesC03}
\cite{CairesC04} \cite{Caires04} are of particular interest providing
several important features for expressing and reasoning about
properties (i.e. classes) of knots. We hint here at how this may be done.

%\begin{description}
%\item [structural connectives] 
\subsubsection{Structural connectives} The spatial logics enjoy
structural connectives corresponding, at the logical level, to the
parallel composition ($P | Q$) and new name ($(\nu \; x)P$)
connectives for processes. As illustrated in the examples below, these
connectives are extremely expressive given the shape of our encoding.
%\item [decideable satisfaction]

\subsubsection{Decideable satisfaction}
In \cite{Caires04} the satisfaction relation is shown to be decideable
for a rich class of processes. It further turns out that the image of
the our encoding is a proper subset of that class. This result
provides the basis for an algorithm by which to search for knots
enjoying a given property.
%\item [characteristic formulae]

\subsubsection{Characteristic formulae}
In the same paper \cite{Caires04} , Caires presents a means of calculating
characteristic formulae, selecting equivalence classes of processes
up to a pre--specified depth limit on the support set of names. Composed with our
encoding, this characteristic formula can be used to select
characteristic formulae for knots.
%\end{description}

\subsubsection{Spatial logic formulae}

The grammar below (segmented for comprehension) summarizes the syntax
of spatial logic formulae. We employ illustrative examples in the
sequel to provide an intuitive understanding of their meaning
referring the reader to \cite{Caires04} for a more detailed explication
of the semantics.

\begin{mathpar}
  \inferrule* [lab=boolean] {} {{A,B} \bc T \;|\; \neg A \;|\; A \wedge B \;|\; \eta = \eta'}
  \and
  \inferrule* [lab=spatial] {} {|\; \pzero \;|\; A | B \;|\; x \text{\textregistered} A \;|\; \forall x . A \;|\;  H x . A}
  \and
  \inferrule* [lab=behavioral] {} {|\; \alpha . A}
  \and 
  \inferrule* [lab=recursion] {} {|\; X(\vec{u}) \;|\; \mu X(\vec{u}) . A}
  \and
  \inferrule* [lab=action] {} {\alpha \bc \langle x?(\vec{y}) \rangle \;|\; \langle x!(\vec{y}) \rangle \;|\; \langle \tau \rangle}
  \and 
  \inferrule* [lab=name] {} {\eta \bc x \;|\; \tau}
\end{mathpar} 

% subsection characteristic_formulae (end)   	 

\subsection{Example formulae}\label{sub:example_formulae_} % (fold)

\subsubsection{Crossing as formula.}
% 
% \begin{align*}
%   \frac{d}{dx} \sin x &= \cos x 
%   & \frac{d}{dx} e^x &= e^x \\
%   \frac{d}{dx} \cos x &= - \sin x 
%   & \frac{d}{dx} \log x &= \frac{1}{x} \\
% \end{align*} 

\begin{align*}
 \mu C(x_{0},x_{1},y_{0},y_{1},u).&(\langle x_{0}?(z) \rangle(\langle u! \rangle\langle y_{1}!z \rangle C(x_{0},x_{1},y_{0},y_{1},u)) & \\
  & \wedge \langle y_{1}?(z) \rangle (\langle u! \rangle \langle x_{0}!z \rangle C(x_{0},x_{1},y_{0},y_{1},u)) & \\
  & \wedge \langle x_{1}?(z) \rangle (\langle u? \rangle \langle y_{0}!z \rangle C(x_{0},x_{1},y_{0},y_{1},u)) & \\
  & \wedge \langle y_{0}?(z) \rangle (\langle u? \rangle \langle x_{1}!z \rangle C(x_{0},x_{1},y_{0},y_{1},u))) &
\end{align*}

The lexicographical similarity between the shape of this formulae and
the shape of definition of the process representing a crossing reveals
the intuitive meaning of this formulae. It describes the capabilities
of a process that has the right to represent a crossing. For example
it picks out processes that may perform an input on the port $x_0$ in
its initial menu of capabilities. What differentiates the formula
from the process, however, is that the crossing process is the
smallest candidate to satisfy the formula. Infinitely many other
processes -- with internal behavior hidden behind this interface, so
to speak -- also satisfy this formula. Even this simple formula,
then, can be seen to open a new view onto knots, providing a
computational interpretation of \emph{virtual} knots.

Note that this formula is derived by hand. A similar formula can be
derived by employing Caires' calculation of characteristic formula
\cite{Caires04} to the process representing a crossing. In light of
this discussion, we let
$\meaningof{C}_{\phi}(x0,x1,y0,y1,u)$ denote a formula specifying the
dynamics we wish to capture of a crossing. To guarantee we preserve
the shape of the interface and minimal semantics we demand that
$\meaningof{C}_{\phi}(x0,x1,y0,y1,u) \Rightarrow
\textbf{C}(x0,x1,y0,y1,u)$ where $\textbf{C}(x0,x1,y0,y1,u)$ denotes
the formula above.
                            
\subsubsection{Crossing number constraints.}
The moral content of the context lemma (Lemma \ref{context}) is that the notion of
``locality'' in the Reidemeister moves is effectively captured by the
parallel composition operator of the process calculus. This intuition
extends through the logic. Given a formula,
$\meaningof{C}_{\phi}(x0,x1,y0,y1,u)$, we can use the structural
connectives to specify constraints on crossing numbers, such as at
least $n$ crossings, or exactly $n$ crossings.
\begin{mathpar}
  \inferrule* [lab=at-least-n] {} { K^{\geq n}_{\phi}(\vec{xs},\vec{ys}) := \Pi_{i=0}^{n-1} Hu . \meaningof{C}_{\phi}(xs_i,ys_i,u) | T }
  \and 
  \inferrule* [lab=exactly-n] {} { K^{= n}_{\phi}(\vec{xs},\vec{ys}) := \Pi_{i=0}^{n-1} Hu . \meaningof{C}_{\phi}(xs_i,ys_i,u) | \neg (\forall x_0,y_0,x_1,y_1,u . \meaningof{C}_{\phi}(x_0,y_0,x_1,y_1,u) | T) }
\end{mathpar}

To round out this section, recall that the encoding of an $n$-crossing
knot decomposes into a parallel composition of $n$ \emph{copies} of a
crossing process together with a wiring harness. To specify different
knot classes with the same crossing number amounts to specifying
logical constraints on the wiring harness. In the interest of space,
we defer examples to a forthcoming paper. Suffice it to say that both
the conditions ``alternating knot'' and ``contains the tangle
corresponding to 5/3'' are expressible. For example, it is possible to
calculate the characteristic formula of a process corresponding to the
tangle 5/3 and conjoin it into the classifying formula via the
composition connective of the logic.

Finally, we wish to observe that it is entirely within reason to
contemplate a more domain-specific version of spatial logic tailored
to the shape of processes in the image of the encoding. Such a
domain-specific logic would have a better claim to the title formal
language of knot properties.

% subsection example_formulae_ (end)

% section knots_as_processes (end) 

% section spatial logic via knots (end)

\section{Conclusions and future work}

\paragraph{Testing physical space}
You, gentle reader, may wonder why of all the theorems to be proved
given this set up we pick the one above. In some sense it's hardly
central to quantum mechanics. We see it as central in the sense that
it firmly establishes a notion of physical space arising from a notion
of the equivalence of behavior. Relating bisimulation to a metric is a
big step forward, but one is faced with interpreting the relationship
of that metric space to something more physical. Quantum mechanical
notions of ``physical'' space are still far from intuitive, but by
relating this idea of distance as testing to calculations that predict
physical circumstances we are making a not insignificant step forward
toward an understanding of the physical space we inhabit as
essentially dynamic.

\paragraph{Effectivity and simulation}
One of the observations we have yet to make is that the entire program
spelled out here is effective. We have built various interpreters for
the reflective calculus at work in this interpretation. In principle,
then, we can simulate quantum mechanics on a computer. The place where
the simulation may lose fidelity is the infinitely branching summation
for the annihilator.

In this connection i also want to point out that the evaluation style
calculation of the inner product puts the non-determinism of the
summation right at the heart of measurement. This suggests that
Milner's original reduction-based formulation of the dynamics of his
calculi in terms of sums was not just notationally suggestive of a
notion of measure-and-continue but captured some significant part of
the physics.

\paragraph{Quantum continuations}
In light of this last observation i want to point out that the
predominant account of quantum mechanics is missing a key aspect of a
truly compositional story of the physical situation. In a real lab,
when a measurement is made the observation can be made to feed into
another device that then makes another measurement conditioned on the
results of the first. This means that after the superposition was
collapsed the entire experimental set up remained in
superposition. While QM offers a means of writing this down it doesn't
quite line up well with the well-trodden formulation of computation
and continuation that we see so succinctly expressed in Milner's
calculi. This suggests that there might be advantages to this account
of dynamics waiting to be explored.

\paragraph{Quantum logic}
In this connection, we also note that by virtue of having the
Hennessy-Milner construction, we can pull the construction through the
interpretation of QM. This gives us a natural candidate for a quantum
logic that enjoys an extremely tight connection with it's domain of
interpretation, making the construction much less ad hoc (rather it is
the image of functor!).

\paragraph{Quantum probabiity}
i have questions about the basis of the interpretation of inner
product as probability amplitude. In particular, using which
axiomatization of probability theory does the notion of probability
amplitude earn the right to be so dubbed? In other words, where is the
proof that the operation for calculating a probability amplitude (and
then squaring) satisfies the axioms of what it means to calculate a
probability? Even if such a proof exists (i have yet to find it in the
literature), i wonder if it might not be possible to turn things on
their heads. Can we view the calculation of the probability amplitude
as an axiomatization of probability? If so, then the definition we
give for calculating probability amplitude may provide the basis for
an \emph{effective} theory of probability.

\paragraph{Quantum vs ``biological'' information}
Finally, i want to conclude with a more philosophical observation. At
a recent workshop in which QM was a predominant topic i noticed
something about quantum information. The speaker was giving a riveting
discussion of axiomatic QM and showing how properties of ``no
cloning'' and ``no deleting'' emerged as consequences of the
axiomatization. Theorems of this form are necessary to give us a sense
of confidence that our axioms characterize the physical theory. What
struck me, though, was that if quantum information is neither erasable
nor replicable it is markedly different from \emph{life}. Two of the
things we know about life is that

\begin{itemize}
  \item it ends;
  \item to gain some measure of persistence, to transcend it's
    finitude it is imminently copyable.
\end{itemize}

Both of these qualities are summarized succinctly in the aphorism: all
flesh is grass. For me these two kinds of ``information'' -- call them
quantum and biological -- are end points on a spectrum of strategies
for persistence. At one end, we have those curious entities that enjoy
uniqueness and permanence; at the other, we have those who in the face
of a certain end and an uncertain present make a go of passing
something on. To me one of the more remarkable aspects of the latter
strategy is that in the presence of noise (and certain features of
copying) we get a kind of dynamism, a chance for improvement against a
given persistent condition.

% subsection other_calculi_other_bisimulations_and_geometry_as_behavior (end)




% section conclusion (end)

%\documentclass[12pt]{llncs}
%\documentclass{jktr}

\usepackage[pdftex]{hyperref}                   
\usepackage {listings}
\usepackage {mathpartir}
\usepackage{bcprules}
%\usepackage{listings}
                       
\usepackage{graphicx} 
%\usepackage[margins=2.5cm,nohead,nofoot]{geometry}
%\usepackage{geometry}
\usepackage{amsfonts}
\usepackage{amstext}
\usepackage{latexsym}
\usepackage{amssymb}
\usepackage{color}


%\include{myPreamble}
\include{qm2pi.local} 

%\ifpdf
%\usepackage[pdftex]{graphicx}
%\else
%\usepackage{graphicx}
%\fi

 % \ifpdf
%  \usepackage{pdfsync}
%  \if


%\title{Brief Article}
%\author{David F. Snyder}
%\author{L.G. Meredith}

%\address{Dept. of Math., Texas State University--San Marcos, San Marcos, TX 78666}
       
\pagestyle{empty}


\begin{document}

\lstset{language=[Objective]Caml,frame=shadowbox}

\input{qm2pi.front}

% section front matter (end)

\input{qm2pi.intro} 
 
% section introduction (end)

% \input{qm2pi.knotations} 

% section notation (end)

\input{qm2pi.process.calculi} 

% section concurrent_process_calculi_and_spatial_logics_ (end)
    
%\input{qm2pi.knots2pi} 

%\input{qm2pi.trefoil} 

%\input{qm2pi.mainthm} 

% subsection basic_interpretation (end)

%\input{qm2pi.rho.presentation} 
\subsection{The syntax and semantics of the notation system}\label{sub:the_syntax_and_semantics_of_the_notation_system} % (fold)

We now summarize a technical presentation of the calculus that
embodies our theory of dynamics. The typical presentation of such a
calculus follows the style of giving generators and relations on
them. The grammar, below, describing term constructors, freely
generates the set of processes, $\Proc$. This set is then quotiented
by a relation known as structural congruence and it is over this set
that the notion of dynamics is expressed. This presentation is
essentially that of \cite{MeredithR05} with the addition of
polyadicity and summation. For readability we have relegated some of
the technical subtleties to an appendix.

\subsubsection{Process grammar}\label{subsub:process_grammar}

\begin{mathpar}
  \inferrule* [lab=synchronization] {} {{M} \bc \pzero \;|\; x?F \;|\; x!C }
  \and
  \inferrule* [lab=abstraction] {} {{F} \bc (x)P}
  \and
  \inferrule* [lab=concretion] {} {{C} \bc \langle Q \rangle}
  \and
  \inferrule* [lab=process] {} {{P,Q} \bc M \;| \;P|Q \;|\; @{x}}
  \and
  \inferrule* [lab=name] {} {{x} \bc \quotep{P}}
\end{mathpar} 

Note that $\vec{x}$ (resp. $\vec{P}$) denotes a vector of names
(resp. processes) of length $|\vec{x}|$ (resp. $|\vec{P}|$). We adopt
the following useful abbreviations.

\begin{mathpar}
   x?(\vec{y}).P := x.(\vec{y})P \and  x\clift{\vec{P}} := x.\clift{\vec{P}}
   \and x!(y) := \lift{x}{\dropn{y}}
   \and \Pi_{i=0}^{n-1}P_i := P_0 | \ldots | P_{n-1}
\end{mathpar}

\subsubsection{Structural congruence}

\paragraph{Free and bound names and alpha-equivalence.} At the
core of structural equivalence is alpha-equivalence which identifies
process that are the same up to a change of variable. Formally, we
recognize the distinction between free and bound names. The free names
of a process, $\freenames{P}$, may be calculated recursively as
follows:

\begin{mathpar}
\freenames{\pzero} := \emptyset
  \and \\
  \freenames{x?(y).P} := \{ x \} \cup (\freenames{P} \setminus \{ y \})
  \and 
  \freenames{x!\langle P \rangle} := \{ x \} \cup \{ P \} 
  \and \\
  \freenames{P|Q} := \freenames{P} \cup \freenames{Q}
  \and \\
  \freenames{@{x}} := \{ x \}
\end{mathpar}

$\pi$
$\quotep{\pi}$

$\freenames{-} : \pi \to \mathcal{P}(\quotep{\pi})$

\begin{eqnarray*}
  \freenames{\pzero} & := & \emptyset \\
  \freenames{x?(y).P} & := & \{ x \} \cup (\freenames{P} \setminus \{ y \}) \\
  \freenames{x!\langle P \rangle} & := & \{ x \} \cup \{ P \} \\
  \freenames{P|Q} & := & \freenames{P} \cup \freenames{Q} \\
  \freenames{\dropn{x}} & := & \{ x \}
\end{eqnarray*}

The bound names of a process, $\boundnames{P}$, are those names occurring in $P$
that are not free. For example, in $x?(y).0$, the name $x$ is free, while $y$ is bound.

\begin{mathpar}
  \inferrule* [lab=monoidal-laws] {} { P|Q \equiv Q|P \and P|0 \equiv P \and P|(Q|R) \equiv (P|Q)|R }
\end{mathpar}

\begin{mathpar}
  \inferrule* [lab=alpha-equivalence] {} { (x)P \equiv (y)P\{y/x\} \and y \not\in \freenames{P} }
\end{mathpar}

\begin{definition}
Then two processes, $P,Q$, are alpha-equivalent if $P = Q\{\vec{y}/\vec{x}\}$ for
some $\vec{x} \in \boundnames{Q},\vec{y} \in \boundnames{P}$, where $Q\{\vec{y}/\vec{x}\}$
denotes the capture-avoiding substitution of $\vec{y}$ for $\vec{x}$ in $Q$.
\end{definition}

\begin{definition}
  The {\em structural congruence} \cite{SangiorgiWalker} , $\equiv$,
  between processes is the least congruence containing
  alpha-equivalence, satisfying the abelian monoid laws
  (associativity, commutativity and $\pzero$ as identity) for parallel
  composition $|$ and for summation $+$.
\end{definition}

\subsection{Name equivalence}

We take name equivalence, written $\nameeq$, to be the smallest
equivalence relation generated by the following rules.

\begin{mathpar}
\inferrule*[lab=Quote-drop]
{ }
{ \quotep{@{x}} \nameeq x }

\inferrule*[lab=Struct-equiv]
{ P \scong Q }
{ \quotep{P} \nameeq \quotep{Q} }
\end{mathpar}

The astute reader will have noticed that the mutual recursion of names
and processes imposes a mutual recursion on alpha-equivalence and
structural equivalence via name-equivalence. Fortunately, all of this
works out pleasantly and we may calculate in the natural way, free of
concern. The reader interested in the details is referred to the
appendix \ref{appendix:rho_details}.

\subsection{Substitution}

We use $\Proc$ for the set of processes, $\QProc$ for the set of
names, and $\id{\{}\vec{y} / \vec{x} \id{\}}$ to denote partial maps,
$s : \QProc \rightarrow \QProc$. A map, $s$ lifts, uniquely, to a map
on process terms, $\widehat{s} : \Proc \rightarrow \Proc$ by the
following equations.

\begin{mathpar}
  (0) \psubstp{Q}{P} := 0 \\
  (R \juxtap S) \psubstp{Q}{P}
  :=    
  (R)\psubstp{Q}{P} \juxtap (S) \psubstp{Q}{P} \\
  (x?(y).R) \psubstp{Q}{P}    
  :=    
  (x)\substp{Q}{P} (z)\concat( (R \psubstn{z}{y}) \psubstp{Q}{P} ) \\
  (\lift{x}{R}) \psubstp{Q}{P}  
  :=
  \lift{(x)\substp{Q}{P}}{ R \psubstp{Q}{P} } \\
%   (\dropn{x})  \psubstp{Q}{P}       
%   := 
%   \left\{ 
%     \begin{array}{ccc} 
%       \dropn{\quotep{Q}} & & x \nameeq \quotep{P} \\
%       \dropn{x} & & otherwise \\
%     \end{array}
%   \right. 
  (\dropn{x})  \psubstp{Q}{P}       
  := 
  \left\{ 
    \begin{array}{ccc} 
      Q & & x \nameeq \quotep{P} \\
      \dropn{x} & & otherwise \\
    \end{array}
  \right.
\end{mathpar}
 

where

\begin{eqnarray}
  (x)\id{\{} \lpquote Q \rpquote / \lpquote P \rpquote \id{\}}            = 
  \left\{ 
    \begin{array}{ccc}
      \lpquote Q \rpquote & & x \nameeq \lpquote P \rpquote \\
      x & & otherwise \\
    \end{array}
  \right. \nonumber
\end{eqnarray}

and $z$ is chosen distinct from $\quotep{P}$, $\quotep{Q}$, the free
names in $Q$, and all the names in $R$. Our $\alpha$-equivalence will
be built in the standard way from this substitution.

\begin{remark}\label{rem:no_self_referential_names}
  One consequence of these definitions is that $\forall P. \quotep{P}
  \not\in \freenames{P}$.
\end{remark}

\subsection{ Dynamic quote: an example }

Anticipating something of what's to come, consider applying the
substitution, $\widehat{\id{\{}u / z \id{\}}}$, to the following pair
of processes, $\lift{w}{y!(z)}$ and $w[ \lpquote y!(z) \rpquote ]$.

\begin{eqnarray}
	\lift{w}{y!(z)}\widehat{\id{\{}u / z \id{\}}}
		& = &
		\lift{w}{y!(u)} \nonumber\\
	w[ \lpquote y!(z) \rpquote ] \widehat{ \id{\{}u / z \id{\}} }
		& = &
		w[ \lpquote y!(z) \rpquote ] \nonumber
\end{eqnarray}

Because the body of the process between quotes is impervious to
substitution, we get radically different answers. In fact, by
examining the first process in an input context,
e.g. $x?(z).\lift{w}{y!(z)}$, we see that the process under the lift
operator may be shaped by prefixed inputs binding a name inside it. In
this sense, the lift operator will be seen as a way to dynamically
construct processes before reifying them as names.

Finally equipped with these standard features we can present the
dynamics of the calculus.

\subsubsection{Operational semantics} 

Finally, we introduce the computational dynamics. What marks these
algebras as distinct from other more traditionally studied algebraic
structures, e.g. vector spaces or polynomial rings, is the manner in
which dynamics is captured. In traditional structures, dynamics is typically
expressed through morphisms between such structures, as in linear maps
between vector spaces or morphisms between rings. In algebras
associated with the semantics of computation, the dynamics is
expressed as part of the algebraic structure itself, through a
reduction reduction relation typically denoted by $\red$. Below, we
give a recursive presentation of this relation for the calculus used
in the encoding.

$\red \subseteq \pi \times \pi$
$\red : \pi \to \mathcal{P}(\pi)$

\begin{mathpar}
  \inferrule* [lab=Comm] { \textsf{match}( x_{src}, x_{trgt} ) } { x_{trgt}?(y)P \; | \; x_{src}!\langle {Q} \rangle \red P\{\quotep{Q}/y}\} }
  \and \\
  \inferrule* [lab=Par] {{P} \red {P}'} {{{P} | {Q}} \red {{P}' | {Q}}}
  \and
  \inferrule* [lab=Equiv]{{{P} \scong {P}'} \andalso {{P}' \red {Q}'} \andalso {{Q}' \scong {Q}}}{{P} \red {Q}}
\end{mathpar}

\begin{eqnarray*}
  match_{\equiv} (\quotep{P},\quotep{Q}) & := & P \equiv Q \\
  match_{\dagger}(\quotep{P},\quotep{Q}) & := & \forall R. P|Q \red^{*} R => R \red^{*} 0 \\
  match_{K}(\quotep{P},\quotep{Q}) & := & K \mbox{ for some context } K
\end{eqnarray*}

$u?(x)P | u!\langle Q \rangle \red P\{\quotep{Q}/x\}$

%We write $\wred$ for $\red^*$, and $P\red$ if $\exists Q $ such that $ P \red Q$.
We write $P\red$ if $\exists Q $ such that $ P \red Q$ and $P\not\red$, otherwise.

\section{Replication}

As mentioned before, it is known that replication (and hence
recursion) can be implemented in a higher-order process algebra
\cite{SangiorgiWalker}. As our first example of calculation with the
machinery thus far presented we give the construction explicitly in
the {\rhoc}.

\begin{eqnarray}
	D_{x} & := & \prefix{x}{y}{(\binpar{\outputp{x}{y}}{@{y}})} \nonumber\\
	\bangp_{x}{P} & := & \binpar{{x}!\langle{\binpar{D_{x}}{P}}\rangle}{D_{x}} \nonumber
\end{eqnarray}

\begin{eqnarray}
	\bangp_{x}{P} & & \nonumber\\
	=
	& {x}!\langle{(\prefix{x}{y}{(\outputp{x}{y} | @{y})) | P}}\rangle 
	      | \prefix{x}{y}{(\outputp{x}{y} | @{y})} & \nonumber\\
	\red
	& (\outputp{x}{y} | @{y})\substn{\quotep{(\prefix{x}{y}{(@{y} | \outputp{x}{y})) | P}}}{y} & \nonumber\\
	=
	& \outputp{x}{\quotep{(\prefix{x}{y}{(\outputp{x}{y} | @{y})) | P}}}
	  | {(\prefix{x}{y}{(\outputp{x}{y} | @{y})) | P}} & \nonumber\\
	\red
	& \ldots & \nonumber\\
	\red^*
	& P | P | \ldots & \nonumber
\end{eqnarray}

Of course, this encoding, as an implementation, runs away, unfolding
$\bangp{P}$ eagerly. A lazier and more implementable replication
operator, restricted to input-guarded processes, may be obtained as follows.

\begin{eqnarray}
\bangp{\prefix{u}{v}{P}} 
	:= 
	\binpar{\lift{x}{\prefix{u}{v}{(\binpar{D(x)}{P})}}}{D(x)} \nonumber
\end{eqnarray}

\begin{remark}
  Note that the lazier definition still does not deal with summation
  or mixed summation (i.e. sums over input and output). The reader is
  invited to construct definitions of replication that deal with these
  features. 

  Further, the definitions are parameterized in a name, $x$. Can you,
  gentle reader, make a definition that eliminates this parameter and
  guarantees no accidental interaction between the replication
  machinery and the process being replicated -- i.e. no accidental
  sharing of names used by the process to get its work done and the
  name(s) used by the replication to effect copying. This latter
  revision of the definition of replication is crucial to obtaining
  the expected identity $!!P \sim !P$.
\end{remark}

\begin{remark}\label{rem:paradoxical_combinator}
  The reader familiar with the lambda calculus will have noticed the
  similarity between $D$ and the paradoxical combinator.

  [Ed. note: the existence of this seems to suggest we have to be more
  restrictive on the set of processes and names we admit if we are to
  support no-cloning.]
\end{remark}

\subsubsection{Bisimulation}

The computational dynamics gives rise to another kind of equivalence,
the equivalence of computational behavior. As previously mentioned
this is typically captured \emph{via} some form of bisimulation.

% The notion we use in this paper is weak barbed bisimulation
% \cite{milner91polyadicpi}.

The notion we use in this paper is derived from weak barbed
bisimulation \cite{milner91polyadicpi}. 

\begin{definition}
An \emph{observation relation}, $\downarrow_{\mathcal N}$, over a set
of names, $\mathcal N$, is the smallest relation satisfying the rules
below.

\infrule[Out-barb]{y \in {\mathcal N}, \; x \nameeq y}
		  {\outputp{x}{v} \downarrow_{\mathcal N} x}
\infrule[Par-barb]{\mbox{$P\downarrow_{\mathcal N} x$ or $Q\downarrow_{\mathcal N} x$}}
		  {\binpar{P}{Q} \downarrow_{\mathcal N} x}

We write $P \Downarrow_{\mathcal N} x$ if there is $Q$ such that 
$P \wred Q$ and $Q \downarrow_{\mathcal N} x$.
\end{definition}

\begin{definition}
%\label{def.bbisim}
An  ${\mathcal N}$-\emph{barbed bisimulation} over a set of names, ${\mathcal N}$, is a symmetric binary relation 
${\mathcal S}_{\mathcal N}$ between agents such that $P\rel{S}_{\mathcal N}Q$ implies:
\begin{enumerate}
\item If $P \red P'$ then $Q \wred Q'$ and $P'\rel{S}_{\mathcal N} Q'$.
\item If $P\downarrow_{\mathcal N} x$, then $Q\Downarrow_{\mathcal N} x$.
\end{enumerate}
$P$ is ${\mathcal N}$-barbed bisimilar to $Q$, written
$P \wbbisim_{\mathcal N} Q$, if $P \rel{S}_{\mathcal N} Q$ for some ${\mathcal N}$-barbed bisimulation ${\mathcal S}_{\mathcal N}$.
\end{definition}

$\mathcal{R} \subseteq \pi \times \pi$

$P \mathcal{R} Q => \forall P'. P \red P' \Rightarrow \exists Q'. Q \red Q', P' \mathcal{R} Q'$

$P \vdash x \Rightarrow Q \vdash x$

\begin{mathpar}
  \inferrule*[lab=Out-barb]{x \nameeq y}{{y}!\langle{Q}\rangle \vdash x}
  \and
  \inferrule*[lab=Par-barb]{\mbox{$P\vdash x$ or $Q\vdash x$}}{\binpar{P}{Q} \vdash x}
\end{mathpar}

\subsubsection{Contexts}

One of the principle advantages of computational calculi like the
$\pi$-calculus is a well-defined notion of context,
contextual-equivalence and a correlation between
contextual-equivalence and notions of bisimulation. The notion of
context allows the decomposition of a process into (sub-)process and
its syntactic environment, its context. Thus, a context may be
thought of as a process with a ``hole'' (written $\Box$) in it. The
application of a context $M$ to a process $P$, written $M[P]$, is
tantamount to filling the hole in $M$ with $P$. In this paper we do
not need the full weight of this theory, but do make use of the notion
of context in the proof the main theorem. 

\begin{mathpar}
  \inferrule* [lab=summation] {} {{M_{M},M_{N}} \bc \Box \;|\; x.M_{A} \;|\; M_{M}+M_{N}}
  \and
  \inferrule* [lab=agent] {} {{M_{A}} \bc (\vec{x})M_{P} \;| \; \clift{P_0,\ldots,M_{P},\ldots,P_N}}
  \and \\
  \inferrule* [lab=process] {} {{M_{P}} \bc M_{N} \;| \;P|M_{P} }
\end{mathpar} 

\begin{mathpar}
  \inferrule* [lab=sychronization] {} {M_{N} \bc \Box \;|\; x?M_{F} \;|\; x!M_{C}}
  \and
  \inferrule* [lab=abstraction] {} {{M_{F}} \bc (x)M_{P} }
  \and
  \inferrule* [lab=concretion] {} {{M_{C}} \bc \langle M_{P} \rangle }
  \and \\
  \inferrule* [lab=process] {} {{M_{P}} \bc M_{N} \;| \;P|M_{P} }
\end{mathpar}

\begin{definition}[contextual application] Given a context $M$, and
  process $P$, we define the \emph{contextual application}, $M[P] :=
  M\{P/\Box\}$. That is, the contextual application of M to P is the
  substitution of $P$ for $\Box$ in $M$.
\end{definition}

$\meaningof{-} : L \to \mathcal{P}(\pi)$

\begin{mathpar}
  \inferrule* [lab=collection] {} {\meaningof{true} = \pi, \and \meaningof{~E} = \pi \setminus \meaningof{E}, \and \meaningof{E_{1} \& E_{2}} = \meaningof{E_{1}} \cap \meaningof{E_{2}}}
\end{mathpar}

\begin{mathpar}
  \inferrule* [lab=structure] {} {\meaningof{0} = \{ P \in \pi | P \equiv 0 \}, \and \\ \meaningof{E_1 | E_2} = \{ P \in \pi | P \equiv P_{1} | P_{2}, P_{1} \in \meaningof{E_{1}}, P_{2} \in \meaningof{E_2}\} }
\end{mathpar}

\begin{mathpar}
 \inferrule* [lab=behavior] {} {\meaningof{\langle a?b \rangle E} = \{ P \in \pi | P \equiv Q | u?(y)P', \\ \and \\\\ \and \\ \;\;\; u \in \meaningof{a}, \forall z.P'\{z/y\} \in \meaningof{E\{z/b\}}\}, \and \\ \meaningof{a!E} = \{ P \in \pi | P \equiv Q | x!\langle P' \rangle, x \in \meaningof{a} P' \in \meaningof{E}\} }
\end{mathpar}

\begin{mathpar}
 \inferrule* [lab=nominal] {} {\meaningof{\quotep{E}} = \{ \quotep{P} \in \quotep{\pi} | P \in \meaningof{E} \}, \and \meaningof{\quotep{P}} = \{ \quotep{Q} \in \quotep{\pi} | P \equiv Q \} \and \\ \meaningof{@\quotep{E}} = \{ P \in \pi | P \equiv @x, x \in \meaningof{E} \}}
\end{mathpar}

\begin{eqnarray*}
  \\
  \meaningof{-} : TS \to ST
\end{eqnarray*}

\begin{eqnarray*}
  \\
  L : TS \to ST
\end{eqnarray*}

\begin{eqnarray*}
  \\
  P \models E \iff P \in \meaningof{E}
\end{eqnarray*}

\begin{eqnarray*}
  P \approx_{L} Q \iff \forall E \in L. P \models E \iff Q \models E
\end{eqnarray*}

\begin{eqnarray*}
  P \approx_{K} Q
\end{eqnarray*}

\begin{eqnarray*}
  P \approx Q
\end{eqnarray*}

$\approx_{K} = \approx = \approx_{L}$

\subsubsection{Contextual duality}

Note that contexts extend the quotation operation to a family of
operations from processes to names. Given a context, $M$, we can
define a \emph{nominal context}, $\quotep{M}$ by $\quotep{M}[P] :=
\quotep{M[P]}$. To foreshadow what is to come we observe that these
operations enjoy a duality with processes very much like the duality
between vectors and maps from vectors to scalars.

Further, because the calculus is essentially higher-order, we have a
correspondence between contexts and processes. More specifically,
given a name $x$ and a context $M$ we can construct $M^{*}_{x}$ such
that 

\begin{mathpar}
  M^{*}_{x} | \lift{x}{P} \red M[P]
\end{mathpar}

namely,

\begin{mathpar}
  M^{*}_{x} := x?(u).M[\dropn{u}]
\end{mathpar}

The dependence of $M^{*}_{x}$ on a name makes it an abstraction, 

\begin{mathpar}
  M^{*} := (x)x?(u).M[\dropn{u}]
\end{mathpar}

\subsection{Additional notation}

It will sometimes be convenient to denote the process a name
quotes. We already have the notation $x = \quotep{P}$, but it will be
convenient to introduce an alternate notation, $\procn{x}$, when we
want to emphasize the connection to the use of the name. Note that, by
virtue of name equivalence, $\quotep{\procn{x}} \nameeq x$; so, the
notation is consistent with previous definitions.

Further, because names have structure it is possible to effect
substitutions on the basis of that structure. This means we need to
upgrade our notation for substitutions, which we accomplish by
adapting comprehension notation. Thus,

\begin{mathpar}
  P\{ y / x : x \in S \}
\end{mathpar}

is interpreted to mean the process derived from P by replacing (in a
capture-avoiding manner) each occurrence of $x$ in $S$ by $y$. For example,

\begin{mathpar}
  P\{ \quotep{\procn{x}|\procn{x}} / x : x \in \freenames{P} \}
\end{mathpar}

will replace each (occurrence) of a free name $x$ in $P$ by
$\quotep{\procn{x}|\procn{x}}$.

Also, we will avail ourselves of the notation $x^{L}$ and $x^{R}$ to
denote injections of a name into disjoint copies of the name
space. There are numerous ways to accomplish this. One example can be
found in \cite{MeredithR05}. This notation overloads to vectors of
names: $\vec{x}^{\pi} := (x_{i}^{\pi} \; : \; 0 \leq i < |\vec{x}| )$ where $\pi \in \{L,R\}$.

We also use $P^{\Box} := P|\Box$.

In \cite{MeredithR05} an interpretation of the new operator is
given. It turns out that there are several possible interpretations
all enjoying the requisite algebraic properties of the operator (see
\cite{milner91polyadicpi}). We will therefore make liberal use of
$(\nu\; \vec{x})P$.

% subsection the_syntax_and_semantics_of_the_notation_system (end)   

\input{qm2pi.qmops} 

\input{qm2pi.sterngerlach} 

\input{qm2pi.metric} 

% section concurrent_process_calculi (end)

%\input{qm2pi.proofsketch}

% section proof sketch (end)

%\input{qm2pi.slviaknots} 

% section spatial logic via knots (end)

\input{qm2pi.conclusion}

% section conclusion (end)

%\input{qm2pi.dtcodes} 

% section wiring algorithm (end)

\input{qm2pi.ack} 

% section acknowledgments (end)

\newpage


\bibliographystyle{plain}   
\bibliography{../../biblios/main.bib}

\input{qm2pi.rhodetails}

\end{document}

 

% section wiring algorithm (end)

\documentclass[12pt]{llncs}
%\documentclass{jktr}

\usepackage[pdftex]{hyperref}                   
\usepackage {listings}
\usepackage {mathpartir}
\usepackage{bcprules}
%\usepackage{listings}
                       
\usepackage{graphicx} 
%\usepackage[margins=2.5cm,nohead,nofoot]{geometry}
%\usepackage{geometry}
\usepackage{amsfonts}
\usepackage{amstext}
\usepackage{latexsym}
\usepackage{amssymb}
\usepackage{color}


%\include{myPreamble}
\include{qm2pi.local} 

%\ifpdf
%\usepackage[pdftex]{graphicx}
%\else
%\usepackage{graphicx}
%\fi

 % \ifpdf
%  \usepackage{pdfsync}
%  \if


%\title{Brief Article}
%\author{David F. Snyder}
%\author{L.G. Meredith}

%\address{Dept. of Math., Texas State University--San Marcos, San Marcos, TX 78666}
       
\pagestyle{empty}


\begin{document}

\lstset{language=[Objective]Caml,frame=shadowbox}

\input{qm2pi.front}

% section front matter (end)

\input{qm2pi.intro} 
 
% section introduction (end)

% \input{qm2pi.knotations} 

% section notation (end)

\input{qm2pi.process.calculi} 

% section concurrent_process_calculi_and_spatial_logics_ (end)
    
%\input{qm2pi.knots2pi} 

%\input{qm2pi.trefoil} 

%\input{qm2pi.mainthm} 

% subsection basic_interpretation (end)

%\input{qm2pi.rho.presentation} 
\subsection{The syntax and semantics of the notation system}\label{sub:the_syntax_and_semantics_of_the_notation_system} % (fold)

We now summarize a technical presentation of the calculus that
embodies our theory of dynamics. The typical presentation of such a
calculus follows the style of giving generators and relations on
them. The grammar, below, describing term constructors, freely
generates the set of processes, $\Proc$. This set is then quotiented
by a relation known as structural congruence and it is over this set
that the notion of dynamics is expressed. This presentation is
essentially that of \cite{MeredithR05} with the addition of
polyadicity and summation. For readability we have relegated some of
the technical subtleties to an appendix.

\subsubsection{Process grammar}\label{subsub:process_grammar}

\begin{mathpar}
  \inferrule* [lab=synchronization] {} {{M} \bc \pzero \;|\; x?F \;|\; x!C }
  \and
  \inferrule* [lab=abstraction] {} {{F} \bc (x)P}
  \and
  \inferrule* [lab=concretion] {} {{C} \bc \langle Q \rangle}
  \and
  \inferrule* [lab=process] {} {{P,Q} \bc M \;| \;P|Q \;|\; @{x}}
  \and
  \inferrule* [lab=name] {} {{x} \bc \quotep{P}}
\end{mathpar} 

Note that $\vec{x}$ (resp. $\vec{P}$) denotes a vector of names
(resp. processes) of length $|\vec{x}|$ (resp. $|\vec{P}|$). We adopt
the following useful abbreviations.

\begin{mathpar}
   x?(\vec{y}).P := x.(\vec{y})P \and  x\clift{\vec{P}} := x.\clift{\vec{P}}
   \and x!(y) := \lift{x}{\dropn{y}}
   \and \Pi_{i=0}^{n-1}P_i := P_0 | \ldots | P_{n-1}
\end{mathpar}

\subsubsection{Structural congruence}

\paragraph{Free and bound names and alpha-equivalence.} At the
core of structural equivalence is alpha-equivalence which identifies
process that are the same up to a change of variable. Formally, we
recognize the distinction between free and bound names. The free names
of a process, $\freenames{P}$, may be calculated recursively as
follows:

\begin{mathpar}
\freenames{\pzero} := \emptyset
  \and \\
  \freenames{x?(y).P} := \{ x \} \cup (\freenames{P} \setminus \{ y \})
  \and 
  \freenames{x!\langle P \rangle} := \{ x \} \cup \{ P \} 
  \and \\
  \freenames{P|Q} := \freenames{P} \cup \freenames{Q}
  \and \\
  \freenames{@{x}} := \{ x \}
\end{mathpar}

$\pi$
$\quotep{\pi}$

$\freenames{-} : \pi \to \mathcal{P}(\quotep{\pi})$

\begin{eqnarray*}
  \freenames{\pzero} & := & \emptyset \\
  \freenames{x?(y).P} & := & \{ x \} \cup (\freenames{P} \setminus \{ y \}) \\
  \freenames{x!\langle P \rangle} & := & \{ x \} \cup \{ P \} \\
  \freenames{P|Q} & := & \freenames{P} \cup \freenames{Q} \\
  \freenames{\dropn{x}} & := & \{ x \}
\end{eqnarray*}

The bound names of a process, $\boundnames{P}$, are those names occurring in $P$
that are not free. For example, in $x?(y).0$, the name $x$ is free, while $y$ is bound.

\begin{mathpar}
  \inferrule* [lab=monoidal-laws] {} { P|Q \equiv Q|P \and P|0 \equiv P \and P|(Q|R) \equiv (P|Q)|R }
\end{mathpar}

\begin{mathpar}
  \inferrule* [lab=alpha-equivalence] {} { (x)P \equiv (y)P\{y/x\} \and y \not\in \freenames{P} }
\end{mathpar}

\begin{definition}
Then two processes, $P,Q$, are alpha-equivalent if $P = Q\{\vec{y}/\vec{x}\}$ for
some $\vec{x} \in \boundnames{Q},\vec{y} \in \boundnames{P}$, where $Q\{\vec{y}/\vec{x}\}$
denotes the capture-avoiding substitution of $\vec{y}$ for $\vec{x}$ in $Q$.
\end{definition}

\begin{definition}
  The {\em structural congruence} \cite{SangiorgiWalker} , $\equiv$,
  between processes is the least congruence containing
  alpha-equivalence, satisfying the abelian monoid laws
  (associativity, commutativity and $\pzero$ as identity) for parallel
  composition $|$ and for summation $+$.
\end{definition}

\subsection{Name equivalence}

We take name equivalence, written $\nameeq$, to be the smallest
equivalence relation generated by the following rules.

\begin{mathpar}
\inferrule*[lab=Quote-drop]
{ }
{ \quotep{@{x}} \nameeq x }

\inferrule*[lab=Struct-equiv]
{ P \scong Q }
{ \quotep{P} \nameeq \quotep{Q} }
\end{mathpar}

The astute reader will have noticed that the mutual recursion of names
and processes imposes a mutual recursion on alpha-equivalence and
structural equivalence via name-equivalence. Fortunately, all of this
works out pleasantly and we may calculate in the natural way, free of
concern. The reader interested in the details is referred to the
appendix \ref{appendix:rho_details}.

\subsection{Substitution}

We use $\Proc$ for the set of processes, $\QProc$ for the set of
names, and $\id{\{}\vec{y} / \vec{x} \id{\}}$ to denote partial maps,
$s : \QProc \rightarrow \QProc$. A map, $s$ lifts, uniquely, to a map
on process terms, $\widehat{s} : \Proc \rightarrow \Proc$ by the
following equations.

\begin{mathpar}
  (0) \psubstp{Q}{P} := 0 \\
  (R \juxtap S) \psubstp{Q}{P}
  :=    
  (R)\psubstp{Q}{P} \juxtap (S) \psubstp{Q}{P} \\
  (x?(y).R) \psubstp{Q}{P}    
  :=    
  (x)\substp{Q}{P} (z)\concat( (R \psubstn{z}{y}) \psubstp{Q}{P} ) \\
  (\lift{x}{R}) \psubstp{Q}{P}  
  :=
  \lift{(x)\substp{Q}{P}}{ R \psubstp{Q}{P} } \\
%   (\dropn{x})  \psubstp{Q}{P}       
%   := 
%   \left\{ 
%     \begin{array}{ccc} 
%       \dropn{\quotep{Q}} & & x \nameeq \quotep{P} \\
%       \dropn{x} & & otherwise \\
%     \end{array}
%   \right. 
  (\dropn{x})  \psubstp{Q}{P}       
  := 
  \left\{ 
    \begin{array}{ccc} 
      Q & & x \nameeq \quotep{P} \\
      \dropn{x} & & otherwise \\
    \end{array}
  \right.
\end{mathpar}
 

where

\begin{eqnarray}
  (x)\id{\{} \lpquote Q \rpquote / \lpquote P \rpquote \id{\}}            = 
  \left\{ 
    \begin{array}{ccc}
      \lpquote Q \rpquote & & x \nameeq \lpquote P \rpquote \\
      x & & otherwise \\
    \end{array}
  \right. \nonumber
\end{eqnarray}

and $z$ is chosen distinct from $\quotep{P}$, $\quotep{Q}$, the free
names in $Q$, and all the names in $R$. Our $\alpha$-equivalence will
be built in the standard way from this substitution.

\begin{remark}\label{rem:no_self_referential_names}
  One consequence of these definitions is that $\forall P. \quotep{P}
  \not\in \freenames{P}$.
\end{remark}

\subsection{ Dynamic quote: an example }

Anticipating something of what's to come, consider applying the
substitution, $\widehat{\id{\{}u / z \id{\}}}$, to the following pair
of processes, $\lift{w}{y!(z)}$ and $w[ \lpquote y!(z) \rpquote ]$.

\begin{eqnarray}
	\lift{w}{y!(z)}\widehat{\id{\{}u / z \id{\}}}
		& = &
		\lift{w}{y!(u)} \nonumber\\
	w[ \lpquote y!(z) \rpquote ] \widehat{ \id{\{}u / z \id{\}} }
		& = &
		w[ \lpquote y!(z) \rpquote ] \nonumber
\end{eqnarray}

Because the body of the process between quotes is impervious to
substitution, we get radically different answers. In fact, by
examining the first process in an input context,
e.g. $x?(z).\lift{w}{y!(z)}$, we see that the process under the lift
operator may be shaped by prefixed inputs binding a name inside it. In
this sense, the lift operator will be seen as a way to dynamically
construct processes before reifying them as names.

Finally equipped with these standard features we can present the
dynamics of the calculus.

\subsubsection{Operational semantics} 

Finally, we introduce the computational dynamics. What marks these
algebras as distinct from other more traditionally studied algebraic
structures, e.g. vector spaces or polynomial rings, is the manner in
which dynamics is captured. In traditional structures, dynamics is typically
expressed through morphisms between such structures, as in linear maps
between vector spaces or morphisms between rings. In algebras
associated with the semantics of computation, the dynamics is
expressed as part of the algebraic structure itself, through a
reduction reduction relation typically denoted by $\red$. Below, we
give a recursive presentation of this relation for the calculus used
in the encoding.

$\red \subseteq \pi \times \pi$
$\red : \pi \to \mathcal{P}(\pi)$

\begin{mathpar}
  \inferrule* [lab=Comm] { \textsf{match}( x_{src}, x_{trgt} ) } { x_{trgt}?(y)P \; | \; x_{src}!\langle {Q} \rangle \red P\{\quotep{Q}/y}\} }
  \and \\
  \inferrule* [lab=Par] {{P} \red {P}'} {{{P} | {Q}} \red {{P}' | {Q}}}
  \and
  \inferrule* [lab=Equiv]{{{P} \scong {P}'} \andalso {{P}' \red {Q}'} \andalso {{Q}' \scong {Q}}}{{P} \red {Q}}
\end{mathpar}

\begin{eqnarray*}
  match_{\equiv} (\quotep{P},\quotep{Q}) & := & P \equiv Q \\
  match_{\dagger}(\quotep{P},\quotep{Q}) & := & \forall R. P|Q \red^{*} R => R \red^{*} 0 \\
  match_{K}(\quotep{P},\quotep{Q}) & := & K \mbox{ for some context } K
\end{eqnarray*}

$u?(x)P | u!\langle Q \rangle \red P\{\quotep{Q}/x\}$

%We write $\wred$ for $\red^*$, and $P\red$ if $\exists Q $ such that $ P \red Q$.
We write $P\red$ if $\exists Q $ such that $ P \red Q$ and $P\not\red$, otherwise.

\section{Replication}

As mentioned before, it is known that replication (and hence
recursion) can be implemented in a higher-order process algebra
\cite{SangiorgiWalker}. As our first example of calculation with the
machinery thus far presented we give the construction explicitly in
the {\rhoc}.

\begin{eqnarray}
	D_{x} & := & \prefix{x}{y}{(\binpar{\outputp{x}{y}}{@{y}})} \nonumber\\
	\bangp_{x}{P} & := & \binpar{{x}!\langle{\binpar{D_{x}}{P}}\rangle}{D_{x}} \nonumber
\end{eqnarray}

\begin{eqnarray}
	\bangp_{x}{P} & & \nonumber\\
	=
	& {x}!\langle{(\prefix{x}{y}{(\outputp{x}{y} | @{y})) | P}}\rangle 
	      | \prefix{x}{y}{(\outputp{x}{y} | @{y})} & \nonumber\\
	\red
	& (\outputp{x}{y} | @{y})\substn{\quotep{(\prefix{x}{y}{(@{y} | \outputp{x}{y})) | P}}}{y} & \nonumber\\
	=
	& \outputp{x}{\quotep{(\prefix{x}{y}{(\outputp{x}{y} | @{y})) | P}}}
	  | {(\prefix{x}{y}{(\outputp{x}{y} | @{y})) | P}} & \nonumber\\
	\red
	& \ldots & \nonumber\\
	\red^*
	& P | P | \ldots & \nonumber
\end{eqnarray}

Of course, this encoding, as an implementation, runs away, unfolding
$\bangp{P}$ eagerly. A lazier and more implementable replication
operator, restricted to input-guarded processes, may be obtained as follows.

\begin{eqnarray}
\bangp{\prefix{u}{v}{P}} 
	:= 
	\binpar{\lift{x}{\prefix{u}{v}{(\binpar{D(x)}{P})}}}{D(x)} \nonumber
\end{eqnarray}

\begin{remark}
  Note that the lazier definition still does not deal with summation
  or mixed summation (i.e. sums over input and output). The reader is
  invited to construct definitions of replication that deal with these
  features. 

  Further, the definitions are parameterized in a name, $x$. Can you,
  gentle reader, make a definition that eliminates this parameter and
  guarantees no accidental interaction between the replication
  machinery and the process being replicated -- i.e. no accidental
  sharing of names used by the process to get its work done and the
  name(s) used by the replication to effect copying. This latter
  revision of the definition of replication is crucial to obtaining
  the expected identity $!!P \sim !P$.
\end{remark}

\begin{remark}\label{rem:paradoxical_combinator}
  The reader familiar with the lambda calculus will have noticed the
  similarity between $D$ and the paradoxical combinator.

  [Ed. note: the existence of this seems to suggest we have to be more
  restrictive on the set of processes and names we admit if we are to
  support no-cloning.]
\end{remark}

\subsubsection{Bisimulation}

The computational dynamics gives rise to another kind of equivalence,
the equivalence of computational behavior. As previously mentioned
this is typically captured \emph{via} some form of bisimulation.

% The notion we use in this paper is weak barbed bisimulation
% \cite{milner91polyadicpi}.

The notion we use in this paper is derived from weak barbed
bisimulation \cite{milner91polyadicpi}. 

\begin{definition}
An \emph{observation relation}, $\downarrow_{\mathcal N}$, over a set
of names, $\mathcal N$, is the smallest relation satisfying the rules
below.

\infrule[Out-barb]{y \in {\mathcal N}, \; x \nameeq y}
		  {\outputp{x}{v} \downarrow_{\mathcal N} x}
\infrule[Par-barb]{\mbox{$P\downarrow_{\mathcal N} x$ or $Q\downarrow_{\mathcal N} x$}}
		  {\binpar{P}{Q} \downarrow_{\mathcal N} x}

We write $P \Downarrow_{\mathcal N} x$ if there is $Q$ such that 
$P \wred Q$ and $Q \downarrow_{\mathcal N} x$.
\end{definition}

\begin{definition}
%\label{def.bbisim}
An  ${\mathcal N}$-\emph{barbed bisimulation} over a set of names, ${\mathcal N}$, is a symmetric binary relation 
${\mathcal S}_{\mathcal N}$ between agents such that $P\rel{S}_{\mathcal N}Q$ implies:
\begin{enumerate}
\item If $P \red P'$ then $Q \wred Q'$ and $P'\rel{S}_{\mathcal N} Q'$.
\item If $P\downarrow_{\mathcal N} x$, then $Q\Downarrow_{\mathcal N} x$.
\end{enumerate}
$P$ is ${\mathcal N}$-barbed bisimilar to $Q$, written
$P \wbbisim_{\mathcal N} Q$, if $P \rel{S}_{\mathcal N} Q$ for some ${\mathcal N}$-barbed bisimulation ${\mathcal S}_{\mathcal N}$.
\end{definition}

$\mathcal{R} \subseteq \pi \times \pi$

$P \mathcal{R} Q => \forall P'. P \red P' \Rightarrow \exists Q'. Q \red Q', P' \mathcal{R} Q'$

$P \vdash x \Rightarrow Q \vdash x$

\begin{mathpar}
  \inferrule*[lab=Out-barb]{x \nameeq y}{{y}!\langle{Q}\rangle \vdash x}
  \and
  \inferrule*[lab=Par-barb]{\mbox{$P\vdash x$ or $Q\vdash x$}}{\binpar{P}{Q} \vdash x}
\end{mathpar}

\subsubsection{Contexts}

One of the principle advantages of computational calculi like the
$\pi$-calculus is a well-defined notion of context,
contextual-equivalence and a correlation between
contextual-equivalence and notions of bisimulation. The notion of
context allows the decomposition of a process into (sub-)process and
its syntactic environment, its context. Thus, a context may be
thought of as a process with a ``hole'' (written $\Box$) in it. The
application of a context $M$ to a process $P$, written $M[P]$, is
tantamount to filling the hole in $M$ with $P$. In this paper we do
not need the full weight of this theory, but do make use of the notion
of context in the proof the main theorem. 

\begin{mathpar}
  \inferrule* [lab=summation] {} {{M_{M},M_{N}} \bc \Box \;|\; x.M_{A} \;|\; M_{M}+M_{N}}
  \and
  \inferrule* [lab=agent] {} {{M_{A}} \bc (\vec{x})M_{P} \;| \; \clift{P_0,\ldots,M_{P},\ldots,P_N}}
  \and \\
  \inferrule* [lab=process] {} {{M_{P}} \bc M_{N} \;| \;P|M_{P} }
\end{mathpar} 

\begin{mathpar}
  \inferrule* [lab=sychronization] {} {M_{N} \bc \Box \;|\; x?M_{F} \;|\; x!M_{C}}
  \and
  \inferrule* [lab=abstraction] {} {{M_{F}} \bc (x)M_{P} }
  \and
  \inferrule* [lab=concretion] {} {{M_{C}} \bc \langle M_{P} \rangle }
  \and \\
  \inferrule* [lab=process] {} {{M_{P}} \bc M_{N} \;| \;P|M_{P} }
\end{mathpar}

\begin{definition}[contextual application] Given a context $M$, and
  process $P$, we define the \emph{contextual application}, $M[P] :=
  M\{P/\Box\}$. That is, the contextual application of M to P is the
  substitution of $P$ for $\Box$ in $M$.
\end{definition}

$\meaningof{-} : L \to \mathcal{P}(\pi)$

\begin{mathpar}
  \inferrule* [lab=collection] {} {\meaningof{true} = \pi, \and \meaningof{~E} = \pi \setminus \meaningof{E}, \and \meaningof{E_{1} \& E_{2}} = \meaningof{E_{1}} \cap \meaningof{E_{2}}}
\end{mathpar}

\begin{mathpar}
  \inferrule* [lab=structure] {} {\meaningof{0} = \{ P \in \pi | P \equiv 0 \}, \and \\ \meaningof{E_1 | E_2} = \{ P \in \pi | P \equiv P_{1} | P_{2}, P_{1} \in \meaningof{E_{1}}, P_{2} \in \meaningof{E_2}\} }
\end{mathpar}

\begin{mathpar}
 \inferrule* [lab=behavior] {} {\meaningof{\langle a?b \rangle E} = \{ P \in \pi | P \equiv Q | u?(y)P', \\ \and \\\\ \and \\ \;\;\; u \in \meaningof{a}, \forall z.P'\{z/y\} \in \meaningof{E\{z/b\}}\}, \and \\ \meaningof{a!E} = \{ P \in \pi | P \equiv Q | x!\langle P' \rangle, x \in \meaningof{a} P' \in \meaningof{E}\} }
\end{mathpar}

\begin{mathpar}
 \inferrule* [lab=nominal] {} {\meaningof{\quotep{E}} = \{ \quotep{P} \in \quotep{\pi} | P \in \meaningof{E} \}, \and \meaningof{\quotep{P}} = \{ \quotep{Q} \in \quotep{\pi} | P \equiv Q \} \and \\ \meaningof{@\quotep{E}} = \{ P \in \pi | P \equiv @x, x \in \meaningof{E} \}}
\end{mathpar}

\begin{eqnarray*}
  \\
  \meaningof{-} : TS \to ST
\end{eqnarray*}

\begin{eqnarray*}
  \\
  L : TS \to ST
\end{eqnarray*}

\begin{eqnarray*}
  \\
  P \models E \iff P \in \meaningof{E}
\end{eqnarray*}

\begin{eqnarray*}
  P \approx_{L} Q \iff \forall E \in L. P \models E \iff Q \models E
\end{eqnarray*}

\begin{eqnarray*}
  P \approx_{K} Q
\end{eqnarray*}

\begin{eqnarray*}
  P \approx Q
\end{eqnarray*}

$\approx_{K} = \approx = \approx_{L}$

\subsubsection{Contextual duality}

Note that contexts extend the quotation operation to a family of
operations from processes to names. Given a context, $M$, we can
define a \emph{nominal context}, $\quotep{M}$ by $\quotep{M}[P] :=
\quotep{M[P]}$. To foreshadow what is to come we observe that these
operations enjoy a duality with processes very much like the duality
between vectors and maps from vectors to scalars.

Further, because the calculus is essentially higher-order, we have a
correspondence between contexts and processes. More specifically,
given a name $x$ and a context $M$ we can construct $M^{*}_{x}$ such
that 

\begin{mathpar}
  M^{*}_{x} | \lift{x}{P} \red M[P]
\end{mathpar}

namely,

\begin{mathpar}
  M^{*}_{x} := x?(u).M[\dropn{u}]
\end{mathpar}

The dependence of $M^{*}_{x}$ on a name makes it an abstraction, 

\begin{mathpar}
  M^{*} := (x)x?(u).M[\dropn{u}]
\end{mathpar}

\subsection{Additional notation}

It will sometimes be convenient to denote the process a name
quotes. We already have the notation $x = \quotep{P}$, but it will be
convenient to introduce an alternate notation, $\procn{x}$, when we
want to emphasize the connection to the use of the name. Note that, by
virtue of name equivalence, $\quotep{\procn{x}} \nameeq x$; so, the
notation is consistent with previous definitions.

Further, because names have structure it is possible to effect
substitutions on the basis of that structure. This means we need to
upgrade our notation for substitutions, which we accomplish by
adapting comprehension notation. Thus,

\begin{mathpar}
  P\{ y / x : x \in S \}
\end{mathpar}

is interpreted to mean the process derived from P by replacing (in a
capture-avoiding manner) each occurrence of $x$ in $S$ by $y$. For example,

\begin{mathpar}
  P\{ \quotep{\procn{x}|\procn{x}} / x : x \in \freenames{P} \}
\end{mathpar}

will replace each (occurrence) of a free name $x$ in $P$ by
$\quotep{\procn{x}|\procn{x}}$.

Also, we will avail ourselves of the notation $x^{L}$ and $x^{R}$ to
denote injections of a name into disjoint copies of the name
space. There are numerous ways to accomplish this. One example can be
found in \cite{MeredithR05}. This notation overloads to vectors of
names: $\vec{x}^{\pi} := (x_{i}^{\pi} \; : \; 0 \leq i < |\vec{x}| )$ where $\pi \in \{L,R\}$.

We also use $P^{\Box} := P|\Box$.

In \cite{MeredithR05} an interpretation of the new operator is
given. It turns out that there are several possible interpretations
all enjoying the requisite algebraic properties of the operator (see
\cite{milner91polyadicpi}). We will therefore make liberal use of
$(\nu\; \vec{x})P$.

% subsection the_syntax_and_semantics_of_the_notation_system (end)   

\input{qm2pi.qmops} 

\input{qm2pi.sterngerlach} 

\input{qm2pi.metric} 

% section concurrent_process_calculi (end)

%\input{qm2pi.proofsketch}

% section proof sketch (end)

%\input{qm2pi.slviaknots} 

% section spatial logic via knots (end)

\input{qm2pi.conclusion}

% section conclusion (end)

%\input{qm2pi.dtcodes} 

% section wiring algorithm (end)

\input{qm2pi.ack} 

% section acknowledgments (end)

\newpage


\bibliographystyle{plain}   
\bibliography{../../biblios/main.bib}

\input{qm2pi.rhodetails}

\end{document}

 

% section acknowledgments (end)

\newpage


\bibliographystyle{plain}   
\bibliography{../../biblios/main.bib}

\documentclass[12pt]{llncs}
%\documentclass{jktr}

\usepackage[pdftex]{hyperref}                   
\usepackage {listings}
\usepackage {mathpartir}
\usepackage{bcprules}
%\usepackage{listings}
                       
\usepackage{graphicx} 
%\usepackage[margins=2.5cm,nohead,nofoot]{geometry}
%\usepackage{geometry}
\usepackage{amsfonts}
\usepackage{amstext}
\usepackage{latexsym}
\usepackage{amssymb}
\usepackage{color}


%\include{myPreamble}
\include{qm2pi.local} 

%\ifpdf
%\usepackage[pdftex]{graphicx}
%\else
%\usepackage{graphicx}
%\fi

 % \ifpdf
%  \usepackage{pdfsync}
%  \if


%\title{Brief Article}
%\author{David F. Snyder}
%\author{L.G. Meredith}

%\address{Dept. of Math., Texas State University--San Marcos, San Marcos, TX 78666}
       
\pagestyle{empty}


\begin{document}

\lstset{language=[Objective]Caml,frame=shadowbox}

\input{qm2pi.front}

% section front matter (end)

\input{qm2pi.intro} 
 
% section introduction (end)

% \input{qm2pi.knotations} 

% section notation (end)

\input{qm2pi.process.calculi} 

% section concurrent_process_calculi_and_spatial_logics_ (end)
    
%\input{qm2pi.knots2pi} 

%\input{qm2pi.trefoil} 

%\input{qm2pi.mainthm} 

% subsection basic_interpretation (end)

%\input{qm2pi.rho.presentation} 
\subsection{The syntax and semantics of the notation system}\label{sub:the_syntax_and_semantics_of_the_notation_system} % (fold)

We now summarize a technical presentation of the calculus that
embodies our theory of dynamics. The typical presentation of such a
calculus follows the style of giving generators and relations on
them. The grammar, below, describing term constructors, freely
generates the set of processes, $\Proc$. This set is then quotiented
by a relation known as structural congruence and it is over this set
that the notion of dynamics is expressed. This presentation is
essentially that of \cite{MeredithR05} with the addition of
polyadicity and summation. For readability we have relegated some of
the technical subtleties to an appendix.

\subsubsection{Process grammar}\label{subsub:process_grammar}

\begin{mathpar}
  \inferrule* [lab=synchronization] {} {{M} \bc \pzero \;|\; x?F \;|\; x!C }
  \and
  \inferrule* [lab=abstraction] {} {{F} \bc (x)P}
  \and
  \inferrule* [lab=concretion] {} {{C} \bc \langle Q \rangle}
  \and
  \inferrule* [lab=process] {} {{P,Q} \bc M \;| \;P|Q \;|\; @{x}}
  \and
  \inferrule* [lab=name] {} {{x} \bc \quotep{P}}
\end{mathpar} 

Note that $\vec{x}$ (resp. $\vec{P}$) denotes a vector of names
(resp. processes) of length $|\vec{x}|$ (resp. $|\vec{P}|$). We adopt
the following useful abbreviations.

\begin{mathpar}
   x?(\vec{y}).P := x.(\vec{y})P \and  x\clift{\vec{P}} := x.\clift{\vec{P}}
   \and x!(y) := \lift{x}{\dropn{y}}
   \and \Pi_{i=0}^{n-1}P_i := P_0 | \ldots | P_{n-1}
\end{mathpar}

\subsubsection{Structural congruence}

\paragraph{Free and bound names and alpha-equivalence.} At the
core of structural equivalence is alpha-equivalence which identifies
process that are the same up to a change of variable. Formally, we
recognize the distinction between free and bound names. The free names
of a process, $\freenames{P}$, may be calculated recursively as
follows:

\begin{mathpar}
\freenames{\pzero} := \emptyset
  \and \\
  \freenames{x?(y).P} := \{ x \} \cup (\freenames{P} \setminus \{ y \})
  \and 
  \freenames{x!\langle P \rangle} := \{ x \} \cup \{ P \} 
  \and \\
  \freenames{P|Q} := \freenames{P} \cup \freenames{Q}
  \and \\
  \freenames{@{x}} := \{ x \}
\end{mathpar}

$\pi$
$\quotep{\pi}$

$\freenames{-} : \pi \to \mathcal{P}(\quotep{\pi})$

\begin{eqnarray*}
  \freenames{\pzero} & := & \emptyset \\
  \freenames{x?(y).P} & := & \{ x \} \cup (\freenames{P} \setminus \{ y \}) \\
  \freenames{x!\langle P \rangle} & := & \{ x \} \cup \{ P \} \\
  \freenames{P|Q} & := & \freenames{P} \cup \freenames{Q} \\
  \freenames{\dropn{x}} & := & \{ x \}
\end{eqnarray*}

The bound names of a process, $\boundnames{P}$, are those names occurring in $P$
that are not free. For example, in $x?(y).0$, the name $x$ is free, while $y$ is bound.

\begin{mathpar}
  \inferrule* [lab=monoidal-laws] {} { P|Q \equiv Q|P \and P|0 \equiv P \and P|(Q|R) \equiv (P|Q)|R }
\end{mathpar}

\begin{mathpar}
  \inferrule* [lab=alpha-equivalence] {} { (x)P \equiv (y)P\{y/x\} \and y \not\in \freenames{P} }
\end{mathpar}

\begin{definition}
Then two processes, $P,Q$, are alpha-equivalent if $P = Q\{\vec{y}/\vec{x}\}$ for
some $\vec{x} \in \boundnames{Q},\vec{y} \in \boundnames{P}$, where $Q\{\vec{y}/\vec{x}\}$
denotes the capture-avoiding substitution of $\vec{y}$ for $\vec{x}$ in $Q$.
\end{definition}

\begin{definition}
  The {\em structural congruence} \cite{SangiorgiWalker} , $\equiv$,
  between processes is the least congruence containing
  alpha-equivalence, satisfying the abelian monoid laws
  (associativity, commutativity and $\pzero$ as identity) for parallel
  composition $|$ and for summation $+$.
\end{definition}

\subsection{Name equivalence}

We take name equivalence, written $\nameeq$, to be the smallest
equivalence relation generated by the following rules.

\begin{mathpar}
\inferrule*[lab=Quote-drop]
{ }
{ \quotep{@{x}} \nameeq x }

\inferrule*[lab=Struct-equiv]
{ P \scong Q }
{ \quotep{P} \nameeq \quotep{Q} }
\end{mathpar}

The astute reader will have noticed that the mutual recursion of names
and processes imposes a mutual recursion on alpha-equivalence and
structural equivalence via name-equivalence. Fortunately, all of this
works out pleasantly and we may calculate in the natural way, free of
concern. The reader interested in the details is referred to the
appendix \ref{appendix:rho_details}.

\subsection{Substitution}

We use $\Proc$ for the set of processes, $\QProc$ for the set of
names, and $\id{\{}\vec{y} / \vec{x} \id{\}}$ to denote partial maps,
$s : \QProc \rightarrow \QProc$. A map, $s$ lifts, uniquely, to a map
on process terms, $\widehat{s} : \Proc \rightarrow \Proc$ by the
following equations.

\begin{mathpar}
  (0) \psubstp{Q}{P} := 0 \\
  (R \juxtap S) \psubstp{Q}{P}
  :=    
  (R)\psubstp{Q}{P} \juxtap (S) \psubstp{Q}{P} \\
  (x?(y).R) \psubstp{Q}{P}    
  :=    
  (x)\substp{Q}{P} (z)\concat( (R \psubstn{z}{y}) \psubstp{Q}{P} ) \\
  (\lift{x}{R}) \psubstp{Q}{P}  
  :=
  \lift{(x)\substp{Q}{P}}{ R \psubstp{Q}{P} } \\
%   (\dropn{x})  \psubstp{Q}{P}       
%   := 
%   \left\{ 
%     \begin{array}{ccc} 
%       \dropn{\quotep{Q}} & & x \nameeq \quotep{P} \\
%       \dropn{x} & & otherwise \\
%     \end{array}
%   \right. 
  (\dropn{x})  \psubstp{Q}{P}       
  := 
  \left\{ 
    \begin{array}{ccc} 
      Q & & x \nameeq \quotep{P} \\
      \dropn{x} & & otherwise \\
    \end{array}
  \right.
\end{mathpar}
 

where

\begin{eqnarray}
  (x)\id{\{} \lpquote Q \rpquote / \lpquote P \rpquote \id{\}}            = 
  \left\{ 
    \begin{array}{ccc}
      \lpquote Q \rpquote & & x \nameeq \lpquote P \rpquote \\
      x & & otherwise \\
    \end{array}
  \right. \nonumber
\end{eqnarray}

and $z$ is chosen distinct from $\quotep{P}$, $\quotep{Q}$, the free
names in $Q$, and all the names in $R$. Our $\alpha$-equivalence will
be built in the standard way from this substitution.

\begin{remark}\label{rem:no_self_referential_names}
  One consequence of these definitions is that $\forall P. \quotep{P}
  \not\in \freenames{P}$.
\end{remark}

\subsection{ Dynamic quote: an example }

Anticipating something of what's to come, consider applying the
substitution, $\widehat{\id{\{}u / z \id{\}}}$, to the following pair
of processes, $\lift{w}{y!(z)}$ and $w[ \lpquote y!(z) \rpquote ]$.

\begin{eqnarray}
	\lift{w}{y!(z)}\widehat{\id{\{}u / z \id{\}}}
		& = &
		\lift{w}{y!(u)} \nonumber\\
	w[ \lpquote y!(z) \rpquote ] \widehat{ \id{\{}u / z \id{\}} }
		& = &
		w[ \lpquote y!(z) \rpquote ] \nonumber
\end{eqnarray}

Because the body of the process between quotes is impervious to
substitution, we get radically different answers. In fact, by
examining the first process in an input context,
e.g. $x?(z).\lift{w}{y!(z)}$, we see that the process under the lift
operator may be shaped by prefixed inputs binding a name inside it. In
this sense, the lift operator will be seen as a way to dynamically
construct processes before reifying them as names.

Finally equipped with these standard features we can present the
dynamics of the calculus.

\subsubsection{Operational semantics} 

Finally, we introduce the computational dynamics. What marks these
algebras as distinct from other more traditionally studied algebraic
structures, e.g. vector spaces or polynomial rings, is the manner in
which dynamics is captured. In traditional structures, dynamics is typically
expressed through morphisms between such structures, as in linear maps
between vector spaces or morphisms between rings. In algebras
associated with the semantics of computation, the dynamics is
expressed as part of the algebraic structure itself, through a
reduction reduction relation typically denoted by $\red$. Below, we
give a recursive presentation of this relation for the calculus used
in the encoding.

$\red \subseteq \pi \times \pi$
$\red : \pi \to \mathcal{P}(\pi)$

\begin{mathpar}
  \inferrule* [lab=Comm] { \textsf{match}( x_{src}, x_{trgt} ) } { x_{trgt}?(y)P \; | \; x_{src}!\langle {Q} \rangle \red P\{\quotep{Q}/y}\} }
  \and \\
  \inferrule* [lab=Par] {{P} \red {P}'} {{{P} | {Q}} \red {{P}' | {Q}}}
  \and
  \inferrule* [lab=Equiv]{{{P} \scong {P}'} \andalso {{P}' \red {Q}'} \andalso {{Q}' \scong {Q}}}{{P} \red {Q}}
\end{mathpar}

\begin{eqnarray*}
  match_{\equiv} (\quotep{P},\quotep{Q}) & := & P \equiv Q \\
  match_{\dagger}(\quotep{P},\quotep{Q}) & := & \forall R. P|Q \red^{*} R => R \red^{*} 0 \\
  match_{K}(\quotep{P},\quotep{Q}) & := & K \mbox{ for some context } K
\end{eqnarray*}

$u?(x)P | u!\langle Q \rangle \red P\{\quotep{Q}/x\}$

%We write $\wred$ for $\red^*$, and $P\red$ if $\exists Q $ such that $ P \red Q$.
We write $P\red$ if $\exists Q $ such that $ P \red Q$ and $P\not\red$, otherwise.

\section{Replication}

As mentioned before, it is known that replication (and hence
recursion) can be implemented in a higher-order process algebra
\cite{SangiorgiWalker}. As our first example of calculation with the
machinery thus far presented we give the construction explicitly in
the {\rhoc}.

\begin{eqnarray}
	D_{x} & := & \prefix{x}{y}{(\binpar{\outputp{x}{y}}{@{y}})} \nonumber\\
	\bangp_{x}{P} & := & \binpar{{x}!\langle{\binpar{D_{x}}{P}}\rangle}{D_{x}} \nonumber
\end{eqnarray}

\begin{eqnarray}
	\bangp_{x}{P} & & \nonumber\\
	=
	& {x}!\langle{(\prefix{x}{y}{(\outputp{x}{y} | @{y})) | P}}\rangle 
	      | \prefix{x}{y}{(\outputp{x}{y} | @{y})} & \nonumber\\
	\red
	& (\outputp{x}{y} | @{y})\substn{\quotep{(\prefix{x}{y}{(@{y} | \outputp{x}{y})) | P}}}{y} & \nonumber\\
	=
	& \outputp{x}{\quotep{(\prefix{x}{y}{(\outputp{x}{y} | @{y})) | P}}}
	  | {(\prefix{x}{y}{(\outputp{x}{y} | @{y})) | P}} & \nonumber\\
	\red
	& \ldots & \nonumber\\
	\red^*
	& P | P | \ldots & \nonumber
\end{eqnarray}

Of course, this encoding, as an implementation, runs away, unfolding
$\bangp{P}$ eagerly. A lazier and more implementable replication
operator, restricted to input-guarded processes, may be obtained as follows.

\begin{eqnarray}
\bangp{\prefix{u}{v}{P}} 
	:= 
	\binpar{\lift{x}{\prefix{u}{v}{(\binpar{D(x)}{P})}}}{D(x)} \nonumber
\end{eqnarray}

\begin{remark}
  Note that the lazier definition still does not deal with summation
  or mixed summation (i.e. sums over input and output). The reader is
  invited to construct definitions of replication that deal with these
  features. 

  Further, the definitions are parameterized in a name, $x$. Can you,
  gentle reader, make a definition that eliminates this parameter and
  guarantees no accidental interaction between the replication
  machinery and the process being replicated -- i.e. no accidental
  sharing of names used by the process to get its work done and the
  name(s) used by the replication to effect copying. This latter
  revision of the definition of replication is crucial to obtaining
  the expected identity $!!P \sim !P$.
\end{remark}

\begin{remark}\label{rem:paradoxical_combinator}
  The reader familiar with the lambda calculus will have noticed the
  similarity between $D$ and the paradoxical combinator.

  [Ed. note: the existence of this seems to suggest we have to be more
  restrictive on the set of processes and names we admit if we are to
  support no-cloning.]
\end{remark}

\subsubsection{Bisimulation}

The computational dynamics gives rise to another kind of equivalence,
the equivalence of computational behavior. As previously mentioned
this is typically captured \emph{via} some form of bisimulation.

% The notion we use in this paper is weak barbed bisimulation
% \cite{milner91polyadicpi}.

The notion we use in this paper is derived from weak barbed
bisimulation \cite{milner91polyadicpi}. 

\begin{definition}
An \emph{observation relation}, $\downarrow_{\mathcal N}$, over a set
of names, $\mathcal N$, is the smallest relation satisfying the rules
below.

\infrule[Out-barb]{y \in {\mathcal N}, \; x \nameeq y}
		  {\outputp{x}{v} \downarrow_{\mathcal N} x}
\infrule[Par-barb]{\mbox{$P\downarrow_{\mathcal N} x$ or $Q\downarrow_{\mathcal N} x$}}
		  {\binpar{P}{Q} \downarrow_{\mathcal N} x}

We write $P \Downarrow_{\mathcal N} x$ if there is $Q$ such that 
$P \wred Q$ and $Q \downarrow_{\mathcal N} x$.
\end{definition}

\begin{definition}
%\label{def.bbisim}
An  ${\mathcal N}$-\emph{barbed bisimulation} over a set of names, ${\mathcal N}$, is a symmetric binary relation 
${\mathcal S}_{\mathcal N}$ between agents such that $P\rel{S}_{\mathcal N}Q$ implies:
\begin{enumerate}
\item If $P \red P'$ then $Q \wred Q'$ and $P'\rel{S}_{\mathcal N} Q'$.
\item If $P\downarrow_{\mathcal N} x$, then $Q\Downarrow_{\mathcal N} x$.
\end{enumerate}
$P$ is ${\mathcal N}$-barbed bisimilar to $Q$, written
$P \wbbisim_{\mathcal N} Q$, if $P \rel{S}_{\mathcal N} Q$ for some ${\mathcal N}$-barbed bisimulation ${\mathcal S}_{\mathcal N}$.
\end{definition}

$\mathcal{R} \subseteq \pi \times \pi$

$P \mathcal{R} Q => \forall P'. P \red P' \Rightarrow \exists Q'. Q \red Q', P' \mathcal{R} Q'$

$P \vdash x \Rightarrow Q \vdash x$

\begin{mathpar}
  \inferrule*[lab=Out-barb]{x \nameeq y}{{y}!\langle{Q}\rangle \vdash x}
  \and
  \inferrule*[lab=Par-barb]{\mbox{$P\vdash x$ or $Q\vdash x$}}{\binpar{P}{Q} \vdash x}
\end{mathpar}

\subsubsection{Contexts}

One of the principle advantages of computational calculi like the
$\pi$-calculus is a well-defined notion of context,
contextual-equivalence and a correlation between
contextual-equivalence and notions of bisimulation. The notion of
context allows the decomposition of a process into (sub-)process and
its syntactic environment, its context. Thus, a context may be
thought of as a process with a ``hole'' (written $\Box$) in it. The
application of a context $M$ to a process $P$, written $M[P]$, is
tantamount to filling the hole in $M$ with $P$. In this paper we do
not need the full weight of this theory, but do make use of the notion
of context in the proof the main theorem. 

\begin{mathpar}
  \inferrule* [lab=summation] {} {{M_{M},M_{N}} \bc \Box \;|\; x.M_{A} \;|\; M_{M}+M_{N}}
  \and
  \inferrule* [lab=agent] {} {{M_{A}} \bc (\vec{x})M_{P} \;| \; \clift{P_0,\ldots,M_{P},\ldots,P_N}}
  \and \\
  \inferrule* [lab=process] {} {{M_{P}} \bc M_{N} \;| \;P|M_{P} }
\end{mathpar} 

\begin{mathpar}
  \inferrule* [lab=sychronization] {} {M_{N} \bc \Box \;|\; x?M_{F} \;|\; x!M_{C}}
  \and
  \inferrule* [lab=abstraction] {} {{M_{F}} \bc (x)M_{P} }
  \and
  \inferrule* [lab=concretion] {} {{M_{C}} \bc \langle M_{P} \rangle }
  \and \\
  \inferrule* [lab=process] {} {{M_{P}} \bc M_{N} \;| \;P|M_{P} }
\end{mathpar}

\begin{definition}[contextual application] Given a context $M$, and
  process $P$, we define the \emph{contextual application}, $M[P] :=
  M\{P/\Box\}$. That is, the contextual application of M to P is the
  substitution of $P$ for $\Box$ in $M$.
\end{definition}

$\meaningof{-} : L \to \mathcal{P}(\pi)$

\begin{mathpar}
  \inferrule* [lab=collection] {} {\meaningof{true} = \pi, \and \meaningof{~E} = \pi \setminus \meaningof{E}, \and \meaningof{E_{1} \& E_{2}} = \meaningof{E_{1}} \cap \meaningof{E_{2}}}
\end{mathpar}

\begin{mathpar}
  \inferrule* [lab=structure] {} {\meaningof{0} = \{ P \in \pi | P \equiv 0 \}, \and \\ \meaningof{E_1 | E_2} = \{ P \in \pi | P \equiv P_{1} | P_{2}, P_{1} \in \meaningof{E_{1}}, P_{2} \in \meaningof{E_2}\} }
\end{mathpar}

\begin{mathpar}
 \inferrule* [lab=behavior] {} {\meaningof{\langle a?b \rangle E} = \{ P \in \pi | P \equiv Q | u?(y)P', \\ \and \\\\ \and \\ \;\;\; u \in \meaningof{a}, \forall z.P'\{z/y\} \in \meaningof{E\{z/b\}}\}, \and \\ \meaningof{a!E} = \{ P \in \pi | P \equiv Q | x!\langle P' \rangle, x \in \meaningof{a} P' \in \meaningof{E}\} }
\end{mathpar}

\begin{mathpar}
 \inferrule* [lab=nominal] {} {\meaningof{\quotep{E}} = \{ \quotep{P} \in \quotep{\pi} | P \in \meaningof{E} \}, \and \meaningof{\quotep{P}} = \{ \quotep{Q} \in \quotep{\pi} | P \equiv Q \} \and \\ \meaningof{@\quotep{E}} = \{ P \in \pi | P \equiv @x, x \in \meaningof{E} \}}
\end{mathpar}

\begin{eqnarray*}
  \\
  \meaningof{-} : TS \to ST
\end{eqnarray*}

\begin{eqnarray*}
  \\
  L : TS \to ST
\end{eqnarray*}

\begin{eqnarray*}
  \\
  P \models E \iff P \in \meaningof{E}
\end{eqnarray*}

\begin{eqnarray*}
  P \approx_{L} Q \iff \forall E \in L. P \models E \iff Q \models E
\end{eqnarray*}

\begin{eqnarray*}
  P \approx_{K} Q
\end{eqnarray*}

\begin{eqnarray*}
  P \approx Q
\end{eqnarray*}

$\approx_{K} = \approx = \approx_{L}$

\subsubsection{Contextual duality}

Note that contexts extend the quotation operation to a family of
operations from processes to names. Given a context, $M$, we can
define a \emph{nominal context}, $\quotep{M}$ by $\quotep{M}[P] :=
\quotep{M[P]}$. To foreshadow what is to come we observe that these
operations enjoy a duality with processes very much like the duality
between vectors and maps from vectors to scalars.

Further, because the calculus is essentially higher-order, we have a
correspondence between contexts and processes. More specifically,
given a name $x$ and a context $M$ we can construct $M^{*}_{x}$ such
that 

\begin{mathpar}
  M^{*}_{x} | \lift{x}{P} \red M[P]
\end{mathpar}

namely,

\begin{mathpar}
  M^{*}_{x} := x?(u).M[\dropn{u}]
\end{mathpar}

The dependence of $M^{*}_{x}$ on a name makes it an abstraction, 

\begin{mathpar}
  M^{*} := (x)x?(u).M[\dropn{u}]
\end{mathpar}

\subsection{Additional notation}

It will sometimes be convenient to denote the process a name
quotes. We already have the notation $x = \quotep{P}$, but it will be
convenient to introduce an alternate notation, $\procn{x}$, when we
want to emphasize the connection to the use of the name. Note that, by
virtue of name equivalence, $\quotep{\procn{x}} \nameeq x$; so, the
notation is consistent with previous definitions.

Further, because names have structure it is possible to effect
substitutions on the basis of that structure. This means we need to
upgrade our notation for substitutions, which we accomplish by
adapting comprehension notation. Thus,

\begin{mathpar}
  P\{ y / x : x \in S \}
\end{mathpar}

is interpreted to mean the process derived from P by replacing (in a
capture-avoiding manner) each occurrence of $x$ in $S$ by $y$. For example,

\begin{mathpar}
  P\{ \quotep{\procn{x}|\procn{x}} / x : x \in \freenames{P} \}
\end{mathpar}

will replace each (occurrence) of a free name $x$ in $P$ by
$\quotep{\procn{x}|\procn{x}}$.

Also, we will avail ourselves of the notation $x^{L}$ and $x^{R}$ to
denote injections of a name into disjoint copies of the name
space. There are numerous ways to accomplish this. One example can be
found in \cite{MeredithR05}. This notation overloads to vectors of
names: $\vec{x}^{\pi} := (x_{i}^{\pi} \; : \; 0 \leq i < |\vec{x}| )$ where $\pi \in \{L,R\}$.

We also use $P^{\Box} := P|\Box$.

In \cite{MeredithR05} an interpretation of the new operator is
given. It turns out that there are several possible interpretations
all enjoying the requisite algebraic properties of the operator (see
\cite{milner91polyadicpi}). We will therefore make liberal use of
$(\nu\; \vec{x})P$.

% subsection the_syntax_and_semantics_of_the_notation_system (end)   

\input{qm2pi.qmops} 

\input{qm2pi.sterngerlach} 

\input{qm2pi.metric} 

% section concurrent_process_calculi (end)

%\input{qm2pi.proofsketch}

% section proof sketch (end)

%\input{qm2pi.slviaknots} 

% section spatial logic via knots (end)

\input{qm2pi.conclusion}

% section conclusion (end)

%\input{qm2pi.dtcodes} 

% section wiring algorithm (end)

\input{qm2pi.ack} 

% section acknowledgments (end)

\newpage


\bibliographystyle{plain}   
\bibliography{../../biblios/main.bib}

\input{qm2pi.rhodetails}

\end{document}



\end{document}

 

% section acknowledgments (end)

\newpage


\bibliographystyle{plain}   
\bibliography{../../biblios/main.bib}

\documentclass[12pt]{llncs}
%\documentclass{jktr}

\usepackage[pdftex]{hyperref}                   
\usepackage {listings}
\usepackage {mathpartir}
\usepackage{bcprules}
%\usepackage{listings}
                       
\usepackage{graphicx} 
%\usepackage[margins=2.5cm,nohead,nofoot]{geometry}
%\usepackage{geometry}
\usepackage{amsfonts}
\usepackage{amstext}
\usepackage{latexsym}
\usepackage{amssymb}
\usepackage{color}


%\include{myPreamble}
\documentclass[12pt]{llncs}
%\documentclass{jktr}

\usepackage[pdftex]{hyperref}                   
\usepackage {listings}
\usepackage {mathpartir}
\usepackage{bcprules}
%\usepackage{listings}
                       
\usepackage{graphicx} 
%\usepackage[margins=2.5cm,nohead,nofoot]{geometry}
%\usepackage{geometry}
\usepackage{amsfonts}
\usepackage{amstext}
\usepackage{latexsym}
\usepackage{amssymb}
\usepackage{color}


%\include{myPreamble}
\include{qm2pi.local} 

%\ifpdf
%\usepackage[pdftex]{graphicx}
%\else
%\usepackage{graphicx}
%\fi

 % \ifpdf
%  \usepackage{pdfsync}
%  \if


%\title{Brief Article}
%\author{David F. Snyder}
%\author{L.G. Meredith}

%\address{Dept. of Math., Texas State University--San Marcos, San Marcos, TX 78666}
       
\pagestyle{empty}


\begin{document}

\lstset{language=[Objective]Caml,frame=shadowbox}

\input{qm2pi.front}

% section front matter (end)

\input{qm2pi.intro} 
 
% section introduction (end)

% \input{qm2pi.knotations} 

% section notation (end)

\input{qm2pi.process.calculi} 

% section concurrent_process_calculi_and_spatial_logics_ (end)
    
%\input{qm2pi.knots2pi} 

%\input{qm2pi.trefoil} 

%\input{qm2pi.mainthm} 

% subsection basic_interpretation (end)

%\input{qm2pi.rho.presentation} 
\subsection{The syntax and semantics of the notation system}\label{sub:the_syntax_and_semantics_of_the_notation_system} % (fold)

We now summarize a technical presentation of the calculus that
embodies our theory of dynamics. The typical presentation of such a
calculus follows the style of giving generators and relations on
them. The grammar, below, describing term constructors, freely
generates the set of processes, $\Proc$. This set is then quotiented
by a relation known as structural congruence and it is over this set
that the notion of dynamics is expressed. This presentation is
essentially that of \cite{MeredithR05} with the addition of
polyadicity and summation. For readability we have relegated some of
the technical subtleties to an appendix.

\subsubsection{Process grammar}\label{subsub:process_grammar}

\begin{mathpar}
  \inferrule* [lab=synchronization] {} {{M} \bc \pzero \;|\; x?F \;|\; x!C }
  \and
  \inferrule* [lab=abstraction] {} {{F} \bc (x)P}
  \and
  \inferrule* [lab=concretion] {} {{C} \bc \langle Q \rangle}
  \and
  \inferrule* [lab=process] {} {{P,Q} \bc M \;| \;P|Q \;|\; @{x}}
  \and
  \inferrule* [lab=name] {} {{x} \bc \quotep{P}}
\end{mathpar} 

Note that $\vec{x}$ (resp. $\vec{P}$) denotes a vector of names
(resp. processes) of length $|\vec{x}|$ (resp. $|\vec{P}|$). We adopt
the following useful abbreviations.

\begin{mathpar}
   x?(\vec{y}).P := x.(\vec{y})P \and  x\clift{\vec{P}} := x.\clift{\vec{P}}
   \and x!(y) := \lift{x}{\dropn{y}}
   \and \Pi_{i=0}^{n-1}P_i := P_0 | \ldots | P_{n-1}
\end{mathpar}

\subsubsection{Structural congruence}

\paragraph{Free and bound names and alpha-equivalence.} At the
core of structural equivalence is alpha-equivalence which identifies
process that are the same up to a change of variable. Formally, we
recognize the distinction between free and bound names. The free names
of a process, $\freenames{P}$, may be calculated recursively as
follows:

\begin{mathpar}
\freenames{\pzero} := \emptyset
  \and \\
  \freenames{x?(y).P} := \{ x \} \cup (\freenames{P} \setminus \{ y \})
  \and 
  \freenames{x!\langle P \rangle} := \{ x \} \cup \{ P \} 
  \and \\
  \freenames{P|Q} := \freenames{P} \cup \freenames{Q}
  \and \\
  \freenames{@{x}} := \{ x \}
\end{mathpar}

$\pi$
$\quotep{\pi}$

$\freenames{-} : \pi \to \mathcal{P}(\quotep{\pi})$

\begin{eqnarray*}
  \freenames{\pzero} & := & \emptyset \\
  \freenames{x?(y).P} & := & \{ x \} \cup (\freenames{P} \setminus \{ y \}) \\
  \freenames{x!\langle P \rangle} & := & \{ x \} \cup \{ P \} \\
  \freenames{P|Q} & := & \freenames{P} \cup \freenames{Q} \\
  \freenames{\dropn{x}} & := & \{ x \}
\end{eqnarray*}

The bound names of a process, $\boundnames{P}$, are those names occurring in $P$
that are not free. For example, in $x?(y).0$, the name $x$ is free, while $y$ is bound.

\begin{mathpar}
  \inferrule* [lab=monoidal-laws] {} { P|Q \equiv Q|P \and P|0 \equiv P \and P|(Q|R) \equiv (P|Q)|R }
\end{mathpar}

\begin{mathpar}
  \inferrule* [lab=alpha-equivalence] {} { (x)P \equiv (y)P\{y/x\} \and y \not\in \freenames{P} }
\end{mathpar}

\begin{definition}
Then two processes, $P,Q$, are alpha-equivalent if $P = Q\{\vec{y}/\vec{x}\}$ for
some $\vec{x} \in \boundnames{Q},\vec{y} \in \boundnames{P}$, where $Q\{\vec{y}/\vec{x}\}$
denotes the capture-avoiding substitution of $\vec{y}$ for $\vec{x}$ in $Q$.
\end{definition}

\begin{definition}
  The {\em structural congruence} \cite{SangiorgiWalker} , $\equiv$,
  between processes is the least congruence containing
  alpha-equivalence, satisfying the abelian monoid laws
  (associativity, commutativity and $\pzero$ as identity) for parallel
  composition $|$ and for summation $+$.
\end{definition}

\subsection{Name equivalence}

We take name equivalence, written $\nameeq$, to be the smallest
equivalence relation generated by the following rules.

\begin{mathpar}
\inferrule*[lab=Quote-drop]
{ }
{ \quotep{@{x}} \nameeq x }

\inferrule*[lab=Struct-equiv]
{ P \scong Q }
{ \quotep{P} \nameeq \quotep{Q} }
\end{mathpar}

The astute reader will have noticed that the mutual recursion of names
and processes imposes a mutual recursion on alpha-equivalence and
structural equivalence via name-equivalence. Fortunately, all of this
works out pleasantly and we may calculate in the natural way, free of
concern. The reader interested in the details is referred to the
appendix \ref{appendix:rho_details}.

\subsection{Substitution}

We use $\Proc$ for the set of processes, $\QProc$ for the set of
names, and $\id{\{}\vec{y} / \vec{x} \id{\}}$ to denote partial maps,
$s : \QProc \rightarrow \QProc$. A map, $s$ lifts, uniquely, to a map
on process terms, $\widehat{s} : \Proc \rightarrow \Proc$ by the
following equations.

\begin{mathpar}
  (0) \psubstp{Q}{P} := 0 \\
  (R \juxtap S) \psubstp{Q}{P}
  :=    
  (R)\psubstp{Q}{P} \juxtap (S) \psubstp{Q}{P} \\
  (x?(y).R) \psubstp{Q}{P}    
  :=    
  (x)\substp{Q}{P} (z)\concat( (R \psubstn{z}{y}) \psubstp{Q}{P} ) \\
  (\lift{x}{R}) \psubstp{Q}{P}  
  :=
  \lift{(x)\substp{Q}{P}}{ R \psubstp{Q}{P} } \\
%   (\dropn{x})  \psubstp{Q}{P}       
%   := 
%   \left\{ 
%     \begin{array}{ccc} 
%       \dropn{\quotep{Q}} & & x \nameeq \quotep{P} \\
%       \dropn{x} & & otherwise \\
%     \end{array}
%   \right. 
  (\dropn{x})  \psubstp{Q}{P}       
  := 
  \left\{ 
    \begin{array}{ccc} 
      Q & & x \nameeq \quotep{P} \\
      \dropn{x} & & otherwise \\
    \end{array}
  \right.
\end{mathpar}
 

where

\begin{eqnarray}
  (x)\id{\{} \lpquote Q \rpquote / \lpquote P \rpquote \id{\}}            = 
  \left\{ 
    \begin{array}{ccc}
      \lpquote Q \rpquote & & x \nameeq \lpquote P \rpquote \\
      x & & otherwise \\
    \end{array}
  \right. \nonumber
\end{eqnarray}

and $z$ is chosen distinct from $\quotep{P}$, $\quotep{Q}$, the free
names in $Q$, and all the names in $R$. Our $\alpha$-equivalence will
be built in the standard way from this substitution.

\begin{remark}\label{rem:no_self_referential_names}
  One consequence of these definitions is that $\forall P. \quotep{P}
  \not\in \freenames{P}$.
\end{remark}

\subsection{ Dynamic quote: an example }

Anticipating something of what's to come, consider applying the
substitution, $\widehat{\id{\{}u / z \id{\}}}$, to the following pair
of processes, $\lift{w}{y!(z)}$ and $w[ \lpquote y!(z) \rpquote ]$.

\begin{eqnarray}
	\lift{w}{y!(z)}\widehat{\id{\{}u / z \id{\}}}
		& = &
		\lift{w}{y!(u)} \nonumber\\
	w[ \lpquote y!(z) \rpquote ] \widehat{ \id{\{}u / z \id{\}} }
		& = &
		w[ \lpquote y!(z) \rpquote ] \nonumber
\end{eqnarray}

Because the body of the process between quotes is impervious to
substitution, we get radically different answers. In fact, by
examining the first process in an input context,
e.g. $x?(z).\lift{w}{y!(z)}$, we see that the process under the lift
operator may be shaped by prefixed inputs binding a name inside it. In
this sense, the lift operator will be seen as a way to dynamically
construct processes before reifying them as names.

Finally equipped with these standard features we can present the
dynamics of the calculus.

\subsubsection{Operational semantics} 

Finally, we introduce the computational dynamics. What marks these
algebras as distinct from other more traditionally studied algebraic
structures, e.g. vector spaces or polynomial rings, is the manner in
which dynamics is captured. In traditional structures, dynamics is typically
expressed through morphisms between such structures, as in linear maps
between vector spaces or morphisms between rings. In algebras
associated with the semantics of computation, the dynamics is
expressed as part of the algebraic structure itself, through a
reduction reduction relation typically denoted by $\red$. Below, we
give a recursive presentation of this relation for the calculus used
in the encoding.

$\red \subseteq \pi \times \pi$
$\red : \pi \to \mathcal{P}(\pi)$

\begin{mathpar}
  \inferrule* [lab=Comm] { \textsf{match}( x_{src}, x_{trgt} ) } { x_{trgt}?(y)P \; | \; x_{src}!\langle {Q} \rangle \red P\{\quotep{Q}/y}\} }
  \and \\
  \inferrule* [lab=Par] {{P} \red {P}'} {{{P} | {Q}} \red {{P}' | {Q}}}
  \and
  \inferrule* [lab=Equiv]{{{P} \scong {P}'} \andalso {{P}' \red {Q}'} \andalso {{Q}' \scong {Q}}}{{P} \red {Q}}
\end{mathpar}

\begin{eqnarray*}
  match_{\equiv} (\quotep{P},\quotep{Q}) & := & P \equiv Q \\
  match_{\dagger}(\quotep{P},\quotep{Q}) & := & \forall R. P|Q \red^{*} R => R \red^{*} 0 \\
  match_{K}(\quotep{P},\quotep{Q}) & := & K \mbox{ for some context } K
\end{eqnarray*}

$u?(x)P | u!\langle Q \rangle \red P\{\quotep{Q}/x\}$

%We write $\wred$ for $\red^*$, and $P\red$ if $\exists Q $ such that $ P \red Q$.
We write $P\red$ if $\exists Q $ such that $ P \red Q$ and $P\not\red$, otherwise.

\section{Replication}

As mentioned before, it is known that replication (and hence
recursion) can be implemented in a higher-order process algebra
\cite{SangiorgiWalker}. As our first example of calculation with the
machinery thus far presented we give the construction explicitly in
the {\rhoc}.

\begin{eqnarray}
	D_{x} & := & \prefix{x}{y}{(\binpar{\outputp{x}{y}}{@{y}})} \nonumber\\
	\bangp_{x}{P} & := & \binpar{{x}!\langle{\binpar{D_{x}}{P}}\rangle}{D_{x}} \nonumber
\end{eqnarray}

\begin{eqnarray}
	\bangp_{x}{P} & & \nonumber\\
	=
	& {x}!\langle{(\prefix{x}{y}{(\outputp{x}{y} | @{y})) | P}}\rangle 
	      | \prefix{x}{y}{(\outputp{x}{y} | @{y})} & \nonumber\\
	\red
	& (\outputp{x}{y} | @{y})\substn{\quotep{(\prefix{x}{y}{(@{y} | \outputp{x}{y})) | P}}}{y} & \nonumber\\
	=
	& \outputp{x}{\quotep{(\prefix{x}{y}{(\outputp{x}{y} | @{y})) | P}}}
	  | {(\prefix{x}{y}{(\outputp{x}{y} | @{y})) | P}} & \nonumber\\
	\red
	& \ldots & \nonumber\\
	\red^*
	& P | P | \ldots & \nonumber
\end{eqnarray}

Of course, this encoding, as an implementation, runs away, unfolding
$\bangp{P}$ eagerly. A lazier and more implementable replication
operator, restricted to input-guarded processes, may be obtained as follows.

\begin{eqnarray}
\bangp{\prefix{u}{v}{P}} 
	:= 
	\binpar{\lift{x}{\prefix{u}{v}{(\binpar{D(x)}{P})}}}{D(x)} \nonumber
\end{eqnarray}

\begin{remark}
  Note that the lazier definition still does not deal with summation
  or mixed summation (i.e. sums over input and output). The reader is
  invited to construct definitions of replication that deal with these
  features. 

  Further, the definitions are parameterized in a name, $x$. Can you,
  gentle reader, make a definition that eliminates this parameter and
  guarantees no accidental interaction between the replication
  machinery and the process being replicated -- i.e. no accidental
  sharing of names used by the process to get its work done and the
  name(s) used by the replication to effect copying. This latter
  revision of the definition of replication is crucial to obtaining
  the expected identity $!!P \sim !P$.
\end{remark}

\begin{remark}\label{rem:paradoxical_combinator}
  The reader familiar with the lambda calculus will have noticed the
  similarity between $D$ and the paradoxical combinator.

  [Ed. note: the existence of this seems to suggest we have to be more
  restrictive on the set of processes and names we admit if we are to
  support no-cloning.]
\end{remark}

\subsubsection{Bisimulation}

The computational dynamics gives rise to another kind of equivalence,
the equivalence of computational behavior. As previously mentioned
this is typically captured \emph{via} some form of bisimulation.

% The notion we use in this paper is weak barbed bisimulation
% \cite{milner91polyadicpi}.

The notion we use in this paper is derived from weak barbed
bisimulation \cite{milner91polyadicpi}. 

\begin{definition}
An \emph{observation relation}, $\downarrow_{\mathcal N}$, over a set
of names, $\mathcal N$, is the smallest relation satisfying the rules
below.

\infrule[Out-barb]{y \in {\mathcal N}, \; x \nameeq y}
		  {\outputp{x}{v} \downarrow_{\mathcal N} x}
\infrule[Par-barb]{\mbox{$P\downarrow_{\mathcal N} x$ or $Q\downarrow_{\mathcal N} x$}}
		  {\binpar{P}{Q} \downarrow_{\mathcal N} x}

We write $P \Downarrow_{\mathcal N} x$ if there is $Q$ such that 
$P \wred Q$ and $Q \downarrow_{\mathcal N} x$.
\end{definition}

\begin{definition}
%\label{def.bbisim}
An  ${\mathcal N}$-\emph{barbed bisimulation} over a set of names, ${\mathcal N}$, is a symmetric binary relation 
${\mathcal S}_{\mathcal N}$ between agents such that $P\rel{S}_{\mathcal N}Q$ implies:
\begin{enumerate}
\item If $P \red P'$ then $Q \wred Q'$ and $P'\rel{S}_{\mathcal N} Q'$.
\item If $P\downarrow_{\mathcal N} x$, then $Q\Downarrow_{\mathcal N} x$.
\end{enumerate}
$P$ is ${\mathcal N}$-barbed bisimilar to $Q$, written
$P \wbbisim_{\mathcal N} Q$, if $P \rel{S}_{\mathcal N} Q$ for some ${\mathcal N}$-barbed bisimulation ${\mathcal S}_{\mathcal N}$.
\end{definition}

$\mathcal{R} \subseteq \pi \times \pi$

$P \mathcal{R} Q => \forall P'. P \red P' \Rightarrow \exists Q'. Q \red Q', P' \mathcal{R} Q'$

$P \vdash x \Rightarrow Q \vdash x$

\begin{mathpar}
  \inferrule*[lab=Out-barb]{x \nameeq y}{{y}!\langle{Q}\rangle \vdash x}
  \and
  \inferrule*[lab=Par-barb]{\mbox{$P\vdash x$ or $Q\vdash x$}}{\binpar{P}{Q} \vdash x}
\end{mathpar}

\subsubsection{Contexts}

One of the principle advantages of computational calculi like the
$\pi$-calculus is a well-defined notion of context,
contextual-equivalence and a correlation between
contextual-equivalence and notions of bisimulation. The notion of
context allows the decomposition of a process into (sub-)process and
its syntactic environment, its context. Thus, a context may be
thought of as a process with a ``hole'' (written $\Box$) in it. The
application of a context $M$ to a process $P$, written $M[P]$, is
tantamount to filling the hole in $M$ with $P$. In this paper we do
not need the full weight of this theory, but do make use of the notion
of context in the proof the main theorem. 

\begin{mathpar}
  \inferrule* [lab=summation] {} {{M_{M},M_{N}} \bc \Box \;|\; x.M_{A} \;|\; M_{M}+M_{N}}
  \and
  \inferrule* [lab=agent] {} {{M_{A}} \bc (\vec{x})M_{P} \;| \; \clift{P_0,\ldots,M_{P},\ldots,P_N}}
  \and \\
  \inferrule* [lab=process] {} {{M_{P}} \bc M_{N} \;| \;P|M_{P} }
\end{mathpar} 

\begin{mathpar}
  \inferrule* [lab=sychronization] {} {M_{N} \bc \Box \;|\; x?M_{F} \;|\; x!M_{C}}
  \and
  \inferrule* [lab=abstraction] {} {{M_{F}} \bc (x)M_{P} }
  \and
  \inferrule* [lab=concretion] {} {{M_{C}} \bc \langle M_{P} \rangle }
  \and \\
  \inferrule* [lab=process] {} {{M_{P}} \bc M_{N} \;| \;P|M_{P} }
\end{mathpar}

\begin{definition}[contextual application] Given a context $M$, and
  process $P$, we define the \emph{contextual application}, $M[P] :=
  M\{P/\Box\}$. That is, the contextual application of M to P is the
  substitution of $P$ for $\Box$ in $M$.
\end{definition}

$\meaningof{-} : L \to \mathcal{P}(\pi)$

\begin{mathpar}
  \inferrule* [lab=collection] {} {\meaningof{true} = \pi, \and \meaningof{~E} = \pi \setminus \meaningof{E}, \and \meaningof{E_{1} \& E_{2}} = \meaningof{E_{1}} \cap \meaningof{E_{2}}}
\end{mathpar}

\begin{mathpar}
  \inferrule* [lab=structure] {} {\meaningof{0} = \{ P \in \pi | P \equiv 0 \}, \and \\ \meaningof{E_1 | E_2} = \{ P \in \pi | P \equiv P_{1} | P_{2}, P_{1} \in \meaningof{E_{1}}, P_{2} \in \meaningof{E_2}\} }
\end{mathpar}

\begin{mathpar}
 \inferrule* [lab=behavior] {} {\meaningof{\langle a?b \rangle E} = \{ P \in \pi | P \equiv Q | u?(y)P', \\ \and \\\\ \and \\ \;\;\; u \in \meaningof{a}, \forall z.P'\{z/y\} \in \meaningof{E\{z/b\}}\}, \and \\ \meaningof{a!E} = \{ P \in \pi | P \equiv Q | x!\langle P' \rangle, x \in \meaningof{a} P' \in \meaningof{E}\} }
\end{mathpar}

\begin{mathpar}
 \inferrule* [lab=nominal] {} {\meaningof{\quotep{E}} = \{ \quotep{P} \in \quotep{\pi} | P \in \meaningof{E} \}, \and \meaningof{\quotep{P}} = \{ \quotep{Q} \in \quotep{\pi} | P \equiv Q \} \and \\ \meaningof{@\quotep{E}} = \{ P \in \pi | P \equiv @x, x \in \meaningof{E} \}}
\end{mathpar}

\begin{eqnarray*}
  \\
  \meaningof{-} : TS \to ST
\end{eqnarray*}

\begin{eqnarray*}
  \\
  L : TS \to ST
\end{eqnarray*}

\begin{eqnarray*}
  \\
  P \models E \iff P \in \meaningof{E}
\end{eqnarray*}

\begin{eqnarray*}
  P \approx_{L} Q \iff \forall E \in L. P \models E \iff Q \models E
\end{eqnarray*}

\begin{eqnarray*}
  P \approx_{K} Q
\end{eqnarray*}

\begin{eqnarray*}
  P \approx Q
\end{eqnarray*}

$\approx_{K} = \approx = \approx_{L}$

\subsubsection{Contextual duality}

Note that contexts extend the quotation operation to a family of
operations from processes to names. Given a context, $M$, we can
define a \emph{nominal context}, $\quotep{M}$ by $\quotep{M}[P] :=
\quotep{M[P]}$. To foreshadow what is to come we observe that these
operations enjoy a duality with processes very much like the duality
between vectors and maps from vectors to scalars.

Further, because the calculus is essentially higher-order, we have a
correspondence between contexts and processes. More specifically,
given a name $x$ and a context $M$ we can construct $M^{*}_{x}$ such
that 

\begin{mathpar}
  M^{*}_{x} | \lift{x}{P} \red M[P]
\end{mathpar}

namely,

\begin{mathpar}
  M^{*}_{x} := x?(u).M[\dropn{u}]
\end{mathpar}

The dependence of $M^{*}_{x}$ on a name makes it an abstraction, 

\begin{mathpar}
  M^{*} := (x)x?(u).M[\dropn{u}]
\end{mathpar}

\subsection{Additional notation}

It will sometimes be convenient to denote the process a name
quotes. We already have the notation $x = \quotep{P}$, but it will be
convenient to introduce an alternate notation, $\procn{x}$, when we
want to emphasize the connection to the use of the name. Note that, by
virtue of name equivalence, $\quotep{\procn{x}} \nameeq x$; so, the
notation is consistent with previous definitions.

Further, because names have structure it is possible to effect
substitutions on the basis of that structure. This means we need to
upgrade our notation for substitutions, which we accomplish by
adapting comprehension notation. Thus,

\begin{mathpar}
  P\{ y / x : x \in S \}
\end{mathpar}

is interpreted to mean the process derived from P by replacing (in a
capture-avoiding manner) each occurrence of $x$ in $S$ by $y$. For example,

\begin{mathpar}
  P\{ \quotep{\procn{x}|\procn{x}} / x : x \in \freenames{P} \}
\end{mathpar}

will replace each (occurrence) of a free name $x$ in $P$ by
$\quotep{\procn{x}|\procn{x}}$.

Also, we will avail ourselves of the notation $x^{L}$ and $x^{R}$ to
denote injections of a name into disjoint copies of the name
space. There are numerous ways to accomplish this. One example can be
found in \cite{MeredithR05}. This notation overloads to vectors of
names: $\vec{x}^{\pi} := (x_{i}^{\pi} \; : \; 0 \leq i < |\vec{x}| )$ where $\pi \in \{L,R\}$.

We also use $P^{\Box} := P|\Box$.

In \cite{MeredithR05} an interpretation of the new operator is
given. It turns out that there are several possible interpretations
all enjoying the requisite algebraic properties of the operator (see
\cite{milner91polyadicpi}). We will therefore make liberal use of
$(\nu\; \vec{x})P$.

% subsection the_syntax_and_semantics_of_the_notation_system (end)   

\input{qm2pi.qmops} 

\input{qm2pi.sterngerlach} 

\input{qm2pi.metric} 

% section concurrent_process_calculi (end)

%\input{qm2pi.proofsketch}

% section proof sketch (end)

%\input{qm2pi.slviaknots} 

% section spatial logic via knots (end)

\input{qm2pi.conclusion}

% section conclusion (end)

%\input{qm2pi.dtcodes} 

% section wiring algorithm (end)

\input{qm2pi.ack} 

% section acknowledgments (end)

\newpage


\bibliographystyle{plain}   
\bibliography{../../biblios/main.bib}

\input{qm2pi.rhodetails}

\end{document}

 

%\ifpdf
%\usepackage[pdftex]{graphicx}
%\else
%\usepackage{graphicx}
%\fi

 % \ifpdf
%  \usepackage{pdfsync}
%  \if


%\title{Brief Article}
%\author{David F. Snyder}
%\author{L.G. Meredith}

%\address{Dept. of Math., Texas State University--San Marcos, San Marcos, TX 78666}
       
\pagestyle{empty}


\begin{document}

\lstset{language=[Objective]Caml,frame=shadowbox}

\documentclass[12pt]{llncs}
%\documentclass{jktr}

\usepackage[pdftex]{hyperref}                   
\usepackage {listings}
\usepackage {mathpartir}
\usepackage{bcprules}
%\usepackage{listings}
                       
\usepackage{graphicx} 
%\usepackage[margins=2.5cm,nohead,nofoot]{geometry}
%\usepackage{geometry}
\usepackage{amsfonts}
\usepackage{amstext}
\usepackage{latexsym}
\usepackage{amssymb}
\usepackage{color}


%\include{myPreamble}
\include{qm2pi.local} 

%\ifpdf
%\usepackage[pdftex]{graphicx}
%\else
%\usepackage{graphicx}
%\fi

 % \ifpdf
%  \usepackage{pdfsync}
%  \if


%\title{Brief Article}
%\author{David F. Snyder}
%\author{L.G. Meredith}

%\address{Dept. of Math., Texas State University--San Marcos, San Marcos, TX 78666}
       
\pagestyle{empty}


\begin{document}

\lstset{language=[Objective]Caml,frame=shadowbox}

\input{qm2pi.front}

% section front matter (end)

\input{qm2pi.intro} 
 
% section introduction (end)

% \input{qm2pi.knotations} 

% section notation (end)

\input{qm2pi.process.calculi} 

% section concurrent_process_calculi_and_spatial_logics_ (end)
    
%\input{qm2pi.knots2pi} 

%\input{qm2pi.trefoil} 

%\input{qm2pi.mainthm} 

% subsection basic_interpretation (end)

%\input{qm2pi.rho.presentation} 
\subsection{The syntax and semantics of the notation system}\label{sub:the_syntax_and_semantics_of_the_notation_system} % (fold)

We now summarize a technical presentation of the calculus that
embodies our theory of dynamics. The typical presentation of such a
calculus follows the style of giving generators and relations on
them. The grammar, below, describing term constructors, freely
generates the set of processes, $\Proc$. This set is then quotiented
by a relation known as structural congruence and it is over this set
that the notion of dynamics is expressed. This presentation is
essentially that of \cite{MeredithR05} with the addition of
polyadicity and summation. For readability we have relegated some of
the technical subtleties to an appendix.

\subsubsection{Process grammar}\label{subsub:process_grammar}

\begin{mathpar}
  \inferrule* [lab=synchronization] {} {{M} \bc \pzero \;|\; x?F \;|\; x!C }
  \and
  \inferrule* [lab=abstraction] {} {{F} \bc (x)P}
  \and
  \inferrule* [lab=concretion] {} {{C} \bc \langle Q \rangle}
  \and
  \inferrule* [lab=process] {} {{P,Q} \bc M \;| \;P|Q \;|\; @{x}}
  \and
  \inferrule* [lab=name] {} {{x} \bc \quotep{P}}
\end{mathpar} 

Note that $\vec{x}$ (resp. $\vec{P}$) denotes a vector of names
(resp. processes) of length $|\vec{x}|$ (resp. $|\vec{P}|$). We adopt
the following useful abbreviations.

\begin{mathpar}
   x?(\vec{y}).P := x.(\vec{y})P \and  x\clift{\vec{P}} := x.\clift{\vec{P}}
   \and x!(y) := \lift{x}{\dropn{y}}
   \and \Pi_{i=0}^{n-1}P_i := P_0 | \ldots | P_{n-1}
\end{mathpar}

\subsubsection{Structural congruence}

\paragraph{Free and bound names and alpha-equivalence.} At the
core of structural equivalence is alpha-equivalence which identifies
process that are the same up to a change of variable. Formally, we
recognize the distinction between free and bound names. The free names
of a process, $\freenames{P}$, may be calculated recursively as
follows:

\begin{mathpar}
\freenames{\pzero} := \emptyset
  \and \\
  \freenames{x?(y).P} := \{ x \} \cup (\freenames{P} \setminus \{ y \})
  \and 
  \freenames{x!\langle P \rangle} := \{ x \} \cup \{ P \} 
  \and \\
  \freenames{P|Q} := \freenames{P} \cup \freenames{Q}
  \and \\
  \freenames{@{x}} := \{ x \}
\end{mathpar}

$\pi$
$\quotep{\pi}$

$\freenames{-} : \pi \to \mathcal{P}(\quotep{\pi})$

\begin{eqnarray*}
  \freenames{\pzero} & := & \emptyset \\
  \freenames{x?(y).P} & := & \{ x \} \cup (\freenames{P} \setminus \{ y \}) \\
  \freenames{x!\langle P \rangle} & := & \{ x \} \cup \{ P \} \\
  \freenames{P|Q} & := & \freenames{P} \cup \freenames{Q} \\
  \freenames{\dropn{x}} & := & \{ x \}
\end{eqnarray*}

The bound names of a process, $\boundnames{P}$, are those names occurring in $P$
that are not free. For example, in $x?(y).0$, the name $x$ is free, while $y$ is bound.

\begin{mathpar}
  \inferrule* [lab=monoidal-laws] {} { P|Q \equiv Q|P \and P|0 \equiv P \and P|(Q|R) \equiv (P|Q)|R }
\end{mathpar}

\begin{mathpar}
  \inferrule* [lab=alpha-equivalence] {} { (x)P \equiv (y)P\{y/x\} \and y \not\in \freenames{P} }
\end{mathpar}

\begin{definition}
Then two processes, $P,Q$, are alpha-equivalent if $P = Q\{\vec{y}/\vec{x}\}$ for
some $\vec{x} \in \boundnames{Q},\vec{y} \in \boundnames{P}$, where $Q\{\vec{y}/\vec{x}\}$
denotes the capture-avoiding substitution of $\vec{y}$ for $\vec{x}$ in $Q$.
\end{definition}

\begin{definition}
  The {\em structural congruence} \cite{SangiorgiWalker} , $\equiv$,
  between processes is the least congruence containing
  alpha-equivalence, satisfying the abelian monoid laws
  (associativity, commutativity and $\pzero$ as identity) for parallel
  composition $|$ and for summation $+$.
\end{definition}

\subsection{Name equivalence}

We take name equivalence, written $\nameeq$, to be the smallest
equivalence relation generated by the following rules.

\begin{mathpar}
\inferrule*[lab=Quote-drop]
{ }
{ \quotep{@{x}} \nameeq x }

\inferrule*[lab=Struct-equiv]
{ P \scong Q }
{ \quotep{P} \nameeq \quotep{Q} }
\end{mathpar}

The astute reader will have noticed that the mutual recursion of names
and processes imposes a mutual recursion on alpha-equivalence and
structural equivalence via name-equivalence. Fortunately, all of this
works out pleasantly and we may calculate in the natural way, free of
concern. The reader interested in the details is referred to the
appendix \ref{appendix:rho_details}.

\subsection{Substitution}

We use $\Proc$ for the set of processes, $\QProc$ for the set of
names, and $\id{\{}\vec{y} / \vec{x} \id{\}}$ to denote partial maps,
$s : \QProc \rightarrow \QProc$. A map, $s$ lifts, uniquely, to a map
on process terms, $\widehat{s} : \Proc \rightarrow \Proc$ by the
following equations.

\begin{mathpar}
  (0) \psubstp{Q}{P} := 0 \\
  (R \juxtap S) \psubstp{Q}{P}
  :=    
  (R)\psubstp{Q}{P} \juxtap (S) \psubstp{Q}{P} \\
  (x?(y).R) \psubstp{Q}{P}    
  :=    
  (x)\substp{Q}{P} (z)\concat( (R \psubstn{z}{y}) \psubstp{Q}{P} ) \\
  (\lift{x}{R}) \psubstp{Q}{P}  
  :=
  \lift{(x)\substp{Q}{P}}{ R \psubstp{Q}{P} } \\
%   (\dropn{x})  \psubstp{Q}{P}       
%   := 
%   \left\{ 
%     \begin{array}{ccc} 
%       \dropn{\quotep{Q}} & & x \nameeq \quotep{P} \\
%       \dropn{x} & & otherwise \\
%     \end{array}
%   \right. 
  (\dropn{x})  \psubstp{Q}{P}       
  := 
  \left\{ 
    \begin{array}{ccc} 
      Q & & x \nameeq \quotep{P} \\
      \dropn{x} & & otherwise \\
    \end{array}
  \right.
\end{mathpar}
 

where

\begin{eqnarray}
  (x)\id{\{} \lpquote Q \rpquote / \lpquote P \rpquote \id{\}}            = 
  \left\{ 
    \begin{array}{ccc}
      \lpquote Q \rpquote & & x \nameeq \lpquote P \rpquote \\
      x & & otherwise \\
    \end{array}
  \right. \nonumber
\end{eqnarray}

and $z$ is chosen distinct from $\quotep{P}$, $\quotep{Q}$, the free
names in $Q$, and all the names in $R$. Our $\alpha$-equivalence will
be built in the standard way from this substitution.

\begin{remark}\label{rem:no_self_referential_names}
  One consequence of these definitions is that $\forall P. \quotep{P}
  \not\in \freenames{P}$.
\end{remark}

\subsection{ Dynamic quote: an example }

Anticipating something of what's to come, consider applying the
substitution, $\widehat{\id{\{}u / z \id{\}}}$, to the following pair
of processes, $\lift{w}{y!(z)}$ and $w[ \lpquote y!(z) \rpquote ]$.

\begin{eqnarray}
	\lift{w}{y!(z)}\widehat{\id{\{}u / z \id{\}}}
		& = &
		\lift{w}{y!(u)} \nonumber\\
	w[ \lpquote y!(z) \rpquote ] \widehat{ \id{\{}u / z \id{\}} }
		& = &
		w[ \lpquote y!(z) \rpquote ] \nonumber
\end{eqnarray}

Because the body of the process between quotes is impervious to
substitution, we get radically different answers. In fact, by
examining the first process in an input context,
e.g. $x?(z).\lift{w}{y!(z)}$, we see that the process under the lift
operator may be shaped by prefixed inputs binding a name inside it. In
this sense, the lift operator will be seen as a way to dynamically
construct processes before reifying them as names.

Finally equipped with these standard features we can present the
dynamics of the calculus.

\subsubsection{Operational semantics} 

Finally, we introduce the computational dynamics. What marks these
algebras as distinct from other more traditionally studied algebraic
structures, e.g. vector spaces or polynomial rings, is the manner in
which dynamics is captured. In traditional structures, dynamics is typically
expressed through morphisms between such structures, as in linear maps
between vector spaces or morphisms between rings. In algebras
associated with the semantics of computation, the dynamics is
expressed as part of the algebraic structure itself, through a
reduction reduction relation typically denoted by $\red$. Below, we
give a recursive presentation of this relation for the calculus used
in the encoding.

$\red \subseteq \pi \times \pi$
$\red : \pi \to \mathcal{P}(\pi)$

\begin{mathpar}
  \inferrule* [lab=Comm] { \textsf{match}( x_{src}, x_{trgt} ) } { x_{trgt}?(y)P \; | \; x_{src}!\langle {Q} \rangle \red P\{\quotep{Q}/y}\} }
  \and \\
  \inferrule* [lab=Par] {{P} \red {P}'} {{{P} | {Q}} \red {{P}' | {Q}}}
  \and
  \inferrule* [lab=Equiv]{{{P} \scong {P}'} \andalso {{P}' \red {Q}'} \andalso {{Q}' \scong {Q}}}{{P} \red {Q}}
\end{mathpar}

\begin{eqnarray*}
  match_{\equiv} (\quotep{P},\quotep{Q}) & := & P \equiv Q \\
  match_{\dagger}(\quotep{P},\quotep{Q}) & := & \forall R. P|Q \red^{*} R => R \red^{*} 0 \\
  match_{K}(\quotep{P},\quotep{Q}) & := & K \mbox{ for some context } K
\end{eqnarray*}

$u?(x)P | u!\langle Q \rangle \red P\{\quotep{Q}/x\}$

%We write $\wred$ for $\red^*$, and $P\red$ if $\exists Q $ such that $ P \red Q$.
We write $P\red$ if $\exists Q $ such that $ P \red Q$ and $P\not\red$, otherwise.

\section{Replication}

As mentioned before, it is known that replication (and hence
recursion) can be implemented in a higher-order process algebra
\cite{SangiorgiWalker}. As our first example of calculation with the
machinery thus far presented we give the construction explicitly in
the {\rhoc}.

\begin{eqnarray}
	D_{x} & := & \prefix{x}{y}{(\binpar{\outputp{x}{y}}{@{y}})} \nonumber\\
	\bangp_{x}{P} & := & \binpar{{x}!\langle{\binpar{D_{x}}{P}}\rangle}{D_{x}} \nonumber
\end{eqnarray}

\begin{eqnarray}
	\bangp_{x}{P} & & \nonumber\\
	=
	& {x}!\langle{(\prefix{x}{y}{(\outputp{x}{y} | @{y})) | P}}\rangle 
	      | \prefix{x}{y}{(\outputp{x}{y} | @{y})} & \nonumber\\
	\red
	& (\outputp{x}{y} | @{y})\substn{\quotep{(\prefix{x}{y}{(@{y} | \outputp{x}{y})) | P}}}{y} & \nonumber\\
	=
	& \outputp{x}{\quotep{(\prefix{x}{y}{(\outputp{x}{y} | @{y})) | P}}}
	  | {(\prefix{x}{y}{(\outputp{x}{y} | @{y})) | P}} & \nonumber\\
	\red
	& \ldots & \nonumber\\
	\red^*
	& P | P | \ldots & \nonumber
\end{eqnarray}

Of course, this encoding, as an implementation, runs away, unfolding
$\bangp{P}$ eagerly. A lazier and more implementable replication
operator, restricted to input-guarded processes, may be obtained as follows.

\begin{eqnarray}
\bangp{\prefix{u}{v}{P}} 
	:= 
	\binpar{\lift{x}{\prefix{u}{v}{(\binpar{D(x)}{P})}}}{D(x)} \nonumber
\end{eqnarray}

\begin{remark}
  Note that the lazier definition still does not deal with summation
  or mixed summation (i.e. sums over input and output). The reader is
  invited to construct definitions of replication that deal with these
  features. 

  Further, the definitions are parameterized in a name, $x$. Can you,
  gentle reader, make a definition that eliminates this parameter and
  guarantees no accidental interaction between the replication
  machinery and the process being replicated -- i.e. no accidental
  sharing of names used by the process to get its work done and the
  name(s) used by the replication to effect copying. This latter
  revision of the definition of replication is crucial to obtaining
  the expected identity $!!P \sim !P$.
\end{remark}

\begin{remark}\label{rem:paradoxical_combinator}
  The reader familiar with the lambda calculus will have noticed the
  similarity between $D$ and the paradoxical combinator.

  [Ed. note: the existence of this seems to suggest we have to be more
  restrictive on the set of processes and names we admit if we are to
  support no-cloning.]
\end{remark}

\subsubsection{Bisimulation}

The computational dynamics gives rise to another kind of equivalence,
the equivalence of computational behavior. As previously mentioned
this is typically captured \emph{via} some form of bisimulation.

% The notion we use in this paper is weak barbed bisimulation
% \cite{milner91polyadicpi}.

The notion we use in this paper is derived from weak barbed
bisimulation \cite{milner91polyadicpi}. 

\begin{definition}
An \emph{observation relation}, $\downarrow_{\mathcal N}$, over a set
of names, $\mathcal N$, is the smallest relation satisfying the rules
below.

\infrule[Out-barb]{y \in {\mathcal N}, \; x \nameeq y}
		  {\outputp{x}{v} \downarrow_{\mathcal N} x}
\infrule[Par-barb]{\mbox{$P\downarrow_{\mathcal N} x$ or $Q\downarrow_{\mathcal N} x$}}
		  {\binpar{P}{Q} \downarrow_{\mathcal N} x}

We write $P \Downarrow_{\mathcal N} x$ if there is $Q$ such that 
$P \wred Q$ and $Q \downarrow_{\mathcal N} x$.
\end{definition}

\begin{definition}
%\label{def.bbisim}
An  ${\mathcal N}$-\emph{barbed bisimulation} over a set of names, ${\mathcal N}$, is a symmetric binary relation 
${\mathcal S}_{\mathcal N}$ between agents such that $P\rel{S}_{\mathcal N}Q$ implies:
\begin{enumerate}
\item If $P \red P'$ then $Q \wred Q'$ and $P'\rel{S}_{\mathcal N} Q'$.
\item If $P\downarrow_{\mathcal N} x$, then $Q\Downarrow_{\mathcal N} x$.
\end{enumerate}
$P$ is ${\mathcal N}$-barbed bisimilar to $Q$, written
$P \wbbisim_{\mathcal N} Q$, if $P \rel{S}_{\mathcal N} Q$ for some ${\mathcal N}$-barbed bisimulation ${\mathcal S}_{\mathcal N}$.
\end{definition}

$\mathcal{R} \subseteq \pi \times \pi$

$P \mathcal{R} Q => \forall P'. P \red P' \Rightarrow \exists Q'. Q \red Q', P' \mathcal{R} Q'$

$P \vdash x \Rightarrow Q \vdash x$

\begin{mathpar}
  \inferrule*[lab=Out-barb]{x \nameeq y}{{y}!\langle{Q}\rangle \vdash x}
  \and
  \inferrule*[lab=Par-barb]{\mbox{$P\vdash x$ or $Q\vdash x$}}{\binpar{P}{Q} \vdash x}
\end{mathpar}

\subsubsection{Contexts}

One of the principle advantages of computational calculi like the
$\pi$-calculus is a well-defined notion of context,
contextual-equivalence and a correlation between
contextual-equivalence and notions of bisimulation. The notion of
context allows the decomposition of a process into (sub-)process and
its syntactic environment, its context. Thus, a context may be
thought of as a process with a ``hole'' (written $\Box$) in it. The
application of a context $M$ to a process $P$, written $M[P]$, is
tantamount to filling the hole in $M$ with $P$. In this paper we do
not need the full weight of this theory, but do make use of the notion
of context in the proof the main theorem. 

\begin{mathpar}
  \inferrule* [lab=summation] {} {{M_{M},M_{N}} \bc \Box \;|\; x.M_{A} \;|\; M_{M}+M_{N}}
  \and
  \inferrule* [lab=agent] {} {{M_{A}} \bc (\vec{x})M_{P} \;| \; \clift{P_0,\ldots,M_{P},\ldots,P_N}}
  \and \\
  \inferrule* [lab=process] {} {{M_{P}} \bc M_{N} \;| \;P|M_{P} }
\end{mathpar} 

\begin{mathpar}
  \inferrule* [lab=sychronization] {} {M_{N} \bc \Box \;|\; x?M_{F} \;|\; x!M_{C}}
  \and
  \inferrule* [lab=abstraction] {} {{M_{F}} \bc (x)M_{P} }
  \and
  \inferrule* [lab=concretion] {} {{M_{C}} \bc \langle M_{P} \rangle }
  \and \\
  \inferrule* [lab=process] {} {{M_{P}} \bc M_{N} \;| \;P|M_{P} }
\end{mathpar}

\begin{definition}[contextual application] Given a context $M$, and
  process $P$, we define the \emph{contextual application}, $M[P] :=
  M\{P/\Box\}$. That is, the contextual application of M to P is the
  substitution of $P$ for $\Box$ in $M$.
\end{definition}

$\meaningof{-} : L \to \mathcal{P}(\pi)$

\begin{mathpar}
  \inferrule* [lab=collection] {} {\meaningof{true} = \pi, \and \meaningof{~E} = \pi \setminus \meaningof{E}, \and \meaningof{E_{1} \& E_{2}} = \meaningof{E_{1}} \cap \meaningof{E_{2}}}
\end{mathpar}

\begin{mathpar}
  \inferrule* [lab=structure] {} {\meaningof{0} = \{ P \in \pi | P \equiv 0 \}, \and \\ \meaningof{E_1 | E_2} = \{ P \in \pi | P \equiv P_{1} | P_{2}, P_{1} \in \meaningof{E_{1}}, P_{2} \in \meaningof{E_2}\} }
\end{mathpar}

\begin{mathpar}
 \inferrule* [lab=behavior] {} {\meaningof{\langle a?b \rangle E} = \{ P \in \pi | P \equiv Q | u?(y)P', \\ \and \\\\ \and \\ \;\;\; u \in \meaningof{a}, \forall z.P'\{z/y\} \in \meaningof{E\{z/b\}}\}, \and \\ \meaningof{a!E} = \{ P \in \pi | P \equiv Q | x!\langle P' \rangle, x \in \meaningof{a} P' \in \meaningof{E}\} }
\end{mathpar}

\begin{mathpar}
 \inferrule* [lab=nominal] {} {\meaningof{\quotep{E}} = \{ \quotep{P} \in \quotep{\pi} | P \in \meaningof{E} \}, \and \meaningof{\quotep{P}} = \{ \quotep{Q} \in \quotep{\pi} | P \equiv Q \} \and \\ \meaningof{@\quotep{E}} = \{ P \in \pi | P \equiv @x, x \in \meaningof{E} \}}
\end{mathpar}

\begin{eqnarray*}
  \\
  \meaningof{-} : TS \to ST
\end{eqnarray*}

\begin{eqnarray*}
  \\
  L : TS \to ST
\end{eqnarray*}

\begin{eqnarray*}
  \\
  P \models E \iff P \in \meaningof{E}
\end{eqnarray*}

\begin{eqnarray*}
  P \approx_{L} Q \iff \forall E \in L. P \models E \iff Q \models E
\end{eqnarray*}

\begin{eqnarray*}
  P \approx_{K} Q
\end{eqnarray*}

\begin{eqnarray*}
  P \approx Q
\end{eqnarray*}

$\approx_{K} = \approx = \approx_{L}$

\subsubsection{Contextual duality}

Note that contexts extend the quotation operation to a family of
operations from processes to names. Given a context, $M$, we can
define a \emph{nominal context}, $\quotep{M}$ by $\quotep{M}[P] :=
\quotep{M[P]}$. To foreshadow what is to come we observe that these
operations enjoy a duality with processes very much like the duality
between vectors and maps from vectors to scalars.

Further, because the calculus is essentially higher-order, we have a
correspondence between contexts and processes. More specifically,
given a name $x$ and a context $M$ we can construct $M^{*}_{x}$ such
that 

\begin{mathpar}
  M^{*}_{x} | \lift{x}{P} \red M[P]
\end{mathpar}

namely,

\begin{mathpar}
  M^{*}_{x} := x?(u).M[\dropn{u}]
\end{mathpar}

The dependence of $M^{*}_{x}$ on a name makes it an abstraction, 

\begin{mathpar}
  M^{*} := (x)x?(u).M[\dropn{u}]
\end{mathpar}

\subsection{Additional notation}

It will sometimes be convenient to denote the process a name
quotes. We already have the notation $x = \quotep{P}$, but it will be
convenient to introduce an alternate notation, $\procn{x}$, when we
want to emphasize the connection to the use of the name. Note that, by
virtue of name equivalence, $\quotep{\procn{x}} \nameeq x$; so, the
notation is consistent with previous definitions.

Further, because names have structure it is possible to effect
substitutions on the basis of that structure. This means we need to
upgrade our notation for substitutions, which we accomplish by
adapting comprehension notation. Thus,

\begin{mathpar}
  P\{ y / x : x \in S \}
\end{mathpar}

is interpreted to mean the process derived from P by replacing (in a
capture-avoiding manner) each occurrence of $x$ in $S$ by $y$. For example,

\begin{mathpar}
  P\{ \quotep{\procn{x}|\procn{x}} / x : x \in \freenames{P} \}
\end{mathpar}

will replace each (occurrence) of a free name $x$ in $P$ by
$\quotep{\procn{x}|\procn{x}}$.

Also, we will avail ourselves of the notation $x^{L}$ and $x^{R}$ to
denote injections of a name into disjoint copies of the name
space. There are numerous ways to accomplish this. One example can be
found in \cite{MeredithR05}. This notation overloads to vectors of
names: $\vec{x}^{\pi} := (x_{i}^{\pi} \; : \; 0 \leq i < |\vec{x}| )$ where $\pi \in \{L,R\}$.

We also use $P^{\Box} := P|\Box$.

In \cite{MeredithR05} an interpretation of the new operator is
given. It turns out that there are several possible interpretations
all enjoying the requisite algebraic properties of the operator (see
\cite{milner91polyadicpi}). We will therefore make liberal use of
$(\nu\; \vec{x})P$.

% subsection the_syntax_and_semantics_of_the_notation_system (end)   

\input{qm2pi.qmops} 

\input{qm2pi.sterngerlach} 

\input{qm2pi.metric} 

% section concurrent_process_calculi (end)

%\input{qm2pi.proofsketch}

% section proof sketch (end)

%\input{qm2pi.slviaknots} 

% section spatial logic via knots (end)

\input{qm2pi.conclusion}

% section conclusion (end)

%\input{qm2pi.dtcodes} 

% section wiring algorithm (end)

\input{qm2pi.ack} 

% section acknowledgments (end)

\newpage


\bibliographystyle{plain}   
\bibliography{../../biblios/main.bib}

\input{qm2pi.rhodetails}

\end{document}



% section front matter (end)

\section{Introduction}\label{sec:introduction} % (fold)
In this draft of the material i am going to have to dispense with the
usual writing conventions adopted in papers on these topics. i'm going
to have adopt whatever tone i need at the time i'm writing up the
calculations. Sometimes this may be very conversational; others it may
be the barest mathematical grunts; others still it may be that i have
lifted text from one of my other papers because the exposition of some
point was better said there. i hope that my readers are not unduly put
out by this decision. i'm not doing this to flout convention or be
rebellious. i find these calculations very technically challenging. To
keep everything going technically, something has to give; i have to
let go of some cognitive burden. So, the academic writing style --
with all of its trade-offs in terms of facilitating technical
communication -- is what i'm letting go of. Perhaps subsequent drafts
can be tightened and polished, but for now, i'm going to speak as if
we were sitting together in a coffee shop with a laptop, wifi and a
pad of paper and a pencil.

So, here's what i have to say. We -- you and i, comfortably ensconced
in our coffee shop and well-equipped with our tools -- can realize and
carry out the calculations of quantum mechanics over a very different
formal theory of dynamics, a formal theory of dynamics that
corresponds to a theory of concurrent computation with
\emph{reflection}. It has the advantage that the underlying theory is
already `quantized', but supports analogues all of the continuuous
operations. Strikingly, this underlying theory has recently been
connected with a notion of metric that we can show, by calculating
together, coincides with the metric induced by the inner product.

There are a lot of reasons why you might be interested in seeing
calculations of this form. Here's why i'm interested. For the past
several centuries there has been no competitor to the ``Newtonian''
account of dynamics. As a result the predominant share of accounts of
dynamical systems and situations have had to be formulated in terms of
the Newtonian machinery. i view this as an intellectually dangerous
position to occupy. Everything, despite it's intrinsic shape, turns
into a nail to be hit with this hammer. Recently, however, the theory
of computation has matured to the point where we have candidates for
theories of dynamics that offer very different perspective on
reasoning about dynamical systems and situations. Testing these
candidates against very successful accounts of dynamical situations,
like quantum mechanics, is going to give us some sense of how mature
they are and some measure of the quality of these accounts of
dynamics.

\subsection{Summary of contributions and outline of paper}

So, we're going to develop an interpretation of the operations of
quantum mechanics normally interpreted by Hilbert spaces and
operators. We're going to do this over a theory of computation. Note
that this is very different than the usual quantum computation program
which develops notions of computation over quantum mechanics. Rather,
we are developing a story that aligns with Wheeler's slogan: It from
Bit. To do this we will first provide an account of the theory of
computation at play here. Then we will dive into a calculation-driven
interpretation of the operations of quantum mechanics.

The reason we take this approach is that -- until very recently --
there hasn't been an axiomatic account of quantum mechanics. As a
result there has been no sharp delineation of the mathematical theory
supporting interpretation of the physical theory and the physical
theory, itself. So, ambient features of the maths are free to be
exploited (or supressed) without a real accounting of their physical
relevance. There is no sharp statement ``here's the physical theory''
qua \emph{theory} and ``here's the mathematical interpretation''
enabling a judgment of how faithful the interpretation is -- apart
from experimental observation. When there is an axiomatic account we
can judge how well a given mathematical formalism supports an
interpretation of the axioms, independent of
experimentation. Likewise, we can judge how well we have captured our
physical evidence and experience with our axiomatics, independent of
any specific mathematical implementation, with accidental detail that
may or may not have physical significance. 

In lieu of a fully fleshed out and vetted axiomatic account of quantum
mechanics, interpreting the operational notions in service of modeling
physical systems will have to suffice. In other words, we are not in
the business of providing a model of Hilbert spaces and operators. We
are in the business of providing a model of quantum mechanics because
we are motivated by testing our notions of dynamics against physical
theory; and, the predictive calculations of the physical theory must
serve as the best formulation -- shy of a fully fleshed out axiomatic
account -- of the physical theory itself (as they have for scientific
theories since time immemorial). Put another way, despite a
whole-hearted commitment to an It-from-Bit ontology, we are firmly
aligned with the shut-up-and-calculate camp as the best way to obtain
results either from the physical perspective or as a quality assurance
measure of our fledgling theory of dynamics.

In detail, we present a reflective process calculus. Then we develop
intuitive correspondences between the notions available in this
calculus and the usual physical notions supporting quantum mechanical
calculations. Thus, 

\begin{table}[htp]
  \center{
    \fbox{
      \begin{tabular}{c|c}
        quantum mechanics & process calculus \\
        \hline
        scalar & name \\
        state vector & process \\
        dual & contextual duals \\
        matrix & formal sums of process-context-dual pairs \\
        orthogonality & process annihilation \\
        inner product & execution-formula + quoting
      \end{tabular}
    }
  }
  \caption{QM - process calculi correspondences}
\end{table}

Then we tighten up these intuitions to operational definitions. We
employ the Dirac notation as the best proxy we can find for an
abstract syntax of the quantum mechanical notions. The definitions we
develop put us in contact with equational constraints coming from the
theory that we demonstrate the definitions and calculations satisfy.

This puts us in a position to shut up and calculate for the
Stern-Gerlach experimental set up, showing how these predictive
calculations become calculations on processes in our theory of a
reflective process calculus.

Penultimately, we demonstrate that the notion of metric coming from
the inner product coincides with the notion of metric available from
the theory of bisimulation. This demonstration gives us the right to
think of space as arising from behavior. Finally, we consider where we
might go from the new vantage point we have obtained.

% section introduction (end) 
 
% section introduction (end)

% \documentclass[12pt]{llncs}
%\documentclass{jktr}

\usepackage[pdftex]{hyperref}                   
\usepackage {listings}
\usepackage {mathpartir}
\usepackage{bcprules}
%\usepackage{listings}
                       
\usepackage{graphicx} 
%\usepackage[margins=2.5cm,nohead,nofoot]{geometry}
%\usepackage{geometry}
\usepackage{amsfonts}
\usepackage{amstext}
\usepackage{latexsym}
\usepackage{amssymb}
\usepackage{color}


%\include{myPreamble}
\include{qm2pi.local} 

%\ifpdf
%\usepackage[pdftex]{graphicx}
%\else
%\usepackage{graphicx}
%\fi

 % \ifpdf
%  \usepackage{pdfsync}
%  \if


%\title{Brief Article}
%\author{David F. Snyder}
%\author{L.G. Meredith}

%\address{Dept. of Math., Texas State University--San Marcos, San Marcos, TX 78666}
       
\pagestyle{empty}


\begin{document}

\lstset{language=[Objective]Caml,frame=shadowbox}

\input{qm2pi.front}

% section front matter (end)

\input{qm2pi.intro} 
 
% section introduction (end)

% \input{qm2pi.knotations} 

% section notation (end)

\input{qm2pi.process.calculi} 

% section concurrent_process_calculi_and_spatial_logics_ (end)
    
%\input{qm2pi.knots2pi} 

%\input{qm2pi.trefoil} 

%\input{qm2pi.mainthm} 

% subsection basic_interpretation (end)

%\input{qm2pi.rho.presentation} 
\subsection{The syntax and semantics of the notation system}\label{sub:the_syntax_and_semantics_of_the_notation_system} % (fold)

We now summarize a technical presentation of the calculus that
embodies our theory of dynamics. The typical presentation of such a
calculus follows the style of giving generators and relations on
them. The grammar, below, describing term constructors, freely
generates the set of processes, $\Proc$. This set is then quotiented
by a relation known as structural congruence and it is over this set
that the notion of dynamics is expressed. This presentation is
essentially that of \cite{MeredithR05} with the addition of
polyadicity and summation. For readability we have relegated some of
the technical subtleties to an appendix.

\subsubsection{Process grammar}\label{subsub:process_grammar}

\begin{mathpar}
  \inferrule* [lab=synchronization] {} {{M} \bc \pzero \;|\; x?F \;|\; x!C }
  \and
  \inferrule* [lab=abstraction] {} {{F} \bc (x)P}
  \and
  \inferrule* [lab=concretion] {} {{C} \bc \langle Q \rangle}
  \and
  \inferrule* [lab=process] {} {{P,Q} \bc M \;| \;P|Q \;|\; @{x}}
  \and
  \inferrule* [lab=name] {} {{x} \bc \quotep{P}}
\end{mathpar} 

Note that $\vec{x}$ (resp. $\vec{P}$) denotes a vector of names
(resp. processes) of length $|\vec{x}|$ (resp. $|\vec{P}|$). We adopt
the following useful abbreviations.

\begin{mathpar}
   x?(\vec{y}).P := x.(\vec{y})P \and  x\clift{\vec{P}} := x.\clift{\vec{P}}
   \and x!(y) := \lift{x}{\dropn{y}}
   \and \Pi_{i=0}^{n-1}P_i := P_0 | \ldots | P_{n-1}
\end{mathpar}

\subsubsection{Structural congruence}

\paragraph{Free and bound names and alpha-equivalence.} At the
core of structural equivalence is alpha-equivalence which identifies
process that are the same up to a change of variable. Formally, we
recognize the distinction between free and bound names. The free names
of a process, $\freenames{P}$, may be calculated recursively as
follows:

\begin{mathpar}
\freenames{\pzero} := \emptyset
  \and \\
  \freenames{x?(y).P} := \{ x \} \cup (\freenames{P} \setminus \{ y \})
  \and 
  \freenames{x!\langle P \rangle} := \{ x \} \cup \{ P \} 
  \and \\
  \freenames{P|Q} := \freenames{P} \cup \freenames{Q}
  \and \\
  \freenames{@{x}} := \{ x \}
\end{mathpar}

$\pi$
$\quotep{\pi}$

$\freenames{-} : \pi \to \mathcal{P}(\quotep{\pi})$

\begin{eqnarray*}
  \freenames{\pzero} & := & \emptyset \\
  \freenames{x?(y).P} & := & \{ x \} \cup (\freenames{P} \setminus \{ y \}) \\
  \freenames{x!\langle P \rangle} & := & \{ x \} \cup \{ P \} \\
  \freenames{P|Q} & := & \freenames{P} \cup \freenames{Q} \\
  \freenames{\dropn{x}} & := & \{ x \}
\end{eqnarray*}

The bound names of a process, $\boundnames{P}$, are those names occurring in $P$
that are not free. For example, in $x?(y).0$, the name $x$ is free, while $y$ is bound.

\begin{mathpar}
  \inferrule* [lab=monoidal-laws] {} { P|Q \equiv Q|P \and P|0 \equiv P \and P|(Q|R) \equiv (P|Q)|R }
\end{mathpar}

\begin{mathpar}
  \inferrule* [lab=alpha-equivalence] {} { (x)P \equiv (y)P\{y/x\} \and y \not\in \freenames{P} }
\end{mathpar}

\begin{definition}
Then two processes, $P,Q$, are alpha-equivalent if $P = Q\{\vec{y}/\vec{x}\}$ for
some $\vec{x} \in \boundnames{Q},\vec{y} \in \boundnames{P}$, where $Q\{\vec{y}/\vec{x}\}$
denotes the capture-avoiding substitution of $\vec{y}$ for $\vec{x}$ in $Q$.
\end{definition}

\begin{definition}
  The {\em structural congruence} \cite{SangiorgiWalker} , $\equiv$,
  between processes is the least congruence containing
  alpha-equivalence, satisfying the abelian monoid laws
  (associativity, commutativity and $\pzero$ as identity) for parallel
  composition $|$ and for summation $+$.
\end{definition}

\subsection{Name equivalence}

We take name equivalence, written $\nameeq$, to be the smallest
equivalence relation generated by the following rules.

\begin{mathpar}
\inferrule*[lab=Quote-drop]
{ }
{ \quotep{@{x}} \nameeq x }

\inferrule*[lab=Struct-equiv]
{ P \scong Q }
{ \quotep{P} \nameeq \quotep{Q} }
\end{mathpar}

The astute reader will have noticed that the mutual recursion of names
and processes imposes a mutual recursion on alpha-equivalence and
structural equivalence via name-equivalence. Fortunately, all of this
works out pleasantly and we may calculate in the natural way, free of
concern. The reader interested in the details is referred to the
appendix \ref{appendix:rho_details}.

\subsection{Substitution}

We use $\Proc$ for the set of processes, $\QProc$ for the set of
names, and $\id{\{}\vec{y} / \vec{x} \id{\}}$ to denote partial maps,
$s : \QProc \rightarrow \QProc$. A map, $s$ lifts, uniquely, to a map
on process terms, $\widehat{s} : \Proc \rightarrow \Proc$ by the
following equations.

\begin{mathpar}
  (0) \psubstp{Q}{P} := 0 \\
  (R \juxtap S) \psubstp{Q}{P}
  :=    
  (R)\psubstp{Q}{P} \juxtap (S) \psubstp{Q}{P} \\
  (x?(y).R) \psubstp{Q}{P}    
  :=    
  (x)\substp{Q}{P} (z)\concat( (R \psubstn{z}{y}) \psubstp{Q}{P} ) \\
  (\lift{x}{R}) \psubstp{Q}{P}  
  :=
  \lift{(x)\substp{Q}{P}}{ R \psubstp{Q}{P} } \\
%   (\dropn{x})  \psubstp{Q}{P}       
%   := 
%   \left\{ 
%     \begin{array}{ccc} 
%       \dropn{\quotep{Q}} & & x \nameeq \quotep{P} \\
%       \dropn{x} & & otherwise \\
%     \end{array}
%   \right. 
  (\dropn{x})  \psubstp{Q}{P}       
  := 
  \left\{ 
    \begin{array}{ccc} 
      Q & & x \nameeq \quotep{P} \\
      \dropn{x} & & otherwise \\
    \end{array}
  \right.
\end{mathpar}
 

where

\begin{eqnarray}
  (x)\id{\{} \lpquote Q \rpquote / \lpquote P \rpquote \id{\}}            = 
  \left\{ 
    \begin{array}{ccc}
      \lpquote Q \rpquote & & x \nameeq \lpquote P \rpquote \\
      x & & otherwise \\
    \end{array}
  \right. \nonumber
\end{eqnarray}

and $z$ is chosen distinct from $\quotep{P}$, $\quotep{Q}$, the free
names in $Q$, and all the names in $R$. Our $\alpha$-equivalence will
be built in the standard way from this substitution.

\begin{remark}\label{rem:no_self_referential_names}
  One consequence of these definitions is that $\forall P. \quotep{P}
  \not\in \freenames{P}$.
\end{remark}

\subsection{ Dynamic quote: an example }

Anticipating something of what's to come, consider applying the
substitution, $\widehat{\id{\{}u / z \id{\}}}$, to the following pair
of processes, $\lift{w}{y!(z)}$ and $w[ \lpquote y!(z) \rpquote ]$.

\begin{eqnarray}
	\lift{w}{y!(z)}\widehat{\id{\{}u / z \id{\}}}
		& = &
		\lift{w}{y!(u)} \nonumber\\
	w[ \lpquote y!(z) \rpquote ] \widehat{ \id{\{}u / z \id{\}} }
		& = &
		w[ \lpquote y!(z) \rpquote ] \nonumber
\end{eqnarray}

Because the body of the process between quotes is impervious to
substitution, we get radically different answers. In fact, by
examining the first process in an input context,
e.g. $x?(z).\lift{w}{y!(z)}$, we see that the process under the lift
operator may be shaped by prefixed inputs binding a name inside it. In
this sense, the lift operator will be seen as a way to dynamically
construct processes before reifying them as names.

Finally equipped with these standard features we can present the
dynamics of the calculus.

\subsubsection{Operational semantics} 

Finally, we introduce the computational dynamics. What marks these
algebras as distinct from other more traditionally studied algebraic
structures, e.g. vector spaces or polynomial rings, is the manner in
which dynamics is captured. In traditional structures, dynamics is typically
expressed through morphisms between such structures, as in linear maps
between vector spaces or morphisms between rings. In algebras
associated with the semantics of computation, the dynamics is
expressed as part of the algebraic structure itself, through a
reduction reduction relation typically denoted by $\red$. Below, we
give a recursive presentation of this relation for the calculus used
in the encoding.

$\red \subseteq \pi \times \pi$
$\red : \pi \to \mathcal{P}(\pi)$

\begin{mathpar}
  \inferrule* [lab=Comm] { \textsf{match}( x_{src}, x_{trgt} ) } { x_{trgt}?(y)P \; | \; x_{src}!\langle {Q} \rangle \red P\{\quotep{Q}/y}\} }
  \and \\
  \inferrule* [lab=Par] {{P} \red {P}'} {{{P} | {Q}} \red {{P}' | {Q}}}
  \and
  \inferrule* [lab=Equiv]{{{P} \scong {P}'} \andalso {{P}' \red {Q}'} \andalso {{Q}' \scong {Q}}}{{P} \red {Q}}
\end{mathpar}

\begin{eqnarray*}
  match_{\equiv} (\quotep{P},\quotep{Q}) & := & P \equiv Q \\
  match_{\dagger}(\quotep{P},\quotep{Q}) & := & \forall R. P|Q \red^{*} R => R \red^{*} 0 \\
  match_{K}(\quotep{P},\quotep{Q}) & := & K \mbox{ for some context } K
\end{eqnarray*}

$u?(x)P | u!\langle Q \rangle \red P\{\quotep{Q}/x\}$

%We write $\wred$ for $\red^*$, and $P\red$ if $\exists Q $ such that $ P \red Q$.
We write $P\red$ if $\exists Q $ such that $ P \red Q$ and $P\not\red$, otherwise.

\section{Replication}

As mentioned before, it is known that replication (and hence
recursion) can be implemented in a higher-order process algebra
\cite{SangiorgiWalker}. As our first example of calculation with the
machinery thus far presented we give the construction explicitly in
the {\rhoc}.

\begin{eqnarray}
	D_{x} & := & \prefix{x}{y}{(\binpar{\outputp{x}{y}}{@{y}})} \nonumber\\
	\bangp_{x}{P} & := & \binpar{{x}!\langle{\binpar{D_{x}}{P}}\rangle}{D_{x}} \nonumber
\end{eqnarray}

\begin{eqnarray}
	\bangp_{x}{P} & & \nonumber\\
	=
	& {x}!\langle{(\prefix{x}{y}{(\outputp{x}{y} | @{y})) | P}}\rangle 
	      | \prefix{x}{y}{(\outputp{x}{y} | @{y})} & \nonumber\\
	\red
	& (\outputp{x}{y} | @{y})\substn{\quotep{(\prefix{x}{y}{(@{y} | \outputp{x}{y})) | P}}}{y} & \nonumber\\
	=
	& \outputp{x}{\quotep{(\prefix{x}{y}{(\outputp{x}{y} | @{y})) | P}}}
	  | {(\prefix{x}{y}{(\outputp{x}{y} | @{y})) | P}} & \nonumber\\
	\red
	& \ldots & \nonumber\\
	\red^*
	& P | P | \ldots & \nonumber
\end{eqnarray}

Of course, this encoding, as an implementation, runs away, unfolding
$\bangp{P}$ eagerly. A lazier and more implementable replication
operator, restricted to input-guarded processes, may be obtained as follows.

\begin{eqnarray}
\bangp{\prefix{u}{v}{P}} 
	:= 
	\binpar{\lift{x}{\prefix{u}{v}{(\binpar{D(x)}{P})}}}{D(x)} \nonumber
\end{eqnarray}

\begin{remark}
  Note that the lazier definition still does not deal with summation
  or mixed summation (i.e. sums over input and output). The reader is
  invited to construct definitions of replication that deal with these
  features. 

  Further, the definitions are parameterized in a name, $x$. Can you,
  gentle reader, make a definition that eliminates this parameter and
  guarantees no accidental interaction between the replication
  machinery and the process being replicated -- i.e. no accidental
  sharing of names used by the process to get its work done and the
  name(s) used by the replication to effect copying. This latter
  revision of the definition of replication is crucial to obtaining
  the expected identity $!!P \sim !P$.
\end{remark}

\begin{remark}\label{rem:paradoxical_combinator}
  The reader familiar with the lambda calculus will have noticed the
  similarity between $D$ and the paradoxical combinator.

  [Ed. note: the existence of this seems to suggest we have to be more
  restrictive on the set of processes and names we admit if we are to
  support no-cloning.]
\end{remark}

\subsubsection{Bisimulation}

The computational dynamics gives rise to another kind of equivalence,
the equivalence of computational behavior. As previously mentioned
this is typically captured \emph{via} some form of bisimulation.

% The notion we use in this paper is weak barbed bisimulation
% \cite{milner91polyadicpi}.

The notion we use in this paper is derived from weak barbed
bisimulation \cite{milner91polyadicpi}. 

\begin{definition}
An \emph{observation relation}, $\downarrow_{\mathcal N}$, over a set
of names, $\mathcal N$, is the smallest relation satisfying the rules
below.

\infrule[Out-barb]{y \in {\mathcal N}, \; x \nameeq y}
		  {\outputp{x}{v} \downarrow_{\mathcal N} x}
\infrule[Par-barb]{\mbox{$P\downarrow_{\mathcal N} x$ or $Q\downarrow_{\mathcal N} x$}}
		  {\binpar{P}{Q} \downarrow_{\mathcal N} x}

We write $P \Downarrow_{\mathcal N} x$ if there is $Q$ such that 
$P \wred Q$ and $Q \downarrow_{\mathcal N} x$.
\end{definition}

\begin{definition}
%\label{def.bbisim}
An  ${\mathcal N}$-\emph{barbed bisimulation} over a set of names, ${\mathcal N}$, is a symmetric binary relation 
${\mathcal S}_{\mathcal N}$ between agents such that $P\rel{S}_{\mathcal N}Q$ implies:
\begin{enumerate}
\item If $P \red P'$ then $Q \wred Q'$ and $P'\rel{S}_{\mathcal N} Q'$.
\item If $P\downarrow_{\mathcal N} x$, then $Q\Downarrow_{\mathcal N} x$.
\end{enumerate}
$P$ is ${\mathcal N}$-barbed bisimilar to $Q$, written
$P \wbbisim_{\mathcal N} Q$, if $P \rel{S}_{\mathcal N} Q$ for some ${\mathcal N}$-barbed bisimulation ${\mathcal S}_{\mathcal N}$.
\end{definition}

$\mathcal{R} \subseteq \pi \times \pi$

$P \mathcal{R} Q => \forall P'. P \red P' \Rightarrow \exists Q'. Q \red Q', P' \mathcal{R} Q'$

$P \vdash x \Rightarrow Q \vdash x$

\begin{mathpar}
  \inferrule*[lab=Out-barb]{x \nameeq y}{{y}!\langle{Q}\rangle \vdash x}
  \and
  \inferrule*[lab=Par-barb]{\mbox{$P\vdash x$ or $Q\vdash x$}}{\binpar{P}{Q} \vdash x}
\end{mathpar}

\subsubsection{Contexts}

One of the principle advantages of computational calculi like the
$\pi$-calculus is a well-defined notion of context,
contextual-equivalence and a correlation between
contextual-equivalence and notions of bisimulation. The notion of
context allows the decomposition of a process into (sub-)process and
its syntactic environment, its context. Thus, a context may be
thought of as a process with a ``hole'' (written $\Box$) in it. The
application of a context $M$ to a process $P$, written $M[P]$, is
tantamount to filling the hole in $M$ with $P$. In this paper we do
not need the full weight of this theory, but do make use of the notion
of context in the proof the main theorem. 

\begin{mathpar}
  \inferrule* [lab=summation] {} {{M_{M},M_{N}} \bc \Box \;|\; x.M_{A} \;|\; M_{M}+M_{N}}
  \and
  \inferrule* [lab=agent] {} {{M_{A}} \bc (\vec{x})M_{P} \;| \; \clift{P_0,\ldots,M_{P},\ldots,P_N}}
  \and \\
  \inferrule* [lab=process] {} {{M_{P}} \bc M_{N} \;| \;P|M_{P} }
\end{mathpar} 

\begin{mathpar}
  \inferrule* [lab=sychronization] {} {M_{N} \bc \Box \;|\; x?M_{F} \;|\; x!M_{C}}
  \and
  \inferrule* [lab=abstraction] {} {{M_{F}} \bc (x)M_{P} }
  \and
  \inferrule* [lab=concretion] {} {{M_{C}} \bc \langle M_{P} \rangle }
  \and \\
  \inferrule* [lab=process] {} {{M_{P}} \bc M_{N} \;| \;P|M_{P} }
\end{mathpar}

\begin{definition}[contextual application] Given a context $M$, and
  process $P$, we define the \emph{contextual application}, $M[P] :=
  M\{P/\Box\}$. That is, the contextual application of M to P is the
  substitution of $P$ for $\Box$ in $M$.
\end{definition}

$\meaningof{-} : L \to \mathcal{P}(\pi)$

\begin{mathpar}
  \inferrule* [lab=collection] {} {\meaningof{true} = \pi, \and \meaningof{~E} = \pi \setminus \meaningof{E}, \and \meaningof{E_{1} \& E_{2}} = \meaningof{E_{1}} \cap \meaningof{E_{2}}}
\end{mathpar}

\begin{mathpar}
  \inferrule* [lab=structure] {} {\meaningof{0} = \{ P \in \pi | P \equiv 0 \}, \and \\ \meaningof{E_1 | E_2} = \{ P \in \pi | P \equiv P_{1} | P_{2}, P_{1} \in \meaningof{E_{1}}, P_{2} \in \meaningof{E_2}\} }
\end{mathpar}

\begin{mathpar}
 \inferrule* [lab=behavior] {} {\meaningof{\langle a?b \rangle E} = \{ P \in \pi | P \equiv Q | u?(y)P', \\ \and \\\\ \and \\ \;\;\; u \in \meaningof{a}, \forall z.P'\{z/y\} \in \meaningof{E\{z/b\}}\}, \and \\ \meaningof{a!E} = \{ P \in \pi | P \equiv Q | x!\langle P' \rangle, x \in \meaningof{a} P' \in \meaningof{E}\} }
\end{mathpar}

\begin{mathpar}
 \inferrule* [lab=nominal] {} {\meaningof{\quotep{E}} = \{ \quotep{P} \in \quotep{\pi} | P \in \meaningof{E} \}, \and \meaningof{\quotep{P}} = \{ \quotep{Q} \in \quotep{\pi} | P \equiv Q \} \and \\ \meaningof{@\quotep{E}} = \{ P \in \pi | P \equiv @x, x \in \meaningof{E} \}}
\end{mathpar}

\begin{eqnarray*}
  \\
  \meaningof{-} : TS \to ST
\end{eqnarray*}

\begin{eqnarray*}
  \\
  L : TS \to ST
\end{eqnarray*}

\begin{eqnarray*}
  \\
  P \models E \iff P \in \meaningof{E}
\end{eqnarray*}

\begin{eqnarray*}
  P \approx_{L} Q \iff \forall E \in L. P \models E \iff Q \models E
\end{eqnarray*}

\begin{eqnarray*}
  P \approx_{K} Q
\end{eqnarray*}

\begin{eqnarray*}
  P \approx Q
\end{eqnarray*}

$\approx_{K} = \approx = \approx_{L}$

\subsubsection{Contextual duality}

Note that contexts extend the quotation operation to a family of
operations from processes to names. Given a context, $M$, we can
define a \emph{nominal context}, $\quotep{M}$ by $\quotep{M}[P] :=
\quotep{M[P]}$. To foreshadow what is to come we observe that these
operations enjoy a duality with processes very much like the duality
between vectors and maps from vectors to scalars.

Further, because the calculus is essentially higher-order, we have a
correspondence between contexts and processes. More specifically,
given a name $x$ and a context $M$ we can construct $M^{*}_{x}$ such
that 

\begin{mathpar}
  M^{*}_{x} | \lift{x}{P} \red M[P]
\end{mathpar}

namely,

\begin{mathpar}
  M^{*}_{x} := x?(u).M[\dropn{u}]
\end{mathpar}

The dependence of $M^{*}_{x}$ on a name makes it an abstraction, 

\begin{mathpar}
  M^{*} := (x)x?(u).M[\dropn{u}]
\end{mathpar}

\subsection{Additional notation}

It will sometimes be convenient to denote the process a name
quotes. We already have the notation $x = \quotep{P}$, but it will be
convenient to introduce an alternate notation, $\procn{x}$, when we
want to emphasize the connection to the use of the name. Note that, by
virtue of name equivalence, $\quotep{\procn{x}} \nameeq x$; so, the
notation is consistent with previous definitions.

Further, because names have structure it is possible to effect
substitutions on the basis of that structure. This means we need to
upgrade our notation for substitutions, which we accomplish by
adapting comprehension notation. Thus,

\begin{mathpar}
  P\{ y / x : x \in S \}
\end{mathpar}

is interpreted to mean the process derived from P by replacing (in a
capture-avoiding manner) each occurrence of $x$ in $S$ by $y$. For example,

\begin{mathpar}
  P\{ \quotep{\procn{x}|\procn{x}} / x : x \in \freenames{P} \}
\end{mathpar}

will replace each (occurrence) of a free name $x$ in $P$ by
$\quotep{\procn{x}|\procn{x}}$.

Also, we will avail ourselves of the notation $x^{L}$ and $x^{R}$ to
denote injections of a name into disjoint copies of the name
space. There are numerous ways to accomplish this. One example can be
found in \cite{MeredithR05}. This notation overloads to vectors of
names: $\vec{x}^{\pi} := (x_{i}^{\pi} \; : \; 0 \leq i < |\vec{x}| )$ where $\pi \in \{L,R\}$.

We also use $P^{\Box} := P|\Box$.

In \cite{MeredithR05} an interpretation of the new operator is
given. It turns out that there are several possible interpretations
all enjoying the requisite algebraic properties of the operator (see
\cite{milner91polyadicpi}). We will therefore make liberal use of
$(\nu\; \vec{x})P$.

% subsection the_syntax_and_semantics_of_the_notation_system (end)   

\input{qm2pi.qmops} 

\input{qm2pi.sterngerlach} 

\input{qm2pi.metric} 

% section concurrent_process_calculi (end)

%\input{qm2pi.proofsketch}

% section proof sketch (end)

%\input{qm2pi.slviaknots} 

% section spatial logic via knots (end)

\input{qm2pi.conclusion}

% section conclusion (end)

%\input{qm2pi.dtcodes} 

% section wiring algorithm (end)

\input{qm2pi.ack} 

% section acknowledgments (end)

\newpage


\bibliographystyle{plain}   
\bibliography{../../biblios/main.bib}

\input{qm2pi.rhodetails}

\end{document}

 

% section notation (end)

\input{qm2pi.process.calculi} 

% section concurrent_process_calculi_and_spatial_logics_ (end)
    
%\documentclass[12pt]{llncs}
%\documentclass{jktr}

\usepackage[pdftex]{hyperref}                   
\usepackage {listings}
\usepackage {mathpartir}
\usepackage{bcprules}
%\usepackage{listings}
                       
\usepackage{graphicx} 
%\usepackage[margins=2.5cm,nohead,nofoot]{geometry}
%\usepackage{geometry}
\usepackage{amsfonts}
\usepackage{amstext}
\usepackage{latexsym}
\usepackage{amssymb}
\usepackage{color}


%\include{myPreamble}
\include{qm2pi.local} 

%\ifpdf
%\usepackage[pdftex]{graphicx}
%\else
%\usepackage{graphicx}
%\fi

 % \ifpdf
%  \usepackage{pdfsync}
%  \if


%\title{Brief Article}
%\author{David F. Snyder}
%\author{L.G. Meredith}

%\address{Dept. of Math., Texas State University--San Marcos, San Marcos, TX 78666}
       
\pagestyle{empty}


\begin{document}

\lstset{language=[Objective]Caml,frame=shadowbox}

\input{qm2pi.front}

% section front matter (end)

\input{qm2pi.intro} 
 
% section introduction (end)

% \input{qm2pi.knotations} 

% section notation (end)

\input{qm2pi.process.calculi} 

% section concurrent_process_calculi_and_spatial_logics_ (end)
    
%\input{qm2pi.knots2pi} 

%\input{qm2pi.trefoil} 

%\input{qm2pi.mainthm} 

% subsection basic_interpretation (end)

%\input{qm2pi.rho.presentation} 
\subsection{The syntax and semantics of the notation system}\label{sub:the_syntax_and_semantics_of_the_notation_system} % (fold)

We now summarize a technical presentation of the calculus that
embodies our theory of dynamics. The typical presentation of such a
calculus follows the style of giving generators and relations on
them. The grammar, below, describing term constructors, freely
generates the set of processes, $\Proc$. This set is then quotiented
by a relation known as structural congruence and it is over this set
that the notion of dynamics is expressed. This presentation is
essentially that of \cite{MeredithR05} with the addition of
polyadicity and summation. For readability we have relegated some of
the technical subtleties to an appendix.

\subsubsection{Process grammar}\label{subsub:process_grammar}

\begin{mathpar}
  \inferrule* [lab=synchronization] {} {{M} \bc \pzero \;|\; x?F \;|\; x!C }
  \and
  \inferrule* [lab=abstraction] {} {{F} \bc (x)P}
  \and
  \inferrule* [lab=concretion] {} {{C} \bc \langle Q \rangle}
  \and
  \inferrule* [lab=process] {} {{P,Q} \bc M \;| \;P|Q \;|\; @{x}}
  \and
  \inferrule* [lab=name] {} {{x} \bc \quotep{P}}
\end{mathpar} 

Note that $\vec{x}$ (resp. $\vec{P}$) denotes a vector of names
(resp. processes) of length $|\vec{x}|$ (resp. $|\vec{P}|$). We adopt
the following useful abbreviations.

\begin{mathpar}
   x?(\vec{y}).P := x.(\vec{y})P \and  x\clift{\vec{P}} := x.\clift{\vec{P}}
   \and x!(y) := \lift{x}{\dropn{y}}
   \and \Pi_{i=0}^{n-1}P_i := P_0 | \ldots | P_{n-1}
\end{mathpar}

\subsubsection{Structural congruence}

\paragraph{Free and bound names and alpha-equivalence.} At the
core of structural equivalence is alpha-equivalence which identifies
process that are the same up to a change of variable. Formally, we
recognize the distinction between free and bound names. The free names
of a process, $\freenames{P}$, may be calculated recursively as
follows:

\begin{mathpar}
\freenames{\pzero} := \emptyset
  \and \\
  \freenames{x?(y).P} := \{ x \} \cup (\freenames{P} \setminus \{ y \})
  \and 
  \freenames{x!\langle P \rangle} := \{ x \} \cup \{ P \} 
  \and \\
  \freenames{P|Q} := \freenames{P} \cup \freenames{Q}
  \and \\
  \freenames{@{x}} := \{ x \}
\end{mathpar}

$\pi$
$\quotep{\pi}$

$\freenames{-} : \pi \to \mathcal{P}(\quotep{\pi})$

\begin{eqnarray*}
  \freenames{\pzero} & := & \emptyset \\
  \freenames{x?(y).P} & := & \{ x \} \cup (\freenames{P} \setminus \{ y \}) \\
  \freenames{x!\langle P \rangle} & := & \{ x \} \cup \{ P \} \\
  \freenames{P|Q} & := & \freenames{P} \cup \freenames{Q} \\
  \freenames{\dropn{x}} & := & \{ x \}
\end{eqnarray*}

The bound names of a process, $\boundnames{P}$, are those names occurring in $P$
that are not free. For example, in $x?(y).0$, the name $x$ is free, while $y$ is bound.

\begin{mathpar}
  \inferrule* [lab=monoidal-laws] {} { P|Q \equiv Q|P \and P|0 \equiv P \and P|(Q|R) \equiv (P|Q)|R }
\end{mathpar}

\begin{mathpar}
  \inferrule* [lab=alpha-equivalence] {} { (x)P \equiv (y)P\{y/x\} \and y \not\in \freenames{P} }
\end{mathpar}

\begin{definition}
Then two processes, $P,Q$, are alpha-equivalent if $P = Q\{\vec{y}/\vec{x}\}$ for
some $\vec{x} \in \boundnames{Q},\vec{y} \in \boundnames{P}$, where $Q\{\vec{y}/\vec{x}\}$
denotes the capture-avoiding substitution of $\vec{y}$ for $\vec{x}$ in $Q$.
\end{definition}

\begin{definition}
  The {\em structural congruence} \cite{SangiorgiWalker} , $\equiv$,
  between processes is the least congruence containing
  alpha-equivalence, satisfying the abelian monoid laws
  (associativity, commutativity and $\pzero$ as identity) for parallel
  composition $|$ and for summation $+$.
\end{definition}

\subsection{Name equivalence}

We take name equivalence, written $\nameeq$, to be the smallest
equivalence relation generated by the following rules.

\begin{mathpar}
\inferrule*[lab=Quote-drop]
{ }
{ \quotep{@{x}} \nameeq x }

\inferrule*[lab=Struct-equiv]
{ P \scong Q }
{ \quotep{P} \nameeq \quotep{Q} }
\end{mathpar}

The astute reader will have noticed that the mutual recursion of names
and processes imposes a mutual recursion on alpha-equivalence and
structural equivalence via name-equivalence. Fortunately, all of this
works out pleasantly and we may calculate in the natural way, free of
concern. The reader interested in the details is referred to the
appendix \ref{appendix:rho_details}.

\subsection{Substitution}

We use $\Proc$ for the set of processes, $\QProc$ for the set of
names, and $\id{\{}\vec{y} / \vec{x} \id{\}}$ to denote partial maps,
$s : \QProc \rightarrow \QProc$. A map, $s$ lifts, uniquely, to a map
on process terms, $\widehat{s} : \Proc \rightarrow \Proc$ by the
following equations.

\begin{mathpar}
  (0) \psubstp{Q}{P} := 0 \\
  (R \juxtap S) \psubstp{Q}{P}
  :=    
  (R)\psubstp{Q}{P} \juxtap (S) \psubstp{Q}{P} \\
  (x?(y).R) \psubstp{Q}{P}    
  :=    
  (x)\substp{Q}{P} (z)\concat( (R \psubstn{z}{y}) \psubstp{Q}{P} ) \\
  (\lift{x}{R}) \psubstp{Q}{P}  
  :=
  \lift{(x)\substp{Q}{P}}{ R \psubstp{Q}{P} } \\
%   (\dropn{x})  \psubstp{Q}{P}       
%   := 
%   \left\{ 
%     \begin{array}{ccc} 
%       \dropn{\quotep{Q}} & & x \nameeq \quotep{P} \\
%       \dropn{x} & & otherwise \\
%     \end{array}
%   \right. 
  (\dropn{x})  \psubstp{Q}{P}       
  := 
  \left\{ 
    \begin{array}{ccc} 
      Q & & x \nameeq \quotep{P} \\
      \dropn{x} & & otherwise \\
    \end{array}
  \right.
\end{mathpar}
 

where

\begin{eqnarray}
  (x)\id{\{} \lpquote Q \rpquote / \lpquote P \rpquote \id{\}}            = 
  \left\{ 
    \begin{array}{ccc}
      \lpquote Q \rpquote & & x \nameeq \lpquote P \rpquote \\
      x & & otherwise \\
    \end{array}
  \right. \nonumber
\end{eqnarray}

and $z$ is chosen distinct from $\quotep{P}$, $\quotep{Q}$, the free
names in $Q$, and all the names in $R$. Our $\alpha$-equivalence will
be built in the standard way from this substitution.

\begin{remark}\label{rem:no_self_referential_names}
  One consequence of these definitions is that $\forall P. \quotep{P}
  \not\in \freenames{P}$.
\end{remark}

\subsection{ Dynamic quote: an example }

Anticipating something of what's to come, consider applying the
substitution, $\widehat{\id{\{}u / z \id{\}}}$, to the following pair
of processes, $\lift{w}{y!(z)}$ and $w[ \lpquote y!(z) \rpquote ]$.

\begin{eqnarray}
	\lift{w}{y!(z)}\widehat{\id{\{}u / z \id{\}}}
		& = &
		\lift{w}{y!(u)} \nonumber\\
	w[ \lpquote y!(z) \rpquote ] \widehat{ \id{\{}u / z \id{\}} }
		& = &
		w[ \lpquote y!(z) \rpquote ] \nonumber
\end{eqnarray}

Because the body of the process between quotes is impervious to
substitution, we get radically different answers. In fact, by
examining the first process in an input context,
e.g. $x?(z).\lift{w}{y!(z)}$, we see that the process under the lift
operator may be shaped by prefixed inputs binding a name inside it. In
this sense, the lift operator will be seen as a way to dynamically
construct processes before reifying them as names.

Finally equipped with these standard features we can present the
dynamics of the calculus.

\subsubsection{Operational semantics} 

Finally, we introduce the computational dynamics. What marks these
algebras as distinct from other more traditionally studied algebraic
structures, e.g. vector spaces or polynomial rings, is the manner in
which dynamics is captured. In traditional structures, dynamics is typically
expressed through morphisms between such structures, as in linear maps
between vector spaces or morphisms between rings. In algebras
associated with the semantics of computation, the dynamics is
expressed as part of the algebraic structure itself, through a
reduction reduction relation typically denoted by $\red$. Below, we
give a recursive presentation of this relation for the calculus used
in the encoding.

$\red \subseteq \pi \times \pi$
$\red : \pi \to \mathcal{P}(\pi)$

\begin{mathpar}
  \inferrule* [lab=Comm] { \textsf{match}( x_{src}, x_{trgt} ) } { x_{trgt}?(y)P \; | \; x_{src}!\langle {Q} \rangle \red P\{\quotep{Q}/y}\} }
  \and \\
  \inferrule* [lab=Par] {{P} \red {P}'} {{{P} | {Q}} \red {{P}' | {Q}}}
  \and
  \inferrule* [lab=Equiv]{{{P} \scong {P}'} \andalso {{P}' \red {Q}'} \andalso {{Q}' \scong {Q}}}{{P} \red {Q}}
\end{mathpar}

\begin{eqnarray*}
  match_{\equiv} (\quotep{P},\quotep{Q}) & := & P \equiv Q \\
  match_{\dagger}(\quotep{P},\quotep{Q}) & := & \forall R. P|Q \red^{*} R => R \red^{*} 0 \\
  match_{K}(\quotep{P},\quotep{Q}) & := & K \mbox{ for some context } K
\end{eqnarray*}

$u?(x)P | u!\langle Q \rangle \red P\{\quotep{Q}/x\}$

%We write $\wred$ for $\red^*$, and $P\red$ if $\exists Q $ such that $ P \red Q$.
We write $P\red$ if $\exists Q $ such that $ P \red Q$ and $P\not\red$, otherwise.

\section{Replication}

As mentioned before, it is known that replication (and hence
recursion) can be implemented in a higher-order process algebra
\cite{SangiorgiWalker}. As our first example of calculation with the
machinery thus far presented we give the construction explicitly in
the {\rhoc}.

\begin{eqnarray}
	D_{x} & := & \prefix{x}{y}{(\binpar{\outputp{x}{y}}{@{y}})} \nonumber\\
	\bangp_{x}{P} & := & \binpar{{x}!\langle{\binpar{D_{x}}{P}}\rangle}{D_{x}} \nonumber
\end{eqnarray}

\begin{eqnarray}
	\bangp_{x}{P} & & \nonumber\\
	=
	& {x}!\langle{(\prefix{x}{y}{(\outputp{x}{y} | @{y})) | P}}\rangle 
	      | \prefix{x}{y}{(\outputp{x}{y} | @{y})} & \nonumber\\
	\red
	& (\outputp{x}{y} | @{y})\substn{\quotep{(\prefix{x}{y}{(@{y} | \outputp{x}{y})) | P}}}{y} & \nonumber\\
	=
	& \outputp{x}{\quotep{(\prefix{x}{y}{(\outputp{x}{y} | @{y})) | P}}}
	  | {(\prefix{x}{y}{(\outputp{x}{y} | @{y})) | P}} & \nonumber\\
	\red
	& \ldots & \nonumber\\
	\red^*
	& P | P | \ldots & \nonumber
\end{eqnarray}

Of course, this encoding, as an implementation, runs away, unfolding
$\bangp{P}$ eagerly. A lazier and more implementable replication
operator, restricted to input-guarded processes, may be obtained as follows.

\begin{eqnarray}
\bangp{\prefix{u}{v}{P}} 
	:= 
	\binpar{\lift{x}{\prefix{u}{v}{(\binpar{D(x)}{P})}}}{D(x)} \nonumber
\end{eqnarray}

\begin{remark}
  Note that the lazier definition still does not deal with summation
  or mixed summation (i.e. sums over input and output). The reader is
  invited to construct definitions of replication that deal with these
  features. 

  Further, the definitions are parameterized in a name, $x$. Can you,
  gentle reader, make a definition that eliminates this parameter and
  guarantees no accidental interaction between the replication
  machinery and the process being replicated -- i.e. no accidental
  sharing of names used by the process to get its work done and the
  name(s) used by the replication to effect copying. This latter
  revision of the definition of replication is crucial to obtaining
  the expected identity $!!P \sim !P$.
\end{remark}

\begin{remark}\label{rem:paradoxical_combinator}
  The reader familiar with the lambda calculus will have noticed the
  similarity between $D$ and the paradoxical combinator.

  [Ed. note: the existence of this seems to suggest we have to be more
  restrictive on the set of processes and names we admit if we are to
  support no-cloning.]
\end{remark}

\subsubsection{Bisimulation}

The computational dynamics gives rise to another kind of equivalence,
the equivalence of computational behavior. As previously mentioned
this is typically captured \emph{via} some form of bisimulation.

% The notion we use in this paper is weak barbed bisimulation
% \cite{milner91polyadicpi}.

The notion we use in this paper is derived from weak barbed
bisimulation \cite{milner91polyadicpi}. 

\begin{definition}
An \emph{observation relation}, $\downarrow_{\mathcal N}$, over a set
of names, $\mathcal N$, is the smallest relation satisfying the rules
below.

\infrule[Out-barb]{y \in {\mathcal N}, \; x \nameeq y}
		  {\outputp{x}{v} \downarrow_{\mathcal N} x}
\infrule[Par-barb]{\mbox{$P\downarrow_{\mathcal N} x$ or $Q\downarrow_{\mathcal N} x$}}
		  {\binpar{P}{Q} \downarrow_{\mathcal N} x}

We write $P \Downarrow_{\mathcal N} x$ if there is $Q$ such that 
$P \wred Q$ and $Q \downarrow_{\mathcal N} x$.
\end{definition}

\begin{definition}
%\label{def.bbisim}
An  ${\mathcal N}$-\emph{barbed bisimulation} over a set of names, ${\mathcal N}$, is a symmetric binary relation 
${\mathcal S}_{\mathcal N}$ between agents such that $P\rel{S}_{\mathcal N}Q$ implies:
\begin{enumerate}
\item If $P \red P'$ then $Q \wred Q'$ and $P'\rel{S}_{\mathcal N} Q'$.
\item If $P\downarrow_{\mathcal N} x$, then $Q\Downarrow_{\mathcal N} x$.
\end{enumerate}
$P$ is ${\mathcal N}$-barbed bisimilar to $Q$, written
$P \wbbisim_{\mathcal N} Q$, if $P \rel{S}_{\mathcal N} Q$ for some ${\mathcal N}$-barbed bisimulation ${\mathcal S}_{\mathcal N}$.
\end{definition}

$\mathcal{R} \subseteq \pi \times \pi$

$P \mathcal{R} Q => \forall P'. P \red P' \Rightarrow \exists Q'. Q \red Q', P' \mathcal{R} Q'$

$P \vdash x \Rightarrow Q \vdash x$

\begin{mathpar}
  \inferrule*[lab=Out-barb]{x \nameeq y}{{y}!\langle{Q}\rangle \vdash x}
  \and
  \inferrule*[lab=Par-barb]{\mbox{$P\vdash x$ or $Q\vdash x$}}{\binpar{P}{Q} \vdash x}
\end{mathpar}

\subsubsection{Contexts}

One of the principle advantages of computational calculi like the
$\pi$-calculus is a well-defined notion of context,
contextual-equivalence and a correlation between
contextual-equivalence and notions of bisimulation. The notion of
context allows the decomposition of a process into (sub-)process and
its syntactic environment, its context. Thus, a context may be
thought of as a process with a ``hole'' (written $\Box$) in it. The
application of a context $M$ to a process $P$, written $M[P]$, is
tantamount to filling the hole in $M$ with $P$. In this paper we do
not need the full weight of this theory, but do make use of the notion
of context in the proof the main theorem. 

\begin{mathpar}
  \inferrule* [lab=summation] {} {{M_{M},M_{N}} \bc \Box \;|\; x.M_{A} \;|\; M_{M}+M_{N}}
  \and
  \inferrule* [lab=agent] {} {{M_{A}} \bc (\vec{x})M_{P} \;| \; \clift{P_0,\ldots,M_{P},\ldots,P_N}}
  \and \\
  \inferrule* [lab=process] {} {{M_{P}} \bc M_{N} \;| \;P|M_{P} }
\end{mathpar} 

\begin{mathpar}
  \inferrule* [lab=sychronization] {} {M_{N} \bc \Box \;|\; x?M_{F} \;|\; x!M_{C}}
  \and
  \inferrule* [lab=abstraction] {} {{M_{F}} \bc (x)M_{P} }
  \and
  \inferrule* [lab=concretion] {} {{M_{C}} \bc \langle M_{P} \rangle }
  \and \\
  \inferrule* [lab=process] {} {{M_{P}} \bc M_{N} \;| \;P|M_{P} }
\end{mathpar}

\begin{definition}[contextual application] Given a context $M$, and
  process $P$, we define the \emph{contextual application}, $M[P] :=
  M\{P/\Box\}$. That is, the contextual application of M to P is the
  substitution of $P$ for $\Box$ in $M$.
\end{definition}

$\meaningof{-} : L \to \mathcal{P}(\pi)$

\begin{mathpar}
  \inferrule* [lab=collection] {} {\meaningof{true} = \pi, \and \meaningof{~E} = \pi \setminus \meaningof{E}, \and \meaningof{E_{1} \& E_{2}} = \meaningof{E_{1}} \cap \meaningof{E_{2}}}
\end{mathpar}

\begin{mathpar}
  \inferrule* [lab=structure] {} {\meaningof{0} = \{ P \in \pi | P \equiv 0 \}, \and \\ \meaningof{E_1 | E_2} = \{ P \in \pi | P \equiv P_{1} | P_{2}, P_{1} \in \meaningof{E_{1}}, P_{2} \in \meaningof{E_2}\} }
\end{mathpar}

\begin{mathpar}
 \inferrule* [lab=behavior] {} {\meaningof{\langle a?b \rangle E} = \{ P \in \pi | P \equiv Q | u?(y)P', \\ \and \\\\ \and \\ \;\;\; u \in \meaningof{a}, \forall z.P'\{z/y\} \in \meaningof{E\{z/b\}}\}, \and \\ \meaningof{a!E} = \{ P \in \pi | P \equiv Q | x!\langle P' \rangle, x \in \meaningof{a} P' \in \meaningof{E}\} }
\end{mathpar}

\begin{mathpar}
 \inferrule* [lab=nominal] {} {\meaningof{\quotep{E}} = \{ \quotep{P} \in \quotep{\pi} | P \in \meaningof{E} \}, \and \meaningof{\quotep{P}} = \{ \quotep{Q} \in \quotep{\pi} | P \equiv Q \} \and \\ \meaningof{@\quotep{E}} = \{ P \in \pi | P \equiv @x, x \in \meaningof{E} \}}
\end{mathpar}

\begin{eqnarray*}
  \\
  \meaningof{-} : TS \to ST
\end{eqnarray*}

\begin{eqnarray*}
  \\
  L : TS \to ST
\end{eqnarray*}

\begin{eqnarray*}
  \\
  P \models E \iff P \in \meaningof{E}
\end{eqnarray*}

\begin{eqnarray*}
  P \approx_{L} Q \iff \forall E \in L. P \models E \iff Q \models E
\end{eqnarray*}

\begin{eqnarray*}
  P \approx_{K} Q
\end{eqnarray*}

\begin{eqnarray*}
  P \approx Q
\end{eqnarray*}

$\approx_{K} = \approx = \approx_{L}$

\subsubsection{Contextual duality}

Note that contexts extend the quotation operation to a family of
operations from processes to names. Given a context, $M$, we can
define a \emph{nominal context}, $\quotep{M}$ by $\quotep{M}[P] :=
\quotep{M[P]}$. To foreshadow what is to come we observe that these
operations enjoy a duality with processes very much like the duality
between vectors and maps from vectors to scalars.

Further, because the calculus is essentially higher-order, we have a
correspondence between contexts and processes. More specifically,
given a name $x$ and a context $M$ we can construct $M^{*}_{x}$ such
that 

\begin{mathpar}
  M^{*}_{x} | \lift{x}{P} \red M[P]
\end{mathpar}

namely,

\begin{mathpar}
  M^{*}_{x} := x?(u).M[\dropn{u}]
\end{mathpar}

The dependence of $M^{*}_{x}$ on a name makes it an abstraction, 

\begin{mathpar}
  M^{*} := (x)x?(u).M[\dropn{u}]
\end{mathpar}

\subsection{Additional notation}

It will sometimes be convenient to denote the process a name
quotes. We already have the notation $x = \quotep{P}$, but it will be
convenient to introduce an alternate notation, $\procn{x}$, when we
want to emphasize the connection to the use of the name. Note that, by
virtue of name equivalence, $\quotep{\procn{x}} \nameeq x$; so, the
notation is consistent with previous definitions.

Further, because names have structure it is possible to effect
substitutions on the basis of that structure. This means we need to
upgrade our notation for substitutions, which we accomplish by
adapting comprehension notation. Thus,

\begin{mathpar}
  P\{ y / x : x \in S \}
\end{mathpar}

is interpreted to mean the process derived from P by replacing (in a
capture-avoiding manner) each occurrence of $x$ in $S$ by $y$. For example,

\begin{mathpar}
  P\{ \quotep{\procn{x}|\procn{x}} / x : x \in \freenames{P} \}
\end{mathpar}

will replace each (occurrence) of a free name $x$ in $P$ by
$\quotep{\procn{x}|\procn{x}}$.

Also, we will avail ourselves of the notation $x^{L}$ and $x^{R}$ to
denote injections of a name into disjoint copies of the name
space. There are numerous ways to accomplish this. One example can be
found in \cite{MeredithR05}. This notation overloads to vectors of
names: $\vec{x}^{\pi} := (x_{i}^{\pi} \; : \; 0 \leq i < |\vec{x}| )$ where $\pi \in \{L,R\}$.

We also use $P^{\Box} := P|\Box$.

In \cite{MeredithR05} an interpretation of the new operator is
given. It turns out that there are several possible interpretations
all enjoying the requisite algebraic properties of the operator (see
\cite{milner91polyadicpi}). We will therefore make liberal use of
$(\nu\; \vec{x})P$.

% subsection the_syntax_and_semantics_of_the_notation_system (end)   

\input{qm2pi.qmops} 

\input{qm2pi.sterngerlach} 

\input{qm2pi.metric} 

% section concurrent_process_calculi (end)

%\input{qm2pi.proofsketch}

% section proof sketch (end)

%\input{qm2pi.slviaknots} 

% section spatial logic via knots (end)

\input{qm2pi.conclusion}

% section conclusion (end)

%\input{qm2pi.dtcodes} 

% section wiring algorithm (end)

\input{qm2pi.ack} 

% section acknowledgments (end)

\newpage


\bibliographystyle{plain}   
\bibliography{../../biblios/main.bib}

\input{qm2pi.rhodetails}

\end{document}

 

%\documentclass[12pt]{llncs}
%\documentclass{jktr}

\usepackage[pdftex]{hyperref}                   
\usepackage {listings}
\usepackage {mathpartir}
\usepackage{bcprules}
%\usepackage{listings}
                       
\usepackage{graphicx} 
%\usepackage[margins=2.5cm,nohead,nofoot]{geometry}
%\usepackage{geometry}
\usepackage{amsfonts}
\usepackage{amstext}
\usepackage{latexsym}
\usepackage{amssymb}
\usepackage{color}


%\include{myPreamble}
\include{qm2pi.local} 

%\ifpdf
%\usepackage[pdftex]{graphicx}
%\else
%\usepackage{graphicx}
%\fi

 % \ifpdf
%  \usepackage{pdfsync}
%  \if


%\title{Brief Article}
%\author{David F. Snyder}
%\author{L.G. Meredith}

%\address{Dept. of Math., Texas State University--San Marcos, San Marcos, TX 78666}
       
\pagestyle{empty}


\begin{document}

\lstset{language=[Objective]Caml,frame=shadowbox}

\input{qm2pi.front}

% section front matter (end)

\input{qm2pi.intro} 
 
% section introduction (end)

% \input{qm2pi.knotations} 

% section notation (end)

\input{qm2pi.process.calculi} 

% section concurrent_process_calculi_and_spatial_logics_ (end)
    
%\input{qm2pi.knots2pi} 

%\input{qm2pi.trefoil} 

%\input{qm2pi.mainthm} 

% subsection basic_interpretation (end)

%\input{qm2pi.rho.presentation} 
\subsection{The syntax and semantics of the notation system}\label{sub:the_syntax_and_semantics_of_the_notation_system} % (fold)

We now summarize a technical presentation of the calculus that
embodies our theory of dynamics. The typical presentation of such a
calculus follows the style of giving generators and relations on
them. The grammar, below, describing term constructors, freely
generates the set of processes, $\Proc$. This set is then quotiented
by a relation known as structural congruence and it is over this set
that the notion of dynamics is expressed. This presentation is
essentially that of \cite{MeredithR05} with the addition of
polyadicity and summation. For readability we have relegated some of
the technical subtleties to an appendix.

\subsubsection{Process grammar}\label{subsub:process_grammar}

\begin{mathpar}
  \inferrule* [lab=synchronization] {} {{M} \bc \pzero \;|\; x?F \;|\; x!C }
  \and
  \inferrule* [lab=abstraction] {} {{F} \bc (x)P}
  \and
  \inferrule* [lab=concretion] {} {{C} \bc \langle Q \rangle}
  \and
  \inferrule* [lab=process] {} {{P,Q} \bc M \;| \;P|Q \;|\; @{x}}
  \and
  \inferrule* [lab=name] {} {{x} \bc \quotep{P}}
\end{mathpar} 

Note that $\vec{x}$ (resp. $\vec{P}$) denotes a vector of names
(resp. processes) of length $|\vec{x}|$ (resp. $|\vec{P}|$). We adopt
the following useful abbreviations.

\begin{mathpar}
   x?(\vec{y}).P := x.(\vec{y})P \and  x\clift{\vec{P}} := x.\clift{\vec{P}}
   \and x!(y) := \lift{x}{\dropn{y}}
   \and \Pi_{i=0}^{n-1}P_i := P_0 | \ldots | P_{n-1}
\end{mathpar}

\subsubsection{Structural congruence}

\paragraph{Free and bound names and alpha-equivalence.} At the
core of structural equivalence is alpha-equivalence which identifies
process that are the same up to a change of variable. Formally, we
recognize the distinction between free and bound names. The free names
of a process, $\freenames{P}$, may be calculated recursively as
follows:

\begin{mathpar}
\freenames{\pzero} := \emptyset
  \and \\
  \freenames{x?(y).P} := \{ x \} \cup (\freenames{P} \setminus \{ y \})
  \and 
  \freenames{x!\langle P \rangle} := \{ x \} \cup \{ P \} 
  \and \\
  \freenames{P|Q} := \freenames{P} \cup \freenames{Q}
  \and \\
  \freenames{@{x}} := \{ x \}
\end{mathpar}

$\pi$
$\quotep{\pi}$

$\freenames{-} : \pi \to \mathcal{P}(\quotep{\pi})$

\begin{eqnarray*}
  \freenames{\pzero} & := & \emptyset \\
  \freenames{x?(y).P} & := & \{ x \} \cup (\freenames{P} \setminus \{ y \}) \\
  \freenames{x!\langle P \rangle} & := & \{ x \} \cup \{ P \} \\
  \freenames{P|Q} & := & \freenames{P} \cup \freenames{Q} \\
  \freenames{\dropn{x}} & := & \{ x \}
\end{eqnarray*}

The bound names of a process, $\boundnames{P}$, are those names occurring in $P$
that are not free. For example, in $x?(y).0$, the name $x$ is free, while $y$ is bound.

\begin{mathpar}
  \inferrule* [lab=monoidal-laws] {} { P|Q \equiv Q|P \and P|0 \equiv P \and P|(Q|R) \equiv (P|Q)|R }
\end{mathpar}

\begin{mathpar}
  \inferrule* [lab=alpha-equivalence] {} { (x)P \equiv (y)P\{y/x\} \and y \not\in \freenames{P} }
\end{mathpar}

\begin{definition}
Then two processes, $P,Q$, are alpha-equivalent if $P = Q\{\vec{y}/\vec{x}\}$ for
some $\vec{x} \in \boundnames{Q},\vec{y} \in \boundnames{P}$, where $Q\{\vec{y}/\vec{x}\}$
denotes the capture-avoiding substitution of $\vec{y}$ for $\vec{x}$ in $Q$.
\end{definition}

\begin{definition}
  The {\em structural congruence} \cite{SangiorgiWalker} , $\equiv$,
  between processes is the least congruence containing
  alpha-equivalence, satisfying the abelian monoid laws
  (associativity, commutativity and $\pzero$ as identity) for parallel
  composition $|$ and for summation $+$.
\end{definition}

\subsection{Name equivalence}

We take name equivalence, written $\nameeq$, to be the smallest
equivalence relation generated by the following rules.

\begin{mathpar}
\inferrule*[lab=Quote-drop]
{ }
{ \quotep{@{x}} \nameeq x }

\inferrule*[lab=Struct-equiv]
{ P \scong Q }
{ \quotep{P} \nameeq \quotep{Q} }
\end{mathpar}

The astute reader will have noticed that the mutual recursion of names
and processes imposes a mutual recursion on alpha-equivalence and
structural equivalence via name-equivalence. Fortunately, all of this
works out pleasantly and we may calculate in the natural way, free of
concern. The reader interested in the details is referred to the
appendix \ref{appendix:rho_details}.

\subsection{Substitution}

We use $\Proc$ for the set of processes, $\QProc$ for the set of
names, and $\id{\{}\vec{y} / \vec{x} \id{\}}$ to denote partial maps,
$s : \QProc \rightarrow \QProc$. A map, $s$ lifts, uniquely, to a map
on process terms, $\widehat{s} : \Proc \rightarrow \Proc$ by the
following equations.

\begin{mathpar}
  (0) \psubstp{Q}{P} := 0 \\
  (R \juxtap S) \psubstp{Q}{P}
  :=    
  (R)\psubstp{Q}{P} \juxtap (S) \psubstp{Q}{P} \\
  (x?(y).R) \psubstp{Q}{P}    
  :=    
  (x)\substp{Q}{P} (z)\concat( (R \psubstn{z}{y}) \psubstp{Q}{P} ) \\
  (\lift{x}{R}) \psubstp{Q}{P}  
  :=
  \lift{(x)\substp{Q}{P}}{ R \psubstp{Q}{P} } \\
%   (\dropn{x})  \psubstp{Q}{P}       
%   := 
%   \left\{ 
%     \begin{array}{ccc} 
%       \dropn{\quotep{Q}} & & x \nameeq \quotep{P} \\
%       \dropn{x} & & otherwise \\
%     \end{array}
%   \right. 
  (\dropn{x})  \psubstp{Q}{P}       
  := 
  \left\{ 
    \begin{array}{ccc} 
      Q & & x \nameeq \quotep{P} \\
      \dropn{x} & & otherwise \\
    \end{array}
  \right.
\end{mathpar}
 

where

\begin{eqnarray}
  (x)\id{\{} \lpquote Q \rpquote / \lpquote P \rpquote \id{\}}            = 
  \left\{ 
    \begin{array}{ccc}
      \lpquote Q \rpquote & & x \nameeq \lpquote P \rpquote \\
      x & & otherwise \\
    \end{array}
  \right. \nonumber
\end{eqnarray}

and $z$ is chosen distinct from $\quotep{P}$, $\quotep{Q}$, the free
names in $Q$, and all the names in $R$. Our $\alpha$-equivalence will
be built in the standard way from this substitution.

\begin{remark}\label{rem:no_self_referential_names}
  One consequence of these definitions is that $\forall P. \quotep{P}
  \not\in \freenames{P}$.
\end{remark}

\subsection{ Dynamic quote: an example }

Anticipating something of what's to come, consider applying the
substitution, $\widehat{\id{\{}u / z \id{\}}}$, to the following pair
of processes, $\lift{w}{y!(z)}$ and $w[ \lpquote y!(z) \rpquote ]$.

\begin{eqnarray}
	\lift{w}{y!(z)}\widehat{\id{\{}u / z \id{\}}}
		& = &
		\lift{w}{y!(u)} \nonumber\\
	w[ \lpquote y!(z) \rpquote ] \widehat{ \id{\{}u / z \id{\}} }
		& = &
		w[ \lpquote y!(z) \rpquote ] \nonumber
\end{eqnarray}

Because the body of the process between quotes is impervious to
substitution, we get radically different answers. In fact, by
examining the first process in an input context,
e.g. $x?(z).\lift{w}{y!(z)}$, we see that the process under the lift
operator may be shaped by prefixed inputs binding a name inside it. In
this sense, the lift operator will be seen as a way to dynamically
construct processes before reifying them as names.

Finally equipped with these standard features we can present the
dynamics of the calculus.

\subsubsection{Operational semantics} 

Finally, we introduce the computational dynamics. What marks these
algebras as distinct from other more traditionally studied algebraic
structures, e.g. vector spaces or polynomial rings, is the manner in
which dynamics is captured. In traditional structures, dynamics is typically
expressed through morphisms between such structures, as in linear maps
between vector spaces or morphisms between rings. In algebras
associated with the semantics of computation, the dynamics is
expressed as part of the algebraic structure itself, through a
reduction reduction relation typically denoted by $\red$. Below, we
give a recursive presentation of this relation for the calculus used
in the encoding.

$\red \subseteq \pi \times \pi$
$\red : \pi \to \mathcal{P}(\pi)$

\begin{mathpar}
  \inferrule* [lab=Comm] { \textsf{match}( x_{src}, x_{trgt} ) } { x_{trgt}?(y)P \; | \; x_{src}!\langle {Q} \rangle \red P\{\quotep{Q}/y}\} }
  \and \\
  \inferrule* [lab=Par] {{P} \red {P}'} {{{P} | {Q}} \red {{P}' | {Q}}}
  \and
  \inferrule* [lab=Equiv]{{{P} \scong {P}'} \andalso {{P}' \red {Q}'} \andalso {{Q}' \scong {Q}}}{{P} \red {Q}}
\end{mathpar}

\begin{eqnarray*}
  match_{\equiv} (\quotep{P},\quotep{Q}) & := & P \equiv Q \\
  match_{\dagger}(\quotep{P},\quotep{Q}) & := & \forall R. P|Q \red^{*} R => R \red^{*} 0 \\
  match_{K}(\quotep{P},\quotep{Q}) & := & K \mbox{ for some context } K
\end{eqnarray*}

$u?(x)P | u!\langle Q \rangle \red P\{\quotep{Q}/x\}$

%We write $\wred$ for $\red^*$, and $P\red$ if $\exists Q $ such that $ P \red Q$.
We write $P\red$ if $\exists Q $ such that $ P \red Q$ and $P\not\red$, otherwise.

\section{Replication}

As mentioned before, it is known that replication (and hence
recursion) can be implemented in a higher-order process algebra
\cite{SangiorgiWalker}. As our first example of calculation with the
machinery thus far presented we give the construction explicitly in
the {\rhoc}.

\begin{eqnarray}
	D_{x} & := & \prefix{x}{y}{(\binpar{\outputp{x}{y}}{@{y}})} \nonumber\\
	\bangp_{x}{P} & := & \binpar{{x}!\langle{\binpar{D_{x}}{P}}\rangle}{D_{x}} \nonumber
\end{eqnarray}

\begin{eqnarray}
	\bangp_{x}{P} & & \nonumber\\
	=
	& {x}!\langle{(\prefix{x}{y}{(\outputp{x}{y} | @{y})) | P}}\rangle 
	      | \prefix{x}{y}{(\outputp{x}{y} | @{y})} & \nonumber\\
	\red
	& (\outputp{x}{y} | @{y})\substn{\quotep{(\prefix{x}{y}{(@{y} | \outputp{x}{y})) | P}}}{y} & \nonumber\\
	=
	& \outputp{x}{\quotep{(\prefix{x}{y}{(\outputp{x}{y} | @{y})) | P}}}
	  | {(\prefix{x}{y}{(\outputp{x}{y} | @{y})) | P}} & \nonumber\\
	\red
	& \ldots & \nonumber\\
	\red^*
	& P | P | \ldots & \nonumber
\end{eqnarray}

Of course, this encoding, as an implementation, runs away, unfolding
$\bangp{P}$ eagerly. A lazier and more implementable replication
operator, restricted to input-guarded processes, may be obtained as follows.

\begin{eqnarray}
\bangp{\prefix{u}{v}{P}} 
	:= 
	\binpar{\lift{x}{\prefix{u}{v}{(\binpar{D(x)}{P})}}}{D(x)} \nonumber
\end{eqnarray}

\begin{remark}
  Note that the lazier definition still does not deal with summation
  or mixed summation (i.e. sums over input and output). The reader is
  invited to construct definitions of replication that deal with these
  features. 

  Further, the definitions are parameterized in a name, $x$. Can you,
  gentle reader, make a definition that eliminates this parameter and
  guarantees no accidental interaction between the replication
  machinery and the process being replicated -- i.e. no accidental
  sharing of names used by the process to get its work done and the
  name(s) used by the replication to effect copying. This latter
  revision of the definition of replication is crucial to obtaining
  the expected identity $!!P \sim !P$.
\end{remark}

\begin{remark}\label{rem:paradoxical_combinator}
  The reader familiar with the lambda calculus will have noticed the
  similarity between $D$ and the paradoxical combinator.

  [Ed. note: the existence of this seems to suggest we have to be more
  restrictive on the set of processes and names we admit if we are to
  support no-cloning.]
\end{remark}

\subsubsection{Bisimulation}

The computational dynamics gives rise to another kind of equivalence,
the equivalence of computational behavior. As previously mentioned
this is typically captured \emph{via} some form of bisimulation.

% The notion we use in this paper is weak barbed bisimulation
% \cite{milner91polyadicpi}.

The notion we use in this paper is derived from weak barbed
bisimulation \cite{milner91polyadicpi}. 

\begin{definition}
An \emph{observation relation}, $\downarrow_{\mathcal N}$, over a set
of names, $\mathcal N$, is the smallest relation satisfying the rules
below.

\infrule[Out-barb]{y \in {\mathcal N}, \; x \nameeq y}
		  {\outputp{x}{v} \downarrow_{\mathcal N} x}
\infrule[Par-barb]{\mbox{$P\downarrow_{\mathcal N} x$ or $Q\downarrow_{\mathcal N} x$}}
		  {\binpar{P}{Q} \downarrow_{\mathcal N} x}

We write $P \Downarrow_{\mathcal N} x$ if there is $Q$ such that 
$P \wred Q$ and $Q \downarrow_{\mathcal N} x$.
\end{definition}

\begin{definition}
%\label{def.bbisim}
An  ${\mathcal N}$-\emph{barbed bisimulation} over a set of names, ${\mathcal N}$, is a symmetric binary relation 
${\mathcal S}_{\mathcal N}$ between agents such that $P\rel{S}_{\mathcal N}Q$ implies:
\begin{enumerate}
\item If $P \red P'$ then $Q \wred Q'$ and $P'\rel{S}_{\mathcal N} Q'$.
\item If $P\downarrow_{\mathcal N} x$, then $Q\Downarrow_{\mathcal N} x$.
\end{enumerate}
$P$ is ${\mathcal N}$-barbed bisimilar to $Q$, written
$P \wbbisim_{\mathcal N} Q$, if $P \rel{S}_{\mathcal N} Q$ for some ${\mathcal N}$-barbed bisimulation ${\mathcal S}_{\mathcal N}$.
\end{definition}

$\mathcal{R} \subseteq \pi \times \pi$

$P \mathcal{R} Q => \forall P'. P \red P' \Rightarrow \exists Q'. Q \red Q', P' \mathcal{R} Q'$

$P \vdash x \Rightarrow Q \vdash x$

\begin{mathpar}
  \inferrule*[lab=Out-barb]{x \nameeq y}{{y}!\langle{Q}\rangle \vdash x}
  \and
  \inferrule*[lab=Par-barb]{\mbox{$P\vdash x$ or $Q\vdash x$}}{\binpar{P}{Q} \vdash x}
\end{mathpar}

\subsubsection{Contexts}

One of the principle advantages of computational calculi like the
$\pi$-calculus is a well-defined notion of context,
contextual-equivalence and a correlation between
contextual-equivalence and notions of bisimulation. The notion of
context allows the decomposition of a process into (sub-)process and
its syntactic environment, its context. Thus, a context may be
thought of as a process with a ``hole'' (written $\Box$) in it. The
application of a context $M$ to a process $P$, written $M[P]$, is
tantamount to filling the hole in $M$ with $P$. In this paper we do
not need the full weight of this theory, but do make use of the notion
of context in the proof the main theorem. 

\begin{mathpar}
  \inferrule* [lab=summation] {} {{M_{M},M_{N}} \bc \Box \;|\; x.M_{A} \;|\; M_{M}+M_{N}}
  \and
  \inferrule* [lab=agent] {} {{M_{A}} \bc (\vec{x})M_{P} \;| \; \clift{P_0,\ldots,M_{P},\ldots,P_N}}
  \and \\
  \inferrule* [lab=process] {} {{M_{P}} \bc M_{N} \;| \;P|M_{P} }
\end{mathpar} 

\begin{mathpar}
  \inferrule* [lab=sychronization] {} {M_{N} \bc \Box \;|\; x?M_{F} \;|\; x!M_{C}}
  \and
  \inferrule* [lab=abstraction] {} {{M_{F}} \bc (x)M_{P} }
  \and
  \inferrule* [lab=concretion] {} {{M_{C}} \bc \langle M_{P} \rangle }
  \and \\
  \inferrule* [lab=process] {} {{M_{P}} \bc M_{N} \;| \;P|M_{P} }
\end{mathpar}

\begin{definition}[contextual application] Given a context $M$, and
  process $P$, we define the \emph{contextual application}, $M[P] :=
  M\{P/\Box\}$. That is, the contextual application of M to P is the
  substitution of $P$ for $\Box$ in $M$.
\end{definition}

$\meaningof{-} : L \to \mathcal{P}(\pi)$

\begin{mathpar}
  \inferrule* [lab=collection] {} {\meaningof{true} = \pi, \and \meaningof{~E} = \pi \setminus \meaningof{E}, \and \meaningof{E_{1} \& E_{2}} = \meaningof{E_{1}} \cap \meaningof{E_{2}}}
\end{mathpar}

\begin{mathpar}
  \inferrule* [lab=structure] {} {\meaningof{0} = \{ P \in \pi | P \equiv 0 \}, \and \\ \meaningof{E_1 | E_2} = \{ P \in \pi | P \equiv P_{1} | P_{2}, P_{1} \in \meaningof{E_{1}}, P_{2} \in \meaningof{E_2}\} }
\end{mathpar}

\begin{mathpar}
 \inferrule* [lab=behavior] {} {\meaningof{\langle a?b \rangle E} = \{ P \in \pi | P \equiv Q | u?(y)P', \\ \and \\\\ \and \\ \;\;\; u \in \meaningof{a}, \forall z.P'\{z/y\} \in \meaningof{E\{z/b\}}\}, \and \\ \meaningof{a!E} = \{ P \in \pi | P \equiv Q | x!\langle P' \rangle, x \in \meaningof{a} P' \in \meaningof{E}\} }
\end{mathpar}

\begin{mathpar}
 \inferrule* [lab=nominal] {} {\meaningof{\quotep{E}} = \{ \quotep{P} \in \quotep{\pi} | P \in \meaningof{E} \}, \and \meaningof{\quotep{P}} = \{ \quotep{Q} \in \quotep{\pi} | P \equiv Q \} \and \\ \meaningof{@\quotep{E}} = \{ P \in \pi | P \equiv @x, x \in \meaningof{E} \}}
\end{mathpar}

\begin{eqnarray*}
  \\
  \meaningof{-} : TS \to ST
\end{eqnarray*}

\begin{eqnarray*}
  \\
  L : TS \to ST
\end{eqnarray*}

\begin{eqnarray*}
  \\
  P \models E \iff P \in \meaningof{E}
\end{eqnarray*}

\begin{eqnarray*}
  P \approx_{L} Q \iff \forall E \in L. P \models E \iff Q \models E
\end{eqnarray*}

\begin{eqnarray*}
  P \approx_{K} Q
\end{eqnarray*}

\begin{eqnarray*}
  P \approx Q
\end{eqnarray*}

$\approx_{K} = \approx = \approx_{L}$

\subsubsection{Contextual duality}

Note that contexts extend the quotation operation to a family of
operations from processes to names. Given a context, $M$, we can
define a \emph{nominal context}, $\quotep{M}$ by $\quotep{M}[P] :=
\quotep{M[P]}$. To foreshadow what is to come we observe that these
operations enjoy a duality with processes very much like the duality
between vectors and maps from vectors to scalars.

Further, because the calculus is essentially higher-order, we have a
correspondence between contexts and processes. More specifically,
given a name $x$ and a context $M$ we can construct $M^{*}_{x}$ such
that 

\begin{mathpar}
  M^{*}_{x} | \lift{x}{P} \red M[P]
\end{mathpar}

namely,

\begin{mathpar}
  M^{*}_{x} := x?(u).M[\dropn{u}]
\end{mathpar}

The dependence of $M^{*}_{x}$ on a name makes it an abstraction, 

\begin{mathpar}
  M^{*} := (x)x?(u).M[\dropn{u}]
\end{mathpar}

\subsection{Additional notation}

It will sometimes be convenient to denote the process a name
quotes. We already have the notation $x = \quotep{P}$, but it will be
convenient to introduce an alternate notation, $\procn{x}$, when we
want to emphasize the connection to the use of the name. Note that, by
virtue of name equivalence, $\quotep{\procn{x}} \nameeq x$; so, the
notation is consistent with previous definitions.

Further, because names have structure it is possible to effect
substitutions on the basis of that structure. This means we need to
upgrade our notation for substitutions, which we accomplish by
adapting comprehension notation. Thus,

\begin{mathpar}
  P\{ y / x : x \in S \}
\end{mathpar}

is interpreted to mean the process derived from P by replacing (in a
capture-avoiding manner) each occurrence of $x$ in $S$ by $y$. For example,

\begin{mathpar}
  P\{ \quotep{\procn{x}|\procn{x}} / x : x \in \freenames{P} \}
\end{mathpar}

will replace each (occurrence) of a free name $x$ in $P$ by
$\quotep{\procn{x}|\procn{x}}$.

Also, we will avail ourselves of the notation $x^{L}$ and $x^{R}$ to
denote injections of a name into disjoint copies of the name
space. There are numerous ways to accomplish this. One example can be
found in \cite{MeredithR05}. This notation overloads to vectors of
names: $\vec{x}^{\pi} := (x_{i}^{\pi} \; : \; 0 \leq i < |\vec{x}| )$ where $\pi \in \{L,R\}$.

We also use $P^{\Box} := P|\Box$.

In \cite{MeredithR05} an interpretation of the new operator is
given. It turns out that there are several possible interpretations
all enjoying the requisite algebraic properties of the operator (see
\cite{milner91polyadicpi}). We will therefore make liberal use of
$(\nu\; \vec{x})P$.

% subsection the_syntax_and_semantics_of_the_notation_system (end)   

\input{qm2pi.qmops} 

\input{qm2pi.sterngerlach} 

\input{qm2pi.metric} 

% section concurrent_process_calculi (end)

%\input{qm2pi.proofsketch}

% section proof sketch (end)

%\input{qm2pi.slviaknots} 

% section spatial logic via knots (end)

\input{qm2pi.conclusion}

% section conclusion (end)

%\input{qm2pi.dtcodes} 

% section wiring algorithm (end)

\input{qm2pi.ack} 

% section acknowledgments (end)

\newpage


\bibliographystyle{plain}   
\bibliography{../../biblios/main.bib}

\input{qm2pi.rhodetails}

\end{document}

 

%\documentclass[12pt]{llncs}
%\documentclass{jktr}

\usepackage[pdftex]{hyperref}                   
\usepackage {listings}
\usepackage {mathpartir}
\usepackage{bcprules}
%\usepackage{listings}
                       
\usepackage{graphicx} 
%\usepackage[margins=2.5cm,nohead,nofoot]{geometry}
%\usepackage{geometry}
\usepackage{amsfonts}
\usepackage{amstext}
\usepackage{latexsym}
\usepackage{amssymb}
\usepackage{color}


%\include{myPreamble}
\include{qm2pi.local} 

%\ifpdf
%\usepackage[pdftex]{graphicx}
%\else
%\usepackage{graphicx}
%\fi

 % \ifpdf
%  \usepackage{pdfsync}
%  \if


%\title{Brief Article}
%\author{David F. Snyder}
%\author{L.G. Meredith}

%\address{Dept. of Math., Texas State University--San Marcos, San Marcos, TX 78666}
       
\pagestyle{empty}


\begin{document}

\lstset{language=[Objective]Caml,frame=shadowbox}

\input{qm2pi.front}

% section front matter (end)

\input{qm2pi.intro} 
 
% section introduction (end)

% \input{qm2pi.knotations} 

% section notation (end)

\input{qm2pi.process.calculi} 

% section concurrent_process_calculi_and_spatial_logics_ (end)
    
%\input{qm2pi.knots2pi} 

%\input{qm2pi.trefoil} 

%\input{qm2pi.mainthm} 

% subsection basic_interpretation (end)

%\input{qm2pi.rho.presentation} 
\subsection{The syntax and semantics of the notation system}\label{sub:the_syntax_and_semantics_of_the_notation_system} % (fold)

We now summarize a technical presentation of the calculus that
embodies our theory of dynamics. The typical presentation of such a
calculus follows the style of giving generators and relations on
them. The grammar, below, describing term constructors, freely
generates the set of processes, $\Proc$. This set is then quotiented
by a relation known as structural congruence and it is over this set
that the notion of dynamics is expressed. This presentation is
essentially that of \cite{MeredithR05} with the addition of
polyadicity and summation. For readability we have relegated some of
the technical subtleties to an appendix.

\subsubsection{Process grammar}\label{subsub:process_grammar}

\begin{mathpar}
  \inferrule* [lab=synchronization] {} {{M} \bc \pzero \;|\; x?F \;|\; x!C }
  \and
  \inferrule* [lab=abstraction] {} {{F} \bc (x)P}
  \and
  \inferrule* [lab=concretion] {} {{C} \bc \langle Q \rangle}
  \and
  \inferrule* [lab=process] {} {{P,Q} \bc M \;| \;P|Q \;|\; @{x}}
  \and
  \inferrule* [lab=name] {} {{x} \bc \quotep{P}}
\end{mathpar} 

Note that $\vec{x}$ (resp. $\vec{P}$) denotes a vector of names
(resp. processes) of length $|\vec{x}|$ (resp. $|\vec{P}|$). We adopt
the following useful abbreviations.

\begin{mathpar}
   x?(\vec{y}).P := x.(\vec{y})P \and  x\clift{\vec{P}} := x.\clift{\vec{P}}
   \and x!(y) := \lift{x}{\dropn{y}}
   \and \Pi_{i=0}^{n-1}P_i := P_0 | \ldots | P_{n-1}
\end{mathpar}

\subsubsection{Structural congruence}

\paragraph{Free and bound names and alpha-equivalence.} At the
core of structural equivalence is alpha-equivalence which identifies
process that are the same up to a change of variable. Formally, we
recognize the distinction between free and bound names. The free names
of a process, $\freenames{P}$, may be calculated recursively as
follows:

\begin{mathpar}
\freenames{\pzero} := \emptyset
  \and \\
  \freenames{x?(y).P} := \{ x \} \cup (\freenames{P} \setminus \{ y \})
  \and 
  \freenames{x!\langle P \rangle} := \{ x \} \cup \{ P \} 
  \and \\
  \freenames{P|Q} := \freenames{P} \cup \freenames{Q}
  \and \\
  \freenames{@{x}} := \{ x \}
\end{mathpar}

$\pi$
$\quotep{\pi}$

$\freenames{-} : \pi \to \mathcal{P}(\quotep{\pi})$

\begin{eqnarray*}
  \freenames{\pzero} & := & \emptyset \\
  \freenames{x?(y).P} & := & \{ x \} \cup (\freenames{P} \setminus \{ y \}) \\
  \freenames{x!\langle P \rangle} & := & \{ x \} \cup \{ P \} \\
  \freenames{P|Q} & := & \freenames{P} \cup \freenames{Q} \\
  \freenames{\dropn{x}} & := & \{ x \}
\end{eqnarray*}

The bound names of a process, $\boundnames{P}$, are those names occurring in $P$
that are not free. For example, in $x?(y).0$, the name $x$ is free, while $y$ is bound.

\begin{mathpar}
  \inferrule* [lab=monoidal-laws] {} { P|Q \equiv Q|P \and P|0 \equiv P \and P|(Q|R) \equiv (P|Q)|R }
\end{mathpar}

\begin{mathpar}
  \inferrule* [lab=alpha-equivalence] {} { (x)P \equiv (y)P\{y/x\} \and y \not\in \freenames{P} }
\end{mathpar}

\begin{definition}
Then two processes, $P,Q$, are alpha-equivalent if $P = Q\{\vec{y}/\vec{x}\}$ for
some $\vec{x} \in \boundnames{Q},\vec{y} \in \boundnames{P}$, where $Q\{\vec{y}/\vec{x}\}$
denotes the capture-avoiding substitution of $\vec{y}$ for $\vec{x}$ in $Q$.
\end{definition}

\begin{definition}
  The {\em structural congruence} \cite{SangiorgiWalker} , $\equiv$,
  between processes is the least congruence containing
  alpha-equivalence, satisfying the abelian monoid laws
  (associativity, commutativity and $\pzero$ as identity) for parallel
  composition $|$ and for summation $+$.
\end{definition}

\subsection{Name equivalence}

We take name equivalence, written $\nameeq$, to be the smallest
equivalence relation generated by the following rules.

\begin{mathpar}
\inferrule*[lab=Quote-drop]
{ }
{ \quotep{@{x}} \nameeq x }

\inferrule*[lab=Struct-equiv]
{ P \scong Q }
{ \quotep{P} \nameeq \quotep{Q} }
\end{mathpar}

The astute reader will have noticed that the mutual recursion of names
and processes imposes a mutual recursion on alpha-equivalence and
structural equivalence via name-equivalence. Fortunately, all of this
works out pleasantly and we may calculate in the natural way, free of
concern. The reader interested in the details is referred to the
appendix \ref{appendix:rho_details}.

\subsection{Substitution}

We use $\Proc$ for the set of processes, $\QProc$ for the set of
names, and $\id{\{}\vec{y} / \vec{x} \id{\}}$ to denote partial maps,
$s : \QProc \rightarrow \QProc$. A map, $s$ lifts, uniquely, to a map
on process terms, $\widehat{s} : \Proc \rightarrow \Proc$ by the
following equations.

\begin{mathpar}
  (0) \psubstp{Q}{P} := 0 \\
  (R \juxtap S) \psubstp{Q}{P}
  :=    
  (R)\psubstp{Q}{P} \juxtap (S) \psubstp{Q}{P} \\
  (x?(y).R) \psubstp{Q}{P}    
  :=    
  (x)\substp{Q}{P} (z)\concat( (R \psubstn{z}{y}) \psubstp{Q}{P} ) \\
  (\lift{x}{R}) \psubstp{Q}{P}  
  :=
  \lift{(x)\substp{Q}{P}}{ R \psubstp{Q}{P} } \\
%   (\dropn{x})  \psubstp{Q}{P}       
%   := 
%   \left\{ 
%     \begin{array}{ccc} 
%       \dropn{\quotep{Q}} & & x \nameeq \quotep{P} \\
%       \dropn{x} & & otherwise \\
%     \end{array}
%   \right. 
  (\dropn{x})  \psubstp{Q}{P}       
  := 
  \left\{ 
    \begin{array}{ccc} 
      Q & & x \nameeq \quotep{P} \\
      \dropn{x} & & otherwise \\
    \end{array}
  \right.
\end{mathpar}
 

where

\begin{eqnarray}
  (x)\id{\{} \lpquote Q \rpquote / \lpquote P \rpquote \id{\}}            = 
  \left\{ 
    \begin{array}{ccc}
      \lpquote Q \rpquote & & x \nameeq \lpquote P \rpquote \\
      x & & otherwise \\
    \end{array}
  \right. \nonumber
\end{eqnarray}

and $z$ is chosen distinct from $\quotep{P}$, $\quotep{Q}$, the free
names in $Q$, and all the names in $R$. Our $\alpha$-equivalence will
be built in the standard way from this substitution.

\begin{remark}\label{rem:no_self_referential_names}
  One consequence of these definitions is that $\forall P. \quotep{P}
  \not\in \freenames{P}$.
\end{remark}

\subsection{ Dynamic quote: an example }

Anticipating something of what's to come, consider applying the
substitution, $\widehat{\id{\{}u / z \id{\}}}$, to the following pair
of processes, $\lift{w}{y!(z)}$ and $w[ \lpquote y!(z) \rpquote ]$.

\begin{eqnarray}
	\lift{w}{y!(z)}\widehat{\id{\{}u / z \id{\}}}
		& = &
		\lift{w}{y!(u)} \nonumber\\
	w[ \lpquote y!(z) \rpquote ] \widehat{ \id{\{}u / z \id{\}} }
		& = &
		w[ \lpquote y!(z) \rpquote ] \nonumber
\end{eqnarray}

Because the body of the process between quotes is impervious to
substitution, we get radically different answers. In fact, by
examining the first process in an input context,
e.g. $x?(z).\lift{w}{y!(z)}$, we see that the process under the lift
operator may be shaped by prefixed inputs binding a name inside it. In
this sense, the lift operator will be seen as a way to dynamically
construct processes before reifying them as names.

Finally equipped with these standard features we can present the
dynamics of the calculus.

\subsubsection{Operational semantics} 

Finally, we introduce the computational dynamics. What marks these
algebras as distinct from other more traditionally studied algebraic
structures, e.g. vector spaces or polynomial rings, is the manner in
which dynamics is captured. In traditional structures, dynamics is typically
expressed through morphisms between such structures, as in linear maps
between vector spaces or morphisms between rings. In algebras
associated with the semantics of computation, the dynamics is
expressed as part of the algebraic structure itself, through a
reduction reduction relation typically denoted by $\red$. Below, we
give a recursive presentation of this relation for the calculus used
in the encoding.

$\red \subseteq \pi \times \pi$
$\red : \pi \to \mathcal{P}(\pi)$

\begin{mathpar}
  \inferrule* [lab=Comm] { \textsf{match}( x_{src}, x_{trgt} ) } { x_{trgt}?(y)P \; | \; x_{src}!\langle {Q} \rangle \red P\{\quotep{Q}/y}\} }
  \and \\
  \inferrule* [lab=Par] {{P} \red {P}'} {{{P} | {Q}} \red {{P}' | {Q}}}
  \and
  \inferrule* [lab=Equiv]{{{P} \scong {P}'} \andalso {{P}' \red {Q}'} \andalso {{Q}' \scong {Q}}}{{P} \red {Q}}
\end{mathpar}

\begin{eqnarray*}
  match_{\equiv} (\quotep{P},\quotep{Q}) & := & P \equiv Q \\
  match_{\dagger}(\quotep{P},\quotep{Q}) & := & \forall R. P|Q \red^{*} R => R \red^{*} 0 \\
  match_{K}(\quotep{P},\quotep{Q}) & := & K \mbox{ for some context } K
\end{eqnarray*}

$u?(x)P | u!\langle Q \rangle \red P\{\quotep{Q}/x\}$

%We write $\wred$ for $\red^*$, and $P\red$ if $\exists Q $ such that $ P \red Q$.
We write $P\red$ if $\exists Q $ such that $ P \red Q$ and $P\not\red$, otherwise.

\section{Replication}

As mentioned before, it is known that replication (and hence
recursion) can be implemented in a higher-order process algebra
\cite{SangiorgiWalker}. As our first example of calculation with the
machinery thus far presented we give the construction explicitly in
the {\rhoc}.

\begin{eqnarray}
	D_{x} & := & \prefix{x}{y}{(\binpar{\outputp{x}{y}}{@{y}})} \nonumber\\
	\bangp_{x}{P} & := & \binpar{{x}!\langle{\binpar{D_{x}}{P}}\rangle}{D_{x}} \nonumber
\end{eqnarray}

\begin{eqnarray}
	\bangp_{x}{P} & & \nonumber\\
	=
	& {x}!\langle{(\prefix{x}{y}{(\outputp{x}{y} | @{y})) | P}}\rangle 
	      | \prefix{x}{y}{(\outputp{x}{y} | @{y})} & \nonumber\\
	\red
	& (\outputp{x}{y} | @{y})\substn{\quotep{(\prefix{x}{y}{(@{y} | \outputp{x}{y})) | P}}}{y} & \nonumber\\
	=
	& \outputp{x}{\quotep{(\prefix{x}{y}{(\outputp{x}{y} | @{y})) | P}}}
	  | {(\prefix{x}{y}{(\outputp{x}{y} | @{y})) | P}} & \nonumber\\
	\red
	& \ldots & \nonumber\\
	\red^*
	& P | P | \ldots & \nonumber
\end{eqnarray}

Of course, this encoding, as an implementation, runs away, unfolding
$\bangp{P}$ eagerly. A lazier and more implementable replication
operator, restricted to input-guarded processes, may be obtained as follows.

\begin{eqnarray}
\bangp{\prefix{u}{v}{P}} 
	:= 
	\binpar{\lift{x}{\prefix{u}{v}{(\binpar{D(x)}{P})}}}{D(x)} \nonumber
\end{eqnarray}

\begin{remark}
  Note that the lazier definition still does not deal with summation
  or mixed summation (i.e. sums over input and output). The reader is
  invited to construct definitions of replication that deal with these
  features. 

  Further, the definitions are parameterized in a name, $x$. Can you,
  gentle reader, make a definition that eliminates this parameter and
  guarantees no accidental interaction between the replication
  machinery and the process being replicated -- i.e. no accidental
  sharing of names used by the process to get its work done and the
  name(s) used by the replication to effect copying. This latter
  revision of the definition of replication is crucial to obtaining
  the expected identity $!!P \sim !P$.
\end{remark}

\begin{remark}\label{rem:paradoxical_combinator}
  The reader familiar with the lambda calculus will have noticed the
  similarity between $D$ and the paradoxical combinator.

  [Ed. note: the existence of this seems to suggest we have to be more
  restrictive on the set of processes and names we admit if we are to
  support no-cloning.]
\end{remark}

\subsubsection{Bisimulation}

The computational dynamics gives rise to another kind of equivalence,
the equivalence of computational behavior. As previously mentioned
this is typically captured \emph{via} some form of bisimulation.

% The notion we use in this paper is weak barbed bisimulation
% \cite{milner91polyadicpi}.

The notion we use in this paper is derived from weak barbed
bisimulation \cite{milner91polyadicpi}. 

\begin{definition}
An \emph{observation relation}, $\downarrow_{\mathcal N}$, over a set
of names, $\mathcal N$, is the smallest relation satisfying the rules
below.

\infrule[Out-barb]{y \in {\mathcal N}, \; x \nameeq y}
		  {\outputp{x}{v} \downarrow_{\mathcal N} x}
\infrule[Par-barb]{\mbox{$P\downarrow_{\mathcal N} x$ or $Q\downarrow_{\mathcal N} x$}}
		  {\binpar{P}{Q} \downarrow_{\mathcal N} x}

We write $P \Downarrow_{\mathcal N} x$ if there is $Q$ such that 
$P \wred Q$ and $Q \downarrow_{\mathcal N} x$.
\end{definition}

\begin{definition}
%\label{def.bbisim}
An  ${\mathcal N}$-\emph{barbed bisimulation} over a set of names, ${\mathcal N}$, is a symmetric binary relation 
${\mathcal S}_{\mathcal N}$ between agents such that $P\rel{S}_{\mathcal N}Q$ implies:
\begin{enumerate}
\item If $P \red P'$ then $Q \wred Q'$ and $P'\rel{S}_{\mathcal N} Q'$.
\item If $P\downarrow_{\mathcal N} x$, then $Q\Downarrow_{\mathcal N} x$.
\end{enumerate}
$P$ is ${\mathcal N}$-barbed bisimilar to $Q$, written
$P \wbbisim_{\mathcal N} Q$, if $P \rel{S}_{\mathcal N} Q$ for some ${\mathcal N}$-barbed bisimulation ${\mathcal S}_{\mathcal N}$.
\end{definition}

$\mathcal{R} \subseteq \pi \times \pi$

$P \mathcal{R} Q => \forall P'. P \red P' \Rightarrow \exists Q'. Q \red Q', P' \mathcal{R} Q'$

$P \vdash x \Rightarrow Q \vdash x$

\begin{mathpar}
  \inferrule*[lab=Out-barb]{x \nameeq y}{{y}!\langle{Q}\rangle \vdash x}
  \and
  \inferrule*[lab=Par-barb]{\mbox{$P\vdash x$ or $Q\vdash x$}}{\binpar{P}{Q} \vdash x}
\end{mathpar}

\subsubsection{Contexts}

One of the principle advantages of computational calculi like the
$\pi$-calculus is a well-defined notion of context,
contextual-equivalence and a correlation between
contextual-equivalence and notions of bisimulation. The notion of
context allows the decomposition of a process into (sub-)process and
its syntactic environment, its context. Thus, a context may be
thought of as a process with a ``hole'' (written $\Box$) in it. The
application of a context $M$ to a process $P$, written $M[P]$, is
tantamount to filling the hole in $M$ with $P$. In this paper we do
not need the full weight of this theory, but do make use of the notion
of context in the proof the main theorem. 

\begin{mathpar}
  \inferrule* [lab=summation] {} {{M_{M},M_{N}} \bc \Box \;|\; x.M_{A} \;|\; M_{M}+M_{N}}
  \and
  \inferrule* [lab=agent] {} {{M_{A}} \bc (\vec{x})M_{P} \;| \; \clift{P_0,\ldots,M_{P},\ldots,P_N}}
  \and \\
  \inferrule* [lab=process] {} {{M_{P}} \bc M_{N} \;| \;P|M_{P} }
\end{mathpar} 

\begin{mathpar}
  \inferrule* [lab=sychronization] {} {M_{N} \bc \Box \;|\; x?M_{F} \;|\; x!M_{C}}
  \and
  \inferrule* [lab=abstraction] {} {{M_{F}} \bc (x)M_{P} }
  \and
  \inferrule* [lab=concretion] {} {{M_{C}} \bc \langle M_{P} \rangle }
  \and \\
  \inferrule* [lab=process] {} {{M_{P}} \bc M_{N} \;| \;P|M_{P} }
\end{mathpar}

\begin{definition}[contextual application] Given a context $M$, and
  process $P$, we define the \emph{contextual application}, $M[P] :=
  M\{P/\Box\}$. That is, the contextual application of M to P is the
  substitution of $P$ for $\Box$ in $M$.
\end{definition}

$\meaningof{-} : L \to \mathcal{P}(\pi)$

\begin{mathpar}
  \inferrule* [lab=collection] {} {\meaningof{true} = \pi, \and \meaningof{~E} = \pi \setminus \meaningof{E}, \and \meaningof{E_{1} \& E_{2}} = \meaningof{E_{1}} \cap \meaningof{E_{2}}}
\end{mathpar}

\begin{mathpar}
  \inferrule* [lab=structure] {} {\meaningof{0} = \{ P \in \pi | P \equiv 0 \}, \and \\ \meaningof{E_1 | E_2} = \{ P \in \pi | P \equiv P_{1} | P_{2}, P_{1} \in \meaningof{E_{1}}, P_{2} \in \meaningof{E_2}\} }
\end{mathpar}

\begin{mathpar}
 \inferrule* [lab=behavior] {} {\meaningof{\langle a?b \rangle E} = \{ P \in \pi | P \equiv Q | u?(y)P', \\ \and \\\\ \and \\ \;\;\; u \in \meaningof{a}, \forall z.P'\{z/y\} \in \meaningof{E\{z/b\}}\}, \and \\ \meaningof{a!E} = \{ P \in \pi | P \equiv Q | x!\langle P' \rangle, x \in \meaningof{a} P' \in \meaningof{E}\} }
\end{mathpar}

\begin{mathpar}
 \inferrule* [lab=nominal] {} {\meaningof{\quotep{E}} = \{ \quotep{P} \in \quotep{\pi} | P \in \meaningof{E} \}, \and \meaningof{\quotep{P}} = \{ \quotep{Q} \in \quotep{\pi} | P \equiv Q \} \and \\ \meaningof{@\quotep{E}} = \{ P \in \pi | P \equiv @x, x \in \meaningof{E} \}}
\end{mathpar}

\begin{eqnarray*}
  \\
  \meaningof{-} : TS \to ST
\end{eqnarray*}

\begin{eqnarray*}
  \\
  L : TS \to ST
\end{eqnarray*}

\begin{eqnarray*}
  \\
  P \models E \iff P \in \meaningof{E}
\end{eqnarray*}

\begin{eqnarray*}
  P \approx_{L} Q \iff \forall E \in L. P \models E \iff Q \models E
\end{eqnarray*}

\begin{eqnarray*}
  P \approx_{K} Q
\end{eqnarray*}

\begin{eqnarray*}
  P \approx Q
\end{eqnarray*}

$\approx_{K} = \approx = \approx_{L}$

\subsubsection{Contextual duality}

Note that contexts extend the quotation operation to a family of
operations from processes to names. Given a context, $M$, we can
define a \emph{nominal context}, $\quotep{M}$ by $\quotep{M}[P] :=
\quotep{M[P]}$. To foreshadow what is to come we observe that these
operations enjoy a duality with processes very much like the duality
between vectors and maps from vectors to scalars.

Further, because the calculus is essentially higher-order, we have a
correspondence between contexts and processes. More specifically,
given a name $x$ and a context $M$ we can construct $M^{*}_{x}$ such
that 

\begin{mathpar}
  M^{*}_{x} | \lift{x}{P} \red M[P]
\end{mathpar}

namely,

\begin{mathpar}
  M^{*}_{x} := x?(u).M[\dropn{u}]
\end{mathpar}

The dependence of $M^{*}_{x}$ on a name makes it an abstraction, 

\begin{mathpar}
  M^{*} := (x)x?(u).M[\dropn{u}]
\end{mathpar}

\subsection{Additional notation}

It will sometimes be convenient to denote the process a name
quotes. We already have the notation $x = \quotep{P}$, but it will be
convenient to introduce an alternate notation, $\procn{x}$, when we
want to emphasize the connection to the use of the name. Note that, by
virtue of name equivalence, $\quotep{\procn{x}} \nameeq x$; so, the
notation is consistent with previous definitions.

Further, because names have structure it is possible to effect
substitutions on the basis of that structure. This means we need to
upgrade our notation for substitutions, which we accomplish by
adapting comprehension notation. Thus,

\begin{mathpar}
  P\{ y / x : x \in S \}
\end{mathpar}

is interpreted to mean the process derived from P by replacing (in a
capture-avoiding manner) each occurrence of $x$ in $S$ by $y$. For example,

\begin{mathpar}
  P\{ \quotep{\procn{x}|\procn{x}} / x : x \in \freenames{P} \}
\end{mathpar}

will replace each (occurrence) of a free name $x$ in $P$ by
$\quotep{\procn{x}|\procn{x}}$.

Also, we will avail ourselves of the notation $x^{L}$ and $x^{R}$ to
denote injections of a name into disjoint copies of the name
space. There are numerous ways to accomplish this. One example can be
found in \cite{MeredithR05}. This notation overloads to vectors of
names: $\vec{x}^{\pi} := (x_{i}^{\pi} \; : \; 0 \leq i < |\vec{x}| )$ where $\pi \in \{L,R\}$.

We also use $P^{\Box} := P|\Box$.

In \cite{MeredithR05} an interpretation of the new operator is
given. It turns out that there are several possible interpretations
all enjoying the requisite algebraic properties of the operator (see
\cite{milner91polyadicpi}). We will therefore make liberal use of
$(\nu\; \vec{x})P$.

% subsection the_syntax_and_semantics_of_the_notation_system (end)   

\input{qm2pi.qmops} 

\input{qm2pi.sterngerlach} 

\input{qm2pi.metric} 

% section concurrent_process_calculi (end)

%\input{qm2pi.proofsketch}

% section proof sketch (end)

%\input{qm2pi.slviaknots} 

% section spatial logic via knots (end)

\input{qm2pi.conclusion}

% section conclusion (end)

%\input{qm2pi.dtcodes} 

% section wiring algorithm (end)

\input{qm2pi.ack} 

% section acknowledgments (end)

\newpage


\bibliographystyle{plain}   
\bibliography{../../biblios/main.bib}

\input{qm2pi.rhodetails}

\end{document}

 

% subsection basic_interpretation (end)

%\input{qm2pi.rho.presentation} 
\subsection{The syntax and semantics of the notation system}\label{sub:the_syntax_and_semantics_of_the_notation_system} % (fold)

We now summarize a technical presentation of the calculus that
embodies our theory of dynamics. The typical presentation of such a
calculus follows the style of giving generators and relations on
them. The grammar, below, describing term constructors, freely
generates the set of processes, $\Proc$. This set is then quotiented
by a relation known as structural congruence and it is over this set
that the notion of dynamics is expressed. This presentation is
essentially that of \cite{MeredithR05} with the addition of
polyadicity and summation. For readability we have relegated some of
the technical subtleties to an appendix.

\subsubsection{Process grammar}\label{subsub:process_grammar}

\begin{mathpar}
  \inferrule* [lab=synchronization] {} {{M} \bc \pzero \;|\; x?F \;|\; x!C }
  \and
  \inferrule* [lab=abstraction] {} {{F} \bc (x)P}
  \and
  \inferrule* [lab=concretion] {} {{C} \bc \langle Q \rangle}
  \and
  \inferrule* [lab=process] {} {{P,Q} \bc M \;| \;P|Q \;|\; @{x}}
  \and
  \inferrule* [lab=name] {} {{x} \bc \quotep{P}}
\end{mathpar} 

Note that $\vec{x}$ (resp. $\vec{P}$) denotes a vector of names
(resp. processes) of length $|\vec{x}|$ (resp. $|\vec{P}|$). We adopt
the following useful abbreviations.

\begin{mathpar}
   x?(\vec{y}).P := x.(\vec{y})P \and  x\clift{\vec{P}} := x.\clift{\vec{P}}
   \and x!(y) := \lift{x}{\dropn{y}}
   \and \Pi_{i=0}^{n-1}P_i := P_0 | \ldots | P_{n-1}
\end{mathpar}

\subsubsection{Structural congruence}

\paragraph{Free and bound names and alpha-equivalence.} At the
core of structural equivalence is alpha-equivalence which identifies
process that are the same up to a change of variable. Formally, we
recognize the distinction between free and bound names. The free names
of a process, $\freenames{P}$, may be calculated recursively as
follows:

\begin{mathpar}
\freenames{\pzero} := \emptyset
  \and \\
  \freenames{x?(y).P} := \{ x \} \cup (\freenames{P} \setminus \{ y \})
  \and 
  \freenames{x!\langle P \rangle} := \{ x \} \cup \{ P \} 
  \and \\
  \freenames{P|Q} := \freenames{P} \cup \freenames{Q}
  \and \\
  \freenames{@{x}} := \{ x \}
\end{mathpar}

$\pi$
$\quotep{\pi}$

$\freenames{-} : \pi \to \mathcal{P}(\quotep{\pi})$

\begin{eqnarray*}
  \freenames{\pzero} & := & \emptyset \\
  \freenames{x?(y).P} & := & \{ x \} \cup (\freenames{P} \setminus \{ y \}) \\
  \freenames{x!\langle P \rangle} & := & \{ x \} \cup \{ P \} \\
  \freenames{P|Q} & := & \freenames{P} \cup \freenames{Q} \\
  \freenames{\dropn{x}} & := & \{ x \}
\end{eqnarray*}

The bound names of a process, $\boundnames{P}$, are those names occurring in $P$
that are not free. For example, in $x?(y).0$, the name $x$ is free, while $y$ is bound.

\begin{mathpar}
  \inferrule* [lab=monoidal-laws] {} { P|Q \equiv Q|P \and P|0 \equiv P \and P|(Q|R) \equiv (P|Q)|R }
\end{mathpar}

\begin{mathpar}
  \inferrule* [lab=alpha-equivalence] {} { (x)P \equiv (y)P\{y/x\} \and y \not\in \freenames{P} }
\end{mathpar}

\begin{definition}
Then two processes, $P,Q$, are alpha-equivalent if $P = Q\{\vec{y}/\vec{x}\}$ for
some $\vec{x} \in \boundnames{Q},\vec{y} \in \boundnames{P}$, where $Q\{\vec{y}/\vec{x}\}$
denotes the capture-avoiding substitution of $\vec{y}$ for $\vec{x}$ in $Q$.
\end{definition}

\begin{definition}
  The {\em structural congruence} \cite{SangiorgiWalker} , $\equiv$,
  between processes is the least congruence containing
  alpha-equivalence, satisfying the abelian monoid laws
  (associativity, commutativity and $\pzero$ as identity) for parallel
  composition $|$ and for summation $+$.
\end{definition}

\subsection{Name equivalence}

We take name equivalence, written $\nameeq$, to be the smallest
equivalence relation generated by the following rules.

\begin{mathpar}
\inferrule*[lab=Quote-drop]
{ }
{ \quotep{@{x}} \nameeq x }

\inferrule*[lab=Struct-equiv]
{ P \scong Q }
{ \quotep{P} \nameeq \quotep{Q} }
\end{mathpar}

The astute reader will have noticed that the mutual recursion of names
and processes imposes a mutual recursion on alpha-equivalence and
structural equivalence via name-equivalence. Fortunately, all of this
works out pleasantly and we may calculate in the natural way, free of
concern. The reader interested in the details is referred to the
appendix \ref{appendix:rho_details}.

\subsection{Substitution}

We use $\Proc$ for the set of processes, $\QProc$ for the set of
names, and $\id{\{}\vec{y} / \vec{x} \id{\}}$ to denote partial maps,
$s : \QProc \rightarrow \QProc$. A map, $s$ lifts, uniquely, to a map
on process terms, $\widehat{s} : \Proc \rightarrow \Proc$ by the
following equations.

\begin{mathpar}
  (0) \psubstp{Q}{P} := 0 \\
  (R \juxtap S) \psubstp{Q}{P}
  :=    
  (R)\psubstp{Q}{P} \juxtap (S) \psubstp{Q}{P} \\
  (x?(y).R) \psubstp{Q}{P}    
  :=    
  (x)\substp{Q}{P} (z)\concat( (R \psubstn{z}{y}) \psubstp{Q}{P} ) \\
  (\lift{x}{R}) \psubstp{Q}{P}  
  :=
  \lift{(x)\substp{Q}{P}}{ R \psubstp{Q}{P} } \\
%   (\dropn{x})  \psubstp{Q}{P}       
%   := 
%   \left\{ 
%     \begin{array}{ccc} 
%       \dropn{\quotep{Q}} & & x \nameeq \quotep{P} \\
%       \dropn{x} & & otherwise \\
%     \end{array}
%   \right. 
  (\dropn{x})  \psubstp{Q}{P}       
  := 
  \left\{ 
    \begin{array}{ccc} 
      Q & & x \nameeq \quotep{P} \\
      \dropn{x} & & otherwise \\
    \end{array}
  \right.
\end{mathpar}
 

where

\begin{eqnarray}
  (x)\id{\{} \lpquote Q \rpquote / \lpquote P \rpquote \id{\}}            = 
  \left\{ 
    \begin{array}{ccc}
      \lpquote Q \rpquote & & x \nameeq \lpquote P \rpquote \\
      x & & otherwise \\
    \end{array}
  \right. \nonumber
\end{eqnarray}

and $z$ is chosen distinct from $\quotep{P}$, $\quotep{Q}$, the free
names in $Q$, and all the names in $R$. Our $\alpha$-equivalence will
be built in the standard way from this substitution.

\begin{remark}\label{rem:no_self_referential_names}
  One consequence of these definitions is that $\forall P. \quotep{P}
  \not\in \freenames{P}$.
\end{remark}

\subsection{ Dynamic quote: an example }

Anticipating something of what's to come, consider applying the
substitution, $\widehat{\id{\{}u / z \id{\}}}$, to the following pair
of processes, $\lift{w}{y!(z)}$ and $w[ \lpquote y!(z) \rpquote ]$.

\begin{eqnarray}
	\lift{w}{y!(z)}\widehat{\id{\{}u / z \id{\}}}
		& = &
		\lift{w}{y!(u)} \nonumber\\
	w[ \lpquote y!(z) \rpquote ] \widehat{ \id{\{}u / z \id{\}} }
		& = &
		w[ \lpquote y!(z) \rpquote ] \nonumber
\end{eqnarray}

Because the body of the process between quotes is impervious to
substitution, we get radically different answers. In fact, by
examining the first process in an input context,
e.g. $x?(z).\lift{w}{y!(z)}$, we see that the process under the lift
operator may be shaped by prefixed inputs binding a name inside it. In
this sense, the lift operator will be seen as a way to dynamically
construct processes before reifying them as names.

Finally equipped with these standard features we can present the
dynamics of the calculus.

\subsubsection{Operational semantics} 

Finally, we introduce the computational dynamics. What marks these
algebras as distinct from other more traditionally studied algebraic
structures, e.g. vector spaces or polynomial rings, is the manner in
which dynamics is captured. In traditional structures, dynamics is typically
expressed through morphisms between such structures, as in linear maps
between vector spaces or morphisms between rings. In algebras
associated with the semantics of computation, the dynamics is
expressed as part of the algebraic structure itself, through a
reduction reduction relation typically denoted by $\red$. Below, we
give a recursive presentation of this relation for the calculus used
in the encoding.

$\red \subseteq \pi \times \pi$
$\red : \pi \to \mathcal{P}(\pi)$

\begin{mathpar}
  \inferrule* [lab=Comm] { \textsf{match}( x_{src}, x_{trgt} ) } { x_{trgt}?(y)P \; | \; x_{src}!\langle {Q} \rangle \red P\{\quotep{Q}/y}\} }
  \and \\
  \inferrule* [lab=Par] {{P} \red {P}'} {{{P} | {Q}} \red {{P}' | {Q}}}
  \and
  \inferrule* [lab=Equiv]{{{P} \scong {P}'} \andalso {{P}' \red {Q}'} \andalso {{Q}' \scong {Q}}}{{P} \red {Q}}
\end{mathpar}

\begin{eqnarray*}
  match_{\equiv} (\quotep{P},\quotep{Q}) & := & P \equiv Q \\
  match_{\dagger}(\quotep{P},\quotep{Q}) & := & \forall R. P|Q \red^{*} R => R \red^{*} 0 \\
  match_{K}(\quotep{P},\quotep{Q}) & := & K \mbox{ for some context } K
\end{eqnarray*}

$u?(x)P | u!\langle Q \rangle \red P\{\quotep{Q}/x\}$

%We write $\wred$ for $\red^*$, and $P\red$ if $\exists Q $ such that $ P \red Q$.
We write $P\red$ if $\exists Q $ such that $ P \red Q$ and $P\not\red$, otherwise.

\section{Replication}

As mentioned before, it is known that replication (and hence
recursion) can be implemented in a higher-order process algebra
\cite{SangiorgiWalker}. As our first example of calculation with the
machinery thus far presented we give the construction explicitly in
the {\rhoc}.

\begin{eqnarray}
	D_{x} & := & \prefix{x}{y}{(\binpar{\outputp{x}{y}}{@{y}})} \nonumber\\
	\bangp_{x}{P} & := & \binpar{{x}!\langle{\binpar{D_{x}}{P}}\rangle}{D_{x}} \nonumber
\end{eqnarray}

\begin{eqnarray}
	\bangp_{x}{P} & & \nonumber\\
	=
	& {x}!\langle{(\prefix{x}{y}{(\outputp{x}{y} | @{y})) | P}}\rangle 
	      | \prefix{x}{y}{(\outputp{x}{y} | @{y})} & \nonumber\\
	\red
	& (\outputp{x}{y} | @{y})\substn{\quotep{(\prefix{x}{y}{(@{y} | \outputp{x}{y})) | P}}}{y} & \nonumber\\
	=
	& \outputp{x}{\quotep{(\prefix{x}{y}{(\outputp{x}{y} | @{y})) | P}}}
	  | {(\prefix{x}{y}{(\outputp{x}{y} | @{y})) | P}} & \nonumber\\
	\red
	& \ldots & \nonumber\\
	\red^*
	& P | P | \ldots & \nonumber
\end{eqnarray}

Of course, this encoding, as an implementation, runs away, unfolding
$\bangp{P}$ eagerly. A lazier and more implementable replication
operator, restricted to input-guarded processes, may be obtained as follows.

\begin{eqnarray}
\bangp{\prefix{u}{v}{P}} 
	:= 
	\binpar{\lift{x}{\prefix{u}{v}{(\binpar{D(x)}{P})}}}{D(x)} \nonumber
\end{eqnarray}

\begin{remark}
  Note that the lazier definition still does not deal with summation
  or mixed summation (i.e. sums over input and output). The reader is
  invited to construct definitions of replication that deal with these
  features. 

  Further, the definitions are parameterized in a name, $x$. Can you,
  gentle reader, make a definition that eliminates this parameter and
  guarantees no accidental interaction between the replication
  machinery and the process being replicated -- i.e. no accidental
  sharing of names used by the process to get its work done and the
  name(s) used by the replication to effect copying. This latter
  revision of the definition of replication is crucial to obtaining
  the expected identity $!!P \sim !P$.
\end{remark}

\begin{remark}\label{rem:paradoxical_combinator}
  The reader familiar with the lambda calculus will have noticed the
  similarity between $D$ and the paradoxical combinator.

  [Ed. note: the existence of this seems to suggest we have to be more
  restrictive on the set of processes and names we admit if we are to
  support no-cloning.]
\end{remark}

\subsubsection{Bisimulation}

The computational dynamics gives rise to another kind of equivalence,
the equivalence of computational behavior. As previously mentioned
this is typically captured \emph{via} some form of bisimulation.

% The notion we use in this paper is weak barbed bisimulation
% \cite{milner91polyadicpi}.

The notion we use in this paper is derived from weak barbed
bisimulation \cite{milner91polyadicpi}. 

\begin{definition}
An \emph{observation relation}, $\downarrow_{\mathcal N}$, over a set
of names, $\mathcal N$, is the smallest relation satisfying the rules
below.

\infrule[Out-barb]{y \in {\mathcal N}, \; x \nameeq y}
		  {\outputp{x}{v} \downarrow_{\mathcal N} x}
\infrule[Par-barb]{\mbox{$P\downarrow_{\mathcal N} x$ or $Q\downarrow_{\mathcal N} x$}}
		  {\binpar{P}{Q} \downarrow_{\mathcal N} x}

We write $P \Downarrow_{\mathcal N} x$ if there is $Q$ such that 
$P \wred Q$ and $Q \downarrow_{\mathcal N} x$.
\end{definition}

\begin{definition}
%\label{def.bbisim}
An  ${\mathcal N}$-\emph{barbed bisimulation} over a set of names, ${\mathcal N}$, is a symmetric binary relation 
${\mathcal S}_{\mathcal N}$ between agents such that $P\rel{S}_{\mathcal N}Q$ implies:
\begin{enumerate}
\item If $P \red P'$ then $Q \wred Q'$ and $P'\rel{S}_{\mathcal N} Q'$.
\item If $P\downarrow_{\mathcal N} x$, then $Q\Downarrow_{\mathcal N} x$.
\end{enumerate}
$P$ is ${\mathcal N}$-barbed bisimilar to $Q$, written
$P \wbbisim_{\mathcal N} Q$, if $P \rel{S}_{\mathcal N} Q$ for some ${\mathcal N}$-barbed bisimulation ${\mathcal S}_{\mathcal N}$.
\end{definition}

$\mathcal{R} \subseteq \pi \times \pi$

$P \mathcal{R} Q => \forall P'. P \red P' \Rightarrow \exists Q'. Q \red Q', P' \mathcal{R} Q'$

$P \vdash x \Rightarrow Q \vdash x$

\begin{mathpar}
  \inferrule*[lab=Out-barb]{x \nameeq y}{{y}!\langle{Q}\rangle \vdash x}
  \and
  \inferrule*[lab=Par-barb]{\mbox{$P\vdash x$ or $Q\vdash x$}}{\binpar{P}{Q} \vdash x}
\end{mathpar}

\subsubsection{Contexts}

One of the principle advantages of computational calculi like the
$\pi$-calculus is a well-defined notion of context,
contextual-equivalence and a correlation between
contextual-equivalence and notions of bisimulation. The notion of
context allows the decomposition of a process into (sub-)process and
its syntactic environment, its context. Thus, a context may be
thought of as a process with a ``hole'' (written $\Box$) in it. The
application of a context $M$ to a process $P$, written $M[P]$, is
tantamount to filling the hole in $M$ with $P$. In this paper we do
not need the full weight of this theory, but do make use of the notion
of context in the proof the main theorem. 

\begin{mathpar}
  \inferrule* [lab=summation] {} {{M_{M},M_{N}} \bc \Box \;|\; x.M_{A} \;|\; M_{M}+M_{N}}
  \and
  \inferrule* [lab=agent] {} {{M_{A}} \bc (\vec{x})M_{P} \;| \; \clift{P_0,\ldots,M_{P},\ldots,P_N}}
  \and \\
  \inferrule* [lab=process] {} {{M_{P}} \bc M_{N} \;| \;P|M_{P} }
\end{mathpar} 

\begin{mathpar}
  \inferrule* [lab=sychronization] {} {M_{N} \bc \Box \;|\; x?M_{F} \;|\; x!M_{C}}
  \and
  \inferrule* [lab=abstraction] {} {{M_{F}} \bc (x)M_{P} }
  \and
  \inferrule* [lab=concretion] {} {{M_{C}} \bc \langle M_{P} \rangle }
  \and \\
  \inferrule* [lab=process] {} {{M_{P}} \bc M_{N} \;| \;P|M_{P} }
\end{mathpar}

\begin{definition}[contextual application] Given a context $M$, and
  process $P$, we define the \emph{contextual application}, $M[P] :=
  M\{P/\Box\}$. That is, the contextual application of M to P is the
  substitution of $P$ for $\Box$ in $M$.
\end{definition}

$\meaningof{-} : L \to \mathcal{P}(\pi)$

\begin{mathpar}
  \inferrule* [lab=collection] {} {\meaningof{true} = \pi, \and \meaningof{~E} = \pi \setminus \meaningof{E}, \and \meaningof{E_{1} \& E_{2}} = \meaningof{E_{1}} \cap \meaningof{E_{2}}}
\end{mathpar}

\begin{mathpar}
  \inferrule* [lab=structure] {} {\meaningof{0} = \{ P \in \pi | P \equiv 0 \}, \and \\ \meaningof{E_1 | E_2} = \{ P \in \pi | P \equiv P_{1} | P_{2}, P_{1} \in \meaningof{E_{1}}, P_{2} \in \meaningof{E_2}\} }
\end{mathpar}

\begin{mathpar}
 \inferrule* [lab=behavior] {} {\meaningof{\langle a?b \rangle E} = \{ P \in \pi | P \equiv Q | u?(y)P', \\ \and \\\\ \and \\ \;\;\; u \in \meaningof{a}, \forall z.P'\{z/y\} \in \meaningof{E\{z/b\}}\}, \and \\ \meaningof{a!E} = \{ P \in \pi | P \equiv Q | x!\langle P' \rangle, x \in \meaningof{a} P' \in \meaningof{E}\} }
\end{mathpar}

\begin{mathpar}
 \inferrule* [lab=nominal] {} {\meaningof{\quotep{E}} = \{ \quotep{P} \in \quotep{\pi} | P \in \meaningof{E} \}, \and \meaningof{\quotep{P}} = \{ \quotep{Q} \in \quotep{\pi} | P \equiv Q \} \and \\ \meaningof{@\quotep{E}} = \{ P \in \pi | P \equiv @x, x \in \meaningof{E} \}}
\end{mathpar}

\begin{eqnarray*}
  \\
  \meaningof{-} : TS \to ST
\end{eqnarray*}

\begin{eqnarray*}
  \\
  L : TS \to ST
\end{eqnarray*}

\begin{eqnarray*}
  \\
  P \models E \iff P \in \meaningof{E}
\end{eqnarray*}

\begin{eqnarray*}
  P \approx_{L} Q \iff \forall E \in L. P \models E \iff Q \models E
\end{eqnarray*}

\begin{eqnarray*}
  P \approx_{K} Q
\end{eqnarray*}

\begin{eqnarray*}
  P \approx Q
\end{eqnarray*}

$\approx_{K} = \approx = \approx_{L}$

\subsubsection{Contextual duality}

Note that contexts extend the quotation operation to a family of
operations from processes to names. Given a context, $M$, we can
define a \emph{nominal context}, $\quotep{M}$ by $\quotep{M}[P] :=
\quotep{M[P]}$. To foreshadow what is to come we observe that these
operations enjoy a duality with processes very much like the duality
between vectors and maps from vectors to scalars.

Further, because the calculus is essentially higher-order, we have a
correspondence between contexts and processes. More specifically,
given a name $x$ and a context $M$ we can construct $M^{*}_{x}$ such
that 

\begin{mathpar}
  M^{*}_{x} | \lift{x}{P} \red M[P]
\end{mathpar}

namely,

\begin{mathpar}
  M^{*}_{x} := x?(u).M[\dropn{u}]
\end{mathpar}

The dependence of $M^{*}_{x}$ on a name makes it an abstraction, 

\begin{mathpar}
  M^{*} := (x)x?(u).M[\dropn{u}]
\end{mathpar}

\subsection{Additional notation}

It will sometimes be convenient to denote the process a name
quotes. We already have the notation $x = \quotep{P}$, but it will be
convenient to introduce an alternate notation, $\procn{x}$, when we
want to emphasize the connection to the use of the name. Note that, by
virtue of name equivalence, $\quotep{\procn{x}} \nameeq x$; so, the
notation is consistent with previous definitions.

Further, because names have structure it is possible to effect
substitutions on the basis of that structure. This means we need to
upgrade our notation for substitutions, which we accomplish by
adapting comprehension notation. Thus,

\begin{mathpar}
  P\{ y / x : x \in S \}
\end{mathpar}

is interpreted to mean the process derived from P by replacing (in a
capture-avoiding manner) each occurrence of $x$ in $S$ by $y$. For example,

\begin{mathpar}
  P\{ \quotep{\procn{x}|\procn{x}} / x : x \in \freenames{P} \}
\end{mathpar}

will replace each (occurrence) of a free name $x$ in $P$ by
$\quotep{\procn{x}|\procn{x}}$.

Also, we will avail ourselves of the notation $x^{L}$ and $x^{R}$ to
denote injections of a name into disjoint copies of the name
space. There are numerous ways to accomplish this. One example can be
found in \cite{MeredithR05}. This notation overloads to vectors of
names: $\vec{x}^{\pi} := (x_{i}^{\pi} \; : \; 0 \leq i < |\vec{x}| )$ where $\pi \in \{L,R\}$.

We also use $P^{\Box} := P|\Box$.

In \cite{MeredithR05} an interpretation of the new operator is
given. It turns out that there are several possible interpretations
all enjoying the requisite algebraic properties of the operator (see
\cite{milner91polyadicpi}). We will therefore make liberal use of
$(\nu\; \vec{x})P$.

% subsection the_syntax_and_semantics_of_the_notation_system (end)   

\section{Interpretation of QM}
\subsection{Supporting definitions}
\subsubsection{Multiplication}
\begin{mathpar}
  \quotep{Q} \cdot \quotep{R} := \quotep{Q|R}
  \and \\
  \quotep{Q} \cdot P := P\{ \quotep{Q|R} / \quotep{R} : \quotep{R} \in \freenames{P} \}
\end{mathpar}

\paragraph{Discussion}
The first line needs little explanation. The second line says that
each free name of the process is replaced with the multiplication of
that name by the scalar. Multiplication of a scalar (name) by a state
(process) results in a process all the names of which have been `moved
over' by parallel composition with the process the scalar
quotes. There is a subtlety that the bound names have to be
manipulated so that multiplied names aren't accidentally
captured. There are many ways to achieve this.

\begin{remark}\label{rem:multiplication_identities}
  The reader is invited to verify that for all $x,y,z \in \QProc$ and $P \in \Proc$
  \begin{mathpar}
    x \cdot \quotep{0} \equiv x 
    \and
    x \cdot y \equiv y \cdot x
    \and
    x \cdot (y \cdot z) \equiv (x \cdot y) \cdot z
    \and \\
    \quotep{0} \cdot P \equiv P
    \and \\
    x \cdot (y \cdot P) \equiv (x \cdot y) \cdot P
    \and \\
    x \cdot (P|Q) \equiv (x \cdot P) | (x \cdot Q)
    \and \\    
  \end{mathpar}
\end{remark}

\subsubsection{Tensor product}

We define a tensor product on processes by structural induction.

\paragraph{Tensor of sums} First note that all summations, including
$\pzero$ and sequence, can be written $\Sigma_{i} x_{i}.A_{i} +
\Sigma_{j} x_{j}.C_{j}$, where we have grouped input-guarded processes
together and output-guarded processes together.

Thus, we can define the tensor product of two summations, $N_{1}\otimes N_{2}$, where

\begin{mathpar}
  N_{1} := \Sigma_{i} x_{i}.A_{i} + \Sigma_{j} x_{j}.C_{j}
  \and
  N_{2} := \Sigma_{i'} y_{i'}.B_{i'} + \Sigma_{j'} y_{j'}.D_{j'} 
\end{mathpar}

as follows.

\begin{mathpar}
  \Sigma_{i} x_{i}.A_{i} + \Sigma_{j} x_{j}.C_{j} \otimes \Sigma_{i'}
  y_{i'}.B_{i'} + \Sigma_{j'} y_{j'}.D_{j'} 
  \and \\
  := \; \Sigma_{i} \Sigma_{i'} \quotep{\stackrel{\vee}{x_{i}}| \stackrel{\vee}{y_{i'}}}.(A_{i}\otimes B_{i'}) \; | \; \Sigma_{i'} \Sigma_{i} \quotep{\stackrel{\vee}{y_{i'}}|\stackrel{\vee}{x_{i}}}.(B_{i'}\otimes A_{i})
  \and
  \;\; | \;\; \Sigma_{j} \Sigma_{j'} \quotep{\stackrel{\vee}{x_{j}}|\stackrel{\vee}{y_{j'}}}.(A_{j}\otimes B_{j'}) \; | \; \Sigma_{j'} \Sigma_{j} \quotep{\stackrel{\vee}{y_{j'}}|\stackrel{\vee}{x_{j}}}.(B_{j'}\otimes A_{j})
\end{mathpar}

\begin{remark}
  Do we need to $x^{L}$ and $y^{R}$ for this construction as well?
\end{remark}

\paragraph{Tensor of parallel compositions} Next, we distribute tensor
over par.

\begin{mathpar}
  P_{1}|P_{2} \otimes Q_{1}|Q_{2} := (P_{1} \otimes Q_{1}) | (P_{1}
  \otimes Q_{2}) | (P_{2} \otimes Q_{1}) | (P_{2} \otimes Q_{2})
\end{mathpar}

\paragraph{Tensor with dropped names} We treat tensor of a
process with a dropped name as parallel composition.

\begin{mathpar}
  P \otimes \dropn{x} := P | \dropn{x}
\end{mathpar}

\paragraph{Tensor of agents}

Finally, we need to define tensor on agents. Note that the definition
of tensor on normal products only tensors inputs with inputs and
outputs with outputs. Thus, we only have to define the operation on
``homogeneous'' pairings.

\begin{mathpar}
  (\vec{x})P \otimes (\vec{y})Q
  \and \\
  := (x_{0}^{L}|y_{0}^{R},\ldots,x_{0}^{L}|y_{n}^{R},\ldots,x_{m}^{L}|y_{0}^{R},\ldots,x_{m}^{L}|y_{n}^R)(P\{ \vec{x}^{L}/\vec{x}\} \otimes Q \{ \vec{y}^{R}/\vec{y}\})
  \and \\
  \clift{\vec{P}} \otimes \clift{\vec{Q}}
  \and \\
  := \clift{P_{0}\otimes Q_{0},\ldots,P_{0}\otimes Q_{n},\ldots,P_{m}\otimes Q_{0},\ldots,P_{m}\otimes Q_{n}}
\end{mathpar}

\begin{remark}
  Observe that arities of tensored abstractions matches arities of
  tensored concretions if the original arities matched. Note also that
  the length of the arities corresponds to the increase in dimension
  we see in ordinary vector space tensor product.
\end{remark}

\begin{remark}
  Operationally, this definition distributes the tensor down to
  components ``linked'' by summation. Tensor over summation is
  intriguing in that it mixes names. Moreover, as a consequence of the
  way it mixes names we have the identities for all $x \in \QProc$ and
  $P,Q \in \Proc$

  \begin{mathpar}
    (x \cdot P) \otimes Q \equiv x \cdot (P \otimes Q) \equiv P \otimes (x \cdot Q)
    \and
    P \otimes \pzero \equiv P
  \end{mathpar}

  that the reader is invited to verify.
\end{remark}

\subsubsection{Annihilation}
\begin{mathpar}
  P^{\perp} := \{ Q | \forall R. P|Q \red^{*} R \Rightarrow R \red^{*} \pzero \}
  \and \\
  P^{\underline{\perp}} := \Sigma_{Q \in P^{\perp}} \quotep{Q}?(y).(\dropn{y}|Q) | \Sigma_{Q \in P^{\perp}} \quotep{Q}\clift{\Box}
\end{mathpar}

\paragraph{Discussion} The reader will note that $P^{\perp}$ is a
\emph{set} of processes, while $P^{\underline{\perp}}$ is a
\emph{context}. We call the set $P^{\perp}$ the \emph{annihilators} of
$P$. The parallel composition of a process in the annihilators of $P$
with $P$ will result in a process, the state space of which has all
paths eventually leading to $\pzero$. Execution may endure loops; but
under reasonable conditions of fairness (naturally guaranteed under
most notions of bisimulation) such a composite process cannot get
stuck in such a loop and will, eventually pop out and terminate.

The context $P^{\underline{\perp}}$ is ready and willing to ``take the
$P$ out of'' the process to which it is applied. It will effectively
transmit the code of the process to which it is applied to one of the
annihilators and run the process against it.

\subsubsection{Evaluation}
We fix $M$ a domain of fully abstract interpretation with an equality
coincident with bisimulation. We take $\meaningof{\cdot} : \Proc \to
M$ to be the map interpreting processes and $\nmeaningof{\cdot} : \M
\to Proc$ to be the map running the other way. Then we define

\begin{mathpar}
  \int P := \nmeaningof{\meaningof{P}}
\end{mathpar}

\paragraph{Discussion}
There are many fully abstract interpretations of Milner's
$\pi$-calculus. Any of them can be used as a basis for interpreting
the reflective calculus here. Equipped with such a domain it is
largely a matter of grinding through to check that the Yoneda
construction for the normalization-by-evaluation program can be
extended to this setting.

\begin{remark}
  The reader is invited to verify that $\int (P^{\underline{\perp}}[P]) = 0$.
\end{remark}

\subsection{Quantum mechanics}

Table \ref{tbl:core_qm_op_defns} gives the core operational definitions

\begin{table}[htp]\label{tbl:core_qm_op_defns}
  \center{
    \fbox{
      \begin{tabular}{c|c}
        quantum mechanics & process calculus \\
        \hline
        scalar & $x := \quotep{P}$ \\
        state vector & $\state{P} := P$ \\
        dual & $\state{P}^{*} := \event{P^{\underline{\perp}}} := \quotep{P^{\underline{\perp}}}[-]$ \\
        matrix & $ \Sigma_{\alpha} \state{P_{\alpha}}x_{\alpha}\event{Q_{\alpha}}$ \\
        vector addition & $\state{P} + \state{Q} := \state{P | Q}$ \\
        tensor product & $\state{P} \otimes \state{Q} := \state{P \otimes Q}$ \\
        inner product & $\innerprod{P}{Q} := \quotep{\int P^{\underline{\perp}}[Q]}$ \\
      \end{tabular}
    }
  }
  \caption{QM - operational definitions}
\end{table}

where

\begin{mathpar}
  \prmatrix{P}{Q} := \fprmatrix{P}{\quotep{\pzero}}{Q}
  \and
  \fprmatrix{P}{x}{Q} := (\state{P},x,\event{Q})
  \and
  (\fprmatrix{P}{x}{Q})(\state{R}) := x \cdot \innerprod{Q}{R} \cdot \state{P}
  \and
  (\fprmatrix{P}{x}{Q})(\event{R}) := x \cdot \innerprod{R}{P} \cdot \event{Q}
\end{mathpar}

\paragraph{Discussion}
As promised: vectors (aka states) are represented as processes; duals
as contextual duals; inner product definition should be compared with
standard inner product definition for ....

\begin{remark}
  Assuming $\int (P^{\underline{\perp}}[P]) = 0$, the reader is
  invited to verify that $(\fprmatrix{P}{x}{P})(\state{P}) = x \cdot \state{P}$.
\end{remark}

\begin{remark}
  The reader is invited to verify that $\innerprod{P}{Q}$ could
  equally well have been written $\quotep{\int \stackrel{\vee}{x}}$
  where $x = \event{P^{\underline{\perp}}}(Q)$.

  One of the motivations for this remark is that there is another way
  to factor these operations. We could package up evaluation in the dual:

  \begin{mathpar}
    \state{P}^{*} := \event{\int P^{\underline{\perp}}} := \quotep{\int P^{\underline{\perp}}}[-]
  \end{mathpar}

  and then have inner product defined by
  
  \begin{mathpar}
    \innerprod{P}{Q} := \event{P}(Q)
  \end{mathpar}

  Hopefully, experience with the calculations will provide guidance on
  the best factoring.
\end{remark}

\begin{remark}
  Assuming $\int (P^{\underline{\perp}}[P]) = 0$, the reader is
  invited to verify that $\forall P,Q. (\prmatrix{0}{Q})(\state{0}) =
  \state{0}$ and dually $(\prmatrix{P}{0})(\event{0}) = \event{0}$.
\end{remark}

\begin{remark}
  i'm a little worried that i don't (yet) have proper support for
  complex conjugacy. But, the observation above may give us a
  clue. According to Abramsky, it must be the case that the scalars
  are iso to the homset of the identity for the tensor -- which the
  observation above characterizes. 

  For now, we will simply bookmark the notion with $\overline{x}$.
\end{remark}

\subsubsection{Adjointness}

We need to give a definition of $(\cdot)^{\dagger}$ for matrices. The
obvious candidate definition is
\begin{mathpar}
(\Sigma_{\alpha}\fprmatrix{P_{\alpha}}{x_{\alpha}}{Q_{\alpha}})^{\dagger}
= \Sigma_{\alpha}\fprmatrix{(Q_{\alpha}^{\underline{\perp}})^{*}}{\overline{x}_{\alpha}}{P_{\alpha}^{\underline{\perp}}} 
\end{mathpar}

But, $(Q_{\alpha}^{\underline{\perp}})^{*}$ requires a name along
which to communicate the process to achieve the context application.

\subsubsection{Basis for a basis}
If processes label states and ``addition'' of states (a.k.a. vector
addition) is interpreted as parallel composition, what corresponds to
notions of linear independence and basis? Here, we recall that Yoshida
has developed a set of \emph{combinators} for an asynchronous verison
of Milner's $\pi$-calculus. These are a finite set of processes such
any process can be expressed as parallel composition of these
combinators together with liberal uses of the new operator and
replication. We can simply give a translation of these into the
present calculus and have reasonable expectation that the property
carries over. That is, that the resultant set allows to express all
processes via parallel composition. Note, however, that there is no
new operator or replication in this calculus. As a result, we expect
that the corresponding set is actually infinite. That is, we expect
that the space is actually infinite dimensional.

\begin{remark}
  The attentive reader may be a bit concerned. Certainly, the
  collection $S$, $K$ and $I$ is a finite set of
  combinators. Shouldn't we expect to see a finite set of combinators
  for an effectively equivalent system? i am very sympathetic to this
  critique and feel it warrants full attention. On the other hand, i
  also have in mind the following analogy. The natural numbers, as a
  monoid under addition, has exactly $1$ generator, while the natural
  numbers, as a monoid under multiplication, has countably many
  generators (the primes). We observe that the application of the
  lambda calculus is much less resource sensitive than the parallel
  composition of the $\pi$-calculus. Could it be the case that we have
  an analogy of the form
  
  \begin{mathpar}
    m + n : MN :: m*n : M|N
  \end{mathpar}

  giving a similar blow up in the set of ``primes''?  This is such a
  wonderful thought that, even if it's not true, i think it's worth
  writing down.
\end{remark}
 

\documentclass[12pt]{llncs}
%\documentclass{jktr}

\usepackage[pdftex]{hyperref}                   
\usepackage {listings}
\usepackage {mathpartir}
\usepackage{bcprules}
%\usepackage{listings}
                       
\usepackage{graphicx} 
%\usepackage[margins=2.5cm,nohead,nofoot]{geometry}
%\usepackage{geometry}
\usepackage{amsfonts}
\usepackage{amstext}
\usepackage{latexsym}
\usepackage{amssymb}
\usepackage{color}


%\include{myPreamble}
\include{qm2pi.local} 

%\ifpdf
%\usepackage[pdftex]{graphicx}
%\else
%\usepackage{graphicx}
%\fi

 % \ifpdf
%  \usepackage{pdfsync}
%  \if


%\title{Brief Article}
%\author{David F. Snyder}
%\author{L.G. Meredith}

%\address{Dept. of Math., Texas State University--San Marcos, San Marcos, TX 78666}
       
\pagestyle{empty}


\begin{document}

\lstset{language=[Objective]Caml,frame=shadowbox}

\input{qm2pi.front}

% section front matter (end)

\input{qm2pi.intro} 
 
% section introduction (end)

% \input{qm2pi.knotations} 

% section notation (end)

\input{qm2pi.process.calculi} 

% section concurrent_process_calculi_and_spatial_logics_ (end)
    
%\input{qm2pi.knots2pi} 

%\input{qm2pi.trefoil} 

%\input{qm2pi.mainthm} 

% subsection basic_interpretation (end)

%\input{qm2pi.rho.presentation} 
\subsection{The syntax and semantics of the notation system}\label{sub:the_syntax_and_semantics_of_the_notation_system} % (fold)

We now summarize a technical presentation of the calculus that
embodies our theory of dynamics. The typical presentation of such a
calculus follows the style of giving generators and relations on
them. The grammar, below, describing term constructors, freely
generates the set of processes, $\Proc$. This set is then quotiented
by a relation known as structural congruence and it is over this set
that the notion of dynamics is expressed. This presentation is
essentially that of \cite{MeredithR05} with the addition of
polyadicity and summation. For readability we have relegated some of
the technical subtleties to an appendix.

\subsubsection{Process grammar}\label{subsub:process_grammar}

\begin{mathpar}
  \inferrule* [lab=synchronization] {} {{M} \bc \pzero \;|\; x?F \;|\; x!C }
  \and
  \inferrule* [lab=abstraction] {} {{F} \bc (x)P}
  \and
  \inferrule* [lab=concretion] {} {{C} \bc \langle Q \rangle}
  \and
  \inferrule* [lab=process] {} {{P,Q} \bc M \;| \;P|Q \;|\; @{x}}
  \and
  \inferrule* [lab=name] {} {{x} \bc \quotep{P}}
\end{mathpar} 

Note that $\vec{x}$ (resp. $\vec{P}$) denotes a vector of names
(resp. processes) of length $|\vec{x}|$ (resp. $|\vec{P}|$). We adopt
the following useful abbreviations.

\begin{mathpar}
   x?(\vec{y}).P := x.(\vec{y})P \and  x\clift{\vec{P}} := x.\clift{\vec{P}}
   \and x!(y) := \lift{x}{\dropn{y}}
   \and \Pi_{i=0}^{n-1}P_i := P_0 | \ldots | P_{n-1}
\end{mathpar}

\subsubsection{Structural congruence}

\paragraph{Free and bound names and alpha-equivalence.} At the
core of structural equivalence is alpha-equivalence which identifies
process that are the same up to a change of variable. Formally, we
recognize the distinction between free and bound names. The free names
of a process, $\freenames{P}$, may be calculated recursively as
follows:

\begin{mathpar}
\freenames{\pzero} := \emptyset
  \and \\
  \freenames{x?(y).P} := \{ x \} \cup (\freenames{P} \setminus \{ y \})
  \and 
  \freenames{x!\langle P \rangle} := \{ x \} \cup \{ P \} 
  \and \\
  \freenames{P|Q} := \freenames{P} \cup \freenames{Q}
  \and \\
  \freenames{@{x}} := \{ x \}
\end{mathpar}

$\pi$
$\quotep{\pi}$

$\freenames{-} : \pi \to \mathcal{P}(\quotep{\pi})$

\begin{eqnarray*}
  \freenames{\pzero} & := & \emptyset \\
  \freenames{x?(y).P} & := & \{ x \} \cup (\freenames{P} \setminus \{ y \}) \\
  \freenames{x!\langle P \rangle} & := & \{ x \} \cup \{ P \} \\
  \freenames{P|Q} & := & \freenames{P} \cup \freenames{Q} \\
  \freenames{\dropn{x}} & := & \{ x \}
\end{eqnarray*}

The bound names of a process, $\boundnames{P}$, are those names occurring in $P$
that are not free. For example, in $x?(y).0$, the name $x$ is free, while $y$ is bound.

\begin{mathpar}
  \inferrule* [lab=monoidal-laws] {} { P|Q \equiv Q|P \and P|0 \equiv P \and P|(Q|R) \equiv (P|Q)|R }
\end{mathpar}

\begin{mathpar}
  \inferrule* [lab=alpha-equivalence] {} { (x)P \equiv (y)P\{y/x\} \and y \not\in \freenames{P} }
\end{mathpar}

\begin{definition}
Then two processes, $P,Q$, are alpha-equivalent if $P = Q\{\vec{y}/\vec{x}\}$ for
some $\vec{x} \in \boundnames{Q},\vec{y} \in \boundnames{P}$, where $Q\{\vec{y}/\vec{x}\}$
denotes the capture-avoiding substitution of $\vec{y}$ for $\vec{x}$ in $Q$.
\end{definition}

\begin{definition}
  The {\em structural congruence} \cite{SangiorgiWalker} , $\equiv$,
  between processes is the least congruence containing
  alpha-equivalence, satisfying the abelian monoid laws
  (associativity, commutativity and $\pzero$ as identity) for parallel
  composition $|$ and for summation $+$.
\end{definition}

\subsection{Name equivalence}

We take name equivalence, written $\nameeq$, to be the smallest
equivalence relation generated by the following rules.

\begin{mathpar}
\inferrule*[lab=Quote-drop]
{ }
{ \quotep{@{x}} \nameeq x }

\inferrule*[lab=Struct-equiv]
{ P \scong Q }
{ \quotep{P} \nameeq \quotep{Q} }
\end{mathpar}

The astute reader will have noticed that the mutual recursion of names
and processes imposes a mutual recursion on alpha-equivalence and
structural equivalence via name-equivalence. Fortunately, all of this
works out pleasantly and we may calculate in the natural way, free of
concern. The reader interested in the details is referred to the
appendix \ref{appendix:rho_details}.

\subsection{Substitution}

We use $\Proc$ for the set of processes, $\QProc$ for the set of
names, and $\id{\{}\vec{y} / \vec{x} \id{\}}$ to denote partial maps,
$s : \QProc \rightarrow \QProc$. A map, $s$ lifts, uniquely, to a map
on process terms, $\widehat{s} : \Proc \rightarrow \Proc$ by the
following equations.

\begin{mathpar}
  (0) \psubstp{Q}{P} := 0 \\
  (R \juxtap S) \psubstp{Q}{P}
  :=    
  (R)\psubstp{Q}{P} \juxtap (S) \psubstp{Q}{P} \\
  (x?(y).R) \psubstp{Q}{P}    
  :=    
  (x)\substp{Q}{P} (z)\concat( (R \psubstn{z}{y}) \psubstp{Q}{P} ) \\
  (\lift{x}{R}) \psubstp{Q}{P}  
  :=
  \lift{(x)\substp{Q}{P}}{ R \psubstp{Q}{P} } \\
%   (\dropn{x})  \psubstp{Q}{P}       
%   := 
%   \left\{ 
%     \begin{array}{ccc} 
%       \dropn{\quotep{Q}} & & x \nameeq \quotep{P} \\
%       \dropn{x} & & otherwise \\
%     \end{array}
%   \right. 
  (\dropn{x})  \psubstp{Q}{P}       
  := 
  \left\{ 
    \begin{array}{ccc} 
      Q & & x \nameeq \quotep{P} \\
      \dropn{x} & & otherwise \\
    \end{array}
  \right.
\end{mathpar}
 

where

\begin{eqnarray}
  (x)\id{\{} \lpquote Q \rpquote / \lpquote P \rpquote \id{\}}            = 
  \left\{ 
    \begin{array}{ccc}
      \lpquote Q \rpquote & & x \nameeq \lpquote P \rpquote \\
      x & & otherwise \\
    \end{array}
  \right. \nonumber
\end{eqnarray}

and $z$ is chosen distinct from $\quotep{P}$, $\quotep{Q}$, the free
names in $Q$, and all the names in $R$. Our $\alpha$-equivalence will
be built in the standard way from this substitution.

\begin{remark}\label{rem:no_self_referential_names}
  One consequence of these definitions is that $\forall P. \quotep{P}
  \not\in \freenames{P}$.
\end{remark}

\subsection{ Dynamic quote: an example }

Anticipating something of what's to come, consider applying the
substitution, $\widehat{\id{\{}u / z \id{\}}}$, to the following pair
of processes, $\lift{w}{y!(z)}$ and $w[ \lpquote y!(z) \rpquote ]$.

\begin{eqnarray}
	\lift{w}{y!(z)}\widehat{\id{\{}u / z \id{\}}}
		& = &
		\lift{w}{y!(u)} \nonumber\\
	w[ \lpquote y!(z) \rpquote ] \widehat{ \id{\{}u / z \id{\}} }
		& = &
		w[ \lpquote y!(z) \rpquote ] \nonumber
\end{eqnarray}

Because the body of the process between quotes is impervious to
substitution, we get radically different answers. In fact, by
examining the first process in an input context,
e.g. $x?(z).\lift{w}{y!(z)}$, we see that the process under the lift
operator may be shaped by prefixed inputs binding a name inside it. In
this sense, the lift operator will be seen as a way to dynamically
construct processes before reifying them as names.

Finally equipped with these standard features we can present the
dynamics of the calculus.

\subsubsection{Operational semantics} 

Finally, we introduce the computational dynamics. What marks these
algebras as distinct from other more traditionally studied algebraic
structures, e.g. vector spaces or polynomial rings, is the manner in
which dynamics is captured. In traditional structures, dynamics is typically
expressed through morphisms between such structures, as in linear maps
between vector spaces or morphisms between rings. In algebras
associated with the semantics of computation, the dynamics is
expressed as part of the algebraic structure itself, through a
reduction reduction relation typically denoted by $\red$. Below, we
give a recursive presentation of this relation for the calculus used
in the encoding.

$\red \subseteq \pi \times \pi$
$\red : \pi \to \mathcal{P}(\pi)$

\begin{mathpar}
  \inferrule* [lab=Comm] { \textsf{match}( x_{src}, x_{trgt} ) } { x_{trgt}?(y)P \; | \; x_{src}!\langle {Q} \rangle \red P\{\quotep{Q}/y}\} }
  \and \\
  \inferrule* [lab=Par] {{P} \red {P}'} {{{P} | {Q}} \red {{P}' | {Q}}}
  \and
  \inferrule* [lab=Equiv]{{{P} \scong {P}'} \andalso {{P}' \red {Q}'} \andalso {{Q}' \scong {Q}}}{{P} \red {Q}}
\end{mathpar}

\begin{eqnarray*}
  match_{\equiv} (\quotep{P},\quotep{Q}) & := & P \equiv Q \\
  match_{\dagger}(\quotep{P},\quotep{Q}) & := & \forall R. P|Q \red^{*} R => R \red^{*} 0 \\
  match_{K}(\quotep{P},\quotep{Q}) & := & K \mbox{ for some context } K
\end{eqnarray*}

$u?(x)P | u!\langle Q \rangle \red P\{\quotep{Q}/x\}$

%We write $\wred$ for $\red^*$, and $P\red$ if $\exists Q $ such that $ P \red Q$.
We write $P\red$ if $\exists Q $ such that $ P \red Q$ and $P\not\red$, otherwise.

\section{Replication}

As mentioned before, it is known that replication (and hence
recursion) can be implemented in a higher-order process algebra
\cite{SangiorgiWalker}. As our first example of calculation with the
machinery thus far presented we give the construction explicitly in
the {\rhoc}.

\begin{eqnarray}
	D_{x} & := & \prefix{x}{y}{(\binpar{\outputp{x}{y}}{@{y}})} \nonumber\\
	\bangp_{x}{P} & := & \binpar{{x}!\langle{\binpar{D_{x}}{P}}\rangle}{D_{x}} \nonumber
\end{eqnarray}

\begin{eqnarray}
	\bangp_{x}{P} & & \nonumber\\
	=
	& {x}!\langle{(\prefix{x}{y}{(\outputp{x}{y} | @{y})) | P}}\rangle 
	      | \prefix{x}{y}{(\outputp{x}{y} | @{y})} & \nonumber\\
	\red
	& (\outputp{x}{y} | @{y})\substn{\quotep{(\prefix{x}{y}{(@{y} | \outputp{x}{y})) | P}}}{y} & \nonumber\\
	=
	& \outputp{x}{\quotep{(\prefix{x}{y}{(\outputp{x}{y} | @{y})) | P}}}
	  | {(\prefix{x}{y}{(\outputp{x}{y} | @{y})) | P}} & \nonumber\\
	\red
	& \ldots & \nonumber\\
	\red^*
	& P | P | \ldots & \nonumber
\end{eqnarray}

Of course, this encoding, as an implementation, runs away, unfolding
$\bangp{P}$ eagerly. A lazier and more implementable replication
operator, restricted to input-guarded processes, may be obtained as follows.

\begin{eqnarray}
\bangp{\prefix{u}{v}{P}} 
	:= 
	\binpar{\lift{x}{\prefix{u}{v}{(\binpar{D(x)}{P})}}}{D(x)} \nonumber
\end{eqnarray}

\begin{remark}
  Note that the lazier definition still does not deal with summation
  or mixed summation (i.e. sums over input and output). The reader is
  invited to construct definitions of replication that deal with these
  features. 

  Further, the definitions are parameterized in a name, $x$. Can you,
  gentle reader, make a definition that eliminates this parameter and
  guarantees no accidental interaction between the replication
  machinery and the process being replicated -- i.e. no accidental
  sharing of names used by the process to get its work done and the
  name(s) used by the replication to effect copying. This latter
  revision of the definition of replication is crucial to obtaining
  the expected identity $!!P \sim !P$.
\end{remark}

\begin{remark}\label{rem:paradoxical_combinator}
  The reader familiar with the lambda calculus will have noticed the
  similarity between $D$ and the paradoxical combinator.

  [Ed. note: the existence of this seems to suggest we have to be more
  restrictive on the set of processes and names we admit if we are to
  support no-cloning.]
\end{remark}

\subsubsection{Bisimulation}

The computational dynamics gives rise to another kind of equivalence,
the equivalence of computational behavior. As previously mentioned
this is typically captured \emph{via} some form of bisimulation.

% The notion we use in this paper is weak barbed bisimulation
% \cite{milner91polyadicpi}.

The notion we use in this paper is derived from weak barbed
bisimulation \cite{milner91polyadicpi}. 

\begin{definition}
An \emph{observation relation}, $\downarrow_{\mathcal N}$, over a set
of names, $\mathcal N$, is the smallest relation satisfying the rules
below.

\infrule[Out-barb]{y \in {\mathcal N}, \; x \nameeq y}
		  {\outputp{x}{v} \downarrow_{\mathcal N} x}
\infrule[Par-barb]{\mbox{$P\downarrow_{\mathcal N} x$ or $Q\downarrow_{\mathcal N} x$}}
		  {\binpar{P}{Q} \downarrow_{\mathcal N} x}

We write $P \Downarrow_{\mathcal N} x$ if there is $Q$ such that 
$P \wred Q$ and $Q \downarrow_{\mathcal N} x$.
\end{definition}

\begin{definition}
%\label{def.bbisim}
An  ${\mathcal N}$-\emph{barbed bisimulation} over a set of names, ${\mathcal N}$, is a symmetric binary relation 
${\mathcal S}_{\mathcal N}$ between agents such that $P\rel{S}_{\mathcal N}Q$ implies:
\begin{enumerate}
\item If $P \red P'$ then $Q \wred Q'$ and $P'\rel{S}_{\mathcal N} Q'$.
\item If $P\downarrow_{\mathcal N} x$, then $Q\Downarrow_{\mathcal N} x$.
\end{enumerate}
$P$ is ${\mathcal N}$-barbed bisimilar to $Q$, written
$P \wbbisim_{\mathcal N} Q$, if $P \rel{S}_{\mathcal N} Q$ for some ${\mathcal N}$-barbed bisimulation ${\mathcal S}_{\mathcal N}$.
\end{definition}

$\mathcal{R} \subseteq \pi \times \pi$

$P \mathcal{R} Q => \forall P'. P \red P' \Rightarrow \exists Q'. Q \red Q', P' \mathcal{R} Q'$

$P \vdash x \Rightarrow Q \vdash x$

\begin{mathpar}
  \inferrule*[lab=Out-barb]{x \nameeq y}{{y}!\langle{Q}\rangle \vdash x}
  \and
  \inferrule*[lab=Par-barb]{\mbox{$P\vdash x$ or $Q\vdash x$}}{\binpar{P}{Q} \vdash x}
\end{mathpar}

\subsubsection{Contexts}

One of the principle advantages of computational calculi like the
$\pi$-calculus is a well-defined notion of context,
contextual-equivalence and a correlation between
contextual-equivalence and notions of bisimulation. The notion of
context allows the decomposition of a process into (sub-)process and
its syntactic environment, its context. Thus, a context may be
thought of as a process with a ``hole'' (written $\Box$) in it. The
application of a context $M$ to a process $P$, written $M[P]$, is
tantamount to filling the hole in $M$ with $P$. In this paper we do
not need the full weight of this theory, but do make use of the notion
of context in the proof the main theorem. 

\begin{mathpar}
  \inferrule* [lab=summation] {} {{M_{M},M_{N}} \bc \Box \;|\; x.M_{A} \;|\; M_{M}+M_{N}}
  \and
  \inferrule* [lab=agent] {} {{M_{A}} \bc (\vec{x})M_{P} \;| \; \clift{P_0,\ldots,M_{P},\ldots,P_N}}
  \and \\
  \inferrule* [lab=process] {} {{M_{P}} \bc M_{N} \;| \;P|M_{P} }
\end{mathpar} 

\begin{mathpar}
  \inferrule* [lab=sychronization] {} {M_{N} \bc \Box \;|\; x?M_{F} \;|\; x!M_{C}}
  \and
  \inferrule* [lab=abstraction] {} {{M_{F}} \bc (x)M_{P} }
  \and
  \inferrule* [lab=concretion] {} {{M_{C}} \bc \langle M_{P} \rangle }
  \and \\
  \inferrule* [lab=process] {} {{M_{P}} \bc M_{N} \;| \;P|M_{P} }
\end{mathpar}

\begin{definition}[contextual application] Given a context $M$, and
  process $P$, we define the \emph{contextual application}, $M[P] :=
  M\{P/\Box\}$. That is, the contextual application of M to P is the
  substitution of $P$ for $\Box$ in $M$.
\end{definition}

$\meaningof{-} : L \to \mathcal{P}(\pi)$

\begin{mathpar}
  \inferrule* [lab=collection] {} {\meaningof{true} = \pi, \and \meaningof{~E} = \pi \setminus \meaningof{E}, \and \meaningof{E_{1} \& E_{2}} = \meaningof{E_{1}} \cap \meaningof{E_{2}}}
\end{mathpar}

\begin{mathpar}
  \inferrule* [lab=structure] {} {\meaningof{0} = \{ P \in \pi | P \equiv 0 \}, \and \\ \meaningof{E_1 | E_2} = \{ P \in \pi | P \equiv P_{1} | P_{2}, P_{1} \in \meaningof{E_{1}}, P_{2} \in \meaningof{E_2}\} }
\end{mathpar}

\begin{mathpar}
 \inferrule* [lab=behavior] {} {\meaningof{\langle a?b \rangle E} = \{ P \in \pi | P \equiv Q | u?(y)P', \\ \and \\\\ \and \\ \;\;\; u \in \meaningof{a}, \forall z.P'\{z/y\} \in \meaningof{E\{z/b\}}\}, \and \\ \meaningof{a!E} = \{ P \in \pi | P \equiv Q | x!\langle P' \rangle, x \in \meaningof{a} P' \in \meaningof{E}\} }
\end{mathpar}

\begin{mathpar}
 \inferrule* [lab=nominal] {} {\meaningof{\quotep{E}} = \{ \quotep{P} \in \quotep{\pi} | P \in \meaningof{E} \}, \and \meaningof{\quotep{P}} = \{ \quotep{Q} \in \quotep{\pi} | P \equiv Q \} \and \\ \meaningof{@\quotep{E}} = \{ P \in \pi | P \equiv @x, x \in \meaningof{E} \}}
\end{mathpar}

\begin{eqnarray*}
  \\
  \meaningof{-} : TS \to ST
\end{eqnarray*}

\begin{eqnarray*}
  \\
  L : TS \to ST
\end{eqnarray*}

\begin{eqnarray*}
  \\
  P \models E \iff P \in \meaningof{E}
\end{eqnarray*}

\begin{eqnarray*}
  P \approx_{L} Q \iff \forall E \in L. P \models E \iff Q \models E
\end{eqnarray*}

\begin{eqnarray*}
  P \approx_{K} Q
\end{eqnarray*}

\begin{eqnarray*}
  P \approx Q
\end{eqnarray*}

$\approx_{K} = \approx = \approx_{L}$

\subsubsection{Contextual duality}

Note that contexts extend the quotation operation to a family of
operations from processes to names. Given a context, $M$, we can
define a \emph{nominal context}, $\quotep{M}$ by $\quotep{M}[P] :=
\quotep{M[P]}$. To foreshadow what is to come we observe that these
operations enjoy a duality with processes very much like the duality
between vectors and maps from vectors to scalars.

Further, because the calculus is essentially higher-order, we have a
correspondence between contexts and processes. More specifically,
given a name $x$ and a context $M$ we can construct $M^{*}_{x}$ such
that 

\begin{mathpar}
  M^{*}_{x} | \lift{x}{P} \red M[P]
\end{mathpar}

namely,

\begin{mathpar}
  M^{*}_{x} := x?(u).M[\dropn{u}]
\end{mathpar}

The dependence of $M^{*}_{x}$ on a name makes it an abstraction, 

\begin{mathpar}
  M^{*} := (x)x?(u).M[\dropn{u}]
\end{mathpar}

\subsection{Additional notation}

It will sometimes be convenient to denote the process a name
quotes. We already have the notation $x = \quotep{P}$, but it will be
convenient to introduce an alternate notation, $\procn{x}$, when we
want to emphasize the connection to the use of the name. Note that, by
virtue of name equivalence, $\quotep{\procn{x}} \nameeq x$; so, the
notation is consistent with previous definitions.

Further, because names have structure it is possible to effect
substitutions on the basis of that structure. This means we need to
upgrade our notation for substitutions, which we accomplish by
adapting comprehension notation. Thus,

\begin{mathpar}
  P\{ y / x : x \in S \}
\end{mathpar}

is interpreted to mean the process derived from P by replacing (in a
capture-avoiding manner) each occurrence of $x$ in $S$ by $y$. For example,

\begin{mathpar}
  P\{ \quotep{\procn{x}|\procn{x}} / x : x \in \freenames{P} \}
\end{mathpar}

will replace each (occurrence) of a free name $x$ in $P$ by
$\quotep{\procn{x}|\procn{x}}$.

Also, we will avail ourselves of the notation $x^{L}$ and $x^{R}$ to
denote injections of a name into disjoint copies of the name
space. There are numerous ways to accomplish this. One example can be
found in \cite{MeredithR05}. This notation overloads to vectors of
names: $\vec{x}^{\pi} := (x_{i}^{\pi} \; : \; 0 \leq i < |\vec{x}| )$ where $\pi \in \{L,R\}$.

We also use $P^{\Box} := P|\Box$.

In \cite{MeredithR05} an interpretation of the new operator is
given. It turns out that there are several possible interpretations
all enjoying the requisite algebraic properties of the operator (see
\cite{milner91polyadicpi}). We will therefore make liberal use of
$(\nu\; \vec{x})P$.

% subsection the_syntax_and_semantics_of_the_notation_system (end)   

\input{qm2pi.qmops} 

\input{qm2pi.sterngerlach} 

\input{qm2pi.metric} 

% section concurrent_process_calculi (end)

%\input{qm2pi.proofsketch}

% section proof sketch (end)

%\input{qm2pi.slviaknots} 

% section spatial logic via knots (end)

\input{qm2pi.conclusion}

% section conclusion (end)

%\input{qm2pi.dtcodes} 

% section wiring algorithm (end)

\input{qm2pi.ack} 

% section acknowledgments (end)

\newpage


\bibliographystyle{plain}   
\bibliography{../../biblios/main.bib}

\input{qm2pi.rhodetails}

\end{document}

 

\documentclass[12pt]{llncs}
%\documentclass{jktr}

\usepackage[pdftex]{hyperref}                   
\usepackage {listings}
\usepackage {mathpartir}
\usepackage{bcprules}
%\usepackage{listings}
                       
\usepackage{graphicx} 
%\usepackage[margins=2.5cm,nohead,nofoot]{geometry}
%\usepackage{geometry}
\usepackage{amsfonts}
\usepackage{amstext}
\usepackage{latexsym}
\usepackage{amssymb}
\usepackage{color}


%\include{myPreamble}
\include{qm2pi.local} 

%\ifpdf
%\usepackage[pdftex]{graphicx}
%\else
%\usepackage{graphicx}
%\fi

 % \ifpdf
%  \usepackage{pdfsync}
%  \if


%\title{Brief Article}
%\author{David F. Snyder}
%\author{L.G. Meredith}

%\address{Dept. of Math., Texas State University--San Marcos, San Marcos, TX 78666}
       
\pagestyle{empty}


\begin{document}

\lstset{language=[Objective]Caml,frame=shadowbox}

\input{qm2pi.front}

% section front matter (end)

\input{qm2pi.intro} 
 
% section introduction (end)

% \input{qm2pi.knotations} 

% section notation (end)

\input{qm2pi.process.calculi} 

% section concurrent_process_calculi_and_spatial_logics_ (end)
    
%\input{qm2pi.knots2pi} 

%\input{qm2pi.trefoil} 

%\input{qm2pi.mainthm} 

% subsection basic_interpretation (end)

%\input{qm2pi.rho.presentation} 
\subsection{The syntax and semantics of the notation system}\label{sub:the_syntax_and_semantics_of_the_notation_system} % (fold)

We now summarize a technical presentation of the calculus that
embodies our theory of dynamics. The typical presentation of such a
calculus follows the style of giving generators and relations on
them. The grammar, below, describing term constructors, freely
generates the set of processes, $\Proc$. This set is then quotiented
by a relation known as structural congruence and it is over this set
that the notion of dynamics is expressed. This presentation is
essentially that of \cite{MeredithR05} with the addition of
polyadicity and summation. For readability we have relegated some of
the technical subtleties to an appendix.

\subsubsection{Process grammar}\label{subsub:process_grammar}

\begin{mathpar}
  \inferrule* [lab=synchronization] {} {{M} \bc \pzero \;|\; x?F \;|\; x!C }
  \and
  \inferrule* [lab=abstraction] {} {{F} \bc (x)P}
  \and
  \inferrule* [lab=concretion] {} {{C} \bc \langle Q \rangle}
  \and
  \inferrule* [lab=process] {} {{P,Q} \bc M \;| \;P|Q \;|\; @{x}}
  \and
  \inferrule* [lab=name] {} {{x} \bc \quotep{P}}
\end{mathpar} 

Note that $\vec{x}$ (resp. $\vec{P}$) denotes a vector of names
(resp. processes) of length $|\vec{x}|$ (resp. $|\vec{P}|$). We adopt
the following useful abbreviations.

\begin{mathpar}
   x?(\vec{y}).P := x.(\vec{y})P \and  x\clift{\vec{P}} := x.\clift{\vec{P}}
   \and x!(y) := \lift{x}{\dropn{y}}
   \and \Pi_{i=0}^{n-1}P_i := P_0 | \ldots | P_{n-1}
\end{mathpar}

\subsubsection{Structural congruence}

\paragraph{Free and bound names and alpha-equivalence.} At the
core of structural equivalence is alpha-equivalence which identifies
process that are the same up to a change of variable. Formally, we
recognize the distinction between free and bound names. The free names
of a process, $\freenames{P}$, may be calculated recursively as
follows:

\begin{mathpar}
\freenames{\pzero} := \emptyset
  \and \\
  \freenames{x?(y).P} := \{ x \} \cup (\freenames{P} \setminus \{ y \})
  \and 
  \freenames{x!\langle P \rangle} := \{ x \} \cup \{ P \} 
  \and \\
  \freenames{P|Q} := \freenames{P} \cup \freenames{Q}
  \and \\
  \freenames{@{x}} := \{ x \}
\end{mathpar}

$\pi$
$\quotep{\pi}$

$\freenames{-} : \pi \to \mathcal{P}(\quotep{\pi})$

\begin{eqnarray*}
  \freenames{\pzero} & := & \emptyset \\
  \freenames{x?(y).P} & := & \{ x \} \cup (\freenames{P} \setminus \{ y \}) \\
  \freenames{x!\langle P \rangle} & := & \{ x \} \cup \{ P \} \\
  \freenames{P|Q} & := & \freenames{P} \cup \freenames{Q} \\
  \freenames{\dropn{x}} & := & \{ x \}
\end{eqnarray*}

The bound names of a process, $\boundnames{P}$, are those names occurring in $P$
that are not free. For example, in $x?(y).0$, the name $x$ is free, while $y$ is bound.

\begin{mathpar}
  \inferrule* [lab=monoidal-laws] {} { P|Q \equiv Q|P \and P|0 \equiv P \and P|(Q|R) \equiv (P|Q)|R }
\end{mathpar}

\begin{mathpar}
  \inferrule* [lab=alpha-equivalence] {} { (x)P \equiv (y)P\{y/x\} \and y \not\in \freenames{P} }
\end{mathpar}

\begin{definition}
Then two processes, $P,Q$, are alpha-equivalent if $P = Q\{\vec{y}/\vec{x}\}$ for
some $\vec{x} \in \boundnames{Q},\vec{y} \in \boundnames{P}$, where $Q\{\vec{y}/\vec{x}\}$
denotes the capture-avoiding substitution of $\vec{y}$ for $\vec{x}$ in $Q$.
\end{definition}

\begin{definition}
  The {\em structural congruence} \cite{SangiorgiWalker} , $\equiv$,
  between processes is the least congruence containing
  alpha-equivalence, satisfying the abelian monoid laws
  (associativity, commutativity and $\pzero$ as identity) for parallel
  composition $|$ and for summation $+$.
\end{definition}

\subsection{Name equivalence}

We take name equivalence, written $\nameeq$, to be the smallest
equivalence relation generated by the following rules.

\begin{mathpar}
\inferrule*[lab=Quote-drop]
{ }
{ \quotep{@{x}} \nameeq x }

\inferrule*[lab=Struct-equiv]
{ P \scong Q }
{ \quotep{P} \nameeq \quotep{Q} }
\end{mathpar}

The astute reader will have noticed that the mutual recursion of names
and processes imposes a mutual recursion on alpha-equivalence and
structural equivalence via name-equivalence. Fortunately, all of this
works out pleasantly and we may calculate in the natural way, free of
concern. The reader interested in the details is referred to the
appendix \ref{appendix:rho_details}.

\subsection{Substitution}

We use $\Proc$ for the set of processes, $\QProc$ for the set of
names, and $\id{\{}\vec{y} / \vec{x} \id{\}}$ to denote partial maps,
$s : \QProc \rightarrow \QProc$. A map, $s$ lifts, uniquely, to a map
on process terms, $\widehat{s} : \Proc \rightarrow \Proc$ by the
following equations.

\begin{mathpar}
  (0) \psubstp{Q}{P} := 0 \\
  (R \juxtap S) \psubstp{Q}{P}
  :=    
  (R)\psubstp{Q}{P} \juxtap (S) \psubstp{Q}{P} \\
  (x?(y).R) \psubstp{Q}{P}    
  :=    
  (x)\substp{Q}{P} (z)\concat( (R \psubstn{z}{y}) \psubstp{Q}{P} ) \\
  (\lift{x}{R}) \psubstp{Q}{P}  
  :=
  \lift{(x)\substp{Q}{P}}{ R \psubstp{Q}{P} } \\
%   (\dropn{x})  \psubstp{Q}{P}       
%   := 
%   \left\{ 
%     \begin{array}{ccc} 
%       \dropn{\quotep{Q}} & & x \nameeq \quotep{P} \\
%       \dropn{x} & & otherwise \\
%     \end{array}
%   \right. 
  (\dropn{x})  \psubstp{Q}{P}       
  := 
  \left\{ 
    \begin{array}{ccc} 
      Q & & x \nameeq \quotep{P} \\
      \dropn{x} & & otherwise \\
    \end{array}
  \right.
\end{mathpar}
 

where

\begin{eqnarray}
  (x)\id{\{} \lpquote Q \rpquote / \lpquote P \rpquote \id{\}}            = 
  \left\{ 
    \begin{array}{ccc}
      \lpquote Q \rpquote & & x \nameeq \lpquote P \rpquote \\
      x & & otherwise \\
    \end{array}
  \right. \nonumber
\end{eqnarray}

and $z$ is chosen distinct from $\quotep{P}$, $\quotep{Q}$, the free
names in $Q$, and all the names in $R$. Our $\alpha$-equivalence will
be built in the standard way from this substitution.

\begin{remark}\label{rem:no_self_referential_names}
  One consequence of these definitions is that $\forall P. \quotep{P}
  \not\in \freenames{P}$.
\end{remark}

\subsection{ Dynamic quote: an example }

Anticipating something of what's to come, consider applying the
substitution, $\widehat{\id{\{}u / z \id{\}}}$, to the following pair
of processes, $\lift{w}{y!(z)}$ and $w[ \lpquote y!(z) \rpquote ]$.

\begin{eqnarray}
	\lift{w}{y!(z)}\widehat{\id{\{}u / z \id{\}}}
		& = &
		\lift{w}{y!(u)} \nonumber\\
	w[ \lpquote y!(z) \rpquote ] \widehat{ \id{\{}u / z \id{\}} }
		& = &
		w[ \lpquote y!(z) \rpquote ] \nonumber
\end{eqnarray}

Because the body of the process between quotes is impervious to
substitution, we get radically different answers. In fact, by
examining the first process in an input context,
e.g. $x?(z).\lift{w}{y!(z)}$, we see that the process under the lift
operator may be shaped by prefixed inputs binding a name inside it. In
this sense, the lift operator will be seen as a way to dynamically
construct processes before reifying them as names.

Finally equipped with these standard features we can present the
dynamics of the calculus.

\subsubsection{Operational semantics} 

Finally, we introduce the computational dynamics. What marks these
algebras as distinct from other more traditionally studied algebraic
structures, e.g. vector spaces or polynomial rings, is the manner in
which dynamics is captured. In traditional structures, dynamics is typically
expressed through morphisms between such structures, as in linear maps
between vector spaces or morphisms between rings. In algebras
associated with the semantics of computation, the dynamics is
expressed as part of the algebraic structure itself, through a
reduction reduction relation typically denoted by $\red$. Below, we
give a recursive presentation of this relation for the calculus used
in the encoding.

$\red \subseteq \pi \times \pi$
$\red : \pi \to \mathcal{P}(\pi)$

\begin{mathpar}
  \inferrule* [lab=Comm] { \textsf{match}( x_{src}, x_{trgt} ) } { x_{trgt}?(y)P \; | \; x_{src}!\langle {Q} \rangle \red P\{\quotep{Q}/y}\} }
  \and \\
  \inferrule* [lab=Par] {{P} \red {P}'} {{{P} | {Q}} \red {{P}' | {Q}}}
  \and
  \inferrule* [lab=Equiv]{{{P} \scong {P}'} \andalso {{P}' \red {Q}'} \andalso {{Q}' \scong {Q}}}{{P} \red {Q}}
\end{mathpar}

\begin{eqnarray*}
  match_{\equiv} (\quotep{P},\quotep{Q}) & := & P \equiv Q \\
  match_{\dagger}(\quotep{P},\quotep{Q}) & := & \forall R. P|Q \red^{*} R => R \red^{*} 0 \\
  match_{K}(\quotep{P},\quotep{Q}) & := & K \mbox{ for some context } K
\end{eqnarray*}

$u?(x)P | u!\langle Q \rangle \red P\{\quotep{Q}/x\}$

%We write $\wred$ for $\red^*$, and $P\red$ if $\exists Q $ such that $ P \red Q$.
We write $P\red$ if $\exists Q $ such that $ P \red Q$ and $P\not\red$, otherwise.

\section{Replication}

As mentioned before, it is known that replication (and hence
recursion) can be implemented in a higher-order process algebra
\cite{SangiorgiWalker}. As our first example of calculation with the
machinery thus far presented we give the construction explicitly in
the {\rhoc}.

\begin{eqnarray}
	D_{x} & := & \prefix{x}{y}{(\binpar{\outputp{x}{y}}{@{y}})} \nonumber\\
	\bangp_{x}{P} & := & \binpar{{x}!\langle{\binpar{D_{x}}{P}}\rangle}{D_{x}} \nonumber
\end{eqnarray}

\begin{eqnarray}
	\bangp_{x}{P} & & \nonumber\\
	=
	& {x}!\langle{(\prefix{x}{y}{(\outputp{x}{y} | @{y})) | P}}\rangle 
	      | \prefix{x}{y}{(\outputp{x}{y} | @{y})} & \nonumber\\
	\red
	& (\outputp{x}{y} | @{y})\substn{\quotep{(\prefix{x}{y}{(@{y} | \outputp{x}{y})) | P}}}{y} & \nonumber\\
	=
	& \outputp{x}{\quotep{(\prefix{x}{y}{(\outputp{x}{y} | @{y})) | P}}}
	  | {(\prefix{x}{y}{(\outputp{x}{y} | @{y})) | P}} & \nonumber\\
	\red
	& \ldots & \nonumber\\
	\red^*
	& P | P | \ldots & \nonumber
\end{eqnarray}

Of course, this encoding, as an implementation, runs away, unfolding
$\bangp{P}$ eagerly. A lazier and more implementable replication
operator, restricted to input-guarded processes, may be obtained as follows.

\begin{eqnarray}
\bangp{\prefix{u}{v}{P}} 
	:= 
	\binpar{\lift{x}{\prefix{u}{v}{(\binpar{D(x)}{P})}}}{D(x)} \nonumber
\end{eqnarray}

\begin{remark}
  Note that the lazier definition still does not deal with summation
  or mixed summation (i.e. sums over input and output). The reader is
  invited to construct definitions of replication that deal with these
  features. 

  Further, the definitions are parameterized in a name, $x$. Can you,
  gentle reader, make a definition that eliminates this parameter and
  guarantees no accidental interaction between the replication
  machinery and the process being replicated -- i.e. no accidental
  sharing of names used by the process to get its work done and the
  name(s) used by the replication to effect copying. This latter
  revision of the definition of replication is crucial to obtaining
  the expected identity $!!P \sim !P$.
\end{remark}

\begin{remark}\label{rem:paradoxical_combinator}
  The reader familiar with the lambda calculus will have noticed the
  similarity between $D$ and the paradoxical combinator.

  [Ed. note: the existence of this seems to suggest we have to be more
  restrictive on the set of processes and names we admit if we are to
  support no-cloning.]
\end{remark}

\subsubsection{Bisimulation}

The computational dynamics gives rise to another kind of equivalence,
the equivalence of computational behavior. As previously mentioned
this is typically captured \emph{via} some form of bisimulation.

% The notion we use in this paper is weak barbed bisimulation
% \cite{milner91polyadicpi}.

The notion we use in this paper is derived from weak barbed
bisimulation \cite{milner91polyadicpi}. 

\begin{definition}
An \emph{observation relation}, $\downarrow_{\mathcal N}$, over a set
of names, $\mathcal N$, is the smallest relation satisfying the rules
below.

\infrule[Out-barb]{y \in {\mathcal N}, \; x \nameeq y}
		  {\outputp{x}{v} \downarrow_{\mathcal N} x}
\infrule[Par-barb]{\mbox{$P\downarrow_{\mathcal N} x$ or $Q\downarrow_{\mathcal N} x$}}
		  {\binpar{P}{Q} \downarrow_{\mathcal N} x}

We write $P \Downarrow_{\mathcal N} x$ if there is $Q$ such that 
$P \wred Q$ and $Q \downarrow_{\mathcal N} x$.
\end{definition}

\begin{definition}
%\label{def.bbisim}
An  ${\mathcal N}$-\emph{barbed bisimulation} over a set of names, ${\mathcal N}$, is a symmetric binary relation 
${\mathcal S}_{\mathcal N}$ between agents such that $P\rel{S}_{\mathcal N}Q$ implies:
\begin{enumerate}
\item If $P \red P'$ then $Q \wred Q'$ and $P'\rel{S}_{\mathcal N} Q'$.
\item If $P\downarrow_{\mathcal N} x$, then $Q\Downarrow_{\mathcal N} x$.
\end{enumerate}
$P$ is ${\mathcal N}$-barbed bisimilar to $Q$, written
$P \wbbisim_{\mathcal N} Q$, if $P \rel{S}_{\mathcal N} Q$ for some ${\mathcal N}$-barbed bisimulation ${\mathcal S}_{\mathcal N}$.
\end{definition}

$\mathcal{R} \subseteq \pi \times \pi$

$P \mathcal{R} Q => \forall P'. P \red P' \Rightarrow \exists Q'. Q \red Q', P' \mathcal{R} Q'$

$P \vdash x \Rightarrow Q \vdash x$

\begin{mathpar}
  \inferrule*[lab=Out-barb]{x \nameeq y}{{y}!\langle{Q}\rangle \vdash x}
  \and
  \inferrule*[lab=Par-barb]{\mbox{$P\vdash x$ or $Q\vdash x$}}{\binpar{P}{Q} \vdash x}
\end{mathpar}

\subsubsection{Contexts}

One of the principle advantages of computational calculi like the
$\pi$-calculus is a well-defined notion of context,
contextual-equivalence and a correlation between
contextual-equivalence and notions of bisimulation. The notion of
context allows the decomposition of a process into (sub-)process and
its syntactic environment, its context. Thus, a context may be
thought of as a process with a ``hole'' (written $\Box$) in it. The
application of a context $M$ to a process $P$, written $M[P]$, is
tantamount to filling the hole in $M$ with $P$. In this paper we do
not need the full weight of this theory, but do make use of the notion
of context in the proof the main theorem. 

\begin{mathpar}
  \inferrule* [lab=summation] {} {{M_{M},M_{N}} \bc \Box \;|\; x.M_{A} \;|\; M_{M}+M_{N}}
  \and
  \inferrule* [lab=agent] {} {{M_{A}} \bc (\vec{x})M_{P} \;| \; \clift{P_0,\ldots,M_{P},\ldots,P_N}}
  \and \\
  \inferrule* [lab=process] {} {{M_{P}} \bc M_{N} \;| \;P|M_{P} }
\end{mathpar} 

\begin{mathpar}
  \inferrule* [lab=sychronization] {} {M_{N} \bc \Box \;|\; x?M_{F} \;|\; x!M_{C}}
  \and
  \inferrule* [lab=abstraction] {} {{M_{F}} \bc (x)M_{P} }
  \and
  \inferrule* [lab=concretion] {} {{M_{C}} \bc \langle M_{P} \rangle }
  \and \\
  \inferrule* [lab=process] {} {{M_{P}} \bc M_{N} \;| \;P|M_{P} }
\end{mathpar}

\begin{definition}[contextual application] Given a context $M$, and
  process $P$, we define the \emph{contextual application}, $M[P] :=
  M\{P/\Box\}$. That is, the contextual application of M to P is the
  substitution of $P$ for $\Box$ in $M$.
\end{definition}

$\meaningof{-} : L \to \mathcal{P}(\pi)$

\begin{mathpar}
  \inferrule* [lab=collection] {} {\meaningof{true} = \pi, \and \meaningof{~E} = \pi \setminus \meaningof{E}, \and \meaningof{E_{1} \& E_{2}} = \meaningof{E_{1}} \cap \meaningof{E_{2}}}
\end{mathpar}

\begin{mathpar}
  \inferrule* [lab=structure] {} {\meaningof{0} = \{ P \in \pi | P \equiv 0 \}, \and \\ \meaningof{E_1 | E_2} = \{ P \in \pi | P \equiv P_{1} | P_{2}, P_{1} \in \meaningof{E_{1}}, P_{2} \in \meaningof{E_2}\} }
\end{mathpar}

\begin{mathpar}
 \inferrule* [lab=behavior] {} {\meaningof{\langle a?b \rangle E} = \{ P \in \pi | P \equiv Q | u?(y)P', \\ \and \\\\ \and \\ \;\;\; u \in \meaningof{a}, \forall z.P'\{z/y\} \in \meaningof{E\{z/b\}}\}, \and \\ \meaningof{a!E} = \{ P \in \pi | P \equiv Q | x!\langle P' \rangle, x \in \meaningof{a} P' \in \meaningof{E}\} }
\end{mathpar}

\begin{mathpar}
 \inferrule* [lab=nominal] {} {\meaningof{\quotep{E}} = \{ \quotep{P} \in \quotep{\pi} | P \in \meaningof{E} \}, \and \meaningof{\quotep{P}} = \{ \quotep{Q} \in \quotep{\pi} | P \equiv Q \} \and \\ \meaningof{@\quotep{E}} = \{ P \in \pi | P \equiv @x, x \in \meaningof{E} \}}
\end{mathpar}

\begin{eqnarray*}
  \\
  \meaningof{-} : TS \to ST
\end{eqnarray*}

\begin{eqnarray*}
  \\
  L : TS \to ST
\end{eqnarray*}

\begin{eqnarray*}
  \\
  P \models E \iff P \in \meaningof{E}
\end{eqnarray*}

\begin{eqnarray*}
  P \approx_{L} Q \iff \forall E \in L. P \models E \iff Q \models E
\end{eqnarray*}

\begin{eqnarray*}
  P \approx_{K} Q
\end{eqnarray*}

\begin{eqnarray*}
  P \approx Q
\end{eqnarray*}

$\approx_{K} = \approx = \approx_{L}$

\subsubsection{Contextual duality}

Note that contexts extend the quotation operation to a family of
operations from processes to names. Given a context, $M$, we can
define a \emph{nominal context}, $\quotep{M}$ by $\quotep{M}[P] :=
\quotep{M[P]}$. To foreshadow what is to come we observe that these
operations enjoy a duality with processes very much like the duality
between vectors and maps from vectors to scalars.

Further, because the calculus is essentially higher-order, we have a
correspondence between contexts and processes. More specifically,
given a name $x$ and a context $M$ we can construct $M^{*}_{x}$ such
that 

\begin{mathpar}
  M^{*}_{x} | \lift{x}{P} \red M[P]
\end{mathpar}

namely,

\begin{mathpar}
  M^{*}_{x} := x?(u).M[\dropn{u}]
\end{mathpar}

The dependence of $M^{*}_{x}$ on a name makes it an abstraction, 

\begin{mathpar}
  M^{*} := (x)x?(u).M[\dropn{u}]
\end{mathpar}

\subsection{Additional notation}

It will sometimes be convenient to denote the process a name
quotes. We already have the notation $x = \quotep{P}$, but it will be
convenient to introduce an alternate notation, $\procn{x}$, when we
want to emphasize the connection to the use of the name. Note that, by
virtue of name equivalence, $\quotep{\procn{x}} \nameeq x$; so, the
notation is consistent with previous definitions.

Further, because names have structure it is possible to effect
substitutions on the basis of that structure. This means we need to
upgrade our notation for substitutions, which we accomplish by
adapting comprehension notation. Thus,

\begin{mathpar}
  P\{ y / x : x \in S \}
\end{mathpar}

is interpreted to mean the process derived from P by replacing (in a
capture-avoiding manner) each occurrence of $x$ in $S$ by $y$. For example,

\begin{mathpar}
  P\{ \quotep{\procn{x}|\procn{x}} / x : x \in \freenames{P} \}
\end{mathpar}

will replace each (occurrence) of a free name $x$ in $P$ by
$\quotep{\procn{x}|\procn{x}}$.

Also, we will avail ourselves of the notation $x^{L}$ and $x^{R}$ to
denote injections of a name into disjoint copies of the name
space. There are numerous ways to accomplish this. One example can be
found in \cite{MeredithR05}. This notation overloads to vectors of
names: $\vec{x}^{\pi} := (x_{i}^{\pi} \; : \; 0 \leq i < |\vec{x}| )$ where $\pi \in \{L,R\}$.

We also use $P^{\Box} := P|\Box$.

In \cite{MeredithR05} an interpretation of the new operator is
given. It turns out that there are several possible interpretations
all enjoying the requisite algebraic properties of the operator (see
\cite{milner91polyadicpi}). We will therefore make liberal use of
$(\nu\; \vec{x})P$.

% subsection the_syntax_and_semantics_of_the_notation_system (end)   

\input{qm2pi.qmops} 

\input{qm2pi.sterngerlach} 

\input{qm2pi.metric} 

% section concurrent_process_calculi (end)

%\input{qm2pi.proofsketch}

% section proof sketch (end)

%\input{qm2pi.slviaknots} 

% section spatial logic via knots (end)

\input{qm2pi.conclusion}

% section conclusion (end)

%\input{qm2pi.dtcodes} 

% section wiring algorithm (end)

\input{qm2pi.ack} 

% section acknowledgments (end)

\newpage


\bibliographystyle{plain}   
\bibliography{../../biblios/main.bib}

\input{qm2pi.rhodetails}

\end{document}

 

% section concurrent_process_calculi (end)

%\documentclass[12pt]{llncs}
%\documentclass{jktr}

\usepackage[pdftex]{hyperref}                   
\usepackage {listings}
\usepackage {mathpartir}
\usepackage{bcprules}
%\usepackage{listings}
                       
\usepackage{graphicx} 
%\usepackage[margins=2.5cm,nohead,nofoot]{geometry}
%\usepackage{geometry}
\usepackage{amsfonts}
\usepackage{amstext}
\usepackage{latexsym}
\usepackage{amssymb}
\usepackage{color}


%\include{myPreamble}
\include{qm2pi.local} 

%\ifpdf
%\usepackage[pdftex]{graphicx}
%\else
%\usepackage{graphicx}
%\fi

 % \ifpdf
%  \usepackage{pdfsync}
%  \if


%\title{Brief Article}
%\author{David F. Snyder}
%\author{L.G. Meredith}

%\address{Dept. of Math., Texas State University--San Marcos, San Marcos, TX 78666}
       
\pagestyle{empty}


\begin{document}

\lstset{language=[Objective]Caml,frame=shadowbox}

\input{qm2pi.front}

% section front matter (end)

\input{qm2pi.intro} 
 
% section introduction (end)

% \input{qm2pi.knotations} 

% section notation (end)

\input{qm2pi.process.calculi} 

% section concurrent_process_calculi_and_spatial_logics_ (end)
    
%\input{qm2pi.knots2pi} 

%\input{qm2pi.trefoil} 

%\input{qm2pi.mainthm} 

% subsection basic_interpretation (end)

%\input{qm2pi.rho.presentation} 
\subsection{The syntax and semantics of the notation system}\label{sub:the_syntax_and_semantics_of_the_notation_system} % (fold)

We now summarize a technical presentation of the calculus that
embodies our theory of dynamics. The typical presentation of such a
calculus follows the style of giving generators and relations on
them. The grammar, below, describing term constructors, freely
generates the set of processes, $\Proc$. This set is then quotiented
by a relation known as structural congruence and it is over this set
that the notion of dynamics is expressed. This presentation is
essentially that of \cite{MeredithR05} with the addition of
polyadicity and summation. For readability we have relegated some of
the technical subtleties to an appendix.

\subsubsection{Process grammar}\label{subsub:process_grammar}

\begin{mathpar}
  \inferrule* [lab=synchronization] {} {{M} \bc \pzero \;|\; x?F \;|\; x!C }
  \and
  \inferrule* [lab=abstraction] {} {{F} \bc (x)P}
  \and
  \inferrule* [lab=concretion] {} {{C} \bc \langle Q \rangle}
  \and
  \inferrule* [lab=process] {} {{P,Q} \bc M \;| \;P|Q \;|\; @{x}}
  \and
  \inferrule* [lab=name] {} {{x} \bc \quotep{P}}
\end{mathpar} 

Note that $\vec{x}$ (resp. $\vec{P}$) denotes a vector of names
(resp. processes) of length $|\vec{x}|$ (resp. $|\vec{P}|$). We adopt
the following useful abbreviations.

\begin{mathpar}
   x?(\vec{y}).P := x.(\vec{y})P \and  x\clift{\vec{P}} := x.\clift{\vec{P}}
   \and x!(y) := \lift{x}{\dropn{y}}
   \and \Pi_{i=0}^{n-1}P_i := P_0 | \ldots | P_{n-1}
\end{mathpar}

\subsubsection{Structural congruence}

\paragraph{Free and bound names and alpha-equivalence.} At the
core of structural equivalence is alpha-equivalence which identifies
process that are the same up to a change of variable. Formally, we
recognize the distinction between free and bound names. The free names
of a process, $\freenames{P}$, may be calculated recursively as
follows:

\begin{mathpar}
\freenames{\pzero} := \emptyset
  \and \\
  \freenames{x?(y).P} := \{ x \} \cup (\freenames{P} \setminus \{ y \})
  \and 
  \freenames{x!\langle P \rangle} := \{ x \} \cup \{ P \} 
  \and \\
  \freenames{P|Q} := \freenames{P} \cup \freenames{Q}
  \and \\
  \freenames{@{x}} := \{ x \}
\end{mathpar}

$\pi$
$\quotep{\pi}$

$\freenames{-} : \pi \to \mathcal{P}(\quotep{\pi})$

\begin{eqnarray*}
  \freenames{\pzero} & := & \emptyset \\
  \freenames{x?(y).P} & := & \{ x \} \cup (\freenames{P} \setminus \{ y \}) \\
  \freenames{x!\langle P \rangle} & := & \{ x \} \cup \{ P \} \\
  \freenames{P|Q} & := & \freenames{P} \cup \freenames{Q} \\
  \freenames{\dropn{x}} & := & \{ x \}
\end{eqnarray*}

The bound names of a process, $\boundnames{P}$, are those names occurring in $P$
that are not free. For example, in $x?(y).0$, the name $x$ is free, while $y$ is bound.

\begin{mathpar}
  \inferrule* [lab=monoidal-laws] {} { P|Q \equiv Q|P \and P|0 \equiv P \and P|(Q|R) \equiv (P|Q)|R }
\end{mathpar}

\begin{mathpar}
  \inferrule* [lab=alpha-equivalence] {} { (x)P \equiv (y)P\{y/x\} \and y \not\in \freenames{P} }
\end{mathpar}

\begin{definition}
Then two processes, $P,Q$, are alpha-equivalent if $P = Q\{\vec{y}/\vec{x}\}$ for
some $\vec{x} \in \boundnames{Q},\vec{y} \in \boundnames{P}$, where $Q\{\vec{y}/\vec{x}\}$
denotes the capture-avoiding substitution of $\vec{y}$ for $\vec{x}$ in $Q$.
\end{definition}

\begin{definition}
  The {\em structural congruence} \cite{SangiorgiWalker} , $\equiv$,
  between processes is the least congruence containing
  alpha-equivalence, satisfying the abelian monoid laws
  (associativity, commutativity and $\pzero$ as identity) for parallel
  composition $|$ and for summation $+$.
\end{definition}

\subsection{Name equivalence}

We take name equivalence, written $\nameeq$, to be the smallest
equivalence relation generated by the following rules.

\begin{mathpar}
\inferrule*[lab=Quote-drop]
{ }
{ \quotep{@{x}} \nameeq x }

\inferrule*[lab=Struct-equiv]
{ P \scong Q }
{ \quotep{P} \nameeq \quotep{Q} }
\end{mathpar}

The astute reader will have noticed that the mutual recursion of names
and processes imposes a mutual recursion on alpha-equivalence and
structural equivalence via name-equivalence. Fortunately, all of this
works out pleasantly and we may calculate in the natural way, free of
concern. The reader interested in the details is referred to the
appendix \ref{appendix:rho_details}.

\subsection{Substitution}

We use $\Proc$ for the set of processes, $\QProc$ for the set of
names, and $\id{\{}\vec{y} / \vec{x} \id{\}}$ to denote partial maps,
$s : \QProc \rightarrow \QProc$. A map, $s$ lifts, uniquely, to a map
on process terms, $\widehat{s} : \Proc \rightarrow \Proc$ by the
following equations.

\begin{mathpar}
  (0) \psubstp{Q}{P} := 0 \\
  (R \juxtap S) \psubstp{Q}{P}
  :=    
  (R)\psubstp{Q}{P} \juxtap (S) \psubstp{Q}{P} \\
  (x?(y).R) \psubstp{Q}{P}    
  :=    
  (x)\substp{Q}{P} (z)\concat( (R \psubstn{z}{y}) \psubstp{Q}{P} ) \\
  (\lift{x}{R}) \psubstp{Q}{P}  
  :=
  \lift{(x)\substp{Q}{P}}{ R \psubstp{Q}{P} } \\
%   (\dropn{x})  \psubstp{Q}{P}       
%   := 
%   \left\{ 
%     \begin{array}{ccc} 
%       \dropn{\quotep{Q}} & & x \nameeq \quotep{P} \\
%       \dropn{x} & & otherwise \\
%     \end{array}
%   \right. 
  (\dropn{x})  \psubstp{Q}{P}       
  := 
  \left\{ 
    \begin{array}{ccc} 
      Q & & x \nameeq \quotep{P} \\
      \dropn{x} & & otherwise \\
    \end{array}
  \right.
\end{mathpar}
 

where

\begin{eqnarray}
  (x)\id{\{} \lpquote Q \rpquote / \lpquote P \rpquote \id{\}}            = 
  \left\{ 
    \begin{array}{ccc}
      \lpquote Q \rpquote & & x \nameeq \lpquote P \rpquote \\
      x & & otherwise \\
    \end{array}
  \right. \nonumber
\end{eqnarray}

and $z$ is chosen distinct from $\quotep{P}$, $\quotep{Q}$, the free
names in $Q$, and all the names in $R$. Our $\alpha$-equivalence will
be built in the standard way from this substitution.

\begin{remark}\label{rem:no_self_referential_names}
  One consequence of these definitions is that $\forall P. \quotep{P}
  \not\in \freenames{P}$.
\end{remark}

\subsection{ Dynamic quote: an example }

Anticipating something of what's to come, consider applying the
substitution, $\widehat{\id{\{}u / z \id{\}}}$, to the following pair
of processes, $\lift{w}{y!(z)}$ and $w[ \lpquote y!(z) \rpquote ]$.

\begin{eqnarray}
	\lift{w}{y!(z)}\widehat{\id{\{}u / z \id{\}}}
		& = &
		\lift{w}{y!(u)} \nonumber\\
	w[ \lpquote y!(z) \rpquote ] \widehat{ \id{\{}u / z \id{\}} }
		& = &
		w[ \lpquote y!(z) \rpquote ] \nonumber
\end{eqnarray}

Because the body of the process between quotes is impervious to
substitution, we get radically different answers. In fact, by
examining the first process in an input context,
e.g. $x?(z).\lift{w}{y!(z)}$, we see that the process under the lift
operator may be shaped by prefixed inputs binding a name inside it. In
this sense, the lift operator will be seen as a way to dynamically
construct processes before reifying them as names.

Finally equipped with these standard features we can present the
dynamics of the calculus.

\subsubsection{Operational semantics} 

Finally, we introduce the computational dynamics. What marks these
algebras as distinct from other more traditionally studied algebraic
structures, e.g. vector spaces or polynomial rings, is the manner in
which dynamics is captured. In traditional structures, dynamics is typically
expressed through morphisms between such structures, as in linear maps
between vector spaces or morphisms between rings. In algebras
associated with the semantics of computation, the dynamics is
expressed as part of the algebraic structure itself, through a
reduction reduction relation typically denoted by $\red$. Below, we
give a recursive presentation of this relation for the calculus used
in the encoding.

$\red \subseteq \pi \times \pi$
$\red : \pi \to \mathcal{P}(\pi)$

\begin{mathpar}
  \inferrule* [lab=Comm] { \textsf{match}( x_{src}, x_{trgt} ) } { x_{trgt}?(y)P \; | \; x_{src}!\langle {Q} \rangle \red P\{\quotep{Q}/y}\} }
  \and \\
  \inferrule* [lab=Par] {{P} \red {P}'} {{{P} | {Q}} \red {{P}' | {Q}}}
  \and
  \inferrule* [lab=Equiv]{{{P} \scong {P}'} \andalso {{P}' \red {Q}'} \andalso {{Q}' \scong {Q}}}{{P} \red {Q}}
\end{mathpar}

\begin{eqnarray*}
  match_{\equiv} (\quotep{P},\quotep{Q}) & := & P \equiv Q \\
  match_{\dagger}(\quotep{P},\quotep{Q}) & := & \forall R. P|Q \red^{*} R => R \red^{*} 0 \\
  match_{K}(\quotep{P},\quotep{Q}) & := & K \mbox{ for some context } K
\end{eqnarray*}

$u?(x)P | u!\langle Q \rangle \red P\{\quotep{Q}/x\}$

%We write $\wred$ for $\red^*$, and $P\red$ if $\exists Q $ such that $ P \red Q$.
We write $P\red$ if $\exists Q $ such that $ P \red Q$ and $P\not\red$, otherwise.

\section{Replication}

As mentioned before, it is known that replication (and hence
recursion) can be implemented in a higher-order process algebra
\cite{SangiorgiWalker}. As our first example of calculation with the
machinery thus far presented we give the construction explicitly in
the {\rhoc}.

\begin{eqnarray}
	D_{x} & := & \prefix{x}{y}{(\binpar{\outputp{x}{y}}{@{y}})} \nonumber\\
	\bangp_{x}{P} & := & \binpar{{x}!\langle{\binpar{D_{x}}{P}}\rangle}{D_{x}} \nonumber
\end{eqnarray}

\begin{eqnarray}
	\bangp_{x}{P} & & \nonumber\\
	=
	& {x}!\langle{(\prefix{x}{y}{(\outputp{x}{y} | @{y})) | P}}\rangle 
	      | \prefix{x}{y}{(\outputp{x}{y} | @{y})} & \nonumber\\
	\red
	& (\outputp{x}{y} | @{y})\substn{\quotep{(\prefix{x}{y}{(@{y} | \outputp{x}{y})) | P}}}{y} & \nonumber\\
	=
	& \outputp{x}{\quotep{(\prefix{x}{y}{(\outputp{x}{y} | @{y})) | P}}}
	  | {(\prefix{x}{y}{(\outputp{x}{y} | @{y})) | P}} & \nonumber\\
	\red
	& \ldots & \nonumber\\
	\red^*
	& P | P | \ldots & \nonumber
\end{eqnarray}

Of course, this encoding, as an implementation, runs away, unfolding
$\bangp{P}$ eagerly. A lazier and more implementable replication
operator, restricted to input-guarded processes, may be obtained as follows.

\begin{eqnarray}
\bangp{\prefix{u}{v}{P}} 
	:= 
	\binpar{\lift{x}{\prefix{u}{v}{(\binpar{D(x)}{P})}}}{D(x)} \nonumber
\end{eqnarray}

\begin{remark}
  Note that the lazier definition still does not deal with summation
  or mixed summation (i.e. sums over input and output). The reader is
  invited to construct definitions of replication that deal with these
  features. 

  Further, the definitions are parameterized in a name, $x$. Can you,
  gentle reader, make a definition that eliminates this parameter and
  guarantees no accidental interaction between the replication
  machinery and the process being replicated -- i.e. no accidental
  sharing of names used by the process to get its work done and the
  name(s) used by the replication to effect copying. This latter
  revision of the definition of replication is crucial to obtaining
  the expected identity $!!P \sim !P$.
\end{remark}

\begin{remark}\label{rem:paradoxical_combinator}
  The reader familiar with the lambda calculus will have noticed the
  similarity between $D$ and the paradoxical combinator.

  [Ed. note: the existence of this seems to suggest we have to be more
  restrictive on the set of processes and names we admit if we are to
  support no-cloning.]
\end{remark}

\subsubsection{Bisimulation}

The computational dynamics gives rise to another kind of equivalence,
the equivalence of computational behavior. As previously mentioned
this is typically captured \emph{via} some form of bisimulation.

% The notion we use in this paper is weak barbed bisimulation
% \cite{milner91polyadicpi}.

The notion we use in this paper is derived from weak barbed
bisimulation \cite{milner91polyadicpi}. 

\begin{definition}
An \emph{observation relation}, $\downarrow_{\mathcal N}$, over a set
of names, $\mathcal N$, is the smallest relation satisfying the rules
below.

\infrule[Out-barb]{y \in {\mathcal N}, \; x \nameeq y}
		  {\outputp{x}{v} \downarrow_{\mathcal N} x}
\infrule[Par-barb]{\mbox{$P\downarrow_{\mathcal N} x$ or $Q\downarrow_{\mathcal N} x$}}
		  {\binpar{P}{Q} \downarrow_{\mathcal N} x}

We write $P \Downarrow_{\mathcal N} x$ if there is $Q$ such that 
$P \wred Q$ and $Q \downarrow_{\mathcal N} x$.
\end{definition}

\begin{definition}
%\label{def.bbisim}
An  ${\mathcal N}$-\emph{barbed bisimulation} over a set of names, ${\mathcal N}$, is a symmetric binary relation 
${\mathcal S}_{\mathcal N}$ between agents such that $P\rel{S}_{\mathcal N}Q$ implies:
\begin{enumerate}
\item If $P \red P'$ then $Q \wred Q'$ and $P'\rel{S}_{\mathcal N} Q'$.
\item If $P\downarrow_{\mathcal N} x$, then $Q\Downarrow_{\mathcal N} x$.
\end{enumerate}
$P$ is ${\mathcal N}$-barbed bisimilar to $Q$, written
$P \wbbisim_{\mathcal N} Q$, if $P \rel{S}_{\mathcal N} Q$ for some ${\mathcal N}$-barbed bisimulation ${\mathcal S}_{\mathcal N}$.
\end{definition}

$\mathcal{R} \subseteq \pi \times \pi$

$P \mathcal{R} Q => \forall P'. P \red P' \Rightarrow \exists Q'. Q \red Q', P' \mathcal{R} Q'$

$P \vdash x \Rightarrow Q \vdash x$

\begin{mathpar}
  \inferrule*[lab=Out-barb]{x \nameeq y}{{y}!\langle{Q}\rangle \vdash x}
  \and
  \inferrule*[lab=Par-barb]{\mbox{$P\vdash x$ or $Q\vdash x$}}{\binpar{P}{Q} \vdash x}
\end{mathpar}

\subsubsection{Contexts}

One of the principle advantages of computational calculi like the
$\pi$-calculus is a well-defined notion of context,
contextual-equivalence and a correlation between
contextual-equivalence and notions of bisimulation. The notion of
context allows the decomposition of a process into (sub-)process and
its syntactic environment, its context. Thus, a context may be
thought of as a process with a ``hole'' (written $\Box$) in it. The
application of a context $M$ to a process $P$, written $M[P]$, is
tantamount to filling the hole in $M$ with $P$. In this paper we do
not need the full weight of this theory, but do make use of the notion
of context in the proof the main theorem. 

\begin{mathpar}
  \inferrule* [lab=summation] {} {{M_{M},M_{N}} \bc \Box \;|\; x.M_{A} \;|\; M_{M}+M_{N}}
  \and
  \inferrule* [lab=agent] {} {{M_{A}} \bc (\vec{x})M_{P} \;| \; \clift{P_0,\ldots,M_{P},\ldots,P_N}}
  \and \\
  \inferrule* [lab=process] {} {{M_{P}} \bc M_{N} \;| \;P|M_{P} }
\end{mathpar} 

\begin{mathpar}
  \inferrule* [lab=sychronization] {} {M_{N} \bc \Box \;|\; x?M_{F} \;|\; x!M_{C}}
  \and
  \inferrule* [lab=abstraction] {} {{M_{F}} \bc (x)M_{P} }
  \and
  \inferrule* [lab=concretion] {} {{M_{C}} \bc \langle M_{P} \rangle }
  \and \\
  \inferrule* [lab=process] {} {{M_{P}} \bc M_{N} \;| \;P|M_{P} }
\end{mathpar}

\begin{definition}[contextual application] Given a context $M$, and
  process $P$, we define the \emph{contextual application}, $M[P] :=
  M\{P/\Box\}$. That is, the contextual application of M to P is the
  substitution of $P$ for $\Box$ in $M$.
\end{definition}

$\meaningof{-} : L \to \mathcal{P}(\pi)$

\begin{mathpar}
  \inferrule* [lab=collection] {} {\meaningof{true} = \pi, \and \meaningof{~E} = \pi \setminus \meaningof{E}, \and \meaningof{E_{1} \& E_{2}} = \meaningof{E_{1}} \cap \meaningof{E_{2}}}
\end{mathpar}

\begin{mathpar}
  \inferrule* [lab=structure] {} {\meaningof{0} = \{ P \in \pi | P \equiv 0 \}, \and \\ \meaningof{E_1 | E_2} = \{ P \in \pi | P \equiv P_{1} | P_{2}, P_{1} \in \meaningof{E_{1}}, P_{2} \in \meaningof{E_2}\} }
\end{mathpar}

\begin{mathpar}
 \inferrule* [lab=behavior] {} {\meaningof{\langle a?b \rangle E} = \{ P \in \pi | P \equiv Q | u?(y)P', \\ \and \\\\ \and \\ \;\;\; u \in \meaningof{a}, \forall z.P'\{z/y\} \in \meaningof{E\{z/b\}}\}, \and \\ \meaningof{a!E} = \{ P \in \pi | P \equiv Q | x!\langle P' \rangle, x \in \meaningof{a} P' \in \meaningof{E}\} }
\end{mathpar}

\begin{mathpar}
 \inferrule* [lab=nominal] {} {\meaningof{\quotep{E}} = \{ \quotep{P} \in \quotep{\pi} | P \in \meaningof{E} \}, \and \meaningof{\quotep{P}} = \{ \quotep{Q} \in \quotep{\pi} | P \equiv Q \} \and \\ \meaningof{@\quotep{E}} = \{ P \in \pi | P \equiv @x, x \in \meaningof{E} \}}
\end{mathpar}

\begin{eqnarray*}
  \\
  \meaningof{-} : TS \to ST
\end{eqnarray*}

\begin{eqnarray*}
  \\
  L : TS \to ST
\end{eqnarray*}

\begin{eqnarray*}
  \\
  P \models E \iff P \in \meaningof{E}
\end{eqnarray*}

\begin{eqnarray*}
  P \approx_{L} Q \iff \forall E \in L. P \models E \iff Q \models E
\end{eqnarray*}

\begin{eqnarray*}
  P \approx_{K} Q
\end{eqnarray*}

\begin{eqnarray*}
  P \approx Q
\end{eqnarray*}

$\approx_{K} = \approx = \approx_{L}$

\subsubsection{Contextual duality}

Note that contexts extend the quotation operation to a family of
operations from processes to names. Given a context, $M$, we can
define a \emph{nominal context}, $\quotep{M}$ by $\quotep{M}[P] :=
\quotep{M[P]}$. To foreshadow what is to come we observe that these
operations enjoy a duality with processes very much like the duality
between vectors and maps from vectors to scalars.

Further, because the calculus is essentially higher-order, we have a
correspondence between contexts and processes. More specifically,
given a name $x$ and a context $M$ we can construct $M^{*}_{x}$ such
that 

\begin{mathpar}
  M^{*}_{x} | \lift{x}{P} \red M[P]
\end{mathpar}

namely,

\begin{mathpar}
  M^{*}_{x} := x?(u).M[\dropn{u}]
\end{mathpar}

The dependence of $M^{*}_{x}$ on a name makes it an abstraction, 

\begin{mathpar}
  M^{*} := (x)x?(u).M[\dropn{u}]
\end{mathpar}

\subsection{Additional notation}

It will sometimes be convenient to denote the process a name
quotes. We already have the notation $x = \quotep{P}$, but it will be
convenient to introduce an alternate notation, $\procn{x}$, when we
want to emphasize the connection to the use of the name. Note that, by
virtue of name equivalence, $\quotep{\procn{x}} \nameeq x$; so, the
notation is consistent with previous definitions.

Further, because names have structure it is possible to effect
substitutions on the basis of that structure. This means we need to
upgrade our notation for substitutions, which we accomplish by
adapting comprehension notation. Thus,

\begin{mathpar}
  P\{ y / x : x \in S \}
\end{mathpar}

is interpreted to mean the process derived from P by replacing (in a
capture-avoiding manner) each occurrence of $x$ in $S$ by $y$. For example,

\begin{mathpar}
  P\{ \quotep{\procn{x}|\procn{x}} / x : x \in \freenames{P} \}
\end{mathpar}

will replace each (occurrence) of a free name $x$ in $P$ by
$\quotep{\procn{x}|\procn{x}}$.

Also, we will avail ourselves of the notation $x^{L}$ and $x^{R}$ to
denote injections of a name into disjoint copies of the name
space. There are numerous ways to accomplish this. One example can be
found in \cite{MeredithR05}. This notation overloads to vectors of
names: $\vec{x}^{\pi} := (x_{i}^{\pi} \; : \; 0 \leq i < |\vec{x}| )$ where $\pi \in \{L,R\}$.

We also use $P^{\Box} := P|\Box$.

In \cite{MeredithR05} an interpretation of the new operator is
given. It turns out that there are several possible interpretations
all enjoying the requisite algebraic properties of the operator (see
\cite{milner91polyadicpi}). We will therefore make liberal use of
$(\nu\; \vec{x})P$.

% subsection the_syntax_and_semantics_of_the_notation_system (end)   

\input{qm2pi.qmops} 

\input{qm2pi.sterngerlach} 

\input{qm2pi.metric} 

% section concurrent_process_calculi (end)

%\input{qm2pi.proofsketch}

% section proof sketch (end)

%\input{qm2pi.slviaknots} 

% section spatial logic via knots (end)

\input{qm2pi.conclusion}

% section conclusion (end)

%\input{qm2pi.dtcodes} 

% section wiring algorithm (end)

\input{qm2pi.ack} 

% section acknowledgments (end)

\newpage


\bibliographystyle{plain}   
\bibliography{../../biblios/main.bib}

\input{qm2pi.rhodetails}

\end{document}



% section proof sketch (end)

%\section{Unlikely characters: spatial logic for
  knots}\label{sub:characteristic_formulae} % (fold)

Associated to the mobile process calculi are a family of logics known
as the Hennessy-Milner logics. These logics typically enjoy a
semantics interpreting formulae as sets of processes that when
factored through the encoding outlined above allows an identification
of classes of knots with logical formulae. In the context of this
encoding the sub-family known as the spatial logics \cite{CairesC03}
\cite{CairesC04} \cite{Caires04} are of particular interest providing
several important features for expressing and reasoning about
properties (i.e. classes) of knots. We hint here at how this may be done.

%\begin{description}
%\item [structural connectives] 
\subsubsection{Structural connectives} The spatial logics enjoy
structural connectives corresponding, at the logical level, to the
parallel composition ($P | Q$) and new name ($(\nu \; x)P$)
connectives for processes. As illustrated in the examples below, these
connectives are extremely expressive given the shape of our encoding.
%\item [decideable satisfaction]

\subsubsection{Decideable satisfaction}
In \cite{Caires04} the satisfaction relation is shown to be decideable
for a rich class of processes. It further turns out that the image of
the our encoding is a proper subset of that class. This result
provides the basis for an algorithm by which to search for knots
enjoying a given property.
%\item [characteristic formulae]

\subsubsection{Characteristic formulae}
In the same paper \cite{Caires04} , Caires presents a means of calculating
characteristic formulae, selecting equivalence classes of processes
up to a pre--specified depth limit on the support set of names. Composed with our
encoding, this characteristic formula can be used to select
characteristic formulae for knots.
%\end{description}

\subsubsection{Spatial logic formulae}

The grammar below (segmented for comprehension) summarizes the syntax
of spatial logic formulae. We employ illustrative examples in the
sequel to provide an intuitive understanding of their meaning
referring the reader to \cite{Caires04} for a more detailed explication
of the semantics.

\begin{mathpar}
  \inferrule* [lab=boolean] {} {{A,B} \bc T \;|\; \neg A \;|\; A \wedge B \;|\; \eta = \eta'}
  \and
  \inferrule* [lab=spatial] {} {|\; \pzero \;|\; A | B \;|\; x \text{\textregistered} A \;|\; \forall x . A \;|\;  H x . A}
  \and
  \inferrule* [lab=behavioral] {} {|\; \alpha . A}
  \and 
  \inferrule* [lab=recursion] {} {|\; X(\vec{u}) \;|\; \mu X(\vec{u}) . A}
  \and
  \inferrule* [lab=action] {} {\alpha \bc \langle x?(\vec{y}) \rangle \;|\; \langle x!(\vec{y}) \rangle \;|\; \langle \tau \rangle}
  \and 
  \inferrule* [lab=name] {} {\eta \bc x \;|\; \tau}
\end{mathpar} 

% subsection characteristic_formulae (end)   	 

\subsection{Example formulae}\label{sub:example_formulae_} % (fold)

\subsubsection{Crossing as formula.}
% 
% \begin{align*}
%   \frac{d}{dx} \sin x &= \cos x 
%   & \frac{d}{dx} e^x &= e^x \\
%   \frac{d}{dx} \cos x &= - \sin x 
%   & \frac{d}{dx} \log x &= \frac{1}{x} \\
% \end{align*} 

\begin{align*}
 \mu C(x_{0},x_{1},y_{0},y_{1},u).&(\langle x_{0}?(z) \rangle(\langle u! \rangle\langle y_{1}!z \rangle C(x_{0},x_{1},y_{0},y_{1},u)) & \\
  & \wedge \langle y_{1}?(z) \rangle (\langle u! \rangle \langle x_{0}!z \rangle C(x_{0},x_{1},y_{0},y_{1},u)) & \\
  & \wedge \langle x_{1}?(z) \rangle (\langle u? \rangle \langle y_{0}!z \rangle C(x_{0},x_{1},y_{0},y_{1},u)) & \\
  & \wedge \langle y_{0}?(z) \rangle (\langle u? \rangle \langle x_{1}!z \rangle C(x_{0},x_{1},y_{0},y_{1},u))) &
\end{align*}

The lexicographical similarity between the shape of this formulae and
the shape of definition of the process representing a crossing reveals
the intuitive meaning of this formulae. It describes the capabilities
of a process that has the right to represent a crossing. For example
it picks out processes that may perform an input on the port $x_0$ in
its initial menu of capabilities. What differentiates the formula
from the process, however, is that the crossing process is the
smallest candidate to satisfy the formula. Infinitely many other
processes -- with internal behavior hidden behind this interface, so
to speak -- also satisfy this formula. Even this simple formula,
then, can be seen to open a new view onto knots, providing a
computational interpretation of \emph{virtual} knots.

Note that this formula is derived by hand. A similar formula can be
derived by employing Caires' calculation of characteristic formula
\cite{Caires04} to the process representing a crossing. In light of
this discussion, we let
$\meaningof{C}_{\phi}(x0,x1,y0,y1,u)$ denote a formula specifying the
dynamics we wish to capture of a crossing. To guarantee we preserve
the shape of the interface and minimal semantics we demand that
$\meaningof{C}_{\phi}(x0,x1,y0,y1,u) \Rightarrow
\textbf{C}(x0,x1,y0,y1,u)$ where $\textbf{C}(x0,x1,y0,y1,u)$ denotes
the formula above.
                            
\subsubsection{Crossing number constraints.}
The moral content of the context lemma (Lemma \ref{context}) is that the notion of
``locality'' in the Reidemeister moves is effectively captured by the
parallel composition operator of the process calculus. This intuition
extends through the logic. Given a formula,
$\meaningof{C}_{\phi}(x0,x1,y0,y1,u)$, we can use the structural
connectives to specify constraints on crossing numbers, such as at
least $n$ crossings, or exactly $n$ crossings.
\begin{mathpar}
  \inferrule* [lab=at-least-n] {} { K^{\geq n}_{\phi}(\vec{xs},\vec{ys}) := \Pi_{i=0}^{n-1} Hu . \meaningof{C}_{\phi}(xs_i,ys_i,u) | T }
  \and 
  \inferrule* [lab=exactly-n] {} { K^{= n}_{\phi}(\vec{xs},\vec{ys}) := \Pi_{i=0}^{n-1} Hu . \meaningof{C}_{\phi}(xs_i,ys_i,u) | \neg (\forall x_0,y_0,x_1,y_1,u . \meaningof{C}_{\phi}(x_0,y_0,x_1,y_1,u) | T) }
\end{mathpar}

To round out this section, recall that the encoding of an $n$-crossing
knot decomposes into a parallel composition of $n$ \emph{copies} of a
crossing process together with a wiring harness. To specify different
knot classes with the same crossing number amounts to specifying
logical constraints on the wiring harness. In the interest of space,
we defer examples to a forthcoming paper. Suffice it to say that both
the conditions ``alternating knot'' and ``contains the tangle
corresponding to 5/3'' are expressible. For example, it is possible to
calculate the characteristic formula of a process corresponding to the
tangle 5/3 and conjoin it into the classifying formula via the
composition connective of the logic.

Finally, we wish to observe that it is entirely within reason to
contemplate a more domain-specific version of spatial logic tailored
to the shape of processes in the image of the encoding. Such a
domain-specific logic would have a better claim to the title formal
language of knot properties.

% subsection example_formulae_ (end)

% section knots_as_processes (end) 

% section spatial logic via knots (end)

\section{Conclusions and future work}

\paragraph{Testing physical space}
You, gentle reader, may wonder why of all the theorems to be proved
given this set up we pick the one above. In some sense it's hardly
central to quantum mechanics. We see it as central in the sense that
it firmly establishes a notion of physical space arising from a notion
of the equivalence of behavior. Relating bisimulation to a metric is a
big step forward, but one is faced with interpreting the relationship
of that metric space to something more physical. Quantum mechanical
notions of ``physical'' space are still far from intuitive, but by
relating this idea of distance as testing to calculations that predict
physical circumstances we are making a not insignificant step forward
toward an understanding of the physical space we inhabit as
essentially dynamic.

\paragraph{Effectivity and simulation}
One of the observations we have yet to make is that the entire program
spelled out here is effective. We have built various interpreters for
the reflective calculus at work in this interpretation. In principle,
then, we can simulate quantum mechanics on a computer. The place where
the simulation may lose fidelity is the infinitely branching summation
for the annihilator.

In this connection i also want to point out that the evaluation style
calculation of the inner product puts the non-determinism of the
summation right at the heart of measurement. This suggests that
Milner's original reduction-based formulation of the dynamics of his
calculi in terms of sums was not just notationally suggestive of a
notion of measure-and-continue but captured some significant part of
the physics.

\paragraph{Quantum continuations}
In light of this last observation i want to point out that the
predominant account of quantum mechanics is missing a key aspect of a
truly compositional story of the physical situation. In a real lab,
when a measurement is made the observation can be made to feed into
another device that then makes another measurement conditioned on the
results of the first. This means that after the superposition was
collapsed the entire experimental set up remained in
superposition. While QM offers a means of writing this down it doesn't
quite line up well with the well-trodden formulation of computation
and continuation that we see so succinctly expressed in Milner's
calculi. This suggests that there might be advantages to this account
of dynamics waiting to be explored.

\paragraph{Quantum logic}
In this connection, we also note that by virtue of having the
Hennessy-Milner construction, we can pull the construction through the
interpretation of QM. This gives us a natural candidate for a quantum
logic that enjoys an extremely tight connection with it's domain of
interpretation, making the construction much less ad hoc (rather it is
the image of functor!).

\paragraph{Quantum probabiity}
i have questions about the basis of the interpretation of inner
product as probability amplitude. In particular, using which
axiomatization of probability theory does the notion of probability
amplitude earn the right to be so dubbed? In other words, where is the
proof that the operation for calculating a probability amplitude (and
then squaring) satisfies the axioms of what it means to calculate a
probability? Even if such a proof exists (i have yet to find it in the
literature), i wonder if it might not be possible to turn things on
their heads. Can we view the calculation of the probability amplitude
as an axiomatization of probability? If so, then the definition we
give for calculating probability amplitude may provide the basis for
an \emph{effective} theory of probability.

\paragraph{Quantum vs ``biological'' information}
Finally, i want to conclude with a more philosophical observation. At
a recent workshop in which QM was a predominant topic i noticed
something about quantum information. The speaker was giving a riveting
discussion of axiomatic QM and showing how properties of ``no
cloning'' and ``no deleting'' emerged as consequences of the
axiomatization. Theorems of this form are necessary to give us a sense
of confidence that our axioms characterize the physical theory. What
struck me, though, was that if quantum information is neither erasable
nor replicable it is markedly different from \emph{life}. Two of the
things we know about life is that

\begin{itemize}
  \item it ends;
  \item to gain some measure of persistence, to transcend it's
    finitude it is imminently copyable.
\end{itemize}

Both of these qualities are summarized succinctly in the aphorism: all
flesh is grass. For me these two kinds of ``information'' -- call them
quantum and biological -- are end points on a spectrum of strategies
for persistence. At one end, we have those curious entities that enjoy
uniqueness and permanence; at the other, we have those who in the face
of a certain end and an uncertain present make a go of passing
something on. To me one of the more remarkable aspects of the latter
strategy is that in the presence of noise (and certain features of
copying) we get a kind of dynamism, a chance for improvement against a
given persistent condition.

% subsection other_calculi_other_bisimulations_and_geometry_as_behavior (end)




% section conclusion (end)

%\documentclass[12pt]{llncs}
%\documentclass{jktr}

\usepackage[pdftex]{hyperref}                   
\usepackage {listings}
\usepackage {mathpartir}
\usepackage{bcprules}
%\usepackage{listings}
                       
\usepackage{graphicx} 
%\usepackage[margins=2.5cm,nohead,nofoot]{geometry}
%\usepackage{geometry}
\usepackage{amsfonts}
\usepackage{amstext}
\usepackage{latexsym}
\usepackage{amssymb}
\usepackage{color}


%\include{myPreamble}
\include{qm2pi.local} 

%\ifpdf
%\usepackage[pdftex]{graphicx}
%\else
%\usepackage{graphicx}
%\fi

 % \ifpdf
%  \usepackage{pdfsync}
%  \if


%\title{Brief Article}
%\author{David F. Snyder}
%\author{L.G. Meredith}

%\address{Dept. of Math., Texas State University--San Marcos, San Marcos, TX 78666}
       
\pagestyle{empty}


\begin{document}

\lstset{language=[Objective]Caml,frame=shadowbox}

\input{qm2pi.front}

% section front matter (end)

\input{qm2pi.intro} 
 
% section introduction (end)

% \input{qm2pi.knotations} 

% section notation (end)

\input{qm2pi.process.calculi} 

% section concurrent_process_calculi_and_spatial_logics_ (end)
    
%\input{qm2pi.knots2pi} 

%\input{qm2pi.trefoil} 

%\input{qm2pi.mainthm} 

% subsection basic_interpretation (end)

%\input{qm2pi.rho.presentation} 
\subsection{The syntax and semantics of the notation system}\label{sub:the_syntax_and_semantics_of_the_notation_system} % (fold)

We now summarize a technical presentation of the calculus that
embodies our theory of dynamics. The typical presentation of such a
calculus follows the style of giving generators and relations on
them. The grammar, below, describing term constructors, freely
generates the set of processes, $\Proc$. This set is then quotiented
by a relation known as structural congruence and it is over this set
that the notion of dynamics is expressed. This presentation is
essentially that of \cite{MeredithR05} with the addition of
polyadicity and summation. For readability we have relegated some of
the technical subtleties to an appendix.

\subsubsection{Process grammar}\label{subsub:process_grammar}

\begin{mathpar}
  \inferrule* [lab=synchronization] {} {{M} \bc \pzero \;|\; x?F \;|\; x!C }
  \and
  \inferrule* [lab=abstraction] {} {{F} \bc (x)P}
  \and
  \inferrule* [lab=concretion] {} {{C} \bc \langle Q \rangle}
  \and
  \inferrule* [lab=process] {} {{P,Q} \bc M \;| \;P|Q \;|\; @{x}}
  \and
  \inferrule* [lab=name] {} {{x} \bc \quotep{P}}
\end{mathpar} 

Note that $\vec{x}$ (resp. $\vec{P}$) denotes a vector of names
(resp. processes) of length $|\vec{x}|$ (resp. $|\vec{P}|$). We adopt
the following useful abbreviations.

\begin{mathpar}
   x?(\vec{y}).P := x.(\vec{y})P \and  x\clift{\vec{P}} := x.\clift{\vec{P}}
   \and x!(y) := \lift{x}{\dropn{y}}
   \and \Pi_{i=0}^{n-1}P_i := P_0 | \ldots | P_{n-1}
\end{mathpar}

\subsubsection{Structural congruence}

\paragraph{Free and bound names and alpha-equivalence.} At the
core of structural equivalence is alpha-equivalence which identifies
process that are the same up to a change of variable. Formally, we
recognize the distinction between free and bound names. The free names
of a process, $\freenames{P}$, may be calculated recursively as
follows:

\begin{mathpar}
\freenames{\pzero} := \emptyset
  \and \\
  \freenames{x?(y).P} := \{ x \} \cup (\freenames{P} \setminus \{ y \})
  \and 
  \freenames{x!\langle P \rangle} := \{ x \} \cup \{ P \} 
  \and \\
  \freenames{P|Q} := \freenames{P} \cup \freenames{Q}
  \and \\
  \freenames{@{x}} := \{ x \}
\end{mathpar}

$\pi$
$\quotep{\pi}$

$\freenames{-} : \pi \to \mathcal{P}(\quotep{\pi})$

\begin{eqnarray*}
  \freenames{\pzero} & := & \emptyset \\
  \freenames{x?(y).P} & := & \{ x \} \cup (\freenames{P} \setminus \{ y \}) \\
  \freenames{x!\langle P \rangle} & := & \{ x \} \cup \{ P \} \\
  \freenames{P|Q} & := & \freenames{P} \cup \freenames{Q} \\
  \freenames{\dropn{x}} & := & \{ x \}
\end{eqnarray*}

The bound names of a process, $\boundnames{P}$, are those names occurring in $P$
that are not free. For example, in $x?(y).0$, the name $x$ is free, while $y$ is bound.

\begin{mathpar}
  \inferrule* [lab=monoidal-laws] {} { P|Q \equiv Q|P \and P|0 \equiv P \and P|(Q|R) \equiv (P|Q)|R }
\end{mathpar}

\begin{mathpar}
  \inferrule* [lab=alpha-equivalence] {} { (x)P \equiv (y)P\{y/x\} \and y \not\in \freenames{P} }
\end{mathpar}

\begin{definition}
Then two processes, $P,Q$, are alpha-equivalent if $P = Q\{\vec{y}/\vec{x}\}$ for
some $\vec{x} \in \boundnames{Q},\vec{y} \in \boundnames{P}$, where $Q\{\vec{y}/\vec{x}\}$
denotes the capture-avoiding substitution of $\vec{y}$ for $\vec{x}$ in $Q$.
\end{definition}

\begin{definition}
  The {\em structural congruence} \cite{SangiorgiWalker} , $\equiv$,
  between processes is the least congruence containing
  alpha-equivalence, satisfying the abelian monoid laws
  (associativity, commutativity and $\pzero$ as identity) for parallel
  composition $|$ and for summation $+$.
\end{definition}

\subsection{Name equivalence}

We take name equivalence, written $\nameeq$, to be the smallest
equivalence relation generated by the following rules.

\begin{mathpar}
\inferrule*[lab=Quote-drop]
{ }
{ \quotep{@{x}} \nameeq x }

\inferrule*[lab=Struct-equiv]
{ P \scong Q }
{ \quotep{P} \nameeq \quotep{Q} }
\end{mathpar}

The astute reader will have noticed that the mutual recursion of names
and processes imposes a mutual recursion on alpha-equivalence and
structural equivalence via name-equivalence. Fortunately, all of this
works out pleasantly and we may calculate in the natural way, free of
concern. The reader interested in the details is referred to the
appendix \ref{appendix:rho_details}.

\subsection{Substitution}

We use $\Proc$ for the set of processes, $\QProc$ for the set of
names, and $\id{\{}\vec{y} / \vec{x} \id{\}}$ to denote partial maps,
$s : \QProc \rightarrow \QProc$. A map, $s$ lifts, uniquely, to a map
on process terms, $\widehat{s} : \Proc \rightarrow \Proc$ by the
following equations.

\begin{mathpar}
  (0) \psubstp{Q}{P} := 0 \\
  (R \juxtap S) \psubstp{Q}{P}
  :=    
  (R)\psubstp{Q}{P} \juxtap (S) \psubstp{Q}{P} \\
  (x?(y).R) \psubstp{Q}{P}    
  :=    
  (x)\substp{Q}{P} (z)\concat( (R \psubstn{z}{y}) \psubstp{Q}{P} ) \\
  (\lift{x}{R}) \psubstp{Q}{P}  
  :=
  \lift{(x)\substp{Q}{P}}{ R \psubstp{Q}{P} } \\
%   (\dropn{x})  \psubstp{Q}{P}       
%   := 
%   \left\{ 
%     \begin{array}{ccc} 
%       \dropn{\quotep{Q}} & & x \nameeq \quotep{P} \\
%       \dropn{x} & & otherwise \\
%     \end{array}
%   \right. 
  (\dropn{x})  \psubstp{Q}{P}       
  := 
  \left\{ 
    \begin{array}{ccc} 
      Q & & x \nameeq \quotep{P} \\
      \dropn{x} & & otherwise \\
    \end{array}
  \right.
\end{mathpar}
 

where

\begin{eqnarray}
  (x)\id{\{} \lpquote Q \rpquote / \lpquote P \rpquote \id{\}}            = 
  \left\{ 
    \begin{array}{ccc}
      \lpquote Q \rpquote & & x \nameeq \lpquote P \rpquote \\
      x & & otherwise \\
    \end{array}
  \right. \nonumber
\end{eqnarray}

and $z$ is chosen distinct from $\quotep{P}$, $\quotep{Q}$, the free
names in $Q$, and all the names in $R$. Our $\alpha$-equivalence will
be built in the standard way from this substitution.

\begin{remark}\label{rem:no_self_referential_names}
  One consequence of these definitions is that $\forall P. \quotep{P}
  \not\in \freenames{P}$.
\end{remark}

\subsection{ Dynamic quote: an example }

Anticipating something of what's to come, consider applying the
substitution, $\widehat{\id{\{}u / z \id{\}}}$, to the following pair
of processes, $\lift{w}{y!(z)}$ and $w[ \lpquote y!(z) \rpquote ]$.

\begin{eqnarray}
	\lift{w}{y!(z)}\widehat{\id{\{}u / z \id{\}}}
		& = &
		\lift{w}{y!(u)} \nonumber\\
	w[ \lpquote y!(z) \rpquote ] \widehat{ \id{\{}u / z \id{\}} }
		& = &
		w[ \lpquote y!(z) \rpquote ] \nonumber
\end{eqnarray}

Because the body of the process between quotes is impervious to
substitution, we get radically different answers. In fact, by
examining the first process in an input context,
e.g. $x?(z).\lift{w}{y!(z)}$, we see that the process under the lift
operator may be shaped by prefixed inputs binding a name inside it. In
this sense, the lift operator will be seen as a way to dynamically
construct processes before reifying them as names.

Finally equipped with these standard features we can present the
dynamics of the calculus.

\subsubsection{Operational semantics} 

Finally, we introduce the computational dynamics. What marks these
algebras as distinct from other more traditionally studied algebraic
structures, e.g. vector spaces or polynomial rings, is the manner in
which dynamics is captured. In traditional structures, dynamics is typically
expressed through morphisms between such structures, as in linear maps
between vector spaces or morphisms between rings. In algebras
associated with the semantics of computation, the dynamics is
expressed as part of the algebraic structure itself, through a
reduction reduction relation typically denoted by $\red$. Below, we
give a recursive presentation of this relation for the calculus used
in the encoding.

$\red \subseteq \pi \times \pi$
$\red : \pi \to \mathcal{P}(\pi)$

\begin{mathpar}
  \inferrule* [lab=Comm] { \textsf{match}( x_{src}, x_{trgt} ) } { x_{trgt}?(y)P \; | \; x_{src}!\langle {Q} \rangle \red P\{\quotep{Q}/y}\} }
  \and \\
  \inferrule* [lab=Par] {{P} \red {P}'} {{{P} | {Q}} \red {{P}' | {Q}}}
  \and
  \inferrule* [lab=Equiv]{{{P} \scong {P}'} \andalso {{P}' \red {Q}'} \andalso {{Q}' \scong {Q}}}{{P} \red {Q}}
\end{mathpar}

\begin{eqnarray*}
  match_{\equiv} (\quotep{P},\quotep{Q}) & := & P \equiv Q \\
  match_{\dagger}(\quotep{P},\quotep{Q}) & := & \forall R. P|Q \red^{*} R => R \red^{*} 0 \\
  match_{K}(\quotep{P},\quotep{Q}) & := & K \mbox{ for some context } K
\end{eqnarray*}

$u?(x)P | u!\langle Q \rangle \red P\{\quotep{Q}/x\}$

%We write $\wred$ for $\red^*$, and $P\red$ if $\exists Q $ such that $ P \red Q$.
We write $P\red$ if $\exists Q $ such that $ P \red Q$ and $P\not\red$, otherwise.

\section{Replication}

As mentioned before, it is known that replication (and hence
recursion) can be implemented in a higher-order process algebra
\cite{SangiorgiWalker}. As our first example of calculation with the
machinery thus far presented we give the construction explicitly in
the {\rhoc}.

\begin{eqnarray}
	D_{x} & := & \prefix{x}{y}{(\binpar{\outputp{x}{y}}{@{y}})} \nonumber\\
	\bangp_{x}{P} & := & \binpar{{x}!\langle{\binpar{D_{x}}{P}}\rangle}{D_{x}} \nonumber
\end{eqnarray}

\begin{eqnarray}
	\bangp_{x}{P} & & \nonumber\\
	=
	& {x}!\langle{(\prefix{x}{y}{(\outputp{x}{y} | @{y})) | P}}\rangle 
	      | \prefix{x}{y}{(\outputp{x}{y} | @{y})} & \nonumber\\
	\red
	& (\outputp{x}{y} | @{y})\substn{\quotep{(\prefix{x}{y}{(@{y} | \outputp{x}{y})) | P}}}{y} & \nonumber\\
	=
	& \outputp{x}{\quotep{(\prefix{x}{y}{(\outputp{x}{y} | @{y})) | P}}}
	  | {(\prefix{x}{y}{(\outputp{x}{y} | @{y})) | P}} & \nonumber\\
	\red
	& \ldots & \nonumber\\
	\red^*
	& P | P | \ldots & \nonumber
\end{eqnarray}

Of course, this encoding, as an implementation, runs away, unfolding
$\bangp{P}$ eagerly. A lazier and more implementable replication
operator, restricted to input-guarded processes, may be obtained as follows.

\begin{eqnarray}
\bangp{\prefix{u}{v}{P}} 
	:= 
	\binpar{\lift{x}{\prefix{u}{v}{(\binpar{D(x)}{P})}}}{D(x)} \nonumber
\end{eqnarray}

\begin{remark}
  Note that the lazier definition still does not deal with summation
  or mixed summation (i.e. sums over input and output). The reader is
  invited to construct definitions of replication that deal with these
  features. 

  Further, the definitions are parameterized in a name, $x$. Can you,
  gentle reader, make a definition that eliminates this parameter and
  guarantees no accidental interaction between the replication
  machinery and the process being replicated -- i.e. no accidental
  sharing of names used by the process to get its work done and the
  name(s) used by the replication to effect copying. This latter
  revision of the definition of replication is crucial to obtaining
  the expected identity $!!P \sim !P$.
\end{remark}

\begin{remark}\label{rem:paradoxical_combinator}
  The reader familiar with the lambda calculus will have noticed the
  similarity between $D$ and the paradoxical combinator.

  [Ed. note: the existence of this seems to suggest we have to be more
  restrictive on the set of processes and names we admit if we are to
  support no-cloning.]
\end{remark}

\subsubsection{Bisimulation}

The computational dynamics gives rise to another kind of equivalence,
the equivalence of computational behavior. As previously mentioned
this is typically captured \emph{via} some form of bisimulation.

% The notion we use in this paper is weak barbed bisimulation
% \cite{milner91polyadicpi}.

The notion we use in this paper is derived from weak barbed
bisimulation \cite{milner91polyadicpi}. 

\begin{definition}
An \emph{observation relation}, $\downarrow_{\mathcal N}$, over a set
of names, $\mathcal N$, is the smallest relation satisfying the rules
below.

\infrule[Out-barb]{y \in {\mathcal N}, \; x \nameeq y}
		  {\outputp{x}{v} \downarrow_{\mathcal N} x}
\infrule[Par-barb]{\mbox{$P\downarrow_{\mathcal N} x$ or $Q\downarrow_{\mathcal N} x$}}
		  {\binpar{P}{Q} \downarrow_{\mathcal N} x}

We write $P \Downarrow_{\mathcal N} x$ if there is $Q$ such that 
$P \wred Q$ and $Q \downarrow_{\mathcal N} x$.
\end{definition}

\begin{definition}
%\label{def.bbisim}
An  ${\mathcal N}$-\emph{barbed bisimulation} over a set of names, ${\mathcal N}$, is a symmetric binary relation 
${\mathcal S}_{\mathcal N}$ between agents such that $P\rel{S}_{\mathcal N}Q$ implies:
\begin{enumerate}
\item If $P \red P'$ then $Q \wred Q'$ and $P'\rel{S}_{\mathcal N} Q'$.
\item If $P\downarrow_{\mathcal N} x$, then $Q\Downarrow_{\mathcal N} x$.
\end{enumerate}
$P$ is ${\mathcal N}$-barbed bisimilar to $Q$, written
$P \wbbisim_{\mathcal N} Q$, if $P \rel{S}_{\mathcal N} Q$ for some ${\mathcal N}$-barbed bisimulation ${\mathcal S}_{\mathcal N}$.
\end{definition}

$\mathcal{R} \subseteq \pi \times \pi$

$P \mathcal{R} Q => \forall P'. P \red P' \Rightarrow \exists Q'. Q \red Q', P' \mathcal{R} Q'$

$P \vdash x \Rightarrow Q \vdash x$

\begin{mathpar}
  \inferrule*[lab=Out-barb]{x \nameeq y}{{y}!\langle{Q}\rangle \vdash x}
  \and
  \inferrule*[lab=Par-barb]{\mbox{$P\vdash x$ or $Q\vdash x$}}{\binpar{P}{Q} \vdash x}
\end{mathpar}

\subsubsection{Contexts}

One of the principle advantages of computational calculi like the
$\pi$-calculus is a well-defined notion of context,
contextual-equivalence and a correlation between
contextual-equivalence and notions of bisimulation. The notion of
context allows the decomposition of a process into (sub-)process and
its syntactic environment, its context. Thus, a context may be
thought of as a process with a ``hole'' (written $\Box$) in it. The
application of a context $M$ to a process $P$, written $M[P]$, is
tantamount to filling the hole in $M$ with $P$. In this paper we do
not need the full weight of this theory, but do make use of the notion
of context in the proof the main theorem. 

\begin{mathpar}
  \inferrule* [lab=summation] {} {{M_{M},M_{N}} \bc \Box \;|\; x.M_{A} \;|\; M_{M}+M_{N}}
  \and
  \inferrule* [lab=agent] {} {{M_{A}} \bc (\vec{x})M_{P} \;| \; \clift{P_0,\ldots,M_{P},\ldots,P_N}}
  \and \\
  \inferrule* [lab=process] {} {{M_{P}} \bc M_{N} \;| \;P|M_{P} }
\end{mathpar} 

\begin{mathpar}
  \inferrule* [lab=sychronization] {} {M_{N} \bc \Box \;|\; x?M_{F} \;|\; x!M_{C}}
  \and
  \inferrule* [lab=abstraction] {} {{M_{F}} \bc (x)M_{P} }
  \and
  \inferrule* [lab=concretion] {} {{M_{C}} \bc \langle M_{P} \rangle }
  \and \\
  \inferrule* [lab=process] {} {{M_{P}} \bc M_{N} \;| \;P|M_{P} }
\end{mathpar}

\begin{definition}[contextual application] Given a context $M$, and
  process $P$, we define the \emph{contextual application}, $M[P] :=
  M\{P/\Box\}$. That is, the contextual application of M to P is the
  substitution of $P$ for $\Box$ in $M$.
\end{definition}

$\meaningof{-} : L \to \mathcal{P}(\pi)$

\begin{mathpar}
  \inferrule* [lab=collection] {} {\meaningof{true} = \pi, \and \meaningof{~E} = \pi \setminus \meaningof{E}, \and \meaningof{E_{1} \& E_{2}} = \meaningof{E_{1}} \cap \meaningof{E_{2}}}
\end{mathpar}

\begin{mathpar}
  \inferrule* [lab=structure] {} {\meaningof{0} = \{ P \in \pi | P \equiv 0 \}, \and \\ \meaningof{E_1 | E_2} = \{ P \in \pi | P \equiv P_{1} | P_{2}, P_{1} \in \meaningof{E_{1}}, P_{2} \in \meaningof{E_2}\} }
\end{mathpar}

\begin{mathpar}
 \inferrule* [lab=behavior] {} {\meaningof{\langle a?b \rangle E} = \{ P \in \pi | P \equiv Q | u?(y)P', \\ \and \\\\ \and \\ \;\;\; u \in \meaningof{a}, \forall z.P'\{z/y\} \in \meaningof{E\{z/b\}}\}, \and \\ \meaningof{a!E} = \{ P \in \pi | P \equiv Q | x!\langle P' \rangle, x \in \meaningof{a} P' \in \meaningof{E}\} }
\end{mathpar}

\begin{mathpar}
 \inferrule* [lab=nominal] {} {\meaningof{\quotep{E}} = \{ \quotep{P} \in \quotep{\pi} | P \in \meaningof{E} \}, \and \meaningof{\quotep{P}} = \{ \quotep{Q} \in \quotep{\pi} | P \equiv Q \} \and \\ \meaningof{@\quotep{E}} = \{ P \in \pi | P \equiv @x, x \in \meaningof{E} \}}
\end{mathpar}

\begin{eqnarray*}
  \\
  \meaningof{-} : TS \to ST
\end{eqnarray*}

\begin{eqnarray*}
  \\
  L : TS \to ST
\end{eqnarray*}

\begin{eqnarray*}
  \\
  P \models E \iff P \in \meaningof{E}
\end{eqnarray*}

\begin{eqnarray*}
  P \approx_{L} Q \iff \forall E \in L. P \models E \iff Q \models E
\end{eqnarray*}

\begin{eqnarray*}
  P \approx_{K} Q
\end{eqnarray*}

\begin{eqnarray*}
  P \approx Q
\end{eqnarray*}

$\approx_{K} = \approx = \approx_{L}$

\subsubsection{Contextual duality}

Note that contexts extend the quotation operation to a family of
operations from processes to names. Given a context, $M$, we can
define a \emph{nominal context}, $\quotep{M}$ by $\quotep{M}[P] :=
\quotep{M[P]}$. To foreshadow what is to come we observe that these
operations enjoy a duality with processes very much like the duality
between vectors and maps from vectors to scalars.

Further, because the calculus is essentially higher-order, we have a
correspondence between contexts and processes. More specifically,
given a name $x$ and a context $M$ we can construct $M^{*}_{x}$ such
that 

\begin{mathpar}
  M^{*}_{x} | \lift{x}{P} \red M[P]
\end{mathpar}

namely,

\begin{mathpar}
  M^{*}_{x} := x?(u).M[\dropn{u}]
\end{mathpar}

The dependence of $M^{*}_{x}$ on a name makes it an abstraction, 

\begin{mathpar}
  M^{*} := (x)x?(u).M[\dropn{u}]
\end{mathpar}

\subsection{Additional notation}

It will sometimes be convenient to denote the process a name
quotes. We already have the notation $x = \quotep{P}$, but it will be
convenient to introduce an alternate notation, $\procn{x}$, when we
want to emphasize the connection to the use of the name. Note that, by
virtue of name equivalence, $\quotep{\procn{x}} \nameeq x$; so, the
notation is consistent with previous definitions.

Further, because names have structure it is possible to effect
substitutions on the basis of that structure. This means we need to
upgrade our notation for substitutions, which we accomplish by
adapting comprehension notation. Thus,

\begin{mathpar}
  P\{ y / x : x \in S \}
\end{mathpar}

is interpreted to mean the process derived from P by replacing (in a
capture-avoiding manner) each occurrence of $x$ in $S$ by $y$. For example,

\begin{mathpar}
  P\{ \quotep{\procn{x}|\procn{x}} / x : x \in \freenames{P} \}
\end{mathpar}

will replace each (occurrence) of a free name $x$ in $P$ by
$\quotep{\procn{x}|\procn{x}}$.

Also, we will avail ourselves of the notation $x^{L}$ and $x^{R}$ to
denote injections of a name into disjoint copies of the name
space. There are numerous ways to accomplish this. One example can be
found in \cite{MeredithR05}. This notation overloads to vectors of
names: $\vec{x}^{\pi} := (x_{i}^{\pi} \; : \; 0 \leq i < |\vec{x}| )$ where $\pi \in \{L,R\}$.

We also use $P^{\Box} := P|\Box$.

In \cite{MeredithR05} an interpretation of the new operator is
given. It turns out that there are several possible interpretations
all enjoying the requisite algebraic properties of the operator (see
\cite{milner91polyadicpi}). We will therefore make liberal use of
$(\nu\; \vec{x})P$.

% subsection the_syntax_and_semantics_of_the_notation_system (end)   

\input{qm2pi.qmops} 

\input{qm2pi.sterngerlach} 

\input{qm2pi.metric} 

% section concurrent_process_calculi (end)

%\input{qm2pi.proofsketch}

% section proof sketch (end)

%\input{qm2pi.slviaknots} 

% section spatial logic via knots (end)

\input{qm2pi.conclusion}

% section conclusion (end)

%\input{qm2pi.dtcodes} 

% section wiring algorithm (end)

\input{qm2pi.ack} 

% section acknowledgments (end)

\newpage


\bibliographystyle{plain}   
\bibliography{../../biblios/main.bib}

\input{qm2pi.rhodetails}

\end{document}

 

% section wiring algorithm (end)

\documentclass[12pt]{llncs}
%\documentclass{jktr}

\usepackage[pdftex]{hyperref}                   
\usepackage {listings}
\usepackage {mathpartir}
\usepackage{bcprules}
%\usepackage{listings}
                       
\usepackage{graphicx} 
%\usepackage[margins=2.5cm,nohead,nofoot]{geometry}
%\usepackage{geometry}
\usepackage{amsfonts}
\usepackage{amstext}
\usepackage{latexsym}
\usepackage{amssymb}
\usepackage{color}


%\include{myPreamble}
\include{qm2pi.local} 

%\ifpdf
%\usepackage[pdftex]{graphicx}
%\else
%\usepackage{graphicx}
%\fi

 % \ifpdf
%  \usepackage{pdfsync}
%  \if


%\title{Brief Article}
%\author{David F. Snyder}
%\author{L.G. Meredith}

%\address{Dept. of Math., Texas State University--San Marcos, San Marcos, TX 78666}
       
\pagestyle{empty}


\begin{document}

\lstset{language=[Objective]Caml,frame=shadowbox}

\input{qm2pi.front}

% section front matter (end)

\input{qm2pi.intro} 
 
% section introduction (end)

% \input{qm2pi.knotations} 

% section notation (end)

\input{qm2pi.process.calculi} 

% section concurrent_process_calculi_and_spatial_logics_ (end)
    
%\input{qm2pi.knots2pi} 

%\input{qm2pi.trefoil} 

%\input{qm2pi.mainthm} 

% subsection basic_interpretation (end)

%\input{qm2pi.rho.presentation} 
\subsection{The syntax and semantics of the notation system}\label{sub:the_syntax_and_semantics_of_the_notation_system} % (fold)

We now summarize a technical presentation of the calculus that
embodies our theory of dynamics. The typical presentation of such a
calculus follows the style of giving generators and relations on
them. The grammar, below, describing term constructors, freely
generates the set of processes, $\Proc$. This set is then quotiented
by a relation known as structural congruence and it is over this set
that the notion of dynamics is expressed. This presentation is
essentially that of \cite{MeredithR05} with the addition of
polyadicity and summation. For readability we have relegated some of
the technical subtleties to an appendix.

\subsubsection{Process grammar}\label{subsub:process_grammar}

\begin{mathpar}
  \inferrule* [lab=synchronization] {} {{M} \bc \pzero \;|\; x?F \;|\; x!C }
  \and
  \inferrule* [lab=abstraction] {} {{F} \bc (x)P}
  \and
  \inferrule* [lab=concretion] {} {{C} \bc \langle Q \rangle}
  \and
  \inferrule* [lab=process] {} {{P,Q} \bc M \;| \;P|Q \;|\; @{x}}
  \and
  \inferrule* [lab=name] {} {{x} \bc \quotep{P}}
\end{mathpar} 

Note that $\vec{x}$ (resp. $\vec{P}$) denotes a vector of names
(resp. processes) of length $|\vec{x}|$ (resp. $|\vec{P}|$). We adopt
the following useful abbreviations.

\begin{mathpar}
   x?(\vec{y}).P := x.(\vec{y})P \and  x\clift{\vec{P}} := x.\clift{\vec{P}}
   \and x!(y) := \lift{x}{\dropn{y}}
   \and \Pi_{i=0}^{n-1}P_i := P_0 | \ldots | P_{n-1}
\end{mathpar}

\subsubsection{Structural congruence}

\paragraph{Free and bound names and alpha-equivalence.} At the
core of structural equivalence is alpha-equivalence which identifies
process that are the same up to a change of variable. Formally, we
recognize the distinction between free and bound names. The free names
of a process, $\freenames{P}$, may be calculated recursively as
follows:

\begin{mathpar}
\freenames{\pzero} := \emptyset
  \and \\
  \freenames{x?(y).P} := \{ x \} \cup (\freenames{P} \setminus \{ y \})
  \and 
  \freenames{x!\langle P \rangle} := \{ x \} \cup \{ P \} 
  \and \\
  \freenames{P|Q} := \freenames{P} \cup \freenames{Q}
  \and \\
  \freenames{@{x}} := \{ x \}
\end{mathpar}

$\pi$
$\quotep{\pi}$

$\freenames{-} : \pi \to \mathcal{P}(\quotep{\pi})$

\begin{eqnarray*}
  \freenames{\pzero} & := & \emptyset \\
  \freenames{x?(y).P} & := & \{ x \} \cup (\freenames{P} \setminus \{ y \}) \\
  \freenames{x!\langle P \rangle} & := & \{ x \} \cup \{ P \} \\
  \freenames{P|Q} & := & \freenames{P} \cup \freenames{Q} \\
  \freenames{\dropn{x}} & := & \{ x \}
\end{eqnarray*}

The bound names of a process, $\boundnames{P}$, are those names occurring in $P$
that are not free. For example, in $x?(y).0$, the name $x$ is free, while $y$ is bound.

\begin{mathpar}
  \inferrule* [lab=monoidal-laws] {} { P|Q \equiv Q|P \and P|0 \equiv P \and P|(Q|R) \equiv (P|Q)|R }
\end{mathpar}

\begin{mathpar}
  \inferrule* [lab=alpha-equivalence] {} { (x)P \equiv (y)P\{y/x\} \and y \not\in \freenames{P} }
\end{mathpar}

\begin{definition}
Then two processes, $P,Q$, are alpha-equivalent if $P = Q\{\vec{y}/\vec{x}\}$ for
some $\vec{x} \in \boundnames{Q},\vec{y} \in \boundnames{P}$, where $Q\{\vec{y}/\vec{x}\}$
denotes the capture-avoiding substitution of $\vec{y}$ for $\vec{x}$ in $Q$.
\end{definition}

\begin{definition}
  The {\em structural congruence} \cite{SangiorgiWalker} , $\equiv$,
  between processes is the least congruence containing
  alpha-equivalence, satisfying the abelian monoid laws
  (associativity, commutativity and $\pzero$ as identity) for parallel
  composition $|$ and for summation $+$.
\end{definition}

\subsection{Name equivalence}

We take name equivalence, written $\nameeq$, to be the smallest
equivalence relation generated by the following rules.

\begin{mathpar}
\inferrule*[lab=Quote-drop]
{ }
{ \quotep{@{x}} \nameeq x }

\inferrule*[lab=Struct-equiv]
{ P \scong Q }
{ \quotep{P} \nameeq \quotep{Q} }
\end{mathpar}

The astute reader will have noticed that the mutual recursion of names
and processes imposes a mutual recursion on alpha-equivalence and
structural equivalence via name-equivalence. Fortunately, all of this
works out pleasantly and we may calculate in the natural way, free of
concern. The reader interested in the details is referred to the
appendix \ref{appendix:rho_details}.

\subsection{Substitution}

We use $\Proc$ for the set of processes, $\QProc$ for the set of
names, and $\id{\{}\vec{y} / \vec{x} \id{\}}$ to denote partial maps,
$s : \QProc \rightarrow \QProc$. A map, $s$ lifts, uniquely, to a map
on process terms, $\widehat{s} : \Proc \rightarrow \Proc$ by the
following equations.

\begin{mathpar}
  (0) \psubstp{Q}{P} := 0 \\
  (R \juxtap S) \psubstp{Q}{P}
  :=    
  (R)\psubstp{Q}{P} \juxtap (S) \psubstp{Q}{P} \\
  (x?(y).R) \psubstp{Q}{P}    
  :=    
  (x)\substp{Q}{P} (z)\concat( (R \psubstn{z}{y}) \psubstp{Q}{P} ) \\
  (\lift{x}{R}) \psubstp{Q}{P}  
  :=
  \lift{(x)\substp{Q}{P}}{ R \psubstp{Q}{P} } \\
%   (\dropn{x})  \psubstp{Q}{P}       
%   := 
%   \left\{ 
%     \begin{array}{ccc} 
%       \dropn{\quotep{Q}} & & x \nameeq \quotep{P} \\
%       \dropn{x} & & otherwise \\
%     \end{array}
%   \right. 
  (\dropn{x})  \psubstp{Q}{P}       
  := 
  \left\{ 
    \begin{array}{ccc} 
      Q & & x \nameeq \quotep{P} \\
      \dropn{x} & & otherwise \\
    \end{array}
  \right.
\end{mathpar}
 

where

\begin{eqnarray}
  (x)\id{\{} \lpquote Q \rpquote / \lpquote P \rpquote \id{\}}            = 
  \left\{ 
    \begin{array}{ccc}
      \lpquote Q \rpquote & & x \nameeq \lpquote P \rpquote \\
      x & & otherwise \\
    \end{array}
  \right. \nonumber
\end{eqnarray}

and $z$ is chosen distinct from $\quotep{P}$, $\quotep{Q}$, the free
names in $Q$, and all the names in $R$. Our $\alpha$-equivalence will
be built in the standard way from this substitution.

\begin{remark}\label{rem:no_self_referential_names}
  One consequence of these definitions is that $\forall P. \quotep{P}
  \not\in \freenames{P}$.
\end{remark}

\subsection{ Dynamic quote: an example }

Anticipating something of what's to come, consider applying the
substitution, $\widehat{\id{\{}u / z \id{\}}}$, to the following pair
of processes, $\lift{w}{y!(z)}$ and $w[ \lpquote y!(z) \rpquote ]$.

\begin{eqnarray}
	\lift{w}{y!(z)}\widehat{\id{\{}u / z \id{\}}}
		& = &
		\lift{w}{y!(u)} \nonumber\\
	w[ \lpquote y!(z) \rpquote ] \widehat{ \id{\{}u / z \id{\}} }
		& = &
		w[ \lpquote y!(z) \rpquote ] \nonumber
\end{eqnarray}

Because the body of the process between quotes is impervious to
substitution, we get radically different answers. In fact, by
examining the first process in an input context,
e.g. $x?(z).\lift{w}{y!(z)}$, we see that the process under the lift
operator may be shaped by prefixed inputs binding a name inside it. In
this sense, the lift operator will be seen as a way to dynamically
construct processes before reifying them as names.

Finally equipped with these standard features we can present the
dynamics of the calculus.

\subsubsection{Operational semantics} 

Finally, we introduce the computational dynamics. What marks these
algebras as distinct from other more traditionally studied algebraic
structures, e.g. vector spaces or polynomial rings, is the manner in
which dynamics is captured. In traditional structures, dynamics is typically
expressed through morphisms between such structures, as in linear maps
between vector spaces or morphisms between rings. In algebras
associated with the semantics of computation, the dynamics is
expressed as part of the algebraic structure itself, through a
reduction reduction relation typically denoted by $\red$. Below, we
give a recursive presentation of this relation for the calculus used
in the encoding.

$\red \subseteq \pi \times \pi$
$\red : \pi \to \mathcal{P}(\pi)$

\begin{mathpar}
  \inferrule* [lab=Comm] { \textsf{match}( x_{src}, x_{trgt} ) } { x_{trgt}?(y)P \; | \; x_{src}!\langle {Q} \rangle \red P\{\quotep{Q}/y}\} }
  \and \\
  \inferrule* [lab=Par] {{P} \red {P}'} {{{P} | {Q}} \red {{P}' | {Q}}}
  \and
  \inferrule* [lab=Equiv]{{{P} \scong {P}'} \andalso {{P}' \red {Q}'} \andalso {{Q}' \scong {Q}}}{{P} \red {Q}}
\end{mathpar}

\begin{eqnarray*}
  match_{\equiv} (\quotep{P},\quotep{Q}) & := & P \equiv Q \\
  match_{\dagger}(\quotep{P},\quotep{Q}) & := & \forall R. P|Q \red^{*} R => R \red^{*} 0 \\
  match_{K}(\quotep{P},\quotep{Q}) & := & K \mbox{ for some context } K
\end{eqnarray*}

$u?(x)P | u!\langle Q \rangle \red P\{\quotep{Q}/x\}$

%We write $\wred$ for $\red^*$, and $P\red$ if $\exists Q $ such that $ P \red Q$.
We write $P\red$ if $\exists Q $ such that $ P \red Q$ and $P\not\red$, otherwise.

\section{Replication}

As mentioned before, it is known that replication (and hence
recursion) can be implemented in a higher-order process algebra
\cite{SangiorgiWalker}. As our first example of calculation with the
machinery thus far presented we give the construction explicitly in
the {\rhoc}.

\begin{eqnarray}
	D_{x} & := & \prefix{x}{y}{(\binpar{\outputp{x}{y}}{@{y}})} \nonumber\\
	\bangp_{x}{P} & := & \binpar{{x}!\langle{\binpar{D_{x}}{P}}\rangle}{D_{x}} \nonumber
\end{eqnarray}

\begin{eqnarray}
	\bangp_{x}{P} & & \nonumber\\
	=
	& {x}!\langle{(\prefix{x}{y}{(\outputp{x}{y} | @{y})) | P}}\rangle 
	      | \prefix{x}{y}{(\outputp{x}{y} | @{y})} & \nonumber\\
	\red
	& (\outputp{x}{y} | @{y})\substn{\quotep{(\prefix{x}{y}{(@{y} | \outputp{x}{y})) | P}}}{y} & \nonumber\\
	=
	& \outputp{x}{\quotep{(\prefix{x}{y}{(\outputp{x}{y} | @{y})) | P}}}
	  | {(\prefix{x}{y}{(\outputp{x}{y} | @{y})) | P}} & \nonumber\\
	\red
	& \ldots & \nonumber\\
	\red^*
	& P | P | \ldots & \nonumber
\end{eqnarray}

Of course, this encoding, as an implementation, runs away, unfolding
$\bangp{P}$ eagerly. A lazier and more implementable replication
operator, restricted to input-guarded processes, may be obtained as follows.

\begin{eqnarray}
\bangp{\prefix{u}{v}{P}} 
	:= 
	\binpar{\lift{x}{\prefix{u}{v}{(\binpar{D(x)}{P})}}}{D(x)} \nonumber
\end{eqnarray}

\begin{remark}
  Note that the lazier definition still does not deal with summation
  or mixed summation (i.e. sums over input and output). The reader is
  invited to construct definitions of replication that deal with these
  features. 

  Further, the definitions are parameterized in a name, $x$. Can you,
  gentle reader, make a definition that eliminates this parameter and
  guarantees no accidental interaction between the replication
  machinery and the process being replicated -- i.e. no accidental
  sharing of names used by the process to get its work done and the
  name(s) used by the replication to effect copying. This latter
  revision of the definition of replication is crucial to obtaining
  the expected identity $!!P \sim !P$.
\end{remark}

\begin{remark}\label{rem:paradoxical_combinator}
  The reader familiar with the lambda calculus will have noticed the
  similarity between $D$ and the paradoxical combinator.

  [Ed. note: the existence of this seems to suggest we have to be more
  restrictive on the set of processes and names we admit if we are to
  support no-cloning.]
\end{remark}

\subsubsection{Bisimulation}

The computational dynamics gives rise to another kind of equivalence,
the equivalence of computational behavior. As previously mentioned
this is typically captured \emph{via} some form of bisimulation.

% The notion we use in this paper is weak barbed bisimulation
% \cite{milner91polyadicpi}.

The notion we use in this paper is derived from weak barbed
bisimulation \cite{milner91polyadicpi}. 

\begin{definition}
An \emph{observation relation}, $\downarrow_{\mathcal N}$, over a set
of names, $\mathcal N$, is the smallest relation satisfying the rules
below.

\infrule[Out-barb]{y \in {\mathcal N}, \; x \nameeq y}
		  {\outputp{x}{v} \downarrow_{\mathcal N} x}
\infrule[Par-barb]{\mbox{$P\downarrow_{\mathcal N} x$ or $Q\downarrow_{\mathcal N} x$}}
		  {\binpar{P}{Q} \downarrow_{\mathcal N} x}

We write $P \Downarrow_{\mathcal N} x$ if there is $Q$ such that 
$P \wred Q$ and $Q \downarrow_{\mathcal N} x$.
\end{definition}

\begin{definition}
%\label{def.bbisim}
An  ${\mathcal N}$-\emph{barbed bisimulation} over a set of names, ${\mathcal N}$, is a symmetric binary relation 
${\mathcal S}_{\mathcal N}$ between agents such that $P\rel{S}_{\mathcal N}Q$ implies:
\begin{enumerate}
\item If $P \red P'$ then $Q \wred Q'$ and $P'\rel{S}_{\mathcal N} Q'$.
\item If $P\downarrow_{\mathcal N} x$, then $Q\Downarrow_{\mathcal N} x$.
\end{enumerate}
$P$ is ${\mathcal N}$-barbed bisimilar to $Q$, written
$P \wbbisim_{\mathcal N} Q$, if $P \rel{S}_{\mathcal N} Q$ for some ${\mathcal N}$-barbed bisimulation ${\mathcal S}_{\mathcal N}$.
\end{definition}

$\mathcal{R} \subseteq \pi \times \pi$

$P \mathcal{R} Q => \forall P'. P \red P' \Rightarrow \exists Q'. Q \red Q', P' \mathcal{R} Q'$

$P \vdash x \Rightarrow Q \vdash x$

\begin{mathpar}
  \inferrule*[lab=Out-barb]{x \nameeq y}{{y}!\langle{Q}\rangle \vdash x}
  \and
  \inferrule*[lab=Par-barb]{\mbox{$P\vdash x$ or $Q\vdash x$}}{\binpar{P}{Q} \vdash x}
\end{mathpar}

\subsubsection{Contexts}

One of the principle advantages of computational calculi like the
$\pi$-calculus is a well-defined notion of context,
contextual-equivalence and a correlation between
contextual-equivalence and notions of bisimulation. The notion of
context allows the decomposition of a process into (sub-)process and
its syntactic environment, its context. Thus, a context may be
thought of as a process with a ``hole'' (written $\Box$) in it. The
application of a context $M$ to a process $P$, written $M[P]$, is
tantamount to filling the hole in $M$ with $P$. In this paper we do
not need the full weight of this theory, but do make use of the notion
of context in the proof the main theorem. 

\begin{mathpar}
  \inferrule* [lab=summation] {} {{M_{M},M_{N}} \bc \Box \;|\; x.M_{A} \;|\; M_{M}+M_{N}}
  \and
  \inferrule* [lab=agent] {} {{M_{A}} \bc (\vec{x})M_{P} \;| \; \clift{P_0,\ldots,M_{P},\ldots,P_N}}
  \and \\
  \inferrule* [lab=process] {} {{M_{P}} \bc M_{N} \;| \;P|M_{P} }
\end{mathpar} 

\begin{mathpar}
  \inferrule* [lab=sychronization] {} {M_{N} \bc \Box \;|\; x?M_{F} \;|\; x!M_{C}}
  \and
  \inferrule* [lab=abstraction] {} {{M_{F}} \bc (x)M_{P} }
  \and
  \inferrule* [lab=concretion] {} {{M_{C}} \bc \langle M_{P} \rangle }
  \and \\
  \inferrule* [lab=process] {} {{M_{P}} \bc M_{N} \;| \;P|M_{P} }
\end{mathpar}

\begin{definition}[contextual application] Given a context $M$, and
  process $P$, we define the \emph{contextual application}, $M[P] :=
  M\{P/\Box\}$. That is, the contextual application of M to P is the
  substitution of $P$ for $\Box$ in $M$.
\end{definition}

$\meaningof{-} : L \to \mathcal{P}(\pi)$

\begin{mathpar}
  \inferrule* [lab=collection] {} {\meaningof{true} = \pi, \and \meaningof{~E} = \pi \setminus \meaningof{E}, \and \meaningof{E_{1} \& E_{2}} = \meaningof{E_{1}} \cap \meaningof{E_{2}}}
\end{mathpar}

\begin{mathpar}
  \inferrule* [lab=structure] {} {\meaningof{0} = \{ P \in \pi | P \equiv 0 \}, \and \\ \meaningof{E_1 | E_2} = \{ P \in \pi | P \equiv P_{1} | P_{2}, P_{1} \in \meaningof{E_{1}}, P_{2} \in \meaningof{E_2}\} }
\end{mathpar}

\begin{mathpar}
 \inferrule* [lab=behavior] {} {\meaningof{\langle a?b \rangle E} = \{ P \in \pi | P \equiv Q | u?(y)P', \\ \and \\\\ \and \\ \;\;\; u \in \meaningof{a}, \forall z.P'\{z/y\} \in \meaningof{E\{z/b\}}\}, \and \\ \meaningof{a!E} = \{ P \in \pi | P \equiv Q | x!\langle P' \rangle, x \in \meaningof{a} P' \in \meaningof{E}\} }
\end{mathpar}

\begin{mathpar}
 \inferrule* [lab=nominal] {} {\meaningof{\quotep{E}} = \{ \quotep{P} \in \quotep{\pi} | P \in \meaningof{E} \}, \and \meaningof{\quotep{P}} = \{ \quotep{Q} \in \quotep{\pi} | P \equiv Q \} \and \\ \meaningof{@\quotep{E}} = \{ P \in \pi | P \equiv @x, x \in \meaningof{E} \}}
\end{mathpar}

\begin{eqnarray*}
  \\
  \meaningof{-} : TS \to ST
\end{eqnarray*}

\begin{eqnarray*}
  \\
  L : TS \to ST
\end{eqnarray*}

\begin{eqnarray*}
  \\
  P \models E \iff P \in \meaningof{E}
\end{eqnarray*}

\begin{eqnarray*}
  P \approx_{L} Q \iff \forall E \in L. P \models E \iff Q \models E
\end{eqnarray*}

\begin{eqnarray*}
  P \approx_{K} Q
\end{eqnarray*}

\begin{eqnarray*}
  P \approx Q
\end{eqnarray*}

$\approx_{K} = \approx = \approx_{L}$

\subsubsection{Contextual duality}

Note that contexts extend the quotation operation to a family of
operations from processes to names. Given a context, $M$, we can
define a \emph{nominal context}, $\quotep{M}$ by $\quotep{M}[P] :=
\quotep{M[P]}$. To foreshadow what is to come we observe that these
operations enjoy a duality with processes very much like the duality
between vectors and maps from vectors to scalars.

Further, because the calculus is essentially higher-order, we have a
correspondence between contexts and processes. More specifically,
given a name $x$ and a context $M$ we can construct $M^{*}_{x}$ such
that 

\begin{mathpar}
  M^{*}_{x} | \lift{x}{P} \red M[P]
\end{mathpar}

namely,

\begin{mathpar}
  M^{*}_{x} := x?(u).M[\dropn{u}]
\end{mathpar}

The dependence of $M^{*}_{x}$ on a name makes it an abstraction, 

\begin{mathpar}
  M^{*} := (x)x?(u).M[\dropn{u}]
\end{mathpar}

\subsection{Additional notation}

It will sometimes be convenient to denote the process a name
quotes. We already have the notation $x = \quotep{P}$, but it will be
convenient to introduce an alternate notation, $\procn{x}$, when we
want to emphasize the connection to the use of the name. Note that, by
virtue of name equivalence, $\quotep{\procn{x}} \nameeq x$; so, the
notation is consistent with previous definitions.

Further, because names have structure it is possible to effect
substitutions on the basis of that structure. This means we need to
upgrade our notation for substitutions, which we accomplish by
adapting comprehension notation. Thus,

\begin{mathpar}
  P\{ y / x : x \in S \}
\end{mathpar}

is interpreted to mean the process derived from P by replacing (in a
capture-avoiding manner) each occurrence of $x$ in $S$ by $y$. For example,

\begin{mathpar}
  P\{ \quotep{\procn{x}|\procn{x}} / x : x \in \freenames{P} \}
\end{mathpar}

will replace each (occurrence) of a free name $x$ in $P$ by
$\quotep{\procn{x}|\procn{x}}$.

Also, we will avail ourselves of the notation $x^{L}$ and $x^{R}$ to
denote injections of a name into disjoint copies of the name
space. There are numerous ways to accomplish this. One example can be
found in \cite{MeredithR05}. This notation overloads to vectors of
names: $\vec{x}^{\pi} := (x_{i}^{\pi} \; : \; 0 \leq i < |\vec{x}| )$ where $\pi \in \{L,R\}$.

We also use $P^{\Box} := P|\Box$.

In \cite{MeredithR05} an interpretation of the new operator is
given. It turns out that there are several possible interpretations
all enjoying the requisite algebraic properties of the operator (see
\cite{milner91polyadicpi}). We will therefore make liberal use of
$(\nu\; \vec{x})P$.

% subsection the_syntax_and_semantics_of_the_notation_system (end)   

\input{qm2pi.qmops} 

\input{qm2pi.sterngerlach} 

\input{qm2pi.metric} 

% section concurrent_process_calculi (end)

%\input{qm2pi.proofsketch}

% section proof sketch (end)

%\input{qm2pi.slviaknots} 

% section spatial logic via knots (end)

\input{qm2pi.conclusion}

% section conclusion (end)

%\input{qm2pi.dtcodes} 

% section wiring algorithm (end)

\input{qm2pi.ack} 

% section acknowledgments (end)

\newpage


\bibliographystyle{plain}   
\bibliography{../../biblios/main.bib}

\input{qm2pi.rhodetails}

\end{document}

 

% section acknowledgments (end)

\newpage


\bibliographystyle{plain}   
\bibliography{../../biblios/main.bib}

\documentclass[12pt]{llncs}
%\documentclass{jktr}

\usepackage[pdftex]{hyperref}                   
\usepackage {listings}
\usepackage {mathpartir}
\usepackage{bcprules}
%\usepackage{listings}
                       
\usepackage{graphicx} 
%\usepackage[margins=2.5cm,nohead,nofoot]{geometry}
%\usepackage{geometry}
\usepackage{amsfonts}
\usepackage{amstext}
\usepackage{latexsym}
\usepackage{amssymb}
\usepackage{color}


%\include{myPreamble}
\include{qm2pi.local} 

%\ifpdf
%\usepackage[pdftex]{graphicx}
%\else
%\usepackage{graphicx}
%\fi

 % \ifpdf
%  \usepackage{pdfsync}
%  \if


%\title{Brief Article}
%\author{David F. Snyder}
%\author{L.G. Meredith}

%\address{Dept. of Math., Texas State University--San Marcos, San Marcos, TX 78666}
       
\pagestyle{empty}


\begin{document}

\lstset{language=[Objective]Caml,frame=shadowbox}

\input{qm2pi.front}

% section front matter (end)

\input{qm2pi.intro} 
 
% section introduction (end)

% \input{qm2pi.knotations} 

% section notation (end)

\input{qm2pi.process.calculi} 

% section concurrent_process_calculi_and_spatial_logics_ (end)
    
%\input{qm2pi.knots2pi} 

%\input{qm2pi.trefoil} 

%\input{qm2pi.mainthm} 

% subsection basic_interpretation (end)

%\input{qm2pi.rho.presentation} 
\subsection{The syntax and semantics of the notation system}\label{sub:the_syntax_and_semantics_of_the_notation_system} % (fold)

We now summarize a technical presentation of the calculus that
embodies our theory of dynamics. The typical presentation of such a
calculus follows the style of giving generators and relations on
them. The grammar, below, describing term constructors, freely
generates the set of processes, $\Proc$. This set is then quotiented
by a relation known as structural congruence and it is over this set
that the notion of dynamics is expressed. This presentation is
essentially that of \cite{MeredithR05} with the addition of
polyadicity and summation. For readability we have relegated some of
the technical subtleties to an appendix.

\subsubsection{Process grammar}\label{subsub:process_grammar}

\begin{mathpar}
  \inferrule* [lab=synchronization] {} {{M} \bc \pzero \;|\; x?F \;|\; x!C }
  \and
  \inferrule* [lab=abstraction] {} {{F} \bc (x)P}
  \and
  \inferrule* [lab=concretion] {} {{C} \bc \langle Q \rangle}
  \and
  \inferrule* [lab=process] {} {{P,Q} \bc M \;| \;P|Q \;|\; @{x}}
  \and
  \inferrule* [lab=name] {} {{x} \bc \quotep{P}}
\end{mathpar} 

Note that $\vec{x}$ (resp. $\vec{P}$) denotes a vector of names
(resp. processes) of length $|\vec{x}|$ (resp. $|\vec{P}|$). We adopt
the following useful abbreviations.

\begin{mathpar}
   x?(\vec{y}).P := x.(\vec{y})P \and  x\clift{\vec{P}} := x.\clift{\vec{P}}
   \and x!(y) := \lift{x}{\dropn{y}}
   \and \Pi_{i=0}^{n-1}P_i := P_0 | \ldots | P_{n-1}
\end{mathpar}

\subsubsection{Structural congruence}

\paragraph{Free and bound names and alpha-equivalence.} At the
core of structural equivalence is alpha-equivalence which identifies
process that are the same up to a change of variable. Formally, we
recognize the distinction between free and bound names. The free names
of a process, $\freenames{P}$, may be calculated recursively as
follows:

\begin{mathpar}
\freenames{\pzero} := \emptyset
  \and \\
  \freenames{x?(y).P} := \{ x \} \cup (\freenames{P} \setminus \{ y \})
  \and 
  \freenames{x!\langle P \rangle} := \{ x \} \cup \{ P \} 
  \and \\
  \freenames{P|Q} := \freenames{P} \cup \freenames{Q}
  \and \\
  \freenames{@{x}} := \{ x \}
\end{mathpar}

$\pi$
$\quotep{\pi}$

$\freenames{-} : \pi \to \mathcal{P}(\quotep{\pi})$

\begin{eqnarray*}
  \freenames{\pzero} & := & \emptyset \\
  \freenames{x?(y).P} & := & \{ x \} \cup (\freenames{P} \setminus \{ y \}) \\
  \freenames{x!\langle P \rangle} & := & \{ x \} \cup \{ P \} \\
  \freenames{P|Q} & := & \freenames{P} \cup \freenames{Q} \\
  \freenames{\dropn{x}} & := & \{ x \}
\end{eqnarray*}

The bound names of a process, $\boundnames{P}$, are those names occurring in $P$
that are not free. For example, in $x?(y).0$, the name $x$ is free, while $y$ is bound.

\begin{mathpar}
  \inferrule* [lab=monoidal-laws] {} { P|Q \equiv Q|P \and P|0 \equiv P \and P|(Q|R) \equiv (P|Q)|R }
\end{mathpar}

\begin{mathpar}
  \inferrule* [lab=alpha-equivalence] {} { (x)P \equiv (y)P\{y/x\} \and y \not\in \freenames{P} }
\end{mathpar}

\begin{definition}
Then two processes, $P,Q$, are alpha-equivalent if $P = Q\{\vec{y}/\vec{x}\}$ for
some $\vec{x} \in \boundnames{Q},\vec{y} \in \boundnames{P}$, where $Q\{\vec{y}/\vec{x}\}$
denotes the capture-avoiding substitution of $\vec{y}$ for $\vec{x}$ in $Q$.
\end{definition}

\begin{definition}
  The {\em structural congruence} \cite{SangiorgiWalker} , $\equiv$,
  between processes is the least congruence containing
  alpha-equivalence, satisfying the abelian monoid laws
  (associativity, commutativity and $\pzero$ as identity) for parallel
  composition $|$ and for summation $+$.
\end{definition}

\subsection{Name equivalence}

We take name equivalence, written $\nameeq$, to be the smallest
equivalence relation generated by the following rules.

\begin{mathpar}
\inferrule*[lab=Quote-drop]
{ }
{ \quotep{@{x}} \nameeq x }

\inferrule*[lab=Struct-equiv]
{ P \scong Q }
{ \quotep{P} \nameeq \quotep{Q} }
\end{mathpar}

The astute reader will have noticed that the mutual recursion of names
and processes imposes a mutual recursion on alpha-equivalence and
structural equivalence via name-equivalence. Fortunately, all of this
works out pleasantly and we may calculate in the natural way, free of
concern. The reader interested in the details is referred to the
appendix \ref{appendix:rho_details}.

\subsection{Substitution}

We use $\Proc$ for the set of processes, $\QProc$ for the set of
names, and $\id{\{}\vec{y} / \vec{x} \id{\}}$ to denote partial maps,
$s : \QProc \rightarrow \QProc$. A map, $s$ lifts, uniquely, to a map
on process terms, $\widehat{s} : \Proc \rightarrow \Proc$ by the
following equations.

\begin{mathpar}
  (0) \psubstp{Q}{P} := 0 \\
  (R \juxtap S) \psubstp{Q}{P}
  :=    
  (R)\psubstp{Q}{P} \juxtap (S) \psubstp{Q}{P} \\
  (x?(y).R) \psubstp{Q}{P}    
  :=    
  (x)\substp{Q}{P} (z)\concat( (R \psubstn{z}{y}) \psubstp{Q}{P} ) \\
  (\lift{x}{R}) \psubstp{Q}{P}  
  :=
  \lift{(x)\substp{Q}{P}}{ R \psubstp{Q}{P} } \\
%   (\dropn{x})  \psubstp{Q}{P}       
%   := 
%   \left\{ 
%     \begin{array}{ccc} 
%       \dropn{\quotep{Q}} & & x \nameeq \quotep{P} \\
%       \dropn{x} & & otherwise \\
%     \end{array}
%   \right. 
  (\dropn{x})  \psubstp{Q}{P}       
  := 
  \left\{ 
    \begin{array}{ccc} 
      Q & & x \nameeq \quotep{P} \\
      \dropn{x} & & otherwise \\
    \end{array}
  \right.
\end{mathpar}
 

where

\begin{eqnarray}
  (x)\id{\{} \lpquote Q \rpquote / \lpquote P \rpquote \id{\}}            = 
  \left\{ 
    \begin{array}{ccc}
      \lpquote Q \rpquote & & x \nameeq \lpquote P \rpquote \\
      x & & otherwise \\
    \end{array}
  \right. \nonumber
\end{eqnarray}

and $z$ is chosen distinct from $\quotep{P}$, $\quotep{Q}$, the free
names in $Q$, and all the names in $R$. Our $\alpha$-equivalence will
be built in the standard way from this substitution.

\begin{remark}\label{rem:no_self_referential_names}
  One consequence of these definitions is that $\forall P. \quotep{P}
  \not\in \freenames{P}$.
\end{remark}

\subsection{ Dynamic quote: an example }

Anticipating something of what's to come, consider applying the
substitution, $\widehat{\id{\{}u / z \id{\}}}$, to the following pair
of processes, $\lift{w}{y!(z)}$ and $w[ \lpquote y!(z) \rpquote ]$.

\begin{eqnarray}
	\lift{w}{y!(z)}\widehat{\id{\{}u / z \id{\}}}
		& = &
		\lift{w}{y!(u)} \nonumber\\
	w[ \lpquote y!(z) \rpquote ] \widehat{ \id{\{}u / z \id{\}} }
		& = &
		w[ \lpquote y!(z) \rpquote ] \nonumber
\end{eqnarray}

Because the body of the process between quotes is impervious to
substitution, we get radically different answers. In fact, by
examining the first process in an input context,
e.g. $x?(z).\lift{w}{y!(z)}$, we see that the process under the lift
operator may be shaped by prefixed inputs binding a name inside it. In
this sense, the lift operator will be seen as a way to dynamically
construct processes before reifying them as names.

Finally equipped with these standard features we can present the
dynamics of the calculus.

\subsubsection{Operational semantics} 

Finally, we introduce the computational dynamics. What marks these
algebras as distinct from other more traditionally studied algebraic
structures, e.g. vector spaces or polynomial rings, is the manner in
which dynamics is captured. In traditional structures, dynamics is typically
expressed through morphisms between such structures, as in linear maps
between vector spaces or morphisms between rings. In algebras
associated with the semantics of computation, the dynamics is
expressed as part of the algebraic structure itself, through a
reduction reduction relation typically denoted by $\red$. Below, we
give a recursive presentation of this relation for the calculus used
in the encoding.

$\red \subseteq \pi \times \pi$
$\red : \pi \to \mathcal{P}(\pi)$

\begin{mathpar}
  \inferrule* [lab=Comm] { \textsf{match}( x_{src}, x_{trgt} ) } { x_{trgt}?(y)P \; | \; x_{src}!\langle {Q} \rangle \red P\{\quotep{Q}/y}\} }
  \and \\
  \inferrule* [lab=Par] {{P} \red {P}'} {{{P} | {Q}} \red {{P}' | {Q}}}
  \and
  \inferrule* [lab=Equiv]{{{P} \scong {P}'} \andalso {{P}' \red {Q}'} \andalso {{Q}' \scong {Q}}}{{P} \red {Q}}
\end{mathpar}

\begin{eqnarray*}
  match_{\equiv} (\quotep{P},\quotep{Q}) & := & P \equiv Q \\
  match_{\dagger}(\quotep{P},\quotep{Q}) & := & \forall R. P|Q \red^{*} R => R \red^{*} 0 \\
  match_{K}(\quotep{P},\quotep{Q}) & := & K \mbox{ for some context } K
\end{eqnarray*}

$u?(x)P | u!\langle Q \rangle \red P\{\quotep{Q}/x\}$

%We write $\wred$ for $\red^*$, and $P\red$ if $\exists Q $ such that $ P \red Q$.
We write $P\red$ if $\exists Q $ such that $ P \red Q$ and $P\not\red$, otherwise.

\section{Replication}

As mentioned before, it is known that replication (and hence
recursion) can be implemented in a higher-order process algebra
\cite{SangiorgiWalker}. As our first example of calculation with the
machinery thus far presented we give the construction explicitly in
the {\rhoc}.

\begin{eqnarray}
	D_{x} & := & \prefix{x}{y}{(\binpar{\outputp{x}{y}}{@{y}})} \nonumber\\
	\bangp_{x}{P} & := & \binpar{{x}!\langle{\binpar{D_{x}}{P}}\rangle}{D_{x}} \nonumber
\end{eqnarray}

\begin{eqnarray}
	\bangp_{x}{P} & & \nonumber\\
	=
	& {x}!\langle{(\prefix{x}{y}{(\outputp{x}{y} | @{y})) | P}}\rangle 
	      | \prefix{x}{y}{(\outputp{x}{y} | @{y})} & \nonumber\\
	\red
	& (\outputp{x}{y} | @{y})\substn{\quotep{(\prefix{x}{y}{(@{y} | \outputp{x}{y})) | P}}}{y} & \nonumber\\
	=
	& \outputp{x}{\quotep{(\prefix{x}{y}{(\outputp{x}{y} | @{y})) | P}}}
	  | {(\prefix{x}{y}{(\outputp{x}{y} | @{y})) | P}} & \nonumber\\
	\red
	& \ldots & \nonumber\\
	\red^*
	& P | P | \ldots & \nonumber
\end{eqnarray}

Of course, this encoding, as an implementation, runs away, unfolding
$\bangp{P}$ eagerly. A lazier and more implementable replication
operator, restricted to input-guarded processes, may be obtained as follows.

\begin{eqnarray}
\bangp{\prefix{u}{v}{P}} 
	:= 
	\binpar{\lift{x}{\prefix{u}{v}{(\binpar{D(x)}{P})}}}{D(x)} \nonumber
\end{eqnarray}

\begin{remark}
  Note that the lazier definition still does not deal with summation
  or mixed summation (i.e. sums over input and output). The reader is
  invited to construct definitions of replication that deal with these
  features. 

  Further, the definitions are parameterized in a name, $x$. Can you,
  gentle reader, make a definition that eliminates this parameter and
  guarantees no accidental interaction between the replication
  machinery and the process being replicated -- i.e. no accidental
  sharing of names used by the process to get its work done and the
  name(s) used by the replication to effect copying. This latter
  revision of the definition of replication is crucial to obtaining
  the expected identity $!!P \sim !P$.
\end{remark}

\begin{remark}\label{rem:paradoxical_combinator}
  The reader familiar with the lambda calculus will have noticed the
  similarity between $D$ and the paradoxical combinator.

  [Ed. note: the existence of this seems to suggest we have to be more
  restrictive on the set of processes and names we admit if we are to
  support no-cloning.]
\end{remark}

\subsubsection{Bisimulation}

The computational dynamics gives rise to another kind of equivalence,
the equivalence of computational behavior. As previously mentioned
this is typically captured \emph{via} some form of bisimulation.

% The notion we use in this paper is weak barbed bisimulation
% \cite{milner91polyadicpi}.

The notion we use in this paper is derived from weak barbed
bisimulation \cite{milner91polyadicpi}. 

\begin{definition}
An \emph{observation relation}, $\downarrow_{\mathcal N}$, over a set
of names, $\mathcal N$, is the smallest relation satisfying the rules
below.

\infrule[Out-barb]{y \in {\mathcal N}, \; x \nameeq y}
		  {\outputp{x}{v} \downarrow_{\mathcal N} x}
\infrule[Par-barb]{\mbox{$P\downarrow_{\mathcal N} x$ or $Q\downarrow_{\mathcal N} x$}}
		  {\binpar{P}{Q} \downarrow_{\mathcal N} x}

We write $P \Downarrow_{\mathcal N} x$ if there is $Q$ such that 
$P \wred Q$ and $Q \downarrow_{\mathcal N} x$.
\end{definition}

\begin{definition}
%\label{def.bbisim}
An  ${\mathcal N}$-\emph{barbed bisimulation} over a set of names, ${\mathcal N}$, is a symmetric binary relation 
${\mathcal S}_{\mathcal N}$ between agents such that $P\rel{S}_{\mathcal N}Q$ implies:
\begin{enumerate}
\item If $P \red P'$ then $Q \wred Q'$ and $P'\rel{S}_{\mathcal N} Q'$.
\item If $P\downarrow_{\mathcal N} x$, then $Q\Downarrow_{\mathcal N} x$.
\end{enumerate}
$P$ is ${\mathcal N}$-barbed bisimilar to $Q$, written
$P \wbbisim_{\mathcal N} Q$, if $P \rel{S}_{\mathcal N} Q$ for some ${\mathcal N}$-barbed bisimulation ${\mathcal S}_{\mathcal N}$.
\end{definition}

$\mathcal{R} \subseteq \pi \times \pi$

$P \mathcal{R} Q => \forall P'. P \red P' \Rightarrow \exists Q'. Q \red Q', P' \mathcal{R} Q'$

$P \vdash x \Rightarrow Q \vdash x$

\begin{mathpar}
  \inferrule*[lab=Out-barb]{x \nameeq y}{{y}!\langle{Q}\rangle \vdash x}
  \and
  \inferrule*[lab=Par-barb]{\mbox{$P\vdash x$ or $Q\vdash x$}}{\binpar{P}{Q} \vdash x}
\end{mathpar}

\subsubsection{Contexts}

One of the principle advantages of computational calculi like the
$\pi$-calculus is a well-defined notion of context,
contextual-equivalence and a correlation between
contextual-equivalence and notions of bisimulation. The notion of
context allows the decomposition of a process into (sub-)process and
its syntactic environment, its context. Thus, a context may be
thought of as a process with a ``hole'' (written $\Box$) in it. The
application of a context $M$ to a process $P$, written $M[P]$, is
tantamount to filling the hole in $M$ with $P$. In this paper we do
not need the full weight of this theory, but do make use of the notion
of context in the proof the main theorem. 

\begin{mathpar}
  \inferrule* [lab=summation] {} {{M_{M},M_{N}} \bc \Box \;|\; x.M_{A} \;|\; M_{M}+M_{N}}
  \and
  \inferrule* [lab=agent] {} {{M_{A}} \bc (\vec{x})M_{P} \;| \; \clift{P_0,\ldots,M_{P},\ldots,P_N}}
  \and \\
  \inferrule* [lab=process] {} {{M_{P}} \bc M_{N} \;| \;P|M_{P} }
\end{mathpar} 

\begin{mathpar}
  \inferrule* [lab=sychronization] {} {M_{N} \bc \Box \;|\; x?M_{F} \;|\; x!M_{C}}
  \and
  \inferrule* [lab=abstraction] {} {{M_{F}} \bc (x)M_{P} }
  \and
  \inferrule* [lab=concretion] {} {{M_{C}} \bc \langle M_{P} \rangle }
  \and \\
  \inferrule* [lab=process] {} {{M_{P}} \bc M_{N} \;| \;P|M_{P} }
\end{mathpar}

\begin{definition}[contextual application] Given a context $M$, and
  process $P$, we define the \emph{contextual application}, $M[P] :=
  M\{P/\Box\}$. That is, the contextual application of M to P is the
  substitution of $P$ for $\Box$ in $M$.
\end{definition}

$\meaningof{-} : L \to \mathcal{P}(\pi)$

\begin{mathpar}
  \inferrule* [lab=collection] {} {\meaningof{true} = \pi, \and \meaningof{~E} = \pi \setminus \meaningof{E}, \and \meaningof{E_{1} \& E_{2}} = \meaningof{E_{1}} \cap \meaningof{E_{2}}}
\end{mathpar}

\begin{mathpar}
  \inferrule* [lab=structure] {} {\meaningof{0} = \{ P \in \pi | P \equiv 0 \}, \and \\ \meaningof{E_1 | E_2} = \{ P \in \pi | P \equiv P_{1} | P_{2}, P_{1} \in \meaningof{E_{1}}, P_{2} \in \meaningof{E_2}\} }
\end{mathpar}

\begin{mathpar}
 \inferrule* [lab=behavior] {} {\meaningof{\langle a?b \rangle E} = \{ P \in \pi | P \equiv Q | u?(y)P', \\ \and \\\\ \and \\ \;\;\; u \in \meaningof{a}, \forall z.P'\{z/y\} \in \meaningof{E\{z/b\}}\}, \and \\ \meaningof{a!E} = \{ P \in \pi | P \equiv Q | x!\langle P' \rangle, x \in \meaningof{a} P' \in \meaningof{E}\} }
\end{mathpar}

\begin{mathpar}
 \inferrule* [lab=nominal] {} {\meaningof{\quotep{E}} = \{ \quotep{P} \in \quotep{\pi} | P \in \meaningof{E} \}, \and \meaningof{\quotep{P}} = \{ \quotep{Q} \in \quotep{\pi} | P \equiv Q \} \and \\ \meaningof{@\quotep{E}} = \{ P \in \pi | P \equiv @x, x \in \meaningof{E} \}}
\end{mathpar}

\begin{eqnarray*}
  \\
  \meaningof{-} : TS \to ST
\end{eqnarray*}

\begin{eqnarray*}
  \\
  L : TS \to ST
\end{eqnarray*}

\begin{eqnarray*}
  \\
  P \models E \iff P \in \meaningof{E}
\end{eqnarray*}

\begin{eqnarray*}
  P \approx_{L} Q \iff \forall E \in L. P \models E \iff Q \models E
\end{eqnarray*}

\begin{eqnarray*}
  P \approx_{K} Q
\end{eqnarray*}

\begin{eqnarray*}
  P \approx Q
\end{eqnarray*}

$\approx_{K} = \approx = \approx_{L}$

\subsubsection{Contextual duality}

Note that contexts extend the quotation operation to a family of
operations from processes to names. Given a context, $M$, we can
define a \emph{nominal context}, $\quotep{M}$ by $\quotep{M}[P] :=
\quotep{M[P]}$. To foreshadow what is to come we observe that these
operations enjoy a duality with processes very much like the duality
between vectors and maps from vectors to scalars.

Further, because the calculus is essentially higher-order, we have a
correspondence between contexts and processes. More specifically,
given a name $x$ and a context $M$ we can construct $M^{*}_{x}$ such
that 

\begin{mathpar}
  M^{*}_{x} | \lift{x}{P} \red M[P]
\end{mathpar}

namely,

\begin{mathpar}
  M^{*}_{x} := x?(u).M[\dropn{u}]
\end{mathpar}

The dependence of $M^{*}_{x}$ on a name makes it an abstraction, 

\begin{mathpar}
  M^{*} := (x)x?(u).M[\dropn{u}]
\end{mathpar}

\subsection{Additional notation}

It will sometimes be convenient to denote the process a name
quotes. We already have the notation $x = \quotep{P}$, but it will be
convenient to introduce an alternate notation, $\procn{x}$, when we
want to emphasize the connection to the use of the name. Note that, by
virtue of name equivalence, $\quotep{\procn{x}} \nameeq x$; so, the
notation is consistent with previous definitions.

Further, because names have structure it is possible to effect
substitutions on the basis of that structure. This means we need to
upgrade our notation for substitutions, which we accomplish by
adapting comprehension notation. Thus,

\begin{mathpar}
  P\{ y / x : x \in S \}
\end{mathpar}

is interpreted to mean the process derived from P by replacing (in a
capture-avoiding manner) each occurrence of $x$ in $S$ by $y$. For example,

\begin{mathpar}
  P\{ \quotep{\procn{x}|\procn{x}} / x : x \in \freenames{P} \}
\end{mathpar}

will replace each (occurrence) of a free name $x$ in $P$ by
$\quotep{\procn{x}|\procn{x}}$.

Also, we will avail ourselves of the notation $x^{L}$ and $x^{R}$ to
denote injections of a name into disjoint copies of the name
space. There are numerous ways to accomplish this. One example can be
found in \cite{MeredithR05}. This notation overloads to vectors of
names: $\vec{x}^{\pi} := (x_{i}^{\pi} \; : \; 0 \leq i < |\vec{x}| )$ where $\pi \in \{L,R\}$.

We also use $P^{\Box} := P|\Box$.

In \cite{MeredithR05} an interpretation of the new operator is
given. It turns out that there are several possible interpretations
all enjoying the requisite algebraic properties of the operator (see
\cite{milner91polyadicpi}). We will therefore make liberal use of
$(\nu\; \vec{x})P$.

% subsection the_syntax_and_semantics_of_the_notation_system (end)   

\input{qm2pi.qmops} 

\input{qm2pi.sterngerlach} 

\input{qm2pi.metric} 

% section concurrent_process_calculi (end)

%\input{qm2pi.proofsketch}

% section proof sketch (end)

%\input{qm2pi.slviaknots} 

% section spatial logic via knots (end)

\input{qm2pi.conclusion}

% section conclusion (end)

%\input{qm2pi.dtcodes} 

% section wiring algorithm (end)

\input{qm2pi.ack} 

% section acknowledgments (end)

\newpage


\bibliographystyle{plain}   
\bibliography{../../biblios/main.bib}

\input{qm2pi.rhodetails}

\end{document}



\end{document}



\end{document}



% section front matter (end)

\section{Introduction}\label{sec:introduction} % (fold)
In this draft of the material i am going to have to dispense with the
usual writing conventions adopted in papers on these topics. i'm going
to have adopt whatever tone i need at the time i'm writing up the
calculations. Sometimes this may be very conversational; others it may
be the barest mathematical grunts; others still it may be that i have
lifted text from one of my other papers because the exposition of some
point was better said there. i hope that my readers are not unduly put
out by this decision. i'm not doing this to flout convention or be
rebellious. i find these calculations very technically challenging. To
keep everything going technically, something has to give; i have to
let go of some cognitive burden. So, the academic writing style --
with all of its trade-offs in terms of facilitating technical
communication -- is what i'm letting go of. Perhaps subsequent drafts
can be tightened and polished, but for now, i'm going to speak as if
we were sitting together in a coffee shop with a laptop, wifi and a
pad of paper and a pencil.

So, here's what i have to say. We -- you and i, comfortably ensconced
in our coffee shop and well-equipped with our tools -- can realize and
carry out the calculations of quantum mechanics over a very different
formal theory of dynamics, a formal theory of dynamics that
corresponds to a theory of concurrent computation with
\emph{reflection}. It has the advantage that the underlying theory is
already `quantized', but supports analogues all of the continuuous
operations. Strikingly, this underlying theory has recently been
connected with a notion of metric that we can show, by calculating
together, coincides with the metric induced by the inner product.

There are a lot of reasons why you might be interested in seeing
calculations of this form. Here's why i'm interested. For the past
several centuries there has been no competitor to the ``Newtonian''
account of dynamics. As a result the predominant share of accounts of
dynamical systems and situations have had to be formulated in terms of
the Newtonian machinery. i view this as an intellectually dangerous
position to occupy. Everything, despite it's intrinsic shape, turns
into a nail to be hit with this hammer. Recently, however, the theory
of computation has matured to the point where we have candidates for
theories of dynamics that offer very different perspective on
reasoning about dynamical systems and situations. Testing these
candidates against very successful accounts of dynamical situations,
like quantum mechanics, is going to give us some sense of how mature
they are and some measure of the quality of these accounts of
dynamics.

\subsection{Summary of contributions and outline of paper}

So, we're going to develop an interpretation of the operations of
quantum mechanics normally interpreted by Hilbert spaces and
operators. We're going to do this over a theory of computation. Note
that this is very different than the usual quantum computation program
which develops notions of computation over quantum mechanics. Rather,
we are developing a story that aligns with Wheeler's slogan: It from
Bit. To do this we will first provide an account of the theory of
computation at play here. Then we will dive into a calculation-driven
interpretation of the operations of quantum mechanics.

The reason we take this approach is that -- until very recently --
there hasn't been an axiomatic account of quantum mechanics. As a
result there has been no sharp delineation of the mathematical theory
supporting interpretation of the physical theory and the physical
theory, itself. So, ambient features of the maths are free to be
exploited (or supressed) without a real accounting of their physical
relevance. There is no sharp statement ``here's the physical theory''
qua \emph{theory} and ``here's the mathematical interpretation''
enabling a judgment of how faithful the interpretation is -- apart
from experimental observation. When there is an axiomatic account we
can judge how well a given mathematical formalism supports an
interpretation of the axioms, independent of
experimentation. Likewise, we can judge how well we have captured our
physical evidence and experience with our axiomatics, independent of
any specific mathematical implementation, with accidental detail that
may or may not have physical significance. 

In lieu of a fully fleshed out and vetted axiomatic account of quantum
mechanics, interpreting the operational notions in service of modeling
physical systems will have to suffice. In other words, we are not in
the business of providing a model of Hilbert spaces and operators. We
are in the business of providing a model of quantum mechanics because
we are motivated by testing our notions of dynamics against physical
theory; and, the predictive calculations of the physical theory must
serve as the best formulation -- shy of a fully fleshed out axiomatic
account -- of the physical theory itself (as they have for scientific
theories since time immemorial). Put another way, despite a
whole-hearted commitment to an It-from-Bit ontology, we are firmly
aligned with the shut-up-and-calculate camp as the best way to obtain
results either from the physical perspective or as a quality assurance
measure of our fledgling theory of dynamics.

In detail, we present a reflective process calculus. Then we develop
intuitive correspondences between the notions available in this
calculus and the usual physical notions supporting quantum mechanical
calculations. Thus, 

\begin{table}[htp]
  \center{
    \fbox{
      \begin{tabular}{c|c}
        quantum mechanics & process calculus \\
        \hline
        scalar & name \\
        state vector & process \\
        dual & contextual duals \\
        matrix & formal sums of process-context-dual pairs \\
        orthogonality & process annihilation \\
        inner product & execution-formula + quoting
      \end{tabular}
    }
  }
  \caption{QM - process calculi correspondences}
\end{table}

Then we tighten up these intuitions to operational definitions. We
employ the Dirac notation as the best proxy we can find for an
abstract syntax of the quantum mechanical notions. The definitions we
develop put us in contact with equational constraints coming from the
theory that we demonstrate the definitions and calculations satisfy.

This puts us in a position to shut up and calculate for the
Stern-Gerlach experimental set up, showing how these predictive
calculations become calculations on processes in our theory of a
reflective process calculus.

Penultimately, we demonstrate that the notion of metric coming from
the inner product coincides with the notion of metric available from
the theory of bisimulation. This demonstration gives us the right to
think of space as arising from behavior. Finally, we consider where we
might go from the new vantage point we have obtained.

% section introduction (end) 
 
% section introduction (end)

% \documentclass[12pt]{llncs}
%\documentclass{jktr}

\usepackage[pdftex]{hyperref}                   
\usepackage {listings}
\usepackage {mathpartir}
\usepackage{bcprules}
%\usepackage{listings}
                       
\usepackage{graphicx} 
%\usepackage[margins=2.5cm,nohead,nofoot]{geometry}
%\usepackage{geometry}
\usepackage{amsfonts}
\usepackage{amstext}
\usepackage{latexsym}
\usepackage{amssymb}
\usepackage{color}


%\include{myPreamble}
\documentclass[12pt]{llncs}
%\documentclass{jktr}

\usepackage[pdftex]{hyperref}                   
\usepackage {listings}
\usepackage {mathpartir}
\usepackage{bcprules}
%\usepackage{listings}
                       
\usepackage{graphicx} 
%\usepackage[margins=2.5cm,nohead,nofoot]{geometry}
%\usepackage{geometry}
\usepackage{amsfonts}
\usepackage{amstext}
\usepackage{latexsym}
\usepackage{amssymb}
\usepackage{color}


%\include{myPreamble}
\documentclass[12pt]{llncs}
%\documentclass{jktr}

\usepackage[pdftex]{hyperref}                   
\usepackage {listings}
\usepackage {mathpartir}
\usepackage{bcprules}
%\usepackage{listings}
                       
\usepackage{graphicx} 
%\usepackage[margins=2.5cm,nohead,nofoot]{geometry}
%\usepackage{geometry}
\usepackage{amsfonts}
\usepackage{amstext}
\usepackage{latexsym}
\usepackage{amssymb}
\usepackage{color}


%\include{myPreamble}
\include{qm2pi.local} 

%\ifpdf
%\usepackage[pdftex]{graphicx}
%\else
%\usepackage{graphicx}
%\fi

 % \ifpdf
%  \usepackage{pdfsync}
%  \if


%\title{Brief Article}
%\author{David F. Snyder}
%\author{L.G. Meredith}

%\address{Dept. of Math., Texas State University--San Marcos, San Marcos, TX 78666}
       
\pagestyle{empty}


\begin{document}

\lstset{language=[Objective]Caml,frame=shadowbox}

\input{qm2pi.front}

% section front matter (end)

\input{qm2pi.intro} 
 
% section introduction (end)

% \input{qm2pi.knotations} 

% section notation (end)

\input{qm2pi.process.calculi} 

% section concurrent_process_calculi_and_spatial_logics_ (end)
    
%\input{qm2pi.knots2pi} 

%\input{qm2pi.trefoil} 

%\input{qm2pi.mainthm} 

% subsection basic_interpretation (end)

%\input{qm2pi.rho.presentation} 
\subsection{The syntax and semantics of the notation system}\label{sub:the_syntax_and_semantics_of_the_notation_system} % (fold)

We now summarize a technical presentation of the calculus that
embodies our theory of dynamics. The typical presentation of such a
calculus follows the style of giving generators and relations on
them. The grammar, below, describing term constructors, freely
generates the set of processes, $\Proc$. This set is then quotiented
by a relation known as structural congruence and it is over this set
that the notion of dynamics is expressed. This presentation is
essentially that of \cite{MeredithR05} with the addition of
polyadicity and summation. For readability we have relegated some of
the technical subtleties to an appendix.

\subsubsection{Process grammar}\label{subsub:process_grammar}

\begin{mathpar}
  \inferrule* [lab=synchronization] {} {{M} \bc \pzero \;|\; x?F \;|\; x!C }
  \and
  \inferrule* [lab=abstraction] {} {{F} \bc (x)P}
  \and
  \inferrule* [lab=concretion] {} {{C} \bc \langle Q \rangle}
  \and
  \inferrule* [lab=process] {} {{P,Q} \bc M \;| \;P|Q \;|\; @{x}}
  \and
  \inferrule* [lab=name] {} {{x} \bc \quotep{P}}
\end{mathpar} 

Note that $\vec{x}$ (resp. $\vec{P}$) denotes a vector of names
(resp. processes) of length $|\vec{x}|$ (resp. $|\vec{P}|$). We adopt
the following useful abbreviations.

\begin{mathpar}
   x?(\vec{y}).P := x.(\vec{y})P \and  x\clift{\vec{P}} := x.\clift{\vec{P}}
   \and x!(y) := \lift{x}{\dropn{y}}
   \and \Pi_{i=0}^{n-1}P_i := P_0 | \ldots | P_{n-1}
\end{mathpar}

\subsubsection{Structural congruence}

\paragraph{Free and bound names and alpha-equivalence.} At the
core of structural equivalence is alpha-equivalence which identifies
process that are the same up to a change of variable. Formally, we
recognize the distinction between free and bound names. The free names
of a process, $\freenames{P}$, may be calculated recursively as
follows:

\begin{mathpar}
\freenames{\pzero} := \emptyset
  \and \\
  \freenames{x?(y).P} := \{ x \} \cup (\freenames{P} \setminus \{ y \})
  \and 
  \freenames{x!\langle P \rangle} := \{ x \} \cup \{ P \} 
  \and \\
  \freenames{P|Q} := \freenames{P} \cup \freenames{Q}
  \and \\
  \freenames{@{x}} := \{ x \}
\end{mathpar}

$\pi$
$\quotep{\pi}$

$\freenames{-} : \pi \to \mathcal{P}(\quotep{\pi})$

\begin{eqnarray*}
  \freenames{\pzero} & := & \emptyset \\
  \freenames{x?(y).P} & := & \{ x \} \cup (\freenames{P} \setminus \{ y \}) \\
  \freenames{x!\langle P \rangle} & := & \{ x \} \cup \{ P \} \\
  \freenames{P|Q} & := & \freenames{P} \cup \freenames{Q} \\
  \freenames{\dropn{x}} & := & \{ x \}
\end{eqnarray*}

The bound names of a process, $\boundnames{P}$, are those names occurring in $P$
that are not free. For example, in $x?(y).0$, the name $x$ is free, while $y$ is bound.

\begin{mathpar}
  \inferrule* [lab=monoidal-laws] {} { P|Q \equiv Q|P \and P|0 \equiv P \and P|(Q|R) \equiv (P|Q)|R }
\end{mathpar}

\begin{mathpar}
  \inferrule* [lab=alpha-equivalence] {} { (x)P \equiv (y)P\{y/x\} \and y \not\in \freenames{P} }
\end{mathpar}

\begin{definition}
Then two processes, $P,Q$, are alpha-equivalent if $P = Q\{\vec{y}/\vec{x}\}$ for
some $\vec{x} \in \boundnames{Q},\vec{y} \in \boundnames{P}$, where $Q\{\vec{y}/\vec{x}\}$
denotes the capture-avoiding substitution of $\vec{y}$ for $\vec{x}$ in $Q$.
\end{definition}

\begin{definition}
  The {\em structural congruence} \cite{SangiorgiWalker} , $\equiv$,
  between processes is the least congruence containing
  alpha-equivalence, satisfying the abelian monoid laws
  (associativity, commutativity and $\pzero$ as identity) for parallel
  composition $|$ and for summation $+$.
\end{definition}

\subsection{Name equivalence}

We take name equivalence, written $\nameeq$, to be the smallest
equivalence relation generated by the following rules.

\begin{mathpar}
\inferrule*[lab=Quote-drop]
{ }
{ \quotep{@{x}} \nameeq x }

\inferrule*[lab=Struct-equiv]
{ P \scong Q }
{ \quotep{P} \nameeq \quotep{Q} }
\end{mathpar}

The astute reader will have noticed that the mutual recursion of names
and processes imposes a mutual recursion on alpha-equivalence and
structural equivalence via name-equivalence. Fortunately, all of this
works out pleasantly and we may calculate in the natural way, free of
concern. The reader interested in the details is referred to the
appendix \ref{appendix:rho_details}.

\subsection{Substitution}

We use $\Proc$ for the set of processes, $\QProc$ for the set of
names, and $\id{\{}\vec{y} / \vec{x} \id{\}}$ to denote partial maps,
$s : \QProc \rightarrow \QProc$. A map, $s$ lifts, uniquely, to a map
on process terms, $\widehat{s} : \Proc \rightarrow \Proc$ by the
following equations.

\begin{mathpar}
  (0) \psubstp{Q}{P} := 0 \\
  (R \juxtap S) \psubstp{Q}{P}
  :=    
  (R)\psubstp{Q}{P} \juxtap (S) \psubstp{Q}{P} \\
  (x?(y).R) \psubstp{Q}{P}    
  :=    
  (x)\substp{Q}{P} (z)\concat( (R \psubstn{z}{y}) \psubstp{Q}{P} ) \\
  (\lift{x}{R}) \psubstp{Q}{P}  
  :=
  \lift{(x)\substp{Q}{P}}{ R \psubstp{Q}{P} } \\
%   (\dropn{x})  \psubstp{Q}{P}       
%   := 
%   \left\{ 
%     \begin{array}{ccc} 
%       \dropn{\quotep{Q}} & & x \nameeq \quotep{P} \\
%       \dropn{x} & & otherwise \\
%     \end{array}
%   \right. 
  (\dropn{x})  \psubstp{Q}{P}       
  := 
  \left\{ 
    \begin{array}{ccc} 
      Q & & x \nameeq \quotep{P} \\
      \dropn{x} & & otherwise \\
    \end{array}
  \right.
\end{mathpar}
 

where

\begin{eqnarray}
  (x)\id{\{} \lpquote Q \rpquote / \lpquote P \rpquote \id{\}}            = 
  \left\{ 
    \begin{array}{ccc}
      \lpquote Q \rpquote & & x \nameeq \lpquote P \rpquote \\
      x & & otherwise \\
    \end{array}
  \right. \nonumber
\end{eqnarray}

and $z$ is chosen distinct from $\quotep{P}$, $\quotep{Q}$, the free
names in $Q$, and all the names in $R$. Our $\alpha$-equivalence will
be built in the standard way from this substitution.

\begin{remark}\label{rem:no_self_referential_names}
  One consequence of these definitions is that $\forall P. \quotep{P}
  \not\in \freenames{P}$.
\end{remark}

\subsection{ Dynamic quote: an example }

Anticipating something of what's to come, consider applying the
substitution, $\widehat{\id{\{}u / z \id{\}}}$, to the following pair
of processes, $\lift{w}{y!(z)}$ and $w[ \lpquote y!(z) \rpquote ]$.

\begin{eqnarray}
	\lift{w}{y!(z)}\widehat{\id{\{}u / z \id{\}}}
		& = &
		\lift{w}{y!(u)} \nonumber\\
	w[ \lpquote y!(z) \rpquote ] \widehat{ \id{\{}u / z \id{\}} }
		& = &
		w[ \lpquote y!(z) \rpquote ] \nonumber
\end{eqnarray}

Because the body of the process between quotes is impervious to
substitution, we get radically different answers. In fact, by
examining the first process in an input context,
e.g. $x?(z).\lift{w}{y!(z)}$, we see that the process under the lift
operator may be shaped by prefixed inputs binding a name inside it. In
this sense, the lift operator will be seen as a way to dynamically
construct processes before reifying them as names.

Finally equipped with these standard features we can present the
dynamics of the calculus.

\subsubsection{Operational semantics} 

Finally, we introduce the computational dynamics. What marks these
algebras as distinct from other more traditionally studied algebraic
structures, e.g. vector spaces or polynomial rings, is the manner in
which dynamics is captured. In traditional structures, dynamics is typically
expressed through morphisms between such structures, as in linear maps
between vector spaces or morphisms between rings. In algebras
associated with the semantics of computation, the dynamics is
expressed as part of the algebraic structure itself, through a
reduction reduction relation typically denoted by $\red$. Below, we
give a recursive presentation of this relation for the calculus used
in the encoding.

$\red \subseteq \pi \times \pi$
$\red : \pi \to \mathcal{P}(\pi)$

\begin{mathpar}
  \inferrule* [lab=Comm] { \textsf{match}( x_{src}, x_{trgt} ) } { x_{trgt}?(y)P \; | \; x_{src}!\langle {Q} \rangle \red P\{\quotep{Q}/y}\} }
  \and \\
  \inferrule* [lab=Par] {{P} \red {P}'} {{{P} | {Q}} \red {{P}' | {Q}}}
  \and
  \inferrule* [lab=Equiv]{{{P} \scong {P}'} \andalso {{P}' \red {Q}'} \andalso {{Q}' \scong {Q}}}{{P} \red {Q}}
\end{mathpar}

\begin{eqnarray*}
  match_{\equiv} (\quotep{P},\quotep{Q}) & := & P \equiv Q \\
  match_{\dagger}(\quotep{P},\quotep{Q}) & := & \forall R. P|Q \red^{*} R => R \red^{*} 0 \\
  match_{K}(\quotep{P},\quotep{Q}) & := & K \mbox{ for some context } K
\end{eqnarray*}

$u?(x)P | u!\langle Q \rangle \red P\{\quotep{Q}/x\}$

%We write $\wred$ for $\red^*$, and $P\red$ if $\exists Q $ such that $ P \red Q$.
We write $P\red$ if $\exists Q $ such that $ P \red Q$ and $P\not\red$, otherwise.

\section{Replication}

As mentioned before, it is known that replication (and hence
recursion) can be implemented in a higher-order process algebra
\cite{SangiorgiWalker}. As our first example of calculation with the
machinery thus far presented we give the construction explicitly in
the {\rhoc}.

\begin{eqnarray}
	D_{x} & := & \prefix{x}{y}{(\binpar{\outputp{x}{y}}{@{y}})} \nonumber\\
	\bangp_{x}{P} & := & \binpar{{x}!\langle{\binpar{D_{x}}{P}}\rangle}{D_{x}} \nonumber
\end{eqnarray}

\begin{eqnarray}
	\bangp_{x}{P} & & \nonumber\\
	=
	& {x}!\langle{(\prefix{x}{y}{(\outputp{x}{y} | @{y})) | P}}\rangle 
	      | \prefix{x}{y}{(\outputp{x}{y} | @{y})} & \nonumber\\
	\red
	& (\outputp{x}{y} | @{y})\substn{\quotep{(\prefix{x}{y}{(@{y} | \outputp{x}{y})) | P}}}{y} & \nonumber\\
	=
	& \outputp{x}{\quotep{(\prefix{x}{y}{(\outputp{x}{y} | @{y})) | P}}}
	  | {(\prefix{x}{y}{(\outputp{x}{y} | @{y})) | P}} & \nonumber\\
	\red
	& \ldots & \nonumber\\
	\red^*
	& P | P | \ldots & \nonumber
\end{eqnarray}

Of course, this encoding, as an implementation, runs away, unfolding
$\bangp{P}$ eagerly. A lazier and more implementable replication
operator, restricted to input-guarded processes, may be obtained as follows.

\begin{eqnarray}
\bangp{\prefix{u}{v}{P}} 
	:= 
	\binpar{\lift{x}{\prefix{u}{v}{(\binpar{D(x)}{P})}}}{D(x)} \nonumber
\end{eqnarray}

\begin{remark}
  Note that the lazier definition still does not deal with summation
  or mixed summation (i.e. sums over input and output). The reader is
  invited to construct definitions of replication that deal with these
  features. 

  Further, the definitions are parameterized in a name, $x$. Can you,
  gentle reader, make a definition that eliminates this parameter and
  guarantees no accidental interaction between the replication
  machinery and the process being replicated -- i.e. no accidental
  sharing of names used by the process to get its work done and the
  name(s) used by the replication to effect copying. This latter
  revision of the definition of replication is crucial to obtaining
  the expected identity $!!P \sim !P$.
\end{remark}

\begin{remark}\label{rem:paradoxical_combinator}
  The reader familiar with the lambda calculus will have noticed the
  similarity between $D$ and the paradoxical combinator.

  [Ed. note: the existence of this seems to suggest we have to be more
  restrictive on the set of processes and names we admit if we are to
  support no-cloning.]
\end{remark}

\subsubsection{Bisimulation}

The computational dynamics gives rise to another kind of equivalence,
the equivalence of computational behavior. As previously mentioned
this is typically captured \emph{via} some form of bisimulation.

% The notion we use in this paper is weak barbed bisimulation
% \cite{milner91polyadicpi}.

The notion we use in this paper is derived from weak barbed
bisimulation \cite{milner91polyadicpi}. 

\begin{definition}
An \emph{observation relation}, $\downarrow_{\mathcal N}$, over a set
of names, $\mathcal N$, is the smallest relation satisfying the rules
below.

\infrule[Out-barb]{y \in {\mathcal N}, \; x \nameeq y}
		  {\outputp{x}{v} \downarrow_{\mathcal N} x}
\infrule[Par-barb]{\mbox{$P\downarrow_{\mathcal N} x$ or $Q\downarrow_{\mathcal N} x$}}
		  {\binpar{P}{Q} \downarrow_{\mathcal N} x}

We write $P \Downarrow_{\mathcal N} x$ if there is $Q$ such that 
$P \wred Q$ and $Q \downarrow_{\mathcal N} x$.
\end{definition}

\begin{definition}
%\label{def.bbisim}
An  ${\mathcal N}$-\emph{barbed bisimulation} over a set of names, ${\mathcal N}$, is a symmetric binary relation 
${\mathcal S}_{\mathcal N}$ between agents such that $P\rel{S}_{\mathcal N}Q$ implies:
\begin{enumerate}
\item If $P \red P'$ then $Q \wred Q'$ and $P'\rel{S}_{\mathcal N} Q'$.
\item If $P\downarrow_{\mathcal N} x$, then $Q\Downarrow_{\mathcal N} x$.
\end{enumerate}
$P$ is ${\mathcal N}$-barbed bisimilar to $Q$, written
$P \wbbisim_{\mathcal N} Q$, if $P \rel{S}_{\mathcal N} Q$ for some ${\mathcal N}$-barbed bisimulation ${\mathcal S}_{\mathcal N}$.
\end{definition}

$\mathcal{R} \subseteq \pi \times \pi$

$P \mathcal{R} Q => \forall P'. P \red P' \Rightarrow \exists Q'. Q \red Q', P' \mathcal{R} Q'$

$P \vdash x \Rightarrow Q \vdash x$

\begin{mathpar}
  \inferrule*[lab=Out-barb]{x \nameeq y}{{y}!\langle{Q}\rangle \vdash x}
  \and
  \inferrule*[lab=Par-barb]{\mbox{$P\vdash x$ or $Q\vdash x$}}{\binpar{P}{Q} \vdash x}
\end{mathpar}

\subsubsection{Contexts}

One of the principle advantages of computational calculi like the
$\pi$-calculus is a well-defined notion of context,
contextual-equivalence and a correlation between
contextual-equivalence and notions of bisimulation. The notion of
context allows the decomposition of a process into (sub-)process and
its syntactic environment, its context. Thus, a context may be
thought of as a process with a ``hole'' (written $\Box$) in it. The
application of a context $M$ to a process $P$, written $M[P]$, is
tantamount to filling the hole in $M$ with $P$. In this paper we do
not need the full weight of this theory, but do make use of the notion
of context in the proof the main theorem. 

\begin{mathpar}
  \inferrule* [lab=summation] {} {{M_{M},M_{N}} \bc \Box \;|\; x.M_{A} \;|\; M_{M}+M_{N}}
  \and
  \inferrule* [lab=agent] {} {{M_{A}} \bc (\vec{x})M_{P} \;| \; \clift{P_0,\ldots,M_{P},\ldots,P_N}}
  \and \\
  \inferrule* [lab=process] {} {{M_{P}} \bc M_{N} \;| \;P|M_{P} }
\end{mathpar} 

\begin{mathpar}
  \inferrule* [lab=sychronization] {} {M_{N} \bc \Box \;|\; x?M_{F} \;|\; x!M_{C}}
  \and
  \inferrule* [lab=abstraction] {} {{M_{F}} \bc (x)M_{P} }
  \and
  \inferrule* [lab=concretion] {} {{M_{C}} \bc \langle M_{P} \rangle }
  \and \\
  \inferrule* [lab=process] {} {{M_{P}} \bc M_{N} \;| \;P|M_{P} }
\end{mathpar}

\begin{definition}[contextual application] Given a context $M$, and
  process $P$, we define the \emph{contextual application}, $M[P] :=
  M\{P/\Box\}$. That is, the contextual application of M to P is the
  substitution of $P$ for $\Box$ in $M$.
\end{definition}

$\meaningof{-} : L \to \mathcal{P}(\pi)$

\begin{mathpar}
  \inferrule* [lab=collection] {} {\meaningof{true} = \pi, \and \meaningof{~E} = \pi \setminus \meaningof{E}, \and \meaningof{E_{1} \& E_{2}} = \meaningof{E_{1}} \cap \meaningof{E_{2}}}
\end{mathpar}

\begin{mathpar}
  \inferrule* [lab=structure] {} {\meaningof{0} = \{ P \in \pi | P \equiv 0 \}, \and \\ \meaningof{E_1 | E_2} = \{ P \in \pi | P \equiv P_{1} | P_{2}, P_{1} \in \meaningof{E_{1}}, P_{2} \in \meaningof{E_2}\} }
\end{mathpar}

\begin{mathpar}
 \inferrule* [lab=behavior] {} {\meaningof{\langle a?b \rangle E} = \{ P \in \pi | P \equiv Q | u?(y)P', \\ \and \\\\ \and \\ \;\;\; u \in \meaningof{a}, \forall z.P'\{z/y\} \in \meaningof{E\{z/b\}}\}, \and \\ \meaningof{a!E} = \{ P \in \pi | P \equiv Q | x!\langle P' \rangle, x \in \meaningof{a} P' \in \meaningof{E}\} }
\end{mathpar}

\begin{mathpar}
 \inferrule* [lab=nominal] {} {\meaningof{\quotep{E}} = \{ \quotep{P} \in \quotep{\pi} | P \in \meaningof{E} \}, \and \meaningof{\quotep{P}} = \{ \quotep{Q} \in \quotep{\pi} | P \equiv Q \} \and \\ \meaningof{@\quotep{E}} = \{ P \in \pi | P \equiv @x, x \in \meaningof{E} \}}
\end{mathpar}

\begin{eqnarray*}
  \\
  \meaningof{-} : TS \to ST
\end{eqnarray*}

\begin{eqnarray*}
  \\
  L : TS \to ST
\end{eqnarray*}

\begin{eqnarray*}
  \\
  P \models E \iff P \in \meaningof{E}
\end{eqnarray*}

\begin{eqnarray*}
  P \approx_{L} Q \iff \forall E \in L. P \models E \iff Q \models E
\end{eqnarray*}

\begin{eqnarray*}
  P \approx_{K} Q
\end{eqnarray*}

\begin{eqnarray*}
  P \approx Q
\end{eqnarray*}

$\approx_{K} = \approx = \approx_{L}$

\subsubsection{Contextual duality}

Note that contexts extend the quotation operation to a family of
operations from processes to names. Given a context, $M$, we can
define a \emph{nominal context}, $\quotep{M}$ by $\quotep{M}[P] :=
\quotep{M[P]}$. To foreshadow what is to come we observe that these
operations enjoy a duality with processes very much like the duality
between vectors and maps from vectors to scalars.

Further, because the calculus is essentially higher-order, we have a
correspondence between contexts and processes. More specifically,
given a name $x$ and a context $M$ we can construct $M^{*}_{x}$ such
that 

\begin{mathpar}
  M^{*}_{x} | \lift{x}{P} \red M[P]
\end{mathpar}

namely,

\begin{mathpar}
  M^{*}_{x} := x?(u).M[\dropn{u}]
\end{mathpar}

The dependence of $M^{*}_{x}$ on a name makes it an abstraction, 

\begin{mathpar}
  M^{*} := (x)x?(u).M[\dropn{u}]
\end{mathpar}

\subsection{Additional notation}

It will sometimes be convenient to denote the process a name
quotes. We already have the notation $x = \quotep{P}$, but it will be
convenient to introduce an alternate notation, $\procn{x}$, when we
want to emphasize the connection to the use of the name. Note that, by
virtue of name equivalence, $\quotep{\procn{x}} \nameeq x$; so, the
notation is consistent with previous definitions.

Further, because names have structure it is possible to effect
substitutions on the basis of that structure. This means we need to
upgrade our notation for substitutions, which we accomplish by
adapting comprehension notation. Thus,

\begin{mathpar}
  P\{ y / x : x \in S \}
\end{mathpar}

is interpreted to mean the process derived from P by replacing (in a
capture-avoiding manner) each occurrence of $x$ in $S$ by $y$. For example,

\begin{mathpar}
  P\{ \quotep{\procn{x}|\procn{x}} / x : x \in \freenames{P} \}
\end{mathpar}

will replace each (occurrence) of a free name $x$ in $P$ by
$\quotep{\procn{x}|\procn{x}}$.

Also, we will avail ourselves of the notation $x^{L}$ and $x^{R}$ to
denote injections of a name into disjoint copies of the name
space. There are numerous ways to accomplish this. One example can be
found in \cite{MeredithR05}. This notation overloads to vectors of
names: $\vec{x}^{\pi} := (x_{i}^{\pi} \; : \; 0 \leq i < |\vec{x}| )$ where $\pi \in \{L,R\}$.

We also use $P^{\Box} := P|\Box$.

In \cite{MeredithR05} an interpretation of the new operator is
given. It turns out that there are several possible interpretations
all enjoying the requisite algebraic properties of the operator (see
\cite{milner91polyadicpi}). We will therefore make liberal use of
$(\nu\; \vec{x})P$.

% subsection the_syntax_and_semantics_of_the_notation_system (end)   

\input{qm2pi.qmops} 

\input{qm2pi.sterngerlach} 

\input{qm2pi.metric} 

% section concurrent_process_calculi (end)

%\input{qm2pi.proofsketch}

% section proof sketch (end)

%\input{qm2pi.slviaknots} 

% section spatial logic via knots (end)

\input{qm2pi.conclusion}

% section conclusion (end)

%\input{qm2pi.dtcodes} 

% section wiring algorithm (end)

\input{qm2pi.ack} 

% section acknowledgments (end)

\newpage


\bibliographystyle{plain}   
\bibliography{../../biblios/main.bib}

\input{qm2pi.rhodetails}

\end{document}

 

%\ifpdf
%\usepackage[pdftex]{graphicx}
%\else
%\usepackage{graphicx}
%\fi

 % \ifpdf
%  \usepackage{pdfsync}
%  \if


%\title{Brief Article}
%\author{David F. Snyder}
%\author{L.G. Meredith}

%\address{Dept. of Math., Texas State University--San Marcos, San Marcos, TX 78666}
       
\pagestyle{empty}


\begin{document}

\lstset{language=[Objective]Caml,frame=shadowbox}

\documentclass[12pt]{llncs}
%\documentclass{jktr}

\usepackage[pdftex]{hyperref}                   
\usepackage {listings}
\usepackage {mathpartir}
\usepackage{bcprules}
%\usepackage{listings}
                       
\usepackage{graphicx} 
%\usepackage[margins=2.5cm,nohead,nofoot]{geometry}
%\usepackage{geometry}
\usepackage{amsfonts}
\usepackage{amstext}
\usepackage{latexsym}
\usepackage{amssymb}
\usepackage{color}


%\include{myPreamble}
\include{qm2pi.local} 

%\ifpdf
%\usepackage[pdftex]{graphicx}
%\else
%\usepackage{graphicx}
%\fi

 % \ifpdf
%  \usepackage{pdfsync}
%  \if


%\title{Brief Article}
%\author{David F. Snyder}
%\author{L.G. Meredith}

%\address{Dept. of Math., Texas State University--San Marcos, San Marcos, TX 78666}
       
\pagestyle{empty}


\begin{document}

\lstset{language=[Objective]Caml,frame=shadowbox}

\input{qm2pi.front}

% section front matter (end)

\input{qm2pi.intro} 
 
% section introduction (end)

% \input{qm2pi.knotations} 

% section notation (end)

\input{qm2pi.process.calculi} 

% section concurrent_process_calculi_and_spatial_logics_ (end)
    
%\input{qm2pi.knots2pi} 

%\input{qm2pi.trefoil} 

%\input{qm2pi.mainthm} 

% subsection basic_interpretation (end)

%\input{qm2pi.rho.presentation} 
\subsection{The syntax and semantics of the notation system}\label{sub:the_syntax_and_semantics_of_the_notation_system} % (fold)

We now summarize a technical presentation of the calculus that
embodies our theory of dynamics. The typical presentation of such a
calculus follows the style of giving generators and relations on
them. The grammar, below, describing term constructors, freely
generates the set of processes, $\Proc$. This set is then quotiented
by a relation known as structural congruence and it is over this set
that the notion of dynamics is expressed. This presentation is
essentially that of \cite{MeredithR05} with the addition of
polyadicity and summation. For readability we have relegated some of
the technical subtleties to an appendix.

\subsubsection{Process grammar}\label{subsub:process_grammar}

\begin{mathpar}
  \inferrule* [lab=synchronization] {} {{M} \bc \pzero \;|\; x?F \;|\; x!C }
  \and
  \inferrule* [lab=abstraction] {} {{F} \bc (x)P}
  \and
  \inferrule* [lab=concretion] {} {{C} \bc \langle Q \rangle}
  \and
  \inferrule* [lab=process] {} {{P,Q} \bc M \;| \;P|Q \;|\; @{x}}
  \and
  \inferrule* [lab=name] {} {{x} \bc \quotep{P}}
\end{mathpar} 

Note that $\vec{x}$ (resp. $\vec{P}$) denotes a vector of names
(resp. processes) of length $|\vec{x}|$ (resp. $|\vec{P}|$). We adopt
the following useful abbreviations.

\begin{mathpar}
   x?(\vec{y}).P := x.(\vec{y})P \and  x\clift{\vec{P}} := x.\clift{\vec{P}}
   \and x!(y) := \lift{x}{\dropn{y}}
   \and \Pi_{i=0}^{n-1}P_i := P_0 | \ldots | P_{n-1}
\end{mathpar}

\subsubsection{Structural congruence}

\paragraph{Free and bound names and alpha-equivalence.} At the
core of structural equivalence is alpha-equivalence which identifies
process that are the same up to a change of variable. Formally, we
recognize the distinction between free and bound names. The free names
of a process, $\freenames{P}$, may be calculated recursively as
follows:

\begin{mathpar}
\freenames{\pzero} := \emptyset
  \and \\
  \freenames{x?(y).P} := \{ x \} \cup (\freenames{P} \setminus \{ y \})
  \and 
  \freenames{x!\langle P \rangle} := \{ x \} \cup \{ P \} 
  \and \\
  \freenames{P|Q} := \freenames{P} \cup \freenames{Q}
  \and \\
  \freenames{@{x}} := \{ x \}
\end{mathpar}

$\pi$
$\quotep{\pi}$

$\freenames{-} : \pi \to \mathcal{P}(\quotep{\pi})$

\begin{eqnarray*}
  \freenames{\pzero} & := & \emptyset \\
  \freenames{x?(y).P} & := & \{ x \} \cup (\freenames{P} \setminus \{ y \}) \\
  \freenames{x!\langle P \rangle} & := & \{ x \} \cup \{ P \} \\
  \freenames{P|Q} & := & \freenames{P} \cup \freenames{Q} \\
  \freenames{\dropn{x}} & := & \{ x \}
\end{eqnarray*}

The bound names of a process, $\boundnames{P}$, are those names occurring in $P$
that are not free. For example, in $x?(y).0$, the name $x$ is free, while $y$ is bound.

\begin{mathpar}
  \inferrule* [lab=monoidal-laws] {} { P|Q \equiv Q|P \and P|0 \equiv P \and P|(Q|R) \equiv (P|Q)|R }
\end{mathpar}

\begin{mathpar}
  \inferrule* [lab=alpha-equivalence] {} { (x)P \equiv (y)P\{y/x\} \and y \not\in \freenames{P} }
\end{mathpar}

\begin{definition}
Then two processes, $P,Q$, are alpha-equivalent if $P = Q\{\vec{y}/\vec{x}\}$ for
some $\vec{x} \in \boundnames{Q},\vec{y} \in \boundnames{P}$, where $Q\{\vec{y}/\vec{x}\}$
denotes the capture-avoiding substitution of $\vec{y}$ for $\vec{x}$ in $Q$.
\end{definition}

\begin{definition}
  The {\em structural congruence} \cite{SangiorgiWalker} , $\equiv$,
  between processes is the least congruence containing
  alpha-equivalence, satisfying the abelian monoid laws
  (associativity, commutativity and $\pzero$ as identity) for parallel
  composition $|$ and for summation $+$.
\end{definition}

\subsection{Name equivalence}

We take name equivalence, written $\nameeq$, to be the smallest
equivalence relation generated by the following rules.

\begin{mathpar}
\inferrule*[lab=Quote-drop]
{ }
{ \quotep{@{x}} \nameeq x }

\inferrule*[lab=Struct-equiv]
{ P \scong Q }
{ \quotep{P} \nameeq \quotep{Q} }
\end{mathpar}

The astute reader will have noticed that the mutual recursion of names
and processes imposes a mutual recursion on alpha-equivalence and
structural equivalence via name-equivalence. Fortunately, all of this
works out pleasantly and we may calculate in the natural way, free of
concern. The reader interested in the details is referred to the
appendix \ref{appendix:rho_details}.

\subsection{Substitution}

We use $\Proc$ for the set of processes, $\QProc$ for the set of
names, and $\id{\{}\vec{y} / \vec{x} \id{\}}$ to denote partial maps,
$s : \QProc \rightarrow \QProc$. A map, $s$ lifts, uniquely, to a map
on process terms, $\widehat{s} : \Proc \rightarrow \Proc$ by the
following equations.

\begin{mathpar}
  (0) \psubstp{Q}{P} := 0 \\
  (R \juxtap S) \psubstp{Q}{P}
  :=    
  (R)\psubstp{Q}{P} \juxtap (S) \psubstp{Q}{P} \\
  (x?(y).R) \psubstp{Q}{P}    
  :=    
  (x)\substp{Q}{P} (z)\concat( (R \psubstn{z}{y}) \psubstp{Q}{P} ) \\
  (\lift{x}{R}) \psubstp{Q}{P}  
  :=
  \lift{(x)\substp{Q}{P}}{ R \psubstp{Q}{P} } \\
%   (\dropn{x})  \psubstp{Q}{P}       
%   := 
%   \left\{ 
%     \begin{array}{ccc} 
%       \dropn{\quotep{Q}} & & x \nameeq \quotep{P} \\
%       \dropn{x} & & otherwise \\
%     \end{array}
%   \right. 
  (\dropn{x})  \psubstp{Q}{P}       
  := 
  \left\{ 
    \begin{array}{ccc} 
      Q & & x \nameeq \quotep{P} \\
      \dropn{x} & & otherwise \\
    \end{array}
  \right.
\end{mathpar}
 

where

\begin{eqnarray}
  (x)\id{\{} \lpquote Q \rpquote / \lpquote P \rpquote \id{\}}            = 
  \left\{ 
    \begin{array}{ccc}
      \lpquote Q \rpquote & & x \nameeq \lpquote P \rpquote \\
      x & & otherwise \\
    \end{array}
  \right. \nonumber
\end{eqnarray}

and $z$ is chosen distinct from $\quotep{P}$, $\quotep{Q}$, the free
names in $Q$, and all the names in $R$. Our $\alpha$-equivalence will
be built in the standard way from this substitution.

\begin{remark}\label{rem:no_self_referential_names}
  One consequence of these definitions is that $\forall P. \quotep{P}
  \not\in \freenames{P}$.
\end{remark}

\subsection{ Dynamic quote: an example }

Anticipating something of what's to come, consider applying the
substitution, $\widehat{\id{\{}u / z \id{\}}}$, to the following pair
of processes, $\lift{w}{y!(z)}$ and $w[ \lpquote y!(z) \rpquote ]$.

\begin{eqnarray}
	\lift{w}{y!(z)}\widehat{\id{\{}u / z \id{\}}}
		& = &
		\lift{w}{y!(u)} \nonumber\\
	w[ \lpquote y!(z) \rpquote ] \widehat{ \id{\{}u / z \id{\}} }
		& = &
		w[ \lpquote y!(z) \rpquote ] \nonumber
\end{eqnarray}

Because the body of the process between quotes is impervious to
substitution, we get radically different answers. In fact, by
examining the first process in an input context,
e.g. $x?(z).\lift{w}{y!(z)}$, we see that the process under the lift
operator may be shaped by prefixed inputs binding a name inside it. In
this sense, the lift operator will be seen as a way to dynamically
construct processes before reifying them as names.

Finally equipped with these standard features we can present the
dynamics of the calculus.

\subsubsection{Operational semantics} 

Finally, we introduce the computational dynamics. What marks these
algebras as distinct from other more traditionally studied algebraic
structures, e.g. vector spaces or polynomial rings, is the manner in
which dynamics is captured. In traditional structures, dynamics is typically
expressed through morphisms between such structures, as in linear maps
between vector spaces or morphisms between rings. In algebras
associated with the semantics of computation, the dynamics is
expressed as part of the algebraic structure itself, through a
reduction reduction relation typically denoted by $\red$. Below, we
give a recursive presentation of this relation for the calculus used
in the encoding.

$\red \subseteq \pi \times \pi$
$\red : \pi \to \mathcal{P}(\pi)$

\begin{mathpar}
  \inferrule* [lab=Comm] { \textsf{match}( x_{src}, x_{trgt} ) } { x_{trgt}?(y)P \; | \; x_{src}!\langle {Q} \rangle \red P\{\quotep{Q}/y}\} }
  \and \\
  \inferrule* [lab=Par] {{P} \red {P}'} {{{P} | {Q}} \red {{P}' | {Q}}}
  \and
  \inferrule* [lab=Equiv]{{{P} \scong {P}'} \andalso {{P}' \red {Q}'} \andalso {{Q}' \scong {Q}}}{{P} \red {Q}}
\end{mathpar}

\begin{eqnarray*}
  match_{\equiv} (\quotep{P},\quotep{Q}) & := & P \equiv Q \\
  match_{\dagger}(\quotep{P},\quotep{Q}) & := & \forall R. P|Q \red^{*} R => R \red^{*} 0 \\
  match_{K}(\quotep{P},\quotep{Q}) & := & K \mbox{ for some context } K
\end{eqnarray*}

$u?(x)P | u!\langle Q \rangle \red P\{\quotep{Q}/x\}$

%We write $\wred$ for $\red^*$, and $P\red$ if $\exists Q $ such that $ P \red Q$.
We write $P\red$ if $\exists Q $ such that $ P \red Q$ and $P\not\red$, otherwise.

\section{Replication}

As mentioned before, it is known that replication (and hence
recursion) can be implemented in a higher-order process algebra
\cite{SangiorgiWalker}. As our first example of calculation with the
machinery thus far presented we give the construction explicitly in
the {\rhoc}.

\begin{eqnarray}
	D_{x} & := & \prefix{x}{y}{(\binpar{\outputp{x}{y}}{@{y}})} \nonumber\\
	\bangp_{x}{P} & := & \binpar{{x}!\langle{\binpar{D_{x}}{P}}\rangle}{D_{x}} \nonumber
\end{eqnarray}

\begin{eqnarray}
	\bangp_{x}{P} & & \nonumber\\
	=
	& {x}!\langle{(\prefix{x}{y}{(\outputp{x}{y} | @{y})) | P}}\rangle 
	      | \prefix{x}{y}{(\outputp{x}{y} | @{y})} & \nonumber\\
	\red
	& (\outputp{x}{y} | @{y})\substn{\quotep{(\prefix{x}{y}{(@{y} | \outputp{x}{y})) | P}}}{y} & \nonumber\\
	=
	& \outputp{x}{\quotep{(\prefix{x}{y}{(\outputp{x}{y} | @{y})) | P}}}
	  | {(\prefix{x}{y}{(\outputp{x}{y} | @{y})) | P}} & \nonumber\\
	\red
	& \ldots & \nonumber\\
	\red^*
	& P | P | \ldots & \nonumber
\end{eqnarray}

Of course, this encoding, as an implementation, runs away, unfolding
$\bangp{P}$ eagerly. A lazier and more implementable replication
operator, restricted to input-guarded processes, may be obtained as follows.

\begin{eqnarray}
\bangp{\prefix{u}{v}{P}} 
	:= 
	\binpar{\lift{x}{\prefix{u}{v}{(\binpar{D(x)}{P})}}}{D(x)} \nonumber
\end{eqnarray}

\begin{remark}
  Note that the lazier definition still does not deal with summation
  or mixed summation (i.e. sums over input and output). The reader is
  invited to construct definitions of replication that deal with these
  features. 

  Further, the definitions are parameterized in a name, $x$. Can you,
  gentle reader, make a definition that eliminates this parameter and
  guarantees no accidental interaction between the replication
  machinery and the process being replicated -- i.e. no accidental
  sharing of names used by the process to get its work done and the
  name(s) used by the replication to effect copying. This latter
  revision of the definition of replication is crucial to obtaining
  the expected identity $!!P \sim !P$.
\end{remark}

\begin{remark}\label{rem:paradoxical_combinator}
  The reader familiar with the lambda calculus will have noticed the
  similarity between $D$ and the paradoxical combinator.

  [Ed. note: the existence of this seems to suggest we have to be more
  restrictive on the set of processes and names we admit if we are to
  support no-cloning.]
\end{remark}

\subsubsection{Bisimulation}

The computational dynamics gives rise to another kind of equivalence,
the equivalence of computational behavior. As previously mentioned
this is typically captured \emph{via} some form of bisimulation.

% The notion we use in this paper is weak barbed bisimulation
% \cite{milner91polyadicpi}.

The notion we use in this paper is derived from weak barbed
bisimulation \cite{milner91polyadicpi}. 

\begin{definition}
An \emph{observation relation}, $\downarrow_{\mathcal N}$, over a set
of names, $\mathcal N$, is the smallest relation satisfying the rules
below.

\infrule[Out-barb]{y \in {\mathcal N}, \; x \nameeq y}
		  {\outputp{x}{v} \downarrow_{\mathcal N} x}
\infrule[Par-barb]{\mbox{$P\downarrow_{\mathcal N} x$ or $Q\downarrow_{\mathcal N} x$}}
		  {\binpar{P}{Q} \downarrow_{\mathcal N} x}

We write $P \Downarrow_{\mathcal N} x$ if there is $Q$ such that 
$P \wred Q$ and $Q \downarrow_{\mathcal N} x$.
\end{definition}

\begin{definition}
%\label{def.bbisim}
An  ${\mathcal N}$-\emph{barbed bisimulation} over a set of names, ${\mathcal N}$, is a symmetric binary relation 
${\mathcal S}_{\mathcal N}$ between agents such that $P\rel{S}_{\mathcal N}Q$ implies:
\begin{enumerate}
\item If $P \red P'$ then $Q \wred Q'$ and $P'\rel{S}_{\mathcal N} Q'$.
\item If $P\downarrow_{\mathcal N} x$, then $Q\Downarrow_{\mathcal N} x$.
\end{enumerate}
$P$ is ${\mathcal N}$-barbed bisimilar to $Q$, written
$P \wbbisim_{\mathcal N} Q$, if $P \rel{S}_{\mathcal N} Q$ for some ${\mathcal N}$-barbed bisimulation ${\mathcal S}_{\mathcal N}$.
\end{definition}

$\mathcal{R} \subseteq \pi \times \pi$

$P \mathcal{R} Q => \forall P'. P \red P' \Rightarrow \exists Q'. Q \red Q', P' \mathcal{R} Q'$

$P \vdash x \Rightarrow Q \vdash x$

\begin{mathpar}
  \inferrule*[lab=Out-barb]{x \nameeq y}{{y}!\langle{Q}\rangle \vdash x}
  \and
  \inferrule*[lab=Par-barb]{\mbox{$P\vdash x$ or $Q\vdash x$}}{\binpar{P}{Q} \vdash x}
\end{mathpar}

\subsubsection{Contexts}

One of the principle advantages of computational calculi like the
$\pi$-calculus is a well-defined notion of context,
contextual-equivalence and a correlation between
contextual-equivalence and notions of bisimulation. The notion of
context allows the decomposition of a process into (sub-)process and
its syntactic environment, its context. Thus, a context may be
thought of as a process with a ``hole'' (written $\Box$) in it. The
application of a context $M$ to a process $P$, written $M[P]$, is
tantamount to filling the hole in $M$ with $P$. In this paper we do
not need the full weight of this theory, but do make use of the notion
of context in the proof the main theorem. 

\begin{mathpar}
  \inferrule* [lab=summation] {} {{M_{M},M_{N}} \bc \Box \;|\; x.M_{A} \;|\; M_{M}+M_{N}}
  \and
  \inferrule* [lab=agent] {} {{M_{A}} \bc (\vec{x})M_{P} \;| \; \clift{P_0,\ldots,M_{P},\ldots,P_N}}
  \and \\
  \inferrule* [lab=process] {} {{M_{P}} \bc M_{N} \;| \;P|M_{P} }
\end{mathpar} 

\begin{mathpar}
  \inferrule* [lab=sychronization] {} {M_{N} \bc \Box \;|\; x?M_{F} \;|\; x!M_{C}}
  \and
  \inferrule* [lab=abstraction] {} {{M_{F}} \bc (x)M_{P} }
  \and
  \inferrule* [lab=concretion] {} {{M_{C}} \bc \langle M_{P} \rangle }
  \and \\
  \inferrule* [lab=process] {} {{M_{P}} \bc M_{N} \;| \;P|M_{P} }
\end{mathpar}

\begin{definition}[contextual application] Given a context $M$, and
  process $P$, we define the \emph{contextual application}, $M[P] :=
  M\{P/\Box\}$. That is, the contextual application of M to P is the
  substitution of $P$ for $\Box$ in $M$.
\end{definition}

$\meaningof{-} : L \to \mathcal{P}(\pi)$

\begin{mathpar}
  \inferrule* [lab=collection] {} {\meaningof{true} = \pi, \and \meaningof{~E} = \pi \setminus \meaningof{E}, \and \meaningof{E_{1} \& E_{2}} = \meaningof{E_{1}} \cap \meaningof{E_{2}}}
\end{mathpar}

\begin{mathpar}
  \inferrule* [lab=structure] {} {\meaningof{0} = \{ P \in \pi | P \equiv 0 \}, \and \\ \meaningof{E_1 | E_2} = \{ P \in \pi | P \equiv P_{1} | P_{2}, P_{1} \in \meaningof{E_{1}}, P_{2} \in \meaningof{E_2}\} }
\end{mathpar}

\begin{mathpar}
 \inferrule* [lab=behavior] {} {\meaningof{\langle a?b \rangle E} = \{ P \in \pi | P \equiv Q | u?(y)P', \\ \and \\\\ \and \\ \;\;\; u \in \meaningof{a}, \forall z.P'\{z/y\} \in \meaningof{E\{z/b\}}\}, \and \\ \meaningof{a!E} = \{ P \in \pi | P \equiv Q | x!\langle P' \rangle, x \in \meaningof{a} P' \in \meaningof{E}\} }
\end{mathpar}

\begin{mathpar}
 \inferrule* [lab=nominal] {} {\meaningof{\quotep{E}} = \{ \quotep{P} \in \quotep{\pi} | P \in \meaningof{E} \}, \and \meaningof{\quotep{P}} = \{ \quotep{Q} \in \quotep{\pi} | P \equiv Q \} \and \\ \meaningof{@\quotep{E}} = \{ P \in \pi | P \equiv @x, x \in \meaningof{E} \}}
\end{mathpar}

\begin{eqnarray*}
  \\
  \meaningof{-} : TS \to ST
\end{eqnarray*}

\begin{eqnarray*}
  \\
  L : TS \to ST
\end{eqnarray*}

\begin{eqnarray*}
  \\
  P \models E \iff P \in \meaningof{E}
\end{eqnarray*}

\begin{eqnarray*}
  P \approx_{L} Q \iff \forall E \in L. P \models E \iff Q \models E
\end{eqnarray*}

\begin{eqnarray*}
  P \approx_{K} Q
\end{eqnarray*}

\begin{eqnarray*}
  P \approx Q
\end{eqnarray*}

$\approx_{K} = \approx = \approx_{L}$

\subsubsection{Contextual duality}

Note that contexts extend the quotation operation to a family of
operations from processes to names. Given a context, $M$, we can
define a \emph{nominal context}, $\quotep{M}$ by $\quotep{M}[P] :=
\quotep{M[P]}$. To foreshadow what is to come we observe that these
operations enjoy a duality with processes very much like the duality
between vectors and maps from vectors to scalars.

Further, because the calculus is essentially higher-order, we have a
correspondence between contexts and processes. More specifically,
given a name $x$ and a context $M$ we can construct $M^{*}_{x}$ such
that 

\begin{mathpar}
  M^{*}_{x} | \lift{x}{P} \red M[P]
\end{mathpar}

namely,

\begin{mathpar}
  M^{*}_{x} := x?(u).M[\dropn{u}]
\end{mathpar}

The dependence of $M^{*}_{x}$ on a name makes it an abstraction, 

\begin{mathpar}
  M^{*} := (x)x?(u).M[\dropn{u}]
\end{mathpar}

\subsection{Additional notation}

It will sometimes be convenient to denote the process a name
quotes. We already have the notation $x = \quotep{P}$, but it will be
convenient to introduce an alternate notation, $\procn{x}$, when we
want to emphasize the connection to the use of the name. Note that, by
virtue of name equivalence, $\quotep{\procn{x}} \nameeq x$; so, the
notation is consistent with previous definitions.

Further, because names have structure it is possible to effect
substitutions on the basis of that structure. This means we need to
upgrade our notation for substitutions, which we accomplish by
adapting comprehension notation. Thus,

\begin{mathpar}
  P\{ y / x : x \in S \}
\end{mathpar}

is interpreted to mean the process derived from P by replacing (in a
capture-avoiding manner) each occurrence of $x$ in $S$ by $y$. For example,

\begin{mathpar}
  P\{ \quotep{\procn{x}|\procn{x}} / x : x \in \freenames{P} \}
\end{mathpar}

will replace each (occurrence) of a free name $x$ in $P$ by
$\quotep{\procn{x}|\procn{x}}$.

Also, we will avail ourselves of the notation $x^{L}$ and $x^{R}$ to
denote injections of a name into disjoint copies of the name
space. There are numerous ways to accomplish this. One example can be
found in \cite{MeredithR05}. This notation overloads to vectors of
names: $\vec{x}^{\pi} := (x_{i}^{\pi} \; : \; 0 \leq i < |\vec{x}| )$ where $\pi \in \{L,R\}$.

We also use $P^{\Box} := P|\Box$.

In \cite{MeredithR05} an interpretation of the new operator is
given. It turns out that there are several possible interpretations
all enjoying the requisite algebraic properties of the operator (see
\cite{milner91polyadicpi}). We will therefore make liberal use of
$(\nu\; \vec{x})P$.

% subsection the_syntax_and_semantics_of_the_notation_system (end)   

\input{qm2pi.qmops} 

\input{qm2pi.sterngerlach} 

\input{qm2pi.metric} 

% section concurrent_process_calculi (end)

%\input{qm2pi.proofsketch}

% section proof sketch (end)

%\input{qm2pi.slviaknots} 

% section spatial logic via knots (end)

\input{qm2pi.conclusion}

% section conclusion (end)

%\input{qm2pi.dtcodes} 

% section wiring algorithm (end)

\input{qm2pi.ack} 

% section acknowledgments (end)

\newpage


\bibliographystyle{plain}   
\bibliography{../../biblios/main.bib}

\input{qm2pi.rhodetails}

\end{document}



% section front matter (end)

\section{Introduction}\label{sec:introduction} % (fold)
In this draft of the material i am going to have to dispense with the
usual writing conventions adopted in papers on these topics. i'm going
to have adopt whatever tone i need at the time i'm writing up the
calculations. Sometimes this may be very conversational; others it may
be the barest mathematical grunts; others still it may be that i have
lifted text from one of my other papers because the exposition of some
point was better said there. i hope that my readers are not unduly put
out by this decision. i'm not doing this to flout convention or be
rebellious. i find these calculations very technically challenging. To
keep everything going technically, something has to give; i have to
let go of some cognitive burden. So, the academic writing style --
with all of its trade-offs in terms of facilitating technical
communication -- is what i'm letting go of. Perhaps subsequent drafts
can be tightened and polished, but for now, i'm going to speak as if
we were sitting together in a coffee shop with a laptop, wifi and a
pad of paper and a pencil.

So, here's what i have to say. We -- you and i, comfortably ensconced
in our coffee shop and well-equipped with our tools -- can realize and
carry out the calculations of quantum mechanics over a very different
formal theory of dynamics, a formal theory of dynamics that
corresponds to a theory of concurrent computation with
\emph{reflection}. It has the advantage that the underlying theory is
already `quantized', but supports analogues all of the continuuous
operations. Strikingly, this underlying theory has recently been
connected with a notion of metric that we can show, by calculating
together, coincides with the metric induced by the inner product.

There are a lot of reasons why you might be interested in seeing
calculations of this form. Here's why i'm interested. For the past
several centuries there has been no competitor to the ``Newtonian''
account of dynamics. As a result the predominant share of accounts of
dynamical systems and situations have had to be formulated in terms of
the Newtonian machinery. i view this as an intellectually dangerous
position to occupy. Everything, despite it's intrinsic shape, turns
into a nail to be hit with this hammer. Recently, however, the theory
of computation has matured to the point where we have candidates for
theories of dynamics that offer very different perspective on
reasoning about dynamical systems and situations. Testing these
candidates against very successful accounts of dynamical situations,
like quantum mechanics, is going to give us some sense of how mature
they are and some measure of the quality of these accounts of
dynamics.

\subsection{Summary of contributions and outline of paper}

So, we're going to develop an interpretation of the operations of
quantum mechanics normally interpreted by Hilbert spaces and
operators. We're going to do this over a theory of computation. Note
that this is very different than the usual quantum computation program
which develops notions of computation over quantum mechanics. Rather,
we are developing a story that aligns with Wheeler's slogan: It from
Bit. To do this we will first provide an account of the theory of
computation at play here. Then we will dive into a calculation-driven
interpretation of the operations of quantum mechanics.

The reason we take this approach is that -- until very recently --
there hasn't been an axiomatic account of quantum mechanics. As a
result there has been no sharp delineation of the mathematical theory
supporting interpretation of the physical theory and the physical
theory, itself. So, ambient features of the maths are free to be
exploited (or supressed) without a real accounting of their physical
relevance. There is no sharp statement ``here's the physical theory''
qua \emph{theory} and ``here's the mathematical interpretation''
enabling a judgment of how faithful the interpretation is -- apart
from experimental observation. When there is an axiomatic account we
can judge how well a given mathematical formalism supports an
interpretation of the axioms, independent of
experimentation. Likewise, we can judge how well we have captured our
physical evidence and experience with our axiomatics, independent of
any specific mathematical implementation, with accidental detail that
may or may not have physical significance. 

In lieu of a fully fleshed out and vetted axiomatic account of quantum
mechanics, interpreting the operational notions in service of modeling
physical systems will have to suffice. In other words, we are not in
the business of providing a model of Hilbert spaces and operators. We
are in the business of providing a model of quantum mechanics because
we are motivated by testing our notions of dynamics against physical
theory; and, the predictive calculations of the physical theory must
serve as the best formulation -- shy of a fully fleshed out axiomatic
account -- of the physical theory itself (as they have for scientific
theories since time immemorial). Put another way, despite a
whole-hearted commitment to an It-from-Bit ontology, we are firmly
aligned with the shut-up-and-calculate camp as the best way to obtain
results either from the physical perspective or as a quality assurance
measure of our fledgling theory of dynamics.

In detail, we present a reflective process calculus. Then we develop
intuitive correspondences between the notions available in this
calculus and the usual physical notions supporting quantum mechanical
calculations. Thus, 

\begin{table}[htp]
  \center{
    \fbox{
      \begin{tabular}{c|c}
        quantum mechanics & process calculus \\
        \hline
        scalar & name \\
        state vector & process \\
        dual & contextual duals \\
        matrix & formal sums of process-context-dual pairs \\
        orthogonality & process annihilation \\
        inner product & execution-formula + quoting
      \end{tabular}
    }
  }
  \caption{QM - process calculi correspondences}
\end{table}

Then we tighten up these intuitions to operational definitions. We
employ the Dirac notation as the best proxy we can find for an
abstract syntax of the quantum mechanical notions. The definitions we
develop put us in contact with equational constraints coming from the
theory that we demonstrate the definitions and calculations satisfy.

This puts us in a position to shut up and calculate for the
Stern-Gerlach experimental set up, showing how these predictive
calculations become calculations on processes in our theory of a
reflective process calculus.

Penultimately, we demonstrate that the notion of metric coming from
the inner product coincides with the notion of metric available from
the theory of bisimulation. This demonstration gives us the right to
think of space as arising from behavior. Finally, we consider where we
might go from the new vantage point we have obtained.

% section introduction (end) 
 
% section introduction (end)

% \documentclass[12pt]{llncs}
%\documentclass{jktr}

\usepackage[pdftex]{hyperref}                   
\usepackage {listings}
\usepackage {mathpartir}
\usepackage{bcprules}
%\usepackage{listings}
                       
\usepackage{graphicx} 
%\usepackage[margins=2.5cm,nohead,nofoot]{geometry}
%\usepackage{geometry}
\usepackage{amsfonts}
\usepackage{amstext}
\usepackage{latexsym}
\usepackage{amssymb}
\usepackage{color}


%\include{myPreamble}
\include{qm2pi.local} 

%\ifpdf
%\usepackage[pdftex]{graphicx}
%\else
%\usepackage{graphicx}
%\fi

 % \ifpdf
%  \usepackage{pdfsync}
%  \if


%\title{Brief Article}
%\author{David F. Snyder}
%\author{L.G. Meredith}

%\address{Dept. of Math., Texas State University--San Marcos, San Marcos, TX 78666}
       
\pagestyle{empty}


\begin{document}

\lstset{language=[Objective]Caml,frame=shadowbox}

\input{qm2pi.front}

% section front matter (end)

\input{qm2pi.intro} 
 
% section introduction (end)

% \input{qm2pi.knotations} 

% section notation (end)

\input{qm2pi.process.calculi} 

% section concurrent_process_calculi_and_spatial_logics_ (end)
    
%\input{qm2pi.knots2pi} 

%\input{qm2pi.trefoil} 

%\input{qm2pi.mainthm} 

% subsection basic_interpretation (end)

%\input{qm2pi.rho.presentation} 
\subsection{The syntax and semantics of the notation system}\label{sub:the_syntax_and_semantics_of_the_notation_system} % (fold)

We now summarize a technical presentation of the calculus that
embodies our theory of dynamics. The typical presentation of such a
calculus follows the style of giving generators and relations on
them. The grammar, below, describing term constructors, freely
generates the set of processes, $\Proc$. This set is then quotiented
by a relation known as structural congruence and it is over this set
that the notion of dynamics is expressed. This presentation is
essentially that of \cite{MeredithR05} with the addition of
polyadicity and summation. For readability we have relegated some of
the technical subtleties to an appendix.

\subsubsection{Process grammar}\label{subsub:process_grammar}

\begin{mathpar}
  \inferrule* [lab=synchronization] {} {{M} \bc \pzero \;|\; x?F \;|\; x!C }
  \and
  \inferrule* [lab=abstraction] {} {{F} \bc (x)P}
  \and
  \inferrule* [lab=concretion] {} {{C} \bc \langle Q \rangle}
  \and
  \inferrule* [lab=process] {} {{P,Q} \bc M \;| \;P|Q \;|\; @{x}}
  \and
  \inferrule* [lab=name] {} {{x} \bc \quotep{P}}
\end{mathpar} 

Note that $\vec{x}$ (resp. $\vec{P}$) denotes a vector of names
(resp. processes) of length $|\vec{x}|$ (resp. $|\vec{P}|$). We adopt
the following useful abbreviations.

\begin{mathpar}
   x?(\vec{y}).P := x.(\vec{y})P \and  x\clift{\vec{P}} := x.\clift{\vec{P}}
   \and x!(y) := \lift{x}{\dropn{y}}
   \and \Pi_{i=0}^{n-1}P_i := P_0 | \ldots | P_{n-1}
\end{mathpar}

\subsubsection{Structural congruence}

\paragraph{Free and bound names and alpha-equivalence.} At the
core of structural equivalence is alpha-equivalence which identifies
process that are the same up to a change of variable. Formally, we
recognize the distinction between free and bound names. The free names
of a process, $\freenames{P}$, may be calculated recursively as
follows:

\begin{mathpar}
\freenames{\pzero} := \emptyset
  \and \\
  \freenames{x?(y).P} := \{ x \} \cup (\freenames{P} \setminus \{ y \})
  \and 
  \freenames{x!\langle P \rangle} := \{ x \} \cup \{ P \} 
  \and \\
  \freenames{P|Q} := \freenames{P} \cup \freenames{Q}
  \and \\
  \freenames{@{x}} := \{ x \}
\end{mathpar}

$\pi$
$\quotep{\pi}$

$\freenames{-} : \pi \to \mathcal{P}(\quotep{\pi})$

\begin{eqnarray*}
  \freenames{\pzero} & := & \emptyset \\
  \freenames{x?(y).P} & := & \{ x \} \cup (\freenames{P} \setminus \{ y \}) \\
  \freenames{x!\langle P \rangle} & := & \{ x \} \cup \{ P \} \\
  \freenames{P|Q} & := & \freenames{P} \cup \freenames{Q} \\
  \freenames{\dropn{x}} & := & \{ x \}
\end{eqnarray*}

The bound names of a process, $\boundnames{P}$, are those names occurring in $P$
that are not free. For example, in $x?(y).0$, the name $x$ is free, while $y$ is bound.

\begin{mathpar}
  \inferrule* [lab=monoidal-laws] {} { P|Q \equiv Q|P \and P|0 \equiv P \and P|(Q|R) \equiv (P|Q)|R }
\end{mathpar}

\begin{mathpar}
  \inferrule* [lab=alpha-equivalence] {} { (x)P \equiv (y)P\{y/x\} \and y \not\in \freenames{P} }
\end{mathpar}

\begin{definition}
Then two processes, $P,Q$, are alpha-equivalent if $P = Q\{\vec{y}/\vec{x}\}$ for
some $\vec{x} \in \boundnames{Q},\vec{y} \in \boundnames{P}$, where $Q\{\vec{y}/\vec{x}\}$
denotes the capture-avoiding substitution of $\vec{y}$ for $\vec{x}$ in $Q$.
\end{definition}

\begin{definition}
  The {\em structural congruence} \cite{SangiorgiWalker} , $\equiv$,
  between processes is the least congruence containing
  alpha-equivalence, satisfying the abelian monoid laws
  (associativity, commutativity and $\pzero$ as identity) for parallel
  composition $|$ and for summation $+$.
\end{definition}

\subsection{Name equivalence}

We take name equivalence, written $\nameeq$, to be the smallest
equivalence relation generated by the following rules.

\begin{mathpar}
\inferrule*[lab=Quote-drop]
{ }
{ \quotep{@{x}} \nameeq x }

\inferrule*[lab=Struct-equiv]
{ P \scong Q }
{ \quotep{P} \nameeq \quotep{Q} }
\end{mathpar}

The astute reader will have noticed that the mutual recursion of names
and processes imposes a mutual recursion on alpha-equivalence and
structural equivalence via name-equivalence. Fortunately, all of this
works out pleasantly and we may calculate in the natural way, free of
concern. The reader interested in the details is referred to the
appendix \ref{appendix:rho_details}.

\subsection{Substitution}

We use $\Proc$ for the set of processes, $\QProc$ for the set of
names, and $\id{\{}\vec{y} / \vec{x} \id{\}}$ to denote partial maps,
$s : \QProc \rightarrow \QProc$. A map, $s$ lifts, uniquely, to a map
on process terms, $\widehat{s} : \Proc \rightarrow \Proc$ by the
following equations.

\begin{mathpar}
  (0) \psubstp{Q}{P} := 0 \\
  (R \juxtap S) \psubstp{Q}{P}
  :=    
  (R)\psubstp{Q}{P} \juxtap (S) \psubstp{Q}{P} \\
  (x?(y).R) \psubstp{Q}{P}    
  :=    
  (x)\substp{Q}{P} (z)\concat( (R \psubstn{z}{y}) \psubstp{Q}{P} ) \\
  (\lift{x}{R}) \psubstp{Q}{P}  
  :=
  \lift{(x)\substp{Q}{P}}{ R \psubstp{Q}{P} } \\
%   (\dropn{x})  \psubstp{Q}{P}       
%   := 
%   \left\{ 
%     \begin{array}{ccc} 
%       \dropn{\quotep{Q}} & & x \nameeq \quotep{P} \\
%       \dropn{x} & & otherwise \\
%     \end{array}
%   \right. 
  (\dropn{x})  \psubstp{Q}{P}       
  := 
  \left\{ 
    \begin{array}{ccc} 
      Q & & x \nameeq \quotep{P} \\
      \dropn{x} & & otherwise \\
    \end{array}
  \right.
\end{mathpar}
 

where

\begin{eqnarray}
  (x)\id{\{} \lpquote Q \rpquote / \lpquote P \rpquote \id{\}}            = 
  \left\{ 
    \begin{array}{ccc}
      \lpquote Q \rpquote & & x \nameeq \lpquote P \rpquote \\
      x & & otherwise \\
    \end{array}
  \right. \nonumber
\end{eqnarray}

and $z$ is chosen distinct from $\quotep{P}$, $\quotep{Q}$, the free
names in $Q$, and all the names in $R$. Our $\alpha$-equivalence will
be built in the standard way from this substitution.

\begin{remark}\label{rem:no_self_referential_names}
  One consequence of these definitions is that $\forall P. \quotep{P}
  \not\in \freenames{P}$.
\end{remark}

\subsection{ Dynamic quote: an example }

Anticipating something of what's to come, consider applying the
substitution, $\widehat{\id{\{}u / z \id{\}}}$, to the following pair
of processes, $\lift{w}{y!(z)}$ and $w[ \lpquote y!(z) \rpquote ]$.

\begin{eqnarray}
	\lift{w}{y!(z)}\widehat{\id{\{}u / z \id{\}}}
		& = &
		\lift{w}{y!(u)} \nonumber\\
	w[ \lpquote y!(z) \rpquote ] \widehat{ \id{\{}u / z \id{\}} }
		& = &
		w[ \lpquote y!(z) \rpquote ] \nonumber
\end{eqnarray}

Because the body of the process between quotes is impervious to
substitution, we get radically different answers. In fact, by
examining the first process in an input context,
e.g. $x?(z).\lift{w}{y!(z)}$, we see that the process under the lift
operator may be shaped by prefixed inputs binding a name inside it. In
this sense, the lift operator will be seen as a way to dynamically
construct processes before reifying them as names.

Finally equipped with these standard features we can present the
dynamics of the calculus.

\subsubsection{Operational semantics} 

Finally, we introduce the computational dynamics. What marks these
algebras as distinct from other more traditionally studied algebraic
structures, e.g. vector spaces or polynomial rings, is the manner in
which dynamics is captured. In traditional structures, dynamics is typically
expressed through morphisms between such structures, as in linear maps
between vector spaces or morphisms between rings. In algebras
associated with the semantics of computation, the dynamics is
expressed as part of the algebraic structure itself, through a
reduction reduction relation typically denoted by $\red$. Below, we
give a recursive presentation of this relation for the calculus used
in the encoding.

$\red \subseteq \pi \times \pi$
$\red : \pi \to \mathcal{P}(\pi)$

\begin{mathpar}
  \inferrule* [lab=Comm] { \textsf{match}( x_{src}, x_{trgt} ) } { x_{trgt}?(y)P \; | \; x_{src}!\langle {Q} \rangle \red P\{\quotep{Q}/y}\} }
  \and \\
  \inferrule* [lab=Par] {{P} \red {P}'} {{{P} | {Q}} \red {{P}' | {Q}}}
  \and
  \inferrule* [lab=Equiv]{{{P} \scong {P}'} \andalso {{P}' \red {Q}'} \andalso {{Q}' \scong {Q}}}{{P} \red {Q}}
\end{mathpar}

\begin{eqnarray*}
  match_{\equiv} (\quotep{P},\quotep{Q}) & := & P \equiv Q \\
  match_{\dagger}(\quotep{P},\quotep{Q}) & := & \forall R. P|Q \red^{*} R => R \red^{*} 0 \\
  match_{K}(\quotep{P},\quotep{Q}) & := & K \mbox{ for some context } K
\end{eqnarray*}

$u?(x)P | u!\langle Q \rangle \red P\{\quotep{Q}/x\}$

%We write $\wred$ for $\red^*$, and $P\red$ if $\exists Q $ such that $ P \red Q$.
We write $P\red$ if $\exists Q $ such that $ P \red Q$ and $P\not\red$, otherwise.

\section{Replication}

As mentioned before, it is known that replication (and hence
recursion) can be implemented in a higher-order process algebra
\cite{SangiorgiWalker}. As our first example of calculation with the
machinery thus far presented we give the construction explicitly in
the {\rhoc}.

\begin{eqnarray}
	D_{x} & := & \prefix{x}{y}{(\binpar{\outputp{x}{y}}{@{y}})} \nonumber\\
	\bangp_{x}{P} & := & \binpar{{x}!\langle{\binpar{D_{x}}{P}}\rangle}{D_{x}} \nonumber
\end{eqnarray}

\begin{eqnarray}
	\bangp_{x}{P} & & \nonumber\\
	=
	& {x}!\langle{(\prefix{x}{y}{(\outputp{x}{y} | @{y})) | P}}\rangle 
	      | \prefix{x}{y}{(\outputp{x}{y} | @{y})} & \nonumber\\
	\red
	& (\outputp{x}{y} | @{y})\substn{\quotep{(\prefix{x}{y}{(@{y} | \outputp{x}{y})) | P}}}{y} & \nonumber\\
	=
	& \outputp{x}{\quotep{(\prefix{x}{y}{(\outputp{x}{y} | @{y})) | P}}}
	  | {(\prefix{x}{y}{(\outputp{x}{y} | @{y})) | P}} & \nonumber\\
	\red
	& \ldots & \nonumber\\
	\red^*
	& P | P | \ldots & \nonumber
\end{eqnarray}

Of course, this encoding, as an implementation, runs away, unfolding
$\bangp{P}$ eagerly. A lazier and more implementable replication
operator, restricted to input-guarded processes, may be obtained as follows.

\begin{eqnarray}
\bangp{\prefix{u}{v}{P}} 
	:= 
	\binpar{\lift{x}{\prefix{u}{v}{(\binpar{D(x)}{P})}}}{D(x)} \nonumber
\end{eqnarray}

\begin{remark}
  Note that the lazier definition still does not deal with summation
  or mixed summation (i.e. sums over input and output). The reader is
  invited to construct definitions of replication that deal with these
  features. 

  Further, the definitions are parameterized in a name, $x$. Can you,
  gentle reader, make a definition that eliminates this parameter and
  guarantees no accidental interaction between the replication
  machinery and the process being replicated -- i.e. no accidental
  sharing of names used by the process to get its work done and the
  name(s) used by the replication to effect copying. This latter
  revision of the definition of replication is crucial to obtaining
  the expected identity $!!P \sim !P$.
\end{remark}

\begin{remark}\label{rem:paradoxical_combinator}
  The reader familiar with the lambda calculus will have noticed the
  similarity between $D$ and the paradoxical combinator.

  [Ed. note: the existence of this seems to suggest we have to be more
  restrictive on the set of processes and names we admit if we are to
  support no-cloning.]
\end{remark}

\subsubsection{Bisimulation}

The computational dynamics gives rise to another kind of equivalence,
the equivalence of computational behavior. As previously mentioned
this is typically captured \emph{via} some form of bisimulation.

% The notion we use in this paper is weak barbed bisimulation
% \cite{milner91polyadicpi}.

The notion we use in this paper is derived from weak barbed
bisimulation \cite{milner91polyadicpi}. 

\begin{definition}
An \emph{observation relation}, $\downarrow_{\mathcal N}$, over a set
of names, $\mathcal N$, is the smallest relation satisfying the rules
below.

\infrule[Out-barb]{y \in {\mathcal N}, \; x \nameeq y}
		  {\outputp{x}{v} \downarrow_{\mathcal N} x}
\infrule[Par-barb]{\mbox{$P\downarrow_{\mathcal N} x$ or $Q\downarrow_{\mathcal N} x$}}
		  {\binpar{P}{Q} \downarrow_{\mathcal N} x}

We write $P \Downarrow_{\mathcal N} x$ if there is $Q$ such that 
$P \wred Q$ and $Q \downarrow_{\mathcal N} x$.
\end{definition}

\begin{definition}
%\label{def.bbisim}
An  ${\mathcal N}$-\emph{barbed bisimulation} over a set of names, ${\mathcal N}$, is a symmetric binary relation 
${\mathcal S}_{\mathcal N}$ between agents such that $P\rel{S}_{\mathcal N}Q$ implies:
\begin{enumerate}
\item If $P \red P'$ then $Q \wred Q'$ and $P'\rel{S}_{\mathcal N} Q'$.
\item If $P\downarrow_{\mathcal N} x$, then $Q\Downarrow_{\mathcal N} x$.
\end{enumerate}
$P$ is ${\mathcal N}$-barbed bisimilar to $Q$, written
$P \wbbisim_{\mathcal N} Q$, if $P \rel{S}_{\mathcal N} Q$ for some ${\mathcal N}$-barbed bisimulation ${\mathcal S}_{\mathcal N}$.
\end{definition}

$\mathcal{R} \subseteq \pi \times \pi$

$P \mathcal{R} Q => \forall P'. P \red P' \Rightarrow \exists Q'. Q \red Q', P' \mathcal{R} Q'$

$P \vdash x \Rightarrow Q \vdash x$

\begin{mathpar}
  \inferrule*[lab=Out-barb]{x \nameeq y}{{y}!\langle{Q}\rangle \vdash x}
  \and
  \inferrule*[lab=Par-barb]{\mbox{$P\vdash x$ or $Q\vdash x$}}{\binpar{P}{Q} \vdash x}
\end{mathpar}

\subsubsection{Contexts}

One of the principle advantages of computational calculi like the
$\pi$-calculus is a well-defined notion of context,
contextual-equivalence and a correlation between
contextual-equivalence and notions of bisimulation. The notion of
context allows the decomposition of a process into (sub-)process and
its syntactic environment, its context. Thus, a context may be
thought of as a process with a ``hole'' (written $\Box$) in it. The
application of a context $M$ to a process $P$, written $M[P]$, is
tantamount to filling the hole in $M$ with $P$. In this paper we do
not need the full weight of this theory, but do make use of the notion
of context in the proof the main theorem. 

\begin{mathpar}
  \inferrule* [lab=summation] {} {{M_{M},M_{N}} \bc \Box \;|\; x.M_{A} \;|\; M_{M}+M_{N}}
  \and
  \inferrule* [lab=agent] {} {{M_{A}} \bc (\vec{x})M_{P} \;| \; \clift{P_0,\ldots,M_{P},\ldots,P_N}}
  \and \\
  \inferrule* [lab=process] {} {{M_{P}} \bc M_{N} \;| \;P|M_{P} }
\end{mathpar} 

\begin{mathpar}
  \inferrule* [lab=sychronization] {} {M_{N} \bc \Box \;|\; x?M_{F} \;|\; x!M_{C}}
  \and
  \inferrule* [lab=abstraction] {} {{M_{F}} \bc (x)M_{P} }
  \and
  \inferrule* [lab=concretion] {} {{M_{C}} \bc \langle M_{P} \rangle }
  \and \\
  \inferrule* [lab=process] {} {{M_{P}} \bc M_{N} \;| \;P|M_{P} }
\end{mathpar}

\begin{definition}[contextual application] Given a context $M$, and
  process $P$, we define the \emph{contextual application}, $M[P] :=
  M\{P/\Box\}$. That is, the contextual application of M to P is the
  substitution of $P$ for $\Box$ in $M$.
\end{definition}

$\meaningof{-} : L \to \mathcal{P}(\pi)$

\begin{mathpar}
  \inferrule* [lab=collection] {} {\meaningof{true} = \pi, \and \meaningof{~E} = \pi \setminus \meaningof{E}, \and \meaningof{E_{1} \& E_{2}} = \meaningof{E_{1}} \cap \meaningof{E_{2}}}
\end{mathpar}

\begin{mathpar}
  \inferrule* [lab=structure] {} {\meaningof{0} = \{ P \in \pi | P \equiv 0 \}, \and \\ \meaningof{E_1 | E_2} = \{ P \in \pi | P \equiv P_{1} | P_{2}, P_{1} \in \meaningof{E_{1}}, P_{2} \in \meaningof{E_2}\} }
\end{mathpar}

\begin{mathpar}
 \inferrule* [lab=behavior] {} {\meaningof{\langle a?b \rangle E} = \{ P \in \pi | P \equiv Q | u?(y)P', \\ \and \\\\ \and \\ \;\;\; u \in \meaningof{a}, \forall z.P'\{z/y\} \in \meaningof{E\{z/b\}}\}, \and \\ \meaningof{a!E} = \{ P \in \pi | P \equiv Q | x!\langle P' \rangle, x \in \meaningof{a} P' \in \meaningof{E}\} }
\end{mathpar}

\begin{mathpar}
 \inferrule* [lab=nominal] {} {\meaningof{\quotep{E}} = \{ \quotep{P} \in \quotep{\pi} | P \in \meaningof{E} \}, \and \meaningof{\quotep{P}} = \{ \quotep{Q} \in \quotep{\pi} | P \equiv Q \} \and \\ \meaningof{@\quotep{E}} = \{ P \in \pi | P \equiv @x, x \in \meaningof{E} \}}
\end{mathpar}

\begin{eqnarray*}
  \\
  \meaningof{-} : TS \to ST
\end{eqnarray*}

\begin{eqnarray*}
  \\
  L : TS \to ST
\end{eqnarray*}

\begin{eqnarray*}
  \\
  P \models E \iff P \in \meaningof{E}
\end{eqnarray*}

\begin{eqnarray*}
  P \approx_{L} Q \iff \forall E \in L. P \models E \iff Q \models E
\end{eqnarray*}

\begin{eqnarray*}
  P \approx_{K} Q
\end{eqnarray*}

\begin{eqnarray*}
  P \approx Q
\end{eqnarray*}

$\approx_{K} = \approx = \approx_{L}$

\subsubsection{Contextual duality}

Note that contexts extend the quotation operation to a family of
operations from processes to names. Given a context, $M$, we can
define a \emph{nominal context}, $\quotep{M}$ by $\quotep{M}[P] :=
\quotep{M[P]}$. To foreshadow what is to come we observe that these
operations enjoy a duality with processes very much like the duality
between vectors and maps from vectors to scalars.

Further, because the calculus is essentially higher-order, we have a
correspondence between contexts and processes. More specifically,
given a name $x$ and a context $M$ we can construct $M^{*}_{x}$ such
that 

\begin{mathpar}
  M^{*}_{x} | \lift{x}{P} \red M[P]
\end{mathpar}

namely,

\begin{mathpar}
  M^{*}_{x} := x?(u).M[\dropn{u}]
\end{mathpar}

The dependence of $M^{*}_{x}$ on a name makes it an abstraction, 

\begin{mathpar}
  M^{*} := (x)x?(u).M[\dropn{u}]
\end{mathpar}

\subsection{Additional notation}

It will sometimes be convenient to denote the process a name
quotes. We already have the notation $x = \quotep{P}$, but it will be
convenient to introduce an alternate notation, $\procn{x}$, when we
want to emphasize the connection to the use of the name. Note that, by
virtue of name equivalence, $\quotep{\procn{x}} \nameeq x$; so, the
notation is consistent with previous definitions.

Further, because names have structure it is possible to effect
substitutions on the basis of that structure. This means we need to
upgrade our notation for substitutions, which we accomplish by
adapting comprehension notation. Thus,

\begin{mathpar}
  P\{ y / x : x \in S \}
\end{mathpar}

is interpreted to mean the process derived from P by replacing (in a
capture-avoiding manner) each occurrence of $x$ in $S$ by $y$. For example,

\begin{mathpar}
  P\{ \quotep{\procn{x}|\procn{x}} / x : x \in \freenames{P} \}
\end{mathpar}

will replace each (occurrence) of a free name $x$ in $P$ by
$\quotep{\procn{x}|\procn{x}}$.

Also, we will avail ourselves of the notation $x^{L}$ and $x^{R}$ to
denote injections of a name into disjoint copies of the name
space. There are numerous ways to accomplish this. One example can be
found in \cite{MeredithR05}. This notation overloads to vectors of
names: $\vec{x}^{\pi} := (x_{i}^{\pi} \; : \; 0 \leq i < |\vec{x}| )$ where $\pi \in \{L,R\}$.

We also use $P^{\Box} := P|\Box$.

In \cite{MeredithR05} an interpretation of the new operator is
given. It turns out that there are several possible interpretations
all enjoying the requisite algebraic properties of the operator (see
\cite{milner91polyadicpi}). We will therefore make liberal use of
$(\nu\; \vec{x})P$.

% subsection the_syntax_and_semantics_of_the_notation_system (end)   

\input{qm2pi.qmops} 

\input{qm2pi.sterngerlach} 

\input{qm2pi.metric} 

% section concurrent_process_calculi (end)

%\input{qm2pi.proofsketch}

% section proof sketch (end)

%\input{qm2pi.slviaknots} 

% section spatial logic via knots (end)

\input{qm2pi.conclusion}

% section conclusion (end)

%\input{qm2pi.dtcodes} 

% section wiring algorithm (end)

\input{qm2pi.ack} 

% section acknowledgments (end)

\newpage


\bibliographystyle{plain}   
\bibliography{../../biblios/main.bib}

\input{qm2pi.rhodetails}

\end{document}

 

% section notation (end)

\input{qm2pi.process.calculi} 

% section concurrent_process_calculi_and_spatial_logics_ (end)
    
%\documentclass[12pt]{llncs}
%\documentclass{jktr}

\usepackage[pdftex]{hyperref}                   
\usepackage {listings}
\usepackage {mathpartir}
\usepackage{bcprules}
%\usepackage{listings}
                       
\usepackage{graphicx} 
%\usepackage[margins=2.5cm,nohead,nofoot]{geometry}
%\usepackage{geometry}
\usepackage{amsfonts}
\usepackage{amstext}
\usepackage{latexsym}
\usepackage{amssymb}
\usepackage{color}


%\include{myPreamble}
\include{qm2pi.local} 

%\ifpdf
%\usepackage[pdftex]{graphicx}
%\else
%\usepackage{graphicx}
%\fi

 % \ifpdf
%  \usepackage{pdfsync}
%  \if


%\title{Brief Article}
%\author{David F. Snyder}
%\author{L.G. Meredith}

%\address{Dept. of Math., Texas State University--San Marcos, San Marcos, TX 78666}
       
\pagestyle{empty}


\begin{document}

\lstset{language=[Objective]Caml,frame=shadowbox}

\input{qm2pi.front}

% section front matter (end)

\input{qm2pi.intro} 
 
% section introduction (end)

% \input{qm2pi.knotations} 

% section notation (end)

\input{qm2pi.process.calculi} 

% section concurrent_process_calculi_and_spatial_logics_ (end)
    
%\input{qm2pi.knots2pi} 

%\input{qm2pi.trefoil} 

%\input{qm2pi.mainthm} 

% subsection basic_interpretation (end)

%\input{qm2pi.rho.presentation} 
\subsection{The syntax and semantics of the notation system}\label{sub:the_syntax_and_semantics_of_the_notation_system} % (fold)

We now summarize a technical presentation of the calculus that
embodies our theory of dynamics. The typical presentation of such a
calculus follows the style of giving generators and relations on
them. The grammar, below, describing term constructors, freely
generates the set of processes, $\Proc$. This set is then quotiented
by a relation known as structural congruence and it is over this set
that the notion of dynamics is expressed. This presentation is
essentially that of \cite{MeredithR05} with the addition of
polyadicity and summation. For readability we have relegated some of
the technical subtleties to an appendix.

\subsubsection{Process grammar}\label{subsub:process_grammar}

\begin{mathpar}
  \inferrule* [lab=synchronization] {} {{M} \bc \pzero \;|\; x?F \;|\; x!C }
  \and
  \inferrule* [lab=abstraction] {} {{F} \bc (x)P}
  \and
  \inferrule* [lab=concretion] {} {{C} \bc \langle Q \rangle}
  \and
  \inferrule* [lab=process] {} {{P,Q} \bc M \;| \;P|Q \;|\; @{x}}
  \and
  \inferrule* [lab=name] {} {{x} \bc \quotep{P}}
\end{mathpar} 

Note that $\vec{x}$ (resp. $\vec{P}$) denotes a vector of names
(resp. processes) of length $|\vec{x}|$ (resp. $|\vec{P}|$). We adopt
the following useful abbreviations.

\begin{mathpar}
   x?(\vec{y}).P := x.(\vec{y})P \and  x\clift{\vec{P}} := x.\clift{\vec{P}}
   \and x!(y) := \lift{x}{\dropn{y}}
   \and \Pi_{i=0}^{n-1}P_i := P_0 | \ldots | P_{n-1}
\end{mathpar}

\subsubsection{Structural congruence}

\paragraph{Free and bound names and alpha-equivalence.} At the
core of structural equivalence is alpha-equivalence which identifies
process that are the same up to a change of variable. Formally, we
recognize the distinction between free and bound names. The free names
of a process, $\freenames{P}$, may be calculated recursively as
follows:

\begin{mathpar}
\freenames{\pzero} := \emptyset
  \and \\
  \freenames{x?(y).P} := \{ x \} \cup (\freenames{P} \setminus \{ y \})
  \and 
  \freenames{x!\langle P \rangle} := \{ x \} \cup \{ P \} 
  \and \\
  \freenames{P|Q} := \freenames{P} \cup \freenames{Q}
  \and \\
  \freenames{@{x}} := \{ x \}
\end{mathpar}

$\pi$
$\quotep{\pi}$

$\freenames{-} : \pi \to \mathcal{P}(\quotep{\pi})$

\begin{eqnarray*}
  \freenames{\pzero} & := & \emptyset \\
  \freenames{x?(y).P} & := & \{ x \} \cup (\freenames{P} \setminus \{ y \}) \\
  \freenames{x!\langle P \rangle} & := & \{ x \} \cup \{ P \} \\
  \freenames{P|Q} & := & \freenames{P} \cup \freenames{Q} \\
  \freenames{\dropn{x}} & := & \{ x \}
\end{eqnarray*}

The bound names of a process, $\boundnames{P}$, are those names occurring in $P$
that are not free. For example, in $x?(y).0$, the name $x$ is free, while $y$ is bound.

\begin{mathpar}
  \inferrule* [lab=monoidal-laws] {} { P|Q \equiv Q|P \and P|0 \equiv P \and P|(Q|R) \equiv (P|Q)|R }
\end{mathpar}

\begin{mathpar}
  \inferrule* [lab=alpha-equivalence] {} { (x)P \equiv (y)P\{y/x\} \and y \not\in \freenames{P} }
\end{mathpar}

\begin{definition}
Then two processes, $P,Q$, are alpha-equivalent if $P = Q\{\vec{y}/\vec{x}\}$ for
some $\vec{x} \in \boundnames{Q},\vec{y} \in \boundnames{P}$, where $Q\{\vec{y}/\vec{x}\}$
denotes the capture-avoiding substitution of $\vec{y}$ for $\vec{x}$ in $Q$.
\end{definition}

\begin{definition}
  The {\em structural congruence} \cite{SangiorgiWalker} , $\equiv$,
  between processes is the least congruence containing
  alpha-equivalence, satisfying the abelian monoid laws
  (associativity, commutativity and $\pzero$ as identity) for parallel
  composition $|$ and for summation $+$.
\end{definition}

\subsection{Name equivalence}

We take name equivalence, written $\nameeq$, to be the smallest
equivalence relation generated by the following rules.

\begin{mathpar}
\inferrule*[lab=Quote-drop]
{ }
{ \quotep{@{x}} \nameeq x }

\inferrule*[lab=Struct-equiv]
{ P \scong Q }
{ \quotep{P} \nameeq \quotep{Q} }
\end{mathpar}

The astute reader will have noticed that the mutual recursion of names
and processes imposes a mutual recursion on alpha-equivalence and
structural equivalence via name-equivalence. Fortunately, all of this
works out pleasantly and we may calculate in the natural way, free of
concern. The reader interested in the details is referred to the
appendix \ref{appendix:rho_details}.

\subsection{Substitution}

We use $\Proc$ for the set of processes, $\QProc$ for the set of
names, and $\id{\{}\vec{y} / \vec{x} \id{\}}$ to denote partial maps,
$s : \QProc \rightarrow \QProc$. A map, $s$ lifts, uniquely, to a map
on process terms, $\widehat{s} : \Proc \rightarrow \Proc$ by the
following equations.

\begin{mathpar}
  (0) \psubstp{Q}{P} := 0 \\
  (R \juxtap S) \psubstp{Q}{P}
  :=    
  (R)\psubstp{Q}{P} \juxtap (S) \psubstp{Q}{P} \\
  (x?(y).R) \psubstp{Q}{P}    
  :=    
  (x)\substp{Q}{P} (z)\concat( (R \psubstn{z}{y}) \psubstp{Q}{P} ) \\
  (\lift{x}{R}) \psubstp{Q}{P}  
  :=
  \lift{(x)\substp{Q}{P}}{ R \psubstp{Q}{P} } \\
%   (\dropn{x})  \psubstp{Q}{P}       
%   := 
%   \left\{ 
%     \begin{array}{ccc} 
%       \dropn{\quotep{Q}} & & x \nameeq \quotep{P} \\
%       \dropn{x} & & otherwise \\
%     \end{array}
%   \right. 
  (\dropn{x})  \psubstp{Q}{P}       
  := 
  \left\{ 
    \begin{array}{ccc} 
      Q & & x \nameeq \quotep{P} \\
      \dropn{x} & & otherwise \\
    \end{array}
  \right.
\end{mathpar}
 

where

\begin{eqnarray}
  (x)\id{\{} \lpquote Q \rpquote / \lpquote P \rpquote \id{\}}            = 
  \left\{ 
    \begin{array}{ccc}
      \lpquote Q \rpquote & & x \nameeq \lpquote P \rpquote \\
      x & & otherwise \\
    \end{array}
  \right. \nonumber
\end{eqnarray}

and $z$ is chosen distinct from $\quotep{P}$, $\quotep{Q}$, the free
names in $Q$, and all the names in $R$. Our $\alpha$-equivalence will
be built in the standard way from this substitution.

\begin{remark}\label{rem:no_self_referential_names}
  One consequence of these definitions is that $\forall P. \quotep{P}
  \not\in \freenames{P}$.
\end{remark}

\subsection{ Dynamic quote: an example }

Anticipating something of what's to come, consider applying the
substitution, $\widehat{\id{\{}u / z \id{\}}}$, to the following pair
of processes, $\lift{w}{y!(z)}$ and $w[ \lpquote y!(z) \rpquote ]$.

\begin{eqnarray}
	\lift{w}{y!(z)}\widehat{\id{\{}u / z \id{\}}}
		& = &
		\lift{w}{y!(u)} \nonumber\\
	w[ \lpquote y!(z) \rpquote ] \widehat{ \id{\{}u / z \id{\}} }
		& = &
		w[ \lpquote y!(z) \rpquote ] \nonumber
\end{eqnarray}

Because the body of the process between quotes is impervious to
substitution, we get radically different answers. In fact, by
examining the first process in an input context,
e.g. $x?(z).\lift{w}{y!(z)}$, we see that the process under the lift
operator may be shaped by prefixed inputs binding a name inside it. In
this sense, the lift operator will be seen as a way to dynamically
construct processes before reifying them as names.

Finally equipped with these standard features we can present the
dynamics of the calculus.

\subsubsection{Operational semantics} 

Finally, we introduce the computational dynamics. What marks these
algebras as distinct from other more traditionally studied algebraic
structures, e.g. vector spaces or polynomial rings, is the manner in
which dynamics is captured. In traditional structures, dynamics is typically
expressed through morphisms between such structures, as in linear maps
between vector spaces or morphisms between rings. In algebras
associated with the semantics of computation, the dynamics is
expressed as part of the algebraic structure itself, through a
reduction reduction relation typically denoted by $\red$. Below, we
give a recursive presentation of this relation for the calculus used
in the encoding.

$\red \subseteq \pi \times \pi$
$\red : \pi \to \mathcal{P}(\pi)$

\begin{mathpar}
  \inferrule* [lab=Comm] { \textsf{match}( x_{src}, x_{trgt} ) } { x_{trgt}?(y)P \; | \; x_{src}!\langle {Q} \rangle \red P\{\quotep{Q}/y}\} }
  \and \\
  \inferrule* [lab=Par] {{P} \red {P}'} {{{P} | {Q}} \red {{P}' | {Q}}}
  \and
  \inferrule* [lab=Equiv]{{{P} \scong {P}'} \andalso {{P}' \red {Q}'} \andalso {{Q}' \scong {Q}}}{{P} \red {Q}}
\end{mathpar}

\begin{eqnarray*}
  match_{\equiv} (\quotep{P},\quotep{Q}) & := & P \equiv Q \\
  match_{\dagger}(\quotep{P},\quotep{Q}) & := & \forall R. P|Q \red^{*} R => R \red^{*} 0 \\
  match_{K}(\quotep{P},\quotep{Q}) & := & K \mbox{ for some context } K
\end{eqnarray*}

$u?(x)P | u!\langle Q \rangle \red P\{\quotep{Q}/x\}$

%We write $\wred$ for $\red^*$, and $P\red$ if $\exists Q $ such that $ P \red Q$.
We write $P\red$ if $\exists Q $ such that $ P \red Q$ and $P\not\red$, otherwise.

\section{Replication}

As mentioned before, it is known that replication (and hence
recursion) can be implemented in a higher-order process algebra
\cite{SangiorgiWalker}. As our first example of calculation with the
machinery thus far presented we give the construction explicitly in
the {\rhoc}.

\begin{eqnarray}
	D_{x} & := & \prefix{x}{y}{(\binpar{\outputp{x}{y}}{@{y}})} \nonumber\\
	\bangp_{x}{P} & := & \binpar{{x}!\langle{\binpar{D_{x}}{P}}\rangle}{D_{x}} \nonumber
\end{eqnarray}

\begin{eqnarray}
	\bangp_{x}{P} & & \nonumber\\
	=
	& {x}!\langle{(\prefix{x}{y}{(\outputp{x}{y} | @{y})) | P}}\rangle 
	      | \prefix{x}{y}{(\outputp{x}{y} | @{y})} & \nonumber\\
	\red
	& (\outputp{x}{y} | @{y})\substn{\quotep{(\prefix{x}{y}{(@{y} | \outputp{x}{y})) | P}}}{y} & \nonumber\\
	=
	& \outputp{x}{\quotep{(\prefix{x}{y}{(\outputp{x}{y} | @{y})) | P}}}
	  | {(\prefix{x}{y}{(\outputp{x}{y} | @{y})) | P}} & \nonumber\\
	\red
	& \ldots & \nonumber\\
	\red^*
	& P | P | \ldots & \nonumber
\end{eqnarray}

Of course, this encoding, as an implementation, runs away, unfolding
$\bangp{P}$ eagerly. A lazier and more implementable replication
operator, restricted to input-guarded processes, may be obtained as follows.

\begin{eqnarray}
\bangp{\prefix{u}{v}{P}} 
	:= 
	\binpar{\lift{x}{\prefix{u}{v}{(\binpar{D(x)}{P})}}}{D(x)} \nonumber
\end{eqnarray}

\begin{remark}
  Note that the lazier definition still does not deal with summation
  or mixed summation (i.e. sums over input and output). The reader is
  invited to construct definitions of replication that deal with these
  features. 

  Further, the definitions are parameterized in a name, $x$. Can you,
  gentle reader, make a definition that eliminates this parameter and
  guarantees no accidental interaction between the replication
  machinery and the process being replicated -- i.e. no accidental
  sharing of names used by the process to get its work done and the
  name(s) used by the replication to effect copying. This latter
  revision of the definition of replication is crucial to obtaining
  the expected identity $!!P \sim !P$.
\end{remark}

\begin{remark}\label{rem:paradoxical_combinator}
  The reader familiar with the lambda calculus will have noticed the
  similarity between $D$ and the paradoxical combinator.

  [Ed. note: the existence of this seems to suggest we have to be more
  restrictive on the set of processes and names we admit if we are to
  support no-cloning.]
\end{remark}

\subsubsection{Bisimulation}

The computational dynamics gives rise to another kind of equivalence,
the equivalence of computational behavior. As previously mentioned
this is typically captured \emph{via} some form of bisimulation.

% The notion we use in this paper is weak barbed bisimulation
% \cite{milner91polyadicpi}.

The notion we use in this paper is derived from weak barbed
bisimulation \cite{milner91polyadicpi}. 

\begin{definition}
An \emph{observation relation}, $\downarrow_{\mathcal N}$, over a set
of names, $\mathcal N$, is the smallest relation satisfying the rules
below.

\infrule[Out-barb]{y \in {\mathcal N}, \; x \nameeq y}
		  {\outputp{x}{v} \downarrow_{\mathcal N} x}
\infrule[Par-barb]{\mbox{$P\downarrow_{\mathcal N} x$ or $Q\downarrow_{\mathcal N} x$}}
		  {\binpar{P}{Q} \downarrow_{\mathcal N} x}

We write $P \Downarrow_{\mathcal N} x$ if there is $Q$ such that 
$P \wred Q$ and $Q \downarrow_{\mathcal N} x$.
\end{definition}

\begin{definition}
%\label{def.bbisim}
An  ${\mathcal N}$-\emph{barbed bisimulation} over a set of names, ${\mathcal N}$, is a symmetric binary relation 
${\mathcal S}_{\mathcal N}$ between agents such that $P\rel{S}_{\mathcal N}Q$ implies:
\begin{enumerate}
\item If $P \red P'$ then $Q \wred Q'$ and $P'\rel{S}_{\mathcal N} Q'$.
\item If $P\downarrow_{\mathcal N} x$, then $Q\Downarrow_{\mathcal N} x$.
\end{enumerate}
$P$ is ${\mathcal N}$-barbed bisimilar to $Q$, written
$P \wbbisim_{\mathcal N} Q$, if $P \rel{S}_{\mathcal N} Q$ for some ${\mathcal N}$-barbed bisimulation ${\mathcal S}_{\mathcal N}$.
\end{definition}

$\mathcal{R} \subseteq \pi \times \pi$

$P \mathcal{R} Q => \forall P'. P \red P' \Rightarrow \exists Q'. Q \red Q', P' \mathcal{R} Q'$

$P \vdash x \Rightarrow Q \vdash x$

\begin{mathpar}
  \inferrule*[lab=Out-barb]{x \nameeq y}{{y}!\langle{Q}\rangle \vdash x}
  \and
  \inferrule*[lab=Par-barb]{\mbox{$P\vdash x$ or $Q\vdash x$}}{\binpar{P}{Q} \vdash x}
\end{mathpar}

\subsubsection{Contexts}

One of the principle advantages of computational calculi like the
$\pi$-calculus is a well-defined notion of context,
contextual-equivalence and a correlation between
contextual-equivalence and notions of bisimulation. The notion of
context allows the decomposition of a process into (sub-)process and
its syntactic environment, its context. Thus, a context may be
thought of as a process with a ``hole'' (written $\Box$) in it. The
application of a context $M$ to a process $P$, written $M[P]$, is
tantamount to filling the hole in $M$ with $P$. In this paper we do
not need the full weight of this theory, but do make use of the notion
of context in the proof the main theorem. 

\begin{mathpar}
  \inferrule* [lab=summation] {} {{M_{M},M_{N}} \bc \Box \;|\; x.M_{A} \;|\; M_{M}+M_{N}}
  \and
  \inferrule* [lab=agent] {} {{M_{A}} \bc (\vec{x})M_{P} \;| \; \clift{P_0,\ldots,M_{P},\ldots,P_N}}
  \and \\
  \inferrule* [lab=process] {} {{M_{P}} \bc M_{N} \;| \;P|M_{P} }
\end{mathpar} 

\begin{mathpar}
  \inferrule* [lab=sychronization] {} {M_{N} \bc \Box \;|\; x?M_{F} \;|\; x!M_{C}}
  \and
  \inferrule* [lab=abstraction] {} {{M_{F}} \bc (x)M_{P} }
  \and
  \inferrule* [lab=concretion] {} {{M_{C}} \bc \langle M_{P} \rangle }
  \and \\
  \inferrule* [lab=process] {} {{M_{P}} \bc M_{N} \;| \;P|M_{P} }
\end{mathpar}

\begin{definition}[contextual application] Given a context $M$, and
  process $P$, we define the \emph{contextual application}, $M[P] :=
  M\{P/\Box\}$. That is, the contextual application of M to P is the
  substitution of $P$ for $\Box$ in $M$.
\end{definition}

$\meaningof{-} : L \to \mathcal{P}(\pi)$

\begin{mathpar}
  \inferrule* [lab=collection] {} {\meaningof{true} = \pi, \and \meaningof{~E} = \pi \setminus \meaningof{E}, \and \meaningof{E_{1} \& E_{2}} = \meaningof{E_{1}} \cap \meaningof{E_{2}}}
\end{mathpar}

\begin{mathpar}
  \inferrule* [lab=structure] {} {\meaningof{0} = \{ P \in \pi | P \equiv 0 \}, \and \\ \meaningof{E_1 | E_2} = \{ P \in \pi | P \equiv P_{1} | P_{2}, P_{1} \in \meaningof{E_{1}}, P_{2} \in \meaningof{E_2}\} }
\end{mathpar}

\begin{mathpar}
 \inferrule* [lab=behavior] {} {\meaningof{\langle a?b \rangle E} = \{ P \in \pi | P \equiv Q | u?(y)P', \\ \and \\\\ \and \\ \;\;\; u \in \meaningof{a}, \forall z.P'\{z/y\} \in \meaningof{E\{z/b\}}\}, \and \\ \meaningof{a!E} = \{ P \in \pi | P \equiv Q | x!\langle P' \rangle, x \in \meaningof{a} P' \in \meaningof{E}\} }
\end{mathpar}

\begin{mathpar}
 \inferrule* [lab=nominal] {} {\meaningof{\quotep{E}} = \{ \quotep{P} \in \quotep{\pi} | P \in \meaningof{E} \}, \and \meaningof{\quotep{P}} = \{ \quotep{Q} \in \quotep{\pi} | P \equiv Q \} \and \\ \meaningof{@\quotep{E}} = \{ P \in \pi | P \equiv @x, x \in \meaningof{E} \}}
\end{mathpar}

\begin{eqnarray*}
  \\
  \meaningof{-} : TS \to ST
\end{eqnarray*}

\begin{eqnarray*}
  \\
  L : TS \to ST
\end{eqnarray*}

\begin{eqnarray*}
  \\
  P \models E \iff P \in \meaningof{E}
\end{eqnarray*}

\begin{eqnarray*}
  P \approx_{L} Q \iff \forall E \in L. P \models E \iff Q \models E
\end{eqnarray*}

\begin{eqnarray*}
  P \approx_{K} Q
\end{eqnarray*}

\begin{eqnarray*}
  P \approx Q
\end{eqnarray*}

$\approx_{K} = \approx = \approx_{L}$

\subsubsection{Contextual duality}

Note that contexts extend the quotation operation to a family of
operations from processes to names. Given a context, $M$, we can
define a \emph{nominal context}, $\quotep{M}$ by $\quotep{M}[P] :=
\quotep{M[P]}$. To foreshadow what is to come we observe that these
operations enjoy a duality with processes very much like the duality
between vectors and maps from vectors to scalars.

Further, because the calculus is essentially higher-order, we have a
correspondence between contexts and processes. More specifically,
given a name $x$ and a context $M$ we can construct $M^{*}_{x}$ such
that 

\begin{mathpar}
  M^{*}_{x} | \lift{x}{P} \red M[P]
\end{mathpar}

namely,

\begin{mathpar}
  M^{*}_{x} := x?(u).M[\dropn{u}]
\end{mathpar}

The dependence of $M^{*}_{x}$ on a name makes it an abstraction, 

\begin{mathpar}
  M^{*} := (x)x?(u).M[\dropn{u}]
\end{mathpar}

\subsection{Additional notation}

It will sometimes be convenient to denote the process a name
quotes. We already have the notation $x = \quotep{P}$, but it will be
convenient to introduce an alternate notation, $\procn{x}$, when we
want to emphasize the connection to the use of the name. Note that, by
virtue of name equivalence, $\quotep{\procn{x}} \nameeq x$; so, the
notation is consistent with previous definitions.

Further, because names have structure it is possible to effect
substitutions on the basis of that structure. This means we need to
upgrade our notation for substitutions, which we accomplish by
adapting comprehension notation. Thus,

\begin{mathpar}
  P\{ y / x : x \in S \}
\end{mathpar}

is interpreted to mean the process derived from P by replacing (in a
capture-avoiding manner) each occurrence of $x$ in $S$ by $y$. For example,

\begin{mathpar}
  P\{ \quotep{\procn{x}|\procn{x}} / x : x \in \freenames{P} \}
\end{mathpar}

will replace each (occurrence) of a free name $x$ in $P$ by
$\quotep{\procn{x}|\procn{x}}$.

Also, we will avail ourselves of the notation $x^{L}$ and $x^{R}$ to
denote injections of a name into disjoint copies of the name
space. There are numerous ways to accomplish this. One example can be
found in \cite{MeredithR05}. This notation overloads to vectors of
names: $\vec{x}^{\pi} := (x_{i}^{\pi} \; : \; 0 \leq i < |\vec{x}| )$ where $\pi \in \{L,R\}$.

We also use $P^{\Box} := P|\Box$.

In \cite{MeredithR05} an interpretation of the new operator is
given. It turns out that there are several possible interpretations
all enjoying the requisite algebraic properties of the operator (see
\cite{milner91polyadicpi}). We will therefore make liberal use of
$(\nu\; \vec{x})P$.

% subsection the_syntax_and_semantics_of_the_notation_system (end)   

\input{qm2pi.qmops} 

\input{qm2pi.sterngerlach} 

\input{qm2pi.metric} 

% section concurrent_process_calculi (end)

%\input{qm2pi.proofsketch}

% section proof sketch (end)

%\input{qm2pi.slviaknots} 

% section spatial logic via knots (end)

\input{qm2pi.conclusion}

% section conclusion (end)

%\input{qm2pi.dtcodes} 

% section wiring algorithm (end)

\input{qm2pi.ack} 

% section acknowledgments (end)

\newpage


\bibliographystyle{plain}   
\bibliography{../../biblios/main.bib}

\input{qm2pi.rhodetails}

\end{document}

 

%\documentclass[12pt]{llncs}
%\documentclass{jktr}

\usepackage[pdftex]{hyperref}                   
\usepackage {listings}
\usepackage {mathpartir}
\usepackage{bcprules}
%\usepackage{listings}
                       
\usepackage{graphicx} 
%\usepackage[margins=2.5cm,nohead,nofoot]{geometry}
%\usepackage{geometry}
\usepackage{amsfonts}
\usepackage{amstext}
\usepackage{latexsym}
\usepackage{amssymb}
\usepackage{color}


%\include{myPreamble}
\include{qm2pi.local} 

%\ifpdf
%\usepackage[pdftex]{graphicx}
%\else
%\usepackage{graphicx}
%\fi

 % \ifpdf
%  \usepackage{pdfsync}
%  \if


%\title{Brief Article}
%\author{David F. Snyder}
%\author{L.G. Meredith}

%\address{Dept. of Math., Texas State University--San Marcos, San Marcos, TX 78666}
       
\pagestyle{empty}


\begin{document}

\lstset{language=[Objective]Caml,frame=shadowbox}

\input{qm2pi.front}

% section front matter (end)

\input{qm2pi.intro} 
 
% section introduction (end)

% \input{qm2pi.knotations} 

% section notation (end)

\input{qm2pi.process.calculi} 

% section concurrent_process_calculi_and_spatial_logics_ (end)
    
%\input{qm2pi.knots2pi} 

%\input{qm2pi.trefoil} 

%\input{qm2pi.mainthm} 

% subsection basic_interpretation (end)

%\input{qm2pi.rho.presentation} 
\subsection{The syntax and semantics of the notation system}\label{sub:the_syntax_and_semantics_of_the_notation_system} % (fold)

We now summarize a technical presentation of the calculus that
embodies our theory of dynamics. The typical presentation of such a
calculus follows the style of giving generators and relations on
them. The grammar, below, describing term constructors, freely
generates the set of processes, $\Proc$. This set is then quotiented
by a relation known as structural congruence and it is over this set
that the notion of dynamics is expressed. This presentation is
essentially that of \cite{MeredithR05} with the addition of
polyadicity and summation. For readability we have relegated some of
the technical subtleties to an appendix.

\subsubsection{Process grammar}\label{subsub:process_grammar}

\begin{mathpar}
  \inferrule* [lab=synchronization] {} {{M} \bc \pzero \;|\; x?F \;|\; x!C }
  \and
  \inferrule* [lab=abstraction] {} {{F} \bc (x)P}
  \and
  \inferrule* [lab=concretion] {} {{C} \bc \langle Q \rangle}
  \and
  \inferrule* [lab=process] {} {{P,Q} \bc M \;| \;P|Q \;|\; @{x}}
  \and
  \inferrule* [lab=name] {} {{x} \bc \quotep{P}}
\end{mathpar} 

Note that $\vec{x}$ (resp. $\vec{P}$) denotes a vector of names
(resp. processes) of length $|\vec{x}|$ (resp. $|\vec{P}|$). We adopt
the following useful abbreviations.

\begin{mathpar}
   x?(\vec{y}).P := x.(\vec{y})P \and  x\clift{\vec{P}} := x.\clift{\vec{P}}
   \and x!(y) := \lift{x}{\dropn{y}}
   \and \Pi_{i=0}^{n-1}P_i := P_0 | \ldots | P_{n-1}
\end{mathpar}

\subsubsection{Structural congruence}

\paragraph{Free and bound names and alpha-equivalence.} At the
core of structural equivalence is alpha-equivalence which identifies
process that are the same up to a change of variable. Formally, we
recognize the distinction between free and bound names. The free names
of a process, $\freenames{P}$, may be calculated recursively as
follows:

\begin{mathpar}
\freenames{\pzero} := \emptyset
  \and \\
  \freenames{x?(y).P} := \{ x \} \cup (\freenames{P} \setminus \{ y \})
  \and 
  \freenames{x!\langle P \rangle} := \{ x \} \cup \{ P \} 
  \and \\
  \freenames{P|Q} := \freenames{P} \cup \freenames{Q}
  \and \\
  \freenames{@{x}} := \{ x \}
\end{mathpar}

$\pi$
$\quotep{\pi}$

$\freenames{-} : \pi \to \mathcal{P}(\quotep{\pi})$

\begin{eqnarray*}
  \freenames{\pzero} & := & \emptyset \\
  \freenames{x?(y).P} & := & \{ x \} \cup (\freenames{P} \setminus \{ y \}) \\
  \freenames{x!\langle P \rangle} & := & \{ x \} \cup \{ P \} \\
  \freenames{P|Q} & := & \freenames{P} \cup \freenames{Q} \\
  \freenames{\dropn{x}} & := & \{ x \}
\end{eqnarray*}

The bound names of a process, $\boundnames{P}$, are those names occurring in $P$
that are not free. For example, in $x?(y).0$, the name $x$ is free, while $y$ is bound.

\begin{mathpar}
  \inferrule* [lab=monoidal-laws] {} { P|Q \equiv Q|P \and P|0 \equiv P \and P|(Q|R) \equiv (P|Q)|R }
\end{mathpar}

\begin{mathpar}
  \inferrule* [lab=alpha-equivalence] {} { (x)P \equiv (y)P\{y/x\} \and y \not\in \freenames{P} }
\end{mathpar}

\begin{definition}
Then two processes, $P,Q$, are alpha-equivalent if $P = Q\{\vec{y}/\vec{x}\}$ for
some $\vec{x} \in \boundnames{Q},\vec{y} \in \boundnames{P}$, where $Q\{\vec{y}/\vec{x}\}$
denotes the capture-avoiding substitution of $\vec{y}$ for $\vec{x}$ in $Q$.
\end{definition}

\begin{definition}
  The {\em structural congruence} \cite{SangiorgiWalker} , $\equiv$,
  between processes is the least congruence containing
  alpha-equivalence, satisfying the abelian monoid laws
  (associativity, commutativity and $\pzero$ as identity) for parallel
  composition $|$ and for summation $+$.
\end{definition}

\subsection{Name equivalence}

We take name equivalence, written $\nameeq$, to be the smallest
equivalence relation generated by the following rules.

\begin{mathpar}
\inferrule*[lab=Quote-drop]
{ }
{ \quotep{@{x}} \nameeq x }

\inferrule*[lab=Struct-equiv]
{ P \scong Q }
{ \quotep{P} \nameeq \quotep{Q} }
\end{mathpar}

The astute reader will have noticed that the mutual recursion of names
and processes imposes a mutual recursion on alpha-equivalence and
structural equivalence via name-equivalence. Fortunately, all of this
works out pleasantly and we may calculate in the natural way, free of
concern. The reader interested in the details is referred to the
appendix \ref{appendix:rho_details}.

\subsection{Substitution}

We use $\Proc$ for the set of processes, $\QProc$ for the set of
names, and $\id{\{}\vec{y} / \vec{x} \id{\}}$ to denote partial maps,
$s : \QProc \rightarrow \QProc$. A map, $s$ lifts, uniquely, to a map
on process terms, $\widehat{s} : \Proc \rightarrow \Proc$ by the
following equations.

\begin{mathpar}
  (0) \psubstp{Q}{P} := 0 \\
  (R \juxtap S) \psubstp{Q}{P}
  :=    
  (R)\psubstp{Q}{P} \juxtap (S) \psubstp{Q}{P} \\
  (x?(y).R) \psubstp{Q}{P}    
  :=    
  (x)\substp{Q}{P} (z)\concat( (R \psubstn{z}{y}) \psubstp{Q}{P} ) \\
  (\lift{x}{R}) \psubstp{Q}{P}  
  :=
  \lift{(x)\substp{Q}{P}}{ R \psubstp{Q}{P} } \\
%   (\dropn{x})  \psubstp{Q}{P}       
%   := 
%   \left\{ 
%     \begin{array}{ccc} 
%       \dropn{\quotep{Q}} & & x \nameeq \quotep{P} \\
%       \dropn{x} & & otherwise \\
%     \end{array}
%   \right. 
  (\dropn{x})  \psubstp{Q}{P}       
  := 
  \left\{ 
    \begin{array}{ccc} 
      Q & & x \nameeq \quotep{P} \\
      \dropn{x} & & otherwise \\
    \end{array}
  \right.
\end{mathpar}
 

where

\begin{eqnarray}
  (x)\id{\{} \lpquote Q \rpquote / \lpquote P \rpquote \id{\}}            = 
  \left\{ 
    \begin{array}{ccc}
      \lpquote Q \rpquote & & x \nameeq \lpquote P \rpquote \\
      x & & otherwise \\
    \end{array}
  \right. \nonumber
\end{eqnarray}

and $z$ is chosen distinct from $\quotep{P}$, $\quotep{Q}$, the free
names in $Q$, and all the names in $R$. Our $\alpha$-equivalence will
be built in the standard way from this substitution.

\begin{remark}\label{rem:no_self_referential_names}
  One consequence of these definitions is that $\forall P. \quotep{P}
  \not\in \freenames{P}$.
\end{remark}

\subsection{ Dynamic quote: an example }

Anticipating something of what's to come, consider applying the
substitution, $\widehat{\id{\{}u / z \id{\}}}$, to the following pair
of processes, $\lift{w}{y!(z)}$ and $w[ \lpquote y!(z) \rpquote ]$.

\begin{eqnarray}
	\lift{w}{y!(z)}\widehat{\id{\{}u / z \id{\}}}
		& = &
		\lift{w}{y!(u)} \nonumber\\
	w[ \lpquote y!(z) \rpquote ] \widehat{ \id{\{}u / z \id{\}} }
		& = &
		w[ \lpquote y!(z) \rpquote ] \nonumber
\end{eqnarray}

Because the body of the process between quotes is impervious to
substitution, we get radically different answers. In fact, by
examining the first process in an input context,
e.g. $x?(z).\lift{w}{y!(z)}$, we see that the process under the lift
operator may be shaped by prefixed inputs binding a name inside it. In
this sense, the lift operator will be seen as a way to dynamically
construct processes before reifying them as names.

Finally equipped with these standard features we can present the
dynamics of the calculus.

\subsubsection{Operational semantics} 

Finally, we introduce the computational dynamics. What marks these
algebras as distinct from other more traditionally studied algebraic
structures, e.g. vector spaces or polynomial rings, is the manner in
which dynamics is captured. In traditional structures, dynamics is typically
expressed through morphisms between such structures, as in linear maps
between vector spaces or morphisms between rings. In algebras
associated with the semantics of computation, the dynamics is
expressed as part of the algebraic structure itself, through a
reduction reduction relation typically denoted by $\red$. Below, we
give a recursive presentation of this relation for the calculus used
in the encoding.

$\red \subseteq \pi \times \pi$
$\red : \pi \to \mathcal{P}(\pi)$

\begin{mathpar}
  \inferrule* [lab=Comm] { \textsf{match}( x_{src}, x_{trgt} ) } { x_{trgt}?(y)P \; | \; x_{src}!\langle {Q} \rangle \red P\{\quotep{Q}/y}\} }
  \and \\
  \inferrule* [lab=Par] {{P} \red {P}'} {{{P} | {Q}} \red {{P}' | {Q}}}
  \and
  \inferrule* [lab=Equiv]{{{P} \scong {P}'} \andalso {{P}' \red {Q}'} \andalso {{Q}' \scong {Q}}}{{P} \red {Q}}
\end{mathpar}

\begin{eqnarray*}
  match_{\equiv} (\quotep{P},\quotep{Q}) & := & P \equiv Q \\
  match_{\dagger}(\quotep{P},\quotep{Q}) & := & \forall R. P|Q \red^{*} R => R \red^{*} 0 \\
  match_{K}(\quotep{P},\quotep{Q}) & := & K \mbox{ for some context } K
\end{eqnarray*}

$u?(x)P | u!\langle Q \rangle \red P\{\quotep{Q}/x\}$

%We write $\wred$ for $\red^*$, and $P\red$ if $\exists Q $ such that $ P \red Q$.
We write $P\red$ if $\exists Q $ such that $ P \red Q$ and $P\not\red$, otherwise.

\section{Replication}

As mentioned before, it is known that replication (and hence
recursion) can be implemented in a higher-order process algebra
\cite{SangiorgiWalker}. As our first example of calculation with the
machinery thus far presented we give the construction explicitly in
the {\rhoc}.

\begin{eqnarray}
	D_{x} & := & \prefix{x}{y}{(\binpar{\outputp{x}{y}}{@{y}})} \nonumber\\
	\bangp_{x}{P} & := & \binpar{{x}!\langle{\binpar{D_{x}}{P}}\rangle}{D_{x}} \nonumber
\end{eqnarray}

\begin{eqnarray}
	\bangp_{x}{P} & & \nonumber\\
	=
	& {x}!\langle{(\prefix{x}{y}{(\outputp{x}{y} | @{y})) | P}}\rangle 
	      | \prefix{x}{y}{(\outputp{x}{y} | @{y})} & \nonumber\\
	\red
	& (\outputp{x}{y} | @{y})\substn{\quotep{(\prefix{x}{y}{(@{y} | \outputp{x}{y})) | P}}}{y} & \nonumber\\
	=
	& \outputp{x}{\quotep{(\prefix{x}{y}{(\outputp{x}{y} | @{y})) | P}}}
	  | {(\prefix{x}{y}{(\outputp{x}{y} | @{y})) | P}} & \nonumber\\
	\red
	& \ldots & \nonumber\\
	\red^*
	& P | P | \ldots & \nonumber
\end{eqnarray}

Of course, this encoding, as an implementation, runs away, unfolding
$\bangp{P}$ eagerly. A lazier and more implementable replication
operator, restricted to input-guarded processes, may be obtained as follows.

\begin{eqnarray}
\bangp{\prefix{u}{v}{P}} 
	:= 
	\binpar{\lift{x}{\prefix{u}{v}{(\binpar{D(x)}{P})}}}{D(x)} \nonumber
\end{eqnarray}

\begin{remark}
  Note that the lazier definition still does not deal with summation
  or mixed summation (i.e. sums over input and output). The reader is
  invited to construct definitions of replication that deal with these
  features. 

  Further, the definitions are parameterized in a name, $x$. Can you,
  gentle reader, make a definition that eliminates this parameter and
  guarantees no accidental interaction between the replication
  machinery and the process being replicated -- i.e. no accidental
  sharing of names used by the process to get its work done and the
  name(s) used by the replication to effect copying. This latter
  revision of the definition of replication is crucial to obtaining
  the expected identity $!!P \sim !P$.
\end{remark}

\begin{remark}\label{rem:paradoxical_combinator}
  The reader familiar with the lambda calculus will have noticed the
  similarity between $D$ and the paradoxical combinator.

  [Ed. note: the existence of this seems to suggest we have to be more
  restrictive on the set of processes and names we admit if we are to
  support no-cloning.]
\end{remark}

\subsubsection{Bisimulation}

The computational dynamics gives rise to another kind of equivalence,
the equivalence of computational behavior. As previously mentioned
this is typically captured \emph{via} some form of bisimulation.

% The notion we use in this paper is weak barbed bisimulation
% \cite{milner91polyadicpi}.

The notion we use in this paper is derived from weak barbed
bisimulation \cite{milner91polyadicpi}. 

\begin{definition}
An \emph{observation relation}, $\downarrow_{\mathcal N}$, over a set
of names, $\mathcal N$, is the smallest relation satisfying the rules
below.

\infrule[Out-barb]{y \in {\mathcal N}, \; x \nameeq y}
		  {\outputp{x}{v} \downarrow_{\mathcal N} x}
\infrule[Par-barb]{\mbox{$P\downarrow_{\mathcal N} x$ or $Q\downarrow_{\mathcal N} x$}}
		  {\binpar{P}{Q} \downarrow_{\mathcal N} x}

We write $P \Downarrow_{\mathcal N} x$ if there is $Q$ such that 
$P \wred Q$ and $Q \downarrow_{\mathcal N} x$.
\end{definition}

\begin{definition}
%\label{def.bbisim}
An  ${\mathcal N}$-\emph{barbed bisimulation} over a set of names, ${\mathcal N}$, is a symmetric binary relation 
${\mathcal S}_{\mathcal N}$ between agents such that $P\rel{S}_{\mathcal N}Q$ implies:
\begin{enumerate}
\item If $P \red P'$ then $Q \wred Q'$ and $P'\rel{S}_{\mathcal N} Q'$.
\item If $P\downarrow_{\mathcal N} x$, then $Q\Downarrow_{\mathcal N} x$.
\end{enumerate}
$P$ is ${\mathcal N}$-barbed bisimilar to $Q$, written
$P \wbbisim_{\mathcal N} Q$, if $P \rel{S}_{\mathcal N} Q$ for some ${\mathcal N}$-barbed bisimulation ${\mathcal S}_{\mathcal N}$.
\end{definition}

$\mathcal{R} \subseteq \pi \times \pi$

$P \mathcal{R} Q => \forall P'. P \red P' \Rightarrow \exists Q'. Q \red Q', P' \mathcal{R} Q'$

$P \vdash x \Rightarrow Q \vdash x$

\begin{mathpar}
  \inferrule*[lab=Out-barb]{x \nameeq y}{{y}!\langle{Q}\rangle \vdash x}
  \and
  \inferrule*[lab=Par-barb]{\mbox{$P\vdash x$ or $Q\vdash x$}}{\binpar{P}{Q} \vdash x}
\end{mathpar}

\subsubsection{Contexts}

One of the principle advantages of computational calculi like the
$\pi$-calculus is a well-defined notion of context,
contextual-equivalence and a correlation between
contextual-equivalence and notions of bisimulation. The notion of
context allows the decomposition of a process into (sub-)process and
its syntactic environment, its context. Thus, a context may be
thought of as a process with a ``hole'' (written $\Box$) in it. The
application of a context $M$ to a process $P$, written $M[P]$, is
tantamount to filling the hole in $M$ with $P$. In this paper we do
not need the full weight of this theory, but do make use of the notion
of context in the proof the main theorem. 

\begin{mathpar}
  \inferrule* [lab=summation] {} {{M_{M},M_{N}} \bc \Box \;|\; x.M_{A} \;|\; M_{M}+M_{N}}
  \and
  \inferrule* [lab=agent] {} {{M_{A}} \bc (\vec{x})M_{P} \;| \; \clift{P_0,\ldots,M_{P},\ldots,P_N}}
  \and \\
  \inferrule* [lab=process] {} {{M_{P}} \bc M_{N} \;| \;P|M_{P} }
\end{mathpar} 

\begin{mathpar}
  \inferrule* [lab=sychronization] {} {M_{N} \bc \Box \;|\; x?M_{F} \;|\; x!M_{C}}
  \and
  \inferrule* [lab=abstraction] {} {{M_{F}} \bc (x)M_{P} }
  \and
  \inferrule* [lab=concretion] {} {{M_{C}} \bc \langle M_{P} \rangle }
  \and \\
  \inferrule* [lab=process] {} {{M_{P}} \bc M_{N} \;| \;P|M_{P} }
\end{mathpar}

\begin{definition}[contextual application] Given a context $M$, and
  process $P$, we define the \emph{contextual application}, $M[P] :=
  M\{P/\Box\}$. That is, the contextual application of M to P is the
  substitution of $P$ for $\Box$ in $M$.
\end{definition}

$\meaningof{-} : L \to \mathcal{P}(\pi)$

\begin{mathpar}
  \inferrule* [lab=collection] {} {\meaningof{true} = \pi, \and \meaningof{~E} = \pi \setminus \meaningof{E}, \and \meaningof{E_{1} \& E_{2}} = \meaningof{E_{1}} \cap \meaningof{E_{2}}}
\end{mathpar}

\begin{mathpar}
  \inferrule* [lab=structure] {} {\meaningof{0} = \{ P \in \pi | P \equiv 0 \}, \and \\ \meaningof{E_1 | E_2} = \{ P \in \pi | P \equiv P_{1} | P_{2}, P_{1} \in \meaningof{E_{1}}, P_{2} \in \meaningof{E_2}\} }
\end{mathpar}

\begin{mathpar}
 \inferrule* [lab=behavior] {} {\meaningof{\langle a?b \rangle E} = \{ P \in \pi | P \equiv Q | u?(y)P', \\ \and \\\\ \and \\ \;\;\; u \in \meaningof{a}, \forall z.P'\{z/y\} \in \meaningof{E\{z/b\}}\}, \and \\ \meaningof{a!E} = \{ P \in \pi | P \equiv Q | x!\langle P' \rangle, x \in \meaningof{a} P' \in \meaningof{E}\} }
\end{mathpar}

\begin{mathpar}
 \inferrule* [lab=nominal] {} {\meaningof{\quotep{E}} = \{ \quotep{P} \in \quotep{\pi} | P \in \meaningof{E} \}, \and \meaningof{\quotep{P}} = \{ \quotep{Q} \in \quotep{\pi} | P \equiv Q \} \and \\ \meaningof{@\quotep{E}} = \{ P \in \pi | P \equiv @x, x \in \meaningof{E} \}}
\end{mathpar}

\begin{eqnarray*}
  \\
  \meaningof{-} : TS \to ST
\end{eqnarray*}

\begin{eqnarray*}
  \\
  L : TS \to ST
\end{eqnarray*}

\begin{eqnarray*}
  \\
  P \models E \iff P \in \meaningof{E}
\end{eqnarray*}

\begin{eqnarray*}
  P \approx_{L} Q \iff \forall E \in L. P \models E \iff Q \models E
\end{eqnarray*}

\begin{eqnarray*}
  P \approx_{K} Q
\end{eqnarray*}

\begin{eqnarray*}
  P \approx Q
\end{eqnarray*}

$\approx_{K} = \approx = \approx_{L}$

\subsubsection{Contextual duality}

Note that contexts extend the quotation operation to a family of
operations from processes to names. Given a context, $M$, we can
define a \emph{nominal context}, $\quotep{M}$ by $\quotep{M}[P] :=
\quotep{M[P]}$. To foreshadow what is to come we observe that these
operations enjoy a duality with processes very much like the duality
between vectors and maps from vectors to scalars.

Further, because the calculus is essentially higher-order, we have a
correspondence between contexts and processes. More specifically,
given a name $x$ and a context $M$ we can construct $M^{*}_{x}$ such
that 

\begin{mathpar}
  M^{*}_{x} | \lift{x}{P} \red M[P]
\end{mathpar}

namely,

\begin{mathpar}
  M^{*}_{x} := x?(u).M[\dropn{u}]
\end{mathpar}

The dependence of $M^{*}_{x}$ on a name makes it an abstraction, 

\begin{mathpar}
  M^{*} := (x)x?(u).M[\dropn{u}]
\end{mathpar}

\subsection{Additional notation}

It will sometimes be convenient to denote the process a name
quotes. We already have the notation $x = \quotep{P}$, but it will be
convenient to introduce an alternate notation, $\procn{x}$, when we
want to emphasize the connection to the use of the name. Note that, by
virtue of name equivalence, $\quotep{\procn{x}} \nameeq x$; so, the
notation is consistent with previous definitions.

Further, because names have structure it is possible to effect
substitutions on the basis of that structure. This means we need to
upgrade our notation for substitutions, which we accomplish by
adapting comprehension notation. Thus,

\begin{mathpar}
  P\{ y / x : x \in S \}
\end{mathpar}

is interpreted to mean the process derived from P by replacing (in a
capture-avoiding manner) each occurrence of $x$ in $S$ by $y$. For example,

\begin{mathpar}
  P\{ \quotep{\procn{x}|\procn{x}} / x : x \in \freenames{P} \}
\end{mathpar}

will replace each (occurrence) of a free name $x$ in $P$ by
$\quotep{\procn{x}|\procn{x}}$.

Also, we will avail ourselves of the notation $x^{L}$ and $x^{R}$ to
denote injections of a name into disjoint copies of the name
space. There are numerous ways to accomplish this. One example can be
found in \cite{MeredithR05}. This notation overloads to vectors of
names: $\vec{x}^{\pi} := (x_{i}^{\pi} \; : \; 0 \leq i < |\vec{x}| )$ where $\pi \in \{L,R\}$.

We also use $P^{\Box} := P|\Box$.

In \cite{MeredithR05} an interpretation of the new operator is
given. It turns out that there are several possible interpretations
all enjoying the requisite algebraic properties of the operator (see
\cite{milner91polyadicpi}). We will therefore make liberal use of
$(\nu\; \vec{x})P$.

% subsection the_syntax_and_semantics_of_the_notation_system (end)   

\input{qm2pi.qmops} 

\input{qm2pi.sterngerlach} 

\input{qm2pi.metric} 

% section concurrent_process_calculi (end)

%\input{qm2pi.proofsketch}

% section proof sketch (end)

%\input{qm2pi.slviaknots} 

% section spatial logic via knots (end)

\input{qm2pi.conclusion}

% section conclusion (end)

%\input{qm2pi.dtcodes} 

% section wiring algorithm (end)

\input{qm2pi.ack} 

% section acknowledgments (end)

\newpage


\bibliographystyle{plain}   
\bibliography{../../biblios/main.bib}

\input{qm2pi.rhodetails}

\end{document}

 

%\documentclass[12pt]{llncs}
%\documentclass{jktr}

\usepackage[pdftex]{hyperref}                   
\usepackage {listings}
\usepackage {mathpartir}
\usepackage{bcprules}
%\usepackage{listings}
                       
\usepackage{graphicx} 
%\usepackage[margins=2.5cm,nohead,nofoot]{geometry}
%\usepackage{geometry}
\usepackage{amsfonts}
\usepackage{amstext}
\usepackage{latexsym}
\usepackage{amssymb}
\usepackage{color}


%\include{myPreamble}
\include{qm2pi.local} 

%\ifpdf
%\usepackage[pdftex]{graphicx}
%\else
%\usepackage{graphicx}
%\fi

 % \ifpdf
%  \usepackage{pdfsync}
%  \if


%\title{Brief Article}
%\author{David F. Snyder}
%\author{L.G. Meredith}

%\address{Dept. of Math., Texas State University--San Marcos, San Marcos, TX 78666}
       
\pagestyle{empty}


\begin{document}

\lstset{language=[Objective]Caml,frame=shadowbox}

\input{qm2pi.front}

% section front matter (end)

\input{qm2pi.intro} 
 
% section introduction (end)

% \input{qm2pi.knotations} 

% section notation (end)

\input{qm2pi.process.calculi} 

% section concurrent_process_calculi_and_spatial_logics_ (end)
    
%\input{qm2pi.knots2pi} 

%\input{qm2pi.trefoil} 

%\input{qm2pi.mainthm} 

% subsection basic_interpretation (end)

%\input{qm2pi.rho.presentation} 
\subsection{The syntax and semantics of the notation system}\label{sub:the_syntax_and_semantics_of_the_notation_system} % (fold)

We now summarize a technical presentation of the calculus that
embodies our theory of dynamics. The typical presentation of such a
calculus follows the style of giving generators and relations on
them. The grammar, below, describing term constructors, freely
generates the set of processes, $\Proc$. This set is then quotiented
by a relation known as structural congruence and it is over this set
that the notion of dynamics is expressed. This presentation is
essentially that of \cite{MeredithR05} with the addition of
polyadicity and summation. For readability we have relegated some of
the technical subtleties to an appendix.

\subsubsection{Process grammar}\label{subsub:process_grammar}

\begin{mathpar}
  \inferrule* [lab=synchronization] {} {{M} \bc \pzero \;|\; x?F \;|\; x!C }
  \and
  \inferrule* [lab=abstraction] {} {{F} \bc (x)P}
  \and
  \inferrule* [lab=concretion] {} {{C} \bc \langle Q \rangle}
  \and
  \inferrule* [lab=process] {} {{P,Q} \bc M \;| \;P|Q \;|\; @{x}}
  \and
  \inferrule* [lab=name] {} {{x} \bc \quotep{P}}
\end{mathpar} 

Note that $\vec{x}$ (resp. $\vec{P}$) denotes a vector of names
(resp. processes) of length $|\vec{x}|$ (resp. $|\vec{P}|$). We adopt
the following useful abbreviations.

\begin{mathpar}
   x?(\vec{y}).P := x.(\vec{y})P \and  x\clift{\vec{P}} := x.\clift{\vec{P}}
   \and x!(y) := \lift{x}{\dropn{y}}
   \and \Pi_{i=0}^{n-1}P_i := P_0 | \ldots | P_{n-1}
\end{mathpar}

\subsubsection{Structural congruence}

\paragraph{Free and bound names and alpha-equivalence.} At the
core of structural equivalence is alpha-equivalence which identifies
process that are the same up to a change of variable. Formally, we
recognize the distinction between free and bound names. The free names
of a process, $\freenames{P}$, may be calculated recursively as
follows:

\begin{mathpar}
\freenames{\pzero} := \emptyset
  \and \\
  \freenames{x?(y).P} := \{ x \} \cup (\freenames{P} \setminus \{ y \})
  \and 
  \freenames{x!\langle P \rangle} := \{ x \} \cup \{ P \} 
  \and \\
  \freenames{P|Q} := \freenames{P} \cup \freenames{Q}
  \and \\
  \freenames{@{x}} := \{ x \}
\end{mathpar}

$\pi$
$\quotep{\pi}$

$\freenames{-} : \pi \to \mathcal{P}(\quotep{\pi})$

\begin{eqnarray*}
  \freenames{\pzero} & := & \emptyset \\
  \freenames{x?(y).P} & := & \{ x \} \cup (\freenames{P} \setminus \{ y \}) \\
  \freenames{x!\langle P \rangle} & := & \{ x \} \cup \{ P \} \\
  \freenames{P|Q} & := & \freenames{P} \cup \freenames{Q} \\
  \freenames{\dropn{x}} & := & \{ x \}
\end{eqnarray*}

The bound names of a process, $\boundnames{P}$, are those names occurring in $P$
that are not free. For example, in $x?(y).0$, the name $x$ is free, while $y$ is bound.

\begin{mathpar}
  \inferrule* [lab=monoidal-laws] {} { P|Q \equiv Q|P \and P|0 \equiv P \and P|(Q|R) \equiv (P|Q)|R }
\end{mathpar}

\begin{mathpar}
  \inferrule* [lab=alpha-equivalence] {} { (x)P \equiv (y)P\{y/x\} \and y \not\in \freenames{P} }
\end{mathpar}

\begin{definition}
Then two processes, $P,Q$, are alpha-equivalent if $P = Q\{\vec{y}/\vec{x}\}$ for
some $\vec{x} \in \boundnames{Q},\vec{y} \in \boundnames{P}$, where $Q\{\vec{y}/\vec{x}\}$
denotes the capture-avoiding substitution of $\vec{y}$ for $\vec{x}$ in $Q$.
\end{definition}

\begin{definition}
  The {\em structural congruence} \cite{SangiorgiWalker} , $\equiv$,
  between processes is the least congruence containing
  alpha-equivalence, satisfying the abelian monoid laws
  (associativity, commutativity and $\pzero$ as identity) for parallel
  composition $|$ and for summation $+$.
\end{definition}

\subsection{Name equivalence}

We take name equivalence, written $\nameeq$, to be the smallest
equivalence relation generated by the following rules.

\begin{mathpar}
\inferrule*[lab=Quote-drop]
{ }
{ \quotep{@{x}} \nameeq x }

\inferrule*[lab=Struct-equiv]
{ P \scong Q }
{ \quotep{P} \nameeq \quotep{Q} }
\end{mathpar}

The astute reader will have noticed that the mutual recursion of names
and processes imposes a mutual recursion on alpha-equivalence and
structural equivalence via name-equivalence. Fortunately, all of this
works out pleasantly and we may calculate in the natural way, free of
concern. The reader interested in the details is referred to the
appendix \ref{appendix:rho_details}.

\subsection{Substitution}

We use $\Proc$ for the set of processes, $\QProc$ for the set of
names, and $\id{\{}\vec{y} / \vec{x} \id{\}}$ to denote partial maps,
$s : \QProc \rightarrow \QProc$. A map, $s$ lifts, uniquely, to a map
on process terms, $\widehat{s} : \Proc \rightarrow \Proc$ by the
following equations.

\begin{mathpar}
  (0) \psubstp{Q}{P} := 0 \\
  (R \juxtap S) \psubstp{Q}{P}
  :=    
  (R)\psubstp{Q}{P} \juxtap (S) \psubstp{Q}{P} \\
  (x?(y).R) \psubstp{Q}{P}    
  :=    
  (x)\substp{Q}{P} (z)\concat( (R \psubstn{z}{y}) \psubstp{Q}{P} ) \\
  (\lift{x}{R}) \psubstp{Q}{P}  
  :=
  \lift{(x)\substp{Q}{P}}{ R \psubstp{Q}{P} } \\
%   (\dropn{x})  \psubstp{Q}{P}       
%   := 
%   \left\{ 
%     \begin{array}{ccc} 
%       \dropn{\quotep{Q}} & & x \nameeq \quotep{P} \\
%       \dropn{x} & & otherwise \\
%     \end{array}
%   \right. 
  (\dropn{x})  \psubstp{Q}{P}       
  := 
  \left\{ 
    \begin{array}{ccc} 
      Q & & x \nameeq \quotep{P} \\
      \dropn{x} & & otherwise \\
    \end{array}
  \right.
\end{mathpar}
 

where

\begin{eqnarray}
  (x)\id{\{} \lpquote Q \rpquote / \lpquote P \rpquote \id{\}}            = 
  \left\{ 
    \begin{array}{ccc}
      \lpquote Q \rpquote & & x \nameeq \lpquote P \rpquote \\
      x & & otherwise \\
    \end{array}
  \right. \nonumber
\end{eqnarray}

and $z$ is chosen distinct from $\quotep{P}$, $\quotep{Q}$, the free
names in $Q$, and all the names in $R$. Our $\alpha$-equivalence will
be built in the standard way from this substitution.

\begin{remark}\label{rem:no_self_referential_names}
  One consequence of these definitions is that $\forall P. \quotep{P}
  \not\in \freenames{P}$.
\end{remark}

\subsection{ Dynamic quote: an example }

Anticipating something of what's to come, consider applying the
substitution, $\widehat{\id{\{}u / z \id{\}}}$, to the following pair
of processes, $\lift{w}{y!(z)}$ and $w[ \lpquote y!(z) \rpquote ]$.

\begin{eqnarray}
	\lift{w}{y!(z)}\widehat{\id{\{}u / z \id{\}}}
		& = &
		\lift{w}{y!(u)} \nonumber\\
	w[ \lpquote y!(z) \rpquote ] \widehat{ \id{\{}u / z \id{\}} }
		& = &
		w[ \lpquote y!(z) \rpquote ] \nonumber
\end{eqnarray}

Because the body of the process between quotes is impervious to
substitution, we get radically different answers. In fact, by
examining the first process in an input context,
e.g. $x?(z).\lift{w}{y!(z)}$, we see that the process under the lift
operator may be shaped by prefixed inputs binding a name inside it. In
this sense, the lift operator will be seen as a way to dynamically
construct processes before reifying them as names.

Finally equipped with these standard features we can present the
dynamics of the calculus.

\subsubsection{Operational semantics} 

Finally, we introduce the computational dynamics. What marks these
algebras as distinct from other more traditionally studied algebraic
structures, e.g. vector spaces or polynomial rings, is the manner in
which dynamics is captured. In traditional structures, dynamics is typically
expressed through morphisms between such structures, as in linear maps
between vector spaces or morphisms between rings. In algebras
associated with the semantics of computation, the dynamics is
expressed as part of the algebraic structure itself, through a
reduction reduction relation typically denoted by $\red$. Below, we
give a recursive presentation of this relation for the calculus used
in the encoding.

$\red \subseteq \pi \times \pi$
$\red : \pi \to \mathcal{P}(\pi)$

\begin{mathpar}
  \inferrule* [lab=Comm] { \textsf{match}( x_{src}, x_{trgt} ) } { x_{trgt}?(y)P \; | \; x_{src}!\langle {Q} \rangle \red P\{\quotep{Q}/y}\} }
  \and \\
  \inferrule* [lab=Par] {{P} \red {P}'} {{{P} | {Q}} \red {{P}' | {Q}}}
  \and
  \inferrule* [lab=Equiv]{{{P} \scong {P}'} \andalso {{P}' \red {Q}'} \andalso {{Q}' \scong {Q}}}{{P} \red {Q}}
\end{mathpar}

\begin{eqnarray*}
  match_{\equiv} (\quotep{P},\quotep{Q}) & := & P \equiv Q \\
  match_{\dagger}(\quotep{P},\quotep{Q}) & := & \forall R. P|Q \red^{*} R => R \red^{*} 0 \\
  match_{K}(\quotep{P},\quotep{Q}) & := & K \mbox{ for some context } K
\end{eqnarray*}

$u?(x)P | u!\langle Q \rangle \red P\{\quotep{Q}/x\}$

%We write $\wred$ for $\red^*$, and $P\red$ if $\exists Q $ such that $ P \red Q$.
We write $P\red$ if $\exists Q $ such that $ P \red Q$ and $P\not\red$, otherwise.

\section{Replication}

As mentioned before, it is known that replication (and hence
recursion) can be implemented in a higher-order process algebra
\cite{SangiorgiWalker}. As our first example of calculation with the
machinery thus far presented we give the construction explicitly in
the {\rhoc}.

\begin{eqnarray}
	D_{x} & := & \prefix{x}{y}{(\binpar{\outputp{x}{y}}{@{y}})} \nonumber\\
	\bangp_{x}{P} & := & \binpar{{x}!\langle{\binpar{D_{x}}{P}}\rangle}{D_{x}} \nonumber
\end{eqnarray}

\begin{eqnarray}
	\bangp_{x}{P} & & \nonumber\\
	=
	& {x}!\langle{(\prefix{x}{y}{(\outputp{x}{y} | @{y})) | P}}\rangle 
	      | \prefix{x}{y}{(\outputp{x}{y} | @{y})} & \nonumber\\
	\red
	& (\outputp{x}{y} | @{y})\substn{\quotep{(\prefix{x}{y}{(@{y} | \outputp{x}{y})) | P}}}{y} & \nonumber\\
	=
	& \outputp{x}{\quotep{(\prefix{x}{y}{(\outputp{x}{y} | @{y})) | P}}}
	  | {(\prefix{x}{y}{(\outputp{x}{y} | @{y})) | P}} & \nonumber\\
	\red
	& \ldots & \nonumber\\
	\red^*
	& P | P | \ldots & \nonumber
\end{eqnarray}

Of course, this encoding, as an implementation, runs away, unfolding
$\bangp{P}$ eagerly. A lazier and more implementable replication
operator, restricted to input-guarded processes, may be obtained as follows.

\begin{eqnarray}
\bangp{\prefix{u}{v}{P}} 
	:= 
	\binpar{\lift{x}{\prefix{u}{v}{(\binpar{D(x)}{P})}}}{D(x)} \nonumber
\end{eqnarray}

\begin{remark}
  Note that the lazier definition still does not deal with summation
  or mixed summation (i.e. sums over input and output). The reader is
  invited to construct definitions of replication that deal with these
  features. 

  Further, the definitions are parameterized in a name, $x$. Can you,
  gentle reader, make a definition that eliminates this parameter and
  guarantees no accidental interaction between the replication
  machinery and the process being replicated -- i.e. no accidental
  sharing of names used by the process to get its work done and the
  name(s) used by the replication to effect copying. This latter
  revision of the definition of replication is crucial to obtaining
  the expected identity $!!P \sim !P$.
\end{remark}

\begin{remark}\label{rem:paradoxical_combinator}
  The reader familiar with the lambda calculus will have noticed the
  similarity between $D$ and the paradoxical combinator.

  [Ed. note: the existence of this seems to suggest we have to be more
  restrictive on the set of processes and names we admit if we are to
  support no-cloning.]
\end{remark}

\subsubsection{Bisimulation}

The computational dynamics gives rise to another kind of equivalence,
the equivalence of computational behavior. As previously mentioned
this is typically captured \emph{via} some form of bisimulation.

% The notion we use in this paper is weak barbed bisimulation
% \cite{milner91polyadicpi}.

The notion we use in this paper is derived from weak barbed
bisimulation \cite{milner91polyadicpi}. 

\begin{definition}
An \emph{observation relation}, $\downarrow_{\mathcal N}$, over a set
of names, $\mathcal N$, is the smallest relation satisfying the rules
below.

\infrule[Out-barb]{y \in {\mathcal N}, \; x \nameeq y}
		  {\outputp{x}{v} \downarrow_{\mathcal N} x}
\infrule[Par-barb]{\mbox{$P\downarrow_{\mathcal N} x$ or $Q\downarrow_{\mathcal N} x$}}
		  {\binpar{P}{Q} \downarrow_{\mathcal N} x}

We write $P \Downarrow_{\mathcal N} x$ if there is $Q$ such that 
$P \wred Q$ and $Q \downarrow_{\mathcal N} x$.
\end{definition}

\begin{definition}
%\label{def.bbisim}
An  ${\mathcal N}$-\emph{barbed bisimulation} over a set of names, ${\mathcal N}$, is a symmetric binary relation 
${\mathcal S}_{\mathcal N}$ between agents such that $P\rel{S}_{\mathcal N}Q$ implies:
\begin{enumerate}
\item If $P \red P'$ then $Q \wred Q'$ and $P'\rel{S}_{\mathcal N} Q'$.
\item If $P\downarrow_{\mathcal N} x$, then $Q\Downarrow_{\mathcal N} x$.
\end{enumerate}
$P$ is ${\mathcal N}$-barbed bisimilar to $Q$, written
$P \wbbisim_{\mathcal N} Q$, if $P \rel{S}_{\mathcal N} Q$ for some ${\mathcal N}$-barbed bisimulation ${\mathcal S}_{\mathcal N}$.
\end{definition}

$\mathcal{R} \subseteq \pi \times \pi$

$P \mathcal{R} Q => \forall P'. P \red P' \Rightarrow \exists Q'. Q \red Q', P' \mathcal{R} Q'$

$P \vdash x \Rightarrow Q \vdash x$

\begin{mathpar}
  \inferrule*[lab=Out-barb]{x \nameeq y}{{y}!\langle{Q}\rangle \vdash x}
  \and
  \inferrule*[lab=Par-barb]{\mbox{$P\vdash x$ or $Q\vdash x$}}{\binpar{P}{Q} \vdash x}
\end{mathpar}

\subsubsection{Contexts}

One of the principle advantages of computational calculi like the
$\pi$-calculus is a well-defined notion of context,
contextual-equivalence and a correlation between
contextual-equivalence and notions of bisimulation. The notion of
context allows the decomposition of a process into (sub-)process and
its syntactic environment, its context. Thus, a context may be
thought of as a process with a ``hole'' (written $\Box$) in it. The
application of a context $M$ to a process $P$, written $M[P]$, is
tantamount to filling the hole in $M$ with $P$. In this paper we do
not need the full weight of this theory, but do make use of the notion
of context in the proof the main theorem. 

\begin{mathpar}
  \inferrule* [lab=summation] {} {{M_{M},M_{N}} \bc \Box \;|\; x.M_{A} \;|\; M_{M}+M_{N}}
  \and
  \inferrule* [lab=agent] {} {{M_{A}} \bc (\vec{x})M_{P} \;| \; \clift{P_0,\ldots,M_{P},\ldots,P_N}}
  \and \\
  \inferrule* [lab=process] {} {{M_{P}} \bc M_{N} \;| \;P|M_{P} }
\end{mathpar} 

\begin{mathpar}
  \inferrule* [lab=sychronization] {} {M_{N} \bc \Box \;|\; x?M_{F} \;|\; x!M_{C}}
  \and
  \inferrule* [lab=abstraction] {} {{M_{F}} \bc (x)M_{P} }
  \and
  \inferrule* [lab=concretion] {} {{M_{C}} \bc \langle M_{P} \rangle }
  \and \\
  \inferrule* [lab=process] {} {{M_{P}} \bc M_{N} \;| \;P|M_{P} }
\end{mathpar}

\begin{definition}[contextual application] Given a context $M$, and
  process $P$, we define the \emph{contextual application}, $M[P] :=
  M\{P/\Box\}$. That is, the contextual application of M to P is the
  substitution of $P$ for $\Box$ in $M$.
\end{definition}

$\meaningof{-} : L \to \mathcal{P}(\pi)$

\begin{mathpar}
  \inferrule* [lab=collection] {} {\meaningof{true} = \pi, \and \meaningof{~E} = \pi \setminus \meaningof{E}, \and \meaningof{E_{1} \& E_{2}} = \meaningof{E_{1}} \cap \meaningof{E_{2}}}
\end{mathpar}

\begin{mathpar}
  \inferrule* [lab=structure] {} {\meaningof{0} = \{ P \in \pi | P \equiv 0 \}, \and \\ \meaningof{E_1 | E_2} = \{ P \in \pi | P \equiv P_{1} | P_{2}, P_{1} \in \meaningof{E_{1}}, P_{2} \in \meaningof{E_2}\} }
\end{mathpar}

\begin{mathpar}
 \inferrule* [lab=behavior] {} {\meaningof{\langle a?b \rangle E} = \{ P \in \pi | P \equiv Q | u?(y)P', \\ \and \\\\ \and \\ \;\;\; u \in \meaningof{a}, \forall z.P'\{z/y\} \in \meaningof{E\{z/b\}}\}, \and \\ \meaningof{a!E} = \{ P \in \pi | P \equiv Q | x!\langle P' \rangle, x \in \meaningof{a} P' \in \meaningof{E}\} }
\end{mathpar}

\begin{mathpar}
 \inferrule* [lab=nominal] {} {\meaningof{\quotep{E}} = \{ \quotep{P} \in \quotep{\pi} | P \in \meaningof{E} \}, \and \meaningof{\quotep{P}} = \{ \quotep{Q} \in \quotep{\pi} | P \equiv Q \} \and \\ \meaningof{@\quotep{E}} = \{ P \in \pi | P \equiv @x, x \in \meaningof{E} \}}
\end{mathpar}

\begin{eqnarray*}
  \\
  \meaningof{-} : TS \to ST
\end{eqnarray*}

\begin{eqnarray*}
  \\
  L : TS \to ST
\end{eqnarray*}

\begin{eqnarray*}
  \\
  P \models E \iff P \in \meaningof{E}
\end{eqnarray*}

\begin{eqnarray*}
  P \approx_{L} Q \iff \forall E \in L. P \models E \iff Q \models E
\end{eqnarray*}

\begin{eqnarray*}
  P \approx_{K} Q
\end{eqnarray*}

\begin{eqnarray*}
  P \approx Q
\end{eqnarray*}

$\approx_{K} = \approx = \approx_{L}$

\subsubsection{Contextual duality}

Note that contexts extend the quotation operation to a family of
operations from processes to names. Given a context, $M$, we can
define a \emph{nominal context}, $\quotep{M}$ by $\quotep{M}[P] :=
\quotep{M[P]}$. To foreshadow what is to come we observe that these
operations enjoy a duality with processes very much like the duality
between vectors and maps from vectors to scalars.

Further, because the calculus is essentially higher-order, we have a
correspondence between contexts and processes. More specifically,
given a name $x$ and a context $M$ we can construct $M^{*}_{x}$ such
that 

\begin{mathpar}
  M^{*}_{x} | \lift{x}{P} \red M[P]
\end{mathpar}

namely,

\begin{mathpar}
  M^{*}_{x} := x?(u).M[\dropn{u}]
\end{mathpar}

The dependence of $M^{*}_{x}$ on a name makes it an abstraction, 

\begin{mathpar}
  M^{*} := (x)x?(u).M[\dropn{u}]
\end{mathpar}

\subsection{Additional notation}

It will sometimes be convenient to denote the process a name
quotes. We already have the notation $x = \quotep{P}$, but it will be
convenient to introduce an alternate notation, $\procn{x}$, when we
want to emphasize the connection to the use of the name. Note that, by
virtue of name equivalence, $\quotep{\procn{x}} \nameeq x$; so, the
notation is consistent with previous definitions.

Further, because names have structure it is possible to effect
substitutions on the basis of that structure. This means we need to
upgrade our notation for substitutions, which we accomplish by
adapting comprehension notation. Thus,

\begin{mathpar}
  P\{ y / x : x \in S \}
\end{mathpar}

is interpreted to mean the process derived from P by replacing (in a
capture-avoiding manner) each occurrence of $x$ in $S$ by $y$. For example,

\begin{mathpar}
  P\{ \quotep{\procn{x}|\procn{x}} / x : x \in \freenames{P} \}
\end{mathpar}

will replace each (occurrence) of a free name $x$ in $P$ by
$\quotep{\procn{x}|\procn{x}}$.

Also, we will avail ourselves of the notation $x^{L}$ and $x^{R}$ to
denote injections of a name into disjoint copies of the name
space. There are numerous ways to accomplish this. One example can be
found in \cite{MeredithR05}. This notation overloads to vectors of
names: $\vec{x}^{\pi} := (x_{i}^{\pi} \; : \; 0 \leq i < |\vec{x}| )$ where $\pi \in \{L,R\}$.

We also use $P^{\Box} := P|\Box$.

In \cite{MeredithR05} an interpretation of the new operator is
given. It turns out that there are several possible interpretations
all enjoying the requisite algebraic properties of the operator (see
\cite{milner91polyadicpi}). We will therefore make liberal use of
$(\nu\; \vec{x})P$.

% subsection the_syntax_and_semantics_of_the_notation_system (end)   

\input{qm2pi.qmops} 

\input{qm2pi.sterngerlach} 

\input{qm2pi.metric} 

% section concurrent_process_calculi (end)

%\input{qm2pi.proofsketch}

% section proof sketch (end)

%\input{qm2pi.slviaknots} 

% section spatial logic via knots (end)

\input{qm2pi.conclusion}

% section conclusion (end)

%\input{qm2pi.dtcodes} 

% section wiring algorithm (end)

\input{qm2pi.ack} 

% section acknowledgments (end)

\newpage


\bibliographystyle{plain}   
\bibliography{../../biblios/main.bib}

\input{qm2pi.rhodetails}

\end{document}

 

% subsection basic_interpretation (end)

%\input{qm2pi.rho.presentation} 
\subsection{The syntax and semantics of the notation system}\label{sub:the_syntax_and_semantics_of_the_notation_system} % (fold)

We now summarize a technical presentation of the calculus that
embodies our theory of dynamics. The typical presentation of such a
calculus follows the style of giving generators and relations on
them. The grammar, below, describing term constructors, freely
generates the set of processes, $\Proc$. This set is then quotiented
by a relation known as structural congruence and it is over this set
that the notion of dynamics is expressed. This presentation is
essentially that of \cite{MeredithR05} with the addition of
polyadicity and summation. For readability we have relegated some of
the technical subtleties to an appendix.

\subsubsection{Process grammar}\label{subsub:process_grammar}

\begin{mathpar}
  \inferrule* [lab=synchronization] {} {{M} \bc \pzero \;|\; x?F \;|\; x!C }
  \and
  \inferrule* [lab=abstraction] {} {{F} \bc (x)P}
  \and
  \inferrule* [lab=concretion] {} {{C} \bc \langle Q \rangle}
  \and
  \inferrule* [lab=process] {} {{P,Q} \bc M \;| \;P|Q \;|\; @{x}}
  \and
  \inferrule* [lab=name] {} {{x} \bc \quotep{P}}
\end{mathpar} 

Note that $\vec{x}$ (resp. $\vec{P}$) denotes a vector of names
(resp. processes) of length $|\vec{x}|$ (resp. $|\vec{P}|$). We adopt
the following useful abbreviations.

\begin{mathpar}
   x?(\vec{y}).P := x.(\vec{y})P \and  x\clift{\vec{P}} := x.\clift{\vec{P}}
   \and x!(y) := \lift{x}{\dropn{y}}
   \and \Pi_{i=0}^{n-1}P_i := P_0 | \ldots | P_{n-1}
\end{mathpar}

\subsubsection{Structural congruence}

\paragraph{Free and bound names and alpha-equivalence.} At the
core of structural equivalence is alpha-equivalence which identifies
process that are the same up to a change of variable. Formally, we
recognize the distinction between free and bound names. The free names
of a process, $\freenames{P}$, may be calculated recursively as
follows:

\begin{mathpar}
\freenames{\pzero} := \emptyset
  \and \\
  \freenames{x?(y).P} := \{ x \} \cup (\freenames{P} \setminus \{ y \})
  \and 
  \freenames{x!\langle P \rangle} := \{ x \} \cup \{ P \} 
  \and \\
  \freenames{P|Q} := \freenames{P} \cup \freenames{Q}
  \and \\
  \freenames{@{x}} := \{ x \}
\end{mathpar}

$\pi$
$\quotep{\pi}$

$\freenames{-} : \pi \to \mathcal{P}(\quotep{\pi})$

\begin{eqnarray*}
  \freenames{\pzero} & := & \emptyset \\
  \freenames{x?(y).P} & := & \{ x \} \cup (\freenames{P} \setminus \{ y \}) \\
  \freenames{x!\langle P \rangle} & := & \{ x \} \cup \{ P \} \\
  \freenames{P|Q} & := & \freenames{P} \cup \freenames{Q} \\
  \freenames{\dropn{x}} & := & \{ x \}
\end{eqnarray*}

The bound names of a process, $\boundnames{P}$, are those names occurring in $P$
that are not free. For example, in $x?(y).0$, the name $x$ is free, while $y$ is bound.

\begin{mathpar}
  \inferrule* [lab=monoidal-laws] {} { P|Q \equiv Q|P \and P|0 \equiv P \and P|(Q|R) \equiv (P|Q)|R }
\end{mathpar}

\begin{mathpar}
  \inferrule* [lab=alpha-equivalence] {} { (x)P \equiv (y)P\{y/x\} \and y \not\in \freenames{P} }
\end{mathpar}

\begin{definition}
Then two processes, $P,Q$, are alpha-equivalent if $P = Q\{\vec{y}/\vec{x}\}$ for
some $\vec{x} \in \boundnames{Q},\vec{y} \in \boundnames{P}$, where $Q\{\vec{y}/\vec{x}\}$
denotes the capture-avoiding substitution of $\vec{y}$ for $\vec{x}$ in $Q$.
\end{definition}

\begin{definition}
  The {\em structural congruence} \cite{SangiorgiWalker} , $\equiv$,
  between processes is the least congruence containing
  alpha-equivalence, satisfying the abelian monoid laws
  (associativity, commutativity and $\pzero$ as identity) for parallel
  composition $|$ and for summation $+$.
\end{definition}

\subsection{Name equivalence}

We take name equivalence, written $\nameeq$, to be the smallest
equivalence relation generated by the following rules.

\begin{mathpar}
\inferrule*[lab=Quote-drop]
{ }
{ \quotep{@{x}} \nameeq x }

\inferrule*[lab=Struct-equiv]
{ P \scong Q }
{ \quotep{P} \nameeq \quotep{Q} }
\end{mathpar}

The astute reader will have noticed that the mutual recursion of names
and processes imposes a mutual recursion on alpha-equivalence and
structural equivalence via name-equivalence. Fortunately, all of this
works out pleasantly and we may calculate in the natural way, free of
concern. The reader interested in the details is referred to the
appendix \ref{appendix:rho_details}.

\subsection{Substitution}

We use $\Proc$ for the set of processes, $\QProc$ for the set of
names, and $\id{\{}\vec{y} / \vec{x} \id{\}}$ to denote partial maps,
$s : \QProc \rightarrow \QProc$. A map, $s$ lifts, uniquely, to a map
on process terms, $\widehat{s} : \Proc \rightarrow \Proc$ by the
following equations.

\begin{mathpar}
  (0) \psubstp{Q}{P} := 0 \\
  (R \juxtap S) \psubstp{Q}{P}
  :=    
  (R)\psubstp{Q}{P} \juxtap (S) \psubstp{Q}{P} \\
  (x?(y).R) \psubstp{Q}{P}    
  :=    
  (x)\substp{Q}{P} (z)\concat( (R \psubstn{z}{y}) \psubstp{Q}{P} ) \\
  (\lift{x}{R}) \psubstp{Q}{P}  
  :=
  \lift{(x)\substp{Q}{P}}{ R \psubstp{Q}{P} } \\
%   (\dropn{x})  \psubstp{Q}{P}       
%   := 
%   \left\{ 
%     \begin{array}{ccc} 
%       \dropn{\quotep{Q}} & & x \nameeq \quotep{P} \\
%       \dropn{x} & & otherwise \\
%     \end{array}
%   \right. 
  (\dropn{x})  \psubstp{Q}{P}       
  := 
  \left\{ 
    \begin{array}{ccc} 
      Q & & x \nameeq \quotep{P} \\
      \dropn{x} & & otherwise \\
    \end{array}
  \right.
\end{mathpar}
 

where

\begin{eqnarray}
  (x)\id{\{} \lpquote Q \rpquote / \lpquote P \rpquote \id{\}}            = 
  \left\{ 
    \begin{array}{ccc}
      \lpquote Q \rpquote & & x \nameeq \lpquote P \rpquote \\
      x & & otherwise \\
    \end{array}
  \right. \nonumber
\end{eqnarray}

and $z$ is chosen distinct from $\quotep{P}$, $\quotep{Q}$, the free
names in $Q$, and all the names in $R$. Our $\alpha$-equivalence will
be built in the standard way from this substitution.

\begin{remark}\label{rem:no_self_referential_names}
  One consequence of these definitions is that $\forall P. \quotep{P}
  \not\in \freenames{P}$.
\end{remark}

\subsection{ Dynamic quote: an example }

Anticipating something of what's to come, consider applying the
substitution, $\widehat{\id{\{}u / z \id{\}}}$, to the following pair
of processes, $\lift{w}{y!(z)}$ and $w[ \lpquote y!(z) \rpquote ]$.

\begin{eqnarray}
	\lift{w}{y!(z)}\widehat{\id{\{}u / z \id{\}}}
		& = &
		\lift{w}{y!(u)} \nonumber\\
	w[ \lpquote y!(z) \rpquote ] \widehat{ \id{\{}u / z \id{\}} }
		& = &
		w[ \lpquote y!(z) \rpquote ] \nonumber
\end{eqnarray}

Because the body of the process between quotes is impervious to
substitution, we get radically different answers. In fact, by
examining the first process in an input context,
e.g. $x?(z).\lift{w}{y!(z)}$, we see that the process under the lift
operator may be shaped by prefixed inputs binding a name inside it. In
this sense, the lift operator will be seen as a way to dynamically
construct processes before reifying them as names.

Finally equipped with these standard features we can present the
dynamics of the calculus.

\subsubsection{Operational semantics} 

Finally, we introduce the computational dynamics. What marks these
algebras as distinct from other more traditionally studied algebraic
structures, e.g. vector spaces or polynomial rings, is the manner in
which dynamics is captured. In traditional structures, dynamics is typically
expressed through morphisms between such structures, as in linear maps
between vector spaces or morphisms between rings. In algebras
associated with the semantics of computation, the dynamics is
expressed as part of the algebraic structure itself, through a
reduction reduction relation typically denoted by $\red$. Below, we
give a recursive presentation of this relation for the calculus used
in the encoding.

$\red \subseteq \pi \times \pi$
$\red : \pi \to \mathcal{P}(\pi)$

\begin{mathpar}
  \inferrule* [lab=Comm] { \textsf{match}( x_{src}, x_{trgt} ) } { x_{trgt}?(y)P \; | \; x_{src}!\langle {Q} \rangle \red P\{\quotep{Q}/y}\} }
  \and \\
  \inferrule* [lab=Par] {{P} \red {P}'} {{{P} | {Q}} \red {{P}' | {Q}}}
  \and
  \inferrule* [lab=Equiv]{{{P} \scong {P}'} \andalso {{P}' \red {Q}'} \andalso {{Q}' \scong {Q}}}{{P} \red {Q}}
\end{mathpar}

\begin{eqnarray*}
  match_{\equiv} (\quotep{P},\quotep{Q}) & := & P \equiv Q \\
  match_{\dagger}(\quotep{P},\quotep{Q}) & := & \forall R. P|Q \red^{*} R => R \red^{*} 0 \\
  match_{K}(\quotep{P},\quotep{Q}) & := & K \mbox{ for some context } K
\end{eqnarray*}

$u?(x)P | u!\langle Q \rangle \red P\{\quotep{Q}/x\}$

%We write $\wred$ for $\red^*$, and $P\red$ if $\exists Q $ such that $ P \red Q$.
We write $P\red$ if $\exists Q $ such that $ P \red Q$ and $P\not\red$, otherwise.

\section{Replication}

As mentioned before, it is known that replication (and hence
recursion) can be implemented in a higher-order process algebra
\cite{SangiorgiWalker}. As our first example of calculation with the
machinery thus far presented we give the construction explicitly in
the {\rhoc}.

\begin{eqnarray}
	D_{x} & := & \prefix{x}{y}{(\binpar{\outputp{x}{y}}{@{y}})} \nonumber\\
	\bangp_{x}{P} & := & \binpar{{x}!\langle{\binpar{D_{x}}{P}}\rangle}{D_{x}} \nonumber
\end{eqnarray}

\begin{eqnarray}
	\bangp_{x}{P} & & \nonumber\\
	=
	& {x}!\langle{(\prefix{x}{y}{(\outputp{x}{y} | @{y})) | P}}\rangle 
	      | \prefix{x}{y}{(\outputp{x}{y} | @{y})} & \nonumber\\
	\red
	& (\outputp{x}{y} | @{y})\substn{\quotep{(\prefix{x}{y}{(@{y} | \outputp{x}{y})) | P}}}{y} & \nonumber\\
	=
	& \outputp{x}{\quotep{(\prefix{x}{y}{(\outputp{x}{y} | @{y})) | P}}}
	  | {(\prefix{x}{y}{(\outputp{x}{y} | @{y})) | P}} & \nonumber\\
	\red
	& \ldots & \nonumber\\
	\red^*
	& P | P | \ldots & \nonumber
\end{eqnarray}

Of course, this encoding, as an implementation, runs away, unfolding
$\bangp{P}$ eagerly. A lazier and more implementable replication
operator, restricted to input-guarded processes, may be obtained as follows.

\begin{eqnarray}
\bangp{\prefix{u}{v}{P}} 
	:= 
	\binpar{\lift{x}{\prefix{u}{v}{(\binpar{D(x)}{P})}}}{D(x)} \nonumber
\end{eqnarray}

\begin{remark}
  Note that the lazier definition still does not deal with summation
  or mixed summation (i.e. sums over input and output). The reader is
  invited to construct definitions of replication that deal with these
  features. 

  Further, the definitions are parameterized in a name, $x$. Can you,
  gentle reader, make a definition that eliminates this parameter and
  guarantees no accidental interaction between the replication
  machinery and the process being replicated -- i.e. no accidental
  sharing of names used by the process to get its work done and the
  name(s) used by the replication to effect copying. This latter
  revision of the definition of replication is crucial to obtaining
  the expected identity $!!P \sim !P$.
\end{remark}

\begin{remark}\label{rem:paradoxical_combinator}
  The reader familiar with the lambda calculus will have noticed the
  similarity between $D$ and the paradoxical combinator.

  [Ed. note: the existence of this seems to suggest we have to be more
  restrictive on the set of processes and names we admit if we are to
  support no-cloning.]
\end{remark}

\subsubsection{Bisimulation}

The computational dynamics gives rise to another kind of equivalence,
the equivalence of computational behavior. As previously mentioned
this is typically captured \emph{via} some form of bisimulation.

% The notion we use in this paper is weak barbed bisimulation
% \cite{milner91polyadicpi}.

The notion we use in this paper is derived from weak barbed
bisimulation \cite{milner91polyadicpi}. 

\begin{definition}
An \emph{observation relation}, $\downarrow_{\mathcal N}$, over a set
of names, $\mathcal N$, is the smallest relation satisfying the rules
below.

\infrule[Out-barb]{y \in {\mathcal N}, \; x \nameeq y}
		  {\outputp{x}{v} \downarrow_{\mathcal N} x}
\infrule[Par-barb]{\mbox{$P\downarrow_{\mathcal N} x$ or $Q\downarrow_{\mathcal N} x$}}
		  {\binpar{P}{Q} \downarrow_{\mathcal N} x}

We write $P \Downarrow_{\mathcal N} x$ if there is $Q$ such that 
$P \wred Q$ and $Q \downarrow_{\mathcal N} x$.
\end{definition}

\begin{definition}
%\label{def.bbisim}
An  ${\mathcal N}$-\emph{barbed bisimulation} over a set of names, ${\mathcal N}$, is a symmetric binary relation 
${\mathcal S}_{\mathcal N}$ between agents such that $P\rel{S}_{\mathcal N}Q$ implies:
\begin{enumerate}
\item If $P \red P'$ then $Q \wred Q'$ and $P'\rel{S}_{\mathcal N} Q'$.
\item If $P\downarrow_{\mathcal N} x$, then $Q\Downarrow_{\mathcal N} x$.
\end{enumerate}
$P$ is ${\mathcal N}$-barbed bisimilar to $Q$, written
$P \wbbisim_{\mathcal N} Q$, if $P \rel{S}_{\mathcal N} Q$ for some ${\mathcal N}$-barbed bisimulation ${\mathcal S}_{\mathcal N}$.
\end{definition}

$\mathcal{R} \subseteq \pi \times \pi$

$P \mathcal{R} Q => \forall P'. P \red P' \Rightarrow \exists Q'. Q \red Q', P' \mathcal{R} Q'$

$P \vdash x \Rightarrow Q \vdash x$

\begin{mathpar}
  \inferrule*[lab=Out-barb]{x \nameeq y}{{y}!\langle{Q}\rangle \vdash x}
  \and
  \inferrule*[lab=Par-barb]{\mbox{$P\vdash x$ or $Q\vdash x$}}{\binpar{P}{Q} \vdash x}
\end{mathpar}

\subsubsection{Contexts}

One of the principle advantages of computational calculi like the
$\pi$-calculus is a well-defined notion of context,
contextual-equivalence and a correlation between
contextual-equivalence and notions of bisimulation. The notion of
context allows the decomposition of a process into (sub-)process and
its syntactic environment, its context. Thus, a context may be
thought of as a process with a ``hole'' (written $\Box$) in it. The
application of a context $M$ to a process $P$, written $M[P]$, is
tantamount to filling the hole in $M$ with $P$. In this paper we do
not need the full weight of this theory, but do make use of the notion
of context in the proof the main theorem. 

\begin{mathpar}
  \inferrule* [lab=summation] {} {{M_{M},M_{N}} \bc \Box \;|\; x.M_{A} \;|\; M_{M}+M_{N}}
  \and
  \inferrule* [lab=agent] {} {{M_{A}} \bc (\vec{x})M_{P} \;| \; \clift{P_0,\ldots,M_{P},\ldots,P_N}}
  \and \\
  \inferrule* [lab=process] {} {{M_{P}} \bc M_{N} \;| \;P|M_{P} }
\end{mathpar} 

\begin{mathpar}
  \inferrule* [lab=sychronization] {} {M_{N} \bc \Box \;|\; x?M_{F} \;|\; x!M_{C}}
  \and
  \inferrule* [lab=abstraction] {} {{M_{F}} \bc (x)M_{P} }
  \and
  \inferrule* [lab=concretion] {} {{M_{C}} \bc \langle M_{P} \rangle }
  \and \\
  \inferrule* [lab=process] {} {{M_{P}} \bc M_{N} \;| \;P|M_{P} }
\end{mathpar}

\begin{definition}[contextual application] Given a context $M$, and
  process $P$, we define the \emph{contextual application}, $M[P] :=
  M\{P/\Box\}$. That is, the contextual application of M to P is the
  substitution of $P$ for $\Box$ in $M$.
\end{definition}

$\meaningof{-} : L \to \mathcal{P}(\pi)$

\begin{mathpar}
  \inferrule* [lab=collection] {} {\meaningof{true} = \pi, \and \meaningof{~E} = \pi \setminus \meaningof{E}, \and \meaningof{E_{1} \& E_{2}} = \meaningof{E_{1}} \cap \meaningof{E_{2}}}
\end{mathpar}

\begin{mathpar}
  \inferrule* [lab=structure] {} {\meaningof{0} = \{ P \in \pi | P \equiv 0 \}, \and \\ \meaningof{E_1 | E_2} = \{ P \in \pi | P \equiv P_{1} | P_{2}, P_{1} \in \meaningof{E_{1}}, P_{2} \in \meaningof{E_2}\} }
\end{mathpar}

\begin{mathpar}
 \inferrule* [lab=behavior] {} {\meaningof{\langle a?b \rangle E} = \{ P \in \pi | P \equiv Q | u?(y)P', \\ \and \\\\ \and \\ \;\;\; u \in \meaningof{a}, \forall z.P'\{z/y\} \in \meaningof{E\{z/b\}}\}, \and \\ \meaningof{a!E} = \{ P \in \pi | P \equiv Q | x!\langle P' \rangle, x \in \meaningof{a} P' \in \meaningof{E}\} }
\end{mathpar}

\begin{mathpar}
 \inferrule* [lab=nominal] {} {\meaningof{\quotep{E}} = \{ \quotep{P} \in \quotep{\pi} | P \in \meaningof{E} \}, \and \meaningof{\quotep{P}} = \{ \quotep{Q} \in \quotep{\pi} | P \equiv Q \} \and \\ \meaningof{@\quotep{E}} = \{ P \in \pi | P \equiv @x, x \in \meaningof{E} \}}
\end{mathpar}

\begin{eqnarray*}
  \\
  \meaningof{-} : TS \to ST
\end{eqnarray*}

\begin{eqnarray*}
  \\
  L : TS \to ST
\end{eqnarray*}

\begin{eqnarray*}
  \\
  P \models E \iff P \in \meaningof{E}
\end{eqnarray*}

\begin{eqnarray*}
  P \approx_{L} Q \iff \forall E \in L. P \models E \iff Q \models E
\end{eqnarray*}

\begin{eqnarray*}
  P \approx_{K} Q
\end{eqnarray*}

\begin{eqnarray*}
  P \approx Q
\end{eqnarray*}

$\approx_{K} = \approx = \approx_{L}$

\subsubsection{Contextual duality}

Note that contexts extend the quotation operation to a family of
operations from processes to names. Given a context, $M$, we can
define a \emph{nominal context}, $\quotep{M}$ by $\quotep{M}[P] :=
\quotep{M[P]}$. To foreshadow what is to come we observe that these
operations enjoy a duality with processes very much like the duality
between vectors and maps from vectors to scalars.

Further, because the calculus is essentially higher-order, we have a
correspondence between contexts and processes. More specifically,
given a name $x$ and a context $M$ we can construct $M^{*}_{x}$ such
that 

\begin{mathpar}
  M^{*}_{x} | \lift{x}{P} \red M[P]
\end{mathpar}

namely,

\begin{mathpar}
  M^{*}_{x} := x?(u).M[\dropn{u}]
\end{mathpar}

The dependence of $M^{*}_{x}$ on a name makes it an abstraction, 

\begin{mathpar}
  M^{*} := (x)x?(u).M[\dropn{u}]
\end{mathpar}

\subsection{Additional notation}

It will sometimes be convenient to denote the process a name
quotes. We already have the notation $x = \quotep{P}$, but it will be
convenient to introduce an alternate notation, $\procn{x}$, when we
want to emphasize the connection to the use of the name. Note that, by
virtue of name equivalence, $\quotep{\procn{x}} \nameeq x$; so, the
notation is consistent with previous definitions.

Further, because names have structure it is possible to effect
substitutions on the basis of that structure. This means we need to
upgrade our notation for substitutions, which we accomplish by
adapting comprehension notation. Thus,

\begin{mathpar}
  P\{ y / x : x \in S \}
\end{mathpar}

is interpreted to mean the process derived from P by replacing (in a
capture-avoiding manner) each occurrence of $x$ in $S$ by $y$. For example,

\begin{mathpar}
  P\{ \quotep{\procn{x}|\procn{x}} / x : x \in \freenames{P} \}
\end{mathpar}

will replace each (occurrence) of a free name $x$ in $P$ by
$\quotep{\procn{x}|\procn{x}}$.

Also, we will avail ourselves of the notation $x^{L}$ and $x^{R}$ to
denote injections of a name into disjoint copies of the name
space. There are numerous ways to accomplish this. One example can be
found in \cite{MeredithR05}. This notation overloads to vectors of
names: $\vec{x}^{\pi} := (x_{i}^{\pi} \; : \; 0 \leq i < |\vec{x}| )$ where $\pi \in \{L,R\}$.

We also use $P^{\Box} := P|\Box$.

In \cite{MeredithR05} an interpretation of the new operator is
given. It turns out that there are several possible interpretations
all enjoying the requisite algebraic properties of the operator (see
\cite{milner91polyadicpi}). We will therefore make liberal use of
$(\nu\; \vec{x})P$.

% subsection the_syntax_and_semantics_of_the_notation_system (end)   

\section{Interpretation of QM}
\subsection{Supporting definitions}
\subsubsection{Multiplication}
\begin{mathpar}
  \quotep{Q} \cdot \quotep{R} := \quotep{Q|R}
  \and \\
  \quotep{Q} \cdot P := P\{ \quotep{Q|R} / \quotep{R} : \quotep{R} \in \freenames{P} \}
\end{mathpar}

\paragraph{Discussion}
The first line needs little explanation. The second line says that
each free name of the process is replaced with the multiplication of
that name by the scalar. Multiplication of a scalar (name) by a state
(process) results in a process all the names of which have been `moved
over' by parallel composition with the process the scalar
quotes. There is a subtlety that the bound names have to be
manipulated so that multiplied names aren't accidentally
captured. There are many ways to achieve this.

\begin{remark}\label{rem:multiplication_identities}
  The reader is invited to verify that for all $x,y,z \in \QProc$ and $P \in \Proc$
  \begin{mathpar}
    x \cdot \quotep{0} \equiv x 
    \and
    x \cdot y \equiv y \cdot x
    \and
    x \cdot (y \cdot z) \equiv (x \cdot y) \cdot z
    \and \\
    \quotep{0} \cdot P \equiv P
    \and \\
    x \cdot (y \cdot P) \equiv (x \cdot y) \cdot P
    \and \\
    x \cdot (P|Q) \equiv (x \cdot P) | (x \cdot Q)
    \and \\    
  \end{mathpar}
\end{remark}

\subsubsection{Tensor product}

We define a tensor product on processes by structural induction.

\paragraph{Tensor of sums} First note that all summations, including
$\pzero$ and sequence, can be written $\Sigma_{i} x_{i}.A_{i} +
\Sigma_{j} x_{j}.C_{j}$, where we have grouped input-guarded processes
together and output-guarded processes together.

Thus, we can define the tensor product of two summations, $N_{1}\otimes N_{2}$, where

\begin{mathpar}
  N_{1} := \Sigma_{i} x_{i}.A_{i} + \Sigma_{j} x_{j}.C_{j}
  \and
  N_{2} := \Sigma_{i'} y_{i'}.B_{i'} + \Sigma_{j'} y_{j'}.D_{j'} 
\end{mathpar}

as follows.

\begin{mathpar}
  \Sigma_{i} x_{i}.A_{i} + \Sigma_{j} x_{j}.C_{j} \otimes \Sigma_{i'}
  y_{i'}.B_{i'} + \Sigma_{j'} y_{j'}.D_{j'} 
  \and \\
  := \; \Sigma_{i} \Sigma_{i'} \quotep{\stackrel{\vee}{x_{i}}| \stackrel{\vee}{y_{i'}}}.(A_{i}\otimes B_{i'}) \; | \; \Sigma_{i'} \Sigma_{i} \quotep{\stackrel{\vee}{y_{i'}}|\stackrel{\vee}{x_{i}}}.(B_{i'}\otimes A_{i})
  \and
  \;\; | \;\; \Sigma_{j} \Sigma_{j'} \quotep{\stackrel{\vee}{x_{j}}|\stackrel{\vee}{y_{j'}}}.(A_{j}\otimes B_{j'}) \; | \; \Sigma_{j'} \Sigma_{j} \quotep{\stackrel{\vee}{y_{j'}}|\stackrel{\vee}{x_{j}}}.(B_{j'}\otimes A_{j})
\end{mathpar}

\begin{remark}
  Do we need to $x^{L}$ and $y^{R}$ for this construction as well?
\end{remark}

\paragraph{Tensor of parallel compositions} Next, we distribute tensor
over par.

\begin{mathpar}
  P_{1}|P_{2} \otimes Q_{1}|Q_{2} := (P_{1} \otimes Q_{1}) | (P_{1}
  \otimes Q_{2}) | (P_{2} \otimes Q_{1}) | (P_{2} \otimes Q_{2})
\end{mathpar}

\paragraph{Tensor with dropped names} We treat tensor of a
process with a dropped name as parallel composition.

\begin{mathpar}
  P \otimes \dropn{x} := P | \dropn{x}
\end{mathpar}

\paragraph{Tensor of agents}

Finally, we need to define tensor on agents. Note that the definition
of tensor on normal products only tensors inputs with inputs and
outputs with outputs. Thus, we only have to define the operation on
``homogeneous'' pairings.

\begin{mathpar}
  (\vec{x})P \otimes (\vec{y})Q
  \and \\
  := (x_{0}^{L}|y_{0}^{R},\ldots,x_{0}^{L}|y_{n}^{R},\ldots,x_{m}^{L}|y_{0}^{R},\ldots,x_{m}^{L}|y_{n}^R)(P\{ \vec{x}^{L}/\vec{x}\} \otimes Q \{ \vec{y}^{R}/\vec{y}\})
  \and \\
  \clift{\vec{P}} \otimes \clift{\vec{Q}}
  \and \\
  := \clift{P_{0}\otimes Q_{0},\ldots,P_{0}\otimes Q_{n},\ldots,P_{m}\otimes Q_{0},\ldots,P_{m}\otimes Q_{n}}
\end{mathpar}

\begin{remark}
  Observe that arities of tensored abstractions matches arities of
  tensored concretions if the original arities matched. Note also that
  the length of the arities corresponds to the increase in dimension
  we see in ordinary vector space tensor product.
\end{remark}

\begin{remark}
  Operationally, this definition distributes the tensor down to
  components ``linked'' by summation. Tensor over summation is
  intriguing in that it mixes names. Moreover, as a consequence of the
  way it mixes names we have the identities for all $x \in \QProc$ and
  $P,Q \in \Proc$

  \begin{mathpar}
    (x \cdot P) \otimes Q \equiv x \cdot (P \otimes Q) \equiv P \otimes (x \cdot Q)
    \and
    P \otimes \pzero \equiv P
  \end{mathpar}

  that the reader is invited to verify.
\end{remark}

\subsubsection{Annihilation}
\begin{mathpar}
  P^{\perp} := \{ Q | \forall R. P|Q \red^{*} R \Rightarrow R \red^{*} \pzero \}
  \and \\
  P^{\underline{\perp}} := \Sigma_{Q \in P^{\perp}} \quotep{Q}?(y).(\dropn{y}|Q) | \Sigma_{Q \in P^{\perp}} \quotep{Q}\clift{\Box}
\end{mathpar}

\paragraph{Discussion} The reader will note that $P^{\perp}$ is a
\emph{set} of processes, while $P^{\underline{\perp}}$ is a
\emph{context}. We call the set $P^{\perp}$ the \emph{annihilators} of
$P$. The parallel composition of a process in the annihilators of $P$
with $P$ will result in a process, the state space of which has all
paths eventually leading to $\pzero$. Execution may endure loops; but
under reasonable conditions of fairness (naturally guaranteed under
most notions of bisimulation) such a composite process cannot get
stuck in such a loop and will, eventually pop out and terminate.

The context $P^{\underline{\perp}}$ is ready and willing to ``take the
$P$ out of'' the process to which it is applied. It will effectively
transmit the code of the process to which it is applied to one of the
annihilators and run the process against it.

\subsubsection{Evaluation}
We fix $M$ a domain of fully abstract interpretation with an equality
coincident with bisimulation. We take $\meaningof{\cdot} : \Proc \to
M$ to be the map interpreting processes and $\nmeaningof{\cdot} : \M
\to Proc$ to be the map running the other way. Then we define

\begin{mathpar}
  \int P := \nmeaningof{\meaningof{P}}
\end{mathpar}

\paragraph{Discussion}
There are many fully abstract interpretations of Milner's
$\pi$-calculus. Any of them can be used as a basis for interpreting
the reflective calculus here. Equipped with such a domain it is
largely a matter of grinding through to check that the Yoneda
construction for the normalization-by-evaluation program can be
extended to this setting.

\begin{remark}
  The reader is invited to verify that $\int (P^{\underline{\perp}}[P]) = 0$.
\end{remark}

\subsection{Quantum mechanics}

Table \ref{tbl:core_qm_op_defns} gives the core operational definitions

\begin{table}[htp]\label{tbl:core_qm_op_defns}
  \center{
    \fbox{
      \begin{tabular}{c|c}
        quantum mechanics & process calculus \\
        \hline
        scalar & $x := \quotep{P}$ \\
        state vector & $\state{P} := P$ \\
        dual & $\state{P}^{*} := \event{P^{\underline{\perp}}} := \quotep{P^{\underline{\perp}}}[-]$ \\
        matrix & $ \Sigma_{\alpha} \state{P_{\alpha}}x_{\alpha}\event{Q_{\alpha}}$ \\
        vector addition & $\state{P} + \state{Q} := \state{P | Q}$ \\
        tensor product & $\state{P} \otimes \state{Q} := \state{P \otimes Q}$ \\
        inner product & $\innerprod{P}{Q} := \quotep{\int P^{\underline{\perp}}[Q]}$ \\
      \end{tabular}
    }
  }
  \caption{QM - operational definitions}
\end{table}

where

\begin{mathpar}
  \prmatrix{P}{Q} := \fprmatrix{P}{\quotep{\pzero}}{Q}
  \and
  \fprmatrix{P}{x}{Q} := (\state{P},x,\event{Q})
  \and
  (\fprmatrix{P}{x}{Q})(\state{R}) := x \cdot \innerprod{Q}{R} \cdot \state{P}
  \and
  (\fprmatrix{P}{x}{Q})(\event{R}) := x \cdot \innerprod{R}{P} \cdot \event{Q}
\end{mathpar}

\paragraph{Discussion}
As promised: vectors (aka states) are represented as processes; duals
as contextual duals; inner product definition should be compared with
standard inner product definition for ....

\begin{remark}
  Assuming $\int (P^{\underline{\perp}}[P]) = 0$, the reader is
  invited to verify that $(\fprmatrix{P}{x}{P})(\state{P}) = x \cdot \state{P}$.
\end{remark}

\begin{remark}
  The reader is invited to verify that $\innerprod{P}{Q}$ could
  equally well have been written $\quotep{\int \stackrel{\vee}{x}}$
  where $x = \event{P^{\underline{\perp}}}(Q)$.

  One of the motivations for this remark is that there is another way
  to factor these operations. We could package up evaluation in the dual:

  \begin{mathpar}
    \state{P}^{*} := \event{\int P^{\underline{\perp}}} := \quotep{\int P^{\underline{\perp}}}[-]
  \end{mathpar}

  and then have inner product defined by
  
  \begin{mathpar}
    \innerprod{P}{Q} := \event{P}(Q)
  \end{mathpar}

  Hopefully, experience with the calculations will provide guidance on
  the best factoring.
\end{remark}

\begin{remark}
  Assuming $\int (P^{\underline{\perp}}[P]) = 0$, the reader is
  invited to verify that $\forall P,Q. (\prmatrix{0}{Q})(\state{0}) =
  \state{0}$ and dually $(\prmatrix{P}{0})(\event{0}) = \event{0}$.
\end{remark}

\begin{remark}
  i'm a little worried that i don't (yet) have proper support for
  complex conjugacy. But, the observation above may give us a
  clue. According to Abramsky, it must be the case that the scalars
  are iso to the homset of the identity for the tensor -- which the
  observation above characterizes. 

  For now, we will simply bookmark the notion with $\overline{x}$.
\end{remark}

\subsubsection{Adjointness}

We need to give a definition of $(\cdot)^{\dagger}$ for matrices. The
obvious candidate definition is
\begin{mathpar}
(\Sigma_{\alpha}\fprmatrix{P_{\alpha}}{x_{\alpha}}{Q_{\alpha}})^{\dagger}
= \Sigma_{\alpha}\fprmatrix{(Q_{\alpha}^{\underline{\perp}})^{*}}{\overline{x}_{\alpha}}{P_{\alpha}^{\underline{\perp}}} 
\end{mathpar}

But, $(Q_{\alpha}^{\underline{\perp}})^{*}$ requires a name along
which to communicate the process to achieve the context application.

\subsubsection{Basis for a basis}
If processes label states and ``addition'' of states (a.k.a. vector
addition) is interpreted as parallel composition, what corresponds to
notions of linear independence and basis? Here, we recall that Yoshida
has developed a set of \emph{combinators} for an asynchronous verison
of Milner's $\pi$-calculus. These are a finite set of processes such
any process can be expressed as parallel composition of these
combinators together with liberal uses of the new operator and
replication. We can simply give a translation of these into the
present calculus and have reasonable expectation that the property
carries over. That is, that the resultant set allows to express all
processes via parallel composition. Note, however, that there is no
new operator or replication in this calculus. As a result, we expect
that the corresponding set is actually infinite. That is, we expect
that the space is actually infinite dimensional.

\begin{remark}
  The attentive reader may be a bit concerned. Certainly, the
  collection $S$, $K$ and $I$ is a finite set of
  combinators. Shouldn't we expect to see a finite set of combinators
  for an effectively equivalent system? i am very sympathetic to this
  critique and feel it warrants full attention. On the other hand, i
  also have in mind the following analogy. The natural numbers, as a
  monoid under addition, has exactly $1$ generator, while the natural
  numbers, as a monoid under multiplication, has countably many
  generators (the primes). We observe that the application of the
  lambda calculus is much less resource sensitive than the parallel
  composition of the $\pi$-calculus. Could it be the case that we have
  an analogy of the form
  
  \begin{mathpar}
    m + n : MN :: m*n : M|N
  \end{mathpar}

  giving a similar blow up in the set of ``primes''?  This is such a
  wonderful thought that, even if it's not true, i think it's worth
  writing down.
\end{remark}
 

\documentclass[12pt]{llncs}
%\documentclass{jktr}

\usepackage[pdftex]{hyperref}                   
\usepackage {listings}
\usepackage {mathpartir}
\usepackage{bcprules}
%\usepackage{listings}
                       
\usepackage{graphicx} 
%\usepackage[margins=2.5cm,nohead,nofoot]{geometry}
%\usepackage{geometry}
\usepackage{amsfonts}
\usepackage{amstext}
\usepackage{latexsym}
\usepackage{amssymb}
\usepackage{color}


%\include{myPreamble}
\include{qm2pi.local} 

%\ifpdf
%\usepackage[pdftex]{graphicx}
%\else
%\usepackage{graphicx}
%\fi

 % \ifpdf
%  \usepackage{pdfsync}
%  \if


%\title{Brief Article}
%\author{David F. Snyder}
%\author{L.G. Meredith}

%\address{Dept. of Math., Texas State University--San Marcos, San Marcos, TX 78666}
       
\pagestyle{empty}


\begin{document}

\lstset{language=[Objective]Caml,frame=shadowbox}

\input{qm2pi.front}

% section front matter (end)

\input{qm2pi.intro} 
 
% section introduction (end)

% \input{qm2pi.knotations} 

% section notation (end)

\input{qm2pi.process.calculi} 

% section concurrent_process_calculi_and_spatial_logics_ (end)
    
%\input{qm2pi.knots2pi} 

%\input{qm2pi.trefoil} 

%\input{qm2pi.mainthm} 

% subsection basic_interpretation (end)

%\input{qm2pi.rho.presentation} 
\subsection{The syntax and semantics of the notation system}\label{sub:the_syntax_and_semantics_of_the_notation_system} % (fold)

We now summarize a technical presentation of the calculus that
embodies our theory of dynamics. The typical presentation of such a
calculus follows the style of giving generators and relations on
them. The grammar, below, describing term constructors, freely
generates the set of processes, $\Proc$. This set is then quotiented
by a relation known as structural congruence and it is over this set
that the notion of dynamics is expressed. This presentation is
essentially that of \cite{MeredithR05} with the addition of
polyadicity and summation. For readability we have relegated some of
the technical subtleties to an appendix.

\subsubsection{Process grammar}\label{subsub:process_grammar}

\begin{mathpar}
  \inferrule* [lab=synchronization] {} {{M} \bc \pzero \;|\; x?F \;|\; x!C }
  \and
  \inferrule* [lab=abstraction] {} {{F} \bc (x)P}
  \and
  \inferrule* [lab=concretion] {} {{C} \bc \langle Q \rangle}
  \and
  \inferrule* [lab=process] {} {{P,Q} \bc M \;| \;P|Q \;|\; @{x}}
  \and
  \inferrule* [lab=name] {} {{x} \bc \quotep{P}}
\end{mathpar} 

Note that $\vec{x}$ (resp. $\vec{P}$) denotes a vector of names
(resp. processes) of length $|\vec{x}|$ (resp. $|\vec{P}|$). We adopt
the following useful abbreviations.

\begin{mathpar}
   x?(\vec{y}).P := x.(\vec{y})P \and  x\clift{\vec{P}} := x.\clift{\vec{P}}
   \and x!(y) := \lift{x}{\dropn{y}}
   \and \Pi_{i=0}^{n-1}P_i := P_0 | \ldots | P_{n-1}
\end{mathpar}

\subsubsection{Structural congruence}

\paragraph{Free and bound names and alpha-equivalence.} At the
core of structural equivalence is alpha-equivalence which identifies
process that are the same up to a change of variable. Formally, we
recognize the distinction between free and bound names. The free names
of a process, $\freenames{P}$, may be calculated recursively as
follows:

\begin{mathpar}
\freenames{\pzero} := \emptyset
  \and \\
  \freenames{x?(y).P} := \{ x \} \cup (\freenames{P} \setminus \{ y \})
  \and 
  \freenames{x!\langle P \rangle} := \{ x \} \cup \{ P \} 
  \and \\
  \freenames{P|Q} := \freenames{P} \cup \freenames{Q}
  \and \\
  \freenames{@{x}} := \{ x \}
\end{mathpar}

$\pi$
$\quotep{\pi}$

$\freenames{-} : \pi \to \mathcal{P}(\quotep{\pi})$

\begin{eqnarray*}
  \freenames{\pzero} & := & \emptyset \\
  \freenames{x?(y).P} & := & \{ x \} \cup (\freenames{P} \setminus \{ y \}) \\
  \freenames{x!\langle P \rangle} & := & \{ x \} \cup \{ P \} \\
  \freenames{P|Q} & := & \freenames{P} \cup \freenames{Q} \\
  \freenames{\dropn{x}} & := & \{ x \}
\end{eqnarray*}

The bound names of a process, $\boundnames{P}$, are those names occurring in $P$
that are not free. For example, in $x?(y).0$, the name $x$ is free, while $y$ is bound.

\begin{mathpar}
  \inferrule* [lab=monoidal-laws] {} { P|Q \equiv Q|P \and P|0 \equiv P \and P|(Q|R) \equiv (P|Q)|R }
\end{mathpar}

\begin{mathpar}
  \inferrule* [lab=alpha-equivalence] {} { (x)P \equiv (y)P\{y/x\} \and y \not\in \freenames{P} }
\end{mathpar}

\begin{definition}
Then two processes, $P,Q$, are alpha-equivalent if $P = Q\{\vec{y}/\vec{x}\}$ for
some $\vec{x} \in \boundnames{Q},\vec{y} \in \boundnames{P}$, where $Q\{\vec{y}/\vec{x}\}$
denotes the capture-avoiding substitution of $\vec{y}$ for $\vec{x}$ in $Q$.
\end{definition}

\begin{definition}
  The {\em structural congruence} \cite{SangiorgiWalker} , $\equiv$,
  between processes is the least congruence containing
  alpha-equivalence, satisfying the abelian monoid laws
  (associativity, commutativity and $\pzero$ as identity) for parallel
  composition $|$ and for summation $+$.
\end{definition}

\subsection{Name equivalence}

We take name equivalence, written $\nameeq$, to be the smallest
equivalence relation generated by the following rules.

\begin{mathpar}
\inferrule*[lab=Quote-drop]
{ }
{ \quotep{@{x}} \nameeq x }

\inferrule*[lab=Struct-equiv]
{ P \scong Q }
{ \quotep{P} \nameeq \quotep{Q} }
\end{mathpar}

The astute reader will have noticed that the mutual recursion of names
and processes imposes a mutual recursion on alpha-equivalence and
structural equivalence via name-equivalence. Fortunately, all of this
works out pleasantly and we may calculate in the natural way, free of
concern. The reader interested in the details is referred to the
appendix \ref{appendix:rho_details}.

\subsection{Substitution}

We use $\Proc$ for the set of processes, $\QProc$ for the set of
names, and $\id{\{}\vec{y} / \vec{x} \id{\}}$ to denote partial maps,
$s : \QProc \rightarrow \QProc$. A map, $s$ lifts, uniquely, to a map
on process terms, $\widehat{s} : \Proc \rightarrow \Proc$ by the
following equations.

\begin{mathpar}
  (0) \psubstp{Q}{P} := 0 \\
  (R \juxtap S) \psubstp{Q}{P}
  :=    
  (R)\psubstp{Q}{P} \juxtap (S) \psubstp{Q}{P} \\
  (x?(y).R) \psubstp{Q}{P}    
  :=    
  (x)\substp{Q}{P} (z)\concat( (R \psubstn{z}{y}) \psubstp{Q}{P} ) \\
  (\lift{x}{R}) \psubstp{Q}{P}  
  :=
  \lift{(x)\substp{Q}{P}}{ R \psubstp{Q}{P} } \\
%   (\dropn{x})  \psubstp{Q}{P}       
%   := 
%   \left\{ 
%     \begin{array}{ccc} 
%       \dropn{\quotep{Q}} & & x \nameeq \quotep{P} \\
%       \dropn{x} & & otherwise \\
%     \end{array}
%   \right. 
  (\dropn{x})  \psubstp{Q}{P}       
  := 
  \left\{ 
    \begin{array}{ccc} 
      Q & & x \nameeq \quotep{P} \\
      \dropn{x} & & otherwise \\
    \end{array}
  \right.
\end{mathpar}
 

where

\begin{eqnarray}
  (x)\id{\{} \lpquote Q \rpquote / \lpquote P \rpquote \id{\}}            = 
  \left\{ 
    \begin{array}{ccc}
      \lpquote Q \rpquote & & x \nameeq \lpquote P \rpquote \\
      x & & otherwise \\
    \end{array}
  \right. \nonumber
\end{eqnarray}

and $z$ is chosen distinct from $\quotep{P}$, $\quotep{Q}$, the free
names in $Q$, and all the names in $R$. Our $\alpha$-equivalence will
be built in the standard way from this substitution.

\begin{remark}\label{rem:no_self_referential_names}
  One consequence of these definitions is that $\forall P. \quotep{P}
  \not\in \freenames{P}$.
\end{remark}

\subsection{ Dynamic quote: an example }

Anticipating something of what's to come, consider applying the
substitution, $\widehat{\id{\{}u / z \id{\}}}$, to the following pair
of processes, $\lift{w}{y!(z)}$ and $w[ \lpquote y!(z) \rpquote ]$.

\begin{eqnarray}
	\lift{w}{y!(z)}\widehat{\id{\{}u / z \id{\}}}
		& = &
		\lift{w}{y!(u)} \nonumber\\
	w[ \lpquote y!(z) \rpquote ] \widehat{ \id{\{}u / z \id{\}} }
		& = &
		w[ \lpquote y!(z) \rpquote ] \nonumber
\end{eqnarray}

Because the body of the process between quotes is impervious to
substitution, we get radically different answers. In fact, by
examining the first process in an input context,
e.g. $x?(z).\lift{w}{y!(z)}$, we see that the process under the lift
operator may be shaped by prefixed inputs binding a name inside it. In
this sense, the lift operator will be seen as a way to dynamically
construct processes before reifying them as names.

Finally equipped with these standard features we can present the
dynamics of the calculus.

\subsubsection{Operational semantics} 

Finally, we introduce the computational dynamics. What marks these
algebras as distinct from other more traditionally studied algebraic
structures, e.g. vector spaces or polynomial rings, is the manner in
which dynamics is captured. In traditional structures, dynamics is typically
expressed through morphisms between such structures, as in linear maps
between vector spaces or morphisms between rings. In algebras
associated with the semantics of computation, the dynamics is
expressed as part of the algebraic structure itself, through a
reduction reduction relation typically denoted by $\red$. Below, we
give a recursive presentation of this relation for the calculus used
in the encoding.

$\red \subseteq \pi \times \pi$
$\red : \pi \to \mathcal{P}(\pi)$

\begin{mathpar}
  \inferrule* [lab=Comm] { \textsf{match}( x_{src}, x_{trgt} ) } { x_{trgt}?(y)P \; | \; x_{src}!\langle {Q} \rangle \red P\{\quotep{Q}/y}\} }
  \and \\
  \inferrule* [lab=Par] {{P} \red {P}'} {{{P} | {Q}} \red {{P}' | {Q}}}
  \and
  \inferrule* [lab=Equiv]{{{P} \scong {P}'} \andalso {{P}' \red {Q}'} \andalso {{Q}' \scong {Q}}}{{P} \red {Q}}
\end{mathpar}

\begin{eqnarray*}
  match_{\equiv} (\quotep{P},\quotep{Q}) & := & P \equiv Q \\
  match_{\dagger}(\quotep{P},\quotep{Q}) & := & \forall R. P|Q \red^{*} R => R \red^{*} 0 \\
  match_{K}(\quotep{P},\quotep{Q}) & := & K \mbox{ for some context } K
\end{eqnarray*}

$u?(x)P | u!\langle Q \rangle \red P\{\quotep{Q}/x\}$

%We write $\wred$ for $\red^*$, and $P\red$ if $\exists Q $ such that $ P \red Q$.
We write $P\red$ if $\exists Q $ such that $ P \red Q$ and $P\not\red$, otherwise.

\section{Replication}

As mentioned before, it is known that replication (and hence
recursion) can be implemented in a higher-order process algebra
\cite{SangiorgiWalker}. As our first example of calculation with the
machinery thus far presented we give the construction explicitly in
the {\rhoc}.

\begin{eqnarray}
	D_{x} & := & \prefix{x}{y}{(\binpar{\outputp{x}{y}}{@{y}})} \nonumber\\
	\bangp_{x}{P} & := & \binpar{{x}!\langle{\binpar{D_{x}}{P}}\rangle}{D_{x}} \nonumber
\end{eqnarray}

\begin{eqnarray}
	\bangp_{x}{P} & & \nonumber\\
	=
	& {x}!\langle{(\prefix{x}{y}{(\outputp{x}{y} | @{y})) | P}}\rangle 
	      | \prefix{x}{y}{(\outputp{x}{y} | @{y})} & \nonumber\\
	\red
	& (\outputp{x}{y} | @{y})\substn{\quotep{(\prefix{x}{y}{(@{y} | \outputp{x}{y})) | P}}}{y} & \nonumber\\
	=
	& \outputp{x}{\quotep{(\prefix{x}{y}{(\outputp{x}{y} | @{y})) | P}}}
	  | {(\prefix{x}{y}{(\outputp{x}{y} | @{y})) | P}} & \nonumber\\
	\red
	& \ldots & \nonumber\\
	\red^*
	& P | P | \ldots & \nonumber
\end{eqnarray}

Of course, this encoding, as an implementation, runs away, unfolding
$\bangp{P}$ eagerly. A lazier and more implementable replication
operator, restricted to input-guarded processes, may be obtained as follows.

\begin{eqnarray}
\bangp{\prefix{u}{v}{P}} 
	:= 
	\binpar{\lift{x}{\prefix{u}{v}{(\binpar{D(x)}{P})}}}{D(x)} \nonumber
\end{eqnarray}

\begin{remark}
  Note that the lazier definition still does not deal with summation
  or mixed summation (i.e. sums over input and output). The reader is
  invited to construct definitions of replication that deal with these
  features. 

  Further, the definitions are parameterized in a name, $x$. Can you,
  gentle reader, make a definition that eliminates this parameter and
  guarantees no accidental interaction between the replication
  machinery and the process being replicated -- i.e. no accidental
  sharing of names used by the process to get its work done and the
  name(s) used by the replication to effect copying. This latter
  revision of the definition of replication is crucial to obtaining
  the expected identity $!!P \sim !P$.
\end{remark}

\begin{remark}\label{rem:paradoxical_combinator}
  The reader familiar with the lambda calculus will have noticed the
  similarity between $D$ and the paradoxical combinator.

  [Ed. note: the existence of this seems to suggest we have to be more
  restrictive on the set of processes and names we admit if we are to
  support no-cloning.]
\end{remark}

\subsubsection{Bisimulation}

The computational dynamics gives rise to another kind of equivalence,
the equivalence of computational behavior. As previously mentioned
this is typically captured \emph{via} some form of bisimulation.

% The notion we use in this paper is weak barbed bisimulation
% \cite{milner91polyadicpi}.

The notion we use in this paper is derived from weak barbed
bisimulation \cite{milner91polyadicpi}. 

\begin{definition}
An \emph{observation relation}, $\downarrow_{\mathcal N}$, over a set
of names, $\mathcal N$, is the smallest relation satisfying the rules
below.

\infrule[Out-barb]{y \in {\mathcal N}, \; x \nameeq y}
		  {\outputp{x}{v} \downarrow_{\mathcal N} x}
\infrule[Par-barb]{\mbox{$P\downarrow_{\mathcal N} x$ or $Q\downarrow_{\mathcal N} x$}}
		  {\binpar{P}{Q} \downarrow_{\mathcal N} x}

We write $P \Downarrow_{\mathcal N} x$ if there is $Q$ such that 
$P \wred Q$ and $Q \downarrow_{\mathcal N} x$.
\end{definition}

\begin{definition}
%\label{def.bbisim}
An  ${\mathcal N}$-\emph{barbed bisimulation} over a set of names, ${\mathcal N}$, is a symmetric binary relation 
${\mathcal S}_{\mathcal N}$ between agents such that $P\rel{S}_{\mathcal N}Q$ implies:
\begin{enumerate}
\item If $P \red P'$ then $Q \wred Q'$ and $P'\rel{S}_{\mathcal N} Q'$.
\item If $P\downarrow_{\mathcal N} x$, then $Q\Downarrow_{\mathcal N} x$.
\end{enumerate}
$P$ is ${\mathcal N}$-barbed bisimilar to $Q$, written
$P \wbbisim_{\mathcal N} Q$, if $P \rel{S}_{\mathcal N} Q$ for some ${\mathcal N}$-barbed bisimulation ${\mathcal S}_{\mathcal N}$.
\end{definition}

$\mathcal{R} \subseteq \pi \times \pi$

$P \mathcal{R} Q => \forall P'. P \red P' \Rightarrow \exists Q'. Q \red Q', P' \mathcal{R} Q'$

$P \vdash x \Rightarrow Q \vdash x$

\begin{mathpar}
  \inferrule*[lab=Out-barb]{x \nameeq y}{{y}!\langle{Q}\rangle \vdash x}
  \and
  \inferrule*[lab=Par-barb]{\mbox{$P\vdash x$ or $Q\vdash x$}}{\binpar{P}{Q} \vdash x}
\end{mathpar}

\subsubsection{Contexts}

One of the principle advantages of computational calculi like the
$\pi$-calculus is a well-defined notion of context,
contextual-equivalence and a correlation between
contextual-equivalence and notions of bisimulation. The notion of
context allows the decomposition of a process into (sub-)process and
its syntactic environment, its context. Thus, a context may be
thought of as a process with a ``hole'' (written $\Box$) in it. The
application of a context $M$ to a process $P$, written $M[P]$, is
tantamount to filling the hole in $M$ with $P$. In this paper we do
not need the full weight of this theory, but do make use of the notion
of context in the proof the main theorem. 

\begin{mathpar}
  \inferrule* [lab=summation] {} {{M_{M},M_{N}} \bc \Box \;|\; x.M_{A} \;|\; M_{M}+M_{N}}
  \and
  \inferrule* [lab=agent] {} {{M_{A}} \bc (\vec{x})M_{P} \;| \; \clift{P_0,\ldots,M_{P},\ldots,P_N}}
  \and \\
  \inferrule* [lab=process] {} {{M_{P}} \bc M_{N} \;| \;P|M_{P} }
\end{mathpar} 

\begin{mathpar}
  \inferrule* [lab=sychronization] {} {M_{N} \bc \Box \;|\; x?M_{F} \;|\; x!M_{C}}
  \and
  \inferrule* [lab=abstraction] {} {{M_{F}} \bc (x)M_{P} }
  \and
  \inferrule* [lab=concretion] {} {{M_{C}} \bc \langle M_{P} \rangle }
  \and \\
  \inferrule* [lab=process] {} {{M_{P}} \bc M_{N} \;| \;P|M_{P} }
\end{mathpar}

\begin{definition}[contextual application] Given a context $M$, and
  process $P$, we define the \emph{contextual application}, $M[P] :=
  M\{P/\Box\}$. That is, the contextual application of M to P is the
  substitution of $P$ for $\Box$ in $M$.
\end{definition}

$\meaningof{-} : L \to \mathcal{P}(\pi)$

\begin{mathpar}
  \inferrule* [lab=collection] {} {\meaningof{true} = \pi, \and \meaningof{~E} = \pi \setminus \meaningof{E}, \and \meaningof{E_{1} \& E_{2}} = \meaningof{E_{1}} \cap \meaningof{E_{2}}}
\end{mathpar}

\begin{mathpar}
  \inferrule* [lab=structure] {} {\meaningof{0} = \{ P \in \pi | P \equiv 0 \}, \and \\ \meaningof{E_1 | E_2} = \{ P \in \pi | P \equiv P_{1} | P_{2}, P_{1} \in \meaningof{E_{1}}, P_{2} \in \meaningof{E_2}\} }
\end{mathpar}

\begin{mathpar}
 \inferrule* [lab=behavior] {} {\meaningof{\langle a?b \rangle E} = \{ P \in \pi | P \equiv Q | u?(y)P', \\ \and \\\\ \and \\ \;\;\; u \in \meaningof{a}, \forall z.P'\{z/y\} \in \meaningof{E\{z/b\}}\}, \and \\ \meaningof{a!E} = \{ P \in \pi | P \equiv Q | x!\langle P' \rangle, x \in \meaningof{a} P' \in \meaningof{E}\} }
\end{mathpar}

\begin{mathpar}
 \inferrule* [lab=nominal] {} {\meaningof{\quotep{E}} = \{ \quotep{P} \in \quotep{\pi} | P \in \meaningof{E} \}, \and \meaningof{\quotep{P}} = \{ \quotep{Q} \in \quotep{\pi} | P \equiv Q \} \and \\ \meaningof{@\quotep{E}} = \{ P \in \pi | P \equiv @x, x \in \meaningof{E} \}}
\end{mathpar}

\begin{eqnarray*}
  \\
  \meaningof{-} : TS \to ST
\end{eqnarray*}

\begin{eqnarray*}
  \\
  L : TS \to ST
\end{eqnarray*}

\begin{eqnarray*}
  \\
  P \models E \iff P \in \meaningof{E}
\end{eqnarray*}

\begin{eqnarray*}
  P \approx_{L} Q \iff \forall E \in L. P \models E \iff Q \models E
\end{eqnarray*}

\begin{eqnarray*}
  P \approx_{K} Q
\end{eqnarray*}

\begin{eqnarray*}
  P \approx Q
\end{eqnarray*}

$\approx_{K} = \approx = \approx_{L}$

\subsubsection{Contextual duality}

Note that contexts extend the quotation operation to a family of
operations from processes to names. Given a context, $M$, we can
define a \emph{nominal context}, $\quotep{M}$ by $\quotep{M}[P] :=
\quotep{M[P]}$. To foreshadow what is to come we observe that these
operations enjoy a duality with processes very much like the duality
between vectors and maps from vectors to scalars.

Further, because the calculus is essentially higher-order, we have a
correspondence between contexts and processes. More specifically,
given a name $x$ and a context $M$ we can construct $M^{*}_{x}$ such
that 

\begin{mathpar}
  M^{*}_{x} | \lift{x}{P} \red M[P]
\end{mathpar}

namely,

\begin{mathpar}
  M^{*}_{x} := x?(u).M[\dropn{u}]
\end{mathpar}

The dependence of $M^{*}_{x}$ on a name makes it an abstraction, 

\begin{mathpar}
  M^{*} := (x)x?(u).M[\dropn{u}]
\end{mathpar}

\subsection{Additional notation}

It will sometimes be convenient to denote the process a name
quotes. We already have the notation $x = \quotep{P}$, but it will be
convenient to introduce an alternate notation, $\procn{x}$, when we
want to emphasize the connection to the use of the name. Note that, by
virtue of name equivalence, $\quotep{\procn{x}} \nameeq x$; so, the
notation is consistent with previous definitions.

Further, because names have structure it is possible to effect
substitutions on the basis of that structure. This means we need to
upgrade our notation for substitutions, which we accomplish by
adapting comprehension notation. Thus,

\begin{mathpar}
  P\{ y / x : x \in S \}
\end{mathpar}

is interpreted to mean the process derived from P by replacing (in a
capture-avoiding manner) each occurrence of $x$ in $S$ by $y$. For example,

\begin{mathpar}
  P\{ \quotep{\procn{x}|\procn{x}} / x : x \in \freenames{P} \}
\end{mathpar}

will replace each (occurrence) of a free name $x$ in $P$ by
$\quotep{\procn{x}|\procn{x}}$.

Also, we will avail ourselves of the notation $x^{L}$ and $x^{R}$ to
denote injections of a name into disjoint copies of the name
space. There are numerous ways to accomplish this. One example can be
found in \cite{MeredithR05}. This notation overloads to vectors of
names: $\vec{x}^{\pi} := (x_{i}^{\pi} \; : \; 0 \leq i < |\vec{x}| )$ where $\pi \in \{L,R\}$.

We also use $P^{\Box} := P|\Box$.

In \cite{MeredithR05} an interpretation of the new operator is
given. It turns out that there are several possible interpretations
all enjoying the requisite algebraic properties of the operator (see
\cite{milner91polyadicpi}). We will therefore make liberal use of
$(\nu\; \vec{x})P$.

% subsection the_syntax_and_semantics_of_the_notation_system (end)   

\input{qm2pi.qmops} 

\input{qm2pi.sterngerlach} 

\input{qm2pi.metric} 

% section concurrent_process_calculi (end)

%\input{qm2pi.proofsketch}

% section proof sketch (end)

%\input{qm2pi.slviaknots} 

% section spatial logic via knots (end)

\input{qm2pi.conclusion}

% section conclusion (end)

%\input{qm2pi.dtcodes} 

% section wiring algorithm (end)

\input{qm2pi.ack} 

% section acknowledgments (end)

\newpage


\bibliographystyle{plain}   
\bibliography{../../biblios/main.bib}

\input{qm2pi.rhodetails}

\end{document}

 

\documentclass[12pt]{llncs}
%\documentclass{jktr}

\usepackage[pdftex]{hyperref}                   
\usepackage {listings}
\usepackage {mathpartir}
\usepackage{bcprules}
%\usepackage{listings}
                       
\usepackage{graphicx} 
%\usepackage[margins=2.5cm,nohead,nofoot]{geometry}
%\usepackage{geometry}
\usepackage{amsfonts}
\usepackage{amstext}
\usepackage{latexsym}
\usepackage{amssymb}
\usepackage{color}


%\include{myPreamble}
\include{qm2pi.local} 

%\ifpdf
%\usepackage[pdftex]{graphicx}
%\else
%\usepackage{graphicx}
%\fi

 % \ifpdf
%  \usepackage{pdfsync}
%  \if


%\title{Brief Article}
%\author{David F. Snyder}
%\author{L.G. Meredith}

%\address{Dept. of Math., Texas State University--San Marcos, San Marcos, TX 78666}
       
\pagestyle{empty}


\begin{document}

\lstset{language=[Objective]Caml,frame=shadowbox}

\input{qm2pi.front}

% section front matter (end)

\input{qm2pi.intro} 
 
% section introduction (end)

% \input{qm2pi.knotations} 

% section notation (end)

\input{qm2pi.process.calculi} 

% section concurrent_process_calculi_and_spatial_logics_ (end)
    
%\input{qm2pi.knots2pi} 

%\input{qm2pi.trefoil} 

%\input{qm2pi.mainthm} 

% subsection basic_interpretation (end)

%\input{qm2pi.rho.presentation} 
\subsection{The syntax and semantics of the notation system}\label{sub:the_syntax_and_semantics_of_the_notation_system} % (fold)

We now summarize a technical presentation of the calculus that
embodies our theory of dynamics. The typical presentation of such a
calculus follows the style of giving generators and relations on
them. The grammar, below, describing term constructors, freely
generates the set of processes, $\Proc$. This set is then quotiented
by a relation known as structural congruence and it is over this set
that the notion of dynamics is expressed. This presentation is
essentially that of \cite{MeredithR05} with the addition of
polyadicity and summation. For readability we have relegated some of
the technical subtleties to an appendix.

\subsubsection{Process grammar}\label{subsub:process_grammar}

\begin{mathpar}
  \inferrule* [lab=synchronization] {} {{M} \bc \pzero \;|\; x?F \;|\; x!C }
  \and
  \inferrule* [lab=abstraction] {} {{F} \bc (x)P}
  \and
  \inferrule* [lab=concretion] {} {{C} \bc \langle Q \rangle}
  \and
  \inferrule* [lab=process] {} {{P,Q} \bc M \;| \;P|Q \;|\; @{x}}
  \and
  \inferrule* [lab=name] {} {{x} \bc \quotep{P}}
\end{mathpar} 

Note that $\vec{x}$ (resp. $\vec{P}$) denotes a vector of names
(resp. processes) of length $|\vec{x}|$ (resp. $|\vec{P}|$). We adopt
the following useful abbreviations.

\begin{mathpar}
   x?(\vec{y}).P := x.(\vec{y})P \and  x\clift{\vec{P}} := x.\clift{\vec{P}}
   \and x!(y) := \lift{x}{\dropn{y}}
   \and \Pi_{i=0}^{n-1}P_i := P_0 | \ldots | P_{n-1}
\end{mathpar}

\subsubsection{Structural congruence}

\paragraph{Free and bound names and alpha-equivalence.} At the
core of structural equivalence is alpha-equivalence which identifies
process that are the same up to a change of variable. Formally, we
recognize the distinction between free and bound names. The free names
of a process, $\freenames{P}$, may be calculated recursively as
follows:

\begin{mathpar}
\freenames{\pzero} := \emptyset
  \and \\
  \freenames{x?(y).P} := \{ x \} \cup (\freenames{P} \setminus \{ y \})
  \and 
  \freenames{x!\langle P \rangle} := \{ x \} \cup \{ P \} 
  \and \\
  \freenames{P|Q} := \freenames{P} \cup \freenames{Q}
  \and \\
  \freenames{@{x}} := \{ x \}
\end{mathpar}

$\pi$
$\quotep{\pi}$

$\freenames{-} : \pi \to \mathcal{P}(\quotep{\pi})$

\begin{eqnarray*}
  \freenames{\pzero} & := & \emptyset \\
  \freenames{x?(y).P} & := & \{ x \} \cup (\freenames{P} \setminus \{ y \}) \\
  \freenames{x!\langle P \rangle} & := & \{ x \} \cup \{ P \} \\
  \freenames{P|Q} & := & \freenames{P} \cup \freenames{Q} \\
  \freenames{\dropn{x}} & := & \{ x \}
\end{eqnarray*}

The bound names of a process, $\boundnames{P}$, are those names occurring in $P$
that are not free. For example, in $x?(y).0$, the name $x$ is free, while $y$ is bound.

\begin{mathpar}
  \inferrule* [lab=monoidal-laws] {} { P|Q \equiv Q|P \and P|0 \equiv P \and P|(Q|R) \equiv (P|Q)|R }
\end{mathpar}

\begin{mathpar}
  \inferrule* [lab=alpha-equivalence] {} { (x)P \equiv (y)P\{y/x\} \and y \not\in \freenames{P} }
\end{mathpar}

\begin{definition}
Then two processes, $P,Q$, are alpha-equivalent if $P = Q\{\vec{y}/\vec{x}\}$ for
some $\vec{x} \in \boundnames{Q},\vec{y} \in \boundnames{P}$, where $Q\{\vec{y}/\vec{x}\}$
denotes the capture-avoiding substitution of $\vec{y}$ for $\vec{x}$ in $Q$.
\end{definition}

\begin{definition}
  The {\em structural congruence} \cite{SangiorgiWalker} , $\equiv$,
  between processes is the least congruence containing
  alpha-equivalence, satisfying the abelian monoid laws
  (associativity, commutativity and $\pzero$ as identity) for parallel
  composition $|$ and for summation $+$.
\end{definition}

\subsection{Name equivalence}

We take name equivalence, written $\nameeq$, to be the smallest
equivalence relation generated by the following rules.

\begin{mathpar}
\inferrule*[lab=Quote-drop]
{ }
{ \quotep{@{x}} \nameeq x }

\inferrule*[lab=Struct-equiv]
{ P \scong Q }
{ \quotep{P} \nameeq \quotep{Q} }
\end{mathpar}

The astute reader will have noticed that the mutual recursion of names
and processes imposes a mutual recursion on alpha-equivalence and
structural equivalence via name-equivalence. Fortunately, all of this
works out pleasantly and we may calculate in the natural way, free of
concern. The reader interested in the details is referred to the
appendix \ref{appendix:rho_details}.

\subsection{Substitution}

We use $\Proc$ for the set of processes, $\QProc$ for the set of
names, and $\id{\{}\vec{y} / \vec{x} \id{\}}$ to denote partial maps,
$s : \QProc \rightarrow \QProc$. A map, $s$ lifts, uniquely, to a map
on process terms, $\widehat{s} : \Proc \rightarrow \Proc$ by the
following equations.

\begin{mathpar}
  (0) \psubstp{Q}{P} := 0 \\
  (R \juxtap S) \psubstp{Q}{P}
  :=    
  (R)\psubstp{Q}{P} \juxtap (S) \psubstp{Q}{P} \\
  (x?(y).R) \psubstp{Q}{P}    
  :=    
  (x)\substp{Q}{P} (z)\concat( (R \psubstn{z}{y}) \psubstp{Q}{P} ) \\
  (\lift{x}{R}) \psubstp{Q}{P}  
  :=
  \lift{(x)\substp{Q}{P}}{ R \psubstp{Q}{P} } \\
%   (\dropn{x})  \psubstp{Q}{P}       
%   := 
%   \left\{ 
%     \begin{array}{ccc} 
%       \dropn{\quotep{Q}} & & x \nameeq \quotep{P} \\
%       \dropn{x} & & otherwise \\
%     \end{array}
%   \right. 
  (\dropn{x})  \psubstp{Q}{P}       
  := 
  \left\{ 
    \begin{array}{ccc} 
      Q & & x \nameeq \quotep{P} \\
      \dropn{x} & & otherwise \\
    \end{array}
  \right.
\end{mathpar}
 

where

\begin{eqnarray}
  (x)\id{\{} \lpquote Q \rpquote / \lpquote P \rpquote \id{\}}            = 
  \left\{ 
    \begin{array}{ccc}
      \lpquote Q \rpquote & & x \nameeq \lpquote P \rpquote \\
      x & & otherwise \\
    \end{array}
  \right. \nonumber
\end{eqnarray}

and $z$ is chosen distinct from $\quotep{P}$, $\quotep{Q}$, the free
names in $Q$, and all the names in $R$. Our $\alpha$-equivalence will
be built in the standard way from this substitution.

\begin{remark}\label{rem:no_self_referential_names}
  One consequence of these definitions is that $\forall P. \quotep{P}
  \not\in \freenames{P}$.
\end{remark}

\subsection{ Dynamic quote: an example }

Anticipating something of what's to come, consider applying the
substitution, $\widehat{\id{\{}u / z \id{\}}}$, to the following pair
of processes, $\lift{w}{y!(z)}$ and $w[ \lpquote y!(z) \rpquote ]$.

\begin{eqnarray}
	\lift{w}{y!(z)}\widehat{\id{\{}u / z \id{\}}}
		& = &
		\lift{w}{y!(u)} \nonumber\\
	w[ \lpquote y!(z) \rpquote ] \widehat{ \id{\{}u / z \id{\}} }
		& = &
		w[ \lpquote y!(z) \rpquote ] \nonumber
\end{eqnarray}

Because the body of the process between quotes is impervious to
substitution, we get radically different answers. In fact, by
examining the first process in an input context,
e.g. $x?(z).\lift{w}{y!(z)}$, we see that the process under the lift
operator may be shaped by prefixed inputs binding a name inside it. In
this sense, the lift operator will be seen as a way to dynamically
construct processes before reifying them as names.

Finally equipped with these standard features we can present the
dynamics of the calculus.

\subsubsection{Operational semantics} 

Finally, we introduce the computational dynamics. What marks these
algebras as distinct from other more traditionally studied algebraic
structures, e.g. vector spaces or polynomial rings, is the manner in
which dynamics is captured. In traditional structures, dynamics is typically
expressed through morphisms between such structures, as in linear maps
between vector spaces or morphisms between rings. In algebras
associated with the semantics of computation, the dynamics is
expressed as part of the algebraic structure itself, through a
reduction reduction relation typically denoted by $\red$. Below, we
give a recursive presentation of this relation for the calculus used
in the encoding.

$\red \subseteq \pi \times \pi$
$\red : \pi \to \mathcal{P}(\pi)$

\begin{mathpar}
  \inferrule* [lab=Comm] { \textsf{match}( x_{src}, x_{trgt} ) } { x_{trgt}?(y)P \; | \; x_{src}!\langle {Q} \rangle \red P\{\quotep{Q}/y}\} }
  \and \\
  \inferrule* [lab=Par] {{P} \red {P}'} {{{P} | {Q}} \red {{P}' | {Q}}}
  \and
  \inferrule* [lab=Equiv]{{{P} \scong {P}'} \andalso {{P}' \red {Q}'} \andalso {{Q}' \scong {Q}}}{{P} \red {Q}}
\end{mathpar}

\begin{eqnarray*}
  match_{\equiv} (\quotep{P},\quotep{Q}) & := & P \equiv Q \\
  match_{\dagger}(\quotep{P},\quotep{Q}) & := & \forall R. P|Q \red^{*} R => R \red^{*} 0 \\
  match_{K}(\quotep{P},\quotep{Q}) & := & K \mbox{ for some context } K
\end{eqnarray*}

$u?(x)P | u!\langle Q \rangle \red P\{\quotep{Q}/x\}$

%We write $\wred$ for $\red^*$, and $P\red$ if $\exists Q $ such that $ P \red Q$.
We write $P\red$ if $\exists Q $ such that $ P \red Q$ and $P\not\red$, otherwise.

\section{Replication}

As mentioned before, it is known that replication (and hence
recursion) can be implemented in a higher-order process algebra
\cite{SangiorgiWalker}. As our first example of calculation with the
machinery thus far presented we give the construction explicitly in
the {\rhoc}.

\begin{eqnarray}
	D_{x} & := & \prefix{x}{y}{(\binpar{\outputp{x}{y}}{@{y}})} \nonumber\\
	\bangp_{x}{P} & := & \binpar{{x}!\langle{\binpar{D_{x}}{P}}\rangle}{D_{x}} \nonumber
\end{eqnarray}

\begin{eqnarray}
	\bangp_{x}{P} & & \nonumber\\
	=
	& {x}!\langle{(\prefix{x}{y}{(\outputp{x}{y} | @{y})) | P}}\rangle 
	      | \prefix{x}{y}{(\outputp{x}{y} | @{y})} & \nonumber\\
	\red
	& (\outputp{x}{y} | @{y})\substn{\quotep{(\prefix{x}{y}{(@{y} | \outputp{x}{y})) | P}}}{y} & \nonumber\\
	=
	& \outputp{x}{\quotep{(\prefix{x}{y}{(\outputp{x}{y} | @{y})) | P}}}
	  | {(\prefix{x}{y}{(\outputp{x}{y} | @{y})) | P}} & \nonumber\\
	\red
	& \ldots & \nonumber\\
	\red^*
	& P | P | \ldots & \nonumber
\end{eqnarray}

Of course, this encoding, as an implementation, runs away, unfolding
$\bangp{P}$ eagerly. A lazier and more implementable replication
operator, restricted to input-guarded processes, may be obtained as follows.

\begin{eqnarray}
\bangp{\prefix{u}{v}{P}} 
	:= 
	\binpar{\lift{x}{\prefix{u}{v}{(\binpar{D(x)}{P})}}}{D(x)} \nonumber
\end{eqnarray}

\begin{remark}
  Note that the lazier definition still does not deal with summation
  or mixed summation (i.e. sums over input and output). The reader is
  invited to construct definitions of replication that deal with these
  features. 

  Further, the definitions are parameterized in a name, $x$. Can you,
  gentle reader, make a definition that eliminates this parameter and
  guarantees no accidental interaction between the replication
  machinery and the process being replicated -- i.e. no accidental
  sharing of names used by the process to get its work done and the
  name(s) used by the replication to effect copying. This latter
  revision of the definition of replication is crucial to obtaining
  the expected identity $!!P \sim !P$.
\end{remark}

\begin{remark}\label{rem:paradoxical_combinator}
  The reader familiar with the lambda calculus will have noticed the
  similarity between $D$ and the paradoxical combinator.

  [Ed. note: the existence of this seems to suggest we have to be more
  restrictive on the set of processes and names we admit if we are to
  support no-cloning.]
\end{remark}

\subsubsection{Bisimulation}

The computational dynamics gives rise to another kind of equivalence,
the equivalence of computational behavior. As previously mentioned
this is typically captured \emph{via} some form of bisimulation.

% The notion we use in this paper is weak barbed bisimulation
% \cite{milner91polyadicpi}.

The notion we use in this paper is derived from weak barbed
bisimulation \cite{milner91polyadicpi}. 

\begin{definition}
An \emph{observation relation}, $\downarrow_{\mathcal N}$, over a set
of names, $\mathcal N$, is the smallest relation satisfying the rules
below.

\infrule[Out-barb]{y \in {\mathcal N}, \; x \nameeq y}
		  {\outputp{x}{v} \downarrow_{\mathcal N} x}
\infrule[Par-barb]{\mbox{$P\downarrow_{\mathcal N} x$ or $Q\downarrow_{\mathcal N} x$}}
		  {\binpar{P}{Q} \downarrow_{\mathcal N} x}

We write $P \Downarrow_{\mathcal N} x$ if there is $Q$ such that 
$P \wred Q$ and $Q \downarrow_{\mathcal N} x$.
\end{definition}

\begin{definition}
%\label{def.bbisim}
An  ${\mathcal N}$-\emph{barbed bisimulation} over a set of names, ${\mathcal N}$, is a symmetric binary relation 
${\mathcal S}_{\mathcal N}$ between agents such that $P\rel{S}_{\mathcal N}Q$ implies:
\begin{enumerate}
\item If $P \red P'$ then $Q \wred Q'$ and $P'\rel{S}_{\mathcal N} Q'$.
\item If $P\downarrow_{\mathcal N} x$, then $Q\Downarrow_{\mathcal N} x$.
\end{enumerate}
$P$ is ${\mathcal N}$-barbed bisimilar to $Q$, written
$P \wbbisim_{\mathcal N} Q$, if $P \rel{S}_{\mathcal N} Q$ for some ${\mathcal N}$-barbed bisimulation ${\mathcal S}_{\mathcal N}$.
\end{definition}

$\mathcal{R} \subseteq \pi \times \pi$

$P \mathcal{R} Q => \forall P'. P \red P' \Rightarrow \exists Q'. Q \red Q', P' \mathcal{R} Q'$

$P \vdash x \Rightarrow Q \vdash x$

\begin{mathpar}
  \inferrule*[lab=Out-barb]{x \nameeq y}{{y}!\langle{Q}\rangle \vdash x}
  \and
  \inferrule*[lab=Par-barb]{\mbox{$P\vdash x$ or $Q\vdash x$}}{\binpar{P}{Q} \vdash x}
\end{mathpar}

\subsubsection{Contexts}

One of the principle advantages of computational calculi like the
$\pi$-calculus is a well-defined notion of context,
contextual-equivalence and a correlation between
contextual-equivalence and notions of bisimulation. The notion of
context allows the decomposition of a process into (sub-)process and
its syntactic environment, its context. Thus, a context may be
thought of as a process with a ``hole'' (written $\Box$) in it. The
application of a context $M$ to a process $P$, written $M[P]$, is
tantamount to filling the hole in $M$ with $P$. In this paper we do
not need the full weight of this theory, but do make use of the notion
of context in the proof the main theorem. 

\begin{mathpar}
  \inferrule* [lab=summation] {} {{M_{M},M_{N}} \bc \Box \;|\; x.M_{A} \;|\; M_{M}+M_{N}}
  \and
  \inferrule* [lab=agent] {} {{M_{A}} \bc (\vec{x})M_{P} \;| \; \clift{P_0,\ldots,M_{P},\ldots,P_N}}
  \and \\
  \inferrule* [lab=process] {} {{M_{P}} \bc M_{N} \;| \;P|M_{P} }
\end{mathpar} 

\begin{mathpar}
  \inferrule* [lab=sychronization] {} {M_{N} \bc \Box \;|\; x?M_{F} \;|\; x!M_{C}}
  \and
  \inferrule* [lab=abstraction] {} {{M_{F}} \bc (x)M_{P} }
  \and
  \inferrule* [lab=concretion] {} {{M_{C}} \bc \langle M_{P} \rangle }
  \and \\
  \inferrule* [lab=process] {} {{M_{P}} \bc M_{N} \;| \;P|M_{P} }
\end{mathpar}

\begin{definition}[contextual application] Given a context $M$, and
  process $P$, we define the \emph{contextual application}, $M[P] :=
  M\{P/\Box\}$. That is, the contextual application of M to P is the
  substitution of $P$ for $\Box$ in $M$.
\end{definition}

$\meaningof{-} : L \to \mathcal{P}(\pi)$

\begin{mathpar}
  \inferrule* [lab=collection] {} {\meaningof{true} = \pi, \and \meaningof{~E} = \pi \setminus \meaningof{E}, \and \meaningof{E_{1} \& E_{2}} = \meaningof{E_{1}} \cap \meaningof{E_{2}}}
\end{mathpar}

\begin{mathpar}
  \inferrule* [lab=structure] {} {\meaningof{0} = \{ P \in \pi | P \equiv 0 \}, \and \\ \meaningof{E_1 | E_2} = \{ P \in \pi | P \equiv P_{1} | P_{2}, P_{1} \in \meaningof{E_{1}}, P_{2} \in \meaningof{E_2}\} }
\end{mathpar}

\begin{mathpar}
 \inferrule* [lab=behavior] {} {\meaningof{\langle a?b \rangle E} = \{ P \in \pi | P \equiv Q | u?(y)P', \\ \and \\\\ \and \\ \;\;\; u \in \meaningof{a}, \forall z.P'\{z/y\} \in \meaningof{E\{z/b\}}\}, \and \\ \meaningof{a!E} = \{ P \in \pi | P \equiv Q | x!\langle P' \rangle, x \in \meaningof{a} P' \in \meaningof{E}\} }
\end{mathpar}

\begin{mathpar}
 \inferrule* [lab=nominal] {} {\meaningof{\quotep{E}} = \{ \quotep{P} \in \quotep{\pi} | P \in \meaningof{E} \}, \and \meaningof{\quotep{P}} = \{ \quotep{Q} \in \quotep{\pi} | P \equiv Q \} \and \\ \meaningof{@\quotep{E}} = \{ P \in \pi | P \equiv @x, x \in \meaningof{E} \}}
\end{mathpar}

\begin{eqnarray*}
  \\
  \meaningof{-} : TS \to ST
\end{eqnarray*}

\begin{eqnarray*}
  \\
  L : TS \to ST
\end{eqnarray*}

\begin{eqnarray*}
  \\
  P \models E \iff P \in \meaningof{E}
\end{eqnarray*}

\begin{eqnarray*}
  P \approx_{L} Q \iff \forall E \in L. P \models E \iff Q \models E
\end{eqnarray*}

\begin{eqnarray*}
  P \approx_{K} Q
\end{eqnarray*}

\begin{eqnarray*}
  P \approx Q
\end{eqnarray*}

$\approx_{K} = \approx = \approx_{L}$

\subsubsection{Contextual duality}

Note that contexts extend the quotation operation to a family of
operations from processes to names. Given a context, $M$, we can
define a \emph{nominal context}, $\quotep{M}$ by $\quotep{M}[P] :=
\quotep{M[P]}$. To foreshadow what is to come we observe that these
operations enjoy a duality with processes very much like the duality
between vectors and maps from vectors to scalars.

Further, because the calculus is essentially higher-order, we have a
correspondence between contexts and processes. More specifically,
given a name $x$ and a context $M$ we can construct $M^{*}_{x}$ such
that 

\begin{mathpar}
  M^{*}_{x} | \lift{x}{P} \red M[P]
\end{mathpar}

namely,

\begin{mathpar}
  M^{*}_{x} := x?(u).M[\dropn{u}]
\end{mathpar}

The dependence of $M^{*}_{x}$ on a name makes it an abstraction, 

\begin{mathpar}
  M^{*} := (x)x?(u).M[\dropn{u}]
\end{mathpar}

\subsection{Additional notation}

It will sometimes be convenient to denote the process a name
quotes. We already have the notation $x = \quotep{P}$, but it will be
convenient to introduce an alternate notation, $\procn{x}$, when we
want to emphasize the connection to the use of the name. Note that, by
virtue of name equivalence, $\quotep{\procn{x}} \nameeq x$; so, the
notation is consistent with previous definitions.

Further, because names have structure it is possible to effect
substitutions on the basis of that structure. This means we need to
upgrade our notation for substitutions, which we accomplish by
adapting comprehension notation. Thus,

\begin{mathpar}
  P\{ y / x : x \in S \}
\end{mathpar}

is interpreted to mean the process derived from P by replacing (in a
capture-avoiding manner) each occurrence of $x$ in $S$ by $y$. For example,

\begin{mathpar}
  P\{ \quotep{\procn{x}|\procn{x}} / x : x \in \freenames{P} \}
\end{mathpar}

will replace each (occurrence) of a free name $x$ in $P$ by
$\quotep{\procn{x}|\procn{x}}$.

Also, we will avail ourselves of the notation $x^{L}$ and $x^{R}$ to
denote injections of a name into disjoint copies of the name
space. There are numerous ways to accomplish this. One example can be
found in \cite{MeredithR05}. This notation overloads to vectors of
names: $\vec{x}^{\pi} := (x_{i}^{\pi} \; : \; 0 \leq i < |\vec{x}| )$ where $\pi \in \{L,R\}$.

We also use $P^{\Box} := P|\Box$.

In \cite{MeredithR05} an interpretation of the new operator is
given. It turns out that there are several possible interpretations
all enjoying the requisite algebraic properties of the operator (see
\cite{milner91polyadicpi}). We will therefore make liberal use of
$(\nu\; \vec{x})P$.

% subsection the_syntax_and_semantics_of_the_notation_system (end)   

\input{qm2pi.qmops} 

\input{qm2pi.sterngerlach} 

\input{qm2pi.metric} 

% section concurrent_process_calculi (end)

%\input{qm2pi.proofsketch}

% section proof sketch (end)

%\input{qm2pi.slviaknots} 

% section spatial logic via knots (end)

\input{qm2pi.conclusion}

% section conclusion (end)

%\input{qm2pi.dtcodes} 

% section wiring algorithm (end)

\input{qm2pi.ack} 

% section acknowledgments (end)

\newpage


\bibliographystyle{plain}   
\bibliography{../../biblios/main.bib}

\input{qm2pi.rhodetails}

\end{document}

 

% section concurrent_process_calculi (end)

%\documentclass[12pt]{llncs}
%\documentclass{jktr}

\usepackage[pdftex]{hyperref}                   
\usepackage {listings}
\usepackage {mathpartir}
\usepackage{bcprules}
%\usepackage{listings}
                       
\usepackage{graphicx} 
%\usepackage[margins=2.5cm,nohead,nofoot]{geometry}
%\usepackage{geometry}
\usepackage{amsfonts}
\usepackage{amstext}
\usepackage{latexsym}
\usepackage{amssymb}
\usepackage{color}


%\include{myPreamble}
\include{qm2pi.local} 

%\ifpdf
%\usepackage[pdftex]{graphicx}
%\else
%\usepackage{graphicx}
%\fi

 % \ifpdf
%  \usepackage{pdfsync}
%  \if


%\title{Brief Article}
%\author{David F. Snyder}
%\author{L.G. Meredith}

%\address{Dept. of Math., Texas State University--San Marcos, San Marcos, TX 78666}
       
\pagestyle{empty}


\begin{document}

\lstset{language=[Objective]Caml,frame=shadowbox}

\input{qm2pi.front}

% section front matter (end)

\input{qm2pi.intro} 
 
% section introduction (end)

% \input{qm2pi.knotations} 

% section notation (end)

\input{qm2pi.process.calculi} 

% section concurrent_process_calculi_and_spatial_logics_ (end)
    
%\input{qm2pi.knots2pi} 

%\input{qm2pi.trefoil} 

%\input{qm2pi.mainthm} 

% subsection basic_interpretation (end)

%\input{qm2pi.rho.presentation} 
\subsection{The syntax and semantics of the notation system}\label{sub:the_syntax_and_semantics_of_the_notation_system} % (fold)

We now summarize a technical presentation of the calculus that
embodies our theory of dynamics. The typical presentation of such a
calculus follows the style of giving generators and relations on
them. The grammar, below, describing term constructors, freely
generates the set of processes, $\Proc$. This set is then quotiented
by a relation known as structural congruence and it is over this set
that the notion of dynamics is expressed. This presentation is
essentially that of \cite{MeredithR05} with the addition of
polyadicity and summation. For readability we have relegated some of
the technical subtleties to an appendix.

\subsubsection{Process grammar}\label{subsub:process_grammar}

\begin{mathpar}
  \inferrule* [lab=synchronization] {} {{M} \bc \pzero \;|\; x?F \;|\; x!C }
  \and
  \inferrule* [lab=abstraction] {} {{F} \bc (x)P}
  \and
  \inferrule* [lab=concretion] {} {{C} \bc \langle Q \rangle}
  \and
  \inferrule* [lab=process] {} {{P,Q} \bc M \;| \;P|Q \;|\; @{x}}
  \and
  \inferrule* [lab=name] {} {{x} \bc \quotep{P}}
\end{mathpar} 

Note that $\vec{x}$ (resp. $\vec{P}$) denotes a vector of names
(resp. processes) of length $|\vec{x}|$ (resp. $|\vec{P}|$). We adopt
the following useful abbreviations.

\begin{mathpar}
   x?(\vec{y}).P := x.(\vec{y})P \and  x\clift{\vec{P}} := x.\clift{\vec{P}}
   \and x!(y) := \lift{x}{\dropn{y}}
   \and \Pi_{i=0}^{n-1}P_i := P_0 | \ldots | P_{n-1}
\end{mathpar}

\subsubsection{Structural congruence}

\paragraph{Free and bound names and alpha-equivalence.} At the
core of structural equivalence is alpha-equivalence which identifies
process that are the same up to a change of variable. Formally, we
recognize the distinction between free and bound names. The free names
of a process, $\freenames{P}$, may be calculated recursively as
follows:

\begin{mathpar}
\freenames{\pzero} := \emptyset
  \and \\
  \freenames{x?(y).P} := \{ x \} \cup (\freenames{P} \setminus \{ y \})
  \and 
  \freenames{x!\langle P \rangle} := \{ x \} \cup \{ P \} 
  \and \\
  \freenames{P|Q} := \freenames{P} \cup \freenames{Q}
  \and \\
  \freenames{@{x}} := \{ x \}
\end{mathpar}

$\pi$
$\quotep{\pi}$

$\freenames{-} : \pi \to \mathcal{P}(\quotep{\pi})$

\begin{eqnarray*}
  \freenames{\pzero} & := & \emptyset \\
  \freenames{x?(y).P} & := & \{ x \} \cup (\freenames{P} \setminus \{ y \}) \\
  \freenames{x!\langle P \rangle} & := & \{ x \} \cup \{ P \} \\
  \freenames{P|Q} & := & \freenames{P} \cup \freenames{Q} \\
  \freenames{\dropn{x}} & := & \{ x \}
\end{eqnarray*}

The bound names of a process, $\boundnames{P}$, are those names occurring in $P$
that are not free. For example, in $x?(y).0$, the name $x$ is free, while $y$ is bound.

\begin{mathpar}
  \inferrule* [lab=monoidal-laws] {} { P|Q \equiv Q|P \and P|0 \equiv P \and P|(Q|R) \equiv (P|Q)|R }
\end{mathpar}

\begin{mathpar}
  \inferrule* [lab=alpha-equivalence] {} { (x)P \equiv (y)P\{y/x\} \and y \not\in \freenames{P} }
\end{mathpar}

\begin{definition}
Then two processes, $P,Q$, are alpha-equivalent if $P = Q\{\vec{y}/\vec{x}\}$ for
some $\vec{x} \in \boundnames{Q},\vec{y} \in \boundnames{P}$, where $Q\{\vec{y}/\vec{x}\}$
denotes the capture-avoiding substitution of $\vec{y}$ for $\vec{x}$ in $Q$.
\end{definition}

\begin{definition}
  The {\em structural congruence} \cite{SangiorgiWalker} , $\equiv$,
  between processes is the least congruence containing
  alpha-equivalence, satisfying the abelian monoid laws
  (associativity, commutativity and $\pzero$ as identity) for parallel
  composition $|$ and for summation $+$.
\end{definition}

\subsection{Name equivalence}

We take name equivalence, written $\nameeq$, to be the smallest
equivalence relation generated by the following rules.

\begin{mathpar}
\inferrule*[lab=Quote-drop]
{ }
{ \quotep{@{x}} \nameeq x }

\inferrule*[lab=Struct-equiv]
{ P \scong Q }
{ \quotep{P} \nameeq \quotep{Q} }
\end{mathpar}

The astute reader will have noticed that the mutual recursion of names
and processes imposes a mutual recursion on alpha-equivalence and
structural equivalence via name-equivalence. Fortunately, all of this
works out pleasantly and we may calculate in the natural way, free of
concern. The reader interested in the details is referred to the
appendix \ref{appendix:rho_details}.

\subsection{Substitution}

We use $\Proc$ for the set of processes, $\QProc$ for the set of
names, and $\id{\{}\vec{y} / \vec{x} \id{\}}$ to denote partial maps,
$s : \QProc \rightarrow \QProc$. A map, $s$ lifts, uniquely, to a map
on process terms, $\widehat{s} : \Proc \rightarrow \Proc$ by the
following equations.

\begin{mathpar}
  (0) \psubstp{Q}{P} := 0 \\
  (R \juxtap S) \psubstp{Q}{P}
  :=    
  (R)\psubstp{Q}{P} \juxtap (S) \psubstp{Q}{P} \\
  (x?(y).R) \psubstp{Q}{P}    
  :=    
  (x)\substp{Q}{P} (z)\concat( (R \psubstn{z}{y}) \psubstp{Q}{P} ) \\
  (\lift{x}{R}) \psubstp{Q}{P}  
  :=
  \lift{(x)\substp{Q}{P}}{ R \psubstp{Q}{P} } \\
%   (\dropn{x})  \psubstp{Q}{P}       
%   := 
%   \left\{ 
%     \begin{array}{ccc} 
%       \dropn{\quotep{Q}} & & x \nameeq \quotep{P} \\
%       \dropn{x} & & otherwise \\
%     \end{array}
%   \right. 
  (\dropn{x})  \psubstp{Q}{P}       
  := 
  \left\{ 
    \begin{array}{ccc} 
      Q & & x \nameeq \quotep{P} \\
      \dropn{x} & & otherwise \\
    \end{array}
  \right.
\end{mathpar}
 

where

\begin{eqnarray}
  (x)\id{\{} \lpquote Q \rpquote / \lpquote P \rpquote \id{\}}            = 
  \left\{ 
    \begin{array}{ccc}
      \lpquote Q \rpquote & & x \nameeq \lpquote P \rpquote \\
      x & & otherwise \\
    \end{array}
  \right. \nonumber
\end{eqnarray}

and $z$ is chosen distinct from $\quotep{P}$, $\quotep{Q}$, the free
names in $Q$, and all the names in $R$. Our $\alpha$-equivalence will
be built in the standard way from this substitution.

\begin{remark}\label{rem:no_self_referential_names}
  One consequence of these definitions is that $\forall P. \quotep{P}
  \not\in \freenames{P}$.
\end{remark}

\subsection{ Dynamic quote: an example }

Anticipating something of what's to come, consider applying the
substitution, $\widehat{\id{\{}u / z \id{\}}}$, to the following pair
of processes, $\lift{w}{y!(z)}$ and $w[ \lpquote y!(z) \rpquote ]$.

\begin{eqnarray}
	\lift{w}{y!(z)}\widehat{\id{\{}u / z \id{\}}}
		& = &
		\lift{w}{y!(u)} \nonumber\\
	w[ \lpquote y!(z) \rpquote ] \widehat{ \id{\{}u / z \id{\}} }
		& = &
		w[ \lpquote y!(z) \rpquote ] \nonumber
\end{eqnarray}

Because the body of the process between quotes is impervious to
substitution, we get radically different answers. In fact, by
examining the first process in an input context,
e.g. $x?(z).\lift{w}{y!(z)}$, we see that the process under the lift
operator may be shaped by prefixed inputs binding a name inside it. In
this sense, the lift operator will be seen as a way to dynamically
construct processes before reifying them as names.

Finally equipped with these standard features we can present the
dynamics of the calculus.

\subsubsection{Operational semantics} 

Finally, we introduce the computational dynamics. What marks these
algebras as distinct from other more traditionally studied algebraic
structures, e.g. vector spaces or polynomial rings, is the manner in
which dynamics is captured. In traditional structures, dynamics is typically
expressed through morphisms between such structures, as in linear maps
between vector spaces or morphisms between rings. In algebras
associated with the semantics of computation, the dynamics is
expressed as part of the algebraic structure itself, through a
reduction reduction relation typically denoted by $\red$. Below, we
give a recursive presentation of this relation for the calculus used
in the encoding.

$\red \subseteq \pi \times \pi$
$\red : \pi \to \mathcal{P}(\pi)$

\begin{mathpar}
  \inferrule* [lab=Comm] { \textsf{match}( x_{src}, x_{trgt} ) } { x_{trgt}?(y)P \; | \; x_{src}!\langle {Q} \rangle \red P\{\quotep{Q}/y}\} }
  \and \\
  \inferrule* [lab=Par] {{P} \red {P}'} {{{P} | {Q}} \red {{P}' | {Q}}}
  \and
  \inferrule* [lab=Equiv]{{{P} \scong {P}'} \andalso {{P}' \red {Q}'} \andalso {{Q}' \scong {Q}}}{{P} \red {Q}}
\end{mathpar}

\begin{eqnarray*}
  match_{\equiv} (\quotep{P},\quotep{Q}) & := & P \equiv Q \\
  match_{\dagger}(\quotep{P},\quotep{Q}) & := & \forall R. P|Q \red^{*} R => R \red^{*} 0 \\
  match_{K}(\quotep{P},\quotep{Q}) & := & K \mbox{ for some context } K
\end{eqnarray*}

$u?(x)P | u!\langle Q \rangle \red P\{\quotep{Q}/x\}$

%We write $\wred$ for $\red^*$, and $P\red$ if $\exists Q $ such that $ P \red Q$.
We write $P\red$ if $\exists Q $ such that $ P \red Q$ and $P\not\red$, otherwise.

\section{Replication}

As mentioned before, it is known that replication (and hence
recursion) can be implemented in a higher-order process algebra
\cite{SangiorgiWalker}. As our first example of calculation with the
machinery thus far presented we give the construction explicitly in
the {\rhoc}.

\begin{eqnarray}
	D_{x} & := & \prefix{x}{y}{(\binpar{\outputp{x}{y}}{@{y}})} \nonumber\\
	\bangp_{x}{P} & := & \binpar{{x}!\langle{\binpar{D_{x}}{P}}\rangle}{D_{x}} \nonumber
\end{eqnarray}

\begin{eqnarray}
	\bangp_{x}{P} & & \nonumber\\
	=
	& {x}!\langle{(\prefix{x}{y}{(\outputp{x}{y} | @{y})) | P}}\rangle 
	      | \prefix{x}{y}{(\outputp{x}{y} | @{y})} & \nonumber\\
	\red
	& (\outputp{x}{y} | @{y})\substn{\quotep{(\prefix{x}{y}{(@{y} | \outputp{x}{y})) | P}}}{y} & \nonumber\\
	=
	& \outputp{x}{\quotep{(\prefix{x}{y}{(\outputp{x}{y} | @{y})) | P}}}
	  | {(\prefix{x}{y}{(\outputp{x}{y} | @{y})) | P}} & \nonumber\\
	\red
	& \ldots & \nonumber\\
	\red^*
	& P | P | \ldots & \nonumber
\end{eqnarray}

Of course, this encoding, as an implementation, runs away, unfolding
$\bangp{P}$ eagerly. A lazier and more implementable replication
operator, restricted to input-guarded processes, may be obtained as follows.

\begin{eqnarray}
\bangp{\prefix{u}{v}{P}} 
	:= 
	\binpar{\lift{x}{\prefix{u}{v}{(\binpar{D(x)}{P})}}}{D(x)} \nonumber
\end{eqnarray}

\begin{remark}
  Note that the lazier definition still does not deal with summation
  or mixed summation (i.e. sums over input and output). The reader is
  invited to construct definitions of replication that deal with these
  features. 

  Further, the definitions are parameterized in a name, $x$. Can you,
  gentle reader, make a definition that eliminates this parameter and
  guarantees no accidental interaction between the replication
  machinery and the process being replicated -- i.e. no accidental
  sharing of names used by the process to get its work done and the
  name(s) used by the replication to effect copying. This latter
  revision of the definition of replication is crucial to obtaining
  the expected identity $!!P \sim !P$.
\end{remark}

\begin{remark}\label{rem:paradoxical_combinator}
  The reader familiar with the lambda calculus will have noticed the
  similarity between $D$ and the paradoxical combinator.

  [Ed. note: the existence of this seems to suggest we have to be more
  restrictive on the set of processes and names we admit if we are to
  support no-cloning.]
\end{remark}

\subsubsection{Bisimulation}

The computational dynamics gives rise to another kind of equivalence,
the equivalence of computational behavior. As previously mentioned
this is typically captured \emph{via} some form of bisimulation.

% The notion we use in this paper is weak barbed bisimulation
% \cite{milner91polyadicpi}.

The notion we use in this paper is derived from weak barbed
bisimulation \cite{milner91polyadicpi}. 

\begin{definition}
An \emph{observation relation}, $\downarrow_{\mathcal N}$, over a set
of names, $\mathcal N$, is the smallest relation satisfying the rules
below.

\infrule[Out-barb]{y \in {\mathcal N}, \; x \nameeq y}
		  {\outputp{x}{v} \downarrow_{\mathcal N} x}
\infrule[Par-barb]{\mbox{$P\downarrow_{\mathcal N} x$ or $Q\downarrow_{\mathcal N} x$}}
		  {\binpar{P}{Q} \downarrow_{\mathcal N} x}

We write $P \Downarrow_{\mathcal N} x$ if there is $Q$ such that 
$P \wred Q$ and $Q \downarrow_{\mathcal N} x$.
\end{definition}

\begin{definition}
%\label{def.bbisim}
An  ${\mathcal N}$-\emph{barbed bisimulation} over a set of names, ${\mathcal N}$, is a symmetric binary relation 
${\mathcal S}_{\mathcal N}$ between agents such that $P\rel{S}_{\mathcal N}Q$ implies:
\begin{enumerate}
\item If $P \red P'$ then $Q \wred Q'$ and $P'\rel{S}_{\mathcal N} Q'$.
\item If $P\downarrow_{\mathcal N} x$, then $Q\Downarrow_{\mathcal N} x$.
\end{enumerate}
$P$ is ${\mathcal N}$-barbed bisimilar to $Q$, written
$P \wbbisim_{\mathcal N} Q$, if $P \rel{S}_{\mathcal N} Q$ for some ${\mathcal N}$-barbed bisimulation ${\mathcal S}_{\mathcal N}$.
\end{definition}

$\mathcal{R} \subseteq \pi \times \pi$

$P \mathcal{R} Q => \forall P'. P \red P' \Rightarrow \exists Q'. Q \red Q', P' \mathcal{R} Q'$

$P \vdash x \Rightarrow Q \vdash x$

\begin{mathpar}
  \inferrule*[lab=Out-barb]{x \nameeq y}{{y}!\langle{Q}\rangle \vdash x}
  \and
  \inferrule*[lab=Par-barb]{\mbox{$P\vdash x$ or $Q\vdash x$}}{\binpar{P}{Q} \vdash x}
\end{mathpar}

\subsubsection{Contexts}

One of the principle advantages of computational calculi like the
$\pi$-calculus is a well-defined notion of context,
contextual-equivalence and a correlation between
contextual-equivalence and notions of bisimulation. The notion of
context allows the decomposition of a process into (sub-)process and
its syntactic environment, its context. Thus, a context may be
thought of as a process with a ``hole'' (written $\Box$) in it. The
application of a context $M$ to a process $P$, written $M[P]$, is
tantamount to filling the hole in $M$ with $P$. In this paper we do
not need the full weight of this theory, but do make use of the notion
of context in the proof the main theorem. 

\begin{mathpar}
  \inferrule* [lab=summation] {} {{M_{M},M_{N}} \bc \Box \;|\; x.M_{A} \;|\; M_{M}+M_{N}}
  \and
  \inferrule* [lab=agent] {} {{M_{A}} \bc (\vec{x})M_{P} \;| \; \clift{P_0,\ldots,M_{P},\ldots,P_N}}
  \and \\
  \inferrule* [lab=process] {} {{M_{P}} \bc M_{N} \;| \;P|M_{P} }
\end{mathpar} 

\begin{mathpar}
  \inferrule* [lab=sychronization] {} {M_{N} \bc \Box \;|\; x?M_{F} \;|\; x!M_{C}}
  \and
  \inferrule* [lab=abstraction] {} {{M_{F}} \bc (x)M_{P} }
  \and
  \inferrule* [lab=concretion] {} {{M_{C}} \bc \langle M_{P} \rangle }
  \and \\
  \inferrule* [lab=process] {} {{M_{P}} \bc M_{N} \;| \;P|M_{P} }
\end{mathpar}

\begin{definition}[contextual application] Given a context $M$, and
  process $P$, we define the \emph{contextual application}, $M[P] :=
  M\{P/\Box\}$. That is, the contextual application of M to P is the
  substitution of $P$ for $\Box$ in $M$.
\end{definition}

$\meaningof{-} : L \to \mathcal{P}(\pi)$

\begin{mathpar}
  \inferrule* [lab=collection] {} {\meaningof{true} = \pi, \and \meaningof{~E} = \pi \setminus \meaningof{E}, \and \meaningof{E_{1} \& E_{2}} = \meaningof{E_{1}} \cap \meaningof{E_{2}}}
\end{mathpar}

\begin{mathpar}
  \inferrule* [lab=structure] {} {\meaningof{0} = \{ P \in \pi | P \equiv 0 \}, \and \\ \meaningof{E_1 | E_2} = \{ P \in \pi | P \equiv P_{1} | P_{2}, P_{1} \in \meaningof{E_{1}}, P_{2} \in \meaningof{E_2}\} }
\end{mathpar}

\begin{mathpar}
 \inferrule* [lab=behavior] {} {\meaningof{\langle a?b \rangle E} = \{ P \in \pi | P \equiv Q | u?(y)P', \\ \and \\\\ \and \\ \;\;\; u \in \meaningof{a}, \forall z.P'\{z/y\} \in \meaningof{E\{z/b\}}\}, \and \\ \meaningof{a!E} = \{ P \in \pi | P \equiv Q | x!\langle P' \rangle, x \in \meaningof{a} P' \in \meaningof{E}\} }
\end{mathpar}

\begin{mathpar}
 \inferrule* [lab=nominal] {} {\meaningof{\quotep{E}} = \{ \quotep{P} \in \quotep{\pi} | P \in \meaningof{E} \}, \and \meaningof{\quotep{P}} = \{ \quotep{Q} \in \quotep{\pi} | P \equiv Q \} \and \\ \meaningof{@\quotep{E}} = \{ P \in \pi | P \equiv @x, x \in \meaningof{E} \}}
\end{mathpar}

\begin{eqnarray*}
  \\
  \meaningof{-} : TS \to ST
\end{eqnarray*}

\begin{eqnarray*}
  \\
  L : TS \to ST
\end{eqnarray*}

\begin{eqnarray*}
  \\
  P \models E \iff P \in \meaningof{E}
\end{eqnarray*}

\begin{eqnarray*}
  P \approx_{L} Q \iff \forall E \in L. P \models E \iff Q \models E
\end{eqnarray*}

\begin{eqnarray*}
  P \approx_{K} Q
\end{eqnarray*}

\begin{eqnarray*}
  P \approx Q
\end{eqnarray*}

$\approx_{K} = \approx = \approx_{L}$

\subsubsection{Contextual duality}

Note that contexts extend the quotation operation to a family of
operations from processes to names. Given a context, $M$, we can
define a \emph{nominal context}, $\quotep{M}$ by $\quotep{M}[P] :=
\quotep{M[P]}$. To foreshadow what is to come we observe that these
operations enjoy a duality with processes very much like the duality
between vectors and maps from vectors to scalars.

Further, because the calculus is essentially higher-order, we have a
correspondence between contexts and processes. More specifically,
given a name $x$ and a context $M$ we can construct $M^{*}_{x}$ such
that 

\begin{mathpar}
  M^{*}_{x} | \lift{x}{P} \red M[P]
\end{mathpar}

namely,

\begin{mathpar}
  M^{*}_{x} := x?(u).M[\dropn{u}]
\end{mathpar}

The dependence of $M^{*}_{x}$ on a name makes it an abstraction, 

\begin{mathpar}
  M^{*} := (x)x?(u).M[\dropn{u}]
\end{mathpar}

\subsection{Additional notation}

It will sometimes be convenient to denote the process a name
quotes. We already have the notation $x = \quotep{P}$, but it will be
convenient to introduce an alternate notation, $\procn{x}$, when we
want to emphasize the connection to the use of the name. Note that, by
virtue of name equivalence, $\quotep{\procn{x}} \nameeq x$; so, the
notation is consistent with previous definitions.

Further, because names have structure it is possible to effect
substitutions on the basis of that structure. This means we need to
upgrade our notation for substitutions, which we accomplish by
adapting comprehension notation. Thus,

\begin{mathpar}
  P\{ y / x : x \in S \}
\end{mathpar}

is interpreted to mean the process derived from P by replacing (in a
capture-avoiding manner) each occurrence of $x$ in $S$ by $y$. For example,

\begin{mathpar}
  P\{ \quotep{\procn{x}|\procn{x}} / x : x \in \freenames{P} \}
\end{mathpar}

will replace each (occurrence) of a free name $x$ in $P$ by
$\quotep{\procn{x}|\procn{x}}$.

Also, we will avail ourselves of the notation $x^{L}$ and $x^{R}$ to
denote injections of a name into disjoint copies of the name
space. There are numerous ways to accomplish this. One example can be
found in \cite{MeredithR05}. This notation overloads to vectors of
names: $\vec{x}^{\pi} := (x_{i}^{\pi} \; : \; 0 \leq i < |\vec{x}| )$ where $\pi \in \{L,R\}$.

We also use $P^{\Box} := P|\Box$.

In \cite{MeredithR05} an interpretation of the new operator is
given. It turns out that there are several possible interpretations
all enjoying the requisite algebraic properties of the operator (see
\cite{milner91polyadicpi}). We will therefore make liberal use of
$(\nu\; \vec{x})P$.

% subsection the_syntax_and_semantics_of_the_notation_system (end)   

\input{qm2pi.qmops} 

\input{qm2pi.sterngerlach} 

\input{qm2pi.metric} 

% section concurrent_process_calculi (end)

%\input{qm2pi.proofsketch}

% section proof sketch (end)

%\input{qm2pi.slviaknots} 

% section spatial logic via knots (end)

\input{qm2pi.conclusion}

% section conclusion (end)

%\input{qm2pi.dtcodes} 

% section wiring algorithm (end)

\input{qm2pi.ack} 

% section acknowledgments (end)

\newpage


\bibliographystyle{plain}   
\bibliography{../../biblios/main.bib}

\input{qm2pi.rhodetails}

\end{document}



% section proof sketch (end)

%\section{Unlikely characters: spatial logic for
  knots}\label{sub:characteristic_formulae} % (fold)

Associated to the mobile process calculi are a family of logics known
as the Hennessy-Milner logics. These logics typically enjoy a
semantics interpreting formulae as sets of processes that when
factored through the encoding outlined above allows an identification
of classes of knots with logical formulae. In the context of this
encoding the sub-family known as the spatial logics \cite{CairesC03}
\cite{CairesC04} \cite{Caires04} are of particular interest providing
several important features for expressing and reasoning about
properties (i.e. classes) of knots. We hint here at how this may be done.

%\begin{description}
%\item [structural connectives] 
\subsubsection{Structural connectives} The spatial logics enjoy
structural connectives corresponding, at the logical level, to the
parallel composition ($P | Q$) and new name ($(\nu \; x)P$)
connectives for processes. As illustrated in the examples below, these
connectives are extremely expressive given the shape of our encoding.
%\item [decideable satisfaction]

\subsubsection{Decideable satisfaction}
In \cite{Caires04} the satisfaction relation is shown to be decideable
for a rich class of processes. It further turns out that the image of
the our encoding is a proper subset of that class. This result
provides the basis for an algorithm by which to search for knots
enjoying a given property.
%\item [characteristic formulae]

\subsubsection{Characteristic formulae}
In the same paper \cite{Caires04} , Caires presents a means of calculating
characteristic formulae, selecting equivalence classes of processes
up to a pre--specified depth limit on the support set of names. Composed with our
encoding, this characteristic formula can be used to select
characteristic formulae for knots.
%\end{description}

\subsubsection{Spatial logic formulae}

The grammar below (segmented for comprehension) summarizes the syntax
of spatial logic formulae. We employ illustrative examples in the
sequel to provide an intuitive understanding of their meaning
referring the reader to \cite{Caires04} for a more detailed explication
of the semantics.

\begin{mathpar}
  \inferrule* [lab=boolean] {} {{A,B} \bc T \;|\; \neg A \;|\; A \wedge B \;|\; \eta = \eta'}
  \and
  \inferrule* [lab=spatial] {} {|\; \pzero \;|\; A | B \;|\; x \text{\textregistered} A \;|\; \forall x . A \;|\;  H x . A}
  \and
  \inferrule* [lab=behavioral] {} {|\; \alpha . A}
  \and 
  \inferrule* [lab=recursion] {} {|\; X(\vec{u}) \;|\; \mu X(\vec{u}) . A}
  \and
  \inferrule* [lab=action] {} {\alpha \bc \langle x?(\vec{y}) \rangle \;|\; \langle x!(\vec{y}) \rangle \;|\; \langle \tau \rangle}
  \and 
  \inferrule* [lab=name] {} {\eta \bc x \;|\; \tau}
\end{mathpar} 

% subsection characteristic_formulae (end)   	 

\subsection{Example formulae}\label{sub:example_formulae_} % (fold)

\subsubsection{Crossing as formula.}
% 
% \begin{align*}
%   \frac{d}{dx} \sin x &= \cos x 
%   & \frac{d}{dx} e^x &= e^x \\
%   \frac{d}{dx} \cos x &= - \sin x 
%   & \frac{d}{dx} \log x &= \frac{1}{x} \\
% \end{align*} 

\begin{align*}
 \mu C(x_{0},x_{1},y_{0},y_{1},u).&(\langle x_{0}?(z) \rangle(\langle u! \rangle\langle y_{1}!z \rangle C(x_{0},x_{1},y_{0},y_{1},u)) & \\
  & \wedge \langle y_{1}?(z) \rangle (\langle u! \rangle \langle x_{0}!z \rangle C(x_{0},x_{1},y_{0},y_{1},u)) & \\
  & \wedge \langle x_{1}?(z) \rangle (\langle u? \rangle \langle y_{0}!z \rangle C(x_{0},x_{1},y_{0},y_{1},u)) & \\
  & \wedge \langle y_{0}?(z) \rangle (\langle u? \rangle \langle x_{1}!z \rangle C(x_{0},x_{1},y_{0},y_{1},u))) &
\end{align*}

The lexicographical similarity between the shape of this formulae and
the shape of definition of the process representing a crossing reveals
the intuitive meaning of this formulae. It describes the capabilities
of a process that has the right to represent a crossing. For example
it picks out processes that may perform an input on the port $x_0$ in
its initial menu of capabilities. What differentiates the formula
from the process, however, is that the crossing process is the
smallest candidate to satisfy the formula. Infinitely many other
processes -- with internal behavior hidden behind this interface, so
to speak -- also satisfy this formula. Even this simple formula,
then, can be seen to open a new view onto knots, providing a
computational interpretation of \emph{virtual} knots.

Note that this formula is derived by hand. A similar formula can be
derived by employing Caires' calculation of characteristic formula
\cite{Caires04} to the process representing a crossing. In light of
this discussion, we let
$\meaningof{C}_{\phi}(x0,x1,y0,y1,u)$ denote a formula specifying the
dynamics we wish to capture of a crossing. To guarantee we preserve
the shape of the interface and minimal semantics we demand that
$\meaningof{C}_{\phi}(x0,x1,y0,y1,u) \Rightarrow
\textbf{C}(x0,x1,y0,y1,u)$ where $\textbf{C}(x0,x1,y0,y1,u)$ denotes
the formula above.
                            
\subsubsection{Crossing number constraints.}
The moral content of the context lemma (Lemma \ref{context}) is that the notion of
``locality'' in the Reidemeister moves is effectively captured by the
parallel composition operator of the process calculus. This intuition
extends through the logic. Given a formula,
$\meaningof{C}_{\phi}(x0,x1,y0,y1,u)$, we can use the structural
connectives to specify constraints on crossing numbers, such as at
least $n$ crossings, or exactly $n$ crossings.
\begin{mathpar}
  \inferrule* [lab=at-least-n] {} { K^{\geq n}_{\phi}(\vec{xs},\vec{ys}) := \Pi_{i=0}^{n-1} Hu . \meaningof{C}_{\phi}(xs_i,ys_i,u) | T }
  \and 
  \inferrule* [lab=exactly-n] {} { K^{= n}_{\phi}(\vec{xs},\vec{ys}) := \Pi_{i=0}^{n-1} Hu . \meaningof{C}_{\phi}(xs_i,ys_i,u) | \neg (\forall x_0,y_0,x_1,y_1,u . \meaningof{C}_{\phi}(x_0,y_0,x_1,y_1,u) | T) }
\end{mathpar}

To round out this section, recall that the encoding of an $n$-crossing
knot decomposes into a parallel composition of $n$ \emph{copies} of a
crossing process together with a wiring harness. To specify different
knot classes with the same crossing number amounts to specifying
logical constraints on the wiring harness. In the interest of space,
we defer examples to a forthcoming paper. Suffice it to say that both
the conditions ``alternating knot'' and ``contains the tangle
corresponding to 5/3'' are expressible. For example, it is possible to
calculate the characteristic formula of a process corresponding to the
tangle 5/3 and conjoin it into the classifying formula via the
composition connective of the logic.

Finally, we wish to observe that it is entirely within reason to
contemplate a more domain-specific version of spatial logic tailored
to the shape of processes in the image of the encoding. Such a
domain-specific logic would have a better claim to the title formal
language of knot properties.

% subsection example_formulae_ (end)

% section knots_as_processes (end) 

% section spatial logic via knots (end)

\section{Conclusions and future work}

\paragraph{Testing physical space}
You, gentle reader, may wonder why of all the theorems to be proved
given this set up we pick the one above. In some sense it's hardly
central to quantum mechanics. We see it as central in the sense that
it firmly establishes a notion of physical space arising from a notion
of the equivalence of behavior. Relating bisimulation to a metric is a
big step forward, but one is faced with interpreting the relationship
of that metric space to something more physical. Quantum mechanical
notions of ``physical'' space are still far from intuitive, but by
relating this idea of distance as testing to calculations that predict
physical circumstances we are making a not insignificant step forward
toward an understanding of the physical space we inhabit as
essentially dynamic.

\paragraph{Effectivity and simulation}
One of the observations we have yet to make is that the entire program
spelled out here is effective. We have built various interpreters for
the reflective calculus at work in this interpretation. In principle,
then, we can simulate quantum mechanics on a computer. The place where
the simulation may lose fidelity is the infinitely branching summation
for the annihilator.

In this connection i also want to point out that the evaluation style
calculation of the inner product puts the non-determinism of the
summation right at the heart of measurement. This suggests that
Milner's original reduction-based formulation of the dynamics of his
calculi in terms of sums was not just notationally suggestive of a
notion of measure-and-continue but captured some significant part of
the physics.

\paragraph{Quantum continuations}
In light of this last observation i want to point out that the
predominant account of quantum mechanics is missing a key aspect of a
truly compositional story of the physical situation. In a real lab,
when a measurement is made the observation can be made to feed into
another device that then makes another measurement conditioned on the
results of the first. This means that after the superposition was
collapsed the entire experimental set up remained in
superposition. While QM offers a means of writing this down it doesn't
quite line up well with the well-trodden formulation of computation
and continuation that we see so succinctly expressed in Milner's
calculi. This suggests that there might be advantages to this account
of dynamics waiting to be explored.

\paragraph{Quantum logic}
In this connection, we also note that by virtue of having the
Hennessy-Milner construction, we can pull the construction through the
interpretation of QM. This gives us a natural candidate for a quantum
logic that enjoys an extremely tight connection with it's domain of
interpretation, making the construction much less ad hoc (rather it is
the image of functor!).

\paragraph{Quantum probabiity}
i have questions about the basis of the interpretation of inner
product as probability amplitude. In particular, using which
axiomatization of probability theory does the notion of probability
amplitude earn the right to be so dubbed? In other words, where is the
proof that the operation for calculating a probability amplitude (and
then squaring) satisfies the axioms of what it means to calculate a
probability? Even if such a proof exists (i have yet to find it in the
literature), i wonder if it might not be possible to turn things on
their heads. Can we view the calculation of the probability amplitude
as an axiomatization of probability? If so, then the definition we
give for calculating probability amplitude may provide the basis for
an \emph{effective} theory of probability.

\paragraph{Quantum vs ``biological'' information}
Finally, i want to conclude with a more philosophical observation. At
a recent workshop in which QM was a predominant topic i noticed
something about quantum information. The speaker was giving a riveting
discussion of axiomatic QM and showing how properties of ``no
cloning'' and ``no deleting'' emerged as consequences of the
axiomatization. Theorems of this form are necessary to give us a sense
of confidence that our axioms characterize the physical theory. What
struck me, though, was that if quantum information is neither erasable
nor replicable it is markedly different from \emph{life}. Two of the
things we know about life is that

\begin{itemize}
  \item it ends;
  \item to gain some measure of persistence, to transcend it's
    finitude it is imminently copyable.
\end{itemize}

Both of these qualities are summarized succinctly in the aphorism: all
flesh is grass. For me these two kinds of ``information'' -- call them
quantum and biological -- are end points on a spectrum of strategies
for persistence. At one end, we have those curious entities that enjoy
uniqueness and permanence; at the other, we have those who in the face
of a certain end and an uncertain present make a go of passing
something on. To me one of the more remarkable aspects of the latter
strategy is that in the presence of noise (and certain features of
copying) we get a kind of dynamism, a chance for improvement against a
given persistent condition.

% subsection other_calculi_other_bisimulations_and_geometry_as_behavior (end)




% section conclusion (end)

%\documentclass[12pt]{llncs}
%\documentclass{jktr}

\usepackage[pdftex]{hyperref}                   
\usepackage {listings}
\usepackage {mathpartir}
\usepackage{bcprules}
%\usepackage{listings}
                       
\usepackage{graphicx} 
%\usepackage[margins=2.5cm,nohead,nofoot]{geometry}
%\usepackage{geometry}
\usepackage{amsfonts}
\usepackage{amstext}
\usepackage{latexsym}
\usepackage{amssymb}
\usepackage{color}


%\include{myPreamble}
\include{qm2pi.local} 

%\ifpdf
%\usepackage[pdftex]{graphicx}
%\else
%\usepackage{graphicx}
%\fi

 % \ifpdf
%  \usepackage{pdfsync}
%  \if


%\title{Brief Article}
%\author{David F. Snyder}
%\author{L.G. Meredith}

%\address{Dept. of Math., Texas State University--San Marcos, San Marcos, TX 78666}
       
\pagestyle{empty}


\begin{document}

\lstset{language=[Objective]Caml,frame=shadowbox}

\input{qm2pi.front}

% section front matter (end)

\input{qm2pi.intro} 
 
% section introduction (end)

% \input{qm2pi.knotations} 

% section notation (end)

\input{qm2pi.process.calculi} 

% section concurrent_process_calculi_and_spatial_logics_ (end)
    
%\input{qm2pi.knots2pi} 

%\input{qm2pi.trefoil} 

%\input{qm2pi.mainthm} 

% subsection basic_interpretation (end)

%\input{qm2pi.rho.presentation} 
\subsection{The syntax and semantics of the notation system}\label{sub:the_syntax_and_semantics_of_the_notation_system} % (fold)

We now summarize a technical presentation of the calculus that
embodies our theory of dynamics. The typical presentation of such a
calculus follows the style of giving generators and relations on
them. The grammar, below, describing term constructors, freely
generates the set of processes, $\Proc$. This set is then quotiented
by a relation known as structural congruence and it is over this set
that the notion of dynamics is expressed. This presentation is
essentially that of \cite{MeredithR05} with the addition of
polyadicity and summation. For readability we have relegated some of
the technical subtleties to an appendix.

\subsubsection{Process grammar}\label{subsub:process_grammar}

\begin{mathpar}
  \inferrule* [lab=synchronization] {} {{M} \bc \pzero \;|\; x?F \;|\; x!C }
  \and
  \inferrule* [lab=abstraction] {} {{F} \bc (x)P}
  \and
  \inferrule* [lab=concretion] {} {{C} \bc \langle Q \rangle}
  \and
  \inferrule* [lab=process] {} {{P,Q} \bc M \;| \;P|Q \;|\; @{x}}
  \and
  \inferrule* [lab=name] {} {{x} \bc \quotep{P}}
\end{mathpar} 

Note that $\vec{x}$ (resp. $\vec{P}$) denotes a vector of names
(resp. processes) of length $|\vec{x}|$ (resp. $|\vec{P}|$). We adopt
the following useful abbreviations.

\begin{mathpar}
   x?(\vec{y}).P := x.(\vec{y})P \and  x\clift{\vec{P}} := x.\clift{\vec{P}}
   \and x!(y) := \lift{x}{\dropn{y}}
   \and \Pi_{i=0}^{n-1}P_i := P_0 | \ldots | P_{n-1}
\end{mathpar}

\subsubsection{Structural congruence}

\paragraph{Free and bound names and alpha-equivalence.} At the
core of structural equivalence is alpha-equivalence which identifies
process that are the same up to a change of variable. Formally, we
recognize the distinction between free and bound names. The free names
of a process, $\freenames{P}$, may be calculated recursively as
follows:

\begin{mathpar}
\freenames{\pzero} := \emptyset
  \and \\
  \freenames{x?(y).P} := \{ x \} \cup (\freenames{P} \setminus \{ y \})
  \and 
  \freenames{x!\langle P \rangle} := \{ x \} \cup \{ P \} 
  \and \\
  \freenames{P|Q} := \freenames{P} \cup \freenames{Q}
  \and \\
  \freenames{@{x}} := \{ x \}
\end{mathpar}

$\pi$
$\quotep{\pi}$

$\freenames{-} : \pi \to \mathcal{P}(\quotep{\pi})$

\begin{eqnarray*}
  \freenames{\pzero} & := & \emptyset \\
  \freenames{x?(y).P} & := & \{ x \} \cup (\freenames{P} \setminus \{ y \}) \\
  \freenames{x!\langle P \rangle} & := & \{ x \} \cup \{ P \} \\
  \freenames{P|Q} & := & \freenames{P} \cup \freenames{Q} \\
  \freenames{\dropn{x}} & := & \{ x \}
\end{eqnarray*}

The bound names of a process, $\boundnames{P}$, are those names occurring in $P$
that are not free. For example, in $x?(y).0$, the name $x$ is free, while $y$ is bound.

\begin{mathpar}
  \inferrule* [lab=monoidal-laws] {} { P|Q \equiv Q|P \and P|0 \equiv P \and P|(Q|R) \equiv (P|Q)|R }
\end{mathpar}

\begin{mathpar}
  \inferrule* [lab=alpha-equivalence] {} { (x)P \equiv (y)P\{y/x\} \and y \not\in \freenames{P} }
\end{mathpar}

\begin{definition}
Then two processes, $P,Q$, are alpha-equivalent if $P = Q\{\vec{y}/\vec{x}\}$ for
some $\vec{x} \in \boundnames{Q},\vec{y} \in \boundnames{P}$, where $Q\{\vec{y}/\vec{x}\}$
denotes the capture-avoiding substitution of $\vec{y}$ for $\vec{x}$ in $Q$.
\end{definition}

\begin{definition}
  The {\em structural congruence} \cite{SangiorgiWalker} , $\equiv$,
  between processes is the least congruence containing
  alpha-equivalence, satisfying the abelian monoid laws
  (associativity, commutativity and $\pzero$ as identity) for parallel
  composition $|$ and for summation $+$.
\end{definition}

\subsection{Name equivalence}

We take name equivalence, written $\nameeq$, to be the smallest
equivalence relation generated by the following rules.

\begin{mathpar}
\inferrule*[lab=Quote-drop]
{ }
{ \quotep{@{x}} \nameeq x }

\inferrule*[lab=Struct-equiv]
{ P \scong Q }
{ \quotep{P} \nameeq \quotep{Q} }
\end{mathpar}

The astute reader will have noticed that the mutual recursion of names
and processes imposes a mutual recursion on alpha-equivalence and
structural equivalence via name-equivalence. Fortunately, all of this
works out pleasantly and we may calculate in the natural way, free of
concern. The reader interested in the details is referred to the
appendix \ref{appendix:rho_details}.

\subsection{Substitution}

We use $\Proc$ for the set of processes, $\QProc$ for the set of
names, and $\id{\{}\vec{y} / \vec{x} \id{\}}$ to denote partial maps,
$s : \QProc \rightarrow \QProc$. A map, $s$ lifts, uniquely, to a map
on process terms, $\widehat{s} : \Proc \rightarrow \Proc$ by the
following equations.

\begin{mathpar}
  (0) \psubstp{Q}{P} := 0 \\
  (R \juxtap S) \psubstp{Q}{P}
  :=    
  (R)\psubstp{Q}{P} \juxtap (S) \psubstp{Q}{P} \\
  (x?(y).R) \psubstp{Q}{P}    
  :=    
  (x)\substp{Q}{P} (z)\concat( (R \psubstn{z}{y}) \psubstp{Q}{P} ) \\
  (\lift{x}{R}) \psubstp{Q}{P}  
  :=
  \lift{(x)\substp{Q}{P}}{ R \psubstp{Q}{P} } \\
%   (\dropn{x})  \psubstp{Q}{P}       
%   := 
%   \left\{ 
%     \begin{array}{ccc} 
%       \dropn{\quotep{Q}} & & x \nameeq \quotep{P} \\
%       \dropn{x} & & otherwise \\
%     \end{array}
%   \right. 
  (\dropn{x})  \psubstp{Q}{P}       
  := 
  \left\{ 
    \begin{array}{ccc} 
      Q & & x \nameeq \quotep{P} \\
      \dropn{x} & & otherwise \\
    \end{array}
  \right.
\end{mathpar}
 

where

\begin{eqnarray}
  (x)\id{\{} \lpquote Q \rpquote / \lpquote P \rpquote \id{\}}            = 
  \left\{ 
    \begin{array}{ccc}
      \lpquote Q \rpquote & & x \nameeq \lpquote P \rpquote \\
      x & & otherwise \\
    \end{array}
  \right. \nonumber
\end{eqnarray}

and $z$ is chosen distinct from $\quotep{P}$, $\quotep{Q}$, the free
names in $Q$, and all the names in $R$. Our $\alpha$-equivalence will
be built in the standard way from this substitution.

\begin{remark}\label{rem:no_self_referential_names}
  One consequence of these definitions is that $\forall P. \quotep{P}
  \not\in \freenames{P}$.
\end{remark}

\subsection{ Dynamic quote: an example }

Anticipating something of what's to come, consider applying the
substitution, $\widehat{\id{\{}u / z \id{\}}}$, to the following pair
of processes, $\lift{w}{y!(z)}$ and $w[ \lpquote y!(z) \rpquote ]$.

\begin{eqnarray}
	\lift{w}{y!(z)}\widehat{\id{\{}u / z \id{\}}}
		& = &
		\lift{w}{y!(u)} \nonumber\\
	w[ \lpquote y!(z) \rpquote ] \widehat{ \id{\{}u / z \id{\}} }
		& = &
		w[ \lpquote y!(z) \rpquote ] \nonumber
\end{eqnarray}

Because the body of the process between quotes is impervious to
substitution, we get radically different answers. In fact, by
examining the first process in an input context,
e.g. $x?(z).\lift{w}{y!(z)}$, we see that the process under the lift
operator may be shaped by prefixed inputs binding a name inside it. In
this sense, the lift operator will be seen as a way to dynamically
construct processes before reifying them as names.

Finally equipped with these standard features we can present the
dynamics of the calculus.

\subsubsection{Operational semantics} 

Finally, we introduce the computational dynamics. What marks these
algebras as distinct from other more traditionally studied algebraic
structures, e.g. vector spaces or polynomial rings, is the manner in
which dynamics is captured. In traditional structures, dynamics is typically
expressed through morphisms between such structures, as in linear maps
between vector spaces or morphisms between rings. In algebras
associated with the semantics of computation, the dynamics is
expressed as part of the algebraic structure itself, through a
reduction reduction relation typically denoted by $\red$. Below, we
give a recursive presentation of this relation for the calculus used
in the encoding.

$\red \subseteq \pi \times \pi$
$\red : \pi \to \mathcal{P}(\pi)$

\begin{mathpar}
  \inferrule* [lab=Comm] { \textsf{match}( x_{src}, x_{trgt} ) } { x_{trgt}?(y)P \; | \; x_{src}!\langle {Q} \rangle \red P\{\quotep{Q}/y}\} }
  \and \\
  \inferrule* [lab=Par] {{P} \red {P}'} {{{P} | {Q}} \red {{P}' | {Q}}}
  \and
  \inferrule* [lab=Equiv]{{{P} \scong {P}'} \andalso {{P}' \red {Q}'} \andalso {{Q}' \scong {Q}}}{{P} \red {Q}}
\end{mathpar}

\begin{eqnarray*}
  match_{\equiv} (\quotep{P},\quotep{Q}) & := & P \equiv Q \\
  match_{\dagger}(\quotep{P},\quotep{Q}) & := & \forall R. P|Q \red^{*} R => R \red^{*} 0 \\
  match_{K}(\quotep{P},\quotep{Q}) & := & K \mbox{ for some context } K
\end{eqnarray*}

$u?(x)P | u!\langle Q \rangle \red P\{\quotep{Q}/x\}$

%We write $\wred$ for $\red^*$, and $P\red$ if $\exists Q $ such that $ P \red Q$.
We write $P\red$ if $\exists Q $ such that $ P \red Q$ and $P\not\red$, otherwise.

\section{Replication}

As mentioned before, it is known that replication (and hence
recursion) can be implemented in a higher-order process algebra
\cite{SangiorgiWalker}. As our first example of calculation with the
machinery thus far presented we give the construction explicitly in
the {\rhoc}.

\begin{eqnarray}
	D_{x} & := & \prefix{x}{y}{(\binpar{\outputp{x}{y}}{@{y}})} \nonumber\\
	\bangp_{x}{P} & := & \binpar{{x}!\langle{\binpar{D_{x}}{P}}\rangle}{D_{x}} \nonumber
\end{eqnarray}

\begin{eqnarray}
	\bangp_{x}{P} & & \nonumber\\
	=
	& {x}!\langle{(\prefix{x}{y}{(\outputp{x}{y} | @{y})) | P}}\rangle 
	      | \prefix{x}{y}{(\outputp{x}{y} | @{y})} & \nonumber\\
	\red
	& (\outputp{x}{y} | @{y})\substn{\quotep{(\prefix{x}{y}{(@{y} | \outputp{x}{y})) | P}}}{y} & \nonumber\\
	=
	& \outputp{x}{\quotep{(\prefix{x}{y}{(\outputp{x}{y} | @{y})) | P}}}
	  | {(\prefix{x}{y}{(\outputp{x}{y} | @{y})) | P}} & \nonumber\\
	\red
	& \ldots & \nonumber\\
	\red^*
	& P | P | \ldots & \nonumber
\end{eqnarray}

Of course, this encoding, as an implementation, runs away, unfolding
$\bangp{P}$ eagerly. A lazier and more implementable replication
operator, restricted to input-guarded processes, may be obtained as follows.

\begin{eqnarray}
\bangp{\prefix{u}{v}{P}} 
	:= 
	\binpar{\lift{x}{\prefix{u}{v}{(\binpar{D(x)}{P})}}}{D(x)} \nonumber
\end{eqnarray}

\begin{remark}
  Note that the lazier definition still does not deal with summation
  or mixed summation (i.e. sums over input and output). The reader is
  invited to construct definitions of replication that deal with these
  features. 

  Further, the definitions are parameterized in a name, $x$. Can you,
  gentle reader, make a definition that eliminates this parameter and
  guarantees no accidental interaction between the replication
  machinery and the process being replicated -- i.e. no accidental
  sharing of names used by the process to get its work done and the
  name(s) used by the replication to effect copying. This latter
  revision of the definition of replication is crucial to obtaining
  the expected identity $!!P \sim !P$.
\end{remark}

\begin{remark}\label{rem:paradoxical_combinator}
  The reader familiar with the lambda calculus will have noticed the
  similarity between $D$ and the paradoxical combinator.

  [Ed. note: the existence of this seems to suggest we have to be more
  restrictive on the set of processes and names we admit if we are to
  support no-cloning.]
\end{remark}

\subsubsection{Bisimulation}

The computational dynamics gives rise to another kind of equivalence,
the equivalence of computational behavior. As previously mentioned
this is typically captured \emph{via} some form of bisimulation.

% The notion we use in this paper is weak barbed bisimulation
% \cite{milner91polyadicpi}.

The notion we use in this paper is derived from weak barbed
bisimulation \cite{milner91polyadicpi}. 

\begin{definition}
An \emph{observation relation}, $\downarrow_{\mathcal N}$, over a set
of names, $\mathcal N$, is the smallest relation satisfying the rules
below.

\infrule[Out-barb]{y \in {\mathcal N}, \; x \nameeq y}
		  {\outputp{x}{v} \downarrow_{\mathcal N} x}
\infrule[Par-barb]{\mbox{$P\downarrow_{\mathcal N} x$ or $Q\downarrow_{\mathcal N} x$}}
		  {\binpar{P}{Q} \downarrow_{\mathcal N} x}

We write $P \Downarrow_{\mathcal N} x$ if there is $Q$ such that 
$P \wred Q$ and $Q \downarrow_{\mathcal N} x$.
\end{definition}

\begin{definition}
%\label{def.bbisim}
An  ${\mathcal N}$-\emph{barbed bisimulation} over a set of names, ${\mathcal N}$, is a symmetric binary relation 
${\mathcal S}_{\mathcal N}$ between agents such that $P\rel{S}_{\mathcal N}Q$ implies:
\begin{enumerate}
\item If $P \red P'$ then $Q \wred Q'$ and $P'\rel{S}_{\mathcal N} Q'$.
\item If $P\downarrow_{\mathcal N} x$, then $Q\Downarrow_{\mathcal N} x$.
\end{enumerate}
$P$ is ${\mathcal N}$-barbed bisimilar to $Q$, written
$P \wbbisim_{\mathcal N} Q$, if $P \rel{S}_{\mathcal N} Q$ for some ${\mathcal N}$-barbed bisimulation ${\mathcal S}_{\mathcal N}$.
\end{definition}

$\mathcal{R} \subseteq \pi \times \pi$

$P \mathcal{R} Q => \forall P'. P \red P' \Rightarrow \exists Q'. Q \red Q', P' \mathcal{R} Q'$

$P \vdash x \Rightarrow Q \vdash x$

\begin{mathpar}
  \inferrule*[lab=Out-barb]{x \nameeq y}{{y}!\langle{Q}\rangle \vdash x}
  \and
  \inferrule*[lab=Par-barb]{\mbox{$P\vdash x$ or $Q\vdash x$}}{\binpar{P}{Q} \vdash x}
\end{mathpar}

\subsubsection{Contexts}

One of the principle advantages of computational calculi like the
$\pi$-calculus is a well-defined notion of context,
contextual-equivalence and a correlation between
contextual-equivalence and notions of bisimulation. The notion of
context allows the decomposition of a process into (sub-)process and
its syntactic environment, its context. Thus, a context may be
thought of as a process with a ``hole'' (written $\Box$) in it. The
application of a context $M$ to a process $P$, written $M[P]$, is
tantamount to filling the hole in $M$ with $P$. In this paper we do
not need the full weight of this theory, but do make use of the notion
of context in the proof the main theorem. 

\begin{mathpar}
  \inferrule* [lab=summation] {} {{M_{M},M_{N}} \bc \Box \;|\; x.M_{A} \;|\; M_{M}+M_{N}}
  \and
  \inferrule* [lab=agent] {} {{M_{A}} \bc (\vec{x})M_{P} \;| \; \clift{P_0,\ldots,M_{P},\ldots,P_N}}
  \and \\
  \inferrule* [lab=process] {} {{M_{P}} \bc M_{N} \;| \;P|M_{P} }
\end{mathpar} 

\begin{mathpar}
  \inferrule* [lab=sychronization] {} {M_{N} \bc \Box \;|\; x?M_{F} \;|\; x!M_{C}}
  \and
  \inferrule* [lab=abstraction] {} {{M_{F}} \bc (x)M_{P} }
  \and
  \inferrule* [lab=concretion] {} {{M_{C}} \bc \langle M_{P} \rangle }
  \and \\
  \inferrule* [lab=process] {} {{M_{P}} \bc M_{N} \;| \;P|M_{P} }
\end{mathpar}

\begin{definition}[contextual application] Given a context $M$, and
  process $P$, we define the \emph{contextual application}, $M[P] :=
  M\{P/\Box\}$. That is, the contextual application of M to P is the
  substitution of $P$ for $\Box$ in $M$.
\end{definition}

$\meaningof{-} : L \to \mathcal{P}(\pi)$

\begin{mathpar}
  \inferrule* [lab=collection] {} {\meaningof{true} = \pi, \and \meaningof{~E} = \pi \setminus \meaningof{E}, \and \meaningof{E_{1} \& E_{2}} = \meaningof{E_{1}} \cap \meaningof{E_{2}}}
\end{mathpar}

\begin{mathpar}
  \inferrule* [lab=structure] {} {\meaningof{0} = \{ P \in \pi | P \equiv 0 \}, \and \\ \meaningof{E_1 | E_2} = \{ P \in \pi | P \equiv P_{1} | P_{2}, P_{1} \in \meaningof{E_{1}}, P_{2} \in \meaningof{E_2}\} }
\end{mathpar}

\begin{mathpar}
 \inferrule* [lab=behavior] {} {\meaningof{\langle a?b \rangle E} = \{ P \in \pi | P \equiv Q | u?(y)P', \\ \and \\\\ \and \\ \;\;\; u \in \meaningof{a}, \forall z.P'\{z/y\} \in \meaningof{E\{z/b\}}\}, \and \\ \meaningof{a!E} = \{ P \in \pi | P \equiv Q | x!\langle P' \rangle, x \in \meaningof{a} P' \in \meaningof{E}\} }
\end{mathpar}

\begin{mathpar}
 \inferrule* [lab=nominal] {} {\meaningof{\quotep{E}} = \{ \quotep{P} \in \quotep{\pi} | P \in \meaningof{E} \}, \and \meaningof{\quotep{P}} = \{ \quotep{Q} \in \quotep{\pi} | P \equiv Q \} \and \\ \meaningof{@\quotep{E}} = \{ P \in \pi | P \equiv @x, x \in \meaningof{E} \}}
\end{mathpar}

\begin{eqnarray*}
  \\
  \meaningof{-} : TS \to ST
\end{eqnarray*}

\begin{eqnarray*}
  \\
  L : TS \to ST
\end{eqnarray*}

\begin{eqnarray*}
  \\
  P \models E \iff P \in \meaningof{E}
\end{eqnarray*}

\begin{eqnarray*}
  P \approx_{L} Q \iff \forall E \in L. P \models E \iff Q \models E
\end{eqnarray*}

\begin{eqnarray*}
  P \approx_{K} Q
\end{eqnarray*}

\begin{eqnarray*}
  P \approx Q
\end{eqnarray*}

$\approx_{K} = \approx = \approx_{L}$

\subsubsection{Contextual duality}

Note that contexts extend the quotation operation to a family of
operations from processes to names. Given a context, $M$, we can
define a \emph{nominal context}, $\quotep{M}$ by $\quotep{M}[P] :=
\quotep{M[P]}$. To foreshadow what is to come we observe that these
operations enjoy a duality with processes very much like the duality
between vectors and maps from vectors to scalars.

Further, because the calculus is essentially higher-order, we have a
correspondence between contexts and processes. More specifically,
given a name $x$ and a context $M$ we can construct $M^{*}_{x}$ such
that 

\begin{mathpar}
  M^{*}_{x} | \lift{x}{P} \red M[P]
\end{mathpar}

namely,

\begin{mathpar}
  M^{*}_{x} := x?(u).M[\dropn{u}]
\end{mathpar}

The dependence of $M^{*}_{x}$ on a name makes it an abstraction, 

\begin{mathpar}
  M^{*} := (x)x?(u).M[\dropn{u}]
\end{mathpar}

\subsection{Additional notation}

It will sometimes be convenient to denote the process a name
quotes. We already have the notation $x = \quotep{P}$, but it will be
convenient to introduce an alternate notation, $\procn{x}$, when we
want to emphasize the connection to the use of the name. Note that, by
virtue of name equivalence, $\quotep{\procn{x}} \nameeq x$; so, the
notation is consistent with previous definitions.

Further, because names have structure it is possible to effect
substitutions on the basis of that structure. This means we need to
upgrade our notation for substitutions, which we accomplish by
adapting comprehension notation. Thus,

\begin{mathpar}
  P\{ y / x : x \in S \}
\end{mathpar}

is interpreted to mean the process derived from P by replacing (in a
capture-avoiding manner) each occurrence of $x$ in $S$ by $y$. For example,

\begin{mathpar}
  P\{ \quotep{\procn{x}|\procn{x}} / x : x \in \freenames{P} \}
\end{mathpar}

will replace each (occurrence) of a free name $x$ in $P$ by
$\quotep{\procn{x}|\procn{x}}$.

Also, we will avail ourselves of the notation $x^{L}$ and $x^{R}$ to
denote injections of a name into disjoint copies of the name
space. There are numerous ways to accomplish this. One example can be
found in \cite{MeredithR05}. This notation overloads to vectors of
names: $\vec{x}^{\pi} := (x_{i}^{\pi} \; : \; 0 \leq i < |\vec{x}| )$ where $\pi \in \{L,R\}$.

We also use $P^{\Box} := P|\Box$.

In \cite{MeredithR05} an interpretation of the new operator is
given. It turns out that there are several possible interpretations
all enjoying the requisite algebraic properties of the operator (see
\cite{milner91polyadicpi}). We will therefore make liberal use of
$(\nu\; \vec{x})P$.

% subsection the_syntax_and_semantics_of_the_notation_system (end)   

\input{qm2pi.qmops} 

\input{qm2pi.sterngerlach} 

\input{qm2pi.metric} 

% section concurrent_process_calculi (end)

%\input{qm2pi.proofsketch}

% section proof sketch (end)

%\input{qm2pi.slviaknots} 

% section spatial logic via knots (end)

\input{qm2pi.conclusion}

% section conclusion (end)

%\input{qm2pi.dtcodes} 

% section wiring algorithm (end)

\input{qm2pi.ack} 

% section acknowledgments (end)

\newpage


\bibliographystyle{plain}   
\bibliography{../../biblios/main.bib}

\input{qm2pi.rhodetails}

\end{document}

 

% section wiring algorithm (end)

\documentclass[12pt]{llncs}
%\documentclass{jktr}

\usepackage[pdftex]{hyperref}                   
\usepackage {listings}
\usepackage {mathpartir}
\usepackage{bcprules}
%\usepackage{listings}
                       
\usepackage{graphicx} 
%\usepackage[margins=2.5cm,nohead,nofoot]{geometry}
%\usepackage{geometry}
\usepackage{amsfonts}
\usepackage{amstext}
\usepackage{latexsym}
\usepackage{amssymb}
\usepackage{color}


%\include{myPreamble}
\include{qm2pi.local} 

%\ifpdf
%\usepackage[pdftex]{graphicx}
%\else
%\usepackage{graphicx}
%\fi

 % \ifpdf
%  \usepackage{pdfsync}
%  \if


%\title{Brief Article}
%\author{David F. Snyder}
%\author{L.G. Meredith}

%\address{Dept. of Math., Texas State University--San Marcos, San Marcos, TX 78666}
       
\pagestyle{empty}


\begin{document}

\lstset{language=[Objective]Caml,frame=shadowbox}

\input{qm2pi.front}

% section front matter (end)

\input{qm2pi.intro} 
 
% section introduction (end)

% \input{qm2pi.knotations} 

% section notation (end)

\input{qm2pi.process.calculi} 

% section concurrent_process_calculi_and_spatial_logics_ (end)
    
%\input{qm2pi.knots2pi} 

%\input{qm2pi.trefoil} 

%\input{qm2pi.mainthm} 

% subsection basic_interpretation (end)

%\input{qm2pi.rho.presentation} 
\subsection{The syntax and semantics of the notation system}\label{sub:the_syntax_and_semantics_of_the_notation_system} % (fold)

We now summarize a technical presentation of the calculus that
embodies our theory of dynamics. The typical presentation of such a
calculus follows the style of giving generators and relations on
them. The grammar, below, describing term constructors, freely
generates the set of processes, $\Proc$. This set is then quotiented
by a relation known as structural congruence and it is over this set
that the notion of dynamics is expressed. This presentation is
essentially that of \cite{MeredithR05} with the addition of
polyadicity and summation. For readability we have relegated some of
the technical subtleties to an appendix.

\subsubsection{Process grammar}\label{subsub:process_grammar}

\begin{mathpar}
  \inferrule* [lab=synchronization] {} {{M} \bc \pzero \;|\; x?F \;|\; x!C }
  \and
  \inferrule* [lab=abstraction] {} {{F} \bc (x)P}
  \and
  \inferrule* [lab=concretion] {} {{C} \bc \langle Q \rangle}
  \and
  \inferrule* [lab=process] {} {{P,Q} \bc M \;| \;P|Q \;|\; @{x}}
  \and
  \inferrule* [lab=name] {} {{x} \bc \quotep{P}}
\end{mathpar} 

Note that $\vec{x}$ (resp. $\vec{P}$) denotes a vector of names
(resp. processes) of length $|\vec{x}|$ (resp. $|\vec{P}|$). We adopt
the following useful abbreviations.

\begin{mathpar}
   x?(\vec{y}).P := x.(\vec{y})P \and  x\clift{\vec{P}} := x.\clift{\vec{P}}
   \and x!(y) := \lift{x}{\dropn{y}}
   \and \Pi_{i=0}^{n-1}P_i := P_0 | \ldots | P_{n-1}
\end{mathpar}

\subsubsection{Structural congruence}

\paragraph{Free and bound names and alpha-equivalence.} At the
core of structural equivalence is alpha-equivalence which identifies
process that are the same up to a change of variable. Formally, we
recognize the distinction between free and bound names. The free names
of a process, $\freenames{P}$, may be calculated recursively as
follows:

\begin{mathpar}
\freenames{\pzero} := \emptyset
  \and \\
  \freenames{x?(y).P} := \{ x \} \cup (\freenames{P} \setminus \{ y \})
  \and 
  \freenames{x!\langle P \rangle} := \{ x \} \cup \{ P \} 
  \and \\
  \freenames{P|Q} := \freenames{P} \cup \freenames{Q}
  \and \\
  \freenames{@{x}} := \{ x \}
\end{mathpar}

$\pi$
$\quotep{\pi}$

$\freenames{-} : \pi \to \mathcal{P}(\quotep{\pi})$

\begin{eqnarray*}
  \freenames{\pzero} & := & \emptyset \\
  \freenames{x?(y).P} & := & \{ x \} \cup (\freenames{P} \setminus \{ y \}) \\
  \freenames{x!\langle P \rangle} & := & \{ x \} \cup \{ P \} \\
  \freenames{P|Q} & := & \freenames{P} \cup \freenames{Q} \\
  \freenames{\dropn{x}} & := & \{ x \}
\end{eqnarray*}

The bound names of a process, $\boundnames{P}$, are those names occurring in $P$
that are not free. For example, in $x?(y).0$, the name $x$ is free, while $y$ is bound.

\begin{mathpar}
  \inferrule* [lab=monoidal-laws] {} { P|Q \equiv Q|P \and P|0 \equiv P \and P|(Q|R) \equiv (P|Q)|R }
\end{mathpar}

\begin{mathpar}
  \inferrule* [lab=alpha-equivalence] {} { (x)P \equiv (y)P\{y/x\} \and y \not\in \freenames{P} }
\end{mathpar}

\begin{definition}
Then two processes, $P,Q$, are alpha-equivalent if $P = Q\{\vec{y}/\vec{x}\}$ for
some $\vec{x} \in \boundnames{Q},\vec{y} \in \boundnames{P}$, where $Q\{\vec{y}/\vec{x}\}$
denotes the capture-avoiding substitution of $\vec{y}$ for $\vec{x}$ in $Q$.
\end{definition}

\begin{definition}
  The {\em structural congruence} \cite{SangiorgiWalker} , $\equiv$,
  between processes is the least congruence containing
  alpha-equivalence, satisfying the abelian monoid laws
  (associativity, commutativity and $\pzero$ as identity) for parallel
  composition $|$ and for summation $+$.
\end{definition}

\subsection{Name equivalence}

We take name equivalence, written $\nameeq$, to be the smallest
equivalence relation generated by the following rules.

\begin{mathpar}
\inferrule*[lab=Quote-drop]
{ }
{ \quotep{@{x}} \nameeq x }

\inferrule*[lab=Struct-equiv]
{ P \scong Q }
{ \quotep{P} \nameeq \quotep{Q} }
\end{mathpar}

The astute reader will have noticed that the mutual recursion of names
and processes imposes a mutual recursion on alpha-equivalence and
structural equivalence via name-equivalence. Fortunately, all of this
works out pleasantly and we may calculate in the natural way, free of
concern. The reader interested in the details is referred to the
appendix \ref{appendix:rho_details}.

\subsection{Substitution}

We use $\Proc$ for the set of processes, $\QProc$ for the set of
names, and $\id{\{}\vec{y} / \vec{x} \id{\}}$ to denote partial maps,
$s : \QProc \rightarrow \QProc$. A map, $s$ lifts, uniquely, to a map
on process terms, $\widehat{s} : \Proc \rightarrow \Proc$ by the
following equations.

\begin{mathpar}
  (0) \psubstp{Q}{P} := 0 \\
  (R \juxtap S) \psubstp{Q}{P}
  :=    
  (R)\psubstp{Q}{P} \juxtap (S) \psubstp{Q}{P} \\
  (x?(y).R) \psubstp{Q}{P}    
  :=    
  (x)\substp{Q}{P} (z)\concat( (R \psubstn{z}{y}) \psubstp{Q}{P} ) \\
  (\lift{x}{R}) \psubstp{Q}{P}  
  :=
  \lift{(x)\substp{Q}{P}}{ R \psubstp{Q}{P} } \\
%   (\dropn{x})  \psubstp{Q}{P}       
%   := 
%   \left\{ 
%     \begin{array}{ccc} 
%       \dropn{\quotep{Q}} & & x \nameeq \quotep{P} \\
%       \dropn{x} & & otherwise \\
%     \end{array}
%   \right. 
  (\dropn{x})  \psubstp{Q}{P}       
  := 
  \left\{ 
    \begin{array}{ccc} 
      Q & & x \nameeq \quotep{P} \\
      \dropn{x} & & otherwise \\
    \end{array}
  \right.
\end{mathpar}
 

where

\begin{eqnarray}
  (x)\id{\{} \lpquote Q \rpquote / \lpquote P \rpquote \id{\}}            = 
  \left\{ 
    \begin{array}{ccc}
      \lpquote Q \rpquote & & x \nameeq \lpquote P \rpquote \\
      x & & otherwise \\
    \end{array}
  \right. \nonumber
\end{eqnarray}

and $z$ is chosen distinct from $\quotep{P}$, $\quotep{Q}$, the free
names in $Q$, and all the names in $R$. Our $\alpha$-equivalence will
be built in the standard way from this substitution.

\begin{remark}\label{rem:no_self_referential_names}
  One consequence of these definitions is that $\forall P. \quotep{P}
  \not\in \freenames{P}$.
\end{remark}

\subsection{ Dynamic quote: an example }

Anticipating something of what's to come, consider applying the
substitution, $\widehat{\id{\{}u / z \id{\}}}$, to the following pair
of processes, $\lift{w}{y!(z)}$ and $w[ \lpquote y!(z) \rpquote ]$.

\begin{eqnarray}
	\lift{w}{y!(z)}\widehat{\id{\{}u / z \id{\}}}
		& = &
		\lift{w}{y!(u)} \nonumber\\
	w[ \lpquote y!(z) \rpquote ] \widehat{ \id{\{}u / z \id{\}} }
		& = &
		w[ \lpquote y!(z) \rpquote ] \nonumber
\end{eqnarray}

Because the body of the process between quotes is impervious to
substitution, we get radically different answers. In fact, by
examining the first process in an input context,
e.g. $x?(z).\lift{w}{y!(z)}$, we see that the process under the lift
operator may be shaped by prefixed inputs binding a name inside it. In
this sense, the lift operator will be seen as a way to dynamically
construct processes before reifying them as names.

Finally equipped with these standard features we can present the
dynamics of the calculus.

\subsubsection{Operational semantics} 

Finally, we introduce the computational dynamics. What marks these
algebras as distinct from other more traditionally studied algebraic
structures, e.g. vector spaces or polynomial rings, is the manner in
which dynamics is captured. In traditional structures, dynamics is typically
expressed through morphisms between such structures, as in linear maps
between vector spaces or morphisms between rings. In algebras
associated with the semantics of computation, the dynamics is
expressed as part of the algebraic structure itself, through a
reduction reduction relation typically denoted by $\red$. Below, we
give a recursive presentation of this relation for the calculus used
in the encoding.

$\red \subseteq \pi \times \pi$
$\red : \pi \to \mathcal{P}(\pi)$

\begin{mathpar}
  \inferrule* [lab=Comm] { \textsf{match}( x_{src}, x_{trgt} ) } { x_{trgt}?(y)P \; | \; x_{src}!\langle {Q} \rangle \red P\{\quotep{Q}/y}\} }
  \and \\
  \inferrule* [lab=Par] {{P} \red {P}'} {{{P} | {Q}} \red {{P}' | {Q}}}
  \and
  \inferrule* [lab=Equiv]{{{P} \scong {P}'} \andalso {{P}' \red {Q}'} \andalso {{Q}' \scong {Q}}}{{P} \red {Q}}
\end{mathpar}

\begin{eqnarray*}
  match_{\equiv} (\quotep{P},\quotep{Q}) & := & P \equiv Q \\
  match_{\dagger}(\quotep{P},\quotep{Q}) & := & \forall R. P|Q \red^{*} R => R \red^{*} 0 \\
  match_{K}(\quotep{P},\quotep{Q}) & := & K \mbox{ for some context } K
\end{eqnarray*}

$u?(x)P | u!\langle Q \rangle \red P\{\quotep{Q}/x\}$

%We write $\wred$ for $\red^*$, and $P\red$ if $\exists Q $ such that $ P \red Q$.
We write $P\red$ if $\exists Q $ such that $ P \red Q$ and $P\not\red$, otherwise.

\section{Replication}

As mentioned before, it is known that replication (and hence
recursion) can be implemented in a higher-order process algebra
\cite{SangiorgiWalker}. As our first example of calculation with the
machinery thus far presented we give the construction explicitly in
the {\rhoc}.

\begin{eqnarray}
	D_{x} & := & \prefix{x}{y}{(\binpar{\outputp{x}{y}}{@{y}})} \nonumber\\
	\bangp_{x}{P} & := & \binpar{{x}!\langle{\binpar{D_{x}}{P}}\rangle}{D_{x}} \nonumber
\end{eqnarray}

\begin{eqnarray}
	\bangp_{x}{P} & & \nonumber\\
	=
	& {x}!\langle{(\prefix{x}{y}{(\outputp{x}{y} | @{y})) | P}}\rangle 
	      | \prefix{x}{y}{(\outputp{x}{y} | @{y})} & \nonumber\\
	\red
	& (\outputp{x}{y} | @{y})\substn{\quotep{(\prefix{x}{y}{(@{y} | \outputp{x}{y})) | P}}}{y} & \nonumber\\
	=
	& \outputp{x}{\quotep{(\prefix{x}{y}{(\outputp{x}{y} | @{y})) | P}}}
	  | {(\prefix{x}{y}{(\outputp{x}{y} | @{y})) | P}} & \nonumber\\
	\red
	& \ldots & \nonumber\\
	\red^*
	& P | P | \ldots & \nonumber
\end{eqnarray}

Of course, this encoding, as an implementation, runs away, unfolding
$\bangp{P}$ eagerly. A lazier and more implementable replication
operator, restricted to input-guarded processes, may be obtained as follows.

\begin{eqnarray}
\bangp{\prefix{u}{v}{P}} 
	:= 
	\binpar{\lift{x}{\prefix{u}{v}{(\binpar{D(x)}{P})}}}{D(x)} \nonumber
\end{eqnarray}

\begin{remark}
  Note that the lazier definition still does not deal with summation
  or mixed summation (i.e. sums over input and output). The reader is
  invited to construct definitions of replication that deal with these
  features. 

  Further, the definitions are parameterized in a name, $x$. Can you,
  gentle reader, make a definition that eliminates this parameter and
  guarantees no accidental interaction between the replication
  machinery and the process being replicated -- i.e. no accidental
  sharing of names used by the process to get its work done and the
  name(s) used by the replication to effect copying. This latter
  revision of the definition of replication is crucial to obtaining
  the expected identity $!!P \sim !P$.
\end{remark}

\begin{remark}\label{rem:paradoxical_combinator}
  The reader familiar with the lambda calculus will have noticed the
  similarity between $D$ and the paradoxical combinator.

  [Ed. note: the existence of this seems to suggest we have to be more
  restrictive on the set of processes and names we admit if we are to
  support no-cloning.]
\end{remark}

\subsubsection{Bisimulation}

The computational dynamics gives rise to another kind of equivalence,
the equivalence of computational behavior. As previously mentioned
this is typically captured \emph{via} some form of bisimulation.

% The notion we use in this paper is weak barbed bisimulation
% \cite{milner91polyadicpi}.

The notion we use in this paper is derived from weak barbed
bisimulation \cite{milner91polyadicpi}. 

\begin{definition}
An \emph{observation relation}, $\downarrow_{\mathcal N}$, over a set
of names, $\mathcal N$, is the smallest relation satisfying the rules
below.

\infrule[Out-barb]{y \in {\mathcal N}, \; x \nameeq y}
		  {\outputp{x}{v} \downarrow_{\mathcal N} x}
\infrule[Par-barb]{\mbox{$P\downarrow_{\mathcal N} x$ or $Q\downarrow_{\mathcal N} x$}}
		  {\binpar{P}{Q} \downarrow_{\mathcal N} x}

We write $P \Downarrow_{\mathcal N} x$ if there is $Q$ such that 
$P \wred Q$ and $Q \downarrow_{\mathcal N} x$.
\end{definition}

\begin{definition}
%\label{def.bbisim}
An  ${\mathcal N}$-\emph{barbed bisimulation} over a set of names, ${\mathcal N}$, is a symmetric binary relation 
${\mathcal S}_{\mathcal N}$ between agents such that $P\rel{S}_{\mathcal N}Q$ implies:
\begin{enumerate}
\item If $P \red P'$ then $Q \wred Q'$ and $P'\rel{S}_{\mathcal N} Q'$.
\item If $P\downarrow_{\mathcal N} x$, then $Q\Downarrow_{\mathcal N} x$.
\end{enumerate}
$P$ is ${\mathcal N}$-barbed bisimilar to $Q$, written
$P \wbbisim_{\mathcal N} Q$, if $P \rel{S}_{\mathcal N} Q$ for some ${\mathcal N}$-barbed bisimulation ${\mathcal S}_{\mathcal N}$.
\end{definition}

$\mathcal{R} \subseteq \pi \times \pi$

$P \mathcal{R} Q => \forall P'. P \red P' \Rightarrow \exists Q'. Q \red Q', P' \mathcal{R} Q'$

$P \vdash x \Rightarrow Q \vdash x$

\begin{mathpar}
  \inferrule*[lab=Out-barb]{x \nameeq y}{{y}!\langle{Q}\rangle \vdash x}
  \and
  \inferrule*[lab=Par-barb]{\mbox{$P\vdash x$ or $Q\vdash x$}}{\binpar{P}{Q} \vdash x}
\end{mathpar}

\subsubsection{Contexts}

One of the principle advantages of computational calculi like the
$\pi$-calculus is a well-defined notion of context,
contextual-equivalence and a correlation between
contextual-equivalence and notions of bisimulation. The notion of
context allows the decomposition of a process into (sub-)process and
its syntactic environment, its context. Thus, a context may be
thought of as a process with a ``hole'' (written $\Box$) in it. The
application of a context $M$ to a process $P$, written $M[P]$, is
tantamount to filling the hole in $M$ with $P$. In this paper we do
not need the full weight of this theory, but do make use of the notion
of context in the proof the main theorem. 

\begin{mathpar}
  \inferrule* [lab=summation] {} {{M_{M},M_{N}} \bc \Box \;|\; x.M_{A} \;|\; M_{M}+M_{N}}
  \and
  \inferrule* [lab=agent] {} {{M_{A}} \bc (\vec{x})M_{P} \;| \; \clift{P_0,\ldots,M_{P},\ldots,P_N}}
  \and \\
  \inferrule* [lab=process] {} {{M_{P}} \bc M_{N} \;| \;P|M_{P} }
\end{mathpar} 

\begin{mathpar}
  \inferrule* [lab=sychronization] {} {M_{N} \bc \Box \;|\; x?M_{F} \;|\; x!M_{C}}
  \and
  \inferrule* [lab=abstraction] {} {{M_{F}} \bc (x)M_{P} }
  \and
  \inferrule* [lab=concretion] {} {{M_{C}} \bc \langle M_{P} \rangle }
  \and \\
  \inferrule* [lab=process] {} {{M_{P}} \bc M_{N} \;| \;P|M_{P} }
\end{mathpar}

\begin{definition}[contextual application] Given a context $M$, and
  process $P$, we define the \emph{contextual application}, $M[P] :=
  M\{P/\Box\}$. That is, the contextual application of M to P is the
  substitution of $P$ for $\Box$ in $M$.
\end{definition}

$\meaningof{-} : L \to \mathcal{P}(\pi)$

\begin{mathpar}
  \inferrule* [lab=collection] {} {\meaningof{true} = \pi, \and \meaningof{~E} = \pi \setminus \meaningof{E}, \and \meaningof{E_{1} \& E_{2}} = \meaningof{E_{1}} \cap \meaningof{E_{2}}}
\end{mathpar}

\begin{mathpar}
  \inferrule* [lab=structure] {} {\meaningof{0} = \{ P \in \pi | P \equiv 0 \}, \and \\ \meaningof{E_1 | E_2} = \{ P \in \pi | P \equiv P_{1} | P_{2}, P_{1} \in \meaningof{E_{1}}, P_{2} \in \meaningof{E_2}\} }
\end{mathpar}

\begin{mathpar}
 \inferrule* [lab=behavior] {} {\meaningof{\langle a?b \rangle E} = \{ P \in \pi | P \equiv Q | u?(y)P', \\ \and \\\\ \and \\ \;\;\; u \in \meaningof{a}, \forall z.P'\{z/y\} \in \meaningof{E\{z/b\}}\}, \and \\ \meaningof{a!E} = \{ P \in \pi | P \equiv Q | x!\langle P' \rangle, x \in \meaningof{a} P' \in \meaningof{E}\} }
\end{mathpar}

\begin{mathpar}
 \inferrule* [lab=nominal] {} {\meaningof{\quotep{E}} = \{ \quotep{P} \in \quotep{\pi} | P \in \meaningof{E} \}, \and \meaningof{\quotep{P}} = \{ \quotep{Q} \in \quotep{\pi} | P \equiv Q \} \and \\ \meaningof{@\quotep{E}} = \{ P \in \pi | P \equiv @x, x \in \meaningof{E} \}}
\end{mathpar}

\begin{eqnarray*}
  \\
  \meaningof{-} : TS \to ST
\end{eqnarray*}

\begin{eqnarray*}
  \\
  L : TS \to ST
\end{eqnarray*}

\begin{eqnarray*}
  \\
  P \models E \iff P \in \meaningof{E}
\end{eqnarray*}

\begin{eqnarray*}
  P \approx_{L} Q \iff \forall E \in L. P \models E \iff Q \models E
\end{eqnarray*}

\begin{eqnarray*}
  P \approx_{K} Q
\end{eqnarray*}

\begin{eqnarray*}
  P \approx Q
\end{eqnarray*}

$\approx_{K} = \approx = \approx_{L}$

\subsubsection{Contextual duality}

Note that contexts extend the quotation operation to a family of
operations from processes to names. Given a context, $M$, we can
define a \emph{nominal context}, $\quotep{M}$ by $\quotep{M}[P] :=
\quotep{M[P]}$. To foreshadow what is to come we observe that these
operations enjoy a duality with processes very much like the duality
between vectors and maps from vectors to scalars.

Further, because the calculus is essentially higher-order, we have a
correspondence between contexts and processes. More specifically,
given a name $x$ and a context $M$ we can construct $M^{*}_{x}$ such
that 

\begin{mathpar}
  M^{*}_{x} | \lift{x}{P} \red M[P]
\end{mathpar}

namely,

\begin{mathpar}
  M^{*}_{x} := x?(u).M[\dropn{u}]
\end{mathpar}

The dependence of $M^{*}_{x}$ on a name makes it an abstraction, 

\begin{mathpar}
  M^{*} := (x)x?(u).M[\dropn{u}]
\end{mathpar}

\subsection{Additional notation}

It will sometimes be convenient to denote the process a name
quotes. We already have the notation $x = \quotep{P}$, but it will be
convenient to introduce an alternate notation, $\procn{x}$, when we
want to emphasize the connection to the use of the name. Note that, by
virtue of name equivalence, $\quotep{\procn{x}} \nameeq x$; so, the
notation is consistent with previous definitions.

Further, because names have structure it is possible to effect
substitutions on the basis of that structure. This means we need to
upgrade our notation for substitutions, which we accomplish by
adapting comprehension notation. Thus,

\begin{mathpar}
  P\{ y / x : x \in S \}
\end{mathpar}

is interpreted to mean the process derived from P by replacing (in a
capture-avoiding manner) each occurrence of $x$ in $S$ by $y$. For example,

\begin{mathpar}
  P\{ \quotep{\procn{x}|\procn{x}} / x : x \in \freenames{P} \}
\end{mathpar}

will replace each (occurrence) of a free name $x$ in $P$ by
$\quotep{\procn{x}|\procn{x}}$.

Also, we will avail ourselves of the notation $x^{L}$ and $x^{R}$ to
denote injections of a name into disjoint copies of the name
space. There are numerous ways to accomplish this. One example can be
found in \cite{MeredithR05}. This notation overloads to vectors of
names: $\vec{x}^{\pi} := (x_{i}^{\pi} \; : \; 0 \leq i < |\vec{x}| )$ where $\pi \in \{L,R\}$.

We also use $P^{\Box} := P|\Box$.

In \cite{MeredithR05} an interpretation of the new operator is
given. It turns out that there are several possible interpretations
all enjoying the requisite algebraic properties of the operator (see
\cite{milner91polyadicpi}). We will therefore make liberal use of
$(\nu\; \vec{x})P$.

% subsection the_syntax_and_semantics_of_the_notation_system (end)   

\input{qm2pi.qmops} 

\input{qm2pi.sterngerlach} 

\input{qm2pi.metric} 

% section concurrent_process_calculi (end)

%\input{qm2pi.proofsketch}

% section proof sketch (end)

%\input{qm2pi.slviaknots} 

% section spatial logic via knots (end)

\input{qm2pi.conclusion}

% section conclusion (end)

%\input{qm2pi.dtcodes} 

% section wiring algorithm (end)

\input{qm2pi.ack} 

% section acknowledgments (end)

\newpage


\bibliographystyle{plain}   
\bibliography{../../biblios/main.bib}

\input{qm2pi.rhodetails}

\end{document}

 

% section acknowledgments (end)

\newpage


\bibliographystyle{plain}   
\bibliography{../../biblios/main.bib}

\documentclass[12pt]{llncs}
%\documentclass{jktr}

\usepackage[pdftex]{hyperref}                   
\usepackage {listings}
\usepackage {mathpartir}
\usepackage{bcprules}
%\usepackage{listings}
                       
\usepackage{graphicx} 
%\usepackage[margins=2.5cm,nohead,nofoot]{geometry}
%\usepackage{geometry}
\usepackage{amsfonts}
\usepackage{amstext}
\usepackage{latexsym}
\usepackage{amssymb}
\usepackage{color}


%\include{myPreamble}
\include{qm2pi.local} 

%\ifpdf
%\usepackage[pdftex]{graphicx}
%\else
%\usepackage{graphicx}
%\fi

 % \ifpdf
%  \usepackage{pdfsync}
%  \if


%\title{Brief Article}
%\author{David F. Snyder}
%\author{L.G. Meredith}

%\address{Dept. of Math., Texas State University--San Marcos, San Marcos, TX 78666}
       
\pagestyle{empty}


\begin{document}

\lstset{language=[Objective]Caml,frame=shadowbox}

\input{qm2pi.front}

% section front matter (end)

\input{qm2pi.intro} 
 
% section introduction (end)

% \input{qm2pi.knotations} 

% section notation (end)

\input{qm2pi.process.calculi} 

% section concurrent_process_calculi_and_spatial_logics_ (end)
    
%\input{qm2pi.knots2pi} 

%\input{qm2pi.trefoil} 

%\input{qm2pi.mainthm} 

% subsection basic_interpretation (end)

%\input{qm2pi.rho.presentation} 
\subsection{The syntax and semantics of the notation system}\label{sub:the_syntax_and_semantics_of_the_notation_system} % (fold)

We now summarize a technical presentation of the calculus that
embodies our theory of dynamics. The typical presentation of such a
calculus follows the style of giving generators and relations on
them. The grammar, below, describing term constructors, freely
generates the set of processes, $\Proc$. This set is then quotiented
by a relation known as structural congruence and it is over this set
that the notion of dynamics is expressed. This presentation is
essentially that of \cite{MeredithR05} with the addition of
polyadicity and summation. For readability we have relegated some of
the technical subtleties to an appendix.

\subsubsection{Process grammar}\label{subsub:process_grammar}

\begin{mathpar}
  \inferrule* [lab=synchronization] {} {{M} \bc \pzero \;|\; x?F \;|\; x!C }
  \and
  \inferrule* [lab=abstraction] {} {{F} \bc (x)P}
  \and
  \inferrule* [lab=concretion] {} {{C} \bc \langle Q \rangle}
  \and
  \inferrule* [lab=process] {} {{P,Q} \bc M \;| \;P|Q \;|\; @{x}}
  \and
  \inferrule* [lab=name] {} {{x} \bc \quotep{P}}
\end{mathpar} 

Note that $\vec{x}$ (resp. $\vec{P}$) denotes a vector of names
(resp. processes) of length $|\vec{x}|$ (resp. $|\vec{P}|$). We adopt
the following useful abbreviations.

\begin{mathpar}
   x?(\vec{y}).P := x.(\vec{y})P \and  x\clift{\vec{P}} := x.\clift{\vec{P}}
   \and x!(y) := \lift{x}{\dropn{y}}
   \and \Pi_{i=0}^{n-1}P_i := P_0 | \ldots | P_{n-1}
\end{mathpar}

\subsubsection{Structural congruence}

\paragraph{Free and bound names and alpha-equivalence.} At the
core of structural equivalence is alpha-equivalence which identifies
process that are the same up to a change of variable. Formally, we
recognize the distinction between free and bound names. The free names
of a process, $\freenames{P}$, may be calculated recursively as
follows:

\begin{mathpar}
\freenames{\pzero} := \emptyset
  \and \\
  \freenames{x?(y).P} := \{ x \} \cup (\freenames{P} \setminus \{ y \})
  \and 
  \freenames{x!\langle P \rangle} := \{ x \} \cup \{ P \} 
  \and \\
  \freenames{P|Q} := \freenames{P} \cup \freenames{Q}
  \and \\
  \freenames{@{x}} := \{ x \}
\end{mathpar}

$\pi$
$\quotep{\pi}$

$\freenames{-} : \pi \to \mathcal{P}(\quotep{\pi})$

\begin{eqnarray*}
  \freenames{\pzero} & := & \emptyset \\
  \freenames{x?(y).P} & := & \{ x \} \cup (\freenames{P} \setminus \{ y \}) \\
  \freenames{x!\langle P \rangle} & := & \{ x \} \cup \{ P \} \\
  \freenames{P|Q} & := & \freenames{P} \cup \freenames{Q} \\
  \freenames{\dropn{x}} & := & \{ x \}
\end{eqnarray*}

The bound names of a process, $\boundnames{P}$, are those names occurring in $P$
that are not free. For example, in $x?(y).0$, the name $x$ is free, while $y$ is bound.

\begin{mathpar}
  \inferrule* [lab=monoidal-laws] {} { P|Q \equiv Q|P \and P|0 \equiv P \and P|(Q|R) \equiv (P|Q)|R }
\end{mathpar}

\begin{mathpar}
  \inferrule* [lab=alpha-equivalence] {} { (x)P \equiv (y)P\{y/x\} \and y \not\in \freenames{P} }
\end{mathpar}

\begin{definition}
Then two processes, $P,Q$, are alpha-equivalent if $P = Q\{\vec{y}/\vec{x}\}$ for
some $\vec{x} \in \boundnames{Q},\vec{y} \in \boundnames{P}$, where $Q\{\vec{y}/\vec{x}\}$
denotes the capture-avoiding substitution of $\vec{y}$ for $\vec{x}$ in $Q$.
\end{definition}

\begin{definition}
  The {\em structural congruence} \cite{SangiorgiWalker} , $\equiv$,
  between processes is the least congruence containing
  alpha-equivalence, satisfying the abelian monoid laws
  (associativity, commutativity and $\pzero$ as identity) for parallel
  composition $|$ and for summation $+$.
\end{definition}

\subsection{Name equivalence}

We take name equivalence, written $\nameeq$, to be the smallest
equivalence relation generated by the following rules.

\begin{mathpar}
\inferrule*[lab=Quote-drop]
{ }
{ \quotep{@{x}} \nameeq x }

\inferrule*[lab=Struct-equiv]
{ P \scong Q }
{ \quotep{P} \nameeq \quotep{Q} }
\end{mathpar}

The astute reader will have noticed that the mutual recursion of names
and processes imposes a mutual recursion on alpha-equivalence and
structural equivalence via name-equivalence. Fortunately, all of this
works out pleasantly and we may calculate in the natural way, free of
concern. The reader interested in the details is referred to the
appendix \ref{appendix:rho_details}.

\subsection{Substitution}

We use $\Proc$ for the set of processes, $\QProc$ for the set of
names, and $\id{\{}\vec{y} / \vec{x} \id{\}}$ to denote partial maps,
$s : \QProc \rightarrow \QProc$. A map, $s$ lifts, uniquely, to a map
on process terms, $\widehat{s} : \Proc \rightarrow \Proc$ by the
following equations.

\begin{mathpar}
  (0) \psubstp{Q}{P} := 0 \\
  (R \juxtap S) \psubstp{Q}{P}
  :=    
  (R)\psubstp{Q}{P} \juxtap (S) \psubstp{Q}{P} \\
  (x?(y).R) \psubstp{Q}{P}    
  :=    
  (x)\substp{Q}{P} (z)\concat( (R \psubstn{z}{y}) \psubstp{Q}{P} ) \\
  (\lift{x}{R}) \psubstp{Q}{P}  
  :=
  \lift{(x)\substp{Q}{P}}{ R \psubstp{Q}{P} } \\
%   (\dropn{x})  \psubstp{Q}{P}       
%   := 
%   \left\{ 
%     \begin{array}{ccc} 
%       \dropn{\quotep{Q}} & & x \nameeq \quotep{P} \\
%       \dropn{x} & & otherwise \\
%     \end{array}
%   \right. 
  (\dropn{x})  \psubstp{Q}{P}       
  := 
  \left\{ 
    \begin{array}{ccc} 
      Q & & x \nameeq \quotep{P} \\
      \dropn{x} & & otherwise \\
    \end{array}
  \right.
\end{mathpar}
 

where

\begin{eqnarray}
  (x)\id{\{} \lpquote Q \rpquote / \lpquote P \rpquote \id{\}}            = 
  \left\{ 
    \begin{array}{ccc}
      \lpquote Q \rpquote & & x \nameeq \lpquote P \rpquote \\
      x & & otherwise \\
    \end{array}
  \right. \nonumber
\end{eqnarray}

and $z$ is chosen distinct from $\quotep{P}$, $\quotep{Q}$, the free
names in $Q$, and all the names in $R$. Our $\alpha$-equivalence will
be built in the standard way from this substitution.

\begin{remark}\label{rem:no_self_referential_names}
  One consequence of these definitions is that $\forall P. \quotep{P}
  \not\in \freenames{P}$.
\end{remark}

\subsection{ Dynamic quote: an example }

Anticipating something of what's to come, consider applying the
substitution, $\widehat{\id{\{}u / z \id{\}}}$, to the following pair
of processes, $\lift{w}{y!(z)}$ and $w[ \lpquote y!(z) \rpquote ]$.

\begin{eqnarray}
	\lift{w}{y!(z)}\widehat{\id{\{}u / z \id{\}}}
		& = &
		\lift{w}{y!(u)} \nonumber\\
	w[ \lpquote y!(z) \rpquote ] \widehat{ \id{\{}u / z \id{\}} }
		& = &
		w[ \lpquote y!(z) \rpquote ] \nonumber
\end{eqnarray}

Because the body of the process between quotes is impervious to
substitution, we get radically different answers. In fact, by
examining the first process in an input context,
e.g. $x?(z).\lift{w}{y!(z)}$, we see that the process under the lift
operator may be shaped by prefixed inputs binding a name inside it. In
this sense, the lift operator will be seen as a way to dynamically
construct processes before reifying them as names.

Finally equipped with these standard features we can present the
dynamics of the calculus.

\subsubsection{Operational semantics} 

Finally, we introduce the computational dynamics. What marks these
algebras as distinct from other more traditionally studied algebraic
structures, e.g. vector spaces or polynomial rings, is the manner in
which dynamics is captured. In traditional structures, dynamics is typically
expressed through morphisms between such structures, as in linear maps
between vector spaces or morphisms between rings. In algebras
associated with the semantics of computation, the dynamics is
expressed as part of the algebraic structure itself, through a
reduction reduction relation typically denoted by $\red$. Below, we
give a recursive presentation of this relation for the calculus used
in the encoding.

$\red \subseteq \pi \times \pi$
$\red : \pi \to \mathcal{P}(\pi)$

\begin{mathpar}
  \inferrule* [lab=Comm] { \textsf{match}( x_{src}, x_{trgt} ) } { x_{trgt}?(y)P \; | \; x_{src}!\langle {Q} \rangle \red P\{\quotep{Q}/y}\} }
  \and \\
  \inferrule* [lab=Par] {{P} \red {P}'} {{{P} | {Q}} \red {{P}' | {Q}}}
  \and
  \inferrule* [lab=Equiv]{{{P} \scong {P}'} \andalso {{P}' \red {Q}'} \andalso {{Q}' \scong {Q}}}{{P} \red {Q}}
\end{mathpar}

\begin{eqnarray*}
  match_{\equiv} (\quotep{P},\quotep{Q}) & := & P \equiv Q \\
  match_{\dagger}(\quotep{P},\quotep{Q}) & := & \forall R. P|Q \red^{*} R => R \red^{*} 0 \\
  match_{K}(\quotep{P},\quotep{Q}) & := & K \mbox{ for some context } K
\end{eqnarray*}

$u?(x)P | u!\langle Q \rangle \red P\{\quotep{Q}/x\}$

%We write $\wred$ for $\red^*$, and $P\red$ if $\exists Q $ such that $ P \red Q$.
We write $P\red$ if $\exists Q $ such that $ P \red Q$ and $P\not\red$, otherwise.

\section{Replication}

As mentioned before, it is known that replication (and hence
recursion) can be implemented in a higher-order process algebra
\cite{SangiorgiWalker}. As our first example of calculation with the
machinery thus far presented we give the construction explicitly in
the {\rhoc}.

\begin{eqnarray}
	D_{x} & := & \prefix{x}{y}{(\binpar{\outputp{x}{y}}{@{y}})} \nonumber\\
	\bangp_{x}{P} & := & \binpar{{x}!\langle{\binpar{D_{x}}{P}}\rangle}{D_{x}} \nonumber
\end{eqnarray}

\begin{eqnarray}
	\bangp_{x}{P} & & \nonumber\\
	=
	& {x}!\langle{(\prefix{x}{y}{(\outputp{x}{y} | @{y})) | P}}\rangle 
	      | \prefix{x}{y}{(\outputp{x}{y} | @{y})} & \nonumber\\
	\red
	& (\outputp{x}{y} | @{y})\substn{\quotep{(\prefix{x}{y}{(@{y} | \outputp{x}{y})) | P}}}{y} & \nonumber\\
	=
	& \outputp{x}{\quotep{(\prefix{x}{y}{(\outputp{x}{y} | @{y})) | P}}}
	  | {(\prefix{x}{y}{(\outputp{x}{y} | @{y})) | P}} & \nonumber\\
	\red
	& \ldots & \nonumber\\
	\red^*
	& P | P | \ldots & \nonumber
\end{eqnarray}

Of course, this encoding, as an implementation, runs away, unfolding
$\bangp{P}$ eagerly. A lazier and more implementable replication
operator, restricted to input-guarded processes, may be obtained as follows.

\begin{eqnarray}
\bangp{\prefix{u}{v}{P}} 
	:= 
	\binpar{\lift{x}{\prefix{u}{v}{(\binpar{D(x)}{P})}}}{D(x)} \nonumber
\end{eqnarray}

\begin{remark}
  Note that the lazier definition still does not deal with summation
  or mixed summation (i.e. sums over input and output). The reader is
  invited to construct definitions of replication that deal with these
  features. 

  Further, the definitions are parameterized in a name, $x$. Can you,
  gentle reader, make a definition that eliminates this parameter and
  guarantees no accidental interaction between the replication
  machinery and the process being replicated -- i.e. no accidental
  sharing of names used by the process to get its work done and the
  name(s) used by the replication to effect copying. This latter
  revision of the definition of replication is crucial to obtaining
  the expected identity $!!P \sim !P$.
\end{remark}

\begin{remark}\label{rem:paradoxical_combinator}
  The reader familiar with the lambda calculus will have noticed the
  similarity between $D$ and the paradoxical combinator.

  [Ed. note: the existence of this seems to suggest we have to be more
  restrictive on the set of processes and names we admit if we are to
  support no-cloning.]
\end{remark}

\subsubsection{Bisimulation}

The computational dynamics gives rise to another kind of equivalence,
the equivalence of computational behavior. As previously mentioned
this is typically captured \emph{via} some form of bisimulation.

% The notion we use in this paper is weak barbed bisimulation
% \cite{milner91polyadicpi}.

The notion we use in this paper is derived from weak barbed
bisimulation \cite{milner91polyadicpi}. 

\begin{definition}
An \emph{observation relation}, $\downarrow_{\mathcal N}$, over a set
of names, $\mathcal N$, is the smallest relation satisfying the rules
below.

\infrule[Out-barb]{y \in {\mathcal N}, \; x \nameeq y}
		  {\outputp{x}{v} \downarrow_{\mathcal N} x}
\infrule[Par-barb]{\mbox{$P\downarrow_{\mathcal N} x$ or $Q\downarrow_{\mathcal N} x$}}
		  {\binpar{P}{Q} \downarrow_{\mathcal N} x}

We write $P \Downarrow_{\mathcal N} x$ if there is $Q$ such that 
$P \wred Q$ and $Q \downarrow_{\mathcal N} x$.
\end{definition}

\begin{definition}
%\label{def.bbisim}
An  ${\mathcal N}$-\emph{barbed bisimulation} over a set of names, ${\mathcal N}$, is a symmetric binary relation 
${\mathcal S}_{\mathcal N}$ between agents such that $P\rel{S}_{\mathcal N}Q$ implies:
\begin{enumerate}
\item If $P \red P'$ then $Q \wred Q'$ and $P'\rel{S}_{\mathcal N} Q'$.
\item If $P\downarrow_{\mathcal N} x$, then $Q\Downarrow_{\mathcal N} x$.
\end{enumerate}
$P$ is ${\mathcal N}$-barbed bisimilar to $Q$, written
$P \wbbisim_{\mathcal N} Q$, if $P \rel{S}_{\mathcal N} Q$ for some ${\mathcal N}$-barbed bisimulation ${\mathcal S}_{\mathcal N}$.
\end{definition}

$\mathcal{R} \subseteq \pi \times \pi$

$P \mathcal{R} Q => \forall P'. P \red P' \Rightarrow \exists Q'. Q \red Q', P' \mathcal{R} Q'$

$P \vdash x \Rightarrow Q \vdash x$

\begin{mathpar}
  \inferrule*[lab=Out-barb]{x \nameeq y}{{y}!\langle{Q}\rangle \vdash x}
  \and
  \inferrule*[lab=Par-barb]{\mbox{$P\vdash x$ or $Q\vdash x$}}{\binpar{P}{Q} \vdash x}
\end{mathpar}

\subsubsection{Contexts}

One of the principle advantages of computational calculi like the
$\pi$-calculus is a well-defined notion of context,
contextual-equivalence and a correlation between
contextual-equivalence and notions of bisimulation. The notion of
context allows the decomposition of a process into (sub-)process and
its syntactic environment, its context. Thus, a context may be
thought of as a process with a ``hole'' (written $\Box$) in it. The
application of a context $M$ to a process $P$, written $M[P]$, is
tantamount to filling the hole in $M$ with $P$. In this paper we do
not need the full weight of this theory, but do make use of the notion
of context in the proof the main theorem. 

\begin{mathpar}
  \inferrule* [lab=summation] {} {{M_{M},M_{N}} \bc \Box \;|\; x.M_{A} \;|\; M_{M}+M_{N}}
  \and
  \inferrule* [lab=agent] {} {{M_{A}} \bc (\vec{x})M_{P} \;| \; \clift{P_0,\ldots,M_{P},\ldots,P_N}}
  \and \\
  \inferrule* [lab=process] {} {{M_{P}} \bc M_{N} \;| \;P|M_{P} }
\end{mathpar} 

\begin{mathpar}
  \inferrule* [lab=sychronization] {} {M_{N} \bc \Box \;|\; x?M_{F} \;|\; x!M_{C}}
  \and
  \inferrule* [lab=abstraction] {} {{M_{F}} \bc (x)M_{P} }
  \and
  \inferrule* [lab=concretion] {} {{M_{C}} \bc \langle M_{P} \rangle }
  \and \\
  \inferrule* [lab=process] {} {{M_{P}} \bc M_{N} \;| \;P|M_{P} }
\end{mathpar}

\begin{definition}[contextual application] Given a context $M$, and
  process $P$, we define the \emph{contextual application}, $M[P] :=
  M\{P/\Box\}$. That is, the contextual application of M to P is the
  substitution of $P$ for $\Box$ in $M$.
\end{definition}

$\meaningof{-} : L \to \mathcal{P}(\pi)$

\begin{mathpar}
  \inferrule* [lab=collection] {} {\meaningof{true} = \pi, \and \meaningof{~E} = \pi \setminus \meaningof{E}, \and \meaningof{E_{1} \& E_{2}} = \meaningof{E_{1}} \cap \meaningof{E_{2}}}
\end{mathpar}

\begin{mathpar}
  \inferrule* [lab=structure] {} {\meaningof{0} = \{ P \in \pi | P \equiv 0 \}, \and \\ \meaningof{E_1 | E_2} = \{ P \in \pi | P \equiv P_{1} | P_{2}, P_{1} \in \meaningof{E_{1}}, P_{2} \in \meaningof{E_2}\} }
\end{mathpar}

\begin{mathpar}
 \inferrule* [lab=behavior] {} {\meaningof{\langle a?b \rangle E} = \{ P \in \pi | P \equiv Q | u?(y)P', \\ \and \\\\ \and \\ \;\;\; u \in \meaningof{a}, \forall z.P'\{z/y\} \in \meaningof{E\{z/b\}}\}, \and \\ \meaningof{a!E} = \{ P \in \pi | P \equiv Q | x!\langle P' \rangle, x \in \meaningof{a} P' \in \meaningof{E}\} }
\end{mathpar}

\begin{mathpar}
 \inferrule* [lab=nominal] {} {\meaningof{\quotep{E}} = \{ \quotep{P} \in \quotep{\pi} | P \in \meaningof{E} \}, \and \meaningof{\quotep{P}} = \{ \quotep{Q} \in \quotep{\pi} | P \equiv Q \} \and \\ \meaningof{@\quotep{E}} = \{ P \in \pi | P \equiv @x, x \in \meaningof{E} \}}
\end{mathpar}

\begin{eqnarray*}
  \\
  \meaningof{-} : TS \to ST
\end{eqnarray*}

\begin{eqnarray*}
  \\
  L : TS \to ST
\end{eqnarray*}

\begin{eqnarray*}
  \\
  P \models E \iff P \in \meaningof{E}
\end{eqnarray*}

\begin{eqnarray*}
  P \approx_{L} Q \iff \forall E \in L. P \models E \iff Q \models E
\end{eqnarray*}

\begin{eqnarray*}
  P \approx_{K} Q
\end{eqnarray*}

\begin{eqnarray*}
  P \approx Q
\end{eqnarray*}

$\approx_{K} = \approx = \approx_{L}$

\subsubsection{Contextual duality}

Note that contexts extend the quotation operation to a family of
operations from processes to names. Given a context, $M$, we can
define a \emph{nominal context}, $\quotep{M}$ by $\quotep{M}[P] :=
\quotep{M[P]}$. To foreshadow what is to come we observe that these
operations enjoy a duality with processes very much like the duality
between vectors and maps from vectors to scalars.

Further, because the calculus is essentially higher-order, we have a
correspondence between contexts and processes. More specifically,
given a name $x$ and a context $M$ we can construct $M^{*}_{x}$ such
that 

\begin{mathpar}
  M^{*}_{x} | \lift{x}{P} \red M[P]
\end{mathpar}

namely,

\begin{mathpar}
  M^{*}_{x} := x?(u).M[\dropn{u}]
\end{mathpar}

The dependence of $M^{*}_{x}$ on a name makes it an abstraction, 

\begin{mathpar}
  M^{*} := (x)x?(u).M[\dropn{u}]
\end{mathpar}

\subsection{Additional notation}

It will sometimes be convenient to denote the process a name
quotes. We already have the notation $x = \quotep{P}$, but it will be
convenient to introduce an alternate notation, $\procn{x}$, when we
want to emphasize the connection to the use of the name. Note that, by
virtue of name equivalence, $\quotep{\procn{x}} \nameeq x$; so, the
notation is consistent with previous definitions.

Further, because names have structure it is possible to effect
substitutions on the basis of that structure. This means we need to
upgrade our notation for substitutions, which we accomplish by
adapting comprehension notation. Thus,

\begin{mathpar}
  P\{ y / x : x \in S \}
\end{mathpar}

is interpreted to mean the process derived from P by replacing (in a
capture-avoiding manner) each occurrence of $x$ in $S$ by $y$. For example,

\begin{mathpar}
  P\{ \quotep{\procn{x}|\procn{x}} / x : x \in \freenames{P} \}
\end{mathpar}

will replace each (occurrence) of a free name $x$ in $P$ by
$\quotep{\procn{x}|\procn{x}}$.

Also, we will avail ourselves of the notation $x^{L}$ and $x^{R}$ to
denote injections of a name into disjoint copies of the name
space. There are numerous ways to accomplish this. One example can be
found in \cite{MeredithR05}. This notation overloads to vectors of
names: $\vec{x}^{\pi} := (x_{i}^{\pi} \; : \; 0 \leq i < |\vec{x}| )$ where $\pi \in \{L,R\}$.

We also use $P^{\Box} := P|\Box$.

In \cite{MeredithR05} an interpretation of the new operator is
given. It turns out that there are several possible interpretations
all enjoying the requisite algebraic properties of the operator (see
\cite{milner91polyadicpi}). We will therefore make liberal use of
$(\nu\; \vec{x})P$.

% subsection the_syntax_and_semantics_of_the_notation_system (end)   

\input{qm2pi.qmops} 

\input{qm2pi.sterngerlach} 

\input{qm2pi.metric} 

% section concurrent_process_calculi (end)

%\input{qm2pi.proofsketch}

% section proof sketch (end)

%\input{qm2pi.slviaknots} 

% section spatial logic via knots (end)

\input{qm2pi.conclusion}

% section conclusion (end)

%\input{qm2pi.dtcodes} 

% section wiring algorithm (end)

\input{qm2pi.ack} 

% section acknowledgments (end)

\newpage


\bibliographystyle{plain}   
\bibliography{../../biblios/main.bib}

\input{qm2pi.rhodetails}

\end{document}



\end{document}

 

%\ifpdf
%\usepackage[pdftex]{graphicx}
%\else
%\usepackage{graphicx}
%\fi

 % \ifpdf
%  \usepackage{pdfsync}
%  \if


%\title{Brief Article}
%\author{David F. Snyder}
%\author{L.G. Meredith}

%\address{Dept. of Math., Texas State University--San Marcos, San Marcos, TX 78666}
       
\pagestyle{empty}


\begin{document}

\lstset{language=[Objective]Caml,frame=shadowbox}

\documentclass[12pt]{llncs}
%\documentclass{jktr}

\usepackage[pdftex]{hyperref}                   
\usepackage {listings}
\usepackage {mathpartir}
\usepackage{bcprules}
%\usepackage{listings}
                       
\usepackage{graphicx} 
%\usepackage[margins=2.5cm,nohead,nofoot]{geometry}
%\usepackage{geometry}
\usepackage{amsfonts}
\usepackage{amstext}
\usepackage{latexsym}
\usepackage{amssymb}
\usepackage{color}


%\include{myPreamble}
\documentclass[12pt]{llncs}
%\documentclass{jktr}

\usepackage[pdftex]{hyperref}                   
\usepackage {listings}
\usepackage {mathpartir}
\usepackage{bcprules}
%\usepackage{listings}
                       
\usepackage{graphicx} 
%\usepackage[margins=2.5cm,nohead,nofoot]{geometry}
%\usepackage{geometry}
\usepackage{amsfonts}
\usepackage{amstext}
\usepackage{latexsym}
\usepackage{amssymb}
\usepackage{color}


%\include{myPreamble}
\include{qm2pi.local} 

%\ifpdf
%\usepackage[pdftex]{graphicx}
%\else
%\usepackage{graphicx}
%\fi

 % \ifpdf
%  \usepackage{pdfsync}
%  \if


%\title{Brief Article}
%\author{David F. Snyder}
%\author{L.G. Meredith}

%\address{Dept. of Math., Texas State University--San Marcos, San Marcos, TX 78666}
       
\pagestyle{empty}


\begin{document}

\lstset{language=[Objective]Caml,frame=shadowbox}

\input{qm2pi.front}

% section front matter (end)

\input{qm2pi.intro} 
 
% section introduction (end)

% \input{qm2pi.knotations} 

% section notation (end)

\input{qm2pi.process.calculi} 

% section concurrent_process_calculi_and_spatial_logics_ (end)
    
%\input{qm2pi.knots2pi} 

%\input{qm2pi.trefoil} 

%\input{qm2pi.mainthm} 

% subsection basic_interpretation (end)

%\input{qm2pi.rho.presentation} 
\subsection{The syntax and semantics of the notation system}\label{sub:the_syntax_and_semantics_of_the_notation_system} % (fold)

We now summarize a technical presentation of the calculus that
embodies our theory of dynamics. The typical presentation of such a
calculus follows the style of giving generators and relations on
them. The grammar, below, describing term constructors, freely
generates the set of processes, $\Proc$. This set is then quotiented
by a relation known as structural congruence and it is over this set
that the notion of dynamics is expressed. This presentation is
essentially that of \cite{MeredithR05} with the addition of
polyadicity and summation. For readability we have relegated some of
the technical subtleties to an appendix.

\subsubsection{Process grammar}\label{subsub:process_grammar}

\begin{mathpar}
  \inferrule* [lab=synchronization] {} {{M} \bc \pzero \;|\; x?F \;|\; x!C }
  \and
  \inferrule* [lab=abstraction] {} {{F} \bc (x)P}
  \and
  \inferrule* [lab=concretion] {} {{C} \bc \langle Q \rangle}
  \and
  \inferrule* [lab=process] {} {{P,Q} \bc M \;| \;P|Q \;|\; @{x}}
  \and
  \inferrule* [lab=name] {} {{x} \bc \quotep{P}}
\end{mathpar} 

Note that $\vec{x}$ (resp. $\vec{P}$) denotes a vector of names
(resp. processes) of length $|\vec{x}|$ (resp. $|\vec{P}|$). We adopt
the following useful abbreviations.

\begin{mathpar}
   x?(\vec{y}).P := x.(\vec{y})P \and  x\clift{\vec{P}} := x.\clift{\vec{P}}
   \and x!(y) := \lift{x}{\dropn{y}}
   \and \Pi_{i=0}^{n-1}P_i := P_0 | \ldots | P_{n-1}
\end{mathpar}

\subsubsection{Structural congruence}

\paragraph{Free and bound names and alpha-equivalence.} At the
core of structural equivalence is alpha-equivalence which identifies
process that are the same up to a change of variable. Formally, we
recognize the distinction between free and bound names. The free names
of a process, $\freenames{P}$, may be calculated recursively as
follows:

\begin{mathpar}
\freenames{\pzero} := \emptyset
  \and \\
  \freenames{x?(y).P} := \{ x \} \cup (\freenames{P} \setminus \{ y \})
  \and 
  \freenames{x!\langle P \rangle} := \{ x \} \cup \{ P \} 
  \and \\
  \freenames{P|Q} := \freenames{P} \cup \freenames{Q}
  \and \\
  \freenames{@{x}} := \{ x \}
\end{mathpar}

$\pi$
$\quotep{\pi}$

$\freenames{-} : \pi \to \mathcal{P}(\quotep{\pi})$

\begin{eqnarray*}
  \freenames{\pzero} & := & \emptyset \\
  \freenames{x?(y).P} & := & \{ x \} \cup (\freenames{P} \setminus \{ y \}) \\
  \freenames{x!\langle P \rangle} & := & \{ x \} \cup \{ P \} \\
  \freenames{P|Q} & := & \freenames{P} \cup \freenames{Q} \\
  \freenames{\dropn{x}} & := & \{ x \}
\end{eqnarray*}

The bound names of a process, $\boundnames{P}$, are those names occurring in $P$
that are not free. For example, in $x?(y).0$, the name $x$ is free, while $y$ is bound.

\begin{mathpar}
  \inferrule* [lab=monoidal-laws] {} { P|Q \equiv Q|P \and P|0 \equiv P \and P|(Q|R) \equiv (P|Q)|R }
\end{mathpar}

\begin{mathpar}
  \inferrule* [lab=alpha-equivalence] {} { (x)P \equiv (y)P\{y/x\} \and y \not\in \freenames{P} }
\end{mathpar}

\begin{definition}
Then two processes, $P,Q$, are alpha-equivalent if $P = Q\{\vec{y}/\vec{x}\}$ for
some $\vec{x} \in \boundnames{Q},\vec{y} \in \boundnames{P}$, where $Q\{\vec{y}/\vec{x}\}$
denotes the capture-avoiding substitution of $\vec{y}$ for $\vec{x}$ in $Q$.
\end{definition}

\begin{definition}
  The {\em structural congruence} \cite{SangiorgiWalker} , $\equiv$,
  between processes is the least congruence containing
  alpha-equivalence, satisfying the abelian monoid laws
  (associativity, commutativity and $\pzero$ as identity) for parallel
  composition $|$ and for summation $+$.
\end{definition}

\subsection{Name equivalence}

We take name equivalence, written $\nameeq$, to be the smallest
equivalence relation generated by the following rules.

\begin{mathpar}
\inferrule*[lab=Quote-drop]
{ }
{ \quotep{@{x}} \nameeq x }

\inferrule*[lab=Struct-equiv]
{ P \scong Q }
{ \quotep{P} \nameeq \quotep{Q} }
\end{mathpar}

The astute reader will have noticed that the mutual recursion of names
and processes imposes a mutual recursion on alpha-equivalence and
structural equivalence via name-equivalence. Fortunately, all of this
works out pleasantly and we may calculate in the natural way, free of
concern. The reader interested in the details is referred to the
appendix \ref{appendix:rho_details}.

\subsection{Substitution}

We use $\Proc$ for the set of processes, $\QProc$ for the set of
names, and $\id{\{}\vec{y} / \vec{x} \id{\}}$ to denote partial maps,
$s : \QProc \rightarrow \QProc$. A map, $s$ lifts, uniquely, to a map
on process terms, $\widehat{s} : \Proc \rightarrow \Proc$ by the
following equations.

\begin{mathpar}
  (0) \psubstp{Q}{P} := 0 \\
  (R \juxtap S) \psubstp{Q}{P}
  :=    
  (R)\psubstp{Q}{P} \juxtap (S) \psubstp{Q}{P} \\
  (x?(y).R) \psubstp{Q}{P}    
  :=    
  (x)\substp{Q}{P} (z)\concat( (R \psubstn{z}{y}) \psubstp{Q}{P} ) \\
  (\lift{x}{R}) \psubstp{Q}{P}  
  :=
  \lift{(x)\substp{Q}{P}}{ R \psubstp{Q}{P} } \\
%   (\dropn{x})  \psubstp{Q}{P}       
%   := 
%   \left\{ 
%     \begin{array}{ccc} 
%       \dropn{\quotep{Q}} & & x \nameeq \quotep{P} \\
%       \dropn{x} & & otherwise \\
%     \end{array}
%   \right. 
  (\dropn{x})  \psubstp{Q}{P}       
  := 
  \left\{ 
    \begin{array}{ccc} 
      Q & & x \nameeq \quotep{P} \\
      \dropn{x} & & otherwise \\
    \end{array}
  \right.
\end{mathpar}
 

where

\begin{eqnarray}
  (x)\id{\{} \lpquote Q \rpquote / \lpquote P \rpquote \id{\}}            = 
  \left\{ 
    \begin{array}{ccc}
      \lpquote Q \rpquote & & x \nameeq \lpquote P \rpquote \\
      x & & otherwise \\
    \end{array}
  \right. \nonumber
\end{eqnarray}

and $z$ is chosen distinct from $\quotep{P}$, $\quotep{Q}$, the free
names in $Q$, and all the names in $R$. Our $\alpha$-equivalence will
be built in the standard way from this substitution.

\begin{remark}\label{rem:no_self_referential_names}
  One consequence of these definitions is that $\forall P. \quotep{P}
  \not\in \freenames{P}$.
\end{remark}

\subsection{ Dynamic quote: an example }

Anticipating something of what's to come, consider applying the
substitution, $\widehat{\id{\{}u / z \id{\}}}$, to the following pair
of processes, $\lift{w}{y!(z)}$ and $w[ \lpquote y!(z) \rpquote ]$.

\begin{eqnarray}
	\lift{w}{y!(z)}\widehat{\id{\{}u / z \id{\}}}
		& = &
		\lift{w}{y!(u)} \nonumber\\
	w[ \lpquote y!(z) \rpquote ] \widehat{ \id{\{}u / z \id{\}} }
		& = &
		w[ \lpquote y!(z) \rpquote ] \nonumber
\end{eqnarray}

Because the body of the process between quotes is impervious to
substitution, we get radically different answers. In fact, by
examining the first process in an input context,
e.g. $x?(z).\lift{w}{y!(z)}$, we see that the process under the lift
operator may be shaped by prefixed inputs binding a name inside it. In
this sense, the lift operator will be seen as a way to dynamically
construct processes before reifying them as names.

Finally equipped with these standard features we can present the
dynamics of the calculus.

\subsubsection{Operational semantics} 

Finally, we introduce the computational dynamics. What marks these
algebras as distinct from other more traditionally studied algebraic
structures, e.g. vector spaces or polynomial rings, is the manner in
which dynamics is captured. In traditional structures, dynamics is typically
expressed through morphisms between such structures, as in linear maps
between vector spaces or morphisms between rings. In algebras
associated with the semantics of computation, the dynamics is
expressed as part of the algebraic structure itself, through a
reduction reduction relation typically denoted by $\red$. Below, we
give a recursive presentation of this relation for the calculus used
in the encoding.

$\red \subseteq \pi \times \pi$
$\red : \pi \to \mathcal{P}(\pi)$

\begin{mathpar}
  \inferrule* [lab=Comm] { \textsf{match}( x_{src}, x_{trgt} ) } { x_{trgt}?(y)P \; | \; x_{src}!\langle {Q} \rangle \red P\{\quotep{Q}/y}\} }
  \and \\
  \inferrule* [lab=Par] {{P} \red {P}'} {{{P} | {Q}} \red {{P}' | {Q}}}
  \and
  \inferrule* [lab=Equiv]{{{P} \scong {P}'} \andalso {{P}' \red {Q}'} \andalso {{Q}' \scong {Q}}}{{P} \red {Q}}
\end{mathpar}

\begin{eqnarray*}
  match_{\equiv} (\quotep{P},\quotep{Q}) & := & P \equiv Q \\
  match_{\dagger}(\quotep{P},\quotep{Q}) & := & \forall R. P|Q \red^{*} R => R \red^{*} 0 \\
  match_{K}(\quotep{P},\quotep{Q}) & := & K \mbox{ for some context } K
\end{eqnarray*}

$u?(x)P | u!\langle Q \rangle \red P\{\quotep{Q}/x\}$

%We write $\wred$ for $\red^*$, and $P\red$ if $\exists Q $ such that $ P \red Q$.
We write $P\red$ if $\exists Q $ such that $ P \red Q$ and $P\not\red$, otherwise.

\section{Replication}

As mentioned before, it is known that replication (and hence
recursion) can be implemented in a higher-order process algebra
\cite{SangiorgiWalker}. As our first example of calculation with the
machinery thus far presented we give the construction explicitly in
the {\rhoc}.

\begin{eqnarray}
	D_{x} & := & \prefix{x}{y}{(\binpar{\outputp{x}{y}}{@{y}})} \nonumber\\
	\bangp_{x}{P} & := & \binpar{{x}!\langle{\binpar{D_{x}}{P}}\rangle}{D_{x}} \nonumber
\end{eqnarray}

\begin{eqnarray}
	\bangp_{x}{P} & & \nonumber\\
	=
	& {x}!\langle{(\prefix{x}{y}{(\outputp{x}{y} | @{y})) | P}}\rangle 
	      | \prefix{x}{y}{(\outputp{x}{y} | @{y})} & \nonumber\\
	\red
	& (\outputp{x}{y} | @{y})\substn{\quotep{(\prefix{x}{y}{(@{y} | \outputp{x}{y})) | P}}}{y} & \nonumber\\
	=
	& \outputp{x}{\quotep{(\prefix{x}{y}{(\outputp{x}{y} | @{y})) | P}}}
	  | {(\prefix{x}{y}{(\outputp{x}{y} | @{y})) | P}} & \nonumber\\
	\red
	& \ldots & \nonumber\\
	\red^*
	& P | P | \ldots & \nonumber
\end{eqnarray}

Of course, this encoding, as an implementation, runs away, unfolding
$\bangp{P}$ eagerly. A lazier and more implementable replication
operator, restricted to input-guarded processes, may be obtained as follows.

\begin{eqnarray}
\bangp{\prefix{u}{v}{P}} 
	:= 
	\binpar{\lift{x}{\prefix{u}{v}{(\binpar{D(x)}{P})}}}{D(x)} \nonumber
\end{eqnarray}

\begin{remark}
  Note that the lazier definition still does not deal with summation
  or mixed summation (i.e. sums over input and output). The reader is
  invited to construct definitions of replication that deal with these
  features. 

  Further, the definitions are parameterized in a name, $x$. Can you,
  gentle reader, make a definition that eliminates this parameter and
  guarantees no accidental interaction between the replication
  machinery and the process being replicated -- i.e. no accidental
  sharing of names used by the process to get its work done and the
  name(s) used by the replication to effect copying. This latter
  revision of the definition of replication is crucial to obtaining
  the expected identity $!!P \sim !P$.
\end{remark}

\begin{remark}\label{rem:paradoxical_combinator}
  The reader familiar with the lambda calculus will have noticed the
  similarity between $D$ and the paradoxical combinator.

  [Ed. note: the existence of this seems to suggest we have to be more
  restrictive on the set of processes and names we admit if we are to
  support no-cloning.]
\end{remark}

\subsubsection{Bisimulation}

The computational dynamics gives rise to another kind of equivalence,
the equivalence of computational behavior. As previously mentioned
this is typically captured \emph{via} some form of bisimulation.

% The notion we use in this paper is weak barbed bisimulation
% \cite{milner91polyadicpi}.

The notion we use in this paper is derived from weak barbed
bisimulation \cite{milner91polyadicpi}. 

\begin{definition}
An \emph{observation relation}, $\downarrow_{\mathcal N}$, over a set
of names, $\mathcal N$, is the smallest relation satisfying the rules
below.

\infrule[Out-barb]{y \in {\mathcal N}, \; x \nameeq y}
		  {\outputp{x}{v} \downarrow_{\mathcal N} x}
\infrule[Par-barb]{\mbox{$P\downarrow_{\mathcal N} x$ or $Q\downarrow_{\mathcal N} x$}}
		  {\binpar{P}{Q} \downarrow_{\mathcal N} x}

We write $P \Downarrow_{\mathcal N} x$ if there is $Q$ such that 
$P \wred Q$ and $Q \downarrow_{\mathcal N} x$.
\end{definition}

\begin{definition}
%\label{def.bbisim}
An  ${\mathcal N}$-\emph{barbed bisimulation} over a set of names, ${\mathcal N}$, is a symmetric binary relation 
${\mathcal S}_{\mathcal N}$ between agents such that $P\rel{S}_{\mathcal N}Q$ implies:
\begin{enumerate}
\item If $P \red P'$ then $Q \wred Q'$ and $P'\rel{S}_{\mathcal N} Q'$.
\item If $P\downarrow_{\mathcal N} x$, then $Q\Downarrow_{\mathcal N} x$.
\end{enumerate}
$P$ is ${\mathcal N}$-barbed bisimilar to $Q$, written
$P \wbbisim_{\mathcal N} Q$, if $P \rel{S}_{\mathcal N} Q$ for some ${\mathcal N}$-barbed bisimulation ${\mathcal S}_{\mathcal N}$.
\end{definition}

$\mathcal{R} \subseteq \pi \times \pi$

$P \mathcal{R} Q => \forall P'. P \red P' \Rightarrow \exists Q'. Q \red Q', P' \mathcal{R} Q'$

$P \vdash x \Rightarrow Q \vdash x$

\begin{mathpar}
  \inferrule*[lab=Out-barb]{x \nameeq y}{{y}!\langle{Q}\rangle \vdash x}
  \and
  \inferrule*[lab=Par-barb]{\mbox{$P\vdash x$ or $Q\vdash x$}}{\binpar{P}{Q} \vdash x}
\end{mathpar}

\subsubsection{Contexts}

One of the principle advantages of computational calculi like the
$\pi$-calculus is a well-defined notion of context,
contextual-equivalence and a correlation between
contextual-equivalence and notions of bisimulation. The notion of
context allows the decomposition of a process into (sub-)process and
its syntactic environment, its context. Thus, a context may be
thought of as a process with a ``hole'' (written $\Box$) in it. The
application of a context $M$ to a process $P$, written $M[P]$, is
tantamount to filling the hole in $M$ with $P$. In this paper we do
not need the full weight of this theory, but do make use of the notion
of context in the proof the main theorem. 

\begin{mathpar}
  \inferrule* [lab=summation] {} {{M_{M},M_{N}} \bc \Box \;|\; x.M_{A} \;|\; M_{M}+M_{N}}
  \and
  \inferrule* [lab=agent] {} {{M_{A}} \bc (\vec{x})M_{P} \;| \; \clift{P_0,\ldots,M_{P},\ldots,P_N}}
  \and \\
  \inferrule* [lab=process] {} {{M_{P}} \bc M_{N} \;| \;P|M_{P} }
\end{mathpar} 

\begin{mathpar}
  \inferrule* [lab=sychronization] {} {M_{N} \bc \Box \;|\; x?M_{F} \;|\; x!M_{C}}
  \and
  \inferrule* [lab=abstraction] {} {{M_{F}} \bc (x)M_{P} }
  \and
  \inferrule* [lab=concretion] {} {{M_{C}} \bc \langle M_{P} \rangle }
  \and \\
  \inferrule* [lab=process] {} {{M_{P}} \bc M_{N} \;| \;P|M_{P} }
\end{mathpar}

\begin{definition}[contextual application] Given a context $M$, and
  process $P$, we define the \emph{contextual application}, $M[P] :=
  M\{P/\Box\}$. That is, the contextual application of M to P is the
  substitution of $P$ for $\Box$ in $M$.
\end{definition}

$\meaningof{-} : L \to \mathcal{P}(\pi)$

\begin{mathpar}
  \inferrule* [lab=collection] {} {\meaningof{true} = \pi, \and \meaningof{~E} = \pi \setminus \meaningof{E}, \and \meaningof{E_{1} \& E_{2}} = \meaningof{E_{1}} \cap \meaningof{E_{2}}}
\end{mathpar}

\begin{mathpar}
  \inferrule* [lab=structure] {} {\meaningof{0} = \{ P \in \pi | P \equiv 0 \}, \and \\ \meaningof{E_1 | E_2} = \{ P \in \pi | P \equiv P_{1} | P_{2}, P_{1} \in \meaningof{E_{1}}, P_{2} \in \meaningof{E_2}\} }
\end{mathpar}

\begin{mathpar}
 \inferrule* [lab=behavior] {} {\meaningof{\langle a?b \rangle E} = \{ P \in \pi | P \equiv Q | u?(y)P', \\ \and \\\\ \and \\ \;\;\; u \in \meaningof{a}, \forall z.P'\{z/y\} \in \meaningof{E\{z/b\}}\}, \and \\ \meaningof{a!E} = \{ P \in \pi | P \equiv Q | x!\langle P' \rangle, x \in \meaningof{a} P' \in \meaningof{E}\} }
\end{mathpar}

\begin{mathpar}
 \inferrule* [lab=nominal] {} {\meaningof{\quotep{E}} = \{ \quotep{P} \in \quotep{\pi} | P \in \meaningof{E} \}, \and \meaningof{\quotep{P}} = \{ \quotep{Q} \in \quotep{\pi} | P \equiv Q \} \and \\ \meaningof{@\quotep{E}} = \{ P \in \pi | P \equiv @x, x \in \meaningof{E} \}}
\end{mathpar}

\begin{eqnarray*}
  \\
  \meaningof{-} : TS \to ST
\end{eqnarray*}

\begin{eqnarray*}
  \\
  L : TS \to ST
\end{eqnarray*}

\begin{eqnarray*}
  \\
  P \models E \iff P \in \meaningof{E}
\end{eqnarray*}

\begin{eqnarray*}
  P \approx_{L} Q \iff \forall E \in L. P \models E \iff Q \models E
\end{eqnarray*}

\begin{eqnarray*}
  P \approx_{K} Q
\end{eqnarray*}

\begin{eqnarray*}
  P \approx Q
\end{eqnarray*}

$\approx_{K} = \approx = \approx_{L}$

\subsubsection{Contextual duality}

Note that contexts extend the quotation operation to a family of
operations from processes to names. Given a context, $M$, we can
define a \emph{nominal context}, $\quotep{M}$ by $\quotep{M}[P] :=
\quotep{M[P]}$. To foreshadow what is to come we observe that these
operations enjoy a duality with processes very much like the duality
between vectors and maps from vectors to scalars.

Further, because the calculus is essentially higher-order, we have a
correspondence between contexts and processes. More specifically,
given a name $x$ and a context $M$ we can construct $M^{*}_{x}$ such
that 

\begin{mathpar}
  M^{*}_{x} | \lift{x}{P} \red M[P]
\end{mathpar}

namely,

\begin{mathpar}
  M^{*}_{x} := x?(u).M[\dropn{u}]
\end{mathpar}

The dependence of $M^{*}_{x}$ on a name makes it an abstraction, 

\begin{mathpar}
  M^{*} := (x)x?(u).M[\dropn{u}]
\end{mathpar}

\subsection{Additional notation}

It will sometimes be convenient to denote the process a name
quotes. We already have the notation $x = \quotep{P}$, but it will be
convenient to introduce an alternate notation, $\procn{x}$, when we
want to emphasize the connection to the use of the name. Note that, by
virtue of name equivalence, $\quotep{\procn{x}} \nameeq x$; so, the
notation is consistent with previous definitions.

Further, because names have structure it is possible to effect
substitutions on the basis of that structure. This means we need to
upgrade our notation for substitutions, which we accomplish by
adapting comprehension notation. Thus,

\begin{mathpar}
  P\{ y / x : x \in S \}
\end{mathpar}

is interpreted to mean the process derived from P by replacing (in a
capture-avoiding manner) each occurrence of $x$ in $S$ by $y$. For example,

\begin{mathpar}
  P\{ \quotep{\procn{x}|\procn{x}} / x : x \in \freenames{P} \}
\end{mathpar}

will replace each (occurrence) of a free name $x$ in $P$ by
$\quotep{\procn{x}|\procn{x}}$.

Also, we will avail ourselves of the notation $x^{L}$ and $x^{R}$ to
denote injections of a name into disjoint copies of the name
space. There are numerous ways to accomplish this. One example can be
found in \cite{MeredithR05}. This notation overloads to vectors of
names: $\vec{x}^{\pi} := (x_{i}^{\pi} \; : \; 0 \leq i < |\vec{x}| )$ where $\pi \in \{L,R\}$.

We also use $P^{\Box} := P|\Box$.

In \cite{MeredithR05} an interpretation of the new operator is
given. It turns out that there are several possible interpretations
all enjoying the requisite algebraic properties of the operator (see
\cite{milner91polyadicpi}). We will therefore make liberal use of
$(\nu\; \vec{x})P$.

% subsection the_syntax_and_semantics_of_the_notation_system (end)   

\input{qm2pi.qmops} 

\input{qm2pi.sterngerlach} 

\input{qm2pi.metric} 

% section concurrent_process_calculi (end)

%\input{qm2pi.proofsketch}

% section proof sketch (end)

%\input{qm2pi.slviaknots} 

% section spatial logic via knots (end)

\input{qm2pi.conclusion}

% section conclusion (end)

%\input{qm2pi.dtcodes} 

% section wiring algorithm (end)

\input{qm2pi.ack} 

% section acknowledgments (end)

\newpage


\bibliographystyle{plain}   
\bibliography{../../biblios/main.bib}

\input{qm2pi.rhodetails}

\end{document}

 

%\ifpdf
%\usepackage[pdftex]{graphicx}
%\else
%\usepackage{graphicx}
%\fi

 % \ifpdf
%  \usepackage{pdfsync}
%  \if


%\title{Brief Article}
%\author{David F. Snyder}
%\author{L.G. Meredith}

%\address{Dept. of Math., Texas State University--San Marcos, San Marcos, TX 78666}
       
\pagestyle{empty}


\begin{document}

\lstset{language=[Objective]Caml,frame=shadowbox}

\documentclass[12pt]{llncs}
%\documentclass{jktr}

\usepackage[pdftex]{hyperref}                   
\usepackage {listings}
\usepackage {mathpartir}
\usepackage{bcprules}
%\usepackage{listings}
                       
\usepackage{graphicx} 
%\usepackage[margins=2.5cm,nohead,nofoot]{geometry}
%\usepackage{geometry}
\usepackage{amsfonts}
\usepackage{amstext}
\usepackage{latexsym}
\usepackage{amssymb}
\usepackage{color}


%\include{myPreamble}
\include{qm2pi.local} 

%\ifpdf
%\usepackage[pdftex]{graphicx}
%\else
%\usepackage{graphicx}
%\fi

 % \ifpdf
%  \usepackage{pdfsync}
%  \if


%\title{Brief Article}
%\author{David F. Snyder}
%\author{L.G. Meredith}

%\address{Dept. of Math., Texas State University--San Marcos, San Marcos, TX 78666}
       
\pagestyle{empty}


\begin{document}

\lstset{language=[Objective]Caml,frame=shadowbox}

\input{qm2pi.front}

% section front matter (end)

\input{qm2pi.intro} 
 
% section introduction (end)

% \input{qm2pi.knotations} 

% section notation (end)

\input{qm2pi.process.calculi} 

% section concurrent_process_calculi_and_spatial_logics_ (end)
    
%\input{qm2pi.knots2pi} 

%\input{qm2pi.trefoil} 

%\input{qm2pi.mainthm} 

% subsection basic_interpretation (end)

%\input{qm2pi.rho.presentation} 
\subsection{The syntax and semantics of the notation system}\label{sub:the_syntax_and_semantics_of_the_notation_system} % (fold)

We now summarize a technical presentation of the calculus that
embodies our theory of dynamics. The typical presentation of such a
calculus follows the style of giving generators and relations on
them. The grammar, below, describing term constructors, freely
generates the set of processes, $\Proc$. This set is then quotiented
by a relation known as structural congruence and it is over this set
that the notion of dynamics is expressed. This presentation is
essentially that of \cite{MeredithR05} with the addition of
polyadicity and summation. For readability we have relegated some of
the technical subtleties to an appendix.

\subsubsection{Process grammar}\label{subsub:process_grammar}

\begin{mathpar}
  \inferrule* [lab=synchronization] {} {{M} \bc \pzero \;|\; x?F \;|\; x!C }
  \and
  \inferrule* [lab=abstraction] {} {{F} \bc (x)P}
  \and
  \inferrule* [lab=concretion] {} {{C} \bc \langle Q \rangle}
  \and
  \inferrule* [lab=process] {} {{P,Q} \bc M \;| \;P|Q \;|\; @{x}}
  \and
  \inferrule* [lab=name] {} {{x} \bc \quotep{P}}
\end{mathpar} 

Note that $\vec{x}$ (resp. $\vec{P}$) denotes a vector of names
(resp. processes) of length $|\vec{x}|$ (resp. $|\vec{P}|$). We adopt
the following useful abbreviations.

\begin{mathpar}
   x?(\vec{y}).P := x.(\vec{y})P \and  x\clift{\vec{P}} := x.\clift{\vec{P}}
   \and x!(y) := \lift{x}{\dropn{y}}
   \and \Pi_{i=0}^{n-1}P_i := P_0 | \ldots | P_{n-1}
\end{mathpar}

\subsubsection{Structural congruence}

\paragraph{Free and bound names and alpha-equivalence.} At the
core of structural equivalence is alpha-equivalence which identifies
process that are the same up to a change of variable. Formally, we
recognize the distinction between free and bound names. The free names
of a process, $\freenames{P}$, may be calculated recursively as
follows:

\begin{mathpar}
\freenames{\pzero} := \emptyset
  \and \\
  \freenames{x?(y).P} := \{ x \} \cup (\freenames{P} \setminus \{ y \})
  \and 
  \freenames{x!\langle P \rangle} := \{ x \} \cup \{ P \} 
  \and \\
  \freenames{P|Q} := \freenames{P} \cup \freenames{Q}
  \and \\
  \freenames{@{x}} := \{ x \}
\end{mathpar}

$\pi$
$\quotep{\pi}$

$\freenames{-} : \pi \to \mathcal{P}(\quotep{\pi})$

\begin{eqnarray*}
  \freenames{\pzero} & := & \emptyset \\
  \freenames{x?(y).P} & := & \{ x \} \cup (\freenames{P} \setminus \{ y \}) \\
  \freenames{x!\langle P \rangle} & := & \{ x \} \cup \{ P \} \\
  \freenames{P|Q} & := & \freenames{P} \cup \freenames{Q} \\
  \freenames{\dropn{x}} & := & \{ x \}
\end{eqnarray*}

The bound names of a process, $\boundnames{P}$, are those names occurring in $P$
that are not free. For example, in $x?(y).0$, the name $x$ is free, while $y$ is bound.

\begin{mathpar}
  \inferrule* [lab=monoidal-laws] {} { P|Q \equiv Q|P \and P|0 \equiv P \and P|(Q|R) \equiv (P|Q)|R }
\end{mathpar}

\begin{mathpar}
  \inferrule* [lab=alpha-equivalence] {} { (x)P \equiv (y)P\{y/x\} \and y \not\in \freenames{P} }
\end{mathpar}

\begin{definition}
Then two processes, $P,Q$, are alpha-equivalent if $P = Q\{\vec{y}/\vec{x}\}$ for
some $\vec{x} \in \boundnames{Q},\vec{y} \in \boundnames{P}$, where $Q\{\vec{y}/\vec{x}\}$
denotes the capture-avoiding substitution of $\vec{y}$ for $\vec{x}$ in $Q$.
\end{definition}

\begin{definition}
  The {\em structural congruence} \cite{SangiorgiWalker} , $\equiv$,
  between processes is the least congruence containing
  alpha-equivalence, satisfying the abelian monoid laws
  (associativity, commutativity and $\pzero$ as identity) for parallel
  composition $|$ and for summation $+$.
\end{definition}

\subsection{Name equivalence}

We take name equivalence, written $\nameeq$, to be the smallest
equivalence relation generated by the following rules.

\begin{mathpar}
\inferrule*[lab=Quote-drop]
{ }
{ \quotep{@{x}} \nameeq x }

\inferrule*[lab=Struct-equiv]
{ P \scong Q }
{ \quotep{P} \nameeq \quotep{Q} }
\end{mathpar}

The astute reader will have noticed that the mutual recursion of names
and processes imposes a mutual recursion on alpha-equivalence and
structural equivalence via name-equivalence. Fortunately, all of this
works out pleasantly and we may calculate in the natural way, free of
concern. The reader interested in the details is referred to the
appendix \ref{appendix:rho_details}.

\subsection{Substitution}

We use $\Proc$ for the set of processes, $\QProc$ for the set of
names, and $\id{\{}\vec{y} / \vec{x} \id{\}}$ to denote partial maps,
$s : \QProc \rightarrow \QProc$. A map, $s$ lifts, uniquely, to a map
on process terms, $\widehat{s} : \Proc \rightarrow \Proc$ by the
following equations.

\begin{mathpar}
  (0) \psubstp{Q}{P} := 0 \\
  (R \juxtap S) \psubstp{Q}{P}
  :=    
  (R)\psubstp{Q}{P} \juxtap (S) \psubstp{Q}{P} \\
  (x?(y).R) \psubstp{Q}{P}    
  :=    
  (x)\substp{Q}{P} (z)\concat( (R \psubstn{z}{y}) \psubstp{Q}{P} ) \\
  (\lift{x}{R}) \psubstp{Q}{P}  
  :=
  \lift{(x)\substp{Q}{P}}{ R \psubstp{Q}{P} } \\
%   (\dropn{x})  \psubstp{Q}{P}       
%   := 
%   \left\{ 
%     \begin{array}{ccc} 
%       \dropn{\quotep{Q}} & & x \nameeq \quotep{P} \\
%       \dropn{x} & & otherwise \\
%     \end{array}
%   \right. 
  (\dropn{x})  \psubstp{Q}{P}       
  := 
  \left\{ 
    \begin{array}{ccc} 
      Q & & x \nameeq \quotep{P} \\
      \dropn{x} & & otherwise \\
    \end{array}
  \right.
\end{mathpar}
 

where

\begin{eqnarray}
  (x)\id{\{} \lpquote Q \rpquote / \lpquote P \rpquote \id{\}}            = 
  \left\{ 
    \begin{array}{ccc}
      \lpquote Q \rpquote & & x \nameeq \lpquote P \rpquote \\
      x & & otherwise \\
    \end{array}
  \right. \nonumber
\end{eqnarray}

and $z$ is chosen distinct from $\quotep{P}$, $\quotep{Q}$, the free
names in $Q$, and all the names in $R$. Our $\alpha$-equivalence will
be built in the standard way from this substitution.

\begin{remark}\label{rem:no_self_referential_names}
  One consequence of these definitions is that $\forall P. \quotep{P}
  \not\in \freenames{P}$.
\end{remark}

\subsection{ Dynamic quote: an example }

Anticipating something of what's to come, consider applying the
substitution, $\widehat{\id{\{}u / z \id{\}}}$, to the following pair
of processes, $\lift{w}{y!(z)}$ and $w[ \lpquote y!(z) \rpquote ]$.

\begin{eqnarray}
	\lift{w}{y!(z)}\widehat{\id{\{}u / z \id{\}}}
		& = &
		\lift{w}{y!(u)} \nonumber\\
	w[ \lpquote y!(z) \rpquote ] \widehat{ \id{\{}u / z \id{\}} }
		& = &
		w[ \lpquote y!(z) \rpquote ] \nonumber
\end{eqnarray}

Because the body of the process between quotes is impervious to
substitution, we get radically different answers. In fact, by
examining the first process in an input context,
e.g. $x?(z).\lift{w}{y!(z)}$, we see that the process under the lift
operator may be shaped by prefixed inputs binding a name inside it. In
this sense, the lift operator will be seen as a way to dynamically
construct processes before reifying them as names.

Finally equipped with these standard features we can present the
dynamics of the calculus.

\subsubsection{Operational semantics} 

Finally, we introduce the computational dynamics. What marks these
algebras as distinct from other more traditionally studied algebraic
structures, e.g. vector spaces or polynomial rings, is the manner in
which dynamics is captured. In traditional structures, dynamics is typically
expressed through morphisms between such structures, as in linear maps
between vector spaces or morphisms between rings. In algebras
associated with the semantics of computation, the dynamics is
expressed as part of the algebraic structure itself, through a
reduction reduction relation typically denoted by $\red$. Below, we
give a recursive presentation of this relation for the calculus used
in the encoding.

$\red \subseteq \pi \times \pi$
$\red : \pi \to \mathcal{P}(\pi)$

\begin{mathpar}
  \inferrule* [lab=Comm] { \textsf{match}( x_{src}, x_{trgt} ) } { x_{trgt}?(y)P \; | \; x_{src}!\langle {Q} \rangle \red P\{\quotep{Q}/y}\} }
  \and \\
  \inferrule* [lab=Par] {{P} \red {P}'} {{{P} | {Q}} \red {{P}' | {Q}}}
  \and
  \inferrule* [lab=Equiv]{{{P} \scong {P}'} \andalso {{P}' \red {Q}'} \andalso {{Q}' \scong {Q}}}{{P} \red {Q}}
\end{mathpar}

\begin{eqnarray*}
  match_{\equiv} (\quotep{P},\quotep{Q}) & := & P \equiv Q \\
  match_{\dagger}(\quotep{P},\quotep{Q}) & := & \forall R. P|Q \red^{*} R => R \red^{*} 0 \\
  match_{K}(\quotep{P},\quotep{Q}) & := & K \mbox{ for some context } K
\end{eqnarray*}

$u?(x)P | u!\langle Q \rangle \red P\{\quotep{Q}/x\}$

%We write $\wred$ for $\red^*$, and $P\red$ if $\exists Q $ such that $ P \red Q$.
We write $P\red$ if $\exists Q $ such that $ P \red Q$ and $P\not\red$, otherwise.

\section{Replication}

As mentioned before, it is known that replication (and hence
recursion) can be implemented in a higher-order process algebra
\cite{SangiorgiWalker}. As our first example of calculation with the
machinery thus far presented we give the construction explicitly in
the {\rhoc}.

\begin{eqnarray}
	D_{x} & := & \prefix{x}{y}{(\binpar{\outputp{x}{y}}{@{y}})} \nonumber\\
	\bangp_{x}{P} & := & \binpar{{x}!\langle{\binpar{D_{x}}{P}}\rangle}{D_{x}} \nonumber
\end{eqnarray}

\begin{eqnarray}
	\bangp_{x}{P} & & \nonumber\\
	=
	& {x}!\langle{(\prefix{x}{y}{(\outputp{x}{y} | @{y})) | P}}\rangle 
	      | \prefix{x}{y}{(\outputp{x}{y} | @{y})} & \nonumber\\
	\red
	& (\outputp{x}{y} | @{y})\substn{\quotep{(\prefix{x}{y}{(@{y} | \outputp{x}{y})) | P}}}{y} & \nonumber\\
	=
	& \outputp{x}{\quotep{(\prefix{x}{y}{(\outputp{x}{y} | @{y})) | P}}}
	  | {(\prefix{x}{y}{(\outputp{x}{y} | @{y})) | P}} & \nonumber\\
	\red
	& \ldots & \nonumber\\
	\red^*
	& P | P | \ldots & \nonumber
\end{eqnarray}

Of course, this encoding, as an implementation, runs away, unfolding
$\bangp{P}$ eagerly. A lazier and more implementable replication
operator, restricted to input-guarded processes, may be obtained as follows.

\begin{eqnarray}
\bangp{\prefix{u}{v}{P}} 
	:= 
	\binpar{\lift{x}{\prefix{u}{v}{(\binpar{D(x)}{P})}}}{D(x)} \nonumber
\end{eqnarray}

\begin{remark}
  Note that the lazier definition still does not deal with summation
  or mixed summation (i.e. sums over input and output). The reader is
  invited to construct definitions of replication that deal with these
  features. 

  Further, the definitions are parameterized in a name, $x$. Can you,
  gentle reader, make a definition that eliminates this parameter and
  guarantees no accidental interaction between the replication
  machinery and the process being replicated -- i.e. no accidental
  sharing of names used by the process to get its work done and the
  name(s) used by the replication to effect copying. This latter
  revision of the definition of replication is crucial to obtaining
  the expected identity $!!P \sim !P$.
\end{remark}

\begin{remark}\label{rem:paradoxical_combinator}
  The reader familiar with the lambda calculus will have noticed the
  similarity between $D$ and the paradoxical combinator.

  [Ed. note: the existence of this seems to suggest we have to be more
  restrictive on the set of processes and names we admit if we are to
  support no-cloning.]
\end{remark}

\subsubsection{Bisimulation}

The computational dynamics gives rise to another kind of equivalence,
the equivalence of computational behavior. As previously mentioned
this is typically captured \emph{via} some form of bisimulation.

% The notion we use in this paper is weak barbed bisimulation
% \cite{milner91polyadicpi}.

The notion we use in this paper is derived from weak barbed
bisimulation \cite{milner91polyadicpi}. 

\begin{definition}
An \emph{observation relation}, $\downarrow_{\mathcal N}$, over a set
of names, $\mathcal N$, is the smallest relation satisfying the rules
below.

\infrule[Out-barb]{y \in {\mathcal N}, \; x \nameeq y}
		  {\outputp{x}{v} \downarrow_{\mathcal N} x}
\infrule[Par-barb]{\mbox{$P\downarrow_{\mathcal N} x$ or $Q\downarrow_{\mathcal N} x$}}
		  {\binpar{P}{Q} \downarrow_{\mathcal N} x}

We write $P \Downarrow_{\mathcal N} x$ if there is $Q$ such that 
$P \wred Q$ and $Q \downarrow_{\mathcal N} x$.
\end{definition}

\begin{definition}
%\label{def.bbisim}
An  ${\mathcal N}$-\emph{barbed bisimulation} over a set of names, ${\mathcal N}$, is a symmetric binary relation 
${\mathcal S}_{\mathcal N}$ between agents such that $P\rel{S}_{\mathcal N}Q$ implies:
\begin{enumerate}
\item If $P \red P'$ then $Q \wred Q'$ and $P'\rel{S}_{\mathcal N} Q'$.
\item If $P\downarrow_{\mathcal N} x$, then $Q\Downarrow_{\mathcal N} x$.
\end{enumerate}
$P$ is ${\mathcal N}$-barbed bisimilar to $Q$, written
$P \wbbisim_{\mathcal N} Q$, if $P \rel{S}_{\mathcal N} Q$ for some ${\mathcal N}$-barbed bisimulation ${\mathcal S}_{\mathcal N}$.
\end{definition}

$\mathcal{R} \subseteq \pi \times \pi$

$P \mathcal{R} Q => \forall P'. P \red P' \Rightarrow \exists Q'. Q \red Q', P' \mathcal{R} Q'$

$P \vdash x \Rightarrow Q \vdash x$

\begin{mathpar}
  \inferrule*[lab=Out-barb]{x \nameeq y}{{y}!\langle{Q}\rangle \vdash x}
  \and
  \inferrule*[lab=Par-barb]{\mbox{$P\vdash x$ or $Q\vdash x$}}{\binpar{P}{Q} \vdash x}
\end{mathpar}

\subsubsection{Contexts}

One of the principle advantages of computational calculi like the
$\pi$-calculus is a well-defined notion of context,
contextual-equivalence and a correlation between
contextual-equivalence and notions of bisimulation. The notion of
context allows the decomposition of a process into (sub-)process and
its syntactic environment, its context. Thus, a context may be
thought of as a process with a ``hole'' (written $\Box$) in it. The
application of a context $M$ to a process $P$, written $M[P]$, is
tantamount to filling the hole in $M$ with $P$. In this paper we do
not need the full weight of this theory, but do make use of the notion
of context in the proof the main theorem. 

\begin{mathpar}
  \inferrule* [lab=summation] {} {{M_{M},M_{N}} \bc \Box \;|\; x.M_{A} \;|\; M_{M}+M_{N}}
  \and
  \inferrule* [lab=agent] {} {{M_{A}} \bc (\vec{x})M_{P} \;| \; \clift{P_0,\ldots,M_{P},\ldots,P_N}}
  \and \\
  \inferrule* [lab=process] {} {{M_{P}} \bc M_{N} \;| \;P|M_{P} }
\end{mathpar} 

\begin{mathpar}
  \inferrule* [lab=sychronization] {} {M_{N} \bc \Box \;|\; x?M_{F} \;|\; x!M_{C}}
  \and
  \inferrule* [lab=abstraction] {} {{M_{F}} \bc (x)M_{P} }
  \and
  \inferrule* [lab=concretion] {} {{M_{C}} \bc \langle M_{P} \rangle }
  \and \\
  \inferrule* [lab=process] {} {{M_{P}} \bc M_{N} \;| \;P|M_{P} }
\end{mathpar}

\begin{definition}[contextual application] Given a context $M$, and
  process $P$, we define the \emph{contextual application}, $M[P] :=
  M\{P/\Box\}$. That is, the contextual application of M to P is the
  substitution of $P$ for $\Box$ in $M$.
\end{definition}

$\meaningof{-} : L \to \mathcal{P}(\pi)$

\begin{mathpar}
  \inferrule* [lab=collection] {} {\meaningof{true} = \pi, \and \meaningof{~E} = \pi \setminus \meaningof{E}, \and \meaningof{E_{1} \& E_{2}} = \meaningof{E_{1}} \cap \meaningof{E_{2}}}
\end{mathpar}

\begin{mathpar}
  \inferrule* [lab=structure] {} {\meaningof{0} = \{ P \in \pi | P \equiv 0 \}, \and \\ \meaningof{E_1 | E_2} = \{ P \in \pi | P \equiv P_{1} | P_{2}, P_{1} \in \meaningof{E_{1}}, P_{2} \in \meaningof{E_2}\} }
\end{mathpar}

\begin{mathpar}
 \inferrule* [lab=behavior] {} {\meaningof{\langle a?b \rangle E} = \{ P \in \pi | P \equiv Q | u?(y)P', \\ \and \\\\ \and \\ \;\;\; u \in \meaningof{a}, \forall z.P'\{z/y\} \in \meaningof{E\{z/b\}}\}, \and \\ \meaningof{a!E} = \{ P \in \pi | P \equiv Q | x!\langle P' \rangle, x \in \meaningof{a} P' \in \meaningof{E}\} }
\end{mathpar}

\begin{mathpar}
 \inferrule* [lab=nominal] {} {\meaningof{\quotep{E}} = \{ \quotep{P} \in \quotep{\pi} | P \in \meaningof{E} \}, \and \meaningof{\quotep{P}} = \{ \quotep{Q} \in \quotep{\pi} | P \equiv Q \} \and \\ \meaningof{@\quotep{E}} = \{ P \in \pi | P \equiv @x, x \in \meaningof{E} \}}
\end{mathpar}

\begin{eqnarray*}
  \\
  \meaningof{-} : TS \to ST
\end{eqnarray*}

\begin{eqnarray*}
  \\
  L : TS \to ST
\end{eqnarray*}

\begin{eqnarray*}
  \\
  P \models E \iff P \in \meaningof{E}
\end{eqnarray*}

\begin{eqnarray*}
  P \approx_{L} Q \iff \forall E \in L. P \models E \iff Q \models E
\end{eqnarray*}

\begin{eqnarray*}
  P \approx_{K} Q
\end{eqnarray*}

\begin{eqnarray*}
  P \approx Q
\end{eqnarray*}

$\approx_{K} = \approx = \approx_{L}$

\subsubsection{Contextual duality}

Note that contexts extend the quotation operation to a family of
operations from processes to names. Given a context, $M$, we can
define a \emph{nominal context}, $\quotep{M}$ by $\quotep{M}[P] :=
\quotep{M[P]}$. To foreshadow what is to come we observe that these
operations enjoy a duality with processes very much like the duality
between vectors and maps from vectors to scalars.

Further, because the calculus is essentially higher-order, we have a
correspondence between contexts and processes. More specifically,
given a name $x$ and a context $M$ we can construct $M^{*}_{x}$ such
that 

\begin{mathpar}
  M^{*}_{x} | \lift{x}{P} \red M[P]
\end{mathpar}

namely,

\begin{mathpar}
  M^{*}_{x} := x?(u).M[\dropn{u}]
\end{mathpar}

The dependence of $M^{*}_{x}$ on a name makes it an abstraction, 

\begin{mathpar}
  M^{*} := (x)x?(u).M[\dropn{u}]
\end{mathpar}

\subsection{Additional notation}

It will sometimes be convenient to denote the process a name
quotes. We already have the notation $x = \quotep{P}$, but it will be
convenient to introduce an alternate notation, $\procn{x}$, when we
want to emphasize the connection to the use of the name. Note that, by
virtue of name equivalence, $\quotep{\procn{x}} \nameeq x$; so, the
notation is consistent with previous definitions.

Further, because names have structure it is possible to effect
substitutions on the basis of that structure. This means we need to
upgrade our notation for substitutions, which we accomplish by
adapting comprehension notation. Thus,

\begin{mathpar}
  P\{ y / x : x \in S \}
\end{mathpar}

is interpreted to mean the process derived from P by replacing (in a
capture-avoiding manner) each occurrence of $x$ in $S$ by $y$. For example,

\begin{mathpar}
  P\{ \quotep{\procn{x}|\procn{x}} / x : x \in \freenames{P} \}
\end{mathpar}

will replace each (occurrence) of a free name $x$ in $P$ by
$\quotep{\procn{x}|\procn{x}}$.

Also, we will avail ourselves of the notation $x^{L}$ and $x^{R}$ to
denote injections of a name into disjoint copies of the name
space. There are numerous ways to accomplish this. One example can be
found in \cite{MeredithR05}. This notation overloads to vectors of
names: $\vec{x}^{\pi} := (x_{i}^{\pi} \; : \; 0 \leq i < |\vec{x}| )$ where $\pi \in \{L,R\}$.

We also use $P^{\Box} := P|\Box$.

In \cite{MeredithR05} an interpretation of the new operator is
given. It turns out that there are several possible interpretations
all enjoying the requisite algebraic properties of the operator (see
\cite{milner91polyadicpi}). We will therefore make liberal use of
$(\nu\; \vec{x})P$.

% subsection the_syntax_and_semantics_of_the_notation_system (end)   

\input{qm2pi.qmops} 

\input{qm2pi.sterngerlach} 

\input{qm2pi.metric} 

% section concurrent_process_calculi (end)

%\input{qm2pi.proofsketch}

% section proof sketch (end)

%\input{qm2pi.slviaknots} 

% section spatial logic via knots (end)

\input{qm2pi.conclusion}

% section conclusion (end)

%\input{qm2pi.dtcodes} 

% section wiring algorithm (end)

\input{qm2pi.ack} 

% section acknowledgments (end)

\newpage


\bibliographystyle{plain}   
\bibliography{../../biblios/main.bib}

\input{qm2pi.rhodetails}

\end{document}



% section front matter (end)

\section{Introduction}\label{sec:introduction} % (fold)
In this draft of the material i am going to have to dispense with the
usual writing conventions adopted in papers on these topics. i'm going
to have adopt whatever tone i need at the time i'm writing up the
calculations. Sometimes this may be very conversational; others it may
be the barest mathematical grunts; others still it may be that i have
lifted text from one of my other papers because the exposition of some
point was better said there. i hope that my readers are not unduly put
out by this decision. i'm not doing this to flout convention or be
rebellious. i find these calculations very technically challenging. To
keep everything going technically, something has to give; i have to
let go of some cognitive burden. So, the academic writing style --
with all of its trade-offs in terms of facilitating technical
communication -- is what i'm letting go of. Perhaps subsequent drafts
can be tightened and polished, but for now, i'm going to speak as if
we were sitting together in a coffee shop with a laptop, wifi and a
pad of paper and a pencil.

So, here's what i have to say. We -- you and i, comfortably ensconced
in our coffee shop and well-equipped with our tools -- can realize and
carry out the calculations of quantum mechanics over a very different
formal theory of dynamics, a formal theory of dynamics that
corresponds to a theory of concurrent computation with
\emph{reflection}. It has the advantage that the underlying theory is
already `quantized', but supports analogues all of the continuuous
operations. Strikingly, this underlying theory has recently been
connected with a notion of metric that we can show, by calculating
together, coincides with the metric induced by the inner product.

There are a lot of reasons why you might be interested in seeing
calculations of this form. Here's why i'm interested. For the past
several centuries there has been no competitor to the ``Newtonian''
account of dynamics. As a result the predominant share of accounts of
dynamical systems and situations have had to be formulated in terms of
the Newtonian machinery. i view this as an intellectually dangerous
position to occupy. Everything, despite it's intrinsic shape, turns
into a nail to be hit with this hammer. Recently, however, the theory
of computation has matured to the point where we have candidates for
theories of dynamics that offer very different perspective on
reasoning about dynamical systems and situations. Testing these
candidates against very successful accounts of dynamical situations,
like quantum mechanics, is going to give us some sense of how mature
they are and some measure of the quality of these accounts of
dynamics.

\subsection{Summary of contributions and outline of paper}

So, we're going to develop an interpretation of the operations of
quantum mechanics normally interpreted by Hilbert spaces and
operators. We're going to do this over a theory of computation. Note
that this is very different than the usual quantum computation program
which develops notions of computation over quantum mechanics. Rather,
we are developing a story that aligns with Wheeler's slogan: It from
Bit. To do this we will first provide an account of the theory of
computation at play here. Then we will dive into a calculation-driven
interpretation of the operations of quantum mechanics.

The reason we take this approach is that -- until very recently --
there hasn't been an axiomatic account of quantum mechanics. As a
result there has been no sharp delineation of the mathematical theory
supporting interpretation of the physical theory and the physical
theory, itself. So, ambient features of the maths are free to be
exploited (or supressed) without a real accounting of their physical
relevance. There is no sharp statement ``here's the physical theory''
qua \emph{theory} and ``here's the mathematical interpretation''
enabling a judgment of how faithful the interpretation is -- apart
from experimental observation. When there is an axiomatic account we
can judge how well a given mathematical formalism supports an
interpretation of the axioms, independent of
experimentation. Likewise, we can judge how well we have captured our
physical evidence and experience with our axiomatics, independent of
any specific mathematical implementation, with accidental detail that
may or may not have physical significance. 

In lieu of a fully fleshed out and vetted axiomatic account of quantum
mechanics, interpreting the operational notions in service of modeling
physical systems will have to suffice. In other words, we are not in
the business of providing a model of Hilbert spaces and operators. We
are in the business of providing a model of quantum mechanics because
we are motivated by testing our notions of dynamics against physical
theory; and, the predictive calculations of the physical theory must
serve as the best formulation -- shy of a fully fleshed out axiomatic
account -- of the physical theory itself (as they have for scientific
theories since time immemorial). Put another way, despite a
whole-hearted commitment to an It-from-Bit ontology, we are firmly
aligned with the shut-up-and-calculate camp as the best way to obtain
results either from the physical perspective or as a quality assurance
measure of our fledgling theory of dynamics.

In detail, we present a reflective process calculus. Then we develop
intuitive correspondences between the notions available in this
calculus and the usual physical notions supporting quantum mechanical
calculations. Thus, 

\begin{table}[htp]
  \center{
    \fbox{
      \begin{tabular}{c|c}
        quantum mechanics & process calculus \\
        \hline
        scalar & name \\
        state vector & process \\
        dual & contextual duals \\
        matrix & formal sums of process-context-dual pairs \\
        orthogonality & process annihilation \\
        inner product & execution-formula + quoting
      \end{tabular}
    }
  }
  \caption{QM - process calculi correspondences}
\end{table}

Then we tighten up these intuitions to operational definitions. We
employ the Dirac notation as the best proxy we can find for an
abstract syntax of the quantum mechanical notions. The definitions we
develop put us in contact with equational constraints coming from the
theory that we demonstrate the definitions and calculations satisfy.

This puts us in a position to shut up and calculate for the
Stern-Gerlach experimental set up, showing how these predictive
calculations become calculations on processes in our theory of a
reflective process calculus.

Penultimately, we demonstrate that the notion of metric coming from
the inner product coincides with the notion of metric available from
the theory of bisimulation. This demonstration gives us the right to
think of space as arising from behavior. Finally, we consider where we
might go from the new vantage point we have obtained.

% section introduction (end) 
 
% section introduction (end)

% \documentclass[12pt]{llncs}
%\documentclass{jktr}

\usepackage[pdftex]{hyperref}                   
\usepackage {listings}
\usepackage {mathpartir}
\usepackage{bcprules}
%\usepackage{listings}
                       
\usepackage{graphicx} 
%\usepackage[margins=2.5cm,nohead,nofoot]{geometry}
%\usepackage{geometry}
\usepackage{amsfonts}
\usepackage{amstext}
\usepackage{latexsym}
\usepackage{amssymb}
\usepackage{color}


%\include{myPreamble}
\include{qm2pi.local} 

%\ifpdf
%\usepackage[pdftex]{graphicx}
%\else
%\usepackage{graphicx}
%\fi

 % \ifpdf
%  \usepackage{pdfsync}
%  \if


%\title{Brief Article}
%\author{David F. Snyder}
%\author{L.G. Meredith}

%\address{Dept. of Math., Texas State University--San Marcos, San Marcos, TX 78666}
       
\pagestyle{empty}


\begin{document}

\lstset{language=[Objective]Caml,frame=shadowbox}

\input{qm2pi.front}

% section front matter (end)

\input{qm2pi.intro} 
 
% section introduction (end)

% \input{qm2pi.knotations} 

% section notation (end)

\input{qm2pi.process.calculi} 

% section concurrent_process_calculi_and_spatial_logics_ (end)
    
%\input{qm2pi.knots2pi} 

%\input{qm2pi.trefoil} 

%\input{qm2pi.mainthm} 

% subsection basic_interpretation (end)

%\input{qm2pi.rho.presentation} 
\subsection{The syntax and semantics of the notation system}\label{sub:the_syntax_and_semantics_of_the_notation_system} % (fold)

We now summarize a technical presentation of the calculus that
embodies our theory of dynamics. The typical presentation of such a
calculus follows the style of giving generators and relations on
them. The grammar, below, describing term constructors, freely
generates the set of processes, $\Proc$. This set is then quotiented
by a relation known as structural congruence and it is over this set
that the notion of dynamics is expressed. This presentation is
essentially that of \cite{MeredithR05} with the addition of
polyadicity and summation. For readability we have relegated some of
the technical subtleties to an appendix.

\subsubsection{Process grammar}\label{subsub:process_grammar}

\begin{mathpar}
  \inferrule* [lab=synchronization] {} {{M} \bc \pzero \;|\; x?F \;|\; x!C }
  \and
  \inferrule* [lab=abstraction] {} {{F} \bc (x)P}
  \and
  \inferrule* [lab=concretion] {} {{C} \bc \langle Q \rangle}
  \and
  \inferrule* [lab=process] {} {{P,Q} \bc M \;| \;P|Q \;|\; @{x}}
  \and
  \inferrule* [lab=name] {} {{x} \bc \quotep{P}}
\end{mathpar} 

Note that $\vec{x}$ (resp. $\vec{P}$) denotes a vector of names
(resp. processes) of length $|\vec{x}|$ (resp. $|\vec{P}|$). We adopt
the following useful abbreviations.

\begin{mathpar}
   x?(\vec{y}).P := x.(\vec{y})P \and  x\clift{\vec{P}} := x.\clift{\vec{P}}
   \and x!(y) := \lift{x}{\dropn{y}}
   \and \Pi_{i=0}^{n-1}P_i := P_0 | \ldots | P_{n-1}
\end{mathpar}

\subsubsection{Structural congruence}

\paragraph{Free and bound names and alpha-equivalence.} At the
core of structural equivalence is alpha-equivalence which identifies
process that are the same up to a change of variable. Formally, we
recognize the distinction between free and bound names. The free names
of a process, $\freenames{P}$, may be calculated recursively as
follows:

\begin{mathpar}
\freenames{\pzero} := \emptyset
  \and \\
  \freenames{x?(y).P} := \{ x \} \cup (\freenames{P} \setminus \{ y \})
  \and 
  \freenames{x!\langle P \rangle} := \{ x \} \cup \{ P \} 
  \and \\
  \freenames{P|Q} := \freenames{P} \cup \freenames{Q}
  \and \\
  \freenames{@{x}} := \{ x \}
\end{mathpar}

$\pi$
$\quotep{\pi}$

$\freenames{-} : \pi \to \mathcal{P}(\quotep{\pi})$

\begin{eqnarray*}
  \freenames{\pzero} & := & \emptyset \\
  \freenames{x?(y).P} & := & \{ x \} \cup (\freenames{P} \setminus \{ y \}) \\
  \freenames{x!\langle P \rangle} & := & \{ x \} \cup \{ P \} \\
  \freenames{P|Q} & := & \freenames{P} \cup \freenames{Q} \\
  \freenames{\dropn{x}} & := & \{ x \}
\end{eqnarray*}

The bound names of a process, $\boundnames{P}$, are those names occurring in $P$
that are not free. For example, in $x?(y).0$, the name $x$ is free, while $y$ is bound.

\begin{mathpar}
  \inferrule* [lab=monoidal-laws] {} { P|Q \equiv Q|P \and P|0 \equiv P \and P|(Q|R) \equiv (P|Q)|R }
\end{mathpar}

\begin{mathpar}
  \inferrule* [lab=alpha-equivalence] {} { (x)P \equiv (y)P\{y/x\} \and y \not\in \freenames{P} }
\end{mathpar}

\begin{definition}
Then two processes, $P,Q$, are alpha-equivalent if $P = Q\{\vec{y}/\vec{x}\}$ for
some $\vec{x} \in \boundnames{Q},\vec{y} \in \boundnames{P}$, where $Q\{\vec{y}/\vec{x}\}$
denotes the capture-avoiding substitution of $\vec{y}$ for $\vec{x}$ in $Q$.
\end{definition}

\begin{definition}
  The {\em structural congruence} \cite{SangiorgiWalker} , $\equiv$,
  between processes is the least congruence containing
  alpha-equivalence, satisfying the abelian monoid laws
  (associativity, commutativity and $\pzero$ as identity) for parallel
  composition $|$ and for summation $+$.
\end{definition}

\subsection{Name equivalence}

We take name equivalence, written $\nameeq$, to be the smallest
equivalence relation generated by the following rules.

\begin{mathpar}
\inferrule*[lab=Quote-drop]
{ }
{ \quotep{@{x}} \nameeq x }

\inferrule*[lab=Struct-equiv]
{ P \scong Q }
{ \quotep{P} \nameeq \quotep{Q} }
\end{mathpar}

The astute reader will have noticed that the mutual recursion of names
and processes imposes a mutual recursion on alpha-equivalence and
structural equivalence via name-equivalence. Fortunately, all of this
works out pleasantly and we may calculate in the natural way, free of
concern. The reader interested in the details is referred to the
appendix \ref{appendix:rho_details}.

\subsection{Substitution}

We use $\Proc$ for the set of processes, $\QProc$ for the set of
names, and $\id{\{}\vec{y} / \vec{x} \id{\}}$ to denote partial maps,
$s : \QProc \rightarrow \QProc$. A map, $s$ lifts, uniquely, to a map
on process terms, $\widehat{s} : \Proc \rightarrow \Proc$ by the
following equations.

\begin{mathpar}
  (0) \psubstp{Q}{P} := 0 \\
  (R \juxtap S) \psubstp{Q}{P}
  :=    
  (R)\psubstp{Q}{P} \juxtap (S) \psubstp{Q}{P} \\
  (x?(y).R) \psubstp{Q}{P}    
  :=    
  (x)\substp{Q}{P} (z)\concat( (R \psubstn{z}{y}) \psubstp{Q}{P} ) \\
  (\lift{x}{R}) \psubstp{Q}{P}  
  :=
  \lift{(x)\substp{Q}{P}}{ R \psubstp{Q}{P} } \\
%   (\dropn{x})  \psubstp{Q}{P}       
%   := 
%   \left\{ 
%     \begin{array}{ccc} 
%       \dropn{\quotep{Q}} & & x \nameeq \quotep{P} \\
%       \dropn{x} & & otherwise \\
%     \end{array}
%   \right. 
  (\dropn{x})  \psubstp{Q}{P}       
  := 
  \left\{ 
    \begin{array}{ccc} 
      Q & & x \nameeq \quotep{P} \\
      \dropn{x} & & otherwise \\
    \end{array}
  \right.
\end{mathpar}
 

where

\begin{eqnarray}
  (x)\id{\{} \lpquote Q \rpquote / \lpquote P \rpquote \id{\}}            = 
  \left\{ 
    \begin{array}{ccc}
      \lpquote Q \rpquote & & x \nameeq \lpquote P \rpquote \\
      x & & otherwise \\
    \end{array}
  \right. \nonumber
\end{eqnarray}

and $z$ is chosen distinct from $\quotep{P}$, $\quotep{Q}$, the free
names in $Q$, and all the names in $R$. Our $\alpha$-equivalence will
be built in the standard way from this substitution.

\begin{remark}\label{rem:no_self_referential_names}
  One consequence of these definitions is that $\forall P. \quotep{P}
  \not\in \freenames{P}$.
\end{remark}

\subsection{ Dynamic quote: an example }

Anticipating something of what's to come, consider applying the
substitution, $\widehat{\id{\{}u / z \id{\}}}$, to the following pair
of processes, $\lift{w}{y!(z)}$ and $w[ \lpquote y!(z) \rpquote ]$.

\begin{eqnarray}
	\lift{w}{y!(z)}\widehat{\id{\{}u / z \id{\}}}
		& = &
		\lift{w}{y!(u)} \nonumber\\
	w[ \lpquote y!(z) \rpquote ] \widehat{ \id{\{}u / z \id{\}} }
		& = &
		w[ \lpquote y!(z) \rpquote ] \nonumber
\end{eqnarray}

Because the body of the process between quotes is impervious to
substitution, we get radically different answers. In fact, by
examining the first process in an input context,
e.g. $x?(z).\lift{w}{y!(z)}$, we see that the process under the lift
operator may be shaped by prefixed inputs binding a name inside it. In
this sense, the lift operator will be seen as a way to dynamically
construct processes before reifying them as names.

Finally equipped with these standard features we can present the
dynamics of the calculus.

\subsubsection{Operational semantics} 

Finally, we introduce the computational dynamics. What marks these
algebras as distinct from other more traditionally studied algebraic
structures, e.g. vector spaces or polynomial rings, is the manner in
which dynamics is captured. In traditional structures, dynamics is typically
expressed through morphisms between such structures, as in linear maps
between vector spaces or morphisms between rings. In algebras
associated with the semantics of computation, the dynamics is
expressed as part of the algebraic structure itself, through a
reduction reduction relation typically denoted by $\red$. Below, we
give a recursive presentation of this relation for the calculus used
in the encoding.

$\red \subseteq \pi \times \pi$
$\red : \pi \to \mathcal{P}(\pi)$

\begin{mathpar}
  \inferrule* [lab=Comm] { \textsf{match}( x_{src}, x_{trgt} ) } { x_{trgt}?(y)P \; | \; x_{src}!\langle {Q} \rangle \red P\{\quotep{Q}/y}\} }
  \and \\
  \inferrule* [lab=Par] {{P} \red {P}'} {{{P} | {Q}} \red {{P}' | {Q}}}
  \and
  \inferrule* [lab=Equiv]{{{P} \scong {P}'} \andalso {{P}' \red {Q}'} \andalso {{Q}' \scong {Q}}}{{P} \red {Q}}
\end{mathpar}

\begin{eqnarray*}
  match_{\equiv} (\quotep{P},\quotep{Q}) & := & P \equiv Q \\
  match_{\dagger}(\quotep{P},\quotep{Q}) & := & \forall R. P|Q \red^{*} R => R \red^{*} 0 \\
  match_{K}(\quotep{P},\quotep{Q}) & := & K \mbox{ for some context } K
\end{eqnarray*}

$u?(x)P | u!\langle Q \rangle \red P\{\quotep{Q}/x\}$

%We write $\wred$ for $\red^*$, and $P\red$ if $\exists Q $ such that $ P \red Q$.
We write $P\red$ if $\exists Q $ such that $ P \red Q$ and $P\not\red$, otherwise.

\section{Replication}

As mentioned before, it is known that replication (and hence
recursion) can be implemented in a higher-order process algebra
\cite{SangiorgiWalker}. As our first example of calculation with the
machinery thus far presented we give the construction explicitly in
the {\rhoc}.

\begin{eqnarray}
	D_{x} & := & \prefix{x}{y}{(\binpar{\outputp{x}{y}}{@{y}})} \nonumber\\
	\bangp_{x}{P} & := & \binpar{{x}!\langle{\binpar{D_{x}}{P}}\rangle}{D_{x}} \nonumber
\end{eqnarray}

\begin{eqnarray}
	\bangp_{x}{P} & & \nonumber\\
	=
	& {x}!\langle{(\prefix{x}{y}{(\outputp{x}{y} | @{y})) | P}}\rangle 
	      | \prefix{x}{y}{(\outputp{x}{y} | @{y})} & \nonumber\\
	\red
	& (\outputp{x}{y} | @{y})\substn{\quotep{(\prefix{x}{y}{(@{y} | \outputp{x}{y})) | P}}}{y} & \nonumber\\
	=
	& \outputp{x}{\quotep{(\prefix{x}{y}{(\outputp{x}{y} | @{y})) | P}}}
	  | {(\prefix{x}{y}{(\outputp{x}{y} | @{y})) | P}} & \nonumber\\
	\red
	& \ldots & \nonumber\\
	\red^*
	& P | P | \ldots & \nonumber
\end{eqnarray}

Of course, this encoding, as an implementation, runs away, unfolding
$\bangp{P}$ eagerly. A lazier and more implementable replication
operator, restricted to input-guarded processes, may be obtained as follows.

\begin{eqnarray}
\bangp{\prefix{u}{v}{P}} 
	:= 
	\binpar{\lift{x}{\prefix{u}{v}{(\binpar{D(x)}{P})}}}{D(x)} \nonumber
\end{eqnarray}

\begin{remark}
  Note that the lazier definition still does not deal with summation
  or mixed summation (i.e. sums over input and output). The reader is
  invited to construct definitions of replication that deal with these
  features. 

  Further, the definitions are parameterized in a name, $x$. Can you,
  gentle reader, make a definition that eliminates this parameter and
  guarantees no accidental interaction between the replication
  machinery and the process being replicated -- i.e. no accidental
  sharing of names used by the process to get its work done and the
  name(s) used by the replication to effect copying. This latter
  revision of the definition of replication is crucial to obtaining
  the expected identity $!!P \sim !P$.
\end{remark}

\begin{remark}\label{rem:paradoxical_combinator}
  The reader familiar with the lambda calculus will have noticed the
  similarity between $D$ and the paradoxical combinator.

  [Ed. note: the existence of this seems to suggest we have to be more
  restrictive on the set of processes and names we admit if we are to
  support no-cloning.]
\end{remark}

\subsubsection{Bisimulation}

The computational dynamics gives rise to another kind of equivalence,
the equivalence of computational behavior. As previously mentioned
this is typically captured \emph{via} some form of bisimulation.

% The notion we use in this paper is weak barbed bisimulation
% \cite{milner91polyadicpi}.

The notion we use in this paper is derived from weak barbed
bisimulation \cite{milner91polyadicpi}. 

\begin{definition}
An \emph{observation relation}, $\downarrow_{\mathcal N}$, over a set
of names, $\mathcal N$, is the smallest relation satisfying the rules
below.

\infrule[Out-barb]{y \in {\mathcal N}, \; x \nameeq y}
		  {\outputp{x}{v} \downarrow_{\mathcal N} x}
\infrule[Par-barb]{\mbox{$P\downarrow_{\mathcal N} x$ or $Q\downarrow_{\mathcal N} x$}}
		  {\binpar{P}{Q} \downarrow_{\mathcal N} x}

We write $P \Downarrow_{\mathcal N} x$ if there is $Q$ such that 
$P \wred Q$ and $Q \downarrow_{\mathcal N} x$.
\end{definition}

\begin{definition}
%\label{def.bbisim}
An  ${\mathcal N}$-\emph{barbed bisimulation} over a set of names, ${\mathcal N}$, is a symmetric binary relation 
${\mathcal S}_{\mathcal N}$ between agents such that $P\rel{S}_{\mathcal N}Q$ implies:
\begin{enumerate}
\item If $P \red P'$ then $Q \wred Q'$ and $P'\rel{S}_{\mathcal N} Q'$.
\item If $P\downarrow_{\mathcal N} x$, then $Q\Downarrow_{\mathcal N} x$.
\end{enumerate}
$P$ is ${\mathcal N}$-barbed bisimilar to $Q$, written
$P \wbbisim_{\mathcal N} Q$, if $P \rel{S}_{\mathcal N} Q$ for some ${\mathcal N}$-barbed bisimulation ${\mathcal S}_{\mathcal N}$.
\end{definition}

$\mathcal{R} \subseteq \pi \times \pi$

$P \mathcal{R} Q => \forall P'. P \red P' \Rightarrow \exists Q'. Q \red Q', P' \mathcal{R} Q'$

$P \vdash x \Rightarrow Q \vdash x$

\begin{mathpar}
  \inferrule*[lab=Out-barb]{x \nameeq y}{{y}!\langle{Q}\rangle \vdash x}
  \and
  \inferrule*[lab=Par-barb]{\mbox{$P\vdash x$ or $Q\vdash x$}}{\binpar{P}{Q} \vdash x}
\end{mathpar}

\subsubsection{Contexts}

One of the principle advantages of computational calculi like the
$\pi$-calculus is a well-defined notion of context,
contextual-equivalence and a correlation between
contextual-equivalence and notions of bisimulation. The notion of
context allows the decomposition of a process into (sub-)process and
its syntactic environment, its context. Thus, a context may be
thought of as a process with a ``hole'' (written $\Box$) in it. The
application of a context $M$ to a process $P$, written $M[P]$, is
tantamount to filling the hole in $M$ with $P$. In this paper we do
not need the full weight of this theory, but do make use of the notion
of context in the proof the main theorem. 

\begin{mathpar}
  \inferrule* [lab=summation] {} {{M_{M},M_{N}} \bc \Box \;|\; x.M_{A} \;|\; M_{M}+M_{N}}
  \and
  \inferrule* [lab=agent] {} {{M_{A}} \bc (\vec{x})M_{P} \;| \; \clift{P_0,\ldots,M_{P},\ldots,P_N}}
  \and \\
  \inferrule* [lab=process] {} {{M_{P}} \bc M_{N} \;| \;P|M_{P} }
\end{mathpar} 

\begin{mathpar}
  \inferrule* [lab=sychronization] {} {M_{N} \bc \Box \;|\; x?M_{F} \;|\; x!M_{C}}
  \and
  \inferrule* [lab=abstraction] {} {{M_{F}} \bc (x)M_{P} }
  \and
  \inferrule* [lab=concretion] {} {{M_{C}} \bc \langle M_{P} \rangle }
  \and \\
  \inferrule* [lab=process] {} {{M_{P}} \bc M_{N} \;| \;P|M_{P} }
\end{mathpar}

\begin{definition}[contextual application] Given a context $M$, and
  process $P$, we define the \emph{contextual application}, $M[P] :=
  M\{P/\Box\}$. That is, the contextual application of M to P is the
  substitution of $P$ for $\Box$ in $M$.
\end{definition}

$\meaningof{-} : L \to \mathcal{P}(\pi)$

\begin{mathpar}
  \inferrule* [lab=collection] {} {\meaningof{true} = \pi, \and \meaningof{~E} = \pi \setminus \meaningof{E}, \and \meaningof{E_{1} \& E_{2}} = \meaningof{E_{1}} \cap \meaningof{E_{2}}}
\end{mathpar}

\begin{mathpar}
  \inferrule* [lab=structure] {} {\meaningof{0} = \{ P \in \pi | P \equiv 0 \}, \and \\ \meaningof{E_1 | E_2} = \{ P \in \pi | P \equiv P_{1} | P_{2}, P_{1} \in \meaningof{E_{1}}, P_{2} \in \meaningof{E_2}\} }
\end{mathpar}

\begin{mathpar}
 \inferrule* [lab=behavior] {} {\meaningof{\langle a?b \rangle E} = \{ P \in \pi | P \equiv Q | u?(y)P', \\ \and \\\\ \and \\ \;\;\; u \in \meaningof{a}, \forall z.P'\{z/y\} \in \meaningof{E\{z/b\}}\}, \and \\ \meaningof{a!E} = \{ P \in \pi | P \equiv Q | x!\langle P' \rangle, x \in \meaningof{a} P' \in \meaningof{E}\} }
\end{mathpar}

\begin{mathpar}
 \inferrule* [lab=nominal] {} {\meaningof{\quotep{E}} = \{ \quotep{P} \in \quotep{\pi} | P \in \meaningof{E} \}, \and \meaningof{\quotep{P}} = \{ \quotep{Q} \in \quotep{\pi} | P \equiv Q \} \and \\ \meaningof{@\quotep{E}} = \{ P \in \pi | P \equiv @x, x \in \meaningof{E} \}}
\end{mathpar}

\begin{eqnarray*}
  \\
  \meaningof{-} : TS \to ST
\end{eqnarray*}

\begin{eqnarray*}
  \\
  L : TS \to ST
\end{eqnarray*}

\begin{eqnarray*}
  \\
  P \models E \iff P \in \meaningof{E}
\end{eqnarray*}

\begin{eqnarray*}
  P \approx_{L} Q \iff \forall E \in L. P \models E \iff Q \models E
\end{eqnarray*}

\begin{eqnarray*}
  P \approx_{K} Q
\end{eqnarray*}

\begin{eqnarray*}
  P \approx Q
\end{eqnarray*}

$\approx_{K} = \approx = \approx_{L}$

\subsubsection{Contextual duality}

Note that contexts extend the quotation operation to a family of
operations from processes to names. Given a context, $M$, we can
define a \emph{nominal context}, $\quotep{M}$ by $\quotep{M}[P] :=
\quotep{M[P]}$. To foreshadow what is to come we observe that these
operations enjoy a duality with processes very much like the duality
between vectors and maps from vectors to scalars.

Further, because the calculus is essentially higher-order, we have a
correspondence between contexts and processes. More specifically,
given a name $x$ and a context $M$ we can construct $M^{*}_{x}$ such
that 

\begin{mathpar}
  M^{*}_{x} | \lift{x}{P} \red M[P]
\end{mathpar}

namely,

\begin{mathpar}
  M^{*}_{x} := x?(u).M[\dropn{u}]
\end{mathpar}

The dependence of $M^{*}_{x}$ on a name makes it an abstraction, 

\begin{mathpar}
  M^{*} := (x)x?(u).M[\dropn{u}]
\end{mathpar}

\subsection{Additional notation}

It will sometimes be convenient to denote the process a name
quotes. We already have the notation $x = \quotep{P}$, but it will be
convenient to introduce an alternate notation, $\procn{x}$, when we
want to emphasize the connection to the use of the name. Note that, by
virtue of name equivalence, $\quotep{\procn{x}} \nameeq x$; so, the
notation is consistent with previous definitions.

Further, because names have structure it is possible to effect
substitutions on the basis of that structure. This means we need to
upgrade our notation for substitutions, which we accomplish by
adapting comprehension notation. Thus,

\begin{mathpar}
  P\{ y / x : x \in S \}
\end{mathpar}

is interpreted to mean the process derived from P by replacing (in a
capture-avoiding manner) each occurrence of $x$ in $S$ by $y$. For example,

\begin{mathpar}
  P\{ \quotep{\procn{x}|\procn{x}} / x : x \in \freenames{P} \}
\end{mathpar}

will replace each (occurrence) of a free name $x$ in $P$ by
$\quotep{\procn{x}|\procn{x}}$.

Also, we will avail ourselves of the notation $x^{L}$ and $x^{R}$ to
denote injections of a name into disjoint copies of the name
space. There are numerous ways to accomplish this. One example can be
found in \cite{MeredithR05}. This notation overloads to vectors of
names: $\vec{x}^{\pi} := (x_{i}^{\pi} \; : \; 0 \leq i < |\vec{x}| )$ where $\pi \in \{L,R\}$.

We also use $P^{\Box} := P|\Box$.

In \cite{MeredithR05} an interpretation of the new operator is
given. It turns out that there are several possible interpretations
all enjoying the requisite algebraic properties of the operator (see
\cite{milner91polyadicpi}). We will therefore make liberal use of
$(\nu\; \vec{x})P$.

% subsection the_syntax_and_semantics_of_the_notation_system (end)   

\input{qm2pi.qmops} 

\input{qm2pi.sterngerlach} 

\input{qm2pi.metric} 

% section concurrent_process_calculi (end)

%\input{qm2pi.proofsketch}

% section proof sketch (end)

%\input{qm2pi.slviaknots} 

% section spatial logic via knots (end)

\input{qm2pi.conclusion}

% section conclusion (end)

%\input{qm2pi.dtcodes} 

% section wiring algorithm (end)

\input{qm2pi.ack} 

% section acknowledgments (end)

\newpage


\bibliographystyle{plain}   
\bibliography{../../biblios/main.bib}

\input{qm2pi.rhodetails}

\end{document}

 

% section notation (end)

\input{qm2pi.process.calculi} 

% section concurrent_process_calculi_and_spatial_logics_ (end)
    
%\documentclass[12pt]{llncs}
%\documentclass{jktr}

\usepackage[pdftex]{hyperref}                   
\usepackage {listings}
\usepackage {mathpartir}
\usepackage{bcprules}
%\usepackage{listings}
                       
\usepackage{graphicx} 
%\usepackage[margins=2.5cm,nohead,nofoot]{geometry}
%\usepackage{geometry}
\usepackage{amsfonts}
\usepackage{amstext}
\usepackage{latexsym}
\usepackage{amssymb}
\usepackage{color}


%\include{myPreamble}
\include{qm2pi.local} 

%\ifpdf
%\usepackage[pdftex]{graphicx}
%\else
%\usepackage{graphicx}
%\fi

 % \ifpdf
%  \usepackage{pdfsync}
%  \if


%\title{Brief Article}
%\author{David F. Snyder}
%\author{L.G. Meredith}

%\address{Dept. of Math., Texas State University--San Marcos, San Marcos, TX 78666}
       
\pagestyle{empty}


\begin{document}

\lstset{language=[Objective]Caml,frame=shadowbox}

\input{qm2pi.front}

% section front matter (end)

\input{qm2pi.intro} 
 
% section introduction (end)

% \input{qm2pi.knotations} 

% section notation (end)

\input{qm2pi.process.calculi} 

% section concurrent_process_calculi_and_spatial_logics_ (end)
    
%\input{qm2pi.knots2pi} 

%\input{qm2pi.trefoil} 

%\input{qm2pi.mainthm} 

% subsection basic_interpretation (end)

%\input{qm2pi.rho.presentation} 
\subsection{The syntax and semantics of the notation system}\label{sub:the_syntax_and_semantics_of_the_notation_system} % (fold)

We now summarize a technical presentation of the calculus that
embodies our theory of dynamics. The typical presentation of such a
calculus follows the style of giving generators and relations on
them. The grammar, below, describing term constructors, freely
generates the set of processes, $\Proc$. This set is then quotiented
by a relation known as structural congruence and it is over this set
that the notion of dynamics is expressed. This presentation is
essentially that of \cite{MeredithR05} with the addition of
polyadicity and summation. For readability we have relegated some of
the technical subtleties to an appendix.

\subsubsection{Process grammar}\label{subsub:process_grammar}

\begin{mathpar}
  \inferrule* [lab=synchronization] {} {{M} \bc \pzero \;|\; x?F \;|\; x!C }
  \and
  \inferrule* [lab=abstraction] {} {{F} \bc (x)P}
  \and
  \inferrule* [lab=concretion] {} {{C} \bc \langle Q \rangle}
  \and
  \inferrule* [lab=process] {} {{P,Q} \bc M \;| \;P|Q \;|\; @{x}}
  \and
  \inferrule* [lab=name] {} {{x} \bc \quotep{P}}
\end{mathpar} 

Note that $\vec{x}$ (resp. $\vec{P}$) denotes a vector of names
(resp. processes) of length $|\vec{x}|$ (resp. $|\vec{P}|$). We adopt
the following useful abbreviations.

\begin{mathpar}
   x?(\vec{y}).P := x.(\vec{y})P \and  x\clift{\vec{P}} := x.\clift{\vec{P}}
   \and x!(y) := \lift{x}{\dropn{y}}
   \and \Pi_{i=0}^{n-1}P_i := P_0 | \ldots | P_{n-1}
\end{mathpar}

\subsubsection{Structural congruence}

\paragraph{Free and bound names and alpha-equivalence.} At the
core of structural equivalence is alpha-equivalence which identifies
process that are the same up to a change of variable. Formally, we
recognize the distinction between free and bound names. The free names
of a process, $\freenames{P}$, may be calculated recursively as
follows:

\begin{mathpar}
\freenames{\pzero} := \emptyset
  \and \\
  \freenames{x?(y).P} := \{ x \} \cup (\freenames{P} \setminus \{ y \})
  \and 
  \freenames{x!\langle P \rangle} := \{ x \} \cup \{ P \} 
  \and \\
  \freenames{P|Q} := \freenames{P} \cup \freenames{Q}
  \and \\
  \freenames{@{x}} := \{ x \}
\end{mathpar}

$\pi$
$\quotep{\pi}$

$\freenames{-} : \pi \to \mathcal{P}(\quotep{\pi})$

\begin{eqnarray*}
  \freenames{\pzero} & := & \emptyset \\
  \freenames{x?(y).P} & := & \{ x \} \cup (\freenames{P} \setminus \{ y \}) \\
  \freenames{x!\langle P \rangle} & := & \{ x \} \cup \{ P \} \\
  \freenames{P|Q} & := & \freenames{P} \cup \freenames{Q} \\
  \freenames{\dropn{x}} & := & \{ x \}
\end{eqnarray*}

The bound names of a process, $\boundnames{P}$, are those names occurring in $P$
that are not free. For example, in $x?(y).0$, the name $x$ is free, while $y$ is bound.

\begin{mathpar}
  \inferrule* [lab=monoidal-laws] {} { P|Q \equiv Q|P \and P|0 \equiv P \and P|(Q|R) \equiv (P|Q)|R }
\end{mathpar}

\begin{mathpar}
  \inferrule* [lab=alpha-equivalence] {} { (x)P \equiv (y)P\{y/x\} \and y \not\in \freenames{P} }
\end{mathpar}

\begin{definition}
Then two processes, $P,Q$, are alpha-equivalent if $P = Q\{\vec{y}/\vec{x}\}$ for
some $\vec{x} \in \boundnames{Q},\vec{y} \in \boundnames{P}$, where $Q\{\vec{y}/\vec{x}\}$
denotes the capture-avoiding substitution of $\vec{y}$ for $\vec{x}$ in $Q$.
\end{definition}

\begin{definition}
  The {\em structural congruence} \cite{SangiorgiWalker} , $\equiv$,
  between processes is the least congruence containing
  alpha-equivalence, satisfying the abelian monoid laws
  (associativity, commutativity and $\pzero$ as identity) for parallel
  composition $|$ and for summation $+$.
\end{definition}

\subsection{Name equivalence}

We take name equivalence, written $\nameeq$, to be the smallest
equivalence relation generated by the following rules.

\begin{mathpar}
\inferrule*[lab=Quote-drop]
{ }
{ \quotep{@{x}} \nameeq x }

\inferrule*[lab=Struct-equiv]
{ P \scong Q }
{ \quotep{P} \nameeq \quotep{Q} }
\end{mathpar}

The astute reader will have noticed that the mutual recursion of names
and processes imposes a mutual recursion on alpha-equivalence and
structural equivalence via name-equivalence. Fortunately, all of this
works out pleasantly and we may calculate in the natural way, free of
concern. The reader interested in the details is referred to the
appendix \ref{appendix:rho_details}.

\subsection{Substitution}

We use $\Proc$ for the set of processes, $\QProc$ for the set of
names, and $\id{\{}\vec{y} / \vec{x} \id{\}}$ to denote partial maps,
$s : \QProc \rightarrow \QProc$. A map, $s$ lifts, uniquely, to a map
on process terms, $\widehat{s} : \Proc \rightarrow \Proc$ by the
following equations.

\begin{mathpar}
  (0) \psubstp{Q}{P} := 0 \\
  (R \juxtap S) \psubstp{Q}{P}
  :=    
  (R)\psubstp{Q}{P} \juxtap (S) \psubstp{Q}{P} \\
  (x?(y).R) \psubstp{Q}{P}    
  :=    
  (x)\substp{Q}{P} (z)\concat( (R \psubstn{z}{y}) \psubstp{Q}{P} ) \\
  (\lift{x}{R}) \psubstp{Q}{P}  
  :=
  \lift{(x)\substp{Q}{P}}{ R \psubstp{Q}{P} } \\
%   (\dropn{x})  \psubstp{Q}{P}       
%   := 
%   \left\{ 
%     \begin{array}{ccc} 
%       \dropn{\quotep{Q}} & & x \nameeq \quotep{P} \\
%       \dropn{x} & & otherwise \\
%     \end{array}
%   \right. 
  (\dropn{x})  \psubstp{Q}{P}       
  := 
  \left\{ 
    \begin{array}{ccc} 
      Q & & x \nameeq \quotep{P} \\
      \dropn{x} & & otherwise \\
    \end{array}
  \right.
\end{mathpar}
 

where

\begin{eqnarray}
  (x)\id{\{} \lpquote Q \rpquote / \lpquote P \rpquote \id{\}}            = 
  \left\{ 
    \begin{array}{ccc}
      \lpquote Q \rpquote & & x \nameeq \lpquote P \rpquote \\
      x & & otherwise \\
    \end{array}
  \right. \nonumber
\end{eqnarray}

and $z$ is chosen distinct from $\quotep{P}$, $\quotep{Q}$, the free
names in $Q$, and all the names in $R$. Our $\alpha$-equivalence will
be built in the standard way from this substitution.

\begin{remark}\label{rem:no_self_referential_names}
  One consequence of these definitions is that $\forall P. \quotep{P}
  \not\in \freenames{P}$.
\end{remark}

\subsection{ Dynamic quote: an example }

Anticipating something of what's to come, consider applying the
substitution, $\widehat{\id{\{}u / z \id{\}}}$, to the following pair
of processes, $\lift{w}{y!(z)}$ and $w[ \lpquote y!(z) \rpquote ]$.

\begin{eqnarray}
	\lift{w}{y!(z)}\widehat{\id{\{}u / z \id{\}}}
		& = &
		\lift{w}{y!(u)} \nonumber\\
	w[ \lpquote y!(z) \rpquote ] \widehat{ \id{\{}u / z \id{\}} }
		& = &
		w[ \lpquote y!(z) \rpquote ] \nonumber
\end{eqnarray}

Because the body of the process between quotes is impervious to
substitution, we get radically different answers. In fact, by
examining the first process in an input context,
e.g. $x?(z).\lift{w}{y!(z)}$, we see that the process under the lift
operator may be shaped by prefixed inputs binding a name inside it. In
this sense, the lift operator will be seen as a way to dynamically
construct processes before reifying them as names.

Finally equipped with these standard features we can present the
dynamics of the calculus.

\subsubsection{Operational semantics} 

Finally, we introduce the computational dynamics. What marks these
algebras as distinct from other more traditionally studied algebraic
structures, e.g. vector spaces or polynomial rings, is the manner in
which dynamics is captured. In traditional structures, dynamics is typically
expressed through morphisms between such structures, as in linear maps
between vector spaces or morphisms between rings. In algebras
associated with the semantics of computation, the dynamics is
expressed as part of the algebraic structure itself, through a
reduction reduction relation typically denoted by $\red$. Below, we
give a recursive presentation of this relation for the calculus used
in the encoding.

$\red \subseteq \pi \times \pi$
$\red : \pi \to \mathcal{P}(\pi)$

\begin{mathpar}
  \inferrule* [lab=Comm] { \textsf{match}( x_{src}, x_{trgt} ) } { x_{trgt}?(y)P \; | \; x_{src}!\langle {Q} \rangle \red P\{\quotep{Q}/y}\} }
  \and \\
  \inferrule* [lab=Par] {{P} \red {P}'} {{{P} | {Q}} \red {{P}' | {Q}}}
  \and
  \inferrule* [lab=Equiv]{{{P} \scong {P}'} \andalso {{P}' \red {Q}'} \andalso {{Q}' \scong {Q}}}{{P} \red {Q}}
\end{mathpar}

\begin{eqnarray*}
  match_{\equiv} (\quotep{P},\quotep{Q}) & := & P \equiv Q \\
  match_{\dagger}(\quotep{P},\quotep{Q}) & := & \forall R. P|Q \red^{*} R => R \red^{*} 0 \\
  match_{K}(\quotep{P},\quotep{Q}) & := & K \mbox{ for some context } K
\end{eqnarray*}

$u?(x)P | u!\langle Q \rangle \red P\{\quotep{Q}/x\}$

%We write $\wred$ for $\red^*$, and $P\red$ if $\exists Q $ such that $ P \red Q$.
We write $P\red$ if $\exists Q $ such that $ P \red Q$ and $P\not\red$, otherwise.

\section{Replication}

As mentioned before, it is known that replication (and hence
recursion) can be implemented in a higher-order process algebra
\cite{SangiorgiWalker}. As our first example of calculation with the
machinery thus far presented we give the construction explicitly in
the {\rhoc}.

\begin{eqnarray}
	D_{x} & := & \prefix{x}{y}{(\binpar{\outputp{x}{y}}{@{y}})} \nonumber\\
	\bangp_{x}{P} & := & \binpar{{x}!\langle{\binpar{D_{x}}{P}}\rangle}{D_{x}} \nonumber
\end{eqnarray}

\begin{eqnarray}
	\bangp_{x}{P} & & \nonumber\\
	=
	& {x}!\langle{(\prefix{x}{y}{(\outputp{x}{y} | @{y})) | P}}\rangle 
	      | \prefix{x}{y}{(\outputp{x}{y} | @{y})} & \nonumber\\
	\red
	& (\outputp{x}{y} | @{y})\substn{\quotep{(\prefix{x}{y}{(@{y} | \outputp{x}{y})) | P}}}{y} & \nonumber\\
	=
	& \outputp{x}{\quotep{(\prefix{x}{y}{(\outputp{x}{y} | @{y})) | P}}}
	  | {(\prefix{x}{y}{(\outputp{x}{y} | @{y})) | P}} & \nonumber\\
	\red
	& \ldots & \nonumber\\
	\red^*
	& P | P | \ldots & \nonumber
\end{eqnarray}

Of course, this encoding, as an implementation, runs away, unfolding
$\bangp{P}$ eagerly. A lazier and more implementable replication
operator, restricted to input-guarded processes, may be obtained as follows.

\begin{eqnarray}
\bangp{\prefix{u}{v}{P}} 
	:= 
	\binpar{\lift{x}{\prefix{u}{v}{(\binpar{D(x)}{P})}}}{D(x)} \nonumber
\end{eqnarray}

\begin{remark}
  Note that the lazier definition still does not deal with summation
  or mixed summation (i.e. sums over input and output). The reader is
  invited to construct definitions of replication that deal with these
  features. 

  Further, the definitions are parameterized in a name, $x$. Can you,
  gentle reader, make a definition that eliminates this parameter and
  guarantees no accidental interaction between the replication
  machinery and the process being replicated -- i.e. no accidental
  sharing of names used by the process to get its work done and the
  name(s) used by the replication to effect copying. This latter
  revision of the definition of replication is crucial to obtaining
  the expected identity $!!P \sim !P$.
\end{remark}

\begin{remark}\label{rem:paradoxical_combinator}
  The reader familiar with the lambda calculus will have noticed the
  similarity between $D$ and the paradoxical combinator.

  [Ed. note: the existence of this seems to suggest we have to be more
  restrictive on the set of processes and names we admit if we are to
  support no-cloning.]
\end{remark}

\subsubsection{Bisimulation}

The computational dynamics gives rise to another kind of equivalence,
the equivalence of computational behavior. As previously mentioned
this is typically captured \emph{via} some form of bisimulation.

% The notion we use in this paper is weak barbed bisimulation
% \cite{milner91polyadicpi}.

The notion we use in this paper is derived from weak barbed
bisimulation \cite{milner91polyadicpi}. 

\begin{definition}
An \emph{observation relation}, $\downarrow_{\mathcal N}$, over a set
of names, $\mathcal N$, is the smallest relation satisfying the rules
below.

\infrule[Out-barb]{y \in {\mathcal N}, \; x \nameeq y}
		  {\outputp{x}{v} \downarrow_{\mathcal N} x}
\infrule[Par-barb]{\mbox{$P\downarrow_{\mathcal N} x$ or $Q\downarrow_{\mathcal N} x$}}
		  {\binpar{P}{Q} \downarrow_{\mathcal N} x}

We write $P \Downarrow_{\mathcal N} x$ if there is $Q$ such that 
$P \wred Q$ and $Q \downarrow_{\mathcal N} x$.
\end{definition}

\begin{definition}
%\label{def.bbisim}
An  ${\mathcal N}$-\emph{barbed bisimulation} over a set of names, ${\mathcal N}$, is a symmetric binary relation 
${\mathcal S}_{\mathcal N}$ between agents such that $P\rel{S}_{\mathcal N}Q$ implies:
\begin{enumerate}
\item If $P \red P'$ then $Q \wred Q'$ and $P'\rel{S}_{\mathcal N} Q'$.
\item If $P\downarrow_{\mathcal N} x$, then $Q\Downarrow_{\mathcal N} x$.
\end{enumerate}
$P$ is ${\mathcal N}$-barbed bisimilar to $Q$, written
$P \wbbisim_{\mathcal N} Q$, if $P \rel{S}_{\mathcal N} Q$ for some ${\mathcal N}$-barbed bisimulation ${\mathcal S}_{\mathcal N}$.
\end{definition}

$\mathcal{R} \subseteq \pi \times \pi$

$P \mathcal{R} Q => \forall P'. P \red P' \Rightarrow \exists Q'. Q \red Q', P' \mathcal{R} Q'$

$P \vdash x \Rightarrow Q \vdash x$

\begin{mathpar}
  \inferrule*[lab=Out-barb]{x \nameeq y}{{y}!\langle{Q}\rangle \vdash x}
  \and
  \inferrule*[lab=Par-barb]{\mbox{$P\vdash x$ or $Q\vdash x$}}{\binpar{P}{Q} \vdash x}
\end{mathpar}

\subsubsection{Contexts}

One of the principle advantages of computational calculi like the
$\pi$-calculus is a well-defined notion of context,
contextual-equivalence and a correlation between
contextual-equivalence and notions of bisimulation. The notion of
context allows the decomposition of a process into (sub-)process and
its syntactic environment, its context. Thus, a context may be
thought of as a process with a ``hole'' (written $\Box$) in it. The
application of a context $M$ to a process $P$, written $M[P]$, is
tantamount to filling the hole in $M$ with $P$. In this paper we do
not need the full weight of this theory, but do make use of the notion
of context in the proof the main theorem. 

\begin{mathpar}
  \inferrule* [lab=summation] {} {{M_{M},M_{N}} \bc \Box \;|\; x.M_{A} \;|\; M_{M}+M_{N}}
  \and
  \inferrule* [lab=agent] {} {{M_{A}} \bc (\vec{x})M_{P} \;| \; \clift{P_0,\ldots,M_{P},\ldots,P_N}}
  \and \\
  \inferrule* [lab=process] {} {{M_{P}} \bc M_{N} \;| \;P|M_{P} }
\end{mathpar} 

\begin{mathpar}
  \inferrule* [lab=sychronization] {} {M_{N} \bc \Box \;|\; x?M_{F} \;|\; x!M_{C}}
  \and
  \inferrule* [lab=abstraction] {} {{M_{F}} \bc (x)M_{P} }
  \and
  \inferrule* [lab=concretion] {} {{M_{C}} \bc \langle M_{P} \rangle }
  \and \\
  \inferrule* [lab=process] {} {{M_{P}} \bc M_{N} \;| \;P|M_{P} }
\end{mathpar}

\begin{definition}[contextual application] Given a context $M$, and
  process $P$, we define the \emph{contextual application}, $M[P] :=
  M\{P/\Box\}$. That is, the contextual application of M to P is the
  substitution of $P$ for $\Box$ in $M$.
\end{definition}

$\meaningof{-} : L \to \mathcal{P}(\pi)$

\begin{mathpar}
  \inferrule* [lab=collection] {} {\meaningof{true} = \pi, \and \meaningof{~E} = \pi \setminus \meaningof{E}, \and \meaningof{E_{1} \& E_{2}} = \meaningof{E_{1}} \cap \meaningof{E_{2}}}
\end{mathpar}

\begin{mathpar}
  \inferrule* [lab=structure] {} {\meaningof{0} = \{ P \in \pi | P \equiv 0 \}, \and \\ \meaningof{E_1 | E_2} = \{ P \in \pi | P \equiv P_{1} | P_{2}, P_{1} \in \meaningof{E_{1}}, P_{2} \in \meaningof{E_2}\} }
\end{mathpar}

\begin{mathpar}
 \inferrule* [lab=behavior] {} {\meaningof{\langle a?b \rangle E} = \{ P \in \pi | P \equiv Q | u?(y)P', \\ \and \\\\ \and \\ \;\;\; u \in \meaningof{a}, \forall z.P'\{z/y\} \in \meaningof{E\{z/b\}}\}, \and \\ \meaningof{a!E} = \{ P \in \pi | P \equiv Q | x!\langle P' \rangle, x \in \meaningof{a} P' \in \meaningof{E}\} }
\end{mathpar}

\begin{mathpar}
 \inferrule* [lab=nominal] {} {\meaningof{\quotep{E}} = \{ \quotep{P} \in \quotep{\pi} | P \in \meaningof{E} \}, \and \meaningof{\quotep{P}} = \{ \quotep{Q} \in \quotep{\pi} | P \equiv Q \} \and \\ \meaningof{@\quotep{E}} = \{ P \in \pi | P \equiv @x, x \in \meaningof{E} \}}
\end{mathpar}

\begin{eqnarray*}
  \\
  \meaningof{-} : TS \to ST
\end{eqnarray*}

\begin{eqnarray*}
  \\
  L : TS \to ST
\end{eqnarray*}

\begin{eqnarray*}
  \\
  P \models E \iff P \in \meaningof{E}
\end{eqnarray*}

\begin{eqnarray*}
  P \approx_{L} Q \iff \forall E \in L. P \models E \iff Q \models E
\end{eqnarray*}

\begin{eqnarray*}
  P \approx_{K} Q
\end{eqnarray*}

\begin{eqnarray*}
  P \approx Q
\end{eqnarray*}

$\approx_{K} = \approx = \approx_{L}$

\subsubsection{Contextual duality}

Note that contexts extend the quotation operation to a family of
operations from processes to names. Given a context, $M$, we can
define a \emph{nominal context}, $\quotep{M}$ by $\quotep{M}[P] :=
\quotep{M[P]}$. To foreshadow what is to come we observe that these
operations enjoy a duality with processes very much like the duality
between vectors and maps from vectors to scalars.

Further, because the calculus is essentially higher-order, we have a
correspondence between contexts and processes. More specifically,
given a name $x$ and a context $M$ we can construct $M^{*}_{x}$ such
that 

\begin{mathpar}
  M^{*}_{x} | \lift{x}{P} \red M[P]
\end{mathpar}

namely,

\begin{mathpar}
  M^{*}_{x} := x?(u).M[\dropn{u}]
\end{mathpar}

The dependence of $M^{*}_{x}$ on a name makes it an abstraction, 

\begin{mathpar}
  M^{*} := (x)x?(u).M[\dropn{u}]
\end{mathpar}

\subsection{Additional notation}

It will sometimes be convenient to denote the process a name
quotes. We already have the notation $x = \quotep{P}$, but it will be
convenient to introduce an alternate notation, $\procn{x}$, when we
want to emphasize the connection to the use of the name. Note that, by
virtue of name equivalence, $\quotep{\procn{x}} \nameeq x$; so, the
notation is consistent with previous definitions.

Further, because names have structure it is possible to effect
substitutions on the basis of that structure. This means we need to
upgrade our notation for substitutions, which we accomplish by
adapting comprehension notation. Thus,

\begin{mathpar}
  P\{ y / x : x \in S \}
\end{mathpar}

is interpreted to mean the process derived from P by replacing (in a
capture-avoiding manner) each occurrence of $x$ in $S$ by $y$. For example,

\begin{mathpar}
  P\{ \quotep{\procn{x}|\procn{x}} / x : x \in \freenames{P} \}
\end{mathpar}

will replace each (occurrence) of a free name $x$ in $P$ by
$\quotep{\procn{x}|\procn{x}}$.

Also, we will avail ourselves of the notation $x^{L}$ and $x^{R}$ to
denote injections of a name into disjoint copies of the name
space. There are numerous ways to accomplish this. One example can be
found in \cite{MeredithR05}. This notation overloads to vectors of
names: $\vec{x}^{\pi} := (x_{i}^{\pi} \; : \; 0 \leq i < |\vec{x}| )$ where $\pi \in \{L,R\}$.

We also use $P^{\Box} := P|\Box$.

In \cite{MeredithR05} an interpretation of the new operator is
given. It turns out that there are several possible interpretations
all enjoying the requisite algebraic properties of the operator (see
\cite{milner91polyadicpi}). We will therefore make liberal use of
$(\nu\; \vec{x})P$.

% subsection the_syntax_and_semantics_of_the_notation_system (end)   

\input{qm2pi.qmops} 

\input{qm2pi.sterngerlach} 

\input{qm2pi.metric} 

% section concurrent_process_calculi (end)

%\input{qm2pi.proofsketch}

% section proof sketch (end)

%\input{qm2pi.slviaknots} 

% section spatial logic via knots (end)

\input{qm2pi.conclusion}

% section conclusion (end)

%\input{qm2pi.dtcodes} 

% section wiring algorithm (end)

\input{qm2pi.ack} 

% section acknowledgments (end)

\newpage


\bibliographystyle{plain}   
\bibliography{../../biblios/main.bib}

\input{qm2pi.rhodetails}

\end{document}

 

%\documentclass[12pt]{llncs}
%\documentclass{jktr}

\usepackage[pdftex]{hyperref}                   
\usepackage {listings}
\usepackage {mathpartir}
\usepackage{bcprules}
%\usepackage{listings}
                       
\usepackage{graphicx} 
%\usepackage[margins=2.5cm,nohead,nofoot]{geometry}
%\usepackage{geometry}
\usepackage{amsfonts}
\usepackage{amstext}
\usepackage{latexsym}
\usepackage{amssymb}
\usepackage{color}


%\include{myPreamble}
\include{qm2pi.local} 

%\ifpdf
%\usepackage[pdftex]{graphicx}
%\else
%\usepackage{graphicx}
%\fi

 % \ifpdf
%  \usepackage{pdfsync}
%  \if


%\title{Brief Article}
%\author{David F. Snyder}
%\author{L.G. Meredith}

%\address{Dept. of Math., Texas State University--San Marcos, San Marcos, TX 78666}
       
\pagestyle{empty}


\begin{document}

\lstset{language=[Objective]Caml,frame=shadowbox}

\input{qm2pi.front}

% section front matter (end)

\input{qm2pi.intro} 
 
% section introduction (end)

% \input{qm2pi.knotations} 

% section notation (end)

\input{qm2pi.process.calculi} 

% section concurrent_process_calculi_and_spatial_logics_ (end)
    
%\input{qm2pi.knots2pi} 

%\input{qm2pi.trefoil} 

%\input{qm2pi.mainthm} 

% subsection basic_interpretation (end)

%\input{qm2pi.rho.presentation} 
\subsection{The syntax and semantics of the notation system}\label{sub:the_syntax_and_semantics_of_the_notation_system} % (fold)

We now summarize a technical presentation of the calculus that
embodies our theory of dynamics. The typical presentation of such a
calculus follows the style of giving generators and relations on
them. The grammar, below, describing term constructors, freely
generates the set of processes, $\Proc$. This set is then quotiented
by a relation known as structural congruence and it is over this set
that the notion of dynamics is expressed. This presentation is
essentially that of \cite{MeredithR05} with the addition of
polyadicity and summation. For readability we have relegated some of
the technical subtleties to an appendix.

\subsubsection{Process grammar}\label{subsub:process_grammar}

\begin{mathpar}
  \inferrule* [lab=synchronization] {} {{M} \bc \pzero \;|\; x?F \;|\; x!C }
  \and
  \inferrule* [lab=abstraction] {} {{F} \bc (x)P}
  \and
  \inferrule* [lab=concretion] {} {{C} \bc \langle Q \rangle}
  \and
  \inferrule* [lab=process] {} {{P,Q} \bc M \;| \;P|Q \;|\; @{x}}
  \and
  \inferrule* [lab=name] {} {{x} \bc \quotep{P}}
\end{mathpar} 

Note that $\vec{x}$ (resp. $\vec{P}$) denotes a vector of names
(resp. processes) of length $|\vec{x}|$ (resp. $|\vec{P}|$). We adopt
the following useful abbreviations.

\begin{mathpar}
   x?(\vec{y}).P := x.(\vec{y})P \and  x\clift{\vec{P}} := x.\clift{\vec{P}}
   \and x!(y) := \lift{x}{\dropn{y}}
   \and \Pi_{i=0}^{n-1}P_i := P_0 | \ldots | P_{n-1}
\end{mathpar}

\subsubsection{Structural congruence}

\paragraph{Free and bound names and alpha-equivalence.} At the
core of structural equivalence is alpha-equivalence which identifies
process that are the same up to a change of variable. Formally, we
recognize the distinction between free and bound names. The free names
of a process, $\freenames{P}$, may be calculated recursively as
follows:

\begin{mathpar}
\freenames{\pzero} := \emptyset
  \and \\
  \freenames{x?(y).P} := \{ x \} \cup (\freenames{P} \setminus \{ y \})
  \and 
  \freenames{x!\langle P \rangle} := \{ x \} \cup \{ P \} 
  \and \\
  \freenames{P|Q} := \freenames{P} \cup \freenames{Q}
  \and \\
  \freenames{@{x}} := \{ x \}
\end{mathpar}

$\pi$
$\quotep{\pi}$

$\freenames{-} : \pi \to \mathcal{P}(\quotep{\pi})$

\begin{eqnarray*}
  \freenames{\pzero} & := & \emptyset \\
  \freenames{x?(y).P} & := & \{ x \} \cup (\freenames{P} \setminus \{ y \}) \\
  \freenames{x!\langle P \rangle} & := & \{ x \} \cup \{ P \} \\
  \freenames{P|Q} & := & \freenames{P} \cup \freenames{Q} \\
  \freenames{\dropn{x}} & := & \{ x \}
\end{eqnarray*}

The bound names of a process, $\boundnames{P}$, are those names occurring in $P$
that are not free. For example, in $x?(y).0$, the name $x$ is free, while $y$ is bound.

\begin{mathpar}
  \inferrule* [lab=monoidal-laws] {} { P|Q \equiv Q|P \and P|0 \equiv P \and P|(Q|R) \equiv (P|Q)|R }
\end{mathpar}

\begin{mathpar}
  \inferrule* [lab=alpha-equivalence] {} { (x)P \equiv (y)P\{y/x\} \and y \not\in \freenames{P} }
\end{mathpar}

\begin{definition}
Then two processes, $P,Q$, are alpha-equivalent if $P = Q\{\vec{y}/\vec{x}\}$ for
some $\vec{x} \in \boundnames{Q},\vec{y} \in \boundnames{P}$, where $Q\{\vec{y}/\vec{x}\}$
denotes the capture-avoiding substitution of $\vec{y}$ for $\vec{x}$ in $Q$.
\end{definition}

\begin{definition}
  The {\em structural congruence} \cite{SangiorgiWalker} , $\equiv$,
  between processes is the least congruence containing
  alpha-equivalence, satisfying the abelian monoid laws
  (associativity, commutativity and $\pzero$ as identity) for parallel
  composition $|$ and for summation $+$.
\end{definition}

\subsection{Name equivalence}

We take name equivalence, written $\nameeq$, to be the smallest
equivalence relation generated by the following rules.

\begin{mathpar}
\inferrule*[lab=Quote-drop]
{ }
{ \quotep{@{x}} \nameeq x }

\inferrule*[lab=Struct-equiv]
{ P \scong Q }
{ \quotep{P} \nameeq \quotep{Q} }
\end{mathpar}

The astute reader will have noticed that the mutual recursion of names
and processes imposes a mutual recursion on alpha-equivalence and
structural equivalence via name-equivalence. Fortunately, all of this
works out pleasantly and we may calculate in the natural way, free of
concern. The reader interested in the details is referred to the
appendix \ref{appendix:rho_details}.

\subsection{Substitution}

We use $\Proc$ for the set of processes, $\QProc$ for the set of
names, and $\id{\{}\vec{y} / \vec{x} \id{\}}$ to denote partial maps,
$s : \QProc \rightarrow \QProc$. A map, $s$ lifts, uniquely, to a map
on process terms, $\widehat{s} : \Proc \rightarrow \Proc$ by the
following equations.

\begin{mathpar}
  (0) \psubstp{Q}{P} := 0 \\
  (R \juxtap S) \psubstp{Q}{P}
  :=    
  (R)\psubstp{Q}{P} \juxtap (S) \psubstp{Q}{P} \\
  (x?(y).R) \psubstp{Q}{P}    
  :=    
  (x)\substp{Q}{P} (z)\concat( (R \psubstn{z}{y}) \psubstp{Q}{P} ) \\
  (\lift{x}{R}) \psubstp{Q}{P}  
  :=
  \lift{(x)\substp{Q}{P}}{ R \psubstp{Q}{P} } \\
%   (\dropn{x})  \psubstp{Q}{P}       
%   := 
%   \left\{ 
%     \begin{array}{ccc} 
%       \dropn{\quotep{Q}} & & x \nameeq \quotep{P} \\
%       \dropn{x} & & otherwise \\
%     \end{array}
%   \right. 
  (\dropn{x})  \psubstp{Q}{P}       
  := 
  \left\{ 
    \begin{array}{ccc} 
      Q & & x \nameeq \quotep{P} \\
      \dropn{x} & & otherwise \\
    \end{array}
  \right.
\end{mathpar}
 

where

\begin{eqnarray}
  (x)\id{\{} \lpquote Q \rpquote / \lpquote P \rpquote \id{\}}            = 
  \left\{ 
    \begin{array}{ccc}
      \lpquote Q \rpquote & & x \nameeq \lpquote P \rpquote \\
      x & & otherwise \\
    \end{array}
  \right. \nonumber
\end{eqnarray}

and $z$ is chosen distinct from $\quotep{P}$, $\quotep{Q}$, the free
names in $Q$, and all the names in $R$. Our $\alpha$-equivalence will
be built in the standard way from this substitution.

\begin{remark}\label{rem:no_self_referential_names}
  One consequence of these definitions is that $\forall P. \quotep{P}
  \not\in \freenames{P}$.
\end{remark}

\subsection{ Dynamic quote: an example }

Anticipating something of what's to come, consider applying the
substitution, $\widehat{\id{\{}u / z \id{\}}}$, to the following pair
of processes, $\lift{w}{y!(z)}$ and $w[ \lpquote y!(z) \rpquote ]$.

\begin{eqnarray}
	\lift{w}{y!(z)}\widehat{\id{\{}u / z \id{\}}}
		& = &
		\lift{w}{y!(u)} \nonumber\\
	w[ \lpquote y!(z) \rpquote ] \widehat{ \id{\{}u / z \id{\}} }
		& = &
		w[ \lpquote y!(z) \rpquote ] \nonumber
\end{eqnarray}

Because the body of the process between quotes is impervious to
substitution, we get radically different answers. In fact, by
examining the first process in an input context,
e.g. $x?(z).\lift{w}{y!(z)}$, we see that the process under the lift
operator may be shaped by prefixed inputs binding a name inside it. In
this sense, the lift operator will be seen as a way to dynamically
construct processes before reifying them as names.

Finally equipped with these standard features we can present the
dynamics of the calculus.

\subsubsection{Operational semantics} 

Finally, we introduce the computational dynamics. What marks these
algebras as distinct from other more traditionally studied algebraic
structures, e.g. vector spaces or polynomial rings, is the manner in
which dynamics is captured. In traditional structures, dynamics is typically
expressed through morphisms between such structures, as in linear maps
between vector spaces or morphisms between rings. In algebras
associated with the semantics of computation, the dynamics is
expressed as part of the algebraic structure itself, through a
reduction reduction relation typically denoted by $\red$. Below, we
give a recursive presentation of this relation for the calculus used
in the encoding.

$\red \subseteq \pi \times \pi$
$\red : \pi \to \mathcal{P}(\pi)$

\begin{mathpar}
  \inferrule* [lab=Comm] { \textsf{match}( x_{src}, x_{trgt} ) } { x_{trgt}?(y)P \; | \; x_{src}!\langle {Q} \rangle \red P\{\quotep{Q}/y}\} }
  \and \\
  \inferrule* [lab=Par] {{P} \red {P}'} {{{P} | {Q}} \red {{P}' | {Q}}}
  \and
  \inferrule* [lab=Equiv]{{{P} \scong {P}'} \andalso {{P}' \red {Q}'} \andalso {{Q}' \scong {Q}}}{{P} \red {Q}}
\end{mathpar}

\begin{eqnarray*}
  match_{\equiv} (\quotep{P},\quotep{Q}) & := & P \equiv Q \\
  match_{\dagger}(\quotep{P},\quotep{Q}) & := & \forall R. P|Q \red^{*} R => R \red^{*} 0 \\
  match_{K}(\quotep{P},\quotep{Q}) & := & K \mbox{ for some context } K
\end{eqnarray*}

$u?(x)P | u!\langle Q \rangle \red P\{\quotep{Q}/x\}$

%We write $\wred$ for $\red^*$, and $P\red$ if $\exists Q $ such that $ P \red Q$.
We write $P\red$ if $\exists Q $ such that $ P \red Q$ and $P\not\red$, otherwise.

\section{Replication}

As mentioned before, it is known that replication (and hence
recursion) can be implemented in a higher-order process algebra
\cite{SangiorgiWalker}. As our first example of calculation with the
machinery thus far presented we give the construction explicitly in
the {\rhoc}.

\begin{eqnarray}
	D_{x} & := & \prefix{x}{y}{(\binpar{\outputp{x}{y}}{@{y}})} \nonumber\\
	\bangp_{x}{P} & := & \binpar{{x}!\langle{\binpar{D_{x}}{P}}\rangle}{D_{x}} \nonumber
\end{eqnarray}

\begin{eqnarray}
	\bangp_{x}{P} & & \nonumber\\
	=
	& {x}!\langle{(\prefix{x}{y}{(\outputp{x}{y} | @{y})) | P}}\rangle 
	      | \prefix{x}{y}{(\outputp{x}{y} | @{y})} & \nonumber\\
	\red
	& (\outputp{x}{y} | @{y})\substn{\quotep{(\prefix{x}{y}{(@{y} | \outputp{x}{y})) | P}}}{y} & \nonumber\\
	=
	& \outputp{x}{\quotep{(\prefix{x}{y}{(\outputp{x}{y} | @{y})) | P}}}
	  | {(\prefix{x}{y}{(\outputp{x}{y} | @{y})) | P}} & \nonumber\\
	\red
	& \ldots & \nonumber\\
	\red^*
	& P | P | \ldots & \nonumber
\end{eqnarray}

Of course, this encoding, as an implementation, runs away, unfolding
$\bangp{P}$ eagerly. A lazier and more implementable replication
operator, restricted to input-guarded processes, may be obtained as follows.

\begin{eqnarray}
\bangp{\prefix{u}{v}{P}} 
	:= 
	\binpar{\lift{x}{\prefix{u}{v}{(\binpar{D(x)}{P})}}}{D(x)} \nonumber
\end{eqnarray}

\begin{remark}
  Note that the lazier definition still does not deal with summation
  or mixed summation (i.e. sums over input and output). The reader is
  invited to construct definitions of replication that deal with these
  features. 

  Further, the definitions are parameterized in a name, $x$. Can you,
  gentle reader, make a definition that eliminates this parameter and
  guarantees no accidental interaction between the replication
  machinery and the process being replicated -- i.e. no accidental
  sharing of names used by the process to get its work done and the
  name(s) used by the replication to effect copying. This latter
  revision of the definition of replication is crucial to obtaining
  the expected identity $!!P \sim !P$.
\end{remark}

\begin{remark}\label{rem:paradoxical_combinator}
  The reader familiar with the lambda calculus will have noticed the
  similarity between $D$ and the paradoxical combinator.

  [Ed. note: the existence of this seems to suggest we have to be more
  restrictive on the set of processes and names we admit if we are to
  support no-cloning.]
\end{remark}

\subsubsection{Bisimulation}

The computational dynamics gives rise to another kind of equivalence,
the equivalence of computational behavior. As previously mentioned
this is typically captured \emph{via} some form of bisimulation.

% The notion we use in this paper is weak barbed bisimulation
% \cite{milner91polyadicpi}.

The notion we use in this paper is derived from weak barbed
bisimulation \cite{milner91polyadicpi}. 

\begin{definition}
An \emph{observation relation}, $\downarrow_{\mathcal N}$, over a set
of names, $\mathcal N$, is the smallest relation satisfying the rules
below.

\infrule[Out-barb]{y \in {\mathcal N}, \; x \nameeq y}
		  {\outputp{x}{v} \downarrow_{\mathcal N} x}
\infrule[Par-barb]{\mbox{$P\downarrow_{\mathcal N} x$ or $Q\downarrow_{\mathcal N} x$}}
		  {\binpar{P}{Q} \downarrow_{\mathcal N} x}

We write $P \Downarrow_{\mathcal N} x$ if there is $Q$ such that 
$P \wred Q$ and $Q \downarrow_{\mathcal N} x$.
\end{definition}

\begin{definition}
%\label{def.bbisim}
An  ${\mathcal N}$-\emph{barbed bisimulation} over a set of names, ${\mathcal N}$, is a symmetric binary relation 
${\mathcal S}_{\mathcal N}$ between agents such that $P\rel{S}_{\mathcal N}Q$ implies:
\begin{enumerate}
\item If $P \red P'$ then $Q \wred Q'$ and $P'\rel{S}_{\mathcal N} Q'$.
\item If $P\downarrow_{\mathcal N} x$, then $Q\Downarrow_{\mathcal N} x$.
\end{enumerate}
$P$ is ${\mathcal N}$-barbed bisimilar to $Q$, written
$P \wbbisim_{\mathcal N} Q$, if $P \rel{S}_{\mathcal N} Q$ for some ${\mathcal N}$-barbed bisimulation ${\mathcal S}_{\mathcal N}$.
\end{definition}

$\mathcal{R} \subseteq \pi \times \pi$

$P \mathcal{R} Q => \forall P'. P \red P' \Rightarrow \exists Q'. Q \red Q', P' \mathcal{R} Q'$

$P \vdash x \Rightarrow Q \vdash x$

\begin{mathpar}
  \inferrule*[lab=Out-barb]{x \nameeq y}{{y}!\langle{Q}\rangle \vdash x}
  \and
  \inferrule*[lab=Par-barb]{\mbox{$P\vdash x$ or $Q\vdash x$}}{\binpar{P}{Q} \vdash x}
\end{mathpar}

\subsubsection{Contexts}

One of the principle advantages of computational calculi like the
$\pi$-calculus is a well-defined notion of context,
contextual-equivalence and a correlation between
contextual-equivalence and notions of bisimulation. The notion of
context allows the decomposition of a process into (sub-)process and
its syntactic environment, its context. Thus, a context may be
thought of as a process with a ``hole'' (written $\Box$) in it. The
application of a context $M$ to a process $P$, written $M[P]$, is
tantamount to filling the hole in $M$ with $P$. In this paper we do
not need the full weight of this theory, but do make use of the notion
of context in the proof the main theorem. 

\begin{mathpar}
  \inferrule* [lab=summation] {} {{M_{M},M_{N}} \bc \Box \;|\; x.M_{A} \;|\; M_{M}+M_{N}}
  \and
  \inferrule* [lab=agent] {} {{M_{A}} \bc (\vec{x})M_{P} \;| \; \clift{P_0,\ldots,M_{P},\ldots,P_N}}
  \and \\
  \inferrule* [lab=process] {} {{M_{P}} \bc M_{N} \;| \;P|M_{P} }
\end{mathpar} 

\begin{mathpar}
  \inferrule* [lab=sychronization] {} {M_{N} \bc \Box \;|\; x?M_{F} \;|\; x!M_{C}}
  \and
  \inferrule* [lab=abstraction] {} {{M_{F}} \bc (x)M_{P} }
  \and
  \inferrule* [lab=concretion] {} {{M_{C}} \bc \langle M_{P} \rangle }
  \and \\
  \inferrule* [lab=process] {} {{M_{P}} \bc M_{N} \;| \;P|M_{P} }
\end{mathpar}

\begin{definition}[contextual application] Given a context $M$, and
  process $P$, we define the \emph{contextual application}, $M[P] :=
  M\{P/\Box\}$. That is, the contextual application of M to P is the
  substitution of $P$ for $\Box$ in $M$.
\end{definition}

$\meaningof{-} : L \to \mathcal{P}(\pi)$

\begin{mathpar}
  \inferrule* [lab=collection] {} {\meaningof{true} = \pi, \and \meaningof{~E} = \pi \setminus \meaningof{E}, \and \meaningof{E_{1} \& E_{2}} = \meaningof{E_{1}} \cap \meaningof{E_{2}}}
\end{mathpar}

\begin{mathpar}
  \inferrule* [lab=structure] {} {\meaningof{0} = \{ P \in \pi | P \equiv 0 \}, \and \\ \meaningof{E_1 | E_2} = \{ P \in \pi | P \equiv P_{1} | P_{2}, P_{1} \in \meaningof{E_{1}}, P_{2} \in \meaningof{E_2}\} }
\end{mathpar}

\begin{mathpar}
 \inferrule* [lab=behavior] {} {\meaningof{\langle a?b \rangle E} = \{ P \in \pi | P \equiv Q | u?(y)P', \\ \and \\\\ \and \\ \;\;\; u \in \meaningof{a}, \forall z.P'\{z/y\} \in \meaningof{E\{z/b\}}\}, \and \\ \meaningof{a!E} = \{ P \in \pi | P \equiv Q | x!\langle P' \rangle, x \in \meaningof{a} P' \in \meaningof{E}\} }
\end{mathpar}

\begin{mathpar}
 \inferrule* [lab=nominal] {} {\meaningof{\quotep{E}} = \{ \quotep{P} \in \quotep{\pi} | P \in \meaningof{E} \}, \and \meaningof{\quotep{P}} = \{ \quotep{Q} \in \quotep{\pi} | P \equiv Q \} \and \\ \meaningof{@\quotep{E}} = \{ P \in \pi | P \equiv @x, x \in \meaningof{E} \}}
\end{mathpar}

\begin{eqnarray*}
  \\
  \meaningof{-} : TS \to ST
\end{eqnarray*}

\begin{eqnarray*}
  \\
  L : TS \to ST
\end{eqnarray*}

\begin{eqnarray*}
  \\
  P \models E \iff P \in \meaningof{E}
\end{eqnarray*}

\begin{eqnarray*}
  P \approx_{L} Q \iff \forall E \in L. P \models E \iff Q \models E
\end{eqnarray*}

\begin{eqnarray*}
  P \approx_{K} Q
\end{eqnarray*}

\begin{eqnarray*}
  P \approx Q
\end{eqnarray*}

$\approx_{K} = \approx = \approx_{L}$

\subsubsection{Contextual duality}

Note that contexts extend the quotation operation to a family of
operations from processes to names. Given a context, $M$, we can
define a \emph{nominal context}, $\quotep{M}$ by $\quotep{M}[P] :=
\quotep{M[P]}$. To foreshadow what is to come we observe that these
operations enjoy a duality with processes very much like the duality
between vectors and maps from vectors to scalars.

Further, because the calculus is essentially higher-order, we have a
correspondence between contexts and processes. More specifically,
given a name $x$ and a context $M$ we can construct $M^{*}_{x}$ such
that 

\begin{mathpar}
  M^{*}_{x} | \lift{x}{P} \red M[P]
\end{mathpar}

namely,

\begin{mathpar}
  M^{*}_{x} := x?(u).M[\dropn{u}]
\end{mathpar}

The dependence of $M^{*}_{x}$ on a name makes it an abstraction, 

\begin{mathpar}
  M^{*} := (x)x?(u).M[\dropn{u}]
\end{mathpar}

\subsection{Additional notation}

It will sometimes be convenient to denote the process a name
quotes. We already have the notation $x = \quotep{P}$, but it will be
convenient to introduce an alternate notation, $\procn{x}$, when we
want to emphasize the connection to the use of the name. Note that, by
virtue of name equivalence, $\quotep{\procn{x}} \nameeq x$; so, the
notation is consistent with previous definitions.

Further, because names have structure it is possible to effect
substitutions on the basis of that structure. This means we need to
upgrade our notation for substitutions, which we accomplish by
adapting comprehension notation. Thus,

\begin{mathpar}
  P\{ y / x : x \in S \}
\end{mathpar}

is interpreted to mean the process derived from P by replacing (in a
capture-avoiding manner) each occurrence of $x$ in $S$ by $y$. For example,

\begin{mathpar}
  P\{ \quotep{\procn{x}|\procn{x}} / x : x \in \freenames{P} \}
\end{mathpar}

will replace each (occurrence) of a free name $x$ in $P$ by
$\quotep{\procn{x}|\procn{x}}$.

Also, we will avail ourselves of the notation $x^{L}$ and $x^{R}$ to
denote injections of a name into disjoint copies of the name
space. There are numerous ways to accomplish this. One example can be
found in \cite{MeredithR05}. This notation overloads to vectors of
names: $\vec{x}^{\pi} := (x_{i}^{\pi} \; : \; 0 \leq i < |\vec{x}| )$ where $\pi \in \{L,R\}$.

We also use $P^{\Box} := P|\Box$.

In \cite{MeredithR05} an interpretation of the new operator is
given. It turns out that there are several possible interpretations
all enjoying the requisite algebraic properties of the operator (see
\cite{milner91polyadicpi}). We will therefore make liberal use of
$(\nu\; \vec{x})P$.

% subsection the_syntax_and_semantics_of_the_notation_system (end)   

\input{qm2pi.qmops} 

\input{qm2pi.sterngerlach} 

\input{qm2pi.metric} 

% section concurrent_process_calculi (end)

%\input{qm2pi.proofsketch}

% section proof sketch (end)

%\input{qm2pi.slviaknots} 

% section spatial logic via knots (end)

\input{qm2pi.conclusion}

% section conclusion (end)

%\input{qm2pi.dtcodes} 

% section wiring algorithm (end)

\input{qm2pi.ack} 

% section acknowledgments (end)

\newpage


\bibliographystyle{plain}   
\bibliography{../../biblios/main.bib}

\input{qm2pi.rhodetails}

\end{document}

 

%\documentclass[12pt]{llncs}
%\documentclass{jktr}

\usepackage[pdftex]{hyperref}                   
\usepackage {listings}
\usepackage {mathpartir}
\usepackage{bcprules}
%\usepackage{listings}
                       
\usepackage{graphicx} 
%\usepackage[margins=2.5cm,nohead,nofoot]{geometry}
%\usepackage{geometry}
\usepackage{amsfonts}
\usepackage{amstext}
\usepackage{latexsym}
\usepackage{amssymb}
\usepackage{color}


%\include{myPreamble}
\include{qm2pi.local} 

%\ifpdf
%\usepackage[pdftex]{graphicx}
%\else
%\usepackage{graphicx}
%\fi

 % \ifpdf
%  \usepackage{pdfsync}
%  \if


%\title{Brief Article}
%\author{David F. Snyder}
%\author{L.G. Meredith}

%\address{Dept. of Math., Texas State University--San Marcos, San Marcos, TX 78666}
       
\pagestyle{empty}


\begin{document}

\lstset{language=[Objective]Caml,frame=shadowbox}

\input{qm2pi.front}

% section front matter (end)

\input{qm2pi.intro} 
 
% section introduction (end)

% \input{qm2pi.knotations} 

% section notation (end)

\input{qm2pi.process.calculi} 

% section concurrent_process_calculi_and_spatial_logics_ (end)
    
%\input{qm2pi.knots2pi} 

%\input{qm2pi.trefoil} 

%\input{qm2pi.mainthm} 

% subsection basic_interpretation (end)

%\input{qm2pi.rho.presentation} 
\subsection{The syntax and semantics of the notation system}\label{sub:the_syntax_and_semantics_of_the_notation_system} % (fold)

We now summarize a technical presentation of the calculus that
embodies our theory of dynamics. The typical presentation of such a
calculus follows the style of giving generators and relations on
them. The grammar, below, describing term constructors, freely
generates the set of processes, $\Proc$. This set is then quotiented
by a relation known as structural congruence and it is over this set
that the notion of dynamics is expressed. This presentation is
essentially that of \cite{MeredithR05} with the addition of
polyadicity and summation. For readability we have relegated some of
the technical subtleties to an appendix.

\subsubsection{Process grammar}\label{subsub:process_grammar}

\begin{mathpar}
  \inferrule* [lab=synchronization] {} {{M} \bc \pzero \;|\; x?F \;|\; x!C }
  \and
  \inferrule* [lab=abstraction] {} {{F} \bc (x)P}
  \and
  \inferrule* [lab=concretion] {} {{C} \bc \langle Q \rangle}
  \and
  \inferrule* [lab=process] {} {{P,Q} \bc M \;| \;P|Q \;|\; @{x}}
  \and
  \inferrule* [lab=name] {} {{x} \bc \quotep{P}}
\end{mathpar} 

Note that $\vec{x}$ (resp. $\vec{P}$) denotes a vector of names
(resp. processes) of length $|\vec{x}|$ (resp. $|\vec{P}|$). We adopt
the following useful abbreviations.

\begin{mathpar}
   x?(\vec{y}).P := x.(\vec{y})P \and  x\clift{\vec{P}} := x.\clift{\vec{P}}
   \and x!(y) := \lift{x}{\dropn{y}}
   \and \Pi_{i=0}^{n-1}P_i := P_0 | \ldots | P_{n-1}
\end{mathpar}

\subsubsection{Structural congruence}

\paragraph{Free and bound names and alpha-equivalence.} At the
core of structural equivalence is alpha-equivalence which identifies
process that are the same up to a change of variable. Formally, we
recognize the distinction between free and bound names. The free names
of a process, $\freenames{P}$, may be calculated recursively as
follows:

\begin{mathpar}
\freenames{\pzero} := \emptyset
  \and \\
  \freenames{x?(y).P} := \{ x \} \cup (\freenames{P} \setminus \{ y \})
  \and 
  \freenames{x!\langle P \rangle} := \{ x \} \cup \{ P \} 
  \and \\
  \freenames{P|Q} := \freenames{P} \cup \freenames{Q}
  \and \\
  \freenames{@{x}} := \{ x \}
\end{mathpar}

$\pi$
$\quotep{\pi}$

$\freenames{-} : \pi \to \mathcal{P}(\quotep{\pi})$

\begin{eqnarray*}
  \freenames{\pzero} & := & \emptyset \\
  \freenames{x?(y).P} & := & \{ x \} \cup (\freenames{P} \setminus \{ y \}) \\
  \freenames{x!\langle P \rangle} & := & \{ x \} \cup \{ P \} \\
  \freenames{P|Q} & := & \freenames{P} \cup \freenames{Q} \\
  \freenames{\dropn{x}} & := & \{ x \}
\end{eqnarray*}

The bound names of a process, $\boundnames{P}$, are those names occurring in $P$
that are not free. For example, in $x?(y).0$, the name $x$ is free, while $y$ is bound.

\begin{mathpar}
  \inferrule* [lab=monoidal-laws] {} { P|Q \equiv Q|P \and P|0 \equiv P \and P|(Q|R) \equiv (P|Q)|R }
\end{mathpar}

\begin{mathpar}
  \inferrule* [lab=alpha-equivalence] {} { (x)P \equiv (y)P\{y/x\} \and y \not\in \freenames{P} }
\end{mathpar}

\begin{definition}
Then two processes, $P,Q$, are alpha-equivalent if $P = Q\{\vec{y}/\vec{x}\}$ for
some $\vec{x} \in \boundnames{Q},\vec{y} \in \boundnames{P}$, where $Q\{\vec{y}/\vec{x}\}$
denotes the capture-avoiding substitution of $\vec{y}$ for $\vec{x}$ in $Q$.
\end{definition}

\begin{definition}
  The {\em structural congruence} \cite{SangiorgiWalker} , $\equiv$,
  between processes is the least congruence containing
  alpha-equivalence, satisfying the abelian monoid laws
  (associativity, commutativity and $\pzero$ as identity) for parallel
  composition $|$ and for summation $+$.
\end{definition}

\subsection{Name equivalence}

We take name equivalence, written $\nameeq$, to be the smallest
equivalence relation generated by the following rules.

\begin{mathpar}
\inferrule*[lab=Quote-drop]
{ }
{ \quotep{@{x}} \nameeq x }

\inferrule*[lab=Struct-equiv]
{ P \scong Q }
{ \quotep{P} \nameeq \quotep{Q} }
\end{mathpar}

The astute reader will have noticed that the mutual recursion of names
and processes imposes a mutual recursion on alpha-equivalence and
structural equivalence via name-equivalence. Fortunately, all of this
works out pleasantly and we may calculate in the natural way, free of
concern. The reader interested in the details is referred to the
appendix \ref{appendix:rho_details}.

\subsection{Substitution}

We use $\Proc$ for the set of processes, $\QProc$ for the set of
names, and $\id{\{}\vec{y} / \vec{x} \id{\}}$ to denote partial maps,
$s : \QProc \rightarrow \QProc$. A map, $s$ lifts, uniquely, to a map
on process terms, $\widehat{s} : \Proc \rightarrow \Proc$ by the
following equations.

\begin{mathpar}
  (0) \psubstp{Q}{P} := 0 \\
  (R \juxtap S) \psubstp{Q}{P}
  :=    
  (R)\psubstp{Q}{P} \juxtap (S) \psubstp{Q}{P} \\
  (x?(y).R) \psubstp{Q}{P}    
  :=    
  (x)\substp{Q}{P} (z)\concat( (R \psubstn{z}{y}) \psubstp{Q}{P} ) \\
  (\lift{x}{R}) \psubstp{Q}{P}  
  :=
  \lift{(x)\substp{Q}{P}}{ R \psubstp{Q}{P} } \\
%   (\dropn{x})  \psubstp{Q}{P}       
%   := 
%   \left\{ 
%     \begin{array}{ccc} 
%       \dropn{\quotep{Q}} & & x \nameeq \quotep{P} \\
%       \dropn{x} & & otherwise \\
%     \end{array}
%   \right. 
  (\dropn{x})  \psubstp{Q}{P}       
  := 
  \left\{ 
    \begin{array}{ccc} 
      Q & & x \nameeq \quotep{P} \\
      \dropn{x} & & otherwise \\
    \end{array}
  \right.
\end{mathpar}
 

where

\begin{eqnarray}
  (x)\id{\{} \lpquote Q \rpquote / \lpquote P \rpquote \id{\}}            = 
  \left\{ 
    \begin{array}{ccc}
      \lpquote Q \rpquote & & x \nameeq \lpquote P \rpquote \\
      x & & otherwise \\
    \end{array}
  \right. \nonumber
\end{eqnarray}

and $z$ is chosen distinct from $\quotep{P}$, $\quotep{Q}$, the free
names in $Q$, and all the names in $R$. Our $\alpha$-equivalence will
be built in the standard way from this substitution.

\begin{remark}\label{rem:no_self_referential_names}
  One consequence of these definitions is that $\forall P. \quotep{P}
  \not\in \freenames{P}$.
\end{remark}

\subsection{ Dynamic quote: an example }

Anticipating something of what's to come, consider applying the
substitution, $\widehat{\id{\{}u / z \id{\}}}$, to the following pair
of processes, $\lift{w}{y!(z)}$ and $w[ \lpquote y!(z) \rpquote ]$.

\begin{eqnarray}
	\lift{w}{y!(z)}\widehat{\id{\{}u / z \id{\}}}
		& = &
		\lift{w}{y!(u)} \nonumber\\
	w[ \lpquote y!(z) \rpquote ] \widehat{ \id{\{}u / z \id{\}} }
		& = &
		w[ \lpquote y!(z) \rpquote ] \nonumber
\end{eqnarray}

Because the body of the process between quotes is impervious to
substitution, we get radically different answers. In fact, by
examining the first process in an input context,
e.g. $x?(z).\lift{w}{y!(z)}$, we see that the process under the lift
operator may be shaped by prefixed inputs binding a name inside it. In
this sense, the lift operator will be seen as a way to dynamically
construct processes before reifying them as names.

Finally equipped with these standard features we can present the
dynamics of the calculus.

\subsubsection{Operational semantics} 

Finally, we introduce the computational dynamics. What marks these
algebras as distinct from other more traditionally studied algebraic
structures, e.g. vector spaces or polynomial rings, is the manner in
which dynamics is captured. In traditional structures, dynamics is typically
expressed through morphisms between such structures, as in linear maps
between vector spaces or morphisms between rings. In algebras
associated with the semantics of computation, the dynamics is
expressed as part of the algebraic structure itself, through a
reduction reduction relation typically denoted by $\red$. Below, we
give a recursive presentation of this relation for the calculus used
in the encoding.

$\red \subseteq \pi \times \pi$
$\red : \pi \to \mathcal{P}(\pi)$

\begin{mathpar}
  \inferrule* [lab=Comm] { \textsf{match}( x_{src}, x_{trgt} ) } { x_{trgt}?(y)P \; | \; x_{src}!\langle {Q} \rangle \red P\{\quotep{Q}/y}\} }
  \and \\
  \inferrule* [lab=Par] {{P} \red {P}'} {{{P} | {Q}} \red {{P}' | {Q}}}
  \and
  \inferrule* [lab=Equiv]{{{P} \scong {P}'} \andalso {{P}' \red {Q}'} \andalso {{Q}' \scong {Q}}}{{P} \red {Q}}
\end{mathpar}

\begin{eqnarray*}
  match_{\equiv} (\quotep{P},\quotep{Q}) & := & P \equiv Q \\
  match_{\dagger}(\quotep{P},\quotep{Q}) & := & \forall R. P|Q \red^{*} R => R \red^{*} 0 \\
  match_{K}(\quotep{P},\quotep{Q}) & := & K \mbox{ for some context } K
\end{eqnarray*}

$u?(x)P | u!\langle Q \rangle \red P\{\quotep{Q}/x\}$

%We write $\wred$ for $\red^*$, and $P\red$ if $\exists Q $ such that $ P \red Q$.
We write $P\red$ if $\exists Q $ such that $ P \red Q$ and $P\not\red$, otherwise.

\section{Replication}

As mentioned before, it is known that replication (and hence
recursion) can be implemented in a higher-order process algebra
\cite{SangiorgiWalker}. As our first example of calculation with the
machinery thus far presented we give the construction explicitly in
the {\rhoc}.

\begin{eqnarray}
	D_{x} & := & \prefix{x}{y}{(\binpar{\outputp{x}{y}}{@{y}})} \nonumber\\
	\bangp_{x}{P} & := & \binpar{{x}!\langle{\binpar{D_{x}}{P}}\rangle}{D_{x}} \nonumber
\end{eqnarray}

\begin{eqnarray}
	\bangp_{x}{P} & & \nonumber\\
	=
	& {x}!\langle{(\prefix{x}{y}{(\outputp{x}{y} | @{y})) | P}}\rangle 
	      | \prefix{x}{y}{(\outputp{x}{y} | @{y})} & \nonumber\\
	\red
	& (\outputp{x}{y} | @{y})\substn{\quotep{(\prefix{x}{y}{(@{y} | \outputp{x}{y})) | P}}}{y} & \nonumber\\
	=
	& \outputp{x}{\quotep{(\prefix{x}{y}{(\outputp{x}{y} | @{y})) | P}}}
	  | {(\prefix{x}{y}{(\outputp{x}{y} | @{y})) | P}} & \nonumber\\
	\red
	& \ldots & \nonumber\\
	\red^*
	& P | P | \ldots & \nonumber
\end{eqnarray}

Of course, this encoding, as an implementation, runs away, unfolding
$\bangp{P}$ eagerly. A lazier and more implementable replication
operator, restricted to input-guarded processes, may be obtained as follows.

\begin{eqnarray}
\bangp{\prefix{u}{v}{P}} 
	:= 
	\binpar{\lift{x}{\prefix{u}{v}{(\binpar{D(x)}{P})}}}{D(x)} \nonumber
\end{eqnarray}

\begin{remark}
  Note that the lazier definition still does not deal with summation
  or mixed summation (i.e. sums over input and output). The reader is
  invited to construct definitions of replication that deal with these
  features. 

  Further, the definitions are parameterized in a name, $x$. Can you,
  gentle reader, make a definition that eliminates this parameter and
  guarantees no accidental interaction between the replication
  machinery and the process being replicated -- i.e. no accidental
  sharing of names used by the process to get its work done and the
  name(s) used by the replication to effect copying. This latter
  revision of the definition of replication is crucial to obtaining
  the expected identity $!!P \sim !P$.
\end{remark}

\begin{remark}\label{rem:paradoxical_combinator}
  The reader familiar with the lambda calculus will have noticed the
  similarity between $D$ and the paradoxical combinator.

  [Ed. note: the existence of this seems to suggest we have to be more
  restrictive on the set of processes and names we admit if we are to
  support no-cloning.]
\end{remark}

\subsubsection{Bisimulation}

The computational dynamics gives rise to another kind of equivalence,
the equivalence of computational behavior. As previously mentioned
this is typically captured \emph{via} some form of bisimulation.

% The notion we use in this paper is weak barbed bisimulation
% \cite{milner91polyadicpi}.

The notion we use in this paper is derived from weak barbed
bisimulation \cite{milner91polyadicpi}. 

\begin{definition}
An \emph{observation relation}, $\downarrow_{\mathcal N}$, over a set
of names, $\mathcal N$, is the smallest relation satisfying the rules
below.

\infrule[Out-barb]{y \in {\mathcal N}, \; x \nameeq y}
		  {\outputp{x}{v} \downarrow_{\mathcal N} x}
\infrule[Par-barb]{\mbox{$P\downarrow_{\mathcal N} x$ or $Q\downarrow_{\mathcal N} x$}}
		  {\binpar{P}{Q} \downarrow_{\mathcal N} x}

We write $P \Downarrow_{\mathcal N} x$ if there is $Q$ such that 
$P \wred Q$ and $Q \downarrow_{\mathcal N} x$.
\end{definition}

\begin{definition}
%\label{def.bbisim}
An  ${\mathcal N}$-\emph{barbed bisimulation} over a set of names, ${\mathcal N}$, is a symmetric binary relation 
${\mathcal S}_{\mathcal N}$ between agents such that $P\rel{S}_{\mathcal N}Q$ implies:
\begin{enumerate}
\item If $P \red P'$ then $Q \wred Q'$ and $P'\rel{S}_{\mathcal N} Q'$.
\item If $P\downarrow_{\mathcal N} x$, then $Q\Downarrow_{\mathcal N} x$.
\end{enumerate}
$P$ is ${\mathcal N}$-barbed bisimilar to $Q$, written
$P \wbbisim_{\mathcal N} Q$, if $P \rel{S}_{\mathcal N} Q$ for some ${\mathcal N}$-barbed bisimulation ${\mathcal S}_{\mathcal N}$.
\end{definition}

$\mathcal{R} \subseteq \pi \times \pi$

$P \mathcal{R} Q => \forall P'. P \red P' \Rightarrow \exists Q'. Q \red Q', P' \mathcal{R} Q'$

$P \vdash x \Rightarrow Q \vdash x$

\begin{mathpar}
  \inferrule*[lab=Out-barb]{x \nameeq y}{{y}!\langle{Q}\rangle \vdash x}
  \and
  \inferrule*[lab=Par-barb]{\mbox{$P\vdash x$ or $Q\vdash x$}}{\binpar{P}{Q} \vdash x}
\end{mathpar}

\subsubsection{Contexts}

One of the principle advantages of computational calculi like the
$\pi$-calculus is a well-defined notion of context,
contextual-equivalence and a correlation between
contextual-equivalence and notions of bisimulation. The notion of
context allows the decomposition of a process into (sub-)process and
its syntactic environment, its context. Thus, a context may be
thought of as a process with a ``hole'' (written $\Box$) in it. The
application of a context $M$ to a process $P$, written $M[P]$, is
tantamount to filling the hole in $M$ with $P$. In this paper we do
not need the full weight of this theory, but do make use of the notion
of context in the proof the main theorem. 

\begin{mathpar}
  \inferrule* [lab=summation] {} {{M_{M},M_{N}} \bc \Box \;|\; x.M_{A} \;|\; M_{M}+M_{N}}
  \and
  \inferrule* [lab=agent] {} {{M_{A}} \bc (\vec{x})M_{P} \;| \; \clift{P_0,\ldots,M_{P},\ldots,P_N}}
  \and \\
  \inferrule* [lab=process] {} {{M_{P}} \bc M_{N} \;| \;P|M_{P} }
\end{mathpar} 

\begin{mathpar}
  \inferrule* [lab=sychronization] {} {M_{N} \bc \Box \;|\; x?M_{F} \;|\; x!M_{C}}
  \and
  \inferrule* [lab=abstraction] {} {{M_{F}} \bc (x)M_{P} }
  \and
  \inferrule* [lab=concretion] {} {{M_{C}} \bc \langle M_{P} \rangle }
  \and \\
  \inferrule* [lab=process] {} {{M_{P}} \bc M_{N} \;| \;P|M_{P} }
\end{mathpar}

\begin{definition}[contextual application] Given a context $M$, and
  process $P$, we define the \emph{contextual application}, $M[P] :=
  M\{P/\Box\}$. That is, the contextual application of M to P is the
  substitution of $P$ for $\Box$ in $M$.
\end{definition}

$\meaningof{-} : L \to \mathcal{P}(\pi)$

\begin{mathpar}
  \inferrule* [lab=collection] {} {\meaningof{true} = \pi, \and \meaningof{~E} = \pi \setminus \meaningof{E}, \and \meaningof{E_{1} \& E_{2}} = \meaningof{E_{1}} \cap \meaningof{E_{2}}}
\end{mathpar}

\begin{mathpar}
  \inferrule* [lab=structure] {} {\meaningof{0} = \{ P \in \pi | P \equiv 0 \}, \and \\ \meaningof{E_1 | E_2} = \{ P \in \pi | P \equiv P_{1} | P_{2}, P_{1} \in \meaningof{E_{1}}, P_{2} \in \meaningof{E_2}\} }
\end{mathpar}

\begin{mathpar}
 \inferrule* [lab=behavior] {} {\meaningof{\langle a?b \rangle E} = \{ P \in \pi | P \equiv Q | u?(y)P', \\ \and \\\\ \and \\ \;\;\; u \in \meaningof{a}, \forall z.P'\{z/y\} \in \meaningof{E\{z/b\}}\}, \and \\ \meaningof{a!E} = \{ P \in \pi | P \equiv Q | x!\langle P' \rangle, x \in \meaningof{a} P' \in \meaningof{E}\} }
\end{mathpar}

\begin{mathpar}
 \inferrule* [lab=nominal] {} {\meaningof{\quotep{E}} = \{ \quotep{P} \in \quotep{\pi} | P \in \meaningof{E} \}, \and \meaningof{\quotep{P}} = \{ \quotep{Q} \in \quotep{\pi} | P \equiv Q \} \and \\ \meaningof{@\quotep{E}} = \{ P \in \pi | P \equiv @x, x \in \meaningof{E} \}}
\end{mathpar}

\begin{eqnarray*}
  \\
  \meaningof{-} : TS \to ST
\end{eqnarray*}

\begin{eqnarray*}
  \\
  L : TS \to ST
\end{eqnarray*}

\begin{eqnarray*}
  \\
  P \models E \iff P \in \meaningof{E}
\end{eqnarray*}

\begin{eqnarray*}
  P \approx_{L} Q \iff \forall E \in L. P \models E \iff Q \models E
\end{eqnarray*}

\begin{eqnarray*}
  P \approx_{K} Q
\end{eqnarray*}

\begin{eqnarray*}
  P \approx Q
\end{eqnarray*}

$\approx_{K} = \approx = \approx_{L}$

\subsubsection{Contextual duality}

Note that contexts extend the quotation operation to a family of
operations from processes to names. Given a context, $M$, we can
define a \emph{nominal context}, $\quotep{M}$ by $\quotep{M}[P] :=
\quotep{M[P]}$. To foreshadow what is to come we observe that these
operations enjoy a duality with processes very much like the duality
between vectors and maps from vectors to scalars.

Further, because the calculus is essentially higher-order, we have a
correspondence between contexts and processes. More specifically,
given a name $x$ and a context $M$ we can construct $M^{*}_{x}$ such
that 

\begin{mathpar}
  M^{*}_{x} | \lift{x}{P} \red M[P]
\end{mathpar}

namely,

\begin{mathpar}
  M^{*}_{x} := x?(u).M[\dropn{u}]
\end{mathpar}

The dependence of $M^{*}_{x}$ on a name makes it an abstraction, 

\begin{mathpar}
  M^{*} := (x)x?(u).M[\dropn{u}]
\end{mathpar}

\subsection{Additional notation}

It will sometimes be convenient to denote the process a name
quotes. We already have the notation $x = \quotep{P}$, but it will be
convenient to introduce an alternate notation, $\procn{x}$, when we
want to emphasize the connection to the use of the name. Note that, by
virtue of name equivalence, $\quotep{\procn{x}} \nameeq x$; so, the
notation is consistent with previous definitions.

Further, because names have structure it is possible to effect
substitutions on the basis of that structure. This means we need to
upgrade our notation for substitutions, which we accomplish by
adapting comprehension notation. Thus,

\begin{mathpar}
  P\{ y / x : x \in S \}
\end{mathpar}

is interpreted to mean the process derived from P by replacing (in a
capture-avoiding manner) each occurrence of $x$ in $S$ by $y$. For example,

\begin{mathpar}
  P\{ \quotep{\procn{x}|\procn{x}} / x : x \in \freenames{P} \}
\end{mathpar}

will replace each (occurrence) of a free name $x$ in $P$ by
$\quotep{\procn{x}|\procn{x}}$.

Also, we will avail ourselves of the notation $x^{L}$ and $x^{R}$ to
denote injections of a name into disjoint copies of the name
space. There are numerous ways to accomplish this. One example can be
found in \cite{MeredithR05}. This notation overloads to vectors of
names: $\vec{x}^{\pi} := (x_{i}^{\pi} \; : \; 0 \leq i < |\vec{x}| )$ where $\pi \in \{L,R\}$.

We also use $P^{\Box} := P|\Box$.

In \cite{MeredithR05} an interpretation of the new operator is
given. It turns out that there are several possible interpretations
all enjoying the requisite algebraic properties of the operator (see
\cite{milner91polyadicpi}). We will therefore make liberal use of
$(\nu\; \vec{x})P$.

% subsection the_syntax_and_semantics_of_the_notation_system (end)   

\input{qm2pi.qmops} 

\input{qm2pi.sterngerlach} 

\input{qm2pi.metric} 

% section concurrent_process_calculi (end)

%\input{qm2pi.proofsketch}

% section proof sketch (end)

%\input{qm2pi.slviaknots} 

% section spatial logic via knots (end)

\input{qm2pi.conclusion}

% section conclusion (end)

%\input{qm2pi.dtcodes} 

% section wiring algorithm (end)

\input{qm2pi.ack} 

% section acknowledgments (end)

\newpage


\bibliographystyle{plain}   
\bibliography{../../biblios/main.bib}

\input{qm2pi.rhodetails}

\end{document}

 

% subsection basic_interpretation (end)

%\input{qm2pi.rho.presentation} 
\subsection{The syntax and semantics of the notation system}\label{sub:the_syntax_and_semantics_of_the_notation_system} % (fold)

We now summarize a technical presentation of the calculus that
embodies our theory of dynamics. The typical presentation of such a
calculus follows the style of giving generators and relations on
them. The grammar, below, describing term constructors, freely
generates the set of processes, $\Proc$. This set is then quotiented
by a relation known as structural congruence and it is over this set
that the notion of dynamics is expressed. This presentation is
essentially that of \cite{MeredithR05} with the addition of
polyadicity and summation. For readability we have relegated some of
the technical subtleties to an appendix.

\subsubsection{Process grammar}\label{subsub:process_grammar}

\begin{mathpar}
  \inferrule* [lab=synchronization] {} {{M} \bc \pzero \;|\; x?F \;|\; x!C }
  \and
  \inferrule* [lab=abstraction] {} {{F} \bc (x)P}
  \and
  \inferrule* [lab=concretion] {} {{C} \bc \langle Q \rangle}
  \and
  \inferrule* [lab=process] {} {{P,Q} \bc M \;| \;P|Q \;|\; @{x}}
  \and
  \inferrule* [lab=name] {} {{x} \bc \quotep{P}}
\end{mathpar} 

Note that $\vec{x}$ (resp. $\vec{P}$) denotes a vector of names
(resp. processes) of length $|\vec{x}|$ (resp. $|\vec{P}|$). We adopt
the following useful abbreviations.

\begin{mathpar}
   x?(\vec{y}).P := x.(\vec{y})P \and  x\clift{\vec{P}} := x.\clift{\vec{P}}
   \and x!(y) := \lift{x}{\dropn{y}}
   \and \Pi_{i=0}^{n-1}P_i := P_0 | \ldots | P_{n-1}
\end{mathpar}

\subsubsection{Structural congruence}

\paragraph{Free and bound names and alpha-equivalence.} At the
core of structural equivalence is alpha-equivalence which identifies
process that are the same up to a change of variable. Formally, we
recognize the distinction between free and bound names. The free names
of a process, $\freenames{P}$, may be calculated recursively as
follows:

\begin{mathpar}
\freenames{\pzero} := \emptyset
  \and \\
  \freenames{x?(y).P} := \{ x \} \cup (\freenames{P} \setminus \{ y \})
  \and 
  \freenames{x!\langle P \rangle} := \{ x \} \cup \{ P \} 
  \and \\
  \freenames{P|Q} := \freenames{P} \cup \freenames{Q}
  \and \\
  \freenames{@{x}} := \{ x \}
\end{mathpar}

$\pi$
$\quotep{\pi}$

$\freenames{-} : \pi \to \mathcal{P}(\quotep{\pi})$

\begin{eqnarray*}
  \freenames{\pzero} & := & \emptyset \\
  \freenames{x?(y).P} & := & \{ x \} \cup (\freenames{P} \setminus \{ y \}) \\
  \freenames{x!\langle P \rangle} & := & \{ x \} \cup \{ P \} \\
  \freenames{P|Q} & := & \freenames{P} \cup \freenames{Q} \\
  \freenames{\dropn{x}} & := & \{ x \}
\end{eqnarray*}

The bound names of a process, $\boundnames{P}$, are those names occurring in $P$
that are not free. For example, in $x?(y).0$, the name $x$ is free, while $y$ is bound.

\begin{mathpar}
  \inferrule* [lab=monoidal-laws] {} { P|Q \equiv Q|P \and P|0 \equiv P \and P|(Q|R) \equiv (P|Q)|R }
\end{mathpar}

\begin{mathpar}
  \inferrule* [lab=alpha-equivalence] {} { (x)P \equiv (y)P\{y/x\} \and y \not\in \freenames{P} }
\end{mathpar}

\begin{definition}
Then two processes, $P,Q$, are alpha-equivalent if $P = Q\{\vec{y}/\vec{x}\}$ for
some $\vec{x} \in \boundnames{Q},\vec{y} \in \boundnames{P}$, where $Q\{\vec{y}/\vec{x}\}$
denotes the capture-avoiding substitution of $\vec{y}$ for $\vec{x}$ in $Q$.
\end{definition}

\begin{definition}
  The {\em structural congruence} \cite{SangiorgiWalker} , $\equiv$,
  between processes is the least congruence containing
  alpha-equivalence, satisfying the abelian monoid laws
  (associativity, commutativity and $\pzero$ as identity) for parallel
  composition $|$ and for summation $+$.
\end{definition}

\subsection{Name equivalence}

We take name equivalence, written $\nameeq$, to be the smallest
equivalence relation generated by the following rules.

\begin{mathpar}
\inferrule*[lab=Quote-drop]
{ }
{ \quotep{@{x}} \nameeq x }

\inferrule*[lab=Struct-equiv]
{ P \scong Q }
{ \quotep{P} \nameeq \quotep{Q} }
\end{mathpar}

The astute reader will have noticed that the mutual recursion of names
and processes imposes a mutual recursion on alpha-equivalence and
structural equivalence via name-equivalence. Fortunately, all of this
works out pleasantly and we may calculate in the natural way, free of
concern. The reader interested in the details is referred to the
appendix \ref{appendix:rho_details}.

\subsection{Substitution}

We use $\Proc$ for the set of processes, $\QProc$ for the set of
names, and $\id{\{}\vec{y} / \vec{x} \id{\}}$ to denote partial maps,
$s : \QProc \rightarrow \QProc$. A map, $s$ lifts, uniquely, to a map
on process terms, $\widehat{s} : \Proc \rightarrow \Proc$ by the
following equations.

\begin{mathpar}
  (0) \psubstp{Q}{P} := 0 \\
  (R \juxtap S) \psubstp{Q}{P}
  :=    
  (R)\psubstp{Q}{P} \juxtap (S) \psubstp{Q}{P} \\
  (x?(y).R) \psubstp{Q}{P}    
  :=    
  (x)\substp{Q}{P} (z)\concat( (R \psubstn{z}{y}) \psubstp{Q}{P} ) \\
  (\lift{x}{R}) \psubstp{Q}{P}  
  :=
  \lift{(x)\substp{Q}{P}}{ R \psubstp{Q}{P} } \\
%   (\dropn{x})  \psubstp{Q}{P}       
%   := 
%   \left\{ 
%     \begin{array}{ccc} 
%       \dropn{\quotep{Q}} & & x \nameeq \quotep{P} \\
%       \dropn{x} & & otherwise \\
%     \end{array}
%   \right. 
  (\dropn{x})  \psubstp{Q}{P}       
  := 
  \left\{ 
    \begin{array}{ccc} 
      Q & & x \nameeq \quotep{P} \\
      \dropn{x} & & otherwise \\
    \end{array}
  \right.
\end{mathpar}
 

where

\begin{eqnarray}
  (x)\id{\{} \lpquote Q \rpquote / \lpquote P \rpquote \id{\}}            = 
  \left\{ 
    \begin{array}{ccc}
      \lpquote Q \rpquote & & x \nameeq \lpquote P \rpquote \\
      x & & otherwise \\
    \end{array}
  \right. \nonumber
\end{eqnarray}

and $z$ is chosen distinct from $\quotep{P}$, $\quotep{Q}$, the free
names in $Q$, and all the names in $R$. Our $\alpha$-equivalence will
be built in the standard way from this substitution.

\begin{remark}\label{rem:no_self_referential_names}
  One consequence of these definitions is that $\forall P. \quotep{P}
  \not\in \freenames{P}$.
\end{remark}

\subsection{ Dynamic quote: an example }

Anticipating something of what's to come, consider applying the
substitution, $\widehat{\id{\{}u / z \id{\}}}$, to the following pair
of processes, $\lift{w}{y!(z)}$ and $w[ \lpquote y!(z) \rpquote ]$.

\begin{eqnarray}
	\lift{w}{y!(z)}\widehat{\id{\{}u / z \id{\}}}
		& = &
		\lift{w}{y!(u)} \nonumber\\
	w[ \lpquote y!(z) \rpquote ] \widehat{ \id{\{}u / z \id{\}} }
		& = &
		w[ \lpquote y!(z) \rpquote ] \nonumber
\end{eqnarray}

Because the body of the process between quotes is impervious to
substitution, we get radically different answers. In fact, by
examining the first process in an input context,
e.g. $x?(z).\lift{w}{y!(z)}$, we see that the process under the lift
operator may be shaped by prefixed inputs binding a name inside it. In
this sense, the lift operator will be seen as a way to dynamically
construct processes before reifying them as names.

Finally equipped with these standard features we can present the
dynamics of the calculus.

\subsubsection{Operational semantics} 

Finally, we introduce the computational dynamics. What marks these
algebras as distinct from other more traditionally studied algebraic
structures, e.g. vector spaces or polynomial rings, is the manner in
which dynamics is captured. In traditional structures, dynamics is typically
expressed through morphisms between such structures, as in linear maps
between vector spaces or morphisms between rings. In algebras
associated with the semantics of computation, the dynamics is
expressed as part of the algebraic structure itself, through a
reduction reduction relation typically denoted by $\red$. Below, we
give a recursive presentation of this relation for the calculus used
in the encoding.

$\red \subseteq \pi \times \pi$
$\red : \pi \to \mathcal{P}(\pi)$

\begin{mathpar}
  \inferrule* [lab=Comm] { \textsf{match}( x_{src}, x_{trgt} ) } { x_{trgt}?(y)P \; | \; x_{src}!\langle {Q} \rangle \red P\{\quotep{Q}/y}\} }
  \and \\
  \inferrule* [lab=Par] {{P} \red {P}'} {{{P} | {Q}} \red {{P}' | {Q}}}
  \and
  \inferrule* [lab=Equiv]{{{P} \scong {P}'} \andalso {{P}' \red {Q}'} \andalso {{Q}' \scong {Q}}}{{P} \red {Q}}
\end{mathpar}

\begin{eqnarray*}
  match_{\equiv} (\quotep{P},\quotep{Q}) & := & P \equiv Q \\
  match_{\dagger}(\quotep{P},\quotep{Q}) & := & \forall R. P|Q \red^{*} R => R \red^{*} 0 \\
  match_{K}(\quotep{P},\quotep{Q}) & := & K \mbox{ for some context } K
\end{eqnarray*}

$u?(x)P | u!\langle Q \rangle \red P\{\quotep{Q}/x\}$

%We write $\wred$ for $\red^*$, and $P\red$ if $\exists Q $ such that $ P \red Q$.
We write $P\red$ if $\exists Q $ such that $ P \red Q$ and $P\not\red$, otherwise.

\section{Replication}

As mentioned before, it is known that replication (and hence
recursion) can be implemented in a higher-order process algebra
\cite{SangiorgiWalker}. As our first example of calculation with the
machinery thus far presented we give the construction explicitly in
the {\rhoc}.

\begin{eqnarray}
	D_{x} & := & \prefix{x}{y}{(\binpar{\outputp{x}{y}}{@{y}})} \nonumber\\
	\bangp_{x}{P} & := & \binpar{{x}!\langle{\binpar{D_{x}}{P}}\rangle}{D_{x}} \nonumber
\end{eqnarray}

\begin{eqnarray}
	\bangp_{x}{P} & & \nonumber\\
	=
	& {x}!\langle{(\prefix{x}{y}{(\outputp{x}{y} | @{y})) | P}}\rangle 
	      | \prefix{x}{y}{(\outputp{x}{y} | @{y})} & \nonumber\\
	\red
	& (\outputp{x}{y} | @{y})\substn{\quotep{(\prefix{x}{y}{(@{y} | \outputp{x}{y})) | P}}}{y} & \nonumber\\
	=
	& \outputp{x}{\quotep{(\prefix{x}{y}{(\outputp{x}{y} | @{y})) | P}}}
	  | {(\prefix{x}{y}{(\outputp{x}{y} | @{y})) | P}} & \nonumber\\
	\red
	& \ldots & \nonumber\\
	\red^*
	& P | P | \ldots & \nonumber
\end{eqnarray}

Of course, this encoding, as an implementation, runs away, unfolding
$\bangp{P}$ eagerly. A lazier and more implementable replication
operator, restricted to input-guarded processes, may be obtained as follows.

\begin{eqnarray}
\bangp{\prefix{u}{v}{P}} 
	:= 
	\binpar{\lift{x}{\prefix{u}{v}{(\binpar{D(x)}{P})}}}{D(x)} \nonumber
\end{eqnarray}

\begin{remark}
  Note that the lazier definition still does not deal with summation
  or mixed summation (i.e. sums over input and output). The reader is
  invited to construct definitions of replication that deal with these
  features. 

  Further, the definitions are parameterized in a name, $x$. Can you,
  gentle reader, make a definition that eliminates this parameter and
  guarantees no accidental interaction between the replication
  machinery and the process being replicated -- i.e. no accidental
  sharing of names used by the process to get its work done and the
  name(s) used by the replication to effect copying. This latter
  revision of the definition of replication is crucial to obtaining
  the expected identity $!!P \sim !P$.
\end{remark}

\begin{remark}\label{rem:paradoxical_combinator}
  The reader familiar with the lambda calculus will have noticed the
  similarity between $D$ and the paradoxical combinator.

  [Ed. note: the existence of this seems to suggest we have to be more
  restrictive on the set of processes and names we admit if we are to
  support no-cloning.]
\end{remark}

\subsubsection{Bisimulation}

The computational dynamics gives rise to another kind of equivalence,
the equivalence of computational behavior. As previously mentioned
this is typically captured \emph{via} some form of bisimulation.

% The notion we use in this paper is weak barbed bisimulation
% \cite{milner91polyadicpi}.

The notion we use in this paper is derived from weak barbed
bisimulation \cite{milner91polyadicpi}. 

\begin{definition}
An \emph{observation relation}, $\downarrow_{\mathcal N}$, over a set
of names, $\mathcal N$, is the smallest relation satisfying the rules
below.

\infrule[Out-barb]{y \in {\mathcal N}, \; x \nameeq y}
		  {\outputp{x}{v} \downarrow_{\mathcal N} x}
\infrule[Par-barb]{\mbox{$P\downarrow_{\mathcal N} x$ or $Q\downarrow_{\mathcal N} x$}}
		  {\binpar{P}{Q} \downarrow_{\mathcal N} x}

We write $P \Downarrow_{\mathcal N} x$ if there is $Q$ such that 
$P \wred Q$ and $Q \downarrow_{\mathcal N} x$.
\end{definition}

\begin{definition}
%\label{def.bbisim}
An  ${\mathcal N}$-\emph{barbed bisimulation} over a set of names, ${\mathcal N}$, is a symmetric binary relation 
${\mathcal S}_{\mathcal N}$ between agents such that $P\rel{S}_{\mathcal N}Q$ implies:
\begin{enumerate}
\item If $P \red P'$ then $Q \wred Q'$ and $P'\rel{S}_{\mathcal N} Q'$.
\item If $P\downarrow_{\mathcal N} x$, then $Q\Downarrow_{\mathcal N} x$.
\end{enumerate}
$P$ is ${\mathcal N}$-barbed bisimilar to $Q$, written
$P \wbbisim_{\mathcal N} Q$, if $P \rel{S}_{\mathcal N} Q$ for some ${\mathcal N}$-barbed bisimulation ${\mathcal S}_{\mathcal N}$.
\end{definition}

$\mathcal{R} \subseteq \pi \times \pi$

$P \mathcal{R} Q => \forall P'. P \red P' \Rightarrow \exists Q'. Q \red Q', P' \mathcal{R} Q'$

$P \vdash x \Rightarrow Q \vdash x$

\begin{mathpar}
  \inferrule*[lab=Out-barb]{x \nameeq y}{{y}!\langle{Q}\rangle \vdash x}
  \and
  \inferrule*[lab=Par-barb]{\mbox{$P\vdash x$ or $Q\vdash x$}}{\binpar{P}{Q} \vdash x}
\end{mathpar}

\subsubsection{Contexts}

One of the principle advantages of computational calculi like the
$\pi$-calculus is a well-defined notion of context,
contextual-equivalence and a correlation between
contextual-equivalence and notions of bisimulation. The notion of
context allows the decomposition of a process into (sub-)process and
its syntactic environment, its context. Thus, a context may be
thought of as a process with a ``hole'' (written $\Box$) in it. The
application of a context $M$ to a process $P$, written $M[P]$, is
tantamount to filling the hole in $M$ with $P$. In this paper we do
not need the full weight of this theory, but do make use of the notion
of context in the proof the main theorem. 

\begin{mathpar}
  \inferrule* [lab=summation] {} {{M_{M},M_{N}} \bc \Box \;|\; x.M_{A} \;|\; M_{M}+M_{N}}
  \and
  \inferrule* [lab=agent] {} {{M_{A}} \bc (\vec{x})M_{P} \;| \; \clift{P_0,\ldots,M_{P},\ldots,P_N}}
  \and \\
  \inferrule* [lab=process] {} {{M_{P}} \bc M_{N} \;| \;P|M_{P} }
\end{mathpar} 

\begin{mathpar}
  \inferrule* [lab=sychronization] {} {M_{N} \bc \Box \;|\; x?M_{F} \;|\; x!M_{C}}
  \and
  \inferrule* [lab=abstraction] {} {{M_{F}} \bc (x)M_{P} }
  \and
  \inferrule* [lab=concretion] {} {{M_{C}} \bc \langle M_{P} \rangle }
  \and \\
  \inferrule* [lab=process] {} {{M_{P}} \bc M_{N} \;| \;P|M_{P} }
\end{mathpar}

\begin{definition}[contextual application] Given a context $M$, and
  process $P$, we define the \emph{contextual application}, $M[P] :=
  M\{P/\Box\}$. That is, the contextual application of M to P is the
  substitution of $P$ for $\Box$ in $M$.
\end{definition}

$\meaningof{-} : L \to \mathcal{P}(\pi)$

\begin{mathpar}
  \inferrule* [lab=collection] {} {\meaningof{true} = \pi, \and \meaningof{~E} = \pi \setminus \meaningof{E}, \and \meaningof{E_{1} \& E_{2}} = \meaningof{E_{1}} \cap \meaningof{E_{2}}}
\end{mathpar}

\begin{mathpar}
  \inferrule* [lab=structure] {} {\meaningof{0} = \{ P \in \pi | P \equiv 0 \}, \and \\ \meaningof{E_1 | E_2} = \{ P \in \pi | P \equiv P_{1} | P_{2}, P_{1} \in \meaningof{E_{1}}, P_{2} \in \meaningof{E_2}\} }
\end{mathpar}

\begin{mathpar}
 \inferrule* [lab=behavior] {} {\meaningof{\langle a?b \rangle E} = \{ P \in \pi | P \equiv Q | u?(y)P', \\ \and \\\\ \and \\ \;\;\; u \in \meaningof{a}, \forall z.P'\{z/y\} \in \meaningof{E\{z/b\}}\}, \and \\ \meaningof{a!E} = \{ P \in \pi | P \equiv Q | x!\langle P' \rangle, x \in \meaningof{a} P' \in \meaningof{E}\} }
\end{mathpar}

\begin{mathpar}
 \inferrule* [lab=nominal] {} {\meaningof{\quotep{E}} = \{ \quotep{P} \in \quotep{\pi} | P \in \meaningof{E} \}, \and \meaningof{\quotep{P}} = \{ \quotep{Q} \in \quotep{\pi} | P \equiv Q \} \and \\ \meaningof{@\quotep{E}} = \{ P \in \pi | P \equiv @x, x \in \meaningof{E} \}}
\end{mathpar}

\begin{eqnarray*}
  \\
  \meaningof{-} : TS \to ST
\end{eqnarray*}

\begin{eqnarray*}
  \\
  L : TS \to ST
\end{eqnarray*}

\begin{eqnarray*}
  \\
  P \models E \iff P \in \meaningof{E}
\end{eqnarray*}

\begin{eqnarray*}
  P \approx_{L} Q \iff \forall E \in L. P \models E \iff Q \models E
\end{eqnarray*}

\begin{eqnarray*}
  P \approx_{K} Q
\end{eqnarray*}

\begin{eqnarray*}
  P \approx Q
\end{eqnarray*}

$\approx_{K} = \approx = \approx_{L}$

\subsubsection{Contextual duality}

Note that contexts extend the quotation operation to a family of
operations from processes to names. Given a context, $M$, we can
define a \emph{nominal context}, $\quotep{M}$ by $\quotep{M}[P] :=
\quotep{M[P]}$. To foreshadow what is to come we observe that these
operations enjoy a duality with processes very much like the duality
between vectors and maps from vectors to scalars.

Further, because the calculus is essentially higher-order, we have a
correspondence between contexts and processes. More specifically,
given a name $x$ and a context $M$ we can construct $M^{*}_{x}$ such
that 

\begin{mathpar}
  M^{*}_{x} | \lift{x}{P} \red M[P]
\end{mathpar}

namely,

\begin{mathpar}
  M^{*}_{x} := x?(u).M[\dropn{u}]
\end{mathpar}

The dependence of $M^{*}_{x}$ on a name makes it an abstraction, 

\begin{mathpar}
  M^{*} := (x)x?(u).M[\dropn{u}]
\end{mathpar}

\subsection{Additional notation}

It will sometimes be convenient to denote the process a name
quotes. We already have the notation $x = \quotep{P}$, but it will be
convenient to introduce an alternate notation, $\procn{x}$, when we
want to emphasize the connection to the use of the name. Note that, by
virtue of name equivalence, $\quotep{\procn{x}} \nameeq x$; so, the
notation is consistent with previous definitions.

Further, because names have structure it is possible to effect
substitutions on the basis of that structure. This means we need to
upgrade our notation for substitutions, which we accomplish by
adapting comprehension notation. Thus,

\begin{mathpar}
  P\{ y / x : x \in S \}
\end{mathpar}

is interpreted to mean the process derived from P by replacing (in a
capture-avoiding manner) each occurrence of $x$ in $S$ by $y$. For example,

\begin{mathpar}
  P\{ \quotep{\procn{x}|\procn{x}} / x : x \in \freenames{P} \}
\end{mathpar}

will replace each (occurrence) of a free name $x$ in $P$ by
$\quotep{\procn{x}|\procn{x}}$.

Also, we will avail ourselves of the notation $x^{L}$ and $x^{R}$ to
denote injections of a name into disjoint copies of the name
space. There are numerous ways to accomplish this. One example can be
found in \cite{MeredithR05}. This notation overloads to vectors of
names: $\vec{x}^{\pi} := (x_{i}^{\pi} \; : \; 0 \leq i < |\vec{x}| )$ where $\pi \in \{L,R\}$.

We also use $P^{\Box} := P|\Box$.

In \cite{MeredithR05} an interpretation of the new operator is
given. It turns out that there are several possible interpretations
all enjoying the requisite algebraic properties of the operator (see
\cite{milner91polyadicpi}). We will therefore make liberal use of
$(\nu\; \vec{x})P$.

% subsection the_syntax_and_semantics_of_the_notation_system (end)   

\section{Interpretation of QM}
\subsection{Supporting definitions}
\subsubsection{Multiplication}
\begin{mathpar}
  \quotep{Q} \cdot \quotep{R} := \quotep{Q|R}
  \and \\
  \quotep{Q} \cdot P := P\{ \quotep{Q|R} / \quotep{R} : \quotep{R} \in \freenames{P} \}
\end{mathpar}

\paragraph{Discussion}
The first line needs little explanation. The second line says that
each free name of the process is replaced with the multiplication of
that name by the scalar. Multiplication of a scalar (name) by a state
(process) results in a process all the names of which have been `moved
over' by parallel composition with the process the scalar
quotes. There is a subtlety that the bound names have to be
manipulated so that multiplied names aren't accidentally
captured. There are many ways to achieve this.

\begin{remark}\label{rem:multiplication_identities}
  The reader is invited to verify that for all $x,y,z \in \QProc$ and $P \in \Proc$
  \begin{mathpar}
    x \cdot \quotep{0} \equiv x 
    \and
    x \cdot y \equiv y \cdot x
    \and
    x \cdot (y \cdot z) \equiv (x \cdot y) \cdot z
    \and \\
    \quotep{0} \cdot P \equiv P
    \and \\
    x \cdot (y \cdot P) \equiv (x \cdot y) \cdot P
    \and \\
    x \cdot (P|Q) \equiv (x \cdot P) | (x \cdot Q)
    \and \\    
  \end{mathpar}
\end{remark}

\subsubsection{Tensor product}

We define a tensor product on processes by structural induction.

\paragraph{Tensor of sums} First note that all summations, including
$\pzero$ and sequence, can be written $\Sigma_{i} x_{i}.A_{i} +
\Sigma_{j} x_{j}.C_{j}$, where we have grouped input-guarded processes
together and output-guarded processes together.

Thus, we can define the tensor product of two summations, $N_{1}\otimes N_{2}$, where

\begin{mathpar}
  N_{1} := \Sigma_{i} x_{i}.A_{i} + \Sigma_{j} x_{j}.C_{j}
  \and
  N_{2} := \Sigma_{i'} y_{i'}.B_{i'} + \Sigma_{j'} y_{j'}.D_{j'} 
\end{mathpar}

as follows.

\begin{mathpar}
  \Sigma_{i} x_{i}.A_{i} + \Sigma_{j} x_{j}.C_{j} \otimes \Sigma_{i'}
  y_{i'}.B_{i'} + \Sigma_{j'} y_{j'}.D_{j'} 
  \and \\
  := \; \Sigma_{i} \Sigma_{i'} \quotep{\stackrel{\vee}{x_{i}}| \stackrel{\vee}{y_{i'}}}.(A_{i}\otimes B_{i'}) \; | \; \Sigma_{i'} \Sigma_{i} \quotep{\stackrel{\vee}{y_{i'}}|\stackrel{\vee}{x_{i}}}.(B_{i'}\otimes A_{i})
  \and
  \;\; | \;\; \Sigma_{j} \Sigma_{j'} \quotep{\stackrel{\vee}{x_{j}}|\stackrel{\vee}{y_{j'}}}.(A_{j}\otimes B_{j'}) \; | \; \Sigma_{j'} \Sigma_{j} \quotep{\stackrel{\vee}{y_{j'}}|\stackrel{\vee}{x_{j}}}.(B_{j'}\otimes A_{j})
\end{mathpar}

\begin{remark}
  Do we need to $x^{L}$ and $y^{R}$ for this construction as well?
\end{remark}

\paragraph{Tensor of parallel compositions} Next, we distribute tensor
over par.

\begin{mathpar}
  P_{1}|P_{2} \otimes Q_{1}|Q_{2} := (P_{1} \otimes Q_{1}) | (P_{1}
  \otimes Q_{2}) | (P_{2} \otimes Q_{1}) | (P_{2} \otimes Q_{2})
\end{mathpar}

\paragraph{Tensor with dropped names} We treat tensor of a
process with a dropped name as parallel composition.

\begin{mathpar}
  P \otimes \dropn{x} := P | \dropn{x}
\end{mathpar}

\paragraph{Tensor of agents}

Finally, we need to define tensor on agents. Note that the definition
of tensor on normal products only tensors inputs with inputs and
outputs with outputs. Thus, we only have to define the operation on
``homogeneous'' pairings.

\begin{mathpar}
  (\vec{x})P \otimes (\vec{y})Q
  \and \\
  := (x_{0}^{L}|y_{0}^{R},\ldots,x_{0}^{L}|y_{n}^{R},\ldots,x_{m}^{L}|y_{0}^{R},\ldots,x_{m}^{L}|y_{n}^R)(P\{ \vec{x}^{L}/\vec{x}\} \otimes Q \{ \vec{y}^{R}/\vec{y}\})
  \and \\
  \clift{\vec{P}} \otimes \clift{\vec{Q}}
  \and \\
  := \clift{P_{0}\otimes Q_{0},\ldots,P_{0}\otimes Q_{n},\ldots,P_{m}\otimes Q_{0},\ldots,P_{m}\otimes Q_{n}}
\end{mathpar}

\begin{remark}
  Observe that arities of tensored abstractions matches arities of
  tensored concretions if the original arities matched. Note also that
  the length of the arities corresponds to the increase in dimension
  we see in ordinary vector space tensor product.
\end{remark}

\begin{remark}
  Operationally, this definition distributes the tensor down to
  components ``linked'' by summation. Tensor over summation is
  intriguing in that it mixes names. Moreover, as a consequence of the
  way it mixes names we have the identities for all $x \in \QProc$ and
  $P,Q \in \Proc$

  \begin{mathpar}
    (x \cdot P) \otimes Q \equiv x \cdot (P \otimes Q) \equiv P \otimes (x \cdot Q)
    \and
    P \otimes \pzero \equiv P
  \end{mathpar}

  that the reader is invited to verify.
\end{remark}

\subsubsection{Annihilation}
\begin{mathpar}
  P^{\perp} := \{ Q | \forall R. P|Q \red^{*} R \Rightarrow R \red^{*} \pzero \}
  \and \\
  P^{\underline{\perp}} := \Sigma_{Q \in P^{\perp}} \quotep{Q}?(y).(\dropn{y}|Q) | \Sigma_{Q \in P^{\perp}} \quotep{Q}\clift{\Box}
\end{mathpar}

\paragraph{Discussion} The reader will note that $P^{\perp}$ is a
\emph{set} of processes, while $P^{\underline{\perp}}$ is a
\emph{context}. We call the set $P^{\perp}$ the \emph{annihilators} of
$P$. The parallel composition of a process in the annihilators of $P$
with $P$ will result in a process, the state space of which has all
paths eventually leading to $\pzero$. Execution may endure loops; but
under reasonable conditions of fairness (naturally guaranteed under
most notions of bisimulation) such a composite process cannot get
stuck in such a loop and will, eventually pop out and terminate.

The context $P^{\underline{\perp}}$ is ready and willing to ``take the
$P$ out of'' the process to which it is applied. It will effectively
transmit the code of the process to which it is applied to one of the
annihilators and run the process against it.

\subsubsection{Evaluation}
We fix $M$ a domain of fully abstract interpretation with an equality
coincident with bisimulation. We take $\meaningof{\cdot} : \Proc \to
M$ to be the map interpreting processes and $\nmeaningof{\cdot} : \M
\to Proc$ to be the map running the other way. Then we define

\begin{mathpar}
  \int P := \nmeaningof{\meaningof{P}}
\end{mathpar}

\paragraph{Discussion}
There are many fully abstract interpretations of Milner's
$\pi$-calculus. Any of them can be used as a basis for interpreting
the reflective calculus here. Equipped with such a domain it is
largely a matter of grinding through to check that the Yoneda
construction for the normalization-by-evaluation program can be
extended to this setting.

\begin{remark}
  The reader is invited to verify that $\int (P^{\underline{\perp}}[P]) = 0$.
\end{remark}

\subsection{Quantum mechanics}

Table \ref{tbl:core_qm_op_defns} gives the core operational definitions

\begin{table}[htp]\label{tbl:core_qm_op_defns}
  \center{
    \fbox{
      \begin{tabular}{c|c}
        quantum mechanics & process calculus \\
        \hline
        scalar & $x := \quotep{P}$ \\
        state vector & $\state{P} := P$ \\
        dual & $\state{P}^{*} := \event{P^{\underline{\perp}}} := \quotep{P^{\underline{\perp}}}[-]$ \\
        matrix & $ \Sigma_{\alpha} \state{P_{\alpha}}x_{\alpha}\event{Q_{\alpha}}$ \\
        vector addition & $\state{P} + \state{Q} := \state{P | Q}$ \\
        tensor product & $\state{P} \otimes \state{Q} := \state{P \otimes Q}$ \\
        inner product & $\innerprod{P}{Q} := \quotep{\int P^{\underline{\perp}}[Q]}$ \\
      \end{tabular}
    }
  }
  \caption{QM - operational definitions}
\end{table}

where

\begin{mathpar}
  \prmatrix{P}{Q} := \fprmatrix{P}{\quotep{\pzero}}{Q}
  \and
  \fprmatrix{P}{x}{Q} := (\state{P},x,\event{Q})
  \and
  (\fprmatrix{P}{x}{Q})(\state{R}) := x \cdot \innerprod{Q}{R} \cdot \state{P}
  \and
  (\fprmatrix{P}{x}{Q})(\event{R}) := x \cdot \innerprod{R}{P} \cdot \event{Q}
\end{mathpar}

\paragraph{Discussion}
As promised: vectors (aka states) are represented as processes; duals
as contextual duals; inner product definition should be compared with
standard inner product definition for ....

\begin{remark}
  Assuming $\int (P^{\underline{\perp}}[P]) = 0$, the reader is
  invited to verify that $(\fprmatrix{P}{x}{P})(\state{P}) = x \cdot \state{P}$.
\end{remark}

\begin{remark}
  The reader is invited to verify that $\innerprod{P}{Q}$ could
  equally well have been written $\quotep{\int \stackrel{\vee}{x}}$
  where $x = \event{P^{\underline{\perp}}}(Q)$.

  One of the motivations for this remark is that there is another way
  to factor these operations. We could package up evaluation in the dual:

  \begin{mathpar}
    \state{P}^{*} := \event{\int P^{\underline{\perp}}} := \quotep{\int P^{\underline{\perp}}}[-]
  \end{mathpar}

  and then have inner product defined by
  
  \begin{mathpar}
    \innerprod{P}{Q} := \event{P}(Q)
  \end{mathpar}

  Hopefully, experience with the calculations will provide guidance on
  the best factoring.
\end{remark}

\begin{remark}
  Assuming $\int (P^{\underline{\perp}}[P]) = 0$, the reader is
  invited to verify that $\forall P,Q. (\prmatrix{0}{Q})(\state{0}) =
  \state{0}$ and dually $(\prmatrix{P}{0})(\event{0}) = \event{0}$.
\end{remark}

\begin{remark}
  i'm a little worried that i don't (yet) have proper support for
  complex conjugacy. But, the observation above may give us a
  clue. According to Abramsky, it must be the case that the scalars
  are iso to the homset of the identity for the tensor -- which the
  observation above characterizes. 

  For now, we will simply bookmark the notion with $\overline{x}$.
\end{remark}

\subsubsection{Adjointness}

We need to give a definition of $(\cdot)^{\dagger}$ for matrices. The
obvious candidate definition is
\begin{mathpar}
(\Sigma_{\alpha}\fprmatrix{P_{\alpha}}{x_{\alpha}}{Q_{\alpha}})^{\dagger}
= \Sigma_{\alpha}\fprmatrix{(Q_{\alpha}^{\underline{\perp}})^{*}}{\overline{x}_{\alpha}}{P_{\alpha}^{\underline{\perp}}} 
\end{mathpar}

But, $(Q_{\alpha}^{\underline{\perp}})^{*}$ requires a name along
which to communicate the process to achieve the context application.

\subsubsection{Basis for a basis}
If processes label states and ``addition'' of states (a.k.a. vector
addition) is interpreted as parallel composition, what corresponds to
notions of linear independence and basis? Here, we recall that Yoshida
has developed a set of \emph{combinators} for an asynchronous verison
of Milner's $\pi$-calculus. These are a finite set of processes such
any process can be expressed as parallel composition of these
combinators together with liberal uses of the new operator and
replication. We can simply give a translation of these into the
present calculus and have reasonable expectation that the property
carries over. That is, that the resultant set allows to express all
processes via parallel composition. Note, however, that there is no
new operator or replication in this calculus. As a result, we expect
that the corresponding set is actually infinite. That is, we expect
that the space is actually infinite dimensional.

\begin{remark}
  The attentive reader may be a bit concerned. Certainly, the
  collection $S$, $K$ and $I$ is a finite set of
  combinators. Shouldn't we expect to see a finite set of combinators
  for an effectively equivalent system? i am very sympathetic to this
  critique and feel it warrants full attention. On the other hand, i
  also have in mind the following analogy. The natural numbers, as a
  monoid under addition, has exactly $1$ generator, while the natural
  numbers, as a monoid under multiplication, has countably many
  generators (the primes). We observe that the application of the
  lambda calculus is much less resource sensitive than the parallel
  composition of the $\pi$-calculus. Could it be the case that we have
  an analogy of the form
  
  \begin{mathpar}
    m + n : MN :: m*n : M|N
  \end{mathpar}

  giving a similar blow up in the set of ``primes''?  This is such a
  wonderful thought that, even if it's not true, i think it's worth
  writing down.
\end{remark}
 

\documentclass[12pt]{llncs}
%\documentclass{jktr}

\usepackage[pdftex]{hyperref}                   
\usepackage {listings}
\usepackage {mathpartir}
\usepackage{bcprules}
%\usepackage{listings}
                       
\usepackage{graphicx} 
%\usepackage[margins=2.5cm,nohead,nofoot]{geometry}
%\usepackage{geometry}
\usepackage{amsfonts}
\usepackage{amstext}
\usepackage{latexsym}
\usepackage{amssymb}
\usepackage{color}


%\include{myPreamble}
\include{qm2pi.local} 

%\ifpdf
%\usepackage[pdftex]{graphicx}
%\else
%\usepackage{graphicx}
%\fi

 % \ifpdf
%  \usepackage{pdfsync}
%  \if


%\title{Brief Article}
%\author{David F. Snyder}
%\author{L.G. Meredith}

%\address{Dept. of Math., Texas State University--San Marcos, San Marcos, TX 78666}
       
\pagestyle{empty}


\begin{document}

\lstset{language=[Objective]Caml,frame=shadowbox}

\input{qm2pi.front}

% section front matter (end)

\input{qm2pi.intro} 
 
% section introduction (end)

% \input{qm2pi.knotations} 

% section notation (end)

\input{qm2pi.process.calculi} 

% section concurrent_process_calculi_and_spatial_logics_ (end)
    
%\input{qm2pi.knots2pi} 

%\input{qm2pi.trefoil} 

%\input{qm2pi.mainthm} 

% subsection basic_interpretation (end)

%\input{qm2pi.rho.presentation} 
\subsection{The syntax and semantics of the notation system}\label{sub:the_syntax_and_semantics_of_the_notation_system} % (fold)

We now summarize a technical presentation of the calculus that
embodies our theory of dynamics. The typical presentation of such a
calculus follows the style of giving generators and relations on
them. The grammar, below, describing term constructors, freely
generates the set of processes, $\Proc$. This set is then quotiented
by a relation known as structural congruence and it is over this set
that the notion of dynamics is expressed. This presentation is
essentially that of \cite{MeredithR05} with the addition of
polyadicity and summation. For readability we have relegated some of
the technical subtleties to an appendix.

\subsubsection{Process grammar}\label{subsub:process_grammar}

\begin{mathpar}
  \inferrule* [lab=synchronization] {} {{M} \bc \pzero \;|\; x?F \;|\; x!C }
  \and
  \inferrule* [lab=abstraction] {} {{F} \bc (x)P}
  \and
  \inferrule* [lab=concretion] {} {{C} \bc \langle Q \rangle}
  \and
  \inferrule* [lab=process] {} {{P,Q} \bc M \;| \;P|Q \;|\; @{x}}
  \and
  \inferrule* [lab=name] {} {{x} \bc \quotep{P}}
\end{mathpar} 

Note that $\vec{x}$ (resp. $\vec{P}$) denotes a vector of names
(resp. processes) of length $|\vec{x}|$ (resp. $|\vec{P}|$). We adopt
the following useful abbreviations.

\begin{mathpar}
   x?(\vec{y}).P := x.(\vec{y})P \and  x\clift{\vec{P}} := x.\clift{\vec{P}}
   \and x!(y) := \lift{x}{\dropn{y}}
   \and \Pi_{i=0}^{n-1}P_i := P_0 | \ldots | P_{n-1}
\end{mathpar}

\subsubsection{Structural congruence}

\paragraph{Free and bound names and alpha-equivalence.} At the
core of structural equivalence is alpha-equivalence which identifies
process that are the same up to a change of variable. Formally, we
recognize the distinction between free and bound names. The free names
of a process, $\freenames{P}$, may be calculated recursively as
follows:

\begin{mathpar}
\freenames{\pzero} := \emptyset
  \and \\
  \freenames{x?(y).P} := \{ x \} \cup (\freenames{P} \setminus \{ y \})
  \and 
  \freenames{x!\langle P \rangle} := \{ x \} \cup \{ P \} 
  \and \\
  \freenames{P|Q} := \freenames{P} \cup \freenames{Q}
  \and \\
  \freenames{@{x}} := \{ x \}
\end{mathpar}

$\pi$
$\quotep{\pi}$

$\freenames{-} : \pi \to \mathcal{P}(\quotep{\pi})$

\begin{eqnarray*}
  \freenames{\pzero} & := & \emptyset \\
  \freenames{x?(y).P} & := & \{ x \} \cup (\freenames{P} \setminus \{ y \}) \\
  \freenames{x!\langle P \rangle} & := & \{ x \} \cup \{ P \} \\
  \freenames{P|Q} & := & \freenames{P} \cup \freenames{Q} \\
  \freenames{\dropn{x}} & := & \{ x \}
\end{eqnarray*}

The bound names of a process, $\boundnames{P}$, are those names occurring in $P$
that are not free. For example, in $x?(y).0$, the name $x$ is free, while $y$ is bound.

\begin{mathpar}
  \inferrule* [lab=monoidal-laws] {} { P|Q \equiv Q|P \and P|0 \equiv P \and P|(Q|R) \equiv (P|Q)|R }
\end{mathpar}

\begin{mathpar}
  \inferrule* [lab=alpha-equivalence] {} { (x)P \equiv (y)P\{y/x\} \and y \not\in \freenames{P} }
\end{mathpar}

\begin{definition}
Then two processes, $P,Q$, are alpha-equivalent if $P = Q\{\vec{y}/\vec{x}\}$ for
some $\vec{x} \in \boundnames{Q},\vec{y} \in \boundnames{P}$, where $Q\{\vec{y}/\vec{x}\}$
denotes the capture-avoiding substitution of $\vec{y}$ for $\vec{x}$ in $Q$.
\end{definition}

\begin{definition}
  The {\em structural congruence} \cite{SangiorgiWalker} , $\equiv$,
  between processes is the least congruence containing
  alpha-equivalence, satisfying the abelian monoid laws
  (associativity, commutativity and $\pzero$ as identity) for parallel
  composition $|$ and for summation $+$.
\end{definition}

\subsection{Name equivalence}

We take name equivalence, written $\nameeq$, to be the smallest
equivalence relation generated by the following rules.

\begin{mathpar}
\inferrule*[lab=Quote-drop]
{ }
{ \quotep{@{x}} \nameeq x }

\inferrule*[lab=Struct-equiv]
{ P \scong Q }
{ \quotep{P} \nameeq \quotep{Q} }
\end{mathpar}

The astute reader will have noticed that the mutual recursion of names
and processes imposes a mutual recursion on alpha-equivalence and
structural equivalence via name-equivalence. Fortunately, all of this
works out pleasantly and we may calculate in the natural way, free of
concern. The reader interested in the details is referred to the
appendix \ref{appendix:rho_details}.

\subsection{Substitution}

We use $\Proc$ for the set of processes, $\QProc$ for the set of
names, and $\id{\{}\vec{y} / \vec{x} \id{\}}$ to denote partial maps,
$s : \QProc \rightarrow \QProc$. A map, $s$ lifts, uniquely, to a map
on process terms, $\widehat{s} : \Proc \rightarrow \Proc$ by the
following equations.

\begin{mathpar}
  (0) \psubstp{Q}{P} := 0 \\
  (R \juxtap S) \psubstp{Q}{P}
  :=    
  (R)\psubstp{Q}{P} \juxtap (S) \psubstp{Q}{P} \\
  (x?(y).R) \psubstp{Q}{P}    
  :=    
  (x)\substp{Q}{P} (z)\concat( (R \psubstn{z}{y}) \psubstp{Q}{P} ) \\
  (\lift{x}{R}) \psubstp{Q}{P}  
  :=
  \lift{(x)\substp{Q}{P}}{ R \psubstp{Q}{P} } \\
%   (\dropn{x})  \psubstp{Q}{P}       
%   := 
%   \left\{ 
%     \begin{array}{ccc} 
%       \dropn{\quotep{Q}} & & x \nameeq \quotep{P} \\
%       \dropn{x} & & otherwise \\
%     \end{array}
%   \right. 
  (\dropn{x})  \psubstp{Q}{P}       
  := 
  \left\{ 
    \begin{array}{ccc} 
      Q & & x \nameeq \quotep{P} \\
      \dropn{x} & & otherwise \\
    \end{array}
  \right.
\end{mathpar}
 

where

\begin{eqnarray}
  (x)\id{\{} \lpquote Q \rpquote / \lpquote P \rpquote \id{\}}            = 
  \left\{ 
    \begin{array}{ccc}
      \lpquote Q \rpquote & & x \nameeq \lpquote P \rpquote \\
      x & & otherwise \\
    \end{array}
  \right. \nonumber
\end{eqnarray}

and $z$ is chosen distinct from $\quotep{P}$, $\quotep{Q}$, the free
names in $Q$, and all the names in $R$. Our $\alpha$-equivalence will
be built in the standard way from this substitution.

\begin{remark}\label{rem:no_self_referential_names}
  One consequence of these definitions is that $\forall P. \quotep{P}
  \not\in \freenames{P}$.
\end{remark}

\subsection{ Dynamic quote: an example }

Anticipating something of what's to come, consider applying the
substitution, $\widehat{\id{\{}u / z \id{\}}}$, to the following pair
of processes, $\lift{w}{y!(z)}$ and $w[ \lpquote y!(z) \rpquote ]$.

\begin{eqnarray}
	\lift{w}{y!(z)}\widehat{\id{\{}u / z \id{\}}}
		& = &
		\lift{w}{y!(u)} \nonumber\\
	w[ \lpquote y!(z) \rpquote ] \widehat{ \id{\{}u / z \id{\}} }
		& = &
		w[ \lpquote y!(z) \rpquote ] \nonumber
\end{eqnarray}

Because the body of the process between quotes is impervious to
substitution, we get radically different answers. In fact, by
examining the first process in an input context,
e.g. $x?(z).\lift{w}{y!(z)}$, we see that the process under the lift
operator may be shaped by prefixed inputs binding a name inside it. In
this sense, the lift operator will be seen as a way to dynamically
construct processes before reifying them as names.

Finally equipped with these standard features we can present the
dynamics of the calculus.

\subsubsection{Operational semantics} 

Finally, we introduce the computational dynamics. What marks these
algebras as distinct from other more traditionally studied algebraic
structures, e.g. vector spaces or polynomial rings, is the manner in
which dynamics is captured. In traditional structures, dynamics is typically
expressed through morphisms between such structures, as in linear maps
between vector spaces or morphisms between rings. In algebras
associated with the semantics of computation, the dynamics is
expressed as part of the algebraic structure itself, through a
reduction reduction relation typically denoted by $\red$. Below, we
give a recursive presentation of this relation for the calculus used
in the encoding.

$\red \subseteq \pi \times \pi$
$\red : \pi \to \mathcal{P}(\pi)$

\begin{mathpar}
  \inferrule* [lab=Comm] { \textsf{match}( x_{src}, x_{trgt} ) } { x_{trgt}?(y)P \; | \; x_{src}!\langle {Q} \rangle \red P\{\quotep{Q}/y}\} }
  \and \\
  \inferrule* [lab=Par] {{P} \red {P}'} {{{P} | {Q}} \red {{P}' | {Q}}}
  \and
  \inferrule* [lab=Equiv]{{{P} \scong {P}'} \andalso {{P}' \red {Q}'} \andalso {{Q}' \scong {Q}}}{{P} \red {Q}}
\end{mathpar}

\begin{eqnarray*}
  match_{\equiv} (\quotep{P},\quotep{Q}) & := & P \equiv Q \\
  match_{\dagger}(\quotep{P},\quotep{Q}) & := & \forall R. P|Q \red^{*} R => R \red^{*} 0 \\
  match_{K}(\quotep{P},\quotep{Q}) & := & K \mbox{ for some context } K
\end{eqnarray*}

$u?(x)P | u!\langle Q \rangle \red P\{\quotep{Q}/x\}$

%We write $\wred$ for $\red^*$, and $P\red$ if $\exists Q $ such that $ P \red Q$.
We write $P\red$ if $\exists Q $ such that $ P \red Q$ and $P\not\red$, otherwise.

\section{Replication}

As mentioned before, it is known that replication (and hence
recursion) can be implemented in a higher-order process algebra
\cite{SangiorgiWalker}. As our first example of calculation with the
machinery thus far presented we give the construction explicitly in
the {\rhoc}.

\begin{eqnarray}
	D_{x} & := & \prefix{x}{y}{(\binpar{\outputp{x}{y}}{@{y}})} \nonumber\\
	\bangp_{x}{P} & := & \binpar{{x}!\langle{\binpar{D_{x}}{P}}\rangle}{D_{x}} \nonumber
\end{eqnarray}

\begin{eqnarray}
	\bangp_{x}{P} & & \nonumber\\
	=
	& {x}!\langle{(\prefix{x}{y}{(\outputp{x}{y} | @{y})) | P}}\rangle 
	      | \prefix{x}{y}{(\outputp{x}{y} | @{y})} & \nonumber\\
	\red
	& (\outputp{x}{y} | @{y})\substn{\quotep{(\prefix{x}{y}{(@{y} | \outputp{x}{y})) | P}}}{y} & \nonumber\\
	=
	& \outputp{x}{\quotep{(\prefix{x}{y}{(\outputp{x}{y} | @{y})) | P}}}
	  | {(\prefix{x}{y}{(\outputp{x}{y} | @{y})) | P}} & \nonumber\\
	\red
	& \ldots & \nonumber\\
	\red^*
	& P | P | \ldots & \nonumber
\end{eqnarray}

Of course, this encoding, as an implementation, runs away, unfolding
$\bangp{P}$ eagerly. A lazier and more implementable replication
operator, restricted to input-guarded processes, may be obtained as follows.

\begin{eqnarray}
\bangp{\prefix{u}{v}{P}} 
	:= 
	\binpar{\lift{x}{\prefix{u}{v}{(\binpar{D(x)}{P})}}}{D(x)} \nonumber
\end{eqnarray}

\begin{remark}
  Note that the lazier definition still does not deal with summation
  or mixed summation (i.e. sums over input and output). The reader is
  invited to construct definitions of replication that deal with these
  features. 

  Further, the definitions are parameterized in a name, $x$. Can you,
  gentle reader, make a definition that eliminates this parameter and
  guarantees no accidental interaction between the replication
  machinery and the process being replicated -- i.e. no accidental
  sharing of names used by the process to get its work done and the
  name(s) used by the replication to effect copying. This latter
  revision of the definition of replication is crucial to obtaining
  the expected identity $!!P \sim !P$.
\end{remark}

\begin{remark}\label{rem:paradoxical_combinator}
  The reader familiar with the lambda calculus will have noticed the
  similarity between $D$ and the paradoxical combinator.

  [Ed. note: the existence of this seems to suggest we have to be more
  restrictive on the set of processes and names we admit if we are to
  support no-cloning.]
\end{remark}

\subsubsection{Bisimulation}

The computational dynamics gives rise to another kind of equivalence,
the equivalence of computational behavior. As previously mentioned
this is typically captured \emph{via} some form of bisimulation.

% The notion we use in this paper is weak barbed bisimulation
% \cite{milner91polyadicpi}.

The notion we use in this paper is derived from weak barbed
bisimulation \cite{milner91polyadicpi}. 

\begin{definition}
An \emph{observation relation}, $\downarrow_{\mathcal N}$, over a set
of names, $\mathcal N$, is the smallest relation satisfying the rules
below.

\infrule[Out-barb]{y \in {\mathcal N}, \; x \nameeq y}
		  {\outputp{x}{v} \downarrow_{\mathcal N} x}
\infrule[Par-barb]{\mbox{$P\downarrow_{\mathcal N} x$ or $Q\downarrow_{\mathcal N} x$}}
		  {\binpar{P}{Q} \downarrow_{\mathcal N} x}

We write $P \Downarrow_{\mathcal N} x$ if there is $Q$ such that 
$P \wred Q$ and $Q \downarrow_{\mathcal N} x$.
\end{definition}

\begin{definition}
%\label{def.bbisim}
An  ${\mathcal N}$-\emph{barbed bisimulation} over a set of names, ${\mathcal N}$, is a symmetric binary relation 
${\mathcal S}_{\mathcal N}$ between agents such that $P\rel{S}_{\mathcal N}Q$ implies:
\begin{enumerate}
\item If $P \red P'$ then $Q \wred Q'$ and $P'\rel{S}_{\mathcal N} Q'$.
\item If $P\downarrow_{\mathcal N} x$, then $Q\Downarrow_{\mathcal N} x$.
\end{enumerate}
$P$ is ${\mathcal N}$-barbed bisimilar to $Q$, written
$P \wbbisim_{\mathcal N} Q$, if $P \rel{S}_{\mathcal N} Q$ for some ${\mathcal N}$-barbed bisimulation ${\mathcal S}_{\mathcal N}$.
\end{definition}

$\mathcal{R} \subseteq \pi \times \pi$

$P \mathcal{R} Q => \forall P'. P \red P' \Rightarrow \exists Q'. Q \red Q', P' \mathcal{R} Q'$

$P \vdash x \Rightarrow Q \vdash x$

\begin{mathpar}
  \inferrule*[lab=Out-barb]{x \nameeq y}{{y}!\langle{Q}\rangle \vdash x}
  \and
  \inferrule*[lab=Par-barb]{\mbox{$P\vdash x$ or $Q\vdash x$}}{\binpar{P}{Q} \vdash x}
\end{mathpar}

\subsubsection{Contexts}

One of the principle advantages of computational calculi like the
$\pi$-calculus is a well-defined notion of context,
contextual-equivalence and a correlation between
contextual-equivalence and notions of bisimulation. The notion of
context allows the decomposition of a process into (sub-)process and
its syntactic environment, its context. Thus, a context may be
thought of as a process with a ``hole'' (written $\Box$) in it. The
application of a context $M$ to a process $P$, written $M[P]$, is
tantamount to filling the hole in $M$ with $P$. In this paper we do
not need the full weight of this theory, but do make use of the notion
of context in the proof the main theorem. 

\begin{mathpar}
  \inferrule* [lab=summation] {} {{M_{M},M_{N}} \bc \Box \;|\; x.M_{A} \;|\; M_{M}+M_{N}}
  \and
  \inferrule* [lab=agent] {} {{M_{A}} \bc (\vec{x})M_{P} \;| \; \clift{P_0,\ldots,M_{P},\ldots,P_N}}
  \and \\
  \inferrule* [lab=process] {} {{M_{P}} \bc M_{N} \;| \;P|M_{P} }
\end{mathpar} 

\begin{mathpar}
  \inferrule* [lab=sychronization] {} {M_{N} \bc \Box \;|\; x?M_{F} \;|\; x!M_{C}}
  \and
  \inferrule* [lab=abstraction] {} {{M_{F}} \bc (x)M_{P} }
  \and
  \inferrule* [lab=concretion] {} {{M_{C}} \bc \langle M_{P} \rangle }
  \and \\
  \inferrule* [lab=process] {} {{M_{P}} \bc M_{N} \;| \;P|M_{P} }
\end{mathpar}

\begin{definition}[contextual application] Given a context $M$, and
  process $P$, we define the \emph{contextual application}, $M[P] :=
  M\{P/\Box\}$. That is, the contextual application of M to P is the
  substitution of $P$ for $\Box$ in $M$.
\end{definition}

$\meaningof{-} : L \to \mathcal{P}(\pi)$

\begin{mathpar}
  \inferrule* [lab=collection] {} {\meaningof{true} = \pi, \and \meaningof{~E} = \pi \setminus \meaningof{E}, \and \meaningof{E_{1} \& E_{2}} = \meaningof{E_{1}} \cap \meaningof{E_{2}}}
\end{mathpar}

\begin{mathpar}
  \inferrule* [lab=structure] {} {\meaningof{0} = \{ P \in \pi | P \equiv 0 \}, \and \\ \meaningof{E_1 | E_2} = \{ P \in \pi | P \equiv P_{1} | P_{2}, P_{1} \in \meaningof{E_{1}}, P_{2} \in \meaningof{E_2}\} }
\end{mathpar}

\begin{mathpar}
 \inferrule* [lab=behavior] {} {\meaningof{\langle a?b \rangle E} = \{ P \in \pi | P \equiv Q | u?(y)P', \\ \and \\\\ \and \\ \;\;\; u \in \meaningof{a}, \forall z.P'\{z/y\} \in \meaningof{E\{z/b\}}\}, \and \\ \meaningof{a!E} = \{ P \in \pi | P \equiv Q | x!\langle P' \rangle, x \in \meaningof{a} P' \in \meaningof{E}\} }
\end{mathpar}

\begin{mathpar}
 \inferrule* [lab=nominal] {} {\meaningof{\quotep{E}} = \{ \quotep{P} \in \quotep{\pi} | P \in \meaningof{E} \}, \and \meaningof{\quotep{P}} = \{ \quotep{Q} \in \quotep{\pi} | P \equiv Q \} \and \\ \meaningof{@\quotep{E}} = \{ P \in \pi | P \equiv @x, x \in \meaningof{E} \}}
\end{mathpar}

\begin{eqnarray*}
  \\
  \meaningof{-} : TS \to ST
\end{eqnarray*}

\begin{eqnarray*}
  \\
  L : TS \to ST
\end{eqnarray*}

\begin{eqnarray*}
  \\
  P \models E \iff P \in \meaningof{E}
\end{eqnarray*}

\begin{eqnarray*}
  P \approx_{L} Q \iff \forall E \in L. P \models E \iff Q \models E
\end{eqnarray*}

\begin{eqnarray*}
  P \approx_{K} Q
\end{eqnarray*}

\begin{eqnarray*}
  P \approx Q
\end{eqnarray*}

$\approx_{K} = \approx = \approx_{L}$

\subsubsection{Contextual duality}

Note that contexts extend the quotation operation to a family of
operations from processes to names. Given a context, $M$, we can
define a \emph{nominal context}, $\quotep{M}$ by $\quotep{M}[P] :=
\quotep{M[P]}$. To foreshadow what is to come we observe that these
operations enjoy a duality with processes very much like the duality
between vectors and maps from vectors to scalars.

Further, because the calculus is essentially higher-order, we have a
correspondence between contexts and processes. More specifically,
given a name $x$ and a context $M$ we can construct $M^{*}_{x}$ such
that 

\begin{mathpar}
  M^{*}_{x} | \lift{x}{P} \red M[P]
\end{mathpar}

namely,

\begin{mathpar}
  M^{*}_{x} := x?(u).M[\dropn{u}]
\end{mathpar}

The dependence of $M^{*}_{x}$ on a name makes it an abstraction, 

\begin{mathpar}
  M^{*} := (x)x?(u).M[\dropn{u}]
\end{mathpar}

\subsection{Additional notation}

It will sometimes be convenient to denote the process a name
quotes. We already have the notation $x = \quotep{P}$, but it will be
convenient to introduce an alternate notation, $\procn{x}$, when we
want to emphasize the connection to the use of the name. Note that, by
virtue of name equivalence, $\quotep{\procn{x}} \nameeq x$; so, the
notation is consistent with previous definitions.

Further, because names have structure it is possible to effect
substitutions on the basis of that structure. This means we need to
upgrade our notation for substitutions, which we accomplish by
adapting comprehension notation. Thus,

\begin{mathpar}
  P\{ y / x : x \in S \}
\end{mathpar}

is interpreted to mean the process derived from P by replacing (in a
capture-avoiding manner) each occurrence of $x$ in $S$ by $y$. For example,

\begin{mathpar}
  P\{ \quotep{\procn{x}|\procn{x}} / x : x \in \freenames{P} \}
\end{mathpar}

will replace each (occurrence) of a free name $x$ in $P$ by
$\quotep{\procn{x}|\procn{x}}$.

Also, we will avail ourselves of the notation $x^{L}$ and $x^{R}$ to
denote injections of a name into disjoint copies of the name
space. There are numerous ways to accomplish this. One example can be
found in \cite{MeredithR05}. This notation overloads to vectors of
names: $\vec{x}^{\pi} := (x_{i}^{\pi} \; : \; 0 \leq i < |\vec{x}| )$ where $\pi \in \{L,R\}$.

We also use $P^{\Box} := P|\Box$.

In \cite{MeredithR05} an interpretation of the new operator is
given. It turns out that there are several possible interpretations
all enjoying the requisite algebraic properties of the operator (see
\cite{milner91polyadicpi}). We will therefore make liberal use of
$(\nu\; \vec{x})P$.

% subsection the_syntax_and_semantics_of_the_notation_system (end)   

\input{qm2pi.qmops} 

\input{qm2pi.sterngerlach} 

\input{qm2pi.metric} 

% section concurrent_process_calculi (end)

%\input{qm2pi.proofsketch}

% section proof sketch (end)

%\input{qm2pi.slviaknots} 

% section spatial logic via knots (end)

\input{qm2pi.conclusion}

% section conclusion (end)

%\input{qm2pi.dtcodes} 

% section wiring algorithm (end)

\input{qm2pi.ack} 

% section acknowledgments (end)

\newpage


\bibliographystyle{plain}   
\bibliography{../../biblios/main.bib}

\input{qm2pi.rhodetails}

\end{document}

 

\documentclass[12pt]{llncs}
%\documentclass{jktr}

\usepackage[pdftex]{hyperref}                   
\usepackage {listings}
\usepackage {mathpartir}
\usepackage{bcprules}
%\usepackage{listings}
                       
\usepackage{graphicx} 
%\usepackage[margins=2.5cm,nohead,nofoot]{geometry}
%\usepackage{geometry}
\usepackage{amsfonts}
\usepackage{amstext}
\usepackage{latexsym}
\usepackage{amssymb}
\usepackage{color}


%\include{myPreamble}
\include{qm2pi.local} 

%\ifpdf
%\usepackage[pdftex]{graphicx}
%\else
%\usepackage{graphicx}
%\fi

 % \ifpdf
%  \usepackage{pdfsync}
%  \if


%\title{Brief Article}
%\author{David F. Snyder}
%\author{L.G. Meredith}

%\address{Dept. of Math., Texas State University--San Marcos, San Marcos, TX 78666}
       
\pagestyle{empty}


\begin{document}

\lstset{language=[Objective]Caml,frame=shadowbox}

\input{qm2pi.front}

% section front matter (end)

\input{qm2pi.intro} 
 
% section introduction (end)

% \input{qm2pi.knotations} 

% section notation (end)

\input{qm2pi.process.calculi} 

% section concurrent_process_calculi_and_spatial_logics_ (end)
    
%\input{qm2pi.knots2pi} 

%\input{qm2pi.trefoil} 

%\input{qm2pi.mainthm} 

% subsection basic_interpretation (end)

%\input{qm2pi.rho.presentation} 
\subsection{The syntax and semantics of the notation system}\label{sub:the_syntax_and_semantics_of_the_notation_system} % (fold)

We now summarize a technical presentation of the calculus that
embodies our theory of dynamics. The typical presentation of such a
calculus follows the style of giving generators and relations on
them. The grammar, below, describing term constructors, freely
generates the set of processes, $\Proc$. This set is then quotiented
by a relation known as structural congruence and it is over this set
that the notion of dynamics is expressed. This presentation is
essentially that of \cite{MeredithR05} with the addition of
polyadicity and summation. For readability we have relegated some of
the technical subtleties to an appendix.

\subsubsection{Process grammar}\label{subsub:process_grammar}

\begin{mathpar}
  \inferrule* [lab=synchronization] {} {{M} \bc \pzero \;|\; x?F \;|\; x!C }
  \and
  \inferrule* [lab=abstraction] {} {{F} \bc (x)P}
  \and
  \inferrule* [lab=concretion] {} {{C} \bc \langle Q \rangle}
  \and
  \inferrule* [lab=process] {} {{P,Q} \bc M \;| \;P|Q \;|\; @{x}}
  \and
  \inferrule* [lab=name] {} {{x} \bc \quotep{P}}
\end{mathpar} 

Note that $\vec{x}$ (resp. $\vec{P}$) denotes a vector of names
(resp. processes) of length $|\vec{x}|$ (resp. $|\vec{P}|$). We adopt
the following useful abbreviations.

\begin{mathpar}
   x?(\vec{y}).P := x.(\vec{y})P \and  x\clift{\vec{P}} := x.\clift{\vec{P}}
   \and x!(y) := \lift{x}{\dropn{y}}
   \and \Pi_{i=0}^{n-1}P_i := P_0 | \ldots | P_{n-1}
\end{mathpar}

\subsubsection{Structural congruence}

\paragraph{Free and bound names and alpha-equivalence.} At the
core of structural equivalence is alpha-equivalence which identifies
process that are the same up to a change of variable. Formally, we
recognize the distinction between free and bound names. The free names
of a process, $\freenames{P}$, may be calculated recursively as
follows:

\begin{mathpar}
\freenames{\pzero} := \emptyset
  \and \\
  \freenames{x?(y).P} := \{ x \} \cup (\freenames{P} \setminus \{ y \})
  \and 
  \freenames{x!\langle P \rangle} := \{ x \} \cup \{ P \} 
  \and \\
  \freenames{P|Q} := \freenames{P} \cup \freenames{Q}
  \and \\
  \freenames{@{x}} := \{ x \}
\end{mathpar}

$\pi$
$\quotep{\pi}$

$\freenames{-} : \pi \to \mathcal{P}(\quotep{\pi})$

\begin{eqnarray*}
  \freenames{\pzero} & := & \emptyset \\
  \freenames{x?(y).P} & := & \{ x \} \cup (\freenames{P} \setminus \{ y \}) \\
  \freenames{x!\langle P \rangle} & := & \{ x \} \cup \{ P \} \\
  \freenames{P|Q} & := & \freenames{P} \cup \freenames{Q} \\
  \freenames{\dropn{x}} & := & \{ x \}
\end{eqnarray*}

The bound names of a process, $\boundnames{P}$, are those names occurring in $P$
that are not free. For example, in $x?(y).0$, the name $x$ is free, while $y$ is bound.

\begin{mathpar}
  \inferrule* [lab=monoidal-laws] {} { P|Q \equiv Q|P \and P|0 \equiv P \and P|(Q|R) \equiv (P|Q)|R }
\end{mathpar}

\begin{mathpar}
  \inferrule* [lab=alpha-equivalence] {} { (x)P \equiv (y)P\{y/x\} \and y \not\in \freenames{P} }
\end{mathpar}

\begin{definition}
Then two processes, $P,Q$, are alpha-equivalent if $P = Q\{\vec{y}/\vec{x}\}$ for
some $\vec{x} \in \boundnames{Q},\vec{y} \in \boundnames{P}$, where $Q\{\vec{y}/\vec{x}\}$
denotes the capture-avoiding substitution of $\vec{y}$ for $\vec{x}$ in $Q$.
\end{definition}

\begin{definition}
  The {\em structural congruence} \cite{SangiorgiWalker} , $\equiv$,
  between processes is the least congruence containing
  alpha-equivalence, satisfying the abelian monoid laws
  (associativity, commutativity and $\pzero$ as identity) for parallel
  composition $|$ and for summation $+$.
\end{definition}

\subsection{Name equivalence}

We take name equivalence, written $\nameeq$, to be the smallest
equivalence relation generated by the following rules.

\begin{mathpar}
\inferrule*[lab=Quote-drop]
{ }
{ \quotep{@{x}} \nameeq x }

\inferrule*[lab=Struct-equiv]
{ P \scong Q }
{ \quotep{P} \nameeq \quotep{Q} }
\end{mathpar}

The astute reader will have noticed that the mutual recursion of names
and processes imposes a mutual recursion on alpha-equivalence and
structural equivalence via name-equivalence. Fortunately, all of this
works out pleasantly and we may calculate in the natural way, free of
concern. The reader interested in the details is referred to the
appendix \ref{appendix:rho_details}.

\subsection{Substitution}

We use $\Proc$ for the set of processes, $\QProc$ for the set of
names, and $\id{\{}\vec{y} / \vec{x} \id{\}}$ to denote partial maps,
$s : \QProc \rightarrow \QProc$. A map, $s$ lifts, uniquely, to a map
on process terms, $\widehat{s} : \Proc \rightarrow \Proc$ by the
following equations.

\begin{mathpar}
  (0) \psubstp{Q}{P} := 0 \\
  (R \juxtap S) \psubstp{Q}{P}
  :=    
  (R)\psubstp{Q}{P} \juxtap (S) \psubstp{Q}{P} \\
  (x?(y).R) \psubstp{Q}{P}    
  :=    
  (x)\substp{Q}{P} (z)\concat( (R \psubstn{z}{y}) \psubstp{Q}{P} ) \\
  (\lift{x}{R}) \psubstp{Q}{P}  
  :=
  \lift{(x)\substp{Q}{P}}{ R \psubstp{Q}{P} } \\
%   (\dropn{x})  \psubstp{Q}{P}       
%   := 
%   \left\{ 
%     \begin{array}{ccc} 
%       \dropn{\quotep{Q}} & & x \nameeq \quotep{P} \\
%       \dropn{x} & & otherwise \\
%     \end{array}
%   \right. 
  (\dropn{x})  \psubstp{Q}{P}       
  := 
  \left\{ 
    \begin{array}{ccc} 
      Q & & x \nameeq \quotep{P} \\
      \dropn{x} & & otherwise \\
    \end{array}
  \right.
\end{mathpar}
 

where

\begin{eqnarray}
  (x)\id{\{} \lpquote Q \rpquote / \lpquote P \rpquote \id{\}}            = 
  \left\{ 
    \begin{array}{ccc}
      \lpquote Q \rpquote & & x \nameeq \lpquote P \rpquote \\
      x & & otherwise \\
    \end{array}
  \right. \nonumber
\end{eqnarray}

and $z$ is chosen distinct from $\quotep{P}$, $\quotep{Q}$, the free
names in $Q$, and all the names in $R$. Our $\alpha$-equivalence will
be built in the standard way from this substitution.

\begin{remark}\label{rem:no_self_referential_names}
  One consequence of these definitions is that $\forall P. \quotep{P}
  \not\in \freenames{P}$.
\end{remark}

\subsection{ Dynamic quote: an example }

Anticipating something of what's to come, consider applying the
substitution, $\widehat{\id{\{}u / z \id{\}}}$, to the following pair
of processes, $\lift{w}{y!(z)}$ and $w[ \lpquote y!(z) \rpquote ]$.

\begin{eqnarray}
	\lift{w}{y!(z)}\widehat{\id{\{}u / z \id{\}}}
		& = &
		\lift{w}{y!(u)} \nonumber\\
	w[ \lpquote y!(z) \rpquote ] \widehat{ \id{\{}u / z \id{\}} }
		& = &
		w[ \lpquote y!(z) \rpquote ] \nonumber
\end{eqnarray}

Because the body of the process between quotes is impervious to
substitution, we get radically different answers. In fact, by
examining the first process in an input context,
e.g. $x?(z).\lift{w}{y!(z)}$, we see that the process under the lift
operator may be shaped by prefixed inputs binding a name inside it. In
this sense, the lift operator will be seen as a way to dynamically
construct processes before reifying them as names.

Finally equipped with these standard features we can present the
dynamics of the calculus.

\subsubsection{Operational semantics} 

Finally, we introduce the computational dynamics. What marks these
algebras as distinct from other more traditionally studied algebraic
structures, e.g. vector spaces or polynomial rings, is the manner in
which dynamics is captured. In traditional structures, dynamics is typically
expressed through morphisms between such structures, as in linear maps
between vector spaces or morphisms between rings. In algebras
associated with the semantics of computation, the dynamics is
expressed as part of the algebraic structure itself, through a
reduction reduction relation typically denoted by $\red$. Below, we
give a recursive presentation of this relation for the calculus used
in the encoding.

$\red \subseteq \pi \times \pi$
$\red : \pi \to \mathcal{P}(\pi)$

\begin{mathpar}
  \inferrule* [lab=Comm] { \textsf{match}( x_{src}, x_{trgt} ) } { x_{trgt}?(y)P \; | \; x_{src}!\langle {Q} \rangle \red P\{\quotep{Q}/y}\} }
  \and \\
  \inferrule* [lab=Par] {{P} \red {P}'} {{{P} | {Q}} \red {{P}' | {Q}}}
  \and
  \inferrule* [lab=Equiv]{{{P} \scong {P}'} \andalso {{P}' \red {Q}'} \andalso {{Q}' \scong {Q}}}{{P} \red {Q}}
\end{mathpar}

\begin{eqnarray*}
  match_{\equiv} (\quotep{P},\quotep{Q}) & := & P \equiv Q \\
  match_{\dagger}(\quotep{P},\quotep{Q}) & := & \forall R. P|Q \red^{*} R => R \red^{*} 0 \\
  match_{K}(\quotep{P},\quotep{Q}) & := & K \mbox{ for some context } K
\end{eqnarray*}

$u?(x)P | u!\langle Q \rangle \red P\{\quotep{Q}/x\}$

%We write $\wred$ for $\red^*$, and $P\red$ if $\exists Q $ such that $ P \red Q$.
We write $P\red$ if $\exists Q $ such that $ P \red Q$ and $P\not\red$, otherwise.

\section{Replication}

As mentioned before, it is known that replication (and hence
recursion) can be implemented in a higher-order process algebra
\cite{SangiorgiWalker}. As our first example of calculation with the
machinery thus far presented we give the construction explicitly in
the {\rhoc}.

\begin{eqnarray}
	D_{x} & := & \prefix{x}{y}{(\binpar{\outputp{x}{y}}{@{y}})} \nonumber\\
	\bangp_{x}{P} & := & \binpar{{x}!\langle{\binpar{D_{x}}{P}}\rangle}{D_{x}} \nonumber
\end{eqnarray}

\begin{eqnarray}
	\bangp_{x}{P} & & \nonumber\\
	=
	& {x}!\langle{(\prefix{x}{y}{(\outputp{x}{y} | @{y})) | P}}\rangle 
	      | \prefix{x}{y}{(\outputp{x}{y} | @{y})} & \nonumber\\
	\red
	& (\outputp{x}{y} | @{y})\substn{\quotep{(\prefix{x}{y}{(@{y} | \outputp{x}{y})) | P}}}{y} & \nonumber\\
	=
	& \outputp{x}{\quotep{(\prefix{x}{y}{(\outputp{x}{y} | @{y})) | P}}}
	  | {(\prefix{x}{y}{(\outputp{x}{y} | @{y})) | P}} & \nonumber\\
	\red
	& \ldots & \nonumber\\
	\red^*
	& P | P | \ldots & \nonumber
\end{eqnarray}

Of course, this encoding, as an implementation, runs away, unfolding
$\bangp{P}$ eagerly. A lazier and more implementable replication
operator, restricted to input-guarded processes, may be obtained as follows.

\begin{eqnarray}
\bangp{\prefix{u}{v}{P}} 
	:= 
	\binpar{\lift{x}{\prefix{u}{v}{(\binpar{D(x)}{P})}}}{D(x)} \nonumber
\end{eqnarray}

\begin{remark}
  Note that the lazier definition still does not deal with summation
  or mixed summation (i.e. sums over input and output). The reader is
  invited to construct definitions of replication that deal with these
  features. 

  Further, the definitions are parameterized in a name, $x$. Can you,
  gentle reader, make a definition that eliminates this parameter and
  guarantees no accidental interaction between the replication
  machinery and the process being replicated -- i.e. no accidental
  sharing of names used by the process to get its work done and the
  name(s) used by the replication to effect copying. This latter
  revision of the definition of replication is crucial to obtaining
  the expected identity $!!P \sim !P$.
\end{remark}

\begin{remark}\label{rem:paradoxical_combinator}
  The reader familiar with the lambda calculus will have noticed the
  similarity between $D$ and the paradoxical combinator.

  [Ed. note: the existence of this seems to suggest we have to be more
  restrictive on the set of processes and names we admit if we are to
  support no-cloning.]
\end{remark}

\subsubsection{Bisimulation}

The computational dynamics gives rise to another kind of equivalence,
the equivalence of computational behavior. As previously mentioned
this is typically captured \emph{via} some form of bisimulation.

% The notion we use in this paper is weak barbed bisimulation
% \cite{milner91polyadicpi}.

The notion we use in this paper is derived from weak barbed
bisimulation \cite{milner91polyadicpi}. 

\begin{definition}
An \emph{observation relation}, $\downarrow_{\mathcal N}$, over a set
of names, $\mathcal N$, is the smallest relation satisfying the rules
below.

\infrule[Out-barb]{y \in {\mathcal N}, \; x \nameeq y}
		  {\outputp{x}{v} \downarrow_{\mathcal N} x}
\infrule[Par-barb]{\mbox{$P\downarrow_{\mathcal N} x$ or $Q\downarrow_{\mathcal N} x$}}
		  {\binpar{P}{Q} \downarrow_{\mathcal N} x}

We write $P \Downarrow_{\mathcal N} x$ if there is $Q$ such that 
$P \wred Q$ and $Q \downarrow_{\mathcal N} x$.
\end{definition}

\begin{definition}
%\label{def.bbisim}
An  ${\mathcal N}$-\emph{barbed bisimulation} over a set of names, ${\mathcal N}$, is a symmetric binary relation 
${\mathcal S}_{\mathcal N}$ between agents such that $P\rel{S}_{\mathcal N}Q$ implies:
\begin{enumerate}
\item If $P \red P'$ then $Q \wred Q'$ and $P'\rel{S}_{\mathcal N} Q'$.
\item If $P\downarrow_{\mathcal N} x$, then $Q\Downarrow_{\mathcal N} x$.
\end{enumerate}
$P$ is ${\mathcal N}$-barbed bisimilar to $Q$, written
$P \wbbisim_{\mathcal N} Q$, if $P \rel{S}_{\mathcal N} Q$ for some ${\mathcal N}$-barbed bisimulation ${\mathcal S}_{\mathcal N}$.
\end{definition}

$\mathcal{R} \subseteq \pi \times \pi$

$P \mathcal{R} Q => \forall P'. P \red P' \Rightarrow \exists Q'. Q \red Q', P' \mathcal{R} Q'$

$P \vdash x \Rightarrow Q \vdash x$

\begin{mathpar}
  \inferrule*[lab=Out-barb]{x \nameeq y}{{y}!\langle{Q}\rangle \vdash x}
  \and
  \inferrule*[lab=Par-barb]{\mbox{$P\vdash x$ or $Q\vdash x$}}{\binpar{P}{Q} \vdash x}
\end{mathpar}

\subsubsection{Contexts}

One of the principle advantages of computational calculi like the
$\pi$-calculus is a well-defined notion of context,
contextual-equivalence and a correlation between
contextual-equivalence and notions of bisimulation. The notion of
context allows the decomposition of a process into (sub-)process and
its syntactic environment, its context. Thus, a context may be
thought of as a process with a ``hole'' (written $\Box$) in it. The
application of a context $M$ to a process $P$, written $M[P]$, is
tantamount to filling the hole in $M$ with $P$. In this paper we do
not need the full weight of this theory, but do make use of the notion
of context in the proof the main theorem. 

\begin{mathpar}
  \inferrule* [lab=summation] {} {{M_{M},M_{N}} \bc \Box \;|\; x.M_{A} \;|\; M_{M}+M_{N}}
  \and
  \inferrule* [lab=agent] {} {{M_{A}} \bc (\vec{x})M_{P} \;| \; \clift{P_0,\ldots,M_{P},\ldots,P_N}}
  \and \\
  \inferrule* [lab=process] {} {{M_{P}} \bc M_{N} \;| \;P|M_{P} }
\end{mathpar} 

\begin{mathpar}
  \inferrule* [lab=sychronization] {} {M_{N} \bc \Box \;|\; x?M_{F} \;|\; x!M_{C}}
  \and
  \inferrule* [lab=abstraction] {} {{M_{F}} \bc (x)M_{P} }
  \and
  \inferrule* [lab=concretion] {} {{M_{C}} \bc \langle M_{P} \rangle }
  \and \\
  \inferrule* [lab=process] {} {{M_{P}} \bc M_{N} \;| \;P|M_{P} }
\end{mathpar}

\begin{definition}[contextual application] Given a context $M$, and
  process $P$, we define the \emph{contextual application}, $M[P] :=
  M\{P/\Box\}$. That is, the contextual application of M to P is the
  substitution of $P$ for $\Box$ in $M$.
\end{definition}

$\meaningof{-} : L \to \mathcal{P}(\pi)$

\begin{mathpar}
  \inferrule* [lab=collection] {} {\meaningof{true} = \pi, \and \meaningof{~E} = \pi \setminus \meaningof{E}, \and \meaningof{E_{1} \& E_{2}} = \meaningof{E_{1}} \cap \meaningof{E_{2}}}
\end{mathpar}

\begin{mathpar}
  \inferrule* [lab=structure] {} {\meaningof{0} = \{ P \in \pi | P \equiv 0 \}, \and \\ \meaningof{E_1 | E_2} = \{ P \in \pi | P \equiv P_{1} | P_{2}, P_{1} \in \meaningof{E_{1}}, P_{2} \in \meaningof{E_2}\} }
\end{mathpar}

\begin{mathpar}
 \inferrule* [lab=behavior] {} {\meaningof{\langle a?b \rangle E} = \{ P \in \pi | P \equiv Q | u?(y)P', \\ \and \\\\ \and \\ \;\;\; u \in \meaningof{a}, \forall z.P'\{z/y\} \in \meaningof{E\{z/b\}}\}, \and \\ \meaningof{a!E} = \{ P \in \pi | P \equiv Q | x!\langle P' \rangle, x \in \meaningof{a} P' \in \meaningof{E}\} }
\end{mathpar}

\begin{mathpar}
 \inferrule* [lab=nominal] {} {\meaningof{\quotep{E}} = \{ \quotep{P} \in \quotep{\pi} | P \in \meaningof{E} \}, \and \meaningof{\quotep{P}} = \{ \quotep{Q} \in \quotep{\pi} | P \equiv Q \} \and \\ \meaningof{@\quotep{E}} = \{ P \in \pi | P \equiv @x, x \in \meaningof{E} \}}
\end{mathpar}

\begin{eqnarray*}
  \\
  \meaningof{-} : TS \to ST
\end{eqnarray*}

\begin{eqnarray*}
  \\
  L : TS \to ST
\end{eqnarray*}

\begin{eqnarray*}
  \\
  P \models E \iff P \in \meaningof{E}
\end{eqnarray*}

\begin{eqnarray*}
  P \approx_{L} Q \iff \forall E \in L. P \models E \iff Q \models E
\end{eqnarray*}

\begin{eqnarray*}
  P \approx_{K} Q
\end{eqnarray*}

\begin{eqnarray*}
  P \approx Q
\end{eqnarray*}

$\approx_{K} = \approx = \approx_{L}$

\subsubsection{Contextual duality}

Note that contexts extend the quotation operation to a family of
operations from processes to names. Given a context, $M$, we can
define a \emph{nominal context}, $\quotep{M}$ by $\quotep{M}[P] :=
\quotep{M[P]}$. To foreshadow what is to come we observe that these
operations enjoy a duality with processes very much like the duality
between vectors and maps from vectors to scalars.

Further, because the calculus is essentially higher-order, we have a
correspondence between contexts and processes. More specifically,
given a name $x$ and a context $M$ we can construct $M^{*}_{x}$ such
that 

\begin{mathpar}
  M^{*}_{x} | \lift{x}{P} \red M[P]
\end{mathpar}

namely,

\begin{mathpar}
  M^{*}_{x} := x?(u).M[\dropn{u}]
\end{mathpar}

The dependence of $M^{*}_{x}$ on a name makes it an abstraction, 

\begin{mathpar}
  M^{*} := (x)x?(u).M[\dropn{u}]
\end{mathpar}

\subsection{Additional notation}

It will sometimes be convenient to denote the process a name
quotes. We already have the notation $x = \quotep{P}$, but it will be
convenient to introduce an alternate notation, $\procn{x}$, when we
want to emphasize the connection to the use of the name. Note that, by
virtue of name equivalence, $\quotep{\procn{x}} \nameeq x$; so, the
notation is consistent with previous definitions.

Further, because names have structure it is possible to effect
substitutions on the basis of that structure. This means we need to
upgrade our notation for substitutions, which we accomplish by
adapting comprehension notation. Thus,

\begin{mathpar}
  P\{ y / x : x \in S \}
\end{mathpar}

is interpreted to mean the process derived from P by replacing (in a
capture-avoiding manner) each occurrence of $x$ in $S$ by $y$. For example,

\begin{mathpar}
  P\{ \quotep{\procn{x}|\procn{x}} / x : x \in \freenames{P} \}
\end{mathpar}

will replace each (occurrence) of a free name $x$ in $P$ by
$\quotep{\procn{x}|\procn{x}}$.

Also, we will avail ourselves of the notation $x^{L}$ and $x^{R}$ to
denote injections of a name into disjoint copies of the name
space. There are numerous ways to accomplish this. One example can be
found in \cite{MeredithR05}. This notation overloads to vectors of
names: $\vec{x}^{\pi} := (x_{i}^{\pi} \; : \; 0 \leq i < |\vec{x}| )$ where $\pi \in \{L,R\}$.

We also use $P^{\Box} := P|\Box$.

In \cite{MeredithR05} an interpretation of the new operator is
given. It turns out that there are several possible interpretations
all enjoying the requisite algebraic properties of the operator (see
\cite{milner91polyadicpi}). We will therefore make liberal use of
$(\nu\; \vec{x})P$.

% subsection the_syntax_and_semantics_of_the_notation_system (end)   

\input{qm2pi.qmops} 

\input{qm2pi.sterngerlach} 

\input{qm2pi.metric} 

% section concurrent_process_calculi (end)

%\input{qm2pi.proofsketch}

% section proof sketch (end)

%\input{qm2pi.slviaknots} 

% section spatial logic via knots (end)

\input{qm2pi.conclusion}

% section conclusion (end)

%\input{qm2pi.dtcodes} 

% section wiring algorithm (end)

\input{qm2pi.ack} 

% section acknowledgments (end)

\newpage


\bibliographystyle{plain}   
\bibliography{../../biblios/main.bib}

\input{qm2pi.rhodetails}

\end{document}

 

% section concurrent_process_calculi (end)

%\documentclass[12pt]{llncs}
%\documentclass{jktr}

\usepackage[pdftex]{hyperref}                   
\usepackage {listings}
\usepackage {mathpartir}
\usepackage{bcprules}
%\usepackage{listings}
                       
\usepackage{graphicx} 
%\usepackage[margins=2.5cm,nohead,nofoot]{geometry}
%\usepackage{geometry}
\usepackage{amsfonts}
\usepackage{amstext}
\usepackage{latexsym}
\usepackage{amssymb}
\usepackage{color}


%\include{myPreamble}
\include{qm2pi.local} 

%\ifpdf
%\usepackage[pdftex]{graphicx}
%\else
%\usepackage{graphicx}
%\fi

 % \ifpdf
%  \usepackage{pdfsync}
%  \if


%\title{Brief Article}
%\author{David F. Snyder}
%\author{L.G. Meredith}

%\address{Dept. of Math., Texas State University--San Marcos, San Marcos, TX 78666}
       
\pagestyle{empty}


\begin{document}

\lstset{language=[Objective]Caml,frame=shadowbox}

\input{qm2pi.front}

% section front matter (end)

\input{qm2pi.intro} 
 
% section introduction (end)

% \input{qm2pi.knotations} 

% section notation (end)

\input{qm2pi.process.calculi} 

% section concurrent_process_calculi_and_spatial_logics_ (end)
    
%\input{qm2pi.knots2pi} 

%\input{qm2pi.trefoil} 

%\input{qm2pi.mainthm} 

% subsection basic_interpretation (end)

%\input{qm2pi.rho.presentation} 
\subsection{The syntax and semantics of the notation system}\label{sub:the_syntax_and_semantics_of_the_notation_system} % (fold)

We now summarize a technical presentation of the calculus that
embodies our theory of dynamics. The typical presentation of such a
calculus follows the style of giving generators and relations on
them. The grammar, below, describing term constructors, freely
generates the set of processes, $\Proc$. This set is then quotiented
by a relation known as structural congruence and it is over this set
that the notion of dynamics is expressed. This presentation is
essentially that of \cite{MeredithR05} with the addition of
polyadicity and summation. For readability we have relegated some of
the technical subtleties to an appendix.

\subsubsection{Process grammar}\label{subsub:process_grammar}

\begin{mathpar}
  \inferrule* [lab=synchronization] {} {{M} \bc \pzero \;|\; x?F \;|\; x!C }
  \and
  \inferrule* [lab=abstraction] {} {{F} \bc (x)P}
  \and
  \inferrule* [lab=concretion] {} {{C} \bc \langle Q \rangle}
  \and
  \inferrule* [lab=process] {} {{P,Q} \bc M \;| \;P|Q \;|\; @{x}}
  \and
  \inferrule* [lab=name] {} {{x} \bc \quotep{P}}
\end{mathpar} 

Note that $\vec{x}$ (resp. $\vec{P}$) denotes a vector of names
(resp. processes) of length $|\vec{x}|$ (resp. $|\vec{P}|$). We adopt
the following useful abbreviations.

\begin{mathpar}
   x?(\vec{y}).P := x.(\vec{y})P \and  x\clift{\vec{P}} := x.\clift{\vec{P}}
   \and x!(y) := \lift{x}{\dropn{y}}
   \and \Pi_{i=0}^{n-1}P_i := P_0 | \ldots | P_{n-1}
\end{mathpar}

\subsubsection{Structural congruence}

\paragraph{Free and bound names and alpha-equivalence.} At the
core of structural equivalence is alpha-equivalence which identifies
process that are the same up to a change of variable. Formally, we
recognize the distinction between free and bound names. The free names
of a process, $\freenames{P}$, may be calculated recursively as
follows:

\begin{mathpar}
\freenames{\pzero} := \emptyset
  \and \\
  \freenames{x?(y).P} := \{ x \} \cup (\freenames{P} \setminus \{ y \})
  \and 
  \freenames{x!\langle P \rangle} := \{ x \} \cup \{ P \} 
  \and \\
  \freenames{P|Q} := \freenames{P} \cup \freenames{Q}
  \and \\
  \freenames{@{x}} := \{ x \}
\end{mathpar}

$\pi$
$\quotep{\pi}$

$\freenames{-} : \pi \to \mathcal{P}(\quotep{\pi})$

\begin{eqnarray*}
  \freenames{\pzero} & := & \emptyset \\
  \freenames{x?(y).P} & := & \{ x \} \cup (\freenames{P} \setminus \{ y \}) \\
  \freenames{x!\langle P \rangle} & := & \{ x \} \cup \{ P \} \\
  \freenames{P|Q} & := & \freenames{P} \cup \freenames{Q} \\
  \freenames{\dropn{x}} & := & \{ x \}
\end{eqnarray*}

The bound names of a process, $\boundnames{P}$, are those names occurring in $P$
that are not free. For example, in $x?(y).0$, the name $x$ is free, while $y$ is bound.

\begin{mathpar}
  \inferrule* [lab=monoidal-laws] {} { P|Q \equiv Q|P \and P|0 \equiv P \and P|(Q|R) \equiv (P|Q)|R }
\end{mathpar}

\begin{mathpar}
  \inferrule* [lab=alpha-equivalence] {} { (x)P \equiv (y)P\{y/x\} \and y \not\in \freenames{P} }
\end{mathpar}

\begin{definition}
Then two processes, $P,Q$, are alpha-equivalent if $P = Q\{\vec{y}/\vec{x}\}$ for
some $\vec{x} \in \boundnames{Q},\vec{y} \in \boundnames{P}$, where $Q\{\vec{y}/\vec{x}\}$
denotes the capture-avoiding substitution of $\vec{y}$ for $\vec{x}$ in $Q$.
\end{definition}

\begin{definition}
  The {\em structural congruence} \cite{SangiorgiWalker} , $\equiv$,
  between processes is the least congruence containing
  alpha-equivalence, satisfying the abelian monoid laws
  (associativity, commutativity and $\pzero$ as identity) for parallel
  composition $|$ and for summation $+$.
\end{definition}

\subsection{Name equivalence}

We take name equivalence, written $\nameeq$, to be the smallest
equivalence relation generated by the following rules.

\begin{mathpar}
\inferrule*[lab=Quote-drop]
{ }
{ \quotep{@{x}} \nameeq x }

\inferrule*[lab=Struct-equiv]
{ P \scong Q }
{ \quotep{P} \nameeq \quotep{Q} }
\end{mathpar}

The astute reader will have noticed that the mutual recursion of names
and processes imposes a mutual recursion on alpha-equivalence and
structural equivalence via name-equivalence. Fortunately, all of this
works out pleasantly and we may calculate in the natural way, free of
concern. The reader interested in the details is referred to the
appendix \ref{appendix:rho_details}.

\subsection{Substitution}

We use $\Proc$ for the set of processes, $\QProc$ for the set of
names, and $\id{\{}\vec{y} / \vec{x} \id{\}}$ to denote partial maps,
$s : \QProc \rightarrow \QProc$. A map, $s$ lifts, uniquely, to a map
on process terms, $\widehat{s} : \Proc \rightarrow \Proc$ by the
following equations.

\begin{mathpar}
  (0) \psubstp{Q}{P} := 0 \\
  (R \juxtap S) \psubstp{Q}{P}
  :=    
  (R)\psubstp{Q}{P} \juxtap (S) \psubstp{Q}{P} \\
  (x?(y).R) \psubstp{Q}{P}    
  :=    
  (x)\substp{Q}{P} (z)\concat( (R \psubstn{z}{y}) \psubstp{Q}{P} ) \\
  (\lift{x}{R}) \psubstp{Q}{P}  
  :=
  \lift{(x)\substp{Q}{P}}{ R \psubstp{Q}{P} } \\
%   (\dropn{x})  \psubstp{Q}{P}       
%   := 
%   \left\{ 
%     \begin{array}{ccc} 
%       \dropn{\quotep{Q}} & & x \nameeq \quotep{P} \\
%       \dropn{x} & & otherwise \\
%     \end{array}
%   \right. 
  (\dropn{x})  \psubstp{Q}{P}       
  := 
  \left\{ 
    \begin{array}{ccc} 
      Q & & x \nameeq \quotep{P} \\
      \dropn{x} & & otherwise \\
    \end{array}
  \right.
\end{mathpar}
 

where

\begin{eqnarray}
  (x)\id{\{} \lpquote Q \rpquote / \lpquote P \rpquote \id{\}}            = 
  \left\{ 
    \begin{array}{ccc}
      \lpquote Q \rpquote & & x \nameeq \lpquote P \rpquote \\
      x & & otherwise \\
    \end{array}
  \right. \nonumber
\end{eqnarray}

and $z$ is chosen distinct from $\quotep{P}$, $\quotep{Q}$, the free
names in $Q$, and all the names in $R$. Our $\alpha$-equivalence will
be built in the standard way from this substitution.

\begin{remark}\label{rem:no_self_referential_names}
  One consequence of these definitions is that $\forall P. \quotep{P}
  \not\in \freenames{P}$.
\end{remark}

\subsection{ Dynamic quote: an example }

Anticipating something of what's to come, consider applying the
substitution, $\widehat{\id{\{}u / z \id{\}}}$, to the following pair
of processes, $\lift{w}{y!(z)}$ and $w[ \lpquote y!(z) \rpquote ]$.

\begin{eqnarray}
	\lift{w}{y!(z)}\widehat{\id{\{}u / z \id{\}}}
		& = &
		\lift{w}{y!(u)} \nonumber\\
	w[ \lpquote y!(z) \rpquote ] \widehat{ \id{\{}u / z \id{\}} }
		& = &
		w[ \lpquote y!(z) \rpquote ] \nonumber
\end{eqnarray}

Because the body of the process between quotes is impervious to
substitution, we get radically different answers. In fact, by
examining the first process in an input context,
e.g. $x?(z).\lift{w}{y!(z)}$, we see that the process under the lift
operator may be shaped by prefixed inputs binding a name inside it. In
this sense, the lift operator will be seen as a way to dynamically
construct processes before reifying them as names.

Finally equipped with these standard features we can present the
dynamics of the calculus.

\subsubsection{Operational semantics} 

Finally, we introduce the computational dynamics. What marks these
algebras as distinct from other more traditionally studied algebraic
structures, e.g. vector spaces or polynomial rings, is the manner in
which dynamics is captured. In traditional structures, dynamics is typically
expressed through morphisms between such structures, as in linear maps
between vector spaces or morphisms between rings. In algebras
associated with the semantics of computation, the dynamics is
expressed as part of the algebraic structure itself, through a
reduction reduction relation typically denoted by $\red$. Below, we
give a recursive presentation of this relation for the calculus used
in the encoding.

$\red \subseteq \pi \times \pi$
$\red : \pi \to \mathcal{P}(\pi)$

\begin{mathpar}
  \inferrule* [lab=Comm] { \textsf{match}( x_{src}, x_{trgt} ) } { x_{trgt}?(y)P \; | \; x_{src}!\langle {Q} \rangle \red P\{\quotep{Q}/y}\} }
  \and \\
  \inferrule* [lab=Par] {{P} \red {P}'} {{{P} | {Q}} \red {{P}' | {Q}}}
  \and
  \inferrule* [lab=Equiv]{{{P} \scong {P}'} \andalso {{P}' \red {Q}'} \andalso {{Q}' \scong {Q}}}{{P} \red {Q}}
\end{mathpar}

\begin{eqnarray*}
  match_{\equiv} (\quotep{P},\quotep{Q}) & := & P \equiv Q \\
  match_{\dagger}(\quotep{P},\quotep{Q}) & := & \forall R. P|Q \red^{*} R => R \red^{*} 0 \\
  match_{K}(\quotep{P},\quotep{Q}) & := & K \mbox{ for some context } K
\end{eqnarray*}

$u?(x)P | u!\langle Q \rangle \red P\{\quotep{Q}/x\}$

%We write $\wred$ for $\red^*$, and $P\red$ if $\exists Q $ such that $ P \red Q$.
We write $P\red$ if $\exists Q $ such that $ P \red Q$ and $P\not\red$, otherwise.

\section{Replication}

As mentioned before, it is known that replication (and hence
recursion) can be implemented in a higher-order process algebra
\cite{SangiorgiWalker}. As our first example of calculation with the
machinery thus far presented we give the construction explicitly in
the {\rhoc}.

\begin{eqnarray}
	D_{x} & := & \prefix{x}{y}{(\binpar{\outputp{x}{y}}{@{y}})} \nonumber\\
	\bangp_{x}{P} & := & \binpar{{x}!\langle{\binpar{D_{x}}{P}}\rangle}{D_{x}} \nonumber
\end{eqnarray}

\begin{eqnarray}
	\bangp_{x}{P} & & \nonumber\\
	=
	& {x}!\langle{(\prefix{x}{y}{(\outputp{x}{y} | @{y})) | P}}\rangle 
	      | \prefix{x}{y}{(\outputp{x}{y} | @{y})} & \nonumber\\
	\red
	& (\outputp{x}{y} | @{y})\substn{\quotep{(\prefix{x}{y}{(@{y} | \outputp{x}{y})) | P}}}{y} & \nonumber\\
	=
	& \outputp{x}{\quotep{(\prefix{x}{y}{(\outputp{x}{y} | @{y})) | P}}}
	  | {(\prefix{x}{y}{(\outputp{x}{y} | @{y})) | P}} & \nonumber\\
	\red
	& \ldots & \nonumber\\
	\red^*
	& P | P | \ldots & \nonumber
\end{eqnarray}

Of course, this encoding, as an implementation, runs away, unfolding
$\bangp{P}$ eagerly. A lazier and more implementable replication
operator, restricted to input-guarded processes, may be obtained as follows.

\begin{eqnarray}
\bangp{\prefix{u}{v}{P}} 
	:= 
	\binpar{\lift{x}{\prefix{u}{v}{(\binpar{D(x)}{P})}}}{D(x)} \nonumber
\end{eqnarray}

\begin{remark}
  Note that the lazier definition still does not deal with summation
  or mixed summation (i.e. sums over input and output). The reader is
  invited to construct definitions of replication that deal with these
  features. 

  Further, the definitions are parameterized in a name, $x$. Can you,
  gentle reader, make a definition that eliminates this parameter and
  guarantees no accidental interaction between the replication
  machinery and the process being replicated -- i.e. no accidental
  sharing of names used by the process to get its work done and the
  name(s) used by the replication to effect copying. This latter
  revision of the definition of replication is crucial to obtaining
  the expected identity $!!P \sim !P$.
\end{remark}

\begin{remark}\label{rem:paradoxical_combinator}
  The reader familiar with the lambda calculus will have noticed the
  similarity between $D$ and the paradoxical combinator.

  [Ed. note: the existence of this seems to suggest we have to be more
  restrictive on the set of processes and names we admit if we are to
  support no-cloning.]
\end{remark}

\subsubsection{Bisimulation}

The computational dynamics gives rise to another kind of equivalence,
the equivalence of computational behavior. As previously mentioned
this is typically captured \emph{via} some form of bisimulation.

% The notion we use in this paper is weak barbed bisimulation
% \cite{milner91polyadicpi}.

The notion we use in this paper is derived from weak barbed
bisimulation \cite{milner91polyadicpi}. 

\begin{definition}
An \emph{observation relation}, $\downarrow_{\mathcal N}$, over a set
of names, $\mathcal N$, is the smallest relation satisfying the rules
below.

\infrule[Out-barb]{y \in {\mathcal N}, \; x \nameeq y}
		  {\outputp{x}{v} \downarrow_{\mathcal N} x}
\infrule[Par-barb]{\mbox{$P\downarrow_{\mathcal N} x$ or $Q\downarrow_{\mathcal N} x$}}
		  {\binpar{P}{Q} \downarrow_{\mathcal N} x}

We write $P \Downarrow_{\mathcal N} x$ if there is $Q$ such that 
$P \wred Q$ and $Q \downarrow_{\mathcal N} x$.
\end{definition}

\begin{definition}
%\label{def.bbisim}
An  ${\mathcal N}$-\emph{barbed bisimulation} over a set of names, ${\mathcal N}$, is a symmetric binary relation 
${\mathcal S}_{\mathcal N}$ between agents such that $P\rel{S}_{\mathcal N}Q$ implies:
\begin{enumerate}
\item If $P \red P'$ then $Q \wred Q'$ and $P'\rel{S}_{\mathcal N} Q'$.
\item If $P\downarrow_{\mathcal N} x$, then $Q\Downarrow_{\mathcal N} x$.
\end{enumerate}
$P$ is ${\mathcal N}$-barbed bisimilar to $Q$, written
$P \wbbisim_{\mathcal N} Q$, if $P \rel{S}_{\mathcal N} Q$ for some ${\mathcal N}$-barbed bisimulation ${\mathcal S}_{\mathcal N}$.
\end{definition}

$\mathcal{R} \subseteq \pi \times \pi$

$P \mathcal{R} Q => \forall P'. P \red P' \Rightarrow \exists Q'. Q \red Q', P' \mathcal{R} Q'$

$P \vdash x \Rightarrow Q \vdash x$

\begin{mathpar}
  \inferrule*[lab=Out-barb]{x \nameeq y}{{y}!\langle{Q}\rangle \vdash x}
  \and
  \inferrule*[lab=Par-barb]{\mbox{$P\vdash x$ or $Q\vdash x$}}{\binpar{P}{Q} \vdash x}
\end{mathpar}

\subsubsection{Contexts}

One of the principle advantages of computational calculi like the
$\pi$-calculus is a well-defined notion of context,
contextual-equivalence and a correlation between
contextual-equivalence and notions of bisimulation. The notion of
context allows the decomposition of a process into (sub-)process and
its syntactic environment, its context. Thus, a context may be
thought of as a process with a ``hole'' (written $\Box$) in it. The
application of a context $M$ to a process $P$, written $M[P]$, is
tantamount to filling the hole in $M$ with $P$. In this paper we do
not need the full weight of this theory, but do make use of the notion
of context in the proof the main theorem. 

\begin{mathpar}
  \inferrule* [lab=summation] {} {{M_{M},M_{N}} \bc \Box \;|\; x.M_{A} \;|\; M_{M}+M_{N}}
  \and
  \inferrule* [lab=agent] {} {{M_{A}} \bc (\vec{x})M_{P} \;| \; \clift{P_0,\ldots,M_{P},\ldots,P_N}}
  \and \\
  \inferrule* [lab=process] {} {{M_{P}} \bc M_{N} \;| \;P|M_{P} }
\end{mathpar} 

\begin{mathpar}
  \inferrule* [lab=sychronization] {} {M_{N} \bc \Box \;|\; x?M_{F} \;|\; x!M_{C}}
  \and
  \inferrule* [lab=abstraction] {} {{M_{F}} \bc (x)M_{P} }
  \and
  \inferrule* [lab=concretion] {} {{M_{C}} \bc \langle M_{P} \rangle }
  \and \\
  \inferrule* [lab=process] {} {{M_{P}} \bc M_{N} \;| \;P|M_{P} }
\end{mathpar}

\begin{definition}[contextual application] Given a context $M$, and
  process $P$, we define the \emph{contextual application}, $M[P] :=
  M\{P/\Box\}$. That is, the contextual application of M to P is the
  substitution of $P$ for $\Box$ in $M$.
\end{definition}

$\meaningof{-} : L \to \mathcal{P}(\pi)$

\begin{mathpar}
  \inferrule* [lab=collection] {} {\meaningof{true} = \pi, \and \meaningof{~E} = \pi \setminus \meaningof{E}, \and \meaningof{E_{1} \& E_{2}} = \meaningof{E_{1}} \cap \meaningof{E_{2}}}
\end{mathpar}

\begin{mathpar}
  \inferrule* [lab=structure] {} {\meaningof{0} = \{ P \in \pi | P \equiv 0 \}, \and \\ \meaningof{E_1 | E_2} = \{ P \in \pi | P \equiv P_{1} | P_{2}, P_{1} \in \meaningof{E_{1}}, P_{2} \in \meaningof{E_2}\} }
\end{mathpar}

\begin{mathpar}
 \inferrule* [lab=behavior] {} {\meaningof{\langle a?b \rangle E} = \{ P \in \pi | P \equiv Q | u?(y)P', \\ \and \\\\ \and \\ \;\;\; u \in \meaningof{a}, \forall z.P'\{z/y\} \in \meaningof{E\{z/b\}}\}, \and \\ \meaningof{a!E} = \{ P \in \pi | P \equiv Q | x!\langle P' \rangle, x \in \meaningof{a} P' \in \meaningof{E}\} }
\end{mathpar}

\begin{mathpar}
 \inferrule* [lab=nominal] {} {\meaningof{\quotep{E}} = \{ \quotep{P} \in \quotep{\pi} | P \in \meaningof{E} \}, \and \meaningof{\quotep{P}} = \{ \quotep{Q} \in \quotep{\pi} | P \equiv Q \} \and \\ \meaningof{@\quotep{E}} = \{ P \in \pi | P \equiv @x, x \in \meaningof{E} \}}
\end{mathpar}

\begin{eqnarray*}
  \\
  \meaningof{-} : TS \to ST
\end{eqnarray*}

\begin{eqnarray*}
  \\
  L : TS \to ST
\end{eqnarray*}

\begin{eqnarray*}
  \\
  P \models E \iff P \in \meaningof{E}
\end{eqnarray*}

\begin{eqnarray*}
  P \approx_{L} Q \iff \forall E \in L. P \models E \iff Q \models E
\end{eqnarray*}

\begin{eqnarray*}
  P \approx_{K} Q
\end{eqnarray*}

\begin{eqnarray*}
  P \approx Q
\end{eqnarray*}

$\approx_{K} = \approx = \approx_{L}$

\subsubsection{Contextual duality}

Note that contexts extend the quotation operation to a family of
operations from processes to names. Given a context, $M$, we can
define a \emph{nominal context}, $\quotep{M}$ by $\quotep{M}[P] :=
\quotep{M[P]}$. To foreshadow what is to come we observe that these
operations enjoy a duality with processes very much like the duality
between vectors and maps from vectors to scalars.

Further, because the calculus is essentially higher-order, we have a
correspondence between contexts and processes. More specifically,
given a name $x$ and a context $M$ we can construct $M^{*}_{x}$ such
that 

\begin{mathpar}
  M^{*}_{x} | \lift{x}{P} \red M[P]
\end{mathpar}

namely,

\begin{mathpar}
  M^{*}_{x} := x?(u).M[\dropn{u}]
\end{mathpar}

The dependence of $M^{*}_{x}$ on a name makes it an abstraction, 

\begin{mathpar}
  M^{*} := (x)x?(u).M[\dropn{u}]
\end{mathpar}

\subsection{Additional notation}

It will sometimes be convenient to denote the process a name
quotes. We already have the notation $x = \quotep{P}$, but it will be
convenient to introduce an alternate notation, $\procn{x}$, when we
want to emphasize the connection to the use of the name. Note that, by
virtue of name equivalence, $\quotep{\procn{x}} \nameeq x$; so, the
notation is consistent with previous definitions.

Further, because names have structure it is possible to effect
substitutions on the basis of that structure. This means we need to
upgrade our notation for substitutions, which we accomplish by
adapting comprehension notation. Thus,

\begin{mathpar}
  P\{ y / x : x \in S \}
\end{mathpar}

is interpreted to mean the process derived from P by replacing (in a
capture-avoiding manner) each occurrence of $x$ in $S$ by $y$. For example,

\begin{mathpar}
  P\{ \quotep{\procn{x}|\procn{x}} / x : x \in \freenames{P} \}
\end{mathpar}

will replace each (occurrence) of a free name $x$ in $P$ by
$\quotep{\procn{x}|\procn{x}}$.

Also, we will avail ourselves of the notation $x^{L}$ and $x^{R}$ to
denote injections of a name into disjoint copies of the name
space. There are numerous ways to accomplish this. One example can be
found in \cite{MeredithR05}. This notation overloads to vectors of
names: $\vec{x}^{\pi} := (x_{i}^{\pi} \; : \; 0 \leq i < |\vec{x}| )$ where $\pi \in \{L,R\}$.

We also use $P^{\Box} := P|\Box$.

In \cite{MeredithR05} an interpretation of the new operator is
given. It turns out that there are several possible interpretations
all enjoying the requisite algebraic properties of the operator (see
\cite{milner91polyadicpi}). We will therefore make liberal use of
$(\nu\; \vec{x})P$.

% subsection the_syntax_and_semantics_of_the_notation_system (end)   

\input{qm2pi.qmops} 

\input{qm2pi.sterngerlach} 

\input{qm2pi.metric} 

% section concurrent_process_calculi (end)

%\input{qm2pi.proofsketch}

% section proof sketch (end)

%\input{qm2pi.slviaknots} 

% section spatial logic via knots (end)

\input{qm2pi.conclusion}

% section conclusion (end)

%\input{qm2pi.dtcodes} 

% section wiring algorithm (end)

\input{qm2pi.ack} 

% section acknowledgments (end)

\newpage


\bibliographystyle{plain}   
\bibliography{../../biblios/main.bib}

\input{qm2pi.rhodetails}

\end{document}



% section proof sketch (end)

%\section{Unlikely characters: spatial logic for
  knots}\label{sub:characteristic_formulae} % (fold)

Associated to the mobile process calculi are a family of logics known
as the Hennessy-Milner logics. These logics typically enjoy a
semantics interpreting formulae as sets of processes that when
factored through the encoding outlined above allows an identification
of classes of knots with logical formulae. In the context of this
encoding the sub-family known as the spatial logics \cite{CairesC03}
\cite{CairesC04} \cite{Caires04} are of particular interest providing
several important features for expressing and reasoning about
properties (i.e. classes) of knots. We hint here at how this may be done.

%\begin{description}
%\item [structural connectives] 
\subsubsection{Structural connectives} The spatial logics enjoy
structural connectives corresponding, at the logical level, to the
parallel composition ($P | Q$) and new name ($(\nu \; x)P$)
connectives for processes. As illustrated in the examples below, these
connectives are extremely expressive given the shape of our encoding.
%\item [decideable satisfaction]

\subsubsection{Decideable satisfaction}
In \cite{Caires04} the satisfaction relation is shown to be decideable
for a rich class of processes. It further turns out that the image of
the our encoding is a proper subset of that class. This result
provides the basis for an algorithm by which to search for knots
enjoying a given property.
%\item [characteristic formulae]

\subsubsection{Characteristic formulae}
In the same paper \cite{Caires04} , Caires presents a means of calculating
characteristic formulae, selecting equivalence classes of processes
up to a pre--specified depth limit on the support set of names. Composed with our
encoding, this characteristic formula can be used to select
characteristic formulae for knots.
%\end{description}

\subsubsection{Spatial logic formulae}

The grammar below (segmented for comprehension) summarizes the syntax
of spatial logic formulae. We employ illustrative examples in the
sequel to provide an intuitive understanding of their meaning
referring the reader to \cite{Caires04} for a more detailed explication
of the semantics.

\begin{mathpar}
  \inferrule* [lab=boolean] {} {{A,B} \bc T \;|\; \neg A \;|\; A \wedge B \;|\; \eta = \eta'}
  \and
  \inferrule* [lab=spatial] {} {|\; \pzero \;|\; A | B \;|\; x \text{\textregistered} A \;|\; \forall x . A \;|\;  H x . A}
  \and
  \inferrule* [lab=behavioral] {} {|\; \alpha . A}
  \and 
  \inferrule* [lab=recursion] {} {|\; X(\vec{u}) \;|\; \mu X(\vec{u}) . A}
  \and
  \inferrule* [lab=action] {} {\alpha \bc \langle x?(\vec{y}) \rangle \;|\; \langle x!(\vec{y}) \rangle \;|\; \langle \tau \rangle}
  \and 
  \inferrule* [lab=name] {} {\eta \bc x \;|\; \tau}
\end{mathpar} 

% subsection characteristic_formulae (end)   	 

\subsection{Example formulae}\label{sub:example_formulae_} % (fold)

\subsubsection{Crossing as formula.}
% 
% \begin{align*}
%   \frac{d}{dx} \sin x &= \cos x 
%   & \frac{d}{dx} e^x &= e^x \\
%   \frac{d}{dx} \cos x &= - \sin x 
%   & \frac{d}{dx} \log x &= \frac{1}{x} \\
% \end{align*} 

\begin{align*}
 \mu C(x_{0},x_{1},y_{0},y_{1},u).&(\langle x_{0}?(z) \rangle(\langle u! \rangle\langle y_{1}!z \rangle C(x_{0},x_{1},y_{0},y_{1},u)) & \\
  & \wedge \langle y_{1}?(z) \rangle (\langle u! \rangle \langle x_{0}!z \rangle C(x_{0},x_{1},y_{0},y_{1},u)) & \\
  & \wedge \langle x_{1}?(z) \rangle (\langle u? \rangle \langle y_{0}!z \rangle C(x_{0},x_{1},y_{0},y_{1},u)) & \\
  & \wedge \langle y_{0}?(z) \rangle (\langle u? \rangle \langle x_{1}!z \rangle C(x_{0},x_{1},y_{0},y_{1},u))) &
\end{align*}

The lexicographical similarity between the shape of this formulae and
the shape of definition of the process representing a crossing reveals
the intuitive meaning of this formulae. It describes the capabilities
of a process that has the right to represent a crossing. For example
it picks out processes that may perform an input on the port $x_0$ in
its initial menu of capabilities. What differentiates the formula
from the process, however, is that the crossing process is the
smallest candidate to satisfy the formula. Infinitely many other
processes -- with internal behavior hidden behind this interface, so
to speak -- also satisfy this formula. Even this simple formula,
then, can be seen to open a new view onto knots, providing a
computational interpretation of \emph{virtual} knots.

Note that this formula is derived by hand. A similar formula can be
derived by employing Caires' calculation of characteristic formula
\cite{Caires04} to the process representing a crossing. In light of
this discussion, we let
$\meaningof{C}_{\phi}(x0,x1,y0,y1,u)$ denote a formula specifying the
dynamics we wish to capture of a crossing. To guarantee we preserve
the shape of the interface and minimal semantics we demand that
$\meaningof{C}_{\phi}(x0,x1,y0,y1,u) \Rightarrow
\textbf{C}(x0,x1,y0,y1,u)$ where $\textbf{C}(x0,x1,y0,y1,u)$ denotes
the formula above.
                            
\subsubsection{Crossing number constraints.}
The moral content of the context lemma (Lemma \ref{context}) is that the notion of
``locality'' in the Reidemeister moves is effectively captured by the
parallel composition operator of the process calculus. This intuition
extends through the logic. Given a formula,
$\meaningof{C}_{\phi}(x0,x1,y0,y1,u)$, we can use the structural
connectives to specify constraints on crossing numbers, such as at
least $n$ crossings, or exactly $n$ crossings.
\begin{mathpar}
  \inferrule* [lab=at-least-n] {} { K^{\geq n}_{\phi}(\vec{xs},\vec{ys}) := \Pi_{i=0}^{n-1} Hu . \meaningof{C}_{\phi}(xs_i,ys_i,u) | T }
  \and 
  \inferrule* [lab=exactly-n] {} { K^{= n}_{\phi}(\vec{xs},\vec{ys}) := \Pi_{i=0}^{n-1} Hu . \meaningof{C}_{\phi}(xs_i,ys_i,u) | \neg (\forall x_0,y_0,x_1,y_1,u . \meaningof{C}_{\phi}(x_0,y_0,x_1,y_1,u) | T) }
\end{mathpar}

To round out this section, recall that the encoding of an $n$-crossing
knot decomposes into a parallel composition of $n$ \emph{copies} of a
crossing process together with a wiring harness. To specify different
knot classes with the same crossing number amounts to specifying
logical constraints on the wiring harness. In the interest of space,
we defer examples to a forthcoming paper. Suffice it to say that both
the conditions ``alternating knot'' and ``contains the tangle
corresponding to 5/3'' are expressible. For example, it is possible to
calculate the characteristic formula of a process corresponding to the
tangle 5/3 and conjoin it into the classifying formula via the
composition connective of the logic.

Finally, we wish to observe that it is entirely within reason to
contemplate a more domain-specific version of spatial logic tailored
to the shape of processes in the image of the encoding. Such a
domain-specific logic would have a better claim to the title formal
language of knot properties.

% subsection example_formulae_ (end)

% section knots_as_processes (end) 

% section spatial logic via knots (end)

\section{Conclusions and future work}

\paragraph{Testing physical space}
You, gentle reader, may wonder why of all the theorems to be proved
given this set up we pick the one above. In some sense it's hardly
central to quantum mechanics. We see it as central in the sense that
it firmly establishes a notion of physical space arising from a notion
of the equivalence of behavior. Relating bisimulation to a metric is a
big step forward, but one is faced with interpreting the relationship
of that metric space to something more physical. Quantum mechanical
notions of ``physical'' space are still far from intuitive, but by
relating this idea of distance as testing to calculations that predict
physical circumstances we are making a not insignificant step forward
toward an understanding of the physical space we inhabit as
essentially dynamic.

\paragraph{Effectivity and simulation}
One of the observations we have yet to make is that the entire program
spelled out here is effective. We have built various interpreters for
the reflective calculus at work in this interpretation. In principle,
then, we can simulate quantum mechanics on a computer. The place where
the simulation may lose fidelity is the infinitely branching summation
for the annihilator.

In this connection i also want to point out that the evaluation style
calculation of the inner product puts the non-determinism of the
summation right at the heart of measurement. This suggests that
Milner's original reduction-based formulation of the dynamics of his
calculi in terms of sums was not just notationally suggestive of a
notion of measure-and-continue but captured some significant part of
the physics.

\paragraph{Quantum continuations}
In light of this last observation i want to point out that the
predominant account of quantum mechanics is missing a key aspect of a
truly compositional story of the physical situation. In a real lab,
when a measurement is made the observation can be made to feed into
another device that then makes another measurement conditioned on the
results of the first. This means that after the superposition was
collapsed the entire experimental set up remained in
superposition. While QM offers a means of writing this down it doesn't
quite line up well with the well-trodden formulation of computation
and continuation that we see so succinctly expressed in Milner's
calculi. This suggests that there might be advantages to this account
of dynamics waiting to be explored.

\paragraph{Quantum logic}
In this connection, we also note that by virtue of having the
Hennessy-Milner construction, we can pull the construction through the
interpretation of QM. This gives us a natural candidate for a quantum
logic that enjoys an extremely tight connection with it's domain of
interpretation, making the construction much less ad hoc (rather it is
the image of functor!).

\paragraph{Quantum probabiity}
i have questions about the basis of the interpretation of inner
product as probability amplitude. In particular, using which
axiomatization of probability theory does the notion of probability
amplitude earn the right to be so dubbed? In other words, where is the
proof that the operation for calculating a probability amplitude (and
then squaring) satisfies the axioms of what it means to calculate a
probability? Even if such a proof exists (i have yet to find it in the
literature), i wonder if it might not be possible to turn things on
their heads. Can we view the calculation of the probability amplitude
as an axiomatization of probability? If so, then the definition we
give for calculating probability amplitude may provide the basis for
an \emph{effective} theory of probability.

\paragraph{Quantum vs ``biological'' information}
Finally, i want to conclude with a more philosophical observation. At
a recent workshop in which QM was a predominant topic i noticed
something about quantum information. The speaker was giving a riveting
discussion of axiomatic QM and showing how properties of ``no
cloning'' and ``no deleting'' emerged as consequences of the
axiomatization. Theorems of this form are necessary to give us a sense
of confidence that our axioms characterize the physical theory. What
struck me, though, was that if quantum information is neither erasable
nor replicable it is markedly different from \emph{life}. Two of the
things we know about life is that

\begin{itemize}
  \item it ends;
  \item to gain some measure of persistence, to transcend it's
    finitude it is imminently copyable.
\end{itemize}

Both of these qualities are summarized succinctly in the aphorism: all
flesh is grass. For me these two kinds of ``information'' -- call them
quantum and biological -- are end points on a spectrum of strategies
for persistence. At one end, we have those curious entities that enjoy
uniqueness and permanence; at the other, we have those who in the face
of a certain end and an uncertain present make a go of passing
something on. To me one of the more remarkable aspects of the latter
strategy is that in the presence of noise (and certain features of
copying) we get a kind of dynamism, a chance for improvement against a
given persistent condition.

% subsection other_calculi_other_bisimulations_and_geometry_as_behavior (end)




% section conclusion (end)

%\documentclass[12pt]{llncs}
%\documentclass{jktr}

\usepackage[pdftex]{hyperref}                   
\usepackage {listings}
\usepackage {mathpartir}
\usepackage{bcprules}
%\usepackage{listings}
                       
\usepackage{graphicx} 
%\usepackage[margins=2.5cm,nohead,nofoot]{geometry}
%\usepackage{geometry}
\usepackage{amsfonts}
\usepackage{amstext}
\usepackage{latexsym}
\usepackage{amssymb}
\usepackage{color}


%\include{myPreamble}
\include{qm2pi.local} 

%\ifpdf
%\usepackage[pdftex]{graphicx}
%\else
%\usepackage{graphicx}
%\fi

 % \ifpdf
%  \usepackage{pdfsync}
%  \if


%\title{Brief Article}
%\author{David F. Snyder}
%\author{L.G. Meredith}

%\address{Dept. of Math., Texas State University--San Marcos, San Marcos, TX 78666}
       
\pagestyle{empty}


\begin{document}

\lstset{language=[Objective]Caml,frame=shadowbox}

\input{qm2pi.front}

% section front matter (end)

\input{qm2pi.intro} 
 
% section introduction (end)

% \input{qm2pi.knotations} 

% section notation (end)

\input{qm2pi.process.calculi} 

% section concurrent_process_calculi_and_spatial_logics_ (end)
    
%\input{qm2pi.knots2pi} 

%\input{qm2pi.trefoil} 

%\input{qm2pi.mainthm} 

% subsection basic_interpretation (end)

%\input{qm2pi.rho.presentation} 
\subsection{The syntax and semantics of the notation system}\label{sub:the_syntax_and_semantics_of_the_notation_system} % (fold)

We now summarize a technical presentation of the calculus that
embodies our theory of dynamics. The typical presentation of such a
calculus follows the style of giving generators and relations on
them. The grammar, below, describing term constructors, freely
generates the set of processes, $\Proc$. This set is then quotiented
by a relation known as structural congruence and it is over this set
that the notion of dynamics is expressed. This presentation is
essentially that of \cite{MeredithR05} with the addition of
polyadicity and summation. For readability we have relegated some of
the technical subtleties to an appendix.

\subsubsection{Process grammar}\label{subsub:process_grammar}

\begin{mathpar}
  \inferrule* [lab=synchronization] {} {{M} \bc \pzero \;|\; x?F \;|\; x!C }
  \and
  \inferrule* [lab=abstraction] {} {{F} \bc (x)P}
  \and
  \inferrule* [lab=concretion] {} {{C} \bc \langle Q \rangle}
  \and
  \inferrule* [lab=process] {} {{P,Q} \bc M \;| \;P|Q \;|\; @{x}}
  \and
  \inferrule* [lab=name] {} {{x} \bc \quotep{P}}
\end{mathpar} 

Note that $\vec{x}$ (resp. $\vec{P}$) denotes a vector of names
(resp. processes) of length $|\vec{x}|$ (resp. $|\vec{P}|$). We adopt
the following useful abbreviations.

\begin{mathpar}
   x?(\vec{y}).P := x.(\vec{y})P \and  x\clift{\vec{P}} := x.\clift{\vec{P}}
   \and x!(y) := \lift{x}{\dropn{y}}
   \and \Pi_{i=0}^{n-1}P_i := P_0 | \ldots | P_{n-1}
\end{mathpar}

\subsubsection{Structural congruence}

\paragraph{Free and bound names and alpha-equivalence.} At the
core of structural equivalence is alpha-equivalence which identifies
process that are the same up to a change of variable. Formally, we
recognize the distinction between free and bound names. The free names
of a process, $\freenames{P}$, may be calculated recursively as
follows:

\begin{mathpar}
\freenames{\pzero} := \emptyset
  \and \\
  \freenames{x?(y).P} := \{ x \} \cup (\freenames{P} \setminus \{ y \})
  \and 
  \freenames{x!\langle P \rangle} := \{ x \} \cup \{ P \} 
  \and \\
  \freenames{P|Q} := \freenames{P} \cup \freenames{Q}
  \and \\
  \freenames{@{x}} := \{ x \}
\end{mathpar}

$\pi$
$\quotep{\pi}$

$\freenames{-} : \pi \to \mathcal{P}(\quotep{\pi})$

\begin{eqnarray*}
  \freenames{\pzero} & := & \emptyset \\
  \freenames{x?(y).P} & := & \{ x \} \cup (\freenames{P} \setminus \{ y \}) \\
  \freenames{x!\langle P \rangle} & := & \{ x \} \cup \{ P \} \\
  \freenames{P|Q} & := & \freenames{P} \cup \freenames{Q} \\
  \freenames{\dropn{x}} & := & \{ x \}
\end{eqnarray*}

The bound names of a process, $\boundnames{P}$, are those names occurring in $P$
that are not free. For example, in $x?(y).0$, the name $x$ is free, while $y$ is bound.

\begin{mathpar}
  \inferrule* [lab=monoidal-laws] {} { P|Q \equiv Q|P \and P|0 \equiv P \and P|(Q|R) \equiv (P|Q)|R }
\end{mathpar}

\begin{mathpar}
  \inferrule* [lab=alpha-equivalence] {} { (x)P \equiv (y)P\{y/x\} \and y \not\in \freenames{P} }
\end{mathpar}

\begin{definition}
Then two processes, $P,Q$, are alpha-equivalent if $P = Q\{\vec{y}/\vec{x}\}$ for
some $\vec{x} \in \boundnames{Q},\vec{y} \in \boundnames{P}$, where $Q\{\vec{y}/\vec{x}\}$
denotes the capture-avoiding substitution of $\vec{y}$ for $\vec{x}$ in $Q$.
\end{definition}

\begin{definition}
  The {\em structural congruence} \cite{SangiorgiWalker} , $\equiv$,
  between processes is the least congruence containing
  alpha-equivalence, satisfying the abelian monoid laws
  (associativity, commutativity and $\pzero$ as identity) for parallel
  composition $|$ and for summation $+$.
\end{definition}

\subsection{Name equivalence}

We take name equivalence, written $\nameeq$, to be the smallest
equivalence relation generated by the following rules.

\begin{mathpar}
\inferrule*[lab=Quote-drop]
{ }
{ \quotep{@{x}} \nameeq x }

\inferrule*[lab=Struct-equiv]
{ P \scong Q }
{ \quotep{P} \nameeq \quotep{Q} }
\end{mathpar}

The astute reader will have noticed that the mutual recursion of names
and processes imposes a mutual recursion on alpha-equivalence and
structural equivalence via name-equivalence. Fortunately, all of this
works out pleasantly and we may calculate in the natural way, free of
concern. The reader interested in the details is referred to the
appendix \ref{appendix:rho_details}.

\subsection{Substitution}

We use $\Proc$ for the set of processes, $\QProc$ for the set of
names, and $\id{\{}\vec{y} / \vec{x} \id{\}}$ to denote partial maps,
$s : \QProc \rightarrow \QProc$. A map, $s$ lifts, uniquely, to a map
on process terms, $\widehat{s} : \Proc \rightarrow \Proc$ by the
following equations.

\begin{mathpar}
  (0) \psubstp{Q}{P} := 0 \\
  (R \juxtap S) \psubstp{Q}{P}
  :=    
  (R)\psubstp{Q}{P} \juxtap (S) \psubstp{Q}{P} \\
  (x?(y).R) \psubstp{Q}{P}    
  :=    
  (x)\substp{Q}{P} (z)\concat( (R \psubstn{z}{y}) \psubstp{Q}{P} ) \\
  (\lift{x}{R}) \psubstp{Q}{P}  
  :=
  \lift{(x)\substp{Q}{P}}{ R \psubstp{Q}{P} } \\
%   (\dropn{x})  \psubstp{Q}{P}       
%   := 
%   \left\{ 
%     \begin{array}{ccc} 
%       \dropn{\quotep{Q}} & & x \nameeq \quotep{P} \\
%       \dropn{x} & & otherwise \\
%     \end{array}
%   \right. 
  (\dropn{x})  \psubstp{Q}{P}       
  := 
  \left\{ 
    \begin{array}{ccc} 
      Q & & x \nameeq \quotep{P} \\
      \dropn{x} & & otherwise \\
    \end{array}
  \right.
\end{mathpar}
 

where

\begin{eqnarray}
  (x)\id{\{} \lpquote Q \rpquote / \lpquote P \rpquote \id{\}}            = 
  \left\{ 
    \begin{array}{ccc}
      \lpquote Q \rpquote & & x \nameeq \lpquote P \rpquote \\
      x & & otherwise \\
    \end{array}
  \right. \nonumber
\end{eqnarray}

and $z$ is chosen distinct from $\quotep{P}$, $\quotep{Q}$, the free
names in $Q$, and all the names in $R$. Our $\alpha$-equivalence will
be built in the standard way from this substitution.

\begin{remark}\label{rem:no_self_referential_names}
  One consequence of these definitions is that $\forall P. \quotep{P}
  \not\in \freenames{P}$.
\end{remark}

\subsection{ Dynamic quote: an example }

Anticipating something of what's to come, consider applying the
substitution, $\widehat{\id{\{}u / z \id{\}}}$, to the following pair
of processes, $\lift{w}{y!(z)}$ and $w[ \lpquote y!(z) \rpquote ]$.

\begin{eqnarray}
	\lift{w}{y!(z)}\widehat{\id{\{}u / z \id{\}}}
		& = &
		\lift{w}{y!(u)} \nonumber\\
	w[ \lpquote y!(z) \rpquote ] \widehat{ \id{\{}u / z \id{\}} }
		& = &
		w[ \lpquote y!(z) \rpquote ] \nonumber
\end{eqnarray}

Because the body of the process between quotes is impervious to
substitution, we get radically different answers. In fact, by
examining the first process in an input context,
e.g. $x?(z).\lift{w}{y!(z)}$, we see that the process under the lift
operator may be shaped by prefixed inputs binding a name inside it. In
this sense, the lift operator will be seen as a way to dynamically
construct processes before reifying them as names.

Finally equipped with these standard features we can present the
dynamics of the calculus.

\subsubsection{Operational semantics} 

Finally, we introduce the computational dynamics. What marks these
algebras as distinct from other more traditionally studied algebraic
structures, e.g. vector spaces or polynomial rings, is the manner in
which dynamics is captured. In traditional structures, dynamics is typically
expressed through morphisms between such structures, as in linear maps
between vector spaces or morphisms between rings. In algebras
associated with the semantics of computation, the dynamics is
expressed as part of the algebraic structure itself, through a
reduction reduction relation typically denoted by $\red$. Below, we
give a recursive presentation of this relation for the calculus used
in the encoding.

$\red \subseteq \pi \times \pi$
$\red : \pi \to \mathcal{P}(\pi)$

\begin{mathpar}
  \inferrule* [lab=Comm] { \textsf{match}( x_{src}, x_{trgt} ) } { x_{trgt}?(y)P \; | \; x_{src}!\langle {Q} \rangle \red P\{\quotep{Q}/y}\} }
  \and \\
  \inferrule* [lab=Par] {{P} \red {P}'} {{{P} | {Q}} \red {{P}' | {Q}}}
  \and
  \inferrule* [lab=Equiv]{{{P} \scong {P}'} \andalso {{P}' \red {Q}'} \andalso {{Q}' \scong {Q}}}{{P} \red {Q}}
\end{mathpar}

\begin{eqnarray*}
  match_{\equiv} (\quotep{P},\quotep{Q}) & := & P \equiv Q \\
  match_{\dagger}(\quotep{P},\quotep{Q}) & := & \forall R. P|Q \red^{*} R => R \red^{*} 0 \\
  match_{K}(\quotep{P},\quotep{Q}) & := & K \mbox{ for some context } K
\end{eqnarray*}

$u?(x)P | u!\langle Q \rangle \red P\{\quotep{Q}/x\}$

%We write $\wred$ for $\red^*$, and $P\red$ if $\exists Q $ such that $ P \red Q$.
We write $P\red$ if $\exists Q $ such that $ P \red Q$ and $P\not\red$, otherwise.

\section{Replication}

As mentioned before, it is known that replication (and hence
recursion) can be implemented in a higher-order process algebra
\cite{SangiorgiWalker}. As our first example of calculation with the
machinery thus far presented we give the construction explicitly in
the {\rhoc}.

\begin{eqnarray}
	D_{x} & := & \prefix{x}{y}{(\binpar{\outputp{x}{y}}{@{y}})} \nonumber\\
	\bangp_{x}{P} & := & \binpar{{x}!\langle{\binpar{D_{x}}{P}}\rangle}{D_{x}} \nonumber
\end{eqnarray}

\begin{eqnarray}
	\bangp_{x}{P} & & \nonumber\\
	=
	& {x}!\langle{(\prefix{x}{y}{(\outputp{x}{y} | @{y})) | P}}\rangle 
	      | \prefix{x}{y}{(\outputp{x}{y} | @{y})} & \nonumber\\
	\red
	& (\outputp{x}{y} | @{y})\substn{\quotep{(\prefix{x}{y}{(@{y} | \outputp{x}{y})) | P}}}{y} & \nonumber\\
	=
	& \outputp{x}{\quotep{(\prefix{x}{y}{(\outputp{x}{y} | @{y})) | P}}}
	  | {(\prefix{x}{y}{(\outputp{x}{y} | @{y})) | P}} & \nonumber\\
	\red
	& \ldots & \nonumber\\
	\red^*
	& P | P | \ldots & \nonumber
\end{eqnarray}

Of course, this encoding, as an implementation, runs away, unfolding
$\bangp{P}$ eagerly. A lazier and more implementable replication
operator, restricted to input-guarded processes, may be obtained as follows.

\begin{eqnarray}
\bangp{\prefix{u}{v}{P}} 
	:= 
	\binpar{\lift{x}{\prefix{u}{v}{(\binpar{D(x)}{P})}}}{D(x)} \nonumber
\end{eqnarray}

\begin{remark}
  Note that the lazier definition still does not deal with summation
  or mixed summation (i.e. sums over input and output). The reader is
  invited to construct definitions of replication that deal with these
  features. 

  Further, the definitions are parameterized in a name, $x$. Can you,
  gentle reader, make a definition that eliminates this parameter and
  guarantees no accidental interaction between the replication
  machinery and the process being replicated -- i.e. no accidental
  sharing of names used by the process to get its work done and the
  name(s) used by the replication to effect copying. This latter
  revision of the definition of replication is crucial to obtaining
  the expected identity $!!P \sim !P$.
\end{remark}

\begin{remark}\label{rem:paradoxical_combinator}
  The reader familiar with the lambda calculus will have noticed the
  similarity between $D$ and the paradoxical combinator.

  [Ed. note: the existence of this seems to suggest we have to be more
  restrictive on the set of processes and names we admit if we are to
  support no-cloning.]
\end{remark}

\subsubsection{Bisimulation}

The computational dynamics gives rise to another kind of equivalence,
the equivalence of computational behavior. As previously mentioned
this is typically captured \emph{via} some form of bisimulation.

% The notion we use in this paper is weak barbed bisimulation
% \cite{milner91polyadicpi}.

The notion we use in this paper is derived from weak barbed
bisimulation \cite{milner91polyadicpi}. 

\begin{definition}
An \emph{observation relation}, $\downarrow_{\mathcal N}$, over a set
of names, $\mathcal N$, is the smallest relation satisfying the rules
below.

\infrule[Out-barb]{y \in {\mathcal N}, \; x \nameeq y}
		  {\outputp{x}{v} \downarrow_{\mathcal N} x}
\infrule[Par-barb]{\mbox{$P\downarrow_{\mathcal N} x$ or $Q\downarrow_{\mathcal N} x$}}
		  {\binpar{P}{Q} \downarrow_{\mathcal N} x}

We write $P \Downarrow_{\mathcal N} x$ if there is $Q$ such that 
$P \wred Q$ and $Q \downarrow_{\mathcal N} x$.
\end{definition}

\begin{definition}
%\label{def.bbisim}
An  ${\mathcal N}$-\emph{barbed bisimulation} over a set of names, ${\mathcal N}$, is a symmetric binary relation 
${\mathcal S}_{\mathcal N}$ between agents such that $P\rel{S}_{\mathcal N}Q$ implies:
\begin{enumerate}
\item If $P \red P'$ then $Q \wred Q'$ and $P'\rel{S}_{\mathcal N} Q'$.
\item If $P\downarrow_{\mathcal N} x$, then $Q\Downarrow_{\mathcal N} x$.
\end{enumerate}
$P$ is ${\mathcal N}$-barbed bisimilar to $Q$, written
$P \wbbisim_{\mathcal N} Q$, if $P \rel{S}_{\mathcal N} Q$ for some ${\mathcal N}$-barbed bisimulation ${\mathcal S}_{\mathcal N}$.
\end{definition}

$\mathcal{R} \subseteq \pi \times \pi$

$P \mathcal{R} Q => \forall P'. P \red P' \Rightarrow \exists Q'. Q \red Q', P' \mathcal{R} Q'$

$P \vdash x \Rightarrow Q \vdash x$

\begin{mathpar}
  \inferrule*[lab=Out-barb]{x \nameeq y}{{y}!\langle{Q}\rangle \vdash x}
  \and
  \inferrule*[lab=Par-barb]{\mbox{$P\vdash x$ or $Q\vdash x$}}{\binpar{P}{Q} \vdash x}
\end{mathpar}

\subsubsection{Contexts}

One of the principle advantages of computational calculi like the
$\pi$-calculus is a well-defined notion of context,
contextual-equivalence and a correlation between
contextual-equivalence and notions of bisimulation. The notion of
context allows the decomposition of a process into (sub-)process and
its syntactic environment, its context. Thus, a context may be
thought of as a process with a ``hole'' (written $\Box$) in it. The
application of a context $M$ to a process $P$, written $M[P]$, is
tantamount to filling the hole in $M$ with $P$. In this paper we do
not need the full weight of this theory, but do make use of the notion
of context in the proof the main theorem. 

\begin{mathpar}
  \inferrule* [lab=summation] {} {{M_{M},M_{N}} \bc \Box \;|\; x.M_{A} \;|\; M_{M}+M_{N}}
  \and
  \inferrule* [lab=agent] {} {{M_{A}} \bc (\vec{x})M_{P} \;| \; \clift{P_0,\ldots,M_{P},\ldots,P_N}}
  \and \\
  \inferrule* [lab=process] {} {{M_{P}} \bc M_{N} \;| \;P|M_{P} }
\end{mathpar} 

\begin{mathpar}
  \inferrule* [lab=sychronization] {} {M_{N} \bc \Box \;|\; x?M_{F} \;|\; x!M_{C}}
  \and
  \inferrule* [lab=abstraction] {} {{M_{F}} \bc (x)M_{P} }
  \and
  \inferrule* [lab=concretion] {} {{M_{C}} \bc \langle M_{P} \rangle }
  \and \\
  \inferrule* [lab=process] {} {{M_{P}} \bc M_{N} \;| \;P|M_{P} }
\end{mathpar}

\begin{definition}[contextual application] Given a context $M$, and
  process $P$, we define the \emph{contextual application}, $M[P] :=
  M\{P/\Box\}$. That is, the contextual application of M to P is the
  substitution of $P$ for $\Box$ in $M$.
\end{definition}

$\meaningof{-} : L \to \mathcal{P}(\pi)$

\begin{mathpar}
  \inferrule* [lab=collection] {} {\meaningof{true} = \pi, \and \meaningof{~E} = \pi \setminus \meaningof{E}, \and \meaningof{E_{1} \& E_{2}} = \meaningof{E_{1}} \cap \meaningof{E_{2}}}
\end{mathpar}

\begin{mathpar}
  \inferrule* [lab=structure] {} {\meaningof{0} = \{ P \in \pi | P \equiv 0 \}, \and \\ \meaningof{E_1 | E_2} = \{ P \in \pi | P \equiv P_{1} | P_{2}, P_{1} \in \meaningof{E_{1}}, P_{2} \in \meaningof{E_2}\} }
\end{mathpar}

\begin{mathpar}
 \inferrule* [lab=behavior] {} {\meaningof{\langle a?b \rangle E} = \{ P \in \pi | P \equiv Q | u?(y)P', \\ \and \\\\ \and \\ \;\;\; u \in \meaningof{a}, \forall z.P'\{z/y\} \in \meaningof{E\{z/b\}}\}, \and \\ \meaningof{a!E} = \{ P \in \pi | P \equiv Q | x!\langle P' \rangle, x \in \meaningof{a} P' \in \meaningof{E}\} }
\end{mathpar}

\begin{mathpar}
 \inferrule* [lab=nominal] {} {\meaningof{\quotep{E}} = \{ \quotep{P} \in \quotep{\pi} | P \in \meaningof{E} \}, \and \meaningof{\quotep{P}} = \{ \quotep{Q} \in \quotep{\pi} | P \equiv Q \} \and \\ \meaningof{@\quotep{E}} = \{ P \in \pi | P \equiv @x, x \in \meaningof{E} \}}
\end{mathpar}

\begin{eqnarray*}
  \\
  \meaningof{-} : TS \to ST
\end{eqnarray*}

\begin{eqnarray*}
  \\
  L : TS \to ST
\end{eqnarray*}

\begin{eqnarray*}
  \\
  P \models E \iff P \in \meaningof{E}
\end{eqnarray*}

\begin{eqnarray*}
  P \approx_{L} Q \iff \forall E \in L. P \models E \iff Q \models E
\end{eqnarray*}

\begin{eqnarray*}
  P \approx_{K} Q
\end{eqnarray*}

\begin{eqnarray*}
  P \approx Q
\end{eqnarray*}

$\approx_{K} = \approx = \approx_{L}$

\subsubsection{Contextual duality}

Note that contexts extend the quotation operation to a family of
operations from processes to names. Given a context, $M$, we can
define a \emph{nominal context}, $\quotep{M}$ by $\quotep{M}[P] :=
\quotep{M[P]}$. To foreshadow what is to come we observe that these
operations enjoy a duality with processes very much like the duality
between vectors and maps from vectors to scalars.

Further, because the calculus is essentially higher-order, we have a
correspondence between contexts and processes. More specifically,
given a name $x$ and a context $M$ we can construct $M^{*}_{x}$ such
that 

\begin{mathpar}
  M^{*}_{x} | \lift{x}{P} \red M[P]
\end{mathpar}

namely,

\begin{mathpar}
  M^{*}_{x} := x?(u).M[\dropn{u}]
\end{mathpar}

The dependence of $M^{*}_{x}$ on a name makes it an abstraction, 

\begin{mathpar}
  M^{*} := (x)x?(u).M[\dropn{u}]
\end{mathpar}

\subsection{Additional notation}

It will sometimes be convenient to denote the process a name
quotes. We already have the notation $x = \quotep{P}$, but it will be
convenient to introduce an alternate notation, $\procn{x}$, when we
want to emphasize the connection to the use of the name. Note that, by
virtue of name equivalence, $\quotep{\procn{x}} \nameeq x$; so, the
notation is consistent with previous definitions.

Further, because names have structure it is possible to effect
substitutions on the basis of that structure. This means we need to
upgrade our notation for substitutions, which we accomplish by
adapting comprehension notation. Thus,

\begin{mathpar}
  P\{ y / x : x \in S \}
\end{mathpar}

is interpreted to mean the process derived from P by replacing (in a
capture-avoiding manner) each occurrence of $x$ in $S$ by $y$. For example,

\begin{mathpar}
  P\{ \quotep{\procn{x}|\procn{x}} / x : x \in \freenames{P} \}
\end{mathpar}

will replace each (occurrence) of a free name $x$ in $P$ by
$\quotep{\procn{x}|\procn{x}}$.

Also, we will avail ourselves of the notation $x^{L}$ and $x^{R}$ to
denote injections of a name into disjoint copies of the name
space. There are numerous ways to accomplish this. One example can be
found in \cite{MeredithR05}. This notation overloads to vectors of
names: $\vec{x}^{\pi} := (x_{i}^{\pi} \; : \; 0 \leq i < |\vec{x}| )$ where $\pi \in \{L,R\}$.

We also use $P^{\Box} := P|\Box$.

In \cite{MeredithR05} an interpretation of the new operator is
given. It turns out that there are several possible interpretations
all enjoying the requisite algebraic properties of the operator (see
\cite{milner91polyadicpi}). We will therefore make liberal use of
$(\nu\; \vec{x})P$.

% subsection the_syntax_and_semantics_of_the_notation_system (end)   

\input{qm2pi.qmops} 

\input{qm2pi.sterngerlach} 

\input{qm2pi.metric} 

% section concurrent_process_calculi (end)

%\input{qm2pi.proofsketch}

% section proof sketch (end)

%\input{qm2pi.slviaknots} 

% section spatial logic via knots (end)

\input{qm2pi.conclusion}

% section conclusion (end)

%\input{qm2pi.dtcodes} 

% section wiring algorithm (end)

\input{qm2pi.ack} 

% section acknowledgments (end)

\newpage


\bibliographystyle{plain}   
\bibliography{../../biblios/main.bib}

\input{qm2pi.rhodetails}

\end{document}

 

% section wiring algorithm (end)

\documentclass[12pt]{llncs}
%\documentclass{jktr}

\usepackage[pdftex]{hyperref}                   
\usepackage {listings}
\usepackage {mathpartir}
\usepackage{bcprules}
%\usepackage{listings}
                       
\usepackage{graphicx} 
%\usepackage[margins=2.5cm,nohead,nofoot]{geometry}
%\usepackage{geometry}
\usepackage{amsfonts}
\usepackage{amstext}
\usepackage{latexsym}
\usepackage{amssymb}
\usepackage{color}


%\include{myPreamble}
\include{qm2pi.local} 

%\ifpdf
%\usepackage[pdftex]{graphicx}
%\else
%\usepackage{graphicx}
%\fi

 % \ifpdf
%  \usepackage{pdfsync}
%  \if


%\title{Brief Article}
%\author{David F. Snyder}
%\author{L.G. Meredith}

%\address{Dept. of Math., Texas State University--San Marcos, San Marcos, TX 78666}
       
\pagestyle{empty}


\begin{document}

\lstset{language=[Objective]Caml,frame=shadowbox}

\input{qm2pi.front}

% section front matter (end)

\input{qm2pi.intro} 
 
% section introduction (end)

% \input{qm2pi.knotations} 

% section notation (end)

\input{qm2pi.process.calculi} 

% section concurrent_process_calculi_and_spatial_logics_ (end)
    
%\input{qm2pi.knots2pi} 

%\input{qm2pi.trefoil} 

%\input{qm2pi.mainthm} 

% subsection basic_interpretation (end)

%\input{qm2pi.rho.presentation} 
\subsection{The syntax and semantics of the notation system}\label{sub:the_syntax_and_semantics_of_the_notation_system} % (fold)

We now summarize a technical presentation of the calculus that
embodies our theory of dynamics. The typical presentation of such a
calculus follows the style of giving generators and relations on
them. The grammar, below, describing term constructors, freely
generates the set of processes, $\Proc$. This set is then quotiented
by a relation known as structural congruence and it is over this set
that the notion of dynamics is expressed. This presentation is
essentially that of \cite{MeredithR05} with the addition of
polyadicity and summation. For readability we have relegated some of
the technical subtleties to an appendix.

\subsubsection{Process grammar}\label{subsub:process_grammar}

\begin{mathpar}
  \inferrule* [lab=synchronization] {} {{M} \bc \pzero \;|\; x?F \;|\; x!C }
  \and
  \inferrule* [lab=abstraction] {} {{F} \bc (x)P}
  \and
  \inferrule* [lab=concretion] {} {{C} \bc \langle Q \rangle}
  \and
  \inferrule* [lab=process] {} {{P,Q} \bc M \;| \;P|Q \;|\; @{x}}
  \and
  \inferrule* [lab=name] {} {{x} \bc \quotep{P}}
\end{mathpar} 

Note that $\vec{x}$ (resp. $\vec{P}$) denotes a vector of names
(resp. processes) of length $|\vec{x}|$ (resp. $|\vec{P}|$). We adopt
the following useful abbreviations.

\begin{mathpar}
   x?(\vec{y}).P := x.(\vec{y})P \and  x\clift{\vec{P}} := x.\clift{\vec{P}}
   \and x!(y) := \lift{x}{\dropn{y}}
   \and \Pi_{i=0}^{n-1}P_i := P_0 | \ldots | P_{n-1}
\end{mathpar}

\subsubsection{Structural congruence}

\paragraph{Free and bound names and alpha-equivalence.} At the
core of structural equivalence is alpha-equivalence which identifies
process that are the same up to a change of variable. Formally, we
recognize the distinction between free and bound names. The free names
of a process, $\freenames{P}$, may be calculated recursively as
follows:

\begin{mathpar}
\freenames{\pzero} := \emptyset
  \and \\
  \freenames{x?(y).P} := \{ x \} \cup (\freenames{P} \setminus \{ y \})
  \and 
  \freenames{x!\langle P \rangle} := \{ x \} \cup \{ P \} 
  \and \\
  \freenames{P|Q} := \freenames{P} \cup \freenames{Q}
  \and \\
  \freenames{@{x}} := \{ x \}
\end{mathpar}

$\pi$
$\quotep{\pi}$

$\freenames{-} : \pi \to \mathcal{P}(\quotep{\pi})$

\begin{eqnarray*}
  \freenames{\pzero} & := & \emptyset \\
  \freenames{x?(y).P} & := & \{ x \} \cup (\freenames{P} \setminus \{ y \}) \\
  \freenames{x!\langle P \rangle} & := & \{ x \} \cup \{ P \} \\
  \freenames{P|Q} & := & \freenames{P} \cup \freenames{Q} \\
  \freenames{\dropn{x}} & := & \{ x \}
\end{eqnarray*}

The bound names of a process, $\boundnames{P}$, are those names occurring in $P$
that are not free. For example, in $x?(y).0$, the name $x$ is free, while $y$ is bound.

\begin{mathpar}
  \inferrule* [lab=monoidal-laws] {} { P|Q \equiv Q|P \and P|0 \equiv P \and P|(Q|R) \equiv (P|Q)|R }
\end{mathpar}

\begin{mathpar}
  \inferrule* [lab=alpha-equivalence] {} { (x)P \equiv (y)P\{y/x\} \and y \not\in \freenames{P} }
\end{mathpar}

\begin{definition}
Then two processes, $P,Q$, are alpha-equivalent if $P = Q\{\vec{y}/\vec{x}\}$ for
some $\vec{x} \in \boundnames{Q},\vec{y} \in \boundnames{P}$, where $Q\{\vec{y}/\vec{x}\}$
denotes the capture-avoiding substitution of $\vec{y}$ for $\vec{x}$ in $Q$.
\end{definition}

\begin{definition}
  The {\em structural congruence} \cite{SangiorgiWalker} , $\equiv$,
  between processes is the least congruence containing
  alpha-equivalence, satisfying the abelian monoid laws
  (associativity, commutativity and $\pzero$ as identity) for parallel
  composition $|$ and for summation $+$.
\end{definition}

\subsection{Name equivalence}

We take name equivalence, written $\nameeq$, to be the smallest
equivalence relation generated by the following rules.

\begin{mathpar}
\inferrule*[lab=Quote-drop]
{ }
{ \quotep{@{x}} \nameeq x }

\inferrule*[lab=Struct-equiv]
{ P \scong Q }
{ \quotep{P} \nameeq \quotep{Q} }
\end{mathpar}

The astute reader will have noticed that the mutual recursion of names
and processes imposes a mutual recursion on alpha-equivalence and
structural equivalence via name-equivalence. Fortunately, all of this
works out pleasantly and we may calculate in the natural way, free of
concern. The reader interested in the details is referred to the
appendix \ref{appendix:rho_details}.

\subsection{Substitution}

We use $\Proc$ for the set of processes, $\QProc$ for the set of
names, and $\id{\{}\vec{y} / \vec{x} \id{\}}$ to denote partial maps,
$s : \QProc \rightarrow \QProc$. A map, $s$ lifts, uniquely, to a map
on process terms, $\widehat{s} : \Proc \rightarrow \Proc$ by the
following equations.

\begin{mathpar}
  (0) \psubstp{Q}{P} := 0 \\
  (R \juxtap S) \psubstp{Q}{P}
  :=    
  (R)\psubstp{Q}{P} \juxtap (S) \psubstp{Q}{P} \\
  (x?(y).R) \psubstp{Q}{P}    
  :=    
  (x)\substp{Q}{P} (z)\concat( (R \psubstn{z}{y}) \psubstp{Q}{P} ) \\
  (\lift{x}{R}) \psubstp{Q}{P}  
  :=
  \lift{(x)\substp{Q}{P}}{ R \psubstp{Q}{P} } \\
%   (\dropn{x})  \psubstp{Q}{P}       
%   := 
%   \left\{ 
%     \begin{array}{ccc} 
%       \dropn{\quotep{Q}} & & x \nameeq \quotep{P} \\
%       \dropn{x} & & otherwise \\
%     \end{array}
%   \right. 
  (\dropn{x})  \psubstp{Q}{P}       
  := 
  \left\{ 
    \begin{array}{ccc} 
      Q & & x \nameeq \quotep{P} \\
      \dropn{x} & & otherwise \\
    \end{array}
  \right.
\end{mathpar}
 

where

\begin{eqnarray}
  (x)\id{\{} \lpquote Q \rpquote / \lpquote P \rpquote \id{\}}            = 
  \left\{ 
    \begin{array}{ccc}
      \lpquote Q \rpquote & & x \nameeq \lpquote P \rpquote \\
      x & & otherwise \\
    \end{array}
  \right. \nonumber
\end{eqnarray}

and $z$ is chosen distinct from $\quotep{P}$, $\quotep{Q}$, the free
names in $Q$, and all the names in $R$. Our $\alpha$-equivalence will
be built in the standard way from this substitution.

\begin{remark}\label{rem:no_self_referential_names}
  One consequence of these definitions is that $\forall P. \quotep{P}
  \not\in \freenames{P}$.
\end{remark}

\subsection{ Dynamic quote: an example }

Anticipating something of what's to come, consider applying the
substitution, $\widehat{\id{\{}u / z \id{\}}}$, to the following pair
of processes, $\lift{w}{y!(z)}$ and $w[ \lpquote y!(z) \rpquote ]$.

\begin{eqnarray}
	\lift{w}{y!(z)}\widehat{\id{\{}u / z \id{\}}}
		& = &
		\lift{w}{y!(u)} \nonumber\\
	w[ \lpquote y!(z) \rpquote ] \widehat{ \id{\{}u / z \id{\}} }
		& = &
		w[ \lpquote y!(z) \rpquote ] \nonumber
\end{eqnarray}

Because the body of the process between quotes is impervious to
substitution, we get radically different answers. In fact, by
examining the first process in an input context,
e.g. $x?(z).\lift{w}{y!(z)}$, we see that the process under the lift
operator may be shaped by prefixed inputs binding a name inside it. In
this sense, the lift operator will be seen as a way to dynamically
construct processes before reifying them as names.

Finally equipped with these standard features we can present the
dynamics of the calculus.

\subsubsection{Operational semantics} 

Finally, we introduce the computational dynamics. What marks these
algebras as distinct from other more traditionally studied algebraic
structures, e.g. vector spaces or polynomial rings, is the manner in
which dynamics is captured. In traditional structures, dynamics is typically
expressed through morphisms between such structures, as in linear maps
between vector spaces or morphisms between rings. In algebras
associated with the semantics of computation, the dynamics is
expressed as part of the algebraic structure itself, through a
reduction reduction relation typically denoted by $\red$. Below, we
give a recursive presentation of this relation for the calculus used
in the encoding.

$\red \subseteq \pi \times \pi$
$\red : \pi \to \mathcal{P}(\pi)$

\begin{mathpar}
  \inferrule* [lab=Comm] { \textsf{match}( x_{src}, x_{trgt} ) } { x_{trgt}?(y)P \; | \; x_{src}!\langle {Q} \rangle \red P\{\quotep{Q}/y}\} }
  \and \\
  \inferrule* [lab=Par] {{P} \red {P}'} {{{P} | {Q}} \red {{P}' | {Q}}}
  \and
  \inferrule* [lab=Equiv]{{{P} \scong {P}'} \andalso {{P}' \red {Q}'} \andalso {{Q}' \scong {Q}}}{{P} \red {Q}}
\end{mathpar}

\begin{eqnarray*}
  match_{\equiv} (\quotep{P},\quotep{Q}) & := & P \equiv Q \\
  match_{\dagger}(\quotep{P},\quotep{Q}) & := & \forall R. P|Q \red^{*} R => R \red^{*} 0 \\
  match_{K}(\quotep{P},\quotep{Q}) & := & K \mbox{ for some context } K
\end{eqnarray*}

$u?(x)P | u!\langle Q \rangle \red P\{\quotep{Q}/x\}$

%We write $\wred$ for $\red^*$, and $P\red$ if $\exists Q $ such that $ P \red Q$.
We write $P\red$ if $\exists Q $ such that $ P \red Q$ and $P\not\red$, otherwise.

\section{Replication}

As mentioned before, it is known that replication (and hence
recursion) can be implemented in a higher-order process algebra
\cite{SangiorgiWalker}. As our first example of calculation with the
machinery thus far presented we give the construction explicitly in
the {\rhoc}.

\begin{eqnarray}
	D_{x} & := & \prefix{x}{y}{(\binpar{\outputp{x}{y}}{@{y}})} \nonumber\\
	\bangp_{x}{P} & := & \binpar{{x}!\langle{\binpar{D_{x}}{P}}\rangle}{D_{x}} \nonumber
\end{eqnarray}

\begin{eqnarray}
	\bangp_{x}{P} & & \nonumber\\
	=
	& {x}!\langle{(\prefix{x}{y}{(\outputp{x}{y} | @{y})) | P}}\rangle 
	      | \prefix{x}{y}{(\outputp{x}{y} | @{y})} & \nonumber\\
	\red
	& (\outputp{x}{y} | @{y})\substn{\quotep{(\prefix{x}{y}{(@{y} | \outputp{x}{y})) | P}}}{y} & \nonumber\\
	=
	& \outputp{x}{\quotep{(\prefix{x}{y}{(\outputp{x}{y} | @{y})) | P}}}
	  | {(\prefix{x}{y}{(\outputp{x}{y} | @{y})) | P}} & \nonumber\\
	\red
	& \ldots & \nonumber\\
	\red^*
	& P | P | \ldots & \nonumber
\end{eqnarray}

Of course, this encoding, as an implementation, runs away, unfolding
$\bangp{P}$ eagerly. A lazier and more implementable replication
operator, restricted to input-guarded processes, may be obtained as follows.

\begin{eqnarray}
\bangp{\prefix{u}{v}{P}} 
	:= 
	\binpar{\lift{x}{\prefix{u}{v}{(\binpar{D(x)}{P})}}}{D(x)} \nonumber
\end{eqnarray}

\begin{remark}
  Note that the lazier definition still does not deal with summation
  or mixed summation (i.e. sums over input and output). The reader is
  invited to construct definitions of replication that deal with these
  features. 

  Further, the definitions are parameterized in a name, $x$. Can you,
  gentle reader, make a definition that eliminates this parameter and
  guarantees no accidental interaction between the replication
  machinery and the process being replicated -- i.e. no accidental
  sharing of names used by the process to get its work done and the
  name(s) used by the replication to effect copying. This latter
  revision of the definition of replication is crucial to obtaining
  the expected identity $!!P \sim !P$.
\end{remark}

\begin{remark}\label{rem:paradoxical_combinator}
  The reader familiar with the lambda calculus will have noticed the
  similarity between $D$ and the paradoxical combinator.

  [Ed. note: the existence of this seems to suggest we have to be more
  restrictive on the set of processes and names we admit if we are to
  support no-cloning.]
\end{remark}

\subsubsection{Bisimulation}

The computational dynamics gives rise to another kind of equivalence,
the equivalence of computational behavior. As previously mentioned
this is typically captured \emph{via} some form of bisimulation.

% The notion we use in this paper is weak barbed bisimulation
% \cite{milner91polyadicpi}.

The notion we use in this paper is derived from weak barbed
bisimulation \cite{milner91polyadicpi}. 

\begin{definition}
An \emph{observation relation}, $\downarrow_{\mathcal N}$, over a set
of names, $\mathcal N$, is the smallest relation satisfying the rules
below.

\infrule[Out-barb]{y \in {\mathcal N}, \; x \nameeq y}
		  {\outputp{x}{v} \downarrow_{\mathcal N} x}
\infrule[Par-barb]{\mbox{$P\downarrow_{\mathcal N} x$ or $Q\downarrow_{\mathcal N} x$}}
		  {\binpar{P}{Q} \downarrow_{\mathcal N} x}

We write $P \Downarrow_{\mathcal N} x$ if there is $Q$ such that 
$P \wred Q$ and $Q \downarrow_{\mathcal N} x$.
\end{definition}

\begin{definition}
%\label{def.bbisim}
An  ${\mathcal N}$-\emph{barbed bisimulation} over a set of names, ${\mathcal N}$, is a symmetric binary relation 
${\mathcal S}_{\mathcal N}$ between agents such that $P\rel{S}_{\mathcal N}Q$ implies:
\begin{enumerate}
\item If $P \red P'$ then $Q \wred Q'$ and $P'\rel{S}_{\mathcal N} Q'$.
\item If $P\downarrow_{\mathcal N} x$, then $Q\Downarrow_{\mathcal N} x$.
\end{enumerate}
$P$ is ${\mathcal N}$-barbed bisimilar to $Q$, written
$P \wbbisim_{\mathcal N} Q$, if $P \rel{S}_{\mathcal N} Q$ for some ${\mathcal N}$-barbed bisimulation ${\mathcal S}_{\mathcal N}$.
\end{definition}

$\mathcal{R} \subseteq \pi \times \pi$

$P \mathcal{R} Q => \forall P'. P \red P' \Rightarrow \exists Q'. Q \red Q', P' \mathcal{R} Q'$

$P \vdash x \Rightarrow Q \vdash x$

\begin{mathpar}
  \inferrule*[lab=Out-barb]{x \nameeq y}{{y}!\langle{Q}\rangle \vdash x}
  \and
  \inferrule*[lab=Par-barb]{\mbox{$P\vdash x$ or $Q\vdash x$}}{\binpar{P}{Q} \vdash x}
\end{mathpar}

\subsubsection{Contexts}

One of the principle advantages of computational calculi like the
$\pi$-calculus is a well-defined notion of context,
contextual-equivalence and a correlation between
contextual-equivalence and notions of bisimulation. The notion of
context allows the decomposition of a process into (sub-)process and
its syntactic environment, its context. Thus, a context may be
thought of as a process with a ``hole'' (written $\Box$) in it. The
application of a context $M$ to a process $P$, written $M[P]$, is
tantamount to filling the hole in $M$ with $P$. In this paper we do
not need the full weight of this theory, but do make use of the notion
of context in the proof the main theorem. 

\begin{mathpar}
  \inferrule* [lab=summation] {} {{M_{M},M_{N}} \bc \Box \;|\; x.M_{A} \;|\; M_{M}+M_{N}}
  \and
  \inferrule* [lab=agent] {} {{M_{A}} \bc (\vec{x})M_{P} \;| \; \clift{P_0,\ldots,M_{P},\ldots,P_N}}
  \and \\
  \inferrule* [lab=process] {} {{M_{P}} \bc M_{N} \;| \;P|M_{P} }
\end{mathpar} 

\begin{mathpar}
  \inferrule* [lab=sychronization] {} {M_{N} \bc \Box \;|\; x?M_{F} \;|\; x!M_{C}}
  \and
  \inferrule* [lab=abstraction] {} {{M_{F}} \bc (x)M_{P} }
  \and
  \inferrule* [lab=concretion] {} {{M_{C}} \bc \langle M_{P} \rangle }
  \and \\
  \inferrule* [lab=process] {} {{M_{P}} \bc M_{N} \;| \;P|M_{P} }
\end{mathpar}

\begin{definition}[contextual application] Given a context $M$, and
  process $P$, we define the \emph{contextual application}, $M[P] :=
  M\{P/\Box\}$. That is, the contextual application of M to P is the
  substitution of $P$ for $\Box$ in $M$.
\end{definition}

$\meaningof{-} : L \to \mathcal{P}(\pi)$

\begin{mathpar}
  \inferrule* [lab=collection] {} {\meaningof{true} = \pi, \and \meaningof{~E} = \pi \setminus \meaningof{E}, \and \meaningof{E_{1} \& E_{2}} = \meaningof{E_{1}} \cap \meaningof{E_{2}}}
\end{mathpar}

\begin{mathpar}
  \inferrule* [lab=structure] {} {\meaningof{0} = \{ P \in \pi | P \equiv 0 \}, \and \\ \meaningof{E_1 | E_2} = \{ P \in \pi | P \equiv P_{1} | P_{2}, P_{1} \in \meaningof{E_{1}}, P_{2} \in \meaningof{E_2}\} }
\end{mathpar}

\begin{mathpar}
 \inferrule* [lab=behavior] {} {\meaningof{\langle a?b \rangle E} = \{ P \in \pi | P \equiv Q | u?(y)P', \\ \and \\\\ \and \\ \;\;\; u \in \meaningof{a}, \forall z.P'\{z/y\} \in \meaningof{E\{z/b\}}\}, \and \\ \meaningof{a!E} = \{ P \in \pi | P \equiv Q | x!\langle P' \rangle, x \in \meaningof{a} P' \in \meaningof{E}\} }
\end{mathpar}

\begin{mathpar}
 \inferrule* [lab=nominal] {} {\meaningof{\quotep{E}} = \{ \quotep{P} \in \quotep{\pi} | P \in \meaningof{E} \}, \and \meaningof{\quotep{P}} = \{ \quotep{Q} \in \quotep{\pi} | P \equiv Q \} \and \\ \meaningof{@\quotep{E}} = \{ P \in \pi | P \equiv @x, x \in \meaningof{E} \}}
\end{mathpar}

\begin{eqnarray*}
  \\
  \meaningof{-} : TS \to ST
\end{eqnarray*}

\begin{eqnarray*}
  \\
  L : TS \to ST
\end{eqnarray*}

\begin{eqnarray*}
  \\
  P \models E \iff P \in \meaningof{E}
\end{eqnarray*}

\begin{eqnarray*}
  P \approx_{L} Q \iff \forall E \in L. P \models E \iff Q \models E
\end{eqnarray*}

\begin{eqnarray*}
  P \approx_{K} Q
\end{eqnarray*}

\begin{eqnarray*}
  P \approx Q
\end{eqnarray*}

$\approx_{K} = \approx = \approx_{L}$

\subsubsection{Contextual duality}

Note that contexts extend the quotation operation to a family of
operations from processes to names. Given a context, $M$, we can
define a \emph{nominal context}, $\quotep{M}$ by $\quotep{M}[P] :=
\quotep{M[P]}$. To foreshadow what is to come we observe that these
operations enjoy a duality with processes very much like the duality
between vectors and maps from vectors to scalars.

Further, because the calculus is essentially higher-order, we have a
correspondence between contexts and processes. More specifically,
given a name $x$ and a context $M$ we can construct $M^{*}_{x}$ such
that 

\begin{mathpar}
  M^{*}_{x} | \lift{x}{P} \red M[P]
\end{mathpar}

namely,

\begin{mathpar}
  M^{*}_{x} := x?(u).M[\dropn{u}]
\end{mathpar}

The dependence of $M^{*}_{x}$ on a name makes it an abstraction, 

\begin{mathpar}
  M^{*} := (x)x?(u).M[\dropn{u}]
\end{mathpar}

\subsection{Additional notation}

It will sometimes be convenient to denote the process a name
quotes. We already have the notation $x = \quotep{P}$, but it will be
convenient to introduce an alternate notation, $\procn{x}$, when we
want to emphasize the connection to the use of the name. Note that, by
virtue of name equivalence, $\quotep{\procn{x}} \nameeq x$; so, the
notation is consistent with previous definitions.

Further, because names have structure it is possible to effect
substitutions on the basis of that structure. This means we need to
upgrade our notation for substitutions, which we accomplish by
adapting comprehension notation. Thus,

\begin{mathpar}
  P\{ y / x : x \in S \}
\end{mathpar}

is interpreted to mean the process derived from P by replacing (in a
capture-avoiding manner) each occurrence of $x$ in $S$ by $y$. For example,

\begin{mathpar}
  P\{ \quotep{\procn{x}|\procn{x}} / x : x \in \freenames{P} \}
\end{mathpar}

will replace each (occurrence) of a free name $x$ in $P$ by
$\quotep{\procn{x}|\procn{x}}$.

Also, we will avail ourselves of the notation $x^{L}$ and $x^{R}$ to
denote injections of a name into disjoint copies of the name
space. There are numerous ways to accomplish this. One example can be
found in \cite{MeredithR05}. This notation overloads to vectors of
names: $\vec{x}^{\pi} := (x_{i}^{\pi} \; : \; 0 \leq i < |\vec{x}| )$ where $\pi \in \{L,R\}$.

We also use $P^{\Box} := P|\Box$.

In \cite{MeredithR05} an interpretation of the new operator is
given. It turns out that there are several possible interpretations
all enjoying the requisite algebraic properties of the operator (see
\cite{milner91polyadicpi}). We will therefore make liberal use of
$(\nu\; \vec{x})P$.

% subsection the_syntax_and_semantics_of_the_notation_system (end)   

\input{qm2pi.qmops} 

\input{qm2pi.sterngerlach} 

\input{qm2pi.metric} 

% section concurrent_process_calculi (end)

%\input{qm2pi.proofsketch}

% section proof sketch (end)

%\input{qm2pi.slviaknots} 

% section spatial logic via knots (end)

\input{qm2pi.conclusion}

% section conclusion (end)

%\input{qm2pi.dtcodes} 

% section wiring algorithm (end)

\input{qm2pi.ack} 

% section acknowledgments (end)

\newpage


\bibliographystyle{plain}   
\bibliography{../../biblios/main.bib}

\input{qm2pi.rhodetails}

\end{document}

 

% section acknowledgments (end)

\newpage


\bibliographystyle{plain}   
\bibliography{../../biblios/main.bib}

\documentclass[12pt]{llncs}
%\documentclass{jktr}

\usepackage[pdftex]{hyperref}                   
\usepackage {listings}
\usepackage {mathpartir}
\usepackage{bcprules}
%\usepackage{listings}
                       
\usepackage{graphicx} 
%\usepackage[margins=2.5cm,nohead,nofoot]{geometry}
%\usepackage{geometry}
\usepackage{amsfonts}
\usepackage{amstext}
\usepackage{latexsym}
\usepackage{amssymb}
\usepackage{color}


%\include{myPreamble}
\include{qm2pi.local} 

%\ifpdf
%\usepackage[pdftex]{graphicx}
%\else
%\usepackage{graphicx}
%\fi

 % \ifpdf
%  \usepackage{pdfsync}
%  \if


%\title{Brief Article}
%\author{David F. Snyder}
%\author{L.G. Meredith}

%\address{Dept. of Math., Texas State University--San Marcos, San Marcos, TX 78666}
       
\pagestyle{empty}


\begin{document}

\lstset{language=[Objective]Caml,frame=shadowbox}

\input{qm2pi.front}

% section front matter (end)

\input{qm2pi.intro} 
 
% section introduction (end)

% \input{qm2pi.knotations} 

% section notation (end)

\input{qm2pi.process.calculi} 

% section concurrent_process_calculi_and_spatial_logics_ (end)
    
%\input{qm2pi.knots2pi} 

%\input{qm2pi.trefoil} 

%\input{qm2pi.mainthm} 

% subsection basic_interpretation (end)

%\input{qm2pi.rho.presentation} 
\subsection{The syntax and semantics of the notation system}\label{sub:the_syntax_and_semantics_of_the_notation_system} % (fold)

We now summarize a technical presentation of the calculus that
embodies our theory of dynamics. The typical presentation of such a
calculus follows the style of giving generators and relations on
them. The grammar, below, describing term constructors, freely
generates the set of processes, $\Proc$. This set is then quotiented
by a relation known as structural congruence and it is over this set
that the notion of dynamics is expressed. This presentation is
essentially that of \cite{MeredithR05} with the addition of
polyadicity and summation. For readability we have relegated some of
the technical subtleties to an appendix.

\subsubsection{Process grammar}\label{subsub:process_grammar}

\begin{mathpar}
  \inferrule* [lab=synchronization] {} {{M} \bc \pzero \;|\; x?F \;|\; x!C }
  \and
  \inferrule* [lab=abstraction] {} {{F} \bc (x)P}
  \and
  \inferrule* [lab=concretion] {} {{C} \bc \langle Q \rangle}
  \and
  \inferrule* [lab=process] {} {{P,Q} \bc M \;| \;P|Q \;|\; @{x}}
  \and
  \inferrule* [lab=name] {} {{x} \bc \quotep{P}}
\end{mathpar} 

Note that $\vec{x}$ (resp. $\vec{P}$) denotes a vector of names
(resp. processes) of length $|\vec{x}|$ (resp. $|\vec{P}|$). We adopt
the following useful abbreviations.

\begin{mathpar}
   x?(\vec{y}).P := x.(\vec{y})P \and  x\clift{\vec{P}} := x.\clift{\vec{P}}
   \and x!(y) := \lift{x}{\dropn{y}}
   \and \Pi_{i=0}^{n-1}P_i := P_0 | \ldots | P_{n-1}
\end{mathpar}

\subsubsection{Structural congruence}

\paragraph{Free and bound names and alpha-equivalence.} At the
core of structural equivalence is alpha-equivalence which identifies
process that are the same up to a change of variable. Formally, we
recognize the distinction between free and bound names. The free names
of a process, $\freenames{P}$, may be calculated recursively as
follows:

\begin{mathpar}
\freenames{\pzero} := \emptyset
  \and \\
  \freenames{x?(y).P} := \{ x \} \cup (\freenames{P} \setminus \{ y \})
  \and 
  \freenames{x!\langle P \rangle} := \{ x \} \cup \{ P \} 
  \and \\
  \freenames{P|Q} := \freenames{P} \cup \freenames{Q}
  \and \\
  \freenames{@{x}} := \{ x \}
\end{mathpar}

$\pi$
$\quotep{\pi}$

$\freenames{-} : \pi \to \mathcal{P}(\quotep{\pi})$

\begin{eqnarray*}
  \freenames{\pzero} & := & \emptyset \\
  \freenames{x?(y).P} & := & \{ x \} \cup (\freenames{P} \setminus \{ y \}) \\
  \freenames{x!\langle P \rangle} & := & \{ x \} \cup \{ P \} \\
  \freenames{P|Q} & := & \freenames{P} \cup \freenames{Q} \\
  \freenames{\dropn{x}} & := & \{ x \}
\end{eqnarray*}

The bound names of a process, $\boundnames{P}$, are those names occurring in $P$
that are not free. For example, in $x?(y).0$, the name $x$ is free, while $y$ is bound.

\begin{mathpar}
  \inferrule* [lab=monoidal-laws] {} { P|Q \equiv Q|P \and P|0 \equiv P \and P|(Q|R) \equiv (P|Q)|R }
\end{mathpar}

\begin{mathpar}
  \inferrule* [lab=alpha-equivalence] {} { (x)P \equiv (y)P\{y/x\} \and y \not\in \freenames{P} }
\end{mathpar}

\begin{definition}
Then two processes, $P,Q$, are alpha-equivalent if $P = Q\{\vec{y}/\vec{x}\}$ for
some $\vec{x} \in \boundnames{Q},\vec{y} \in \boundnames{P}$, where $Q\{\vec{y}/\vec{x}\}$
denotes the capture-avoiding substitution of $\vec{y}$ for $\vec{x}$ in $Q$.
\end{definition}

\begin{definition}
  The {\em structural congruence} \cite{SangiorgiWalker} , $\equiv$,
  between processes is the least congruence containing
  alpha-equivalence, satisfying the abelian monoid laws
  (associativity, commutativity and $\pzero$ as identity) for parallel
  composition $|$ and for summation $+$.
\end{definition}

\subsection{Name equivalence}

We take name equivalence, written $\nameeq$, to be the smallest
equivalence relation generated by the following rules.

\begin{mathpar}
\inferrule*[lab=Quote-drop]
{ }
{ \quotep{@{x}} \nameeq x }

\inferrule*[lab=Struct-equiv]
{ P \scong Q }
{ \quotep{P} \nameeq \quotep{Q} }
\end{mathpar}

The astute reader will have noticed that the mutual recursion of names
and processes imposes a mutual recursion on alpha-equivalence and
structural equivalence via name-equivalence. Fortunately, all of this
works out pleasantly and we may calculate in the natural way, free of
concern. The reader interested in the details is referred to the
appendix \ref{appendix:rho_details}.

\subsection{Substitution}

We use $\Proc$ for the set of processes, $\QProc$ for the set of
names, and $\id{\{}\vec{y} / \vec{x} \id{\}}$ to denote partial maps,
$s : \QProc \rightarrow \QProc$. A map, $s$ lifts, uniquely, to a map
on process terms, $\widehat{s} : \Proc \rightarrow \Proc$ by the
following equations.

\begin{mathpar}
  (0) \psubstp{Q}{P} := 0 \\
  (R \juxtap S) \psubstp{Q}{P}
  :=    
  (R)\psubstp{Q}{P} \juxtap (S) \psubstp{Q}{P} \\
  (x?(y).R) \psubstp{Q}{P}    
  :=    
  (x)\substp{Q}{P} (z)\concat( (R \psubstn{z}{y}) \psubstp{Q}{P} ) \\
  (\lift{x}{R}) \psubstp{Q}{P}  
  :=
  \lift{(x)\substp{Q}{P}}{ R \psubstp{Q}{P} } \\
%   (\dropn{x})  \psubstp{Q}{P}       
%   := 
%   \left\{ 
%     \begin{array}{ccc} 
%       \dropn{\quotep{Q}} & & x \nameeq \quotep{P} \\
%       \dropn{x} & & otherwise \\
%     \end{array}
%   \right. 
  (\dropn{x})  \psubstp{Q}{P}       
  := 
  \left\{ 
    \begin{array}{ccc} 
      Q & & x \nameeq \quotep{P} \\
      \dropn{x} & & otherwise \\
    \end{array}
  \right.
\end{mathpar}
 

where

\begin{eqnarray}
  (x)\id{\{} \lpquote Q \rpquote / \lpquote P \rpquote \id{\}}            = 
  \left\{ 
    \begin{array}{ccc}
      \lpquote Q \rpquote & & x \nameeq \lpquote P \rpquote \\
      x & & otherwise \\
    \end{array}
  \right. \nonumber
\end{eqnarray}

and $z$ is chosen distinct from $\quotep{P}$, $\quotep{Q}$, the free
names in $Q$, and all the names in $R$. Our $\alpha$-equivalence will
be built in the standard way from this substitution.

\begin{remark}\label{rem:no_self_referential_names}
  One consequence of these definitions is that $\forall P. \quotep{P}
  \not\in \freenames{P}$.
\end{remark}

\subsection{ Dynamic quote: an example }

Anticipating something of what's to come, consider applying the
substitution, $\widehat{\id{\{}u / z \id{\}}}$, to the following pair
of processes, $\lift{w}{y!(z)}$ and $w[ \lpquote y!(z) \rpquote ]$.

\begin{eqnarray}
	\lift{w}{y!(z)}\widehat{\id{\{}u / z \id{\}}}
		& = &
		\lift{w}{y!(u)} \nonumber\\
	w[ \lpquote y!(z) \rpquote ] \widehat{ \id{\{}u / z \id{\}} }
		& = &
		w[ \lpquote y!(z) \rpquote ] \nonumber
\end{eqnarray}

Because the body of the process between quotes is impervious to
substitution, we get radically different answers. In fact, by
examining the first process in an input context,
e.g. $x?(z).\lift{w}{y!(z)}$, we see that the process under the lift
operator may be shaped by prefixed inputs binding a name inside it. In
this sense, the lift operator will be seen as a way to dynamically
construct processes before reifying them as names.

Finally equipped with these standard features we can present the
dynamics of the calculus.

\subsubsection{Operational semantics} 

Finally, we introduce the computational dynamics. What marks these
algebras as distinct from other more traditionally studied algebraic
structures, e.g. vector spaces or polynomial rings, is the manner in
which dynamics is captured. In traditional structures, dynamics is typically
expressed through morphisms between such structures, as in linear maps
between vector spaces or morphisms between rings. In algebras
associated with the semantics of computation, the dynamics is
expressed as part of the algebraic structure itself, through a
reduction reduction relation typically denoted by $\red$. Below, we
give a recursive presentation of this relation for the calculus used
in the encoding.

$\red \subseteq \pi \times \pi$
$\red : \pi \to \mathcal{P}(\pi)$

\begin{mathpar}
  \inferrule* [lab=Comm] { \textsf{match}( x_{src}, x_{trgt} ) } { x_{trgt}?(y)P \; | \; x_{src}!\langle {Q} \rangle \red P\{\quotep{Q}/y}\} }
  \and \\
  \inferrule* [lab=Par] {{P} \red {P}'} {{{P} | {Q}} \red {{P}' | {Q}}}
  \and
  \inferrule* [lab=Equiv]{{{P} \scong {P}'} \andalso {{P}' \red {Q}'} \andalso {{Q}' \scong {Q}}}{{P} \red {Q}}
\end{mathpar}

\begin{eqnarray*}
  match_{\equiv} (\quotep{P},\quotep{Q}) & := & P \equiv Q \\
  match_{\dagger}(\quotep{P},\quotep{Q}) & := & \forall R. P|Q \red^{*} R => R \red^{*} 0 \\
  match_{K}(\quotep{P},\quotep{Q}) & := & K \mbox{ for some context } K
\end{eqnarray*}

$u?(x)P | u!\langle Q \rangle \red P\{\quotep{Q}/x\}$

%We write $\wred$ for $\red^*$, and $P\red$ if $\exists Q $ such that $ P \red Q$.
We write $P\red$ if $\exists Q $ such that $ P \red Q$ and $P\not\red$, otherwise.

\section{Replication}

As mentioned before, it is known that replication (and hence
recursion) can be implemented in a higher-order process algebra
\cite{SangiorgiWalker}. As our first example of calculation with the
machinery thus far presented we give the construction explicitly in
the {\rhoc}.

\begin{eqnarray}
	D_{x} & := & \prefix{x}{y}{(\binpar{\outputp{x}{y}}{@{y}})} \nonumber\\
	\bangp_{x}{P} & := & \binpar{{x}!\langle{\binpar{D_{x}}{P}}\rangle}{D_{x}} \nonumber
\end{eqnarray}

\begin{eqnarray}
	\bangp_{x}{P} & & \nonumber\\
	=
	& {x}!\langle{(\prefix{x}{y}{(\outputp{x}{y} | @{y})) | P}}\rangle 
	      | \prefix{x}{y}{(\outputp{x}{y} | @{y})} & \nonumber\\
	\red
	& (\outputp{x}{y} | @{y})\substn{\quotep{(\prefix{x}{y}{(@{y} | \outputp{x}{y})) | P}}}{y} & \nonumber\\
	=
	& \outputp{x}{\quotep{(\prefix{x}{y}{(\outputp{x}{y} | @{y})) | P}}}
	  | {(\prefix{x}{y}{(\outputp{x}{y} | @{y})) | P}} & \nonumber\\
	\red
	& \ldots & \nonumber\\
	\red^*
	& P | P | \ldots & \nonumber
\end{eqnarray}

Of course, this encoding, as an implementation, runs away, unfolding
$\bangp{P}$ eagerly. A lazier and more implementable replication
operator, restricted to input-guarded processes, may be obtained as follows.

\begin{eqnarray}
\bangp{\prefix{u}{v}{P}} 
	:= 
	\binpar{\lift{x}{\prefix{u}{v}{(\binpar{D(x)}{P})}}}{D(x)} \nonumber
\end{eqnarray}

\begin{remark}
  Note that the lazier definition still does not deal with summation
  or mixed summation (i.e. sums over input and output). The reader is
  invited to construct definitions of replication that deal with these
  features. 

  Further, the definitions are parameterized in a name, $x$. Can you,
  gentle reader, make a definition that eliminates this parameter and
  guarantees no accidental interaction between the replication
  machinery and the process being replicated -- i.e. no accidental
  sharing of names used by the process to get its work done and the
  name(s) used by the replication to effect copying. This latter
  revision of the definition of replication is crucial to obtaining
  the expected identity $!!P \sim !P$.
\end{remark}

\begin{remark}\label{rem:paradoxical_combinator}
  The reader familiar with the lambda calculus will have noticed the
  similarity between $D$ and the paradoxical combinator.

  [Ed. note: the existence of this seems to suggest we have to be more
  restrictive on the set of processes and names we admit if we are to
  support no-cloning.]
\end{remark}

\subsubsection{Bisimulation}

The computational dynamics gives rise to another kind of equivalence,
the equivalence of computational behavior. As previously mentioned
this is typically captured \emph{via} some form of bisimulation.

% The notion we use in this paper is weak barbed bisimulation
% \cite{milner91polyadicpi}.

The notion we use in this paper is derived from weak barbed
bisimulation \cite{milner91polyadicpi}. 

\begin{definition}
An \emph{observation relation}, $\downarrow_{\mathcal N}$, over a set
of names, $\mathcal N$, is the smallest relation satisfying the rules
below.

\infrule[Out-barb]{y \in {\mathcal N}, \; x \nameeq y}
		  {\outputp{x}{v} \downarrow_{\mathcal N} x}
\infrule[Par-barb]{\mbox{$P\downarrow_{\mathcal N} x$ or $Q\downarrow_{\mathcal N} x$}}
		  {\binpar{P}{Q} \downarrow_{\mathcal N} x}

We write $P \Downarrow_{\mathcal N} x$ if there is $Q$ such that 
$P \wred Q$ and $Q \downarrow_{\mathcal N} x$.
\end{definition}

\begin{definition}
%\label{def.bbisim}
An  ${\mathcal N}$-\emph{barbed bisimulation} over a set of names, ${\mathcal N}$, is a symmetric binary relation 
${\mathcal S}_{\mathcal N}$ between agents such that $P\rel{S}_{\mathcal N}Q$ implies:
\begin{enumerate}
\item If $P \red P'$ then $Q \wred Q'$ and $P'\rel{S}_{\mathcal N} Q'$.
\item If $P\downarrow_{\mathcal N} x$, then $Q\Downarrow_{\mathcal N} x$.
\end{enumerate}
$P$ is ${\mathcal N}$-barbed bisimilar to $Q$, written
$P \wbbisim_{\mathcal N} Q$, if $P \rel{S}_{\mathcal N} Q$ for some ${\mathcal N}$-barbed bisimulation ${\mathcal S}_{\mathcal N}$.
\end{definition}

$\mathcal{R} \subseteq \pi \times \pi$

$P \mathcal{R} Q => \forall P'. P \red P' \Rightarrow \exists Q'. Q \red Q', P' \mathcal{R} Q'$

$P \vdash x \Rightarrow Q \vdash x$

\begin{mathpar}
  \inferrule*[lab=Out-barb]{x \nameeq y}{{y}!\langle{Q}\rangle \vdash x}
  \and
  \inferrule*[lab=Par-barb]{\mbox{$P\vdash x$ or $Q\vdash x$}}{\binpar{P}{Q} \vdash x}
\end{mathpar}

\subsubsection{Contexts}

One of the principle advantages of computational calculi like the
$\pi$-calculus is a well-defined notion of context,
contextual-equivalence and a correlation between
contextual-equivalence and notions of bisimulation. The notion of
context allows the decomposition of a process into (sub-)process and
its syntactic environment, its context. Thus, a context may be
thought of as a process with a ``hole'' (written $\Box$) in it. The
application of a context $M$ to a process $P$, written $M[P]$, is
tantamount to filling the hole in $M$ with $P$. In this paper we do
not need the full weight of this theory, but do make use of the notion
of context in the proof the main theorem. 

\begin{mathpar}
  \inferrule* [lab=summation] {} {{M_{M},M_{N}} \bc \Box \;|\; x.M_{A} \;|\; M_{M}+M_{N}}
  \and
  \inferrule* [lab=agent] {} {{M_{A}} \bc (\vec{x})M_{P} \;| \; \clift{P_0,\ldots,M_{P},\ldots,P_N}}
  \and \\
  \inferrule* [lab=process] {} {{M_{P}} \bc M_{N} \;| \;P|M_{P} }
\end{mathpar} 

\begin{mathpar}
  \inferrule* [lab=sychronization] {} {M_{N} \bc \Box \;|\; x?M_{F} \;|\; x!M_{C}}
  \and
  \inferrule* [lab=abstraction] {} {{M_{F}} \bc (x)M_{P} }
  \and
  \inferrule* [lab=concretion] {} {{M_{C}} \bc \langle M_{P} \rangle }
  \and \\
  \inferrule* [lab=process] {} {{M_{P}} \bc M_{N} \;| \;P|M_{P} }
\end{mathpar}

\begin{definition}[contextual application] Given a context $M$, and
  process $P$, we define the \emph{contextual application}, $M[P] :=
  M\{P/\Box\}$. That is, the contextual application of M to P is the
  substitution of $P$ for $\Box$ in $M$.
\end{definition}

$\meaningof{-} : L \to \mathcal{P}(\pi)$

\begin{mathpar}
  \inferrule* [lab=collection] {} {\meaningof{true} = \pi, \and \meaningof{~E} = \pi \setminus \meaningof{E}, \and \meaningof{E_{1} \& E_{2}} = \meaningof{E_{1}} \cap \meaningof{E_{2}}}
\end{mathpar}

\begin{mathpar}
  \inferrule* [lab=structure] {} {\meaningof{0} = \{ P \in \pi | P \equiv 0 \}, \and \\ \meaningof{E_1 | E_2} = \{ P \in \pi | P \equiv P_{1} | P_{2}, P_{1} \in \meaningof{E_{1}}, P_{2} \in \meaningof{E_2}\} }
\end{mathpar}

\begin{mathpar}
 \inferrule* [lab=behavior] {} {\meaningof{\langle a?b \rangle E} = \{ P \in \pi | P \equiv Q | u?(y)P', \\ \and \\\\ \and \\ \;\;\; u \in \meaningof{a}, \forall z.P'\{z/y\} \in \meaningof{E\{z/b\}}\}, \and \\ \meaningof{a!E} = \{ P \in \pi | P \equiv Q | x!\langle P' \rangle, x \in \meaningof{a} P' \in \meaningof{E}\} }
\end{mathpar}

\begin{mathpar}
 \inferrule* [lab=nominal] {} {\meaningof{\quotep{E}} = \{ \quotep{P} \in \quotep{\pi} | P \in \meaningof{E} \}, \and \meaningof{\quotep{P}} = \{ \quotep{Q} \in \quotep{\pi} | P \equiv Q \} \and \\ \meaningof{@\quotep{E}} = \{ P \in \pi | P \equiv @x, x \in \meaningof{E} \}}
\end{mathpar}

\begin{eqnarray*}
  \\
  \meaningof{-} : TS \to ST
\end{eqnarray*}

\begin{eqnarray*}
  \\
  L : TS \to ST
\end{eqnarray*}

\begin{eqnarray*}
  \\
  P \models E \iff P \in \meaningof{E}
\end{eqnarray*}

\begin{eqnarray*}
  P \approx_{L} Q \iff \forall E \in L. P \models E \iff Q \models E
\end{eqnarray*}

\begin{eqnarray*}
  P \approx_{K} Q
\end{eqnarray*}

\begin{eqnarray*}
  P \approx Q
\end{eqnarray*}

$\approx_{K} = \approx = \approx_{L}$

\subsubsection{Contextual duality}

Note that contexts extend the quotation operation to a family of
operations from processes to names. Given a context, $M$, we can
define a \emph{nominal context}, $\quotep{M}$ by $\quotep{M}[P] :=
\quotep{M[P]}$. To foreshadow what is to come we observe that these
operations enjoy a duality with processes very much like the duality
between vectors and maps from vectors to scalars.

Further, because the calculus is essentially higher-order, we have a
correspondence between contexts and processes. More specifically,
given a name $x$ and a context $M$ we can construct $M^{*}_{x}$ such
that 

\begin{mathpar}
  M^{*}_{x} | \lift{x}{P} \red M[P]
\end{mathpar}

namely,

\begin{mathpar}
  M^{*}_{x} := x?(u).M[\dropn{u}]
\end{mathpar}

The dependence of $M^{*}_{x}$ on a name makes it an abstraction, 

\begin{mathpar}
  M^{*} := (x)x?(u).M[\dropn{u}]
\end{mathpar}

\subsection{Additional notation}

It will sometimes be convenient to denote the process a name
quotes. We already have the notation $x = \quotep{P}$, but it will be
convenient to introduce an alternate notation, $\procn{x}$, when we
want to emphasize the connection to the use of the name. Note that, by
virtue of name equivalence, $\quotep{\procn{x}} \nameeq x$; so, the
notation is consistent with previous definitions.

Further, because names have structure it is possible to effect
substitutions on the basis of that structure. This means we need to
upgrade our notation for substitutions, which we accomplish by
adapting comprehension notation. Thus,

\begin{mathpar}
  P\{ y / x : x \in S \}
\end{mathpar}

is interpreted to mean the process derived from P by replacing (in a
capture-avoiding manner) each occurrence of $x$ in $S$ by $y$. For example,

\begin{mathpar}
  P\{ \quotep{\procn{x}|\procn{x}} / x : x \in \freenames{P} \}
\end{mathpar}

will replace each (occurrence) of a free name $x$ in $P$ by
$\quotep{\procn{x}|\procn{x}}$.

Also, we will avail ourselves of the notation $x^{L}$ and $x^{R}$ to
denote injections of a name into disjoint copies of the name
space. There are numerous ways to accomplish this. One example can be
found in \cite{MeredithR05}. This notation overloads to vectors of
names: $\vec{x}^{\pi} := (x_{i}^{\pi} \; : \; 0 \leq i < |\vec{x}| )$ where $\pi \in \{L,R\}$.

We also use $P^{\Box} := P|\Box$.

In \cite{MeredithR05} an interpretation of the new operator is
given. It turns out that there are several possible interpretations
all enjoying the requisite algebraic properties of the operator (see
\cite{milner91polyadicpi}). We will therefore make liberal use of
$(\nu\; \vec{x})P$.

% subsection the_syntax_and_semantics_of_the_notation_system (end)   

\input{qm2pi.qmops} 

\input{qm2pi.sterngerlach} 

\input{qm2pi.metric} 

% section concurrent_process_calculi (end)

%\input{qm2pi.proofsketch}

% section proof sketch (end)

%\input{qm2pi.slviaknots} 

% section spatial logic via knots (end)

\input{qm2pi.conclusion}

% section conclusion (end)

%\input{qm2pi.dtcodes} 

% section wiring algorithm (end)

\input{qm2pi.ack} 

% section acknowledgments (end)

\newpage


\bibliographystyle{plain}   
\bibliography{../../biblios/main.bib}

\input{qm2pi.rhodetails}

\end{document}



\end{document}



% section front matter (end)

\section{Introduction}\label{sec:introduction} % (fold)
In this draft of the material i am going to have to dispense with the
usual writing conventions adopted in papers on these topics. i'm going
to have adopt whatever tone i need at the time i'm writing up the
calculations. Sometimes this may be very conversational; others it may
be the barest mathematical grunts; others still it may be that i have
lifted text from one of my other papers because the exposition of some
point was better said there. i hope that my readers are not unduly put
out by this decision. i'm not doing this to flout convention or be
rebellious. i find these calculations very technically challenging. To
keep everything going technically, something has to give; i have to
let go of some cognitive burden. So, the academic writing style --
with all of its trade-offs in terms of facilitating technical
communication -- is what i'm letting go of. Perhaps subsequent drafts
can be tightened and polished, but for now, i'm going to speak as if
we were sitting together in a coffee shop with a laptop, wifi and a
pad of paper and a pencil.

So, here's what i have to say. We -- you and i, comfortably ensconced
in our coffee shop and well-equipped with our tools -- can realize and
carry out the calculations of quantum mechanics over a very different
formal theory of dynamics, a formal theory of dynamics that
corresponds to a theory of concurrent computation with
\emph{reflection}. It has the advantage that the underlying theory is
already `quantized', but supports analogues all of the continuuous
operations. Strikingly, this underlying theory has recently been
connected with a notion of metric that we can show, by calculating
together, coincides with the metric induced by the inner product.

There are a lot of reasons why you might be interested in seeing
calculations of this form. Here's why i'm interested. For the past
several centuries there has been no competitor to the ``Newtonian''
account of dynamics. As a result the predominant share of accounts of
dynamical systems and situations have had to be formulated in terms of
the Newtonian machinery. i view this as an intellectually dangerous
position to occupy. Everything, despite it's intrinsic shape, turns
into a nail to be hit with this hammer. Recently, however, the theory
of computation has matured to the point where we have candidates for
theories of dynamics that offer very different perspective on
reasoning about dynamical systems and situations. Testing these
candidates against very successful accounts of dynamical situations,
like quantum mechanics, is going to give us some sense of how mature
they are and some measure of the quality of these accounts of
dynamics.

\subsection{Summary of contributions and outline of paper}

So, we're going to develop an interpretation of the operations of
quantum mechanics normally interpreted by Hilbert spaces and
operators. We're going to do this over a theory of computation. Note
that this is very different than the usual quantum computation program
which develops notions of computation over quantum mechanics. Rather,
we are developing a story that aligns with Wheeler's slogan: It from
Bit. To do this we will first provide an account of the theory of
computation at play here. Then we will dive into a calculation-driven
interpretation of the operations of quantum mechanics.

The reason we take this approach is that -- until very recently --
there hasn't been an axiomatic account of quantum mechanics. As a
result there has been no sharp delineation of the mathematical theory
supporting interpretation of the physical theory and the physical
theory, itself. So, ambient features of the maths are free to be
exploited (or supressed) without a real accounting of their physical
relevance. There is no sharp statement ``here's the physical theory''
qua \emph{theory} and ``here's the mathematical interpretation''
enabling a judgment of how faithful the interpretation is -- apart
from experimental observation. When there is an axiomatic account we
can judge how well a given mathematical formalism supports an
interpretation of the axioms, independent of
experimentation. Likewise, we can judge how well we have captured our
physical evidence and experience with our axiomatics, independent of
any specific mathematical implementation, with accidental detail that
may or may not have physical significance. 

In lieu of a fully fleshed out and vetted axiomatic account of quantum
mechanics, interpreting the operational notions in service of modeling
physical systems will have to suffice. In other words, we are not in
the business of providing a model of Hilbert spaces and operators. We
are in the business of providing a model of quantum mechanics because
we are motivated by testing our notions of dynamics against physical
theory; and, the predictive calculations of the physical theory must
serve as the best formulation -- shy of a fully fleshed out axiomatic
account -- of the physical theory itself (as they have for scientific
theories since time immemorial). Put another way, despite a
whole-hearted commitment to an It-from-Bit ontology, we are firmly
aligned with the shut-up-and-calculate camp as the best way to obtain
results either from the physical perspective or as a quality assurance
measure of our fledgling theory of dynamics.

In detail, we present a reflective process calculus. Then we develop
intuitive correspondences between the notions available in this
calculus and the usual physical notions supporting quantum mechanical
calculations. Thus, 

\begin{table}[htp]
  \center{
    \fbox{
      \begin{tabular}{c|c}
        quantum mechanics & process calculus \\
        \hline
        scalar & name \\
        state vector & process \\
        dual & contextual duals \\
        matrix & formal sums of process-context-dual pairs \\
        orthogonality & process annihilation \\
        inner product & execution-formula + quoting
      \end{tabular}
    }
  }
  \caption{QM - process calculi correspondences}
\end{table}

Then we tighten up these intuitions to operational definitions. We
employ the Dirac notation as the best proxy we can find for an
abstract syntax of the quantum mechanical notions. The definitions we
develop put us in contact with equational constraints coming from the
theory that we demonstrate the definitions and calculations satisfy.

This puts us in a position to shut up and calculate for the
Stern-Gerlach experimental set up, showing how these predictive
calculations become calculations on processes in our theory of a
reflective process calculus.

Penultimately, we demonstrate that the notion of metric coming from
the inner product coincides with the notion of metric available from
the theory of bisimulation. This demonstration gives us the right to
think of space as arising from behavior. Finally, we consider where we
might go from the new vantage point we have obtained.

% section introduction (end) 
 
% section introduction (end)

% \documentclass[12pt]{llncs}
%\documentclass{jktr}

\usepackage[pdftex]{hyperref}                   
\usepackage {listings}
\usepackage {mathpartir}
\usepackage{bcprules}
%\usepackage{listings}
                       
\usepackage{graphicx} 
%\usepackage[margins=2.5cm,nohead,nofoot]{geometry}
%\usepackage{geometry}
\usepackage{amsfonts}
\usepackage{amstext}
\usepackage{latexsym}
\usepackage{amssymb}
\usepackage{color}


%\include{myPreamble}
\documentclass[12pt]{llncs}
%\documentclass{jktr}

\usepackage[pdftex]{hyperref}                   
\usepackage {listings}
\usepackage {mathpartir}
\usepackage{bcprules}
%\usepackage{listings}
                       
\usepackage{graphicx} 
%\usepackage[margins=2.5cm,nohead,nofoot]{geometry}
%\usepackage{geometry}
\usepackage{amsfonts}
\usepackage{amstext}
\usepackage{latexsym}
\usepackage{amssymb}
\usepackage{color}


%\include{myPreamble}
\include{qm2pi.local} 

%\ifpdf
%\usepackage[pdftex]{graphicx}
%\else
%\usepackage{graphicx}
%\fi

 % \ifpdf
%  \usepackage{pdfsync}
%  \if


%\title{Brief Article}
%\author{David F. Snyder}
%\author{L.G. Meredith}

%\address{Dept. of Math., Texas State University--San Marcos, San Marcos, TX 78666}
       
\pagestyle{empty}


\begin{document}

\lstset{language=[Objective]Caml,frame=shadowbox}

\input{qm2pi.front}

% section front matter (end)

\input{qm2pi.intro} 
 
% section introduction (end)

% \input{qm2pi.knotations} 

% section notation (end)

\input{qm2pi.process.calculi} 

% section concurrent_process_calculi_and_spatial_logics_ (end)
    
%\input{qm2pi.knots2pi} 

%\input{qm2pi.trefoil} 

%\input{qm2pi.mainthm} 

% subsection basic_interpretation (end)

%\input{qm2pi.rho.presentation} 
\subsection{The syntax and semantics of the notation system}\label{sub:the_syntax_and_semantics_of_the_notation_system} % (fold)

We now summarize a technical presentation of the calculus that
embodies our theory of dynamics. The typical presentation of such a
calculus follows the style of giving generators and relations on
them. The grammar, below, describing term constructors, freely
generates the set of processes, $\Proc$. This set is then quotiented
by a relation known as structural congruence and it is over this set
that the notion of dynamics is expressed. This presentation is
essentially that of \cite{MeredithR05} with the addition of
polyadicity and summation. For readability we have relegated some of
the technical subtleties to an appendix.

\subsubsection{Process grammar}\label{subsub:process_grammar}

\begin{mathpar}
  \inferrule* [lab=synchronization] {} {{M} \bc \pzero \;|\; x?F \;|\; x!C }
  \and
  \inferrule* [lab=abstraction] {} {{F} \bc (x)P}
  \and
  \inferrule* [lab=concretion] {} {{C} \bc \langle Q \rangle}
  \and
  \inferrule* [lab=process] {} {{P,Q} \bc M \;| \;P|Q \;|\; @{x}}
  \and
  \inferrule* [lab=name] {} {{x} \bc \quotep{P}}
\end{mathpar} 

Note that $\vec{x}$ (resp. $\vec{P}$) denotes a vector of names
(resp. processes) of length $|\vec{x}|$ (resp. $|\vec{P}|$). We adopt
the following useful abbreviations.

\begin{mathpar}
   x?(\vec{y}).P := x.(\vec{y})P \and  x\clift{\vec{P}} := x.\clift{\vec{P}}
   \and x!(y) := \lift{x}{\dropn{y}}
   \and \Pi_{i=0}^{n-1}P_i := P_0 | \ldots | P_{n-1}
\end{mathpar}

\subsubsection{Structural congruence}

\paragraph{Free and bound names and alpha-equivalence.} At the
core of structural equivalence is alpha-equivalence which identifies
process that are the same up to a change of variable. Formally, we
recognize the distinction between free and bound names. The free names
of a process, $\freenames{P}$, may be calculated recursively as
follows:

\begin{mathpar}
\freenames{\pzero} := \emptyset
  \and \\
  \freenames{x?(y).P} := \{ x \} \cup (\freenames{P} \setminus \{ y \})
  \and 
  \freenames{x!\langle P \rangle} := \{ x \} \cup \{ P \} 
  \and \\
  \freenames{P|Q} := \freenames{P} \cup \freenames{Q}
  \and \\
  \freenames{@{x}} := \{ x \}
\end{mathpar}

$\pi$
$\quotep{\pi}$

$\freenames{-} : \pi \to \mathcal{P}(\quotep{\pi})$

\begin{eqnarray*}
  \freenames{\pzero} & := & \emptyset \\
  \freenames{x?(y).P} & := & \{ x \} \cup (\freenames{P} \setminus \{ y \}) \\
  \freenames{x!\langle P \rangle} & := & \{ x \} \cup \{ P \} \\
  \freenames{P|Q} & := & \freenames{P} \cup \freenames{Q} \\
  \freenames{\dropn{x}} & := & \{ x \}
\end{eqnarray*}

The bound names of a process, $\boundnames{P}$, are those names occurring in $P$
that are not free. For example, in $x?(y).0$, the name $x$ is free, while $y$ is bound.

\begin{mathpar}
  \inferrule* [lab=monoidal-laws] {} { P|Q \equiv Q|P \and P|0 \equiv P \and P|(Q|R) \equiv (P|Q)|R }
\end{mathpar}

\begin{mathpar}
  \inferrule* [lab=alpha-equivalence] {} { (x)P \equiv (y)P\{y/x\} \and y \not\in \freenames{P} }
\end{mathpar}

\begin{definition}
Then two processes, $P,Q$, are alpha-equivalent if $P = Q\{\vec{y}/\vec{x}\}$ for
some $\vec{x} \in \boundnames{Q},\vec{y} \in \boundnames{P}$, where $Q\{\vec{y}/\vec{x}\}$
denotes the capture-avoiding substitution of $\vec{y}$ for $\vec{x}$ in $Q$.
\end{definition}

\begin{definition}
  The {\em structural congruence} \cite{SangiorgiWalker} , $\equiv$,
  between processes is the least congruence containing
  alpha-equivalence, satisfying the abelian monoid laws
  (associativity, commutativity and $\pzero$ as identity) for parallel
  composition $|$ and for summation $+$.
\end{definition}

\subsection{Name equivalence}

We take name equivalence, written $\nameeq$, to be the smallest
equivalence relation generated by the following rules.

\begin{mathpar}
\inferrule*[lab=Quote-drop]
{ }
{ \quotep{@{x}} \nameeq x }

\inferrule*[lab=Struct-equiv]
{ P \scong Q }
{ \quotep{P} \nameeq \quotep{Q} }
\end{mathpar}

The astute reader will have noticed that the mutual recursion of names
and processes imposes a mutual recursion on alpha-equivalence and
structural equivalence via name-equivalence. Fortunately, all of this
works out pleasantly and we may calculate in the natural way, free of
concern. The reader interested in the details is referred to the
appendix \ref{appendix:rho_details}.

\subsection{Substitution}

We use $\Proc$ for the set of processes, $\QProc$ for the set of
names, and $\id{\{}\vec{y} / \vec{x} \id{\}}$ to denote partial maps,
$s : \QProc \rightarrow \QProc$. A map, $s$ lifts, uniquely, to a map
on process terms, $\widehat{s} : \Proc \rightarrow \Proc$ by the
following equations.

\begin{mathpar}
  (0) \psubstp{Q}{P} := 0 \\
  (R \juxtap S) \psubstp{Q}{P}
  :=    
  (R)\psubstp{Q}{P} \juxtap (S) \psubstp{Q}{P} \\
  (x?(y).R) \psubstp{Q}{P}    
  :=    
  (x)\substp{Q}{P} (z)\concat( (R \psubstn{z}{y}) \psubstp{Q}{P} ) \\
  (\lift{x}{R}) \psubstp{Q}{P}  
  :=
  \lift{(x)\substp{Q}{P}}{ R \psubstp{Q}{P} } \\
%   (\dropn{x})  \psubstp{Q}{P}       
%   := 
%   \left\{ 
%     \begin{array}{ccc} 
%       \dropn{\quotep{Q}} & & x \nameeq \quotep{P} \\
%       \dropn{x} & & otherwise \\
%     \end{array}
%   \right. 
  (\dropn{x})  \psubstp{Q}{P}       
  := 
  \left\{ 
    \begin{array}{ccc} 
      Q & & x \nameeq \quotep{P} \\
      \dropn{x} & & otherwise \\
    \end{array}
  \right.
\end{mathpar}
 

where

\begin{eqnarray}
  (x)\id{\{} \lpquote Q \rpquote / \lpquote P \rpquote \id{\}}            = 
  \left\{ 
    \begin{array}{ccc}
      \lpquote Q \rpquote & & x \nameeq \lpquote P \rpquote \\
      x & & otherwise \\
    \end{array}
  \right. \nonumber
\end{eqnarray}

and $z$ is chosen distinct from $\quotep{P}$, $\quotep{Q}$, the free
names in $Q$, and all the names in $R$. Our $\alpha$-equivalence will
be built in the standard way from this substitution.

\begin{remark}\label{rem:no_self_referential_names}
  One consequence of these definitions is that $\forall P. \quotep{P}
  \not\in \freenames{P}$.
\end{remark}

\subsection{ Dynamic quote: an example }

Anticipating something of what's to come, consider applying the
substitution, $\widehat{\id{\{}u / z \id{\}}}$, to the following pair
of processes, $\lift{w}{y!(z)}$ and $w[ \lpquote y!(z) \rpquote ]$.

\begin{eqnarray}
	\lift{w}{y!(z)}\widehat{\id{\{}u / z \id{\}}}
		& = &
		\lift{w}{y!(u)} \nonumber\\
	w[ \lpquote y!(z) \rpquote ] \widehat{ \id{\{}u / z \id{\}} }
		& = &
		w[ \lpquote y!(z) \rpquote ] \nonumber
\end{eqnarray}

Because the body of the process between quotes is impervious to
substitution, we get radically different answers. In fact, by
examining the first process in an input context,
e.g. $x?(z).\lift{w}{y!(z)}$, we see that the process under the lift
operator may be shaped by prefixed inputs binding a name inside it. In
this sense, the lift operator will be seen as a way to dynamically
construct processes before reifying them as names.

Finally equipped with these standard features we can present the
dynamics of the calculus.

\subsubsection{Operational semantics} 

Finally, we introduce the computational dynamics. What marks these
algebras as distinct from other more traditionally studied algebraic
structures, e.g. vector spaces or polynomial rings, is the manner in
which dynamics is captured. In traditional structures, dynamics is typically
expressed through morphisms between such structures, as in linear maps
between vector spaces or morphisms between rings. In algebras
associated with the semantics of computation, the dynamics is
expressed as part of the algebraic structure itself, through a
reduction reduction relation typically denoted by $\red$. Below, we
give a recursive presentation of this relation for the calculus used
in the encoding.

$\red \subseteq \pi \times \pi$
$\red : \pi \to \mathcal{P}(\pi)$

\begin{mathpar}
  \inferrule* [lab=Comm] { \textsf{match}( x_{src}, x_{trgt} ) } { x_{trgt}?(y)P \; | \; x_{src}!\langle {Q} \rangle \red P\{\quotep{Q}/y}\} }
  \and \\
  \inferrule* [lab=Par] {{P} \red {P}'} {{{P} | {Q}} \red {{P}' | {Q}}}
  \and
  \inferrule* [lab=Equiv]{{{P} \scong {P}'} \andalso {{P}' \red {Q}'} \andalso {{Q}' \scong {Q}}}{{P} \red {Q}}
\end{mathpar}

\begin{eqnarray*}
  match_{\equiv} (\quotep{P},\quotep{Q}) & := & P \equiv Q \\
  match_{\dagger}(\quotep{P},\quotep{Q}) & := & \forall R. P|Q \red^{*} R => R \red^{*} 0 \\
  match_{K}(\quotep{P},\quotep{Q}) & := & K \mbox{ for some context } K
\end{eqnarray*}

$u?(x)P | u!\langle Q \rangle \red P\{\quotep{Q}/x\}$

%We write $\wred$ for $\red^*$, and $P\red$ if $\exists Q $ such that $ P \red Q$.
We write $P\red$ if $\exists Q $ such that $ P \red Q$ and $P\not\red$, otherwise.

\section{Replication}

As mentioned before, it is known that replication (and hence
recursion) can be implemented in a higher-order process algebra
\cite{SangiorgiWalker}. As our first example of calculation with the
machinery thus far presented we give the construction explicitly in
the {\rhoc}.

\begin{eqnarray}
	D_{x} & := & \prefix{x}{y}{(\binpar{\outputp{x}{y}}{@{y}})} \nonumber\\
	\bangp_{x}{P} & := & \binpar{{x}!\langle{\binpar{D_{x}}{P}}\rangle}{D_{x}} \nonumber
\end{eqnarray}

\begin{eqnarray}
	\bangp_{x}{P} & & \nonumber\\
	=
	& {x}!\langle{(\prefix{x}{y}{(\outputp{x}{y} | @{y})) | P}}\rangle 
	      | \prefix{x}{y}{(\outputp{x}{y} | @{y})} & \nonumber\\
	\red
	& (\outputp{x}{y} | @{y})\substn{\quotep{(\prefix{x}{y}{(@{y} | \outputp{x}{y})) | P}}}{y} & \nonumber\\
	=
	& \outputp{x}{\quotep{(\prefix{x}{y}{(\outputp{x}{y} | @{y})) | P}}}
	  | {(\prefix{x}{y}{(\outputp{x}{y} | @{y})) | P}} & \nonumber\\
	\red
	& \ldots & \nonumber\\
	\red^*
	& P | P | \ldots & \nonumber
\end{eqnarray}

Of course, this encoding, as an implementation, runs away, unfolding
$\bangp{P}$ eagerly. A lazier and more implementable replication
operator, restricted to input-guarded processes, may be obtained as follows.

\begin{eqnarray}
\bangp{\prefix{u}{v}{P}} 
	:= 
	\binpar{\lift{x}{\prefix{u}{v}{(\binpar{D(x)}{P})}}}{D(x)} \nonumber
\end{eqnarray}

\begin{remark}
  Note that the lazier definition still does not deal with summation
  or mixed summation (i.e. sums over input and output). The reader is
  invited to construct definitions of replication that deal with these
  features. 

  Further, the definitions are parameterized in a name, $x$. Can you,
  gentle reader, make a definition that eliminates this parameter and
  guarantees no accidental interaction between the replication
  machinery and the process being replicated -- i.e. no accidental
  sharing of names used by the process to get its work done and the
  name(s) used by the replication to effect copying. This latter
  revision of the definition of replication is crucial to obtaining
  the expected identity $!!P \sim !P$.
\end{remark}

\begin{remark}\label{rem:paradoxical_combinator}
  The reader familiar with the lambda calculus will have noticed the
  similarity between $D$ and the paradoxical combinator.

  [Ed. note: the existence of this seems to suggest we have to be more
  restrictive on the set of processes and names we admit if we are to
  support no-cloning.]
\end{remark}

\subsubsection{Bisimulation}

The computational dynamics gives rise to another kind of equivalence,
the equivalence of computational behavior. As previously mentioned
this is typically captured \emph{via} some form of bisimulation.

% The notion we use in this paper is weak barbed bisimulation
% \cite{milner91polyadicpi}.

The notion we use in this paper is derived from weak barbed
bisimulation \cite{milner91polyadicpi}. 

\begin{definition}
An \emph{observation relation}, $\downarrow_{\mathcal N}$, over a set
of names, $\mathcal N$, is the smallest relation satisfying the rules
below.

\infrule[Out-barb]{y \in {\mathcal N}, \; x \nameeq y}
		  {\outputp{x}{v} \downarrow_{\mathcal N} x}
\infrule[Par-barb]{\mbox{$P\downarrow_{\mathcal N} x$ or $Q\downarrow_{\mathcal N} x$}}
		  {\binpar{P}{Q} \downarrow_{\mathcal N} x}

We write $P \Downarrow_{\mathcal N} x$ if there is $Q$ such that 
$P \wred Q$ and $Q \downarrow_{\mathcal N} x$.
\end{definition}

\begin{definition}
%\label{def.bbisim}
An  ${\mathcal N}$-\emph{barbed bisimulation} over a set of names, ${\mathcal N}$, is a symmetric binary relation 
${\mathcal S}_{\mathcal N}$ between agents such that $P\rel{S}_{\mathcal N}Q$ implies:
\begin{enumerate}
\item If $P \red P'$ then $Q \wred Q'$ and $P'\rel{S}_{\mathcal N} Q'$.
\item If $P\downarrow_{\mathcal N} x$, then $Q\Downarrow_{\mathcal N} x$.
\end{enumerate}
$P$ is ${\mathcal N}$-barbed bisimilar to $Q$, written
$P \wbbisim_{\mathcal N} Q$, if $P \rel{S}_{\mathcal N} Q$ for some ${\mathcal N}$-barbed bisimulation ${\mathcal S}_{\mathcal N}$.
\end{definition}

$\mathcal{R} \subseteq \pi \times \pi$

$P \mathcal{R} Q => \forall P'. P \red P' \Rightarrow \exists Q'. Q \red Q', P' \mathcal{R} Q'$

$P \vdash x \Rightarrow Q \vdash x$

\begin{mathpar}
  \inferrule*[lab=Out-barb]{x \nameeq y}{{y}!\langle{Q}\rangle \vdash x}
  \and
  \inferrule*[lab=Par-barb]{\mbox{$P\vdash x$ or $Q\vdash x$}}{\binpar{P}{Q} \vdash x}
\end{mathpar}

\subsubsection{Contexts}

One of the principle advantages of computational calculi like the
$\pi$-calculus is a well-defined notion of context,
contextual-equivalence and a correlation between
contextual-equivalence and notions of bisimulation. The notion of
context allows the decomposition of a process into (sub-)process and
its syntactic environment, its context. Thus, a context may be
thought of as a process with a ``hole'' (written $\Box$) in it. The
application of a context $M$ to a process $P$, written $M[P]$, is
tantamount to filling the hole in $M$ with $P$. In this paper we do
not need the full weight of this theory, but do make use of the notion
of context in the proof the main theorem. 

\begin{mathpar}
  \inferrule* [lab=summation] {} {{M_{M},M_{N}} \bc \Box \;|\; x.M_{A} \;|\; M_{M}+M_{N}}
  \and
  \inferrule* [lab=agent] {} {{M_{A}} \bc (\vec{x})M_{P} \;| \; \clift{P_0,\ldots,M_{P},\ldots,P_N}}
  \and \\
  \inferrule* [lab=process] {} {{M_{P}} \bc M_{N} \;| \;P|M_{P} }
\end{mathpar} 

\begin{mathpar}
  \inferrule* [lab=sychronization] {} {M_{N} \bc \Box \;|\; x?M_{F} \;|\; x!M_{C}}
  \and
  \inferrule* [lab=abstraction] {} {{M_{F}} \bc (x)M_{P} }
  \and
  \inferrule* [lab=concretion] {} {{M_{C}} \bc \langle M_{P} \rangle }
  \and \\
  \inferrule* [lab=process] {} {{M_{P}} \bc M_{N} \;| \;P|M_{P} }
\end{mathpar}

\begin{definition}[contextual application] Given a context $M$, and
  process $P$, we define the \emph{contextual application}, $M[P] :=
  M\{P/\Box\}$. That is, the contextual application of M to P is the
  substitution of $P$ for $\Box$ in $M$.
\end{definition}

$\meaningof{-} : L \to \mathcal{P}(\pi)$

\begin{mathpar}
  \inferrule* [lab=collection] {} {\meaningof{true} = \pi, \and \meaningof{~E} = \pi \setminus \meaningof{E}, \and \meaningof{E_{1} \& E_{2}} = \meaningof{E_{1}} \cap \meaningof{E_{2}}}
\end{mathpar}

\begin{mathpar}
  \inferrule* [lab=structure] {} {\meaningof{0} = \{ P \in \pi | P \equiv 0 \}, \and \\ \meaningof{E_1 | E_2} = \{ P \in \pi | P \equiv P_{1} | P_{2}, P_{1} \in \meaningof{E_{1}}, P_{2} \in \meaningof{E_2}\} }
\end{mathpar}

\begin{mathpar}
 \inferrule* [lab=behavior] {} {\meaningof{\langle a?b \rangle E} = \{ P \in \pi | P \equiv Q | u?(y)P', \\ \and \\\\ \and \\ \;\;\; u \in \meaningof{a}, \forall z.P'\{z/y\} \in \meaningof{E\{z/b\}}\}, \and \\ \meaningof{a!E} = \{ P \in \pi | P \equiv Q | x!\langle P' \rangle, x \in \meaningof{a} P' \in \meaningof{E}\} }
\end{mathpar}

\begin{mathpar}
 \inferrule* [lab=nominal] {} {\meaningof{\quotep{E}} = \{ \quotep{P} \in \quotep{\pi} | P \in \meaningof{E} \}, \and \meaningof{\quotep{P}} = \{ \quotep{Q} \in \quotep{\pi} | P \equiv Q \} \and \\ \meaningof{@\quotep{E}} = \{ P \in \pi | P \equiv @x, x \in \meaningof{E} \}}
\end{mathpar}

\begin{eqnarray*}
  \\
  \meaningof{-} : TS \to ST
\end{eqnarray*}

\begin{eqnarray*}
  \\
  L : TS \to ST
\end{eqnarray*}

\begin{eqnarray*}
  \\
  P \models E \iff P \in \meaningof{E}
\end{eqnarray*}

\begin{eqnarray*}
  P \approx_{L} Q \iff \forall E \in L. P \models E \iff Q \models E
\end{eqnarray*}

\begin{eqnarray*}
  P \approx_{K} Q
\end{eqnarray*}

\begin{eqnarray*}
  P \approx Q
\end{eqnarray*}

$\approx_{K} = \approx = \approx_{L}$

\subsubsection{Contextual duality}

Note that contexts extend the quotation operation to a family of
operations from processes to names. Given a context, $M$, we can
define a \emph{nominal context}, $\quotep{M}$ by $\quotep{M}[P] :=
\quotep{M[P]}$. To foreshadow what is to come we observe that these
operations enjoy a duality with processes very much like the duality
between vectors and maps from vectors to scalars.

Further, because the calculus is essentially higher-order, we have a
correspondence between contexts and processes. More specifically,
given a name $x$ and a context $M$ we can construct $M^{*}_{x}$ such
that 

\begin{mathpar}
  M^{*}_{x} | \lift{x}{P} \red M[P]
\end{mathpar}

namely,

\begin{mathpar}
  M^{*}_{x} := x?(u).M[\dropn{u}]
\end{mathpar}

The dependence of $M^{*}_{x}$ on a name makes it an abstraction, 

\begin{mathpar}
  M^{*} := (x)x?(u).M[\dropn{u}]
\end{mathpar}

\subsection{Additional notation}

It will sometimes be convenient to denote the process a name
quotes. We already have the notation $x = \quotep{P}$, but it will be
convenient to introduce an alternate notation, $\procn{x}$, when we
want to emphasize the connection to the use of the name. Note that, by
virtue of name equivalence, $\quotep{\procn{x}} \nameeq x$; so, the
notation is consistent with previous definitions.

Further, because names have structure it is possible to effect
substitutions on the basis of that structure. This means we need to
upgrade our notation for substitutions, which we accomplish by
adapting comprehension notation. Thus,

\begin{mathpar}
  P\{ y / x : x \in S \}
\end{mathpar}

is interpreted to mean the process derived from P by replacing (in a
capture-avoiding manner) each occurrence of $x$ in $S$ by $y$. For example,

\begin{mathpar}
  P\{ \quotep{\procn{x}|\procn{x}} / x : x \in \freenames{P} \}
\end{mathpar}

will replace each (occurrence) of a free name $x$ in $P$ by
$\quotep{\procn{x}|\procn{x}}$.

Also, we will avail ourselves of the notation $x^{L}$ and $x^{R}$ to
denote injections of a name into disjoint copies of the name
space. There are numerous ways to accomplish this. One example can be
found in \cite{MeredithR05}. This notation overloads to vectors of
names: $\vec{x}^{\pi} := (x_{i}^{\pi} \; : \; 0 \leq i < |\vec{x}| )$ where $\pi \in \{L,R\}$.

We also use $P^{\Box} := P|\Box$.

In \cite{MeredithR05} an interpretation of the new operator is
given. It turns out that there are several possible interpretations
all enjoying the requisite algebraic properties of the operator (see
\cite{milner91polyadicpi}). We will therefore make liberal use of
$(\nu\; \vec{x})P$.

% subsection the_syntax_and_semantics_of_the_notation_system (end)   

\input{qm2pi.qmops} 

\input{qm2pi.sterngerlach} 

\input{qm2pi.metric} 

% section concurrent_process_calculi (end)

%\input{qm2pi.proofsketch}

% section proof sketch (end)

%\input{qm2pi.slviaknots} 

% section spatial logic via knots (end)

\input{qm2pi.conclusion}

% section conclusion (end)

%\input{qm2pi.dtcodes} 

% section wiring algorithm (end)

\input{qm2pi.ack} 

% section acknowledgments (end)

\newpage


\bibliographystyle{plain}   
\bibliography{../../biblios/main.bib}

\input{qm2pi.rhodetails}

\end{document}

 

%\ifpdf
%\usepackage[pdftex]{graphicx}
%\else
%\usepackage{graphicx}
%\fi

 % \ifpdf
%  \usepackage{pdfsync}
%  \if


%\title{Brief Article}
%\author{David F. Snyder}
%\author{L.G. Meredith}

%\address{Dept. of Math., Texas State University--San Marcos, San Marcos, TX 78666}
       
\pagestyle{empty}


\begin{document}

\lstset{language=[Objective]Caml,frame=shadowbox}

\documentclass[12pt]{llncs}
%\documentclass{jktr}

\usepackage[pdftex]{hyperref}                   
\usepackage {listings}
\usepackage {mathpartir}
\usepackage{bcprules}
%\usepackage{listings}
                       
\usepackage{graphicx} 
%\usepackage[margins=2.5cm,nohead,nofoot]{geometry}
%\usepackage{geometry}
\usepackage{amsfonts}
\usepackage{amstext}
\usepackage{latexsym}
\usepackage{amssymb}
\usepackage{color}


%\include{myPreamble}
\include{qm2pi.local} 

%\ifpdf
%\usepackage[pdftex]{graphicx}
%\else
%\usepackage{graphicx}
%\fi

 % \ifpdf
%  \usepackage{pdfsync}
%  \if


%\title{Brief Article}
%\author{David F. Snyder}
%\author{L.G. Meredith}

%\address{Dept. of Math., Texas State University--San Marcos, San Marcos, TX 78666}
       
\pagestyle{empty}


\begin{document}

\lstset{language=[Objective]Caml,frame=shadowbox}

\input{qm2pi.front}

% section front matter (end)

\input{qm2pi.intro} 
 
% section introduction (end)

% \input{qm2pi.knotations} 

% section notation (end)

\input{qm2pi.process.calculi} 

% section concurrent_process_calculi_and_spatial_logics_ (end)
    
%\input{qm2pi.knots2pi} 

%\input{qm2pi.trefoil} 

%\input{qm2pi.mainthm} 

% subsection basic_interpretation (end)

%\input{qm2pi.rho.presentation} 
\subsection{The syntax and semantics of the notation system}\label{sub:the_syntax_and_semantics_of_the_notation_system} % (fold)

We now summarize a technical presentation of the calculus that
embodies our theory of dynamics. The typical presentation of such a
calculus follows the style of giving generators and relations on
them. The grammar, below, describing term constructors, freely
generates the set of processes, $\Proc$. This set is then quotiented
by a relation known as structural congruence and it is over this set
that the notion of dynamics is expressed. This presentation is
essentially that of \cite{MeredithR05} with the addition of
polyadicity and summation. For readability we have relegated some of
the technical subtleties to an appendix.

\subsubsection{Process grammar}\label{subsub:process_grammar}

\begin{mathpar}
  \inferrule* [lab=synchronization] {} {{M} \bc \pzero \;|\; x?F \;|\; x!C }
  \and
  \inferrule* [lab=abstraction] {} {{F} \bc (x)P}
  \and
  \inferrule* [lab=concretion] {} {{C} \bc \langle Q \rangle}
  \and
  \inferrule* [lab=process] {} {{P,Q} \bc M \;| \;P|Q \;|\; @{x}}
  \and
  \inferrule* [lab=name] {} {{x} \bc \quotep{P}}
\end{mathpar} 

Note that $\vec{x}$ (resp. $\vec{P}$) denotes a vector of names
(resp. processes) of length $|\vec{x}|$ (resp. $|\vec{P}|$). We adopt
the following useful abbreviations.

\begin{mathpar}
   x?(\vec{y}).P := x.(\vec{y})P \and  x\clift{\vec{P}} := x.\clift{\vec{P}}
   \and x!(y) := \lift{x}{\dropn{y}}
   \and \Pi_{i=0}^{n-1}P_i := P_0 | \ldots | P_{n-1}
\end{mathpar}

\subsubsection{Structural congruence}

\paragraph{Free and bound names and alpha-equivalence.} At the
core of structural equivalence is alpha-equivalence which identifies
process that are the same up to a change of variable. Formally, we
recognize the distinction between free and bound names. The free names
of a process, $\freenames{P}$, may be calculated recursively as
follows:

\begin{mathpar}
\freenames{\pzero} := \emptyset
  \and \\
  \freenames{x?(y).P} := \{ x \} \cup (\freenames{P} \setminus \{ y \})
  \and 
  \freenames{x!\langle P \rangle} := \{ x \} \cup \{ P \} 
  \and \\
  \freenames{P|Q} := \freenames{P} \cup \freenames{Q}
  \and \\
  \freenames{@{x}} := \{ x \}
\end{mathpar}

$\pi$
$\quotep{\pi}$

$\freenames{-} : \pi \to \mathcal{P}(\quotep{\pi})$

\begin{eqnarray*}
  \freenames{\pzero} & := & \emptyset \\
  \freenames{x?(y).P} & := & \{ x \} \cup (\freenames{P} \setminus \{ y \}) \\
  \freenames{x!\langle P \rangle} & := & \{ x \} \cup \{ P \} \\
  \freenames{P|Q} & := & \freenames{P} \cup \freenames{Q} \\
  \freenames{\dropn{x}} & := & \{ x \}
\end{eqnarray*}

The bound names of a process, $\boundnames{P}$, are those names occurring in $P$
that are not free. For example, in $x?(y).0$, the name $x$ is free, while $y$ is bound.

\begin{mathpar}
  \inferrule* [lab=monoidal-laws] {} { P|Q \equiv Q|P \and P|0 \equiv P \and P|(Q|R) \equiv (P|Q)|R }
\end{mathpar}

\begin{mathpar}
  \inferrule* [lab=alpha-equivalence] {} { (x)P \equiv (y)P\{y/x\} \and y \not\in \freenames{P} }
\end{mathpar}

\begin{definition}
Then two processes, $P,Q$, are alpha-equivalent if $P = Q\{\vec{y}/\vec{x}\}$ for
some $\vec{x} \in \boundnames{Q},\vec{y} \in \boundnames{P}$, where $Q\{\vec{y}/\vec{x}\}$
denotes the capture-avoiding substitution of $\vec{y}$ for $\vec{x}$ in $Q$.
\end{definition}

\begin{definition}
  The {\em structural congruence} \cite{SangiorgiWalker} , $\equiv$,
  between processes is the least congruence containing
  alpha-equivalence, satisfying the abelian monoid laws
  (associativity, commutativity and $\pzero$ as identity) for parallel
  composition $|$ and for summation $+$.
\end{definition}

\subsection{Name equivalence}

We take name equivalence, written $\nameeq$, to be the smallest
equivalence relation generated by the following rules.

\begin{mathpar}
\inferrule*[lab=Quote-drop]
{ }
{ \quotep{@{x}} \nameeq x }

\inferrule*[lab=Struct-equiv]
{ P \scong Q }
{ \quotep{P} \nameeq \quotep{Q} }
\end{mathpar}

The astute reader will have noticed that the mutual recursion of names
and processes imposes a mutual recursion on alpha-equivalence and
structural equivalence via name-equivalence. Fortunately, all of this
works out pleasantly and we may calculate in the natural way, free of
concern. The reader interested in the details is referred to the
appendix \ref{appendix:rho_details}.

\subsection{Substitution}

We use $\Proc$ for the set of processes, $\QProc$ for the set of
names, and $\id{\{}\vec{y} / \vec{x} \id{\}}$ to denote partial maps,
$s : \QProc \rightarrow \QProc$. A map, $s$ lifts, uniquely, to a map
on process terms, $\widehat{s} : \Proc \rightarrow \Proc$ by the
following equations.

\begin{mathpar}
  (0) \psubstp{Q}{P} := 0 \\
  (R \juxtap S) \psubstp{Q}{P}
  :=    
  (R)\psubstp{Q}{P} \juxtap (S) \psubstp{Q}{P} \\
  (x?(y).R) \psubstp{Q}{P}    
  :=    
  (x)\substp{Q}{P} (z)\concat( (R \psubstn{z}{y}) \psubstp{Q}{P} ) \\
  (\lift{x}{R}) \psubstp{Q}{P}  
  :=
  \lift{(x)\substp{Q}{P}}{ R \psubstp{Q}{P} } \\
%   (\dropn{x})  \psubstp{Q}{P}       
%   := 
%   \left\{ 
%     \begin{array}{ccc} 
%       \dropn{\quotep{Q}} & & x \nameeq \quotep{P} \\
%       \dropn{x} & & otherwise \\
%     \end{array}
%   \right. 
  (\dropn{x})  \psubstp{Q}{P}       
  := 
  \left\{ 
    \begin{array}{ccc} 
      Q & & x \nameeq \quotep{P} \\
      \dropn{x} & & otherwise \\
    \end{array}
  \right.
\end{mathpar}
 

where

\begin{eqnarray}
  (x)\id{\{} \lpquote Q \rpquote / \lpquote P \rpquote \id{\}}            = 
  \left\{ 
    \begin{array}{ccc}
      \lpquote Q \rpquote & & x \nameeq \lpquote P \rpquote \\
      x & & otherwise \\
    \end{array}
  \right. \nonumber
\end{eqnarray}

and $z$ is chosen distinct from $\quotep{P}$, $\quotep{Q}$, the free
names in $Q$, and all the names in $R$. Our $\alpha$-equivalence will
be built in the standard way from this substitution.

\begin{remark}\label{rem:no_self_referential_names}
  One consequence of these definitions is that $\forall P. \quotep{P}
  \not\in \freenames{P}$.
\end{remark}

\subsection{ Dynamic quote: an example }

Anticipating something of what's to come, consider applying the
substitution, $\widehat{\id{\{}u / z \id{\}}}$, to the following pair
of processes, $\lift{w}{y!(z)}$ and $w[ \lpquote y!(z) \rpquote ]$.

\begin{eqnarray}
	\lift{w}{y!(z)}\widehat{\id{\{}u / z \id{\}}}
		& = &
		\lift{w}{y!(u)} \nonumber\\
	w[ \lpquote y!(z) \rpquote ] \widehat{ \id{\{}u / z \id{\}} }
		& = &
		w[ \lpquote y!(z) \rpquote ] \nonumber
\end{eqnarray}

Because the body of the process between quotes is impervious to
substitution, we get radically different answers. In fact, by
examining the first process in an input context,
e.g. $x?(z).\lift{w}{y!(z)}$, we see that the process under the lift
operator may be shaped by prefixed inputs binding a name inside it. In
this sense, the lift operator will be seen as a way to dynamically
construct processes before reifying them as names.

Finally equipped with these standard features we can present the
dynamics of the calculus.

\subsubsection{Operational semantics} 

Finally, we introduce the computational dynamics. What marks these
algebras as distinct from other more traditionally studied algebraic
structures, e.g. vector spaces or polynomial rings, is the manner in
which dynamics is captured. In traditional structures, dynamics is typically
expressed through morphisms between such structures, as in linear maps
between vector spaces or morphisms between rings. In algebras
associated with the semantics of computation, the dynamics is
expressed as part of the algebraic structure itself, through a
reduction reduction relation typically denoted by $\red$. Below, we
give a recursive presentation of this relation for the calculus used
in the encoding.

$\red \subseteq \pi \times \pi$
$\red : \pi \to \mathcal{P}(\pi)$

\begin{mathpar}
  \inferrule* [lab=Comm] { \textsf{match}( x_{src}, x_{trgt} ) } { x_{trgt}?(y)P \; | \; x_{src}!\langle {Q} \rangle \red P\{\quotep{Q}/y}\} }
  \and \\
  \inferrule* [lab=Par] {{P} \red {P}'} {{{P} | {Q}} \red {{P}' | {Q}}}
  \and
  \inferrule* [lab=Equiv]{{{P} \scong {P}'} \andalso {{P}' \red {Q}'} \andalso {{Q}' \scong {Q}}}{{P} \red {Q}}
\end{mathpar}

\begin{eqnarray*}
  match_{\equiv} (\quotep{P},\quotep{Q}) & := & P \equiv Q \\
  match_{\dagger}(\quotep{P},\quotep{Q}) & := & \forall R. P|Q \red^{*} R => R \red^{*} 0 \\
  match_{K}(\quotep{P},\quotep{Q}) & := & K \mbox{ for some context } K
\end{eqnarray*}

$u?(x)P | u!\langle Q \rangle \red P\{\quotep{Q}/x\}$

%We write $\wred$ for $\red^*$, and $P\red$ if $\exists Q $ such that $ P \red Q$.
We write $P\red$ if $\exists Q $ such that $ P \red Q$ and $P\not\red$, otherwise.

\section{Replication}

As mentioned before, it is known that replication (and hence
recursion) can be implemented in a higher-order process algebra
\cite{SangiorgiWalker}. As our first example of calculation with the
machinery thus far presented we give the construction explicitly in
the {\rhoc}.

\begin{eqnarray}
	D_{x} & := & \prefix{x}{y}{(\binpar{\outputp{x}{y}}{@{y}})} \nonumber\\
	\bangp_{x}{P} & := & \binpar{{x}!\langle{\binpar{D_{x}}{P}}\rangle}{D_{x}} \nonumber
\end{eqnarray}

\begin{eqnarray}
	\bangp_{x}{P} & & \nonumber\\
	=
	& {x}!\langle{(\prefix{x}{y}{(\outputp{x}{y} | @{y})) | P}}\rangle 
	      | \prefix{x}{y}{(\outputp{x}{y} | @{y})} & \nonumber\\
	\red
	& (\outputp{x}{y} | @{y})\substn{\quotep{(\prefix{x}{y}{(@{y} | \outputp{x}{y})) | P}}}{y} & \nonumber\\
	=
	& \outputp{x}{\quotep{(\prefix{x}{y}{(\outputp{x}{y} | @{y})) | P}}}
	  | {(\prefix{x}{y}{(\outputp{x}{y} | @{y})) | P}} & \nonumber\\
	\red
	& \ldots & \nonumber\\
	\red^*
	& P | P | \ldots & \nonumber
\end{eqnarray}

Of course, this encoding, as an implementation, runs away, unfolding
$\bangp{P}$ eagerly. A lazier and more implementable replication
operator, restricted to input-guarded processes, may be obtained as follows.

\begin{eqnarray}
\bangp{\prefix{u}{v}{P}} 
	:= 
	\binpar{\lift{x}{\prefix{u}{v}{(\binpar{D(x)}{P})}}}{D(x)} \nonumber
\end{eqnarray}

\begin{remark}
  Note that the lazier definition still does not deal with summation
  or mixed summation (i.e. sums over input and output). The reader is
  invited to construct definitions of replication that deal with these
  features. 

  Further, the definitions are parameterized in a name, $x$. Can you,
  gentle reader, make a definition that eliminates this parameter and
  guarantees no accidental interaction between the replication
  machinery and the process being replicated -- i.e. no accidental
  sharing of names used by the process to get its work done and the
  name(s) used by the replication to effect copying. This latter
  revision of the definition of replication is crucial to obtaining
  the expected identity $!!P \sim !P$.
\end{remark}

\begin{remark}\label{rem:paradoxical_combinator}
  The reader familiar with the lambda calculus will have noticed the
  similarity between $D$ and the paradoxical combinator.

  [Ed. note: the existence of this seems to suggest we have to be more
  restrictive on the set of processes and names we admit if we are to
  support no-cloning.]
\end{remark}

\subsubsection{Bisimulation}

The computational dynamics gives rise to another kind of equivalence,
the equivalence of computational behavior. As previously mentioned
this is typically captured \emph{via} some form of bisimulation.

% The notion we use in this paper is weak barbed bisimulation
% \cite{milner91polyadicpi}.

The notion we use in this paper is derived from weak barbed
bisimulation \cite{milner91polyadicpi}. 

\begin{definition}
An \emph{observation relation}, $\downarrow_{\mathcal N}$, over a set
of names, $\mathcal N$, is the smallest relation satisfying the rules
below.

\infrule[Out-barb]{y \in {\mathcal N}, \; x \nameeq y}
		  {\outputp{x}{v} \downarrow_{\mathcal N} x}
\infrule[Par-barb]{\mbox{$P\downarrow_{\mathcal N} x$ or $Q\downarrow_{\mathcal N} x$}}
		  {\binpar{P}{Q} \downarrow_{\mathcal N} x}

We write $P \Downarrow_{\mathcal N} x$ if there is $Q$ such that 
$P \wred Q$ and $Q \downarrow_{\mathcal N} x$.
\end{definition}

\begin{definition}
%\label{def.bbisim}
An  ${\mathcal N}$-\emph{barbed bisimulation} over a set of names, ${\mathcal N}$, is a symmetric binary relation 
${\mathcal S}_{\mathcal N}$ between agents such that $P\rel{S}_{\mathcal N}Q$ implies:
\begin{enumerate}
\item If $P \red P'$ then $Q \wred Q'$ and $P'\rel{S}_{\mathcal N} Q'$.
\item If $P\downarrow_{\mathcal N} x$, then $Q\Downarrow_{\mathcal N} x$.
\end{enumerate}
$P$ is ${\mathcal N}$-barbed bisimilar to $Q$, written
$P \wbbisim_{\mathcal N} Q$, if $P \rel{S}_{\mathcal N} Q$ for some ${\mathcal N}$-barbed bisimulation ${\mathcal S}_{\mathcal N}$.
\end{definition}

$\mathcal{R} \subseteq \pi \times \pi$

$P \mathcal{R} Q => \forall P'. P \red P' \Rightarrow \exists Q'. Q \red Q', P' \mathcal{R} Q'$

$P \vdash x \Rightarrow Q \vdash x$

\begin{mathpar}
  \inferrule*[lab=Out-barb]{x \nameeq y}{{y}!\langle{Q}\rangle \vdash x}
  \and
  \inferrule*[lab=Par-barb]{\mbox{$P\vdash x$ or $Q\vdash x$}}{\binpar{P}{Q} \vdash x}
\end{mathpar}

\subsubsection{Contexts}

One of the principle advantages of computational calculi like the
$\pi$-calculus is a well-defined notion of context,
contextual-equivalence and a correlation between
contextual-equivalence and notions of bisimulation. The notion of
context allows the decomposition of a process into (sub-)process and
its syntactic environment, its context. Thus, a context may be
thought of as a process with a ``hole'' (written $\Box$) in it. The
application of a context $M$ to a process $P$, written $M[P]$, is
tantamount to filling the hole in $M$ with $P$. In this paper we do
not need the full weight of this theory, but do make use of the notion
of context in the proof the main theorem. 

\begin{mathpar}
  \inferrule* [lab=summation] {} {{M_{M},M_{N}} \bc \Box \;|\; x.M_{A} \;|\; M_{M}+M_{N}}
  \and
  \inferrule* [lab=agent] {} {{M_{A}} \bc (\vec{x})M_{P} \;| \; \clift{P_0,\ldots,M_{P},\ldots,P_N}}
  \and \\
  \inferrule* [lab=process] {} {{M_{P}} \bc M_{N} \;| \;P|M_{P} }
\end{mathpar} 

\begin{mathpar}
  \inferrule* [lab=sychronization] {} {M_{N} \bc \Box \;|\; x?M_{F} \;|\; x!M_{C}}
  \and
  \inferrule* [lab=abstraction] {} {{M_{F}} \bc (x)M_{P} }
  \and
  \inferrule* [lab=concretion] {} {{M_{C}} \bc \langle M_{P} \rangle }
  \and \\
  \inferrule* [lab=process] {} {{M_{P}} \bc M_{N} \;| \;P|M_{P} }
\end{mathpar}

\begin{definition}[contextual application] Given a context $M$, and
  process $P$, we define the \emph{contextual application}, $M[P] :=
  M\{P/\Box\}$. That is, the contextual application of M to P is the
  substitution of $P$ for $\Box$ in $M$.
\end{definition}

$\meaningof{-} : L \to \mathcal{P}(\pi)$

\begin{mathpar}
  \inferrule* [lab=collection] {} {\meaningof{true} = \pi, \and \meaningof{~E} = \pi \setminus \meaningof{E}, \and \meaningof{E_{1} \& E_{2}} = \meaningof{E_{1}} \cap \meaningof{E_{2}}}
\end{mathpar}

\begin{mathpar}
  \inferrule* [lab=structure] {} {\meaningof{0} = \{ P \in \pi | P \equiv 0 \}, \and \\ \meaningof{E_1 | E_2} = \{ P \in \pi | P \equiv P_{1} | P_{2}, P_{1} \in \meaningof{E_{1}}, P_{2} \in \meaningof{E_2}\} }
\end{mathpar}

\begin{mathpar}
 \inferrule* [lab=behavior] {} {\meaningof{\langle a?b \rangle E} = \{ P \in \pi | P \equiv Q | u?(y)P', \\ \and \\\\ \and \\ \;\;\; u \in \meaningof{a}, \forall z.P'\{z/y\} \in \meaningof{E\{z/b\}}\}, \and \\ \meaningof{a!E} = \{ P \in \pi | P \equiv Q | x!\langle P' \rangle, x \in \meaningof{a} P' \in \meaningof{E}\} }
\end{mathpar}

\begin{mathpar}
 \inferrule* [lab=nominal] {} {\meaningof{\quotep{E}} = \{ \quotep{P} \in \quotep{\pi} | P \in \meaningof{E} \}, \and \meaningof{\quotep{P}} = \{ \quotep{Q} \in \quotep{\pi} | P \equiv Q \} \and \\ \meaningof{@\quotep{E}} = \{ P \in \pi | P \equiv @x, x \in \meaningof{E} \}}
\end{mathpar}

\begin{eqnarray*}
  \\
  \meaningof{-} : TS \to ST
\end{eqnarray*}

\begin{eqnarray*}
  \\
  L : TS \to ST
\end{eqnarray*}

\begin{eqnarray*}
  \\
  P \models E \iff P \in \meaningof{E}
\end{eqnarray*}

\begin{eqnarray*}
  P \approx_{L} Q \iff \forall E \in L. P \models E \iff Q \models E
\end{eqnarray*}

\begin{eqnarray*}
  P \approx_{K} Q
\end{eqnarray*}

\begin{eqnarray*}
  P \approx Q
\end{eqnarray*}

$\approx_{K} = \approx = \approx_{L}$

\subsubsection{Contextual duality}

Note that contexts extend the quotation operation to a family of
operations from processes to names. Given a context, $M$, we can
define a \emph{nominal context}, $\quotep{M}$ by $\quotep{M}[P] :=
\quotep{M[P]}$. To foreshadow what is to come we observe that these
operations enjoy a duality with processes very much like the duality
between vectors and maps from vectors to scalars.

Further, because the calculus is essentially higher-order, we have a
correspondence between contexts and processes. More specifically,
given a name $x$ and a context $M$ we can construct $M^{*}_{x}$ such
that 

\begin{mathpar}
  M^{*}_{x} | \lift{x}{P} \red M[P]
\end{mathpar}

namely,

\begin{mathpar}
  M^{*}_{x} := x?(u).M[\dropn{u}]
\end{mathpar}

The dependence of $M^{*}_{x}$ on a name makes it an abstraction, 

\begin{mathpar}
  M^{*} := (x)x?(u).M[\dropn{u}]
\end{mathpar}

\subsection{Additional notation}

It will sometimes be convenient to denote the process a name
quotes. We already have the notation $x = \quotep{P}$, but it will be
convenient to introduce an alternate notation, $\procn{x}$, when we
want to emphasize the connection to the use of the name. Note that, by
virtue of name equivalence, $\quotep{\procn{x}} \nameeq x$; so, the
notation is consistent with previous definitions.

Further, because names have structure it is possible to effect
substitutions on the basis of that structure. This means we need to
upgrade our notation for substitutions, which we accomplish by
adapting comprehension notation. Thus,

\begin{mathpar}
  P\{ y / x : x \in S \}
\end{mathpar}

is interpreted to mean the process derived from P by replacing (in a
capture-avoiding manner) each occurrence of $x$ in $S$ by $y$. For example,

\begin{mathpar}
  P\{ \quotep{\procn{x}|\procn{x}} / x : x \in \freenames{P} \}
\end{mathpar}

will replace each (occurrence) of a free name $x$ in $P$ by
$\quotep{\procn{x}|\procn{x}}$.

Also, we will avail ourselves of the notation $x^{L}$ and $x^{R}$ to
denote injections of a name into disjoint copies of the name
space. There are numerous ways to accomplish this. One example can be
found in \cite{MeredithR05}. This notation overloads to vectors of
names: $\vec{x}^{\pi} := (x_{i}^{\pi} \; : \; 0 \leq i < |\vec{x}| )$ where $\pi \in \{L,R\}$.

We also use $P^{\Box} := P|\Box$.

In \cite{MeredithR05} an interpretation of the new operator is
given. It turns out that there are several possible interpretations
all enjoying the requisite algebraic properties of the operator (see
\cite{milner91polyadicpi}). We will therefore make liberal use of
$(\nu\; \vec{x})P$.

% subsection the_syntax_and_semantics_of_the_notation_system (end)   

\input{qm2pi.qmops} 

\input{qm2pi.sterngerlach} 

\input{qm2pi.metric} 

% section concurrent_process_calculi (end)

%\input{qm2pi.proofsketch}

% section proof sketch (end)

%\input{qm2pi.slviaknots} 

% section spatial logic via knots (end)

\input{qm2pi.conclusion}

% section conclusion (end)

%\input{qm2pi.dtcodes} 

% section wiring algorithm (end)

\input{qm2pi.ack} 

% section acknowledgments (end)

\newpage


\bibliographystyle{plain}   
\bibliography{../../biblios/main.bib}

\input{qm2pi.rhodetails}

\end{document}



% section front matter (end)

\section{Introduction}\label{sec:introduction} % (fold)
In this draft of the material i am going to have to dispense with the
usual writing conventions adopted in papers on these topics. i'm going
to have adopt whatever tone i need at the time i'm writing up the
calculations. Sometimes this may be very conversational; others it may
be the barest mathematical grunts; others still it may be that i have
lifted text from one of my other papers because the exposition of some
point was better said there. i hope that my readers are not unduly put
out by this decision. i'm not doing this to flout convention or be
rebellious. i find these calculations very technically challenging. To
keep everything going technically, something has to give; i have to
let go of some cognitive burden. So, the academic writing style --
with all of its trade-offs in terms of facilitating technical
communication -- is what i'm letting go of. Perhaps subsequent drafts
can be tightened and polished, but for now, i'm going to speak as if
we were sitting together in a coffee shop with a laptop, wifi and a
pad of paper and a pencil.

So, here's what i have to say. We -- you and i, comfortably ensconced
in our coffee shop and well-equipped with our tools -- can realize and
carry out the calculations of quantum mechanics over a very different
formal theory of dynamics, a formal theory of dynamics that
corresponds to a theory of concurrent computation with
\emph{reflection}. It has the advantage that the underlying theory is
already `quantized', but supports analogues all of the continuuous
operations. Strikingly, this underlying theory has recently been
connected with a notion of metric that we can show, by calculating
together, coincides with the metric induced by the inner product.

There are a lot of reasons why you might be interested in seeing
calculations of this form. Here's why i'm interested. For the past
several centuries there has been no competitor to the ``Newtonian''
account of dynamics. As a result the predominant share of accounts of
dynamical systems and situations have had to be formulated in terms of
the Newtonian machinery. i view this as an intellectually dangerous
position to occupy. Everything, despite it's intrinsic shape, turns
into a nail to be hit with this hammer. Recently, however, the theory
of computation has matured to the point where we have candidates for
theories of dynamics that offer very different perspective on
reasoning about dynamical systems and situations. Testing these
candidates against very successful accounts of dynamical situations,
like quantum mechanics, is going to give us some sense of how mature
they are and some measure of the quality of these accounts of
dynamics.

\subsection{Summary of contributions and outline of paper}

So, we're going to develop an interpretation of the operations of
quantum mechanics normally interpreted by Hilbert spaces and
operators. We're going to do this over a theory of computation. Note
that this is very different than the usual quantum computation program
which develops notions of computation over quantum mechanics. Rather,
we are developing a story that aligns with Wheeler's slogan: It from
Bit. To do this we will first provide an account of the theory of
computation at play here. Then we will dive into a calculation-driven
interpretation of the operations of quantum mechanics.

The reason we take this approach is that -- until very recently --
there hasn't been an axiomatic account of quantum mechanics. As a
result there has been no sharp delineation of the mathematical theory
supporting interpretation of the physical theory and the physical
theory, itself. So, ambient features of the maths are free to be
exploited (or supressed) without a real accounting of their physical
relevance. There is no sharp statement ``here's the physical theory''
qua \emph{theory} and ``here's the mathematical interpretation''
enabling a judgment of how faithful the interpretation is -- apart
from experimental observation. When there is an axiomatic account we
can judge how well a given mathematical formalism supports an
interpretation of the axioms, independent of
experimentation. Likewise, we can judge how well we have captured our
physical evidence and experience with our axiomatics, independent of
any specific mathematical implementation, with accidental detail that
may or may not have physical significance. 

In lieu of a fully fleshed out and vetted axiomatic account of quantum
mechanics, interpreting the operational notions in service of modeling
physical systems will have to suffice. In other words, we are not in
the business of providing a model of Hilbert spaces and operators. We
are in the business of providing a model of quantum mechanics because
we are motivated by testing our notions of dynamics against physical
theory; and, the predictive calculations of the physical theory must
serve as the best formulation -- shy of a fully fleshed out axiomatic
account -- of the physical theory itself (as they have for scientific
theories since time immemorial). Put another way, despite a
whole-hearted commitment to an It-from-Bit ontology, we are firmly
aligned with the shut-up-and-calculate camp as the best way to obtain
results either from the physical perspective or as a quality assurance
measure of our fledgling theory of dynamics.

In detail, we present a reflective process calculus. Then we develop
intuitive correspondences between the notions available in this
calculus and the usual physical notions supporting quantum mechanical
calculations. Thus, 

\begin{table}[htp]
  \center{
    \fbox{
      \begin{tabular}{c|c}
        quantum mechanics & process calculus \\
        \hline
        scalar & name \\
        state vector & process \\
        dual & contextual duals \\
        matrix & formal sums of process-context-dual pairs \\
        orthogonality & process annihilation \\
        inner product & execution-formula + quoting
      \end{tabular}
    }
  }
  \caption{QM - process calculi correspondences}
\end{table}

Then we tighten up these intuitions to operational definitions. We
employ the Dirac notation as the best proxy we can find for an
abstract syntax of the quantum mechanical notions. The definitions we
develop put us in contact with equational constraints coming from the
theory that we demonstrate the definitions and calculations satisfy.

This puts us in a position to shut up and calculate for the
Stern-Gerlach experimental set up, showing how these predictive
calculations become calculations on processes in our theory of a
reflective process calculus.

Penultimately, we demonstrate that the notion of metric coming from
the inner product coincides with the notion of metric available from
the theory of bisimulation. This demonstration gives us the right to
think of space as arising from behavior. Finally, we consider where we
might go from the new vantage point we have obtained.

% section introduction (end) 
 
% section introduction (end)

% \documentclass[12pt]{llncs}
%\documentclass{jktr}

\usepackage[pdftex]{hyperref}                   
\usepackage {listings}
\usepackage {mathpartir}
\usepackage{bcprules}
%\usepackage{listings}
                       
\usepackage{graphicx} 
%\usepackage[margins=2.5cm,nohead,nofoot]{geometry}
%\usepackage{geometry}
\usepackage{amsfonts}
\usepackage{amstext}
\usepackage{latexsym}
\usepackage{amssymb}
\usepackage{color}


%\include{myPreamble}
\include{qm2pi.local} 

%\ifpdf
%\usepackage[pdftex]{graphicx}
%\else
%\usepackage{graphicx}
%\fi

 % \ifpdf
%  \usepackage{pdfsync}
%  \if


%\title{Brief Article}
%\author{David F. Snyder}
%\author{L.G. Meredith}

%\address{Dept. of Math., Texas State University--San Marcos, San Marcos, TX 78666}
       
\pagestyle{empty}


\begin{document}

\lstset{language=[Objective]Caml,frame=shadowbox}

\input{qm2pi.front}

% section front matter (end)

\input{qm2pi.intro} 
 
% section introduction (end)

% \input{qm2pi.knotations} 

% section notation (end)

\input{qm2pi.process.calculi} 

% section concurrent_process_calculi_and_spatial_logics_ (end)
    
%\input{qm2pi.knots2pi} 

%\input{qm2pi.trefoil} 

%\input{qm2pi.mainthm} 

% subsection basic_interpretation (end)

%\input{qm2pi.rho.presentation} 
\subsection{The syntax and semantics of the notation system}\label{sub:the_syntax_and_semantics_of_the_notation_system} % (fold)

We now summarize a technical presentation of the calculus that
embodies our theory of dynamics. The typical presentation of such a
calculus follows the style of giving generators and relations on
them. The grammar, below, describing term constructors, freely
generates the set of processes, $\Proc$. This set is then quotiented
by a relation known as structural congruence and it is over this set
that the notion of dynamics is expressed. This presentation is
essentially that of \cite{MeredithR05} with the addition of
polyadicity and summation. For readability we have relegated some of
the technical subtleties to an appendix.

\subsubsection{Process grammar}\label{subsub:process_grammar}

\begin{mathpar}
  \inferrule* [lab=synchronization] {} {{M} \bc \pzero \;|\; x?F \;|\; x!C }
  \and
  \inferrule* [lab=abstraction] {} {{F} \bc (x)P}
  \and
  \inferrule* [lab=concretion] {} {{C} \bc \langle Q \rangle}
  \and
  \inferrule* [lab=process] {} {{P,Q} \bc M \;| \;P|Q \;|\; @{x}}
  \and
  \inferrule* [lab=name] {} {{x} \bc \quotep{P}}
\end{mathpar} 

Note that $\vec{x}$ (resp. $\vec{P}$) denotes a vector of names
(resp. processes) of length $|\vec{x}|$ (resp. $|\vec{P}|$). We adopt
the following useful abbreviations.

\begin{mathpar}
   x?(\vec{y}).P := x.(\vec{y})P \and  x\clift{\vec{P}} := x.\clift{\vec{P}}
   \and x!(y) := \lift{x}{\dropn{y}}
   \and \Pi_{i=0}^{n-1}P_i := P_0 | \ldots | P_{n-1}
\end{mathpar}

\subsubsection{Structural congruence}

\paragraph{Free and bound names and alpha-equivalence.} At the
core of structural equivalence is alpha-equivalence which identifies
process that are the same up to a change of variable. Formally, we
recognize the distinction between free and bound names. The free names
of a process, $\freenames{P}$, may be calculated recursively as
follows:

\begin{mathpar}
\freenames{\pzero} := \emptyset
  \and \\
  \freenames{x?(y).P} := \{ x \} \cup (\freenames{P} \setminus \{ y \})
  \and 
  \freenames{x!\langle P \rangle} := \{ x \} \cup \{ P \} 
  \and \\
  \freenames{P|Q} := \freenames{P} \cup \freenames{Q}
  \and \\
  \freenames{@{x}} := \{ x \}
\end{mathpar}

$\pi$
$\quotep{\pi}$

$\freenames{-} : \pi \to \mathcal{P}(\quotep{\pi})$

\begin{eqnarray*}
  \freenames{\pzero} & := & \emptyset \\
  \freenames{x?(y).P} & := & \{ x \} \cup (\freenames{P} \setminus \{ y \}) \\
  \freenames{x!\langle P \rangle} & := & \{ x \} \cup \{ P \} \\
  \freenames{P|Q} & := & \freenames{P} \cup \freenames{Q} \\
  \freenames{\dropn{x}} & := & \{ x \}
\end{eqnarray*}

The bound names of a process, $\boundnames{P}$, are those names occurring in $P$
that are not free. For example, in $x?(y).0$, the name $x$ is free, while $y$ is bound.

\begin{mathpar}
  \inferrule* [lab=monoidal-laws] {} { P|Q \equiv Q|P \and P|0 \equiv P \and P|(Q|R) \equiv (P|Q)|R }
\end{mathpar}

\begin{mathpar}
  \inferrule* [lab=alpha-equivalence] {} { (x)P \equiv (y)P\{y/x\} \and y \not\in \freenames{P} }
\end{mathpar}

\begin{definition}
Then two processes, $P,Q$, are alpha-equivalent if $P = Q\{\vec{y}/\vec{x}\}$ for
some $\vec{x} \in \boundnames{Q},\vec{y} \in \boundnames{P}$, where $Q\{\vec{y}/\vec{x}\}$
denotes the capture-avoiding substitution of $\vec{y}$ for $\vec{x}$ in $Q$.
\end{definition}

\begin{definition}
  The {\em structural congruence} \cite{SangiorgiWalker} , $\equiv$,
  between processes is the least congruence containing
  alpha-equivalence, satisfying the abelian monoid laws
  (associativity, commutativity and $\pzero$ as identity) for parallel
  composition $|$ and for summation $+$.
\end{definition}

\subsection{Name equivalence}

We take name equivalence, written $\nameeq$, to be the smallest
equivalence relation generated by the following rules.

\begin{mathpar}
\inferrule*[lab=Quote-drop]
{ }
{ \quotep{@{x}} \nameeq x }

\inferrule*[lab=Struct-equiv]
{ P \scong Q }
{ \quotep{P} \nameeq \quotep{Q} }
\end{mathpar}

The astute reader will have noticed that the mutual recursion of names
and processes imposes a mutual recursion on alpha-equivalence and
structural equivalence via name-equivalence. Fortunately, all of this
works out pleasantly and we may calculate in the natural way, free of
concern. The reader interested in the details is referred to the
appendix \ref{appendix:rho_details}.

\subsection{Substitution}

We use $\Proc$ for the set of processes, $\QProc$ for the set of
names, and $\id{\{}\vec{y} / \vec{x} \id{\}}$ to denote partial maps,
$s : \QProc \rightarrow \QProc$. A map, $s$ lifts, uniquely, to a map
on process terms, $\widehat{s} : \Proc \rightarrow \Proc$ by the
following equations.

\begin{mathpar}
  (0) \psubstp{Q}{P} := 0 \\
  (R \juxtap S) \psubstp{Q}{P}
  :=    
  (R)\psubstp{Q}{P} \juxtap (S) \psubstp{Q}{P} \\
  (x?(y).R) \psubstp{Q}{P}    
  :=    
  (x)\substp{Q}{P} (z)\concat( (R \psubstn{z}{y}) \psubstp{Q}{P} ) \\
  (\lift{x}{R}) \psubstp{Q}{P}  
  :=
  \lift{(x)\substp{Q}{P}}{ R \psubstp{Q}{P} } \\
%   (\dropn{x})  \psubstp{Q}{P}       
%   := 
%   \left\{ 
%     \begin{array}{ccc} 
%       \dropn{\quotep{Q}} & & x \nameeq \quotep{P} \\
%       \dropn{x} & & otherwise \\
%     \end{array}
%   \right. 
  (\dropn{x})  \psubstp{Q}{P}       
  := 
  \left\{ 
    \begin{array}{ccc} 
      Q & & x \nameeq \quotep{P} \\
      \dropn{x} & & otherwise \\
    \end{array}
  \right.
\end{mathpar}
 

where

\begin{eqnarray}
  (x)\id{\{} \lpquote Q \rpquote / \lpquote P \rpquote \id{\}}            = 
  \left\{ 
    \begin{array}{ccc}
      \lpquote Q \rpquote & & x \nameeq \lpquote P \rpquote \\
      x & & otherwise \\
    \end{array}
  \right. \nonumber
\end{eqnarray}

and $z$ is chosen distinct from $\quotep{P}$, $\quotep{Q}$, the free
names in $Q$, and all the names in $R$. Our $\alpha$-equivalence will
be built in the standard way from this substitution.

\begin{remark}\label{rem:no_self_referential_names}
  One consequence of these definitions is that $\forall P. \quotep{P}
  \not\in \freenames{P}$.
\end{remark}

\subsection{ Dynamic quote: an example }

Anticipating something of what's to come, consider applying the
substitution, $\widehat{\id{\{}u / z \id{\}}}$, to the following pair
of processes, $\lift{w}{y!(z)}$ and $w[ \lpquote y!(z) \rpquote ]$.

\begin{eqnarray}
	\lift{w}{y!(z)}\widehat{\id{\{}u / z \id{\}}}
		& = &
		\lift{w}{y!(u)} \nonumber\\
	w[ \lpquote y!(z) \rpquote ] \widehat{ \id{\{}u / z \id{\}} }
		& = &
		w[ \lpquote y!(z) \rpquote ] \nonumber
\end{eqnarray}

Because the body of the process between quotes is impervious to
substitution, we get radically different answers. In fact, by
examining the first process in an input context,
e.g. $x?(z).\lift{w}{y!(z)}$, we see that the process under the lift
operator may be shaped by prefixed inputs binding a name inside it. In
this sense, the lift operator will be seen as a way to dynamically
construct processes before reifying them as names.

Finally equipped with these standard features we can present the
dynamics of the calculus.

\subsubsection{Operational semantics} 

Finally, we introduce the computational dynamics. What marks these
algebras as distinct from other more traditionally studied algebraic
structures, e.g. vector spaces or polynomial rings, is the manner in
which dynamics is captured. In traditional structures, dynamics is typically
expressed through morphisms between such structures, as in linear maps
between vector spaces or morphisms between rings. In algebras
associated with the semantics of computation, the dynamics is
expressed as part of the algebraic structure itself, through a
reduction reduction relation typically denoted by $\red$. Below, we
give a recursive presentation of this relation for the calculus used
in the encoding.

$\red \subseteq \pi \times \pi$
$\red : \pi \to \mathcal{P}(\pi)$

\begin{mathpar}
  \inferrule* [lab=Comm] { \textsf{match}( x_{src}, x_{trgt} ) } { x_{trgt}?(y)P \; | \; x_{src}!\langle {Q} \rangle \red P\{\quotep{Q}/y}\} }
  \and \\
  \inferrule* [lab=Par] {{P} \red {P}'} {{{P} | {Q}} \red {{P}' | {Q}}}
  \and
  \inferrule* [lab=Equiv]{{{P} \scong {P}'} \andalso {{P}' \red {Q}'} \andalso {{Q}' \scong {Q}}}{{P} \red {Q}}
\end{mathpar}

\begin{eqnarray*}
  match_{\equiv} (\quotep{P},\quotep{Q}) & := & P \equiv Q \\
  match_{\dagger}(\quotep{P},\quotep{Q}) & := & \forall R. P|Q \red^{*} R => R \red^{*} 0 \\
  match_{K}(\quotep{P},\quotep{Q}) & := & K \mbox{ for some context } K
\end{eqnarray*}

$u?(x)P | u!\langle Q \rangle \red P\{\quotep{Q}/x\}$

%We write $\wred$ for $\red^*$, and $P\red$ if $\exists Q $ such that $ P \red Q$.
We write $P\red$ if $\exists Q $ such that $ P \red Q$ and $P\not\red$, otherwise.

\section{Replication}

As mentioned before, it is known that replication (and hence
recursion) can be implemented in a higher-order process algebra
\cite{SangiorgiWalker}. As our first example of calculation with the
machinery thus far presented we give the construction explicitly in
the {\rhoc}.

\begin{eqnarray}
	D_{x} & := & \prefix{x}{y}{(\binpar{\outputp{x}{y}}{@{y}})} \nonumber\\
	\bangp_{x}{P} & := & \binpar{{x}!\langle{\binpar{D_{x}}{P}}\rangle}{D_{x}} \nonumber
\end{eqnarray}

\begin{eqnarray}
	\bangp_{x}{P} & & \nonumber\\
	=
	& {x}!\langle{(\prefix{x}{y}{(\outputp{x}{y} | @{y})) | P}}\rangle 
	      | \prefix{x}{y}{(\outputp{x}{y} | @{y})} & \nonumber\\
	\red
	& (\outputp{x}{y} | @{y})\substn{\quotep{(\prefix{x}{y}{(@{y} | \outputp{x}{y})) | P}}}{y} & \nonumber\\
	=
	& \outputp{x}{\quotep{(\prefix{x}{y}{(\outputp{x}{y} | @{y})) | P}}}
	  | {(\prefix{x}{y}{(\outputp{x}{y} | @{y})) | P}} & \nonumber\\
	\red
	& \ldots & \nonumber\\
	\red^*
	& P | P | \ldots & \nonumber
\end{eqnarray}

Of course, this encoding, as an implementation, runs away, unfolding
$\bangp{P}$ eagerly. A lazier and more implementable replication
operator, restricted to input-guarded processes, may be obtained as follows.

\begin{eqnarray}
\bangp{\prefix{u}{v}{P}} 
	:= 
	\binpar{\lift{x}{\prefix{u}{v}{(\binpar{D(x)}{P})}}}{D(x)} \nonumber
\end{eqnarray}

\begin{remark}
  Note that the lazier definition still does not deal with summation
  or mixed summation (i.e. sums over input and output). The reader is
  invited to construct definitions of replication that deal with these
  features. 

  Further, the definitions are parameterized in a name, $x$. Can you,
  gentle reader, make a definition that eliminates this parameter and
  guarantees no accidental interaction between the replication
  machinery and the process being replicated -- i.e. no accidental
  sharing of names used by the process to get its work done and the
  name(s) used by the replication to effect copying. This latter
  revision of the definition of replication is crucial to obtaining
  the expected identity $!!P \sim !P$.
\end{remark}

\begin{remark}\label{rem:paradoxical_combinator}
  The reader familiar with the lambda calculus will have noticed the
  similarity between $D$ and the paradoxical combinator.

  [Ed. note: the existence of this seems to suggest we have to be more
  restrictive on the set of processes and names we admit if we are to
  support no-cloning.]
\end{remark}

\subsubsection{Bisimulation}

The computational dynamics gives rise to another kind of equivalence,
the equivalence of computational behavior. As previously mentioned
this is typically captured \emph{via} some form of bisimulation.

% The notion we use in this paper is weak barbed bisimulation
% \cite{milner91polyadicpi}.

The notion we use in this paper is derived from weak barbed
bisimulation \cite{milner91polyadicpi}. 

\begin{definition}
An \emph{observation relation}, $\downarrow_{\mathcal N}$, over a set
of names, $\mathcal N$, is the smallest relation satisfying the rules
below.

\infrule[Out-barb]{y \in {\mathcal N}, \; x \nameeq y}
		  {\outputp{x}{v} \downarrow_{\mathcal N} x}
\infrule[Par-barb]{\mbox{$P\downarrow_{\mathcal N} x$ or $Q\downarrow_{\mathcal N} x$}}
		  {\binpar{P}{Q} \downarrow_{\mathcal N} x}

We write $P \Downarrow_{\mathcal N} x$ if there is $Q$ such that 
$P \wred Q$ and $Q \downarrow_{\mathcal N} x$.
\end{definition}

\begin{definition}
%\label{def.bbisim}
An  ${\mathcal N}$-\emph{barbed bisimulation} over a set of names, ${\mathcal N}$, is a symmetric binary relation 
${\mathcal S}_{\mathcal N}$ between agents such that $P\rel{S}_{\mathcal N}Q$ implies:
\begin{enumerate}
\item If $P \red P'$ then $Q \wred Q'$ and $P'\rel{S}_{\mathcal N} Q'$.
\item If $P\downarrow_{\mathcal N} x$, then $Q\Downarrow_{\mathcal N} x$.
\end{enumerate}
$P$ is ${\mathcal N}$-barbed bisimilar to $Q$, written
$P \wbbisim_{\mathcal N} Q$, if $P \rel{S}_{\mathcal N} Q$ for some ${\mathcal N}$-barbed bisimulation ${\mathcal S}_{\mathcal N}$.
\end{definition}

$\mathcal{R} \subseteq \pi \times \pi$

$P \mathcal{R} Q => \forall P'. P \red P' \Rightarrow \exists Q'. Q \red Q', P' \mathcal{R} Q'$

$P \vdash x \Rightarrow Q \vdash x$

\begin{mathpar}
  \inferrule*[lab=Out-barb]{x \nameeq y}{{y}!\langle{Q}\rangle \vdash x}
  \and
  \inferrule*[lab=Par-barb]{\mbox{$P\vdash x$ or $Q\vdash x$}}{\binpar{P}{Q} \vdash x}
\end{mathpar}

\subsubsection{Contexts}

One of the principle advantages of computational calculi like the
$\pi$-calculus is a well-defined notion of context,
contextual-equivalence and a correlation between
contextual-equivalence and notions of bisimulation. The notion of
context allows the decomposition of a process into (sub-)process and
its syntactic environment, its context. Thus, a context may be
thought of as a process with a ``hole'' (written $\Box$) in it. The
application of a context $M$ to a process $P$, written $M[P]$, is
tantamount to filling the hole in $M$ with $P$. In this paper we do
not need the full weight of this theory, but do make use of the notion
of context in the proof the main theorem. 

\begin{mathpar}
  \inferrule* [lab=summation] {} {{M_{M},M_{N}} \bc \Box \;|\; x.M_{A} \;|\; M_{M}+M_{N}}
  \and
  \inferrule* [lab=agent] {} {{M_{A}} \bc (\vec{x})M_{P} \;| \; \clift{P_0,\ldots,M_{P},\ldots,P_N}}
  \and \\
  \inferrule* [lab=process] {} {{M_{P}} \bc M_{N} \;| \;P|M_{P} }
\end{mathpar} 

\begin{mathpar}
  \inferrule* [lab=sychronization] {} {M_{N} \bc \Box \;|\; x?M_{F} \;|\; x!M_{C}}
  \and
  \inferrule* [lab=abstraction] {} {{M_{F}} \bc (x)M_{P} }
  \and
  \inferrule* [lab=concretion] {} {{M_{C}} \bc \langle M_{P} \rangle }
  \and \\
  \inferrule* [lab=process] {} {{M_{P}} \bc M_{N} \;| \;P|M_{P} }
\end{mathpar}

\begin{definition}[contextual application] Given a context $M$, and
  process $P$, we define the \emph{contextual application}, $M[P] :=
  M\{P/\Box\}$. That is, the contextual application of M to P is the
  substitution of $P$ for $\Box$ in $M$.
\end{definition}

$\meaningof{-} : L \to \mathcal{P}(\pi)$

\begin{mathpar}
  \inferrule* [lab=collection] {} {\meaningof{true} = \pi, \and \meaningof{~E} = \pi \setminus \meaningof{E}, \and \meaningof{E_{1} \& E_{2}} = \meaningof{E_{1}} \cap \meaningof{E_{2}}}
\end{mathpar}

\begin{mathpar}
  \inferrule* [lab=structure] {} {\meaningof{0} = \{ P \in \pi | P \equiv 0 \}, \and \\ \meaningof{E_1 | E_2} = \{ P \in \pi | P \equiv P_{1} | P_{2}, P_{1} \in \meaningof{E_{1}}, P_{2} \in \meaningof{E_2}\} }
\end{mathpar}

\begin{mathpar}
 \inferrule* [lab=behavior] {} {\meaningof{\langle a?b \rangle E} = \{ P \in \pi | P \equiv Q | u?(y)P', \\ \and \\\\ \and \\ \;\;\; u \in \meaningof{a}, \forall z.P'\{z/y\} \in \meaningof{E\{z/b\}}\}, \and \\ \meaningof{a!E} = \{ P \in \pi | P \equiv Q | x!\langle P' \rangle, x \in \meaningof{a} P' \in \meaningof{E}\} }
\end{mathpar}

\begin{mathpar}
 \inferrule* [lab=nominal] {} {\meaningof{\quotep{E}} = \{ \quotep{P} \in \quotep{\pi} | P \in \meaningof{E} \}, \and \meaningof{\quotep{P}} = \{ \quotep{Q} \in \quotep{\pi} | P \equiv Q \} \and \\ \meaningof{@\quotep{E}} = \{ P \in \pi | P \equiv @x, x \in \meaningof{E} \}}
\end{mathpar}

\begin{eqnarray*}
  \\
  \meaningof{-} : TS \to ST
\end{eqnarray*}

\begin{eqnarray*}
  \\
  L : TS \to ST
\end{eqnarray*}

\begin{eqnarray*}
  \\
  P \models E \iff P \in \meaningof{E}
\end{eqnarray*}

\begin{eqnarray*}
  P \approx_{L} Q \iff \forall E \in L. P \models E \iff Q \models E
\end{eqnarray*}

\begin{eqnarray*}
  P \approx_{K} Q
\end{eqnarray*}

\begin{eqnarray*}
  P \approx Q
\end{eqnarray*}

$\approx_{K} = \approx = \approx_{L}$

\subsubsection{Contextual duality}

Note that contexts extend the quotation operation to a family of
operations from processes to names. Given a context, $M$, we can
define a \emph{nominal context}, $\quotep{M}$ by $\quotep{M}[P] :=
\quotep{M[P]}$. To foreshadow what is to come we observe that these
operations enjoy a duality with processes very much like the duality
between vectors and maps from vectors to scalars.

Further, because the calculus is essentially higher-order, we have a
correspondence between contexts and processes. More specifically,
given a name $x$ and a context $M$ we can construct $M^{*}_{x}$ such
that 

\begin{mathpar}
  M^{*}_{x} | \lift{x}{P} \red M[P]
\end{mathpar}

namely,

\begin{mathpar}
  M^{*}_{x} := x?(u).M[\dropn{u}]
\end{mathpar}

The dependence of $M^{*}_{x}$ on a name makes it an abstraction, 

\begin{mathpar}
  M^{*} := (x)x?(u).M[\dropn{u}]
\end{mathpar}

\subsection{Additional notation}

It will sometimes be convenient to denote the process a name
quotes. We already have the notation $x = \quotep{P}$, but it will be
convenient to introduce an alternate notation, $\procn{x}$, when we
want to emphasize the connection to the use of the name. Note that, by
virtue of name equivalence, $\quotep{\procn{x}} \nameeq x$; so, the
notation is consistent with previous definitions.

Further, because names have structure it is possible to effect
substitutions on the basis of that structure. This means we need to
upgrade our notation for substitutions, which we accomplish by
adapting comprehension notation. Thus,

\begin{mathpar}
  P\{ y / x : x \in S \}
\end{mathpar}

is interpreted to mean the process derived from P by replacing (in a
capture-avoiding manner) each occurrence of $x$ in $S$ by $y$. For example,

\begin{mathpar}
  P\{ \quotep{\procn{x}|\procn{x}} / x : x \in \freenames{P} \}
\end{mathpar}

will replace each (occurrence) of a free name $x$ in $P$ by
$\quotep{\procn{x}|\procn{x}}$.

Also, we will avail ourselves of the notation $x^{L}$ and $x^{R}$ to
denote injections of a name into disjoint copies of the name
space. There are numerous ways to accomplish this. One example can be
found in \cite{MeredithR05}. This notation overloads to vectors of
names: $\vec{x}^{\pi} := (x_{i}^{\pi} \; : \; 0 \leq i < |\vec{x}| )$ where $\pi \in \{L,R\}$.

We also use $P^{\Box} := P|\Box$.

In \cite{MeredithR05} an interpretation of the new operator is
given. It turns out that there are several possible interpretations
all enjoying the requisite algebraic properties of the operator (see
\cite{milner91polyadicpi}). We will therefore make liberal use of
$(\nu\; \vec{x})P$.

% subsection the_syntax_and_semantics_of_the_notation_system (end)   

\input{qm2pi.qmops} 

\input{qm2pi.sterngerlach} 

\input{qm2pi.metric} 

% section concurrent_process_calculi (end)

%\input{qm2pi.proofsketch}

% section proof sketch (end)

%\input{qm2pi.slviaknots} 

% section spatial logic via knots (end)

\input{qm2pi.conclusion}

% section conclusion (end)

%\input{qm2pi.dtcodes} 

% section wiring algorithm (end)

\input{qm2pi.ack} 

% section acknowledgments (end)

\newpage


\bibliographystyle{plain}   
\bibliography{../../biblios/main.bib}

\input{qm2pi.rhodetails}

\end{document}

 

% section notation (end)

\input{qm2pi.process.calculi} 

% section concurrent_process_calculi_and_spatial_logics_ (end)
    
%\documentclass[12pt]{llncs}
%\documentclass{jktr}

\usepackage[pdftex]{hyperref}                   
\usepackage {listings}
\usepackage {mathpartir}
\usepackage{bcprules}
%\usepackage{listings}
                       
\usepackage{graphicx} 
%\usepackage[margins=2.5cm,nohead,nofoot]{geometry}
%\usepackage{geometry}
\usepackage{amsfonts}
\usepackage{amstext}
\usepackage{latexsym}
\usepackage{amssymb}
\usepackage{color}


%\include{myPreamble}
\include{qm2pi.local} 

%\ifpdf
%\usepackage[pdftex]{graphicx}
%\else
%\usepackage{graphicx}
%\fi

 % \ifpdf
%  \usepackage{pdfsync}
%  \if


%\title{Brief Article}
%\author{David F. Snyder}
%\author{L.G. Meredith}

%\address{Dept. of Math., Texas State University--San Marcos, San Marcos, TX 78666}
       
\pagestyle{empty}


\begin{document}

\lstset{language=[Objective]Caml,frame=shadowbox}

\input{qm2pi.front}

% section front matter (end)

\input{qm2pi.intro} 
 
% section introduction (end)

% \input{qm2pi.knotations} 

% section notation (end)

\input{qm2pi.process.calculi} 

% section concurrent_process_calculi_and_spatial_logics_ (end)
    
%\input{qm2pi.knots2pi} 

%\input{qm2pi.trefoil} 

%\input{qm2pi.mainthm} 

% subsection basic_interpretation (end)

%\input{qm2pi.rho.presentation} 
\subsection{The syntax and semantics of the notation system}\label{sub:the_syntax_and_semantics_of_the_notation_system} % (fold)

We now summarize a technical presentation of the calculus that
embodies our theory of dynamics. The typical presentation of such a
calculus follows the style of giving generators and relations on
them. The grammar, below, describing term constructors, freely
generates the set of processes, $\Proc$. This set is then quotiented
by a relation known as structural congruence and it is over this set
that the notion of dynamics is expressed. This presentation is
essentially that of \cite{MeredithR05} with the addition of
polyadicity and summation. For readability we have relegated some of
the technical subtleties to an appendix.

\subsubsection{Process grammar}\label{subsub:process_grammar}

\begin{mathpar}
  \inferrule* [lab=synchronization] {} {{M} \bc \pzero \;|\; x?F \;|\; x!C }
  \and
  \inferrule* [lab=abstraction] {} {{F} \bc (x)P}
  \and
  \inferrule* [lab=concretion] {} {{C} \bc \langle Q \rangle}
  \and
  \inferrule* [lab=process] {} {{P,Q} \bc M \;| \;P|Q \;|\; @{x}}
  \and
  \inferrule* [lab=name] {} {{x} \bc \quotep{P}}
\end{mathpar} 

Note that $\vec{x}$ (resp. $\vec{P}$) denotes a vector of names
(resp. processes) of length $|\vec{x}|$ (resp. $|\vec{P}|$). We adopt
the following useful abbreviations.

\begin{mathpar}
   x?(\vec{y}).P := x.(\vec{y})P \and  x\clift{\vec{P}} := x.\clift{\vec{P}}
   \and x!(y) := \lift{x}{\dropn{y}}
   \and \Pi_{i=0}^{n-1}P_i := P_0 | \ldots | P_{n-1}
\end{mathpar}

\subsubsection{Structural congruence}

\paragraph{Free and bound names and alpha-equivalence.} At the
core of structural equivalence is alpha-equivalence which identifies
process that are the same up to a change of variable. Formally, we
recognize the distinction between free and bound names. The free names
of a process, $\freenames{P}$, may be calculated recursively as
follows:

\begin{mathpar}
\freenames{\pzero} := \emptyset
  \and \\
  \freenames{x?(y).P} := \{ x \} \cup (\freenames{P} \setminus \{ y \})
  \and 
  \freenames{x!\langle P \rangle} := \{ x \} \cup \{ P \} 
  \and \\
  \freenames{P|Q} := \freenames{P} \cup \freenames{Q}
  \and \\
  \freenames{@{x}} := \{ x \}
\end{mathpar}

$\pi$
$\quotep{\pi}$

$\freenames{-} : \pi \to \mathcal{P}(\quotep{\pi})$

\begin{eqnarray*}
  \freenames{\pzero} & := & \emptyset \\
  \freenames{x?(y).P} & := & \{ x \} \cup (\freenames{P} \setminus \{ y \}) \\
  \freenames{x!\langle P \rangle} & := & \{ x \} \cup \{ P \} \\
  \freenames{P|Q} & := & \freenames{P} \cup \freenames{Q} \\
  \freenames{\dropn{x}} & := & \{ x \}
\end{eqnarray*}

The bound names of a process, $\boundnames{P}$, are those names occurring in $P$
that are not free. For example, in $x?(y).0$, the name $x$ is free, while $y$ is bound.

\begin{mathpar}
  \inferrule* [lab=monoidal-laws] {} { P|Q \equiv Q|P \and P|0 \equiv P \and P|(Q|R) \equiv (P|Q)|R }
\end{mathpar}

\begin{mathpar}
  \inferrule* [lab=alpha-equivalence] {} { (x)P \equiv (y)P\{y/x\} \and y \not\in \freenames{P} }
\end{mathpar}

\begin{definition}
Then two processes, $P,Q$, are alpha-equivalent if $P = Q\{\vec{y}/\vec{x}\}$ for
some $\vec{x} \in \boundnames{Q},\vec{y} \in \boundnames{P}$, where $Q\{\vec{y}/\vec{x}\}$
denotes the capture-avoiding substitution of $\vec{y}$ for $\vec{x}$ in $Q$.
\end{definition}

\begin{definition}
  The {\em structural congruence} \cite{SangiorgiWalker} , $\equiv$,
  between processes is the least congruence containing
  alpha-equivalence, satisfying the abelian monoid laws
  (associativity, commutativity and $\pzero$ as identity) for parallel
  composition $|$ and for summation $+$.
\end{definition}

\subsection{Name equivalence}

We take name equivalence, written $\nameeq$, to be the smallest
equivalence relation generated by the following rules.

\begin{mathpar}
\inferrule*[lab=Quote-drop]
{ }
{ \quotep{@{x}} \nameeq x }

\inferrule*[lab=Struct-equiv]
{ P \scong Q }
{ \quotep{P} \nameeq \quotep{Q} }
\end{mathpar}

The astute reader will have noticed that the mutual recursion of names
and processes imposes a mutual recursion on alpha-equivalence and
structural equivalence via name-equivalence. Fortunately, all of this
works out pleasantly and we may calculate in the natural way, free of
concern. The reader interested in the details is referred to the
appendix \ref{appendix:rho_details}.

\subsection{Substitution}

We use $\Proc$ for the set of processes, $\QProc$ for the set of
names, and $\id{\{}\vec{y} / \vec{x} \id{\}}$ to denote partial maps,
$s : \QProc \rightarrow \QProc$. A map, $s$ lifts, uniquely, to a map
on process terms, $\widehat{s} : \Proc \rightarrow \Proc$ by the
following equations.

\begin{mathpar}
  (0) \psubstp{Q}{P} := 0 \\
  (R \juxtap S) \psubstp{Q}{P}
  :=    
  (R)\psubstp{Q}{P} \juxtap (S) \psubstp{Q}{P} \\
  (x?(y).R) \psubstp{Q}{P}    
  :=    
  (x)\substp{Q}{P} (z)\concat( (R \psubstn{z}{y}) \psubstp{Q}{P} ) \\
  (\lift{x}{R}) \psubstp{Q}{P}  
  :=
  \lift{(x)\substp{Q}{P}}{ R \psubstp{Q}{P} } \\
%   (\dropn{x})  \psubstp{Q}{P}       
%   := 
%   \left\{ 
%     \begin{array}{ccc} 
%       \dropn{\quotep{Q}} & & x \nameeq \quotep{P} \\
%       \dropn{x} & & otherwise \\
%     \end{array}
%   \right. 
  (\dropn{x})  \psubstp{Q}{P}       
  := 
  \left\{ 
    \begin{array}{ccc} 
      Q & & x \nameeq \quotep{P} \\
      \dropn{x} & & otherwise \\
    \end{array}
  \right.
\end{mathpar}
 

where

\begin{eqnarray}
  (x)\id{\{} \lpquote Q \rpquote / \lpquote P \rpquote \id{\}}            = 
  \left\{ 
    \begin{array}{ccc}
      \lpquote Q \rpquote & & x \nameeq \lpquote P \rpquote \\
      x & & otherwise \\
    \end{array}
  \right. \nonumber
\end{eqnarray}

and $z$ is chosen distinct from $\quotep{P}$, $\quotep{Q}$, the free
names in $Q$, and all the names in $R$. Our $\alpha$-equivalence will
be built in the standard way from this substitution.

\begin{remark}\label{rem:no_self_referential_names}
  One consequence of these definitions is that $\forall P. \quotep{P}
  \not\in \freenames{P}$.
\end{remark}

\subsection{ Dynamic quote: an example }

Anticipating something of what's to come, consider applying the
substitution, $\widehat{\id{\{}u / z \id{\}}}$, to the following pair
of processes, $\lift{w}{y!(z)}$ and $w[ \lpquote y!(z) \rpquote ]$.

\begin{eqnarray}
	\lift{w}{y!(z)}\widehat{\id{\{}u / z \id{\}}}
		& = &
		\lift{w}{y!(u)} \nonumber\\
	w[ \lpquote y!(z) \rpquote ] \widehat{ \id{\{}u / z \id{\}} }
		& = &
		w[ \lpquote y!(z) \rpquote ] \nonumber
\end{eqnarray}

Because the body of the process between quotes is impervious to
substitution, we get radically different answers. In fact, by
examining the first process in an input context,
e.g. $x?(z).\lift{w}{y!(z)}$, we see that the process under the lift
operator may be shaped by prefixed inputs binding a name inside it. In
this sense, the lift operator will be seen as a way to dynamically
construct processes before reifying them as names.

Finally equipped with these standard features we can present the
dynamics of the calculus.

\subsubsection{Operational semantics} 

Finally, we introduce the computational dynamics. What marks these
algebras as distinct from other more traditionally studied algebraic
structures, e.g. vector spaces or polynomial rings, is the manner in
which dynamics is captured. In traditional structures, dynamics is typically
expressed through morphisms between such structures, as in linear maps
between vector spaces or morphisms between rings. In algebras
associated with the semantics of computation, the dynamics is
expressed as part of the algebraic structure itself, through a
reduction reduction relation typically denoted by $\red$. Below, we
give a recursive presentation of this relation for the calculus used
in the encoding.

$\red \subseteq \pi \times \pi$
$\red : \pi \to \mathcal{P}(\pi)$

\begin{mathpar}
  \inferrule* [lab=Comm] { \textsf{match}( x_{src}, x_{trgt} ) } { x_{trgt}?(y)P \; | \; x_{src}!\langle {Q} \rangle \red P\{\quotep{Q}/y}\} }
  \and \\
  \inferrule* [lab=Par] {{P} \red {P}'} {{{P} | {Q}} \red {{P}' | {Q}}}
  \and
  \inferrule* [lab=Equiv]{{{P} \scong {P}'} \andalso {{P}' \red {Q}'} \andalso {{Q}' \scong {Q}}}{{P} \red {Q}}
\end{mathpar}

\begin{eqnarray*}
  match_{\equiv} (\quotep{P},\quotep{Q}) & := & P \equiv Q \\
  match_{\dagger}(\quotep{P},\quotep{Q}) & := & \forall R. P|Q \red^{*} R => R \red^{*} 0 \\
  match_{K}(\quotep{P},\quotep{Q}) & := & K \mbox{ for some context } K
\end{eqnarray*}

$u?(x)P | u!\langle Q \rangle \red P\{\quotep{Q}/x\}$

%We write $\wred$ for $\red^*$, and $P\red$ if $\exists Q $ such that $ P \red Q$.
We write $P\red$ if $\exists Q $ such that $ P \red Q$ and $P\not\red$, otherwise.

\section{Replication}

As mentioned before, it is known that replication (and hence
recursion) can be implemented in a higher-order process algebra
\cite{SangiorgiWalker}. As our first example of calculation with the
machinery thus far presented we give the construction explicitly in
the {\rhoc}.

\begin{eqnarray}
	D_{x} & := & \prefix{x}{y}{(\binpar{\outputp{x}{y}}{@{y}})} \nonumber\\
	\bangp_{x}{P} & := & \binpar{{x}!\langle{\binpar{D_{x}}{P}}\rangle}{D_{x}} \nonumber
\end{eqnarray}

\begin{eqnarray}
	\bangp_{x}{P} & & \nonumber\\
	=
	& {x}!\langle{(\prefix{x}{y}{(\outputp{x}{y} | @{y})) | P}}\rangle 
	      | \prefix{x}{y}{(\outputp{x}{y} | @{y})} & \nonumber\\
	\red
	& (\outputp{x}{y} | @{y})\substn{\quotep{(\prefix{x}{y}{(@{y} | \outputp{x}{y})) | P}}}{y} & \nonumber\\
	=
	& \outputp{x}{\quotep{(\prefix{x}{y}{(\outputp{x}{y} | @{y})) | P}}}
	  | {(\prefix{x}{y}{(\outputp{x}{y} | @{y})) | P}} & \nonumber\\
	\red
	& \ldots & \nonumber\\
	\red^*
	& P | P | \ldots & \nonumber
\end{eqnarray}

Of course, this encoding, as an implementation, runs away, unfolding
$\bangp{P}$ eagerly. A lazier and more implementable replication
operator, restricted to input-guarded processes, may be obtained as follows.

\begin{eqnarray}
\bangp{\prefix{u}{v}{P}} 
	:= 
	\binpar{\lift{x}{\prefix{u}{v}{(\binpar{D(x)}{P})}}}{D(x)} \nonumber
\end{eqnarray}

\begin{remark}
  Note that the lazier definition still does not deal with summation
  or mixed summation (i.e. sums over input and output). The reader is
  invited to construct definitions of replication that deal with these
  features. 

  Further, the definitions are parameterized in a name, $x$. Can you,
  gentle reader, make a definition that eliminates this parameter and
  guarantees no accidental interaction between the replication
  machinery and the process being replicated -- i.e. no accidental
  sharing of names used by the process to get its work done and the
  name(s) used by the replication to effect copying. This latter
  revision of the definition of replication is crucial to obtaining
  the expected identity $!!P \sim !P$.
\end{remark}

\begin{remark}\label{rem:paradoxical_combinator}
  The reader familiar with the lambda calculus will have noticed the
  similarity between $D$ and the paradoxical combinator.

  [Ed. note: the existence of this seems to suggest we have to be more
  restrictive on the set of processes and names we admit if we are to
  support no-cloning.]
\end{remark}

\subsubsection{Bisimulation}

The computational dynamics gives rise to another kind of equivalence,
the equivalence of computational behavior. As previously mentioned
this is typically captured \emph{via} some form of bisimulation.

% The notion we use in this paper is weak barbed bisimulation
% \cite{milner91polyadicpi}.

The notion we use in this paper is derived from weak barbed
bisimulation \cite{milner91polyadicpi}. 

\begin{definition}
An \emph{observation relation}, $\downarrow_{\mathcal N}$, over a set
of names, $\mathcal N$, is the smallest relation satisfying the rules
below.

\infrule[Out-barb]{y \in {\mathcal N}, \; x \nameeq y}
		  {\outputp{x}{v} \downarrow_{\mathcal N} x}
\infrule[Par-barb]{\mbox{$P\downarrow_{\mathcal N} x$ or $Q\downarrow_{\mathcal N} x$}}
		  {\binpar{P}{Q} \downarrow_{\mathcal N} x}

We write $P \Downarrow_{\mathcal N} x$ if there is $Q$ such that 
$P \wred Q$ and $Q \downarrow_{\mathcal N} x$.
\end{definition}

\begin{definition}
%\label{def.bbisim}
An  ${\mathcal N}$-\emph{barbed bisimulation} over a set of names, ${\mathcal N}$, is a symmetric binary relation 
${\mathcal S}_{\mathcal N}$ between agents such that $P\rel{S}_{\mathcal N}Q$ implies:
\begin{enumerate}
\item If $P \red P'$ then $Q \wred Q'$ and $P'\rel{S}_{\mathcal N} Q'$.
\item If $P\downarrow_{\mathcal N} x$, then $Q\Downarrow_{\mathcal N} x$.
\end{enumerate}
$P$ is ${\mathcal N}$-barbed bisimilar to $Q$, written
$P \wbbisim_{\mathcal N} Q$, if $P \rel{S}_{\mathcal N} Q$ for some ${\mathcal N}$-barbed bisimulation ${\mathcal S}_{\mathcal N}$.
\end{definition}

$\mathcal{R} \subseteq \pi \times \pi$

$P \mathcal{R} Q => \forall P'. P \red P' \Rightarrow \exists Q'. Q \red Q', P' \mathcal{R} Q'$

$P \vdash x \Rightarrow Q \vdash x$

\begin{mathpar}
  \inferrule*[lab=Out-barb]{x \nameeq y}{{y}!\langle{Q}\rangle \vdash x}
  \and
  \inferrule*[lab=Par-barb]{\mbox{$P\vdash x$ or $Q\vdash x$}}{\binpar{P}{Q} \vdash x}
\end{mathpar}

\subsubsection{Contexts}

One of the principle advantages of computational calculi like the
$\pi$-calculus is a well-defined notion of context,
contextual-equivalence and a correlation between
contextual-equivalence and notions of bisimulation. The notion of
context allows the decomposition of a process into (sub-)process and
its syntactic environment, its context. Thus, a context may be
thought of as a process with a ``hole'' (written $\Box$) in it. The
application of a context $M$ to a process $P$, written $M[P]$, is
tantamount to filling the hole in $M$ with $P$. In this paper we do
not need the full weight of this theory, but do make use of the notion
of context in the proof the main theorem. 

\begin{mathpar}
  \inferrule* [lab=summation] {} {{M_{M},M_{N}} \bc \Box \;|\; x.M_{A} \;|\; M_{M}+M_{N}}
  \and
  \inferrule* [lab=agent] {} {{M_{A}} \bc (\vec{x})M_{P} \;| \; \clift{P_0,\ldots,M_{P},\ldots,P_N}}
  \and \\
  \inferrule* [lab=process] {} {{M_{P}} \bc M_{N} \;| \;P|M_{P} }
\end{mathpar} 

\begin{mathpar}
  \inferrule* [lab=sychronization] {} {M_{N} \bc \Box \;|\; x?M_{F} \;|\; x!M_{C}}
  \and
  \inferrule* [lab=abstraction] {} {{M_{F}} \bc (x)M_{P} }
  \and
  \inferrule* [lab=concretion] {} {{M_{C}} \bc \langle M_{P} \rangle }
  \and \\
  \inferrule* [lab=process] {} {{M_{P}} \bc M_{N} \;| \;P|M_{P} }
\end{mathpar}

\begin{definition}[contextual application] Given a context $M$, and
  process $P$, we define the \emph{contextual application}, $M[P] :=
  M\{P/\Box\}$. That is, the contextual application of M to P is the
  substitution of $P$ for $\Box$ in $M$.
\end{definition}

$\meaningof{-} : L \to \mathcal{P}(\pi)$

\begin{mathpar}
  \inferrule* [lab=collection] {} {\meaningof{true} = \pi, \and \meaningof{~E} = \pi \setminus \meaningof{E}, \and \meaningof{E_{1} \& E_{2}} = \meaningof{E_{1}} \cap \meaningof{E_{2}}}
\end{mathpar}

\begin{mathpar}
  \inferrule* [lab=structure] {} {\meaningof{0} = \{ P \in \pi | P \equiv 0 \}, \and \\ \meaningof{E_1 | E_2} = \{ P \in \pi | P \equiv P_{1} | P_{2}, P_{1} \in \meaningof{E_{1}}, P_{2} \in \meaningof{E_2}\} }
\end{mathpar}

\begin{mathpar}
 \inferrule* [lab=behavior] {} {\meaningof{\langle a?b \rangle E} = \{ P \in \pi | P \equiv Q | u?(y)P', \\ \and \\\\ \and \\ \;\;\; u \in \meaningof{a}, \forall z.P'\{z/y\} \in \meaningof{E\{z/b\}}\}, \and \\ \meaningof{a!E} = \{ P \in \pi | P \equiv Q | x!\langle P' \rangle, x \in \meaningof{a} P' \in \meaningof{E}\} }
\end{mathpar}

\begin{mathpar}
 \inferrule* [lab=nominal] {} {\meaningof{\quotep{E}} = \{ \quotep{P} \in \quotep{\pi} | P \in \meaningof{E} \}, \and \meaningof{\quotep{P}} = \{ \quotep{Q} \in \quotep{\pi} | P \equiv Q \} \and \\ \meaningof{@\quotep{E}} = \{ P \in \pi | P \equiv @x, x \in \meaningof{E} \}}
\end{mathpar}

\begin{eqnarray*}
  \\
  \meaningof{-} : TS \to ST
\end{eqnarray*}

\begin{eqnarray*}
  \\
  L : TS \to ST
\end{eqnarray*}

\begin{eqnarray*}
  \\
  P \models E \iff P \in \meaningof{E}
\end{eqnarray*}

\begin{eqnarray*}
  P \approx_{L} Q \iff \forall E \in L. P \models E \iff Q \models E
\end{eqnarray*}

\begin{eqnarray*}
  P \approx_{K} Q
\end{eqnarray*}

\begin{eqnarray*}
  P \approx Q
\end{eqnarray*}

$\approx_{K} = \approx = \approx_{L}$

\subsubsection{Contextual duality}

Note that contexts extend the quotation operation to a family of
operations from processes to names. Given a context, $M$, we can
define a \emph{nominal context}, $\quotep{M}$ by $\quotep{M}[P] :=
\quotep{M[P]}$. To foreshadow what is to come we observe that these
operations enjoy a duality with processes very much like the duality
between vectors and maps from vectors to scalars.

Further, because the calculus is essentially higher-order, we have a
correspondence between contexts and processes. More specifically,
given a name $x$ and a context $M$ we can construct $M^{*}_{x}$ such
that 

\begin{mathpar}
  M^{*}_{x} | \lift{x}{P} \red M[P]
\end{mathpar}

namely,

\begin{mathpar}
  M^{*}_{x} := x?(u).M[\dropn{u}]
\end{mathpar}

The dependence of $M^{*}_{x}$ on a name makes it an abstraction, 

\begin{mathpar}
  M^{*} := (x)x?(u).M[\dropn{u}]
\end{mathpar}

\subsection{Additional notation}

It will sometimes be convenient to denote the process a name
quotes. We already have the notation $x = \quotep{P}$, but it will be
convenient to introduce an alternate notation, $\procn{x}$, when we
want to emphasize the connection to the use of the name. Note that, by
virtue of name equivalence, $\quotep{\procn{x}} \nameeq x$; so, the
notation is consistent with previous definitions.

Further, because names have structure it is possible to effect
substitutions on the basis of that structure. This means we need to
upgrade our notation for substitutions, which we accomplish by
adapting comprehension notation. Thus,

\begin{mathpar}
  P\{ y / x : x \in S \}
\end{mathpar}

is interpreted to mean the process derived from P by replacing (in a
capture-avoiding manner) each occurrence of $x$ in $S$ by $y$. For example,

\begin{mathpar}
  P\{ \quotep{\procn{x}|\procn{x}} / x : x \in \freenames{P} \}
\end{mathpar}

will replace each (occurrence) of a free name $x$ in $P$ by
$\quotep{\procn{x}|\procn{x}}$.

Also, we will avail ourselves of the notation $x^{L}$ and $x^{R}$ to
denote injections of a name into disjoint copies of the name
space. There are numerous ways to accomplish this. One example can be
found in \cite{MeredithR05}. This notation overloads to vectors of
names: $\vec{x}^{\pi} := (x_{i}^{\pi} \; : \; 0 \leq i < |\vec{x}| )$ where $\pi \in \{L,R\}$.

We also use $P^{\Box} := P|\Box$.

In \cite{MeredithR05} an interpretation of the new operator is
given. It turns out that there are several possible interpretations
all enjoying the requisite algebraic properties of the operator (see
\cite{milner91polyadicpi}). We will therefore make liberal use of
$(\nu\; \vec{x})P$.

% subsection the_syntax_and_semantics_of_the_notation_system (end)   

\input{qm2pi.qmops} 

\input{qm2pi.sterngerlach} 

\input{qm2pi.metric} 

% section concurrent_process_calculi (end)

%\input{qm2pi.proofsketch}

% section proof sketch (end)

%\input{qm2pi.slviaknots} 

% section spatial logic via knots (end)

\input{qm2pi.conclusion}

% section conclusion (end)

%\input{qm2pi.dtcodes} 

% section wiring algorithm (end)

\input{qm2pi.ack} 

% section acknowledgments (end)

\newpage


\bibliographystyle{plain}   
\bibliography{../../biblios/main.bib}

\input{qm2pi.rhodetails}

\end{document}

 

%\documentclass[12pt]{llncs}
%\documentclass{jktr}

\usepackage[pdftex]{hyperref}                   
\usepackage {listings}
\usepackage {mathpartir}
\usepackage{bcprules}
%\usepackage{listings}
                       
\usepackage{graphicx} 
%\usepackage[margins=2.5cm,nohead,nofoot]{geometry}
%\usepackage{geometry}
\usepackage{amsfonts}
\usepackage{amstext}
\usepackage{latexsym}
\usepackage{amssymb}
\usepackage{color}


%\include{myPreamble}
\include{qm2pi.local} 

%\ifpdf
%\usepackage[pdftex]{graphicx}
%\else
%\usepackage{graphicx}
%\fi

 % \ifpdf
%  \usepackage{pdfsync}
%  \if


%\title{Brief Article}
%\author{David F. Snyder}
%\author{L.G. Meredith}

%\address{Dept. of Math., Texas State University--San Marcos, San Marcos, TX 78666}
       
\pagestyle{empty}


\begin{document}

\lstset{language=[Objective]Caml,frame=shadowbox}

\input{qm2pi.front}

% section front matter (end)

\input{qm2pi.intro} 
 
% section introduction (end)

% \input{qm2pi.knotations} 

% section notation (end)

\input{qm2pi.process.calculi} 

% section concurrent_process_calculi_and_spatial_logics_ (end)
    
%\input{qm2pi.knots2pi} 

%\input{qm2pi.trefoil} 

%\input{qm2pi.mainthm} 

% subsection basic_interpretation (end)

%\input{qm2pi.rho.presentation} 
\subsection{The syntax and semantics of the notation system}\label{sub:the_syntax_and_semantics_of_the_notation_system} % (fold)

We now summarize a technical presentation of the calculus that
embodies our theory of dynamics. The typical presentation of such a
calculus follows the style of giving generators and relations on
them. The grammar, below, describing term constructors, freely
generates the set of processes, $\Proc$. This set is then quotiented
by a relation known as structural congruence and it is over this set
that the notion of dynamics is expressed. This presentation is
essentially that of \cite{MeredithR05} with the addition of
polyadicity and summation. For readability we have relegated some of
the technical subtleties to an appendix.

\subsubsection{Process grammar}\label{subsub:process_grammar}

\begin{mathpar}
  \inferrule* [lab=synchronization] {} {{M} \bc \pzero \;|\; x?F \;|\; x!C }
  \and
  \inferrule* [lab=abstraction] {} {{F} \bc (x)P}
  \and
  \inferrule* [lab=concretion] {} {{C} \bc \langle Q \rangle}
  \and
  \inferrule* [lab=process] {} {{P,Q} \bc M \;| \;P|Q \;|\; @{x}}
  \and
  \inferrule* [lab=name] {} {{x} \bc \quotep{P}}
\end{mathpar} 

Note that $\vec{x}$ (resp. $\vec{P}$) denotes a vector of names
(resp. processes) of length $|\vec{x}|$ (resp. $|\vec{P}|$). We adopt
the following useful abbreviations.

\begin{mathpar}
   x?(\vec{y}).P := x.(\vec{y})P \and  x\clift{\vec{P}} := x.\clift{\vec{P}}
   \and x!(y) := \lift{x}{\dropn{y}}
   \and \Pi_{i=0}^{n-1}P_i := P_0 | \ldots | P_{n-1}
\end{mathpar}

\subsubsection{Structural congruence}

\paragraph{Free and bound names and alpha-equivalence.} At the
core of structural equivalence is alpha-equivalence which identifies
process that are the same up to a change of variable. Formally, we
recognize the distinction between free and bound names. The free names
of a process, $\freenames{P}$, may be calculated recursively as
follows:

\begin{mathpar}
\freenames{\pzero} := \emptyset
  \and \\
  \freenames{x?(y).P} := \{ x \} \cup (\freenames{P} \setminus \{ y \})
  \and 
  \freenames{x!\langle P \rangle} := \{ x \} \cup \{ P \} 
  \and \\
  \freenames{P|Q} := \freenames{P} \cup \freenames{Q}
  \and \\
  \freenames{@{x}} := \{ x \}
\end{mathpar}

$\pi$
$\quotep{\pi}$

$\freenames{-} : \pi \to \mathcal{P}(\quotep{\pi})$

\begin{eqnarray*}
  \freenames{\pzero} & := & \emptyset \\
  \freenames{x?(y).P} & := & \{ x \} \cup (\freenames{P} \setminus \{ y \}) \\
  \freenames{x!\langle P \rangle} & := & \{ x \} \cup \{ P \} \\
  \freenames{P|Q} & := & \freenames{P} \cup \freenames{Q} \\
  \freenames{\dropn{x}} & := & \{ x \}
\end{eqnarray*}

The bound names of a process, $\boundnames{P}$, are those names occurring in $P$
that are not free. For example, in $x?(y).0$, the name $x$ is free, while $y$ is bound.

\begin{mathpar}
  \inferrule* [lab=monoidal-laws] {} { P|Q \equiv Q|P \and P|0 \equiv P \and P|(Q|R) \equiv (P|Q)|R }
\end{mathpar}

\begin{mathpar}
  \inferrule* [lab=alpha-equivalence] {} { (x)P \equiv (y)P\{y/x\} \and y \not\in \freenames{P} }
\end{mathpar}

\begin{definition}
Then two processes, $P,Q$, are alpha-equivalent if $P = Q\{\vec{y}/\vec{x}\}$ for
some $\vec{x} \in \boundnames{Q},\vec{y} \in \boundnames{P}$, where $Q\{\vec{y}/\vec{x}\}$
denotes the capture-avoiding substitution of $\vec{y}$ for $\vec{x}$ in $Q$.
\end{definition}

\begin{definition}
  The {\em structural congruence} \cite{SangiorgiWalker} , $\equiv$,
  between processes is the least congruence containing
  alpha-equivalence, satisfying the abelian monoid laws
  (associativity, commutativity and $\pzero$ as identity) for parallel
  composition $|$ and for summation $+$.
\end{definition}

\subsection{Name equivalence}

We take name equivalence, written $\nameeq$, to be the smallest
equivalence relation generated by the following rules.

\begin{mathpar}
\inferrule*[lab=Quote-drop]
{ }
{ \quotep{@{x}} \nameeq x }

\inferrule*[lab=Struct-equiv]
{ P \scong Q }
{ \quotep{P} \nameeq \quotep{Q} }
\end{mathpar}

The astute reader will have noticed that the mutual recursion of names
and processes imposes a mutual recursion on alpha-equivalence and
structural equivalence via name-equivalence. Fortunately, all of this
works out pleasantly and we may calculate in the natural way, free of
concern. The reader interested in the details is referred to the
appendix \ref{appendix:rho_details}.

\subsection{Substitution}

We use $\Proc$ for the set of processes, $\QProc$ for the set of
names, and $\id{\{}\vec{y} / \vec{x} \id{\}}$ to denote partial maps,
$s : \QProc \rightarrow \QProc$. A map, $s$ lifts, uniquely, to a map
on process terms, $\widehat{s} : \Proc \rightarrow \Proc$ by the
following equations.

\begin{mathpar}
  (0) \psubstp{Q}{P} := 0 \\
  (R \juxtap S) \psubstp{Q}{P}
  :=    
  (R)\psubstp{Q}{P} \juxtap (S) \psubstp{Q}{P} \\
  (x?(y).R) \psubstp{Q}{P}    
  :=    
  (x)\substp{Q}{P} (z)\concat( (R \psubstn{z}{y}) \psubstp{Q}{P} ) \\
  (\lift{x}{R}) \psubstp{Q}{P}  
  :=
  \lift{(x)\substp{Q}{P}}{ R \psubstp{Q}{P} } \\
%   (\dropn{x})  \psubstp{Q}{P}       
%   := 
%   \left\{ 
%     \begin{array}{ccc} 
%       \dropn{\quotep{Q}} & & x \nameeq \quotep{P} \\
%       \dropn{x} & & otherwise \\
%     \end{array}
%   \right. 
  (\dropn{x})  \psubstp{Q}{P}       
  := 
  \left\{ 
    \begin{array}{ccc} 
      Q & & x \nameeq \quotep{P} \\
      \dropn{x} & & otherwise \\
    \end{array}
  \right.
\end{mathpar}
 

where

\begin{eqnarray}
  (x)\id{\{} \lpquote Q \rpquote / \lpquote P \rpquote \id{\}}            = 
  \left\{ 
    \begin{array}{ccc}
      \lpquote Q \rpquote & & x \nameeq \lpquote P \rpquote \\
      x & & otherwise \\
    \end{array}
  \right. \nonumber
\end{eqnarray}

and $z$ is chosen distinct from $\quotep{P}$, $\quotep{Q}$, the free
names in $Q$, and all the names in $R$. Our $\alpha$-equivalence will
be built in the standard way from this substitution.

\begin{remark}\label{rem:no_self_referential_names}
  One consequence of these definitions is that $\forall P. \quotep{P}
  \not\in \freenames{P}$.
\end{remark}

\subsection{ Dynamic quote: an example }

Anticipating something of what's to come, consider applying the
substitution, $\widehat{\id{\{}u / z \id{\}}}$, to the following pair
of processes, $\lift{w}{y!(z)}$ and $w[ \lpquote y!(z) \rpquote ]$.

\begin{eqnarray}
	\lift{w}{y!(z)}\widehat{\id{\{}u / z \id{\}}}
		& = &
		\lift{w}{y!(u)} \nonumber\\
	w[ \lpquote y!(z) \rpquote ] \widehat{ \id{\{}u / z \id{\}} }
		& = &
		w[ \lpquote y!(z) \rpquote ] \nonumber
\end{eqnarray}

Because the body of the process between quotes is impervious to
substitution, we get radically different answers. In fact, by
examining the first process in an input context,
e.g. $x?(z).\lift{w}{y!(z)}$, we see that the process under the lift
operator may be shaped by prefixed inputs binding a name inside it. In
this sense, the lift operator will be seen as a way to dynamically
construct processes before reifying them as names.

Finally equipped with these standard features we can present the
dynamics of the calculus.

\subsubsection{Operational semantics} 

Finally, we introduce the computational dynamics. What marks these
algebras as distinct from other more traditionally studied algebraic
structures, e.g. vector spaces or polynomial rings, is the manner in
which dynamics is captured. In traditional structures, dynamics is typically
expressed through morphisms between such structures, as in linear maps
between vector spaces or morphisms between rings. In algebras
associated with the semantics of computation, the dynamics is
expressed as part of the algebraic structure itself, through a
reduction reduction relation typically denoted by $\red$. Below, we
give a recursive presentation of this relation for the calculus used
in the encoding.

$\red \subseteq \pi \times \pi$
$\red : \pi \to \mathcal{P}(\pi)$

\begin{mathpar}
  \inferrule* [lab=Comm] { \textsf{match}( x_{src}, x_{trgt} ) } { x_{trgt}?(y)P \; | \; x_{src}!\langle {Q} \rangle \red P\{\quotep{Q}/y}\} }
  \and \\
  \inferrule* [lab=Par] {{P} \red {P}'} {{{P} | {Q}} \red {{P}' | {Q}}}
  \and
  \inferrule* [lab=Equiv]{{{P} \scong {P}'} \andalso {{P}' \red {Q}'} \andalso {{Q}' \scong {Q}}}{{P} \red {Q}}
\end{mathpar}

\begin{eqnarray*}
  match_{\equiv} (\quotep{P},\quotep{Q}) & := & P \equiv Q \\
  match_{\dagger}(\quotep{P},\quotep{Q}) & := & \forall R. P|Q \red^{*} R => R \red^{*} 0 \\
  match_{K}(\quotep{P},\quotep{Q}) & := & K \mbox{ for some context } K
\end{eqnarray*}

$u?(x)P | u!\langle Q \rangle \red P\{\quotep{Q}/x\}$

%We write $\wred$ for $\red^*$, and $P\red$ if $\exists Q $ such that $ P \red Q$.
We write $P\red$ if $\exists Q $ such that $ P \red Q$ and $P\not\red$, otherwise.

\section{Replication}

As mentioned before, it is known that replication (and hence
recursion) can be implemented in a higher-order process algebra
\cite{SangiorgiWalker}. As our first example of calculation with the
machinery thus far presented we give the construction explicitly in
the {\rhoc}.

\begin{eqnarray}
	D_{x} & := & \prefix{x}{y}{(\binpar{\outputp{x}{y}}{@{y}})} \nonumber\\
	\bangp_{x}{P} & := & \binpar{{x}!\langle{\binpar{D_{x}}{P}}\rangle}{D_{x}} \nonumber
\end{eqnarray}

\begin{eqnarray}
	\bangp_{x}{P} & & \nonumber\\
	=
	& {x}!\langle{(\prefix{x}{y}{(\outputp{x}{y} | @{y})) | P}}\rangle 
	      | \prefix{x}{y}{(\outputp{x}{y} | @{y})} & \nonumber\\
	\red
	& (\outputp{x}{y} | @{y})\substn{\quotep{(\prefix{x}{y}{(@{y} | \outputp{x}{y})) | P}}}{y} & \nonumber\\
	=
	& \outputp{x}{\quotep{(\prefix{x}{y}{(\outputp{x}{y} | @{y})) | P}}}
	  | {(\prefix{x}{y}{(\outputp{x}{y} | @{y})) | P}} & \nonumber\\
	\red
	& \ldots & \nonumber\\
	\red^*
	& P | P | \ldots & \nonumber
\end{eqnarray}

Of course, this encoding, as an implementation, runs away, unfolding
$\bangp{P}$ eagerly. A lazier and more implementable replication
operator, restricted to input-guarded processes, may be obtained as follows.

\begin{eqnarray}
\bangp{\prefix{u}{v}{P}} 
	:= 
	\binpar{\lift{x}{\prefix{u}{v}{(\binpar{D(x)}{P})}}}{D(x)} \nonumber
\end{eqnarray}

\begin{remark}
  Note that the lazier definition still does not deal with summation
  or mixed summation (i.e. sums over input and output). The reader is
  invited to construct definitions of replication that deal with these
  features. 

  Further, the definitions are parameterized in a name, $x$. Can you,
  gentle reader, make a definition that eliminates this parameter and
  guarantees no accidental interaction between the replication
  machinery and the process being replicated -- i.e. no accidental
  sharing of names used by the process to get its work done and the
  name(s) used by the replication to effect copying. This latter
  revision of the definition of replication is crucial to obtaining
  the expected identity $!!P \sim !P$.
\end{remark}

\begin{remark}\label{rem:paradoxical_combinator}
  The reader familiar with the lambda calculus will have noticed the
  similarity between $D$ and the paradoxical combinator.

  [Ed. note: the existence of this seems to suggest we have to be more
  restrictive on the set of processes and names we admit if we are to
  support no-cloning.]
\end{remark}

\subsubsection{Bisimulation}

The computational dynamics gives rise to another kind of equivalence,
the equivalence of computational behavior. As previously mentioned
this is typically captured \emph{via} some form of bisimulation.

% The notion we use in this paper is weak barbed bisimulation
% \cite{milner91polyadicpi}.

The notion we use in this paper is derived from weak barbed
bisimulation \cite{milner91polyadicpi}. 

\begin{definition}
An \emph{observation relation}, $\downarrow_{\mathcal N}$, over a set
of names, $\mathcal N$, is the smallest relation satisfying the rules
below.

\infrule[Out-barb]{y \in {\mathcal N}, \; x \nameeq y}
		  {\outputp{x}{v} \downarrow_{\mathcal N} x}
\infrule[Par-barb]{\mbox{$P\downarrow_{\mathcal N} x$ or $Q\downarrow_{\mathcal N} x$}}
		  {\binpar{P}{Q} \downarrow_{\mathcal N} x}

We write $P \Downarrow_{\mathcal N} x$ if there is $Q$ such that 
$P \wred Q$ and $Q \downarrow_{\mathcal N} x$.
\end{definition}

\begin{definition}
%\label{def.bbisim}
An  ${\mathcal N}$-\emph{barbed bisimulation} over a set of names, ${\mathcal N}$, is a symmetric binary relation 
${\mathcal S}_{\mathcal N}$ between agents such that $P\rel{S}_{\mathcal N}Q$ implies:
\begin{enumerate}
\item If $P \red P'$ then $Q \wred Q'$ and $P'\rel{S}_{\mathcal N} Q'$.
\item If $P\downarrow_{\mathcal N} x$, then $Q\Downarrow_{\mathcal N} x$.
\end{enumerate}
$P$ is ${\mathcal N}$-barbed bisimilar to $Q$, written
$P \wbbisim_{\mathcal N} Q$, if $P \rel{S}_{\mathcal N} Q$ for some ${\mathcal N}$-barbed bisimulation ${\mathcal S}_{\mathcal N}$.
\end{definition}

$\mathcal{R} \subseteq \pi \times \pi$

$P \mathcal{R} Q => \forall P'. P \red P' \Rightarrow \exists Q'. Q \red Q', P' \mathcal{R} Q'$

$P \vdash x \Rightarrow Q \vdash x$

\begin{mathpar}
  \inferrule*[lab=Out-barb]{x \nameeq y}{{y}!\langle{Q}\rangle \vdash x}
  \and
  \inferrule*[lab=Par-barb]{\mbox{$P\vdash x$ or $Q\vdash x$}}{\binpar{P}{Q} \vdash x}
\end{mathpar}

\subsubsection{Contexts}

One of the principle advantages of computational calculi like the
$\pi$-calculus is a well-defined notion of context,
contextual-equivalence and a correlation between
contextual-equivalence and notions of bisimulation. The notion of
context allows the decomposition of a process into (sub-)process and
its syntactic environment, its context. Thus, a context may be
thought of as a process with a ``hole'' (written $\Box$) in it. The
application of a context $M$ to a process $P$, written $M[P]$, is
tantamount to filling the hole in $M$ with $P$. In this paper we do
not need the full weight of this theory, but do make use of the notion
of context in the proof the main theorem. 

\begin{mathpar}
  \inferrule* [lab=summation] {} {{M_{M},M_{N}} \bc \Box \;|\; x.M_{A} \;|\; M_{M}+M_{N}}
  \and
  \inferrule* [lab=agent] {} {{M_{A}} \bc (\vec{x})M_{P} \;| \; \clift{P_0,\ldots,M_{P},\ldots,P_N}}
  \and \\
  \inferrule* [lab=process] {} {{M_{P}} \bc M_{N} \;| \;P|M_{P} }
\end{mathpar} 

\begin{mathpar}
  \inferrule* [lab=sychronization] {} {M_{N} \bc \Box \;|\; x?M_{F} \;|\; x!M_{C}}
  \and
  \inferrule* [lab=abstraction] {} {{M_{F}} \bc (x)M_{P} }
  \and
  \inferrule* [lab=concretion] {} {{M_{C}} \bc \langle M_{P} \rangle }
  \and \\
  \inferrule* [lab=process] {} {{M_{P}} \bc M_{N} \;| \;P|M_{P} }
\end{mathpar}

\begin{definition}[contextual application] Given a context $M$, and
  process $P$, we define the \emph{contextual application}, $M[P] :=
  M\{P/\Box\}$. That is, the contextual application of M to P is the
  substitution of $P$ for $\Box$ in $M$.
\end{definition}

$\meaningof{-} : L \to \mathcal{P}(\pi)$

\begin{mathpar}
  \inferrule* [lab=collection] {} {\meaningof{true} = \pi, \and \meaningof{~E} = \pi \setminus \meaningof{E}, \and \meaningof{E_{1} \& E_{2}} = \meaningof{E_{1}} \cap \meaningof{E_{2}}}
\end{mathpar}

\begin{mathpar}
  \inferrule* [lab=structure] {} {\meaningof{0} = \{ P \in \pi | P \equiv 0 \}, \and \\ \meaningof{E_1 | E_2} = \{ P \in \pi | P \equiv P_{1} | P_{2}, P_{1} \in \meaningof{E_{1}}, P_{2} \in \meaningof{E_2}\} }
\end{mathpar}

\begin{mathpar}
 \inferrule* [lab=behavior] {} {\meaningof{\langle a?b \rangle E} = \{ P \in \pi | P \equiv Q | u?(y)P', \\ \and \\\\ \and \\ \;\;\; u \in \meaningof{a}, \forall z.P'\{z/y\} \in \meaningof{E\{z/b\}}\}, \and \\ \meaningof{a!E} = \{ P \in \pi | P \equiv Q | x!\langle P' \rangle, x \in \meaningof{a} P' \in \meaningof{E}\} }
\end{mathpar}

\begin{mathpar}
 \inferrule* [lab=nominal] {} {\meaningof{\quotep{E}} = \{ \quotep{P} \in \quotep{\pi} | P \in \meaningof{E} \}, \and \meaningof{\quotep{P}} = \{ \quotep{Q} \in \quotep{\pi} | P \equiv Q \} \and \\ \meaningof{@\quotep{E}} = \{ P \in \pi | P \equiv @x, x \in \meaningof{E} \}}
\end{mathpar}

\begin{eqnarray*}
  \\
  \meaningof{-} : TS \to ST
\end{eqnarray*}

\begin{eqnarray*}
  \\
  L : TS \to ST
\end{eqnarray*}

\begin{eqnarray*}
  \\
  P \models E \iff P \in \meaningof{E}
\end{eqnarray*}

\begin{eqnarray*}
  P \approx_{L} Q \iff \forall E \in L. P \models E \iff Q \models E
\end{eqnarray*}

\begin{eqnarray*}
  P \approx_{K} Q
\end{eqnarray*}

\begin{eqnarray*}
  P \approx Q
\end{eqnarray*}

$\approx_{K} = \approx = \approx_{L}$

\subsubsection{Contextual duality}

Note that contexts extend the quotation operation to a family of
operations from processes to names. Given a context, $M$, we can
define a \emph{nominal context}, $\quotep{M}$ by $\quotep{M}[P] :=
\quotep{M[P]}$. To foreshadow what is to come we observe that these
operations enjoy a duality with processes very much like the duality
between vectors and maps from vectors to scalars.

Further, because the calculus is essentially higher-order, we have a
correspondence between contexts and processes. More specifically,
given a name $x$ and a context $M$ we can construct $M^{*}_{x}$ such
that 

\begin{mathpar}
  M^{*}_{x} | \lift{x}{P} \red M[P]
\end{mathpar}

namely,

\begin{mathpar}
  M^{*}_{x} := x?(u).M[\dropn{u}]
\end{mathpar}

The dependence of $M^{*}_{x}$ on a name makes it an abstraction, 

\begin{mathpar}
  M^{*} := (x)x?(u).M[\dropn{u}]
\end{mathpar}

\subsection{Additional notation}

It will sometimes be convenient to denote the process a name
quotes. We already have the notation $x = \quotep{P}$, but it will be
convenient to introduce an alternate notation, $\procn{x}$, when we
want to emphasize the connection to the use of the name. Note that, by
virtue of name equivalence, $\quotep{\procn{x}} \nameeq x$; so, the
notation is consistent with previous definitions.

Further, because names have structure it is possible to effect
substitutions on the basis of that structure. This means we need to
upgrade our notation for substitutions, which we accomplish by
adapting comprehension notation. Thus,

\begin{mathpar}
  P\{ y / x : x \in S \}
\end{mathpar}

is interpreted to mean the process derived from P by replacing (in a
capture-avoiding manner) each occurrence of $x$ in $S$ by $y$. For example,

\begin{mathpar}
  P\{ \quotep{\procn{x}|\procn{x}} / x : x \in \freenames{P} \}
\end{mathpar}

will replace each (occurrence) of a free name $x$ in $P$ by
$\quotep{\procn{x}|\procn{x}}$.

Also, we will avail ourselves of the notation $x^{L}$ and $x^{R}$ to
denote injections of a name into disjoint copies of the name
space. There are numerous ways to accomplish this. One example can be
found in \cite{MeredithR05}. This notation overloads to vectors of
names: $\vec{x}^{\pi} := (x_{i}^{\pi} \; : \; 0 \leq i < |\vec{x}| )$ where $\pi \in \{L,R\}$.

We also use $P^{\Box} := P|\Box$.

In \cite{MeredithR05} an interpretation of the new operator is
given. It turns out that there are several possible interpretations
all enjoying the requisite algebraic properties of the operator (see
\cite{milner91polyadicpi}). We will therefore make liberal use of
$(\nu\; \vec{x})P$.

% subsection the_syntax_and_semantics_of_the_notation_system (end)   

\input{qm2pi.qmops} 

\input{qm2pi.sterngerlach} 

\input{qm2pi.metric} 

% section concurrent_process_calculi (end)

%\input{qm2pi.proofsketch}

% section proof sketch (end)

%\input{qm2pi.slviaknots} 

% section spatial logic via knots (end)

\input{qm2pi.conclusion}

% section conclusion (end)

%\input{qm2pi.dtcodes} 

% section wiring algorithm (end)

\input{qm2pi.ack} 

% section acknowledgments (end)

\newpage


\bibliographystyle{plain}   
\bibliography{../../biblios/main.bib}

\input{qm2pi.rhodetails}

\end{document}

 

%\documentclass[12pt]{llncs}
%\documentclass{jktr}

\usepackage[pdftex]{hyperref}                   
\usepackage {listings}
\usepackage {mathpartir}
\usepackage{bcprules}
%\usepackage{listings}
                       
\usepackage{graphicx} 
%\usepackage[margins=2.5cm,nohead,nofoot]{geometry}
%\usepackage{geometry}
\usepackage{amsfonts}
\usepackage{amstext}
\usepackage{latexsym}
\usepackage{amssymb}
\usepackage{color}


%\include{myPreamble}
\include{qm2pi.local} 

%\ifpdf
%\usepackage[pdftex]{graphicx}
%\else
%\usepackage{graphicx}
%\fi

 % \ifpdf
%  \usepackage{pdfsync}
%  \if


%\title{Brief Article}
%\author{David F. Snyder}
%\author{L.G. Meredith}

%\address{Dept. of Math., Texas State University--San Marcos, San Marcos, TX 78666}
       
\pagestyle{empty}


\begin{document}

\lstset{language=[Objective]Caml,frame=shadowbox}

\input{qm2pi.front}

% section front matter (end)

\input{qm2pi.intro} 
 
% section introduction (end)

% \input{qm2pi.knotations} 

% section notation (end)

\input{qm2pi.process.calculi} 

% section concurrent_process_calculi_and_spatial_logics_ (end)
    
%\input{qm2pi.knots2pi} 

%\input{qm2pi.trefoil} 

%\input{qm2pi.mainthm} 

% subsection basic_interpretation (end)

%\input{qm2pi.rho.presentation} 
\subsection{The syntax and semantics of the notation system}\label{sub:the_syntax_and_semantics_of_the_notation_system} % (fold)

We now summarize a technical presentation of the calculus that
embodies our theory of dynamics. The typical presentation of such a
calculus follows the style of giving generators and relations on
them. The grammar, below, describing term constructors, freely
generates the set of processes, $\Proc$. This set is then quotiented
by a relation known as structural congruence and it is over this set
that the notion of dynamics is expressed. This presentation is
essentially that of \cite{MeredithR05} with the addition of
polyadicity and summation. For readability we have relegated some of
the technical subtleties to an appendix.

\subsubsection{Process grammar}\label{subsub:process_grammar}

\begin{mathpar}
  \inferrule* [lab=synchronization] {} {{M} \bc \pzero \;|\; x?F \;|\; x!C }
  \and
  \inferrule* [lab=abstraction] {} {{F} \bc (x)P}
  \and
  \inferrule* [lab=concretion] {} {{C} \bc \langle Q \rangle}
  \and
  \inferrule* [lab=process] {} {{P,Q} \bc M \;| \;P|Q \;|\; @{x}}
  \and
  \inferrule* [lab=name] {} {{x} \bc \quotep{P}}
\end{mathpar} 

Note that $\vec{x}$ (resp. $\vec{P}$) denotes a vector of names
(resp. processes) of length $|\vec{x}|$ (resp. $|\vec{P}|$). We adopt
the following useful abbreviations.

\begin{mathpar}
   x?(\vec{y}).P := x.(\vec{y})P \and  x\clift{\vec{P}} := x.\clift{\vec{P}}
   \and x!(y) := \lift{x}{\dropn{y}}
   \and \Pi_{i=0}^{n-1}P_i := P_0 | \ldots | P_{n-1}
\end{mathpar}

\subsubsection{Structural congruence}

\paragraph{Free and bound names and alpha-equivalence.} At the
core of structural equivalence is alpha-equivalence which identifies
process that are the same up to a change of variable. Formally, we
recognize the distinction between free and bound names. The free names
of a process, $\freenames{P}$, may be calculated recursively as
follows:

\begin{mathpar}
\freenames{\pzero} := \emptyset
  \and \\
  \freenames{x?(y).P} := \{ x \} \cup (\freenames{P} \setminus \{ y \})
  \and 
  \freenames{x!\langle P \rangle} := \{ x \} \cup \{ P \} 
  \and \\
  \freenames{P|Q} := \freenames{P} \cup \freenames{Q}
  \and \\
  \freenames{@{x}} := \{ x \}
\end{mathpar}

$\pi$
$\quotep{\pi}$

$\freenames{-} : \pi \to \mathcal{P}(\quotep{\pi})$

\begin{eqnarray*}
  \freenames{\pzero} & := & \emptyset \\
  \freenames{x?(y).P} & := & \{ x \} \cup (\freenames{P} \setminus \{ y \}) \\
  \freenames{x!\langle P \rangle} & := & \{ x \} \cup \{ P \} \\
  \freenames{P|Q} & := & \freenames{P} \cup \freenames{Q} \\
  \freenames{\dropn{x}} & := & \{ x \}
\end{eqnarray*}

The bound names of a process, $\boundnames{P}$, are those names occurring in $P$
that are not free. For example, in $x?(y).0$, the name $x$ is free, while $y$ is bound.

\begin{mathpar}
  \inferrule* [lab=monoidal-laws] {} { P|Q \equiv Q|P \and P|0 \equiv P \and P|(Q|R) \equiv (P|Q)|R }
\end{mathpar}

\begin{mathpar}
  \inferrule* [lab=alpha-equivalence] {} { (x)P \equiv (y)P\{y/x\} \and y \not\in \freenames{P} }
\end{mathpar}

\begin{definition}
Then two processes, $P,Q$, are alpha-equivalent if $P = Q\{\vec{y}/\vec{x}\}$ for
some $\vec{x} \in \boundnames{Q},\vec{y} \in \boundnames{P}$, where $Q\{\vec{y}/\vec{x}\}$
denotes the capture-avoiding substitution of $\vec{y}$ for $\vec{x}$ in $Q$.
\end{definition}

\begin{definition}
  The {\em structural congruence} \cite{SangiorgiWalker} , $\equiv$,
  between processes is the least congruence containing
  alpha-equivalence, satisfying the abelian monoid laws
  (associativity, commutativity and $\pzero$ as identity) for parallel
  composition $|$ and for summation $+$.
\end{definition}

\subsection{Name equivalence}

We take name equivalence, written $\nameeq$, to be the smallest
equivalence relation generated by the following rules.

\begin{mathpar}
\inferrule*[lab=Quote-drop]
{ }
{ \quotep{@{x}} \nameeq x }

\inferrule*[lab=Struct-equiv]
{ P \scong Q }
{ \quotep{P} \nameeq \quotep{Q} }
\end{mathpar}

The astute reader will have noticed that the mutual recursion of names
and processes imposes a mutual recursion on alpha-equivalence and
structural equivalence via name-equivalence. Fortunately, all of this
works out pleasantly and we may calculate in the natural way, free of
concern. The reader interested in the details is referred to the
appendix \ref{appendix:rho_details}.

\subsection{Substitution}

We use $\Proc$ for the set of processes, $\QProc$ for the set of
names, and $\id{\{}\vec{y} / \vec{x} \id{\}}$ to denote partial maps,
$s : \QProc \rightarrow \QProc$. A map, $s$ lifts, uniquely, to a map
on process terms, $\widehat{s} : \Proc \rightarrow \Proc$ by the
following equations.

\begin{mathpar}
  (0) \psubstp{Q}{P} := 0 \\
  (R \juxtap S) \psubstp{Q}{P}
  :=    
  (R)\psubstp{Q}{P} \juxtap (S) \psubstp{Q}{P} \\
  (x?(y).R) \psubstp{Q}{P}    
  :=    
  (x)\substp{Q}{P} (z)\concat( (R \psubstn{z}{y}) \psubstp{Q}{P} ) \\
  (\lift{x}{R}) \psubstp{Q}{P}  
  :=
  \lift{(x)\substp{Q}{P}}{ R \psubstp{Q}{P} } \\
%   (\dropn{x})  \psubstp{Q}{P}       
%   := 
%   \left\{ 
%     \begin{array}{ccc} 
%       \dropn{\quotep{Q}} & & x \nameeq \quotep{P} \\
%       \dropn{x} & & otherwise \\
%     \end{array}
%   \right. 
  (\dropn{x})  \psubstp{Q}{P}       
  := 
  \left\{ 
    \begin{array}{ccc} 
      Q & & x \nameeq \quotep{P} \\
      \dropn{x} & & otherwise \\
    \end{array}
  \right.
\end{mathpar}
 

where

\begin{eqnarray}
  (x)\id{\{} \lpquote Q \rpquote / \lpquote P \rpquote \id{\}}            = 
  \left\{ 
    \begin{array}{ccc}
      \lpquote Q \rpquote & & x \nameeq \lpquote P \rpquote \\
      x & & otherwise \\
    \end{array}
  \right. \nonumber
\end{eqnarray}

and $z$ is chosen distinct from $\quotep{P}$, $\quotep{Q}$, the free
names in $Q$, and all the names in $R$. Our $\alpha$-equivalence will
be built in the standard way from this substitution.

\begin{remark}\label{rem:no_self_referential_names}
  One consequence of these definitions is that $\forall P. \quotep{P}
  \not\in \freenames{P}$.
\end{remark}

\subsection{ Dynamic quote: an example }

Anticipating something of what's to come, consider applying the
substitution, $\widehat{\id{\{}u / z \id{\}}}$, to the following pair
of processes, $\lift{w}{y!(z)}$ and $w[ \lpquote y!(z) \rpquote ]$.

\begin{eqnarray}
	\lift{w}{y!(z)}\widehat{\id{\{}u / z \id{\}}}
		& = &
		\lift{w}{y!(u)} \nonumber\\
	w[ \lpquote y!(z) \rpquote ] \widehat{ \id{\{}u / z \id{\}} }
		& = &
		w[ \lpquote y!(z) \rpquote ] \nonumber
\end{eqnarray}

Because the body of the process between quotes is impervious to
substitution, we get radically different answers. In fact, by
examining the first process in an input context,
e.g. $x?(z).\lift{w}{y!(z)}$, we see that the process under the lift
operator may be shaped by prefixed inputs binding a name inside it. In
this sense, the lift operator will be seen as a way to dynamically
construct processes before reifying them as names.

Finally equipped with these standard features we can present the
dynamics of the calculus.

\subsubsection{Operational semantics} 

Finally, we introduce the computational dynamics. What marks these
algebras as distinct from other more traditionally studied algebraic
structures, e.g. vector spaces or polynomial rings, is the manner in
which dynamics is captured. In traditional structures, dynamics is typically
expressed through morphisms between such structures, as in linear maps
between vector spaces or morphisms between rings. In algebras
associated with the semantics of computation, the dynamics is
expressed as part of the algebraic structure itself, through a
reduction reduction relation typically denoted by $\red$. Below, we
give a recursive presentation of this relation for the calculus used
in the encoding.

$\red \subseteq \pi \times \pi$
$\red : \pi \to \mathcal{P}(\pi)$

\begin{mathpar}
  \inferrule* [lab=Comm] { \textsf{match}( x_{src}, x_{trgt} ) } { x_{trgt}?(y)P \; | \; x_{src}!\langle {Q} \rangle \red P\{\quotep{Q}/y}\} }
  \and \\
  \inferrule* [lab=Par] {{P} \red {P}'} {{{P} | {Q}} \red {{P}' | {Q}}}
  \and
  \inferrule* [lab=Equiv]{{{P} \scong {P}'} \andalso {{P}' \red {Q}'} \andalso {{Q}' \scong {Q}}}{{P} \red {Q}}
\end{mathpar}

\begin{eqnarray*}
  match_{\equiv} (\quotep{P},\quotep{Q}) & := & P \equiv Q \\
  match_{\dagger}(\quotep{P},\quotep{Q}) & := & \forall R. P|Q \red^{*} R => R \red^{*} 0 \\
  match_{K}(\quotep{P},\quotep{Q}) & := & K \mbox{ for some context } K
\end{eqnarray*}

$u?(x)P | u!\langle Q \rangle \red P\{\quotep{Q}/x\}$

%We write $\wred$ for $\red^*$, and $P\red$ if $\exists Q $ such that $ P \red Q$.
We write $P\red$ if $\exists Q $ such that $ P \red Q$ and $P\not\red$, otherwise.

\section{Replication}

As mentioned before, it is known that replication (and hence
recursion) can be implemented in a higher-order process algebra
\cite{SangiorgiWalker}. As our first example of calculation with the
machinery thus far presented we give the construction explicitly in
the {\rhoc}.

\begin{eqnarray}
	D_{x} & := & \prefix{x}{y}{(\binpar{\outputp{x}{y}}{@{y}})} \nonumber\\
	\bangp_{x}{P} & := & \binpar{{x}!\langle{\binpar{D_{x}}{P}}\rangle}{D_{x}} \nonumber
\end{eqnarray}

\begin{eqnarray}
	\bangp_{x}{P} & & \nonumber\\
	=
	& {x}!\langle{(\prefix{x}{y}{(\outputp{x}{y} | @{y})) | P}}\rangle 
	      | \prefix{x}{y}{(\outputp{x}{y} | @{y})} & \nonumber\\
	\red
	& (\outputp{x}{y} | @{y})\substn{\quotep{(\prefix{x}{y}{(@{y} | \outputp{x}{y})) | P}}}{y} & \nonumber\\
	=
	& \outputp{x}{\quotep{(\prefix{x}{y}{(\outputp{x}{y} | @{y})) | P}}}
	  | {(\prefix{x}{y}{(\outputp{x}{y} | @{y})) | P}} & \nonumber\\
	\red
	& \ldots & \nonumber\\
	\red^*
	& P | P | \ldots & \nonumber
\end{eqnarray}

Of course, this encoding, as an implementation, runs away, unfolding
$\bangp{P}$ eagerly. A lazier and more implementable replication
operator, restricted to input-guarded processes, may be obtained as follows.

\begin{eqnarray}
\bangp{\prefix{u}{v}{P}} 
	:= 
	\binpar{\lift{x}{\prefix{u}{v}{(\binpar{D(x)}{P})}}}{D(x)} \nonumber
\end{eqnarray}

\begin{remark}
  Note that the lazier definition still does not deal with summation
  or mixed summation (i.e. sums over input and output). The reader is
  invited to construct definitions of replication that deal with these
  features. 

  Further, the definitions are parameterized in a name, $x$. Can you,
  gentle reader, make a definition that eliminates this parameter and
  guarantees no accidental interaction between the replication
  machinery and the process being replicated -- i.e. no accidental
  sharing of names used by the process to get its work done and the
  name(s) used by the replication to effect copying. This latter
  revision of the definition of replication is crucial to obtaining
  the expected identity $!!P \sim !P$.
\end{remark}

\begin{remark}\label{rem:paradoxical_combinator}
  The reader familiar with the lambda calculus will have noticed the
  similarity between $D$ and the paradoxical combinator.

  [Ed. note: the existence of this seems to suggest we have to be more
  restrictive on the set of processes and names we admit if we are to
  support no-cloning.]
\end{remark}

\subsubsection{Bisimulation}

The computational dynamics gives rise to another kind of equivalence,
the equivalence of computational behavior. As previously mentioned
this is typically captured \emph{via} some form of bisimulation.

% The notion we use in this paper is weak barbed bisimulation
% \cite{milner91polyadicpi}.

The notion we use in this paper is derived from weak barbed
bisimulation \cite{milner91polyadicpi}. 

\begin{definition}
An \emph{observation relation}, $\downarrow_{\mathcal N}$, over a set
of names, $\mathcal N$, is the smallest relation satisfying the rules
below.

\infrule[Out-barb]{y \in {\mathcal N}, \; x \nameeq y}
		  {\outputp{x}{v} \downarrow_{\mathcal N} x}
\infrule[Par-barb]{\mbox{$P\downarrow_{\mathcal N} x$ or $Q\downarrow_{\mathcal N} x$}}
		  {\binpar{P}{Q} \downarrow_{\mathcal N} x}

We write $P \Downarrow_{\mathcal N} x$ if there is $Q$ such that 
$P \wred Q$ and $Q \downarrow_{\mathcal N} x$.
\end{definition}

\begin{definition}
%\label{def.bbisim}
An  ${\mathcal N}$-\emph{barbed bisimulation} over a set of names, ${\mathcal N}$, is a symmetric binary relation 
${\mathcal S}_{\mathcal N}$ between agents such that $P\rel{S}_{\mathcal N}Q$ implies:
\begin{enumerate}
\item If $P \red P'$ then $Q \wred Q'$ and $P'\rel{S}_{\mathcal N} Q'$.
\item If $P\downarrow_{\mathcal N} x$, then $Q\Downarrow_{\mathcal N} x$.
\end{enumerate}
$P$ is ${\mathcal N}$-barbed bisimilar to $Q$, written
$P \wbbisim_{\mathcal N} Q$, if $P \rel{S}_{\mathcal N} Q$ for some ${\mathcal N}$-barbed bisimulation ${\mathcal S}_{\mathcal N}$.
\end{definition}

$\mathcal{R} \subseteq \pi \times \pi$

$P \mathcal{R} Q => \forall P'. P \red P' \Rightarrow \exists Q'. Q \red Q', P' \mathcal{R} Q'$

$P \vdash x \Rightarrow Q \vdash x$

\begin{mathpar}
  \inferrule*[lab=Out-barb]{x \nameeq y}{{y}!\langle{Q}\rangle \vdash x}
  \and
  \inferrule*[lab=Par-barb]{\mbox{$P\vdash x$ or $Q\vdash x$}}{\binpar{P}{Q} \vdash x}
\end{mathpar}

\subsubsection{Contexts}

One of the principle advantages of computational calculi like the
$\pi$-calculus is a well-defined notion of context,
contextual-equivalence and a correlation between
contextual-equivalence and notions of bisimulation. The notion of
context allows the decomposition of a process into (sub-)process and
its syntactic environment, its context. Thus, a context may be
thought of as a process with a ``hole'' (written $\Box$) in it. The
application of a context $M$ to a process $P$, written $M[P]$, is
tantamount to filling the hole in $M$ with $P$. In this paper we do
not need the full weight of this theory, but do make use of the notion
of context in the proof the main theorem. 

\begin{mathpar}
  \inferrule* [lab=summation] {} {{M_{M},M_{N}} \bc \Box \;|\; x.M_{A} \;|\; M_{M}+M_{N}}
  \and
  \inferrule* [lab=agent] {} {{M_{A}} \bc (\vec{x})M_{P} \;| \; \clift{P_0,\ldots,M_{P},\ldots,P_N}}
  \and \\
  \inferrule* [lab=process] {} {{M_{P}} \bc M_{N} \;| \;P|M_{P} }
\end{mathpar} 

\begin{mathpar}
  \inferrule* [lab=sychronization] {} {M_{N} \bc \Box \;|\; x?M_{F} \;|\; x!M_{C}}
  \and
  \inferrule* [lab=abstraction] {} {{M_{F}} \bc (x)M_{P} }
  \and
  \inferrule* [lab=concretion] {} {{M_{C}} \bc \langle M_{P} \rangle }
  \and \\
  \inferrule* [lab=process] {} {{M_{P}} \bc M_{N} \;| \;P|M_{P} }
\end{mathpar}

\begin{definition}[contextual application] Given a context $M$, and
  process $P$, we define the \emph{contextual application}, $M[P] :=
  M\{P/\Box\}$. That is, the contextual application of M to P is the
  substitution of $P$ for $\Box$ in $M$.
\end{definition}

$\meaningof{-} : L \to \mathcal{P}(\pi)$

\begin{mathpar}
  \inferrule* [lab=collection] {} {\meaningof{true} = \pi, \and \meaningof{~E} = \pi \setminus \meaningof{E}, \and \meaningof{E_{1} \& E_{2}} = \meaningof{E_{1}} \cap \meaningof{E_{2}}}
\end{mathpar}

\begin{mathpar}
  \inferrule* [lab=structure] {} {\meaningof{0} = \{ P \in \pi | P \equiv 0 \}, \and \\ \meaningof{E_1 | E_2} = \{ P \in \pi | P \equiv P_{1} | P_{2}, P_{1} \in \meaningof{E_{1}}, P_{2} \in \meaningof{E_2}\} }
\end{mathpar}

\begin{mathpar}
 \inferrule* [lab=behavior] {} {\meaningof{\langle a?b \rangle E} = \{ P \in \pi | P \equiv Q | u?(y)P', \\ \and \\\\ \and \\ \;\;\; u \in \meaningof{a}, \forall z.P'\{z/y\} \in \meaningof{E\{z/b\}}\}, \and \\ \meaningof{a!E} = \{ P \in \pi | P \equiv Q | x!\langle P' \rangle, x \in \meaningof{a} P' \in \meaningof{E}\} }
\end{mathpar}

\begin{mathpar}
 \inferrule* [lab=nominal] {} {\meaningof{\quotep{E}} = \{ \quotep{P} \in \quotep{\pi} | P \in \meaningof{E} \}, \and \meaningof{\quotep{P}} = \{ \quotep{Q} \in \quotep{\pi} | P \equiv Q \} \and \\ \meaningof{@\quotep{E}} = \{ P \in \pi | P \equiv @x, x \in \meaningof{E} \}}
\end{mathpar}

\begin{eqnarray*}
  \\
  \meaningof{-} : TS \to ST
\end{eqnarray*}

\begin{eqnarray*}
  \\
  L : TS \to ST
\end{eqnarray*}

\begin{eqnarray*}
  \\
  P \models E \iff P \in \meaningof{E}
\end{eqnarray*}

\begin{eqnarray*}
  P \approx_{L} Q \iff \forall E \in L. P \models E \iff Q \models E
\end{eqnarray*}

\begin{eqnarray*}
  P \approx_{K} Q
\end{eqnarray*}

\begin{eqnarray*}
  P \approx Q
\end{eqnarray*}

$\approx_{K} = \approx = \approx_{L}$

\subsubsection{Contextual duality}

Note that contexts extend the quotation operation to a family of
operations from processes to names. Given a context, $M$, we can
define a \emph{nominal context}, $\quotep{M}$ by $\quotep{M}[P] :=
\quotep{M[P]}$. To foreshadow what is to come we observe that these
operations enjoy a duality with processes very much like the duality
between vectors and maps from vectors to scalars.

Further, because the calculus is essentially higher-order, we have a
correspondence between contexts and processes. More specifically,
given a name $x$ and a context $M$ we can construct $M^{*}_{x}$ such
that 

\begin{mathpar}
  M^{*}_{x} | \lift{x}{P} \red M[P]
\end{mathpar}

namely,

\begin{mathpar}
  M^{*}_{x} := x?(u).M[\dropn{u}]
\end{mathpar}

The dependence of $M^{*}_{x}$ on a name makes it an abstraction, 

\begin{mathpar}
  M^{*} := (x)x?(u).M[\dropn{u}]
\end{mathpar}

\subsection{Additional notation}

It will sometimes be convenient to denote the process a name
quotes. We already have the notation $x = \quotep{P}$, but it will be
convenient to introduce an alternate notation, $\procn{x}$, when we
want to emphasize the connection to the use of the name. Note that, by
virtue of name equivalence, $\quotep{\procn{x}} \nameeq x$; so, the
notation is consistent with previous definitions.

Further, because names have structure it is possible to effect
substitutions on the basis of that structure. This means we need to
upgrade our notation for substitutions, which we accomplish by
adapting comprehension notation. Thus,

\begin{mathpar}
  P\{ y / x : x \in S \}
\end{mathpar}

is interpreted to mean the process derived from P by replacing (in a
capture-avoiding manner) each occurrence of $x$ in $S$ by $y$. For example,

\begin{mathpar}
  P\{ \quotep{\procn{x}|\procn{x}} / x : x \in \freenames{P} \}
\end{mathpar}

will replace each (occurrence) of a free name $x$ in $P$ by
$\quotep{\procn{x}|\procn{x}}$.

Also, we will avail ourselves of the notation $x^{L}$ and $x^{R}$ to
denote injections of a name into disjoint copies of the name
space. There are numerous ways to accomplish this. One example can be
found in \cite{MeredithR05}. This notation overloads to vectors of
names: $\vec{x}^{\pi} := (x_{i}^{\pi} \; : \; 0 \leq i < |\vec{x}| )$ where $\pi \in \{L,R\}$.

We also use $P^{\Box} := P|\Box$.

In \cite{MeredithR05} an interpretation of the new operator is
given. It turns out that there are several possible interpretations
all enjoying the requisite algebraic properties of the operator (see
\cite{milner91polyadicpi}). We will therefore make liberal use of
$(\nu\; \vec{x})P$.

% subsection the_syntax_and_semantics_of_the_notation_system (end)   

\input{qm2pi.qmops} 

\input{qm2pi.sterngerlach} 

\input{qm2pi.metric} 

% section concurrent_process_calculi (end)

%\input{qm2pi.proofsketch}

% section proof sketch (end)

%\input{qm2pi.slviaknots} 

% section spatial logic via knots (end)

\input{qm2pi.conclusion}

% section conclusion (end)

%\input{qm2pi.dtcodes} 

% section wiring algorithm (end)

\input{qm2pi.ack} 

% section acknowledgments (end)

\newpage


\bibliographystyle{plain}   
\bibliography{../../biblios/main.bib}

\input{qm2pi.rhodetails}

\end{document}

 

% subsection basic_interpretation (end)

%\input{qm2pi.rho.presentation} 
\subsection{The syntax and semantics of the notation system}\label{sub:the_syntax_and_semantics_of_the_notation_system} % (fold)

We now summarize a technical presentation of the calculus that
embodies our theory of dynamics. The typical presentation of such a
calculus follows the style of giving generators and relations on
them. The grammar, below, describing term constructors, freely
generates the set of processes, $\Proc$. This set is then quotiented
by a relation known as structural congruence and it is over this set
that the notion of dynamics is expressed. This presentation is
essentially that of \cite{MeredithR05} with the addition of
polyadicity and summation. For readability we have relegated some of
the technical subtleties to an appendix.

\subsubsection{Process grammar}\label{subsub:process_grammar}

\begin{mathpar}
  \inferrule* [lab=synchronization] {} {{M} \bc \pzero \;|\; x?F \;|\; x!C }
  \and
  \inferrule* [lab=abstraction] {} {{F} \bc (x)P}
  \and
  \inferrule* [lab=concretion] {} {{C} \bc \langle Q \rangle}
  \and
  \inferrule* [lab=process] {} {{P,Q} \bc M \;| \;P|Q \;|\; @{x}}
  \and
  \inferrule* [lab=name] {} {{x} \bc \quotep{P}}
\end{mathpar} 

Note that $\vec{x}$ (resp. $\vec{P}$) denotes a vector of names
(resp. processes) of length $|\vec{x}|$ (resp. $|\vec{P}|$). We adopt
the following useful abbreviations.

\begin{mathpar}
   x?(\vec{y}).P := x.(\vec{y})P \and  x\clift{\vec{P}} := x.\clift{\vec{P}}
   \and x!(y) := \lift{x}{\dropn{y}}
   \and \Pi_{i=0}^{n-1}P_i := P_0 | \ldots | P_{n-1}
\end{mathpar}

\subsubsection{Structural congruence}

\paragraph{Free and bound names and alpha-equivalence.} At the
core of structural equivalence is alpha-equivalence which identifies
process that are the same up to a change of variable. Formally, we
recognize the distinction between free and bound names. The free names
of a process, $\freenames{P}$, may be calculated recursively as
follows:

\begin{mathpar}
\freenames{\pzero} := \emptyset
  \and \\
  \freenames{x?(y).P} := \{ x \} \cup (\freenames{P} \setminus \{ y \})
  \and 
  \freenames{x!\langle P \rangle} := \{ x \} \cup \{ P \} 
  \and \\
  \freenames{P|Q} := \freenames{P} \cup \freenames{Q}
  \and \\
  \freenames{@{x}} := \{ x \}
\end{mathpar}

$\pi$
$\quotep{\pi}$

$\freenames{-} : \pi \to \mathcal{P}(\quotep{\pi})$

\begin{eqnarray*}
  \freenames{\pzero} & := & \emptyset \\
  \freenames{x?(y).P} & := & \{ x \} \cup (\freenames{P} \setminus \{ y \}) \\
  \freenames{x!\langle P \rangle} & := & \{ x \} \cup \{ P \} \\
  \freenames{P|Q} & := & \freenames{P} \cup \freenames{Q} \\
  \freenames{\dropn{x}} & := & \{ x \}
\end{eqnarray*}

The bound names of a process, $\boundnames{P}$, are those names occurring in $P$
that are not free. For example, in $x?(y).0$, the name $x$ is free, while $y$ is bound.

\begin{mathpar}
  \inferrule* [lab=monoidal-laws] {} { P|Q \equiv Q|P \and P|0 \equiv P \and P|(Q|R) \equiv (P|Q)|R }
\end{mathpar}

\begin{mathpar}
  \inferrule* [lab=alpha-equivalence] {} { (x)P \equiv (y)P\{y/x\} \and y \not\in \freenames{P} }
\end{mathpar}

\begin{definition}
Then two processes, $P,Q$, are alpha-equivalent if $P = Q\{\vec{y}/\vec{x}\}$ for
some $\vec{x} \in \boundnames{Q},\vec{y} \in \boundnames{P}$, where $Q\{\vec{y}/\vec{x}\}$
denotes the capture-avoiding substitution of $\vec{y}$ for $\vec{x}$ in $Q$.
\end{definition}

\begin{definition}
  The {\em structural congruence} \cite{SangiorgiWalker} , $\equiv$,
  between processes is the least congruence containing
  alpha-equivalence, satisfying the abelian monoid laws
  (associativity, commutativity and $\pzero$ as identity) for parallel
  composition $|$ and for summation $+$.
\end{definition}

\subsection{Name equivalence}

We take name equivalence, written $\nameeq$, to be the smallest
equivalence relation generated by the following rules.

\begin{mathpar}
\inferrule*[lab=Quote-drop]
{ }
{ \quotep{@{x}} \nameeq x }

\inferrule*[lab=Struct-equiv]
{ P \scong Q }
{ \quotep{P} \nameeq \quotep{Q} }
\end{mathpar}

The astute reader will have noticed that the mutual recursion of names
and processes imposes a mutual recursion on alpha-equivalence and
structural equivalence via name-equivalence. Fortunately, all of this
works out pleasantly and we may calculate in the natural way, free of
concern. The reader interested in the details is referred to the
appendix \ref{appendix:rho_details}.

\subsection{Substitution}

We use $\Proc$ for the set of processes, $\QProc$ for the set of
names, and $\id{\{}\vec{y} / \vec{x} \id{\}}$ to denote partial maps,
$s : \QProc \rightarrow \QProc$. A map, $s$ lifts, uniquely, to a map
on process terms, $\widehat{s} : \Proc \rightarrow \Proc$ by the
following equations.

\begin{mathpar}
  (0) \psubstp{Q}{P} := 0 \\
  (R \juxtap S) \psubstp{Q}{P}
  :=    
  (R)\psubstp{Q}{P} \juxtap (S) \psubstp{Q}{P} \\
  (x?(y).R) \psubstp{Q}{P}    
  :=    
  (x)\substp{Q}{P} (z)\concat( (R \psubstn{z}{y}) \psubstp{Q}{P} ) \\
  (\lift{x}{R}) \psubstp{Q}{P}  
  :=
  \lift{(x)\substp{Q}{P}}{ R \psubstp{Q}{P} } \\
%   (\dropn{x})  \psubstp{Q}{P}       
%   := 
%   \left\{ 
%     \begin{array}{ccc} 
%       \dropn{\quotep{Q}} & & x \nameeq \quotep{P} \\
%       \dropn{x} & & otherwise \\
%     \end{array}
%   \right. 
  (\dropn{x})  \psubstp{Q}{P}       
  := 
  \left\{ 
    \begin{array}{ccc} 
      Q & & x \nameeq \quotep{P} \\
      \dropn{x} & & otherwise \\
    \end{array}
  \right.
\end{mathpar}
 

where

\begin{eqnarray}
  (x)\id{\{} \lpquote Q \rpquote / \lpquote P \rpquote \id{\}}            = 
  \left\{ 
    \begin{array}{ccc}
      \lpquote Q \rpquote & & x \nameeq \lpquote P \rpquote \\
      x & & otherwise \\
    \end{array}
  \right. \nonumber
\end{eqnarray}

and $z$ is chosen distinct from $\quotep{P}$, $\quotep{Q}$, the free
names in $Q$, and all the names in $R$. Our $\alpha$-equivalence will
be built in the standard way from this substitution.

\begin{remark}\label{rem:no_self_referential_names}
  One consequence of these definitions is that $\forall P. \quotep{P}
  \not\in \freenames{P}$.
\end{remark}

\subsection{ Dynamic quote: an example }

Anticipating something of what's to come, consider applying the
substitution, $\widehat{\id{\{}u / z \id{\}}}$, to the following pair
of processes, $\lift{w}{y!(z)}$ and $w[ \lpquote y!(z) \rpquote ]$.

\begin{eqnarray}
	\lift{w}{y!(z)}\widehat{\id{\{}u / z \id{\}}}
		& = &
		\lift{w}{y!(u)} \nonumber\\
	w[ \lpquote y!(z) \rpquote ] \widehat{ \id{\{}u / z \id{\}} }
		& = &
		w[ \lpquote y!(z) \rpquote ] \nonumber
\end{eqnarray}

Because the body of the process between quotes is impervious to
substitution, we get radically different answers. In fact, by
examining the first process in an input context,
e.g. $x?(z).\lift{w}{y!(z)}$, we see that the process under the lift
operator may be shaped by prefixed inputs binding a name inside it. In
this sense, the lift operator will be seen as a way to dynamically
construct processes before reifying them as names.

Finally equipped with these standard features we can present the
dynamics of the calculus.

\subsubsection{Operational semantics} 

Finally, we introduce the computational dynamics. What marks these
algebras as distinct from other more traditionally studied algebraic
structures, e.g. vector spaces or polynomial rings, is the manner in
which dynamics is captured. In traditional structures, dynamics is typically
expressed through morphisms between such structures, as in linear maps
between vector spaces or morphisms between rings. In algebras
associated with the semantics of computation, the dynamics is
expressed as part of the algebraic structure itself, through a
reduction reduction relation typically denoted by $\red$. Below, we
give a recursive presentation of this relation for the calculus used
in the encoding.

$\red \subseteq \pi \times \pi$
$\red : \pi \to \mathcal{P}(\pi)$

\begin{mathpar}
  \inferrule* [lab=Comm] { \textsf{match}( x_{src}, x_{trgt} ) } { x_{trgt}?(y)P \; | \; x_{src}!\langle {Q} \rangle \red P\{\quotep{Q}/y}\} }
  \and \\
  \inferrule* [lab=Par] {{P} \red {P}'} {{{P} | {Q}} \red {{P}' | {Q}}}
  \and
  \inferrule* [lab=Equiv]{{{P} \scong {P}'} \andalso {{P}' \red {Q}'} \andalso {{Q}' \scong {Q}}}{{P} \red {Q}}
\end{mathpar}

\begin{eqnarray*}
  match_{\equiv} (\quotep{P},\quotep{Q}) & := & P \equiv Q \\
  match_{\dagger}(\quotep{P},\quotep{Q}) & := & \forall R. P|Q \red^{*} R => R \red^{*} 0 \\
  match_{K}(\quotep{P},\quotep{Q}) & := & K \mbox{ for some context } K
\end{eqnarray*}

$u?(x)P | u!\langle Q \rangle \red P\{\quotep{Q}/x\}$

%We write $\wred$ for $\red^*$, and $P\red$ if $\exists Q $ such that $ P \red Q$.
We write $P\red$ if $\exists Q $ such that $ P \red Q$ and $P\not\red$, otherwise.

\section{Replication}

As mentioned before, it is known that replication (and hence
recursion) can be implemented in a higher-order process algebra
\cite{SangiorgiWalker}. As our first example of calculation with the
machinery thus far presented we give the construction explicitly in
the {\rhoc}.

\begin{eqnarray}
	D_{x} & := & \prefix{x}{y}{(\binpar{\outputp{x}{y}}{@{y}})} \nonumber\\
	\bangp_{x}{P} & := & \binpar{{x}!\langle{\binpar{D_{x}}{P}}\rangle}{D_{x}} \nonumber
\end{eqnarray}

\begin{eqnarray}
	\bangp_{x}{P} & & \nonumber\\
	=
	& {x}!\langle{(\prefix{x}{y}{(\outputp{x}{y} | @{y})) | P}}\rangle 
	      | \prefix{x}{y}{(\outputp{x}{y} | @{y})} & \nonumber\\
	\red
	& (\outputp{x}{y} | @{y})\substn{\quotep{(\prefix{x}{y}{(@{y} | \outputp{x}{y})) | P}}}{y} & \nonumber\\
	=
	& \outputp{x}{\quotep{(\prefix{x}{y}{(\outputp{x}{y} | @{y})) | P}}}
	  | {(\prefix{x}{y}{(\outputp{x}{y} | @{y})) | P}} & \nonumber\\
	\red
	& \ldots & \nonumber\\
	\red^*
	& P | P | \ldots & \nonumber
\end{eqnarray}

Of course, this encoding, as an implementation, runs away, unfolding
$\bangp{P}$ eagerly. A lazier and more implementable replication
operator, restricted to input-guarded processes, may be obtained as follows.

\begin{eqnarray}
\bangp{\prefix{u}{v}{P}} 
	:= 
	\binpar{\lift{x}{\prefix{u}{v}{(\binpar{D(x)}{P})}}}{D(x)} \nonumber
\end{eqnarray}

\begin{remark}
  Note that the lazier definition still does not deal with summation
  or mixed summation (i.e. sums over input and output). The reader is
  invited to construct definitions of replication that deal with these
  features. 

  Further, the definitions are parameterized in a name, $x$. Can you,
  gentle reader, make a definition that eliminates this parameter and
  guarantees no accidental interaction between the replication
  machinery and the process being replicated -- i.e. no accidental
  sharing of names used by the process to get its work done and the
  name(s) used by the replication to effect copying. This latter
  revision of the definition of replication is crucial to obtaining
  the expected identity $!!P \sim !P$.
\end{remark}

\begin{remark}\label{rem:paradoxical_combinator}
  The reader familiar with the lambda calculus will have noticed the
  similarity between $D$ and the paradoxical combinator.

  [Ed. note: the existence of this seems to suggest we have to be more
  restrictive on the set of processes and names we admit if we are to
  support no-cloning.]
\end{remark}

\subsubsection{Bisimulation}

The computational dynamics gives rise to another kind of equivalence,
the equivalence of computational behavior. As previously mentioned
this is typically captured \emph{via} some form of bisimulation.

% The notion we use in this paper is weak barbed bisimulation
% \cite{milner91polyadicpi}.

The notion we use in this paper is derived from weak barbed
bisimulation \cite{milner91polyadicpi}. 

\begin{definition}
An \emph{observation relation}, $\downarrow_{\mathcal N}$, over a set
of names, $\mathcal N$, is the smallest relation satisfying the rules
below.

\infrule[Out-barb]{y \in {\mathcal N}, \; x \nameeq y}
		  {\outputp{x}{v} \downarrow_{\mathcal N} x}
\infrule[Par-barb]{\mbox{$P\downarrow_{\mathcal N} x$ or $Q\downarrow_{\mathcal N} x$}}
		  {\binpar{P}{Q} \downarrow_{\mathcal N} x}

We write $P \Downarrow_{\mathcal N} x$ if there is $Q$ such that 
$P \wred Q$ and $Q \downarrow_{\mathcal N} x$.
\end{definition}

\begin{definition}
%\label{def.bbisim}
An  ${\mathcal N}$-\emph{barbed bisimulation} over a set of names, ${\mathcal N}$, is a symmetric binary relation 
${\mathcal S}_{\mathcal N}$ between agents such that $P\rel{S}_{\mathcal N}Q$ implies:
\begin{enumerate}
\item If $P \red P'$ then $Q \wred Q'$ and $P'\rel{S}_{\mathcal N} Q'$.
\item If $P\downarrow_{\mathcal N} x$, then $Q\Downarrow_{\mathcal N} x$.
\end{enumerate}
$P$ is ${\mathcal N}$-barbed bisimilar to $Q$, written
$P \wbbisim_{\mathcal N} Q$, if $P \rel{S}_{\mathcal N} Q$ for some ${\mathcal N}$-barbed bisimulation ${\mathcal S}_{\mathcal N}$.
\end{definition}

$\mathcal{R} \subseteq \pi \times \pi$

$P \mathcal{R} Q => \forall P'. P \red P' \Rightarrow \exists Q'. Q \red Q', P' \mathcal{R} Q'$

$P \vdash x \Rightarrow Q \vdash x$

\begin{mathpar}
  \inferrule*[lab=Out-barb]{x \nameeq y}{{y}!\langle{Q}\rangle \vdash x}
  \and
  \inferrule*[lab=Par-barb]{\mbox{$P\vdash x$ or $Q\vdash x$}}{\binpar{P}{Q} \vdash x}
\end{mathpar}

\subsubsection{Contexts}

One of the principle advantages of computational calculi like the
$\pi$-calculus is a well-defined notion of context,
contextual-equivalence and a correlation between
contextual-equivalence and notions of bisimulation. The notion of
context allows the decomposition of a process into (sub-)process and
its syntactic environment, its context. Thus, a context may be
thought of as a process with a ``hole'' (written $\Box$) in it. The
application of a context $M$ to a process $P$, written $M[P]$, is
tantamount to filling the hole in $M$ with $P$. In this paper we do
not need the full weight of this theory, but do make use of the notion
of context in the proof the main theorem. 

\begin{mathpar}
  \inferrule* [lab=summation] {} {{M_{M},M_{N}} \bc \Box \;|\; x.M_{A} \;|\; M_{M}+M_{N}}
  \and
  \inferrule* [lab=agent] {} {{M_{A}} \bc (\vec{x})M_{P} \;| \; \clift{P_0,\ldots,M_{P},\ldots,P_N}}
  \and \\
  \inferrule* [lab=process] {} {{M_{P}} \bc M_{N} \;| \;P|M_{P} }
\end{mathpar} 

\begin{mathpar}
  \inferrule* [lab=sychronization] {} {M_{N} \bc \Box \;|\; x?M_{F} \;|\; x!M_{C}}
  \and
  \inferrule* [lab=abstraction] {} {{M_{F}} \bc (x)M_{P} }
  \and
  \inferrule* [lab=concretion] {} {{M_{C}} \bc \langle M_{P} \rangle }
  \and \\
  \inferrule* [lab=process] {} {{M_{P}} \bc M_{N} \;| \;P|M_{P} }
\end{mathpar}

\begin{definition}[contextual application] Given a context $M$, and
  process $P$, we define the \emph{contextual application}, $M[P] :=
  M\{P/\Box\}$. That is, the contextual application of M to P is the
  substitution of $P$ for $\Box$ in $M$.
\end{definition}

$\meaningof{-} : L \to \mathcal{P}(\pi)$

\begin{mathpar}
  \inferrule* [lab=collection] {} {\meaningof{true} = \pi, \and \meaningof{~E} = \pi \setminus \meaningof{E}, \and \meaningof{E_{1} \& E_{2}} = \meaningof{E_{1}} \cap \meaningof{E_{2}}}
\end{mathpar}

\begin{mathpar}
  \inferrule* [lab=structure] {} {\meaningof{0} = \{ P \in \pi | P \equiv 0 \}, \and \\ \meaningof{E_1 | E_2} = \{ P \in \pi | P \equiv P_{1} | P_{2}, P_{1} \in \meaningof{E_{1}}, P_{2} \in \meaningof{E_2}\} }
\end{mathpar}

\begin{mathpar}
 \inferrule* [lab=behavior] {} {\meaningof{\langle a?b \rangle E} = \{ P \in \pi | P \equiv Q | u?(y)P', \\ \and \\\\ \and \\ \;\;\; u \in \meaningof{a}, \forall z.P'\{z/y\} \in \meaningof{E\{z/b\}}\}, \and \\ \meaningof{a!E} = \{ P \in \pi | P \equiv Q | x!\langle P' \rangle, x \in \meaningof{a} P' \in \meaningof{E}\} }
\end{mathpar}

\begin{mathpar}
 \inferrule* [lab=nominal] {} {\meaningof{\quotep{E}} = \{ \quotep{P} \in \quotep{\pi} | P \in \meaningof{E} \}, \and \meaningof{\quotep{P}} = \{ \quotep{Q} \in \quotep{\pi} | P \equiv Q \} \and \\ \meaningof{@\quotep{E}} = \{ P \in \pi | P \equiv @x, x \in \meaningof{E} \}}
\end{mathpar}

\begin{eqnarray*}
  \\
  \meaningof{-} : TS \to ST
\end{eqnarray*}

\begin{eqnarray*}
  \\
  L : TS \to ST
\end{eqnarray*}

\begin{eqnarray*}
  \\
  P \models E \iff P \in \meaningof{E}
\end{eqnarray*}

\begin{eqnarray*}
  P \approx_{L} Q \iff \forall E \in L. P \models E \iff Q \models E
\end{eqnarray*}

\begin{eqnarray*}
  P \approx_{K} Q
\end{eqnarray*}

\begin{eqnarray*}
  P \approx Q
\end{eqnarray*}

$\approx_{K} = \approx = \approx_{L}$

\subsubsection{Contextual duality}

Note that contexts extend the quotation operation to a family of
operations from processes to names. Given a context, $M$, we can
define a \emph{nominal context}, $\quotep{M}$ by $\quotep{M}[P] :=
\quotep{M[P]}$. To foreshadow what is to come we observe that these
operations enjoy a duality with processes very much like the duality
between vectors and maps from vectors to scalars.

Further, because the calculus is essentially higher-order, we have a
correspondence between contexts and processes. More specifically,
given a name $x$ and a context $M$ we can construct $M^{*}_{x}$ such
that 

\begin{mathpar}
  M^{*}_{x} | \lift{x}{P} \red M[P]
\end{mathpar}

namely,

\begin{mathpar}
  M^{*}_{x} := x?(u).M[\dropn{u}]
\end{mathpar}

The dependence of $M^{*}_{x}$ on a name makes it an abstraction, 

\begin{mathpar}
  M^{*} := (x)x?(u).M[\dropn{u}]
\end{mathpar}

\subsection{Additional notation}

It will sometimes be convenient to denote the process a name
quotes. We already have the notation $x = \quotep{P}$, but it will be
convenient to introduce an alternate notation, $\procn{x}$, when we
want to emphasize the connection to the use of the name. Note that, by
virtue of name equivalence, $\quotep{\procn{x}} \nameeq x$; so, the
notation is consistent with previous definitions.

Further, because names have structure it is possible to effect
substitutions on the basis of that structure. This means we need to
upgrade our notation for substitutions, which we accomplish by
adapting comprehension notation. Thus,

\begin{mathpar}
  P\{ y / x : x \in S \}
\end{mathpar}

is interpreted to mean the process derived from P by replacing (in a
capture-avoiding manner) each occurrence of $x$ in $S$ by $y$. For example,

\begin{mathpar}
  P\{ \quotep{\procn{x}|\procn{x}} / x : x \in \freenames{P} \}
\end{mathpar}

will replace each (occurrence) of a free name $x$ in $P$ by
$\quotep{\procn{x}|\procn{x}}$.

Also, we will avail ourselves of the notation $x^{L}$ and $x^{R}$ to
denote injections of a name into disjoint copies of the name
space. There are numerous ways to accomplish this. One example can be
found in \cite{MeredithR05}. This notation overloads to vectors of
names: $\vec{x}^{\pi} := (x_{i}^{\pi} \; : \; 0 \leq i < |\vec{x}| )$ where $\pi \in \{L,R\}$.

We also use $P^{\Box} := P|\Box$.

In \cite{MeredithR05} an interpretation of the new operator is
given. It turns out that there are several possible interpretations
all enjoying the requisite algebraic properties of the operator (see
\cite{milner91polyadicpi}). We will therefore make liberal use of
$(\nu\; \vec{x})P$.

% subsection the_syntax_and_semantics_of_the_notation_system (end)   

\section{Interpretation of QM}
\subsection{Supporting definitions}
\subsubsection{Multiplication}
\begin{mathpar}
  \quotep{Q} \cdot \quotep{R} := \quotep{Q|R}
  \and \\
  \quotep{Q} \cdot P := P\{ \quotep{Q|R} / \quotep{R} : \quotep{R} \in \freenames{P} \}
\end{mathpar}

\paragraph{Discussion}
The first line needs little explanation. The second line says that
each free name of the process is replaced with the multiplication of
that name by the scalar. Multiplication of a scalar (name) by a state
(process) results in a process all the names of which have been `moved
over' by parallel composition with the process the scalar
quotes. There is a subtlety that the bound names have to be
manipulated so that multiplied names aren't accidentally
captured. There are many ways to achieve this.

\begin{remark}\label{rem:multiplication_identities}
  The reader is invited to verify that for all $x,y,z \in \QProc$ and $P \in \Proc$
  \begin{mathpar}
    x \cdot \quotep{0} \equiv x 
    \and
    x \cdot y \equiv y \cdot x
    \and
    x \cdot (y \cdot z) \equiv (x \cdot y) \cdot z
    \and \\
    \quotep{0} \cdot P \equiv P
    \and \\
    x \cdot (y \cdot P) \equiv (x \cdot y) \cdot P
    \and \\
    x \cdot (P|Q) \equiv (x \cdot P) | (x \cdot Q)
    \and \\    
  \end{mathpar}
\end{remark}

\subsubsection{Tensor product}

We define a tensor product on processes by structural induction.

\paragraph{Tensor of sums} First note that all summations, including
$\pzero$ and sequence, can be written $\Sigma_{i} x_{i}.A_{i} +
\Sigma_{j} x_{j}.C_{j}$, where we have grouped input-guarded processes
together and output-guarded processes together.

Thus, we can define the tensor product of two summations, $N_{1}\otimes N_{2}$, where

\begin{mathpar}
  N_{1} := \Sigma_{i} x_{i}.A_{i} + \Sigma_{j} x_{j}.C_{j}
  \and
  N_{2} := \Sigma_{i'} y_{i'}.B_{i'} + \Sigma_{j'} y_{j'}.D_{j'} 
\end{mathpar}

as follows.

\begin{mathpar}
  \Sigma_{i} x_{i}.A_{i} + \Sigma_{j} x_{j}.C_{j} \otimes \Sigma_{i'}
  y_{i'}.B_{i'} + \Sigma_{j'} y_{j'}.D_{j'} 
  \and \\
  := \; \Sigma_{i} \Sigma_{i'} \quotep{\stackrel{\vee}{x_{i}}| \stackrel{\vee}{y_{i'}}}.(A_{i}\otimes B_{i'}) \; | \; \Sigma_{i'} \Sigma_{i} \quotep{\stackrel{\vee}{y_{i'}}|\stackrel{\vee}{x_{i}}}.(B_{i'}\otimes A_{i})
  \and
  \;\; | \;\; \Sigma_{j} \Sigma_{j'} \quotep{\stackrel{\vee}{x_{j}}|\stackrel{\vee}{y_{j'}}}.(A_{j}\otimes B_{j'}) \; | \; \Sigma_{j'} \Sigma_{j} \quotep{\stackrel{\vee}{y_{j'}}|\stackrel{\vee}{x_{j}}}.(B_{j'}\otimes A_{j})
\end{mathpar}

\begin{remark}
  Do we need to $x^{L}$ and $y^{R}$ for this construction as well?
\end{remark}

\paragraph{Tensor of parallel compositions} Next, we distribute tensor
over par.

\begin{mathpar}
  P_{1}|P_{2} \otimes Q_{1}|Q_{2} := (P_{1} \otimes Q_{1}) | (P_{1}
  \otimes Q_{2}) | (P_{2} \otimes Q_{1}) | (P_{2} \otimes Q_{2})
\end{mathpar}

\paragraph{Tensor with dropped names} We treat tensor of a
process with a dropped name as parallel composition.

\begin{mathpar}
  P \otimes \dropn{x} := P | \dropn{x}
\end{mathpar}

\paragraph{Tensor of agents}

Finally, we need to define tensor on agents. Note that the definition
of tensor on normal products only tensors inputs with inputs and
outputs with outputs. Thus, we only have to define the operation on
``homogeneous'' pairings.

\begin{mathpar}
  (\vec{x})P \otimes (\vec{y})Q
  \and \\
  := (x_{0}^{L}|y_{0}^{R},\ldots,x_{0}^{L}|y_{n}^{R},\ldots,x_{m}^{L}|y_{0}^{R},\ldots,x_{m}^{L}|y_{n}^R)(P\{ \vec{x}^{L}/\vec{x}\} \otimes Q \{ \vec{y}^{R}/\vec{y}\})
  \and \\
  \clift{\vec{P}} \otimes \clift{\vec{Q}}
  \and \\
  := \clift{P_{0}\otimes Q_{0},\ldots,P_{0}\otimes Q_{n},\ldots,P_{m}\otimes Q_{0},\ldots,P_{m}\otimes Q_{n}}
\end{mathpar}

\begin{remark}
  Observe that arities of tensored abstractions matches arities of
  tensored concretions if the original arities matched. Note also that
  the length of the arities corresponds to the increase in dimension
  we see in ordinary vector space tensor product.
\end{remark}

\begin{remark}
  Operationally, this definition distributes the tensor down to
  components ``linked'' by summation. Tensor over summation is
  intriguing in that it mixes names. Moreover, as a consequence of the
  way it mixes names we have the identities for all $x \in \QProc$ and
  $P,Q \in \Proc$

  \begin{mathpar}
    (x \cdot P) \otimes Q \equiv x \cdot (P \otimes Q) \equiv P \otimes (x \cdot Q)
    \and
    P \otimes \pzero \equiv P
  \end{mathpar}

  that the reader is invited to verify.
\end{remark}

\subsubsection{Annihilation}
\begin{mathpar}
  P^{\perp} := \{ Q | \forall R. P|Q \red^{*} R \Rightarrow R \red^{*} \pzero \}
  \and \\
  P^{\underline{\perp}} := \Sigma_{Q \in P^{\perp}} \quotep{Q}?(y).(\dropn{y}|Q) | \Sigma_{Q \in P^{\perp}} \quotep{Q}\clift{\Box}
\end{mathpar}

\paragraph{Discussion} The reader will note that $P^{\perp}$ is a
\emph{set} of processes, while $P^{\underline{\perp}}$ is a
\emph{context}. We call the set $P^{\perp}$ the \emph{annihilators} of
$P$. The parallel composition of a process in the annihilators of $P$
with $P$ will result in a process, the state space of which has all
paths eventually leading to $\pzero$. Execution may endure loops; but
under reasonable conditions of fairness (naturally guaranteed under
most notions of bisimulation) such a composite process cannot get
stuck in such a loop and will, eventually pop out and terminate.

The context $P^{\underline{\perp}}$ is ready and willing to ``take the
$P$ out of'' the process to which it is applied. It will effectively
transmit the code of the process to which it is applied to one of the
annihilators and run the process against it.

\subsubsection{Evaluation}
We fix $M$ a domain of fully abstract interpretation with an equality
coincident with bisimulation. We take $\meaningof{\cdot} : \Proc \to
M$ to be the map interpreting processes and $\nmeaningof{\cdot} : \M
\to Proc$ to be the map running the other way. Then we define

\begin{mathpar}
  \int P := \nmeaningof{\meaningof{P}}
\end{mathpar}

\paragraph{Discussion}
There are many fully abstract interpretations of Milner's
$\pi$-calculus. Any of them can be used as a basis for interpreting
the reflective calculus here. Equipped with such a domain it is
largely a matter of grinding through to check that the Yoneda
construction for the normalization-by-evaluation program can be
extended to this setting.

\begin{remark}
  The reader is invited to verify that $\int (P^{\underline{\perp}}[P]) = 0$.
\end{remark}

\subsection{Quantum mechanics}

Table \ref{tbl:core_qm_op_defns} gives the core operational definitions

\begin{table}[htp]\label{tbl:core_qm_op_defns}
  \center{
    \fbox{
      \begin{tabular}{c|c}
        quantum mechanics & process calculus \\
        \hline
        scalar & $x := \quotep{P}$ \\
        state vector & $\state{P} := P$ \\
        dual & $\state{P}^{*} := \event{P^{\underline{\perp}}} := \quotep{P^{\underline{\perp}}}[-]$ \\
        matrix & $ \Sigma_{\alpha} \state{P_{\alpha}}x_{\alpha}\event{Q_{\alpha}}$ \\
        vector addition & $\state{P} + \state{Q} := \state{P | Q}$ \\
        tensor product & $\state{P} \otimes \state{Q} := \state{P \otimes Q}$ \\
        inner product & $\innerprod{P}{Q} := \quotep{\int P^{\underline{\perp}}[Q]}$ \\
      \end{tabular}
    }
  }
  \caption{QM - operational definitions}
\end{table}

where

\begin{mathpar}
  \prmatrix{P}{Q} := \fprmatrix{P}{\quotep{\pzero}}{Q}
  \and
  \fprmatrix{P}{x}{Q} := (\state{P},x,\event{Q})
  \and
  (\fprmatrix{P}{x}{Q})(\state{R}) := x \cdot \innerprod{Q}{R} \cdot \state{P}
  \and
  (\fprmatrix{P}{x}{Q})(\event{R}) := x \cdot \innerprod{R}{P} \cdot \event{Q}
\end{mathpar}

\paragraph{Discussion}
As promised: vectors (aka states) are represented as processes; duals
as contextual duals; inner product definition should be compared with
standard inner product definition for ....

\begin{remark}
  Assuming $\int (P^{\underline{\perp}}[P]) = 0$, the reader is
  invited to verify that $(\fprmatrix{P}{x}{P})(\state{P}) = x \cdot \state{P}$.
\end{remark}

\begin{remark}
  The reader is invited to verify that $\innerprod{P}{Q}$ could
  equally well have been written $\quotep{\int \stackrel{\vee}{x}}$
  where $x = \event{P^{\underline{\perp}}}(Q)$.

  One of the motivations for this remark is that there is another way
  to factor these operations. We could package up evaluation in the dual:

  \begin{mathpar}
    \state{P}^{*} := \event{\int P^{\underline{\perp}}} := \quotep{\int P^{\underline{\perp}}}[-]
  \end{mathpar}

  and then have inner product defined by
  
  \begin{mathpar}
    \innerprod{P}{Q} := \event{P}(Q)
  \end{mathpar}

  Hopefully, experience with the calculations will provide guidance on
  the best factoring.
\end{remark}

\begin{remark}
  Assuming $\int (P^{\underline{\perp}}[P]) = 0$, the reader is
  invited to verify that $\forall P,Q. (\prmatrix{0}{Q})(\state{0}) =
  \state{0}$ and dually $(\prmatrix{P}{0})(\event{0}) = \event{0}$.
\end{remark}

\begin{remark}
  i'm a little worried that i don't (yet) have proper support for
  complex conjugacy. But, the observation above may give us a
  clue. According to Abramsky, it must be the case that the scalars
  are iso to the homset of the identity for the tensor -- which the
  observation above characterizes. 

  For now, we will simply bookmark the notion with $\overline{x}$.
\end{remark}

\subsubsection{Adjointness}

We need to give a definition of $(\cdot)^{\dagger}$ for matrices. The
obvious candidate definition is
\begin{mathpar}
(\Sigma_{\alpha}\fprmatrix{P_{\alpha}}{x_{\alpha}}{Q_{\alpha}})^{\dagger}
= \Sigma_{\alpha}\fprmatrix{(Q_{\alpha}^{\underline{\perp}})^{*}}{\overline{x}_{\alpha}}{P_{\alpha}^{\underline{\perp}}} 
\end{mathpar}

But, $(Q_{\alpha}^{\underline{\perp}})^{*}$ requires a name along
which to communicate the process to achieve the context application.

\subsubsection{Basis for a basis}
If processes label states and ``addition'' of states (a.k.a. vector
addition) is interpreted as parallel composition, what corresponds to
notions of linear independence and basis? Here, we recall that Yoshida
has developed a set of \emph{combinators} for an asynchronous verison
of Milner's $\pi$-calculus. These are a finite set of processes such
any process can be expressed as parallel composition of these
combinators together with liberal uses of the new operator and
replication. We can simply give a translation of these into the
present calculus and have reasonable expectation that the property
carries over. That is, that the resultant set allows to express all
processes via parallel composition. Note, however, that there is no
new operator or replication in this calculus. As a result, we expect
that the corresponding set is actually infinite. That is, we expect
that the space is actually infinite dimensional.

\begin{remark}
  The attentive reader may be a bit concerned. Certainly, the
  collection $S$, $K$ and $I$ is a finite set of
  combinators. Shouldn't we expect to see a finite set of combinators
  for an effectively equivalent system? i am very sympathetic to this
  critique and feel it warrants full attention. On the other hand, i
  also have in mind the following analogy. The natural numbers, as a
  monoid under addition, has exactly $1$ generator, while the natural
  numbers, as a monoid under multiplication, has countably many
  generators (the primes). We observe that the application of the
  lambda calculus is much less resource sensitive than the parallel
  composition of the $\pi$-calculus. Could it be the case that we have
  an analogy of the form
  
  \begin{mathpar}
    m + n : MN :: m*n : M|N
  \end{mathpar}

  giving a similar blow up in the set of ``primes''?  This is such a
  wonderful thought that, even if it's not true, i think it's worth
  writing down.
\end{remark}
 

\documentclass[12pt]{llncs}
%\documentclass{jktr}

\usepackage[pdftex]{hyperref}                   
\usepackage {listings}
\usepackage {mathpartir}
\usepackage{bcprules}
%\usepackage{listings}
                       
\usepackage{graphicx} 
%\usepackage[margins=2.5cm,nohead,nofoot]{geometry}
%\usepackage{geometry}
\usepackage{amsfonts}
\usepackage{amstext}
\usepackage{latexsym}
\usepackage{amssymb}
\usepackage{color}


%\include{myPreamble}
\include{qm2pi.local} 

%\ifpdf
%\usepackage[pdftex]{graphicx}
%\else
%\usepackage{graphicx}
%\fi

 % \ifpdf
%  \usepackage{pdfsync}
%  \if


%\title{Brief Article}
%\author{David F. Snyder}
%\author{L.G. Meredith}

%\address{Dept. of Math., Texas State University--San Marcos, San Marcos, TX 78666}
       
\pagestyle{empty}


\begin{document}

\lstset{language=[Objective]Caml,frame=shadowbox}

\input{qm2pi.front}

% section front matter (end)

\input{qm2pi.intro} 
 
% section introduction (end)

% \input{qm2pi.knotations} 

% section notation (end)

\input{qm2pi.process.calculi} 

% section concurrent_process_calculi_and_spatial_logics_ (end)
    
%\input{qm2pi.knots2pi} 

%\input{qm2pi.trefoil} 

%\input{qm2pi.mainthm} 

% subsection basic_interpretation (end)

%\input{qm2pi.rho.presentation} 
\subsection{The syntax and semantics of the notation system}\label{sub:the_syntax_and_semantics_of_the_notation_system} % (fold)

We now summarize a technical presentation of the calculus that
embodies our theory of dynamics. The typical presentation of such a
calculus follows the style of giving generators and relations on
them. The grammar, below, describing term constructors, freely
generates the set of processes, $\Proc$. This set is then quotiented
by a relation known as structural congruence and it is over this set
that the notion of dynamics is expressed. This presentation is
essentially that of \cite{MeredithR05} with the addition of
polyadicity and summation. For readability we have relegated some of
the technical subtleties to an appendix.

\subsubsection{Process grammar}\label{subsub:process_grammar}

\begin{mathpar}
  \inferrule* [lab=synchronization] {} {{M} \bc \pzero \;|\; x?F \;|\; x!C }
  \and
  \inferrule* [lab=abstraction] {} {{F} \bc (x)P}
  \and
  \inferrule* [lab=concretion] {} {{C} \bc \langle Q \rangle}
  \and
  \inferrule* [lab=process] {} {{P,Q} \bc M \;| \;P|Q \;|\; @{x}}
  \and
  \inferrule* [lab=name] {} {{x} \bc \quotep{P}}
\end{mathpar} 

Note that $\vec{x}$ (resp. $\vec{P}$) denotes a vector of names
(resp. processes) of length $|\vec{x}|$ (resp. $|\vec{P}|$). We adopt
the following useful abbreviations.

\begin{mathpar}
   x?(\vec{y}).P := x.(\vec{y})P \and  x\clift{\vec{P}} := x.\clift{\vec{P}}
   \and x!(y) := \lift{x}{\dropn{y}}
   \and \Pi_{i=0}^{n-1}P_i := P_0 | \ldots | P_{n-1}
\end{mathpar}

\subsubsection{Structural congruence}

\paragraph{Free and bound names and alpha-equivalence.} At the
core of structural equivalence is alpha-equivalence which identifies
process that are the same up to a change of variable. Formally, we
recognize the distinction between free and bound names. The free names
of a process, $\freenames{P}$, may be calculated recursively as
follows:

\begin{mathpar}
\freenames{\pzero} := \emptyset
  \and \\
  \freenames{x?(y).P} := \{ x \} \cup (\freenames{P} \setminus \{ y \})
  \and 
  \freenames{x!\langle P \rangle} := \{ x \} \cup \{ P \} 
  \and \\
  \freenames{P|Q} := \freenames{P} \cup \freenames{Q}
  \and \\
  \freenames{@{x}} := \{ x \}
\end{mathpar}

$\pi$
$\quotep{\pi}$

$\freenames{-} : \pi \to \mathcal{P}(\quotep{\pi})$

\begin{eqnarray*}
  \freenames{\pzero} & := & \emptyset \\
  \freenames{x?(y).P} & := & \{ x \} \cup (\freenames{P} \setminus \{ y \}) \\
  \freenames{x!\langle P \rangle} & := & \{ x \} \cup \{ P \} \\
  \freenames{P|Q} & := & \freenames{P} \cup \freenames{Q} \\
  \freenames{\dropn{x}} & := & \{ x \}
\end{eqnarray*}

The bound names of a process, $\boundnames{P}$, are those names occurring in $P$
that are not free. For example, in $x?(y).0$, the name $x$ is free, while $y$ is bound.

\begin{mathpar}
  \inferrule* [lab=monoidal-laws] {} { P|Q \equiv Q|P \and P|0 \equiv P \and P|(Q|R) \equiv (P|Q)|R }
\end{mathpar}

\begin{mathpar}
  \inferrule* [lab=alpha-equivalence] {} { (x)P \equiv (y)P\{y/x\} \and y \not\in \freenames{P} }
\end{mathpar}

\begin{definition}
Then two processes, $P,Q$, are alpha-equivalent if $P = Q\{\vec{y}/\vec{x}\}$ for
some $\vec{x} \in \boundnames{Q},\vec{y} \in \boundnames{P}$, where $Q\{\vec{y}/\vec{x}\}$
denotes the capture-avoiding substitution of $\vec{y}$ for $\vec{x}$ in $Q$.
\end{definition}

\begin{definition}
  The {\em structural congruence} \cite{SangiorgiWalker} , $\equiv$,
  between processes is the least congruence containing
  alpha-equivalence, satisfying the abelian monoid laws
  (associativity, commutativity and $\pzero$ as identity) for parallel
  composition $|$ and for summation $+$.
\end{definition}

\subsection{Name equivalence}

We take name equivalence, written $\nameeq$, to be the smallest
equivalence relation generated by the following rules.

\begin{mathpar}
\inferrule*[lab=Quote-drop]
{ }
{ \quotep{@{x}} \nameeq x }

\inferrule*[lab=Struct-equiv]
{ P \scong Q }
{ \quotep{P} \nameeq \quotep{Q} }
\end{mathpar}

The astute reader will have noticed that the mutual recursion of names
and processes imposes a mutual recursion on alpha-equivalence and
structural equivalence via name-equivalence. Fortunately, all of this
works out pleasantly and we may calculate in the natural way, free of
concern. The reader interested in the details is referred to the
appendix \ref{appendix:rho_details}.

\subsection{Substitution}

We use $\Proc$ for the set of processes, $\QProc$ for the set of
names, and $\id{\{}\vec{y} / \vec{x} \id{\}}$ to denote partial maps,
$s : \QProc \rightarrow \QProc$. A map, $s$ lifts, uniquely, to a map
on process terms, $\widehat{s} : \Proc \rightarrow \Proc$ by the
following equations.

\begin{mathpar}
  (0) \psubstp{Q}{P} := 0 \\
  (R \juxtap S) \psubstp{Q}{P}
  :=    
  (R)\psubstp{Q}{P} \juxtap (S) \psubstp{Q}{P} \\
  (x?(y).R) \psubstp{Q}{P}    
  :=    
  (x)\substp{Q}{P} (z)\concat( (R \psubstn{z}{y}) \psubstp{Q}{P} ) \\
  (\lift{x}{R}) \psubstp{Q}{P}  
  :=
  \lift{(x)\substp{Q}{P}}{ R \psubstp{Q}{P} } \\
%   (\dropn{x})  \psubstp{Q}{P}       
%   := 
%   \left\{ 
%     \begin{array}{ccc} 
%       \dropn{\quotep{Q}} & & x \nameeq \quotep{P} \\
%       \dropn{x} & & otherwise \\
%     \end{array}
%   \right. 
  (\dropn{x})  \psubstp{Q}{P}       
  := 
  \left\{ 
    \begin{array}{ccc} 
      Q & & x \nameeq \quotep{P} \\
      \dropn{x} & & otherwise \\
    \end{array}
  \right.
\end{mathpar}
 

where

\begin{eqnarray}
  (x)\id{\{} \lpquote Q \rpquote / \lpquote P \rpquote \id{\}}            = 
  \left\{ 
    \begin{array}{ccc}
      \lpquote Q \rpquote & & x \nameeq \lpquote P \rpquote \\
      x & & otherwise \\
    \end{array}
  \right. \nonumber
\end{eqnarray}

and $z$ is chosen distinct from $\quotep{P}$, $\quotep{Q}$, the free
names in $Q$, and all the names in $R$. Our $\alpha$-equivalence will
be built in the standard way from this substitution.

\begin{remark}\label{rem:no_self_referential_names}
  One consequence of these definitions is that $\forall P. \quotep{P}
  \not\in \freenames{P}$.
\end{remark}

\subsection{ Dynamic quote: an example }

Anticipating something of what's to come, consider applying the
substitution, $\widehat{\id{\{}u / z \id{\}}}$, to the following pair
of processes, $\lift{w}{y!(z)}$ and $w[ \lpquote y!(z) \rpquote ]$.

\begin{eqnarray}
	\lift{w}{y!(z)}\widehat{\id{\{}u / z \id{\}}}
		& = &
		\lift{w}{y!(u)} \nonumber\\
	w[ \lpquote y!(z) \rpquote ] \widehat{ \id{\{}u / z \id{\}} }
		& = &
		w[ \lpquote y!(z) \rpquote ] \nonumber
\end{eqnarray}

Because the body of the process between quotes is impervious to
substitution, we get radically different answers. In fact, by
examining the first process in an input context,
e.g. $x?(z).\lift{w}{y!(z)}$, we see that the process under the lift
operator may be shaped by prefixed inputs binding a name inside it. In
this sense, the lift operator will be seen as a way to dynamically
construct processes before reifying them as names.

Finally equipped with these standard features we can present the
dynamics of the calculus.

\subsubsection{Operational semantics} 

Finally, we introduce the computational dynamics. What marks these
algebras as distinct from other more traditionally studied algebraic
structures, e.g. vector spaces or polynomial rings, is the manner in
which dynamics is captured. In traditional structures, dynamics is typically
expressed through morphisms between such structures, as in linear maps
between vector spaces or morphisms between rings. In algebras
associated with the semantics of computation, the dynamics is
expressed as part of the algebraic structure itself, through a
reduction reduction relation typically denoted by $\red$. Below, we
give a recursive presentation of this relation for the calculus used
in the encoding.

$\red \subseteq \pi \times \pi$
$\red : \pi \to \mathcal{P}(\pi)$

\begin{mathpar}
  \inferrule* [lab=Comm] { \textsf{match}( x_{src}, x_{trgt} ) } { x_{trgt}?(y)P \; | \; x_{src}!\langle {Q} \rangle \red P\{\quotep{Q}/y}\} }
  \and \\
  \inferrule* [lab=Par] {{P} \red {P}'} {{{P} | {Q}} \red {{P}' | {Q}}}
  \and
  \inferrule* [lab=Equiv]{{{P} \scong {P}'} \andalso {{P}' \red {Q}'} \andalso {{Q}' \scong {Q}}}{{P} \red {Q}}
\end{mathpar}

\begin{eqnarray*}
  match_{\equiv} (\quotep{P},\quotep{Q}) & := & P \equiv Q \\
  match_{\dagger}(\quotep{P},\quotep{Q}) & := & \forall R. P|Q \red^{*} R => R \red^{*} 0 \\
  match_{K}(\quotep{P},\quotep{Q}) & := & K \mbox{ for some context } K
\end{eqnarray*}

$u?(x)P | u!\langle Q \rangle \red P\{\quotep{Q}/x\}$

%We write $\wred$ for $\red^*$, and $P\red$ if $\exists Q $ such that $ P \red Q$.
We write $P\red$ if $\exists Q $ such that $ P \red Q$ and $P\not\red$, otherwise.

\section{Replication}

As mentioned before, it is known that replication (and hence
recursion) can be implemented in a higher-order process algebra
\cite{SangiorgiWalker}. As our first example of calculation with the
machinery thus far presented we give the construction explicitly in
the {\rhoc}.

\begin{eqnarray}
	D_{x} & := & \prefix{x}{y}{(\binpar{\outputp{x}{y}}{@{y}})} \nonumber\\
	\bangp_{x}{P} & := & \binpar{{x}!\langle{\binpar{D_{x}}{P}}\rangle}{D_{x}} \nonumber
\end{eqnarray}

\begin{eqnarray}
	\bangp_{x}{P} & & \nonumber\\
	=
	& {x}!\langle{(\prefix{x}{y}{(\outputp{x}{y} | @{y})) | P}}\rangle 
	      | \prefix{x}{y}{(\outputp{x}{y} | @{y})} & \nonumber\\
	\red
	& (\outputp{x}{y} | @{y})\substn{\quotep{(\prefix{x}{y}{(@{y} | \outputp{x}{y})) | P}}}{y} & \nonumber\\
	=
	& \outputp{x}{\quotep{(\prefix{x}{y}{(\outputp{x}{y} | @{y})) | P}}}
	  | {(\prefix{x}{y}{(\outputp{x}{y} | @{y})) | P}} & \nonumber\\
	\red
	& \ldots & \nonumber\\
	\red^*
	& P | P | \ldots & \nonumber
\end{eqnarray}

Of course, this encoding, as an implementation, runs away, unfolding
$\bangp{P}$ eagerly. A lazier and more implementable replication
operator, restricted to input-guarded processes, may be obtained as follows.

\begin{eqnarray}
\bangp{\prefix{u}{v}{P}} 
	:= 
	\binpar{\lift{x}{\prefix{u}{v}{(\binpar{D(x)}{P})}}}{D(x)} \nonumber
\end{eqnarray}

\begin{remark}
  Note that the lazier definition still does not deal with summation
  or mixed summation (i.e. sums over input and output). The reader is
  invited to construct definitions of replication that deal with these
  features. 

  Further, the definitions are parameterized in a name, $x$. Can you,
  gentle reader, make a definition that eliminates this parameter and
  guarantees no accidental interaction between the replication
  machinery and the process being replicated -- i.e. no accidental
  sharing of names used by the process to get its work done and the
  name(s) used by the replication to effect copying. This latter
  revision of the definition of replication is crucial to obtaining
  the expected identity $!!P \sim !P$.
\end{remark}

\begin{remark}\label{rem:paradoxical_combinator}
  The reader familiar with the lambda calculus will have noticed the
  similarity between $D$ and the paradoxical combinator.

  [Ed. note: the existence of this seems to suggest we have to be more
  restrictive on the set of processes and names we admit if we are to
  support no-cloning.]
\end{remark}

\subsubsection{Bisimulation}

The computational dynamics gives rise to another kind of equivalence,
the equivalence of computational behavior. As previously mentioned
this is typically captured \emph{via} some form of bisimulation.

% The notion we use in this paper is weak barbed bisimulation
% \cite{milner91polyadicpi}.

The notion we use in this paper is derived from weak barbed
bisimulation \cite{milner91polyadicpi}. 

\begin{definition}
An \emph{observation relation}, $\downarrow_{\mathcal N}$, over a set
of names, $\mathcal N$, is the smallest relation satisfying the rules
below.

\infrule[Out-barb]{y \in {\mathcal N}, \; x \nameeq y}
		  {\outputp{x}{v} \downarrow_{\mathcal N} x}
\infrule[Par-barb]{\mbox{$P\downarrow_{\mathcal N} x$ or $Q\downarrow_{\mathcal N} x$}}
		  {\binpar{P}{Q} \downarrow_{\mathcal N} x}

We write $P \Downarrow_{\mathcal N} x$ if there is $Q$ such that 
$P \wred Q$ and $Q \downarrow_{\mathcal N} x$.
\end{definition}

\begin{definition}
%\label{def.bbisim}
An  ${\mathcal N}$-\emph{barbed bisimulation} over a set of names, ${\mathcal N}$, is a symmetric binary relation 
${\mathcal S}_{\mathcal N}$ between agents such that $P\rel{S}_{\mathcal N}Q$ implies:
\begin{enumerate}
\item If $P \red P'$ then $Q \wred Q'$ and $P'\rel{S}_{\mathcal N} Q'$.
\item If $P\downarrow_{\mathcal N} x$, then $Q\Downarrow_{\mathcal N} x$.
\end{enumerate}
$P$ is ${\mathcal N}$-barbed bisimilar to $Q$, written
$P \wbbisim_{\mathcal N} Q$, if $P \rel{S}_{\mathcal N} Q$ for some ${\mathcal N}$-barbed bisimulation ${\mathcal S}_{\mathcal N}$.
\end{definition}

$\mathcal{R} \subseteq \pi \times \pi$

$P \mathcal{R} Q => \forall P'. P \red P' \Rightarrow \exists Q'. Q \red Q', P' \mathcal{R} Q'$

$P \vdash x \Rightarrow Q \vdash x$

\begin{mathpar}
  \inferrule*[lab=Out-barb]{x \nameeq y}{{y}!\langle{Q}\rangle \vdash x}
  \and
  \inferrule*[lab=Par-barb]{\mbox{$P\vdash x$ or $Q\vdash x$}}{\binpar{P}{Q} \vdash x}
\end{mathpar}

\subsubsection{Contexts}

One of the principle advantages of computational calculi like the
$\pi$-calculus is a well-defined notion of context,
contextual-equivalence and a correlation between
contextual-equivalence and notions of bisimulation. The notion of
context allows the decomposition of a process into (sub-)process and
its syntactic environment, its context. Thus, a context may be
thought of as a process with a ``hole'' (written $\Box$) in it. The
application of a context $M$ to a process $P$, written $M[P]$, is
tantamount to filling the hole in $M$ with $P$. In this paper we do
not need the full weight of this theory, but do make use of the notion
of context in the proof the main theorem. 

\begin{mathpar}
  \inferrule* [lab=summation] {} {{M_{M},M_{N}} \bc \Box \;|\; x.M_{A} \;|\; M_{M}+M_{N}}
  \and
  \inferrule* [lab=agent] {} {{M_{A}} \bc (\vec{x})M_{P} \;| \; \clift{P_0,\ldots,M_{P},\ldots,P_N}}
  \and \\
  \inferrule* [lab=process] {} {{M_{P}} \bc M_{N} \;| \;P|M_{P} }
\end{mathpar} 

\begin{mathpar}
  \inferrule* [lab=sychronization] {} {M_{N} \bc \Box \;|\; x?M_{F} \;|\; x!M_{C}}
  \and
  \inferrule* [lab=abstraction] {} {{M_{F}} \bc (x)M_{P} }
  \and
  \inferrule* [lab=concretion] {} {{M_{C}} \bc \langle M_{P} \rangle }
  \and \\
  \inferrule* [lab=process] {} {{M_{P}} \bc M_{N} \;| \;P|M_{P} }
\end{mathpar}

\begin{definition}[contextual application] Given a context $M$, and
  process $P$, we define the \emph{contextual application}, $M[P] :=
  M\{P/\Box\}$. That is, the contextual application of M to P is the
  substitution of $P$ for $\Box$ in $M$.
\end{definition}

$\meaningof{-} : L \to \mathcal{P}(\pi)$

\begin{mathpar}
  \inferrule* [lab=collection] {} {\meaningof{true} = \pi, \and \meaningof{~E} = \pi \setminus \meaningof{E}, \and \meaningof{E_{1} \& E_{2}} = \meaningof{E_{1}} \cap \meaningof{E_{2}}}
\end{mathpar}

\begin{mathpar}
  \inferrule* [lab=structure] {} {\meaningof{0} = \{ P \in \pi | P \equiv 0 \}, \and \\ \meaningof{E_1 | E_2} = \{ P \in \pi | P \equiv P_{1} | P_{2}, P_{1} \in \meaningof{E_{1}}, P_{2} \in \meaningof{E_2}\} }
\end{mathpar}

\begin{mathpar}
 \inferrule* [lab=behavior] {} {\meaningof{\langle a?b \rangle E} = \{ P \in \pi | P \equiv Q | u?(y)P', \\ \and \\\\ \and \\ \;\;\; u \in \meaningof{a}, \forall z.P'\{z/y\} \in \meaningof{E\{z/b\}}\}, \and \\ \meaningof{a!E} = \{ P \in \pi | P \equiv Q | x!\langle P' \rangle, x \in \meaningof{a} P' \in \meaningof{E}\} }
\end{mathpar}

\begin{mathpar}
 \inferrule* [lab=nominal] {} {\meaningof{\quotep{E}} = \{ \quotep{P} \in \quotep{\pi} | P \in \meaningof{E} \}, \and \meaningof{\quotep{P}} = \{ \quotep{Q} \in \quotep{\pi} | P \equiv Q \} \and \\ \meaningof{@\quotep{E}} = \{ P \in \pi | P \equiv @x, x \in \meaningof{E} \}}
\end{mathpar}

\begin{eqnarray*}
  \\
  \meaningof{-} : TS \to ST
\end{eqnarray*}

\begin{eqnarray*}
  \\
  L : TS \to ST
\end{eqnarray*}

\begin{eqnarray*}
  \\
  P \models E \iff P \in \meaningof{E}
\end{eqnarray*}

\begin{eqnarray*}
  P \approx_{L} Q \iff \forall E \in L. P \models E \iff Q \models E
\end{eqnarray*}

\begin{eqnarray*}
  P \approx_{K} Q
\end{eqnarray*}

\begin{eqnarray*}
  P \approx Q
\end{eqnarray*}

$\approx_{K} = \approx = \approx_{L}$

\subsubsection{Contextual duality}

Note that contexts extend the quotation operation to a family of
operations from processes to names. Given a context, $M$, we can
define a \emph{nominal context}, $\quotep{M}$ by $\quotep{M}[P] :=
\quotep{M[P]}$. To foreshadow what is to come we observe that these
operations enjoy a duality with processes very much like the duality
between vectors and maps from vectors to scalars.

Further, because the calculus is essentially higher-order, we have a
correspondence between contexts and processes. More specifically,
given a name $x$ and a context $M$ we can construct $M^{*}_{x}$ such
that 

\begin{mathpar}
  M^{*}_{x} | \lift{x}{P} \red M[P]
\end{mathpar}

namely,

\begin{mathpar}
  M^{*}_{x} := x?(u).M[\dropn{u}]
\end{mathpar}

The dependence of $M^{*}_{x}$ on a name makes it an abstraction, 

\begin{mathpar}
  M^{*} := (x)x?(u).M[\dropn{u}]
\end{mathpar}

\subsection{Additional notation}

It will sometimes be convenient to denote the process a name
quotes. We already have the notation $x = \quotep{P}$, but it will be
convenient to introduce an alternate notation, $\procn{x}$, when we
want to emphasize the connection to the use of the name. Note that, by
virtue of name equivalence, $\quotep{\procn{x}} \nameeq x$; so, the
notation is consistent with previous definitions.

Further, because names have structure it is possible to effect
substitutions on the basis of that structure. This means we need to
upgrade our notation for substitutions, which we accomplish by
adapting comprehension notation. Thus,

\begin{mathpar}
  P\{ y / x : x \in S \}
\end{mathpar}

is interpreted to mean the process derived from P by replacing (in a
capture-avoiding manner) each occurrence of $x$ in $S$ by $y$. For example,

\begin{mathpar}
  P\{ \quotep{\procn{x}|\procn{x}} / x : x \in \freenames{P} \}
\end{mathpar}

will replace each (occurrence) of a free name $x$ in $P$ by
$\quotep{\procn{x}|\procn{x}}$.

Also, we will avail ourselves of the notation $x^{L}$ and $x^{R}$ to
denote injections of a name into disjoint copies of the name
space. There are numerous ways to accomplish this. One example can be
found in \cite{MeredithR05}. This notation overloads to vectors of
names: $\vec{x}^{\pi} := (x_{i}^{\pi} \; : \; 0 \leq i < |\vec{x}| )$ where $\pi \in \{L,R\}$.

We also use $P^{\Box} := P|\Box$.

In \cite{MeredithR05} an interpretation of the new operator is
given. It turns out that there are several possible interpretations
all enjoying the requisite algebraic properties of the operator (see
\cite{milner91polyadicpi}). We will therefore make liberal use of
$(\nu\; \vec{x})P$.

% subsection the_syntax_and_semantics_of_the_notation_system (end)   

\input{qm2pi.qmops} 

\input{qm2pi.sterngerlach} 

\input{qm2pi.metric} 

% section concurrent_process_calculi (end)

%\input{qm2pi.proofsketch}

% section proof sketch (end)

%\input{qm2pi.slviaknots} 

% section spatial logic via knots (end)

\input{qm2pi.conclusion}

% section conclusion (end)

%\input{qm2pi.dtcodes} 

% section wiring algorithm (end)

\input{qm2pi.ack} 

% section acknowledgments (end)

\newpage


\bibliographystyle{plain}   
\bibliography{../../biblios/main.bib}

\input{qm2pi.rhodetails}

\end{document}

 

\documentclass[12pt]{llncs}
%\documentclass{jktr}

\usepackage[pdftex]{hyperref}                   
\usepackage {listings}
\usepackage {mathpartir}
\usepackage{bcprules}
%\usepackage{listings}
                       
\usepackage{graphicx} 
%\usepackage[margins=2.5cm,nohead,nofoot]{geometry}
%\usepackage{geometry}
\usepackage{amsfonts}
\usepackage{amstext}
\usepackage{latexsym}
\usepackage{amssymb}
\usepackage{color}


%\include{myPreamble}
\include{qm2pi.local} 

%\ifpdf
%\usepackage[pdftex]{graphicx}
%\else
%\usepackage{graphicx}
%\fi

 % \ifpdf
%  \usepackage{pdfsync}
%  \if


%\title{Brief Article}
%\author{David F. Snyder}
%\author{L.G. Meredith}

%\address{Dept. of Math., Texas State University--San Marcos, San Marcos, TX 78666}
       
\pagestyle{empty}


\begin{document}

\lstset{language=[Objective]Caml,frame=shadowbox}

\input{qm2pi.front}

% section front matter (end)

\input{qm2pi.intro} 
 
% section introduction (end)

% \input{qm2pi.knotations} 

% section notation (end)

\input{qm2pi.process.calculi} 

% section concurrent_process_calculi_and_spatial_logics_ (end)
    
%\input{qm2pi.knots2pi} 

%\input{qm2pi.trefoil} 

%\input{qm2pi.mainthm} 

% subsection basic_interpretation (end)

%\input{qm2pi.rho.presentation} 
\subsection{The syntax and semantics of the notation system}\label{sub:the_syntax_and_semantics_of_the_notation_system} % (fold)

We now summarize a technical presentation of the calculus that
embodies our theory of dynamics. The typical presentation of such a
calculus follows the style of giving generators and relations on
them. The grammar, below, describing term constructors, freely
generates the set of processes, $\Proc$. This set is then quotiented
by a relation known as structural congruence and it is over this set
that the notion of dynamics is expressed. This presentation is
essentially that of \cite{MeredithR05} with the addition of
polyadicity and summation. For readability we have relegated some of
the technical subtleties to an appendix.

\subsubsection{Process grammar}\label{subsub:process_grammar}

\begin{mathpar}
  \inferrule* [lab=synchronization] {} {{M} \bc \pzero \;|\; x?F \;|\; x!C }
  \and
  \inferrule* [lab=abstraction] {} {{F} \bc (x)P}
  \and
  \inferrule* [lab=concretion] {} {{C} \bc \langle Q \rangle}
  \and
  \inferrule* [lab=process] {} {{P,Q} \bc M \;| \;P|Q \;|\; @{x}}
  \and
  \inferrule* [lab=name] {} {{x} \bc \quotep{P}}
\end{mathpar} 

Note that $\vec{x}$ (resp. $\vec{P}$) denotes a vector of names
(resp. processes) of length $|\vec{x}|$ (resp. $|\vec{P}|$). We adopt
the following useful abbreviations.

\begin{mathpar}
   x?(\vec{y}).P := x.(\vec{y})P \and  x\clift{\vec{P}} := x.\clift{\vec{P}}
   \and x!(y) := \lift{x}{\dropn{y}}
   \and \Pi_{i=0}^{n-1}P_i := P_0 | \ldots | P_{n-1}
\end{mathpar}

\subsubsection{Structural congruence}

\paragraph{Free and bound names and alpha-equivalence.} At the
core of structural equivalence is alpha-equivalence which identifies
process that are the same up to a change of variable. Formally, we
recognize the distinction between free and bound names. The free names
of a process, $\freenames{P}$, may be calculated recursively as
follows:

\begin{mathpar}
\freenames{\pzero} := \emptyset
  \and \\
  \freenames{x?(y).P} := \{ x \} \cup (\freenames{P} \setminus \{ y \})
  \and 
  \freenames{x!\langle P \rangle} := \{ x \} \cup \{ P \} 
  \and \\
  \freenames{P|Q} := \freenames{P} \cup \freenames{Q}
  \and \\
  \freenames{@{x}} := \{ x \}
\end{mathpar}

$\pi$
$\quotep{\pi}$

$\freenames{-} : \pi \to \mathcal{P}(\quotep{\pi})$

\begin{eqnarray*}
  \freenames{\pzero} & := & \emptyset \\
  \freenames{x?(y).P} & := & \{ x \} \cup (\freenames{P} \setminus \{ y \}) \\
  \freenames{x!\langle P \rangle} & := & \{ x \} \cup \{ P \} \\
  \freenames{P|Q} & := & \freenames{P} \cup \freenames{Q} \\
  \freenames{\dropn{x}} & := & \{ x \}
\end{eqnarray*}

The bound names of a process, $\boundnames{P}$, are those names occurring in $P$
that are not free. For example, in $x?(y).0$, the name $x$ is free, while $y$ is bound.

\begin{mathpar}
  \inferrule* [lab=monoidal-laws] {} { P|Q \equiv Q|P \and P|0 \equiv P \and P|(Q|R) \equiv (P|Q)|R }
\end{mathpar}

\begin{mathpar}
  \inferrule* [lab=alpha-equivalence] {} { (x)P \equiv (y)P\{y/x\} \and y \not\in \freenames{P} }
\end{mathpar}

\begin{definition}
Then two processes, $P,Q$, are alpha-equivalent if $P = Q\{\vec{y}/\vec{x}\}$ for
some $\vec{x} \in \boundnames{Q},\vec{y} \in \boundnames{P}$, where $Q\{\vec{y}/\vec{x}\}$
denotes the capture-avoiding substitution of $\vec{y}$ for $\vec{x}$ in $Q$.
\end{definition}

\begin{definition}
  The {\em structural congruence} \cite{SangiorgiWalker} , $\equiv$,
  between processes is the least congruence containing
  alpha-equivalence, satisfying the abelian monoid laws
  (associativity, commutativity and $\pzero$ as identity) for parallel
  composition $|$ and for summation $+$.
\end{definition}

\subsection{Name equivalence}

We take name equivalence, written $\nameeq$, to be the smallest
equivalence relation generated by the following rules.

\begin{mathpar}
\inferrule*[lab=Quote-drop]
{ }
{ \quotep{@{x}} \nameeq x }

\inferrule*[lab=Struct-equiv]
{ P \scong Q }
{ \quotep{P} \nameeq \quotep{Q} }
\end{mathpar}

The astute reader will have noticed that the mutual recursion of names
and processes imposes a mutual recursion on alpha-equivalence and
structural equivalence via name-equivalence. Fortunately, all of this
works out pleasantly and we may calculate in the natural way, free of
concern. The reader interested in the details is referred to the
appendix \ref{appendix:rho_details}.

\subsection{Substitution}

We use $\Proc$ for the set of processes, $\QProc$ for the set of
names, and $\id{\{}\vec{y} / \vec{x} \id{\}}$ to denote partial maps,
$s : \QProc \rightarrow \QProc$. A map, $s$ lifts, uniquely, to a map
on process terms, $\widehat{s} : \Proc \rightarrow \Proc$ by the
following equations.

\begin{mathpar}
  (0) \psubstp{Q}{P} := 0 \\
  (R \juxtap S) \psubstp{Q}{P}
  :=    
  (R)\psubstp{Q}{P} \juxtap (S) \psubstp{Q}{P} \\
  (x?(y).R) \psubstp{Q}{P}    
  :=    
  (x)\substp{Q}{P} (z)\concat( (R \psubstn{z}{y}) \psubstp{Q}{P} ) \\
  (\lift{x}{R}) \psubstp{Q}{P}  
  :=
  \lift{(x)\substp{Q}{P}}{ R \psubstp{Q}{P} } \\
%   (\dropn{x})  \psubstp{Q}{P}       
%   := 
%   \left\{ 
%     \begin{array}{ccc} 
%       \dropn{\quotep{Q}} & & x \nameeq \quotep{P} \\
%       \dropn{x} & & otherwise \\
%     \end{array}
%   \right. 
  (\dropn{x})  \psubstp{Q}{P}       
  := 
  \left\{ 
    \begin{array}{ccc} 
      Q & & x \nameeq \quotep{P} \\
      \dropn{x} & & otherwise \\
    \end{array}
  \right.
\end{mathpar}
 

where

\begin{eqnarray}
  (x)\id{\{} \lpquote Q \rpquote / \lpquote P \rpquote \id{\}}            = 
  \left\{ 
    \begin{array}{ccc}
      \lpquote Q \rpquote & & x \nameeq \lpquote P \rpquote \\
      x & & otherwise \\
    \end{array}
  \right. \nonumber
\end{eqnarray}

and $z$ is chosen distinct from $\quotep{P}$, $\quotep{Q}$, the free
names in $Q$, and all the names in $R$. Our $\alpha$-equivalence will
be built in the standard way from this substitution.

\begin{remark}\label{rem:no_self_referential_names}
  One consequence of these definitions is that $\forall P. \quotep{P}
  \not\in \freenames{P}$.
\end{remark}

\subsection{ Dynamic quote: an example }

Anticipating something of what's to come, consider applying the
substitution, $\widehat{\id{\{}u / z \id{\}}}$, to the following pair
of processes, $\lift{w}{y!(z)}$ and $w[ \lpquote y!(z) \rpquote ]$.

\begin{eqnarray}
	\lift{w}{y!(z)}\widehat{\id{\{}u / z \id{\}}}
		& = &
		\lift{w}{y!(u)} \nonumber\\
	w[ \lpquote y!(z) \rpquote ] \widehat{ \id{\{}u / z \id{\}} }
		& = &
		w[ \lpquote y!(z) \rpquote ] \nonumber
\end{eqnarray}

Because the body of the process between quotes is impervious to
substitution, we get radically different answers. In fact, by
examining the first process in an input context,
e.g. $x?(z).\lift{w}{y!(z)}$, we see that the process under the lift
operator may be shaped by prefixed inputs binding a name inside it. In
this sense, the lift operator will be seen as a way to dynamically
construct processes before reifying them as names.

Finally equipped with these standard features we can present the
dynamics of the calculus.

\subsubsection{Operational semantics} 

Finally, we introduce the computational dynamics. What marks these
algebras as distinct from other more traditionally studied algebraic
structures, e.g. vector spaces or polynomial rings, is the manner in
which dynamics is captured. In traditional structures, dynamics is typically
expressed through morphisms between such structures, as in linear maps
between vector spaces or morphisms between rings. In algebras
associated with the semantics of computation, the dynamics is
expressed as part of the algebraic structure itself, through a
reduction reduction relation typically denoted by $\red$. Below, we
give a recursive presentation of this relation for the calculus used
in the encoding.

$\red \subseteq \pi \times \pi$
$\red : \pi \to \mathcal{P}(\pi)$

\begin{mathpar}
  \inferrule* [lab=Comm] { \textsf{match}( x_{src}, x_{trgt} ) } { x_{trgt}?(y)P \; | \; x_{src}!\langle {Q} \rangle \red P\{\quotep{Q}/y}\} }
  \and \\
  \inferrule* [lab=Par] {{P} \red {P}'} {{{P} | {Q}} \red {{P}' | {Q}}}
  \and
  \inferrule* [lab=Equiv]{{{P} \scong {P}'} \andalso {{P}' \red {Q}'} \andalso {{Q}' \scong {Q}}}{{P} \red {Q}}
\end{mathpar}

\begin{eqnarray*}
  match_{\equiv} (\quotep{P},\quotep{Q}) & := & P \equiv Q \\
  match_{\dagger}(\quotep{P},\quotep{Q}) & := & \forall R. P|Q \red^{*} R => R \red^{*} 0 \\
  match_{K}(\quotep{P},\quotep{Q}) & := & K \mbox{ for some context } K
\end{eqnarray*}

$u?(x)P | u!\langle Q \rangle \red P\{\quotep{Q}/x\}$

%We write $\wred$ for $\red^*$, and $P\red$ if $\exists Q $ such that $ P \red Q$.
We write $P\red$ if $\exists Q $ such that $ P \red Q$ and $P\not\red$, otherwise.

\section{Replication}

As mentioned before, it is known that replication (and hence
recursion) can be implemented in a higher-order process algebra
\cite{SangiorgiWalker}. As our first example of calculation with the
machinery thus far presented we give the construction explicitly in
the {\rhoc}.

\begin{eqnarray}
	D_{x} & := & \prefix{x}{y}{(\binpar{\outputp{x}{y}}{@{y}})} \nonumber\\
	\bangp_{x}{P} & := & \binpar{{x}!\langle{\binpar{D_{x}}{P}}\rangle}{D_{x}} \nonumber
\end{eqnarray}

\begin{eqnarray}
	\bangp_{x}{P} & & \nonumber\\
	=
	& {x}!\langle{(\prefix{x}{y}{(\outputp{x}{y} | @{y})) | P}}\rangle 
	      | \prefix{x}{y}{(\outputp{x}{y} | @{y})} & \nonumber\\
	\red
	& (\outputp{x}{y} | @{y})\substn{\quotep{(\prefix{x}{y}{(@{y} | \outputp{x}{y})) | P}}}{y} & \nonumber\\
	=
	& \outputp{x}{\quotep{(\prefix{x}{y}{(\outputp{x}{y} | @{y})) | P}}}
	  | {(\prefix{x}{y}{(\outputp{x}{y} | @{y})) | P}} & \nonumber\\
	\red
	& \ldots & \nonumber\\
	\red^*
	& P | P | \ldots & \nonumber
\end{eqnarray}

Of course, this encoding, as an implementation, runs away, unfolding
$\bangp{P}$ eagerly. A lazier and more implementable replication
operator, restricted to input-guarded processes, may be obtained as follows.

\begin{eqnarray}
\bangp{\prefix{u}{v}{P}} 
	:= 
	\binpar{\lift{x}{\prefix{u}{v}{(\binpar{D(x)}{P})}}}{D(x)} \nonumber
\end{eqnarray}

\begin{remark}
  Note that the lazier definition still does not deal with summation
  or mixed summation (i.e. sums over input and output). The reader is
  invited to construct definitions of replication that deal with these
  features. 

  Further, the definitions are parameterized in a name, $x$. Can you,
  gentle reader, make a definition that eliminates this parameter and
  guarantees no accidental interaction between the replication
  machinery and the process being replicated -- i.e. no accidental
  sharing of names used by the process to get its work done and the
  name(s) used by the replication to effect copying. This latter
  revision of the definition of replication is crucial to obtaining
  the expected identity $!!P \sim !P$.
\end{remark}

\begin{remark}\label{rem:paradoxical_combinator}
  The reader familiar with the lambda calculus will have noticed the
  similarity between $D$ and the paradoxical combinator.

  [Ed. note: the existence of this seems to suggest we have to be more
  restrictive on the set of processes and names we admit if we are to
  support no-cloning.]
\end{remark}

\subsubsection{Bisimulation}

The computational dynamics gives rise to another kind of equivalence,
the equivalence of computational behavior. As previously mentioned
this is typically captured \emph{via} some form of bisimulation.

% The notion we use in this paper is weak barbed bisimulation
% \cite{milner91polyadicpi}.

The notion we use in this paper is derived from weak barbed
bisimulation \cite{milner91polyadicpi}. 

\begin{definition}
An \emph{observation relation}, $\downarrow_{\mathcal N}$, over a set
of names, $\mathcal N$, is the smallest relation satisfying the rules
below.

\infrule[Out-barb]{y \in {\mathcal N}, \; x \nameeq y}
		  {\outputp{x}{v} \downarrow_{\mathcal N} x}
\infrule[Par-barb]{\mbox{$P\downarrow_{\mathcal N} x$ or $Q\downarrow_{\mathcal N} x$}}
		  {\binpar{P}{Q} \downarrow_{\mathcal N} x}

We write $P \Downarrow_{\mathcal N} x$ if there is $Q$ such that 
$P \wred Q$ and $Q \downarrow_{\mathcal N} x$.
\end{definition}

\begin{definition}
%\label{def.bbisim}
An  ${\mathcal N}$-\emph{barbed bisimulation} over a set of names, ${\mathcal N}$, is a symmetric binary relation 
${\mathcal S}_{\mathcal N}$ between agents such that $P\rel{S}_{\mathcal N}Q$ implies:
\begin{enumerate}
\item If $P \red P'$ then $Q \wred Q'$ and $P'\rel{S}_{\mathcal N} Q'$.
\item If $P\downarrow_{\mathcal N} x$, then $Q\Downarrow_{\mathcal N} x$.
\end{enumerate}
$P$ is ${\mathcal N}$-barbed bisimilar to $Q$, written
$P \wbbisim_{\mathcal N} Q$, if $P \rel{S}_{\mathcal N} Q$ for some ${\mathcal N}$-barbed bisimulation ${\mathcal S}_{\mathcal N}$.
\end{definition}

$\mathcal{R} \subseteq \pi \times \pi$

$P \mathcal{R} Q => \forall P'. P \red P' \Rightarrow \exists Q'. Q \red Q', P' \mathcal{R} Q'$

$P \vdash x \Rightarrow Q \vdash x$

\begin{mathpar}
  \inferrule*[lab=Out-barb]{x \nameeq y}{{y}!\langle{Q}\rangle \vdash x}
  \and
  \inferrule*[lab=Par-barb]{\mbox{$P\vdash x$ or $Q\vdash x$}}{\binpar{P}{Q} \vdash x}
\end{mathpar}

\subsubsection{Contexts}

One of the principle advantages of computational calculi like the
$\pi$-calculus is a well-defined notion of context,
contextual-equivalence and a correlation between
contextual-equivalence and notions of bisimulation. The notion of
context allows the decomposition of a process into (sub-)process and
its syntactic environment, its context. Thus, a context may be
thought of as a process with a ``hole'' (written $\Box$) in it. The
application of a context $M$ to a process $P$, written $M[P]$, is
tantamount to filling the hole in $M$ with $P$. In this paper we do
not need the full weight of this theory, but do make use of the notion
of context in the proof the main theorem. 

\begin{mathpar}
  \inferrule* [lab=summation] {} {{M_{M},M_{N}} \bc \Box \;|\; x.M_{A} \;|\; M_{M}+M_{N}}
  \and
  \inferrule* [lab=agent] {} {{M_{A}} \bc (\vec{x})M_{P} \;| \; \clift{P_0,\ldots,M_{P},\ldots,P_N}}
  \and \\
  \inferrule* [lab=process] {} {{M_{P}} \bc M_{N} \;| \;P|M_{P} }
\end{mathpar} 

\begin{mathpar}
  \inferrule* [lab=sychronization] {} {M_{N} \bc \Box \;|\; x?M_{F} \;|\; x!M_{C}}
  \and
  \inferrule* [lab=abstraction] {} {{M_{F}} \bc (x)M_{P} }
  \and
  \inferrule* [lab=concretion] {} {{M_{C}} \bc \langle M_{P} \rangle }
  \and \\
  \inferrule* [lab=process] {} {{M_{P}} \bc M_{N} \;| \;P|M_{P} }
\end{mathpar}

\begin{definition}[contextual application] Given a context $M$, and
  process $P$, we define the \emph{contextual application}, $M[P] :=
  M\{P/\Box\}$. That is, the contextual application of M to P is the
  substitution of $P$ for $\Box$ in $M$.
\end{definition}

$\meaningof{-} : L \to \mathcal{P}(\pi)$

\begin{mathpar}
  \inferrule* [lab=collection] {} {\meaningof{true} = \pi, \and \meaningof{~E} = \pi \setminus \meaningof{E}, \and \meaningof{E_{1} \& E_{2}} = \meaningof{E_{1}} \cap \meaningof{E_{2}}}
\end{mathpar}

\begin{mathpar}
  \inferrule* [lab=structure] {} {\meaningof{0} = \{ P \in \pi | P \equiv 0 \}, \and \\ \meaningof{E_1 | E_2} = \{ P \in \pi | P \equiv P_{1} | P_{2}, P_{1} \in \meaningof{E_{1}}, P_{2} \in \meaningof{E_2}\} }
\end{mathpar}

\begin{mathpar}
 \inferrule* [lab=behavior] {} {\meaningof{\langle a?b \rangle E} = \{ P \in \pi | P \equiv Q | u?(y)P', \\ \and \\\\ \and \\ \;\;\; u \in \meaningof{a}, \forall z.P'\{z/y\} \in \meaningof{E\{z/b\}}\}, \and \\ \meaningof{a!E} = \{ P \in \pi | P \equiv Q | x!\langle P' \rangle, x \in \meaningof{a} P' \in \meaningof{E}\} }
\end{mathpar}

\begin{mathpar}
 \inferrule* [lab=nominal] {} {\meaningof{\quotep{E}} = \{ \quotep{P} \in \quotep{\pi} | P \in \meaningof{E} \}, \and \meaningof{\quotep{P}} = \{ \quotep{Q} \in \quotep{\pi} | P \equiv Q \} \and \\ \meaningof{@\quotep{E}} = \{ P \in \pi | P \equiv @x, x \in \meaningof{E} \}}
\end{mathpar}

\begin{eqnarray*}
  \\
  \meaningof{-} : TS \to ST
\end{eqnarray*}

\begin{eqnarray*}
  \\
  L : TS \to ST
\end{eqnarray*}

\begin{eqnarray*}
  \\
  P \models E \iff P \in \meaningof{E}
\end{eqnarray*}

\begin{eqnarray*}
  P \approx_{L} Q \iff \forall E \in L. P \models E \iff Q \models E
\end{eqnarray*}

\begin{eqnarray*}
  P \approx_{K} Q
\end{eqnarray*}

\begin{eqnarray*}
  P \approx Q
\end{eqnarray*}

$\approx_{K} = \approx = \approx_{L}$

\subsubsection{Contextual duality}

Note that contexts extend the quotation operation to a family of
operations from processes to names. Given a context, $M$, we can
define a \emph{nominal context}, $\quotep{M}$ by $\quotep{M}[P] :=
\quotep{M[P]}$. To foreshadow what is to come we observe that these
operations enjoy a duality with processes very much like the duality
between vectors and maps from vectors to scalars.

Further, because the calculus is essentially higher-order, we have a
correspondence between contexts and processes. More specifically,
given a name $x$ and a context $M$ we can construct $M^{*}_{x}$ such
that 

\begin{mathpar}
  M^{*}_{x} | \lift{x}{P} \red M[P]
\end{mathpar}

namely,

\begin{mathpar}
  M^{*}_{x} := x?(u).M[\dropn{u}]
\end{mathpar}

The dependence of $M^{*}_{x}$ on a name makes it an abstraction, 

\begin{mathpar}
  M^{*} := (x)x?(u).M[\dropn{u}]
\end{mathpar}

\subsection{Additional notation}

It will sometimes be convenient to denote the process a name
quotes. We already have the notation $x = \quotep{P}$, but it will be
convenient to introduce an alternate notation, $\procn{x}$, when we
want to emphasize the connection to the use of the name. Note that, by
virtue of name equivalence, $\quotep{\procn{x}} \nameeq x$; so, the
notation is consistent with previous definitions.

Further, because names have structure it is possible to effect
substitutions on the basis of that structure. This means we need to
upgrade our notation for substitutions, which we accomplish by
adapting comprehension notation. Thus,

\begin{mathpar}
  P\{ y / x : x \in S \}
\end{mathpar}

is interpreted to mean the process derived from P by replacing (in a
capture-avoiding manner) each occurrence of $x$ in $S$ by $y$. For example,

\begin{mathpar}
  P\{ \quotep{\procn{x}|\procn{x}} / x : x \in \freenames{P} \}
\end{mathpar}

will replace each (occurrence) of a free name $x$ in $P$ by
$\quotep{\procn{x}|\procn{x}}$.

Also, we will avail ourselves of the notation $x^{L}$ and $x^{R}$ to
denote injections of a name into disjoint copies of the name
space. There are numerous ways to accomplish this. One example can be
found in \cite{MeredithR05}. This notation overloads to vectors of
names: $\vec{x}^{\pi} := (x_{i}^{\pi} \; : \; 0 \leq i < |\vec{x}| )$ where $\pi \in \{L,R\}$.

We also use $P^{\Box} := P|\Box$.

In \cite{MeredithR05} an interpretation of the new operator is
given. It turns out that there are several possible interpretations
all enjoying the requisite algebraic properties of the operator (see
\cite{milner91polyadicpi}). We will therefore make liberal use of
$(\nu\; \vec{x})P$.

% subsection the_syntax_and_semantics_of_the_notation_system (end)   

\input{qm2pi.qmops} 

\input{qm2pi.sterngerlach} 

\input{qm2pi.metric} 

% section concurrent_process_calculi (end)

%\input{qm2pi.proofsketch}

% section proof sketch (end)

%\input{qm2pi.slviaknots} 

% section spatial logic via knots (end)

\input{qm2pi.conclusion}

% section conclusion (end)

%\input{qm2pi.dtcodes} 

% section wiring algorithm (end)

\input{qm2pi.ack} 

% section acknowledgments (end)

\newpage


\bibliographystyle{plain}   
\bibliography{../../biblios/main.bib}

\input{qm2pi.rhodetails}

\end{document}

 

% section concurrent_process_calculi (end)

%\documentclass[12pt]{llncs}
%\documentclass{jktr}

\usepackage[pdftex]{hyperref}                   
\usepackage {listings}
\usepackage {mathpartir}
\usepackage{bcprules}
%\usepackage{listings}
                       
\usepackage{graphicx} 
%\usepackage[margins=2.5cm,nohead,nofoot]{geometry}
%\usepackage{geometry}
\usepackage{amsfonts}
\usepackage{amstext}
\usepackage{latexsym}
\usepackage{amssymb}
\usepackage{color}


%\include{myPreamble}
\include{qm2pi.local} 

%\ifpdf
%\usepackage[pdftex]{graphicx}
%\else
%\usepackage{graphicx}
%\fi

 % \ifpdf
%  \usepackage{pdfsync}
%  \if


%\title{Brief Article}
%\author{David F. Snyder}
%\author{L.G. Meredith}

%\address{Dept. of Math., Texas State University--San Marcos, San Marcos, TX 78666}
       
\pagestyle{empty}


\begin{document}

\lstset{language=[Objective]Caml,frame=shadowbox}

\input{qm2pi.front}

% section front matter (end)

\input{qm2pi.intro} 
 
% section introduction (end)

% \input{qm2pi.knotations} 

% section notation (end)

\input{qm2pi.process.calculi} 

% section concurrent_process_calculi_and_spatial_logics_ (end)
    
%\input{qm2pi.knots2pi} 

%\input{qm2pi.trefoil} 

%\input{qm2pi.mainthm} 

% subsection basic_interpretation (end)

%\input{qm2pi.rho.presentation} 
\subsection{The syntax and semantics of the notation system}\label{sub:the_syntax_and_semantics_of_the_notation_system} % (fold)

We now summarize a technical presentation of the calculus that
embodies our theory of dynamics. The typical presentation of such a
calculus follows the style of giving generators and relations on
them. The grammar, below, describing term constructors, freely
generates the set of processes, $\Proc$. This set is then quotiented
by a relation known as structural congruence and it is over this set
that the notion of dynamics is expressed. This presentation is
essentially that of \cite{MeredithR05} with the addition of
polyadicity and summation. For readability we have relegated some of
the technical subtleties to an appendix.

\subsubsection{Process grammar}\label{subsub:process_grammar}

\begin{mathpar}
  \inferrule* [lab=synchronization] {} {{M} \bc \pzero \;|\; x?F \;|\; x!C }
  \and
  \inferrule* [lab=abstraction] {} {{F} \bc (x)P}
  \and
  \inferrule* [lab=concretion] {} {{C} \bc \langle Q \rangle}
  \and
  \inferrule* [lab=process] {} {{P,Q} \bc M \;| \;P|Q \;|\; @{x}}
  \and
  \inferrule* [lab=name] {} {{x} \bc \quotep{P}}
\end{mathpar} 

Note that $\vec{x}$ (resp. $\vec{P}$) denotes a vector of names
(resp. processes) of length $|\vec{x}|$ (resp. $|\vec{P}|$). We adopt
the following useful abbreviations.

\begin{mathpar}
   x?(\vec{y}).P := x.(\vec{y})P \and  x\clift{\vec{P}} := x.\clift{\vec{P}}
   \and x!(y) := \lift{x}{\dropn{y}}
   \and \Pi_{i=0}^{n-1}P_i := P_0 | \ldots | P_{n-1}
\end{mathpar}

\subsubsection{Structural congruence}

\paragraph{Free and bound names and alpha-equivalence.} At the
core of structural equivalence is alpha-equivalence which identifies
process that are the same up to a change of variable. Formally, we
recognize the distinction between free and bound names. The free names
of a process, $\freenames{P}$, may be calculated recursively as
follows:

\begin{mathpar}
\freenames{\pzero} := \emptyset
  \and \\
  \freenames{x?(y).P} := \{ x \} \cup (\freenames{P} \setminus \{ y \})
  \and 
  \freenames{x!\langle P \rangle} := \{ x \} \cup \{ P \} 
  \and \\
  \freenames{P|Q} := \freenames{P} \cup \freenames{Q}
  \and \\
  \freenames{@{x}} := \{ x \}
\end{mathpar}

$\pi$
$\quotep{\pi}$

$\freenames{-} : \pi \to \mathcal{P}(\quotep{\pi})$

\begin{eqnarray*}
  \freenames{\pzero} & := & \emptyset \\
  \freenames{x?(y).P} & := & \{ x \} \cup (\freenames{P} \setminus \{ y \}) \\
  \freenames{x!\langle P \rangle} & := & \{ x \} \cup \{ P \} \\
  \freenames{P|Q} & := & \freenames{P} \cup \freenames{Q} \\
  \freenames{\dropn{x}} & := & \{ x \}
\end{eqnarray*}

The bound names of a process, $\boundnames{P}$, are those names occurring in $P$
that are not free. For example, in $x?(y).0$, the name $x$ is free, while $y$ is bound.

\begin{mathpar}
  \inferrule* [lab=monoidal-laws] {} { P|Q \equiv Q|P \and P|0 \equiv P \and P|(Q|R) \equiv (P|Q)|R }
\end{mathpar}

\begin{mathpar}
  \inferrule* [lab=alpha-equivalence] {} { (x)P \equiv (y)P\{y/x\} \and y \not\in \freenames{P} }
\end{mathpar}

\begin{definition}
Then two processes, $P,Q$, are alpha-equivalent if $P = Q\{\vec{y}/\vec{x}\}$ for
some $\vec{x} \in \boundnames{Q},\vec{y} \in \boundnames{P}$, where $Q\{\vec{y}/\vec{x}\}$
denotes the capture-avoiding substitution of $\vec{y}$ for $\vec{x}$ in $Q$.
\end{definition}

\begin{definition}
  The {\em structural congruence} \cite{SangiorgiWalker} , $\equiv$,
  between processes is the least congruence containing
  alpha-equivalence, satisfying the abelian monoid laws
  (associativity, commutativity and $\pzero$ as identity) for parallel
  composition $|$ and for summation $+$.
\end{definition}

\subsection{Name equivalence}

We take name equivalence, written $\nameeq$, to be the smallest
equivalence relation generated by the following rules.

\begin{mathpar}
\inferrule*[lab=Quote-drop]
{ }
{ \quotep{@{x}} \nameeq x }

\inferrule*[lab=Struct-equiv]
{ P \scong Q }
{ \quotep{P} \nameeq \quotep{Q} }
\end{mathpar}

The astute reader will have noticed that the mutual recursion of names
and processes imposes a mutual recursion on alpha-equivalence and
structural equivalence via name-equivalence. Fortunately, all of this
works out pleasantly and we may calculate in the natural way, free of
concern. The reader interested in the details is referred to the
appendix \ref{appendix:rho_details}.

\subsection{Substitution}

We use $\Proc$ for the set of processes, $\QProc$ for the set of
names, and $\id{\{}\vec{y} / \vec{x} \id{\}}$ to denote partial maps,
$s : \QProc \rightarrow \QProc$. A map, $s$ lifts, uniquely, to a map
on process terms, $\widehat{s} : \Proc \rightarrow \Proc$ by the
following equations.

\begin{mathpar}
  (0) \psubstp{Q}{P} := 0 \\
  (R \juxtap S) \psubstp{Q}{P}
  :=    
  (R)\psubstp{Q}{P} \juxtap (S) \psubstp{Q}{P} \\
  (x?(y).R) \psubstp{Q}{P}    
  :=    
  (x)\substp{Q}{P} (z)\concat( (R \psubstn{z}{y}) \psubstp{Q}{P} ) \\
  (\lift{x}{R}) \psubstp{Q}{P}  
  :=
  \lift{(x)\substp{Q}{P}}{ R \psubstp{Q}{P} } \\
%   (\dropn{x})  \psubstp{Q}{P}       
%   := 
%   \left\{ 
%     \begin{array}{ccc} 
%       \dropn{\quotep{Q}} & & x \nameeq \quotep{P} \\
%       \dropn{x} & & otherwise \\
%     \end{array}
%   \right. 
  (\dropn{x})  \psubstp{Q}{P}       
  := 
  \left\{ 
    \begin{array}{ccc} 
      Q & & x \nameeq \quotep{P} \\
      \dropn{x} & & otherwise \\
    \end{array}
  \right.
\end{mathpar}
 

where

\begin{eqnarray}
  (x)\id{\{} \lpquote Q \rpquote / \lpquote P \rpquote \id{\}}            = 
  \left\{ 
    \begin{array}{ccc}
      \lpquote Q \rpquote & & x \nameeq \lpquote P \rpquote \\
      x & & otherwise \\
    \end{array}
  \right. \nonumber
\end{eqnarray}

and $z$ is chosen distinct from $\quotep{P}$, $\quotep{Q}$, the free
names in $Q$, and all the names in $R$. Our $\alpha$-equivalence will
be built in the standard way from this substitution.

\begin{remark}\label{rem:no_self_referential_names}
  One consequence of these definitions is that $\forall P. \quotep{P}
  \not\in \freenames{P}$.
\end{remark}

\subsection{ Dynamic quote: an example }

Anticipating something of what's to come, consider applying the
substitution, $\widehat{\id{\{}u / z \id{\}}}$, to the following pair
of processes, $\lift{w}{y!(z)}$ and $w[ \lpquote y!(z) \rpquote ]$.

\begin{eqnarray}
	\lift{w}{y!(z)}\widehat{\id{\{}u / z \id{\}}}
		& = &
		\lift{w}{y!(u)} \nonumber\\
	w[ \lpquote y!(z) \rpquote ] \widehat{ \id{\{}u / z \id{\}} }
		& = &
		w[ \lpquote y!(z) \rpquote ] \nonumber
\end{eqnarray}

Because the body of the process between quotes is impervious to
substitution, we get radically different answers. In fact, by
examining the first process in an input context,
e.g. $x?(z).\lift{w}{y!(z)}$, we see that the process under the lift
operator may be shaped by prefixed inputs binding a name inside it. In
this sense, the lift operator will be seen as a way to dynamically
construct processes before reifying them as names.

Finally equipped with these standard features we can present the
dynamics of the calculus.

\subsubsection{Operational semantics} 

Finally, we introduce the computational dynamics. What marks these
algebras as distinct from other more traditionally studied algebraic
structures, e.g. vector spaces or polynomial rings, is the manner in
which dynamics is captured. In traditional structures, dynamics is typically
expressed through morphisms between such structures, as in linear maps
between vector spaces or morphisms between rings. In algebras
associated with the semantics of computation, the dynamics is
expressed as part of the algebraic structure itself, through a
reduction reduction relation typically denoted by $\red$. Below, we
give a recursive presentation of this relation for the calculus used
in the encoding.

$\red \subseteq \pi \times \pi$
$\red : \pi \to \mathcal{P}(\pi)$

\begin{mathpar}
  \inferrule* [lab=Comm] { \textsf{match}( x_{src}, x_{trgt} ) } { x_{trgt}?(y)P \; | \; x_{src}!\langle {Q} \rangle \red P\{\quotep{Q}/y}\} }
  \and \\
  \inferrule* [lab=Par] {{P} \red {P}'} {{{P} | {Q}} \red {{P}' | {Q}}}
  \and
  \inferrule* [lab=Equiv]{{{P} \scong {P}'} \andalso {{P}' \red {Q}'} \andalso {{Q}' \scong {Q}}}{{P} \red {Q}}
\end{mathpar}

\begin{eqnarray*}
  match_{\equiv} (\quotep{P},\quotep{Q}) & := & P \equiv Q \\
  match_{\dagger}(\quotep{P},\quotep{Q}) & := & \forall R. P|Q \red^{*} R => R \red^{*} 0 \\
  match_{K}(\quotep{P},\quotep{Q}) & := & K \mbox{ for some context } K
\end{eqnarray*}

$u?(x)P | u!\langle Q \rangle \red P\{\quotep{Q}/x\}$

%We write $\wred$ for $\red^*$, and $P\red$ if $\exists Q $ such that $ P \red Q$.
We write $P\red$ if $\exists Q $ such that $ P \red Q$ and $P\not\red$, otherwise.

\section{Replication}

As mentioned before, it is known that replication (and hence
recursion) can be implemented in a higher-order process algebra
\cite{SangiorgiWalker}. As our first example of calculation with the
machinery thus far presented we give the construction explicitly in
the {\rhoc}.

\begin{eqnarray}
	D_{x} & := & \prefix{x}{y}{(\binpar{\outputp{x}{y}}{@{y}})} \nonumber\\
	\bangp_{x}{P} & := & \binpar{{x}!\langle{\binpar{D_{x}}{P}}\rangle}{D_{x}} \nonumber
\end{eqnarray}

\begin{eqnarray}
	\bangp_{x}{P} & & \nonumber\\
	=
	& {x}!\langle{(\prefix{x}{y}{(\outputp{x}{y} | @{y})) | P}}\rangle 
	      | \prefix{x}{y}{(\outputp{x}{y} | @{y})} & \nonumber\\
	\red
	& (\outputp{x}{y} | @{y})\substn{\quotep{(\prefix{x}{y}{(@{y} | \outputp{x}{y})) | P}}}{y} & \nonumber\\
	=
	& \outputp{x}{\quotep{(\prefix{x}{y}{(\outputp{x}{y} | @{y})) | P}}}
	  | {(\prefix{x}{y}{(\outputp{x}{y} | @{y})) | P}} & \nonumber\\
	\red
	& \ldots & \nonumber\\
	\red^*
	& P | P | \ldots & \nonumber
\end{eqnarray}

Of course, this encoding, as an implementation, runs away, unfolding
$\bangp{P}$ eagerly. A lazier and more implementable replication
operator, restricted to input-guarded processes, may be obtained as follows.

\begin{eqnarray}
\bangp{\prefix{u}{v}{P}} 
	:= 
	\binpar{\lift{x}{\prefix{u}{v}{(\binpar{D(x)}{P})}}}{D(x)} \nonumber
\end{eqnarray}

\begin{remark}
  Note that the lazier definition still does not deal with summation
  or mixed summation (i.e. sums over input and output). The reader is
  invited to construct definitions of replication that deal with these
  features. 

  Further, the definitions are parameterized in a name, $x$. Can you,
  gentle reader, make a definition that eliminates this parameter and
  guarantees no accidental interaction between the replication
  machinery and the process being replicated -- i.e. no accidental
  sharing of names used by the process to get its work done and the
  name(s) used by the replication to effect copying. This latter
  revision of the definition of replication is crucial to obtaining
  the expected identity $!!P \sim !P$.
\end{remark}

\begin{remark}\label{rem:paradoxical_combinator}
  The reader familiar with the lambda calculus will have noticed the
  similarity between $D$ and the paradoxical combinator.

  [Ed. note: the existence of this seems to suggest we have to be more
  restrictive on the set of processes and names we admit if we are to
  support no-cloning.]
\end{remark}

\subsubsection{Bisimulation}

The computational dynamics gives rise to another kind of equivalence,
the equivalence of computational behavior. As previously mentioned
this is typically captured \emph{via} some form of bisimulation.

% The notion we use in this paper is weak barbed bisimulation
% \cite{milner91polyadicpi}.

The notion we use in this paper is derived from weak barbed
bisimulation \cite{milner91polyadicpi}. 

\begin{definition}
An \emph{observation relation}, $\downarrow_{\mathcal N}$, over a set
of names, $\mathcal N$, is the smallest relation satisfying the rules
below.

\infrule[Out-barb]{y \in {\mathcal N}, \; x \nameeq y}
		  {\outputp{x}{v} \downarrow_{\mathcal N} x}
\infrule[Par-barb]{\mbox{$P\downarrow_{\mathcal N} x$ or $Q\downarrow_{\mathcal N} x$}}
		  {\binpar{P}{Q} \downarrow_{\mathcal N} x}

We write $P \Downarrow_{\mathcal N} x$ if there is $Q$ such that 
$P \wred Q$ and $Q \downarrow_{\mathcal N} x$.
\end{definition}

\begin{definition}
%\label{def.bbisim}
An  ${\mathcal N}$-\emph{barbed bisimulation} over a set of names, ${\mathcal N}$, is a symmetric binary relation 
${\mathcal S}_{\mathcal N}$ between agents such that $P\rel{S}_{\mathcal N}Q$ implies:
\begin{enumerate}
\item If $P \red P'$ then $Q \wred Q'$ and $P'\rel{S}_{\mathcal N} Q'$.
\item If $P\downarrow_{\mathcal N} x$, then $Q\Downarrow_{\mathcal N} x$.
\end{enumerate}
$P$ is ${\mathcal N}$-barbed bisimilar to $Q$, written
$P \wbbisim_{\mathcal N} Q$, if $P \rel{S}_{\mathcal N} Q$ for some ${\mathcal N}$-barbed bisimulation ${\mathcal S}_{\mathcal N}$.
\end{definition}

$\mathcal{R} \subseteq \pi \times \pi$

$P \mathcal{R} Q => \forall P'. P \red P' \Rightarrow \exists Q'. Q \red Q', P' \mathcal{R} Q'$

$P \vdash x \Rightarrow Q \vdash x$

\begin{mathpar}
  \inferrule*[lab=Out-barb]{x \nameeq y}{{y}!\langle{Q}\rangle \vdash x}
  \and
  \inferrule*[lab=Par-barb]{\mbox{$P\vdash x$ or $Q\vdash x$}}{\binpar{P}{Q} \vdash x}
\end{mathpar}

\subsubsection{Contexts}

One of the principle advantages of computational calculi like the
$\pi$-calculus is a well-defined notion of context,
contextual-equivalence and a correlation between
contextual-equivalence and notions of bisimulation. The notion of
context allows the decomposition of a process into (sub-)process and
its syntactic environment, its context. Thus, a context may be
thought of as a process with a ``hole'' (written $\Box$) in it. The
application of a context $M$ to a process $P$, written $M[P]$, is
tantamount to filling the hole in $M$ with $P$. In this paper we do
not need the full weight of this theory, but do make use of the notion
of context in the proof the main theorem. 

\begin{mathpar}
  \inferrule* [lab=summation] {} {{M_{M},M_{N}} \bc \Box \;|\; x.M_{A} \;|\; M_{M}+M_{N}}
  \and
  \inferrule* [lab=agent] {} {{M_{A}} \bc (\vec{x})M_{P} \;| \; \clift{P_0,\ldots,M_{P},\ldots,P_N}}
  \and \\
  \inferrule* [lab=process] {} {{M_{P}} \bc M_{N} \;| \;P|M_{P} }
\end{mathpar} 

\begin{mathpar}
  \inferrule* [lab=sychronization] {} {M_{N} \bc \Box \;|\; x?M_{F} \;|\; x!M_{C}}
  \and
  \inferrule* [lab=abstraction] {} {{M_{F}} \bc (x)M_{P} }
  \and
  \inferrule* [lab=concretion] {} {{M_{C}} \bc \langle M_{P} \rangle }
  \and \\
  \inferrule* [lab=process] {} {{M_{P}} \bc M_{N} \;| \;P|M_{P} }
\end{mathpar}

\begin{definition}[contextual application] Given a context $M$, and
  process $P$, we define the \emph{contextual application}, $M[P] :=
  M\{P/\Box\}$. That is, the contextual application of M to P is the
  substitution of $P$ for $\Box$ in $M$.
\end{definition}

$\meaningof{-} : L \to \mathcal{P}(\pi)$

\begin{mathpar}
  \inferrule* [lab=collection] {} {\meaningof{true} = \pi, \and \meaningof{~E} = \pi \setminus \meaningof{E}, \and \meaningof{E_{1} \& E_{2}} = \meaningof{E_{1}} \cap \meaningof{E_{2}}}
\end{mathpar}

\begin{mathpar}
  \inferrule* [lab=structure] {} {\meaningof{0} = \{ P \in \pi | P \equiv 0 \}, \and \\ \meaningof{E_1 | E_2} = \{ P \in \pi | P \equiv P_{1} | P_{2}, P_{1} \in \meaningof{E_{1}}, P_{2} \in \meaningof{E_2}\} }
\end{mathpar}

\begin{mathpar}
 \inferrule* [lab=behavior] {} {\meaningof{\langle a?b \rangle E} = \{ P \in \pi | P \equiv Q | u?(y)P', \\ \and \\\\ \and \\ \;\;\; u \in \meaningof{a}, \forall z.P'\{z/y\} \in \meaningof{E\{z/b\}}\}, \and \\ \meaningof{a!E} = \{ P \in \pi | P \equiv Q | x!\langle P' \rangle, x \in \meaningof{a} P' \in \meaningof{E}\} }
\end{mathpar}

\begin{mathpar}
 \inferrule* [lab=nominal] {} {\meaningof{\quotep{E}} = \{ \quotep{P} \in \quotep{\pi} | P \in \meaningof{E} \}, \and \meaningof{\quotep{P}} = \{ \quotep{Q} \in \quotep{\pi} | P \equiv Q \} \and \\ \meaningof{@\quotep{E}} = \{ P \in \pi | P \equiv @x, x \in \meaningof{E} \}}
\end{mathpar}

\begin{eqnarray*}
  \\
  \meaningof{-} : TS \to ST
\end{eqnarray*}

\begin{eqnarray*}
  \\
  L : TS \to ST
\end{eqnarray*}

\begin{eqnarray*}
  \\
  P \models E \iff P \in \meaningof{E}
\end{eqnarray*}

\begin{eqnarray*}
  P \approx_{L} Q \iff \forall E \in L. P \models E \iff Q \models E
\end{eqnarray*}

\begin{eqnarray*}
  P \approx_{K} Q
\end{eqnarray*}

\begin{eqnarray*}
  P \approx Q
\end{eqnarray*}

$\approx_{K} = \approx = \approx_{L}$

\subsubsection{Contextual duality}

Note that contexts extend the quotation operation to a family of
operations from processes to names. Given a context, $M$, we can
define a \emph{nominal context}, $\quotep{M}$ by $\quotep{M}[P] :=
\quotep{M[P]}$. To foreshadow what is to come we observe that these
operations enjoy a duality with processes very much like the duality
between vectors and maps from vectors to scalars.

Further, because the calculus is essentially higher-order, we have a
correspondence between contexts and processes. More specifically,
given a name $x$ and a context $M$ we can construct $M^{*}_{x}$ such
that 

\begin{mathpar}
  M^{*}_{x} | \lift{x}{P} \red M[P]
\end{mathpar}

namely,

\begin{mathpar}
  M^{*}_{x} := x?(u).M[\dropn{u}]
\end{mathpar}

The dependence of $M^{*}_{x}$ on a name makes it an abstraction, 

\begin{mathpar}
  M^{*} := (x)x?(u).M[\dropn{u}]
\end{mathpar}

\subsection{Additional notation}

It will sometimes be convenient to denote the process a name
quotes. We already have the notation $x = \quotep{P}$, but it will be
convenient to introduce an alternate notation, $\procn{x}$, when we
want to emphasize the connection to the use of the name. Note that, by
virtue of name equivalence, $\quotep{\procn{x}} \nameeq x$; so, the
notation is consistent with previous definitions.

Further, because names have structure it is possible to effect
substitutions on the basis of that structure. This means we need to
upgrade our notation for substitutions, which we accomplish by
adapting comprehension notation. Thus,

\begin{mathpar}
  P\{ y / x : x \in S \}
\end{mathpar}

is interpreted to mean the process derived from P by replacing (in a
capture-avoiding manner) each occurrence of $x$ in $S$ by $y$. For example,

\begin{mathpar}
  P\{ \quotep{\procn{x}|\procn{x}} / x : x \in \freenames{P} \}
\end{mathpar}

will replace each (occurrence) of a free name $x$ in $P$ by
$\quotep{\procn{x}|\procn{x}}$.

Also, we will avail ourselves of the notation $x^{L}$ and $x^{R}$ to
denote injections of a name into disjoint copies of the name
space. There are numerous ways to accomplish this. One example can be
found in \cite{MeredithR05}. This notation overloads to vectors of
names: $\vec{x}^{\pi} := (x_{i}^{\pi} \; : \; 0 \leq i < |\vec{x}| )$ where $\pi \in \{L,R\}$.

We also use $P^{\Box} := P|\Box$.

In \cite{MeredithR05} an interpretation of the new operator is
given. It turns out that there are several possible interpretations
all enjoying the requisite algebraic properties of the operator (see
\cite{milner91polyadicpi}). We will therefore make liberal use of
$(\nu\; \vec{x})P$.

% subsection the_syntax_and_semantics_of_the_notation_system (end)   

\input{qm2pi.qmops} 

\input{qm2pi.sterngerlach} 

\input{qm2pi.metric} 

% section concurrent_process_calculi (end)

%\input{qm2pi.proofsketch}

% section proof sketch (end)

%\input{qm2pi.slviaknots} 

% section spatial logic via knots (end)

\input{qm2pi.conclusion}

% section conclusion (end)

%\input{qm2pi.dtcodes} 

% section wiring algorithm (end)

\input{qm2pi.ack} 

% section acknowledgments (end)

\newpage


\bibliographystyle{plain}   
\bibliography{../../biblios/main.bib}

\input{qm2pi.rhodetails}

\end{document}



% section proof sketch (end)

%\section{Unlikely characters: spatial logic for
  knots}\label{sub:characteristic_formulae} % (fold)

Associated to the mobile process calculi are a family of logics known
as the Hennessy-Milner logics. These logics typically enjoy a
semantics interpreting formulae as sets of processes that when
factored through the encoding outlined above allows an identification
of classes of knots with logical formulae. In the context of this
encoding the sub-family known as the spatial logics \cite{CairesC03}
\cite{CairesC04} \cite{Caires04} are of particular interest providing
several important features for expressing and reasoning about
properties (i.e. classes) of knots. We hint here at how this may be done.

%\begin{description}
%\item [structural connectives] 
\subsubsection{Structural connectives} The spatial logics enjoy
structural connectives corresponding, at the logical level, to the
parallel composition ($P | Q$) and new name ($(\nu \; x)P$)
connectives for processes. As illustrated in the examples below, these
connectives are extremely expressive given the shape of our encoding.
%\item [decideable satisfaction]

\subsubsection{Decideable satisfaction}
In \cite{Caires04} the satisfaction relation is shown to be decideable
for a rich class of processes. It further turns out that the image of
the our encoding is a proper subset of that class. This result
provides the basis for an algorithm by which to search for knots
enjoying a given property.
%\item [characteristic formulae]

\subsubsection{Characteristic formulae}
In the same paper \cite{Caires04} , Caires presents a means of calculating
characteristic formulae, selecting equivalence classes of processes
up to a pre--specified depth limit on the support set of names. Composed with our
encoding, this characteristic formula can be used to select
characteristic formulae for knots.
%\end{description}

\subsubsection{Spatial logic formulae}

The grammar below (segmented for comprehension) summarizes the syntax
of spatial logic formulae. We employ illustrative examples in the
sequel to provide an intuitive understanding of their meaning
referring the reader to \cite{Caires04} for a more detailed explication
of the semantics.

\begin{mathpar}
  \inferrule* [lab=boolean] {} {{A,B} \bc T \;|\; \neg A \;|\; A \wedge B \;|\; \eta = \eta'}
  \and
  \inferrule* [lab=spatial] {} {|\; \pzero \;|\; A | B \;|\; x \text{\textregistered} A \;|\; \forall x . A \;|\;  H x . A}
  \and
  \inferrule* [lab=behavioral] {} {|\; \alpha . A}
  \and 
  \inferrule* [lab=recursion] {} {|\; X(\vec{u}) \;|\; \mu X(\vec{u}) . A}
  \and
  \inferrule* [lab=action] {} {\alpha \bc \langle x?(\vec{y}) \rangle \;|\; \langle x!(\vec{y}) \rangle \;|\; \langle \tau \rangle}
  \and 
  \inferrule* [lab=name] {} {\eta \bc x \;|\; \tau}
\end{mathpar} 

% subsection characteristic_formulae (end)   	 

\subsection{Example formulae}\label{sub:example_formulae_} % (fold)

\subsubsection{Crossing as formula.}
% 
% \begin{align*}
%   \frac{d}{dx} \sin x &= \cos x 
%   & \frac{d}{dx} e^x &= e^x \\
%   \frac{d}{dx} \cos x &= - \sin x 
%   & \frac{d}{dx} \log x &= \frac{1}{x} \\
% \end{align*} 

\begin{align*}
 \mu C(x_{0},x_{1},y_{0},y_{1},u).&(\langle x_{0}?(z) \rangle(\langle u! \rangle\langle y_{1}!z \rangle C(x_{0},x_{1},y_{0},y_{1},u)) & \\
  & \wedge \langle y_{1}?(z) \rangle (\langle u! \rangle \langle x_{0}!z \rangle C(x_{0},x_{1},y_{0},y_{1},u)) & \\
  & \wedge \langle x_{1}?(z) \rangle (\langle u? \rangle \langle y_{0}!z \rangle C(x_{0},x_{1},y_{0},y_{1},u)) & \\
  & \wedge \langle y_{0}?(z) \rangle (\langle u? \rangle \langle x_{1}!z \rangle C(x_{0},x_{1},y_{0},y_{1},u))) &
\end{align*}

The lexicographical similarity between the shape of this formulae and
the shape of definition of the process representing a crossing reveals
the intuitive meaning of this formulae. It describes the capabilities
of a process that has the right to represent a crossing. For example
it picks out processes that may perform an input on the port $x_0$ in
its initial menu of capabilities. What differentiates the formula
from the process, however, is that the crossing process is the
smallest candidate to satisfy the formula. Infinitely many other
processes -- with internal behavior hidden behind this interface, so
to speak -- also satisfy this formula. Even this simple formula,
then, can be seen to open a new view onto knots, providing a
computational interpretation of \emph{virtual} knots.

Note that this formula is derived by hand. A similar formula can be
derived by employing Caires' calculation of characteristic formula
\cite{Caires04} to the process representing a crossing. In light of
this discussion, we let
$\meaningof{C}_{\phi}(x0,x1,y0,y1,u)$ denote a formula specifying the
dynamics we wish to capture of a crossing. To guarantee we preserve
the shape of the interface and minimal semantics we demand that
$\meaningof{C}_{\phi}(x0,x1,y0,y1,u) \Rightarrow
\textbf{C}(x0,x1,y0,y1,u)$ where $\textbf{C}(x0,x1,y0,y1,u)$ denotes
the formula above.
                            
\subsubsection{Crossing number constraints.}
The moral content of the context lemma (Lemma \ref{context}) is that the notion of
``locality'' in the Reidemeister moves is effectively captured by the
parallel composition operator of the process calculus. This intuition
extends through the logic. Given a formula,
$\meaningof{C}_{\phi}(x0,x1,y0,y1,u)$, we can use the structural
connectives to specify constraints on crossing numbers, such as at
least $n$ crossings, or exactly $n$ crossings.
\begin{mathpar}
  \inferrule* [lab=at-least-n] {} { K^{\geq n}_{\phi}(\vec{xs},\vec{ys}) := \Pi_{i=0}^{n-1} Hu . \meaningof{C}_{\phi}(xs_i,ys_i,u) | T }
  \and 
  \inferrule* [lab=exactly-n] {} { K^{= n}_{\phi}(\vec{xs},\vec{ys}) := \Pi_{i=0}^{n-1} Hu . \meaningof{C}_{\phi}(xs_i,ys_i,u) | \neg (\forall x_0,y_0,x_1,y_1,u . \meaningof{C}_{\phi}(x_0,y_0,x_1,y_1,u) | T) }
\end{mathpar}

To round out this section, recall that the encoding of an $n$-crossing
knot decomposes into a parallel composition of $n$ \emph{copies} of a
crossing process together with a wiring harness. To specify different
knot classes with the same crossing number amounts to specifying
logical constraints on the wiring harness. In the interest of space,
we defer examples to a forthcoming paper. Suffice it to say that both
the conditions ``alternating knot'' and ``contains the tangle
corresponding to 5/3'' are expressible. For example, it is possible to
calculate the characteristic formula of a process corresponding to the
tangle 5/3 and conjoin it into the classifying formula via the
composition connective of the logic.

Finally, we wish to observe that it is entirely within reason to
contemplate a more domain-specific version of spatial logic tailored
to the shape of processes in the image of the encoding. Such a
domain-specific logic would have a better claim to the title formal
language of knot properties.

% subsection example_formulae_ (end)

% section knots_as_processes (end) 

% section spatial logic via knots (end)

\section{Conclusions and future work}

\paragraph{Testing physical space}
You, gentle reader, may wonder why of all the theorems to be proved
given this set up we pick the one above. In some sense it's hardly
central to quantum mechanics. We see it as central in the sense that
it firmly establishes a notion of physical space arising from a notion
of the equivalence of behavior. Relating bisimulation to a metric is a
big step forward, but one is faced with interpreting the relationship
of that metric space to something more physical. Quantum mechanical
notions of ``physical'' space are still far from intuitive, but by
relating this idea of distance as testing to calculations that predict
physical circumstances we are making a not insignificant step forward
toward an understanding of the physical space we inhabit as
essentially dynamic.

\paragraph{Effectivity and simulation}
One of the observations we have yet to make is that the entire program
spelled out here is effective. We have built various interpreters for
the reflective calculus at work in this interpretation. In principle,
then, we can simulate quantum mechanics on a computer. The place where
the simulation may lose fidelity is the infinitely branching summation
for the annihilator.

In this connection i also want to point out that the evaluation style
calculation of the inner product puts the non-determinism of the
summation right at the heart of measurement. This suggests that
Milner's original reduction-based formulation of the dynamics of his
calculi in terms of sums was not just notationally suggestive of a
notion of measure-and-continue but captured some significant part of
the physics.

\paragraph{Quantum continuations}
In light of this last observation i want to point out that the
predominant account of quantum mechanics is missing a key aspect of a
truly compositional story of the physical situation. In a real lab,
when a measurement is made the observation can be made to feed into
another device that then makes another measurement conditioned on the
results of the first. This means that after the superposition was
collapsed the entire experimental set up remained in
superposition. While QM offers a means of writing this down it doesn't
quite line up well with the well-trodden formulation of computation
and continuation that we see so succinctly expressed in Milner's
calculi. This suggests that there might be advantages to this account
of dynamics waiting to be explored.

\paragraph{Quantum logic}
In this connection, we also note that by virtue of having the
Hennessy-Milner construction, we can pull the construction through the
interpretation of QM. This gives us a natural candidate for a quantum
logic that enjoys an extremely tight connection with it's domain of
interpretation, making the construction much less ad hoc (rather it is
the image of functor!).

\paragraph{Quantum probabiity}
i have questions about the basis of the interpretation of inner
product as probability amplitude. In particular, using which
axiomatization of probability theory does the notion of probability
amplitude earn the right to be so dubbed? In other words, where is the
proof that the operation for calculating a probability amplitude (and
then squaring) satisfies the axioms of what it means to calculate a
probability? Even if such a proof exists (i have yet to find it in the
literature), i wonder if it might not be possible to turn things on
their heads. Can we view the calculation of the probability amplitude
as an axiomatization of probability? If so, then the definition we
give for calculating probability amplitude may provide the basis for
an \emph{effective} theory of probability.

\paragraph{Quantum vs ``biological'' information}
Finally, i want to conclude with a more philosophical observation. At
a recent workshop in which QM was a predominant topic i noticed
something about quantum information. The speaker was giving a riveting
discussion of axiomatic QM and showing how properties of ``no
cloning'' and ``no deleting'' emerged as consequences of the
axiomatization. Theorems of this form are necessary to give us a sense
of confidence that our axioms characterize the physical theory. What
struck me, though, was that if quantum information is neither erasable
nor replicable it is markedly different from \emph{life}. Two of the
things we know about life is that

\begin{itemize}
  \item it ends;
  \item to gain some measure of persistence, to transcend it's
    finitude it is imminently copyable.
\end{itemize}

Both of these qualities are summarized succinctly in the aphorism: all
flesh is grass. For me these two kinds of ``information'' -- call them
quantum and biological -- are end points on a spectrum of strategies
for persistence. At one end, we have those curious entities that enjoy
uniqueness and permanence; at the other, we have those who in the face
of a certain end and an uncertain present make a go of passing
something on. To me one of the more remarkable aspects of the latter
strategy is that in the presence of noise (and certain features of
copying) we get a kind of dynamism, a chance for improvement against a
given persistent condition.

% subsection other_calculi_other_bisimulations_and_geometry_as_behavior (end)




% section conclusion (end)

%\documentclass[12pt]{llncs}
%\documentclass{jktr}

\usepackage[pdftex]{hyperref}                   
\usepackage {listings}
\usepackage {mathpartir}
\usepackage{bcprules}
%\usepackage{listings}
                       
\usepackage{graphicx} 
%\usepackage[margins=2.5cm,nohead,nofoot]{geometry}
%\usepackage{geometry}
\usepackage{amsfonts}
\usepackage{amstext}
\usepackage{latexsym}
\usepackage{amssymb}
\usepackage{color}


%\include{myPreamble}
\include{qm2pi.local} 

%\ifpdf
%\usepackage[pdftex]{graphicx}
%\else
%\usepackage{graphicx}
%\fi

 % \ifpdf
%  \usepackage{pdfsync}
%  \if


%\title{Brief Article}
%\author{David F. Snyder}
%\author{L.G. Meredith}

%\address{Dept. of Math., Texas State University--San Marcos, San Marcos, TX 78666}
       
\pagestyle{empty}


\begin{document}

\lstset{language=[Objective]Caml,frame=shadowbox}

\input{qm2pi.front}

% section front matter (end)

\input{qm2pi.intro} 
 
% section introduction (end)

% \input{qm2pi.knotations} 

% section notation (end)

\input{qm2pi.process.calculi} 

% section concurrent_process_calculi_and_spatial_logics_ (end)
    
%\input{qm2pi.knots2pi} 

%\input{qm2pi.trefoil} 

%\input{qm2pi.mainthm} 

% subsection basic_interpretation (end)

%\input{qm2pi.rho.presentation} 
\subsection{The syntax and semantics of the notation system}\label{sub:the_syntax_and_semantics_of_the_notation_system} % (fold)

We now summarize a technical presentation of the calculus that
embodies our theory of dynamics. The typical presentation of such a
calculus follows the style of giving generators and relations on
them. The grammar, below, describing term constructors, freely
generates the set of processes, $\Proc$. This set is then quotiented
by a relation known as structural congruence and it is over this set
that the notion of dynamics is expressed. This presentation is
essentially that of \cite{MeredithR05} with the addition of
polyadicity and summation. For readability we have relegated some of
the technical subtleties to an appendix.

\subsubsection{Process grammar}\label{subsub:process_grammar}

\begin{mathpar}
  \inferrule* [lab=synchronization] {} {{M} \bc \pzero \;|\; x?F \;|\; x!C }
  \and
  \inferrule* [lab=abstraction] {} {{F} \bc (x)P}
  \and
  \inferrule* [lab=concretion] {} {{C} \bc \langle Q \rangle}
  \and
  \inferrule* [lab=process] {} {{P,Q} \bc M \;| \;P|Q \;|\; @{x}}
  \and
  \inferrule* [lab=name] {} {{x} \bc \quotep{P}}
\end{mathpar} 

Note that $\vec{x}$ (resp. $\vec{P}$) denotes a vector of names
(resp. processes) of length $|\vec{x}|$ (resp. $|\vec{P}|$). We adopt
the following useful abbreviations.

\begin{mathpar}
   x?(\vec{y}).P := x.(\vec{y})P \and  x\clift{\vec{P}} := x.\clift{\vec{P}}
   \and x!(y) := \lift{x}{\dropn{y}}
   \and \Pi_{i=0}^{n-1}P_i := P_0 | \ldots | P_{n-1}
\end{mathpar}

\subsubsection{Structural congruence}

\paragraph{Free and bound names and alpha-equivalence.} At the
core of structural equivalence is alpha-equivalence which identifies
process that are the same up to a change of variable. Formally, we
recognize the distinction between free and bound names. The free names
of a process, $\freenames{P}$, may be calculated recursively as
follows:

\begin{mathpar}
\freenames{\pzero} := \emptyset
  \and \\
  \freenames{x?(y).P} := \{ x \} \cup (\freenames{P} \setminus \{ y \})
  \and 
  \freenames{x!\langle P \rangle} := \{ x \} \cup \{ P \} 
  \and \\
  \freenames{P|Q} := \freenames{P} \cup \freenames{Q}
  \and \\
  \freenames{@{x}} := \{ x \}
\end{mathpar}

$\pi$
$\quotep{\pi}$

$\freenames{-} : \pi \to \mathcal{P}(\quotep{\pi})$

\begin{eqnarray*}
  \freenames{\pzero} & := & \emptyset \\
  \freenames{x?(y).P} & := & \{ x \} \cup (\freenames{P} \setminus \{ y \}) \\
  \freenames{x!\langle P \rangle} & := & \{ x \} \cup \{ P \} \\
  \freenames{P|Q} & := & \freenames{P} \cup \freenames{Q} \\
  \freenames{\dropn{x}} & := & \{ x \}
\end{eqnarray*}

The bound names of a process, $\boundnames{P}$, are those names occurring in $P$
that are not free. For example, in $x?(y).0$, the name $x$ is free, while $y$ is bound.

\begin{mathpar}
  \inferrule* [lab=monoidal-laws] {} { P|Q \equiv Q|P \and P|0 \equiv P \and P|(Q|R) \equiv (P|Q)|R }
\end{mathpar}

\begin{mathpar}
  \inferrule* [lab=alpha-equivalence] {} { (x)P \equiv (y)P\{y/x\} \and y \not\in \freenames{P} }
\end{mathpar}

\begin{definition}
Then two processes, $P,Q$, are alpha-equivalent if $P = Q\{\vec{y}/\vec{x}\}$ for
some $\vec{x} \in \boundnames{Q},\vec{y} \in \boundnames{P}$, where $Q\{\vec{y}/\vec{x}\}$
denotes the capture-avoiding substitution of $\vec{y}$ for $\vec{x}$ in $Q$.
\end{definition}

\begin{definition}
  The {\em structural congruence} \cite{SangiorgiWalker} , $\equiv$,
  between processes is the least congruence containing
  alpha-equivalence, satisfying the abelian monoid laws
  (associativity, commutativity and $\pzero$ as identity) for parallel
  composition $|$ and for summation $+$.
\end{definition}

\subsection{Name equivalence}

We take name equivalence, written $\nameeq$, to be the smallest
equivalence relation generated by the following rules.

\begin{mathpar}
\inferrule*[lab=Quote-drop]
{ }
{ \quotep{@{x}} \nameeq x }

\inferrule*[lab=Struct-equiv]
{ P \scong Q }
{ \quotep{P} \nameeq \quotep{Q} }
\end{mathpar}

The astute reader will have noticed that the mutual recursion of names
and processes imposes a mutual recursion on alpha-equivalence and
structural equivalence via name-equivalence. Fortunately, all of this
works out pleasantly and we may calculate in the natural way, free of
concern. The reader interested in the details is referred to the
appendix \ref{appendix:rho_details}.

\subsection{Substitution}

We use $\Proc$ for the set of processes, $\QProc$ for the set of
names, and $\id{\{}\vec{y} / \vec{x} \id{\}}$ to denote partial maps,
$s : \QProc \rightarrow \QProc$. A map, $s$ lifts, uniquely, to a map
on process terms, $\widehat{s} : \Proc \rightarrow \Proc$ by the
following equations.

\begin{mathpar}
  (0) \psubstp{Q}{P} := 0 \\
  (R \juxtap S) \psubstp{Q}{P}
  :=    
  (R)\psubstp{Q}{P} \juxtap (S) \psubstp{Q}{P} \\
  (x?(y).R) \psubstp{Q}{P}    
  :=    
  (x)\substp{Q}{P} (z)\concat( (R \psubstn{z}{y}) \psubstp{Q}{P} ) \\
  (\lift{x}{R}) \psubstp{Q}{P}  
  :=
  \lift{(x)\substp{Q}{P}}{ R \psubstp{Q}{P} } \\
%   (\dropn{x})  \psubstp{Q}{P}       
%   := 
%   \left\{ 
%     \begin{array}{ccc} 
%       \dropn{\quotep{Q}} & & x \nameeq \quotep{P} \\
%       \dropn{x} & & otherwise \\
%     \end{array}
%   \right. 
  (\dropn{x})  \psubstp{Q}{P}       
  := 
  \left\{ 
    \begin{array}{ccc} 
      Q & & x \nameeq \quotep{P} \\
      \dropn{x} & & otherwise \\
    \end{array}
  \right.
\end{mathpar}
 

where

\begin{eqnarray}
  (x)\id{\{} \lpquote Q \rpquote / \lpquote P \rpquote \id{\}}            = 
  \left\{ 
    \begin{array}{ccc}
      \lpquote Q \rpquote & & x \nameeq \lpquote P \rpquote \\
      x & & otherwise \\
    \end{array}
  \right. \nonumber
\end{eqnarray}

and $z$ is chosen distinct from $\quotep{P}$, $\quotep{Q}$, the free
names in $Q$, and all the names in $R$. Our $\alpha$-equivalence will
be built in the standard way from this substitution.

\begin{remark}\label{rem:no_self_referential_names}
  One consequence of these definitions is that $\forall P. \quotep{P}
  \not\in \freenames{P}$.
\end{remark}

\subsection{ Dynamic quote: an example }

Anticipating something of what's to come, consider applying the
substitution, $\widehat{\id{\{}u / z \id{\}}}$, to the following pair
of processes, $\lift{w}{y!(z)}$ and $w[ \lpquote y!(z) \rpquote ]$.

\begin{eqnarray}
	\lift{w}{y!(z)}\widehat{\id{\{}u / z \id{\}}}
		& = &
		\lift{w}{y!(u)} \nonumber\\
	w[ \lpquote y!(z) \rpquote ] \widehat{ \id{\{}u / z \id{\}} }
		& = &
		w[ \lpquote y!(z) \rpquote ] \nonumber
\end{eqnarray}

Because the body of the process between quotes is impervious to
substitution, we get radically different answers. In fact, by
examining the first process in an input context,
e.g. $x?(z).\lift{w}{y!(z)}$, we see that the process under the lift
operator may be shaped by prefixed inputs binding a name inside it. In
this sense, the lift operator will be seen as a way to dynamically
construct processes before reifying them as names.

Finally equipped with these standard features we can present the
dynamics of the calculus.

\subsubsection{Operational semantics} 

Finally, we introduce the computational dynamics. What marks these
algebras as distinct from other more traditionally studied algebraic
structures, e.g. vector spaces or polynomial rings, is the manner in
which dynamics is captured. In traditional structures, dynamics is typically
expressed through morphisms between such structures, as in linear maps
between vector spaces or morphisms between rings. In algebras
associated with the semantics of computation, the dynamics is
expressed as part of the algebraic structure itself, through a
reduction reduction relation typically denoted by $\red$. Below, we
give a recursive presentation of this relation for the calculus used
in the encoding.

$\red \subseteq \pi \times \pi$
$\red : \pi \to \mathcal{P}(\pi)$

\begin{mathpar}
  \inferrule* [lab=Comm] { \textsf{match}( x_{src}, x_{trgt} ) } { x_{trgt}?(y)P \; | \; x_{src}!\langle {Q} \rangle \red P\{\quotep{Q}/y}\} }
  \and \\
  \inferrule* [lab=Par] {{P} \red {P}'} {{{P} | {Q}} \red {{P}' | {Q}}}
  \and
  \inferrule* [lab=Equiv]{{{P} \scong {P}'} \andalso {{P}' \red {Q}'} \andalso {{Q}' \scong {Q}}}{{P} \red {Q}}
\end{mathpar}

\begin{eqnarray*}
  match_{\equiv} (\quotep{P},\quotep{Q}) & := & P \equiv Q \\
  match_{\dagger}(\quotep{P},\quotep{Q}) & := & \forall R. P|Q \red^{*} R => R \red^{*} 0 \\
  match_{K}(\quotep{P},\quotep{Q}) & := & K \mbox{ for some context } K
\end{eqnarray*}

$u?(x)P | u!\langle Q \rangle \red P\{\quotep{Q}/x\}$

%We write $\wred$ for $\red^*$, and $P\red$ if $\exists Q $ such that $ P \red Q$.
We write $P\red$ if $\exists Q $ such that $ P \red Q$ and $P\not\red$, otherwise.

\section{Replication}

As mentioned before, it is known that replication (and hence
recursion) can be implemented in a higher-order process algebra
\cite{SangiorgiWalker}. As our first example of calculation with the
machinery thus far presented we give the construction explicitly in
the {\rhoc}.

\begin{eqnarray}
	D_{x} & := & \prefix{x}{y}{(\binpar{\outputp{x}{y}}{@{y}})} \nonumber\\
	\bangp_{x}{P} & := & \binpar{{x}!\langle{\binpar{D_{x}}{P}}\rangle}{D_{x}} \nonumber
\end{eqnarray}

\begin{eqnarray}
	\bangp_{x}{P} & & \nonumber\\
	=
	& {x}!\langle{(\prefix{x}{y}{(\outputp{x}{y} | @{y})) | P}}\rangle 
	      | \prefix{x}{y}{(\outputp{x}{y} | @{y})} & \nonumber\\
	\red
	& (\outputp{x}{y} | @{y})\substn{\quotep{(\prefix{x}{y}{(@{y} | \outputp{x}{y})) | P}}}{y} & \nonumber\\
	=
	& \outputp{x}{\quotep{(\prefix{x}{y}{(\outputp{x}{y} | @{y})) | P}}}
	  | {(\prefix{x}{y}{(\outputp{x}{y} | @{y})) | P}} & \nonumber\\
	\red
	& \ldots & \nonumber\\
	\red^*
	& P | P | \ldots & \nonumber
\end{eqnarray}

Of course, this encoding, as an implementation, runs away, unfolding
$\bangp{P}$ eagerly. A lazier and more implementable replication
operator, restricted to input-guarded processes, may be obtained as follows.

\begin{eqnarray}
\bangp{\prefix{u}{v}{P}} 
	:= 
	\binpar{\lift{x}{\prefix{u}{v}{(\binpar{D(x)}{P})}}}{D(x)} \nonumber
\end{eqnarray}

\begin{remark}
  Note that the lazier definition still does not deal with summation
  or mixed summation (i.e. sums over input and output). The reader is
  invited to construct definitions of replication that deal with these
  features. 

  Further, the definitions are parameterized in a name, $x$. Can you,
  gentle reader, make a definition that eliminates this parameter and
  guarantees no accidental interaction between the replication
  machinery and the process being replicated -- i.e. no accidental
  sharing of names used by the process to get its work done and the
  name(s) used by the replication to effect copying. This latter
  revision of the definition of replication is crucial to obtaining
  the expected identity $!!P \sim !P$.
\end{remark}

\begin{remark}\label{rem:paradoxical_combinator}
  The reader familiar with the lambda calculus will have noticed the
  similarity between $D$ and the paradoxical combinator.

  [Ed. note: the existence of this seems to suggest we have to be more
  restrictive on the set of processes and names we admit if we are to
  support no-cloning.]
\end{remark}

\subsubsection{Bisimulation}

The computational dynamics gives rise to another kind of equivalence,
the equivalence of computational behavior. As previously mentioned
this is typically captured \emph{via} some form of bisimulation.

% The notion we use in this paper is weak barbed bisimulation
% \cite{milner91polyadicpi}.

The notion we use in this paper is derived from weak barbed
bisimulation \cite{milner91polyadicpi}. 

\begin{definition}
An \emph{observation relation}, $\downarrow_{\mathcal N}$, over a set
of names, $\mathcal N$, is the smallest relation satisfying the rules
below.

\infrule[Out-barb]{y \in {\mathcal N}, \; x \nameeq y}
		  {\outputp{x}{v} \downarrow_{\mathcal N} x}
\infrule[Par-barb]{\mbox{$P\downarrow_{\mathcal N} x$ or $Q\downarrow_{\mathcal N} x$}}
		  {\binpar{P}{Q} \downarrow_{\mathcal N} x}

We write $P \Downarrow_{\mathcal N} x$ if there is $Q$ such that 
$P \wred Q$ and $Q \downarrow_{\mathcal N} x$.
\end{definition}

\begin{definition}
%\label{def.bbisim}
An  ${\mathcal N}$-\emph{barbed bisimulation} over a set of names, ${\mathcal N}$, is a symmetric binary relation 
${\mathcal S}_{\mathcal N}$ between agents such that $P\rel{S}_{\mathcal N}Q$ implies:
\begin{enumerate}
\item If $P \red P'$ then $Q \wred Q'$ and $P'\rel{S}_{\mathcal N} Q'$.
\item If $P\downarrow_{\mathcal N} x$, then $Q\Downarrow_{\mathcal N} x$.
\end{enumerate}
$P$ is ${\mathcal N}$-barbed bisimilar to $Q$, written
$P \wbbisim_{\mathcal N} Q$, if $P \rel{S}_{\mathcal N} Q$ for some ${\mathcal N}$-barbed bisimulation ${\mathcal S}_{\mathcal N}$.
\end{definition}

$\mathcal{R} \subseteq \pi \times \pi$

$P \mathcal{R} Q => \forall P'. P \red P' \Rightarrow \exists Q'. Q \red Q', P' \mathcal{R} Q'$

$P \vdash x \Rightarrow Q \vdash x$

\begin{mathpar}
  \inferrule*[lab=Out-barb]{x \nameeq y}{{y}!\langle{Q}\rangle \vdash x}
  \and
  \inferrule*[lab=Par-barb]{\mbox{$P\vdash x$ or $Q\vdash x$}}{\binpar{P}{Q} \vdash x}
\end{mathpar}

\subsubsection{Contexts}

One of the principle advantages of computational calculi like the
$\pi$-calculus is a well-defined notion of context,
contextual-equivalence and a correlation between
contextual-equivalence and notions of bisimulation. The notion of
context allows the decomposition of a process into (sub-)process and
its syntactic environment, its context. Thus, a context may be
thought of as a process with a ``hole'' (written $\Box$) in it. The
application of a context $M$ to a process $P$, written $M[P]$, is
tantamount to filling the hole in $M$ with $P$. In this paper we do
not need the full weight of this theory, but do make use of the notion
of context in the proof the main theorem. 

\begin{mathpar}
  \inferrule* [lab=summation] {} {{M_{M},M_{N}} \bc \Box \;|\; x.M_{A} \;|\; M_{M}+M_{N}}
  \and
  \inferrule* [lab=agent] {} {{M_{A}} \bc (\vec{x})M_{P} \;| \; \clift{P_0,\ldots,M_{P},\ldots,P_N}}
  \and \\
  \inferrule* [lab=process] {} {{M_{P}} \bc M_{N} \;| \;P|M_{P} }
\end{mathpar} 

\begin{mathpar}
  \inferrule* [lab=sychronization] {} {M_{N} \bc \Box \;|\; x?M_{F} \;|\; x!M_{C}}
  \and
  \inferrule* [lab=abstraction] {} {{M_{F}} \bc (x)M_{P} }
  \and
  \inferrule* [lab=concretion] {} {{M_{C}} \bc \langle M_{P} \rangle }
  \and \\
  \inferrule* [lab=process] {} {{M_{P}} \bc M_{N} \;| \;P|M_{P} }
\end{mathpar}

\begin{definition}[contextual application] Given a context $M$, and
  process $P$, we define the \emph{contextual application}, $M[P] :=
  M\{P/\Box\}$. That is, the contextual application of M to P is the
  substitution of $P$ for $\Box$ in $M$.
\end{definition}

$\meaningof{-} : L \to \mathcal{P}(\pi)$

\begin{mathpar}
  \inferrule* [lab=collection] {} {\meaningof{true} = \pi, \and \meaningof{~E} = \pi \setminus \meaningof{E}, \and \meaningof{E_{1} \& E_{2}} = \meaningof{E_{1}} \cap \meaningof{E_{2}}}
\end{mathpar}

\begin{mathpar}
  \inferrule* [lab=structure] {} {\meaningof{0} = \{ P \in \pi | P \equiv 0 \}, \and \\ \meaningof{E_1 | E_2} = \{ P \in \pi | P \equiv P_{1} | P_{2}, P_{1} \in \meaningof{E_{1}}, P_{2} \in \meaningof{E_2}\} }
\end{mathpar}

\begin{mathpar}
 \inferrule* [lab=behavior] {} {\meaningof{\langle a?b \rangle E} = \{ P \in \pi | P \equiv Q | u?(y)P', \\ \and \\\\ \and \\ \;\;\; u \in \meaningof{a}, \forall z.P'\{z/y\} \in \meaningof{E\{z/b\}}\}, \and \\ \meaningof{a!E} = \{ P \in \pi | P \equiv Q | x!\langle P' \rangle, x \in \meaningof{a} P' \in \meaningof{E}\} }
\end{mathpar}

\begin{mathpar}
 \inferrule* [lab=nominal] {} {\meaningof{\quotep{E}} = \{ \quotep{P} \in \quotep{\pi} | P \in \meaningof{E} \}, \and \meaningof{\quotep{P}} = \{ \quotep{Q} \in \quotep{\pi} | P \equiv Q \} \and \\ \meaningof{@\quotep{E}} = \{ P \in \pi | P \equiv @x, x \in \meaningof{E} \}}
\end{mathpar}

\begin{eqnarray*}
  \\
  \meaningof{-} : TS \to ST
\end{eqnarray*}

\begin{eqnarray*}
  \\
  L : TS \to ST
\end{eqnarray*}

\begin{eqnarray*}
  \\
  P \models E \iff P \in \meaningof{E}
\end{eqnarray*}

\begin{eqnarray*}
  P \approx_{L} Q \iff \forall E \in L. P \models E \iff Q \models E
\end{eqnarray*}

\begin{eqnarray*}
  P \approx_{K} Q
\end{eqnarray*}

\begin{eqnarray*}
  P \approx Q
\end{eqnarray*}

$\approx_{K} = \approx = \approx_{L}$

\subsubsection{Contextual duality}

Note that contexts extend the quotation operation to a family of
operations from processes to names. Given a context, $M$, we can
define a \emph{nominal context}, $\quotep{M}$ by $\quotep{M}[P] :=
\quotep{M[P]}$. To foreshadow what is to come we observe that these
operations enjoy a duality with processes very much like the duality
between vectors and maps from vectors to scalars.

Further, because the calculus is essentially higher-order, we have a
correspondence between contexts and processes. More specifically,
given a name $x$ and a context $M$ we can construct $M^{*}_{x}$ such
that 

\begin{mathpar}
  M^{*}_{x} | \lift{x}{P} \red M[P]
\end{mathpar}

namely,

\begin{mathpar}
  M^{*}_{x} := x?(u).M[\dropn{u}]
\end{mathpar}

The dependence of $M^{*}_{x}$ on a name makes it an abstraction, 

\begin{mathpar}
  M^{*} := (x)x?(u).M[\dropn{u}]
\end{mathpar}

\subsection{Additional notation}

It will sometimes be convenient to denote the process a name
quotes. We already have the notation $x = \quotep{P}$, but it will be
convenient to introduce an alternate notation, $\procn{x}$, when we
want to emphasize the connection to the use of the name. Note that, by
virtue of name equivalence, $\quotep{\procn{x}} \nameeq x$; so, the
notation is consistent with previous definitions.

Further, because names have structure it is possible to effect
substitutions on the basis of that structure. This means we need to
upgrade our notation for substitutions, which we accomplish by
adapting comprehension notation. Thus,

\begin{mathpar}
  P\{ y / x : x \in S \}
\end{mathpar}

is interpreted to mean the process derived from P by replacing (in a
capture-avoiding manner) each occurrence of $x$ in $S$ by $y$. For example,

\begin{mathpar}
  P\{ \quotep{\procn{x}|\procn{x}} / x : x \in \freenames{P} \}
\end{mathpar}

will replace each (occurrence) of a free name $x$ in $P$ by
$\quotep{\procn{x}|\procn{x}}$.

Also, we will avail ourselves of the notation $x^{L}$ and $x^{R}$ to
denote injections of a name into disjoint copies of the name
space. There are numerous ways to accomplish this. One example can be
found in \cite{MeredithR05}. This notation overloads to vectors of
names: $\vec{x}^{\pi} := (x_{i}^{\pi} \; : \; 0 \leq i < |\vec{x}| )$ where $\pi \in \{L,R\}$.

We also use $P^{\Box} := P|\Box$.

In \cite{MeredithR05} an interpretation of the new operator is
given. It turns out that there are several possible interpretations
all enjoying the requisite algebraic properties of the operator (see
\cite{milner91polyadicpi}). We will therefore make liberal use of
$(\nu\; \vec{x})P$.

% subsection the_syntax_and_semantics_of_the_notation_system (end)   

\input{qm2pi.qmops} 

\input{qm2pi.sterngerlach} 

\input{qm2pi.metric} 

% section concurrent_process_calculi (end)

%\input{qm2pi.proofsketch}

% section proof sketch (end)

%\input{qm2pi.slviaknots} 

% section spatial logic via knots (end)

\input{qm2pi.conclusion}

% section conclusion (end)

%\input{qm2pi.dtcodes} 

% section wiring algorithm (end)

\input{qm2pi.ack} 

% section acknowledgments (end)

\newpage


\bibliographystyle{plain}   
\bibliography{../../biblios/main.bib}

\input{qm2pi.rhodetails}

\end{document}

 

% section wiring algorithm (end)

\documentclass[12pt]{llncs}
%\documentclass{jktr}

\usepackage[pdftex]{hyperref}                   
\usepackage {listings}
\usepackage {mathpartir}
\usepackage{bcprules}
%\usepackage{listings}
                       
\usepackage{graphicx} 
%\usepackage[margins=2.5cm,nohead,nofoot]{geometry}
%\usepackage{geometry}
\usepackage{amsfonts}
\usepackage{amstext}
\usepackage{latexsym}
\usepackage{amssymb}
\usepackage{color}


%\include{myPreamble}
\include{qm2pi.local} 

%\ifpdf
%\usepackage[pdftex]{graphicx}
%\else
%\usepackage{graphicx}
%\fi

 % \ifpdf
%  \usepackage{pdfsync}
%  \if


%\title{Brief Article}
%\author{David F. Snyder}
%\author{L.G. Meredith}

%\address{Dept. of Math., Texas State University--San Marcos, San Marcos, TX 78666}
       
\pagestyle{empty}


\begin{document}

\lstset{language=[Objective]Caml,frame=shadowbox}

\input{qm2pi.front}

% section front matter (end)

\input{qm2pi.intro} 
 
% section introduction (end)

% \input{qm2pi.knotations} 

% section notation (end)

\input{qm2pi.process.calculi} 

% section concurrent_process_calculi_and_spatial_logics_ (end)
    
%\input{qm2pi.knots2pi} 

%\input{qm2pi.trefoil} 

%\input{qm2pi.mainthm} 

% subsection basic_interpretation (end)

%\input{qm2pi.rho.presentation} 
\subsection{The syntax and semantics of the notation system}\label{sub:the_syntax_and_semantics_of_the_notation_system} % (fold)

We now summarize a technical presentation of the calculus that
embodies our theory of dynamics. The typical presentation of such a
calculus follows the style of giving generators and relations on
them. The grammar, below, describing term constructors, freely
generates the set of processes, $\Proc$. This set is then quotiented
by a relation known as structural congruence and it is over this set
that the notion of dynamics is expressed. This presentation is
essentially that of \cite{MeredithR05} with the addition of
polyadicity and summation. For readability we have relegated some of
the technical subtleties to an appendix.

\subsubsection{Process grammar}\label{subsub:process_grammar}

\begin{mathpar}
  \inferrule* [lab=synchronization] {} {{M} \bc \pzero \;|\; x?F \;|\; x!C }
  \and
  \inferrule* [lab=abstraction] {} {{F} \bc (x)P}
  \and
  \inferrule* [lab=concretion] {} {{C} \bc \langle Q \rangle}
  \and
  \inferrule* [lab=process] {} {{P,Q} \bc M \;| \;P|Q \;|\; @{x}}
  \and
  \inferrule* [lab=name] {} {{x} \bc \quotep{P}}
\end{mathpar} 

Note that $\vec{x}$ (resp. $\vec{P}$) denotes a vector of names
(resp. processes) of length $|\vec{x}|$ (resp. $|\vec{P}|$). We adopt
the following useful abbreviations.

\begin{mathpar}
   x?(\vec{y}).P := x.(\vec{y})P \and  x\clift{\vec{P}} := x.\clift{\vec{P}}
   \and x!(y) := \lift{x}{\dropn{y}}
   \and \Pi_{i=0}^{n-1}P_i := P_0 | \ldots | P_{n-1}
\end{mathpar}

\subsubsection{Structural congruence}

\paragraph{Free and bound names and alpha-equivalence.} At the
core of structural equivalence is alpha-equivalence which identifies
process that are the same up to a change of variable. Formally, we
recognize the distinction between free and bound names. The free names
of a process, $\freenames{P}$, may be calculated recursively as
follows:

\begin{mathpar}
\freenames{\pzero} := \emptyset
  \and \\
  \freenames{x?(y).P} := \{ x \} \cup (\freenames{P} \setminus \{ y \})
  \and 
  \freenames{x!\langle P \rangle} := \{ x \} \cup \{ P \} 
  \and \\
  \freenames{P|Q} := \freenames{P} \cup \freenames{Q}
  \and \\
  \freenames{@{x}} := \{ x \}
\end{mathpar}

$\pi$
$\quotep{\pi}$

$\freenames{-} : \pi \to \mathcal{P}(\quotep{\pi})$

\begin{eqnarray*}
  \freenames{\pzero} & := & \emptyset \\
  \freenames{x?(y).P} & := & \{ x \} \cup (\freenames{P} \setminus \{ y \}) \\
  \freenames{x!\langle P \rangle} & := & \{ x \} \cup \{ P \} \\
  \freenames{P|Q} & := & \freenames{P} \cup \freenames{Q} \\
  \freenames{\dropn{x}} & := & \{ x \}
\end{eqnarray*}

The bound names of a process, $\boundnames{P}$, are those names occurring in $P$
that are not free. For example, in $x?(y).0$, the name $x$ is free, while $y$ is bound.

\begin{mathpar}
  \inferrule* [lab=monoidal-laws] {} { P|Q \equiv Q|P \and P|0 \equiv P \and P|(Q|R) \equiv (P|Q)|R }
\end{mathpar}

\begin{mathpar}
  \inferrule* [lab=alpha-equivalence] {} { (x)P \equiv (y)P\{y/x\} \and y \not\in \freenames{P} }
\end{mathpar}

\begin{definition}
Then two processes, $P,Q$, are alpha-equivalent if $P = Q\{\vec{y}/\vec{x}\}$ for
some $\vec{x} \in \boundnames{Q},\vec{y} \in \boundnames{P}$, where $Q\{\vec{y}/\vec{x}\}$
denotes the capture-avoiding substitution of $\vec{y}$ for $\vec{x}$ in $Q$.
\end{definition}

\begin{definition}
  The {\em structural congruence} \cite{SangiorgiWalker} , $\equiv$,
  between processes is the least congruence containing
  alpha-equivalence, satisfying the abelian monoid laws
  (associativity, commutativity and $\pzero$ as identity) for parallel
  composition $|$ and for summation $+$.
\end{definition}

\subsection{Name equivalence}

We take name equivalence, written $\nameeq$, to be the smallest
equivalence relation generated by the following rules.

\begin{mathpar}
\inferrule*[lab=Quote-drop]
{ }
{ \quotep{@{x}} \nameeq x }

\inferrule*[lab=Struct-equiv]
{ P \scong Q }
{ \quotep{P} \nameeq \quotep{Q} }
\end{mathpar}

The astute reader will have noticed that the mutual recursion of names
and processes imposes a mutual recursion on alpha-equivalence and
structural equivalence via name-equivalence. Fortunately, all of this
works out pleasantly and we may calculate in the natural way, free of
concern. The reader interested in the details is referred to the
appendix \ref{appendix:rho_details}.

\subsection{Substitution}

We use $\Proc$ for the set of processes, $\QProc$ for the set of
names, and $\id{\{}\vec{y} / \vec{x} \id{\}}$ to denote partial maps,
$s : \QProc \rightarrow \QProc$. A map, $s$ lifts, uniquely, to a map
on process terms, $\widehat{s} : \Proc \rightarrow \Proc$ by the
following equations.

\begin{mathpar}
  (0) \psubstp{Q}{P} := 0 \\
  (R \juxtap S) \psubstp{Q}{P}
  :=    
  (R)\psubstp{Q}{P} \juxtap (S) \psubstp{Q}{P} \\
  (x?(y).R) \psubstp{Q}{P}    
  :=    
  (x)\substp{Q}{P} (z)\concat( (R \psubstn{z}{y}) \psubstp{Q}{P} ) \\
  (\lift{x}{R}) \psubstp{Q}{P}  
  :=
  \lift{(x)\substp{Q}{P}}{ R \psubstp{Q}{P} } \\
%   (\dropn{x})  \psubstp{Q}{P}       
%   := 
%   \left\{ 
%     \begin{array}{ccc} 
%       \dropn{\quotep{Q}} & & x \nameeq \quotep{P} \\
%       \dropn{x} & & otherwise \\
%     \end{array}
%   \right. 
  (\dropn{x})  \psubstp{Q}{P}       
  := 
  \left\{ 
    \begin{array}{ccc} 
      Q & & x \nameeq \quotep{P} \\
      \dropn{x} & & otherwise \\
    \end{array}
  \right.
\end{mathpar}
 

where

\begin{eqnarray}
  (x)\id{\{} \lpquote Q \rpquote / \lpquote P \rpquote \id{\}}            = 
  \left\{ 
    \begin{array}{ccc}
      \lpquote Q \rpquote & & x \nameeq \lpquote P \rpquote \\
      x & & otherwise \\
    \end{array}
  \right. \nonumber
\end{eqnarray}

and $z$ is chosen distinct from $\quotep{P}$, $\quotep{Q}$, the free
names in $Q$, and all the names in $R$. Our $\alpha$-equivalence will
be built in the standard way from this substitution.

\begin{remark}\label{rem:no_self_referential_names}
  One consequence of these definitions is that $\forall P. \quotep{P}
  \not\in \freenames{P}$.
\end{remark}

\subsection{ Dynamic quote: an example }

Anticipating something of what's to come, consider applying the
substitution, $\widehat{\id{\{}u / z \id{\}}}$, to the following pair
of processes, $\lift{w}{y!(z)}$ and $w[ \lpquote y!(z) \rpquote ]$.

\begin{eqnarray}
	\lift{w}{y!(z)}\widehat{\id{\{}u / z \id{\}}}
		& = &
		\lift{w}{y!(u)} \nonumber\\
	w[ \lpquote y!(z) \rpquote ] \widehat{ \id{\{}u / z \id{\}} }
		& = &
		w[ \lpquote y!(z) \rpquote ] \nonumber
\end{eqnarray}

Because the body of the process between quotes is impervious to
substitution, we get radically different answers. In fact, by
examining the first process in an input context,
e.g. $x?(z).\lift{w}{y!(z)}$, we see that the process under the lift
operator may be shaped by prefixed inputs binding a name inside it. In
this sense, the lift operator will be seen as a way to dynamically
construct processes before reifying them as names.

Finally equipped with these standard features we can present the
dynamics of the calculus.

\subsubsection{Operational semantics} 

Finally, we introduce the computational dynamics. What marks these
algebras as distinct from other more traditionally studied algebraic
structures, e.g. vector spaces or polynomial rings, is the manner in
which dynamics is captured. In traditional structures, dynamics is typically
expressed through morphisms between such structures, as in linear maps
between vector spaces or morphisms between rings. In algebras
associated with the semantics of computation, the dynamics is
expressed as part of the algebraic structure itself, through a
reduction reduction relation typically denoted by $\red$. Below, we
give a recursive presentation of this relation for the calculus used
in the encoding.

$\red \subseteq \pi \times \pi$
$\red : \pi \to \mathcal{P}(\pi)$

\begin{mathpar}
  \inferrule* [lab=Comm] { \textsf{match}( x_{src}, x_{trgt} ) } { x_{trgt}?(y)P \; | \; x_{src}!\langle {Q} \rangle \red P\{\quotep{Q}/y}\} }
  \and \\
  \inferrule* [lab=Par] {{P} \red {P}'} {{{P} | {Q}} \red {{P}' | {Q}}}
  \and
  \inferrule* [lab=Equiv]{{{P} \scong {P}'} \andalso {{P}' \red {Q}'} \andalso {{Q}' \scong {Q}}}{{P} \red {Q}}
\end{mathpar}

\begin{eqnarray*}
  match_{\equiv} (\quotep{P},\quotep{Q}) & := & P \equiv Q \\
  match_{\dagger}(\quotep{P},\quotep{Q}) & := & \forall R. P|Q \red^{*} R => R \red^{*} 0 \\
  match_{K}(\quotep{P},\quotep{Q}) & := & K \mbox{ for some context } K
\end{eqnarray*}

$u?(x)P | u!\langle Q \rangle \red P\{\quotep{Q}/x\}$

%We write $\wred$ for $\red^*$, and $P\red$ if $\exists Q $ such that $ P \red Q$.
We write $P\red$ if $\exists Q $ such that $ P \red Q$ and $P\not\red$, otherwise.

\section{Replication}

As mentioned before, it is known that replication (and hence
recursion) can be implemented in a higher-order process algebra
\cite{SangiorgiWalker}. As our first example of calculation with the
machinery thus far presented we give the construction explicitly in
the {\rhoc}.

\begin{eqnarray}
	D_{x} & := & \prefix{x}{y}{(\binpar{\outputp{x}{y}}{@{y}})} \nonumber\\
	\bangp_{x}{P} & := & \binpar{{x}!\langle{\binpar{D_{x}}{P}}\rangle}{D_{x}} \nonumber
\end{eqnarray}

\begin{eqnarray}
	\bangp_{x}{P} & & \nonumber\\
	=
	& {x}!\langle{(\prefix{x}{y}{(\outputp{x}{y} | @{y})) | P}}\rangle 
	      | \prefix{x}{y}{(\outputp{x}{y} | @{y})} & \nonumber\\
	\red
	& (\outputp{x}{y} | @{y})\substn{\quotep{(\prefix{x}{y}{(@{y} | \outputp{x}{y})) | P}}}{y} & \nonumber\\
	=
	& \outputp{x}{\quotep{(\prefix{x}{y}{(\outputp{x}{y} | @{y})) | P}}}
	  | {(\prefix{x}{y}{(\outputp{x}{y} | @{y})) | P}} & \nonumber\\
	\red
	& \ldots & \nonumber\\
	\red^*
	& P | P | \ldots & \nonumber
\end{eqnarray}

Of course, this encoding, as an implementation, runs away, unfolding
$\bangp{P}$ eagerly. A lazier and more implementable replication
operator, restricted to input-guarded processes, may be obtained as follows.

\begin{eqnarray}
\bangp{\prefix{u}{v}{P}} 
	:= 
	\binpar{\lift{x}{\prefix{u}{v}{(\binpar{D(x)}{P})}}}{D(x)} \nonumber
\end{eqnarray}

\begin{remark}
  Note that the lazier definition still does not deal with summation
  or mixed summation (i.e. sums over input and output). The reader is
  invited to construct definitions of replication that deal with these
  features. 

  Further, the definitions are parameterized in a name, $x$. Can you,
  gentle reader, make a definition that eliminates this parameter and
  guarantees no accidental interaction between the replication
  machinery and the process being replicated -- i.e. no accidental
  sharing of names used by the process to get its work done and the
  name(s) used by the replication to effect copying. This latter
  revision of the definition of replication is crucial to obtaining
  the expected identity $!!P \sim !P$.
\end{remark}

\begin{remark}\label{rem:paradoxical_combinator}
  The reader familiar with the lambda calculus will have noticed the
  similarity between $D$ and the paradoxical combinator.

  [Ed. note: the existence of this seems to suggest we have to be more
  restrictive on the set of processes and names we admit if we are to
  support no-cloning.]
\end{remark}

\subsubsection{Bisimulation}

The computational dynamics gives rise to another kind of equivalence,
the equivalence of computational behavior. As previously mentioned
this is typically captured \emph{via} some form of bisimulation.

% The notion we use in this paper is weak barbed bisimulation
% \cite{milner91polyadicpi}.

The notion we use in this paper is derived from weak barbed
bisimulation \cite{milner91polyadicpi}. 

\begin{definition}
An \emph{observation relation}, $\downarrow_{\mathcal N}$, over a set
of names, $\mathcal N$, is the smallest relation satisfying the rules
below.

\infrule[Out-barb]{y \in {\mathcal N}, \; x \nameeq y}
		  {\outputp{x}{v} \downarrow_{\mathcal N} x}
\infrule[Par-barb]{\mbox{$P\downarrow_{\mathcal N} x$ or $Q\downarrow_{\mathcal N} x$}}
		  {\binpar{P}{Q} \downarrow_{\mathcal N} x}

We write $P \Downarrow_{\mathcal N} x$ if there is $Q$ such that 
$P \wred Q$ and $Q \downarrow_{\mathcal N} x$.
\end{definition}

\begin{definition}
%\label{def.bbisim}
An  ${\mathcal N}$-\emph{barbed bisimulation} over a set of names, ${\mathcal N}$, is a symmetric binary relation 
${\mathcal S}_{\mathcal N}$ between agents such that $P\rel{S}_{\mathcal N}Q$ implies:
\begin{enumerate}
\item If $P \red P'$ then $Q \wred Q'$ and $P'\rel{S}_{\mathcal N} Q'$.
\item If $P\downarrow_{\mathcal N} x$, then $Q\Downarrow_{\mathcal N} x$.
\end{enumerate}
$P$ is ${\mathcal N}$-barbed bisimilar to $Q$, written
$P \wbbisim_{\mathcal N} Q$, if $P \rel{S}_{\mathcal N} Q$ for some ${\mathcal N}$-barbed bisimulation ${\mathcal S}_{\mathcal N}$.
\end{definition}

$\mathcal{R} \subseteq \pi \times \pi$

$P \mathcal{R} Q => \forall P'. P \red P' \Rightarrow \exists Q'. Q \red Q', P' \mathcal{R} Q'$

$P \vdash x \Rightarrow Q \vdash x$

\begin{mathpar}
  \inferrule*[lab=Out-barb]{x \nameeq y}{{y}!\langle{Q}\rangle \vdash x}
  \and
  \inferrule*[lab=Par-barb]{\mbox{$P\vdash x$ or $Q\vdash x$}}{\binpar{P}{Q} \vdash x}
\end{mathpar}

\subsubsection{Contexts}

One of the principle advantages of computational calculi like the
$\pi$-calculus is a well-defined notion of context,
contextual-equivalence and a correlation between
contextual-equivalence and notions of bisimulation. The notion of
context allows the decomposition of a process into (sub-)process and
its syntactic environment, its context. Thus, a context may be
thought of as a process with a ``hole'' (written $\Box$) in it. The
application of a context $M$ to a process $P$, written $M[P]$, is
tantamount to filling the hole in $M$ with $P$. In this paper we do
not need the full weight of this theory, but do make use of the notion
of context in the proof the main theorem. 

\begin{mathpar}
  \inferrule* [lab=summation] {} {{M_{M},M_{N}} \bc \Box \;|\; x.M_{A} \;|\; M_{M}+M_{N}}
  \and
  \inferrule* [lab=agent] {} {{M_{A}} \bc (\vec{x})M_{P} \;| \; \clift{P_0,\ldots,M_{P},\ldots,P_N}}
  \and \\
  \inferrule* [lab=process] {} {{M_{P}} \bc M_{N} \;| \;P|M_{P} }
\end{mathpar} 

\begin{mathpar}
  \inferrule* [lab=sychronization] {} {M_{N} \bc \Box \;|\; x?M_{F} \;|\; x!M_{C}}
  \and
  \inferrule* [lab=abstraction] {} {{M_{F}} \bc (x)M_{P} }
  \and
  \inferrule* [lab=concretion] {} {{M_{C}} \bc \langle M_{P} \rangle }
  \and \\
  \inferrule* [lab=process] {} {{M_{P}} \bc M_{N} \;| \;P|M_{P} }
\end{mathpar}

\begin{definition}[contextual application] Given a context $M$, and
  process $P$, we define the \emph{contextual application}, $M[P] :=
  M\{P/\Box\}$. That is, the contextual application of M to P is the
  substitution of $P$ for $\Box$ in $M$.
\end{definition}

$\meaningof{-} : L \to \mathcal{P}(\pi)$

\begin{mathpar}
  \inferrule* [lab=collection] {} {\meaningof{true} = \pi, \and \meaningof{~E} = \pi \setminus \meaningof{E}, \and \meaningof{E_{1} \& E_{2}} = \meaningof{E_{1}} \cap \meaningof{E_{2}}}
\end{mathpar}

\begin{mathpar}
  \inferrule* [lab=structure] {} {\meaningof{0} = \{ P \in \pi | P \equiv 0 \}, \and \\ \meaningof{E_1 | E_2} = \{ P \in \pi | P \equiv P_{1} | P_{2}, P_{1} \in \meaningof{E_{1}}, P_{2} \in \meaningof{E_2}\} }
\end{mathpar}

\begin{mathpar}
 \inferrule* [lab=behavior] {} {\meaningof{\langle a?b \rangle E} = \{ P \in \pi | P \equiv Q | u?(y)P', \\ \and \\\\ \and \\ \;\;\; u \in \meaningof{a}, \forall z.P'\{z/y\} \in \meaningof{E\{z/b\}}\}, \and \\ \meaningof{a!E} = \{ P \in \pi | P \equiv Q | x!\langle P' \rangle, x \in \meaningof{a} P' \in \meaningof{E}\} }
\end{mathpar}

\begin{mathpar}
 \inferrule* [lab=nominal] {} {\meaningof{\quotep{E}} = \{ \quotep{P} \in \quotep{\pi} | P \in \meaningof{E} \}, \and \meaningof{\quotep{P}} = \{ \quotep{Q} \in \quotep{\pi} | P \equiv Q \} \and \\ \meaningof{@\quotep{E}} = \{ P \in \pi | P \equiv @x, x \in \meaningof{E} \}}
\end{mathpar}

\begin{eqnarray*}
  \\
  \meaningof{-} : TS \to ST
\end{eqnarray*}

\begin{eqnarray*}
  \\
  L : TS \to ST
\end{eqnarray*}

\begin{eqnarray*}
  \\
  P \models E \iff P \in \meaningof{E}
\end{eqnarray*}

\begin{eqnarray*}
  P \approx_{L} Q \iff \forall E \in L. P \models E \iff Q \models E
\end{eqnarray*}

\begin{eqnarray*}
  P \approx_{K} Q
\end{eqnarray*}

\begin{eqnarray*}
  P \approx Q
\end{eqnarray*}

$\approx_{K} = \approx = \approx_{L}$

\subsubsection{Contextual duality}

Note that contexts extend the quotation operation to a family of
operations from processes to names. Given a context, $M$, we can
define a \emph{nominal context}, $\quotep{M}$ by $\quotep{M}[P] :=
\quotep{M[P]}$. To foreshadow what is to come we observe that these
operations enjoy a duality with processes very much like the duality
between vectors and maps from vectors to scalars.

Further, because the calculus is essentially higher-order, we have a
correspondence between contexts and processes. More specifically,
given a name $x$ and a context $M$ we can construct $M^{*}_{x}$ such
that 

\begin{mathpar}
  M^{*}_{x} | \lift{x}{P} \red M[P]
\end{mathpar}

namely,

\begin{mathpar}
  M^{*}_{x} := x?(u).M[\dropn{u}]
\end{mathpar}

The dependence of $M^{*}_{x}$ on a name makes it an abstraction, 

\begin{mathpar}
  M^{*} := (x)x?(u).M[\dropn{u}]
\end{mathpar}

\subsection{Additional notation}

It will sometimes be convenient to denote the process a name
quotes. We already have the notation $x = \quotep{P}$, but it will be
convenient to introduce an alternate notation, $\procn{x}$, when we
want to emphasize the connection to the use of the name. Note that, by
virtue of name equivalence, $\quotep{\procn{x}} \nameeq x$; so, the
notation is consistent with previous definitions.

Further, because names have structure it is possible to effect
substitutions on the basis of that structure. This means we need to
upgrade our notation for substitutions, which we accomplish by
adapting comprehension notation. Thus,

\begin{mathpar}
  P\{ y / x : x \in S \}
\end{mathpar}

is interpreted to mean the process derived from P by replacing (in a
capture-avoiding manner) each occurrence of $x$ in $S$ by $y$. For example,

\begin{mathpar}
  P\{ \quotep{\procn{x}|\procn{x}} / x : x \in \freenames{P} \}
\end{mathpar}

will replace each (occurrence) of a free name $x$ in $P$ by
$\quotep{\procn{x}|\procn{x}}$.

Also, we will avail ourselves of the notation $x^{L}$ and $x^{R}$ to
denote injections of a name into disjoint copies of the name
space. There are numerous ways to accomplish this. One example can be
found in \cite{MeredithR05}. This notation overloads to vectors of
names: $\vec{x}^{\pi} := (x_{i}^{\pi} \; : \; 0 \leq i < |\vec{x}| )$ where $\pi \in \{L,R\}$.

We also use $P^{\Box} := P|\Box$.

In \cite{MeredithR05} an interpretation of the new operator is
given. It turns out that there are several possible interpretations
all enjoying the requisite algebraic properties of the operator (see
\cite{milner91polyadicpi}). We will therefore make liberal use of
$(\nu\; \vec{x})P$.

% subsection the_syntax_and_semantics_of_the_notation_system (end)   

\input{qm2pi.qmops} 

\input{qm2pi.sterngerlach} 

\input{qm2pi.metric} 

% section concurrent_process_calculi (end)

%\input{qm2pi.proofsketch}

% section proof sketch (end)

%\input{qm2pi.slviaknots} 

% section spatial logic via knots (end)

\input{qm2pi.conclusion}

% section conclusion (end)

%\input{qm2pi.dtcodes} 

% section wiring algorithm (end)

\input{qm2pi.ack} 

% section acknowledgments (end)

\newpage


\bibliographystyle{plain}   
\bibliography{../../biblios/main.bib}

\input{qm2pi.rhodetails}

\end{document}

 

% section acknowledgments (end)

\newpage


\bibliographystyle{plain}   
\bibliography{../../biblios/main.bib}

\documentclass[12pt]{llncs}
%\documentclass{jktr}

\usepackage[pdftex]{hyperref}                   
\usepackage {listings}
\usepackage {mathpartir}
\usepackage{bcprules}
%\usepackage{listings}
                       
\usepackage{graphicx} 
%\usepackage[margins=2.5cm,nohead,nofoot]{geometry}
%\usepackage{geometry}
\usepackage{amsfonts}
\usepackage{amstext}
\usepackage{latexsym}
\usepackage{amssymb}
\usepackage{color}


%\include{myPreamble}
\include{qm2pi.local} 

%\ifpdf
%\usepackage[pdftex]{graphicx}
%\else
%\usepackage{graphicx}
%\fi

 % \ifpdf
%  \usepackage{pdfsync}
%  \if


%\title{Brief Article}
%\author{David F. Snyder}
%\author{L.G. Meredith}

%\address{Dept. of Math., Texas State University--San Marcos, San Marcos, TX 78666}
       
\pagestyle{empty}


\begin{document}

\lstset{language=[Objective]Caml,frame=shadowbox}

\input{qm2pi.front}

% section front matter (end)

\input{qm2pi.intro} 
 
% section introduction (end)

% \input{qm2pi.knotations} 

% section notation (end)

\input{qm2pi.process.calculi} 

% section concurrent_process_calculi_and_spatial_logics_ (end)
    
%\input{qm2pi.knots2pi} 

%\input{qm2pi.trefoil} 

%\input{qm2pi.mainthm} 

% subsection basic_interpretation (end)

%\input{qm2pi.rho.presentation} 
\subsection{The syntax and semantics of the notation system}\label{sub:the_syntax_and_semantics_of_the_notation_system} % (fold)

We now summarize a technical presentation of the calculus that
embodies our theory of dynamics. The typical presentation of such a
calculus follows the style of giving generators and relations on
them. The grammar, below, describing term constructors, freely
generates the set of processes, $\Proc$. This set is then quotiented
by a relation known as structural congruence and it is over this set
that the notion of dynamics is expressed. This presentation is
essentially that of \cite{MeredithR05} with the addition of
polyadicity and summation. For readability we have relegated some of
the technical subtleties to an appendix.

\subsubsection{Process grammar}\label{subsub:process_grammar}

\begin{mathpar}
  \inferrule* [lab=synchronization] {} {{M} \bc \pzero \;|\; x?F \;|\; x!C }
  \and
  \inferrule* [lab=abstraction] {} {{F} \bc (x)P}
  \and
  \inferrule* [lab=concretion] {} {{C} \bc \langle Q \rangle}
  \and
  \inferrule* [lab=process] {} {{P,Q} \bc M \;| \;P|Q \;|\; @{x}}
  \and
  \inferrule* [lab=name] {} {{x} \bc \quotep{P}}
\end{mathpar} 

Note that $\vec{x}$ (resp. $\vec{P}$) denotes a vector of names
(resp. processes) of length $|\vec{x}|$ (resp. $|\vec{P}|$). We adopt
the following useful abbreviations.

\begin{mathpar}
   x?(\vec{y}).P := x.(\vec{y})P \and  x\clift{\vec{P}} := x.\clift{\vec{P}}
   \and x!(y) := \lift{x}{\dropn{y}}
   \and \Pi_{i=0}^{n-1}P_i := P_0 | \ldots | P_{n-1}
\end{mathpar}

\subsubsection{Structural congruence}

\paragraph{Free and bound names and alpha-equivalence.} At the
core of structural equivalence is alpha-equivalence which identifies
process that are the same up to a change of variable. Formally, we
recognize the distinction between free and bound names. The free names
of a process, $\freenames{P}$, may be calculated recursively as
follows:

\begin{mathpar}
\freenames{\pzero} := \emptyset
  \and \\
  \freenames{x?(y).P} := \{ x \} \cup (\freenames{P} \setminus \{ y \})
  \and 
  \freenames{x!\langle P \rangle} := \{ x \} \cup \{ P \} 
  \and \\
  \freenames{P|Q} := \freenames{P} \cup \freenames{Q}
  \and \\
  \freenames{@{x}} := \{ x \}
\end{mathpar}

$\pi$
$\quotep{\pi}$

$\freenames{-} : \pi \to \mathcal{P}(\quotep{\pi})$

\begin{eqnarray*}
  \freenames{\pzero} & := & \emptyset \\
  \freenames{x?(y).P} & := & \{ x \} \cup (\freenames{P} \setminus \{ y \}) \\
  \freenames{x!\langle P \rangle} & := & \{ x \} \cup \{ P \} \\
  \freenames{P|Q} & := & \freenames{P} \cup \freenames{Q} \\
  \freenames{\dropn{x}} & := & \{ x \}
\end{eqnarray*}

The bound names of a process, $\boundnames{P}$, are those names occurring in $P$
that are not free. For example, in $x?(y).0$, the name $x$ is free, while $y$ is bound.

\begin{mathpar}
  \inferrule* [lab=monoidal-laws] {} { P|Q \equiv Q|P \and P|0 \equiv P \and P|(Q|R) \equiv (P|Q)|R }
\end{mathpar}

\begin{mathpar}
  \inferrule* [lab=alpha-equivalence] {} { (x)P \equiv (y)P\{y/x\} \and y \not\in \freenames{P} }
\end{mathpar}

\begin{definition}
Then two processes, $P,Q$, are alpha-equivalent if $P = Q\{\vec{y}/\vec{x}\}$ for
some $\vec{x} \in \boundnames{Q},\vec{y} \in \boundnames{P}$, where $Q\{\vec{y}/\vec{x}\}$
denotes the capture-avoiding substitution of $\vec{y}$ for $\vec{x}$ in $Q$.
\end{definition}

\begin{definition}
  The {\em structural congruence} \cite{SangiorgiWalker} , $\equiv$,
  between processes is the least congruence containing
  alpha-equivalence, satisfying the abelian monoid laws
  (associativity, commutativity and $\pzero$ as identity) for parallel
  composition $|$ and for summation $+$.
\end{definition}

\subsection{Name equivalence}

We take name equivalence, written $\nameeq$, to be the smallest
equivalence relation generated by the following rules.

\begin{mathpar}
\inferrule*[lab=Quote-drop]
{ }
{ \quotep{@{x}} \nameeq x }

\inferrule*[lab=Struct-equiv]
{ P \scong Q }
{ \quotep{P} \nameeq \quotep{Q} }
\end{mathpar}

The astute reader will have noticed that the mutual recursion of names
and processes imposes a mutual recursion on alpha-equivalence and
structural equivalence via name-equivalence. Fortunately, all of this
works out pleasantly and we may calculate in the natural way, free of
concern. The reader interested in the details is referred to the
appendix \ref{appendix:rho_details}.

\subsection{Substitution}

We use $\Proc$ for the set of processes, $\QProc$ for the set of
names, and $\id{\{}\vec{y} / \vec{x} \id{\}}$ to denote partial maps,
$s : \QProc \rightarrow \QProc$. A map, $s$ lifts, uniquely, to a map
on process terms, $\widehat{s} : \Proc \rightarrow \Proc$ by the
following equations.

\begin{mathpar}
  (0) \psubstp{Q}{P} := 0 \\
  (R \juxtap S) \psubstp{Q}{P}
  :=    
  (R)\psubstp{Q}{P} \juxtap (S) \psubstp{Q}{P} \\
  (x?(y).R) \psubstp{Q}{P}    
  :=    
  (x)\substp{Q}{P} (z)\concat( (R \psubstn{z}{y}) \psubstp{Q}{P} ) \\
  (\lift{x}{R}) \psubstp{Q}{P}  
  :=
  \lift{(x)\substp{Q}{P}}{ R \psubstp{Q}{P} } \\
%   (\dropn{x})  \psubstp{Q}{P}       
%   := 
%   \left\{ 
%     \begin{array}{ccc} 
%       \dropn{\quotep{Q}} & & x \nameeq \quotep{P} \\
%       \dropn{x} & & otherwise \\
%     \end{array}
%   \right. 
  (\dropn{x})  \psubstp{Q}{P}       
  := 
  \left\{ 
    \begin{array}{ccc} 
      Q & & x \nameeq \quotep{P} \\
      \dropn{x} & & otherwise \\
    \end{array}
  \right.
\end{mathpar}
 

where

\begin{eqnarray}
  (x)\id{\{} \lpquote Q \rpquote / \lpquote P \rpquote \id{\}}            = 
  \left\{ 
    \begin{array}{ccc}
      \lpquote Q \rpquote & & x \nameeq \lpquote P \rpquote \\
      x & & otherwise \\
    \end{array}
  \right. \nonumber
\end{eqnarray}

and $z$ is chosen distinct from $\quotep{P}$, $\quotep{Q}$, the free
names in $Q$, and all the names in $R$. Our $\alpha$-equivalence will
be built in the standard way from this substitution.

\begin{remark}\label{rem:no_self_referential_names}
  One consequence of these definitions is that $\forall P. \quotep{P}
  \not\in \freenames{P}$.
\end{remark}

\subsection{ Dynamic quote: an example }

Anticipating something of what's to come, consider applying the
substitution, $\widehat{\id{\{}u / z \id{\}}}$, to the following pair
of processes, $\lift{w}{y!(z)}$ and $w[ \lpquote y!(z) \rpquote ]$.

\begin{eqnarray}
	\lift{w}{y!(z)}\widehat{\id{\{}u / z \id{\}}}
		& = &
		\lift{w}{y!(u)} \nonumber\\
	w[ \lpquote y!(z) \rpquote ] \widehat{ \id{\{}u / z \id{\}} }
		& = &
		w[ \lpquote y!(z) \rpquote ] \nonumber
\end{eqnarray}

Because the body of the process between quotes is impervious to
substitution, we get radically different answers. In fact, by
examining the first process in an input context,
e.g. $x?(z).\lift{w}{y!(z)}$, we see that the process under the lift
operator may be shaped by prefixed inputs binding a name inside it. In
this sense, the lift operator will be seen as a way to dynamically
construct processes before reifying them as names.

Finally equipped with these standard features we can present the
dynamics of the calculus.

\subsubsection{Operational semantics} 

Finally, we introduce the computational dynamics. What marks these
algebras as distinct from other more traditionally studied algebraic
structures, e.g. vector spaces or polynomial rings, is the manner in
which dynamics is captured. In traditional structures, dynamics is typically
expressed through morphisms between such structures, as in linear maps
between vector spaces or morphisms between rings. In algebras
associated with the semantics of computation, the dynamics is
expressed as part of the algebraic structure itself, through a
reduction reduction relation typically denoted by $\red$. Below, we
give a recursive presentation of this relation for the calculus used
in the encoding.

$\red \subseteq \pi \times \pi$
$\red : \pi \to \mathcal{P}(\pi)$

\begin{mathpar}
  \inferrule* [lab=Comm] { \textsf{match}( x_{src}, x_{trgt} ) } { x_{trgt}?(y)P \; | \; x_{src}!\langle {Q} \rangle \red P\{\quotep{Q}/y}\} }
  \and \\
  \inferrule* [lab=Par] {{P} \red {P}'} {{{P} | {Q}} \red {{P}' | {Q}}}
  \and
  \inferrule* [lab=Equiv]{{{P} \scong {P}'} \andalso {{P}' \red {Q}'} \andalso {{Q}' \scong {Q}}}{{P} \red {Q}}
\end{mathpar}

\begin{eqnarray*}
  match_{\equiv} (\quotep{P},\quotep{Q}) & := & P \equiv Q \\
  match_{\dagger}(\quotep{P},\quotep{Q}) & := & \forall R. P|Q \red^{*} R => R \red^{*} 0 \\
  match_{K}(\quotep{P},\quotep{Q}) & := & K \mbox{ for some context } K
\end{eqnarray*}

$u?(x)P | u!\langle Q \rangle \red P\{\quotep{Q}/x\}$

%We write $\wred$ for $\red^*$, and $P\red$ if $\exists Q $ such that $ P \red Q$.
We write $P\red$ if $\exists Q $ such that $ P \red Q$ and $P\not\red$, otherwise.

\section{Replication}

As mentioned before, it is known that replication (and hence
recursion) can be implemented in a higher-order process algebra
\cite{SangiorgiWalker}. As our first example of calculation with the
machinery thus far presented we give the construction explicitly in
the {\rhoc}.

\begin{eqnarray}
	D_{x} & := & \prefix{x}{y}{(\binpar{\outputp{x}{y}}{@{y}})} \nonumber\\
	\bangp_{x}{P} & := & \binpar{{x}!\langle{\binpar{D_{x}}{P}}\rangle}{D_{x}} \nonumber
\end{eqnarray}

\begin{eqnarray}
	\bangp_{x}{P} & & \nonumber\\
	=
	& {x}!\langle{(\prefix{x}{y}{(\outputp{x}{y} | @{y})) | P}}\rangle 
	      | \prefix{x}{y}{(\outputp{x}{y} | @{y})} & \nonumber\\
	\red
	& (\outputp{x}{y} | @{y})\substn{\quotep{(\prefix{x}{y}{(@{y} | \outputp{x}{y})) | P}}}{y} & \nonumber\\
	=
	& \outputp{x}{\quotep{(\prefix{x}{y}{(\outputp{x}{y} | @{y})) | P}}}
	  | {(\prefix{x}{y}{(\outputp{x}{y} | @{y})) | P}} & \nonumber\\
	\red
	& \ldots & \nonumber\\
	\red^*
	& P | P | \ldots & \nonumber
\end{eqnarray}

Of course, this encoding, as an implementation, runs away, unfolding
$\bangp{P}$ eagerly. A lazier and more implementable replication
operator, restricted to input-guarded processes, may be obtained as follows.

\begin{eqnarray}
\bangp{\prefix{u}{v}{P}} 
	:= 
	\binpar{\lift{x}{\prefix{u}{v}{(\binpar{D(x)}{P})}}}{D(x)} \nonumber
\end{eqnarray}

\begin{remark}
  Note that the lazier definition still does not deal with summation
  or mixed summation (i.e. sums over input and output). The reader is
  invited to construct definitions of replication that deal with these
  features. 

  Further, the definitions are parameterized in a name, $x$. Can you,
  gentle reader, make a definition that eliminates this parameter and
  guarantees no accidental interaction between the replication
  machinery and the process being replicated -- i.e. no accidental
  sharing of names used by the process to get its work done and the
  name(s) used by the replication to effect copying. This latter
  revision of the definition of replication is crucial to obtaining
  the expected identity $!!P \sim !P$.
\end{remark}

\begin{remark}\label{rem:paradoxical_combinator}
  The reader familiar with the lambda calculus will have noticed the
  similarity between $D$ and the paradoxical combinator.

  [Ed. note: the existence of this seems to suggest we have to be more
  restrictive on the set of processes and names we admit if we are to
  support no-cloning.]
\end{remark}

\subsubsection{Bisimulation}

The computational dynamics gives rise to another kind of equivalence,
the equivalence of computational behavior. As previously mentioned
this is typically captured \emph{via} some form of bisimulation.

% The notion we use in this paper is weak barbed bisimulation
% \cite{milner91polyadicpi}.

The notion we use in this paper is derived from weak barbed
bisimulation \cite{milner91polyadicpi}. 

\begin{definition}
An \emph{observation relation}, $\downarrow_{\mathcal N}$, over a set
of names, $\mathcal N$, is the smallest relation satisfying the rules
below.

\infrule[Out-barb]{y \in {\mathcal N}, \; x \nameeq y}
		  {\outputp{x}{v} \downarrow_{\mathcal N} x}
\infrule[Par-barb]{\mbox{$P\downarrow_{\mathcal N} x$ or $Q\downarrow_{\mathcal N} x$}}
		  {\binpar{P}{Q} \downarrow_{\mathcal N} x}

We write $P \Downarrow_{\mathcal N} x$ if there is $Q$ such that 
$P \wred Q$ and $Q \downarrow_{\mathcal N} x$.
\end{definition}

\begin{definition}
%\label{def.bbisim}
An  ${\mathcal N}$-\emph{barbed bisimulation} over a set of names, ${\mathcal N}$, is a symmetric binary relation 
${\mathcal S}_{\mathcal N}$ between agents such that $P\rel{S}_{\mathcal N}Q$ implies:
\begin{enumerate}
\item If $P \red P'$ then $Q \wred Q'$ and $P'\rel{S}_{\mathcal N} Q'$.
\item If $P\downarrow_{\mathcal N} x$, then $Q\Downarrow_{\mathcal N} x$.
\end{enumerate}
$P$ is ${\mathcal N}$-barbed bisimilar to $Q$, written
$P \wbbisim_{\mathcal N} Q$, if $P \rel{S}_{\mathcal N} Q$ for some ${\mathcal N}$-barbed bisimulation ${\mathcal S}_{\mathcal N}$.
\end{definition}

$\mathcal{R} \subseteq \pi \times \pi$

$P \mathcal{R} Q => \forall P'. P \red P' \Rightarrow \exists Q'. Q \red Q', P' \mathcal{R} Q'$

$P \vdash x \Rightarrow Q \vdash x$

\begin{mathpar}
  \inferrule*[lab=Out-barb]{x \nameeq y}{{y}!\langle{Q}\rangle \vdash x}
  \and
  \inferrule*[lab=Par-barb]{\mbox{$P\vdash x$ or $Q\vdash x$}}{\binpar{P}{Q} \vdash x}
\end{mathpar}

\subsubsection{Contexts}

One of the principle advantages of computational calculi like the
$\pi$-calculus is a well-defined notion of context,
contextual-equivalence and a correlation between
contextual-equivalence and notions of bisimulation. The notion of
context allows the decomposition of a process into (sub-)process and
its syntactic environment, its context. Thus, a context may be
thought of as a process with a ``hole'' (written $\Box$) in it. The
application of a context $M$ to a process $P$, written $M[P]$, is
tantamount to filling the hole in $M$ with $P$. In this paper we do
not need the full weight of this theory, but do make use of the notion
of context in the proof the main theorem. 

\begin{mathpar}
  \inferrule* [lab=summation] {} {{M_{M},M_{N}} \bc \Box \;|\; x.M_{A} \;|\; M_{M}+M_{N}}
  \and
  \inferrule* [lab=agent] {} {{M_{A}} \bc (\vec{x})M_{P} \;| \; \clift{P_0,\ldots,M_{P},\ldots,P_N}}
  \and \\
  \inferrule* [lab=process] {} {{M_{P}} \bc M_{N} \;| \;P|M_{P} }
\end{mathpar} 

\begin{mathpar}
  \inferrule* [lab=sychronization] {} {M_{N} \bc \Box \;|\; x?M_{F} \;|\; x!M_{C}}
  \and
  \inferrule* [lab=abstraction] {} {{M_{F}} \bc (x)M_{P} }
  \and
  \inferrule* [lab=concretion] {} {{M_{C}} \bc \langle M_{P} \rangle }
  \and \\
  \inferrule* [lab=process] {} {{M_{P}} \bc M_{N} \;| \;P|M_{P} }
\end{mathpar}

\begin{definition}[contextual application] Given a context $M$, and
  process $P$, we define the \emph{contextual application}, $M[P] :=
  M\{P/\Box\}$. That is, the contextual application of M to P is the
  substitution of $P$ for $\Box$ in $M$.
\end{definition}

$\meaningof{-} : L \to \mathcal{P}(\pi)$

\begin{mathpar}
  \inferrule* [lab=collection] {} {\meaningof{true} = \pi, \and \meaningof{~E} = \pi \setminus \meaningof{E}, \and \meaningof{E_{1} \& E_{2}} = \meaningof{E_{1}} \cap \meaningof{E_{2}}}
\end{mathpar}

\begin{mathpar}
  \inferrule* [lab=structure] {} {\meaningof{0} = \{ P \in \pi | P \equiv 0 \}, \and \\ \meaningof{E_1 | E_2} = \{ P \in \pi | P \equiv P_{1} | P_{2}, P_{1} \in \meaningof{E_{1}}, P_{2} \in \meaningof{E_2}\} }
\end{mathpar}

\begin{mathpar}
 \inferrule* [lab=behavior] {} {\meaningof{\langle a?b \rangle E} = \{ P \in \pi | P \equiv Q | u?(y)P', \\ \and \\\\ \and \\ \;\;\; u \in \meaningof{a}, \forall z.P'\{z/y\} \in \meaningof{E\{z/b\}}\}, \and \\ \meaningof{a!E} = \{ P \in \pi | P \equiv Q | x!\langle P' \rangle, x \in \meaningof{a} P' \in \meaningof{E}\} }
\end{mathpar}

\begin{mathpar}
 \inferrule* [lab=nominal] {} {\meaningof{\quotep{E}} = \{ \quotep{P} \in \quotep{\pi} | P \in \meaningof{E} \}, \and \meaningof{\quotep{P}} = \{ \quotep{Q} \in \quotep{\pi} | P \equiv Q \} \and \\ \meaningof{@\quotep{E}} = \{ P \in \pi | P \equiv @x, x \in \meaningof{E} \}}
\end{mathpar}

\begin{eqnarray*}
  \\
  \meaningof{-} : TS \to ST
\end{eqnarray*}

\begin{eqnarray*}
  \\
  L : TS \to ST
\end{eqnarray*}

\begin{eqnarray*}
  \\
  P \models E \iff P \in \meaningof{E}
\end{eqnarray*}

\begin{eqnarray*}
  P \approx_{L} Q \iff \forall E \in L. P \models E \iff Q \models E
\end{eqnarray*}

\begin{eqnarray*}
  P \approx_{K} Q
\end{eqnarray*}

\begin{eqnarray*}
  P \approx Q
\end{eqnarray*}

$\approx_{K} = \approx = \approx_{L}$

\subsubsection{Contextual duality}

Note that contexts extend the quotation operation to a family of
operations from processes to names. Given a context, $M$, we can
define a \emph{nominal context}, $\quotep{M}$ by $\quotep{M}[P] :=
\quotep{M[P]}$. To foreshadow what is to come we observe that these
operations enjoy a duality with processes very much like the duality
between vectors and maps from vectors to scalars.

Further, because the calculus is essentially higher-order, we have a
correspondence between contexts and processes. More specifically,
given a name $x$ and a context $M$ we can construct $M^{*}_{x}$ such
that 

\begin{mathpar}
  M^{*}_{x} | \lift{x}{P} \red M[P]
\end{mathpar}

namely,

\begin{mathpar}
  M^{*}_{x} := x?(u).M[\dropn{u}]
\end{mathpar}

The dependence of $M^{*}_{x}$ on a name makes it an abstraction, 

\begin{mathpar}
  M^{*} := (x)x?(u).M[\dropn{u}]
\end{mathpar}

\subsection{Additional notation}

It will sometimes be convenient to denote the process a name
quotes. We already have the notation $x = \quotep{P}$, but it will be
convenient to introduce an alternate notation, $\procn{x}$, when we
want to emphasize the connection to the use of the name. Note that, by
virtue of name equivalence, $\quotep{\procn{x}} \nameeq x$; so, the
notation is consistent with previous definitions.

Further, because names have structure it is possible to effect
substitutions on the basis of that structure. This means we need to
upgrade our notation for substitutions, which we accomplish by
adapting comprehension notation. Thus,

\begin{mathpar}
  P\{ y / x : x \in S \}
\end{mathpar}

is interpreted to mean the process derived from P by replacing (in a
capture-avoiding manner) each occurrence of $x$ in $S$ by $y$. For example,

\begin{mathpar}
  P\{ \quotep{\procn{x}|\procn{x}} / x : x \in \freenames{P} \}
\end{mathpar}

will replace each (occurrence) of a free name $x$ in $P$ by
$\quotep{\procn{x}|\procn{x}}$.

Also, we will avail ourselves of the notation $x^{L}$ and $x^{R}$ to
denote injections of a name into disjoint copies of the name
space. There are numerous ways to accomplish this. One example can be
found in \cite{MeredithR05}. This notation overloads to vectors of
names: $\vec{x}^{\pi} := (x_{i}^{\pi} \; : \; 0 \leq i < |\vec{x}| )$ where $\pi \in \{L,R\}$.

We also use $P^{\Box} := P|\Box$.

In \cite{MeredithR05} an interpretation of the new operator is
given. It turns out that there are several possible interpretations
all enjoying the requisite algebraic properties of the operator (see
\cite{milner91polyadicpi}). We will therefore make liberal use of
$(\nu\; \vec{x})P$.

% subsection the_syntax_and_semantics_of_the_notation_system (end)   

\input{qm2pi.qmops} 

\input{qm2pi.sterngerlach} 

\input{qm2pi.metric} 

% section concurrent_process_calculi (end)

%\input{qm2pi.proofsketch}

% section proof sketch (end)

%\input{qm2pi.slviaknots} 

% section spatial logic via knots (end)

\input{qm2pi.conclusion}

% section conclusion (end)

%\input{qm2pi.dtcodes} 

% section wiring algorithm (end)

\input{qm2pi.ack} 

% section acknowledgments (end)

\newpage


\bibliographystyle{plain}   
\bibliography{../../biblios/main.bib}

\input{qm2pi.rhodetails}

\end{document}



\end{document}

 

% section notation (end)

\input{qm2pi.process.calculi} 

% section concurrent_process_calculi_and_spatial_logics_ (end)
    
%\documentclass[12pt]{llncs}
%\documentclass{jktr}

\usepackage[pdftex]{hyperref}                   
\usepackage {listings}
\usepackage {mathpartir}
\usepackage{bcprules}
%\usepackage{listings}
                       
\usepackage{graphicx} 
%\usepackage[margins=2.5cm,nohead,nofoot]{geometry}
%\usepackage{geometry}
\usepackage{amsfonts}
\usepackage{amstext}
\usepackage{latexsym}
\usepackage{amssymb}
\usepackage{color}


%\include{myPreamble}
\documentclass[12pt]{llncs}
%\documentclass{jktr}

\usepackage[pdftex]{hyperref}                   
\usepackage {listings}
\usepackage {mathpartir}
\usepackage{bcprules}
%\usepackage{listings}
                       
\usepackage{graphicx} 
%\usepackage[margins=2.5cm,nohead,nofoot]{geometry}
%\usepackage{geometry}
\usepackage{amsfonts}
\usepackage{amstext}
\usepackage{latexsym}
\usepackage{amssymb}
\usepackage{color}


%\include{myPreamble}
\include{qm2pi.local} 

%\ifpdf
%\usepackage[pdftex]{graphicx}
%\else
%\usepackage{graphicx}
%\fi

 % \ifpdf
%  \usepackage{pdfsync}
%  \if


%\title{Brief Article}
%\author{David F. Snyder}
%\author{L.G. Meredith}

%\address{Dept. of Math., Texas State University--San Marcos, San Marcos, TX 78666}
       
\pagestyle{empty}


\begin{document}

\lstset{language=[Objective]Caml,frame=shadowbox}

\input{qm2pi.front}

% section front matter (end)

\input{qm2pi.intro} 
 
% section introduction (end)

% \input{qm2pi.knotations} 

% section notation (end)

\input{qm2pi.process.calculi} 

% section concurrent_process_calculi_and_spatial_logics_ (end)
    
%\input{qm2pi.knots2pi} 

%\input{qm2pi.trefoil} 

%\input{qm2pi.mainthm} 

% subsection basic_interpretation (end)

%\input{qm2pi.rho.presentation} 
\subsection{The syntax and semantics of the notation system}\label{sub:the_syntax_and_semantics_of_the_notation_system} % (fold)

We now summarize a technical presentation of the calculus that
embodies our theory of dynamics. The typical presentation of such a
calculus follows the style of giving generators and relations on
them. The grammar, below, describing term constructors, freely
generates the set of processes, $\Proc$. This set is then quotiented
by a relation known as structural congruence and it is over this set
that the notion of dynamics is expressed. This presentation is
essentially that of \cite{MeredithR05} with the addition of
polyadicity and summation. For readability we have relegated some of
the technical subtleties to an appendix.

\subsubsection{Process grammar}\label{subsub:process_grammar}

\begin{mathpar}
  \inferrule* [lab=synchronization] {} {{M} \bc \pzero \;|\; x?F \;|\; x!C }
  \and
  \inferrule* [lab=abstraction] {} {{F} \bc (x)P}
  \and
  \inferrule* [lab=concretion] {} {{C} \bc \langle Q \rangle}
  \and
  \inferrule* [lab=process] {} {{P,Q} \bc M \;| \;P|Q \;|\; @{x}}
  \and
  \inferrule* [lab=name] {} {{x} \bc \quotep{P}}
\end{mathpar} 

Note that $\vec{x}$ (resp. $\vec{P}$) denotes a vector of names
(resp. processes) of length $|\vec{x}|$ (resp. $|\vec{P}|$). We adopt
the following useful abbreviations.

\begin{mathpar}
   x?(\vec{y}).P := x.(\vec{y})P \and  x\clift{\vec{P}} := x.\clift{\vec{P}}
   \and x!(y) := \lift{x}{\dropn{y}}
   \and \Pi_{i=0}^{n-1}P_i := P_0 | \ldots | P_{n-1}
\end{mathpar}

\subsubsection{Structural congruence}

\paragraph{Free and bound names and alpha-equivalence.} At the
core of structural equivalence is alpha-equivalence which identifies
process that are the same up to a change of variable. Formally, we
recognize the distinction between free and bound names. The free names
of a process, $\freenames{P}$, may be calculated recursively as
follows:

\begin{mathpar}
\freenames{\pzero} := \emptyset
  \and \\
  \freenames{x?(y).P} := \{ x \} \cup (\freenames{P} \setminus \{ y \})
  \and 
  \freenames{x!\langle P \rangle} := \{ x \} \cup \{ P \} 
  \and \\
  \freenames{P|Q} := \freenames{P} \cup \freenames{Q}
  \and \\
  \freenames{@{x}} := \{ x \}
\end{mathpar}

$\pi$
$\quotep{\pi}$

$\freenames{-} : \pi \to \mathcal{P}(\quotep{\pi})$

\begin{eqnarray*}
  \freenames{\pzero} & := & \emptyset \\
  \freenames{x?(y).P} & := & \{ x \} \cup (\freenames{P} \setminus \{ y \}) \\
  \freenames{x!\langle P \rangle} & := & \{ x \} \cup \{ P \} \\
  \freenames{P|Q} & := & \freenames{P} \cup \freenames{Q} \\
  \freenames{\dropn{x}} & := & \{ x \}
\end{eqnarray*}

The bound names of a process, $\boundnames{P}$, are those names occurring in $P$
that are not free. For example, in $x?(y).0$, the name $x$ is free, while $y$ is bound.

\begin{mathpar}
  \inferrule* [lab=monoidal-laws] {} { P|Q \equiv Q|P \and P|0 \equiv P \and P|(Q|R) \equiv (P|Q)|R }
\end{mathpar}

\begin{mathpar}
  \inferrule* [lab=alpha-equivalence] {} { (x)P \equiv (y)P\{y/x\} \and y \not\in \freenames{P} }
\end{mathpar}

\begin{definition}
Then two processes, $P,Q$, are alpha-equivalent if $P = Q\{\vec{y}/\vec{x}\}$ for
some $\vec{x} \in \boundnames{Q},\vec{y} \in \boundnames{P}$, where $Q\{\vec{y}/\vec{x}\}$
denotes the capture-avoiding substitution of $\vec{y}$ for $\vec{x}$ in $Q$.
\end{definition}

\begin{definition}
  The {\em structural congruence} \cite{SangiorgiWalker} , $\equiv$,
  between processes is the least congruence containing
  alpha-equivalence, satisfying the abelian monoid laws
  (associativity, commutativity and $\pzero$ as identity) for parallel
  composition $|$ and for summation $+$.
\end{definition}

\subsection{Name equivalence}

We take name equivalence, written $\nameeq$, to be the smallest
equivalence relation generated by the following rules.

\begin{mathpar}
\inferrule*[lab=Quote-drop]
{ }
{ \quotep{@{x}} \nameeq x }

\inferrule*[lab=Struct-equiv]
{ P \scong Q }
{ \quotep{P} \nameeq \quotep{Q} }
\end{mathpar}

The astute reader will have noticed that the mutual recursion of names
and processes imposes a mutual recursion on alpha-equivalence and
structural equivalence via name-equivalence. Fortunately, all of this
works out pleasantly and we may calculate in the natural way, free of
concern. The reader interested in the details is referred to the
appendix \ref{appendix:rho_details}.

\subsection{Substitution}

We use $\Proc$ for the set of processes, $\QProc$ for the set of
names, and $\id{\{}\vec{y} / \vec{x} \id{\}}$ to denote partial maps,
$s : \QProc \rightarrow \QProc$. A map, $s$ lifts, uniquely, to a map
on process terms, $\widehat{s} : \Proc \rightarrow \Proc$ by the
following equations.

\begin{mathpar}
  (0) \psubstp{Q}{P} := 0 \\
  (R \juxtap S) \psubstp{Q}{P}
  :=    
  (R)\psubstp{Q}{P} \juxtap (S) \psubstp{Q}{P} \\
  (x?(y).R) \psubstp{Q}{P}    
  :=    
  (x)\substp{Q}{P} (z)\concat( (R \psubstn{z}{y}) \psubstp{Q}{P} ) \\
  (\lift{x}{R}) \psubstp{Q}{P}  
  :=
  \lift{(x)\substp{Q}{P}}{ R \psubstp{Q}{P} } \\
%   (\dropn{x})  \psubstp{Q}{P}       
%   := 
%   \left\{ 
%     \begin{array}{ccc} 
%       \dropn{\quotep{Q}} & & x \nameeq \quotep{P} \\
%       \dropn{x} & & otherwise \\
%     \end{array}
%   \right. 
  (\dropn{x})  \psubstp{Q}{P}       
  := 
  \left\{ 
    \begin{array}{ccc} 
      Q & & x \nameeq \quotep{P} \\
      \dropn{x} & & otherwise \\
    \end{array}
  \right.
\end{mathpar}
 

where

\begin{eqnarray}
  (x)\id{\{} \lpquote Q \rpquote / \lpquote P \rpquote \id{\}}            = 
  \left\{ 
    \begin{array}{ccc}
      \lpquote Q \rpquote & & x \nameeq \lpquote P \rpquote \\
      x & & otherwise \\
    \end{array}
  \right. \nonumber
\end{eqnarray}

and $z$ is chosen distinct from $\quotep{P}$, $\quotep{Q}$, the free
names in $Q$, and all the names in $R$. Our $\alpha$-equivalence will
be built in the standard way from this substitution.

\begin{remark}\label{rem:no_self_referential_names}
  One consequence of these definitions is that $\forall P. \quotep{P}
  \not\in \freenames{P}$.
\end{remark}

\subsection{ Dynamic quote: an example }

Anticipating something of what's to come, consider applying the
substitution, $\widehat{\id{\{}u / z \id{\}}}$, to the following pair
of processes, $\lift{w}{y!(z)}$ and $w[ \lpquote y!(z) \rpquote ]$.

\begin{eqnarray}
	\lift{w}{y!(z)}\widehat{\id{\{}u / z \id{\}}}
		& = &
		\lift{w}{y!(u)} \nonumber\\
	w[ \lpquote y!(z) \rpquote ] \widehat{ \id{\{}u / z \id{\}} }
		& = &
		w[ \lpquote y!(z) \rpquote ] \nonumber
\end{eqnarray}

Because the body of the process between quotes is impervious to
substitution, we get radically different answers. In fact, by
examining the first process in an input context,
e.g. $x?(z).\lift{w}{y!(z)}$, we see that the process under the lift
operator may be shaped by prefixed inputs binding a name inside it. In
this sense, the lift operator will be seen as a way to dynamically
construct processes before reifying them as names.

Finally equipped with these standard features we can present the
dynamics of the calculus.

\subsubsection{Operational semantics} 

Finally, we introduce the computational dynamics. What marks these
algebras as distinct from other more traditionally studied algebraic
structures, e.g. vector spaces or polynomial rings, is the manner in
which dynamics is captured. In traditional structures, dynamics is typically
expressed through morphisms between such structures, as in linear maps
between vector spaces or morphisms between rings. In algebras
associated with the semantics of computation, the dynamics is
expressed as part of the algebraic structure itself, through a
reduction reduction relation typically denoted by $\red$. Below, we
give a recursive presentation of this relation for the calculus used
in the encoding.

$\red \subseteq \pi \times \pi$
$\red : \pi \to \mathcal{P}(\pi)$

\begin{mathpar}
  \inferrule* [lab=Comm] { \textsf{match}( x_{src}, x_{trgt} ) } { x_{trgt}?(y)P \; | \; x_{src}!\langle {Q} \rangle \red P\{\quotep{Q}/y}\} }
  \and \\
  \inferrule* [lab=Par] {{P} \red {P}'} {{{P} | {Q}} \red {{P}' | {Q}}}
  \and
  \inferrule* [lab=Equiv]{{{P} \scong {P}'} \andalso {{P}' \red {Q}'} \andalso {{Q}' \scong {Q}}}{{P} \red {Q}}
\end{mathpar}

\begin{eqnarray*}
  match_{\equiv} (\quotep{P},\quotep{Q}) & := & P \equiv Q \\
  match_{\dagger}(\quotep{P},\quotep{Q}) & := & \forall R. P|Q \red^{*} R => R \red^{*} 0 \\
  match_{K}(\quotep{P},\quotep{Q}) & := & K \mbox{ for some context } K
\end{eqnarray*}

$u?(x)P | u!\langle Q \rangle \red P\{\quotep{Q}/x\}$

%We write $\wred$ for $\red^*$, and $P\red$ if $\exists Q $ such that $ P \red Q$.
We write $P\red$ if $\exists Q $ such that $ P \red Q$ and $P\not\red$, otherwise.

\section{Replication}

As mentioned before, it is known that replication (and hence
recursion) can be implemented in a higher-order process algebra
\cite{SangiorgiWalker}. As our first example of calculation with the
machinery thus far presented we give the construction explicitly in
the {\rhoc}.

\begin{eqnarray}
	D_{x} & := & \prefix{x}{y}{(\binpar{\outputp{x}{y}}{@{y}})} \nonumber\\
	\bangp_{x}{P} & := & \binpar{{x}!\langle{\binpar{D_{x}}{P}}\rangle}{D_{x}} \nonumber
\end{eqnarray}

\begin{eqnarray}
	\bangp_{x}{P} & & \nonumber\\
	=
	& {x}!\langle{(\prefix{x}{y}{(\outputp{x}{y} | @{y})) | P}}\rangle 
	      | \prefix{x}{y}{(\outputp{x}{y} | @{y})} & \nonumber\\
	\red
	& (\outputp{x}{y} | @{y})\substn{\quotep{(\prefix{x}{y}{(@{y} | \outputp{x}{y})) | P}}}{y} & \nonumber\\
	=
	& \outputp{x}{\quotep{(\prefix{x}{y}{(\outputp{x}{y} | @{y})) | P}}}
	  | {(\prefix{x}{y}{(\outputp{x}{y} | @{y})) | P}} & \nonumber\\
	\red
	& \ldots & \nonumber\\
	\red^*
	& P | P | \ldots & \nonumber
\end{eqnarray}

Of course, this encoding, as an implementation, runs away, unfolding
$\bangp{P}$ eagerly. A lazier and more implementable replication
operator, restricted to input-guarded processes, may be obtained as follows.

\begin{eqnarray}
\bangp{\prefix{u}{v}{P}} 
	:= 
	\binpar{\lift{x}{\prefix{u}{v}{(\binpar{D(x)}{P})}}}{D(x)} \nonumber
\end{eqnarray}

\begin{remark}
  Note that the lazier definition still does not deal with summation
  or mixed summation (i.e. sums over input and output). The reader is
  invited to construct definitions of replication that deal with these
  features. 

  Further, the definitions are parameterized in a name, $x$. Can you,
  gentle reader, make a definition that eliminates this parameter and
  guarantees no accidental interaction between the replication
  machinery and the process being replicated -- i.e. no accidental
  sharing of names used by the process to get its work done and the
  name(s) used by the replication to effect copying. This latter
  revision of the definition of replication is crucial to obtaining
  the expected identity $!!P \sim !P$.
\end{remark}

\begin{remark}\label{rem:paradoxical_combinator}
  The reader familiar with the lambda calculus will have noticed the
  similarity between $D$ and the paradoxical combinator.

  [Ed. note: the existence of this seems to suggest we have to be more
  restrictive on the set of processes and names we admit if we are to
  support no-cloning.]
\end{remark}

\subsubsection{Bisimulation}

The computational dynamics gives rise to another kind of equivalence,
the equivalence of computational behavior. As previously mentioned
this is typically captured \emph{via} some form of bisimulation.

% The notion we use in this paper is weak barbed bisimulation
% \cite{milner91polyadicpi}.

The notion we use in this paper is derived from weak barbed
bisimulation \cite{milner91polyadicpi}. 

\begin{definition}
An \emph{observation relation}, $\downarrow_{\mathcal N}$, over a set
of names, $\mathcal N$, is the smallest relation satisfying the rules
below.

\infrule[Out-barb]{y \in {\mathcal N}, \; x \nameeq y}
		  {\outputp{x}{v} \downarrow_{\mathcal N} x}
\infrule[Par-barb]{\mbox{$P\downarrow_{\mathcal N} x$ or $Q\downarrow_{\mathcal N} x$}}
		  {\binpar{P}{Q} \downarrow_{\mathcal N} x}

We write $P \Downarrow_{\mathcal N} x$ if there is $Q$ such that 
$P \wred Q$ and $Q \downarrow_{\mathcal N} x$.
\end{definition}

\begin{definition}
%\label{def.bbisim}
An  ${\mathcal N}$-\emph{barbed bisimulation} over a set of names, ${\mathcal N}$, is a symmetric binary relation 
${\mathcal S}_{\mathcal N}$ between agents such that $P\rel{S}_{\mathcal N}Q$ implies:
\begin{enumerate}
\item If $P \red P'$ then $Q \wred Q'$ and $P'\rel{S}_{\mathcal N} Q'$.
\item If $P\downarrow_{\mathcal N} x$, then $Q\Downarrow_{\mathcal N} x$.
\end{enumerate}
$P$ is ${\mathcal N}$-barbed bisimilar to $Q$, written
$P \wbbisim_{\mathcal N} Q$, if $P \rel{S}_{\mathcal N} Q$ for some ${\mathcal N}$-barbed bisimulation ${\mathcal S}_{\mathcal N}$.
\end{definition}

$\mathcal{R} \subseteq \pi \times \pi$

$P \mathcal{R} Q => \forall P'. P \red P' \Rightarrow \exists Q'. Q \red Q', P' \mathcal{R} Q'$

$P \vdash x \Rightarrow Q \vdash x$

\begin{mathpar}
  \inferrule*[lab=Out-barb]{x \nameeq y}{{y}!\langle{Q}\rangle \vdash x}
  \and
  \inferrule*[lab=Par-barb]{\mbox{$P\vdash x$ or $Q\vdash x$}}{\binpar{P}{Q} \vdash x}
\end{mathpar}

\subsubsection{Contexts}

One of the principle advantages of computational calculi like the
$\pi$-calculus is a well-defined notion of context,
contextual-equivalence and a correlation between
contextual-equivalence and notions of bisimulation. The notion of
context allows the decomposition of a process into (sub-)process and
its syntactic environment, its context. Thus, a context may be
thought of as a process with a ``hole'' (written $\Box$) in it. The
application of a context $M$ to a process $P$, written $M[P]$, is
tantamount to filling the hole in $M$ with $P$. In this paper we do
not need the full weight of this theory, but do make use of the notion
of context in the proof the main theorem. 

\begin{mathpar}
  \inferrule* [lab=summation] {} {{M_{M},M_{N}} \bc \Box \;|\; x.M_{A} \;|\; M_{M}+M_{N}}
  \and
  \inferrule* [lab=agent] {} {{M_{A}} \bc (\vec{x})M_{P} \;| \; \clift{P_0,\ldots,M_{P},\ldots,P_N}}
  \and \\
  \inferrule* [lab=process] {} {{M_{P}} \bc M_{N} \;| \;P|M_{P} }
\end{mathpar} 

\begin{mathpar}
  \inferrule* [lab=sychronization] {} {M_{N} \bc \Box \;|\; x?M_{F} \;|\; x!M_{C}}
  \and
  \inferrule* [lab=abstraction] {} {{M_{F}} \bc (x)M_{P} }
  \and
  \inferrule* [lab=concretion] {} {{M_{C}} \bc \langle M_{P} \rangle }
  \and \\
  \inferrule* [lab=process] {} {{M_{P}} \bc M_{N} \;| \;P|M_{P} }
\end{mathpar}

\begin{definition}[contextual application] Given a context $M$, and
  process $P$, we define the \emph{contextual application}, $M[P] :=
  M\{P/\Box\}$. That is, the contextual application of M to P is the
  substitution of $P$ for $\Box$ in $M$.
\end{definition}

$\meaningof{-} : L \to \mathcal{P}(\pi)$

\begin{mathpar}
  \inferrule* [lab=collection] {} {\meaningof{true} = \pi, \and \meaningof{~E} = \pi \setminus \meaningof{E}, \and \meaningof{E_{1} \& E_{2}} = \meaningof{E_{1}} \cap \meaningof{E_{2}}}
\end{mathpar}

\begin{mathpar}
  \inferrule* [lab=structure] {} {\meaningof{0} = \{ P \in \pi | P \equiv 0 \}, \and \\ \meaningof{E_1 | E_2} = \{ P \in \pi | P \equiv P_{1} | P_{2}, P_{1} \in \meaningof{E_{1}}, P_{2} \in \meaningof{E_2}\} }
\end{mathpar}

\begin{mathpar}
 \inferrule* [lab=behavior] {} {\meaningof{\langle a?b \rangle E} = \{ P \in \pi | P \equiv Q | u?(y)P', \\ \and \\\\ \and \\ \;\;\; u \in \meaningof{a}, \forall z.P'\{z/y\} \in \meaningof{E\{z/b\}}\}, \and \\ \meaningof{a!E} = \{ P \in \pi | P \equiv Q | x!\langle P' \rangle, x \in \meaningof{a} P' \in \meaningof{E}\} }
\end{mathpar}

\begin{mathpar}
 \inferrule* [lab=nominal] {} {\meaningof{\quotep{E}} = \{ \quotep{P} \in \quotep{\pi} | P \in \meaningof{E} \}, \and \meaningof{\quotep{P}} = \{ \quotep{Q} \in \quotep{\pi} | P \equiv Q \} \and \\ \meaningof{@\quotep{E}} = \{ P \in \pi | P \equiv @x, x \in \meaningof{E} \}}
\end{mathpar}

\begin{eqnarray*}
  \\
  \meaningof{-} : TS \to ST
\end{eqnarray*}

\begin{eqnarray*}
  \\
  L : TS \to ST
\end{eqnarray*}

\begin{eqnarray*}
  \\
  P \models E \iff P \in \meaningof{E}
\end{eqnarray*}

\begin{eqnarray*}
  P \approx_{L} Q \iff \forall E \in L. P \models E \iff Q \models E
\end{eqnarray*}

\begin{eqnarray*}
  P \approx_{K} Q
\end{eqnarray*}

\begin{eqnarray*}
  P \approx Q
\end{eqnarray*}

$\approx_{K} = \approx = \approx_{L}$

\subsubsection{Contextual duality}

Note that contexts extend the quotation operation to a family of
operations from processes to names. Given a context, $M$, we can
define a \emph{nominal context}, $\quotep{M}$ by $\quotep{M}[P] :=
\quotep{M[P]}$. To foreshadow what is to come we observe that these
operations enjoy a duality with processes very much like the duality
between vectors and maps from vectors to scalars.

Further, because the calculus is essentially higher-order, we have a
correspondence between contexts and processes. More specifically,
given a name $x$ and a context $M$ we can construct $M^{*}_{x}$ such
that 

\begin{mathpar}
  M^{*}_{x} | \lift{x}{P} \red M[P]
\end{mathpar}

namely,

\begin{mathpar}
  M^{*}_{x} := x?(u).M[\dropn{u}]
\end{mathpar}

The dependence of $M^{*}_{x}$ on a name makes it an abstraction, 

\begin{mathpar}
  M^{*} := (x)x?(u).M[\dropn{u}]
\end{mathpar}

\subsection{Additional notation}

It will sometimes be convenient to denote the process a name
quotes. We already have the notation $x = \quotep{P}$, but it will be
convenient to introduce an alternate notation, $\procn{x}$, when we
want to emphasize the connection to the use of the name. Note that, by
virtue of name equivalence, $\quotep{\procn{x}} \nameeq x$; so, the
notation is consistent with previous definitions.

Further, because names have structure it is possible to effect
substitutions on the basis of that structure. This means we need to
upgrade our notation for substitutions, which we accomplish by
adapting comprehension notation. Thus,

\begin{mathpar}
  P\{ y / x : x \in S \}
\end{mathpar}

is interpreted to mean the process derived from P by replacing (in a
capture-avoiding manner) each occurrence of $x$ in $S$ by $y$. For example,

\begin{mathpar}
  P\{ \quotep{\procn{x}|\procn{x}} / x : x \in \freenames{P} \}
\end{mathpar}

will replace each (occurrence) of a free name $x$ in $P$ by
$\quotep{\procn{x}|\procn{x}}$.

Also, we will avail ourselves of the notation $x^{L}$ and $x^{R}$ to
denote injections of a name into disjoint copies of the name
space. There are numerous ways to accomplish this. One example can be
found in \cite{MeredithR05}. This notation overloads to vectors of
names: $\vec{x}^{\pi} := (x_{i}^{\pi} \; : \; 0 \leq i < |\vec{x}| )$ where $\pi \in \{L,R\}$.

We also use $P^{\Box} := P|\Box$.

In \cite{MeredithR05} an interpretation of the new operator is
given. It turns out that there are several possible interpretations
all enjoying the requisite algebraic properties of the operator (see
\cite{milner91polyadicpi}). We will therefore make liberal use of
$(\nu\; \vec{x})P$.

% subsection the_syntax_and_semantics_of_the_notation_system (end)   

\input{qm2pi.qmops} 

\input{qm2pi.sterngerlach} 

\input{qm2pi.metric} 

% section concurrent_process_calculi (end)

%\input{qm2pi.proofsketch}

% section proof sketch (end)

%\input{qm2pi.slviaknots} 

% section spatial logic via knots (end)

\input{qm2pi.conclusion}

% section conclusion (end)

%\input{qm2pi.dtcodes} 

% section wiring algorithm (end)

\input{qm2pi.ack} 

% section acknowledgments (end)

\newpage


\bibliographystyle{plain}   
\bibliography{../../biblios/main.bib}

\input{qm2pi.rhodetails}

\end{document}

 

%\ifpdf
%\usepackage[pdftex]{graphicx}
%\else
%\usepackage{graphicx}
%\fi

 % \ifpdf
%  \usepackage{pdfsync}
%  \if


%\title{Brief Article}
%\author{David F. Snyder}
%\author{L.G. Meredith}

%\address{Dept. of Math., Texas State University--San Marcos, San Marcos, TX 78666}
       
\pagestyle{empty}


\begin{document}

\lstset{language=[Objective]Caml,frame=shadowbox}

\documentclass[12pt]{llncs}
%\documentclass{jktr}

\usepackage[pdftex]{hyperref}                   
\usepackage {listings}
\usepackage {mathpartir}
\usepackage{bcprules}
%\usepackage{listings}
                       
\usepackage{graphicx} 
%\usepackage[margins=2.5cm,nohead,nofoot]{geometry}
%\usepackage{geometry}
\usepackage{amsfonts}
\usepackage{amstext}
\usepackage{latexsym}
\usepackage{amssymb}
\usepackage{color}


%\include{myPreamble}
\include{qm2pi.local} 

%\ifpdf
%\usepackage[pdftex]{graphicx}
%\else
%\usepackage{graphicx}
%\fi

 % \ifpdf
%  \usepackage{pdfsync}
%  \if


%\title{Brief Article}
%\author{David F. Snyder}
%\author{L.G. Meredith}

%\address{Dept. of Math., Texas State University--San Marcos, San Marcos, TX 78666}
       
\pagestyle{empty}


\begin{document}

\lstset{language=[Objective]Caml,frame=shadowbox}

\input{qm2pi.front}

% section front matter (end)

\input{qm2pi.intro} 
 
% section introduction (end)

% \input{qm2pi.knotations} 

% section notation (end)

\input{qm2pi.process.calculi} 

% section concurrent_process_calculi_and_spatial_logics_ (end)
    
%\input{qm2pi.knots2pi} 

%\input{qm2pi.trefoil} 

%\input{qm2pi.mainthm} 

% subsection basic_interpretation (end)

%\input{qm2pi.rho.presentation} 
\subsection{The syntax and semantics of the notation system}\label{sub:the_syntax_and_semantics_of_the_notation_system} % (fold)

We now summarize a technical presentation of the calculus that
embodies our theory of dynamics. The typical presentation of such a
calculus follows the style of giving generators and relations on
them. The grammar, below, describing term constructors, freely
generates the set of processes, $\Proc$. This set is then quotiented
by a relation known as structural congruence and it is over this set
that the notion of dynamics is expressed. This presentation is
essentially that of \cite{MeredithR05} with the addition of
polyadicity and summation. For readability we have relegated some of
the technical subtleties to an appendix.

\subsubsection{Process grammar}\label{subsub:process_grammar}

\begin{mathpar}
  \inferrule* [lab=synchronization] {} {{M} \bc \pzero \;|\; x?F \;|\; x!C }
  \and
  \inferrule* [lab=abstraction] {} {{F} \bc (x)P}
  \and
  \inferrule* [lab=concretion] {} {{C} \bc \langle Q \rangle}
  \and
  \inferrule* [lab=process] {} {{P,Q} \bc M \;| \;P|Q \;|\; @{x}}
  \and
  \inferrule* [lab=name] {} {{x} \bc \quotep{P}}
\end{mathpar} 

Note that $\vec{x}$ (resp. $\vec{P}$) denotes a vector of names
(resp. processes) of length $|\vec{x}|$ (resp. $|\vec{P}|$). We adopt
the following useful abbreviations.

\begin{mathpar}
   x?(\vec{y}).P := x.(\vec{y})P \and  x\clift{\vec{P}} := x.\clift{\vec{P}}
   \and x!(y) := \lift{x}{\dropn{y}}
   \and \Pi_{i=0}^{n-1}P_i := P_0 | \ldots | P_{n-1}
\end{mathpar}

\subsubsection{Structural congruence}

\paragraph{Free and bound names and alpha-equivalence.} At the
core of structural equivalence is alpha-equivalence which identifies
process that are the same up to a change of variable. Formally, we
recognize the distinction between free and bound names. The free names
of a process, $\freenames{P}$, may be calculated recursively as
follows:

\begin{mathpar}
\freenames{\pzero} := \emptyset
  \and \\
  \freenames{x?(y).P} := \{ x \} \cup (\freenames{P} \setminus \{ y \})
  \and 
  \freenames{x!\langle P \rangle} := \{ x \} \cup \{ P \} 
  \and \\
  \freenames{P|Q} := \freenames{P} \cup \freenames{Q}
  \and \\
  \freenames{@{x}} := \{ x \}
\end{mathpar}

$\pi$
$\quotep{\pi}$

$\freenames{-} : \pi \to \mathcal{P}(\quotep{\pi})$

\begin{eqnarray*}
  \freenames{\pzero} & := & \emptyset \\
  \freenames{x?(y).P} & := & \{ x \} \cup (\freenames{P} \setminus \{ y \}) \\
  \freenames{x!\langle P \rangle} & := & \{ x \} \cup \{ P \} \\
  \freenames{P|Q} & := & \freenames{P} \cup \freenames{Q} \\
  \freenames{\dropn{x}} & := & \{ x \}
\end{eqnarray*}

The bound names of a process, $\boundnames{P}$, are those names occurring in $P$
that are not free. For example, in $x?(y).0$, the name $x$ is free, while $y$ is bound.

\begin{mathpar}
  \inferrule* [lab=monoidal-laws] {} { P|Q \equiv Q|P \and P|0 \equiv P \and P|(Q|R) \equiv (P|Q)|R }
\end{mathpar}

\begin{mathpar}
  \inferrule* [lab=alpha-equivalence] {} { (x)P \equiv (y)P\{y/x\} \and y \not\in \freenames{P} }
\end{mathpar}

\begin{definition}
Then two processes, $P,Q$, are alpha-equivalent if $P = Q\{\vec{y}/\vec{x}\}$ for
some $\vec{x} \in \boundnames{Q},\vec{y} \in \boundnames{P}$, where $Q\{\vec{y}/\vec{x}\}$
denotes the capture-avoiding substitution of $\vec{y}$ for $\vec{x}$ in $Q$.
\end{definition}

\begin{definition}
  The {\em structural congruence} \cite{SangiorgiWalker} , $\equiv$,
  between processes is the least congruence containing
  alpha-equivalence, satisfying the abelian monoid laws
  (associativity, commutativity and $\pzero$ as identity) for parallel
  composition $|$ and for summation $+$.
\end{definition}

\subsection{Name equivalence}

We take name equivalence, written $\nameeq$, to be the smallest
equivalence relation generated by the following rules.

\begin{mathpar}
\inferrule*[lab=Quote-drop]
{ }
{ \quotep{@{x}} \nameeq x }

\inferrule*[lab=Struct-equiv]
{ P \scong Q }
{ \quotep{P} \nameeq \quotep{Q} }
\end{mathpar}

The astute reader will have noticed that the mutual recursion of names
and processes imposes a mutual recursion on alpha-equivalence and
structural equivalence via name-equivalence. Fortunately, all of this
works out pleasantly and we may calculate in the natural way, free of
concern. The reader interested in the details is referred to the
appendix \ref{appendix:rho_details}.

\subsection{Substitution}

We use $\Proc$ for the set of processes, $\QProc$ for the set of
names, and $\id{\{}\vec{y} / \vec{x} \id{\}}$ to denote partial maps,
$s : \QProc \rightarrow \QProc$. A map, $s$ lifts, uniquely, to a map
on process terms, $\widehat{s} : \Proc \rightarrow \Proc$ by the
following equations.

\begin{mathpar}
  (0) \psubstp{Q}{P} := 0 \\
  (R \juxtap S) \psubstp{Q}{P}
  :=    
  (R)\psubstp{Q}{P} \juxtap (S) \psubstp{Q}{P} \\
  (x?(y).R) \psubstp{Q}{P}    
  :=    
  (x)\substp{Q}{P} (z)\concat( (R \psubstn{z}{y}) \psubstp{Q}{P} ) \\
  (\lift{x}{R}) \psubstp{Q}{P}  
  :=
  \lift{(x)\substp{Q}{P}}{ R \psubstp{Q}{P} } \\
%   (\dropn{x})  \psubstp{Q}{P}       
%   := 
%   \left\{ 
%     \begin{array}{ccc} 
%       \dropn{\quotep{Q}} & & x \nameeq \quotep{P} \\
%       \dropn{x} & & otherwise \\
%     \end{array}
%   \right. 
  (\dropn{x})  \psubstp{Q}{P}       
  := 
  \left\{ 
    \begin{array}{ccc} 
      Q & & x \nameeq \quotep{P} \\
      \dropn{x} & & otherwise \\
    \end{array}
  \right.
\end{mathpar}
 

where

\begin{eqnarray}
  (x)\id{\{} \lpquote Q \rpquote / \lpquote P \rpquote \id{\}}            = 
  \left\{ 
    \begin{array}{ccc}
      \lpquote Q \rpquote & & x \nameeq \lpquote P \rpquote \\
      x & & otherwise \\
    \end{array}
  \right. \nonumber
\end{eqnarray}

and $z$ is chosen distinct from $\quotep{P}$, $\quotep{Q}$, the free
names in $Q$, and all the names in $R$. Our $\alpha$-equivalence will
be built in the standard way from this substitution.

\begin{remark}\label{rem:no_self_referential_names}
  One consequence of these definitions is that $\forall P. \quotep{P}
  \not\in \freenames{P}$.
\end{remark}

\subsection{ Dynamic quote: an example }

Anticipating something of what's to come, consider applying the
substitution, $\widehat{\id{\{}u / z \id{\}}}$, to the following pair
of processes, $\lift{w}{y!(z)}$ and $w[ \lpquote y!(z) \rpquote ]$.

\begin{eqnarray}
	\lift{w}{y!(z)}\widehat{\id{\{}u / z \id{\}}}
		& = &
		\lift{w}{y!(u)} \nonumber\\
	w[ \lpquote y!(z) \rpquote ] \widehat{ \id{\{}u / z \id{\}} }
		& = &
		w[ \lpquote y!(z) \rpquote ] \nonumber
\end{eqnarray}

Because the body of the process between quotes is impervious to
substitution, we get radically different answers. In fact, by
examining the first process in an input context,
e.g. $x?(z).\lift{w}{y!(z)}$, we see that the process under the lift
operator may be shaped by prefixed inputs binding a name inside it. In
this sense, the lift operator will be seen as a way to dynamically
construct processes before reifying them as names.

Finally equipped with these standard features we can present the
dynamics of the calculus.

\subsubsection{Operational semantics} 

Finally, we introduce the computational dynamics. What marks these
algebras as distinct from other more traditionally studied algebraic
structures, e.g. vector spaces or polynomial rings, is the manner in
which dynamics is captured. In traditional structures, dynamics is typically
expressed through morphisms between such structures, as in linear maps
between vector spaces or morphisms between rings. In algebras
associated with the semantics of computation, the dynamics is
expressed as part of the algebraic structure itself, through a
reduction reduction relation typically denoted by $\red$. Below, we
give a recursive presentation of this relation for the calculus used
in the encoding.

$\red \subseteq \pi \times \pi$
$\red : \pi \to \mathcal{P}(\pi)$

\begin{mathpar}
  \inferrule* [lab=Comm] { \textsf{match}( x_{src}, x_{trgt} ) } { x_{trgt}?(y)P \; | \; x_{src}!\langle {Q} \rangle \red P\{\quotep{Q}/y}\} }
  \and \\
  \inferrule* [lab=Par] {{P} \red {P}'} {{{P} | {Q}} \red {{P}' | {Q}}}
  \and
  \inferrule* [lab=Equiv]{{{P} \scong {P}'} \andalso {{P}' \red {Q}'} \andalso {{Q}' \scong {Q}}}{{P} \red {Q}}
\end{mathpar}

\begin{eqnarray*}
  match_{\equiv} (\quotep{P},\quotep{Q}) & := & P \equiv Q \\
  match_{\dagger}(\quotep{P},\quotep{Q}) & := & \forall R. P|Q \red^{*} R => R \red^{*} 0 \\
  match_{K}(\quotep{P},\quotep{Q}) & := & K \mbox{ for some context } K
\end{eqnarray*}

$u?(x)P | u!\langle Q \rangle \red P\{\quotep{Q}/x\}$

%We write $\wred$ for $\red^*$, and $P\red$ if $\exists Q $ such that $ P \red Q$.
We write $P\red$ if $\exists Q $ such that $ P \red Q$ and $P\not\red$, otherwise.

\section{Replication}

As mentioned before, it is known that replication (and hence
recursion) can be implemented in a higher-order process algebra
\cite{SangiorgiWalker}. As our first example of calculation with the
machinery thus far presented we give the construction explicitly in
the {\rhoc}.

\begin{eqnarray}
	D_{x} & := & \prefix{x}{y}{(\binpar{\outputp{x}{y}}{@{y}})} \nonumber\\
	\bangp_{x}{P} & := & \binpar{{x}!\langle{\binpar{D_{x}}{P}}\rangle}{D_{x}} \nonumber
\end{eqnarray}

\begin{eqnarray}
	\bangp_{x}{P} & & \nonumber\\
	=
	& {x}!\langle{(\prefix{x}{y}{(\outputp{x}{y} | @{y})) | P}}\rangle 
	      | \prefix{x}{y}{(\outputp{x}{y} | @{y})} & \nonumber\\
	\red
	& (\outputp{x}{y} | @{y})\substn{\quotep{(\prefix{x}{y}{(@{y} | \outputp{x}{y})) | P}}}{y} & \nonumber\\
	=
	& \outputp{x}{\quotep{(\prefix{x}{y}{(\outputp{x}{y} | @{y})) | P}}}
	  | {(\prefix{x}{y}{(\outputp{x}{y} | @{y})) | P}} & \nonumber\\
	\red
	& \ldots & \nonumber\\
	\red^*
	& P | P | \ldots & \nonumber
\end{eqnarray}

Of course, this encoding, as an implementation, runs away, unfolding
$\bangp{P}$ eagerly. A lazier and more implementable replication
operator, restricted to input-guarded processes, may be obtained as follows.

\begin{eqnarray}
\bangp{\prefix{u}{v}{P}} 
	:= 
	\binpar{\lift{x}{\prefix{u}{v}{(\binpar{D(x)}{P})}}}{D(x)} \nonumber
\end{eqnarray}

\begin{remark}
  Note that the lazier definition still does not deal with summation
  or mixed summation (i.e. sums over input and output). The reader is
  invited to construct definitions of replication that deal with these
  features. 

  Further, the definitions are parameterized in a name, $x$. Can you,
  gentle reader, make a definition that eliminates this parameter and
  guarantees no accidental interaction between the replication
  machinery and the process being replicated -- i.e. no accidental
  sharing of names used by the process to get its work done and the
  name(s) used by the replication to effect copying. This latter
  revision of the definition of replication is crucial to obtaining
  the expected identity $!!P \sim !P$.
\end{remark}

\begin{remark}\label{rem:paradoxical_combinator}
  The reader familiar with the lambda calculus will have noticed the
  similarity between $D$ and the paradoxical combinator.

  [Ed. note: the existence of this seems to suggest we have to be more
  restrictive on the set of processes and names we admit if we are to
  support no-cloning.]
\end{remark}

\subsubsection{Bisimulation}

The computational dynamics gives rise to another kind of equivalence,
the equivalence of computational behavior. As previously mentioned
this is typically captured \emph{via} some form of bisimulation.

% The notion we use in this paper is weak barbed bisimulation
% \cite{milner91polyadicpi}.

The notion we use in this paper is derived from weak barbed
bisimulation \cite{milner91polyadicpi}. 

\begin{definition}
An \emph{observation relation}, $\downarrow_{\mathcal N}$, over a set
of names, $\mathcal N$, is the smallest relation satisfying the rules
below.

\infrule[Out-barb]{y \in {\mathcal N}, \; x \nameeq y}
		  {\outputp{x}{v} \downarrow_{\mathcal N} x}
\infrule[Par-barb]{\mbox{$P\downarrow_{\mathcal N} x$ or $Q\downarrow_{\mathcal N} x$}}
		  {\binpar{P}{Q} \downarrow_{\mathcal N} x}

We write $P \Downarrow_{\mathcal N} x$ if there is $Q$ such that 
$P \wred Q$ and $Q \downarrow_{\mathcal N} x$.
\end{definition}

\begin{definition}
%\label{def.bbisim}
An  ${\mathcal N}$-\emph{barbed bisimulation} over a set of names, ${\mathcal N}$, is a symmetric binary relation 
${\mathcal S}_{\mathcal N}$ between agents such that $P\rel{S}_{\mathcal N}Q$ implies:
\begin{enumerate}
\item If $P \red P'$ then $Q \wred Q'$ and $P'\rel{S}_{\mathcal N} Q'$.
\item If $P\downarrow_{\mathcal N} x$, then $Q\Downarrow_{\mathcal N} x$.
\end{enumerate}
$P$ is ${\mathcal N}$-barbed bisimilar to $Q$, written
$P \wbbisim_{\mathcal N} Q$, if $P \rel{S}_{\mathcal N} Q$ for some ${\mathcal N}$-barbed bisimulation ${\mathcal S}_{\mathcal N}$.
\end{definition}

$\mathcal{R} \subseteq \pi \times \pi$

$P \mathcal{R} Q => \forall P'. P \red P' \Rightarrow \exists Q'. Q \red Q', P' \mathcal{R} Q'$

$P \vdash x \Rightarrow Q \vdash x$

\begin{mathpar}
  \inferrule*[lab=Out-barb]{x \nameeq y}{{y}!\langle{Q}\rangle \vdash x}
  \and
  \inferrule*[lab=Par-barb]{\mbox{$P\vdash x$ or $Q\vdash x$}}{\binpar{P}{Q} \vdash x}
\end{mathpar}

\subsubsection{Contexts}

One of the principle advantages of computational calculi like the
$\pi$-calculus is a well-defined notion of context,
contextual-equivalence and a correlation between
contextual-equivalence and notions of bisimulation. The notion of
context allows the decomposition of a process into (sub-)process and
its syntactic environment, its context. Thus, a context may be
thought of as a process with a ``hole'' (written $\Box$) in it. The
application of a context $M$ to a process $P$, written $M[P]$, is
tantamount to filling the hole in $M$ with $P$. In this paper we do
not need the full weight of this theory, but do make use of the notion
of context in the proof the main theorem. 

\begin{mathpar}
  \inferrule* [lab=summation] {} {{M_{M},M_{N}} \bc \Box \;|\; x.M_{A} \;|\; M_{M}+M_{N}}
  \and
  \inferrule* [lab=agent] {} {{M_{A}} \bc (\vec{x})M_{P} \;| \; \clift{P_0,\ldots,M_{P},\ldots,P_N}}
  \and \\
  \inferrule* [lab=process] {} {{M_{P}} \bc M_{N} \;| \;P|M_{P} }
\end{mathpar} 

\begin{mathpar}
  \inferrule* [lab=sychronization] {} {M_{N} \bc \Box \;|\; x?M_{F} \;|\; x!M_{C}}
  \and
  \inferrule* [lab=abstraction] {} {{M_{F}} \bc (x)M_{P} }
  \and
  \inferrule* [lab=concretion] {} {{M_{C}} \bc \langle M_{P} \rangle }
  \and \\
  \inferrule* [lab=process] {} {{M_{P}} \bc M_{N} \;| \;P|M_{P} }
\end{mathpar}

\begin{definition}[contextual application] Given a context $M$, and
  process $P$, we define the \emph{contextual application}, $M[P] :=
  M\{P/\Box\}$. That is, the contextual application of M to P is the
  substitution of $P$ for $\Box$ in $M$.
\end{definition}

$\meaningof{-} : L \to \mathcal{P}(\pi)$

\begin{mathpar}
  \inferrule* [lab=collection] {} {\meaningof{true} = \pi, \and \meaningof{~E} = \pi \setminus \meaningof{E}, \and \meaningof{E_{1} \& E_{2}} = \meaningof{E_{1}} \cap \meaningof{E_{2}}}
\end{mathpar}

\begin{mathpar}
  \inferrule* [lab=structure] {} {\meaningof{0} = \{ P \in \pi | P \equiv 0 \}, \and \\ \meaningof{E_1 | E_2} = \{ P \in \pi | P \equiv P_{1} | P_{2}, P_{1} \in \meaningof{E_{1}}, P_{2} \in \meaningof{E_2}\} }
\end{mathpar}

\begin{mathpar}
 \inferrule* [lab=behavior] {} {\meaningof{\langle a?b \rangle E} = \{ P \in \pi | P \equiv Q | u?(y)P', \\ \and \\\\ \and \\ \;\;\; u \in \meaningof{a}, \forall z.P'\{z/y\} \in \meaningof{E\{z/b\}}\}, \and \\ \meaningof{a!E} = \{ P \in \pi | P \equiv Q | x!\langle P' \rangle, x \in \meaningof{a} P' \in \meaningof{E}\} }
\end{mathpar}

\begin{mathpar}
 \inferrule* [lab=nominal] {} {\meaningof{\quotep{E}} = \{ \quotep{P} \in \quotep{\pi} | P \in \meaningof{E} \}, \and \meaningof{\quotep{P}} = \{ \quotep{Q} \in \quotep{\pi} | P \equiv Q \} \and \\ \meaningof{@\quotep{E}} = \{ P \in \pi | P \equiv @x, x \in \meaningof{E} \}}
\end{mathpar}

\begin{eqnarray*}
  \\
  \meaningof{-} : TS \to ST
\end{eqnarray*}

\begin{eqnarray*}
  \\
  L : TS \to ST
\end{eqnarray*}

\begin{eqnarray*}
  \\
  P \models E \iff P \in \meaningof{E}
\end{eqnarray*}

\begin{eqnarray*}
  P \approx_{L} Q \iff \forall E \in L. P \models E \iff Q \models E
\end{eqnarray*}

\begin{eqnarray*}
  P \approx_{K} Q
\end{eqnarray*}

\begin{eqnarray*}
  P \approx Q
\end{eqnarray*}

$\approx_{K} = \approx = \approx_{L}$

\subsubsection{Contextual duality}

Note that contexts extend the quotation operation to a family of
operations from processes to names. Given a context, $M$, we can
define a \emph{nominal context}, $\quotep{M}$ by $\quotep{M}[P] :=
\quotep{M[P]}$. To foreshadow what is to come we observe that these
operations enjoy a duality with processes very much like the duality
between vectors and maps from vectors to scalars.

Further, because the calculus is essentially higher-order, we have a
correspondence between contexts and processes. More specifically,
given a name $x$ and a context $M$ we can construct $M^{*}_{x}$ such
that 

\begin{mathpar}
  M^{*}_{x} | \lift{x}{P} \red M[P]
\end{mathpar}

namely,

\begin{mathpar}
  M^{*}_{x} := x?(u).M[\dropn{u}]
\end{mathpar}

The dependence of $M^{*}_{x}$ on a name makes it an abstraction, 

\begin{mathpar}
  M^{*} := (x)x?(u).M[\dropn{u}]
\end{mathpar}

\subsection{Additional notation}

It will sometimes be convenient to denote the process a name
quotes. We already have the notation $x = \quotep{P}$, but it will be
convenient to introduce an alternate notation, $\procn{x}$, when we
want to emphasize the connection to the use of the name. Note that, by
virtue of name equivalence, $\quotep{\procn{x}} \nameeq x$; so, the
notation is consistent with previous definitions.

Further, because names have structure it is possible to effect
substitutions on the basis of that structure. This means we need to
upgrade our notation for substitutions, which we accomplish by
adapting comprehension notation. Thus,

\begin{mathpar}
  P\{ y / x : x \in S \}
\end{mathpar}

is interpreted to mean the process derived from P by replacing (in a
capture-avoiding manner) each occurrence of $x$ in $S$ by $y$. For example,

\begin{mathpar}
  P\{ \quotep{\procn{x}|\procn{x}} / x : x \in \freenames{P} \}
\end{mathpar}

will replace each (occurrence) of a free name $x$ in $P$ by
$\quotep{\procn{x}|\procn{x}}$.

Also, we will avail ourselves of the notation $x^{L}$ and $x^{R}$ to
denote injections of a name into disjoint copies of the name
space. There are numerous ways to accomplish this. One example can be
found in \cite{MeredithR05}. This notation overloads to vectors of
names: $\vec{x}^{\pi} := (x_{i}^{\pi} \; : \; 0 \leq i < |\vec{x}| )$ where $\pi \in \{L,R\}$.

We also use $P^{\Box} := P|\Box$.

In \cite{MeredithR05} an interpretation of the new operator is
given. It turns out that there are several possible interpretations
all enjoying the requisite algebraic properties of the operator (see
\cite{milner91polyadicpi}). We will therefore make liberal use of
$(\nu\; \vec{x})P$.

% subsection the_syntax_and_semantics_of_the_notation_system (end)   

\input{qm2pi.qmops} 

\input{qm2pi.sterngerlach} 

\input{qm2pi.metric} 

% section concurrent_process_calculi (end)

%\input{qm2pi.proofsketch}

% section proof sketch (end)

%\input{qm2pi.slviaknots} 

% section spatial logic via knots (end)

\input{qm2pi.conclusion}

% section conclusion (end)

%\input{qm2pi.dtcodes} 

% section wiring algorithm (end)

\input{qm2pi.ack} 

% section acknowledgments (end)

\newpage


\bibliographystyle{plain}   
\bibliography{../../biblios/main.bib}

\input{qm2pi.rhodetails}

\end{document}



% section front matter (end)

\section{Introduction}\label{sec:introduction} % (fold)
In this draft of the material i am going to have to dispense with the
usual writing conventions adopted in papers on these topics. i'm going
to have adopt whatever tone i need at the time i'm writing up the
calculations. Sometimes this may be very conversational; others it may
be the barest mathematical grunts; others still it may be that i have
lifted text from one of my other papers because the exposition of some
point was better said there. i hope that my readers are not unduly put
out by this decision. i'm not doing this to flout convention or be
rebellious. i find these calculations very technically challenging. To
keep everything going technically, something has to give; i have to
let go of some cognitive burden. So, the academic writing style --
with all of its trade-offs in terms of facilitating technical
communication -- is what i'm letting go of. Perhaps subsequent drafts
can be tightened and polished, but for now, i'm going to speak as if
we were sitting together in a coffee shop with a laptop, wifi and a
pad of paper and a pencil.

So, here's what i have to say. We -- you and i, comfortably ensconced
in our coffee shop and well-equipped with our tools -- can realize and
carry out the calculations of quantum mechanics over a very different
formal theory of dynamics, a formal theory of dynamics that
corresponds to a theory of concurrent computation with
\emph{reflection}. It has the advantage that the underlying theory is
already `quantized', but supports analogues all of the continuuous
operations. Strikingly, this underlying theory has recently been
connected with a notion of metric that we can show, by calculating
together, coincides with the metric induced by the inner product.

There are a lot of reasons why you might be interested in seeing
calculations of this form. Here's why i'm interested. For the past
several centuries there has been no competitor to the ``Newtonian''
account of dynamics. As a result the predominant share of accounts of
dynamical systems and situations have had to be formulated in terms of
the Newtonian machinery. i view this as an intellectually dangerous
position to occupy. Everything, despite it's intrinsic shape, turns
into a nail to be hit with this hammer. Recently, however, the theory
of computation has matured to the point where we have candidates for
theories of dynamics that offer very different perspective on
reasoning about dynamical systems and situations. Testing these
candidates against very successful accounts of dynamical situations,
like quantum mechanics, is going to give us some sense of how mature
they are and some measure of the quality of these accounts of
dynamics.

\subsection{Summary of contributions and outline of paper}

So, we're going to develop an interpretation of the operations of
quantum mechanics normally interpreted by Hilbert spaces and
operators. We're going to do this over a theory of computation. Note
that this is very different than the usual quantum computation program
which develops notions of computation over quantum mechanics. Rather,
we are developing a story that aligns with Wheeler's slogan: It from
Bit. To do this we will first provide an account of the theory of
computation at play here. Then we will dive into a calculation-driven
interpretation of the operations of quantum mechanics.

The reason we take this approach is that -- until very recently --
there hasn't been an axiomatic account of quantum mechanics. As a
result there has been no sharp delineation of the mathematical theory
supporting interpretation of the physical theory and the physical
theory, itself. So, ambient features of the maths are free to be
exploited (or supressed) without a real accounting of their physical
relevance. There is no sharp statement ``here's the physical theory''
qua \emph{theory} and ``here's the mathematical interpretation''
enabling a judgment of how faithful the interpretation is -- apart
from experimental observation. When there is an axiomatic account we
can judge how well a given mathematical formalism supports an
interpretation of the axioms, independent of
experimentation. Likewise, we can judge how well we have captured our
physical evidence and experience with our axiomatics, independent of
any specific mathematical implementation, with accidental detail that
may or may not have physical significance. 

In lieu of a fully fleshed out and vetted axiomatic account of quantum
mechanics, interpreting the operational notions in service of modeling
physical systems will have to suffice. In other words, we are not in
the business of providing a model of Hilbert spaces and operators. We
are in the business of providing a model of quantum mechanics because
we are motivated by testing our notions of dynamics against physical
theory; and, the predictive calculations of the physical theory must
serve as the best formulation -- shy of a fully fleshed out axiomatic
account -- of the physical theory itself (as they have for scientific
theories since time immemorial). Put another way, despite a
whole-hearted commitment to an It-from-Bit ontology, we are firmly
aligned with the shut-up-and-calculate camp as the best way to obtain
results either from the physical perspective or as a quality assurance
measure of our fledgling theory of dynamics.

In detail, we present a reflective process calculus. Then we develop
intuitive correspondences between the notions available in this
calculus and the usual physical notions supporting quantum mechanical
calculations. Thus, 

\begin{table}[htp]
  \center{
    \fbox{
      \begin{tabular}{c|c}
        quantum mechanics & process calculus \\
        \hline
        scalar & name \\
        state vector & process \\
        dual & contextual duals \\
        matrix & formal sums of process-context-dual pairs \\
        orthogonality & process annihilation \\
        inner product & execution-formula + quoting
      \end{tabular}
    }
  }
  \caption{QM - process calculi correspondences}
\end{table}

Then we tighten up these intuitions to operational definitions. We
employ the Dirac notation as the best proxy we can find for an
abstract syntax of the quantum mechanical notions. The definitions we
develop put us in contact with equational constraints coming from the
theory that we demonstrate the definitions and calculations satisfy.

This puts us in a position to shut up and calculate for the
Stern-Gerlach experimental set up, showing how these predictive
calculations become calculations on processes in our theory of a
reflective process calculus.

Penultimately, we demonstrate that the notion of metric coming from
the inner product coincides with the notion of metric available from
the theory of bisimulation. This demonstration gives us the right to
think of space as arising from behavior. Finally, we consider where we
might go from the new vantage point we have obtained.

% section introduction (end) 
 
% section introduction (end)

% \documentclass[12pt]{llncs}
%\documentclass{jktr}

\usepackage[pdftex]{hyperref}                   
\usepackage {listings}
\usepackage {mathpartir}
\usepackage{bcprules}
%\usepackage{listings}
                       
\usepackage{graphicx} 
%\usepackage[margins=2.5cm,nohead,nofoot]{geometry}
%\usepackage{geometry}
\usepackage{amsfonts}
\usepackage{amstext}
\usepackage{latexsym}
\usepackage{amssymb}
\usepackage{color}


%\include{myPreamble}
\include{qm2pi.local} 

%\ifpdf
%\usepackage[pdftex]{graphicx}
%\else
%\usepackage{graphicx}
%\fi

 % \ifpdf
%  \usepackage{pdfsync}
%  \if


%\title{Brief Article}
%\author{David F. Snyder}
%\author{L.G. Meredith}

%\address{Dept. of Math., Texas State University--San Marcos, San Marcos, TX 78666}
       
\pagestyle{empty}


\begin{document}

\lstset{language=[Objective]Caml,frame=shadowbox}

\input{qm2pi.front}

% section front matter (end)

\input{qm2pi.intro} 
 
% section introduction (end)

% \input{qm2pi.knotations} 

% section notation (end)

\input{qm2pi.process.calculi} 

% section concurrent_process_calculi_and_spatial_logics_ (end)
    
%\input{qm2pi.knots2pi} 

%\input{qm2pi.trefoil} 

%\input{qm2pi.mainthm} 

% subsection basic_interpretation (end)

%\input{qm2pi.rho.presentation} 
\subsection{The syntax and semantics of the notation system}\label{sub:the_syntax_and_semantics_of_the_notation_system} % (fold)

We now summarize a technical presentation of the calculus that
embodies our theory of dynamics. The typical presentation of such a
calculus follows the style of giving generators and relations on
them. The grammar, below, describing term constructors, freely
generates the set of processes, $\Proc$. This set is then quotiented
by a relation known as structural congruence and it is over this set
that the notion of dynamics is expressed. This presentation is
essentially that of \cite{MeredithR05} with the addition of
polyadicity and summation. For readability we have relegated some of
the technical subtleties to an appendix.

\subsubsection{Process grammar}\label{subsub:process_grammar}

\begin{mathpar}
  \inferrule* [lab=synchronization] {} {{M} \bc \pzero \;|\; x?F \;|\; x!C }
  \and
  \inferrule* [lab=abstraction] {} {{F} \bc (x)P}
  \and
  \inferrule* [lab=concretion] {} {{C} \bc \langle Q \rangle}
  \and
  \inferrule* [lab=process] {} {{P,Q} \bc M \;| \;P|Q \;|\; @{x}}
  \and
  \inferrule* [lab=name] {} {{x} \bc \quotep{P}}
\end{mathpar} 

Note that $\vec{x}$ (resp. $\vec{P}$) denotes a vector of names
(resp. processes) of length $|\vec{x}|$ (resp. $|\vec{P}|$). We adopt
the following useful abbreviations.

\begin{mathpar}
   x?(\vec{y}).P := x.(\vec{y})P \and  x\clift{\vec{P}} := x.\clift{\vec{P}}
   \and x!(y) := \lift{x}{\dropn{y}}
   \and \Pi_{i=0}^{n-1}P_i := P_0 | \ldots | P_{n-1}
\end{mathpar}

\subsubsection{Structural congruence}

\paragraph{Free and bound names and alpha-equivalence.} At the
core of structural equivalence is alpha-equivalence which identifies
process that are the same up to a change of variable. Formally, we
recognize the distinction between free and bound names. The free names
of a process, $\freenames{P}$, may be calculated recursively as
follows:

\begin{mathpar}
\freenames{\pzero} := \emptyset
  \and \\
  \freenames{x?(y).P} := \{ x \} \cup (\freenames{P} \setminus \{ y \})
  \and 
  \freenames{x!\langle P \rangle} := \{ x \} \cup \{ P \} 
  \and \\
  \freenames{P|Q} := \freenames{P} \cup \freenames{Q}
  \and \\
  \freenames{@{x}} := \{ x \}
\end{mathpar}

$\pi$
$\quotep{\pi}$

$\freenames{-} : \pi \to \mathcal{P}(\quotep{\pi})$

\begin{eqnarray*}
  \freenames{\pzero} & := & \emptyset \\
  \freenames{x?(y).P} & := & \{ x \} \cup (\freenames{P} \setminus \{ y \}) \\
  \freenames{x!\langle P \rangle} & := & \{ x \} \cup \{ P \} \\
  \freenames{P|Q} & := & \freenames{P} \cup \freenames{Q} \\
  \freenames{\dropn{x}} & := & \{ x \}
\end{eqnarray*}

The bound names of a process, $\boundnames{P}$, are those names occurring in $P$
that are not free. For example, in $x?(y).0$, the name $x$ is free, while $y$ is bound.

\begin{mathpar}
  \inferrule* [lab=monoidal-laws] {} { P|Q \equiv Q|P \and P|0 \equiv P \and P|(Q|R) \equiv (P|Q)|R }
\end{mathpar}

\begin{mathpar}
  \inferrule* [lab=alpha-equivalence] {} { (x)P \equiv (y)P\{y/x\} \and y \not\in \freenames{P} }
\end{mathpar}

\begin{definition}
Then two processes, $P,Q$, are alpha-equivalent if $P = Q\{\vec{y}/\vec{x}\}$ for
some $\vec{x} \in \boundnames{Q},\vec{y} \in \boundnames{P}$, where $Q\{\vec{y}/\vec{x}\}$
denotes the capture-avoiding substitution of $\vec{y}$ for $\vec{x}$ in $Q$.
\end{definition}

\begin{definition}
  The {\em structural congruence} \cite{SangiorgiWalker} , $\equiv$,
  between processes is the least congruence containing
  alpha-equivalence, satisfying the abelian monoid laws
  (associativity, commutativity and $\pzero$ as identity) for parallel
  composition $|$ and for summation $+$.
\end{definition}

\subsection{Name equivalence}

We take name equivalence, written $\nameeq$, to be the smallest
equivalence relation generated by the following rules.

\begin{mathpar}
\inferrule*[lab=Quote-drop]
{ }
{ \quotep{@{x}} \nameeq x }

\inferrule*[lab=Struct-equiv]
{ P \scong Q }
{ \quotep{P} \nameeq \quotep{Q} }
\end{mathpar}

The astute reader will have noticed that the mutual recursion of names
and processes imposes a mutual recursion on alpha-equivalence and
structural equivalence via name-equivalence. Fortunately, all of this
works out pleasantly and we may calculate in the natural way, free of
concern. The reader interested in the details is referred to the
appendix \ref{appendix:rho_details}.

\subsection{Substitution}

We use $\Proc$ for the set of processes, $\QProc$ for the set of
names, and $\id{\{}\vec{y} / \vec{x} \id{\}}$ to denote partial maps,
$s : \QProc \rightarrow \QProc$. A map, $s$ lifts, uniquely, to a map
on process terms, $\widehat{s} : \Proc \rightarrow \Proc$ by the
following equations.

\begin{mathpar}
  (0) \psubstp{Q}{P} := 0 \\
  (R \juxtap S) \psubstp{Q}{P}
  :=    
  (R)\psubstp{Q}{P} \juxtap (S) \psubstp{Q}{P} \\
  (x?(y).R) \psubstp{Q}{P}    
  :=    
  (x)\substp{Q}{P} (z)\concat( (R \psubstn{z}{y}) \psubstp{Q}{P} ) \\
  (\lift{x}{R}) \psubstp{Q}{P}  
  :=
  \lift{(x)\substp{Q}{P}}{ R \psubstp{Q}{P} } \\
%   (\dropn{x})  \psubstp{Q}{P}       
%   := 
%   \left\{ 
%     \begin{array}{ccc} 
%       \dropn{\quotep{Q}} & & x \nameeq \quotep{P} \\
%       \dropn{x} & & otherwise \\
%     \end{array}
%   \right. 
  (\dropn{x})  \psubstp{Q}{P}       
  := 
  \left\{ 
    \begin{array}{ccc} 
      Q & & x \nameeq \quotep{P} \\
      \dropn{x} & & otherwise \\
    \end{array}
  \right.
\end{mathpar}
 

where

\begin{eqnarray}
  (x)\id{\{} \lpquote Q \rpquote / \lpquote P \rpquote \id{\}}            = 
  \left\{ 
    \begin{array}{ccc}
      \lpquote Q \rpquote & & x \nameeq \lpquote P \rpquote \\
      x & & otherwise \\
    \end{array}
  \right. \nonumber
\end{eqnarray}

and $z$ is chosen distinct from $\quotep{P}$, $\quotep{Q}$, the free
names in $Q$, and all the names in $R$. Our $\alpha$-equivalence will
be built in the standard way from this substitution.

\begin{remark}\label{rem:no_self_referential_names}
  One consequence of these definitions is that $\forall P. \quotep{P}
  \not\in \freenames{P}$.
\end{remark}

\subsection{ Dynamic quote: an example }

Anticipating something of what's to come, consider applying the
substitution, $\widehat{\id{\{}u / z \id{\}}}$, to the following pair
of processes, $\lift{w}{y!(z)}$ and $w[ \lpquote y!(z) \rpquote ]$.

\begin{eqnarray}
	\lift{w}{y!(z)}\widehat{\id{\{}u / z \id{\}}}
		& = &
		\lift{w}{y!(u)} \nonumber\\
	w[ \lpquote y!(z) \rpquote ] \widehat{ \id{\{}u / z \id{\}} }
		& = &
		w[ \lpquote y!(z) \rpquote ] \nonumber
\end{eqnarray}

Because the body of the process between quotes is impervious to
substitution, we get radically different answers. In fact, by
examining the first process in an input context,
e.g. $x?(z).\lift{w}{y!(z)}$, we see that the process under the lift
operator may be shaped by prefixed inputs binding a name inside it. In
this sense, the lift operator will be seen as a way to dynamically
construct processes before reifying them as names.

Finally equipped with these standard features we can present the
dynamics of the calculus.

\subsubsection{Operational semantics} 

Finally, we introduce the computational dynamics. What marks these
algebras as distinct from other more traditionally studied algebraic
structures, e.g. vector spaces or polynomial rings, is the manner in
which dynamics is captured. In traditional structures, dynamics is typically
expressed through morphisms between such structures, as in linear maps
between vector spaces or morphisms between rings. In algebras
associated with the semantics of computation, the dynamics is
expressed as part of the algebraic structure itself, through a
reduction reduction relation typically denoted by $\red$. Below, we
give a recursive presentation of this relation for the calculus used
in the encoding.

$\red \subseteq \pi \times \pi$
$\red : \pi \to \mathcal{P}(\pi)$

\begin{mathpar}
  \inferrule* [lab=Comm] { \textsf{match}( x_{src}, x_{trgt} ) } { x_{trgt}?(y)P \; | \; x_{src}!\langle {Q} \rangle \red P\{\quotep{Q}/y}\} }
  \and \\
  \inferrule* [lab=Par] {{P} \red {P}'} {{{P} | {Q}} \red {{P}' | {Q}}}
  \and
  \inferrule* [lab=Equiv]{{{P} \scong {P}'} \andalso {{P}' \red {Q}'} \andalso {{Q}' \scong {Q}}}{{P} \red {Q}}
\end{mathpar}

\begin{eqnarray*}
  match_{\equiv} (\quotep{P},\quotep{Q}) & := & P \equiv Q \\
  match_{\dagger}(\quotep{P},\quotep{Q}) & := & \forall R. P|Q \red^{*} R => R \red^{*} 0 \\
  match_{K}(\quotep{P},\quotep{Q}) & := & K \mbox{ for some context } K
\end{eqnarray*}

$u?(x)P | u!\langle Q \rangle \red P\{\quotep{Q}/x\}$

%We write $\wred$ for $\red^*$, and $P\red$ if $\exists Q $ such that $ P \red Q$.
We write $P\red$ if $\exists Q $ such that $ P \red Q$ and $P\not\red$, otherwise.

\section{Replication}

As mentioned before, it is known that replication (and hence
recursion) can be implemented in a higher-order process algebra
\cite{SangiorgiWalker}. As our first example of calculation with the
machinery thus far presented we give the construction explicitly in
the {\rhoc}.

\begin{eqnarray}
	D_{x} & := & \prefix{x}{y}{(\binpar{\outputp{x}{y}}{@{y}})} \nonumber\\
	\bangp_{x}{P} & := & \binpar{{x}!\langle{\binpar{D_{x}}{P}}\rangle}{D_{x}} \nonumber
\end{eqnarray}

\begin{eqnarray}
	\bangp_{x}{P} & & \nonumber\\
	=
	& {x}!\langle{(\prefix{x}{y}{(\outputp{x}{y} | @{y})) | P}}\rangle 
	      | \prefix{x}{y}{(\outputp{x}{y} | @{y})} & \nonumber\\
	\red
	& (\outputp{x}{y} | @{y})\substn{\quotep{(\prefix{x}{y}{(@{y} | \outputp{x}{y})) | P}}}{y} & \nonumber\\
	=
	& \outputp{x}{\quotep{(\prefix{x}{y}{(\outputp{x}{y} | @{y})) | P}}}
	  | {(\prefix{x}{y}{(\outputp{x}{y} | @{y})) | P}} & \nonumber\\
	\red
	& \ldots & \nonumber\\
	\red^*
	& P | P | \ldots & \nonumber
\end{eqnarray}

Of course, this encoding, as an implementation, runs away, unfolding
$\bangp{P}$ eagerly. A lazier and more implementable replication
operator, restricted to input-guarded processes, may be obtained as follows.

\begin{eqnarray}
\bangp{\prefix{u}{v}{P}} 
	:= 
	\binpar{\lift{x}{\prefix{u}{v}{(\binpar{D(x)}{P})}}}{D(x)} \nonumber
\end{eqnarray}

\begin{remark}
  Note that the lazier definition still does not deal with summation
  or mixed summation (i.e. sums over input and output). The reader is
  invited to construct definitions of replication that deal with these
  features. 

  Further, the definitions are parameterized in a name, $x$. Can you,
  gentle reader, make a definition that eliminates this parameter and
  guarantees no accidental interaction between the replication
  machinery and the process being replicated -- i.e. no accidental
  sharing of names used by the process to get its work done and the
  name(s) used by the replication to effect copying. This latter
  revision of the definition of replication is crucial to obtaining
  the expected identity $!!P \sim !P$.
\end{remark}

\begin{remark}\label{rem:paradoxical_combinator}
  The reader familiar with the lambda calculus will have noticed the
  similarity between $D$ and the paradoxical combinator.

  [Ed. note: the existence of this seems to suggest we have to be more
  restrictive on the set of processes and names we admit if we are to
  support no-cloning.]
\end{remark}

\subsubsection{Bisimulation}

The computational dynamics gives rise to another kind of equivalence,
the equivalence of computational behavior. As previously mentioned
this is typically captured \emph{via} some form of bisimulation.

% The notion we use in this paper is weak barbed bisimulation
% \cite{milner91polyadicpi}.

The notion we use in this paper is derived from weak barbed
bisimulation \cite{milner91polyadicpi}. 

\begin{definition}
An \emph{observation relation}, $\downarrow_{\mathcal N}$, over a set
of names, $\mathcal N$, is the smallest relation satisfying the rules
below.

\infrule[Out-barb]{y \in {\mathcal N}, \; x \nameeq y}
		  {\outputp{x}{v} \downarrow_{\mathcal N} x}
\infrule[Par-barb]{\mbox{$P\downarrow_{\mathcal N} x$ or $Q\downarrow_{\mathcal N} x$}}
		  {\binpar{P}{Q} \downarrow_{\mathcal N} x}

We write $P \Downarrow_{\mathcal N} x$ if there is $Q$ such that 
$P \wred Q$ and $Q \downarrow_{\mathcal N} x$.
\end{definition}

\begin{definition}
%\label{def.bbisim}
An  ${\mathcal N}$-\emph{barbed bisimulation} over a set of names, ${\mathcal N}$, is a symmetric binary relation 
${\mathcal S}_{\mathcal N}$ between agents such that $P\rel{S}_{\mathcal N}Q$ implies:
\begin{enumerate}
\item If $P \red P'$ then $Q \wred Q'$ and $P'\rel{S}_{\mathcal N} Q'$.
\item If $P\downarrow_{\mathcal N} x$, then $Q\Downarrow_{\mathcal N} x$.
\end{enumerate}
$P$ is ${\mathcal N}$-barbed bisimilar to $Q$, written
$P \wbbisim_{\mathcal N} Q$, if $P \rel{S}_{\mathcal N} Q$ for some ${\mathcal N}$-barbed bisimulation ${\mathcal S}_{\mathcal N}$.
\end{definition}

$\mathcal{R} \subseteq \pi \times \pi$

$P \mathcal{R} Q => \forall P'. P \red P' \Rightarrow \exists Q'. Q \red Q', P' \mathcal{R} Q'$

$P \vdash x \Rightarrow Q \vdash x$

\begin{mathpar}
  \inferrule*[lab=Out-barb]{x \nameeq y}{{y}!\langle{Q}\rangle \vdash x}
  \and
  \inferrule*[lab=Par-barb]{\mbox{$P\vdash x$ or $Q\vdash x$}}{\binpar{P}{Q} \vdash x}
\end{mathpar}

\subsubsection{Contexts}

One of the principle advantages of computational calculi like the
$\pi$-calculus is a well-defined notion of context,
contextual-equivalence and a correlation between
contextual-equivalence and notions of bisimulation. The notion of
context allows the decomposition of a process into (sub-)process and
its syntactic environment, its context. Thus, a context may be
thought of as a process with a ``hole'' (written $\Box$) in it. The
application of a context $M$ to a process $P$, written $M[P]$, is
tantamount to filling the hole in $M$ with $P$. In this paper we do
not need the full weight of this theory, but do make use of the notion
of context in the proof the main theorem. 

\begin{mathpar}
  \inferrule* [lab=summation] {} {{M_{M},M_{N}} \bc \Box \;|\; x.M_{A} \;|\; M_{M}+M_{N}}
  \and
  \inferrule* [lab=agent] {} {{M_{A}} \bc (\vec{x})M_{P} \;| \; \clift{P_0,\ldots,M_{P},\ldots,P_N}}
  \and \\
  \inferrule* [lab=process] {} {{M_{P}} \bc M_{N} \;| \;P|M_{P} }
\end{mathpar} 

\begin{mathpar}
  \inferrule* [lab=sychronization] {} {M_{N} \bc \Box \;|\; x?M_{F} \;|\; x!M_{C}}
  \and
  \inferrule* [lab=abstraction] {} {{M_{F}} \bc (x)M_{P} }
  \and
  \inferrule* [lab=concretion] {} {{M_{C}} \bc \langle M_{P} \rangle }
  \and \\
  \inferrule* [lab=process] {} {{M_{P}} \bc M_{N} \;| \;P|M_{P} }
\end{mathpar}

\begin{definition}[contextual application] Given a context $M$, and
  process $P$, we define the \emph{contextual application}, $M[P] :=
  M\{P/\Box\}$. That is, the contextual application of M to P is the
  substitution of $P$ for $\Box$ in $M$.
\end{definition}

$\meaningof{-} : L \to \mathcal{P}(\pi)$

\begin{mathpar}
  \inferrule* [lab=collection] {} {\meaningof{true} = \pi, \and \meaningof{~E} = \pi \setminus \meaningof{E}, \and \meaningof{E_{1} \& E_{2}} = \meaningof{E_{1}} \cap \meaningof{E_{2}}}
\end{mathpar}

\begin{mathpar}
  \inferrule* [lab=structure] {} {\meaningof{0} = \{ P \in \pi | P \equiv 0 \}, \and \\ \meaningof{E_1 | E_2} = \{ P \in \pi | P \equiv P_{1} | P_{2}, P_{1} \in \meaningof{E_{1}}, P_{2} \in \meaningof{E_2}\} }
\end{mathpar}

\begin{mathpar}
 \inferrule* [lab=behavior] {} {\meaningof{\langle a?b \rangle E} = \{ P \in \pi | P \equiv Q | u?(y)P', \\ \and \\\\ \and \\ \;\;\; u \in \meaningof{a}, \forall z.P'\{z/y\} \in \meaningof{E\{z/b\}}\}, \and \\ \meaningof{a!E} = \{ P \in \pi | P \equiv Q | x!\langle P' \rangle, x \in \meaningof{a} P' \in \meaningof{E}\} }
\end{mathpar}

\begin{mathpar}
 \inferrule* [lab=nominal] {} {\meaningof{\quotep{E}} = \{ \quotep{P} \in \quotep{\pi} | P \in \meaningof{E} \}, \and \meaningof{\quotep{P}} = \{ \quotep{Q} \in \quotep{\pi} | P \equiv Q \} \and \\ \meaningof{@\quotep{E}} = \{ P \in \pi | P \equiv @x, x \in \meaningof{E} \}}
\end{mathpar}

\begin{eqnarray*}
  \\
  \meaningof{-} : TS \to ST
\end{eqnarray*}

\begin{eqnarray*}
  \\
  L : TS \to ST
\end{eqnarray*}

\begin{eqnarray*}
  \\
  P \models E \iff P \in \meaningof{E}
\end{eqnarray*}

\begin{eqnarray*}
  P \approx_{L} Q \iff \forall E \in L. P \models E \iff Q \models E
\end{eqnarray*}

\begin{eqnarray*}
  P \approx_{K} Q
\end{eqnarray*}

\begin{eqnarray*}
  P \approx Q
\end{eqnarray*}

$\approx_{K} = \approx = \approx_{L}$

\subsubsection{Contextual duality}

Note that contexts extend the quotation operation to a family of
operations from processes to names. Given a context, $M$, we can
define a \emph{nominal context}, $\quotep{M}$ by $\quotep{M}[P] :=
\quotep{M[P]}$. To foreshadow what is to come we observe that these
operations enjoy a duality with processes very much like the duality
between vectors and maps from vectors to scalars.

Further, because the calculus is essentially higher-order, we have a
correspondence between contexts and processes. More specifically,
given a name $x$ and a context $M$ we can construct $M^{*}_{x}$ such
that 

\begin{mathpar}
  M^{*}_{x} | \lift{x}{P} \red M[P]
\end{mathpar}

namely,

\begin{mathpar}
  M^{*}_{x} := x?(u).M[\dropn{u}]
\end{mathpar}

The dependence of $M^{*}_{x}$ on a name makes it an abstraction, 

\begin{mathpar}
  M^{*} := (x)x?(u).M[\dropn{u}]
\end{mathpar}

\subsection{Additional notation}

It will sometimes be convenient to denote the process a name
quotes. We already have the notation $x = \quotep{P}$, but it will be
convenient to introduce an alternate notation, $\procn{x}$, when we
want to emphasize the connection to the use of the name. Note that, by
virtue of name equivalence, $\quotep{\procn{x}} \nameeq x$; so, the
notation is consistent with previous definitions.

Further, because names have structure it is possible to effect
substitutions on the basis of that structure. This means we need to
upgrade our notation for substitutions, which we accomplish by
adapting comprehension notation. Thus,

\begin{mathpar}
  P\{ y / x : x \in S \}
\end{mathpar}

is interpreted to mean the process derived from P by replacing (in a
capture-avoiding manner) each occurrence of $x$ in $S$ by $y$. For example,

\begin{mathpar}
  P\{ \quotep{\procn{x}|\procn{x}} / x : x \in \freenames{P} \}
\end{mathpar}

will replace each (occurrence) of a free name $x$ in $P$ by
$\quotep{\procn{x}|\procn{x}}$.

Also, we will avail ourselves of the notation $x^{L}$ and $x^{R}$ to
denote injections of a name into disjoint copies of the name
space. There are numerous ways to accomplish this. One example can be
found in \cite{MeredithR05}. This notation overloads to vectors of
names: $\vec{x}^{\pi} := (x_{i}^{\pi} \; : \; 0 \leq i < |\vec{x}| )$ where $\pi \in \{L,R\}$.

We also use $P^{\Box} := P|\Box$.

In \cite{MeredithR05} an interpretation of the new operator is
given. It turns out that there are several possible interpretations
all enjoying the requisite algebraic properties of the operator (see
\cite{milner91polyadicpi}). We will therefore make liberal use of
$(\nu\; \vec{x})P$.

% subsection the_syntax_and_semantics_of_the_notation_system (end)   

\input{qm2pi.qmops} 

\input{qm2pi.sterngerlach} 

\input{qm2pi.metric} 

% section concurrent_process_calculi (end)

%\input{qm2pi.proofsketch}

% section proof sketch (end)

%\input{qm2pi.slviaknots} 

% section spatial logic via knots (end)

\input{qm2pi.conclusion}

% section conclusion (end)

%\input{qm2pi.dtcodes} 

% section wiring algorithm (end)

\input{qm2pi.ack} 

% section acknowledgments (end)

\newpage


\bibliographystyle{plain}   
\bibliography{../../biblios/main.bib}

\input{qm2pi.rhodetails}

\end{document}

 

% section notation (end)

\input{qm2pi.process.calculi} 

% section concurrent_process_calculi_and_spatial_logics_ (end)
    
%\documentclass[12pt]{llncs}
%\documentclass{jktr}

\usepackage[pdftex]{hyperref}                   
\usepackage {listings}
\usepackage {mathpartir}
\usepackage{bcprules}
%\usepackage{listings}
                       
\usepackage{graphicx} 
%\usepackage[margins=2.5cm,nohead,nofoot]{geometry}
%\usepackage{geometry}
\usepackage{amsfonts}
\usepackage{amstext}
\usepackage{latexsym}
\usepackage{amssymb}
\usepackage{color}


%\include{myPreamble}
\include{qm2pi.local} 

%\ifpdf
%\usepackage[pdftex]{graphicx}
%\else
%\usepackage{graphicx}
%\fi

 % \ifpdf
%  \usepackage{pdfsync}
%  \if


%\title{Brief Article}
%\author{David F. Snyder}
%\author{L.G. Meredith}

%\address{Dept. of Math., Texas State University--San Marcos, San Marcos, TX 78666}
       
\pagestyle{empty}


\begin{document}

\lstset{language=[Objective]Caml,frame=shadowbox}

\input{qm2pi.front}

% section front matter (end)

\input{qm2pi.intro} 
 
% section introduction (end)

% \input{qm2pi.knotations} 

% section notation (end)

\input{qm2pi.process.calculi} 

% section concurrent_process_calculi_and_spatial_logics_ (end)
    
%\input{qm2pi.knots2pi} 

%\input{qm2pi.trefoil} 

%\input{qm2pi.mainthm} 

% subsection basic_interpretation (end)

%\input{qm2pi.rho.presentation} 
\subsection{The syntax and semantics of the notation system}\label{sub:the_syntax_and_semantics_of_the_notation_system} % (fold)

We now summarize a technical presentation of the calculus that
embodies our theory of dynamics. The typical presentation of such a
calculus follows the style of giving generators and relations on
them. The grammar, below, describing term constructors, freely
generates the set of processes, $\Proc$. This set is then quotiented
by a relation known as structural congruence and it is over this set
that the notion of dynamics is expressed. This presentation is
essentially that of \cite{MeredithR05} with the addition of
polyadicity and summation. For readability we have relegated some of
the technical subtleties to an appendix.

\subsubsection{Process grammar}\label{subsub:process_grammar}

\begin{mathpar}
  \inferrule* [lab=synchronization] {} {{M} \bc \pzero \;|\; x?F \;|\; x!C }
  \and
  \inferrule* [lab=abstraction] {} {{F} \bc (x)P}
  \and
  \inferrule* [lab=concretion] {} {{C} \bc \langle Q \rangle}
  \and
  \inferrule* [lab=process] {} {{P,Q} \bc M \;| \;P|Q \;|\; @{x}}
  \and
  \inferrule* [lab=name] {} {{x} \bc \quotep{P}}
\end{mathpar} 

Note that $\vec{x}$ (resp. $\vec{P}$) denotes a vector of names
(resp. processes) of length $|\vec{x}|$ (resp. $|\vec{P}|$). We adopt
the following useful abbreviations.

\begin{mathpar}
   x?(\vec{y}).P := x.(\vec{y})P \and  x\clift{\vec{P}} := x.\clift{\vec{P}}
   \and x!(y) := \lift{x}{\dropn{y}}
   \and \Pi_{i=0}^{n-1}P_i := P_0 | \ldots | P_{n-1}
\end{mathpar}

\subsubsection{Structural congruence}

\paragraph{Free and bound names and alpha-equivalence.} At the
core of structural equivalence is alpha-equivalence which identifies
process that are the same up to a change of variable. Formally, we
recognize the distinction between free and bound names. The free names
of a process, $\freenames{P}$, may be calculated recursively as
follows:

\begin{mathpar}
\freenames{\pzero} := \emptyset
  \and \\
  \freenames{x?(y).P} := \{ x \} \cup (\freenames{P} \setminus \{ y \})
  \and 
  \freenames{x!\langle P \rangle} := \{ x \} \cup \{ P \} 
  \and \\
  \freenames{P|Q} := \freenames{P} \cup \freenames{Q}
  \and \\
  \freenames{@{x}} := \{ x \}
\end{mathpar}

$\pi$
$\quotep{\pi}$

$\freenames{-} : \pi \to \mathcal{P}(\quotep{\pi})$

\begin{eqnarray*}
  \freenames{\pzero} & := & \emptyset \\
  \freenames{x?(y).P} & := & \{ x \} \cup (\freenames{P} \setminus \{ y \}) \\
  \freenames{x!\langle P \rangle} & := & \{ x \} \cup \{ P \} \\
  \freenames{P|Q} & := & \freenames{P} \cup \freenames{Q} \\
  \freenames{\dropn{x}} & := & \{ x \}
\end{eqnarray*}

The bound names of a process, $\boundnames{P}$, are those names occurring in $P$
that are not free. For example, in $x?(y).0$, the name $x$ is free, while $y$ is bound.

\begin{mathpar}
  \inferrule* [lab=monoidal-laws] {} { P|Q \equiv Q|P \and P|0 \equiv P \and P|(Q|R) \equiv (P|Q)|R }
\end{mathpar}

\begin{mathpar}
  \inferrule* [lab=alpha-equivalence] {} { (x)P \equiv (y)P\{y/x\} \and y \not\in \freenames{P} }
\end{mathpar}

\begin{definition}
Then two processes, $P,Q$, are alpha-equivalent if $P = Q\{\vec{y}/\vec{x}\}$ for
some $\vec{x} \in \boundnames{Q},\vec{y} \in \boundnames{P}$, where $Q\{\vec{y}/\vec{x}\}$
denotes the capture-avoiding substitution of $\vec{y}$ for $\vec{x}$ in $Q$.
\end{definition}

\begin{definition}
  The {\em structural congruence} \cite{SangiorgiWalker} , $\equiv$,
  between processes is the least congruence containing
  alpha-equivalence, satisfying the abelian monoid laws
  (associativity, commutativity and $\pzero$ as identity) for parallel
  composition $|$ and for summation $+$.
\end{definition}

\subsection{Name equivalence}

We take name equivalence, written $\nameeq$, to be the smallest
equivalence relation generated by the following rules.

\begin{mathpar}
\inferrule*[lab=Quote-drop]
{ }
{ \quotep{@{x}} \nameeq x }

\inferrule*[lab=Struct-equiv]
{ P \scong Q }
{ \quotep{P} \nameeq \quotep{Q} }
\end{mathpar}

The astute reader will have noticed that the mutual recursion of names
and processes imposes a mutual recursion on alpha-equivalence and
structural equivalence via name-equivalence. Fortunately, all of this
works out pleasantly and we may calculate in the natural way, free of
concern. The reader interested in the details is referred to the
appendix \ref{appendix:rho_details}.

\subsection{Substitution}

We use $\Proc$ for the set of processes, $\QProc$ for the set of
names, and $\id{\{}\vec{y} / \vec{x} \id{\}}$ to denote partial maps,
$s : \QProc \rightarrow \QProc$. A map, $s$ lifts, uniquely, to a map
on process terms, $\widehat{s} : \Proc \rightarrow \Proc$ by the
following equations.

\begin{mathpar}
  (0) \psubstp{Q}{P} := 0 \\
  (R \juxtap S) \psubstp{Q}{P}
  :=    
  (R)\psubstp{Q}{P} \juxtap (S) \psubstp{Q}{P} \\
  (x?(y).R) \psubstp{Q}{P}    
  :=    
  (x)\substp{Q}{P} (z)\concat( (R \psubstn{z}{y}) \psubstp{Q}{P} ) \\
  (\lift{x}{R}) \psubstp{Q}{P}  
  :=
  \lift{(x)\substp{Q}{P}}{ R \psubstp{Q}{P} } \\
%   (\dropn{x})  \psubstp{Q}{P}       
%   := 
%   \left\{ 
%     \begin{array}{ccc} 
%       \dropn{\quotep{Q}} & & x \nameeq \quotep{P} \\
%       \dropn{x} & & otherwise \\
%     \end{array}
%   \right. 
  (\dropn{x})  \psubstp{Q}{P}       
  := 
  \left\{ 
    \begin{array}{ccc} 
      Q & & x \nameeq \quotep{P} \\
      \dropn{x} & & otherwise \\
    \end{array}
  \right.
\end{mathpar}
 

where

\begin{eqnarray}
  (x)\id{\{} \lpquote Q \rpquote / \lpquote P \rpquote \id{\}}            = 
  \left\{ 
    \begin{array}{ccc}
      \lpquote Q \rpquote & & x \nameeq \lpquote P \rpquote \\
      x & & otherwise \\
    \end{array}
  \right. \nonumber
\end{eqnarray}

and $z$ is chosen distinct from $\quotep{P}$, $\quotep{Q}$, the free
names in $Q$, and all the names in $R$. Our $\alpha$-equivalence will
be built in the standard way from this substitution.

\begin{remark}\label{rem:no_self_referential_names}
  One consequence of these definitions is that $\forall P. \quotep{P}
  \not\in \freenames{P}$.
\end{remark}

\subsection{ Dynamic quote: an example }

Anticipating something of what's to come, consider applying the
substitution, $\widehat{\id{\{}u / z \id{\}}}$, to the following pair
of processes, $\lift{w}{y!(z)}$ and $w[ \lpquote y!(z) \rpquote ]$.

\begin{eqnarray}
	\lift{w}{y!(z)}\widehat{\id{\{}u / z \id{\}}}
		& = &
		\lift{w}{y!(u)} \nonumber\\
	w[ \lpquote y!(z) \rpquote ] \widehat{ \id{\{}u / z \id{\}} }
		& = &
		w[ \lpquote y!(z) \rpquote ] \nonumber
\end{eqnarray}

Because the body of the process between quotes is impervious to
substitution, we get radically different answers. In fact, by
examining the first process in an input context,
e.g. $x?(z).\lift{w}{y!(z)}$, we see that the process under the lift
operator may be shaped by prefixed inputs binding a name inside it. In
this sense, the lift operator will be seen as a way to dynamically
construct processes before reifying them as names.

Finally equipped with these standard features we can present the
dynamics of the calculus.

\subsubsection{Operational semantics} 

Finally, we introduce the computational dynamics. What marks these
algebras as distinct from other more traditionally studied algebraic
structures, e.g. vector spaces or polynomial rings, is the manner in
which dynamics is captured. In traditional structures, dynamics is typically
expressed through morphisms between such structures, as in linear maps
between vector spaces or morphisms between rings. In algebras
associated with the semantics of computation, the dynamics is
expressed as part of the algebraic structure itself, through a
reduction reduction relation typically denoted by $\red$. Below, we
give a recursive presentation of this relation for the calculus used
in the encoding.

$\red \subseteq \pi \times \pi$
$\red : \pi \to \mathcal{P}(\pi)$

\begin{mathpar}
  \inferrule* [lab=Comm] { \textsf{match}( x_{src}, x_{trgt} ) } { x_{trgt}?(y)P \; | \; x_{src}!\langle {Q} \rangle \red P\{\quotep{Q}/y}\} }
  \and \\
  \inferrule* [lab=Par] {{P} \red {P}'} {{{P} | {Q}} \red {{P}' | {Q}}}
  \and
  \inferrule* [lab=Equiv]{{{P} \scong {P}'} \andalso {{P}' \red {Q}'} \andalso {{Q}' \scong {Q}}}{{P} \red {Q}}
\end{mathpar}

\begin{eqnarray*}
  match_{\equiv} (\quotep{P},\quotep{Q}) & := & P \equiv Q \\
  match_{\dagger}(\quotep{P},\quotep{Q}) & := & \forall R. P|Q \red^{*} R => R \red^{*} 0 \\
  match_{K}(\quotep{P},\quotep{Q}) & := & K \mbox{ for some context } K
\end{eqnarray*}

$u?(x)P | u!\langle Q \rangle \red P\{\quotep{Q}/x\}$

%We write $\wred$ for $\red^*$, and $P\red$ if $\exists Q $ such that $ P \red Q$.
We write $P\red$ if $\exists Q $ such that $ P \red Q$ and $P\not\red$, otherwise.

\section{Replication}

As mentioned before, it is known that replication (and hence
recursion) can be implemented in a higher-order process algebra
\cite{SangiorgiWalker}. As our first example of calculation with the
machinery thus far presented we give the construction explicitly in
the {\rhoc}.

\begin{eqnarray}
	D_{x} & := & \prefix{x}{y}{(\binpar{\outputp{x}{y}}{@{y}})} \nonumber\\
	\bangp_{x}{P} & := & \binpar{{x}!\langle{\binpar{D_{x}}{P}}\rangle}{D_{x}} \nonumber
\end{eqnarray}

\begin{eqnarray}
	\bangp_{x}{P} & & \nonumber\\
	=
	& {x}!\langle{(\prefix{x}{y}{(\outputp{x}{y} | @{y})) | P}}\rangle 
	      | \prefix{x}{y}{(\outputp{x}{y} | @{y})} & \nonumber\\
	\red
	& (\outputp{x}{y} | @{y})\substn{\quotep{(\prefix{x}{y}{(@{y} | \outputp{x}{y})) | P}}}{y} & \nonumber\\
	=
	& \outputp{x}{\quotep{(\prefix{x}{y}{(\outputp{x}{y} | @{y})) | P}}}
	  | {(\prefix{x}{y}{(\outputp{x}{y} | @{y})) | P}} & \nonumber\\
	\red
	& \ldots & \nonumber\\
	\red^*
	& P | P | \ldots & \nonumber
\end{eqnarray}

Of course, this encoding, as an implementation, runs away, unfolding
$\bangp{P}$ eagerly. A lazier and more implementable replication
operator, restricted to input-guarded processes, may be obtained as follows.

\begin{eqnarray}
\bangp{\prefix{u}{v}{P}} 
	:= 
	\binpar{\lift{x}{\prefix{u}{v}{(\binpar{D(x)}{P})}}}{D(x)} \nonumber
\end{eqnarray}

\begin{remark}
  Note that the lazier definition still does not deal with summation
  or mixed summation (i.e. sums over input and output). The reader is
  invited to construct definitions of replication that deal with these
  features. 

  Further, the definitions are parameterized in a name, $x$. Can you,
  gentle reader, make a definition that eliminates this parameter and
  guarantees no accidental interaction between the replication
  machinery and the process being replicated -- i.e. no accidental
  sharing of names used by the process to get its work done and the
  name(s) used by the replication to effect copying. This latter
  revision of the definition of replication is crucial to obtaining
  the expected identity $!!P \sim !P$.
\end{remark}

\begin{remark}\label{rem:paradoxical_combinator}
  The reader familiar with the lambda calculus will have noticed the
  similarity between $D$ and the paradoxical combinator.

  [Ed. note: the existence of this seems to suggest we have to be more
  restrictive on the set of processes and names we admit if we are to
  support no-cloning.]
\end{remark}

\subsubsection{Bisimulation}

The computational dynamics gives rise to another kind of equivalence,
the equivalence of computational behavior. As previously mentioned
this is typically captured \emph{via} some form of bisimulation.

% The notion we use in this paper is weak barbed bisimulation
% \cite{milner91polyadicpi}.

The notion we use in this paper is derived from weak barbed
bisimulation \cite{milner91polyadicpi}. 

\begin{definition}
An \emph{observation relation}, $\downarrow_{\mathcal N}$, over a set
of names, $\mathcal N$, is the smallest relation satisfying the rules
below.

\infrule[Out-barb]{y \in {\mathcal N}, \; x \nameeq y}
		  {\outputp{x}{v} \downarrow_{\mathcal N} x}
\infrule[Par-barb]{\mbox{$P\downarrow_{\mathcal N} x$ or $Q\downarrow_{\mathcal N} x$}}
		  {\binpar{P}{Q} \downarrow_{\mathcal N} x}

We write $P \Downarrow_{\mathcal N} x$ if there is $Q$ such that 
$P \wred Q$ and $Q \downarrow_{\mathcal N} x$.
\end{definition}

\begin{definition}
%\label{def.bbisim}
An  ${\mathcal N}$-\emph{barbed bisimulation} over a set of names, ${\mathcal N}$, is a symmetric binary relation 
${\mathcal S}_{\mathcal N}$ between agents such that $P\rel{S}_{\mathcal N}Q$ implies:
\begin{enumerate}
\item If $P \red P'$ then $Q \wred Q'$ and $P'\rel{S}_{\mathcal N} Q'$.
\item If $P\downarrow_{\mathcal N} x$, then $Q\Downarrow_{\mathcal N} x$.
\end{enumerate}
$P$ is ${\mathcal N}$-barbed bisimilar to $Q$, written
$P \wbbisim_{\mathcal N} Q$, if $P \rel{S}_{\mathcal N} Q$ for some ${\mathcal N}$-barbed bisimulation ${\mathcal S}_{\mathcal N}$.
\end{definition}

$\mathcal{R} \subseteq \pi \times \pi$

$P \mathcal{R} Q => \forall P'. P \red P' \Rightarrow \exists Q'. Q \red Q', P' \mathcal{R} Q'$

$P \vdash x \Rightarrow Q \vdash x$

\begin{mathpar}
  \inferrule*[lab=Out-barb]{x \nameeq y}{{y}!\langle{Q}\rangle \vdash x}
  \and
  \inferrule*[lab=Par-barb]{\mbox{$P\vdash x$ or $Q\vdash x$}}{\binpar{P}{Q} \vdash x}
\end{mathpar}

\subsubsection{Contexts}

One of the principle advantages of computational calculi like the
$\pi$-calculus is a well-defined notion of context,
contextual-equivalence and a correlation between
contextual-equivalence and notions of bisimulation. The notion of
context allows the decomposition of a process into (sub-)process and
its syntactic environment, its context. Thus, a context may be
thought of as a process with a ``hole'' (written $\Box$) in it. The
application of a context $M$ to a process $P$, written $M[P]$, is
tantamount to filling the hole in $M$ with $P$. In this paper we do
not need the full weight of this theory, but do make use of the notion
of context in the proof the main theorem. 

\begin{mathpar}
  \inferrule* [lab=summation] {} {{M_{M},M_{N}} \bc \Box \;|\; x.M_{A} \;|\; M_{M}+M_{N}}
  \and
  \inferrule* [lab=agent] {} {{M_{A}} \bc (\vec{x})M_{P} \;| \; \clift{P_0,\ldots,M_{P},\ldots,P_N}}
  \and \\
  \inferrule* [lab=process] {} {{M_{P}} \bc M_{N} \;| \;P|M_{P} }
\end{mathpar} 

\begin{mathpar}
  \inferrule* [lab=sychronization] {} {M_{N} \bc \Box \;|\; x?M_{F} \;|\; x!M_{C}}
  \and
  \inferrule* [lab=abstraction] {} {{M_{F}} \bc (x)M_{P} }
  \and
  \inferrule* [lab=concretion] {} {{M_{C}} \bc \langle M_{P} \rangle }
  \and \\
  \inferrule* [lab=process] {} {{M_{P}} \bc M_{N} \;| \;P|M_{P} }
\end{mathpar}

\begin{definition}[contextual application] Given a context $M$, and
  process $P$, we define the \emph{contextual application}, $M[P] :=
  M\{P/\Box\}$. That is, the contextual application of M to P is the
  substitution of $P$ for $\Box$ in $M$.
\end{definition}

$\meaningof{-} : L \to \mathcal{P}(\pi)$

\begin{mathpar}
  \inferrule* [lab=collection] {} {\meaningof{true} = \pi, \and \meaningof{~E} = \pi \setminus \meaningof{E}, \and \meaningof{E_{1} \& E_{2}} = \meaningof{E_{1}} \cap \meaningof{E_{2}}}
\end{mathpar}

\begin{mathpar}
  \inferrule* [lab=structure] {} {\meaningof{0} = \{ P \in \pi | P \equiv 0 \}, \and \\ \meaningof{E_1 | E_2} = \{ P \in \pi | P \equiv P_{1} | P_{2}, P_{1} \in \meaningof{E_{1}}, P_{2} \in \meaningof{E_2}\} }
\end{mathpar}

\begin{mathpar}
 \inferrule* [lab=behavior] {} {\meaningof{\langle a?b \rangle E} = \{ P \in \pi | P \equiv Q | u?(y)P', \\ \and \\\\ \and \\ \;\;\; u \in \meaningof{a}, \forall z.P'\{z/y\} \in \meaningof{E\{z/b\}}\}, \and \\ \meaningof{a!E} = \{ P \in \pi | P \equiv Q | x!\langle P' \rangle, x \in \meaningof{a} P' \in \meaningof{E}\} }
\end{mathpar}

\begin{mathpar}
 \inferrule* [lab=nominal] {} {\meaningof{\quotep{E}} = \{ \quotep{P} \in \quotep{\pi} | P \in \meaningof{E} \}, \and \meaningof{\quotep{P}} = \{ \quotep{Q} \in \quotep{\pi} | P \equiv Q \} \and \\ \meaningof{@\quotep{E}} = \{ P \in \pi | P \equiv @x, x \in \meaningof{E} \}}
\end{mathpar}

\begin{eqnarray*}
  \\
  \meaningof{-} : TS \to ST
\end{eqnarray*}

\begin{eqnarray*}
  \\
  L : TS \to ST
\end{eqnarray*}

\begin{eqnarray*}
  \\
  P \models E \iff P \in \meaningof{E}
\end{eqnarray*}

\begin{eqnarray*}
  P \approx_{L} Q \iff \forall E \in L. P \models E \iff Q \models E
\end{eqnarray*}

\begin{eqnarray*}
  P \approx_{K} Q
\end{eqnarray*}

\begin{eqnarray*}
  P \approx Q
\end{eqnarray*}

$\approx_{K} = \approx = \approx_{L}$

\subsubsection{Contextual duality}

Note that contexts extend the quotation operation to a family of
operations from processes to names. Given a context, $M$, we can
define a \emph{nominal context}, $\quotep{M}$ by $\quotep{M}[P] :=
\quotep{M[P]}$. To foreshadow what is to come we observe that these
operations enjoy a duality with processes very much like the duality
between vectors and maps from vectors to scalars.

Further, because the calculus is essentially higher-order, we have a
correspondence between contexts and processes. More specifically,
given a name $x$ and a context $M$ we can construct $M^{*}_{x}$ such
that 

\begin{mathpar}
  M^{*}_{x} | \lift{x}{P} \red M[P]
\end{mathpar}

namely,

\begin{mathpar}
  M^{*}_{x} := x?(u).M[\dropn{u}]
\end{mathpar}

The dependence of $M^{*}_{x}$ on a name makes it an abstraction, 

\begin{mathpar}
  M^{*} := (x)x?(u).M[\dropn{u}]
\end{mathpar}

\subsection{Additional notation}

It will sometimes be convenient to denote the process a name
quotes. We already have the notation $x = \quotep{P}$, but it will be
convenient to introduce an alternate notation, $\procn{x}$, when we
want to emphasize the connection to the use of the name. Note that, by
virtue of name equivalence, $\quotep{\procn{x}} \nameeq x$; so, the
notation is consistent with previous definitions.

Further, because names have structure it is possible to effect
substitutions on the basis of that structure. This means we need to
upgrade our notation for substitutions, which we accomplish by
adapting comprehension notation. Thus,

\begin{mathpar}
  P\{ y / x : x \in S \}
\end{mathpar}

is interpreted to mean the process derived from P by replacing (in a
capture-avoiding manner) each occurrence of $x$ in $S$ by $y$. For example,

\begin{mathpar}
  P\{ \quotep{\procn{x}|\procn{x}} / x : x \in \freenames{P} \}
\end{mathpar}

will replace each (occurrence) of a free name $x$ in $P$ by
$\quotep{\procn{x}|\procn{x}}$.

Also, we will avail ourselves of the notation $x^{L}$ and $x^{R}$ to
denote injections of a name into disjoint copies of the name
space. There are numerous ways to accomplish this. One example can be
found in \cite{MeredithR05}. This notation overloads to vectors of
names: $\vec{x}^{\pi} := (x_{i}^{\pi} \; : \; 0 \leq i < |\vec{x}| )$ where $\pi \in \{L,R\}$.

We also use $P^{\Box} := P|\Box$.

In \cite{MeredithR05} an interpretation of the new operator is
given. It turns out that there are several possible interpretations
all enjoying the requisite algebraic properties of the operator (see
\cite{milner91polyadicpi}). We will therefore make liberal use of
$(\nu\; \vec{x})P$.

% subsection the_syntax_and_semantics_of_the_notation_system (end)   

\input{qm2pi.qmops} 

\input{qm2pi.sterngerlach} 

\input{qm2pi.metric} 

% section concurrent_process_calculi (end)

%\input{qm2pi.proofsketch}

% section proof sketch (end)

%\input{qm2pi.slviaknots} 

% section spatial logic via knots (end)

\input{qm2pi.conclusion}

% section conclusion (end)

%\input{qm2pi.dtcodes} 

% section wiring algorithm (end)

\input{qm2pi.ack} 

% section acknowledgments (end)

\newpage


\bibliographystyle{plain}   
\bibliography{../../biblios/main.bib}

\input{qm2pi.rhodetails}

\end{document}

 

%\documentclass[12pt]{llncs}
%\documentclass{jktr}

\usepackage[pdftex]{hyperref}                   
\usepackage {listings}
\usepackage {mathpartir}
\usepackage{bcprules}
%\usepackage{listings}
                       
\usepackage{graphicx} 
%\usepackage[margins=2.5cm,nohead,nofoot]{geometry}
%\usepackage{geometry}
\usepackage{amsfonts}
\usepackage{amstext}
\usepackage{latexsym}
\usepackage{amssymb}
\usepackage{color}


%\include{myPreamble}
\include{qm2pi.local} 

%\ifpdf
%\usepackage[pdftex]{graphicx}
%\else
%\usepackage{graphicx}
%\fi

 % \ifpdf
%  \usepackage{pdfsync}
%  \if


%\title{Brief Article}
%\author{David F. Snyder}
%\author{L.G. Meredith}

%\address{Dept. of Math., Texas State University--San Marcos, San Marcos, TX 78666}
       
\pagestyle{empty}


\begin{document}

\lstset{language=[Objective]Caml,frame=shadowbox}

\input{qm2pi.front}

% section front matter (end)

\input{qm2pi.intro} 
 
% section introduction (end)

% \input{qm2pi.knotations} 

% section notation (end)

\input{qm2pi.process.calculi} 

% section concurrent_process_calculi_and_spatial_logics_ (end)
    
%\input{qm2pi.knots2pi} 

%\input{qm2pi.trefoil} 

%\input{qm2pi.mainthm} 

% subsection basic_interpretation (end)

%\input{qm2pi.rho.presentation} 
\subsection{The syntax and semantics of the notation system}\label{sub:the_syntax_and_semantics_of_the_notation_system} % (fold)

We now summarize a technical presentation of the calculus that
embodies our theory of dynamics. The typical presentation of such a
calculus follows the style of giving generators and relations on
them. The grammar, below, describing term constructors, freely
generates the set of processes, $\Proc$. This set is then quotiented
by a relation known as structural congruence and it is over this set
that the notion of dynamics is expressed. This presentation is
essentially that of \cite{MeredithR05} with the addition of
polyadicity and summation. For readability we have relegated some of
the technical subtleties to an appendix.

\subsubsection{Process grammar}\label{subsub:process_grammar}

\begin{mathpar}
  \inferrule* [lab=synchronization] {} {{M} \bc \pzero \;|\; x?F \;|\; x!C }
  \and
  \inferrule* [lab=abstraction] {} {{F} \bc (x)P}
  \and
  \inferrule* [lab=concretion] {} {{C} \bc \langle Q \rangle}
  \and
  \inferrule* [lab=process] {} {{P,Q} \bc M \;| \;P|Q \;|\; @{x}}
  \and
  \inferrule* [lab=name] {} {{x} \bc \quotep{P}}
\end{mathpar} 

Note that $\vec{x}$ (resp. $\vec{P}$) denotes a vector of names
(resp. processes) of length $|\vec{x}|$ (resp. $|\vec{P}|$). We adopt
the following useful abbreviations.

\begin{mathpar}
   x?(\vec{y}).P := x.(\vec{y})P \and  x\clift{\vec{P}} := x.\clift{\vec{P}}
   \and x!(y) := \lift{x}{\dropn{y}}
   \and \Pi_{i=0}^{n-1}P_i := P_0 | \ldots | P_{n-1}
\end{mathpar}

\subsubsection{Structural congruence}

\paragraph{Free and bound names and alpha-equivalence.} At the
core of structural equivalence is alpha-equivalence which identifies
process that are the same up to a change of variable. Formally, we
recognize the distinction between free and bound names. The free names
of a process, $\freenames{P}$, may be calculated recursively as
follows:

\begin{mathpar}
\freenames{\pzero} := \emptyset
  \and \\
  \freenames{x?(y).P} := \{ x \} \cup (\freenames{P} \setminus \{ y \})
  \and 
  \freenames{x!\langle P \rangle} := \{ x \} \cup \{ P \} 
  \and \\
  \freenames{P|Q} := \freenames{P} \cup \freenames{Q}
  \and \\
  \freenames{@{x}} := \{ x \}
\end{mathpar}

$\pi$
$\quotep{\pi}$

$\freenames{-} : \pi \to \mathcal{P}(\quotep{\pi})$

\begin{eqnarray*}
  \freenames{\pzero} & := & \emptyset \\
  \freenames{x?(y).P} & := & \{ x \} \cup (\freenames{P} \setminus \{ y \}) \\
  \freenames{x!\langle P \rangle} & := & \{ x \} \cup \{ P \} \\
  \freenames{P|Q} & := & \freenames{P} \cup \freenames{Q} \\
  \freenames{\dropn{x}} & := & \{ x \}
\end{eqnarray*}

The bound names of a process, $\boundnames{P}$, are those names occurring in $P$
that are not free. For example, in $x?(y).0$, the name $x$ is free, while $y$ is bound.

\begin{mathpar}
  \inferrule* [lab=monoidal-laws] {} { P|Q \equiv Q|P \and P|0 \equiv P \and P|(Q|R) \equiv (P|Q)|R }
\end{mathpar}

\begin{mathpar}
  \inferrule* [lab=alpha-equivalence] {} { (x)P \equiv (y)P\{y/x\} \and y \not\in \freenames{P} }
\end{mathpar}

\begin{definition}
Then two processes, $P,Q$, are alpha-equivalent if $P = Q\{\vec{y}/\vec{x}\}$ for
some $\vec{x} \in \boundnames{Q},\vec{y} \in \boundnames{P}$, where $Q\{\vec{y}/\vec{x}\}$
denotes the capture-avoiding substitution of $\vec{y}$ for $\vec{x}$ in $Q$.
\end{definition}

\begin{definition}
  The {\em structural congruence} \cite{SangiorgiWalker} , $\equiv$,
  between processes is the least congruence containing
  alpha-equivalence, satisfying the abelian monoid laws
  (associativity, commutativity and $\pzero$ as identity) for parallel
  composition $|$ and for summation $+$.
\end{definition}

\subsection{Name equivalence}

We take name equivalence, written $\nameeq$, to be the smallest
equivalence relation generated by the following rules.

\begin{mathpar}
\inferrule*[lab=Quote-drop]
{ }
{ \quotep{@{x}} \nameeq x }

\inferrule*[lab=Struct-equiv]
{ P \scong Q }
{ \quotep{P} \nameeq \quotep{Q} }
\end{mathpar}

The astute reader will have noticed that the mutual recursion of names
and processes imposes a mutual recursion on alpha-equivalence and
structural equivalence via name-equivalence. Fortunately, all of this
works out pleasantly and we may calculate in the natural way, free of
concern. The reader interested in the details is referred to the
appendix \ref{appendix:rho_details}.

\subsection{Substitution}

We use $\Proc$ for the set of processes, $\QProc$ for the set of
names, and $\id{\{}\vec{y} / \vec{x} \id{\}}$ to denote partial maps,
$s : \QProc \rightarrow \QProc$. A map, $s$ lifts, uniquely, to a map
on process terms, $\widehat{s} : \Proc \rightarrow \Proc$ by the
following equations.

\begin{mathpar}
  (0) \psubstp{Q}{P} := 0 \\
  (R \juxtap S) \psubstp{Q}{P}
  :=    
  (R)\psubstp{Q}{P} \juxtap (S) \psubstp{Q}{P} \\
  (x?(y).R) \psubstp{Q}{P}    
  :=    
  (x)\substp{Q}{P} (z)\concat( (R \psubstn{z}{y}) \psubstp{Q}{P} ) \\
  (\lift{x}{R}) \psubstp{Q}{P}  
  :=
  \lift{(x)\substp{Q}{P}}{ R \psubstp{Q}{P} } \\
%   (\dropn{x})  \psubstp{Q}{P}       
%   := 
%   \left\{ 
%     \begin{array}{ccc} 
%       \dropn{\quotep{Q}} & & x \nameeq \quotep{P} \\
%       \dropn{x} & & otherwise \\
%     \end{array}
%   \right. 
  (\dropn{x})  \psubstp{Q}{P}       
  := 
  \left\{ 
    \begin{array}{ccc} 
      Q & & x \nameeq \quotep{P} \\
      \dropn{x} & & otherwise \\
    \end{array}
  \right.
\end{mathpar}
 

where

\begin{eqnarray}
  (x)\id{\{} \lpquote Q \rpquote / \lpquote P \rpquote \id{\}}            = 
  \left\{ 
    \begin{array}{ccc}
      \lpquote Q \rpquote & & x \nameeq \lpquote P \rpquote \\
      x & & otherwise \\
    \end{array}
  \right. \nonumber
\end{eqnarray}

and $z$ is chosen distinct from $\quotep{P}$, $\quotep{Q}$, the free
names in $Q$, and all the names in $R$. Our $\alpha$-equivalence will
be built in the standard way from this substitution.

\begin{remark}\label{rem:no_self_referential_names}
  One consequence of these definitions is that $\forall P. \quotep{P}
  \not\in \freenames{P}$.
\end{remark}

\subsection{ Dynamic quote: an example }

Anticipating something of what's to come, consider applying the
substitution, $\widehat{\id{\{}u / z \id{\}}}$, to the following pair
of processes, $\lift{w}{y!(z)}$ and $w[ \lpquote y!(z) \rpquote ]$.

\begin{eqnarray}
	\lift{w}{y!(z)}\widehat{\id{\{}u / z \id{\}}}
		& = &
		\lift{w}{y!(u)} \nonumber\\
	w[ \lpquote y!(z) \rpquote ] \widehat{ \id{\{}u / z \id{\}} }
		& = &
		w[ \lpquote y!(z) \rpquote ] \nonumber
\end{eqnarray}

Because the body of the process between quotes is impervious to
substitution, we get radically different answers. In fact, by
examining the first process in an input context,
e.g. $x?(z).\lift{w}{y!(z)}$, we see that the process under the lift
operator may be shaped by prefixed inputs binding a name inside it. In
this sense, the lift operator will be seen as a way to dynamically
construct processes before reifying them as names.

Finally equipped with these standard features we can present the
dynamics of the calculus.

\subsubsection{Operational semantics} 

Finally, we introduce the computational dynamics. What marks these
algebras as distinct from other more traditionally studied algebraic
structures, e.g. vector spaces or polynomial rings, is the manner in
which dynamics is captured. In traditional structures, dynamics is typically
expressed through morphisms between such structures, as in linear maps
between vector spaces or morphisms between rings. In algebras
associated with the semantics of computation, the dynamics is
expressed as part of the algebraic structure itself, through a
reduction reduction relation typically denoted by $\red$. Below, we
give a recursive presentation of this relation for the calculus used
in the encoding.

$\red \subseteq \pi \times \pi$
$\red : \pi \to \mathcal{P}(\pi)$

\begin{mathpar}
  \inferrule* [lab=Comm] { \textsf{match}( x_{src}, x_{trgt} ) } { x_{trgt}?(y)P \; | \; x_{src}!\langle {Q} \rangle \red P\{\quotep{Q}/y}\} }
  \and \\
  \inferrule* [lab=Par] {{P} \red {P}'} {{{P} | {Q}} \red {{P}' | {Q}}}
  \and
  \inferrule* [lab=Equiv]{{{P} \scong {P}'} \andalso {{P}' \red {Q}'} \andalso {{Q}' \scong {Q}}}{{P} \red {Q}}
\end{mathpar}

\begin{eqnarray*}
  match_{\equiv} (\quotep{P},\quotep{Q}) & := & P \equiv Q \\
  match_{\dagger}(\quotep{P},\quotep{Q}) & := & \forall R. P|Q \red^{*} R => R \red^{*} 0 \\
  match_{K}(\quotep{P},\quotep{Q}) & := & K \mbox{ for some context } K
\end{eqnarray*}

$u?(x)P | u!\langle Q \rangle \red P\{\quotep{Q}/x\}$

%We write $\wred$ for $\red^*$, and $P\red$ if $\exists Q $ such that $ P \red Q$.
We write $P\red$ if $\exists Q $ such that $ P \red Q$ and $P\not\red$, otherwise.

\section{Replication}

As mentioned before, it is known that replication (and hence
recursion) can be implemented in a higher-order process algebra
\cite{SangiorgiWalker}. As our first example of calculation with the
machinery thus far presented we give the construction explicitly in
the {\rhoc}.

\begin{eqnarray}
	D_{x} & := & \prefix{x}{y}{(\binpar{\outputp{x}{y}}{@{y}})} \nonumber\\
	\bangp_{x}{P} & := & \binpar{{x}!\langle{\binpar{D_{x}}{P}}\rangle}{D_{x}} \nonumber
\end{eqnarray}

\begin{eqnarray}
	\bangp_{x}{P} & & \nonumber\\
	=
	& {x}!\langle{(\prefix{x}{y}{(\outputp{x}{y} | @{y})) | P}}\rangle 
	      | \prefix{x}{y}{(\outputp{x}{y} | @{y})} & \nonumber\\
	\red
	& (\outputp{x}{y} | @{y})\substn{\quotep{(\prefix{x}{y}{(@{y} | \outputp{x}{y})) | P}}}{y} & \nonumber\\
	=
	& \outputp{x}{\quotep{(\prefix{x}{y}{(\outputp{x}{y} | @{y})) | P}}}
	  | {(\prefix{x}{y}{(\outputp{x}{y} | @{y})) | P}} & \nonumber\\
	\red
	& \ldots & \nonumber\\
	\red^*
	& P | P | \ldots & \nonumber
\end{eqnarray}

Of course, this encoding, as an implementation, runs away, unfolding
$\bangp{P}$ eagerly. A lazier and more implementable replication
operator, restricted to input-guarded processes, may be obtained as follows.

\begin{eqnarray}
\bangp{\prefix{u}{v}{P}} 
	:= 
	\binpar{\lift{x}{\prefix{u}{v}{(\binpar{D(x)}{P})}}}{D(x)} \nonumber
\end{eqnarray}

\begin{remark}
  Note that the lazier definition still does not deal with summation
  or mixed summation (i.e. sums over input and output). The reader is
  invited to construct definitions of replication that deal with these
  features. 

  Further, the definitions are parameterized in a name, $x$. Can you,
  gentle reader, make a definition that eliminates this parameter and
  guarantees no accidental interaction between the replication
  machinery and the process being replicated -- i.e. no accidental
  sharing of names used by the process to get its work done and the
  name(s) used by the replication to effect copying. This latter
  revision of the definition of replication is crucial to obtaining
  the expected identity $!!P \sim !P$.
\end{remark}

\begin{remark}\label{rem:paradoxical_combinator}
  The reader familiar with the lambda calculus will have noticed the
  similarity between $D$ and the paradoxical combinator.

  [Ed. note: the existence of this seems to suggest we have to be more
  restrictive on the set of processes and names we admit if we are to
  support no-cloning.]
\end{remark}

\subsubsection{Bisimulation}

The computational dynamics gives rise to another kind of equivalence,
the equivalence of computational behavior. As previously mentioned
this is typically captured \emph{via} some form of bisimulation.

% The notion we use in this paper is weak barbed bisimulation
% \cite{milner91polyadicpi}.

The notion we use in this paper is derived from weak barbed
bisimulation \cite{milner91polyadicpi}. 

\begin{definition}
An \emph{observation relation}, $\downarrow_{\mathcal N}$, over a set
of names, $\mathcal N$, is the smallest relation satisfying the rules
below.

\infrule[Out-barb]{y \in {\mathcal N}, \; x \nameeq y}
		  {\outputp{x}{v} \downarrow_{\mathcal N} x}
\infrule[Par-barb]{\mbox{$P\downarrow_{\mathcal N} x$ or $Q\downarrow_{\mathcal N} x$}}
		  {\binpar{P}{Q} \downarrow_{\mathcal N} x}

We write $P \Downarrow_{\mathcal N} x$ if there is $Q$ such that 
$P \wred Q$ and $Q \downarrow_{\mathcal N} x$.
\end{definition}

\begin{definition}
%\label{def.bbisim}
An  ${\mathcal N}$-\emph{barbed bisimulation} over a set of names, ${\mathcal N}$, is a symmetric binary relation 
${\mathcal S}_{\mathcal N}$ between agents such that $P\rel{S}_{\mathcal N}Q$ implies:
\begin{enumerate}
\item If $P \red P'$ then $Q \wred Q'$ and $P'\rel{S}_{\mathcal N} Q'$.
\item If $P\downarrow_{\mathcal N} x$, then $Q\Downarrow_{\mathcal N} x$.
\end{enumerate}
$P$ is ${\mathcal N}$-barbed bisimilar to $Q$, written
$P \wbbisim_{\mathcal N} Q$, if $P \rel{S}_{\mathcal N} Q$ for some ${\mathcal N}$-barbed bisimulation ${\mathcal S}_{\mathcal N}$.
\end{definition}

$\mathcal{R} \subseteq \pi \times \pi$

$P \mathcal{R} Q => \forall P'. P \red P' \Rightarrow \exists Q'. Q \red Q', P' \mathcal{R} Q'$

$P \vdash x \Rightarrow Q \vdash x$

\begin{mathpar}
  \inferrule*[lab=Out-barb]{x \nameeq y}{{y}!\langle{Q}\rangle \vdash x}
  \and
  \inferrule*[lab=Par-barb]{\mbox{$P\vdash x$ or $Q\vdash x$}}{\binpar{P}{Q} \vdash x}
\end{mathpar}

\subsubsection{Contexts}

One of the principle advantages of computational calculi like the
$\pi$-calculus is a well-defined notion of context,
contextual-equivalence and a correlation between
contextual-equivalence and notions of bisimulation. The notion of
context allows the decomposition of a process into (sub-)process and
its syntactic environment, its context. Thus, a context may be
thought of as a process with a ``hole'' (written $\Box$) in it. The
application of a context $M$ to a process $P$, written $M[P]$, is
tantamount to filling the hole in $M$ with $P$. In this paper we do
not need the full weight of this theory, but do make use of the notion
of context in the proof the main theorem. 

\begin{mathpar}
  \inferrule* [lab=summation] {} {{M_{M},M_{N}} \bc \Box \;|\; x.M_{A} \;|\; M_{M}+M_{N}}
  \and
  \inferrule* [lab=agent] {} {{M_{A}} \bc (\vec{x})M_{P} \;| \; \clift{P_0,\ldots,M_{P},\ldots,P_N}}
  \and \\
  \inferrule* [lab=process] {} {{M_{P}} \bc M_{N} \;| \;P|M_{P} }
\end{mathpar} 

\begin{mathpar}
  \inferrule* [lab=sychronization] {} {M_{N} \bc \Box \;|\; x?M_{F} \;|\; x!M_{C}}
  \and
  \inferrule* [lab=abstraction] {} {{M_{F}} \bc (x)M_{P} }
  \and
  \inferrule* [lab=concretion] {} {{M_{C}} \bc \langle M_{P} \rangle }
  \and \\
  \inferrule* [lab=process] {} {{M_{P}} \bc M_{N} \;| \;P|M_{P} }
\end{mathpar}

\begin{definition}[contextual application] Given a context $M$, and
  process $P$, we define the \emph{contextual application}, $M[P] :=
  M\{P/\Box\}$. That is, the contextual application of M to P is the
  substitution of $P$ for $\Box$ in $M$.
\end{definition}

$\meaningof{-} : L \to \mathcal{P}(\pi)$

\begin{mathpar}
  \inferrule* [lab=collection] {} {\meaningof{true} = \pi, \and \meaningof{~E} = \pi \setminus \meaningof{E}, \and \meaningof{E_{1} \& E_{2}} = \meaningof{E_{1}} \cap \meaningof{E_{2}}}
\end{mathpar}

\begin{mathpar}
  \inferrule* [lab=structure] {} {\meaningof{0} = \{ P \in \pi | P \equiv 0 \}, \and \\ \meaningof{E_1 | E_2} = \{ P \in \pi | P \equiv P_{1} | P_{2}, P_{1} \in \meaningof{E_{1}}, P_{2} \in \meaningof{E_2}\} }
\end{mathpar}

\begin{mathpar}
 \inferrule* [lab=behavior] {} {\meaningof{\langle a?b \rangle E} = \{ P \in \pi | P \equiv Q | u?(y)P', \\ \and \\\\ \and \\ \;\;\; u \in \meaningof{a}, \forall z.P'\{z/y\} \in \meaningof{E\{z/b\}}\}, \and \\ \meaningof{a!E} = \{ P \in \pi | P \equiv Q | x!\langle P' \rangle, x \in \meaningof{a} P' \in \meaningof{E}\} }
\end{mathpar}

\begin{mathpar}
 \inferrule* [lab=nominal] {} {\meaningof{\quotep{E}} = \{ \quotep{P} \in \quotep{\pi} | P \in \meaningof{E} \}, \and \meaningof{\quotep{P}} = \{ \quotep{Q} \in \quotep{\pi} | P \equiv Q \} \and \\ \meaningof{@\quotep{E}} = \{ P \in \pi | P \equiv @x, x \in \meaningof{E} \}}
\end{mathpar}

\begin{eqnarray*}
  \\
  \meaningof{-} : TS \to ST
\end{eqnarray*}

\begin{eqnarray*}
  \\
  L : TS \to ST
\end{eqnarray*}

\begin{eqnarray*}
  \\
  P \models E \iff P \in \meaningof{E}
\end{eqnarray*}

\begin{eqnarray*}
  P \approx_{L} Q \iff \forall E \in L. P \models E \iff Q \models E
\end{eqnarray*}

\begin{eqnarray*}
  P \approx_{K} Q
\end{eqnarray*}

\begin{eqnarray*}
  P \approx Q
\end{eqnarray*}

$\approx_{K} = \approx = \approx_{L}$

\subsubsection{Contextual duality}

Note that contexts extend the quotation operation to a family of
operations from processes to names. Given a context, $M$, we can
define a \emph{nominal context}, $\quotep{M}$ by $\quotep{M}[P] :=
\quotep{M[P]}$. To foreshadow what is to come we observe that these
operations enjoy a duality with processes very much like the duality
between vectors and maps from vectors to scalars.

Further, because the calculus is essentially higher-order, we have a
correspondence between contexts and processes. More specifically,
given a name $x$ and a context $M$ we can construct $M^{*}_{x}$ such
that 

\begin{mathpar}
  M^{*}_{x} | \lift{x}{P} \red M[P]
\end{mathpar}

namely,

\begin{mathpar}
  M^{*}_{x} := x?(u).M[\dropn{u}]
\end{mathpar}

The dependence of $M^{*}_{x}$ on a name makes it an abstraction, 

\begin{mathpar}
  M^{*} := (x)x?(u).M[\dropn{u}]
\end{mathpar}

\subsection{Additional notation}

It will sometimes be convenient to denote the process a name
quotes. We already have the notation $x = \quotep{P}$, but it will be
convenient to introduce an alternate notation, $\procn{x}$, when we
want to emphasize the connection to the use of the name. Note that, by
virtue of name equivalence, $\quotep{\procn{x}} \nameeq x$; so, the
notation is consistent with previous definitions.

Further, because names have structure it is possible to effect
substitutions on the basis of that structure. This means we need to
upgrade our notation for substitutions, which we accomplish by
adapting comprehension notation. Thus,

\begin{mathpar}
  P\{ y / x : x \in S \}
\end{mathpar}

is interpreted to mean the process derived from P by replacing (in a
capture-avoiding manner) each occurrence of $x$ in $S$ by $y$. For example,

\begin{mathpar}
  P\{ \quotep{\procn{x}|\procn{x}} / x : x \in \freenames{P} \}
\end{mathpar}

will replace each (occurrence) of a free name $x$ in $P$ by
$\quotep{\procn{x}|\procn{x}}$.

Also, we will avail ourselves of the notation $x^{L}$ and $x^{R}$ to
denote injections of a name into disjoint copies of the name
space. There are numerous ways to accomplish this. One example can be
found in \cite{MeredithR05}. This notation overloads to vectors of
names: $\vec{x}^{\pi} := (x_{i}^{\pi} \; : \; 0 \leq i < |\vec{x}| )$ where $\pi \in \{L,R\}$.

We also use $P^{\Box} := P|\Box$.

In \cite{MeredithR05} an interpretation of the new operator is
given. It turns out that there are several possible interpretations
all enjoying the requisite algebraic properties of the operator (see
\cite{milner91polyadicpi}). We will therefore make liberal use of
$(\nu\; \vec{x})P$.

% subsection the_syntax_and_semantics_of_the_notation_system (end)   

\input{qm2pi.qmops} 

\input{qm2pi.sterngerlach} 

\input{qm2pi.metric} 

% section concurrent_process_calculi (end)

%\input{qm2pi.proofsketch}

% section proof sketch (end)

%\input{qm2pi.slviaknots} 

% section spatial logic via knots (end)

\input{qm2pi.conclusion}

% section conclusion (end)

%\input{qm2pi.dtcodes} 

% section wiring algorithm (end)

\input{qm2pi.ack} 

% section acknowledgments (end)

\newpage


\bibliographystyle{plain}   
\bibliography{../../biblios/main.bib}

\input{qm2pi.rhodetails}

\end{document}

 

%\documentclass[12pt]{llncs}
%\documentclass{jktr}

\usepackage[pdftex]{hyperref}                   
\usepackage {listings}
\usepackage {mathpartir}
\usepackage{bcprules}
%\usepackage{listings}
                       
\usepackage{graphicx} 
%\usepackage[margins=2.5cm,nohead,nofoot]{geometry}
%\usepackage{geometry}
\usepackage{amsfonts}
\usepackage{amstext}
\usepackage{latexsym}
\usepackage{amssymb}
\usepackage{color}


%\include{myPreamble}
\include{qm2pi.local} 

%\ifpdf
%\usepackage[pdftex]{graphicx}
%\else
%\usepackage{graphicx}
%\fi

 % \ifpdf
%  \usepackage{pdfsync}
%  \if


%\title{Brief Article}
%\author{David F. Snyder}
%\author{L.G. Meredith}

%\address{Dept. of Math., Texas State University--San Marcos, San Marcos, TX 78666}
       
\pagestyle{empty}


\begin{document}

\lstset{language=[Objective]Caml,frame=shadowbox}

\input{qm2pi.front}

% section front matter (end)

\input{qm2pi.intro} 
 
% section introduction (end)

% \input{qm2pi.knotations} 

% section notation (end)

\input{qm2pi.process.calculi} 

% section concurrent_process_calculi_and_spatial_logics_ (end)
    
%\input{qm2pi.knots2pi} 

%\input{qm2pi.trefoil} 

%\input{qm2pi.mainthm} 

% subsection basic_interpretation (end)

%\input{qm2pi.rho.presentation} 
\subsection{The syntax and semantics of the notation system}\label{sub:the_syntax_and_semantics_of_the_notation_system} % (fold)

We now summarize a technical presentation of the calculus that
embodies our theory of dynamics. The typical presentation of such a
calculus follows the style of giving generators and relations on
them. The grammar, below, describing term constructors, freely
generates the set of processes, $\Proc$. This set is then quotiented
by a relation known as structural congruence and it is over this set
that the notion of dynamics is expressed. This presentation is
essentially that of \cite{MeredithR05} with the addition of
polyadicity and summation. For readability we have relegated some of
the technical subtleties to an appendix.

\subsubsection{Process grammar}\label{subsub:process_grammar}

\begin{mathpar}
  \inferrule* [lab=synchronization] {} {{M} \bc \pzero \;|\; x?F \;|\; x!C }
  \and
  \inferrule* [lab=abstraction] {} {{F} \bc (x)P}
  \and
  \inferrule* [lab=concretion] {} {{C} \bc \langle Q \rangle}
  \and
  \inferrule* [lab=process] {} {{P,Q} \bc M \;| \;P|Q \;|\; @{x}}
  \and
  \inferrule* [lab=name] {} {{x} \bc \quotep{P}}
\end{mathpar} 

Note that $\vec{x}$ (resp. $\vec{P}$) denotes a vector of names
(resp. processes) of length $|\vec{x}|$ (resp. $|\vec{P}|$). We adopt
the following useful abbreviations.

\begin{mathpar}
   x?(\vec{y}).P := x.(\vec{y})P \and  x\clift{\vec{P}} := x.\clift{\vec{P}}
   \and x!(y) := \lift{x}{\dropn{y}}
   \and \Pi_{i=0}^{n-1}P_i := P_0 | \ldots | P_{n-1}
\end{mathpar}

\subsubsection{Structural congruence}

\paragraph{Free and bound names and alpha-equivalence.} At the
core of structural equivalence is alpha-equivalence which identifies
process that are the same up to a change of variable. Formally, we
recognize the distinction between free and bound names. The free names
of a process, $\freenames{P}$, may be calculated recursively as
follows:

\begin{mathpar}
\freenames{\pzero} := \emptyset
  \and \\
  \freenames{x?(y).P} := \{ x \} \cup (\freenames{P} \setminus \{ y \})
  \and 
  \freenames{x!\langle P \rangle} := \{ x \} \cup \{ P \} 
  \and \\
  \freenames{P|Q} := \freenames{P} \cup \freenames{Q}
  \and \\
  \freenames{@{x}} := \{ x \}
\end{mathpar}

$\pi$
$\quotep{\pi}$

$\freenames{-} : \pi \to \mathcal{P}(\quotep{\pi})$

\begin{eqnarray*}
  \freenames{\pzero} & := & \emptyset \\
  \freenames{x?(y).P} & := & \{ x \} \cup (\freenames{P} \setminus \{ y \}) \\
  \freenames{x!\langle P \rangle} & := & \{ x \} \cup \{ P \} \\
  \freenames{P|Q} & := & \freenames{P} \cup \freenames{Q} \\
  \freenames{\dropn{x}} & := & \{ x \}
\end{eqnarray*}

The bound names of a process, $\boundnames{P}$, are those names occurring in $P$
that are not free. For example, in $x?(y).0$, the name $x$ is free, while $y$ is bound.

\begin{mathpar}
  \inferrule* [lab=monoidal-laws] {} { P|Q \equiv Q|P \and P|0 \equiv P \and P|(Q|R) \equiv (P|Q)|R }
\end{mathpar}

\begin{mathpar}
  \inferrule* [lab=alpha-equivalence] {} { (x)P \equiv (y)P\{y/x\} \and y \not\in \freenames{P} }
\end{mathpar}

\begin{definition}
Then two processes, $P,Q$, are alpha-equivalent if $P = Q\{\vec{y}/\vec{x}\}$ for
some $\vec{x} \in \boundnames{Q},\vec{y} \in \boundnames{P}$, where $Q\{\vec{y}/\vec{x}\}$
denotes the capture-avoiding substitution of $\vec{y}$ for $\vec{x}$ in $Q$.
\end{definition}

\begin{definition}
  The {\em structural congruence} \cite{SangiorgiWalker} , $\equiv$,
  between processes is the least congruence containing
  alpha-equivalence, satisfying the abelian monoid laws
  (associativity, commutativity and $\pzero$ as identity) for parallel
  composition $|$ and for summation $+$.
\end{definition}

\subsection{Name equivalence}

We take name equivalence, written $\nameeq$, to be the smallest
equivalence relation generated by the following rules.

\begin{mathpar}
\inferrule*[lab=Quote-drop]
{ }
{ \quotep{@{x}} \nameeq x }

\inferrule*[lab=Struct-equiv]
{ P \scong Q }
{ \quotep{P} \nameeq \quotep{Q} }
\end{mathpar}

The astute reader will have noticed that the mutual recursion of names
and processes imposes a mutual recursion on alpha-equivalence and
structural equivalence via name-equivalence. Fortunately, all of this
works out pleasantly and we may calculate in the natural way, free of
concern. The reader interested in the details is referred to the
appendix \ref{appendix:rho_details}.

\subsection{Substitution}

We use $\Proc$ for the set of processes, $\QProc$ for the set of
names, and $\id{\{}\vec{y} / \vec{x} \id{\}}$ to denote partial maps,
$s : \QProc \rightarrow \QProc$. A map, $s$ lifts, uniquely, to a map
on process terms, $\widehat{s} : \Proc \rightarrow \Proc$ by the
following equations.

\begin{mathpar}
  (0) \psubstp{Q}{P} := 0 \\
  (R \juxtap S) \psubstp{Q}{P}
  :=    
  (R)\psubstp{Q}{P} \juxtap (S) \psubstp{Q}{P} \\
  (x?(y).R) \psubstp{Q}{P}    
  :=    
  (x)\substp{Q}{P} (z)\concat( (R \psubstn{z}{y}) \psubstp{Q}{P} ) \\
  (\lift{x}{R}) \psubstp{Q}{P}  
  :=
  \lift{(x)\substp{Q}{P}}{ R \psubstp{Q}{P} } \\
%   (\dropn{x})  \psubstp{Q}{P}       
%   := 
%   \left\{ 
%     \begin{array}{ccc} 
%       \dropn{\quotep{Q}} & & x \nameeq \quotep{P} \\
%       \dropn{x} & & otherwise \\
%     \end{array}
%   \right. 
  (\dropn{x})  \psubstp{Q}{P}       
  := 
  \left\{ 
    \begin{array}{ccc} 
      Q & & x \nameeq \quotep{P} \\
      \dropn{x} & & otherwise \\
    \end{array}
  \right.
\end{mathpar}
 

where

\begin{eqnarray}
  (x)\id{\{} \lpquote Q \rpquote / \lpquote P \rpquote \id{\}}            = 
  \left\{ 
    \begin{array}{ccc}
      \lpquote Q \rpquote & & x \nameeq \lpquote P \rpquote \\
      x & & otherwise \\
    \end{array}
  \right. \nonumber
\end{eqnarray}

and $z$ is chosen distinct from $\quotep{P}$, $\quotep{Q}$, the free
names in $Q$, and all the names in $R$. Our $\alpha$-equivalence will
be built in the standard way from this substitution.

\begin{remark}\label{rem:no_self_referential_names}
  One consequence of these definitions is that $\forall P. \quotep{P}
  \not\in \freenames{P}$.
\end{remark}

\subsection{ Dynamic quote: an example }

Anticipating something of what's to come, consider applying the
substitution, $\widehat{\id{\{}u / z \id{\}}}$, to the following pair
of processes, $\lift{w}{y!(z)}$ and $w[ \lpquote y!(z) \rpquote ]$.

\begin{eqnarray}
	\lift{w}{y!(z)}\widehat{\id{\{}u / z \id{\}}}
		& = &
		\lift{w}{y!(u)} \nonumber\\
	w[ \lpquote y!(z) \rpquote ] \widehat{ \id{\{}u / z \id{\}} }
		& = &
		w[ \lpquote y!(z) \rpquote ] \nonumber
\end{eqnarray}

Because the body of the process between quotes is impervious to
substitution, we get radically different answers. In fact, by
examining the first process in an input context,
e.g. $x?(z).\lift{w}{y!(z)}$, we see that the process under the lift
operator may be shaped by prefixed inputs binding a name inside it. In
this sense, the lift operator will be seen as a way to dynamically
construct processes before reifying them as names.

Finally equipped with these standard features we can present the
dynamics of the calculus.

\subsubsection{Operational semantics} 

Finally, we introduce the computational dynamics. What marks these
algebras as distinct from other more traditionally studied algebraic
structures, e.g. vector spaces or polynomial rings, is the manner in
which dynamics is captured. In traditional structures, dynamics is typically
expressed through morphisms between such structures, as in linear maps
between vector spaces or morphisms between rings. In algebras
associated with the semantics of computation, the dynamics is
expressed as part of the algebraic structure itself, through a
reduction reduction relation typically denoted by $\red$. Below, we
give a recursive presentation of this relation for the calculus used
in the encoding.

$\red \subseteq \pi \times \pi$
$\red : \pi \to \mathcal{P}(\pi)$

\begin{mathpar}
  \inferrule* [lab=Comm] { \textsf{match}( x_{src}, x_{trgt} ) } { x_{trgt}?(y)P \; | \; x_{src}!\langle {Q} \rangle \red P\{\quotep{Q}/y}\} }
  \and \\
  \inferrule* [lab=Par] {{P} \red {P}'} {{{P} | {Q}} \red {{P}' | {Q}}}
  \and
  \inferrule* [lab=Equiv]{{{P} \scong {P}'} \andalso {{P}' \red {Q}'} \andalso {{Q}' \scong {Q}}}{{P} \red {Q}}
\end{mathpar}

\begin{eqnarray*}
  match_{\equiv} (\quotep{P},\quotep{Q}) & := & P \equiv Q \\
  match_{\dagger}(\quotep{P},\quotep{Q}) & := & \forall R. P|Q \red^{*} R => R \red^{*} 0 \\
  match_{K}(\quotep{P},\quotep{Q}) & := & K \mbox{ for some context } K
\end{eqnarray*}

$u?(x)P | u!\langle Q \rangle \red P\{\quotep{Q}/x\}$

%We write $\wred$ for $\red^*$, and $P\red$ if $\exists Q $ such that $ P \red Q$.
We write $P\red$ if $\exists Q $ such that $ P \red Q$ and $P\not\red$, otherwise.

\section{Replication}

As mentioned before, it is known that replication (and hence
recursion) can be implemented in a higher-order process algebra
\cite{SangiorgiWalker}. As our first example of calculation with the
machinery thus far presented we give the construction explicitly in
the {\rhoc}.

\begin{eqnarray}
	D_{x} & := & \prefix{x}{y}{(\binpar{\outputp{x}{y}}{@{y}})} \nonumber\\
	\bangp_{x}{P} & := & \binpar{{x}!\langle{\binpar{D_{x}}{P}}\rangle}{D_{x}} \nonumber
\end{eqnarray}

\begin{eqnarray}
	\bangp_{x}{P} & & \nonumber\\
	=
	& {x}!\langle{(\prefix{x}{y}{(\outputp{x}{y} | @{y})) | P}}\rangle 
	      | \prefix{x}{y}{(\outputp{x}{y} | @{y})} & \nonumber\\
	\red
	& (\outputp{x}{y} | @{y})\substn{\quotep{(\prefix{x}{y}{(@{y} | \outputp{x}{y})) | P}}}{y} & \nonumber\\
	=
	& \outputp{x}{\quotep{(\prefix{x}{y}{(\outputp{x}{y} | @{y})) | P}}}
	  | {(\prefix{x}{y}{(\outputp{x}{y} | @{y})) | P}} & \nonumber\\
	\red
	& \ldots & \nonumber\\
	\red^*
	& P | P | \ldots & \nonumber
\end{eqnarray}

Of course, this encoding, as an implementation, runs away, unfolding
$\bangp{P}$ eagerly. A lazier and more implementable replication
operator, restricted to input-guarded processes, may be obtained as follows.

\begin{eqnarray}
\bangp{\prefix{u}{v}{P}} 
	:= 
	\binpar{\lift{x}{\prefix{u}{v}{(\binpar{D(x)}{P})}}}{D(x)} \nonumber
\end{eqnarray}

\begin{remark}
  Note that the lazier definition still does not deal with summation
  or mixed summation (i.e. sums over input and output). The reader is
  invited to construct definitions of replication that deal with these
  features. 

  Further, the definitions are parameterized in a name, $x$. Can you,
  gentle reader, make a definition that eliminates this parameter and
  guarantees no accidental interaction between the replication
  machinery and the process being replicated -- i.e. no accidental
  sharing of names used by the process to get its work done and the
  name(s) used by the replication to effect copying. This latter
  revision of the definition of replication is crucial to obtaining
  the expected identity $!!P \sim !P$.
\end{remark}

\begin{remark}\label{rem:paradoxical_combinator}
  The reader familiar with the lambda calculus will have noticed the
  similarity between $D$ and the paradoxical combinator.

  [Ed. note: the existence of this seems to suggest we have to be more
  restrictive on the set of processes and names we admit if we are to
  support no-cloning.]
\end{remark}

\subsubsection{Bisimulation}

The computational dynamics gives rise to another kind of equivalence,
the equivalence of computational behavior. As previously mentioned
this is typically captured \emph{via} some form of bisimulation.

% The notion we use in this paper is weak barbed bisimulation
% \cite{milner91polyadicpi}.

The notion we use in this paper is derived from weak barbed
bisimulation \cite{milner91polyadicpi}. 

\begin{definition}
An \emph{observation relation}, $\downarrow_{\mathcal N}$, over a set
of names, $\mathcal N$, is the smallest relation satisfying the rules
below.

\infrule[Out-barb]{y \in {\mathcal N}, \; x \nameeq y}
		  {\outputp{x}{v} \downarrow_{\mathcal N} x}
\infrule[Par-barb]{\mbox{$P\downarrow_{\mathcal N} x$ or $Q\downarrow_{\mathcal N} x$}}
		  {\binpar{P}{Q} \downarrow_{\mathcal N} x}

We write $P \Downarrow_{\mathcal N} x$ if there is $Q$ such that 
$P \wred Q$ and $Q \downarrow_{\mathcal N} x$.
\end{definition}

\begin{definition}
%\label{def.bbisim}
An  ${\mathcal N}$-\emph{barbed bisimulation} over a set of names, ${\mathcal N}$, is a symmetric binary relation 
${\mathcal S}_{\mathcal N}$ between agents such that $P\rel{S}_{\mathcal N}Q$ implies:
\begin{enumerate}
\item If $P \red P'$ then $Q \wred Q'$ and $P'\rel{S}_{\mathcal N} Q'$.
\item If $P\downarrow_{\mathcal N} x$, then $Q\Downarrow_{\mathcal N} x$.
\end{enumerate}
$P$ is ${\mathcal N}$-barbed bisimilar to $Q$, written
$P \wbbisim_{\mathcal N} Q$, if $P \rel{S}_{\mathcal N} Q$ for some ${\mathcal N}$-barbed bisimulation ${\mathcal S}_{\mathcal N}$.
\end{definition}

$\mathcal{R} \subseteq \pi \times \pi$

$P \mathcal{R} Q => \forall P'. P \red P' \Rightarrow \exists Q'. Q \red Q', P' \mathcal{R} Q'$

$P \vdash x \Rightarrow Q \vdash x$

\begin{mathpar}
  \inferrule*[lab=Out-barb]{x \nameeq y}{{y}!\langle{Q}\rangle \vdash x}
  \and
  \inferrule*[lab=Par-barb]{\mbox{$P\vdash x$ or $Q\vdash x$}}{\binpar{P}{Q} \vdash x}
\end{mathpar}

\subsubsection{Contexts}

One of the principle advantages of computational calculi like the
$\pi$-calculus is a well-defined notion of context,
contextual-equivalence and a correlation between
contextual-equivalence and notions of bisimulation. The notion of
context allows the decomposition of a process into (sub-)process and
its syntactic environment, its context. Thus, a context may be
thought of as a process with a ``hole'' (written $\Box$) in it. The
application of a context $M$ to a process $P$, written $M[P]$, is
tantamount to filling the hole in $M$ with $P$. In this paper we do
not need the full weight of this theory, but do make use of the notion
of context in the proof the main theorem. 

\begin{mathpar}
  \inferrule* [lab=summation] {} {{M_{M},M_{N}} \bc \Box \;|\; x.M_{A} \;|\; M_{M}+M_{N}}
  \and
  \inferrule* [lab=agent] {} {{M_{A}} \bc (\vec{x})M_{P} \;| \; \clift{P_0,\ldots,M_{P},\ldots,P_N}}
  \and \\
  \inferrule* [lab=process] {} {{M_{P}} \bc M_{N} \;| \;P|M_{P} }
\end{mathpar} 

\begin{mathpar}
  \inferrule* [lab=sychronization] {} {M_{N} \bc \Box \;|\; x?M_{F} \;|\; x!M_{C}}
  \and
  \inferrule* [lab=abstraction] {} {{M_{F}} \bc (x)M_{P} }
  \and
  \inferrule* [lab=concretion] {} {{M_{C}} \bc \langle M_{P} \rangle }
  \and \\
  \inferrule* [lab=process] {} {{M_{P}} \bc M_{N} \;| \;P|M_{P} }
\end{mathpar}

\begin{definition}[contextual application] Given a context $M$, and
  process $P$, we define the \emph{contextual application}, $M[P] :=
  M\{P/\Box\}$. That is, the contextual application of M to P is the
  substitution of $P$ for $\Box$ in $M$.
\end{definition}

$\meaningof{-} : L \to \mathcal{P}(\pi)$

\begin{mathpar}
  \inferrule* [lab=collection] {} {\meaningof{true} = \pi, \and \meaningof{~E} = \pi \setminus \meaningof{E}, \and \meaningof{E_{1} \& E_{2}} = \meaningof{E_{1}} \cap \meaningof{E_{2}}}
\end{mathpar}

\begin{mathpar}
  \inferrule* [lab=structure] {} {\meaningof{0} = \{ P \in \pi | P \equiv 0 \}, \and \\ \meaningof{E_1 | E_2} = \{ P \in \pi | P \equiv P_{1} | P_{2}, P_{1} \in \meaningof{E_{1}}, P_{2} \in \meaningof{E_2}\} }
\end{mathpar}

\begin{mathpar}
 \inferrule* [lab=behavior] {} {\meaningof{\langle a?b \rangle E} = \{ P \in \pi | P \equiv Q | u?(y)P', \\ \and \\\\ \and \\ \;\;\; u \in \meaningof{a}, \forall z.P'\{z/y\} \in \meaningof{E\{z/b\}}\}, \and \\ \meaningof{a!E} = \{ P \in \pi | P \equiv Q | x!\langle P' \rangle, x \in \meaningof{a} P' \in \meaningof{E}\} }
\end{mathpar}

\begin{mathpar}
 \inferrule* [lab=nominal] {} {\meaningof{\quotep{E}} = \{ \quotep{P} \in \quotep{\pi} | P \in \meaningof{E} \}, \and \meaningof{\quotep{P}} = \{ \quotep{Q} \in \quotep{\pi} | P \equiv Q \} \and \\ \meaningof{@\quotep{E}} = \{ P \in \pi | P \equiv @x, x \in \meaningof{E} \}}
\end{mathpar}

\begin{eqnarray*}
  \\
  \meaningof{-} : TS \to ST
\end{eqnarray*}

\begin{eqnarray*}
  \\
  L : TS \to ST
\end{eqnarray*}

\begin{eqnarray*}
  \\
  P \models E \iff P \in \meaningof{E}
\end{eqnarray*}

\begin{eqnarray*}
  P \approx_{L} Q \iff \forall E \in L. P \models E \iff Q \models E
\end{eqnarray*}

\begin{eqnarray*}
  P \approx_{K} Q
\end{eqnarray*}

\begin{eqnarray*}
  P \approx Q
\end{eqnarray*}

$\approx_{K} = \approx = \approx_{L}$

\subsubsection{Contextual duality}

Note that contexts extend the quotation operation to a family of
operations from processes to names. Given a context, $M$, we can
define a \emph{nominal context}, $\quotep{M}$ by $\quotep{M}[P] :=
\quotep{M[P]}$. To foreshadow what is to come we observe that these
operations enjoy a duality with processes very much like the duality
between vectors and maps from vectors to scalars.

Further, because the calculus is essentially higher-order, we have a
correspondence between contexts and processes. More specifically,
given a name $x$ and a context $M$ we can construct $M^{*}_{x}$ such
that 

\begin{mathpar}
  M^{*}_{x} | \lift{x}{P} \red M[P]
\end{mathpar}

namely,

\begin{mathpar}
  M^{*}_{x} := x?(u).M[\dropn{u}]
\end{mathpar}

The dependence of $M^{*}_{x}$ on a name makes it an abstraction, 

\begin{mathpar}
  M^{*} := (x)x?(u).M[\dropn{u}]
\end{mathpar}

\subsection{Additional notation}

It will sometimes be convenient to denote the process a name
quotes. We already have the notation $x = \quotep{P}$, but it will be
convenient to introduce an alternate notation, $\procn{x}$, when we
want to emphasize the connection to the use of the name. Note that, by
virtue of name equivalence, $\quotep{\procn{x}} \nameeq x$; so, the
notation is consistent with previous definitions.

Further, because names have structure it is possible to effect
substitutions on the basis of that structure. This means we need to
upgrade our notation for substitutions, which we accomplish by
adapting comprehension notation. Thus,

\begin{mathpar}
  P\{ y / x : x \in S \}
\end{mathpar}

is interpreted to mean the process derived from P by replacing (in a
capture-avoiding manner) each occurrence of $x$ in $S$ by $y$. For example,

\begin{mathpar}
  P\{ \quotep{\procn{x}|\procn{x}} / x : x \in \freenames{P} \}
\end{mathpar}

will replace each (occurrence) of a free name $x$ in $P$ by
$\quotep{\procn{x}|\procn{x}}$.

Also, we will avail ourselves of the notation $x^{L}$ and $x^{R}$ to
denote injections of a name into disjoint copies of the name
space. There are numerous ways to accomplish this. One example can be
found in \cite{MeredithR05}. This notation overloads to vectors of
names: $\vec{x}^{\pi} := (x_{i}^{\pi} \; : \; 0 \leq i < |\vec{x}| )$ where $\pi \in \{L,R\}$.

We also use $P^{\Box} := P|\Box$.

In \cite{MeredithR05} an interpretation of the new operator is
given. It turns out that there are several possible interpretations
all enjoying the requisite algebraic properties of the operator (see
\cite{milner91polyadicpi}). We will therefore make liberal use of
$(\nu\; \vec{x})P$.

% subsection the_syntax_and_semantics_of_the_notation_system (end)   

\input{qm2pi.qmops} 

\input{qm2pi.sterngerlach} 

\input{qm2pi.metric} 

% section concurrent_process_calculi (end)

%\input{qm2pi.proofsketch}

% section proof sketch (end)

%\input{qm2pi.slviaknots} 

% section spatial logic via knots (end)

\input{qm2pi.conclusion}

% section conclusion (end)

%\input{qm2pi.dtcodes} 

% section wiring algorithm (end)

\input{qm2pi.ack} 

% section acknowledgments (end)

\newpage


\bibliographystyle{plain}   
\bibliography{../../biblios/main.bib}

\input{qm2pi.rhodetails}

\end{document}

 

% subsection basic_interpretation (end)

%\input{qm2pi.rho.presentation} 
\subsection{The syntax and semantics of the notation system}\label{sub:the_syntax_and_semantics_of_the_notation_system} % (fold)

We now summarize a technical presentation of the calculus that
embodies our theory of dynamics. The typical presentation of such a
calculus follows the style of giving generators and relations on
them. The grammar, below, describing term constructors, freely
generates the set of processes, $\Proc$. This set is then quotiented
by a relation known as structural congruence and it is over this set
that the notion of dynamics is expressed. This presentation is
essentially that of \cite{MeredithR05} with the addition of
polyadicity and summation. For readability we have relegated some of
the technical subtleties to an appendix.

\subsubsection{Process grammar}\label{subsub:process_grammar}

\begin{mathpar}
  \inferrule* [lab=synchronization] {} {{M} \bc \pzero \;|\; x?F \;|\; x!C }
  \and
  \inferrule* [lab=abstraction] {} {{F} \bc (x)P}
  \and
  \inferrule* [lab=concretion] {} {{C} \bc \langle Q \rangle}
  \and
  \inferrule* [lab=process] {} {{P,Q} \bc M \;| \;P|Q \;|\; @{x}}
  \and
  \inferrule* [lab=name] {} {{x} \bc \quotep{P}}
\end{mathpar} 

Note that $\vec{x}$ (resp. $\vec{P}$) denotes a vector of names
(resp. processes) of length $|\vec{x}|$ (resp. $|\vec{P}|$). We adopt
the following useful abbreviations.

\begin{mathpar}
   x?(\vec{y}).P := x.(\vec{y})P \and  x\clift{\vec{P}} := x.\clift{\vec{P}}
   \and x!(y) := \lift{x}{\dropn{y}}
   \and \Pi_{i=0}^{n-1}P_i := P_0 | \ldots | P_{n-1}
\end{mathpar}

\subsubsection{Structural congruence}

\paragraph{Free and bound names and alpha-equivalence.} At the
core of structural equivalence is alpha-equivalence which identifies
process that are the same up to a change of variable. Formally, we
recognize the distinction between free and bound names. The free names
of a process, $\freenames{P}$, may be calculated recursively as
follows:

\begin{mathpar}
\freenames{\pzero} := \emptyset
  \and \\
  \freenames{x?(y).P} := \{ x \} \cup (\freenames{P} \setminus \{ y \})
  \and 
  \freenames{x!\langle P \rangle} := \{ x \} \cup \{ P \} 
  \and \\
  \freenames{P|Q} := \freenames{P} \cup \freenames{Q}
  \and \\
  \freenames{@{x}} := \{ x \}
\end{mathpar}

$\pi$
$\quotep{\pi}$

$\freenames{-} : \pi \to \mathcal{P}(\quotep{\pi})$

\begin{eqnarray*}
  \freenames{\pzero} & := & \emptyset \\
  \freenames{x?(y).P} & := & \{ x \} \cup (\freenames{P} \setminus \{ y \}) \\
  \freenames{x!\langle P \rangle} & := & \{ x \} \cup \{ P \} \\
  \freenames{P|Q} & := & \freenames{P} \cup \freenames{Q} \\
  \freenames{\dropn{x}} & := & \{ x \}
\end{eqnarray*}

The bound names of a process, $\boundnames{P}$, are those names occurring in $P$
that are not free. For example, in $x?(y).0$, the name $x$ is free, while $y$ is bound.

\begin{mathpar}
  \inferrule* [lab=monoidal-laws] {} { P|Q \equiv Q|P \and P|0 \equiv P \and P|(Q|R) \equiv (P|Q)|R }
\end{mathpar}

\begin{mathpar}
  \inferrule* [lab=alpha-equivalence] {} { (x)P \equiv (y)P\{y/x\} \and y \not\in \freenames{P} }
\end{mathpar}

\begin{definition}
Then two processes, $P,Q$, are alpha-equivalent if $P = Q\{\vec{y}/\vec{x}\}$ for
some $\vec{x} \in \boundnames{Q},\vec{y} \in \boundnames{P}$, where $Q\{\vec{y}/\vec{x}\}$
denotes the capture-avoiding substitution of $\vec{y}$ for $\vec{x}$ in $Q$.
\end{definition}

\begin{definition}
  The {\em structural congruence} \cite{SangiorgiWalker} , $\equiv$,
  between processes is the least congruence containing
  alpha-equivalence, satisfying the abelian monoid laws
  (associativity, commutativity and $\pzero$ as identity) for parallel
  composition $|$ and for summation $+$.
\end{definition}

\subsection{Name equivalence}

We take name equivalence, written $\nameeq$, to be the smallest
equivalence relation generated by the following rules.

\begin{mathpar}
\inferrule*[lab=Quote-drop]
{ }
{ \quotep{@{x}} \nameeq x }

\inferrule*[lab=Struct-equiv]
{ P \scong Q }
{ \quotep{P} \nameeq \quotep{Q} }
\end{mathpar}

The astute reader will have noticed that the mutual recursion of names
and processes imposes a mutual recursion on alpha-equivalence and
structural equivalence via name-equivalence. Fortunately, all of this
works out pleasantly and we may calculate in the natural way, free of
concern. The reader interested in the details is referred to the
appendix \ref{appendix:rho_details}.

\subsection{Substitution}

We use $\Proc$ for the set of processes, $\QProc$ for the set of
names, and $\id{\{}\vec{y} / \vec{x} \id{\}}$ to denote partial maps,
$s : \QProc \rightarrow \QProc$. A map, $s$ lifts, uniquely, to a map
on process terms, $\widehat{s} : \Proc \rightarrow \Proc$ by the
following equations.

\begin{mathpar}
  (0) \psubstp{Q}{P} := 0 \\
  (R \juxtap S) \psubstp{Q}{P}
  :=    
  (R)\psubstp{Q}{P} \juxtap (S) \psubstp{Q}{P} \\
  (x?(y).R) \psubstp{Q}{P}    
  :=    
  (x)\substp{Q}{P} (z)\concat( (R \psubstn{z}{y}) \psubstp{Q}{P} ) \\
  (\lift{x}{R}) \psubstp{Q}{P}  
  :=
  \lift{(x)\substp{Q}{P}}{ R \psubstp{Q}{P} } \\
%   (\dropn{x})  \psubstp{Q}{P}       
%   := 
%   \left\{ 
%     \begin{array}{ccc} 
%       \dropn{\quotep{Q}} & & x \nameeq \quotep{P} \\
%       \dropn{x} & & otherwise \\
%     \end{array}
%   \right. 
  (\dropn{x})  \psubstp{Q}{P}       
  := 
  \left\{ 
    \begin{array}{ccc} 
      Q & & x \nameeq \quotep{P} \\
      \dropn{x} & & otherwise \\
    \end{array}
  \right.
\end{mathpar}
 

where

\begin{eqnarray}
  (x)\id{\{} \lpquote Q \rpquote / \lpquote P \rpquote \id{\}}            = 
  \left\{ 
    \begin{array}{ccc}
      \lpquote Q \rpquote & & x \nameeq \lpquote P \rpquote \\
      x & & otherwise \\
    \end{array}
  \right. \nonumber
\end{eqnarray}

and $z$ is chosen distinct from $\quotep{P}$, $\quotep{Q}$, the free
names in $Q$, and all the names in $R$. Our $\alpha$-equivalence will
be built in the standard way from this substitution.

\begin{remark}\label{rem:no_self_referential_names}
  One consequence of these definitions is that $\forall P. \quotep{P}
  \not\in \freenames{P}$.
\end{remark}

\subsection{ Dynamic quote: an example }

Anticipating something of what's to come, consider applying the
substitution, $\widehat{\id{\{}u / z \id{\}}}$, to the following pair
of processes, $\lift{w}{y!(z)}$ and $w[ \lpquote y!(z) \rpquote ]$.

\begin{eqnarray}
	\lift{w}{y!(z)}\widehat{\id{\{}u / z \id{\}}}
		& = &
		\lift{w}{y!(u)} \nonumber\\
	w[ \lpquote y!(z) \rpquote ] \widehat{ \id{\{}u / z \id{\}} }
		& = &
		w[ \lpquote y!(z) \rpquote ] \nonumber
\end{eqnarray}

Because the body of the process between quotes is impervious to
substitution, we get radically different answers. In fact, by
examining the first process in an input context,
e.g. $x?(z).\lift{w}{y!(z)}$, we see that the process under the lift
operator may be shaped by prefixed inputs binding a name inside it. In
this sense, the lift operator will be seen as a way to dynamically
construct processes before reifying them as names.

Finally equipped with these standard features we can present the
dynamics of the calculus.

\subsubsection{Operational semantics} 

Finally, we introduce the computational dynamics. What marks these
algebras as distinct from other more traditionally studied algebraic
structures, e.g. vector spaces or polynomial rings, is the manner in
which dynamics is captured. In traditional structures, dynamics is typically
expressed through morphisms between such structures, as in linear maps
between vector spaces or morphisms between rings. In algebras
associated with the semantics of computation, the dynamics is
expressed as part of the algebraic structure itself, through a
reduction reduction relation typically denoted by $\red$. Below, we
give a recursive presentation of this relation for the calculus used
in the encoding.

$\red \subseteq \pi \times \pi$
$\red : \pi \to \mathcal{P}(\pi)$

\begin{mathpar}
  \inferrule* [lab=Comm] { \textsf{match}( x_{src}, x_{trgt} ) } { x_{trgt}?(y)P \; | \; x_{src}!\langle {Q} \rangle \red P\{\quotep{Q}/y}\} }
  \and \\
  \inferrule* [lab=Par] {{P} \red {P}'} {{{P} | {Q}} \red {{P}' | {Q}}}
  \and
  \inferrule* [lab=Equiv]{{{P} \scong {P}'} \andalso {{P}' \red {Q}'} \andalso {{Q}' \scong {Q}}}{{P} \red {Q}}
\end{mathpar}

\begin{eqnarray*}
  match_{\equiv} (\quotep{P},\quotep{Q}) & := & P \equiv Q \\
  match_{\dagger}(\quotep{P},\quotep{Q}) & := & \forall R. P|Q \red^{*} R => R \red^{*} 0 \\
  match_{K}(\quotep{P},\quotep{Q}) & := & K \mbox{ for some context } K
\end{eqnarray*}

$u?(x)P | u!\langle Q \rangle \red P\{\quotep{Q}/x\}$

%We write $\wred$ for $\red^*$, and $P\red$ if $\exists Q $ such that $ P \red Q$.
We write $P\red$ if $\exists Q $ such that $ P \red Q$ and $P\not\red$, otherwise.

\section{Replication}

As mentioned before, it is known that replication (and hence
recursion) can be implemented in a higher-order process algebra
\cite{SangiorgiWalker}. As our first example of calculation with the
machinery thus far presented we give the construction explicitly in
the {\rhoc}.

\begin{eqnarray}
	D_{x} & := & \prefix{x}{y}{(\binpar{\outputp{x}{y}}{@{y}})} \nonumber\\
	\bangp_{x}{P} & := & \binpar{{x}!\langle{\binpar{D_{x}}{P}}\rangle}{D_{x}} \nonumber
\end{eqnarray}

\begin{eqnarray}
	\bangp_{x}{P} & & \nonumber\\
	=
	& {x}!\langle{(\prefix{x}{y}{(\outputp{x}{y} | @{y})) | P}}\rangle 
	      | \prefix{x}{y}{(\outputp{x}{y} | @{y})} & \nonumber\\
	\red
	& (\outputp{x}{y} | @{y})\substn{\quotep{(\prefix{x}{y}{(@{y} | \outputp{x}{y})) | P}}}{y} & \nonumber\\
	=
	& \outputp{x}{\quotep{(\prefix{x}{y}{(\outputp{x}{y} | @{y})) | P}}}
	  | {(\prefix{x}{y}{(\outputp{x}{y} | @{y})) | P}} & \nonumber\\
	\red
	& \ldots & \nonumber\\
	\red^*
	& P | P | \ldots & \nonumber
\end{eqnarray}

Of course, this encoding, as an implementation, runs away, unfolding
$\bangp{P}$ eagerly. A lazier and more implementable replication
operator, restricted to input-guarded processes, may be obtained as follows.

\begin{eqnarray}
\bangp{\prefix{u}{v}{P}} 
	:= 
	\binpar{\lift{x}{\prefix{u}{v}{(\binpar{D(x)}{P})}}}{D(x)} \nonumber
\end{eqnarray}

\begin{remark}
  Note that the lazier definition still does not deal with summation
  or mixed summation (i.e. sums over input and output). The reader is
  invited to construct definitions of replication that deal with these
  features. 

  Further, the definitions are parameterized in a name, $x$. Can you,
  gentle reader, make a definition that eliminates this parameter and
  guarantees no accidental interaction between the replication
  machinery and the process being replicated -- i.e. no accidental
  sharing of names used by the process to get its work done and the
  name(s) used by the replication to effect copying. This latter
  revision of the definition of replication is crucial to obtaining
  the expected identity $!!P \sim !P$.
\end{remark}

\begin{remark}\label{rem:paradoxical_combinator}
  The reader familiar with the lambda calculus will have noticed the
  similarity between $D$ and the paradoxical combinator.

  [Ed. note: the existence of this seems to suggest we have to be more
  restrictive on the set of processes and names we admit if we are to
  support no-cloning.]
\end{remark}

\subsubsection{Bisimulation}

The computational dynamics gives rise to another kind of equivalence,
the equivalence of computational behavior. As previously mentioned
this is typically captured \emph{via} some form of bisimulation.

% The notion we use in this paper is weak barbed bisimulation
% \cite{milner91polyadicpi}.

The notion we use in this paper is derived from weak barbed
bisimulation \cite{milner91polyadicpi}. 

\begin{definition}
An \emph{observation relation}, $\downarrow_{\mathcal N}$, over a set
of names, $\mathcal N$, is the smallest relation satisfying the rules
below.

\infrule[Out-barb]{y \in {\mathcal N}, \; x \nameeq y}
		  {\outputp{x}{v} \downarrow_{\mathcal N} x}
\infrule[Par-barb]{\mbox{$P\downarrow_{\mathcal N} x$ or $Q\downarrow_{\mathcal N} x$}}
		  {\binpar{P}{Q} \downarrow_{\mathcal N} x}

We write $P \Downarrow_{\mathcal N} x$ if there is $Q$ such that 
$P \wred Q$ and $Q \downarrow_{\mathcal N} x$.
\end{definition}

\begin{definition}
%\label{def.bbisim}
An  ${\mathcal N}$-\emph{barbed bisimulation} over a set of names, ${\mathcal N}$, is a symmetric binary relation 
${\mathcal S}_{\mathcal N}$ between agents such that $P\rel{S}_{\mathcal N}Q$ implies:
\begin{enumerate}
\item If $P \red P'$ then $Q \wred Q'$ and $P'\rel{S}_{\mathcal N} Q'$.
\item If $P\downarrow_{\mathcal N} x$, then $Q\Downarrow_{\mathcal N} x$.
\end{enumerate}
$P$ is ${\mathcal N}$-barbed bisimilar to $Q$, written
$P \wbbisim_{\mathcal N} Q$, if $P \rel{S}_{\mathcal N} Q$ for some ${\mathcal N}$-barbed bisimulation ${\mathcal S}_{\mathcal N}$.
\end{definition}

$\mathcal{R} \subseteq \pi \times \pi$

$P \mathcal{R} Q => \forall P'. P \red P' \Rightarrow \exists Q'. Q \red Q', P' \mathcal{R} Q'$

$P \vdash x \Rightarrow Q \vdash x$

\begin{mathpar}
  \inferrule*[lab=Out-barb]{x \nameeq y}{{y}!\langle{Q}\rangle \vdash x}
  \and
  \inferrule*[lab=Par-barb]{\mbox{$P\vdash x$ or $Q\vdash x$}}{\binpar{P}{Q} \vdash x}
\end{mathpar}

\subsubsection{Contexts}

One of the principle advantages of computational calculi like the
$\pi$-calculus is a well-defined notion of context,
contextual-equivalence and a correlation between
contextual-equivalence and notions of bisimulation. The notion of
context allows the decomposition of a process into (sub-)process and
its syntactic environment, its context. Thus, a context may be
thought of as a process with a ``hole'' (written $\Box$) in it. The
application of a context $M$ to a process $P$, written $M[P]$, is
tantamount to filling the hole in $M$ with $P$. In this paper we do
not need the full weight of this theory, but do make use of the notion
of context in the proof the main theorem. 

\begin{mathpar}
  \inferrule* [lab=summation] {} {{M_{M},M_{N}} \bc \Box \;|\; x.M_{A} \;|\; M_{M}+M_{N}}
  \and
  \inferrule* [lab=agent] {} {{M_{A}} \bc (\vec{x})M_{P} \;| \; \clift{P_0,\ldots,M_{P},\ldots,P_N}}
  \and \\
  \inferrule* [lab=process] {} {{M_{P}} \bc M_{N} \;| \;P|M_{P} }
\end{mathpar} 

\begin{mathpar}
  \inferrule* [lab=sychronization] {} {M_{N} \bc \Box \;|\; x?M_{F} \;|\; x!M_{C}}
  \and
  \inferrule* [lab=abstraction] {} {{M_{F}} \bc (x)M_{P} }
  \and
  \inferrule* [lab=concretion] {} {{M_{C}} \bc \langle M_{P} \rangle }
  \and \\
  \inferrule* [lab=process] {} {{M_{P}} \bc M_{N} \;| \;P|M_{P} }
\end{mathpar}

\begin{definition}[contextual application] Given a context $M$, and
  process $P$, we define the \emph{contextual application}, $M[P] :=
  M\{P/\Box\}$. That is, the contextual application of M to P is the
  substitution of $P$ for $\Box$ in $M$.
\end{definition}

$\meaningof{-} : L \to \mathcal{P}(\pi)$

\begin{mathpar}
  \inferrule* [lab=collection] {} {\meaningof{true} = \pi, \and \meaningof{~E} = \pi \setminus \meaningof{E}, \and \meaningof{E_{1} \& E_{2}} = \meaningof{E_{1}} \cap \meaningof{E_{2}}}
\end{mathpar}

\begin{mathpar}
  \inferrule* [lab=structure] {} {\meaningof{0} = \{ P \in \pi | P \equiv 0 \}, \and \\ \meaningof{E_1 | E_2} = \{ P \in \pi | P \equiv P_{1} | P_{2}, P_{1} \in \meaningof{E_{1}}, P_{2} \in \meaningof{E_2}\} }
\end{mathpar}

\begin{mathpar}
 \inferrule* [lab=behavior] {} {\meaningof{\langle a?b \rangle E} = \{ P \in \pi | P \equiv Q | u?(y)P', \\ \and \\\\ \and \\ \;\;\; u \in \meaningof{a}, \forall z.P'\{z/y\} \in \meaningof{E\{z/b\}}\}, \and \\ \meaningof{a!E} = \{ P \in \pi | P \equiv Q | x!\langle P' \rangle, x \in \meaningof{a} P' \in \meaningof{E}\} }
\end{mathpar}

\begin{mathpar}
 \inferrule* [lab=nominal] {} {\meaningof{\quotep{E}} = \{ \quotep{P} \in \quotep{\pi} | P \in \meaningof{E} \}, \and \meaningof{\quotep{P}} = \{ \quotep{Q} \in \quotep{\pi} | P \equiv Q \} \and \\ \meaningof{@\quotep{E}} = \{ P \in \pi | P \equiv @x, x \in \meaningof{E} \}}
\end{mathpar}

\begin{eqnarray*}
  \\
  \meaningof{-} : TS \to ST
\end{eqnarray*}

\begin{eqnarray*}
  \\
  L : TS \to ST
\end{eqnarray*}

\begin{eqnarray*}
  \\
  P \models E \iff P \in \meaningof{E}
\end{eqnarray*}

\begin{eqnarray*}
  P \approx_{L} Q \iff \forall E \in L. P \models E \iff Q \models E
\end{eqnarray*}

\begin{eqnarray*}
  P \approx_{K} Q
\end{eqnarray*}

\begin{eqnarray*}
  P \approx Q
\end{eqnarray*}

$\approx_{K} = \approx = \approx_{L}$

\subsubsection{Contextual duality}

Note that contexts extend the quotation operation to a family of
operations from processes to names. Given a context, $M$, we can
define a \emph{nominal context}, $\quotep{M}$ by $\quotep{M}[P] :=
\quotep{M[P]}$. To foreshadow what is to come we observe that these
operations enjoy a duality with processes very much like the duality
between vectors and maps from vectors to scalars.

Further, because the calculus is essentially higher-order, we have a
correspondence between contexts and processes. More specifically,
given a name $x$ and a context $M$ we can construct $M^{*}_{x}$ such
that 

\begin{mathpar}
  M^{*}_{x} | \lift{x}{P} \red M[P]
\end{mathpar}

namely,

\begin{mathpar}
  M^{*}_{x} := x?(u).M[\dropn{u}]
\end{mathpar}

The dependence of $M^{*}_{x}$ on a name makes it an abstraction, 

\begin{mathpar}
  M^{*} := (x)x?(u).M[\dropn{u}]
\end{mathpar}

\subsection{Additional notation}

It will sometimes be convenient to denote the process a name
quotes. We already have the notation $x = \quotep{P}$, but it will be
convenient to introduce an alternate notation, $\procn{x}$, when we
want to emphasize the connection to the use of the name. Note that, by
virtue of name equivalence, $\quotep{\procn{x}} \nameeq x$; so, the
notation is consistent with previous definitions.

Further, because names have structure it is possible to effect
substitutions on the basis of that structure. This means we need to
upgrade our notation for substitutions, which we accomplish by
adapting comprehension notation. Thus,

\begin{mathpar}
  P\{ y / x : x \in S \}
\end{mathpar}

is interpreted to mean the process derived from P by replacing (in a
capture-avoiding manner) each occurrence of $x$ in $S$ by $y$. For example,

\begin{mathpar}
  P\{ \quotep{\procn{x}|\procn{x}} / x : x \in \freenames{P} \}
\end{mathpar}

will replace each (occurrence) of a free name $x$ in $P$ by
$\quotep{\procn{x}|\procn{x}}$.

Also, we will avail ourselves of the notation $x^{L}$ and $x^{R}$ to
denote injections of a name into disjoint copies of the name
space. There are numerous ways to accomplish this. One example can be
found in \cite{MeredithR05}. This notation overloads to vectors of
names: $\vec{x}^{\pi} := (x_{i}^{\pi} \; : \; 0 \leq i < |\vec{x}| )$ where $\pi \in \{L,R\}$.

We also use $P^{\Box} := P|\Box$.

In \cite{MeredithR05} an interpretation of the new operator is
given. It turns out that there are several possible interpretations
all enjoying the requisite algebraic properties of the operator (see
\cite{milner91polyadicpi}). We will therefore make liberal use of
$(\nu\; \vec{x})P$.

% subsection the_syntax_and_semantics_of_the_notation_system (end)   

\section{Interpretation of QM}
\subsection{Supporting definitions}
\subsubsection{Multiplication}
\begin{mathpar}
  \quotep{Q} \cdot \quotep{R} := \quotep{Q|R}
  \and \\
  \quotep{Q} \cdot P := P\{ \quotep{Q|R} / \quotep{R} : \quotep{R} \in \freenames{P} \}
\end{mathpar}

\paragraph{Discussion}
The first line needs little explanation. The second line says that
each free name of the process is replaced with the multiplication of
that name by the scalar. Multiplication of a scalar (name) by a state
(process) results in a process all the names of which have been `moved
over' by parallel composition with the process the scalar
quotes. There is a subtlety that the bound names have to be
manipulated so that multiplied names aren't accidentally
captured. There are many ways to achieve this.

\begin{remark}\label{rem:multiplication_identities}
  The reader is invited to verify that for all $x,y,z \in \QProc$ and $P \in \Proc$
  \begin{mathpar}
    x \cdot \quotep{0} \equiv x 
    \and
    x \cdot y \equiv y \cdot x
    \and
    x \cdot (y \cdot z) \equiv (x \cdot y) \cdot z
    \and \\
    \quotep{0} \cdot P \equiv P
    \and \\
    x \cdot (y \cdot P) \equiv (x \cdot y) \cdot P
    \and \\
    x \cdot (P|Q) \equiv (x \cdot P) | (x \cdot Q)
    \and \\    
  \end{mathpar}
\end{remark}

\subsubsection{Tensor product}

We define a tensor product on processes by structural induction.

\paragraph{Tensor of sums} First note that all summations, including
$\pzero$ and sequence, can be written $\Sigma_{i} x_{i}.A_{i} +
\Sigma_{j} x_{j}.C_{j}$, where we have grouped input-guarded processes
together and output-guarded processes together.

Thus, we can define the tensor product of two summations, $N_{1}\otimes N_{2}$, where

\begin{mathpar}
  N_{1} := \Sigma_{i} x_{i}.A_{i} + \Sigma_{j} x_{j}.C_{j}
  \and
  N_{2} := \Sigma_{i'} y_{i'}.B_{i'} + \Sigma_{j'} y_{j'}.D_{j'} 
\end{mathpar}

as follows.

\begin{mathpar}
  \Sigma_{i} x_{i}.A_{i} + \Sigma_{j} x_{j}.C_{j} \otimes \Sigma_{i'}
  y_{i'}.B_{i'} + \Sigma_{j'} y_{j'}.D_{j'} 
  \and \\
  := \; \Sigma_{i} \Sigma_{i'} \quotep{\stackrel{\vee}{x_{i}}| \stackrel{\vee}{y_{i'}}}.(A_{i}\otimes B_{i'}) \; | \; \Sigma_{i'} \Sigma_{i} \quotep{\stackrel{\vee}{y_{i'}}|\stackrel{\vee}{x_{i}}}.(B_{i'}\otimes A_{i})
  \and
  \;\; | \;\; \Sigma_{j} \Sigma_{j'} \quotep{\stackrel{\vee}{x_{j}}|\stackrel{\vee}{y_{j'}}}.(A_{j}\otimes B_{j'}) \; | \; \Sigma_{j'} \Sigma_{j} \quotep{\stackrel{\vee}{y_{j'}}|\stackrel{\vee}{x_{j}}}.(B_{j'}\otimes A_{j})
\end{mathpar}

\begin{remark}
  Do we need to $x^{L}$ and $y^{R}$ for this construction as well?
\end{remark}

\paragraph{Tensor of parallel compositions} Next, we distribute tensor
over par.

\begin{mathpar}
  P_{1}|P_{2} \otimes Q_{1}|Q_{2} := (P_{1} \otimes Q_{1}) | (P_{1}
  \otimes Q_{2}) | (P_{2} \otimes Q_{1}) | (P_{2} \otimes Q_{2})
\end{mathpar}

\paragraph{Tensor with dropped names} We treat tensor of a
process with a dropped name as parallel composition.

\begin{mathpar}
  P \otimes \dropn{x} := P | \dropn{x}
\end{mathpar}

\paragraph{Tensor of agents}

Finally, we need to define tensor on agents. Note that the definition
of tensor on normal products only tensors inputs with inputs and
outputs with outputs. Thus, we only have to define the operation on
``homogeneous'' pairings.

\begin{mathpar}
  (\vec{x})P \otimes (\vec{y})Q
  \and \\
  := (x_{0}^{L}|y_{0}^{R},\ldots,x_{0}^{L}|y_{n}^{R},\ldots,x_{m}^{L}|y_{0}^{R},\ldots,x_{m}^{L}|y_{n}^R)(P\{ \vec{x}^{L}/\vec{x}\} \otimes Q \{ \vec{y}^{R}/\vec{y}\})
  \and \\
  \clift{\vec{P}} \otimes \clift{\vec{Q}}
  \and \\
  := \clift{P_{0}\otimes Q_{0},\ldots,P_{0}\otimes Q_{n},\ldots,P_{m}\otimes Q_{0},\ldots,P_{m}\otimes Q_{n}}
\end{mathpar}

\begin{remark}
  Observe that arities of tensored abstractions matches arities of
  tensored concretions if the original arities matched. Note also that
  the length of the arities corresponds to the increase in dimension
  we see in ordinary vector space tensor product.
\end{remark}

\begin{remark}
  Operationally, this definition distributes the tensor down to
  components ``linked'' by summation. Tensor over summation is
  intriguing in that it mixes names. Moreover, as a consequence of the
  way it mixes names we have the identities for all $x \in \QProc$ and
  $P,Q \in \Proc$

  \begin{mathpar}
    (x \cdot P) \otimes Q \equiv x \cdot (P \otimes Q) \equiv P \otimes (x \cdot Q)
    \and
    P \otimes \pzero \equiv P
  \end{mathpar}

  that the reader is invited to verify.
\end{remark}

\subsubsection{Annihilation}
\begin{mathpar}
  P^{\perp} := \{ Q | \forall R. P|Q \red^{*} R \Rightarrow R \red^{*} \pzero \}
  \and \\
  P^{\underline{\perp}} := \Sigma_{Q \in P^{\perp}} \quotep{Q}?(y).(\dropn{y}|Q) | \Sigma_{Q \in P^{\perp}} \quotep{Q}\clift{\Box}
\end{mathpar}

\paragraph{Discussion} The reader will note that $P^{\perp}$ is a
\emph{set} of processes, while $P^{\underline{\perp}}$ is a
\emph{context}. We call the set $P^{\perp}$ the \emph{annihilators} of
$P$. The parallel composition of a process in the annihilators of $P$
with $P$ will result in a process, the state space of which has all
paths eventually leading to $\pzero$. Execution may endure loops; but
under reasonable conditions of fairness (naturally guaranteed under
most notions of bisimulation) such a composite process cannot get
stuck in such a loop and will, eventually pop out and terminate.

The context $P^{\underline{\perp}}$ is ready and willing to ``take the
$P$ out of'' the process to which it is applied. It will effectively
transmit the code of the process to which it is applied to one of the
annihilators and run the process against it.

\subsubsection{Evaluation}
We fix $M$ a domain of fully abstract interpretation with an equality
coincident with bisimulation. We take $\meaningof{\cdot} : \Proc \to
M$ to be the map interpreting processes and $\nmeaningof{\cdot} : \M
\to Proc$ to be the map running the other way. Then we define

\begin{mathpar}
  \int P := \nmeaningof{\meaningof{P}}
\end{mathpar}

\paragraph{Discussion}
There are many fully abstract interpretations of Milner's
$\pi$-calculus. Any of them can be used as a basis for interpreting
the reflective calculus here. Equipped with such a domain it is
largely a matter of grinding through to check that the Yoneda
construction for the normalization-by-evaluation program can be
extended to this setting.

\begin{remark}
  The reader is invited to verify that $\int (P^{\underline{\perp}}[P]) = 0$.
\end{remark}

\subsection{Quantum mechanics}

Table \ref{tbl:core_qm_op_defns} gives the core operational definitions

\begin{table}[htp]\label{tbl:core_qm_op_defns}
  \center{
    \fbox{
      \begin{tabular}{c|c}
        quantum mechanics & process calculus \\
        \hline
        scalar & $x := \quotep{P}$ \\
        state vector & $\state{P} := P$ \\
        dual & $\state{P}^{*} := \event{P^{\underline{\perp}}} := \quotep{P^{\underline{\perp}}}[-]$ \\
        matrix & $ \Sigma_{\alpha} \state{P_{\alpha}}x_{\alpha}\event{Q_{\alpha}}$ \\
        vector addition & $\state{P} + \state{Q} := \state{P | Q}$ \\
        tensor product & $\state{P} \otimes \state{Q} := \state{P \otimes Q}$ \\
        inner product & $\innerprod{P}{Q} := \quotep{\int P^{\underline{\perp}}[Q]}$ \\
      \end{tabular}
    }
  }
  \caption{QM - operational definitions}
\end{table}

where

\begin{mathpar}
  \prmatrix{P}{Q} := \fprmatrix{P}{\quotep{\pzero}}{Q}
  \and
  \fprmatrix{P}{x}{Q} := (\state{P},x,\event{Q})
  \and
  (\fprmatrix{P}{x}{Q})(\state{R}) := x \cdot \innerprod{Q}{R} \cdot \state{P}
  \and
  (\fprmatrix{P}{x}{Q})(\event{R}) := x \cdot \innerprod{R}{P} \cdot \event{Q}
\end{mathpar}

\paragraph{Discussion}
As promised: vectors (aka states) are represented as processes; duals
as contextual duals; inner product definition should be compared with
standard inner product definition for ....

\begin{remark}
  Assuming $\int (P^{\underline{\perp}}[P]) = 0$, the reader is
  invited to verify that $(\fprmatrix{P}{x}{P})(\state{P}) = x \cdot \state{P}$.
\end{remark}

\begin{remark}
  The reader is invited to verify that $\innerprod{P}{Q}$ could
  equally well have been written $\quotep{\int \stackrel{\vee}{x}}$
  where $x = \event{P^{\underline{\perp}}}(Q)$.

  One of the motivations for this remark is that there is another way
  to factor these operations. We could package up evaluation in the dual:

  \begin{mathpar}
    \state{P}^{*} := \event{\int P^{\underline{\perp}}} := \quotep{\int P^{\underline{\perp}}}[-]
  \end{mathpar}

  and then have inner product defined by
  
  \begin{mathpar}
    \innerprod{P}{Q} := \event{P}(Q)
  \end{mathpar}

  Hopefully, experience with the calculations will provide guidance on
  the best factoring.
\end{remark}

\begin{remark}
  Assuming $\int (P^{\underline{\perp}}[P]) = 0$, the reader is
  invited to verify that $\forall P,Q. (\prmatrix{0}{Q})(\state{0}) =
  \state{0}$ and dually $(\prmatrix{P}{0})(\event{0}) = \event{0}$.
\end{remark}

\begin{remark}
  i'm a little worried that i don't (yet) have proper support for
  complex conjugacy. But, the observation above may give us a
  clue. According to Abramsky, it must be the case that the scalars
  are iso to the homset of the identity for the tensor -- which the
  observation above characterizes. 

  For now, we will simply bookmark the notion with $\overline{x}$.
\end{remark}

\subsubsection{Adjointness}

We need to give a definition of $(\cdot)^{\dagger}$ for matrices. The
obvious candidate definition is
\begin{mathpar}
(\Sigma_{\alpha}\fprmatrix{P_{\alpha}}{x_{\alpha}}{Q_{\alpha}})^{\dagger}
= \Sigma_{\alpha}\fprmatrix{(Q_{\alpha}^{\underline{\perp}})^{*}}{\overline{x}_{\alpha}}{P_{\alpha}^{\underline{\perp}}} 
\end{mathpar}

But, $(Q_{\alpha}^{\underline{\perp}})^{*}$ requires a name along
which to communicate the process to achieve the context application.

\subsubsection{Basis for a basis}
If processes label states and ``addition'' of states (a.k.a. vector
addition) is interpreted as parallel composition, what corresponds to
notions of linear independence and basis? Here, we recall that Yoshida
has developed a set of \emph{combinators} for an asynchronous verison
of Milner's $\pi$-calculus. These are a finite set of processes such
any process can be expressed as parallel composition of these
combinators together with liberal uses of the new operator and
replication. We can simply give a translation of these into the
present calculus and have reasonable expectation that the property
carries over. That is, that the resultant set allows to express all
processes via parallel composition. Note, however, that there is no
new operator or replication in this calculus. As a result, we expect
that the corresponding set is actually infinite. That is, we expect
that the space is actually infinite dimensional.

\begin{remark}
  The attentive reader may be a bit concerned. Certainly, the
  collection $S$, $K$ and $I$ is a finite set of
  combinators. Shouldn't we expect to see a finite set of combinators
  for an effectively equivalent system? i am very sympathetic to this
  critique and feel it warrants full attention. On the other hand, i
  also have in mind the following analogy. The natural numbers, as a
  monoid under addition, has exactly $1$ generator, while the natural
  numbers, as a monoid under multiplication, has countably many
  generators (the primes). We observe that the application of the
  lambda calculus is much less resource sensitive than the parallel
  composition of the $\pi$-calculus. Could it be the case that we have
  an analogy of the form
  
  \begin{mathpar}
    m + n : MN :: m*n : M|N
  \end{mathpar}

  giving a similar blow up in the set of ``primes''?  This is such a
  wonderful thought that, even if it's not true, i think it's worth
  writing down.
\end{remark}
 

\documentclass[12pt]{llncs}
%\documentclass{jktr}

\usepackage[pdftex]{hyperref}                   
\usepackage {listings}
\usepackage {mathpartir}
\usepackage{bcprules}
%\usepackage{listings}
                       
\usepackage{graphicx} 
%\usepackage[margins=2.5cm,nohead,nofoot]{geometry}
%\usepackage{geometry}
\usepackage{amsfonts}
\usepackage{amstext}
\usepackage{latexsym}
\usepackage{amssymb}
\usepackage{color}


%\include{myPreamble}
\include{qm2pi.local} 

%\ifpdf
%\usepackage[pdftex]{graphicx}
%\else
%\usepackage{graphicx}
%\fi

 % \ifpdf
%  \usepackage{pdfsync}
%  \if


%\title{Brief Article}
%\author{David F. Snyder}
%\author{L.G. Meredith}

%\address{Dept. of Math., Texas State University--San Marcos, San Marcos, TX 78666}
       
\pagestyle{empty}


\begin{document}

\lstset{language=[Objective]Caml,frame=shadowbox}

\input{qm2pi.front}

% section front matter (end)

\input{qm2pi.intro} 
 
% section introduction (end)

% \input{qm2pi.knotations} 

% section notation (end)

\input{qm2pi.process.calculi} 

% section concurrent_process_calculi_and_spatial_logics_ (end)
    
%\input{qm2pi.knots2pi} 

%\input{qm2pi.trefoil} 

%\input{qm2pi.mainthm} 

% subsection basic_interpretation (end)

%\input{qm2pi.rho.presentation} 
\subsection{The syntax and semantics of the notation system}\label{sub:the_syntax_and_semantics_of_the_notation_system} % (fold)

We now summarize a technical presentation of the calculus that
embodies our theory of dynamics. The typical presentation of such a
calculus follows the style of giving generators and relations on
them. The grammar, below, describing term constructors, freely
generates the set of processes, $\Proc$. This set is then quotiented
by a relation known as structural congruence and it is over this set
that the notion of dynamics is expressed. This presentation is
essentially that of \cite{MeredithR05} with the addition of
polyadicity and summation. For readability we have relegated some of
the technical subtleties to an appendix.

\subsubsection{Process grammar}\label{subsub:process_grammar}

\begin{mathpar}
  \inferrule* [lab=synchronization] {} {{M} \bc \pzero \;|\; x?F \;|\; x!C }
  \and
  \inferrule* [lab=abstraction] {} {{F} \bc (x)P}
  \and
  \inferrule* [lab=concretion] {} {{C} \bc \langle Q \rangle}
  \and
  \inferrule* [lab=process] {} {{P,Q} \bc M \;| \;P|Q \;|\; @{x}}
  \and
  \inferrule* [lab=name] {} {{x} \bc \quotep{P}}
\end{mathpar} 

Note that $\vec{x}$ (resp. $\vec{P}$) denotes a vector of names
(resp. processes) of length $|\vec{x}|$ (resp. $|\vec{P}|$). We adopt
the following useful abbreviations.

\begin{mathpar}
   x?(\vec{y}).P := x.(\vec{y})P \and  x\clift{\vec{P}} := x.\clift{\vec{P}}
   \and x!(y) := \lift{x}{\dropn{y}}
   \and \Pi_{i=0}^{n-1}P_i := P_0 | \ldots | P_{n-1}
\end{mathpar}

\subsubsection{Structural congruence}

\paragraph{Free and bound names and alpha-equivalence.} At the
core of structural equivalence is alpha-equivalence which identifies
process that are the same up to a change of variable. Formally, we
recognize the distinction between free and bound names. The free names
of a process, $\freenames{P}$, may be calculated recursively as
follows:

\begin{mathpar}
\freenames{\pzero} := \emptyset
  \and \\
  \freenames{x?(y).P} := \{ x \} \cup (\freenames{P} \setminus \{ y \})
  \and 
  \freenames{x!\langle P \rangle} := \{ x \} \cup \{ P \} 
  \and \\
  \freenames{P|Q} := \freenames{P} \cup \freenames{Q}
  \and \\
  \freenames{@{x}} := \{ x \}
\end{mathpar}

$\pi$
$\quotep{\pi}$

$\freenames{-} : \pi \to \mathcal{P}(\quotep{\pi})$

\begin{eqnarray*}
  \freenames{\pzero} & := & \emptyset \\
  \freenames{x?(y).P} & := & \{ x \} \cup (\freenames{P} \setminus \{ y \}) \\
  \freenames{x!\langle P \rangle} & := & \{ x \} \cup \{ P \} \\
  \freenames{P|Q} & := & \freenames{P} \cup \freenames{Q} \\
  \freenames{\dropn{x}} & := & \{ x \}
\end{eqnarray*}

The bound names of a process, $\boundnames{P}$, are those names occurring in $P$
that are not free. For example, in $x?(y).0$, the name $x$ is free, while $y$ is bound.

\begin{mathpar}
  \inferrule* [lab=monoidal-laws] {} { P|Q \equiv Q|P \and P|0 \equiv P \and P|(Q|R) \equiv (P|Q)|R }
\end{mathpar}

\begin{mathpar}
  \inferrule* [lab=alpha-equivalence] {} { (x)P \equiv (y)P\{y/x\} \and y \not\in \freenames{P} }
\end{mathpar}

\begin{definition}
Then two processes, $P,Q$, are alpha-equivalent if $P = Q\{\vec{y}/\vec{x}\}$ for
some $\vec{x} \in \boundnames{Q},\vec{y} \in \boundnames{P}$, where $Q\{\vec{y}/\vec{x}\}$
denotes the capture-avoiding substitution of $\vec{y}$ for $\vec{x}$ in $Q$.
\end{definition}

\begin{definition}
  The {\em structural congruence} \cite{SangiorgiWalker} , $\equiv$,
  between processes is the least congruence containing
  alpha-equivalence, satisfying the abelian monoid laws
  (associativity, commutativity and $\pzero$ as identity) for parallel
  composition $|$ and for summation $+$.
\end{definition}

\subsection{Name equivalence}

We take name equivalence, written $\nameeq$, to be the smallest
equivalence relation generated by the following rules.

\begin{mathpar}
\inferrule*[lab=Quote-drop]
{ }
{ \quotep{@{x}} \nameeq x }

\inferrule*[lab=Struct-equiv]
{ P \scong Q }
{ \quotep{P} \nameeq \quotep{Q} }
\end{mathpar}

The astute reader will have noticed that the mutual recursion of names
and processes imposes a mutual recursion on alpha-equivalence and
structural equivalence via name-equivalence. Fortunately, all of this
works out pleasantly and we may calculate in the natural way, free of
concern. The reader interested in the details is referred to the
appendix \ref{appendix:rho_details}.

\subsection{Substitution}

We use $\Proc$ for the set of processes, $\QProc$ for the set of
names, and $\id{\{}\vec{y} / \vec{x} \id{\}}$ to denote partial maps,
$s : \QProc \rightarrow \QProc$. A map, $s$ lifts, uniquely, to a map
on process terms, $\widehat{s} : \Proc \rightarrow \Proc$ by the
following equations.

\begin{mathpar}
  (0) \psubstp{Q}{P} := 0 \\
  (R \juxtap S) \psubstp{Q}{P}
  :=    
  (R)\psubstp{Q}{P} \juxtap (S) \psubstp{Q}{P} \\
  (x?(y).R) \psubstp{Q}{P}    
  :=    
  (x)\substp{Q}{P} (z)\concat( (R \psubstn{z}{y}) \psubstp{Q}{P} ) \\
  (\lift{x}{R}) \psubstp{Q}{P}  
  :=
  \lift{(x)\substp{Q}{P}}{ R \psubstp{Q}{P} } \\
%   (\dropn{x})  \psubstp{Q}{P}       
%   := 
%   \left\{ 
%     \begin{array}{ccc} 
%       \dropn{\quotep{Q}} & & x \nameeq \quotep{P} \\
%       \dropn{x} & & otherwise \\
%     \end{array}
%   \right. 
  (\dropn{x})  \psubstp{Q}{P}       
  := 
  \left\{ 
    \begin{array}{ccc} 
      Q & & x \nameeq \quotep{P} \\
      \dropn{x} & & otherwise \\
    \end{array}
  \right.
\end{mathpar}
 

where

\begin{eqnarray}
  (x)\id{\{} \lpquote Q \rpquote / \lpquote P \rpquote \id{\}}            = 
  \left\{ 
    \begin{array}{ccc}
      \lpquote Q \rpquote & & x \nameeq \lpquote P \rpquote \\
      x & & otherwise \\
    \end{array}
  \right. \nonumber
\end{eqnarray}

and $z$ is chosen distinct from $\quotep{P}$, $\quotep{Q}$, the free
names in $Q$, and all the names in $R$. Our $\alpha$-equivalence will
be built in the standard way from this substitution.

\begin{remark}\label{rem:no_self_referential_names}
  One consequence of these definitions is that $\forall P. \quotep{P}
  \not\in \freenames{P}$.
\end{remark}

\subsection{ Dynamic quote: an example }

Anticipating something of what's to come, consider applying the
substitution, $\widehat{\id{\{}u / z \id{\}}}$, to the following pair
of processes, $\lift{w}{y!(z)}$ and $w[ \lpquote y!(z) \rpquote ]$.

\begin{eqnarray}
	\lift{w}{y!(z)}\widehat{\id{\{}u / z \id{\}}}
		& = &
		\lift{w}{y!(u)} \nonumber\\
	w[ \lpquote y!(z) \rpquote ] \widehat{ \id{\{}u / z \id{\}} }
		& = &
		w[ \lpquote y!(z) \rpquote ] \nonumber
\end{eqnarray}

Because the body of the process between quotes is impervious to
substitution, we get radically different answers. In fact, by
examining the first process in an input context,
e.g. $x?(z).\lift{w}{y!(z)}$, we see that the process under the lift
operator may be shaped by prefixed inputs binding a name inside it. In
this sense, the lift operator will be seen as a way to dynamically
construct processes before reifying them as names.

Finally equipped with these standard features we can present the
dynamics of the calculus.

\subsubsection{Operational semantics} 

Finally, we introduce the computational dynamics. What marks these
algebras as distinct from other more traditionally studied algebraic
structures, e.g. vector spaces or polynomial rings, is the manner in
which dynamics is captured. In traditional structures, dynamics is typically
expressed through morphisms between such structures, as in linear maps
between vector spaces or morphisms between rings. In algebras
associated with the semantics of computation, the dynamics is
expressed as part of the algebraic structure itself, through a
reduction reduction relation typically denoted by $\red$. Below, we
give a recursive presentation of this relation for the calculus used
in the encoding.

$\red \subseteq \pi \times \pi$
$\red : \pi \to \mathcal{P}(\pi)$

\begin{mathpar}
  \inferrule* [lab=Comm] { \textsf{match}( x_{src}, x_{trgt} ) } { x_{trgt}?(y)P \; | \; x_{src}!\langle {Q} \rangle \red P\{\quotep{Q}/y}\} }
  \and \\
  \inferrule* [lab=Par] {{P} \red {P}'} {{{P} | {Q}} \red {{P}' | {Q}}}
  \and
  \inferrule* [lab=Equiv]{{{P} \scong {P}'} \andalso {{P}' \red {Q}'} \andalso {{Q}' \scong {Q}}}{{P} \red {Q}}
\end{mathpar}

\begin{eqnarray*}
  match_{\equiv} (\quotep{P},\quotep{Q}) & := & P \equiv Q \\
  match_{\dagger}(\quotep{P},\quotep{Q}) & := & \forall R. P|Q \red^{*} R => R \red^{*} 0 \\
  match_{K}(\quotep{P},\quotep{Q}) & := & K \mbox{ for some context } K
\end{eqnarray*}

$u?(x)P | u!\langle Q \rangle \red P\{\quotep{Q}/x\}$

%We write $\wred$ for $\red^*$, and $P\red$ if $\exists Q $ such that $ P \red Q$.
We write $P\red$ if $\exists Q $ such that $ P \red Q$ and $P\not\red$, otherwise.

\section{Replication}

As mentioned before, it is known that replication (and hence
recursion) can be implemented in a higher-order process algebra
\cite{SangiorgiWalker}. As our first example of calculation with the
machinery thus far presented we give the construction explicitly in
the {\rhoc}.

\begin{eqnarray}
	D_{x} & := & \prefix{x}{y}{(\binpar{\outputp{x}{y}}{@{y}})} \nonumber\\
	\bangp_{x}{P} & := & \binpar{{x}!\langle{\binpar{D_{x}}{P}}\rangle}{D_{x}} \nonumber
\end{eqnarray}

\begin{eqnarray}
	\bangp_{x}{P} & & \nonumber\\
	=
	& {x}!\langle{(\prefix{x}{y}{(\outputp{x}{y} | @{y})) | P}}\rangle 
	      | \prefix{x}{y}{(\outputp{x}{y} | @{y})} & \nonumber\\
	\red
	& (\outputp{x}{y} | @{y})\substn{\quotep{(\prefix{x}{y}{(@{y} | \outputp{x}{y})) | P}}}{y} & \nonumber\\
	=
	& \outputp{x}{\quotep{(\prefix{x}{y}{(\outputp{x}{y} | @{y})) | P}}}
	  | {(\prefix{x}{y}{(\outputp{x}{y} | @{y})) | P}} & \nonumber\\
	\red
	& \ldots & \nonumber\\
	\red^*
	& P | P | \ldots & \nonumber
\end{eqnarray}

Of course, this encoding, as an implementation, runs away, unfolding
$\bangp{P}$ eagerly. A lazier and more implementable replication
operator, restricted to input-guarded processes, may be obtained as follows.

\begin{eqnarray}
\bangp{\prefix{u}{v}{P}} 
	:= 
	\binpar{\lift{x}{\prefix{u}{v}{(\binpar{D(x)}{P})}}}{D(x)} \nonumber
\end{eqnarray}

\begin{remark}
  Note that the lazier definition still does not deal with summation
  or mixed summation (i.e. sums over input and output). The reader is
  invited to construct definitions of replication that deal with these
  features. 

  Further, the definitions are parameterized in a name, $x$. Can you,
  gentle reader, make a definition that eliminates this parameter and
  guarantees no accidental interaction between the replication
  machinery and the process being replicated -- i.e. no accidental
  sharing of names used by the process to get its work done and the
  name(s) used by the replication to effect copying. This latter
  revision of the definition of replication is crucial to obtaining
  the expected identity $!!P \sim !P$.
\end{remark}

\begin{remark}\label{rem:paradoxical_combinator}
  The reader familiar with the lambda calculus will have noticed the
  similarity between $D$ and the paradoxical combinator.

  [Ed. note: the existence of this seems to suggest we have to be more
  restrictive on the set of processes and names we admit if we are to
  support no-cloning.]
\end{remark}

\subsubsection{Bisimulation}

The computational dynamics gives rise to another kind of equivalence,
the equivalence of computational behavior. As previously mentioned
this is typically captured \emph{via} some form of bisimulation.

% The notion we use in this paper is weak barbed bisimulation
% \cite{milner91polyadicpi}.

The notion we use in this paper is derived from weak barbed
bisimulation \cite{milner91polyadicpi}. 

\begin{definition}
An \emph{observation relation}, $\downarrow_{\mathcal N}$, over a set
of names, $\mathcal N$, is the smallest relation satisfying the rules
below.

\infrule[Out-barb]{y \in {\mathcal N}, \; x \nameeq y}
		  {\outputp{x}{v} \downarrow_{\mathcal N} x}
\infrule[Par-barb]{\mbox{$P\downarrow_{\mathcal N} x$ or $Q\downarrow_{\mathcal N} x$}}
		  {\binpar{P}{Q} \downarrow_{\mathcal N} x}

We write $P \Downarrow_{\mathcal N} x$ if there is $Q$ such that 
$P \wred Q$ and $Q \downarrow_{\mathcal N} x$.
\end{definition}

\begin{definition}
%\label{def.bbisim}
An  ${\mathcal N}$-\emph{barbed bisimulation} over a set of names, ${\mathcal N}$, is a symmetric binary relation 
${\mathcal S}_{\mathcal N}$ between agents such that $P\rel{S}_{\mathcal N}Q$ implies:
\begin{enumerate}
\item If $P \red P'$ then $Q \wred Q'$ and $P'\rel{S}_{\mathcal N} Q'$.
\item If $P\downarrow_{\mathcal N} x$, then $Q\Downarrow_{\mathcal N} x$.
\end{enumerate}
$P$ is ${\mathcal N}$-barbed bisimilar to $Q$, written
$P \wbbisim_{\mathcal N} Q$, if $P \rel{S}_{\mathcal N} Q$ for some ${\mathcal N}$-barbed bisimulation ${\mathcal S}_{\mathcal N}$.
\end{definition}

$\mathcal{R} \subseteq \pi \times \pi$

$P \mathcal{R} Q => \forall P'. P \red P' \Rightarrow \exists Q'. Q \red Q', P' \mathcal{R} Q'$

$P \vdash x \Rightarrow Q \vdash x$

\begin{mathpar}
  \inferrule*[lab=Out-barb]{x \nameeq y}{{y}!\langle{Q}\rangle \vdash x}
  \and
  \inferrule*[lab=Par-barb]{\mbox{$P\vdash x$ or $Q\vdash x$}}{\binpar{P}{Q} \vdash x}
\end{mathpar}

\subsubsection{Contexts}

One of the principle advantages of computational calculi like the
$\pi$-calculus is a well-defined notion of context,
contextual-equivalence and a correlation between
contextual-equivalence and notions of bisimulation. The notion of
context allows the decomposition of a process into (sub-)process and
its syntactic environment, its context. Thus, a context may be
thought of as a process with a ``hole'' (written $\Box$) in it. The
application of a context $M$ to a process $P$, written $M[P]$, is
tantamount to filling the hole in $M$ with $P$. In this paper we do
not need the full weight of this theory, but do make use of the notion
of context in the proof the main theorem. 

\begin{mathpar}
  \inferrule* [lab=summation] {} {{M_{M},M_{N}} \bc \Box \;|\; x.M_{A} \;|\; M_{M}+M_{N}}
  \and
  \inferrule* [lab=agent] {} {{M_{A}} \bc (\vec{x})M_{P} \;| \; \clift{P_0,\ldots,M_{P},\ldots,P_N}}
  \and \\
  \inferrule* [lab=process] {} {{M_{P}} \bc M_{N} \;| \;P|M_{P} }
\end{mathpar} 

\begin{mathpar}
  \inferrule* [lab=sychronization] {} {M_{N} \bc \Box \;|\; x?M_{F} \;|\; x!M_{C}}
  \and
  \inferrule* [lab=abstraction] {} {{M_{F}} \bc (x)M_{P} }
  \and
  \inferrule* [lab=concretion] {} {{M_{C}} \bc \langle M_{P} \rangle }
  \and \\
  \inferrule* [lab=process] {} {{M_{P}} \bc M_{N} \;| \;P|M_{P} }
\end{mathpar}

\begin{definition}[contextual application] Given a context $M$, and
  process $P$, we define the \emph{contextual application}, $M[P] :=
  M\{P/\Box\}$. That is, the contextual application of M to P is the
  substitution of $P$ for $\Box$ in $M$.
\end{definition}

$\meaningof{-} : L \to \mathcal{P}(\pi)$

\begin{mathpar}
  \inferrule* [lab=collection] {} {\meaningof{true} = \pi, \and \meaningof{~E} = \pi \setminus \meaningof{E}, \and \meaningof{E_{1} \& E_{2}} = \meaningof{E_{1}} \cap \meaningof{E_{2}}}
\end{mathpar}

\begin{mathpar}
  \inferrule* [lab=structure] {} {\meaningof{0} = \{ P \in \pi | P \equiv 0 \}, \and \\ \meaningof{E_1 | E_2} = \{ P \in \pi | P \equiv P_{1} | P_{2}, P_{1} \in \meaningof{E_{1}}, P_{2} \in \meaningof{E_2}\} }
\end{mathpar}

\begin{mathpar}
 \inferrule* [lab=behavior] {} {\meaningof{\langle a?b \rangle E} = \{ P \in \pi | P \equiv Q | u?(y)P', \\ \and \\\\ \and \\ \;\;\; u \in \meaningof{a}, \forall z.P'\{z/y\} \in \meaningof{E\{z/b\}}\}, \and \\ \meaningof{a!E} = \{ P \in \pi | P \equiv Q | x!\langle P' \rangle, x \in \meaningof{a} P' \in \meaningof{E}\} }
\end{mathpar}

\begin{mathpar}
 \inferrule* [lab=nominal] {} {\meaningof{\quotep{E}} = \{ \quotep{P} \in \quotep{\pi} | P \in \meaningof{E} \}, \and \meaningof{\quotep{P}} = \{ \quotep{Q} \in \quotep{\pi} | P \equiv Q \} \and \\ \meaningof{@\quotep{E}} = \{ P \in \pi | P \equiv @x, x \in \meaningof{E} \}}
\end{mathpar}

\begin{eqnarray*}
  \\
  \meaningof{-} : TS \to ST
\end{eqnarray*}

\begin{eqnarray*}
  \\
  L : TS \to ST
\end{eqnarray*}

\begin{eqnarray*}
  \\
  P \models E \iff P \in \meaningof{E}
\end{eqnarray*}

\begin{eqnarray*}
  P \approx_{L} Q \iff \forall E \in L. P \models E \iff Q \models E
\end{eqnarray*}

\begin{eqnarray*}
  P \approx_{K} Q
\end{eqnarray*}

\begin{eqnarray*}
  P \approx Q
\end{eqnarray*}

$\approx_{K} = \approx = \approx_{L}$

\subsubsection{Contextual duality}

Note that contexts extend the quotation operation to a family of
operations from processes to names. Given a context, $M$, we can
define a \emph{nominal context}, $\quotep{M}$ by $\quotep{M}[P] :=
\quotep{M[P]}$. To foreshadow what is to come we observe that these
operations enjoy a duality with processes very much like the duality
between vectors and maps from vectors to scalars.

Further, because the calculus is essentially higher-order, we have a
correspondence between contexts and processes. More specifically,
given a name $x$ and a context $M$ we can construct $M^{*}_{x}$ such
that 

\begin{mathpar}
  M^{*}_{x} | \lift{x}{P} \red M[P]
\end{mathpar}

namely,

\begin{mathpar}
  M^{*}_{x} := x?(u).M[\dropn{u}]
\end{mathpar}

The dependence of $M^{*}_{x}$ on a name makes it an abstraction, 

\begin{mathpar}
  M^{*} := (x)x?(u).M[\dropn{u}]
\end{mathpar}

\subsection{Additional notation}

It will sometimes be convenient to denote the process a name
quotes. We already have the notation $x = \quotep{P}$, but it will be
convenient to introduce an alternate notation, $\procn{x}$, when we
want to emphasize the connection to the use of the name. Note that, by
virtue of name equivalence, $\quotep{\procn{x}} \nameeq x$; so, the
notation is consistent with previous definitions.

Further, because names have structure it is possible to effect
substitutions on the basis of that structure. This means we need to
upgrade our notation for substitutions, which we accomplish by
adapting comprehension notation. Thus,

\begin{mathpar}
  P\{ y / x : x \in S \}
\end{mathpar}

is interpreted to mean the process derived from P by replacing (in a
capture-avoiding manner) each occurrence of $x$ in $S$ by $y$. For example,

\begin{mathpar}
  P\{ \quotep{\procn{x}|\procn{x}} / x : x \in \freenames{P} \}
\end{mathpar}

will replace each (occurrence) of a free name $x$ in $P$ by
$\quotep{\procn{x}|\procn{x}}$.

Also, we will avail ourselves of the notation $x^{L}$ and $x^{R}$ to
denote injections of a name into disjoint copies of the name
space. There are numerous ways to accomplish this. One example can be
found in \cite{MeredithR05}. This notation overloads to vectors of
names: $\vec{x}^{\pi} := (x_{i}^{\pi} \; : \; 0 \leq i < |\vec{x}| )$ where $\pi \in \{L,R\}$.

We also use $P^{\Box} := P|\Box$.

In \cite{MeredithR05} an interpretation of the new operator is
given. It turns out that there are several possible interpretations
all enjoying the requisite algebraic properties of the operator (see
\cite{milner91polyadicpi}). We will therefore make liberal use of
$(\nu\; \vec{x})P$.

% subsection the_syntax_and_semantics_of_the_notation_system (end)   

\input{qm2pi.qmops} 

\input{qm2pi.sterngerlach} 

\input{qm2pi.metric} 

% section concurrent_process_calculi (end)

%\input{qm2pi.proofsketch}

% section proof sketch (end)

%\input{qm2pi.slviaknots} 

% section spatial logic via knots (end)

\input{qm2pi.conclusion}

% section conclusion (end)

%\input{qm2pi.dtcodes} 

% section wiring algorithm (end)

\input{qm2pi.ack} 

% section acknowledgments (end)

\newpage


\bibliographystyle{plain}   
\bibliography{../../biblios/main.bib}

\input{qm2pi.rhodetails}

\end{document}

 

\documentclass[12pt]{llncs}
%\documentclass{jktr}

\usepackage[pdftex]{hyperref}                   
\usepackage {listings}
\usepackage {mathpartir}
\usepackage{bcprules}
%\usepackage{listings}
                       
\usepackage{graphicx} 
%\usepackage[margins=2.5cm,nohead,nofoot]{geometry}
%\usepackage{geometry}
\usepackage{amsfonts}
\usepackage{amstext}
\usepackage{latexsym}
\usepackage{amssymb}
\usepackage{color}


%\include{myPreamble}
\include{qm2pi.local} 

%\ifpdf
%\usepackage[pdftex]{graphicx}
%\else
%\usepackage{graphicx}
%\fi

 % \ifpdf
%  \usepackage{pdfsync}
%  \if


%\title{Brief Article}
%\author{David F. Snyder}
%\author{L.G. Meredith}

%\address{Dept. of Math., Texas State University--San Marcos, San Marcos, TX 78666}
       
\pagestyle{empty}


\begin{document}

\lstset{language=[Objective]Caml,frame=shadowbox}

\input{qm2pi.front}

% section front matter (end)

\input{qm2pi.intro} 
 
% section introduction (end)

% \input{qm2pi.knotations} 

% section notation (end)

\input{qm2pi.process.calculi} 

% section concurrent_process_calculi_and_spatial_logics_ (end)
    
%\input{qm2pi.knots2pi} 

%\input{qm2pi.trefoil} 

%\input{qm2pi.mainthm} 

% subsection basic_interpretation (end)

%\input{qm2pi.rho.presentation} 
\subsection{The syntax and semantics of the notation system}\label{sub:the_syntax_and_semantics_of_the_notation_system} % (fold)

We now summarize a technical presentation of the calculus that
embodies our theory of dynamics. The typical presentation of such a
calculus follows the style of giving generators and relations on
them. The grammar, below, describing term constructors, freely
generates the set of processes, $\Proc$. This set is then quotiented
by a relation known as structural congruence and it is over this set
that the notion of dynamics is expressed. This presentation is
essentially that of \cite{MeredithR05} with the addition of
polyadicity and summation. For readability we have relegated some of
the technical subtleties to an appendix.

\subsubsection{Process grammar}\label{subsub:process_grammar}

\begin{mathpar}
  \inferrule* [lab=synchronization] {} {{M} \bc \pzero \;|\; x?F \;|\; x!C }
  \and
  \inferrule* [lab=abstraction] {} {{F} \bc (x)P}
  \and
  \inferrule* [lab=concretion] {} {{C} \bc \langle Q \rangle}
  \and
  \inferrule* [lab=process] {} {{P,Q} \bc M \;| \;P|Q \;|\; @{x}}
  \and
  \inferrule* [lab=name] {} {{x} \bc \quotep{P}}
\end{mathpar} 

Note that $\vec{x}$ (resp. $\vec{P}$) denotes a vector of names
(resp. processes) of length $|\vec{x}|$ (resp. $|\vec{P}|$). We adopt
the following useful abbreviations.

\begin{mathpar}
   x?(\vec{y}).P := x.(\vec{y})P \and  x\clift{\vec{P}} := x.\clift{\vec{P}}
   \and x!(y) := \lift{x}{\dropn{y}}
   \and \Pi_{i=0}^{n-1}P_i := P_0 | \ldots | P_{n-1}
\end{mathpar}

\subsubsection{Structural congruence}

\paragraph{Free and bound names and alpha-equivalence.} At the
core of structural equivalence is alpha-equivalence which identifies
process that are the same up to a change of variable. Formally, we
recognize the distinction between free and bound names. The free names
of a process, $\freenames{P}$, may be calculated recursively as
follows:

\begin{mathpar}
\freenames{\pzero} := \emptyset
  \and \\
  \freenames{x?(y).P} := \{ x \} \cup (\freenames{P} \setminus \{ y \})
  \and 
  \freenames{x!\langle P \rangle} := \{ x \} \cup \{ P \} 
  \and \\
  \freenames{P|Q} := \freenames{P} \cup \freenames{Q}
  \and \\
  \freenames{@{x}} := \{ x \}
\end{mathpar}

$\pi$
$\quotep{\pi}$

$\freenames{-} : \pi \to \mathcal{P}(\quotep{\pi})$

\begin{eqnarray*}
  \freenames{\pzero} & := & \emptyset \\
  \freenames{x?(y).P} & := & \{ x \} \cup (\freenames{P} \setminus \{ y \}) \\
  \freenames{x!\langle P \rangle} & := & \{ x \} \cup \{ P \} \\
  \freenames{P|Q} & := & \freenames{P} \cup \freenames{Q} \\
  \freenames{\dropn{x}} & := & \{ x \}
\end{eqnarray*}

The bound names of a process, $\boundnames{P}$, are those names occurring in $P$
that are not free. For example, in $x?(y).0$, the name $x$ is free, while $y$ is bound.

\begin{mathpar}
  \inferrule* [lab=monoidal-laws] {} { P|Q \equiv Q|P \and P|0 \equiv P \and P|(Q|R) \equiv (P|Q)|R }
\end{mathpar}

\begin{mathpar}
  \inferrule* [lab=alpha-equivalence] {} { (x)P \equiv (y)P\{y/x\} \and y \not\in \freenames{P} }
\end{mathpar}

\begin{definition}
Then two processes, $P,Q$, are alpha-equivalent if $P = Q\{\vec{y}/\vec{x}\}$ for
some $\vec{x} \in \boundnames{Q},\vec{y} \in \boundnames{P}$, where $Q\{\vec{y}/\vec{x}\}$
denotes the capture-avoiding substitution of $\vec{y}$ for $\vec{x}$ in $Q$.
\end{definition}

\begin{definition}
  The {\em structural congruence} \cite{SangiorgiWalker} , $\equiv$,
  between processes is the least congruence containing
  alpha-equivalence, satisfying the abelian monoid laws
  (associativity, commutativity and $\pzero$ as identity) for parallel
  composition $|$ and for summation $+$.
\end{definition}

\subsection{Name equivalence}

We take name equivalence, written $\nameeq$, to be the smallest
equivalence relation generated by the following rules.

\begin{mathpar}
\inferrule*[lab=Quote-drop]
{ }
{ \quotep{@{x}} \nameeq x }

\inferrule*[lab=Struct-equiv]
{ P \scong Q }
{ \quotep{P} \nameeq \quotep{Q} }
\end{mathpar}

The astute reader will have noticed that the mutual recursion of names
and processes imposes a mutual recursion on alpha-equivalence and
structural equivalence via name-equivalence. Fortunately, all of this
works out pleasantly and we may calculate in the natural way, free of
concern. The reader interested in the details is referred to the
appendix \ref{appendix:rho_details}.

\subsection{Substitution}

We use $\Proc$ for the set of processes, $\QProc$ for the set of
names, and $\id{\{}\vec{y} / \vec{x} \id{\}}$ to denote partial maps,
$s : \QProc \rightarrow \QProc$. A map, $s$ lifts, uniquely, to a map
on process terms, $\widehat{s} : \Proc \rightarrow \Proc$ by the
following equations.

\begin{mathpar}
  (0) \psubstp{Q}{P} := 0 \\
  (R \juxtap S) \psubstp{Q}{P}
  :=    
  (R)\psubstp{Q}{P} \juxtap (S) \psubstp{Q}{P} \\
  (x?(y).R) \psubstp{Q}{P}    
  :=    
  (x)\substp{Q}{P} (z)\concat( (R \psubstn{z}{y}) \psubstp{Q}{P} ) \\
  (\lift{x}{R}) \psubstp{Q}{P}  
  :=
  \lift{(x)\substp{Q}{P}}{ R \psubstp{Q}{P} } \\
%   (\dropn{x})  \psubstp{Q}{P}       
%   := 
%   \left\{ 
%     \begin{array}{ccc} 
%       \dropn{\quotep{Q}} & & x \nameeq \quotep{P} \\
%       \dropn{x} & & otherwise \\
%     \end{array}
%   \right. 
  (\dropn{x})  \psubstp{Q}{P}       
  := 
  \left\{ 
    \begin{array}{ccc} 
      Q & & x \nameeq \quotep{P} \\
      \dropn{x} & & otherwise \\
    \end{array}
  \right.
\end{mathpar}
 

where

\begin{eqnarray}
  (x)\id{\{} \lpquote Q \rpquote / \lpquote P \rpquote \id{\}}            = 
  \left\{ 
    \begin{array}{ccc}
      \lpquote Q \rpquote & & x \nameeq \lpquote P \rpquote \\
      x & & otherwise \\
    \end{array}
  \right. \nonumber
\end{eqnarray}

and $z$ is chosen distinct from $\quotep{P}$, $\quotep{Q}$, the free
names in $Q$, and all the names in $R$. Our $\alpha$-equivalence will
be built in the standard way from this substitution.

\begin{remark}\label{rem:no_self_referential_names}
  One consequence of these definitions is that $\forall P. \quotep{P}
  \not\in \freenames{P}$.
\end{remark}

\subsection{ Dynamic quote: an example }

Anticipating something of what's to come, consider applying the
substitution, $\widehat{\id{\{}u / z \id{\}}}$, to the following pair
of processes, $\lift{w}{y!(z)}$ and $w[ \lpquote y!(z) \rpquote ]$.

\begin{eqnarray}
	\lift{w}{y!(z)}\widehat{\id{\{}u / z \id{\}}}
		& = &
		\lift{w}{y!(u)} \nonumber\\
	w[ \lpquote y!(z) \rpquote ] \widehat{ \id{\{}u / z \id{\}} }
		& = &
		w[ \lpquote y!(z) \rpquote ] \nonumber
\end{eqnarray}

Because the body of the process between quotes is impervious to
substitution, we get radically different answers. In fact, by
examining the first process in an input context,
e.g. $x?(z).\lift{w}{y!(z)}$, we see that the process under the lift
operator may be shaped by prefixed inputs binding a name inside it. In
this sense, the lift operator will be seen as a way to dynamically
construct processes before reifying them as names.

Finally equipped with these standard features we can present the
dynamics of the calculus.

\subsubsection{Operational semantics} 

Finally, we introduce the computational dynamics. What marks these
algebras as distinct from other more traditionally studied algebraic
structures, e.g. vector spaces or polynomial rings, is the manner in
which dynamics is captured. In traditional structures, dynamics is typically
expressed through morphisms between such structures, as in linear maps
between vector spaces or morphisms between rings. In algebras
associated with the semantics of computation, the dynamics is
expressed as part of the algebraic structure itself, through a
reduction reduction relation typically denoted by $\red$. Below, we
give a recursive presentation of this relation for the calculus used
in the encoding.

$\red \subseteq \pi \times \pi$
$\red : \pi \to \mathcal{P}(\pi)$

\begin{mathpar}
  \inferrule* [lab=Comm] { \textsf{match}( x_{src}, x_{trgt} ) } { x_{trgt}?(y)P \; | \; x_{src}!\langle {Q} \rangle \red P\{\quotep{Q}/y}\} }
  \and \\
  \inferrule* [lab=Par] {{P} \red {P}'} {{{P} | {Q}} \red {{P}' | {Q}}}
  \and
  \inferrule* [lab=Equiv]{{{P} \scong {P}'} \andalso {{P}' \red {Q}'} \andalso {{Q}' \scong {Q}}}{{P} \red {Q}}
\end{mathpar}

\begin{eqnarray*}
  match_{\equiv} (\quotep{P},\quotep{Q}) & := & P \equiv Q \\
  match_{\dagger}(\quotep{P},\quotep{Q}) & := & \forall R. P|Q \red^{*} R => R \red^{*} 0 \\
  match_{K}(\quotep{P},\quotep{Q}) & := & K \mbox{ for some context } K
\end{eqnarray*}

$u?(x)P | u!\langle Q \rangle \red P\{\quotep{Q}/x\}$

%We write $\wred$ for $\red^*$, and $P\red$ if $\exists Q $ such that $ P \red Q$.
We write $P\red$ if $\exists Q $ such that $ P \red Q$ and $P\not\red$, otherwise.

\section{Replication}

As mentioned before, it is known that replication (and hence
recursion) can be implemented in a higher-order process algebra
\cite{SangiorgiWalker}. As our first example of calculation with the
machinery thus far presented we give the construction explicitly in
the {\rhoc}.

\begin{eqnarray}
	D_{x} & := & \prefix{x}{y}{(\binpar{\outputp{x}{y}}{@{y}})} \nonumber\\
	\bangp_{x}{P} & := & \binpar{{x}!\langle{\binpar{D_{x}}{P}}\rangle}{D_{x}} \nonumber
\end{eqnarray}

\begin{eqnarray}
	\bangp_{x}{P} & & \nonumber\\
	=
	& {x}!\langle{(\prefix{x}{y}{(\outputp{x}{y} | @{y})) | P}}\rangle 
	      | \prefix{x}{y}{(\outputp{x}{y} | @{y})} & \nonumber\\
	\red
	& (\outputp{x}{y} | @{y})\substn{\quotep{(\prefix{x}{y}{(@{y} | \outputp{x}{y})) | P}}}{y} & \nonumber\\
	=
	& \outputp{x}{\quotep{(\prefix{x}{y}{(\outputp{x}{y} | @{y})) | P}}}
	  | {(\prefix{x}{y}{(\outputp{x}{y} | @{y})) | P}} & \nonumber\\
	\red
	& \ldots & \nonumber\\
	\red^*
	& P | P | \ldots & \nonumber
\end{eqnarray}

Of course, this encoding, as an implementation, runs away, unfolding
$\bangp{P}$ eagerly. A lazier and more implementable replication
operator, restricted to input-guarded processes, may be obtained as follows.

\begin{eqnarray}
\bangp{\prefix{u}{v}{P}} 
	:= 
	\binpar{\lift{x}{\prefix{u}{v}{(\binpar{D(x)}{P})}}}{D(x)} \nonumber
\end{eqnarray}

\begin{remark}
  Note that the lazier definition still does not deal with summation
  or mixed summation (i.e. sums over input and output). The reader is
  invited to construct definitions of replication that deal with these
  features. 

  Further, the definitions are parameterized in a name, $x$. Can you,
  gentle reader, make a definition that eliminates this parameter and
  guarantees no accidental interaction between the replication
  machinery and the process being replicated -- i.e. no accidental
  sharing of names used by the process to get its work done and the
  name(s) used by the replication to effect copying. This latter
  revision of the definition of replication is crucial to obtaining
  the expected identity $!!P \sim !P$.
\end{remark}

\begin{remark}\label{rem:paradoxical_combinator}
  The reader familiar with the lambda calculus will have noticed the
  similarity between $D$ and the paradoxical combinator.

  [Ed. note: the existence of this seems to suggest we have to be more
  restrictive on the set of processes and names we admit if we are to
  support no-cloning.]
\end{remark}

\subsubsection{Bisimulation}

The computational dynamics gives rise to another kind of equivalence,
the equivalence of computational behavior. As previously mentioned
this is typically captured \emph{via} some form of bisimulation.

% The notion we use in this paper is weak barbed bisimulation
% \cite{milner91polyadicpi}.

The notion we use in this paper is derived from weak barbed
bisimulation \cite{milner91polyadicpi}. 

\begin{definition}
An \emph{observation relation}, $\downarrow_{\mathcal N}$, over a set
of names, $\mathcal N$, is the smallest relation satisfying the rules
below.

\infrule[Out-barb]{y \in {\mathcal N}, \; x \nameeq y}
		  {\outputp{x}{v} \downarrow_{\mathcal N} x}
\infrule[Par-barb]{\mbox{$P\downarrow_{\mathcal N} x$ or $Q\downarrow_{\mathcal N} x$}}
		  {\binpar{P}{Q} \downarrow_{\mathcal N} x}

We write $P \Downarrow_{\mathcal N} x$ if there is $Q$ such that 
$P \wred Q$ and $Q \downarrow_{\mathcal N} x$.
\end{definition}

\begin{definition}
%\label{def.bbisim}
An  ${\mathcal N}$-\emph{barbed bisimulation} over a set of names, ${\mathcal N}$, is a symmetric binary relation 
${\mathcal S}_{\mathcal N}$ between agents such that $P\rel{S}_{\mathcal N}Q$ implies:
\begin{enumerate}
\item If $P \red P'$ then $Q \wred Q'$ and $P'\rel{S}_{\mathcal N} Q'$.
\item If $P\downarrow_{\mathcal N} x$, then $Q\Downarrow_{\mathcal N} x$.
\end{enumerate}
$P$ is ${\mathcal N}$-barbed bisimilar to $Q$, written
$P \wbbisim_{\mathcal N} Q$, if $P \rel{S}_{\mathcal N} Q$ for some ${\mathcal N}$-barbed bisimulation ${\mathcal S}_{\mathcal N}$.
\end{definition}

$\mathcal{R} \subseteq \pi \times \pi$

$P \mathcal{R} Q => \forall P'. P \red P' \Rightarrow \exists Q'. Q \red Q', P' \mathcal{R} Q'$

$P \vdash x \Rightarrow Q \vdash x$

\begin{mathpar}
  \inferrule*[lab=Out-barb]{x \nameeq y}{{y}!\langle{Q}\rangle \vdash x}
  \and
  \inferrule*[lab=Par-barb]{\mbox{$P\vdash x$ or $Q\vdash x$}}{\binpar{P}{Q} \vdash x}
\end{mathpar}

\subsubsection{Contexts}

One of the principle advantages of computational calculi like the
$\pi$-calculus is a well-defined notion of context,
contextual-equivalence and a correlation between
contextual-equivalence and notions of bisimulation. The notion of
context allows the decomposition of a process into (sub-)process and
its syntactic environment, its context. Thus, a context may be
thought of as a process with a ``hole'' (written $\Box$) in it. The
application of a context $M$ to a process $P$, written $M[P]$, is
tantamount to filling the hole in $M$ with $P$. In this paper we do
not need the full weight of this theory, but do make use of the notion
of context in the proof the main theorem. 

\begin{mathpar}
  \inferrule* [lab=summation] {} {{M_{M},M_{N}} \bc \Box \;|\; x.M_{A} \;|\; M_{M}+M_{N}}
  \and
  \inferrule* [lab=agent] {} {{M_{A}} \bc (\vec{x})M_{P} \;| \; \clift{P_0,\ldots,M_{P},\ldots,P_N}}
  \and \\
  \inferrule* [lab=process] {} {{M_{P}} \bc M_{N} \;| \;P|M_{P} }
\end{mathpar} 

\begin{mathpar}
  \inferrule* [lab=sychronization] {} {M_{N} \bc \Box \;|\; x?M_{F} \;|\; x!M_{C}}
  \and
  \inferrule* [lab=abstraction] {} {{M_{F}} \bc (x)M_{P} }
  \and
  \inferrule* [lab=concretion] {} {{M_{C}} \bc \langle M_{P} \rangle }
  \and \\
  \inferrule* [lab=process] {} {{M_{P}} \bc M_{N} \;| \;P|M_{P} }
\end{mathpar}

\begin{definition}[contextual application] Given a context $M$, and
  process $P$, we define the \emph{contextual application}, $M[P] :=
  M\{P/\Box\}$. That is, the contextual application of M to P is the
  substitution of $P$ for $\Box$ in $M$.
\end{definition}

$\meaningof{-} : L \to \mathcal{P}(\pi)$

\begin{mathpar}
  \inferrule* [lab=collection] {} {\meaningof{true} = \pi, \and \meaningof{~E} = \pi \setminus \meaningof{E}, \and \meaningof{E_{1} \& E_{2}} = \meaningof{E_{1}} \cap \meaningof{E_{2}}}
\end{mathpar}

\begin{mathpar}
  \inferrule* [lab=structure] {} {\meaningof{0} = \{ P \in \pi | P \equiv 0 \}, \and \\ \meaningof{E_1 | E_2} = \{ P \in \pi | P \equiv P_{1} | P_{2}, P_{1} \in \meaningof{E_{1}}, P_{2} \in \meaningof{E_2}\} }
\end{mathpar}

\begin{mathpar}
 \inferrule* [lab=behavior] {} {\meaningof{\langle a?b \rangle E} = \{ P \in \pi | P \equiv Q | u?(y)P', \\ \and \\\\ \and \\ \;\;\; u \in \meaningof{a}, \forall z.P'\{z/y\} \in \meaningof{E\{z/b\}}\}, \and \\ \meaningof{a!E} = \{ P \in \pi | P \equiv Q | x!\langle P' \rangle, x \in \meaningof{a} P' \in \meaningof{E}\} }
\end{mathpar}

\begin{mathpar}
 \inferrule* [lab=nominal] {} {\meaningof{\quotep{E}} = \{ \quotep{P} \in \quotep{\pi} | P \in \meaningof{E} \}, \and \meaningof{\quotep{P}} = \{ \quotep{Q} \in \quotep{\pi} | P \equiv Q \} \and \\ \meaningof{@\quotep{E}} = \{ P \in \pi | P \equiv @x, x \in \meaningof{E} \}}
\end{mathpar}

\begin{eqnarray*}
  \\
  \meaningof{-} : TS \to ST
\end{eqnarray*}

\begin{eqnarray*}
  \\
  L : TS \to ST
\end{eqnarray*}

\begin{eqnarray*}
  \\
  P \models E \iff P \in \meaningof{E}
\end{eqnarray*}

\begin{eqnarray*}
  P \approx_{L} Q \iff \forall E \in L. P \models E \iff Q \models E
\end{eqnarray*}

\begin{eqnarray*}
  P \approx_{K} Q
\end{eqnarray*}

\begin{eqnarray*}
  P \approx Q
\end{eqnarray*}

$\approx_{K} = \approx = \approx_{L}$

\subsubsection{Contextual duality}

Note that contexts extend the quotation operation to a family of
operations from processes to names. Given a context, $M$, we can
define a \emph{nominal context}, $\quotep{M}$ by $\quotep{M}[P] :=
\quotep{M[P]}$. To foreshadow what is to come we observe that these
operations enjoy a duality with processes very much like the duality
between vectors and maps from vectors to scalars.

Further, because the calculus is essentially higher-order, we have a
correspondence between contexts and processes. More specifically,
given a name $x$ and a context $M$ we can construct $M^{*}_{x}$ such
that 

\begin{mathpar}
  M^{*}_{x} | \lift{x}{P} \red M[P]
\end{mathpar}

namely,

\begin{mathpar}
  M^{*}_{x} := x?(u).M[\dropn{u}]
\end{mathpar}

The dependence of $M^{*}_{x}$ on a name makes it an abstraction, 

\begin{mathpar}
  M^{*} := (x)x?(u).M[\dropn{u}]
\end{mathpar}

\subsection{Additional notation}

It will sometimes be convenient to denote the process a name
quotes. We already have the notation $x = \quotep{P}$, but it will be
convenient to introduce an alternate notation, $\procn{x}$, when we
want to emphasize the connection to the use of the name. Note that, by
virtue of name equivalence, $\quotep{\procn{x}} \nameeq x$; so, the
notation is consistent with previous definitions.

Further, because names have structure it is possible to effect
substitutions on the basis of that structure. This means we need to
upgrade our notation for substitutions, which we accomplish by
adapting comprehension notation. Thus,

\begin{mathpar}
  P\{ y / x : x \in S \}
\end{mathpar}

is interpreted to mean the process derived from P by replacing (in a
capture-avoiding manner) each occurrence of $x$ in $S$ by $y$. For example,

\begin{mathpar}
  P\{ \quotep{\procn{x}|\procn{x}} / x : x \in \freenames{P} \}
\end{mathpar}

will replace each (occurrence) of a free name $x$ in $P$ by
$\quotep{\procn{x}|\procn{x}}$.

Also, we will avail ourselves of the notation $x^{L}$ and $x^{R}$ to
denote injections of a name into disjoint copies of the name
space. There are numerous ways to accomplish this. One example can be
found in \cite{MeredithR05}. This notation overloads to vectors of
names: $\vec{x}^{\pi} := (x_{i}^{\pi} \; : \; 0 \leq i < |\vec{x}| )$ where $\pi \in \{L,R\}$.

We also use $P^{\Box} := P|\Box$.

In \cite{MeredithR05} an interpretation of the new operator is
given. It turns out that there are several possible interpretations
all enjoying the requisite algebraic properties of the operator (see
\cite{milner91polyadicpi}). We will therefore make liberal use of
$(\nu\; \vec{x})P$.

% subsection the_syntax_and_semantics_of_the_notation_system (end)   

\input{qm2pi.qmops} 

\input{qm2pi.sterngerlach} 

\input{qm2pi.metric} 

% section concurrent_process_calculi (end)

%\input{qm2pi.proofsketch}

% section proof sketch (end)

%\input{qm2pi.slviaknots} 

% section spatial logic via knots (end)

\input{qm2pi.conclusion}

% section conclusion (end)

%\input{qm2pi.dtcodes} 

% section wiring algorithm (end)

\input{qm2pi.ack} 

% section acknowledgments (end)

\newpage


\bibliographystyle{plain}   
\bibliography{../../biblios/main.bib}

\input{qm2pi.rhodetails}

\end{document}

 

% section concurrent_process_calculi (end)

%\documentclass[12pt]{llncs}
%\documentclass{jktr}

\usepackage[pdftex]{hyperref}                   
\usepackage {listings}
\usepackage {mathpartir}
\usepackage{bcprules}
%\usepackage{listings}
                       
\usepackage{graphicx} 
%\usepackage[margins=2.5cm,nohead,nofoot]{geometry}
%\usepackage{geometry}
\usepackage{amsfonts}
\usepackage{amstext}
\usepackage{latexsym}
\usepackage{amssymb}
\usepackage{color}


%\include{myPreamble}
\include{qm2pi.local} 

%\ifpdf
%\usepackage[pdftex]{graphicx}
%\else
%\usepackage{graphicx}
%\fi

 % \ifpdf
%  \usepackage{pdfsync}
%  \if


%\title{Brief Article}
%\author{David F. Snyder}
%\author{L.G. Meredith}

%\address{Dept. of Math., Texas State University--San Marcos, San Marcos, TX 78666}
       
\pagestyle{empty}


\begin{document}

\lstset{language=[Objective]Caml,frame=shadowbox}

\input{qm2pi.front}

% section front matter (end)

\input{qm2pi.intro} 
 
% section introduction (end)

% \input{qm2pi.knotations} 

% section notation (end)

\input{qm2pi.process.calculi} 

% section concurrent_process_calculi_and_spatial_logics_ (end)
    
%\input{qm2pi.knots2pi} 

%\input{qm2pi.trefoil} 

%\input{qm2pi.mainthm} 

% subsection basic_interpretation (end)

%\input{qm2pi.rho.presentation} 
\subsection{The syntax and semantics of the notation system}\label{sub:the_syntax_and_semantics_of_the_notation_system} % (fold)

We now summarize a technical presentation of the calculus that
embodies our theory of dynamics. The typical presentation of such a
calculus follows the style of giving generators and relations on
them. The grammar, below, describing term constructors, freely
generates the set of processes, $\Proc$. This set is then quotiented
by a relation known as structural congruence and it is over this set
that the notion of dynamics is expressed. This presentation is
essentially that of \cite{MeredithR05} with the addition of
polyadicity and summation. For readability we have relegated some of
the technical subtleties to an appendix.

\subsubsection{Process grammar}\label{subsub:process_grammar}

\begin{mathpar}
  \inferrule* [lab=synchronization] {} {{M} \bc \pzero \;|\; x?F \;|\; x!C }
  \and
  \inferrule* [lab=abstraction] {} {{F} \bc (x)P}
  \and
  \inferrule* [lab=concretion] {} {{C} \bc \langle Q \rangle}
  \and
  \inferrule* [lab=process] {} {{P,Q} \bc M \;| \;P|Q \;|\; @{x}}
  \and
  \inferrule* [lab=name] {} {{x} \bc \quotep{P}}
\end{mathpar} 

Note that $\vec{x}$ (resp. $\vec{P}$) denotes a vector of names
(resp. processes) of length $|\vec{x}|$ (resp. $|\vec{P}|$). We adopt
the following useful abbreviations.

\begin{mathpar}
   x?(\vec{y}).P := x.(\vec{y})P \and  x\clift{\vec{P}} := x.\clift{\vec{P}}
   \and x!(y) := \lift{x}{\dropn{y}}
   \and \Pi_{i=0}^{n-1}P_i := P_0 | \ldots | P_{n-1}
\end{mathpar}

\subsubsection{Structural congruence}

\paragraph{Free and bound names and alpha-equivalence.} At the
core of structural equivalence is alpha-equivalence which identifies
process that are the same up to a change of variable. Formally, we
recognize the distinction between free and bound names. The free names
of a process, $\freenames{P}$, may be calculated recursively as
follows:

\begin{mathpar}
\freenames{\pzero} := \emptyset
  \and \\
  \freenames{x?(y).P} := \{ x \} \cup (\freenames{P} \setminus \{ y \})
  \and 
  \freenames{x!\langle P \rangle} := \{ x \} \cup \{ P \} 
  \and \\
  \freenames{P|Q} := \freenames{P} \cup \freenames{Q}
  \and \\
  \freenames{@{x}} := \{ x \}
\end{mathpar}

$\pi$
$\quotep{\pi}$

$\freenames{-} : \pi \to \mathcal{P}(\quotep{\pi})$

\begin{eqnarray*}
  \freenames{\pzero} & := & \emptyset \\
  \freenames{x?(y).P} & := & \{ x \} \cup (\freenames{P} \setminus \{ y \}) \\
  \freenames{x!\langle P \rangle} & := & \{ x \} \cup \{ P \} \\
  \freenames{P|Q} & := & \freenames{P} \cup \freenames{Q} \\
  \freenames{\dropn{x}} & := & \{ x \}
\end{eqnarray*}

The bound names of a process, $\boundnames{P}$, are those names occurring in $P$
that are not free. For example, in $x?(y).0$, the name $x$ is free, while $y$ is bound.

\begin{mathpar}
  \inferrule* [lab=monoidal-laws] {} { P|Q \equiv Q|P \and P|0 \equiv P \and P|(Q|R) \equiv (P|Q)|R }
\end{mathpar}

\begin{mathpar}
  \inferrule* [lab=alpha-equivalence] {} { (x)P \equiv (y)P\{y/x\} \and y \not\in \freenames{P} }
\end{mathpar}

\begin{definition}
Then two processes, $P,Q$, are alpha-equivalent if $P = Q\{\vec{y}/\vec{x}\}$ for
some $\vec{x} \in \boundnames{Q},\vec{y} \in \boundnames{P}$, where $Q\{\vec{y}/\vec{x}\}$
denotes the capture-avoiding substitution of $\vec{y}$ for $\vec{x}$ in $Q$.
\end{definition}

\begin{definition}
  The {\em structural congruence} \cite{SangiorgiWalker} , $\equiv$,
  between processes is the least congruence containing
  alpha-equivalence, satisfying the abelian monoid laws
  (associativity, commutativity and $\pzero$ as identity) for parallel
  composition $|$ and for summation $+$.
\end{definition}

\subsection{Name equivalence}

We take name equivalence, written $\nameeq$, to be the smallest
equivalence relation generated by the following rules.

\begin{mathpar}
\inferrule*[lab=Quote-drop]
{ }
{ \quotep{@{x}} \nameeq x }

\inferrule*[lab=Struct-equiv]
{ P \scong Q }
{ \quotep{P} \nameeq \quotep{Q} }
\end{mathpar}

The astute reader will have noticed that the mutual recursion of names
and processes imposes a mutual recursion on alpha-equivalence and
structural equivalence via name-equivalence. Fortunately, all of this
works out pleasantly and we may calculate in the natural way, free of
concern. The reader interested in the details is referred to the
appendix \ref{appendix:rho_details}.

\subsection{Substitution}

We use $\Proc$ for the set of processes, $\QProc$ for the set of
names, and $\id{\{}\vec{y} / \vec{x} \id{\}}$ to denote partial maps,
$s : \QProc \rightarrow \QProc$. A map, $s$ lifts, uniquely, to a map
on process terms, $\widehat{s} : \Proc \rightarrow \Proc$ by the
following equations.

\begin{mathpar}
  (0) \psubstp{Q}{P} := 0 \\
  (R \juxtap S) \psubstp{Q}{P}
  :=    
  (R)\psubstp{Q}{P} \juxtap (S) \psubstp{Q}{P} \\
  (x?(y).R) \psubstp{Q}{P}    
  :=    
  (x)\substp{Q}{P} (z)\concat( (R \psubstn{z}{y}) \psubstp{Q}{P} ) \\
  (\lift{x}{R}) \psubstp{Q}{P}  
  :=
  \lift{(x)\substp{Q}{P}}{ R \psubstp{Q}{P} } \\
%   (\dropn{x})  \psubstp{Q}{P}       
%   := 
%   \left\{ 
%     \begin{array}{ccc} 
%       \dropn{\quotep{Q}} & & x \nameeq \quotep{P} \\
%       \dropn{x} & & otherwise \\
%     \end{array}
%   \right. 
  (\dropn{x})  \psubstp{Q}{P}       
  := 
  \left\{ 
    \begin{array}{ccc} 
      Q & & x \nameeq \quotep{P} \\
      \dropn{x} & & otherwise \\
    \end{array}
  \right.
\end{mathpar}
 

where

\begin{eqnarray}
  (x)\id{\{} \lpquote Q \rpquote / \lpquote P \rpquote \id{\}}            = 
  \left\{ 
    \begin{array}{ccc}
      \lpquote Q \rpquote & & x \nameeq \lpquote P \rpquote \\
      x & & otherwise \\
    \end{array}
  \right. \nonumber
\end{eqnarray}

and $z$ is chosen distinct from $\quotep{P}$, $\quotep{Q}$, the free
names in $Q$, and all the names in $R$. Our $\alpha$-equivalence will
be built in the standard way from this substitution.

\begin{remark}\label{rem:no_self_referential_names}
  One consequence of these definitions is that $\forall P. \quotep{P}
  \not\in \freenames{P}$.
\end{remark}

\subsection{ Dynamic quote: an example }

Anticipating something of what's to come, consider applying the
substitution, $\widehat{\id{\{}u / z \id{\}}}$, to the following pair
of processes, $\lift{w}{y!(z)}$ and $w[ \lpquote y!(z) \rpquote ]$.

\begin{eqnarray}
	\lift{w}{y!(z)}\widehat{\id{\{}u / z \id{\}}}
		& = &
		\lift{w}{y!(u)} \nonumber\\
	w[ \lpquote y!(z) \rpquote ] \widehat{ \id{\{}u / z \id{\}} }
		& = &
		w[ \lpquote y!(z) \rpquote ] \nonumber
\end{eqnarray}

Because the body of the process between quotes is impervious to
substitution, we get radically different answers. In fact, by
examining the first process in an input context,
e.g. $x?(z).\lift{w}{y!(z)}$, we see that the process under the lift
operator may be shaped by prefixed inputs binding a name inside it. In
this sense, the lift operator will be seen as a way to dynamically
construct processes before reifying them as names.

Finally equipped with these standard features we can present the
dynamics of the calculus.

\subsubsection{Operational semantics} 

Finally, we introduce the computational dynamics. What marks these
algebras as distinct from other more traditionally studied algebraic
structures, e.g. vector spaces or polynomial rings, is the manner in
which dynamics is captured. In traditional structures, dynamics is typically
expressed through morphisms between such structures, as in linear maps
between vector spaces or morphisms between rings. In algebras
associated with the semantics of computation, the dynamics is
expressed as part of the algebraic structure itself, through a
reduction reduction relation typically denoted by $\red$. Below, we
give a recursive presentation of this relation for the calculus used
in the encoding.

$\red \subseteq \pi \times \pi$
$\red : \pi \to \mathcal{P}(\pi)$

\begin{mathpar}
  \inferrule* [lab=Comm] { \textsf{match}( x_{src}, x_{trgt} ) } { x_{trgt}?(y)P \; | \; x_{src}!\langle {Q} \rangle \red P\{\quotep{Q}/y}\} }
  \and \\
  \inferrule* [lab=Par] {{P} \red {P}'} {{{P} | {Q}} \red {{P}' | {Q}}}
  \and
  \inferrule* [lab=Equiv]{{{P} \scong {P}'} \andalso {{P}' \red {Q}'} \andalso {{Q}' \scong {Q}}}{{P} \red {Q}}
\end{mathpar}

\begin{eqnarray*}
  match_{\equiv} (\quotep{P},\quotep{Q}) & := & P \equiv Q \\
  match_{\dagger}(\quotep{P},\quotep{Q}) & := & \forall R. P|Q \red^{*} R => R \red^{*} 0 \\
  match_{K}(\quotep{P},\quotep{Q}) & := & K \mbox{ for some context } K
\end{eqnarray*}

$u?(x)P | u!\langle Q \rangle \red P\{\quotep{Q}/x\}$

%We write $\wred$ for $\red^*$, and $P\red$ if $\exists Q $ such that $ P \red Q$.
We write $P\red$ if $\exists Q $ such that $ P \red Q$ and $P\not\red$, otherwise.

\section{Replication}

As mentioned before, it is known that replication (and hence
recursion) can be implemented in a higher-order process algebra
\cite{SangiorgiWalker}. As our first example of calculation with the
machinery thus far presented we give the construction explicitly in
the {\rhoc}.

\begin{eqnarray}
	D_{x} & := & \prefix{x}{y}{(\binpar{\outputp{x}{y}}{@{y}})} \nonumber\\
	\bangp_{x}{P} & := & \binpar{{x}!\langle{\binpar{D_{x}}{P}}\rangle}{D_{x}} \nonumber
\end{eqnarray}

\begin{eqnarray}
	\bangp_{x}{P} & & \nonumber\\
	=
	& {x}!\langle{(\prefix{x}{y}{(\outputp{x}{y} | @{y})) | P}}\rangle 
	      | \prefix{x}{y}{(\outputp{x}{y} | @{y})} & \nonumber\\
	\red
	& (\outputp{x}{y} | @{y})\substn{\quotep{(\prefix{x}{y}{(@{y} | \outputp{x}{y})) | P}}}{y} & \nonumber\\
	=
	& \outputp{x}{\quotep{(\prefix{x}{y}{(\outputp{x}{y} | @{y})) | P}}}
	  | {(\prefix{x}{y}{(\outputp{x}{y} | @{y})) | P}} & \nonumber\\
	\red
	& \ldots & \nonumber\\
	\red^*
	& P | P | \ldots & \nonumber
\end{eqnarray}

Of course, this encoding, as an implementation, runs away, unfolding
$\bangp{P}$ eagerly. A lazier and more implementable replication
operator, restricted to input-guarded processes, may be obtained as follows.

\begin{eqnarray}
\bangp{\prefix{u}{v}{P}} 
	:= 
	\binpar{\lift{x}{\prefix{u}{v}{(\binpar{D(x)}{P})}}}{D(x)} \nonumber
\end{eqnarray}

\begin{remark}
  Note that the lazier definition still does not deal with summation
  or mixed summation (i.e. sums over input and output). The reader is
  invited to construct definitions of replication that deal with these
  features. 

  Further, the definitions are parameterized in a name, $x$. Can you,
  gentle reader, make a definition that eliminates this parameter and
  guarantees no accidental interaction between the replication
  machinery and the process being replicated -- i.e. no accidental
  sharing of names used by the process to get its work done and the
  name(s) used by the replication to effect copying. This latter
  revision of the definition of replication is crucial to obtaining
  the expected identity $!!P \sim !P$.
\end{remark}

\begin{remark}\label{rem:paradoxical_combinator}
  The reader familiar with the lambda calculus will have noticed the
  similarity between $D$ and the paradoxical combinator.

  [Ed. note: the existence of this seems to suggest we have to be more
  restrictive on the set of processes and names we admit if we are to
  support no-cloning.]
\end{remark}

\subsubsection{Bisimulation}

The computational dynamics gives rise to another kind of equivalence,
the equivalence of computational behavior. As previously mentioned
this is typically captured \emph{via} some form of bisimulation.

% The notion we use in this paper is weak barbed bisimulation
% \cite{milner91polyadicpi}.

The notion we use in this paper is derived from weak barbed
bisimulation \cite{milner91polyadicpi}. 

\begin{definition}
An \emph{observation relation}, $\downarrow_{\mathcal N}$, over a set
of names, $\mathcal N$, is the smallest relation satisfying the rules
below.

\infrule[Out-barb]{y \in {\mathcal N}, \; x \nameeq y}
		  {\outputp{x}{v} \downarrow_{\mathcal N} x}
\infrule[Par-barb]{\mbox{$P\downarrow_{\mathcal N} x$ or $Q\downarrow_{\mathcal N} x$}}
		  {\binpar{P}{Q} \downarrow_{\mathcal N} x}

We write $P \Downarrow_{\mathcal N} x$ if there is $Q$ such that 
$P \wred Q$ and $Q \downarrow_{\mathcal N} x$.
\end{definition}

\begin{definition}
%\label{def.bbisim}
An  ${\mathcal N}$-\emph{barbed bisimulation} over a set of names, ${\mathcal N}$, is a symmetric binary relation 
${\mathcal S}_{\mathcal N}$ between agents such that $P\rel{S}_{\mathcal N}Q$ implies:
\begin{enumerate}
\item If $P \red P'$ then $Q \wred Q'$ and $P'\rel{S}_{\mathcal N} Q'$.
\item If $P\downarrow_{\mathcal N} x$, then $Q\Downarrow_{\mathcal N} x$.
\end{enumerate}
$P$ is ${\mathcal N}$-barbed bisimilar to $Q$, written
$P \wbbisim_{\mathcal N} Q$, if $P \rel{S}_{\mathcal N} Q$ for some ${\mathcal N}$-barbed bisimulation ${\mathcal S}_{\mathcal N}$.
\end{definition}

$\mathcal{R} \subseteq \pi \times \pi$

$P \mathcal{R} Q => \forall P'. P \red P' \Rightarrow \exists Q'. Q \red Q', P' \mathcal{R} Q'$

$P \vdash x \Rightarrow Q \vdash x$

\begin{mathpar}
  \inferrule*[lab=Out-barb]{x \nameeq y}{{y}!\langle{Q}\rangle \vdash x}
  \and
  \inferrule*[lab=Par-barb]{\mbox{$P\vdash x$ or $Q\vdash x$}}{\binpar{P}{Q} \vdash x}
\end{mathpar}

\subsubsection{Contexts}

One of the principle advantages of computational calculi like the
$\pi$-calculus is a well-defined notion of context,
contextual-equivalence and a correlation between
contextual-equivalence and notions of bisimulation. The notion of
context allows the decomposition of a process into (sub-)process and
its syntactic environment, its context. Thus, a context may be
thought of as a process with a ``hole'' (written $\Box$) in it. The
application of a context $M$ to a process $P$, written $M[P]$, is
tantamount to filling the hole in $M$ with $P$. In this paper we do
not need the full weight of this theory, but do make use of the notion
of context in the proof the main theorem. 

\begin{mathpar}
  \inferrule* [lab=summation] {} {{M_{M},M_{N}} \bc \Box \;|\; x.M_{A} \;|\; M_{M}+M_{N}}
  \and
  \inferrule* [lab=agent] {} {{M_{A}} \bc (\vec{x})M_{P} \;| \; \clift{P_0,\ldots,M_{P},\ldots,P_N}}
  \and \\
  \inferrule* [lab=process] {} {{M_{P}} \bc M_{N} \;| \;P|M_{P} }
\end{mathpar} 

\begin{mathpar}
  \inferrule* [lab=sychronization] {} {M_{N} \bc \Box \;|\; x?M_{F} \;|\; x!M_{C}}
  \and
  \inferrule* [lab=abstraction] {} {{M_{F}} \bc (x)M_{P} }
  \and
  \inferrule* [lab=concretion] {} {{M_{C}} \bc \langle M_{P} \rangle }
  \and \\
  \inferrule* [lab=process] {} {{M_{P}} \bc M_{N} \;| \;P|M_{P} }
\end{mathpar}

\begin{definition}[contextual application] Given a context $M$, and
  process $P$, we define the \emph{contextual application}, $M[P] :=
  M\{P/\Box\}$. That is, the contextual application of M to P is the
  substitution of $P$ for $\Box$ in $M$.
\end{definition}

$\meaningof{-} : L \to \mathcal{P}(\pi)$

\begin{mathpar}
  \inferrule* [lab=collection] {} {\meaningof{true} = \pi, \and \meaningof{~E} = \pi \setminus \meaningof{E}, \and \meaningof{E_{1} \& E_{2}} = \meaningof{E_{1}} \cap \meaningof{E_{2}}}
\end{mathpar}

\begin{mathpar}
  \inferrule* [lab=structure] {} {\meaningof{0} = \{ P \in \pi | P \equiv 0 \}, \and \\ \meaningof{E_1 | E_2} = \{ P \in \pi | P \equiv P_{1} | P_{2}, P_{1} \in \meaningof{E_{1}}, P_{2} \in \meaningof{E_2}\} }
\end{mathpar}

\begin{mathpar}
 \inferrule* [lab=behavior] {} {\meaningof{\langle a?b \rangle E} = \{ P \in \pi | P \equiv Q | u?(y)P', \\ \and \\\\ \and \\ \;\;\; u \in \meaningof{a}, \forall z.P'\{z/y\} \in \meaningof{E\{z/b\}}\}, \and \\ \meaningof{a!E} = \{ P \in \pi | P \equiv Q | x!\langle P' \rangle, x \in \meaningof{a} P' \in \meaningof{E}\} }
\end{mathpar}

\begin{mathpar}
 \inferrule* [lab=nominal] {} {\meaningof{\quotep{E}} = \{ \quotep{P} \in \quotep{\pi} | P \in \meaningof{E} \}, \and \meaningof{\quotep{P}} = \{ \quotep{Q} \in \quotep{\pi} | P \equiv Q \} \and \\ \meaningof{@\quotep{E}} = \{ P \in \pi | P \equiv @x, x \in \meaningof{E} \}}
\end{mathpar}

\begin{eqnarray*}
  \\
  \meaningof{-} : TS \to ST
\end{eqnarray*}

\begin{eqnarray*}
  \\
  L : TS \to ST
\end{eqnarray*}

\begin{eqnarray*}
  \\
  P \models E \iff P \in \meaningof{E}
\end{eqnarray*}

\begin{eqnarray*}
  P \approx_{L} Q \iff \forall E \in L. P \models E \iff Q \models E
\end{eqnarray*}

\begin{eqnarray*}
  P \approx_{K} Q
\end{eqnarray*}

\begin{eqnarray*}
  P \approx Q
\end{eqnarray*}

$\approx_{K} = \approx = \approx_{L}$

\subsubsection{Contextual duality}

Note that contexts extend the quotation operation to a family of
operations from processes to names. Given a context, $M$, we can
define a \emph{nominal context}, $\quotep{M}$ by $\quotep{M}[P] :=
\quotep{M[P]}$. To foreshadow what is to come we observe that these
operations enjoy a duality with processes very much like the duality
between vectors and maps from vectors to scalars.

Further, because the calculus is essentially higher-order, we have a
correspondence between contexts and processes. More specifically,
given a name $x$ and a context $M$ we can construct $M^{*}_{x}$ such
that 

\begin{mathpar}
  M^{*}_{x} | \lift{x}{P} \red M[P]
\end{mathpar}

namely,

\begin{mathpar}
  M^{*}_{x} := x?(u).M[\dropn{u}]
\end{mathpar}

The dependence of $M^{*}_{x}$ on a name makes it an abstraction, 

\begin{mathpar}
  M^{*} := (x)x?(u).M[\dropn{u}]
\end{mathpar}

\subsection{Additional notation}

It will sometimes be convenient to denote the process a name
quotes. We already have the notation $x = \quotep{P}$, but it will be
convenient to introduce an alternate notation, $\procn{x}$, when we
want to emphasize the connection to the use of the name. Note that, by
virtue of name equivalence, $\quotep{\procn{x}} \nameeq x$; so, the
notation is consistent with previous definitions.

Further, because names have structure it is possible to effect
substitutions on the basis of that structure. This means we need to
upgrade our notation for substitutions, which we accomplish by
adapting comprehension notation. Thus,

\begin{mathpar}
  P\{ y / x : x \in S \}
\end{mathpar}

is interpreted to mean the process derived from P by replacing (in a
capture-avoiding manner) each occurrence of $x$ in $S$ by $y$. For example,

\begin{mathpar}
  P\{ \quotep{\procn{x}|\procn{x}} / x : x \in \freenames{P} \}
\end{mathpar}

will replace each (occurrence) of a free name $x$ in $P$ by
$\quotep{\procn{x}|\procn{x}}$.

Also, we will avail ourselves of the notation $x^{L}$ and $x^{R}$ to
denote injections of a name into disjoint copies of the name
space. There are numerous ways to accomplish this. One example can be
found in \cite{MeredithR05}. This notation overloads to vectors of
names: $\vec{x}^{\pi} := (x_{i}^{\pi} \; : \; 0 \leq i < |\vec{x}| )$ where $\pi \in \{L,R\}$.

We also use $P^{\Box} := P|\Box$.

In \cite{MeredithR05} an interpretation of the new operator is
given. It turns out that there are several possible interpretations
all enjoying the requisite algebraic properties of the operator (see
\cite{milner91polyadicpi}). We will therefore make liberal use of
$(\nu\; \vec{x})P$.

% subsection the_syntax_and_semantics_of_the_notation_system (end)   

\input{qm2pi.qmops} 

\input{qm2pi.sterngerlach} 

\input{qm2pi.metric} 

% section concurrent_process_calculi (end)

%\input{qm2pi.proofsketch}

% section proof sketch (end)

%\input{qm2pi.slviaknots} 

% section spatial logic via knots (end)

\input{qm2pi.conclusion}

% section conclusion (end)

%\input{qm2pi.dtcodes} 

% section wiring algorithm (end)

\input{qm2pi.ack} 

% section acknowledgments (end)

\newpage


\bibliographystyle{plain}   
\bibliography{../../biblios/main.bib}

\input{qm2pi.rhodetails}

\end{document}



% section proof sketch (end)

%\section{Unlikely characters: spatial logic for
  knots}\label{sub:characteristic_formulae} % (fold)

Associated to the mobile process calculi are a family of logics known
as the Hennessy-Milner logics. These logics typically enjoy a
semantics interpreting formulae as sets of processes that when
factored through the encoding outlined above allows an identification
of classes of knots with logical formulae. In the context of this
encoding the sub-family known as the spatial logics \cite{CairesC03}
\cite{CairesC04} \cite{Caires04} are of particular interest providing
several important features for expressing and reasoning about
properties (i.e. classes) of knots. We hint here at how this may be done.

%\begin{description}
%\item [structural connectives] 
\subsubsection{Structural connectives} The spatial logics enjoy
structural connectives corresponding, at the logical level, to the
parallel composition ($P | Q$) and new name ($(\nu \; x)P$)
connectives for processes. As illustrated in the examples below, these
connectives are extremely expressive given the shape of our encoding.
%\item [decideable satisfaction]

\subsubsection{Decideable satisfaction}
In \cite{Caires04} the satisfaction relation is shown to be decideable
for a rich class of processes. It further turns out that the image of
the our encoding is a proper subset of that class. This result
provides the basis for an algorithm by which to search for knots
enjoying a given property.
%\item [characteristic formulae]

\subsubsection{Characteristic formulae}
In the same paper \cite{Caires04} , Caires presents a means of calculating
characteristic formulae, selecting equivalence classes of processes
up to a pre--specified depth limit on the support set of names. Composed with our
encoding, this characteristic formula can be used to select
characteristic formulae for knots.
%\end{description}

\subsubsection{Spatial logic formulae}

The grammar below (segmented for comprehension) summarizes the syntax
of spatial logic formulae. We employ illustrative examples in the
sequel to provide an intuitive understanding of their meaning
referring the reader to \cite{Caires04} for a more detailed explication
of the semantics.

\begin{mathpar}
  \inferrule* [lab=boolean] {} {{A,B} \bc T \;|\; \neg A \;|\; A \wedge B \;|\; \eta = \eta'}
  \and
  \inferrule* [lab=spatial] {} {|\; \pzero \;|\; A | B \;|\; x \text{\textregistered} A \;|\; \forall x . A \;|\;  H x . A}
  \and
  \inferrule* [lab=behavioral] {} {|\; \alpha . A}
  \and 
  \inferrule* [lab=recursion] {} {|\; X(\vec{u}) \;|\; \mu X(\vec{u}) . A}
  \and
  \inferrule* [lab=action] {} {\alpha \bc \langle x?(\vec{y}) \rangle \;|\; \langle x!(\vec{y}) \rangle \;|\; \langle \tau \rangle}
  \and 
  \inferrule* [lab=name] {} {\eta \bc x \;|\; \tau}
\end{mathpar} 

% subsection characteristic_formulae (end)   	 

\subsection{Example formulae}\label{sub:example_formulae_} % (fold)

\subsubsection{Crossing as formula.}
% 
% \begin{align*}
%   \frac{d}{dx} \sin x &= \cos x 
%   & \frac{d}{dx} e^x &= e^x \\
%   \frac{d}{dx} \cos x &= - \sin x 
%   & \frac{d}{dx} \log x &= \frac{1}{x} \\
% \end{align*} 

\begin{align*}
 \mu C(x_{0},x_{1},y_{0},y_{1},u).&(\langle x_{0}?(z) \rangle(\langle u! \rangle\langle y_{1}!z \rangle C(x_{0},x_{1},y_{0},y_{1},u)) & \\
  & \wedge \langle y_{1}?(z) \rangle (\langle u! \rangle \langle x_{0}!z \rangle C(x_{0},x_{1},y_{0},y_{1},u)) & \\
  & \wedge \langle x_{1}?(z) \rangle (\langle u? \rangle \langle y_{0}!z \rangle C(x_{0},x_{1},y_{0},y_{1},u)) & \\
  & \wedge \langle y_{0}?(z) \rangle (\langle u? \rangle \langle x_{1}!z \rangle C(x_{0},x_{1},y_{0},y_{1},u))) &
\end{align*}

The lexicographical similarity between the shape of this formulae and
the shape of definition of the process representing a crossing reveals
the intuitive meaning of this formulae. It describes the capabilities
of a process that has the right to represent a crossing. For example
it picks out processes that may perform an input on the port $x_0$ in
its initial menu of capabilities. What differentiates the formula
from the process, however, is that the crossing process is the
smallest candidate to satisfy the formula. Infinitely many other
processes -- with internal behavior hidden behind this interface, so
to speak -- also satisfy this formula. Even this simple formula,
then, can be seen to open a new view onto knots, providing a
computational interpretation of \emph{virtual} knots.

Note that this formula is derived by hand. A similar formula can be
derived by employing Caires' calculation of characteristic formula
\cite{Caires04} to the process representing a crossing. In light of
this discussion, we let
$\meaningof{C}_{\phi}(x0,x1,y0,y1,u)$ denote a formula specifying the
dynamics we wish to capture of a crossing. To guarantee we preserve
the shape of the interface and minimal semantics we demand that
$\meaningof{C}_{\phi}(x0,x1,y0,y1,u) \Rightarrow
\textbf{C}(x0,x1,y0,y1,u)$ where $\textbf{C}(x0,x1,y0,y1,u)$ denotes
the formula above.
                            
\subsubsection{Crossing number constraints.}
The moral content of the context lemma (Lemma \ref{context}) is that the notion of
``locality'' in the Reidemeister moves is effectively captured by the
parallel composition operator of the process calculus. This intuition
extends through the logic. Given a formula,
$\meaningof{C}_{\phi}(x0,x1,y0,y1,u)$, we can use the structural
connectives to specify constraints on crossing numbers, such as at
least $n$ crossings, or exactly $n$ crossings.
\begin{mathpar}
  \inferrule* [lab=at-least-n] {} { K^{\geq n}_{\phi}(\vec{xs},\vec{ys}) := \Pi_{i=0}^{n-1} Hu . \meaningof{C}_{\phi}(xs_i,ys_i,u) | T }
  \and 
  \inferrule* [lab=exactly-n] {} { K^{= n}_{\phi}(\vec{xs},\vec{ys}) := \Pi_{i=0}^{n-1} Hu . \meaningof{C}_{\phi}(xs_i,ys_i,u) | \neg (\forall x_0,y_0,x_1,y_1,u . \meaningof{C}_{\phi}(x_0,y_0,x_1,y_1,u) | T) }
\end{mathpar}

To round out this section, recall that the encoding of an $n$-crossing
knot decomposes into a parallel composition of $n$ \emph{copies} of a
crossing process together with a wiring harness. To specify different
knot classes with the same crossing number amounts to specifying
logical constraints on the wiring harness. In the interest of space,
we defer examples to a forthcoming paper. Suffice it to say that both
the conditions ``alternating knot'' and ``contains the tangle
corresponding to 5/3'' are expressible. For example, it is possible to
calculate the characteristic formula of a process corresponding to the
tangle 5/3 and conjoin it into the classifying formula via the
composition connective of the logic.

Finally, we wish to observe that it is entirely within reason to
contemplate a more domain-specific version of spatial logic tailored
to the shape of processes in the image of the encoding. Such a
domain-specific logic would have a better claim to the title formal
language of knot properties.

% subsection example_formulae_ (end)

% section knots_as_processes (end) 

% section spatial logic via knots (end)

\section{Conclusions and future work}

\paragraph{Testing physical space}
You, gentle reader, may wonder why of all the theorems to be proved
given this set up we pick the one above. In some sense it's hardly
central to quantum mechanics. We see it as central in the sense that
it firmly establishes a notion of physical space arising from a notion
of the equivalence of behavior. Relating bisimulation to a metric is a
big step forward, but one is faced with interpreting the relationship
of that metric space to something more physical. Quantum mechanical
notions of ``physical'' space are still far from intuitive, but by
relating this idea of distance as testing to calculations that predict
physical circumstances we are making a not insignificant step forward
toward an understanding of the physical space we inhabit as
essentially dynamic.

\paragraph{Effectivity and simulation}
One of the observations we have yet to make is that the entire program
spelled out here is effective. We have built various interpreters for
the reflective calculus at work in this interpretation. In principle,
then, we can simulate quantum mechanics on a computer. The place where
the simulation may lose fidelity is the infinitely branching summation
for the annihilator.

In this connection i also want to point out that the evaluation style
calculation of the inner product puts the non-determinism of the
summation right at the heart of measurement. This suggests that
Milner's original reduction-based formulation of the dynamics of his
calculi in terms of sums was not just notationally suggestive of a
notion of measure-and-continue but captured some significant part of
the physics.

\paragraph{Quantum continuations}
In light of this last observation i want to point out that the
predominant account of quantum mechanics is missing a key aspect of a
truly compositional story of the physical situation. In a real lab,
when a measurement is made the observation can be made to feed into
another device that then makes another measurement conditioned on the
results of the first. This means that after the superposition was
collapsed the entire experimental set up remained in
superposition. While QM offers a means of writing this down it doesn't
quite line up well with the well-trodden formulation of computation
and continuation that we see so succinctly expressed in Milner's
calculi. This suggests that there might be advantages to this account
of dynamics waiting to be explored.

\paragraph{Quantum logic}
In this connection, we also note that by virtue of having the
Hennessy-Milner construction, we can pull the construction through the
interpretation of QM. This gives us a natural candidate for a quantum
logic that enjoys an extremely tight connection with it's domain of
interpretation, making the construction much less ad hoc (rather it is
the image of functor!).

\paragraph{Quantum probabiity}
i have questions about the basis of the interpretation of inner
product as probability amplitude. In particular, using which
axiomatization of probability theory does the notion of probability
amplitude earn the right to be so dubbed? In other words, where is the
proof that the operation for calculating a probability amplitude (and
then squaring) satisfies the axioms of what it means to calculate a
probability? Even if such a proof exists (i have yet to find it in the
literature), i wonder if it might not be possible to turn things on
their heads. Can we view the calculation of the probability amplitude
as an axiomatization of probability? If so, then the definition we
give for calculating probability amplitude may provide the basis for
an \emph{effective} theory of probability.

\paragraph{Quantum vs ``biological'' information}
Finally, i want to conclude with a more philosophical observation. At
a recent workshop in which QM was a predominant topic i noticed
something about quantum information. The speaker was giving a riveting
discussion of axiomatic QM and showing how properties of ``no
cloning'' and ``no deleting'' emerged as consequences of the
axiomatization. Theorems of this form are necessary to give us a sense
of confidence that our axioms characterize the physical theory. What
struck me, though, was that if quantum information is neither erasable
nor replicable it is markedly different from \emph{life}. Two of the
things we know about life is that

\begin{itemize}
  \item it ends;
  \item to gain some measure of persistence, to transcend it's
    finitude it is imminently copyable.
\end{itemize}

Both of these qualities are summarized succinctly in the aphorism: all
flesh is grass. For me these two kinds of ``information'' -- call them
quantum and biological -- are end points on a spectrum of strategies
for persistence. At one end, we have those curious entities that enjoy
uniqueness and permanence; at the other, we have those who in the face
of a certain end and an uncertain present make a go of passing
something on. To me one of the more remarkable aspects of the latter
strategy is that in the presence of noise (and certain features of
copying) we get a kind of dynamism, a chance for improvement against a
given persistent condition.

% subsection other_calculi_other_bisimulations_and_geometry_as_behavior (end)




% section conclusion (end)

%\documentclass[12pt]{llncs}
%\documentclass{jktr}

\usepackage[pdftex]{hyperref}                   
\usepackage {listings}
\usepackage {mathpartir}
\usepackage{bcprules}
%\usepackage{listings}
                       
\usepackage{graphicx} 
%\usepackage[margins=2.5cm,nohead,nofoot]{geometry}
%\usepackage{geometry}
\usepackage{amsfonts}
\usepackage{amstext}
\usepackage{latexsym}
\usepackage{amssymb}
\usepackage{color}


%\include{myPreamble}
\include{qm2pi.local} 

%\ifpdf
%\usepackage[pdftex]{graphicx}
%\else
%\usepackage{graphicx}
%\fi

 % \ifpdf
%  \usepackage{pdfsync}
%  \if


%\title{Brief Article}
%\author{David F. Snyder}
%\author{L.G. Meredith}

%\address{Dept. of Math., Texas State University--San Marcos, San Marcos, TX 78666}
       
\pagestyle{empty}


\begin{document}

\lstset{language=[Objective]Caml,frame=shadowbox}

\input{qm2pi.front}

% section front matter (end)

\input{qm2pi.intro} 
 
% section introduction (end)

% \input{qm2pi.knotations} 

% section notation (end)

\input{qm2pi.process.calculi} 

% section concurrent_process_calculi_and_spatial_logics_ (end)
    
%\input{qm2pi.knots2pi} 

%\input{qm2pi.trefoil} 

%\input{qm2pi.mainthm} 

% subsection basic_interpretation (end)

%\input{qm2pi.rho.presentation} 
\subsection{The syntax and semantics of the notation system}\label{sub:the_syntax_and_semantics_of_the_notation_system} % (fold)

We now summarize a technical presentation of the calculus that
embodies our theory of dynamics. The typical presentation of such a
calculus follows the style of giving generators and relations on
them. The grammar, below, describing term constructors, freely
generates the set of processes, $\Proc$. This set is then quotiented
by a relation known as structural congruence and it is over this set
that the notion of dynamics is expressed. This presentation is
essentially that of \cite{MeredithR05} with the addition of
polyadicity and summation. For readability we have relegated some of
the technical subtleties to an appendix.

\subsubsection{Process grammar}\label{subsub:process_grammar}

\begin{mathpar}
  \inferrule* [lab=synchronization] {} {{M} \bc \pzero \;|\; x?F \;|\; x!C }
  \and
  \inferrule* [lab=abstraction] {} {{F} \bc (x)P}
  \and
  \inferrule* [lab=concretion] {} {{C} \bc \langle Q \rangle}
  \and
  \inferrule* [lab=process] {} {{P,Q} \bc M \;| \;P|Q \;|\; @{x}}
  \and
  \inferrule* [lab=name] {} {{x} \bc \quotep{P}}
\end{mathpar} 

Note that $\vec{x}$ (resp. $\vec{P}$) denotes a vector of names
(resp. processes) of length $|\vec{x}|$ (resp. $|\vec{P}|$). We adopt
the following useful abbreviations.

\begin{mathpar}
   x?(\vec{y}).P := x.(\vec{y})P \and  x\clift{\vec{P}} := x.\clift{\vec{P}}
   \and x!(y) := \lift{x}{\dropn{y}}
   \and \Pi_{i=0}^{n-1}P_i := P_0 | \ldots | P_{n-1}
\end{mathpar}

\subsubsection{Structural congruence}

\paragraph{Free and bound names and alpha-equivalence.} At the
core of structural equivalence is alpha-equivalence which identifies
process that are the same up to a change of variable. Formally, we
recognize the distinction between free and bound names. The free names
of a process, $\freenames{P}$, may be calculated recursively as
follows:

\begin{mathpar}
\freenames{\pzero} := \emptyset
  \and \\
  \freenames{x?(y).P} := \{ x \} \cup (\freenames{P} \setminus \{ y \})
  \and 
  \freenames{x!\langle P \rangle} := \{ x \} \cup \{ P \} 
  \and \\
  \freenames{P|Q} := \freenames{P} \cup \freenames{Q}
  \and \\
  \freenames{@{x}} := \{ x \}
\end{mathpar}

$\pi$
$\quotep{\pi}$

$\freenames{-} : \pi \to \mathcal{P}(\quotep{\pi})$

\begin{eqnarray*}
  \freenames{\pzero} & := & \emptyset \\
  \freenames{x?(y).P} & := & \{ x \} \cup (\freenames{P} \setminus \{ y \}) \\
  \freenames{x!\langle P \rangle} & := & \{ x \} \cup \{ P \} \\
  \freenames{P|Q} & := & \freenames{P} \cup \freenames{Q} \\
  \freenames{\dropn{x}} & := & \{ x \}
\end{eqnarray*}

The bound names of a process, $\boundnames{P}$, are those names occurring in $P$
that are not free. For example, in $x?(y).0$, the name $x$ is free, while $y$ is bound.

\begin{mathpar}
  \inferrule* [lab=monoidal-laws] {} { P|Q \equiv Q|P \and P|0 \equiv P \and P|(Q|R) \equiv (P|Q)|R }
\end{mathpar}

\begin{mathpar}
  \inferrule* [lab=alpha-equivalence] {} { (x)P \equiv (y)P\{y/x\} \and y \not\in \freenames{P} }
\end{mathpar}

\begin{definition}
Then two processes, $P,Q$, are alpha-equivalent if $P = Q\{\vec{y}/\vec{x}\}$ for
some $\vec{x} \in \boundnames{Q},\vec{y} \in \boundnames{P}$, where $Q\{\vec{y}/\vec{x}\}$
denotes the capture-avoiding substitution of $\vec{y}$ for $\vec{x}$ in $Q$.
\end{definition}

\begin{definition}
  The {\em structural congruence} \cite{SangiorgiWalker} , $\equiv$,
  between processes is the least congruence containing
  alpha-equivalence, satisfying the abelian monoid laws
  (associativity, commutativity and $\pzero$ as identity) for parallel
  composition $|$ and for summation $+$.
\end{definition}

\subsection{Name equivalence}

We take name equivalence, written $\nameeq$, to be the smallest
equivalence relation generated by the following rules.

\begin{mathpar}
\inferrule*[lab=Quote-drop]
{ }
{ \quotep{@{x}} \nameeq x }

\inferrule*[lab=Struct-equiv]
{ P \scong Q }
{ \quotep{P} \nameeq \quotep{Q} }
\end{mathpar}

The astute reader will have noticed that the mutual recursion of names
and processes imposes a mutual recursion on alpha-equivalence and
structural equivalence via name-equivalence. Fortunately, all of this
works out pleasantly and we may calculate in the natural way, free of
concern. The reader interested in the details is referred to the
appendix \ref{appendix:rho_details}.

\subsection{Substitution}

We use $\Proc$ for the set of processes, $\QProc$ for the set of
names, and $\id{\{}\vec{y} / \vec{x} \id{\}}$ to denote partial maps,
$s : \QProc \rightarrow \QProc$. A map, $s$ lifts, uniquely, to a map
on process terms, $\widehat{s} : \Proc \rightarrow \Proc$ by the
following equations.

\begin{mathpar}
  (0) \psubstp{Q}{P} := 0 \\
  (R \juxtap S) \psubstp{Q}{P}
  :=    
  (R)\psubstp{Q}{P} \juxtap (S) \psubstp{Q}{P} \\
  (x?(y).R) \psubstp{Q}{P}    
  :=    
  (x)\substp{Q}{P} (z)\concat( (R \psubstn{z}{y}) \psubstp{Q}{P} ) \\
  (\lift{x}{R}) \psubstp{Q}{P}  
  :=
  \lift{(x)\substp{Q}{P}}{ R \psubstp{Q}{P} } \\
%   (\dropn{x})  \psubstp{Q}{P}       
%   := 
%   \left\{ 
%     \begin{array}{ccc} 
%       \dropn{\quotep{Q}} & & x \nameeq \quotep{P} \\
%       \dropn{x} & & otherwise \\
%     \end{array}
%   \right. 
  (\dropn{x})  \psubstp{Q}{P}       
  := 
  \left\{ 
    \begin{array}{ccc} 
      Q & & x \nameeq \quotep{P} \\
      \dropn{x} & & otherwise \\
    \end{array}
  \right.
\end{mathpar}
 

where

\begin{eqnarray}
  (x)\id{\{} \lpquote Q \rpquote / \lpquote P \rpquote \id{\}}            = 
  \left\{ 
    \begin{array}{ccc}
      \lpquote Q \rpquote & & x \nameeq \lpquote P \rpquote \\
      x & & otherwise \\
    \end{array}
  \right. \nonumber
\end{eqnarray}

and $z$ is chosen distinct from $\quotep{P}$, $\quotep{Q}$, the free
names in $Q$, and all the names in $R$. Our $\alpha$-equivalence will
be built in the standard way from this substitution.

\begin{remark}\label{rem:no_self_referential_names}
  One consequence of these definitions is that $\forall P. \quotep{P}
  \not\in \freenames{P}$.
\end{remark}

\subsection{ Dynamic quote: an example }

Anticipating something of what's to come, consider applying the
substitution, $\widehat{\id{\{}u / z \id{\}}}$, to the following pair
of processes, $\lift{w}{y!(z)}$ and $w[ \lpquote y!(z) \rpquote ]$.

\begin{eqnarray}
	\lift{w}{y!(z)}\widehat{\id{\{}u / z \id{\}}}
		& = &
		\lift{w}{y!(u)} \nonumber\\
	w[ \lpquote y!(z) \rpquote ] \widehat{ \id{\{}u / z \id{\}} }
		& = &
		w[ \lpquote y!(z) \rpquote ] \nonumber
\end{eqnarray}

Because the body of the process between quotes is impervious to
substitution, we get radically different answers. In fact, by
examining the first process in an input context,
e.g. $x?(z).\lift{w}{y!(z)}$, we see that the process under the lift
operator may be shaped by prefixed inputs binding a name inside it. In
this sense, the lift operator will be seen as a way to dynamically
construct processes before reifying them as names.

Finally equipped with these standard features we can present the
dynamics of the calculus.

\subsubsection{Operational semantics} 

Finally, we introduce the computational dynamics. What marks these
algebras as distinct from other more traditionally studied algebraic
structures, e.g. vector spaces or polynomial rings, is the manner in
which dynamics is captured. In traditional structures, dynamics is typically
expressed through morphisms between such structures, as in linear maps
between vector spaces or morphisms between rings. In algebras
associated with the semantics of computation, the dynamics is
expressed as part of the algebraic structure itself, through a
reduction reduction relation typically denoted by $\red$. Below, we
give a recursive presentation of this relation for the calculus used
in the encoding.

$\red \subseteq \pi \times \pi$
$\red : \pi \to \mathcal{P}(\pi)$

\begin{mathpar}
  \inferrule* [lab=Comm] { \textsf{match}( x_{src}, x_{trgt} ) } { x_{trgt}?(y)P \; | \; x_{src}!\langle {Q} \rangle \red P\{\quotep{Q}/y}\} }
  \and \\
  \inferrule* [lab=Par] {{P} \red {P}'} {{{P} | {Q}} \red {{P}' | {Q}}}
  \and
  \inferrule* [lab=Equiv]{{{P} \scong {P}'} \andalso {{P}' \red {Q}'} \andalso {{Q}' \scong {Q}}}{{P} \red {Q}}
\end{mathpar}

\begin{eqnarray*}
  match_{\equiv} (\quotep{P},\quotep{Q}) & := & P \equiv Q \\
  match_{\dagger}(\quotep{P},\quotep{Q}) & := & \forall R. P|Q \red^{*} R => R \red^{*} 0 \\
  match_{K}(\quotep{P},\quotep{Q}) & := & K \mbox{ for some context } K
\end{eqnarray*}

$u?(x)P | u!\langle Q \rangle \red P\{\quotep{Q}/x\}$

%We write $\wred$ for $\red^*$, and $P\red$ if $\exists Q $ such that $ P \red Q$.
We write $P\red$ if $\exists Q $ such that $ P \red Q$ and $P\not\red$, otherwise.

\section{Replication}

As mentioned before, it is known that replication (and hence
recursion) can be implemented in a higher-order process algebra
\cite{SangiorgiWalker}. As our first example of calculation with the
machinery thus far presented we give the construction explicitly in
the {\rhoc}.

\begin{eqnarray}
	D_{x} & := & \prefix{x}{y}{(\binpar{\outputp{x}{y}}{@{y}})} \nonumber\\
	\bangp_{x}{P} & := & \binpar{{x}!\langle{\binpar{D_{x}}{P}}\rangle}{D_{x}} \nonumber
\end{eqnarray}

\begin{eqnarray}
	\bangp_{x}{P} & & \nonumber\\
	=
	& {x}!\langle{(\prefix{x}{y}{(\outputp{x}{y} | @{y})) | P}}\rangle 
	      | \prefix{x}{y}{(\outputp{x}{y} | @{y})} & \nonumber\\
	\red
	& (\outputp{x}{y} | @{y})\substn{\quotep{(\prefix{x}{y}{(@{y} | \outputp{x}{y})) | P}}}{y} & \nonumber\\
	=
	& \outputp{x}{\quotep{(\prefix{x}{y}{(\outputp{x}{y} | @{y})) | P}}}
	  | {(\prefix{x}{y}{(\outputp{x}{y} | @{y})) | P}} & \nonumber\\
	\red
	& \ldots & \nonumber\\
	\red^*
	& P | P | \ldots & \nonumber
\end{eqnarray}

Of course, this encoding, as an implementation, runs away, unfolding
$\bangp{P}$ eagerly. A lazier and more implementable replication
operator, restricted to input-guarded processes, may be obtained as follows.

\begin{eqnarray}
\bangp{\prefix{u}{v}{P}} 
	:= 
	\binpar{\lift{x}{\prefix{u}{v}{(\binpar{D(x)}{P})}}}{D(x)} \nonumber
\end{eqnarray}

\begin{remark}
  Note that the lazier definition still does not deal with summation
  or mixed summation (i.e. sums over input and output). The reader is
  invited to construct definitions of replication that deal with these
  features. 

  Further, the definitions are parameterized in a name, $x$. Can you,
  gentle reader, make a definition that eliminates this parameter and
  guarantees no accidental interaction between the replication
  machinery and the process being replicated -- i.e. no accidental
  sharing of names used by the process to get its work done and the
  name(s) used by the replication to effect copying. This latter
  revision of the definition of replication is crucial to obtaining
  the expected identity $!!P \sim !P$.
\end{remark}

\begin{remark}\label{rem:paradoxical_combinator}
  The reader familiar with the lambda calculus will have noticed the
  similarity between $D$ and the paradoxical combinator.

  [Ed. note: the existence of this seems to suggest we have to be more
  restrictive on the set of processes and names we admit if we are to
  support no-cloning.]
\end{remark}

\subsubsection{Bisimulation}

The computational dynamics gives rise to another kind of equivalence,
the equivalence of computational behavior. As previously mentioned
this is typically captured \emph{via} some form of bisimulation.

% The notion we use in this paper is weak barbed bisimulation
% \cite{milner91polyadicpi}.

The notion we use in this paper is derived from weak barbed
bisimulation \cite{milner91polyadicpi}. 

\begin{definition}
An \emph{observation relation}, $\downarrow_{\mathcal N}$, over a set
of names, $\mathcal N$, is the smallest relation satisfying the rules
below.

\infrule[Out-barb]{y \in {\mathcal N}, \; x \nameeq y}
		  {\outputp{x}{v} \downarrow_{\mathcal N} x}
\infrule[Par-barb]{\mbox{$P\downarrow_{\mathcal N} x$ or $Q\downarrow_{\mathcal N} x$}}
		  {\binpar{P}{Q} \downarrow_{\mathcal N} x}

We write $P \Downarrow_{\mathcal N} x$ if there is $Q$ such that 
$P \wred Q$ and $Q \downarrow_{\mathcal N} x$.
\end{definition}

\begin{definition}
%\label{def.bbisim}
An  ${\mathcal N}$-\emph{barbed bisimulation} over a set of names, ${\mathcal N}$, is a symmetric binary relation 
${\mathcal S}_{\mathcal N}$ between agents such that $P\rel{S}_{\mathcal N}Q$ implies:
\begin{enumerate}
\item If $P \red P'$ then $Q \wred Q'$ and $P'\rel{S}_{\mathcal N} Q'$.
\item If $P\downarrow_{\mathcal N} x$, then $Q\Downarrow_{\mathcal N} x$.
\end{enumerate}
$P$ is ${\mathcal N}$-barbed bisimilar to $Q$, written
$P \wbbisim_{\mathcal N} Q$, if $P \rel{S}_{\mathcal N} Q$ for some ${\mathcal N}$-barbed bisimulation ${\mathcal S}_{\mathcal N}$.
\end{definition}

$\mathcal{R} \subseteq \pi \times \pi$

$P \mathcal{R} Q => \forall P'. P \red P' \Rightarrow \exists Q'. Q \red Q', P' \mathcal{R} Q'$

$P \vdash x \Rightarrow Q \vdash x$

\begin{mathpar}
  \inferrule*[lab=Out-barb]{x \nameeq y}{{y}!\langle{Q}\rangle \vdash x}
  \and
  \inferrule*[lab=Par-barb]{\mbox{$P\vdash x$ or $Q\vdash x$}}{\binpar{P}{Q} \vdash x}
\end{mathpar}

\subsubsection{Contexts}

One of the principle advantages of computational calculi like the
$\pi$-calculus is a well-defined notion of context,
contextual-equivalence and a correlation between
contextual-equivalence and notions of bisimulation. The notion of
context allows the decomposition of a process into (sub-)process and
its syntactic environment, its context. Thus, a context may be
thought of as a process with a ``hole'' (written $\Box$) in it. The
application of a context $M$ to a process $P$, written $M[P]$, is
tantamount to filling the hole in $M$ with $P$. In this paper we do
not need the full weight of this theory, but do make use of the notion
of context in the proof the main theorem. 

\begin{mathpar}
  \inferrule* [lab=summation] {} {{M_{M},M_{N}} \bc \Box \;|\; x.M_{A} \;|\; M_{M}+M_{N}}
  \and
  \inferrule* [lab=agent] {} {{M_{A}} \bc (\vec{x})M_{P} \;| \; \clift{P_0,\ldots,M_{P},\ldots,P_N}}
  \and \\
  \inferrule* [lab=process] {} {{M_{P}} \bc M_{N} \;| \;P|M_{P} }
\end{mathpar} 

\begin{mathpar}
  \inferrule* [lab=sychronization] {} {M_{N} \bc \Box \;|\; x?M_{F} \;|\; x!M_{C}}
  \and
  \inferrule* [lab=abstraction] {} {{M_{F}} \bc (x)M_{P} }
  \and
  \inferrule* [lab=concretion] {} {{M_{C}} \bc \langle M_{P} \rangle }
  \and \\
  \inferrule* [lab=process] {} {{M_{P}} \bc M_{N} \;| \;P|M_{P} }
\end{mathpar}

\begin{definition}[contextual application] Given a context $M$, and
  process $P$, we define the \emph{contextual application}, $M[P] :=
  M\{P/\Box\}$. That is, the contextual application of M to P is the
  substitution of $P$ for $\Box$ in $M$.
\end{definition}

$\meaningof{-} : L \to \mathcal{P}(\pi)$

\begin{mathpar}
  \inferrule* [lab=collection] {} {\meaningof{true} = \pi, \and \meaningof{~E} = \pi \setminus \meaningof{E}, \and \meaningof{E_{1} \& E_{2}} = \meaningof{E_{1}} \cap \meaningof{E_{2}}}
\end{mathpar}

\begin{mathpar}
  \inferrule* [lab=structure] {} {\meaningof{0} = \{ P \in \pi | P \equiv 0 \}, \and \\ \meaningof{E_1 | E_2} = \{ P \in \pi | P \equiv P_{1} | P_{2}, P_{1} \in \meaningof{E_{1}}, P_{2} \in \meaningof{E_2}\} }
\end{mathpar}

\begin{mathpar}
 \inferrule* [lab=behavior] {} {\meaningof{\langle a?b \rangle E} = \{ P \in \pi | P \equiv Q | u?(y)P', \\ \and \\\\ \and \\ \;\;\; u \in \meaningof{a}, \forall z.P'\{z/y\} \in \meaningof{E\{z/b\}}\}, \and \\ \meaningof{a!E} = \{ P \in \pi | P \equiv Q | x!\langle P' \rangle, x \in \meaningof{a} P' \in \meaningof{E}\} }
\end{mathpar}

\begin{mathpar}
 \inferrule* [lab=nominal] {} {\meaningof{\quotep{E}} = \{ \quotep{P} \in \quotep{\pi} | P \in \meaningof{E} \}, \and \meaningof{\quotep{P}} = \{ \quotep{Q} \in \quotep{\pi} | P \equiv Q \} \and \\ \meaningof{@\quotep{E}} = \{ P \in \pi | P \equiv @x, x \in \meaningof{E} \}}
\end{mathpar}

\begin{eqnarray*}
  \\
  \meaningof{-} : TS \to ST
\end{eqnarray*}

\begin{eqnarray*}
  \\
  L : TS \to ST
\end{eqnarray*}

\begin{eqnarray*}
  \\
  P \models E \iff P \in \meaningof{E}
\end{eqnarray*}

\begin{eqnarray*}
  P \approx_{L} Q \iff \forall E \in L. P \models E \iff Q \models E
\end{eqnarray*}

\begin{eqnarray*}
  P \approx_{K} Q
\end{eqnarray*}

\begin{eqnarray*}
  P \approx Q
\end{eqnarray*}

$\approx_{K} = \approx = \approx_{L}$

\subsubsection{Contextual duality}

Note that contexts extend the quotation operation to a family of
operations from processes to names. Given a context, $M$, we can
define a \emph{nominal context}, $\quotep{M}$ by $\quotep{M}[P] :=
\quotep{M[P]}$. To foreshadow what is to come we observe that these
operations enjoy a duality with processes very much like the duality
between vectors and maps from vectors to scalars.

Further, because the calculus is essentially higher-order, we have a
correspondence between contexts and processes. More specifically,
given a name $x$ and a context $M$ we can construct $M^{*}_{x}$ such
that 

\begin{mathpar}
  M^{*}_{x} | \lift{x}{P} \red M[P]
\end{mathpar}

namely,

\begin{mathpar}
  M^{*}_{x} := x?(u).M[\dropn{u}]
\end{mathpar}

The dependence of $M^{*}_{x}$ on a name makes it an abstraction, 

\begin{mathpar}
  M^{*} := (x)x?(u).M[\dropn{u}]
\end{mathpar}

\subsection{Additional notation}

It will sometimes be convenient to denote the process a name
quotes. We already have the notation $x = \quotep{P}$, but it will be
convenient to introduce an alternate notation, $\procn{x}$, when we
want to emphasize the connection to the use of the name. Note that, by
virtue of name equivalence, $\quotep{\procn{x}} \nameeq x$; so, the
notation is consistent with previous definitions.

Further, because names have structure it is possible to effect
substitutions on the basis of that structure. This means we need to
upgrade our notation for substitutions, which we accomplish by
adapting comprehension notation. Thus,

\begin{mathpar}
  P\{ y / x : x \in S \}
\end{mathpar}

is interpreted to mean the process derived from P by replacing (in a
capture-avoiding manner) each occurrence of $x$ in $S$ by $y$. For example,

\begin{mathpar}
  P\{ \quotep{\procn{x}|\procn{x}} / x : x \in \freenames{P} \}
\end{mathpar}

will replace each (occurrence) of a free name $x$ in $P$ by
$\quotep{\procn{x}|\procn{x}}$.

Also, we will avail ourselves of the notation $x^{L}$ and $x^{R}$ to
denote injections of a name into disjoint copies of the name
space. There are numerous ways to accomplish this. One example can be
found in \cite{MeredithR05}. This notation overloads to vectors of
names: $\vec{x}^{\pi} := (x_{i}^{\pi} \; : \; 0 \leq i < |\vec{x}| )$ where $\pi \in \{L,R\}$.

We also use $P^{\Box} := P|\Box$.

In \cite{MeredithR05} an interpretation of the new operator is
given. It turns out that there are several possible interpretations
all enjoying the requisite algebraic properties of the operator (see
\cite{milner91polyadicpi}). We will therefore make liberal use of
$(\nu\; \vec{x})P$.

% subsection the_syntax_and_semantics_of_the_notation_system (end)   

\input{qm2pi.qmops} 

\input{qm2pi.sterngerlach} 

\input{qm2pi.metric} 

% section concurrent_process_calculi (end)

%\input{qm2pi.proofsketch}

% section proof sketch (end)

%\input{qm2pi.slviaknots} 

% section spatial logic via knots (end)

\input{qm2pi.conclusion}

% section conclusion (end)

%\input{qm2pi.dtcodes} 

% section wiring algorithm (end)

\input{qm2pi.ack} 

% section acknowledgments (end)

\newpage


\bibliographystyle{plain}   
\bibliography{../../biblios/main.bib}

\input{qm2pi.rhodetails}

\end{document}

 

% section wiring algorithm (end)

\documentclass[12pt]{llncs}
%\documentclass{jktr}

\usepackage[pdftex]{hyperref}                   
\usepackage {listings}
\usepackage {mathpartir}
\usepackage{bcprules}
%\usepackage{listings}
                       
\usepackage{graphicx} 
%\usepackage[margins=2.5cm,nohead,nofoot]{geometry}
%\usepackage{geometry}
\usepackage{amsfonts}
\usepackage{amstext}
\usepackage{latexsym}
\usepackage{amssymb}
\usepackage{color}


%\include{myPreamble}
\include{qm2pi.local} 

%\ifpdf
%\usepackage[pdftex]{graphicx}
%\else
%\usepackage{graphicx}
%\fi

 % \ifpdf
%  \usepackage{pdfsync}
%  \if


%\title{Brief Article}
%\author{David F. Snyder}
%\author{L.G. Meredith}

%\address{Dept. of Math., Texas State University--San Marcos, San Marcos, TX 78666}
       
\pagestyle{empty}


\begin{document}

\lstset{language=[Objective]Caml,frame=shadowbox}

\input{qm2pi.front}

% section front matter (end)

\input{qm2pi.intro} 
 
% section introduction (end)

% \input{qm2pi.knotations} 

% section notation (end)

\input{qm2pi.process.calculi} 

% section concurrent_process_calculi_and_spatial_logics_ (end)
    
%\input{qm2pi.knots2pi} 

%\input{qm2pi.trefoil} 

%\input{qm2pi.mainthm} 

% subsection basic_interpretation (end)

%\input{qm2pi.rho.presentation} 
\subsection{The syntax and semantics of the notation system}\label{sub:the_syntax_and_semantics_of_the_notation_system} % (fold)

We now summarize a technical presentation of the calculus that
embodies our theory of dynamics. The typical presentation of such a
calculus follows the style of giving generators and relations on
them. The grammar, below, describing term constructors, freely
generates the set of processes, $\Proc$. This set is then quotiented
by a relation known as structural congruence and it is over this set
that the notion of dynamics is expressed. This presentation is
essentially that of \cite{MeredithR05} with the addition of
polyadicity and summation. For readability we have relegated some of
the technical subtleties to an appendix.

\subsubsection{Process grammar}\label{subsub:process_grammar}

\begin{mathpar}
  \inferrule* [lab=synchronization] {} {{M} \bc \pzero \;|\; x?F \;|\; x!C }
  \and
  \inferrule* [lab=abstraction] {} {{F} \bc (x)P}
  \and
  \inferrule* [lab=concretion] {} {{C} \bc \langle Q \rangle}
  \and
  \inferrule* [lab=process] {} {{P,Q} \bc M \;| \;P|Q \;|\; @{x}}
  \and
  \inferrule* [lab=name] {} {{x} \bc \quotep{P}}
\end{mathpar} 

Note that $\vec{x}$ (resp. $\vec{P}$) denotes a vector of names
(resp. processes) of length $|\vec{x}|$ (resp. $|\vec{P}|$). We adopt
the following useful abbreviations.

\begin{mathpar}
   x?(\vec{y}).P := x.(\vec{y})P \and  x\clift{\vec{P}} := x.\clift{\vec{P}}
   \and x!(y) := \lift{x}{\dropn{y}}
   \and \Pi_{i=0}^{n-1}P_i := P_0 | \ldots | P_{n-1}
\end{mathpar}

\subsubsection{Structural congruence}

\paragraph{Free and bound names and alpha-equivalence.} At the
core of structural equivalence is alpha-equivalence which identifies
process that are the same up to a change of variable. Formally, we
recognize the distinction between free and bound names. The free names
of a process, $\freenames{P}$, may be calculated recursively as
follows:

\begin{mathpar}
\freenames{\pzero} := \emptyset
  \and \\
  \freenames{x?(y).P} := \{ x \} \cup (\freenames{P} \setminus \{ y \})
  \and 
  \freenames{x!\langle P \rangle} := \{ x \} \cup \{ P \} 
  \and \\
  \freenames{P|Q} := \freenames{P} \cup \freenames{Q}
  \and \\
  \freenames{@{x}} := \{ x \}
\end{mathpar}

$\pi$
$\quotep{\pi}$

$\freenames{-} : \pi \to \mathcal{P}(\quotep{\pi})$

\begin{eqnarray*}
  \freenames{\pzero} & := & \emptyset \\
  \freenames{x?(y).P} & := & \{ x \} \cup (\freenames{P} \setminus \{ y \}) \\
  \freenames{x!\langle P \rangle} & := & \{ x \} \cup \{ P \} \\
  \freenames{P|Q} & := & \freenames{P} \cup \freenames{Q} \\
  \freenames{\dropn{x}} & := & \{ x \}
\end{eqnarray*}

The bound names of a process, $\boundnames{P}$, are those names occurring in $P$
that are not free. For example, in $x?(y).0$, the name $x$ is free, while $y$ is bound.

\begin{mathpar}
  \inferrule* [lab=monoidal-laws] {} { P|Q \equiv Q|P \and P|0 \equiv P \and P|(Q|R) \equiv (P|Q)|R }
\end{mathpar}

\begin{mathpar}
  \inferrule* [lab=alpha-equivalence] {} { (x)P \equiv (y)P\{y/x\} \and y \not\in \freenames{P} }
\end{mathpar}

\begin{definition}
Then two processes, $P,Q$, are alpha-equivalent if $P = Q\{\vec{y}/\vec{x}\}$ for
some $\vec{x} \in \boundnames{Q},\vec{y} \in \boundnames{P}$, where $Q\{\vec{y}/\vec{x}\}$
denotes the capture-avoiding substitution of $\vec{y}$ for $\vec{x}$ in $Q$.
\end{definition}

\begin{definition}
  The {\em structural congruence} \cite{SangiorgiWalker} , $\equiv$,
  between processes is the least congruence containing
  alpha-equivalence, satisfying the abelian monoid laws
  (associativity, commutativity and $\pzero$ as identity) for parallel
  composition $|$ and for summation $+$.
\end{definition}

\subsection{Name equivalence}

We take name equivalence, written $\nameeq$, to be the smallest
equivalence relation generated by the following rules.

\begin{mathpar}
\inferrule*[lab=Quote-drop]
{ }
{ \quotep{@{x}} \nameeq x }

\inferrule*[lab=Struct-equiv]
{ P \scong Q }
{ \quotep{P} \nameeq \quotep{Q} }
\end{mathpar}

The astute reader will have noticed that the mutual recursion of names
and processes imposes a mutual recursion on alpha-equivalence and
structural equivalence via name-equivalence. Fortunately, all of this
works out pleasantly and we may calculate in the natural way, free of
concern. The reader interested in the details is referred to the
appendix \ref{appendix:rho_details}.

\subsection{Substitution}

We use $\Proc$ for the set of processes, $\QProc$ for the set of
names, and $\id{\{}\vec{y} / \vec{x} \id{\}}$ to denote partial maps,
$s : \QProc \rightarrow \QProc$. A map, $s$ lifts, uniquely, to a map
on process terms, $\widehat{s} : \Proc \rightarrow \Proc$ by the
following equations.

\begin{mathpar}
  (0) \psubstp{Q}{P} := 0 \\
  (R \juxtap S) \psubstp{Q}{P}
  :=    
  (R)\psubstp{Q}{P} \juxtap (S) \psubstp{Q}{P} \\
  (x?(y).R) \psubstp{Q}{P}    
  :=    
  (x)\substp{Q}{P} (z)\concat( (R \psubstn{z}{y}) \psubstp{Q}{P} ) \\
  (\lift{x}{R}) \psubstp{Q}{P}  
  :=
  \lift{(x)\substp{Q}{P}}{ R \psubstp{Q}{P} } \\
%   (\dropn{x})  \psubstp{Q}{P}       
%   := 
%   \left\{ 
%     \begin{array}{ccc} 
%       \dropn{\quotep{Q}} & & x \nameeq \quotep{P} \\
%       \dropn{x} & & otherwise \\
%     \end{array}
%   \right. 
  (\dropn{x})  \psubstp{Q}{P}       
  := 
  \left\{ 
    \begin{array}{ccc} 
      Q & & x \nameeq \quotep{P} \\
      \dropn{x} & & otherwise \\
    \end{array}
  \right.
\end{mathpar}
 

where

\begin{eqnarray}
  (x)\id{\{} \lpquote Q \rpquote / \lpquote P \rpquote \id{\}}            = 
  \left\{ 
    \begin{array}{ccc}
      \lpquote Q \rpquote & & x \nameeq \lpquote P \rpquote \\
      x & & otherwise \\
    \end{array}
  \right. \nonumber
\end{eqnarray}

and $z$ is chosen distinct from $\quotep{P}$, $\quotep{Q}$, the free
names in $Q$, and all the names in $R$. Our $\alpha$-equivalence will
be built in the standard way from this substitution.

\begin{remark}\label{rem:no_self_referential_names}
  One consequence of these definitions is that $\forall P. \quotep{P}
  \not\in \freenames{P}$.
\end{remark}

\subsection{ Dynamic quote: an example }

Anticipating something of what's to come, consider applying the
substitution, $\widehat{\id{\{}u / z \id{\}}}$, to the following pair
of processes, $\lift{w}{y!(z)}$ and $w[ \lpquote y!(z) \rpquote ]$.

\begin{eqnarray}
	\lift{w}{y!(z)}\widehat{\id{\{}u / z \id{\}}}
		& = &
		\lift{w}{y!(u)} \nonumber\\
	w[ \lpquote y!(z) \rpquote ] \widehat{ \id{\{}u / z \id{\}} }
		& = &
		w[ \lpquote y!(z) \rpquote ] \nonumber
\end{eqnarray}

Because the body of the process between quotes is impervious to
substitution, we get radically different answers. In fact, by
examining the first process in an input context,
e.g. $x?(z).\lift{w}{y!(z)}$, we see that the process under the lift
operator may be shaped by prefixed inputs binding a name inside it. In
this sense, the lift operator will be seen as a way to dynamically
construct processes before reifying them as names.

Finally equipped with these standard features we can present the
dynamics of the calculus.

\subsubsection{Operational semantics} 

Finally, we introduce the computational dynamics. What marks these
algebras as distinct from other more traditionally studied algebraic
structures, e.g. vector spaces or polynomial rings, is the manner in
which dynamics is captured. In traditional structures, dynamics is typically
expressed through morphisms between such structures, as in linear maps
between vector spaces or morphisms between rings. In algebras
associated with the semantics of computation, the dynamics is
expressed as part of the algebraic structure itself, through a
reduction reduction relation typically denoted by $\red$. Below, we
give a recursive presentation of this relation for the calculus used
in the encoding.

$\red \subseteq \pi \times \pi$
$\red : \pi \to \mathcal{P}(\pi)$

\begin{mathpar}
  \inferrule* [lab=Comm] { \textsf{match}( x_{src}, x_{trgt} ) } { x_{trgt}?(y)P \; | \; x_{src}!\langle {Q} \rangle \red P\{\quotep{Q}/y}\} }
  \and \\
  \inferrule* [lab=Par] {{P} \red {P}'} {{{P} | {Q}} \red {{P}' | {Q}}}
  \and
  \inferrule* [lab=Equiv]{{{P} \scong {P}'} \andalso {{P}' \red {Q}'} \andalso {{Q}' \scong {Q}}}{{P} \red {Q}}
\end{mathpar}

\begin{eqnarray*}
  match_{\equiv} (\quotep{P},\quotep{Q}) & := & P \equiv Q \\
  match_{\dagger}(\quotep{P},\quotep{Q}) & := & \forall R. P|Q \red^{*} R => R \red^{*} 0 \\
  match_{K}(\quotep{P},\quotep{Q}) & := & K \mbox{ for some context } K
\end{eqnarray*}

$u?(x)P | u!\langle Q \rangle \red P\{\quotep{Q}/x\}$

%We write $\wred$ for $\red^*$, and $P\red$ if $\exists Q $ such that $ P \red Q$.
We write $P\red$ if $\exists Q $ such that $ P \red Q$ and $P\not\red$, otherwise.

\section{Replication}

As mentioned before, it is known that replication (and hence
recursion) can be implemented in a higher-order process algebra
\cite{SangiorgiWalker}. As our first example of calculation with the
machinery thus far presented we give the construction explicitly in
the {\rhoc}.

\begin{eqnarray}
	D_{x} & := & \prefix{x}{y}{(\binpar{\outputp{x}{y}}{@{y}})} \nonumber\\
	\bangp_{x}{P} & := & \binpar{{x}!\langle{\binpar{D_{x}}{P}}\rangle}{D_{x}} \nonumber
\end{eqnarray}

\begin{eqnarray}
	\bangp_{x}{P} & & \nonumber\\
	=
	& {x}!\langle{(\prefix{x}{y}{(\outputp{x}{y} | @{y})) | P}}\rangle 
	      | \prefix{x}{y}{(\outputp{x}{y} | @{y})} & \nonumber\\
	\red
	& (\outputp{x}{y} | @{y})\substn{\quotep{(\prefix{x}{y}{(@{y} | \outputp{x}{y})) | P}}}{y} & \nonumber\\
	=
	& \outputp{x}{\quotep{(\prefix{x}{y}{(\outputp{x}{y} | @{y})) | P}}}
	  | {(\prefix{x}{y}{(\outputp{x}{y} | @{y})) | P}} & \nonumber\\
	\red
	& \ldots & \nonumber\\
	\red^*
	& P | P | \ldots & \nonumber
\end{eqnarray}

Of course, this encoding, as an implementation, runs away, unfolding
$\bangp{P}$ eagerly. A lazier and more implementable replication
operator, restricted to input-guarded processes, may be obtained as follows.

\begin{eqnarray}
\bangp{\prefix{u}{v}{P}} 
	:= 
	\binpar{\lift{x}{\prefix{u}{v}{(\binpar{D(x)}{P})}}}{D(x)} \nonumber
\end{eqnarray}

\begin{remark}
  Note that the lazier definition still does not deal with summation
  or mixed summation (i.e. sums over input and output). The reader is
  invited to construct definitions of replication that deal with these
  features. 

  Further, the definitions are parameterized in a name, $x$. Can you,
  gentle reader, make a definition that eliminates this parameter and
  guarantees no accidental interaction between the replication
  machinery and the process being replicated -- i.e. no accidental
  sharing of names used by the process to get its work done and the
  name(s) used by the replication to effect copying. This latter
  revision of the definition of replication is crucial to obtaining
  the expected identity $!!P \sim !P$.
\end{remark}

\begin{remark}\label{rem:paradoxical_combinator}
  The reader familiar with the lambda calculus will have noticed the
  similarity between $D$ and the paradoxical combinator.

  [Ed. note: the existence of this seems to suggest we have to be more
  restrictive on the set of processes and names we admit if we are to
  support no-cloning.]
\end{remark}

\subsubsection{Bisimulation}

The computational dynamics gives rise to another kind of equivalence,
the equivalence of computational behavior. As previously mentioned
this is typically captured \emph{via} some form of bisimulation.

% The notion we use in this paper is weak barbed bisimulation
% \cite{milner91polyadicpi}.

The notion we use in this paper is derived from weak barbed
bisimulation \cite{milner91polyadicpi}. 

\begin{definition}
An \emph{observation relation}, $\downarrow_{\mathcal N}$, over a set
of names, $\mathcal N$, is the smallest relation satisfying the rules
below.

\infrule[Out-barb]{y \in {\mathcal N}, \; x \nameeq y}
		  {\outputp{x}{v} \downarrow_{\mathcal N} x}
\infrule[Par-barb]{\mbox{$P\downarrow_{\mathcal N} x$ or $Q\downarrow_{\mathcal N} x$}}
		  {\binpar{P}{Q} \downarrow_{\mathcal N} x}

We write $P \Downarrow_{\mathcal N} x$ if there is $Q$ such that 
$P \wred Q$ and $Q \downarrow_{\mathcal N} x$.
\end{definition}

\begin{definition}
%\label{def.bbisim}
An  ${\mathcal N}$-\emph{barbed bisimulation} over a set of names, ${\mathcal N}$, is a symmetric binary relation 
${\mathcal S}_{\mathcal N}$ between agents such that $P\rel{S}_{\mathcal N}Q$ implies:
\begin{enumerate}
\item If $P \red P'$ then $Q \wred Q'$ and $P'\rel{S}_{\mathcal N} Q'$.
\item If $P\downarrow_{\mathcal N} x$, then $Q\Downarrow_{\mathcal N} x$.
\end{enumerate}
$P$ is ${\mathcal N}$-barbed bisimilar to $Q$, written
$P \wbbisim_{\mathcal N} Q$, if $P \rel{S}_{\mathcal N} Q$ for some ${\mathcal N}$-barbed bisimulation ${\mathcal S}_{\mathcal N}$.
\end{definition}

$\mathcal{R} \subseteq \pi \times \pi$

$P \mathcal{R} Q => \forall P'. P \red P' \Rightarrow \exists Q'. Q \red Q', P' \mathcal{R} Q'$

$P \vdash x \Rightarrow Q \vdash x$

\begin{mathpar}
  \inferrule*[lab=Out-barb]{x \nameeq y}{{y}!\langle{Q}\rangle \vdash x}
  \and
  \inferrule*[lab=Par-barb]{\mbox{$P\vdash x$ or $Q\vdash x$}}{\binpar{P}{Q} \vdash x}
\end{mathpar}

\subsubsection{Contexts}

One of the principle advantages of computational calculi like the
$\pi$-calculus is a well-defined notion of context,
contextual-equivalence and a correlation between
contextual-equivalence and notions of bisimulation. The notion of
context allows the decomposition of a process into (sub-)process and
its syntactic environment, its context. Thus, a context may be
thought of as a process with a ``hole'' (written $\Box$) in it. The
application of a context $M$ to a process $P$, written $M[P]$, is
tantamount to filling the hole in $M$ with $P$. In this paper we do
not need the full weight of this theory, but do make use of the notion
of context in the proof the main theorem. 

\begin{mathpar}
  \inferrule* [lab=summation] {} {{M_{M},M_{N}} \bc \Box \;|\; x.M_{A} \;|\; M_{M}+M_{N}}
  \and
  \inferrule* [lab=agent] {} {{M_{A}} \bc (\vec{x})M_{P} \;| \; \clift{P_0,\ldots,M_{P},\ldots,P_N}}
  \and \\
  \inferrule* [lab=process] {} {{M_{P}} \bc M_{N} \;| \;P|M_{P} }
\end{mathpar} 

\begin{mathpar}
  \inferrule* [lab=sychronization] {} {M_{N} \bc \Box \;|\; x?M_{F} \;|\; x!M_{C}}
  \and
  \inferrule* [lab=abstraction] {} {{M_{F}} \bc (x)M_{P} }
  \and
  \inferrule* [lab=concretion] {} {{M_{C}} \bc \langle M_{P} \rangle }
  \and \\
  \inferrule* [lab=process] {} {{M_{P}} \bc M_{N} \;| \;P|M_{P} }
\end{mathpar}

\begin{definition}[contextual application] Given a context $M$, and
  process $P$, we define the \emph{contextual application}, $M[P] :=
  M\{P/\Box\}$. That is, the contextual application of M to P is the
  substitution of $P$ for $\Box$ in $M$.
\end{definition}

$\meaningof{-} : L \to \mathcal{P}(\pi)$

\begin{mathpar}
  \inferrule* [lab=collection] {} {\meaningof{true} = \pi, \and \meaningof{~E} = \pi \setminus \meaningof{E}, \and \meaningof{E_{1} \& E_{2}} = \meaningof{E_{1}} \cap \meaningof{E_{2}}}
\end{mathpar}

\begin{mathpar}
  \inferrule* [lab=structure] {} {\meaningof{0} = \{ P \in \pi | P \equiv 0 \}, \and \\ \meaningof{E_1 | E_2} = \{ P \in \pi | P \equiv P_{1} | P_{2}, P_{1} \in \meaningof{E_{1}}, P_{2} \in \meaningof{E_2}\} }
\end{mathpar}

\begin{mathpar}
 \inferrule* [lab=behavior] {} {\meaningof{\langle a?b \rangle E} = \{ P \in \pi | P \equiv Q | u?(y)P', \\ \and \\\\ \and \\ \;\;\; u \in \meaningof{a}, \forall z.P'\{z/y\} \in \meaningof{E\{z/b\}}\}, \and \\ \meaningof{a!E} = \{ P \in \pi | P \equiv Q | x!\langle P' \rangle, x \in \meaningof{a} P' \in \meaningof{E}\} }
\end{mathpar}

\begin{mathpar}
 \inferrule* [lab=nominal] {} {\meaningof{\quotep{E}} = \{ \quotep{P} \in \quotep{\pi} | P \in \meaningof{E} \}, \and \meaningof{\quotep{P}} = \{ \quotep{Q} \in \quotep{\pi} | P \equiv Q \} \and \\ \meaningof{@\quotep{E}} = \{ P \in \pi | P \equiv @x, x \in \meaningof{E} \}}
\end{mathpar}

\begin{eqnarray*}
  \\
  \meaningof{-} : TS \to ST
\end{eqnarray*}

\begin{eqnarray*}
  \\
  L : TS \to ST
\end{eqnarray*}

\begin{eqnarray*}
  \\
  P \models E \iff P \in \meaningof{E}
\end{eqnarray*}

\begin{eqnarray*}
  P \approx_{L} Q \iff \forall E \in L. P \models E \iff Q \models E
\end{eqnarray*}

\begin{eqnarray*}
  P \approx_{K} Q
\end{eqnarray*}

\begin{eqnarray*}
  P \approx Q
\end{eqnarray*}

$\approx_{K} = \approx = \approx_{L}$

\subsubsection{Contextual duality}

Note that contexts extend the quotation operation to a family of
operations from processes to names. Given a context, $M$, we can
define a \emph{nominal context}, $\quotep{M}$ by $\quotep{M}[P] :=
\quotep{M[P]}$. To foreshadow what is to come we observe that these
operations enjoy a duality with processes very much like the duality
between vectors and maps from vectors to scalars.

Further, because the calculus is essentially higher-order, we have a
correspondence between contexts and processes. More specifically,
given a name $x$ and a context $M$ we can construct $M^{*}_{x}$ such
that 

\begin{mathpar}
  M^{*}_{x} | \lift{x}{P} \red M[P]
\end{mathpar}

namely,

\begin{mathpar}
  M^{*}_{x} := x?(u).M[\dropn{u}]
\end{mathpar}

The dependence of $M^{*}_{x}$ on a name makes it an abstraction, 

\begin{mathpar}
  M^{*} := (x)x?(u).M[\dropn{u}]
\end{mathpar}

\subsection{Additional notation}

It will sometimes be convenient to denote the process a name
quotes. We already have the notation $x = \quotep{P}$, but it will be
convenient to introduce an alternate notation, $\procn{x}$, when we
want to emphasize the connection to the use of the name. Note that, by
virtue of name equivalence, $\quotep{\procn{x}} \nameeq x$; so, the
notation is consistent with previous definitions.

Further, because names have structure it is possible to effect
substitutions on the basis of that structure. This means we need to
upgrade our notation for substitutions, which we accomplish by
adapting comprehension notation. Thus,

\begin{mathpar}
  P\{ y / x : x \in S \}
\end{mathpar}

is interpreted to mean the process derived from P by replacing (in a
capture-avoiding manner) each occurrence of $x$ in $S$ by $y$. For example,

\begin{mathpar}
  P\{ \quotep{\procn{x}|\procn{x}} / x : x \in \freenames{P} \}
\end{mathpar}

will replace each (occurrence) of a free name $x$ in $P$ by
$\quotep{\procn{x}|\procn{x}}$.

Also, we will avail ourselves of the notation $x^{L}$ and $x^{R}$ to
denote injections of a name into disjoint copies of the name
space. There are numerous ways to accomplish this. One example can be
found in \cite{MeredithR05}. This notation overloads to vectors of
names: $\vec{x}^{\pi} := (x_{i}^{\pi} \; : \; 0 \leq i < |\vec{x}| )$ where $\pi \in \{L,R\}$.

We also use $P^{\Box} := P|\Box$.

In \cite{MeredithR05} an interpretation of the new operator is
given. It turns out that there are several possible interpretations
all enjoying the requisite algebraic properties of the operator (see
\cite{milner91polyadicpi}). We will therefore make liberal use of
$(\nu\; \vec{x})P$.

% subsection the_syntax_and_semantics_of_the_notation_system (end)   

\input{qm2pi.qmops} 

\input{qm2pi.sterngerlach} 

\input{qm2pi.metric} 

% section concurrent_process_calculi (end)

%\input{qm2pi.proofsketch}

% section proof sketch (end)

%\input{qm2pi.slviaknots} 

% section spatial logic via knots (end)

\input{qm2pi.conclusion}

% section conclusion (end)

%\input{qm2pi.dtcodes} 

% section wiring algorithm (end)

\input{qm2pi.ack} 

% section acknowledgments (end)

\newpage


\bibliographystyle{plain}   
\bibliography{../../biblios/main.bib}

\input{qm2pi.rhodetails}

\end{document}

 

% section acknowledgments (end)

\newpage


\bibliographystyle{plain}   
\bibliography{../../biblios/main.bib}

\documentclass[12pt]{llncs}
%\documentclass{jktr}

\usepackage[pdftex]{hyperref}                   
\usepackage {listings}
\usepackage {mathpartir}
\usepackage{bcprules}
%\usepackage{listings}
                       
\usepackage{graphicx} 
%\usepackage[margins=2.5cm,nohead,nofoot]{geometry}
%\usepackage{geometry}
\usepackage{amsfonts}
\usepackage{amstext}
\usepackage{latexsym}
\usepackage{amssymb}
\usepackage{color}


%\include{myPreamble}
\include{qm2pi.local} 

%\ifpdf
%\usepackage[pdftex]{graphicx}
%\else
%\usepackage{graphicx}
%\fi

 % \ifpdf
%  \usepackage{pdfsync}
%  \if


%\title{Brief Article}
%\author{David F. Snyder}
%\author{L.G. Meredith}

%\address{Dept. of Math., Texas State University--San Marcos, San Marcos, TX 78666}
       
\pagestyle{empty}


\begin{document}

\lstset{language=[Objective]Caml,frame=shadowbox}

\input{qm2pi.front}

% section front matter (end)

\input{qm2pi.intro} 
 
% section introduction (end)

% \input{qm2pi.knotations} 

% section notation (end)

\input{qm2pi.process.calculi} 

% section concurrent_process_calculi_and_spatial_logics_ (end)
    
%\input{qm2pi.knots2pi} 

%\input{qm2pi.trefoil} 

%\input{qm2pi.mainthm} 

% subsection basic_interpretation (end)

%\input{qm2pi.rho.presentation} 
\subsection{The syntax and semantics of the notation system}\label{sub:the_syntax_and_semantics_of_the_notation_system} % (fold)

We now summarize a technical presentation of the calculus that
embodies our theory of dynamics. The typical presentation of such a
calculus follows the style of giving generators and relations on
them. The grammar, below, describing term constructors, freely
generates the set of processes, $\Proc$. This set is then quotiented
by a relation known as structural congruence and it is over this set
that the notion of dynamics is expressed. This presentation is
essentially that of \cite{MeredithR05} with the addition of
polyadicity and summation. For readability we have relegated some of
the technical subtleties to an appendix.

\subsubsection{Process grammar}\label{subsub:process_grammar}

\begin{mathpar}
  \inferrule* [lab=synchronization] {} {{M} \bc \pzero \;|\; x?F \;|\; x!C }
  \and
  \inferrule* [lab=abstraction] {} {{F} \bc (x)P}
  \and
  \inferrule* [lab=concretion] {} {{C} \bc \langle Q \rangle}
  \and
  \inferrule* [lab=process] {} {{P,Q} \bc M \;| \;P|Q \;|\; @{x}}
  \and
  \inferrule* [lab=name] {} {{x} \bc \quotep{P}}
\end{mathpar} 

Note that $\vec{x}$ (resp. $\vec{P}$) denotes a vector of names
(resp. processes) of length $|\vec{x}|$ (resp. $|\vec{P}|$). We adopt
the following useful abbreviations.

\begin{mathpar}
   x?(\vec{y}).P := x.(\vec{y})P \and  x\clift{\vec{P}} := x.\clift{\vec{P}}
   \and x!(y) := \lift{x}{\dropn{y}}
   \and \Pi_{i=0}^{n-1}P_i := P_0 | \ldots | P_{n-1}
\end{mathpar}

\subsubsection{Structural congruence}

\paragraph{Free and bound names and alpha-equivalence.} At the
core of structural equivalence is alpha-equivalence which identifies
process that are the same up to a change of variable. Formally, we
recognize the distinction between free and bound names. The free names
of a process, $\freenames{P}$, may be calculated recursively as
follows:

\begin{mathpar}
\freenames{\pzero} := \emptyset
  \and \\
  \freenames{x?(y).P} := \{ x \} \cup (\freenames{P} \setminus \{ y \})
  \and 
  \freenames{x!\langle P \rangle} := \{ x \} \cup \{ P \} 
  \and \\
  \freenames{P|Q} := \freenames{P} \cup \freenames{Q}
  \and \\
  \freenames{@{x}} := \{ x \}
\end{mathpar}

$\pi$
$\quotep{\pi}$

$\freenames{-} : \pi \to \mathcal{P}(\quotep{\pi})$

\begin{eqnarray*}
  \freenames{\pzero} & := & \emptyset \\
  \freenames{x?(y).P} & := & \{ x \} \cup (\freenames{P} \setminus \{ y \}) \\
  \freenames{x!\langle P \rangle} & := & \{ x \} \cup \{ P \} \\
  \freenames{P|Q} & := & \freenames{P} \cup \freenames{Q} \\
  \freenames{\dropn{x}} & := & \{ x \}
\end{eqnarray*}

The bound names of a process, $\boundnames{P}$, are those names occurring in $P$
that are not free. For example, in $x?(y).0$, the name $x$ is free, while $y$ is bound.

\begin{mathpar}
  \inferrule* [lab=monoidal-laws] {} { P|Q \equiv Q|P \and P|0 \equiv P \and P|(Q|R) \equiv (P|Q)|R }
\end{mathpar}

\begin{mathpar}
  \inferrule* [lab=alpha-equivalence] {} { (x)P \equiv (y)P\{y/x\} \and y \not\in \freenames{P} }
\end{mathpar}

\begin{definition}
Then two processes, $P,Q$, are alpha-equivalent if $P = Q\{\vec{y}/\vec{x}\}$ for
some $\vec{x} \in \boundnames{Q},\vec{y} \in \boundnames{P}$, where $Q\{\vec{y}/\vec{x}\}$
denotes the capture-avoiding substitution of $\vec{y}$ for $\vec{x}$ in $Q$.
\end{definition}

\begin{definition}
  The {\em structural congruence} \cite{SangiorgiWalker} , $\equiv$,
  between processes is the least congruence containing
  alpha-equivalence, satisfying the abelian monoid laws
  (associativity, commutativity and $\pzero$ as identity) for parallel
  composition $|$ and for summation $+$.
\end{definition}

\subsection{Name equivalence}

We take name equivalence, written $\nameeq$, to be the smallest
equivalence relation generated by the following rules.

\begin{mathpar}
\inferrule*[lab=Quote-drop]
{ }
{ \quotep{@{x}} \nameeq x }

\inferrule*[lab=Struct-equiv]
{ P \scong Q }
{ \quotep{P} \nameeq \quotep{Q} }
\end{mathpar}

The astute reader will have noticed that the mutual recursion of names
and processes imposes a mutual recursion on alpha-equivalence and
structural equivalence via name-equivalence. Fortunately, all of this
works out pleasantly and we may calculate in the natural way, free of
concern. The reader interested in the details is referred to the
appendix \ref{appendix:rho_details}.

\subsection{Substitution}

We use $\Proc$ for the set of processes, $\QProc$ for the set of
names, and $\id{\{}\vec{y} / \vec{x} \id{\}}$ to denote partial maps,
$s : \QProc \rightarrow \QProc$. A map, $s$ lifts, uniquely, to a map
on process terms, $\widehat{s} : \Proc \rightarrow \Proc$ by the
following equations.

\begin{mathpar}
  (0) \psubstp{Q}{P} := 0 \\
  (R \juxtap S) \psubstp{Q}{P}
  :=    
  (R)\psubstp{Q}{P} \juxtap (S) \psubstp{Q}{P} \\
  (x?(y).R) \psubstp{Q}{P}    
  :=    
  (x)\substp{Q}{P} (z)\concat( (R \psubstn{z}{y}) \psubstp{Q}{P} ) \\
  (\lift{x}{R}) \psubstp{Q}{P}  
  :=
  \lift{(x)\substp{Q}{P}}{ R \psubstp{Q}{P} } \\
%   (\dropn{x})  \psubstp{Q}{P}       
%   := 
%   \left\{ 
%     \begin{array}{ccc} 
%       \dropn{\quotep{Q}} & & x \nameeq \quotep{P} \\
%       \dropn{x} & & otherwise \\
%     \end{array}
%   \right. 
  (\dropn{x})  \psubstp{Q}{P}       
  := 
  \left\{ 
    \begin{array}{ccc} 
      Q & & x \nameeq \quotep{P} \\
      \dropn{x} & & otherwise \\
    \end{array}
  \right.
\end{mathpar}
 

where

\begin{eqnarray}
  (x)\id{\{} \lpquote Q \rpquote / \lpquote P \rpquote \id{\}}            = 
  \left\{ 
    \begin{array}{ccc}
      \lpquote Q \rpquote & & x \nameeq \lpquote P \rpquote \\
      x & & otherwise \\
    \end{array}
  \right. \nonumber
\end{eqnarray}

and $z$ is chosen distinct from $\quotep{P}$, $\quotep{Q}$, the free
names in $Q$, and all the names in $R$. Our $\alpha$-equivalence will
be built in the standard way from this substitution.

\begin{remark}\label{rem:no_self_referential_names}
  One consequence of these definitions is that $\forall P. \quotep{P}
  \not\in \freenames{P}$.
\end{remark}

\subsection{ Dynamic quote: an example }

Anticipating something of what's to come, consider applying the
substitution, $\widehat{\id{\{}u / z \id{\}}}$, to the following pair
of processes, $\lift{w}{y!(z)}$ and $w[ \lpquote y!(z) \rpquote ]$.

\begin{eqnarray}
	\lift{w}{y!(z)}\widehat{\id{\{}u / z \id{\}}}
		& = &
		\lift{w}{y!(u)} \nonumber\\
	w[ \lpquote y!(z) \rpquote ] \widehat{ \id{\{}u / z \id{\}} }
		& = &
		w[ \lpquote y!(z) \rpquote ] \nonumber
\end{eqnarray}

Because the body of the process between quotes is impervious to
substitution, we get radically different answers. In fact, by
examining the first process in an input context,
e.g. $x?(z).\lift{w}{y!(z)}$, we see that the process under the lift
operator may be shaped by prefixed inputs binding a name inside it. In
this sense, the lift operator will be seen as a way to dynamically
construct processes before reifying them as names.

Finally equipped with these standard features we can present the
dynamics of the calculus.

\subsubsection{Operational semantics} 

Finally, we introduce the computational dynamics. What marks these
algebras as distinct from other more traditionally studied algebraic
structures, e.g. vector spaces or polynomial rings, is the manner in
which dynamics is captured. In traditional structures, dynamics is typically
expressed through morphisms between such structures, as in linear maps
between vector spaces or morphisms between rings. In algebras
associated with the semantics of computation, the dynamics is
expressed as part of the algebraic structure itself, through a
reduction reduction relation typically denoted by $\red$. Below, we
give a recursive presentation of this relation for the calculus used
in the encoding.

$\red \subseteq \pi \times \pi$
$\red : \pi \to \mathcal{P}(\pi)$

\begin{mathpar}
  \inferrule* [lab=Comm] { \textsf{match}( x_{src}, x_{trgt} ) } { x_{trgt}?(y)P \; | \; x_{src}!\langle {Q} \rangle \red P\{\quotep{Q}/y}\} }
  \and \\
  \inferrule* [lab=Par] {{P} \red {P}'} {{{P} | {Q}} \red {{P}' | {Q}}}
  \and
  \inferrule* [lab=Equiv]{{{P} \scong {P}'} \andalso {{P}' \red {Q}'} \andalso {{Q}' \scong {Q}}}{{P} \red {Q}}
\end{mathpar}

\begin{eqnarray*}
  match_{\equiv} (\quotep{P},\quotep{Q}) & := & P \equiv Q \\
  match_{\dagger}(\quotep{P},\quotep{Q}) & := & \forall R. P|Q \red^{*} R => R \red^{*} 0 \\
  match_{K}(\quotep{P},\quotep{Q}) & := & K \mbox{ for some context } K
\end{eqnarray*}

$u?(x)P | u!\langle Q \rangle \red P\{\quotep{Q}/x\}$

%We write $\wred$ for $\red^*$, and $P\red$ if $\exists Q $ such that $ P \red Q$.
We write $P\red$ if $\exists Q $ such that $ P \red Q$ and $P\not\red$, otherwise.

\section{Replication}

As mentioned before, it is known that replication (and hence
recursion) can be implemented in a higher-order process algebra
\cite{SangiorgiWalker}. As our first example of calculation with the
machinery thus far presented we give the construction explicitly in
the {\rhoc}.

\begin{eqnarray}
	D_{x} & := & \prefix{x}{y}{(\binpar{\outputp{x}{y}}{@{y}})} \nonumber\\
	\bangp_{x}{P} & := & \binpar{{x}!\langle{\binpar{D_{x}}{P}}\rangle}{D_{x}} \nonumber
\end{eqnarray}

\begin{eqnarray}
	\bangp_{x}{P} & & \nonumber\\
	=
	& {x}!\langle{(\prefix{x}{y}{(\outputp{x}{y} | @{y})) | P}}\rangle 
	      | \prefix{x}{y}{(\outputp{x}{y} | @{y})} & \nonumber\\
	\red
	& (\outputp{x}{y} | @{y})\substn{\quotep{(\prefix{x}{y}{(@{y} | \outputp{x}{y})) | P}}}{y} & \nonumber\\
	=
	& \outputp{x}{\quotep{(\prefix{x}{y}{(\outputp{x}{y} | @{y})) | P}}}
	  | {(\prefix{x}{y}{(\outputp{x}{y} | @{y})) | P}} & \nonumber\\
	\red
	& \ldots & \nonumber\\
	\red^*
	& P | P | \ldots & \nonumber
\end{eqnarray}

Of course, this encoding, as an implementation, runs away, unfolding
$\bangp{P}$ eagerly. A lazier and more implementable replication
operator, restricted to input-guarded processes, may be obtained as follows.

\begin{eqnarray}
\bangp{\prefix{u}{v}{P}} 
	:= 
	\binpar{\lift{x}{\prefix{u}{v}{(\binpar{D(x)}{P})}}}{D(x)} \nonumber
\end{eqnarray}

\begin{remark}
  Note that the lazier definition still does not deal with summation
  or mixed summation (i.e. sums over input and output). The reader is
  invited to construct definitions of replication that deal with these
  features. 

  Further, the definitions are parameterized in a name, $x$. Can you,
  gentle reader, make a definition that eliminates this parameter and
  guarantees no accidental interaction between the replication
  machinery and the process being replicated -- i.e. no accidental
  sharing of names used by the process to get its work done and the
  name(s) used by the replication to effect copying. This latter
  revision of the definition of replication is crucial to obtaining
  the expected identity $!!P \sim !P$.
\end{remark}

\begin{remark}\label{rem:paradoxical_combinator}
  The reader familiar with the lambda calculus will have noticed the
  similarity between $D$ and the paradoxical combinator.

  [Ed. note: the existence of this seems to suggest we have to be more
  restrictive on the set of processes and names we admit if we are to
  support no-cloning.]
\end{remark}

\subsubsection{Bisimulation}

The computational dynamics gives rise to another kind of equivalence,
the equivalence of computational behavior. As previously mentioned
this is typically captured \emph{via} some form of bisimulation.

% The notion we use in this paper is weak barbed bisimulation
% \cite{milner91polyadicpi}.

The notion we use in this paper is derived from weak barbed
bisimulation \cite{milner91polyadicpi}. 

\begin{definition}
An \emph{observation relation}, $\downarrow_{\mathcal N}$, over a set
of names, $\mathcal N$, is the smallest relation satisfying the rules
below.

\infrule[Out-barb]{y \in {\mathcal N}, \; x \nameeq y}
		  {\outputp{x}{v} \downarrow_{\mathcal N} x}
\infrule[Par-barb]{\mbox{$P\downarrow_{\mathcal N} x$ or $Q\downarrow_{\mathcal N} x$}}
		  {\binpar{P}{Q} \downarrow_{\mathcal N} x}

We write $P \Downarrow_{\mathcal N} x$ if there is $Q$ such that 
$P \wred Q$ and $Q \downarrow_{\mathcal N} x$.
\end{definition}

\begin{definition}
%\label{def.bbisim}
An  ${\mathcal N}$-\emph{barbed bisimulation} over a set of names, ${\mathcal N}$, is a symmetric binary relation 
${\mathcal S}_{\mathcal N}$ between agents such that $P\rel{S}_{\mathcal N}Q$ implies:
\begin{enumerate}
\item If $P \red P'$ then $Q \wred Q'$ and $P'\rel{S}_{\mathcal N} Q'$.
\item If $P\downarrow_{\mathcal N} x$, then $Q\Downarrow_{\mathcal N} x$.
\end{enumerate}
$P$ is ${\mathcal N}$-barbed bisimilar to $Q$, written
$P \wbbisim_{\mathcal N} Q$, if $P \rel{S}_{\mathcal N} Q$ for some ${\mathcal N}$-barbed bisimulation ${\mathcal S}_{\mathcal N}$.
\end{definition}

$\mathcal{R} \subseteq \pi \times \pi$

$P \mathcal{R} Q => \forall P'. P \red P' \Rightarrow \exists Q'. Q \red Q', P' \mathcal{R} Q'$

$P \vdash x \Rightarrow Q \vdash x$

\begin{mathpar}
  \inferrule*[lab=Out-barb]{x \nameeq y}{{y}!\langle{Q}\rangle \vdash x}
  \and
  \inferrule*[lab=Par-barb]{\mbox{$P\vdash x$ or $Q\vdash x$}}{\binpar{P}{Q} \vdash x}
\end{mathpar}

\subsubsection{Contexts}

One of the principle advantages of computational calculi like the
$\pi$-calculus is a well-defined notion of context,
contextual-equivalence and a correlation between
contextual-equivalence and notions of bisimulation. The notion of
context allows the decomposition of a process into (sub-)process and
its syntactic environment, its context. Thus, a context may be
thought of as a process with a ``hole'' (written $\Box$) in it. The
application of a context $M$ to a process $P$, written $M[P]$, is
tantamount to filling the hole in $M$ with $P$. In this paper we do
not need the full weight of this theory, but do make use of the notion
of context in the proof the main theorem. 

\begin{mathpar}
  \inferrule* [lab=summation] {} {{M_{M},M_{N}} \bc \Box \;|\; x.M_{A} \;|\; M_{M}+M_{N}}
  \and
  \inferrule* [lab=agent] {} {{M_{A}} \bc (\vec{x})M_{P} \;| \; \clift{P_0,\ldots,M_{P},\ldots,P_N}}
  \and \\
  \inferrule* [lab=process] {} {{M_{P}} \bc M_{N} \;| \;P|M_{P} }
\end{mathpar} 

\begin{mathpar}
  \inferrule* [lab=sychronization] {} {M_{N} \bc \Box \;|\; x?M_{F} \;|\; x!M_{C}}
  \and
  \inferrule* [lab=abstraction] {} {{M_{F}} \bc (x)M_{P} }
  \and
  \inferrule* [lab=concretion] {} {{M_{C}} \bc \langle M_{P} \rangle }
  \and \\
  \inferrule* [lab=process] {} {{M_{P}} \bc M_{N} \;| \;P|M_{P} }
\end{mathpar}

\begin{definition}[contextual application] Given a context $M$, and
  process $P$, we define the \emph{contextual application}, $M[P] :=
  M\{P/\Box\}$. That is, the contextual application of M to P is the
  substitution of $P$ for $\Box$ in $M$.
\end{definition}

$\meaningof{-} : L \to \mathcal{P}(\pi)$

\begin{mathpar}
  \inferrule* [lab=collection] {} {\meaningof{true} = \pi, \and \meaningof{~E} = \pi \setminus \meaningof{E}, \and \meaningof{E_{1} \& E_{2}} = \meaningof{E_{1}} \cap \meaningof{E_{2}}}
\end{mathpar}

\begin{mathpar}
  \inferrule* [lab=structure] {} {\meaningof{0} = \{ P \in \pi | P \equiv 0 \}, \and \\ \meaningof{E_1 | E_2} = \{ P \in \pi | P \equiv P_{1} | P_{2}, P_{1} \in \meaningof{E_{1}}, P_{2} \in \meaningof{E_2}\} }
\end{mathpar}

\begin{mathpar}
 \inferrule* [lab=behavior] {} {\meaningof{\langle a?b \rangle E} = \{ P \in \pi | P \equiv Q | u?(y)P', \\ \and \\\\ \and \\ \;\;\; u \in \meaningof{a}, \forall z.P'\{z/y\} \in \meaningof{E\{z/b\}}\}, \and \\ \meaningof{a!E} = \{ P \in \pi | P \equiv Q | x!\langle P' \rangle, x \in \meaningof{a} P' \in \meaningof{E}\} }
\end{mathpar}

\begin{mathpar}
 \inferrule* [lab=nominal] {} {\meaningof{\quotep{E}} = \{ \quotep{P} \in \quotep{\pi} | P \in \meaningof{E} \}, \and \meaningof{\quotep{P}} = \{ \quotep{Q} \in \quotep{\pi} | P \equiv Q \} \and \\ \meaningof{@\quotep{E}} = \{ P \in \pi | P \equiv @x, x \in \meaningof{E} \}}
\end{mathpar}

\begin{eqnarray*}
  \\
  \meaningof{-} : TS \to ST
\end{eqnarray*}

\begin{eqnarray*}
  \\
  L : TS \to ST
\end{eqnarray*}

\begin{eqnarray*}
  \\
  P \models E \iff P \in \meaningof{E}
\end{eqnarray*}

\begin{eqnarray*}
  P \approx_{L} Q \iff \forall E \in L. P \models E \iff Q \models E
\end{eqnarray*}

\begin{eqnarray*}
  P \approx_{K} Q
\end{eqnarray*}

\begin{eqnarray*}
  P \approx Q
\end{eqnarray*}

$\approx_{K} = \approx = \approx_{L}$

\subsubsection{Contextual duality}

Note that contexts extend the quotation operation to a family of
operations from processes to names. Given a context, $M$, we can
define a \emph{nominal context}, $\quotep{M}$ by $\quotep{M}[P] :=
\quotep{M[P]}$. To foreshadow what is to come we observe that these
operations enjoy a duality with processes very much like the duality
between vectors and maps from vectors to scalars.

Further, because the calculus is essentially higher-order, we have a
correspondence between contexts and processes. More specifically,
given a name $x$ and a context $M$ we can construct $M^{*}_{x}$ such
that 

\begin{mathpar}
  M^{*}_{x} | \lift{x}{P} \red M[P]
\end{mathpar}

namely,

\begin{mathpar}
  M^{*}_{x} := x?(u).M[\dropn{u}]
\end{mathpar}

The dependence of $M^{*}_{x}$ on a name makes it an abstraction, 

\begin{mathpar}
  M^{*} := (x)x?(u).M[\dropn{u}]
\end{mathpar}

\subsection{Additional notation}

It will sometimes be convenient to denote the process a name
quotes. We already have the notation $x = \quotep{P}$, but it will be
convenient to introduce an alternate notation, $\procn{x}$, when we
want to emphasize the connection to the use of the name. Note that, by
virtue of name equivalence, $\quotep{\procn{x}} \nameeq x$; so, the
notation is consistent with previous definitions.

Further, because names have structure it is possible to effect
substitutions on the basis of that structure. This means we need to
upgrade our notation for substitutions, which we accomplish by
adapting comprehension notation. Thus,

\begin{mathpar}
  P\{ y / x : x \in S \}
\end{mathpar}

is interpreted to mean the process derived from P by replacing (in a
capture-avoiding manner) each occurrence of $x$ in $S$ by $y$. For example,

\begin{mathpar}
  P\{ \quotep{\procn{x}|\procn{x}} / x : x \in \freenames{P} \}
\end{mathpar}

will replace each (occurrence) of a free name $x$ in $P$ by
$\quotep{\procn{x}|\procn{x}}$.

Also, we will avail ourselves of the notation $x^{L}$ and $x^{R}$ to
denote injections of a name into disjoint copies of the name
space. There are numerous ways to accomplish this. One example can be
found in \cite{MeredithR05}. This notation overloads to vectors of
names: $\vec{x}^{\pi} := (x_{i}^{\pi} \; : \; 0 \leq i < |\vec{x}| )$ where $\pi \in \{L,R\}$.

We also use $P^{\Box} := P|\Box$.

In \cite{MeredithR05} an interpretation of the new operator is
given. It turns out that there are several possible interpretations
all enjoying the requisite algebraic properties of the operator (see
\cite{milner91polyadicpi}). We will therefore make liberal use of
$(\nu\; \vec{x})P$.

% subsection the_syntax_and_semantics_of_the_notation_system (end)   

\input{qm2pi.qmops} 

\input{qm2pi.sterngerlach} 

\input{qm2pi.metric} 

% section concurrent_process_calculi (end)

%\input{qm2pi.proofsketch}

% section proof sketch (end)

%\input{qm2pi.slviaknots} 

% section spatial logic via knots (end)

\input{qm2pi.conclusion}

% section conclusion (end)

%\input{qm2pi.dtcodes} 

% section wiring algorithm (end)

\input{qm2pi.ack} 

% section acknowledgments (end)

\newpage


\bibliographystyle{plain}   
\bibliography{../../biblios/main.bib}

\input{qm2pi.rhodetails}

\end{document}



\end{document}

 

%\documentclass[12pt]{llncs}
%\documentclass{jktr}

\usepackage[pdftex]{hyperref}                   
\usepackage {listings}
\usepackage {mathpartir}
\usepackage{bcprules}
%\usepackage{listings}
                       
\usepackage{graphicx} 
%\usepackage[margins=2.5cm,nohead,nofoot]{geometry}
%\usepackage{geometry}
\usepackage{amsfonts}
\usepackage{amstext}
\usepackage{latexsym}
\usepackage{amssymb}
\usepackage{color}


%\include{myPreamble}
\documentclass[12pt]{llncs}
%\documentclass{jktr}

\usepackage[pdftex]{hyperref}                   
\usepackage {listings}
\usepackage {mathpartir}
\usepackage{bcprules}
%\usepackage{listings}
                       
\usepackage{graphicx} 
%\usepackage[margins=2.5cm,nohead,nofoot]{geometry}
%\usepackage{geometry}
\usepackage{amsfonts}
\usepackage{amstext}
\usepackage{latexsym}
\usepackage{amssymb}
\usepackage{color}


%\include{myPreamble}
\include{qm2pi.local} 

%\ifpdf
%\usepackage[pdftex]{graphicx}
%\else
%\usepackage{graphicx}
%\fi

 % \ifpdf
%  \usepackage{pdfsync}
%  \if


%\title{Brief Article}
%\author{David F. Snyder}
%\author{L.G. Meredith}

%\address{Dept. of Math., Texas State University--San Marcos, San Marcos, TX 78666}
       
\pagestyle{empty}


\begin{document}

\lstset{language=[Objective]Caml,frame=shadowbox}

\input{qm2pi.front}

% section front matter (end)

\input{qm2pi.intro} 
 
% section introduction (end)

% \input{qm2pi.knotations} 

% section notation (end)

\input{qm2pi.process.calculi} 

% section concurrent_process_calculi_and_spatial_logics_ (end)
    
%\input{qm2pi.knots2pi} 

%\input{qm2pi.trefoil} 

%\input{qm2pi.mainthm} 

% subsection basic_interpretation (end)

%\input{qm2pi.rho.presentation} 
\subsection{The syntax and semantics of the notation system}\label{sub:the_syntax_and_semantics_of_the_notation_system} % (fold)

We now summarize a technical presentation of the calculus that
embodies our theory of dynamics. The typical presentation of such a
calculus follows the style of giving generators and relations on
them. The grammar, below, describing term constructors, freely
generates the set of processes, $\Proc$. This set is then quotiented
by a relation known as structural congruence and it is over this set
that the notion of dynamics is expressed. This presentation is
essentially that of \cite{MeredithR05} with the addition of
polyadicity and summation. For readability we have relegated some of
the technical subtleties to an appendix.

\subsubsection{Process grammar}\label{subsub:process_grammar}

\begin{mathpar}
  \inferrule* [lab=synchronization] {} {{M} \bc \pzero \;|\; x?F \;|\; x!C }
  \and
  \inferrule* [lab=abstraction] {} {{F} \bc (x)P}
  \and
  \inferrule* [lab=concretion] {} {{C} \bc \langle Q \rangle}
  \and
  \inferrule* [lab=process] {} {{P,Q} \bc M \;| \;P|Q \;|\; @{x}}
  \and
  \inferrule* [lab=name] {} {{x} \bc \quotep{P}}
\end{mathpar} 

Note that $\vec{x}$ (resp. $\vec{P}$) denotes a vector of names
(resp. processes) of length $|\vec{x}|$ (resp. $|\vec{P}|$). We adopt
the following useful abbreviations.

\begin{mathpar}
   x?(\vec{y}).P := x.(\vec{y})P \and  x\clift{\vec{P}} := x.\clift{\vec{P}}
   \and x!(y) := \lift{x}{\dropn{y}}
   \and \Pi_{i=0}^{n-1}P_i := P_0 | \ldots | P_{n-1}
\end{mathpar}

\subsubsection{Structural congruence}

\paragraph{Free and bound names and alpha-equivalence.} At the
core of structural equivalence is alpha-equivalence which identifies
process that are the same up to a change of variable. Formally, we
recognize the distinction between free and bound names. The free names
of a process, $\freenames{P}$, may be calculated recursively as
follows:

\begin{mathpar}
\freenames{\pzero} := \emptyset
  \and \\
  \freenames{x?(y).P} := \{ x \} \cup (\freenames{P} \setminus \{ y \})
  \and 
  \freenames{x!\langle P \rangle} := \{ x \} \cup \{ P \} 
  \and \\
  \freenames{P|Q} := \freenames{P} \cup \freenames{Q}
  \and \\
  \freenames{@{x}} := \{ x \}
\end{mathpar}

$\pi$
$\quotep{\pi}$

$\freenames{-} : \pi \to \mathcal{P}(\quotep{\pi})$

\begin{eqnarray*}
  \freenames{\pzero} & := & \emptyset \\
  \freenames{x?(y).P} & := & \{ x \} \cup (\freenames{P} \setminus \{ y \}) \\
  \freenames{x!\langle P \rangle} & := & \{ x \} \cup \{ P \} \\
  \freenames{P|Q} & := & \freenames{P} \cup \freenames{Q} \\
  \freenames{\dropn{x}} & := & \{ x \}
\end{eqnarray*}

The bound names of a process, $\boundnames{P}$, are those names occurring in $P$
that are not free. For example, in $x?(y).0$, the name $x$ is free, while $y$ is bound.

\begin{mathpar}
  \inferrule* [lab=monoidal-laws] {} { P|Q \equiv Q|P \and P|0 \equiv P \and P|(Q|R) \equiv (P|Q)|R }
\end{mathpar}

\begin{mathpar}
  \inferrule* [lab=alpha-equivalence] {} { (x)P \equiv (y)P\{y/x\} \and y \not\in \freenames{P} }
\end{mathpar}

\begin{definition}
Then two processes, $P,Q$, are alpha-equivalent if $P = Q\{\vec{y}/\vec{x}\}$ for
some $\vec{x} \in \boundnames{Q},\vec{y} \in \boundnames{P}$, where $Q\{\vec{y}/\vec{x}\}$
denotes the capture-avoiding substitution of $\vec{y}$ for $\vec{x}$ in $Q$.
\end{definition}

\begin{definition}
  The {\em structural congruence} \cite{SangiorgiWalker} , $\equiv$,
  between processes is the least congruence containing
  alpha-equivalence, satisfying the abelian monoid laws
  (associativity, commutativity and $\pzero$ as identity) for parallel
  composition $|$ and for summation $+$.
\end{definition}

\subsection{Name equivalence}

We take name equivalence, written $\nameeq$, to be the smallest
equivalence relation generated by the following rules.

\begin{mathpar}
\inferrule*[lab=Quote-drop]
{ }
{ \quotep{@{x}} \nameeq x }

\inferrule*[lab=Struct-equiv]
{ P \scong Q }
{ \quotep{P} \nameeq \quotep{Q} }
\end{mathpar}

The astute reader will have noticed that the mutual recursion of names
and processes imposes a mutual recursion on alpha-equivalence and
structural equivalence via name-equivalence. Fortunately, all of this
works out pleasantly and we may calculate in the natural way, free of
concern. The reader interested in the details is referred to the
appendix \ref{appendix:rho_details}.

\subsection{Substitution}

We use $\Proc$ for the set of processes, $\QProc$ for the set of
names, and $\id{\{}\vec{y} / \vec{x} \id{\}}$ to denote partial maps,
$s : \QProc \rightarrow \QProc$. A map, $s$ lifts, uniquely, to a map
on process terms, $\widehat{s} : \Proc \rightarrow \Proc$ by the
following equations.

\begin{mathpar}
  (0) \psubstp{Q}{P} := 0 \\
  (R \juxtap S) \psubstp{Q}{P}
  :=    
  (R)\psubstp{Q}{P} \juxtap (S) \psubstp{Q}{P} \\
  (x?(y).R) \psubstp{Q}{P}    
  :=    
  (x)\substp{Q}{P} (z)\concat( (R \psubstn{z}{y}) \psubstp{Q}{P} ) \\
  (\lift{x}{R}) \psubstp{Q}{P}  
  :=
  \lift{(x)\substp{Q}{P}}{ R \psubstp{Q}{P} } \\
%   (\dropn{x})  \psubstp{Q}{P}       
%   := 
%   \left\{ 
%     \begin{array}{ccc} 
%       \dropn{\quotep{Q}} & & x \nameeq \quotep{P} \\
%       \dropn{x} & & otherwise \\
%     \end{array}
%   \right. 
  (\dropn{x})  \psubstp{Q}{P}       
  := 
  \left\{ 
    \begin{array}{ccc} 
      Q & & x \nameeq \quotep{P} \\
      \dropn{x} & & otherwise \\
    \end{array}
  \right.
\end{mathpar}
 

where

\begin{eqnarray}
  (x)\id{\{} \lpquote Q \rpquote / \lpquote P \rpquote \id{\}}            = 
  \left\{ 
    \begin{array}{ccc}
      \lpquote Q \rpquote & & x \nameeq \lpquote P \rpquote \\
      x & & otherwise \\
    \end{array}
  \right. \nonumber
\end{eqnarray}

and $z$ is chosen distinct from $\quotep{P}$, $\quotep{Q}$, the free
names in $Q$, and all the names in $R$. Our $\alpha$-equivalence will
be built in the standard way from this substitution.

\begin{remark}\label{rem:no_self_referential_names}
  One consequence of these definitions is that $\forall P. \quotep{P}
  \not\in \freenames{P}$.
\end{remark}

\subsection{ Dynamic quote: an example }

Anticipating something of what's to come, consider applying the
substitution, $\widehat{\id{\{}u / z \id{\}}}$, to the following pair
of processes, $\lift{w}{y!(z)}$ and $w[ \lpquote y!(z) \rpquote ]$.

\begin{eqnarray}
	\lift{w}{y!(z)}\widehat{\id{\{}u / z \id{\}}}
		& = &
		\lift{w}{y!(u)} \nonumber\\
	w[ \lpquote y!(z) \rpquote ] \widehat{ \id{\{}u / z \id{\}} }
		& = &
		w[ \lpquote y!(z) \rpquote ] \nonumber
\end{eqnarray}

Because the body of the process between quotes is impervious to
substitution, we get radically different answers. In fact, by
examining the first process in an input context,
e.g. $x?(z).\lift{w}{y!(z)}$, we see that the process under the lift
operator may be shaped by prefixed inputs binding a name inside it. In
this sense, the lift operator will be seen as a way to dynamically
construct processes before reifying them as names.

Finally equipped with these standard features we can present the
dynamics of the calculus.

\subsubsection{Operational semantics} 

Finally, we introduce the computational dynamics. What marks these
algebras as distinct from other more traditionally studied algebraic
structures, e.g. vector spaces or polynomial rings, is the manner in
which dynamics is captured. In traditional structures, dynamics is typically
expressed through morphisms between such structures, as in linear maps
between vector spaces or morphisms between rings. In algebras
associated with the semantics of computation, the dynamics is
expressed as part of the algebraic structure itself, through a
reduction reduction relation typically denoted by $\red$. Below, we
give a recursive presentation of this relation for the calculus used
in the encoding.

$\red \subseteq \pi \times \pi$
$\red : \pi \to \mathcal{P}(\pi)$

\begin{mathpar}
  \inferrule* [lab=Comm] { \textsf{match}( x_{src}, x_{trgt} ) } { x_{trgt}?(y)P \; | \; x_{src}!\langle {Q} \rangle \red P\{\quotep{Q}/y}\} }
  \and \\
  \inferrule* [lab=Par] {{P} \red {P}'} {{{P} | {Q}} \red {{P}' | {Q}}}
  \and
  \inferrule* [lab=Equiv]{{{P} \scong {P}'} \andalso {{P}' \red {Q}'} \andalso {{Q}' \scong {Q}}}{{P} \red {Q}}
\end{mathpar}

\begin{eqnarray*}
  match_{\equiv} (\quotep{P},\quotep{Q}) & := & P \equiv Q \\
  match_{\dagger}(\quotep{P},\quotep{Q}) & := & \forall R. P|Q \red^{*} R => R \red^{*} 0 \\
  match_{K}(\quotep{P},\quotep{Q}) & := & K \mbox{ for some context } K
\end{eqnarray*}

$u?(x)P | u!\langle Q \rangle \red P\{\quotep{Q}/x\}$

%We write $\wred$ for $\red^*$, and $P\red$ if $\exists Q $ such that $ P \red Q$.
We write $P\red$ if $\exists Q $ such that $ P \red Q$ and $P\not\red$, otherwise.

\section{Replication}

As mentioned before, it is known that replication (and hence
recursion) can be implemented in a higher-order process algebra
\cite{SangiorgiWalker}. As our first example of calculation with the
machinery thus far presented we give the construction explicitly in
the {\rhoc}.

\begin{eqnarray}
	D_{x} & := & \prefix{x}{y}{(\binpar{\outputp{x}{y}}{@{y}})} \nonumber\\
	\bangp_{x}{P} & := & \binpar{{x}!\langle{\binpar{D_{x}}{P}}\rangle}{D_{x}} \nonumber
\end{eqnarray}

\begin{eqnarray}
	\bangp_{x}{P} & & \nonumber\\
	=
	& {x}!\langle{(\prefix{x}{y}{(\outputp{x}{y} | @{y})) | P}}\rangle 
	      | \prefix{x}{y}{(\outputp{x}{y} | @{y})} & \nonumber\\
	\red
	& (\outputp{x}{y} | @{y})\substn{\quotep{(\prefix{x}{y}{(@{y} | \outputp{x}{y})) | P}}}{y} & \nonumber\\
	=
	& \outputp{x}{\quotep{(\prefix{x}{y}{(\outputp{x}{y} | @{y})) | P}}}
	  | {(\prefix{x}{y}{(\outputp{x}{y} | @{y})) | P}} & \nonumber\\
	\red
	& \ldots & \nonumber\\
	\red^*
	& P | P | \ldots & \nonumber
\end{eqnarray}

Of course, this encoding, as an implementation, runs away, unfolding
$\bangp{P}$ eagerly. A lazier and more implementable replication
operator, restricted to input-guarded processes, may be obtained as follows.

\begin{eqnarray}
\bangp{\prefix{u}{v}{P}} 
	:= 
	\binpar{\lift{x}{\prefix{u}{v}{(\binpar{D(x)}{P})}}}{D(x)} \nonumber
\end{eqnarray}

\begin{remark}
  Note that the lazier definition still does not deal with summation
  or mixed summation (i.e. sums over input and output). The reader is
  invited to construct definitions of replication that deal with these
  features. 

  Further, the definitions are parameterized in a name, $x$. Can you,
  gentle reader, make a definition that eliminates this parameter and
  guarantees no accidental interaction between the replication
  machinery and the process being replicated -- i.e. no accidental
  sharing of names used by the process to get its work done and the
  name(s) used by the replication to effect copying. This latter
  revision of the definition of replication is crucial to obtaining
  the expected identity $!!P \sim !P$.
\end{remark}

\begin{remark}\label{rem:paradoxical_combinator}
  The reader familiar with the lambda calculus will have noticed the
  similarity between $D$ and the paradoxical combinator.

  [Ed. note: the existence of this seems to suggest we have to be more
  restrictive on the set of processes and names we admit if we are to
  support no-cloning.]
\end{remark}

\subsubsection{Bisimulation}

The computational dynamics gives rise to another kind of equivalence,
the equivalence of computational behavior. As previously mentioned
this is typically captured \emph{via} some form of bisimulation.

% The notion we use in this paper is weak barbed bisimulation
% \cite{milner91polyadicpi}.

The notion we use in this paper is derived from weak barbed
bisimulation \cite{milner91polyadicpi}. 

\begin{definition}
An \emph{observation relation}, $\downarrow_{\mathcal N}$, over a set
of names, $\mathcal N$, is the smallest relation satisfying the rules
below.

\infrule[Out-barb]{y \in {\mathcal N}, \; x \nameeq y}
		  {\outputp{x}{v} \downarrow_{\mathcal N} x}
\infrule[Par-barb]{\mbox{$P\downarrow_{\mathcal N} x$ or $Q\downarrow_{\mathcal N} x$}}
		  {\binpar{P}{Q} \downarrow_{\mathcal N} x}

We write $P \Downarrow_{\mathcal N} x$ if there is $Q$ such that 
$P \wred Q$ and $Q \downarrow_{\mathcal N} x$.
\end{definition}

\begin{definition}
%\label{def.bbisim}
An  ${\mathcal N}$-\emph{barbed bisimulation} over a set of names, ${\mathcal N}$, is a symmetric binary relation 
${\mathcal S}_{\mathcal N}$ between agents such that $P\rel{S}_{\mathcal N}Q$ implies:
\begin{enumerate}
\item If $P \red P'$ then $Q \wred Q'$ and $P'\rel{S}_{\mathcal N} Q'$.
\item If $P\downarrow_{\mathcal N} x$, then $Q\Downarrow_{\mathcal N} x$.
\end{enumerate}
$P$ is ${\mathcal N}$-barbed bisimilar to $Q$, written
$P \wbbisim_{\mathcal N} Q$, if $P \rel{S}_{\mathcal N} Q$ for some ${\mathcal N}$-barbed bisimulation ${\mathcal S}_{\mathcal N}$.
\end{definition}

$\mathcal{R} \subseteq \pi \times \pi$

$P \mathcal{R} Q => \forall P'. P \red P' \Rightarrow \exists Q'. Q \red Q', P' \mathcal{R} Q'$

$P \vdash x \Rightarrow Q \vdash x$

\begin{mathpar}
  \inferrule*[lab=Out-barb]{x \nameeq y}{{y}!\langle{Q}\rangle \vdash x}
  \and
  \inferrule*[lab=Par-barb]{\mbox{$P\vdash x$ or $Q\vdash x$}}{\binpar{P}{Q} \vdash x}
\end{mathpar}

\subsubsection{Contexts}

One of the principle advantages of computational calculi like the
$\pi$-calculus is a well-defined notion of context,
contextual-equivalence and a correlation between
contextual-equivalence and notions of bisimulation. The notion of
context allows the decomposition of a process into (sub-)process and
its syntactic environment, its context. Thus, a context may be
thought of as a process with a ``hole'' (written $\Box$) in it. The
application of a context $M$ to a process $P$, written $M[P]$, is
tantamount to filling the hole in $M$ with $P$. In this paper we do
not need the full weight of this theory, but do make use of the notion
of context in the proof the main theorem. 

\begin{mathpar}
  \inferrule* [lab=summation] {} {{M_{M},M_{N}} \bc \Box \;|\; x.M_{A} \;|\; M_{M}+M_{N}}
  \and
  \inferrule* [lab=agent] {} {{M_{A}} \bc (\vec{x})M_{P} \;| \; \clift{P_0,\ldots,M_{P},\ldots,P_N}}
  \and \\
  \inferrule* [lab=process] {} {{M_{P}} \bc M_{N} \;| \;P|M_{P} }
\end{mathpar} 

\begin{mathpar}
  \inferrule* [lab=sychronization] {} {M_{N} \bc \Box \;|\; x?M_{F} \;|\; x!M_{C}}
  \and
  \inferrule* [lab=abstraction] {} {{M_{F}} \bc (x)M_{P} }
  \and
  \inferrule* [lab=concretion] {} {{M_{C}} \bc \langle M_{P} \rangle }
  \and \\
  \inferrule* [lab=process] {} {{M_{P}} \bc M_{N} \;| \;P|M_{P} }
\end{mathpar}

\begin{definition}[contextual application] Given a context $M$, and
  process $P$, we define the \emph{contextual application}, $M[P] :=
  M\{P/\Box\}$. That is, the contextual application of M to P is the
  substitution of $P$ for $\Box$ in $M$.
\end{definition}

$\meaningof{-} : L \to \mathcal{P}(\pi)$

\begin{mathpar}
  \inferrule* [lab=collection] {} {\meaningof{true} = \pi, \and \meaningof{~E} = \pi \setminus \meaningof{E}, \and \meaningof{E_{1} \& E_{2}} = \meaningof{E_{1}} \cap \meaningof{E_{2}}}
\end{mathpar}

\begin{mathpar}
  \inferrule* [lab=structure] {} {\meaningof{0} = \{ P \in \pi | P \equiv 0 \}, \and \\ \meaningof{E_1 | E_2} = \{ P \in \pi | P \equiv P_{1} | P_{2}, P_{1} \in \meaningof{E_{1}}, P_{2} \in \meaningof{E_2}\} }
\end{mathpar}

\begin{mathpar}
 \inferrule* [lab=behavior] {} {\meaningof{\langle a?b \rangle E} = \{ P \in \pi | P \equiv Q | u?(y)P', \\ \and \\\\ \and \\ \;\;\; u \in \meaningof{a}, \forall z.P'\{z/y\} \in \meaningof{E\{z/b\}}\}, \and \\ \meaningof{a!E} = \{ P \in \pi | P \equiv Q | x!\langle P' \rangle, x \in \meaningof{a} P' \in \meaningof{E}\} }
\end{mathpar}

\begin{mathpar}
 \inferrule* [lab=nominal] {} {\meaningof{\quotep{E}} = \{ \quotep{P} \in \quotep{\pi} | P \in \meaningof{E} \}, \and \meaningof{\quotep{P}} = \{ \quotep{Q} \in \quotep{\pi} | P \equiv Q \} \and \\ \meaningof{@\quotep{E}} = \{ P \in \pi | P \equiv @x, x \in \meaningof{E} \}}
\end{mathpar}

\begin{eqnarray*}
  \\
  \meaningof{-} : TS \to ST
\end{eqnarray*}

\begin{eqnarray*}
  \\
  L : TS \to ST
\end{eqnarray*}

\begin{eqnarray*}
  \\
  P \models E \iff P \in \meaningof{E}
\end{eqnarray*}

\begin{eqnarray*}
  P \approx_{L} Q \iff \forall E \in L. P \models E \iff Q \models E
\end{eqnarray*}

\begin{eqnarray*}
  P \approx_{K} Q
\end{eqnarray*}

\begin{eqnarray*}
  P \approx Q
\end{eqnarray*}

$\approx_{K} = \approx = \approx_{L}$

\subsubsection{Contextual duality}

Note that contexts extend the quotation operation to a family of
operations from processes to names. Given a context, $M$, we can
define a \emph{nominal context}, $\quotep{M}$ by $\quotep{M}[P] :=
\quotep{M[P]}$. To foreshadow what is to come we observe that these
operations enjoy a duality with processes very much like the duality
between vectors and maps from vectors to scalars.

Further, because the calculus is essentially higher-order, we have a
correspondence between contexts and processes. More specifically,
given a name $x$ and a context $M$ we can construct $M^{*}_{x}$ such
that 

\begin{mathpar}
  M^{*}_{x} | \lift{x}{P} \red M[P]
\end{mathpar}

namely,

\begin{mathpar}
  M^{*}_{x} := x?(u).M[\dropn{u}]
\end{mathpar}

The dependence of $M^{*}_{x}$ on a name makes it an abstraction, 

\begin{mathpar}
  M^{*} := (x)x?(u).M[\dropn{u}]
\end{mathpar}

\subsection{Additional notation}

It will sometimes be convenient to denote the process a name
quotes. We already have the notation $x = \quotep{P}$, but it will be
convenient to introduce an alternate notation, $\procn{x}$, when we
want to emphasize the connection to the use of the name. Note that, by
virtue of name equivalence, $\quotep{\procn{x}} \nameeq x$; so, the
notation is consistent with previous definitions.

Further, because names have structure it is possible to effect
substitutions on the basis of that structure. This means we need to
upgrade our notation for substitutions, which we accomplish by
adapting comprehension notation. Thus,

\begin{mathpar}
  P\{ y / x : x \in S \}
\end{mathpar}

is interpreted to mean the process derived from P by replacing (in a
capture-avoiding manner) each occurrence of $x$ in $S$ by $y$. For example,

\begin{mathpar}
  P\{ \quotep{\procn{x}|\procn{x}} / x : x \in \freenames{P} \}
\end{mathpar}

will replace each (occurrence) of a free name $x$ in $P$ by
$\quotep{\procn{x}|\procn{x}}$.

Also, we will avail ourselves of the notation $x^{L}$ and $x^{R}$ to
denote injections of a name into disjoint copies of the name
space. There are numerous ways to accomplish this. One example can be
found in \cite{MeredithR05}. This notation overloads to vectors of
names: $\vec{x}^{\pi} := (x_{i}^{\pi} \; : \; 0 \leq i < |\vec{x}| )$ where $\pi \in \{L,R\}$.

We also use $P^{\Box} := P|\Box$.

In \cite{MeredithR05} an interpretation of the new operator is
given. It turns out that there are several possible interpretations
all enjoying the requisite algebraic properties of the operator (see
\cite{milner91polyadicpi}). We will therefore make liberal use of
$(\nu\; \vec{x})P$.

% subsection the_syntax_and_semantics_of_the_notation_system (end)   

\input{qm2pi.qmops} 

\input{qm2pi.sterngerlach} 

\input{qm2pi.metric} 

% section concurrent_process_calculi (end)

%\input{qm2pi.proofsketch}

% section proof sketch (end)

%\input{qm2pi.slviaknots} 

% section spatial logic via knots (end)

\input{qm2pi.conclusion}

% section conclusion (end)

%\input{qm2pi.dtcodes} 

% section wiring algorithm (end)

\input{qm2pi.ack} 

% section acknowledgments (end)

\newpage


\bibliographystyle{plain}   
\bibliography{../../biblios/main.bib}

\input{qm2pi.rhodetails}

\end{document}

 

%\ifpdf
%\usepackage[pdftex]{graphicx}
%\else
%\usepackage{graphicx}
%\fi

 % \ifpdf
%  \usepackage{pdfsync}
%  \if


%\title{Brief Article}
%\author{David F. Snyder}
%\author{L.G. Meredith}

%\address{Dept. of Math., Texas State University--San Marcos, San Marcos, TX 78666}
       
\pagestyle{empty}


\begin{document}

\lstset{language=[Objective]Caml,frame=shadowbox}

\documentclass[12pt]{llncs}
%\documentclass{jktr}

\usepackage[pdftex]{hyperref}                   
\usepackage {listings}
\usepackage {mathpartir}
\usepackage{bcprules}
%\usepackage{listings}
                       
\usepackage{graphicx} 
%\usepackage[margins=2.5cm,nohead,nofoot]{geometry}
%\usepackage{geometry}
\usepackage{amsfonts}
\usepackage{amstext}
\usepackage{latexsym}
\usepackage{amssymb}
\usepackage{color}


%\include{myPreamble}
\include{qm2pi.local} 

%\ifpdf
%\usepackage[pdftex]{graphicx}
%\else
%\usepackage{graphicx}
%\fi

 % \ifpdf
%  \usepackage{pdfsync}
%  \if


%\title{Brief Article}
%\author{David F. Snyder}
%\author{L.G. Meredith}

%\address{Dept. of Math., Texas State University--San Marcos, San Marcos, TX 78666}
       
\pagestyle{empty}


\begin{document}

\lstset{language=[Objective]Caml,frame=shadowbox}

\input{qm2pi.front}

% section front matter (end)

\input{qm2pi.intro} 
 
% section introduction (end)

% \input{qm2pi.knotations} 

% section notation (end)

\input{qm2pi.process.calculi} 

% section concurrent_process_calculi_and_spatial_logics_ (end)
    
%\input{qm2pi.knots2pi} 

%\input{qm2pi.trefoil} 

%\input{qm2pi.mainthm} 

% subsection basic_interpretation (end)

%\input{qm2pi.rho.presentation} 
\subsection{The syntax and semantics of the notation system}\label{sub:the_syntax_and_semantics_of_the_notation_system} % (fold)

We now summarize a technical presentation of the calculus that
embodies our theory of dynamics. The typical presentation of such a
calculus follows the style of giving generators and relations on
them. The grammar, below, describing term constructors, freely
generates the set of processes, $\Proc$. This set is then quotiented
by a relation known as structural congruence and it is over this set
that the notion of dynamics is expressed. This presentation is
essentially that of \cite{MeredithR05} with the addition of
polyadicity and summation. For readability we have relegated some of
the technical subtleties to an appendix.

\subsubsection{Process grammar}\label{subsub:process_grammar}

\begin{mathpar}
  \inferrule* [lab=synchronization] {} {{M} \bc \pzero \;|\; x?F \;|\; x!C }
  \and
  \inferrule* [lab=abstraction] {} {{F} \bc (x)P}
  \and
  \inferrule* [lab=concretion] {} {{C} \bc \langle Q \rangle}
  \and
  \inferrule* [lab=process] {} {{P,Q} \bc M \;| \;P|Q \;|\; @{x}}
  \and
  \inferrule* [lab=name] {} {{x} \bc \quotep{P}}
\end{mathpar} 

Note that $\vec{x}$ (resp. $\vec{P}$) denotes a vector of names
(resp. processes) of length $|\vec{x}|$ (resp. $|\vec{P}|$). We adopt
the following useful abbreviations.

\begin{mathpar}
   x?(\vec{y}).P := x.(\vec{y})P \and  x\clift{\vec{P}} := x.\clift{\vec{P}}
   \and x!(y) := \lift{x}{\dropn{y}}
   \and \Pi_{i=0}^{n-1}P_i := P_0 | \ldots | P_{n-1}
\end{mathpar}

\subsubsection{Structural congruence}

\paragraph{Free and bound names and alpha-equivalence.} At the
core of structural equivalence is alpha-equivalence which identifies
process that are the same up to a change of variable. Formally, we
recognize the distinction between free and bound names. The free names
of a process, $\freenames{P}$, may be calculated recursively as
follows:

\begin{mathpar}
\freenames{\pzero} := \emptyset
  \and \\
  \freenames{x?(y).P} := \{ x \} \cup (\freenames{P} \setminus \{ y \})
  \and 
  \freenames{x!\langle P \rangle} := \{ x \} \cup \{ P \} 
  \and \\
  \freenames{P|Q} := \freenames{P} \cup \freenames{Q}
  \and \\
  \freenames{@{x}} := \{ x \}
\end{mathpar}

$\pi$
$\quotep{\pi}$

$\freenames{-} : \pi \to \mathcal{P}(\quotep{\pi})$

\begin{eqnarray*}
  \freenames{\pzero} & := & \emptyset \\
  \freenames{x?(y).P} & := & \{ x \} \cup (\freenames{P} \setminus \{ y \}) \\
  \freenames{x!\langle P \rangle} & := & \{ x \} \cup \{ P \} \\
  \freenames{P|Q} & := & \freenames{P} \cup \freenames{Q} \\
  \freenames{\dropn{x}} & := & \{ x \}
\end{eqnarray*}

The bound names of a process, $\boundnames{P}$, are those names occurring in $P$
that are not free. For example, in $x?(y).0$, the name $x$ is free, while $y$ is bound.

\begin{mathpar}
  \inferrule* [lab=monoidal-laws] {} { P|Q \equiv Q|P \and P|0 \equiv P \and P|(Q|R) \equiv (P|Q)|R }
\end{mathpar}

\begin{mathpar}
  \inferrule* [lab=alpha-equivalence] {} { (x)P \equiv (y)P\{y/x\} \and y \not\in \freenames{P} }
\end{mathpar}

\begin{definition}
Then two processes, $P,Q$, are alpha-equivalent if $P = Q\{\vec{y}/\vec{x}\}$ for
some $\vec{x} \in \boundnames{Q},\vec{y} \in \boundnames{P}$, where $Q\{\vec{y}/\vec{x}\}$
denotes the capture-avoiding substitution of $\vec{y}$ for $\vec{x}$ in $Q$.
\end{definition}

\begin{definition}
  The {\em structural congruence} \cite{SangiorgiWalker} , $\equiv$,
  between processes is the least congruence containing
  alpha-equivalence, satisfying the abelian monoid laws
  (associativity, commutativity and $\pzero$ as identity) for parallel
  composition $|$ and for summation $+$.
\end{definition}

\subsection{Name equivalence}

We take name equivalence, written $\nameeq$, to be the smallest
equivalence relation generated by the following rules.

\begin{mathpar}
\inferrule*[lab=Quote-drop]
{ }
{ \quotep{@{x}} \nameeq x }

\inferrule*[lab=Struct-equiv]
{ P \scong Q }
{ \quotep{P} \nameeq \quotep{Q} }
\end{mathpar}

The astute reader will have noticed that the mutual recursion of names
and processes imposes a mutual recursion on alpha-equivalence and
structural equivalence via name-equivalence. Fortunately, all of this
works out pleasantly and we may calculate in the natural way, free of
concern. The reader interested in the details is referred to the
appendix \ref{appendix:rho_details}.

\subsection{Substitution}

We use $\Proc$ for the set of processes, $\QProc$ for the set of
names, and $\id{\{}\vec{y} / \vec{x} \id{\}}$ to denote partial maps,
$s : \QProc \rightarrow \QProc$. A map, $s$ lifts, uniquely, to a map
on process terms, $\widehat{s} : \Proc \rightarrow \Proc$ by the
following equations.

\begin{mathpar}
  (0) \psubstp{Q}{P} := 0 \\
  (R \juxtap S) \psubstp{Q}{P}
  :=    
  (R)\psubstp{Q}{P} \juxtap (S) \psubstp{Q}{P} \\
  (x?(y).R) \psubstp{Q}{P}    
  :=    
  (x)\substp{Q}{P} (z)\concat( (R \psubstn{z}{y}) \psubstp{Q}{P} ) \\
  (\lift{x}{R}) \psubstp{Q}{P}  
  :=
  \lift{(x)\substp{Q}{P}}{ R \psubstp{Q}{P} } \\
%   (\dropn{x})  \psubstp{Q}{P}       
%   := 
%   \left\{ 
%     \begin{array}{ccc} 
%       \dropn{\quotep{Q}} & & x \nameeq \quotep{P} \\
%       \dropn{x} & & otherwise \\
%     \end{array}
%   \right. 
  (\dropn{x})  \psubstp{Q}{P}       
  := 
  \left\{ 
    \begin{array}{ccc} 
      Q & & x \nameeq \quotep{P} \\
      \dropn{x} & & otherwise \\
    \end{array}
  \right.
\end{mathpar}
 

where

\begin{eqnarray}
  (x)\id{\{} \lpquote Q \rpquote / \lpquote P \rpquote \id{\}}            = 
  \left\{ 
    \begin{array}{ccc}
      \lpquote Q \rpquote & & x \nameeq \lpquote P \rpquote \\
      x & & otherwise \\
    \end{array}
  \right. \nonumber
\end{eqnarray}

and $z$ is chosen distinct from $\quotep{P}$, $\quotep{Q}$, the free
names in $Q$, and all the names in $R$. Our $\alpha$-equivalence will
be built in the standard way from this substitution.

\begin{remark}\label{rem:no_self_referential_names}
  One consequence of these definitions is that $\forall P. \quotep{P}
  \not\in \freenames{P}$.
\end{remark}

\subsection{ Dynamic quote: an example }

Anticipating something of what's to come, consider applying the
substitution, $\widehat{\id{\{}u / z \id{\}}}$, to the following pair
of processes, $\lift{w}{y!(z)}$ and $w[ \lpquote y!(z) \rpquote ]$.

\begin{eqnarray}
	\lift{w}{y!(z)}\widehat{\id{\{}u / z \id{\}}}
		& = &
		\lift{w}{y!(u)} \nonumber\\
	w[ \lpquote y!(z) \rpquote ] \widehat{ \id{\{}u / z \id{\}} }
		& = &
		w[ \lpquote y!(z) \rpquote ] \nonumber
\end{eqnarray}

Because the body of the process between quotes is impervious to
substitution, we get radically different answers. In fact, by
examining the first process in an input context,
e.g. $x?(z).\lift{w}{y!(z)}$, we see that the process under the lift
operator may be shaped by prefixed inputs binding a name inside it. In
this sense, the lift operator will be seen as a way to dynamically
construct processes before reifying them as names.

Finally equipped with these standard features we can present the
dynamics of the calculus.

\subsubsection{Operational semantics} 

Finally, we introduce the computational dynamics. What marks these
algebras as distinct from other more traditionally studied algebraic
structures, e.g. vector spaces or polynomial rings, is the manner in
which dynamics is captured. In traditional structures, dynamics is typically
expressed through morphisms between such structures, as in linear maps
between vector spaces or morphisms between rings. In algebras
associated with the semantics of computation, the dynamics is
expressed as part of the algebraic structure itself, through a
reduction reduction relation typically denoted by $\red$. Below, we
give a recursive presentation of this relation for the calculus used
in the encoding.

$\red \subseteq \pi \times \pi$
$\red : \pi \to \mathcal{P}(\pi)$

\begin{mathpar}
  \inferrule* [lab=Comm] { \textsf{match}( x_{src}, x_{trgt} ) } { x_{trgt}?(y)P \; | \; x_{src}!\langle {Q} \rangle \red P\{\quotep{Q}/y}\} }
  \and \\
  \inferrule* [lab=Par] {{P} \red {P}'} {{{P} | {Q}} \red {{P}' | {Q}}}
  \and
  \inferrule* [lab=Equiv]{{{P} \scong {P}'} \andalso {{P}' \red {Q}'} \andalso {{Q}' \scong {Q}}}{{P} \red {Q}}
\end{mathpar}

\begin{eqnarray*}
  match_{\equiv} (\quotep{P},\quotep{Q}) & := & P \equiv Q \\
  match_{\dagger}(\quotep{P},\quotep{Q}) & := & \forall R. P|Q \red^{*} R => R \red^{*} 0 \\
  match_{K}(\quotep{P},\quotep{Q}) & := & K \mbox{ for some context } K
\end{eqnarray*}

$u?(x)P | u!\langle Q \rangle \red P\{\quotep{Q}/x\}$

%We write $\wred$ for $\red^*$, and $P\red$ if $\exists Q $ such that $ P \red Q$.
We write $P\red$ if $\exists Q $ such that $ P \red Q$ and $P\not\red$, otherwise.

\section{Replication}

As mentioned before, it is known that replication (and hence
recursion) can be implemented in a higher-order process algebra
\cite{SangiorgiWalker}. As our first example of calculation with the
machinery thus far presented we give the construction explicitly in
the {\rhoc}.

\begin{eqnarray}
	D_{x} & := & \prefix{x}{y}{(\binpar{\outputp{x}{y}}{@{y}})} \nonumber\\
	\bangp_{x}{P} & := & \binpar{{x}!\langle{\binpar{D_{x}}{P}}\rangle}{D_{x}} \nonumber
\end{eqnarray}

\begin{eqnarray}
	\bangp_{x}{P} & & \nonumber\\
	=
	& {x}!\langle{(\prefix{x}{y}{(\outputp{x}{y} | @{y})) | P}}\rangle 
	      | \prefix{x}{y}{(\outputp{x}{y} | @{y})} & \nonumber\\
	\red
	& (\outputp{x}{y} | @{y})\substn{\quotep{(\prefix{x}{y}{(@{y} | \outputp{x}{y})) | P}}}{y} & \nonumber\\
	=
	& \outputp{x}{\quotep{(\prefix{x}{y}{(\outputp{x}{y} | @{y})) | P}}}
	  | {(\prefix{x}{y}{(\outputp{x}{y} | @{y})) | P}} & \nonumber\\
	\red
	& \ldots & \nonumber\\
	\red^*
	& P | P | \ldots & \nonumber
\end{eqnarray}

Of course, this encoding, as an implementation, runs away, unfolding
$\bangp{P}$ eagerly. A lazier and more implementable replication
operator, restricted to input-guarded processes, may be obtained as follows.

\begin{eqnarray}
\bangp{\prefix{u}{v}{P}} 
	:= 
	\binpar{\lift{x}{\prefix{u}{v}{(\binpar{D(x)}{P})}}}{D(x)} \nonumber
\end{eqnarray}

\begin{remark}
  Note that the lazier definition still does not deal with summation
  or mixed summation (i.e. sums over input and output). The reader is
  invited to construct definitions of replication that deal with these
  features. 

  Further, the definitions are parameterized in a name, $x$. Can you,
  gentle reader, make a definition that eliminates this parameter and
  guarantees no accidental interaction between the replication
  machinery and the process being replicated -- i.e. no accidental
  sharing of names used by the process to get its work done and the
  name(s) used by the replication to effect copying. This latter
  revision of the definition of replication is crucial to obtaining
  the expected identity $!!P \sim !P$.
\end{remark}

\begin{remark}\label{rem:paradoxical_combinator}
  The reader familiar with the lambda calculus will have noticed the
  similarity between $D$ and the paradoxical combinator.

  [Ed. note: the existence of this seems to suggest we have to be more
  restrictive on the set of processes and names we admit if we are to
  support no-cloning.]
\end{remark}

\subsubsection{Bisimulation}

The computational dynamics gives rise to another kind of equivalence,
the equivalence of computational behavior. As previously mentioned
this is typically captured \emph{via} some form of bisimulation.

% The notion we use in this paper is weak barbed bisimulation
% \cite{milner91polyadicpi}.

The notion we use in this paper is derived from weak barbed
bisimulation \cite{milner91polyadicpi}. 

\begin{definition}
An \emph{observation relation}, $\downarrow_{\mathcal N}$, over a set
of names, $\mathcal N$, is the smallest relation satisfying the rules
below.

\infrule[Out-barb]{y \in {\mathcal N}, \; x \nameeq y}
		  {\outputp{x}{v} \downarrow_{\mathcal N} x}
\infrule[Par-barb]{\mbox{$P\downarrow_{\mathcal N} x$ or $Q\downarrow_{\mathcal N} x$}}
		  {\binpar{P}{Q} \downarrow_{\mathcal N} x}

We write $P \Downarrow_{\mathcal N} x$ if there is $Q$ such that 
$P \wred Q$ and $Q \downarrow_{\mathcal N} x$.
\end{definition}

\begin{definition}
%\label{def.bbisim}
An  ${\mathcal N}$-\emph{barbed bisimulation} over a set of names, ${\mathcal N}$, is a symmetric binary relation 
${\mathcal S}_{\mathcal N}$ between agents such that $P\rel{S}_{\mathcal N}Q$ implies:
\begin{enumerate}
\item If $P \red P'$ then $Q \wred Q'$ and $P'\rel{S}_{\mathcal N} Q'$.
\item If $P\downarrow_{\mathcal N} x$, then $Q\Downarrow_{\mathcal N} x$.
\end{enumerate}
$P$ is ${\mathcal N}$-barbed bisimilar to $Q$, written
$P \wbbisim_{\mathcal N} Q$, if $P \rel{S}_{\mathcal N} Q$ for some ${\mathcal N}$-barbed bisimulation ${\mathcal S}_{\mathcal N}$.
\end{definition}

$\mathcal{R} \subseteq \pi \times \pi$

$P \mathcal{R} Q => \forall P'. P \red P' \Rightarrow \exists Q'. Q \red Q', P' \mathcal{R} Q'$

$P \vdash x \Rightarrow Q \vdash x$

\begin{mathpar}
  \inferrule*[lab=Out-barb]{x \nameeq y}{{y}!\langle{Q}\rangle \vdash x}
  \and
  \inferrule*[lab=Par-barb]{\mbox{$P\vdash x$ or $Q\vdash x$}}{\binpar{P}{Q} \vdash x}
\end{mathpar}

\subsubsection{Contexts}

One of the principle advantages of computational calculi like the
$\pi$-calculus is a well-defined notion of context,
contextual-equivalence and a correlation between
contextual-equivalence and notions of bisimulation. The notion of
context allows the decomposition of a process into (sub-)process and
its syntactic environment, its context. Thus, a context may be
thought of as a process with a ``hole'' (written $\Box$) in it. The
application of a context $M$ to a process $P$, written $M[P]$, is
tantamount to filling the hole in $M$ with $P$. In this paper we do
not need the full weight of this theory, but do make use of the notion
of context in the proof the main theorem. 

\begin{mathpar}
  \inferrule* [lab=summation] {} {{M_{M},M_{N}} \bc \Box \;|\; x.M_{A} \;|\; M_{M}+M_{N}}
  \and
  \inferrule* [lab=agent] {} {{M_{A}} \bc (\vec{x})M_{P} \;| \; \clift{P_0,\ldots,M_{P},\ldots,P_N}}
  \and \\
  \inferrule* [lab=process] {} {{M_{P}} \bc M_{N} \;| \;P|M_{P} }
\end{mathpar} 

\begin{mathpar}
  \inferrule* [lab=sychronization] {} {M_{N} \bc \Box \;|\; x?M_{F} \;|\; x!M_{C}}
  \and
  \inferrule* [lab=abstraction] {} {{M_{F}} \bc (x)M_{P} }
  \and
  \inferrule* [lab=concretion] {} {{M_{C}} \bc \langle M_{P} \rangle }
  \and \\
  \inferrule* [lab=process] {} {{M_{P}} \bc M_{N} \;| \;P|M_{P} }
\end{mathpar}

\begin{definition}[contextual application] Given a context $M$, and
  process $P$, we define the \emph{contextual application}, $M[P] :=
  M\{P/\Box\}$. That is, the contextual application of M to P is the
  substitution of $P$ for $\Box$ in $M$.
\end{definition}

$\meaningof{-} : L \to \mathcal{P}(\pi)$

\begin{mathpar}
  \inferrule* [lab=collection] {} {\meaningof{true} = \pi, \and \meaningof{~E} = \pi \setminus \meaningof{E}, \and \meaningof{E_{1} \& E_{2}} = \meaningof{E_{1}} \cap \meaningof{E_{2}}}
\end{mathpar}

\begin{mathpar}
  \inferrule* [lab=structure] {} {\meaningof{0} = \{ P \in \pi | P \equiv 0 \}, \and \\ \meaningof{E_1 | E_2} = \{ P \in \pi | P \equiv P_{1} | P_{2}, P_{1} \in \meaningof{E_{1}}, P_{2} \in \meaningof{E_2}\} }
\end{mathpar}

\begin{mathpar}
 \inferrule* [lab=behavior] {} {\meaningof{\langle a?b \rangle E} = \{ P \in \pi | P \equiv Q | u?(y)P', \\ \and \\\\ \and \\ \;\;\; u \in \meaningof{a}, \forall z.P'\{z/y\} \in \meaningof{E\{z/b\}}\}, \and \\ \meaningof{a!E} = \{ P \in \pi | P \equiv Q | x!\langle P' \rangle, x \in \meaningof{a} P' \in \meaningof{E}\} }
\end{mathpar}

\begin{mathpar}
 \inferrule* [lab=nominal] {} {\meaningof{\quotep{E}} = \{ \quotep{P} \in \quotep{\pi} | P \in \meaningof{E} \}, \and \meaningof{\quotep{P}} = \{ \quotep{Q} \in \quotep{\pi} | P \equiv Q \} \and \\ \meaningof{@\quotep{E}} = \{ P \in \pi | P \equiv @x, x \in \meaningof{E} \}}
\end{mathpar}

\begin{eqnarray*}
  \\
  \meaningof{-} : TS \to ST
\end{eqnarray*}

\begin{eqnarray*}
  \\
  L : TS \to ST
\end{eqnarray*}

\begin{eqnarray*}
  \\
  P \models E \iff P \in \meaningof{E}
\end{eqnarray*}

\begin{eqnarray*}
  P \approx_{L} Q \iff \forall E \in L. P \models E \iff Q \models E
\end{eqnarray*}

\begin{eqnarray*}
  P \approx_{K} Q
\end{eqnarray*}

\begin{eqnarray*}
  P \approx Q
\end{eqnarray*}

$\approx_{K} = \approx = \approx_{L}$

\subsubsection{Contextual duality}

Note that contexts extend the quotation operation to a family of
operations from processes to names. Given a context, $M$, we can
define a \emph{nominal context}, $\quotep{M}$ by $\quotep{M}[P] :=
\quotep{M[P]}$. To foreshadow what is to come we observe that these
operations enjoy a duality with processes very much like the duality
between vectors and maps from vectors to scalars.

Further, because the calculus is essentially higher-order, we have a
correspondence between contexts and processes. More specifically,
given a name $x$ and a context $M$ we can construct $M^{*}_{x}$ such
that 

\begin{mathpar}
  M^{*}_{x} | \lift{x}{P} \red M[P]
\end{mathpar}

namely,

\begin{mathpar}
  M^{*}_{x} := x?(u).M[\dropn{u}]
\end{mathpar}

The dependence of $M^{*}_{x}$ on a name makes it an abstraction, 

\begin{mathpar}
  M^{*} := (x)x?(u).M[\dropn{u}]
\end{mathpar}

\subsection{Additional notation}

It will sometimes be convenient to denote the process a name
quotes. We already have the notation $x = \quotep{P}$, but it will be
convenient to introduce an alternate notation, $\procn{x}$, when we
want to emphasize the connection to the use of the name. Note that, by
virtue of name equivalence, $\quotep{\procn{x}} \nameeq x$; so, the
notation is consistent with previous definitions.

Further, because names have structure it is possible to effect
substitutions on the basis of that structure. This means we need to
upgrade our notation for substitutions, which we accomplish by
adapting comprehension notation. Thus,

\begin{mathpar}
  P\{ y / x : x \in S \}
\end{mathpar}

is interpreted to mean the process derived from P by replacing (in a
capture-avoiding manner) each occurrence of $x$ in $S$ by $y$. For example,

\begin{mathpar}
  P\{ \quotep{\procn{x}|\procn{x}} / x : x \in \freenames{P} \}
\end{mathpar}

will replace each (occurrence) of a free name $x$ in $P$ by
$\quotep{\procn{x}|\procn{x}}$.

Also, we will avail ourselves of the notation $x^{L}$ and $x^{R}$ to
denote injections of a name into disjoint copies of the name
space. There are numerous ways to accomplish this. One example can be
found in \cite{MeredithR05}. This notation overloads to vectors of
names: $\vec{x}^{\pi} := (x_{i}^{\pi} \; : \; 0 \leq i < |\vec{x}| )$ where $\pi \in \{L,R\}$.

We also use $P^{\Box} := P|\Box$.

In \cite{MeredithR05} an interpretation of the new operator is
given. It turns out that there are several possible interpretations
all enjoying the requisite algebraic properties of the operator (see
\cite{milner91polyadicpi}). We will therefore make liberal use of
$(\nu\; \vec{x})P$.

% subsection the_syntax_and_semantics_of_the_notation_system (end)   

\input{qm2pi.qmops} 

\input{qm2pi.sterngerlach} 

\input{qm2pi.metric} 

% section concurrent_process_calculi (end)

%\input{qm2pi.proofsketch}

% section proof sketch (end)

%\input{qm2pi.slviaknots} 

% section spatial logic via knots (end)

\input{qm2pi.conclusion}

% section conclusion (end)

%\input{qm2pi.dtcodes} 

% section wiring algorithm (end)

\input{qm2pi.ack} 

% section acknowledgments (end)

\newpage


\bibliographystyle{plain}   
\bibliography{../../biblios/main.bib}

\input{qm2pi.rhodetails}

\end{document}



% section front matter (end)

\section{Introduction}\label{sec:introduction} % (fold)
In this draft of the material i am going to have to dispense with the
usual writing conventions adopted in papers on these topics. i'm going
to have adopt whatever tone i need at the time i'm writing up the
calculations. Sometimes this may be very conversational; others it may
be the barest mathematical grunts; others still it may be that i have
lifted text from one of my other papers because the exposition of some
point was better said there. i hope that my readers are not unduly put
out by this decision. i'm not doing this to flout convention or be
rebellious. i find these calculations very technically challenging. To
keep everything going technically, something has to give; i have to
let go of some cognitive burden. So, the academic writing style --
with all of its trade-offs in terms of facilitating technical
communication -- is what i'm letting go of. Perhaps subsequent drafts
can be tightened and polished, but for now, i'm going to speak as if
we were sitting together in a coffee shop with a laptop, wifi and a
pad of paper and a pencil.

So, here's what i have to say. We -- you and i, comfortably ensconced
in our coffee shop and well-equipped with our tools -- can realize and
carry out the calculations of quantum mechanics over a very different
formal theory of dynamics, a formal theory of dynamics that
corresponds to a theory of concurrent computation with
\emph{reflection}. It has the advantage that the underlying theory is
already `quantized', but supports analogues all of the continuuous
operations. Strikingly, this underlying theory has recently been
connected with a notion of metric that we can show, by calculating
together, coincides with the metric induced by the inner product.

There are a lot of reasons why you might be interested in seeing
calculations of this form. Here's why i'm interested. For the past
several centuries there has been no competitor to the ``Newtonian''
account of dynamics. As a result the predominant share of accounts of
dynamical systems and situations have had to be formulated in terms of
the Newtonian machinery. i view this as an intellectually dangerous
position to occupy. Everything, despite it's intrinsic shape, turns
into a nail to be hit with this hammer. Recently, however, the theory
of computation has matured to the point where we have candidates for
theories of dynamics that offer very different perspective on
reasoning about dynamical systems and situations. Testing these
candidates against very successful accounts of dynamical situations,
like quantum mechanics, is going to give us some sense of how mature
they are and some measure of the quality of these accounts of
dynamics.

\subsection{Summary of contributions and outline of paper}

So, we're going to develop an interpretation of the operations of
quantum mechanics normally interpreted by Hilbert spaces and
operators. We're going to do this over a theory of computation. Note
that this is very different than the usual quantum computation program
which develops notions of computation over quantum mechanics. Rather,
we are developing a story that aligns with Wheeler's slogan: It from
Bit. To do this we will first provide an account of the theory of
computation at play here. Then we will dive into a calculation-driven
interpretation of the operations of quantum mechanics.

The reason we take this approach is that -- until very recently --
there hasn't been an axiomatic account of quantum mechanics. As a
result there has been no sharp delineation of the mathematical theory
supporting interpretation of the physical theory and the physical
theory, itself. So, ambient features of the maths are free to be
exploited (or supressed) without a real accounting of their physical
relevance. There is no sharp statement ``here's the physical theory''
qua \emph{theory} and ``here's the mathematical interpretation''
enabling a judgment of how faithful the interpretation is -- apart
from experimental observation. When there is an axiomatic account we
can judge how well a given mathematical formalism supports an
interpretation of the axioms, independent of
experimentation. Likewise, we can judge how well we have captured our
physical evidence and experience with our axiomatics, independent of
any specific mathematical implementation, with accidental detail that
may or may not have physical significance. 

In lieu of a fully fleshed out and vetted axiomatic account of quantum
mechanics, interpreting the operational notions in service of modeling
physical systems will have to suffice. In other words, we are not in
the business of providing a model of Hilbert spaces and operators. We
are in the business of providing a model of quantum mechanics because
we are motivated by testing our notions of dynamics against physical
theory; and, the predictive calculations of the physical theory must
serve as the best formulation -- shy of a fully fleshed out axiomatic
account -- of the physical theory itself (as they have for scientific
theories since time immemorial). Put another way, despite a
whole-hearted commitment to an It-from-Bit ontology, we are firmly
aligned with the shut-up-and-calculate camp as the best way to obtain
results either from the physical perspective or as a quality assurance
measure of our fledgling theory of dynamics.

In detail, we present a reflective process calculus. Then we develop
intuitive correspondences between the notions available in this
calculus and the usual physical notions supporting quantum mechanical
calculations. Thus, 

\begin{table}[htp]
  \center{
    \fbox{
      \begin{tabular}{c|c}
        quantum mechanics & process calculus \\
        \hline
        scalar & name \\
        state vector & process \\
        dual & contextual duals \\
        matrix & formal sums of process-context-dual pairs \\
        orthogonality & process annihilation \\
        inner product & execution-formula + quoting
      \end{tabular}
    }
  }
  \caption{QM - process calculi correspondences}
\end{table}

Then we tighten up these intuitions to operational definitions. We
employ the Dirac notation as the best proxy we can find for an
abstract syntax of the quantum mechanical notions. The definitions we
develop put us in contact with equational constraints coming from the
theory that we demonstrate the definitions and calculations satisfy.

This puts us in a position to shut up and calculate for the
Stern-Gerlach experimental set up, showing how these predictive
calculations become calculations on processes in our theory of a
reflective process calculus.

Penultimately, we demonstrate that the notion of metric coming from
the inner product coincides with the notion of metric available from
the theory of bisimulation. This demonstration gives us the right to
think of space as arising from behavior. Finally, we consider where we
might go from the new vantage point we have obtained.

% section introduction (end) 
 
% section introduction (end)

% \documentclass[12pt]{llncs}
%\documentclass{jktr}

\usepackage[pdftex]{hyperref}                   
\usepackage {listings}
\usepackage {mathpartir}
\usepackage{bcprules}
%\usepackage{listings}
                       
\usepackage{graphicx} 
%\usepackage[margins=2.5cm,nohead,nofoot]{geometry}
%\usepackage{geometry}
\usepackage{amsfonts}
\usepackage{amstext}
\usepackage{latexsym}
\usepackage{amssymb}
\usepackage{color}


%\include{myPreamble}
\include{qm2pi.local} 

%\ifpdf
%\usepackage[pdftex]{graphicx}
%\else
%\usepackage{graphicx}
%\fi

 % \ifpdf
%  \usepackage{pdfsync}
%  \if


%\title{Brief Article}
%\author{David F. Snyder}
%\author{L.G. Meredith}

%\address{Dept. of Math., Texas State University--San Marcos, San Marcos, TX 78666}
       
\pagestyle{empty}


\begin{document}

\lstset{language=[Objective]Caml,frame=shadowbox}

\input{qm2pi.front}

% section front matter (end)

\input{qm2pi.intro} 
 
% section introduction (end)

% \input{qm2pi.knotations} 

% section notation (end)

\input{qm2pi.process.calculi} 

% section concurrent_process_calculi_and_spatial_logics_ (end)
    
%\input{qm2pi.knots2pi} 

%\input{qm2pi.trefoil} 

%\input{qm2pi.mainthm} 

% subsection basic_interpretation (end)

%\input{qm2pi.rho.presentation} 
\subsection{The syntax and semantics of the notation system}\label{sub:the_syntax_and_semantics_of_the_notation_system} % (fold)

We now summarize a technical presentation of the calculus that
embodies our theory of dynamics. The typical presentation of such a
calculus follows the style of giving generators and relations on
them. The grammar, below, describing term constructors, freely
generates the set of processes, $\Proc$. This set is then quotiented
by a relation known as structural congruence and it is over this set
that the notion of dynamics is expressed. This presentation is
essentially that of \cite{MeredithR05} with the addition of
polyadicity and summation. For readability we have relegated some of
the technical subtleties to an appendix.

\subsubsection{Process grammar}\label{subsub:process_grammar}

\begin{mathpar}
  \inferrule* [lab=synchronization] {} {{M} \bc \pzero \;|\; x?F \;|\; x!C }
  \and
  \inferrule* [lab=abstraction] {} {{F} \bc (x)P}
  \and
  \inferrule* [lab=concretion] {} {{C} \bc \langle Q \rangle}
  \and
  \inferrule* [lab=process] {} {{P,Q} \bc M \;| \;P|Q \;|\; @{x}}
  \and
  \inferrule* [lab=name] {} {{x} \bc \quotep{P}}
\end{mathpar} 

Note that $\vec{x}$ (resp. $\vec{P}$) denotes a vector of names
(resp. processes) of length $|\vec{x}|$ (resp. $|\vec{P}|$). We adopt
the following useful abbreviations.

\begin{mathpar}
   x?(\vec{y}).P := x.(\vec{y})P \and  x\clift{\vec{P}} := x.\clift{\vec{P}}
   \and x!(y) := \lift{x}{\dropn{y}}
   \and \Pi_{i=0}^{n-1}P_i := P_0 | \ldots | P_{n-1}
\end{mathpar}

\subsubsection{Structural congruence}

\paragraph{Free and bound names and alpha-equivalence.} At the
core of structural equivalence is alpha-equivalence which identifies
process that are the same up to a change of variable. Formally, we
recognize the distinction between free and bound names. The free names
of a process, $\freenames{P}$, may be calculated recursively as
follows:

\begin{mathpar}
\freenames{\pzero} := \emptyset
  \and \\
  \freenames{x?(y).P} := \{ x \} \cup (\freenames{P} \setminus \{ y \})
  \and 
  \freenames{x!\langle P \rangle} := \{ x \} \cup \{ P \} 
  \and \\
  \freenames{P|Q} := \freenames{P} \cup \freenames{Q}
  \and \\
  \freenames{@{x}} := \{ x \}
\end{mathpar}

$\pi$
$\quotep{\pi}$

$\freenames{-} : \pi \to \mathcal{P}(\quotep{\pi})$

\begin{eqnarray*}
  \freenames{\pzero} & := & \emptyset \\
  \freenames{x?(y).P} & := & \{ x \} \cup (\freenames{P} \setminus \{ y \}) \\
  \freenames{x!\langle P \rangle} & := & \{ x \} \cup \{ P \} \\
  \freenames{P|Q} & := & \freenames{P} \cup \freenames{Q} \\
  \freenames{\dropn{x}} & := & \{ x \}
\end{eqnarray*}

The bound names of a process, $\boundnames{P}$, are those names occurring in $P$
that are not free. For example, in $x?(y).0$, the name $x$ is free, while $y$ is bound.

\begin{mathpar}
  \inferrule* [lab=monoidal-laws] {} { P|Q \equiv Q|P \and P|0 \equiv P \and P|(Q|R) \equiv (P|Q)|R }
\end{mathpar}

\begin{mathpar}
  \inferrule* [lab=alpha-equivalence] {} { (x)P \equiv (y)P\{y/x\} \and y \not\in \freenames{P} }
\end{mathpar}

\begin{definition}
Then two processes, $P,Q$, are alpha-equivalent if $P = Q\{\vec{y}/\vec{x}\}$ for
some $\vec{x} \in \boundnames{Q},\vec{y} \in \boundnames{P}$, where $Q\{\vec{y}/\vec{x}\}$
denotes the capture-avoiding substitution of $\vec{y}$ for $\vec{x}$ in $Q$.
\end{definition}

\begin{definition}
  The {\em structural congruence} \cite{SangiorgiWalker} , $\equiv$,
  between processes is the least congruence containing
  alpha-equivalence, satisfying the abelian monoid laws
  (associativity, commutativity and $\pzero$ as identity) for parallel
  composition $|$ and for summation $+$.
\end{definition}

\subsection{Name equivalence}

We take name equivalence, written $\nameeq$, to be the smallest
equivalence relation generated by the following rules.

\begin{mathpar}
\inferrule*[lab=Quote-drop]
{ }
{ \quotep{@{x}} \nameeq x }

\inferrule*[lab=Struct-equiv]
{ P \scong Q }
{ \quotep{P} \nameeq \quotep{Q} }
\end{mathpar}

The astute reader will have noticed that the mutual recursion of names
and processes imposes a mutual recursion on alpha-equivalence and
structural equivalence via name-equivalence. Fortunately, all of this
works out pleasantly and we may calculate in the natural way, free of
concern. The reader interested in the details is referred to the
appendix \ref{appendix:rho_details}.

\subsection{Substitution}

We use $\Proc$ for the set of processes, $\QProc$ for the set of
names, and $\id{\{}\vec{y} / \vec{x} \id{\}}$ to denote partial maps,
$s : \QProc \rightarrow \QProc$. A map, $s$ lifts, uniquely, to a map
on process terms, $\widehat{s} : \Proc \rightarrow \Proc$ by the
following equations.

\begin{mathpar}
  (0) \psubstp{Q}{P} := 0 \\
  (R \juxtap S) \psubstp{Q}{P}
  :=    
  (R)\psubstp{Q}{P} \juxtap (S) \psubstp{Q}{P} \\
  (x?(y).R) \psubstp{Q}{P}    
  :=    
  (x)\substp{Q}{P} (z)\concat( (R \psubstn{z}{y}) \psubstp{Q}{P} ) \\
  (\lift{x}{R}) \psubstp{Q}{P}  
  :=
  \lift{(x)\substp{Q}{P}}{ R \psubstp{Q}{P} } \\
%   (\dropn{x})  \psubstp{Q}{P}       
%   := 
%   \left\{ 
%     \begin{array}{ccc} 
%       \dropn{\quotep{Q}} & & x \nameeq \quotep{P} \\
%       \dropn{x} & & otherwise \\
%     \end{array}
%   \right. 
  (\dropn{x})  \psubstp{Q}{P}       
  := 
  \left\{ 
    \begin{array}{ccc} 
      Q & & x \nameeq \quotep{P} \\
      \dropn{x} & & otherwise \\
    \end{array}
  \right.
\end{mathpar}
 

where

\begin{eqnarray}
  (x)\id{\{} \lpquote Q \rpquote / \lpquote P \rpquote \id{\}}            = 
  \left\{ 
    \begin{array}{ccc}
      \lpquote Q \rpquote & & x \nameeq \lpquote P \rpquote \\
      x & & otherwise \\
    \end{array}
  \right. \nonumber
\end{eqnarray}

and $z$ is chosen distinct from $\quotep{P}$, $\quotep{Q}$, the free
names in $Q$, and all the names in $R$. Our $\alpha$-equivalence will
be built in the standard way from this substitution.

\begin{remark}\label{rem:no_self_referential_names}
  One consequence of these definitions is that $\forall P. \quotep{P}
  \not\in \freenames{P}$.
\end{remark}

\subsection{ Dynamic quote: an example }

Anticipating something of what's to come, consider applying the
substitution, $\widehat{\id{\{}u / z \id{\}}}$, to the following pair
of processes, $\lift{w}{y!(z)}$ and $w[ \lpquote y!(z) \rpquote ]$.

\begin{eqnarray}
	\lift{w}{y!(z)}\widehat{\id{\{}u / z \id{\}}}
		& = &
		\lift{w}{y!(u)} \nonumber\\
	w[ \lpquote y!(z) \rpquote ] \widehat{ \id{\{}u / z \id{\}} }
		& = &
		w[ \lpquote y!(z) \rpquote ] \nonumber
\end{eqnarray}

Because the body of the process between quotes is impervious to
substitution, we get radically different answers. In fact, by
examining the first process in an input context,
e.g. $x?(z).\lift{w}{y!(z)}$, we see that the process under the lift
operator may be shaped by prefixed inputs binding a name inside it. In
this sense, the lift operator will be seen as a way to dynamically
construct processes before reifying them as names.

Finally equipped with these standard features we can present the
dynamics of the calculus.

\subsubsection{Operational semantics} 

Finally, we introduce the computational dynamics. What marks these
algebras as distinct from other more traditionally studied algebraic
structures, e.g. vector spaces or polynomial rings, is the manner in
which dynamics is captured. In traditional structures, dynamics is typically
expressed through morphisms between such structures, as in linear maps
between vector spaces or morphisms between rings. In algebras
associated with the semantics of computation, the dynamics is
expressed as part of the algebraic structure itself, through a
reduction reduction relation typically denoted by $\red$. Below, we
give a recursive presentation of this relation for the calculus used
in the encoding.

$\red \subseteq \pi \times \pi$
$\red : \pi \to \mathcal{P}(\pi)$

\begin{mathpar}
  \inferrule* [lab=Comm] { \textsf{match}( x_{src}, x_{trgt} ) } { x_{trgt}?(y)P \; | \; x_{src}!\langle {Q} \rangle \red P\{\quotep{Q}/y}\} }
  \and \\
  \inferrule* [lab=Par] {{P} \red {P}'} {{{P} | {Q}} \red {{P}' | {Q}}}
  \and
  \inferrule* [lab=Equiv]{{{P} \scong {P}'} \andalso {{P}' \red {Q}'} \andalso {{Q}' \scong {Q}}}{{P} \red {Q}}
\end{mathpar}

\begin{eqnarray*}
  match_{\equiv} (\quotep{P},\quotep{Q}) & := & P \equiv Q \\
  match_{\dagger}(\quotep{P},\quotep{Q}) & := & \forall R. P|Q \red^{*} R => R \red^{*} 0 \\
  match_{K}(\quotep{P},\quotep{Q}) & := & K \mbox{ for some context } K
\end{eqnarray*}

$u?(x)P | u!\langle Q \rangle \red P\{\quotep{Q}/x\}$

%We write $\wred$ for $\red^*$, and $P\red$ if $\exists Q $ such that $ P \red Q$.
We write $P\red$ if $\exists Q $ such that $ P \red Q$ and $P\not\red$, otherwise.

\section{Replication}

As mentioned before, it is known that replication (and hence
recursion) can be implemented in a higher-order process algebra
\cite{SangiorgiWalker}. As our first example of calculation with the
machinery thus far presented we give the construction explicitly in
the {\rhoc}.

\begin{eqnarray}
	D_{x} & := & \prefix{x}{y}{(\binpar{\outputp{x}{y}}{@{y}})} \nonumber\\
	\bangp_{x}{P} & := & \binpar{{x}!\langle{\binpar{D_{x}}{P}}\rangle}{D_{x}} \nonumber
\end{eqnarray}

\begin{eqnarray}
	\bangp_{x}{P} & & \nonumber\\
	=
	& {x}!\langle{(\prefix{x}{y}{(\outputp{x}{y} | @{y})) | P}}\rangle 
	      | \prefix{x}{y}{(\outputp{x}{y} | @{y})} & \nonumber\\
	\red
	& (\outputp{x}{y} | @{y})\substn{\quotep{(\prefix{x}{y}{(@{y} | \outputp{x}{y})) | P}}}{y} & \nonumber\\
	=
	& \outputp{x}{\quotep{(\prefix{x}{y}{(\outputp{x}{y} | @{y})) | P}}}
	  | {(\prefix{x}{y}{(\outputp{x}{y} | @{y})) | P}} & \nonumber\\
	\red
	& \ldots & \nonumber\\
	\red^*
	& P | P | \ldots & \nonumber
\end{eqnarray}

Of course, this encoding, as an implementation, runs away, unfolding
$\bangp{P}$ eagerly. A lazier and more implementable replication
operator, restricted to input-guarded processes, may be obtained as follows.

\begin{eqnarray}
\bangp{\prefix{u}{v}{P}} 
	:= 
	\binpar{\lift{x}{\prefix{u}{v}{(\binpar{D(x)}{P})}}}{D(x)} \nonumber
\end{eqnarray}

\begin{remark}
  Note that the lazier definition still does not deal with summation
  or mixed summation (i.e. sums over input and output). The reader is
  invited to construct definitions of replication that deal with these
  features. 

  Further, the definitions are parameterized in a name, $x$. Can you,
  gentle reader, make a definition that eliminates this parameter and
  guarantees no accidental interaction between the replication
  machinery and the process being replicated -- i.e. no accidental
  sharing of names used by the process to get its work done and the
  name(s) used by the replication to effect copying. This latter
  revision of the definition of replication is crucial to obtaining
  the expected identity $!!P \sim !P$.
\end{remark}

\begin{remark}\label{rem:paradoxical_combinator}
  The reader familiar with the lambda calculus will have noticed the
  similarity between $D$ and the paradoxical combinator.

  [Ed. note: the existence of this seems to suggest we have to be more
  restrictive on the set of processes and names we admit if we are to
  support no-cloning.]
\end{remark}

\subsubsection{Bisimulation}

The computational dynamics gives rise to another kind of equivalence,
the equivalence of computational behavior. As previously mentioned
this is typically captured \emph{via} some form of bisimulation.

% The notion we use in this paper is weak barbed bisimulation
% \cite{milner91polyadicpi}.

The notion we use in this paper is derived from weak barbed
bisimulation \cite{milner91polyadicpi}. 

\begin{definition}
An \emph{observation relation}, $\downarrow_{\mathcal N}$, over a set
of names, $\mathcal N$, is the smallest relation satisfying the rules
below.

\infrule[Out-barb]{y \in {\mathcal N}, \; x \nameeq y}
		  {\outputp{x}{v} \downarrow_{\mathcal N} x}
\infrule[Par-barb]{\mbox{$P\downarrow_{\mathcal N} x$ or $Q\downarrow_{\mathcal N} x$}}
		  {\binpar{P}{Q} \downarrow_{\mathcal N} x}

We write $P \Downarrow_{\mathcal N} x$ if there is $Q$ such that 
$P \wred Q$ and $Q \downarrow_{\mathcal N} x$.
\end{definition}

\begin{definition}
%\label{def.bbisim}
An  ${\mathcal N}$-\emph{barbed bisimulation} over a set of names, ${\mathcal N}$, is a symmetric binary relation 
${\mathcal S}_{\mathcal N}$ between agents such that $P\rel{S}_{\mathcal N}Q$ implies:
\begin{enumerate}
\item If $P \red P'$ then $Q \wred Q'$ and $P'\rel{S}_{\mathcal N} Q'$.
\item If $P\downarrow_{\mathcal N} x$, then $Q\Downarrow_{\mathcal N} x$.
\end{enumerate}
$P$ is ${\mathcal N}$-barbed bisimilar to $Q$, written
$P \wbbisim_{\mathcal N} Q$, if $P \rel{S}_{\mathcal N} Q$ for some ${\mathcal N}$-barbed bisimulation ${\mathcal S}_{\mathcal N}$.
\end{definition}

$\mathcal{R} \subseteq \pi \times \pi$

$P \mathcal{R} Q => \forall P'. P \red P' \Rightarrow \exists Q'. Q \red Q', P' \mathcal{R} Q'$

$P \vdash x \Rightarrow Q \vdash x$

\begin{mathpar}
  \inferrule*[lab=Out-barb]{x \nameeq y}{{y}!\langle{Q}\rangle \vdash x}
  \and
  \inferrule*[lab=Par-barb]{\mbox{$P\vdash x$ or $Q\vdash x$}}{\binpar{P}{Q} \vdash x}
\end{mathpar}

\subsubsection{Contexts}

One of the principle advantages of computational calculi like the
$\pi$-calculus is a well-defined notion of context,
contextual-equivalence and a correlation between
contextual-equivalence and notions of bisimulation. The notion of
context allows the decomposition of a process into (sub-)process and
its syntactic environment, its context. Thus, a context may be
thought of as a process with a ``hole'' (written $\Box$) in it. The
application of a context $M$ to a process $P$, written $M[P]$, is
tantamount to filling the hole in $M$ with $P$. In this paper we do
not need the full weight of this theory, but do make use of the notion
of context in the proof the main theorem. 

\begin{mathpar}
  \inferrule* [lab=summation] {} {{M_{M},M_{N}} \bc \Box \;|\; x.M_{A} \;|\; M_{M}+M_{N}}
  \and
  \inferrule* [lab=agent] {} {{M_{A}} \bc (\vec{x})M_{P} \;| \; \clift{P_0,\ldots,M_{P},\ldots,P_N}}
  \and \\
  \inferrule* [lab=process] {} {{M_{P}} \bc M_{N} \;| \;P|M_{P} }
\end{mathpar} 

\begin{mathpar}
  \inferrule* [lab=sychronization] {} {M_{N} \bc \Box \;|\; x?M_{F} \;|\; x!M_{C}}
  \and
  \inferrule* [lab=abstraction] {} {{M_{F}} \bc (x)M_{P} }
  \and
  \inferrule* [lab=concretion] {} {{M_{C}} \bc \langle M_{P} \rangle }
  \and \\
  \inferrule* [lab=process] {} {{M_{P}} \bc M_{N} \;| \;P|M_{P} }
\end{mathpar}

\begin{definition}[contextual application] Given a context $M$, and
  process $P$, we define the \emph{contextual application}, $M[P] :=
  M\{P/\Box\}$. That is, the contextual application of M to P is the
  substitution of $P$ for $\Box$ in $M$.
\end{definition}

$\meaningof{-} : L \to \mathcal{P}(\pi)$

\begin{mathpar}
  \inferrule* [lab=collection] {} {\meaningof{true} = \pi, \and \meaningof{~E} = \pi \setminus \meaningof{E}, \and \meaningof{E_{1} \& E_{2}} = \meaningof{E_{1}} \cap \meaningof{E_{2}}}
\end{mathpar}

\begin{mathpar}
  \inferrule* [lab=structure] {} {\meaningof{0} = \{ P \in \pi | P \equiv 0 \}, \and \\ \meaningof{E_1 | E_2} = \{ P \in \pi | P \equiv P_{1} | P_{2}, P_{1} \in \meaningof{E_{1}}, P_{2} \in \meaningof{E_2}\} }
\end{mathpar}

\begin{mathpar}
 \inferrule* [lab=behavior] {} {\meaningof{\langle a?b \rangle E} = \{ P \in \pi | P \equiv Q | u?(y)P', \\ \and \\\\ \and \\ \;\;\; u \in \meaningof{a}, \forall z.P'\{z/y\} \in \meaningof{E\{z/b\}}\}, \and \\ \meaningof{a!E} = \{ P \in \pi | P \equiv Q | x!\langle P' \rangle, x \in \meaningof{a} P' \in \meaningof{E}\} }
\end{mathpar}

\begin{mathpar}
 \inferrule* [lab=nominal] {} {\meaningof{\quotep{E}} = \{ \quotep{P} \in \quotep{\pi} | P \in \meaningof{E} \}, \and \meaningof{\quotep{P}} = \{ \quotep{Q} \in \quotep{\pi} | P \equiv Q \} \and \\ \meaningof{@\quotep{E}} = \{ P \in \pi | P \equiv @x, x \in \meaningof{E} \}}
\end{mathpar}

\begin{eqnarray*}
  \\
  \meaningof{-} : TS \to ST
\end{eqnarray*}

\begin{eqnarray*}
  \\
  L : TS \to ST
\end{eqnarray*}

\begin{eqnarray*}
  \\
  P \models E \iff P \in \meaningof{E}
\end{eqnarray*}

\begin{eqnarray*}
  P \approx_{L} Q \iff \forall E \in L. P \models E \iff Q \models E
\end{eqnarray*}

\begin{eqnarray*}
  P \approx_{K} Q
\end{eqnarray*}

\begin{eqnarray*}
  P \approx Q
\end{eqnarray*}

$\approx_{K} = \approx = \approx_{L}$

\subsubsection{Contextual duality}

Note that contexts extend the quotation operation to a family of
operations from processes to names. Given a context, $M$, we can
define a \emph{nominal context}, $\quotep{M}$ by $\quotep{M}[P] :=
\quotep{M[P]}$. To foreshadow what is to come we observe that these
operations enjoy a duality with processes very much like the duality
between vectors and maps from vectors to scalars.

Further, because the calculus is essentially higher-order, we have a
correspondence between contexts and processes. More specifically,
given a name $x$ and a context $M$ we can construct $M^{*}_{x}$ such
that 

\begin{mathpar}
  M^{*}_{x} | \lift{x}{P} \red M[P]
\end{mathpar}

namely,

\begin{mathpar}
  M^{*}_{x} := x?(u).M[\dropn{u}]
\end{mathpar}

The dependence of $M^{*}_{x}$ on a name makes it an abstraction, 

\begin{mathpar}
  M^{*} := (x)x?(u).M[\dropn{u}]
\end{mathpar}

\subsection{Additional notation}

It will sometimes be convenient to denote the process a name
quotes. We already have the notation $x = \quotep{P}$, but it will be
convenient to introduce an alternate notation, $\procn{x}$, when we
want to emphasize the connection to the use of the name. Note that, by
virtue of name equivalence, $\quotep{\procn{x}} \nameeq x$; so, the
notation is consistent with previous definitions.

Further, because names have structure it is possible to effect
substitutions on the basis of that structure. This means we need to
upgrade our notation for substitutions, which we accomplish by
adapting comprehension notation. Thus,

\begin{mathpar}
  P\{ y / x : x \in S \}
\end{mathpar}

is interpreted to mean the process derived from P by replacing (in a
capture-avoiding manner) each occurrence of $x$ in $S$ by $y$. For example,

\begin{mathpar}
  P\{ \quotep{\procn{x}|\procn{x}} / x : x \in \freenames{P} \}
\end{mathpar}

will replace each (occurrence) of a free name $x$ in $P$ by
$\quotep{\procn{x}|\procn{x}}$.

Also, we will avail ourselves of the notation $x^{L}$ and $x^{R}$ to
denote injections of a name into disjoint copies of the name
space. There are numerous ways to accomplish this. One example can be
found in \cite{MeredithR05}. This notation overloads to vectors of
names: $\vec{x}^{\pi} := (x_{i}^{\pi} \; : \; 0 \leq i < |\vec{x}| )$ where $\pi \in \{L,R\}$.

We also use $P^{\Box} := P|\Box$.

In \cite{MeredithR05} an interpretation of the new operator is
given. It turns out that there are several possible interpretations
all enjoying the requisite algebraic properties of the operator (see
\cite{milner91polyadicpi}). We will therefore make liberal use of
$(\nu\; \vec{x})P$.

% subsection the_syntax_and_semantics_of_the_notation_system (end)   

\input{qm2pi.qmops} 

\input{qm2pi.sterngerlach} 

\input{qm2pi.metric} 

% section concurrent_process_calculi (end)

%\input{qm2pi.proofsketch}

% section proof sketch (end)

%\input{qm2pi.slviaknots} 

% section spatial logic via knots (end)

\input{qm2pi.conclusion}

% section conclusion (end)

%\input{qm2pi.dtcodes} 

% section wiring algorithm (end)

\input{qm2pi.ack} 

% section acknowledgments (end)

\newpage


\bibliographystyle{plain}   
\bibliography{../../biblios/main.bib}

\input{qm2pi.rhodetails}

\end{document}

 

% section notation (end)

\input{qm2pi.process.calculi} 

% section concurrent_process_calculi_and_spatial_logics_ (end)
    
%\documentclass[12pt]{llncs}
%\documentclass{jktr}

\usepackage[pdftex]{hyperref}                   
\usepackage {listings}
\usepackage {mathpartir}
\usepackage{bcprules}
%\usepackage{listings}
                       
\usepackage{graphicx} 
%\usepackage[margins=2.5cm,nohead,nofoot]{geometry}
%\usepackage{geometry}
\usepackage{amsfonts}
\usepackage{amstext}
\usepackage{latexsym}
\usepackage{amssymb}
\usepackage{color}


%\include{myPreamble}
\include{qm2pi.local} 

%\ifpdf
%\usepackage[pdftex]{graphicx}
%\else
%\usepackage{graphicx}
%\fi

 % \ifpdf
%  \usepackage{pdfsync}
%  \if


%\title{Brief Article}
%\author{David F. Snyder}
%\author{L.G. Meredith}

%\address{Dept. of Math., Texas State University--San Marcos, San Marcos, TX 78666}
       
\pagestyle{empty}


\begin{document}

\lstset{language=[Objective]Caml,frame=shadowbox}

\input{qm2pi.front}

% section front matter (end)

\input{qm2pi.intro} 
 
% section introduction (end)

% \input{qm2pi.knotations} 

% section notation (end)

\input{qm2pi.process.calculi} 

% section concurrent_process_calculi_and_spatial_logics_ (end)
    
%\input{qm2pi.knots2pi} 

%\input{qm2pi.trefoil} 

%\input{qm2pi.mainthm} 

% subsection basic_interpretation (end)

%\input{qm2pi.rho.presentation} 
\subsection{The syntax and semantics of the notation system}\label{sub:the_syntax_and_semantics_of_the_notation_system} % (fold)

We now summarize a technical presentation of the calculus that
embodies our theory of dynamics. The typical presentation of such a
calculus follows the style of giving generators and relations on
them. The grammar, below, describing term constructors, freely
generates the set of processes, $\Proc$. This set is then quotiented
by a relation known as structural congruence and it is over this set
that the notion of dynamics is expressed. This presentation is
essentially that of \cite{MeredithR05} with the addition of
polyadicity and summation. For readability we have relegated some of
the technical subtleties to an appendix.

\subsubsection{Process grammar}\label{subsub:process_grammar}

\begin{mathpar}
  \inferrule* [lab=synchronization] {} {{M} \bc \pzero \;|\; x?F \;|\; x!C }
  \and
  \inferrule* [lab=abstraction] {} {{F} \bc (x)P}
  \and
  \inferrule* [lab=concretion] {} {{C} \bc \langle Q \rangle}
  \and
  \inferrule* [lab=process] {} {{P,Q} \bc M \;| \;P|Q \;|\; @{x}}
  \and
  \inferrule* [lab=name] {} {{x} \bc \quotep{P}}
\end{mathpar} 

Note that $\vec{x}$ (resp. $\vec{P}$) denotes a vector of names
(resp. processes) of length $|\vec{x}|$ (resp. $|\vec{P}|$). We adopt
the following useful abbreviations.

\begin{mathpar}
   x?(\vec{y}).P := x.(\vec{y})P \and  x\clift{\vec{P}} := x.\clift{\vec{P}}
   \and x!(y) := \lift{x}{\dropn{y}}
   \and \Pi_{i=0}^{n-1}P_i := P_0 | \ldots | P_{n-1}
\end{mathpar}

\subsubsection{Structural congruence}

\paragraph{Free and bound names and alpha-equivalence.} At the
core of structural equivalence is alpha-equivalence which identifies
process that are the same up to a change of variable. Formally, we
recognize the distinction between free and bound names. The free names
of a process, $\freenames{P}$, may be calculated recursively as
follows:

\begin{mathpar}
\freenames{\pzero} := \emptyset
  \and \\
  \freenames{x?(y).P} := \{ x \} \cup (\freenames{P} \setminus \{ y \})
  \and 
  \freenames{x!\langle P \rangle} := \{ x \} \cup \{ P \} 
  \and \\
  \freenames{P|Q} := \freenames{P} \cup \freenames{Q}
  \and \\
  \freenames{@{x}} := \{ x \}
\end{mathpar}

$\pi$
$\quotep{\pi}$

$\freenames{-} : \pi \to \mathcal{P}(\quotep{\pi})$

\begin{eqnarray*}
  \freenames{\pzero} & := & \emptyset \\
  \freenames{x?(y).P} & := & \{ x \} \cup (\freenames{P} \setminus \{ y \}) \\
  \freenames{x!\langle P \rangle} & := & \{ x \} \cup \{ P \} \\
  \freenames{P|Q} & := & \freenames{P} \cup \freenames{Q} \\
  \freenames{\dropn{x}} & := & \{ x \}
\end{eqnarray*}

The bound names of a process, $\boundnames{P}$, are those names occurring in $P$
that are not free. For example, in $x?(y).0$, the name $x$ is free, while $y$ is bound.

\begin{mathpar}
  \inferrule* [lab=monoidal-laws] {} { P|Q \equiv Q|P \and P|0 \equiv P \and P|(Q|R) \equiv (P|Q)|R }
\end{mathpar}

\begin{mathpar}
  \inferrule* [lab=alpha-equivalence] {} { (x)P \equiv (y)P\{y/x\} \and y \not\in \freenames{P} }
\end{mathpar}

\begin{definition}
Then two processes, $P,Q$, are alpha-equivalent if $P = Q\{\vec{y}/\vec{x}\}$ for
some $\vec{x} \in \boundnames{Q},\vec{y} \in \boundnames{P}$, where $Q\{\vec{y}/\vec{x}\}$
denotes the capture-avoiding substitution of $\vec{y}$ for $\vec{x}$ in $Q$.
\end{definition}

\begin{definition}
  The {\em structural congruence} \cite{SangiorgiWalker} , $\equiv$,
  between processes is the least congruence containing
  alpha-equivalence, satisfying the abelian monoid laws
  (associativity, commutativity and $\pzero$ as identity) for parallel
  composition $|$ and for summation $+$.
\end{definition}

\subsection{Name equivalence}

We take name equivalence, written $\nameeq$, to be the smallest
equivalence relation generated by the following rules.

\begin{mathpar}
\inferrule*[lab=Quote-drop]
{ }
{ \quotep{@{x}} \nameeq x }

\inferrule*[lab=Struct-equiv]
{ P \scong Q }
{ \quotep{P} \nameeq \quotep{Q} }
\end{mathpar}

The astute reader will have noticed that the mutual recursion of names
and processes imposes a mutual recursion on alpha-equivalence and
structural equivalence via name-equivalence. Fortunately, all of this
works out pleasantly and we may calculate in the natural way, free of
concern. The reader interested in the details is referred to the
appendix \ref{appendix:rho_details}.

\subsection{Substitution}

We use $\Proc$ for the set of processes, $\QProc$ for the set of
names, and $\id{\{}\vec{y} / \vec{x} \id{\}}$ to denote partial maps,
$s : \QProc \rightarrow \QProc$. A map, $s$ lifts, uniquely, to a map
on process terms, $\widehat{s} : \Proc \rightarrow \Proc$ by the
following equations.

\begin{mathpar}
  (0) \psubstp{Q}{P} := 0 \\
  (R \juxtap S) \psubstp{Q}{P}
  :=    
  (R)\psubstp{Q}{P} \juxtap (S) \psubstp{Q}{P} \\
  (x?(y).R) \psubstp{Q}{P}    
  :=    
  (x)\substp{Q}{P} (z)\concat( (R \psubstn{z}{y}) \psubstp{Q}{P} ) \\
  (\lift{x}{R}) \psubstp{Q}{P}  
  :=
  \lift{(x)\substp{Q}{P}}{ R \psubstp{Q}{P} } \\
%   (\dropn{x})  \psubstp{Q}{P}       
%   := 
%   \left\{ 
%     \begin{array}{ccc} 
%       \dropn{\quotep{Q}} & & x \nameeq \quotep{P} \\
%       \dropn{x} & & otherwise \\
%     \end{array}
%   \right. 
  (\dropn{x})  \psubstp{Q}{P}       
  := 
  \left\{ 
    \begin{array}{ccc} 
      Q & & x \nameeq \quotep{P} \\
      \dropn{x} & & otherwise \\
    \end{array}
  \right.
\end{mathpar}
 

where

\begin{eqnarray}
  (x)\id{\{} \lpquote Q \rpquote / \lpquote P \rpquote \id{\}}            = 
  \left\{ 
    \begin{array}{ccc}
      \lpquote Q \rpquote & & x \nameeq \lpquote P \rpquote \\
      x & & otherwise \\
    \end{array}
  \right. \nonumber
\end{eqnarray}

and $z$ is chosen distinct from $\quotep{P}$, $\quotep{Q}$, the free
names in $Q$, and all the names in $R$. Our $\alpha$-equivalence will
be built in the standard way from this substitution.

\begin{remark}\label{rem:no_self_referential_names}
  One consequence of these definitions is that $\forall P. \quotep{P}
  \not\in \freenames{P}$.
\end{remark}

\subsection{ Dynamic quote: an example }

Anticipating something of what's to come, consider applying the
substitution, $\widehat{\id{\{}u / z \id{\}}}$, to the following pair
of processes, $\lift{w}{y!(z)}$ and $w[ \lpquote y!(z) \rpquote ]$.

\begin{eqnarray}
	\lift{w}{y!(z)}\widehat{\id{\{}u / z \id{\}}}
		& = &
		\lift{w}{y!(u)} \nonumber\\
	w[ \lpquote y!(z) \rpquote ] \widehat{ \id{\{}u / z \id{\}} }
		& = &
		w[ \lpquote y!(z) \rpquote ] \nonumber
\end{eqnarray}

Because the body of the process between quotes is impervious to
substitution, we get radically different answers. In fact, by
examining the first process in an input context,
e.g. $x?(z).\lift{w}{y!(z)}$, we see that the process under the lift
operator may be shaped by prefixed inputs binding a name inside it. In
this sense, the lift operator will be seen as a way to dynamically
construct processes before reifying them as names.

Finally equipped with these standard features we can present the
dynamics of the calculus.

\subsubsection{Operational semantics} 

Finally, we introduce the computational dynamics. What marks these
algebras as distinct from other more traditionally studied algebraic
structures, e.g. vector spaces or polynomial rings, is the manner in
which dynamics is captured. In traditional structures, dynamics is typically
expressed through morphisms between such structures, as in linear maps
between vector spaces or morphisms between rings. In algebras
associated with the semantics of computation, the dynamics is
expressed as part of the algebraic structure itself, through a
reduction reduction relation typically denoted by $\red$. Below, we
give a recursive presentation of this relation for the calculus used
in the encoding.

$\red \subseteq \pi \times \pi$
$\red : \pi \to \mathcal{P}(\pi)$

\begin{mathpar}
  \inferrule* [lab=Comm] { \textsf{match}( x_{src}, x_{trgt} ) } { x_{trgt}?(y)P \; | \; x_{src}!\langle {Q} \rangle \red P\{\quotep{Q}/y}\} }
  \and \\
  \inferrule* [lab=Par] {{P} \red {P}'} {{{P} | {Q}} \red {{P}' | {Q}}}
  \and
  \inferrule* [lab=Equiv]{{{P} \scong {P}'} \andalso {{P}' \red {Q}'} \andalso {{Q}' \scong {Q}}}{{P} \red {Q}}
\end{mathpar}

\begin{eqnarray*}
  match_{\equiv} (\quotep{P},\quotep{Q}) & := & P \equiv Q \\
  match_{\dagger}(\quotep{P},\quotep{Q}) & := & \forall R. P|Q \red^{*} R => R \red^{*} 0 \\
  match_{K}(\quotep{P},\quotep{Q}) & := & K \mbox{ for some context } K
\end{eqnarray*}

$u?(x)P | u!\langle Q \rangle \red P\{\quotep{Q}/x\}$

%We write $\wred$ for $\red^*$, and $P\red$ if $\exists Q $ such that $ P \red Q$.
We write $P\red$ if $\exists Q $ such that $ P \red Q$ and $P\not\red$, otherwise.

\section{Replication}

As mentioned before, it is known that replication (and hence
recursion) can be implemented in a higher-order process algebra
\cite{SangiorgiWalker}. As our first example of calculation with the
machinery thus far presented we give the construction explicitly in
the {\rhoc}.

\begin{eqnarray}
	D_{x} & := & \prefix{x}{y}{(\binpar{\outputp{x}{y}}{@{y}})} \nonumber\\
	\bangp_{x}{P} & := & \binpar{{x}!\langle{\binpar{D_{x}}{P}}\rangle}{D_{x}} \nonumber
\end{eqnarray}

\begin{eqnarray}
	\bangp_{x}{P} & & \nonumber\\
	=
	& {x}!\langle{(\prefix{x}{y}{(\outputp{x}{y} | @{y})) | P}}\rangle 
	      | \prefix{x}{y}{(\outputp{x}{y} | @{y})} & \nonumber\\
	\red
	& (\outputp{x}{y} | @{y})\substn{\quotep{(\prefix{x}{y}{(@{y} | \outputp{x}{y})) | P}}}{y} & \nonumber\\
	=
	& \outputp{x}{\quotep{(\prefix{x}{y}{(\outputp{x}{y} | @{y})) | P}}}
	  | {(\prefix{x}{y}{(\outputp{x}{y} | @{y})) | P}} & \nonumber\\
	\red
	& \ldots & \nonumber\\
	\red^*
	& P | P | \ldots & \nonumber
\end{eqnarray}

Of course, this encoding, as an implementation, runs away, unfolding
$\bangp{P}$ eagerly. A lazier and more implementable replication
operator, restricted to input-guarded processes, may be obtained as follows.

\begin{eqnarray}
\bangp{\prefix{u}{v}{P}} 
	:= 
	\binpar{\lift{x}{\prefix{u}{v}{(\binpar{D(x)}{P})}}}{D(x)} \nonumber
\end{eqnarray}

\begin{remark}
  Note that the lazier definition still does not deal with summation
  or mixed summation (i.e. sums over input and output). The reader is
  invited to construct definitions of replication that deal with these
  features. 

  Further, the definitions are parameterized in a name, $x$. Can you,
  gentle reader, make a definition that eliminates this parameter and
  guarantees no accidental interaction between the replication
  machinery and the process being replicated -- i.e. no accidental
  sharing of names used by the process to get its work done and the
  name(s) used by the replication to effect copying. This latter
  revision of the definition of replication is crucial to obtaining
  the expected identity $!!P \sim !P$.
\end{remark}

\begin{remark}\label{rem:paradoxical_combinator}
  The reader familiar with the lambda calculus will have noticed the
  similarity between $D$ and the paradoxical combinator.

  [Ed. note: the existence of this seems to suggest we have to be more
  restrictive on the set of processes and names we admit if we are to
  support no-cloning.]
\end{remark}

\subsubsection{Bisimulation}

The computational dynamics gives rise to another kind of equivalence,
the equivalence of computational behavior. As previously mentioned
this is typically captured \emph{via} some form of bisimulation.

% The notion we use in this paper is weak barbed bisimulation
% \cite{milner91polyadicpi}.

The notion we use in this paper is derived from weak barbed
bisimulation \cite{milner91polyadicpi}. 

\begin{definition}
An \emph{observation relation}, $\downarrow_{\mathcal N}$, over a set
of names, $\mathcal N$, is the smallest relation satisfying the rules
below.

\infrule[Out-barb]{y \in {\mathcal N}, \; x \nameeq y}
		  {\outputp{x}{v} \downarrow_{\mathcal N} x}
\infrule[Par-barb]{\mbox{$P\downarrow_{\mathcal N} x$ or $Q\downarrow_{\mathcal N} x$}}
		  {\binpar{P}{Q} \downarrow_{\mathcal N} x}

We write $P \Downarrow_{\mathcal N} x$ if there is $Q$ such that 
$P \wred Q$ and $Q \downarrow_{\mathcal N} x$.
\end{definition}

\begin{definition}
%\label{def.bbisim}
An  ${\mathcal N}$-\emph{barbed bisimulation} over a set of names, ${\mathcal N}$, is a symmetric binary relation 
${\mathcal S}_{\mathcal N}$ between agents such that $P\rel{S}_{\mathcal N}Q$ implies:
\begin{enumerate}
\item If $P \red P'$ then $Q \wred Q'$ and $P'\rel{S}_{\mathcal N} Q'$.
\item If $P\downarrow_{\mathcal N} x$, then $Q\Downarrow_{\mathcal N} x$.
\end{enumerate}
$P$ is ${\mathcal N}$-barbed bisimilar to $Q$, written
$P \wbbisim_{\mathcal N} Q$, if $P \rel{S}_{\mathcal N} Q$ for some ${\mathcal N}$-barbed bisimulation ${\mathcal S}_{\mathcal N}$.
\end{definition}

$\mathcal{R} \subseteq \pi \times \pi$

$P \mathcal{R} Q => \forall P'. P \red P' \Rightarrow \exists Q'. Q \red Q', P' \mathcal{R} Q'$

$P \vdash x \Rightarrow Q \vdash x$

\begin{mathpar}
  \inferrule*[lab=Out-barb]{x \nameeq y}{{y}!\langle{Q}\rangle \vdash x}
  \and
  \inferrule*[lab=Par-barb]{\mbox{$P\vdash x$ or $Q\vdash x$}}{\binpar{P}{Q} \vdash x}
\end{mathpar}

\subsubsection{Contexts}

One of the principle advantages of computational calculi like the
$\pi$-calculus is a well-defined notion of context,
contextual-equivalence and a correlation between
contextual-equivalence and notions of bisimulation. The notion of
context allows the decomposition of a process into (sub-)process and
its syntactic environment, its context. Thus, a context may be
thought of as a process with a ``hole'' (written $\Box$) in it. The
application of a context $M$ to a process $P$, written $M[P]$, is
tantamount to filling the hole in $M$ with $P$. In this paper we do
not need the full weight of this theory, but do make use of the notion
of context in the proof the main theorem. 

\begin{mathpar}
  \inferrule* [lab=summation] {} {{M_{M},M_{N}} \bc \Box \;|\; x.M_{A} \;|\; M_{M}+M_{N}}
  \and
  \inferrule* [lab=agent] {} {{M_{A}} \bc (\vec{x})M_{P} \;| \; \clift{P_0,\ldots,M_{P},\ldots,P_N}}
  \and \\
  \inferrule* [lab=process] {} {{M_{P}} \bc M_{N} \;| \;P|M_{P} }
\end{mathpar} 

\begin{mathpar}
  \inferrule* [lab=sychronization] {} {M_{N} \bc \Box \;|\; x?M_{F} \;|\; x!M_{C}}
  \and
  \inferrule* [lab=abstraction] {} {{M_{F}} \bc (x)M_{P} }
  \and
  \inferrule* [lab=concretion] {} {{M_{C}} \bc \langle M_{P} \rangle }
  \and \\
  \inferrule* [lab=process] {} {{M_{P}} \bc M_{N} \;| \;P|M_{P} }
\end{mathpar}

\begin{definition}[contextual application] Given a context $M$, and
  process $P$, we define the \emph{contextual application}, $M[P] :=
  M\{P/\Box\}$. That is, the contextual application of M to P is the
  substitution of $P$ for $\Box$ in $M$.
\end{definition}

$\meaningof{-} : L \to \mathcal{P}(\pi)$

\begin{mathpar}
  \inferrule* [lab=collection] {} {\meaningof{true} = \pi, \and \meaningof{~E} = \pi \setminus \meaningof{E}, \and \meaningof{E_{1} \& E_{2}} = \meaningof{E_{1}} \cap \meaningof{E_{2}}}
\end{mathpar}

\begin{mathpar}
  \inferrule* [lab=structure] {} {\meaningof{0} = \{ P \in \pi | P \equiv 0 \}, \and \\ \meaningof{E_1 | E_2} = \{ P \in \pi | P \equiv P_{1} | P_{2}, P_{1} \in \meaningof{E_{1}}, P_{2} \in \meaningof{E_2}\} }
\end{mathpar}

\begin{mathpar}
 \inferrule* [lab=behavior] {} {\meaningof{\langle a?b \rangle E} = \{ P \in \pi | P \equiv Q | u?(y)P', \\ \and \\\\ \and \\ \;\;\; u \in \meaningof{a}, \forall z.P'\{z/y\} \in \meaningof{E\{z/b\}}\}, \and \\ \meaningof{a!E} = \{ P \in \pi | P \equiv Q | x!\langle P' \rangle, x \in \meaningof{a} P' \in \meaningof{E}\} }
\end{mathpar}

\begin{mathpar}
 \inferrule* [lab=nominal] {} {\meaningof{\quotep{E}} = \{ \quotep{P} \in \quotep{\pi} | P \in \meaningof{E} \}, \and \meaningof{\quotep{P}} = \{ \quotep{Q} \in \quotep{\pi} | P \equiv Q \} \and \\ \meaningof{@\quotep{E}} = \{ P \in \pi | P \equiv @x, x \in \meaningof{E} \}}
\end{mathpar}

\begin{eqnarray*}
  \\
  \meaningof{-} : TS \to ST
\end{eqnarray*}

\begin{eqnarray*}
  \\
  L : TS \to ST
\end{eqnarray*}

\begin{eqnarray*}
  \\
  P \models E \iff P \in \meaningof{E}
\end{eqnarray*}

\begin{eqnarray*}
  P \approx_{L} Q \iff \forall E \in L. P \models E \iff Q \models E
\end{eqnarray*}

\begin{eqnarray*}
  P \approx_{K} Q
\end{eqnarray*}

\begin{eqnarray*}
  P \approx Q
\end{eqnarray*}

$\approx_{K} = \approx = \approx_{L}$

\subsubsection{Contextual duality}

Note that contexts extend the quotation operation to a family of
operations from processes to names. Given a context, $M$, we can
define a \emph{nominal context}, $\quotep{M}$ by $\quotep{M}[P] :=
\quotep{M[P]}$. To foreshadow what is to come we observe that these
operations enjoy a duality with processes very much like the duality
between vectors and maps from vectors to scalars.

Further, because the calculus is essentially higher-order, we have a
correspondence between contexts and processes. More specifically,
given a name $x$ and a context $M$ we can construct $M^{*}_{x}$ such
that 

\begin{mathpar}
  M^{*}_{x} | \lift{x}{P} \red M[P]
\end{mathpar}

namely,

\begin{mathpar}
  M^{*}_{x} := x?(u).M[\dropn{u}]
\end{mathpar}

The dependence of $M^{*}_{x}$ on a name makes it an abstraction, 

\begin{mathpar}
  M^{*} := (x)x?(u).M[\dropn{u}]
\end{mathpar}

\subsection{Additional notation}

It will sometimes be convenient to denote the process a name
quotes. We already have the notation $x = \quotep{P}$, but it will be
convenient to introduce an alternate notation, $\procn{x}$, when we
want to emphasize the connection to the use of the name. Note that, by
virtue of name equivalence, $\quotep{\procn{x}} \nameeq x$; so, the
notation is consistent with previous definitions.

Further, because names have structure it is possible to effect
substitutions on the basis of that structure. This means we need to
upgrade our notation for substitutions, which we accomplish by
adapting comprehension notation. Thus,

\begin{mathpar}
  P\{ y / x : x \in S \}
\end{mathpar}

is interpreted to mean the process derived from P by replacing (in a
capture-avoiding manner) each occurrence of $x$ in $S$ by $y$. For example,

\begin{mathpar}
  P\{ \quotep{\procn{x}|\procn{x}} / x : x \in \freenames{P} \}
\end{mathpar}

will replace each (occurrence) of a free name $x$ in $P$ by
$\quotep{\procn{x}|\procn{x}}$.

Also, we will avail ourselves of the notation $x^{L}$ and $x^{R}$ to
denote injections of a name into disjoint copies of the name
space. There are numerous ways to accomplish this. One example can be
found in \cite{MeredithR05}. This notation overloads to vectors of
names: $\vec{x}^{\pi} := (x_{i}^{\pi} \; : \; 0 \leq i < |\vec{x}| )$ where $\pi \in \{L,R\}$.

We also use $P^{\Box} := P|\Box$.

In \cite{MeredithR05} an interpretation of the new operator is
given. It turns out that there are several possible interpretations
all enjoying the requisite algebraic properties of the operator (see
\cite{milner91polyadicpi}). We will therefore make liberal use of
$(\nu\; \vec{x})P$.

% subsection the_syntax_and_semantics_of_the_notation_system (end)   

\input{qm2pi.qmops} 

\input{qm2pi.sterngerlach} 

\input{qm2pi.metric} 

% section concurrent_process_calculi (end)

%\input{qm2pi.proofsketch}

% section proof sketch (end)

%\input{qm2pi.slviaknots} 

% section spatial logic via knots (end)

\input{qm2pi.conclusion}

% section conclusion (end)

%\input{qm2pi.dtcodes} 

% section wiring algorithm (end)

\input{qm2pi.ack} 

% section acknowledgments (end)

\newpage


\bibliographystyle{plain}   
\bibliography{../../biblios/main.bib}

\input{qm2pi.rhodetails}

\end{document}

 

%\documentclass[12pt]{llncs}
%\documentclass{jktr}

\usepackage[pdftex]{hyperref}                   
\usepackage {listings}
\usepackage {mathpartir}
\usepackage{bcprules}
%\usepackage{listings}
                       
\usepackage{graphicx} 
%\usepackage[margins=2.5cm,nohead,nofoot]{geometry}
%\usepackage{geometry}
\usepackage{amsfonts}
\usepackage{amstext}
\usepackage{latexsym}
\usepackage{amssymb}
\usepackage{color}


%\include{myPreamble}
\include{qm2pi.local} 

%\ifpdf
%\usepackage[pdftex]{graphicx}
%\else
%\usepackage{graphicx}
%\fi

 % \ifpdf
%  \usepackage{pdfsync}
%  \if


%\title{Brief Article}
%\author{David F. Snyder}
%\author{L.G. Meredith}

%\address{Dept. of Math., Texas State University--San Marcos, San Marcos, TX 78666}
       
\pagestyle{empty}


\begin{document}

\lstset{language=[Objective]Caml,frame=shadowbox}

\input{qm2pi.front}

% section front matter (end)

\input{qm2pi.intro} 
 
% section introduction (end)

% \input{qm2pi.knotations} 

% section notation (end)

\input{qm2pi.process.calculi} 

% section concurrent_process_calculi_and_spatial_logics_ (end)
    
%\input{qm2pi.knots2pi} 

%\input{qm2pi.trefoil} 

%\input{qm2pi.mainthm} 

% subsection basic_interpretation (end)

%\input{qm2pi.rho.presentation} 
\subsection{The syntax and semantics of the notation system}\label{sub:the_syntax_and_semantics_of_the_notation_system} % (fold)

We now summarize a technical presentation of the calculus that
embodies our theory of dynamics. The typical presentation of such a
calculus follows the style of giving generators and relations on
them. The grammar, below, describing term constructors, freely
generates the set of processes, $\Proc$. This set is then quotiented
by a relation known as structural congruence and it is over this set
that the notion of dynamics is expressed. This presentation is
essentially that of \cite{MeredithR05} with the addition of
polyadicity and summation. For readability we have relegated some of
the technical subtleties to an appendix.

\subsubsection{Process grammar}\label{subsub:process_grammar}

\begin{mathpar}
  \inferrule* [lab=synchronization] {} {{M} \bc \pzero \;|\; x?F \;|\; x!C }
  \and
  \inferrule* [lab=abstraction] {} {{F} \bc (x)P}
  \and
  \inferrule* [lab=concretion] {} {{C} \bc \langle Q \rangle}
  \and
  \inferrule* [lab=process] {} {{P,Q} \bc M \;| \;P|Q \;|\; @{x}}
  \and
  \inferrule* [lab=name] {} {{x} \bc \quotep{P}}
\end{mathpar} 

Note that $\vec{x}$ (resp. $\vec{P}$) denotes a vector of names
(resp. processes) of length $|\vec{x}|$ (resp. $|\vec{P}|$). We adopt
the following useful abbreviations.

\begin{mathpar}
   x?(\vec{y}).P := x.(\vec{y})P \and  x\clift{\vec{P}} := x.\clift{\vec{P}}
   \and x!(y) := \lift{x}{\dropn{y}}
   \and \Pi_{i=0}^{n-1}P_i := P_0 | \ldots | P_{n-1}
\end{mathpar}

\subsubsection{Structural congruence}

\paragraph{Free and bound names and alpha-equivalence.} At the
core of structural equivalence is alpha-equivalence which identifies
process that are the same up to a change of variable. Formally, we
recognize the distinction between free and bound names. The free names
of a process, $\freenames{P}$, may be calculated recursively as
follows:

\begin{mathpar}
\freenames{\pzero} := \emptyset
  \and \\
  \freenames{x?(y).P} := \{ x \} \cup (\freenames{P} \setminus \{ y \})
  \and 
  \freenames{x!\langle P \rangle} := \{ x \} \cup \{ P \} 
  \and \\
  \freenames{P|Q} := \freenames{P} \cup \freenames{Q}
  \and \\
  \freenames{@{x}} := \{ x \}
\end{mathpar}

$\pi$
$\quotep{\pi}$

$\freenames{-} : \pi \to \mathcal{P}(\quotep{\pi})$

\begin{eqnarray*}
  \freenames{\pzero} & := & \emptyset \\
  \freenames{x?(y).P} & := & \{ x \} \cup (\freenames{P} \setminus \{ y \}) \\
  \freenames{x!\langle P \rangle} & := & \{ x \} \cup \{ P \} \\
  \freenames{P|Q} & := & \freenames{P} \cup \freenames{Q} \\
  \freenames{\dropn{x}} & := & \{ x \}
\end{eqnarray*}

The bound names of a process, $\boundnames{P}$, are those names occurring in $P$
that are not free. For example, in $x?(y).0$, the name $x$ is free, while $y$ is bound.

\begin{mathpar}
  \inferrule* [lab=monoidal-laws] {} { P|Q \equiv Q|P \and P|0 \equiv P \and P|(Q|R) \equiv (P|Q)|R }
\end{mathpar}

\begin{mathpar}
  \inferrule* [lab=alpha-equivalence] {} { (x)P \equiv (y)P\{y/x\} \and y \not\in \freenames{P} }
\end{mathpar}

\begin{definition}
Then two processes, $P,Q$, are alpha-equivalent if $P = Q\{\vec{y}/\vec{x}\}$ for
some $\vec{x} \in \boundnames{Q},\vec{y} \in \boundnames{P}$, where $Q\{\vec{y}/\vec{x}\}$
denotes the capture-avoiding substitution of $\vec{y}$ for $\vec{x}$ in $Q$.
\end{definition}

\begin{definition}
  The {\em structural congruence} \cite{SangiorgiWalker} , $\equiv$,
  between processes is the least congruence containing
  alpha-equivalence, satisfying the abelian monoid laws
  (associativity, commutativity and $\pzero$ as identity) for parallel
  composition $|$ and for summation $+$.
\end{definition}

\subsection{Name equivalence}

We take name equivalence, written $\nameeq$, to be the smallest
equivalence relation generated by the following rules.

\begin{mathpar}
\inferrule*[lab=Quote-drop]
{ }
{ \quotep{@{x}} \nameeq x }

\inferrule*[lab=Struct-equiv]
{ P \scong Q }
{ \quotep{P} \nameeq \quotep{Q} }
\end{mathpar}

The astute reader will have noticed that the mutual recursion of names
and processes imposes a mutual recursion on alpha-equivalence and
structural equivalence via name-equivalence. Fortunately, all of this
works out pleasantly and we may calculate in the natural way, free of
concern. The reader interested in the details is referred to the
appendix \ref{appendix:rho_details}.

\subsection{Substitution}

We use $\Proc$ for the set of processes, $\QProc$ for the set of
names, and $\id{\{}\vec{y} / \vec{x} \id{\}}$ to denote partial maps,
$s : \QProc \rightarrow \QProc$. A map, $s$ lifts, uniquely, to a map
on process terms, $\widehat{s} : \Proc \rightarrow \Proc$ by the
following equations.

\begin{mathpar}
  (0) \psubstp{Q}{P} := 0 \\
  (R \juxtap S) \psubstp{Q}{P}
  :=    
  (R)\psubstp{Q}{P} \juxtap (S) \psubstp{Q}{P} \\
  (x?(y).R) \psubstp{Q}{P}    
  :=    
  (x)\substp{Q}{P} (z)\concat( (R \psubstn{z}{y}) \psubstp{Q}{P} ) \\
  (\lift{x}{R}) \psubstp{Q}{P}  
  :=
  \lift{(x)\substp{Q}{P}}{ R \psubstp{Q}{P} } \\
%   (\dropn{x})  \psubstp{Q}{P}       
%   := 
%   \left\{ 
%     \begin{array}{ccc} 
%       \dropn{\quotep{Q}} & & x \nameeq \quotep{P} \\
%       \dropn{x} & & otherwise \\
%     \end{array}
%   \right. 
  (\dropn{x})  \psubstp{Q}{P}       
  := 
  \left\{ 
    \begin{array}{ccc} 
      Q & & x \nameeq \quotep{P} \\
      \dropn{x} & & otherwise \\
    \end{array}
  \right.
\end{mathpar}
 

where

\begin{eqnarray}
  (x)\id{\{} \lpquote Q \rpquote / \lpquote P \rpquote \id{\}}            = 
  \left\{ 
    \begin{array}{ccc}
      \lpquote Q \rpquote & & x \nameeq \lpquote P \rpquote \\
      x & & otherwise \\
    \end{array}
  \right. \nonumber
\end{eqnarray}

and $z$ is chosen distinct from $\quotep{P}$, $\quotep{Q}$, the free
names in $Q$, and all the names in $R$. Our $\alpha$-equivalence will
be built in the standard way from this substitution.

\begin{remark}\label{rem:no_self_referential_names}
  One consequence of these definitions is that $\forall P. \quotep{P}
  \not\in \freenames{P}$.
\end{remark}

\subsection{ Dynamic quote: an example }

Anticipating something of what's to come, consider applying the
substitution, $\widehat{\id{\{}u / z \id{\}}}$, to the following pair
of processes, $\lift{w}{y!(z)}$ and $w[ \lpquote y!(z) \rpquote ]$.

\begin{eqnarray}
	\lift{w}{y!(z)}\widehat{\id{\{}u / z \id{\}}}
		& = &
		\lift{w}{y!(u)} \nonumber\\
	w[ \lpquote y!(z) \rpquote ] \widehat{ \id{\{}u / z \id{\}} }
		& = &
		w[ \lpquote y!(z) \rpquote ] \nonumber
\end{eqnarray}

Because the body of the process between quotes is impervious to
substitution, we get radically different answers. In fact, by
examining the first process in an input context,
e.g. $x?(z).\lift{w}{y!(z)}$, we see that the process under the lift
operator may be shaped by prefixed inputs binding a name inside it. In
this sense, the lift operator will be seen as a way to dynamically
construct processes before reifying them as names.

Finally equipped with these standard features we can present the
dynamics of the calculus.

\subsubsection{Operational semantics} 

Finally, we introduce the computational dynamics. What marks these
algebras as distinct from other more traditionally studied algebraic
structures, e.g. vector spaces or polynomial rings, is the manner in
which dynamics is captured. In traditional structures, dynamics is typically
expressed through morphisms between such structures, as in linear maps
between vector spaces or morphisms between rings. In algebras
associated with the semantics of computation, the dynamics is
expressed as part of the algebraic structure itself, through a
reduction reduction relation typically denoted by $\red$. Below, we
give a recursive presentation of this relation for the calculus used
in the encoding.

$\red \subseteq \pi \times \pi$
$\red : \pi \to \mathcal{P}(\pi)$

\begin{mathpar}
  \inferrule* [lab=Comm] { \textsf{match}( x_{src}, x_{trgt} ) } { x_{trgt}?(y)P \; | \; x_{src}!\langle {Q} \rangle \red P\{\quotep{Q}/y}\} }
  \and \\
  \inferrule* [lab=Par] {{P} \red {P}'} {{{P} | {Q}} \red {{P}' | {Q}}}
  \and
  \inferrule* [lab=Equiv]{{{P} \scong {P}'} \andalso {{P}' \red {Q}'} \andalso {{Q}' \scong {Q}}}{{P} \red {Q}}
\end{mathpar}

\begin{eqnarray*}
  match_{\equiv} (\quotep{P},\quotep{Q}) & := & P \equiv Q \\
  match_{\dagger}(\quotep{P},\quotep{Q}) & := & \forall R. P|Q \red^{*} R => R \red^{*} 0 \\
  match_{K}(\quotep{P},\quotep{Q}) & := & K \mbox{ for some context } K
\end{eqnarray*}

$u?(x)P | u!\langle Q \rangle \red P\{\quotep{Q}/x\}$

%We write $\wred$ for $\red^*$, and $P\red$ if $\exists Q $ such that $ P \red Q$.
We write $P\red$ if $\exists Q $ such that $ P \red Q$ and $P\not\red$, otherwise.

\section{Replication}

As mentioned before, it is known that replication (and hence
recursion) can be implemented in a higher-order process algebra
\cite{SangiorgiWalker}. As our first example of calculation with the
machinery thus far presented we give the construction explicitly in
the {\rhoc}.

\begin{eqnarray}
	D_{x} & := & \prefix{x}{y}{(\binpar{\outputp{x}{y}}{@{y}})} \nonumber\\
	\bangp_{x}{P} & := & \binpar{{x}!\langle{\binpar{D_{x}}{P}}\rangle}{D_{x}} \nonumber
\end{eqnarray}

\begin{eqnarray}
	\bangp_{x}{P} & & \nonumber\\
	=
	& {x}!\langle{(\prefix{x}{y}{(\outputp{x}{y} | @{y})) | P}}\rangle 
	      | \prefix{x}{y}{(\outputp{x}{y} | @{y})} & \nonumber\\
	\red
	& (\outputp{x}{y} | @{y})\substn{\quotep{(\prefix{x}{y}{(@{y} | \outputp{x}{y})) | P}}}{y} & \nonumber\\
	=
	& \outputp{x}{\quotep{(\prefix{x}{y}{(\outputp{x}{y} | @{y})) | P}}}
	  | {(\prefix{x}{y}{(\outputp{x}{y} | @{y})) | P}} & \nonumber\\
	\red
	& \ldots & \nonumber\\
	\red^*
	& P | P | \ldots & \nonumber
\end{eqnarray}

Of course, this encoding, as an implementation, runs away, unfolding
$\bangp{P}$ eagerly. A lazier and more implementable replication
operator, restricted to input-guarded processes, may be obtained as follows.

\begin{eqnarray}
\bangp{\prefix{u}{v}{P}} 
	:= 
	\binpar{\lift{x}{\prefix{u}{v}{(\binpar{D(x)}{P})}}}{D(x)} \nonumber
\end{eqnarray}

\begin{remark}
  Note that the lazier definition still does not deal with summation
  or mixed summation (i.e. sums over input and output). The reader is
  invited to construct definitions of replication that deal with these
  features. 

  Further, the definitions are parameterized in a name, $x$. Can you,
  gentle reader, make a definition that eliminates this parameter and
  guarantees no accidental interaction between the replication
  machinery and the process being replicated -- i.e. no accidental
  sharing of names used by the process to get its work done and the
  name(s) used by the replication to effect copying. This latter
  revision of the definition of replication is crucial to obtaining
  the expected identity $!!P \sim !P$.
\end{remark}

\begin{remark}\label{rem:paradoxical_combinator}
  The reader familiar with the lambda calculus will have noticed the
  similarity between $D$ and the paradoxical combinator.

  [Ed. note: the existence of this seems to suggest we have to be more
  restrictive on the set of processes and names we admit if we are to
  support no-cloning.]
\end{remark}

\subsubsection{Bisimulation}

The computational dynamics gives rise to another kind of equivalence,
the equivalence of computational behavior. As previously mentioned
this is typically captured \emph{via} some form of bisimulation.

% The notion we use in this paper is weak barbed bisimulation
% \cite{milner91polyadicpi}.

The notion we use in this paper is derived from weak barbed
bisimulation \cite{milner91polyadicpi}. 

\begin{definition}
An \emph{observation relation}, $\downarrow_{\mathcal N}$, over a set
of names, $\mathcal N$, is the smallest relation satisfying the rules
below.

\infrule[Out-barb]{y \in {\mathcal N}, \; x \nameeq y}
		  {\outputp{x}{v} \downarrow_{\mathcal N} x}
\infrule[Par-barb]{\mbox{$P\downarrow_{\mathcal N} x$ or $Q\downarrow_{\mathcal N} x$}}
		  {\binpar{P}{Q} \downarrow_{\mathcal N} x}

We write $P \Downarrow_{\mathcal N} x$ if there is $Q$ such that 
$P \wred Q$ and $Q \downarrow_{\mathcal N} x$.
\end{definition}

\begin{definition}
%\label{def.bbisim}
An  ${\mathcal N}$-\emph{barbed bisimulation} over a set of names, ${\mathcal N}$, is a symmetric binary relation 
${\mathcal S}_{\mathcal N}$ between agents such that $P\rel{S}_{\mathcal N}Q$ implies:
\begin{enumerate}
\item If $P \red P'$ then $Q \wred Q'$ and $P'\rel{S}_{\mathcal N} Q'$.
\item If $P\downarrow_{\mathcal N} x$, then $Q\Downarrow_{\mathcal N} x$.
\end{enumerate}
$P$ is ${\mathcal N}$-barbed bisimilar to $Q$, written
$P \wbbisim_{\mathcal N} Q$, if $P \rel{S}_{\mathcal N} Q$ for some ${\mathcal N}$-barbed bisimulation ${\mathcal S}_{\mathcal N}$.
\end{definition}

$\mathcal{R} \subseteq \pi \times \pi$

$P \mathcal{R} Q => \forall P'. P \red P' \Rightarrow \exists Q'. Q \red Q', P' \mathcal{R} Q'$

$P \vdash x \Rightarrow Q \vdash x$

\begin{mathpar}
  \inferrule*[lab=Out-barb]{x \nameeq y}{{y}!\langle{Q}\rangle \vdash x}
  \and
  \inferrule*[lab=Par-barb]{\mbox{$P\vdash x$ or $Q\vdash x$}}{\binpar{P}{Q} \vdash x}
\end{mathpar}

\subsubsection{Contexts}

One of the principle advantages of computational calculi like the
$\pi$-calculus is a well-defined notion of context,
contextual-equivalence and a correlation between
contextual-equivalence and notions of bisimulation. The notion of
context allows the decomposition of a process into (sub-)process and
its syntactic environment, its context. Thus, a context may be
thought of as a process with a ``hole'' (written $\Box$) in it. The
application of a context $M$ to a process $P$, written $M[P]$, is
tantamount to filling the hole in $M$ with $P$. In this paper we do
not need the full weight of this theory, but do make use of the notion
of context in the proof the main theorem. 

\begin{mathpar}
  \inferrule* [lab=summation] {} {{M_{M},M_{N}} \bc \Box \;|\; x.M_{A} \;|\; M_{M}+M_{N}}
  \and
  \inferrule* [lab=agent] {} {{M_{A}} \bc (\vec{x})M_{P} \;| \; \clift{P_0,\ldots,M_{P},\ldots,P_N}}
  \and \\
  \inferrule* [lab=process] {} {{M_{P}} \bc M_{N} \;| \;P|M_{P} }
\end{mathpar} 

\begin{mathpar}
  \inferrule* [lab=sychronization] {} {M_{N} \bc \Box \;|\; x?M_{F} \;|\; x!M_{C}}
  \and
  \inferrule* [lab=abstraction] {} {{M_{F}} \bc (x)M_{P} }
  \and
  \inferrule* [lab=concretion] {} {{M_{C}} \bc \langle M_{P} \rangle }
  \and \\
  \inferrule* [lab=process] {} {{M_{P}} \bc M_{N} \;| \;P|M_{P} }
\end{mathpar}

\begin{definition}[contextual application] Given a context $M$, and
  process $P$, we define the \emph{contextual application}, $M[P] :=
  M\{P/\Box\}$. That is, the contextual application of M to P is the
  substitution of $P$ for $\Box$ in $M$.
\end{definition}

$\meaningof{-} : L \to \mathcal{P}(\pi)$

\begin{mathpar}
  \inferrule* [lab=collection] {} {\meaningof{true} = \pi, \and \meaningof{~E} = \pi \setminus \meaningof{E}, \and \meaningof{E_{1} \& E_{2}} = \meaningof{E_{1}} \cap \meaningof{E_{2}}}
\end{mathpar}

\begin{mathpar}
  \inferrule* [lab=structure] {} {\meaningof{0} = \{ P \in \pi | P \equiv 0 \}, \and \\ \meaningof{E_1 | E_2} = \{ P \in \pi | P \equiv P_{1} | P_{2}, P_{1} \in \meaningof{E_{1}}, P_{2} \in \meaningof{E_2}\} }
\end{mathpar}

\begin{mathpar}
 \inferrule* [lab=behavior] {} {\meaningof{\langle a?b \rangle E} = \{ P \in \pi | P \equiv Q | u?(y)P', \\ \and \\\\ \and \\ \;\;\; u \in \meaningof{a}, \forall z.P'\{z/y\} \in \meaningof{E\{z/b\}}\}, \and \\ \meaningof{a!E} = \{ P \in \pi | P \equiv Q | x!\langle P' \rangle, x \in \meaningof{a} P' \in \meaningof{E}\} }
\end{mathpar}

\begin{mathpar}
 \inferrule* [lab=nominal] {} {\meaningof{\quotep{E}} = \{ \quotep{P} \in \quotep{\pi} | P \in \meaningof{E} \}, \and \meaningof{\quotep{P}} = \{ \quotep{Q} \in \quotep{\pi} | P \equiv Q \} \and \\ \meaningof{@\quotep{E}} = \{ P \in \pi | P \equiv @x, x \in \meaningof{E} \}}
\end{mathpar}

\begin{eqnarray*}
  \\
  \meaningof{-} : TS \to ST
\end{eqnarray*}

\begin{eqnarray*}
  \\
  L : TS \to ST
\end{eqnarray*}

\begin{eqnarray*}
  \\
  P \models E \iff P \in \meaningof{E}
\end{eqnarray*}

\begin{eqnarray*}
  P \approx_{L} Q \iff \forall E \in L. P \models E \iff Q \models E
\end{eqnarray*}

\begin{eqnarray*}
  P \approx_{K} Q
\end{eqnarray*}

\begin{eqnarray*}
  P \approx Q
\end{eqnarray*}

$\approx_{K} = \approx = \approx_{L}$

\subsubsection{Contextual duality}

Note that contexts extend the quotation operation to a family of
operations from processes to names. Given a context, $M$, we can
define a \emph{nominal context}, $\quotep{M}$ by $\quotep{M}[P] :=
\quotep{M[P]}$. To foreshadow what is to come we observe that these
operations enjoy a duality with processes very much like the duality
between vectors and maps from vectors to scalars.

Further, because the calculus is essentially higher-order, we have a
correspondence between contexts and processes. More specifically,
given a name $x$ and a context $M$ we can construct $M^{*}_{x}$ such
that 

\begin{mathpar}
  M^{*}_{x} | \lift{x}{P} \red M[P]
\end{mathpar}

namely,

\begin{mathpar}
  M^{*}_{x} := x?(u).M[\dropn{u}]
\end{mathpar}

The dependence of $M^{*}_{x}$ on a name makes it an abstraction, 

\begin{mathpar}
  M^{*} := (x)x?(u).M[\dropn{u}]
\end{mathpar}

\subsection{Additional notation}

It will sometimes be convenient to denote the process a name
quotes. We already have the notation $x = \quotep{P}$, but it will be
convenient to introduce an alternate notation, $\procn{x}$, when we
want to emphasize the connection to the use of the name. Note that, by
virtue of name equivalence, $\quotep{\procn{x}} \nameeq x$; so, the
notation is consistent with previous definitions.

Further, because names have structure it is possible to effect
substitutions on the basis of that structure. This means we need to
upgrade our notation for substitutions, which we accomplish by
adapting comprehension notation. Thus,

\begin{mathpar}
  P\{ y / x : x \in S \}
\end{mathpar}

is interpreted to mean the process derived from P by replacing (in a
capture-avoiding manner) each occurrence of $x$ in $S$ by $y$. For example,

\begin{mathpar}
  P\{ \quotep{\procn{x}|\procn{x}} / x : x \in \freenames{P} \}
\end{mathpar}

will replace each (occurrence) of a free name $x$ in $P$ by
$\quotep{\procn{x}|\procn{x}}$.

Also, we will avail ourselves of the notation $x^{L}$ and $x^{R}$ to
denote injections of a name into disjoint copies of the name
space. There are numerous ways to accomplish this. One example can be
found in \cite{MeredithR05}. This notation overloads to vectors of
names: $\vec{x}^{\pi} := (x_{i}^{\pi} \; : \; 0 \leq i < |\vec{x}| )$ where $\pi \in \{L,R\}$.

We also use $P^{\Box} := P|\Box$.

In \cite{MeredithR05} an interpretation of the new operator is
given. It turns out that there are several possible interpretations
all enjoying the requisite algebraic properties of the operator (see
\cite{milner91polyadicpi}). We will therefore make liberal use of
$(\nu\; \vec{x})P$.

% subsection the_syntax_and_semantics_of_the_notation_system (end)   

\input{qm2pi.qmops} 

\input{qm2pi.sterngerlach} 

\input{qm2pi.metric} 

% section concurrent_process_calculi (end)

%\input{qm2pi.proofsketch}

% section proof sketch (end)

%\input{qm2pi.slviaknots} 

% section spatial logic via knots (end)

\input{qm2pi.conclusion}

% section conclusion (end)

%\input{qm2pi.dtcodes} 

% section wiring algorithm (end)

\input{qm2pi.ack} 

% section acknowledgments (end)

\newpage


\bibliographystyle{plain}   
\bibliography{../../biblios/main.bib}

\input{qm2pi.rhodetails}

\end{document}

 

%\documentclass[12pt]{llncs}
%\documentclass{jktr}

\usepackage[pdftex]{hyperref}                   
\usepackage {listings}
\usepackage {mathpartir}
\usepackage{bcprules}
%\usepackage{listings}
                       
\usepackage{graphicx} 
%\usepackage[margins=2.5cm,nohead,nofoot]{geometry}
%\usepackage{geometry}
\usepackage{amsfonts}
\usepackage{amstext}
\usepackage{latexsym}
\usepackage{amssymb}
\usepackage{color}


%\include{myPreamble}
\include{qm2pi.local} 

%\ifpdf
%\usepackage[pdftex]{graphicx}
%\else
%\usepackage{graphicx}
%\fi

 % \ifpdf
%  \usepackage{pdfsync}
%  \if


%\title{Brief Article}
%\author{David F. Snyder}
%\author{L.G. Meredith}

%\address{Dept. of Math., Texas State University--San Marcos, San Marcos, TX 78666}
       
\pagestyle{empty}


\begin{document}

\lstset{language=[Objective]Caml,frame=shadowbox}

\input{qm2pi.front}

% section front matter (end)

\input{qm2pi.intro} 
 
% section introduction (end)

% \input{qm2pi.knotations} 

% section notation (end)

\input{qm2pi.process.calculi} 

% section concurrent_process_calculi_and_spatial_logics_ (end)
    
%\input{qm2pi.knots2pi} 

%\input{qm2pi.trefoil} 

%\input{qm2pi.mainthm} 

% subsection basic_interpretation (end)

%\input{qm2pi.rho.presentation} 
\subsection{The syntax and semantics of the notation system}\label{sub:the_syntax_and_semantics_of_the_notation_system} % (fold)

We now summarize a technical presentation of the calculus that
embodies our theory of dynamics. The typical presentation of such a
calculus follows the style of giving generators and relations on
them. The grammar, below, describing term constructors, freely
generates the set of processes, $\Proc$. This set is then quotiented
by a relation known as structural congruence and it is over this set
that the notion of dynamics is expressed. This presentation is
essentially that of \cite{MeredithR05} with the addition of
polyadicity and summation. For readability we have relegated some of
the technical subtleties to an appendix.

\subsubsection{Process grammar}\label{subsub:process_grammar}

\begin{mathpar}
  \inferrule* [lab=synchronization] {} {{M} \bc \pzero \;|\; x?F \;|\; x!C }
  \and
  \inferrule* [lab=abstraction] {} {{F} \bc (x)P}
  \and
  \inferrule* [lab=concretion] {} {{C} \bc \langle Q \rangle}
  \and
  \inferrule* [lab=process] {} {{P,Q} \bc M \;| \;P|Q \;|\; @{x}}
  \and
  \inferrule* [lab=name] {} {{x} \bc \quotep{P}}
\end{mathpar} 

Note that $\vec{x}$ (resp. $\vec{P}$) denotes a vector of names
(resp. processes) of length $|\vec{x}|$ (resp. $|\vec{P}|$). We adopt
the following useful abbreviations.

\begin{mathpar}
   x?(\vec{y}).P := x.(\vec{y})P \and  x\clift{\vec{P}} := x.\clift{\vec{P}}
   \and x!(y) := \lift{x}{\dropn{y}}
   \and \Pi_{i=0}^{n-1}P_i := P_0 | \ldots | P_{n-1}
\end{mathpar}

\subsubsection{Structural congruence}

\paragraph{Free and bound names and alpha-equivalence.} At the
core of structural equivalence is alpha-equivalence which identifies
process that are the same up to a change of variable. Formally, we
recognize the distinction between free and bound names. The free names
of a process, $\freenames{P}$, may be calculated recursively as
follows:

\begin{mathpar}
\freenames{\pzero} := \emptyset
  \and \\
  \freenames{x?(y).P} := \{ x \} \cup (\freenames{P} \setminus \{ y \})
  \and 
  \freenames{x!\langle P \rangle} := \{ x \} \cup \{ P \} 
  \and \\
  \freenames{P|Q} := \freenames{P} \cup \freenames{Q}
  \and \\
  \freenames{@{x}} := \{ x \}
\end{mathpar}

$\pi$
$\quotep{\pi}$

$\freenames{-} : \pi \to \mathcal{P}(\quotep{\pi})$

\begin{eqnarray*}
  \freenames{\pzero} & := & \emptyset \\
  \freenames{x?(y).P} & := & \{ x \} \cup (\freenames{P} \setminus \{ y \}) \\
  \freenames{x!\langle P \rangle} & := & \{ x \} \cup \{ P \} \\
  \freenames{P|Q} & := & \freenames{P} \cup \freenames{Q} \\
  \freenames{\dropn{x}} & := & \{ x \}
\end{eqnarray*}

The bound names of a process, $\boundnames{P}$, are those names occurring in $P$
that are not free. For example, in $x?(y).0$, the name $x$ is free, while $y$ is bound.

\begin{mathpar}
  \inferrule* [lab=monoidal-laws] {} { P|Q \equiv Q|P \and P|0 \equiv P \and P|(Q|R) \equiv (P|Q)|R }
\end{mathpar}

\begin{mathpar}
  \inferrule* [lab=alpha-equivalence] {} { (x)P \equiv (y)P\{y/x\} \and y \not\in \freenames{P} }
\end{mathpar}

\begin{definition}
Then two processes, $P,Q$, are alpha-equivalent if $P = Q\{\vec{y}/\vec{x}\}$ for
some $\vec{x} \in \boundnames{Q},\vec{y} \in \boundnames{P}$, where $Q\{\vec{y}/\vec{x}\}$
denotes the capture-avoiding substitution of $\vec{y}$ for $\vec{x}$ in $Q$.
\end{definition}

\begin{definition}
  The {\em structural congruence} \cite{SangiorgiWalker} , $\equiv$,
  between processes is the least congruence containing
  alpha-equivalence, satisfying the abelian monoid laws
  (associativity, commutativity and $\pzero$ as identity) for parallel
  composition $|$ and for summation $+$.
\end{definition}

\subsection{Name equivalence}

We take name equivalence, written $\nameeq$, to be the smallest
equivalence relation generated by the following rules.

\begin{mathpar}
\inferrule*[lab=Quote-drop]
{ }
{ \quotep{@{x}} \nameeq x }

\inferrule*[lab=Struct-equiv]
{ P \scong Q }
{ \quotep{P} \nameeq \quotep{Q} }
\end{mathpar}

The astute reader will have noticed that the mutual recursion of names
and processes imposes a mutual recursion on alpha-equivalence and
structural equivalence via name-equivalence. Fortunately, all of this
works out pleasantly and we may calculate in the natural way, free of
concern. The reader interested in the details is referred to the
appendix \ref{appendix:rho_details}.

\subsection{Substitution}

We use $\Proc$ for the set of processes, $\QProc$ for the set of
names, and $\id{\{}\vec{y} / \vec{x} \id{\}}$ to denote partial maps,
$s : \QProc \rightarrow \QProc$. A map, $s$ lifts, uniquely, to a map
on process terms, $\widehat{s} : \Proc \rightarrow \Proc$ by the
following equations.

\begin{mathpar}
  (0) \psubstp{Q}{P} := 0 \\
  (R \juxtap S) \psubstp{Q}{P}
  :=    
  (R)\psubstp{Q}{P} \juxtap (S) \psubstp{Q}{P} \\
  (x?(y).R) \psubstp{Q}{P}    
  :=    
  (x)\substp{Q}{P} (z)\concat( (R \psubstn{z}{y}) \psubstp{Q}{P} ) \\
  (\lift{x}{R}) \psubstp{Q}{P}  
  :=
  \lift{(x)\substp{Q}{P}}{ R \psubstp{Q}{P} } \\
%   (\dropn{x})  \psubstp{Q}{P}       
%   := 
%   \left\{ 
%     \begin{array}{ccc} 
%       \dropn{\quotep{Q}} & & x \nameeq \quotep{P} \\
%       \dropn{x} & & otherwise \\
%     \end{array}
%   \right. 
  (\dropn{x})  \psubstp{Q}{P}       
  := 
  \left\{ 
    \begin{array}{ccc} 
      Q & & x \nameeq \quotep{P} \\
      \dropn{x} & & otherwise \\
    \end{array}
  \right.
\end{mathpar}
 

where

\begin{eqnarray}
  (x)\id{\{} \lpquote Q \rpquote / \lpquote P \rpquote \id{\}}            = 
  \left\{ 
    \begin{array}{ccc}
      \lpquote Q \rpquote & & x \nameeq \lpquote P \rpquote \\
      x & & otherwise \\
    \end{array}
  \right. \nonumber
\end{eqnarray}

and $z$ is chosen distinct from $\quotep{P}$, $\quotep{Q}$, the free
names in $Q$, and all the names in $R$. Our $\alpha$-equivalence will
be built in the standard way from this substitution.

\begin{remark}\label{rem:no_self_referential_names}
  One consequence of these definitions is that $\forall P. \quotep{P}
  \not\in \freenames{P}$.
\end{remark}

\subsection{ Dynamic quote: an example }

Anticipating something of what's to come, consider applying the
substitution, $\widehat{\id{\{}u / z \id{\}}}$, to the following pair
of processes, $\lift{w}{y!(z)}$ and $w[ \lpquote y!(z) \rpquote ]$.

\begin{eqnarray}
	\lift{w}{y!(z)}\widehat{\id{\{}u / z \id{\}}}
		& = &
		\lift{w}{y!(u)} \nonumber\\
	w[ \lpquote y!(z) \rpquote ] \widehat{ \id{\{}u / z \id{\}} }
		& = &
		w[ \lpquote y!(z) \rpquote ] \nonumber
\end{eqnarray}

Because the body of the process between quotes is impervious to
substitution, we get radically different answers. In fact, by
examining the first process in an input context,
e.g. $x?(z).\lift{w}{y!(z)}$, we see that the process under the lift
operator may be shaped by prefixed inputs binding a name inside it. In
this sense, the lift operator will be seen as a way to dynamically
construct processes before reifying them as names.

Finally equipped with these standard features we can present the
dynamics of the calculus.

\subsubsection{Operational semantics} 

Finally, we introduce the computational dynamics. What marks these
algebras as distinct from other more traditionally studied algebraic
structures, e.g. vector spaces or polynomial rings, is the manner in
which dynamics is captured. In traditional structures, dynamics is typically
expressed through morphisms between such structures, as in linear maps
between vector spaces or morphisms between rings. In algebras
associated with the semantics of computation, the dynamics is
expressed as part of the algebraic structure itself, through a
reduction reduction relation typically denoted by $\red$. Below, we
give a recursive presentation of this relation for the calculus used
in the encoding.

$\red \subseteq \pi \times \pi$
$\red : \pi \to \mathcal{P}(\pi)$

\begin{mathpar}
  \inferrule* [lab=Comm] { \textsf{match}( x_{src}, x_{trgt} ) } { x_{trgt}?(y)P \; | \; x_{src}!\langle {Q} \rangle \red P\{\quotep{Q}/y}\} }
  \and \\
  \inferrule* [lab=Par] {{P} \red {P}'} {{{P} | {Q}} \red {{P}' | {Q}}}
  \and
  \inferrule* [lab=Equiv]{{{P} \scong {P}'} \andalso {{P}' \red {Q}'} \andalso {{Q}' \scong {Q}}}{{P} \red {Q}}
\end{mathpar}

\begin{eqnarray*}
  match_{\equiv} (\quotep{P},\quotep{Q}) & := & P \equiv Q \\
  match_{\dagger}(\quotep{P},\quotep{Q}) & := & \forall R. P|Q \red^{*} R => R \red^{*} 0 \\
  match_{K}(\quotep{P},\quotep{Q}) & := & K \mbox{ for some context } K
\end{eqnarray*}

$u?(x)P | u!\langle Q \rangle \red P\{\quotep{Q}/x\}$

%We write $\wred$ for $\red^*$, and $P\red$ if $\exists Q $ such that $ P \red Q$.
We write $P\red$ if $\exists Q $ such that $ P \red Q$ and $P\not\red$, otherwise.

\section{Replication}

As mentioned before, it is known that replication (and hence
recursion) can be implemented in a higher-order process algebra
\cite{SangiorgiWalker}. As our first example of calculation with the
machinery thus far presented we give the construction explicitly in
the {\rhoc}.

\begin{eqnarray}
	D_{x} & := & \prefix{x}{y}{(\binpar{\outputp{x}{y}}{@{y}})} \nonumber\\
	\bangp_{x}{P} & := & \binpar{{x}!\langle{\binpar{D_{x}}{P}}\rangle}{D_{x}} \nonumber
\end{eqnarray}

\begin{eqnarray}
	\bangp_{x}{P} & & \nonumber\\
	=
	& {x}!\langle{(\prefix{x}{y}{(\outputp{x}{y} | @{y})) | P}}\rangle 
	      | \prefix{x}{y}{(\outputp{x}{y} | @{y})} & \nonumber\\
	\red
	& (\outputp{x}{y} | @{y})\substn{\quotep{(\prefix{x}{y}{(@{y} | \outputp{x}{y})) | P}}}{y} & \nonumber\\
	=
	& \outputp{x}{\quotep{(\prefix{x}{y}{(\outputp{x}{y} | @{y})) | P}}}
	  | {(\prefix{x}{y}{(\outputp{x}{y} | @{y})) | P}} & \nonumber\\
	\red
	& \ldots & \nonumber\\
	\red^*
	& P | P | \ldots & \nonumber
\end{eqnarray}

Of course, this encoding, as an implementation, runs away, unfolding
$\bangp{P}$ eagerly. A lazier and more implementable replication
operator, restricted to input-guarded processes, may be obtained as follows.

\begin{eqnarray}
\bangp{\prefix{u}{v}{P}} 
	:= 
	\binpar{\lift{x}{\prefix{u}{v}{(\binpar{D(x)}{P})}}}{D(x)} \nonumber
\end{eqnarray}

\begin{remark}
  Note that the lazier definition still does not deal with summation
  or mixed summation (i.e. sums over input and output). The reader is
  invited to construct definitions of replication that deal with these
  features. 

  Further, the definitions are parameterized in a name, $x$. Can you,
  gentle reader, make a definition that eliminates this parameter and
  guarantees no accidental interaction between the replication
  machinery and the process being replicated -- i.e. no accidental
  sharing of names used by the process to get its work done and the
  name(s) used by the replication to effect copying. This latter
  revision of the definition of replication is crucial to obtaining
  the expected identity $!!P \sim !P$.
\end{remark}

\begin{remark}\label{rem:paradoxical_combinator}
  The reader familiar with the lambda calculus will have noticed the
  similarity between $D$ and the paradoxical combinator.

  [Ed. note: the existence of this seems to suggest we have to be more
  restrictive on the set of processes and names we admit if we are to
  support no-cloning.]
\end{remark}

\subsubsection{Bisimulation}

The computational dynamics gives rise to another kind of equivalence,
the equivalence of computational behavior. As previously mentioned
this is typically captured \emph{via} some form of bisimulation.

% The notion we use in this paper is weak barbed bisimulation
% \cite{milner91polyadicpi}.

The notion we use in this paper is derived from weak barbed
bisimulation \cite{milner91polyadicpi}. 

\begin{definition}
An \emph{observation relation}, $\downarrow_{\mathcal N}$, over a set
of names, $\mathcal N$, is the smallest relation satisfying the rules
below.

\infrule[Out-barb]{y \in {\mathcal N}, \; x \nameeq y}
		  {\outputp{x}{v} \downarrow_{\mathcal N} x}
\infrule[Par-barb]{\mbox{$P\downarrow_{\mathcal N} x$ or $Q\downarrow_{\mathcal N} x$}}
		  {\binpar{P}{Q} \downarrow_{\mathcal N} x}

We write $P \Downarrow_{\mathcal N} x$ if there is $Q$ such that 
$P \wred Q$ and $Q \downarrow_{\mathcal N} x$.
\end{definition}

\begin{definition}
%\label{def.bbisim}
An  ${\mathcal N}$-\emph{barbed bisimulation} over a set of names, ${\mathcal N}$, is a symmetric binary relation 
${\mathcal S}_{\mathcal N}$ between agents such that $P\rel{S}_{\mathcal N}Q$ implies:
\begin{enumerate}
\item If $P \red P'$ then $Q \wred Q'$ and $P'\rel{S}_{\mathcal N} Q'$.
\item If $P\downarrow_{\mathcal N} x$, then $Q\Downarrow_{\mathcal N} x$.
\end{enumerate}
$P$ is ${\mathcal N}$-barbed bisimilar to $Q$, written
$P \wbbisim_{\mathcal N} Q$, if $P \rel{S}_{\mathcal N} Q$ for some ${\mathcal N}$-barbed bisimulation ${\mathcal S}_{\mathcal N}$.
\end{definition}

$\mathcal{R} \subseteq \pi \times \pi$

$P \mathcal{R} Q => \forall P'. P \red P' \Rightarrow \exists Q'. Q \red Q', P' \mathcal{R} Q'$

$P \vdash x \Rightarrow Q \vdash x$

\begin{mathpar}
  \inferrule*[lab=Out-barb]{x \nameeq y}{{y}!\langle{Q}\rangle \vdash x}
  \and
  \inferrule*[lab=Par-barb]{\mbox{$P\vdash x$ or $Q\vdash x$}}{\binpar{P}{Q} \vdash x}
\end{mathpar}

\subsubsection{Contexts}

One of the principle advantages of computational calculi like the
$\pi$-calculus is a well-defined notion of context,
contextual-equivalence and a correlation between
contextual-equivalence and notions of bisimulation. The notion of
context allows the decomposition of a process into (sub-)process and
its syntactic environment, its context. Thus, a context may be
thought of as a process with a ``hole'' (written $\Box$) in it. The
application of a context $M$ to a process $P$, written $M[P]$, is
tantamount to filling the hole in $M$ with $P$. In this paper we do
not need the full weight of this theory, but do make use of the notion
of context in the proof the main theorem. 

\begin{mathpar}
  \inferrule* [lab=summation] {} {{M_{M},M_{N}} \bc \Box \;|\; x.M_{A} \;|\; M_{M}+M_{N}}
  \and
  \inferrule* [lab=agent] {} {{M_{A}} \bc (\vec{x})M_{P} \;| \; \clift{P_0,\ldots,M_{P},\ldots,P_N}}
  \and \\
  \inferrule* [lab=process] {} {{M_{P}} \bc M_{N} \;| \;P|M_{P} }
\end{mathpar} 

\begin{mathpar}
  \inferrule* [lab=sychronization] {} {M_{N} \bc \Box \;|\; x?M_{F} \;|\; x!M_{C}}
  \and
  \inferrule* [lab=abstraction] {} {{M_{F}} \bc (x)M_{P} }
  \and
  \inferrule* [lab=concretion] {} {{M_{C}} \bc \langle M_{P} \rangle }
  \and \\
  \inferrule* [lab=process] {} {{M_{P}} \bc M_{N} \;| \;P|M_{P} }
\end{mathpar}

\begin{definition}[contextual application] Given a context $M$, and
  process $P$, we define the \emph{contextual application}, $M[P] :=
  M\{P/\Box\}$. That is, the contextual application of M to P is the
  substitution of $P$ for $\Box$ in $M$.
\end{definition}

$\meaningof{-} : L \to \mathcal{P}(\pi)$

\begin{mathpar}
  \inferrule* [lab=collection] {} {\meaningof{true} = \pi, \and \meaningof{~E} = \pi \setminus \meaningof{E}, \and \meaningof{E_{1} \& E_{2}} = \meaningof{E_{1}} \cap \meaningof{E_{2}}}
\end{mathpar}

\begin{mathpar}
  \inferrule* [lab=structure] {} {\meaningof{0} = \{ P \in \pi | P \equiv 0 \}, \and \\ \meaningof{E_1 | E_2} = \{ P \in \pi | P \equiv P_{1} | P_{2}, P_{1} \in \meaningof{E_{1}}, P_{2} \in \meaningof{E_2}\} }
\end{mathpar}

\begin{mathpar}
 \inferrule* [lab=behavior] {} {\meaningof{\langle a?b \rangle E} = \{ P \in \pi | P \equiv Q | u?(y)P', \\ \and \\\\ \and \\ \;\;\; u \in \meaningof{a}, \forall z.P'\{z/y\} \in \meaningof{E\{z/b\}}\}, \and \\ \meaningof{a!E} = \{ P \in \pi | P \equiv Q | x!\langle P' \rangle, x \in \meaningof{a} P' \in \meaningof{E}\} }
\end{mathpar}

\begin{mathpar}
 \inferrule* [lab=nominal] {} {\meaningof{\quotep{E}} = \{ \quotep{P} \in \quotep{\pi} | P \in \meaningof{E} \}, \and \meaningof{\quotep{P}} = \{ \quotep{Q} \in \quotep{\pi} | P \equiv Q \} \and \\ \meaningof{@\quotep{E}} = \{ P \in \pi | P \equiv @x, x \in \meaningof{E} \}}
\end{mathpar}

\begin{eqnarray*}
  \\
  \meaningof{-} : TS \to ST
\end{eqnarray*}

\begin{eqnarray*}
  \\
  L : TS \to ST
\end{eqnarray*}

\begin{eqnarray*}
  \\
  P \models E \iff P \in \meaningof{E}
\end{eqnarray*}

\begin{eqnarray*}
  P \approx_{L} Q \iff \forall E \in L. P \models E \iff Q \models E
\end{eqnarray*}

\begin{eqnarray*}
  P \approx_{K} Q
\end{eqnarray*}

\begin{eqnarray*}
  P \approx Q
\end{eqnarray*}

$\approx_{K} = \approx = \approx_{L}$

\subsubsection{Contextual duality}

Note that contexts extend the quotation operation to a family of
operations from processes to names. Given a context, $M$, we can
define a \emph{nominal context}, $\quotep{M}$ by $\quotep{M}[P] :=
\quotep{M[P]}$. To foreshadow what is to come we observe that these
operations enjoy a duality with processes very much like the duality
between vectors and maps from vectors to scalars.

Further, because the calculus is essentially higher-order, we have a
correspondence between contexts and processes. More specifically,
given a name $x$ and a context $M$ we can construct $M^{*}_{x}$ such
that 

\begin{mathpar}
  M^{*}_{x} | \lift{x}{P} \red M[P]
\end{mathpar}

namely,

\begin{mathpar}
  M^{*}_{x} := x?(u).M[\dropn{u}]
\end{mathpar}

The dependence of $M^{*}_{x}$ on a name makes it an abstraction, 

\begin{mathpar}
  M^{*} := (x)x?(u).M[\dropn{u}]
\end{mathpar}

\subsection{Additional notation}

It will sometimes be convenient to denote the process a name
quotes. We already have the notation $x = \quotep{P}$, but it will be
convenient to introduce an alternate notation, $\procn{x}$, when we
want to emphasize the connection to the use of the name. Note that, by
virtue of name equivalence, $\quotep{\procn{x}} \nameeq x$; so, the
notation is consistent with previous definitions.

Further, because names have structure it is possible to effect
substitutions on the basis of that structure. This means we need to
upgrade our notation for substitutions, which we accomplish by
adapting comprehension notation. Thus,

\begin{mathpar}
  P\{ y / x : x \in S \}
\end{mathpar}

is interpreted to mean the process derived from P by replacing (in a
capture-avoiding manner) each occurrence of $x$ in $S$ by $y$. For example,

\begin{mathpar}
  P\{ \quotep{\procn{x}|\procn{x}} / x : x \in \freenames{P} \}
\end{mathpar}

will replace each (occurrence) of a free name $x$ in $P$ by
$\quotep{\procn{x}|\procn{x}}$.

Also, we will avail ourselves of the notation $x^{L}$ and $x^{R}$ to
denote injections of a name into disjoint copies of the name
space. There are numerous ways to accomplish this. One example can be
found in \cite{MeredithR05}. This notation overloads to vectors of
names: $\vec{x}^{\pi} := (x_{i}^{\pi} \; : \; 0 \leq i < |\vec{x}| )$ where $\pi \in \{L,R\}$.

We also use $P^{\Box} := P|\Box$.

In \cite{MeredithR05} an interpretation of the new operator is
given. It turns out that there are several possible interpretations
all enjoying the requisite algebraic properties of the operator (see
\cite{milner91polyadicpi}). We will therefore make liberal use of
$(\nu\; \vec{x})P$.

% subsection the_syntax_and_semantics_of_the_notation_system (end)   

\input{qm2pi.qmops} 

\input{qm2pi.sterngerlach} 

\input{qm2pi.metric} 

% section concurrent_process_calculi (end)

%\input{qm2pi.proofsketch}

% section proof sketch (end)

%\input{qm2pi.slviaknots} 

% section spatial logic via knots (end)

\input{qm2pi.conclusion}

% section conclusion (end)

%\input{qm2pi.dtcodes} 

% section wiring algorithm (end)

\input{qm2pi.ack} 

% section acknowledgments (end)

\newpage


\bibliographystyle{plain}   
\bibliography{../../biblios/main.bib}

\input{qm2pi.rhodetails}

\end{document}

 

% subsection basic_interpretation (end)

%\input{qm2pi.rho.presentation} 
\subsection{The syntax and semantics of the notation system}\label{sub:the_syntax_and_semantics_of_the_notation_system} % (fold)

We now summarize a technical presentation of the calculus that
embodies our theory of dynamics. The typical presentation of such a
calculus follows the style of giving generators and relations on
them. The grammar, below, describing term constructors, freely
generates the set of processes, $\Proc$. This set is then quotiented
by a relation known as structural congruence and it is over this set
that the notion of dynamics is expressed. This presentation is
essentially that of \cite{MeredithR05} with the addition of
polyadicity and summation. For readability we have relegated some of
the technical subtleties to an appendix.

\subsubsection{Process grammar}\label{subsub:process_grammar}

\begin{mathpar}
  \inferrule* [lab=synchronization] {} {{M} \bc \pzero \;|\; x?F \;|\; x!C }
  \and
  \inferrule* [lab=abstraction] {} {{F} \bc (x)P}
  \and
  \inferrule* [lab=concretion] {} {{C} \bc \langle Q \rangle}
  \and
  \inferrule* [lab=process] {} {{P,Q} \bc M \;| \;P|Q \;|\; @{x}}
  \and
  \inferrule* [lab=name] {} {{x} \bc \quotep{P}}
\end{mathpar} 

Note that $\vec{x}$ (resp. $\vec{P}$) denotes a vector of names
(resp. processes) of length $|\vec{x}|$ (resp. $|\vec{P}|$). We adopt
the following useful abbreviations.

\begin{mathpar}
   x?(\vec{y}).P := x.(\vec{y})P \and  x\clift{\vec{P}} := x.\clift{\vec{P}}
   \and x!(y) := \lift{x}{\dropn{y}}
   \and \Pi_{i=0}^{n-1}P_i := P_0 | \ldots | P_{n-1}
\end{mathpar}

\subsubsection{Structural congruence}

\paragraph{Free and bound names and alpha-equivalence.} At the
core of structural equivalence is alpha-equivalence which identifies
process that are the same up to a change of variable. Formally, we
recognize the distinction between free and bound names. The free names
of a process, $\freenames{P}$, may be calculated recursively as
follows:

\begin{mathpar}
\freenames{\pzero} := \emptyset
  \and \\
  \freenames{x?(y).P} := \{ x \} \cup (\freenames{P} \setminus \{ y \})
  \and 
  \freenames{x!\langle P \rangle} := \{ x \} \cup \{ P \} 
  \and \\
  \freenames{P|Q} := \freenames{P} \cup \freenames{Q}
  \and \\
  \freenames{@{x}} := \{ x \}
\end{mathpar}

$\pi$
$\quotep{\pi}$

$\freenames{-} : \pi \to \mathcal{P}(\quotep{\pi})$

\begin{eqnarray*}
  \freenames{\pzero} & := & \emptyset \\
  \freenames{x?(y).P} & := & \{ x \} \cup (\freenames{P} \setminus \{ y \}) \\
  \freenames{x!\langle P \rangle} & := & \{ x \} \cup \{ P \} \\
  \freenames{P|Q} & := & \freenames{P} \cup \freenames{Q} \\
  \freenames{\dropn{x}} & := & \{ x \}
\end{eqnarray*}

The bound names of a process, $\boundnames{P}$, are those names occurring in $P$
that are not free. For example, in $x?(y).0$, the name $x$ is free, while $y$ is bound.

\begin{mathpar}
  \inferrule* [lab=monoidal-laws] {} { P|Q \equiv Q|P \and P|0 \equiv P \and P|(Q|R) \equiv (P|Q)|R }
\end{mathpar}

\begin{mathpar}
  \inferrule* [lab=alpha-equivalence] {} { (x)P \equiv (y)P\{y/x\} \and y \not\in \freenames{P} }
\end{mathpar}

\begin{definition}
Then two processes, $P,Q$, are alpha-equivalent if $P = Q\{\vec{y}/\vec{x}\}$ for
some $\vec{x} \in \boundnames{Q},\vec{y} \in \boundnames{P}$, where $Q\{\vec{y}/\vec{x}\}$
denotes the capture-avoiding substitution of $\vec{y}$ for $\vec{x}$ in $Q$.
\end{definition}

\begin{definition}
  The {\em structural congruence} \cite{SangiorgiWalker} , $\equiv$,
  between processes is the least congruence containing
  alpha-equivalence, satisfying the abelian monoid laws
  (associativity, commutativity and $\pzero$ as identity) for parallel
  composition $|$ and for summation $+$.
\end{definition}

\subsection{Name equivalence}

We take name equivalence, written $\nameeq$, to be the smallest
equivalence relation generated by the following rules.

\begin{mathpar}
\inferrule*[lab=Quote-drop]
{ }
{ \quotep{@{x}} \nameeq x }

\inferrule*[lab=Struct-equiv]
{ P \scong Q }
{ \quotep{P} \nameeq \quotep{Q} }
\end{mathpar}

The astute reader will have noticed that the mutual recursion of names
and processes imposes a mutual recursion on alpha-equivalence and
structural equivalence via name-equivalence. Fortunately, all of this
works out pleasantly and we may calculate in the natural way, free of
concern. The reader interested in the details is referred to the
appendix \ref{appendix:rho_details}.

\subsection{Substitution}

We use $\Proc$ for the set of processes, $\QProc$ for the set of
names, and $\id{\{}\vec{y} / \vec{x} \id{\}}$ to denote partial maps,
$s : \QProc \rightarrow \QProc$. A map, $s$ lifts, uniquely, to a map
on process terms, $\widehat{s} : \Proc \rightarrow \Proc$ by the
following equations.

\begin{mathpar}
  (0) \psubstp{Q}{P} := 0 \\
  (R \juxtap S) \psubstp{Q}{P}
  :=    
  (R)\psubstp{Q}{P} \juxtap (S) \psubstp{Q}{P} \\
  (x?(y).R) \psubstp{Q}{P}    
  :=    
  (x)\substp{Q}{P} (z)\concat( (R \psubstn{z}{y}) \psubstp{Q}{P} ) \\
  (\lift{x}{R}) \psubstp{Q}{P}  
  :=
  \lift{(x)\substp{Q}{P}}{ R \psubstp{Q}{P} } \\
%   (\dropn{x})  \psubstp{Q}{P}       
%   := 
%   \left\{ 
%     \begin{array}{ccc} 
%       \dropn{\quotep{Q}} & & x \nameeq \quotep{P} \\
%       \dropn{x} & & otherwise \\
%     \end{array}
%   \right. 
  (\dropn{x})  \psubstp{Q}{P}       
  := 
  \left\{ 
    \begin{array}{ccc} 
      Q & & x \nameeq \quotep{P} \\
      \dropn{x} & & otherwise \\
    \end{array}
  \right.
\end{mathpar}
 

where

\begin{eqnarray}
  (x)\id{\{} \lpquote Q \rpquote / \lpquote P \rpquote \id{\}}            = 
  \left\{ 
    \begin{array}{ccc}
      \lpquote Q \rpquote & & x \nameeq \lpquote P \rpquote \\
      x & & otherwise \\
    \end{array}
  \right. \nonumber
\end{eqnarray}

and $z$ is chosen distinct from $\quotep{P}$, $\quotep{Q}$, the free
names in $Q$, and all the names in $R$. Our $\alpha$-equivalence will
be built in the standard way from this substitution.

\begin{remark}\label{rem:no_self_referential_names}
  One consequence of these definitions is that $\forall P. \quotep{P}
  \not\in \freenames{P}$.
\end{remark}

\subsection{ Dynamic quote: an example }

Anticipating something of what's to come, consider applying the
substitution, $\widehat{\id{\{}u / z \id{\}}}$, to the following pair
of processes, $\lift{w}{y!(z)}$ and $w[ \lpquote y!(z) \rpquote ]$.

\begin{eqnarray}
	\lift{w}{y!(z)}\widehat{\id{\{}u / z \id{\}}}
		& = &
		\lift{w}{y!(u)} \nonumber\\
	w[ \lpquote y!(z) \rpquote ] \widehat{ \id{\{}u / z \id{\}} }
		& = &
		w[ \lpquote y!(z) \rpquote ] \nonumber
\end{eqnarray}

Because the body of the process between quotes is impervious to
substitution, we get radically different answers. In fact, by
examining the first process in an input context,
e.g. $x?(z).\lift{w}{y!(z)}$, we see that the process under the lift
operator may be shaped by prefixed inputs binding a name inside it. In
this sense, the lift operator will be seen as a way to dynamically
construct processes before reifying them as names.

Finally equipped with these standard features we can present the
dynamics of the calculus.

\subsubsection{Operational semantics} 

Finally, we introduce the computational dynamics. What marks these
algebras as distinct from other more traditionally studied algebraic
structures, e.g. vector spaces or polynomial rings, is the manner in
which dynamics is captured. In traditional structures, dynamics is typically
expressed through morphisms between such structures, as in linear maps
between vector spaces or morphisms between rings. In algebras
associated with the semantics of computation, the dynamics is
expressed as part of the algebraic structure itself, through a
reduction reduction relation typically denoted by $\red$. Below, we
give a recursive presentation of this relation for the calculus used
in the encoding.

$\red \subseteq \pi \times \pi$
$\red : \pi \to \mathcal{P}(\pi)$

\begin{mathpar}
  \inferrule* [lab=Comm] { \textsf{match}( x_{src}, x_{trgt} ) } { x_{trgt}?(y)P \; | \; x_{src}!\langle {Q} \rangle \red P\{\quotep{Q}/y}\} }
  \and \\
  \inferrule* [lab=Par] {{P} \red {P}'} {{{P} | {Q}} \red {{P}' | {Q}}}
  \and
  \inferrule* [lab=Equiv]{{{P} \scong {P}'} \andalso {{P}' \red {Q}'} \andalso {{Q}' \scong {Q}}}{{P} \red {Q}}
\end{mathpar}

\begin{eqnarray*}
  match_{\equiv} (\quotep{P},\quotep{Q}) & := & P \equiv Q \\
  match_{\dagger}(\quotep{P},\quotep{Q}) & := & \forall R. P|Q \red^{*} R => R \red^{*} 0 \\
  match_{K}(\quotep{P},\quotep{Q}) & := & K \mbox{ for some context } K
\end{eqnarray*}

$u?(x)P | u!\langle Q \rangle \red P\{\quotep{Q}/x\}$

%We write $\wred$ for $\red^*$, and $P\red$ if $\exists Q $ such that $ P \red Q$.
We write $P\red$ if $\exists Q $ such that $ P \red Q$ and $P\not\red$, otherwise.

\section{Replication}

As mentioned before, it is known that replication (and hence
recursion) can be implemented in a higher-order process algebra
\cite{SangiorgiWalker}. As our first example of calculation with the
machinery thus far presented we give the construction explicitly in
the {\rhoc}.

\begin{eqnarray}
	D_{x} & := & \prefix{x}{y}{(\binpar{\outputp{x}{y}}{@{y}})} \nonumber\\
	\bangp_{x}{P} & := & \binpar{{x}!\langle{\binpar{D_{x}}{P}}\rangle}{D_{x}} \nonumber
\end{eqnarray}

\begin{eqnarray}
	\bangp_{x}{P} & & \nonumber\\
	=
	& {x}!\langle{(\prefix{x}{y}{(\outputp{x}{y} | @{y})) | P}}\rangle 
	      | \prefix{x}{y}{(\outputp{x}{y} | @{y})} & \nonumber\\
	\red
	& (\outputp{x}{y} | @{y})\substn{\quotep{(\prefix{x}{y}{(@{y} | \outputp{x}{y})) | P}}}{y} & \nonumber\\
	=
	& \outputp{x}{\quotep{(\prefix{x}{y}{(\outputp{x}{y} | @{y})) | P}}}
	  | {(\prefix{x}{y}{(\outputp{x}{y} | @{y})) | P}} & \nonumber\\
	\red
	& \ldots & \nonumber\\
	\red^*
	& P | P | \ldots & \nonumber
\end{eqnarray}

Of course, this encoding, as an implementation, runs away, unfolding
$\bangp{P}$ eagerly. A lazier and more implementable replication
operator, restricted to input-guarded processes, may be obtained as follows.

\begin{eqnarray}
\bangp{\prefix{u}{v}{P}} 
	:= 
	\binpar{\lift{x}{\prefix{u}{v}{(\binpar{D(x)}{P})}}}{D(x)} \nonumber
\end{eqnarray}

\begin{remark}
  Note that the lazier definition still does not deal with summation
  or mixed summation (i.e. sums over input and output). The reader is
  invited to construct definitions of replication that deal with these
  features. 

  Further, the definitions are parameterized in a name, $x$. Can you,
  gentle reader, make a definition that eliminates this parameter and
  guarantees no accidental interaction between the replication
  machinery and the process being replicated -- i.e. no accidental
  sharing of names used by the process to get its work done and the
  name(s) used by the replication to effect copying. This latter
  revision of the definition of replication is crucial to obtaining
  the expected identity $!!P \sim !P$.
\end{remark}

\begin{remark}\label{rem:paradoxical_combinator}
  The reader familiar with the lambda calculus will have noticed the
  similarity between $D$ and the paradoxical combinator.

  [Ed. note: the existence of this seems to suggest we have to be more
  restrictive on the set of processes and names we admit if we are to
  support no-cloning.]
\end{remark}

\subsubsection{Bisimulation}

The computational dynamics gives rise to another kind of equivalence,
the equivalence of computational behavior. As previously mentioned
this is typically captured \emph{via} some form of bisimulation.

% The notion we use in this paper is weak barbed bisimulation
% \cite{milner91polyadicpi}.

The notion we use in this paper is derived from weak barbed
bisimulation \cite{milner91polyadicpi}. 

\begin{definition}
An \emph{observation relation}, $\downarrow_{\mathcal N}$, over a set
of names, $\mathcal N$, is the smallest relation satisfying the rules
below.

\infrule[Out-barb]{y \in {\mathcal N}, \; x \nameeq y}
		  {\outputp{x}{v} \downarrow_{\mathcal N} x}
\infrule[Par-barb]{\mbox{$P\downarrow_{\mathcal N} x$ or $Q\downarrow_{\mathcal N} x$}}
		  {\binpar{P}{Q} \downarrow_{\mathcal N} x}

We write $P \Downarrow_{\mathcal N} x$ if there is $Q$ such that 
$P \wred Q$ and $Q \downarrow_{\mathcal N} x$.
\end{definition}

\begin{definition}
%\label{def.bbisim}
An  ${\mathcal N}$-\emph{barbed bisimulation} over a set of names, ${\mathcal N}$, is a symmetric binary relation 
${\mathcal S}_{\mathcal N}$ between agents such that $P\rel{S}_{\mathcal N}Q$ implies:
\begin{enumerate}
\item If $P \red P'$ then $Q \wred Q'$ and $P'\rel{S}_{\mathcal N} Q'$.
\item If $P\downarrow_{\mathcal N} x$, then $Q\Downarrow_{\mathcal N} x$.
\end{enumerate}
$P$ is ${\mathcal N}$-barbed bisimilar to $Q$, written
$P \wbbisim_{\mathcal N} Q$, if $P \rel{S}_{\mathcal N} Q$ for some ${\mathcal N}$-barbed bisimulation ${\mathcal S}_{\mathcal N}$.
\end{definition}

$\mathcal{R} \subseteq \pi \times \pi$

$P \mathcal{R} Q => \forall P'. P \red P' \Rightarrow \exists Q'. Q \red Q', P' \mathcal{R} Q'$

$P \vdash x \Rightarrow Q \vdash x$

\begin{mathpar}
  \inferrule*[lab=Out-barb]{x \nameeq y}{{y}!\langle{Q}\rangle \vdash x}
  \and
  \inferrule*[lab=Par-barb]{\mbox{$P\vdash x$ or $Q\vdash x$}}{\binpar{P}{Q} \vdash x}
\end{mathpar}

\subsubsection{Contexts}

One of the principle advantages of computational calculi like the
$\pi$-calculus is a well-defined notion of context,
contextual-equivalence and a correlation between
contextual-equivalence and notions of bisimulation. The notion of
context allows the decomposition of a process into (sub-)process and
its syntactic environment, its context. Thus, a context may be
thought of as a process with a ``hole'' (written $\Box$) in it. The
application of a context $M$ to a process $P$, written $M[P]$, is
tantamount to filling the hole in $M$ with $P$. In this paper we do
not need the full weight of this theory, but do make use of the notion
of context in the proof the main theorem. 

\begin{mathpar}
  \inferrule* [lab=summation] {} {{M_{M},M_{N}} \bc \Box \;|\; x.M_{A} \;|\; M_{M}+M_{N}}
  \and
  \inferrule* [lab=agent] {} {{M_{A}} \bc (\vec{x})M_{P} \;| \; \clift{P_0,\ldots,M_{P},\ldots,P_N}}
  \and \\
  \inferrule* [lab=process] {} {{M_{P}} \bc M_{N} \;| \;P|M_{P} }
\end{mathpar} 

\begin{mathpar}
  \inferrule* [lab=sychronization] {} {M_{N} \bc \Box \;|\; x?M_{F} \;|\; x!M_{C}}
  \and
  \inferrule* [lab=abstraction] {} {{M_{F}} \bc (x)M_{P} }
  \and
  \inferrule* [lab=concretion] {} {{M_{C}} \bc \langle M_{P} \rangle }
  \and \\
  \inferrule* [lab=process] {} {{M_{P}} \bc M_{N} \;| \;P|M_{P} }
\end{mathpar}

\begin{definition}[contextual application] Given a context $M$, and
  process $P$, we define the \emph{contextual application}, $M[P] :=
  M\{P/\Box\}$. That is, the contextual application of M to P is the
  substitution of $P$ for $\Box$ in $M$.
\end{definition}

$\meaningof{-} : L \to \mathcal{P}(\pi)$

\begin{mathpar}
  \inferrule* [lab=collection] {} {\meaningof{true} = \pi, \and \meaningof{~E} = \pi \setminus \meaningof{E}, \and \meaningof{E_{1} \& E_{2}} = \meaningof{E_{1}} \cap \meaningof{E_{2}}}
\end{mathpar}

\begin{mathpar}
  \inferrule* [lab=structure] {} {\meaningof{0} = \{ P \in \pi | P \equiv 0 \}, \and \\ \meaningof{E_1 | E_2} = \{ P \in \pi | P \equiv P_{1} | P_{2}, P_{1} \in \meaningof{E_{1}}, P_{2} \in \meaningof{E_2}\} }
\end{mathpar}

\begin{mathpar}
 \inferrule* [lab=behavior] {} {\meaningof{\langle a?b \rangle E} = \{ P \in \pi | P \equiv Q | u?(y)P', \\ \and \\\\ \and \\ \;\;\; u \in \meaningof{a}, \forall z.P'\{z/y\} \in \meaningof{E\{z/b\}}\}, \and \\ \meaningof{a!E} = \{ P \in \pi | P \equiv Q | x!\langle P' \rangle, x \in \meaningof{a} P' \in \meaningof{E}\} }
\end{mathpar}

\begin{mathpar}
 \inferrule* [lab=nominal] {} {\meaningof{\quotep{E}} = \{ \quotep{P} \in \quotep{\pi} | P \in \meaningof{E} \}, \and \meaningof{\quotep{P}} = \{ \quotep{Q} \in \quotep{\pi} | P \equiv Q \} \and \\ \meaningof{@\quotep{E}} = \{ P \in \pi | P \equiv @x, x \in \meaningof{E} \}}
\end{mathpar}

\begin{eqnarray*}
  \\
  \meaningof{-} : TS \to ST
\end{eqnarray*}

\begin{eqnarray*}
  \\
  L : TS \to ST
\end{eqnarray*}

\begin{eqnarray*}
  \\
  P \models E \iff P \in \meaningof{E}
\end{eqnarray*}

\begin{eqnarray*}
  P \approx_{L} Q \iff \forall E \in L. P \models E \iff Q \models E
\end{eqnarray*}

\begin{eqnarray*}
  P \approx_{K} Q
\end{eqnarray*}

\begin{eqnarray*}
  P \approx Q
\end{eqnarray*}

$\approx_{K} = \approx = \approx_{L}$

\subsubsection{Contextual duality}

Note that contexts extend the quotation operation to a family of
operations from processes to names. Given a context, $M$, we can
define a \emph{nominal context}, $\quotep{M}$ by $\quotep{M}[P] :=
\quotep{M[P]}$. To foreshadow what is to come we observe that these
operations enjoy a duality with processes very much like the duality
between vectors and maps from vectors to scalars.

Further, because the calculus is essentially higher-order, we have a
correspondence between contexts and processes. More specifically,
given a name $x$ and a context $M$ we can construct $M^{*}_{x}$ such
that 

\begin{mathpar}
  M^{*}_{x} | \lift{x}{P} \red M[P]
\end{mathpar}

namely,

\begin{mathpar}
  M^{*}_{x} := x?(u).M[\dropn{u}]
\end{mathpar}

The dependence of $M^{*}_{x}$ on a name makes it an abstraction, 

\begin{mathpar}
  M^{*} := (x)x?(u).M[\dropn{u}]
\end{mathpar}

\subsection{Additional notation}

It will sometimes be convenient to denote the process a name
quotes. We already have the notation $x = \quotep{P}$, but it will be
convenient to introduce an alternate notation, $\procn{x}$, when we
want to emphasize the connection to the use of the name. Note that, by
virtue of name equivalence, $\quotep{\procn{x}} \nameeq x$; so, the
notation is consistent with previous definitions.

Further, because names have structure it is possible to effect
substitutions on the basis of that structure. This means we need to
upgrade our notation for substitutions, which we accomplish by
adapting comprehension notation. Thus,

\begin{mathpar}
  P\{ y / x : x \in S \}
\end{mathpar}

is interpreted to mean the process derived from P by replacing (in a
capture-avoiding manner) each occurrence of $x$ in $S$ by $y$. For example,

\begin{mathpar}
  P\{ \quotep{\procn{x}|\procn{x}} / x : x \in \freenames{P} \}
\end{mathpar}

will replace each (occurrence) of a free name $x$ in $P$ by
$\quotep{\procn{x}|\procn{x}}$.

Also, we will avail ourselves of the notation $x^{L}$ and $x^{R}$ to
denote injections of a name into disjoint copies of the name
space. There are numerous ways to accomplish this. One example can be
found in \cite{MeredithR05}. This notation overloads to vectors of
names: $\vec{x}^{\pi} := (x_{i}^{\pi} \; : \; 0 \leq i < |\vec{x}| )$ where $\pi \in \{L,R\}$.

We also use $P^{\Box} := P|\Box$.

In \cite{MeredithR05} an interpretation of the new operator is
given. It turns out that there are several possible interpretations
all enjoying the requisite algebraic properties of the operator (see
\cite{milner91polyadicpi}). We will therefore make liberal use of
$(\nu\; \vec{x})P$.

% subsection the_syntax_and_semantics_of_the_notation_system (end)   

\section{Interpretation of QM}
\subsection{Supporting definitions}
\subsubsection{Multiplication}
\begin{mathpar}
  \quotep{Q} \cdot \quotep{R} := \quotep{Q|R}
  \and \\
  \quotep{Q} \cdot P := P\{ \quotep{Q|R} / \quotep{R} : \quotep{R} \in \freenames{P} \}
\end{mathpar}

\paragraph{Discussion}
The first line needs little explanation. The second line says that
each free name of the process is replaced with the multiplication of
that name by the scalar. Multiplication of a scalar (name) by a state
(process) results in a process all the names of which have been `moved
over' by parallel composition with the process the scalar
quotes. There is a subtlety that the bound names have to be
manipulated so that multiplied names aren't accidentally
captured. There are many ways to achieve this.

\begin{remark}\label{rem:multiplication_identities}
  The reader is invited to verify that for all $x,y,z \in \QProc$ and $P \in \Proc$
  \begin{mathpar}
    x \cdot \quotep{0} \equiv x 
    \and
    x \cdot y \equiv y \cdot x
    \and
    x \cdot (y \cdot z) \equiv (x \cdot y) \cdot z
    \and \\
    \quotep{0} \cdot P \equiv P
    \and \\
    x \cdot (y \cdot P) \equiv (x \cdot y) \cdot P
    \and \\
    x \cdot (P|Q) \equiv (x \cdot P) | (x \cdot Q)
    \and \\    
  \end{mathpar}
\end{remark}

\subsubsection{Tensor product}

We define a tensor product on processes by structural induction.

\paragraph{Tensor of sums} First note that all summations, including
$\pzero$ and sequence, can be written $\Sigma_{i} x_{i}.A_{i} +
\Sigma_{j} x_{j}.C_{j}$, where we have grouped input-guarded processes
together and output-guarded processes together.

Thus, we can define the tensor product of two summations, $N_{1}\otimes N_{2}$, where

\begin{mathpar}
  N_{1} := \Sigma_{i} x_{i}.A_{i} + \Sigma_{j} x_{j}.C_{j}
  \and
  N_{2} := \Sigma_{i'} y_{i'}.B_{i'} + \Sigma_{j'} y_{j'}.D_{j'} 
\end{mathpar}

as follows.

\begin{mathpar}
  \Sigma_{i} x_{i}.A_{i} + \Sigma_{j} x_{j}.C_{j} \otimes \Sigma_{i'}
  y_{i'}.B_{i'} + \Sigma_{j'} y_{j'}.D_{j'} 
  \and \\
  := \; \Sigma_{i} \Sigma_{i'} \quotep{\stackrel{\vee}{x_{i}}| \stackrel{\vee}{y_{i'}}}.(A_{i}\otimes B_{i'}) \; | \; \Sigma_{i'} \Sigma_{i} \quotep{\stackrel{\vee}{y_{i'}}|\stackrel{\vee}{x_{i}}}.(B_{i'}\otimes A_{i})
  \and
  \;\; | \;\; \Sigma_{j} \Sigma_{j'} \quotep{\stackrel{\vee}{x_{j}}|\stackrel{\vee}{y_{j'}}}.(A_{j}\otimes B_{j'}) \; | \; \Sigma_{j'} \Sigma_{j} \quotep{\stackrel{\vee}{y_{j'}}|\stackrel{\vee}{x_{j}}}.(B_{j'}\otimes A_{j})
\end{mathpar}

\begin{remark}
  Do we need to $x^{L}$ and $y^{R}$ for this construction as well?
\end{remark}

\paragraph{Tensor of parallel compositions} Next, we distribute tensor
over par.

\begin{mathpar}
  P_{1}|P_{2} \otimes Q_{1}|Q_{2} := (P_{1} \otimes Q_{1}) | (P_{1}
  \otimes Q_{2}) | (P_{2} \otimes Q_{1}) | (P_{2} \otimes Q_{2})
\end{mathpar}

\paragraph{Tensor with dropped names} We treat tensor of a
process with a dropped name as parallel composition.

\begin{mathpar}
  P \otimes \dropn{x} := P | \dropn{x}
\end{mathpar}

\paragraph{Tensor of agents}

Finally, we need to define tensor on agents. Note that the definition
of tensor on normal products only tensors inputs with inputs and
outputs with outputs. Thus, we only have to define the operation on
``homogeneous'' pairings.

\begin{mathpar}
  (\vec{x})P \otimes (\vec{y})Q
  \and \\
  := (x_{0}^{L}|y_{0}^{R},\ldots,x_{0}^{L}|y_{n}^{R},\ldots,x_{m}^{L}|y_{0}^{R},\ldots,x_{m}^{L}|y_{n}^R)(P\{ \vec{x}^{L}/\vec{x}\} \otimes Q \{ \vec{y}^{R}/\vec{y}\})
  \and \\
  \clift{\vec{P}} \otimes \clift{\vec{Q}}
  \and \\
  := \clift{P_{0}\otimes Q_{0},\ldots,P_{0}\otimes Q_{n},\ldots,P_{m}\otimes Q_{0},\ldots,P_{m}\otimes Q_{n}}
\end{mathpar}

\begin{remark}
  Observe that arities of tensored abstractions matches arities of
  tensored concretions if the original arities matched. Note also that
  the length of the arities corresponds to the increase in dimension
  we see in ordinary vector space tensor product.
\end{remark}

\begin{remark}
  Operationally, this definition distributes the tensor down to
  components ``linked'' by summation. Tensor over summation is
  intriguing in that it mixes names. Moreover, as a consequence of the
  way it mixes names we have the identities for all $x \in \QProc$ and
  $P,Q \in \Proc$

  \begin{mathpar}
    (x \cdot P) \otimes Q \equiv x \cdot (P \otimes Q) \equiv P \otimes (x \cdot Q)
    \and
    P \otimes \pzero \equiv P
  \end{mathpar}

  that the reader is invited to verify.
\end{remark}

\subsubsection{Annihilation}
\begin{mathpar}
  P^{\perp} := \{ Q | \forall R. P|Q \red^{*} R \Rightarrow R \red^{*} \pzero \}
  \and \\
  P^{\underline{\perp}} := \Sigma_{Q \in P^{\perp}} \quotep{Q}?(y).(\dropn{y}|Q) | \Sigma_{Q \in P^{\perp}} \quotep{Q}\clift{\Box}
\end{mathpar}

\paragraph{Discussion} The reader will note that $P^{\perp}$ is a
\emph{set} of processes, while $P^{\underline{\perp}}$ is a
\emph{context}. We call the set $P^{\perp}$ the \emph{annihilators} of
$P$. The parallel composition of a process in the annihilators of $P$
with $P$ will result in a process, the state space of which has all
paths eventually leading to $\pzero$. Execution may endure loops; but
under reasonable conditions of fairness (naturally guaranteed under
most notions of bisimulation) such a composite process cannot get
stuck in such a loop and will, eventually pop out and terminate.

The context $P^{\underline{\perp}}$ is ready and willing to ``take the
$P$ out of'' the process to which it is applied. It will effectively
transmit the code of the process to which it is applied to one of the
annihilators and run the process against it.

\subsubsection{Evaluation}
We fix $M$ a domain of fully abstract interpretation with an equality
coincident with bisimulation. We take $\meaningof{\cdot} : \Proc \to
M$ to be the map interpreting processes and $\nmeaningof{\cdot} : \M
\to Proc$ to be the map running the other way. Then we define

\begin{mathpar}
  \int P := \nmeaningof{\meaningof{P}}
\end{mathpar}

\paragraph{Discussion}
There are many fully abstract interpretations of Milner's
$\pi$-calculus. Any of them can be used as a basis for interpreting
the reflective calculus here. Equipped with such a domain it is
largely a matter of grinding through to check that the Yoneda
construction for the normalization-by-evaluation program can be
extended to this setting.

\begin{remark}
  The reader is invited to verify that $\int (P^{\underline{\perp}}[P]) = 0$.
\end{remark}

\subsection{Quantum mechanics}

Table \ref{tbl:core_qm_op_defns} gives the core operational definitions

\begin{table}[htp]\label{tbl:core_qm_op_defns}
  \center{
    \fbox{
      \begin{tabular}{c|c}
        quantum mechanics & process calculus \\
        \hline
        scalar & $x := \quotep{P}$ \\
        state vector & $\state{P} := P$ \\
        dual & $\state{P}^{*} := \event{P^{\underline{\perp}}} := \quotep{P^{\underline{\perp}}}[-]$ \\
        matrix & $ \Sigma_{\alpha} \state{P_{\alpha}}x_{\alpha}\event{Q_{\alpha}}$ \\
        vector addition & $\state{P} + \state{Q} := \state{P | Q}$ \\
        tensor product & $\state{P} \otimes \state{Q} := \state{P \otimes Q}$ \\
        inner product & $\innerprod{P}{Q} := \quotep{\int P^{\underline{\perp}}[Q]}$ \\
      \end{tabular}
    }
  }
  \caption{QM - operational definitions}
\end{table}

where

\begin{mathpar}
  \prmatrix{P}{Q} := \fprmatrix{P}{\quotep{\pzero}}{Q}
  \and
  \fprmatrix{P}{x}{Q} := (\state{P},x,\event{Q})
  \and
  (\fprmatrix{P}{x}{Q})(\state{R}) := x \cdot \innerprod{Q}{R} \cdot \state{P}
  \and
  (\fprmatrix{P}{x}{Q})(\event{R}) := x \cdot \innerprod{R}{P} \cdot \event{Q}
\end{mathpar}

\paragraph{Discussion}
As promised: vectors (aka states) are represented as processes; duals
as contextual duals; inner product definition should be compared with
standard inner product definition for ....

\begin{remark}
  Assuming $\int (P^{\underline{\perp}}[P]) = 0$, the reader is
  invited to verify that $(\fprmatrix{P}{x}{P})(\state{P}) = x \cdot \state{P}$.
\end{remark}

\begin{remark}
  The reader is invited to verify that $\innerprod{P}{Q}$ could
  equally well have been written $\quotep{\int \stackrel{\vee}{x}}$
  where $x = \event{P^{\underline{\perp}}}(Q)$.

  One of the motivations for this remark is that there is another way
  to factor these operations. We could package up evaluation in the dual:

  \begin{mathpar}
    \state{P}^{*} := \event{\int P^{\underline{\perp}}} := \quotep{\int P^{\underline{\perp}}}[-]
  \end{mathpar}

  and then have inner product defined by
  
  \begin{mathpar}
    \innerprod{P}{Q} := \event{P}(Q)
  \end{mathpar}

  Hopefully, experience with the calculations will provide guidance on
  the best factoring.
\end{remark}

\begin{remark}
  Assuming $\int (P^{\underline{\perp}}[P]) = 0$, the reader is
  invited to verify that $\forall P,Q. (\prmatrix{0}{Q})(\state{0}) =
  \state{0}$ and dually $(\prmatrix{P}{0})(\event{0}) = \event{0}$.
\end{remark}

\begin{remark}
  i'm a little worried that i don't (yet) have proper support for
  complex conjugacy. But, the observation above may give us a
  clue. According to Abramsky, it must be the case that the scalars
  are iso to the homset of the identity for the tensor -- which the
  observation above characterizes. 

  For now, we will simply bookmark the notion with $\overline{x}$.
\end{remark}

\subsubsection{Adjointness}

We need to give a definition of $(\cdot)^{\dagger}$ for matrices. The
obvious candidate definition is
\begin{mathpar}
(\Sigma_{\alpha}\fprmatrix{P_{\alpha}}{x_{\alpha}}{Q_{\alpha}})^{\dagger}
= \Sigma_{\alpha}\fprmatrix{(Q_{\alpha}^{\underline{\perp}})^{*}}{\overline{x}_{\alpha}}{P_{\alpha}^{\underline{\perp}}} 
\end{mathpar}

But, $(Q_{\alpha}^{\underline{\perp}})^{*}$ requires a name along
which to communicate the process to achieve the context application.

\subsubsection{Basis for a basis}
If processes label states and ``addition'' of states (a.k.a. vector
addition) is interpreted as parallel composition, what corresponds to
notions of linear independence and basis? Here, we recall that Yoshida
has developed a set of \emph{combinators} for an asynchronous verison
of Milner's $\pi$-calculus. These are a finite set of processes such
any process can be expressed as parallel composition of these
combinators together with liberal uses of the new operator and
replication. We can simply give a translation of these into the
present calculus and have reasonable expectation that the property
carries over. That is, that the resultant set allows to express all
processes via parallel composition. Note, however, that there is no
new operator or replication in this calculus. As a result, we expect
that the corresponding set is actually infinite. That is, we expect
that the space is actually infinite dimensional.

\begin{remark}
  The attentive reader may be a bit concerned. Certainly, the
  collection $S$, $K$ and $I$ is a finite set of
  combinators. Shouldn't we expect to see a finite set of combinators
  for an effectively equivalent system? i am very sympathetic to this
  critique and feel it warrants full attention. On the other hand, i
  also have in mind the following analogy. The natural numbers, as a
  monoid under addition, has exactly $1$ generator, while the natural
  numbers, as a monoid under multiplication, has countably many
  generators (the primes). We observe that the application of the
  lambda calculus is much less resource sensitive than the parallel
  composition of the $\pi$-calculus. Could it be the case that we have
  an analogy of the form
  
  \begin{mathpar}
    m + n : MN :: m*n : M|N
  \end{mathpar}

  giving a similar blow up in the set of ``primes''?  This is such a
  wonderful thought that, even if it's not true, i think it's worth
  writing down.
\end{remark}
 

\documentclass[12pt]{llncs}
%\documentclass{jktr}

\usepackage[pdftex]{hyperref}                   
\usepackage {listings}
\usepackage {mathpartir}
\usepackage{bcprules}
%\usepackage{listings}
                       
\usepackage{graphicx} 
%\usepackage[margins=2.5cm,nohead,nofoot]{geometry}
%\usepackage{geometry}
\usepackage{amsfonts}
\usepackage{amstext}
\usepackage{latexsym}
\usepackage{amssymb}
\usepackage{color}


%\include{myPreamble}
\include{qm2pi.local} 

%\ifpdf
%\usepackage[pdftex]{graphicx}
%\else
%\usepackage{graphicx}
%\fi

 % \ifpdf
%  \usepackage{pdfsync}
%  \if


%\title{Brief Article}
%\author{David F. Snyder}
%\author{L.G. Meredith}

%\address{Dept. of Math., Texas State University--San Marcos, San Marcos, TX 78666}
       
\pagestyle{empty}


\begin{document}

\lstset{language=[Objective]Caml,frame=shadowbox}

\input{qm2pi.front}

% section front matter (end)

\input{qm2pi.intro} 
 
% section introduction (end)

% \input{qm2pi.knotations} 

% section notation (end)

\input{qm2pi.process.calculi} 

% section concurrent_process_calculi_and_spatial_logics_ (end)
    
%\input{qm2pi.knots2pi} 

%\input{qm2pi.trefoil} 

%\input{qm2pi.mainthm} 

% subsection basic_interpretation (end)

%\input{qm2pi.rho.presentation} 
\subsection{The syntax and semantics of the notation system}\label{sub:the_syntax_and_semantics_of_the_notation_system} % (fold)

We now summarize a technical presentation of the calculus that
embodies our theory of dynamics. The typical presentation of such a
calculus follows the style of giving generators and relations on
them. The grammar, below, describing term constructors, freely
generates the set of processes, $\Proc$. This set is then quotiented
by a relation known as structural congruence and it is over this set
that the notion of dynamics is expressed. This presentation is
essentially that of \cite{MeredithR05} with the addition of
polyadicity and summation. For readability we have relegated some of
the technical subtleties to an appendix.

\subsubsection{Process grammar}\label{subsub:process_grammar}

\begin{mathpar}
  \inferrule* [lab=synchronization] {} {{M} \bc \pzero \;|\; x?F \;|\; x!C }
  \and
  \inferrule* [lab=abstraction] {} {{F} \bc (x)P}
  \and
  \inferrule* [lab=concretion] {} {{C} \bc \langle Q \rangle}
  \and
  \inferrule* [lab=process] {} {{P,Q} \bc M \;| \;P|Q \;|\; @{x}}
  \and
  \inferrule* [lab=name] {} {{x} \bc \quotep{P}}
\end{mathpar} 

Note that $\vec{x}$ (resp. $\vec{P}$) denotes a vector of names
(resp. processes) of length $|\vec{x}|$ (resp. $|\vec{P}|$). We adopt
the following useful abbreviations.

\begin{mathpar}
   x?(\vec{y}).P := x.(\vec{y})P \and  x\clift{\vec{P}} := x.\clift{\vec{P}}
   \and x!(y) := \lift{x}{\dropn{y}}
   \and \Pi_{i=0}^{n-1}P_i := P_0 | \ldots | P_{n-1}
\end{mathpar}

\subsubsection{Structural congruence}

\paragraph{Free and bound names and alpha-equivalence.} At the
core of structural equivalence is alpha-equivalence which identifies
process that are the same up to a change of variable. Formally, we
recognize the distinction between free and bound names. The free names
of a process, $\freenames{P}$, may be calculated recursively as
follows:

\begin{mathpar}
\freenames{\pzero} := \emptyset
  \and \\
  \freenames{x?(y).P} := \{ x \} \cup (\freenames{P} \setminus \{ y \})
  \and 
  \freenames{x!\langle P \rangle} := \{ x \} \cup \{ P \} 
  \and \\
  \freenames{P|Q} := \freenames{P} \cup \freenames{Q}
  \and \\
  \freenames{@{x}} := \{ x \}
\end{mathpar}

$\pi$
$\quotep{\pi}$

$\freenames{-} : \pi \to \mathcal{P}(\quotep{\pi})$

\begin{eqnarray*}
  \freenames{\pzero} & := & \emptyset \\
  \freenames{x?(y).P} & := & \{ x \} \cup (\freenames{P} \setminus \{ y \}) \\
  \freenames{x!\langle P \rangle} & := & \{ x \} \cup \{ P \} \\
  \freenames{P|Q} & := & \freenames{P} \cup \freenames{Q} \\
  \freenames{\dropn{x}} & := & \{ x \}
\end{eqnarray*}

The bound names of a process, $\boundnames{P}$, are those names occurring in $P$
that are not free. For example, in $x?(y).0$, the name $x$ is free, while $y$ is bound.

\begin{mathpar}
  \inferrule* [lab=monoidal-laws] {} { P|Q \equiv Q|P \and P|0 \equiv P \and P|(Q|R) \equiv (P|Q)|R }
\end{mathpar}

\begin{mathpar}
  \inferrule* [lab=alpha-equivalence] {} { (x)P \equiv (y)P\{y/x\} \and y \not\in \freenames{P} }
\end{mathpar}

\begin{definition}
Then two processes, $P,Q$, are alpha-equivalent if $P = Q\{\vec{y}/\vec{x}\}$ for
some $\vec{x} \in \boundnames{Q},\vec{y} \in \boundnames{P}$, where $Q\{\vec{y}/\vec{x}\}$
denotes the capture-avoiding substitution of $\vec{y}$ for $\vec{x}$ in $Q$.
\end{definition}

\begin{definition}
  The {\em structural congruence} \cite{SangiorgiWalker} , $\equiv$,
  between processes is the least congruence containing
  alpha-equivalence, satisfying the abelian monoid laws
  (associativity, commutativity and $\pzero$ as identity) for parallel
  composition $|$ and for summation $+$.
\end{definition}

\subsection{Name equivalence}

We take name equivalence, written $\nameeq$, to be the smallest
equivalence relation generated by the following rules.

\begin{mathpar}
\inferrule*[lab=Quote-drop]
{ }
{ \quotep{@{x}} \nameeq x }

\inferrule*[lab=Struct-equiv]
{ P \scong Q }
{ \quotep{P} \nameeq \quotep{Q} }
\end{mathpar}

The astute reader will have noticed that the mutual recursion of names
and processes imposes a mutual recursion on alpha-equivalence and
structural equivalence via name-equivalence. Fortunately, all of this
works out pleasantly and we may calculate in the natural way, free of
concern. The reader interested in the details is referred to the
appendix \ref{appendix:rho_details}.

\subsection{Substitution}

We use $\Proc$ for the set of processes, $\QProc$ for the set of
names, and $\id{\{}\vec{y} / \vec{x} \id{\}}$ to denote partial maps,
$s : \QProc \rightarrow \QProc$. A map, $s$ lifts, uniquely, to a map
on process terms, $\widehat{s} : \Proc \rightarrow \Proc$ by the
following equations.

\begin{mathpar}
  (0) \psubstp{Q}{P} := 0 \\
  (R \juxtap S) \psubstp{Q}{P}
  :=    
  (R)\psubstp{Q}{P} \juxtap (S) \psubstp{Q}{P} \\
  (x?(y).R) \psubstp{Q}{P}    
  :=    
  (x)\substp{Q}{P} (z)\concat( (R \psubstn{z}{y}) \psubstp{Q}{P} ) \\
  (\lift{x}{R}) \psubstp{Q}{P}  
  :=
  \lift{(x)\substp{Q}{P}}{ R \psubstp{Q}{P} } \\
%   (\dropn{x})  \psubstp{Q}{P}       
%   := 
%   \left\{ 
%     \begin{array}{ccc} 
%       \dropn{\quotep{Q}} & & x \nameeq \quotep{P} \\
%       \dropn{x} & & otherwise \\
%     \end{array}
%   \right. 
  (\dropn{x})  \psubstp{Q}{P}       
  := 
  \left\{ 
    \begin{array}{ccc} 
      Q & & x \nameeq \quotep{P} \\
      \dropn{x} & & otherwise \\
    \end{array}
  \right.
\end{mathpar}
 

where

\begin{eqnarray}
  (x)\id{\{} \lpquote Q \rpquote / \lpquote P \rpquote \id{\}}            = 
  \left\{ 
    \begin{array}{ccc}
      \lpquote Q \rpquote & & x \nameeq \lpquote P \rpquote \\
      x & & otherwise \\
    \end{array}
  \right. \nonumber
\end{eqnarray}

and $z$ is chosen distinct from $\quotep{P}$, $\quotep{Q}$, the free
names in $Q$, and all the names in $R$. Our $\alpha$-equivalence will
be built in the standard way from this substitution.

\begin{remark}\label{rem:no_self_referential_names}
  One consequence of these definitions is that $\forall P. \quotep{P}
  \not\in \freenames{P}$.
\end{remark}

\subsection{ Dynamic quote: an example }

Anticipating something of what's to come, consider applying the
substitution, $\widehat{\id{\{}u / z \id{\}}}$, to the following pair
of processes, $\lift{w}{y!(z)}$ and $w[ \lpquote y!(z) \rpquote ]$.

\begin{eqnarray}
	\lift{w}{y!(z)}\widehat{\id{\{}u / z \id{\}}}
		& = &
		\lift{w}{y!(u)} \nonumber\\
	w[ \lpquote y!(z) \rpquote ] \widehat{ \id{\{}u / z \id{\}} }
		& = &
		w[ \lpquote y!(z) \rpquote ] \nonumber
\end{eqnarray}

Because the body of the process between quotes is impervious to
substitution, we get radically different answers. In fact, by
examining the first process in an input context,
e.g. $x?(z).\lift{w}{y!(z)}$, we see that the process under the lift
operator may be shaped by prefixed inputs binding a name inside it. In
this sense, the lift operator will be seen as a way to dynamically
construct processes before reifying them as names.

Finally equipped with these standard features we can present the
dynamics of the calculus.

\subsubsection{Operational semantics} 

Finally, we introduce the computational dynamics. What marks these
algebras as distinct from other more traditionally studied algebraic
structures, e.g. vector spaces or polynomial rings, is the manner in
which dynamics is captured. In traditional structures, dynamics is typically
expressed through morphisms between such structures, as in linear maps
between vector spaces or morphisms between rings. In algebras
associated with the semantics of computation, the dynamics is
expressed as part of the algebraic structure itself, through a
reduction reduction relation typically denoted by $\red$. Below, we
give a recursive presentation of this relation for the calculus used
in the encoding.

$\red \subseteq \pi \times \pi$
$\red : \pi \to \mathcal{P}(\pi)$

\begin{mathpar}
  \inferrule* [lab=Comm] { \textsf{match}( x_{src}, x_{trgt} ) } { x_{trgt}?(y)P \; | \; x_{src}!\langle {Q} \rangle \red P\{\quotep{Q}/y}\} }
  \and \\
  \inferrule* [lab=Par] {{P} \red {P}'} {{{P} | {Q}} \red {{P}' | {Q}}}
  \and
  \inferrule* [lab=Equiv]{{{P} \scong {P}'} \andalso {{P}' \red {Q}'} \andalso {{Q}' \scong {Q}}}{{P} \red {Q}}
\end{mathpar}

\begin{eqnarray*}
  match_{\equiv} (\quotep{P},\quotep{Q}) & := & P \equiv Q \\
  match_{\dagger}(\quotep{P},\quotep{Q}) & := & \forall R. P|Q \red^{*} R => R \red^{*} 0 \\
  match_{K}(\quotep{P},\quotep{Q}) & := & K \mbox{ for some context } K
\end{eqnarray*}

$u?(x)P | u!\langle Q \rangle \red P\{\quotep{Q}/x\}$

%We write $\wred$ for $\red^*$, and $P\red$ if $\exists Q $ such that $ P \red Q$.
We write $P\red$ if $\exists Q $ such that $ P \red Q$ and $P\not\red$, otherwise.

\section{Replication}

As mentioned before, it is known that replication (and hence
recursion) can be implemented in a higher-order process algebra
\cite{SangiorgiWalker}. As our first example of calculation with the
machinery thus far presented we give the construction explicitly in
the {\rhoc}.

\begin{eqnarray}
	D_{x} & := & \prefix{x}{y}{(\binpar{\outputp{x}{y}}{@{y}})} \nonumber\\
	\bangp_{x}{P} & := & \binpar{{x}!\langle{\binpar{D_{x}}{P}}\rangle}{D_{x}} \nonumber
\end{eqnarray}

\begin{eqnarray}
	\bangp_{x}{P} & & \nonumber\\
	=
	& {x}!\langle{(\prefix{x}{y}{(\outputp{x}{y} | @{y})) | P}}\rangle 
	      | \prefix{x}{y}{(\outputp{x}{y} | @{y})} & \nonumber\\
	\red
	& (\outputp{x}{y} | @{y})\substn{\quotep{(\prefix{x}{y}{(@{y} | \outputp{x}{y})) | P}}}{y} & \nonumber\\
	=
	& \outputp{x}{\quotep{(\prefix{x}{y}{(\outputp{x}{y} | @{y})) | P}}}
	  | {(\prefix{x}{y}{(\outputp{x}{y} | @{y})) | P}} & \nonumber\\
	\red
	& \ldots & \nonumber\\
	\red^*
	& P | P | \ldots & \nonumber
\end{eqnarray}

Of course, this encoding, as an implementation, runs away, unfolding
$\bangp{P}$ eagerly. A lazier and more implementable replication
operator, restricted to input-guarded processes, may be obtained as follows.

\begin{eqnarray}
\bangp{\prefix{u}{v}{P}} 
	:= 
	\binpar{\lift{x}{\prefix{u}{v}{(\binpar{D(x)}{P})}}}{D(x)} \nonumber
\end{eqnarray}

\begin{remark}
  Note that the lazier definition still does not deal with summation
  or mixed summation (i.e. sums over input and output). The reader is
  invited to construct definitions of replication that deal with these
  features. 

  Further, the definitions are parameterized in a name, $x$. Can you,
  gentle reader, make a definition that eliminates this parameter and
  guarantees no accidental interaction between the replication
  machinery and the process being replicated -- i.e. no accidental
  sharing of names used by the process to get its work done and the
  name(s) used by the replication to effect copying. This latter
  revision of the definition of replication is crucial to obtaining
  the expected identity $!!P \sim !P$.
\end{remark}

\begin{remark}\label{rem:paradoxical_combinator}
  The reader familiar with the lambda calculus will have noticed the
  similarity between $D$ and the paradoxical combinator.

  [Ed. note: the existence of this seems to suggest we have to be more
  restrictive on the set of processes and names we admit if we are to
  support no-cloning.]
\end{remark}

\subsubsection{Bisimulation}

The computational dynamics gives rise to another kind of equivalence,
the equivalence of computational behavior. As previously mentioned
this is typically captured \emph{via} some form of bisimulation.

% The notion we use in this paper is weak barbed bisimulation
% \cite{milner91polyadicpi}.

The notion we use in this paper is derived from weak barbed
bisimulation \cite{milner91polyadicpi}. 

\begin{definition}
An \emph{observation relation}, $\downarrow_{\mathcal N}$, over a set
of names, $\mathcal N$, is the smallest relation satisfying the rules
below.

\infrule[Out-barb]{y \in {\mathcal N}, \; x \nameeq y}
		  {\outputp{x}{v} \downarrow_{\mathcal N} x}
\infrule[Par-barb]{\mbox{$P\downarrow_{\mathcal N} x$ or $Q\downarrow_{\mathcal N} x$}}
		  {\binpar{P}{Q} \downarrow_{\mathcal N} x}

We write $P \Downarrow_{\mathcal N} x$ if there is $Q$ such that 
$P \wred Q$ and $Q \downarrow_{\mathcal N} x$.
\end{definition}

\begin{definition}
%\label{def.bbisim}
An  ${\mathcal N}$-\emph{barbed bisimulation} over a set of names, ${\mathcal N}$, is a symmetric binary relation 
${\mathcal S}_{\mathcal N}$ between agents such that $P\rel{S}_{\mathcal N}Q$ implies:
\begin{enumerate}
\item If $P \red P'$ then $Q \wred Q'$ and $P'\rel{S}_{\mathcal N} Q'$.
\item If $P\downarrow_{\mathcal N} x$, then $Q\Downarrow_{\mathcal N} x$.
\end{enumerate}
$P$ is ${\mathcal N}$-barbed bisimilar to $Q$, written
$P \wbbisim_{\mathcal N} Q$, if $P \rel{S}_{\mathcal N} Q$ for some ${\mathcal N}$-barbed bisimulation ${\mathcal S}_{\mathcal N}$.
\end{definition}

$\mathcal{R} \subseteq \pi \times \pi$

$P \mathcal{R} Q => \forall P'. P \red P' \Rightarrow \exists Q'. Q \red Q', P' \mathcal{R} Q'$

$P \vdash x \Rightarrow Q \vdash x$

\begin{mathpar}
  \inferrule*[lab=Out-barb]{x \nameeq y}{{y}!\langle{Q}\rangle \vdash x}
  \and
  \inferrule*[lab=Par-barb]{\mbox{$P\vdash x$ or $Q\vdash x$}}{\binpar{P}{Q} \vdash x}
\end{mathpar}

\subsubsection{Contexts}

One of the principle advantages of computational calculi like the
$\pi$-calculus is a well-defined notion of context,
contextual-equivalence and a correlation between
contextual-equivalence and notions of bisimulation. The notion of
context allows the decomposition of a process into (sub-)process and
its syntactic environment, its context. Thus, a context may be
thought of as a process with a ``hole'' (written $\Box$) in it. The
application of a context $M$ to a process $P$, written $M[P]$, is
tantamount to filling the hole in $M$ with $P$. In this paper we do
not need the full weight of this theory, but do make use of the notion
of context in the proof the main theorem. 

\begin{mathpar}
  \inferrule* [lab=summation] {} {{M_{M},M_{N}} \bc \Box \;|\; x.M_{A} \;|\; M_{M}+M_{N}}
  \and
  \inferrule* [lab=agent] {} {{M_{A}} \bc (\vec{x})M_{P} \;| \; \clift{P_0,\ldots,M_{P},\ldots,P_N}}
  \and \\
  \inferrule* [lab=process] {} {{M_{P}} \bc M_{N} \;| \;P|M_{P} }
\end{mathpar} 

\begin{mathpar}
  \inferrule* [lab=sychronization] {} {M_{N} \bc \Box \;|\; x?M_{F} \;|\; x!M_{C}}
  \and
  \inferrule* [lab=abstraction] {} {{M_{F}} \bc (x)M_{P} }
  \and
  \inferrule* [lab=concretion] {} {{M_{C}} \bc \langle M_{P} \rangle }
  \and \\
  \inferrule* [lab=process] {} {{M_{P}} \bc M_{N} \;| \;P|M_{P} }
\end{mathpar}

\begin{definition}[contextual application] Given a context $M$, and
  process $P$, we define the \emph{contextual application}, $M[P] :=
  M\{P/\Box\}$. That is, the contextual application of M to P is the
  substitution of $P$ for $\Box$ in $M$.
\end{definition}

$\meaningof{-} : L \to \mathcal{P}(\pi)$

\begin{mathpar}
  \inferrule* [lab=collection] {} {\meaningof{true} = \pi, \and \meaningof{~E} = \pi \setminus \meaningof{E}, \and \meaningof{E_{1} \& E_{2}} = \meaningof{E_{1}} \cap \meaningof{E_{2}}}
\end{mathpar}

\begin{mathpar}
  \inferrule* [lab=structure] {} {\meaningof{0} = \{ P \in \pi | P \equiv 0 \}, \and \\ \meaningof{E_1 | E_2} = \{ P \in \pi | P \equiv P_{1} | P_{2}, P_{1} \in \meaningof{E_{1}}, P_{2} \in \meaningof{E_2}\} }
\end{mathpar}

\begin{mathpar}
 \inferrule* [lab=behavior] {} {\meaningof{\langle a?b \rangle E} = \{ P \in \pi | P \equiv Q | u?(y)P', \\ \and \\\\ \and \\ \;\;\; u \in \meaningof{a}, \forall z.P'\{z/y\} \in \meaningof{E\{z/b\}}\}, \and \\ \meaningof{a!E} = \{ P \in \pi | P \equiv Q | x!\langle P' \rangle, x \in \meaningof{a} P' \in \meaningof{E}\} }
\end{mathpar}

\begin{mathpar}
 \inferrule* [lab=nominal] {} {\meaningof{\quotep{E}} = \{ \quotep{P} \in \quotep{\pi} | P \in \meaningof{E} \}, \and \meaningof{\quotep{P}} = \{ \quotep{Q} \in \quotep{\pi} | P \equiv Q \} \and \\ \meaningof{@\quotep{E}} = \{ P \in \pi | P \equiv @x, x \in \meaningof{E} \}}
\end{mathpar}

\begin{eqnarray*}
  \\
  \meaningof{-} : TS \to ST
\end{eqnarray*}

\begin{eqnarray*}
  \\
  L : TS \to ST
\end{eqnarray*}

\begin{eqnarray*}
  \\
  P \models E \iff P \in \meaningof{E}
\end{eqnarray*}

\begin{eqnarray*}
  P \approx_{L} Q \iff \forall E \in L. P \models E \iff Q \models E
\end{eqnarray*}

\begin{eqnarray*}
  P \approx_{K} Q
\end{eqnarray*}

\begin{eqnarray*}
  P \approx Q
\end{eqnarray*}

$\approx_{K} = \approx = \approx_{L}$

\subsubsection{Contextual duality}

Note that contexts extend the quotation operation to a family of
operations from processes to names. Given a context, $M$, we can
define a \emph{nominal context}, $\quotep{M}$ by $\quotep{M}[P] :=
\quotep{M[P]}$. To foreshadow what is to come we observe that these
operations enjoy a duality with processes very much like the duality
between vectors and maps from vectors to scalars.

Further, because the calculus is essentially higher-order, we have a
correspondence between contexts and processes. More specifically,
given a name $x$ and a context $M$ we can construct $M^{*}_{x}$ such
that 

\begin{mathpar}
  M^{*}_{x} | \lift{x}{P} \red M[P]
\end{mathpar}

namely,

\begin{mathpar}
  M^{*}_{x} := x?(u).M[\dropn{u}]
\end{mathpar}

The dependence of $M^{*}_{x}$ on a name makes it an abstraction, 

\begin{mathpar}
  M^{*} := (x)x?(u).M[\dropn{u}]
\end{mathpar}

\subsection{Additional notation}

It will sometimes be convenient to denote the process a name
quotes. We already have the notation $x = \quotep{P}$, but it will be
convenient to introduce an alternate notation, $\procn{x}$, when we
want to emphasize the connection to the use of the name. Note that, by
virtue of name equivalence, $\quotep{\procn{x}} \nameeq x$; so, the
notation is consistent with previous definitions.

Further, because names have structure it is possible to effect
substitutions on the basis of that structure. This means we need to
upgrade our notation for substitutions, which we accomplish by
adapting comprehension notation. Thus,

\begin{mathpar}
  P\{ y / x : x \in S \}
\end{mathpar}

is interpreted to mean the process derived from P by replacing (in a
capture-avoiding manner) each occurrence of $x$ in $S$ by $y$. For example,

\begin{mathpar}
  P\{ \quotep{\procn{x}|\procn{x}} / x : x \in \freenames{P} \}
\end{mathpar}

will replace each (occurrence) of a free name $x$ in $P$ by
$\quotep{\procn{x}|\procn{x}}$.

Also, we will avail ourselves of the notation $x^{L}$ and $x^{R}$ to
denote injections of a name into disjoint copies of the name
space. There are numerous ways to accomplish this. One example can be
found in \cite{MeredithR05}. This notation overloads to vectors of
names: $\vec{x}^{\pi} := (x_{i}^{\pi} \; : \; 0 \leq i < |\vec{x}| )$ where $\pi \in \{L,R\}$.

We also use $P^{\Box} := P|\Box$.

In \cite{MeredithR05} an interpretation of the new operator is
given. It turns out that there are several possible interpretations
all enjoying the requisite algebraic properties of the operator (see
\cite{milner91polyadicpi}). We will therefore make liberal use of
$(\nu\; \vec{x})P$.

% subsection the_syntax_and_semantics_of_the_notation_system (end)   

\input{qm2pi.qmops} 

\input{qm2pi.sterngerlach} 

\input{qm2pi.metric} 

% section concurrent_process_calculi (end)

%\input{qm2pi.proofsketch}

% section proof sketch (end)

%\input{qm2pi.slviaknots} 

% section spatial logic via knots (end)

\input{qm2pi.conclusion}

% section conclusion (end)

%\input{qm2pi.dtcodes} 

% section wiring algorithm (end)

\input{qm2pi.ack} 

% section acknowledgments (end)

\newpage


\bibliographystyle{plain}   
\bibliography{../../biblios/main.bib}

\input{qm2pi.rhodetails}

\end{document}

 

\documentclass[12pt]{llncs}
%\documentclass{jktr}

\usepackage[pdftex]{hyperref}                   
\usepackage {listings}
\usepackage {mathpartir}
\usepackage{bcprules}
%\usepackage{listings}
                       
\usepackage{graphicx} 
%\usepackage[margins=2.5cm,nohead,nofoot]{geometry}
%\usepackage{geometry}
\usepackage{amsfonts}
\usepackage{amstext}
\usepackage{latexsym}
\usepackage{amssymb}
\usepackage{color}


%\include{myPreamble}
\include{qm2pi.local} 

%\ifpdf
%\usepackage[pdftex]{graphicx}
%\else
%\usepackage{graphicx}
%\fi

 % \ifpdf
%  \usepackage{pdfsync}
%  \if


%\title{Brief Article}
%\author{David F. Snyder}
%\author{L.G. Meredith}

%\address{Dept. of Math., Texas State University--San Marcos, San Marcos, TX 78666}
       
\pagestyle{empty}


\begin{document}

\lstset{language=[Objective]Caml,frame=shadowbox}

\input{qm2pi.front}

% section front matter (end)

\input{qm2pi.intro} 
 
% section introduction (end)

% \input{qm2pi.knotations} 

% section notation (end)

\input{qm2pi.process.calculi} 

% section concurrent_process_calculi_and_spatial_logics_ (end)
    
%\input{qm2pi.knots2pi} 

%\input{qm2pi.trefoil} 

%\input{qm2pi.mainthm} 

% subsection basic_interpretation (end)

%\input{qm2pi.rho.presentation} 
\subsection{The syntax and semantics of the notation system}\label{sub:the_syntax_and_semantics_of_the_notation_system} % (fold)

We now summarize a technical presentation of the calculus that
embodies our theory of dynamics. The typical presentation of such a
calculus follows the style of giving generators and relations on
them. The grammar, below, describing term constructors, freely
generates the set of processes, $\Proc$. This set is then quotiented
by a relation known as structural congruence and it is over this set
that the notion of dynamics is expressed. This presentation is
essentially that of \cite{MeredithR05} with the addition of
polyadicity and summation. For readability we have relegated some of
the technical subtleties to an appendix.

\subsubsection{Process grammar}\label{subsub:process_grammar}

\begin{mathpar}
  \inferrule* [lab=synchronization] {} {{M} \bc \pzero \;|\; x?F \;|\; x!C }
  \and
  \inferrule* [lab=abstraction] {} {{F} \bc (x)P}
  \and
  \inferrule* [lab=concretion] {} {{C} \bc \langle Q \rangle}
  \and
  \inferrule* [lab=process] {} {{P,Q} \bc M \;| \;P|Q \;|\; @{x}}
  \and
  \inferrule* [lab=name] {} {{x} \bc \quotep{P}}
\end{mathpar} 

Note that $\vec{x}$ (resp. $\vec{P}$) denotes a vector of names
(resp. processes) of length $|\vec{x}|$ (resp. $|\vec{P}|$). We adopt
the following useful abbreviations.

\begin{mathpar}
   x?(\vec{y}).P := x.(\vec{y})P \and  x\clift{\vec{P}} := x.\clift{\vec{P}}
   \and x!(y) := \lift{x}{\dropn{y}}
   \and \Pi_{i=0}^{n-1}P_i := P_0 | \ldots | P_{n-1}
\end{mathpar}

\subsubsection{Structural congruence}

\paragraph{Free and bound names and alpha-equivalence.} At the
core of structural equivalence is alpha-equivalence which identifies
process that are the same up to a change of variable. Formally, we
recognize the distinction between free and bound names. The free names
of a process, $\freenames{P}$, may be calculated recursively as
follows:

\begin{mathpar}
\freenames{\pzero} := \emptyset
  \and \\
  \freenames{x?(y).P} := \{ x \} \cup (\freenames{P} \setminus \{ y \})
  \and 
  \freenames{x!\langle P \rangle} := \{ x \} \cup \{ P \} 
  \and \\
  \freenames{P|Q} := \freenames{P} \cup \freenames{Q}
  \and \\
  \freenames{@{x}} := \{ x \}
\end{mathpar}

$\pi$
$\quotep{\pi}$

$\freenames{-} : \pi \to \mathcal{P}(\quotep{\pi})$

\begin{eqnarray*}
  \freenames{\pzero} & := & \emptyset \\
  \freenames{x?(y).P} & := & \{ x \} \cup (\freenames{P} \setminus \{ y \}) \\
  \freenames{x!\langle P \rangle} & := & \{ x \} \cup \{ P \} \\
  \freenames{P|Q} & := & \freenames{P} \cup \freenames{Q} \\
  \freenames{\dropn{x}} & := & \{ x \}
\end{eqnarray*}

The bound names of a process, $\boundnames{P}$, are those names occurring in $P$
that are not free. For example, in $x?(y).0$, the name $x$ is free, while $y$ is bound.

\begin{mathpar}
  \inferrule* [lab=monoidal-laws] {} { P|Q \equiv Q|P \and P|0 \equiv P \and P|(Q|R) \equiv (P|Q)|R }
\end{mathpar}

\begin{mathpar}
  \inferrule* [lab=alpha-equivalence] {} { (x)P \equiv (y)P\{y/x\} \and y \not\in \freenames{P} }
\end{mathpar}

\begin{definition}
Then two processes, $P,Q$, are alpha-equivalent if $P = Q\{\vec{y}/\vec{x}\}$ for
some $\vec{x} \in \boundnames{Q},\vec{y} \in \boundnames{P}$, where $Q\{\vec{y}/\vec{x}\}$
denotes the capture-avoiding substitution of $\vec{y}$ for $\vec{x}$ in $Q$.
\end{definition}

\begin{definition}
  The {\em structural congruence} \cite{SangiorgiWalker} , $\equiv$,
  between processes is the least congruence containing
  alpha-equivalence, satisfying the abelian monoid laws
  (associativity, commutativity and $\pzero$ as identity) for parallel
  composition $|$ and for summation $+$.
\end{definition}

\subsection{Name equivalence}

We take name equivalence, written $\nameeq$, to be the smallest
equivalence relation generated by the following rules.

\begin{mathpar}
\inferrule*[lab=Quote-drop]
{ }
{ \quotep{@{x}} \nameeq x }

\inferrule*[lab=Struct-equiv]
{ P \scong Q }
{ \quotep{P} \nameeq \quotep{Q} }
\end{mathpar}

The astute reader will have noticed that the mutual recursion of names
and processes imposes a mutual recursion on alpha-equivalence and
structural equivalence via name-equivalence. Fortunately, all of this
works out pleasantly and we may calculate in the natural way, free of
concern. The reader interested in the details is referred to the
appendix \ref{appendix:rho_details}.

\subsection{Substitution}

We use $\Proc$ for the set of processes, $\QProc$ for the set of
names, and $\id{\{}\vec{y} / \vec{x} \id{\}}$ to denote partial maps,
$s : \QProc \rightarrow \QProc$. A map, $s$ lifts, uniquely, to a map
on process terms, $\widehat{s} : \Proc \rightarrow \Proc$ by the
following equations.

\begin{mathpar}
  (0) \psubstp{Q}{P} := 0 \\
  (R \juxtap S) \psubstp{Q}{P}
  :=    
  (R)\psubstp{Q}{P} \juxtap (S) \psubstp{Q}{P} \\
  (x?(y).R) \psubstp{Q}{P}    
  :=    
  (x)\substp{Q}{P} (z)\concat( (R \psubstn{z}{y}) \psubstp{Q}{P} ) \\
  (\lift{x}{R}) \psubstp{Q}{P}  
  :=
  \lift{(x)\substp{Q}{P}}{ R \psubstp{Q}{P} } \\
%   (\dropn{x})  \psubstp{Q}{P}       
%   := 
%   \left\{ 
%     \begin{array}{ccc} 
%       \dropn{\quotep{Q}} & & x \nameeq \quotep{P} \\
%       \dropn{x} & & otherwise \\
%     \end{array}
%   \right. 
  (\dropn{x})  \psubstp{Q}{P}       
  := 
  \left\{ 
    \begin{array}{ccc} 
      Q & & x \nameeq \quotep{P} \\
      \dropn{x} & & otherwise \\
    \end{array}
  \right.
\end{mathpar}
 

where

\begin{eqnarray}
  (x)\id{\{} \lpquote Q \rpquote / \lpquote P \rpquote \id{\}}            = 
  \left\{ 
    \begin{array}{ccc}
      \lpquote Q \rpquote & & x \nameeq \lpquote P \rpquote \\
      x & & otherwise \\
    \end{array}
  \right. \nonumber
\end{eqnarray}

and $z$ is chosen distinct from $\quotep{P}$, $\quotep{Q}$, the free
names in $Q$, and all the names in $R$. Our $\alpha$-equivalence will
be built in the standard way from this substitution.

\begin{remark}\label{rem:no_self_referential_names}
  One consequence of these definitions is that $\forall P. \quotep{P}
  \not\in \freenames{P}$.
\end{remark}

\subsection{ Dynamic quote: an example }

Anticipating something of what's to come, consider applying the
substitution, $\widehat{\id{\{}u / z \id{\}}}$, to the following pair
of processes, $\lift{w}{y!(z)}$ and $w[ \lpquote y!(z) \rpquote ]$.

\begin{eqnarray}
	\lift{w}{y!(z)}\widehat{\id{\{}u / z \id{\}}}
		& = &
		\lift{w}{y!(u)} \nonumber\\
	w[ \lpquote y!(z) \rpquote ] \widehat{ \id{\{}u / z \id{\}} }
		& = &
		w[ \lpquote y!(z) \rpquote ] \nonumber
\end{eqnarray}

Because the body of the process between quotes is impervious to
substitution, we get radically different answers. In fact, by
examining the first process in an input context,
e.g. $x?(z).\lift{w}{y!(z)}$, we see that the process under the lift
operator may be shaped by prefixed inputs binding a name inside it. In
this sense, the lift operator will be seen as a way to dynamically
construct processes before reifying them as names.

Finally equipped with these standard features we can present the
dynamics of the calculus.

\subsubsection{Operational semantics} 

Finally, we introduce the computational dynamics. What marks these
algebras as distinct from other more traditionally studied algebraic
structures, e.g. vector spaces or polynomial rings, is the manner in
which dynamics is captured. In traditional structures, dynamics is typically
expressed through morphisms between such structures, as in linear maps
between vector spaces or morphisms between rings. In algebras
associated with the semantics of computation, the dynamics is
expressed as part of the algebraic structure itself, through a
reduction reduction relation typically denoted by $\red$. Below, we
give a recursive presentation of this relation for the calculus used
in the encoding.

$\red \subseteq \pi \times \pi$
$\red : \pi \to \mathcal{P}(\pi)$

\begin{mathpar}
  \inferrule* [lab=Comm] { \textsf{match}( x_{src}, x_{trgt} ) } { x_{trgt}?(y)P \; | \; x_{src}!\langle {Q} \rangle \red P\{\quotep{Q}/y}\} }
  \and \\
  \inferrule* [lab=Par] {{P} \red {P}'} {{{P} | {Q}} \red {{P}' | {Q}}}
  \and
  \inferrule* [lab=Equiv]{{{P} \scong {P}'} \andalso {{P}' \red {Q}'} \andalso {{Q}' \scong {Q}}}{{P} \red {Q}}
\end{mathpar}

\begin{eqnarray*}
  match_{\equiv} (\quotep{P},\quotep{Q}) & := & P \equiv Q \\
  match_{\dagger}(\quotep{P},\quotep{Q}) & := & \forall R. P|Q \red^{*} R => R \red^{*} 0 \\
  match_{K}(\quotep{P},\quotep{Q}) & := & K \mbox{ for some context } K
\end{eqnarray*}

$u?(x)P | u!\langle Q \rangle \red P\{\quotep{Q}/x\}$

%We write $\wred$ for $\red^*$, and $P\red$ if $\exists Q $ such that $ P \red Q$.
We write $P\red$ if $\exists Q $ such that $ P \red Q$ and $P\not\red$, otherwise.

\section{Replication}

As mentioned before, it is known that replication (and hence
recursion) can be implemented in a higher-order process algebra
\cite{SangiorgiWalker}. As our first example of calculation with the
machinery thus far presented we give the construction explicitly in
the {\rhoc}.

\begin{eqnarray}
	D_{x} & := & \prefix{x}{y}{(\binpar{\outputp{x}{y}}{@{y}})} \nonumber\\
	\bangp_{x}{P} & := & \binpar{{x}!\langle{\binpar{D_{x}}{P}}\rangle}{D_{x}} \nonumber
\end{eqnarray}

\begin{eqnarray}
	\bangp_{x}{P} & & \nonumber\\
	=
	& {x}!\langle{(\prefix{x}{y}{(\outputp{x}{y} | @{y})) | P}}\rangle 
	      | \prefix{x}{y}{(\outputp{x}{y} | @{y})} & \nonumber\\
	\red
	& (\outputp{x}{y} | @{y})\substn{\quotep{(\prefix{x}{y}{(@{y} | \outputp{x}{y})) | P}}}{y} & \nonumber\\
	=
	& \outputp{x}{\quotep{(\prefix{x}{y}{(\outputp{x}{y} | @{y})) | P}}}
	  | {(\prefix{x}{y}{(\outputp{x}{y} | @{y})) | P}} & \nonumber\\
	\red
	& \ldots & \nonumber\\
	\red^*
	& P | P | \ldots & \nonumber
\end{eqnarray}

Of course, this encoding, as an implementation, runs away, unfolding
$\bangp{P}$ eagerly. A lazier and more implementable replication
operator, restricted to input-guarded processes, may be obtained as follows.

\begin{eqnarray}
\bangp{\prefix{u}{v}{P}} 
	:= 
	\binpar{\lift{x}{\prefix{u}{v}{(\binpar{D(x)}{P})}}}{D(x)} \nonumber
\end{eqnarray}

\begin{remark}
  Note that the lazier definition still does not deal with summation
  or mixed summation (i.e. sums over input and output). The reader is
  invited to construct definitions of replication that deal with these
  features. 

  Further, the definitions are parameterized in a name, $x$. Can you,
  gentle reader, make a definition that eliminates this parameter and
  guarantees no accidental interaction between the replication
  machinery and the process being replicated -- i.e. no accidental
  sharing of names used by the process to get its work done and the
  name(s) used by the replication to effect copying. This latter
  revision of the definition of replication is crucial to obtaining
  the expected identity $!!P \sim !P$.
\end{remark}

\begin{remark}\label{rem:paradoxical_combinator}
  The reader familiar with the lambda calculus will have noticed the
  similarity between $D$ and the paradoxical combinator.

  [Ed. note: the existence of this seems to suggest we have to be more
  restrictive on the set of processes and names we admit if we are to
  support no-cloning.]
\end{remark}

\subsubsection{Bisimulation}

The computational dynamics gives rise to another kind of equivalence,
the equivalence of computational behavior. As previously mentioned
this is typically captured \emph{via} some form of bisimulation.

% The notion we use in this paper is weak barbed bisimulation
% \cite{milner91polyadicpi}.

The notion we use in this paper is derived from weak barbed
bisimulation \cite{milner91polyadicpi}. 

\begin{definition}
An \emph{observation relation}, $\downarrow_{\mathcal N}$, over a set
of names, $\mathcal N$, is the smallest relation satisfying the rules
below.

\infrule[Out-barb]{y \in {\mathcal N}, \; x \nameeq y}
		  {\outputp{x}{v} \downarrow_{\mathcal N} x}
\infrule[Par-barb]{\mbox{$P\downarrow_{\mathcal N} x$ or $Q\downarrow_{\mathcal N} x$}}
		  {\binpar{P}{Q} \downarrow_{\mathcal N} x}

We write $P \Downarrow_{\mathcal N} x$ if there is $Q$ such that 
$P \wred Q$ and $Q \downarrow_{\mathcal N} x$.
\end{definition}

\begin{definition}
%\label{def.bbisim}
An  ${\mathcal N}$-\emph{barbed bisimulation} over a set of names, ${\mathcal N}$, is a symmetric binary relation 
${\mathcal S}_{\mathcal N}$ between agents such that $P\rel{S}_{\mathcal N}Q$ implies:
\begin{enumerate}
\item If $P \red P'$ then $Q \wred Q'$ and $P'\rel{S}_{\mathcal N} Q'$.
\item If $P\downarrow_{\mathcal N} x$, then $Q\Downarrow_{\mathcal N} x$.
\end{enumerate}
$P$ is ${\mathcal N}$-barbed bisimilar to $Q$, written
$P \wbbisim_{\mathcal N} Q$, if $P \rel{S}_{\mathcal N} Q$ for some ${\mathcal N}$-barbed bisimulation ${\mathcal S}_{\mathcal N}$.
\end{definition}

$\mathcal{R} \subseteq \pi \times \pi$

$P \mathcal{R} Q => \forall P'. P \red P' \Rightarrow \exists Q'. Q \red Q', P' \mathcal{R} Q'$

$P \vdash x \Rightarrow Q \vdash x$

\begin{mathpar}
  \inferrule*[lab=Out-barb]{x \nameeq y}{{y}!\langle{Q}\rangle \vdash x}
  \and
  \inferrule*[lab=Par-barb]{\mbox{$P\vdash x$ or $Q\vdash x$}}{\binpar{P}{Q} \vdash x}
\end{mathpar}

\subsubsection{Contexts}

One of the principle advantages of computational calculi like the
$\pi$-calculus is a well-defined notion of context,
contextual-equivalence and a correlation between
contextual-equivalence and notions of bisimulation. The notion of
context allows the decomposition of a process into (sub-)process and
its syntactic environment, its context. Thus, a context may be
thought of as a process with a ``hole'' (written $\Box$) in it. The
application of a context $M$ to a process $P$, written $M[P]$, is
tantamount to filling the hole in $M$ with $P$. In this paper we do
not need the full weight of this theory, but do make use of the notion
of context in the proof the main theorem. 

\begin{mathpar}
  \inferrule* [lab=summation] {} {{M_{M},M_{N}} \bc \Box \;|\; x.M_{A} \;|\; M_{M}+M_{N}}
  \and
  \inferrule* [lab=agent] {} {{M_{A}} \bc (\vec{x})M_{P} \;| \; \clift{P_0,\ldots,M_{P},\ldots,P_N}}
  \and \\
  \inferrule* [lab=process] {} {{M_{P}} \bc M_{N} \;| \;P|M_{P} }
\end{mathpar} 

\begin{mathpar}
  \inferrule* [lab=sychronization] {} {M_{N} \bc \Box \;|\; x?M_{F} \;|\; x!M_{C}}
  \and
  \inferrule* [lab=abstraction] {} {{M_{F}} \bc (x)M_{P} }
  \and
  \inferrule* [lab=concretion] {} {{M_{C}} \bc \langle M_{P} \rangle }
  \and \\
  \inferrule* [lab=process] {} {{M_{P}} \bc M_{N} \;| \;P|M_{P} }
\end{mathpar}

\begin{definition}[contextual application] Given a context $M$, and
  process $P$, we define the \emph{contextual application}, $M[P] :=
  M\{P/\Box\}$. That is, the contextual application of M to P is the
  substitution of $P$ for $\Box$ in $M$.
\end{definition}

$\meaningof{-} : L \to \mathcal{P}(\pi)$

\begin{mathpar}
  \inferrule* [lab=collection] {} {\meaningof{true} = \pi, \and \meaningof{~E} = \pi \setminus \meaningof{E}, \and \meaningof{E_{1} \& E_{2}} = \meaningof{E_{1}} \cap \meaningof{E_{2}}}
\end{mathpar}

\begin{mathpar}
  \inferrule* [lab=structure] {} {\meaningof{0} = \{ P \in \pi | P \equiv 0 \}, \and \\ \meaningof{E_1 | E_2} = \{ P \in \pi | P \equiv P_{1} | P_{2}, P_{1} \in \meaningof{E_{1}}, P_{2} \in \meaningof{E_2}\} }
\end{mathpar}

\begin{mathpar}
 \inferrule* [lab=behavior] {} {\meaningof{\langle a?b \rangle E} = \{ P \in \pi | P \equiv Q | u?(y)P', \\ \and \\\\ \and \\ \;\;\; u \in \meaningof{a}, \forall z.P'\{z/y\} \in \meaningof{E\{z/b\}}\}, \and \\ \meaningof{a!E} = \{ P \in \pi | P \equiv Q | x!\langle P' \rangle, x \in \meaningof{a} P' \in \meaningof{E}\} }
\end{mathpar}

\begin{mathpar}
 \inferrule* [lab=nominal] {} {\meaningof{\quotep{E}} = \{ \quotep{P} \in \quotep{\pi} | P \in \meaningof{E} \}, \and \meaningof{\quotep{P}} = \{ \quotep{Q} \in \quotep{\pi} | P \equiv Q \} \and \\ \meaningof{@\quotep{E}} = \{ P \in \pi | P \equiv @x, x \in \meaningof{E} \}}
\end{mathpar}

\begin{eqnarray*}
  \\
  \meaningof{-} : TS \to ST
\end{eqnarray*}

\begin{eqnarray*}
  \\
  L : TS \to ST
\end{eqnarray*}

\begin{eqnarray*}
  \\
  P \models E \iff P \in \meaningof{E}
\end{eqnarray*}

\begin{eqnarray*}
  P \approx_{L} Q \iff \forall E \in L. P \models E \iff Q \models E
\end{eqnarray*}

\begin{eqnarray*}
  P \approx_{K} Q
\end{eqnarray*}

\begin{eqnarray*}
  P \approx Q
\end{eqnarray*}

$\approx_{K} = \approx = \approx_{L}$

\subsubsection{Contextual duality}

Note that contexts extend the quotation operation to a family of
operations from processes to names. Given a context, $M$, we can
define a \emph{nominal context}, $\quotep{M}$ by $\quotep{M}[P] :=
\quotep{M[P]}$. To foreshadow what is to come we observe that these
operations enjoy a duality with processes very much like the duality
between vectors and maps from vectors to scalars.

Further, because the calculus is essentially higher-order, we have a
correspondence between contexts and processes. More specifically,
given a name $x$ and a context $M$ we can construct $M^{*}_{x}$ such
that 

\begin{mathpar}
  M^{*}_{x} | \lift{x}{P} \red M[P]
\end{mathpar}

namely,

\begin{mathpar}
  M^{*}_{x} := x?(u).M[\dropn{u}]
\end{mathpar}

The dependence of $M^{*}_{x}$ on a name makes it an abstraction, 

\begin{mathpar}
  M^{*} := (x)x?(u).M[\dropn{u}]
\end{mathpar}

\subsection{Additional notation}

It will sometimes be convenient to denote the process a name
quotes. We already have the notation $x = \quotep{P}$, but it will be
convenient to introduce an alternate notation, $\procn{x}$, when we
want to emphasize the connection to the use of the name. Note that, by
virtue of name equivalence, $\quotep{\procn{x}} \nameeq x$; so, the
notation is consistent with previous definitions.

Further, because names have structure it is possible to effect
substitutions on the basis of that structure. This means we need to
upgrade our notation for substitutions, which we accomplish by
adapting comprehension notation. Thus,

\begin{mathpar}
  P\{ y / x : x \in S \}
\end{mathpar}

is interpreted to mean the process derived from P by replacing (in a
capture-avoiding manner) each occurrence of $x$ in $S$ by $y$. For example,

\begin{mathpar}
  P\{ \quotep{\procn{x}|\procn{x}} / x : x \in \freenames{P} \}
\end{mathpar}

will replace each (occurrence) of a free name $x$ in $P$ by
$\quotep{\procn{x}|\procn{x}}$.

Also, we will avail ourselves of the notation $x^{L}$ and $x^{R}$ to
denote injections of a name into disjoint copies of the name
space. There are numerous ways to accomplish this. One example can be
found in \cite{MeredithR05}. This notation overloads to vectors of
names: $\vec{x}^{\pi} := (x_{i}^{\pi} \; : \; 0 \leq i < |\vec{x}| )$ where $\pi \in \{L,R\}$.

We also use $P^{\Box} := P|\Box$.

In \cite{MeredithR05} an interpretation of the new operator is
given. It turns out that there are several possible interpretations
all enjoying the requisite algebraic properties of the operator (see
\cite{milner91polyadicpi}). We will therefore make liberal use of
$(\nu\; \vec{x})P$.

% subsection the_syntax_and_semantics_of_the_notation_system (end)   

\input{qm2pi.qmops} 

\input{qm2pi.sterngerlach} 

\input{qm2pi.metric} 

% section concurrent_process_calculi (end)

%\input{qm2pi.proofsketch}

% section proof sketch (end)

%\input{qm2pi.slviaknots} 

% section spatial logic via knots (end)

\input{qm2pi.conclusion}

% section conclusion (end)

%\input{qm2pi.dtcodes} 

% section wiring algorithm (end)

\input{qm2pi.ack} 

% section acknowledgments (end)

\newpage


\bibliographystyle{plain}   
\bibliography{../../biblios/main.bib}

\input{qm2pi.rhodetails}

\end{document}

 

% section concurrent_process_calculi (end)

%\documentclass[12pt]{llncs}
%\documentclass{jktr}

\usepackage[pdftex]{hyperref}                   
\usepackage {listings}
\usepackage {mathpartir}
\usepackage{bcprules}
%\usepackage{listings}
                       
\usepackage{graphicx} 
%\usepackage[margins=2.5cm,nohead,nofoot]{geometry}
%\usepackage{geometry}
\usepackage{amsfonts}
\usepackage{amstext}
\usepackage{latexsym}
\usepackage{amssymb}
\usepackage{color}


%\include{myPreamble}
\include{qm2pi.local} 

%\ifpdf
%\usepackage[pdftex]{graphicx}
%\else
%\usepackage{graphicx}
%\fi

 % \ifpdf
%  \usepackage{pdfsync}
%  \if


%\title{Brief Article}
%\author{David F. Snyder}
%\author{L.G. Meredith}

%\address{Dept. of Math., Texas State University--San Marcos, San Marcos, TX 78666}
       
\pagestyle{empty}


\begin{document}

\lstset{language=[Objective]Caml,frame=shadowbox}

\input{qm2pi.front}

% section front matter (end)

\input{qm2pi.intro} 
 
% section introduction (end)

% \input{qm2pi.knotations} 

% section notation (end)

\input{qm2pi.process.calculi} 

% section concurrent_process_calculi_and_spatial_logics_ (end)
    
%\input{qm2pi.knots2pi} 

%\input{qm2pi.trefoil} 

%\input{qm2pi.mainthm} 

% subsection basic_interpretation (end)

%\input{qm2pi.rho.presentation} 
\subsection{The syntax and semantics of the notation system}\label{sub:the_syntax_and_semantics_of_the_notation_system} % (fold)

We now summarize a technical presentation of the calculus that
embodies our theory of dynamics. The typical presentation of such a
calculus follows the style of giving generators and relations on
them. The grammar, below, describing term constructors, freely
generates the set of processes, $\Proc$. This set is then quotiented
by a relation known as structural congruence and it is over this set
that the notion of dynamics is expressed. This presentation is
essentially that of \cite{MeredithR05} with the addition of
polyadicity and summation. For readability we have relegated some of
the technical subtleties to an appendix.

\subsubsection{Process grammar}\label{subsub:process_grammar}

\begin{mathpar}
  \inferrule* [lab=synchronization] {} {{M} \bc \pzero \;|\; x?F \;|\; x!C }
  \and
  \inferrule* [lab=abstraction] {} {{F} \bc (x)P}
  \and
  \inferrule* [lab=concretion] {} {{C} \bc \langle Q \rangle}
  \and
  \inferrule* [lab=process] {} {{P,Q} \bc M \;| \;P|Q \;|\; @{x}}
  \and
  \inferrule* [lab=name] {} {{x} \bc \quotep{P}}
\end{mathpar} 

Note that $\vec{x}$ (resp. $\vec{P}$) denotes a vector of names
(resp. processes) of length $|\vec{x}|$ (resp. $|\vec{P}|$). We adopt
the following useful abbreviations.

\begin{mathpar}
   x?(\vec{y}).P := x.(\vec{y})P \and  x\clift{\vec{P}} := x.\clift{\vec{P}}
   \and x!(y) := \lift{x}{\dropn{y}}
   \and \Pi_{i=0}^{n-1}P_i := P_0 | \ldots | P_{n-1}
\end{mathpar}

\subsubsection{Structural congruence}

\paragraph{Free and bound names and alpha-equivalence.} At the
core of structural equivalence is alpha-equivalence which identifies
process that are the same up to a change of variable. Formally, we
recognize the distinction between free and bound names. The free names
of a process, $\freenames{P}$, may be calculated recursively as
follows:

\begin{mathpar}
\freenames{\pzero} := \emptyset
  \and \\
  \freenames{x?(y).P} := \{ x \} \cup (\freenames{P} \setminus \{ y \})
  \and 
  \freenames{x!\langle P \rangle} := \{ x \} \cup \{ P \} 
  \and \\
  \freenames{P|Q} := \freenames{P} \cup \freenames{Q}
  \and \\
  \freenames{@{x}} := \{ x \}
\end{mathpar}

$\pi$
$\quotep{\pi}$

$\freenames{-} : \pi \to \mathcal{P}(\quotep{\pi})$

\begin{eqnarray*}
  \freenames{\pzero} & := & \emptyset \\
  \freenames{x?(y).P} & := & \{ x \} \cup (\freenames{P} \setminus \{ y \}) \\
  \freenames{x!\langle P \rangle} & := & \{ x \} \cup \{ P \} \\
  \freenames{P|Q} & := & \freenames{P} \cup \freenames{Q} \\
  \freenames{\dropn{x}} & := & \{ x \}
\end{eqnarray*}

The bound names of a process, $\boundnames{P}$, are those names occurring in $P$
that are not free. For example, in $x?(y).0$, the name $x$ is free, while $y$ is bound.

\begin{mathpar}
  \inferrule* [lab=monoidal-laws] {} { P|Q \equiv Q|P \and P|0 \equiv P \and P|(Q|R) \equiv (P|Q)|R }
\end{mathpar}

\begin{mathpar}
  \inferrule* [lab=alpha-equivalence] {} { (x)P \equiv (y)P\{y/x\} \and y \not\in \freenames{P} }
\end{mathpar}

\begin{definition}
Then two processes, $P,Q$, are alpha-equivalent if $P = Q\{\vec{y}/\vec{x}\}$ for
some $\vec{x} \in \boundnames{Q},\vec{y} \in \boundnames{P}$, where $Q\{\vec{y}/\vec{x}\}$
denotes the capture-avoiding substitution of $\vec{y}$ for $\vec{x}$ in $Q$.
\end{definition}

\begin{definition}
  The {\em structural congruence} \cite{SangiorgiWalker} , $\equiv$,
  between processes is the least congruence containing
  alpha-equivalence, satisfying the abelian monoid laws
  (associativity, commutativity and $\pzero$ as identity) for parallel
  composition $|$ and for summation $+$.
\end{definition}

\subsection{Name equivalence}

We take name equivalence, written $\nameeq$, to be the smallest
equivalence relation generated by the following rules.

\begin{mathpar}
\inferrule*[lab=Quote-drop]
{ }
{ \quotep{@{x}} \nameeq x }

\inferrule*[lab=Struct-equiv]
{ P \scong Q }
{ \quotep{P} \nameeq \quotep{Q} }
\end{mathpar}

The astute reader will have noticed that the mutual recursion of names
and processes imposes a mutual recursion on alpha-equivalence and
structural equivalence via name-equivalence. Fortunately, all of this
works out pleasantly and we may calculate in the natural way, free of
concern. The reader interested in the details is referred to the
appendix \ref{appendix:rho_details}.

\subsection{Substitution}

We use $\Proc$ for the set of processes, $\QProc$ for the set of
names, and $\id{\{}\vec{y} / \vec{x} \id{\}}$ to denote partial maps,
$s : \QProc \rightarrow \QProc$. A map, $s$ lifts, uniquely, to a map
on process terms, $\widehat{s} : \Proc \rightarrow \Proc$ by the
following equations.

\begin{mathpar}
  (0) \psubstp{Q}{P} := 0 \\
  (R \juxtap S) \psubstp{Q}{P}
  :=    
  (R)\psubstp{Q}{P} \juxtap (S) \psubstp{Q}{P} \\
  (x?(y).R) \psubstp{Q}{P}    
  :=    
  (x)\substp{Q}{P} (z)\concat( (R \psubstn{z}{y}) \psubstp{Q}{P} ) \\
  (\lift{x}{R}) \psubstp{Q}{P}  
  :=
  \lift{(x)\substp{Q}{P}}{ R \psubstp{Q}{P} } \\
%   (\dropn{x})  \psubstp{Q}{P}       
%   := 
%   \left\{ 
%     \begin{array}{ccc} 
%       \dropn{\quotep{Q}} & & x \nameeq \quotep{P} \\
%       \dropn{x} & & otherwise \\
%     \end{array}
%   \right. 
  (\dropn{x})  \psubstp{Q}{P}       
  := 
  \left\{ 
    \begin{array}{ccc} 
      Q & & x \nameeq \quotep{P} \\
      \dropn{x} & & otherwise \\
    \end{array}
  \right.
\end{mathpar}
 

where

\begin{eqnarray}
  (x)\id{\{} \lpquote Q \rpquote / \lpquote P \rpquote \id{\}}            = 
  \left\{ 
    \begin{array}{ccc}
      \lpquote Q \rpquote & & x \nameeq \lpquote P \rpquote \\
      x & & otherwise \\
    \end{array}
  \right. \nonumber
\end{eqnarray}

and $z$ is chosen distinct from $\quotep{P}$, $\quotep{Q}$, the free
names in $Q$, and all the names in $R$. Our $\alpha$-equivalence will
be built in the standard way from this substitution.

\begin{remark}\label{rem:no_self_referential_names}
  One consequence of these definitions is that $\forall P. \quotep{P}
  \not\in \freenames{P}$.
\end{remark}

\subsection{ Dynamic quote: an example }

Anticipating something of what's to come, consider applying the
substitution, $\widehat{\id{\{}u / z \id{\}}}$, to the following pair
of processes, $\lift{w}{y!(z)}$ and $w[ \lpquote y!(z) \rpquote ]$.

\begin{eqnarray}
	\lift{w}{y!(z)}\widehat{\id{\{}u / z \id{\}}}
		& = &
		\lift{w}{y!(u)} \nonumber\\
	w[ \lpquote y!(z) \rpquote ] \widehat{ \id{\{}u / z \id{\}} }
		& = &
		w[ \lpquote y!(z) \rpquote ] \nonumber
\end{eqnarray}

Because the body of the process between quotes is impervious to
substitution, we get radically different answers. In fact, by
examining the first process in an input context,
e.g. $x?(z).\lift{w}{y!(z)}$, we see that the process under the lift
operator may be shaped by prefixed inputs binding a name inside it. In
this sense, the lift operator will be seen as a way to dynamically
construct processes before reifying them as names.

Finally equipped with these standard features we can present the
dynamics of the calculus.

\subsubsection{Operational semantics} 

Finally, we introduce the computational dynamics. What marks these
algebras as distinct from other more traditionally studied algebraic
structures, e.g. vector spaces or polynomial rings, is the manner in
which dynamics is captured. In traditional structures, dynamics is typically
expressed through morphisms between such structures, as in linear maps
between vector spaces or morphisms between rings. In algebras
associated with the semantics of computation, the dynamics is
expressed as part of the algebraic structure itself, through a
reduction reduction relation typically denoted by $\red$. Below, we
give a recursive presentation of this relation for the calculus used
in the encoding.

$\red \subseteq \pi \times \pi$
$\red : \pi \to \mathcal{P}(\pi)$

\begin{mathpar}
  \inferrule* [lab=Comm] { \textsf{match}( x_{src}, x_{trgt} ) } { x_{trgt}?(y)P \; | \; x_{src}!\langle {Q} \rangle \red P\{\quotep{Q}/y}\} }
  \and \\
  \inferrule* [lab=Par] {{P} \red {P}'} {{{P} | {Q}} \red {{P}' | {Q}}}
  \and
  \inferrule* [lab=Equiv]{{{P} \scong {P}'} \andalso {{P}' \red {Q}'} \andalso {{Q}' \scong {Q}}}{{P} \red {Q}}
\end{mathpar}

\begin{eqnarray*}
  match_{\equiv} (\quotep{P},\quotep{Q}) & := & P \equiv Q \\
  match_{\dagger}(\quotep{P},\quotep{Q}) & := & \forall R. P|Q \red^{*} R => R \red^{*} 0 \\
  match_{K}(\quotep{P},\quotep{Q}) & := & K \mbox{ for some context } K
\end{eqnarray*}

$u?(x)P | u!\langle Q \rangle \red P\{\quotep{Q}/x\}$

%We write $\wred$ for $\red^*$, and $P\red$ if $\exists Q $ such that $ P \red Q$.
We write $P\red$ if $\exists Q $ such that $ P \red Q$ and $P\not\red$, otherwise.

\section{Replication}

As mentioned before, it is known that replication (and hence
recursion) can be implemented in a higher-order process algebra
\cite{SangiorgiWalker}. As our first example of calculation with the
machinery thus far presented we give the construction explicitly in
the {\rhoc}.

\begin{eqnarray}
	D_{x} & := & \prefix{x}{y}{(\binpar{\outputp{x}{y}}{@{y}})} \nonumber\\
	\bangp_{x}{P} & := & \binpar{{x}!\langle{\binpar{D_{x}}{P}}\rangle}{D_{x}} \nonumber
\end{eqnarray}

\begin{eqnarray}
	\bangp_{x}{P} & & \nonumber\\
	=
	& {x}!\langle{(\prefix{x}{y}{(\outputp{x}{y} | @{y})) | P}}\rangle 
	      | \prefix{x}{y}{(\outputp{x}{y} | @{y})} & \nonumber\\
	\red
	& (\outputp{x}{y} | @{y})\substn{\quotep{(\prefix{x}{y}{(@{y} | \outputp{x}{y})) | P}}}{y} & \nonumber\\
	=
	& \outputp{x}{\quotep{(\prefix{x}{y}{(\outputp{x}{y} | @{y})) | P}}}
	  | {(\prefix{x}{y}{(\outputp{x}{y} | @{y})) | P}} & \nonumber\\
	\red
	& \ldots & \nonumber\\
	\red^*
	& P | P | \ldots & \nonumber
\end{eqnarray}

Of course, this encoding, as an implementation, runs away, unfolding
$\bangp{P}$ eagerly. A lazier and more implementable replication
operator, restricted to input-guarded processes, may be obtained as follows.

\begin{eqnarray}
\bangp{\prefix{u}{v}{P}} 
	:= 
	\binpar{\lift{x}{\prefix{u}{v}{(\binpar{D(x)}{P})}}}{D(x)} \nonumber
\end{eqnarray}

\begin{remark}
  Note that the lazier definition still does not deal with summation
  or mixed summation (i.e. sums over input and output). The reader is
  invited to construct definitions of replication that deal with these
  features. 

  Further, the definitions are parameterized in a name, $x$. Can you,
  gentle reader, make a definition that eliminates this parameter and
  guarantees no accidental interaction between the replication
  machinery and the process being replicated -- i.e. no accidental
  sharing of names used by the process to get its work done and the
  name(s) used by the replication to effect copying. This latter
  revision of the definition of replication is crucial to obtaining
  the expected identity $!!P \sim !P$.
\end{remark}

\begin{remark}\label{rem:paradoxical_combinator}
  The reader familiar with the lambda calculus will have noticed the
  similarity between $D$ and the paradoxical combinator.

  [Ed. note: the existence of this seems to suggest we have to be more
  restrictive on the set of processes and names we admit if we are to
  support no-cloning.]
\end{remark}

\subsubsection{Bisimulation}

The computational dynamics gives rise to another kind of equivalence,
the equivalence of computational behavior. As previously mentioned
this is typically captured \emph{via} some form of bisimulation.

% The notion we use in this paper is weak barbed bisimulation
% \cite{milner91polyadicpi}.

The notion we use in this paper is derived from weak barbed
bisimulation \cite{milner91polyadicpi}. 

\begin{definition}
An \emph{observation relation}, $\downarrow_{\mathcal N}$, over a set
of names, $\mathcal N$, is the smallest relation satisfying the rules
below.

\infrule[Out-barb]{y \in {\mathcal N}, \; x \nameeq y}
		  {\outputp{x}{v} \downarrow_{\mathcal N} x}
\infrule[Par-barb]{\mbox{$P\downarrow_{\mathcal N} x$ or $Q\downarrow_{\mathcal N} x$}}
		  {\binpar{P}{Q} \downarrow_{\mathcal N} x}

We write $P \Downarrow_{\mathcal N} x$ if there is $Q$ such that 
$P \wred Q$ and $Q \downarrow_{\mathcal N} x$.
\end{definition}

\begin{definition}
%\label{def.bbisim}
An  ${\mathcal N}$-\emph{barbed bisimulation} over a set of names, ${\mathcal N}$, is a symmetric binary relation 
${\mathcal S}_{\mathcal N}$ between agents such that $P\rel{S}_{\mathcal N}Q$ implies:
\begin{enumerate}
\item If $P \red P'$ then $Q \wred Q'$ and $P'\rel{S}_{\mathcal N} Q'$.
\item If $P\downarrow_{\mathcal N} x$, then $Q\Downarrow_{\mathcal N} x$.
\end{enumerate}
$P$ is ${\mathcal N}$-barbed bisimilar to $Q$, written
$P \wbbisim_{\mathcal N} Q$, if $P \rel{S}_{\mathcal N} Q$ for some ${\mathcal N}$-barbed bisimulation ${\mathcal S}_{\mathcal N}$.
\end{definition}

$\mathcal{R} \subseteq \pi \times \pi$

$P \mathcal{R} Q => \forall P'. P \red P' \Rightarrow \exists Q'. Q \red Q', P' \mathcal{R} Q'$

$P \vdash x \Rightarrow Q \vdash x$

\begin{mathpar}
  \inferrule*[lab=Out-barb]{x \nameeq y}{{y}!\langle{Q}\rangle \vdash x}
  \and
  \inferrule*[lab=Par-barb]{\mbox{$P\vdash x$ or $Q\vdash x$}}{\binpar{P}{Q} \vdash x}
\end{mathpar}

\subsubsection{Contexts}

One of the principle advantages of computational calculi like the
$\pi$-calculus is a well-defined notion of context,
contextual-equivalence and a correlation between
contextual-equivalence and notions of bisimulation. The notion of
context allows the decomposition of a process into (sub-)process and
its syntactic environment, its context. Thus, a context may be
thought of as a process with a ``hole'' (written $\Box$) in it. The
application of a context $M$ to a process $P$, written $M[P]$, is
tantamount to filling the hole in $M$ with $P$. In this paper we do
not need the full weight of this theory, but do make use of the notion
of context in the proof the main theorem. 

\begin{mathpar}
  \inferrule* [lab=summation] {} {{M_{M},M_{N}} \bc \Box \;|\; x.M_{A} \;|\; M_{M}+M_{N}}
  \and
  \inferrule* [lab=agent] {} {{M_{A}} \bc (\vec{x})M_{P} \;| \; \clift{P_0,\ldots,M_{P},\ldots,P_N}}
  \and \\
  \inferrule* [lab=process] {} {{M_{P}} \bc M_{N} \;| \;P|M_{P} }
\end{mathpar} 

\begin{mathpar}
  \inferrule* [lab=sychronization] {} {M_{N} \bc \Box \;|\; x?M_{F} \;|\; x!M_{C}}
  \and
  \inferrule* [lab=abstraction] {} {{M_{F}} \bc (x)M_{P} }
  \and
  \inferrule* [lab=concretion] {} {{M_{C}} \bc \langle M_{P} \rangle }
  \and \\
  \inferrule* [lab=process] {} {{M_{P}} \bc M_{N} \;| \;P|M_{P} }
\end{mathpar}

\begin{definition}[contextual application] Given a context $M$, and
  process $P$, we define the \emph{contextual application}, $M[P] :=
  M\{P/\Box\}$. That is, the contextual application of M to P is the
  substitution of $P$ for $\Box$ in $M$.
\end{definition}

$\meaningof{-} : L \to \mathcal{P}(\pi)$

\begin{mathpar}
  \inferrule* [lab=collection] {} {\meaningof{true} = \pi, \and \meaningof{~E} = \pi \setminus \meaningof{E}, \and \meaningof{E_{1} \& E_{2}} = \meaningof{E_{1}} \cap \meaningof{E_{2}}}
\end{mathpar}

\begin{mathpar}
  \inferrule* [lab=structure] {} {\meaningof{0} = \{ P \in \pi | P \equiv 0 \}, \and \\ \meaningof{E_1 | E_2} = \{ P \in \pi | P \equiv P_{1} | P_{2}, P_{1} \in \meaningof{E_{1}}, P_{2} \in \meaningof{E_2}\} }
\end{mathpar}

\begin{mathpar}
 \inferrule* [lab=behavior] {} {\meaningof{\langle a?b \rangle E} = \{ P \in \pi | P \equiv Q | u?(y)P', \\ \and \\\\ \and \\ \;\;\; u \in \meaningof{a}, \forall z.P'\{z/y\} \in \meaningof{E\{z/b\}}\}, \and \\ \meaningof{a!E} = \{ P \in \pi | P \equiv Q | x!\langle P' \rangle, x \in \meaningof{a} P' \in \meaningof{E}\} }
\end{mathpar}

\begin{mathpar}
 \inferrule* [lab=nominal] {} {\meaningof{\quotep{E}} = \{ \quotep{P} \in \quotep{\pi} | P \in \meaningof{E} \}, \and \meaningof{\quotep{P}} = \{ \quotep{Q} \in \quotep{\pi} | P \equiv Q \} \and \\ \meaningof{@\quotep{E}} = \{ P \in \pi | P \equiv @x, x \in \meaningof{E} \}}
\end{mathpar}

\begin{eqnarray*}
  \\
  \meaningof{-} : TS \to ST
\end{eqnarray*}

\begin{eqnarray*}
  \\
  L : TS \to ST
\end{eqnarray*}

\begin{eqnarray*}
  \\
  P \models E \iff P \in \meaningof{E}
\end{eqnarray*}

\begin{eqnarray*}
  P \approx_{L} Q \iff \forall E \in L. P \models E \iff Q \models E
\end{eqnarray*}

\begin{eqnarray*}
  P \approx_{K} Q
\end{eqnarray*}

\begin{eqnarray*}
  P \approx Q
\end{eqnarray*}

$\approx_{K} = \approx = \approx_{L}$

\subsubsection{Contextual duality}

Note that contexts extend the quotation operation to a family of
operations from processes to names. Given a context, $M$, we can
define a \emph{nominal context}, $\quotep{M}$ by $\quotep{M}[P] :=
\quotep{M[P]}$. To foreshadow what is to come we observe that these
operations enjoy a duality with processes very much like the duality
between vectors and maps from vectors to scalars.

Further, because the calculus is essentially higher-order, we have a
correspondence between contexts and processes. More specifically,
given a name $x$ and a context $M$ we can construct $M^{*}_{x}$ such
that 

\begin{mathpar}
  M^{*}_{x} | \lift{x}{P} \red M[P]
\end{mathpar}

namely,

\begin{mathpar}
  M^{*}_{x} := x?(u).M[\dropn{u}]
\end{mathpar}

The dependence of $M^{*}_{x}$ on a name makes it an abstraction, 

\begin{mathpar}
  M^{*} := (x)x?(u).M[\dropn{u}]
\end{mathpar}

\subsection{Additional notation}

It will sometimes be convenient to denote the process a name
quotes. We already have the notation $x = \quotep{P}$, but it will be
convenient to introduce an alternate notation, $\procn{x}$, when we
want to emphasize the connection to the use of the name. Note that, by
virtue of name equivalence, $\quotep{\procn{x}} \nameeq x$; so, the
notation is consistent with previous definitions.

Further, because names have structure it is possible to effect
substitutions on the basis of that structure. This means we need to
upgrade our notation for substitutions, which we accomplish by
adapting comprehension notation. Thus,

\begin{mathpar}
  P\{ y / x : x \in S \}
\end{mathpar}

is interpreted to mean the process derived from P by replacing (in a
capture-avoiding manner) each occurrence of $x$ in $S$ by $y$. For example,

\begin{mathpar}
  P\{ \quotep{\procn{x}|\procn{x}} / x : x \in \freenames{P} \}
\end{mathpar}

will replace each (occurrence) of a free name $x$ in $P$ by
$\quotep{\procn{x}|\procn{x}}$.

Also, we will avail ourselves of the notation $x^{L}$ and $x^{R}$ to
denote injections of a name into disjoint copies of the name
space. There are numerous ways to accomplish this. One example can be
found in \cite{MeredithR05}. This notation overloads to vectors of
names: $\vec{x}^{\pi} := (x_{i}^{\pi} \; : \; 0 \leq i < |\vec{x}| )$ where $\pi \in \{L,R\}$.

We also use $P^{\Box} := P|\Box$.

In \cite{MeredithR05} an interpretation of the new operator is
given. It turns out that there are several possible interpretations
all enjoying the requisite algebraic properties of the operator (see
\cite{milner91polyadicpi}). We will therefore make liberal use of
$(\nu\; \vec{x})P$.

% subsection the_syntax_and_semantics_of_the_notation_system (end)   

\input{qm2pi.qmops} 

\input{qm2pi.sterngerlach} 

\input{qm2pi.metric} 

% section concurrent_process_calculi (end)

%\input{qm2pi.proofsketch}

% section proof sketch (end)

%\input{qm2pi.slviaknots} 

% section spatial logic via knots (end)

\input{qm2pi.conclusion}

% section conclusion (end)

%\input{qm2pi.dtcodes} 

% section wiring algorithm (end)

\input{qm2pi.ack} 

% section acknowledgments (end)

\newpage


\bibliographystyle{plain}   
\bibliography{../../biblios/main.bib}

\input{qm2pi.rhodetails}

\end{document}



% section proof sketch (end)

%\section{Unlikely characters: spatial logic for
  knots}\label{sub:characteristic_formulae} % (fold)

Associated to the mobile process calculi are a family of logics known
as the Hennessy-Milner logics. These logics typically enjoy a
semantics interpreting formulae as sets of processes that when
factored through the encoding outlined above allows an identification
of classes of knots with logical formulae. In the context of this
encoding the sub-family known as the spatial logics \cite{CairesC03}
\cite{CairesC04} \cite{Caires04} are of particular interest providing
several important features for expressing and reasoning about
properties (i.e. classes) of knots. We hint here at how this may be done.

%\begin{description}
%\item [structural connectives] 
\subsubsection{Structural connectives} The spatial logics enjoy
structural connectives corresponding, at the logical level, to the
parallel composition ($P | Q$) and new name ($(\nu \; x)P$)
connectives for processes. As illustrated in the examples below, these
connectives are extremely expressive given the shape of our encoding.
%\item [decideable satisfaction]

\subsubsection{Decideable satisfaction}
In \cite{Caires04} the satisfaction relation is shown to be decideable
for a rich class of processes. It further turns out that the image of
the our encoding is a proper subset of that class. This result
provides the basis for an algorithm by which to search for knots
enjoying a given property.
%\item [characteristic formulae]

\subsubsection{Characteristic formulae}
In the same paper \cite{Caires04} , Caires presents a means of calculating
characteristic formulae, selecting equivalence classes of processes
up to a pre--specified depth limit on the support set of names. Composed with our
encoding, this characteristic formula can be used to select
characteristic formulae for knots.
%\end{description}

\subsubsection{Spatial logic formulae}

The grammar below (segmented for comprehension) summarizes the syntax
of spatial logic formulae. We employ illustrative examples in the
sequel to provide an intuitive understanding of their meaning
referring the reader to \cite{Caires04} for a more detailed explication
of the semantics.

\begin{mathpar}
  \inferrule* [lab=boolean] {} {{A,B} \bc T \;|\; \neg A \;|\; A \wedge B \;|\; \eta = \eta'}
  \and
  \inferrule* [lab=spatial] {} {|\; \pzero \;|\; A | B \;|\; x \text{\textregistered} A \;|\; \forall x . A \;|\;  H x . A}
  \and
  \inferrule* [lab=behavioral] {} {|\; \alpha . A}
  \and 
  \inferrule* [lab=recursion] {} {|\; X(\vec{u}) \;|\; \mu X(\vec{u}) . A}
  \and
  \inferrule* [lab=action] {} {\alpha \bc \langle x?(\vec{y}) \rangle \;|\; \langle x!(\vec{y}) \rangle \;|\; \langle \tau \rangle}
  \and 
  \inferrule* [lab=name] {} {\eta \bc x \;|\; \tau}
\end{mathpar} 

% subsection characteristic_formulae (end)   	 

\subsection{Example formulae}\label{sub:example_formulae_} % (fold)

\subsubsection{Crossing as formula.}
% 
% \begin{align*}
%   \frac{d}{dx} \sin x &= \cos x 
%   & \frac{d}{dx} e^x &= e^x \\
%   \frac{d}{dx} \cos x &= - \sin x 
%   & \frac{d}{dx} \log x &= \frac{1}{x} \\
% \end{align*} 

\begin{align*}
 \mu C(x_{0},x_{1},y_{0},y_{1},u).&(\langle x_{0}?(z) \rangle(\langle u! \rangle\langle y_{1}!z \rangle C(x_{0},x_{1},y_{0},y_{1},u)) & \\
  & \wedge \langle y_{1}?(z) \rangle (\langle u! \rangle \langle x_{0}!z \rangle C(x_{0},x_{1},y_{0},y_{1},u)) & \\
  & \wedge \langle x_{1}?(z) \rangle (\langle u? \rangle \langle y_{0}!z \rangle C(x_{0},x_{1},y_{0},y_{1},u)) & \\
  & \wedge \langle y_{0}?(z) \rangle (\langle u? \rangle \langle x_{1}!z \rangle C(x_{0},x_{1},y_{0},y_{1},u))) &
\end{align*}

The lexicographical similarity between the shape of this formulae and
the shape of definition of the process representing a crossing reveals
the intuitive meaning of this formulae. It describes the capabilities
of a process that has the right to represent a crossing. For example
it picks out processes that may perform an input on the port $x_0$ in
its initial menu of capabilities. What differentiates the formula
from the process, however, is that the crossing process is the
smallest candidate to satisfy the formula. Infinitely many other
processes -- with internal behavior hidden behind this interface, so
to speak -- also satisfy this formula. Even this simple formula,
then, can be seen to open a new view onto knots, providing a
computational interpretation of \emph{virtual} knots.

Note that this formula is derived by hand. A similar formula can be
derived by employing Caires' calculation of characteristic formula
\cite{Caires04} to the process representing a crossing. In light of
this discussion, we let
$\meaningof{C}_{\phi}(x0,x1,y0,y1,u)$ denote a formula specifying the
dynamics we wish to capture of a crossing. To guarantee we preserve
the shape of the interface and minimal semantics we demand that
$\meaningof{C}_{\phi}(x0,x1,y0,y1,u) \Rightarrow
\textbf{C}(x0,x1,y0,y1,u)$ where $\textbf{C}(x0,x1,y0,y1,u)$ denotes
the formula above.
                            
\subsubsection{Crossing number constraints.}
The moral content of the context lemma (Lemma \ref{context}) is that the notion of
``locality'' in the Reidemeister moves is effectively captured by the
parallel composition operator of the process calculus. This intuition
extends through the logic. Given a formula,
$\meaningof{C}_{\phi}(x0,x1,y0,y1,u)$, we can use the structural
connectives to specify constraints on crossing numbers, such as at
least $n$ crossings, or exactly $n$ crossings.
\begin{mathpar}
  \inferrule* [lab=at-least-n] {} { K^{\geq n}_{\phi}(\vec{xs},\vec{ys}) := \Pi_{i=0}^{n-1} Hu . \meaningof{C}_{\phi}(xs_i,ys_i,u) | T }
  \and 
  \inferrule* [lab=exactly-n] {} { K^{= n}_{\phi}(\vec{xs},\vec{ys}) := \Pi_{i=0}^{n-1} Hu . \meaningof{C}_{\phi}(xs_i,ys_i,u) | \neg (\forall x_0,y_0,x_1,y_1,u . \meaningof{C}_{\phi}(x_0,y_0,x_1,y_1,u) | T) }
\end{mathpar}

To round out this section, recall that the encoding of an $n$-crossing
knot decomposes into a parallel composition of $n$ \emph{copies} of a
crossing process together with a wiring harness. To specify different
knot classes with the same crossing number amounts to specifying
logical constraints on the wiring harness. In the interest of space,
we defer examples to a forthcoming paper. Suffice it to say that both
the conditions ``alternating knot'' and ``contains the tangle
corresponding to 5/3'' are expressible. For example, it is possible to
calculate the characteristic formula of a process corresponding to the
tangle 5/3 and conjoin it into the classifying formula via the
composition connective of the logic.

Finally, we wish to observe that it is entirely within reason to
contemplate a more domain-specific version of spatial logic tailored
to the shape of processes in the image of the encoding. Such a
domain-specific logic would have a better claim to the title formal
language of knot properties.

% subsection example_formulae_ (end)

% section knots_as_processes (end) 

% section spatial logic via knots (end)

\section{Conclusions and future work}

\paragraph{Testing physical space}
You, gentle reader, may wonder why of all the theorems to be proved
given this set up we pick the one above. In some sense it's hardly
central to quantum mechanics. We see it as central in the sense that
it firmly establishes a notion of physical space arising from a notion
of the equivalence of behavior. Relating bisimulation to a metric is a
big step forward, but one is faced with interpreting the relationship
of that metric space to something more physical. Quantum mechanical
notions of ``physical'' space are still far from intuitive, but by
relating this idea of distance as testing to calculations that predict
physical circumstances we are making a not insignificant step forward
toward an understanding of the physical space we inhabit as
essentially dynamic.

\paragraph{Effectivity and simulation}
One of the observations we have yet to make is that the entire program
spelled out here is effective. We have built various interpreters for
the reflective calculus at work in this interpretation. In principle,
then, we can simulate quantum mechanics on a computer. The place where
the simulation may lose fidelity is the infinitely branching summation
for the annihilator.

In this connection i also want to point out that the evaluation style
calculation of the inner product puts the non-determinism of the
summation right at the heart of measurement. This suggests that
Milner's original reduction-based formulation of the dynamics of his
calculi in terms of sums was not just notationally suggestive of a
notion of measure-and-continue but captured some significant part of
the physics.

\paragraph{Quantum continuations}
In light of this last observation i want to point out that the
predominant account of quantum mechanics is missing a key aspect of a
truly compositional story of the physical situation. In a real lab,
when a measurement is made the observation can be made to feed into
another device that then makes another measurement conditioned on the
results of the first. This means that after the superposition was
collapsed the entire experimental set up remained in
superposition. While QM offers a means of writing this down it doesn't
quite line up well with the well-trodden formulation of computation
and continuation that we see so succinctly expressed in Milner's
calculi. This suggests that there might be advantages to this account
of dynamics waiting to be explored.

\paragraph{Quantum logic}
In this connection, we also note that by virtue of having the
Hennessy-Milner construction, we can pull the construction through the
interpretation of QM. This gives us a natural candidate for a quantum
logic that enjoys an extremely tight connection with it's domain of
interpretation, making the construction much less ad hoc (rather it is
the image of functor!).

\paragraph{Quantum probabiity}
i have questions about the basis of the interpretation of inner
product as probability amplitude. In particular, using which
axiomatization of probability theory does the notion of probability
amplitude earn the right to be so dubbed? In other words, where is the
proof that the operation for calculating a probability amplitude (and
then squaring) satisfies the axioms of what it means to calculate a
probability? Even if such a proof exists (i have yet to find it in the
literature), i wonder if it might not be possible to turn things on
their heads. Can we view the calculation of the probability amplitude
as an axiomatization of probability? If so, then the definition we
give for calculating probability amplitude may provide the basis for
an \emph{effective} theory of probability.

\paragraph{Quantum vs ``biological'' information}
Finally, i want to conclude with a more philosophical observation. At
a recent workshop in which QM was a predominant topic i noticed
something about quantum information. The speaker was giving a riveting
discussion of axiomatic QM and showing how properties of ``no
cloning'' and ``no deleting'' emerged as consequences of the
axiomatization. Theorems of this form are necessary to give us a sense
of confidence that our axioms characterize the physical theory. What
struck me, though, was that if quantum information is neither erasable
nor replicable it is markedly different from \emph{life}. Two of the
things we know about life is that

\begin{itemize}
  \item it ends;
  \item to gain some measure of persistence, to transcend it's
    finitude it is imminently copyable.
\end{itemize}

Both of these qualities are summarized succinctly in the aphorism: all
flesh is grass. For me these two kinds of ``information'' -- call them
quantum and biological -- are end points on a spectrum of strategies
for persistence. At one end, we have those curious entities that enjoy
uniqueness and permanence; at the other, we have those who in the face
of a certain end and an uncertain present make a go of passing
something on. To me one of the more remarkable aspects of the latter
strategy is that in the presence of noise (and certain features of
copying) we get a kind of dynamism, a chance for improvement against a
given persistent condition.

% subsection other_calculi_other_bisimulations_and_geometry_as_behavior (end)




% section conclusion (end)

%\documentclass[12pt]{llncs}
%\documentclass{jktr}

\usepackage[pdftex]{hyperref}                   
\usepackage {listings}
\usepackage {mathpartir}
\usepackage{bcprules}
%\usepackage{listings}
                       
\usepackage{graphicx} 
%\usepackage[margins=2.5cm,nohead,nofoot]{geometry}
%\usepackage{geometry}
\usepackage{amsfonts}
\usepackage{amstext}
\usepackage{latexsym}
\usepackage{amssymb}
\usepackage{color}


%\include{myPreamble}
\include{qm2pi.local} 

%\ifpdf
%\usepackage[pdftex]{graphicx}
%\else
%\usepackage{graphicx}
%\fi

 % \ifpdf
%  \usepackage{pdfsync}
%  \if


%\title{Brief Article}
%\author{David F. Snyder}
%\author{L.G. Meredith}

%\address{Dept. of Math., Texas State University--San Marcos, San Marcos, TX 78666}
       
\pagestyle{empty}


\begin{document}

\lstset{language=[Objective]Caml,frame=shadowbox}

\input{qm2pi.front}

% section front matter (end)

\input{qm2pi.intro} 
 
% section introduction (end)

% \input{qm2pi.knotations} 

% section notation (end)

\input{qm2pi.process.calculi} 

% section concurrent_process_calculi_and_spatial_logics_ (end)
    
%\input{qm2pi.knots2pi} 

%\input{qm2pi.trefoil} 

%\input{qm2pi.mainthm} 

% subsection basic_interpretation (end)

%\input{qm2pi.rho.presentation} 
\subsection{The syntax and semantics of the notation system}\label{sub:the_syntax_and_semantics_of_the_notation_system} % (fold)

We now summarize a technical presentation of the calculus that
embodies our theory of dynamics. The typical presentation of such a
calculus follows the style of giving generators and relations on
them. The grammar, below, describing term constructors, freely
generates the set of processes, $\Proc$. This set is then quotiented
by a relation known as structural congruence and it is over this set
that the notion of dynamics is expressed. This presentation is
essentially that of \cite{MeredithR05} with the addition of
polyadicity and summation. For readability we have relegated some of
the technical subtleties to an appendix.

\subsubsection{Process grammar}\label{subsub:process_grammar}

\begin{mathpar}
  \inferrule* [lab=synchronization] {} {{M} \bc \pzero \;|\; x?F \;|\; x!C }
  \and
  \inferrule* [lab=abstraction] {} {{F} \bc (x)P}
  \and
  \inferrule* [lab=concretion] {} {{C} \bc \langle Q \rangle}
  \and
  \inferrule* [lab=process] {} {{P,Q} \bc M \;| \;P|Q \;|\; @{x}}
  \and
  \inferrule* [lab=name] {} {{x} \bc \quotep{P}}
\end{mathpar} 

Note that $\vec{x}$ (resp. $\vec{P}$) denotes a vector of names
(resp. processes) of length $|\vec{x}|$ (resp. $|\vec{P}|$). We adopt
the following useful abbreviations.

\begin{mathpar}
   x?(\vec{y}).P := x.(\vec{y})P \and  x\clift{\vec{P}} := x.\clift{\vec{P}}
   \and x!(y) := \lift{x}{\dropn{y}}
   \and \Pi_{i=0}^{n-1}P_i := P_0 | \ldots | P_{n-1}
\end{mathpar}

\subsubsection{Structural congruence}

\paragraph{Free and bound names and alpha-equivalence.} At the
core of structural equivalence is alpha-equivalence which identifies
process that are the same up to a change of variable. Formally, we
recognize the distinction between free and bound names. The free names
of a process, $\freenames{P}$, may be calculated recursively as
follows:

\begin{mathpar}
\freenames{\pzero} := \emptyset
  \and \\
  \freenames{x?(y).P} := \{ x \} \cup (\freenames{P} \setminus \{ y \})
  \and 
  \freenames{x!\langle P \rangle} := \{ x \} \cup \{ P \} 
  \and \\
  \freenames{P|Q} := \freenames{P} \cup \freenames{Q}
  \and \\
  \freenames{@{x}} := \{ x \}
\end{mathpar}

$\pi$
$\quotep{\pi}$

$\freenames{-} : \pi \to \mathcal{P}(\quotep{\pi})$

\begin{eqnarray*}
  \freenames{\pzero} & := & \emptyset \\
  \freenames{x?(y).P} & := & \{ x \} \cup (\freenames{P} \setminus \{ y \}) \\
  \freenames{x!\langle P \rangle} & := & \{ x \} \cup \{ P \} \\
  \freenames{P|Q} & := & \freenames{P} \cup \freenames{Q} \\
  \freenames{\dropn{x}} & := & \{ x \}
\end{eqnarray*}

The bound names of a process, $\boundnames{P}$, are those names occurring in $P$
that are not free. For example, in $x?(y).0$, the name $x$ is free, while $y$ is bound.

\begin{mathpar}
  \inferrule* [lab=monoidal-laws] {} { P|Q \equiv Q|P \and P|0 \equiv P \and P|(Q|R) \equiv (P|Q)|R }
\end{mathpar}

\begin{mathpar}
  \inferrule* [lab=alpha-equivalence] {} { (x)P \equiv (y)P\{y/x\} \and y \not\in \freenames{P} }
\end{mathpar}

\begin{definition}
Then two processes, $P,Q$, are alpha-equivalent if $P = Q\{\vec{y}/\vec{x}\}$ for
some $\vec{x} \in \boundnames{Q},\vec{y} \in \boundnames{P}$, where $Q\{\vec{y}/\vec{x}\}$
denotes the capture-avoiding substitution of $\vec{y}$ for $\vec{x}$ in $Q$.
\end{definition}

\begin{definition}
  The {\em structural congruence} \cite{SangiorgiWalker} , $\equiv$,
  between processes is the least congruence containing
  alpha-equivalence, satisfying the abelian monoid laws
  (associativity, commutativity and $\pzero$ as identity) for parallel
  composition $|$ and for summation $+$.
\end{definition}

\subsection{Name equivalence}

We take name equivalence, written $\nameeq$, to be the smallest
equivalence relation generated by the following rules.

\begin{mathpar}
\inferrule*[lab=Quote-drop]
{ }
{ \quotep{@{x}} \nameeq x }

\inferrule*[lab=Struct-equiv]
{ P \scong Q }
{ \quotep{P} \nameeq \quotep{Q} }
\end{mathpar}

The astute reader will have noticed that the mutual recursion of names
and processes imposes a mutual recursion on alpha-equivalence and
structural equivalence via name-equivalence. Fortunately, all of this
works out pleasantly and we may calculate in the natural way, free of
concern. The reader interested in the details is referred to the
appendix \ref{appendix:rho_details}.

\subsection{Substitution}

We use $\Proc$ for the set of processes, $\QProc$ for the set of
names, and $\id{\{}\vec{y} / \vec{x} \id{\}}$ to denote partial maps,
$s : \QProc \rightarrow \QProc$. A map, $s$ lifts, uniquely, to a map
on process terms, $\widehat{s} : \Proc \rightarrow \Proc$ by the
following equations.

\begin{mathpar}
  (0) \psubstp{Q}{P} := 0 \\
  (R \juxtap S) \psubstp{Q}{P}
  :=    
  (R)\psubstp{Q}{P} \juxtap (S) \psubstp{Q}{P} \\
  (x?(y).R) \psubstp{Q}{P}    
  :=    
  (x)\substp{Q}{P} (z)\concat( (R \psubstn{z}{y}) \psubstp{Q}{P} ) \\
  (\lift{x}{R}) \psubstp{Q}{P}  
  :=
  \lift{(x)\substp{Q}{P}}{ R \psubstp{Q}{P} } \\
%   (\dropn{x})  \psubstp{Q}{P}       
%   := 
%   \left\{ 
%     \begin{array}{ccc} 
%       \dropn{\quotep{Q}} & & x \nameeq \quotep{P} \\
%       \dropn{x} & & otherwise \\
%     \end{array}
%   \right. 
  (\dropn{x})  \psubstp{Q}{P}       
  := 
  \left\{ 
    \begin{array}{ccc} 
      Q & & x \nameeq \quotep{P} \\
      \dropn{x} & & otherwise \\
    \end{array}
  \right.
\end{mathpar}
 

where

\begin{eqnarray}
  (x)\id{\{} \lpquote Q \rpquote / \lpquote P \rpquote \id{\}}            = 
  \left\{ 
    \begin{array}{ccc}
      \lpquote Q \rpquote & & x \nameeq \lpquote P \rpquote \\
      x & & otherwise \\
    \end{array}
  \right. \nonumber
\end{eqnarray}

and $z$ is chosen distinct from $\quotep{P}$, $\quotep{Q}$, the free
names in $Q$, and all the names in $R$. Our $\alpha$-equivalence will
be built in the standard way from this substitution.

\begin{remark}\label{rem:no_self_referential_names}
  One consequence of these definitions is that $\forall P. \quotep{P}
  \not\in \freenames{P}$.
\end{remark}

\subsection{ Dynamic quote: an example }

Anticipating something of what's to come, consider applying the
substitution, $\widehat{\id{\{}u / z \id{\}}}$, to the following pair
of processes, $\lift{w}{y!(z)}$ and $w[ \lpquote y!(z) \rpquote ]$.

\begin{eqnarray}
	\lift{w}{y!(z)}\widehat{\id{\{}u / z \id{\}}}
		& = &
		\lift{w}{y!(u)} \nonumber\\
	w[ \lpquote y!(z) \rpquote ] \widehat{ \id{\{}u / z \id{\}} }
		& = &
		w[ \lpquote y!(z) \rpquote ] \nonumber
\end{eqnarray}

Because the body of the process between quotes is impervious to
substitution, we get radically different answers. In fact, by
examining the first process in an input context,
e.g. $x?(z).\lift{w}{y!(z)}$, we see that the process under the lift
operator may be shaped by prefixed inputs binding a name inside it. In
this sense, the lift operator will be seen as a way to dynamically
construct processes before reifying them as names.

Finally equipped with these standard features we can present the
dynamics of the calculus.

\subsubsection{Operational semantics} 

Finally, we introduce the computational dynamics. What marks these
algebras as distinct from other more traditionally studied algebraic
structures, e.g. vector spaces or polynomial rings, is the manner in
which dynamics is captured. In traditional structures, dynamics is typically
expressed through morphisms between such structures, as in linear maps
between vector spaces or morphisms between rings. In algebras
associated with the semantics of computation, the dynamics is
expressed as part of the algebraic structure itself, through a
reduction reduction relation typically denoted by $\red$. Below, we
give a recursive presentation of this relation for the calculus used
in the encoding.

$\red \subseteq \pi \times \pi$
$\red : \pi \to \mathcal{P}(\pi)$

\begin{mathpar}
  \inferrule* [lab=Comm] { \textsf{match}( x_{src}, x_{trgt} ) } { x_{trgt}?(y)P \; | \; x_{src}!\langle {Q} \rangle \red P\{\quotep{Q}/y}\} }
  \and \\
  \inferrule* [lab=Par] {{P} \red {P}'} {{{P} | {Q}} \red {{P}' | {Q}}}
  \and
  \inferrule* [lab=Equiv]{{{P} \scong {P}'} \andalso {{P}' \red {Q}'} \andalso {{Q}' \scong {Q}}}{{P} \red {Q}}
\end{mathpar}

\begin{eqnarray*}
  match_{\equiv} (\quotep{P},\quotep{Q}) & := & P \equiv Q \\
  match_{\dagger}(\quotep{P},\quotep{Q}) & := & \forall R. P|Q \red^{*} R => R \red^{*} 0 \\
  match_{K}(\quotep{P},\quotep{Q}) & := & K \mbox{ for some context } K
\end{eqnarray*}

$u?(x)P | u!\langle Q \rangle \red P\{\quotep{Q}/x\}$

%We write $\wred$ for $\red^*$, and $P\red$ if $\exists Q $ such that $ P \red Q$.
We write $P\red$ if $\exists Q $ such that $ P \red Q$ and $P\not\red$, otherwise.

\section{Replication}

As mentioned before, it is known that replication (and hence
recursion) can be implemented in a higher-order process algebra
\cite{SangiorgiWalker}. As our first example of calculation with the
machinery thus far presented we give the construction explicitly in
the {\rhoc}.

\begin{eqnarray}
	D_{x} & := & \prefix{x}{y}{(\binpar{\outputp{x}{y}}{@{y}})} \nonumber\\
	\bangp_{x}{P} & := & \binpar{{x}!\langle{\binpar{D_{x}}{P}}\rangle}{D_{x}} \nonumber
\end{eqnarray}

\begin{eqnarray}
	\bangp_{x}{P} & & \nonumber\\
	=
	& {x}!\langle{(\prefix{x}{y}{(\outputp{x}{y} | @{y})) | P}}\rangle 
	      | \prefix{x}{y}{(\outputp{x}{y} | @{y})} & \nonumber\\
	\red
	& (\outputp{x}{y} | @{y})\substn{\quotep{(\prefix{x}{y}{(@{y} | \outputp{x}{y})) | P}}}{y} & \nonumber\\
	=
	& \outputp{x}{\quotep{(\prefix{x}{y}{(\outputp{x}{y} | @{y})) | P}}}
	  | {(\prefix{x}{y}{(\outputp{x}{y} | @{y})) | P}} & \nonumber\\
	\red
	& \ldots & \nonumber\\
	\red^*
	& P | P | \ldots & \nonumber
\end{eqnarray}

Of course, this encoding, as an implementation, runs away, unfolding
$\bangp{P}$ eagerly. A lazier and more implementable replication
operator, restricted to input-guarded processes, may be obtained as follows.

\begin{eqnarray}
\bangp{\prefix{u}{v}{P}} 
	:= 
	\binpar{\lift{x}{\prefix{u}{v}{(\binpar{D(x)}{P})}}}{D(x)} \nonumber
\end{eqnarray}

\begin{remark}
  Note that the lazier definition still does not deal with summation
  or mixed summation (i.e. sums over input and output). The reader is
  invited to construct definitions of replication that deal with these
  features. 

  Further, the definitions are parameterized in a name, $x$. Can you,
  gentle reader, make a definition that eliminates this parameter and
  guarantees no accidental interaction between the replication
  machinery and the process being replicated -- i.e. no accidental
  sharing of names used by the process to get its work done and the
  name(s) used by the replication to effect copying. This latter
  revision of the definition of replication is crucial to obtaining
  the expected identity $!!P \sim !P$.
\end{remark}

\begin{remark}\label{rem:paradoxical_combinator}
  The reader familiar with the lambda calculus will have noticed the
  similarity between $D$ and the paradoxical combinator.

  [Ed. note: the existence of this seems to suggest we have to be more
  restrictive on the set of processes and names we admit if we are to
  support no-cloning.]
\end{remark}

\subsubsection{Bisimulation}

The computational dynamics gives rise to another kind of equivalence,
the equivalence of computational behavior. As previously mentioned
this is typically captured \emph{via} some form of bisimulation.

% The notion we use in this paper is weak barbed bisimulation
% \cite{milner91polyadicpi}.

The notion we use in this paper is derived from weak barbed
bisimulation \cite{milner91polyadicpi}. 

\begin{definition}
An \emph{observation relation}, $\downarrow_{\mathcal N}$, over a set
of names, $\mathcal N$, is the smallest relation satisfying the rules
below.

\infrule[Out-barb]{y \in {\mathcal N}, \; x \nameeq y}
		  {\outputp{x}{v} \downarrow_{\mathcal N} x}
\infrule[Par-barb]{\mbox{$P\downarrow_{\mathcal N} x$ or $Q\downarrow_{\mathcal N} x$}}
		  {\binpar{P}{Q} \downarrow_{\mathcal N} x}

We write $P \Downarrow_{\mathcal N} x$ if there is $Q$ such that 
$P \wred Q$ and $Q \downarrow_{\mathcal N} x$.
\end{definition}

\begin{definition}
%\label{def.bbisim}
An  ${\mathcal N}$-\emph{barbed bisimulation} over a set of names, ${\mathcal N}$, is a symmetric binary relation 
${\mathcal S}_{\mathcal N}$ between agents such that $P\rel{S}_{\mathcal N}Q$ implies:
\begin{enumerate}
\item If $P \red P'$ then $Q \wred Q'$ and $P'\rel{S}_{\mathcal N} Q'$.
\item If $P\downarrow_{\mathcal N} x$, then $Q\Downarrow_{\mathcal N} x$.
\end{enumerate}
$P$ is ${\mathcal N}$-barbed bisimilar to $Q$, written
$P \wbbisim_{\mathcal N} Q$, if $P \rel{S}_{\mathcal N} Q$ for some ${\mathcal N}$-barbed bisimulation ${\mathcal S}_{\mathcal N}$.
\end{definition}

$\mathcal{R} \subseteq \pi \times \pi$

$P \mathcal{R} Q => \forall P'. P \red P' \Rightarrow \exists Q'. Q \red Q', P' \mathcal{R} Q'$

$P \vdash x \Rightarrow Q \vdash x$

\begin{mathpar}
  \inferrule*[lab=Out-barb]{x \nameeq y}{{y}!\langle{Q}\rangle \vdash x}
  \and
  \inferrule*[lab=Par-barb]{\mbox{$P\vdash x$ or $Q\vdash x$}}{\binpar{P}{Q} \vdash x}
\end{mathpar}

\subsubsection{Contexts}

One of the principle advantages of computational calculi like the
$\pi$-calculus is a well-defined notion of context,
contextual-equivalence and a correlation between
contextual-equivalence and notions of bisimulation. The notion of
context allows the decomposition of a process into (sub-)process and
its syntactic environment, its context. Thus, a context may be
thought of as a process with a ``hole'' (written $\Box$) in it. The
application of a context $M$ to a process $P$, written $M[P]$, is
tantamount to filling the hole in $M$ with $P$. In this paper we do
not need the full weight of this theory, but do make use of the notion
of context in the proof the main theorem. 

\begin{mathpar}
  \inferrule* [lab=summation] {} {{M_{M},M_{N}} \bc \Box \;|\; x.M_{A} \;|\; M_{M}+M_{N}}
  \and
  \inferrule* [lab=agent] {} {{M_{A}} \bc (\vec{x})M_{P} \;| \; \clift{P_0,\ldots,M_{P},\ldots,P_N}}
  \and \\
  \inferrule* [lab=process] {} {{M_{P}} \bc M_{N} \;| \;P|M_{P} }
\end{mathpar} 

\begin{mathpar}
  \inferrule* [lab=sychronization] {} {M_{N} \bc \Box \;|\; x?M_{F} \;|\; x!M_{C}}
  \and
  \inferrule* [lab=abstraction] {} {{M_{F}} \bc (x)M_{P} }
  \and
  \inferrule* [lab=concretion] {} {{M_{C}} \bc \langle M_{P} \rangle }
  \and \\
  \inferrule* [lab=process] {} {{M_{P}} \bc M_{N} \;| \;P|M_{P} }
\end{mathpar}

\begin{definition}[contextual application] Given a context $M$, and
  process $P$, we define the \emph{contextual application}, $M[P] :=
  M\{P/\Box\}$. That is, the contextual application of M to P is the
  substitution of $P$ for $\Box$ in $M$.
\end{definition}

$\meaningof{-} : L \to \mathcal{P}(\pi)$

\begin{mathpar}
  \inferrule* [lab=collection] {} {\meaningof{true} = \pi, \and \meaningof{~E} = \pi \setminus \meaningof{E}, \and \meaningof{E_{1} \& E_{2}} = \meaningof{E_{1}} \cap \meaningof{E_{2}}}
\end{mathpar}

\begin{mathpar}
  \inferrule* [lab=structure] {} {\meaningof{0} = \{ P \in \pi | P \equiv 0 \}, \and \\ \meaningof{E_1 | E_2} = \{ P \in \pi | P \equiv P_{1} | P_{2}, P_{1} \in \meaningof{E_{1}}, P_{2} \in \meaningof{E_2}\} }
\end{mathpar}

\begin{mathpar}
 \inferrule* [lab=behavior] {} {\meaningof{\langle a?b \rangle E} = \{ P \in \pi | P \equiv Q | u?(y)P', \\ \and \\\\ \and \\ \;\;\; u \in \meaningof{a}, \forall z.P'\{z/y\} \in \meaningof{E\{z/b\}}\}, \and \\ \meaningof{a!E} = \{ P \in \pi | P \equiv Q | x!\langle P' \rangle, x \in \meaningof{a} P' \in \meaningof{E}\} }
\end{mathpar}

\begin{mathpar}
 \inferrule* [lab=nominal] {} {\meaningof{\quotep{E}} = \{ \quotep{P} \in \quotep{\pi} | P \in \meaningof{E} \}, \and \meaningof{\quotep{P}} = \{ \quotep{Q} \in \quotep{\pi} | P \equiv Q \} \and \\ \meaningof{@\quotep{E}} = \{ P \in \pi | P \equiv @x, x \in \meaningof{E} \}}
\end{mathpar}

\begin{eqnarray*}
  \\
  \meaningof{-} : TS \to ST
\end{eqnarray*}

\begin{eqnarray*}
  \\
  L : TS \to ST
\end{eqnarray*}

\begin{eqnarray*}
  \\
  P \models E \iff P \in \meaningof{E}
\end{eqnarray*}

\begin{eqnarray*}
  P \approx_{L} Q \iff \forall E \in L. P \models E \iff Q \models E
\end{eqnarray*}

\begin{eqnarray*}
  P \approx_{K} Q
\end{eqnarray*}

\begin{eqnarray*}
  P \approx Q
\end{eqnarray*}

$\approx_{K} = \approx = \approx_{L}$

\subsubsection{Contextual duality}

Note that contexts extend the quotation operation to a family of
operations from processes to names. Given a context, $M$, we can
define a \emph{nominal context}, $\quotep{M}$ by $\quotep{M}[P] :=
\quotep{M[P]}$. To foreshadow what is to come we observe that these
operations enjoy a duality with processes very much like the duality
between vectors and maps from vectors to scalars.

Further, because the calculus is essentially higher-order, we have a
correspondence between contexts and processes. More specifically,
given a name $x$ and a context $M$ we can construct $M^{*}_{x}$ such
that 

\begin{mathpar}
  M^{*}_{x} | \lift{x}{P} \red M[P]
\end{mathpar}

namely,

\begin{mathpar}
  M^{*}_{x} := x?(u).M[\dropn{u}]
\end{mathpar}

The dependence of $M^{*}_{x}$ on a name makes it an abstraction, 

\begin{mathpar}
  M^{*} := (x)x?(u).M[\dropn{u}]
\end{mathpar}

\subsection{Additional notation}

It will sometimes be convenient to denote the process a name
quotes. We already have the notation $x = \quotep{P}$, but it will be
convenient to introduce an alternate notation, $\procn{x}$, when we
want to emphasize the connection to the use of the name. Note that, by
virtue of name equivalence, $\quotep{\procn{x}} \nameeq x$; so, the
notation is consistent with previous definitions.

Further, because names have structure it is possible to effect
substitutions on the basis of that structure. This means we need to
upgrade our notation for substitutions, which we accomplish by
adapting comprehension notation. Thus,

\begin{mathpar}
  P\{ y / x : x \in S \}
\end{mathpar}

is interpreted to mean the process derived from P by replacing (in a
capture-avoiding manner) each occurrence of $x$ in $S$ by $y$. For example,

\begin{mathpar}
  P\{ \quotep{\procn{x}|\procn{x}} / x : x \in \freenames{P} \}
\end{mathpar}

will replace each (occurrence) of a free name $x$ in $P$ by
$\quotep{\procn{x}|\procn{x}}$.

Also, we will avail ourselves of the notation $x^{L}$ and $x^{R}$ to
denote injections of a name into disjoint copies of the name
space. There are numerous ways to accomplish this. One example can be
found in \cite{MeredithR05}. This notation overloads to vectors of
names: $\vec{x}^{\pi} := (x_{i}^{\pi} \; : \; 0 \leq i < |\vec{x}| )$ where $\pi \in \{L,R\}$.

We also use $P^{\Box} := P|\Box$.

In \cite{MeredithR05} an interpretation of the new operator is
given. It turns out that there are several possible interpretations
all enjoying the requisite algebraic properties of the operator (see
\cite{milner91polyadicpi}). We will therefore make liberal use of
$(\nu\; \vec{x})P$.

% subsection the_syntax_and_semantics_of_the_notation_system (end)   

\input{qm2pi.qmops} 

\input{qm2pi.sterngerlach} 

\input{qm2pi.metric} 

% section concurrent_process_calculi (end)

%\input{qm2pi.proofsketch}

% section proof sketch (end)

%\input{qm2pi.slviaknots} 

% section spatial logic via knots (end)

\input{qm2pi.conclusion}

% section conclusion (end)

%\input{qm2pi.dtcodes} 

% section wiring algorithm (end)

\input{qm2pi.ack} 

% section acknowledgments (end)

\newpage


\bibliographystyle{plain}   
\bibliography{../../biblios/main.bib}

\input{qm2pi.rhodetails}

\end{document}

 

% section wiring algorithm (end)

\documentclass[12pt]{llncs}
%\documentclass{jktr}

\usepackage[pdftex]{hyperref}                   
\usepackage {listings}
\usepackage {mathpartir}
\usepackage{bcprules}
%\usepackage{listings}
                       
\usepackage{graphicx} 
%\usepackage[margins=2.5cm,nohead,nofoot]{geometry}
%\usepackage{geometry}
\usepackage{amsfonts}
\usepackage{amstext}
\usepackage{latexsym}
\usepackage{amssymb}
\usepackage{color}


%\include{myPreamble}
\include{qm2pi.local} 

%\ifpdf
%\usepackage[pdftex]{graphicx}
%\else
%\usepackage{graphicx}
%\fi

 % \ifpdf
%  \usepackage{pdfsync}
%  \if


%\title{Brief Article}
%\author{David F. Snyder}
%\author{L.G. Meredith}

%\address{Dept. of Math., Texas State University--San Marcos, San Marcos, TX 78666}
       
\pagestyle{empty}


\begin{document}

\lstset{language=[Objective]Caml,frame=shadowbox}

\input{qm2pi.front}

% section front matter (end)

\input{qm2pi.intro} 
 
% section introduction (end)

% \input{qm2pi.knotations} 

% section notation (end)

\input{qm2pi.process.calculi} 

% section concurrent_process_calculi_and_spatial_logics_ (end)
    
%\input{qm2pi.knots2pi} 

%\input{qm2pi.trefoil} 

%\input{qm2pi.mainthm} 

% subsection basic_interpretation (end)

%\input{qm2pi.rho.presentation} 
\subsection{The syntax and semantics of the notation system}\label{sub:the_syntax_and_semantics_of_the_notation_system} % (fold)

We now summarize a technical presentation of the calculus that
embodies our theory of dynamics. The typical presentation of such a
calculus follows the style of giving generators and relations on
them. The grammar, below, describing term constructors, freely
generates the set of processes, $\Proc$. This set is then quotiented
by a relation known as structural congruence and it is over this set
that the notion of dynamics is expressed. This presentation is
essentially that of \cite{MeredithR05} with the addition of
polyadicity and summation. For readability we have relegated some of
the technical subtleties to an appendix.

\subsubsection{Process grammar}\label{subsub:process_grammar}

\begin{mathpar}
  \inferrule* [lab=synchronization] {} {{M} \bc \pzero \;|\; x?F \;|\; x!C }
  \and
  \inferrule* [lab=abstraction] {} {{F} \bc (x)P}
  \and
  \inferrule* [lab=concretion] {} {{C} \bc \langle Q \rangle}
  \and
  \inferrule* [lab=process] {} {{P,Q} \bc M \;| \;P|Q \;|\; @{x}}
  \and
  \inferrule* [lab=name] {} {{x} \bc \quotep{P}}
\end{mathpar} 

Note that $\vec{x}$ (resp. $\vec{P}$) denotes a vector of names
(resp. processes) of length $|\vec{x}|$ (resp. $|\vec{P}|$). We adopt
the following useful abbreviations.

\begin{mathpar}
   x?(\vec{y}).P := x.(\vec{y})P \and  x\clift{\vec{P}} := x.\clift{\vec{P}}
   \and x!(y) := \lift{x}{\dropn{y}}
   \and \Pi_{i=0}^{n-1}P_i := P_0 | \ldots | P_{n-1}
\end{mathpar}

\subsubsection{Structural congruence}

\paragraph{Free and bound names and alpha-equivalence.} At the
core of structural equivalence is alpha-equivalence which identifies
process that are the same up to a change of variable. Formally, we
recognize the distinction between free and bound names. The free names
of a process, $\freenames{P}$, may be calculated recursively as
follows:

\begin{mathpar}
\freenames{\pzero} := \emptyset
  \and \\
  \freenames{x?(y).P} := \{ x \} \cup (\freenames{P} \setminus \{ y \})
  \and 
  \freenames{x!\langle P \rangle} := \{ x \} \cup \{ P \} 
  \and \\
  \freenames{P|Q} := \freenames{P} \cup \freenames{Q}
  \and \\
  \freenames{@{x}} := \{ x \}
\end{mathpar}

$\pi$
$\quotep{\pi}$

$\freenames{-} : \pi \to \mathcal{P}(\quotep{\pi})$

\begin{eqnarray*}
  \freenames{\pzero} & := & \emptyset \\
  \freenames{x?(y).P} & := & \{ x \} \cup (\freenames{P} \setminus \{ y \}) \\
  \freenames{x!\langle P \rangle} & := & \{ x \} \cup \{ P \} \\
  \freenames{P|Q} & := & \freenames{P} \cup \freenames{Q} \\
  \freenames{\dropn{x}} & := & \{ x \}
\end{eqnarray*}

The bound names of a process, $\boundnames{P}$, are those names occurring in $P$
that are not free. For example, in $x?(y).0$, the name $x$ is free, while $y$ is bound.

\begin{mathpar}
  \inferrule* [lab=monoidal-laws] {} { P|Q \equiv Q|P \and P|0 \equiv P \and P|(Q|R) \equiv (P|Q)|R }
\end{mathpar}

\begin{mathpar}
  \inferrule* [lab=alpha-equivalence] {} { (x)P \equiv (y)P\{y/x\} \and y \not\in \freenames{P} }
\end{mathpar}

\begin{definition}
Then two processes, $P,Q$, are alpha-equivalent if $P = Q\{\vec{y}/\vec{x}\}$ for
some $\vec{x} \in \boundnames{Q},\vec{y} \in \boundnames{P}$, where $Q\{\vec{y}/\vec{x}\}$
denotes the capture-avoiding substitution of $\vec{y}$ for $\vec{x}$ in $Q$.
\end{definition}

\begin{definition}
  The {\em structural congruence} \cite{SangiorgiWalker} , $\equiv$,
  between processes is the least congruence containing
  alpha-equivalence, satisfying the abelian monoid laws
  (associativity, commutativity and $\pzero$ as identity) for parallel
  composition $|$ and for summation $+$.
\end{definition}

\subsection{Name equivalence}

We take name equivalence, written $\nameeq$, to be the smallest
equivalence relation generated by the following rules.

\begin{mathpar}
\inferrule*[lab=Quote-drop]
{ }
{ \quotep{@{x}} \nameeq x }

\inferrule*[lab=Struct-equiv]
{ P \scong Q }
{ \quotep{P} \nameeq \quotep{Q} }
\end{mathpar}

The astute reader will have noticed that the mutual recursion of names
and processes imposes a mutual recursion on alpha-equivalence and
structural equivalence via name-equivalence. Fortunately, all of this
works out pleasantly and we may calculate in the natural way, free of
concern. The reader interested in the details is referred to the
appendix \ref{appendix:rho_details}.

\subsection{Substitution}

We use $\Proc$ for the set of processes, $\QProc$ for the set of
names, and $\id{\{}\vec{y} / \vec{x} \id{\}}$ to denote partial maps,
$s : \QProc \rightarrow \QProc$. A map, $s$ lifts, uniquely, to a map
on process terms, $\widehat{s} : \Proc \rightarrow \Proc$ by the
following equations.

\begin{mathpar}
  (0) \psubstp{Q}{P} := 0 \\
  (R \juxtap S) \psubstp{Q}{P}
  :=    
  (R)\psubstp{Q}{P} \juxtap (S) \psubstp{Q}{P} \\
  (x?(y).R) \psubstp{Q}{P}    
  :=    
  (x)\substp{Q}{P} (z)\concat( (R \psubstn{z}{y}) \psubstp{Q}{P} ) \\
  (\lift{x}{R}) \psubstp{Q}{P}  
  :=
  \lift{(x)\substp{Q}{P}}{ R \psubstp{Q}{P} } \\
%   (\dropn{x})  \psubstp{Q}{P}       
%   := 
%   \left\{ 
%     \begin{array}{ccc} 
%       \dropn{\quotep{Q}} & & x \nameeq \quotep{P} \\
%       \dropn{x} & & otherwise \\
%     \end{array}
%   \right. 
  (\dropn{x})  \psubstp{Q}{P}       
  := 
  \left\{ 
    \begin{array}{ccc} 
      Q & & x \nameeq \quotep{P} \\
      \dropn{x} & & otherwise \\
    \end{array}
  \right.
\end{mathpar}
 

where

\begin{eqnarray}
  (x)\id{\{} \lpquote Q \rpquote / \lpquote P \rpquote \id{\}}            = 
  \left\{ 
    \begin{array}{ccc}
      \lpquote Q \rpquote & & x \nameeq \lpquote P \rpquote \\
      x & & otherwise \\
    \end{array}
  \right. \nonumber
\end{eqnarray}

and $z$ is chosen distinct from $\quotep{P}$, $\quotep{Q}$, the free
names in $Q$, and all the names in $R$. Our $\alpha$-equivalence will
be built in the standard way from this substitution.

\begin{remark}\label{rem:no_self_referential_names}
  One consequence of these definitions is that $\forall P. \quotep{P}
  \not\in \freenames{P}$.
\end{remark}

\subsection{ Dynamic quote: an example }

Anticipating something of what's to come, consider applying the
substitution, $\widehat{\id{\{}u / z \id{\}}}$, to the following pair
of processes, $\lift{w}{y!(z)}$ and $w[ \lpquote y!(z) \rpquote ]$.

\begin{eqnarray}
	\lift{w}{y!(z)}\widehat{\id{\{}u / z \id{\}}}
		& = &
		\lift{w}{y!(u)} \nonumber\\
	w[ \lpquote y!(z) \rpquote ] \widehat{ \id{\{}u / z \id{\}} }
		& = &
		w[ \lpquote y!(z) \rpquote ] \nonumber
\end{eqnarray}

Because the body of the process between quotes is impervious to
substitution, we get radically different answers. In fact, by
examining the first process in an input context,
e.g. $x?(z).\lift{w}{y!(z)}$, we see that the process under the lift
operator may be shaped by prefixed inputs binding a name inside it. In
this sense, the lift operator will be seen as a way to dynamically
construct processes before reifying them as names.

Finally equipped with these standard features we can present the
dynamics of the calculus.

\subsubsection{Operational semantics} 

Finally, we introduce the computational dynamics. What marks these
algebras as distinct from other more traditionally studied algebraic
structures, e.g. vector spaces or polynomial rings, is the manner in
which dynamics is captured. In traditional structures, dynamics is typically
expressed through morphisms between such structures, as in linear maps
between vector spaces or morphisms between rings. In algebras
associated with the semantics of computation, the dynamics is
expressed as part of the algebraic structure itself, through a
reduction reduction relation typically denoted by $\red$. Below, we
give a recursive presentation of this relation for the calculus used
in the encoding.

$\red \subseteq \pi \times \pi$
$\red : \pi \to \mathcal{P}(\pi)$

\begin{mathpar}
  \inferrule* [lab=Comm] { \textsf{match}( x_{src}, x_{trgt} ) } { x_{trgt}?(y)P \; | \; x_{src}!\langle {Q} \rangle \red P\{\quotep{Q}/y}\} }
  \and \\
  \inferrule* [lab=Par] {{P} \red {P}'} {{{P} | {Q}} \red {{P}' | {Q}}}
  \and
  \inferrule* [lab=Equiv]{{{P} \scong {P}'} \andalso {{P}' \red {Q}'} \andalso {{Q}' \scong {Q}}}{{P} \red {Q}}
\end{mathpar}

\begin{eqnarray*}
  match_{\equiv} (\quotep{P},\quotep{Q}) & := & P \equiv Q \\
  match_{\dagger}(\quotep{P},\quotep{Q}) & := & \forall R. P|Q \red^{*} R => R \red^{*} 0 \\
  match_{K}(\quotep{P},\quotep{Q}) & := & K \mbox{ for some context } K
\end{eqnarray*}

$u?(x)P | u!\langle Q \rangle \red P\{\quotep{Q}/x\}$

%We write $\wred$ for $\red^*$, and $P\red$ if $\exists Q $ such that $ P \red Q$.
We write $P\red$ if $\exists Q $ such that $ P \red Q$ and $P\not\red$, otherwise.

\section{Replication}

As mentioned before, it is known that replication (and hence
recursion) can be implemented in a higher-order process algebra
\cite{SangiorgiWalker}. As our first example of calculation with the
machinery thus far presented we give the construction explicitly in
the {\rhoc}.

\begin{eqnarray}
	D_{x} & := & \prefix{x}{y}{(\binpar{\outputp{x}{y}}{@{y}})} \nonumber\\
	\bangp_{x}{P} & := & \binpar{{x}!\langle{\binpar{D_{x}}{P}}\rangle}{D_{x}} \nonumber
\end{eqnarray}

\begin{eqnarray}
	\bangp_{x}{P} & & \nonumber\\
	=
	& {x}!\langle{(\prefix{x}{y}{(\outputp{x}{y} | @{y})) | P}}\rangle 
	      | \prefix{x}{y}{(\outputp{x}{y} | @{y})} & \nonumber\\
	\red
	& (\outputp{x}{y} | @{y})\substn{\quotep{(\prefix{x}{y}{(@{y} | \outputp{x}{y})) | P}}}{y} & \nonumber\\
	=
	& \outputp{x}{\quotep{(\prefix{x}{y}{(\outputp{x}{y} | @{y})) | P}}}
	  | {(\prefix{x}{y}{(\outputp{x}{y} | @{y})) | P}} & \nonumber\\
	\red
	& \ldots & \nonumber\\
	\red^*
	& P | P | \ldots & \nonumber
\end{eqnarray}

Of course, this encoding, as an implementation, runs away, unfolding
$\bangp{P}$ eagerly. A lazier and more implementable replication
operator, restricted to input-guarded processes, may be obtained as follows.

\begin{eqnarray}
\bangp{\prefix{u}{v}{P}} 
	:= 
	\binpar{\lift{x}{\prefix{u}{v}{(\binpar{D(x)}{P})}}}{D(x)} \nonumber
\end{eqnarray}

\begin{remark}
  Note that the lazier definition still does not deal with summation
  or mixed summation (i.e. sums over input and output). The reader is
  invited to construct definitions of replication that deal with these
  features. 

  Further, the definitions are parameterized in a name, $x$. Can you,
  gentle reader, make a definition that eliminates this parameter and
  guarantees no accidental interaction between the replication
  machinery and the process being replicated -- i.e. no accidental
  sharing of names used by the process to get its work done and the
  name(s) used by the replication to effect copying. This latter
  revision of the definition of replication is crucial to obtaining
  the expected identity $!!P \sim !P$.
\end{remark}

\begin{remark}\label{rem:paradoxical_combinator}
  The reader familiar with the lambda calculus will have noticed the
  similarity between $D$ and the paradoxical combinator.

  [Ed. note: the existence of this seems to suggest we have to be more
  restrictive on the set of processes and names we admit if we are to
  support no-cloning.]
\end{remark}

\subsubsection{Bisimulation}

The computational dynamics gives rise to another kind of equivalence,
the equivalence of computational behavior. As previously mentioned
this is typically captured \emph{via} some form of bisimulation.

% The notion we use in this paper is weak barbed bisimulation
% \cite{milner91polyadicpi}.

The notion we use in this paper is derived from weak barbed
bisimulation \cite{milner91polyadicpi}. 

\begin{definition}
An \emph{observation relation}, $\downarrow_{\mathcal N}$, over a set
of names, $\mathcal N$, is the smallest relation satisfying the rules
below.

\infrule[Out-barb]{y \in {\mathcal N}, \; x \nameeq y}
		  {\outputp{x}{v} \downarrow_{\mathcal N} x}
\infrule[Par-barb]{\mbox{$P\downarrow_{\mathcal N} x$ or $Q\downarrow_{\mathcal N} x$}}
		  {\binpar{P}{Q} \downarrow_{\mathcal N} x}

We write $P \Downarrow_{\mathcal N} x$ if there is $Q$ such that 
$P \wred Q$ and $Q \downarrow_{\mathcal N} x$.
\end{definition}

\begin{definition}
%\label{def.bbisim}
An  ${\mathcal N}$-\emph{barbed bisimulation} over a set of names, ${\mathcal N}$, is a symmetric binary relation 
${\mathcal S}_{\mathcal N}$ between agents such that $P\rel{S}_{\mathcal N}Q$ implies:
\begin{enumerate}
\item If $P \red P'$ then $Q \wred Q'$ and $P'\rel{S}_{\mathcal N} Q'$.
\item If $P\downarrow_{\mathcal N} x$, then $Q\Downarrow_{\mathcal N} x$.
\end{enumerate}
$P$ is ${\mathcal N}$-barbed bisimilar to $Q$, written
$P \wbbisim_{\mathcal N} Q$, if $P \rel{S}_{\mathcal N} Q$ for some ${\mathcal N}$-barbed bisimulation ${\mathcal S}_{\mathcal N}$.
\end{definition}

$\mathcal{R} \subseteq \pi \times \pi$

$P \mathcal{R} Q => \forall P'. P \red P' \Rightarrow \exists Q'. Q \red Q', P' \mathcal{R} Q'$

$P \vdash x \Rightarrow Q \vdash x$

\begin{mathpar}
  \inferrule*[lab=Out-barb]{x \nameeq y}{{y}!\langle{Q}\rangle \vdash x}
  \and
  \inferrule*[lab=Par-barb]{\mbox{$P\vdash x$ or $Q\vdash x$}}{\binpar{P}{Q} \vdash x}
\end{mathpar}

\subsubsection{Contexts}

One of the principle advantages of computational calculi like the
$\pi$-calculus is a well-defined notion of context,
contextual-equivalence and a correlation between
contextual-equivalence and notions of bisimulation. The notion of
context allows the decomposition of a process into (sub-)process and
its syntactic environment, its context. Thus, a context may be
thought of as a process with a ``hole'' (written $\Box$) in it. The
application of a context $M$ to a process $P$, written $M[P]$, is
tantamount to filling the hole in $M$ with $P$. In this paper we do
not need the full weight of this theory, but do make use of the notion
of context in the proof the main theorem. 

\begin{mathpar}
  \inferrule* [lab=summation] {} {{M_{M},M_{N}} \bc \Box \;|\; x.M_{A} \;|\; M_{M}+M_{N}}
  \and
  \inferrule* [lab=agent] {} {{M_{A}} \bc (\vec{x})M_{P} \;| \; \clift{P_0,\ldots,M_{P},\ldots,P_N}}
  \and \\
  \inferrule* [lab=process] {} {{M_{P}} \bc M_{N} \;| \;P|M_{P} }
\end{mathpar} 

\begin{mathpar}
  \inferrule* [lab=sychronization] {} {M_{N} \bc \Box \;|\; x?M_{F} \;|\; x!M_{C}}
  \and
  \inferrule* [lab=abstraction] {} {{M_{F}} \bc (x)M_{P} }
  \and
  \inferrule* [lab=concretion] {} {{M_{C}} \bc \langle M_{P} \rangle }
  \and \\
  \inferrule* [lab=process] {} {{M_{P}} \bc M_{N} \;| \;P|M_{P} }
\end{mathpar}

\begin{definition}[contextual application] Given a context $M$, and
  process $P$, we define the \emph{contextual application}, $M[P] :=
  M\{P/\Box\}$. That is, the contextual application of M to P is the
  substitution of $P$ for $\Box$ in $M$.
\end{definition}

$\meaningof{-} : L \to \mathcal{P}(\pi)$

\begin{mathpar}
  \inferrule* [lab=collection] {} {\meaningof{true} = \pi, \and \meaningof{~E} = \pi \setminus \meaningof{E}, \and \meaningof{E_{1} \& E_{2}} = \meaningof{E_{1}} \cap \meaningof{E_{2}}}
\end{mathpar}

\begin{mathpar}
  \inferrule* [lab=structure] {} {\meaningof{0} = \{ P \in \pi | P \equiv 0 \}, \and \\ \meaningof{E_1 | E_2} = \{ P \in \pi | P \equiv P_{1} | P_{2}, P_{1} \in \meaningof{E_{1}}, P_{2} \in \meaningof{E_2}\} }
\end{mathpar}

\begin{mathpar}
 \inferrule* [lab=behavior] {} {\meaningof{\langle a?b \rangle E} = \{ P \in \pi | P \equiv Q | u?(y)P', \\ \and \\\\ \and \\ \;\;\; u \in \meaningof{a}, \forall z.P'\{z/y\} \in \meaningof{E\{z/b\}}\}, \and \\ \meaningof{a!E} = \{ P \in \pi | P \equiv Q | x!\langle P' \rangle, x \in \meaningof{a} P' \in \meaningof{E}\} }
\end{mathpar}

\begin{mathpar}
 \inferrule* [lab=nominal] {} {\meaningof{\quotep{E}} = \{ \quotep{P} \in \quotep{\pi} | P \in \meaningof{E} \}, \and \meaningof{\quotep{P}} = \{ \quotep{Q} \in \quotep{\pi} | P \equiv Q \} \and \\ \meaningof{@\quotep{E}} = \{ P \in \pi | P \equiv @x, x \in \meaningof{E} \}}
\end{mathpar}

\begin{eqnarray*}
  \\
  \meaningof{-} : TS \to ST
\end{eqnarray*}

\begin{eqnarray*}
  \\
  L : TS \to ST
\end{eqnarray*}

\begin{eqnarray*}
  \\
  P \models E \iff P \in \meaningof{E}
\end{eqnarray*}

\begin{eqnarray*}
  P \approx_{L} Q \iff \forall E \in L. P \models E \iff Q \models E
\end{eqnarray*}

\begin{eqnarray*}
  P \approx_{K} Q
\end{eqnarray*}

\begin{eqnarray*}
  P \approx Q
\end{eqnarray*}

$\approx_{K} = \approx = \approx_{L}$

\subsubsection{Contextual duality}

Note that contexts extend the quotation operation to a family of
operations from processes to names. Given a context, $M$, we can
define a \emph{nominal context}, $\quotep{M}$ by $\quotep{M}[P] :=
\quotep{M[P]}$. To foreshadow what is to come we observe that these
operations enjoy a duality with processes very much like the duality
between vectors and maps from vectors to scalars.

Further, because the calculus is essentially higher-order, we have a
correspondence between contexts and processes. More specifically,
given a name $x$ and a context $M$ we can construct $M^{*}_{x}$ such
that 

\begin{mathpar}
  M^{*}_{x} | \lift{x}{P} \red M[P]
\end{mathpar}

namely,

\begin{mathpar}
  M^{*}_{x} := x?(u).M[\dropn{u}]
\end{mathpar}

The dependence of $M^{*}_{x}$ on a name makes it an abstraction, 

\begin{mathpar}
  M^{*} := (x)x?(u).M[\dropn{u}]
\end{mathpar}

\subsection{Additional notation}

It will sometimes be convenient to denote the process a name
quotes. We already have the notation $x = \quotep{P}$, but it will be
convenient to introduce an alternate notation, $\procn{x}$, when we
want to emphasize the connection to the use of the name. Note that, by
virtue of name equivalence, $\quotep{\procn{x}} \nameeq x$; so, the
notation is consistent with previous definitions.

Further, because names have structure it is possible to effect
substitutions on the basis of that structure. This means we need to
upgrade our notation for substitutions, which we accomplish by
adapting comprehension notation. Thus,

\begin{mathpar}
  P\{ y / x : x \in S \}
\end{mathpar}

is interpreted to mean the process derived from P by replacing (in a
capture-avoiding manner) each occurrence of $x$ in $S$ by $y$. For example,

\begin{mathpar}
  P\{ \quotep{\procn{x}|\procn{x}} / x : x \in \freenames{P} \}
\end{mathpar}

will replace each (occurrence) of a free name $x$ in $P$ by
$\quotep{\procn{x}|\procn{x}}$.

Also, we will avail ourselves of the notation $x^{L}$ and $x^{R}$ to
denote injections of a name into disjoint copies of the name
space. There are numerous ways to accomplish this. One example can be
found in \cite{MeredithR05}. This notation overloads to vectors of
names: $\vec{x}^{\pi} := (x_{i}^{\pi} \; : \; 0 \leq i < |\vec{x}| )$ where $\pi \in \{L,R\}$.

We also use $P^{\Box} := P|\Box$.

In \cite{MeredithR05} an interpretation of the new operator is
given. It turns out that there are several possible interpretations
all enjoying the requisite algebraic properties of the operator (see
\cite{milner91polyadicpi}). We will therefore make liberal use of
$(\nu\; \vec{x})P$.

% subsection the_syntax_and_semantics_of_the_notation_system (end)   

\input{qm2pi.qmops} 

\input{qm2pi.sterngerlach} 

\input{qm2pi.metric} 

% section concurrent_process_calculi (end)

%\input{qm2pi.proofsketch}

% section proof sketch (end)

%\input{qm2pi.slviaknots} 

% section spatial logic via knots (end)

\input{qm2pi.conclusion}

% section conclusion (end)

%\input{qm2pi.dtcodes} 

% section wiring algorithm (end)

\input{qm2pi.ack} 

% section acknowledgments (end)

\newpage


\bibliographystyle{plain}   
\bibliography{../../biblios/main.bib}

\input{qm2pi.rhodetails}

\end{document}

 

% section acknowledgments (end)

\newpage


\bibliographystyle{plain}   
\bibliography{../../biblios/main.bib}

\documentclass[12pt]{llncs}
%\documentclass{jktr}

\usepackage[pdftex]{hyperref}                   
\usepackage {listings}
\usepackage {mathpartir}
\usepackage{bcprules}
%\usepackage{listings}
                       
\usepackage{graphicx} 
%\usepackage[margins=2.5cm,nohead,nofoot]{geometry}
%\usepackage{geometry}
\usepackage{amsfonts}
\usepackage{amstext}
\usepackage{latexsym}
\usepackage{amssymb}
\usepackage{color}


%\include{myPreamble}
\include{qm2pi.local} 

%\ifpdf
%\usepackage[pdftex]{graphicx}
%\else
%\usepackage{graphicx}
%\fi

 % \ifpdf
%  \usepackage{pdfsync}
%  \if


%\title{Brief Article}
%\author{David F. Snyder}
%\author{L.G. Meredith}

%\address{Dept. of Math., Texas State University--San Marcos, San Marcos, TX 78666}
       
\pagestyle{empty}


\begin{document}

\lstset{language=[Objective]Caml,frame=shadowbox}

\input{qm2pi.front}

% section front matter (end)

\input{qm2pi.intro} 
 
% section introduction (end)

% \input{qm2pi.knotations} 

% section notation (end)

\input{qm2pi.process.calculi} 

% section concurrent_process_calculi_and_spatial_logics_ (end)
    
%\input{qm2pi.knots2pi} 

%\input{qm2pi.trefoil} 

%\input{qm2pi.mainthm} 

% subsection basic_interpretation (end)

%\input{qm2pi.rho.presentation} 
\subsection{The syntax and semantics of the notation system}\label{sub:the_syntax_and_semantics_of_the_notation_system} % (fold)

We now summarize a technical presentation of the calculus that
embodies our theory of dynamics. The typical presentation of such a
calculus follows the style of giving generators and relations on
them. The grammar, below, describing term constructors, freely
generates the set of processes, $\Proc$. This set is then quotiented
by a relation known as structural congruence and it is over this set
that the notion of dynamics is expressed. This presentation is
essentially that of \cite{MeredithR05} with the addition of
polyadicity and summation. For readability we have relegated some of
the technical subtleties to an appendix.

\subsubsection{Process grammar}\label{subsub:process_grammar}

\begin{mathpar}
  \inferrule* [lab=synchronization] {} {{M} \bc \pzero \;|\; x?F \;|\; x!C }
  \and
  \inferrule* [lab=abstraction] {} {{F} \bc (x)P}
  \and
  \inferrule* [lab=concretion] {} {{C} \bc \langle Q \rangle}
  \and
  \inferrule* [lab=process] {} {{P,Q} \bc M \;| \;P|Q \;|\; @{x}}
  \and
  \inferrule* [lab=name] {} {{x} \bc \quotep{P}}
\end{mathpar} 

Note that $\vec{x}$ (resp. $\vec{P}$) denotes a vector of names
(resp. processes) of length $|\vec{x}|$ (resp. $|\vec{P}|$). We adopt
the following useful abbreviations.

\begin{mathpar}
   x?(\vec{y}).P := x.(\vec{y})P \and  x\clift{\vec{P}} := x.\clift{\vec{P}}
   \and x!(y) := \lift{x}{\dropn{y}}
   \and \Pi_{i=0}^{n-1}P_i := P_0 | \ldots | P_{n-1}
\end{mathpar}

\subsubsection{Structural congruence}

\paragraph{Free and bound names and alpha-equivalence.} At the
core of structural equivalence is alpha-equivalence which identifies
process that are the same up to a change of variable. Formally, we
recognize the distinction between free and bound names. The free names
of a process, $\freenames{P}$, may be calculated recursively as
follows:

\begin{mathpar}
\freenames{\pzero} := \emptyset
  \and \\
  \freenames{x?(y).P} := \{ x \} \cup (\freenames{P} \setminus \{ y \})
  \and 
  \freenames{x!\langle P \rangle} := \{ x \} \cup \{ P \} 
  \and \\
  \freenames{P|Q} := \freenames{P} \cup \freenames{Q}
  \and \\
  \freenames{@{x}} := \{ x \}
\end{mathpar}

$\pi$
$\quotep{\pi}$

$\freenames{-} : \pi \to \mathcal{P}(\quotep{\pi})$

\begin{eqnarray*}
  \freenames{\pzero} & := & \emptyset \\
  \freenames{x?(y).P} & := & \{ x \} \cup (\freenames{P} \setminus \{ y \}) \\
  \freenames{x!\langle P \rangle} & := & \{ x \} \cup \{ P \} \\
  \freenames{P|Q} & := & \freenames{P} \cup \freenames{Q} \\
  \freenames{\dropn{x}} & := & \{ x \}
\end{eqnarray*}

The bound names of a process, $\boundnames{P}$, are those names occurring in $P$
that are not free. For example, in $x?(y).0$, the name $x$ is free, while $y$ is bound.

\begin{mathpar}
  \inferrule* [lab=monoidal-laws] {} { P|Q \equiv Q|P \and P|0 \equiv P \and P|(Q|R) \equiv (P|Q)|R }
\end{mathpar}

\begin{mathpar}
  \inferrule* [lab=alpha-equivalence] {} { (x)P \equiv (y)P\{y/x\} \and y \not\in \freenames{P} }
\end{mathpar}

\begin{definition}
Then two processes, $P,Q$, are alpha-equivalent if $P = Q\{\vec{y}/\vec{x}\}$ for
some $\vec{x} \in \boundnames{Q},\vec{y} \in \boundnames{P}$, where $Q\{\vec{y}/\vec{x}\}$
denotes the capture-avoiding substitution of $\vec{y}$ for $\vec{x}$ in $Q$.
\end{definition}

\begin{definition}
  The {\em structural congruence} \cite{SangiorgiWalker} , $\equiv$,
  between processes is the least congruence containing
  alpha-equivalence, satisfying the abelian monoid laws
  (associativity, commutativity and $\pzero$ as identity) for parallel
  composition $|$ and for summation $+$.
\end{definition}

\subsection{Name equivalence}

We take name equivalence, written $\nameeq$, to be the smallest
equivalence relation generated by the following rules.

\begin{mathpar}
\inferrule*[lab=Quote-drop]
{ }
{ \quotep{@{x}} \nameeq x }

\inferrule*[lab=Struct-equiv]
{ P \scong Q }
{ \quotep{P} \nameeq \quotep{Q} }
\end{mathpar}

The astute reader will have noticed that the mutual recursion of names
and processes imposes a mutual recursion on alpha-equivalence and
structural equivalence via name-equivalence. Fortunately, all of this
works out pleasantly and we may calculate in the natural way, free of
concern. The reader interested in the details is referred to the
appendix \ref{appendix:rho_details}.

\subsection{Substitution}

We use $\Proc$ for the set of processes, $\QProc$ for the set of
names, and $\id{\{}\vec{y} / \vec{x} \id{\}}$ to denote partial maps,
$s : \QProc \rightarrow \QProc$. A map, $s$ lifts, uniquely, to a map
on process terms, $\widehat{s} : \Proc \rightarrow \Proc$ by the
following equations.

\begin{mathpar}
  (0) \psubstp{Q}{P} := 0 \\
  (R \juxtap S) \psubstp{Q}{P}
  :=    
  (R)\psubstp{Q}{P} \juxtap (S) \psubstp{Q}{P} \\
  (x?(y).R) \psubstp{Q}{P}    
  :=    
  (x)\substp{Q}{P} (z)\concat( (R \psubstn{z}{y}) \psubstp{Q}{P} ) \\
  (\lift{x}{R}) \psubstp{Q}{P}  
  :=
  \lift{(x)\substp{Q}{P}}{ R \psubstp{Q}{P} } \\
%   (\dropn{x})  \psubstp{Q}{P}       
%   := 
%   \left\{ 
%     \begin{array}{ccc} 
%       \dropn{\quotep{Q}} & & x \nameeq \quotep{P} \\
%       \dropn{x} & & otherwise \\
%     \end{array}
%   \right. 
  (\dropn{x})  \psubstp{Q}{P}       
  := 
  \left\{ 
    \begin{array}{ccc} 
      Q & & x \nameeq \quotep{P} \\
      \dropn{x} & & otherwise \\
    \end{array}
  \right.
\end{mathpar}
 

where

\begin{eqnarray}
  (x)\id{\{} \lpquote Q \rpquote / \lpquote P \rpquote \id{\}}            = 
  \left\{ 
    \begin{array}{ccc}
      \lpquote Q \rpquote & & x \nameeq \lpquote P \rpquote \\
      x & & otherwise \\
    \end{array}
  \right. \nonumber
\end{eqnarray}

and $z$ is chosen distinct from $\quotep{P}$, $\quotep{Q}$, the free
names in $Q$, and all the names in $R$. Our $\alpha$-equivalence will
be built in the standard way from this substitution.

\begin{remark}\label{rem:no_self_referential_names}
  One consequence of these definitions is that $\forall P. \quotep{P}
  \not\in \freenames{P}$.
\end{remark}

\subsection{ Dynamic quote: an example }

Anticipating something of what's to come, consider applying the
substitution, $\widehat{\id{\{}u / z \id{\}}}$, to the following pair
of processes, $\lift{w}{y!(z)}$ and $w[ \lpquote y!(z) \rpquote ]$.

\begin{eqnarray}
	\lift{w}{y!(z)}\widehat{\id{\{}u / z \id{\}}}
		& = &
		\lift{w}{y!(u)} \nonumber\\
	w[ \lpquote y!(z) \rpquote ] \widehat{ \id{\{}u / z \id{\}} }
		& = &
		w[ \lpquote y!(z) \rpquote ] \nonumber
\end{eqnarray}

Because the body of the process between quotes is impervious to
substitution, we get radically different answers. In fact, by
examining the first process in an input context,
e.g. $x?(z).\lift{w}{y!(z)}$, we see that the process under the lift
operator may be shaped by prefixed inputs binding a name inside it. In
this sense, the lift operator will be seen as a way to dynamically
construct processes before reifying them as names.

Finally equipped with these standard features we can present the
dynamics of the calculus.

\subsubsection{Operational semantics} 

Finally, we introduce the computational dynamics. What marks these
algebras as distinct from other more traditionally studied algebraic
structures, e.g. vector spaces or polynomial rings, is the manner in
which dynamics is captured. In traditional structures, dynamics is typically
expressed through morphisms between such structures, as in linear maps
between vector spaces or morphisms between rings. In algebras
associated with the semantics of computation, the dynamics is
expressed as part of the algebraic structure itself, through a
reduction reduction relation typically denoted by $\red$. Below, we
give a recursive presentation of this relation for the calculus used
in the encoding.

$\red \subseteq \pi \times \pi$
$\red : \pi \to \mathcal{P}(\pi)$

\begin{mathpar}
  \inferrule* [lab=Comm] { \textsf{match}( x_{src}, x_{trgt} ) } { x_{trgt}?(y)P \; | \; x_{src}!\langle {Q} \rangle \red P\{\quotep{Q}/y}\} }
  \and \\
  \inferrule* [lab=Par] {{P} \red {P}'} {{{P} | {Q}} \red {{P}' | {Q}}}
  \and
  \inferrule* [lab=Equiv]{{{P} \scong {P}'} \andalso {{P}' \red {Q}'} \andalso {{Q}' \scong {Q}}}{{P} \red {Q}}
\end{mathpar}

\begin{eqnarray*}
  match_{\equiv} (\quotep{P},\quotep{Q}) & := & P \equiv Q \\
  match_{\dagger}(\quotep{P},\quotep{Q}) & := & \forall R. P|Q \red^{*} R => R \red^{*} 0 \\
  match_{K}(\quotep{P},\quotep{Q}) & := & K \mbox{ for some context } K
\end{eqnarray*}

$u?(x)P | u!\langle Q \rangle \red P\{\quotep{Q}/x\}$

%We write $\wred$ for $\red^*$, and $P\red$ if $\exists Q $ such that $ P \red Q$.
We write $P\red$ if $\exists Q $ such that $ P \red Q$ and $P\not\red$, otherwise.

\section{Replication}

As mentioned before, it is known that replication (and hence
recursion) can be implemented in a higher-order process algebra
\cite{SangiorgiWalker}. As our first example of calculation with the
machinery thus far presented we give the construction explicitly in
the {\rhoc}.

\begin{eqnarray}
	D_{x} & := & \prefix{x}{y}{(\binpar{\outputp{x}{y}}{@{y}})} \nonumber\\
	\bangp_{x}{P} & := & \binpar{{x}!\langle{\binpar{D_{x}}{P}}\rangle}{D_{x}} \nonumber
\end{eqnarray}

\begin{eqnarray}
	\bangp_{x}{P} & & \nonumber\\
	=
	& {x}!\langle{(\prefix{x}{y}{(\outputp{x}{y} | @{y})) | P}}\rangle 
	      | \prefix{x}{y}{(\outputp{x}{y} | @{y})} & \nonumber\\
	\red
	& (\outputp{x}{y} | @{y})\substn{\quotep{(\prefix{x}{y}{(@{y} | \outputp{x}{y})) | P}}}{y} & \nonumber\\
	=
	& \outputp{x}{\quotep{(\prefix{x}{y}{(\outputp{x}{y} | @{y})) | P}}}
	  | {(\prefix{x}{y}{(\outputp{x}{y} | @{y})) | P}} & \nonumber\\
	\red
	& \ldots & \nonumber\\
	\red^*
	& P | P | \ldots & \nonumber
\end{eqnarray}

Of course, this encoding, as an implementation, runs away, unfolding
$\bangp{P}$ eagerly. A lazier and more implementable replication
operator, restricted to input-guarded processes, may be obtained as follows.

\begin{eqnarray}
\bangp{\prefix{u}{v}{P}} 
	:= 
	\binpar{\lift{x}{\prefix{u}{v}{(\binpar{D(x)}{P})}}}{D(x)} \nonumber
\end{eqnarray}

\begin{remark}
  Note that the lazier definition still does not deal with summation
  or mixed summation (i.e. sums over input and output). The reader is
  invited to construct definitions of replication that deal with these
  features. 

  Further, the definitions are parameterized in a name, $x$. Can you,
  gentle reader, make a definition that eliminates this parameter and
  guarantees no accidental interaction between the replication
  machinery and the process being replicated -- i.e. no accidental
  sharing of names used by the process to get its work done and the
  name(s) used by the replication to effect copying. This latter
  revision of the definition of replication is crucial to obtaining
  the expected identity $!!P \sim !P$.
\end{remark}

\begin{remark}\label{rem:paradoxical_combinator}
  The reader familiar with the lambda calculus will have noticed the
  similarity between $D$ and the paradoxical combinator.

  [Ed. note: the existence of this seems to suggest we have to be more
  restrictive on the set of processes and names we admit if we are to
  support no-cloning.]
\end{remark}

\subsubsection{Bisimulation}

The computational dynamics gives rise to another kind of equivalence,
the equivalence of computational behavior. As previously mentioned
this is typically captured \emph{via} some form of bisimulation.

% The notion we use in this paper is weak barbed bisimulation
% \cite{milner91polyadicpi}.

The notion we use in this paper is derived from weak barbed
bisimulation \cite{milner91polyadicpi}. 

\begin{definition}
An \emph{observation relation}, $\downarrow_{\mathcal N}$, over a set
of names, $\mathcal N$, is the smallest relation satisfying the rules
below.

\infrule[Out-barb]{y \in {\mathcal N}, \; x \nameeq y}
		  {\outputp{x}{v} \downarrow_{\mathcal N} x}
\infrule[Par-barb]{\mbox{$P\downarrow_{\mathcal N} x$ or $Q\downarrow_{\mathcal N} x$}}
		  {\binpar{P}{Q} \downarrow_{\mathcal N} x}

We write $P \Downarrow_{\mathcal N} x$ if there is $Q$ such that 
$P \wred Q$ and $Q \downarrow_{\mathcal N} x$.
\end{definition}

\begin{definition}
%\label{def.bbisim}
An  ${\mathcal N}$-\emph{barbed bisimulation} over a set of names, ${\mathcal N}$, is a symmetric binary relation 
${\mathcal S}_{\mathcal N}$ between agents such that $P\rel{S}_{\mathcal N}Q$ implies:
\begin{enumerate}
\item If $P \red P'$ then $Q \wred Q'$ and $P'\rel{S}_{\mathcal N} Q'$.
\item If $P\downarrow_{\mathcal N} x$, then $Q\Downarrow_{\mathcal N} x$.
\end{enumerate}
$P$ is ${\mathcal N}$-barbed bisimilar to $Q$, written
$P \wbbisim_{\mathcal N} Q$, if $P \rel{S}_{\mathcal N} Q$ for some ${\mathcal N}$-barbed bisimulation ${\mathcal S}_{\mathcal N}$.
\end{definition}

$\mathcal{R} \subseteq \pi \times \pi$

$P \mathcal{R} Q => \forall P'. P \red P' \Rightarrow \exists Q'. Q \red Q', P' \mathcal{R} Q'$

$P \vdash x \Rightarrow Q \vdash x$

\begin{mathpar}
  \inferrule*[lab=Out-barb]{x \nameeq y}{{y}!\langle{Q}\rangle \vdash x}
  \and
  \inferrule*[lab=Par-barb]{\mbox{$P\vdash x$ or $Q\vdash x$}}{\binpar{P}{Q} \vdash x}
\end{mathpar}

\subsubsection{Contexts}

One of the principle advantages of computational calculi like the
$\pi$-calculus is a well-defined notion of context,
contextual-equivalence and a correlation between
contextual-equivalence and notions of bisimulation. The notion of
context allows the decomposition of a process into (sub-)process and
its syntactic environment, its context. Thus, a context may be
thought of as a process with a ``hole'' (written $\Box$) in it. The
application of a context $M$ to a process $P$, written $M[P]$, is
tantamount to filling the hole in $M$ with $P$. In this paper we do
not need the full weight of this theory, but do make use of the notion
of context in the proof the main theorem. 

\begin{mathpar}
  \inferrule* [lab=summation] {} {{M_{M},M_{N}} \bc \Box \;|\; x.M_{A} \;|\; M_{M}+M_{N}}
  \and
  \inferrule* [lab=agent] {} {{M_{A}} \bc (\vec{x})M_{P} \;| \; \clift{P_0,\ldots,M_{P},\ldots,P_N}}
  \and \\
  \inferrule* [lab=process] {} {{M_{P}} \bc M_{N} \;| \;P|M_{P} }
\end{mathpar} 

\begin{mathpar}
  \inferrule* [lab=sychronization] {} {M_{N} \bc \Box \;|\; x?M_{F} \;|\; x!M_{C}}
  \and
  \inferrule* [lab=abstraction] {} {{M_{F}} \bc (x)M_{P} }
  \and
  \inferrule* [lab=concretion] {} {{M_{C}} \bc \langle M_{P} \rangle }
  \and \\
  \inferrule* [lab=process] {} {{M_{P}} \bc M_{N} \;| \;P|M_{P} }
\end{mathpar}

\begin{definition}[contextual application] Given a context $M$, and
  process $P$, we define the \emph{contextual application}, $M[P] :=
  M\{P/\Box\}$. That is, the contextual application of M to P is the
  substitution of $P$ for $\Box$ in $M$.
\end{definition}

$\meaningof{-} : L \to \mathcal{P}(\pi)$

\begin{mathpar}
  \inferrule* [lab=collection] {} {\meaningof{true} = \pi, \and \meaningof{~E} = \pi \setminus \meaningof{E}, \and \meaningof{E_{1} \& E_{2}} = \meaningof{E_{1}} \cap \meaningof{E_{2}}}
\end{mathpar}

\begin{mathpar}
  \inferrule* [lab=structure] {} {\meaningof{0} = \{ P \in \pi | P \equiv 0 \}, \and \\ \meaningof{E_1 | E_2} = \{ P \in \pi | P \equiv P_{1} | P_{2}, P_{1} \in \meaningof{E_{1}}, P_{2} \in \meaningof{E_2}\} }
\end{mathpar}

\begin{mathpar}
 \inferrule* [lab=behavior] {} {\meaningof{\langle a?b \rangle E} = \{ P \in \pi | P \equiv Q | u?(y)P', \\ \and \\\\ \and \\ \;\;\; u \in \meaningof{a}, \forall z.P'\{z/y\} \in \meaningof{E\{z/b\}}\}, \and \\ \meaningof{a!E} = \{ P \in \pi | P \equiv Q | x!\langle P' \rangle, x \in \meaningof{a} P' \in \meaningof{E}\} }
\end{mathpar}

\begin{mathpar}
 \inferrule* [lab=nominal] {} {\meaningof{\quotep{E}} = \{ \quotep{P} \in \quotep{\pi} | P \in \meaningof{E} \}, \and \meaningof{\quotep{P}} = \{ \quotep{Q} \in \quotep{\pi} | P \equiv Q \} \and \\ \meaningof{@\quotep{E}} = \{ P \in \pi | P \equiv @x, x \in \meaningof{E} \}}
\end{mathpar}

\begin{eqnarray*}
  \\
  \meaningof{-} : TS \to ST
\end{eqnarray*}

\begin{eqnarray*}
  \\
  L : TS \to ST
\end{eqnarray*}

\begin{eqnarray*}
  \\
  P \models E \iff P \in \meaningof{E}
\end{eqnarray*}

\begin{eqnarray*}
  P \approx_{L} Q \iff \forall E \in L. P \models E \iff Q \models E
\end{eqnarray*}

\begin{eqnarray*}
  P \approx_{K} Q
\end{eqnarray*}

\begin{eqnarray*}
  P \approx Q
\end{eqnarray*}

$\approx_{K} = \approx = \approx_{L}$

\subsubsection{Contextual duality}

Note that contexts extend the quotation operation to a family of
operations from processes to names. Given a context, $M$, we can
define a \emph{nominal context}, $\quotep{M}$ by $\quotep{M}[P] :=
\quotep{M[P]}$. To foreshadow what is to come we observe that these
operations enjoy a duality with processes very much like the duality
between vectors and maps from vectors to scalars.

Further, because the calculus is essentially higher-order, we have a
correspondence between contexts and processes. More specifically,
given a name $x$ and a context $M$ we can construct $M^{*}_{x}$ such
that 

\begin{mathpar}
  M^{*}_{x} | \lift{x}{P} \red M[P]
\end{mathpar}

namely,

\begin{mathpar}
  M^{*}_{x} := x?(u).M[\dropn{u}]
\end{mathpar}

The dependence of $M^{*}_{x}$ on a name makes it an abstraction, 

\begin{mathpar}
  M^{*} := (x)x?(u).M[\dropn{u}]
\end{mathpar}

\subsection{Additional notation}

It will sometimes be convenient to denote the process a name
quotes. We already have the notation $x = \quotep{P}$, but it will be
convenient to introduce an alternate notation, $\procn{x}$, when we
want to emphasize the connection to the use of the name. Note that, by
virtue of name equivalence, $\quotep{\procn{x}} \nameeq x$; so, the
notation is consistent with previous definitions.

Further, because names have structure it is possible to effect
substitutions on the basis of that structure. This means we need to
upgrade our notation for substitutions, which we accomplish by
adapting comprehension notation. Thus,

\begin{mathpar}
  P\{ y / x : x \in S \}
\end{mathpar}

is interpreted to mean the process derived from P by replacing (in a
capture-avoiding manner) each occurrence of $x$ in $S$ by $y$. For example,

\begin{mathpar}
  P\{ \quotep{\procn{x}|\procn{x}} / x : x \in \freenames{P} \}
\end{mathpar}

will replace each (occurrence) of a free name $x$ in $P$ by
$\quotep{\procn{x}|\procn{x}}$.

Also, we will avail ourselves of the notation $x^{L}$ and $x^{R}$ to
denote injections of a name into disjoint copies of the name
space. There are numerous ways to accomplish this. One example can be
found in \cite{MeredithR05}. This notation overloads to vectors of
names: $\vec{x}^{\pi} := (x_{i}^{\pi} \; : \; 0 \leq i < |\vec{x}| )$ where $\pi \in \{L,R\}$.

We also use $P^{\Box} := P|\Box$.

In \cite{MeredithR05} an interpretation of the new operator is
given. It turns out that there are several possible interpretations
all enjoying the requisite algebraic properties of the operator (see
\cite{milner91polyadicpi}). We will therefore make liberal use of
$(\nu\; \vec{x})P$.

% subsection the_syntax_and_semantics_of_the_notation_system (end)   

\input{qm2pi.qmops} 

\input{qm2pi.sterngerlach} 

\input{qm2pi.metric} 

% section concurrent_process_calculi (end)

%\input{qm2pi.proofsketch}

% section proof sketch (end)

%\input{qm2pi.slviaknots} 

% section spatial logic via knots (end)

\input{qm2pi.conclusion}

% section conclusion (end)

%\input{qm2pi.dtcodes} 

% section wiring algorithm (end)

\input{qm2pi.ack} 

% section acknowledgments (end)

\newpage


\bibliographystyle{plain}   
\bibliography{../../biblios/main.bib}

\input{qm2pi.rhodetails}

\end{document}



\end{document}

 

%\documentclass[12pt]{llncs}
%\documentclass{jktr}

\usepackage[pdftex]{hyperref}                   
\usepackage {listings}
\usepackage {mathpartir}
\usepackage{bcprules}
%\usepackage{listings}
                       
\usepackage{graphicx} 
%\usepackage[margins=2.5cm,nohead,nofoot]{geometry}
%\usepackage{geometry}
\usepackage{amsfonts}
\usepackage{amstext}
\usepackage{latexsym}
\usepackage{amssymb}
\usepackage{color}


%\include{myPreamble}
\documentclass[12pt]{llncs}
%\documentclass{jktr}

\usepackage[pdftex]{hyperref}                   
\usepackage {listings}
\usepackage {mathpartir}
\usepackage{bcprules}
%\usepackage{listings}
                       
\usepackage{graphicx} 
%\usepackage[margins=2.5cm,nohead,nofoot]{geometry}
%\usepackage{geometry}
\usepackage{amsfonts}
\usepackage{amstext}
\usepackage{latexsym}
\usepackage{amssymb}
\usepackage{color}


%\include{myPreamble}
\include{qm2pi.local} 

%\ifpdf
%\usepackage[pdftex]{graphicx}
%\else
%\usepackage{graphicx}
%\fi

 % \ifpdf
%  \usepackage{pdfsync}
%  \if


%\title{Brief Article}
%\author{David F. Snyder}
%\author{L.G. Meredith}

%\address{Dept. of Math., Texas State University--San Marcos, San Marcos, TX 78666}
       
\pagestyle{empty}


\begin{document}

\lstset{language=[Objective]Caml,frame=shadowbox}

\input{qm2pi.front}

% section front matter (end)

\input{qm2pi.intro} 
 
% section introduction (end)

% \input{qm2pi.knotations} 

% section notation (end)

\input{qm2pi.process.calculi} 

% section concurrent_process_calculi_and_spatial_logics_ (end)
    
%\input{qm2pi.knots2pi} 

%\input{qm2pi.trefoil} 

%\input{qm2pi.mainthm} 

% subsection basic_interpretation (end)

%\input{qm2pi.rho.presentation} 
\subsection{The syntax and semantics of the notation system}\label{sub:the_syntax_and_semantics_of_the_notation_system} % (fold)

We now summarize a technical presentation of the calculus that
embodies our theory of dynamics. The typical presentation of such a
calculus follows the style of giving generators and relations on
them. The grammar, below, describing term constructors, freely
generates the set of processes, $\Proc$. This set is then quotiented
by a relation known as structural congruence and it is over this set
that the notion of dynamics is expressed. This presentation is
essentially that of \cite{MeredithR05} with the addition of
polyadicity and summation. For readability we have relegated some of
the technical subtleties to an appendix.

\subsubsection{Process grammar}\label{subsub:process_grammar}

\begin{mathpar}
  \inferrule* [lab=synchronization] {} {{M} \bc \pzero \;|\; x?F \;|\; x!C }
  \and
  \inferrule* [lab=abstraction] {} {{F} \bc (x)P}
  \and
  \inferrule* [lab=concretion] {} {{C} \bc \langle Q \rangle}
  \and
  \inferrule* [lab=process] {} {{P,Q} \bc M \;| \;P|Q \;|\; @{x}}
  \and
  \inferrule* [lab=name] {} {{x} \bc \quotep{P}}
\end{mathpar} 

Note that $\vec{x}$ (resp. $\vec{P}$) denotes a vector of names
(resp. processes) of length $|\vec{x}|$ (resp. $|\vec{P}|$). We adopt
the following useful abbreviations.

\begin{mathpar}
   x?(\vec{y}).P := x.(\vec{y})P \and  x\clift{\vec{P}} := x.\clift{\vec{P}}
   \and x!(y) := \lift{x}{\dropn{y}}
   \and \Pi_{i=0}^{n-1}P_i := P_0 | \ldots | P_{n-1}
\end{mathpar}

\subsubsection{Structural congruence}

\paragraph{Free and bound names and alpha-equivalence.} At the
core of structural equivalence is alpha-equivalence which identifies
process that are the same up to a change of variable. Formally, we
recognize the distinction between free and bound names. The free names
of a process, $\freenames{P}$, may be calculated recursively as
follows:

\begin{mathpar}
\freenames{\pzero} := \emptyset
  \and \\
  \freenames{x?(y).P} := \{ x \} \cup (\freenames{P} \setminus \{ y \})
  \and 
  \freenames{x!\langle P \rangle} := \{ x \} \cup \{ P \} 
  \and \\
  \freenames{P|Q} := \freenames{P} \cup \freenames{Q}
  \and \\
  \freenames{@{x}} := \{ x \}
\end{mathpar}

$\pi$
$\quotep{\pi}$

$\freenames{-} : \pi \to \mathcal{P}(\quotep{\pi})$

\begin{eqnarray*}
  \freenames{\pzero} & := & \emptyset \\
  \freenames{x?(y).P} & := & \{ x \} \cup (\freenames{P} \setminus \{ y \}) \\
  \freenames{x!\langle P \rangle} & := & \{ x \} \cup \{ P \} \\
  \freenames{P|Q} & := & \freenames{P} \cup \freenames{Q} \\
  \freenames{\dropn{x}} & := & \{ x \}
\end{eqnarray*}

The bound names of a process, $\boundnames{P}$, are those names occurring in $P$
that are not free. For example, in $x?(y).0$, the name $x$ is free, while $y$ is bound.

\begin{mathpar}
  \inferrule* [lab=monoidal-laws] {} { P|Q \equiv Q|P \and P|0 \equiv P \and P|(Q|R) \equiv (P|Q)|R }
\end{mathpar}

\begin{mathpar}
  \inferrule* [lab=alpha-equivalence] {} { (x)P \equiv (y)P\{y/x\} \and y \not\in \freenames{P} }
\end{mathpar}

\begin{definition}
Then two processes, $P,Q$, are alpha-equivalent if $P = Q\{\vec{y}/\vec{x}\}$ for
some $\vec{x} \in \boundnames{Q},\vec{y} \in \boundnames{P}$, where $Q\{\vec{y}/\vec{x}\}$
denotes the capture-avoiding substitution of $\vec{y}$ for $\vec{x}$ in $Q$.
\end{definition}

\begin{definition}
  The {\em structural congruence} \cite{SangiorgiWalker} , $\equiv$,
  between processes is the least congruence containing
  alpha-equivalence, satisfying the abelian monoid laws
  (associativity, commutativity and $\pzero$ as identity) for parallel
  composition $|$ and for summation $+$.
\end{definition}

\subsection{Name equivalence}

We take name equivalence, written $\nameeq$, to be the smallest
equivalence relation generated by the following rules.

\begin{mathpar}
\inferrule*[lab=Quote-drop]
{ }
{ \quotep{@{x}} \nameeq x }

\inferrule*[lab=Struct-equiv]
{ P \scong Q }
{ \quotep{P} \nameeq \quotep{Q} }
\end{mathpar}

The astute reader will have noticed that the mutual recursion of names
and processes imposes a mutual recursion on alpha-equivalence and
structural equivalence via name-equivalence. Fortunately, all of this
works out pleasantly and we may calculate in the natural way, free of
concern. The reader interested in the details is referred to the
appendix \ref{appendix:rho_details}.

\subsection{Substitution}

We use $\Proc$ for the set of processes, $\QProc$ for the set of
names, and $\id{\{}\vec{y} / \vec{x} \id{\}}$ to denote partial maps,
$s : \QProc \rightarrow \QProc$. A map, $s$ lifts, uniquely, to a map
on process terms, $\widehat{s} : \Proc \rightarrow \Proc$ by the
following equations.

\begin{mathpar}
  (0) \psubstp{Q}{P} := 0 \\
  (R \juxtap S) \psubstp{Q}{P}
  :=    
  (R)\psubstp{Q}{P} \juxtap (S) \psubstp{Q}{P} \\
  (x?(y).R) \psubstp{Q}{P}    
  :=    
  (x)\substp{Q}{P} (z)\concat( (R \psubstn{z}{y}) \psubstp{Q}{P} ) \\
  (\lift{x}{R}) \psubstp{Q}{P}  
  :=
  \lift{(x)\substp{Q}{P}}{ R \psubstp{Q}{P} } \\
%   (\dropn{x})  \psubstp{Q}{P}       
%   := 
%   \left\{ 
%     \begin{array}{ccc} 
%       \dropn{\quotep{Q}} & & x \nameeq \quotep{P} \\
%       \dropn{x} & & otherwise \\
%     \end{array}
%   \right. 
  (\dropn{x})  \psubstp{Q}{P}       
  := 
  \left\{ 
    \begin{array}{ccc} 
      Q & & x \nameeq \quotep{P} \\
      \dropn{x} & & otherwise \\
    \end{array}
  \right.
\end{mathpar}
 

where

\begin{eqnarray}
  (x)\id{\{} \lpquote Q \rpquote / \lpquote P \rpquote \id{\}}            = 
  \left\{ 
    \begin{array}{ccc}
      \lpquote Q \rpquote & & x \nameeq \lpquote P \rpquote \\
      x & & otherwise \\
    \end{array}
  \right. \nonumber
\end{eqnarray}

and $z$ is chosen distinct from $\quotep{P}$, $\quotep{Q}$, the free
names in $Q$, and all the names in $R$. Our $\alpha$-equivalence will
be built in the standard way from this substitution.

\begin{remark}\label{rem:no_self_referential_names}
  One consequence of these definitions is that $\forall P. \quotep{P}
  \not\in \freenames{P}$.
\end{remark}

\subsection{ Dynamic quote: an example }

Anticipating something of what's to come, consider applying the
substitution, $\widehat{\id{\{}u / z \id{\}}}$, to the following pair
of processes, $\lift{w}{y!(z)}$ and $w[ \lpquote y!(z) \rpquote ]$.

\begin{eqnarray}
	\lift{w}{y!(z)}\widehat{\id{\{}u / z \id{\}}}
		& = &
		\lift{w}{y!(u)} \nonumber\\
	w[ \lpquote y!(z) \rpquote ] \widehat{ \id{\{}u / z \id{\}} }
		& = &
		w[ \lpquote y!(z) \rpquote ] \nonumber
\end{eqnarray}

Because the body of the process between quotes is impervious to
substitution, we get radically different answers. In fact, by
examining the first process in an input context,
e.g. $x?(z).\lift{w}{y!(z)}$, we see that the process under the lift
operator may be shaped by prefixed inputs binding a name inside it. In
this sense, the lift operator will be seen as a way to dynamically
construct processes before reifying them as names.

Finally equipped with these standard features we can present the
dynamics of the calculus.

\subsubsection{Operational semantics} 

Finally, we introduce the computational dynamics. What marks these
algebras as distinct from other more traditionally studied algebraic
structures, e.g. vector spaces or polynomial rings, is the manner in
which dynamics is captured. In traditional structures, dynamics is typically
expressed through morphisms between such structures, as in linear maps
between vector spaces or morphisms between rings. In algebras
associated with the semantics of computation, the dynamics is
expressed as part of the algebraic structure itself, through a
reduction reduction relation typically denoted by $\red$. Below, we
give a recursive presentation of this relation for the calculus used
in the encoding.

$\red \subseteq \pi \times \pi$
$\red : \pi \to \mathcal{P}(\pi)$

\begin{mathpar}
  \inferrule* [lab=Comm] { \textsf{match}( x_{src}, x_{trgt} ) } { x_{trgt}?(y)P \; | \; x_{src}!\langle {Q} \rangle \red P\{\quotep{Q}/y}\} }
  \and \\
  \inferrule* [lab=Par] {{P} \red {P}'} {{{P} | {Q}} \red {{P}' | {Q}}}
  \and
  \inferrule* [lab=Equiv]{{{P} \scong {P}'} \andalso {{P}' \red {Q}'} \andalso {{Q}' \scong {Q}}}{{P} \red {Q}}
\end{mathpar}

\begin{eqnarray*}
  match_{\equiv} (\quotep{P},\quotep{Q}) & := & P \equiv Q \\
  match_{\dagger}(\quotep{P},\quotep{Q}) & := & \forall R. P|Q \red^{*} R => R \red^{*} 0 \\
  match_{K}(\quotep{P},\quotep{Q}) & := & K \mbox{ for some context } K
\end{eqnarray*}

$u?(x)P | u!\langle Q \rangle \red P\{\quotep{Q}/x\}$

%We write $\wred$ for $\red^*$, and $P\red$ if $\exists Q $ such that $ P \red Q$.
We write $P\red$ if $\exists Q $ such that $ P \red Q$ and $P\not\red$, otherwise.

\section{Replication}

As mentioned before, it is known that replication (and hence
recursion) can be implemented in a higher-order process algebra
\cite{SangiorgiWalker}. As our first example of calculation with the
machinery thus far presented we give the construction explicitly in
the {\rhoc}.

\begin{eqnarray}
	D_{x} & := & \prefix{x}{y}{(\binpar{\outputp{x}{y}}{@{y}})} \nonumber\\
	\bangp_{x}{P} & := & \binpar{{x}!\langle{\binpar{D_{x}}{P}}\rangle}{D_{x}} \nonumber
\end{eqnarray}

\begin{eqnarray}
	\bangp_{x}{P} & & \nonumber\\
	=
	& {x}!\langle{(\prefix{x}{y}{(\outputp{x}{y} | @{y})) | P}}\rangle 
	      | \prefix{x}{y}{(\outputp{x}{y} | @{y})} & \nonumber\\
	\red
	& (\outputp{x}{y} | @{y})\substn{\quotep{(\prefix{x}{y}{(@{y} | \outputp{x}{y})) | P}}}{y} & \nonumber\\
	=
	& \outputp{x}{\quotep{(\prefix{x}{y}{(\outputp{x}{y} | @{y})) | P}}}
	  | {(\prefix{x}{y}{(\outputp{x}{y} | @{y})) | P}} & \nonumber\\
	\red
	& \ldots & \nonumber\\
	\red^*
	& P | P | \ldots & \nonumber
\end{eqnarray}

Of course, this encoding, as an implementation, runs away, unfolding
$\bangp{P}$ eagerly. A lazier and more implementable replication
operator, restricted to input-guarded processes, may be obtained as follows.

\begin{eqnarray}
\bangp{\prefix{u}{v}{P}} 
	:= 
	\binpar{\lift{x}{\prefix{u}{v}{(\binpar{D(x)}{P})}}}{D(x)} \nonumber
\end{eqnarray}

\begin{remark}
  Note that the lazier definition still does not deal with summation
  or mixed summation (i.e. sums over input and output). The reader is
  invited to construct definitions of replication that deal with these
  features. 

  Further, the definitions are parameterized in a name, $x$. Can you,
  gentle reader, make a definition that eliminates this parameter and
  guarantees no accidental interaction between the replication
  machinery and the process being replicated -- i.e. no accidental
  sharing of names used by the process to get its work done and the
  name(s) used by the replication to effect copying. This latter
  revision of the definition of replication is crucial to obtaining
  the expected identity $!!P \sim !P$.
\end{remark}

\begin{remark}\label{rem:paradoxical_combinator}
  The reader familiar with the lambda calculus will have noticed the
  similarity between $D$ and the paradoxical combinator.

  [Ed. note: the existence of this seems to suggest we have to be more
  restrictive on the set of processes and names we admit if we are to
  support no-cloning.]
\end{remark}

\subsubsection{Bisimulation}

The computational dynamics gives rise to another kind of equivalence,
the equivalence of computational behavior. As previously mentioned
this is typically captured \emph{via} some form of bisimulation.

% The notion we use in this paper is weak barbed bisimulation
% \cite{milner91polyadicpi}.

The notion we use in this paper is derived from weak barbed
bisimulation \cite{milner91polyadicpi}. 

\begin{definition}
An \emph{observation relation}, $\downarrow_{\mathcal N}$, over a set
of names, $\mathcal N$, is the smallest relation satisfying the rules
below.

\infrule[Out-barb]{y \in {\mathcal N}, \; x \nameeq y}
		  {\outputp{x}{v} \downarrow_{\mathcal N} x}
\infrule[Par-barb]{\mbox{$P\downarrow_{\mathcal N} x$ or $Q\downarrow_{\mathcal N} x$}}
		  {\binpar{P}{Q} \downarrow_{\mathcal N} x}

We write $P \Downarrow_{\mathcal N} x$ if there is $Q$ such that 
$P \wred Q$ and $Q \downarrow_{\mathcal N} x$.
\end{definition}

\begin{definition}
%\label{def.bbisim}
An  ${\mathcal N}$-\emph{barbed bisimulation} over a set of names, ${\mathcal N}$, is a symmetric binary relation 
${\mathcal S}_{\mathcal N}$ between agents such that $P\rel{S}_{\mathcal N}Q$ implies:
\begin{enumerate}
\item If $P \red P'$ then $Q \wred Q'$ and $P'\rel{S}_{\mathcal N} Q'$.
\item If $P\downarrow_{\mathcal N} x$, then $Q\Downarrow_{\mathcal N} x$.
\end{enumerate}
$P$ is ${\mathcal N}$-barbed bisimilar to $Q$, written
$P \wbbisim_{\mathcal N} Q$, if $P \rel{S}_{\mathcal N} Q$ for some ${\mathcal N}$-barbed bisimulation ${\mathcal S}_{\mathcal N}$.
\end{definition}

$\mathcal{R} \subseteq \pi \times \pi$

$P \mathcal{R} Q => \forall P'. P \red P' \Rightarrow \exists Q'. Q \red Q', P' \mathcal{R} Q'$

$P \vdash x \Rightarrow Q \vdash x$

\begin{mathpar}
  \inferrule*[lab=Out-barb]{x \nameeq y}{{y}!\langle{Q}\rangle \vdash x}
  \and
  \inferrule*[lab=Par-barb]{\mbox{$P\vdash x$ or $Q\vdash x$}}{\binpar{P}{Q} \vdash x}
\end{mathpar}

\subsubsection{Contexts}

One of the principle advantages of computational calculi like the
$\pi$-calculus is a well-defined notion of context,
contextual-equivalence and a correlation between
contextual-equivalence and notions of bisimulation. The notion of
context allows the decomposition of a process into (sub-)process and
its syntactic environment, its context. Thus, a context may be
thought of as a process with a ``hole'' (written $\Box$) in it. The
application of a context $M$ to a process $P$, written $M[P]$, is
tantamount to filling the hole in $M$ with $P$. In this paper we do
not need the full weight of this theory, but do make use of the notion
of context in the proof the main theorem. 

\begin{mathpar}
  \inferrule* [lab=summation] {} {{M_{M},M_{N}} \bc \Box \;|\; x.M_{A} \;|\; M_{M}+M_{N}}
  \and
  \inferrule* [lab=agent] {} {{M_{A}} \bc (\vec{x})M_{P} \;| \; \clift{P_0,\ldots,M_{P},\ldots,P_N}}
  \and \\
  \inferrule* [lab=process] {} {{M_{P}} \bc M_{N} \;| \;P|M_{P} }
\end{mathpar} 

\begin{mathpar}
  \inferrule* [lab=sychronization] {} {M_{N} \bc \Box \;|\; x?M_{F} \;|\; x!M_{C}}
  \and
  \inferrule* [lab=abstraction] {} {{M_{F}} \bc (x)M_{P} }
  \and
  \inferrule* [lab=concretion] {} {{M_{C}} \bc \langle M_{P} \rangle }
  \and \\
  \inferrule* [lab=process] {} {{M_{P}} \bc M_{N} \;| \;P|M_{P} }
\end{mathpar}

\begin{definition}[contextual application] Given a context $M$, and
  process $P$, we define the \emph{contextual application}, $M[P] :=
  M\{P/\Box\}$. That is, the contextual application of M to P is the
  substitution of $P$ for $\Box$ in $M$.
\end{definition}

$\meaningof{-} : L \to \mathcal{P}(\pi)$

\begin{mathpar}
  \inferrule* [lab=collection] {} {\meaningof{true} = \pi, \and \meaningof{~E} = \pi \setminus \meaningof{E}, \and \meaningof{E_{1} \& E_{2}} = \meaningof{E_{1}} \cap \meaningof{E_{2}}}
\end{mathpar}

\begin{mathpar}
  \inferrule* [lab=structure] {} {\meaningof{0} = \{ P \in \pi | P \equiv 0 \}, \and \\ \meaningof{E_1 | E_2} = \{ P \in \pi | P \equiv P_{1} | P_{2}, P_{1} \in \meaningof{E_{1}}, P_{2} \in \meaningof{E_2}\} }
\end{mathpar}

\begin{mathpar}
 \inferrule* [lab=behavior] {} {\meaningof{\langle a?b \rangle E} = \{ P \in \pi | P \equiv Q | u?(y)P', \\ \and \\\\ \and \\ \;\;\; u \in \meaningof{a}, \forall z.P'\{z/y\} \in \meaningof{E\{z/b\}}\}, \and \\ \meaningof{a!E} = \{ P \in \pi | P \equiv Q | x!\langle P' \rangle, x \in \meaningof{a} P' \in \meaningof{E}\} }
\end{mathpar}

\begin{mathpar}
 \inferrule* [lab=nominal] {} {\meaningof{\quotep{E}} = \{ \quotep{P} \in \quotep{\pi} | P \in \meaningof{E} \}, \and \meaningof{\quotep{P}} = \{ \quotep{Q} \in \quotep{\pi} | P \equiv Q \} \and \\ \meaningof{@\quotep{E}} = \{ P \in \pi | P \equiv @x, x \in \meaningof{E} \}}
\end{mathpar}

\begin{eqnarray*}
  \\
  \meaningof{-} : TS \to ST
\end{eqnarray*}

\begin{eqnarray*}
  \\
  L : TS \to ST
\end{eqnarray*}

\begin{eqnarray*}
  \\
  P \models E \iff P \in \meaningof{E}
\end{eqnarray*}

\begin{eqnarray*}
  P \approx_{L} Q \iff \forall E \in L. P \models E \iff Q \models E
\end{eqnarray*}

\begin{eqnarray*}
  P \approx_{K} Q
\end{eqnarray*}

\begin{eqnarray*}
  P \approx Q
\end{eqnarray*}

$\approx_{K} = \approx = \approx_{L}$

\subsubsection{Contextual duality}

Note that contexts extend the quotation operation to a family of
operations from processes to names. Given a context, $M$, we can
define a \emph{nominal context}, $\quotep{M}$ by $\quotep{M}[P] :=
\quotep{M[P]}$. To foreshadow what is to come we observe that these
operations enjoy a duality with processes very much like the duality
between vectors and maps from vectors to scalars.

Further, because the calculus is essentially higher-order, we have a
correspondence between contexts and processes. More specifically,
given a name $x$ and a context $M$ we can construct $M^{*}_{x}$ such
that 

\begin{mathpar}
  M^{*}_{x} | \lift{x}{P} \red M[P]
\end{mathpar}

namely,

\begin{mathpar}
  M^{*}_{x} := x?(u).M[\dropn{u}]
\end{mathpar}

The dependence of $M^{*}_{x}$ on a name makes it an abstraction, 

\begin{mathpar}
  M^{*} := (x)x?(u).M[\dropn{u}]
\end{mathpar}

\subsection{Additional notation}

It will sometimes be convenient to denote the process a name
quotes. We already have the notation $x = \quotep{P}$, but it will be
convenient to introduce an alternate notation, $\procn{x}$, when we
want to emphasize the connection to the use of the name. Note that, by
virtue of name equivalence, $\quotep{\procn{x}} \nameeq x$; so, the
notation is consistent with previous definitions.

Further, because names have structure it is possible to effect
substitutions on the basis of that structure. This means we need to
upgrade our notation for substitutions, which we accomplish by
adapting comprehension notation. Thus,

\begin{mathpar}
  P\{ y / x : x \in S \}
\end{mathpar}

is interpreted to mean the process derived from P by replacing (in a
capture-avoiding manner) each occurrence of $x$ in $S$ by $y$. For example,

\begin{mathpar}
  P\{ \quotep{\procn{x}|\procn{x}} / x : x \in \freenames{P} \}
\end{mathpar}

will replace each (occurrence) of a free name $x$ in $P$ by
$\quotep{\procn{x}|\procn{x}}$.

Also, we will avail ourselves of the notation $x^{L}$ and $x^{R}$ to
denote injections of a name into disjoint copies of the name
space. There are numerous ways to accomplish this. One example can be
found in \cite{MeredithR05}. This notation overloads to vectors of
names: $\vec{x}^{\pi} := (x_{i}^{\pi} \; : \; 0 \leq i < |\vec{x}| )$ where $\pi \in \{L,R\}$.

We also use $P^{\Box} := P|\Box$.

In \cite{MeredithR05} an interpretation of the new operator is
given. It turns out that there are several possible interpretations
all enjoying the requisite algebraic properties of the operator (see
\cite{milner91polyadicpi}). We will therefore make liberal use of
$(\nu\; \vec{x})P$.

% subsection the_syntax_and_semantics_of_the_notation_system (end)   

\input{qm2pi.qmops} 

\input{qm2pi.sterngerlach} 

\input{qm2pi.metric} 

% section concurrent_process_calculi (end)

%\input{qm2pi.proofsketch}

% section proof sketch (end)

%\input{qm2pi.slviaknots} 

% section spatial logic via knots (end)

\input{qm2pi.conclusion}

% section conclusion (end)

%\input{qm2pi.dtcodes} 

% section wiring algorithm (end)

\input{qm2pi.ack} 

% section acknowledgments (end)

\newpage


\bibliographystyle{plain}   
\bibliography{../../biblios/main.bib}

\input{qm2pi.rhodetails}

\end{document}

 

%\ifpdf
%\usepackage[pdftex]{graphicx}
%\else
%\usepackage{graphicx}
%\fi

 % \ifpdf
%  \usepackage{pdfsync}
%  \if


%\title{Brief Article}
%\author{David F. Snyder}
%\author{L.G. Meredith}

%\address{Dept. of Math., Texas State University--San Marcos, San Marcos, TX 78666}
       
\pagestyle{empty}


\begin{document}

\lstset{language=[Objective]Caml,frame=shadowbox}

\documentclass[12pt]{llncs}
%\documentclass{jktr}

\usepackage[pdftex]{hyperref}                   
\usepackage {listings}
\usepackage {mathpartir}
\usepackage{bcprules}
%\usepackage{listings}
                       
\usepackage{graphicx} 
%\usepackage[margins=2.5cm,nohead,nofoot]{geometry}
%\usepackage{geometry}
\usepackage{amsfonts}
\usepackage{amstext}
\usepackage{latexsym}
\usepackage{amssymb}
\usepackage{color}


%\include{myPreamble}
\include{qm2pi.local} 

%\ifpdf
%\usepackage[pdftex]{graphicx}
%\else
%\usepackage{graphicx}
%\fi

 % \ifpdf
%  \usepackage{pdfsync}
%  \if


%\title{Brief Article}
%\author{David F. Snyder}
%\author{L.G. Meredith}

%\address{Dept. of Math., Texas State University--San Marcos, San Marcos, TX 78666}
       
\pagestyle{empty}


\begin{document}

\lstset{language=[Objective]Caml,frame=shadowbox}

\input{qm2pi.front}

% section front matter (end)

\input{qm2pi.intro} 
 
% section introduction (end)

% \input{qm2pi.knotations} 

% section notation (end)

\input{qm2pi.process.calculi} 

% section concurrent_process_calculi_and_spatial_logics_ (end)
    
%\input{qm2pi.knots2pi} 

%\input{qm2pi.trefoil} 

%\input{qm2pi.mainthm} 

% subsection basic_interpretation (end)

%\input{qm2pi.rho.presentation} 
\subsection{The syntax and semantics of the notation system}\label{sub:the_syntax_and_semantics_of_the_notation_system} % (fold)

We now summarize a technical presentation of the calculus that
embodies our theory of dynamics. The typical presentation of such a
calculus follows the style of giving generators and relations on
them. The grammar, below, describing term constructors, freely
generates the set of processes, $\Proc$. This set is then quotiented
by a relation known as structural congruence and it is over this set
that the notion of dynamics is expressed. This presentation is
essentially that of \cite{MeredithR05} with the addition of
polyadicity and summation. For readability we have relegated some of
the technical subtleties to an appendix.

\subsubsection{Process grammar}\label{subsub:process_grammar}

\begin{mathpar}
  \inferrule* [lab=synchronization] {} {{M} \bc \pzero \;|\; x?F \;|\; x!C }
  \and
  \inferrule* [lab=abstraction] {} {{F} \bc (x)P}
  \and
  \inferrule* [lab=concretion] {} {{C} \bc \langle Q \rangle}
  \and
  \inferrule* [lab=process] {} {{P,Q} \bc M \;| \;P|Q \;|\; @{x}}
  \and
  \inferrule* [lab=name] {} {{x} \bc \quotep{P}}
\end{mathpar} 

Note that $\vec{x}$ (resp. $\vec{P}$) denotes a vector of names
(resp. processes) of length $|\vec{x}|$ (resp. $|\vec{P}|$). We adopt
the following useful abbreviations.

\begin{mathpar}
   x?(\vec{y}).P := x.(\vec{y})P \and  x\clift{\vec{P}} := x.\clift{\vec{P}}
   \and x!(y) := \lift{x}{\dropn{y}}
   \and \Pi_{i=0}^{n-1}P_i := P_0 | \ldots | P_{n-1}
\end{mathpar}

\subsubsection{Structural congruence}

\paragraph{Free and bound names and alpha-equivalence.} At the
core of structural equivalence is alpha-equivalence which identifies
process that are the same up to a change of variable. Formally, we
recognize the distinction between free and bound names. The free names
of a process, $\freenames{P}$, may be calculated recursively as
follows:

\begin{mathpar}
\freenames{\pzero} := \emptyset
  \and \\
  \freenames{x?(y).P} := \{ x \} \cup (\freenames{P} \setminus \{ y \})
  \and 
  \freenames{x!\langle P \rangle} := \{ x \} \cup \{ P \} 
  \and \\
  \freenames{P|Q} := \freenames{P} \cup \freenames{Q}
  \and \\
  \freenames{@{x}} := \{ x \}
\end{mathpar}

$\pi$
$\quotep{\pi}$

$\freenames{-} : \pi \to \mathcal{P}(\quotep{\pi})$

\begin{eqnarray*}
  \freenames{\pzero} & := & \emptyset \\
  \freenames{x?(y).P} & := & \{ x \} \cup (\freenames{P} \setminus \{ y \}) \\
  \freenames{x!\langle P \rangle} & := & \{ x \} \cup \{ P \} \\
  \freenames{P|Q} & := & \freenames{P} \cup \freenames{Q} \\
  \freenames{\dropn{x}} & := & \{ x \}
\end{eqnarray*}

The bound names of a process, $\boundnames{P}$, are those names occurring in $P$
that are not free. For example, in $x?(y).0$, the name $x$ is free, while $y$ is bound.

\begin{mathpar}
  \inferrule* [lab=monoidal-laws] {} { P|Q \equiv Q|P \and P|0 \equiv P \and P|(Q|R) \equiv (P|Q)|R }
\end{mathpar}

\begin{mathpar}
  \inferrule* [lab=alpha-equivalence] {} { (x)P \equiv (y)P\{y/x\} \and y \not\in \freenames{P} }
\end{mathpar}

\begin{definition}
Then two processes, $P,Q$, are alpha-equivalent if $P = Q\{\vec{y}/\vec{x}\}$ for
some $\vec{x} \in \boundnames{Q},\vec{y} \in \boundnames{P}$, where $Q\{\vec{y}/\vec{x}\}$
denotes the capture-avoiding substitution of $\vec{y}$ for $\vec{x}$ in $Q$.
\end{definition}

\begin{definition}
  The {\em structural congruence} \cite{SangiorgiWalker} , $\equiv$,
  between processes is the least congruence containing
  alpha-equivalence, satisfying the abelian monoid laws
  (associativity, commutativity and $\pzero$ as identity) for parallel
  composition $|$ and for summation $+$.
\end{definition}

\subsection{Name equivalence}

We take name equivalence, written $\nameeq$, to be the smallest
equivalence relation generated by the following rules.

\begin{mathpar}
\inferrule*[lab=Quote-drop]
{ }
{ \quotep{@{x}} \nameeq x }

\inferrule*[lab=Struct-equiv]
{ P \scong Q }
{ \quotep{P} \nameeq \quotep{Q} }
\end{mathpar}

The astute reader will have noticed that the mutual recursion of names
and processes imposes a mutual recursion on alpha-equivalence and
structural equivalence via name-equivalence. Fortunately, all of this
works out pleasantly and we may calculate in the natural way, free of
concern. The reader interested in the details is referred to the
appendix \ref{appendix:rho_details}.

\subsection{Substitution}

We use $\Proc$ for the set of processes, $\QProc$ for the set of
names, and $\id{\{}\vec{y} / \vec{x} \id{\}}$ to denote partial maps,
$s : \QProc \rightarrow \QProc$. A map, $s$ lifts, uniquely, to a map
on process terms, $\widehat{s} : \Proc \rightarrow \Proc$ by the
following equations.

\begin{mathpar}
  (0) \psubstp{Q}{P} := 0 \\
  (R \juxtap S) \psubstp{Q}{P}
  :=    
  (R)\psubstp{Q}{P} \juxtap (S) \psubstp{Q}{P} \\
  (x?(y).R) \psubstp{Q}{P}    
  :=    
  (x)\substp{Q}{P} (z)\concat( (R \psubstn{z}{y}) \psubstp{Q}{P} ) \\
  (\lift{x}{R}) \psubstp{Q}{P}  
  :=
  \lift{(x)\substp{Q}{P}}{ R \psubstp{Q}{P} } \\
%   (\dropn{x})  \psubstp{Q}{P}       
%   := 
%   \left\{ 
%     \begin{array}{ccc} 
%       \dropn{\quotep{Q}} & & x \nameeq \quotep{P} \\
%       \dropn{x} & & otherwise \\
%     \end{array}
%   \right. 
  (\dropn{x})  \psubstp{Q}{P}       
  := 
  \left\{ 
    \begin{array}{ccc} 
      Q & & x \nameeq \quotep{P} \\
      \dropn{x} & & otherwise \\
    \end{array}
  \right.
\end{mathpar}
 

where

\begin{eqnarray}
  (x)\id{\{} \lpquote Q \rpquote / \lpquote P \rpquote \id{\}}            = 
  \left\{ 
    \begin{array}{ccc}
      \lpquote Q \rpquote & & x \nameeq \lpquote P \rpquote \\
      x & & otherwise \\
    \end{array}
  \right. \nonumber
\end{eqnarray}

and $z$ is chosen distinct from $\quotep{P}$, $\quotep{Q}$, the free
names in $Q$, and all the names in $R$. Our $\alpha$-equivalence will
be built in the standard way from this substitution.

\begin{remark}\label{rem:no_self_referential_names}
  One consequence of these definitions is that $\forall P. \quotep{P}
  \not\in \freenames{P}$.
\end{remark}

\subsection{ Dynamic quote: an example }

Anticipating something of what's to come, consider applying the
substitution, $\widehat{\id{\{}u / z \id{\}}}$, to the following pair
of processes, $\lift{w}{y!(z)}$ and $w[ \lpquote y!(z) \rpquote ]$.

\begin{eqnarray}
	\lift{w}{y!(z)}\widehat{\id{\{}u / z \id{\}}}
		& = &
		\lift{w}{y!(u)} \nonumber\\
	w[ \lpquote y!(z) \rpquote ] \widehat{ \id{\{}u / z \id{\}} }
		& = &
		w[ \lpquote y!(z) \rpquote ] \nonumber
\end{eqnarray}

Because the body of the process between quotes is impervious to
substitution, we get radically different answers. In fact, by
examining the first process in an input context,
e.g. $x?(z).\lift{w}{y!(z)}$, we see that the process under the lift
operator may be shaped by prefixed inputs binding a name inside it. In
this sense, the lift operator will be seen as a way to dynamically
construct processes before reifying them as names.

Finally equipped with these standard features we can present the
dynamics of the calculus.

\subsubsection{Operational semantics} 

Finally, we introduce the computational dynamics. What marks these
algebras as distinct from other more traditionally studied algebraic
structures, e.g. vector spaces or polynomial rings, is the manner in
which dynamics is captured. In traditional structures, dynamics is typically
expressed through morphisms between such structures, as in linear maps
between vector spaces or morphisms between rings. In algebras
associated with the semantics of computation, the dynamics is
expressed as part of the algebraic structure itself, through a
reduction reduction relation typically denoted by $\red$. Below, we
give a recursive presentation of this relation for the calculus used
in the encoding.

$\red \subseteq \pi \times \pi$
$\red : \pi \to \mathcal{P}(\pi)$

\begin{mathpar}
  \inferrule* [lab=Comm] { \textsf{match}( x_{src}, x_{trgt} ) } { x_{trgt}?(y)P \; | \; x_{src}!\langle {Q} \rangle \red P\{\quotep{Q}/y}\} }
  \and \\
  \inferrule* [lab=Par] {{P} \red {P}'} {{{P} | {Q}} \red {{P}' | {Q}}}
  \and
  \inferrule* [lab=Equiv]{{{P} \scong {P}'} \andalso {{P}' \red {Q}'} \andalso {{Q}' \scong {Q}}}{{P} \red {Q}}
\end{mathpar}

\begin{eqnarray*}
  match_{\equiv} (\quotep{P},\quotep{Q}) & := & P \equiv Q \\
  match_{\dagger}(\quotep{P},\quotep{Q}) & := & \forall R. P|Q \red^{*} R => R \red^{*} 0 \\
  match_{K}(\quotep{P},\quotep{Q}) & := & K \mbox{ for some context } K
\end{eqnarray*}

$u?(x)P | u!\langle Q \rangle \red P\{\quotep{Q}/x\}$

%We write $\wred$ for $\red^*$, and $P\red$ if $\exists Q $ such that $ P \red Q$.
We write $P\red$ if $\exists Q $ such that $ P \red Q$ and $P\not\red$, otherwise.

\section{Replication}

As mentioned before, it is known that replication (and hence
recursion) can be implemented in a higher-order process algebra
\cite{SangiorgiWalker}. As our first example of calculation with the
machinery thus far presented we give the construction explicitly in
the {\rhoc}.

\begin{eqnarray}
	D_{x} & := & \prefix{x}{y}{(\binpar{\outputp{x}{y}}{@{y}})} \nonumber\\
	\bangp_{x}{P} & := & \binpar{{x}!\langle{\binpar{D_{x}}{P}}\rangle}{D_{x}} \nonumber
\end{eqnarray}

\begin{eqnarray}
	\bangp_{x}{P} & & \nonumber\\
	=
	& {x}!\langle{(\prefix{x}{y}{(\outputp{x}{y} | @{y})) | P}}\rangle 
	      | \prefix{x}{y}{(\outputp{x}{y} | @{y})} & \nonumber\\
	\red
	& (\outputp{x}{y} | @{y})\substn{\quotep{(\prefix{x}{y}{(@{y} | \outputp{x}{y})) | P}}}{y} & \nonumber\\
	=
	& \outputp{x}{\quotep{(\prefix{x}{y}{(\outputp{x}{y} | @{y})) | P}}}
	  | {(\prefix{x}{y}{(\outputp{x}{y} | @{y})) | P}} & \nonumber\\
	\red
	& \ldots & \nonumber\\
	\red^*
	& P | P | \ldots & \nonumber
\end{eqnarray}

Of course, this encoding, as an implementation, runs away, unfolding
$\bangp{P}$ eagerly. A lazier and more implementable replication
operator, restricted to input-guarded processes, may be obtained as follows.

\begin{eqnarray}
\bangp{\prefix{u}{v}{P}} 
	:= 
	\binpar{\lift{x}{\prefix{u}{v}{(\binpar{D(x)}{P})}}}{D(x)} \nonumber
\end{eqnarray}

\begin{remark}
  Note that the lazier definition still does not deal with summation
  or mixed summation (i.e. sums over input and output). The reader is
  invited to construct definitions of replication that deal with these
  features. 

  Further, the definitions are parameterized in a name, $x$. Can you,
  gentle reader, make a definition that eliminates this parameter and
  guarantees no accidental interaction between the replication
  machinery and the process being replicated -- i.e. no accidental
  sharing of names used by the process to get its work done and the
  name(s) used by the replication to effect copying. This latter
  revision of the definition of replication is crucial to obtaining
  the expected identity $!!P \sim !P$.
\end{remark}

\begin{remark}\label{rem:paradoxical_combinator}
  The reader familiar with the lambda calculus will have noticed the
  similarity between $D$ and the paradoxical combinator.

  [Ed. note: the existence of this seems to suggest we have to be more
  restrictive on the set of processes and names we admit if we are to
  support no-cloning.]
\end{remark}

\subsubsection{Bisimulation}

The computational dynamics gives rise to another kind of equivalence,
the equivalence of computational behavior. As previously mentioned
this is typically captured \emph{via} some form of bisimulation.

% The notion we use in this paper is weak barbed bisimulation
% \cite{milner91polyadicpi}.

The notion we use in this paper is derived from weak barbed
bisimulation \cite{milner91polyadicpi}. 

\begin{definition}
An \emph{observation relation}, $\downarrow_{\mathcal N}$, over a set
of names, $\mathcal N$, is the smallest relation satisfying the rules
below.

\infrule[Out-barb]{y \in {\mathcal N}, \; x \nameeq y}
		  {\outputp{x}{v} \downarrow_{\mathcal N} x}
\infrule[Par-barb]{\mbox{$P\downarrow_{\mathcal N} x$ or $Q\downarrow_{\mathcal N} x$}}
		  {\binpar{P}{Q} \downarrow_{\mathcal N} x}

We write $P \Downarrow_{\mathcal N} x$ if there is $Q$ such that 
$P \wred Q$ and $Q \downarrow_{\mathcal N} x$.
\end{definition}

\begin{definition}
%\label{def.bbisim}
An  ${\mathcal N}$-\emph{barbed bisimulation} over a set of names, ${\mathcal N}$, is a symmetric binary relation 
${\mathcal S}_{\mathcal N}$ between agents such that $P\rel{S}_{\mathcal N}Q$ implies:
\begin{enumerate}
\item If $P \red P'$ then $Q \wred Q'$ and $P'\rel{S}_{\mathcal N} Q'$.
\item If $P\downarrow_{\mathcal N} x$, then $Q\Downarrow_{\mathcal N} x$.
\end{enumerate}
$P$ is ${\mathcal N}$-barbed bisimilar to $Q$, written
$P \wbbisim_{\mathcal N} Q$, if $P \rel{S}_{\mathcal N} Q$ for some ${\mathcal N}$-barbed bisimulation ${\mathcal S}_{\mathcal N}$.
\end{definition}

$\mathcal{R} \subseteq \pi \times \pi$

$P \mathcal{R} Q => \forall P'. P \red P' \Rightarrow \exists Q'. Q \red Q', P' \mathcal{R} Q'$

$P \vdash x \Rightarrow Q \vdash x$

\begin{mathpar}
  \inferrule*[lab=Out-barb]{x \nameeq y}{{y}!\langle{Q}\rangle \vdash x}
  \and
  \inferrule*[lab=Par-barb]{\mbox{$P\vdash x$ or $Q\vdash x$}}{\binpar{P}{Q} \vdash x}
\end{mathpar}

\subsubsection{Contexts}

One of the principle advantages of computational calculi like the
$\pi$-calculus is a well-defined notion of context,
contextual-equivalence and a correlation between
contextual-equivalence and notions of bisimulation. The notion of
context allows the decomposition of a process into (sub-)process and
its syntactic environment, its context. Thus, a context may be
thought of as a process with a ``hole'' (written $\Box$) in it. The
application of a context $M$ to a process $P$, written $M[P]$, is
tantamount to filling the hole in $M$ with $P$. In this paper we do
not need the full weight of this theory, but do make use of the notion
of context in the proof the main theorem. 

\begin{mathpar}
  \inferrule* [lab=summation] {} {{M_{M},M_{N}} \bc \Box \;|\; x.M_{A} \;|\; M_{M}+M_{N}}
  \and
  \inferrule* [lab=agent] {} {{M_{A}} \bc (\vec{x})M_{P} \;| \; \clift{P_0,\ldots,M_{P},\ldots,P_N}}
  \and \\
  \inferrule* [lab=process] {} {{M_{P}} \bc M_{N} \;| \;P|M_{P} }
\end{mathpar} 

\begin{mathpar}
  \inferrule* [lab=sychronization] {} {M_{N} \bc \Box \;|\; x?M_{F} \;|\; x!M_{C}}
  \and
  \inferrule* [lab=abstraction] {} {{M_{F}} \bc (x)M_{P} }
  \and
  \inferrule* [lab=concretion] {} {{M_{C}} \bc \langle M_{P} \rangle }
  \and \\
  \inferrule* [lab=process] {} {{M_{P}} \bc M_{N} \;| \;P|M_{P} }
\end{mathpar}

\begin{definition}[contextual application] Given a context $M$, and
  process $P$, we define the \emph{contextual application}, $M[P] :=
  M\{P/\Box\}$. That is, the contextual application of M to P is the
  substitution of $P$ for $\Box$ in $M$.
\end{definition}

$\meaningof{-} : L \to \mathcal{P}(\pi)$

\begin{mathpar}
  \inferrule* [lab=collection] {} {\meaningof{true} = \pi, \and \meaningof{~E} = \pi \setminus \meaningof{E}, \and \meaningof{E_{1} \& E_{2}} = \meaningof{E_{1}} \cap \meaningof{E_{2}}}
\end{mathpar}

\begin{mathpar}
  \inferrule* [lab=structure] {} {\meaningof{0} = \{ P \in \pi | P \equiv 0 \}, \and \\ \meaningof{E_1 | E_2} = \{ P \in \pi | P \equiv P_{1} | P_{2}, P_{1} \in \meaningof{E_{1}}, P_{2} \in \meaningof{E_2}\} }
\end{mathpar}

\begin{mathpar}
 \inferrule* [lab=behavior] {} {\meaningof{\langle a?b \rangle E} = \{ P \in \pi | P \equiv Q | u?(y)P', \\ \and \\\\ \and \\ \;\;\; u \in \meaningof{a}, \forall z.P'\{z/y\} \in \meaningof{E\{z/b\}}\}, \and \\ \meaningof{a!E} = \{ P \in \pi | P \equiv Q | x!\langle P' \rangle, x \in \meaningof{a} P' \in \meaningof{E}\} }
\end{mathpar}

\begin{mathpar}
 \inferrule* [lab=nominal] {} {\meaningof{\quotep{E}} = \{ \quotep{P} \in \quotep{\pi} | P \in \meaningof{E} \}, \and \meaningof{\quotep{P}} = \{ \quotep{Q} \in \quotep{\pi} | P \equiv Q \} \and \\ \meaningof{@\quotep{E}} = \{ P \in \pi | P \equiv @x, x \in \meaningof{E} \}}
\end{mathpar}

\begin{eqnarray*}
  \\
  \meaningof{-} : TS \to ST
\end{eqnarray*}

\begin{eqnarray*}
  \\
  L : TS \to ST
\end{eqnarray*}

\begin{eqnarray*}
  \\
  P \models E \iff P \in \meaningof{E}
\end{eqnarray*}

\begin{eqnarray*}
  P \approx_{L} Q \iff \forall E \in L. P \models E \iff Q \models E
\end{eqnarray*}

\begin{eqnarray*}
  P \approx_{K} Q
\end{eqnarray*}

\begin{eqnarray*}
  P \approx Q
\end{eqnarray*}

$\approx_{K} = \approx = \approx_{L}$

\subsubsection{Contextual duality}

Note that contexts extend the quotation operation to a family of
operations from processes to names. Given a context, $M$, we can
define a \emph{nominal context}, $\quotep{M}$ by $\quotep{M}[P] :=
\quotep{M[P]}$. To foreshadow what is to come we observe that these
operations enjoy a duality with processes very much like the duality
between vectors and maps from vectors to scalars.

Further, because the calculus is essentially higher-order, we have a
correspondence between contexts and processes. More specifically,
given a name $x$ and a context $M$ we can construct $M^{*}_{x}$ such
that 

\begin{mathpar}
  M^{*}_{x} | \lift{x}{P} \red M[P]
\end{mathpar}

namely,

\begin{mathpar}
  M^{*}_{x} := x?(u).M[\dropn{u}]
\end{mathpar}

The dependence of $M^{*}_{x}$ on a name makes it an abstraction, 

\begin{mathpar}
  M^{*} := (x)x?(u).M[\dropn{u}]
\end{mathpar}

\subsection{Additional notation}

It will sometimes be convenient to denote the process a name
quotes. We already have the notation $x = \quotep{P}$, but it will be
convenient to introduce an alternate notation, $\procn{x}$, when we
want to emphasize the connection to the use of the name. Note that, by
virtue of name equivalence, $\quotep{\procn{x}} \nameeq x$; so, the
notation is consistent with previous definitions.

Further, because names have structure it is possible to effect
substitutions on the basis of that structure. This means we need to
upgrade our notation for substitutions, which we accomplish by
adapting comprehension notation. Thus,

\begin{mathpar}
  P\{ y / x : x \in S \}
\end{mathpar}

is interpreted to mean the process derived from P by replacing (in a
capture-avoiding manner) each occurrence of $x$ in $S$ by $y$. For example,

\begin{mathpar}
  P\{ \quotep{\procn{x}|\procn{x}} / x : x \in \freenames{P} \}
\end{mathpar}

will replace each (occurrence) of a free name $x$ in $P$ by
$\quotep{\procn{x}|\procn{x}}$.

Also, we will avail ourselves of the notation $x^{L}$ and $x^{R}$ to
denote injections of a name into disjoint copies of the name
space. There are numerous ways to accomplish this. One example can be
found in \cite{MeredithR05}. This notation overloads to vectors of
names: $\vec{x}^{\pi} := (x_{i}^{\pi} \; : \; 0 \leq i < |\vec{x}| )$ where $\pi \in \{L,R\}$.

We also use $P^{\Box} := P|\Box$.

In \cite{MeredithR05} an interpretation of the new operator is
given. It turns out that there are several possible interpretations
all enjoying the requisite algebraic properties of the operator (see
\cite{milner91polyadicpi}). We will therefore make liberal use of
$(\nu\; \vec{x})P$.

% subsection the_syntax_and_semantics_of_the_notation_system (end)   

\input{qm2pi.qmops} 

\input{qm2pi.sterngerlach} 

\input{qm2pi.metric} 

% section concurrent_process_calculi (end)

%\input{qm2pi.proofsketch}

% section proof sketch (end)

%\input{qm2pi.slviaknots} 

% section spatial logic via knots (end)

\input{qm2pi.conclusion}

% section conclusion (end)

%\input{qm2pi.dtcodes} 

% section wiring algorithm (end)

\input{qm2pi.ack} 

% section acknowledgments (end)

\newpage


\bibliographystyle{plain}   
\bibliography{../../biblios/main.bib}

\input{qm2pi.rhodetails}

\end{document}



% section front matter (end)

\section{Introduction}\label{sec:introduction} % (fold)
In this draft of the material i am going to have to dispense with the
usual writing conventions adopted in papers on these topics. i'm going
to have adopt whatever tone i need at the time i'm writing up the
calculations. Sometimes this may be very conversational; others it may
be the barest mathematical grunts; others still it may be that i have
lifted text from one of my other papers because the exposition of some
point was better said there. i hope that my readers are not unduly put
out by this decision. i'm not doing this to flout convention or be
rebellious. i find these calculations very technically challenging. To
keep everything going technically, something has to give; i have to
let go of some cognitive burden. So, the academic writing style --
with all of its trade-offs in terms of facilitating technical
communication -- is what i'm letting go of. Perhaps subsequent drafts
can be tightened and polished, but for now, i'm going to speak as if
we were sitting together in a coffee shop with a laptop, wifi and a
pad of paper and a pencil.

So, here's what i have to say. We -- you and i, comfortably ensconced
in our coffee shop and well-equipped with our tools -- can realize and
carry out the calculations of quantum mechanics over a very different
formal theory of dynamics, a formal theory of dynamics that
corresponds to a theory of concurrent computation with
\emph{reflection}. It has the advantage that the underlying theory is
already `quantized', but supports analogues all of the continuuous
operations. Strikingly, this underlying theory has recently been
connected with a notion of metric that we can show, by calculating
together, coincides with the metric induced by the inner product.

There are a lot of reasons why you might be interested in seeing
calculations of this form. Here's why i'm interested. For the past
several centuries there has been no competitor to the ``Newtonian''
account of dynamics. As a result the predominant share of accounts of
dynamical systems and situations have had to be formulated in terms of
the Newtonian machinery. i view this as an intellectually dangerous
position to occupy. Everything, despite it's intrinsic shape, turns
into a nail to be hit with this hammer. Recently, however, the theory
of computation has matured to the point where we have candidates for
theories of dynamics that offer very different perspective on
reasoning about dynamical systems and situations. Testing these
candidates against very successful accounts of dynamical situations,
like quantum mechanics, is going to give us some sense of how mature
they are and some measure of the quality of these accounts of
dynamics.

\subsection{Summary of contributions and outline of paper}

So, we're going to develop an interpretation of the operations of
quantum mechanics normally interpreted by Hilbert spaces and
operators. We're going to do this over a theory of computation. Note
that this is very different than the usual quantum computation program
which develops notions of computation over quantum mechanics. Rather,
we are developing a story that aligns with Wheeler's slogan: It from
Bit. To do this we will first provide an account of the theory of
computation at play here. Then we will dive into a calculation-driven
interpretation of the operations of quantum mechanics.

The reason we take this approach is that -- until very recently --
there hasn't been an axiomatic account of quantum mechanics. As a
result there has been no sharp delineation of the mathematical theory
supporting interpretation of the physical theory and the physical
theory, itself. So, ambient features of the maths are free to be
exploited (or supressed) without a real accounting of their physical
relevance. There is no sharp statement ``here's the physical theory''
qua \emph{theory} and ``here's the mathematical interpretation''
enabling a judgment of how faithful the interpretation is -- apart
from experimental observation. When there is an axiomatic account we
can judge how well a given mathematical formalism supports an
interpretation of the axioms, independent of
experimentation. Likewise, we can judge how well we have captured our
physical evidence and experience with our axiomatics, independent of
any specific mathematical implementation, with accidental detail that
may or may not have physical significance. 

In lieu of a fully fleshed out and vetted axiomatic account of quantum
mechanics, interpreting the operational notions in service of modeling
physical systems will have to suffice. In other words, we are not in
the business of providing a model of Hilbert spaces and operators. We
are in the business of providing a model of quantum mechanics because
we are motivated by testing our notions of dynamics against physical
theory; and, the predictive calculations of the physical theory must
serve as the best formulation -- shy of a fully fleshed out axiomatic
account -- of the physical theory itself (as they have for scientific
theories since time immemorial). Put another way, despite a
whole-hearted commitment to an It-from-Bit ontology, we are firmly
aligned with the shut-up-and-calculate camp as the best way to obtain
results either from the physical perspective or as a quality assurance
measure of our fledgling theory of dynamics.

In detail, we present a reflective process calculus. Then we develop
intuitive correspondences between the notions available in this
calculus and the usual physical notions supporting quantum mechanical
calculations. Thus, 

\begin{table}[htp]
  \center{
    \fbox{
      \begin{tabular}{c|c}
        quantum mechanics & process calculus \\
        \hline
        scalar & name \\
        state vector & process \\
        dual & contextual duals \\
        matrix & formal sums of process-context-dual pairs \\
        orthogonality & process annihilation \\
        inner product & execution-formula + quoting
      \end{tabular}
    }
  }
  \caption{QM - process calculi correspondences}
\end{table}

Then we tighten up these intuitions to operational definitions. We
employ the Dirac notation as the best proxy we can find for an
abstract syntax of the quantum mechanical notions. The definitions we
develop put us in contact with equational constraints coming from the
theory that we demonstrate the definitions and calculations satisfy.

This puts us in a position to shut up and calculate for the
Stern-Gerlach experimental set up, showing how these predictive
calculations become calculations on processes in our theory of a
reflective process calculus.

Penultimately, we demonstrate that the notion of metric coming from
the inner product coincides with the notion of metric available from
the theory of bisimulation. This demonstration gives us the right to
think of space as arising from behavior. Finally, we consider where we
might go from the new vantage point we have obtained.

% section introduction (end) 
 
% section introduction (end)

% \documentclass[12pt]{llncs}
%\documentclass{jktr}

\usepackage[pdftex]{hyperref}                   
\usepackage {listings}
\usepackage {mathpartir}
\usepackage{bcprules}
%\usepackage{listings}
                       
\usepackage{graphicx} 
%\usepackage[margins=2.5cm,nohead,nofoot]{geometry}
%\usepackage{geometry}
\usepackage{amsfonts}
\usepackage{amstext}
\usepackage{latexsym}
\usepackage{amssymb}
\usepackage{color}


%\include{myPreamble}
\include{qm2pi.local} 

%\ifpdf
%\usepackage[pdftex]{graphicx}
%\else
%\usepackage{graphicx}
%\fi

 % \ifpdf
%  \usepackage{pdfsync}
%  \if


%\title{Brief Article}
%\author{David F. Snyder}
%\author{L.G. Meredith}

%\address{Dept. of Math., Texas State University--San Marcos, San Marcos, TX 78666}
       
\pagestyle{empty}


\begin{document}

\lstset{language=[Objective]Caml,frame=shadowbox}

\input{qm2pi.front}

% section front matter (end)

\input{qm2pi.intro} 
 
% section introduction (end)

% \input{qm2pi.knotations} 

% section notation (end)

\input{qm2pi.process.calculi} 

% section concurrent_process_calculi_and_spatial_logics_ (end)
    
%\input{qm2pi.knots2pi} 

%\input{qm2pi.trefoil} 

%\input{qm2pi.mainthm} 

% subsection basic_interpretation (end)

%\input{qm2pi.rho.presentation} 
\subsection{The syntax and semantics of the notation system}\label{sub:the_syntax_and_semantics_of_the_notation_system} % (fold)

We now summarize a technical presentation of the calculus that
embodies our theory of dynamics. The typical presentation of such a
calculus follows the style of giving generators and relations on
them. The grammar, below, describing term constructors, freely
generates the set of processes, $\Proc$. This set is then quotiented
by a relation known as structural congruence and it is over this set
that the notion of dynamics is expressed. This presentation is
essentially that of \cite{MeredithR05} with the addition of
polyadicity and summation. For readability we have relegated some of
the technical subtleties to an appendix.

\subsubsection{Process grammar}\label{subsub:process_grammar}

\begin{mathpar}
  \inferrule* [lab=synchronization] {} {{M} \bc \pzero \;|\; x?F \;|\; x!C }
  \and
  \inferrule* [lab=abstraction] {} {{F} \bc (x)P}
  \and
  \inferrule* [lab=concretion] {} {{C} \bc \langle Q \rangle}
  \and
  \inferrule* [lab=process] {} {{P,Q} \bc M \;| \;P|Q \;|\; @{x}}
  \and
  \inferrule* [lab=name] {} {{x} \bc \quotep{P}}
\end{mathpar} 

Note that $\vec{x}$ (resp. $\vec{P}$) denotes a vector of names
(resp. processes) of length $|\vec{x}|$ (resp. $|\vec{P}|$). We adopt
the following useful abbreviations.

\begin{mathpar}
   x?(\vec{y}).P := x.(\vec{y})P \and  x\clift{\vec{P}} := x.\clift{\vec{P}}
   \and x!(y) := \lift{x}{\dropn{y}}
   \and \Pi_{i=0}^{n-1}P_i := P_0 | \ldots | P_{n-1}
\end{mathpar}

\subsubsection{Structural congruence}

\paragraph{Free and bound names and alpha-equivalence.} At the
core of structural equivalence is alpha-equivalence which identifies
process that are the same up to a change of variable. Formally, we
recognize the distinction between free and bound names. The free names
of a process, $\freenames{P}$, may be calculated recursively as
follows:

\begin{mathpar}
\freenames{\pzero} := \emptyset
  \and \\
  \freenames{x?(y).P} := \{ x \} \cup (\freenames{P} \setminus \{ y \})
  \and 
  \freenames{x!\langle P \rangle} := \{ x \} \cup \{ P \} 
  \and \\
  \freenames{P|Q} := \freenames{P} \cup \freenames{Q}
  \and \\
  \freenames{@{x}} := \{ x \}
\end{mathpar}

$\pi$
$\quotep{\pi}$

$\freenames{-} : \pi \to \mathcal{P}(\quotep{\pi})$

\begin{eqnarray*}
  \freenames{\pzero} & := & \emptyset \\
  \freenames{x?(y).P} & := & \{ x \} \cup (\freenames{P} \setminus \{ y \}) \\
  \freenames{x!\langle P \rangle} & := & \{ x \} \cup \{ P \} \\
  \freenames{P|Q} & := & \freenames{P} \cup \freenames{Q} \\
  \freenames{\dropn{x}} & := & \{ x \}
\end{eqnarray*}

The bound names of a process, $\boundnames{P}$, are those names occurring in $P$
that are not free. For example, in $x?(y).0$, the name $x$ is free, while $y$ is bound.

\begin{mathpar}
  \inferrule* [lab=monoidal-laws] {} { P|Q \equiv Q|P \and P|0 \equiv P \and P|(Q|R) \equiv (P|Q)|R }
\end{mathpar}

\begin{mathpar}
  \inferrule* [lab=alpha-equivalence] {} { (x)P \equiv (y)P\{y/x\} \and y \not\in \freenames{P} }
\end{mathpar}

\begin{definition}
Then two processes, $P,Q$, are alpha-equivalent if $P = Q\{\vec{y}/\vec{x}\}$ for
some $\vec{x} \in \boundnames{Q},\vec{y} \in \boundnames{P}$, where $Q\{\vec{y}/\vec{x}\}$
denotes the capture-avoiding substitution of $\vec{y}$ for $\vec{x}$ in $Q$.
\end{definition}

\begin{definition}
  The {\em structural congruence} \cite{SangiorgiWalker} , $\equiv$,
  between processes is the least congruence containing
  alpha-equivalence, satisfying the abelian monoid laws
  (associativity, commutativity and $\pzero$ as identity) for parallel
  composition $|$ and for summation $+$.
\end{definition}

\subsection{Name equivalence}

We take name equivalence, written $\nameeq$, to be the smallest
equivalence relation generated by the following rules.

\begin{mathpar}
\inferrule*[lab=Quote-drop]
{ }
{ \quotep{@{x}} \nameeq x }

\inferrule*[lab=Struct-equiv]
{ P \scong Q }
{ \quotep{P} \nameeq \quotep{Q} }
\end{mathpar}

The astute reader will have noticed that the mutual recursion of names
and processes imposes a mutual recursion on alpha-equivalence and
structural equivalence via name-equivalence. Fortunately, all of this
works out pleasantly and we may calculate in the natural way, free of
concern. The reader interested in the details is referred to the
appendix \ref{appendix:rho_details}.

\subsection{Substitution}

We use $\Proc$ for the set of processes, $\QProc$ for the set of
names, and $\id{\{}\vec{y} / \vec{x} \id{\}}$ to denote partial maps,
$s : \QProc \rightarrow \QProc$. A map, $s$ lifts, uniquely, to a map
on process terms, $\widehat{s} : \Proc \rightarrow \Proc$ by the
following equations.

\begin{mathpar}
  (0) \psubstp{Q}{P} := 0 \\
  (R \juxtap S) \psubstp{Q}{P}
  :=    
  (R)\psubstp{Q}{P} \juxtap (S) \psubstp{Q}{P} \\
  (x?(y).R) \psubstp{Q}{P}    
  :=    
  (x)\substp{Q}{P} (z)\concat( (R \psubstn{z}{y}) \psubstp{Q}{P} ) \\
  (\lift{x}{R}) \psubstp{Q}{P}  
  :=
  \lift{(x)\substp{Q}{P}}{ R \psubstp{Q}{P} } \\
%   (\dropn{x})  \psubstp{Q}{P}       
%   := 
%   \left\{ 
%     \begin{array}{ccc} 
%       \dropn{\quotep{Q}} & & x \nameeq \quotep{P} \\
%       \dropn{x} & & otherwise \\
%     \end{array}
%   \right. 
  (\dropn{x})  \psubstp{Q}{P}       
  := 
  \left\{ 
    \begin{array}{ccc} 
      Q & & x \nameeq \quotep{P} \\
      \dropn{x} & & otherwise \\
    \end{array}
  \right.
\end{mathpar}
 

where

\begin{eqnarray}
  (x)\id{\{} \lpquote Q \rpquote / \lpquote P \rpquote \id{\}}            = 
  \left\{ 
    \begin{array}{ccc}
      \lpquote Q \rpquote & & x \nameeq \lpquote P \rpquote \\
      x & & otherwise \\
    \end{array}
  \right. \nonumber
\end{eqnarray}

and $z$ is chosen distinct from $\quotep{P}$, $\quotep{Q}$, the free
names in $Q$, and all the names in $R$. Our $\alpha$-equivalence will
be built in the standard way from this substitution.

\begin{remark}\label{rem:no_self_referential_names}
  One consequence of these definitions is that $\forall P. \quotep{P}
  \not\in \freenames{P}$.
\end{remark}

\subsection{ Dynamic quote: an example }

Anticipating something of what's to come, consider applying the
substitution, $\widehat{\id{\{}u / z \id{\}}}$, to the following pair
of processes, $\lift{w}{y!(z)}$ and $w[ \lpquote y!(z) \rpquote ]$.

\begin{eqnarray}
	\lift{w}{y!(z)}\widehat{\id{\{}u / z \id{\}}}
		& = &
		\lift{w}{y!(u)} \nonumber\\
	w[ \lpquote y!(z) \rpquote ] \widehat{ \id{\{}u / z \id{\}} }
		& = &
		w[ \lpquote y!(z) \rpquote ] \nonumber
\end{eqnarray}

Because the body of the process between quotes is impervious to
substitution, we get radically different answers. In fact, by
examining the first process in an input context,
e.g. $x?(z).\lift{w}{y!(z)}$, we see that the process under the lift
operator may be shaped by prefixed inputs binding a name inside it. In
this sense, the lift operator will be seen as a way to dynamically
construct processes before reifying them as names.

Finally equipped with these standard features we can present the
dynamics of the calculus.

\subsubsection{Operational semantics} 

Finally, we introduce the computational dynamics. What marks these
algebras as distinct from other more traditionally studied algebraic
structures, e.g. vector spaces or polynomial rings, is the manner in
which dynamics is captured. In traditional structures, dynamics is typically
expressed through morphisms between such structures, as in linear maps
between vector spaces or morphisms between rings. In algebras
associated with the semantics of computation, the dynamics is
expressed as part of the algebraic structure itself, through a
reduction reduction relation typically denoted by $\red$. Below, we
give a recursive presentation of this relation for the calculus used
in the encoding.

$\red \subseteq \pi \times \pi$
$\red : \pi \to \mathcal{P}(\pi)$

\begin{mathpar}
  \inferrule* [lab=Comm] { \textsf{match}( x_{src}, x_{trgt} ) } { x_{trgt}?(y)P \; | \; x_{src}!\langle {Q} \rangle \red P\{\quotep{Q}/y}\} }
  \and \\
  \inferrule* [lab=Par] {{P} \red {P}'} {{{P} | {Q}} \red {{P}' | {Q}}}
  \and
  \inferrule* [lab=Equiv]{{{P} \scong {P}'} \andalso {{P}' \red {Q}'} \andalso {{Q}' \scong {Q}}}{{P} \red {Q}}
\end{mathpar}

\begin{eqnarray*}
  match_{\equiv} (\quotep{P},\quotep{Q}) & := & P \equiv Q \\
  match_{\dagger}(\quotep{P},\quotep{Q}) & := & \forall R. P|Q \red^{*} R => R \red^{*} 0 \\
  match_{K}(\quotep{P},\quotep{Q}) & := & K \mbox{ for some context } K
\end{eqnarray*}

$u?(x)P | u!\langle Q \rangle \red P\{\quotep{Q}/x\}$

%We write $\wred$ for $\red^*$, and $P\red$ if $\exists Q $ such that $ P \red Q$.
We write $P\red$ if $\exists Q $ such that $ P \red Q$ and $P\not\red$, otherwise.

\section{Replication}

As mentioned before, it is known that replication (and hence
recursion) can be implemented in a higher-order process algebra
\cite{SangiorgiWalker}. As our first example of calculation with the
machinery thus far presented we give the construction explicitly in
the {\rhoc}.

\begin{eqnarray}
	D_{x} & := & \prefix{x}{y}{(\binpar{\outputp{x}{y}}{@{y}})} \nonumber\\
	\bangp_{x}{P} & := & \binpar{{x}!\langle{\binpar{D_{x}}{P}}\rangle}{D_{x}} \nonumber
\end{eqnarray}

\begin{eqnarray}
	\bangp_{x}{P} & & \nonumber\\
	=
	& {x}!\langle{(\prefix{x}{y}{(\outputp{x}{y} | @{y})) | P}}\rangle 
	      | \prefix{x}{y}{(\outputp{x}{y} | @{y})} & \nonumber\\
	\red
	& (\outputp{x}{y} | @{y})\substn{\quotep{(\prefix{x}{y}{(@{y} | \outputp{x}{y})) | P}}}{y} & \nonumber\\
	=
	& \outputp{x}{\quotep{(\prefix{x}{y}{(\outputp{x}{y} | @{y})) | P}}}
	  | {(\prefix{x}{y}{(\outputp{x}{y} | @{y})) | P}} & \nonumber\\
	\red
	& \ldots & \nonumber\\
	\red^*
	& P | P | \ldots & \nonumber
\end{eqnarray}

Of course, this encoding, as an implementation, runs away, unfolding
$\bangp{P}$ eagerly. A lazier and more implementable replication
operator, restricted to input-guarded processes, may be obtained as follows.

\begin{eqnarray}
\bangp{\prefix{u}{v}{P}} 
	:= 
	\binpar{\lift{x}{\prefix{u}{v}{(\binpar{D(x)}{P})}}}{D(x)} \nonumber
\end{eqnarray}

\begin{remark}
  Note that the lazier definition still does not deal with summation
  or mixed summation (i.e. sums over input and output). The reader is
  invited to construct definitions of replication that deal with these
  features. 

  Further, the definitions are parameterized in a name, $x$. Can you,
  gentle reader, make a definition that eliminates this parameter and
  guarantees no accidental interaction between the replication
  machinery and the process being replicated -- i.e. no accidental
  sharing of names used by the process to get its work done and the
  name(s) used by the replication to effect copying. This latter
  revision of the definition of replication is crucial to obtaining
  the expected identity $!!P \sim !P$.
\end{remark}

\begin{remark}\label{rem:paradoxical_combinator}
  The reader familiar with the lambda calculus will have noticed the
  similarity between $D$ and the paradoxical combinator.

  [Ed. note: the existence of this seems to suggest we have to be more
  restrictive on the set of processes and names we admit if we are to
  support no-cloning.]
\end{remark}

\subsubsection{Bisimulation}

The computational dynamics gives rise to another kind of equivalence,
the equivalence of computational behavior. As previously mentioned
this is typically captured \emph{via} some form of bisimulation.

% The notion we use in this paper is weak barbed bisimulation
% \cite{milner91polyadicpi}.

The notion we use in this paper is derived from weak barbed
bisimulation \cite{milner91polyadicpi}. 

\begin{definition}
An \emph{observation relation}, $\downarrow_{\mathcal N}$, over a set
of names, $\mathcal N$, is the smallest relation satisfying the rules
below.

\infrule[Out-barb]{y \in {\mathcal N}, \; x \nameeq y}
		  {\outputp{x}{v} \downarrow_{\mathcal N} x}
\infrule[Par-barb]{\mbox{$P\downarrow_{\mathcal N} x$ or $Q\downarrow_{\mathcal N} x$}}
		  {\binpar{P}{Q} \downarrow_{\mathcal N} x}

We write $P \Downarrow_{\mathcal N} x$ if there is $Q$ such that 
$P \wred Q$ and $Q \downarrow_{\mathcal N} x$.
\end{definition}

\begin{definition}
%\label{def.bbisim}
An  ${\mathcal N}$-\emph{barbed bisimulation} over a set of names, ${\mathcal N}$, is a symmetric binary relation 
${\mathcal S}_{\mathcal N}$ between agents such that $P\rel{S}_{\mathcal N}Q$ implies:
\begin{enumerate}
\item If $P \red P'$ then $Q \wred Q'$ and $P'\rel{S}_{\mathcal N} Q'$.
\item If $P\downarrow_{\mathcal N} x$, then $Q\Downarrow_{\mathcal N} x$.
\end{enumerate}
$P$ is ${\mathcal N}$-barbed bisimilar to $Q$, written
$P \wbbisim_{\mathcal N} Q$, if $P \rel{S}_{\mathcal N} Q$ for some ${\mathcal N}$-barbed bisimulation ${\mathcal S}_{\mathcal N}$.
\end{definition}

$\mathcal{R} \subseteq \pi \times \pi$

$P \mathcal{R} Q => \forall P'. P \red P' \Rightarrow \exists Q'. Q \red Q', P' \mathcal{R} Q'$

$P \vdash x \Rightarrow Q \vdash x$

\begin{mathpar}
  \inferrule*[lab=Out-barb]{x \nameeq y}{{y}!\langle{Q}\rangle \vdash x}
  \and
  \inferrule*[lab=Par-barb]{\mbox{$P\vdash x$ or $Q\vdash x$}}{\binpar{P}{Q} \vdash x}
\end{mathpar}

\subsubsection{Contexts}

One of the principle advantages of computational calculi like the
$\pi$-calculus is a well-defined notion of context,
contextual-equivalence and a correlation between
contextual-equivalence and notions of bisimulation. The notion of
context allows the decomposition of a process into (sub-)process and
its syntactic environment, its context. Thus, a context may be
thought of as a process with a ``hole'' (written $\Box$) in it. The
application of a context $M$ to a process $P$, written $M[P]$, is
tantamount to filling the hole in $M$ with $P$. In this paper we do
not need the full weight of this theory, but do make use of the notion
of context in the proof the main theorem. 

\begin{mathpar}
  \inferrule* [lab=summation] {} {{M_{M},M_{N}} \bc \Box \;|\; x.M_{A} \;|\; M_{M}+M_{N}}
  \and
  \inferrule* [lab=agent] {} {{M_{A}} \bc (\vec{x})M_{P} \;| \; \clift{P_0,\ldots,M_{P},\ldots,P_N}}
  \and \\
  \inferrule* [lab=process] {} {{M_{P}} \bc M_{N} \;| \;P|M_{P} }
\end{mathpar} 

\begin{mathpar}
  \inferrule* [lab=sychronization] {} {M_{N} \bc \Box \;|\; x?M_{F} \;|\; x!M_{C}}
  \and
  \inferrule* [lab=abstraction] {} {{M_{F}} \bc (x)M_{P} }
  \and
  \inferrule* [lab=concretion] {} {{M_{C}} \bc \langle M_{P} \rangle }
  \and \\
  \inferrule* [lab=process] {} {{M_{P}} \bc M_{N} \;| \;P|M_{P} }
\end{mathpar}

\begin{definition}[contextual application] Given a context $M$, and
  process $P$, we define the \emph{contextual application}, $M[P] :=
  M\{P/\Box\}$. That is, the contextual application of M to P is the
  substitution of $P$ for $\Box$ in $M$.
\end{definition}

$\meaningof{-} : L \to \mathcal{P}(\pi)$

\begin{mathpar}
  \inferrule* [lab=collection] {} {\meaningof{true} = \pi, \and \meaningof{~E} = \pi \setminus \meaningof{E}, \and \meaningof{E_{1} \& E_{2}} = \meaningof{E_{1}} \cap \meaningof{E_{2}}}
\end{mathpar}

\begin{mathpar}
  \inferrule* [lab=structure] {} {\meaningof{0} = \{ P \in \pi | P \equiv 0 \}, \and \\ \meaningof{E_1 | E_2} = \{ P \in \pi | P \equiv P_{1} | P_{2}, P_{1} \in \meaningof{E_{1}}, P_{2} \in \meaningof{E_2}\} }
\end{mathpar}

\begin{mathpar}
 \inferrule* [lab=behavior] {} {\meaningof{\langle a?b \rangle E} = \{ P \in \pi | P \equiv Q | u?(y)P', \\ \and \\\\ \and \\ \;\;\; u \in \meaningof{a}, \forall z.P'\{z/y\} \in \meaningof{E\{z/b\}}\}, \and \\ \meaningof{a!E} = \{ P \in \pi | P \equiv Q | x!\langle P' \rangle, x \in \meaningof{a} P' \in \meaningof{E}\} }
\end{mathpar}

\begin{mathpar}
 \inferrule* [lab=nominal] {} {\meaningof{\quotep{E}} = \{ \quotep{P} \in \quotep{\pi} | P \in \meaningof{E} \}, \and \meaningof{\quotep{P}} = \{ \quotep{Q} \in \quotep{\pi} | P \equiv Q \} \and \\ \meaningof{@\quotep{E}} = \{ P \in \pi | P \equiv @x, x \in \meaningof{E} \}}
\end{mathpar}

\begin{eqnarray*}
  \\
  \meaningof{-} : TS \to ST
\end{eqnarray*}

\begin{eqnarray*}
  \\
  L : TS \to ST
\end{eqnarray*}

\begin{eqnarray*}
  \\
  P \models E \iff P \in \meaningof{E}
\end{eqnarray*}

\begin{eqnarray*}
  P \approx_{L} Q \iff \forall E \in L. P \models E \iff Q \models E
\end{eqnarray*}

\begin{eqnarray*}
  P \approx_{K} Q
\end{eqnarray*}

\begin{eqnarray*}
  P \approx Q
\end{eqnarray*}

$\approx_{K} = \approx = \approx_{L}$

\subsubsection{Contextual duality}

Note that contexts extend the quotation operation to a family of
operations from processes to names. Given a context, $M$, we can
define a \emph{nominal context}, $\quotep{M}$ by $\quotep{M}[P] :=
\quotep{M[P]}$. To foreshadow what is to come we observe that these
operations enjoy a duality with processes very much like the duality
between vectors and maps from vectors to scalars.

Further, because the calculus is essentially higher-order, we have a
correspondence between contexts and processes. More specifically,
given a name $x$ and a context $M$ we can construct $M^{*}_{x}$ such
that 

\begin{mathpar}
  M^{*}_{x} | \lift{x}{P} \red M[P]
\end{mathpar}

namely,

\begin{mathpar}
  M^{*}_{x} := x?(u).M[\dropn{u}]
\end{mathpar}

The dependence of $M^{*}_{x}$ on a name makes it an abstraction, 

\begin{mathpar}
  M^{*} := (x)x?(u).M[\dropn{u}]
\end{mathpar}

\subsection{Additional notation}

It will sometimes be convenient to denote the process a name
quotes. We already have the notation $x = \quotep{P}$, but it will be
convenient to introduce an alternate notation, $\procn{x}$, when we
want to emphasize the connection to the use of the name. Note that, by
virtue of name equivalence, $\quotep{\procn{x}} \nameeq x$; so, the
notation is consistent with previous definitions.

Further, because names have structure it is possible to effect
substitutions on the basis of that structure. This means we need to
upgrade our notation for substitutions, which we accomplish by
adapting comprehension notation. Thus,

\begin{mathpar}
  P\{ y / x : x \in S \}
\end{mathpar}

is interpreted to mean the process derived from P by replacing (in a
capture-avoiding manner) each occurrence of $x$ in $S$ by $y$. For example,

\begin{mathpar}
  P\{ \quotep{\procn{x}|\procn{x}} / x : x \in \freenames{P} \}
\end{mathpar}

will replace each (occurrence) of a free name $x$ in $P$ by
$\quotep{\procn{x}|\procn{x}}$.

Also, we will avail ourselves of the notation $x^{L}$ and $x^{R}$ to
denote injections of a name into disjoint copies of the name
space. There are numerous ways to accomplish this. One example can be
found in \cite{MeredithR05}. This notation overloads to vectors of
names: $\vec{x}^{\pi} := (x_{i}^{\pi} \; : \; 0 \leq i < |\vec{x}| )$ where $\pi \in \{L,R\}$.

We also use $P^{\Box} := P|\Box$.

In \cite{MeredithR05} an interpretation of the new operator is
given. It turns out that there are several possible interpretations
all enjoying the requisite algebraic properties of the operator (see
\cite{milner91polyadicpi}). We will therefore make liberal use of
$(\nu\; \vec{x})P$.

% subsection the_syntax_and_semantics_of_the_notation_system (end)   

\input{qm2pi.qmops} 

\input{qm2pi.sterngerlach} 

\input{qm2pi.metric} 

% section concurrent_process_calculi (end)

%\input{qm2pi.proofsketch}

% section proof sketch (end)

%\input{qm2pi.slviaknots} 

% section spatial logic via knots (end)

\input{qm2pi.conclusion}

% section conclusion (end)

%\input{qm2pi.dtcodes} 

% section wiring algorithm (end)

\input{qm2pi.ack} 

% section acknowledgments (end)

\newpage


\bibliographystyle{plain}   
\bibliography{../../biblios/main.bib}

\input{qm2pi.rhodetails}

\end{document}

 

% section notation (end)

\input{qm2pi.process.calculi} 

% section concurrent_process_calculi_and_spatial_logics_ (end)
    
%\documentclass[12pt]{llncs}
%\documentclass{jktr}

\usepackage[pdftex]{hyperref}                   
\usepackage {listings}
\usepackage {mathpartir}
\usepackage{bcprules}
%\usepackage{listings}
                       
\usepackage{graphicx} 
%\usepackage[margins=2.5cm,nohead,nofoot]{geometry}
%\usepackage{geometry}
\usepackage{amsfonts}
\usepackage{amstext}
\usepackage{latexsym}
\usepackage{amssymb}
\usepackage{color}


%\include{myPreamble}
\include{qm2pi.local} 

%\ifpdf
%\usepackage[pdftex]{graphicx}
%\else
%\usepackage{graphicx}
%\fi

 % \ifpdf
%  \usepackage{pdfsync}
%  \if


%\title{Brief Article}
%\author{David F. Snyder}
%\author{L.G. Meredith}

%\address{Dept. of Math., Texas State University--San Marcos, San Marcos, TX 78666}
       
\pagestyle{empty}


\begin{document}

\lstset{language=[Objective]Caml,frame=shadowbox}

\input{qm2pi.front}

% section front matter (end)

\input{qm2pi.intro} 
 
% section introduction (end)

% \input{qm2pi.knotations} 

% section notation (end)

\input{qm2pi.process.calculi} 

% section concurrent_process_calculi_and_spatial_logics_ (end)
    
%\input{qm2pi.knots2pi} 

%\input{qm2pi.trefoil} 

%\input{qm2pi.mainthm} 

% subsection basic_interpretation (end)

%\input{qm2pi.rho.presentation} 
\subsection{The syntax and semantics of the notation system}\label{sub:the_syntax_and_semantics_of_the_notation_system} % (fold)

We now summarize a technical presentation of the calculus that
embodies our theory of dynamics. The typical presentation of such a
calculus follows the style of giving generators and relations on
them. The grammar, below, describing term constructors, freely
generates the set of processes, $\Proc$. This set is then quotiented
by a relation known as structural congruence and it is over this set
that the notion of dynamics is expressed. This presentation is
essentially that of \cite{MeredithR05} with the addition of
polyadicity and summation. For readability we have relegated some of
the technical subtleties to an appendix.

\subsubsection{Process grammar}\label{subsub:process_grammar}

\begin{mathpar}
  \inferrule* [lab=synchronization] {} {{M} \bc \pzero \;|\; x?F \;|\; x!C }
  \and
  \inferrule* [lab=abstraction] {} {{F} \bc (x)P}
  \and
  \inferrule* [lab=concretion] {} {{C} \bc \langle Q \rangle}
  \and
  \inferrule* [lab=process] {} {{P,Q} \bc M \;| \;P|Q \;|\; @{x}}
  \and
  \inferrule* [lab=name] {} {{x} \bc \quotep{P}}
\end{mathpar} 

Note that $\vec{x}$ (resp. $\vec{P}$) denotes a vector of names
(resp. processes) of length $|\vec{x}|$ (resp. $|\vec{P}|$). We adopt
the following useful abbreviations.

\begin{mathpar}
   x?(\vec{y}).P := x.(\vec{y})P \and  x\clift{\vec{P}} := x.\clift{\vec{P}}
   \and x!(y) := \lift{x}{\dropn{y}}
   \and \Pi_{i=0}^{n-1}P_i := P_0 | \ldots | P_{n-1}
\end{mathpar}

\subsubsection{Structural congruence}

\paragraph{Free and bound names and alpha-equivalence.} At the
core of structural equivalence is alpha-equivalence which identifies
process that are the same up to a change of variable. Formally, we
recognize the distinction between free and bound names. The free names
of a process, $\freenames{P}$, may be calculated recursively as
follows:

\begin{mathpar}
\freenames{\pzero} := \emptyset
  \and \\
  \freenames{x?(y).P} := \{ x \} \cup (\freenames{P} \setminus \{ y \})
  \and 
  \freenames{x!\langle P \rangle} := \{ x \} \cup \{ P \} 
  \and \\
  \freenames{P|Q} := \freenames{P} \cup \freenames{Q}
  \and \\
  \freenames{@{x}} := \{ x \}
\end{mathpar}

$\pi$
$\quotep{\pi}$

$\freenames{-} : \pi \to \mathcal{P}(\quotep{\pi})$

\begin{eqnarray*}
  \freenames{\pzero} & := & \emptyset \\
  \freenames{x?(y).P} & := & \{ x \} \cup (\freenames{P} \setminus \{ y \}) \\
  \freenames{x!\langle P \rangle} & := & \{ x \} \cup \{ P \} \\
  \freenames{P|Q} & := & \freenames{P} \cup \freenames{Q} \\
  \freenames{\dropn{x}} & := & \{ x \}
\end{eqnarray*}

The bound names of a process, $\boundnames{P}$, are those names occurring in $P$
that are not free. For example, in $x?(y).0$, the name $x$ is free, while $y$ is bound.

\begin{mathpar}
  \inferrule* [lab=monoidal-laws] {} { P|Q \equiv Q|P \and P|0 \equiv P \and P|(Q|R) \equiv (P|Q)|R }
\end{mathpar}

\begin{mathpar}
  \inferrule* [lab=alpha-equivalence] {} { (x)P \equiv (y)P\{y/x\} \and y \not\in \freenames{P} }
\end{mathpar}

\begin{definition}
Then two processes, $P,Q$, are alpha-equivalent if $P = Q\{\vec{y}/\vec{x}\}$ for
some $\vec{x} \in \boundnames{Q},\vec{y} \in \boundnames{P}$, where $Q\{\vec{y}/\vec{x}\}$
denotes the capture-avoiding substitution of $\vec{y}$ for $\vec{x}$ in $Q$.
\end{definition}

\begin{definition}
  The {\em structural congruence} \cite{SangiorgiWalker} , $\equiv$,
  between processes is the least congruence containing
  alpha-equivalence, satisfying the abelian monoid laws
  (associativity, commutativity and $\pzero$ as identity) for parallel
  composition $|$ and for summation $+$.
\end{definition}

\subsection{Name equivalence}

We take name equivalence, written $\nameeq$, to be the smallest
equivalence relation generated by the following rules.

\begin{mathpar}
\inferrule*[lab=Quote-drop]
{ }
{ \quotep{@{x}} \nameeq x }

\inferrule*[lab=Struct-equiv]
{ P \scong Q }
{ \quotep{P} \nameeq \quotep{Q} }
\end{mathpar}

The astute reader will have noticed that the mutual recursion of names
and processes imposes a mutual recursion on alpha-equivalence and
structural equivalence via name-equivalence. Fortunately, all of this
works out pleasantly and we may calculate in the natural way, free of
concern. The reader interested in the details is referred to the
appendix \ref{appendix:rho_details}.

\subsection{Substitution}

We use $\Proc$ for the set of processes, $\QProc$ for the set of
names, and $\id{\{}\vec{y} / \vec{x} \id{\}}$ to denote partial maps,
$s : \QProc \rightarrow \QProc$. A map, $s$ lifts, uniquely, to a map
on process terms, $\widehat{s} : \Proc \rightarrow \Proc$ by the
following equations.

\begin{mathpar}
  (0) \psubstp{Q}{P} := 0 \\
  (R \juxtap S) \psubstp{Q}{P}
  :=    
  (R)\psubstp{Q}{P} \juxtap (S) \psubstp{Q}{P} \\
  (x?(y).R) \psubstp{Q}{P}    
  :=    
  (x)\substp{Q}{P} (z)\concat( (R \psubstn{z}{y}) \psubstp{Q}{P} ) \\
  (\lift{x}{R}) \psubstp{Q}{P}  
  :=
  \lift{(x)\substp{Q}{P}}{ R \psubstp{Q}{P} } \\
%   (\dropn{x})  \psubstp{Q}{P}       
%   := 
%   \left\{ 
%     \begin{array}{ccc} 
%       \dropn{\quotep{Q}} & & x \nameeq \quotep{P} \\
%       \dropn{x} & & otherwise \\
%     \end{array}
%   \right. 
  (\dropn{x})  \psubstp{Q}{P}       
  := 
  \left\{ 
    \begin{array}{ccc} 
      Q & & x \nameeq \quotep{P} \\
      \dropn{x} & & otherwise \\
    \end{array}
  \right.
\end{mathpar}
 

where

\begin{eqnarray}
  (x)\id{\{} \lpquote Q \rpquote / \lpquote P \rpquote \id{\}}            = 
  \left\{ 
    \begin{array}{ccc}
      \lpquote Q \rpquote & & x \nameeq \lpquote P \rpquote \\
      x & & otherwise \\
    \end{array}
  \right. \nonumber
\end{eqnarray}

and $z$ is chosen distinct from $\quotep{P}$, $\quotep{Q}$, the free
names in $Q$, and all the names in $R$. Our $\alpha$-equivalence will
be built in the standard way from this substitution.

\begin{remark}\label{rem:no_self_referential_names}
  One consequence of these definitions is that $\forall P. \quotep{P}
  \not\in \freenames{P}$.
\end{remark}

\subsection{ Dynamic quote: an example }

Anticipating something of what's to come, consider applying the
substitution, $\widehat{\id{\{}u / z \id{\}}}$, to the following pair
of processes, $\lift{w}{y!(z)}$ and $w[ \lpquote y!(z) \rpquote ]$.

\begin{eqnarray}
	\lift{w}{y!(z)}\widehat{\id{\{}u / z \id{\}}}
		& = &
		\lift{w}{y!(u)} \nonumber\\
	w[ \lpquote y!(z) \rpquote ] \widehat{ \id{\{}u / z \id{\}} }
		& = &
		w[ \lpquote y!(z) \rpquote ] \nonumber
\end{eqnarray}

Because the body of the process between quotes is impervious to
substitution, we get radically different answers. In fact, by
examining the first process in an input context,
e.g. $x?(z).\lift{w}{y!(z)}$, we see that the process under the lift
operator may be shaped by prefixed inputs binding a name inside it. In
this sense, the lift operator will be seen as a way to dynamically
construct processes before reifying them as names.

Finally equipped with these standard features we can present the
dynamics of the calculus.

\subsubsection{Operational semantics} 

Finally, we introduce the computational dynamics. What marks these
algebras as distinct from other more traditionally studied algebraic
structures, e.g. vector spaces or polynomial rings, is the manner in
which dynamics is captured. In traditional structures, dynamics is typically
expressed through morphisms between such structures, as in linear maps
between vector spaces or morphisms between rings. In algebras
associated with the semantics of computation, the dynamics is
expressed as part of the algebraic structure itself, through a
reduction reduction relation typically denoted by $\red$. Below, we
give a recursive presentation of this relation for the calculus used
in the encoding.

$\red \subseteq \pi \times \pi$
$\red : \pi \to \mathcal{P}(\pi)$

\begin{mathpar}
  \inferrule* [lab=Comm] { \textsf{match}( x_{src}, x_{trgt} ) } { x_{trgt}?(y)P \; | \; x_{src}!\langle {Q} \rangle \red P\{\quotep{Q}/y}\} }
  \and \\
  \inferrule* [lab=Par] {{P} \red {P}'} {{{P} | {Q}} \red {{P}' | {Q}}}
  \and
  \inferrule* [lab=Equiv]{{{P} \scong {P}'} \andalso {{P}' \red {Q}'} \andalso {{Q}' \scong {Q}}}{{P} \red {Q}}
\end{mathpar}

\begin{eqnarray*}
  match_{\equiv} (\quotep{P},\quotep{Q}) & := & P \equiv Q \\
  match_{\dagger}(\quotep{P},\quotep{Q}) & := & \forall R. P|Q \red^{*} R => R \red^{*} 0 \\
  match_{K}(\quotep{P},\quotep{Q}) & := & K \mbox{ for some context } K
\end{eqnarray*}

$u?(x)P | u!\langle Q \rangle \red P\{\quotep{Q}/x\}$

%We write $\wred$ for $\red^*$, and $P\red$ if $\exists Q $ such that $ P \red Q$.
We write $P\red$ if $\exists Q $ such that $ P \red Q$ and $P\not\red$, otherwise.

\section{Replication}

As mentioned before, it is known that replication (and hence
recursion) can be implemented in a higher-order process algebra
\cite{SangiorgiWalker}. As our first example of calculation with the
machinery thus far presented we give the construction explicitly in
the {\rhoc}.

\begin{eqnarray}
	D_{x} & := & \prefix{x}{y}{(\binpar{\outputp{x}{y}}{@{y}})} \nonumber\\
	\bangp_{x}{P} & := & \binpar{{x}!\langle{\binpar{D_{x}}{P}}\rangle}{D_{x}} \nonumber
\end{eqnarray}

\begin{eqnarray}
	\bangp_{x}{P} & & \nonumber\\
	=
	& {x}!\langle{(\prefix{x}{y}{(\outputp{x}{y} | @{y})) | P}}\rangle 
	      | \prefix{x}{y}{(\outputp{x}{y} | @{y})} & \nonumber\\
	\red
	& (\outputp{x}{y} | @{y})\substn{\quotep{(\prefix{x}{y}{(@{y} | \outputp{x}{y})) | P}}}{y} & \nonumber\\
	=
	& \outputp{x}{\quotep{(\prefix{x}{y}{(\outputp{x}{y} | @{y})) | P}}}
	  | {(\prefix{x}{y}{(\outputp{x}{y} | @{y})) | P}} & \nonumber\\
	\red
	& \ldots & \nonumber\\
	\red^*
	& P | P | \ldots & \nonumber
\end{eqnarray}

Of course, this encoding, as an implementation, runs away, unfolding
$\bangp{P}$ eagerly. A lazier and more implementable replication
operator, restricted to input-guarded processes, may be obtained as follows.

\begin{eqnarray}
\bangp{\prefix{u}{v}{P}} 
	:= 
	\binpar{\lift{x}{\prefix{u}{v}{(\binpar{D(x)}{P})}}}{D(x)} \nonumber
\end{eqnarray}

\begin{remark}
  Note that the lazier definition still does not deal with summation
  or mixed summation (i.e. sums over input and output). The reader is
  invited to construct definitions of replication that deal with these
  features. 

  Further, the definitions are parameterized in a name, $x$. Can you,
  gentle reader, make a definition that eliminates this parameter and
  guarantees no accidental interaction between the replication
  machinery and the process being replicated -- i.e. no accidental
  sharing of names used by the process to get its work done and the
  name(s) used by the replication to effect copying. This latter
  revision of the definition of replication is crucial to obtaining
  the expected identity $!!P \sim !P$.
\end{remark}

\begin{remark}\label{rem:paradoxical_combinator}
  The reader familiar with the lambda calculus will have noticed the
  similarity between $D$ and the paradoxical combinator.

  [Ed. note: the existence of this seems to suggest we have to be more
  restrictive on the set of processes and names we admit if we are to
  support no-cloning.]
\end{remark}

\subsubsection{Bisimulation}

The computational dynamics gives rise to another kind of equivalence,
the equivalence of computational behavior. As previously mentioned
this is typically captured \emph{via} some form of bisimulation.

% The notion we use in this paper is weak barbed bisimulation
% \cite{milner91polyadicpi}.

The notion we use in this paper is derived from weak barbed
bisimulation \cite{milner91polyadicpi}. 

\begin{definition}
An \emph{observation relation}, $\downarrow_{\mathcal N}$, over a set
of names, $\mathcal N$, is the smallest relation satisfying the rules
below.

\infrule[Out-barb]{y \in {\mathcal N}, \; x \nameeq y}
		  {\outputp{x}{v} \downarrow_{\mathcal N} x}
\infrule[Par-barb]{\mbox{$P\downarrow_{\mathcal N} x$ or $Q\downarrow_{\mathcal N} x$}}
		  {\binpar{P}{Q} \downarrow_{\mathcal N} x}

We write $P \Downarrow_{\mathcal N} x$ if there is $Q$ such that 
$P \wred Q$ and $Q \downarrow_{\mathcal N} x$.
\end{definition}

\begin{definition}
%\label{def.bbisim}
An  ${\mathcal N}$-\emph{barbed bisimulation} over a set of names, ${\mathcal N}$, is a symmetric binary relation 
${\mathcal S}_{\mathcal N}$ between agents such that $P\rel{S}_{\mathcal N}Q$ implies:
\begin{enumerate}
\item If $P \red P'$ then $Q \wred Q'$ and $P'\rel{S}_{\mathcal N} Q'$.
\item If $P\downarrow_{\mathcal N} x$, then $Q\Downarrow_{\mathcal N} x$.
\end{enumerate}
$P$ is ${\mathcal N}$-barbed bisimilar to $Q$, written
$P \wbbisim_{\mathcal N} Q$, if $P \rel{S}_{\mathcal N} Q$ for some ${\mathcal N}$-barbed bisimulation ${\mathcal S}_{\mathcal N}$.
\end{definition}

$\mathcal{R} \subseteq \pi \times \pi$

$P \mathcal{R} Q => \forall P'. P \red P' \Rightarrow \exists Q'. Q \red Q', P' \mathcal{R} Q'$

$P \vdash x \Rightarrow Q \vdash x$

\begin{mathpar}
  \inferrule*[lab=Out-barb]{x \nameeq y}{{y}!\langle{Q}\rangle \vdash x}
  \and
  \inferrule*[lab=Par-barb]{\mbox{$P\vdash x$ or $Q\vdash x$}}{\binpar{P}{Q} \vdash x}
\end{mathpar}

\subsubsection{Contexts}

One of the principle advantages of computational calculi like the
$\pi$-calculus is a well-defined notion of context,
contextual-equivalence and a correlation between
contextual-equivalence and notions of bisimulation. The notion of
context allows the decomposition of a process into (sub-)process and
its syntactic environment, its context. Thus, a context may be
thought of as a process with a ``hole'' (written $\Box$) in it. The
application of a context $M$ to a process $P$, written $M[P]$, is
tantamount to filling the hole in $M$ with $P$. In this paper we do
not need the full weight of this theory, but do make use of the notion
of context in the proof the main theorem. 

\begin{mathpar}
  \inferrule* [lab=summation] {} {{M_{M},M_{N}} \bc \Box \;|\; x.M_{A} \;|\; M_{M}+M_{N}}
  \and
  \inferrule* [lab=agent] {} {{M_{A}} \bc (\vec{x})M_{P} \;| \; \clift{P_0,\ldots,M_{P},\ldots,P_N}}
  \and \\
  \inferrule* [lab=process] {} {{M_{P}} \bc M_{N} \;| \;P|M_{P} }
\end{mathpar} 

\begin{mathpar}
  \inferrule* [lab=sychronization] {} {M_{N} \bc \Box \;|\; x?M_{F} \;|\; x!M_{C}}
  \and
  \inferrule* [lab=abstraction] {} {{M_{F}} \bc (x)M_{P} }
  \and
  \inferrule* [lab=concretion] {} {{M_{C}} \bc \langle M_{P} \rangle }
  \and \\
  \inferrule* [lab=process] {} {{M_{P}} \bc M_{N} \;| \;P|M_{P} }
\end{mathpar}

\begin{definition}[contextual application] Given a context $M$, and
  process $P$, we define the \emph{contextual application}, $M[P] :=
  M\{P/\Box\}$. That is, the contextual application of M to P is the
  substitution of $P$ for $\Box$ in $M$.
\end{definition}

$\meaningof{-} : L \to \mathcal{P}(\pi)$

\begin{mathpar}
  \inferrule* [lab=collection] {} {\meaningof{true} = \pi, \and \meaningof{~E} = \pi \setminus \meaningof{E}, \and \meaningof{E_{1} \& E_{2}} = \meaningof{E_{1}} \cap \meaningof{E_{2}}}
\end{mathpar}

\begin{mathpar}
  \inferrule* [lab=structure] {} {\meaningof{0} = \{ P \in \pi | P \equiv 0 \}, \and \\ \meaningof{E_1 | E_2} = \{ P \in \pi | P \equiv P_{1} | P_{2}, P_{1} \in \meaningof{E_{1}}, P_{2} \in \meaningof{E_2}\} }
\end{mathpar}

\begin{mathpar}
 \inferrule* [lab=behavior] {} {\meaningof{\langle a?b \rangle E} = \{ P \in \pi | P \equiv Q | u?(y)P', \\ \and \\\\ \and \\ \;\;\; u \in \meaningof{a}, \forall z.P'\{z/y\} \in \meaningof{E\{z/b\}}\}, \and \\ \meaningof{a!E} = \{ P \in \pi | P \equiv Q | x!\langle P' \rangle, x \in \meaningof{a} P' \in \meaningof{E}\} }
\end{mathpar}

\begin{mathpar}
 \inferrule* [lab=nominal] {} {\meaningof{\quotep{E}} = \{ \quotep{P} \in \quotep{\pi} | P \in \meaningof{E} \}, \and \meaningof{\quotep{P}} = \{ \quotep{Q} \in \quotep{\pi} | P \equiv Q \} \and \\ \meaningof{@\quotep{E}} = \{ P \in \pi | P \equiv @x, x \in \meaningof{E} \}}
\end{mathpar}

\begin{eqnarray*}
  \\
  \meaningof{-} : TS \to ST
\end{eqnarray*}

\begin{eqnarray*}
  \\
  L : TS \to ST
\end{eqnarray*}

\begin{eqnarray*}
  \\
  P \models E \iff P \in \meaningof{E}
\end{eqnarray*}

\begin{eqnarray*}
  P \approx_{L} Q \iff \forall E \in L. P \models E \iff Q \models E
\end{eqnarray*}

\begin{eqnarray*}
  P \approx_{K} Q
\end{eqnarray*}

\begin{eqnarray*}
  P \approx Q
\end{eqnarray*}

$\approx_{K} = \approx = \approx_{L}$

\subsubsection{Contextual duality}

Note that contexts extend the quotation operation to a family of
operations from processes to names. Given a context, $M$, we can
define a \emph{nominal context}, $\quotep{M}$ by $\quotep{M}[P] :=
\quotep{M[P]}$. To foreshadow what is to come we observe that these
operations enjoy a duality with processes very much like the duality
between vectors and maps from vectors to scalars.

Further, because the calculus is essentially higher-order, we have a
correspondence between contexts and processes. More specifically,
given a name $x$ and a context $M$ we can construct $M^{*}_{x}$ such
that 

\begin{mathpar}
  M^{*}_{x} | \lift{x}{P} \red M[P]
\end{mathpar}

namely,

\begin{mathpar}
  M^{*}_{x} := x?(u).M[\dropn{u}]
\end{mathpar}

The dependence of $M^{*}_{x}$ on a name makes it an abstraction, 

\begin{mathpar}
  M^{*} := (x)x?(u).M[\dropn{u}]
\end{mathpar}

\subsection{Additional notation}

It will sometimes be convenient to denote the process a name
quotes. We already have the notation $x = \quotep{P}$, but it will be
convenient to introduce an alternate notation, $\procn{x}$, when we
want to emphasize the connection to the use of the name. Note that, by
virtue of name equivalence, $\quotep{\procn{x}} \nameeq x$; so, the
notation is consistent with previous definitions.

Further, because names have structure it is possible to effect
substitutions on the basis of that structure. This means we need to
upgrade our notation for substitutions, which we accomplish by
adapting comprehension notation. Thus,

\begin{mathpar}
  P\{ y / x : x \in S \}
\end{mathpar}

is interpreted to mean the process derived from P by replacing (in a
capture-avoiding manner) each occurrence of $x$ in $S$ by $y$. For example,

\begin{mathpar}
  P\{ \quotep{\procn{x}|\procn{x}} / x : x \in \freenames{P} \}
\end{mathpar}

will replace each (occurrence) of a free name $x$ in $P$ by
$\quotep{\procn{x}|\procn{x}}$.

Also, we will avail ourselves of the notation $x^{L}$ and $x^{R}$ to
denote injections of a name into disjoint copies of the name
space. There are numerous ways to accomplish this. One example can be
found in \cite{MeredithR05}. This notation overloads to vectors of
names: $\vec{x}^{\pi} := (x_{i}^{\pi} \; : \; 0 \leq i < |\vec{x}| )$ where $\pi \in \{L,R\}$.

We also use $P^{\Box} := P|\Box$.

In \cite{MeredithR05} an interpretation of the new operator is
given. It turns out that there are several possible interpretations
all enjoying the requisite algebraic properties of the operator (see
\cite{milner91polyadicpi}). We will therefore make liberal use of
$(\nu\; \vec{x})P$.

% subsection the_syntax_and_semantics_of_the_notation_system (end)   

\input{qm2pi.qmops} 

\input{qm2pi.sterngerlach} 

\input{qm2pi.metric} 

% section concurrent_process_calculi (end)

%\input{qm2pi.proofsketch}

% section proof sketch (end)

%\input{qm2pi.slviaknots} 

% section spatial logic via knots (end)

\input{qm2pi.conclusion}

% section conclusion (end)

%\input{qm2pi.dtcodes} 

% section wiring algorithm (end)

\input{qm2pi.ack} 

% section acknowledgments (end)

\newpage


\bibliographystyle{plain}   
\bibliography{../../biblios/main.bib}

\input{qm2pi.rhodetails}

\end{document}

 

%\documentclass[12pt]{llncs}
%\documentclass{jktr}

\usepackage[pdftex]{hyperref}                   
\usepackage {listings}
\usepackage {mathpartir}
\usepackage{bcprules}
%\usepackage{listings}
                       
\usepackage{graphicx} 
%\usepackage[margins=2.5cm,nohead,nofoot]{geometry}
%\usepackage{geometry}
\usepackage{amsfonts}
\usepackage{amstext}
\usepackage{latexsym}
\usepackage{amssymb}
\usepackage{color}


%\include{myPreamble}
\include{qm2pi.local} 

%\ifpdf
%\usepackage[pdftex]{graphicx}
%\else
%\usepackage{graphicx}
%\fi

 % \ifpdf
%  \usepackage{pdfsync}
%  \if


%\title{Brief Article}
%\author{David F. Snyder}
%\author{L.G. Meredith}

%\address{Dept. of Math., Texas State University--San Marcos, San Marcos, TX 78666}
       
\pagestyle{empty}


\begin{document}

\lstset{language=[Objective]Caml,frame=shadowbox}

\input{qm2pi.front}

% section front matter (end)

\input{qm2pi.intro} 
 
% section introduction (end)

% \input{qm2pi.knotations} 

% section notation (end)

\input{qm2pi.process.calculi} 

% section concurrent_process_calculi_and_spatial_logics_ (end)
    
%\input{qm2pi.knots2pi} 

%\input{qm2pi.trefoil} 

%\input{qm2pi.mainthm} 

% subsection basic_interpretation (end)

%\input{qm2pi.rho.presentation} 
\subsection{The syntax and semantics of the notation system}\label{sub:the_syntax_and_semantics_of_the_notation_system} % (fold)

We now summarize a technical presentation of the calculus that
embodies our theory of dynamics. The typical presentation of such a
calculus follows the style of giving generators and relations on
them. The grammar, below, describing term constructors, freely
generates the set of processes, $\Proc$. This set is then quotiented
by a relation known as structural congruence and it is over this set
that the notion of dynamics is expressed. This presentation is
essentially that of \cite{MeredithR05} with the addition of
polyadicity and summation. For readability we have relegated some of
the technical subtleties to an appendix.

\subsubsection{Process grammar}\label{subsub:process_grammar}

\begin{mathpar}
  \inferrule* [lab=synchronization] {} {{M} \bc \pzero \;|\; x?F \;|\; x!C }
  \and
  \inferrule* [lab=abstraction] {} {{F} \bc (x)P}
  \and
  \inferrule* [lab=concretion] {} {{C} \bc \langle Q \rangle}
  \and
  \inferrule* [lab=process] {} {{P,Q} \bc M \;| \;P|Q \;|\; @{x}}
  \and
  \inferrule* [lab=name] {} {{x} \bc \quotep{P}}
\end{mathpar} 

Note that $\vec{x}$ (resp. $\vec{P}$) denotes a vector of names
(resp. processes) of length $|\vec{x}|$ (resp. $|\vec{P}|$). We adopt
the following useful abbreviations.

\begin{mathpar}
   x?(\vec{y}).P := x.(\vec{y})P \and  x\clift{\vec{P}} := x.\clift{\vec{P}}
   \and x!(y) := \lift{x}{\dropn{y}}
   \and \Pi_{i=0}^{n-1}P_i := P_0 | \ldots | P_{n-1}
\end{mathpar}

\subsubsection{Structural congruence}

\paragraph{Free and bound names and alpha-equivalence.} At the
core of structural equivalence is alpha-equivalence which identifies
process that are the same up to a change of variable. Formally, we
recognize the distinction between free and bound names. The free names
of a process, $\freenames{P}$, may be calculated recursively as
follows:

\begin{mathpar}
\freenames{\pzero} := \emptyset
  \and \\
  \freenames{x?(y).P} := \{ x \} \cup (\freenames{P} \setminus \{ y \})
  \and 
  \freenames{x!\langle P \rangle} := \{ x \} \cup \{ P \} 
  \and \\
  \freenames{P|Q} := \freenames{P} \cup \freenames{Q}
  \and \\
  \freenames{@{x}} := \{ x \}
\end{mathpar}

$\pi$
$\quotep{\pi}$

$\freenames{-} : \pi \to \mathcal{P}(\quotep{\pi})$

\begin{eqnarray*}
  \freenames{\pzero} & := & \emptyset \\
  \freenames{x?(y).P} & := & \{ x \} \cup (\freenames{P} \setminus \{ y \}) \\
  \freenames{x!\langle P \rangle} & := & \{ x \} \cup \{ P \} \\
  \freenames{P|Q} & := & \freenames{P} \cup \freenames{Q} \\
  \freenames{\dropn{x}} & := & \{ x \}
\end{eqnarray*}

The bound names of a process, $\boundnames{P}$, are those names occurring in $P$
that are not free. For example, in $x?(y).0$, the name $x$ is free, while $y$ is bound.

\begin{mathpar}
  \inferrule* [lab=monoidal-laws] {} { P|Q \equiv Q|P \and P|0 \equiv P \and P|(Q|R) \equiv (P|Q)|R }
\end{mathpar}

\begin{mathpar}
  \inferrule* [lab=alpha-equivalence] {} { (x)P \equiv (y)P\{y/x\} \and y \not\in \freenames{P} }
\end{mathpar}

\begin{definition}
Then two processes, $P,Q$, are alpha-equivalent if $P = Q\{\vec{y}/\vec{x}\}$ for
some $\vec{x} \in \boundnames{Q},\vec{y} \in \boundnames{P}$, where $Q\{\vec{y}/\vec{x}\}$
denotes the capture-avoiding substitution of $\vec{y}$ for $\vec{x}$ in $Q$.
\end{definition}

\begin{definition}
  The {\em structural congruence} \cite{SangiorgiWalker} , $\equiv$,
  between processes is the least congruence containing
  alpha-equivalence, satisfying the abelian monoid laws
  (associativity, commutativity and $\pzero$ as identity) for parallel
  composition $|$ and for summation $+$.
\end{definition}

\subsection{Name equivalence}

We take name equivalence, written $\nameeq$, to be the smallest
equivalence relation generated by the following rules.

\begin{mathpar}
\inferrule*[lab=Quote-drop]
{ }
{ \quotep{@{x}} \nameeq x }

\inferrule*[lab=Struct-equiv]
{ P \scong Q }
{ \quotep{P} \nameeq \quotep{Q} }
\end{mathpar}

The astute reader will have noticed that the mutual recursion of names
and processes imposes a mutual recursion on alpha-equivalence and
structural equivalence via name-equivalence. Fortunately, all of this
works out pleasantly and we may calculate in the natural way, free of
concern. The reader interested in the details is referred to the
appendix \ref{appendix:rho_details}.

\subsection{Substitution}

We use $\Proc$ for the set of processes, $\QProc$ for the set of
names, and $\id{\{}\vec{y} / \vec{x} \id{\}}$ to denote partial maps,
$s : \QProc \rightarrow \QProc$. A map, $s$ lifts, uniquely, to a map
on process terms, $\widehat{s} : \Proc \rightarrow \Proc$ by the
following equations.

\begin{mathpar}
  (0) \psubstp{Q}{P} := 0 \\
  (R \juxtap S) \psubstp{Q}{P}
  :=    
  (R)\psubstp{Q}{P} \juxtap (S) \psubstp{Q}{P} \\
  (x?(y).R) \psubstp{Q}{P}    
  :=    
  (x)\substp{Q}{P} (z)\concat( (R \psubstn{z}{y}) \psubstp{Q}{P} ) \\
  (\lift{x}{R}) \psubstp{Q}{P}  
  :=
  \lift{(x)\substp{Q}{P}}{ R \psubstp{Q}{P} } \\
%   (\dropn{x})  \psubstp{Q}{P}       
%   := 
%   \left\{ 
%     \begin{array}{ccc} 
%       \dropn{\quotep{Q}} & & x \nameeq \quotep{P} \\
%       \dropn{x} & & otherwise \\
%     \end{array}
%   \right. 
  (\dropn{x})  \psubstp{Q}{P}       
  := 
  \left\{ 
    \begin{array}{ccc} 
      Q & & x \nameeq \quotep{P} \\
      \dropn{x} & & otherwise \\
    \end{array}
  \right.
\end{mathpar}
 

where

\begin{eqnarray}
  (x)\id{\{} \lpquote Q \rpquote / \lpquote P \rpquote \id{\}}            = 
  \left\{ 
    \begin{array}{ccc}
      \lpquote Q \rpquote & & x \nameeq \lpquote P \rpquote \\
      x & & otherwise \\
    \end{array}
  \right. \nonumber
\end{eqnarray}

and $z$ is chosen distinct from $\quotep{P}$, $\quotep{Q}$, the free
names in $Q$, and all the names in $R$. Our $\alpha$-equivalence will
be built in the standard way from this substitution.

\begin{remark}\label{rem:no_self_referential_names}
  One consequence of these definitions is that $\forall P. \quotep{P}
  \not\in \freenames{P}$.
\end{remark}

\subsection{ Dynamic quote: an example }

Anticipating something of what's to come, consider applying the
substitution, $\widehat{\id{\{}u / z \id{\}}}$, to the following pair
of processes, $\lift{w}{y!(z)}$ and $w[ \lpquote y!(z) \rpquote ]$.

\begin{eqnarray}
	\lift{w}{y!(z)}\widehat{\id{\{}u / z \id{\}}}
		& = &
		\lift{w}{y!(u)} \nonumber\\
	w[ \lpquote y!(z) \rpquote ] \widehat{ \id{\{}u / z \id{\}} }
		& = &
		w[ \lpquote y!(z) \rpquote ] \nonumber
\end{eqnarray}

Because the body of the process between quotes is impervious to
substitution, we get radically different answers. In fact, by
examining the first process in an input context,
e.g. $x?(z).\lift{w}{y!(z)}$, we see that the process under the lift
operator may be shaped by prefixed inputs binding a name inside it. In
this sense, the lift operator will be seen as a way to dynamically
construct processes before reifying them as names.

Finally equipped with these standard features we can present the
dynamics of the calculus.

\subsubsection{Operational semantics} 

Finally, we introduce the computational dynamics. What marks these
algebras as distinct from other more traditionally studied algebraic
structures, e.g. vector spaces or polynomial rings, is the manner in
which dynamics is captured. In traditional structures, dynamics is typically
expressed through morphisms between such structures, as in linear maps
between vector spaces or morphisms between rings. In algebras
associated with the semantics of computation, the dynamics is
expressed as part of the algebraic structure itself, through a
reduction reduction relation typically denoted by $\red$. Below, we
give a recursive presentation of this relation for the calculus used
in the encoding.

$\red \subseteq \pi \times \pi$
$\red : \pi \to \mathcal{P}(\pi)$

\begin{mathpar}
  \inferrule* [lab=Comm] { \textsf{match}( x_{src}, x_{trgt} ) } { x_{trgt}?(y)P \; | \; x_{src}!\langle {Q} \rangle \red P\{\quotep{Q}/y}\} }
  \and \\
  \inferrule* [lab=Par] {{P} \red {P}'} {{{P} | {Q}} \red {{P}' | {Q}}}
  \and
  \inferrule* [lab=Equiv]{{{P} \scong {P}'} \andalso {{P}' \red {Q}'} \andalso {{Q}' \scong {Q}}}{{P} \red {Q}}
\end{mathpar}

\begin{eqnarray*}
  match_{\equiv} (\quotep{P},\quotep{Q}) & := & P \equiv Q \\
  match_{\dagger}(\quotep{P},\quotep{Q}) & := & \forall R. P|Q \red^{*} R => R \red^{*} 0 \\
  match_{K}(\quotep{P},\quotep{Q}) & := & K \mbox{ for some context } K
\end{eqnarray*}

$u?(x)P | u!\langle Q \rangle \red P\{\quotep{Q}/x\}$

%We write $\wred$ for $\red^*$, and $P\red$ if $\exists Q $ such that $ P \red Q$.
We write $P\red$ if $\exists Q $ such that $ P \red Q$ and $P\not\red$, otherwise.

\section{Replication}

As mentioned before, it is known that replication (and hence
recursion) can be implemented in a higher-order process algebra
\cite{SangiorgiWalker}. As our first example of calculation with the
machinery thus far presented we give the construction explicitly in
the {\rhoc}.

\begin{eqnarray}
	D_{x} & := & \prefix{x}{y}{(\binpar{\outputp{x}{y}}{@{y}})} \nonumber\\
	\bangp_{x}{P} & := & \binpar{{x}!\langle{\binpar{D_{x}}{P}}\rangle}{D_{x}} \nonumber
\end{eqnarray}

\begin{eqnarray}
	\bangp_{x}{P} & & \nonumber\\
	=
	& {x}!\langle{(\prefix{x}{y}{(\outputp{x}{y} | @{y})) | P}}\rangle 
	      | \prefix{x}{y}{(\outputp{x}{y} | @{y})} & \nonumber\\
	\red
	& (\outputp{x}{y} | @{y})\substn{\quotep{(\prefix{x}{y}{(@{y} | \outputp{x}{y})) | P}}}{y} & \nonumber\\
	=
	& \outputp{x}{\quotep{(\prefix{x}{y}{(\outputp{x}{y} | @{y})) | P}}}
	  | {(\prefix{x}{y}{(\outputp{x}{y} | @{y})) | P}} & \nonumber\\
	\red
	& \ldots & \nonumber\\
	\red^*
	& P | P | \ldots & \nonumber
\end{eqnarray}

Of course, this encoding, as an implementation, runs away, unfolding
$\bangp{P}$ eagerly. A lazier and more implementable replication
operator, restricted to input-guarded processes, may be obtained as follows.

\begin{eqnarray}
\bangp{\prefix{u}{v}{P}} 
	:= 
	\binpar{\lift{x}{\prefix{u}{v}{(\binpar{D(x)}{P})}}}{D(x)} \nonumber
\end{eqnarray}

\begin{remark}
  Note that the lazier definition still does not deal with summation
  or mixed summation (i.e. sums over input and output). The reader is
  invited to construct definitions of replication that deal with these
  features. 

  Further, the definitions are parameterized in a name, $x$. Can you,
  gentle reader, make a definition that eliminates this parameter and
  guarantees no accidental interaction between the replication
  machinery and the process being replicated -- i.e. no accidental
  sharing of names used by the process to get its work done and the
  name(s) used by the replication to effect copying. This latter
  revision of the definition of replication is crucial to obtaining
  the expected identity $!!P \sim !P$.
\end{remark}

\begin{remark}\label{rem:paradoxical_combinator}
  The reader familiar with the lambda calculus will have noticed the
  similarity between $D$ and the paradoxical combinator.

  [Ed. note: the existence of this seems to suggest we have to be more
  restrictive on the set of processes and names we admit if we are to
  support no-cloning.]
\end{remark}

\subsubsection{Bisimulation}

The computational dynamics gives rise to another kind of equivalence,
the equivalence of computational behavior. As previously mentioned
this is typically captured \emph{via} some form of bisimulation.

% The notion we use in this paper is weak barbed bisimulation
% \cite{milner91polyadicpi}.

The notion we use in this paper is derived from weak barbed
bisimulation \cite{milner91polyadicpi}. 

\begin{definition}
An \emph{observation relation}, $\downarrow_{\mathcal N}$, over a set
of names, $\mathcal N$, is the smallest relation satisfying the rules
below.

\infrule[Out-barb]{y \in {\mathcal N}, \; x \nameeq y}
		  {\outputp{x}{v} \downarrow_{\mathcal N} x}
\infrule[Par-barb]{\mbox{$P\downarrow_{\mathcal N} x$ or $Q\downarrow_{\mathcal N} x$}}
		  {\binpar{P}{Q} \downarrow_{\mathcal N} x}

We write $P \Downarrow_{\mathcal N} x$ if there is $Q$ such that 
$P \wred Q$ and $Q \downarrow_{\mathcal N} x$.
\end{definition}

\begin{definition}
%\label{def.bbisim}
An  ${\mathcal N}$-\emph{barbed bisimulation} over a set of names, ${\mathcal N}$, is a symmetric binary relation 
${\mathcal S}_{\mathcal N}$ between agents such that $P\rel{S}_{\mathcal N}Q$ implies:
\begin{enumerate}
\item If $P \red P'$ then $Q \wred Q'$ and $P'\rel{S}_{\mathcal N} Q'$.
\item If $P\downarrow_{\mathcal N} x$, then $Q\Downarrow_{\mathcal N} x$.
\end{enumerate}
$P$ is ${\mathcal N}$-barbed bisimilar to $Q$, written
$P \wbbisim_{\mathcal N} Q$, if $P \rel{S}_{\mathcal N} Q$ for some ${\mathcal N}$-barbed bisimulation ${\mathcal S}_{\mathcal N}$.
\end{definition}

$\mathcal{R} \subseteq \pi \times \pi$

$P \mathcal{R} Q => \forall P'. P \red P' \Rightarrow \exists Q'. Q \red Q', P' \mathcal{R} Q'$

$P \vdash x \Rightarrow Q \vdash x$

\begin{mathpar}
  \inferrule*[lab=Out-barb]{x \nameeq y}{{y}!\langle{Q}\rangle \vdash x}
  \and
  \inferrule*[lab=Par-barb]{\mbox{$P\vdash x$ or $Q\vdash x$}}{\binpar{P}{Q} \vdash x}
\end{mathpar}

\subsubsection{Contexts}

One of the principle advantages of computational calculi like the
$\pi$-calculus is a well-defined notion of context,
contextual-equivalence and a correlation between
contextual-equivalence and notions of bisimulation. The notion of
context allows the decomposition of a process into (sub-)process and
its syntactic environment, its context. Thus, a context may be
thought of as a process with a ``hole'' (written $\Box$) in it. The
application of a context $M$ to a process $P$, written $M[P]$, is
tantamount to filling the hole in $M$ with $P$. In this paper we do
not need the full weight of this theory, but do make use of the notion
of context in the proof the main theorem. 

\begin{mathpar}
  \inferrule* [lab=summation] {} {{M_{M},M_{N}} \bc \Box \;|\; x.M_{A} \;|\; M_{M}+M_{N}}
  \and
  \inferrule* [lab=agent] {} {{M_{A}} \bc (\vec{x})M_{P} \;| \; \clift{P_0,\ldots,M_{P},\ldots,P_N}}
  \and \\
  \inferrule* [lab=process] {} {{M_{P}} \bc M_{N} \;| \;P|M_{P} }
\end{mathpar} 

\begin{mathpar}
  \inferrule* [lab=sychronization] {} {M_{N} \bc \Box \;|\; x?M_{F} \;|\; x!M_{C}}
  \and
  \inferrule* [lab=abstraction] {} {{M_{F}} \bc (x)M_{P} }
  \and
  \inferrule* [lab=concretion] {} {{M_{C}} \bc \langle M_{P} \rangle }
  \and \\
  \inferrule* [lab=process] {} {{M_{P}} \bc M_{N} \;| \;P|M_{P} }
\end{mathpar}

\begin{definition}[contextual application] Given a context $M$, and
  process $P$, we define the \emph{contextual application}, $M[P] :=
  M\{P/\Box\}$. That is, the contextual application of M to P is the
  substitution of $P$ for $\Box$ in $M$.
\end{definition}

$\meaningof{-} : L \to \mathcal{P}(\pi)$

\begin{mathpar}
  \inferrule* [lab=collection] {} {\meaningof{true} = \pi, \and \meaningof{~E} = \pi \setminus \meaningof{E}, \and \meaningof{E_{1} \& E_{2}} = \meaningof{E_{1}} \cap \meaningof{E_{2}}}
\end{mathpar}

\begin{mathpar}
  \inferrule* [lab=structure] {} {\meaningof{0} = \{ P \in \pi | P \equiv 0 \}, \and \\ \meaningof{E_1 | E_2} = \{ P \in \pi | P \equiv P_{1} | P_{2}, P_{1} \in \meaningof{E_{1}}, P_{2} \in \meaningof{E_2}\} }
\end{mathpar}

\begin{mathpar}
 \inferrule* [lab=behavior] {} {\meaningof{\langle a?b \rangle E} = \{ P \in \pi | P \equiv Q | u?(y)P', \\ \and \\\\ \and \\ \;\;\; u \in \meaningof{a}, \forall z.P'\{z/y\} \in \meaningof{E\{z/b\}}\}, \and \\ \meaningof{a!E} = \{ P \in \pi | P \equiv Q | x!\langle P' \rangle, x \in \meaningof{a} P' \in \meaningof{E}\} }
\end{mathpar}

\begin{mathpar}
 \inferrule* [lab=nominal] {} {\meaningof{\quotep{E}} = \{ \quotep{P} \in \quotep{\pi} | P \in \meaningof{E} \}, \and \meaningof{\quotep{P}} = \{ \quotep{Q} \in \quotep{\pi} | P \equiv Q \} \and \\ \meaningof{@\quotep{E}} = \{ P \in \pi | P \equiv @x, x \in \meaningof{E} \}}
\end{mathpar}

\begin{eqnarray*}
  \\
  \meaningof{-} : TS \to ST
\end{eqnarray*}

\begin{eqnarray*}
  \\
  L : TS \to ST
\end{eqnarray*}

\begin{eqnarray*}
  \\
  P \models E \iff P \in \meaningof{E}
\end{eqnarray*}

\begin{eqnarray*}
  P \approx_{L} Q \iff \forall E \in L. P \models E \iff Q \models E
\end{eqnarray*}

\begin{eqnarray*}
  P \approx_{K} Q
\end{eqnarray*}

\begin{eqnarray*}
  P \approx Q
\end{eqnarray*}

$\approx_{K} = \approx = \approx_{L}$

\subsubsection{Contextual duality}

Note that contexts extend the quotation operation to a family of
operations from processes to names. Given a context, $M$, we can
define a \emph{nominal context}, $\quotep{M}$ by $\quotep{M}[P] :=
\quotep{M[P]}$. To foreshadow what is to come we observe that these
operations enjoy a duality with processes very much like the duality
between vectors and maps from vectors to scalars.

Further, because the calculus is essentially higher-order, we have a
correspondence between contexts and processes. More specifically,
given a name $x$ and a context $M$ we can construct $M^{*}_{x}$ such
that 

\begin{mathpar}
  M^{*}_{x} | \lift{x}{P} \red M[P]
\end{mathpar}

namely,

\begin{mathpar}
  M^{*}_{x} := x?(u).M[\dropn{u}]
\end{mathpar}

The dependence of $M^{*}_{x}$ on a name makes it an abstraction, 

\begin{mathpar}
  M^{*} := (x)x?(u).M[\dropn{u}]
\end{mathpar}

\subsection{Additional notation}

It will sometimes be convenient to denote the process a name
quotes. We already have the notation $x = \quotep{P}$, but it will be
convenient to introduce an alternate notation, $\procn{x}$, when we
want to emphasize the connection to the use of the name. Note that, by
virtue of name equivalence, $\quotep{\procn{x}} \nameeq x$; so, the
notation is consistent with previous definitions.

Further, because names have structure it is possible to effect
substitutions on the basis of that structure. This means we need to
upgrade our notation for substitutions, which we accomplish by
adapting comprehension notation. Thus,

\begin{mathpar}
  P\{ y / x : x \in S \}
\end{mathpar}

is interpreted to mean the process derived from P by replacing (in a
capture-avoiding manner) each occurrence of $x$ in $S$ by $y$. For example,

\begin{mathpar}
  P\{ \quotep{\procn{x}|\procn{x}} / x : x \in \freenames{P} \}
\end{mathpar}

will replace each (occurrence) of a free name $x$ in $P$ by
$\quotep{\procn{x}|\procn{x}}$.

Also, we will avail ourselves of the notation $x^{L}$ and $x^{R}$ to
denote injections of a name into disjoint copies of the name
space. There are numerous ways to accomplish this. One example can be
found in \cite{MeredithR05}. This notation overloads to vectors of
names: $\vec{x}^{\pi} := (x_{i}^{\pi} \; : \; 0 \leq i < |\vec{x}| )$ where $\pi \in \{L,R\}$.

We also use $P^{\Box} := P|\Box$.

In \cite{MeredithR05} an interpretation of the new operator is
given. It turns out that there are several possible interpretations
all enjoying the requisite algebraic properties of the operator (see
\cite{milner91polyadicpi}). We will therefore make liberal use of
$(\nu\; \vec{x})P$.

% subsection the_syntax_and_semantics_of_the_notation_system (end)   

\input{qm2pi.qmops} 

\input{qm2pi.sterngerlach} 

\input{qm2pi.metric} 

% section concurrent_process_calculi (end)

%\input{qm2pi.proofsketch}

% section proof sketch (end)

%\input{qm2pi.slviaknots} 

% section spatial logic via knots (end)

\input{qm2pi.conclusion}

% section conclusion (end)

%\input{qm2pi.dtcodes} 

% section wiring algorithm (end)

\input{qm2pi.ack} 

% section acknowledgments (end)

\newpage


\bibliographystyle{plain}   
\bibliography{../../biblios/main.bib}

\input{qm2pi.rhodetails}

\end{document}

 

%\documentclass[12pt]{llncs}
%\documentclass{jktr}

\usepackage[pdftex]{hyperref}                   
\usepackage {listings}
\usepackage {mathpartir}
\usepackage{bcprules}
%\usepackage{listings}
                       
\usepackage{graphicx} 
%\usepackage[margins=2.5cm,nohead,nofoot]{geometry}
%\usepackage{geometry}
\usepackage{amsfonts}
\usepackage{amstext}
\usepackage{latexsym}
\usepackage{amssymb}
\usepackage{color}


%\include{myPreamble}
\include{qm2pi.local} 

%\ifpdf
%\usepackage[pdftex]{graphicx}
%\else
%\usepackage{graphicx}
%\fi

 % \ifpdf
%  \usepackage{pdfsync}
%  \if


%\title{Brief Article}
%\author{David F. Snyder}
%\author{L.G. Meredith}

%\address{Dept. of Math., Texas State University--San Marcos, San Marcos, TX 78666}
       
\pagestyle{empty}


\begin{document}

\lstset{language=[Objective]Caml,frame=shadowbox}

\input{qm2pi.front}

% section front matter (end)

\input{qm2pi.intro} 
 
% section introduction (end)

% \input{qm2pi.knotations} 

% section notation (end)

\input{qm2pi.process.calculi} 

% section concurrent_process_calculi_and_spatial_logics_ (end)
    
%\input{qm2pi.knots2pi} 

%\input{qm2pi.trefoil} 

%\input{qm2pi.mainthm} 

% subsection basic_interpretation (end)

%\input{qm2pi.rho.presentation} 
\subsection{The syntax and semantics of the notation system}\label{sub:the_syntax_and_semantics_of_the_notation_system} % (fold)

We now summarize a technical presentation of the calculus that
embodies our theory of dynamics. The typical presentation of such a
calculus follows the style of giving generators and relations on
them. The grammar, below, describing term constructors, freely
generates the set of processes, $\Proc$. This set is then quotiented
by a relation known as structural congruence and it is over this set
that the notion of dynamics is expressed. This presentation is
essentially that of \cite{MeredithR05} with the addition of
polyadicity and summation. For readability we have relegated some of
the technical subtleties to an appendix.

\subsubsection{Process grammar}\label{subsub:process_grammar}

\begin{mathpar}
  \inferrule* [lab=synchronization] {} {{M} \bc \pzero \;|\; x?F \;|\; x!C }
  \and
  \inferrule* [lab=abstraction] {} {{F} \bc (x)P}
  \and
  \inferrule* [lab=concretion] {} {{C} \bc \langle Q \rangle}
  \and
  \inferrule* [lab=process] {} {{P,Q} \bc M \;| \;P|Q \;|\; @{x}}
  \and
  \inferrule* [lab=name] {} {{x} \bc \quotep{P}}
\end{mathpar} 

Note that $\vec{x}$ (resp. $\vec{P}$) denotes a vector of names
(resp. processes) of length $|\vec{x}|$ (resp. $|\vec{P}|$). We adopt
the following useful abbreviations.

\begin{mathpar}
   x?(\vec{y}).P := x.(\vec{y})P \and  x\clift{\vec{P}} := x.\clift{\vec{P}}
   \and x!(y) := \lift{x}{\dropn{y}}
   \and \Pi_{i=0}^{n-1}P_i := P_0 | \ldots | P_{n-1}
\end{mathpar}

\subsubsection{Structural congruence}

\paragraph{Free and bound names and alpha-equivalence.} At the
core of structural equivalence is alpha-equivalence which identifies
process that are the same up to a change of variable. Formally, we
recognize the distinction between free and bound names. The free names
of a process, $\freenames{P}$, may be calculated recursively as
follows:

\begin{mathpar}
\freenames{\pzero} := \emptyset
  \and \\
  \freenames{x?(y).P} := \{ x \} \cup (\freenames{P} \setminus \{ y \})
  \and 
  \freenames{x!\langle P \rangle} := \{ x \} \cup \{ P \} 
  \and \\
  \freenames{P|Q} := \freenames{P} \cup \freenames{Q}
  \and \\
  \freenames{@{x}} := \{ x \}
\end{mathpar}

$\pi$
$\quotep{\pi}$

$\freenames{-} : \pi \to \mathcal{P}(\quotep{\pi})$

\begin{eqnarray*}
  \freenames{\pzero} & := & \emptyset \\
  \freenames{x?(y).P} & := & \{ x \} \cup (\freenames{P} \setminus \{ y \}) \\
  \freenames{x!\langle P \rangle} & := & \{ x \} \cup \{ P \} \\
  \freenames{P|Q} & := & \freenames{P} \cup \freenames{Q} \\
  \freenames{\dropn{x}} & := & \{ x \}
\end{eqnarray*}

The bound names of a process, $\boundnames{P}$, are those names occurring in $P$
that are not free. For example, in $x?(y).0$, the name $x$ is free, while $y$ is bound.

\begin{mathpar}
  \inferrule* [lab=monoidal-laws] {} { P|Q \equiv Q|P \and P|0 \equiv P \and P|(Q|R) \equiv (P|Q)|R }
\end{mathpar}

\begin{mathpar}
  \inferrule* [lab=alpha-equivalence] {} { (x)P \equiv (y)P\{y/x\} \and y \not\in \freenames{P} }
\end{mathpar}

\begin{definition}
Then two processes, $P,Q$, are alpha-equivalent if $P = Q\{\vec{y}/\vec{x}\}$ for
some $\vec{x} \in \boundnames{Q},\vec{y} \in \boundnames{P}$, where $Q\{\vec{y}/\vec{x}\}$
denotes the capture-avoiding substitution of $\vec{y}$ for $\vec{x}$ in $Q$.
\end{definition}

\begin{definition}
  The {\em structural congruence} \cite{SangiorgiWalker} , $\equiv$,
  between processes is the least congruence containing
  alpha-equivalence, satisfying the abelian monoid laws
  (associativity, commutativity and $\pzero$ as identity) for parallel
  composition $|$ and for summation $+$.
\end{definition}

\subsection{Name equivalence}

We take name equivalence, written $\nameeq$, to be the smallest
equivalence relation generated by the following rules.

\begin{mathpar}
\inferrule*[lab=Quote-drop]
{ }
{ \quotep{@{x}} \nameeq x }

\inferrule*[lab=Struct-equiv]
{ P \scong Q }
{ \quotep{P} \nameeq \quotep{Q} }
\end{mathpar}

The astute reader will have noticed that the mutual recursion of names
and processes imposes a mutual recursion on alpha-equivalence and
structural equivalence via name-equivalence. Fortunately, all of this
works out pleasantly and we may calculate in the natural way, free of
concern. The reader interested in the details is referred to the
appendix \ref{appendix:rho_details}.

\subsection{Substitution}

We use $\Proc$ for the set of processes, $\QProc$ for the set of
names, and $\id{\{}\vec{y} / \vec{x} \id{\}}$ to denote partial maps,
$s : \QProc \rightarrow \QProc$. A map, $s$ lifts, uniquely, to a map
on process terms, $\widehat{s} : \Proc \rightarrow \Proc$ by the
following equations.

\begin{mathpar}
  (0) \psubstp{Q}{P} := 0 \\
  (R \juxtap S) \psubstp{Q}{P}
  :=    
  (R)\psubstp{Q}{P} \juxtap (S) \psubstp{Q}{P} \\
  (x?(y).R) \psubstp{Q}{P}    
  :=    
  (x)\substp{Q}{P} (z)\concat( (R \psubstn{z}{y}) \psubstp{Q}{P} ) \\
  (\lift{x}{R}) \psubstp{Q}{P}  
  :=
  \lift{(x)\substp{Q}{P}}{ R \psubstp{Q}{P} } \\
%   (\dropn{x})  \psubstp{Q}{P}       
%   := 
%   \left\{ 
%     \begin{array}{ccc} 
%       \dropn{\quotep{Q}} & & x \nameeq \quotep{P} \\
%       \dropn{x} & & otherwise \\
%     \end{array}
%   \right. 
  (\dropn{x})  \psubstp{Q}{P}       
  := 
  \left\{ 
    \begin{array}{ccc} 
      Q & & x \nameeq \quotep{P} \\
      \dropn{x} & & otherwise \\
    \end{array}
  \right.
\end{mathpar}
 

where

\begin{eqnarray}
  (x)\id{\{} \lpquote Q \rpquote / \lpquote P \rpquote \id{\}}            = 
  \left\{ 
    \begin{array}{ccc}
      \lpquote Q \rpquote & & x \nameeq \lpquote P \rpquote \\
      x & & otherwise \\
    \end{array}
  \right. \nonumber
\end{eqnarray}

and $z$ is chosen distinct from $\quotep{P}$, $\quotep{Q}$, the free
names in $Q$, and all the names in $R$. Our $\alpha$-equivalence will
be built in the standard way from this substitution.

\begin{remark}\label{rem:no_self_referential_names}
  One consequence of these definitions is that $\forall P. \quotep{P}
  \not\in \freenames{P}$.
\end{remark}

\subsection{ Dynamic quote: an example }

Anticipating something of what's to come, consider applying the
substitution, $\widehat{\id{\{}u / z \id{\}}}$, to the following pair
of processes, $\lift{w}{y!(z)}$ and $w[ \lpquote y!(z) \rpquote ]$.

\begin{eqnarray}
	\lift{w}{y!(z)}\widehat{\id{\{}u / z \id{\}}}
		& = &
		\lift{w}{y!(u)} \nonumber\\
	w[ \lpquote y!(z) \rpquote ] \widehat{ \id{\{}u / z \id{\}} }
		& = &
		w[ \lpquote y!(z) \rpquote ] \nonumber
\end{eqnarray}

Because the body of the process between quotes is impervious to
substitution, we get radically different answers. In fact, by
examining the first process in an input context,
e.g. $x?(z).\lift{w}{y!(z)}$, we see that the process under the lift
operator may be shaped by prefixed inputs binding a name inside it. In
this sense, the lift operator will be seen as a way to dynamically
construct processes before reifying them as names.

Finally equipped with these standard features we can present the
dynamics of the calculus.

\subsubsection{Operational semantics} 

Finally, we introduce the computational dynamics. What marks these
algebras as distinct from other more traditionally studied algebraic
structures, e.g. vector spaces or polynomial rings, is the manner in
which dynamics is captured. In traditional structures, dynamics is typically
expressed through morphisms between such structures, as in linear maps
between vector spaces or morphisms between rings. In algebras
associated with the semantics of computation, the dynamics is
expressed as part of the algebraic structure itself, through a
reduction reduction relation typically denoted by $\red$. Below, we
give a recursive presentation of this relation for the calculus used
in the encoding.

$\red \subseteq \pi \times \pi$
$\red : \pi \to \mathcal{P}(\pi)$

\begin{mathpar}
  \inferrule* [lab=Comm] { \textsf{match}( x_{src}, x_{trgt} ) } { x_{trgt}?(y)P \; | \; x_{src}!\langle {Q} \rangle \red P\{\quotep{Q}/y}\} }
  \and \\
  \inferrule* [lab=Par] {{P} \red {P}'} {{{P} | {Q}} \red {{P}' | {Q}}}
  \and
  \inferrule* [lab=Equiv]{{{P} \scong {P}'} \andalso {{P}' \red {Q}'} \andalso {{Q}' \scong {Q}}}{{P} \red {Q}}
\end{mathpar}

\begin{eqnarray*}
  match_{\equiv} (\quotep{P},\quotep{Q}) & := & P \equiv Q \\
  match_{\dagger}(\quotep{P},\quotep{Q}) & := & \forall R. P|Q \red^{*} R => R \red^{*} 0 \\
  match_{K}(\quotep{P},\quotep{Q}) & := & K \mbox{ for some context } K
\end{eqnarray*}

$u?(x)P | u!\langle Q \rangle \red P\{\quotep{Q}/x\}$

%We write $\wred$ for $\red^*$, and $P\red$ if $\exists Q $ such that $ P \red Q$.
We write $P\red$ if $\exists Q $ such that $ P \red Q$ and $P\not\red$, otherwise.

\section{Replication}

As mentioned before, it is known that replication (and hence
recursion) can be implemented in a higher-order process algebra
\cite{SangiorgiWalker}. As our first example of calculation with the
machinery thus far presented we give the construction explicitly in
the {\rhoc}.

\begin{eqnarray}
	D_{x} & := & \prefix{x}{y}{(\binpar{\outputp{x}{y}}{@{y}})} \nonumber\\
	\bangp_{x}{P} & := & \binpar{{x}!\langle{\binpar{D_{x}}{P}}\rangle}{D_{x}} \nonumber
\end{eqnarray}

\begin{eqnarray}
	\bangp_{x}{P} & & \nonumber\\
	=
	& {x}!\langle{(\prefix{x}{y}{(\outputp{x}{y} | @{y})) | P}}\rangle 
	      | \prefix{x}{y}{(\outputp{x}{y} | @{y})} & \nonumber\\
	\red
	& (\outputp{x}{y} | @{y})\substn{\quotep{(\prefix{x}{y}{(@{y} | \outputp{x}{y})) | P}}}{y} & \nonumber\\
	=
	& \outputp{x}{\quotep{(\prefix{x}{y}{(\outputp{x}{y} | @{y})) | P}}}
	  | {(\prefix{x}{y}{(\outputp{x}{y} | @{y})) | P}} & \nonumber\\
	\red
	& \ldots & \nonumber\\
	\red^*
	& P | P | \ldots & \nonumber
\end{eqnarray}

Of course, this encoding, as an implementation, runs away, unfolding
$\bangp{P}$ eagerly. A lazier and more implementable replication
operator, restricted to input-guarded processes, may be obtained as follows.

\begin{eqnarray}
\bangp{\prefix{u}{v}{P}} 
	:= 
	\binpar{\lift{x}{\prefix{u}{v}{(\binpar{D(x)}{P})}}}{D(x)} \nonumber
\end{eqnarray}

\begin{remark}
  Note that the lazier definition still does not deal with summation
  or mixed summation (i.e. sums over input and output). The reader is
  invited to construct definitions of replication that deal with these
  features. 

  Further, the definitions are parameterized in a name, $x$. Can you,
  gentle reader, make a definition that eliminates this parameter and
  guarantees no accidental interaction between the replication
  machinery and the process being replicated -- i.e. no accidental
  sharing of names used by the process to get its work done and the
  name(s) used by the replication to effect copying. This latter
  revision of the definition of replication is crucial to obtaining
  the expected identity $!!P \sim !P$.
\end{remark}

\begin{remark}\label{rem:paradoxical_combinator}
  The reader familiar with the lambda calculus will have noticed the
  similarity between $D$ and the paradoxical combinator.

  [Ed. note: the existence of this seems to suggest we have to be more
  restrictive on the set of processes and names we admit if we are to
  support no-cloning.]
\end{remark}

\subsubsection{Bisimulation}

The computational dynamics gives rise to another kind of equivalence,
the equivalence of computational behavior. As previously mentioned
this is typically captured \emph{via} some form of bisimulation.

% The notion we use in this paper is weak barbed bisimulation
% \cite{milner91polyadicpi}.

The notion we use in this paper is derived from weak barbed
bisimulation \cite{milner91polyadicpi}. 

\begin{definition}
An \emph{observation relation}, $\downarrow_{\mathcal N}$, over a set
of names, $\mathcal N$, is the smallest relation satisfying the rules
below.

\infrule[Out-barb]{y \in {\mathcal N}, \; x \nameeq y}
		  {\outputp{x}{v} \downarrow_{\mathcal N} x}
\infrule[Par-barb]{\mbox{$P\downarrow_{\mathcal N} x$ or $Q\downarrow_{\mathcal N} x$}}
		  {\binpar{P}{Q} \downarrow_{\mathcal N} x}

We write $P \Downarrow_{\mathcal N} x$ if there is $Q$ such that 
$P \wred Q$ and $Q \downarrow_{\mathcal N} x$.
\end{definition}

\begin{definition}
%\label{def.bbisim}
An  ${\mathcal N}$-\emph{barbed bisimulation} over a set of names, ${\mathcal N}$, is a symmetric binary relation 
${\mathcal S}_{\mathcal N}$ between agents such that $P\rel{S}_{\mathcal N}Q$ implies:
\begin{enumerate}
\item If $P \red P'$ then $Q \wred Q'$ and $P'\rel{S}_{\mathcal N} Q'$.
\item If $P\downarrow_{\mathcal N} x$, then $Q\Downarrow_{\mathcal N} x$.
\end{enumerate}
$P$ is ${\mathcal N}$-barbed bisimilar to $Q$, written
$P \wbbisim_{\mathcal N} Q$, if $P \rel{S}_{\mathcal N} Q$ for some ${\mathcal N}$-barbed bisimulation ${\mathcal S}_{\mathcal N}$.
\end{definition}

$\mathcal{R} \subseteq \pi \times \pi$

$P \mathcal{R} Q => \forall P'. P \red P' \Rightarrow \exists Q'. Q \red Q', P' \mathcal{R} Q'$

$P \vdash x \Rightarrow Q \vdash x$

\begin{mathpar}
  \inferrule*[lab=Out-barb]{x \nameeq y}{{y}!\langle{Q}\rangle \vdash x}
  \and
  \inferrule*[lab=Par-barb]{\mbox{$P\vdash x$ or $Q\vdash x$}}{\binpar{P}{Q} \vdash x}
\end{mathpar}

\subsubsection{Contexts}

One of the principle advantages of computational calculi like the
$\pi$-calculus is a well-defined notion of context,
contextual-equivalence and a correlation between
contextual-equivalence and notions of bisimulation. The notion of
context allows the decomposition of a process into (sub-)process and
its syntactic environment, its context. Thus, a context may be
thought of as a process with a ``hole'' (written $\Box$) in it. The
application of a context $M$ to a process $P$, written $M[P]$, is
tantamount to filling the hole in $M$ with $P$. In this paper we do
not need the full weight of this theory, but do make use of the notion
of context in the proof the main theorem. 

\begin{mathpar}
  \inferrule* [lab=summation] {} {{M_{M},M_{N}} \bc \Box \;|\; x.M_{A} \;|\; M_{M}+M_{N}}
  \and
  \inferrule* [lab=agent] {} {{M_{A}} \bc (\vec{x})M_{P} \;| \; \clift{P_0,\ldots,M_{P},\ldots,P_N}}
  \and \\
  \inferrule* [lab=process] {} {{M_{P}} \bc M_{N} \;| \;P|M_{P} }
\end{mathpar} 

\begin{mathpar}
  \inferrule* [lab=sychronization] {} {M_{N} \bc \Box \;|\; x?M_{F} \;|\; x!M_{C}}
  \and
  \inferrule* [lab=abstraction] {} {{M_{F}} \bc (x)M_{P} }
  \and
  \inferrule* [lab=concretion] {} {{M_{C}} \bc \langle M_{P} \rangle }
  \and \\
  \inferrule* [lab=process] {} {{M_{P}} \bc M_{N} \;| \;P|M_{P} }
\end{mathpar}

\begin{definition}[contextual application] Given a context $M$, and
  process $P$, we define the \emph{contextual application}, $M[P] :=
  M\{P/\Box\}$. That is, the contextual application of M to P is the
  substitution of $P$ for $\Box$ in $M$.
\end{definition}

$\meaningof{-} : L \to \mathcal{P}(\pi)$

\begin{mathpar}
  \inferrule* [lab=collection] {} {\meaningof{true} = \pi, \and \meaningof{~E} = \pi \setminus \meaningof{E}, \and \meaningof{E_{1} \& E_{2}} = \meaningof{E_{1}} \cap \meaningof{E_{2}}}
\end{mathpar}

\begin{mathpar}
  \inferrule* [lab=structure] {} {\meaningof{0} = \{ P \in \pi | P \equiv 0 \}, \and \\ \meaningof{E_1 | E_2} = \{ P \in \pi | P \equiv P_{1} | P_{2}, P_{1} \in \meaningof{E_{1}}, P_{2} \in \meaningof{E_2}\} }
\end{mathpar}

\begin{mathpar}
 \inferrule* [lab=behavior] {} {\meaningof{\langle a?b \rangle E} = \{ P \in \pi | P \equiv Q | u?(y)P', \\ \and \\\\ \and \\ \;\;\; u \in \meaningof{a}, \forall z.P'\{z/y\} \in \meaningof{E\{z/b\}}\}, \and \\ \meaningof{a!E} = \{ P \in \pi | P \equiv Q | x!\langle P' \rangle, x \in \meaningof{a} P' \in \meaningof{E}\} }
\end{mathpar}

\begin{mathpar}
 \inferrule* [lab=nominal] {} {\meaningof{\quotep{E}} = \{ \quotep{P} \in \quotep{\pi} | P \in \meaningof{E} \}, \and \meaningof{\quotep{P}} = \{ \quotep{Q} \in \quotep{\pi} | P \equiv Q \} \and \\ \meaningof{@\quotep{E}} = \{ P \in \pi | P \equiv @x, x \in \meaningof{E} \}}
\end{mathpar}

\begin{eqnarray*}
  \\
  \meaningof{-} : TS \to ST
\end{eqnarray*}

\begin{eqnarray*}
  \\
  L : TS \to ST
\end{eqnarray*}

\begin{eqnarray*}
  \\
  P \models E \iff P \in \meaningof{E}
\end{eqnarray*}

\begin{eqnarray*}
  P \approx_{L} Q \iff \forall E \in L. P \models E \iff Q \models E
\end{eqnarray*}

\begin{eqnarray*}
  P \approx_{K} Q
\end{eqnarray*}

\begin{eqnarray*}
  P \approx Q
\end{eqnarray*}

$\approx_{K} = \approx = \approx_{L}$

\subsubsection{Contextual duality}

Note that contexts extend the quotation operation to a family of
operations from processes to names. Given a context, $M$, we can
define a \emph{nominal context}, $\quotep{M}$ by $\quotep{M}[P] :=
\quotep{M[P]}$. To foreshadow what is to come we observe that these
operations enjoy a duality with processes very much like the duality
between vectors and maps from vectors to scalars.

Further, because the calculus is essentially higher-order, we have a
correspondence between contexts and processes. More specifically,
given a name $x$ and a context $M$ we can construct $M^{*}_{x}$ such
that 

\begin{mathpar}
  M^{*}_{x} | \lift{x}{P} \red M[P]
\end{mathpar}

namely,

\begin{mathpar}
  M^{*}_{x} := x?(u).M[\dropn{u}]
\end{mathpar}

The dependence of $M^{*}_{x}$ on a name makes it an abstraction, 

\begin{mathpar}
  M^{*} := (x)x?(u).M[\dropn{u}]
\end{mathpar}

\subsection{Additional notation}

It will sometimes be convenient to denote the process a name
quotes. We already have the notation $x = \quotep{P}$, but it will be
convenient to introduce an alternate notation, $\procn{x}$, when we
want to emphasize the connection to the use of the name. Note that, by
virtue of name equivalence, $\quotep{\procn{x}} \nameeq x$; so, the
notation is consistent with previous definitions.

Further, because names have structure it is possible to effect
substitutions on the basis of that structure. This means we need to
upgrade our notation for substitutions, which we accomplish by
adapting comprehension notation. Thus,

\begin{mathpar}
  P\{ y / x : x \in S \}
\end{mathpar}

is interpreted to mean the process derived from P by replacing (in a
capture-avoiding manner) each occurrence of $x$ in $S$ by $y$. For example,

\begin{mathpar}
  P\{ \quotep{\procn{x}|\procn{x}} / x : x \in \freenames{P} \}
\end{mathpar}

will replace each (occurrence) of a free name $x$ in $P$ by
$\quotep{\procn{x}|\procn{x}}$.

Also, we will avail ourselves of the notation $x^{L}$ and $x^{R}$ to
denote injections of a name into disjoint copies of the name
space. There are numerous ways to accomplish this. One example can be
found in \cite{MeredithR05}. This notation overloads to vectors of
names: $\vec{x}^{\pi} := (x_{i}^{\pi} \; : \; 0 \leq i < |\vec{x}| )$ where $\pi \in \{L,R\}$.

We also use $P^{\Box} := P|\Box$.

In \cite{MeredithR05} an interpretation of the new operator is
given. It turns out that there are several possible interpretations
all enjoying the requisite algebraic properties of the operator (see
\cite{milner91polyadicpi}). We will therefore make liberal use of
$(\nu\; \vec{x})P$.

% subsection the_syntax_and_semantics_of_the_notation_system (end)   

\input{qm2pi.qmops} 

\input{qm2pi.sterngerlach} 

\input{qm2pi.metric} 

% section concurrent_process_calculi (end)

%\input{qm2pi.proofsketch}

% section proof sketch (end)

%\input{qm2pi.slviaknots} 

% section spatial logic via knots (end)

\input{qm2pi.conclusion}

% section conclusion (end)

%\input{qm2pi.dtcodes} 

% section wiring algorithm (end)

\input{qm2pi.ack} 

% section acknowledgments (end)

\newpage


\bibliographystyle{plain}   
\bibliography{../../biblios/main.bib}

\input{qm2pi.rhodetails}

\end{document}

 

% subsection basic_interpretation (end)

%\input{qm2pi.rho.presentation} 
\subsection{The syntax and semantics of the notation system}\label{sub:the_syntax_and_semantics_of_the_notation_system} % (fold)

We now summarize a technical presentation of the calculus that
embodies our theory of dynamics. The typical presentation of such a
calculus follows the style of giving generators and relations on
them. The grammar, below, describing term constructors, freely
generates the set of processes, $\Proc$. This set is then quotiented
by a relation known as structural congruence and it is over this set
that the notion of dynamics is expressed. This presentation is
essentially that of \cite{MeredithR05} with the addition of
polyadicity and summation. For readability we have relegated some of
the technical subtleties to an appendix.

\subsubsection{Process grammar}\label{subsub:process_grammar}

\begin{mathpar}
  \inferrule* [lab=synchronization] {} {{M} \bc \pzero \;|\; x?F \;|\; x!C }
  \and
  \inferrule* [lab=abstraction] {} {{F} \bc (x)P}
  \and
  \inferrule* [lab=concretion] {} {{C} \bc \langle Q \rangle}
  \and
  \inferrule* [lab=process] {} {{P,Q} \bc M \;| \;P|Q \;|\; @{x}}
  \and
  \inferrule* [lab=name] {} {{x} \bc \quotep{P}}
\end{mathpar} 

Note that $\vec{x}$ (resp. $\vec{P}$) denotes a vector of names
(resp. processes) of length $|\vec{x}|$ (resp. $|\vec{P}|$). We adopt
the following useful abbreviations.

\begin{mathpar}
   x?(\vec{y}).P := x.(\vec{y})P \and  x\clift{\vec{P}} := x.\clift{\vec{P}}
   \and x!(y) := \lift{x}{\dropn{y}}
   \and \Pi_{i=0}^{n-1}P_i := P_0 | \ldots | P_{n-1}
\end{mathpar}

\subsubsection{Structural congruence}

\paragraph{Free and bound names and alpha-equivalence.} At the
core of structural equivalence is alpha-equivalence which identifies
process that are the same up to a change of variable. Formally, we
recognize the distinction between free and bound names. The free names
of a process, $\freenames{P}$, may be calculated recursively as
follows:

\begin{mathpar}
\freenames{\pzero} := \emptyset
  \and \\
  \freenames{x?(y).P} := \{ x \} \cup (\freenames{P} \setminus \{ y \})
  \and 
  \freenames{x!\langle P \rangle} := \{ x \} \cup \{ P \} 
  \and \\
  \freenames{P|Q} := \freenames{P} \cup \freenames{Q}
  \and \\
  \freenames{@{x}} := \{ x \}
\end{mathpar}

$\pi$
$\quotep{\pi}$

$\freenames{-} : \pi \to \mathcal{P}(\quotep{\pi})$

\begin{eqnarray*}
  \freenames{\pzero} & := & \emptyset \\
  \freenames{x?(y).P} & := & \{ x \} \cup (\freenames{P} \setminus \{ y \}) \\
  \freenames{x!\langle P \rangle} & := & \{ x \} \cup \{ P \} \\
  \freenames{P|Q} & := & \freenames{P} \cup \freenames{Q} \\
  \freenames{\dropn{x}} & := & \{ x \}
\end{eqnarray*}

The bound names of a process, $\boundnames{P}$, are those names occurring in $P$
that are not free. For example, in $x?(y).0$, the name $x$ is free, while $y$ is bound.

\begin{mathpar}
  \inferrule* [lab=monoidal-laws] {} { P|Q \equiv Q|P \and P|0 \equiv P \and P|(Q|R) \equiv (P|Q)|R }
\end{mathpar}

\begin{mathpar}
  \inferrule* [lab=alpha-equivalence] {} { (x)P \equiv (y)P\{y/x\} \and y \not\in \freenames{P} }
\end{mathpar}

\begin{definition}
Then two processes, $P,Q$, are alpha-equivalent if $P = Q\{\vec{y}/\vec{x}\}$ for
some $\vec{x} \in \boundnames{Q},\vec{y} \in \boundnames{P}$, where $Q\{\vec{y}/\vec{x}\}$
denotes the capture-avoiding substitution of $\vec{y}$ for $\vec{x}$ in $Q$.
\end{definition}

\begin{definition}
  The {\em structural congruence} \cite{SangiorgiWalker} , $\equiv$,
  between processes is the least congruence containing
  alpha-equivalence, satisfying the abelian monoid laws
  (associativity, commutativity and $\pzero$ as identity) for parallel
  composition $|$ and for summation $+$.
\end{definition}

\subsection{Name equivalence}

We take name equivalence, written $\nameeq$, to be the smallest
equivalence relation generated by the following rules.

\begin{mathpar}
\inferrule*[lab=Quote-drop]
{ }
{ \quotep{@{x}} \nameeq x }

\inferrule*[lab=Struct-equiv]
{ P \scong Q }
{ \quotep{P} \nameeq \quotep{Q} }
\end{mathpar}

The astute reader will have noticed that the mutual recursion of names
and processes imposes a mutual recursion on alpha-equivalence and
structural equivalence via name-equivalence. Fortunately, all of this
works out pleasantly and we may calculate in the natural way, free of
concern. The reader interested in the details is referred to the
appendix \ref{appendix:rho_details}.

\subsection{Substitution}

We use $\Proc$ for the set of processes, $\QProc$ for the set of
names, and $\id{\{}\vec{y} / \vec{x} \id{\}}$ to denote partial maps,
$s : \QProc \rightarrow \QProc$. A map, $s$ lifts, uniquely, to a map
on process terms, $\widehat{s} : \Proc \rightarrow \Proc$ by the
following equations.

\begin{mathpar}
  (0) \psubstp{Q}{P} := 0 \\
  (R \juxtap S) \psubstp{Q}{P}
  :=    
  (R)\psubstp{Q}{P} \juxtap (S) \psubstp{Q}{P} \\
  (x?(y).R) \psubstp{Q}{P}    
  :=    
  (x)\substp{Q}{P} (z)\concat( (R \psubstn{z}{y}) \psubstp{Q}{P} ) \\
  (\lift{x}{R}) \psubstp{Q}{P}  
  :=
  \lift{(x)\substp{Q}{P}}{ R \psubstp{Q}{P} } \\
%   (\dropn{x})  \psubstp{Q}{P}       
%   := 
%   \left\{ 
%     \begin{array}{ccc} 
%       \dropn{\quotep{Q}} & & x \nameeq \quotep{P} \\
%       \dropn{x} & & otherwise \\
%     \end{array}
%   \right. 
  (\dropn{x})  \psubstp{Q}{P}       
  := 
  \left\{ 
    \begin{array}{ccc} 
      Q & & x \nameeq \quotep{P} \\
      \dropn{x} & & otherwise \\
    \end{array}
  \right.
\end{mathpar}
 

where

\begin{eqnarray}
  (x)\id{\{} \lpquote Q \rpquote / \lpquote P \rpquote \id{\}}            = 
  \left\{ 
    \begin{array}{ccc}
      \lpquote Q \rpquote & & x \nameeq \lpquote P \rpquote \\
      x & & otherwise \\
    \end{array}
  \right. \nonumber
\end{eqnarray}

and $z$ is chosen distinct from $\quotep{P}$, $\quotep{Q}$, the free
names in $Q$, and all the names in $R$. Our $\alpha$-equivalence will
be built in the standard way from this substitution.

\begin{remark}\label{rem:no_self_referential_names}
  One consequence of these definitions is that $\forall P. \quotep{P}
  \not\in \freenames{P}$.
\end{remark}

\subsection{ Dynamic quote: an example }

Anticipating something of what's to come, consider applying the
substitution, $\widehat{\id{\{}u / z \id{\}}}$, to the following pair
of processes, $\lift{w}{y!(z)}$ and $w[ \lpquote y!(z) \rpquote ]$.

\begin{eqnarray}
	\lift{w}{y!(z)}\widehat{\id{\{}u / z \id{\}}}
		& = &
		\lift{w}{y!(u)} \nonumber\\
	w[ \lpquote y!(z) \rpquote ] \widehat{ \id{\{}u / z \id{\}} }
		& = &
		w[ \lpquote y!(z) \rpquote ] \nonumber
\end{eqnarray}

Because the body of the process between quotes is impervious to
substitution, we get radically different answers. In fact, by
examining the first process in an input context,
e.g. $x?(z).\lift{w}{y!(z)}$, we see that the process under the lift
operator may be shaped by prefixed inputs binding a name inside it. In
this sense, the lift operator will be seen as a way to dynamically
construct processes before reifying them as names.

Finally equipped with these standard features we can present the
dynamics of the calculus.

\subsubsection{Operational semantics} 

Finally, we introduce the computational dynamics. What marks these
algebras as distinct from other more traditionally studied algebraic
structures, e.g. vector spaces or polynomial rings, is the manner in
which dynamics is captured. In traditional structures, dynamics is typically
expressed through morphisms between such structures, as in linear maps
between vector spaces or morphisms between rings. In algebras
associated with the semantics of computation, the dynamics is
expressed as part of the algebraic structure itself, through a
reduction reduction relation typically denoted by $\red$. Below, we
give a recursive presentation of this relation for the calculus used
in the encoding.

$\red \subseteq \pi \times \pi$
$\red : \pi \to \mathcal{P}(\pi)$

\begin{mathpar}
  \inferrule* [lab=Comm] { \textsf{match}( x_{src}, x_{trgt} ) } { x_{trgt}?(y)P \; | \; x_{src}!\langle {Q} \rangle \red P\{\quotep{Q}/y}\} }
  \and \\
  \inferrule* [lab=Par] {{P} \red {P}'} {{{P} | {Q}} \red {{P}' | {Q}}}
  \and
  \inferrule* [lab=Equiv]{{{P} \scong {P}'} \andalso {{P}' \red {Q}'} \andalso {{Q}' \scong {Q}}}{{P} \red {Q}}
\end{mathpar}

\begin{eqnarray*}
  match_{\equiv} (\quotep{P},\quotep{Q}) & := & P \equiv Q \\
  match_{\dagger}(\quotep{P},\quotep{Q}) & := & \forall R. P|Q \red^{*} R => R \red^{*} 0 \\
  match_{K}(\quotep{P},\quotep{Q}) & := & K \mbox{ for some context } K
\end{eqnarray*}

$u?(x)P | u!\langle Q \rangle \red P\{\quotep{Q}/x\}$

%We write $\wred$ for $\red^*$, and $P\red$ if $\exists Q $ such that $ P \red Q$.
We write $P\red$ if $\exists Q $ such that $ P \red Q$ and $P\not\red$, otherwise.

\section{Replication}

As mentioned before, it is known that replication (and hence
recursion) can be implemented in a higher-order process algebra
\cite{SangiorgiWalker}. As our first example of calculation with the
machinery thus far presented we give the construction explicitly in
the {\rhoc}.

\begin{eqnarray}
	D_{x} & := & \prefix{x}{y}{(\binpar{\outputp{x}{y}}{@{y}})} \nonumber\\
	\bangp_{x}{P} & := & \binpar{{x}!\langle{\binpar{D_{x}}{P}}\rangle}{D_{x}} \nonumber
\end{eqnarray}

\begin{eqnarray}
	\bangp_{x}{P} & & \nonumber\\
	=
	& {x}!\langle{(\prefix{x}{y}{(\outputp{x}{y} | @{y})) | P}}\rangle 
	      | \prefix{x}{y}{(\outputp{x}{y} | @{y})} & \nonumber\\
	\red
	& (\outputp{x}{y} | @{y})\substn{\quotep{(\prefix{x}{y}{(@{y} | \outputp{x}{y})) | P}}}{y} & \nonumber\\
	=
	& \outputp{x}{\quotep{(\prefix{x}{y}{(\outputp{x}{y} | @{y})) | P}}}
	  | {(\prefix{x}{y}{(\outputp{x}{y} | @{y})) | P}} & \nonumber\\
	\red
	& \ldots & \nonumber\\
	\red^*
	& P | P | \ldots & \nonumber
\end{eqnarray}

Of course, this encoding, as an implementation, runs away, unfolding
$\bangp{P}$ eagerly. A lazier and more implementable replication
operator, restricted to input-guarded processes, may be obtained as follows.

\begin{eqnarray}
\bangp{\prefix{u}{v}{P}} 
	:= 
	\binpar{\lift{x}{\prefix{u}{v}{(\binpar{D(x)}{P})}}}{D(x)} \nonumber
\end{eqnarray}

\begin{remark}
  Note that the lazier definition still does not deal with summation
  or mixed summation (i.e. sums over input and output). The reader is
  invited to construct definitions of replication that deal with these
  features. 

  Further, the definitions are parameterized in a name, $x$. Can you,
  gentle reader, make a definition that eliminates this parameter and
  guarantees no accidental interaction between the replication
  machinery and the process being replicated -- i.e. no accidental
  sharing of names used by the process to get its work done and the
  name(s) used by the replication to effect copying. This latter
  revision of the definition of replication is crucial to obtaining
  the expected identity $!!P \sim !P$.
\end{remark}

\begin{remark}\label{rem:paradoxical_combinator}
  The reader familiar with the lambda calculus will have noticed the
  similarity between $D$ and the paradoxical combinator.

  [Ed. note: the existence of this seems to suggest we have to be more
  restrictive on the set of processes and names we admit if we are to
  support no-cloning.]
\end{remark}

\subsubsection{Bisimulation}

The computational dynamics gives rise to another kind of equivalence,
the equivalence of computational behavior. As previously mentioned
this is typically captured \emph{via} some form of bisimulation.

% The notion we use in this paper is weak barbed bisimulation
% \cite{milner91polyadicpi}.

The notion we use in this paper is derived from weak barbed
bisimulation \cite{milner91polyadicpi}. 

\begin{definition}
An \emph{observation relation}, $\downarrow_{\mathcal N}$, over a set
of names, $\mathcal N$, is the smallest relation satisfying the rules
below.

\infrule[Out-barb]{y \in {\mathcal N}, \; x \nameeq y}
		  {\outputp{x}{v} \downarrow_{\mathcal N} x}
\infrule[Par-barb]{\mbox{$P\downarrow_{\mathcal N} x$ or $Q\downarrow_{\mathcal N} x$}}
		  {\binpar{P}{Q} \downarrow_{\mathcal N} x}

We write $P \Downarrow_{\mathcal N} x$ if there is $Q$ such that 
$P \wred Q$ and $Q \downarrow_{\mathcal N} x$.
\end{definition}

\begin{definition}
%\label{def.bbisim}
An  ${\mathcal N}$-\emph{barbed bisimulation} over a set of names, ${\mathcal N}$, is a symmetric binary relation 
${\mathcal S}_{\mathcal N}$ between agents such that $P\rel{S}_{\mathcal N}Q$ implies:
\begin{enumerate}
\item If $P \red P'$ then $Q \wred Q'$ and $P'\rel{S}_{\mathcal N} Q'$.
\item If $P\downarrow_{\mathcal N} x$, then $Q\Downarrow_{\mathcal N} x$.
\end{enumerate}
$P$ is ${\mathcal N}$-barbed bisimilar to $Q$, written
$P \wbbisim_{\mathcal N} Q$, if $P \rel{S}_{\mathcal N} Q$ for some ${\mathcal N}$-barbed bisimulation ${\mathcal S}_{\mathcal N}$.
\end{definition}

$\mathcal{R} \subseteq \pi \times \pi$

$P \mathcal{R} Q => \forall P'. P \red P' \Rightarrow \exists Q'. Q \red Q', P' \mathcal{R} Q'$

$P \vdash x \Rightarrow Q \vdash x$

\begin{mathpar}
  \inferrule*[lab=Out-barb]{x \nameeq y}{{y}!\langle{Q}\rangle \vdash x}
  \and
  \inferrule*[lab=Par-barb]{\mbox{$P\vdash x$ or $Q\vdash x$}}{\binpar{P}{Q} \vdash x}
\end{mathpar}

\subsubsection{Contexts}

One of the principle advantages of computational calculi like the
$\pi$-calculus is a well-defined notion of context,
contextual-equivalence and a correlation between
contextual-equivalence and notions of bisimulation. The notion of
context allows the decomposition of a process into (sub-)process and
its syntactic environment, its context. Thus, a context may be
thought of as a process with a ``hole'' (written $\Box$) in it. The
application of a context $M$ to a process $P$, written $M[P]$, is
tantamount to filling the hole in $M$ with $P$. In this paper we do
not need the full weight of this theory, but do make use of the notion
of context in the proof the main theorem. 

\begin{mathpar}
  \inferrule* [lab=summation] {} {{M_{M},M_{N}} \bc \Box \;|\; x.M_{A} \;|\; M_{M}+M_{N}}
  \and
  \inferrule* [lab=agent] {} {{M_{A}} \bc (\vec{x})M_{P} \;| \; \clift{P_0,\ldots,M_{P},\ldots,P_N}}
  \and \\
  \inferrule* [lab=process] {} {{M_{P}} \bc M_{N} \;| \;P|M_{P} }
\end{mathpar} 

\begin{mathpar}
  \inferrule* [lab=sychronization] {} {M_{N} \bc \Box \;|\; x?M_{F} \;|\; x!M_{C}}
  \and
  \inferrule* [lab=abstraction] {} {{M_{F}} \bc (x)M_{P} }
  \and
  \inferrule* [lab=concretion] {} {{M_{C}} \bc \langle M_{P} \rangle }
  \and \\
  \inferrule* [lab=process] {} {{M_{P}} \bc M_{N} \;| \;P|M_{P} }
\end{mathpar}

\begin{definition}[contextual application] Given a context $M$, and
  process $P$, we define the \emph{contextual application}, $M[P] :=
  M\{P/\Box\}$. That is, the contextual application of M to P is the
  substitution of $P$ for $\Box$ in $M$.
\end{definition}

$\meaningof{-} : L \to \mathcal{P}(\pi)$

\begin{mathpar}
  \inferrule* [lab=collection] {} {\meaningof{true} = \pi, \and \meaningof{~E} = \pi \setminus \meaningof{E}, \and \meaningof{E_{1} \& E_{2}} = \meaningof{E_{1}} \cap \meaningof{E_{2}}}
\end{mathpar}

\begin{mathpar}
  \inferrule* [lab=structure] {} {\meaningof{0} = \{ P \in \pi | P \equiv 0 \}, \and \\ \meaningof{E_1 | E_2} = \{ P \in \pi | P \equiv P_{1} | P_{2}, P_{1} \in \meaningof{E_{1}}, P_{2} \in \meaningof{E_2}\} }
\end{mathpar}

\begin{mathpar}
 \inferrule* [lab=behavior] {} {\meaningof{\langle a?b \rangle E} = \{ P \in \pi | P \equiv Q | u?(y)P', \\ \and \\\\ \and \\ \;\;\; u \in \meaningof{a}, \forall z.P'\{z/y\} \in \meaningof{E\{z/b\}}\}, \and \\ \meaningof{a!E} = \{ P \in \pi | P \equiv Q | x!\langle P' \rangle, x \in \meaningof{a} P' \in \meaningof{E}\} }
\end{mathpar}

\begin{mathpar}
 \inferrule* [lab=nominal] {} {\meaningof{\quotep{E}} = \{ \quotep{P} \in \quotep{\pi} | P \in \meaningof{E} \}, \and \meaningof{\quotep{P}} = \{ \quotep{Q} \in \quotep{\pi} | P \equiv Q \} \and \\ \meaningof{@\quotep{E}} = \{ P \in \pi | P \equiv @x, x \in \meaningof{E} \}}
\end{mathpar}

\begin{eqnarray*}
  \\
  \meaningof{-} : TS \to ST
\end{eqnarray*}

\begin{eqnarray*}
  \\
  L : TS \to ST
\end{eqnarray*}

\begin{eqnarray*}
  \\
  P \models E \iff P \in \meaningof{E}
\end{eqnarray*}

\begin{eqnarray*}
  P \approx_{L} Q \iff \forall E \in L. P \models E \iff Q \models E
\end{eqnarray*}

\begin{eqnarray*}
  P \approx_{K} Q
\end{eqnarray*}

\begin{eqnarray*}
  P \approx Q
\end{eqnarray*}

$\approx_{K} = \approx = \approx_{L}$

\subsubsection{Contextual duality}

Note that contexts extend the quotation operation to a family of
operations from processes to names. Given a context, $M$, we can
define a \emph{nominal context}, $\quotep{M}$ by $\quotep{M}[P] :=
\quotep{M[P]}$. To foreshadow what is to come we observe that these
operations enjoy a duality with processes very much like the duality
between vectors and maps from vectors to scalars.

Further, because the calculus is essentially higher-order, we have a
correspondence between contexts and processes. More specifically,
given a name $x$ and a context $M$ we can construct $M^{*}_{x}$ such
that 

\begin{mathpar}
  M^{*}_{x} | \lift{x}{P} \red M[P]
\end{mathpar}

namely,

\begin{mathpar}
  M^{*}_{x} := x?(u).M[\dropn{u}]
\end{mathpar}

The dependence of $M^{*}_{x}$ on a name makes it an abstraction, 

\begin{mathpar}
  M^{*} := (x)x?(u).M[\dropn{u}]
\end{mathpar}

\subsection{Additional notation}

It will sometimes be convenient to denote the process a name
quotes. We already have the notation $x = \quotep{P}$, but it will be
convenient to introduce an alternate notation, $\procn{x}$, when we
want to emphasize the connection to the use of the name. Note that, by
virtue of name equivalence, $\quotep{\procn{x}} \nameeq x$; so, the
notation is consistent with previous definitions.

Further, because names have structure it is possible to effect
substitutions on the basis of that structure. This means we need to
upgrade our notation for substitutions, which we accomplish by
adapting comprehension notation. Thus,

\begin{mathpar}
  P\{ y / x : x \in S \}
\end{mathpar}

is interpreted to mean the process derived from P by replacing (in a
capture-avoiding manner) each occurrence of $x$ in $S$ by $y$. For example,

\begin{mathpar}
  P\{ \quotep{\procn{x}|\procn{x}} / x : x \in \freenames{P} \}
\end{mathpar}

will replace each (occurrence) of a free name $x$ in $P$ by
$\quotep{\procn{x}|\procn{x}}$.

Also, we will avail ourselves of the notation $x^{L}$ and $x^{R}$ to
denote injections of a name into disjoint copies of the name
space. There are numerous ways to accomplish this. One example can be
found in \cite{MeredithR05}. This notation overloads to vectors of
names: $\vec{x}^{\pi} := (x_{i}^{\pi} \; : \; 0 \leq i < |\vec{x}| )$ where $\pi \in \{L,R\}$.

We also use $P^{\Box} := P|\Box$.

In \cite{MeredithR05} an interpretation of the new operator is
given. It turns out that there are several possible interpretations
all enjoying the requisite algebraic properties of the operator (see
\cite{milner91polyadicpi}). We will therefore make liberal use of
$(\nu\; \vec{x})P$.

% subsection the_syntax_and_semantics_of_the_notation_system (end)   

\section{Interpretation of QM}
\subsection{Supporting definitions}
\subsubsection{Multiplication}
\begin{mathpar}
  \quotep{Q} \cdot \quotep{R} := \quotep{Q|R}
  \and \\
  \quotep{Q} \cdot P := P\{ \quotep{Q|R} / \quotep{R} : \quotep{R} \in \freenames{P} \}
\end{mathpar}

\paragraph{Discussion}
The first line needs little explanation. The second line says that
each free name of the process is replaced with the multiplication of
that name by the scalar. Multiplication of a scalar (name) by a state
(process) results in a process all the names of which have been `moved
over' by parallel composition with the process the scalar
quotes. There is a subtlety that the bound names have to be
manipulated so that multiplied names aren't accidentally
captured. There are many ways to achieve this.

\begin{remark}\label{rem:multiplication_identities}
  The reader is invited to verify that for all $x,y,z \in \QProc$ and $P \in \Proc$
  \begin{mathpar}
    x \cdot \quotep{0} \equiv x 
    \and
    x \cdot y \equiv y \cdot x
    \and
    x \cdot (y \cdot z) \equiv (x \cdot y) \cdot z
    \and \\
    \quotep{0} \cdot P \equiv P
    \and \\
    x \cdot (y \cdot P) \equiv (x \cdot y) \cdot P
    \and \\
    x \cdot (P|Q) \equiv (x \cdot P) | (x \cdot Q)
    \and \\    
  \end{mathpar}
\end{remark}

\subsubsection{Tensor product}

We define a tensor product on processes by structural induction.

\paragraph{Tensor of sums} First note that all summations, including
$\pzero$ and sequence, can be written $\Sigma_{i} x_{i}.A_{i} +
\Sigma_{j} x_{j}.C_{j}$, where we have grouped input-guarded processes
together and output-guarded processes together.

Thus, we can define the tensor product of two summations, $N_{1}\otimes N_{2}$, where

\begin{mathpar}
  N_{1} := \Sigma_{i} x_{i}.A_{i} + \Sigma_{j} x_{j}.C_{j}
  \and
  N_{2} := \Sigma_{i'} y_{i'}.B_{i'} + \Sigma_{j'} y_{j'}.D_{j'} 
\end{mathpar}

as follows.

\begin{mathpar}
  \Sigma_{i} x_{i}.A_{i} + \Sigma_{j} x_{j}.C_{j} \otimes \Sigma_{i'}
  y_{i'}.B_{i'} + \Sigma_{j'} y_{j'}.D_{j'} 
  \and \\
  := \; \Sigma_{i} \Sigma_{i'} \quotep{\stackrel{\vee}{x_{i}}| \stackrel{\vee}{y_{i'}}}.(A_{i}\otimes B_{i'}) \; | \; \Sigma_{i'} \Sigma_{i} \quotep{\stackrel{\vee}{y_{i'}}|\stackrel{\vee}{x_{i}}}.(B_{i'}\otimes A_{i})
  \and
  \;\; | \;\; \Sigma_{j} \Sigma_{j'} \quotep{\stackrel{\vee}{x_{j}}|\stackrel{\vee}{y_{j'}}}.(A_{j}\otimes B_{j'}) \; | \; \Sigma_{j'} \Sigma_{j} \quotep{\stackrel{\vee}{y_{j'}}|\stackrel{\vee}{x_{j}}}.(B_{j'}\otimes A_{j})
\end{mathpar}

\begin{remark}
  Do we need to $x^{L}$ and $y^{R}$ for this construction as well?
\end{remark}

\paragraph{Tensor of parallel compositions} Next, we distribute tensor
over par.

\begin{mathpar}
  P_{1}|P_{2} \otimes Q_{1}|Q_{2} := (P_{1} \otimes Q_{1}) | (P_{1}
  \otimes Q_{2}) | (P_{2} \otimes Q_{1}) | (P_{2} \otimes Q_{2})
\end{mathpar}

\paragraph{Tensor with dropped names} We treat tensor of a
process with a dropped name as parallel composition.

\begin{mathpar}
  P \otimes \dropn{x} := P | \dropn{x}
\end{mathpar}

\paragraph{Tensor of agents}

Finally, we need to define tensor on agents. Note that the definition
of tensor on normal products only tensors inputs with inputs and
outputs with outputs. Thus, we only have to define the operation on
``homogeneous'' pairings.

\begin{mathpar}
  (\vec{x})P \otimes (\vec{y})Q
  \and \\
  := (x_{0}^{L}|y_{0}^{R},\ldots,x_{0}^{L}|y_{n}^{R},\ldots,x_{m}^{L}|y_{0}^{R},\ldots,x_{m}^{L}|y_{n}^R)(P\{ \vec{x}^{L}/\vec{x}\} \otimes Q \{ \vec{y}^{R}/\vec{y}\})
  \and \\
  \clift{\vec{P}} \otimes \clift{\vec{Q}}
  \and \\
  := \clift{P_{0}\otimes Q_{0},\ldots,P_{0}\otimes Q_{n},\ldots,P_{m}\otimes Q_{0},\ldots,P_{m}\otimes Q_{n}}
\end{mathpar}

\begin{remark}
  Observe that arities of tensored abstractions matches arities of
  tensored concretions if the original arities matched. Note also that
  the length of the arities corresponds to the increase in dimension
  we see in ordinary vector space tensor product.
\end{remark}

\begin{remark}
  Operationally, this definition distributes the tensor down to
  components ``linked'' by summation. Tensor over summation is
  intriguing in that it mixes names. Moreover, as a consequence of the
  way it mixes names we have the identities for all $x \in \QProc$ and
  $P,Q \in \Proc$

  \begin{mathpar}
    (x \cdot P) \otimes Q \equiv x \cdot (P \otimes Q) \equiv P \otimes (x \cdot Q)
    \and
    P \otimes \pzero \equiv P
  \end{mathpar}

  that the reader is invited to verify.
\end{remark}

\subsubsection{Annihilation}
\begin{mathpar}
  P^{\perp} := \{ Q | \forall R. P|Q \red^{*} R \Rightarrow R \red^{*} \pzero \}
  \and \\
  P^{\underline{\perp}} := \Sigma_{Q \in P^{\perp}} \quotep{Q}?(y).(\dropn{y}|Q) | \Sigma_{Q \in P^{\perp}} \quotep{Q}\clift{\Box}
\end{mathpar}

\paragraph{Discussion} The reader will note that $P^{\perp}$ is a
\emph{set} of processes, while $P^{\underline{\perp}}$ is a
\emph{context}. We call the set $P^{\perp}$ the \emph{annihilators} of
$P$. The parallel composition of a process in the annihilators of $P$
with $P$ will result in a process, the state space of which has all
paths eventually leading to $\pzero$. Execution may endure loops; but
under reasonable conditions of fairness (naturally guaranteed under
most notions of bisimulation) such a composite process cannot get
stuck in such a loop and will, eventually pop out and terminate.

The context $P^{\underline{\perp}}$ is ready and willing to ``take the
$P$ out of'' the process to which it is applied. It will effectively
transmit the code of the process to which it is applied to one of the
annihilators and run the process against it.

\subsubsection{Evaluation}
We fix $M$ a domain of fully abstract interpretation with an equality
coincident with bisimulation. We take $\meaningof{\cdot} : \Proc \to
M$ to be the map interpreting processes and $\nmeaningof{\cdot} : \M
\to Proc$ to be the map running the other way. Then we define

\begin{mathpar}
  \int P := \nmeaningof{\meaningof{P}}
\end{mathpar}

\paragraph{Discussion}
There are many fully abstract interpretations of Milner's
$\pi$-calculus. Any of them can be used as a basis for interpreting
the reflective calculus here. Equipped with such a domain it is
largely a matter of grinding through to check that the Yoneda
construction for the normalization-by-evaluation program can be
extended to this setting.

\begin{remark}
  The reader is invited to verify that $\int (P^{\underline{\perp}}[P]) = 0$.
\end{remark}

\subsection{Quantum mechanics}

Table \ref{tbl:core_qm_op_defns} gives the core operational definitions

\begin{table}[htp]\label{tbl:core_qm_op_defns}
  \center{
    \fbox{
      \begin{tabular}{c|c}
        quantum mechanics & process calculus \\
        \hline
        scalar & $x := \quotep{P}$ \\
        state vector & $\state{P} := P$ \\
        dual & $\state{P}^{*} := \event{P^{\underline{\perp}}} := \quotep{P^{\underline{\perp}}}[-]$ \\
        matrix & $ \Sigma_{\alpha} \state{P_{\alpha}}x_{\alpha}\event{Q_{\alpha}}$ \\
        vector addition & $\state{P} + \state{Q} := \state{P | Q}$ \\
        tensor product & $\state{P} \otimes \state{Q} := \state{P \otimes Q}$ \\
        inner product & $\innerprod{P}{Q} := \quotep{\int P^{\underline{\perp}}[Q]}$ \\
      \end{tabular}
    }
  }
  \caption{QM - operational definitions}
\end{table}

where

\begin{mathpar}
  \prmatrix{P}{Q} := \fprmatrix{P}{\quotep{\pzero}}{Q}
  \and
  \fprmatrix{P}{x}{Q} := (\state{P},x,\event{Q})
  \and
  (\fprmatrix{P}{x}{Q})(\state{R}) := x \cdot \innerprod{Q}{R} \cdot \state{P}
  \and
  (\fprmatrix{P}{x}{Q})(\event{R}) := x \cdot \innerprod{R}{P} \cdot \event{Q}
\end{mathpar}

\paragraph{Discussion}
As promised: vectors (aka states) are represented as processes; duals
as contextual duals; inner product definition should be compared with
standard inner product definition for ....

\begin{remark}
  Assuming $\int (P^{\underline{\perp}}[P]) = 0$, the reader is
  invited to verify that $(\fprmatrix{P}{x}{P})(\state{P}) = x \cdot \state{P}$.
\end{remark}

\begin{remark}
  The reader is invited to verify that $\innerprod{P}{Q}$ could
  equally well have been written $\quotep{\int \stackrel{\vee}{x}}$
  where $x = \event{P^{\underline{\perp}}}(Q)$.

  One of the motivations for this remark is that there is another way
  to factor these operations. We could package up evaluation in the dual:

  \begin{mathpar}
    \state{P}^{*} := \event{\int P^{\underline{\perp}}} := \quotep{\int P^{\underline{\perp}}}[-]
  \end{mathpar}

  and then have inner product defined by
  
  \begin{mathpar}
    \innerprod{P}{Q} := \event{P}(Q)
  \end{mathpar}

  Hopefully, experience with the calculations will provide guidance on
  the best factoring.
\end{remark}

\begin{remark}
  Assuming $\int (P^{\underline{\perp}}[P]) = 0$, the reader is
  invited to verify that $\forall P,Q. (\prmatrix{0}{Q})(\state{0}) =
  \state{0}$ and dually $(\prmatrix{P}{0})(\event{0}) = \event{0}$.
\end{remark}

\begin{remark}
  i'm a little worried that i don't (yet) have proper support for
  complex conjugacy. But, the observation above may give us a
  clue. According to Abramsky, it must be the case that the scalars
  are iso to the homset of the identity for the tensor -- which the
  observation above characterizes. 

  For now, we will simply bookmark the notion with $\overline{x}$.
\end{remark}

\subsubsection{Adjointness}

We need to give a definition of $(\cdot)^{\dagger}$ for matrices. The
obvious candidate definition is
\begin{mathpar}
(\Sigma_{\alpha}\fprmatrix{P_{\alpha}}{x_{\alpha}}{Q_{\alpha}})^{\dagger}
= \Sigma_{\alpha}\fprmatrix{(Q_{\alpha}^{\underline{\perp}})^{*}}{\overline{x}_{\alpha}}{P_{\alpha}^{\underline{\perp}}} 
\end{mathpar}

But, $(Q_{\alpha}^{\underline{\perp}})^{*}$ requires a name along
which to communicate the process to achieve the context application.

\subsubsection{Basis for a basis}
If processes label states and ``addition'' of states (a.k.a. vector
addition) is interpreted as parallel composition, what corresponds to
notions of linear independence and basis? Here, we recall that Yoshida
has developed a set of \emph{combinators} for an asynchronous verison
of Milner's $\pi$-calculus. These are a finite set of processes such
any process can be expressed as parallel composition of these
combinators together with liberal uses of the new operator and
replication. We can simply give a translation of these into the
present calculus and have reasonable expectation that the property
carries over. That is, that the resultant set allows to express all
processes via parallel composition. Note, however, that there is no
new operator or replication in this calculus. As a result, we expect
that the corresponding set is actually infinite. That is, we expect
that the space is actually infinite dimensional.

\begin{remark}
  The attentive reader may be a bit concerned. Certainly, the
  collection $S$, $K$ and $I$ is a finite set of
  combinators. Shouldn't we expect to see a finite set of combinators
  for an effectively equivalent system? i am very sympathetic to this
  critique and feel it warrants full attention. On the other hand, i
  also have in mind the following analogy. The natural numbers, as a
  monoid under addition, has exactly $1$ generator, while the natural
  numbers, as a monoid under multiplication, has countably many
  generators (the primes). We observe that the application of the
  lambda calculus is much less resource sensitive than the parallel
  composition of the $\pi$-calculus. Could it be the case that we have
  an analogy of the form
  
  \begin{mathpar}
    m + n : MN :: m*n : M|N
  \end{mathpar}

  giving a similar blow up in the set of ``primes''?  This is such a
  wonderful thought that, even if it's not true, i think it's worth
  writing down.
\end{remark}
 

\documentclass[12pt]{llncs}
%\documentclass{jktr}

\usepackage[pdftex]{hyperref}                   
\usepackage {listings}
\usepackage {mathpartir}
\usepackage{bcprules}
%\usepackage{listings}
                       
\usepackage{graphicx} 
%\usepackage[margins=2.5cm,nohead,nofoot]{geometry}
%\usepackage{geometry}
\usepackage{amsfonts}
\usepackage{amstext}
\usepackage{latexsym}
\usepackage{amssymb}
\usepackage{color}


%\include{myPreamble}
\include{qm2pi.local} 

%\ifpdf
%\usepackage[pdftex]{graphicx}
%\else
%\usepackage{graphicx}
%\fi

 % \ifpdf
%  \usepackage{pdfsync}
%  \if


%\title{Brief Article}
%\author{David F. Snyder}
%\author{L.G. Meredith}

%\address{Dept. of Math., Texas State University--San Marcos, San Marcos, TX 78666}
       
\pagestyle{empty}


\begin{document}

\lstset{language=[Objective]Caml,frame=shadowbox}

\input{qm2pi.front}

% section front matter (end)

\input{qm2pi.intro} 
 
% section introduction (end)

% \input{qm2pi.knotations} 

% section notation (end)

\input{qm2pi.process.calculi} 

% section concurrent_process_calculi_and_spatial_logics_ (end)
    
%\input{qm2pi.knots2pi} 

%\input{qm2pi.trefoil} 

%\input{qm2pi.mainthm} 

% subsection basic_interpretation (end)

%\input{qm2pi.rho.presentation} 
\subsection{The syntax and semantics of the notation system}\label{sub:the_syntax_and_semantics_of_the_notation_system} % (fold)

We now summarize a technical presentation of the calculus that
embodies our theory of dynamics. The typical presentation of such a
calculus follows the style of giving generators and relations on
them. The grammar, below, describing term constructors, freely
generates the set of processes, $\Proc$. This set is then quotiented
by a relation known as structural congruence and it is over this set
that the notion of dynamics is expressed. This presentation is
essentially that of \cite{MeredithR05} with the addition of
polyadicity and summation. For readability we have relegated some of
the technical subtleties to an appendix.

\subsubsection{Process grammar}\label{subsub:process_grammar}

\begin{mathpar}
  \inferrule* [lab=synchronization] {} {{M} \bc \pzero \;|\; x?F \;|\; x!C }
  \and
  \inferrule* [lab=abstraction] {} {{F} \bc (x)P}
  \and
  \inferrule* [lab=concretion] {} {{C} \bc \langle Q \rangle}
  \and
  \inferrule* [lab=process] {} {{P,Q} \bc M \;| \;P|Q \;|\; @{x}}
  \and
  \inferrule* [lab=name] {} {{x} \bc \quotep{P}}
\end{mathpar} 

Note that $\vec{x}$ (resp. $\vec{P}$) denotes a vector of names
(resp. processes) of length $|\vec{x}|$ (resp. $|\vec{P}|$). We adopt
the following useful abbreviations.

\begin{mathpar}
   x?(\vec{y}).P := x.(\vec{y})P \and  x\clift{\vec{P}} := x.\clift{\vec{P}}
   \and x!(y) := \lift{x}{\dropn{y}}
   \and \Pi_{i=0}^{n-1}P_i := P_0 | \ldots | P_{n-1}
\end{mathpar}

\subsubsection{Structural congruence}

\paragraph{Free and bound names and alpha-equivalence.} At the
core of structural equivalence is alpha-equivalence which identifies
process that are the same up to a change of variable. Formally, we
recognize the distinction between free and bound names. The free names
of a process, $\freenames{P}$, may be calculated recursively as
follows:

\begin{mathpar}
\freenames{\pzero} := \emptyset
  \and \\
  \freenames{x?(y).P} := \{ x \} \cup (\freenames{P} \setminus \{ y \})
  \and 
  \freenames{x!\langle P \rangle} := \{ x \} \cup \{ P \} 
  \and \\
  \freenames{P|Q} := \freenames{P} \cup \freenames{Q}
  \and \\
  \freenames{@{x}} := \{ x \}
\end{mathpar}

$\pi$
$\quotep{\pi}$

$\freenames{-} : \pi \to \mathcal{P}(\quotep{\pi})$

\begin{eqnarray*}
  \freenames{\pzero} & := & \emptyset \\
  \freenames{x?(y).P} & := & \{ x \} \cup (\freenames{P} \setminus \{ y \}) \\
  \freenames{x!\langle P \rangle} & := & \{ x \} \cup \{ P \} \\
  \freenames{P|Q} & := & \freenames{P} \cup \freenames{Q} \\
  \freenames{\dropn{x}} & := & \{ x \}
\end{eqnarray*}

The bound names of a process, $\boundnames{P}$, are those names occurring in $P$
that are not free. For example, in $x?(y).0$, the name $x$ is free, while $y$ is bound.

\begin{mathpar}
  \inferrule* [lab=monoidal-laws] {} { P|Q \equiv Q|P \and P|0 \equiv P \and P|(Q|R) \equiv (P|Q)|R }
\end{mathpar}

\begin{mathpar}
  \inferrule* [lab=alpha-equivalence] {} { (x)P \equiv (y)P\{y/x\} \and y \not\in \freenames{P} }
\end{mathpar}

\begin{definition}
Then two processes, $P,Q$, are alpha-equivalent if $P = Q\{\vec{y}/\vec{x}\}$ for
some $\vec{x} \in \boundnames{Q},\vec{y} \in \boundnames{P}$, where $Q\{\vec{y}/\vec{x}\}$
denotes the capture-avoiding substitution of $\vec{y}$ for $\vec{x}$ in $Q$.
\end{definition}

\begin{definition}
  The {\em structural congruence} \cite{SangiorgiWalker} , $\equiv$,
  between processes is the least congruence containing
  alpha-equivalence, satisfying the abelian monoid laws
  (associativity, commutativity and $\pzero$ as identity) for parallel
  composition $|$ and for summation $+$.
\end{definition}

\subsection{Name equivalence}

We take name equivalence, written $\nameeq$, to be the smallest
equivalence relation generated by the following rules.

\begin{mathpar}
\inferrule*[lab=Quote-drop]
{ }
{ \quotep{@{x}} \nameeq x }

\inferrule*[lab=Struct-equiv]
{ P \scong Q }
{ \quotep{P} \nameeq \quotep{Q} }
\end{mathpar}

The astute reader will have noticed that the mutual recursion of names
and processes imposes a mutual recursion on alpha-equivalence and
structural equivalence via name-equivalence. Fortunately, all of this
works out pleasantly and we may calculate in the natural way, free of
concern. The reader interested in the details is referred to the
appendix \ref{appendix:rho_details}.

\subsection{Substitution}

We use $\Proc$ for the set of processes, $\QProc$ for the set of
names, and $\id{\{}\vec{y} / \vec{x} \id{\}}$ to denote partial maps,
$s : \QProc \rightarrow \QProc$. A map, $s$ lifts, uniquely, to a map
on process terms, $\widehat{s} : \Proc \rightarrow \Proc$ by the
following equations.

\begin{mathpar}
  (0) \psubstp{Q}{P} := 0 \\
  (R \juxtap S) \psubstp{Q}{P}
  :=    
  (R)\psubstp{Q}{P} \juxtap (S) \psubstp{Q}{P} \\
  (x?(y).R) \psubstp{Q}{P}    
  :=    
  (x)\substp{Q}{P} (z)\concat( (R \psubstn{z}{y}) \psubstp{Q}{P} ) \\
  (\lift{x}{R}) \psubstp{Q}{P}  
  :=
  \lift{(x)\substp{Q}{P}}{ R \psubstp{Q}{P} } \\
%   (\dropn{x})  \psubstp{Q}{P}       
%   := 
%   \left\{ 
%     \begin{array}{ccc} 
%       \dropn{\quotep{Q}} & & x \nameeq \quotep{P} \\
%       \dropn{x} & & otherwise \\
%     \end{array}
%   \right. 
  (\dropn{x})  \psubstp{Q}{P}       
  := 
  \left\{ 
    \begin{array}{ccc} 
      Q & & x \nameeq \quotep{P} \\
      \dropn{x} & & otherwise \\
    \end{array}
  \right.
\end{mathpar}
 

where

\begin{eqnarray}
  (x)\id{\{} \lpquote Q \rpquote / \lpquote P \rpquote \id{\}}            = 
  \left\{ 
    \begin{array}{ccc}
      \lpquote Q \rpquote & & x \nameeq \lpquote P \rpquote \\
      x & & otherwise \\
    \end{array}
  \right. \nonumber
\end{eqnarray}

and $z$ is chosen distinct from $\quotep{P}$, $\quotep{Q}$, the free
names in $Q$, and all the names in $R$. Our $\alpha$-equivalence will
be built in the standard way from this substitution.

\begin{remark}\label{rem:no_self_referential_names}
  One consequence of these definitions is that $\forall P. \quotep{P}
  \not\in \freenames{P}$.
\end{remark}

\subsection{ Dynamic quote: an example }

Anticipating something of what's to come, consider applying the
substitution, $\widehat{\id{\{}u / z \id{\}}}$, to the following pair
of processes, $\lift{w}{y!(z)}$ and $w[ \lpquote y!(z) \rpquote ]$.

\begin{eqnarray}
	\lift{w}{y!(z)}\widehat{\id{\{}u / z \id{\}}}
		& = &
		\lift{w}{y!(u)} \nonumber\\
	w[ \lpquote y!(z) \rpquote ] \widehat{ \id{\{}u / z \id{\}} }
		& = &
		w[ \lpquote y!(z) \rpquote ] \nonumber
\end{eqnarray}

Because the body of the process between quotes is impervious to
substitution, we get radically different answers. In fact, by
examining the first process in an input context,
e.g. $x?(z).\lift{w}{y!(z)}$, we see that the process under the lift
operator may be shaped by prefixed inputs binding a name inside it. In
this sense, the lift operator will be seen as a way to dynamically
construct processes before reifying them as names.

Finally equipped with these standard features we can present the
dynamics of the calculus.

\subsubsection{Operational semantics} 

Finally, we introduce the computational dynamics. What marks these
algebras as distinct from other more traditionally studied algebraic
structures, e.g. vector spaces or polynomial rings, is the manner in
which dynamics is captured. In traditional structures, dynamics is typically
expressed through morphisms between such structures, as in linear maps
between vector spaces or morphisms between rings. In algebras
associated with the semantics of computation, the dynamics is
expressed as part of the algebraic structure itself, through a
reduction reduction relation typically denoted by $\red$. Below, we
give a recursive presentation of this relation for the calculus used
in the encoding.

$\red \subseteq \pi \times \pi$
$\red : \pi \to \mathcal{P}(\pi)$

\begin{mathpar}
  \inferrule* [lab=Comm] { \textsf{match}( x_{src}, x_{trgt} ) } { x_{trgt}?(y)P \; | \; x_{src}!\langle {Q} \rangle \red P\{\quotep{Q}/y}\} }
  \and \\
  \inferrule* [lab=Par] {{P} \red {P}'} {{{P} | {Q}} \red {{P}' | {Q}}}
  \and
  \inferrule* [lab=Equiv]{{{P} \scong {P}'} \andalso {{P}' \red {Q}'} \andalso {{Q}' \scong {Q}}}{{P} \red {Q}}
\end{mathpar}

\begin{eqnarray*}
  match_{\equiv} (\quotep{P},\quotep{Q}) & := & P \equiv Q \\
  match_{\dagger}(\quotep{P},\quotep{Q}) & := & \forall R. P|Q \red^{*} R => R \red^{*} 0 \\
  match_{K}(\quotep{P},\quotep{Q}) & := & K \mbox{ for some context } K
\end{eqnarray*}

$u?(x)P | u!\langle Q \rangle \red P\{\quotep{Q}/x\}$

%We write $\wred$ for $\red^*$, and $P\red$ if $\exists Q $ such that $ P \red Q$.
We write $P\red$ if $\exists Q $ such that $ P \red Q$ and $P\not\red$, otherwise.

\section{Replication}

As mentioned before, it is known that replication (and hence
recursion) can be implemented in a higher-order process algebra
\cite{SangiorgiWalker}. As our first example of calculation with the
machinery thus far presented we give the construction explicitly in
the {\rhoc}.

\begin{eqnarray}
	D_{x} & := & \prefix{x}{y}{(\binpar{\outputp{x}{y}}{@{y}})} \nonumber\\
	\bangp_{x}{P} & := & \binpar{{x}!\langle{\binpar{D_{x}}{P}}\rangle}{D_{x}} \nonumber
\end{eqnarray}

\begin{eqnarray}
	\bangp_{x}{P} & & \nonumber\\
	=
	& {x}!\langle{(\prefix{x}{y}{(\outputp{x}{y} | @{y})) | P}}\rangle 
	      | \prefix{x}{y}{(\outputp{x}{y} | @{y})} & \nonumber\\
	\red
	& (\outputp{x}{y} | @{y})\substn{\quotep{(\prefix{x}{y}{(@{y} | \outputp{x}{y})) | P}}}{y} & \nonumber\\
	=
	& \outputp{x}{\quotep{(\prefix{x}{y}{(\outputp{x}{y} | @{y})) | P}}}
	  | {(\prefix{x}{y}{(\outputp{x}{y} | @{y})) | P}} & \nonumber\\
	\red
	& \ldots & \nonumber\\
	\red^*
	& P | P | \ldots & \nonumber
\end{eqnarray}

Of course, this encoding, as an implementation, runs away, unfolding
$\bangp{P}$ eagerly. A lazier and more implementable replication
operator, restricted to input-guarded processes, may be obtained as follows.

\begin{eqnarray}
\bangp{\prefix{u}{v}{P}} 
	:= 
	\binpar{\lift{x}{\prefix{u}{v}{(\binpar{D(x)}{P})}}}{D(x)} \nonumber
\end{eqnarray}

\begin{remark}
  Note that the lazier definition still does not deal with summation
  or mixed summation (i.e. sums over input and output). The reader is
  invited to construct definitions of replication that deal with these
  features. 

  Further, the definitions are parameterized in a name, $x$. Can you,
  gentle reader, make a definition that eliminates this parameter and
  guarantees no accidental interaction between the replication
  machinery and the process being replicated -- i.e. no accidental
  sharing of names used by the process to get its work done and the
  name(s) used by the replication to effect copying. This latter
  revision of the definition of replication is crucial to obtaining
  the expected identity $!!P \sim !P$.
\end{remark}

\begin{remark}\label{rem:paradoxical_combinator}
  The reader familiar with the lambda calculus will have noticed the
  similarity between $D$ and the paradoxical combinator.

  [Ed. note: the existence of this seems to suggest we have to be more
  restrictive on the set of processes and names we admit if we are to
  support no-cloning.]
\end{remark}

\subsubsection{Bisimulation}

The computational dynamics gives rise to another kind of equivalence,
the equivalence of computational behavior. As previously mentioned
this is typically captured \emph{via} some form of bisimulation.

% The notion we use in this paper is weak barbed bisimulation
% \cite{milner91polyadicpi}.

The notion we use in this paper is derived from weak barbed
bisimulation \cite{milner91polyadicpi}. 

\begin{definition}
An \emph{observation relation}, $\downarrow_{\mathcal N}$, over a set
of names, $\mathcal N$, is the smallest relation satisfying the rules
below.

\infrule[Out-barb]{y \in {\mathcal N}, \; x \nameeq y}
		  {\outputp{x}{v} \downarrow_{\mathcal N} x}
\infrule[Par-barb]{\mbox{$P\downarrow_{\mathcal N} x$ or $Q\downarrow_{\mathcal N} x$}}
		  {\binpar{P}{Q} \downarrow_{\mathcal N} x}

We write $P \Downarrow_{\mathcal N} x$ if there is $Q$ such that 
$P \wred Q$ and $Q \downarrow_{\mathcal N} x$.
\end{definition}

\begin{definition}
%\label{def.bbisim}
An  ${\mathcal N}$-\emph{barbed bisimulation} over a set of names, ${\mathcal N}$, is a symmetric binary relation 
${\mathcal S}_{\mathcal N}$ between agents such that $P\rel{S}_{\mathcal N}Q$ implies:
\begin{enumerate}
\item If $P \red P'$ then $Q \wred Q'$ and $P'\rel{S}_{\mathcal N} Q'$.
\item If $P\downarrow_{\mathcal N} x$, then $Q\Downarrow_{\mathcal N} x$.
\end{enumerate}
$P$ is ${\mathcal N}$-barbed bisimilar to $Q$, written
$P \wbbisim_{\mathcal N} Q$, if $P \rel{S}_{\mathcal N} Q$ for some ${\mathcal N}$-barbed bisimulation ${\mathcal S}_{\mathcal N}$.
\end{definition}

$\mathcal{R} \subseteq \pi \times \pi$

$P \mathcal{R} Q => \forall P'. P \red P' \Rightarrow \exists Q'. Q \red Q', P' \mathcal{R} Q'$

$P \vdash x \Rightarrow Q \vdash x$

\begin{mathpar}
  \inferrule*[lab=Out-barb]{x \nameeq y}{{y}!\langle{Q}\rangle \vdash x}
  \and
  \inferrule*[lab=Par-barb]{\mbox{$P\vdash x$ or $Q\vdash x$}}{\binpar{P}{Q} \vdash x}
\end{mathpar}

\subsubsection{Contexts}

One of the principle advantages of computational calculi like the
$\pi$-calculus is a well-defined notion of context,
contextual-equivalence and a correlation between
contextual-equivalence and notions of bisimulation. The notion of
context allows the decomposition of a process into (sub-)process and
its syntactic environment, its context. Thus, a context may be
thought of as a process with a ``hole'' (written $\Box$) in it. The
application of a context $M$ to a process $P$, written $M[P]$, is
tantamount to filling the hole in $M$ with $P$. In this paper we do
not need the full weight of this theory, but do make use of the notion
of context in the proof the main theorem. 

\begin{mathpar}
  \inferrule* [lab=summation] {} {{M_{M},M_{N}} \bc \Box \;|\; x.M_{A} \;|\; M_{M}+M_{N}}
  \and
  \inferrule* [lab=agent] {} {{M_{A}} \bc (\vec{x})M_{P} \;| \; \clift{P_0,\ldots,M_{P},\ldots,P_N}}
  \and \\
  \inferrule* [lab=process] {} {{M_{P}} \bc M_{N} \;| \;P|M_{P} }
\end{mathpar} 

\begin{mathpar}
  \inferrule* [lab=sychronization] {} {M_{N} \bc \Box \;|\; x?M_{F} \;|\; x!M_{C}}
  \and
  \inferrule* [lab=abstraction] {} {{M_{F}} \bc (x)M_{P} }
  \and
  \inferrule* [lab=concretion] {} {{M_{C}} \bc \langle M_{P} \rangle }
  \and \\
  \inferrule* [lab=process] {} {{M_{P}} \bc M_{N} \;| \;P|M_{P} }
\end{mathpar}

\begin{definition}[contextual application] Given a context $M$, and
  process $P$, we define the \emph{contextual application}, $M[P] :=
  M\{P/\Box\}$. That is, the contextual application of M to P is the
  substitution of $P$ for $\Box$ in $M$.
\end{definition}

$\meaningof{-} : L \to \mathcal{P}(\pi)$

\begin{mathpar}
  \inferrule* [lab=collection] {} {\meaningof{true} = \pi, \and \meaningof{~E} = \pi \setminus \meaningof{E}, \and \meaningof{E_{1} \& E_{2}} = \meaningof{E_{1}} \cap \meaningof{E_{2}}}
\end{mathpar}

\begin{mathpar}
  \inferrule* [lab=structure] {} {\meaningof{0} = \{ P \in \pi | P \equiv 0 \}, \and \\ \meaningof{E_1 | E_2} = \{ P \in \pi | P \equiv P_{1} | P_{2}, P_{1} \in \meaningof{E_{1}}, P_{2} \in \meaningof{E_2}\} }
\end{mathpar}

\begin{mathpar}
 \inferrule* [lab=behavior] {} {\meaningof{\langle a?b \rangle E} = \{ P \in \pi | P \equiv Q | u?(y)P', \\ \and \\\\ \and \\ \;\;\; u \in \meaningof{a}, \forall z.P'\{z/y\} \in \meaningof{E\{z/b\}}\}, \and \\ \meaningof{a!E} = \{ P \in \pi | P \equiv Q | x!\langle P' \rangle, x \in \meaningof{a} P' \in \meaningof{E}\} }
\end{mathpar}

\begin{mathpar}
 \inferrule* [lab=nominal] {} {\meaningof{\quotep{E}} = \{ \quotep{P} \in \quotep{\pi} | P \in \meaningof{E} \}, \and \meaningof{\quotep{P}} = \{ \quotep{Q} \in \quotep{\pi} | P \equiv Q \} \and \\ \meaningof{@\quotep{E}} = \{ P \in \pi | P \equiv @x, x \in \meaningof{E} \}}
\end{mathpar}

\begin{eqnarray*}
  \\
  \meaningof{-} : TS \to ST
\end{eqnarray*}

\begin{eqnarray*}
  \\
  L : TS \to ST
\end{eqnarray*}

\begin{eqnarray*}
  \\
  P \models E \iff P \in \meaningof{E}
\end{eqnarray*}

\begin{eqnarray*}
  P \approx_{L} Q \iff \forall E \in L. P \models E \iff Q \models E
\end{eqnarray*}

\begin{eqnarray*}
  P \approx_{K} Q
\end{eqnarray*}

\begin{eqnarray*}
  P \approx Q
\end{eqnarray*}

$\approx_{K} = \approx = \approx_{L}$

\subsubsection{Contextual duality}

Note that contexts extend the quotation operation to a family of
operations from processes to names. Given a context, $M$, we can
define a \emph{nominal context}, $\quotep{M}$ by $\quotep{M}[P] :=
\quotep{M[P]}$. To foreshadow what is to come we observe that these
operations enjoy a duality with processes very much like the duality
between vectors and maps from vectors to scalars.

Further, because the calculus is essentially higher-order, we have a
correspondence between contexts and processes. More specifically,
given a name $x$ and a context $M$ we can construct $M^{*}_{x}$ such
that 

\begin{mathpar}
  M^{*}_{x} | \lift{x}{P} \red M[P]
\end{mathpar}

namely,

\begin{mathpar}
  M^{*}_{x} := x?(u).M[\dropn{u}]
\end{mathpar}

The dependence of $M^{*}_{x}$ on a name makes it an abstraction, 

\begin{mathpar}
  M^{*} := (x)x?(u).M[\dropn{u}]
\end{mathpar}

\subsection{Additional notation}

It will sometimes be convenient to denote the process a name
quotes. We already have the notation $x = \quotep{P}$, but it will be
convenient to introduce an alternate notation, $\procn{x}$, when we
want to emphasize the connection to the use of the name. Note that, by
virtue of name equivalence, $\quotep{\procn{x}} \nameeq x$; so, the
notation is consistent with previous definitions.

Further, because names have structure it is possible to effect
substitutions on the basis of that structure. This means we need to
upgrade our notation for substitutions, which we accomplish by
adapting comprehension notation. Thus,

\begin{mathpar}
  P\{ y / x : x \in S \}
\end{mathpar}

is interpreted to mean the process derived from P by replacing (in a
capture-avoiding manner) each occurrence of $x$ in $S$ by $y$. For example,

\begin{mathpar}
  P\{ \quotep{\procn{x}|\procn{x}} / x : x \in \freenames{P} \}
\end{mathpar}

will replace each (occurrence) of a free name $x$ in $P$ by
$\quotep{\procn{x}|\procn{x}}$.

Also, we will avail ourselves of the notation $x^{L}$ and $x^{R}$ to
denote injections of a name into disjoint copies of the name
space. There are numerous ways to accomplish this. One example can be
found in \cite{MeredithR05}. This notation overloads to vectors of
names: $\vec{x}^{\pi} := (x_{i}^{\pi} \; : \; 0 \leq i < |\vec{x}| )$ where $\pi \in \{L,R\}$.

We also use $P^{\Box} := P|\Box$.

In \cite{MeredithR05} an interpretation of the new operator is
given. It turns out that there are several possible interpretations
all enjoying the requisite algebraic properties of the operator (see
\cite{milner91polyadicpi}). We will therefore make liberal use of
$(\nu\; \vec{x})P$.

% subsection the_syntax_and_semantics_of_the_notation_system (end)   

\input{qm2pi.qmops} 

\input{qm2pi.sterngerlach} 

\input{qm2pi.metric} 

% section concurrent_process_calculi (end)

%\input{qm2pi.proofsketch}

% section proof sketch (end)

%\input{qm2pi.slviaknots} 

% section spatial logic via knots (end)

\input{qm2pi.conclusion}

% section conclusion (end)

%\input{qm2pi.dtcodes} 

% section wiring algorithm (end)

\input{qm2pi.ack} 

% section acknowledgments (end)

\newpage


\bibliographystyle{plain}   
\bibliography{../../biblios/main.bib}

\input{qm2pi.rhodetails}

\end{document}

 

\documentclass[12pt]{llncs}
%\documentclass{jktr}

\usepackage[pdftex]{hyperref}                   
\usepackage {listings}
\usepackage {mathpartir}
\usepackage{bcprules}
%\usepackage{listings}
                       
\usepackage{graphicx} 
%\usepackage[margins=2.5cm,nohead,nofoot]{geometry}
%\usepackage{geometry}
\usepackage{amsfonts}
\usepackage{amstext}
\usepackage{latexsym}
\usepackage{amssymb}
\usepackage{color}


%\include{myPreamble}
\include{qm2pi.local} 

%\ifpdf
%\usepackage[pdftex]{graphicx}
%\else
%\usepackage{graphicx}
%\fi

 % \ifpdf
%  \usepackage{pdfsync}
%  \if


%\title{Brief Article}
%\author{David F. Snyder}
%\author{L.G. Meredith}

%\address{Dept. of Math., Texas State University--San Marcos, San Marcos, TX 78666}
       
\pagestyle{empty}


\begin{document}

\lstset{language=[Objective]Caml,frame=shadowbox}

\input{qm2pi.front}

% section front matter (end)

\input{qm2pi.intro} 
 
% section introduction (end)

% \input{qm2pi.knotations} 

% section notation (end)

\input{qm2pi.process.calculi} 

% section concurrent_process_calculi_and_spatial_logics_ (end)
    
%\input{qm2pi.knots2pi} 

%\input{qm2pi.trefoil} 

%\input{qm2pi.mainthm} 

% subsection basic_interpretation (end)

%\input{qm2pi.rho.presentation} 
\subsection{The syntax and semantics of the notation system}\label{sub:the_syntax_and_semantics_of_the_notation_system} % (fold)

We now summarize a technical presentation of the calculus that
embodies our theory of dynamics. The typical presentation of such a
calculus follows the style of giving generators and relations on
them. The grammar, below, describing term constructors, freely
generates the set of processes, $\Proc$. This set is then quotiented
by a relation known as structural congruence and it is over this set
that the notion of dynamics is expressed. This presentation is
essentially that of \cite{MeredithR05} with the addition of
polyadicity and summation. For readability we have relegated some of
the technical subtleties to an appendix.

\subsubsection{Process grammar}\label{subsub:process_grammar}

\begin{mathpar}
  \inferrule* [lab=synchronization] {} {{M} \bc \pzero \;|\; x?F \;|\; x!C }
  \and
  \inferrule* [lab=abstraction] {} {{F} \bc (x)P}
  \and
  \inferrule* [lab=concretion] {} {{C} \bc \langle Q \rangle}
  \and
  \inferrule* [lab=process] {} {{P,Q} \bc M \;| \;P|Q \;|\; @{x}}
  \and
  \inferrule* [lab=name] {} {{x} \bc \quotep{P}}
\end{mathpar} 

Note that $\vec{x}$ (resp. $\vec{P}$) denotes a vector of names
(resp. processes) of length $|\vec{x}|$ (resp. $|\vec{P}|$). We adopt
the following useful abbreviations.

\begin{mathpar}
   x?(\vec{y}).P := x.(\vec{y})P \and  x\clift{\vec{P}} := x.\clift{\vec{P}}
   \and x!(y) := \lift{x}{\dropn{y}}
   \and \Pi_{i=0}^{n-1}P_i := P_0 | \ldots | P_{n-1}
\end{mathpar}

\subsubsection{Structural congruence}

\paragraph{Free and bound names and alpha-equivalence.} At the
core of structural equivalence is alpha-equivalence which identifies
process that are the same up to a change of variable. Formally, we
recognize the distinction between free and bound names. The free names
of a process, $\freenames{P}$, may be calculated recursively as
follows:

\begin{mathpar}
\freenames{\pzero} := \emptyset
  \and \\
  \freenames{x?(y).P} := \{ x \} \cup (\freenames{P} \setminus \{ y \})
  \and 
  \freenames{x!\langle P \rangle} := \{ x \} \cup \{ P \} 
  \and \\
  \freenames{P|Q} := \freenames{P} \cup \freenames{Q}
  \and \\
  \freenames{@{x}} := \{ x \}
\end{mathpar}

$\pi$
$\quotep{\pi}$

$\freenames{-} : \pi \to \mathcal{P}(\quotep{\pi})$

\begin{eqnarray*}
  \freenames{\pzero} & := & \emptyset \\
  \freenames{x?(y).P} & := & \{ x \} \cup (\freenames{P} \setminus \{ y \}) \\
  \freenames{x!\langle P \rangle} & := & \{ x \} \cup \{ P \} \\
  \freenames{P|Q} & := & \freenames{P} \cup \freenames{Q} \\
  \freenames{\dropn{x}} & := & \{ x \}
\end{eqnarray*}

The bound names of a process, $\boundnames{P}$, are those names occurring in $P$
that are not free. For example, in $x?(y).0$, the name $x$ is free, while $y$ is bound.

\begin{mathpar}
  \inferrule* [lab=monoidal-laws] {} { P|Q \equiv Q|P \and P|0 \equiv P \and P|(Q|R) \equiv (P|Q)|R }
\end{mathpar}

\begin{mathpar}
  \inferrule* [lab=alpha-equivalence] {} { (x)P \equiv (y)P\{y/x\} \and y \not\in \freenames{P} }
\end{mathpar}

\begin{definition}
Then two processes, $P,Q$, are alpha-equivalent if $P = Q\{\vec{y}/\vec{x}\}$ for
some $\vec{x} \in \boundnames{Q},\vec{y} \in \boundnames{P}$, where $Q\{\vec{y}/\vec{x}\}$
denotes the capture-avoiding substitution of $\vec{y}$ for $\vec{x}$ in $Q$.
\end{definition}

\begin{definition}
  The {\em structural congruence} \cite{SangiorgiWalker} , $\equiv$,
  between processes is the least congruence containing
  alpha-equivalence, satisfying the abelian monoid laws
  (associativity, commutativity and $\pzero$ as identity) for parallel
  composition $|$ and for summation $+$.
\end{definition}

\subsection{Name equivalence}

We take name equivalence, written $\nameeq$, to be the smallest
equivalence relation generated by the following rules.

\begin{mathpar}
\inferrule*[lab=Quote-drop]
{ }
{ \quotep{@{x}} \nameeq x }

\inferrule*[lab=Struct-equiv]
{ P \scong Q }
{ \quotep{P} \nameeq \quotep{Q} }
\end{mathpar}

The astute reader will have noticed that the mutual recursion of names
and processes imposes a mutual recursion on alpha-equivalence and
structural equivalence via name-equivalence. Fortunately, all of this
works out pleasantly and we may calculate in the natural way, free of
concern. The reader interested in the details is referred to the
appendix \ref{appendix:rho_details}.

\subsection{Substitution}

We use $\Proc$ for the set of processes, $\QProc$ for the set of
names, and $\id{\{}\vec{y} / \vec{x} \id{\}}$ to denote partial maps,
$s : \QProc \rightarrow \QProc$. A map, $s$ lifts, uniquely, to a map
on process terms, $\widehat{s} : \Proc \rightarrow \Proc$ by the
following equations.

\begin{mathpar}
  (0) \psubstp{Q}{P} := 0 \\
  (R \juxtap S) \psubstp{Q}{P}
  :=    
  (R)\psubstp{Q}{P} \juxtap (S) \psubstp{Q}{P} \\
  (x?(y).R) \psubstp{Q}{P}    
  :=    
  (x)\substp{Q}{P} (z)\concat( (R \psubstn{z}{y}) \psubstp{Q}{P} ) \\
  (\lift{x}{R}) \psubstp{Q}{P}  
  :=
  \lift{(x)\substp{Q}{P}}{ R \psubstp{Q}{P} } \\
%   (\dropn{x})  \psubstp{Q}{P}       
%   := 
%   \left\{ 
%     \begin{array}{ccc} 
%       \dropn{\quotep{Q}} & & x \nameeq \quotep{P} \\
%       \dropn{x} & & otherwise \\
%     \end{array}
%   \right. 
  (\dropn{x})  \psubstp{Q}{P}       
  := 
  \left\{ 
    \begin{array}{ccc} 
      Q & & x \nameeq \quotep{P} \\
      \dropn{x} & & otherwise \\
    \end{array}
  \right.
\end{mathpar}
 

where

\begin{eqnarray}
  (x)\id{\{} \lpquote Q \rpquote / \lpquote P \rpquote \id{\}}            = 
  \left\{ 
    \begin{array}{ccc}
      \lpquote Q \rpquote & & x \nameeq \lpquote P \rpquote \\
      x & & otherwise \\
    \end{array}
  \right. \nonumber
\end{eqnarray}

and $z$ is chosen distinct from $\quotep{P}$, $\quotep{Q}$, the free
names in $Q$, and all the names in $R$. Our $\alpha$-equivalence will
be built in the standard way from this substitution.

\begin{remark}\label{rem:no_self_referential_names}
  One consequence of these definitions is that $\forall P. \quotep{P}
  \not\in \freenames{P}$.
\end{remark}

\subsection{ Dynamic quote: an example }

Anticipating something of what's to come, consider applying the
substitution, $\widehat{\id{\{}u / z \id{\}}}$, to the following pair
of processes, $\lift{w}{y!(z)}$ and $w[ \lpquote y!(z) \rpquote ]$.

\begin{eqnarray}
	\lift{w}{y!(z)}\widehat{\id{\{}u / z \id{\}}}
		& = &
		\lift{w}{y!(u)} \nonumber\\
	w[ \lpquote y!(z) \rpquote ] \widehat{ \id{\{}u / z \id{\}} }
		& = &
		w[ \lpquote y!(z) \rpquote ] \nonumber
\end{eqnarray}

Because the body of the process between quotes is impervious to
substitution, we get radically different answers. In fact, by
examining the first process in an input context,
e.g. $x?(z).\lift{w}{y!(z)}$, we see that the process under the lift
operator may be shaped by prefixed inputs binding a name inside it. In
this sense, the lift operator will be seen as a way to dynamically
construct processes before reifying them as names.

Finally equipped with these standard features we can present the
dynamics of the calculus.

\subsubsection{Operational semantics} 

Finally, we introduce the computational dynamics. What marks these
algebras as distinct from other more traditionally studied algebraic
structures, e.g. vector spaces or polynomial rings, is the manner in
which dynamics is captured. In traditional structures, dynamics is typically
expressed through morphisms between such structures, as in linear maps
between vector spaces or morphisms between rings. In algebras
associated with the semantics of computation, the dynamics is
expressed as part of the algebraic structure itself, through a
reduction reduction relation typically denoted by $\red$. Below, we
give a recursive presentation of this relation for the calculus used
in the encoding.

$\red \subseteq \pi \times \pi$
$\red : \pi \to \mathcal{P}(\pi)$

\begin{mathpar}
  \inferrule* [lab=Comm] { \textsf{match}( x_{src}, x_{trgt} ) } { x_{trgt}?(y)P \; | \; x_{src}!\langle {Q} \rangle \red P\{\quotep{Q}/y}\} }
  \and \\
  \inferrule* [lab=Par] {{P} \red {P}'} {{{P} | {Q}} \red {{P}' | {Q}}}
  \and
  \inferrule* [lab=Equiv]{{{P} \scong {P}'} \andalso {{P}' \red {Q}'} \andalso {{Q}' \scong {Q}}}{{P} \red {Q}}
\end{mathpar}

\begin{eqnarray*}
  match_{\equiv} (\quotep{P},\quotep{Q}) & := & P \equiv Q \\
  match_{\dagger}(\quotep{P},\quotep{Q}) & := & \forall R. P|Q \red^{*} R => R \red^{*} 0 \\
  match_{K}(\quotep{P},\quotep{Q}) & := & K \mbox{ for some context } K
\end{eqnarray*}

$u?(x)P | u!\langle Q \rangle \red P\{\quotep{Q}/x\}$

%We write $\wred$ for $\red^*$, and $P\red$ if $\exists Q $ such that $ P \red Q$.
We write $P\red$ if $\exists Q $ such that $ P \red Q$ and $P\not\red$, otherwise.

\section{Replication}

As mentioned before, it is known that replication (and hence
recursion) can be implemented in a higher-order process algebra
\cite{SangiorgiWalker}. As our first example of calculation with the
machinery thus far presented we give the construction explicitly in
the {\rhoc}.

\begin{eqnarray}
	D_{x} & := & \prefix{x}{y}{(\binpar{\outputp{x}{y}}{@{y}})} \nonumber\\
	\bangp_{x}{P} & := & \binpar{{x}!\langle{\binpar{D_{x}}{P}}\rangle}{D_{x}} \nonumber
\end{eqnarray}

\begin{eqnarray}
	\bangp_{x}{P} & & \nonumber\\
	=
	& {x}!\langle{(\prefix{x}{y}{(\outputp{x}{y} | @{y})) | P}}\rangle 
	      | \prefix{x}{y}{(\outputp{x}{y} | @{y})} & \nonumber\\
	\red
	& (\outputp{x}{y} | @{y})\substn{\quotep{(\prefix{x}{y}{(@{y} | \outputp{x}{y})) | P}}}{y} & \nonumber\\
	=
	& \outputp{x}{\quotep{(\prefix{x}{y}{(\outputp{x}{y} | @{y})) | P}}}
	  | {(\prefix{x}{y}{(\outputp{x}{y} | @{y})) | P}} & \nonumber\\
	\red
	& \ldots & \nonumber\\
	\red^*
	& P | P | \ldots & \nonumber
\end{eqnarray}

Of course, this encoding, as an implementation, runs away, unfolding
$\bangp{P}$ eagerly. A lazier and more implementable replication
operator, restricted to input-guarded processes, may be obtained as follows.

\begin{eqnarray}
\bangp{\prefix{u}{v}{P}} 
	:= 
	\binpar{\lift{x}{\prefix{u}{v}{(\binpar{D(x)}{P})}}}{D(x)} \nonumber
\end{eqnarray}

\begin{remark}
  Note that the lazier definition still does not deal with summation
  or mixed summation (i.e. sums over input and output). The reader is
  invited to construct definitions of replication that deal with these
  features. 

  Further, the definitions are parameterized in a name, $x$. Can you,
  gentle reader, make a definition that eliminates this parameter and
  guarantees no accidental interaction between the replication
  machinery and the process being replicated -- i.e. no accidental
  sharing of names used by the process to get its work done and the
  name(s) used by the replication to effect copying. This latter
  revision of the definition of replication is crucial to obtaining
  the expected identity $!!P \sim !P$.
\end{remark}

\begin{remark}\label{rem:paradoxical_combinator}
  The reader familiar with the lambda calculus will have noticed the
  similarity between $D$ and the paradoxical combinator.

  [Ed. note: the existence of this seems to suggest we have to be more
  restrictive on the set of processes and names we admit if we are to
  support no-cloning.]
\end{remark}

\subsubsection{Bisimulation}

The computational dynamics gives rise to another kind of equivalence,
the equivalence of computational behavior. As previously mentioned
this is typically captured \emph{via} some form of bisimulation.

% The notion we use in this paper is weak barbed bisimulation
% \cite{milner91polyadicpi}.

The notion we use in this paper is derived from weak barbed
bisimulation \cite{milner91polyadicpi}. 

\begin{definition}
An \emph{observation relation}, $\downarrow_{\mathcal N}$, over a set
of names, $\mathcal N$, is the smallest relation satisfying the rules
below.

\infrule[Out-barb]{y \in {\mathcal N}, \; x \nameeq y}
		  {\outputp{x}{v} \downarrow_{\mathcal N} x}
\infrule[Par-barb]{\mbox{$P\downarrow_{\mathcal N} x$ or $Q\downarrow_{\mathcal N} x$}}
		  {\binpar{P}{Q} \downarrow_{\mathcal N} x}

We write $P \Downarrow_{\mathcal N} x$ if there is $Q$ such that 
$P \wred Q$ and $Q \downarrow_{\mathcal N} x$.
\end{definition}

\begin{definition}
%\label{def.bbisim}
An  ${\mathcal N}$-\emph{barbed bisimulation} over a set of names, ${\mathcal N}$, is a symmetric binary relation 
${\mathcal S}_{\mathcal N}$ between agents such that $P\rel{S}_{\mathcal N}Q$ implies:
\begin{enumerate}
\item If $P \red P'$ then $Q \wred Q'$ and $P'\rel{S}_{\mathcal N} Q'$.
\item If $P\downarrow_{\mathcal N} x$, then $Q\Downarrow_{\mathcal N} x$.
\end{enumerate}
$P$ is ${\mathcal N}$-barbed bisimilar to $Q$, written
$P \wbbisim_{\mathcal N} Q$, if $P \rel{S}_{\mathcal N} Q$ for some ${\mathcal N}$-barbed bisimulation ${\mathcal S}_{\mathcal N}$.
\end{definition}

$\mathcal{R} \subseteq \pi \times \pi$

$P \mathcal{R} Q => \forall P'. P \red P' \Rightarrow \exists Q'. Q \red Q', P' \mathcal{R} Q'$

$P \vdash x \Rightarrow Q \vdash x$

\begin{mathpar}
  \inferrule*[lab=Out-barb]{x \nameeq y}{{y}!\langle{Q}\rangle \vdash x}
  \and
  \inferrule*[lab=Par-barb]{\mbox{$P\vdash x$ or $Q\vdash x$}}{\binpar{P}{Q} \vdash x}
\end{mathpar}

\subsubsection{Contexts}

One of the principle advantages of computational calculi like the
$\pi$-calculus is a well-defined notion of context,
contextual-equivalence and a correlation between
contextual-equivalence and notions of bisimulation. The notion of
context allows the decomposition of a process into (sub-)process and
its syntactic environment, its context. Thus, a context may be
thought of as a process with a ``hole'' (written $\Box$) in it. The
application of a context $M$ to a process $P$, written $M[P]$, is
tantamount to filling the hole in $M$ with $P$. In this paper we do
not need the full weight of this theory, but do make use of the notion
of context in the proof the main theorem. 

\begin{mathpar}
  \inferrule* [lab=summation] {} {{M_{M},M_{N}} \bc \Box \;|\; x.M_{A} \;|\; M_{M}+M_{N}}
  \and
  \inferrule* [lab=agent] {} {{M_{A}} \bc (\vec{x})M_{P} \;| \; \clift{P_0,\ldots,M_{P},\ldots,P_N}}
  \and \\
  \inferrule* [lab=process] {} {{M_{P}} \bc M_{N} \;| \;P|M_{P} }
\end{mathpar} 

\begin{mathpar}
  \inferrule* [lab=sychronization] {} {M_{N} \bc \Box \;|\; x?M_{F} \;|\; x!M_{C}}
  \and
  \inferrule* [lab=abstraction] {} {{M_{F}} \bc (x)M_{P} }
  \and
  \inferrule* [lab=concretion] {} {{M_{C}} \bc \langle M_{P} \rangle }
  \and \\
  \inferrule* [lab=process] {} {{M_{P}} \bc M_{N} \;| \;P|M_{P} }
\end{mathpar}

\begin{definition}[contextual application] Given a context $M$, and
  process $P$, we define the \emph{contextual application}, $M[P] :=
  M\{P/\Box\}$. That is, the contextual application of M to P is the
  substitution of $P$ for $\Box$ in $M$.
\end{definition}

$\meaningof{-} : L \to \mathcal{P}(\pi)$

\begin{mathpar}
  \inferrule* [lab=collection] {} {\meaningof{true} = \pi, \and \meaningof{~E} = \pi \setminus \meaningof{E}, \and \meaningof{E_{1} \& E_{2}} = \meaningof{E_{1}} \cap \meaningof{E_{2}}}
\end{mathpar}

\begin{mathpar}
  \inferrule* [lab=structure] {} {\meaningof{0} = \{ P \in \pi | P \equiv 0 \}, \and \\ \meaningof{E_1 | E_2} = \{ P \in \pi | P \equiv P_{1} | P_{2}, P_{1} \in \meaningof{E_{1}}, P_{2} \in \meaningof{E_2}\} }
\end{mathpar}

\begin{mathpar}
 \inferrule* [lab=behavior] {} {\meaningof{\langle a?b \rangle E} = \{ P \in \pi | P \equiv Q | u?(y)P', \\ \and \\\\ \and \\ \;\;\; u \in \meaningof{a}, \forall z.P'\{z/y\} \in \meaningof{E\{z/b\}}\}, \and \\ \meaningof{a!E} = \{ P \in \pi | P \equiv Q | x!\langle P' \rangle, x \in \meaningof{a} P' \in \meaningof{E}\} }
\end{mathpar}

\begin{mathpar}
 \inferrule* [lab=nominal] {} {\meaningof{\quotep{E}} = \{ \quotep{P} \in \quotep{\pi} | P \in \meaningof{E} \}, \and \meaningof{\quotep{P}} = \{ \quotep{Q} \in \quotep{\pi} | P \equiv Q \} \and \\ \meaningof{@\quotep{E}} = \{ P \in \pi | P \equiv @x, x \in \meaningof{E} \}}
\end{mathpar}

\begin{eqnarray*}
  \\
  \meaningof{-} : TS \to ST
\end{eqnarray*}

\begin{eqnarray*}
  \\
  L : TS \to ST
\end{eqnarray*}

\begin{eqnarray*}
  \\
  P \models E \iff P \in \meaningof{E}
\end{eqnarray*}

\begin{eqnarray*}
  P \approx_{L} Q \iff \forall E \in L. P \models E \iff Q \models E
\end{eqnarray*}

\begin{eqnarray*}
  P \approx_{K} Q
\end{eqnarray*}

\begin{eqnarray*}
  P \approx Q
\end{eqnarray*}

$\approx_{K} = \approx = \approx_{L}$

\subsubsection{Contextual duality}

Note that contexts extend the quotation operation to a family of
operations from processes to names. Given a context, $M$, we can
define a \emph{nominal context}, $\quotep{M}$ by $\quotep{M}[P] :=
\quotep{M[P]}$. To foreshadow what is to come we observe that these
operations enjoy a duality with processes very much like the duality
between vectors and maps from vectors to scalars.

Further, because the calculus is essentially higher-order, we have a
correspondence between contexts and processes. More specifically,
given a name $x$ and a context $M$ we can construct $M^{*}_{x}$ such
that 

\begin{mathpar}
  M^{*}_{x} | \lift{x}{P} \red M[P]
\end{mathpar}

namely,

\begin{mathpar}
  M^{*}_{x} := x?(u).M[\dropn{u}]
\end{mathpar}

The dependence of $M^{*}_{x}$ on a name makes it an abstraction, 

\begin{mathpar}
  M^{*} := (x)x?(u).M[\dropn{u}]
\end{mathpar}

\subsection{Additional notation}

It will sometimes be convenient to denote the process a name
quotes. We already have the notation $x = \quotep{P}$, but it will be
convenient to introduce an alternate notation, $\procn{x}$, when we
want to emphasize the connection to the use of the name. Note that, by
virtue of name equivalence, $\quotep{\procn{x}} \nameeq x$; so, the
notation is consistent with previous definitions.

Further, because names have structure it is possible to effect
substitutions on the basis of that structure. This means we need to
upgrade our notation for substitutions, which we accomplish by
adapting comprehension notation. Thus,

\begin{mathpar}
  P\{ y / x : x \in S \}
\end{mathpar}

is interpreted to mean the process derived from P by replacing (in a
capture-avoiding manner) each occurrence of $x$ in $S$ by $y$. For example,

\begin{mathpar}
  P\{ \quotep{\procn{x}|\procn{x}} / x : x \in \freenames{P} \}
\end{mathpar}

will replace each (occurrence) of a free name $x$ in $P$ by
$\quotep{\procn{x}|\procn{x}}$.

Also, we will avail ourselves of the notation $x^{L}$ and $x^{R}$ to
denote injections of a name into disjoint copies of the name
space. There are numerous ways to accomplish this. One example can be
found in \cite{MeredithR05}. This notation overloads to vectors of
names: $\vec{x}^{\pi} := (x_{i}^{\pi} \; : \; 0 \leq i < |\vec{x}| )$ where $\pi \in \{L,R\}$.

We also use $P^{\Box} := P|\Box$.

In \cite{MeredithR05} an interpretation of the new operator is
given. It turns out that there are several possible interpretations
all enjoying the requisite algebraic properties of the operator (see
\cite{milner91polyadicpi}). We will therefore make liberal use of
$(\nu\; \vec{x})P$.

% subsection the_syntax_and_semantics_of_the_notation_system (end)   

\input{qm2pi.qmops} 

\input{qm2pi.sterngerlach} 

\input{qm2pi.metric} 

% section concurrent_process_calculi (end)

%\input{qm2pi.proofsketch}

% section proof sketch (end)

%\input{qm2pi.slviaknots} 

% section spatial logic via knots (end)

\input{qm2pi.conclusion}

% section conclusion (end)

%\input{qm2pi.dtcodes} 

% section wiring algorithm (end)

\input{qm2pi.ack} 

% section acknowledgments (end)

\newpage


\bibliographystyle{plain}   
\bibliography{../../biblios/main.bib}

\input{qm2pi.rhodetails}

\end{document}

 

% section concurrent_process_calculi (end)

%\documentclass[12pt]{llncs}
%\documentclass{jktr}

\usepackage[pdftex]{hyperref}                   
\usepackage {listings}
\usepackage {mathpartir}
\usepackage{bcprules}
%\usepackage{listings}
                       
\usepackage{graphicx} 
%\usepackage[margins=2.5cm,nohead,nofoot]{geometry}
%\usepackage{geometry}
\usepackage{amsfonts}
\usepackage{amstext}
\usepackage{latexsym}
\usepackage{amssymb}
\usepackage{color}


%\include{myPreamble}
\include{qm2pi.local} 

%\ifpdf
%\usepackage[pdftex]{graphicx}
%\else
%\usepackage{graphicx}
%\fi

 % \ifpdf
%  \usepackage{pdfsync}
%  \if


%\title{Brief Article}
%\author{David F. Snyder}
%\author{L.G. Meredith}

%\address{Dept. of Math., Texas State University--San Marcos, San Marcos, TX 78666}
       
\pagestyle{empty}


\begin{document}

\lstset{language=[Objective]Caml,frame=shadowbox}

\input{qm2pi.front}

% section front matter (end)

\input{qm2pi.intro} 
 
% section introduction (end)

% \input{qm2pi.knotations} 

% section notation (end)

\input{qm2pi.process.calculi} 

% section concurrent_process_calculi_and_spatial_logics_ (end)
    
%\input{qm2pi.knots2pi} 

%\input{qm2pi.trefoil} 

%\input{qm2pi.mainthm} 

% subsection basic_interpretation (end)

%\input{qm2pi.rho.presentation} 
\subsection{The syntax and semantics of the notation system}\label{sub:the_syntax_and_semantics_of_the_notation_system} % (fold)

We now summarize a technical presentation of the calculus that
embodies our theory of dynamics. The typical presentation of such a
calculus follows the style of giving generators and relations on
them. The grammar, below, describing term constructors, freely
generates the set of processes, $\Proc$. This set is then quotiented
by a relation known as structural congruence and it is over this set
that the notion of dynamics is expressed. This presentation is
essentially that of \cite{MeredithR05} with the addition of
polyadicity and summation. For readability we have relegated some of
the technical subtleties to an appendix.

\subsubsection{Process grammar}\label{subsub:process_grammar}

\begin{mathpar}
  \inferrule* [lab=synchronization] {} {{M} \bc \pzero \;|\; x?F \;|\; x!C }
  \and
  \inferrule* [lab=abstraction] {} {{F} \bc (x)P}
  \and
  \inferrule* [lab=concretion] {} {{C} \bc \langle Q \rangle}
  \and
  \inferrule* [lab=process] {} {{P,Q} \bc M \;| \;P|Q \;|\; @{x}}
  \and
  \inferrule* [lab=name] {} {{x} \bc \quotep{P}}
\end{mathpar} 

Note that $\vec{x}$ (resp. $\vec{P}$) denotes a vector of names
(resp. processes) of length $|\vec{x}|$ (resp. $|\vec{P}|$). We adopt
the following useful abbreviations.

\begin{mathpar}
   x?(\vec{y}).P := x.(\vec{y})P \and  x\clift{\vec{P}} := x.\clift{\vec{P}}
   \and x!(y) := \lift{x}{\dropn{y}}
   \and \Pi_{i=0}^{n-1}P_i := P_0 | \ldots | P_{n-1}
\end{mathpar}

\subsubsection{Structural congruence}

\paragraph{Free and bound names and alpha-equivalence.} At the
core of structural equivalence is alpha-equivalence which identifies
process that are the same up to a change of variable. Formally, we
recognize the distinction between free and bound names. The free names
of a process, $\freenames{P}$, may be calculated recursively as
follows:

\begin{mathpar}
\freenames{\pzero} := \emptyset
  \and \\
  \freenames{x?(y).P} := \{ x \} \cup (\freenames{P} \setminus \{ y \})
  \and 
  \freenames{x!\langle P \rangle} := \{ x \} \cup \{ P \} 
  \and \\
  \freenames{P|Q} := \freenames{P} \cup \freenames{Q}
  \and \\
  \freenames{@{x}} := \{ x \}
\end{mathpar}

$\pi$
$\quotep{\pi}$

$\freenames{-} : \pi \to \mathcal{P}(\quotep{\pi})$

\begin{eqnarray*}
  \freenames{\pzero} & := & \emptyset \\
  \freenames{x?(y).P} & := & \{ x \} \cup (\freenames{P} \setminus \{ y \}) \\
  \freenames{x!\langle P \rangle} & := & \{ x \} \cup \{ P \} \\
  \freenames{P|Q} & := & \freenames{P} \cup \freenames{Q} \\
  \freenames{\dropn{x}} & := & \{ x \}
\end{eqnarray*}

The bound names of a process, $\boundnames{P}$, are those names occurring in $P$
that are not free. For example, in $x?(y).0$, the name $x$ is free, while $y$ is bound.

\begin{mathpar}
  \inferrule* [lab=monoidal-laws] {} { P|Q \equiv Q|P \and P|0 \equiv P \and P|(Q|R) \equiv (P|Q)|R }
\end{mathpar}

\begin{mathpar}
  \inferrule* [lab=alpha-equivalence] {} { (x)P \equiv (y)P\{y/x\} \and y \not\in \freenames{P} }
\end{mathpar}

\begin{definition}
Then two processes, $P,Q$, are alpha-equivalent if $P = Q\{\vec{y}/\vec{x}\}$ for
some $\vec{x} \in \boundnames{Q},\vec{y} \in \boundnames{P}$, where $Q\{\vec{y}/\vec{x}\}$
denotes the capture-avoiding substitution of $\vec{y}$ for $\vec{x}$ in $Q$.
\end{definition}

\begin{definition}
  The {\em structural congruence} \cite{SangiorgiWalker} , $\equiv$,
  between processes is the least congruence containing
  alpha-equivalence, satisfying the abelian monoid laws
  (associativity, commutativity and $\pzero$ as identity) for parallel
  composition $|$ and for summation $+$.
\end{definition}

\subsection{Name equivalence}

We take name equivalence, written $\nameeq$, to be the smallest
equivalence relation generated by the following rules.

\begin{mathpar}
\inferrule*[lab=Quote-drop]
{ }
{ \quotep{@{x}} \nameeq x }

\inferrule*[lab=Struct-equiv]
{ P \scong Q }
{ \quotep{P} \nameeq \quotep{Q} }
\end{mathpar}

The astute reader will have noticed that the mutual recursion of names
and processes imposes a mutual recursion on alpha-equivalence and
structural equivalence via name-equivalence. Fortunately, all of this
works out pleasantly and we may calculate in the natural way, free of
concern. The reader interested in the details is referred to the
appendix \ref{appendix:rho_details}.

\subsection{Substitution}

We use $\Proc$ for the set of processes, $\QProc$ for the set of
names, and $\id{\{}\vec{y} / \vec{x} \id{\}}$ to denote partial maps,
$s : \QProc \rightarrow \QProc$. A map, $s$ lifts, uniquely, to a map
on process terms, $\widehat{s} : \Proc \rightarrow \Proc$ by the
following equations.

\begin{mathpar}
  (0) \psubstp{Q}{P} := 0 \\
  (R \juxtap S) \psubstp{Q}{P}
  :=    
  (R)\psubstp{Q}{P} \juxtap (S) \psubstp{Q}{P} \\
  (x?(y).R) \psubstp{Q}{P}    
  :=    
  (x)\substp{Q}{P} (z)\concat( (R \psubstn{z}{y}) \psubstp{Q}{P} ) \\
  (\lift{x}{R}) \psubstp{Q}{P}  
  :=
  \lift{(x)\substp{Q}{P}}{ R \psubstp{Q}{P} } \\
%   (\dropn{x})  \psubstp{Q}{P}       
%   := 
%   \left\{ 
%     \begin{array}{ccc} 
%       \dropn{\quotep{Q}} & & x \nameeq \quotep{P} \\
%       \dropn{x} & & otherwise \\
%     \end{array}
%   \right. 
  (\dropn{x})  \psubstp{Q}{P}       
  := 
  \left\{ 
    \begin{array}{ccc} 
      Q & & x \nameeq \quotep{P} \\
      \dropn{x} & & otherwise \\
    \end{array}
  \right.
\end{mathpar}
 

where

\begin{eqnarray}
  (x)\id{\{} \lpquote Q \rpquote / \lpquote P \rpquote \id{\}}            = 
  \left\{ 
    \begin{array}{ccc}
      \lpquote Q \rpquote & & x \nameeq \lpquote P \rpquote \\
      x & & otherwise \\
    \end{array}
  \right. \nonumber
\end{eqnarray}

and $z$ is chosen distinct from $\quotep{P}$, $\quotep{Q}$, the free
names in $Q$, and all the names in $R$. Our $\alpha$-equivalence will
be built in the standard way from this substitution.

\begin{remark}\label{rem:no_self_referential_names}
  One consequence of these definitions is that $\forall P. \quotep{P}
  \not\in \freenames{P}$.
\end{remark}

\subsection{ Dynamic quote: an example }

Anticipating something of what's to come, consider applying the
substitution, $\widehat{\id{\{}u / z \id{\}}}$, to the following pair
of processes, $\lift{w}{y!(z)}$ and $w[ \lpquote y!(z) \rpquote ]$.

\begin{eqnarray}
	\lift{w}{y!(z)}\widehat{\id{\{}u / z \id{\}}}
		& = &
		\lift{w}{y!(u)} \nonumber\\
	w[ \lpquote y!(z) \rpquote ] \widehat{ \id{\{}u / z \id{\}} }
		& = &
		w[ \lpquote y!(z) \rpquote ] \nonumber
\end{eqnarray}

Because the body of the process between quotes is impervious to
substitution, we get radically different answers. In fact, by
examining the first process in an input context,
e.g. $x?(z).\lift{w}{y!(z)}$, we see that the process under the lift
operator may be shaped by prefixed inputs binding a name inside it. In
this sense, the lift operator will be seen as a way to dynamically
construct processes before reifying them as names.

Finally equipped with these standard features we can present the
dynamics of the calculus.

\subsubsection{Operational semantics} 

Finally, we introduce the computational dynamics. What marks these
algebras as distinct from other more traditionally studied algebraic
structures, e.g. vector spaces or polynomial rings, is the manner in
which dynamics is captured. In traditional structures, dynamics is typically
expressed through morphisms between such structures, as in linear maps
between vector spaces or morphisms between rings. In algebras
associated with the semantics of computation, the dynamics is
expressed as part of the algebraic structure itself, through a
reduction reduction relation typically denoted by $\red$. Below, we
give a recursive presentation of this relation for the calculus used
in the encoding.

$\red \subseteq \pi \times \pi$
$\red : \pi \to \mathcal{P}(\pi)$

\begin{mathpar}
  \inferrule* [lab=Comm] { \textsf{match}( x_{src}, x_{trgt} ) } { x_{trgt}?(y)P \; | \; x_{src}!\langle {Q} \rangle \red P\{\quotep{Q}/y}\} }
  \and \\
  \inferrule* [lab=Par] {{P} \red {P}'} {{{P} | {Q}} \red {{P}' | {Q}}}
  \and
  \inferrule* [lab=Equiv]{{{P} \scong {P}'} \andalso {{P}' \red {Q}'} \andalso {{Q}' \scong {Q}}}{{P} \red {Q}}
\end{mathpar}

\begin{eqnarray*}
  match_{\equiv} (\quotep{P},\quotep{Q}) & := & P \equiv Q \\
  match_{\dagger}(\quotep{P},\quotep{Q}) & := & \forall R. P|Q \red^{*} R => R \red^{*} 0 \\
  match_{K}(\quotep{P},\quotep{Q}) & := & K \mbox{ for some context } K
\end{eqnarray*}

$u?(x)P | u!\langle Q \rangle \red P\{\quotep{Q}/x\}$

%We write $\wred$ for $\red^*$, and $P\red$ if $\exists Q $ such that $ P \red Q$.
We write $P\red$ if $\exists Q $ such that $ P \red Q$ and $P\not\red$, otherwise.

\section{Replication}

As mentioned before, it is known that replication (and hence
recursion) can be implemented in a higher-order process algebra
\cite{SangiorgiWalker}. As our first example of calculation with the
machinery thus far presented we give the construction explicitly in
the {\rhoc}.

\begin{eqnarray}
	D_{x} & := & \prefix{x}{y}{(\binpar{\outputp{x}{y}}{@{y}})} \nonumber\\
	\bangp_{x}{P} & := & \binpar{{x}!\langle{\binpar{D_{x}}{P}}\rangle}{D_{x}} \nonumber
\end{eqnarray}

\begin{eqnarray}
	\bangp_{x}{P} & & \nonumber\\
	=
	& {x}!\langle{(\prefix{x}{y}{(\outputp{x}{y} | @{y})) | P}}\rangle 
	      | \prefix{x}{y}{(\outputp{x}{y} | @{y})} & \nonumber\\
	\red
	& (\outputp{x}{y} | @{y})\substn{\quotep{(\prefix{x}{y}{(@{y} | \outputp{x}{y})) | P}}}{y} & \nonumber\\
	=
	& \outputp{x}{\quotep{(\prefix{x}{y}{(\outputp{x}{y} | @{y})) | P}}}
	  | {(\prefix{x}{y}{(\outputp{x}{y} | @{y})) | P}} & \nonumber\\
	\red
	& \ldots & \nonumber\\
	\red^*
	& P | P | \ldots & \nonumber
\end{eqnarray}

Of course, this encoding, as an implementation, runs away, unfolding
$\bangp{P}$ eagerly. A lazier and more implementable replication
operator, restricted to input-guarded processes, may be obtained as follows.

\begin{eqnarray}
\bangp{\prefix{u}{v}{P}} 
	:= 
	\binpar{\lift{x}{\prefix{u}{v}{(\binpar{D(x)}{P})}}}{D(x)} \nonumber
\end{eqnarray}

\begin{remark}
  Note that the lazier definition still does not deal with summation
  or mixed summation (i.e. sums over input and output). The reader is
  invited to construct definitions of replication that deal with these
  features. 

  Further, the definitions are parameterized in a name, $x$. Can you,
  gentle reader, make a definition that eliminates this parameter and
  guarantees no accidental interaction between the replication
  machinery and the process being replicated -- i.e. no accidental
  sharing of names used by the process to get its work done and the
  name(s) used by the replication to effect copying. This latter
  revision of the definition of replication is crucial to obtaining
  the expected identity $!!P \sim !P$.
\end{remark}

\begin{remark}\label{rem:paradoxical_combinator}
  The reader familiar with the lambda calculus will have noticed the
  similarity between $D$ and the paradoxical combinator.

  [Ed. note: the existence of this seems to suggest we have to be more
  restrictive on the set of processes and names we admit if we are to
  support no-cloning.]
\end{remark}

\subsubsection{Bisimulation}

The computational dynamics gives rise to another kind of equivalence,
the equivalence of computational behavior. As previously mentioned
this is typically captured \emph{via} some form of bisimulation.

% The notion we use in this paper is weak barbed bisimulation
% \cite{milner91polyadicpi}.

The notion we use in this paper is derived from weak barbed
bisimulation \cite{milner91polyadicpi}. 

\begin{definition}
An \emph{observation relation}, $\downarrow_{\mathcal N}$, over a set
of names, $\mathcal N$, is the smallest relation satisfying the rules
below.

\infrule[Out-barb]{y \in {\mathcal N}, \; x \nameeq y}
		  {\outputp{x}{v} \downarrow_{\mathcal N} x}
\infrule[Par-barb]{\mbox{$P\downarrow_{\mathcal N} x$ or $Q\downarrow_{\mathcal N} x$}}
		  {\binpar{P}{Q} \downarrow_{\mathcal N} x}

We write $P \Downarrow_{\mathcal N} x$ if there is $Q$ such that 
$P \wred Q$ and $Q \downarrow_{\mathcal N} x$.
\end{definition}

\begin{definition}
%\label{def.bbisim}
An  ${\mathcal N}$-\emph{barbed bisimulation} over a set of names, ${\mathcal N}$, is a symmetric binary relation 
${\mathcal S}_{\mathcal N}$ between agents such that $P\rel{S}_{\mathcal N}Q$ implies:
\begin{enumerate}
\item If $P \red P'$ then $Q \wred Q'$ and $P'\rel{S}_{\mathcal N} Q'$.
\item If $P\downarrow_{\mathcal N} x$, then $Q\Downarrow_{\mathcal N} x$.
\end{enumerate}
$P$ is ${\mathcal N}$-barbed bisimilar to $Q$, written
$P \wbbisim_{\mathcal N} Q$, if $P \rel{S}_{\mathcal N} Q$ for some ${\mathcal N}$-barbed bisimulation ${\mathcal S}_{\mathcal N}$.
\end{definition}

$\mathcal{R} \subseteq \pi \times \pi$

$P \mathcal{R} Q => \forall P'. P \red P' \Rightarrow \exists Q'. Q \red Q', P' \mathcal{R} Q'$

$P \vdash x \Rightarrow Q \vdash x$

\begin{mathpar}
  \inferrule*[lab=Out-barb]{x \nameeq y}{{y}!\langle{Q}\rangle \vdash x}
  \and
  \inferrule*[lab=Par-barb]{\mbox{$P\vdash x$ or $Q\vdash x$}}{\binpar{P}{Q} \vdash x}
\end{mathpar}

\subsubsection{Contexts}

One of the principle advantages of computational calculi like the
$\pi$-calculus is a well-defined notion of context,
contextual-equivalence and a correlation between
contextual-equivalence and notions of bisimulation. The notion of
context allows the decomposition of a process into (sub-)process and
its syntactic environment, its context. Thus, a context may be
thought of as a process with a ``hole'' (written $\Box$) in it. The
application of a context $M$ to a process $P$, written $M[P]$, is
tantamount to filling the hole in $M$ with $P$. In this paper we do
not need the full weight of this theory, but do make use of the notion
of context in the proof the main theorem. 

\begin{mathpar}
  \inferrule* [lab=summation] {} {{M_{M},M_{N}} \bc \Box \;|\; x.M_{A} \;|\; M_{M}+M_{N}}
  \and
  \inferrule* [lab=agent] {} {{M_{A}} \bc (\vec{x})M_{P} \;| \; \clift{P_0,\ldots,M_{P},\ldots,P_N}}
  \and \\
  \inferrule* [lab=process] {} {{M_{P}} \bc M_{N} \;| \;P|M_{P} }
\end{mathpar} 

\begin{mathpar}
  \inferrule* [lab=sychronization] {} {M_{N} \bc \Box \;|\; x?M_{F} \;|\; x!M_{C}}
  \and
  \inferrule* [lab=abstraction] {} {{M_{F}} \bc (x)M_{P} }
  \and
  \inferrule* [lab=concretion] {} {{M_{C}} \bc \langle M_{P} \rangle }
  \and \\
  \inferrule* [lab=process] {} {{M_{P}} \bc M_{N} \;| \;P|M_{P} }
\end{mathpar}

\begin{definition}[contextual application] Given a context $M$, and
  process $P$, we define the \emph{contextual application}, $M[P] :=
  M\{P/\Box\}$. That is, the contextual application of M to P is the
  substitution of $P$ for $\Box$ in $M$.
\end{definition}

$\meaningof{-} : L \to \mathcal{P}(\pi)$

\begin{mathpar}
  \inferrule* [lab=collection] {} {\meaningof{true} = \pi, \and \meaningof{~E} = \pi \setminus \meaningof{E}, \and \meaningof{E_{1} \& E_{2}} = \meaningof{E_{1}} \cap \meaningof{E_{2}}}
\end{mathpar}

\begin{mathpar}
  \inferrule* [lab=structure] {} {\meaningof{0} = \{ P \in \pi | P \equiv 0 \}, \and \\ \meaningof{E_1 | E_2} = \{ P \in \pi | P \equiv P_{1} | P_{2}, P_{1} \in \meaningof{E_{1}}, P_{2} \in \meaningof{E_2}\} }
\end{mathpar}

\begin{mathpar}
 \inferrule* [lab=behavior] {} {\meaningof{\langle a?b \rangle E} = \{ P \in \pi | P \equiv Q | u?(y)P', \\ \and \\\\ \and \\ \;\;\; u \in \meaningof{a}, \forall z.P'\{z/y\} \in \meaningof{E\{z/b\}}\}, \and \\ \meaningof{a!E} = \{ P \in \pi | P \equiv Q | x!\langle P' \rangle, x \in \meaningof{a} P' \in \meaningof{E}\} }
\end{mathpar}

\begin{mathpar}
 \inferrule* [lab=nominal] {} {\meaningof{\quotep{E}} = \{ \quotep{P} \in \quotep{\pi} | P \in \meaningof{E} \}, \and \meaningof{\quotep{P}} = \{ \quotep{Q} \in \quotep{\pi} | P \equiv Q \} \and \\ \meaningof{@\quotep{E}} = \{ P \in \pi | P \equiv @x, x \in \meaningof{E} \}}
\end{mathpar}

\begin{eqnarray*}
  \\
  \meaningof{-} : TS \to ST
\end{eqnarray*}

\begin{eqnarray*}
  \\
  L : TS \to ST
\end{eqnarray*}

\begin{eqnarray*}
  \\
  P \models E \iff P \in \meaningof{E}
\end{eqnarray*}

\begin{eqnarray*}
  P \approx_{L} Q \iff \forall E \in L. P \models E \iff Q \models E
\end{eqnarray*}

\begin{eqnarray*}
  P \approx_{K} Q
\end{eqnarray*}

\begin{eqnarray*}
  P \approx Q
\end{eqnarray*}

$\approx_{K} = \approx = \approx_{L}$

\subsubsection{Contextual duality}

Note that contexts extend the quotation operation to a family of
operations from processes to names. Given a context, $M$, we can
define a \emph{nominal context}, $\quotep{M}$ by $\quotep{M}[P] :=
\quotep{M[P]}$. To foreshadow what is to come we observe that these
operations enjoy a duality with processes very much like the duality
between vectors and maps from vectors to scalars.

Further, because the calculus is essentially higher-order, we have a
correspondence between contexts and processes. More specifically,
given a name $x$ and a context $M$ we can construct $M^{*}_{x}$ such
that 

\begin{mathpar}
  M^{*}_{x} | \lift{x}{P} \red M[P]
\end{mathpar}

namely,

\begin{mathpar}
  M^{*}_{x} := x?(u).M[\dropn{u}]
\end{mathpar}

The dependence of $M^{*}_{x}$ on a name makes it an abstraction, 

\begin{mathpar}
  M^{*} := (x)x?(u).M[\dropn{u}]
\end{mathpar}

\subsection{Additional notation}

It will sometimes be convenient to denote the process a name
quotes. We already have the notation $x = \quotep{P}$, but it will be
convenient to introduce an alternate notation, $\procn{x}$, when we
want to emphasize the connection to the use of the name. Note that, by
virtue of name equivalence, $\quotep{\procn{x}} \nameeq x$; so, the
notation is consistent with previous definitions.

Further, because names have structure it is possible to effect
substitutions on the basis of that structure. This means we need to
upgrade our notation for substitutions, which we accomplish by
adapting comprehension notation. Thus,

\begin{mathpar}
  P\{ y / x : x \in S \}
\end{mathpar}

is interpreted to mean the process derived from P by replacing (in a
capture-avoiding manner) each occurrence of $x$ in $S$ by $y$. For example,

\begin{mathpar}
  P\{ \quotep{\procn{x}|\procn{x}} / x : x \in \freenames{P} \}
\end{mathpar}

will replace each (occurrence) of a free name $x$ in $P$ by
$\quotep{\procn{x}|\procn{x}}$.

Also, we will avail ourselves of the notation $x^{L}$ and $x^{R}$ to
denote injections of a name into disjoint copies of the name
space. There are numerous ways to accomplish this. One example can be
found in \cite{MeredithR05}. This notation overloads to vectors of
names: $\vec{x}^{\pi} := (x_{i}^{\pi} \; : \; 0 \leq i < |\vec{x}| )$ where $\pi \in \{L,R\}$.

We also use $P^{\Box} := P|\Box$.

In \cite{MeredithR05} an interpretation of the new operator is
given. It turns out that there are several possible interpretations
all enjoying the requisite algebraic properties of the operator (see
\cite{milner91polyadicpi}). We will therefore make liberal use of
$(\nu\; \vec{x})P$.

% subsection the_syntax_and_semantics_of_the_notation_system (end)   

\input{qm2pi.qmops} 

\input{qm2pi.sterngerlach} 

\input{qm2pi.metric} 

% section concurrent_process_calculi (end)

%\input{qm2pi.proofsketch}

% section proof sketch (end)

%\input{qm2pi.slviaknots} 

% section spatial logic via knots (end)

\input{qm2pi.conclusion}

% section conclusion (end)

%\input{qm2pi.dtcodes} 

% section wiring algorithm (end)

\input{qm2pi.ack} 

% section acknowledgments (end)

\newpage


\bibliographystyle{plain}   
\bibliography{../../biblios/main.bib}

\input{qm2pi.rhodetails}

\end{document}



% section proof sketch (end)

%\section{Unlikely characters: spatial logic for
  knots}\label{sub:characteristic_formulae} % (fold)

Associated to the mobile process calculi are a family of logics known
as the Hennessy-Milner logics. These logics typically enjoy a
semantics interpreting formulae as sets of processes that when
factored through the encoding outlined above allows an identification
of classes of knots with logical formulae. In the context of this
encoding the sub-family known as the spatial logics \cite{CairesC03}
\cite{CairesC04} \cite{Caires04} are of particular interest providing
several important features for expressing and reasoning about
properties (i.e. classes) of knots. We hint here at how this may be done.

%\begin{description}
%\item [structural connectives] 
\subsubsection{Structural connectives} The spatial logics enjoy
structural connectives corresponding, at the logical level, to the
parallel composition ($P | Q$) and new name ($(\nu \; x)P$)
connectives for processes. As illustrated in the examples below, these
connectives are extremely expressive given the shape of our encoding.
%\item [decideable satisfaction]

\subsubsection{Decideable satisfaction}
In \cite{Caires04} the satisfaction relation is shown to be decideable
for a rich class of processes. It further turns out that the image of
the our encoding is a proper subset of that class. This result
provides the basis for an algorithm by which to search for knots
enjoying a given property.
%\item [characteristic formulae]

\subsubsection{Characteristic formulae}
In the same paper \cite{Caires04} , Caires presents a means of calculating
characteristic formulae, selecting equivalence classes of processes
up to a pre--specified depth limit on the support set of names. Composed with our
encoding, this characteristic formula can be used to select
characteristic formulae for knots.
%\end{description}

\subsubsection{Spatial logic formulae}

The grammar below (segmented for comprehension) summarizes the syntax
of spatial logic formulae. We employ illustrative examples in the
sequel to provide an intuitive understanding of their meaning
referring the reader to \cite{Caires04} for a more detailed explication
of the semantics.

\begin{mathpar}
  \inferrule* [lab=boolean] {} {{A,B} \bc T \;|\; \neg A \;|\; A \wedge B \;|\; \eta = \eta'}
  \and
  \inferrule* [lab=spatial] {} {|\; \pzero \;|\; A | B \;|\; x \text{\textregistered} A \;|\; \forall x . A \;|\;  H x . A}
  \and
  \inferrule* [lab=behavioral] {} {|\; \alpha . A}
  \and 
  \inferrule* [lab=recursion] {} {|\; X(\vec{u}) \;|\; \mu X(\vec{u}) . A}
  \and
  \inferrule* [lab=action] {} {\alpha \bc \langle x?(\vec{y}) \rangle \;|\; \langle x!(\vec{y}) \rangle \;|\; \langle \tau \rangle}
  \and 
  \inferrule* [lab=name] {} {\eta \bc x \;|\; \tau}
\end{mathpar} 

% subsection characteristic_formulae (end)   	 

\subsection{Example formulae}\label{sub:example_formulae_} % (fold)

\subsubsection{Crossing as formula.}
% 
% \begin{align*}
%   \frac{d}{dx} \sin x &= \cos x 
%   & \frac{d}{dx} e^x &= e^x \\
%   \frac{d}{dx} \cos x &= - \sin x 
%   & \frac{d}{dx} \log x &= \frac{1}{x} \\
% \end{align*} 

\begin{align*}
 \mu C(x_{0},x_{1},y_{0},y_{1},u).&(\langle x_{0}?(z) \rangle(\langle u! \rangle\langle y_{1}!z \rangle C(x_{0},x_{1},y_{0},y_{1},u)) & \\
  & \wedge \langle y_{1}?(z) \rangle (\langle u! \rangle \langle x_{0}!z \rangle C(x_{0},x_{1},y_{0},y_{1},u)) & \\
  & \wedge \langle x_{1}?(z) \rangle (\langle u? \rangle \langle y_{0}!z \rangle C(x_{0},x_{1},y_{0},y_{1},u)) & \\
  & \wedge \langle y_{0}?(z) \rangle (\langle u? \rangle \langle x_{1}!z \rangle C(x_{0},x_{1},y_{0},y_{1},u))) &
\end{align*}

The lexicographical similarity between the shape of this formulae and
the shape of definition of the process representing a crossing reveals
the intuitive meaning of this formulae. It describes the capabilities
of a process that has the right to represent a crossing. For example
it picks out processes that may perform an input on the port $x_0$ in
its initial menu of capabilities. What differentiates the formula
from the process, however, is that the crossing process is the
smallest candidate to satisfy the formula. Infinitely many other
processes -- with internal behavior hidden behind this interface, so
to speak -- also satisfy this formula. Even this simple formula,
then, can be seen to open a new view onto knots, providing a
computational interpretation of \emph{virtual} knots.

Note that this formula is derived by hand. A similar formula can be
derived by employing Caires' calculation of characteristic formula
\cite{Caires04} to the process representing a crossing. In light of
this discussion, we let
$\meaningof{C}_{\phi}(x0,x1,y0,y1,u)$ denote a formula specifying the
dynamics we wish to capture of a crossing. To guarantee we preserve
the shape of the interface and minimal semantics we demand that
$\meaningof{C}_{\phi}(x0,x1,y0,y1,u) \Rightarrow
\textbf{C}(x0,x1,y0,y1,u)$ where $\textbf{C}(x0,x1,y0,y1,u)$ denotes
the formula above.
                            
\subsubsection{Crossing number constraints.}
The moral content of the context lemma (Lemma \ref{context}) is that the notion of
``locality'' in the Reidemeister moves is effectively captured by the
parallel composition operator of the process calculus. This intuition
extends through the logic. Given a formula,
$\meaningof{C}_{\phi}(x0,x1,y0,y1,u)$, we can use the structural
connectives to specify constraints on crossing numbers, such as at
least $n$ crossings, or exactly $n$ crossings.
\begin{mathpar}
  \inferrule* [lab=at-least-n] {} { K^{\geq n}_{\phi}(\vec{xs},\vec{ys}) := \Pi_{i=0}^{n-1} Hu . \meaningof{C}_{\phi}(xs_i,ys_i,u) | T }
  \and 
  \inferrule* [lab=exactly-n] {} { K^{= n}_{\phi}(\vec{xs},\vec{ys}) := \Pi_{i=0}^{n-1} Hu . \meaningof{C}_{\phi}(xs_i,ys_i,u) | \neg (\forall x_0,y_0,x_1,y_1,u . \meaningof{C}_{\phi}(x_0,y_0,x_1,y_1,u) | T) }
\end{mathpar}

To round out this section, recall that the encoding of an $n$-crossing
knot decomposes into a parallel composition of $n$ \emph{copies} of a
crossing process together with a wiring harness. To specify different
knot classes with the same crossing number amounts to specifying
logical constraints on the wiring harness. In the interest of space,
we defer examples to a forthcoming paper. Suffice it to say that both
the conditions ``alternating knot'' and ``contains the tangle
corresponding to 5/3'' are expressible. For example, it is possible to
calculate the characteristic formula of a process corresponding to the
tangle 5/3 and conjoin it into the classifying formula via the
composition connective of the logic.

Finally, we wish to observe that it is entirely within reason to
contemplate a more domain-specific version of spatial logic tailored
to the shape of processes in the image of the encoding. Such a
domain-specific logic would have a better claim to the title formal
language of knot properties.

% subsection example_formulae_ (end)

% section knots_as_processes (end) 

% section spatial logic via knots (end)

\section{Conclusions and future work}

\paragraph{Testing physical space}
You, gentle reader, may wonder why of all the theorems to be proved
given this set up we pick the one above. In some sense it's hardly
central to quantum mechanics. We see it as central in the sense that
it firmly establishes a notion of physical space arising from a notion
of the equivalence of behavior. Relating bisimulation to a metric is a
big step forward, but one is faced with interpreting the relationship
of that metric space to something more physical. Quantum mechanical
notions of ``physical'' space are still far from intuitive, but by
relating this idea of distance as testing to calculations that predict
physical circumstances we are making a not insignificant step forward
toward an understanding of the physical space we inhabit as
essentially dynamic.

\paragraph{Effectivity and simulation}
One of the observations we have yet to make is that the entire program
spelled out here is effective. We have built various interpreters for
the reflective calculus at work in this interpretation. In principle,
then, we can simulate quantum mechanics on a computer. The place where
the simulation may lose fidelity is the infinitely branching summation
for the annihilator.

In this connection i also want to point out that the evaluation style
calculation of the inner product puts the non-determinism of the
summation right at the heart of measurement. This suggests that
Milner's original reduction-based formulation of the dynamics of his
calculi in terms of sums was not just notationally suggestive of a
notion of measure-and-continue but captured some significant part of
the physics.

\paragraph{Quantum continuations}
In light of this last observation i want to point out that the
predominant account of quantum mechanics is missing a key aspect of a
truly compositional story of the physical situation. In a real lab,
when a measurement is made the observation can be made to feed into
another device that then makes another measurement conditioned on the
results of the first. This means that after the superposition was
collapsed the entire experimental set up remained in
superposition. While QM offers a means of writing this down it doesn't
quite line up well with the well-trodden formulation of computation
and continuation that we see so succinctly expressed in Milner's
calculi. This suggests that there might be advantages to this account
of dynamics waiting to be explored.

\paragraph{Quantum logic}
In this connection, we also note that by virtue of having the
Hennessy-Milner construction, we can pull the construction through the
interpretation of QM. This gives us a natural candidate for a quantum
logic that enjoys an extremely tight connection with it's domain of
interpretation, making the construction much less ad hoc (rather it is
the image of functor!).

\paragraph{Quantum probabiity}
i have questions about the basis of the interpretation of inner
product as probability amplitude. In particular, using which
axiomatization of probability theory does the notion of probability
amplitude earn the right to be so dubbed? In other words, where is the
proof that the operation for calculating a probability amplitude (and
then squaring) satisfies the axioms of what it means to calculate a
probability? Even if such a proof exists (i have yet to find it in the
literature), i wonder if it might not be possible to turn things on
their heads. Can we view the calculation of the probability amplitude
as an axiomatization of probability? If so, then the definition we
give for calculating probability amplitude may provide the basis for
an \emph{effective} theory of probability.

\paragraph{Quantum vs ``biological'' information}
Finally, i want to conclude with a more philosophical observation. At
a recent workshop in which QM was a predominant topic i noticed
something about quantum information. The speaker was giving a riveting
discussion of axiomatic QM and showing how properties of ``no
cloning'' and ``no deleting'' emerged as consequences of the
axiomatization. Theorems of this form are necessary to give us a sense
of confidence that our axioms characterize the physical theory. What
struck me, though, was that if quantum information is neither erasable
nor replicable it is markedly different from \emph{life}. Two of the
things we know about life is that

\begin{itemize}
  \item it ends;
  \item to gain some measure of persistence, to transcend it's
    finitude it is imminently copyable.
\end{itemize}

Both of these qualities are summarized succinctly in the aphorism: all
flesh is grass. For me these two kinds of ``information'' -- call them
quantum and biological -- are end points on a spectrum of strategies
for persistence. At one end, we have those curious entities that enjoy
uniqueness and permanence; at the other, we have those who in the face
of a certain end and an uncertain present make a go of passing
something on. To me one of the more remarkable aspects of the latter
strategy is that in the presence of noise (and certain features of
copying) we get a kind of dynamism, a chance for improvement against a
given persistent condition.

% subsection other_calculi_other_bisimulations_and_geometry_as_behavior (end)




% section conclusion (end)

%\documentclass[12pt]{llncs}
%\documentclass{jktr}

\usepackage[pdftex]{hyperref}                   
\usepackage {listings}
\usepackage {mathpartir}
\usepackage{bcprules}
%\usepackage{listings}
                       
\usepackage{graphicx} 
%\usepackage[margins=2.5cm,nohead,nofoot]{geometry}
%\usepackage{geometry}
\usepackage{amsfonts}
\usepackage{amstext}
\usepackage{latexsym}
\usepackage{amssymb}
\usepackage{color}


%\include{myPreamble}
\include{qm2pi.local} 

%\ifpdf
%\usepackage[pdftex]{graphicx}
%\else
%\usepackage{graphicx}
%\fi

 % \ifpdf
%  \usepackage{pdfsync}
%  \if


%\title{Brief Article}
%\author{David F. Snyder}
%\author{L.G. Meredith}

%\address{Dept. of Math., Texas State University--San Marcos, San Marcos, TX 78666}
       
\pagestyle{empty}


\begin{document}

\lstset{language=[Objective]Caml,frame=shadowbox}

\input{qm2pi.front}

% section front matter (end)

\input{qm2pi.intro} 
 
% section introduction (end)

% \input{qm2pi.knotations} 

% section notation (end)

\input{qm2pi.process.calculi} 

% section concurrent_process_calculi_and_spatial_logics_ (end)
    
%\input{qm2pi.knots2pi} 

%\input{qm2pi.trefoil} 

%\input{qm2pi.mainthm} 

% subsection basic_interpretation (end)

%\input{qm2pi.rho.presentation} 
\subsection{The syntax and semantics of the notation system}\label{sub:the_syntax_and_semantics_of_the_notation_system} % (fold)

We now summarize a technical presentation of the calculus that
embodies our theory of dynamics. The typical presentation of such a
calculus follows the style of giving generators and relations on
them. The grammar, below, describing term constructors, freely
generates the set of processes, $\Proc$. This set is then quotiented
by a relation known as structural congruence and it is over this set
that the notion of dynamics is expressed. This presentation is
essentially that of \cite{MeredithR05} with the addition of
polyadicity and summation. For readability we have relegated some of
the technical subtleties to an appendix.

\subsubsection{Process grammar}\label{subsub:process_grammar}

\begin{mathpar}
  \inferrule* [lab=synchronization] {} {{M} \bc \pzero \;|\; x?F \;|\; x!C }
  \and
  \inferrule* [lab=abstraction] {} {{F} \bc (x)P}
  \and
  \inferrule* [lab=concretion] {} {{C} \bc \langle Q \rangle}
  \and
  \inferrule* [lab=process] {} {{P,Q} \bc M \;| \;P|Q \;|\; @{x}}
  \and
  \inferrule* [lab=name] {} {{x} \bc \quotep{P}}
\end{mathpar} 

Note that $\vec{x}$ (resp. $\vec{P}$) denotes a vector of names
(resp. processes) of length $|\vec{x}|$ (resp. $|\vec{P}|$). We adopt
the following useful abbreviations.

\begin{mathpar}
   x?(\vec{y}).P := x.(\vec{y})P \and  x\clift{\vec{P}} := x.\clift{\vec{P}}
   \and x!(y) := \lift{x}{\dropn{y}}
   \and \Pi_{i=0}^{n-1}P_i := P_0 | \ldots | P_{n-1}
\end{mathpar}

\subsubsection{Structural congruence}

\paragraph{Free and bound names and alpha-equivalence.} At the
core of structural equivalence is alpha-equivalence which identifies
process that are the same up to a change of variable. Formally, we
recognize the distinction between free and bound names. The free names
of a process, $\freenames{P}$, may be calculated recursively as
follows:

\begin{mathpar}
\freenames{\pzero} := \emptyset
  \and \\
  \freenames{x?(y).P} := \{ x \} \cup (\freenames{P} \setminus \{ y \})
  \and 
  \freenames{x!\langle P \rangle} := \{ x \} \cup \{ P \} 
  \and \\
  \freenames{P|Q} := \freenames{P} \cup \freenames{Q}
  \and \\
  \freenames{@{x}} := \{ x \}
\end{mathpar}

$\pi$
$\quotep{\pi}$

$\freenames{-} : \pi \to \mathcal{P}(\quotep{\pi})$

\begin{eqnarray*}
  \freenames{\pzero} & := & \emptyset \\
  \freenames{x?(y).P} & := & \{ x \} \cup (\freenames{P} \setminus \{ y \}) \\
  \freenames{x!\langle P \rangle} & := & \{ x \} \cup \{ P \} \\
  \freenames{P|Q} & := & \freenames{P} \cup \freenames{Q} \\
  \freenames{\dropn{x}} & := & \{ x \}
\end{eqnarray*}

The bound names of a process, $\boundnames{P}$, are those names occurring in $P$
that are not free. For example, in $x?(y).0$, the name $x$ is free, while $y$ is bound.

\begin{mathpar}
  \inferrule* [lab=monoidal-laws] {} { P|Q \equiv Q|P \and P|0 \equiv P \and P|(Q|R) \equiv (P|Q)|R }
\end{mathpar}

\begin{mathpar}
  \inferrule* [lab=alpha-equivalence] {} { (x)P \equiv (y)P\{y/x\} \and y \not\in \freenames{P} }
\end{mathpar}

\begin{definition}
Then two processes, $P,Q$, are alpha-equivalent if $P = Q\{\vec{y}/\vec{x}\}$ for
some $\vec{x} \in \boundnames{Q},\vec{y} \in \boundnames{P}$, where $Q\{\vec{y}/\vec{x}\}$
denotes the capture-avoiding substitution of $\vec{y}$ for $\vec{x}$ in $Q$.
\end{definition}

\begin{definition}
  The {\em structural congruence} \cite{SangiorgiWalker} , $\equiv$,
  between processes is the least congruence containing
  alpha-equivalence, satisfying the abelian monoid laws
  (associativity, commutativity and $\pzero$ as identity) for parallel
  composition $|$ and for summation $+$.
\end{definition}

\subsection{Name equivalence}

We take name equivalence, written $\nameeq$, to be the smallest
equivalence relation generated by the following rules.

\begin{mathpar}
\inferrule*[lab=Quote-drop]
{ }
{ \quotep{@{x}} \nameeq x }

\inferrule*[lab=Struct-equiv]
{ P \scong Q }
{ \quotep{P} \nameeq \quotep{Q} }
\end{mathpar}

The astute reader will have noticed that the mutual recursion of names
and processes imposes a mutual recursion on alpha-equivalence and
structural equivalence via name-equivalence. Fortunately, all of this
works out pleasantly and we may calculate in the natural way, free of
concern. The reader interested in the details is referred to the
appendix \ref{appendix:rho_details}.

\subsection{Substitution}

We use $\Proc$ for the set of processes, $\QProc$ for the set of
names, and $\id{\{}\vec{y} / \vec{x} \id{\}}$ to denote partial maps,
$s : \QProc \rightarrow \QProc$. A map, $s$ lifts, uniquely, to a map
on process terms, $\widehat{s} : \Proc \rightarrow \Proc$ by the
following equations.

\begin{mathpar}
  (0) \psubstp{Q}{P} := 0 \\
  (R \juxtap S) \psubstp{Q}{P}
  :=    
  (R)\psubstp{Q}{P} \juxtap (S) \psubstp{Q}{P} \\
  (x?(y).R) \psubstp{Q}{P}    
  :=    
  (x)\substp{Q}{P} (z)\concat( (R \psubstn{z}{y}) \psubstp{Q}{P} ) \\
  (\lift{x}{R}) \psubstp{Q}{P}  
  :=
  \lift{(x)\substp{Q}{P}}{ R \psubstp{Q}{P} } \\
%   (\dropn{x})  \psubstp{Q}{P}       
%   := 
%   \left\{ 
%     \begin{array}{ccc} 
%       \dropn{\quotep{Q}} & & x \nameeq \quotep{P} \\
%       \dropn{x} & & otherwise \\
%     \end{array}
%   \right. 
  (\dropn{x})  \psubstp{Q}{P}       
  := 
  \left\{ 
    \begin{array}{ccc} 
      Q & & x \nameeq \quotep{P} \\
      \dropn{x} & & otherwise \\
    \end{array}
  \right.
\end{mathpar}
 

where

\begin{eqnarray}
  (x)\id{\{} \lpquote Q \rpquote / \lpquote P \rpquote \id{\}}            = 
  \left\{ 
    \begin{array}{ccc}
      \lpquote Q \rpquote & & x \nameeq \lpquote P \rpquote \\
      x & & otherwise \\
    \end{array}
  \right. \nonumber
\end{eqnarray}

and $z$ is chosen distinct from $\quotep{P}$, $\quotep{Q}$, the free
names in $Q$, and all the names in $R$. Our $\alpha$-equivalence will
be built in the standard way from this substitution.

\begin{remark}\label{rem:no_self_referential_names}
  One consequence of these definitions is that $\forall P. \quotep{P}
  \not\in \freenames{P}$.
\end{remark}

\subsection{ Dynamic quote: an example }

Anticipating something of what's to come, consider applying the
substitution, $\widehat{\id{\{}u / z \id{\}}}$, to the following pair
of processes, $\lift{w}{y!(z)}$ and $w[ \lpquote y!(z) \rpquote ]$.

\begin{eqnarray}
	\lift{w}{y!(z)}\widehat{\id{\{}u / z \id{\}}}
		& = &
		\lift{w}{y!(u)} \nonumber\\
	w[ \lpquote y!(z) \rpquote ] \widehat{ \id{\{}u / z \id{\}} }
		& = &
		w[ \lpquote y!(z) \rpquote ] \nonumber
\end{eqnarray}

Because the body of the process between quotes is impervious to
substitution, we get radically different answers. In fact, by
examining the first process in an input context,
e.g. $x?(z).\lift{w}{y!(z)}$, we see that the process under the lift
operator may be shaped by prefixed inputs binding a name inside it. In
this sense, the lift operator will be seen as a way to dynamically
construct processes before reifying them as names.

Finally equipped with these standard features we can present the
dynamics of the calculus.

\subsubsection{Operational semantics} 

Finally, we introduce the computational dynamics. What marks these
algebras as distinct from other more traditionally studied algebraic
structures, e.g. vector spaces or polynomial rings, is the manner in
which dynamics is captured. In traditional structures, dynamics is typically
expressed through morphisms between such structures, as in linear maps
between vector spaces or morphisms between rings. In algebras
associated with the semantics of computation, the dynamics is
expressed as part of the algebraic structure itself, through a
reduction reduction relation typically denoted by $\red$. Below, we
give a recursive presentation of this relation for the calculus used
in the encoding.

$\red \subseteq \pi \times \pi$
$\red : \pi \to \mathcal{P}(\pi)$

\begin{mathpar}
  \inferrule* [lab=Comm] { \textsf{match}( x_{src}, x_{trgt} ) } { x_{trgt}?(y)P \; | \; x_{src}!\langle {Q} \rangle \red P\{\quotep{Q}/y}\} }
  \and \\
  \inferrule* [lab=Par] {{P} \red {P}'} {{{P} | {Q}} \red {{P}' | {Q}}}
  \and
  \inferrule* [lab=Equiv]{{{P} \scong {P}'} \andalso {{P}' \red {Q}'} \andalso {{Q}' \scong {Q}}}{{P} \red {Q}}
\end{mathpar}

\begin{eqnarray*}
  match_{\equiv} (\quotep{P},\quotep{Q}) & := & P \equiv Q \\
  match_{\dagger}(\quotep{P},\quotep{Q}) & := & \forall R. P|Q \red^{*} R => R \red^{*} 0 \\
  match_{K}(\quotep{P},\quotep{Q}) & := & K \mbox{ for some context } K
\end{eqnarray*}

$u?(x)P | u!\langle Q \rangle \red P\{\quotep{Q}/x\}$

%We write $\wred$ for $\red^*$, and $P\red$ if $\exists Q $ such that $ P \red Q$.
We write $P\red$ if $\exists Q $ such that $ P \red Q$ and $P\not\red$, otherwise.

\section{Replication}

As mentioned before, it is known that replication (and hence
recursion) can be implemented in a higher-order process algebra
\cite{SangiorgiWalker}. As our first example of calculation with the
machinery thus far presented we give the construction explicitly in
the {\rhoc}.

\begin{eqnarray}
	D_{x} & := & \prefix{x}{y}{(\binpar{\outputp{x}{y}}{@{y}})} \nonumber\\
	\bangp_{x}{P} & := & \binpar{{x}!\langle{\binpar{D_{x}}{P}}\rangle}{D_{x}} \nonumber
\end{eqnarray}

\begin{eqnarray}
	\bangp_{x}{P} & & \nonumber\\
	=
	& {x}!\langle{(\prefix{x}{y}{(\outputp{x}{y} | @{y})) | P}}\rangle 
	      | \prefix{x}{y}{(\outputp{x}{y} | @{y})} & \nonumber\\
	\red
	& (\outputp{x}{y} | @{y})\substn{\quotep{(\prefix{x}{y}{(@{y} | \outputp{x}{y})) | P}}}{y} & \nonumber\\
	=
	& \outputp{x}{\quotep{(\prefix{x}{y}{(\outputp{x}{y} | @{y})) | P}}}
	  | {(\prefix{x}{y}{(\outputp{x}{y} | @{y})) | P}} & \nonumber\\
	\red
	& \ldots & \nonumber\\
	\red^*
	& P | P | \ldots & \nonumber
\end{eqnarray}

Of course, this encoding, as an implementation, runs away, unfolding
$\bangp{P}$ eagerly. A lazier and more implementable replication
operator, restricted to input-guarded processes, may be obtained as follows.

\begin{eqnarray}
\bangp{\prefix{u}{v}{P}} 
	:= 
	\binpar{\lift{x}{\prefix{u}{v}{(\binpar{D(x)}{P})}}}{D(x)} \nonumber
\end{eqnarray}

\begin{remark}
  Note that the lazier definition still does not deal with summation
  or mixed summation (i.e. sums over input and output). The reader is
  invited to construct definitions of replication that deal with these
  features. 

  Further, the definitions are parameterized in a name, $x$. Can you,
  gentle reader, make a definition that eliminates this parameter and
  guarantees no accidental interaction between the replication
  machinery and the process being replicated -- i.e. no accidental
  sharing of names used by the process to get its work done and the
  name(s) used by the replication to effect copying. This latter
  revision of the definition of replication is crucial to obtaining
  the expected identity $!!P \sim !P$.
\end{remark}

\begin{remark}\label{rem:paradoxical_combinator}
  The reader familiar with the lambda calculus will have noticed the
  similarity between $D$ and the paradoxical combinator.

  [Ed. note: the existence of this seems to suggest we have to be more
  restrictive on the set of processes and names we admit if we are to
  support no-cloning.]
\end{remark}

\subsubsection{Bisimulation}

The computational dynamics gives rise to another kind of equivalence,
the equivalence of computational behavior. As previously mentioned
this is typically captured \emph{via} some form of bisimulation.

% The notion we use in this paper is weak barbed bisimulation
% \cite{milner91polyadicpi}.

The notion we use in this paper is derived from weak barbed
bisimulation \cite{milner91polyadicpi}. 

\begin{definition}
An \emph{observation relation}, $\downarrow_{\mathcal N}$, over a set
of names, $\mathcal N$, is the smallest relation satisfying the rules
below.

\infrule[Out-barb]{y \in {\mathcal N}, \; x \nameeq y}
		  {\outputp{x}{v} \downarrow_{\mathcal N} x}
\infrule[Par-barb]{\mbox{$P\downarrow_{\mathcal N} x$ or $Q\downarrow_{\mathcal N} x$}}
		  {\binpar{P}{Q} \downarrow_{\mathcal N} x}

We write $P \Downarrow_{\mathcal N} x$ if there is $Q$ such that 
$P \wred Q$ and $Q \downarrow_{\mathcal N} x$.
\end{definition}

\begin{definition}
%\label{def.bbisim}
An  ${\mathcal N}$-\emph{barbed bisimulation} over a set of names, ${\mathcal N}$, is a symmetric binary relation 
${\mathcal S}_{\mathcal N}$ between agents such that $P\rel{S}_{\mathcal N}Q$ implies:
\begin{enumerate}
\item If $P \red P'$ then $Q \wred Q'$ and $P'\rel{S}_{\mathcal N} Q'$.
\item If $P\downarrow_{\mathcal N} x$, then $Q\Downarrow_{\mathcal N} x$.
\end{enumerate}
$P$ is ${\mathcal N}$-barbed bisimilar to $Q$, written
$P \wbbisim_{\mathcal N} Q$, if $P \rel{S}_{\mathcal N} Q$ for some ${\mathcal N}$-barbed bisimulation ${\mathcal S}_{\mathcal N}$.
\end{definition}

$\mathcal{R} \subseteq \pi \times \pi$

$P \mathcal{R} Q => \forall P'. P \red P' \Rightarrow \exists Q'. Q \red Q', P' \mathcal{R} Q'$

$P \vdash x \Rightarrow Q \vdash x$

\begin{mathpar}
  \inferrule*[lab=Out-barb]{x \nameeq y}{{y}!\langle{Q}\rangle \vdash x}
  \and
  \inferrule*[lab=Par-barb]{\mbox{$P\vdash x$ or $Q\vdash x$}}{\binpar{P}{Q} \vdash x}
\end{mathpar}

\subsubsection{Contexts}

One of the principle advantages of computational calculi like the
$\pi$-calculus is a well-defined notion of context,
contextual-equivalence and a correlation between
contextual-equivalence and notions of bisimulation. The notion of
context allows the decomposition of a process into (sub-)process and
its syntactic environment, its context. Thus, a context may be
thought of as a process with a ``hole'' (written $\Box$) in it. The
application of a context $M$ to a process $P$, written $M[P]$, is
tantamount to filling the hole in $M$ with $P$. In this paper we do
not need the full weight of this theory, but do make use of the notion
of context in the proof the main theorem. 

\begin{mathpar}
  \inferrule* [lab=summation] {} {{M_{M},M_{N}} \bc \Box \;|\; x.M_{A} \;|\; M_{M}+M_{N}}
  \and
  \inferrule* [lab=agent] {} {{M_{A}} \bc (\vec{x})M_{P} \;| \; \clift{P_0,\ldots,M_{P},\ldots,P_N}}
  \and \\
  \inferrule* [lab=process] {} {{M_{P}} \bc M_{N} \;| \;P|M_{P} }
\end{mathpar} 

\begin{mathpar}
  \inferrule* [lab=sychronization] {} {M_{N} \bc \Box \;|\; x?M_{F} \;|\; x!M_{C}}
  \and
  \inferrule* [lab=abstraction] {} {{M_{F}} \bc (x)M_{P} }
  \and
  \inferrule* [lab=concretion] {} {{M_{C}} \bc \langle M_{P} \rangle }
  \and \\
  \inferrule* [lab=process] {} {{M_{P}} \bc M_{N} \;| \;P|M_{P} }
\end{mathpar}

\begin{definition}[contextual application] Given a context $M$, and
  process $P$, we define the \emph{contextual application}, $M[P] :=
  M\{P/\Box\}$. That is, the contextual application of M to P is the
  substitution of $P$ for $\Box$ in $M$.
\end{definition}

$\meaningof{-} : L \to \mathcal{P}(\pi)$

\begin{mathpar}
  \inferrule* [lab=collection] {} {\meaningof{true} = \pi, \and \meaningof{~E} = \pi \setminus \meaningof{E}, \and \meaningof{E_{1} \& E_{2}} = \meaningof{E_{1}} \cap \meaningof{E_{2}}}
\end{mathpar}

\begin{mathpar}
  \inferrule* [lab=structure] {} {\meaningof{0} = \{ P \in \pi | P \equiv 0 \}, \and \\ \meaningof{E_1 | E_2} = \{ P \in \pi | P \equiv P_{1} | P_{2}, P_{1} \in \meaningof{E_{1}}, P_{2} \in \meaningof{E_2}\} }
\end{mathpar}

\begin{mathpar}
 \inferrule* [lab=behavior] {} {\meaningof{\langle a?b \rangle E} = \{ P \in \pi | P \equiv Q | u?(y)P', \\ \and \\\\ \and \\ \;\;\; u \in \meaningof{a}, \forall z.P'\{z/y\} \in \meaningof{E\{z/b\}}\}, \and \\ \meaningof{a!E} = \{ P \in \pi | P \equiv Q | x!\langle P' \rangle, x \in \meaningof{a} P' \in \meaningof{E}\} }
\end{mathpar}

\begin{mathpar}
 \inferrule* [lab=nominal] {} {\meaningof{\quotep{E}} = \{ \quotep{P} \in \quotep{\pi} | P \in \meaningof{E} \}, \and \meaningof{\quotep{P}} = \{ \quotep{Q} \in \quotep{\pi} | P \equiv Q \} \and \\ \meaningof{@\quotep{E}} = \{ P \in \pi | P \equiv @x, x \in \meaningof{E} \}}
\end{mathpar}

\begin{eqnarray*}
  \\
  \meaningof{-} : TS \to ST
\end{eqnarray*}

\begin{eqnarray*}
  \\
  L : TS \to ST
\end{eqnarray*}

\begin{eqnarray*}
  \\
  P \models E \iff P \in \meaningof{E}
\end{eqnarray*}

\begin{eqnarray*}
  P \approx_{L} Q \iff \forall E \in L. P \models E \iff Q \models E
\end{eqnarray*}

\begin{eqnarray*}
  P \approx_{K} Q
\end{eqnarray*}

\begin{eqnarray*}
  P \approx Q
\end{eqnarray*}

$\approx_{K} = \approx = \approx_{L}$

\subsubsection{Contextual duality}

Note that contexts extend the quotation operation to a family of
operations from processes to names. Given a context, $M$, we can
define a \emph{nominal context}, $\quotep{M}$ by $\quotep{M}[P] :=
\quotep{M[P]}$. To foreshadow what is to come we observe that these
operations enjoy a duality with processes very much like the duality
between vectors and maps from vectors to scalars.

Further, because the calculus is essentially higher-order, we have a
correspondence between contexts and processes. More specifically,
given a name $x$ and a context $M$ we can construct $M^{*}_{x}$ such
that 

\begin{mathpar}
  M^{*}_{x} | \lift{x}{P} \red M[P]
\end{mathpar}

namely,

\begin{mathpar}
  M^{*}_{x} := x?(u).M[\dropn{u}]
\end{mathpar}

The dependence of $M^{*}_{x}$ on a name makes it an abstraction, 

\begin{mathpar}
  M^{*} := (x)x?(u).M[\dropn{u}]
\end{mathpar}

\subsection{Additional notation}

It will sometimes be convenient to denote the process a name
quotes. We already have the notation $x = \quotep{P}$, but it will be
convenient to introduce an alternate notation, $\procn{x}$, when we
want to emphasize the connection to the use of the name. Note that, by
virtue of name equivalence, $\quotep{\procn{x}} \nameeq x$; so, the
notation is consistent with previous definitions.

Further, because names have structure it is possible to effect
substitutions on the basis of that structure. This means we need to
upgrade our notation for substitutions, which we accomplish by
adapting comprehension notation. Thus,

\begin{mathpar}
  P\{ y / x : x \in S \}
\end{mathpar}

is interpreted to mean the process derived from P by replacing (in a
capture-avoiding manner) each occurrence of $x$ in $S$ by $y$. For example,

\begin{mathpar}
  P\{ \quotep{\procn{x}|\procn{x}} / x : x \in \freenames{P} \}
\end{mathpar}

will replace each (occurrence) of a free name $x$ in $P$ by
$\quotep{\procn{x}|\procn{x}}$.

Also, we will avail ourselves of the notation $x^{L}$ and $x^{R}$ to
denote injections of a name into disjoint copies of the name
space. There are numerous ways to accomplish this. One example can be
found in \cite{MeredithR05}. This notation overloads to vectors of
names: $\vec{x}^{\pi} := (x_{i}^{\pi} \; : \; 0 \leq i < |\vec{x}| )$ where $\pi \in \{L,R\}$.

We also use $P^{\Box} := P|\Box$.

In \cite{MeredithR05} an interpretation of the new operator is
given. It turns out that there are several possible interpretations
all enjoying the requisite algebraic properties of the operator (see
\cite{milner91polyadicpi}). We will therefore make liberal use of
$(\nu\; \vec{x})P$.

% subsection the_syntax_and_semantics_of_the_notation_system (end)   

\input{qm2pi.qmops} 

\input{qm2pi.sterngerlach} 

\input{qm2pi.metric} 

% section concurrent_process_calculi (end)

%\input{qm2pi.proofsketch}

% section proof sketch (end)

%\input{qm2pi.slviaknots} 

% section spatial logic via knots (end)

\input{qm2pi.conclusion}

% section conclusion (end)

%\input{qm2pi.dtcodes} 

% section wiring algorithm (end)

\input{qm2pi.ack} 

% section acknowledgments (end)

\newpage


\bibliographystyle{plain}   
\bibliography{../../biblios/main.bib}

\input{qm2pi.rhodetails}

\end{document}

 

% section wiring algorithm (end)

\documentclass[12pt]{llncs}
%\documentclass{jktr}

\usepackage[pdftex]{hyperref}                   
\usepackage {listings}
\usepackage {mathpartir}
\usepackage{bcprules}
%\usepackage{listings}
                       
\usepackage{graphicx} 
%\usepackage[margins=2.5cm,nohead,nofoot]{geometry}
%\usepackage{geometry}
\usepackage{amsfonts}
\usepackage{amstext}
\usepackage{latexsym}
\usepackage{amssymb}
\usepackage{color}


%\include{myPreamble}
\include{qm2pi.local} 

%\ifpdf
%\usepackage[pdftex]{graphicx}
%\else
%\usepackage{graphicx}
%\fi

 % \ifpdf
%  \usepackage{pdfsync}
%  \if


%\title{Brief Article}
%\author{David F. Snyder}
%\author{L.G. Meredith}

%\address{Dept. of Math., Texas State University--San Marcos, San Marcos, TX 78666}
       
\pagestyle{empty}


\begin{document}

\lstset{language=[Objective]Caml,frame=shadowbox}

\input{qm2pi.front}

% section front matter (end)

\input{qm2pi.intro} 
 
% section introduction (end)

% \input{qm2pi.knotations} 

% section notation (end)

\input{qm2pi.process.calculi} 

% section concurrent_process_calculi_and_spatial_logics_ (end)
    
%\input{qm2pi.knots2pi} 

%\input{qm2pi.trefoil} 

%\input{qm2pi.mainthm} 

% subsection basic_interpretation (end)

%\input{qm2pi.rho.presentation} 
\subsection{The syntax and semantics of the notation system}\label{sub:the_syntax_and_semantics_of_the_notation_system} % (fold)

We now summarize a technical presentation of the calculus that
embodies our theory of dynamics. The typical presentation of such a
calculus follows the style of giving generators and relations on
them. The grammar, below, describing term constructors, freely
generates the set of processes, $\Proc$. This set is then quotiented
by a relation known as structural congruence and it is over this set
that the notion of dynamics is expressed. This presentation is
essentially that of \cite{MeredithR05} with the addition of
polyadicity and summation. For readability we have relegated some of
the technical subtleties to an appendix.

\subsubsection{Process grammar}\label{subsub:process_grammar}

\begin{mathpar}
  \inferrule* [lab=synchronization] {} {{M} \bc \pzero \;|\; x?F \;|\; x!C }
  \and
  \inferrule* [lab=abstraction] {} {{F} \bc (x)P}
  \and
  \inferrule* [lab=concretion] {} {{C} \bc \langle Q \rangle}
  \and
  \inferrule* [lab=process] {} {{P,Q} \bc M \;| \;P|Q \;|\; @{x}}
  \and
  \inferrule* [lab=name] {} {{x} \bc \quotep{P}}
\end{mathpar} 

Note that $\vec{x}$ (resp. $\vec{P}$) denotes a vector of names
(resp. processes) of length $|\vec{x}|$ (resp. $|\vec{P}|$). We adopt
the following useful abbreviations.

\begin{mathpar}
   x?(\vec{y}).P := x.(\vec{y})P \and  x\clift{\vec{P}} := x.\clift{\vec{P}}
   \and x!(y) := \lift{x}{\dropn{y}}
   \and \Pi_{i=0}^{n-1}P_i := P_0 | \ldots | P_{n-1}
\end{mathpar}

\subsubsection{Structural congruence}

\paragraph{Free and bound names and alpha-equivalence.} At the
core of structural equivalence is alpha-equivalence which identifies
process that are the same up to a change of variable. Formally, we
recognize the distinction between free and bound names. The free names
of a process, $\freenames{P}$, may be calculated recursively as
follows:

\begin{mathpar}
\freenames{\pzero} := \emptyset
  \and \\
  \freenames{x?(y).P} := \{ x \} \cup (\freenames{P} \setminus \{ y \})
  \and 
  \freenames{x!\langle P \rangle} := \{ x \} \cup \{ P \} 
  \and \\
  \freenames{P|Q} := \freenames{P} \cup \freenames{Q}
  \and \\
  \freenames{@{x}} := \{ x \}
\end{mathpar}

$\pi$
$\quotep{\pi}$

$\freenames{-} : \pi \to \mathcal{P}(\quotep{\pi})$

\begin{eqnarray*}
  \freenames{\pzero} & := & \emptyset \\
  \freenames{x?(y).P} & := & \{ x \} \cup (\freenames{P} \setminus \{ y \}) \\
  \freenames{x!\langle P \rangle} & := & \{ x \} \cup \{ P \} \\
  \freenames{P|Q} & := & \freenames{P} \cup \freenames{Q} \\
  \freenames{\dropn{x}} & := & \{ x \}
\end{eqnarray*}

The bound names of a process, $\boundnames{P}$, are those names occurring in $P$
that are not free. For example, in $x?(y).0$, the name $x$ is free, while $y$ is bound.

\begin{mathpar}
  \inferrule* [lab=monoidal-laws] {} { P|Q \equiv Q|P \and P|0 \equiv P \and P|(Q|R) \equiv (P|Q)|R }
\end{mathpar}

\begin{mathpar}
  \inferrule* [lab=alpha-equivalence] {} { (x)P \equiv (y)P\{y/x\} \and y \not\in \freenames{P} }
\end{mathpar}

\begin{definition}
Then two processes, $P,Q$, are alpha-equivalent if $P = Q\{\vec{y}/\vec{x}\}$ for
some $\vec{x} \in \boundnames{Q},\vec{y} \in \boundnames{P}$, where $Q\{\vec{y}/\vec{x}\}$
denotes the capture-avoiding substitution of $\vec{y}$ for $\vec{x}$ in $Q$.
\end{definition}

\begin{definition}
  The {\em structural congruence} \cite{SangiorgiWalker} , $\equiv$,
  between processes is the least congruence containing
  alpha-equivalence, satisfying the abelian monoid laws
  (associativity, commutativity and $\pzero$ as identity) for parallel
  composition $|$ and for summation $+$.
\end{definition}

\subsection{Name equivalence}

We take name equivalence, written $\nameeq$, to be the smallest
equivalence relation generated by the following rules.

\begin{mathpar}
\inferrule*[lab=Quote-drop]
{ }
{ \quotep{@{x}} \nameeq x }

\inferrule*[lab=Struct-equiv]
{ P \scong Q }
{ \quotep{P} \nameeq \quotep{Q} }
\end{mathpar}

The astute reader will have noticed that the mutual recursion of names
and processes imposes a mutual recursion on alpha-equivalence and
structural equivalence via name-equivalence. Fortunately, all of this
works out pleasantly and we may calculate in the natural way, free of
concern. The reader interested in the details is referred to the
appendix \ref{appendix:rho_details}.

\subsection{Substitution}

We use $\Proc$ for the set of processes, $\QProc$ for the set of
names, and $\id{\{}\vec{y} / \vec{x} \id{\}}$ to denote partial maps,
$s : \QProc \rightarrow \QProc$. A map, $s$ lifts, uniquely, to a map
on process terms, $\widehat{s} : \Proc \rightarrow \Proc$ by the
following equations.

\begin{mathpar}
  (0) \psubstp{Q}{P} := 0 \\
  (R \juxtap S) \psubstp{Q}{P}
  :=    
  (R)\psubstp{Q}{P} \juxtap (S) \psubstp{Q}{P} \\
  (x?(y).R) \psubstp{Q}{P}    
  :=    
  (x)\substp{Q}{P} (z)\concat( (R \psubstn{z}{y}) \psubstp{Q}{P} ) \\
  (\lift{x}{R}) \psubstp{Q}{P}  
  :=
  \lift{(x)\substp{Q}{P}}{ R \psubstp{Q}{P} } \\
%   (\dropn{x})  \psubstp{Q}{P}       
%   := 
%   \left\{ 
%     \begin{array}{ccc} 
%       \dropn{\quotep{Q}} & & x \nameeq \quotep{P} \\
%       \dropn{x} & & otherwise \\
%     \end{array}
%   \right. 
  (\dropn{x})  \psubstp{Q}{P}       
  := 
  \left\{ 
    \begin{array}{ccc} 
      Q & & x \nameeq \quotep{P} \\
      \dropn{x} & & otherwise \\
    \end{array}
  \right.
\end{mathpar}
 

where

\begin{eqnarray}
  (x)\id{\{} \lpquote Q \rpquote / \lpquote P \rpquote \id{\}}            = 
  \left\{ 
    \begin{array}{ccc}
      \lpquote Q \rpquote & & x \nameeq \lpquote P \rpquote \\
      x & & otherwise \\
    \end{array}
  \right. \nonumber
\end{eqnarray}

and $z$ is chosen distinct from $\quotep{P}$, $\quotep{Q}$, the free
names in $Q$, and all the names in $R$. Our $\alpha$-equivalence will
be built in the standard way from this substitution.

\begin{remark}\label{rem:no_self_referential_names}
  One consequence of these definitions is that $\forall P. \quotep{P}
  \not\in \freenames{P}$.
\end{remark}

\subsection{ Dynamic quote: an example }

Anticipating something of what's to come, consider applying the
substitution, $\widehat{\id{\{}u / z \id{\}}}$, to the following pair
of processes, $\lift{w}{y!(z)}$ and $w[ \lpquote y!(z) \rpquote ]$.

\begin{eqnarray}
	\lift{w}{y!(z)}\widehat{\id{\{}u / z \id{\}}}
		& = &
		\lift{w}{y!(u)} \nonumber\\
	w[ \lpquote y!(z) \rpquote ] \widehat{ \id{\{}u / z \id{\}} }
		& = &
		w[ \lpquote y!(z) \rpquote ] \nonumber
\end{eqnarray}

Because the body of the process between quotes is impervious to
substitution, we get radically different answers. In fact, by
examining the first process in an input context,
e.g. $x?(z).\lift{w}{y!(z)}$, we see that the process under the lift
operator may be shaped by prefixed inputs binding a name inside it. In
this sense, the lift operator will be seen as a way to dynamically
construct processes before reifying them as names.

Finally equipped with these standard features we can present the
dynamics of the calculus.

\subsubsection{Operational semantics} 

Finally, we introduce the computational dynamics. What marks these
algebras as distinct from other more traditionally studied algebraic
structures, e.g. vector spaces or polynomial rings, is the manner in
which dynamics is captured. In traditional structures, dynamics is typically
expressed through morphisms between such structures, as in linear maps
between vector spaces or morphisms between rings. In algebras
associated with the semantics of computation, the dynamics is
expressed as part of the algebraic structure itself, through a
reduction reduction relation typically denoted by $\red$. Below, we
give a recursive presentation of this relation for the calculus used
in the encoding.

$\red \subseteq \pi \times \pi$
$\red : \pi \to \mathcal{P}(\pi)$

\begin{mathpar}
  \inferrule* [lab=Comm] { \textsf{match}( x_{src}, x_{trgt} ) } { x_{trgt}?(y)P \; | \; x_{src}!\langle {Q} \rangle \red P\{\quotep{Q}/y}\} }
  \and \\
  \inferrule* [lab=Par] {{P} \red {P}'} {{{P} | {Q}} \red {{P}' | {Q}}}
  \and
  \inferrule* [lab=Equiv]{{{P} \scong {P}'} \andalso {{P}' \red {Q}'} \andalso {{Q}' \scong {Q}}}{{P} \red {Q}}
\end{mathpar}

\begin{eqnarray*}
  match_{\equiv} (\quotep{P},\quotep{Q}) & := & P \equiv Q \\
  match_{\dagger}(\quotep{P},\quotep{Q}) & := & \forall R. P|Q \red^{*} R => R \red^{*} 0 \\
  match_{K}(\quotep{P},\quotep{Q}) & := & K \mbox{ for some context } K
\end{eqnarray*}

$u?(x)P | u!\langle Q \rangle \red P\{\quotep{Q}/x\}$

%We write $\wred$ for $\red^*$, and $P\red$ if $\exists Q $ such that $ P \red Q$.
We write $P\red$ if $\exists Q $ such that $ P \red Q$ and $P\not\red$, otherwise.

\section{Replication}

As mentioned before, it is known that replication (and hence
recursion) can be implemented in a higher-order process algebra
\cite{SangiorgiWalker}. As our first example of calculation with the
machinery thus far presented we give the construction explicitly in
the {\rhoc}.

\begin{eqnarray}
	D_{x} & := & \prefix{x}{y}{(\binpar{\outputp{x}{y}}{@{y}})} \nonumber\\
	\bangp_{x}{P} & := & \binpar{{x}!\langle{\binpar{D_{x}}{P}}\rangle}{D_{x}} \nonumber
\end{eqnarray}

\begin{eqnarray}
	\bangp_{x}{P} & & \nonumber\\
	=
	& {x}!\langle{(\prefix{x}{y}{(\outputp{x}{y} | @{y})) | P}}\rangle 
	      | \prefix{x}{y}{(\outputp{x}{y} | @{y})} & \nonumber\\
	\red
	& (\outputp{x}{y} | @{y})\substn{\quotep{(\prefix{x}{y}{(@{y} | \outputp{x}{y})) | P}}}{y} & \nonumber\\
	=
	& \outputp{x}{\quotep{(\prefix{x}{y}{(\outputp{x}{y} | @{y})) | P}}}
	  | {(\prefix{x}{y}{(\outputp{x}{y} | @{y})) | P}} & \nonumber\\
	\red
	& \ldots & \nonumber\\
	\red^*
	& P | P | \ldots & \nonumber
\end{eqnarray}

Of course, this encoding, as an implementation, runs away, unfolding
$\bangp{P}$ eagerly. A lazier and more implementable replication
operator, restricted to input-guarded processes, may be obtained as follows.

\begin{eqnarray}
\bangp{\prefix{u}{v}{P}} 
	:= 
	\binpar{\lift{x}{\prefix{u}{v}{(\binpar{D(x)}{P})}}}{D(x)} \nonumber
\end{eqnarray}

\begin{remark}
  Note that the lazier definition still does not deal with summation
  or mixed summation (i.e. sums over input and output). The reader is
  invited to construct definitions of replication that deal with these
  features. 

  Further, the definitions are parameterized in a name, $x$. Can you,
  gentle reader, make a definition that eliminates this parameter and
  guarantees no accidental interaction between the replication
  machinery and the process being replicated -- i.e. no accidental
  sharing of names used by the process to get its work done and the
  name(s) used by the replication to effect copying. This latter
  revision of the definition of replication is crucial to obtaining
  the expected identity $!!P \sim !P$.
\end{remark}

\begin{remark}\label{rem:paradoxical_combinator}
  The reader familiar with the lambda calculus will have noticed the
  similarity between $D$ and the paradoxical combinator.

  [Ed. note: the existence of this seems to suggest we have to be more
  restrictive on the set of processes and names we admit if we are to
  support no-cloning.]
\end{remark}

\subsubsection{Bisimulation}

The computational dynamics gives rise to another kind of equivalence,
the equivalence of computational behavior. As previously mentioned
this is typically captured \emph{via} some form of bisimulation.

% The notion we use in this paper is weak barbed bisimulation
% \cite{milner91polyadicpi}.

The notion we use in this paper is derived from weak barbed
bisimulation \cite{milner91polyadicpi}. 

\begin{definition}
An \emph{observation relation}, $\downarrow_{\mathcal N}$, over a set
of names, $\mathcal N$, is the smallest relation satisfying the rules
below.

\infrule[Out-barb]{y \in {\mathcal N}, \; x \nameeq y}
		  {\outputp{x}{v} \downarrow_{\mathcal N} x}
\infrule[Par-barb]{\mbox{$P\downarrow_{\mathcal N} x$ or $Q\downarrow_{\mathcal N} x$}}
		  {\binpar{P}{Q} \downarrow_{\mathcal N} x}

We write $P \Downarrow_{\mathcal N} x$ if there is $Q$ such that 
$P \wred Q$ and $Q \downarrow_{\mathcal N} x$.
\end{definition}

\begin{definition}
%\label{def.bbisim}
An  ${\mathcal N}$-\emph{barbed bisimulation} over a set of names, ${\mathcal N}$, is a symmetric binary relation 
${\mathcal S}_{\mathcal N}$ between agents such that $P\rel{S}_{\mathcal N}Q$ implies:
\begin{enumerate}
\item If $P \red P'$ then $Q \wred Q'$ and $P'\rel{S}_{\mathcal N} Q'$.
\item If $P\downarrow_{\mathcal N} x$, then $Q\Downarrow_{\mathcal N} x$.
\end{enumerate}
$P$ is ${\mathcal N}$-barbed bisimilar to $Q$, written
$P \wbbisim_{\mathcal N} Q$, if $P \rel{S}_{\mathcal N} Q$ for some ${\mathcal N}$-barbed bisimulation ${\mathcal S}_{\mathcal N}$.
\end{definition}

$\mathcal{R} \subseteq \pi \times \pi$

$P \mathcal{R} Q => \forall P'. P \red P' \Rightarrow \exists Q'. Q \red Q', P' \mathcal{R} Q'$

$P \vdash x \Rightarrow Q \vdash x$

\begin{mathpar}
  \inferrule*[lab=Out-barb]{x \nameeq y}{{y}!\langle{Q}\rangle \vdash x}
  \and
  \inferrule*[lab=Par-barb]{\mbox{$P\vdash x$ or $Q\vdash x$}}{\binpar{P}{Q} \vdash x}
\end{mathpar}

\subsubsection{Contexts}

One of the principle advantages of computational calculi like the
$\pi$-calculus is a well-defined notion of context,
contextual-equivalence and a correlation between
contextual-equivalence and notions of bisimulation. The notion of
context allows the decomposition of a process into (sub-)process and
its syntactic environment, its context. Thus, a context may be
thought of as a process with a ``hole'' (written $\Box$) in it. The
application of a context $M$ to a process $P$, written $M[P]$, is
tantamount to filling the hole in $M$ with $P$. In this paper we do
not need the full weight of this theory, but do make use of the notion
of context in the proof the main theorem. 

\begin{mathpar}
  \inferrule* [lab=summation] {} {{M_{M},M_{N}} \bc \Box \;|\; x.M_{A} \;|\; M_{M}+M_{N}}
  \and
  \inferrule* [lab=agent] {} {{M_{A}} \bc (\vec{x})M_{P} \;| \; \clift{P_0,\ldots,M_{P},\ldots,P_N}}
  \and \\
  \inferrule* [lab=process] {} {{M_{P}} \bc M_{N} \;| \;P|M_{P} }
\end{mathpar} 

\begin{mathpar}
  \inferrule* [lab=sychronization] {} {M_{N} \bc \Box \;|\; x?M_{F} \;|\; x!M_{C}}
  \and
  \inferrule* [lab=abstraction] {} {{M_{F}} \bc (x)M_{P} }
  \and
  \inferrule* [lab=concretion] {} {{M_{C}} \bc \langle M_{P} \rangle }
  \and \\
  \inferrule* [lab=process] {} {{M_{P}} \bc M_{N} \;| \;P|M_{P} }
\end{mathpar}

\begin{definition}[contextual application] Given a context $M$, and
  process $P$, we define the \emph{contextual application}, $M[P] :=
  M\{P/\Box\}$. That is, the contextual application of M to P is the
  substitution of $P$ for $\Box$ in $M$.
\end{definition}

$\meaningof{-} : L \to \mathcal{P}(\pi)$

\begin{mathpar}
  \inferrule* [lab=collection] {} {\meaningof{true} = \pi, \and \meaningof{~E} = \pi \setminus \meaningof{E}, \and \meaningof{E_{1} \& E_{2}} = \meaningof{E_{1}} \cap \meaningof{E_{2}}}
\end{mathpar}

\begin{mathpar}
  \inferrule* [lab=structure] {} {\meaningof{0} = \{ P \in \pi | P \equiv 0 \}, \and \\ \meaningof{E_1 | E_2} = \{ P \in \pi | P \equiv P_{1} | P_{2}, P_{1} \in \meaningof{E_{1}}, P_{2} \in \meaningof{E_2}\} }
\end{mathpar}

\begin{mathpar}
 \inferrule* [lab=behavior] {} {\meaningof{\langle a?b \rangle E} = \{ P \in \pi | P \equiv Q | u?(y)P', \\ \and \\\\ \and \\ \;\;\; u \in \meaningof{a}, \forall z.P'\{z/y\} \in \meaningof{E\{z/b\}}\}, \and \\ \meaningof{a!E} = \{ P \in \pi | P \equiv Q | x!\langle P' \rangle, x \in \meaningof{a} P' \in \meaningof{E}\} }
\end{mathpar}

\begin{mathpar}
 \inferrule* [lab=nominal] {} {\meaningof{\quotep{E}} = \{ \quotep{P} \in \quotep{\pi} | P \in \meaningof{E} \}, \and \meaningof{\quotep{P}} = \{ \quotep{Q} \in \quotep{\pi} | P \equiv Q \} \and \\ \meaningof{@\quotep{E}} = \{ P \in \pi | P \equiv @x, x \in \meaningof{E} \}}
\end{mathpar}

\begin{eqnarray*}
  \\
  \meaningof{-} : TS \to ST
\end{eqnarray*}

\begin{eqnarray*}
  \\
  L : TS \to ST
\end{eqnarray*}

\begin{eqnarray*}
  \\
  P \models E \iff P \in \meaningof{E}
\end{eqnarray*}

\begin{eqnarray*}
  P \approx_{L} Q \iff \forall E \in L. P \models E \iff Q \models E
\end{eqnarray*}

\begin{eqnarray*}
  P \approx_{K} Q
\end{eqnarray*}

\begin{eqnarray*}
  P \approx Q
\end{eqnarray*}

$\approx_{K} = \approx = \approx_{L}$

\subsubsection{Contextual duality}

Note that contexts extend the quotation operation to a family of
operations from processes to names. Given a context, $M$, we can
define a \emph{nominal context}, $\quotep{M}$ by $\quotep{M}[P] :=
\quotep{M[P]}$. To foreshadow what is to come we observe that these
operations enjoy a duality with processes very much like the duality
between vectors and maps from vectors to scalars.

Further, because the calculus is essentially higher-order, we have a
correspondence between contexts and processes. More specifically,
given a name $x$ and a context $M$ we can construct $M^{*}_{x}$ such
that 

\begin{mathpar}
  M^{*}_{x} | \lift{x}{P} \red M[P]
\end{mathpar}

namely,

\begin{mathpar}
  M^{*}_{x} := x?(u).M[\dropn{u}]
\end{mathpar}

The dependence of $M^{*}_{x}$ on a name makes it an abstraction, 

\begin{mathpar}
  M^{*} := (x)x?(u).M[\dropn{u}]
\end{mathpar}

\subsection{Additional notation}

It will sometimes be convenient to denote the process a name
quotes. We already have the notation $x = \quotep{P}$, but it will be
convenient to introduce an alternate notation, $\procn{x}$, when we
want to emphasize the connection to the use of the name. Note that, by
virtue of name equivalence, $\quotep{\procn{x}} \nameeq x$; so, the
notation is consistent with previous definitions.

Further, because names have structure it is possible to effect
substitutions on the basis of that structure. This means we need to
upgrade our notation for substitutions, which we accomplish by
adapting comprehension notation. Thus,

\begin{mathpar}
  P\{ y / x : x \in S \}
\end{mathpar}

is interpreted to mean the process derived from P by replacing (in a
capture-avoiding manner) each occurrence of $x$ in $S$ by $y$. For example,

\begin{mathpar}
  P\{ \quotep{\procn{x}|\procn{x}} / x : x \in \freenames{P} \}
\end{mathpar}

will replace each (occurrence) of a free name $x$ in $P$ by
$\quotep{\procn{x}|\procn{x}}$.

Also, we will avail ourselves of the notation $x^{L}$ and $x^{R}$ to
denote injections of a name into disjoint copies of the name
space. There are numerous ways to accomplish this. One example can be
found in \cite{MeredithR05}. This notation overloads to vectors of
names: $\vec{x}^{\pi} := (x_{i}^{\pi} \; : \; 0 \leq i < |\vec{x}| )$ where $\pi \in \{L,R\}$.

We also use $P^{\Box} := P|\Box$.

In \cite{MeredithR05} an interpretation of the new operator is
given. It turns out that there are several possible interpretations
all enjoying the requisite algebraic properties of the operator (see
\cite{milner91polyadicpi}). We will therefore make liberal use of
$(\nu\; \vec{x})P$.

% subsection the_syntax_and_semantics_of_the_notation_system (end)   

\input{qm2pi.qmops} 

\input{qm2pi.sterngerlach} 

\input{qm2pi.metric} 

% section concurrent_process_calculi (end)

%\input{qm2pi.proofsketch}

% section proof sketch (end)

%\input{qm2pi.slviaknots} 

% section spatial logic via knots (end)

\input{qm2pi.conclusion}

% section conclusion (end)

%\input{qm2pi.dtcodes} 

% section wiring algorithm (end)

\input{qm2pi.ack} 

% section acknowledgments (end)

\newpage


\bibliographystyle{plain}   
\bibliography{../../biblios/main.bib}

\input{qm2pi.rhodetails}

\end{document}

 

% section acknowledgments (end)

\newpage


\bibliographystyle{plain}   
\bibliography{../../biblios/main.bib}

\documentclass[12pt]{llncs}
%\documentclass{jktr}

\usepackage[pdftex]{hyperref}                   
\usepackage {listings}
\usepackage {mathpartir}
\usepackage{bcprules}
%\usepackage{listings}
                       
\usepackage{graphicx} 
%\usepackage[margins=2.5cm,nohead,nofoot]{geometry}
%\usepackage{geometry}
\usepackage{amsfonts}
\usepackage{amstext}
\usepackage{latexsym}
\usepackage{amssymb}
\usepackage{color}


%\include{myPreamble}
\include{qm2pi.local} 

%\ifpdf
%\usepackage[pdftex]{graphicx}
%\else
%\usepackage{graphicx}
%\fi

 % \ifpdf
%  \usepackage{pdfsync}
%  \if


%\title{Brief Article}
%\author{David F. Snyder}
%\author{L.G. Meredith}

%\address{Dept. of Math., Texas State University--San Marcos, San Marcos, TX 78666}
       
\pagestyle{empty}


\begin{document}

\lstset{language=[Objective]Caml,frame=shadowbox}

\input{qm2pi.front}

% section front matter (end)

\input{qm2pi.intro} 
 
% section introduction (end)

% \input{qm2pi.knotations} 

% section notation (end)

\input{qm2pi.process.calculi} 

% section concurrent_process_calculi_and_spatial_logics_ (end)
    
%\input{qm2pi.knots2pi} 

%\input{qm2pi.trefoil} 

%\input{qm2pi.mainthm} 

% subsection basic_interpretation (end)

%\input{qm2pi.rho.presentation} 
\subsection{The syntax and semantics of the notation system}\label{sub:the_syntax_and_semantics_of_the_notation_system} % (fold)

We now summarize a technical presentation of the calculus that
embodies our theory of dynamics. The typical presentation of such a
calculus follows the style of giving generators and relations on
them. The grammar, below, describing term constructors, freely
generates the set of processes, $\Proc$. This set is then quotiented
by a relation known as structural congruence and it is over this set
that the notion of dynamics is expressed. This presentation is
essentially that of \cite{MeredithR05} with the addition of
polyadicity and summation. For readability we have relegated some of
the technical subtleties to an appendix.

\subsubsection{Process grammar}\label{subsub:process_grammar}

\begin{mathpar}
  \inferrule* [lab=synchronization] {} {{M} \bc \pzero \;|\; x?F \;|\; x!C }
  \and
  \inferrule* [lab=abstraction] {} {{F} \bc (x)P}
  \and
  \inferrule* [lab=concretion] {} {{C} \bc \langle Q \rangle}
  \and
  \inferrule* [lab=process] {} {{P,Q} \bc M \;| \;P|Q \;|\; @{x}}
  \and
  \inferrule* [lab=name] {} {{x} \bc \quotep{P}}
\end{mathpar} 

Note that $\vec{x}$ (resp. $\vec{P}$) denotes a vector of names
(resp. processes) of length $|\vec{x}|$ (resp. $|\vec{P}|$). We adopt
the following useful abbreviations.

\begin{mathpar}
   x?(\vec{y}).P := x.(\vec{y})P \and  x\clift{\vec{P}} := x.\clift{\vec{P}}
   \and x!(y) := \lift{x}{\dropn{y}}
   \and \Pi_{i=0}^{n-1}P_i := P_0 | \ldots | P_{n-1}
\end{mathpar}

\subsubsection{Structural congruence}

\paragraph{Free and bound names and alpha-equivalence.} At the
core of structural equivalence is alpha-equivalence which identifies
process that are the same up to a change of variable. Formally, we
recognize the distinction between free and bound names. The free names
of a process, $\freenames{P}$, may be calculated recursively as
follows:

\begin{mathpar}
\freenames{\pzero} := \emptyset
  \and \\
  \freenames{x?(y).P} := \{ x \} \cup (\freenames{P} \setminus \{ y \})
  \and 
  \freenames{x!\langle P \rangle} := \{ x \} \cup \{ P \} 
  \and \\
  \freenames{P|Q} := \freenames{P} \cup \freenames{Q}
  \and \\
  \freenames{@{x}} := \{ x \}
\end{mathpar}

$\pi$
$\quotep{\pi}$

$\freenames{-} : \pi \to \mathcal{P}(\quotep{\pi})$

\begin{eqnarray*}
  \freenames{\pzero} & := & \emptyset \\
  \freenames{x?(y).P} & := & \{ x \} \cup (\freenames{P} \setminus \{ y \}) \\
  \freenames{x!\langle P \rangle} & := & \{ x \} \cup \{ P \} \\
  \freenames{P|Q} & := & \freenames{P} \cup \freenames{Q} \\
  \freenames{\dropn{x}} & := & \{ x \}
\end{eqnarray*}

The bound names of a process, $\boundnames{P}$, are those names occurring in $P$
that are not free. For example, in $x?(y).0$, the name $x$ is free, while $y$ is bound.

\begin{mathpar}
  \inferrule* [lab=monoidal-laws] {} { P|Q \equiv Q|P \and P|0 \equiv P \and P|(Q|R) \equiv (P|Q)|R }
\end{mathpar}

\begin{mathpar}
  \inferrule* [lab=alpha-equivalence] {} { (x)P \equiv (y)P\{y/x\} \and y \not\in \freenames{P} }
\end{mathpar}

\begin{definition}
Then two processes, $P,Q$, are alpha-equivalent if $P = Q\{\vec{y}/\vec{x}\}$ for
some $\vec{x} \in \boundnames{Q},\vec{y} \in \boundnames{P}$, where $Q\{\vec{y}/\vec{x}\}$
denotes the capture-avoiding substitution of $\vec{y}$ for $\vec{x}$ in $Q$.
\end{definition}

\begin{definition}
  The {\em structural congruence} \cite{SangiorgiWalker} , $\equiv$,
  between processes is the least congruence containing
  alpha-equivalence, satisfying the abelian monoid laws
  (associativity, commutativity and $\pzero$ as identity) for parallel
  composition $|$ and for summation $+$.
\end{definition}

\subsection{Name equivalence}

We take name equivalence, written $\nameeq$, to be the smallest
equivalence relation generated by the following rules.

\begin{mathpar}
\inferrule*[lab=Quote-drop]
{ }
{ \quotep{@{x}} \nameeq x }

\inferrule*[lab=Struct-equiv]
{ P \scong Q }
{ \quotep{P} \nameeq \quotep{Q} }
\end{mathpar}

The astute reader will have noticed that the mutual recursion of names
and processes imposes a mutual recursion on alpha-equivalence and
structural equivalence via name-equivalence. Fortunately, all of this
works out pleasantly and we may calculate in the natural way, free of
concern. The reader interested in the details is referred to the
appendix \ref{appendix:rho_details}.

\subsection{Substitution}

We use $\Proc$ for the set of processes, $\QProc$ for the set of
names, and $\id{\{}\vec{y} / \vec{x} \id{\}}$ to denote partial maps,
$s : \QProc \rightarrow \QProc$. A map, $s$ lifts, uniquely, to a map
on process terms, $\widehat{s} : \Proc \rightarrow \Proc$ by the
following equations.

\begin{mathpar}
  (0) \psubstp{Q}{P} := 0 \\
  (R \juxtap S) \psubstp{Q}{P}
  :=    
  (R)\psubstp{Q}{P} \juxtap (S) \psubstp{Q}{P} \\
  (x?(y).R) \psubstp{Q}{P}    
  :=    
  (x)\substp{Q}{P} (z)\concat( (R \psubstn{z}{y}) \psubstp{Q}{P} ) \\
  (\lift{x}{R}) \psubstp{Q}{P}  
  :=
  \lift{(x)\substp{Q}{P}}{ R \psubstp{Q}{P} } \\
%   (\dropn{x})  \psubstp{Q}{P}       
%   := 
%   \left\{ 
%     \begin{array}{ccc} 
%       \dropn{\quotep{Q}} & & x \nameeq \quotep{P} \\
%       \dropn{x} & & otherwise \\
%     \end{array}
%   \right. 
  (\dropn{x})  \psubstp{Q}{P}       
  := 
  \left\{ 
    \begin{array}{ccc} 
      Q & & x \nameeq \quotep{P} \\
      \dropn{x} & & otherwise \\
    \end{array}
  \right.
\end{mathpar}
 

where

\begin{eqnarray}
  (x)\id{\{} \lpquote Q \rpquote / \lpquote P \rpquote \id{\}}            = 
  \left\{ 
    \begin{array}{ccc}
      \lpquote Q \rpquote & & x \nameeq \lpquote P \rpquote \\
      x & & otherwise \\
    \end{array}
  \right. \nonumber
\end{eqnarray}

and $z$ is chosen distinct from $\quotep{P}$, $\quotep{Q}$, the free
names in $Q$, and all the names in $R$. Our $\alpha$-equivalence will
be built in the standard way from this substitution.

\begin{remark}\label{rem:no_self_referential_names}
  One consequence of these definitions is that $\forall P. \quotep{P}
  \not\in \freenames{P}$.
\end{remark}

\subsection{ Dynamic quote: an example }

Anticipating something of what's to come, consider applying the
substitution, $\widehat{\id{\{}u / z \id{\}}}$, to the following pair
of processes, $\lift{w}{y!(z)}$ and $w[ \lpquote y!(z) \rpquote ]$.

\begin{eqnarray}
	\lift{w}{y!(z)}\widehat{\id{\{}u / z \id{\}}}
		& = &
		\lift{w}{y!(u)} \nonumber\\
	w[ \lpquote y!(z) \rpquote ] \widehat{ \id{\{}u / z \id{\}} }
		& = &
		w[ \lpquote y!(z) \rpquote ] \nonumber
\end{eqnarray}

Because the body of the process between quotes is impervious to
substitution, we get radically different answers. In fact, by
examining the first process in an input context,
e.g. $x?(z).\lift{w}{y!(z)}$, we see that the process under the lift
operator may be shaped by prefixed inputs binding a name inside it. In
this sense, the lift operator will be seen as a way to dynamically
construct processes before reifying them as names.

Finally equipped with these standard features we can present the
dynamics of the calculus.

\subsubsection{Operational semantics} 

Finally, we introduce the computational dynamics. What marks these
algebras as distinct from other more traditionally studied algebraic
structures, e.g. vector spaces or polynomial rings, is the manner in
which dynamics is captured. In traditional structures, dynamics is typically
expressed through morphisms between such structures, as in linear maps
between vector spaces or morphisms between rings. In algebras
associated with the semantics of computation, the dynamics is
expressed as part of the algebraic structure itself, through a
reduction reduction relation typically denoted by $\red$. Below, we
give a recursive presentation of this relation for the calculus used
in the encoding.

$\red \subseteq \pi \times \pi$
$\red : \pi \to \mathcal{P}(\pi)$

\begin{mathpar}
  \inferrule* [lab=Comm] { \textsf{match}( x_{src}, x_{trgt} ) } { x_{trgt}?(y)P \; | \; x_{src}!\langle {Q} \rangle \red P\{\quotep{Q}/y}\} }
  \and \\
  \inferrule* [lab=Par] {{P} \red {P}'} {{{P} | {Q}} \red {{P}' | {Q}}}
  \and
  \inferrule* [lab=Equiv]{{{P} \scong {P}'} \andalso {{P}' \red {Q}'} \andalso {{Q}' \scong {Q}}}{{P} \red {Q}}
\end{mathpar}

\begin{eqnarray*}
  match_{\equiv} (\quotep{P},\quotep{Q}) & := & P \equiv Q \\
  match_{\dagger}(\quotep{P},\quotep{Q}) & := & \forall R. P|Q \red^{*} R => R \red^{*} 0 \\
  match_{K}(\quotep{P},\quotep{Q}) & := & K \mbox{ for some context } K
\end{eqnarray*}

$u?(x)P | u!\langle Q \rangle \red P\{\quotep{Q}/x\}$

%We write $\wred$ for $\red^*$, and $P\red$ if $\exists Q $ such that $ P \red Q$.
We write $P\red$ if $\exists Q $ such that $ P \red Q$ and $P\not\red$, otherwise.

\section{Replication}

As mentioned before, it is known that replication (and hence
recursion) can be implemented in a higher-order process algebra
\cite{SangiorgiWalker}. As our first example of calculation with the
machinery thus far presented we give the construction explicitly in
the {\rhoc}.

\begin{eqnarray}
	D_{x} & := & \prefix{x}{y}{(\binpar{\outputp{x}{y}}{@{y}})} \nonumber\\
	\bangp_{x}{P} & := & \binpar{{x}!\langle{\binpar{D_{x}}{P}}\rangle}{D_{x}} \nonumber
\end{eqnarray}

\begin{eqnarray}
	\bangp_{x}{P} & & \nonumber\\
	=
	& {x}!\langle{(\prefix{x}{y}{(\outputp{x}{y} | @{y})) | P}}\rangle 
	      | \prefix{x}{y}{(\outputp{x}{y} | @{y})} & \nonumber\\
	\red
	& (\outputp{x}{y} | @{y})\substn{\quotep{(\prefix{x}{y}{(@{y} | \outputp{x}{y})) | P}}}{y} & \nonumber\\
	=
	& \outputp{x}{\quotep{(\prefix{x}{y}{(\outputp{x}{y} | @{y})) | P}}}
	  | {(\prefix{x}{y}{(\outputp{x}{y} | @{y})) | P}} & \nonumber\\
	\red
	& \ldots & \nonumber\\
	\red^*
	& P | P | \ldots & \nonumber
\end{eqnarray}

Of course, this encoding, as an implementation, runs away, unfolding
$\bangp{P}$ eagerly. A lazier and more implementable replication
operator, restricted to input-guarded processes, may be obtained as follows.

\begin{eqnarray}
\bangp{\prefix{u}{v}{P}} 
	:= 
	\binpar{\lift{x}{\prefix{u}{v}{(\binpar{D(x)}{P})}}}{D(x)} \nonumber
\end{eqnarray}

\begin{remark}
  Note that the lazier definition still does not deal with summation
  or mixed summation (i.e. sums over input and output). The reader is
  invited to construct definitions of replication that deal with these
  features. 

  Further, the definitions are parameterized in a name, $x$. Can you,
  gentle reader, make a definition that eliminates this parameter and
  guarantees no accidental interaction between the replication
  machinery and the process being replicated -- i.e. no accidental
  sharing of names used by the process to get its work done and the
  name(s) used by the replication to effect copying. This latter
  revision of the definition of replication is crucial to obtaining
  the expected identity $!!P \sim !P$.
\end{remark}

\begin{remark}\label{rem:paradoxical_combinator}
  The reader familiar with the lambda calculus will have noticed the
  similarity between $D$ and the paradoxical combinator.

  [Ed. note: the existence of this seems to suggest we have to be more
  restrictive on the set of processes and names we admit if we are to
  support no-cloning.]
\end{remark}

\subsubsection{Bisimulation}

The computational dynamics gives rise to another kind of equivalence,
the equivalence of computational behavior. As previously mentioned
this is typically captured \emph{via} some form of bisimulation.

% The notion we use in this paper is weak barbed bisimulation
% \cite{milner91polyadicpi}.

The notion we use in this paper is derived from weak barbed
bisimulation \cite{milner91polyadicpi}. 

\begin{definition}
An \emph{observation relation}, $\downarrow_{\mathcal N}$, over a set
of names, $\mathcal N$, is the smallest relation satisfying the rules
below.

\infrule[Out-barb]{y \in {\mathcal N}, \; x \nameeq y}
		  {\outputp{x}{v} \downarrow_{\mathcal N} x}
\infrule[Par-barb]{\mbox{$P\downarrow_{\mathcal N} x$ or $Q\downarrow_{\mathcal N} x$}}
		  {\binpar{P}{Q} \downarrow_{\mathcal N} x}

We write $P \Downarrow_{\mathcal N} x$ if there is $Q$ such that 
$P \wred Q$ and $Q \downarrow_{\mathcal N} x$.
\end{definition}

\begin{definition}
%\label{def.bbisim}
An  ${\mathcal N}$-\emph{barbed bisimulation} over a set of names, ${\mathcal N}$, is a symmetric binary relation 
${\mathcal S}_{\mathcal N}$ between agents such that $P\rel{S}_{\mathcal N}Q$ implies:
\begin{enumerate}
\item If $P \red P'$ then $Q \wred Q'$ and $P'\rel{S}_{\mathcal N} Q'$.
\item If $P\downarrow_{\mathcal N} x$, then $Q\Downarrow_{\mathcal N} x$.
\end{enumerate}
$P$ is ${\mathcal N}$-barbed bisimilar to $Q$, written
$P \wbbisim_{\mathcal N} Q$, if $P \rel{S}_{\mathcal N} Q$ for some ${\mathcal N}$-barbed bisimulation ${\mathcal S}_{\mathcal N}$.
\end{definition}

$\mathcal{R} \subseteq \pi \times \pi$

$P \mathcal{R} Q => \forall P'. P \red P' \Rightarrow \exists Q'. Q \red Q', P' \mathcal{R} Q'$

$P \vdash x \Rightarrow Q \vdash x$

\begin{mathpar}
  \inferrule*[lab=Out-barb]{x \nameeq y}{{y}!\langle{Q}\rangle \vdash x}
  \and
  \inferrule*[lab=Par-barb]{\mbox{$P\vdash x$ or $Q\vdash x$}}{\binpar{P}{Q} \vdash x}
\end{mathpar}

\subsubsection{Contexts}

One of the principle advantages of computational calculi like the
$\pi$-calculus is a well-defined notion of context,
contextual-equivalence and a correlation between
contextual-equivalence and notions of bisimulation. The notion of
context allows the decomposition of a process into (sub-)process and
its syntactic environment, its context. Thus, a context may be
thought of as a process with a ``hole'' (written $\Box$) in it. The
application of a context $M$ to a process $P$, written $M[P]$, is
tantamount to filling the hole in $M$ with $P$. In this paper we do
not need the full weight of this theory, but do make use of the notion
of context in the proof the main theorem. 

\begin{mathpar}
  \inferrule* [lab=summation] {} {{M_{M},M_{N}} \bc \Box \;|\; x.M_{A} \;|\; M_{M}+M_{N}}
  \and
  \inferrule* [lab=agent] {} {{M_{A}} \bc (\vec{x})M_{P} \;| \; \clift{P_0,\ldots,M_{P},\ldots,P_N}}
  \and \\
  \inferrule* [lab=process] {} {{M_{P}} \bc M_{N} \;| \;P|M_{P} }
\end{mathpar} 

\begin{mathpar}
  \inferrule* [lab=sychronization] {} {M_{N} \bc \Box \;|\; x?M_{F} \;|\; x!M_{C}}
  \and
  \inferrule* [lab=abstraction] {} {{M_{F}} \bc (x)M_{P} }
  \and
  \inferrule* [lab=concretion] {} {{M_{C}} \bc \langle M_{P} \rangle }
  \and \\
  \inferrule* [lab=process] {} {{M_{P}} \bc M_{N} \;| \;P|M_{P} }
\end{mathpar}

\begin{definition}[contextual application] Given a context $M$, and
  process $P$, we define the \emph{contextual application}, $M[P] :=
  M\{P/\Box\}$. That is, the contextual application of M to P is the
  substitution of $P$ for $\Box$ in $M$.
\end{definition}

$\meaningof{-} : L \to \mathcal{P}(\pi)$

\begin{mathpar}
  \inferrule* [lab=collection] {} {\meaningof{true} = \pi, \and \meaningof{~E} = \pi \setminus \meaningof{E}, \and \meaningof{E_{1} \& E_{2}} = \meaningof{E_{1}} \cap \meaningof{E_{2}}}
\end{mathpar}

\begin{mathpar}
  \inferrule* [lab=structure] {} {\meaningof{0} = \{ P \in \pi | P \equiv 0 \}, \and \\ \meaningof{E_1 | E_2} = \{ P \in \pi | P \equiv P_{1} | P_{2}, P_{1} \in \meaningof{E_{1}}, P_{2} \in \meaningof{E_2}\} }
\end{mathpar}

\begin{mathpar}
 \inferrule* [lab=behavior] {} {\meaningof{\langle a?b \rangle E} = \{ P \in \pi | P \equiv Q | u?(y)P', \\ \and \\\\ \and \\ \;\;\; u \in \meaningof{a}, \forall z.P'\{z/y\} \in \meaningof{E\{z/b\}}\}, \and \\ \meaningof{a!E} = \{ P \in \pi | P \equiv Q | x!\langle P' \rangle, x \in \meaningof{a} P' \in \meaningof{E}\} }
\end{mathpar}

\begin{mathpar}
 \inferrule* [lab=nominal] {} {\meaningof{\quotep{E}} = \{ \quotep{P} \in \quotep{\pi} | P \in \meaningof{E} \}, \and \meaningof{\quotep{P}} = \{ \quotep{Q} \in \quotep{\pi} | P \equiv Q \} \and \\ \meaningof{@\quotep{E}} = \{ P \in \pi | P \equiv @x, x \in \meaningof{E} \}}
\end{mathpar}

\begin{eqnarray*}
  \\
  \meaningof{-} : TS \to ST
\end{eqnarray*}

\begin{eqnarray*}
  \\
  L : TS \to ST
\end{eqnarray*}

\begin{eqnarray*}
  \\
  P \models E \iff P \in \meaningof{E}
\end{eqnarray*}

\begin{eqnarray*}
  P \approx_{L} Q \iff \forall E \in L. P \models E \iff Q \models E
\end{eqnarray*}

\begin{eqnarray*}
  P \approx_{K} Q
\end{eqnarray*}

\begin{eqnarray*}
  P \approx Q
\end{eqnarray*}

$\approx_{K} = \approx = \approx_{L}$

\subsubsection{Contextual duality}

Note that contexts extend the quotation operation to a family of
operations from processes to names. Given a context, $M$, we can
define a \emph{nominal context}, $\quotep{M}$ by $\quotep{M}[P] :=
\quotep{M[P]}$. To foreshadow what is to come we observe that these
operations enjoy a duality with processes very much like the duality
between vectors and maps from vectors to scalars.

Further, because the calculus is essentially higher-order, we have a
correspondence between contexts and processes. More specifically,
given a name $x$ and a context $M$ we can construct $M^{*}_{x}$ such
that 

\begin{mathpar}
  M^{*}_{x} | \lift{x}{P} \red M[P]
\end{mathpar}

namely,

\begin{mathpar}
  M^{*}_{x} := x?(u).M[\dropn{u}]
\end{mathpar}

The dependence of $M^{*}_{x}$ on a name makes it an abstraction, 

\begin{mathpar}
  M^{*} := (x)x?(u).M[\dropn{u}]
\end{mathpar}

\subsection{Additional notation}

It will sometimes be convenient to denote the process a name
quotes. We already have the notation $x = \quotep{P}$, but it will be
convenient to introduce an alternate notation, $\procn{x}$, when we
want to emphasize the connection to the use of the name. Note that, by
virtue of name equivalence, $\quotep{\procn{x}} \nameeq x$; so, the
notation is consistent with previous definitions.

Further, because names have structure it is possible to effect
substitutions on the basis of that structure. This means we need to
upgrade our notation for substitutions, which we accomplish by
adapting comprehension notation. Thus,

\begin{mathpar}
  P\{ y / x : x \in S \}
\end{mathpar}

is interpreted to mean the process derived from P by replacing (in a
capture-avoiding manner) each occurrence of $x$ in $S$ by $y$. For example,

\begin{mathpar}
  P\{ \quotep{\procn{x}|\procn{x}} / x : x \in \freenames{P} \}
\end{mathpar}

will replace each (occurrence) of a free name $x$ in $P$ by
$\quotep{\procn{x}|\procn{x}}$.

Also, we will avail ourselves of the notation $x^{L}$ and $x^{R}$ to
denote injections of a name into disjoint copies of the name
space. There are numerous ways to accomplish this. One example can be
found in \cite{MeredithR05}. This notation overloads to vectors of
names: $\vec{x}^{\pi} := (x_{i}^{\pi} \; : \; 0 \leq i < |\vec{x}| )$ where $\pi \in \{L,R\}$.

We also use $P^{\Box} := P|\Box$.

In \cite{MeredithR05} an interpretation of the new operator is
given. It turns out that there are several possible interpretations
all enjoying the requisite algebraic properties of the operator (see
\cite{milner91polyadicpi}). We will therefore make liberal use of
$(\nu\; \vec{x})P$.

% subsection the_syntax_and_semantics_of_the_notation_system (end)   

\input{qm2pi.qmops} 

\input{qm2pi.sterngerlach} 

\input{qm2pi.metric} 

% section concurrent_process_calculi (end)

%\input{qm2pi.proofsketch}

% section proof sketch (end)

%\input{qm2pi.slviaknots} 

% section spatial logic via knots (end)

\input{qm2pi.conclusion}

% section conclusion (end)

%\input{qm2pi.dtcodes} 

% section wiring algorithm (end)

\input{qm2pi.ack} 

% section acknowledgments (end)

\newpage


\bibliographystyle{plain}   
\bibliography{../../biblios/main.bib}

\input{qm2pi.rhodetails}

\end{document}



\end{document}

 

% subsection basic_interpretation (end)

%\input{qm2pi.rho.presentation} 
\subsection{The syntax and semantics of the notation system}\label{sub:the_syntax_and_semantics_of_the_notation_system} % (fold)

We now summarize a technical presentation of the calculus that
embodies our theory of dynamics. The typical presentation of such a
calculus follows the style of giving generators and relations on
them. The grammar, below, describing term constructors, freely
generates the set of processes, $\Proc$. This set is then quotiented
by a relation known as structural congruence and it is over this set
that the notion of dynamics is expressed. This presentation is
essentially that of \cite{MeredithR05} with the addition of
polyadicity and summation. For readability we have relegated some of
the technical subtleties to an appendix.

\subsubsection{Process grammar}\label{subsub:process_grammar}

\begin{mathpar}
  \inferrule* [lab=synchronization] {} {{M} \bc \pzero \;|\; x?F \;|\; x!C }
  \and
  \inferrule* [lab=abstraction] {} {{F} \bc (x)P}
  \and
  \inferrule* [lab=concretion] {} {{C} \bc \langle Q \rangle}
  \and
  \inferrule* [lab=process] {} {{P,Q} \bc M \;| \;P|Q \;|\; @{x}}
  \and
  \inferrule* [lab=name] {} {{x} \bc \quotep{P}}
\end{mathpar} 

Note that $\vec{x}$ (resp. $\vec{P}$) denotes a vector of names
(resp. processes) of length $|\vec{x}|$ (resp. $|\vec{P}|$). We adopt
the following useful abbreviations.

\begin{mathpar}
   x?(\vec{y}).P := x.(\vec{y})P \and  x\clift{\vec{P}} := x.\clift{\vec{P}}
   \and x!(y) := \lift{x}{\dropn{y}}
   \and \Pi_{i=0}^{n-1}P_i := P_0 | \ldots | P_{n-1}
\end{mathpar}

\subsubsection{Structural congruence}

\paragraph{Free and bound names and alpha-equivalence.} At the
core of structural equivalence is alpha-equivalence which identifies
process that are the same up to a change of variable. Formally, we
recognize the distinction between free and bound names. The free names
of a process, $\freenames{P}$, may be calculated recursively as
follows:

\begin{mathpar}
\freenames{\pzero} := \emptyset
  \and \\
  \freenames{x?(y).P} := \{ x \} \cup (\freenames{P} \setminus \{ y \})
  \and 
  \freenames{x!\langle P \rangle} := \{ x \} \cup \{ P \} 
  \and \\
  \freenames{P|Q} := \freenames{P} \cup \freenames{Q}
  \and \\
  \freenames{@{x}} := \{ x \}
\end{mathpar}

$\pi$
$\quotep{\pi}$

$\freenames{-} : \pi \to \mathcal{P}(\quotep{\pi})$

\begin{eqnarray*}
  \freenames{\pzero} & := & \emptyset \\
  \freenames{x?(y).P} & := & \{ x \} \cup (\freenames{P} \setminus \{ y \}) \\
  \freenames{x!\langle P \rangle} & := & \{ x \} \cup \{ P \} \\
  \freenames{P|Q} & := & \freenames{P} \cup \freenames{Q} \\
  \freenames{\dropn{x}} & := & \{ x \}
\end{eqnarray*}

The bound names of a process, $\boundnames{P}$, are those names occurring in $P$
that are not free. For example, in $x?(y).0$, the name $x$ is free, while $y$ is bound.

\begin{mathpar}
  \inferrule* [lab=monoidal-laws] {} { P|Q \equiv Q|P \and P|0 \equiv P \and P|(Q|R) \equiv (P|Q)|R }
\end{mathpar}

\begin{mathpar}
  \inferrule* [lab=alpha-equivalence] {} { (x)P \equiv (y)P\{y/x\} \and y \not\in \freenames{P} }
\end{mathpar}

\begin{definition}
Then two processes, $P,Q$, are alpha-equivalent if $P = Q\{\vec{y}/\vec{x}\}$ for
some $\vec{x} \in \boundnames{Q},\vec{y} \in \boundnames{P}$, where $Q\{\vec{y}/\vec{x}\}$
denotes the capture-avoiding substitution of $\vec{y}$ for $\vec{x}$ in $Q$.
\end{definition}

\begin{definition}
  The {\em structural congruence} \cite{SangiorgiWalker} , $\equiv$,
  between processes is the least congruence containing
  alpha-equivalence, satisfying the abelian monoid laws
  (associativity, commutativity and $\pzero$ as identity) for parallel
  composition $|$ and for summation $+$.
\end{definition}

\subsection{Name equivalence}

We take name equivalence, written $\nameeq$, to be the smallest
equivalence relation generated by the following rules.

\begin{mathpar}
\inferrule*[lab=Quote-drop]
{ }
{ \quotep{@{x}} \nameeq x }

\inferrule*[lab=Struct-equiv]
{ P \scong Q }
{ \quotep{P} \nameeq \quotep{Q} }
\end{mathpar}

The astute reader will have noticed that the mutual recursion of names
and processes imposes a mutual recursion on alpha-equivalence and
structural equivalence via name-equivalence. Fortunately, all of this
works out pleasantly and we may calculate in the natural way, free of
concern. The reader interested in the details is referred to the
appendix \ref{appendix:rho_details}.

\subsection{Substitution}

We use $\Proc$ for the set of processes, $\QProc$ for the set of
names, and $\id{\{}\vec{y} / \vec{x} \id{\}}$ to denote partial maps,
$s : \QProc \rightarrow \QProc$. A map, $s$ lifts, uniquely, to a map
on process terms, $\widehat{s} : \Proc \rightarrow \Proc$ by the
following equations.

\begin{mathpar}
  (0) \psubstp{Q}{P} := 0 \\
  (R \juxtap S) \psubstp{Q}{P}
  :=    
  (R)\psubstp{Q}{P} \juxtap (S) \psubstp{Q}{P} \\
  (x?(y).R) \psubstp{Q}{P}    
  :=    
  (x)\substp{Q}{P} (z)\concat( (R \psubstn{z}{y}) \psubstp{Q}{P} ) \\
  (\lift{x}{R}) \psubstp{Q}{P}  
  :=
  \lift{(x)\substp{Q}{P}}{ R \psubstp{Q}{P} } \\
%   (\dropn{x})  \psubstp{Q}{P}       
%   := 
%   \left\{ 
%     \begin{array}{ccc} 
%       \dropn{\quotep{Q}} & & x \nameeq \quotep{P} \\
%       \dropn{x} & & otherwise \\
%     \end{array}
%   \right. 
  (\dropn{x})  \psubstp{Q}{P}       
  := 
  \left\{ 
    \begin{array}{ccc} 
      Q & & x \nameeq \quotep{P} \\
      \dropn{x} & & otherwise \\
    \end{array}
  \right.
\end{mathpar}
 

where

\begin{eqnarray}
  (x)\id{\{} \lpquote Q \rpquote / \lpquote P \rpquote \id{\}}            = 
  \left\{ 
    \begin{array}{ccc}
      \lpquote Q \rpquote & & x \nameeq \lpquote P \rpquote \\
      x & & otherwise \\
    \end{array}
  \right. \nonumber
\end{eqnarray}

and $z$ is chosen distinct from $\quotep{P}$, $\quotep{Q}$, the free
names in $Q$, and all the names in $R$. Our $\alpha$-equivalence will
be built in the standard way from this substitution.

\begin{remark}\label{rem:no_self_referential_names}
  One consequence of these definitions is that $\forall P. \quotep{P}
  \not\in \freenames{P}$.
\end{remark}

\subsection{ Dynamic quote: an example }

Anticipating something of what's to come, consider applying the
substitution, $\widehat{\id{\{}u / z \id{\}}}$, to the following pair
of processes, $\lift{w}{y!(z)}$ and $w[ \lpquote y!(z) \rpquote ]$.

\begin{eqnarray}
	\lift{w}{y!(z)}\widehat{\id{\{}u / z \id{\}}}
		& = &
		\lift{w}{y!(u)} \nonumber\\
	w[ \lpquote y!(z) \rpquote ] \widehat{ \id{\{}u / z \id{\}} }
		& = &
		w[ \lpquote y!(z) \rpquote ] \nonumber
\end{eqnarray}

Because the body of the process between quotes is impervious to
substitution, we get radically different answers. In fact, by
examining the first process in an input context,
e.g. $x?(z).\lift{w}{y!(z)}$, we see that the process under the lift
operator may be shaped by prefixed inputs binding a name inside it. In
this sense, the lift operator will be seen as a way to dynamically
construct processes before reifying them as names.

Finally equipped with these standard features we can present the
dynamics of the calculus.

\subsubsection{Operational semantics} 

Finally, we introduce the computational dynamics. What marks these
algebras as distinct from other more traditionally studied algebraic
structures, e.g. vector spaces or polynomial rings, is the manner in
which dynamics is captured. In traditional structures, dynamics is typically
expressed through morphisms between such structures, as in linear maps
between vector spaces or morphisms between rings. In algebras
associated with the semantics of computation, the dynamics is
expressed as part of the algebraic structure itself, through a
reduction reduction relation typically denoted by $\red$. Below, we
give a recursive presentation of this relation for the calculus used
in the encoding.

$\red \subseteq \pi \times \pi$
$\red : \pi \to \mathcal{P}(\pi)$

\begin{mathpar}
  \inferrule* [lab=Comm] { \textsf{match}( x_{src}, x_{trgt} ) } { x_{trgt}?(y)P \; | \; x_{src}!\langle {Q} \rangle \red P\{\quotep{Q}/y}\} }
  \and \\
  \inferrule* [lab=Par] {{P} \red {P}'} {{{P} | {Q}} \red {{P}' | {Q}}}
  \and
  \inferrule* [lab=Equiv]{{{P} \scong {P}'} \andalso {{P}' \red {Q}'} \andalso {{Q}' \scong {Q}}}{{P} \red {Q}}
\end{mathpar}

\begin{eqnarray*}
  match_{\equiv} (\quotep{P},\quotep{Q}) & := & P \equiv Q \\
  match_{\dagger}(\quotep{P},\quotep{Q}) & := & \forall R. P|Q \red^{*} R => R \red^{*} 0 \\
  match_{K}(\quotep{P},\quotep{Q}) & := & K \mbox{ for some context } K
\end{eqnarray*}

$u?(x)P | u!\langle Q \rangle \red P\{\quotep{Q}/x\}$

%We write $\wred$ for $\red^*$, and $P\red$ if $\exists Q $ such that $ P \red Q$.
We write $P\red$ if $\exists Q $ such that $ P \red Q$ and $P\not\red$, otherwise.

\section{Replication}

As mentioned before, it is known that replication (and hence
recursion) can be implemented in a higher-order process algebra
\cite{SangiorgiWalker}. As our first example of calculation with the
machinery thus far presented we give the construction explicitly in
the {\rhoc}.

\begin{eqnarray}
	D_{x} & := & \prefix{x}{y}{(\binpar{\outputp{x}{y}}{@{y}})} \nonumber\\
	\bangp_{x}{P} & := & \binpar{{x}!\langle{\binpar{D_{x}}{P}}\rangle}{D_{x}} \nonumber
\end{eqnarray}

\begin{eqnarray}
	\bangp_{x}{P} & & \nonumber\\
	=
	& {x}!\langle{(\prefix{x}{y}{(\outputp{x}{y} | @{y})) | P}}\rangle 
	      | \prefix{x}{y}{(\outputp{x}{y} | @{y})} & \nonumber\\
	\red
	& (\outputp{x}{y} | @{y})\substn{\quotep{(\prefix{x}{y}{(@{y} | \outputp{x}{y})) | P}}}{y} & \nonumber\\
	=
	& \outputp{x}{\quotep{(\prefix{x}{y}{(\outputp{x}{y} | @{y})) | P}}}
	  | {(\prefix{x}{y}{(\outputp{x}{y} | @{y})) | P}} & \nonumber\\
	\red
	& \ldots & \nonumber\\
	\red^*
	& P | P | \ldots & \nonumber
\end{eqnarray}

Of course, this encoding, as an implementation, runs away, unfolding
$\bangp{P}$ eagerly. A lazier and more implementable replication
operator, restricted to input-guarded processes, may be obtained as follows.

\begin{eqnarray}
\bangp{\prefix{u}{v}{P}} 
	:= 
	\binpar{\lift{x}{\prefix{u}{v}{(\binpar{D(x)}{P})}}}{D(x)} \nonumber
\end{eqnarray}

\begin{remark}
  Note that the lazier definition still does not deal with summation
  or mixed summation (i.e. sums over input and output). The reader is
  invited to construct definitions of replication that deal with these
  features. 

  Further, the definitions are parameterized in a name, $x$. Can you,
  gentle reader, make a definition that eliminates this parameter and
  guarantees no accidental interaction between the replication
  machinery and the process being replicated -- i.e. no accidental
  sharing of names used by the process to get its work done and the
  name(s) used by the replication to effect copying. This latter
  revision of the definition of replication is crucial to obtaining
  the expected identity $!!P \sim !P$.
\end{remark}

\begin{remark}\label{rem:paradoxical_combinator}
  The reader familiar with the lambda calculus will have noticed the
  similarity between $D$ and the paradoxical combinator.

  [Ed. note: the existence of this seems to suggest we have to be more
  restrictive on the set of processes and names we admit if we are to
  support no-cloning.]
\end{remark}

\subsubsection{Bisimulation}

The computational dynamics gives rise to another kind of equivalence,
the equivalence of computational behavior. As previously mentioned
this is typically captured \emph{via} some form of bisimulation.

% The notion we use in this paper is weak barbed bisimulation
% \cite{milner91polyadicpi}.

The notion we use in this paper is derived from weak barbed
bisimulation \cite{milner91polyadicpi}. 

\begin{definition}
An \emph{observation relation}, $\downarrow_{\mathcal N}$, over a set
of names, $\mathcal N$, is the smallest relation satisfying the rules
below.

\infrule[Out-barb]{y \in {\mathcal N}, \; x \nameeq y}
		  {\outputp{x}{v} \downarrow_{\mathcal N} x}
\infrule[Par-barb]{\mbox{$P\downarrow_{\mathcal N} x$ or $Q\downarrow_{\mathcal N} x$}}
		  {\binpar{P}{Q} \downarrow_{\mathcal N} x}

We write $P \Downarrow_{\mathcal N} x$ if there is $Q$ such that 
$P \wred Q$ and $Q \downarrow_{\mathcal N} x$.
\end{definition}

\begin{definition}
%\label{def.bbisim}
An  ${\mathcal N}$-\emph{barbed bisimulation} over a set of names, ${\mathcal N}$, is a symmetric binary relation 
${\mathcal S}_{\mathcal N}$ between agents such that $P\rel{S}_{\mathcal N}Q$ implies:
\begin{enumerate}
\item If $P \red P'$ then $Q \wred Q'$ and $P'\rel{S}_{\mathcal N} Q'$.
\item If $P\downarrow_{\mathcal N} x$, then $Q\Downarrow_{\mathcal N} x$.
\end{enumerate}
$P$ is ${\mathcal N}$-barbed bisimilar to $Q$, written
$P \wbbisim_{\mathcal N} Q$, if $P \rel{S}_{\mathcal N} Q$ for some ${\mathcal N}$-barbed bisimulation ${\mathcal S}_{\mathcal N}$.
\end{definition}

$\mathcal{R} \subseteq \pi \times \pi$

$P \mathcal{R} Q => \forall P'. P \red P' \Rightarrow \exists Q'. Q \red Q', P' \mathcal{R} Q'$

$P \vdash x \Rightarrow Q \vdash x$

\begin{mathpar}
  \inferrule*[lab=Out-barb]{x \nameeq y}{{y}!\langle{Q}\rangle \vdash x}
  \and
  \inferrule*[lab=Par-barb]{\mbox{$P\vdash x$ or $Q\vdash x$}}{\binpar{P}{Q} \vdash x}
\end{mathpar}

\subsubsection{Contexts}

One of the principle advantages of computational calculi like the
$\pi$-calculus is a well-defined notion of context,
contextual-equivalence and a correlation between
contextual-equivalence and notions of bisimulation. The notion of
context allows the decomposition of a process into (sub-)process and
its syntactic environment, its context. Thus, a context may be
thought of as a process with a ``hole'' (written $\Box$) in it. The
application of a context $M$ to a process $P$, written $M[P]$, is
tantamount to filling the hole in $M$ with $P$. In this paper we do
not need the full weight of this theory, but do make use of the notion
of context in the proof the main theorem. 

\begin{mathpar}
  \inferrule* [lab=summation] {} {{M_{M},M_{N}} \bc \Box \;|\; x.M_{A} \;|\; M_{M}+M_{N}}
  \and
  \inferrule* [lab=agent] {} {{M_{A}} \bc (\vec{x})M_{P} \;| \; \clift{P_0,\ldots,M_{P},\ldots,P_N}}
  \and \\
  \inferrule* [lab=process] {} {{M_{P}} \bc M_{N} \;| \;P|M_{P} }
\end{mathpar} 

\begin{mathpar}
  \inferrule* [lab=sychronization] {} {M_{N} \bc \Box \;|\; x?M_{F} \;|\; x!M_{C}}
  \and
  \inferrule* [lab=abstraction] {} {{M_{F}} \bc (x)M_{P} }
  \and
  \inferrule* [lab=concretion] {} {{M_{C}} \bc \langle M_{P} \rangle }
  \and \\
  \inferrule* [lab=process] {} {{M_{P}} \bc M_{N} \;| \;P|M_{P} }
\end{mathpar}

\begin{definition}[contextual application] Given a context $M$, and
  process $P$, we define the \emph{contextual application}, $M[P] :=
  M\{P/\Box\}$. That is, the contextual application of M to P is the
  substitution of $P$ for $\Box$ in $M$.
\end{definition}

$\meaningof{-} : L \to \mathcal{P}(\pi)$

\begin{mathpar}
  \inferrule* [lab=collection] {} {\meaningof{true} = \pi, \and \meaningof{~E} = \pi \setminus \meaningof{E}, \and \meaningof{E_{1} \& E_{2}} = \meaningof{E_{1}} \cap \meaningof{E_{2}}}
\end{mathpar}

\begin{mathpar}
  \inferrule* [lab=structure] {} {\meaningof{0} = \{ P \in \pi | P \equiv 0 \}, \and \\ \meaningof{E_1 | E_2} = \{ P \in \pi | P \equiv P_{1} | P_{2}, P_{1} \in \meaningof{E_{1}}, P_{2} \in \meaningof{E_2}\} }
\end{mathpar}

\begin{mathpar}
 \inferrule* [lab=behavior] {} {\meaningof{\langle a?b \rangle E} = \{ P \in \pi | P \equiv Q | u?(y)P', \\ \and \\\\ \and \\ \;\;\; u \in \meaningof{a}, \forall z.P'\{z/y\} \in \meaningof{E\{z/b\}}\}, \and \\ \meaningof{a!E} = \{ P \in \pi | P \equiv Q | x!\langle P' \rangle, x \in \meaningof{a} P' \in \meaningof{E}\} }
\end{mathpar}

\begin{mathpar}
 \inferrule* [lab=nominal] {} {\meaningof{\quotep{E}} = \{ \quotep{P} \in \quotep{\pi} | P \in \meaningof{E} \}, \and \meaningof{\quotep{P}} = \{ \quotep{Q} \in \quotep{\pi} | P \equiv Q \} \and \\ \meaningof{@\quotep{E}} = \{ P \in \pi | P \equiv @x, x \in \meaningof{E} \}}
\end{mathpar}

\begin{eqnarray*}
  \\
  \meaningof{-} : TS \to ST
\end{eqnarray*}

\begin{eqnarray*}
  \\
  L : TS \to ST
\end{eqnarray*}

\begin{eqnarray*}
  \\
  P \models E \iff P \in \meaningof{E}
\end{eqnarray*}

\begin{eqnarray*}
  P \approx_{L} Q \iff \forall E \in L. P \models E \iff Q \models E
\end{eqnarray*}

\begin{eqnarray*}
  P \approx_{K} Q
\end{eqnarray*}

\begin{eqnarray*}
  P \approx Q
\end{eqnarray*}

$\approx_{K} = \approx = \approx_{L}$

\subsubsection{Contextual duality}

Note that contexts extend the quotation operation to a family of
operations from processes to names. Given a context, $M$, we can
define a \emph{nominal context}, $\quotep{M}$ by $\quotep{M}[P] :=
\quotep{M[P]}$. To foreshadow what is to come we observe that these
operations enjoy a duality with processes very much like the duality
between vectors and maps from vectors to scalars.

Further, because the calculus is essentially higher-order, we have a
correspondence between contexts and processes. More specifically,
given a name $x$ and a context $M$ we can construct $M^{*}_{x}$ such
that 

\begin{mathpar}
  M^{*}_{x} | \lift{x}{P} \red M[P]
\end{mathpar}

namely,

\begin{mathpar}
  M^{*}_{x} := x?(u).M[\dropn{u}]
\end{mathpar}

The dependence of $M^{*}_{x}$ on a name makes it an abstraction, 

\begin{mathpar}
  M^{*} := (x)x?(u).M[\dropn{u}]
\end{mathpar}

\subsection{Additional notation}

It will sometimes be convenient to denote the process a name
quotes. We already have the notation $x = \quotep{P}$, but it will be
convenient to introduce an alternate notation, $\procn{x}$, when we
want to emphasize the connection to the use of the name. Note that, by
virtue of name equivalence, $\quotep{\procn{x}} \nameeq x$; so, the
notation is consistent with previous definitions.

Further, because names have structure it is possible to effect
substitutions on the basis of that structure. This means we need to
upgrade our notation for substitutions, which we accomplish by
adapting comprehension notation. Thus,

\begin{mathpar}
  P\{ y / x : x \in S \}
\end{mathpar}

is interpreted to mean the process derived from P by replacing (in a
capture-avoiding manner) each occurrence of $x$ in $S$ by $y$. For example,

\begin{mathpar}
  P\{ \quotep{\procn{x}|\procn{x}} / x : x \in \freenames{P} \}
\end{mathpar}

will replace each (occurrence) of a free name $x$ in $P$ by
$\quotep{\procn{x}|\procn{x}}$.

Also, we will avail ourselves of the notation $x^{L}$ and $x^{R}$ to
denote injections of a name into disjoint copies of the name
space. There are numerous ways to accomplish this. One example can be
found in \cite{MeredithR05}. This notation overloads to vectors of
names: $\vec{x}^{\pi} := (x_{i}^{\pi} \; : \; 0 \leq i < |\vec{x}| )$ where $\pi \in \{L,R\}$.

We also use $P^{\Box} := P|\Box$.

In \cite{MeredithR05} an interpretation of the new operator is
given. It turns out that there are several possible interpretations
all enjoying the requisite algebraic properties of the operator (see
\cite{milner91polyadicpi}). We will therefore make liberal use of
$(\nu\; \vec{x})P$.

% subsection the_syntax_and_semantics_of_the_notation_system (end)   

\section{Interpretation of QM}
\subsection{Supporting definitions}
\subsubsection{Multiplication}
\begin{mathpar}
  \quotep{Q} \cdot \quotep{R} := \quotep{Q|R}
  \and \\
  \quotep{Q} \cdot P := P\{ \quotep{Q|R} / \quotep{R} : \quotep{R} \in \freenames{P} \}
\end{mathpar}

\paragraph{Discussion}
The first line needs little explanation. The second line says that
each free name of the process is replaced with the multiplication of
that name by the scalar. Multiplication of a scalar (name) by a state
(process) results in a process all the names of which have been `moved
over' by parallel composition with the process the scalar
quotes. There is a subtlety that the bound names have to be
manipulated so that multiplied names aren't accidentally
captured. There are many ways to achieve this.

\begin{remark}\label{rem:multiplication_identities}
  The reader is invited to verify that for all $x,y,z \in \QProc$ and $P \in \Proc$
  \begin{mathpar}
    x \cdot \quotep{0} \equiv x 
    \and
    x \cdot y \equiv y \cdot x
    \and
    x \cdot (y \cdot z) \equiv (x \cdot y) \cdot z
    \and \\
    \quotep{0} \cdot P \equiv P
    \and \\
    x \cdot (y \cdot P) \equiv (x \cdot y) \cdot P
    \and \\
    x \cdot (P|Q) \equiv (x \cdot P) | (x \cdot Q)
    \and \\    
  \end{mathpar}
\end{remark}

\subsubsection{Tensor product}

We define a tensor product on processes by structural induction.

\paragraph{Tensor of sums} First note that all summations, including
$\pzero$ and sequence, can be written $\Sigma_{i} x_{i}.A_{i} +
\Sigma_{j} x_{j}.C_{j}$, where we have grouped input-guarded processes
together and output-guarded processes together.

Thus, we can define the tensor product of two summations, $N_{1}\otimes N_{2}$, where

\begin{mathpar}
  N_{1} := \Sigma_{i} x_{i}.A_{i} + \Sigma_{j} x_{j}.C_{j}
  \and
  N_{2} := \Sigma_{i'} y_{i'}.B_{i'} + \Sigma_{j'} y_{j'}.D_{j'} 
\end{mathpar}

as follows.

\begin{mathpar}
  \Sigma_{i} x_{i}.A_{i} + \Sigma_{j} x_{j}.C_{j} \otimes \Sigma_{i'}
  y_{i'}.B_{i'} + \Sigma_{j'} y_{j'}.D_{j'} 
  \and \\
  := \; \Sigma_{i} \Sigma_{i'} \quotep{\stackrel{\vee}{x_{i}}| \stackrel{\vee}{y_{i'}}}.(A_{i}\otimes B_{i'}) \; | \; \Sigma_{i'} \Sigma_{i} \quotep{\stackrel{\vee}{y_{i'}}|\stackrel{\vee}{x_{i}}}.(B_{i'}\otimes A_{i})
  \and
  \;\; | \;\; \Sigma_{j} \Sigma_{j'} \quotep{\stackrel{\vee}{x_{j}}|\stackrel{\vee}{y_{j'}}}.(A_{j}\otimes B_{j'}) \; | \; \Sigma_{j'} \Sigma_{j} \quotep{\stackrel{\vee}{y_{j'}}|\stackrel{\vee}{x_{j}}}.(B_{j'}\otimes A_{j})
\end{mathpar}

\begin{remark}
  Do we need to $x^{L}$ and $y^{R}$ for this construction as well?
\end{remark}

\paragraph{Tensor of parallel compositions} Next, we distribute tensor
over par.

\begin{mathpar}
  P_{1}|P_{2} \otimes Q_{1}|Q_{2} := (P_{1} \otimes Q_{1}) | (P_{1}
  \otimes Q_{2}) | (P_{2} \otimes Q_{1}) | (P_{2} \otimes Q_{2})
\end{mathpar}

\paragraph{Tensor with dropped names} We treat tensor of a
process with a dropped name as parallel composition.

\begin{mathpar}
  P \otimes \dropn{x} := P | \dropn{x}
\end{mathpar}

\paragraph{Tensor of agents}

Finally, we need to define tensor on agents. Note that the definition
of tensor on normal products only tensors inputs with inputs and
outputs with outputs. Thus, we only have to define the operation on
``homogeneous'' pairings.

\begin{mathpar}
  (\vec{x})P \otimes (\vec{y})Q
  \and \\
  := (x_{0}^{L}|y_{0}^{R},\ldots,x_{0}^{L}|y_{n}^{R},\ldots,x_{m}^{L}|y_{0}^{R},\ldots,x_{m}^{L}|y_{n}^R)(P\{ \vec{x}^{L}/\vec{x}\} \otimes Q \{ \vec{y}^{R}/\vec{y}\})
  \and \\
  \clift{\vec{P}} \otimes \clift{\vec{Q}}
  \and \\
  := \clift{P_{0}\otimes Q_{0},\ldots,P_{0}\otimes Q_{n},\ldots,P_{m}\otimes Q_{0},\ldots,P_{m}\otimes Q_{n}}
\end{mathpar}

\begin{remark}
  Observe that arities of tensored abstractions matches arities of
  tensored concretions if the original arities matched. Note also that
  the length of the arities corresponds to the increase in dimension
  we see in ordinary vector space tensor product.
\end{remark}

\begin{remark}
  Operationally, this definition distributes the tensor down to
  components ``linked'' by summation. Tensor over summation is
  intriguing in that it mixes names. Moreover, as a consequence of the
  way it mixes names we have the identities for all $x \in \QProc$ and
  $P,Q \in \Proc$

  \begin{mathpar}
    (x \cdot P) \otimes Q \equiv x \cdot (P \otimes Q) \equiv P \otimes (x \cdot Q)
    \and
    P \otimes \pzero \equiv P
  \end{mathpar}

  that the reader is invited to verify.
\end{remark}

\subsubsection{Annihilation}
\begin{mathpar}
  P^{\perp} := \{ Q | \forall R. P|Q \red^{*} R \Rightarrow R \red^{*} \pzero \}
  \and \\
  P^{\underline{\perp}} := \Sigma_{Q \in P^{\perp}} \quotep{Q}?(y).(\dropn{y}|Q) | \Sigma_{Q \in P^{\perp}} \quotep{Q}\clift{\Box}
\end{mathpar}

\paragraph{Discussion} The reader will note that $P^{\perp}$ is a
\emph{set} of processes, while $P^{\underline{\perp}}$ is a
\emph{context}. We call the set $P^{\perp}$ the \emph{annihilators} of
$P$. The parallel composition of a process in the annihilators of $P$
with $P$ will result in a process, the state space of which has all
paths eventually leading to $\pzero$. Execution may endure loops; but
under reasonable conditions of fairness (naturally guaranteed under
most notions of bisimulation) such a composite process cannot get
stuck in such a loop and will, eventually pop out and terminate.

The context $P^{\underline{\perp}}$ is ready and willing to ``take the
$P$ out of'' the process to which it is applied. It will effectively
transmit the code of the process to which it is applied to one of the
annihilators and run the process against it.

\subsubsection{Evaluation}
We fix $M$ a domain of fully abstract interpretation with an equality
coincident with bisimulation. We take $\meaningof{\cdot} : \Proc \to
M$ to be the map interpreting processes and $\nmeaningof{\cdot} : \M
\to Proc$ to be the map running the other way. Then we define

\begin{mathpar}
  \int P := \nmeaningof{\meaningof{P}}
\end{mathpar}

\paragraph{Discussion}
There are many fully abstract interpretations of Milner's
$\pi$-calculus. Any of them can be used as a basis for interpreting
the reflective calculus here. Equipped with such a domain it is
largely a matter of grinding through to check that the Yoneda
construction for the normalization-by-evaluation program can be
extended to this setting.

\begin{remark}
  The reader is invited to verify that $\int (P^{\underline{\perp}}[P]) = 0$.
\end{remark}

\subsection{Quantum mechanics}

Table \ref{tbl:core_qm_op_defns} gives the core operational definitions

\begin{table}[htp]\label{tbl:core_qm_op_defns}
  \center{
    \fbox{
      \begin{tabular}{c|c}
        quantum mechanics & process calculus \\
        \hline
        scalar & $x := \quotep{P}$ \\
        state vector & $\state{P} := P$ \\
        dual & $\state{P}^{*} := \event{P^{\underline{\perp}}} := \quotep{P^{\underline{\perp}}}[-]$ \\
        matrix & $ \Sigma_{\alpha} \state{P_{\alpha}}x_{\alpha}\event{Q_{\alpha}}$ \\
        vector addition & $\state{P} + \state{Q} := \state{P | Q}$ \\
        tensor product & $\state{P} \otimes \state{Q} := \state{P \otimes Q}$ \\
        inner product & $\innerprod{P}{Q} := \quotep{\int P^{\underline{\perp}}[Q]}$ \\
      \end{tabular}
    }
  }
  \caption{QM - operational definitions}
\end{table}

where

\begin{mathpar}
  \prmatrix{P}{Q} := \fprmatrix{P}{\quotep{\pzero}}{Q}
  \and
  \fprmatrix{P}{x}{Q} := (\state{P},x,\event{Q})
  \and
  (\fprmatrix{P}{x}{Q})(\state{R}) := x \cdot \innerprod{Q}{R} \cdot \state{P}
  \and
  (\fprmatrix{P}{x}{Q})(\event{R}) := x \cdot \innerprod{R}{P} \cdot \event{Q}
\end{mathpar}

\paragraph{Discussion}
As promised: vectors (aka states) are represented as processes; duals
as contextual duals; inner product definition should be compared with
standard inner product definition for ....

\begin{remark}
  Assuming $\int (P^{\underline{\perp}}[P]) = 0$, the reader is
  invited to verify that $(\fprmatrix{P}{x}{P})(\state{P}) = x \cdot \state{P}$.
\end{remark}

\begin{remark}
  The reader is invited to verify that $\innerprod{P}{Q}$ could
  equally well have been written $\quotep{\int \stackrel{\vee}{x}}$
  where $x = \event{P^{\underline{\perp}}}(Q)$.

  One of the motivations for this remark is that there is another way
  to factor these operations. We could package up evaluation in the dual:

  \begin{mathpar}
    \state{P}^{*} := \event{\int P^{\underline{\perp}}} := \quotep{\int P^{\underline{\perp}}}[-]
  \end{mathpar}

  and then have inner product defined by
  
  \begin{mathpar}
    \innerprod{P}{Q} := \event{P}(Q)
  \end{mathpar}

  Hopefully, experience with the calculations will provide guidance on
  the best factoring.
\end{remark}

\begin{remark}
  Assuming $\int (P^{\underline{\perp}}[P]) = 0$, the reader is
  invited to verify that $\forall P,Q. (\prmatrix{0}{Q})(\state{0}) =
  \state{0}$ and dually $(\prmatrix{P}{0})(\event{0}) = \event{0}$.
\end{remark}

\begin{remark}
  i'm a little worried that i don't (yet) have proper support for
  complex conjugacy. But, the observation above may give us a
  clue. According to Abramsky, it must be the case that the scalars
  are iso to the homset of the identity for the tensor -- which the
  observation above characterizes. 

  For now, we will simply bookmark the notion with $\overline{x}$.
\end{remark}

\subsubsection{Adjointness}

We need to give a definition of $(\cdot)^{\dagger}$ for matrices. The
obvious candidate definition is
\begin{mathpar}
(\Sigma_{\alpha}\fprmatrix{P_{\alpha}}{x_{\alpha}}{Q_{\alpha}})^{\dagger}
= \Sigma_{\alpha}\fprmatrix{(Q_{\alpha}^{\underline{\perp}})^{*}}{\overline{x}_{\alpha}}{P_{\alpha}^{\underline{\perp}}} 
\end{mathpar}

But, $(Q_{\alpha}^{\underline{\perp}})^{*}$ requires a name along
which to communicate the process to achieve the context application.

\subsubsection{Basis for a basis}
If processes label states and ``addition'' of states (a.k.a. vector
addition) is interpreted as parallel composition, what corresponds to
notions of linear independence and basis? Here, we recall that Yoshida
has developed a set of \emph{combinators} for an asynchronous verison
of Milner's $\pi$-calculus. These are a finite set of processes such
any process can be expressed as parallel composition of these
combinators together with liberal uses of the new operator and
replication. We can simply give a translation of these into the
present calculus and have reasonable expectation that the property
carries over. That is, that the resultant set allows to express all
processes via parallel composition. Note, however, that there is no
new operator or replication in this calculus. As a result, we expect
that the corresponding set is actually infinite. That is, we expect
that the space is actually infinite dimensional.

\begin{remark}
  The attentive reader may be a bit concerned. Certainly, the
  collection $S$, $K$ and $I$ is a finite set of
  combinators. Shouldn't we expect to see a finite set of combinators
  for an effectively equivalent system? i am very sympathetic to this
  critique and feel it warrants full attention. On the other hand, i
  also have in mind the following analogy. The natural numbers, as a
  monoid under addition, has exactly $1$ generator, while the natural
  numbers, as a monoid under multiplication, has countably many
  generators (the primes). We observe that the application of the
  lambda calculus is much less resource sensitive than the parallel
  composition of the $\pi$-calculus. Could it be the case that we have
  an analogy of the form
  
  \begin{mathpar}
    m + n : MN :: m*n : M|N
  \end{mathpar}

  giving a similar blow up in the set of ``primes''?  This is such a
  wonderful thought that, even if it's not true, i think it's worth
  writing down.
\end{remark}
 

\documentclass[12pt]{llncs}
%\documentclass{jktr}

\usepackage[pdftex]{hyperref}                   
\usepackage {listings}
\usepackage {mathpartir}
\usepackage{bcprules}
%\usepackage{listings}
                       
\usepackage{graphicx} 
%\usepackage[margins=2.5cm,nohead,nofoot]{geometry}
%\usepackage{geometry}
\usepackage{amsfonts}
\usepackage{amstext}
\usepackage{latexsym}
\usepackage{amssymb}
\usepackage{color}


%\include{myPreamble}
\documentclass[12pt]{llncs}
%\documentclass{jktr}

\usepackage[pdftex]{hyperref}                   
\usepackage {listings}
\usepackage {mathpartir}
\usepackage{bcprules}
%\usepackage{listings}
                       
\usepackage{graphicx} 
%\usepackage[margins=2.5cm,nohead,nofoot]{geometry}
%\usepackage{geometry}
\usepackage{amsfonts}
\usepackage{amstext}
\usepackage{latexsym}
\usepackage{amssymb}
\usepackage{color}


%\include{myPreamble}
\include{qm2pi.local} 

%\ifpdf
%\usepackage[pdftex]{graphicx}
%\else
%\usepackage{graphicx}
%\fi

 % \ifpdf
%  \usepackage{pdfsync}
%  \if


%\title{Brief Article}
%\author{David F. Snyder}
%\author{L.G. Meredith}

%\address{Dept. of Math., Texas State University--San Marcos, San Marcos, TX 78666}
       
\pagestyle{empty}


\begin{document}

\lstset{language=[Objective]Caml,frame=shadowbox}

\input{qm2pi.front}

% section front matter (end)

\input{qm2pi.intro} 
 
% section introduction (end)

% \input{qm2pi.knotations} 

% section notation (end)

\input{qm2pi.process.calculi} 

% section concurrent_process_calculi_and_spatial_logics_ (end)
    
%\input{qm2pi.knots2pi} 

%\input{qm2pi.trefoil} 

%\input{qm2pi.mainthm} 

% subsection basic_interpretation (end)

%\input{qm2pi.rho.presentation} 
\subsection{The syntax and semantics of the notation system}\label{sub:the_syntax_and_semantics_of_the_notation_system} % (fold)

We now summarize a technical presentation of the calculus that
embodies our theory of dynamics. The typical presentation of such a
calculus follows the style of giving generators and relations on
them. The grammar, below, describing term constructors, freely
generates the set of processes, $\Proc$. This set is then quotiented
by a relation known as structural congruence and it is over this set
that the notion of dynamics is expressed. This presentation is
essentially that of \cite{MeredithR05} with the addition of
polyadicity and summation. For readability we have relegated some of
the technical subtleties to an appendix.

\subsubsection{Process grammar}\label{subsub:process_grammar}

\begin{mathpar}
  \inferrule* [lab=synchronization] {} {{M} \bc \pzero \;|\; x?F \;|\; x!C }
  \and
  \inferrule* [lab=abstraction] {} {{F} \bc (x)P}
  \and
  \inferrule* [lab=concretion] {} {{C} \bc \langle Q \rangle}
  \and
  \inferrule* [lab=process] {} {{P,Q} \bc M \;| \;P|Q \;|\; @{x}}
  \and
  \inferrule* [lab=name] {} {{x} \bc \quotep{P}}
\end{mathpar} 

Note that $\vec{x}$ (resp. $\vec{P}$) denotes a vector of names
(resp. processes) of length $|\vec{x}|$ (resp. $|\vec{P}|$). We adopt
the following useful abbreviations.

\begin{mathpar}
   x?(\vec{y}).P := x.(\vec{y})P \and  x\clift{\vec{P}} := x.\clift{\vec{P}}
   \and x!(y) := \lift{x}{\dropn{y}}
   \and \Pi_{i=0}^{n-1}P_i := P_0 | \ldots | P_{n-1}
\end{mathpar}

\subsubsection{Structural congruence}

\paragraph{Free and bound names and alpha-equivalence.} At the
core of structural equivalence is alpha-equivalence which identifies
process that are the same up to a change of variable. Formally, we
recognize the distinction between free and bound names. The free names
of a process, $\freenames{P}$, may be calculated recursively as
follows:

\begin{mathpar}
\freenames{\pzero} := \emptyset
  \and \\
  \freenames{x?(y).P} := \{ x \} \cup (\freenames{P} \setminus \{ y \})
  \and 
  \freenames{x!\langle P \rangle} := \{ x \} \cup \{ P \} 
  \and \\
  \freenames{P|Q} := \freenames{P} \cup \freenames{Q}
  \and \\
  \freenames{@{x}} := \{ x \}
\end{mathpar}

$\pi$
$\quotep{\pi}$

$\freenames{-} : \pi \to \mathcal{P}(\quotep{\pi})$

\begin{eqnarray*}
  \freenames{\pzero} & := & \emptyset \\
  \freenames{x?(y).P} & := & \{ x \} \cup (\freenames{P} \setminus \{ y \}) \\
  \freenames{x!\langle P \rangle} & := & \{ x \} \cup \{ P \} \\
  \freenames{P|Q} & := & \freenames{P} \cup \freenames{Q} \\
  \freenames{\dropn{x}} & := & \{ x \}
\end{eqnarray*}

The bound names of a process, $\boundnames{P}$, are those names occurring in $P$
that are not free. For example, in $x?(y).0$, the name $x$ is free, while $y$ is bound.

\begin{mathpar}
  \inferrule* [lab=monoidal-laws] {} { P|Q \equiv Q|P \and P|0 \equiv P \and P|(Q|R) \equiv (P|Q)|R }
\end{mathpar}

\begin{mathpar}
  \inferrule* [lab=alpha-equivalence] {} { (x)P \equiv (y)P\{y/x\} \and y \not\in \freenames{P} }
\end{mathpar}

\begin{definition}
Then two processes, $P,Q$, are alpha-equivalent if $P = Q\{\vec{y}/\vec{x}\}$ for
some $\vec{x} \in \boundnames{Q},\vec{y} \in \boundnames{P}$, where $Q\{\vec{y}/\vec{x}\}$
denotes the capture-avoiding substitution of $\vec{y}$ for $\vec{x}$ in $Q$.
\end{definition}

\begin{definition}
  The {\em structural congruence} \cite{SangiorgiWalker} , $\equiv$,
  between processes is the least congruence containing
  alpha-equivalence, satisfying the abelian monoid laws
  (associativity, commutativity and $\pzero$ as identity) for parallel
  composition $|$ and for summation $+$.
\end{definition}

\subsection{Name equivalence}

We take name equivalence, written $\nameeq$, to be the smallest
equivalence relation generated by the following rules.

\begin{mathpar}
\inferrule*[lab=Quote-drop]
{ }
{ \quotep{@{x}} \nameeq x }

\inferrule*[lab=Struct-equiv]
{ P \scong Q }
{ \quotep{P} \nameeq \quotep{Q} }
\end{mathpar}

The astute reader will have noticed that the mutual recursion of names
and processes imposes a mutual recursion on alpha-equivalence and
structural equivalence via name-equivalence. Fortunately, all of this
works out pleasantly and we may calculate in the natural way, free of
concern. The reader interested in the details is referred to the
appendix \ref{appendix:rho_details}.

\subsection{Substitution}

We use $\Proc$ for the set of processes, $\QProc$ for the set of
names, and $\id{\{}\vec{y} / \vec{x} \id{\}}$ to denote partial maps,
$s : \QProc \rightarrow \QProc$. A map, $s$ lifts, uniquely, to a map
on process terms, $\widehat{s} : \Proc \rightarrow \Proc$ by the
following equations.

\begin{mathpar}
  (0) \psubstp{Q}{P} := 0 \\
  (R \juxtap S) \psubstp{Q}{P}
  :=    
  (R)\psubstp{Q}{P} \juxtap (S) \psubstp{Q}{P} \\
  (x?(y).R) \psubstp{Q}{P}    
  :=    
  (x)\substp{Q}{P} (z)\concat( (R \psubstn{z}{y}) \psubstp{Q}{P} ) \\
  (\lift{x}{R}) \psubstp{Q}{P}  
  :=
  \lift{(x)\substp{Q}{P}}{ R \psubstp{Q}{P} } \\
%   (\dropn{x})  \psubstp{Q}{P}       
%   := 
%   \left\{ 
%     \begin{array}{ccc} 
%       \dropn{\quotep{Q}} & & x \nameeq \quotep{P} \\
%       \dropn{x} & & otherwise \\
%     \end{array}
%   \right. 
  (\dropn{x})  \psubstp{Q}{P}       
  := 
  \left\{ 
    \begin{array}{ccc} 
      Q & & x \nameeq \quotep{P} \\
      \dropn{x} & & otherwise \\
    \end{array}
  \right.
\end{mathpar}
 

where

\begin{eqnarray}
  (x)\id{\{} \lpquote Q \rpquote / \lpquote P \rpquote \id{\}}            = 
  \left\{ 
    \begin{array}{ccc}
      \lpquote Q \rpquote & & x \nameeq \lpquote P \rpquote \\
      x & & otherwise \\
    \end{array}
  \right. \nonumber
\end{eqnarray}

and $z$ is chosen distinct from $\quotep{P}$, $\quotep{Q}$, the free
names in $Q$, and all the names in $R$. Our $\alpha$-equivalence will
be built in the standard way from this substitution.

\begin{remark}\label{rem:no_self_referential_names}
  One consequence of these definitions is that $\forall P. \quotep{P}
  \not\in \freenames{P}$.
\end{remark}

\subsection{ Dynamic quote: an example }

Anticipating something of what's to come, consider applying the
substitution, $\widehat{\id{\{}u / z \id{\}}}$, to the following pair
of processes, $\lift{w}{y!(z)}$ and $w[ \lpquote y!(z) \rpquote ]$.

\begin{eqnarray}
	\lift{w}{y!(z)}\widehat{\id{\{}u / z \id{\}}}
		& = &
		\lift{w}{y!(u)} \nonumber\\
	w[ \lpquote y!(z) \rpquote ] \widehat{ \id{\{}u / z \id{\}} }
		& = &
		w[ \lpquote y!(z) \rpquote ] \nonumber
\end{eqnarray}

Because the body of the process between quotes is impervious to
substitution, we get radically different answers. In fact, by
examining the first process in an input context,
e.g. $x?(z).\lift{w}{y!(z)}$, we see that the process under the lift
operator may be shaped by prefixed inputs binding a name inside it. In
this sense, the lift operator will be seen as a way to dynamically
construct processes before reifying them as names.

Finally equipped with these standard features we can present the
dynamics of the calculus.

\subsubsection{Operational semantics} 

Finally, we introduce the computational dynamics. What marks these
algebras as distinct from other more traditionally studied algebraic
structures, e.g. vector spaces or polynomial rings, is the manner in
which dynamics is captured. In traditional structures, dynamics is typically
expressed through morphisms between such structures, as in linear maps
between vector spaces or morphisms between rings. In algebras
associated with the semantics of computation, the dynamics is
expressed as part of the algebraic structure itself, through a
reduction reduction relation typically denoted by $\red$. Below, we
give a recursive presentation of this relation for the calculus used
in the encoding.

$\red \subseteq \pi \times \pi$
$\red : \pi \to \mathcal{P}(\pi)$

\begin{mathpar}
  \inferrule* [lab=Comm] { \textsf{match}( x_{src}, x_{trgt} ) } { x_{trgt}?(y)P \; | \; x_{src}!\langle {Q} \rangle \red P\{\quotep{Q}/y}\} }
  \and \\
  \inferrule* [lab=Par] {{P} \red {P}'} {{{P} | {Q}} \red {{P}' | {Q}}}
  \and
  \inferrule* [lab=Equiv]{{{P} \scong {P}'} \andalso {{P}' \red {Q}'} \andalso {{Q}' \scong {Q}}}{{P} \red {Q}}
\end{mathpar}

\begin{eqnarray*}
  match_{\equiv} (\quotep{P},\quotep{Q}) & := & P \equiv Q \\
  match_{\dagger}(\quotep{P},\quotep{Q}) & := & \forall R. P|Q \red^{*} R => R \red^{*} 0 \\
  match_{K}(\quotep{P},\quotep{Q}) & := & K \mbox{ for some context } K
\end{eqnarray*}

$u?(x)P | u!\langle Q \rangle \red P\{\quotep{Q}/x\}$

%We write $\wred$ for $\red^*$, and $P\red$ if $\exists Q $ such that $ P \red Q$.
We write $P\red$ if $\exists Q $ such that $ P \red Q$ and $P\not\red$, otherwise.

\section{Replication}

As mentioned before, it is known that replication (and hence
recursion) can be implemented in a higher-order process algebra
\cite{SangiorgiWalker}. As our first example of calculation with the
machinery thus far presented we give the construction explicitly in
the {\rhoc}.

\begin{eqnarray}
	D_{x} & := & \prefix{x}{y}{(\binpar{\outputp{x}{y}}{@{y}})} \nonumber\\
	\bangp_{x}{P} & := & \binpar{{x}!\langle{\binpar{D_{x}}{P}}\rangle}{D_{x}} \nonumber
\end{eqnarray}

\begin{eqnarray}
	\bangp_{x}{P} & & \nonumber\\
	=
	& {x}!\langle{(\prefix{x}{y}{(\outputp{x}{y} | @{y})) | P}}\rangle 
	      | \prefix{x}{y}{(\outputp{x}{y} | @{y})} & \nonumber\\
	\red
	& (\outputp{x}{y} | @{y})\substn{\quotep{(\prefix{x}{y}{(@{y} | \outputp{x}{y})) | P}}}{y} & \nonumber\\
	=
	& \outputp{x}{\quotep{(\prefix{x}{y}{(\outputp{x}{y} | @{y})) | P}}}
	  | {(\prefix{x}{y}{(\outputp{x}{y} | @{y})) | P}} & \nonumber\\
	\red
	& \ldots & \nonumber\\
	\red^*
	& P | P | \ldots & \nonumber
\end{eqnarray}

Of course, this encoding, as an implementation, runs away, unfolding
$\bangp{P}$ eagerly. A lazier and more implementable replication
operator, restricted to input-guarded processes, may be obtained as follows.

\begin{eqnarray}
\bangp{\prefix{u}{v}{P}} 
	:= 
	\binpar{\lift{x}{\prefix{u}{v}{(\binpar{D(x)}{P})}}}{D(x)} \nonumber
\end{eqnarray}

\begin{remark}
  Note that the lazier definition still does not deal with summation
  or mixed summation (i.e. sums over input and output). The reader is
  invited to construct definitions of replication that deal with these
  features. 

  Further, the definitions are parameterized in a name, $x$. Can you,
  gentle reader, make a definition that eliminates this parameter and
  guarantees no accidental interaction between the replication
  machinery and the process being replicated -- i.e. no accidental
  sharing of names used by the process to get its work done and the
  name(s) used by the replication to effect copying. This latter
  revision of the definition of replication is crucial to obtaining
  the expected identity $!!P \sim !P$.
\end{remark}

\begin{remark}\label{rem:paradoxical_combinator}
  The reader familiar with the lambda calculus will have noticed the
  similarity between $D$ and the paradoxical combinator.

  [Ed. note: the existence of this seems to suggest we have to be more
  restrictive on the set of processes and names we admit if we are to
  support no-cloning.]
\end{remark}

\subsubsection{Bisimulation}

The computational dynamics gives rise to another kind of equivalence,
the equivalence of computational behavior. As previously mentioned
this is typically captured \emph{via} some form of bisimulation.

% The notion we use in this paper is weak barbed bisimulation
% \cite{milner91polyadicpi}.

The notion we use in this paper is derived from weak barbed
bisimulation \cite{milner91polyadicpi}. 

\begin{definition}
An \emph{observation relation}, $\downarrow_{\mathcal N}$, over a set
of names, $\mathcal N$, is the smallest relation satisfying the rules
below.

\infrule[Out-barb]{y \in {\mathcal N}, \; x \nameeq y}
		  {\outputp{x}{v} \downarrow_{\mathcal N} x}
\infrule[Par-barb]{\mbox{$P\downarrow_{\mathcal N} x$ or $Q\downarrow_{\mathcal N} x$}}
		  {\binpar{P}{Q} \downarrow_{\mathcal N} x}

We write $P \Downarrow_{\mathcal N} x$ if there is $Q$ such that 
$P \wred Q$ and $Q \downarrow_{\mathcal N} x$.
\end{definition}

\begin{definition}
%\label{def.bbisim}
An  ${\mathcal N}$-\emph{barbed bisimulation} over a set of names, ${\mathcal N}$, is a symmetric binary relation 
${\mathcal S}_{\mathcal N}$ between agents such that $P\rel{S}_{\mathcal N}Q$ implies:
\begin{enumerate}
\item If $P \red P'$ then $Q \wred Q'$ and $P'\rel{S}_{\mathcal N} Q'$.
\item If $P\downarrow_{\mathcal N} x$, then $Q\Downarrow_{\mathcal N} x$.
\end{enumerate}
$P$ is ${\mathcal N}$-barbed bisimilar to $Q$, written
$P \wbbisim_{\mathcal N} Q$, if $P \rel{S}_{\mathcal N} Q$ for some ${\mathcal N}$-barbed bisimulation ${\mathcal S}_{\mathcal N}$.
\end{definition}

$\mathcal{R} \subseteq \pi \times \pi$

$P \mathcal{R} Q => \forall P'. P \red P' \Rightarrow \exists Q'. Q \red Q', P' \mathcal{R} Q'$

$P \vdash x \Rightarrow Q \vdash x$

\begin{mathpar}
  \inferrule*[lab=Out-barb]{x \nameeq y}{{y}!\langle{Q}\rangle \vdash x}
  \and
  \inferrule*[lab=Par-barb]{\mbox{$P\vdash x$ or $Q\vdash x$}}{\binpar{P}{Q} \vdash x}
\end{mathpar}

\subsubsection{Contexts}

One of the principle advantages of computational calculi like the
$\pi$-calculus is a well-defined notion of context,
contextual-equivalence and a correlation between
contextual-equivalence and notions of bisimulation. The notion of
context allows the decomposition of a process into (sub-)process and
its syntactic environment, its context. Thus, a context may be
thought of as a process with a ``hole'' (written $\Box$) in it. The
application of a context $M$ to a process $P$, written $M[P]$, is
tantamount to filling the hole in $M$ with $P$. In this paper we do
not need the full weight of this theory, but do make use of the notion
of context in the proof the main theorem. 

\begin{mathpar}
  \inferrule* [lab=summation] {} {{M_{M},M_{N}} \bc \Box \;|\; x.M_{A} \;|\; M_{M}+M_{N}}
  \and
  \inferrule* [lab=agent] {} {{M_{A}} \bc (\vec{x})M_{P} \;| \; \clift{P_0,\ldots,M_{P},\ldots,P_N}}
  \and \\
  \inferrule* [lab=process] {} {{M_{P}} \bc M_{N} \;| \;P|M_{P} }
\end{mathpar} 

\begin{mathpar}
  \inferrule* [lab=sychronization] {} {M_{N} \bc \Box \;|\; x?M_{F} \;|\; x!M_{C}}
  \and
  \inferrule* [lab=abstraction] {} {{M_{F}} \bc (x)M_{P} }
  \and
  \inferrule* [lab=concretion] {} {{M_{C}} \bc \langle M_{P} \rangle }
  \and \\
  \inferrule* [lab=process] {} {{M_{P}} \bc M_{N} \;| \;P|M_{P} }
\end{mathpar}

\begin{definition}[contextual application] Given a context $M$, and
  process $P$, we define the \emph{contextual application}, $M[P] :=
  M\{P/\Box\}$. That is, the contextual application of M to P is the
  substitution of $P$ for $\Box$ in $M$.
\end{definition}

$\meaningof{-} : L \to \mathcal{P}(\pi)$

\begin{mathpar}
  \inferrule* [lab=collection] {} {\meaningof{true} = \pi, \and \meaningof{~E} = \pi \setminus \meaningof{E}, \and \meaningof{E_{1} \& E_{2}} = \meaningof{E_{1}} \cap \meaningof{E_{2}}}
\end{mathpar}

\begin{mathpar}
  \inferrule* [lab=structure] {} {\meaningof{0} = \{ P \in \pi | P \equiv 0 \}, \and \\ \meaningof{E_1 | E_2} = \{ P \in \pi | P \equiv P_{1} | P_{2}, P_{1} \in \meaningof{E_{1}}, P_{2} \in \meaningof{E_2}\} }
\end{mathpar}

\begin{mathpar}
 \inferrule* [lab=behavior] {} {\meaningof{\langle a?b \rangle E} = \{ P \in \pi | P \equiv Q | u?(y)P', \\ \and \\\\ \and \\ \;\;\; u \in \meaningof{a}, \forall z.P'\{z/y\} \in \meaningof{E\{z/b\}}\}, \and \\ \meaningof{a!E} = \{ P \in \pi | P \equiv Q | x!\langle P' \rangle, x \in \meaningof{a} P' \in \meaningof{E}\} }
\end{mathpar}

\begin{mathpar}
 \inferrule* [lab=nominal] {} {\meaningof{\quotep{E}} = \{ \quotep{P} \in \quotep{\pi} | P \in \meaningof{E} \}, \and \meaningof{\quotep{P}} = \{ \quotep{Q} \in \quotep{\pi} | P \equiv Q \} \and \\ \meaningof{@\quotep{E}} = \{ P \in \pi | P \equiv @x, x \in \meaningof{E} \}}
\end{mathpar}

\begin{eqnarray*}
  \\
  \meaningof{-} : TS \to ST
\end{eqnarray*}

\begin{eqnarray*}
  \\
  L : TS \to ST
\end{eqnarray*}

\begin{eqnarray*}
  \\
  P \models E \iff P \in \meaningof{E}
\end{eqnarray*}

\begin{eqnarray*}
  P \approx_{L} Q \iff \forall E \in L. P \models E \iff Q \models E
\end{eqnarray*}

\begin{eqnarray*}
  P \approx_{K} Q
\end{eqnarray*}

\begin{eqnarray*}
  P \approx Q
\end{eqnarray*}

$\approx_{K} = \approx = \approx_{L}$

\subsubsection{Contextual duality}

Note that contexts extend the quotation operation to a family of
operations from processes to names. Given a context, $M$, we can
define a \emph{nominal context}, $\quotep{M}$ by $\quotep{M}[P] :=
\quotep{M[P]}$. To foreshadow what is to come we observe that these
operations enjoy a duality with processes very much like the duality
between vectors and maps from vectors to scalars.

Further, because the calculus is essentially higher-order, we have a
correspondence between contexts and processes. More specifically,
given a name $x$ and a context $M$ we can construct $M^{*}_{x}$ such
that 

\begin{mathpar}
  M^{*}_{x} | \lift{x}{P} \red M[P]
\end{mathpar}

namely,

\begin{mathpar}
  M^{*}_{x} := x?(u).M[\dropn{u}]
\end{mathpar}

The dependence of $M^{*}_{x}$ on a name makes it an abstraction, 

\begin{mathpar}
  M^{*} := (x)x?(u).M[\dropn{u}]
\end{mathpar}

\subsection{Additional notation}

It will sometimes be convenient to denote the process a name
quotes. We already have the notation $x = \quotep{P}$, but it will be
convenient to introduce an alternate notation, $\procn{x}$, when we
want to emphasize the connection to the use of the name. Note that, by
virtue of name equivalence, $\quotep{\procn{x}} \nameeq x$; so, the
notation is consistent with previous definitions.

Further, because names have structure it is possible to effect
substitutions on the basis of that structure. This means we need to
upgrade our notation for substitutions, which we accomplish by
adapting comprehension notation. Thus,

\begin{mathpar}
  P\{ y / x : x \in S \}
\end{mathpar}

is interpreted to mean the process derived from P by replacing (in a
capture-avoiding manner) each occurrence of $x$ in $S$ by $y$. For example,

\begin{mathpar}
  P\{ \quotep{\procn{x}|\procn{x}} / x : x \in \freenames{P} \}
\end{mathpar}

will replace each (occurrence) of a free name $x$ in $P$ by
$\quotep{\procn{x}|\procn{x}}$.

Also, we will avail ourselves of the notation $x^{L}$ and $x^{R}$ to
denote injections of a name into disjoint copies of the name
space. There are numerous ways to accomplish this. One example can be
found in \cite{MeredithR05}. This notation overloads to vectors of
names: $\vec{x}^{\pi} := (x_{i}^{\pi} \; : \; 0 \leq i < |\vec{x}| )$ where $\pi \in \{L,R\}$.

We also use $P^{\Box} := P|\Box$.

In \cite{MeredithR05} an interpretation of the new operator is
given. It turns out that there are several possible interpretations
all enjoying the requisite algebraic properties of the operator (see
\cite{milner91polyadicpi}). We will therefore make liberal use of
$(\nu\; \vec{x})P$.

% subsection the_syntax_and_semantics_of_the_notation_system (end)   

\input{qm2pi.qmops} 

\input{qm2pi.sterngerlach} 

\input{qm2pi.metric} 

% section concurrent_process_calculi (end)

%\input{qm2pi.proofsketch}

% section proof sketch (end)

%\input{qm2pi.slviaknots} 

% section spatial logic via knots (end)

\input{qm2pi.conclusion}

% section conclusion (end)

%\input{qm2pi.dtcodes} 

% section wiring algorithm (end)

\input{qm2pi.ack} 

% section acknowledgments (end)

\newpage


\bibliographystyle{plain}   
\bibliography{../../biblios/main.bib}

\input{qm2pi.rhodetails}

\end{document}

 

%\ifpdf
%\usepackage[pdftex]{graphicx}
%\else
%\usepackage{graphicx}
%\fi

 % \ifpdf
%  \usepackage{pdfsync}
%  \if


%\title{Brief Article}
%\author{David F. Snyder}
%\author{L.G. Meredith}

%\address{Dept. of Math., Texas State University--San Marcos, San Marcos, TX 78666}
       
\pagestyle{empty}


\begin{document}

\lstset{language=[Objective]Caml,frame=shadowbox}

\documentclass[12pt]{llncs}
%\documentclass{jktr}

\usepackage[pdftex]{hyperref}                   
\usepackage {listings}
\usepackage {mathpartir}
\usepackage{bcprules}
%\usepackage{listings}
                       
\usepackage{graphicx} 
%\usepackage[margins=2.5cm,nohead,nofoot]{geometry}
%\usepackage{geometry}
\usepackage{amsfonts}
\usepackage{amstext}
\usepackage{latexsym}
\usepackage{amssymb}
\usepackage{color}


%\include{myPreamble}
\include{qm2pi.local} 

%\ifpdf
%\usepackage[pdftex]{graphicx}
%\else
%\usepackage{graphicx}
%\fi

 % \ifpdf
%  \usepackage{pdfsync}
%  \if


%\title{Brief Article}
%\author{David F. Snyder}
%\author{L.G. Meredith}

%\address{Dept. of Math., Texas State University--San Marcos, San Marcos, TX 78666}
       
\pagestyle{empty}


\begin{document}

\lstset{language=[Objective]Caml,frame=shadowbox}

\input{qm2pi.front}

% section front matter (end)

\input{qm2pi.intro} 
 
% section introduction (end)

% \input{qm2pi.knotations} 

% section notation (end)

\input{qm2pi.process.calculi} 

% section concurrent_process_calculi_and_spatial_logics_ (end)
    
%\input{qm2pi.knots2pi} 

%\input{qm2pi.trefoil} 

%\input{qm2pi.mainthm} 

% subsection basic_interpretation (end)

%\input{qm2pi.rho.presentation} 
\subsection{The syntax and semantics of the notation system}\label{sub:the_syntax_and_semantics_of_the_notation_system} % (fold)

We now summarize a technical presentation of the calculus that
embodies our theory of dynamics. The typical presentation of such a
calculus follows the style of giving generators and relations on
them. The grammar, below, describing term constructors, freely
generates the set of processes, $\Proc$. This set is then quotiented
by a relation known as structural congruence and it is over this set
that the notion of dynamics is expressed. This presentation is
essentially that of \cite{MeredithR05} with the addition of
polyadicity and summation. For readability we have relegated some of
the technical subtleties to an appendix.

\subsubsection{Process grammar}\label{subsub:process_grammar}

\begin{mathpar}
  \inferrule* [lab=synchronization] {} {{M} \bc \pzero \;|\; x?F \;|\; x!C }
  \and
  \inferrule* [lab=abstraction] {} {{F} \bc (x)P}
  \and
  \inferrule* [lab=concretion] {} {{C} \bc \langle Q \rangle}
  \and
  \inferrule* [lab=process] {} {{P,Q} \bc M \;| \;P|Q \;|\; @{x}}
  \and
  \inferrule* [lab=name] {} {{x} \bc \quotep{P}}
\end{mathpar} 

Note that $\vec{x}$ (resp. $\vec{P}$) denotes a vector of names
(resp. processes) of length $|\vec{x}|$ (resp. $|\vec{P}|$). We adopt
the following useful abbreviations.

\begin{mathpar}
   x?(\vec{y}).P := x.(\vec{y})P \and  x\clift{\vec{P}} := x.\clift{\vec{P}}
   \and x!(y) := \lift{x}{\dropn{y}}
   \and \Pi_{i=0}^{n-1}P_i := P_0 | \ldots | P_{n-1}
\end{mathpar}

\subsubsection{Structural congruence}

\paragraph{Free and bound names and alpha-equivalence.} At the
core of structural equivalence is alpha-equivalence which identifies
process that are the same up to a change of variable. Formally, we
recognize the distinction between free and bound names. The free names
of a process, $\freenames{P}$, may be calculated recursively as
follows:

\begin{mathpar}
\freenames{\pzero} := \emptyset
  \and \\
  \freenames{x?(y).P} := \{ x \} \cup (\freenames{P} \setminus \{ y \})
  \and 
  \freenames{x!\langle P \rangle} := \{ x \} \cup \{ P \} 
  \and \\
  \freenames{P|Q} := \freenames{P} \cup \freenames{Q}
  \and \\
  \freenames{@{x}} := \{ x \}
\end{mathpar}

$\pi$
$\quotep{\pi}$

$\freenames{-} : \pi \to \mathcal{P}(\quotep{\pi})$

\begin{eqnarray*}
  \freenames{\pzero} & := & \emptyset \\
  \freenames{x?(y).P} & := & \{ x \} \cup (\freenames{P} \setminus \{ y \}) \\
  \freenames{x!\langle P \rangle} & := & \{ x \} \cup \{ P \} \\
  \freenames{P|Q} & := & \freenames{P} \cup \freenames{Q} \\
  \freenames{\dropn{x}} & := & \{ x \}
\end{eqnarray*}

The bound names of a process, $\boundnames{P}$, are those names occurring in $P$
that are not free. For example, in $x?(y).0$, the name $x$ is free, while $y$ is bound.

\begin{mathpar}
  \inferrule* [lab=monoidal-laws] {} { P|Q \equiv Q|P \and P|0 \equiv P \and P|(Q|R) \equiv (P|Q)|R }
\end{mathpar}

\begin{mathpar}
  \inferrule* [lab=alpha-equivalence] {} { (x)P \equiv (y)P\{y/x\} \and y \not\in \freenames{P} }
\end{mathpar}

\begin{definition}
Then two processes, $P,Q$, are alpha-equivalent if $P = Q\{\vec{y}/\vec{x}\}$ for
some $\vec{x} \in \boundnames{Q},\vec{y} \in \boundnames{P}$, where $Q\{\vec{y}/\vec{x}\}$
denotes the capture-avoiding substitution of $\vec{y}$ for $\vec{x}$ in $Q$.
\end{definition}

\begin{definition}
  The {\em structural congruence} \cite{SangiorgiWalker} , $\equiv$,
  between processes is the least congruence containing
  alpha-equivalence, satisfying the abelian monoid laws
  (associativity, commutativity and $\pzero$ as identity) for parallel
  composition $|$ and for summation $+$.
\end{definition}

\subsection{Name equivalence}

We take name equivalence, written $\nameeq$, to be the smallest
equivalence relation generated by the following rules.

\begin{mathpar}
\inferrule*[lab=Quote-drop]
{ }
{ \quotep{@{x}} \nameeq x }

\inferrule*[lab=Struct-equiv]
{ P \scong Q }
{ \quotep{P} \nameeq \quotep{Q} }
\end{mathpar}

The astute reader will have noticed that the mutual recursion of names
and processes imposes a mutual recursion on alpha-equivalence and
structural equivalence via name-equivalence. Fortunately, all of this
works out pleasantly and we may calculate in the natural way, free of
concern. The reader interested in the details is referred to the
appendix \ref{appendix:rho_details}.

\subsection{Substitution}

We use $\Proc$ for the set of processes, $\QProc$ for the set of
names, and $\id{\{}\vec{y} / \vec{x} \id{\}}$ to denote partial maps,
$s : \QProc \rightarrow \QProc$. A map, $s$ lifts, uniquely, to a map
on process terms, $\widehat{s} : \Proc \rightarrow \Proc$ by the
following equations.

\begin{mathpar}
  (0) \psubstp{Q}{P} := 0 \\
  (R \juxtap S) \psubstp{Q}{P}
  :=    
  (R)\psubstp{Q}{P} \juxtap (S) \psubstp{Q}{P} \\
  (x?(y).R) \psubstp{Q}{P}    
  :=    
  (x)\substp{Q}{P} (z)\concat( (R \psubstn{z}{y}) \psubstp{Q}{P} ) \\
  (\lift{x}{R}) \psubstp{Q}{P}  
  :=
  \lift{(x)\substp{Q}{P}}{ R \psubstp{Q}{P} } \\
%   (\dropn{x})  \psubstp{Q}{P}       
%   := 
%   \left\{ 
%     \begin{array}{ccc} 
%       \dropn{\quotep{Q}} & & x \nameeq \quotep{P} \\
%       \dropn{x} & & otherwise \\
%     \end{array}
%   \right. 
  (\dropn{x})  \psubstp{Q}{P}       
  := 
  \left\{ 
    \begin{array}{ccc} 
      Q & & x \nameeq \quotep{P} \\
      \dropn{x} & & otherwise \\
    \end{array}
  \right.
\end{mathpar}
 

where

\begin{eqnarray}
  (x)\id{\{} \lpquote Q \rpquote / \lpquote P \rpquote \id{\}}            = 
  \left\{ 
    \begin{array}{ccc}
      \lpquote Q \rpquote & & x \nameeq \lpquote P \rpquote \\
      x & & otherwise \\
    \end{array}
  \right. \nonumber
\end{eqnarray}

and $z$ is chosen distinct from $\quotep{P}$, $\quotep{Q}$, the free
names in $Q$, and all the names in $R$. Our $\alpha$-equivalence will
be built in the standard way from this substitution.

\begin{remark}\label{rem:no_self_referential_names}
  One consequence of these definitions is that $\forall P. \quotep{P}
  \not\in \freenames{P}$.
\end{remark}

\subsection{ Dynamic quote: an example }

Anticipating something of what's to come, consider applying the
substitution, $\widehat{\id{\{}u / z \id{\}}}$, to the following pair
of processes, $\lift{w}{y!(z)}$ and $w[ \lpquote y!(z) \rpquote ]$.

\begin{eqnarray}
	\lift{w}{y!(z)}\widehat{\id{\{}u / z \id{\}}}
		& = &
		\lift{w}{y!(u)} \nonumber\\
	w[ \lpquote y!(z) \rpquote ] \widehat{ \id{\{}u / z \id{\}} }
		& = &
		w[ \lpquote y!(z) \rpquote ] \nonumber
\end{eqnarray}

Because the body of the process between quotes is impervious to
substitution, we get radically different answers. In fact, by
examining the first process in an input context,
e.g. $x?(z).\lift{w}{y!(z)}$, we see that the process under the lift
operator may be shaped by prefixed inputs binding a name inside it. In
this sense, the lift operator will be seen as a way to dynamically
construct processes before reifying them as names.

Finally equipped with these standard features we can present the
dynamics of the calculus.

\subsubsection{Operational semantics} 

Finally, we introduce the computational dynamics. What marks these
algebras as distinct from other more traditionally studied algebraic
structures, e.g. vector spaces or polynomial rings, is the manner in
which dynamics is captured. In traditional structures, dynamics is typically
expressed through morphisms between such structures, as in linear maps
between vector spaces or morphisms between rings. In algebras
associated with the semantics of computation, the dynamics is
expressed as part of the algebraic structure itself, through a
reduction reduction relation typically denoted by $\red$. Below, we
give a recursive presentation of this relation for the calculus used
in the encoding.

$\red \subseteq \pi \times \pi$
$\red : \pi \to \mathcal{P}(\pi)$

\begin{mathpar}
  \inferrule* [lab=Comm] { \textsf{match}( x_{src}, x_{trgt} ) } { x_{trgt}?(y)P \; | \; x_{src}!\langle {Q} \rangle \red P\{\quotep{Q}/y}\} }
  \and \\
  \inferrule* [lab=Par] {{P} \red {P}'} {{{P} | {Q}} \red {{P}' | {Q}}}
  \and
  \inferrule* [lab=Equiv]{{{P} \scong {P}'} \andalso {{P}' \red {Q}'} \andalso {{Q}' \scong {Q}}}{{P} \red {Q}}
\end{mathpar}

\begin{eqnarray*}
  match_{\equiv} (\quotep{P},\quotep{Q}) & := & P \equiv Q \\
  match_{\dagger}(\quotep{P},\quotep{Q}) & := & \forall R. P|Q \red^{*} R => R \red^{*} 0 \\
  match_{K}(\quotep{P},\quotep{Q}) & := & K \mbox{ for some context } K
\end{eqnarray*}

$u?(x)P | u!\langle Q \rangle \red P\{\quotep{Q}/x\}$

%We write $\wred$ for $\red^*$, and $P\red$ if $\exists Q $ such that $ P \red Q$.
We write $P\red$ if $\exists Q $ such that $ P \red Q$ and $P\not\red$, otherwise.

\section{Replication}

As mentioned before, it is known that replication (and hence
recursion) can be implemented in a higher-order process algebra
\cite{SangiorgiWalker}. As our first example of calculation with the
machinery thus far presented we give the construction explicitly in
the {\rhoc}.

\begin{eqnarray}
	D_{x} & := & \prefix{x}{y}{(\binpar{\outputp{x}{y}}{@{y}})} \nonumber\\
	\bangp_{x}{P} & := & \binpar{{x}!\langle{\binpar{D_{x}}{P}}\rangle}{D_{x}} \nonumber
\end{eqnarray}

\begin{eqnarray}
	\bangp_{x}{P} & & \nonumber\\
	=
	& {x}!\langle{(\prefix{x}{y}{(\outputp{x}{y} | @{y})) | P}}\rangle 
	      | \prefix{x}{y}{(\outputp{x}{y} | @{y})} & \nonumber\\
	\red
	& (\outputp{x}{y} | @{y})\substn{\quotep{(\prefix{x}{y}{(@{y} | \outputp{x}{y})) | P}}}{y} & \nonumber\\
	=
	& \outputp{x}{\quotep{(\prefix{x}{y}{(\outputp{x}{y} | @{y})) | P}}}
	  | {(\prefix{x}{y}{(\outputp{x}{y} | @{y})) | P}} & \nonumber\\
	\red
	& \ldots & \nonumber\\
	\red^*
	& P | P | \ldots & \nonumber
\end{eqnarray}

Of course, this encoding, as an implementation, runs away, unfolding
$\bangp{P}$ eagerly. A lazier and more implementable replication
operator, restricted to input-guarded processes, may be obtained as follows.

\begin{eqnarray}
\bangp{\prefix{u}{v}{P}} 
	:= 
	\binpar{\lift{x}{\prefix{u}{v}{(\binpar{D(x)}{P})}}}{D(x)} \nonumber
\end{eqnarray}

\begin{remark}
  Note that the lazier definition still does not deal with summation
  or mixed summation (i.e. sums over input and output). The reader is
  invited to construct definitions of replication that deal with these
  features. 

  Further, the definitions are parameterized in a name, $x$. Can you,
  gentle reader, make a definition that eliminates this parameter and
  guarantees no accidental interaction between the replication
  machinery and the process being replicated -- i.e. no accidental
  sharing of names used by the process to get its work done and the
  name(s) used by the replication to effect copying. This latter
  revision of the definition of replication is crucial to obtaining
  the expected identity $!!P \sim !P$.
\end{remark}

\begin{remark}\label{rem:paradoxical_combinator}
  The reader familiar with the lambda calculus will have noticed the
  similarity between $D$ and the paradoxical combinator.

  [Ed. note: the existence of this seems to suggest we have to be more
  restrictive on the set of processes and names we admit if we are to
  support no-cloning.]
\end{remark}

\subsubsection{Bisimulation}

The computational dynamics gives rise to another kind of equivalence,
the equivalence of computational behavior. As previously mentioned
this is typically captured \emph{via} some form of bisimulation.

% The notion we use in this paper is weak barbed bisimulation
% \cite{milner91polyadicpi}.

The notion we use in this paper is derived from weak barbed
bisimulation \cite{milner91polyadicpi}. 

\begin{definition}
An \emph{observation relation}, $\downarrow_{\mathcal N}$, over a set
of names, $\mathcal N$, is the smallest relation satisfying the rules
below.

\infrule[Out-barb]{y \in {\mathcal N}, \; x \nameeq y}
		  {\outputp{x}{v} \downarrow_{\mathcal N} x}
\infrule[Par-barb]{\mbox{$P\downarrow_{\mathcal N} x$ or $Q\downarrow_{\mathcal N} x$}}
		  {\binpar{P}{Q} \downarrow_{\mathcal N} x}

We write $P \Downarrow_{\mathcal N} x$ if there is $Q$ such that 
$P \wred Q$ and $Q \downarrow_{\mathcal N} x$.
\end{definition}

\begin{definition}
%\label{def.bbisim}
An  ${\mathcal N}$-\emph{barbed bisimulation} over a set of names, ${\mathcal N}$, is a symmetric binary relation 
${\mathcal S}_{\mathcal N}$ between agents such that $P\rel{S}_{\mathcal N}Q$ implies:
\begin{enumerate}
\item If $P \red P'$ then $Q \wred Q'$ and $P'\rel{S}_{\mathcal N} Q'$.
\item If $P\downarrow_{\mathcal N} x$, then $Q\Downarrow_{\mathcal N} x$.
\end{enumerate}
$P$ is ${\mathcal N}$-barbed bisimilar to $Q$, written
$P \wbbisim_{\mathcal N} Q$, if $P \rel{S}_{\mathcal N} Q$ for some ${\mathcal N}$-barbed bisimulation ${\mathcal S}_{\mathcal N}$.
\end{definition}

$\mathcal{R} \subseteq \pi \times \pi$

$P \mathcal{R} Q => \forall P'. P \red P' \Rightarrow \exists Q'. Q \red Q', P' \mathcal{R} Q'$

$P \vdash x \Rightarrow Q \vdash x$

\begin{mathpar}
  \inferrule*[lab=Out-barb]{x \nameeq y}{{y}!\langle{Q}\rangle \vdash x}
  \and
  \inferrule*[lab=Par-barb]{\mbox{$P\vdash x$ or $Q\vdash x$}}{\binpar{P}{Q} \vdash x}
\end{mathpar}

\subsubsection{Contexts}

One of the principle advantages of computational calculi like the
$\pi$-calculus is a well-defined notion of context,
contextual-equivalence and a correlation between
contextual-equivalence and notions of bisimulation. The notion of
context allows the decomposition of a process into (sub-)process and
its syntactic environment, its context. Thus, a context may be
thought of as a process with a ``hole'' (written $\Box$) in it. The
application of a context $M$ to a process $P$, written $M[P]$, is
tantamount to filling the hole in $M$ with $P$. In this paper we do
not need the full weight of this theory, but do make use of the notion
of context in the proof the main theorem. 

\begin{mathpar}
  \inferrule* [lab=summation] {} {{M_{M},M_{N}} \bc \Box \;|\; x.M_{A} \;|\; M_{M}+M_{N}}
  \and
  \inferrule* [lab=agent] {} {{M_{A}} \bc (\vec{x})M_{P} \;| \; \clift{P_0,\ldots,M_{P},\ldots,P_N}}
  \and \\
  \inferrule* [lab=process] {} {{M_{P}} \bc M_{N} \;| \;P|M_{P} }
\end{mathpar} 

\begin{mathpar}
  \inferrule* [lab=sychronization] {} {M_{N} \bc \Box \;|\; x?M_{F} \;|\; x!M_{C}}
  \and
  \inferrule* [lab=abstraction] {} {{M_{F}} \bc (x)M_{P} }
  \and
  \inferrule* [lab=concretion] {} {{M_{C}} \bc \langle M_{P} \rangle }
  \and \\
  \inferrule* [lab=process] {} {{M_{P}} \bc M_{N} \;| \;P|M_{P} }
\end{mathpar}

\begin{definition}[contextual application] Given a context $M$, and
  process $P$, we define the \emph{contextual application}, $M[P] :=
  M\{P/\Box\}$. That is, the contextual application of M to P is the
  substitution of $P$ for $\Box$ in $M$.
\end{definition}

$\meaningof{-} : L \to \mathcal{P}(\pi)$

\begin{mathpar}
  \inferrule* [lab=collection] {} {\meaningof{true} = \pi, \and \meaningof{~E} = \pi \setminus \meaningof{E}, \and \meaningof{E_{1} \& E_{2}} = \meaningof{E_{1}} \cap \meaningof{E_{2}}}
\end{mathpar}

\begin{mathpar}
  \inferrule* [lab=structure] {} {\meaningof{0} = \{ P \in \pi | P \equiv 0 \}, \and \\ \meaningof{E_1 | E_2} = \{ P \in \pi | P \equiv P_{1} | P_{2}, P_{1} \in \meaningof{E_{1}}, P_{2} \in \meaningof{E_2}\} }
\end{mathpar}

\begin{mathpar}
 \inferrule* [lab=behavior] {} {\meaningof{\langle a?b \rangle E} = \{ P \in \pi | P \equiv Q | u?(y)P', \\ \and \\\\ \and \\ \;\;\; u \in \meaningof{a}, \forall z.P'\{z/y\} \in \meaningof{E\{z/b\}}\}, \and \\ \meaningof{a!E} = \{ P \in \pi | P \equiv Q | x!\langle P' \rangle, x \in \meaningof{a} P' \in \meaningof{E}\} }
\end{mathpar}

\begin{mathpar}
 \inferrule* [lab=nominal] {} {\meaningof{\quotep{E}} = \{ \quotep{P} \in \quotep{\pi} | P \in \meaningof{E} \}, \and \meaningof{\quotep{P}} = \{ \quotep{Q} \in \quotep{\pi} | P \equiv Q \} \and \\ \meaningof{@\quotep{E}} = \{ P \in \pi | P \equiv @x, x \in \meaningof{E} \}}
\end{mathpar}

\begin{eqnarray*}
  \\
  \meaningof{-} : TS \to ST
\end{eqnarray*}

\begin{eqnarray*}
  \\
  L : TS \to ST
\end{eqnarray*}

\begin{eqnarray*}
  \\
  P \models E \iff P \in \meaningof{E}
\end{eqnarray*}

\begin{eqnarray*}
  P \approx_{L} Q \iff \forall E \in L. P \models E \iff Q \models E
\end{eqnarray*}

\begin{eqnarray*}
  P \approx_{K} Q
\end{eqnarray*}

\begin{eqnarray*}
  P \approx Q
\end{eqnarray*}

$\approx_{K} = \approx = \approx_{L}$

\subsubsection{Contextual duality}

Note that contexts extend the quotation operation to a family of
operations from processes to names. Given a context, $M$, we can
define a \emph{nominal context}, $\quotep{M}$ by $\quotep{M}[P] :=
\quotep{M[P]}$. To foreshadow what is to come we observe that these
operations enjoy a duality with processes very much like the duality
between vectors and maps from vectors to scalars.

Further, because the calculus is essentially higher-order, we have a
correspondence between contexts and processes. More specifically,
given a name $x$ and a context $M$ we can construct $M^{*}_{x}$ such
that 

\begin{mathpar}
  M^{*}_{x} | \lift{x}{P} \red M[P]
\end{mathpar}

namely,

\begin{mathpar}
  M^{*}_{x} := x?(u).M[\dropn{u}]
\end{mathpar}

The dependence of $M^{*}_{x}$ on a name makes it an abstraction, 

\begin{mathpar}
  M^{*} := (x)x?(u).M[\dropn{u}]
\end{mathpar}

\subsection{Additional notation}

It will sometimes be convenient to denote the process a name
quotes. We already have the notation $x = \quotep{P}$, but it will be
convenient to introduce an alternate notation, $\procn{x}$, when we
want to emphasize the connection to the use of the name. Note that, by
virtue of name equivalence, $\quotep{\procn{x}} \nameeq x$; so, the
notation is consistent with previous definitions.

Further, because names have structure it is possible to effect
substitutions on the basis of that structure. This means we need to
upgrade our notation for substitutions, which we accomplish by
adapting comprehension notation. Thus,

\begin{mathpar}
  P\{ y / x : x \in S \}
\end{mathpar}

is interpreted to mean the process derived from P by replacing (in a
capture-avoiding manner) each occurrence of $x$ in $S$ by $y$. For example,

\begin{mathpar}
  P\{ \quotep{\procn{x}|\procn{x}} / x : x \in \freenames{P} \}
\end{mathpar}

will replace each (occurrence) of a free name $x$ in $P$ by
$\quotep{\procn{x}|\procn{x}}$.

Also, we will avail ourselves of the notation $x^{L}$ and $x^{R}$ to
denote injections of a name into disjoint copies of the name
space. There are numerous ways to accomplish this. One example can be
found in \cite{MeredithR05}. This notation overloads to vectors of
names: $\vec{x}^{\pi} := (x_{i}^{\pi} \; : \; 0 \leq i < |\vec{x}| )$ where $\pi \in \{L,R\}$.

We also use $P^{\Box} := P|\Box$.

In \cite{MeredithR05} an interpretation of the new operator is
given. It turns out that there are several possible interpretations
all enjoying the requisite algebraic properties of the operator (see
\cite{milner91polyadicpi}). We will therefore make liberal use of
$(\nu\; \vec{x})P$.

% subsection the_syntax_and_semantics_of_the_notation_system (end)   

\input{qm2pi.qmops} 

\input{qm2pi.sterngerlach} 

\input{qm2pi.metric} 

% section concurrent_process_calculi (end)

%\input{qm2pi.proofsketch}

% section proof sketch (end)

%\input{qm2pi.slviaknots} 

% section spatial logic via knots (end)

\input{qm2pi.conclusion}

% section conclusion (end)

%\input{qm2pi.dtcodes} 

% section wiring algorithm (end)

\input{qm2pi.ack} 

% section acknowledgments (end)

\newpage


\bibliographystyle{plain}   
\bibliography{../../biblios/main.bib}

\input{qm2pi.rhodetails}

\end{document}



% section front matter (end)

\section{Introduction}\label{sec:introduction} % (fold)
In this draft of the material i am going to have to dispense with the
usual writing conventions adopted in papers on these topics. i'm going
to have adopt whatever tone i need at the time i'm writing up the
calculations. Sometimes this may be very conversational; others it may
be the barest mathematical grunts; others still it may be that i have
lifted text from one of my other papers because the exposition of some
point was better said there. i hope that my readers are not unduly put
out by this decision. i'm not doing this to flout convention or be
rebellious. i find these calculations very technically challenging. To
keep everything going technically, something has to give; i have to
let go of some cognitive burden. So, the academic writing style --
with all of its trade-offs in terms of facilitating technical
communication -- is what i'm letting go of. Perhaps subsequent drafts
can be tightened and polished, but for now, i'm going to speak as if
we were sitting together in a coffee shop with a laptop, wifi and a
pad of paper and a pencil.

So, here's what i have to say. We -- you and i, comfortably ensconced
in our coffee shop and well-equipped with our tools -- can realize and
carry out the calculations of quantum mechanics over a very different
formal theory of dynamics, a formal theory of dynamics that
corresponds to a theory of concurrent computation with
\emph{reflection}. It has the advantage that the underlying theory is
already `quantized', but supports analogues all of the continuuous
operations. Strikingly, this underlying theory has recently been
connected with a notion of metric that we can show, by calculating
together, coincides with the metric induced by the inner product.

There are a lot of reasons why you might be interested in seeing
calculations of this form. Here's why i'm interested. For the past
several centuries there has been no competitor to the ``Newtonian''
account of dynamics. As a result the predominant share of accounts of
dynamical systems and situations have had to be formulated in terms of
the Newtonian machinery. i view this as an intellectually dangerous
position to occupy. Everything, despite it's intrinsic shape, turns
into a nail to be hit with this hammer. Recently, however, the theory
of computation has matured to the point where we have candidates for
theories of dynamics that offer very different perspective on
reasoning about dynamical systems and situations. Testing these
candidates against very successful accounts of dynamical situations,
like quantum mechanics, is going to give us some sense of how mature
they are and some measure of the quality of these accounts of
dynamics.

\subsection{Summary of contributions and outline of paper}

So, we're going to develop an interpretation of the operations of
quantum mechanics normally interpreted by Hilbert spaces and
operators. We're going to do this over a theory of computation. Note
that this is very different than the usual quantum computation program
which develops notions of computation over quantum mechanics. Rather,
we are developing a story that aligns with Wheeler's slogan: It from
Bit. To do this we will first provide an account of the theory of
computation at play here. Then we will dive into a calculation-driven
interpretation of the operations of quantum mechanics.

The reason we take this approach is that -- until very recently --
there hasn't been an axiomatic account of quantum mechanics. As a
result there has been no sharp delineation of the mathematical theory
supporting interpretation of the physical theory and the physical
theory, itself. So, ambient features of the maths are free to be
exploited (or supressed) without a real accounting of their physical
relevance. There is no sharp statement ``here's the physical theory''
qua \emph{theory} and ``here's the mathematical interpretation''
enabling a judgment of how faithful the interpretation is -- apart
from experimental observation. When there is an axiomatic account we
can judge how well a given mathematical formalism supports an
interpretation of the axioms, independent of
experimentation. Likewise, we can judge how well we have captured our
physical evidence and experience with our axiomatics, independent of
any specific mathematical implementation, with accidental detail that
may or may not have physical significance. 

In lieu of a fully fleshed out and vetted axiomatic account of quantum
mechanics, interpreting the operational notions in service of modeling
physical systems will have to suffice. In other words, we are not in
the business of providing a model of Hilbert spaces and operators. We
are in the business of providing a model of quantum mechanics because
we are motivated by testing our notions of dynamics against physical
theory; and, the predictive calculations of the physical theory must
serve as the best formulation -- shy of a fully fleshed out axiomatic
account -- of the physical theory itself (as they have for scientific
theories since time immemorial). Put another way, despite a
whole-hearted commitment to an It-from-Bit ontology, we are firmly
aligned with the shut-up-and-calculate camp as the best way to obtain
results either from the physical perspective or as a quality assurance
measure of our fledgling theory of dynamics.

In detail, we present a reflective process calculus. Then we develop
intuitive correspondences between the notions available in this
calculus and the usual physical notions supporting quantum mechanical
calculations. Thus, 

\begin{table}[htp]
  \center{
    \fbox{
      \begin{tabular}{c|c}
        quantum mechanics & process calculus \\
        \hline
        scalar & name \\
        state vector & process \\
        dual & contextual duals \\
        matrix & formal sums of process-context-dual pairs \\
        orthogonality & process annihilation \\
        inner product & execution-formula + quoting
      \end{tabular}
    }
  }
  \caption{QM - process calculi correspondences}
\end{table}

Then we tighten up these intuitions to operational definitions. We
employ the Dirac notation as the best proxy we can find for an
abstract syntax of the quantum mechanical notions. The definitions we
develop put us in contact with equational constraints coming from the
theory that we demonstrate the definitions and calculations satisfy.

This puts us in a position to shut up and calculate for the
Stern-Gerlach experimental set up, showing how these predictive
calculations become calculations on processes in our theory of a
reflective process calculus.

Penultimately, we demonstrate that the notion of metric coming from
the inner product coincides with the notion of metric available from
the theory of bisimulation. This demonstration gives us the right to
think of space as arising from behavior. Finally, we consider where we
might go from the new vantage point we have obtained.

% section introduction (end) 
 
% section introduction (end)

% \documentclass[12pt]{llncs}
%\documentclass{jktr}

\usepackage[pdftex]{hyperref}                   
\usepackage {listings}
\usepackage {mathpartir}
\usepackage{bcprules}
%\usepackage{listings}
                       
\usepackage{graphicx} 
%\usepackage[margins=2.5cm,nohead,nofoot]{geometry}
%\usepackage{geometry}
\usepackage{amsfonts}
\usepackage{amstext}
\usepackage{latexsym}
\usepackage{amssymb}
\usepackage{color}


%\include{myPreamble}
\include{qm2pi.local} 

%\ifpdf
%\usepackage[pdftex]{graphicx}
%\else
%\usepackage{graphicx}
%\fi

 % \ifpdf
%  \usepackage{pdfsync}
%  \if


%\title{Brief Article}
%\author{David F. Snyder}
%\author{L.G. Meredith}

%\address{Dept. of Math., Texas State University--San Marcos, San Marcos, TX 78666}
       
\pagestyle{empty}


\begin{document}

\lstset{language=[Objective]Caml,frame=shadowbox}

\input{qm2pi.front}

% section front matter (end)

\input{qm2pi.intro} 
 
% section introduction (end)

% \input{qm2pi.knotations} 

% section notation (end)

\input{qm2pi.process.calculi} 

% section concurrent_process_calculi_and_spatial_logics_ (end)
    
%\input{qm2pi.knots2pi} 

%\input{qm2pi.trefoil} 

%\input{qm2pi.mainthm} 

% subsection basic_interpretation (end)

%\input{qm2pi.rho.presentation} 
\subsection{The syntax and semantics of the notation system}\label{sub:the_syntax_and_semantics_of_the_notation_system} % (fold)

We now summarize a technical presentation of the calculus that
embodies our theory of dynamics. The typical presentation of such a
calculus follows the style of giving generators and relations on
them. The grammar, below, describing term constructors, freely
generates the set of processes, $\Proc$. This set is then quotiented
by a relation known as structural congruence and it is over this set
that the notion of dynamics is expressed. This presentation is
essentially that of \cite{MeredithR05} with the addition of
polyadicity and summation. For readability we have relegated some of
the technical subtleties to an appendix.

\subsubsection{Process grammar}\label{subsub:process_grammar}

\begin{mathpar}
  \inferrule* [lab=synchronization] {} {{M} \bc \pzero \;|\; x?F \;|\; x!C }
  \and
  \inferrule* [lab=abstraction] {} {{F} \bc (x)P}
  \and
  \inferrule* [lab=concretion] {} {{C} \bc \langle Q \rangle}
  \and
  \inferrule* [lab=process] {} {{P,Q} \bc M \;| \;P|Q \;|\; @{x}}
  \and
  \inferrule* [lab=name] {} {{x} \bc \quotep{P}}
\end{mathpar} 

Note that $\vec{x}$ (resp. $\vec{P}$) denotes a vector of names
(resp. processes) of length $|\vec{x}|$ (resp. $|\vec{P}|$). We adopt
the following useful abbreviations.

\begin{mathpar}
   x?(\vec{y}).P := x.(\vec{y})P \and  x\clift{\vec{P}} := x.\clift{\vec{P}}
   \and x!(y) := \lift{x}{\dropn{y}}
   \and \Pi_{i=0}^{n-1}P_i := P_0 | \ldots | P_{n-1}
\end{mathpar}

\subsubsection{Structural congruence}

\paragraph{Free and bound names and alpha-equivalence.} At the
core of structural equivalence is alpha-equivalence which identifies
process that are the same up to a change of variable. Formally, we
recognize the distinction between free and bound names. The free names
of a process, $\freenames{P}$, may be calculated recursively as
follows:

\begin{mathpar}
\freenames{\pzero} := \emptyset
  \and \\
  \freenames{x?(y).P} := \{ x \} \cup (\freenames{P} \setminus \{ y \})
  \and 
  \freenames{x!\langle P \rangle} := \{ x \} \cup \{ P \} 
  \and \\
  \freenames{P|Q} := \freenames{P} \cup \freenames{Q}
  \and \\
  \freenames{@{x}} := \{ x \}
\end{mathpar}

$\pi$
$\quotep{\pi}$

$\freenames{-} : \pi \to \mathcal{P}(\quotep{\pi})$

\begin{eqnarray*}
  \freenames{\pzero} & := & \emptyset \\
  \freenames{x?(y).P} & := & \{ x \} \cup (\freenames{P} \setminus \{ y \}) \\
  \freenames{x!\langle P \rangle} & := & \{ x \} \cup \{ P \} \\
  \freenames{P|Q} & := & \freenames{P} \cup \freenames{Q} \\
  \freenames{\dropn{x}} & := & \{ x \}
\end{eqnarray*}

The bound names of a process, $\boundnames{P}$, are those names occurring in $P$
that are not free. For example, in $x?(y).0$, the name $x$ is free, while $y$ is bound.

\begin{mathpar}
  \inferrule* [lab=monoidal-laws] {} { P|Q \equiv Q|P \and P|0 \equiv P \and P|(Q|R) \equiv (P|Q)|R }
\end{mathpar}

\begin{mathpar}
  \inferrule* [lab=alpha-equivalence] {} { (x)P \equiv (y)P\{y/x\} \and y \not\in \freenames{P} }
\end{mathpar}

\begin{definition}
Then two processes, $P,Q$, are alpha-equivalent if $P = Q\{\vec{y}/\vec{x}\}$ for
some $\vec{x} \in \boundnames{Q},\vec{y} \in \boundnames{P}$, where $Q\{\vec{y}/\vec{x}\}$
denotes the capture-avoiding substitution of $\vec{y}$ for $\vec{x}$ in $Q$.
\end{definition}

\begin{definition}
  The {\em structural congruence} \cite{SangiorgiWalker} , $\equiv$,
  between processes is the least congruence containing
  alpha-equivalence, satisfying the abelian monoid laws
  (associativity, commutativity and $\pzero$ as identity) for parallel
  composition $|$ and for summation $+$.
\end{definition}

\subsection{Name equivalence}

We take name equivalence, written $\nameeq$, to be the smallest
equivalence relation generated by the following rules.

\begin{mathpar}
\inferrule*[lab=Quote-drop]
{ }
{ \quotep{@{x}} \nameeq x }

\inferrule*[lab=Struct-equiv]
{ P \scong Q }
{ \quotep{P} \nameeq \quotep{Q} }
\end{mathpar}

The astute reader will have noticed that the mutual recursion of names
and processes imposes a mutual recursion on alpha-equivalence and
structural equivalence via name-equivalence. Fortunately, all of this
works out pleasantly and we may calculate in the natural way, free of
concern. The reader interested in the details is referred to the
appendix \ref{appendix:rho_details}.

\subsection{Substitution}

We use $\Proc$ for the set of processes, $\QProc$ for the set of
names, and $\id{\{}\vec{y} / \vec{x} \id{\}}$ to denote partial maps,
$s : \QProc \rightarrow \QProc$. A map, $s$ lifts, uniquely, to a map
on process terms, $\widehat{s} : \Proc \rightarrow \Proc$ by the
following equations.

\begin{mathpar}
  (0) \psubstp{Q}{P} := 0 \\
  (R \juxtap S) \psubstp{Q}{P}
  :=    
  (R)\psubstp{Q}{P} \juxtap (S) \psubstp{Q}{P} \\
  (x?(y).R) \psubstp{Q}{P}    
  :=    
  (x)\substp{Q}{P} (z)\concat( (R \psubstn{z}{y}) \psubstp{Q}{P} ) \\
  (\lift{x}{R}) \psubstp{Q}{P}  
  :=
  \lift{(x)\substp{Q}{P}}{ R \psubstp{Q}{P} } \\
%   (\dropn{x})  \psubstp{Q}{P}       
%   := 
%   \left\{ 
%     \begin{array}{ccc} 
%       \dropn{\quotep{Q}} & & x \nameeq \quotep{P} \\
%       \dropn{x} & & otherwise \\
%     \end{array}
%   \right. 
  (\dropn{x})  \psubstp{Q}{P}       
  := 
  \left\{ 
    \begin{array}{ccc} 
      Q & & x \nameeq \quotep{P} \\
      \dropn{x} & & otherwise \\
    \end{array}
  \right.
\end{mathpar}
 

where

\begin{eqnarray}
  (x)\id{\{} \lpquote Q \rpquote / \lpquote P \rpquote \id{\}}            = 
  \left\{ 
    \begin{array}{ccc}
      \lpquote Q \rpquote & & x \nameeq \lpquote P \rpquote \\
      x & & otherwise \\
    \end{array}
  \right. \nonumber
\end{eqnarray}

and $z$ is chosen distinct from $\quotep{P}$, $\quotep{Q}$, the free
names in $Q$, and all the names in $R$. Our $\alpha$-equivalence will
be built in the standard way from this substitution.

\begin{remark}\label{rem:no_self_referential_names}
  One consequence of these definitions is that $\forall P. \quotep{P}
  \not\in \freenames{P}$.
\end{remark}

\subsection{ Dynamic quote: an example }

Anticipating something of what's to come, consider applying the
substitution, $\widehat{\id{\{}u / z \id{\}}}$, to the following pair
of processes, $\lift{w}{y!(z)}$ and $w[ \lpquote y!(z) \rpquote ]$.

\begin{eqnarray}
	\lift{w}{y!(z)}\widehat{\id{\{}u / z \id{\}}}
		& = &
		\lift{w}{y!(u)} \nonumber\\
	w[ \lpquote y!(z) \rpquote ] \widehat{ \id{\{}u / z \id{\}} }
		& = &
		w[ \lpquote y!(z) \rpquote ] \nonumber
\end{eqnarray}

Because the body of the process between quotes is impervious to
substitution, we get radically different answers. In fact, by
examining the first process in an input context,
e.g. $x?(z).\lift{w}{y!(z)}$, we see that the process under the lift
operator may be shaped by prefixed inputs binding a name inside it. In
this sense, the lift operator will be seen as a way to dynamically
construct processes before reifying them as names.

Finally equipped with these standard features we can present the
dynamics of the calculus.

\subsubsection{Operational semantics} 

Finally, we introduce the computational dynamics. What marks these
algebras as distinct from other more traditionally studied algebraic
structures, e.g. vector spaces or polynomial rings, is the manner in
which dynamics is captured. In traditional structures, dynamics is typically
expressed through morphisms between such structures, as in linear maps
between vector spaces or morphisms between rings. In algebras
associated with the semantics of computation, the dynamics is
expressed as part of the algebraic structure itself, through a
reduction reduction relation typically denoted by $\red$. Below, we
give a recursive presentation of this relation for the calculus used
in the encoding.

$\red \subseteq \pi \times \pi$
$\red : \pi \to \mathcal{P}(\pi)$

\begin{mathpar}
  \inferrule* [lab=Comm] { \textsf{match}( x_{src}, x_{trgt} ) } { x_{trgt}?(y)P \; | \; x_{src}!\langle {Q} \rangle \red P\{\quotep{Q}/y}\} }
  \and \\
  \inferrule* [lab=Par] {{P} \red {P}'} {{{P} | {Q}} \red {{P}' | {Q}}}
  \and
  \inferrule* [lab=Equiv]{{{P} \scong {P}'} \andalso {{P}' \red {Q}'} \andalso {{Q}' \scong {Q}}}{{P} \red {Q}}
\end{mathpar}

\begin{eqnarray*}
  match_{\equiv} (\quotep{P},\quotep{Q}) & := & P \equiv Q \\
  match_{\dagger}(\quotep{P},\quotep{Q}) & := & \forall R. P|Q \red^{*} R => R \red^{*} 0 \\
  match_{K}(\quotep{P},\quotep{Q}) & := & K \mbox{ for some context } K
\end{eqnarray*}

$u?(x)P | u!\langle Q \rangle \red P\{\quotep{Q}/x\}$

%We write $\wred$ for $\red^*$, and $P\red$ if $\exists Q $ such that $ P \red Q$.
We write $P\red$ if $\exists Q $ such that $ P \red Q$ and $P\not\red$, otherwise.

\section{Replication}

As mentioned before, it is known that replication (and hence
recursion) can be implemented in a higher-order process algebra
\cite{SangiorgiWalker}. As our first example of calculation with the
machinery thus far presented we give the construction explicitly in
the {\rhoc}.

\begin{eqnarray}
	D_{x} & := & \prefix{x}{y}{(\binpar{\outputp{x}{y}}{@{y}})} \nonumber\\
	\bangp_{x}{P} & := & \binpar{{x}!\langle{\binpar{D_{x}}{P}}\rangle}{D_{x}} \nonumber
\end{eqnarray}

\begin{eqnarray}
	\bangp_{x}{P} & & \nonumber\\
	=
	& {x}!\langle{(\prefix{x}{y}{(\outputp{x}{y} | @{y})) | P}}\rangle 
	      | \prefix{x}{y}{(\outputp{x}{y} | @{y})} & \nonumber\\
	\red
	& (\outputp{x}{y} | @{y})\substn{\quotep{(\prefix{x}{y}{(@{y} | \outputp{x}{y})) | P}}}{y} & \nonumber\\
	=
	& \outputp{x}{\quotep{(\prefix{x}{y}{(\outputp{x}{y} | @{y})) | P}}}
	  | {(\prefix{x}{y}{(\outputp{x}{y} | @{y})) | P}} & \nonumber\\
	\red
	& \ldots & \nonumber\\
	\red^*
	& P | P | \ldots & \nonumber
\end{eqnarray}

Of course, this encoding, as an implementation, runs away, unfolding
$\bangp{P}$ eagerly. A lazier and more implementable replication
operator, restricted to input-guarded processes, may be obtained as follows.

\begin{eqnarray}
\bangp{\prefix{u}{v}{P}} 
	:= 
	\binpar{\lift{x}{\prefix{u}{v}{(\binpar{D(x)}{P})}}}{D(x)} \nonumber
\end{eqnarray}

\begin{remark}
  Note that the lazier definition still does not deal with summation
  or mixed summation (i.e. sums over input and output). The reader is
  invited to construct definitions of replication that deal with these
  features. 

  Further, the definitions are parameterized in a name, $x$. Can you,
  gentle reader, make a definition that eliminates this parameter and
  guarantees no accidental interaction between the replication
  machinery and the process being replicated -- i.e. no accidental
  sharing of names used by the process to get its work done and the
  name(s) used by the replication to effect copying. This latter
  revision of the definition of replication is crucial to obtaining
  the expected identity $!!P \sim !P$.
\end{remark}

\begin{remark}\label{rem:paradoxical_combinator}
  The reader familiar with the lambda calculus will have noticed the
  similarity between $D$ and the paradoxical combinator.

  [Ed. note: the existence of this seems to suggest we have to be more
  restrictive on the set of processes and names we admit if we are to
  support no-cloning.]
\end{remark}

\subsubsection{Bisimulation}

The computational dynamics gives rise to another kind of equivalence,
the equivalence of computational behavior. As previously mentioned
this is typically captured \emph{via} some form of bisimulation.

% The notion we use in this paper is weak barbed bisimulation
% \cite{milner91polyadicpi}.

The notion we use in this paper is derived from weak barbed
bisimulation \cite{milner91polyadicpi}. 

\begin{definition}
An \emph{observation relation}, $\downarrow_{\mathcal N}$, over a set
of names, $\mathcal N$, is the smallest relation satisfying the rules
below.

\infrule[Out-barb]{y \in {\mathcal N}, \; x \nameeq y}
		  {\outputp{x}{v} \downarrow_{\mathcal N} x}
\infrule[Par-barb]{\mbox{$P\downarrow_{\mathcal N} x$ or $Q\downarrow_{\mathcal N} x$}}
		  {\binpar{P}{Q} \downarrow_{\mathcal N} x}

We write $P \Downarrow_{\mathcal N} x$ if there is $Q$ such that 
$P \wred Q$ and $Q \downarrow_{\mathcal N} x$.
\end{definition}

\begin{definition}
%\label{def.bbisim}
An  ${\mathcal N}$-\emph{barbed bisimulation} over a set of names, ${\mathcal N}$, is a symmetric binary relation 
${\mathcal S}_{\mathcal N}$ between agents such that $P\rel{S}_{\mathcal N}Q$ implies:
\begin{enumerate}
\item If $P \red P'$ then $Q \wred Q'$ and $P'\rel{S}_{\mathcal N} Q'$.
\item If $P\downarrow_{\mathcal N} x$, then $Q\Downarrow_{\mathcal N} x$.
\end{enumerate}
$P$ is ${\mathcal N}$-barbed bisimilar to $Q$, written
$P \wbbisim_{\mathcal N} Q$, if $P \rel{S}_{\mathcal N} Q$ for some ${\mathcal N}$-barbed bisimulation ${\mathcal S}_{\mathcal N}$.
\end{definition}

$\mathcal{R} \subseteq \pi \times \pi$

$P \mathcal{R} Q => \forall P'. P \red P' \Rightarrow \exists Q'. Q \red Q', P' \mathcal{R} Q'$

$P \vdash x \Rightarrow Q \vdash x$

\begin{mathpar}
  \inferrule*[lab=Out-barb]{x \nameeq y}{{y}!\langle{Q}\rangle \vdash x}
  \and
  \inferrule*[lab=Par-barb]{\mbox{$P\vdash x$ or $Q\vdash x$}}{\binpar{P}{Q} \vdash x}
\end{mathpar}

\subsubsection{Contexts}

One of the principle advantages of computational calculi like the
$\pi$-calculus is a well-defined notion of context,
contextual-equivalence and a correlation between
contextual-equivalence and notions of bisimulation. The notion of
context allows the decomposition of a process into (sub-)process and
its syntactic environment, its context. Thus, a context may be
thought of as a process with a ``hole'' (written $\Box$) in it. The
application of a context $M$ to a process $P$, written $M[P]$, is
tantamount to filling the hole in $M$ with $P$. In this paper we do
not need the full weight of this theory, but do make use of the notion
of context in the proof the main theorem. 

\begin{mathpar}
  \inferrule* [lab=summation] {} {{M_{M},M_{N}} \bc \Box \;|\; x.M_{A} \;|\; M_{M}+M_{N}}
  \and
  \inferrule* [lab=agent] {} {{M_{A}} \bc (\vec{x})M_{P} \;| \; \clift{P_0,\ldots,M_{P},\ldots,P_N}}
  \and \\
  \inferrule* [lab=process] {} {{M_{P}} \bc M_{N} \;| \;P|M_{P} }
\end{mathpar} 

\begin{mathpar}
  \inferrule* [lab=sychronization] {} {M_{N} \bc \Box \;|\; x?M_{F} \;|\; x!M_{C}}
  \and
  \inferrule* [lab=abstraction] {} {{M_{F}} \bc (x)M_{P} }
  \and
  \inferrule* [lab=concretion] {} {{M_{C}} \bc \langle M_{P} \rangle }
  \and \\
  \inferrule* [lab=process] {} {{M_{P}} \bc M_{N} \;| \;P|M_{P} }
\end{mathpar}

\begin{definition}[contextual application] Given a context $M$, and
  process $P$, we define the \emph{contextual application}, $M[P] :=
  M\{P/\Box\}$. That is, the contextual application of M to P is the
  substitution of $P$ for $\Box$ in $M$.
\end{definition}

$\meaningof{-} : L \to \mathcal{P}(\pi)$

\begin{mathpar}
  \inferrule* [lab=collection] {} {\meaningof{true} = \pi, \and \meaningof{~E} = \pi \setminus \meaningof{E}, \and \meaningof{E_{1} \& E_{2}} = \meaningof{E_{1}} \cap \meaningof{E_{2}}}
\end{mathpar}

\begin{mathpar}
  \inferrule* [lab=structure] {} {\meaningof{0} = \{ P \in \pi | P \equiv 0 \}, \and \\ \meaningof{E_1 | E_2} = \{ P \in \pi | P \equiv P_{1} | P_{2}, P_{1} \in \meaningof{E_{1}}, P_{2} \in \meaningof{E_2}\} }
\end{mathpar}

\begin{mathpar}
 \inferrule* [lab=behavior] {} {\meaningof{\langle a?b \rangle E} = \{ P \in \pi | P \equiv Q | u?(y)P', \\ \and \\\\ \and \\ \;\;\; u \in \meaningof{a}, \forall z.P'\{z/y\} \in \meaningof{E\{z/b\}}\}, \and \\ \meaningof{a!E} = \{ P \in \pi | P \equiv Q | x!\langle P' \rangle, x \in \meaningof{a} P' \in \meaningof{E}\} }
\end{mathpar}

\begin{mathpar}
 \inferrule* [lab=nominal] {} {\meaningof{\quotep{E}} = \{ \quotep{P} \in \quotep{\pi} | P \in \meaningof{E} \}, \and \meaningof{\quotep{P}} = \{ \quotep{Q} \in \quotep{\pi} | P \equiv Q \} \and \\ \meaningof{@\quotep{E}} = \{ P \in \pi | P \equiv @x, x \in \meaningof{E} \}}
\end{mathpar}

\begin{eqnarray*}
  \\
  \meaningof{-} : TS \to ST
\end{eqnarray*}

\begin{eqnarray*}
  \\
  L : TS \to ST
\end{eqnarray*}

\begin{eqnarray*}
  \\
  P \models E \iff P \in \meaningof{E}
\end{eqnarray*}

\begin{eqnarray*}
  P \approx_{L} Q \iff \forall E \in L. P \models E \iff Q \models E
\end{eqnarray*}

\begin{eqnarray*}
  P \approx_{K} Q
\end{eqnarray*}

\begin{eqnarray*}
  P \approx Q
\end{eqnarray*}

$\approx_{K} = \approx = \approx_{L}$

\subsubsection{Contextual duality}

Note that contexts extend the quotation operation to a family of
operations from processes to names. Given a context, $M$, we can
define a \emph{nominal context}, $\quotep{M}$ by $\quotep{M}[P] :=
\quotep{M[P]}$. To foreshadow what is to come we observe that these
operations enjoy a duality with processes very much like the duality
between vectors and maps from vectors to scalars.

Further, because the calculus is essentially higher-order, we have a
correspondence between contexts and processes. More specifically,
given a name $x$ and a context $M$ we can construct $M^{*}_{x}$ such
that 

\begin{mathpar}
  M^{*}_{x} | \lift{x}{P} \red M[P]
\end{mathpar}

namely,

\begin{mathpar}
  M^{*}_{x} := x?(u).M[\dropn{u}]
\end{mathpar}

The dependence of $M^{*}_{x}$ on a name makes it an abstraction, 

\begin{mathpar}
  M^{*} := (x)x?(u).M[\dropn{u}]
\end{mathpar}

\subsection{Additional notation}

It will sometimes be convenient to denote the process a name
quotes. We already have the notation $x = \quotep{P}$, but it will be
convenient to introduce an alternate notation, $\procn{x}$, when we
want to emphasize the connection to the use of the name. Note that, by
virtue of name equivalence, $\quotep{\procn{x}} \nameeq x$; so, the
notation is consistent with previous definitions.

Further, because names have structure it is possible to effect
substitutions on the basis of that structure. This means we need to
upgrade our notation for substitutions, which we accomplish by
adapting comprehension notation. Thus,

\begin{mathpar}
  P\{ y / x : x \in S \}
\end{mathpar}

is interpreted to mean the process derived from P by replacing (in a
capture-avoiding manner) each occurrence of $x$ in $S$ by $y$. For example,

\begin{mathpar}
  P\{ \quotep{\procn{x}|\procn{x}} / x : x \in \freenames{P} \}
\end{mathpar}

will replace each (occurrence) of a free name $x$ in $P$ by
$\quotep{\procn{x}|\procn{x}}$.

Also, we will avail ourselves of the notation $x^{L}$ and $x^{R}$ to
denote injections of a name into disjoint copies of the name
space. There are numerous ways to accomplish this. One example can be
found in \cite{MeredithR05}. This notation overloads to vectors of
names: $\vec{x}^{\pi} := (x_{i}^{\pi} \; : \; 0 \leq i < |\vec{x}| )$ where $\pi \in \{L,R\}$.

We also use $P^{\Box} := P|\Box$.

In \cite{MeredithR05} an interpretation of the new operator is
given. It turns out that there are several possible interpretations
all enjoying the requisite algebraic properties of the operator (see
\cite{milner91polyadicpi}). We will therefore make liberal use of
$(\nu\; \vec{x})P$.

% subsection the_syntax_and_semantics_of_the_notation_system (end)   

\input{qm2pi.qmops} 

\input{qm2pi.sterngerlach} 

\input{qm2pi.metric} 

% section concurrent_process_calculi (end)

%\input{qm2pi.proofsketch}

% section proof sketch (end)

%\input{qm2pi.slviaknots} 

% section spatial logic via knots (end)

\input{qm2pi.conclusion}

% section conclusion (end)

%\input{qm2pi.dtcodes} 

% section wiring algorithm (end)

\input{qm2pi.ack} 

% section acknowledgments (end)

\newpage


\bibliographystyle{plain}   
\bibliography{../../biblios/main.bib}

\input{qm2pi.rhodetails}

\end{document}

 

% section notation (end)

\input{qm2pi.process.calculi} 

% section concurrent_process_calculi_and_spatial_logics_ (end)
    
%\documentclass[12pt]{llncs}
%\documentclass{jktr}

\usepackage[pdftex]{hyperref}                   
\usepackage {listings}
\usepackage {mathpartir}
\usepackage{bcprules}
%\usepackage{listings}
                       
\usepackage{graphicx} 
%\usepackage[margins=2.5cm,nohead,nofoot]{geometry}
%\usepackage{geometry}
\usepackage{amsfonts}
\usepackage{amstext}
\usepackage{latexsym}
\usepackage{amssymb}
\usepackage{color}


%\include{myPreamble}
\include{qm2pi.local} 

%\ifpdf
%\usepackage[pdftex]{graphicx}
%\else
%\usepackage{graphicx}
%\fi

 % \ifpdf
%  \usepackage{pdfsync}
%  \if


%\title{Brief Article}
%\author{David F. Snyder}
%\author{L.G. Meredith}

%\address{Dept. of Math., Texas State University--San Marcos, San Marcos, TX 78666}
       
\pagestyle{empty}


\begin{document}

\lstset{language=[Objective]Caml,frame=shadowbox}

\input{qm2pi.front}

% section front matter (end)

\input{qm2pi.intro} 
 
% section introduction (end)

% \input{qm2pi.knotations} 

% section notation (end)

\input{qm2pi.process.calculi} 

% section concurrent_process_calculi_and_spatial_logics_ (end)
    
%\input{qm2pi.knots2pi} 

%\input{qm2pi.trefoil} 

%\input{qm2pi.mainthm} 

% subsection basic_interpretation (end)

%\input{qm2pi.rho.presentation} 
\subsection{The syntax and semantics of the notation system}\label{sub:the_syntax_and_semantics_of_the_notation_system} % (fold)

We now summarize a technical presentation of the calculus that
embodies our theory of dynamics. The typical presentation of such a
calculus follows the style of giving generators and relations on
them. The grammar, below, describing term constructors, freely
generates the set of processes, $\Proc$. This set is then quotiented
by a relation known as structural congruence and it is over this set
that the notion of dynamics is expressed. This presentation is
essentially that of \cite{MeredithR05} with the addition of
polyadicity and summation. For readability we have relegated some of
the technical subtleties to an appendix.

\subsubsection{Process grammar}\label{subsub:process_grammar}

\begin{mathpar}
  \inferrule* [lab=synchronization] {} {{M} \bc \pzero \;|\; x?F \;|\; x!C }
  \and
  \inferrule* [lab=abstraction] {} {{F} \bc (x)P}
  \and
  \inferrule* [lab=concretion] {} {{C} \bc \langle Q \rangle}
  \and
  \inferrule* [lab=process] {} {{P,Q} \bc M \;| \;P|Q \;|\; @{x}}
  \and
  \inferrule* [lab=name] {} {{x} \bc \quotep{P}}
\end{mathpar} 

Note that $\vec{x}$ (resp. $\vec{P}$) denotes a vector of names
(resp. processes) of length $|\vec{x}|$ (resp. $|\vec{P}|$). We adopt
the following useful abbreviations.

\begin{mathpar}
   x?(\vec{y}).P := x.(\vec{y})P \and  x\clift{\vec{P}} := x.\clift{\vec{P}}
   \and x!(y) := \lift{x}{\dropn{y}}
   \and \Pi_{i=0}^{n-1}P_i := P_0 | \ldots | P_{n-1}
\end{mathpar}

\subsubsection{Structural congruence}

\paragraph{Free and bound names and alpha-equivalence.} At the
core of structural equivalence is alpha-equivalence which identifies
process that are the same up to a change of variable. Formally, we
recognize the distinction between free and bound names. The free names
of a process, $\freenames{P}$, may be calculated recursively as
follows:

\begin{mathpar}
\freenames{\pzero} := \emptyset
  \and \\
  \freenames{x?(y).P} := \{ x \} \cup (\freenames{P} \setminus \{ y \})
  \and 
  \freenames{x!\langle P \rangle} := \{ x \} \cup \{ P \} 
  \and \\
  \freenames{P|Q} := \freenames{P} \cup \freenames{Q}
  \and \\
  \freenames{@{x}} := \{ x \}
\end{mathpar}

$\pi$
$\quotep{\pi}$

$\freenames{-} : \pi \to \mathcal{P}(\quotep{\pi})$

\begin{eqnarray*}
  \freenames{\pzero} & := & \emptyset \\
  \freenames{x?(y).P} & := & \{ x \} \cup (\freenames{P} \setminus \{ y \}) \\
  \freenames{x!\langle P \rangle} & := & \{ x \} \cup \{ P \} \\
  \freenames{P|Q} & := & \freenames{P} \cup \freenames{Q} \\
  \freenames{\dropn{x}} & := & \{ x \}
\end{eqnarray*}

The bound names of a process, $\boundnames{P}$, are those names occurring in $P$
that are not free. For example, in $x?(y).0$, the name $x$ is free, while $y$ is bound.

\begin{mathpar}
  \inferrule* [lab=monoidal-laws] {} { P|Q \equiv Q|P \and P|0 \equiv P \and P|(Q|R) \equiv (P|Q)|R }
\end{mathpar}

\begin{mathpar}
  \inferrule* [lab=alpha-equivalence] {} { (x)P \equiv (y)P\{y/x\} \and y \not\in \freenames{P} }
\end{mathpar}

\begin{definition}
Then two processes, $P,Q$, are alpha-equivalent if $P = Q\{\vec{y}/\vec{x}\}$ for
some $\vec{x} \in \boundnames{Q},\vec{y} \in \boundnames{P}$, where $Q\{\vec{y}/\vec{x}\}$
denotes the capture-avoiding substitution of $\vec{y}$ for $\vec{x}$ in $Q$.
\end{definition}

\begin{definition}
  The {\em structural congruence} \cite{SangiorgiWalker} , $\equiv$,
  between processes is the least congruence containing
  alpha-equivalence, satisfying the abelian monoid laws
  (associativity, commutativity and $\pzero$ as identity) for parallel
  composition $|$ and for summation $+$.
\end{definition}

\subsection{Name equivalence}

We take name equivalence, written $\nameeq$, to be the smallest
equivalence relation generated by the following rules.

\begin{mathpar}
\inferrule*[lab=Quote-drop]
{ }
{ \quotep{@{x}} \nameeq x }

\inferrule*[lab=Struct-equiv]
{ P \scong Q }
{ \quotep{P} \nameeq \quotep{Q} }
\end{mathpar}

The astute reader will have noticed that the mutual recursion of names
and processes imposes a mutual recursion on alpha-equivalence and
structural equivalence via name-equivalence. Fortunately, all of this
works out pleasantly and we may calculate in the natural way, free of
concern. The reader interested in the details is referred to the
appendix \ref{appendix:rho_details}.

\subsection{Substitution}

We use $\Proc$ for the set of processes, $\QProc$ for the set of
names, and $\id{\{}\vec{y} / \vec{x} \id{\}}$ to denote partial maps,
$s : \QProc \rightarrow \QProc$. A map, $s$ lifts, uniquely, to a map
on process terms, $\widehat{s} : \Proc \rightarrow \Proc$ by the
following equations.

\begin{mathpar}
  (0) \psubstp{Q}{P} := 0 \\
  (R \juxtap S) \psubstp{Q}{P}
  :=    
  (R)\psubstp{Q}{P} \juxtap (S) \psubstp{Q}{P} \\
  (x?(y).R) \psubstp{Q}{P}    
  :=    
  (x)\substp{Q}{P} (z)\concat( (R \psubstn{z}{y}) \psubstp{Q}{P} ) \\
  (\lift{x}{R}) \psubstp{Q}{P}  
  :=
  \lift{(x)\substp{Q}{P}}{ R \psubstp{Q}{P} } \\
%   (\dropn{x})  \psubstp{Q}{P}       
%   := 
%   \left\{ 
%     \begin{array}{ccc} 
%       \dropn{\quotep{Q}} & & x \nameeq \quotep{P} \\
%       \dropn{x} & & otherwise \\
%     \end{array}
%   \right. 
  (\dropn{x})  \psubstp{Q}{P}       
  := 
  \left\{ 
    \begin{array}{ccc} 
      Q & & x \nameeq \quotep{P} \\
      \dropn{x} & & otherwise \\
    \end{array}
  \right.
\end{mathpar}
 

where

\begin{eqnarray}
  (x)\id{\{} \lpquote Q \rpquote / \lpquote P \rpquote \id{\}}            = 
  \left\{ 
    \begin{array}{ccc}
      \lpquote Q \rpquote & & x \nameeq \lpquote P \rpquote \\
      x & & otherwise \\
    \end{array}
  \right. \nonumber
\end{eqnarray}

and $z$ is chosen distinct from $\quotep{P}$, $\quotep{Q}$, the free
names in $Q$, and all the names in $R$. Our $\alpha$-equivalence will
be built in the standard way from this substitution.

\begin{remark}\label{rem:no_self_referential_names}
  One consequence of these definitions is that $\forall P. \quotep{P}
  \not\in \freenames{P}$.
\end{remark}

\subsection{ Dynamic quote: an example }

Anticipating something of what's to come, consider applying the
substitution, $\widehat{\id{\{}u / z \id{\}}}$, to the following pair
of processes, $\lift{w}{y!(z)}$ and $w[ \lpquote y!(z) \rpquote ]$.

\begin{eqnarray}
	\lift{w}{y!(z)}\widehat{\id{\{}u / z \id{\}}}
		& = &
		\lift{w}{y!(u)} \nonumber\\
	w[ \lpquote y!(z) \rpquote ] \widehat{ \id{\{}u / z \id{\}} }
		& = &
		w[ \lpquote y!(z) \rpquote ] \nonumber
\end{eqnarray}

Because the body of the process between quotes is impervious to
substitution, we get radically different answers. In fact, by
examining the first process in an input context,
e.g. $x?(z).\lift{w}{y!(z)}$, we see that the process under the lift
operator may be shaped by prefixed inputs binding a name inside it. In
this sense, the lift operator will be seen as a way to dynamically
construct processes before reifying them as names.

Finally equipped with these standard features we can present the
dynamics of the calculus.

\subsubsection{Operational semantics} 

Finally, we introduce the computational dynamics. What marks these
algebras as distinct from other more traditionally studied algebraic
structures, e.g. vector spaces or polynomial rings, is the manner in
which dynamics is captured. In traditional structures, dynamics is typically
expressed through morphisms between such structures, as in linear maps
between vector spaces or morphisms between rings. In algebras
associated with the semantics of computation, the dynamics is
expressed as part of the algebraic structure itself, through a
reduction reduction relation typically denoted by $\red$. Below, we
give a recursive presentation of this relation for the calculus used
in the encoding.

$\red \subseteq \pi \times \pi$
$\red : \pi \to \mathcal{P}(\pi)$

\begin{mathpar}
  \inferrule* [lab=Comm] { \textsf{match}( x_{src}, x_{trgt} ) } { x_{trgt}?(y)P \; | \; x_{src}!\langle {Q} \rangle \red P\{\quotep{Q}/y}\} }
  \and \\
  \inferrule* [lab=Par] {{P} \red {P}'} {{{P} | {Q}} \red {{P}' | {Q}}}
  \and
  \inferrule* [lab=Equiv]{{{P} \scong {P}'} \andalso {{P}' \red {Q}'} \andalso {{Q}' \scong {Q}}}{{P} \red {Q}}
\end{mathpar}

\begin{eqnarray*}
  match_{\equiv} (\quotep{P},\quotep{Q}) & := & P \equiv Q \\
  match_{\dagger}(\quotep{P},\quotep{Q}) & := & \forall R. P|Q \red^{*} R => R \red^{*} 0 \\
  match_{K}(\quotep{P},\quotep{Q}) & := & K \mbox{ for some context } K
\end{eqnarray*}

$u?(x)P | u!\langle Q \rangle \red P\{\quotep{Q}/x\}$

%We write $\wred$ for $\red^*$, and $P\red$ if $\exists Q $ such that $ P \red Q$.
We write $P\red$ if $\exists Q $ such that $ P \red Q$ and $P\not\red$, otherwise.

\section{Replication}

As mentioned before, it is known that replication (and hence
recursion) can be implemented in a higher-order process algebra
\cite{SangiorgiWalker}. As our first example of calculation with the
machinery thus far presented we give the construction explicitly in
the {\rhoc}.

\begin{eqnarray}
	D_{x} & := & \prefix{x}{y}{(\binpar{\outputp{x}{y}}{@{y}})} \nonumber\\
	\bangp_{x}{P} & := & \binpar{{x}!\langle{\binpar{D_{x}}{P}}\rangle}{D_{x}} \nonumber
\end{eqnarray}

\begin{eqnarray}
	\bangp_{x}{P} & & \nonumber\\
	=
	& {x}!\langle{(\prefix{x}{y}{(\outputp{x}{y} | @{y})) | P}}\rangle 
	      | \prefix{x}{y}{(\outputp{x}{y} | @{y})} & \nonumber\\
	\red
	& (\outputp{x}{y} | @{y})\substn{\quotep{(\prefix{x}{y}{(@{y} | \outputp{x}{y})) | P}}}{y} & \nonumber\\
	=
	& \outputp{x}{\quotep{(\prefix{x}{y}{(\outputp{x}{y} | @{y})) | P}}}
	  | {(\prefix{x}{y}{(\outputp{x}{y} | @{y})) | P}} & \nonumber\\
	\red
	& \ldots & \nonumber\\
	\red^*
	& P | P | \ldots & \nonumber
\end{eqnarray}

Of course, this encoding, as an implementation, runs away, unfolding
$\bangp{P}$ eagerly. A lazier and more implementable replication
operator, restricted to input-guarded processes, may be obtained as follows.

\begin{eqnarray}
\bangp{\prefix{u}{v}{P}} 
	:= 
	\binpar{\lift{x}{\prefix{u}{v}{(\binpar{D(x)}{P})}}}{D(x)} \nonumber
\end{eqnarray}

\begin{remark}
  Note that the lazier definition still does not deal with summation
  or mixed summation (i.e. sums over input and output). The reader is
  invited to construct definitions of replication that deal with these
  features. 

  Further, the definitions are parameterized in a name, $x$. Can you,
  gentle reader, make a definition that eliminates this parameter and
  guarantees no accidental interaction between the replication
  machinery and the process being replicated -- i.e. no accidental
  sharing of names used by the process to get its work done and the
  name(s) used by the replication to effect copying. This latter
  revision of the definition of replication is crucial to obtaining
  the expected identity $!!P \sim !P$.
\end{remark}

\begin{remark}\label{rem:paradoxical_combinator}
  The reader familiar with the lambda calculus will have noticed the
  similarity between $D$ and the paradoxical combinator.

  [Ed. note: the existence of this seems to suggest we have to be more
  restrictive on the set of processes and names we admit if we are to
  support no-cloning.]
\end{remark}

\subsubsection{Bisimulation}

The computational dynamics gives rise to another kind of equivalence,
the equivalence of computational behavior. As previously mentioned
this is typically captured \emph{via} some form of bisimulation.

% The notion we use in this paper is weak barbed bisimulation
% \cite{milner91polyadicpi}.

The notion we use in this paper is derived from weak barbed
bisimulation \cite{milner91polyadicpi}. 

\begin{definition}
An \emph{observation relation}, $\downarrow_{\mathcal N}$, over a set
of names, $\mathcal N$, is the smallest relation satisfying the rules
below.

\infrule[Out-barb]{y \in {\mathcal N}, \; x \nameeq y}
		  {\outputp{x}{v} \downarrow_{\mathcal N} x}
\infrule[Par-barb]{\mbox{$P\downarrow_{\mathcal N} x$ or $Q\downarrow_{\mathcal N} x$}}
		  {\binpar{P}{Q} \downarrow_{\mathcal N} x}

We write $P \Downarrow_{\mathcal N} x$ if there is $Q$ such that 
$P \wred Q$ and $Q \downarrow_{\mathcal N} x$.
\end{definition}

\begin{definition}
%\label{def.bbisim}
An  ${\mathcal N}$-\emph{barbed bisimulation} over a set of names, ${\mathcal N}$, is a symmetric binary relation 
${\mathcal S}_{\mathcal N}$ between agents such that $P\rel{S}_{\mathcal N}Q$ implies:
\begin{enumerate}
\item If $P \red P'$ then $Q \wred Q'$ and $P'\rel{S}_{\mathcal N} Q'$.
\item If $P\downarrow_{\mathcal N} x$, then $Q\Downarrow_{\mathcal N} x$.
\end{enumerate}
$P$ is ${\mathcal N}$-barbed bisimilar to $Q$, written
$P \wbbisim_{\mathcal N} Q$, if $P \rel{S}_{\mathcal N} Q$ for some ${\mathcal N}$-barbed bisimulation ${\mathcal S}_{\mathcal N}$.
\end{definition}

$\mathcal{R} \subseteq \pi \times \pi$

$P \mathcal{R} Q => \forall P'. P \red P' \Rightarrow \exists Q'. Q \red Q', P' \mathcal{R} Q'$

$P \vdash x \Rightarrow Q \vdash x$

\begin{mathpar}
  \inferrule*[lab=Out-barb]{x \nameeq y}{{y}!\langle{Q}\rangle \vdash x}
  \and
  \inferrule*[lab=Par-barb]{\mbox{$P\vdash x$ or $Q\vdash x$}}{\binpar{P}{Q} \vdash x}
\end{mathpar}

\subsubsection{Contexts}

One of the principle advantages of computational calculi like the
$\pi$-calculus is a well-defined notion of context,
contextual-equivalence and a correlation between
contextual-equivalence and notions of bisimulation. The notion of
context allows the decomposition of a process into (sub-)process and
its syntactic environment, its context. Thus, a context may be
thought of as a process with a ``hole'' (written $\Box$) in it. The
application of a context $M$ to a process $P$, written $M[P]$, is
tantamount to filling the hole in $M$ with $P$. In this paper we do
not need the full weight of this theory, but do make use of the notion
of context in the proof the main theorem. 

\begin{mathpar}
  \inferrule* [lab=summation] {} {{M_{M},M_{N}} \bc \Box \;|\; x.M_{A} \;|\; M_{M}+M_{N}}
  \and
  \inferrule* [lab=agent] {} {{M_{A}} \bc (\vec{x})M_{P} \;| \; \clift{P_0,\ldots,M_{P},\ldots,P_N}}
  \and \\
  \inferrule* [lab=process] {} {{M_{P}} \bc M_{N} \;| \;P|M_{P} }
\end{mathpar} 

\begin{mathpar}
  \inferrule* [lab=sychronization] {} {M_{N} \bc \Box \;|\; x?M_{F} \;|\; x!M_{C}}
  \and
  \inferrule* [lab=abstraction] {} {{M_{F}} \bc (x)M_{P} }
  \and
  \inferrule* [lab=concretion] {} {{M_{C}} \bc \langle M_{P} \rangle }
  \and \\
  \inferrule* [lab=process] {} {{M_{P}} \bc M_{N} \;| \;P|M_{P} }
\end{mathpar}

\begin{definition}[contextual application] Given a context $M$, and
  process $P$, we define the \emph{contextual application}, $M[P] :=
  M\{P/\Box\}$. That is, the contextual application of M to P is the
  substitution of $P$ for $\Box$ in $M$.
\end{definition}

$\meaningof{-} : L \to \mathcal{P}(\pi)$

\begin{mathpar}
  \inferrule* [lab=collection] {} {\meaningof{true} = \pi, \and \meaningof{~E} = \pi \setminus \meaningof{E}, \and \meaningof{E_{1} \& E_{2}} = \meaningof{E_{1}} \cap \meaningof{E_{2}}}
\end{mathpar}

\begin{mathpar}
  \inferrule* [lab=structure] {} {\meaningof{0} = \{ P \in \pi | P \equiv 0 \}, \and \\ \meaningof{E_1 | E_2} = \{ P \in \pi | P \equiv P_{1} | P_{2}, P_{1} \in \meaningof{E_{1}}, P_{2} \in \meaningof{E_2}\} }
\end{mathpar}

\begin{mathpar}
 \inferrule* [lab=behavior] {} {\meaningof{\langle a?b \rangle E} = \{ P \in \pi | P \equiv Q | u?(y)P', \\ \and \\\\ \and \\ \;\;\; u \in \meaningof{a}, \forall z.P'\{z/y\} \in \meaningof{E\{z/b\}}\}, \and \\ \meaningof{a!E} = \{ P \in \pi | P \equiv Q | x!\langle P' \rangle, x \in \meaningof{a} P' \in \meaningof{E}\} }
\end{mathpar}

\begin{mathpar}
 \inferrule* [lab=nominal] {} {\meaningof{\quotep{E}} = \{ \quotep{P} \in \quotep{\pi} | P \in \meaningof{E} \}, \and \meaningof{\quotep{P}} = \{ \quotep{Q} \in \quotep{\pi} | P \equiv Q \} \and \\ \meaningof{@\quotep{E}} = \{ P \in \pi | P \equiv @x, x \in \meaningof{E} \}}
\end{mathpar}

\begin{eqnarray*}
  \\
  \meaningof{-} : TS \to ST
\end{eqnarray*}

\begin{eqnarray*}
  \\
  L : TS \to ST
\end{eqnarray*}

\begin{eqnarray*}
  \\
  P \models E \iff P \in \meaningof{E}
\end{eqnarray*}

\begin{eqnarray*}
  P \approx_{L} Q \iff \forall E \in L. P \models E \iff Q \models E
\end{eqnarray*}

\begin{eqnarray*}
  P \approx_{K} Q
\end{eqnarray*}

\begin{eqnarray*}
  P \approx Q
\end{eqnarray*}

$\approx_{K} = \approx = \approx_{L}$

\subsubsection{Contextual duality}

Note that contexts extend the quotation operation to a family of
operations from processes to names. Given a context, $M$, we can
define a \emph{nominal context}, $\quotep{M}$ by $\quotep{M}[P] :=
\quotep{M[P]}$. To foreshadow what is to come we observe that these
operations enjoy a duality with processes very much like the duality
between vectors and maps from vectors to scalars.

Further, because the calculus is essentially higher-order, we have a
correspondence between contexts and processes. More specifically,
given a name $x$ and a context $M$ we can construct $M^{*}_{x}$ such
that 

\begin{mathpar}
  M^{*}_{x} | \lift{x}{P} \red M[P]
\end{mathpar}

namely,

\begin{mathpar}
  M^{*}_{x} := x?(u).M[\dropn{u}]
\end{mathpar}

The dependence of $M^{*}_{x}$ on a name makes it an abstraction, 

\begin{mathpar}
  M^{*} := (x)x?(u).M[\dropn{u}]
\end{mathpar}

\subsection{Additional notation}

It will sometimes be convenient to denote the process a name
quotes. We already have the notation $x = \quotep{P}$, but it will be
convenient to introduce an alternate notation, $\procn{x}$, when we
want to emphasize the connection to the use of the name. Note that, by
virtue of name equivalence, $\quotep{\procn{x}} \nameeq x$; so, the
notation is consistent with previous definitions.

Further, because names have structure it is possible to effect
substitutions on the basis of that structure. This means we need to
upgrade our notation for substitutions, which we accomplish by
adapting comprehension notation. Thus,

\begin{mathpar}
  P\{ y / x : x \in S \}
\end{mathpar}

is interpreted to mean the process derived from P by replacing (in a
capture-avoiding manner) each occurrence of $x$ in $S$ by $y$. For example,

\begin{mathpar}
  P\{ \quotep{\procn{x}|\procn{x}} / x : x \in \freenames{P} \}
\end{mathpar}

will replace each (occurrence) of a free name $x$ in $P$ by
$\quotep{\procn{x}|\procn{x}}$.

Also, we will avail ourselves of the notation $x^{L}$ and $x^{R}$ to
denote injections of a name into disjoint copies of the name
space. There are numerous ways to accomplish this. One example can be
found in \cite{MeredithR05}. This notation overloads to vectors of
names: $\vec{x}^{\pi} := (x_{i}^{\pi} \; : \; 0 \leq i < |\vec{x}| )$ where $\pi \in \{L,R\}$.

We also use $P^{\Box} := P|\Box$.

In \cite{MeredithR05} an interpretation of the new operator is
given. It turns out that there are several possible interpretations
all enjoying the requisite algebraic properties of the operator (see
\cite{milner91polyadicpi}). We will therefore make liberal use of
$(\nu\; \vec{x})P$.

% subsection the_syntax_and_semantics_of_the_notation_system (end)   

\input{qm2pi.qmops} 

\input{qm2pi.sterngerlach} 

\input{qm2pi.metric} 

% section concurrent_process_calculi (end)

%\input{qm2pi.proofsketch}

% section proof sketch (end)

%\input{qm2pi.slviaknots} 

% section spatial logic via knots (end)

\input{qm2pi.conclusion}

% section conclusion (end)

%\input{qm2pi.dtcodes} 

% section wiring algorithm (end)

\input{qm2pi.ack} 

% section acknowledgments (end)

\newpage


\bibliographystyle{plain}   
\bibliography{../../biblios/main.bib}

\input{qm2pi.rhodetails}

\end{document}

 

%\documentclass[12pt]{llncs}
%\documentclass{jktr}

\usepackage[pdftex]{hyperref}                   
\usepackage {listings}
\usepackage {mathpartir}
\usepackage{bcprules}
%\usepackage{listings}
                       
\usepackage{graphicx} 
%\usepackage[margins=2.5cm,nohead,nofoot]{geometry}
%\usepackage{geometry}
\usepackage{amsfonts}
\usepackage{amstext}
\usepackage{latexsym}
\usepackage{amssymb}
\usepackage{color}


%\include{myPreamble}
\include{qm2pi.local} 

%\ifpdf
%\usepackage[pdftex]{graphicx}
%\else
%\usepackage{graphicx}
%\fi

 % \ifpdf
%  \usepackage{pdfsync}
%  \if


%\title{Brief Article}
%\author{David F. Snyder}
%\author{L.G. Meredith}

%\address{Dept. of Math., Texas State University--San Marcos, San Marcos, TX 78666}
       
\pagestyle{empty}


\begin{document}

\lstset{language=[Objective]Caml,frame=shadowbox}

\input{qm2pi.front}

% section front matter (end)

\input{qm2pi.intro} 
 
% section introduction (end)

% \input{qm2pi.knotations} 

% section notation (end)

\input{qm2pi.process.calculi} 

% section concurrent_process_calculi_and_spatial_logics_ (end)
    
%\input{qm2pi.knots2pi} 

%\input{qm2pi.trefoil} 

%\input{qm2pi.mainthm} 

% subsection basic_interpretation (end)

%\input{qm2pi.rho.presentation} 
\subsection{The syntax and semantics of the notation system}\label{sub:the_syntax_and_semantics_of_the_notation_system} % (fold)

We now summarize a technical presentation of the calculus that
embodies our theory of dynamics. The typical presentation of such a
calculus follows the style of giving generators and relations on
them. The grammar, below, describing term constructors, freely
generates the set of processes, $\Proc$. This set is then quotiented
by a relation known as structural congruence and it is over this set
that the notion of dynamics is expressed. This presentation is
essentially that of \cite{MeredithR05} with the addition of
polyadicity and summation. For readability we have relegated some of
the technical subtleties to an appendix.

\subsubsection{Process grammar}\label{subsub:process_grammar}

\begin{mathpar}
  \inferrule* [lab=synchronization] {} {{M} \bc \pzero \;|\; x?F \;|\; x!C }
  \and
  \inferrule* [lab=abstraction] {} {{F} \bc (x)P}
  \and
  \inferrule* [lab=concretion] {} {{C} \bc \langle Q \rangle}
  \and
  \inferrule* [lab=process] {} {{P,Q} \bc M \;| \;P|Q \;|\; @{x}}
  \and
  \inferrule* [lab=name] {} {{x} \bc \quotep{P}}
\end{mathpar} 

Note that $\vec{x}$ (resp. $\vec{P}$) denotes a vector of names
(resp. processes) of length $|\vec{x}|$ (resp. $|\vec{P}|$). We adopt
the following useful abbreviations.

\begin{mathpar}
   x?(\vec{y}).P := x.(\vec{y})P \and  x\clift{\vec{P}} := x.\clift{\vec{P}}
   \and x!(y) := \lift{x}{\dropn{y}}
   \and \Pi_{i=0}^{n-1}P_i := P_0 | \ldots | P_{n-1}
\end{mathpar}

\subsubsection{Structural congruence}

\paragraph{Free and bound names and alpha-equivalence.} At the
core of structural equivalence is alpha-equivalence which identifies
process that are the same up to a change of variable. Formally, we
recognize the distinction between free and bound names. The free names
of a process, $\freenames{P}$, may be calculated recursively as
follows:

\begin{mathpar}
\freenames{\pzero} := \emptyset
  \and \\
  \freenames{x?(y).P} := \{ x \} \cup (\freenames{P} \setminus \{ y \})
  \and 
  \freenames{x!\langle P \rangle} := \{ x \} \cup \{ P \} 
  \and \\
  \freenames{P|Q} := \freenames{P} \cup \freenames{Q}
  \and \\
  \freenames{@{x}} := \{ x \}
\end{mathpar}

$\pi$
$\quotep{\pi}$

$\freenames{-} : \pi \to \mathcal{P}(\quotep{\pi})$

\begin{eqnarray*}
  \freenames{\pzero} & := & \emptyset \\
  \freenames{x?(y).P} & := & \{ x \} \cup (\freenames{P} \setminus \{ y \}) \\
  \freenames{x!\langle P \rangle} & := & \{ x \} \cup \{ P \} \\
  \freenames{P|Q} & := & \freenames{P} \cup \freenames{Q} \\
  \freenames{\dropn{x}} & := & \{ x \}
\end{eqnarray*}

The bound names of a process, $\boundnames{P}$, are those names occurring in $P$
that are not free. For example, in $x?(y).0$, the name $x$ is free, while $y$ is bound.

\begin{mathpar}
  \inferrule* [lab=monoidal-laws] {} { P|Q \equiv Q|P \and P|0 \equiv P \and P|(Q|R) \equiv (P|Q)|R }
\end{mathpar}

\begin{mathpar}
  \inferrule* [lab=alpha-equivalence] {} { (x)P \equiv (y)P\{y/x\} \and y \not\in \freenames{P} }
\end{mathpar}

\begin{definition}
Then two processes, $P,Q$, are alpha-equivalent if $P = Q\{\vec{y}/\vec{x}\}$ for
some $\vec{x} \in \boundnames{Q},\vec{y} \in \boundnames{P}$, where $Q\{\vec{y}/\vec{x}\}$
denotes the capture-avoiding substitution of $\vec{y}$ for $\vec{x}$ in $Q$.
\end{definition}

\begin{definition}
  The {\em structural congruence} \cite{SangiorgiWalker} , $\equiv$,
  between processes is the least congruence containing
  alpha-equivalence, satisfying the abelian monoid laws
  (associativity, commutativity and $\pzero$ as identity) for parallel
  composition $|$ and for summation $+$.
\end{definition}

\subsection{Name equivalence}

We take name equivalence, written $\nameeq$, to be the smallest
equivalence relation generated by the following rules.

\begin{mathpar}
\inferrule*[lab=Quote-drop]
{ }
{ \quotep{@{x}} \nameeq x }

\inferrule*[lab=Struct-equiv]
{ P \scong Q }
{ \quotep{P} \nameeq \quotep{Q} }
\end{mathpar}

The astute reader will have noticed that the mutual recursion of names
and processes imposes a mutual recursion on alpha-equivalence and
structural equivalence via name-equivalence. Fortunately, all of this
works out pleasantly and we may calculate in the natural way, free of
concern. The reader interested in the details is referred to the
appendix \ref{appendix:rho_details}.

\subsection{Substitution}

We use $\Proc$ for the set of processes, $\QProc$ for the set of
names, and $\id{\{}\vec{y} / \vec{x} \id{\}}$ to denote partial maps,
$s : \QProc \rightarrow \QProc$. A map, $s$ lifts, uniquely, to a map
on process terms, $\widehat{s} : \Proc \rightarrow \Proc$ by the
following equations.

\begin{mathpar}
  (0) \psubstp{Q}{P} := 0 \\
  (R \juxtap S) \psubstp{Q}{P}
  :=    
  (R)\psubstp{Q}{P} \juxtap (S) \psubstp{Q}{P} \\
  (x?(y).R) \psubstp{Q}{P}    
  :=    
  (x)\substp{Q}{P} (z)\concat( (R \psubstn{z}{y}) \psubstp{Q}{P} ) \\
  (\lift{x}{R}) \psubstp{Q}{P}  
  :=
  \lift{(x)\substp{Q}{P}}{ R \psubstp{Q}{P} } \\
%   (\dropn{x})  \psubstp{Q}{P}       
%   := 
%   \left\{ 
%     \begin{array}{ccc} 
%       \dropn{\quotep{Q}} & & x \nameeq \quotep{P} \\
%       \dropn{x} & & otherwise \\
%     \end{array}
%   \right. 
  (\dropn{x})  \psubstp{Q}{P}       
  := 
  \left\{ 
    \begin{array}{ccc} 
      Q & & x \nameeq \quotep{P} \\
      \dropn{x} & & otherwise \\
    \end{array}
  \right.
\end{mathpar}
 

where

\begin{eqnarray}
  (x)\id{\{} \lpquote Q \rpquote / \lpquote P \rpquote \id{\}}            = 
  \left\{ 
    \begin{array}{ccc}
      \lpquote Q \rpquote & & x \nameeq \lpquote P \rpquote \\
      x & & otherwise \\
    \end{array}
  \right. \nonumber
\end{eqnarray}

and $z$ is chosen distinct from $\quotep{P}$, $\quotep{Q}$, the free
names in $Q$, and all the names in $R$. Our $\alpha$-equivalence will
be built in the standard way from this substitution.

\begin{remark}\label{rem:no_self_referential_names}
  One consequence of these definitions is that $\forall P. \quotep{P}
  \not\in \freenames{P}$.
\end{remark}

\subsection{ Dynamic quote: an example }

Anticipating something of what's to come, consider applying the
substitution, $\widehat{\id{\{}u / z \id{\}}}$, to the following pair
of processes, $\lift{w}{y!(z)}$ and $w[ \lpquote y!(z) \rpquote ]$.

\begin{eqnarray}
	\lift{w}{y!(z)}\widehat{\id{\{}u / z \id{\}}}
		& = &
		\lift{w}{y!(u)} \nonumber\\
	w[ \lpquote y!(z) \rpquote ] \widehat{ \id{\{}u / z \id{\}} }
		& = &
		w[ \lpquote y!(z) \rpquote ] \nonumber
\end{eqnarray}

Because the body of the process between quotes is impervious to
substitution, we get radically different answers. In fact, by
examining the first process in an input context,
e.g. $x?(z).\lift{w}{y!(z)}$, we see that the process under the lift
operator may be shaped by prefixed inputs binding a name inside it. In
this sense, the lift operator will be seen as a way to dynamically
construct processes before reifying them as names.

Finally equipped with these standard features we can present the
dynamics of the calculus.

\subsubsection{Operational semantics} 

Finally, we introduce the computational dynamics. What marks these
algebras as distinct from other more traditionally studied algebraic
structures, e.g. vector spaces or polynomial rings, is the manner in
which dynamics is captured. In traditional structures, dynamics is typically
expressed through morphisms between such structures, as in linear maps
between vector spaces or morphisms between rings. In algebras
associated with the semantics of computation, the dynamics is
expressed as part of the algebraic structure itself, through a
reduction reduction relation typically denoted by $\red$. Below, we
give a recursive presentation of this relation for the calculus used
in the encoding.

$\red \subseteq \pi \times \pi$
$\red : \pi \to \mathcal{P}(\pi)$

\begin{mathpar}
  \inferrule* [lab=Comm] { \textsf{match}( x_{src}, x_{trgt} ) } { x_{trgt}?(y)P \; | \; x_{src}!\langle {Q} \rangle \red P\{\quotep{Q}/y}\} }
  \and \\
  \inferrule* [lab=Par] {{P} \red {P}'} {{{P} | {Q}} \red {{P}' | {Q}}}
  \and
  \inferrule* [lab=Equiv]{{{P} \scong {P}'} \andalso {{P}' \red {Q}'} \andalso {{Q}' \scong {Q}}}{{P} \red {Q}}
\end{mathpar}

\begin{eqnarray*}
  match_{\equiv} (\quotep{P},\quotep{Q}) & := & P \equiv Q \\
  match_{\dagger}(\quotep{P},\quotep{Q}) & := & \forall R. P|Q \red^{*} R => R \red^{*} 0 \\
  match_{K}(\quotep{P},\quotep{Q}) & := & K \mbox{ for some context } K
\end{eqnarray*}

$u?(x)P | u!\langle Q \rangle \red P\{\quotep{Q}/x\}$

%We write $\wred$ for $\red^*$, and $P\red$ if $\exists Q $ such that $ P \red Q$.
We write $P\red$ if $\exists Q $ such that $ P \red Q$ and $P\not\red$, otherwise.

\section{Replication}

As mentioned before, it is known that replication (and hence
recursion) can be implemented in a higher-order process algebra
\cite{SangiorgiWalker}. As our first example of calculation with the
machinery thus far presented we give the construction explicitly in
the {\rhoc}.

\begin{eqnarray}
	D_{x} & := & \prefix{x}{y}{(\binpar{\outputp{x}{y}}{@{y}})} \nonumber\\
	\bangp_{x}{P} & := & \binpar{{x}!\langle{\binpar{D_{x}}{P}}\rangle}{D_{x}} \nonumber
\end{eqnarray}

\begin{eqnarray}
	\bangp_{x}{P} & & \nonumber\\
	=
	& {x}!\langle{(\prefix{x}{y}{(\outputp{x}{y} | @{y})) | P}}\rangle 
	      | \prefix{x}{y}{(\outputp{x}{y} | @{y})} & \nonumber\\
	\red
	& (\outputp{x}{y} | @{y})\substn{\quotep{(\prefix{x}{y}{(@{y} | \outputp{x}{y})) | P}}}{y} & \nonumber\\
	=
	& \outputp{x}{\quotep{(\prefix{x}{y}{(\outputp{x}{y} | @{y})) | P}}}
	  | {(\prefix{x}{y}{(\outputp{x}{y} | @{y})) | P}} & \nonumber\\
	\red
	& \ldots & \nonumber\\
	\red^*
	& P | P | \ldots & \nonumber
\end{eqnarray}

Of course, this encoding, as an implementation, runs away, unfolding
$\bangp{P}$ eagerly. A lazier and more implementable replication
operator, restricted to input-guarded processes, may be obtained as follows.

\begin{eqnarray}
\bangp{\prefix{u}{v}{P}} 
	:= 
	\binpar{\lift{x}{\prefix{u}{v}{(\binpar{D(x)}{P})}}}{D(x)} \nonumber
\end{eqnarray}

\begin{remark}
  Note that the lazier definition still does not deal with summation
  or mixed summation (i.e. sums over input and output). The reader is
  invited to construct definitions of replication that deal with these
  features. 

  Further, the definitions are parameterized in a name, $x$. Can you,
  gentle reader, make a definition that eliminates this parameter and
  guarantees no accidental interaction between the replication
  machinery and the process being replicated -- i.e. no accidental
  sharing of names used by the process to get its work done and the
  name(s) used by the replication to effect copying. This latter
  revision of the definition of replication is crucial to obtaining
  the expected identity $!!P \sim !P$.
\end{remark}

\begin{remark}\label{rem:paradoxical_combinator}
  The reader familiar with the lambda calculus will have noticed the
  similarity between $D$ and the paradoxical combinator.

  [Ed. note: the existence of this seems to suggest we have to be more
  restrictive on the set of processes and names we admit if we are to
  support no-cloning.]
\end{remark}

\subsubsection{Bisimulation}

The computational dynamics gives rise to another kind of equivalence,
the equivalence of computational behavior. As previously mentioned
this is typically captured \emph{via} some form of bisimulation.

% The notion we use in this paper is weak barbed bisimulation
% \cite{milner91polyadicpi}.

The notion we use in this paper is derived from weak barbed
bisimulation \cite{milner91polyadicpi}. 

\begin{definition}
An \emph{observation relation}, $\downarrow_{\mathcal N}$, over a set
of names, $\mathcal N$, is the smallest relation satisfying the rules
below.

\infrule[Out-barb]{y \in {\mathcal N}, \; x \nameeq y}
		  {\outputp{x}{v} \downarrow_{\mathcal N} x}
\infrule[Par-barb]{\mbox{$P\downarrow_{\mathcal N} x$ or $Q\downarrow_{\mathcal N} x$}}
		  {\binpar{P}{Q} \downarrow_{\mathcal N} x}

We write $P \Downarrow_{\mathcal N} x$ if there is $Q$ such that 
$P \wred Q$ and $Q \downarrow_{\mathcal N} x$.
\end{definition}

\begin{definition}
%\label{def.bbisim}
An  ${\mathcal N}$-\emph{barbed bisimulation} over a set of names, ${\mathcal N}$, is a symmetric binary relation 
${\mathcal S}_{\mathcal N}$ between agents such that $P\rel{S}_{\mathcal N}Q$ implies:
\begin{enumerate}
\item If $P \red P'$ then $Q \wred Q'$ and $P'\rel{S}_{\mathcal N} Q'$.
\item If $P\downarrow_{\mathcal N} x$, then $Q\Downarrow_{\mathcal N} x$.
\end{enumerate}
$P$ is ${\mathcal N}$-barbed bisimilar to $Q$, written
$P \wbbisim_{\mathcal N} Q$, if $P \rel{S}_{\mathcal N} Q$ for some ${\mathcal N}$-barbed bisimulation ${\mathcal S}_{\mathcal N}$.
\end{definition}

$\mathcal{R} \subseteq \pi \times \pi$

$P \mathcal{R} Q => \forall P'. P \red P' \Rightarrow \exists Q'. Q \red Q', P' \mathcal{R} Q'$

$P \vdash x \Rightarrow Q \vdash x$

\begin{mathpar}
  \inferrule*[lab=Out-barb]{x \nameeq y}{{y}!\langle{Q}\rangle \vdash x}
  \and
  \inferrule*[lab=Par-barb]{\mbox{$P\vdash x$ or $Q\vdash x$}}{\binpar{P}{Q} \vdash x}
\end{mathpar}

\subsubsection{Contexts}

One of the principle advantages of computational calculi like the
$\pi$-calculus is a well-defined notion of context,
contextual-equivalence and a correlation between
contextual-equivalence and notions of bisimulation. The notion of
context allows the decomposition of a process into (sub-)process and
its syntactic environment, its context. Thus, a context may be
thought of as a process with a ``hole'' (written $\Box$) in it. The
application of a context $M$ to a process $P$, written $M[P]$, is
tantamount to filling the hole in $M$ with $P$. In this paper we do
not need the full weight of this theory, but do make use of the notion
of context in the proof the main theorem. 

\begin{mathpar}
  \inferrule* [lab=summation] {} {{M_{M},M_{N}} \bc \Box \;|\; x.M_{A} \;|\; M_{M}+M_{N}}
  \and
  \inferrule* [lab=agent] {} {{M_{A}} \bc (\vec{x})M_{P} \;| \; \clift{P_0,\ldots,M_{P},\ldots,P_N}}
  \and \\
  \inferrule* [lab=process] {} {{M_{P}} \bc M_{N} \;| \;P|M_{P} }
\end{mathpar} 

\begin{mathpar}
  \inferrule* [lab=sychronization] {} {M_{N} \bc \Box \;|\; x?M_{F} \;|\; x!M_{C}}
  \and
  \inferrule* [lab=abstraction] {} {{M_{F}} \bc (x)M_{P} }
  \and
  \inferrule* [lab=concretion] {} {{M_{C}} \bc \langle M_{P} \rangle }
  \and \\
  \inferrule* [lab=process] {} {{M_{P}} \bc M_{N} \;| \;P|M_{P} }
\end{mathpar}

\begin{definition}[contextual application] Given a context $M$, and
  process $P$, we define the \emph{contextual application}, $M[P] :=
  M\{P/\Box\}$. That is, the contextual application of M to P is the
  substitution of $P$ for $\Box$ in $M$.
\end{definition}

$\meaningof{-} : L \to \mathcal{P}(\pi)$

\begin{mathpar}
  \inferrule* [lab=collection] {} {\meaningof{true} = \pi, \and \meaningof{~E} = \pi \setminus \meaningof{E}, \and \meaningof{E_{1} \& E_{2}} = \meaningof{E_{1}} \cap \meaningof{E_{2}}}
\end{mathpar}

\begin{mathpar}
  \inferrule* [lab=structure] {} {\meaningof{0} = \{ P \in \pi | P \equiv 0 \}, \and \\ \meaningof{E_1 | E_2} = \{ P \in \pi | P \equiv P_{1} | P_{2}, P_{1} \in \meaningof{E_{1}}, P_{2} \in \meaningof{E_2}\} }
\end{mathpar}

\begin{mathpar}
 \inferrule* [lab=behavior] {} {\meaningof{\langle a?b \rangle E} = \{ P \in \pi | P \equiv Q | u?(y)P', \\ \and \\\\ \and \\ \;\;\; u \in \meaningof{a}, \forall z.P'\{z/y\} \in \meaningof{E\{z/b\}}\}, \and \\ \meaningof{a!E} = \{ P \in \pi | P \equiv Q | x!\langle P' \rangle, x \in \meaningof{a} P' \in \meaningof{E}\} }
\end{mathpar}

\begin{mathpar}
 \inferrule* [lab=nominal] {} {\meaningof{\quotep{E}} = \{ \quotep{P} \in \quotep{\pi} | P \in \meaningof{E} \}, \and \meaningof{\quotep{P}} = \{ \quotep{Q} \in \quotep{\pi} | P \equiv Q \} \and \\ \meaningof{@\quotep{E}} = \{ P \in \pi | P \equiv @x, x \in \meaningof{E} \}}
\end{mathpar}

\begin{eqnarray*}
  \\
  \meaningof{-} : TS \to ST
\end{eqnarray*}

\begin{eqnarray*}
  \\
  L : TS \to ST
\end{eqnarray*}

\begin{eqnarray*}
  \\
  P \models E \iff P \in \meaningof{E}
\end{eqnarray*}

\begin{eqnarray*}
  P \approx_{L} Q \iff \forall E \in L. P \models E \iff Q \models E
\end{eqnarray*}

\begin{eqnarray*}
  P \approx_{K} Q
\end{eqnarray*}

\begin{eqnarray*}
  P \approx Q
\end{eqnarray*}

$\approx_{K} = \approx = \approx_{L}$

\subsubsection{Contextual duality}

Note that contexts extend the quotation operation to a family of
operations from processes to names. Given a context, $M$, we can
define a \emph{nominal context}, $\quotep{M}$ by $\quotep{M}[P] :=
\quotep{M[P]}$. To foreshadow what is to come we observe that these
operations enjoy a duality with processes very much like the duality
between vectors and maps from vectors to scalars.

Further, because the calculus is essentially higher-order, we have a
correspondence between contexts and processes. More specifically,
given a name $x$ and a context $M$ we can construct $M^{*}_{x}$ such
that 

\begin{mathpar}
  M^{*}_{x} | \lift{x}{P} \red M[P]
\end{mathpar}

namely,

\begin{mathpar}
  M^{*}_{x} := x?(u).M[\dropn{u}]
\end{mathpar}

The dependence of $M^{*}_{x}$ on a name makes it an abstraction, 

\begin{mathpar}
  M^{*} := (x)x?(u).M[\dropn{u}]
\end{mathpar}

\subsection{Additional notation}

It will sometimes be convenient to denote the process a name
quotes. We already have the notation $x = \quotep{P}$, but it will be
convenient to introduce an alternate notation, $\procn{x}$, when we
want to emphasize the connection to the use of the name. Note that, by
virtue of name equivalence, $\quotep{\procn{x}} \nameeq x$; so, the
notation is consistent with previous definitions.

Further, because names have structure it is possible to effect
substitutions on the basis of that structure. This means we need to
upgrade our notation for substitutions, which we accomplish by
adapting comprehension notation. Thus,

\begin{mathpar}
  P\{ y / x : x \in S \}
\end{mathpar}

is interpreted to mean the process derived from P by replacing (in a
capture-avoiding manner) each occurrence of $x$ in $S$ by $y$. For example,

\begin{mathpar}
  P\{ \quotep{\procn{x}|\procn{x}} / x : x \in \freenames{P} \}
\end{mathpar}

will replace each (occurrence) of a free name $x$ in $P$ by
$\quotep{\procn{x}|\procn{x}}$.

Also, we will avail ourselves of the notation $x^{L}$ and $x^{R}$ to
denote injections of a name into disjoint copies of the name
space. There are numerous ways to accomplish this. One example can be
found in \cite{MeredithR05}. This notation overloads to vectors of
names: $\vec{x}^{\pi} := (x_{i}^{\pi} \; : \; 0 \leq i < |\vec{x}| )$ where $\pi \in \{L,R\}$.

We also use $P^{\Box} := P|\Box$.

In \cite{MeredithR05} an interpretation of the new operator is
given. It turns out that there are several possible interpretations
all enjoying the requisite algebraic properties of the operator (see
\cite{milner91polyadicpi}). We will therefore make liberal use of
$(\nu\; \vec{x})P$.

% subsection the_syntax_and_semantics_of_the_notation_system (end)   

\input{qm2pi.qmops} 

\input{qm2pi.sterngerlach} 

\input{qm2pi.metric} 

% section concurrent_process_calculi (end)

%\input{qm2pi.proofsketch}

% section proof sketch (end)

%\input{qm2pi.slviaknots} 

% section spatial logic via knots (end)

\input{qm2pi.conclusion}

% section conclusion (end)

%\input{qm2pi.dtcodes} 

% section wiring algorithm (end)

\input{qm2pi.ack} 

% section acknowledgments (end)

\newpage


\bibliographystyle{plain}   
\bibliography{../../biblios/main.bib}

\input{qm2pi.rhodetails}

\end{document}

 

%\documentclass[12pt]{llncs}
%\documentclass{jktr}

\usepackage[pdftex]{hyperref}                   
\usepackage {listings}
\usepackage {mathpartir}
\usepackage{bcprules}
%\usepackage{listings}
                       
\usepackage{graphicx} 
%\usepackage[margins=2.5cm,nohead,nofoot]{geometry}
%\usepackage{geometry}
\usepackage{amsfonts}
\usepackage{amstext}
\usepackage{latexsym}
\usepackage{amssymb}
\usepackage{color}


%\include{myPreamble}
\include{qm2pi.local} 

%\ifpdf
%\usepackage[pdftex]{graphicx}
%\else
%\usepackage{graphicx}
%\fi

 % \ifpdf
%  \usepackage{pdfsync}
%  \if


%\title{Brief Article}
%\author{David F. Snyder}
%\author{L.G. Meredith}

%\address{Dept. of Math., Texas State University--San Marcos, San Marcos, TX 78666}
       
\pagestyle{empty}


\begin{document}

\lstset{language=[Objective]Caml,frame=shadowbox}

\input{qm2pi.front}

% section front matter (end)

\input{qm2pi.intro} 
 
% section introduction (end)

% \input{qm2pi.knotations} 

% section notation (end)

\input{qm2pi.process.calculi} 

% section concurrent_process_calculi_and_spatial_logics_ (end)
    
%\input{qm2pi.knots2pi} 

%\input{qm2pi.trefoil} 

%\input{qm2pi.mainthm} 

% subsection basic_interpretation (end)

%\input{qm2pi.rho.presentation} 
\subsection{The syntax and semantics of the notation system}\label{sub:the_syntax_and_semantics_of_the_notation_system} % (fold)

We now summarize a technical presentation of the calculus that
embodies our theory of dynamics. The typical presentation of such a
calculus follows the style of giving generators and relations on
them. The grammar, below, describing term constructors, freely
generates the set of processes, $\Proc$. This set is then quotiented
by a relation known as structural congruence and it is over this set
that the notion of dynamics is expressed. This presentation is
essentially that of \cite{MeredithR05} with the addition of
polyadicity and summation. For readability we have relegated some of
the technical subtleties to an appendix.

\subsubsection{Process grammar}\label{subsub:process_grammar}

\begin{mathpar}
  \inferrule* [lab=synchronization] {} {{M} \bc \pzero \;|\; x?F \;|\; x!C }
  \and
  \inferrule* [lab=abstraction] {} {{F} \bc (x)P}
  \and
  \inferrule* [lab=concretion] {} {{C} \bc \langle Q \rangle}
  \and
  \inferrule* [lab=process] {} {{P,Q} \bc M \;| \;P|Q \;|\; @{x}}
  \and
  \inferrule* [lab=name] {} {{x} \bc \quotep{P}}
\end{mathpar} 

Note that $\vec{x}$ (resp. $\vec{P}$) denotes a vector of names
(resp. processes) of length $|\vec{x}|$ (resp. $|\vec{P}|$). We adopt
the following useful abbreviations.

\begin{mathpar}
   x?(\vec{y}).P := x.(\vec{y})P \and  x\clift{\vec{P}} := x.\clift{\vec{P}}
   \and x!(y) := \lift{x}{\dropn{y}}
   \and \Pi_{i=0}^{n-1}P_i := P_0 | \ldots | P_{n-1}
\end{mathpar}

\subsubsection{Structural congruence}

\paragraph{Free and bound names and alpha-equivalence.} At the
core of structural equivalence is alpha-equivalence which identifies
process that are the same up to a change of variable. Formally, we
recognize the distinction between free and bound names. The free names
of a process, $\freenames{P}$, may be calculated recursively as
follows:

\begin{mathpar}
\freenames{\pzero} := \emptyset
  \and \\
  \freenames{x?(y).P} := \{ x \} \cup (\freenames{P} \setminus \{ y \})
  \and 
  \freenames{x!\langle P \rangle} := \{ x \} \cup \{ P \} 
  \and \\
  \freenames{P|Q} := \freenames{P} \cup \freenames{Q}
  \and \\
  \freenames{@{x}} := \{ x \}
\end{mathpar}

$\pi$
$\quotep{\pi}$

$\freenames{-} : \pi \to \mathcal{P}(\quotep{\pi})$

\begin{eqnarray*}
  \freenames{\pzero} & := & \emptyset \\
  \freenames{x?(y).P} & := & \{ x \} \cup (\freenames{P} \setminus \{ y \}) \\
  \freenames{x!\langle P \rangle} & := & \{ x \} \cup \{ P \} \\
  \freenames{P|Q} & := & \freenames{P} \cup \freenames{Q} \\
  \freenames{\dropn{x}} & := & \{ x \}
\end{eqnarray*}

The bound names of a process, $\boundnames{P}$, are those names occurring in $P$
that are not free. For example, in $x?(y).0$, the name $x$ is free, while $y$ is bound.

\begin{mathpar}
  \inferrule* [lab=monoidal-laws] {} { P|Q \equiv Q|P \and P|0 \equiv P \and P|(Q|R) \equiv (P|Q)|R }
\end{mathpar}

\begin{mathpar}
  \inferrule* [lab=alpha-equivalence] {} { (x)P \equiv (y)P\{y/x\} \and y \not\in \freenames{P} }
\end{mathpar}

\begin{definition}
Then two processes, $P,Q$, are alpha-equivalent if $P = Q\{\vec{y}/\vec{x}\}$ for
some $\vec{x} \in \boundnames{Q},\vec{y} \in \boundnames{P}$, where $Q\{\vec{y}/\vec{x}\}$
denotes the capture-avoiding substitution of $\vec{y}$ for $\vec{x}$ in $Q$.
\end{definition}

\begin{definition}
  The {\em structural congruence} \cite{SangiorgiWalker} , $\equiv$,
  between processes is the least congruence containing
  alpha-equivalence, satisfying the abelian monoid laws
  (associativity, commutativity and $\pzero$ as identity) for parallel
  composition $|$ and for summation $+$.
\end{definition}

\subsection{Name equivalence}

We take name equivalence, written $\nameeq$, to be the smallest
equivalence relation generated by the following rules.

\begin{mathpar}
\inferrule*[lab=Quote-drop]
{ }
{ \quotep{@{x}} \nameeq x }

\inferrule*[lab=Struct-equiv]
{ P \scong Q }
{ \quotep{P} \nameeq \quotep{Q} }
\end{mathpar}

The astute reader will have noticed that the mutual recursion of names
and processes imposes a mutual recursion on alpha-equivalence and
structural equivalence via name-equivalence. Fortunately, all of this
works out pleasantly and we may calculate in the natural way, free of
concern. The reader interested in the details is referred to the
appendix \ref{appendix:rho_details}.

\subsection{Substitution}

We use $\Proc$ for the set of processes, $\QProc$ for the set of
names, and $\id{\{}\vec{y} / \vec{x} \id{\}}$ to denote partial maps,
$s : \QProc \rightarrow \QProc$. A map, $s$ lifts, uniquely, to a map
on process terms, $\widehat{s} : \Proc \rightarrow \Proc$ by the
following equations.

\begin{mathpar}
  (0) \psubstp{Q}{P} := 0 \\
  (R \juxtap S) \psubstp{Q}{P}
  :=    
  (R)\psubstp{Q}{P} \juxtap (S) \psubstp{Q}{P} \\
  (x?(y).R) \psubstp{Q}{P}    
  :=    
  (x)\substp{Q}{P} (z)\concat( (R \psubstn{z}{y}) \psubstp{Q}{P} ) \\
  (\lift{x}{R}) \psubstp{Q}{P}  
  :=
  \lift{(x)\substp{Q}{P}}{ R \psubstp{Q}{P} } \\
%   (\dropn{x})  \psubstp{Q}{P}       
%   := 
%   \left\{ 
%     \begin{array}{ccc} 
%       \dropn{\quotep{Q}} & & x \nameeq \quotep{P} \\
%       \dropn{x} & & otherwise \\
%     \end{array}
%   \right. 
  (\dropn{x})  \psubstp{Q}{P}       
  := 
  \left\{ 
    \begin{array}{ccc} 
      Q & & x \nameeq \quotep{P} \\
      \dropn{x} & & otherwise \\
    \end{array}
  \right.
\end{mathpar}
 

where

\begin{eqnarray}
  (x)\id{\{} \lpquote Q \rpquote / \lpquote P \rpquote \id{\}}            = 
  \left\{ 
    \begin{array}{ccc}
      \lpquote Q \rpquote & & x \nameeq \lpquote P \rpquote \\
      x & & otherwise \\
    \end{array}
  \right. \nonumber
\end{eqnarray}

and $z$ is chosen distinct from $\quotep{P}$, $\quotep{Q}$, the free
names in $Q$, and all the names in $R$. Our $\alpha$-equivalence will
be built in the standard way from this substitution.

\begin{remark}\label{rem:no_self_referential_names}
  One consequence of these definitions is that $\forall P. \quotep{P}
  \not\in \freenames{P}$.
\end{remark}

\subsection{ Dynamic quote: an example }

Anticipating something of what's to come, consider applying the
substitution, $\widehat{\id{\{}u / z \id{\}}}$, to the following pair
of processes, $\lift{w}{y!(z)}$ and $w[ \lpquote y!(z) \rpquote ]$.

\begin{eqnarray}
	\lift{w}{y!(z)}\widehat{\id{\{}u / z \id{\}}}
		& = &
		\lift{w}{y!(u)} \nonumber\\
	w[ \lpquote y!(z) \rpquote ] \widehat{ \id{\{}u / z \id{\}} }
		& = &
		w[ \lpquote y!(z) \rpquote ] \nonumber
\end{eqnarray}

Because the body of the process between quotes is impervious to
substitution, we get radically different answers. In fact, by
examining the first process in an input context,
e.g. $x?(z).\lift{w}{y!(z)}$, we see that the process under the lift
operator may be shaped by prefixed inputs binding a name inside it. In
this sense, the lift operator will be seen as a way to dynamically
construct processes before reifying them as names.

Finally equipped with these standard features we can present the
dynamics of the calculus.

\subsubsection{Operational semantics} 

Finally, we introduce the computational dynamics. What marks these
algebras as distinct from other more traditionally studied algebraic
structures, e.g. vector spaces or polynomial rings, is the manner in
which dynamics is captured. In traditional structures, dynamics is typically
expressed through morphisms between such structures, as in linear maps
between vector spaces or morphisms between rings. In algebras
associated with the semantics of computation, the dynamics is
expressed as part of the algebraic structure itself, through a
reduction reduction relation typically denoted by $\red$. Below, we
give a recursive presentation of this relation for the calculus used
in the encoding.

$\red \subseteq \pi \times \pi$
$\red : \pi \to \mathcal{P}(\pi)$

\begin{mathpar}
  \inferrule* [lab=Comm] { \textsf{match}( x_{src}, x_{trgt} ) } { x_{trgt}?(y)P \; | \; x_{src}!\langle {Q} \rangle \red P\{\quotep{Q}/y}\} }
  \and \\
  \inferrule* [lab=Par] {{P} \red {P}'} {{{P} | {Q}} \red {{P}' | {Q}}}
  \and
  \inferrule* [lab=Equiv]{{{P} \scong {P}'} \andalso {{P}' \red {Q}'} \andalso {{Q}' \scong {Q}}}{{P} \red {Q}}
\end{mathpar}

\begin{eqnarray*}
  match_{\equiv} (\quotep{P},\quotep{Q}) & := & P \equiv Q \\
  match_{\dagger}(\quotep{P},\quotep{Q}) & := & \forall R. P|Q \red^{*} R => R \red^{*} 0 \\
  match_{K}(\quotep{P},\quotep{Q}) & := & K \mbox{ for some context } K
\end{eqnarray*}

$u?(x)P | u!\langle Q \rangle \red P\{\quotep{Q}/x\}$

%We write $\wred$ for $\red^*$, and $P\red$ if $\exists Q $ such that $ P \red Q$.
We write $P\red$ if $\exists Q $ such that $ P \red Q$ and $P\not\red$, otherwise.

\section{Replication}

As mentioned before, it is known that replication (and hence
recursion) can be implemented in a higher-order process algebra
\cite{SangiorgiWalker}. As our first example of calculation with the
machinery thus far presented we give the construction explicitly in
the {\rhoc}.

\begin{eqnarray}
	D_{x} & := & \prefix{x}{y}{(\binpar{\outputp{x}{y}}{@{y}})} \nonumber\\
	\bangp_{x}{P} & := & \binpar{{x}!\langle{\binpar{D_{x}}{P}}\rangle}{D_{x}} \nonumber
\end{eqnarray}

\begin{eqnarray}
	\bangp_{x}{P} & & \nonumber\\
	=
	& {x}!\langle{(\prefix{x}{y}{(\outputp{x}{y} | @{y})) | P}}\rangle 
	      | \prefix{x}{y}{(\outputp{x}{y} | @{y})} & \nonumber\\
	\red
	& (\outputp{x}{y} | @{y})\substn{\quotep{(\prefix{x}{y}{(@{y} | \outputp{x}{y})) | P}}}{y} & \nonumber\\
	=
	& \outputp{x}{\quotep{(\prefix{x}{y}{(\outputp{x}{y} | @{y})) | P}}}
	  | {(\prefix{x}{y}{(\outputp{x}{y} | @{y})) | P}} & \nonumber\\
	\red
	& \ldots & \nonumber\\
	\red^*
	& P | P | \ldots & \nonumber
\end{eqnarray}

Of course, this encoding, as an implementation, runs away, unfolding
$\bangp{P}$ eagerly. A lazier and more implementable replication
operator, restricted to input-guarded processes, may be obtained as follows.

\begin{eqnarray}
\bangp{\prefix{u}{v}{P}} 
	:= 
	\binpar{\lift{x}{\prefix{u}{v}{(\binpar{D(x)}{P})}}}{D(x)} \nonumber
\end{eqnarray}

\begin{remark}
  Note that the lazier definition still does not deal with summation
  or mixed summation (i.e. sums over input and output). The reader is
  invited to construct definitions of replication that deal with these
  features. 

  Further, the definitions are parameterized in a name, $x$. Can you,
  gentle reader, make a definition that eliminates this parameter and
  guarantees no accidental interaction between the replication
  machinery and the process being replicated -- i.e. no accidental
  sharing of names used by the process to get its work done and the
  name(s) used by the replication to effect copying. This latter
  revision of the definition of replication is crucial to obtaining
  the expected identity $!!P \sim !P$.
\end{remark}

\begin{remark}\label{rem:paradoxical_combinator}
  The reader familiar with the lambda calculus will have noticed the
  similarity between $D$ and the paradoxical combinator.

  [Ed. note: the existence of this seems to suggest we have to be more
  restrictive on the set of processes and names we admit if we are to
  support no-cloning.]
\end{remark}

\subsubsection{Bisimulation}

The computational dynamics gives rise to another kind of equivalence,
the equivalence of computational behavior. As previously mentioned
this is typically captured \emph{via} some form of bisimulation.

% The notion we use in this paper is weak barbed bisimulation
% \cite{milner91polyadicpi}.

The notion we use in this paper is derived from weak barbed
bisimulation \cite{milner91polyadicpi}. 

\begin{definition}
An \emph{observation relation}, $\downarrow_{\mathcal N}$, over a set
of names, $\mathcal N$, is the smallest relation satisfying the rules
below.

\infrule[Out-barb]{y \in {\mathcal N}, \; x \nameeq y}
		  {\outputp{x}{v} \downarrow_{\mathcal N} x}
\infrule[Par-barb]{\mbox{$P\downarrow_{\mathcal N} x$ or $Q\downarrow_{\mathcal N} x$}}
		  {\binpar{P}{Q} \downarrow_{\mathcal N} x}

We write $P \Downarrow_{\mathcal N} x$ if there is $Q$ such that 
$P \wred Q$ and $Q \downarrow_{\mathcal N} x$.
\end{definition}

\begin{definition}
%\label{def.bbisim}
An  ${\mathcal N}$-\emph{barbed bisimulation} over a set of names, ${\mathcal N}$, is a symmetric binary relation 
${\mathcal S}_{\mathcal N}$ between agents such that $P\rel{S}_{\mathcal N}Q$ implies:
\begin{enumerate}
\item If $P \red P'$ then $Q \wred Q'$ and $P'\rel{S}_{\mathcal N} Q'$.
\item If $P\downarrow_{\mathcal N} x$, then $Q\Downarrow_{\mathcal N} x$.
\end{enumerate}
$P$ is ${\mathcal N}$-barbed bisimilar to $Q$, written
$P \wbbisim_{\mathcal N} Q$, if $P \rel{S}_{\mathcal N} Q$ for some ${\mathcal N}$-barbed bisimulation ${\mathcal S}_{\mathcal N}$.
\end{definition}

$\mathcal{R} \subseteq \pi \times \pi$

$P \mathcal{R} Q => \forall P'. P \red P' \Rightarrow \exists Q'. Q \red Q', P' \mathcal{R} Q'$

$P \vdash x \Rightarrow Q \vdash x$

\begin{mathpar}
  \inferrule*[lab=Out-barb]{x \nameeq y}{{y}!\langle{Q}\rangle \vdash x}
  \and
  \inferrule*[lab=Par-barb]{\mbox{$P\vdash x$ or $Q\vdash x$}}{\binpar{P}{Q} \vdash x}
\end{mathpar}

\subsubsection{Contexts}

One of the principle advantages of computational calculi like the
$\pi$-calculus is a well-defined notion of context,
contextual-equivalence and a correlation between
contextual-equivalence and notions of bisimulation. The notion of
context allows the decomposition of a process into (sub-)process and
its syntactic environment, its context. Thus, a context may be
thought of as a process with a ``hole'' (written $\Box$) in it. The
application of a context $M$ to a process $P$, written $M[P]$, is
tantamount to filling the hole in $M$ with $P$. In this paper we do
not need the full weight of this theory, but do make use of the notion
of context in the proof the main theorem. 

\begin{mathpar}
  \inferrule* [lab=summation] {} {{M_{M},M_{N}} \bc \Box \;|\; x.M_{A} \;|\; M_{M}+M_{N}}
  \and
  \inferrule* [lab=agent] {} {{M_{A}} \bc (\vec{x})M_{P} \;| \; \clift{P_0,\ldots,M_{P},\ldots,P_N}}
  \and \\
  \inferrule* [lab=process] {} {{M_{P}} \bc M_{N} \;| \;P|M_{P} }
\end{mathpar} 

\begin{mathpar}
  \inferrule* [lab=sychronization] {} {M_{N} \bc \Box \;|\; x?M_{F} \;|\; x!M_{C}}
  \and
  \inferrule* [lab=abstraction] {} {{M_{F}} \bc (x)M_{P} }
  \and
  \inferrule* [lab=concretion] {} {{M_{C}} \bc \langle M_{P} \rangle }
  \and \\
  \inferrule* [lab=process] {} {{M_{P}} \bc M_{N} \;| \;P|M_{P} }
\end{mathpar}

\begin{definition}[contextual application] Given a context $M$, and
  process $P$, we define the \emph{contextual application}, $M[P] :=
  M\{P/\Box\}$. That is, the contextual application of M to P is the
  substitution of $P$ for $\Box$ in $M$.
\end{definition}

$\meaningof{-} : L \to \mathcal{P}(\pi)$

\begin{mathpar}
  \inferrule* [lab=collection] {} {\meaningof{true} = \pi, \and \meaningof{~E} = \pi \setminus \meaningof{E}, \and \meaningof{E_{1} \& E_{2}} = \meaningof{E_{1}} \cap \meaningof{E_{2}}}
\end{mathpar}

\begin{mathpar}
  \inferrule* [lab=structure] {} {\meaningof{0} = \{ P \in \pi | P \equiv 0 \}, \and \\ \meaningof{E_1 | E_2} = \{ P \in \pi | P \equiv P_{1} | P_{2}, P_{1} \in \meaningof{E_{1}}, P_{2} \in \meaningof{E_2}\} }
\end{mathpar}

\begin{mathpar}
 \inferrule* [lab=behavior] {} {\meaningof{\langle a?b \rangle E} = \{ P \in \pi | P \equiv Q | u?(y)P', \\ \and \\\\ \and \\ \;\;\; u \in \meaningof{a}, \forall z.P'\{z/y\} \in \meaningof{E\{z/b\}}\}, \and \\ \meaningof{a!E} = \{ P \in \pi | P \equiv Q | x!\langle P' \rangle, x \in \meaningof{a} P' \in \meaningof{E}\} }
\end{mathpar}

\begin{mathpar}
 \inferrule* [lab=nominal] {} {\meaningof{\quotep{E}} = \{ \quotep{P} \in \quotep{\pi} | P \in \meaningof{E} \}, \and \meaningof{\quotep{P}} = \{ \quotep{Q} \in \quotep{\pi} | P \equiv Q \} \and \\ \meaningof{@\quotep{E}} = \{ P \in \pi | P \equiv @x, x \in \meaningof{E} \}}
\end{mathpar}

\begin{eqnarray*}
  \\
  \meaningof{-} : TS \to ST
\end{eqnarray*}

\begin{eqnarray*}
  \\
  L : TS \to ST
\end{eqnarray*}

\begin{eqnarray*}
  \\
  P \models E \iff P \in \meaningof{E}
\end{eqnarray*}

\begin{eqnarray*}
  P \approx_{L} Q \iff \forall E \in L. P \models E \iff Q \models E
\end{eqnarray*}

\begin{eqnarray*}
  P \approx_{K} Q
\end{eqnarray*}

\begin{eqnarray*}
  P \approx Q
\end{eqnarray*}

$\approx_{K} = \approx = \approx_{L}$

\subsubsection{Contextual duality}

Note that contexts extend the quotation operation to a family of
operations from processes to names. Given a context, $M$, we can
define a \emph{nominal context}, $\quotep{M}$ by $\quotep{M}[P] :=
\quotep{M[P]}$. To foreshadow what is to come we observe that these
operations enjoy a duality with processes very much like the duality
between vectors and maps from vectors to scalars.

Further, because the calculus is essentially higher-order, we have a
correspondence between contexts and processes. More specifically,
given a name $x$ and a context $M$ we can construct $M^{*}_{x}$ such
that 

\begin{mathpar}
  M^{*}_{x} | \lift{x}{P} \red M[P]
\end{mathpar}

namely,

\begin{mathpar}
  M^{*}_{x} := x?(u).M[\dropn{u}]
\end{mathpar}

The dependence of $M^{*}_{x}$ on a name makes it an abstraction, 

\begin{mathpar}
  M^{*} := (x)x?(u).M[\dropn{u}]
\end{mathpar}

\subsection{Additional notation}

It will sometimes be convenient to denote the process a name
quotes. We already have the notation $x = \quotep{P}$, but it will be
convenient to introduce an alternate notation, $\procn{x}$, when we
want to emphasize the connection to the use of the name. Note that, by
virtue of name equivalence, $\quotep{\procn{x}} \nameeq x$; so, the
notation is consistent with previous definitions.

Further, because names have structure it is possible to effect
substitutions on the basis of that structure. This means we need to
upgrade our notation for substitutions, which we accomplish by
adapting comprehension notation. Thus,

\begin{mathpar}
  P\{ y / x : x \in S \}
\end{mathpar}

is interpreted to mean the process derived from P by replacing (in a
capture-avoiding manner) each occurrence of $x$ in $S$ by $y$. For example,

\begin{mathpar}
  P\{ \quotep{\procn{x}|\procn{x}} / x : x \in \freenames{P} \}
\end{mathpar}

will replace each (occurrence) of a free name $x$ in $P$ by
$\quotep{\procn{x}|\procn{x}}$.

Also, we will avail ourselves of the notation $x^{L}$ and $x^{R}$ to
denote injections of a name into disjoint copies of the name
space. There are numerous ways to accomplish this. One example can be
found in \cite{MeredithR05}. This notation overloads to vectors of
names: $\vec{x}^{\pi} := (x_{i}^{\pi} \; : \; 0 \leq i < |\vec{x}| )$ where $\pi \in \{L,R\}$.

We also use $P^{\Box} := P|\Box$.

In \cite{MeredithR05} an interpretation of the new operator is
given. It turns out that there are several possible interpretations
all enjoying the requisite algebraic properties of the operator (see
\cite{milner91polyadicpi}). We will therefore make liberal use of
$(\nu\; \vec{x})P$.

% subsection the_syntax_and_semantics_of_the_notation_system (end)   

\input{qm2pi.qmops} 

\input{qm2pi.sterngerlach} 

\input{qm2pi.metric} 

% section concurrent_process_calculi (end)

%\input{qm2pi.proofsketch}

% section proof sketch (end)

%\input{qm2pi.slviaknots} 

% section spatial logic via knots (end)

\input{qm2pi.conclusion}

% section conclusion (end)

%\input{qm2pi.dtcodes} 

% section wiring algorithm (end)

\input{qm2pi.ack} 

% section acknowledgments (end)

\newpage


\bibliographystyle{plain}   
\bibliography{../../biblios/main.bib}

\input{qm2pi.rhodetails}

\end{document}

 

% subsection basic_interpretation (end)

%\input{qm2pi.rho.presentation} 
\subsection{The syntax and semantics of the notation system}\label{sub:the_syntax_and_semantics_of_the_notation_system} % (fold)

We now summarize a technical presentation of the calculus that
embodies our theory of dynamics. The typical presentation of such a
calculus follows the style of giving generators and relations on
them. The grammar, below, describing term constructors, freely
generates the set of processes, $\Proc$. This set is then quotiented
by a relation known as structural congruence and it is over this set
that the notion of dynamics is expressed. This presentation is
essentially that of \cite{MeredithR05} with the addition of
polyadicity and summation. For readability we have relegated some of
the technical subtleties to an appendix.

\subsubsection{Process grammar}\label{subsub:process_grammar}

\begin{mathpar}
  \inferrule* [lab=synchronization] {} {{M} \bc \pzero \;|\; x?F \;|\; x!C }
  \and
  \inferrule* [lab=abstraction] {} {{F} \bc (x)P}
  \and
  \inferrule* [lab=concretion] {} {{C} \bc \langle Q \rangle}
  \and
  \inferrule* [lab=process] {} {{P,Q} \bc M \;| \;P|Q \;|\; @{x}}
  \and
  \inferrule* [lab=name] {} {{x} \bc \quotep{P}}
\end{mathpar} 

Note that $\vec{x}$ (resp. $\vec{P}$) denotes a vector of names
(resp. processes) of length $|\vec{x}|$ (resp. $|\vec{P}|$). We adopt
the following useful abbreviations.

\begin{mathpar}
   x?(\vec{y}).P := x.(\vec{y})P \and  x\clift{\vec{P}} := x.\clift{\vec{P}}
   \and x!(y) := \lift{x}{\dropn{y}}
   \and \Pi_{i=0}^{n-1}P_i := P_0 | \ldots | P_{n-1}
\end{mathpar}

\subsubsection{Structural congruence}

\paragraph{Free and bound names and alpha-equivalence.} At the
core of structural equivalence is alpha-equivalence which identifies
process that are the same up to a change of variable. Formally, we
recognize the distinction between free and bound names. The free names
of a process, $\freenames{P}$, may be calculated recursively as
follows:

\begin{mathpar}
\freenames{\pzero} := \emptyset
  \and \\
  \freenames{x?(y).P} := \{ x \} \cup (\freenames{P} \setminus \{ y \})
  \and 
  \freenames{x!\langle P \rangle} := \{ x \} \cup \{ P \} 
  \and \\
  \freenames{P|Q} := \freenames{P} \cup \freenames{Q}
  \and \\
  \freenames{@{x}} := \{ x \}
\end{mathpar}

$\pi$
$\quotep{\pi}$

$\freenames{-} : \pi \to \mathcal{P}(\quotep{\pi})$

\begin{eqnarray*}
  \freenames{\pzero} & := & \emptyset \\
  \freenames{x?(y).P} & := & \{ x \} \cup (\freenames{P} \setminus \{ y \}) \\
  \freenames{x!\langle P \rangle} & := & \{ x \} \cup \{ P \} \\
  \freenames{P|Q} & := & \freenames{P} \cup \freenames{Q} \\
  \freenames{\dropn{x}} & := & \{ x \}
\end{eqnarray*}

The bound names of a process, $\boundnames{P}$, are those names occurring in $P$
that are not free. For example, in $x?(y).0$, the name $x$ is free, while $y$ is bound.

\begin{mathpar}
  \inferrule* [lab=monoidal-laws] {} { P|Q \equiv Q|P \and P|0 \equiv P \and P|(Q|R) \equiv (P|Q)|R }
\end{mathpar}

\begin{mathpar}
  \inferrule* [lab=alpha-equivalence] {} { (x)P \equiv (y)P\{y/x\} \and y \not\in \freenames{P} }
\end{mathpar}

\begin{definition}
Then two processes, $P,Q$, are alpha-equivalent if $P = Q\{\vec{y}/\vec{x}\}$ for
some $\vec{x} \in \boundnames{Q},\vec{y} \in \boundnames{P}$, where $Q\{\vec{y}/\vec{x}\}$
denotes the capture-avoiding substitution of $\vec{y}$ for $\vec{x}$ in $Q$.
\end{definition}

\begin{definition}
  The {\em structural congruence} \cite{SangiorgiWalker} , $\equiv$,
  between processes is the least congruence containing
  alpha-equivalence, satisfying the abelian monoid laws
  (associativity, commutativity and $\pzero$ as identity) for parallel
  composition $|$ and for summation $+$.
\end{definition}

\subsection{Name equivalence}

We take name equivalence, written $\nameeq$, to be the smallest
equivalence relation generated by the following rules.

\begin{mathpar}
\inferrule*[lab=Quote-drop]
{ }
{ \quotep{@{x}} \nameeq x }

\inferrule*[lab=Struct-equiv]
{ P \scong Q }
{ \quotep{P} \nameeq \quotep{Q} }
\end{mathpar}

The astute reader will have noticed that the mutual recursion of names
and processes imposes a mutual recursion on alpha-equivalence and
structural equivalence via name-equivalence. Fortunately, all of this
works out pleasantly and we may calculate in the natural way, free of
concern. The reader interested in the details is referred to the
appendix \ref{appendix:rho_details}.

\subsection{Substitution}

We use $\Proc$ for the set of processes, $\QProc$ for the set of
names, and $\id{\{}\vec{y} / \vec{x} \id{\}}$ to denote partial maps,
$s : \QProc \rightarrow \QProc$. A map, $s$ lifts, uniquely, to a map
on process terms, $\widehat{s} : \Proc \rightarrow \Proc$ by the
following equations.

\begin{mathpar}
  (0) \psubstp{Q}{P} := 0 \\
  (R \juxtap S) \psubstp{Q}{P}
  :=    
  (R)\psubstp{Q}{P} \juxtap (S) \psubstp{Q}{P} \\
  (x?(y).R) \psubstp{Q}{P}    
  :=    
  (x)\substp{Q}{P} (z)\concat( (R \psubstn{z}{y}) \psubstp{Q}{P} ) \\
  (\lift{x}{R}) \psubstp{Q}{P}  
  :=
  \lift{(x)\substp{Q}{P}}{ R \psubstp{Q}{P} } \\
%   (\dropn{x})  \psubstp{Q}{P}       
%   := 
%   \left\{ 
%     \begin{array}{ccc} 
%       \dropn{\quotep{Q}} & & x \nameeq \quotep{P} \\
%       \dropn{x} & & otherwise \\
%     \end{array}
%   \right. 
  (\dropn{x})  \psubstp{Q}{P}       
  := 
  \left\{ 
    \begin{array}{ccc} 
      Q & & x \nameeq \quotep{P} \\
      \dropn{x} & & otherwise \\
    \end{array}
  \right.
\end{mathpar}
 

where

\begin{eqnarray}
  (x)\id{\{} \lpquote Q \rpquote / \lpquote P \rpquote \id{\}}            = 
  \left\{ 
    \begin{array}{ccc}
      \lpquote Q \rpquote & & x \nameeq \lpquote P \rpquote \\
      x & & otherwise \\
    \end{array}
  \right. \nonumber
\end{eqnarray}

and $z$ is chosen distinct from $\quotep{P}$, $\quotep{Q}$, the free
names in $Q$, and all the names in $R$. Our $\alpha$-equivalence will
be built in the standard way from this substitution.

\begin{remark}\label{rem:no_self_referential_names}
  One consequence of these definitions is that $\forall P. \quotep{P}
  \not\in \freenames{P}$.
\end{remark}

\subsection{ Dynamic quote: an example }

Anticipating something of what's to come, consider applying the
substitution, $\widehat{\id{\{}u / z \id{\}}}$, to the following pair
of processes, $\lift{w}{y!(z)}$ and $w[ \lpquote y!(z) \rpquote ]$.

\begin{eqnarray}
	\lift{w}{y!(z)}\widehat{\id{\{}u / z \id{\}}}
		& = &
		\lift{w}{y!(u)} \nonumber\\
	w[ \lpquote y!(z) \rpquote ] \widehat{ \id{\{}u / z \id{\}} }
		& = &
		w[ \lpquote y!(z) \rpquote ] \nonumber
\end{eqnarray}

Because the body of the process between quotes is impervious to
substitution, we get radically different answers. In fact, by
examining the first process in an input context,
e.g. $x?(z).\lift{w}{y!(z)}$, we see that the process under the lift
operator may be shaped by prefixed inputs binding a name inside it. In
this sense, the lift operator will be seen as a way to dynamically
construct processes before reifying them as names.

Finally equipped with these standard features we can present the
dynamics of the calculus.

\subsubsection{Operational semantics} 

Finally, we introduce the computational dynamics. What marks these
algebras as distinct from other more traditionally studied algebraic
structures, e.g. vector spaces or polynomial rings, is the manner in
which dynamics is captured. In traditional structures, dynamics is typically
expressed through morphisms between such structures, as in linear maps
between vector spaces or morphisms between rings. In algebras
associated with the semantics of computation, the dynamics is
expressed as part of the algebraic structure itself, through a
reduction reduction relation typically denoted by $\red$. Below, we
give a recursive presentation of this relation for the calculus used
in the encoding.

$\red \subseteq \pi \times \pi$
$\red : \pi \to \mathcal{P}(\pi)$

\begin{mathpar}
  \inferrule* [lab=Comm] { \textsf{match}( x_{src}, x_{trgt} ) } { x_{trgt}?(y)P \; | \; x_{src}!\langle {Q} \rangle \red P\{\quotep{Q}/y}\} }
  \and \\
  \inferrule* [lab=Par] {{P} \red {P}'} {{{P} | {Q}} \red {{P}' | {Q}}}
  \and
  \inferrule* [lab=Equiv]{{{P} \scong {P}'} \andalso {{P}' \red {Q}'} \andalso {{Q}' \scong {Q}}}{{P} \red {Q}}
\end{mathpar}

\begin{eqnarray*}
  match_{\equiv} (\quotep{P},\quotep{Q}) & := & P \equiv Q \\
  match_{\dagger}(\quotep{P},\quotep{Q}) & := & \forall R. P|Q \red^{*} R => R \red^{*} 0 \\
  match_{K}(\quotep{P},\quotep{Q}) & := & K \mbox{ for some context } K
\end{eqnarray*}

$u?(x)P | u!\langle Q \rangle \red P\{\quotep{Q}/x\}$

%We write $\wred$ for $\red^*$, and $P\red$ if $\exists Q $ such that $ P \red Q$.
We write $P\red$ if $\exists Q $ such that $ P \red Q$ and $P\not\red$, otherwise.

\section{Replication}

As mentioned before, it is known that replication (and hence
recursion) can be implemented in a higher-order process algebra
\cite{SangiorgiWalker}. As our first example of calculation with the
machinery thus far presented we give the construction explicitly in
the {\rhoc}.

\begin{eqnarray}
	D_{x} & := & \prefix{x}{y}{(\binpar{\outputp{x}{y}}{@{y}})} \nonumber\\
	\bangp_{x}{P} & := & \binpar{{x}!\langle{\binpar{D_{x}}{P}}\rangle}{D_{x}} \nonumber
\end{eqnarray}

\begin{eqnarray}
	\bangp_{x}{P} & & \nonumber\\
	=
	& {x}!\langle{(\prefix{x}{y}{(\outputp{x}{y} | @{y})) | P}}\rangle 
	      | \prefix{x}{y}{(\outputp{x}{y} | @{y})} & \nonumber\\
	\red
	& (\outputp{x}{y} | @{y})\substn{\quotep{(\prefix{x}{y}{(@{y} | \outputp{x}{y})) | P}}}{y} & \nonumber\\
	=
	& \outputp{x}{\quotep{(\prefix{x}{y}{(\outputp{x}{y} | @{y})) | P}}}
	  | {(\prefix{x}{y}{(\outputp{x}{y} | @{y})) | P}} & \nonumber\\
	\red
	& \ldots & \nonumber\\
	\red^*
	& P | P | \ldots & \nonumber
\end{eqnarray}

Of course, this encoding, as an implementation, runs away, unfolding
$\bangp{P}$ eagerly. A lazier and more implementable replication
operator, restricted to input-guarded processes, may be obtained as follows.

\begin{eqnarray}
\bangp{\prefix{u}{v}{P}} 
	:= 
	\binpar{\lift{x}{\prefix{u}{v}{(\binpar{D(x)}{P})}}}{D(x)} \nonumber
\end{eqnarray}

\begin{remark}
  Note that the lazier definition still does not deal with summation
  or mixed summation (i.e. sums over input and output). The reader is
  invited to construct definitions of replication that deal with these
  features. 

  Further, the definitions are parameterized in a name, $x$. Can you,
  gentle reader, make a definition that eliminates this parameter and
  guarantees no accidental interaction between the replication
  machinery and the process being replicated -- i.e. no accidental
  sharing of names used by the process to get its work done and the
  name(s) used by the replication to effect copying. This latter
  revision of the definition of replication is crucial to obtaining
  the expected identity $!!P \sim !P$.
\end{remark}

\begin{remark}\label{rem:paradoxical_combinator}
  The reader familiar with the lambda calculus will have noticed the
  similarity between $D$ and the paradoxical combinator.

  [Ed. note: the existence of this seems to suggest we have to be more
  restrictive on the set of processes and names we admit if we are to
  support no-cloning.]
\end{remark}

\subsubsection{Bisimulation}

The computational dynamics gives rise to another kind of equivalence,
the equivalence of computational behavior. As previously mentioned
this is typically captured \emph{via} some form of bisimulation.

% The notion we use in this paper is weak barbed bisimulation
% \cite{milner91polyadicpi}.

The notion we use in this paper is derived from weak barbed
bisimulation \cite{milner91polyadicpi}. 

\begin{definition}
An \emph{observation relation}, $\downarrow_{\mathcal N}$, over a set
of names, $\mathcal N$, is the smallest relation satisfying the rules
below.

\infrule[Out-barb]{y \in {\mathcal N}, \; x \nameeq y}
		  {\outputp{x}{v} \downarrow_{\mathcal N} x}
\infrule[Par-barb]{\mbox{$P\downarrow_{\mathcal N} x$ or $Q\downarrow_{\mathcal N} x$}}
		  {\binpar{P}{Q} \downarrow_{\mathcal N} x}

We write $P \Downarrow_{\mathcal N} x$ if there is $Q$ such that 
$P \wred Q$ and $Q \downarrow_{\mathcal N} x$.
\end{definition}

\begin{definition}
%\label{def.bbisim}
An  ${\mathcal N}$-\emph{barbed bisimulation} over a set of names, ${\mathcal N}$, is a symmetric binary relation 
${\mathcal S}_{\mathcal N}$ between agents such that $P\rel{S}_{\mathcal N}Q$ implies:
\begin{enumerate}
\item If $P \red P'$ then $Q \wred Q'$ and $P'\rel{S}_{\mathcal N} Q'$.
\item If $P\downarrow_{\mathcal N} x$, then $Q\Downarrow_{\mathcal N} x$.
\end{enumerate}
$P$ is ${\mathcal N}$-barbed bisimilar to $Q$, written
$P \wbbisim_{\mathcal N} Q$, if $P \rel{S}_{\mathcal N} Q$ for some ${\mathcal N}$-barbed bisimulation ${\mathcal S}_{\mathcal N}$.
\end{definition}

$\mathcal{R} \subseteq \pi \times \pi$

$P \mathcal{R} Q => \forall P'. P \red P' \Rightarrow \exists Q'. Q \red Q', P' \mathcal{R} Q'$

$P \vdash x \Rightarrow Q \vdash x$

\begin{mathpar}
  \inferrule*[lab=Out-barb]{x \nameeq y}{{y}!\langle{Q}\rangle \vdash x}
  \and
  \inferrule*[lab=Par-barb]{\mbox{$P\vdash x$ or $Q\vdash x$}}{\binpar{P}{Q} \vdash x}
\end{mathpar}

\subsubsection{Contexts}

One of the principle advantages of computational calculi like the
$\pi$-calculus is a well-defined notion of context,
contextual-equivalence and a correlation between
contextual-equivalence and notions of bisimulation. The notion of
context allows the decomposition of a process into (sub-)process and
its syntactic environment, its context. Thus, a context may be
thought of as a process with a ``hole'' (written $\Box$) in it. The
application of a context $M$ to a process $P$, written $M[P]$, is
tantamount to filling the hole in $M$ with $P$. In this paper we do
not need the full weight of this theory, but do make use of the notion
of context in the proof the main theorem. 

\begin{mathpar}
  \inferrule* [lab=summation] {} {{M_{M},M_{N}} \bc \Box \;|\; x.M_{A} \;|\; M_{M}+M_{N}}
  \and
  \inferrule* [lab=agent] {} {{M_{A}} \bc (\vec{x})M_{P} \;| \; \clift{P_0,\ldots,M_{P},\ldots,P_N}}
  \and \\
  \inferrule* [lab=process] {} {{M_{P}} \bc M_{N} \;| \;P|M_{P} }
\end{mathpar} 

\begin{mathpar}
  \inferrule* [lab=sychronization] {} {M_{N} \bc \Box \;|\; x?M_{F} \;|\; x!M_{C}}
  \and
  \inferrule* [lab=abstraction] {} {{M_{F}} \bc (x)M_{P} }
  \and
  \inferrule* [lab=concretion] {} {{M_{C}} \bc \langle M_{P} \rangle }
  \and \\
  \inferrule* [lab=process] {} {{M_{P}} \bc M_{N} \;| \;P|M_{P} }
\end{mathpar}

\begin{definition}[contextual application] Given a context $M$, and
  process $P$, we define the \emph{contextual application}, $M[P] :=
  M\{P/\Box\}$. That is, the contextual application of M to P is the
  substitution of $P$ for $\Box$ in $M$.
\end{definition}

$\meaningof{-} : L \to \mathcal{P}(\pi)$

\begin{mathpar}
  \inferrule* [lab=collection] {} {\meaningof{true} = \pi, \and \meaningof{~E} = \pi \setminus \meaningof{E}, \and \meaningof{E_{1} \& E_{2}} = \meaningof{E_{1}} \cap \meaningof{E_{2}}}
\end{mathpar}

\begin{mathpar}
  \inferrule* [lab=structure] {} {\meaningof{0} = \{ P \in \pi | P \equiv 0 \}, \and \\ \meaningof{E_1 | E_2} = \{ P \in \pi | P \equiv P_{1} | P_{2}, P_{1} \in \meaningof{E_{1}}, P_{2} \in \meaningof{E_2}\} }
\end{mathpar}

\begin{mathpar}
 \inferrule* [lab=behavior] {} {\meaningof{\langle a?b \rangle E} = \{ P \in \pi | P \equiv Q | u?(y)P', \\ \and \\\\ \and \\ \;\;\; u \in \meaningof{a}, \forall z.P'\{z/y\} \in \meaningof{E\{z/b\}}\}, \and \\ \meaningof{a!E} = \{ P \in \pi | P \equiv Q | x!\langle P' \rangle, x \in \meaningof{a} P' \in \meaningof{E}\} }
\end{mathpar}

\begin{mathpar}
 \inferrule* [lab=nominal] {} {\meaningof{\quotep{E}} = \{ \quotep{P} \in \quotep{\pi} | P \in \meaningof{E} \}, \and \meaningof{\quotep{P}} = \{ \quotep{Q} \in \quotep{\pi} | P \equiv Q \} \and \\ \meaningof{@\quotep{E}} = \{ P \in \pi | P \equiv @x, x \in \meaningof{E} \}}
\end{mathpar}

\begin{eqnarray*}
  \\
  \meaningof{-} : TS \to ST
\end{eqnarray*}

\begin{eqnarray*}
  \\
  L : TS \to ST
\end{eqnarray*}

\begin{eqnarray*}
  \\
  P \models E \iff P \in \meaningof{E}
\end{eqnarray*}

\begin{eqnarray*}
  P \approx_{L} Q \iff \forall E \in L. P \models E \iff Q \models E
\end{eqnarray*}

\begin{eqnarray*}
  P \approx_{K} Q
\end{eqnarray*}

\begin{eqnarray*}
  P \approx Q
\end{eqnarray*}

$\approx_{K} = \approx = \approx_{L}$

\subsubsection{Contextual duality}

Note that contexts extend the quotation operation to a family of
operations from processes to names. Given a context, $M$, we can
define a \emph{nominal context}, $\quotep{M}$ by $\quotep{M}[P] :=
\quotep{M[P]}$. To foreshadow what is to come we observe that these
operations enjoy a duality with processes very much like the duality
between vectors and maps from vectors to scalars.

Further, because the calculus is essentially higher-order, we have a
correspondence between contexts and processes. More specifically,
given a name $x$ and a context $M$ we can construct $M^{*}_{x}$ such
that 

\begin{mathpar}
  M^{*}_{x} | \lift{x}{P} \red M[P]
\end{mathpar}

namely,

\begin{mathpar}
  M^{*}_{x} := x?(u).M[\dropn{u}]
\end{mathpar}

The dependence of $M^{*}_{x}$ on a name makes it an abstraction, 

\begin{mathpar}
  M^{*} := (x)x?(u).M[\dropn{u}]
\end{mathpar}

\subsection{Additional notation}

It will sometimes be convenient to denote the process a name
quotes. We already have the notation $x = \quotep{P}$, but it will be
convenient to introduce an alternate notation, $\procn{x}$, when we
want to emphasize the connection to the use of the name. Note that, by
virtue of name equivalence, $\quotep{\procn{x}} \nameeq x$; so, the
notation is consistent with previous definitions.

Further, because names have structure it is possible to effect
substitutions on the basis of that structure. This means we need to
upgrade our notation for substitutions, which we accomplish by
adapting comprehension notation. Thus,

\begin{mathpar}
  P\{ y / x : x \in S \}
\end{mathpar}

is interpreted to mean the process derived from P by replacing (in a
capture-avoiding manner) each occurrence of $x$ in $S$ by $y$. For example,

\begin{mathpar}
  P\{ \quotep{\procn{x}|\procn{x}} / x : x \in \freenames{P} \}
\end{mathpar}

will replace each (occurrence) of a free name $x$ in $P$ by
$\quotep{\procn{x}|\procn{x}}$.

Also, we will avail ourselves of the notation $x^{L}$ and $x^{R}$ to
denote injections of a name into disjoint copies of the name
space. There are numerous ways to accomplish this. One example can be
found in \cite{MeredithR05}. This notation overloads to vectors of
names: $\vec{x}^{\pi} := (x_{i}^{\pi} \; : \; 0 \leq i < |\vec{x}| )$ where $\pi \in \{L,R\}$.

We also use $P^{\Box} := P|\Box$.

In \cite{MeredithR05} an interpretation of the new operator is
given. It turns out that there are several possible interpretations
all enjoying the requisite algebraic properties of the operator (see
\cite{milner91polyadicpi}). We will therefore make liberal use of
$(\nu\; \vec{x})P$.

% subsection the_syntax_and_semantics_of_the_notation_system (end)   

\section{Interpretation of QM}
\subsection{Supporting definitions}
\subsubsection{Multiplication}
\begin{mathpar}
  \quotep{Q} \cdot \quotep{R} := \quotep{Q|R}
  \and \\
  \quotep{Q} \cdot P := P\{ \quotep{Q|R} / \quotep{R} : \quotep{R} \in \freenames{P} \}
\end{mathpar}

\paragraph{Discussion}
The first line needs little explanation. The second line says that
each free name of the process is replaced with the multiplication of
that name by the scalar. Multiplication of a scalar (name) by a state
(process) results in a process all the names of which have been `moved
over' by parallel composition with the process the scalar
quotes. There is a subtlety that the bound names have to be
manipulated so that multiplied names aren't accidentally
captured. There are many ways to achieve this.

\begin{remark}\label{rem:multiplication_identities}
  The reader is invited to verify that for all $x,y,z \in \QProc$ and $P \in \Proc$
  \begin{mathpar}
    x \cdot \quotep{0} \equiv x 
    \and
    x \cdot y \equiv y \cdot x
    \and
    x \cdot (y \cdot z) \equiv (x \cdot y) \cdot z
    \and \\
    \quotep{0} \cdot P \equiv P
    \and \\
    x \cdot (y \cdot P) \equiv (x \cdot y) \cdot P
    \and \\
    x \cdot (P|Q) \equiv (x \cdot P) | (x \cdot Q)
    \and \\    
  \end{mathpar}
\end{remark}

\subsubsection{Tensor product}

We define a tensor product on processes by structural induction.

\paragraph{Tensor of sums} First note that all summations, including
$\pzero$ and sequence, can be written $\Sigma_{i} x_{i}.A_{i} +
\Sigma_{j} x_{j}.C_{j}$, where we have grouped input-guarded processes
together and output-guarded processes together.

Thus, we can define the tensor product of two summations, $N_{1}\otimes N_{2}$, where

\begin{mathpar}
  N_{1} := \Sigma_{i} x_{i}.A_{i} + \Sigma_{j} x_{j}.C_{j}
  \and
  N_{2} := \Sigma_{i'} y_{i'}.B_{i'} + \Sigma_{j'} y_{j'}.D_{j'} 
\end{mathpar}

as follows.

\begin{mathpar}
  \Sigma_{i} x_{i}.A_{i} + \Sigma_{j} x_{j}.C_{j} \otimes \Sigma_{i'}
  y_{i'}.B_{i'} + \Sigma_{j'} y_{j'}.D_{j'} 
  \and \\
  := \; \Sigma_{i} \Sigma_{i'} \quotep{\stackrel{\vee}{x_{i}}| \stackrel{\vee}{y_{i'}}}.(A_{i}\otimes B_{i'}) \; | \; \Sigma_{i'} \Sigma_{i} \quotep{\stackrel{\vee}{y_{i'}}|\stackrel{\vee}{x_{i}}}.(B_{i'}\otimes A_{i})
  \and
  \;\; | \;\; \Sigma_{j} \Sigma_{j'} \quotep{\stackrel{\vee}{x_{j}}|\stackrel{\vee}{y_{j'}}}.(A_{j}\otimes B_{j'}) \; | \; \Sigma_{j'} \Sigma_{j} \quotep{\stackrel{\vee}{y_{j'}}|\stackrel{\vee}{x_{j}}}.(B_{j'}\otimes A_{j})
\end{mathpar}

\begin{remark}
  Do we need to $x^{L}$ and $y^{R}$ for this construction as well?
\end{remark}

\paragraph{Tensor of parallel compositions} Next, we distribute tensor
over par.

\begin{mathpar}
  P_{1}|P_{2} \otimes Q_{1}|Q_{2} := (P_{1} \otimes Q_{1}) | (P_{1}
  \otimes Q_{2}) | (P_{2} \otimes Q_{1}) | (P_{2} \otimes Q_{2})
\end{mathpar}

\paragraph{Tensor with dropped names} We treat tensor of a
process with a dropped name as parallel composition.

\begin{mathpar}
  P \otimes \dropn{x} := P | \dropn{x}
\end{mathpar}

\paragraph{Tensor of agents}

Finally, we need to define tensor on agents. Note that the definition
of tensor on normal products only tensors inputs with inputs and
outputs with outputs. Thus, we only have to define the operation on
``homogeneous'' pairings.

\begin{mathpar}
  (\vec{x})P \otimes (\vec{y})Q
  \and \\
  := (x_{0}^{L}|y_{0}^{R},\ldots,x_{0}^{L}|y_{n}^{R},\ldots,x_{m}^{L}|y_{0}^{R},\ldots,x_{m}^{L}|y_{n}^R)(P\{ \vec{x}^{L}/\vec{x}\} \otimes Q \{ \vec{y}^{R}/\vec{y}\})
  \and \\
  \clift{\vec{P}} \otimes \clift{\vec{Q}}
  \and \\
  := \clift{P_{0}\otimes Q_{0},\ldots,P_{0}\otimes Q_{n},\ldots,P_{m}\otimes Q_{0},\ldots,P_{m}\otimes Q_{n}}
\end{mathpar}

\begin{remark}
  Observe that arities of tensored abstractions matches arities of
  tensored concretions if the original arities matched. Note also that
  the length of the arities corresponds to the increase in dimension
  we see in ordinary vector space tensor product.
\end{remark}

\begin{remark}
  Operationally, this definition distributes the tensor down to
  components ``linked'' by summation. Tensor over summation is
  intriguing in that it mixes names. Moreover, as a consequence of the
  way it mixes names we have the identities for all $x \in \QProc$ and
  $P,Q \in \Proc$

  \begin{mathpar}
    (x \cdot P) \otimes Q \equiv x \cdot (P \otimes Q) \equiv P \otimes (x \cdot Q)
    \and
    P \otimes \pzero \equiv P
  \end{mathpar}

  that the reader is invited to verify.
\end{remark}

\subsubsection{Annihilation}
\begin{mathpar}
  P^{\perp} := \{ Q | \forall R. P|Q \red^{*} R \Rightarrow R \red^{*} \pzero \}
  \and \\
  P^{\underline{\perp}} := \Sigma_{Q \in P^{\perp}} \quotep{Q}?(y).(\dropn{y}|Q) | \Sigma_{Q \in P^{\perp}} \quotep{Q}\clift{\Box}
\end{mathpar}

\paragraph{Discussion} The reader will note that $P^{\perp}$ is a
\emph{set} of processes, while $P^{\underline{\perp}}$ is a
\emph{context}. We call the set $P^{\perp}$ the \emph{annihilators} of
$P$. The parallel composition of a process in the annihilators of $P$
with $P$ will result in a process, the state space of which has all
paths eventually leading to $\pzero$. Execution may endure loops; but
under reasonable conditions of fairness (naturally guaranteed under
most notions of bisimulation) such a composite process cannot get
stuck in such a loop and will, eventually pop out and terminate.

The context $P^{\underline{\perp}}$ is ready and willing to ``take the
$P$ out of'' the process to which it is applied. It will effectively
transmit the code of the process to which it is applied to one of the
annihilators and run the process against it.

\subsubsection{Evaluation}
We fix $M$ a domain of fully abstract interpretation with an equality
coincident with bisimulation. We take $\meaningof{\cdot} : \Proc \to
M$ to be the map interpreting processes and $\nmeaningof{\cdot} : \M
\to Proc$ to be the map running the other way. Then we define

\begin{mathpar}
  \int P := \nmeaningof{\meaningof{P}}
\end{mathpar}

\paragraph{Discussion}
There are many fully abstract interpretations of Milner's
$\pi$-calculus. Any of them can be used as a basis for interpreting
the reflective calculus here. Equipped with such a domain it is
largely a matter of grinding through to check that the Yoneda
construction for the normalization-by-evaluation program can be
extended to this setting.

\begin{remark}
  The reader is invited to verify that $\int (P^{\underline{\perp}}[P]) = 0$.
\end{remark}

\subsection{Quantum mechanics}

Table \ref{tbl:core_qm_op_defns} gives the core operational definitions

\begin{table}[htp]\label{tbl:core_qm_op_defns}
  \center{
    \fbox{
      \begin{tabular}{c|c}
        quantum mechanics & process calculus \\
        \hline
        scalar & $x := \quotep{P}$ \\
        state vector & $\state{P} := P$ \\
        dual & $\state{P}^{*} := \event{P^{\underline{\perp}}} := \quotep{P^{\underline{\perp}}}[-]$ \\
        matrix & $ \Sigma_{\alpha} \state{P_{\alpha}}x_{\alpha}\event{Q_{\alpha}}$ \\
        vector addition & $\state{P} + \state{Q} := \state{P | Q}$ \\
        tensor product & $\state{P} \otimes \state{Q} := \state{P \otimes Q}$ \\
        inner product & $\innerprod{P}{Q} := \quotep{\int P^{\underline{\perp}}[Q]}$ \\
      \end{tabular}
    }
  }
  \caption{QM - operational definitions}
\end{table}

where

\begin{mathpar}
  \prmatrix{P}{Q} := \fprmatrix{P}{\quotep{\pzero}}{Q}
  \and
  \fprmatrix{P}{x}{Q} := (\state{P},x,\event{Q})
  \and
  (\fprmatrix{P}{x}{Q})(\state{R}) := x \cdot \innerprod{Q}{R} \cdot \state{P}
  \and
  (\fprmatrix{P}{x}{Q})(\event{R}) := x \cdot \innerprod{R}{P} \cdot \event{Q}
\end{mathpar}

\paragraph{Discussion}
As promised: vectors (aka states) are represented as processes; duals
as contextual duals; inner product definition should be compared with
standard inner product definition for ....

\begin{remark}
  Assuming $\int (P^{\underline{\perp}}[P]) = 0$, the reader is
  invited to verify that $(\fprmatrix{P}{x}{P})(\state{P}) = x \cdot \state{P}$.
\end{remark}

\begin{remark}
  The reader is invited to verify that $\innerprod{P}{Q}$ could
  equally well have been written $\quotep{\int \stackrel{\vee}{x}}$
  where $x = \event{P^{\underline{\perp}}}(Q)$.

  One of the motivations for this remark is that there is another way
  to factor these operations. We could package up evaluation in the dual:

  \begin{mathpar}
    \state{P}^{*} := \event{\int P^{\underline{\perp}}} := \quotep{\int P^{\underline{\perp}}}[-]
  \end{mathpar}

  and then have inner product defined by
  
  \begin{mathpar}
    \innerprod{P}{Q} := \event{P}(Q)
  \end{mathpar}

  Hopefully, experience with the calculations will provide guidance on
  the best factoring.
\end{remark}

\begin{remark}
  Assuming $\int (P^{\underline{\perp}}[P]) = 0$, the reader is
  invited to verify that $\forall P,Q. (\prmatrix{0}{Q})(\state{0}) =
  \state{0}$ and dually $(\prmatrix{P}{0})(\event{0}) = \event{0}$.
\end{remark}

\begin{remark}
  i'm a little worried that i don't (yet) have proper support for
  complex conjugacy. But, the observation above may give us a
  clue. According to Abramsky, it must be the case that the scalars
  are iso to the homset of the identity for the tensor -- which the
  observation above characterizes. 

  For now, we will simply bookmark the notion with $\overline{x}$.
\end{remark}

\subsubsection{Adjointness}

We need to give a definition of $(\cdot)^{\dagger}$ for matrices. The
obvious candidate definition is
\begin{mathpar}
(\Sigma_{\alpha}\fprmatrix{P_{\alpha}}{x_{\alpha}}{Q_{\alpha}})^{\dagger}
= \Sigma_{\alpha}\fprmatrix{(Q_{\alpha}^{\underline{\perp}})^{*}}{\overline{x}_{\alpha}}{P_{\alpha}^{\underline{\perp}}} 
\end{mathpar}

But, $(Q_{\alpha}^{\underline{\perp}})^{*}$ requires a name along
which to communicate the process to achieve the context application.

\subsubsection{Basis for a basis}
If processes label states and ``addition'' of states (a.k.a. vector
addition) is interpreted as parallel composition, what corresponds to
notions of linear independence and basis? Here, we recall that Yoshida
has developed a set of \emph{combinators} for an asynchronous verison
of Milner's $\pi$-calculus. These are a finite set of processes such
any process can be expressed as parallel composition of these
combinators together with liberal uses of the new operator and
replication. We can simply give a translation of these into the
present calculus and have reasonable expectation that the property
carries over. That is, that the resultant set allows to express all
processes via parallel composition. Note, however, that there is no
new operator or replication in this calculus. As a result, we expect
that the corresponding set is actually infinite. That is, we expect
that the space is actually infinite dimensional.

\begin{remark}
  The attentive reader may be a bit concerned. Certainly, the
  collection $S$, $K$ and $I$ is a finite set of
  combinators. Shouldn't we expect to see a finite set of combinators
  for an effectively equivalent system? i am very sympathetic to this
  critique and feel it warrants full attention. On the other hand, i
  also have in mind the following analogy. The natural numbers, as a
  monoid under addition, has exactly $1$ generator, while the natural
  numbers, as a monoid under multiplication, has countably many
  generators (the primes). We observe that the application of the
  lambda calculus is much less resource sensitive than the parallel
  composition of the $\pi$-calculus. Could it be the case that we have
  an analogy of the form
  
  \begin{mathpar}
    m + n : MN :: m*n : M|N
  \end{mathpar}

  giving a similar blow up in the set of ``primes''?  This is such a
  wonderful thought that, even if it's not true, i think it's worth
  writing down.
\end{remark}
 

\documentclass[12pt]{llncs}
%\documentclass{jktr}

\usepackage[pdftex]{hyperref}                   
\usepackage {listings}
\usepackage {mathpartir}
\usepackage{bcprules}
%\usepackage{listings}
                       
\usepackage{graphicx} 
%\usepackage[margins=2.5cm,nohead,nofoot]{geometry}
%\usepackage{geometry}
\usepackage{amsfonts}
\usepackage{amstext}
\usepackage{latexsym}
\usepackage{amssymb}
\usepackage{color}


%\include{myPreamble}
\include{qm2pi.local} 

%\ifpdf
%\usepackage[pdftex]{graphicx}
%\else
%\usepackage{graphicx}
%\fi

 % \ifpdf
%  \usepackage{pdfsync}
%  \if


%\title{Brief Article}
%\author{David F. Snyder}
%\author{L.G. Meredith}

%\address{Dept. of Math., Texas State University--San Marcos, San Marcos, TX 78666}
       
\pagestyle{empty}


\begin{document}

\lstset{language=[Objective]Caml,frame=shadowbox}

\input{qm2pi.front}

% section front matter (end)

\input{qm2pi.intro} 
 
% section introduction (end)

% \input{qm2pi.knotations} 

% section notation (end)

\input{qm2pi.process.calculi} 

% section concurrent_process_calculi_and_spatial_logics_ (end)
    
%\input{qm2pi.knots2pi} 

%\input{qm2pi.trefoil} 

%\input{qm2pi.mainthm} 

% subsection basic_interpretation (end)

%\input{qm2pi.rho.presentation} 
\subsection{The syntax and semantics of the notation system}\label{sub:the_syntax_and_semantics_of_the_notation_system} % (fold)

We now summarize a technical presentation of the calculus that
embodies our theory of dynamics. The typical presentation of such a
calculus follows the style of giving generators and relations on
them. The grammar, below, describing term constructors, freely
generates the set of processes, $\Proc$. This set is then quotiented
by a relation known as structural congruence and it is over this set
that the notion of dynamics is expressed. This presentation is
essentially that of \cite{MeredithR05} with the addition of
polyadicity and summation. For readability we have relegated some of
the technical subtleties to an appendix.

\subsubsection{Process grammar}\label{subsub:process_grammar}

\begin{mathpar}
  \inferrule* [lab=synchronization] {} {{M} \bc \pzero \;|\; x?F \;|\; x!C }
  \and
  \inferrule* [lab=abstraction] {} {{F} \bc (x)P}
  \and
  \inferrule* [lab=concretion] {} {{C} \bc \langle Q \rangle}
  \and
  \inferrule* [lab=process] {} {{P,Q} \bc M \;| \;P|Q \;|\; @{x}}
  \and
  \inferrule* [lab=name] {} {{x} \bc \quotep{P}}
\end{mathpar} 

Note that $\vec{x}$ (resp. $\vec{P}$) denotes a vector of names
(resp. processes) of length $|\vec{x}|$ (resp. $|\vec{P}|$). We adopt
the following useful abbreviations.

\begin{mathpar}
   x?(\vec{y}).P := x.(\vec{y})P \and  x\clift{\vec{P}} := x.\clift{\vec{P}}
   \and x!(y) := \lift{x}{\dropn{y}}
   \and \Pi_{i=0}^{n-1}P_i := P_0 | \ldots | P_{n-1}
\end{mathpar}

\subsubsection{Structural congruence}

\paragraph{Free and bound names and alpha-equivalence.} At the
core of structural equivalence is alpha-equivalence which identifies
process that are the same up to a change of variable. Formally, we
recognize the distinction between free and bound names. The free names
of a process, $\freenames{P}$, may be calculated recursively as
follows:

\begin{mathpar}
\freenames{\pzero} := \emptyset
  \and \\
  \freenames{x?(y).P} := \{ x \} \cup (\freenames{P} \setminus \{ y \})
  \and 
  \freenames{x!\langle P \rangle} := \{ x \} \cup \{ P \} 
  \and \\
  \freenames{P|Q} := \freenames{P} \cup \freenames{Q}
  \and \\
  \freenames{@{x}} := \{ x \}
\end{mathpar}

$\pi$
$\quotep{\pi}$

$\freenames{-} : \pi \to \mathcal{P}(\quotep{\pi})$

\begin{eqnarray*}
  \freenames{\pzero} & := & \emptyset \\
  \freenames{x?(y).P} & := & \{ x \} \cup (\freenames{P} \setminus \{ y \}) \\
  \freenames{x!\langle P \rangle} & := & \{ x \} \cup \{ P \} \\
  \freenames{P|Q} & := & \freenames{P} \cup \freenames{Q} \\
  \freenames{\dropn{x}} & := & \{ x \}
\end{eqnarray*}

The bound names of a process, $\boundnames{P}$, are those names occurring in $P$
that are not free. For example, in $x?(y).0$, the name $x$ is free, while $y$ is bound.

\begin{mathpar}
  \inferrule* [lab=monoidal-laws] {} { P|Q \equiv Q|P \and P|0 \equiv P \and P|(Q|R) \equiv (P|Q)|R }
\end{mathpar}

\begin{mathpar}
  \inferrule* [lab=alpha-equivalence] {} { (x)P \equiv (y)P\{y/x\} \and y \not\in \freenames{P} }
\end{mathpar}

\begin{definition}
Then two processes, $P,Q$, are alpha-equivalent if $P = Q\{\vec{y}/\vec{x}\}$ for
some $\vec{x} \in \boundnames{Q},\vec{y} \in \boundnames{P}$, where $Q\{\vec{y}/\vec{x}\}$
denotes the capture-avoiding substitution of $\vec{y}$ for $\vec{x}$ in $Q$.
\end{definition}

\begin{definition}
  The {\em structural congruence} \cite{SangiorgiWalker} , $\equiv$,
  between processes is the least congruence containing
  alpha-equivalence, satisfying the abelian monoid laws
  (associativity, commutativity and $\pzero$ as identity) for parallel
  composition $|$ and for summation $+$.
\end{definition}

\subsection{Name equivalence}

We take name equivalence, written $\nameeq$, to be the smallest
equivalence relation generated by the following rules.

\begin{mathpar}
\inferrule*[lab=Quote-drop]
{ }
{ \quotep{@{x}} \nameeq x }

\inferrule*[lab=Struct-equiv]
{ P \scong Q }
{ \quotep{P} \nameeq \quotep{Q} }
\end{mathpar}

The astute reader will have noticed that the mutual recursion of names
and processes imposes a mutual recursion on alpha-equivalence and
structural equivalence via name-equivalence. Fortunately, all of this
works out pleasantly and we may calculate in the natural way, free of
concern. The reader interested in the details is referred to the
appendix \ref{appendix:rho_details}.

\subsection{Substitution}

We use $\Proc$ for the set of processes, $\QProc$ for the set of
names, and $\id{\{}\vec{y} / \vec{x} \id{\}}$ to denote partial maps,
$s : \QProc \rightarrow \QProc$. A map, $s$ lifts, uniquely, to a map
on process terms, $\widehat{s} : \Proc \rightarrow \Proc$ by the
following equations.

\begin{mathpar}
  (0) \psubstp{Q}{P} := 0 \\
  (R \juxtap S) \psubstp{Q}{P}
  :=    
  (R)\psubstp{Q}{P} \juxtap (S) \psubstp{Q}{P} \\
  (x?(y).R) \psubstp{Q}{P}    
  :=    
  (x)\substp{Q}{P} (z)\concat( (R \psubstn{z}{y}) \psubstp{Q}{P} ) \\
  (\lift{x}{R}) \psubstp{Q}{P}  
  :=
  \lift{(x)\substp{Q}{P}}{ R \psubstp{Q}{P} } \\
%   (\dropn{x})  \psubstp{Q}{P}       
%   := 
%   \left\{ 
%     \begin{array}{ccc} 
%       \dropn{\quotep{Q}} & & x \nameeq \quotep{P} \\
%       \dropn{x} & & otherwise \\
%     \end{array}
%   \right. 
  (\dropn{x})  \psubstp{Q}{P}       
  := 
  \left\{ 
    \begin{array}{ccc} 
      Q & & x \nameeq \quotep{P} \\
      \dropn{x} & & otherwise \\
    \end{array}
  \right.
\end{mathpar}
 

where

\begin{eqnarray}
  (x)\id{\{} \lpquote Q \rpquote / \lpquote P \rpquote \id{\}}            = 
  \left\{ 
    \begin{array}{ccc}
      \lpquote Q \rpquote & & x \nameeq \lpquote P \rpquote \\
      x & & otherwise \\
    \end{array}
  \right. \nonumber
\end{eqnarray}

and $z$ is chosen distinct from $\quotep{P}$, $\quotep{Q}$, the free
names in $Q$, and all the names in $R$. Our $\alpha$-equivalence will
be built in the standard way from this substitution.

\begin{remark}\label{rem:no_self_referential_names}
  One consequence of these definitions is that $\forall P. \quotep{P}
  \not\in \freenames{P}$.
\end{remark}

\subsection{ Dynamic quote: an example }

Anticipating something of what's to come, consider applying the
substitution, $\widehat{\id{\{}u / z \id{\}}}$, to the following pair
of processes, $\lift{w}{y!(z)}$ and $w[ \lpquote y!(z) \rpquote ]$.

\begin{eqnarray}
	\lift{w}{y!(z)}\widehat{\id{\{}u / z \id{\}}}
		& = &
		\lift{w}{y!(u)} \nonumber\\
	w[ \lpquote y!(z) \rpquote ] \widehat{ \id{\{}u / z \id{\}} }
		& = &
		w[ \lpquote y!(z) \rpquote ] \nonumber
\end{eqnarray}

Because the body of the process between quotes is impervious to
substitution, we get radically different answers. In fact, by
examining the first process in an input context,
e.g. $x?(z).\lift{w}{y!(z)}$, we see that the process under the lift
operator may be shaped by prefixed inputs binding a name inside it. In
this sense, the lift operator will be seen as a way to dynamically
construct processes before reifying them as names.

Finally equipped with these standard features we can present the
dynamics of the calculus.

\subsubsection{Operational semantics} 

Finally, we introduce the computational dynamics. What marks these
algebras as distinct from other more traditionally studied algebraic
structures, e.g. vector spaces or polynomial rings, is the manner in
which dynamics is captured. In traditional structures, dynamics is typically
expressed through morphisms between such structures, as in linear maps
between vector spaces or morphisms between rings. In algebras
associated with the semantics of computation, the dynamics is
expressed as part of the algebraic structure itself, through a
reduction reduction relation typically denoted by $\red$. Below, we
give a recursive presentation of this relation for the calculus used
in the encoding.

$\red \subseteq \pi \times \pi$
$\red : \pi \to \mathcal{P}(\pi)$

\begin{mathpar}
  \inferrule* [lab=Comm] { \textsf{match}( x_{src}, x_{trgt} ) } { x_{trgt}?(y)P \; | \; x_{src}!\langle {Q} \rangle \red P\{\quotep{Q}/y}\} }
  \and \\
  \inferrule* [lab=Par] {{P} \red {P}'} {{{P} | {Q}} \red {{P}' | {Q}}}
  \and
  \inferrule* [lab=Equiv]{{{P} \scong {P}'} \andalso {{P}' \red {Q}'} \andalso {{Q}' \scong {Q}}}{{P} \red {Q}}
\end{mathpar}

\begin{eqnarray*}
  match_{\equiv} (\quotep{P},\quotep{Q}) & := & P \equiv Q \\
  match_{\dagger}(\quotep{P},\quotep{Q}) & := & \forall R. P|Q \red^{*} R => R \red^{*} 0 \\
  match_{K}(\quotep{P},\quotep{Q}) & := & K \mbox{ for some context } K
\end{eqnarray*}

$u?(x)P | u!\langle Q \rangle \red P\{\quotep{Q}/x\}$

%We write $\wred$ for $\red^*$, and $P\red$ if $\exists Q $ such that $ P \red Q$.
We write $P\red$ if $\exists Q $ such that $ P \red Q$ and $P\not\red$, otherwise.

\section{Replication}

As mentioned before, it is known that replication (and hence
recursion) can be implemented in a higher-order process algebra
\cite{SangiorgiWalker}. As our first example of calculation with the
machinery thus far presented we give the construction explicitly in
the {\rhoc}.

\begin{eqnarray}
	D_{x} & := & \prefix{x}{y}{(\binpar{\outputp{x}{y}}{@{y}})} \nonumber\\
	\bangp_{x}{P} & := & \binpar{{x}!\langle{\binpar{D_{x}}{P}}\rangle}{D_{x}} \nonumber
\end{eqnarray}

\begin{eqnarray}
	\bangp_{x}{P} & & \nonumber\\
	=
	& {x}!\langle{(\prefix{x}{y}{(\outputp{x}{y} | @{y})) | P}}\rangle 
	      | \prefix{x}{y}{(\outputp{x}{y} | @{y})} & \nonumber\\
	\red
	& (\outputp{x}{y} | @{y})\substn{\quotep{(\prefix{x}{y}{(@{y} | \outputp{x}{y})) | P}}}{y} & \nonumber\\
	=
	& \outputp{x}{\quotep{(\prefix{x}{y}{(\outputp{x}{y} | @{y})) | P}}}
	  | {(\prefix{x}{y}{(\outputp{x}{y} | @{y})) | P}} & \nonumber\\
	\red
	& \ldots & \nonumber\\
	\red^*
	& P | P | \ldots & \nonumber
\end{eqnarray}

Of course, this encoding, as an implementation, runs away, unfolding
$\bangp{P}$ eagerly. A lazier and more implementable replication
operator, restricted to input-guarded processes, may be obtained as follows.

\begin{eqnarray}
\bangp{\prefix{u}{v}{P}} 
	:= 
	\binpar{\lift{x}{\prefix{u}{v}{(\binpar{D(x)}{P})}}}{D(x)} \nonumber
\end{eqnarray}

\begin{remark}
  Note that the lazier definition still does not deal with summation
  or mixed summation (i.e. sums over input and output). The reader is
  invited to construct definitions of replication that deal with these
  features. 

  Further, the definitions are parameterized in a name, $x$. Can you,
  gentle reader, make a definition that eliminates this parameter and
  guarantees no accidental interaction between the replication
  machinery and the process being replicated -- i.e. no accidental
  sharing of names used by the process to get its work done and the
  name(s) used by the replication to effect copying. This latter
  revision of the definition of replication is crucial to obtaining
  the expected identity $!!P \sim !P$.
\end{remark}

\begin{remark}\label{rem:paradoxical_combinator}
  The reader familiar with the lambda calculus will have noticed the
  similarity between $D$ and the paradoxical combinator.

  [Ed. note: the existence of this seems to suggest we have to be more
  restrictive on the set of processes and names we admit if we are to
  support no-cloning.]
\end{remark}

\subsubsection{Bisimulation}

The computational dynamics gives rise to another kind of equivalence,
the equivalence of computational behavior. As previously mentioned
this is typically captured \emph{via} some form of bisimulation.

% The notion we use in this paper is weak barbed bisimulation
% \cite{milner91polyadicpi}.

The notion we use in this paper is derived from weak barbed
bisimulation \cite{milner91polyadicpi}. 

\begin{definition}
An \emph{observation relation}, $\downarrow_{\mathcal N}$, over a set
of names, $\mathcal N$, is the smallest relation satisfying the rules
below.

\infrule[Out-barb]{y \in {\mathcal N}, \; x \nameeq y}
		  {\outputp{x}{v} \downarrow_{\mathcal N} x}
\infrule[Par-barb]{\mbox{$P\downarrow_{\mathcal N} x$ or $Q\downarrow_{\mathcal N} x$}}
		  {\binpar{P}{Q} \downarrow_{\mathcal N} x}

We write $P \Downarrow_{\mathcal N} x$ if there is $Q$ such that 
$P \wred Q$ and $Q \downarrow_{\mathcal N} x$.
\end{definition}

\begin{definition}
%\label{def.bbisim}
An  ${\mathcal N}$-\emph{barbed bisimulation} over a set of names, ${\mathcal N}$, is a symmetric binary relation 
${\mathcal S}_{\mathcal N}$ between agents such that $P\rel{S}_{\mathcal N}Q$ implies:
\begin{enumerate}
\item If $P \red P'$ then $Q \wred Q'$ and $P'\rel{S}_{\mathcal N} Q'$.
\item If $P\downarrow_{\mathcal N} x$, then $Q\Downarrow_{\mathcal N} x$.
\end{enumerate}
$P$ is ${\mathcal N}$-barbed bisimilar to $Q$, written
$P \wbbisim_{\mathcal N} Q$, if $P \rel{S}_{\mathcal N} Q$ for some ${\mathcal N}$-barbed bisimulation ${\mathcal S}_{\mathcal N}$.
\end{definition}

$\mathcal{R} \subseteq \pi \times \pi$

$P \mathcal{R} Q => \forall P'. P \red P' \Rightarrow \exists Q'. Q \red Q', P' \mathcal{R} Q'$

$P \vdash x \Rightarrow Q \vdash x$

\begin{mathpar}
  \inferrule*[lab=Out-barb]{x \nameeq y}{{y}!\langle{Q}\rangle \vdash x}
  \and
  \inferrule*[lab=Par-barb]{\mbox{$P\vdash x$ or $Q\vdash x$}}{\binpar{P}{Q} \vdash x}
\end{mathpar}

\subsubsection{Contexts}

One of the principle advantages of computational calculi like the
$\pi$-calculus is a well-defined notion of context,
contextual-equivalence and a correlation between
contextual-equivalence and notions of bisimulation. The notion of
context allows the decomposition of a process into (sub-)process and
its syntactic environment, its context. Thus, a context may be
thought of as a process with a ``hole'' (written $\Box$) in it. The
application of a context $M$ to a process $P$, written $M[P]$, is
tantamount to filling the hole in $M$ with $P$. In this paper we do
not need the full weight of this theory, but do make use of the notion
of context in the proof the main theorem. 

\begin{mathpar}
  \inferrule* [lab=summation] {} {{M_{M},M_{N}} \bc \Box \;|\; x.M_{A} \;|\; M_{M}+M_{N}}
  \and
  \inferrule* [lab=agent] {} {{M_{A}} \bc (\vec{x})M_{P} \;| \; \clift{P_0,\ldots,M_{P},\ldots,P_N}}
  \and \\
  \inferrule* [lab=process] {} {{M_{P}} \bc M_{N} \;| \;P|M_{P} }
\end{mathpar} 

\begin{mathpar}
  \inferrule* [lab=sychronization] {} {M_{N} \bc \Box \;|\; x?M_{F} \;|\; x!M_{C}}
  \and
  \inferrule* [lab=abstraction] {} {{M_{F}} \bc (x)M_{P} }
  \and
  \inferrule* [lab=concretion] {} {{M_{C}} \bc \langle M_{P} \rangle }
  \and \\
  \inferrule* [lab=process] {} {{M_{P}} \bc M_{N} \;| \;P|M_{P} }
\end{mathpar}

\begin{definition}[contextual application] Given a context $M$, and
  process $P$, we define the \emph{contextual application}, $M[P] :=
  M\{P/\Box\}$. That is, the contextual application of M to P is the
  substitution of $P$ for $\Box$ in $M$.
\end{definition}

$\meaningof{-} : L \to \mathcal{P}(\pi)$

\begin{mathpar}
  \inferrule* [lab=collection] {} {\meaningof{true} = \pi, \and \meaningof{~E} = \pi \setminus \meaningof{E}, \and \meaningof{E_{1} \& E_{2}} = \meaningof{E_{1}} \cap \meaningof{E_{2}}}
\end{mathpar}

\begin{mathpar}
  \inferrule* [lab=structure] {} {\meaningof{0} = \{ P \in \pi | P \equiv 0 \}, \and \\ \meaningof{E_1 | E_2} = \{ P \in \pi | P \equiv P_{1} | P_{2}, P_{1} \in \meaningof{E_{1}}, P_{2} \in \meaningof{E_2}\} }
\end{mathpar}

\begin{mathpar}
 \inferrule* [lab=behavior] {} {\meaningof{\langle a?b \rangle E} = \{ P \in \pi | P \equiv Q | u?(y)P', \\ \and \\\\ \and \\ \;\;\; u \in \meaningof{a}, \forall z.P'\{z/y\} \in \meaningof{E\{z/b\}}\}, \and \\ \meaningof{a!E} = \{ P \in \pi | P \equiv Q | x!\langle P' \rangle, x \in \meaningof{a} P' \in \meaningof{E}\} }
\end{mathpar}

\begin{mathpar}
 \inferrule* [lab=nominal] {} {\meaningof{\quotep{E}} = \{ \quotep{P} \in \quotep{\pi} | P \in \meaningof{E} \}, \and \meaningof{\quotep{P}} = \{ \quotep{Q} \in \quotep{\pi} | P \equiv Q \} \and \\ \meaningof{@\quotep{E}} = \{ P \in \pi | P \equiv @x, x \in \meaningof{E} \}}
\end{mathpar}

\begin{eqnarray*}
  \\
  \meaningof{-} : TS \to ST
\end{eqnarray*}

\begin{eqnarray*}
  \\
  L : TS \to ST
\end{eqnarray*}

\begin{eqnarray*}
  \\
  P \models E \iff P \in \meaningof{E}
\end{eqnarray*}

\begin{eqnarray*}
  P \approx_{L} Q \iff \forall E \in L. P \models E \iff Q \models E
\end{eqnarray*}

\begin{eqnarray*}
  P \approx_{K} Q
\end{eqnarray*}

\begin{eqnarray*}
  P \approx Q
\end{eqnarray*}

$\approx_{K} = \approx = \approx_{L}$

\subsubsection{Contextual duality}

Note that contexts extend the quotation operation to a family of
operations from processes to names. Given a context, $M$, we can
define a \emph{nominal context}, $\quotep{M}$ by $\quotep{M}[P] :=
\quotep{M[P]}$. To foreshadow what is to come we observe that these
operations enjoy a duality with processes very much like the duality
between vectors and maps from vectors to scalars.

Further, because the calculus is essentially higher-order, we have a
correspondence between contexts and processes. More specifically,
given a name $x$ and a context $M$ we can construct $M^{*}_{x}$ such
that 

\begin{mathpar}
  M^{*}_{x} | \lift{x}{P} \red M[P]
\end{mathpar}

namely,

\begin{mathpar}
  M^{*}_{x} := x?(u).M[\dropn{u}]
\end{mathpar}

The dependence of $M^{*}_{x}$ on a name makes it an abstraction, 

\begin{mathpar}
  M^{*} := (x)x?(u).M[\dropn{u}]
\end{mathpar}

\subsection{Additional notation}

It will sometimes be convenient to denote the process a name
quotes. We already have the notation $x = \quotep{P}$, but it will be
convenient to introduce an alternate notation, $\procn{x}$, when we
want to emphasize the connection to the use of the name. Note that, by
virtue of name equivalence, $\quotep{\procn{x}} \nameeq x$; so, the
notation is consistent with previous definitions.

Further, because names have structure it is possible to effect
substitutions on the basis of that structure. This means we need to
upgrade our notation for substitutions, which we accomplish by
adapting comprehension notation. Thus,

\begin{mathpar}
  P\{ y / x : x \in S \}
\end{mathpar}

is interpreted to mean the process derived from P by replacing (in a
capture-avoiding manner) each occurrence of $x$ in $S$ by $y$. For example,

\begin{mathpar}
  P\{ \quotep{\procn{x}|\procn{x}} / x : x \in \freenames{P} \}
\end{mathpar}

will replace each (occurrence) of a free name $x$ in $P$ by
$\quotep{\procn{x}|\procn{x}}$.

Also, we will avail ourselves of the notation $x^{L}$ and $x^{R}$ to
denote injections of a name into disjoint copies of the name
space. There are numerous ways to accomplish this. One example can be
found in \cite{MeredithR05}. This notation overloads to vectors of
names: $\vec{x}^{\pi} := (x_{i}^{\pi} \; : \; 0 \leq i < |\vec{x}| )$ where $\pi \in \{L,R\}$.

We also use $P^{\Box} := P|\Box$.

In \cite{MeredithR05} an interpretation of the new operator is
given. It turns out that there are several possible interpretations
all enjoying the requisite algebraic properties of the operator (see
\cite{milner91polyadicpi}). We will therefore make liberal use of
$(\nu\; \vec{x})P$.

% subsection the_syntax_and_semantics_of_the_notation_system (end)   

\input{qm2pi.qmops} 

\input{qm2pi.sterngerlach} 

\input{qm2pi.metric} 

% section concurrent_process_calculi (end)

%\input{qm2pi.proofsketch}

% section proof sketch (end)

%\input{qm2pi.slviaknots} 

% section spatial logic via knots (end)

\input{qm2pi.conclusion}

% section conclusion (end)

%\input{qm2pi.dtcodes} 

% section wiring algorithm (end)

\input{qm2pi.ack} 

% section acknowledgments (end)

\newpage


\bibliographystyle{plain}   
\bibliography{../../biblios/main.bib}

\input{qm2pi.rhodetails}

\end{document}

 

\documentclass[12pt]{llncs}
%\documentclass{jktr}

\usepackage[pdftex]{hyperref}                   
\usepackage {listings}
\usepackage {mathpartir}
\usepackage{bcprules}
%\usepackage{listings}
                       
\usepackage{graphicx} 
%\usepackage[margins=2.5cm,nohead,nofoot]{geometry}
%\usepackage{geometry}
\usepackage{amsfonts}
\usepackage{amstext}
\usepackage{latexsym}
\usepackage{amssymb}
\usepackage{color}


%\include{myPreamble}
\include{qm2pi.local} 

%\ifpdf
%\usepackage[pdftex]{graphicx}
%\else
%\usepackage{graphicx}
%\fi

 % \ifpdf
%  \usepackage{pdfsync}
%  \if


%\title{Brief Article}
%\author{David F. Snyder}
%\author{L.G. Meredith}

%\address{Dept. of Math., Texas State University--San Marcos, San Marcos, TX 78666}
       
\pagestyle{empty}


\begin{document}

\lstset{language=[Objective]Caml,frame=shadowbox}

\input{qm2pi.front}

% section front matter (end)

\input{qm2pi.intro} 
 
% section introduction (end)

% \input{qm2pi.knotations} 

% section notation (end)

\input{qm2pi.process.calculi} 

% section concurrent_process_calculi_and_spatial_logics_ (end)
    
%\input{qm2pi.knots2pi} 

%\input{qm2pi.trefoil} 

%\input{qm2pi.mainthm} 

% subsection basic_interpretation (end)

%\input{qm2pi.rho.presentation} 
\subsection{The syntax and semantics of the notation system}\label{sub:the_syntax_and_semantics_of_the_notation_system} % (fold)

We now summarize a technical presentation of the calculus that
embodies our theory of dynamics. The typical presentation of such a
calculus follows the style of giving generators and relations on
them. The grammar, below, describing term constructors, freely
generates the set of processes, $\Proc$. This set is then quotiented
by a relation known as structural congruence and it is over this set
that the notion of dynamics is expressed. This presentation is
essentially that of \cite{MeredithR05} with the addition of
polyadicity and summation. For readability we have relegated some of
the technical subtleties to an appendix.

\subsubsection{Process grammar}\label{subsub:process_grammar}

\begin{mathpar}
  \inferrule* [lab=synchronization] {} {{M} \bc \pzero \;|\; x?F \;|\; x!C }
  \and
  \inferrule* [lab=abstraction] {} {{F} \bc (x)P}
  \and
  \inferrule* [lab=concretion] {} {{C} \bc \langle Q \rangle}
  \and
  \inferrule* [lab=process] {} {{P,Q} \bc M \;| \;P|Q \;|\; @{x}}
  \and
  \inferrule* [lab=name] {} {{x} \bc \quotep{P}}
\end{mathpar} 

Note that $\vec{x}$ (resp. $\vec{P}$) denotes a vector of names
(resp. processes) of length $|\vec{x}|$ (resp. $|\vec{P}|$). We adopt
the following useful abbreviations.

\begin{mathpar}
   x?(\vec{y}).P := x.(\vec{y})P \and  x\clift{\vec{P}} := x.\clift{\vec{P}}
   \and x!(y) := \lift{x}{\dropn{y}}
   \and \Pi_{i=0}^{n-1}P_i := P_0 | \ldots | P_{n-1}
\end{mathpar}

\subsubsection{Structural congruence}

\paragraph{Free and bound names and alpha-equivalence.} At the
core of structural equivalence is alpha-equivalence which identifies
process that are the same up to a change of variable. Formally, we
recognize the distinction between free and bound names. The free names
of a process, $\freenames{P}$, may be calculated recursively as
follows:

\begin{mathpar}
\freenames{\pzero} := \emptyset
  \and \\
  \freenames{x?(y).P} := \{ x \} \cup (\freenames{P} \setminus \{ y \})
  \and 
  \freenames{x!\langle P \rangle} := \{ x \} \cup \{ P \} 
  \and \\
  \freenames{P|Q} := \freenames{P} \cup \freenames{Q}
  \and \\
  \freenames{@{x}} := \{ x \}
\end{mathpar}

$\pi$
$\quotep{\pi}$

$\freenames{-} : \pi \to \mathcal{P}(\quotep{\pi})$

\begin{eqnarray*}
  \freenames{\pzero} & := & \emptyset \\
  \freenames{x?(y).P} & := & \{ x \} \cup (\freenames{P} \setminus \{ y \}) \\
  \freenames{x!\langle P \rangle} & := & \{ x \} \cup \{ P \} \\
  \freenames{P|Q} & := & \freenames{P} \cup \freenames{Q} \\
  \freenames{\dropn{x}} & := & \{ x \}
\end{eqnarray*}

The bound names of a process, $\boundnames{P}$, are those names occurring in $P$
that are not free. For example, in $x?(y).0$, the name $x$ is free, while $y$ is bound.

\begin{mathpar}
  \inferrule* [lab=monoidal-laws] {} { P|Q \equiv Q|P \and P|0 \equiv P \and P|(Q|R) \equiv (P|Q)|R }
\end{mathpar}

\begin{mathpar}
  \inferrule* [lab=alpha-equivalence] {} { (x)P \equiv (y)P\{y/x\} \and y \not\in \freenames{P} }
\end{mathpar}

\begin{definition}
Then two processes, $P,Q$, are alpha-equivalent if $P = Q\{\vec{y}/\vec{x}\}$ for
some $\vec{x} \in \boundnames{Q},\vec{y} \in \boundnames{P}$, where $Q\{\vec{y}/\vec{x}\}$
denotes the capture-avoiding substitution of $\vec{y}$ for $\vec{x}$ in $Q$.
\end{definition}

\begin{definition}
  The {\em structural congruence} \cite{SangiorgiWalker} , $\equiv$,
  between processes is the least congruence containing
  alpha-equivalence, satisfying the abelian monoid laws
  (associativity, commutativity and $\pzero$ as identity) for parallel
  composition $|$ and for summation $+$.
\end{definition}

\subsection{Name equivalence}

We take name equivalence, written $\nameeq$, to be the smallest
equivalence relation generated by the following rules.

\begin{mathpar}
\inferrule*[lab=Quote-drop]
{ }
{ \quotep{@{x}} \nameeq x }

\inferrule*[lab=Struct-equiv]
{ P \scong Q }
{ \quotep{P} \nameeq \quotep{Q} }
\end{mathpar}

The astute reader will have noticed that the mutual recursion of names
and processes imposes a mutual recursion on alpha-equivalence and
structural equivalence via name-equivalence. Fortunately, all of this
works out pleasantly and we may calculate in the natural way, free of
concern. The reader interested in the details is referred to the
appendix \ref{appendix:rho_details}.

\subsection{Substitution}

We use $\Proc$ for the set of processes, $\QProc$ for the set of
names, and $\id{\{}\vec{y} / \vec{x} \id{\}}$ to denote partial maps,
$s : \QProc \rightarrow \QProc$. A map, $s$ lifts, uniquely, to a map
on process terms, $\widehat{s} : \Proc \rightarrow \Proc$ by the
following equations.

\begin{mathpar}
  (0) \psubstp{Q}{P} := 0 \\
  (R \juxtap S) \psubstp{Q}{P}
  :=    
  (R)\psubstp{Q}{P} \juxtap (S) \psubstp{Q}{P} \\
  (x?(y).R) \psubstp{Q}{P}    
  :=    
  (x)\substp{Q}{P} (z)\concat( (R \psubstn{z}{y}) \psubstp{Q}{P} ) \\
  (\lift{x}{R}) \psubstp{Q}{P}  
  :=
  \lift{(x)\substp{Q}{P}}{ R \psubstp{Q}{P} } \\
%   (\dropn{x})  \psubstp{Q}{P}       
%   := 
%   \left\{ 
%     \begin{array}{ccc} 
%       \dropn{\quotep{Q}} & & x \nameeq \quotep{P} \\
%       \dropn{x} & & otherwise \\
%     \end{array}
%   \right. 
  (\dropn{x})  \psubstp{Q}{P}       
  := 
  \left\{ 
    \begin{array}{ccc} 
      Q & & x \nameeq \quotep{P} \\
      \dropn{x} & & otherwise \\
    \end{array}
  \right.
\end{mathpar}
 

where

\begin{eqnarray}
  (x)\id{\{} \lpquote Q \rpquote / \lpquote P \rpquote \id{\}}            = 
  \left\{ 
    \begin{array}{ccc}
      \lpquote Q \rpquote & & x \nameeq \lpquote P \rpquote \\
      x & & otherwise \\
    \end{array}
  \right. \nonumber
\end{eqnarray}

and $z$ is chosen distinct from $\quotep{P}$, $\quotep{Q}$, the free
names in $Q$, and all the names in $R$. Our $\alpha$-equivalence will
be built in the standard way from this substitution.

\begin{remark}\label{rem:no_self_referential_names}
  One consequence of these definitions is that $\forall P. \quotep{P}
  \not\in \freenames{P}$.
\end{remark}

\subsection{ Dynamic quote: an example }

Anticipating something of what's to come, consider applying the
substitution, $\widehat{\id{\{}u / z \id{\}}}$, to the following pair
of processes, $\lift{w}{y!(z)}$ and $w[ \lpquote y!(z) \rpquote ]$.

\begin{eqnarray}
	\lift{w}{y!(z)}\widehat{\id{\{}u / z \id{\}}}
		& = &
		\lift{w}{y!(u)} \nonumber\\
	w[ \lpquote y!(z) \rpquote ] \widehat{ \id{\{}u / z \id{\}} }
		& = &
		w[ \lpquote y!(z) \rpquote ] \nonumber
\end{eqnarray}

Because the body of the process between quotes is impervious to
substitution, we get radically different answers. In fact, by
examining the first process in an input context,
e.g. $x?(z).\lift{w}{y!(z)}$, we see that the process under the lift
operator may be shaped by prefixed inputs binding a name inside it. In
this sense, the lift operator will be seen as a way to dynamically
construct processes before reifying them as names.

Finally equipped with these standard features we can present the
dynamics of the calculus.

\subsubsection{Operational semantics} 

Finally, we introduce the computational dynamics. What marks these
algebras as distinct from other more traditionally studied algebraic
structures, e.g. vector spaces or polynomial rings, is the manner in
which dynamics is captured. In traditional structures, dynamics is typically
expressed through morphisms between such structures, as in linear maps
between vector spaces or morphisms between rings. In algebras
associated with the semantics of computation, the dynamics is
expressed as part of the algebraic structure itself, through a
reduction reduction relation typically denoted by $\red$. Below, we
give a recursive presentation of this relation for the calculus used
in the encoding.

$\red \subseteq \pi \times \pi$
$\red : \pi \to \mathcal{P}(\pi)$

\begin{mathpar}
  \inferrule* [lab=Comm] { \textsf{match}( x_{src}, x_{trgt} ) } { x_{trgt}?(y)P \; | \; x_{src}!\langle {Q} \rangle \red P\{\quotep{Q}/y}\} }
  \and \\
  \inferrule* [lab=Par] {{P} \red {P}'} {{{P} | {Q}} \red {{P}' | {Q}}}
  \and
  \inferrule* [lab=Equiv]{{{P} \scong {P}'} \andalso {{P}' \red {Q}'} \andalso {{Q}' \scong {Q}}}{{P} \red {Q}}
\end{mathpar}

\begin{eqnarray*}
  match_{\equiv} (\quotep{P},\quotep{Q}) & := & P \equiv Q \\
  match_{\dagger}(\quotep{P},\quotep{Q}) & := & \forall R. P|Q \red^{*} R => R \red^{*} 0 \\
  match_{K}(\quotep{P},\quotep{Q}) & := & K \mbox{ for some context } K
\end{eqnarray*}

$u?(x)P | u!\langle Q \rangle \red P\{\quotep{Q}/x\}$

%We write $\wred$ for $\red^*$, and $P\red$ if $\exists Q $ such that $ P \red Q$.
We write $P\red$ if $\exists Q $ such that $ P \red Q$ and $P\not\red$, otherwise.

\section{Replication}

As mentioned before, it is known that replication (and hence
recursion) can be implemented in a higher-order process algebra
\cite{SangiorgiWalker}. As our first example of calculation with the
machinery thus far presented we give the construction explicitly in
the {\rhoc}.

\begin{eqnarray}
	D_{x} & := & \prefix{x}{y}{(\binpar{\outputp{x}{y}}{@{y}})} \nonumber\\
	\bangp_{x}{P} & := & \binpar{{x}!\langle{\binpar{D_{x}}{P}}\rangle}{D_{x}} \nonumber
\end{eqnarray}

\begin{eqnarray}
	\bangp_{x}{P} & & \nonumber\\
	=
	& {x}!\langle{(\prefix{x}{y}{(\outputp{x}{y} | @{y})) | P}}\rangle 
	      | \prefix{x}{y}{(\outputp{x}{y} | @{y})} & \nonumber\\
	\red
	& (\outputp{x}{y} | @{y})\substn{\quotep{(\prefix{x}{y}{(@{y} | \outputp{x}{y})) | P}}}{y} & \nonumber\\
	=
	& \outputp{x}{\quotep{(\prefix{x}{y}{(\outputp{x}{y} | @{y})) | P}}}
	  | {(\prefix{x}{y}{(\outputp{x}{y} | @{y})) | P}} & \nonumber\\
	\red
	& \ldots & \nonumber\\
	\red^*
	& P | P | \ldots & \nonumber
\end{eqnarray}

Of course, this encoding, as an implementation, runs away, unfolding
$\bangp{P}$ eagerly. A lazier and more implementable replication
operator, restricted to input-guarded processes, may be obtained as follows.

\begin{eqnarray}
\bangp{\prefix{u}{v}{P}} 
	:= 
	\binpar{\lift{x}{\prefix{u}{v}{(\binpar{D(x)}{P})}}}{D(x)} \nonumber
\end{eqnarray}

\begin{remark}
  Note that the lazier definition still does not deal with summation
  or mixed summation (i.e. sums over input and output). The reader is
  invited to construct definitions of replication that deal with these
  features. 

  Further, the definitions are parameterized in a name, $x$. Can you,
  gentle reader, make a definition that eliminates this parameter and
  guarantees no accidental interaction between the replication
  machinery and the process being replicated -- i.e. no accidental
  sharing of names used by the process to get its work done and the
  name(s) used by the replication to effect copying. This latter
  revision of the definition of replication is crucial to obtaining
  the expected identity $!!P \sim !P$.
\end{remark}

\begin{remark}\label{rem:paradoxical_combinator}
  The reader familiar with the lambda calculus will have noticed the
  similarity between $D$ and the paradoxical combinator.

  [Ed. note: the existence of this seems to suggest we have to be more
  restrictive on the set of processes and names we admit if we are to
  support no-cloning.]
\end{remark}

\subsubsection{Bisimulation}

The computational dynamics gives rise to another kind of equivalence,
the equivalence of computational behavior. As previously mentioned
this is typically captured \emph{via} some form of bisimulation.

% The notion we use in this paper is weak barbed bisimulation
% \cite{milner91polyadicpi}.

The notion we use in this paper is derived from weak barbed
bisimulation \cite{milner91polyadicpi}. 

\begin{definition}
An \emph{observation relation}, $\downarrow_{\mathcal N}$, over a set
of names, $\mathcal N$, is the smallest relation satisfying the rules
below.

\infrule[Out-barb]{y \in {\mathcal N}, \; x \nameeq y}
		  {\outputp{x}{v} \downarrow_{\mathcal N} x}
\infrule[Par-barb]{\mbox{$P\downarrow_{\mathcal N} x$ or $Q\downarrow_{\mathcal N} x$}}
		  {\binpar{P}{Q} \downarrow_{\mathcal N} x}

We write $P \Downarrow_{\mathcal N} x$ if there is $Q$ such that 
$P \wred Q$ and $Q \downarrow_{\mathcal N} x$.
\end{definition}

\begin{definition}
%\label{def.bbisim}
An  ${\mathcal N}$-\emph{barbed bisimulation} over a set of names, ${\mathcal N}$, is a symmetric binary relation 
${\mathcal S}_{\mathcal N}$ between agents such that $P\rel{S}_{\mathcal N}Q$ implies:
\begin{enumerate}
\item If $P \red P'$ then $Q \wred Q'$ and $P'\rel{S}_{\mathcal N} Q'$.
\item If $P\downarrow_{\mathcal N} x$, then $Q\Downarrow_{\mathcal N} x$.
\end{enumerate}
$P$ is ${\mathcal N}$-barbed bisimilar to $Q$, written
$P \wbbisim_{\mathcal N} Q$, if $P \rel{S}_{\mathcal N} Q$ for some ${\mathcal N}$-barbed bisimulation ${\mathcal S}_{\mathcal N}$.
\end{definition}

$\mathcal{R} \subseteq \pi \times \pi$

$P \mathcal{R} Q => \forall P'. P \red P' \Rightarrow \exists Q'. Q \red Q', P' \mathcal{R} Q'$

$P \vdash x \Rightarrow Q \vdash x$

\begin{mathpar}
  \inferrule*[lab=Out-barb]{x \nameeq y}{{y}!\langle{Q}\rangle \vdash x}
  \and
  \inferrule*[lab=Par-barb]{\mbox{$P\vdash x$ or $Q\vdash x$}}{\binpar{P}{Q} \vdash x}
\end{mathpar}

\subsubsection{Contexts}

One of the principle advantages of computational calculi like the
$\pi$-calculus is a well-defined notion of context,
contextual-equivalence and a correlation between
contextual-equivalence and notions of bisimulation. The notion of
context allows the decomposition of a process into (sub-)process and
its syntactic environment, its context. Thus, a context may be
thought of as a process with a ``hole'' (written $\Box$) in it. The
application of a context $M$ to a process $P$, written $M[P]$, is
tantamount to filling the hole in $M$ with $P$. In this paper we do
not need the full weight of this theory, but do make use of the notion
of context in the proof the main theorem. 

\begin{mathpar}
  \inferrule* [lab=summation] {} {{M_{M},M_{N}} \bc \Box \;|\; x.M_{A} \;|\; M_{M}+M_{N}}
  \and
  \inferrule* [lab=agent] {} {{M_{A}} \bc (\vec{x})M_{P} \;| \; \clift{P_0,\ldots,M_{P},\ldots,P_N}}
  \and \\
  \inferrule* [lab=process] {} {{M_{P}} \bc M_{N} \;| \;P|M_{P} }
\end{mathpar} 

\begin{mathpar}
  \inferrule* [lab=sychronization] {} {M_{N} \bc \Box \;|\; x?M_{F} \;|\; x!M_{C}}
  \and
  \inferrule* [lab=abstraction] {} {{M_{F}} \bc (x)M_{P} }
  \and
  \inferrule* [lab=concretion] {} {{M_{C}} \bc \langle M_{P} \rangle }
  \and \\
  \inferrule* [lab=process] {} {{M_{P}} \bc M_{N} \;| \;P|M_{P} }
\end{mathpar}

\begin{definition}[contextual application] Given a context $M$, and
  process $P$, we define the \emph{contextual application}, $M[P] :=
  M\{P/\Box\}$. That is, the contextual application of M to P is the
  substitution of $P$ for $\Box$ in $M$.
\end{definition}

$\meaningof{-} : L \to \mathcal{P}(\pi)$

\begin{mathpar}
  \inferrule* [lab=collection] {} {\meaningof{true} = \pi, \and \meaningof{~E} = \pi \setminus \meaningof{E}, \and \meaningof{E_{1} \& E_{2}} = \meaningof{E_{1}} \cap \meaningof{E_{2}}}
\end{mathpar}

\begin{mathpar}
  \inferrule* [lab=structure] {} {\meaningof{0} = \{ P \in \pi | P \equiv 0 \}, \and \\ \meaningof{E_1 | E_2} = \{ P \in \pi | P \equiv P_{1} | P_{2}, P_{1} \in \meaningof{E_{1}}, P_{2} \in \meaningof{E_2}\} }
\end{mathpar}

\begin{mathpar}
 \inferrule* [lab=behavior] {} {\meaningof{\langle a?b \rangle E} = \{ P \in \pi | P \equiv Q | u?(y)P', \\ \and \\\\ \and \\ \;\;\; u \in \meaningof{a}, \forall z.P'\{z/y\} \in \meaningof{E\{z/b\}}\}, \and \\ \meaningof{a!E} = \{ P \in \pi | P \equiv Q | x!\langle P' \rangle, x \in \meaningof{a} P' \in \meaningof{E}\} }
\end{mathpar}

\begin{mathpar}
 \inferrule* [lab=nominal] {} {\meaningof{\quotep{E}} = \{ \quotep{P} \in \quotep{\pi} | P \in \meaningof{E} \}, \and \meaningof{\quotep{P}} = \{ \quotep{Q} \in \quotep{\pi} | P \equiv Q \} \and \\ \meaningof{@\quotep{E}} = \{ P \in \pi | P \equiv @x, x \in \meaningof{E} \}}
\end{mathpar}

\begin{eqnarray*}
  \\
  \meaningof{-} : TS \to ST
\end{eqnarray*}

\begin{eqnarray*}
  \\
  L : TS \to ST
\end{eqnarray*}

\begin{eqnarray*}
  \\
  P \models E \iff P \in \meaningof{E}
\end{eqnarray*}

\begin{eqnarray*}
  P \approx_{L} Q \iff \forall E \in L. P \models E \iff Q \models E
\end{eqnarray*}

\begin{eqnarray*}
  P \approx_{K} Q
\end{eqnarray*}

\begin{eqnarray*}
  P \approx Q
\end{eqnarray*}

$\approx_{K} = \approx = \approx_{L}$

\subsubsection{Contextual duality}

Note that contexts extend the quotation operation to a family of
operations from processes to names. Given a context, $M$, we can
define a \emph{nominal context}, $\quotep{M}$ by $\quotep{M}[P] :=
\quotep{M[P]}$. To foreshadow what is to come we observe that these
operations enjoy a duality with processes very much like the duality
between vectors and maps from vectors to scalars.

Further, because the calculus is essentially higher-order, we have a
correspondence between contexts and processes. More specifically,
given a name $x$ and a context $M$ we can construct $M^{*}_{x}$ such
that 

\begin{mathpar}
  M^{*}_{x} | \lift{x}{P} \red M[P]
\end{mathpar}

namely,

\begin{mathpar}
  M^{*}_{x} := x?(u).M[\dropn{u}]
\end{mathpar}

The dependence of $M^{*}_{x}$ on a name makes it an abstraction, 

\begin{mathpar}
  M^{*} := (x)x?(u).M[\dropn{u}]
\end{mathpar}

\subsection{Additional notation}

It will sometimes be convenient to denote the process a name
quotes. We already have the notation $x = \quotep{P}$, but it will be
convenient to introduce an alternate notation, $\procn{x}$, when we
want to emphasize the connection to the use of the name. Note that, by
virtue of name equivalence, $\quotep{\procn{x}} \nameeq x$; so, the
notation is consistent with previous definitions.

Further, because names have structure it is possible to effect
substitutions on the basis of that structure. This means we need to
upgrade our notation for substitutions, which we accomplish by
adapting comprehension notation. Thus,

\begin{mathpar}
  P\{ y / x : x \in S \}
\end{mathpar}

is interpreted to mean the process derived from P by replacing (in a
capture-avoiding manner) each occurrence of $x$ in $S$ by $y$. For example,

\begin{mathpar}
  P\{ \quotep{\procn{x}|\procn{x}} / x : x \in \freenames{P} \}
\end{mathpar}

will replace each (occurrence) of a free name $x$ in $P$ by
$\quotep{\procn{x}|\procn{x}}$.

Also, we will avail ourselves of the notation $x^{L}$ and $x^{R}$ to
denote injections of a name into disjoint copies of the name
space. There are numerous ways to accomplish this. One example can be
found in \cite{MeredithR05}. This notation overloads to vectors of
names: $\vec{x}^{\pi} := (x_{i}^{\pi} \; : \; 0 \leq i < |\vec{x}| )$ where $\pi \in \{L,R\}$.

We also use $P^{\Box} := P|\Box$.

In \cite{MeredithR05} an interpretation of the new operator is
given. It turns out that there are several possible interpretations
all enjoying the requisite algebraic properties of the operator (see
\cite{milner91polyadicpi}). We will therefore make liberal use of
$(\nu\; \vec{x})P$.

% subsection the_syntax_and_semantics_of_the_notation_system (end)   

\input{qm2pi.qmops} 

\input{qm2pi.sterngerlach} 

\input{qm2pi.metric} 

% section concurrent_process_calculi (end)

%\input{qm2pi.proofsketch}

% section proof sketch (end)

%\input{qm2pi.slviaknots} 

% section spatial logic via knots (end)

\input{qm2pi.conclusion}

% section conclusion (end)

%\input{qm2pi.dtcodes} 

% section wiring algorithm (end)

\input{qm2pi.ack} 

% section acknowledgments (end)

\newpage


\bibliographystyle{plain}   
\bibliography{../../biblios/main.bib}

\input{qm2pi.rhodetails}

\end{document}

 

% section concurrent_process_calculi (end)

%\documentclass[12pt]{llncs}
%\documentclass{jktr}

\usepackage[pdftex]{hyperref}                   
\usepackage {listings}
\usepackage {mathpartir}
\usepackage{bcprules}
%\usepackage{listings}
                       
\usepackage{graphicx} 
%\usepackage[margins=2.5cm,nohead,nofoot]{geometry}
%\usepackage{geometry}
\usepackage{amsfonts}
\usepackage{amstext}
\usepackage{latexsym}
\usepackage{amssymb}
\usepackage{color}


%\include{myPreamble}
\include{qm2pi.local} 

%\ifpdf
%\usepackage[pdftex]{graphicx}
%\else
%\usepackage{graphicx}
%\fi

 % \ifpdf
%  \usepackage{pdfsync}
%  \if


%\title{Brief Article}
%\author{David F. Snyder}
%\author{L.G. Meredith}

%\address{Dept. of Math., Texas State University--San Marcos, San Marcos, TX 78666}
       
\pagestyle{empty}


\begin{document}

\lstset{language=[Objective]Caml,frame=shadowbox}

\input{qm2pi.front}

% section front matter (end)

\input{qm2pi.intro} 
 
% section introduction (end)

% \input{qm2pi.knotations} 

% section notation (end)

\input{qm2pi.process.calculi} 

% section concurrent_process_calculi_and_spatial_logics_ (end)
    
%\input{qm2pi.knots2pi} 

%\input{qm2pi.trefoil} 

%\input{qm2pi.mainthm} 

% subsection basic_interpretation (end)

%\input{qm2pi.rho.presentation} 
\subsection{The syntax and semantics of the notation system}\label{sub:the_syntax_and_semantics_of_the_notation_system} % (fold)

We now summarize a technical presentation of the calculus that
embodies our theory of dynamics. The typical presentation of such a
calculus follows the style of giving generators and relations on
them. The grammar, below, describing term constructors, freely
generates the set of processes, $\Proc$. This set is then quotiented
by a relation known as structural congruence and it is over this set
that the notion of dynamics is expressed. This presentation is
essentially that of \cite{MeredithR05} with the addition of
polyadicity and summation. For readability we have relegated some of
the technical subtleties to an appendix.

\subsubsection{Process grammar}\label{subsub:process_grammar}

\begin{mathpar}
  \inferrule* [lab=synchronization] {} {{M} \bc \pzero \;|\; x?F \;|\; x!C }
  \and
  \inferrule* [lab=abstraction] {} {{F} \bc (x)P}
  \and
  \inferrule* [lab=concretion] {} {{C} \bc \langle Q \rangle}
  \and
  \inferrule* [lab=process] {} {{P,Q} \bc M \;| \;P|Q \;|\; @{x}}
  \and
  \inferrule* [lab=name] {} {{x} \bc \quotep{P}}
\end{mathpar} 

Note that $\vec{x}$ (resp. $\vec{P}$) denotes a vector of names
(resp. processes) of length $|\vec{x}|$ (resp. $|\vec{P}|$). We adopt
the following useful abbreviations.

\begin{mathpar}
   x?(\vec{y}).P := x.(\vec{y})P \and  x\clift{\vec{P}} := x.\clift{\vec{P}}
   \and x!(y) := \lift{x}{\dropn{y}}
   \and \Pi_{i=0}^{n-1}P_i := P_0 | \ldots | P_{n-1}
\end{mathpar}

\subsubsection{Structural congruence}

\paragraph{Free and bound names and alpha-equivalence.} At the
core of structural equivalence is alpha-equivalence which identifies
process that are the same up to a change of variable. Formally, we
recognize the distinction between free and bound names. The free names
of a process, $\freenames{P}$, may be calculated recursively as
follows:

\begin{mathpar}
\freenames{\pzero} := \emptyset
  \and \\
  \freenames{x?(y).P} := \{ x \} \cup (\freenames{P} \setminus \{ y \})
  \and 
  \freenames{x!\langle P \rangle} := \{ x \} \cup \{ P \} 
  \and \\
  \freenames{P|Q} := \freenames{P} \cup \freenames{Q}
  \and \\
  \freenames{@{x}} := \{ x \}
\end{mathpar}

$\pi$
$\quotep{\pi}$

$\freenames{-} : \pi \to \mathcal{P}(\quotep{\pi})$

\begin{eqnarray*}
  \freenames{\pzero} & := & \emptyset \\
  \freenames{x?(y).P} & := & \{ x \} \cup (\freenames{P} \setminus \{ y \}) \\
  \freenames{x!\langle P \rangle} & := & \{ x \} \cup \{ P \} \\
  \freenames{P|Q} & := & \freenames{P} \cup \freenames{Q} \\
  \freenames{\dropn{x}} & := & \{ x \}
\end{eqnarray*}

The bound names of a process, $\boundnames{P}$, are those names occurring in $P$
that are not free. For example, in $x?(y).0$, the name $x$ is free, while $y$ is bound.

\begin{mathpar}
  \inferrule* [lab=monoidal-laws] {} { P|Q \equiv Q|P \and P|0 \equiv P \and P|(Q|R) \equiv (P|Q)|R }
\end{mathpar}

\begin{mathpar}
  \inferrule* [lab=alpha-equivalence] {} { (x)P \equiv (y)P\{y/x\} \and y \not\in \freenames{P} }
\end{mathpar}

\begin{definition}
Then two processes, $P,Q$, are alpha-equivalent if $P = Q\{\vec{y}/\vec{x}\}$ for
some $\vec{x} \in \boundnames{Q},\vec{y} \in \boundnames{P}$, where $Q\{\vec{y}/\vec{x}\}$
denotes the capture-avoiding substitution of $\vec{y}$ for $\vec{x}$ in $Q$.
\end{definition}

\begin{definition}
  The {\em structural congruence} \cite{SangiorgiWalker} , $\equiv$,
  between processes is the least congruence containing
  alpha-equivalence, satisfying the abelian monoid laws
  (associativity, commutativity and $\pzero$ as identity) for parallel
  composition $|$ and for summation $+$.
\end{definition}

\subsection{Name equivalence}

We take name equivalence, written $\nameeq$, to be the smallest
equivalence relation generated by the following rules.

\begin{mathpar}
\inferrule*[lab=Quote-drop]
{ }
{ \quotep{@{x}} \nameeq x }

\inferrule*[lab=Struct-equiv]
{ P \scong Q }
{ \quotep{P} \nameeq \quotep{Q} }
\end{mathpar}

The astute reader will have noticed that the mutual recursion of names
and processes imposes a mutual recursion on alpha-equivalence and
structural equivalence via name-equivalence. Fortunately, all of this
works out pleasantly and we may calculate in the natural way, free of
concern. The reader interested in the details is referred to the
appendix \ref{appendix:rho_details}.

\subsection{Substitution}

We use $\Proc$ for the set of processes, $\QProc$ for the set of
names, and $\id{\{}\vec{y} / \vec{x} \id{\}}$ to denote partial maps,
$s : \QProc \rightarrow \QProc$. A map, $s$ lifts, uniquely, to a map
on process terms, $\widehat{s} : \Proc \rightarrow \Proc$ by the
following equations.

\begin{mathpar}
  (0) \psubstp{Q}{P} := 0 \\
  (R \juxtap S) \psubstp{Q}{P}
  :=    
  (R)\psubstp{Q}{P} \juxtap (S) \psubstp{Q}{P} \\
  (x?(y).R) \psubstp{Q}{P}    
  :=    
  (x)\substp{Q}{P} (z)\concat( (R \psubstn{z}{y}) \psubstp{Q}{P} ) \\
  (\lift{x}{R}) \psubstp{Q}{P}  
  :=
  \lift{(x)\substp{Q}{P}}{ R \psubstp{Q}{P} } \\
%   (\dropn{x})  \psubstp{Q}{P}       
%   := 
%   \left\{ 
%     \begin{array}{ccc} 
%       \dropn{\quotep{Q}} & & x \nameeq \quotep{P} \\
%       \dropn{x} & & otherwise \\
%     \end{array}
%   \right. 
  (\dropn{x})  \psubstp{Q}{P}       
  := 
  \left\{ 
    \begin{array}{ccc} 
      Q & & x \nameeq \quotep{P} \\
      \dropn{x} & & otherwise \\
    \end{array}
  \right.
\end{mathpar}
 

where

\begin{eqnarray}
  (x)\id{\{} \lpquote Q \rpquote / \lpquote P \rpquote \id{\}}            = 
  \left\{ 
    \begin{array}{ccc}
      \lpquote Q \rpquote & & x \nameeq \lpquote P \rpquote \\
      x & & otherwise \\
    \end{array}
  \right. \nonumber
\end{eqnarray}

and $z$ is chosen distinct from $\quotep{P}$, $\quotep{Q}$, the free
names in $Q$, and all the names in $R$. Our $\alpha$-equivalence will
be built in the standard way from this substitution.

\begin{remark}\label{rem:no_self_referential_names}
  One consequence of these definitions is that $\forall P. \quotep{P}
  \not\in \freenames{P}$.
\end{remark}

\subsection{ Dynamic quote: an example }

Anticipating something of what's to come, consider applying the
substitution, $\widehat{\id{\{}u / z \id{\}}}$, to the following pair
of processes, $\lift{w}{y!(z)}$ and $w[ \lpquote y!(z) \rpquote ]$.

\begin{eqnarray}
	\lift{w}{y!(z)}\widehat{\id{\{}u / z \id{\}}}
		& = &
		\lift{w}{y!(u)} \nonumber\\
	w[ \lpquote y!(z) \rpquote ] \widehat{ \id{\{}u / z \id{\}} }
		& = &
		w[ \lpquote y!(z) \rpquote ] \nonumber
\end{eqnarray}

Because the body of the process between quotes is impervious to
substitution, we get radically different answers. In fact, by
examining the first process in an input context,
e.g. $x?(z).\lift{w}{y!(z)}$, we see that the process under the lift
operator may be shaped by prefixed inputs binding a name inside it. In
this sense, the lift operator will be seen as a way to dynamically
construct processes before reifying them as names.

Finally equipped with these standard features we can present the
dynamics of the calculus.

\subsubsection{Operational semantics} 

Finally, we introduce the computational dynamics. What marks these
algebras as distinct from other more traditionally studied algebraic
structures, e.g. vector spaces or polynomial rings, is the manner in
which dynamics is captured. In traditional structures, dynamics is typically
expressed through morphisms between such structures, as in linear maps
between vector spaces or morphisms between rings. In algebras
associated with the semantics of computation, the dynamics is
expressed as part of the algebraic structure itself, through a
reduction reduction relation typically denoted by $\red$. Below, we
give a recursive presentation of this relation for the calculus used
in the encoding.

$\red \subseteq \pi \times \pi$
$\red : \pi \to \mathcal{P}(\pi)$

\begin{mathpar}
  \inferrule* [lab=Comm] { \textsf{match}( x_{src}, x_{trgt} ) } { x_{trgt}?(y)P \; | \; x_{src}!\langle {Q} \rangle \red P\{\quotep{Q}/y}\} }
  \and \\
  \inferrule* [lab=Par] {{P} \red {P}'} {{{P} | {Q}} \red {{P}' | {Q}}}
  \and
  \inferrule* [lab=Equiv]{{{P} \scong {P}'} \andalso {{P}' \red {Q}'} \andalso {{Q}' \scong {Q}}}{{P} \red {Q}}
\end{mathpar}

\begin{eqnarray*}
  match_{\equiv} (\quotep{P},\quotep{Q}) & := & P \equiv Q \\
  match_{\dagger}(\quotep{P},\quotep{Q}) & := & \forall R. P|Q \red^{*} R => R \red^{*} 0 \\
  match_{K}(\quotep{P},\quotep{Q}) & := & K \mbox{ for some context } K
\end{eqnarray*}

$u?(x)P | u!\langle Q \rangle \red P\{\quotep{Q}/x\}$

%We write $\wred$ for $\red^*$, and $P\red$ if $\exists Q $ such that $ P \red Q$.
We write $P\red$ if $\exists Q $ such that $ P \red Q$ and $P\not\red$, otherwise.

\section{Replication}

As mentioned before, it is known that replication (and hence
recursion) can be implemented in a higher-order process algebra
\cite{SangiorgiWalker}. As our first example of calculation with the
machinery thus far presented we give the construction explicitly in
the {\rhoc}.

\begin{eqnarray}
	D_{x} & := & \prefix{x}{y}{(\binpar{\outputp{x}{y}}{@{y}})} \nonumber\\
	\bangp_{x}{P} & := & \binpar{{x}!\langle{\binpar{D_{x}}{P}}\rangle}{D_{x}} \nonumber
\end{eqnarray}

\begin{eqnarray}
	\bangp_{x}{P} & & \nonumber\\
	=
	& {x}!\langle{(\prefix{x}{y}{(\outputp{x}{y} | @{y})) | P}}\rangle 
	      | \prefix{x}{y}{(\outputp{x}{y} | @{y})} & \nonumber\\
	\red
	& (\outputp{x}{y} | @{y})\substn{\quotep{(\prefix{x}{y}{(@{y} | \outputp{x}{y})) | P}}}{y} & \nonumber\\
	=
	& \outputp{x}{\quotep{(\prefix{x}{y}{(\outputp{x}{y} | @{y})) | P}}}
	  | {(\prefix{x}{y}{(\outputp{x}{y} | @{y})) | P}} & \nonumber\\
	\red
	& \ldots & \nonumber\\
	\red^*
	& P | P | \ldots & \nonumber
\end{eqnarray}

Of course, this encoding, as an implementation, runs away, unfolding
$\bangp{P}$ eagerly. A lazier and more implementable replication
operator, restricted to input-guarded processes, may be obtained as follows.

\begin{eqnarray}
\bangp{\prefix{u}{v}{P}} 
	:= 
	\binpar{\lift{x}{\prefix{u}{v}{(\binpar{D(x)}{P})}}}{D(x)} \nonumber
\end{eqnarray}

\begin{remark}
  Note that the lazier definition still does not deal with summation
  or mixed summation (i.e. sums over input and output). The reader is
  invited to construct definitions of replication that deal with these
  features. 

  Further, the definitions are parameterized in a name, $x$. Can you,
  gentle reader, make a definition that eliminates this parameter and
  guarantees no accidental interaction between the replication
  machinery and the process being replicated -- i.e. no accidental
  sharing of names used by the process to get its work done and the
  name(s) used by the replication to effect copying. This latter
  revision of the definition of replication is crucial to obtaining
  the expected identity $!!P \sim !P$.
\end{remark}

\begin{remark}\label{rem:paradoxical_combinator}
  The reader familiar with the lambda calculus will have noticed the
  similarity between $D$ and the paradoxical combinator.

  [Ed. note: the existence of this seems to suggest we have to be more
  restrictive on the set of processes and names we admit if we are to
  support no-cloning.]
\end{remark}

\subsubsection{Bisimulation}

The computational dynamics gives rise to another kind of equivalence,
the equivalence of computational behavior. As previously mentioned
this is typically captured \emph{via} some form of bisimulation.

% The notion we use in this paper is weak barbed bisimulation
% \cite{milner91polyadicpi}.

The notion we use in this paper is derived from weak barbed
bisimulation \cite{milner91polyadicpi}. 

\begin{definition}
An \emph{observation relation}, $\downarrow_{\mathcal N}$, over a set
of names, $\mathcal N$, is the smallest relation satisfying the rules
below.

\infrule[Out-barb]{y \in {\mathcal N}, \; x \nameeq y}
		  {\outputp{x}{v} \downarrow_{\mathcal N} x}
\infrule[Par-barb]{\mbox{$P\downarrow_{\mathcal N} x$ or $Q\downarrow_{\mathcal N} x$}}
		  {\binpar{P}{Q} \downarrow_{\mathcal N} x}

We write $P \Downarrow_{\mathcal N} x$ if there is $Q$ such that 
$P \wred Q$ and $Q \downarrow_{\mathcal N} x$.
\end{definition}

\begin{definition}
%\label{def.bbisim}
An  ${\mathcal N}$-\emph{barbed bisimulation} over a set of names, ${\mathcal N}$, is a symmetric binary relation 
${\mathcal S}_{\mathcal N}$ between agents such that $P\rel{S}_{\mathcal N}Q$ implies:
\begin{enumerate}
\item If $P \red P'$ then $Q \wred Q'$ and $P'\rel{S}_{\mathcal N} Q'$.
\item If $P\downarrow_{\mathcal N} x$, then $Q\Downarrow_{\mathcal N} x$.
\end{enumerate}
$P$ is ${\mathcal N}$-barbed bisimilar to $Q$, written
$P \wbbisim_{\mathcal N} Q$, if $P \rel{S}_{\mathcal N} Q$ for some ${\mathcal N}$-barbed bisimulation ${\mathcal S}_{\mathcal N}$.
\end{definition}

$\mathcal{R} \subseteq \pi \times \pi$

$P \mathcal{R} Q => \forall P'. P \red P' \Rightarrow \exists Q'. Q \red Q', P' \mathcal{R} Q'$

$P \vdash x \Rightarrow Q \vdash x$

\begin{mathpar}
  \inferrule*[lab=Out-barb]{x \nameeq y}{{y}!\langle{Q}\rangle \vdash x}
  \and
  \inferrule*[lab=Par-barb]{\mbox{$P\vdash x$ or $Q\vdash x$}}{\binpar{P}{Q} \vdash x}
\end{mathpar}

\subsubsection{Contexts}

One of the principle advantages of computational calculi like the
$\pi$-calculus is a well-defined notion of context,
contextual-equivalence and a correlation between
contextual-equivalence and notions of bisimulation. The notion of
context allows the decomposition of a process into (sub-)process and
its syntactic environment, its context. Thus, a context may be
thought of as a process with a ``hole'' (written $\Box$) in it. The
application of a context $M$ to a process $P$, written $M[P]$, is
tantamount to filling the hole in $M$ with $P$. In this paper we do
not need the full weight of this theory, but do make use of the notion
of context in the proof the main theorem. 

\begin{mathpar}
  \inferrule* [lab=summation] {} {{M_{M},M_{N}} \bc \Box \;|\; x.M_{A} \;|\; M_{M}+M_{N}}
  \and
  \inferrule* [lab=agent] {} {{M_{A}} \bc (\vec{x})M_{P} \;| \; \clift{P_0,\ldots,M_{P},\ldots,P_N}}
  \and \\
  \inferrule* [lab=process] {} {{M_{P}} \bc M_{N} \;| \;P|M_{P} }
\end{mathpar} 

\begin{mathpar}
  \inferrule* [lab=sychronization] {} {M_{N} \bc \Box \;|\; x?M_{F} \;|\; x!M_{C}}
  \and
  \inferrule* [lab=abstraction] {} {{M_{F}} \bc (x)M_{P} }
  \and
  \inferrule* [lab=concretion] {} {{M_{C}} \bc \langle M_{P} \rangle }
  \and \\
  \inferrule* [lab=process] {} {{M_{P}} \bc M_{N} \;| \;P|M_{P} }
\end{mathpar}

\begin{definition}[contextual application] Given a context $M$, and
  process $P$, we define the \emph{contextual application}, $M[P] :=
  M\{P/\Box\}$. That is, the contextual application of M to P is the
  substitution of $P$ for $\Box$ in $M$.
\end{definition}

$\meaningof{-} : L \to \mathcal{P}(\pi)$

\begin{mathpar}
  \inferrule* [lab=collection] {} {\meaningof{true} = \pi, \and \meaningof{~E} = \pi \setminus \meaningof{E}, \and \meaningof{E_{1} \& E_{2}} = \meaningof{E_{1}} \cap \meaningof{E_{2}}}
\end{mathpar}

\begin{mathpar}
  \inferrule* [lab=structure] {} {\meaningof{0} = \{ P \in \pi | P \equiv 0 \}, \and \\ \meaningof{E_1 | E_2} = \{ P \in \pi | P \equiv P_{1} | P_{2}, P_{1} \in \meaningof{E_{1}}, P_{2} \in \meaningof{E_2}\} }
\end{mathpar}

\begin{mathpar}
 \inferrule* [lab=behavior] {} {\meaningof{\langle a?b \rangle E} = \{ P \in \pi | P \equiv Q | u?(y)P', \\ \and \\\\ \and \\ \;\;\; u \in \meaningof{a}, \forall z.P'\{z/y\} \in \meaningof{E\{z/b\}}\}, \and \\ \meaningof{a!E} = \{ P \in \pi | P \equiv Q | x!\langle P' \rangle, x \in \meaningof{a} P' \in \meaningof{E}\} }
\end{mathpar}

\begin{mathpar}
 \inferrule* [lab=nominal] {} {\meaningof{\quotep{E}} = \{ \quotep{P} \in \quotep{\pi} | P \in \meaningof{E} \}, \and \meaningof{\quotep{P}} = \{ \quotep{Q} \in \quotep{\pi} | P \equiv Q \} \and \\ \meaningof{@\quotep{E}} = \{ P \in \pi | P \equiv @x, x \in \meaningof{E} \}}
\end{mathpar}

\begin{eqnarray*}
  \\
  \meaningof{-} : TS \to ST
\end{eqnarray*}

\begin{eqnarray*}
  \\
  L : TS \to ST
\end{eqnarray*}

\begin{eqnarray*}
  \\
  P \models E \iff P \in \meaningof{E}
\end{eqnarray*}

\begin{eqnarray*}
  P \approx_{L} Q \iff \forall E \in L. P \models E \iff Q \models E
\end{eqnarray*}

\begin{eqnarray*}
  P \approx_{K} Q
\end{eqnarray*}

\begin{eqnarray*}
  P \approx Q
\end{eqnarray*}

$\approx_{K} = \approx = \approx_{L}$

\subsubsection{Contextual duality}

Note that contexts extend the quotation operation to a family of
operations from processes to names. Given a context, $M$, we can
define a \emph{nominal context}, $\quotep{M}$ by $\quotep{M}[P] :=
\quotep{M[P]}$. To foreshadow what is to come we observe that these
operations enjoy a duality with processes very much like the duality
between vectors and maps from vectors to scalars.

Further, because the calculus is essentially higher-order, we have a
correspondence between contexts and processes. More specifically,
given a name $x$ and a context $M$ we can construct $M^{*}_{x}$ such
that 

\begin{mathpar}
  M^{*}_{x} | \lift{x}{P} \red M[P]
\end{mathpar}

namely,

\begin{mathpar}
  M^{*}_{x} := x?(u).M[\dropn{u}]
\end{mathpar}

The dependence of $M^{*}_{x}$ on a name makes it an abstraction, 

\begin{mathpar}
  M^{*} := (x)x?(u).M[\dropn{u}]
\end{mathpar}

\subsection{Additional notation}

It will sometimes be convenient to denote the process a name
quotes. We already have the notation $x = \quotep{P}$, but it will be
convenient to introduce an alternate notation, $\procn{x}$, when we
want to emphasize the connection to the use of the name. Note that, by
virtue of name equivalence, $\quotep{\procn{x}} \nameeq x$; so, the
notation is consistent with previous definitions.

Further, because names have structure it is possible to effect
substitutions on the basis of that structure. This means we need to
upgrade our notation for substitutions, which we accomplish by
adapting comprehension notation. Thus,

\begin{mathpar}
  P\{ y / x : x \in S \}
\end{mathpar}

is interpreted to mean the process derived from P by replacing (in a
capture-avoiding manner) each occurrence of $x$ in $S$ by $y$. For example,

\begin{mathpar}
  P\{ \quotep{\procn{x}|\procn{x}} / x : x \in \freenames{P} \}
\end{mathpar}

will replace each (occurrence) of a free name $x$ in $P$ by
$\quotep{\procn{x}|\procn{x}}$.

Also, we will avail ourselves of the notation $x^{L}$ and $x^{R}$ to
denote injections of a name into disjoint copies of the name
space. There are numerous ways to accomplish this. One example can be
found in \cite{MeredithR05}. This notation overloads to vectors of
names: $\vec{x}^{\pi} := (x_{i}^{\pi} \; : \; 0 \leq i < |\vec{x}| )$ where $\pi \in \{L,R\}$.

We also use $P^{\Box} := P|\Box$.

In \cite{MeredithR05} an interpretation of the new operator is
given. It turns out that there are several possible interpretations
all enjoying the requisite algebraic properties of the operator (see
\cite{milner91polyadicpi}). We will therefore make liberal use of
$(\nu\; \vec{x})P$.

% subsection the_syntax_and_semantics_of_the_notation_system (end)   

\input{qm2pi.qmops} 

\input{qm2pi.sterngerlach} 

\input{qm2pi.metric} 

% section concurrent_process_calculi (end)

%\input{qm2pi.proofsketch}

% section proof sketch (end)

%\input{qm2pi.slviaknots} 

% section spatial logic via knots (end)

\input{qm2pi.conclusion}

% section conclusion (end)

%\input{qm2pi.dtcodes} 

% section wiring algorithm (end)

\input{qm2pi.ack} 

% section acknowledgments (end)

\newpage


\bibliographystyle{plain}   
\bibliography{../../biblios/main.bib}

\input{qm2pi.rhodetails}

\end{document}



% section proof sketch (end)

%\section{Unlikely characters: spatial logic for
  knots}\label{sub:characteristic_formulae} % (fold)

Associated to the mobile process calculi are a family of logics known
as the Hennessy-Milner logics. These logics typically enjoy a
semantics interpreting formulae as sets of processes that when
factored through the encoding outlined above allows an identification
of classes of knots with logical formulae. In the context of this
encoding the sub-family known as the spatial logics \cite{CairesC03}
\cite{CairesC04} \cite{Caires04} are of particular interest providing
several important features for expressing and reasoning about
properties (i.e. classes) of knots. We hint here at how this may be done.

%\begin{description}
%\item [structural connectives] 
\subsubsection{Structural connectives} The spatial logics enjoy
structural connectives corresponding, at the logical level, to the
parallel composition ($P | Q$) and new name ($(\nu \; x)P$)
connectives for processes. As illustrated in the examples below, these
connectives are extremely expressive given the shape of our encoding.
%\item [decideable satisfaction]

\subsubsection{Decideable satisfaction}
In \cite{Caires04} the satisfaction relation is shown to be decideable
for a rich class of processes. It further turns out that the image of
the our encoding is a proper subset of that class. This result
provides the basis for an algorithm by which to search for knots
enjoying a given property.
%\item [characteristic formulae]

\subsubsection{Characteristic formulae}
In the same paper \cite{Caires04} , Caires presents a means of calculating
characteristic formulae, selecting equivalence classes of processes
up to a pre--specified depth limit on the support set of names. Composed with our
encoding, this characteristic formula can be used to select
characteristic formulae for knots.
%\end{description}

\subsubsection{Spatial logic formulae}

The grammar below (segmented for comprehension) summarizes the syntax
of spatial logic formulae. We employ illustrative examples in the
sequel to provide an intuitive understanding of their meaning
referring the reader to \cite{Caires04} for a more detailed explication
of the semantics.

\begin{mathpar}
  \inferrule* [lab=boolean] {} {{A,B} \bc T \;|\; \neg A \;|\; A \wedge B \;|\; \eta = \eta'}
  \and
  \inferrule* [lab=spatial] {} {|\; \pzero \;|\; A | B \;|\; x \text{\textregistered} A \;|\; \forall x . A \;|\;  H x . A}
  \and
  \inferrule* [lab=behavioral] {} {|\; \alpha . A}
  \and 
  \inferrule* [lab=recursion] {} {|\; X(\vec{u}) \;|\; \mu X(\vec{u}) . A}
  \and
  \inferrule* [lab=action] {} {\alpha \bc \langle x?(\vec{y}) \rangle \;|\; \langle x!(\vec{y}) \rangle \;|\; \langle \tau \rangle}
  \and 
  \inferrule* [lab=name] {} {\eta \bc x \;|\; \tau}
\end{mathpar} 

% subsection characteristic_formulae (end)   	 

\subsection{Example formulae}\label{sub:example_formulae_} % (fold)

\subsubsection{Crossing as formula.}
% 
% \begin{align*}
%   \frac{d}{dx} \sin x &= \cos x 
%   & \frac{d}{dx} e^x &= e^x \\
%   \frac{d}{dx} \cos x &= - \sin x 
%   & \frac{d}{dx} \log x &= \frac{1}{x} \\
% \end{align*} 

\begin{align*}
 \mu C(x_{0},x_{1},y_{0},y_{1},u).&(\langle x_{0}?(z) \rangle(\langle u! \rangle\langle y_{1}!z \rangle C(x_{0},x_{1},y_{0},y_{1},u)) & \\
  & \wedge \langle y_{1}?(z) \rangle (\langle u! \rangle \langle x_{0}!z \rangle C(x_{0},x_{1},y_{0},y_{1},u)) & \\
  & \wedge \langle x_{1}?(z) \rangle (\langle u? \rangle \langle y_{0}!z \rangle C(x_{0},x_{1},y_{0},y_{1},u)) & \\
  & \wedge \langle y_{0}?(z) \rangle (\langle u? \rangle \langle x_{1}!z \rangle C(x_{0},x_{1},y_{0},y_{1},u))) &
\end{align*}

The lexicographical similarity between the shape of this formulae and
the shape of definition of the process representing a crossing reveals
the intuitive meaning of this formulae. It describes the capabilities
of a process that has the right to represent a crossing. For example
it picks out processes that may perform an input on the port $x_0$ in
its initial menu of capabilities. What differentiates the formula
from the process, however, is that the crossing process is the
smallest candidate to satisfy the formula. Infinitely many other
processes -- with internal behavior hidden behind this interface, so
to speak -- also satisfy this formula. Even this simple formula,
then, can be seen to open a new view onto knots, providing a
computational interpretation of \emph{virtual} knots.

Note that this formula is derived by hand. A similar formula can be
derived by employing Caires' calculation of characteristic formula
\cite{Caires04} to the process representing a crossing. In light of
this discussion, we let
$\meaningof{C}_{\phi}(x0,x1,y0,y1,u)$ denote a formula specifying the
dynamics we wish to capture of a crossing. To guarantee we preserve
the shape of the interface and minimal semantics we demand that
$\meaningof{C}_{\phi}(x0,x1,y0,y1,u) \Rightarrow
\textbf{C}(x0,x1,y0,y1,u)$ where $\textbf{C}(x0,x1,y0,y1,u)$ denotes
the formula above.
                            
\subsubsection{Crossing number constraints.}
The moral content of the context lemma (Lemma \ref{context}) is that the notion of
``locality'' in the Reidemeister moves is effectively captured by the
parallel composition operator of the process calculus. This intuition
extends through the logic. Given a formula,
$\meaningof{C}_{\phi}(x0,x1,y0,y1,u)$, we can use the structural
connectives to specify constraints on crossing numbers, such as at
least $n$ crossings, or exactly $n$ crossings.
\begin{mathpar}
  \inferrule* [lab=at-least-n] {} { K^{\geq n}_{\phi}(\vec{xs},\vec{ys}) := \Pi_{i=0}^{n-1} Hu . \meaningof{C}_{\phi}(xs_i,ys_i,u) | T }
  \and 
  \inferrule* [lab=exactly-n] {} { K^{= n}_{\phi}(\vec{xs},\vec{ys}) := \Pi_{i=0}^{n-1} Hu . \meaningof{C}_{\phi}(xs_i,ys_i,u) | \neg (\forall x_0,y_0,x_1,y_1,u . \meaningof{C}_{\phi}(x_0,y_0,x_1,y_1,u) | T) }
\end{mathpar}

To round out this section, recall that the encoding of an $n$-crossing
knot decomposes into a parallel composition of $n$ \emph{copies} of a
crossing process together with a wiring harness. To specify different
knot classes with the same crossing number amounts to specifying
logical constraints on the wiring harness. In the interest of space,
we defer examples to a forthcoming paper. Suffice it to say that both
the conditions ``alternating knot'' and ``contains the tangle
corresponding to 5/3'' are expressible. For example, it is possible to
calculate the characteristic formula of a process corresponding to the
tangle 5/3 and conjoin it into the classifying formula via the
composition connective of the logic.

Finally, we wish to observe that it is entirely within reason to
contemplate a more domain-specific version of spatial logic tailored
to the shape of processes in the image of the encoding. Such a
domain-specific logic would have a better claim to the title formal
language of knot properties.

% subsection example_formulae_ (end)

% section knots_as_processes (end) 

% section spatial logic via knots (end)

\section{Conclusions and future work}

\paragraph{Testing physical space}
You, gentle reader, may wonder why of all the theorems to be proved
given this set up we pick the one above. In some sense it's hardly
central to quantum mechanics. We see it as central in the sense that
it firmly establishes a notion of physical space arising from a notion
of the equivalence of behavior. Relating bisimulation to a metric is a
big step forward, but one is faced with interpreting the relationship
of that metric space to something more physical. Quantum mechanical
notions of ``physical'' space are still far from intuitive, but by
relating this idea of distance as testing to calculations that predict
physical circumstances we are making a not insignificant step forward
toward an understanding of the physical space we inhabit as
essentially dynamic.

\paragraph{Effectivity and simulation}
One of the observations we have yet to make is that the entire program
spelled out here is effective. We have built various interpreters for
the reflective calculus at work in this interpretation. In principle,
then, we can simulate quantum mechanics on a computer. The place where
the simulation may lose fidelity is the infinitely branching summation
for the annihilator.

In this connection i also want to point out that the evaluation style
calculation of the inner product puts the non-determinism of the
summation right at the heart of measurement. This suggests that
Milner's original reduction-based formulation of the dynamics of his
calculi in terms of sums was not just notationally suggestive of a
notion of measure-and-continue but captured some significant part of
the physics.

\paragraph{Quantum continuations}
In light of this last observation i want to point out that the
predominant account of quantum mechanics is missing a key aspect of a
truly compositional story of the physical situation. In a real lab,
when a measurement is made the observation can be made to feed into
another device that then makes another measurement conditioned on the
results of the first. This means that after the superposition was
collapsed the entire experimental set up remained in
superposition. While QM offers a means of writing this down it doesn't
quite line up well with the well-trodden formulation of computation
and continuation that we see so succinctly expressed in Milner's
calculi. This suggests that there might be advantages to this account
of dynamics waiting to be explored.

\paragraph{Quantum logic}
In this connection, we also note that by virtue of having the
Hennessy-Milner construction, we can pull the construction through the
interpretation of QM. This gives us a natural candidate for a quantum
logic that enjoys an extremely tight connection with it's domain of
interpretation, making the construction much less ad hoc (rather it is
the image of functor!).

\paragraph{Quantum probabiity}
i have questions about the basis of the interpretation of inner
product as probability amplitude. In particular, using which
axiomatization of probability theory does the notion of probability
amplitude earn the right to be so dubbed? In other words, where is the
proof that the operation for calculating a probability amplitude (and
then squaring) satisfies the axioms of what it means to calculate a
probability? Even if such a proof exists (i have yet to find it in the
literature), i wonder if it might not be possible to turn things on
their heads. Can we view the calculation of the probability amplitude
as an axiomatization of probability? If so, then the definition we
give for calculating probability amplitude may provide the basis for
an \emph{effective} theory of probability.

\paragraph{Quantum vs ``biological'' information}
Finally, i want to conclude with a more philosophical observation. At
a recent workshop in which QM was a predominant topic i noticed
something about quantum information. The speaker was giving a riveting
discussion of axiomatic QM and showing how properties of ``no
cloning'' and ``no deleting'' emerged as consequences of the
axiomatization. Theorems of this form are necessary to give us a sense
of confidence that our axioms characterize the physical theory. What
struck me, though, was that if quantum information is neither erasable
nor replicable it is markedly different from \emph{life}. Two of the
things we know about life is that

\begin{itemize}
  \item it ends;
  \item to gain some measure of persistence, to transcend it's
    finitude it is imminently copyable.
\end{itemize}

Both of these qualities are summarized succinctly in the aphorism: all
flesh is grass. For me these two kinds of ``information'' -- call them
quantum and biological -- are end points on a spectrum of strategies
for persistence. At one end, we have those curious entities that enjoy
uniqueness and permanence; at the other, we have those who in the face
of a certain end and an uncertain present make a go of passing
something on. To me one of the more remarkable aspects of the latter
strategy is that in the presence of noise (and certain features of
copying) we get a kind of dynamism, a chance for improvement against a
given persistent condition.

% subsection other_calculi_other_bisimulations_and_geometry_as_behavior (end)




% section conclusion (end)

%\documentclass[12pt]{llncs}
%\documentclass{jktr}

\usepackage[pdftex]{hyperref}                   
\usepackage {listings}
\usepackage {mathpartir}
\usepackage{bcprules}
%\usepackage{listings}
                       
\usepackage{graphicx} 
%\usepackage[margins=2.5cm,nohead,nofoot]{geometry}
%\usepackage{geometry}
\usepackage{amsfonts}
\usepackage{amstext}
\usepackage{latexsym}
\usepackage{amssymb}
\usepackage{color}


%\include{myPreamble}
\include{qm2pi.local} 

%\ifpdf
%\usepackage[pdftex]{graphicx}
%\else
%\usepackage{graphicx}
%\fi

 % \ifpdf
%  \usepackage{pdfsync}
%  \if


%\title{Brief Article}
%\author{David F. Snyder}
%\author{L.G. Meredith}

%\address{Dept. of Math., Texas State University--San Marcos, San Marcos, TX 78666}
       
\pagestyle{empty}


\begin{document}

\lstset{language=[Objective]Caml,frame=shadowbox}

\input{qm2pi.front}

% section front matter (end)

\input{qm2pi.intro} 
 
% section introduction (end)

% \input{qm2pi.knotations} 

% section notation (end)

\input{qm2pi.process.calculi} 

% section concurrent_process_calculi_and_spatial_logics_ (end)
    
%\input{qm2pi.knots2pi} 

%\input{qm2pi.trefoil} 

%\input{qm2pi.mainthm} 

% subsection basic_interpretation (end)

%\input{qm2pi.rho.presentation} 
\subsection{The syntax and semantics of the notation system}\label{sub:the_syntax_and_semantics_of_the_notation_system} % (fold)

We now summarize a technical presentation of the calculus that
embodies our theory of dynamics. The typical presentation of such a
calculus follows the style of giving generators and relations on
them. The grammar, below, describing term constructors, freely
generates the set of processes, $\Proc$. This set is then quotiented
by a relation known as structural congruence and it is over this set
that the notion of dynamics is expressed. This presentation is
essentially that of \cite{MeredithR05} with the addition of
polyadicity and summation. For readability we have relegated some of
the technical subtleties to an appendix.

\subsubsection{Process grammar}\label{subsub:process_grammar}

\begin{mathpar}
  \inferrule* [lab=synchronization] {} {{M} \bc \pzero \;|\; x?F \;|\; x!C }
  \and
  \inferrule* [lab=abstraction] {} {{F} \bc (x)P}
  \and
  \inferrule* [lab=concretion] {} {{C} \bc \langle Q \rangle}
  \and
  \inferrule* [lab=process] {} {{P,Q} \bc M \;| \;P|Q \;|\; @{x}}
  \and
  \inferrule* [lab=name] {} {{x} \bc \quotep{P}}
\end{mathpar} 

Note that $\vec{x}$ (resp. $\vec{P}$) denotes a vector of names
(resp. processes) of length $|\vec{x}|$ (resp. $|\vec{P}|$). We adopt
the following useful abbreviations.

\begin{mathpar}
   x?(\vec{y}).P := x.(\vec{y})P \and  x\clift{\vec{P}} := x.\clift{\vec{P}}
   \and x!(y) := \lift{x}{\dropn{y}}
   \and \Pi_{i=0}^{n-1}P_i := P_0 | \ldots | P_{n-1}
\end{mathpar}

\subsubsection{Structural congruence}

\paragraph{Free and bound names and alpha-equivalence.} At the
core of structural equivalence is alpha-equivalence which identifies
process that are the same up to a change of variable. Formally, we
recognize the distinction between free and bound names. The free names
of a process, $\freenames{P}$, may be calculated recursively as
follows:

\begin{mathpar}
\freenames{\pzero} := \emptyset
  \and \\
  \freenames{x?(y).P} := \{ x \} \cup (\freenames{P} \setminus \{ y \})
  \and 
  \freenames{x!\langle P \rangle} := \{ x \} \cup \{ P \} 
  \and \\
  \freenames{P|Q} := \freenames{P} \cup \freenames{Q}
  \and \\
  \freenames{@{x}} := \{ x \}
\end{mathpar}

$\pi$
$\quotep{\pi}$

$\freenames{-} : \pi \to \mathcal{P}(\quotep{\pi})$

\begin{eqnarray*}
  \freenames{\pzero} & := & \emptyset \\
  \freenames{x?(y).P} & := & \{ x \} \cup (\freenames{P} \setminus \{ y \}) \\
  \freenames{x!\langle P \rangle} & := & \{ x \} \cup \{ P \} \\
  \freenames{P|Q} & := & \freenames{P} \cup \freenames{Q} \\
  \freenames{\dropn{x}} & := & \{ x \}
\end{eqnarray*}

The bound names of a process, $\boundnames{P}$, are those names occurring in $P$
that are not free. For example, in $x?(y).0$, the name $x$ is free, while $y$ is bound.

\begin{mathpar}
  \inferrule* [lab=monoidal-laws] {} { P|Q \equiv Q|P \and P|0 \equiv P \and P|(Q|R) \equiv (P|Q)|R }
\end{mathpar}

\begin{mathpar}
  \inferrule* [lab=alpha-equivalence] {} { (x)P \equiv (y)P\{y/x\} \and y \not\in \freenames{P} }
\end{mathpar}

\begin{definition}
Then two processes, $P,Q$, are alpha-equivalent if $P = Q\{\vec{y}/\vec{x}\}$ for
some $\vec{x} \in \boundnames{Q},\vec{y} \in \boundnames{P}$, where $Q\{\vec{y}/\vec{x}\}$
denotes the capture-avoiding substitution of $\vec{y}$ for $\vec{x}$ in $Q$.
\end{definition}

\begin{definition}
  The {\em structural congruence} \cite{SangiorgiWalker} , $\equiv$,
  between processes is the least congruence containing
  alpha-equivalence, satisfying the abelian monoid laws
  (associativity, commutativity and $\pzero$ as identity) for parallel
  composition $|$ and for summation $+$.
\end{definition}

\subsection{Name equivalence}

We take name equivalence, written $\nameeq$, to be the smallest
equivalence relation generated by the following rules.

\begin{mathpar}
\inferrule*[lab=Quote-drop]
{ }
{ \quotep{@{x}} \nameeq x }

\inferrule*[lab=Struct-equiv]
{ P \scong Q }
{ \quotep{P} \nameeq \quotep{Q} }
\end{mathpar}

The astute reader will have noticed that the mutual recursion of names
and processes imposes a mutual recursion on alpha-equivalence and
structural equivalence via name-equivalence. Fortunately, all of this
works out pleasantly and we may calculate in the natural way, free of
concern. The reader interested in the details is referred to the
appendix \ref{appendix:rho_details}.

\subsection{Substitution}

We use $\Proc$ for the set of processes, $\QProc$ for the set of
names, and $\id{\{}\vec{y} / \vec{x} \id{\}}$ to denote partial maps,
$s : \QProc \rightarrow \QProc$. A map, $s$ lifts, uniquely, to a map
on process terms, $\widehat{s} : \Proc \rightarrow \Proc$ by the
following equations.

\begin{mathpar}
  (0) \psubstp{Q}{P} := 0 \\
  (R \juxtap S) \psubstp{Q}{P}
  :=    
  (R)\psubstp{Q}{P} \juxtap (S) \psubstp{Q}{P} \\
  (x?(y).R) \psubstp{Q}{P}    
  :=    
  (x)\substp{Q}{P} (z)\concat( (R \psubstn{z}{y}) \psubstp{Q}{P} ) \\
  (\lift{x}{R}) \psubstp{Q}{P}  
  :=
  \lift{(x)\substp{Q}{P}}{ R \psubstp{Q}{P} } \\
%   (\dropn{x})  \psubstp{Q}{P}       
%   := 
%   \left\{ 
%     \begin{array}{ccc} 
%       \dropn{\quotep{Q}} & & x \nameeq \quotep{P} \\
%       \dropn{x} & & otherwise \\
%     \end{array}
%   \right. 
  (\dropn{x})  \psubstp{Q}{P}       
  := 
  \left\{ 
    \begin{array}{ccc} 
      Q & & x \nameeq \quotep{P} \\
      \dropn{x} & & otherwise \\
    \end{array}
  \right.
\end{mathpar}
 

where

\begin{eqnarray}
  (x)\id{\{} \lpquote Q \rpquote / \lpquote P \rpquote \id{\}}            = 
  \left\{ 
    \begin{array}{ccc}
      \lpquote Q \rpquote & & x \nameeq \lpquote P \rpquote \\
      x & & otherwise \\
    \end{array}
  \right. \nonumber
\end{eqnarray}

and $z$ is chosen distinct from $\quotep{P}$, $\quotep{Q}$, the free
names in $Q$, and all the names in $R$. Our $\alpha$-equivalence will
be built in the standard way from this substitution.

\begin{remark}\label{rem:no_self_referential_names}
  One consequence of these definitions is that $\forall P. \quotep{P}
  \not\in \freenames{P}$.
\end{remark}

\subsection{ Dynamic quote: an example }

Anticipating something of what's to come, consider applying the
substitution, $\widehat{\id{\{}u / z \id{\}}}$, to the following pair
of processes, $\lift{w}{y!(z)}$ and $w[ \lpquote y!(z) \rpquote ]$.

\begin{eqnarray}
	\lift{w}{y!(z)}\widehat{\id{\{}u / z \id{\}}}
		& = &
		\lift{w}{y!(u)} \nonumber\\
	w[ \lpquote y!(z) \rpquote ] \widehat{ \id{\{}u / z \id{\}} }
		& = &
		w[ \lpquote y!(z) \rpquote ] \nonumber
\end{eqnarray}

Because the body of the process between quotes is impervious to
substitution, we get radically different answers. In fact, by
examining the first process in an input context,
e.g. $x?(z).\lift{w}{y!(z)}$, we see that the process under the lift
operator may be shaped by prefixed inputs binding a name inside it. In
this sense, the lift operator will be seen as a way to dynamically
construct processes before reifying them as names.

Finally equipped with these standard features we can present the
dynamics of the calculus.

\subsubsection{Operational semantics} 

Finally, we introduce the computational dynamics. What marks these
algebras as distinct from other more traditionally studied algebraic
structures, e.g. vector spaces or polynomial rings, is the manner in
which dynamics is captured. In traditional structures, dynamics is typically
expressed through morphisms between such structures, as in linear maps
between vector spaces or morphisms between rings. In algebras
associated with the semantics of computation, the dynamics is
expressed as part of the algebraic structure itself, through a
reduction reduction relation typically denoted by $\red$. Below, we
give a recursive presentation of this relation for the calculus used
in the encoding.

$\red \subseteq \pi \times \pi$
$\red : \pi \to \mathcal{P}(\pi)$

\begin{mathpar}
  \inferrule* [lab=Comm] { \textsf{match}( x_{src}, x_{trgt} ) } { x_{trgt}?(y)P \; | \; x_{src}!\langle {Q} \rangle \red P\{\quotep{Q}/y}\} }
  \and \\
  \inferrule* [lab=Par] {{P} \red {P}'} {{{P} | {Q}} \red {{P}' | {Q}}}
  \and
  \inferrule* [lab=Equiv]{{{P} \scong {P}'} \andalso {{P}' \red {Q}'} \andalso {{Q}' \scong {Q}}}{{P} \red {Q}}
\end{mathpar}

\begin{eqnarray*}
  match_{\equiv} (\quotep{P},\quotep{Q}) & := & P \equiv Q \\
  match_{\dagger}(\quotep{P},\quotep{Q}) & := & \forall R. P|Q \red^{*} R => R \red^{*} 0 \\
  match_{K}(\quotep{P},\quotep{Q}) & := & K \mbox{ for some context } K
\end{eqnarray*}

$u?(x)P | u!\langle Q \rangle \red P\{\quotep{Q}/x\}$

%We write $\wred$ for $\red^*$, and $P\red$ if $\exists Q $ such that $ P \red Q$.
We write $P\red$ if $\exists Q $ such that $ P \red Q$ and $P\not\red$, otherwise.

\section{Replication}

As mentioned before, it is known that replication (and hence
recursion) can be implemented in a higher-order process algebra
\cite{SangiorgiWalker}. As our first example of calculation with the
machinery thus far presented we give the construction explicitly in
the {\rhoc}.

\begin{eqnarray}
	D_{x} & := & \prefix{x}{y}{(\binpar{\outputp{x}{y}}{@{y}})} \nonumber\\
	\bangp_{x}{P} & := & \binpar{{x}!\langle{\binpar{D_{x}}{P}}\rangle}{D_{x}} \nonumber
\end{eqnarray}

\begin{eqnarray}
	\bangp_{x}{P} & & \nonumber\\
	=
	& {x}!\langle{(\prefix{x}{y}{(\outputp{x}{y} | @{y})) | P}}\rangle 
	      | \prefix{x}{y}{(\outputp{x}{y} | @{y})} & \nonumber\\
	\red
	& (\outputp{x}{y} | @{y})\substn{\quotep{(\prefix{x}{y}{(@{y} | \outputp{x}{y})) | P}}}{y} & \nonumber\\
	=
	& \outputp{x}{\quotep{(\prefix{x}{y}{(\outputp{x}{y} | @{y})) | P}}}
	  | {(\prefix{x}{y}{(\outputp{x}{y} | @{y})) | P}} & \nonumber\\
	\red
	& \ldots & \nonumber\\
	\red^*
	& P | P | \ldots & \nonumber
\end{eqnarray}

Of course, this encoding, as an implementation, runs away, unfolding
$\bangp{P}$ eagerly. A lazier and more implementable replication
operator, restricted to input-guarded processes, may be obtained as follows.

\begin{eqnarray}
\bangp{\prefix{u}{v}{P}} 
	:= 
	\binpar{\lift{x}{\prefix{u}{v}{(\binpar{D(x)}{P})}}}{D(x)} \nonumber
\end{eqnarray}

\begin{remark}
  Note that the lazier definition still does not deal with summation
  or mixed summation (i.e. sums over input and output). The reader is
  invited to construct definitions of replication that deal with these
  features. 

  Further, the definitions are parameterized in a name, $x$. Can you,
  gentle reader, make a definition that eliminates this parameter and
  guarantees no accidental interaction between the replication
  machinery and the process being replicated -- i.e. no accidental
  sharing of names used by the process to get its work done and the
  name(s) used by the replication to effect copying. This latter
  revision of the definition of replication is crucial to obtaining
  the expected identity $!!P \sim !P$.
\end{remark}

\begin{remark}\label{rem:paradoxical_combinator}
  The reader familiar with the lambda calculus will have noticed the
  similarity between $D$ and the paradoxical combinator.

  [Ed. note: the existence of this seems to suggest we have to be more
  restrictive on the set of processes and names we admit if we are to
  support no-cloning.]
\end{remark}

\subsubsection{Bisimulation}

The computational dynamics gives rise to another kind of equivalence,
the equivalence of computational behavior. As previously mentioned
this is typically captured \emph{via} some form of bisimulation.

% The notion we use in this paper is weak barbed bisimulation
% \cite{milner91polyadicpi}.

The notion we use in this paper is derived from weak barbed
bisimulation \cite{milner91polyadicpi}. 

\begin{definition}
An \emph{observation relation}, $\downarrow_{\mathcal N}$, over a set
of names, $\mathcal N$, is the smallest relation satisfying the rules
below.

\infrule[Out-barb]{y \in {\mathcal N}, \; x \nameeq y}
		  {\outputp{x}{v} \downarrow_{\mathcal N} x}
\infrule[Par-barb]{\mbox{$P\downarrow_{\mathcal N} x$ or $Q\downarrow_{\mathcal N} x$}}
		  {\binpar{P}{Q} \downarrow_{\mathcal N} x}

We write $P \Downarrow_{\mathcal N} x$ if there is $Q$ such that 
$P \wred Q$ and $Q \downarrow_{\mathcal N} x$.
\end{definition}

\begin{definition}
%\label{def.bbisim}
An  ${\mathcal N}$-\emph{barbed bisimulation} over a set of names, ${\mathcal N}$, is a symmetric binary relation 
${\mathcal S}_{\mathcal N}$ between agents such that $P\rel{S}_{\mathcal N}Q$ implies:
\begin{enumerate}
\item If $P \red P'$ then $Q \wred Q'$ and $P'\rel{S}_{\mathcal N} Q'$.
\item If $P\downarrow_{\mathcal N} x$, then $Q\Downarrow_{\mathcal N} x$.
\end{enumerate}
$P$ is ${\mathcal N}$-barbed bisimilar to $Q$, written
$P \wbbisim_{\mathcal N} Q$, if $P \rel{S}_{\mathcal N} Q$ for some ${\mathcal N}$-barbed bisimulation ${\mathcal S}_{\mathcal N}$.
\end{definition}

$\mathcal{R} \subseteq \pi \times \pi$

$P \mathcal{R} Q => \forall P'. P \red P' \Rightarrow \exists Q'. Q \red Q', P' \mathcal{R} Q'$

$P \vdash x \Rightarrow Q \vdash x$

\begin{mathpar}
  \inferrule*[lab=Out-barb]{x \nameeq y}{{y}!\langle{Q}\rangle \vdash x}
  \and
  \inferrule*[lab=Par-barb]{\mbox{$P\vdash x$ or $Q\vdash x$}}{\binpar{P}{Q} \vdash x}
\end{mathpar}

\subsubsection{Contexts}

One of the principle advantages of computational calculi like the
$\pi$-calculus is a well-defined notion of context,
contextual-equivalence and a correlation between
contextual-equivalence and notions of bisimulation. The notion of
context allows the decomposition of a process into (sub-)process and
its syntactic environment, its context. Thus, a context may be
thought of as a process with a ``hole'' (written $\Box$) in it. The
application of a context $M$ to a process $P$, written $M[P]$, is
tantamount to filling the hole in $M$ with $P$. In this paper we do
not need the full weight of this theory, but do make use of the notion
of context in the proof the main theorem. 

\begin{mathpar}
  \inferrule* [lab=summation] {} {{M_{M},M_{N}} \bc \Box \;|\; x.M_{A} \;|\; M_{M}+M_{N}}
  \and
  \inferrule* [lab=agent] {} {{M_{A}} \bc (\vec{x})M_{P} \;| \; \clift{P_0,\ldots,M_{P},\ldots,P_N}}
  \and \\
  \inferrule* [lab=process] {} {{M_{P}} \bc M_{N} \;| \;P|M_{P} }
\end{mathpar} 

\begin{mathpar}
  \inferrule* [lab=sychronization] {} {M_{N} \bc \Box \;|\; x?M_{F} \;|\; x!M_{C}}
  \and
  \inferrule* [lab=abstraction] {} {{M_{F}} \bc (x)M_{P} }
  \and
  \inferrule* [lab=concretion] {} {{M_{C}} \bc \langle M_{P} \rangle }
  \and \\
  \inferrule* [lab=process] {} {{M_{P}} \bc M_{N} \;| \;P|M_{P} }
\end{mathpar}

\begin{definition}[contextual application] Given a context $M$, and
  process $P$, we define the \emph{contextual application}, $M[P] :=
  M\{P/\Box\}$. That is, the contextual application of M to P is the
  substitution of $P$ for $\Box$ in $M$.
\end{definition}

$\meaningof{-} : L \to \mathcal{P}(\pi)$

\begin{mathpar}
  \inferrule* [lab=collection] {} {\meaningof{true} = \pi, \and \meaningof{~E} = \pi \setminus \meaningof{E}, \and \meaningof{E_{1} \& E_{2}} = \meaningof{E_{1}} \cap \meaningof{E_{2}}}
\end{mathpar}

\begin{mathpar}
  \inferrule* [lab=structure] {} {\meaningof{0} = \{ P \in \pi | P \equiv 0 \}, \and \\ \meaningof{E_1 | E_2} = \{ P \in \pi | P \equiv P_{1} | P_{2}, P_{1} \in \meaningof{E_{1}}, P_{2} \in \meaningof{E_2}\} }
\end{mathpar}

\begin{mathpar}
 \inferrule* [lab=behavior] {} {\meaningof{\langle a?b \rangle E} = \{ P \in \pi | P \equiv Q | u?(y)P', \\ \and \\\\ \and \\ \;\;\; u \in \meaningof{a}, \forall z.P'\{z/y\} \in \meaningof{E\{z/b\}}\}, \and \\ \meaningof{a!E} = \{ P \in \pi | P \equiv Q | x!\langle P' \rangle, x \in \meaningof{a} P' \in \meaningof{E}\} }
\end{mathpar}

\begin{mathpar}
 \inferrule* [lab=nominal] {} {\meaningof{\quotep{E}} = \{ \quotep{P} \in \quotep{\pi} | P \in \meaningof{E} \}, \and \meaningof{\quotep{P}} = \{ \quotep{Q} \in \quotep{\pi} | P \equiv Q \} \and \\ \meaningof{@\quotep{E}} = \{ P \in \pi | P \equiv @x, x \in \meaningof{E} \}}
\end{mathpar}

\begin{eqnarray*}
  \\
  \meaningof{-} : TS \to ST
\end{eqnarray*}

\begin{eqnarray*}
  \\
  L : TS \to ST
\end{eqnarray*}

\begin{eqnarray*}
  \\
  P \models E \iff P \in \meaningof{E}
\end{eqnarray*}

\begin{eqnarray*}
  P \approx_{L} Q \iff \forall E \in L. P \models E \iff Q \models E
\end{eqnarray*}

\begin{eqnarray*}
  P \approx_{K} Q
\end{eqnarray*}

\begin{eqnarray*}
  P \approx Q
\end{eqnarray*}

$\approx_{K} = \approx = \approx_{L}$

\subsubsection{Contextual duality}

Note that contexts extend the quotation operation to a family of
operations from processes to names. Given a context, $M$, we can
define a \emph{nominal context}, $\quotep{M}$ by $\quotep{M}[P] :=
\quotep{M[P]}$. To foreshadow what is to come we observe that these
operations enjoy a duality with processes very much like the duality
between vectors and maps from vectors to scalars.

Further, because the calculus is essentially higher-order, we have a
correspondence between contexts and processes. More specifically,
given a name $x$ and a context $M$ we can construct $M^{*}_{x}$ such
that 

\begin{mathpar}
  M^{*}_{x} | \lift{x}{P} \red M[P]
\end{mathpar}

namely,

\begin{mathpar}
  M^{*}_{x} := x?(u).M[\dropn{u}]
\end{mathpar}

The dependence of $M^{*}_{x}$ on a name makes it an abstraction, 

\begin{mathpar}
  M^{*} := (x)x?(u).M[\dropn{u}]
\end{mathpar}

\subsection{Additional notation}

It will sometimes be convenient to denote the process a name
quotes. We already have the notation $x = \quotep{P}$, but it will be
convenient to introduce an alternate notation, $\procn{x}$, when we
want to emphasize the connection to the use of the name. Note that, by
virtue of name equivalence, $\quotep{\procn{x}} \nameeq x$; so, the
notation is consistent with previous definitions.

Further, because names have structure it is possible to effect
substitutions on the basis of that structure. This means we need to
upgrade our notation for substitutions, which we accomplish by
adapting comprehension notation. Thus,

\begin{mathpar}
  P\{ y / x : x \in S \}
\end{mathpar}

is interpreted to mean the process derived from P by replacing (in a
capture-avoiding manner) each occurrence of $x$ in $S$ by $y$. For example,

\begin{mathpar}
  P\{ \quotep{\procn{x}|\procn{x}} / x : x \in \freenames{P} \}
\end{mathpar}

will replace each (occurrence) of a free name $x$ in $P$ by
$\quotep{\procn{x}|\procn{x}}$.

Also, we will avail ourselves of the notation $x^{L}$ and $x^{R}$ to
denote injections of a name into disjoint copies of the name
space. There are numerous ways to accomplish this. One example can be
found in \cite{MeredithR05}. This notation overloads to vectors of
names: $\vec{x}^{\pi} := (x_{i}^{\pi} \; : \; 0 \leq i < |\vec{x}| )$ where $\pi \in \{L,R\}$.

We also use $P^{\Box} := P|\Box$.

In \cite{MeredithR05} an interpretation of the new operator is
given. It turns out that there are several possible interpretations
all enjoying the requisite algebraic properties of the operator (see
\cite{milner91polyadicpi}). We will therefore make liberal use of
$(\nu\; \vec{x})P$.

% subsection the_syntax_and_semantics_of_the_notation_system (end)   

\input{qm2pi.qmops} 

\input{qm2pi.sterngerlach} 

\input{qm2pi.metric} 

% section concurrent_process_calculi (end)

%\input{qm2pi.proofsketch}

% section proof sketch (end)

%\input{qm2pi.slviaknots} 

% section spatial logic via knots (end)

\input{qm2pi.conclusion}

% section conclusion (end)

%\input{qm2pi.dtcodes} 

% section wiring algorithm (end)

\input{qm2pi.ack} 

% section acknowledgments (end)

\newpage


\bibliographystyle{plain}   
\bibliography{../../biblios/main.bib}

\input{qm2pi.rhodetails}

\end{document}

 

% section wiring algorithm (end)

\documentclass[12pt]{llncs}
%\documentclass{jktr}

\usepackage[pdftex]{hyperref}                   
\usepackage {listings}
\usepackage {mathpartir}
\usepackage{bcprules}
%\usepackage{listings}
                       
\usepackage{graphicx} 
%\usepackage[margins=2.5cm,nohead,nofoot]{geometry}
%\usepackage{geometry}
\usepackage{amsfonts}
\usepackage{amstext}
\usepackage{latexsym}
\usepackage{amssymb}
\usepackage{color}


%\include{myPreamble}
\include{qm2pi.local} 

%\ifpdf
%\usepackage[pdftex]{graphicx}
%\else
%\usepackage{graphicx}
%\fi

 % \ifpdf
%  \usepackage{pdfsync}
%  \if


%\title{Brief Article}
%\author{David F. Snyder}
%\author{L.G. Meredith}

%\address{Dept. of Math., Texas State University--San Marcos, San Marcos, TX 78666}
       
\pagestyle{empty}


\begin{document}

\lstset{language=[Objective]Caml,frame=shadowbox}

\input{qm2pi.front}

% section front matter (end)

\input{qm2pi.intro} 
 
% section introduction (end)

% \input{qm2pi.knotations} 

% section notation (end)

\input{qm2pi.process.calculi} 

% section concurrent_process_calculi_and_spatial_logics_ (end)
    
%\input{qm2pi.knots2pi} 

%\input{qm2pi.trefoil} 

%\input{qm2pi.mainthm} 

% subsection basic_interpretation (end)

%\input{qm2pi.rho.presentation} 
\subsection{The syntax and semantics of the notation system}\label{sub:the_syntax_and_semantics_of_the_notation_system} % (fold)

We now summarize a technical presentation of the calculus that
embodies our theory of dynamics. The typical presentation of such a
calculus follows the style of giving generators and relations on
them. The grammar, below, describing term constructors, freely
generates the set of processes, $\Proc$. This set is then quotiented
by a relation known as structural congruence and it is over this set
that the notion of dynamics is expressed. This presentation is
essentially that of \cite{MeredithR05} with the addition of
polyadicity and summation. For readability we have relegated some of
the technical subtleties to an appendix.

\subsubsection{Process grammar}\label{subsub:process_grammar}

\begin{mathpar}
  \inferrule* [lab=synchronization] {} {{M} \bc \pzero \;|\; x?F \;|\; x!C }
  \and
  \inferrule* [lab=abstraction] {} {{F} \bc (x)P}
  \and
  \inferrule* [lab=concretion] {} {{C} \bc \langle Q \rangle}
  \and
  \inferrule* [lab=process] {} {{P,Q} \bc M \;| \;P|Q \;|\; @{x}}
  \and
  \inferrule* [lab=name] {} {{x} \bc \quotep{P}}
\end{mathpar} 

Note that $\vec{x}$ (resp. $\vec{P}$) denotes a vector of names
(resp. processes) of length $|\vec{x}|$ (resp. $|\vec{P}|$). We adopt
the following useful abbreviations.

\begin{mathpar}
   x?(\vec{y}).P := x.(\vec{y})P \and  x\clift{\vec{P}} := x.\clift{\vec{P}}
   \and x!(y) := \lift{x}{\dropn{y}}
   \and \Pi_{i=0}^{n-1}P_i := P_0 | \ldots | P_{n-1}
\end{mathpar}

\subsubsection{Structural congruence}

\paragraph{Free and bound names and alpha-equivalence.} At the
core of structural equivalence is alpha-equivalence which identifies
process that are the same up to a change of variable. Formally, we
recognize the distinction between free and bound names. The free names
of a process, $\freenames{P}$, may be calculated recursively as
follows:

\begin{mathpar}
\freenames{\pzero} := \emptyset
  \and \\
  \freenames{x?(y).P} := \{ x \} \cup (\freenames{P} \setminus \{ y \})
  \and 
  \freenames{x!\langle P \rangle} := \{ x \} \cup \{ P \} 
  \and \\
  \freenames{P|Q} := \freenames{P} \cup \freenames{Q}
  \and \\
  \freenames{@{x}} := \{ x \}
\end{mathpar}

$\pi$
$\quotep{\pi}$

$\freenames{-} : \pi \to \mathcal{P}(\quotep{\pi})$

\begin{eqnarray*}
  \freenames{\pzero} & := & \emptyset \\
  \freenames{x?(y).P} & := & \{ x \} \cup (\freenames{P} \setminus \{ y \}) \\
  \freenames{x!\langle P \rangle} & := & \{ x \} \cup \{ P \} \\
  \freenames{P|Q} & := & \freenames{P} \cup \freenames{Q} \\
  \freenames{\dropn{x}} & := & \{ x \}
\end{eqnarray*}

The bound names of a process, $\boundnames{P}$, are those names occurring in $P$
that are not free. For example, in $x?(y).0$, the name $x$ is free, while $y$ is bound.

\begin{mathpar}
  \inferrule* [lab=monoidal-laws] {} { P|Q \equiv Q|P \and P|0 \equiv P \and P|(Q|R) \equiv (P|Q)|R }
\end{mathpar}

\begin{mathpar}
  \inferrule* [lab=alpha-equivalence] {} { (x)P \equiv (y)P\{y/x\} \and y \not\in \freenames{P} }
\end{mathpar}

\begin{definition}
Then two processes, $P,Q$, are alpha-equivalent if $P = Q\{\vec{y}/\vec{x}\}$ for
some $\vec{x} \in \boundnames{Q},\vec{y} \in \boundnames{P}$, where $Q\{\vec{y}/\vec{x}\}$
denotes the capture-avoiding substitution of $\vec{y}$ for $\vec{x}$ in $Q$.
\end{definition}

\begin{definition}
  The {\em structural congruence} \cite{SangiorgiWalker} , $\equiv$,
  between processes is the least congruence containing
  alpha-equivalence, satisfying the abelian monoid laws
  (associativity, commutativity and $\pzero$ as identity) for parallel
  composition $|$ and for summation $+$.
\end{definition}

\subsection{Name equivalence}

We take name equivalence, written $\nameeq$, to be the smallest
equivalence relation generated by the following rules.

\begin{mathpar}
\inferrule*[lab=Quote-drop]
{ }
{ \quotep{@{x}} \nameeq x }

\inferrule*[lab=Struct-equiv]
{ P \scong Q }
{ \quotep{P} \nameeq \quotep{Q} }
\end{mathpar}

The astute reader will have noticed that the mutual recursion of names
and processes imposes a mutual recursion on alpha-equivalence and
structural equivalence via name-equivalence. Fortunately, all of this
works out pleasantly and we may calculate in the natural way, free of
concern. The reader interested in the details is referred to the
appendix \ref{appendix:rho_details}.

\subsection{Substitution}

We use $\Proc$ for the set of processes, $\QProc$ for the set of
names, and $\id{\{}\vec{y} / \vec{x} \id{\}}$ to denote partial maps,
$s : \QProc \rightarrow \QProc$. A map, $s$ lifts, uniquely, to a map
on process terms, $\widehat{s} : \Proc \rightarrow \Proc$ by the
following equations.

\begin{mathpar}
  (0) \psubstp{Q}{P} := 0 \\
  (R \juxtap S) \psubstp{Q}{P}
  :=    
  (R)\psubstp{Q}{P} \juxtap (S) \psubstp{Q}{P} \\
  (x?(y).R) \psubstp{Q}{P}    
  :=    
  (x)\substp{Q}{P} (z)\concat( (R \psubstn{z}{y}) \psubstp{Q}{P} ) \\
  (\lift{x}{R}) \psubstp{Q}{P}  
  :=
  \lift{(x)\substp{Q}{P}}{ R \psubstp{Q}{P} } \\
%   (\dropn{x})  \psubstp{Q}{P}       
%   := 
%   \left\{ 
%     \begin{array}{ccc} 
%       \dropn{\quotep{Q}} & & x \nameeq \quotep{P} \\
%       \dropn{x} & & otherwise \\
%     \end{array}
%   \right. 
  (\dropn{x})  \psubstp{Q}{P}       
  := 
  \left\{ 
    \begin{array}{ccc} 
      Q & & x \nameeq \quotep{P} \\
      \dropn{x} & & otherwise \\
    \end{array}
  \right.
\end{mathpar}
 

where

\begin{eqnarray}
  (x)\id{\{} \lpquote Q \rpquote / \lpquote P \rpquote \id{\}}            = 
  \left\{ 
    \begin{array}{ccc}
      \lpquote Q \rpquote & & x \nameeq \lpquote P \rpquote \\
      x & & otherwise \\
    \end{array}
  \right. \nonumber
\end{eqnarray}

and $z$ is chosen distinct from $\quotep{P}$, $\quotep{Q}$, the free
names in $Q$, and all the names in $R$. Our $\alpha$-equivalence will
be built in the standard way from this substitution.

\begin{remark}\label{rem:no_self_referential_names}
  One consequence of these definitions is that $\forall P. \quotep{P}
  \not\in \freenames{P}$.
\end{remark}

\subsection{ Dynamic quote: an example }

Anticipating something of what's to come, consider applying the
substitution, $\widehat{\id{\{}u / z \id{\}}}$, to the following pair
of processes, $\lift{w}{y!(z)}$ and $w[ \lpquote y!(z) \rpquote ]$.

\begin{eqnarray}
	\lift{w}{y!(z)}\widehat{\id{\{}u / z \id{\}}}
		& = &
		\lift{w}{y!(u)} \nonumber\\
	w[ \lpquote y!(z) \rpquote ] \widehat{ \id{\{}u / z \id{\}} }
		& = &
		w[ \lpquote y!(z) \rpquote ] \nonumber
\end{eqnarray}

Because the body of the process between quotes is impervious to
substitution, we get radically different answers. In fact, by
examining the first process in an input context,
e.g. $x?(z).\lift{w}{y!(z)}$, we see that the process under the lift
operator may be shaped by prefixed inputs binding a name inside it. In
this sense, the lift operator will be seen as a way to dynamically
construct processes before reifying them as names.

Finally equipped with these standard features we can present the
dynamics of the calculus.

\subsubsection{Operational semantics} 

Finally, we introduce the computational dynamics. What marks these
algebras as distinct from other more traditionally studied algebraic
structures, e.g. vector spaces or polynomial rings, is the manner in
which dynamics is captured. In traditional structures, dynamics is typically
expressed through morphisms between such structures, as in linear maps
between vector spaces or morphisms between rings. In algebras
associated with the semantics of computation, the dynamics is
expressed as part of the algebraic structure itself, through a
reduction reduction relation typically denoted by $\red$. Below, we
give a recursive presentation of this relation for the calculus used
in the encoding.

$\red \subseteq \pi \times \pi$
$\red : \pi \to \mathcal{P}(\pi)$

\begin{mathpar}
  \inferrule* [lab=Comm] { \textsf{match}( x_{src}, x_{trgt} ) } { x_{trgt}?(y)P \; | \; x_{src}!\langle {Q} \rangle \red P\{\quotep{Q}/y}\} }
  \and \\
  \inferrule* [lab=Par] {{P} \red {P}'} {{{P} | {Q}} \red {{P}' | {Q}}}
  \and
  \inferrule* [lab=Equiv]{{{P} \scong {P}'} \andalso {{P}' \red {Q}'} \andalso {{Q}' \scong {Q}}}{{P} \red {Q}}
\end{mathpar}

\begin{eqnarray*}
  match_{\equiv} (\quotep{P},\quotep{Q}) & := & P \equiv Q \\
  match_{\dagger}(\quotep{P},\quotep{Q}) & := & \forall R. P|Q \red^{*} R => R \red^{*} 0 \\
  match_{K}(\quotep{P},\quotep{Q}) & := & K \mbox{ for some context } K
\end{eqnarray*}

$u?(x)P | u!\langle Q \rangle \red P\{\quotep{Q}/x\}$

%We write $\wred$ for $\red^*$, and $P\red$ if $\exists Q $ such that $ P \red Q$.
We write $P\red$ if $\exists Q $ such that $ P \red Q$ and $P\not\red$, otherwise.

\section{Replication}

As mentioned before, it is known that replication (and hence
recursion) can be implemented in a higher-order process algebra
\cite{SangiorgiWalker}. As our first example of calculation with the
machinery thus far presented we give the construction explicitly in
the {\rhoc}.

\begin{eqnarray}
	D_{x} & := & \prefix{x}{y}{(\binpar{\outputp{x}{y}}{@{y}})} \nonumber\\
	\bangp_{x}{P} & := & \binpar{{x}!\langle{\binpar{D_{x}}{P}}\rangle}{D_{x}} \nonumber
\end{eqnarray}

\begin{eqnarray}
	\bangp_{x}{P} & & \nonumber\\
	=
	& {x}!\langle{(\prefix{x}{y}{(\outputp{x}{y} | @{y})) | P}}\rangle 
	      | \prefix{x}{y}{(\outputp{x}{y} | @{y})} & \nonumber\\
	\red
	& (\outputp{x}{y} | @{y})\substn{\quotep{(\prefix{x}{y}{(@{y} | \outputp{x}{y})) | P}}}{y} & \nonumber\\
	=
	& \outputp{x}{\quotep{(\prefix{x}{y}{(\outputp{x}{y} | @{y})) | P}}}
	  | {(\prefix{x}{y}{(\outputp{x}{y} | @{y})) | P}} & \nonumber\\
	\red
	& \ldots & \nonumber\\
	\red^*
	& P | P | \ldots & \nonumber
\end{eqnarray}

Of course, this encoding, as an implementation, runs away, unfolding
$\bangp{P}$ eagerly. A lazier and more implementable replication
operator, restricted to input-guarded processes, may be obtained as follows.

\begin{eqnarray}
\bangp{\prefix{u}{v}{P}} 
	:= 
	\binpar{\lift{x}{\prefix{u}{v}{(\binpar{D(x)}{P})}}}{D(x)} \nonumber
\end{eqnarray}

\begin{remark}
  Note that the lazier definition still does not deal with summation
  or mixed summation (i.e. sums over input and output). The reader is
  invited to construct definitions of replication that deal with these
  features. 

  Further, the definitions are parameterized in a name, $x$. Can you,
  gentle reader, make a definition that eliminates this parameter and
  guarantees no accidental interaction between the replication
  machinery and the process being replicated -- i.e. no accidental
  sharing of names used by the process to get its work done and the
  name(s) used by the replication to effect copying. This latter
  revision of the definition of replication is crucial to obtaining
  the expected identity $!!P \sim !P$.
\end{remark}

\begin{remark}\label{rem:paradoxical_combinator}
  The reader familiar with the lambda calculus will have noticed the
  similarity between $D$ and the paradoxical combinator.

  [Ed. note: the existence of this seems to suggest we have to be more
  restrictive on the set of processes and names we admit if we are to
  support no-cloning.]
\end{remark}

\subsubsection{Bisimulation}

The computational dynamics gives rise to another kind of equivalence,
the equivalence of computational behavior. As previously mentioned
this is typically captured \emph{via} some form of bisimulation.

% The notion we use in this paper is weak barbed bisimulation
% \cite{milner91polyadicpi}.

The notion we use in this paper is derived from weak barbed
bisimulation \cite{milner91polyadicpi}. 

\begin{definition}
An \emph{observation relation}, $\downarrow_{\mathcal N}$, over a set
of names, $\mathcal N$, is the smallest relation satisfying the rules
below.

\infrule[Out-barb]{y \in {\mathcal N}, \; x \nameeq y}
		  {\outputp{x}{v} \downarrow_{\mathcal N} x}
\infrule[Par-barb]{\mbox{$P\downarrow_{\mathcal N} x$ or $Q\downarrow_{\mathcal N} x$}}
		  {\binpar{P}{Q} \downarrow_{\mathcal N} x}

We write $P \Downarrow_{\mathcal N} x$ if there is $Q$ such that 
$P \wred Q$ and $Q \downarrow_{\mathcal N} x$.
\end{definition}

\begin{definition}
%\label{def.bbisim}
An  ${\mathcal N}$-\emph{barbed bisimulation} over a set of names, ${\mathcal N}$, is a symmetric binary relation 
${\mathcal S}_{\mathcal N}$ between agents such that $P\rel{S}_{\mathcal N}Q$ implies:
\begin{enumerate}
\item If $P \red P'$ then $Q \wred Q'$ and $P'\rel{S}_{\mathcal N} Q'$.
\item If $P\downarrow_{\mathcal N} x$, then $Q\Downarrow_{\mathcal N} x$.
\end{enumerate}
$P$ is ${\mathcal N}$-barbed bisimilar to $Q$, written
$P \wbbisim_{\mathcal N} Q$, if $P \rel{S}_{\mathcal N} Q$ for some ${\mathcal N}$-barbed bisimulation ${\mathcal S}_{\mathcal N}$.
\end{definition}

$\mathcal{R} \subseteq \pi \times \pi$

$P \mathcal{R} Q => \forall P'. P \red P' \Rightarrow \exists Q'. Q \red Q', P' \mathcal{R} Q'$

$P \vdash x \Rightarrow Q \vdash x$

\begin{mathpar}
  \inferrule*[lab=Out-barb]{x \nameeq y}{{y}!\langle{Q}\rangle \vdash x}
  \and
  \inferrule*[lab=Par-barb]{\mbox{$P\vdash x$ or $Q\vdash x$}}{\binpar{P}{Q} \vdash x}
\end{mathpar}

\subsubsection{Contexts}

One of the principle advantages of computational calculi like the
$\pi$-calculus is a well-defined notion of context,
contextual-equivalence and a correlation between
contextual-equivalence and notions of bisimulation. The notion of
context allows the decomposition of a process into (sub-)process and
its syntactic environment, its context. Thus, a context may be
thought of as a process with a ``hole'' (written $\Box$) in it. The
application of a context $M$ to a process $P$, written $M[P]$, is
tantamount to filling the hole in $M$ with $P$. In this paper we do
not need the full weight of this theory, but do make use of the notion
of context in the proof the main theorem. 

\begin{mathpar}
  \inferrule* [lab=summation] {} {{M_{M},M_{N}} \bc \Box \;|\; x.M_{A} \;|\; M_{M}+M_{N}}
  \and
  \inferrule* [lab=agent] {} {{M_{A}} \bc (\vec{x})M_{P} \;| \; \clift{P_0,\ldots,M_{P},\ldots,P_N}}
  \and \\
  \inferrule* [lab=process] {} {{M_{P}} \bc M_{N} \;| \;P|M_{P} }
\end{mathpar} 

\begin{mathpar}
  \inferrule* [lab=sychronization] {} {M_{N} \bc \Box \;|\; x?M_{F} \;|\; x!M_{C}}
  \and
  \inferrule* [lab=abstraction] {} {{M_{F}} \bc (x)M_{P} }
  \and
  \inferrule* [lab=concretion] {} {{M_{C}} \bc \langle M_{P} \rangle }
  \and \\
  \inferrule* [lab=process] {} {{M_{P}} \bc M_{N} \;| \;P|M_{P} }
\end{mathpar}

\begin{definition}[contextual application] Given a context $M$, and
  process $P$, we define the \emph{contextual application}, $M[P] :=
  M\{P/\Box\}$. That is, the contextual application of M to P is the
  substitution of $P$ for $\Box$ in $M$.
\end{definition}

$\meaningof{-} : L \to \mathcal{P}(\pi)$

\begin{mathpar}
  \inferrule* [lab=collection] {} {\meaningof{true} = \pi, \and \meaningof{~E} = \pi \setminus \meaningof{E}, \and \meaningof{E_{1} \& E_{2}} = \meaningof{E_{1}} \cap \meaningof{E_{2}}}
\end{mathpar}

\begin{mathpar}
  \inferrule* [lab=structure] {} {\meaningof{0} = \{ P \in \pi | P \equiv 0 \}, \and \\ \meaningof{E_1 | E_2} = \{ P \in \pi | P \equiv P_{1} | P_{2}, P_{1} \in \meaningof{E_{1}}, P_{2} \in \meaningof{E_2}\} }
\end{mathpar}

\begin{mathpar}
 \inferrule* [lab=behavior] {} {\meaningof{\langle a?b \rangle E} = \{ P \in \pi | P \equiv Q | u?(y)P', \\ \and \\\\ \and \\ \;\;\; u \in \meaningof{a}, \forall z.P'\{z/y\} \in \meaningof{E\{z/b\}}\}, \and \\ \meaningof{a!E} = \{ P \in \pi | P \equiv Q | x!\langle P' \rangle, x \in \meaningof{a} P' \in \meaningof{E}\} }
\end{mathpar}

\begin{mathpar}
 \inferrule* [lab=nominal] {} {\meaningof{\quotep{E}} = \{ \quotep{P} \in \quotep{\pi} | P \in \meaningof{E} \}, \and \meaningof{\quotep{P}} = \{ \quotep{Q} \in \quotep{\pi} | P \equiv Q \} \and \\ \meaningof{@\quotep{E}} = \{ P \in \pi | P \equiv @x, x \in \meaningof{E} \}}
\end{mathpar}

\begin{eqnarray*}
  \\
  \meaningof{-} : TS \to ST
\end{eqnarray*}

\begin{eqnarray*}
  \\
  L : TS \to ST
\end{eqnarray*}

\begin{eqnarray*}
  \\
  P \models E \iff P \in \meaningof{E}
\end{eqnarray*}

\begin{eqnarray*}
  P \approx_{L} Q \iff \forall E \in L. P \models E \iff Q \models E
\end{eqnarray*}

\begin{eqnarray*}
  P \approx_{K} Q
\end{eqnarray*}

\begin{eqnarray*}
  P \approx Q
\end{eqnarray*}

$\approx_{K} = \approx = \approx_{L}$

\subsubsection{Contextual duality}

Note that contexts extend the quotation operation to a family of
operations from processes to names. Given a context, $M$, we can
define a \emph{nominal context}, $\quotep{M}$ by $\quotep{M}[P] :=
\quotep{M[P]}$. To foreshadow what is to come we observe that these
operations enjoy a duality with processes very much like the duality
between vectors and maps from vectors to scalars.

Further, because the calculus is essentially higher-order, we have a
correspondence between contexts and processes. More specifically,
given a name $x$ and a context $M$ we can construct $M^{*}_{x}$ such
that 

\begin{mathpar}
  M^{*}_{x} | \lift{x}{P} \red M[P]
\end{mathpar}

namely,

\begin{mathpar}
  M^{*}_{x} := x?(u).M[\dropn{u}]
\end{mathpar}

The dependence of $M^{*}_{x}$ on a name makes it an abstraction, 

\begin{mathpar}
  M^{*} := (x)x?(u).M[\dropn{u}]
\end{mathpar}

\subsection{Additional notation}

It will sometimes be convenient to denote the process a name
quotes. We already have the notation $x = \quotep{P}$, but it will be
convenient to introduce an alternate notation, $\procn{x}$, when we
want to emphasize the connection to the use of the name. Note that, by
virtue of name equivalence, $\quotep{\procn{x}} \nameeq x$; so, the
notation is consistent with previous definitions.

Further, because names have structure it is possible to effect
substitutions on the basis of that structure. This means we need to
upgrade our notation for substitutions, which we accomplish by
adapting comprehension notation. Thus,

\begin{mathpar}
  P\{ y / x : x \in S \}
\end{mathpar}

is interpreted to mean the process derived from P by replacing (in a
capture-avoiding manner) each occurrence of $x$ in $S$ by $y$. For example,

\begin{mathpar}
  P\{ \quotep{\procn{x}|\procn{x}} / x : x \in \freenames{P} \}
\end{mathpar}

will replace each (occurrence) of a free name $x$ in $P$ by
$\quotep{\procn{x}|\procn{x}}$.

Also, we will avail ourselves of the notation $x^{L}$ and $x^{R}$ to
denote injections of a name into disjoint copies of the name
space. There are numerous ways to accomplish this. One example can be
found in \cite{MeredithR05}. This notation overloads to vectors of
names: $\vec{x}^{\pi} := (x_{i}^{\pi} \; : \; 0 \leq i < |\vec{x}| )$ where $\pi \in \{L,R\}$.

We also use $P^{\Box} := P|\Box$.

In \cite{MeredithR05} an interpretation of the new operator is
given. It turns out that there are several possible interpretations
all enjoying the requisite algebraic properties of the operator (see
\cite{milner91polyadicpi}). We will therefore make liberal use of
$(\nu\; \vec{x})P$.

% subsection the_syntax_and_semantics_of_the_notation_system (end)   

\input{qm2pi.qmops} 

\input{qm2pi.sterngerlach} 

\input{qm2pi.metric} 

% section concurrent_process_calculi (end)

%\input{qm2pi.proofsketch}

% section proof sketch (end)

%\input{qm2pi.slviaknots} 

% section spatial logic via knots (end)

\input{qm2pi.conclusion}

% section conclusion (end)

%\input{qm2pi.dtcodes} 

% section wiring algorithm (end)

\input{qm2pi.ack} 

% section acknowledgments (end)

\newpage


\bibliographystyle{plain}   
\bibliography{../../biblios/main.bib}

\input{qm2pi.rhodetails}

\end{document}

 

% section acknowledgments (end)

\newpage


\bibliographystyle{plain}   
\bibliography{../../biblios/main.bib}

\documentclass[12pt]{llncs}
%\documentclass{jktr}

\usepackage[pdftex]{hyperref}                   
\usepackage {listings}
\usepackage {mathpartir}
\usepackage{bcprules}
%\usepackage{listings}
                       
\usepackage{graphicx} 
%\usepackage[margins=2.5cm,nohead,nofoot]{geometry}
%\usepackage{geometry}
\usepackage{amsfonts}
\usepackage{amstext}
\usepackage{latexsym}
\usepackage{amssymb}
\usepackage{color}


%\include{myPreamble}
\include{qm2pi.local} 

%\ifpdf
%\usepackage[pdftex]{graphicx}
%\else
%\usepackage{graphicx}
%\fi

 % \ifpdf
%  \usepackage{pdfsync}
%  \if


%\title{Brief Article}
%\author{David F. Snyder}
%\author{L.G. Meredith}

%\address{Dept. of Math., Texas State University--San Marcos, San Marcos, TX 78666}
       
\pagestyle{empty}


\begin{document}

\lstset{language=[Objective]Caml,frame=shadowbox}

\input{qm2pi.front}

% section front matter (end)

\input{qm2pi.intro} 
 
% section introduction (end)

% \input{qm2pi.knotations} 

% section notation (end)

\input{qm2pi.process.calculi} 

% section concurrent_process_calculi_and_spatial_logics_ (end)
    
%\input{qm2pi.knots2pi} 

%\input{qm2pi.trefoil} 

%\input{qm2pi.mainthm} 

% subsection basic_interpretation (end)

%\input{qm2pi.rho.presentation} 
\subsection{The syntax and semantics of the notation system}\label{sub:the_syntax_and_semantics_of_the_notation_system} % (fold)

We now summarize a technical presentation of the calculus that
embodies our theory of dynamics. The typical presentation of such a
calculus follows the style of giving generators and relations on
them. The grammar, below, describing term constructors, freely
generates the set of processes, $\Proc$. This set is then quotiented
by a relation known as structural congruence and it is over this set
that the notion of dynamics is expressed. This presentation is
essentially that of \cite{MeredithR05} with the addition of
polyadicity and summation. For readability we have relegated some of
the technical subtleties to an appendix.

\subsubsection{Process grammar}\label{subsub:process_grammar}

\begin{mathpar}
  \inferrule* [lab=synchronization] {} {{M} \bc \pzero \;|\; x?F \;|\; x!C }
  \and
  \inferrule* [lab=abstraction] {} {{F} \bc (x)P}
  \and
  \inferrule* [lab=concretion] {} {{C} \bc \langle Q \rangle}
  \and
  \inferrule* [lab=process] {} {{P,Q} \bc M \;| \;P|Q \;|\; @{x}}
  \and
  \inferrule* [lab=name] {} {{x} \bc \quotep{P}}
\end{mathpar} 

Note that $\vec{x}$ (resp. $\vec{P}$) denotes a vector of names
(resp. processes) of length $|\vec{x}|$ (resp. $|\vec{P}|$). We adopt
the following useful abbreviations.

\begin{mathpar}
   x?(\vec{y}).P := x.(\vec{y})P \and  x\clift{\vec{P}} := x.\clift{\vec{P}}
   \and x!(y) := \lift{x}{\dropn{y}}
   \and \Pi_{i=0}^{n-1}P_i := P_0 | \ldots | P_{n-1}
\end{mathpar}

\subsubsection{Structural congruence}

\paragraph{Free and bound names and alpha-equivalence.} At the
core of structural equivalence is alpha-equivalence which identifies
process that are the same up to a change of variable. Formally, we
recognize the distinction between free and bound names. The free names
of a process, $\freenames{P}$, may be calculated recursively as
follows:

\begin{mathpar}
\freenames{\pzero} := \emptyset
  \and \\
  \freenames{x?(y).P} := \{ x \} \cup (\freenames{P} \setminus \{ y \})
  \and 
  \freenames{x!\langle P \rangle} := \{ x \} \cup \{ P \} 
  \and \\
  \freenames{P|Q} := \freenames{P} \cup \freenames{Q}
  \and \\
  \freenames{@{x}} := \{ x \}
\end{mathpar}

$\pi$
$\quotep{\pi}$

$\freenames{-} : \pi \to \mathcal{P}(\quotep{\pi})$

\begin{eqnarray*}
  \freenames{\pzero} & := & \emptyset \\
  \freenames{x?(y).P} & := & \{ x \} \cup (\freenames{P} \setminus \{ y \}) \\
  \freenames{x!\langle P \rangle} & := & \{ x \} \cup \{ P \} \\
  \freenames{P|Q} & := & \freenames{P} \cup \freenames{Q} \\
  \freenames{\dropn{x}} & := & \{ x \}
\end{eqnarray*}

The bound names of a process, $\boundnames{P}$, are those names occurring in $P$
that are not free. For example, in $x?(y).0$, the name $x$ is free, while $y$ is bound.

\begin{mathpar}
  \inferrule* [lab=monoidal-laws] {} { P|Q \equiv Q|P \and P|0 \equiv P \and P|(Q|R) \equiv (P|Q)|R }
\end{mathpar}

\begin{mathpar}
  \inferrule* [lab=alpha-equivalence] {} { (x)P \equiv (y)P\{y/x\} \and y \not\in \freenames{P} }
\end{mathpar}

\begin{definition}
Then two processes, $P,Q$, are alpha-equivalent if $P = Q\{\vec{y}/\vec{x}\}$ for
some $\vec{x} \in \boundnames{Q},\vec{y} \in \boundnames{P}$, where $Q\{\vec{y}/\vec{x}\}$
denotes the capture-avoiding substitution of $\vec{y}$ for $\vec{x}$ in $Q$.
\end{definition}

\begin{definition}
  The {\em structural congruence} \cite{SangiorgiWalker} , $\equiv$,
  between processes is the least congruence containing
  alpha-equivalence, satisfying the abelian monoid laws
  (associativity, commutativity and $\pzero$ as identity) for parallel
  composition $|$ and for summation $+$.
\end{definition}

\subsection{Name equivalence}

We take name equivalence, written $\nameeq$, to be the smallest
equivalence relation generated by the following rules.

\begin{mathpar}
\inferrule*[lab=Quote-drop]
{ }
{ \quotep{@{x}} \nameeq x }

\inferrule*[lab=Struct-equiv]
{ P \scong Q }
{ \quotep{P} \nameeq \quotep{Q} }
\end{mathpar}

The astute reader will have noticed that the mutual recursion of names
and processes imposes a mutual recursion on alpha-equivalence and
structural equivalence via name-equivalence. Fortunately, all of this
works out pleasantly and we may calculate in the natural way, free of
concern. The reader interested in the details is referred to the
appendix \ref{appendix:rho_details}.

\subsection{Substitution}

We use $\Proc$ for the set of processes, $\QProc$ for the set of
names, and $\id{\{}\vec{y} / \vec{x} \id{\}}$ to denote partial maps,
$s : \QProc \rightarrow \QProc$. A map, $s$ lifts, uniquely, to a map
on process terms, $\widehat{s} : \Proc \rightarrow \Proc$ by the
following equations.

\begin{mathpar}
  (0) \psubstp{Q}{P} := 0 \\
  (R \juxtap S) \psubstp{Q}{P}
  :=    
  (R)\psubstp{Q}{P} \juxtap (S) \psubstp{Q}{P} \\
  (x?(y).R) \psubstp{Q}{P}    
  :=    
  (x)\substp{Q}{P} (z)\concat( (R \psubstn{z}{y}) \psubstp{Q}{P} ) \\
  (\lift{x}{R}) \psubstp{Q}{P}  
  :=
  \lift{(x)\substp{Q}{P}}{ R \psubstp{Q}{P} } \\
%   (\dropn{x})  \psubstp{Q}{P}       
%   := 
%   \left\{ 
%     \begin{array}{ccc} 
%       \dropn{\quotep{Q}} & & x \nameeq \quotep{P} \\
%       \dropn{x} & & otherwise \\
%     \end{array}
%   \right. 
  (\dropn{x})  \psubstp{Q}{P}       
  := 
  \left\{ 
    \begin{array}{ccc} 
      Q & & x \nameeq \quotep{P} \\
      \dropn{x} & & otherwise \\
    \end{array}
  \right.
\end{mathpar}
 

where

\begin{eqnarray}
  (x)\id{\{} \lpquote Q \rpquote / \lpquote P \rpquote \id{\}}            = 
  \left\{ 
    \begin{array}{ccc}
      \lpquote Q \rpquote & & x \nameeq \lpquote P \rpquote \\
      x & & otherwise \\
    \end{array}
  \right. \nonumber
\end{eqnarray}

and $z$ is chosen distinct from $\quotep{P}$, $\quotep{Q}$, the free
names in $Q$, and all the names in $R$. Our $\alpha$-equivalence will
be built in the standard way from this substitution.

\begin{remark}\label{rem:no_self_referential_names}
  One consequence of these definitions is that $\forall P. \quotep{P}
  \not\in \freenames{P}$.
\end{remark}

\subsection{ Dynamic quote: an example }

Anticipating something of what's to come, consider applying the
substitution, $\widehat{\id{\{}u / z \id{\}}}$, to the following pair
of processes, $\lift{w}{y!(z)}$ and $w[ \lpquote y!(z) \rpquote ]$.

\begin{eqnarray}
	\lift{w}{y!(z)}\widehat{\id{\{}u / z \id{\}}}
		& = &
		\lift{w}{y!(u)} \nonumber\\
	w[ \lpquote y!(z) \rpquote ] \widehat{ \id{\{}u / z \id{\}} }
		& = &
		w[ \lpquote y!(z) \rpquote ] \nonumber
\end{eqnarray}

Because the body of the process between quotes is impervious to
substitution, we get radically different answers. In fact, by
examining the first process in an input context,
e.g. $x?(z).\lift{w}{y!(z)}$, we see that the process under the lift
operator may be shaped by prefixed inputs binding a name inside it. In
this sense, the lift operator will be seen as a way to dynamically
construct processes before reifying them as names.

Finally equipped with these standard features we can present the
dynamics of the calculus.

\subsubsection{Operational semantics} 

Finally, we introduce the computational dynamics. What marks these
algebras as distinct from other more traditionally studied algebraic
structures, e.g. vector spaces or polynomial rings, is the manner in
which dynamics is captured. In traditional structures, dynamics is typically
expressed through morphisms between such structures, as in linear maps
between vector spaces or morphisms between rings. In algebras
associated with the semantics of computation, the dynamics is
expressed as part of the algebraic structure itself, through a
reduction reduction relation typically denoted by $\red$. Below, we
give a recursive presentation of this relation for the calculus used
in the encoding.

$\red \subseteq \pi \times \pi$
$\red : \pi \to \mathcal{P}(\pi)$

\begin{mathpar}
  \inferrule* [lab=Comm] { \textsf{match}( x_{src}, x_{trgt} ) } { x_{trgt}?(y)P \; | \; x_{src}!\langle {Q} \rangle \red P\{\quotep{Q}/y}\} }
  \and \\
  \inferrule* [lab=Par] {{P} \red {P}'} {{{P} | {Q}} \red {{P}' | {Q}}}
  \and
  \inferrule* [lab=Equiv]{{{P} \scong {P}'} \andalso {{P}' \red {Q}'} \andalso {{Q}' \scong {Q}}}{{P} \red {Q}}
\end{mathpar}

\begin{eqnarray*}
  match_{\equiv} (\quotep{P},\quotep{Q}) & := & P \equiv Q \\
  match_{\dagger}(\quotep{P},\quotep{Q}) & := & \forall R. P|Q \red^{*} R => R \red^{*} 0 \\
  match_{K}(\quotep{P},\quotep{Q}) & := & K \mbox{ for some context } K
\end{eqnarray*}

$u?(x)P | u!\langle Q \rangle \red P\{\quotep{Q}/x\}$

%We write $\wred$ for $\red^*$, and $P\red$ if $\exists Q $ such that $ P \red Q$.
We write $P\red$ if $\exists Q $ such that $ P \red Q$ and $P\not\red$, otherwise.

\section{Replication}

As mentioned before, it is known that replication (and hence
recursion) can be implemented in a higher-order process algebra
\cite{SangiorgiWalker}. As our first example of calculation with the
machinery thus far presented we give the construction explicitly in
the {\rhoc}.

\begin{eqnarray}
	D_{x} & := & \prefix{x}{y}{(\binpar{\outputp{x}{y}}{@{y}})} \nonumber\\
	\bangp_{x}{P} & := & \binpar{{x}!\langle{\binpar{D_{x}}{P}}\rangle}{D_{x}} \nonumber
\end{eqnarray}

\begin{eqnarray}
	\bangp_{x}{P} & & \nonumber\\
	=
	& {x}!\langle{(\prefix{x}{y}{(\outputp{x}{y} | @{y})) | P}}\rangle 
	      | \prefix{x}{y}{(\outputp{x}{y} | @{y})} & \nonumber\\
	\red
	& (\outputp{x}{y} | @{y})\substn{\quotep{(\prefix{x}{y}{(@{y} | \outputp{x}{y})) | P}}}{y} & \nonumber\\
	=
	& \outputp{x}{\quotep{(\prefix{x}{y}{(\outputp{x}{y} | @{y})) | P}}}
	  | {(\prefix{x}{y}{(\outputp{x}{y} | @{y})) | P}} & \nonumber\\
	\red
	& \ldots & \nonumber\\
	\red^*
	& P | P | \ldots & \nonumber
\end{eqnarray}

Of course, this encoding, as an implementation, runs away, unfolding
$\bangp{P}$ eagerly. A lazier and more implementable replication
operator, restricted to input-guarded processes, may be obtained as follows.

\begin{eqnarray}
\bangp{\prefix{u}{v}{P}} 
	:= 
	\binpar{\lift{x}{\prefix{u}{v}{(\binpar{D(x)}{P})}}}{D(x)} \nonumber
\end{eqnarray}

\begin{remark}
  Note that the lazier definition still does not deal with summation
  or mixed summation (i.e. sums over input and output). The reader is
  invited to construct definitions of replication that deal with these
  features. 

  Further, the definitions are parameterized in a name, $x$. Can you,
  gentle reader, make a definition that eliminates this parameter and
  guarantees no accidental interaction between the replication
  machinery and the process being replicated -- i.e. no accidental
  sharing of names used by the process to get its work done and the
  name(s) used by the replication to effect copying. This latter
  revision of the definition of replication is crucial to obtaining
  the expected identity $!!P \sim !P$.
\end{remark}

\begin{remark}\label{rem:paradoxical_combinator}
  The reader familiar with the lambda calculus will have noticed the
  similarity between $D$ and the paradoxical combinator.

  [Ed. note: the existence of this seems to suggest we have to be more
  restrictive on the set of processes and names we admit if we are to
  support no-cloning.]
\end{remark}

\subsubsection{Bisimulation}

The computational dynamics gives rise to another kind of equivalence,
the equivalence of computational behavior. As previously mentioned
this is typically captured \emph{via} some form of bisimulation.

% The notion we use in this paper is weak barbed bisimulation
% \cite{milner91polyadicpi}.

The notion we use in this paper is derived from weak barbed
bisimulation \cite{milner91polyadicpi}. 

\begin{definition}
An \emph{observation relation}, $\downarrow_{\mathcal N}$, over a set
of names, $\mathcal N$, is the smallest relation satisfying the rules
below.

\infrule[Out-barb]{y \in {\mathcal N}, \; x \nameeq y}
		  {\outputp{x}{v} \downarrow_{\mathcal N} x}
\infrule[Par-barb]{\mbox{$P\downarrow_{\mathcal N} x$ or $Q\downarrow_{\mathcal N} x$}}
		  {\binpar{P}{Q} \downarrow_{\mathcal N} x}

We write $P \Downarrow_{\mathcal N} x$ if there is $Q$ such that 
$P \wred Q$ and $Q \downarrow_{\mathcal N} x$.
\end{definition}

\begin{definition}
%\label{def.bbisim}
An  ${\mathcal N}$-\emph{barbed bisimulation} over a set of names, ${\mathcal N}$, is a symmetric binary relation 
${\mathcal S}_{\mathcal N}$ between agents such that $P\rel{S}_{\mathcal N}Q$ implies:
\begin{enumerate}
\item If $P \red P'$ then $Q \wred Q'$ and $P'\rel{S}_{\mathcal N} Q'$.
\item If $P\downarrow_{\mathcal N} x$, then $Q\Downarrow_{\mathcal N} x$.
\end{enumerate}
$P$ is ${\mathcal N}$-barbed bisimilar to $Q$, written
$P \wbbisim_{\mathcal N} Q$, if $P \rel{S}_{\mathcal N} Q$ for some ${\mathcal N}$-barbed bisimulation ${\mathcal S}_{\mathcal N}$.
\end{definition}

$\mathcal{R} \subseteq \pi \times \pi$

$P \mathcal{R} Q => \forall P'. P \red P' \Rightarrow \exists Q'. Q \red Q', P' \mathcal{R} Q'$

$P \vdash x \Rightarrow Q \vdash x$

\begin{mathpar}
  \inferrule*[lab=Out-barb]{x \nameeq y}{{y}!\langle{Q}\rangle \vdash x}
  \and
  \inferrule*[lab=Par-barb]{\mbox{$P\vdash x$ or $Q\vdash x$}}{\binpar{P}{Q} \vdash x}
\end{mathpar}

\subsubsection{Contexts}

One of the principle advantages of computational calculi like the
$\pi$-calculus is a well-defined notion of context,
contextual-equivalence and a correlation between
contextual-equivalence and notions of bisimulation. The notion of
context allows the decomposition of a process into (sub-)process and
its syntactic environment, its context. Thus, a context may be
thought of as a process with a ``hole'' (written $\Box$) in it. The
application of a context $M$ to a process $P$, written $M[P]$, is
tantamount to filling the hole in $M$ with $P$. In this paper we do
not need the full weight of this theory, but do make use of the notion
of context in the proof the main theorem. 

\begin{mathpar}
  \inferrule* [lab=summation] {} {{M_{M},M_{N}} \bc \Box \;|\; x.M_{A} \;|\; M_{M}+M_{N}}
  \and
  \inferrule* [lab=agent] {} {{M_{A}} \bc (\vec{x})M_{P} \;| \; \clift{P_0,\ldots,M_{P},\ldots,P_N}}
  \and \\
  \inferrule* [lab=process] {} {{M_{P}} \bc M_{N} \;| \;P|M_{P} }
\end{mathpar} 

\begin{mathpar}
  \inferrule* [lab=sychronization] {} {M_{N} \bc \Box \;|\; x?M_{F} \;|\; x!M_{C}}
  \and
  \inferrule* [lab=abstraction] {} {{M_{F}} \bc (x)M_{P} }
  \and
  \inferrule* [lab=concretion] {} {{M_{C}} \bc \langle M_{P} \rangle }
  \and \\
  \inferrule* [lab=process] {} {{M_{P}} \bc M_{N} \;| \;P|M_{P} }
\end{mathpar}

\begin{definition}[contextual application] Given a context $M$, and
  process $P$, we define the \emph{contextual application}, $M[P] :=
  M\{P/\Box\}$. That is, the contextual application of M to P is the
  substitution of $P$ for $\Box$ in $M$.
\end{definition}

$\meaningof{-} : L \to \mathcal{P}(\pi)$

\begin{mathpar}
  \inferrule* [lab=collection] {} {\meaningof{true} = \pi, \and \meaningof{~E} = \pi \setminus \meaningof{E}, \and \meaningof{E_{1} \& E_{2}} = \meaningof{E_{1}} \cap \meaningof{E_{2}}}
\end{mathpar}

\begin{mathpar}
  \inferrule* [lab=structure] {} {\meaningof{0} = \{ P \in \pi | P \equiv 0 \}, \and \\ \meaningof{E_1 | E_2} = \{ P \in \pi | P \equiv P_{1} | P_{2}, P_{1} \in \meaningof{E_{1}}, P_{2} \in \meaningof{E_2}\} }
\end{mathpar}

\begin{mathpar}
 \inferrule* [lab=behavior] {} {\meaningof{\langle a?b \rangle E} = \{ P \in \pi | P \equiv Q | u?(y)P', \\ \and \\\\ \and \\ \;\;\; u \in \meaningof{a}, \forall z.P'\{z/y\} \in \meaningof{E\{z/b\}}\}, \and \\ \meaningof{a!E} = \{ P \in \pi | P \equiv Q | x!\langle P' \rangle, x \in \meaningof{a} P' \in \meaningof{E}\} }
\end{mathpar}

\begin{mathpar}
 \inferrule* [lab=nominal] {} {\meaningof{\quotep{E}} = \{ \quotep{P} \in \quotep{\pi} | P \in \meaningof{E} \}, \and \meaningof{\quotep{P}} = \{ \quotep{Q} \in \quotep{\pi} | P \equiv Q \} \and \\ \meaningof{@\quotep{E}} = \{ P \in \pi | P \equiv @x, x \in \meaningof{E} \}}
\end{mathpar}

\begin{eqnarray*}
  \\
  \meaningof{-} : TS \to ST
\end{eqnarray*}

\begin{eqnarray*}
  \\
  L : TS \to ST
\end{eqnarray*}

\begin{eqnarray*}
  \\
  P \models E \iff P \in \meaningof{E}
\end{eqnarray*}

\begin{eqnarray*}
  P \approx_{L} Q \iff \forall E \in L. P \models E \iff Q \models E
\end{eqnarray*}

\begin{eqnarray*}
  P \approx_{K} Q
\end{eqnarray*}

\begin{eqnarray*}
  P \approx Q
\end{eqnarray*}

$\approx_{K} = \approx = \approx_{L}$

\subsubsection{Contextual duality}

Note that contexts extend the quotation operation to a family of
operations from processes to names. Given a context, $M$, we can
define a \emph{nominal context}, $\quotep{M}$ by $\quotep{M}[P] :=
\quotep{M[P]}$. To foreshadow what is to come we observe that these
operations enjoy a duality with processes very much like the duality
between vectors and maps from vectors to scalars.

Further, because the calculus is essentially higher-order, we have a
correspondence between contexts and processes. More specifically,
given a name $x$ and a context $M$ we can construct $M^{*}_{x}$ such
that 

\begin{mathpar}
  M^{*}_{x} | \lift{x}{P} \red M[P]
\end{mathpar}

namely,

\begin{mathpar}
  M^{*}_{x} := x?(u).M[\dropn{u}]
\end{mathpar}

The dependence of $M^{*}_{x}$ on a name makes it an abstraction, 

\begin{mathpar}
  M^{*} := (x)x?(u).M[\dropn{u}]
\end{mathpar}

\subsection{Additional notation}

It will sometimes be convenient to denote the process a name
quotes. We already have the notation $x = \quotep{P}$, but it will be
convenient to introduce an alternate notation, $\procn{x}$, when we
want to emphasize the connection to the use of the name. Note that, by
virtue of name equivalence, $\quotep{\procn{x}} \nameeq x$; so, the
notation is consistent with previous definitions.

Further, because names have structure it is possible to effect
substitutions on the basis of that structure. This means we need to
upgrade our notation for substitutions, which we accomplish by
adapting comprehension notation. Thus,

\begin{mathpar}
  P\{ y / x : x \in S \}
\end{mathpar}

is interpreted to mean the process derived from P by replacing (in a
capture-avoiding manner) each occurrence of $x$ in $S$ by $y$. For example,

\begin{mathpar}
  P\{ \quotep{\procn{x}|\procn{x}} / x : x \in \freenames{P} \}
\end{mathpar}

will replace each (occurrence) of a free name $x$ in $P$ by
$\quotep{\procn{x}|\procn{x}}$.

Also, we will avail ourselves of the notation $x^{L}$ and $x^{R}$ to
denote injections of a name into disjoint copies of the name
space. There are numerous ways to accomplish this. One example can be
found in \cite{MeredithR05}. This notation overloads to vectors of
names: $\vec{x}^{\pi} := (x_{i}^{\pi} \; : \; 0 \leq i < |\vec{x}| )$ where $\pi \in \{L,R\}$.

We also use $P^{\Box} := P|\Box$.

In \cite{MeredithR05} an interpretation of the new operator is
given. It turns out that there are several possible interpretations
all enjoying the requisite algebraic properties of the operator (see
\cite{milner91polyadicpi}). We will therefore make liberal use of
$(\nu\; \vec{x})P$.

% subsection the_syntax_and_semantics_of_the_notation_system (end)   

\input{qm2pi.qmops} 

\input{qm2pi.sterngerlach} 

\input{qm2pi.metric} 

% section concurrent_process_calculi (end)

%\input{qm2pi.proofsketch}

% section proof sketch (end)

%\input{qm2pi.slviaknots} 

% section spatial logic via knots (end)

\input{qm2pi.conclusion}

% section conclusion (end)

%\input{qm2pi.dtcodes} 

% section wiring algorithm (end)

\input{qm2pi.ack} 

% section acknowledgments (end)

\newpage


\bibliographystyle{plain}   
\bibliography{../../biblios/main.bib}

\input{qm2pi.rhodetails}

\end{document}



\end{document}

 

\documentclass[12pt]{llncs}
%\documentclass{jktr}

\usepackage[pdftex]{hyperref}                   
\usepackage {listings}
\usepackage {mathpartir}
\usepackage{bcprules}
%\usepackage{listings}
                       
\usepackage{graphicx} 
%\usepackage[margins=2.5cm,nohead,nofoot]{geometry}
%\usepackage{geometry}
\usepackage{amsfonts}
\usepackage{amstext}
\usepackage{latexsym}
\usepackage{amssymb}
\usepackage{color}


%\include{myPreamble}
\documentclass[12pt]{llncs}
%\documentclass{jktr}

\usepackage[pdftex]{hyperref}                   
\usepackage {listings}
\usepackage {mathpartir}
\usepackage{bcprules}
%\usepackage{listings}
                       
\usepackage{graphicx} 
%\usepackage[margins=2.5cm,nohead,nofoot]{geometry}
%\usepackage{geometry}
\usepackage{amsfonts}
\usepackage{amstext}
\usepackage{latexsym}
\usepackage{amssymb}
\usepackage{color}


%\include{myPreamble}
\include{qm2pi.local} 

%\ifpdf
%\usepackage[pdftex]{graphicx}
%\else
%\usepackage{graphicx}
%\fi

 % \ifpdf
%  \usepackage{pdfsync}
%  \if


%\title{Brief Article}
%\author{David F. Snyder}
%\author{L.G. Meredith}

%\address{Dept. of Math., Texas State University--San Marcos, San Marcos, TX 78666}
       
\pagestyle{empty}


\begin{document}

\lstset{language=[Objective]Caml,frame=shadowbox}

\input{qm2pi.front}

% section front matter (end)

\input{qm2pi.intro} 
 
% section introduction (end)

% \input{qm2pi.knotations} 

% section notation (end)

\input{qm2pi.process.calculi} 

% section concurrent_process_calculi_and_spatial_logics_ (end)
    
%\input{qm2pi.knots2pi} 

%\input{qm2pi.trefoil} 

%\input{qm2pi.mainthm} 

% subsection basic_interpretation (end)

%\input{qm2pi.rho.presentation} 
\subsection{The syntax and semantics of the notation system}\label{sub:the_syntax_and_semantics_of_the_notation_system} % (fold)

We now summarize a technical presentation of the calculus that
embodies our theory of dynamics. The typical presentation of such a
calculus follows the style of giving generators and relations on
them. The grammar, below, describing term constructors, freely
generates the set of processes, $\Proc$. This set is then quotiented
by a relation known as structural congruence and it is over this set
that the notion of dynamics is expressed. This presentation is
essentially that of \cite{MeredithR05} with the addition of
polyadicity and summation. For readability we have relegated some of
the technical subtleties to an appendix.

\subsubsection{Process grammar}\label{subsub:process_grammar}

\begin{mathpar}
  \inferrule* [lab=synchronization] {} {{M} \bc \pzero \;|\; x?F \;|\; x!C }
  \and
  \inferrule* [lab=abstraction] {} {{F} \bc (x)P}
  \and
  \inferrule* [lab=concretion] {} {{C} \bc \langle Q \rangle}
  \and
  \inferrule* [lab=process] {} {{P,Q} \bc M \;| \;P|Q \;|\; @{x}}
  \and
  \inferrule* [lab=name] {} {{x} \bc \quotep{P}}
\end{mathpar} 

Note that $\vec{x}$ (resp. $\vec{P}$) denotes a vector of names
(resp. processes) of length $|\vec{x}|$ (resp. $|\vec{P}|$). We adopt
the following useful abbreviations.

\begin{mathpar}
   x?(\vec{y}).P := x.(\vec{y})P \and  x\clift{\vec{P}} := x.\clift{\vec{P}}
   \and x!(y) := \lift{x}{\dropn{y}}
   \and \Pi_{i=0}^{n-1}P_i := P_0 | \ldots | P_{n-1}
\end{mathpar}

\subsubsection{Structural congruence}

\paragraph{Free and bound names and alpha-equivalence.} At the
core of structural equivalence is alpha-equivalence which identifies
process that are the same up to a change of variable. Formally, we
recognize the distinction between free and bound names. The free names
of a process, $\freenames{P}$, may be calculated recursively as
follows:

\begin{mathpar}
\freenames{\pzero} := \emptyset
  \and \\
  \freenames{x?(y).P} := \{ x \} \cup (\freenames{P} \setminus \{ y \})
  \and 
  \freenames{x!\langle P \rangle} := \{ x \} \cup \{ P \} 
  \and \\
  \freenames{P|Q} := \freenames{P} \cup \freenames{Q}
  \and \\
  \freenames{@{x}} := \{ x \}
\end{mathpar}

$\pi$
$\quotep{\pi}$

$\freenames{-} : \pi \to \mathcal{P}(\quotep{\pi})$

\begin{eqnarray*}
  \freenames{\pzero} & := & \emptyset \\
  \freenames{x?(y).P} & := & \{ x \} \cup (\freenames{P} \setminus \{ y \}) \\
  \freenames{x!\langle P \rangle} & := & \{ x \} \cup \{ P \} \\
  \freenames{P|Q} & := & \freenames{P} \cup \freenames{Q} \\
  \freenames{\dropn{x}} & := & \{ x \}
\end{eqnarray*}

The bound names of a process, $\boundnames{P}$, are those names occurring in $P$
that are not free. For example, in $x?(y).0$, the name $x$ is free, while $y$ is bound.

\begin{mathpar}
  \inferrule* [lab=monoidal-laws] {} { P|Q \equiv Q|P \and P|0 \equiv P \and P|(Q|R) \equiv (P|Q)|R }
\end{mathpar}

\begin{mathpar}
  \inferrule* [lab=alpha-equivalence] {} { (x)P \equiv (y)P\{y/x\} \and y \not\in \freenames{P} }
\end{mathpar}

\begin{definition}
Then two processes, $P,Q$, are alpha-equivalent if $P = Q\{\vec{y}/\vec{x}\}$ for
some $\vec{x} \in \boundnames{Q},\vec{y} \in \boundnames{P}$, where $Q\{\vec{y}/\vec{x}\}$
denotes the capture-avoiding substitution of $\vec{y}$ for $\vec{x}$ in $Q$.
\end{definition}

\begin{definition}
  The {\em structural congruence} \cite{SangiorgiWalker} , $\equiv$,
  between processes is the least congruence containing
  alpha-equivalence, satisfying the abelian monoid laws
  (associativity, commutativity and $\pzero$ as identity) for parallel
  composition $|$ and for summation $+$.
\end{definition}

\subsection{Name equivalence}

We take name equivalence, written $\nameeq$, to be the smallest
equivalence relation generated by the following rules.

\begin{mathpar}
\inferrule*[lab=Quote-drop]
{ }
{ \quotep{@{x}} \nameeq x }

\inferrule*[lab=Struct-equiv]
{ P \scong Q }
{ \quotep{P} \nameeq \quotep{Q} }
\end{mathpar}

The astute reader will have noticed that the mutual recursion of names
and processes imposes a mutual recursion on alpha-equivalence and
structural equivalence via name-equivalence. Fortunately, all of this
works out pleasantly and we may calculate in the natural way, free of
concern. The reader interested in the details is referred to the
appendix \ref{appendix:rho_details}.

\subsection{Substitution}

We use $\Proc$ for the set of processes, $\QProc$ for the set of
names, and $\id{\{}\vec{y} / \vec{x} \id{\}}$ to denote partial maps,
$s : \QProc \rightarrow \QProc$. A map, $s$ lifts, uniquely, to a map
on process terms, $\widehat{s} : \Proc \rightarrow \Proc$ by the
following equations.

\begin{mathpar}
  (0) \psubstp{Q}{P} := 0 \\
  (R \juxtap S) \psubstp{Q}{P}
  :=    
  (R)\psubstp{Q}{P} \juxtap (S) \psubstp{Q}{P} \\
  (x?(y).R) \psubstp{Q}{P}    
  :=    
  (x)\substp{Q}{P} (z)\concat( (R \psubstn{z}{y}) \psubstp{Q}{P} ) \\
  (\lift{x}{R}) \psubstp{Q}{P}  
  :=
  \lift{(x)\substp{Q}{P}}{ R \psubstp{Q}{P} } \\
%   (\dropn{x})  \psubstp{Q}{P}       
%   := 
%   \left\{ 
%     \begin{array}{ccc} 
%       \dropn{\quotep{Q}} & & x \nameeq \quotep{P} \\
%       \dropn{x} & & otherwise \\
%     \end{array}
%   \right. 
  (\dropn{x})  \psubstp{Q}{P}       
  := 
  \left\{ 
    \begin{array}{ccc} 
      Q & & x \nameeq \quotep{P} \\
      \dropn{x} & & otherwise \\
    \end{array}
  \right.
\end{mathpar}
 

where

\begin{eqnarray}
  (x)\id{\{} \lpquote Q \rpquote / \lpquote P \rpquote \id{\}}            = 
  \left\{ 
    \begin{array}{ccc}
      \lpquote Q \rpquote & & x \nameeq \lpquote P \rpquote \\
      x & & otherwise \\
    \end{array}
  \right. \nonumber
\end{eqnarray}

and $z$ is chosen distinct from $\quotep{P}$, $\quotep{Q}$, the free
names in $Q$, and all the names in $R$. Our $\alpha$-equivalence will
be built in the standard way from this substitution.

\begin{remark}\label{rem:no_self_referential_names}
  One consequence of these definitions is that $\forall P. \quotep{P}
  \not\in \freenames{P}$.
\end{remark}

\subsection{ Dynamic quote: an example }

Anticipating something of what's to come, consider applying the
substitution, $\widehat{\id{\{}u / z \id{\}}}$, to the following pair
of processes, $\lift{w}{y!(z)}$ and $w[ \lpquote y!(z) \rpquote ]$.

\begin{eqnarray}
	\lift{w}{y!(z)}\widehat{\id{\{}u / z \id{\}}}
		& = &
		\lift{w}{y!(u)} \nonumber\\
	w[ \lpquote y!(z) \rpquote ] \widehat{ \id{\{}u / z \id{\}} }
		& = &
		w[ \lpquote y!(z) \rpquote ] \nonumber
\end{eqnarray}

Because the body of the process between quotes is impervious to
substitution, we get radically different answers. In fact, by
examining the first process in an input context,
e.g. $x?(z).\lift{w}{y!(z)}$, we see that the process under the lift
operator may be shaped by prefixed inputs binding a name inside it. In
this sense, the lift operator will be seen as a way to dynamically
construct processes before reifying them as names.

Finally equipped with these standard features we can present the
dynamics of the calculus.

\subsubsection{Operational semantics} 

Finally, we introduce the computational dynamics. What marks these
algebras as distinct from other more traditionally studied algebraic
structures, e.g. vector spaces or polynomial rings, is the manner in
which dynamics is captured. In traditional structures, dynamics is typically
expressed through morphisms between such structures, as in linear maps
between vector spaces or morphisms between rings. In algebras
associated with the semantics of computation, the dynamics is
expressed as part of the algebraic structure itself, through a
reduction reduction relation typically denoted by $\red$. Below, we
give a recursive presentation of this relation for the calculus used
in the encoding.

$\red \subseteq \pi \times \pi$
$\red : \pi \to \mathcal{P}(\pi)$

\begin{mathpar}
  \inferrule* [lab=Comm] { \textsf{match}( x_{src}, x_{trgt} ) } { x_{trgt}?(y)P \; | \; x_{src}!\langle {Q} \rangle \red P\{\quotep{Q}/y}\} }
  \and \\
  \inferrule* [lab=Par] {{P} \red {P}'} {{{P} | {Q}} \red {{P}' | {Q}}}
  \and
  \inferrule* [lab=Equiv]{{{P} \scong {P}'} \andalso {{P}' \red {Q}'} \andalso {{Q}' \scong {Q}}}{{P} \red {Q}}
\end{mathpar}

\begin{eqnarray*}
  match_{\equiv} (\quotep{P},\quotep{Q}) & := & P \equiv Q \\
  match_{\dagger}(\quotep{P},\quotep{Q}) & := & \forall R. P|Q \red^{*} R => R \red^{*} 0 \\
  match_{K}(\quotep{P},\quotep{Q}) & := & K \mbox{ for some context } K
\end{eqnarray*}

$u?(x)P | u!\langle Q \rangle \red P\{\quotep{Q}/x\}$

%We write $\wred$ for $\red^*$, and $P\red$ if $\exists Q $ such that $ P \red Q$.
We write $P\red$ if $\exists Q $ such that $ P \red Q$ and $P\not\red$, otherwise.

\section{Replication}

As mentioned before, it is known that replication (and hence
recursion) can be implemented in a higher-order process algebra
\cite{SangiorgiWalker}. As our first example of calculation with the
machinery thus far presented we give the construction explicitly in
the {\rhoc}.

\begin{eqnarray}
	D_{x} & := & \prefix{x}{y}{(\binpar{\outputp{x}{y}}{@{y}})} \nonumber\\
	\bangp_{x}{P} & := & \binpar{{x}!\langle{\binpar{D_{x}}{P}}\rangle}{D_{x}} \nonumber
\end{eqnarray}

\begin{eqnarray}
	\bangp_{x}{P} & & \nonumber\\
	=
	& {x}!\langle{(\prefix{x}{y}{(\outputp{x}{y} | @{y})) | P}}\rangle 
	      | \prefix{x}{y}{(\outputp{x}{y} | @{y})} & \nonumber\\
	\red
	& (\outputp{x}{y} | @{y})\substn{\quotep{(\prefix{x}{y}{(@{y} | \outputp{x}{y})) | P}}}{y} & \nonumber\\
	=
	& \outputp{x}{\quotep{(\prefix{x}{y}{(\outputp{x}{y} | @{y})) | P}}}
	  | {(\prefix{x}{y}{(\outputp{x}{y} | @{y})) | P}} & \nonumber\\
	\red
	& \ldots & \nonumber\\
	\red^*
	& P | P | \ldots & \nonumber
\end{eqnarray}

Of course, this encoding, as an implementation, runs away, unfolding
$\bangp{P}$ eagerly. A lazier and more implementable replication
operator, restricted to input-guarded processes, may be obtained as follows.

\begin{eqnarray}
\bangp{\prefix{u}{v}{P}} 
	:= 
	\binpar{\lift{x}{\prefix{u}{v}{(\binpar{D(x)}{P})}}}{D(x)} \nonumber
\end{eqnarray}

\begin{remark}
  Note that the lazier definition still does not deal with summation
  or mixed summation (i.e. sums over input and output). The reader is
  invited to construct definitions of replication that deal with these
  features. 

  Further, the definitions are parameterized in a name, $x$. Can you,
  gentle reader, make a definition that eliminates this parameter and
  guarantees no accidental interaction between the replication
  machinery and the process being replicated -- i.e. no accidental
  sharing of names used by the process to get its work done and the
  name(s) used by the replication to effect copying. This latter
  revision of the definition of replication is crucial to obtaining
  the expected identity $!!P \sim !P$.
\end{remark}

\begin{remark}\label{rem:paradoxical_combinator}
  The reader familiar with the lambda calculus will have noticed the
  similarity between $D$ and the paradoxical combinator.

  [Ed. note: the existence of this seems to suggest we have to be more
  restrictive on the set of processes and names we admit if we are to
  support no-cloning.]
\end{remark}

\subsubsection{Bisimulation}

The computational dynamics gives rise to another kind of equivalence,
the equivalence of computational behavior. As previously mentioned
this is typically captured \emph{via} some form of bisimulation.

% The notion we use in this paper is weak barbed bisimulation
% \cite{milner91polyadicpi}.

The notion we use in this paper is derived from weak barbed
bisimulation \cite{milner91polyadicpi}. 

\begin{definition}
An \emph{observation relation}, $\downarrow_{\mathcal N}$, over a set
of names, $\mathcal N$, is the smallest relation satisfying the rules
below.

\infrule[Out-barb]{y \in {\mathcal N}, \; x \nameeq y}
		  {\outputp{x}{v} \downarrow_{\mathcal N} x}
\infrule[Par-barb]{\mbox{$P\downarrow_{\mathcal N} x$ or $Q\downarrow_{\mathcal N} x$}}
		  {\binpar{P}{Q} \downarrow_{\mathcal N} x}

We write $P \Downarrow_{\mathcal N} x$ if there is $Q$ such that 
$P \wred Q$ and $Q \downarrow_{\mathcal N} x$.
\end{definition}

\begin{definition}
%\label{def.bbisim}
An  ${\mathcal N}$-\emph{barbed bisimulation} over a set of names, ${\mathcal N}$, is a symmetric binary relation 
${\mathcal S}_{\mathcal N}$ between agents such that $P\rel{S}_{\mathcal N}Q$ implies:
\begin{enumerate}
\item If $P \red P'$ then $Q \wred Q'$ and $P'\rel{S}_{\mathcal N} Q'$.
\item If $P\downarrow_{\mathcal N} x$, then $Q\Downarrow_{\mathcal N} x$.
\end{enumerate}
$P$ is ${\mathcal N}$-barbed bisimilar to $Q$, written
$P \wbbisim_{\mathcal N} Q$, if $P \rel{S}_{\mathcal N} Q$ for some ${\mathcal N}$-barbed bisimulation ${\mathcal S}_{\mathcal N}$.
\end{definition}

$\mathcal{R} \subseteq \pi \times \pi$

$P \mathcal{R} Q => \forall P'. P \red P' \Rightarrow \exists Q'. Q \red Q', P' \mathcal{R} Q'$

$P \vdash x \Rightarrow Q \vdash x$

\begin{mathpar}
  \inferrule*[lab=Out-barb]{x \nameeq y}{{y}!\langle{Q}\rangle \vdash x}
  \and
  \inferrule*[lab=Par-barb]{\mbox{$P\vdash x$ or $Q\vdash x$}}{\binpar{P}{Q} \vdash x}
\end{mathpar}

\subsubsection{Contexts}

One of the principle advantages of computational calculi like the
$\pi$-calculus is a well-defined notion of context,
contextual-equivalence and a correlation between
contextual-equivalence and notions of bisimulation. The notion of
context allows the decomposition of a process into (sub-)process and
its syntactic environment, its context. Thus, a context may be
thought of as a process with a ``hole'' (written $\Box$) in it. The
application of a context $M$ to a process $P$, written $M[P]$, is
tantamount to filling the hole in $M$ with $P$. In this paper we do
not need the full weight of this theory, but do make use of the notion
of context in the proof the main theorem. 

\begin{mathpar}
  \inferrule* [lab=summation] {} {{M_{M},M_{N}} \bc \Box \;|\; x.M_{A} \;|\; M_{M}+M_{N}}
  \and
  \inferrule* [lab=agent] {} {{M_{A}} \bc (\vec{x})M_{P} \;| \; \clift{P_0,\ldots,M_{P},\ldots,P_N}}
  \and \\
  \inferrule* [lab=process] {} {{M_{P}} \bc M_{N} \;| \;P|M_{P} }
\end{mathpar} 

\begin{mathpar}
  \inferrule* [lab=sychronization] {} {M_{N} \bc \Box \;|\; x?M_{F} \;|\; x!M_{C}}
  \and
  \inferrule* [lab=abstraction] {} {{M_{F}} \bc (x)M_{P} }
  \and
  \inferrule* [lab=concretion] {} {{M_{C}} \bc \langle M_{P} \rangle }
  \and \\
  \inferrule* [lab=process] {} {{M_{P}} \bc M_{N} \;| \;P|M_{P} }
\end{mathpar}

\begin{definition}[contextual application] Given a context $M$, and
  process $P$, we define the \emph{contextual application}, $M[P] :=
  M\{P/\Box\}$. That is, the contextual application of M to P is the
  substitution of $P$ for $\Box$ in $M$.
\end{definition}

$\meaningof{-} : L \to \mathcal{P}(\pi)$

\begin{mathpar}
  \inferrule* [lab=collection] {} {\meaningof{true} = \pi, \and \meaningof{~E} = \pi \setminus \meaningof{E}, \and \meaningof{E_{1} \& E_{2}} = \meaningof{E_{1}} \cap \meaningof{E_{2}}}
\end{mathpar}

\begin{mathpar}
  \inferrule* [lab=structure] {} {\meaningof{0} = \{ P \in \pi | P \equiv 0 \}, \and \\ \meaningof{E_1 | E_2} = \{ P \in \pi | P \equiv P_{1} | P_{2}, P_{1} \in \meaningof{E_{1}}, P_{2} \in \meaningof{E_2}\} }
\end{mathpar}

\begin{mathpar}
 \inferrule* [lab=behavior] {} {\meaningof{\langle a?b \rangle E} = \{ P \in \pi | P \equiv Q | u?(y)P', \\ \and \\\\ \and \\ \;\;\; u \in \meaningof{a}, \forall z.P'\{z/y\} \in \meaningof{E\{z/b\}}\}, \and \\ \meaningof{a!E} = \{ P \in \pi | P \equiv Q | x!\langle P' \rangle, x \in \meaningof{a} P' \in \meaningof{E}\} }
\end{mathpar}

\begin{mathpar}
 \inferrule* [lab=nominal] {} {\meaningof{\quotep{E}} = \{ \quotep{P} \in \quotep{\pi} | P \in \meaningof{E} \}, \and \meaningof{\quotep{P}} = \{ \quotep{Q} \in \quotep{\pi} | P \equiv Q \} \and \\ \meaningof{@\quotep{E}} = \{ P \in \pi | P \equiv @x, x \in \meaningof{E} \}}
\end{mathpar}

\begin{eqnarray*}
  \\
  \meaningof{-} : TS \to ST
\end{eqnarray*}

\begin{eqnarray*}
  \\
  L : TS \to ST
\end{eqnarray*}

\begin{eqnarray*}
  \\
  P \models E \iff P \in \meaningof{E}
\end{eqnarray*}

\begin{eqnarray*}
  P \approx_{L} Q \iff \forall E \in L. P \models E \iff Q \models E
\end{eqnarray*}

\begin{eqnarray*}
  P \approx_{K} Q
\end{eqnarray*}

\begin{eqnarray*}
  P \approx Q
\end{eqnarray*}

$\approx_{K} = \approx = \approx_{L}$

\subsubsection{Contextual duality}

Note that contexts extend the quotation operation to a family of
operations from processes to names. Given a context, $M$, we can
define a \emph{nominal context}, $\quotep{M}$ by $\quotep{M}[P] :=
\quotep{M[P]}$. To foreshadow what is to come we observe that these
operations enjoy a duality with processes very much like the duality
between vectors and maps from vectors to scalars.

Further, because the calculus is essentially higher-order, we have a
correspondence between contexts and processes. More specifically,
given a name $x$ and a context $M$ we can construct $M^{*}_{x}$ such
that 

\begin{mathpar}
  M^{*}_{x} | \lift{x}{P} \red M[P]
\end{mathpar}

namely,

\begin{mathpar}
  M^{*}_{x} := x?(u).M[\dropn{u}]
\end{mathpar}

The dependence of $M^{*}_{x}$ on a name makes it an abstraction, 

\begin{mathpar}
  M^{*} := (x)x?(u).M[\dropn{u}]
\end{mathpar}

\subsection{Additional notation}

It will sometimes be convenient to denote the process a name
quotes. We already have the notation $x = \quotep{P}$, but it will be
convenient to introduce an alternate notation, $\procn{x}$, when we
want to emphasize the connection to the use of the name. Note that, by
virtue of name equivalence, $\quotep{\procn{x}} \nameeq x$; so, the
notation is consistent with previous definitions.

Further, because names have structure it is possible to effect
substitutions on the basis of that structure. This means we need to
upgrade our notation for substitutions, which we accomplish by
adapting comprehension notation. Thus,

\begin{mathpar}
  P\{ y / x : x \in S \}
\end{mathpar}

is interpreted to mean the process derived from P by replacing (in a
capture-avoiding manner) each occurrence of $x$ in $S$ by $y$. For example,

\begin{mathpar}
  P\{ \quotep{\procn{x}|\procn{x}} / x : x \in \freenames{P} \}
\end{mathpar}

will replace each (occurrence) of a free name $x$ in $P$ by
$\quotep{\procn{x}|\procn{x}}$.

Also, we will avail ourselves of the notation $x^{L}$ and $x^{R}$ to
denote injections of a name into disjoint copies of the name
space. There are numerous ways to accomplish this. One example can be
found in \cite{MeredithR05}. This notation overloads to vectors of
names: $\vec{x}^{\pi} := (x_{i}^{\pi} \; : \; 0 \leq i < |\vec{x}| )$ where $\pi \in \{L,R\}$.

We also use $P^{\Box} := P|\Box$.

In \cite{MeredithR05} an interpretation of the new operator is
given. It turns out that there are several possible interpretations
all enjoying the requisite algebraic properties of the operator (see
\cite{milner91polyadicpi}). We will therefore make liberal use of
$(\nu\; \vec{x})P$.

% subsection the_syntax_and_semantics_of_the_notation_system (end)   

\input{qm2pi.qmops} 

\input{qm2pi.sterngerlach} 

\input{qm2pi.metric} 

% section concurrent_process_calculi (end)

%\input{qm2pi.proofsketch}

% section proof sketch (end)

%\input{qm2pi.slviaknots} 

% section spatial logic via knots (end)

\input{qm2pi.conclusion}

% section conclusion (end)

%\input{qm2pi.dtcodes} 

% section wiring algorithm (end)

\input{qm2pi.ack} 

% section acknowledgments (end)

\newpage


\bibliographystyle{plain}   
\bibliography{../../biblios/main.bib}

\input{qm2pi.rhodetails}

\end{document}

 

%\ifpdf
%\usepackage[pdftex]{graphicx}
%\else
%\usepackage{graphicx}
%\fi

 % \ifpdf
%  \usepackage{pdfsync}
%  \if


%\title{Brief Article}
%\author{David F. Snyder}
%\author{L.G. Meredith}

%\address{Dept. of Math., Texas State University--San Marcos, San Marcos, TX 78666}
       
\pagestyle{empty}


\begin{document}

\lstset{language=[Objective]Caml,frame=shadowbox}

\documentclass[12pt]{llncs}
%\documentclass{jktr}

\usepackage[pdftex]{hyperref}                   
\usepackage {listings}
\usepackage {mathpartir}
\usepackage{bcprules}
%\usepackage{listings}
                       
\usepackage{graphicx} 
%\usepackage[margins=2.5cm,nohead,nofoot]{geometry}
%\usepackage{geometry}
\usepackage{amsfonts}
\usepackage{amstext}
\usepackage{latexsym}
\usepackage{amssymb}
\usepackage{color}


%\include{myPreamble}
\include{qm2pi.local} 

%\ifpdf
%\usepackage[pdftex]{graphicx}
%\else
%\usepackage{graphicx}
%\fi

 % \ifpdf
%  \usepackage{pdfsync}
%  \if


%\title{Brief Article}
%\author{David F. Snyder}
%\author{L.G. Meredith}

%\address{Dept. of Math., Texas State University--San Marcos, San Marcos, TX 78666}
       
\pagestyle{empty}


\begin{document}

\lstset{language=[Objective]Caml,frame=shadowbox}

\input{qm2pi.front}

% section front matter (end)

\input{qm2pi.intro} 
 
% section introduction (end)

% \input{qm2pi.knotations} 

% section notation (end)

\input{qm2pi.process.calculi} 

% section concurrent_process_calculi_and_spatial_logics_ (end)
    
%\input{qm2pi.knots2pi} 

%\input{qm2pi.trefoil} 

%\input{qm2pi.mainthm} 

% subsection basic_interpretation (end)

%\input{qm2pi.rho.presentation} 
\subsection{The syntax and semantics of the notation system}\label{sub:the_syntax_and_semantics_of_the_notation_system} % (fold)

We now summarize a technical presentation of the calculus that
embodies our theory of dynamics. The typical presentation of such a
calculus follows the style of giving generators and relations on
them. The grammar, below, describing term constructors, freely
generates the set of processes, $\Proc$. This set is then quotiented
by a relation known as structural congruence and it is over this set
that the notion of dynamics is expressed. This presentation is
essentially that of \cite{MeredithR05} with the addition of
polyadicity and summation. For readability we have relegated some of
the technical subtleties to an appendix.

\subsubsection{Process grammar}\label{subsub:process_grammar}

\begin{mathpar}
  \inferrule* [lab=synchronization] {} {{M} \bc \pzero \;|\; x?F \;|\; x!C }
  \and
  \inferrule* [lab=abstraction] {} {{F} \bc (x)P}
  \and
  \inferrule* [lab=concretion] {} {{C} \bc \langle Q \rangle}
  \and
  \inferrule* [lab=process] {} {{P,Q} \bc M \;| \;P|Q \;|\; @{x}}
  \and
  \inferrule* [lab=name] {} {{x} \bc \quotep{P}}
\end{mathpar} 

Note that $\vec{x}$ (resp. $\vec{P}$) denotes a vector of names
(resp. processes) of length $|\vec{x}|$ (resp. $|\vec{P}|$). We adopt
the following useful abbreviations.

\begin{mathpar}
   x?(\vec{y}).P := x.(\vec{y})P \and  x\clift{\vec{P}} := x.\clift{\vec{P}}
   \and x!(y) := \lift{x}{\dropn{y}}
   \and \Pi_{i=0}^{n-1}P_i := P_0 | \ldots | P_{n-1}
\end{mathpar}

\subsubsection{Structural congruence}

\paragraph{Free and bound names and alpha-equivalence.} At the
core of structural equivalence is alpha-equivalence which identifies
process that are the same up to a change of variable. Formally, we
recognize the distinction between free and bound names. The free names
of a process, $\freenames{P}$, may be calculated recursively as
follows:

\begin{mathpar}
\freenames{\pzero} := \emptyset
  \and \\
  \freenames{x?(y).P} := \{ x \} \cup (\freenames{P} \setminus \{ y \})
  \and 
  \freenames{x!\langle P \rangle} := \{ x \} \cup \{ P \} 
  \and \\
  \freenames{P|Q} := \freenames{P} \cup \freenames{Q}
  \and \\
  \freenames{@{x}} := \{ x \}
\end{mathpar}

$\pi$
$\quotep{\pi}$

$\freenames{-} : \pi \to \mathcal{P}(\quotep{\pi})$

\begin{eqnarray*}
  \freenames{\pzero} & := & \emptyset \\
  \freenames{x?(y).P} & := & \{ x \} \cup (\freenames{P} \setminus \{ y \}) \\
  \freenames{x!\langle P \rangle} & := & \{ x \} \cup \{ P \} \\
  \freenames{P|Q} & := & \freenames{P} \cup \freenames{Q} \\
  \freenames{\dropn{x}} & := & \{ x \}
\end{eqnarray*}

The bound names of a process, $\boundnames{P}$, are those names occurring in $P$
that are not free. For example, in $x?(y).0$, the name $x$ is free, while $y$ is bound.

\begin{mathpar}
  \inferrule* [lab=monoidal-laws] {} { P|Q \equiv Q|P \and P|0 \equiv P \and P|(Q|R) \equiv (P|Q)|R }
\end{mathpar}

\begin{mathpar}
  \inferrule* [lab=alpha-equivalence] {} { (x)P \equiv (y)P\{y/x\} \and y \not\in \freenames{P} }
\end{mathpar}

\begin{definition}
Then two processes, $P,Q$, are alpha-equivalent if $P = Q\{\vec{y}/\vec{x}\}$ for
some $\vec{x} \in \boundnames{Q},\vec{y} \in \boundnames{P}$, where $Q\{\vec{y}/\vec{x}\}$
denotes the capture-avoiding substitution of $\vec{y}$ for $\vec{x}$ in $Q$.
\end{definition}

\begin{definition}
  The {\em structural congruence} \cite{SangiorgiWalker} , $\equiv$,
  between processes is the least congruence containing
  alpha-equivalence, satisfying the abelian monoid laws
  (associativity, commutativity and $\pzero$ as identity) for parallel
  composition $|$ and for summation $+$.
\end{definition}

\subsection{Name equivalence}

We take name equivalence, written $\nameeq$, to be the smallest
equivalence relation generated by the following rules.

\begin{mathpar}
\inferrule*[lab=Quote-drop]
{ }
{ \quotep{@{x}} \nameeq x }

\inferrule*[lab=Struct-equiv]
{ P \scong Q }
{ \quotep{P} \nameeq \quotep{Q} }
\end{mathpar}

The astute reader will have noticed that the mutual recursion of names
and processes imposes a mutual recursion on alpha-equivalence and
structural equivalence via name-equivalence. Fortunately, all of this
works out pleasantly and we may calculate in the natural way, free of
concern. The reader interested in the details is referred to the
appendix \ref{appendix:rho_details}.

\subsection{Substitution}

We use $\Proc$ for the set of processes, $\QProc$ for the set of
names, and $\id{\{}\vec{y} / \vec{x} \id{\}}$ to denote partial maps,
$s : \QProc \rightarrow \QProc$. A map, $s$ lifts, uniquely, to a map
on process terms, $\widehat{s} : \Proc \rightarrow \Proc$ by the
following equations.

\begin{mathpar}
  (0) \psubstp{Q}{P} := 0 \\
  (R \juxtap S) \psubstp{Q}{P}
  :=    
  (R)\psubstp{Q}{P} \juxtap (S) \psubstp{Q}{P} \\
  (x?(y).R) \psubstp{Q}{P}    
  :=    
  (x)\substp{Q}{P} (z)\concat( (R \psubstn{z}{y}) \psubstp{Q}{P} ) \\
  (\lift{x}{R}) \psubstp{Q}{P}  
  :=
  \lift{(x)\substp{Q}{P}}{ R \psubstp{Q}{P} } \\
%   (\dropn{x})  \psubstp{Q}{P}       
%   := 
%   \left\{ 
%     \begin{array}{ccc} 
%       \dropn{\quotep{Q}} & & x \nameeq \quotep{P} \\
%       \dropn{x} & & otherwise \\
%     \end{array}
%   \right. 
  (\dropn{x})  \psubstp{Q}{P}       
  := 
  \left\{ 
    \begin{array}{ccc} 
      Q & & x \nameeq \quotep{P} \\
      \dropn{x} & & otherwise \\
    \end{array}
  \right.
\end{mathpar}
 

where

\begin{eqnarray}
  (x)\id{\{} \lpquote Q \rpquote / \lpquote P \rpquote \id{\}}            = 
  \left\{ 
    \begin{array}{ccc}
      \lpquote Q \rpquote & & x \nameeq \lpquote P \rpquote \\
      x & & otherwise \\
    \end{array}
  \right. \nonumber
\end{eqnarray}

and $z$ is chosen distinct from $\quotep{P}$, $\quotep{Q}$, the free
names in $Q$, and all the names in $R$. Our $\alpha$-equivalence will
be built in the standard way from this substitution.

\begin{remark}\label{rem:no_self_referential_names}
  One consequence of these definitions is that $\forall P. \quotep{P}
  \not\in \freenames{P}$.
\end{remark}

\subsection{ Dynamic quote: an example }

Anticipating something of what's to come, consider applying the
substitution, $\widehat{\id{\{}u / z \id{\}}}$, to the following pair
of processes, $\lift{w}{y!(z)}$ and $w[ \lpquote y!(z) \rpquote ]$.

\begin{eqnarray}
	\lift{w}{y!(z)}\widehat{\id{\{}u / z \id{\}}}
		& = &
		\lift{w}{y!(u)} \nonumber\\
	w[ \lpquote y!(z) \rpquote ] \widehat{ \id{\{}u / z \id{\}} }
		& = &
		w[ \lpquote y!(z) \rpquote ] \nonumber
\end{eqnarray}

Because the body of the process between quotes is impervious to
substitution, we get radically different answers. In fact, by
examining the first process in an input context,
e.g. $x?(z).\lift{w}{y!(z)}$, we see that the process under the lift
operator may be shaped by prefixed inputs binding a name inside it. In
this sense, the lift operator will be seen as a way to dynamically
construct processes before reifying them as names.

Finally equipped with these standard features we can present the
dynamics of the calculus.

\subsubsection{Operational semantics} 

Finally, we introduce the computational dynamics. What marks these
algebras as distinct from other more traditionally studied algebraic
structures, e.g. vector spaces or polynomial rings, is the manner in
which dynamics is captured. In traditional structures, dynamics is typically
expressed through morphisms between such structures, as in linear maps
between vector spaces or morphisms between rings. In algebras
associated with the semantics of computation, the dynamics is
expressed as part of the algebraic structure itself, through a
reduction reduction relation typically denoted by $\red$. Below, we
give a recursive presentation of this relation for the calculus used
in the encoding.

$\red \subseteq \pi \times \pi$
$\red : \pi \to \mathcal{P}(\pi)$

\begin{mathpar}
  \inferrule* [lab=Comm] { \textsf{match}( x_{src}, x_{trgt} ) } { x_{trgt}?(y)P \; | \; x_{src}!\langle {Q} \rangle \red P\{\quotep{Q}/y}\} }
  \and \\
  \inferrule* [lab=Par] {{P} \red {P}'} {{{P} | {Q}} \red {{P}' | {Q}}}
  \and
  \inferrule* [lab=Equiv]{{{P} \scong {P}'} \andalso {{P}' \red {Q}'} \andalso {{Q}' \scong {Q}}}{{P} \red {Q}}
\end{mathpar}

\begin{eqnarray*}
  match_{\equiv} (\quotep{P},\quotep{Q}) & := & P \equiv Q \\
  match_{\dagger}(\quotep{P},\quotep{Q}) & := & \forall R. P|Q \red^{*} R => R \red^{*} 0 \\
  match_{K}(\quotep{P},\quotep{Q}) & := & K \mbox{ for some context } K
\end{eqnarray*}

$u?(x)P | u!\langle Q \rangle \red P\{\quotep{Q}/x\}$

%We write $\wred$ for $\red^*$, and $P\red$ if $\exists Q $ such that $ P \red Q$.
We write $P\red$ if $\exists Q $ such that $ P \red Q$ and $P\not\red$, otherwise.

\section{Replication}

As mentioned before, it is known that replication (and hence
recursion) can be implemented in a higher-order process algebra
\cite{SangiorgiWalker}. As our first example of calculation with the
machinery thus far presented we give the construction explicitly in
the {\rhoc}.

\begin{eqnarray}
	D_{x} & := & \prefix{x}{y}{(\binpar{\outputp{x}{y}}{@{y}})} \nonumber\\
	\bangp_{x}{P} & := & \binpar{{x}!\langle{\binpar{D_{x}}{P}}\rangle}{D_{x}} \nonumber
\end{eqnarray}

\begin{eqnarray}
	\bangp_{x}{P} & & \nonumber\\
	=
	& {x}!\langle{(\prefix{x}{y}{(\outputp{x}{y} | @{y})) | P}}\rangle 
	      | \prefix{x}{y}{(\outputp{x}{y} | @{y})} & \nonumber\\
	\red
	& (\outputp{x}{y} | @{y})\substn{\quotep{(\prefix{x}{y}{(@{y} | \outputp{x}{y})) | P}}}{y} & \nonumber\\
	=
	& \outputp{x}{\quotep{(\prefix{x}{y}{(\outputp{x}{y} | @{y})) | P}}}
	  | {(\prefix{x}{y}{(\outputp{x}{y} | @{y})) | P}} & \nonumber\\
	\red
	& \ldots & \nonumber\\
	\red^*
	& P | P | \ldots & \nonumber
\end{eqnarray}

Of course, this encoding, as an implementation, runs away, unfolding
$\bangp{P}$ eagerly. A lazier and more implementable replication
operator, restricted to input-guarded processes, may be obtained as follows.

\begin{eqnarray}
\bangp{\prefix{u}{v}{P}} 
	:= 
	\binpar{\lift{x}{\prefix{u}{v}{(\binpar{D(x)}{P})}}}{D(x)} \nonumber
\end{eqnarray}

\begin{remark}
  Note that the lazier definition still does not deal with summation
  or mixed summation (i.e. sums over input and output). The reader is
  invited to construct definitions of replication that deal with these
  features. 

  Further, the definitions are parameterized in a name, $x$. Can you,
  gentle reader, make a definition that eliminates this parameter and
  guarantees no accidental interaction between the replication
  machinery and the process being replicated -- i.e. no accidental
  sharing of names used by the process to get its work done and the
  name(s) used by the replication to effect copying. This latter
  revision of the definition of replication is crucial to obtaining
  the expected identity $!!P \sim !P$.
\end{remark}

\begin{remark}\label{rem:paradoxical_combinator}
  The reader familiar with the lambda calculus will have noticed the
  similarity between $D$ and the paradoxical combinator.

  [Ed. note: the existence of this seems to suggest we have to be more
  restrictive on the set of processes and names we admit if we are to
  support no-cloning.]
\end{remark}

\subsubsection{Bisimulation}

The computational dynamics gives rise to another kind of equivalence,
the equivalence of computational behavior. As previously mentioned
this is typically captured \emph{via} some form of bisimulation.

% The notion we use in this paper is weak barbed bisimulation
% \cite{milner91polyadicpi}.

The notion we use in this paper is derived from weak barbed
bisimulation \cite{milner91polyadicpi}. 

\begin{definition}
An \emph{observation relation}, $\downarrow_{\mathcal N}$, over a set
of names, $\mathcal N$, is the smallest relation satisfying the rules
below.

\infrule[Out-barb]{y \in {\mathcal N}, \; x \nameeq y}
		  {\outputp{x}{v} \downarrow_{\mathcal N} x}
\infrule[Par-barb]{\mbox{$P\downarrow_{\mathcal N} x$ or $Q\downarrow_{\mathcal N} x$}}
		  {\binpar{P}{Q} \downarrow_{\mathcal N} x}

We write $P \Downarrow_{\mathcal N} x$ if there is $Q$ such that 
$P \wred Q$ and $Q \downarrow_{\mathcal N} x$.
\end{definition}

\begin{definition}
%\label{def.bbisim}
An  ${\mathcal N}$-\emph{barbed bisimulation} over a set of names, ${\mathcal N}$, is a symmetric binary relation 
${\mathcal S}_{\mathcal N}$ between agents such that $P\rel{S}_{\mathcal N}Q$ implies:
\begin{enumerate}
\item If $P \red P'$ then $Q \wred Q'$ and $P'\rel{S}_{\mathcal N} Q'$.
\item If $P\downarrow_{\mathcal N} x$, then $Q\Downarrow_{\mathcal N} x$.
\end{enumerate}
$P$ is ${\mathcal N}$-barbed bisimilar to $Q$, written
$P \wbbisim_{\mathcal N} Q$, if $P \rel{S}_{\mathcal N} Q$ for some ${\mathcal N}$-barbed bisimulation ${\mathcal S}_{\mathcal N}$.
\end{definition}

$\mathcal{R} \subseteq \pi \times \pi$

$P \mathcal{R} Q => \forall P'. P \red P' \Rightarrow \exists Q'. Q \red Q', P' \mathcal{R} Q'$

$P \vdash x \Rightarrow Q \vdash x$

\begin{mathpar}
  \inferrule*[lab=Out-barb]{x \nameeq y}{{y}!\langle{Q}\rangle \vdash x}
  \and
  \inferrule*[lab=Par-barb]{\mbox{$P\vdash x$ or $Q\vdash x$}}{\binpar{P}{Q} \vdash x}
\end{mathpar}

\subsubsection{Contexts}

One of the principle advantages of computational calculi like the
$\pi$-calculus is a well-defined notion of context,
contextual-equivalence and a correlation between
contextual-equivalence and notions of bisimulation. The notion of
context allows the decomposition of a process into (sub-)process and
its syntactic environment, its context. Thus, a context may be
thought of as a process with a ``hole'' (written $\Box$) in it. The
application of a context $M$ to a process $P$, written $M[P]$, is
tantamount to filling the hole in $M$ with $P$. In this paper we do
not need the full weight of this theory, but do make use of the notion
of context in the proof the main theorem. 

\begin{mathpar}
  \inferrule* [lab=summation] {} {{M_{M},M_{N}} \bc \Box \;|\; x.M_{A} \;|\; M_{M}+M_{N}}
  \and
  \inferrule* [lab=agent] {} {{M_{A}} \bc (\vec{x})M_{P} \;| \; \clift{P_0,\ldots,M_{P},\ldots,P_N}}
  \and \\
  \inferrule* [lab=process] {} {{M_{P}} \bc M_{N} \;| \;P|M_{P} }
\end{mathpar} 

\begin{mathpar}
  \inferrule* [lab=sychronization] {} {M_{N} \bc \Box \;|\; x?M_{F} \;|\; x!M_{C}}
  \and
  \inferrule* [lab=abstraction] {} {{M_{F}} \bc (x)M_{P} }
  \and
  \inferrule* [lab=concretion] {} {{M_{C}} \bc \langle M_{P} \rangle }
  \and \\
  \inferrule* [lab=process] {} {{M_{P}} \bc M_{N} \;| \;P|M_{P} }
\end{mathpar}

\begin{definition}[contextual application] Given a context $M$, and
  process $P$, we define the \emph{contextual application}, $M[P] :=
  M\{P/\Box\}$. That is, the contextual application of M to P is the
  substitution of $P$ for $\Box$ in $M$.
\end{definition}

$\meaningof{-} : L \to \mathcal{P}(\pi)$

\begin{mathpar}
  \inferrule* [lab=collection] {} {\meaningof{true} = \pi, \and \meaningof{~E} = \pi \setminus \meaningof{E}, \and \meaningof{E_{1} \& E_{2}} = \meaningof{E_{1}} \cap \meaningof{E_{2}}}
\end{mathpar}

\begin{mathpar}
  \inferrule* [lab=structure] {} {\meaningof{0} = \{ P \in \pi | P \equiv 0 \}, \and \\ \meaningof{E_1 | E_2} = \{ P \in \pi | P \equiv P_{1} | P_{2}, P_{1} \in \meaningof{E_{1}}, P_{2} \in \meaningof{E_2}\} }
\end{mathpar}

\begin{mathpar}
 \inferrule* [lab=behavior] {} {\meaningof{\langle a?b \rangle E} = \{ P \in \pi | P \equiv Q | u?(y)P', \\ \and \\\\ \and \\ \;\;\; u \in \meaningof{a}, \forall z.P'\{z/y\} \in \meaningof{E\{z/b\}}\}, \and \\ \meaningof{a!E} = \{ P \in \pi | P \equiv Q | x!\langle P' \rangle, x \in \meaningof{a} P' \in \meaningof{E}\} }
\end{mathpar}

\begin{mathpar}
 \inferrule* [lab=nominal] {} {\meaningof{\quotep{E}} = \{ \quotep{P} \in \quotep{\pi} | P \in \meaningof{E} \}, \and \meaningof{\quotep{P}} = \{ \quotep{Q} \in \quotep{\pi} | P \equiv Q \} \and \\ \meaningof{@\quotep{E}} = \{ P \in \pi | P \equiv @x, x \in \meaningof{E} \}}
\end{mathpar}

\begin{eqnarray*}
  \\
  \meaningof{-} : TS \to ST
\end{eqnarray*}

\begin{eqnarray*}
  \\
  L : TS \to ST
\end{eqnarray*}

\begin{eqnarray*}
  \\
  P \models E \iff P \in \meaningof{E}
\end{eqnarray*}

\begin{eqnarray*}
  P \approx_{L} Q \iff \forall E \in L. P \models E \iff Q \models E
\end{eqnarray*}

\begin{eqnarray*}
  P \approx_{K} Q
\end{eqnarray*}

\begin{eqnarray*}
  P \approx Q
\end{eqnarray*}

$\approx_{K} = \approx = \approx_{L}$

\subsubsection{Contextual duality}

Note that contexts extend the quotation operation to a family of
operations from processes to names. Given a context, $M$, we can
define a \emph{nominal context}, $\quotep{M}$ by $\quotep{M}[P] :=
\quotep{M[P]}$. To foreshadow what is to come we observe that these
operations enjoy a duality with processes very much like the duality
between vectors and maps from vectors to scalars.

Further, because the calculus is essentially higher-order, we have a
correspondence between contexts and processes. More specifically,
given a name $x$ and a context $M$ we can construct $M^{*}_{x}$ such
that 

\begin{mathpar}
  M^{*}_{x} | \lift{x}{P} \red M[P]
\end{mathpar}

namely,

\begin{mathpar}
  M^{*}_{x} := x?(u).M[\dropn{u}]
\end{mathpar}

The dependence of $M^{*}_{x}$ on a name makes it an abstraction, 

\begin{mathpar}
  M^{*} := (x)x?(u).M[\dropn{u}]
\end{mathpar}

\subsection{Additional notation}

It will sometimes be convenient to denote the process a name
quotes. We already have the notation $x = \quotep{P}$, but it will be
convenient to introduce an alternate notation, $\procn{x}$, when we
want to emphasize the connection to the use of the name. Note that, by
virtue of name equivalence, $\quotep{\procn{x}} \nameeq x$; so, the
notation is consistent with previous definitions.

Further, because names have structure it is possible to effect
substitutions on the basis of that structure. This means we need to
upgrade our notation for substitutions, which we accomplish by
adapting comprehension notation. Thus,

\begin{mathpar}
  P\{ y / x : x \in S \}
\end{mathpar}

is interpreted to mean the process derived from P by replacing (in a
capture-avoiding manner) each occurrence of $x$ in $S$ by $y$. For example,

\begin{mathpar}
  P\{ \quotep{\procn{x}|\procn{x}} / x : x \in \freenames{P} \}
\end{mathpar}

will replace each (occurrence) of a free name $x$ in $P$ by
$\quotep{\procn{x}|\procn{x}}$.

Also, we will avail ourselves of the notation $x^{L}$ and $x^{R}$ to
denote injections of a name into disjoint copies of the name
space. There are numerous ways to accomplish this. One example can be
found in \cite{MeredithR05}. This notation overloads to vectors of
names: $\vec{x}^{\pi} := (x_{i}^{\pi} \; : \; 0 \leq i < |\vec{x}| )$ where $\pi \in \{L,R\}$.

We also use $P^{\Box} := P|\Box$.

In \cite{MeredithR05} an interpretation of the new operator is
given. It turns out that there are several possible interpretations
all enjoying the requisite algebraic properties of the operator (see
\cite{milner91polyadicpi}). We will therefore make liberal use of
$(\nu\; \vec{x})P$.

% subsection the_syntax_and_semantics_of_the_notation_system (end)   

\input{qm2pi.qmops} 

\input{qm2pi.sterngerlach} 

\input{qm2pi.metric} 

% section concurrent_process_calculi (end)

%\input{qm2pi.proofsketch}

% section proof sketch (end)

%\input{qm2pi.slviaknots} 

% section spatial logic via knots (end)

\input{qm2pi.conclusion}

% section conclusion (end)

%\input{qm2pi.dtcodes} 

% section wiring algorithm (end)

\input{qm2pi.ack} 

% section acknowledgments (end)

\newpage


\bibliographystyle{plain}   
\bibliography{../../biblios/main.bib}

\input{qm2pi.rhodetails}

\end{document}



% section front matter (end)

\section{Introduction}\label{sec:introduction} % (fold)
In this draft of the material i am going to have to dispense with the
usual writing conventions adopted in papers on these topics. i'm going
to have adopt whatever tone i need at the time i'm writing up the
calculations. Sometimes this may be very conversational; others it may
be the barest mathematical grunts; others still it may be that i have
lifted text from one of my other papers because the exposition of some
point was better said there. i hope that my readers are not unduly put
out by this decision. i'm not doing this to flout convention or be
rebellious. i find these calculations very technically challenging. To
keep everything going technically, something has to give; i have to
let go of some cognitive burden. So, the academic writing style --
with all of its trade-offs in terms of facilitating technical
communication -- is what i'm letting go of. Perhaps subsequent drafts
can be tightened and polished, but for now, i'm going to speak as if
we were sitting together in a coffee shop with a laptop, wifi and a
pad of paper and a pencil.

So, here's what i have to say. We -- you and i, comfortably ensconced
in our coffee shop and well-equipped with our tools -- can realize and
carry out the calculations of quantum mechanics over a very different
formal theory of dynamics, a formal theory of dynamics that
corresponds to a theory of concurrent computation with
\emph{reflection}. It has the advantage that the underlying theory is
already `quantized', but supports analogues all of the continuuous
operations. Strikingly, this underlying theory has recently been
connected with a notion of metric that we can show, by calculating
together, coincides with the metric induced by the inner product.

There are a lot of reasons why you might be interested in seeing
calculations of this form. Here's why i'm interested. For the past
several centuries there has been no competitor to the ``Newtonian''
account of dynamics. As a result the predominant share of accounts of
dynamical systems and situations have had to be formulated in terms of
the Newtonian machinery. i view this as an intellectually dangerous
position to occupy. Everything, despite it's intrinsic shape, turns
into a nail to be hit with this hammer. Recently, however, the theory
of computation has matured to the point where we have candidates for
theories of dynamics that offer very different perspective on
reasoning about dynamical systems and situations. Testing these
candidates against very successful accounts of dynamical situations,
like quantum mechanics, is going to give us some sense of how mature
they are and some measure of the quality of these accounts of
dynamics.

\subsection{Summary of contributions and outline of paper}

So, we're going to develop an interpretation of the operations of
quantum mechanics normally interpreted by Hilbert spaces and
operators. We're going to do this over a theory of computation. Note
that this is very different than the usual quantum computation program
which develops notions of computation over quantum mechanics. Rather,
we are developing a story that aligns with Wheeler's slogan: It from
Bit. To do this we will first provide an account of the theory of
computation at play here. Then we will dive into a calculation-driven
interpretation of the operations of quantum mechanics.

The reason we take this approach is that -- until very recently --
there hasn't been an axiomatic account of quantum mechanics. As a
result there has been no sharp delineation of the mathematical theory
supporting interpretation of the physical theory and the physical
theory, itself. So, ambient features of the maths are free to be
exploited (or supressed) without a real accounting of their physical
relevance. There is no sharp statement ``here's the physical theory''
qua \emph{theory} and ``here's the mathematical interpretation''
enabling a judgment of how faithful the interpretation is -- apart
from experimental observation. When there is an axiomatic account we
can judge how well a given mathematical formalism supports an
interpretation of the axioms, independent of
experimentation. Likewise, we can judge how well we have captured our
physical evidence and experience with our axiomatics, independent of
any specific mathematical implementation, with accidental detail that
may or may not have physical significance. 

In lieu of a fully fleshed out and vetted axiomatic account of quantum
mechanics, interpreting the operational notions in service of modeling
physical systems will have to suffice. In other words, we are not in
the business of providing a model of Hilbert spaces and operators. We
are in the business of providing a model of quantum mechanics because
we are motivated by testing our notions of dynamics against physical
theory; and, the predictive calculations of the physical theory must
serve as the best formulation -- shy of a fully fleshed out axiomatic
account -- of the physical theory itself (as they have for scientific
theories since time immemorial). Put another way, despite a
whole-hearted commitment to an It-from-Bit ontology, we are firmly
aligned with the shut-up-and-calculate camp as the best way to obtain
results either from the physical perspective or as a quality assurance
measure of our fledgling theory of dynamics.

In detail, we present a reflective process calculus. Then we develop
intuitive correspondences between the notions available in this
calculus and the usual physical notions supporting quantum mechanical
calculations. Thus, 

\begin{table}[htp]
  \center{
    \fbox{
      \begin{tabular}{c|c}
        quantum mechanics & process calculus \\
        \hline
        scalar & name \\
        state vector & process \\
        dual & contextual duals \\
        matrix & formal sums of process-context-dual pairs \\
        orthogonality & process annihilation \\
        inner product & execution-formula + quoting
      \end{tabular}
    }
  }
  \caption{QM - process calculi correspondences}
\end{table}

Then we tighten up these intuitions to operational definitions. We
employ the Dirac notation as the best proxy we can find for an
abstract syntax of the quantum mechanical notions. The definitions we
develop put us in contact with equational constraints coming from the
theory that we demonstrate the definitions and calculations satisfy.

This puts us in a position to shut up and calculate for the
Stern-Gerlach experimental set up, showing how these predictive
calculations become calculations on processes in our theory of a
reflective process calculus.

Penultimately, we demonstrate that the notion of metric coming from
the inner product coincides with the notion of metric available from
the theory of bisimulation. This demonstration gives us the right to
think of space as arising from behavior. Finally, we consider where we
might go from the new vantage point we have obtained.

% section introduction (end) 
 
% section introduction (end)

% \documentclass[12pt]{llncs}
%\documentclass{jktr}

\usepackage[pdftex]{hyperref}                   
\usepackage {listings}
\usepackage {mathpartir}
\usepackage{bcprules}
%\usepackage{listings}
                       
\usepackage{graphicx} 
%\usepackage[margins=2.5cm,nohead,nofoot]{geometry}
%\usepackage{geometry}
\usepackage{amsfonts}
\usepackage{amstext}
\usepackage{latexsym}
\usepackage{amssymb}
\usepackage{color}


%\include{myPreamble}
\include{qm2pi.local} 

%\ifpdf
%\usepackage[pdftex]{graphicx}
%\else
%\usepackage{graphicx}
%\fi

 % \ifpdf
%  \usepackage{pdfsync}
%  \if


%\title{Brief Article}
%\author{David F. Snyder}
%\author{L.G. Meredith}

%\address{Dept. of Math., Texas State University--San Marcos, San Marcos, TX 78666}
       
\pagestyle{empty}


\begin{document}

\lstset{language=[Objective]Caml,frame=shadowbox}

\input{qm2pi.front}

% section front matter (end)

\input{qm2pi.intro} 
 
% section introduction (end)

% \input{qm2pi.knotations} 

% section notation (end)

\input{qm2pi.process.calculi} 

% section concurrent_process_calculi_and_spatial_logics_ (end)
    
%\input{qm2pi.knots2pi} 

%\input{qm2pi.trefoil} 

%\input{qm2pi.mainthm} 

% subsection basic_interpretation (end)

%\input{qm2pi.rho.presentation} 
\subsection{The syntax and semantics of the notation system}\label{sub:the_syntax_and_semantics_of_the_notation_system} % (fold)

We now summarize a technical presentation of the calculus that
embodies our theory of dynamics. The typical presentation of such a
calculus follows the style of giving generators and relations on
them. The grammar, below, describing term constructors, freely
generates the set of processes, $\Proc$. This set is then quotiented
by a relation known as structural congruence and it is over this set
that the notion of dynamics is expressed. This presentation is
essentially that of \cite{MeredithR05} with the addition of
polyadicity and summation. For readability we have relegated some of
the technical subtleties to an appendix.

\subsubsection{Process grammar}\label{subsub:process_grammar}

\begin{mathpar}
  \inferrule* [lab=synchronization] {} {{M} \bc \pzero \;|\; x?F \;|\; x!C }
  \and
  \inferrule* [lab=abstraction] {} {{F} \bc (x)P}
  \and
  \inferrule* [lab=concretion] {} {{C} \bc \langle Q \rangle}
  \and
  \inferrule* [lab=process] {} {{P,Q} \bc M \;| \;P|Q \;|\; @{x}}
  \and
  \inferrule* [lab=name] {} {{x} \bc \quotep{P}}
\end{mathpar} 

Note that $\vec{x}$ (resp. $\vec{P}$) denotes a vector of names
(resp. processes) of length $|\vec{x}|$ (resp. $|\vec{P}|$). We adopt
the following useful abbreviations.

\begin{mathpar}
   x?(\vec{y}).P := x.(\vec{y})P \and  x\clift{\vec{P}} := x.\clift{\vec{P}}
   \and x!(y) := \lift{x}{\dropn{y}}
   \and \Pi_{i=0}^{n-1}P_i := P_0 | \ldots | P_{n-1}
\end{mathpar}

\subsubsection{Structural congruence}

\paragraph{Free and bound names and alpha-equivalence.} At the
core of structural equivalence is alpha-equivalence which identifies
process that are the same up to a change of variable. Formally, we
recognize the distinction between free and bound names. The free names
of a process, $\freenames{P}$, may be calculated recursively as
follows:

\begin{mathpar}
\freenames{\pzero} := \emptyset
  \and \\
  \freenames{x?(y).P} := \{ x \} \cup (\freenames{P} \setminus \{ y \})
  \and 
  \freenames{x!\langle P \rangle} := \{ x \} \cup \{ P \} 
  \and \\
  \freenames{P|Q} := \freenames{P} \cup \freenames{Q}
  \and \\
  \freenames{@{x}} := \{ x \}
\end{mathpar}

$\pi$
$\quotep{\pi}$

$\freenames{-} : \pi \to \mathcal{P}(\quotep{\pi})$

\begin{eqnarray*}
  \freenames{\pzero} & := & \emptyset \\
  \freenames{x?(y).P} & := & \{ x \} \cup (\freenames{P} \setminus \{ y \}) \\
  \freenames{x!\langle P \rangle} & := & \{ x \} \cup \{ P \} \\
  \freenames{P|Q} & := & \freenames{P} \cup \freenames{Q} \\
  \freenames{\dropn{x}} & := & \{ x \}
\end{eqnarray*}

The bound names of a process, $\boundnames{P}$, are those names occurring in $P$
that are not free. For example, in $x?(y).0$, the name $x$ is free, while $y$ is bound.

\begin{mathpar}
  \inferrule* [lab=monoidal-laws] {} { P|Q \equiv Q|P \and P|0 \equiv P \and P|(Q|R) \equiv (P|Q)|R }
\end{mathpar}

\begin{mathpar}
  \inferrule* [lab=alpha-equivalence] {} { (x)P \equiv (y)P\{y/x\} \and y \not\in \freenames{P} }
\end{mathpar}

\begin{definition}
Then two processes, $P,Q$, are alpha-equivalent if $P = Q\{\vec{y}/\vec{x}\}$ for
some $\vec{x} \in \boundnames{Q},\vec{y} \in \boundnames{P}$, where $Q\{\vec{y}/\vec{x}\}$
denotes the capture-avoiding substitution of $\vec{y}$ for $\vec{x}$ in $Q$.
\end{definition}

\begin{definition}
  The {\em structural congruence} \cite{SangiorgiWalker} , $\equiv$,
  between processes is the least congruence containing
  alpha-equivalence, satisfying the abelian monoid laws
  (associativity, commutativity and $\pzero$ as identity) for parallel
  composition $|$ and for summation $+$.
\end{definition}

\subsection{Name equivalence}

We take name equivalence, written $\nameeq$, to be the smallest
equivalence relation generated by the following rules.

\begin{mathpar}
\inferrule*[lab=Quote-drop]
{ }
{ \quotep{@{x}} \nameeq x }

\inferrule*[lab=Struct-equiv]
{ P \scong Q }
{ \quotep{P} \nameeq \quotep{Q} }
\end{mathpar}

The astute reader will have noticed that the mutual recursion of names
and processes imposes a mutual recursion on alpha-equivalence and
structural equivalence via name-equivalence. Fortunately, all of this
works out pleasantly and we may calculate in the natural way, free of
concern. The reader interested in the details is referred to the
appendix \ref{appendix:rho_details}.

\subsection{Substitution}

We use $\Proc$ for the set of processes, $\QProc$ for the set of
names, and $\id{\{}\vec{y} / \vec{x} \id{\}}$ to denote partial maps,
$s : \QProc \rightarrow \QProc$. A map, $s$ lifts, uniquely, to a map
on process terms, $\widehat{s} : \Proc \rightarrow \Proc$ by the
following equations.

\begin{mathpar}
  (0) \psubstp{Q}{P} := 0 \\
  (R \juxtap S) \psubstp{Q}{P}
  :=    
  (R)\psubstp{Q}{P} \juxtap (S) \psubstp{Q}{P} \\
  (x?(y).R) \psubstp{Q}{P}    
  :=    
  (x)\substp{Q}{P} (z)\concat( (R \psubstn{z}{y}) \psubstp{Q}{P} ) \\
  (\lift{x}{R}) \psubstp{Q}{P}  
  :=
  \lift{(x)\substp{Q}{P}}{ R \psubstp{Q}{P} } \\
%   (\dropn{x})  \psubstp{Q}{P}       
%   := 
%   \left\{ 
%     \begin{array}{ccc} 
%       \dropn{\quotep{Q}} & & x \nameeq \quotep{P} \\
%       \dropn{x} & & otherwise \\
%     \end{array}
%   \right. 
  (\dropn{x})  \psubstp{Q}{P}       
  := 
  \left\{ 
    \begin{array}{ccc} 
      Q & & x \nameeq \quotep{P} \\
      \dropn{x} & & otherwise \\
    \end{array}
  \right.
\end{mathpar}
 

where

\begin{eqnarray}
  (x)\id{\{} \lpquote Q \rpquote / \lpquote P \rpquote \id{\}}            = 
  \left\{ 
    \begin{array}{ccc}
      \lpquote Q \rpquote & & x \nameeq \lpquote P \rpquote \\
      x & & otherwise \\
    \end{array}
  \right. \nonumber
\end{eqnarray}

and $z$ is chosen distinct from $\quotep{P}$, $\quotep{Q}$, the free
names in $Q$, and all the names in $R$. Our $\alpha$-equivalence will
be built in the standard way from this substitution.

\begin{remark}\label{rem:no_self_referential_names}
  One consequence of these definitions is that $\forall P. \quotep{P}
  \not\in \freenames{P}$.
\end{remark}

\subsection{ Dynamic quote: an example }

Anticipating something of what's to come, consider applying the
substitution, $\widehat{\id{\{}u / z \id{\}}}$, to the following pair
of processes, $\lift{w}{y!(z)}$ and $w[ \lpquote y!(z) \rpquote ]$.

\begin{eqnarray}
	\lift{w}{y!(z)}\widehat{\id{\{}u / z \id{\}}}
		& = &
		\lift{w}{y!(u)} \nonumber\\
	w[ \lpquote y!(z) \rpquote ] \widehat{ \id{\{}u / z \id{\}} }
		& = &
		w[ \lpquote y!(z) \rpquote ] \nonumber
\end{eqnarray}

Because the body of the process between quotes is impervious to
substitution, we get radically different answers. In fact, by
examining the first process in an input context,
e.g. $x?(z).\lift{w}{y!(z)}$, we see that the process under the lift
operator may be shaped by prefixed inputs binding a name inside it. In
this sense, the lift operator will be seen as a way to dynamically
construct processes before reifying them as names.

Finally equipped with these standard features we can present the
dynamics of the calculus.

\subsubsection{Operational semantics} 

Finally, we introduce the computational dynamics. What marks these
algebras as distinct from other more traditionally studied algebraic
structures, e.g. vector spaces or polynomial rings, is the manner in
which dynamics is captured. In traditional structures, dynamics is typically
expressed through morphisms between such structures, as in linear maps
between vector spaces or morphisms between rings. In algebras
associated with the semantics of computation, the dynamics is
expressed as part of the algebraic structure itself, through a
reduction reduction relation typically denoted by $\red$. Below, we
give a recursive presentation of this relation for the calculus used
in the encoding.

$\red \subseteq \pi \times \pi$
$\red : \pi \to \mathcal{P}(\pi)$

\begin{mathpar}
  \inferrule* [lab=Comm] { \textsf{match}( x_{src}, x_{trgt} ) } { x_{trgt}?(y)P \; | \; x_{src}!\langle {Q} \rangle \red P\{\quotep{Q}/y}\} }
  \and \\
  \inferrule* [lab=Par] {{P} \red {P}'} {{{P} | {Q}} \red {{P}' | {Q}}}
  \and
  \inferrule* [lab=Equiv]{{{P} \scong {P}'} \andalso {{P}' \red {Q}'} \andalso {{Q}' \scong {Q}}}{{P} \red {Q}}
\end{mathpar}

\begin{eqnarray*}
  match_{\equiv} (\quotep{P},\quotep{Q}) & := & P \equiv Q \\
  match_{\dagger}(\quotep{P},\quotep{Q}) & := & \forall R. P|Q \red^{*} R => R \red^{*} 0 \\
  match_{K}(\quotep{P},\quotep{Q}) & := & K \mbox{ for some context } K
\end{eqnarray*}

$u?(x)P | u!\langle Q \rangle \red P\{\quotep{Q}/x\}$

%We write $\wred$ for $\red^*$, and $P\red$ if $\exists Q $ such that $ P \red Q$.
We write $P\red$ if $\exists Q $ such that $ P \red Q$ and $P\not\red$, otherwise.

\section{Replication}

As mentioned before, it is known that replication (and hence
recursion) can be implemented in a higher-order process algebra
\cite{SangiorgiWalker}. As our first example of calculation with the
machinery thus far presented we give the construction explicitly in
the {\rhoc}.

\begin{eqnarray}
	D_{x} & := & \prefix{x}{y}{(\binpar{\outputp{x}{y}}{@{y}})} \nonumber\\
	\bangp_{x}{P} & := & \binpar{{x}!\langle{\binpar{D_{x}}{P}}\rangle}{D_{x}} \nonumber
\end{eqnarray}

\begin{eqnarray}
	\bangp_{x}{P} & & \nonumber\\
	=
	& {x}!\langle{(\prefix{x}{y}{(\outputp{x}{y} | @{y})) | P}}\rangle 
	      | \prefix{x}{y}{(\outputp{x}{y} | @{y})} & \nonumber\\
	\red
	& (\outputp{x}{y} | @{y})\substn{\quotep{(\prefix{x}{y}{(@{y} | \outputp{x}{y})) | P}}}{y} & \nonumber\\
	=
	& \outputp{x}{\quotep{(\prefix{x}{y}{(\outputp{x}{y} | @{y})) | P}}}
	  | {(\prefix{x}{y}{(\outputp{x}{y} | @{y})) | P}} & \nonumber\\
	\red
	& \ldots & \nonumber\\
	\red^*
	& P | P | \ldots & \nonumber
\end{eqnarray}

Of course, this encoding, as an implementation, runs away, unfolding
$\bangp{P}$ eagerly. A lazier and more implementable replication
operator, restricted to input-guarded processes, may be obtained as follows.

\begin{eqnarray}
\bangp{\prefix{u}{v}{P}} 
	:= 
	\binpar{\lift{x}{\prefix{u}{v}{(\binpar{D(x)}{P})}}}{D(x)} \nonumber
\end{eqnarray}

\begin{remark}
  Note that the lazier definition still does not deal with summation
  or mixed summation (i.e. sums over input and output). The reader is
  invited to construct definitions of replication that deal with these
  features. 

  Further, the definitions are parameterized in a name, $x$. Can you,
  gentle reader, make a definition that eliminates this parameter and
  guarantees no accidental interaction between the replication
  machinery and the process being replicated -- i.e. no accidental
  sharing of names used by the process to get its work done and the
  name(s) used by the replication to effect copying. This latter
  revision of the definition of replication is crucial to obtaining
  the expected identity $!!P \sim !P$.
\end{remark}

\begin{remark}\label{rem:paradoxical_combinator}
  The reader familiar with the lambda calculus will have noticed the
  similarity between $D$ and the paradoxical combinator.

  [Ed. note: the existence of this seems to suggest we have to be more
  restrictive on the set of processes and names we admit if we are to
  support no-cloning.]
\end{remark}

\subsubsection{Bisimulation}

The computational dynamics gives rise to another kind of equivalence,
the equivalence of computational behavior. As previously mentioned
this is typically captured \emph{via} some form of bisimulation.

% The notion we use in this paper is weak barbed bisimulation
% \cite{milner91polyadicpi}.

The notion we use in this paper is derived from weak barbed
bisimulation \cite{milner91polyadicpi}. 

\begin{definition}
An \emph{observation relation}, $\downarrow_{\mathcal N}$, over a set
of names, $\mathcal N$, is the smallest relation satisfying the rules
below.

\infrule[Out-barb]{y \in {\mathcal N}, \; x \nameeq y}
		  {\outputp{x}{v} \downarrow_{\mathcal N} x}
\infrule[Par-barb]{\mbox{$P\downarrow_{\mathcal N} x$ or $Q\downarrow_{\mathcal N} x$}}
		  {\binpar{P}{Q} \downarrow_{\mathcal N} x}

We write $P \Downarrow_{\mathcal N} x$ if there is $Q$ such that 
$P \wred Q$ and $Q \downarrow_{\mathcal N} x$.
\end{definition}

\begin{definition}
%\label{def.bbisim}
An  ${\mathcal N}$-\emph{barbed bisimulation} over a set of names, ${\mathcal N}$, is a symmetric binary relation 
${\mathcal S}_{\mathcal N}$ between agents such that $P\rel{S}_{\mathcal N}Q$ implies:
\begin{enumerate}
\item If $P \red P'$ then $Q \wred Q'$ and $P'\rel{S}_{\mathcal N} Q'$.
\item If $P\downarrow_{\mathcal N} x$, then $Q\Downarrow_{\mathcal N} x$.
\end{enumerate}
$P$ is ${\mathcal N}$-barbed bisimilar to $Q$, written
$P \wbbisim_{\mathcal N} Q$, if $P \rel{S}_{\mathcal N} Q$ for some ${\mathcal N}$-barbed bisimulation ${\mathcal S}_{\mathcal N}$.
\end{definition}

$\mathcal{R} \subseteq \pi \times \pi$

$P \mathcal{R} Q => \forall P'. P \red P' \Rightarrow \exists Q'. Q \red Q', P' \mathcal{R} Q'$

$P \vdash x \Rightarrow Q \vdash x$

\begin{mathpar}
  \inferrule*[lab=Out-barb]{x \nameeq y}{{y}!\langle{Q}\rangle \vdash x}
  \and
  \inferrule*[lab=Par-barb]{\mbox{$P\vdash x$ or $Q\vdash x$}}{\binpar{P}{Q} \vdash x}
\end{mathpar}

\subsubsection{Contexts}

One of the principle advantages of computational calculi like the
$\pi$-calculus is a well-defined notion of context,
contextual-equivalence and a correlation between
contextual-equivalence and notions of bisimulation. The notion of
context allows the decomposition of a process into (sub-)process and
its syntactic environment, its context. Thus, a context may be
thought of as a process with a ``hole'' (written $\Box$) in it. The
application of a context $M$ to a process $P$, written $M[P]$, is
tantamount to filling the hole in $M$ with $P$. In this paper we do
not need the full weight of this theory, but do make use of the notion
of context in the proof the main theorem. 

\begin{mathpar}
  \inferrule* [lab=summation] {} {{M_{M},M_{N}} \bc \Box \;|\; x.M_{A} \;|\; M_{M}+M_{N}}
  \and
  \inferrule* [lab=agent] {} {{M_{A}} \bc (\vec{x})M_{P} \;| \; \clift{P_0,\ldots,M_{P},\ldots,P_N}}
  \and \\
  \inferrule* [lab=process] {} {{M_{P}} \bc M_{N} \;| \;P|M_{P} }
\end{mathpar} 

\begin{mathpar}
  \inferrule* [lab=sychronization] {} {M_{N} \bc \Box \;|\; x?M_{F} \;|\; x!M_{C}}
  \and
  \inferrule* [lab=abstraction] {} {{M_{F}} \bc (x)M_{P} }
  \and
  \inferrule* [lab=concretion] {} {{M_{C}} \bc \langle M_{P} \rangle }
  \and \\
  \inferrule* [lab=process] {} {{M_{P}} \bc M_{N} \;| \;P|M_{P} }
\end{mathpar}

\begin{definition}[contextual application] Given a context $M$, and
  process $P$, we define the \emph{contextual application}, $M[P] :=
  M\{P/\Box\}$. That is, the contextual application of M to P is the
  substitution of $P$ for $\Box$ in $M$.
\end{definition}

$\meaningof{-} : L \to \mathcal{P}(\pi)$

\begin{mathpar}
  \inferrule* [lab=collection] {} {\meaningof{true} = \pi, \and \meaningof{~E} = \pi \setminus \meaningof{E}, \and \meaningof{E_{1} \& E_{2}} = \meaningof{E_{1}} \cap \meaningof{E_{2}}}
\end{mathpar}

\begin{mathpar}
  \inferrule* [lab=structure] {} {\meaningof{0} = \{ P \in \pi | P \equiv 0 \}, \and \\ \meaningof{E_1 | E_2} = \{ P \in \pi | P \equiv P_{1} | P_{2}, P_{1} \in \meaningof{E_{1}}, P_{2} \in \meaningof{E_2}\} }
\end{mathpar}

\begin{mathpar}
 \inferrule* [lab=behavior] {} {\meaningof{\langle a?b \rangle E} = \{ P \in \pi | P \equiv Q | u?(y)P', \\ \and \\\\ \and \\ \;\;\; u \in \meaningof{a}, \forall z.P'\{z/y\} \in \meaningof{E\{z/b\}}\}, \and \\ \meaningof{a!E} = \{ P \in \pi | P \equiv Q | x!\langle P' \rangle, x \in \meaningof{a} P' \in \meaningof{E}\} }
\end{mathpar}

\begin{mathpar}
 \inferrule* [lab=nominal] {} {\meaningof{\quotep{E}} = \{ \quotep{P} \in \quotep{\pi} | P \in \meaningof{E} \}, \and \meaningof{\quotep{P}} = \{ \quotep{Q} \in \quotep{\pi} | P \equiv Q \} \and \\ \meaningof{@\quotep{E}} = \{ P \in \pi | P \equiv @x, x \in \meaningof{E} \}}
\end{mathpar}

\begin{eqnarray*}
  \\
  \meaningof{-} : TS \to ST
\end{eqnarray*}

\begin{eqnarray*}
  \\
  L : TS \to ST
\end{eqnarray*}

\begin{eqnarray*}
  \\
  P \models E \iff P \in \meaningof{E}
\end{eqnarray*}

\begin{eqnarray*}
  P \approx_{L} Q \iff \forall E \in L. P \models E \iff Q \models E
\end{eqnarray*}

\begin{eqnarray*}
  P \approx_{K} Q
\end{eqnarray*}

\begin{eqnarray*}
  P \approx Q
\end{eqnarray*}

$\approx_{K} = \approx = \approx_{L}$

\subsubsection{Contextual duality}

Note that contexts extend the quotation operation to a family of
operations from processes to names. Given a context, $M$, we can
define a \emph{nominal context}, $\quotep{M}$ by $\quotep{M}[P] :=
\quotep{M[P]}$. To foreshadow what is to come we observe that these
operations enjoy a duality with processes very much like the duality
between vectors and maps from vectors to scalars.

Further, because the calculus is essentially higher-order, we have a
correspondence between contexts and processes. More specifically,
given a name $x$ and a context $M$ we can construct $M^{*}_{x}$ such
that 

\begin{mathpar}
  M^{*}_{x} | \lift{x}{P} \red M[P]
\end{mathpar}

namely,

\begin{mathpar}
  M^{*}_{x} := x?(u).M[\dropn{u}]
\end{mathpar}

The dependence of $M^{*}_{x}$ on a name makes it an abstraction, 

\begin{mathpar}
  M^{*} := (x)x?(u).M[\dropn{u}]
\end{mathpar}

\subsection{Additional notation}

It will sometimes be convenient to denote the process a name
quotes. We already have the notation $x = \quotep{P}$, but it will be
convenient to introduce an alternate notation, $\procn{x}$, when we
want to emphasize the connection to the use of the name. Note that, by
virtue of name equivalence, $\quotep{\procn{x}} \nameeq x$; so, the
notation is consistent with previous definitions.

Further, because names have structure it is possible to effect
substitutions on the basis of that structure. This means we need to
upgrade our notation for substitutions, which we accomplish by
adapting comprehension notation. Thus,

\begin{mathpar}
  P\{ y / x : x \in S \}
\end{mathpar}

is interpreted to mean the process derived from P by replacing (in a
capture-avoiding manner) each occurrence of $x$ in $S$ by $y$. For example,

\begin{mathpar}
  P\{ \quotep{\procn{x}|\procn{x}} / x : x \in \freenames{P} \}
\end{mathpar}

will replace each (occurrence) of a free name $x$ in $P$ by
$\quotep{\procn{x}|\procn{x}}$.

Also, we will avail ourselves of the notation $x^{L}$ and $x^{R}$ to
denote injections of a name into disjoint copies of the name
space. There are numerous ways to accomplish this. One example can be
found in \cite{MeredithR05}. This notation overloads to vectors of
names: $\vec{x}^{\pi} := (x_{i}^{\pi} \; : \; 0 \leq i < |\vec{x}| )$ where $\pi \in \{L,R\}$.

We also use $P^{\Box} := P|\Box$.

In \cite{MeredithR05} an interpretation of the new operator is
given. It turns out that there are several possible interpretations
all enjoying the requisite algebraic properties of the operator (see
\cite{milner91polyadicpi}). We will therefore make liberal use of
$(\nu\; \vec{x})P$.

% subsection the_syntax_and_semantics_of_the_notation_system (end)   

\input{qm2pi.qmops} 

\input{qm2pi.sterngerlach} 

\input{qm2pi.metric} 

% section concurrent_process_calculi (end)

%\input{qm2pi.proofsketch}

% section proof sketch (end)

%\input{qm2pi.slviaknots} 

% section spatial logic via knots (end)

\input{qm2pi.conclusion}

% section conclusion (end)

%\input{qm2pi.dtcodes} 

% section wiring algorithm (end)

\input{qm2pi.ack} 

% section acknowledgments (end)

\newpage


\bibliographystyle{plain}   
\bibliography{../../biblios/main.bib}

\input{qm2pi.rhodetails}

\end{document}

 

% section notation (end)

\input{qm2pi.process.calculi} 

% section concurrent_process_calculi_and_spatial_logics_ (end)
    
%\documentclass[12pt]{llncs}
%\documentclass{jktr}

\usepackage[pdftex]{hyperref}                   
\usepackage {listings}
\usepackage {mathpartir}
\usepackage{bcprules}
%\usepackage{listings}
                       
\usepackage{graphicx} 
%\usepackage[margins=2.5cm,nohead,nofoot]{geometry}
%\usepackage{geometry}
\usepackage{amsfonts}
\usepackage{amstext}
\usepackage{latexsym}
\usepackage{amssymb}
\usepackage{color}


%\include{myPreamble}
\include{qm2pi.local} 

%\ifpdf
%\usepackage[pdftex]{graphicx}
%\else
%\usepackage{graphicx}
%\fi

 % \ifpdf
%  \usepackage{pdfsync}
%  \if


%\title{Brief Article}
%\author{David F. Snyder}
%\author{L.G. Meredith}

%\address{Dept. of Math., Texas State University--San Marcos, San Marcos, TX 78666}
       
\pagestyle{empty}


\begin{document}

\lstset{language=[Objective]Caml,frame=shadowbox}

\input{qm2pi.front}

% section front matter (end)

\input{qm2pi.intro} 
 
% section introduction (end)

% \input{qm2pi.knotations} 

% section notation (end)

\input{qm2pi.process.calculi} 

% section concurrent_process_calculi_and_spatial_logics_ (end)
    
%\input{qm2pi.knots2pi} 

%\input{qm2pi.trefoil} 

%\input{qm2pi.mainthm} 

% subsection basic_interpretation (end)

%\input{qm2pi.rho.presentation} 
\subsection{The syntax and semantics of the notation system}\label{sub:the_syntax_and_semantics_of_the_notation_system} % (fold)

We now summarize a technical presentation of the calculus that
embodies our theory of dynamics. The typical presentation of such a
calculus follows the style of giving generators and relations on
them. The grammar, below, describing term constructors, freely
generates the set of processes, $\Proc$. This set is then quotiented
by a relation known as structural congruence and it is over this set
that the notion of dynamics is expressed. This presentation is
essentially that of \cite{MeredithR05} with the addition of
polyadicity and summation. For readability we have relegated some of
the technical subtleties to an appendix.

\subsubsection{Process grammar}\label{subsub:process_grammar}

\begin{mathpar}
  \inferrule* [lab=synchronization] {} {{M} \bc \pzero \;|\; x?F \;|\; x!C }
  \and
  \inferrule* [lab=abstraction] {} {{F} \bc (x)P}
  \and
  \inferrule* [lab=concretion] {} {{C} \bc \langle Q \rangle}
  \and
  \inferrule* [lab=process] {} {{P,Q} \bc M \;| \;P|Q \;|\; @{x}}
  \and
  \inferrule* [lab=name] {} {{x} \bc \quotep{P}}
\end{mathpar} 

Note that $\vec{x}$ (resp. $\vec{P}$) denotes a vector of names
(resp. processes) of length $|\vec{x}|$ (resp. $|\vec{P}|$). We adopt
the following useful abbreviations.

\begin{mathpar}
   x?(\vec{y}).P := x.(\vec{y})P \and  x\clift{\vec{P}} := x.\clift{\vec{P}}
   \and x!(y) := \lift{x}{\dropn{y}}
   \and \Pi_{i=0}^{n-1}P_i := P_0 | \ldots | P_{n-1}
\end{mathpar}

\subsubsection{Structural congruence}

\paragraph{Free and bound names and alpha-equivalence.} At the
core of structural equivalence is alpha-equivalence which identifies
process that are the same up to a change of variable. Formally, we
recognize the distinction between free and bound names. The free names
of a process, $\freenames{P}$, may be calculated recursively as
follows:

\begin{mathpar}
\freenames{\pzero} := \emptyset
  \and \\
  \freenames{x?(y).P} := \{ x \} \cup (\freenames{P} \setminus \{ y \})
  \and 
  \freenames{x!\langle P \rangle} := \{ x \} \cup \{ P \} 
  \and \\
  \freenames{P|Q} := \freenames{P} \cup \freenames{Q}
  \and \\
  \freenames{@{x}} := \{ x \}
\end{mathpar}

$\pi$
$\quotep{\pi}$

$\freenames{-} : \pi \to \mathcal{P}(\quotep{\pi})$

\begin{eqnarray*}
  \freenames{\pzero} & := & \emptyset \\
  \freenames{x?(y).P} & := & \{ x \} \cup (\freenames{P} \setminus \{ y \}) \\
  \freenames{x!\langle P \rangle} & := & \{ x \} \cup \{ P \} \\
  \freenames{P|Q} & := & \freenames{P} \cup \freenames{Q} \\
  \freenames{\dropn{x}} & := & \{ x \}
\end{eqnarray*}

The bound names of a process, $\boundnames{P}$, are those names occurring in $P$
that are not free. For example, in $x?(y).0$, the name $x$ is free, while $y$ is bound.

\begin{mathpar}
  \inferrule* [lab=monoidal-laws] {} { P|Q \equiv Q|P \and P|0 \equiv P \and P|(Q|R) \equiv (P|Q)|R }
\end{mathpar}

\begin{mathpar}
  \inferrule* [lab=alpha-equivalence] {} { (x)P \equiv (y)P\{y/x\} \and y \not\in \freenames{P} }
\end{mathpar}

\begin{definition}
Then two processes, $P,Q$, are alpha-equivalent if $P = Q\{\vec{y}/\vec{x}\}$ for
some $\vec{x} \in \boundnames{Q},\vec{y} \in \boundnames{P}$, where $Q\{\vec{y}/\vec{x}\}$
denotes the capture-avoiding substitution of $\vec{y}$ for $\vec{x}$ in $Q$.
\end{definition}

\begin{definition}
  The {\em structural congruence} \cite{SangiorgiWalker} , $\equiv$,
  between processes is the least congruence containing
  alpha-equivalence, satisfying the abelian monoid laws
  (associativity, commutativity and $\pzero$ as identity) for parallel
  composition $|$ and for summation $+$.
\end{definition}

\subsection{Name equivalence}

We take name equivalence, written $\nameeq$, to be the smallest
equivalence relation generated by the following rules.

\begin{mathpar}
\inferrule*[lab=Quote-drop]
{ }
{ \quotep{@{x}} \nameeq x }

\inferrule*[lab=Struct-equiv]
{ P \scong Q }
{ \quotep{P} \nameeq \quotep{Q} }
\end{mathpar}

The astute reader will have noticed that the mutual recursion of names
and processes imposes a mutual recursion on alpha-equivalence and
structural equivalence via name-equivalence. Fortunately, all of this
works out pleasantly and we may calculate in the natural way, free of
concern. The reader interested in the details is referred to the
appendix \ref{appendix:rho_details}.

\subsection{Substitution}

We use $\Proc$ for the set of processes, $\QProc$ for the set of
names, and $\id{\{}\vec{y} / \vec{x} \id{\}}$ to denote partial maps,
$s : \QProc \rightarrow \QProc$. A map, $s$ lifts, uniquely, to a map
on process terms, $\widehat{s} : \Proc \rightarrow \Proc$ by the
following equations.

\begin{mathpar}
  (0) \psubstp{Q}{P} := 0 \\
  (R \juxtap S) \psubstp{Q}{P}
  :=    
  (R)\psubstp{Q}{P} \juxtap (S) \psubstp{Q}{P} \\
  (x?(y).R) \psubstp{Q}{P}    
  :=    
  (x)\substp{Q}{P} (z)\concat( (R \psubstn{z}{y}) \psubstp{Q}{P} ) \\
  (\lift{x}{R}) \psubstp{Q}{P}  
  :=
  \lift{(x)\substp{Q}{P}}{ R \psubstp{Q}{P} } \\
%   (\dropn{x})  \psubstp{Q}{P}       
%   := 
%   \left\{ 
%     \begin{array}{ccc} 
%       \dropn{\quotep{Q}} & & x \nameeq \quotep{P} \\
%       \dropn{x} & & otherwise \\
%     \end{array}
%   \right. 
  (\dropn{x})  \psubstp{Q}{P}       
  := 
  \left\{ 
    \begin{array}{ccc} 
      Q & & x \nameeq \quotep{P} \\
      \dropn{x} & & otherwise \\
    \end{array}
  \right.
\end{mathpar}
 

where

\begin{eqnarray}
  (x)\id{\{} \lpquote Q \rpquote / \lpquote P \rpquote \id{\}}            = 
  \left\{ 
    \begin{array}{ccc}
      \lpquote Q \rpquote & & x \nameeq \lpquote P \rpquote \\
      x & & otherwise \\
    \end{array}
  \right. \nonumber
\end{eqnarray}

and $z$ is chosen distinct from $\quotep{P}$, $\quotep{Q}$, the free
names in $Q$, and all the names in $R$. Our $\alpha$-equivalence will
be built in the standard way from this substitution.

\begin{remark}\label{rem:no_self_referential_names}
  One consequence of these definitions is that $\forall P. \quotep{P}
  \not\in \freenames{P}$.
\end{remark}

\subsection{ Dynamic quote: an example }

Anticipating something of what's to come, consider applying the
substitution, $\widehat{\id{\{}u / z \id{\}}}$, to the following pair
of processes, $\lift{w}{y!(z)}$ and $w[ \lpquote y!(z) \rpquote ]$.

\begin{eqnarray}
	\lift{w}{y!(z)}\widehat{\id{\{}u / z \id{\}}}
		& = &
		\lift{w}{y!(u)} \nonumber\\
	w[ \lpquote y!(z) \rpquote ] \widehat{ \id{\{}u / z \id{\}} }
		& = &
		w[ \lpquote y!(z) \rpquote ] \nonumber
\end{eqnarray}

Because the body of the process between quotes is impervious to
substitution, we get radically different answers. In fact, by
examining the first process in an input context,
e.g. $x?(z).\lift{w}{y!(z)}$, we see that the process under the lift
operator may be shaped by prefixed inputs binding a name inside it. In
this sense, the lift operator will be seen as a way to dynamically
construct processes before reifying them as names.

Finally equipped with these standard features we can present the
dynamics of the calculus.

\subsubsection{Operational semantics} 

Finally, we introduce the computational dynamics. What marks these
algebras as distinct from other more traditionally studied algebraic
structures, e.g. vector spaces or polynomial rings, is the manner in
which dynamics is captured. In traditional structures, dynamics is typically
expressed through morphisms between such structures, as in linear maps
between vector spaces or morphisms between rings. In algebras
associated with the semantics of computation, the dynamics is
expressed as part of the algebraic structure itself, through a
reduction reduction relation typically denoted by $\red$. Below, we
give a recursive presentation of this relation for the calculus used
in the encoding.

$\red \subseteq \pi \times \pi$
$\red : \pi \to \mathcal{P}(\pi)$

\begin{mathpar}
  \inferrule* [lab=Comm] { \textsf{match}( x_{src}, x_{trgt} ) } { x_{trgt}?(y)P \; | \; x_{src}!\langle {Q} \rangle \red P\{\quotep{Q}/y}\} }
  \and \\
  \inferrule* [lab=Par] {{P} \red {P}'} {{{P} | {Q}} \red {{P}' | {Q}}}
  \and
  \inferrule* [lab=Equiv]{{{P} \scong {P}'} \andalso {{P}' \red {Q}'} \andalso {{Q}' \scong {Q}}}{{P} \red {Q}}
\end{mathpar}

\begin{eqnarray*}
  match_{\equiv} (\quotep{P},\quotep{Q}) & := & P \equiv Q \\
  match_{\dagger}(\quotep{P},\quotep{Q}) & := & \forall R. P|Q \red^{*} R => R \red^{*} 0 \\
  match_{K}(\quotep{P},\quotep{Q}) & := & K \mbox{ for some context } K
\end{eqnarray*}

$u?(x)P | u!\langle Q \rangle \red P\{\quotep{Q}/x\}$

%We write $\wred$ for $\red^*$, and $P\red$ if $\exists Q $ such that $ P \red Q$.
We write $P\red$ if $\exists Q $ such that $ P \red Q$ and $P\not\red$, otherwise.

\section{Replication}

As mentioned before, it is known that replication (and hence
recursion) can be implemented in a higher-order process algebra
\cite{SangiorgiWalker}. As our first example of calculation with the
machinery thus far presented we give the construction explicitly in
the {\rhoc}.

\begin{eqnarray}
	D_{x} & := & \prefix{x}{y}{(\binpar{\outputp{x}{y}}{@{y}})} \nonumber\\
	\bangp_{x}{P} & := & \binpar{{x}!\langle{\binpar{D_{x}}{P}}\rangle}{D_{x}} \nonumber
\end{eqnarray}

\begin{eqnarray}
	\bangp_{x}{P} & & \nonumber\\
	=
	& {x}!\langle{(\prefix{x}{y}{(\outputp{x}{y} | @{y})) | P}}\rangle 
	      | \prefix{x}{y}{(\outputp{x}{y} | @{y})} & \nonumber\\
	\red
	& (\outputp{x}{y} | @{y})\substn{\quotep{(\prefix{x}{y}{(@{y} | \outputp{x}{y})) | P}}}{y} & \nonumber\\
	=
	& \outputp{x}{\quotep{(\prefix{x}{y}{(\outputp{x}{y} | @{y})) | P}}}
	  | {(\prefix{x}{y}{(\outputp{x}{y} | @{y})) | P}} & \nonumber\\
	\red
	& \ldots & \nonumber\\
	\red^*
	& P | P | \ldots & \nonumber
\end{eqnarray}

Of course, this encoding, as an implementation, runs away, unfolding
$\bangp{P}$ eagerly. A lazier and more implementable replication
operator, restricted to input-guarded processes, may be obtained as follows.

\begin{eqnarray}
\bangp{\prefix{u}{v}{P}} 
	:= 
	\binpar{\lift{x}{\prefix{u}{v}{(\binpar{D(x)}{P})}}}{D(x)} \nonumber
\end{eqnarray}

\begin{remark}
  Note that the lazier definition still does not deal with summation
  or mixed summation (i.e. sums over input and output). The reader is
  invited to construct definitions of replication that deal with these
  features. 

  Further, the definitions are parameterized in a name, $x$. Can you,
  gentle reader, make a definition that eliminates this parameter and
  guarantees no accidental interaction between the replication
  machinery and the process being replicated -- i.e. no accidental
  sharing of names used by the process to get its work done and the
  name(s) used by the replication to effect copying. This latter
  revision of the definition of replication is crucial to obtaining
  the expected identity $!!P \sim !P$.
\end{remark}

\begin{remark}\label{rem:paradoxical_combinator}
  The reader familiar with the lambda calculus will have noticed the
  similarity between $D$ and the paradoxical combinator.

  [Ed. note: the existence of this seems to suggest we have to be more
  restrictive on the set of processes and names we admit if we are to
  support no-cloning.]
\end{remark}

\subsubsection{Bisimulation}

The computational dynamics gives rise to another kind of equivalence,
the equivalence of computational behavior. As previously mentioned
this is typically captured \emph{via} some form of bisimulation.

% The notion we use in this paper is weak barbed bisimulation
% \cite{milner91polyadicpi}.

The notion we use in this paper is derived from weak barbed
bisimulation \cite{milner91polyadicpi}. 

\begin{definition}
An \emph{observation relation}, $\downarrow_{\mathcal N}$, over a set
of names, $\mathcal N$, is the smallest relation satisfying the rules
below.

\infrule[Out-barb]{y \in {\mathcal N}, \; x \nameeq y}
		  {\outputp{x}{v} \downarrow_{\mathcal N} x}
\infrule[Par-barb]{\mbox{$P\downarrow_{\mathcal N} x$ or $Q\downarrow_{\mathcal N} x$}}
		  {\binpar{P}{Q} \downarrow_{\mathcal N} x}

We write $P \Downarrow_{\mathcal N} x$ if there is $Q$ such that 
$P \wred Q$ and $Q \downarrow_{\mathcal N} x$.
\end{definition}

\begin{definition}
%\label{def.bbisim}
An  ${\mathcal N}$-\emph{barbed bisimulation} over a set of names, ${\mathcal N}$, is a symmetric binary relation 
${\mathcal S}_{\mathcal N}$ between agents such that $P\rel{S}_{\mathcal N}Q$ implies:
\begin{enumerate}
\item If $P \red P'$ then $Q \wred Q'$ and $P'\rel{S}_{\mathcal N} Q'$.
\item If $P\downarrow_{\mathcal N} x$, then $Q\Downarrow_{\mathcal N} x$.
\end{enumerate}
$P$ is ${\mathcal N}$-barbed bisimilar to $Q$, written
$P \wbbisim_{\mathcal N} Q$, if $P \rel{S}_{\mathcal N} Q$ for some ${\mathcal N}$-barbed bisimulation ${\mathcal S}_{\mathcal N}$.
\end{definition}

$\mathcal{R} \subseteq \pi \times \pi$

$P \mathcal{R} Q => \forall P'. P \red P' \Rightarrow \exists Q'. Q \red Q', P' \mathcal{R} Q'$

$P \vdash x \Rightarrow Q \vdash x$

\begin{mathpar}
  \inferrule*[lab=Out-barb]{x \nameeq y}{{y}!\langle{Q}\rangle \vdash x}
  \and
  \inferrule*[lab=Par-barb]{\mbox{$P\vdash x$ or $Q\vdash x$}}{\binpar{P}{Q} \vdash x}
\end{mathpar}

\subsubsection{Contexts}

One of the principle advantages of computational calculi like the
$\pi$-calculus is a well-defined notion of context,
contextual-equivalence and a correlation between
contextual-equivalence and notions of bisimulation. The notion of
context allows the decomposition of a process into (sub-)process and
its syntactic environment, its context. Thus, a context may be
thought of as a process with a ``hole'' (written $\Box$) in it. The
application of a context $M$ to a process $P$, written $M[P]$, is
tantamount to filling the hole in $M$ with $P$. In this paper we do
not need the full weight of this theory, but do make use of the notion
of context in the proof the main theorem. 

\begin{mathpar}
  \inferrule* [lab=summation] {} {{M_{M},M_{N}} \bc \Box \;|\; x.M_{A} \;|\; M_{M}+M_{N}}
  \and
  \inferrule* [lab=agent] {} {{M_{A}} \bc (\vec{x})M_{P} \;| \; \clift{P_0,\ldots,M_{P},\ldots,P_N}}
  \and \\
  \inferrule* [lab=process] {} {{M_{P}} \bc M_{N} \;| \;P|M_{P} }
\end{mathpar} 

\begin{mathpar}
  \inferrule* [lab=sychronization] {} {M_{N} \bc \Box \;|\; x?M_{F} \;|\; x!M_{C}}
  \and
  \inferrule* [lab=abstraction] {} {{M_{F}} \bc (x)M_{P} }
  \and
  \inferrule* [lab=concretion] {} {{M_{C}} \bc \langle M_{P} \rangle }
  \and \\
  \inferrule* [lab=process] {} {{M_{P}} \bc M_{N} \;| \;P|M_{P} }
\end{mathpar}

\begin{definition}[contextual application] Given a context $M$, and
  process $P$, we define the \emph{contextual application}, $M[P] :=
  M\{P/\Box\}$. That is, the contextual application of M to P is the
  substitution of $P$ for $\Box$ in $M$.
\end{definition}

$\meaningof{-} : L \to \mathcal{P}(\pi)$

\begin{mathpar}
  \inferrule* [lab=collection] {} {\meaningof{true} = \pi, \and \meaningof{~E} = \pi \setminus \meaningof{E}, \and \meaningof{E_{1} \& E_{2}} = \meaningof{E_{1}} \cap \meaningof{E_{2}}}
\end{mathpar}

\begin{mathpar}
  \inferrule* [lab=structure] {} {\meaningof{0} = \{ P \in \pi | P \equiv 0 \}, \and \\ \meaningof{E_1 | E_2} = \{ P \in \pi | P \equiv P_{1} | P_{2}, P_{1} \in \meaningof{E_{1}}, P_{2} \in \meaningof{E_2}\} }
\end{mathpar}

\begin{mathpar}
 \inferrule* [lab=behavior] {} {\meaningof{\langle a?b \rangle E} = \{ P \in \pi | P \equiv Q | u?(y)P', \\ \and \\\\ \and \\ \;\;\; u \in \meaningof{a}, \forall z.P'\{z/y\} \in \meaningof{E\{z/b\}}\}, \and \\ \meaningof{a!E} = \{ P \in \pi | P \equiv Q | x!\langle P' \rangle, x \in \meaningof{a} P' \in \meaningof{E}\} }
\end{mathpar}

\begin{mathpar}
 \inferrule* [lab=nominal] {} {\meaningof{\quotep{E}} = \{ \quotep{P} \in \quotep{\pi} | P \in \meaningof{E} \}, \and \meaningof{\quotep{P}} = \{ \quotep{Q} \in \quotep{\pi} | P \equiv Q \} \and \\ \meaningof{@\quotep{E}} = \{ P \in \pi | P \equiv @x, x \in \meaningof{E} \}}
\end{mathpar}

\begin{eqnarray*}
  \\
  \meaningof{-} : TS \to ST
\end{eqnarray*}

\begin{eqnarray*}
  \\
  L : TS \to ST
\end{eqnarray*}

\begin{eqnarray*}
  \\
  P \models E \iff P \in \meaningof{E}
\end{eqnarray*}

\begin{eqnarray*}
  P \approx_{L} Q \iff \forall E \in L. P \models E \iff Q \models E
\end{eqnarray*}

\begin{eqnarray*}
  P \approx_{K} Q
\end{eqnarray*}

\begin{eqnarray*}
  P \approx Q
\end{eqnarray*}

$\approx_{K} = \approx = \approx_{L}$

\subsubsection{Contextual duality}

Note that contexts extend the quotation operation to a family of
operations from processes to names. Given a context, $M$, we can
define a \emph{nominal context}, $\quotep{M}$ by $\quotep{M}[P] :=
\quotep{M[P]}$. To foreshadow what is to come we observe that these
operations enjoy a duality with processes very much like the duality
between vectors and maps from vectors to scalars.

Further, because the calculus is essentially higher-order, we have a
correspondence between contexts and processes. More specifically,
given a name $x$ and a context $M$ we can construct $M^{*}_{x}$ such
that 

\begin{mathpar}
  M^{*}_{x} | \lift{x}{P} \red M[P]
\end{mathpar}

namely,

\begin{mathpar}
  M^{*}_{x} := x?(u).M[\dropn{u}]
\end{mathpar}

The dependence of $M^{*}_{x}$ on a name makes it an abstraction, 

\begin{mathpar}
  M^{*} := (x)x?(u).M[\dropn{u}]
\end{mathpar}

\subsection{Additional notation}

It will sometimes be convenient to denote the process a name
quotes. We already have the notation $x = \quotep{P}$, but it will be
convenient to introduce an alternate notation, $\procn{x}$, when we
want to emphasize the connection to the use of the name. Note that, by
virtue of name equivalence, $\quotep{\procn{x}} \nameeq x$; so, the
notation is consistent with previous definitions.

Further, because names have structure it is possible to effect
substitutions on the basis of that structure. This means we need to
upgrade our notation for substitutions, which we accomplish by
adapting comprehension notation. Thus,

\begin{mathpar}
  P\{ y / x : x \in S \}
\end{mathpar}

is interpreted to mean the process derived from P by replacing (in a
capture-avoiding manner) each occurrence of $x$ in $S$ by $y$. For example,

\begin{mathpar}
  P\{ \quotep{\procn{x}|\procn{x}} / x : x \in \freenames{P} \}
\end{mathpar}

will replace each (occurrence) of a free name $x$ in $P$ by
$\quotep{\procn{x}|\procn{x}}$.

Also, we will avail ourselves of the notation $x^{L}$ and $x^{R}$ to
denote injections of a name into disjoint copies of the name
space. There are numerous ways to accomplish this. One example can be
found in \cite{MeredithR05}. This notation overloads to vectors of
names: $\vec{x}^{\pi} := (x_{i}^{\pi} \; : \; 0 \leq i < |\vec{x}| )$ where $\pi \in \{L,R\}$.

We also use $P^{\Box} := P|\Box$.

In \cite{MeredithR05} an interpretation of the new operator is
given. It turns out that there are several possible interpretations
all enjoying the requisite algebraic properties of the operator (see
\cite{milner91polyadicpi}). We will therefore make liberal use of
$(\nu\; \vec{x})P$.

% subsection the_syntax_and_semantics_of_the_notation_system (end)   

\input{qm2pi.qmops} 

\input{qm2pi.sterngerlach} 

\input{qm2pi.metric} 

% section concurrent_process_calculi (end)

%\input{qm2pi.proofsketch}

% section proof sketch (end)

%\input{qm2pi.slviaknots} 

% section spatial logic via knots (end)

\input{qm2pi.conclusion}

% section conclusion (end)

%\input{qm2pi.dtcodes} 

% section wiring algorithm (end)

\input{qm2pi.ack} 

% section acknowledgments (end)

\newpage


\bibliographystyle{plain}   
\bibliography{../../biblios/main.bib}

\input{qm2pi.rhodetails}

\end{document}

 

%\documentclass[12pt]{llncs}
%\documentclass{jktr}

\usepackage[pdftex]{hyperref}                   
\usepackage {listings}
\usepackage {mathpartir}
\usepackage{bcprules}
%\usepackage{listings}
                       
\usepackage{graphicx} 
%\usepackage[margins=2.5cm,nohead,nofoot]{geometry}
%\usepackage{geometry}
\usepackage{amsfonts}
\usepackage{amstext}
\usepackage{latexsym}
\usepackage{amssymb}
\usepackage{color}


%\include{myPreamble}
\include{qm2pi.local} 

%\ifpdf
%\usepackage[pdftex]{graphicx}
%\else
%\usepackage{graphicx}
%\fi

 % \ifpdf
%  \usepackage{pdfsync}
%  \if


%\title{Brief Article}
%\author{David F. Snyder}
%\author{L.G. Meredith}

%\address{Dept. of Math., Texas State University--San Marcos, San Marcos, TX 78666}
       
\pagestyle{empty}


\begin{document}

\lstset{language=[Objective]Caml,frame=shadowbox}

\input{qm2pi.front}

% section front matter (end)

\input{qm2pi.intro} 
 
% section introduction (end)

% \input{qm2pi.knotations} 

% section notation (end)

\input{qm2pi.process.calculi} 

% section concurrent_process_calculi_and_spatial_logics_ (end)
    
%\input{qm2pi.knots2pi} 

%\input{qm2pi.trefoil} 

%\input{qm2pi.mainthm} 

% subsection basic_interpretation (end)

%\input{qm2pi.rho.presentation} 
\subsection{The syntax and semantics of the notation system}\label{sub:the_syntax_and_semantics_of_the_notation_system} % (fold)

We now summarize a technical presentation of the calculus that
embodies our theory of dynamics. The typical presentation of such a
calculus follows the style of giving generators and relations on
them. The grammar, below, describing term constructors, freely
generates the set of processes, $\Proc$. This set is then quotiented
by a relation known as structural congruence and it is over this set
that the notion of dynamics is expressed. This presentation is
essentially that of \cite{MeredithR05} with the addition of
polyadicity and summation. For readability we have relegated some of
the technical subtleties to an appendix.

\subsubsection{Process grammar}\label{subsub:process_grammar}

\begin{mathpar}
  \inferrule* [lab=synchronization] {} {{M} \bc \pzero \;|\; x?F \;|\; x!C }
  \and
  \inferrule* [lab=abstraction] {} {{F} \bc (x)P}
  \and
  \inferrule* [lab=concretion] {} {{C} \bc \langle Q \rangle}
  \and
  \inferrule* [lab=process] {} {{P,Q} \bc M \;| \;P|Q \;|\; @{x}}
  \and
  \inferrule* [lab=name] {} {{x} \bc \quotep{P}}
\end{mathpar} 

Note that $\vec{x}$ (resp. $\vec{P}$) denotes a vector of names
(resp. processes) of length $|\vec{x}|$ (resp. $|\vec{P}|$). We adopt
the following useful abbreviations.

\begin{mathpar}
   x?(\vec{y}).P := x.(\vec{y})P \and  x\clift{\vec{P}} := x.\clift{\vec{P}}
   \and x!(y) := \lift{x}{\dropn{y}}
   \and \Pi_{i=0}^{n-1}P_i := P_0 | \ldots | P_{n-1}
\end{mathpar}

\subsubsection{Structural congruence}

\paragraph{Free and bound names and alpha-equivalence.} At the
core of structural equivalence is alpha-equivalence which identifies
process that are the same up to a change of variable. Formally, we
recognize the distinction between free and bound names. The free names
of a process, $\freenames{P}$, may be calculated recursively as
follows:

\begin{mathpar}
\freenames{\pzero} := \emptyset
  \and \\
  \freenames{x?(y).P} := \{ x \} \cup (\freenames{P} \setminus \{ y \})
  \and 
  \freenames{x!\langle P \rangle} := \{ x \} \cup \{ P \} 
  \and \\
  \freenames{P|Q} := \freenames{P} \cup \freenames{Q}
  \and \\
  \freenames{@{x}} := \{ x \}
\end{mathpar}

$\pi$
$\quotep{\pi}$

$\freenames{-} : \pi \to \mathcal{P}(\quotep{\pi})$

\begin{eqnarray*}
  \freenames{\pzero} & := & \emptyset \\
  \freenames{x?(y).P} & := & \{ x \} \cup (\freenames{P} \setminus \{ y \}) \\
  \freenames{x!\langle P \rangle} & := & \{ x \} \cup \{ P \} \\
  \freenames{P|Q} & := & \freenames{P} \cup \freenames{Q} \\
  \freenames{\dropn{x}} & := & \{ x \}
\end{eqnarray*}

The bound names of a process, $\boundnames{P}$, are those names occurring in $P$
that are not free. For example, in $x?(y).0$, the name $x$ is free, while $y$ is bound.

\begin{mathpar}
  \inferrule* [lab=monoidal-laws] {} { P|Q \equiv Q|P \and P|0 \equiv P \and P|(Q|R) \equiv (P|Q)|R }
\end{mathpar}

\begin{mathpar}
  \inferrule* [lab=alpha-equivalence] {} { (x)P \equiv (y)P\{y/x\} \and y \not\in \freenames{P} }
\end{mathpar}

\begin{definition}
Then two processes, $P,Q$, are alpha-equivalent if $P = Q\{\vec{y}/\vec{x}\}$ for
some $\vec{x} \in \boundnames{Q},\vec{y} \in \boundnames{P}$, where $Q\{\vec{y}/\vec{x}\}$
denotes the capture-avoiding substitution of $\vec{y}$ for $\vec{x}$ in $Q$.
\end{definition}

\begin{definition}
  The {\em structural congruence} \cite{SangiorgiWalker} , $\equiv$,
  between processes is the least congruence containing
  alpha-equivalence, satisfying the abelian monoid laws
  (associativity, commutativity and $\pzero$ as identity) for parallel
  composition $|$ and for summation $+$.
\end{definition}

\subsection{Name equivalence}

We take name equivalence, written $\nameeq$, to be the smallest
equivalence relation generated by the following rules.

\begin{mathpar}
\inferrule*[lab=Quote-drop]
{ }
{ \quotep{@{x}} \nameeq x }

\inferrule*[lab=Struct-equiv]
{ P \scong Q }
{ \quotep{P} \nameeq \quotep{Q} }
\end{mathpar}

The astute reader will have noticed that the mutual recursion of names
and processes imposes a mutual recursion on alpha-equivalence and
structural equivalence via name-equivalence. Fortunately, all of this
works out pleasantly and we may calculate in the natural way, free of
concern. The reader interested in the details is referred to the
appendix \ref{appendix:rho_details}.

\subsection{Substitution}

We use $\Proc$ for the set of processes, $\QProc$ for the set of
names, and $\id{\{}\vec{y} / \vec{x} \id{\}}$ to denote partial maps,
$s : \QProc \rightarrow \QProc$. A map, $s$ lifts, uniquely, to a map
on process terms, $\widehat{s} : \Proc \rightarrow \Proc$ by the
following equations.

\begin{mathpar}
  (0) \psubstp{Q}{P} := 0 \\
  (R \juxtap S) \psubstp{Q}{P}
  :=    
  (R)\psubstp{Q}{P} \juxtap (S) \psubstp{Q}{P} \\
  (x?(y).R) \psubstp{Q}{P}    
  :=    
  (x)\substp{Q}{P} (z)\concat( (R \psubstn{z}{y}) \psubstp{Q}{P} ) \\
  (\lift{x}{R}) \psubstp{Q}{P}  
  :=
  \lift{(x)\substp{Q}{P}}{ R \psubstp{Q}{P} } \\
%   (\dropn{x})  \psubstp{Q}{P}       
%   := 
%   \left\{ 
%     \begin{array}{ccc} 
%       \dropn{\quotep{Q}} & & x \nameeq \quotep{P} \\
%       \dropn{x} & & otherwise \\
%     \end{array}
%   \right. 
  (\dropn{x})  \psubstp{Q}{P}       
  := 
  \left\{ 
    \begin{array}{ccc} 
      Q & & x \nameeq \quotep{P} \\
      \dropn{x} & & otherwise \\
    \end{array}
  \right.
\end{mathpar}
 

where

\begin{eqnarray}
  (x)\id{\{} \lpquote Q \rpquote / \lpquote P \rpquote \id{\}}            = 
  \left\{ 
    \begin{array}{ccc}
      \lpquote Q \rpquote & & x \nameeq \lpquote P \rpquote \\
      x & & otherwise \\
    \end{array}
  \right. \nonumber
\end{eqnarray}

and $z$ is chosen distinct from $\quotep{P}$, $\quotep{Q}$, the free
names in $Q$, and all the names in $R$. Our $\alpha$-equivalence will
be built in the standard way from this substitution.

\begin{remark}\label{rem:no_self_referential_names}
  One consequence of these definitions is that $\forall P. \quotep{P}
  \not\in \freenames{P}$.
\end{remark}

\subsection{ Dynamic quote: an example }

Anticipating something of what's to come, consider applying the
substitution, $\widehat{\id{\{}u / z \id{\}}}$, to the following pair
of processes, $\lift{w}{y!(z)}$ and $w[ \lpquote y!(z) \rpquote ]$.

\begin{eqnarray}
	\lift{w}{y!(z)}\widehat{\id{\{}u / z \id{\}}}
		& = &
		\lift{w}{y!(u)} \nonumber\\
	w[ \lpquote y!(z) \rpquote ] \widehat{ \id{\{}u / z \id{\}} }
		& = &
		w[ \lpquote y!(z) \rpquote ] \nonumber
\end{eqnarray}

Because the body of the process between quotes is impervious to
substitution, we get radically different answers. In fact, by
examining the first process in an input context,
e.g. $x?(z).\lift{w}{y!(z)}$, we see that the process under the lift
operator may be shaped by prefixed inputs binding a name inside it. In
this sense, the lift operator will be seen as a way to dynamically
construct processes before reifying them as names.

Finally equipped with these standard features we can present the
dynamics of the calculus.

\subsubsection{Operational semantics} 

Finally, we introduce the computational dynamics. What marks these
algebras as distinct from other more traditionally studied algebraic
structures, e.g. vector spaces or polynomial rings, is the manner in
which dynamics is captured. In traditional structures, dynamics is typically
expressed through morphisms between such structures, as in linear maps
between vector spaces or morphisms between rings. In algebras
associated with the semantics of computation, the dynamics is
expressed as part of the algebraic structure itself, through a
reduction reduction relation typically denoted by $\red$. Below, we
give a recursive presentation of this relation for the calculus used
in the encoding.

$\red \subseteq \pi \times \pi$
$\red : \pi \to \mathcal{P}(\pi)$

\begin{mathpar}
  \inferrule* [lab=Comm] { \textsf{match}( x_{src}, x_{trgt} ) } { x_{trgt}?(y)P \; | \; x_{src}!\langle {Q} \rangle \red P\{\quotep{Q}/y}\} }
  \and \\
  \inferrule* [lab=Par] {{P} \red {P}'} {{{P} | {Q}} \red {{P}' | {Q}}}
  \and
  \inferrule* [lab=Equiv]{{{P} \scong {P}'} \andalso {{P}' \red {Q}'} \andalso {{Q}' \scong {Q}}}{{P} \red {Q}}
\end{mathpar}

\begin{eqnarray*}
  match_{\equiv} (\quotep{P},\quotep{Q}) & := & P \equiv Q \\
  match_{\dagger}(\quotep{P},\quotep{Q}) & := & \forall R. P|Q \red^{*} R => R \red^{*} 0 \\
  match_{K}(\quotep{P},\quotep{Q}) & := & K \mbox{ for some context } K
\end{eqnarray*}

$u?(x)P | u!\langle Q \rangle \red P\{\quotep{Q}/x\}$

%We write $\wred$ for $\red^*$, and $P\red$ if $\exists Q $ such that $ P \red Q$.
We write $P\red$ if $\exists Q $ such that $ P \red Q$ and $P\not\red$, otherwise.

\section{Replication}

As mentioned before, it is known that replication (and hence
recursion) can be implemented in a higher-order process algebra
\cite{SangiorgiWalker}. As our first example of calculation with the
machinery thus far presented we give the construction explicitly in
the {\rhoc}.

\begin{eqnarray}
	D_{x} & := & \prefix{x}{y}{(\binpar{\outputp{x}{y}}{@{y}})} \nonumber\\
	\bangp_{x}{P} & := & \binpar{{x}!\langle{\binpar{D_{x}}{P}}\rangle}{D_{x}} \nonumber
\end{eqnarray}

\begin{eqnarray}
	\bangp_{x}{P} & & \nonumber\\
	=
	& {x}!\langle{(\prefix{x}{y}{(\outputp{x}{y} | @{y})) | P}}\rangle 
	      | \prefix{x}{y}{(\outputp{x}{y} | @{y})} & \nonumber\\
	\red
	& (\outputp{x}{y} | @{y})\substn{\quotep{(\prefix{x}{y}{(@{y} | \outputp{x}{y})) | P}}}{y} & \nonumber\\
	=
	& \outputp{x}{\quotep{(\prefix{x}{y}{(\outputp{x}{y} | @{y})) | P}}}
	  | {(\prefix{x}{y}{(\outputp{x}{y} | @{y})) | P}} & \nonumber\\
	\red
	& \ldots & \nonumber\\
	\red^*
	& P | P | \ldots & \nonumber
\end{eqnarray}

Of course, this encoding, as an implementation, runs away, unfolding
$\bangp{P}$ eagerly. A lazier and more implementable replication
operator, restricted to input-guarded processes, may be obtained as follows.

\begin{eqnarray}
\bangp{\prefix{u}{v}{P}} 
	:= 
	\binpar{\lift{x}{\prefix{u}{v}{(\binpar{D(x)}{P})}}}{D(x)} \nonumber
\end{eqnarray}

\begin{remark}
  Note that the lazier definition still does not deal with summation
  or mixed summation (i.e. sums over input and output). The reader is
  invited to construct definitions of replication that deal with these
  features. 

  Further, the definitions are parameterized in a name, $x$. Can you,
  gentle reader, make a definition that eliminates this parameter and
  guarantees no accidental interaction between the replication
  machinery and the process being replicated -- i.e. no accidental
  sharing of names used by the process to get its work done and the
  name(s) used by the replication to effect copying. This latter
  revision of the definition of replication is crucial to obtaining
  the expected identity $!!P \sim !P$.
\end{remark}

\begin{remark}\label{rem:paradoxical_combinator}
  The reader familiar with the lambda calculus will have noticed the
  similarity between $D$ and the paradoxical combinator.

  [Ed. note: the existence of this seems to suggest we have to be more
  restrictive on the set of processes and names we admit if we are to
  support no-cloning.]
\end{remark}

\subsubsection{Bisimulation}

The computational dynamics gives rise to another kind of equivalence,
the equivalence of computational behavior. As previously mentioned
this is typically captured \emph{via} some form of bisimulation.

% The notion we use in this paper is weak barbed bisimulation
% \cite{milner91polyadicpi}.

The notion we use in this paper is derived from weak barbed
bisimulation \cite{milner91polyadicpi}. 

\begin{definition}
An \emph{observation relation}, $\downarrow_{\mathcal N}$, over a set
of names, $\mathcal N$, is the smallest relation satisfying the rules
below.

\infrule[Out-barb]{y \in {\mathcal N}, \; x \nameeq y}
		  {\outputp{x}{v} \downarrow_{\mathcal N} x}
\infrule[Par-barb]{\mbox{$P\downarrow_{\mathcal N} x$ or $Q\downarrow_{\mathcal N} x$}}
		  {\binpar{P}{Q} \downarrow_{\mathcal N} x}

We write $P \Downarrow_{\mathcal N} x$ if there is $Q$ such that 
$P \wred Q$ and $Q \downarrow_{\mathcal N} x$.
\end{definition}

\begin{definition}
%\label{def.bbisim}
An  ${\mathcal N}$-\emph{barbed bisimulation} over a set of names, ${\mathcal N}$, is a symmetric binary relation 
${\mathcal S}_{\mathcal N}$ between agents such that $P\rel{S}_{\mathcal N}Q$ implies:
\begin{enumerate}
\item If $P \red P'$ then $Q \wred Q'$ and $P'\rel{S}_{\mathcal N} Q'$.
\item If $P\downarrow_{\mathcal N} x$, then $Q\Downarrow_{\mathcal N} x$.
\end{enumerate}
$P$ is ${\mathcal N}$-barbed bisimilar to $Q$, written
$P \wbbisim_{\mathcal N} Q$, if $P \rel{S}_{\mathcal N} Q$ for some ${\mathcal N}$-barbed bisimulation ${\mathcal S}_{\mathcal N}$.
\end{definition}

$\mathcal{R} \subseteq \pi \times \pi$

$P \mathcal{R} Q => \forall P'. P \red P' \Rightarrow \exists Q'. Q \red Q', P' \mathcal{R} Q'$

$P \vdash x \Rightarrow Q \vdash x$

\begin{mathpar}
  \inferrule*[lab=Out-barb]{x \nameeq y}{{y}!\langle{Q}\rangle \vdash x}
  \and
  \inferrule*[lab=Par-barb]{\mbox{$P\vdash x$ or $Q\vdash x$}}{\binpar{P}{Q} \vdash x}
\end{mathpar}

\subsubsection{Contexts}

One of the principle advantages of computational calculi like the
$\pi$-calculus is a well-defined notion of context,
contextual-equivalence and a correlation between
contextual-equivalence and notions of bisimulation. The notion of
context allows the decomposition of a process into (sub-)process and
its syntactic environment, its context. Thus, a context may be
thought of as a process with a ``hole'' (written $\Box$) in it. The
application of a context $M$ to a process $P$, written $M[P]$, is
tantamount to filling the hole in $M$ with $P$. In this paper we do
not need the full weight of this theory, but do make use of the notion
of context in the proof the main theorem. 

\begin{mathpar}
  \inferrule* [lab=summation] {} {{M_{M},M_{N}} \bc \Box \;|\; x.M_{A} \;|\; M_{M}+M_{N}}
  \and
  \inferrule* [lab=agent] {} {{M_{A}} \bc (\vec{x})M_{P} \;| \; \clift{P_0,\ldots,M_{P},\ldots,P_N}}
  \and \\
  \inferrule* [lab=process] {} {{M_{P}} \bc M_{N} \;| \;P|M_{P} }
\end{mathpar} 

\begin{mathpar}
  \inferrule* [lab=sychronization] {} {M_{N} \bc \Box \;|\; x?M_{F} \;|\; x!M_{C}}
  \and
  \inferrule* [lab=abstraction] {} {{M_{F}} \bc (x)M_{P} }
  \and
  \inferrule* [lab=concretion] {} {{M_{C}} \bc \langle M_{P} \rangle }
  \and \\
  \inferrule* [lab=process] {} {{M_{P}} \bc M_{N} \;| \;P|M_{P} }
\end{mathpar}

\begin{definition}[contextual application] Given a context $M$, and
  process $P$, we define the \emph{contextual application}, $M[P] :=
  M\{P/\Box\}$. That is, the contextual application of M to P is the
  substitution of $P$ for $\Box$ in $M$.
\end{definition}

$\meaningof{-} : L \to \mathcal{P}(\pi)$

\begin{mathpar}
  \inferrule* [lab=collection] {} {\meaningof{true} = \pi, \and \meaningof{~E} = \pi \setminus \meaningof{E}, \and \meaningof{E_{1} \& E_{2}} = \meaningof{E_{1}} \cap \meaningof{E_{2}}}
\end{mathpar}

\begin{mathpar}
  \inferrule* [lab=structure] {} {\meaningof{0} = \{ P \in \pi | P \equiv 0 \}, \and \\ \meaningof{E_1 | E_2} = \{ P \in \pi | P \equiv P_{1} | P_{2}, P_{1} \in \meaningof{E_{1}}, P_{2} \in \meaningof{E_2}\} }
\end{mathpar}

\begin{mathpar}
 \inferrule* [lab=behavior] {} {\meaningof{\langle a?b \rangle E} = \{ P \in \pi | P \equiv Q | u?(y)P', \\ \and \\\\ \and \\ \;\;\; u \in \meaningof{a}, \forall z.P'\{z/y\} \in \meaningof{E\{z/b\}}\}, \and \\ \meaningof{a!E} = \{ P \in \pi | P \equiv Q | x!\langle P' \rangle, x \in \meaningof{a} P' \in \meaningof{E}\} }
\end{mathpar}

\begin{mathpar}
 \inferrule* [lab=nominal] {} {\meaningof{\quotep{E}} = \{ \quotep{P} \in \quotep{\pi} | P \in \meaningof{E} \}, \and \meaningof{\quotep{P}} = \{ \quotep{Q} \in \quotep{\pi} | P \equiv Q \} \and \\ \meaningof{@\quotep{E}} = \{ P \in \pi | P \equiv @x, x \in \meaningof{E} \}}
\end{mathpar}

\begin{eqnarray*}
  \\
  \meaningof{-} : TS \to ST
\end{eqnarray*}

\begin{eqnarray*}
  \\
  L : TS \to ST
\end{eqnarray*}

\begin{eqnarray*}
  \\
  P \models E \iff P \in \meaningof{E}
\end{eqnarray*}

\begin{eqnarray*}
  P \approx_{L} Q \iff \forall E \in L. P \models E \iff Q \models E
\end{eqnarray*}

\begin{eqnarray*}
  P \approx_{K} Q
\end{eqnarray*}

\begin{eqnarray*}
  P \approx Q
\end{eqnarray*}

$\approx_{K} = \approx = \approx_{L}$

\subsubsection{Contextual duality}

Note that contexts extend the quotation operation to a family of
operations from processes to names. Given a context, $M$, we can
define a \emph{nominal context}, $\quotep{M}$ by $\quotep{M}[P] :=
\quotep{M[P]}$. To foreshadow what is to come we observe that these
operations enjoy a duality with processes very much like the duality
between vectors and maps from vectors to scalars.

Further, because the calculus is essentially higher-order, we have a
correspondence between contexts and processes. More specifically,
given a name $x$ and a context $M$ we can construct $M^{*}_{x}$ such
that 

\begin{mathpar}
  M^{*}_{x} | \lift{x}{P} \red M[P]
\end{mathpar}

namely,

\begin{mathpar}
  M^{*}_{x} := x?(u).M[\dropn{u}]
\end{mathpar}

The dependence of $M^{*}_{x}$ on a name makes it an abstraction, 

\begin{mathpar}
  M^{*} := (x)x?(u).M[\dropn{u}]
\end{mathpar}

\subsection{Additional notation}

It will sometimes be convenient to denote the process a name
quotes. We already have the notation $x = \quotep{P}$, but it will be
convenient to introduce an alternate notation, $\procn{x}$, when we
want to emphasize the connection to the use of the name. Note that, by
virtue of name equivalence, $\quotep{\procn{x}} \nameeq x$; so, the
notation is consistent with previous definitions.

Further, because names have structure it is possible to effect
substitutions on the basis of that structure. This means we need to
upgrade our notation for substitutions, which we accomplish by
adapting comprehension notation. Thus,

\begin{mathpar}
  P\{ y / x : x \in S \}
\end{mathpar}

is interpreted to mean the process derived from P by replacing (in a
capture-avoiding manner) each occurrence of $x$ in $S$ by $y$. For example,

\begin{mathpar}
  P\{ \quotep{\procn{x}|\procn{x}} / x : x \in \freenames{P} \}
\end{mathpar}

will replace each (occurrence) of a free name $x$ in $P$ by
$\quotep{\procn{x}|\procn{x}}$.

Also, we will avail ourselves of the notation $x^{L}$ and $x^{R}$ to
denote injections of a name into disjoint copies of the name
space. There are numerous ways to accomplish this. One example can be
found in \cite{MeredithR05}. This notation overloads to vectors of
names: $\vec{x}^{\pi} := (x_{i}^{\pi} \; : \; 0 \leq i < |\vec{x}| )$ where $\pi \in \{L,R\}$.

We also use $P^{\Box} := P|\Box$.

In \cite{MeredithR05} an interpretation of the new operator is
given. It turns out that there are several possible interpretations
all enjoying the requisite algebraic properties of the operator (see
\cite{milner91polyadicpi}). We will therefore make liberal use of
$(\nu\; \vec{x})P$.

% subsection the_syntax_and_semantics_of_the_notation_system (end)   

\input{qm2pi.qmops} 

\input{qm2pi.sterngerlach} 

\input{qm2pi.metric} 

% section concurrent_process_calculi (end)

%\input{qm2pi.proofsketch}

% section proof sketch (end)

%\input{qm2pi.slviaknots} 

% section spatial logic via knots (end)

\input{qm2pi.conclusion}

% section conclusion (end)

%\input{qm2pi.dtcodes} 

% section wiring algorithm (end)

\input{qm2pi.ack} 

% section acknowledgments (end)

\newpage


\bibliographystyle{plain}   
\bibliography{../../biblios/main.bib}

\input{qm2pi.rhodetails}

\end{document}

 

%\documentclass[12pt]{llncs}
%\documentclass{jktr}

\usepackage[pdftex]{hyperref}                   
\usepackage {listings}
\usepackage {mathpartir}
\usepackage{bcprules}
%\usepackage{listings}
                       
\usepackage{graphicx} 
%\usepackage[margins=2.5cm,nohead,nofoot]{geometry}
%\usepackage{geometry}
\usepackage{amsfonts}
\usepackage{amstext}
\usepackage{latexsym}
\usepackage{amssymb}
\usepackage{color}


%\include{myPreamble}
\include{qm2pi.local} 

%\ifpdf
%\usepackage[pdftex]{graphicx}
%\else
%\usepackage{graphicx}
%\fi

 % \ifpdf
%  \usepackage{pdfsync}
%  \if


%\title{Brief Article}
%\author{David F. Snyder}
%\author{L.G. Meredith}

%\address{Dept. of Math., Texas State University--San Marcos, San Marcos, TX 78666}
       
\pagestyle{empty}


\begin{document}

\lstset{language=[Objective]Caml,frame=shadowbox}

\input{qm2pi.front}

% section front matter (end)

\input{qm2pi.intro} 
 
% section introduction (end)

% \input{qm2pi.knotations} 

% section notation (end)

\input{qm2pi.process.calculi} 

% section concurrent_process_calculi_and_spatial_logics_ (end)
    
%\input{qm2pi.knots2pi} 

%\input{qm2pi.trefoil} 

%\input{qm2pi.mainthm} 

% subsection basic_interpretation (end)

%\input{qm2pi.rho.presentation} 
\subsection{The syntax and semantics of the notation system}\label{sub:the_syntax_and_semantics_of_the_notation_system} % (fold)

We now summarize a technical presentation of the calculus that
embodies our theory of dynamics. The typical presentation of such a
calculus follows the style of giving generators and relations on
them. The grammar, below, describing term constructors, freely
generates the set of processes, $\Proc$. This set is then quotiented
by a relation known as structural congruence and it is over this set
that the notion of dynamics is expressed. This presentation is
essentially that of \cite{MeredithR05} with the addition of
polyadicity and summation. For readability we have relegated some of
the technical subtleties to an appendix.

\subsubsection{Process grammar}\label{subsub:process_grammar}

\begin{mathpar}
  \inferrule* [lab=synchronization] {} {{M} \bc \pzero \;|\; x?F \;|\; x!C }
  \and
  \inferrule* [lab=abstraction] {} {{F} \bc (x)P}
  \and
  \inferrule* [lab=concretion] {} {{C} \bc \langle Q \rangle}
  \and
  \inferrule* [lab=process] {} {{P,Q} \bc M \;| \;P|Q \;|\; @{x}}
  \and
  \inferrule* [lab=name] {} {{x} \bc \quotep{P}}
\end{mathpar} 

Note that $\vec{x}$ (resp. $\vec{P}$) denotes a vector of names
(resp. processes) of length $|\vec{x}|$ (resp. $|\vec{P}|$). We adopt
the following useful abbreviations.

\begin{mathpar}
   x?(\vec{y}).P := x.(\vec{y})P \and  x\clift{\vec{P}} := x.\clift{\vec{P}}
   \and x!(y) := \lift{x}{\dropn{y}}
   \and \Pi_{i=0}^{n-1}P_i := P_0 | \ldots | P_{n-1}
\end{mathpar}

\subsubsection{Structural congruence}

\paragraph{Free and bound names and alpha-equivalence.} At the
core of structural equivalence is alpha-equivalence which identifies
process that are the same up to a change of variable. Formally, we
recognize the distinction between free and bound names. The free names
of a process, $\freenames{P}$, may be calculated recursively as
follows:

\begin{mathpar}
\freenames{\pzero} := \emptyset
  \and \\
  \freenames{x?(y).P} := \{ x \} \cup (\freenames{P} \setminus \{ y \})
  \and 
  \freenames{x!\langle P \rangle} := \{ x \} \cup \{ P \} 
  \and \\
  \freenames{P|Q} := \freenames{P} \cup \freenames{Q}
  \and \\
  \freenames{@{x}} := \{ x \}
\end{mathpar}

$\pi$
$\quotep{\pi}$

$\freenames{-} : \pi \to \mathcal{P}(\quotep{\pi})$

\begin{eqnarray*}
  \freenames{\pzero} & := & \emptyset \\
  \freenames{x?(y).P} & := & \{ x \} \cup (\freenames{P} \setminus \{ y \}) \\
  \freenames{x!\langle P \rangle} & := & \{ x \} \cup \{ P \} \\
  \freenames{P|Q} & := & \freenames{P} \cup \freenames{Q} \\
  \freenames{\dropn{x}} & := & \{ x \}
\end{eqnarray*}

The bound names of a process, $\boundnames{P}$, are those names occurring in $P$
that are not free. For example, in $x?(y).0$, the name $x$ is free, while $y$ is bound.

\begin{mathpar}
  \inferrule* [lab=monoidal-laws] {} { P|Q \equiv Q|P \and P|0 \equiv P \and P|(Q|R) \equiv (P|Q)|R }
\end{mathpar}

\begin{mathpar}
  \inferrule* [lab=alpha-equivalence] {} { (x)P \equiv (y)P\{y/x\} \and y \not\in \freenames{P} }
\end{mathpar}

\begin{definition}
Then two processes, $P,Q$, are alpha-equivalent if $P = Q\{\vec{y}/\vec{x}\}$ for
some $\vec{x} \in \boundnames{Q},\vec{y} \in \boundnames{P}$, where $Q\{\vec{y}/\vec{x}\}$
denotes the capture-avoiding substitution of $\vec{y}$ for $\vec{x}$ in $Q$.
\end{definition}

\begin{definition}
  The {\em structural congruence} \cite{SangiorgiWalker} , $\equiv$,
  between processes is the least congruence containing
  alpha-equivalence, satisfying the abelian monoid laws
  (associativity, commutativity and $\pzero$ as identity) for parallel
  composition $|$ and for summation $+$.
\end{definition}

\subsection{Name equivalence}

We take name equivalence, written $\nameeq$, to be the smallest
equivalence relation generated by the following rules.

\begin{mathpar}
\inferrule*[lab=Quote-drop]
{ }
{ \quotep{@{x}} \nameeq x }

\inferrule*[lab=Struct-equiv]
{ P \scong Q }
{ \quotep{P} \nameeq \quotep{Q} }
\end{mathpar}

The astute reader will have noticed that the mutual recursion of names
and processes imposes a mutual recursion on alpha-equivalence and
structural equivalence via name-equivalence. Fortunately, all of this
works out pleasantly and we may calculate in the natural way, free of
concern. The reader interested in the details is referred to the
appendix \ref{appendix:rho_details}.

\subsection{Substitution}

We use $\Proc$ for the set of processes, $\QProc$ for the set of
names, and $\id{\{}\vec{y} / \vec{x} \id{\}}$ to denote partial maps,
$s : \QProc \rightarrow \QProc$. A map, $s$ lifts, uniquely, to a map
on process terms, $\widehat{s} : \Proc \rightarrow \Proc$ by the
following equations.

\begin{mathpar}
  (0) \psubstp{Q}{P} := 0 \\
  (R \juxtap S) \psubstp{Q}{P}
  :=    
  (R)\psubstp{Q}{P} \juxtap (S) \psubstp{Q}{P} \\
  (x?(y).R) \psubstp{Q}{P}    
  :=    
  (x)\substp{Q}{P} (z)\concat( (R \psubstn{z}{y}) \psubstp{Q}{P} ) \\
  (\lift{x}{R}) \psubstp{Q}{P}  
  :=
  \lift{(x)\substp{Q}{P}}{ R \psubstp{Q}{P} } \\
%   (\dropn{x})  \psubstp{Q}{P}       
%   := 
%   \left\{ 
%     \begin{array}{ccc} 
%       \dropn{\quotep{Q}} & & x \nameeq \quotep{P} \\
%       \dropn{x} & & otherwise \\
%     \end{array}
%   \right. 
  (\dropn{x})  \psubstp{Q}{P}       
  := 
  \left\{ 
    \begin{array}{ccc} 
      Q & & x \nameeq \quotep{P} \\
      \dropn{x} & & otherwise \\
    \end{array}
  \right.
\end{mathpar}
 

where

\begin{eqnarray}
  (x)\id{\{} \lpquote Q \rpquote / \lpquote P \rpquote \id{\}}            = 
  \left\{ 
    \begin{array}{ccc}
      \lpquote Q \rpquote & & x \nameeq \lpquote P \rpquote \\
      x & & otherwise \\
    \end{array}
  \right. \nonumber
\end{eqnarray}

and $z$ is chosen distinct from $\quotep{P}$, $\quotep{Q}$, the free
names in $Q$, and all the names in $R$. Our $\alpha$-equivalence will
be built in the standard way from this substitution.

\begin{remark}\label{rem:no_self_referential_names}
  One consequence of these definitions is that $\forall P. \quotep{P}
  \not\in \freenames{P}$.
\end{remark}

\subsection{ Dynamic quote: an example }

Anticipating something of what's to come, consider applying the
substitution, $\widehat{\id{\{}u / z \id{\}}}$, to the following pair
of processes, $\lift{w}{y!(z)}$ and $w[ \lpquote y!(z) \rpquote ]$.

\begin{eqnarray}
	\lift{w}{y!(z)}\widehat{\id{\{}u / z \id{\}}}
		& = &
		\lift{w}{y!(u)} \nonumber\\
	w[ \lpquote y!(z) \rpquote ] \widehat{ \id{\{}u / z \id{\}} }
		& = &
		w[ \lpquote y!(z) \rpquote ] \nonumber
\end{eqnarray}

Because the body of the process between quotes is impervious to
substitution, we get radically different answers. In fact, by
examining the first process in an input context,
e.g. $x?(z).\lift{w}{y!(z)}$, we see that the process under the lift
operator may be shaped by prefixed inputs binding a name inside it. In
this sense, the lift operator will be seen as a way to dynamically
construct processes before reifying them as names.

Finally equipped with these standard features we can present the
dynamics of the calculus.

\subsubsection{Operational semantics} 

Finally, we introduce the computational dynamics. What marks these
algebras as distinct from other more traditionally studied algebraic
structures, e.g. vector spaces or polynomial rings, is the manner in
which dynamics is captured. In traditional structures, dynamics is typically
expressed through morphisms between such structures, as in linear maps
between vector spaces or morphisms between rings. In algebras
associated with the semantics of computation, the dynamics is
expressed as part of the algebraic structure itself, through a
reduction reduction relation typically denoted by $\red$. Below, we
give a recursive presentation of this relation for the calculus used
in the encoding.

$\red \subseteq \pi \times \pi$
$\red : \pi \to \mathcal{P}(\pi)$

\begin{mathpar}
  \inferrule* [lab=Comm] { \textsf{match}( x_{src}, x_{trgt} ) } { x_{trgt}?(y)P \; | \; x_{src}!\langle {Q} \rangle \red P\{\quotep{Q}/y}\} }
  \and \\
  \inferrule* [lab=Par] {{P} \red {P}'} {{{P} | {Q}} \red {{P}' | {Q}}}
  \and
  \inferrule* [lab=Equiv]{{{P} \scong {P}'} \andalso {{P}' \red {Q}'} \andalso {{Q}' \scong {Q}}}{{P} \red {Q}}
\end{mathpar}

\begin{eqnarray*}
  match_{\equiv} (\quotep{P},\quotep{Q}) & := & P \equiv Q \\
  match_{\dagger}(\quotep{P},\quotep{Q}) & := & \forall R. P|Q \red^{*} R => R \red^{*} 0 \\
  match_{K}(\quotep{P},\quotep{Q}) & := & K \mbox{ for some context } K
\end{eqnarray*}

$u?(x)P | u!\langle Q \rangle \red P\{\quotep{Q}/x\}$

%We write $\wred$ for $\red^*$, and $P\red$ if $\exists Q $ such that $ P \red Q$.
We write $P\red$ if $\exists Q $ such that $ P \red Q$ and $P\not\red$, otherwise.

\section{Replication}

As mentioned before, it is known that replication (and hence
recursion) can be implemented in a higher-order process algebra
\cite{SangiorgiWalker}. As our first example of calculation with the
machinery thus far presented we give the construction explicitly in
the {\rhoc}.

\begin{eqnarray}
	D_{x} & := & \prefix{x}{y}{(\binpar{\outputp{x}{y}}{@{y}})} \nonumber\\
	\bangp_{x}{P} & := & \binpar{{x}!\langle{\binpar{D_{x}}{P}}\rangle}{D_{x}} \nonumber
\end{eqnarray}

\begin{eqnarray}
	\bangp_{x}{P} & & \nonumber\\
	=
	& {x}!\langle{(\prefix{x}{y}{(\outputp{x}{y} | @{y})) | P}}\rangle 
	      | \prefix{x}{y}{(\outputp{x}{y} | @{y})} & \nonumber\\
	\red
	& (\outputp{x}{y} | @{y})\substn{\quotep{(\prefix{x}{y}{(@{y} | \outputp{x}{y})) | P}}}{y} & \nonumber\\
	=
	& \outputp{x}{\quotep{(\prefix{x}{y}{(\outputp{x}{y} | @{y})) | P}}}
	  | {(\prefix{x}{y}{(\outputp{x}{y} | @{y})) | P}} & \nonumber\\
	\red
	& \ldots & \nonumber\\
	\red^*
	& P | P | \ldots & \nonumber
\end{eqnarray}

Of course, this encoding, as an implementation, runs away, unfolding
$\bangp{P}$ eagerly. A lazier and more implementable replication
operator, restricted to input-guarded processes, may be obtained as follows.

\begin{eqnarray}
\bangp{\prefix{u}{v}{P}} 
	:= 
	\binpar{\lift{x}{\prefix{u}{v}{(\binpar{D(x)}{P})}}}{D(x)} \nonumber
\end{eqnarray}

\begin{remark}
  Note that the lazier definition still does not deal with summation
  or mixed summation (i.e. sums over input and output). The reader is
  invited to construct definitions of replication that deal with these
  features. 

  Further, the definitions are parameterized in a name, $x$. Can you,
  gentle reader, make a definition that eliminates this parameter and
  guarantees no accidental interaction between the replication
  machinery and the process being replicated -- i.e. no accidental
  sharing of names used by the process to get its work done and the
  name(s) used by the replication to effect copying. This latter
  revision of the definition of replication is crucial to obtaining
  the expected identity $!!P \sim !P$.
\end{remark}

\begin{remark}\label{rem:paradoxical_combinator}
  The reader familiar with the lambda calculus will have noticed the
  similarity between $D$ and the paradoxical combinator.

  [Ed. note: the existence of this seems to suggest we have to be more
  restrictive on the set of processes and names we admit if we are to
  support no-cloning.]
\end{remark}

\subsubsection{Bisimulation}

The computational dynamics gives rise to another kind of equivalence,
the equivalence of computational behavior. As previously mentioned
this is typically captured \emph{via} some form of bisimulation.

% The notion we use in this paper is weak barbed bisimulation
% \cite{milner91polyadicpi}.

The notion we use in this paper is derived from weak barbed
bisimulation \cite{milner91polyadicpi}. 

\begin{definition}
An \emph{observation relation}, $\downarrow_{\mathcal N}$, over a set
of names, $\mathcal N$, is the smallest relation satisfying the rules
below.

\infrule[Out-barb]{y \in {\mathcal N}, \; x \nameeq y}
		  {\outputp{x}{v} \downarrow_{\mathcal N} x}
\infrule[Par-barb]{\mbox{$P\downarrow_{\mathcal N} x$ or $Q\downarrow_{\mathcal N} x$}}
		  {\binpar{P}{Q} \downarrow_{\mathcal N} x}

We write $P \Downarrow_{\mathcal N} x$ if there is $Q$ such that 
$P \wred Q$ and $Q \downarrow_{\mathcal N} x$.
\end{definition}

\begin{definition}
%\label{def.bbisim}
An  ${\mathcal N}$-\emph{barbed bisimulation} over a set of names, ${\mathcal N}$, is a symmetric binary relation 
${\mathcal S}_{\mathcal N}$ between agents such that $P\rel{S}_{\mathcal N}Q$ implies:
\begin{enumerate}
\item If $P \red P'$ then $Q \wred Q'$ and $P'\rel{S}_{\mathcal N} Q'$.
\item If $P\downarrow_{\mathcal N} x$, then $Q\Downarrow_{\mathcal N} x$.
\end{enumerate}
$P$ is ${\mathcal N}$-barbed bisimilar to $Q$, written
$P \wbbisim_{\mathcal N} Q$, if $P \rel{S}_{\mathcal N} Q$ for some ${\mathcal N}$-barbed bisimulation ${\mathcal S}_{\mathcal N}$.
\end{definition}

$\mathcal{R} \subseteq \pi \times \pi$

$P \mathcal{R} Q => \forall P'. P \red P' \Rightarrow \exists Q'. Q \red Q', P' \mathcal{R} Q'$

$P \vdash x \Rightarrow Q \vdash x$

\begin{mathpar}
  \inferrule*[lab=Out-barb]{x \nameeq y}{{y}!\langle{Q}\rangle \vdash x}
  \and
  \inferrule*[lab=Par-barb]{\mbox{$P\vdash x$ or $Q\vdash x$}}{\binpar{P}{Q} \vdash x}
\end{mathpar}

\subsubsection{Contexts}

One of the principle advantages of computational calculi like the
$\pi$-calculus is a well-defined notion of context,
contextual-equivalence and a correlation between
contextual-equivalence and notions of bisimulation. The notion of
context allows the decomposition of a process into (sub-)process and
its syntactic environment, its context. Thus, a context may be
thought of as a process with a ``hole'' (written $\Box$) in it. The
application of a context $M$ to a process $P$, written $M[P]$, is
tantamount to filling the hole in $M$ with $P$. In this paper we do
not need the full weight of this theory, but do make use of the notion
of context in the proof the main theorem. 

\begin{mathpar}
  \inferrule* [lab=summation] {} {{M_{M},M_{N}} \bc \Box \;|\; x.M_{A} \;|\; M_{M}+M_{N}}
  \and
  \inferrule* [lab=agent] {} {{M_{A}} \bc (\vec{x})M_{P} \;| \; \clift{P_0,\ldots,M_{P},\ldots,P_N}}
  \and \\
  \inferrule* [lab=process] {} {{M_{P}} \bc M_{N} \;| \;P|M_{P} }
\end{mathpar} 

\begin{mathpar}
  \inferrule* [lab=sychronization] {} {M_{N} \bc \Box \;|\; x?M_{F} \;|\; x!M_{C}}
  \and
  \inferrule* [lab=abstraction] {} {{M_{F}} \bc (x)M_{P} }
  \and
  \inferrule* [lab=concretion] {} {{M_{C}} \bc \langle M_{P} \rangle }
  \and \\
  \inferrule* [lab=process] {} {{M_{P}} \bc M_{N} \;| \;P|M_{P} }
\end{mathpar}

\begin{definition}[contextual application] Given a context $M$, and
  process $P$, we define the \emph{contextual application}, $M[P] :=
  M\{P/\Box\}$. That is, the contextual application of M to P is the
  substitution of $P$ for $\Box$ in $M$.
\end{definition}

$\meaningof{-} : L \to \mathcal{P}(\pi)$

\begin{mathpar}
  \inferrule* [lab=collection] {} {\meaningof{true} = \pi, \and \meaningof{~E} = \pi \setminus \meaningof{E}, \and \meaningof{E_{1} \& E_{2}} = \meaningof{E_{1}} \cap \meaningof{E_{2}}}
\end{mathpar}

\begin{mathpar}
  \inferrule* [lab=structure] {} {\meaningof{0} = \{ P \in \pi | P \equiv 0 \}, \and \\ \meaningof{E_1 | E_2} = \{ P \in \pi | P \equiv P_{1} | P_{2}, P_{1} \in \meaningof{E_{1}}, P_{2} \in \meaningof{E_2}\} }
\end{mathpar}

\begin{mathpar}
 \inferrule* [lab=behavior] {} {\meaningof{\langle a?b \rangle E} = \{ P \in \pi | P \equiv Q | u?(y)P', \\ \and \\\\ \and \\ \;\;\; u \in \meaningof{a}, \forall z.P'\{z/y\} \in \meaningof{E\{z/b\}}\}, \and \\ \meaningof{a!E} = \{ P \in \pi | P \equiv Q | x!\langle P' \rangle, x \in \meaningof{a} P' \in \meaningof{E}\} }
\end{mathpar}

\begin{mathpar}
 \inferrule* [lab=nominal] {} {\meaningof{\quotep{E}} = \{ \quotep{P} \in \quotep{\pi} | P \in \meaningof{E} \}, \and \meaningof{\quotep{P}} = \{ \quotep{Q} \in \quotep{\pi} | P \equiv Q \} \and \\ \meaningof{@\quotep{E}} = \{ P \in \pi | P \equiv @x, x \in \meaningof{E} \}}
\end{mathpar}

\begin{eqnarray*}
  \\
  \meaningof{-} : TS \to ST
\end{eqnarray*}

\begin{eqnarray*}
  \\
  L : TS \to ST
\end{eqnarray*}

\begin{eqnarray*}
  \\
  P \models E \iff P \in \meaningof{E}
\end{eqnarray*}

\begin{eqnarray*}
  P \approx_{L} Q \iff \forall E \in L. P \models E \iff Q \models E
\end{eqnarray*}

\begin{eqnarray*}
  P \approx_{K} Q
\end{eqnarray*}

\begin{eqnarray*}
  P \approx Q
\end{eqnarray*}

$\approx_{K} = \approx = \approx_{L}$

\subsubsection{Contextual duality}

Note that contexts extend the quotation operation to a family of
operations from processes to names. Given a context, $M$, we can
define a \emph{nominal context}, $\quotep{M}$ by $\quotep{M}[P] :=
\quotep{M[P]}$. To foreshadow what is to come we observe that these
operations enjoy a duality with processes very much like the duality
between vectors and maps from vectors to scalars.

Further, because the calculus is essentially higher-order, we have a
correspondence between contexts and processes. More specifically,
given a name $x$ and a context $M$ we can construct $M^{*}_{x}$ such
that 

\begin{mathpar}
  M^{*}_{x} | \lift{x}{P} \red M[P]
\end{mathpar}

namely,

\begin{mathpar}
  M^{*}_{x} := x?(u).M[\dropn{u}]
\end{mathpar}

The dependence of $M^{*}_{x}$ on a name makes it an abstraction, 

\begin{mathpar}
  M^{*} := (x)x?(u).M[\dropn{u}]
\end{mathpar}

\subsection{Additional notation}

It will sometimes be convenient to denote the process a name
quotes. We already have the notation $x = \quotep{P}$, but it will be
convenient to introduce an alternate notation, $\procn{x}$, when we
want to emphasize the connection to the use of the name. Note that, by
virtue of name equivalence, $\quotep{\procn{x}} \nameeq x$; so, the
notation is consistent with previous definitions.

Further, because names have structure it is possible to effect
substitutions on the basis of that structure. This means we need to
upgrade our notation for substitutions, which we accomplish by
adapting comprehension notation. Thus,

\begin{mathpar}
  P\{ y / x : x \in S \}
\end{mathpar}

is interpreted to mean the process derived from P by replacing (in a
capture-avoiding manner) each occurrence of $x$ in $S$ by $y$. For example,

\begin{mathpar}
  P\{ \quotep{\procn{x}|\procn{x}} / x : x \in \freenames{P} \}
\end{mathpar}

will replace each (occurrence) of a free name $x$ in $P$ by
$\quotep{\procn{x}|\procn{x}}$.

Also, we will avail ourselves of the notation $x^{L}$ and $x^{R}$ to
denote injections of a name into disjoint copies of the name
space. There are numerous ways to accomplish this. One example can be
found in \cite{MeredithR05}. This notation overloads to vectors of
names: $\vec{x}^{\pi} := (x_{i}^{\pi} \; : \; 0 \leq i < |\vec{x}| )$ where $\pi \in \{L,R\}$.

We also use $P^{\Box} := P|\Box$.

In \cite{MeredithR05} an interpretation of the new operator is
given. It turns out that there are several possible interpretations
all enjoying the requisite algebraic properties of the operator (see
\cite{milner91polyadicpi}). We will therefore make liberal use of
$(\nu\; \vec{x})P$.

% subsection the_syntax_and_semantics_of_the_notation_system (end)   

\input{qm2pi.qmops} 

\input{qm2pi.sterngerlach} 

\input{qm2pi.metric} 

% section concurrent_process_calculi (end)

%\input{qm2pi.proofsketch}

% section proof sketch (end)

%\input{qm2pi.slviaknots} 

% section spatial logic via knots (end)

\input{qm2pi.conclusion}

% section conclusion (end)

%\input{qm2pi.dtcodes} 

% section wiring algorithm (end)

\input{qm2pi.ack} 

% section acknowledgments (end)

\newpage


\bibliographystyle{plain}   
\bibliography{../../biblios/main.bib}

\input{qm2pi.rhodetails}

\end{document}

 

% subsection basic_interpretation (end)

%\input{qm2pi.rho.presentation} 
\subsection{The syntax and semantics of the notation system}\label{sub:the_syntax_and_semantics_of_the_notation_system} % (fold)

We now summarize a technical presentation of the calculus that
embodies our theory of dynamics. The typical presentation of such a
calculus follows the style of giving generators and relations on
them. The grammar, below, describing term constructors, freely
generates the set of processes, $\Proc$. This set is then quotiented
by a relation known as structural congruence and it is over this set
that the notion of dynamics is expressed. This presentation is
essentially that of \cite{MeredithR05} with the addition of
polyadicity and summation. For readability we have relegated some of
the technical subtleties to an appendix.

\subsubsection{Process grammar}\label{subsub:process_grammar}

\begin{mathpar}
  \inferrule* [lab=synchronization] {} {{M} \bc \pzero \;|\; x?F \;|\; x!C }
  \and
  \inferrule* [lab=abstraction] {} {{F} \bc (x)P}
  \and
  \inferrule* [lab=concretion] {} {{C} \bc \langle Q \rangle}
  \and
  \inferrule* [lab=process] {} {{P,Q} \bc M \;| \;P|Q \;|\; @{x}}
  \and
  \inferrule* [lab=name] {} {{x} \bc \quotep{P}}
\end{mathpar} 

Note that $\vec{x}$ (resp. $\vec{P}$) denotes a vector of names
(resp. processes) of length $|\vec{x}|$ (resp. $|\vec{P}|$). We adopt
the following useful abbreviations.

\begin{mathpar}
   x?(\vec{y}).P := x.(\vec{y})P \and  x\clift{\vec{P}} := x.\clift{\vec{P}}
   \and x!(y) := \lift{x}{\dropn{y}}
   \and \Pi_{i=0}^{n-1}P_i := P_0 | \ldots | P_{n-1}
\end{mathpar}

\subsubsection{Structural congruence}

\paragraph{Free and bound names and alpha-equivalence.} At the
core of structural equivalence is alpha-equivalence which identifies
process that are the same up to a change of variable. Formally, we
recognize the distinction between free and bound names. The free names
of a process, $\freenames{P}$, may be calculated recursively as
follows:

\begin{mathpar}
\freenames{\pzero} := \emptyset
  \and \\
  \freenames{x?(y).P} := \{ x \} \cup (\freenames{P} \setminus \{ y \})
  \and 
  \freenames{x!\langle P \rangle} := \{ x \} \cup \{ P \} 
  \and \\
  \freenames{P|Q} := \freenames{P} \cup \freenames{Q}
  \and \\
  \freenames{@{x}} := \{ x \}
\end{mathpar}

$\pi$
$\quotep{\pi}$

$\freenames{-} : \pi \to \mathcal{P}(\quotep{\pi})$

\begin{eqnarray*}
  \freenames{\pzero} & := & \emptyset \\
  \freenames{x?(y).P} & := & \{ x \} \cup (\freenames{P} \setminus \{ y \}) \\
  \freenames{x!\langle P \rangle} & := & \{ x \} \cup \{ P \} \\
  \freenames{P|Q} & := & \freenames{P} \cup \freenames{Q} \\
  \freenames{\dropn{x}} & := & \{ x \}
\end{eqnarray*}

The bound names of a process, $\boundnames{P}$, are those names occurring in $P$
that are not free. For example, in $x?(y).0$, the name $x$ is free, while $y$ is bound.

\begin{mathpar}
  \inferrule* [lab=monoidal-laws] {} { P|Q \equiv Q|P \and P|0 \equiv P \and P|(Q|R) \equiv (P|Q)|R }
\end{mathpar}

\begin{mathpar}
  \inferrule* [lab=alpha-equivalence] {} { (x)P \equiv (y)P\{y/x\} \and y \not\in \freenames{P} }
\end{mathpar}

\begin{definition}
Then two processes, $P,Q$, are alpha-equivalent if $P = Q\{\vec{y}/\vec{x}\}$ for
some $\vec{x} \in \boundnames{Q},\vec{y} \in \boundnames{P}$, where $Q\{\vec{y}/\vec{x}\}$
denotes the capture-avoiding substitution of $\vec{y}$ for $\vec{x}$ in $Q$.
\end{definition}

\begin{definition}
  The {\em structural congruence} \cite{SangiorgiWalker} , $\equiv$,
  between processes is the least congruence containing
  alpha-equivalence, satisfying the abelian monoid laws
  (associativity, commutativity and $\pzero$ as identity) for parallel
  composition $|$ and for summation $+$.
\end{definition}

\subsection{Name equivalence}

We take name equivalence, written $\nameeq$, to be the smallest
equivalence relation generated by the following rules.

\begin{mathpar}
\inferrule*[lab=Quote-drop]
{ }
{ \quotep{@{x}} \nameeq x }

\inferrule*[lab=Struct-equiv]
{ P \scong Q }
{ \quotep{P} \nameeq \quotep{Q} }
\end{mathpar}

The astute reader will have noticed that the mutual recursion of names
and processes imposes a mutual recursion on alpha-equivalence and
structural equivalence via name-equivalence. Fortunately, all of this
works out pleasantly and we may calculate in the natural way, free of
concern. The reader interested in the details is referred to the
appendix \ref{appendix:rho_details}.

\subsection{Substitution}

We use $\Proc$ for the set of processes, $\QProc$ for the set of
names, and $\id{\{}\vec{y} / \vec{x} \id{\}}$ to denote partial maps,
$s : \QProc \rightarrow \QProc$. A map, $s$ lifts, uniquely, to a map
on process terms, $\widehat{s} : \Proc \rightarrow \Proc$ by the
following equations.

\begin{mathpar}
  (0) \psubstp{Q}{P} := 0 \\
  (R \juxtap S) \psubstp{Q}{P}
  :=    
  (R)\psubstp{Q}{P} \juxtap (S) \psubstp{Q}{P} \\
  (x?(y).R) \psubstp{Q}{P}    
  :=    
  (x)\substp{Q}{P} (z)\concat( (R \psubstn{z}{y}) \psubstp{Q}{P} ) \\
  (\lift{x}{R}) \psubstp{Q}{P}  
  :=
  \lift{(x)\substp{Q}{P}}{ R \psubstp{Q}{P} } \\
%   (\dropn{x})  \psubstp{Q}{P}       
%   := 
%   \left\{ 
%     \begin{array}{ccc} 
%       \dropn{\quotep{Q}} & & x \nameeq \quotep{P} \\
%       \dropn{x} & & otherwise \\
%     \end{array}
%   \right. 
  (\dropn{x})  \psubstp{Q}{P}       
  := 
  \left\{ 
    \begin{array}{ccc} 
      Q & & x \nameeq \quotep{P} \\
      \dropn{x} & & otherwise \\
    \end{array}
  \right.
\end{mathpar}
 

where

\begin{eqnarray}
  (x)\id{\{} \lpquote Q \rpquote / \lpquote P \rpquote \id{\}}            = 
  \left\{ 
    \begin{array}{ccc}
      \lpquote Q \rpquote & & x \nameeq \lpquote P \rpquote \\
      x & & otherwise \\
    \end{array}
  \right. \nonumber
\end{eqnarray}

and $z$ is chosen distinct from $\quotep{P}$, $\quotep{Q}$, the free
names in $Q$, and all the names in $R$. Our $\alpha$-equivalence will
be built in the standard way from this substitution.

\begin{remark}\label{rem:no_self_referential_names}
  One consequence of these definitions is that $\forall P. \quotep{P}
  \not\in \freenames{P}$.
\end{remark}

\subsection{ Dynamic quote: an example }

Anticipating something of what's to come, consider applying the
substitution, $\widehat{\id{\{}u / z \id{\}}}$, to the following pair
of processes, $\lift{w}{y!(z)}$ and $w[ \lpquote y!(z) \rpquote ]$.

\begin{eqnarray}
	\lift{w}{y!(z)}\widehat{\id{\{}u / z \id{\}}}
		& = &
		\lift{w}{y!(u)} \nonumber\\
	w[ \lpquote y!(z) \rpquote ] \widehat{ \id{\{}u / z \id{\}} }
		& = &
		w[ \lpquote y!(z) \rpquote ] \nonumber
\end{eqnarray}

Because the body of the process between quotes is impervious to
substitution, we get radically different answers. In fact, by
examining the first process in an input context,
e.g. $x?(z).\lift{w}{y!(z)}$, we see that the process under the lift
operator may be shaped by prefixed inputs binding a name inside it. In
this sense, the lift operator will be seen as a way to dynamically
construct processes before reifying them as names.

Finally equipped with these standard features we can present the
dynamics of the calculus.

\subsubsection{Operational semantics} 

Finally, we introduce the computational dynamics. What marks these
algebras as distinct from other more traditionally studied algebraic
structures, e.g. vector spaces or polynomial rings, is the manner in
which dynamics is captured. In traditional structures, dynamics is typically
expressed through morphisms between such structures, as in linear maps
between vector spaces or morphisms between rings. In algebras
associated with the semantics of computation, the dynamics is
expressed as part of the algebraic structure itself, through a
reduction reduction relation typically denoted by $\red$. Below, we
give a recursive presentation of this relation for the calculus used
in the encoding.

$\red \subseteq \pi \times \pi$
$\red : \pi \to \mathcal{P}(\pi)$

\begin{mathpar}
  \inferrule* [lab=Comm] { \textsf{match}( x_{src}, x_{trgt} ) } { x_{trgt}?(y)P \; | \; x_{src}!\langle {Q} \rangle \red P\{\quotep{Q}/y}\} }
  \and \\
  \inferrule* [lab=Par] {{P} \red {P}'} {{{P} | {Q}} \red {{P}' | {Q}}}
  \and
  \inferrule* [lab=Equiv]{{{P} \scong {P}'} \andalso {{P}' \red {Q}'} \andalso {{Q}' \scong {Q}}}{{P} \red {Q}}
\end{mathpar}

\begin{eqnarray*}
  match_{\equiv} (\quotep{P},\quotep{Q}) & := & P \equiv Q \\
  match_{\dagger}(\quotep{P},\quotep{Q}) & := & \forall R. P|Q \red^{*} R => R \red^{*} 0 \\
  match_{K}(\quotep{P},\quotep{Q}) & := & K \mbox{ for some context } K
\end{eqnarray*}

$u?(x)P | u!\langle Q \rangle \red P\{\quotep{Q}/x\}$

%We write $\wred$ for $\red^*$, and $P\red$ if $\exists Q $ such that $ P \red Q$.
We write $P\red$ if $\exists Q $ such that $ P \red Q$ and $P\not\red$, otherwise.

\section{Replication}

As mentioned before, it is known that replication (and hence
recursion) can be implemented in a higher-order process algebra
\cite{SangiorgiWalker}. As our first example of calculation with the
machinery thus far presented we give the construction explicitly in
the {\rhoc}.

\begin{eqnarray}
	D_{x} & := & \prefix{x}{y}{(\binpar{\outputp{x}{y}}{@{y}})} \nonumber\\
	\bangp_{x}{P} & := & \binpar{{x}!\langle{\binpar{D_{x}}{P}}\rangle}{D_{x}} \nonumber
\end{eqnarray}

\begin{eqnarray}
	\bangp_{x}{P} & & \nonumber\\
	=
	& {x}!\langle{(\prefix{x}{y}{(\outputp{x}{y} | @{y})) | P}}\rangle 
	      | \prefix{x}{y}{(\outputp{x}{y} | @{y})} & \nonumber\\
	\red
	& (\outputp{x}{y} | @{y})\substn{\quotep{(\prefix{x}{y}{(@{y} | \outputp{x}{y})) | P}}}{y} & \nonumber\\
	=
	& \outputp{x}{\quotep{(\prefix{x}{y}{(\outputp{x}{y} | @{y})) | P}}}
	  | {(\prefix{x}{y}{(\outputp{x}{y} | @{y})) | P}} & \nonumber\\
	\red
	& \ldots & \nonumber\\
	\red^*
	& P | P | \ldots & \nonumber
\end{eqnarray}

Of course, this encoding, as an implementation, runs away, unfolding
$\bangp{P}$ eagerly. A lazier and more implementable replication
operator, restricted to input-guarded processes, may be obtained as follows.

\begin{eqnarray}
\bangp{\prefix{u}{v}{P}} 
	:= 
	\binpar{\lift{x}{\prefix{u}{v}{(\binpar{D(x)}{P})}}}{D(x)} \nonumber
\end{eqnarray}

\begin{remark}
  Note that the lazier definition still does not deal with summation
  or mixed summation (i.e. sums over input and output). The reader is
  invited to construct definitions of replication that deal with these
  features. 

  Further, the definitions are parameterized in a name, $x$. Can you,
  gentle reader, make a definition that eliminates this parameter and
  guarantees no accidental interaction between the replication
  machinery and the process being replicated -- i.e. no accidental
  sharing of names used by the process to get its work done and the
  name(s) used by the replication to effect copying. This latter
  revision of the definition of replication is crucial to obtaining
  the expected identity $!!P \sim !P$.
\end{remark}

\begin{remark}\label{rem:paradoxical_combinator}
  The reader familiar with the lambda calculus will have noticed the
  similarity between $D$ and the paradoxical combinator.

  [Ed. note: the existence of this seems to suggest we have to be more
  restrictive on the set of processes and names we admit if we are to
  support no-cloning.]
\end{remark}

\subsubsection{Bisimulation}

The computational dynamics gives rise to another kind of equivalence,
the equivalence of computational behavior. As previously mentioned
this is typically captured \emph{via} some form of bisimulation.

% The notion we use in this paper is weak barbed bisimulation
% \cite{milner91polyadicpi}.

The notion we use in this paper is derived from weak barbed
bisimulation \cite{milner91polyadicpi}. 

\begin{definition}
An \emph{observation relation}, $\downarrow_{\mathcal N}$, over a set
of names, $\mathcal N$, is the smallest relation satisfying the rules
below.

\infrule[Out-barb]{y \in {\mathcal N}, \; x \nameeq y}
		  {\outputp{x}{v} \downarrow_{\mathcal N} x}
\infrule[Par-barb]{\mbox{$P\downarrow_{\mathcal N} x$ or $Q\downarrow_{\mathcal N} x$}}
		  {\binpar{P}{Q} \downarrow_{\mathcal N} x}

We write $P \Downarrow_{\mathcal N} x$ if there is $Q$ such that 
$P \wred Q$ and $Q \downarrow_{\mathcal N} x$.
\end{definition}

\begin{definition}
%\label{def.bbisim}
An  ${\mathcal N}$-\emph{barbed bisimulation} over a set of names, ${\mathcal N}$, is a symmetric binary relation 
${\mathcal S}_{\mathcal N}$ between agents such that $P\rel{S}_{\mathcal N}Q$ implies:
\begin{enumerate}
\item If $P \red P'$ then $Q \wred Q'$ and $P'\rel{S}_{\mathcal N} Q'$.
\item If $P\downarrow_{\mathcal N} x$, then $Q\Downarrow_{\mathcal N} x$.
\end{enumerate}
$P$ is ${\mathcal N}$-barbed bisimilar to $Q$, written
$P \wbbisim_{\mathcal N} Q$, if $P \rel{S}_{\mathcal N} Q$ for some ${\mathcal N}$-barbed bisimulation ${\mathcal S}_{\mathcal N}$.
\end{definition}

$\mathcal{R} \subseteq \pi \times \pi$

$P \mathcal{R} Q => \forall P'. P \red P' \Rightarrow \exists Q'. Q \red Q', P' \mathcal{R} Q'$

$P \vdash x \Rightarrow Q \vdash x$

\begin{mathpar}
  \inferrule*[lab=Out-barb]{x \nameeq y}{{y}!\langle{Q}\rangle \vdash x}
  \and
  \inferrule*[lab=Par-barb]{\mbox{$P\vdash x$ or $Q\vdash x$}}{\binpar{P}{Q} \vdash x}
\end{mathpar}

\subsubsection{Contexts}

One of the principle advantages of computational calculi like the
$\pi$-calculus is a well-defined notion of context,
contextual-equivalence and a correlation between
contextual-equivalence and notions of bisimulation. The notion of
context allows the decomposition of a process into (sub-)process and
its syntactic environment, its context. Thus, a context may be
thought of as a process with a ``hole'' (written $\Box$) in it. The
application of a context $M$ to a process $P$, written $M[P]$, is
tantamount to filling the hole in $M$ with $P$. In this paper we do
not need the full weight of this theory, but do make use of the notion
of context in the proof the main theorem. 

\begin{mathpar}
  \inferrule* [lab=summation] {} {{M_{M},M_{N}} \bc \Box \;|\; x.M_{A} \;|\; M_{M}+M_{N}}
  \and
  \inferrule* [lab=agent] {} {{M_{A}} \bc (\vec{x})M_{P} \;| \; \clift{P_0,\ldots,M_{P},\ldots,P_N}}
  \and \\
  \inferrule* [lab=process] {} {{M_{P}} \bc M_{N} \;| \;P|M_{P} }
\end{mathpar} 

\begin{mathpar}
  \inferrule* [lab=sychronization] {} {M_{N} \bc \Box \;|\; x?M_{F} \;|\; x!M_{C}}
  \and
  \inferrule* [lab=abstraction] {} {{M_{F}} \bc (x)M_{P} }
  \and
  \inferrule* [lab=concretion] {} {{M_{C}} \bc \langle M_{P} \rangle }
  \and \\
  \inferrule* [lab=process] {} {{M_{P}} \bc M_{N} \;| \;P|M_{P} }
\end{mathpar}

\begin{definition}[contextual application] Given a context $M$, and
  process $P$, we define the \emph{contextual application}, $M[P] :=
  M\{P/\Box\}$. That is, the contextual application of M to P is the
  substitution of $P$ for $\Box$ in $M$.
\end{definition}

$\meaningof{-} : L \to \mathcal{P}(\pi)$

\begin{mathpar}
  \inferrule* [lab=collection] {} {\meaningof{true} = \pi, \and \meaningof{~E} = \pi \setminus \meaningof{E}, \and \meaningof{E_{1} \& E_{2}} = \meaningof{E_{1}} \cap \meaningof{E_{2}}}
\end{mathpar}

\begin{mathpar}
  \inferrule* [lab=structure] {} {\meaningof{0} = \{ P \in \pi | P \equiv 0 \}, \and \\ \meaningof{E_1 | E_2} = \{ P \in \pi | P \equiv P_{1} | P_{2}, P_{1} \in \meaningof{E_{1}}, P_{2} \in \meaningof{E_2}\} }
\end{mathpar}

\begin{mathpar}
 \inferrule* [lab=behavior] {} {\meaningof{\langle a?b \rangle E} = \{ P \in \pi | P \equiv Q | u?(y)P', \\ \and \\\\ \and \\ \;\;\; u \in \meaningof{a}, \forall z.P'\{z/y\} \in \meaningof{E\{z/b\}}\}, \and \\ \meaningof{a!E} = \{ P \in \pi | P \equiv Q | x!\langle P' \rangle, x \in \meaningof{a} P' \in \meaningof{E}\} }
\end{mathpar}

\begin{mathpar}
 \inferrule* [lab=nominal] {} {\meaningof{\quotep{E}} = \{ \quotep{P} \in \quotep{\pi} | P \in \meaningof{E} \}, \and \meaningof{\quotep{P}} = \{ \quotep{Q} \in \quotep{\pi} | P \equiv Q \} \and \\ \meaningof{@\quotep{E}} = \{ P \in \pi | P \equiv @x, x \in \meaningof{E} \}}
\end{mathpar}

\begin{eqnarray*}
  \\
  \meaningof{-} : TS \to ST
\end{eqnarray*}

\begin{eqnarray*}
  \\
  L : TS \to ST
\end{eqnarray*}

\begin{eqnarray*}
  \\
  P \models E \iff P \in \meaningof{E}
\end{eqnarray*}

\begin{eqnarray*}
  P \approx_{L} Q \iff \forall E \in L. P \models E \iff Q \models E
\end{eqnarray*}

\begin{eqnarray*}
  P \approx_{K} Q
\end{eqnarray*}

\begin{eqnarray*}
  P \approx Q
\end{eqnarray*}

$\approx_{K} = \approx = \approx_{L}$

\subsubsection{Contextual duality}

Note that contexts extend the quotation operation to a family of
operations from processes to names. Given a context, $M$, we can
define a \emph{nominal context}, $\quotep{M}$ by $\quotep{M}[P] :=
\quotep{M[P]}$. To foreshadow what is to come we observe that these
operations enjoy a duality with processes very much like the duality
between vectors and maps from vectors to scalars.

Further, because the calculus is essentially higher-order, we have a
correspondence between contexts and processes. More specifically,
given a name $x$ and a context $M$ we can construct $M^{*}_{x}$ such
that 

\begin{mathpar}
  M^{*}_{x} | \lift{x}{P} \red M[P]
\end{mathpar}

namely,

\begin{mathpar}
  M^{*}_{x} := x?(u).M[\dropn{u}]
\end{mathpar}

The dependence of $M^{*}_{x}$ on a name makes it an abstraction, 

\begin{mathpar}
  M^{*} := (x)x?(u).M[\dropn{u}]
\end{mathpar}

\subsection{Additional notation}

It will sometimes be convenient to denote the process a name
quotes. We already have the notation $x = \quotep{P}$, but it will be
convenient to introduce an alternate notation, $\procn{x}$, when we
want to emphasize the connection to the use of the name. Note that, by
virtue of name equivalence, $\quotep{\procn{x}} \nameeq x$; so, the
notation is consistent with previous definitions.

Further, because names have structure it is possible to effect
substitutions on the basis of that structure. This means we need to
upgrade our notation for substitutions, which we accomplish by
adapting comprehension notation. Thus,

\begin{mathpar}
  P\{ y / x : x \in S \}
\end{mathpar}

is interpreted to mean the process derived from P by replacing (in a
capture-avoiding manner) each occurrence of $x$ in $S$ by $y$. For example,

\begin{mathpar}
  P\{ \quotep{\procn{x}|\procn{x}} / x : x \in \freenames{P} \}
\end{mathpar}

will replace each (occurrence) of a free name $x$ in $P$ by
$\quotep{\procn{x}|\procn{x}}$.

Also, we will avail ourselves of the notation $x^{L}$ and $x^{R}$ to
denote injections of a name into disjoint copies of the name
space. There are numerous ways to accomplish this. One example can be
found in \cite{MeredithR05}. This notation overloads to vectors of
names: $\vec{x}^{\pi} := (x_{i}^{\pi} \; : \; 0 \leq i < |\vec{x}| )$ where $\pi \in \{L,R\}$.

We also use $P^{\Box} := P|\Box$.

In \cite{MeredithR05} an interpretation of the new operator is
given. It turns out that there are several possible interpretations
all enjoying the requisite algebraic properties of the operator (see
\cite{milner91polyadicpi}). We will therefore make liberal use of
$(\nu\; \vec{x})P$.

% subsection the_syntax_and_semantics_of_the_notation_system (end)   

\section{Interpretation of QM}
\subsection{Supporting definitions}
\subsubsection{Multiplication}
\begin{mathpar}
  \quotep{Q} \cdot \quotep{R} := \quotep{Q|R}
  \and \\
  \quotep{Q} \cdot P := P\{ \quotep{Q|R} / \quotep{R} : \quotep{R} \in \freenames{P} \}
\end{mathpar}

\paragraph{Discussion}
The first line needs little explanation. The second line says that
each free name of the process is replaced with the multiplication of
that name by the scalar. Multiplication of a scalar (name) by a state
(process) results in a process all the names of which have been `moved
over' by parallel composition with the process the scalar
quotes. There is a subtlety that the bound names have to be
manipulated so that multiplied names aren't accidentally
captured. There are many ways to achieve this.

\begin{remark}\label{rem:multiplication_identities}
  The reader is invited to verify that for all $x,y,z \in \QProc$ and $P \in \Proc$
  \begin{mathpar}
    x \cdot \quotep{0} \equiv x 
    \and
    x \cdot y \equiv y \cdot x
    \and
    x \cdot (y \cdot z) \equiv (x \cdot y) \cdot z
    \and \\
    \quotep{0} \cdot P \equiv P
    \and \\
    x \cdot (y \cdot P) \equiv (x \cdot y) \cdot P
    \and \\
    x \cdot (P|Q) \equiv (x \cdot P) | (x \cdot Q)
    \and \\    
  \end{mathpar}
\end{remark}

\subsubsection{Tensor product}

We define a tensor product on processes by structural induction.

\paragraph{Tensor of sums} First note that all summations, including
$\pzero$ and sequence, can be written $\Sigma_{i} x_{i}.A_{i} +
\Sigma_{j} x_{j}.C_{j}$, where we have grouped input-guarded processes
together and output-guarded processes together.

Thus, we can define the tensor product of two summations, $N_{1}\otimes N_{2}$, where

\begin{mathpar}
  N_{1} := \Sigma_{i} x_{i}.A_{i} + \Sigma_{j} x_{j}.C_{j}
  \and
  N_{2} := \Sigma_{i'} y_{i'}.B_{i'} + \Sigma_{j'} y_{j'}.D_{j'} 
\end{mathpar}

as follows.

\begin{mathpar}
  \Sigma_{i} x_{i}.A_{i} + \Sigma_{j} x_{j}.C_{j} \otimes \Sigma_{i'}
  y_{i'}.B_{i'} + \Sigma_{j'} y_{j'}.D_{j'} 
  \and \\
  := \; \Sigma_{i} \Sigma_{i'} \quotep{\stackrel{\vee}{x_{i}}| \stackrel{\vee}{y_{i'}}}.(A_{i}\otimes B_{i'}) \; | \; \Sigma_{i'} \Sigma_{i} \quotep{\stackrel{\vee}{y_{i'}}|\stackrel{\vee}{x_{i}}}.(B_{i'}\otimes A_{i})
  \and
  \;\; | \;\; \Sigma_{j} \Sigma_{j'} \quotep{\stackrel{\vee}{x_{j}}|\stackrel{\vee}{y_{j'}}}.(A_{j}\otimes B_{j'}) \; | \; \Sigma_{j'} \Sigma_{j} \quotep{\stackrel{\vee}{y_{j'}}|\stackrel{\vee}{x_{j}}}.(B_{j'}\otimes A_{j})
\end{mathpar}

\begin{remark}
  Do we need to $x^{L}$ and $y^{R}$ for this construction as well?
\end{remark}

\paragraph{Tensor of parallel compositions} Next, we distribute tensor
over par.

\begin{mathpar}
  P_{1}|P_{2} \otimes Q_{1}|Q_{2} := (P_{1} \otimes Q_{1}) | (P_{1}
  \otimes Q_{2}) | (P_{2} \otimes Q_{1}) | (P_{2} \otimes Q_{2})
\end{mathpar}

\paragraph{Tensor with dropped names} We treat tensor of a
process with a dropped name as parallel composition.

\begin{mathpar}
  P \otimes \dropn{x} := P | \dropn{x}
\end{mathpar}

\paragraph{Tensor of agents}

Finally, we need to define tensor on agents. Note that the definition
of tensor on normal products only tensors inputs with inputs and
outputs with outputs. Thus, we only have to define the operation on
``homogeneous'' pairings.

\begin{mathpar}
  (\vec{x})P \otimes (\vec{y})Q
  \and \\
  := (x_{0}^{L}|y_{0}^{R},\ldots,x_{0}^{L}|y_{n}^{R},\ldots,x_{m}^{L}|y_{0}^{R},\ldots,x_{m}^{L}|y_{n}^R)(P\{ \vec{x}^{L}/\vec{x}\} \otimes Q \{ \vec{y}^{R}/\vec{y}\})
  \and \\
  \clift{\vec{P}} \otimes \clift{\vec{Q}}
  \and \\
  := \clift{P_{0}\otimes Q_{0},\ldots,P_{0}\otimes Q_{n},\ldots,P_{m}\otimes Q_{0},\ldots,P_{m}\otimes Q_{n}}
\end{mathpar}

\begin{remark}
  Observe that arities of tensored abstractions matches arities of
  tensored concretions if the original arities matched. Note also that
  the length of the arities corresponds to the increase in dimension
  we see in ordinary vector space tensor product.
\end{remark}

\begin{remark}
  Operationally, this definition distributes the tensor down to
  components ``linked'' by summation. Tensor over summation is
  intriguing in that it mixes names. Moreover, as a consequence of the
  way it mixes names we have the identities for all $x \in \QProc$ and
  $P,Q \in \Proc$

  \begin{mathpar}
    (x \cdot P) \otimes Q \equiv x \cdot (P \otimes Q) \equiv P \otimes (x \cdot Q)
    \and
    P \otimes \pzero \equiv P
  \end{mathpar}

  that the reader is invited to verify.
\end{remark}

\subsubsection{Annihilation}
\begin{mathpar}
  P^{\perp} := \{ Q | \forall R. P|Q \red^{*} R \Rightarrow R \red^{*} \pzero \}
  \and \\
  P^{\underline{\perp}} := \Sigma_{Q \in P^{\perp}} \quotep{Q}?(y).(\dropn{y}|Q) | \Sigma_{Q \in P^{\perp}} \quotep{Q}\clift{\Box}
\end{mathpar}

\paragraph{Discussion} The reader will note that $P^{\perp}$ is a
\emph{set} of processes, while $P^{\underline{\perp}}$ is a
\emph{context}. We call the set $P^{\perp}$ the \emph{annihilators} of
$P$. The parallel composition of a process in the annihilators of $P$
with $P$ will result in a process, the state space of which has all
paths eventually leading to $\pzero$. Execution may endure loops; but
under reasonable conditions of fairness (naturally guaranteed under
most notions of bisimulation) such a composite process cannot get
stuck in such a loop and will, eventually pop out and terminate.

The context $P^{\underline{\perp}}$ is ready and willing to ``take the
$P$ out of'' the process to which it is applied. It will effectively
transmit the code of the process to which it is applied to one of the
annihilators and run the process against it.

\subsubsection{Evaluation}
We fix $M$ a domain of fully abstract interpretation with an equality
coincident with bisimulation. We take $\meaningof{\cdot} : \Proc \to
M$ to be the map interpreting processes and $\nmeaningof{\cdot} : \M
\to Proc$ to be the map running the other way. Then we define

\begin{mathpar}
  \int P := \nmeaningof{\meaningof{P}}
\end{mathpar}

\paragraph{Discussion}
There are many fully abstract interpretations of Milner's
$\pi$-calculus. Any of them can be used as a basis for interpreting
the reflective calculus here. Equipped with such a domain it is
largely a matter of grinding through to check that the Yoneda
construction for the normalization-by-evaluation program can be
extended to this setting.

\begin{remark}
  The reader is invited to verify that $\int (P^{\underline{\perp}}[P]) = 0$.
\end{remark}

\subsection{Quantum mechanics}

Table \ref{tbl:core_qm_op_defns} gives the core operational definitions

\begin{table}[htp]\label{tbl:core_qm_op_defns}
  \center{
    \fbox{
      \begin{tabular}{c|c}
        quantum mechanics & process calculus \\
        \hline
        scalar & $x := \quotep{P}$ \\
        state vector & $\state{P} := P$ \\
        dual & $\state{P}^{*} := \event{P^{\underline{\perp}}} := \quotep{P^{\underline{\perp}}}[-]$ \\
        matrix & $ \Sigma_{\alpha} \state{P_{\alpha}}x_{\alpha}\event{Q_{\alpha}}$ \\
        vector addition & $\state{P} + \state{Q} := \state{P | Q}$ \\
        tensor product & $\state{P} \otimes \state{Q} := \state{P \otimes Q}$ \\
        inner product & $\innerprod{P}{Q} := \quotep{\int P^{\underline{\perp}}[Q]}$ \\
      \end{tabular}
    }
  }
  \caption{QM - operational definitions}
\end{table}

where

\begin{mathpar}
  \prmatrix{P}{Q} := \fprmatrix{P}{\quotep{\pzero}}{Q}
  \and
  \fprmatrix{P}{x}{Q} := (\state{P},x,\event{Q})
  \and
  (\fprmatrix{P}{x}{Q})(\state{R}) := x \cdot \innerprod{Q}{R} \cdot \state{P}
  \and
  (\fprmatrix{P}{x}{Q})(\event{R}) := x \cdot \innerprod{R}{P} \cdot \event{Q}
\end{mathpar}

\paragraph{Discussion}
As promised: vectors (aka states) are represented as processes; duals
as contextual duals; inner product definition should be compared with
standard inner product definition for ....

\begin{remark}
  Assuming $\int (P^{\underline{\perp}}[P]) = 0$, the reader is
  invited to verify that $(\fprmatrix{P}{x}{P})(\state{P}) = x \cdot \state{P}$.
\end{remark}

\begin{remark}
  The reader is invited to verify that $\innerprod{P}{Q}$ could
  equally well have been written $\quotep{\int \stackrel{\vee}{x}}$
  where $x = \event{P^{\underline{\perp}}}(Q)$.

  One of the motivations for this remark is that there is another way
  to factor these operations. We could package up evaluation in the dual:

  \begin{mathpar}
    \state{P}^{*} := \event{\int P^{\underline{\perp}}} := \quotep{\int P^{\underline{\perp}}}[-]
  \end{mathpar}

  and then have inner product defined by
  
  \begin{mathpar}
    \innerprod{P}{Q} := \event{P}(Q)
  \end{mathpar}

  Hopefully, experience with the calculations will provide guidance on
  the best factoring.
\end{remark}

\begin{remark}
  Assuming $\int (P^{\underline{\perp}}[P]) = 0$, the reader is
  invited to verify that $\forall P,Q. (\prmatrix{0}{Q})(\state{0}) =
  \state{0}$ and dually $(\prmatrix{P}{0})(\event{0}) = \event{0}$.
\end{remark}

\begin{remark}
  i'm a little worried that i don't (yet) have proper support for
  complex conjugacy. But, the observation above may give us a
  clue. According to Abramsky, it must be the case that the scalars
  are iso to the homset of the identity for the tensor -- which the
  observation above characterizes. 

  For now, we will simply bookmark the notion with $\overline{x}$.
\end{remark}

\subsubsection{Adjointness}

We need to give a definition of $(\cdot)^{\dagger}$ for matrices. The
obvious candidate definition is
\begin{mathpar}
(\Sigma_{\alpha}\fprmatrix{P_{\alpha}}{x_{\alpha}}{Q_{\alpha}})^{\dagger}
= \Sigma_{\alpha}\fprmatrix{(Q_{\alpha}^{\underline{\perp}})^{*}}{\overline{x}_{\alpha}}{P_{\alpha}^{\underline{\perp}}} 
\end{mathpar}

But, $(Q_{\alpha}^{\underline{\perp}})^{*}$ requires a name along
which to communicate the process to achieve the context application.

\subsubsection{Basis for a basis}
If processes label states and ``addition'' of states (a.k.a. vector
addition) is interpreted as parallel composition, what corresponds to
notions of linear independence and basis? Here, we recall that Yoshida
has developed a set of \emph{combinators} for an asynchronous verison
of Milner's $\pi$-calculus. These are a finite set of processes such
any process can be expressed as parallel composition of these
combinators together with liberal uses of the new operator and
replication. We can simply give a translation of these into the
present calculus and have reasonable expectation that the property
carries over. That is, that the resultant set allows to express all
processes via parallel composition. Note, however, that there is no
new operator or replication in this calculus. As a result, we expect
that the corresponding set is actually infinite. That is, we expect
that the space is actually infinite dimensional.

\begin{remark}
  The attentive reader may be a bit concerned. Certainly, the
  collection $S$, $K$ and $I$ is a finite set of
  combinators. Shouldn't we expect to see a finite set of combinators
  for an effectively equivalent system? i am very sympathetic to this
  critique and feel it warrants full attention. On the other hand, i
  also have in mind the following analogy. The natural numbers, as a
  monoid under addition, has exactly $1$ generator, while the natural
  numbers, as a monoid under multiplication, has countably many
  generators (the primes). We observe that the application of the
  lambda calculus is much less resource sensitive than the parallel
  composition of the $\pi$-calculus. Could it be the case that we have
  an analogy of the form
  
  \begin{mathpar}
    m + n : MN :: m*n : M|N
  \end{mathpar}

  giving a similar blow up in the set of ``primes''?  This is such a
  wonderful thought that, even if it's not true, i think it's worth
  writing down.
\end{remark}
 

\documentclass[12pt]{llncs}
%\documentclass{jktr}

\usepackage[pdftex]{hyperref}                   
\usepackage {listings}
\usepackage {mathpartir}
\usepackage{bcprules}
%\usepackage{listings}
                       
\usepackage{graphicx} 
%\usepackage[margins=2.5cm,nohead,nofoot]{geometry}
%\usepackage{geometry}
\usepackage{amsfonts}
\usepackage{amstext}
\usepackage{latexsym}
\usepackage{amssymb}
\usepackage{color}


%\include{myPreamble}
\include{qm2pi.local} 

%\ifpdf
%\usepackage[pdftex]{graphicx}
%\else
%\usepackage{graphicx}
%\fi

 % \ifpdf
%  \usepackage{pdfsync}
%  \if


%\title{Brief Article}
%\author{David F. Snyder}
%\author{L.G. Meredith}

%\address{Dept. of Math., Texas State University--San Marcos, San Marcos, TX 78666}
       
\pagestyle{empty}


\begin{document}

\lstset{language=[Objective]Caml,frame=shadowbox}

\input{qm2pi.front}

% section front matter (end)

\input{qm2pi.intro} 
 
% section introduction (end)

% \input{qm2pi.knotations} 

% section notation (end)

\input{qm2pi.process.calculi} 

% section concurrent_process_calculi_and_spatial_logics_ (end)
    
%\input{qm2pi.knots2pi} 

%\input{qm2pi.trefoil} 

%\input{qm2pi.mainthm} 

% subsection basic_interpretation (end)

%\input{qm2pi.rho.presentation} 
\subsection{The syntax and semantics of the notation system}\label{sub:the_syntax_and_semantics_of_the_notation_system} % (fold)

We now summarize a technical presentation of the calculus that
embodies our theory of dynamics. The typical presentation of such a
calculus follows the style of giving generators and relations on
them. The grammar, below, describing term constructors, freely
generates the set of processes, $\Proc$. This set is then quotiented
by a relation known as structural congruence and it is over this set
that the notion of dynamics is expressed. This presentation is
essentially that of \cite{MeredithR05} with the addition of
polyadicity and summation. For readability we have relegated some of
the technical subtleties to an appendix.

\subsubsection{Process grammar}\label{subsub:process_grammar}

\begin{mathpar}
  \inferrule* [lab=synchronization] {} {{M} \bc \pzero \;|\; x?F \;|\; x!C }
  \and
  \inferrule* [lab=abstraction] {} {{F} \bc (x)P}
  \and
  \inferrule* [lab=concretion] {} {{C} \bc \langle Q \rangle}
  \and
  \inferrule* [lab=process] {} {{P,Q} \bc M \;| \;P|Q \;|\; @{x}}
  \and
  \inferrule* [lab=name] {} {{x} \bc \quotep{P}}
\end{mathpar} 

Note that $\vec{x}$ (resp. $\vec{P}$) denotes a vector of names
(resp. processes) of length $|\vec{x}|$ (resp. $|\vec{P}|$). We adopt
the following useful abbreviations.

\begin{mathpar}
   x?(\vec{y}).P := x.(\vec{y})P \and  x\clift{\vec{P}} := x.\clift{\vec{P}}
   \and x!(y) := \lift{x}{\dropn{y}}
   \and \Pi_{i=0}^{n-1}P_i := P_0 | \ldots | P_{n-1}
\end{mathpar}

\subsubsection{Structural congruence}

\paragraph{Free and bound names and alpha-equivalence.} At the
core of structural equivalence is alpha-equivalence which identifies
process that are the same up to a change of variable. Formally, we
recognize the distinction between free and bound names. The free names
of a process, $\freenames{P}$, may be calculated recursively as
follows:

\begin{mathpar}
\freenames{\pzero} := \emptyset
  \and \\
  \freenames{x?(y).P} := \{ x \} \cup (\freenames{P} \setminus \{ y \})
  \and 
  \freenames{x!\langle P \rangle} := \{ x \} \cup \{ P \} 
  \and \\
  \freenames{P|Q} := \freenames{P} \cup \freenames{Q}
  \and \\
  \freenames{@{x}} := \{ x \}
\end{mathpar}

$\pi$
$\quotep{\pi}$

$\freenames{-} : \pi \to \mathcal{P}(\quotep{\pi})$

\begin{eqnarray*}
  \freenames{\pzero} & := & \emptyset \\
  \freenames{x?(y).P} & := & \{ x \} \cup (\freenames{P} \setminus \{ y \}) \\
  \freenames{x!\langle P \rangle} & := & \{ x \} \cup \{ P \} \\
  \freenames{P|Q} & := & \freenames{P} \cup \freenames{Q} \\
  \freenames{\dropn{x}} & := & \{ x \}
\end{eqnarray*}

The bound names of a process, $\boundnames{P}$, are those names occurring in $P$
that are not free. For example, in $x?(y).0$, the name $x$ is free, while $y$ is bound.

\begin{mathpar}
  \inferrule* [lab=monoidal-laws] {} { P|Q \equiv Q|P \and P|0 \equiv P \and P|(Q|R) \equiv (P|Q)|R }
\end{mathpar}

\begin{mathpar}
  \inferrule* [lab=alpha-equivalence] {} { (x)P \equiv (y)P\{y/x\} \and y \not\in \freenames{P} }
\end{mathpar}

\begin{definition}
Then two processes, $P,Q$, are alpha-equivalent if $P = Q\{\vec{y}/\vec{x}\}$ for
some $\vec{x} \in \boundnames{Q},\vec{y} \in \boundnames{P}$, where $Q\{\vec{y}/\vec{x}\}$
denotes the capture-avoiding substitution of $\vec{y}$ for $\vec{x}$ in $Q$.
\end{definition}

\begin{definition}
  The {\em structural congruence} \cite{SangiorgiWalker} , $\equiv$,
  between processes is the least congruence containing
  alpha-equivalence, satisfying the abelian monoid laws
  (associativity, commutativity and $\pzero$ as identity) for parallel
  composition $|$ and for summation $+$.
\end{definition}

\subsection{Name equivalence}

We take name equivalence, written $\nameeq$, to be the smallest
equivalence relation generated by the following rules.

\begin{mathpar}
\inferrule*[lab=Quote-drop]
{ }
{ \quotep{@{x}} \nameeq x }

\inferrule*[lab=Struct-equiv]
{ P \scong Q }
{ \quotep{P} \nameeq \quotep{Q} }
\end{mathpar}

The astute reader will have noticed that the mutual recursion of names
and processes imposes a mutual recursion on alpha-equivalence and
structural equivalence via name-equivalence. Fortunately, all of this
works out pleasantly and we may calculate in the natural way, free of
concern. The reader interested in the details is referred to the
appendix \ref{appendix:rho_details}.

\subsection{Substitution}

We use $\Proc$ for the set of processes, $\QProc$ for the set of
names, and $\id{\{}\vec{y} / \vec{x} \id{\}}$ to denote partial maps,
$s : \QProc \rightarrow \QProc$. A map, $s$ lifts, uniquely, to a map
on process terms, $\widehat{s} : \Proc \rightarrow \Proc$ by the
following equations.

\begin{mathpar}
  (0) \psubstp{Q}{P} := 0 \\
  (R \juxtap S) \psubstp{Q}{P}
  :=    
  (R)\psubstp{Q}{P} \juxtap (S) \psubstp{Q}{P} \\
  (x?(y).R) \psubstp{Q}{P}    
  :=    
  (x)\substp{Q}{P} (z)\concat( (R \psubstn{z}{y}) \psubstp{Q}{P} ) \\
  (\lift{x}{R}) \psubstp{Q}{P}  
  :=
  \lift{(x)\substp{Q}{P}}{ R \psubstp{Q}{P} } \\
%   (\dropn{x})  \psubstp{Q}{P}       
%   := 
%   \left\{ 
%     \begin{array}{ccc} 
%       \dropn{\quotep{Q}} & & x \nameeq \quotep{P} \\
%       \dropn{x} & & otherwise \\
%     \end{array}
%   \right. 
  (\dropn{x})  \psubstp{Q}{P}       
  := 
  \left\{ 
    \begin{array}{ccc} 
      Q & & x \nameeq \quotep{P} \\
      \dropn{x} & & otherwise \\
    \end{array}
  \right.
\end{mathpar}
 

where

\begin{eqnarray}
  (x)\id{\{} \lpquote Q \rpquote / \lpquote P \rpquote \id{\}}            = 
  \left\{ 
    \begin{array}{ccc}
      \lpquote Q \rpquote & & x \nameeq \lpquote P \rpquote \\
      x & & otherwise \\
    \end{array}
  \right. \nonumber
\end{eqnarray}

and $z$ is chosen distinct from $\quotep{P}$, $\quotep{Q}$, the free
names in $Q$, and all the names in $R$. Our $\alpha$-equivalence will
be built in the standard way from this substitution.

\begin{remark}\label{rem:no_self_referential_names}
  One consequence of these definitions is that $\forall P. \quotep{P}
  \not\in \freenames{P}$.
\end{remark}

\subsection{ Dynamic quote: an example }

Anticipating something of what's to come, consider applying the
substitution, $\widehat{\id{\{}u / z \id{\}}}$, to the following pair
of processes, $\lift{w}{y!(z)}$ and $w[ \lpquote y!(z) \rpquote ]$.

\begin{eqnarray}
	\lift{w}{y!(z)}\widehat{\id{\{}u / z \id{\}}}
		& = &
		\lift{w}{y!(u)} \nonumber\\
	w[ \lpquote y!(z) \rpquote ] \widehat{ \id{\{}u / z \id{\}} }
		& = &
		w[ \lpquote y!(z) \rpquote ] \nonumber
\end{eqnarray}

Because the body of the process between quotes is impervious to
substitution, we get radically different answers. In fact, by
examining the first process in an input context,
e.g. $x?(z).\lift{w}{y!(z)}$, we see that the process under the lift
operator may be shaped by prefixed inputs binding a name inside it. In
this sense, the lift operator will be seen as a way to dynamically
construct processes before reifying them as names.

Finally equipped with these standard features we can present the
dynamics of the calculus.

\subsubsection{Operational semantics} 

Finally, we introduce the computational dynamics. What marks these
algebras as distinct from other more traditionally studied algebraic
structures, e.g. vector spaces or polynomial rings, is the manner in
which dynamics is captured. In traditional structures, dynamics is typically
expressed through morphisms between such structures, as in linear maps
between vector spaces or morphisms between rings. In algebras
associated with the semantics of computation, the dynamics is
expressed as part of the algebraic structure itself, through a
reduction reduction relation typically denoted by $\red$. Below, we
give a recursive presentation of this relation for the calculus used
in the encoding.

$\red \subseteq \pi \times \pi$
$\red : \pi \to \mathcal{P}(\pi)$

\begin{mathpar}
  \inferrule* [lab=Comm] { \textsf{match}( x_{src}, x_{trgt} ) } { x_{trgt}?(y)P \; | \; x_{src}!\langle {Q} \rangle \red P\{\quotep{Q}/y}\} }
  \and \\
  \inferrule* [lab=Par] {{P} \red {P}'} {{{P} | {Q}} \red {{P}' | {Q}}}
  \and
  \inferrule* [lab=Equiv]{{{P} \scong {P}'} \andalso {{P}' \red {Q}'} \andalso {{Q}' \scong {Q}}}{{P} \red {Q}}
\end{mathpar}

\begin{eqnarray*}
  match_{\equiv} (\quotep{P},\quotep{Q}) & := & P \equiv Q \\
  match_{\dagger}(\quotep{P},\quotep{Q}) & := & \forall R. P|Q \red^{*} R => R \red^{*} 0 \\
  match_{K}(\quotep{P},\quotep{Q}) & := & K \mbox{ for some context } K
\end{eqnarray*}

$u?(x)P | u!\langle Q \rangle \red P\{\quotep{Q}/x\}$

%We write $\wred$ for $\red^*$, and $P\red$ if $\exists Q $ such that $ P \red Q$.
We write $P\red$ if $\exists Q $ such that $ P \red Q$ and $P\not\red$, otherwise.

\section{Replication}

As mentioned before, it is known that replication (and hence
recursion) can be implemented in a higher-order process algebra
\cite{SangiorgiWalker}. As our first example of calculation with the
machinery thus far presented we give the construction explicitly in
the {\rhoc}.

\begin{eqnarray}
	D_{x} & := & \prefix{x}{y}{(\binpar{\outputp{x}{y}}{@{y}})} \nonumber\\
	\bangp_{x}{P} & := & \binpar{{x}!\langle{\binpar{D_{x}}{P}}\rangle}{D_{x}} \nonumber
\end{eqnarray}

\begin{eqnarray}
	\bangp_{x}{P} & & \nonumber\\
	=
	& {x}!\langle{(\prefix{x}{y}{(\outputp{x}{y} | @{y})) | P}}\rangle 
	      | \prefix{x}{y}{(\outputp{x}{y} | @{y})} & \nonumber\\
	\red
	& (\outputp{x}{y} | @{y})\substn{\quotep{(\prefix{x}{y}{(@{y} | \outputp{x}{y})) | P}}}{y} & \nonumber\\
	=
	& \outputp{x}{\quotep{(\prefix{x}{y}{(\outputp{x}{y} | @{y})) | P}}}
	  | {(\prefix{x}{y}{(\outputp{x}{y} | @{y})) | P}} & \nonumber\\
	\red
	& \ldots & \nonumber\\
	\red^*
	& P | P | \ldots & \nonumber
\end{eqnarray}

Of course, this encoding, as an implementation, runs away, unfolding
$\bangp{P}$ eagerly. A lazier and more implementable replication
operator, restricted to input-guarded processes, may be obtained as follows.

\begin{eqnarray}
\bangp{\prefix{u}{v}{P}} 
	:= 
	\binpar{\lift{x}{\prefix{u}{v}{(\binpar{D(x)}{P})}}}{D(x)} \nonumber
\end{eqnarray}

\begin{remark}
  Note that the lazier definition still does not deal with summation
  or mixed summation (i.e. sums over input and output). The reader is
  invited to construct definitions of replication that deal with these
  features. 

  Further, the definitions are parameterized in a name, $x$. Can you,
  gentle reader, make a definition that eliminates this parameter and
  guarantees no accidental interaction between the replication
  machinery and the process being replicated -- i.e. no accidental
  sharing of names used by the process to get its work done and the
  name(s) used by the replication to effect copying. This latter
  revision of the definition of replication is crucial to obtaining
  the expected identity $!!P \sim !P$.
\end{remark}

\begin{remark}\label{rem:paradoxical_combinator}
  The reader familiar with the lambda calculus will have noticed the
  similarity between $D$ and the paradoxical combinator.

  [Ed. note: the existence of this seems to suggest we have to be more
  restrictive on the set of processes and names we admit if we are to
  support no-cloning.]
\end{remark}

\subsubsection{Bisimulation}

The computational dynamics gives rise to another kind of equivalence,
the equivalence of computational behavior. As previously mentioned
this is typically captured \emph{via} some form of bisimulation.

% The notion we use in this paper is weak barbed bisimulation
% \cite{milner91polyadicpi}.

The notion we use in this paper is derived from weak barbed
bisimulation \cite{milner91polyadicpi}. 

\begin{definition}
An \emph{observation relation}, $\downarrow_{\mathcal N}$, over a set
of names, $\mathcal N$, is the smallest relation satisfying the rules
below.

\infrule[Out-barb]{y \in {\mathcal N}, \; x \nameeq y}
		  {\outputp{x}{v} \downarrow_{\mathcal N} x}
\infrule[Par-barb]{\mbox{$P\downarrow_{\mathcal N} x$ or $Q\downarrow_{\mathcal N} x$}}
		  {\binpar{P}{Q} \downarrow_{\mathcal N} x}

We write $P \Downarrow_{\mathcal N} x$ if there is $Q$ such that 
$P \wred Q$ and $Q \downarrow_{\mathcal N} x$.
\end{definition}

\begin{definition}
%\label{def.bbisim}
An  ${\mathcal N}$-\emph{barbed bisimulation} over a set of names, ${\mathcal N}$, is a symmetric binary relation 
${\mathcal S}_{\mathcal N}$ between agents such that $P\rel{S}_{\mathcal N}Q$ implies:
\begin{enumerate}
\item If $P \red P'$ then $Q \wred Q'$ and $P'\rel{S}_{\mathcal N} Q'$.
\item If $P\downarrow_{\mathcal N} x$, then $Q\Downarrow_{\mathcal N} x$.
\end{enumerate}
$P$ is ${\mathcal N}$-barbed bisimilar to $Q$, written
$P \wbbisim_{\mathcal N} Q$, if $P \rel{S}_{\mathcal N} Q$ for some ${\mathcal N}$-barbed bisimulation ${\mathcal S}_{\mathcal N}$.
\end{definition}

$\mathcal{R} \subseteq \pi \times \pi$

$P \mathcal{R} Q => \forall P'. P \red P' \Rightarrow \exists Q'. Q \red Q', P' \mathcal{R} Q'$

$P \vdash x \Rightarrow Q \vdash x$

\begin{mathpar}
  \inferrule*[lab=Out-barb]{x \nameeq y}{{y}!\langle{Q}\rangle \vdash x}
  \and
  \inferrule*[lab=Par-barb]{\mbox{$P\vdash x$ or $Q\vdash x$}}{\binpar{P}{Q} \vdash x}
\end{mathpar}

\subsubsection{Contexts}

One of the principle advantages of computational calculi like the
$\pi$-calculus is a well-defined notion of context,
contextual-equivalence and a correlation between
contextual-equivalence and notions of bisimulation. The notion of
context allows the decomposition of a process into (sub-)process and
its syntactic environment, its context. Thus, a context may be
thought of as a process with a ``hole'' (written $\Box$) in it. The
application of a context $M$ to a process $P$, written $M[P]$, is
tantamount to filling the hole in $M$ with $P$. In this paper we do
not need the full weight of this theory, but do make use of the notion
of context in the proof the main theorem. 

\begin{mathpar}
  \inferrule* [lab=summation] {} {{M_{M},M_{N}} \bc \Box \;|\; x.M_{A} \;|\; M_{M}+M_{N}}
  \and
  \inferrule* [lab=agent] {} {{M_{A}} \bc (\vec{x})M_{P} \;| \; \clift{P_0,\ldots,M_{P},\ldots,P_N}}
  \and \\
  \inferrule* [lab=process] {} {{M_{P}} \bc M_{N} \;| \;P|M_{P} }
\end{mathpar} 

\begin{mathpar}
  \inferrule* [lab=sychronization] {} {M_{N} \bc \Box \;|\; x?M_{F} \;|\; x!M_{C}}
  \and
  \inferrule* [lab=abstraction] {} {{M_{F}} \bc (x)M_{P} }
  \and
  \inferrule* [lab=concretion] {} {{M_{C}} \bc \langle M_{P} \rangle }
  \and \\
  \inferrule* [lab=process] {} {{M_{P}} \bc M_{N} \;| \;P|M_{P} }
\end{mathpar}

\begin{definition}[contextual application] Given a context $M$, and
  process $P$, we define the \emph{contextual application}, $M[P] :=
  M\{P/\Box\}$. That is, the contextual application of M to P is the
  substitution of $P$ for $\Box$ in $M$.
\end{definition}

$\meaningof{-} : L \to \mathcal{P}(\pi)$

\begin{mathpar}
  \inferrule* [lab=collection] {} {\meaningof{true} = \pi, \and \meaningof{~E} = \pi \setminus \meaningof{E}, \and \meaningof{E_{1} \& E_{2}} = \meaningof{E_{1}} \cap \meaningof{E_{2}}}
\end{mathpar}

\begin{mathpar}
  \inferrule* [lab=structure] {} {\meaningof{0} = \{ P \in \pi | P \equiv 0 \}, \and \\ \meaningof{E_1 | E_2} = \{ P \in \pi | P \equiv P_{1} | P_{2}, P_{1} \in \meaningof{E_{1}}, P_{2} \in \meaningof{E_2}\} }
\end{mathpar}

\begin{mathpar}
 \inferrule* [lab=behavior] {} {\meaningof{\langle a?b \rangle E} = \{ P \in \pi | P \equiv Q | u?(y)P', \\ \and \\\\ \and \\ \;\;\; u \in \meaningof{a}, \forall z.P'\{z/y\} \in \meaningof{E\{z/b\}}\}, \and \\ \meaningof{a!E} = \{ P \in \pi | P \equiv Q | x!\langle P' \rangle, x \in \meaningof{a} P' \in \meaningof{E}\} }
\end{mathpar}

\begin{mathpar}
 \inferrule* [lab=nominal] {} {\meaningof{\quotep{E}} = \{ \quotep{P} \in \quotep{\pi} | P \in \meaningof{E} \}, \and \meaningof{\quotep{P}} = \{ \quotep{Q} \in \quotep{\pi} | P \equiv Q \} \and \\ \meaningof{@\quotep{E}} = \{ P \in \pi | P \equiv @x, x \in \meaningof{E} \}}
\end{mathpar}

\begin{eqnarray*}
  \\
  \meaningof{-} : TS \to ST
\end{eqnarray*}

\begin{eqnarray*}
  \\
  L : TS \to ST
\end{eqnarray*}

\begin{eqnarray*}
  \\
  P \models E \iff P \in \meaningof{E}
\end{eqnarray*}

\begin{eqnarray*}
  P \approx_{L} Q \iff \forall E \in L. P \models E \iff Q \models E
\end{eqnarray*}

\begin{eqnarray*}
  P \approx_{K} Q
\end{eqnarray*}

\begin{eqnarray*}
  P \approx Q
\end{eqnarray*}

$\approx_{K} = \approx = \approx_{L}$

\subsubsection{Contextual duality}

Note that contexts extend the quotation operation to a family of
operations from processes to names. Given a context, $M$, we can
define a \emph{nominal context}, $\quotep{M}$ by $\quotep{M}[P] :=
\quotep{M[P]}$. To foreshadow what is to come we observe that these
operations enjoy a duality with processes very much like the duality
between vectors and maps from vectors to scalars.

Further, because the calculus is essentially higher-order, we have a
correspondence between contexts and processes. More specifically,
given a name $x$ and a context $M$ we can construct $M^{*}_{x}$ such
that 

\begin{mathpar}
  M^{*}_{x} | \lift{x}{P} \red M[P]
\end{mathpar}

namely,

\begin{mathpar}
  M^{*}_{x} := x?(u).M[\dropn{u}]
\end{mathpar}

The dependence of $M^{*}_{x}$ on a name makes it an abstraction, 

\begin{mathpar}
  M^{*} := (x)x?(u).M[\dropn{u}]
\end{mathpar}

\subsection{Additional notation}

It will sometimes be convenient to denote the process a name
quotes. We already have the notation $x = \quotep{P}$, but it will be
convenient to introduce an alternate notation, $\procn{x}$, when we
want to emphasize the connection to the use of the name. Note that, by
virtue of name equivalence, $\quotep{\procn{x}} \nameeq x$; so, the
notation is consistent with previous definitions.

Further, because names have structure it is possible to effect
substitutions on the basis of that structure. This means we need to
upgrade our notation for substitutions, which we accomplish by
adapting comprehension notation. Thus,

\begin{mathpar}
  P\{ y / x : x \in S \}
\end{mathpar}

is interpreted to mean the process derived from P by replacing (in a
capture-avoiding manner) each occurrence of $x$ in $S$ by $y$. For example,

\begin{mathpar}
  P\{ \quotep{\procn{x}|\procn{x}} / x : x \in \freenames{P} \}
\end{mathpar}

will replace each (occurrence) of a free name $x$ in $P$ by
$\quotep{\procn{x}|\procn{x}}$.

Also, we will avail ourselves of the notation $x^{L}$ and $x^{R}$ to
denote injections of a name into disjoint copies of the name
space. There are numerous ways to accomplish this. One example can be
found in \cite{MeredithR05}. This notation overloads to vectors of
names: $\vec{x}^{\pi} := (x_{i}^{\pi} \; : \; 0 \leq i < |\vec{x}| )$ where $\pi \in \{L,R\}$.

We also use $P^{\Box} := P|\Box$.

In \cite{MeredithR05} an interpretation of the new operator is
given. It turns out that there are several possible interpretations
all enjoying the requisite algebraic properties of the operator (see
\cite{milner91polyadicpi}). We will therefore make liberal use of
$(\nu\; \vec{x})P$.

% subsection the_syntax_and_semantics_of_the_notation_system (end)   

\input{qm2pi.qmops} 

\input{qm2pi.sterngerlach} 

\input{qm2pi.metric} 

% section concurrent_process_calculi (end)

%\input{qm2pi.proofsketch}

% section proof sketch (end)

%\input{qm2pi.slviaknots} 

% section spatial logic via knots (end)

\input{qm2pi.conclusion}

% section conclusion (end)

%\input{qm2pi.dtcodes} 

% section wiring algorithm (end)

\input{qm2pi.ack} 

% section acknowledgments (end)

\newpage


\bibliographystyle{plain}   
\bibliography{../../biblios/main.bib}

\input{qm2pi.rhodetails}

\end{document}

 

\documentclass[12pt]{llncs}
%\documentclass{jktr}

\usepackage[pdftex]{hyperref}                   
\usepackage {listings}
\usepackage {mathpartir}
\usepackage{bcprules}
%\usepackage{listings}
                       
\usepackage{graphicx} 
%\usepackage[margins=2.5cm,nohead,nofoot]{geometry}
%\usepackage{geometry}
\usepackage{amsfonts}
\usepackage{amstext}
\usepackage{latexsym}
\usepackage{amssymb}
\usepackage{color}


%\include{myPreamble}
\include{qm2pi.local} 

%\ifpdf
%\usepackage[pdftex]{graphicx}
%\else
%\usepackage{graphicx}
%\fi

 % \ifpdf
%  \usepackage{pdfsync}
%  \if


%\title{Brief Article}
%\author{David F. Snyder}
%\author{L.G. Meredith}

%\address{Dept. of Math., Texas State University--San Marcos, San Marcos, TX 78666}
       
\pagestyle{empty}


\begin{document}

\lstset{language=[Objective]Caml,frame=shadowbox}

\input{qm2pi.front}

% section front matter (end)

\input{qm2pi.intro} 
 
% section introduction (end)

% \input{qm2pi.knotations} 

% section notation (end)

\input{qm2pi.process.calculi} 

% section concurrent_process_calculi_and_spatial_logics_ (end)
    
%\input{qm2pi.knots2pi} 

%\input{qm2pi.trefoil} 

%\input{qm2pi.mainthm} 

% subsection basic_interpretation (end)

%\input{qm2pi.rho.presentation} 
\subsection{The syntax and semantics of the notation system}\label{sub:the_syntax_and_semantics_of_the_notation_system} % (fold)

We now summarize a technical presentation of the calculus that
embodies our theory of dynamics. The typical presentation of such a
calculus follows the style of giving generators and relations on
them. The grammar, below, describing term constructors, freely
generates the set of processes, $\Proc$. This set is then quotiented
by a relation known as structural congruence and it is over this set
that the notion of dynamics is expressed. This presentation is
essentially that of \cite{MeredithR05} with the addition of
polyadicity and summation. For readability we have relegated some of
the technical subtleties to an appendix.

\subsubsection{Process grammar}\label{subsub:process_grammar}

\begin{mathpar}
  \inferrule* [lab=synchronization] {} {{M} \bc \pzero \;|\; x?F \;|\; x!C }
  \and
  \inferrule* [lab=abstraction] {} {{F} \bc (x)P}
  \and
  \inferrule* [lab=concretion] {} {{C} \bc \langle Q \rangle}
  \and
  \inferrule* [lab=process] {} {{P,Q} \bc M \;| \;P|Q \;|\; @{x}}
  \and
  \inferrule* [lab=name] {} {{x} \bc \quotep{P}}
\end{mathpar} 

Note that $\vec{x}$ (resp. $\vec{P}$) denotes a vector of names
(resp. processes) of length $|\vec{x}|$ (resp. $|\vec{P}|$). We adopt
the following useful abbreviations.

\begin{mathpar}
   x?(\vec{y}).P := x.(\vec{y})P \and  x\clift{\vec{P}} := x.\clift{\vec{P}}
   \and x!(y) := \lift{x}{\dropn{y}}
   \and \Pi_{i=0}^{n-1}P_i := P_0 | \ldots | P_{n-1}
\end{mathpar}

\subsubsection{Structural congruence}

\paragraph{Free and bound names and alpha-equivalence.} At the
core of structural equivalence is alpha-equivalence which identifies
process that are the same up to a change of variable. Formally, we
recognize the distinction between free and bound names. The free names
of a process, $\freenames{P}$, may be calculated recursively as
follows:

\begin{mathpar}
\freenames{\pzero} := \emptyset
  \and \\
  \freenames{x?(y).P} := \{ x \} \cup (\freenames{P} \setminus \{ y \})
  \and 
  \freenames{x!\langle P \rangle} := \{ x \} \cup \{ P \} 
  \and \\
  \freenames{P|Q} := \freenames{P} \cup \freenames{Q}
  \and \\
  \freenames{@{x}} := \{ x \}
\end{mathpar}

$\pi$
$\quotep{\pi}$

$\freenames{-} : \pi \to \mathcal{P}(\quotep{\pi})$

\begin{eqnarray*}
  \freenames{\pzero} & := & \emptyset \\
  \freenames{x?(y).P} & := & \{ x \} \cup (\freenames{P} \setminus \{ y \}) \\
  \freenames{x!\langle P \rangle} & := & \{ x \} \cup \{ P \} \\
  \freenames{P|Q} & := & \freenames{P} \cup \freenames{Q} \\
  \freenames{\dropn{x}} & := & \{ x \}
\end{eqnarray*}

The bound names of a process, $\boundnames{P}$, are those names occurring in $P$
that are not free. For example, in $x?(y).0$, the name $x$ is free, while $y$ is bound.

\begin{mathpar}
  \inferrule* [lab=monoidal-laws] {} { P|Q \equiv Q|P \and P|0 \equiv P \and P|(Q|R) \equiv (P|Q)|R }
\end{mathpar}

\begin{mathpar}
  \inferrule* [lab=alpha-equivalence] {} { (x)P \equiv (y)P\{y/x\} \and y \not\in \freenames{P} }
\end{mathpar}

\begin{definition}
Then two processes, $P,Q$, are alpha-equivalent if $P = Q\{\vec{y}/\vec{x}\}$ for
some $\vec{x} \in \boundnames{Q},\vec{y} \in \boundnames{P}$, where $Q\{\vec{y}/\vec{x}\}$
denotes the capture-avoiding substitution of $\vec{y}$ for $\vec{x}$ in $Q$.
\end{definition}

\begin{definition}
  The {\em structural congruence} \cite{SangiorgiWalker} , $\equiv$,
  between processes is the least congruence containing
  alpha-equivalence, satisfying the abelian monoid laws
  (associativity, commutativity and $\pzero$ as identity) for parallel
  composition $|$ and for summation $+$.
\end{definition}

\subsection{Name equivalence}

We take name equivalence, written $\nameeq$, to be the smallest
equivalence relation generated by the following rules.

\begin{mathpar}
\inferrule*[lab=Quote-drop]
{ }
{ \quotep{@{x}} \nameeq x }

\inferrule*[lab=Struct-equiv]
{ P \scong Q }
{ \quotep{P} \nameeq \quotep{Q} }
\end{mathpar}

The astute reader will have noticed that the mutual recursion of names
and processes imposes a mutual recursion on alpha-equivalence and
structural equivalence via name-equivalence. Fortunately, all of this
works out pleasantly and we may calculate in the natural way, free of
concern. The reader interested in the details is referred to the
appendix \ref{appendix:rho_details}.

\subsection{Substitution}

We use $\Proc$ for the set of processes, $\QProc$ for the set of
names, and $\id{\{}\vec{y} / \vec{x} \id{\}}$ to denote partial maps,
$s : \QProc \rightarrow \QProc$. A map, $s$ lifts, uniquely, to a map
on process terms, $\widehat{s} : \Proc \rightarrow \Proc$ by the
following equations.

\begin{mathpar}
  (0) \psubstp{Q}{P} := 0 \\
  (R \juxtap S) \psubstp{Q}{P}
  :=    
  (R)\psubstp{Q}{P} \juxtap (S) \psubstp{Q}{P} \\
  (x?(y).R) \psubstp{Q}{P}    
  :=    
  (x)\substp{Q}{P} (z)\concat( (R \psubstn{z}{y}) \psubstp{Q}{P} ) \\
  (\lift{x}{R}) \psubstp{Q}{P}  
  :=
  \lift{(x)\substp{Q}{P}}{ R \psubstp{Q}{P} } \\
%   (\dropn{x})  \psubstp{Q}{P}       
%   := 
%   \left\{ 
%     \begin{array}{ccc} 
%       \dropn{\quotep{Q}} & & x \nameeq \quotep{P} \\
%       \dropn{x} & & otherwise \\
%     \end{array}
%   \right. 
  (\dropn{x})  \psubstp{Q}{P}       
  := 
  \left\{ 
    \begin{array}{ccc} 
      Q & & x \nameeq \quotep{P} \\
      \dropn{x} & & otherwise \\
    \end{array}
  \right.
\end{mathpar}
 

where

\begin{eqnarray}
  (x)\id{\{} \lpquote Q \rpquote / \lpquote P \rpquote \id{\}}            = 
  \left\{ 
    \begin{array}{ccc}
      \lpquote Q \rpquote & & x \nameeq \lpquote P \rpquote \\
      x & & otherwise \\
    \end{array}
  \right. \nonumber
\end{eqnarray}

and $z$ is chosen distinct from $\quotep{P}$, $\quotep{Q}$, the free
names in $Q$, and all the names in $R$. Our $\alpha$-equivalence will
be built in the standard way from this substitution.

\begin{remark}\label{rem:no_self_referential_names}
  One consequence of these definitions is that $\forall P. \quotep{P}
  \not\in \freenames{P}$.
\end{remark}

\subsection{ Dynamic quote: an example }

Anticipating something of what's to come, consider applying the
substitution, $\widehat{\id{\{}u / z \id{\}}}$, to the following pair
of processes, $\lift{w}{y!(z)}$ and $w[ \lpquote y!(z) \rpquote ]$.

\begin{eqnarray}
	\lift{w}{y!(z)}\widehat{\id{\{}u / z \id{\}}}
		& = &
		\lift{w}{y!(u)} \nonumber\\
	w[ \lpquote y!(z) \rpquote ] \widehat{ \id{\{}u / z \id{\}} }
		& = &
		w[ \lpquote y!(z) \rpquote ] \nonumber
\end{eqnarray}

Because the body of the process between quotes is impervious to
substitution, we get radically different answers. In fact, by
examining the first process in an input context,
e.g. $x?(z).\lift{w}{y!(z)}$, we see that the process under the lift
operator may be shaped by prefixed inputs binding a name inside it. In
this sense, the lift operator will be seen as a way to dynamically
construct processes before reifying them as names.

Finally equipped with these standard features we can present the
dynamics of the calculus.

\subsubsection{Operational semantics} 

Finally, we introduce the computational dynamics. What marks these
algebras as distinct from other more traditionally studied algebraic
structures, e.g. vector spaces or polynomial rings, is the manner in
which dynamics is captured. In traditional structures, dynamics is typically
expressed through morphisms between such structures, as in linear maps
between vector spaces or morphisms between rings. In algebras
associated with the semantics of computation, the dynamics is
expressed as part of the algebraic structure itself, through a
reduction reduction relation typically denoted by $\red$. Below, we
give a recursive presentation of this relation for the calculus used
in the encoding.

$\red \subseteq \pi \times \pi$
$\red : \pi \to \mathcal{P}(\pi)$

\begin{mathpar}
  \inferrule* [lab=Comm] { \textsf{match}( x_{src}, x_{trgt} ) } { x_{trgt}?(y)P \; | \; x_{src}!\langle {Q} \rangle \red P\{\quotep{Q}/y}\} }
  \and \\
  \inferrule* [lab=Par] {{P} \red {P}'} {{{P} | {Q}} \red {{P}' | {Q}}}
  \and
  \inferrule* [lab=Equiv]{{{P} \scong {P}'} \andalso {{P}' \red {Q}'} \andalso {{Q}' \scong {Q}}}{{P} \red {Q}}
\end{mathpar}

\begin{eqnarray*}
  match_{\equiv} (\quotep{P},\quotep{Q}) & := & P \equiv Q \\
  match_{\dagger}(\quotep{P},\quotep{Q}) & := & \forall R. P|Q \red^{*} R => R \red^{*} 0 \\
  match_{K}(\quotep{P},\quotep{Q}) & := & K \mbox{ for some context } K
\end{eqnarray*}

$u?(x)P | u!\langle Q \rangle \red P\{\quotep{Q}/x\}$

%We write $\wred$ for $\red^*$, and $P\red$ if $\exists Q $ such that $ P \red Q$.
We write $P\red$ if $\exists Q $ such that $ P \red Q$ and $P\not\red$, otherwise.

\section{Replication}

As mentioned before, it is known that replication (and hence
recursion) can be implemented in a higher-order process algebra
\cite{SangiorgiWalker}. As our first example of calculation with the
machinery thus far presented we give the construction explicitly in
the {\rhoc}.

\begin{eqnarray}
	D_{x} & := & \prefix{x}{y}{(\binpar{\outputp{x}{y}}{@{y}})} \nonumber\\
	\bangp_{x}{P} & := & \binpar{{x}!\langle{\binpar{D_{x}}{P}}\rangle}{D_{x}} \nonumber
\end{eqnarray}

\begin{eqnarray}
	\bangp_{x}{P} & & \nonumber\\
	=
	& {x}!\langle{(\prefix{x}{y}{(\outputp{x}{y} | @{y})) | P}}\rangle 
	      | \prefix{x}{y}{(\outputp{x}{y} | @{y})} & \nonumber\\
	\red
	& (\outputp{x}{y} | @{y})\substn{\quotep{(\prefix{x}{y}{(@{y} | \outputp{x}{y})) | P}}}{y} & \nonumber\\
	=
	& \outputp{x}{\quotep{(\prefix{x}{y}{(\outputp{x}{y} | @{y})) | P}}}
	  | {(\prefix{x}{y}{(\outputp{x}{y} | @{y})) | P}} & \nonumber\\
	\red
	& \ldots & \nonumber\\
	\red^*
	& P | P | \ldots & \nonumber
\end{eqnarray}

Of course, this encoding, as an implementation, runs away, unfolding
$\bangp{P}$ eagerly. A lazier and more implementable replication
operator, restricted to input-guarded processes, may be obtained as follows.

\begin{eqnarray}
\bangp{\prefix{u}{v}{P}} 
	:= 
	\binpar{\lift{x}{\prefix{u}{v}{(\binpar{D(x)}{P})}}}{D(x)} \nonumber
\end{eqnarray}

\begin{remark}
  Note that the lazier definition still does not deal with summation
  or mixed summation (i.e. sums over input and output). The reader is
  invited to construct definitions of replication that deal with these
  features. 

  Further, the definitions are parameterized in a name, $x$. Can you,
  gentle reader, make a definition that eliminates this parameter and
  guarantees no accidental interaction between the replication
  machinery and the process being replicated -- i.e. no accidental
  sharing of names used by the process to get its work done and the
  name(s) used by the replication to effect copying. This latter
  revision of the definition of replication is crucial to obtaining
  the expected identity $!!P \sim !P$.
\end{remark}

\begin{remark}\label{rem:paradoxical_combinator}
  The reader familiar with the lambda calculus will have noticed the
  similarity between $D$ and the paradoxical combinator.

  [Ed. note: the existence of this seems to suggest we have to be more
  restrictive on the set of processes and names we admit if we are to
  support no-cloning.]
\end{remark}

\subsubsection{Bisimulation}

The computational dynamics gives rise to another kind of equivalence,
the equivalence of computational behavior. As previously mentioned
this is typically captured \emph{via} some form of bisimulation.

% The notion we use in this paper is weak barbed bisimulation
% \cite{milner91polyadicpi}.

The notion we use in this paper is derived from weak barbed
bisimulation \cite{milner91polyadicpi}. 

\begin{definition}
An \emph{observation relation}, $\downarrow_{\mathcal N}$, over a set
of names, $\mathcal N$, is the smallest relation satisfying the rules
below.

\infrule[Out-barb]{y \in {\mathcal N}, \; x \nameeq y}
		  {\outputp{x}{v} \downarrow_{\mathcal N} x}
\infrule[Par-barb]{\mbox{$P\downarrow_{\mathcal N} x$ or $Q\downarrow_{\mathcal N} x$}}
		  {\binpar{P}{Q} \downarrow_{\mathcal N} x}

We write $P \Downarrow_{\mathcal N} x$ if there is $Q$ such that 
$P \wred Q$ and $Q \downarrow_{\mathcal N} x$.
\end{definition}

\begin{definition}
%\label{def.bbisim}
An  ${\mathcal N}$-\emph{barbed bisimulation} over a set of names, ${\mathcal N}$, is a symmetric binary relation 
${\mathcal S}_{\mathcal N}$ between agents such that $P\rel{S}_{\mathcal N}Q$ implies:
\begin{enumerate}
\item If $P \red P'$ then $Q \wred Q'$ and $P'\rel{S}_{\mathcal N} Q'$.
\item If $P\downarrow_{\mathcal N} x$, then $Q\Downarrow_{\mathcal N} x$.
\end{enumerate}
$P$ is ${\mathcal N}$-barbed bisimilar to $Q$, written
$P \wbbisim_{\mathcal N} Q$, if $P \rel{S}_{\mathcal N} Q$ for some ${\mathcal N}$-barbed bisimulation ${\mathcal S}_{\mathcal N}$.
\end{definition}

$\mathcal{R} \subseteq \pi \times \pi$

$P \mathcal{R} Q => \forall P'. P \red P' \Rightarrow \exists Q'. Q \red Q', P' \mathcal{R} Q'$

$P \vdash x \Rightarrow Q \vdash x$

\begin{mathpar}
  \inferrule*[lab=Out-barb]{x \nameeq y}{{y}!\langle{Q}\rangle \vdash x}
  \and
  \inferrule*[lab=Par-barb]{\mbox{$P\vdash x$ or $Q\vdash x$}}{\binpar{P}{Q} \vdash x}
\end{mathpar}

\subsubsection{Contexts}

One of the principle advantages of computational calculi like the
$\pi$-calculus is a well-defined notion of context,
contextual-equivalence and a correlation between
contextual-equivalence and notions of bisimulation. The notion of
context allows the decomposition of a process into (sub-)process and
its syntactic environment, its context. Thus, a context may be
thought of as a process with a ``hole'' (written $\Box$) in it. The
application of a context $M$ to a process $P$, written $M[P]$, is
tantamount to filling the hole in $M$ with $P$. In this paper we do
not need the full weight of this theory, but do make use of the notion
of context in the proof the main theorem. 

\begin{mathpar}
  \inferrule* [lab=summation] {} {{M_{M},M_{N}} \bc \Box \;|\; x.M_{A} \;|\; M_{M}+M_{N}}
  \and
  \inferrule* [lab=agent] {} {{M_{A}} \bc (\vec{x})M_{P} \;| \; \clift{P_0,\ldots,M_{P},\ldots,P_N}}
  \and \\
  \inferrule* [lab=process] {} {{M_{P}} \bc M_{N} \;| \;P|M_{P} }
\end{mathpar} 

\begin{mathpar}
  \inferrule* [lab=sychronization] {} {M_{N} \bc \Box \;|\; x?M_{F} \;|\; x!M_{C}}
  \and
  \inferrule* [lab=abstraction] {} {{M_{F}} \bc (x)M_{P} }
  \and
  \inferrule* [lab=concretion] {} {{M_{C}} \bc \langle M_{P} \rangle }
  \and \\
  \inferrule* [lab=process] {} {{M_{P}} \bc M_{N} \;| \;P|M_{P} }
\end{mathpar}

\begin{definition}[contextual application] Given a context $M$, and
  process $P$, we define the \emph{contextual application}, $M[P] :=
  M\{P/\Box\}$. That is, the contextual application of M to P is the
  substitution of $P$ for $\Box$ in $M$.
\end{definition}

$\meaningof{-} : L \to \mathcal{P}(\pi)$

\begin{mathpar}
  \inferrule* [lab=collection] {} {\meaningof{true} = \pi, \and \meaningof{~E} = \pi \setminus \meaningof{E}, \and \meaningof{E_{1} \& E_{2}} = \meaningof{E_{1}} \cap \meaningof{E_{2}}}
\end{mathpar}

\begin{mathpar}
  \inferrule* [lab=structure] {} {\meaningof{0} = \{ P \in \pi | P \equiv 0 \}, \and \\ \meaningof{E_1 | E_2} = \{ P \in \pi | P \equiv P_{1} | P_{2}, P_{1} \in \meaningof{E_{1}}, P_{2} \in \meaningof{E_2}\} }
\end{mathpar}

\begin{mathpar}
 \inferrule* [lab=behavior] {} {\meaningof{\langle a?b \rangle E} = \{ P \in \pi | P \equiv Q | u?(y)P', \\ \and \\\\ \and \\ \;\;\; u \in \meaningof{a}, \forall z.P'\{z/y\} \in \meaningof{E\{z/b\}}\}, \and \\ \meaningof{a!E} = \{ P \in \pi | P \equiv Q | x!\langle P' \rangle, x \in \meaningof{a} P' \in \meaningof{E}\} }
\end{mathpar}

\begin{mathpar}
 \inferrule* [lab=nominal] {} {\meaningof{\quotep{E}} = \{ \quotep{P} \in \quotep{\pi} | P \in \meaningof{E} \}, \and \meaningof{\quotep{P}} = \{ \quotep{Q} \in \quotep{\pi} | P \equiv Q \} \and \\ \meaningof{@\quotep{E}} = \{ P \in \pi | P \equiv @x, x \in \meaningof{E} \}}
\end{mathpar}

\begin{eqnarray*}
  \\
  \meaningof{-} : TS \to ST
\end{eqnarray*}

\begin{eqnarray*}
  \\
  L : TS \to ST
\end{eqnarray*}

\begin{eqnarray*}
  \\
  P \models E \iff P \in \meaningof{E}
\end{eqnarray*}

\begin{eqnarray*}
  P \approx_{L} Q \iff \forall E \in L. P \models E \iff Q \models E
\end{eqnarray*}

\begin{eqnarray*}
  P \approx_{K} Q
\end{eqnarray*}

\begin{eqnarray*}
  P \approx Q
\end{eqnarray*}

$\approx_{K} = \approx = \approx_{L}$

\subsubsection{Contextual duality}

Note that contexts extend the quotation operation to a family of
operations from processes to names. Given a context, $M$, we can
define a \emph{nominal context}, $\quotep{M}$ by $\quotep{M}[P] :=
\quotep{M[P]}$. To foreshadow what is to come we observe that these
operations enjoy a duality with processes very much like the duality
between vectors and maps from vectors to scalars.

Further, because the calculus is essentially higher-order, we have a
correspondence between contexts and processes. More specifically,
given a name $x$ and a context $M$ we can construct $M^{*}_{x}$ such
that 

\begin{mathpar}
  M^{*}_{x} | \lift{x}{P} \red M[P]
\end{mathpar}

namely,

\begin{mathpar}
  M^{*}_{x} := x?(u).M[\dropn{u}]
\end{mathpar}

The dependence of $M^{*}_{x}$ on a name makes it an abstraction, 

\begin{mathpar}
  M^{*} := (x)x?(u).M[\dropn{u}]
\end{mathpar}

\subsection{Additional notation}

It will sometimes be convenient to denote the process a name
quotes. We already have the notation $x = \quotep{P}$, but it will be
convenient to introduce an alternate notation, $\procn{x}$, when we
want to emphasize the connection to the use of the name. Note that, by
virtue of name equivalence, $\quotep{\procn{x}} \nameeq x$; so, the
notation is consistent with previous definitions.

Further, because names have structure it is possible to effect
substitutions on the basis of that structure. This means we need to
upgrade our notation for substitutions, which we accomplish by
adapting comprehension notation. Thus,

\begin{mathpar}
  P\{ y / x : x \in S \}
\end{mathpar}

is interpreted to mean the process derived from P by replacing (in a
capture-avoiding manner) each occurrence of $x$ in $S$ by $y$. For example,

\begin{mathpar}
  P\{ \quotep{\procn{x}|\procn{x}} / x : x \in \freenames{P} \}
\end{mathpar}

will replace each (occurrence) of a free name $x$ in $P$ by
$\quotep{\procn{x}|\procn{x}}$.

Also, we will avail ourselves of the notation $x^{L}$ and $x^{R}$ to
denote injections of a name into disjoint copies of the name
space. There are numerous ways to accomplish this. One example can be
found in \cite{MeredithR05}. This notation overloads to vectors of
names: $\vec{x}^{\pi} := (x_{i}^{\pi} \; : \; 0 \leq i < |\vec{x}| )$ where $\pi \in \{L,R\}$.

We also use $P^{\Box} := P|\Box$.

In \cite{MeredithR05} an interpretation of the new operator is
given. It turns out that there are several possible interpretations
all enjoying the requisite algebraic properties of the operator (see
\cite{milner91polyadicpi}). We will therefore make liberal use of
$(\nu\; \vec{x})P$.

% subsection the_syntax_and_semantics_of_the_notation_system (end)   

\input{qm2pi.qmops} 

\input{qm2pi.sterngerlach} 

\input{qm2pi.metric} 

% section concurrent_process_calculi (end)

%\input{qm2pi.proofsketch}

% section proof sketch (end)

%\input{qm2pi.slviaknots} 

% section spatial logic via knots (end)

\input{qm2pi.conclusion}

% section conclusion (end)

%\input{qm2pi.dtcodes} 

% section wiring algorithm (end)

\input{qm2pi.ack} 

% section acknowledgments (end)

\newpage


\bibliographystyle{plain}   
\bibliography{../../biblios/main.bib}

\input{qm2pi.rhodetails}

\end{document}

 

% section concurrent_process_calculi (end)

%\documentclass[12pt]{llncs}
%\documentclass{jktr}

\usepackage[pdftex]{hyperref}                   
\usepackage {listings}
\usepackage {mathpartir}
\usepackage{bcprules}
%\usepackage{listings}
                       
\usepackage{graphicx} 
%\usepackage[margins=2.5cm,nohead,nofoot]{geometry}
%\usepackage{geometry}
\usepackage{amsfonts}
\usepackage{amstext}
\usepackage{latexsym}
\usepackage{amssymb}
\usepackage{color}


%\include{myPreamble}
\include{qm2pi.local} 

%\ifpdf
%\usepackage[pdftex]{graphicx}
%\else
%\usepackage{graphicx}
%\fi

 % \ifpdf
%  \usepackage{pdfsync}
%  \if


%\title{Brief Article}
%\author{David F. Snyder}
%\author{L.G. Meredith}

%\address{Dept. of Math., Texas State University--San Marcos, San Marcos, TX 78666}
       
\pagestyle{empty}


\begin{document}

\lstset{language=[Objective]Caml,frame=shadowbox}

\input{qm2pi.front}

% section front matter (end)

\input{qm2pi.intro} 
 
% section introduction (end)

% \input{qm2pi.knotations} 

% section notation (end)

\input{qm2pi.process.calculi} 

% section concurrent_process_calculi_and_spatial_logics_ (end)
    
%\input{qm2pi.knots2pi} 

%\input{qm2pi.trefoil} 

%\input{qm2pi.mainthm} 

% subsection basic_interpretation (end)

%\input{qm2pi.rho.presentation} 
\subsection{The syntax and semantics of the notation system}\label{sub:the_syntax_and_semantics_of_the_notation_system} % (fold)

We now summarize a technical presentation of the calculus that
embodies our theory of dynamics. The typical presentation of such a
calculus follows the style of giving generators and relations on
them. The grammar, below, describing term constructors, freely
generates the set of processes, $\Proc$. This set is then quotiented
by a relation known as structural congruence and it is over this set
that the notion of dynamics is expressed. This presentation is
essentially that of \cite{MeredithR05} with the addition of
polyadicity and summation. For readability we have relegated some of
the technical subtleties to an appendix.

\subsubsection{Process grammar}\label{subsub:process_grammar}

\begin{mathpar}
  \inferrule* [lab=synchronization] {} {{M} \bc \pzero \;|\; x?F \;|\; x!C }
  \and
  \inferrule* [lab=abstraction] {} {{F} \bc (x)P}
  \and
  \inferrule* [lab=concretion] {} {{C} \bc \langle Q \rangle}
  \and
  \inferrule* [lab=process] {} {{P,Q} \bc M \;| \;P|Q \;|\; @{x}}
  \and
  \inferrule* [lab=name] {} {{x} \bc \quotep{P}}
\end{mathpar} 

Note that $\vec{x}$ (resp. $\vec{P}$) denotes a vector of names
(resp. processes) of length $|\vec{x}|$ (resp. $|\vec{P}|$). We adopt
the following useful abbreviations.

\begin{mathpar}
   x?(\vec{y}).P := x.(\vec{y})P \and  x\clift{\vec{P}} := x.\clift{\vec{P}}
   \and x!(y) := \lift{x}{\dropn{y}}
   \and \Pi_{i=0}^{n-1}P_i := P_0 | \ldots | P_{n-1}
\end{mathpar}

\subsubsection{Structural congruence}

\paragraph{Free and bound names and alpha-equivalence.} At the
core of structural equivalence is alpha-equivalence which identifies
process that are the same up to a change of variable. Formally, we
recognize the distinction between free and bound names. The free names
of a process, $\freenames{P}$, may be calculated recursively as
follows:

\begin{mathpar}
\freenames{\pzero} := \emptyset
  \and \\
  \freenames{x?(y).P} := \{ x \} \cup (\freenames{P} \setminus \{ y \})
  \and 
  \freenames{x!\langle P \rangle} := \{ x \} \cup \{ P \} 
  \and \\
  \freenames{P|Q} := \freenames{P} \cup \freenames{Q}
  \and \\
  \freenames{@{x}} := \{ x \}
\end{mathpar}

$\pi$
$\quotep{\pi}$

$\freenames{-} : \pi \to \mathcal{P}(\quotep{\pi})$

\begin{eqnarray*}
  \freenames{\pzero} & := & \emptyset \\
  \freenames{x?(y).P} & := & \{ x \} \cup (\freenames{P} \setminus \{ y \}) \\
  \freenames{x!\langle P \rangle} & := & \{ x \} \cup \{ P \} \\
  \freenames{P|Q} & := & \freenames{P} \cup \freenames{Q} \\
  \freenames{\dropn{x}} & := & \{ x \}
\end{eqnarray*}

The bound names of a process, $\boundnames{P}$, are those names occurring in $P$
that are not free. For example, in $x?(y).0$, the name $x$ is free, while $y$ is bound.

\begin{mathpar}
  \inferrule* [lab=monoidal-laws] {} { P|Q \equiv Q|P \and P|0 \equiv P \and P|(Q|R) \equiv (P|Q)|R }
\end{mathpar}

\begin{mathpar}
  \inferrule* [lab=alpha-equivalence] {} { (x)P \equiv (y)P\{y/x\} \and y \not\in \freenames{P} }
\end{mathpar}

\begin{definition}
Then two processes, $P,Q$, are alpha-equivalent if $P = Q\{\vec{y}/\vec{x}\}$ for
some $\vec{x} \in \boundnames{Q},\vec{y} \in \boundnames{P}$, where $Q\{\vec{y}/\vec{x}\}$
denotes the capture-avoiding substitution of $\vec{y}$ for $\vec{x}$ in $Q$.
\end{definition}

\begin{definition}
  The {\em structural congruence} \cite{SangiorgiWalker} , $\equiv$,
  between processes is the least congruence containing
  alpha-equivalence, satisfying the abelian monoid laws
  (associativity, commutativity and $\pzero$ as identity) for parallel
  composition $|$ and for summation $+$.
\end{definition}

\subsection{Name equivalence}

We take name equivalence, written $\nameeq$, to be the smallest
equivalence relation generated by the following rules.

\begin{mathpar}
\inferrule*[lab=Quote-drop]
{ }
{ \quotep{@{x}} \nameeq x }

\inferrule*[lab=Struct-equiv]
{ P \scong Q }
{ \quotep{P} \nameeq \quotep{Q} }
\end{mathpar}

The astute reader will have noticed that the mutual recursion of names
and processes imposes a mutual recursion on alpha-equivalence and
structural equivalence via name-equivalence. Fortunately, all of this
works out pleasantly and we may calculate in the natural way, free of
concern. The reader interested in the details is referred to the
appendix \ref{appendix:rho_details}.

\subsection{Substitution}

We use $\Proc$ for the set of processes, $\QProc$ for the set of
names, and $\id{\{}\vec{y} / \vec{x} \id{\}}$ to denote partial maps,
$s : \QProc \rightarrow \QProc$. A map, $s$ lifts, uniquely, to a map
on process terms, $\widehat{s} : \Proc \rightarrow \Proc$ by the
following equations.

\begin{mathpar}
  (0) \psubstp{Q}{P} := 0 \\
  (R \juxtap S) \psubstp{Q}{P}
  :=    
  (R)\psubstp{Q}{P} \juxtap (S) \psubstp{Q}{P} \\
  (x?(y).R) \psubstp{Q}{P}    
  :=    
  (x)\substp{Q}{P} (z)\concat( (R \psubstn{z}{y}) \psubstp{Q}{P} ) \\
  (\lift{x}{R}) \psubstp{Q}{P}  
  :=
  \lift{(x)\substp{Q}{P}}{ R \psubstp{Q}{P} } \\
%   (\dropn{x})  \psubstp{Q}{P}       
%   := 
%   \left\{ 
%     \begin{array}{ccc} 
%       \dropn{\quotep{Q}} & & x \nameeq \quotep{P} \\
%       \dropn{x} & & otherwise \\
%     \end{array}
%   \right. 
  (\dropn{x})  \psubstp{Q}{P}       
  := 
  \left\{ 
    \begin{array}{ccc} 
      Q & & x \nameeq \quotep{P} \\
      \dropn{x} & & otherwise \\
    \end{array}
  \right.
\end{mathpar}
 

where

\begin{eqnarray}
  (x)\id{\{} \lpquote Q \rpquote / \lpquote P \rpquote \id{\}}            = 
  \left\{ 
    \begin{array}{ccc}
      \lpquote Q \rpquote & & x \nameeq \lpquote P \rpquote \\
      x & & otherwise \\
    \end{array}
  \right. \nonumber
\end{eqnarray}

and $z$ is chosen distinct from $\quotep{P}$, $\quotep{Q}$, the free
names in $Q$, and all the names in $R$. Our $\alpha$-equivalence will
be built in the standard way from this substitution.

\begin{remark}\label{rem:no_self_referential_names}
  One consequence of these definitions is that $\forall P. \quotep{P}
  \not\in \freenames{P}$.
\end{remark}

\subsection{ Dynamic quote: an example }

Anticipating something of what's to come, consider applying the
substitution, $\widehat{\id{\{}u / z \id{\}}}$, to the following pair
of processes, $\lift{w}{y!(z)}$ and $w[ \lpquote y!(z) \rpquote ]$.

\begin{eqnarray}
	\lift{w}{y!(z)}\widehat{\id{\{}u / z \id{\}}}
		& = &
		\lift{w}{y!(u)} \nonumber\\
	w[ \lpquote y!(z) \rpquote ] \widehat{ \id{\{}u / z \id{\}} }
		& = &
		w[ \lpquote y!(z) \rpquote ] \nonumber
\end{eqnarray}

Because the body of the process between quotes is impervious to
substitution, we get radically different answers. In fact, by
examining the first process in an input context,
e.g. $x?(z).\lift{w}{y!(z)}$, we see that the process under the lift
operator may be shaped by prefixed inputs binding a name inside it. In
this sense, the lift operator will be seen as a way to dynamically
construct processes before reifying them as names.

Finally equipped with these standard features we can present the
dynamics of the calculus.

\subsubsection{Operational semantics} 

Finally, we introduce the computational dynamics. What marks these
algebras as distinct from other more traditionally studied algebraic
structures, e.g. vector spaces or polynomial rings, is the manner in
which dynamics is captured. In traditional structures, dynamics is typically
expressed through morphisms between such structures, as in linear maps
between vector spaces or morphisms between rings. In algebras
associated with the semantics of computation, the dynamics is
expressed as part of the algebraic structure itself, through a
reduction reduction relation typically denoted by $\red$. Below, we
give a recursive presentation of this relation for the calculus used
in the encoding.

$\red \subseteq \pi \times \pi$
$\red : \pi \to \mathcal{P}(\pi)$

\begin{mathpar}
  \inferrule* [lab=Comm] { \textsf{match}( x_{src}, x_{trgt} ) } { x_{trgt}?(y)P \; | \; x_{src}!\langle {Q} \rangle \red P\{\quotep{Q}/y}\} }
  \and \\
  \inferrule* [lab=Par] {{P} \red {P}'} {{{P} | {Q}} \red {{P}' | {Q}}}
  \and
  \inferrule* [lab=Equiv]{{{P} \scong {P}'} \andalso {{P}' \red {Q}'} \andalso {{Q}' \scong {Q}}}{{P} \red {Q}}
\end{mathpar}

\begin{eqnarray*}
  match_{\equiv} (\quotep{P},\quotep{Q}) & := & P \equiv Q \\
  match_{\dagger}(\quotep{P},\quotep{Q}) & := & \forall R. P|Q \red^{*} R => R \red^{*} 0 \\
  match_{K}(\quotep{P},\quotep{Q}) & := & K \mbox{ for some context } K
\end{eqnarray*}

$u?(x)P | u!\langle Q \rangle \red P\{\quotep{Q}/x\}$

%We write $\wred$ for $\red^*$, and $P\red$ if $\exists Q $ such that $ P \red Q$.
We write $P\red$ if $\exists Q $ such that $ P \red Q$ and $P\not\red$, otherwise.

\section{Replication}

As mentioned before, it is known that replication (and hence
recursion) can be implemented in a higher-order process algebra
\cite{SangiorgiWalker}. As our first example of calculation with the
machinery thus far presented we give the construction explicitly in
the {\rhoc}.

\begin{eqnarray}
	D_{x} & := & \prefix{x}{y}{(\binpar{\outputp{x}{y}}{@{y}})} \nonumber\\
	\bangp_{x}{P} & := & \binpar{{x}!\langle{\binpar{D_{x}}{P}}\rangle}{D_{x}} \nonumber
\end{eqnarray}

\begin{eqnarray}
	\bangp_{x}{P} & & \nonumber\\
	=
	& {x}!\langle{(\prefix{x}{y}{(\outputp{x}{y} | @{y})) | P}}\rangle 
	      | \prefix{x}{y}{(\outputp{x}{y} | @{y})} & \nonumber\\
	\red
	& (\outputp{x}{y} | @{y})\substn{\quotep{(\prefix{x}{y}{(@{y} | \outputp{x}{y})) | P}}}{y} & \nonumber\\
	=
	& \outputp{x}{\quotep{(\prefix{x}{y}{(\outputp{x}{y} | @{y})) | P}}}
	  | {(\prefix{x}{y}{(\outputp{x}{y} | @{y})) | P}} & \nonumber\\
	\red
	& \ldots & \nonumber\\
	\red^*
	& P | P | \ldots & \nonumber
\end{eqnarray}

Of course, this encoding, as an implementation, runs away, unfolding
$\bangp{P}$ eagerly. A lazier and more implementable replication
operator, restricted to input-guarded processes, may be obtained as follows.

\begin{eqnarray}
\bangp{\prefix{u}{v}{P}} 
	:= 
	\binpar{\lift{x}{\prefix{u}{v}{(\binpar{D(x)}{P})}}}{D(x)} \nonumber
\end{eqnarray}

\begin{remark}
  Note that the lazier definition still does not deal with summation
  or mixed summation (i.e. sums over input and output). The reader is
  invited to construct definitions of replication that deal with these
  features. 

  Further, the definitions are parameterized in a name, $x$. Can you,
  gentle reader, make a definition that eliminates this parameter and
  guarantees no accidental interaction between the replication
  machinery and the process being replicated -- i.e. no accidental
  sharing of names used by the process to get its work done and the
  name(s) used by the replication to effect copying. This latter
  revision of the definition of replication is crucial to obtaining
  the expected identity $!!P \sim !P$.
\end{remark}

\begin{remark}\label{rem:paradoxical_combinator}
  The reader familiar with the lambda calculus will have noticed the
  similarity between $D$ and the paradoxical combinator.

  [Ed. note: the existence of this seems to suggest we have to be more
  restrictive on the set of processes and names we admit if we are to
  support no-cloning.]
\end{remark}

\subsubsection{Bisimulation}

The computational dynamics gives rise to another kind of equivalence,
the equivalence of computational behavior. As previously mentioned
this is typically captured \emph{via} some form of bisimulation.

% The notion we use in this paper is weak barbed bisimulation
% \cite{milner91polyadicpi}.

The notion we use in this paper is derived from weak barbed
bisimulation \cite{milner91polyadicpi}. 

\begin{definition}
An \emph{observation relation}, $\downarrow_{\mathcal N}$, over a set
of names, $\mathcal N$, is the smallest relation satisfying the rules
below.

\infrule[Out-barb]{y \in {\mathcal N}, \; x \nameeq y}
		  {\outputp{x}{v} \downarrow_{\mathcal N} x}
\infrule[Par-barb]{\mbox{$P\downarrow_{\mathcal N} x$ or $Q\downarrow_{\mathcal N} x$}}
		  {\binpar{P}{Q} \downarrow_{\mathcal N} x}

We write $P \Downarrow_{\mathcal N} x$ if there is $Q$ such that 
$P \wred Q$ and $Q \downarrow_{\mathcal N} x$.
\end{definition}

\begin{definition}
%\label{def.bbisim}
An  ${\mathcal N}$-\emph{barbed bisimulation} over a set of names, ${\mathcal N}$, is a symmetric binary relation 
${\mathcal S}_{\mathcal N}$ between agents such that $P\rel{S}_{\mathcal N}Q$ implies:
\begin{enumerate}
\item If $P \red P'$ then $Q \wred Q'$ and $P'\rel{S}_{\mathcal N} Q'$.
\item If $P\downarrow_{\mathcal N} x$, then $Q\Downarrow_{\mathcal N} x$.
\end{enumerate}
$P$ is ${\mathcal N}$-barbed bisimilar to $Q$, written
$P \wbbisim_{\mathcal N} Q$, if $P \rel{S}_{\mathcal N} Q$ for some ${\mathcal N}$-barbed bisimulation ${\mathcal S}_{\mathcal N}$.
\end{definition}

$\mathcal{R} \subseteq \pi \times \pi$

$P \mathcal{R} Q => \forall P'. P \red P' \Rightarrow \exists Q'. Q \red Q', P' \mathcal{R} Q'$

$P \vdash x \Rightarrow Q \vdash x$

\begin{mathpar}
  \inferrule*[lab=Out-barb]{x \nameeq y}{{y}!\langle{Q}\rangle \vdash x}
  \and
  \inferrule*[lab=Par-barb]{\mbox{$P\vdash x$ or $Q\vdash x$}}{\binpar{P}{Q} \vdash x}
\end{mathpar}

\subsubsection{Contexts}

One of the principle advantages of computational calculi like the
$\pi$-calculus is a well-defined notion of context,
contextual-equivalence and a correlation between
contextual-equivalence and notions of bisimulation. The notion of
context allows the decomposition of a process into (sub-)process and
its syntactic environment, its context. Thus, a context may be
thought of as a process with a ``hole'' (written $\Box$) in it. The
application of a context $M$ to a process $P$, written $M[P]$, is
tantamount to filling the hole in $M$ with $P$. In this paper we do
not need the full weight of this theory, but do make use of the notion
of context in the proof the main theorem. 

\begin{mathpar}
  \inferrule* [lab=summation] {} {{M_{M},M_{N}} \bc \Box \;|\; x.M_{A} \;|\; M_{M}+M_{N}}
  \and
  \inferrule* [lab=agent] {} {{M_{A}} \bc (\vec{x})M_{P} \;| \; \clift{P_0,\ldots,M_{P},\ldots,P_N}}
  \and \\
  \inferrule* [lab=process] {} {{M_{P}} \bc M_{N} \;| \;P|M_{P} }
\end{mathpar} 

\begin{mathpar}
  \inferrule* [lab=sychronization] {} {M_{N} \bc \Box \;|\; x?M_{F} \;|\; x!M_{C}}
  \and
  \inferrule* [lab=abstraction] {} {{M_{F}} \bc (x)M_{P} }
  \and
  \inferrule* [lab=concretion] {} {{M_{C}} \bc \langle M_{P} \rangle }
  \and \\
  \inferrule* [lab=process] {} {{M_{P}} \bc M_{N} \;| \;P|M_{P} }
\end{mathpar}

\begin{definition}[contextual application] Given a context $M$, and
  process $P$, we define the \emph{contextual application}, $M[P] :=
  M\{P/\Box\}$. That is, the contextual application of M to P is the
  substitution of $P$ for $\Box$ in $M$.
\end{definition}

$\meaningof{-} : L \to \mathcal{P}(\pi)$

\begin{mathpar}
  \inferrule* [lab=collection] {} {\meaningof{true} = \pi, \and \meaningof{~E} = \pi \setminus \meaningof{E}, \and \meaningof{E_{1} \& E_{2}} = \meaningof{E_{1}} \cap \meaningof{E_{2}}}
\end{mathpar}

\begin{mathpar}
  \inferrule* [lab=structure] {} {\meaningof{0} = \{ P \in \pi | P \equiv 0 \}, \and \\ \meaningof{E_1 | E_2} = \{ P \in \pi | P \equiv P_{1} | P_{2}, P_{1} \in \meaningof{E_{1}}, P_{2} \in \meaningof{E_2}\} }
\end{mathpar}

\begin{mathpar}
 \inferrule* [lab=behavior] {} {\meaningof{\langle a?b \rangle E} = \{ P \in \pi | P \equiv Q | u?(y)P', \\ \and \\\\ \and \\ \;\;\; u \in \meaningof{a}, \forall z.P'\{z/y\} \in \meaningof{E\{z/b\}}\}, \and \\ \meaningof{a!E} = \{ P \in \pi | P \equiv Q | x!\langle P' \rangle, x \in \meaningof{a} P' \in \meaningof{E}\} }
\end{mathpar}

\begin{mathpar}
 \inferrule* [lab=nominal] {} {\meaningof{\quotep{E}} = \{ \quotep{P} \in \quotep{\pi} | P \in \meaningof{E} \}, \and \meaningof{\quotep{P}} = \{ \quotep{Q} \in \quotep{\pi} | P \equiv Q \} \and \\ \meaningof{@\quotep{E}} = \{ P \in \pi | P \equiv @x, x \in \meaningof{E} \}}
\end{mathpar}

\begin{eqnarray*}
  \\
  \meaningof{-} : TS \to ST
\end{eqnarray*}

\begin{eqnarray*}
  \\
  L : TS \to ST
\end{eqnarray*}

\begin{eqnarray*}
  \\
  P \models E \iff P \in \meaningof{E}
\end{eqnarray*}

\begin{eqnarray*}
  P \approx_{L} Q \iff \forall E \in L. P \models E \iff Q \models E
\end{eqnarray*}

\begin{eqnarray*}
  P \approx_{K} Q
\end{eqnarray*}

\begin{eqnarray*}
  P \approx Q
\end{eqnarray*}

$\approx_{K} = \approx = \approx_{L}$

\subsubsection{Contextual duality}

Note that contexts extend the quotation operation to a family of
operations from processes to names. Given a context, $M$, we can
define a \emph{nominal context}, $\quotep{M}$ by $\quotep{M}[P] :=
\quotep{M[P]}$. To foreshadow what is to come we observe that these
operations enjoy a duality with processes very much like the duality
between vectors and maps from vectors to scalars.

Further, because the calculus is essentially higher-order, we have a
correspondence between contexts and processes. More specifically,
given a name $x$ and a context $M$ we can construct $M^{*}_{x}$ such
that 

\begin{mathpar}
  M^{*}_{x} | \lift{x}{P} \red M[P]
\end{mathpar}

namely,

\begin{mathpar}
  M^{*}_{x} := x?(u).M[\dropn{u}]
\end{mathpar}

The dependence of $M^{*}_{x}$ on a name makes it an abstraction, 

\begin{mathpar}
  M^{*} := (x)x?(u).M[\dropn{u}]
\end{mathpar}

\subsection{Additional notation}

It will sometimes be convenient to denote the process a name
quotes. We already have the notation $x = \quotep{P}$, but it will be
convenient to introduce an alternate notation, $\procn{x}$, when we
want to emphasize the connection to the use of the name. Note that, by
virtue of name equivalence, $\quotep{\procn{x}} \nameeq x$; so, the
notation is consistent with previous definitions.

Further, because names have structure it is possible to effect
substitutions on the basis of that structure. This means we need to
upgrade our notation for substitutions, which we accomplish by
adapting comprehension notation. Thus,

\begin{mathpar}
  P\{ y / x : x \in S \}
\end{mathpar}

is interpreted to mean the process derived from P by replacing (in a
capture-avoiding manner) each occurrence of $x$ in $S$ by $y$. For example,

\begin{mathpar}
  P\{ \quotep{\procn{x}|\procn{x}} / x : x \in \freenames{P} \}
\end{mathpar}

will replace each (occurrence) of a free name $x$ in $P$ by
$\quotep{\procn{x}|\procn{x}}$.

Also, we will avail ourselves of the notation $x^{L}$ and $x^{R}$ to
denote injections of a name into disjoint copies of the name
space. There are numerous ways to accomplish this. One example can be
found in \cite{MeredithR05}. This notation overloads to vectors of
names: $\vec{x}^{\pi} := (x_{i}^{\pi} \; : \; 0 \leq i < |\vec{x}| )$ where $\pi \in \{L,R\}$.

We also use $P^{\Box} := P|\Box$.

In \cite{MeredithR05} an interpretation of the new operator is
given. It turns out that there are several possible interpretations
all enjoying the requisite algebraic properties of the operator (see
\cite{milner91polyadicpi}). We will therefore make liberal use of
$(\nu\; \vec{x})P$.

% subsection the_syntax_and_semantics_of_the_notation_system (end)   

\input{qm2pi.qmops} 

\input{qm2pi.sterngerlach} 

\input{qm2pi.metric} 

% section concurrent_process_calculi (end)

%\input{qm2pi.proofsketch}

% section proof sketch (end)

%\input{qm2pi.slviaknots} 

% section spatial logic via knots (end)

\input{qm2pi.conclusion}

% section conclusion (end)

%\input{qm2pi.dtcodes} 

% section wiring algorithm (end)

\input{qm2pi.ack} 

% section acknowledgments (end)

\newpage


\bibliographystyle{plain}   
\bibliography{../../biblios/main.bib}

\input{qm2pi.rhodetails}

\end{document}



% section proof sketch (end)

%\section{Unlikely characters: spatial logic for
  knots}\label{sub:characteristic_formulae} % (fold)

Associated to the mobile process calculi are a family of logics known
as the Hennessy-Milner logics. These logics typically enjoy a
semantics interpreting formulae as sets of processes that when
factored through the encoding outlined above allows an identification
of classes of knots with logical formulae. In the context of this
encoding the sub-family known as the spatial logics \cite{CairesC03}
\cite{CairesC04} \cite{Caires04} are of particular interest providing
several important features for expressing and reasoning about
properties (i.e. classes) of knots. We hint here at how this may be done.

%\begin{description}
%\item [structural connectives] 
\subsubsection{Structural connectives} The spatial logics enjoy
structural connectives corresponding, at the logical level, to the
parallel composition ($P | Q$) and new name ($(\nu \; x)P$)
connectives for processes. As illustrated in the examples below, these
connectives are extremely expressive given the shape of our encoding.
%\item [decideable satisfaction]

\subsubsection{Decideable satisfaction}
In \cite{Caires04} the satisfaction relation is shown to be decideable
for a rich class of processes. It further turns out that the image of
the our encoding is a proper subset of that class. This result
provides the basis for an algorithm by which to search for knots
enjoying a given property.
%\item [characteristic formulae]

\subsubsection{Characteristic formulae}
In the same paper \cite{Caires04} , Caires presents a means of calculating
characteristic formulae, selecting equivalence classes of processes
up to a pre--specified depth limit on the support set of names. Composed with our
encoding, this characteristic formula can be used to select
characteristic formulae for knots.
%\end{description}

\subsubsection{Spatial logic formulae}

The grammar below (segmented for comprehension) summarizes the syntax
of spatial logic formulae. We employ illustrative examples in the
sequel to provide an intuitive understanding of their meaning
referring the reader to \cite{Caires04} for a more detailed explication
of the semantics.

\begin{mathpar}
  \inferrule* [lab=boolean] {} {{A,B} \bc T \;|\; \neg A \;|\; A \wedge B \;|\; \eta = \eta'}
  \and
  \inferrule* [lab=spatial] {} {|\; \pzero \;|\; A | B \;|\; x \text{\textregistered} A \;|\; \forall x . A \;|\;  H x . A}
  \and
  \inferrule* [lab=behavioral] {} {|\; \alpha . A}
  \and 
  \inferrule* [lab=recursion] {} {|\; X(\vec{u}) \;|\; \mu X(\vec{u}) . A}
  \and
  \inferrule* [lab=action] {} {\alpha \bc \langle x?(\vec{y}) \rangle \;|\; \langle x!(\vec{y}) \rangle \;|\; \langle \tau \rangle}
  \and 
  \inferrule* [lab=name] {} {\eta \bc x \;|\; \tau}
\end{mathpar} 

% subsection characteristic_formulae (end)   	 

\subsection{Example formulae}\label{sub:example_formulae_} % (fold)

\subsubsection{Crossing as formula.}
% 
% \begin{align*}
%   \frac{d}{dx} \sin x &= \cos x 
%   & \frac{d}{dx} e^x &= e^x \\
%   \frac{d}{dx} \cos x &= - \sin x 
%   & \frac{d}{dx} \log x &= \frac{1}{x} \\
% \end{align*} 

\begin{align*}
 \mu C(x_{0},x_{1},y_{0},y_{1},u).&(\langle x_{0}?(z) \rangle(\langle u! \rangle\langle y_{1}!z \rangle C(x_{0},x_{1},y_{0},y_{1},u)) & \\
  & \wedge \langle y_{1}?(z) \rangle (\langle u! \rangle \langle x_{0}!z \rangle C(x_{0},x_{1},y_{0},y_{1},u)) & \\
  & \wedge \langle x_{1}?(z) \rangle (\langle u? \rangle \langle y_{0}!z \rangle C(x_{0},x_{1},y_{0},y_{1},u)) & \\
  & \wedge \langle y_{0}?(z) \rangle (\langle u? \rangle \langle x_{1}!z \rangle C(x_{0},x_{1},y_{0},y_{1},u))) &
\end{align*}

The lexicographical similarity between the shape of this formulae and
the shape of definition of the process representing a crossing reveals
the intuitive meaning of this formulae. It describes the capabilities
of a process that has the right to represent a crossing. For example
it picks out processes that may perform an input on the port $x_0$ in
its initial menu of capabilities. What differentiates the formula
from the process, however, is that the crossing process is the
smallest candidate to satisfy the formula. Infinitely many other
processes -- with internal behavior hidden behind this interface, so
to speak -- also satisfy this formula. Even this simple formula,
then, can be seen to open a new view onto knots, providing a
computational interpretation of \emph{virtual} knots.

Note that this formula is derived by hand. A similar formula can be
derived by employing Caires' calculation of characteristic formula
\cite{Caires04} to the process representing a crossing. In light of
this discussion, we let
$\meaningof{C}_{\phi}(x0,x1,y0,y1,u)$ denote a formula specifying the
dynamics we wish to capture of a crossing. To guarantee we preserve
the shape of the interface and minimal semantics we demand that
$\meaningof{C}_{\phi}(x0,x1,y0,y1,u) \Rightarrow
\textbf{C}(x0,x1,y0,y1,u)$ where $\textbf{C}(x0,x1,y0,y1,u)$ denotes
the formula above.
                            
\subsubsection{Crossing number constraints.}
The moral content of the context lemma (Lemma \ref{context}) is that the notion of
``locality'' in the Reidemeister moves is effectively captured by the
parallel composition operator of the process calculus. This intuition
extends through the logic. Given a formula,
$\meaningof{C}_{\phi}(x0,x1,y0,y1,u)$, we can use the structural
connectives to specify constraints on crossing numbers, such as at
least $n$ crossings, or exactly $n$ crossings.
\begin{mathpar}
  \inferrule* [lab=at-least-n] {} { K^{\geq n}_{\phi}(\vec{xs},\vec{ys}) := \Pi_{i=0}^{n-1} Hu . \meaningof{C}_{\phi}(xs_i,ys_i,u) | T }
  \and 
  \inferrule* [lab=exactly-n] {} { K^{= n}_{\phi}(\vec{xs},\vec{ys}) := \Pi_{i=0}^{n-1} Hu . \meaningof{C}_{\phi}(xs_i,ys_i,u) | \neg (\forall x_0,y_0,x_1,y_1,u . \meaningof{C}_{\phi}(x_0,y_0,x_1,y_1,u) | T) }
\end{mathpar}

To round out this section, recall that the encoding of an $n$-crossing
knot decomposes into a parallel composition of $n$ \emph{copies} of a
crossing process together with a wiring harness. To specify different
knot classes with the same crossing number amounts to specifying
logical constraints on the wiring harness. In the interest of space,
we defer examples to a forthcoming paper. Suffice it to say that both
the conditions ``alternating knot'' and ``contains the tangle
corresponding to 5/3'' are expressible. For example, it is possible to
calculate the characteristic formula of a process corresponding to the
tangle 5/3 and conjoin it into the classifying formula via the
composition connective of the logic.

Finally, we wish to observe that it is entirely within reason to
contemplate a more domain-specific version of spatial logic tailored
to the shape of processes in the image of the encoding. Such a
domain-specific logic would have a better claim to the title formal
language of knot properties.

% subsection example_formulae_ (end)

% section knots_as_processes (end) 

% section spatial logic via knots (end)

\section{Conclusions and future work}

\paragraph{Testing physical space}
You, gentle reader, may wonder why of all the theorems to be proved
given this set up we pick the one above. In some sense it's hardly
central to quantum mechanics. We see it as central in the sense that
it firmly establishes a notion of physical space arising from a notion
of the equivalence of behavior. Relating bisimulation to a metric is a
big step forward, but one is faced with interpreting the relationship
of that metric space to something more physical. Quantum mechanical
notions of ``physical'' space are still far from intuitive, but by
relating this idea of distance as testing to calculations that predict
physical circumstances we are making a not insignificant step forward
toward an understanding of the physical space we inhabit as
essentially dynamic.

\paragraph{Effectivity and simulation}
One of the observations we have yet to make is that the entire program
spelled out here is effective. We have built various interpreters for
the reflective calculus at work in this interpretation. In principle,
then, we can simulate quantum mechanics on a computer. The place where
the simulation may lose fidelity is the infinitely branching summation
for the annihilator.

In this connection i also want to point out that the evaluation style
calculation of the inner product puts the non-determinism of the
summation right at the heart of measurement. This suggests that
Milner's original reduction-based formulation of the dynamics of his
calculi in terms of sums was not just notationally suggestive of a
notion of measure-and-continue but captured some significant part of
the physics.

\paragraph{Quantum continuations}
In light of this last observation i want to point out that the
predominant account of quantum mechanics is missing a key aspect of a
truly compositional story of the physical situation. In a real lab,
when a measurement is made the observation can be made to feed into
another device that then makes another measurement conditioned on the
results of the first. This means that after the superposition was
collapsed the entire experimental set up remained in
superposition. While QM offers a means of writing this down it doesn't
quite line up well with the well-trodden formulation of computation
and continuation that we see so succinctly expressed in Milner's
calculi. This suggests that there might be advantages to this account
of dynamics waiting to be explored.

\paragraph{Quantum logic}
In this connection, we also note that by virtue of having the
Hennessy-Milner construction, we can pull the construction through the
interpretation of QM. This gives us a natural candidate for a quantum
logic that enjoys an extremely tight connection with it's domain of
interpretation, making the construction much less ad hoc (rather it is
the image of functor!).

\paragraph{Quantum probabiity}
i have questions about the basis of the interpretation of inner
product as probability amplitude. In particular, using which
axiomatization of probability theory does the notion of probability
amplitude earn the right to be so dubbed? In other words, where is the
proof that the operation for calculating a probability amplitude (and
then squaring) satisfies the axioms of what it means to calculate a
probability? Even if such a proof exists (i have yet to find it in the
literature), i wonder if it might not be possible to turn things on
their heads. Can we view the calculation of the probability amplitude
as an axiomatization of probability? If so, then the definition we
give for calculating probability amplitude may provide the basis for
an \emph{effective} theory of probability.

\paragraph{Quantum vs ``biological'' information}
Finally, i want to conclude with a more philosophical observation. At
a recent workshop in which QM was a predominant topic i noticed
something about quantum information. The speaker was giving a riveting
discussion of axiomatic QM and showing how properties of ``no
cloning'' and ``no deleting'' emerged as consequences of the
axiomatization. Theorems of this form are necessary to give us a sense
of confidence that our axioms characterize the physical theory. What
struck me, though, was that if quantum information is neither erasable
nor replicable it is markedly different from \emph{life}. Two of the
things we know about life is that

\begin{itemize}
  \item it ends;
  \item to gain some measure of persistence, to transcend it's
    finitude it is imminently copyable.
\end{itemize}

Both of these qualities are summarized succinctly in the aphorism: all
flesh is grass. For me these two kinds of ``information'' -- call them
quantum and biological -- are end points on a spectrum of strategies
for persistence. At one end, we have those curious entities that enjoy
uniqueness and permanence; at the other, we have those who in the face
of a certain end and an uncertain present make a go of passing
something on. To me one of the more remarkable aspects of the latter
strategy is that in the presence of noise (and certain features of
copying) we get a kind of dynamism, a chance for improvement against a
given persistent condition.

% subsection other_calculi_other_bisimulations_and_geometry_as_behavior (end)




% section conclusion (end)

%\documentclass[12pt]{llncs}
%\documentclass{jktr}

\usepackage[pdftex]{hyperref}                   
\usepackage {listings}
\usepackage {mathpartir}
\usepackage{bcprules}
%\usepackage{listings}
                       
\usepackage{graphicx} 
%\usepackage[margins=2.5cm,nohead,nofoot]{geometry}
%\usepackage{geometry}
\usepackage{amsfonts}
\usepackage{amstext}
\usepackage{latexsym}
\usepackage{amssymb}
\usepackage{color}


%\include{myPreamble}
\include{qm2pi.local} 

%\ifpdf
%\usepackage[pdftex]{graphicx}
%\else
%\usepackage{graphicx}
%\fi

 % \ifpdf
%  \usepackage{pdfsync}
%  \if


%\title{Brief Article}
%\author{David F. Snyder}
%\author{L.G. Meredith}

%\address{Dept. of Math., Texas State University--San Marcos, San Marcos, TX 78666}
       
\pagestyle{empty}


\begin{document}

\lstset{language=[Objective]Caml,frame=shadowbox}

\input{qm2pi.front}

% section front matter (end)

\input{qm2pi.intro} 
 
% section introduction (end)

% \input{qm2pi.knotations} 

% section notation (end)

\input{qm2pi.process.calculi} 

% section concurrent_process_calculi_and_spatial_logics_ (end)
    
%\input{qm2pi.knots2pi} 

%\input{qm2pi.trefoil} 

%\input{qm2pi.mainthm} 

% subsection basic_interpretation (end)

%\input{qm2pi.rho.presentation} 
\subsection{The syntax and semantics of the notation system}\label{sub:the_syntax_and_semantics_of_the_notation_system} % (fold)

We now summarize a technical presentation of the calculus that
embodies our theory of dynamics. The typical presentation of such a
calculus follows the style of giving generators and relations on
them. The grammar, below, describing term constructors, freely
generates the set of processes, $\Proc$. This set is then quotiented
by a relation known as structural congruence and it is over this set
that the notion of dynamics is expressed. This presentation is
essentially that of \cite{MeredithR05} with the addition of
polyadicity and summation. For readability we have relegated some of
the technical subtleties to an appendix.

\subsubsection{Process grammar}\label{subsub:process_grammar}

\begin{mathpar}
  \inferrule* [lab=synchronization] {} {{M} \bc \pzero \;|\; x?F \;|\; x!C }
  \and
  \inferrule* [lab=abstraction] {} {{F} \bc (x)P}
  \and
  \inferrule* [lab=concretion] {} {{C} \bc \langle Q \rangle}
  \and
  \inferrule* [lab=process] {} {{P,Q} \bc M \;| \;P|Q \;|\; @{x}}
  \and
  \inferrule* [lab=name] {} {{x} \bc \quotep{P}}
\end{mathpar} 

Note that $\vec{x}$ (resp. $\vec{P}$) denotes a vector of names
(resp. processes) of length $|\vec{x}|$ (resp. $|\vec{P}|$). We adopt
the following useful abbreviations.

\begin{mathpar}
   x?(\vec{y}).P := x.(\vec{y})P \and  x\clift{\vec{P}} := x.\clift{\vec{P}}
   \and x!(y) := \lift{x}{\dropn{y}}
   \and \Pi_{i=0}^{n-1}P_i := P_0 | \ldots | P_{n-1}
\end{mathpar}

\subsubsection{Structural congruence}

\paragraph{Free and bound names and alpha-equivalence.} At the
core of structural equivalence is alpha-equivalence which identifies
process that are the same up to a change of variable. Formally, we
recognize the distinction between free and bound names. The free names
of a process, $\freenames{P}$, may be calculated recursively as
follows:

\begin{mathpar}
\freenames{\pzero} := \emptyset
  \and \\
  \freenames{x?(y).P} := \{ x \} \cup (\freenames{P} \setminus \{ y \})
  \and 
  \freenames{x!\langle P \rangle} := \{ x \} \cup \{ P \} 
  \and \\
  \freenames{P|Q} := \freenames{P} \cup \freenames{Q}
  \and \\
  \freenames{@{x}} := \{ x \}
\end{mathpar}

$\pi$
$\quotep{\pi}$

$\freenames{-} : \pi \to \mathcal{P}(\quotep{\pi})$

\begin{eqnarray*}
  \freenames{\pzero} & := & \emptyset \\
  \freenames{x?(y).P} & := & \{ x \} \cup (\freenames{P} \setminus \{ y \}) \\
  \freenames{x!\langle P \rangle} & := & \{ x \} \cup \{ P \} \\
  \freenames{P|Q} & := & \freenames{P} \cup \freenames{Q} \\
  \freenames{\dropn{x}} & := & \{ x \}
\end{eqnarray*}

The bound names of a process, $\boundnames{P}$, are those names occurring in $P$
that are not free. For example, in $x?(y).0$, the name $x$ is free, while $y$ is bound.

\begin{mathpar}
  \inferrule* [lab=monoidal-laws] {} { P|Q \equiv Q|P \and P|0 \equiv P \and P|(Q|R) \equiv (P|Q)|R }
\end{mathpar}

\begin{mathpar}
  \inferrule* [lab=alpha-equivalence] {} { (x)P \equiv (y)P\{y/x\} \and y \not\in \freenames{P} }
\end{mathpar}

\begin{definition}
Then two processes, $P,Q$, are alpha-equivalent if $P = Q\{\vec{y}/\vec{x}\}$ for
some $\vec{x} \in \boundnames{Q},\vec{y} \in \boundnames{P}$, where $Q\{\vec{y}/\vec{x}\}$
denotes the capture-avoiding substitution of $\vec{y}$ for $\vec{x}$ in $Q$.
\end{definition}

\begin{definition}
  The {\em structural congruence} \cite{SangiorgiWalker} , $\equiv$,
  between processes is the least congruence containing
  alpha-equivalence, satisfying the abelian monoid laws
  (associativity, commutativity and $\pzero$ as identity) for parallel
  composition $|$ and for summation $+$.
\end{definition}

\subsection{Name equivalence}

We take name equivalence, written $\nameeq$, to be the smallest
equivalence relation generated by the following rules.

\begin{mathpar}
\inferrule*[lab=Quote-drop]
{ }
{ \quotep{@{x}} \nameeq x }

\inferrule*[lab=Struct-equiv]
{ P \scong Q }
{ \quotep{P} \nameeq \quotep{Q} }
\end{mathpar}

The astute reader will have noticed that the mutual recursion of names
and processes imposes a mutual recursion on alpha-equivalence and
structural equivalence via name-equivalence. Fortunately, all of this
works out pleasantly and we may calculate in the natural way, free of
concern. The reader interested in the details is referred to the
appendix \ref{appendix:rho_details}.

\subsection{Substitution}

We use $\Proc$ for the set of processes, $\QProc$ for the set of
names, and $\id{\{}\vec{y} / \vec{x} \id{\}}$ to denote partial maps,
$s : \QProc \rightarrow \QProc$. A map, $s$ lifts, uniquely, to a map
on process terms, $\widehat{s} : \Proc \rightarrow \Proc$ by the
following equations.

\begin{mathpar}
  (0) \psubstp{Q}{P} := 0 \\
  (R \juxtap S) \psubstp{Q}{P}
  :=    
  (R)\psubstp{Q}{P} \juxtap (S) \psubstp{Q}{P} \\
  (x?(y).R) \psubstp{Q}{P}    
  :=    
  (x)\substp{Q}{P} (z)\concat( (R \psubstn{z}{y}) \psubstp{Q}{P} ) \\
  (\lift{x}{R}) \psubstp{Q}{P}  
  :=
  \lift{(x)\substp{Q}{P}}{ R \psubstp{Q}{P} } \\
%   (\dropn{x})  \psubstp{Q}{P}       
%   := 
%   \left\{ 
%     \begin{array}{ccc} 
%       \dropn{\quotep{Q}} & & x \nameeq \quotep{P} \\
%       \dropn{x} & & otherwise \\
%     \end{array}
%   \right. 
  (\dropn{x})  \psubstp{Q}{P}       
  := 
  \left\{ 
    \begin{array}{ccc} 
      Q & & x \nameeq \quotep{P} \\
      \dropn{x} & & otherwise \\
    \end{array}
  \right.
\end{mathpar}
 

where

\begin{eqnarray}
  (x)\id{\{} \lpquote Q \rpquote / \lpquote P \rpquote \id{\}}            = 
  \left\{ 
    \begin{array}{ccc}
      \lpquote Q \rpquote & & x \nameeq \lpquote P \rpquote \\
      x & & otherwise \\
    \end{array}
  \right. \nonumber
\end{eqnarray}

and $z$ is chosen distinct from $\quotep{P}$, $\quotep{Q}$, the free
names in $Q$, and all the names in $R$. Our $\alpha$-equivalence will
be built in the standard way from this substitution.

\begin{remark}\label{rem:no_self_referential_names}
  One consequence of these definitions is that $\forall P. \quotep{P}
  \not\in \freenames{P}$.
\end{remark}

\subsection{ Dynamic quote: an example }

Anticipating something of what's to come, consider applying the
substitution, $\widehat{\id{\{}u / z \id{\}}}$, to the following pair
of processes, $\lift{w}{y!(z)}$ and $w[ \lpquote y!(z) \rpquote ]$.

\begin{eqnarray}
	\lift{w}{y!(z)}\widehat{\id{\{}u / z \id{\}}}
		& = &
		\lift{w}{y!(u)} \nonumber\\
	w[ \lpquote y!(z) \rpquote ] \widehat{ \id{\{}u / z \id{\}} }
		& = &
		w[ \lpquote y!(z) \rpquote ] \nonumber
\end{eqnarray}

Because the body of the process between quotes is impervious to
substitution, we get radically different answers. In fact, by
examining the first process in an input context,
e.g. $x?(z).\lift{w}{y!(z)}$, we see that the process under the lift
operator may be shaped by prefixed inputs binding a name inside it. In
this sense, the lift operator will be seen as a way to dynamically
construct processes before reifying them as names.

Finally equipped with these standard features we can present the
dynamics of the calculus.

\subsubsection{Operational semantics} 

Finally, we introduce the computational dynamics. What marks these
algebras as distinct from other more traditionally studied algebraic
structures, e.g. vector spaces or polynomial rings, is the manner in
which dynamics is captured. In traditional structures, dynamics is typically
expressed through morphisms between such structures, as in linear maps
between vector spaces or morphisms between rings. In algebras
associated with the semantics of computation, the dynamics is
expressed as part of the algebraic structure itself, through a
reduction reduction relation typically denoted by $\red$. Below, we
give a recursive presentation of this relation for the calculus used
in the encoding.

$\red \subseteq \pi \times \pi$
$\red : \pi \to \mathcal{P}(\pi)$

\begin{mathpar}
  \inferrule* [lab=Comm] { \textsf{match}( x_{src}, x_{trgt} ) } { x_{trgt}?(y)P \; | \; x_{src}!\langle {Q} \rangle \red P\{\quotep{Q}/y}\} }
  \and \\
  \inferrule* [lab=Par] {{P} \red {P}'} {{{P} | {Q}} \red {{P}' | {Q}}}
  \and
  \inferrule* [lab=Equiv]{{{P} \scong {P}'} \andalso {{P}' \red {Q}'} \andalso {{Q}' \scong {Q}}}{{P} \red {Q}}
\end{mathpar}

\begin{eqnarray*}
  match_{\equiv} (\quotep{P},\quotep{Q}) & := & P \equiv Q \\
  match_{\dagger}(\quotep{P},\quotep{Q}) & := & \forall R. P|Q \red^{*} R => R \red^{*} 0 \\
  match_{K}(\quotep{P},\quotep{Q}) & := & K \mbox{ for some context } K
\end{eqnarray*}

$u?(x)P | u!\langle Q \rangle \red P\{\quotep{Q}/x\}$

%We write $\wred$ for $\red^*$, and $P\red$ if $\exists Q $ such that $ P \red Q$.
We write $P\red$ if $\exists Q $ such that $ P \red Q$ and $P\not\red$, otherwise.

\section{Replication}

As mentioned before, it is known that replication (and hence
recursion) can be implemented in a higher-order process algebra
\cite{SangiorgiWalker}. As our first example of calculation with the
machinery thus far presented we give the construction explicitly in
the {\rhoc}.

\begin{eqnarray}
	D_{x} & := & \prefix{x}{y}{(\binpar{\outputp{x}{y}}{@{y}})} \nonumber\\
	\bangp_{x}{P} & := & \binpar{{x}!\langle{\binpar{D_{x}}{P}}\rangle}{D_{x}} \nonumber
\end{eqnarray}

\begin{eqnarray}
	\bangp_{x}{P} & & \nonumber\\
	=
	& {x}!\langle{(\prefix{x}{y}{(\outputp{x}{y} | @{y})) | P}}\rangle 
	      | \prefix{x}{y}{(\outputp{x}{y} | @{y})} & \nonumber\\
	\red
	& (\outputp{x}{y} | @{y})\substn{\quotep{(\prefix{x}{y}{(@{y} | \outputp{x}{y})) | P}}}{y} & \nonumber\\
	=
	& \outputp{x}{\quotep{(\prefix{x}{y}{(\outputp{x}{y} | @{y})) | P}}}
	  | {(\prefix{x}{y}{(\outputp{x}{y} | @{y})) | P}} & \nonumber\\
	\red
	& \ldots & \nonumber\\
	\red^*
	& P | P | \ldots & \nonumber
\end{eqnarray}

Of course, this encoding, as an implementation, runs away, unfolding
$\bangp{P}$ eagerly. A lazier and more implementable replication
operator, restricted to input-guarded processes, may be obtained as follows.

\begin{eqnarray}
\bangp{\prefix{u}{v}{P}} 
	:= 
	\binpar{\lift{x}{\prefix{u}{v}{(\binpar{D(x)}{P})}}}{D(x)} \nonumber
\end{eqnarray}

\begin{remark}
  Note that the lazier definition still does not deal with summation
  or mixed summation (i.e. sums over input and output). The reader is
  invited to construct definitions of replication that deal with these
  features. 

  Further, the definitions are parameterized in a name, $x$. Can you,
  gentle reader, make a definition that eliminates this parameter and
  guarantees no accidental interaction between the replication
  machinery and the process being replicated -- i.e. no accidental
  sharing of names used by the process to get its work done and the
  name(s) used by the replication to effect copying. This latter
  revision of the definition of replication is crucial to obtaining
  the expected identity $!!P \sim !P$.
\end{remark}

\begin{remark}\label{rem:paradoxical_combinator}
  The reader familiar with the lambda calculus will have noticed the
  similarity between $D$ and the paradoxical combinator.

  [Ed. note: the existence of this seems to suggest we have to be more
  restrictive on the set of processes and names we admit if we are to
  support no-cloning.]
\end{remark}

\subsubsection{Bisimulation}

The computational dynamics gives rise to another kind of equivalence,
the equivalence of computational behavior. As previously mentioned
this is typically captured \emph{via} some form of bisimulation.

% The notion we use in this paper is weak barbed bisimulation
% \cite{milner91polyadicpi}.

The notion we use in this paper is derived from weak barbed
bisimulation \cite{milner91polyadicpi}. 

\begin{definition}
An \emph{observation relation}, $\downarrow_{\mathcal N}$, over a set
of names, $\mathcal N$, is the smallest relation satisfying the rules
below.

\infrule[Out-barb]{y \in {\mathcal N}, \; x \nameeq y}
		  {\outputp{x}{v} \downarrow_{\mathcal N} x}
\infrule[Par-barb]{\mbox{$P\downarrow_{\mathcal N} x$ or $Q\downarrow_{\mathcal N} x$}}
		  {\binpar{P}{Q} \downarrow_{\mathcal N} x}

We write $P \Downarrow_{\mathcal N} x$ if there is $Q$ such that 
$P \wred Q$ and $Q \downarrow_{\mathcal N} x$.
\end{definition}

\begin{definition}
%\label{def.bbisim}
An  ${\mathcal N}$-\emph{barbed bisimulation} over a set of names, ${\mathcal N}$, is a symmetric binary relation 
${\mathcal S}_{\mathcal N}$ between agents such that $P\rel{S}_{\mathcal N}Q$ implies:
\begin{enumerate}
\item If $P \red P'$ then $Q \wred Q'$ and $P'\rel{S}_{\mathcal N} Q'$.
\item If $P\downarrow_{\mathcal N} x$, then $Q\Downarrow_{\mathcal N} x$.
\end{enumerate}
$P$ is ${\mathcal N}$-barbed bisimilar to $Q$, written
$P \wbbisim_{\mathcal N} Q$, if $P \rel{S}_{\mathcal N} Q$ for some ${\mathcal N}$-barbed bisimulation ${\mathcal S}_{\mathcal N}$.
\end{definition}

$\mathcal{R} \subseteq \pi \times \pi$

$P \mathcal{R} Q => \forall P'. P \red P' \Rightarrow \exists Q'. Q \red Q', P' \mathcal{R} Q'$

$P \vdash x \Rightarrow Q \vdash x$

\begin{mathpar}
  \inferrule*[lab=Out-barb]{x \nameeq y}{{y}!\langle{Q}\rangle \vdash x}
  \and
  \inferrule*[lab=Par-barb]{\mbox{$P\vdash x$ or $Q\vdash x$}}{\binpar{P}{Q} \vdash x}
\end{mathpar}

\subsubsection{Contexts}

One of the principle advantages of computational calculi like the
$\pi$-calculus is a well-defined notion of context,
contextual-equivalence and a correlation between
contextual-equivalence and notions of bisimulation. The notion of
context allows the decomposition of a process into (sub-)process and
its syntactic environment, its context. Thus, a context may be
thought of as a process with a ``hole'' (written $\Box$) in it. The
application of a context $M$ to a process $P$, written $M[P]$, is
tantamount to filling the hole in $M$ with $P$. In this paper we do
not need the full weight of this theory, but do make use of the notion
of context in the proof the main theorem. 

\begin{mathpar}
  \inferrule* [lab=summation] {} {{M_{M},M_{N}} \bc \Box \;|\; x.M_{A} \;|\; M_{M}+M_{N}}
  \and
  \inferrule* [lab=agent] {} {{M_{A}} \bc (\vec{x})M_{P} \;| \; \clift{P_0,\ldots,M_{P},\ldots,P_N}}
  \and \\
  \inferrule* [lab=process] {} {{M_{P}} \bc M_{N} \;| \;P|M_{P} }
\end{mathpar} 

\begin{mathpar}
  \inferrule* [lab=sychronization] {} {M_{N} \bc \Box \;|\; x?M_{F} \;|\; x!M_{C}}
  \and
  \inferrule* [lab=abstraction] {} {{M_{F}} \bc (x)M_{P} }
  \and
  \inferrule* [lab=concretion] {} {{M_{C}} \bc \langle M_{P} \rangle }
  \and \\
  \inferrule* [lab=process] {} {{M_{P}} \bc M_{N} \;| \;P|M_{P} }
\end{mathpar}

\begin{definition}[contextual application] Given a context $M$, and
  process $P$, we define the \emph{contextual application}, $M[P] :=
  M\{P/\Box\}$. That is, the contextual application of M to P is the
  substitution of $P$ for $\Box$ in $M$.
\end{definition}

$\meaningof{-} : L \to \mathcal{P}(\pi)$

\begin{mathpar}
  \inferrule* [lab=collection] {} {\meaningof{true} = \pi, \and \meaningof{~E} = \pi \setminus \meaningof{E}, \and \meaningof{E_{1} \& E_{2}} = \meaningof{E_{1}} \cap \meaningof{E_{2}}}
\end{mathpar}

\begin{mathpar}
  \inferrule* [lab=structure] {} {\meaningof{0} = \{ P \in \pi | P \equiv 0 \}, \and \\ \meaningof{E_1 | E_2} = \{ P \in \pi | P \equiv P_{1} | P_{2}, P_{1} \in \meaningof{E_{1}}, P_{2} \in \meaningof{E_2}\} }
\end{mathpar}

\begin{mathpar}
 \inferrule* [lab=behavior] {} {\meaningof{\langle a?b \rangle E} = \{ P \in \pi | P \equiv Q | u?(y)P', \\ \and \\\\ \and \\ \;\;\; u \in \meaningof{a}, \forall z.P'\{z/y\} \in \meaningof{E\{z/b\}}\}, \and \\ \meaningof{a!E} = \{ P \in \pi | P \equiv Q | x!\langle P' \rangle, x \in \meaningof{a} P' \in \meaningof{E}\} }
\end{mathpar}

\begin{mathpar}
 \inferrule* [lab=nominal] {} {\meaningof{\quotep{E}} = \{ \quotep{P} \in \quotep{\pi} | P \in \meaningof{E} \}, \and \meaningof{\quotep{P}} = \{ \quotep{Q} \in \quotep{\pi} | P \equiv Q \} \and \\ \meaningof{@\quotep{E}} = \{ P \in \pi | P \equiv @x, x \in \meaningof{E} \}}
\end{mathpar}

\begin{eqnarray*}
  \\
  \meaningof{-} : TS \to ST
\end{eqnarray*}

\begin{eqnarray*}
  \\
  L : TS \to ST
\end{eqnarray*}

\begin{eqnarray*}
  \\
  P \models E \iff P \in \meaningof{E}
\end{eqnarray*}

\begin{eqnarray*}
  P \approx_{L} Q \iff \forall E \in L. P \models E \iff Q \models E
\end{eqnarray*}

\begin{eqnarray*}
  P \approx_{K} Q
\end{eqnarray*}

\begin{eqnarray*}
  P \approx Q
\end{eqnarray*}

$\approx_{K} = \approx = \approx_{L}$

\subsubsection{Contextual duality}

Note that contexts extend the quotation operation to a family of
operations from processes to names. Given a context, $M$, we can
define a \emph{nominal context}, $\quotep{M}$ by $\quotep{M}[P] :=
\quotep{M[P]}$. To foreshadow what is to come we observe that these
operations enjoy a duality with processes very much like the duality
between vectors and maps from vectors to scalars.

Further, because the calculus is essentially higher-order, we have a
correspondence between contexts and processes. More specifically,
given a name $x$ and a context $M$ we can construct $M^{*}_{x}$ such
that 

\begin{mathpar}
  M^{*}_{x} | \lift{x}{P} \red M[P]
\end{mathpar}

namely,

\begin{mathpar}
  M^{*}_{x} := x?(u).M[\dropn{u}]
\end{mathpar}

The dependence of $M^{*}_{x}$ on a name makes it an abstraction, 

\begin{mathpar}
  M^{*} := (x)x?(u).M[\dropn{u}]
\end{mathpar}

\subsection{Additional notation}

It will sometimes be convenient to denote the process a name
quotes. We already have the notation $x = \quotep{P}$, but it will be
convenient to introduce an alternate notation, $\procn{x}$, when we
want to emphasize the connection to the use of the name. Note that, by
virtue of name equivalence, $\quotep{\procn{x}} \nameeq x$; so, the
notation is consistent with previous definitions.

Further, because names have structure it is possible to effect
substitutions on the basis of that structure. This means we need to
upgrade our notation for substitutions, which we accomplish by
adapting comprehension notation. Thus,

\begin{mathpar}
  P\{ y / x : x \in S \}
\end{mathpar}

is interpreted to mean the process derived from P by replacing (in a
capture-avoiding manner) each occurrence of $x$ in $S$ by $y$. For example,

\begin{mathpar}
  P\{ \quotep{\procn{x}|\procn{x}} / x : x \in \freenames{P} \}
\end{mathpar}

will replace each (occurrence) of a free name $x$ in $P$ by
$\quotep{\procn{x}|\procn{x}}$.

Also, we will avail ourselves of the notation $x^{L}$ and $x^{R}$ to
denote injections of a name into disjoint copies of the name
space. There are numerous ways to accomplish this. One example can be
found in \cite{MeredithR05}. This notation overloads to vectors of
names: $\vec{x}^{\pi} := (x_{i}^{\pi} \; : \; 0 \leq i < |\vec{x}| )$ where $\pi \in \{L,R\}$.

We also use $P^{\Box} := P|\Box$.

In \cite{MeredithR05} an interpretation of the new operator is
given. It turns out that there are several possible interpretations
all enjoying the requisite algebraic properties of the operator (see
\cite{milner91polyadicpi}). We will therefore make liberal use of
$(\nu\; \vec{x})P$.

% subsection the_syntax_and_semantics_of_the_notation_system (end)   

\input{qm2pi.qmops} 

\input{qm2pi.sterngerlach} 

\input{qm2pi.metric} 

% section concurrent_process_calculi (end)

%\input{qm2pi.proofsketch}

% section proof sketch (end)

%\input{qm2pi.slviaknots} 

% section spatial logic via knots (end)

\input{qm2pi.conclusion}

% section conclusion (end)

%\input{qm2pi.dtcodes} 

% section wiring algorithm (end)

\input{qm2pi.ack} 

% section acknowledgments (end)

\newpage


\bibliographystyle{plain}   
\bibliography{../../biblios/main.bib}

\input{qm2pi.rhodetails}

\end{document}

 

% section wiring algorithm (end)

\documentclass[12pt]{llncs}
%\documentclass{jktr}

\usepackage[pdftex]{hyperref}                   
\usepackage {listings}
\usepackage {mathpartir}
\usepackage{bcprules}
%\usepackage{listings}
                       
\usepackage{graphicx} 
%\usepackage[margins=2.5cm,nohead,nofoot]{geometry}
%\usepackage{geometry}
\usepackage{amsfonts}
\usepackage{amstext}
\usepackage{latexsym}
\usepackage{amssymb}
\usepackage{color}


%\include{myPreamble}
\include{qm2pi.local} 

%\ifpdf
%\usepackage[pdftex]{graphicx}
%\else
%\usepackage{graphicx}
%\fi

 % \ifpdf
%  \usepackage{pdfsync}
%  \if


%\title{Brief Article}
%\author{David F. Snyder}
%\author{L.G. Meredith}

%\address{Dept. of Math., Texas State University--San Marcos, San Marcos, TX 78666}
       
\pagestyle{empty}


\begin{document}

\lstset{language=[Objective]Caml,frame=shadowbox}

\input{qm2pi.front}

% section front matter (end)

\input{qm2pi.intro} 
 
% section introduction (end)

% \input{qm2pi.knotations} 

% section notation (end)

\input{qm2pi.process.calculi} 

% section concurrent_process_calculi_and_spatial_logics_ (end)
    
%\input{qm2pi.knots2pi} 

%\input{qm2pi.trefoil} 

%\input{qm2pi.mainthm} 

% subsection basic_interpretation (end)

%\input{qm2pi.rho.presentation} 
\subsection{The syntax and semantics of the notation system}\label{sub:the_syntax_and_semantics_of_the_notation_system} % (fold)

We now summarize a technical presentation of the calculus that
embodies our theory of dynamics. The typical presentation of such a
calculus follows the style of giving generators and relations on
them. The grammar, below, describing term constructors, freely
generates the set of processes, $\Proc$. This set is then quotiented
by a relation known as structural congruence and it is over this set
that the notion of dynamics is expressed. This presentation is
essentially that of \cite{MeredithR05} with the addition of
polyadicity and summation. For readability we have relegated some of
the technical subtleties to an appendix.

\subsubsection{Process grammar}\label{subsub:process_grammar}

\begin{mathpar}
  \inferrule* [lab=synchronization] {} {{M} \bc \pzero \;|\; x?F \;|\; x!C }
  \and
  \inferrule* [lab=abstraction] {} {{F} \bc (x)P}
  \and
  \inferrule* [lab=concretion] {} {{C} \bc \langle Q \rangle}
  \and
  \inferrule* [lab=process] {} {{P,Q} \bc M \;| \;P|Q \;|\; @{x}}
  \and
  \inferrule* [lab=name] {} {{x} \bc \quotep{P}}
\end{mathpar} 

Note that $\vec{x}$ (resp. $\vec{P}$) denotes a vector of names
(resp. processes) of length $|\vec{x}|$ (resp. $|\vec{P}|$). We adopt
the following useful abbreviations.

\begin{mathpar}
   x?(\vec{y}).P := x.(\vec{y})P \and  x\clift{\vec{P}} := x.\clift{\vec{P}}
   \and x!(y) := \lift{x}{\dropn{y}}
   \and \Pi_{i=0}^{n-1}P_i := P_0 | \ldots | P_{n-1}
\end{mathpar}

\subsubsection{Structural congruence}

\paragraph{Free and bound names and alpha-equivalence.} At the
core of structural equivalence is alpha-equivalence which identifies
process that are the same up to a change of variable. Formally, we
recognize the distinction between free and bound names. The free names
of a process, $\freenames{P}$, may be calculated recursively as
follows:

\begin{mathpar}
\freenames{\pzero} := \emptyset
  \and \\
  \freenames{x?(y).P} := \{ x \} \cup (\freenames{P} \setminus \{ y \})
  \and 
  \freenames{x!\langle P \rangle} := \{ x \} \cup \{ P \} 
  \and \\
  \freenames{P|Q} := \freenames{P} \cup \freenames{Q}
  \and \\
  \freenames{@{x}} := \{ x \}
\end{mathpar}

$\pi$
$\quotep{\pi}$

$\freenames{-} : \pi \to \mathcal{P}(\quotep{\pi})$

\begin{eqnarray*}
  \freenames{\pzero} & := & \emptyset \\
  \freenames{x?(y).P} & := & \{ x \} \cup (\freenames{P} \setminus \{ y \}) \\
  \freenames{x!\langle P \rangle} & := & \{ x \} \cup \{ P \} \\
  \freenames{P|Q} & := & \freenames{P} \cup \freenames{Q} \\
  \freenames{\dropn{x}} & := & \{ x \}
\end{eqnarray*}

The bound names of a process, $\boundnames{P}$, are those names occurring in $P$
that are not free. For example, in $x?(y).0$, the name $x$ is free, while $y$ is bound.

\begin{mathpar}
  \inferrule* [lab=monoidal-laws] {} { P|Q \equiv Q|P \and P|0 \equiv P \and P|(Q|R) \equiv (P|Q)|R }
\end{mathpar}

\begin{mathpar}
  \inferrule* [lab=alpha-equivalence] {} { (x)P \equiv (y)P\{y/x\} \and y \not\in \freenames{P} }
\end{mathpar}

\begin{definition}
Then two processes, $P,Q$, are alpha-equivalent if $P = Q\{\vec{y}/\vec{x}\}$ for
some $\vec{x} \in \boundnames{Q},\vec{y} \in \boundnames{P}$, where $Q\{\vec{y}/\vec{x}\}$
denotes the capture-avoiding substitution of $\vec{y}$ for $\vec{x}$ in $Q$.
\end{definition}

\begin{definition}
  The {\em structural congruence} \cite{SangiorgiWalker} , $\equiv$,
  between processes is the least congruence containing
  alpha-equivalence, satisfying the abelian monoid laws
  (associativity, commutativity and $\pzero$ as identity) for parallel
  composition $|$ and for summation $+$.
\end{definition}

\subsection{Name equivalence}

We take name equivalence, written $\nameeq$, to be the smallest
equivalence relation generated by the following rules.

\begin{mathpar}
\inferrule*[lab=Quote-drop]
{ }
{ \quotep{@{x}} \nameeq x }

\inferrule*[lab=Struct-equiv]
{ P \scong Q }
{ \quotep{P} \nameeq \quotep{Q} }
\end{mathpar}

The astute reader will have noticed that the mutual recursion of names
and processes imposes a mutual recursion on alpha-equivalence and
structural equivalence via name-equivalence. Fortunately, all of this
works out pleasantly and we may calculate in the natural way, free of
concern. The reader interested in the details is referred to the
appendix \ref{appendix:rho_details}.

\subsection{Substitution}

We use $\Proc$ for the set of processes, $\QProc$ for the set of
names, and $\id{\{}\vec{y} / \vec{x} \id{\}}$ to denote partial maps,
$s : \QProc \rightarrow \QProc$. A map, $s$ lifts, uniquely, to a map
on process terms, $\widehat{s} : \Proc \rightarrow \Proc$ by the
following equations.

\begin{mathpar}
  (0) \psubstp{Q}{P} := 0 \\
  (R \juxtap S) \psubstp{Q}{P}
  :=    
  (R)\psubstp{Q}{P} \juxtap (S) \psubstp{Q}{P} \\
  (x?(y).R) \psubstp{Q}{P}    
  :=    
  (x)\substp{Q}{P} (z)\concat( (R \psubstn{z}{y}) \psubstp{Q}{P} ) \\
  (\lift{x}{R}) \psubstp{Q}{P}  
  :=
  \lift{(x)\substp{Q}{P}}{ R \psubstp{Q}{P} } \\
%   (\dropn{x})  \psubstp{Q}{P}       
%   := 
%   \left\{ 
%     \begin{array}{ccc} 
%       \dropn{\quotep{Q}} & & x \nameeq \quotep{P} \\
%       \dropn{x} & & otherwise \\
%     \end{array}
%   \right. 
  (\dropn{x})  \psubstp{Q}{P}       
  := 
  \left\{ 
    \begin{array}{ccc} 
      Q & & x \nameeq \quotep{P} \\
      \dropn{x} & & otherwise \\
    \end{array}
  \right.
\end{mathpar}
 

where

\begin{eqnarray}
  (x)\id{\{} \lpquote Q \rpquote / \lpquote P \rpquote \id{\}}            = 
  \left\{ 
    \begin{array}{ccc}
      \lpquote Q \rpquote & & x \nameeq \lpquote P \rpquote \\
      x & & otherwise \\
    \end{array}
  \right. \nonumber
\end{eqnarray}

and $z$ is chosen distinct from $\quotep{P}$, $\quotep{Q}$, the free
names in $Q$, and all the names in $R$. Our $\alpha$-equivalence will
be built in the standard way from this substitution.

\begin{remark}\label{rem:no_self_referential_names}
  One consequence of these definitions is that $\forall P. \quotep{P}
  \not\in \freenames{P}$.
\end{remark}

\subsection{ Dynamic quote: an example }

Anticipating something of what's to come, consider applying the
substitution, $\widehat{\id{\{}u / z \id{\}}}$, to the following pair
of processes, $\lift{w}{y!(z)}$ and $w[ \lpquote y!(z) \rpquote ]$.

\begin{eqnarray}
	\lift{w}{y!(z)}\widehat{\id{\{}u / z \id{\}}}
		& = &
		\lift{w}{y!(u)} \nonumber\\
	w[ \lpquote y!(z) \rpquote ] \widehat{ \id{\{}u / z \id{\}} }
		& = &
		w[ \lpquote y!(z) \rpquote ] \nonumber
\end{eqnarray}

Because the body of the process between quotes is impervious to
substitution, we get radically different answers. In fact, by
examining the first process in an input context,
e.g. $x?(z).\lift{w}{y!(z)}$, we see that the process under the lift
operator may be shaped by prefixed inputs binding a name inside it. In
this sense, the lift operator will be seen as a way to dynamically
construct processes before reifying them as names.

Finally equipped with these standard features we can present the
dynamics of the calculus.

\subsubsection{Operational semantics} 

Finally, we introduce the computational dynamics. What marks these
algebras as distinct from other more traditionally studied algebraic
structures, e.g. vector spaces or polynomial rings, is the manner in
which dynamics is captured. In traditional structures, dynamics is typically
expressed through morphisms between such structures, as in linear maps
between vector spaces or morphisms between rings. In algebras
associated with the semantics of computation, the dynamics is
expressed as part of the algebraic structure itself, through a
reduction reduction relation typically denoted by $\red$. Below, we
give a recursive presentation of this relation for the calculus used
in the encoding.

$\red \subseteq \pi \times \pi$
$\red : \pi \to \mathcal{P}(\pi)$

\begin{mathpar}
  \inferrule* [lab=Comm] { \textsf{match}( x_{src}, x_{trgt} ) } { x_{trgt}?(y)P \; | \; x_{src}!\langle {Q} \rangle \red P\{\quotep{Q}/y}\} }
  \and \\
  \inferrule* [lab=Par] {{P} \red {P}'} {{{P} | {Q}} \red {{P}' | {Q}}}
  \and
  \inferrule* [lab=Equiv]{{{P} \scong {P}'} \andalso {{P}' \red {Q}'} \andalso {{Q}' \scong {Q}}}{{P} \red {Q}}
\end{mathpar}

\begin{eqnarray*}
  match_{\equiv} (\quotep{P},\quotep{Q}) & := & P \equiv Q \\
  match_{\dagger}(\quotep{P},\quotep{Q}) & := & \forall R. P|Q \red^{*} R => R \red^{*} 0 \\
  match_{K}(\quotep{P},\quotep{Q}) & := & K \mbox{ for some context } K
\end{eqnarray*}

$u?(x)P | u!\langle Q \rangle \red P\{\quotep{Q}/x\}$

%We write $\wred$ for $\red^*$, and $P\red$ if $\exists Q $ such that $ P \red Q$.
We write $P\red$ if $\exists Q $ such that $ P \red Q$ and $P\not\red$, otherwise.

\section{Replication}

As mentioned before, it is known that replication (and hence
recursion) can be implemented in a higher-order process algebra
\cite{SangiorgiWalker}. As our first example of calculation with the
machinery thus far presented we give the construction explicitly in
the {\rhoc}.

\begin{eqnarray}
	D_{x} & := & \prefix{x}{y}{(\binpar{\outputp{x}{y}}{@{y}})} \nonumber\\
	\bangp_{x}{P} & := & \binpar{{x}!\langle{\binpar{D_{x}}{P}}\rangle}{D_{x}} \nonumber
\end{eqnarray}

\begin{eqnarray}
	\bangp_{x}{P} & & \nonumber\\
	=
	& {x}!\langle{(\prefix{x}{y}{(\outputp{x}{y} | @{y})) | P}}\rangle 
	      | \prefix{x}{y}{(\outputp{x}{y} | @{y})} & \nonumber\\
	\red
	& (\outputp{x}{y} | @{y})\substn{\quotep{(\prefix{x}{y}{(@{y} | \outputp{x}{y})) | P}}}{y} & \nonumber\\
	=
	& \outputp{x}{\quotep{(\prefix{x}{y}{(\outputp{x}{y} | @{y})) | P}}}
	  | {(\prefix{x}{y}{(\outputp{x}{y} | @{y})) | P}} & \nonumber\\
	\red
	& \ldots & \nonumber\\
	\red^*
	& P | P | \ldots & \nonumber
\end{eqnarray}

Of course, this encoding, as an implementation, runs away, unfolding
$\bangp{P}$ eagerly. A lazier and more implementable replication
operator, restricted to input-guarded processes, may be obtained as follows.

\begin{eqnarray}
\bangp{\prefix{u}{v}{P}} 
	:= 
	\binpar{\lift{x}{\prefix{u}{v}{(\binpar{D(x)}{P})}}}{D(x)} \nonumber
\end{eqnarray}

\begin{remark}
  Note that the lazier definition still does not deal with summation
  or mixed summation (i.e. sums over input and output). The reader is
  invited to construct definitions of replication that deal with these
  features. 

  Further, the definitions are parameterized in a name, $x$. Can you,
  gentle reader, make a definition that eliminates this parameter and
  guarantees no accidental interaction between the replication
  machinery and the process being replicated -- i.e. no accidental
  sharing of names used by the process to get its work done and the
  name(s) used by the replication to effect copying. This latter
  revision of the definition of replication is crucial to obtaining
  the expected identity $!!P \sim !P$.
\end{remark}

\begin{remark}\label{rem:paradoxical_combinator}
  The reader familiar with the lambda calculus will have noticed the
  similarity between $D$ and the paradoxical combinator.

  [Ed. note: the existence of this seems to suggest we have to be more
  restrictive on the set of processes and names we admit if we are to
  support no-cloning.]
\end{remark}

\subsubsection{Bisimulation}

The computational dynamics gives rise to another kind of equivalence,
the equivalence of computational behavior. As previously mentioned
this is typically captured \emph{via} some form of bisimulation.

% The notion we use in this paper is weak barbed bisimulation
% \cite{milner91polyadicpi}.

The notion we use in this paper is derived from weak barbed
bisimulation \cite{milner91polyadicpi}. 

\begin{definition}
An \emph{observation relation}, $\downarrow_{\mathcal N}$, over a set
of names, $\mathcal N$, is the smallest relation satisfying the rules
below.

\infrule[Out-barb]{y \in {\mathcal N}, \; x \nameeq y}
		  {\outputp{x}{v} \downarrow_{\mathcal N} x}
\infrule[Par-barb]{\mbox{$P\downarrow_{\mathcal N} x$ or $Q\downarrow_{\mathcal N} x$}}
		  {\binpar{P}{Q} \downarrow_{\mathcal N} x}

We write $P \Downarrow_{\mathcal N} x$ if there is $Q$ such that 
$P \wred Q$ and $Q \downarrow_{\mathcal N} x$.
\end{definition}

\begin{definition}
%\label{def.bbisim}
An  ${\mathcal N}$-\emph{barbed bisimulation} over a set of names, ${\mathcal N}$, is a symmetric binary relation 
${\mathcal S}_{\mathcal N}$ between agents such that $P\rel{S}_{\mathcal N}Q$ implies:
\begin{enumerate}
\item If $P \red P'$ then $Q \wred Q'$ and $P'\rel{S}_{\mathcal N} Q'$.
\item If $P\downarrow_{\mathcal N} x$, then $Q\Downarrow_{\mathcal N} x$.
\end{enumerate}
$P$ is ${\mathcal N}$-barbed bisimilar to $Q$, written
$P \wbbisim_{\mathcal N} Q$, if $P \rel{S}_{\mathcal N} Q$ for some ${\mathcal N}$-barbed bisimulation ${\mathcal S}_{\mathcal N}$.
\end{definition}

$\mathcal{R} \subseteq \pi \times \pi$

$P \mathcal{R} Q => \forall P'. P \red P' \Rightarrow \exists Q'. Q \red Q', P' \mathcal{R} Q'$

$P \vdash x \Rightarrow Q \vdash x$

\begin{mathpar}
  \inferrule*[lab=Out-barb]{x \nameeq y}{{y}!\langle{Q}\rangle \vdash x}
  \and
  \inferrule*[lab=Par-barb]{\mbox{$P\vdash x$ or $Q\vdash x$}}{\binpar{P}{Q} \vdash x}
\end{mathpar}

\subsubsection{Contexts}

One of the principle advantages of computational calculi like the
$\pi$-calculus is a well-defined notion of context,
contextual-equivalence and a correlation between
contextual-equivalence and notions of bisimulation. The notion of
context allows the decomposition of a process into (sub-)process and
its syntactic environment, its context. Thus, a context may be
thought of as a process with a ``hole'' (written $\Box$) in it. The
application of a context $M$ to a process $P$, written $M[P]$, is
tantamount to filling the hole in $M$ with $P$. In this paper we do
not need the full weight of this theory, but do make use of the notion
of context in the proof the main theorem. 

\begin{mathpar}
  \inferrule* [lab=summation] {} {{M_{M},M_{N}} \bc \Box \;|\; x.M_{A} \;|\; M_{M}+M_{N}}
  \and
  \inferrule* [lab=agent] {} {{M_{A}} \bc (\vec{x})M_{P} \;| \; \clift{P_0,\ldots,M_{P},\ldots,P_N}}
  \and \\
  \inferrule* [lab=process] {} {{M_{P}} \bc M_{N} \;| \;P|M_{P} }
\end{mathpar} 

\begin{mathpar}
  \inferrule* [lab=sychronization] {} {M_{N} \bc \Box \;|\; x?M_{F} \;|\; x!M_{C}}
  \and
  \inferrule* [lab=abstraction] {} {{M_{F}} \bc (x)M_{P} }
  \and
  \inferrule* [lab=concretion] {} {{M_{C}} \bc \langle M_{P} \rangle }
  \and \\
  \inferrule* [lab=process] {} {{M_{P}} \bc M_{N} \;| \;P|M_{P} }
\end{mathpar}

\begin{definition}[contextual application] Given a context $M$, and
  process $P$, we define the \emph{contextual application}, $M[P] :=
  M\{P/\Box\}$. That is, the contextual application of M to P is the
  substitution of $P$ for $\Box$ in $M$.
\end{definition}

$\meaningof{-} : L \to \mathcal{P}(\pi)$

\begin{mathpar}
  \inferrule* [lab=collection] {} {\meaningof{true} = \pi, \and \meaningof{~E} = \pi \setminus \meaningof{E}, \and \meaningof{E_{1} \& E_{2}} = \meaningof{E_{1}} \cap \meaningof{E_{2}}}
\end{mathpar}

\begin{mathpar}
  \inferrule* [lab=structure] {} {\meaningof{0} = \{ P \in \pi | P \equiv 0 \}, \and \\ \meaningof{E_1 | E_2} = \{ P \in \pi | P \equiv P_{1} | P_{2}, P_{1} \in \meaningof{E_{1}}, P_{2} \in \meaningof{E_2}\} }
\end{mathpar}

\begin{mathpar}
 \inferrule* [lab=behavior] {} {\meaningof{\langle a?b \rangle E} = \{ P \in \pi | P \equiv Q | u?(y)P', \\ \and \\\\ \and \\ \;\;\; u \in \meaningof{a}, \forall z.P'\{z/y\} \in \meaningof{E\{z/b\}}\}, \and \\ \meaningof{a!E} = \{ P \in \pi | P \equiv Q | x!\langle P' \rangle, x \in \meaningof{a} P' \in \meaningof{E}\} }
\end{mathpar}

\begin{mathpar}
 \inferrule* [lab=nominal] {} {\meaningof{\quotep{E}} = \{ \quotep{P} \in \quotep{\pi} | P \in \meaningof{E} \}, \and \meaningof{\quotep{P}} = \{ \quotep{Q} \in \quotep{\pi} | P \equiv Q \} \and \\ \meaningof{@\quotep{E}} = \{ P \in \pi | P \equiv @x, x \in \meaningof{E} \}}
\end{mathpar}

\begin{eqnarray*}
  \\
  \meaningof{-} : TS \to ST
\end{eqnarray*}

\begin{eqnarray*}
  \\
  L : TS \to ST
\end{eqnarray*}

\begin{eqnarray*}
  \\
  P \models E \iff P \in \meaningof{E}
\end{eqnarray*}

\begin{eqnarray*}
  P \approx_{L} Q \iff \forall E \in L. P \models E \iff Q \models E
\end{eqnarray*}

\begin{eqnarray*}
  P \approx_{K} Q
\end{eqnarray*}

\begin{eqnarray*}
  P \approx Q
\end{eqnarray*}

$\approx_{K} = \approx = \approx_{L}$

\subsubsection{Contextual duality}

Note that contexts extend the quotation operation to a family of
operations from processes to names. Given a context, $M$, we can
define a \emph{nominal context}, $\quotep{M}$ by $\quotep{M}[P] :=
\quotep{M[P]}$. To foreshadow what is to come we observe that these
operations enjoy a duality with processes very much like the duality
between vectors and maps from vectors to scalars.

Further, because the calculus is essentially higher-order, we have a
correspondence between contexts and processes. More specifically,
given a name $x$ and a context $M$ we can construct $M^{*}_{x}$ such
that 

\begin{mathpar}
  M^{*}_{x} | \lift{x}{P} \red M[P]
\end{mathpar}

namely,

\begin{mathpar}
  M^{*}_{x} := x?(u).M[\dropn{u}]
\end{mathpar}

The dependence of $M^{*}_{x}$ on a name makes it an abstraction, 

\begin{mathpar}
  M^{*} := (x)x?(u).M[\dropn{u}]
\end{mathpar}

\subsection{Additional notation}

It will sometimes be convenient to denote the process a name
quotes. We already have the notation $x = \quotep{P}$, but it will be
convenient to introduce an alternate notation, $\procn{x}$, when we
want to emphasize the connection to the use of the name. Note that, by
virtue of name equivalence, $\quotep{\procn{x}} \nameeq x$; so, the
notation is consistent with previous definitions.

Further, because names have structure it is possible to effect
substitutions on the basis of that structure. This means we need to
upgrade our notation for substitutions, which we accomplish by
adapting comprehension notation. Thus,

\begin{mathpar}
  P\{ y / x : x \in S \}
\end{mathpar}

is interpreted to mean the process derived from P by replacing (in a
capture-avoiding manner) each occurrence of $x$ in $S$ by $y$. For example,

\begin{mathpar}
  P\{ \quotep{\procn{x}|\procn{x}} / x : x \in \freenames{P} \}
\end{mathpar}

will replace each (occurrence) of a free name $x$ in $P$ by
$\quotep{\procn{x}|\procn{x}}$.

Also, we will avail ourselves of the notation $x^{L}$ and $x^{R}$ to
denote injections of a name into disjoint copies of the name
space. There are numerous ways to accomplish this. One example can be
found in \cite{MeredithR05}. This notation overloads to vectors of
names: $\vec{x}^{\pi} := (x_{i}^{\pi} \; : \; 0 \leq i < |\vec{x}| )$ where $\pi \in \{L,R\}$.

We also use $P^{\Box} := P|\Box$.

In \cite{MeredithR05} an interpretation of the new operator is
given. It turns out that there are several possible interpretations
all enjoying the requisite algebraic properties of the operator (see
\cite{milner91polyadicpi}). We will therefore make liberal use of
$(\nu\; \vec{x})P$.

% subsection the_syntax_and_semantics_of_the_notation_system (end)   

\input{qm2pi.qmops} 

\input{qm2pi.sterngerlach} 

\input{qm2pi.metric} 

% section concurrent_process_calculi (end)

%\input{qm2pi.proofsketch}

% section proof sketch (end)

%\input{qm2pi.slviaknots} 

% section spatial logic via knots (end)

\input{qm2pi.conclusion}

% section conclusion (end)

%\input{qm2pi.dtcodes} 

% section wiring algorithm (end)

\input{qm2pi.ack} 

% section acknowledgments (end)

\newpage


\bibliographystyle{plain}   
\bibliography{../../biblios/main.bib}

\input{qm2pi.rhodetails}

\end{document}

 

% section acknowledgments (end)

\newpage


\bibliographystyle{plain}   
\bibliography{../../biblios/main.bib}

\documentclass[12pt]{llncs}
%\documentclass{jktr}

\usepackage[pdftex]{hyperref}                   
\usepackage {listings}
\usepackage {mathpartir}
\usepackage{bcprules}
%\usepackage{listings}
                       
\usepackage{graphicx} 
%\usepackage[margins=2.5cm,nohead,nofoot]{geometry}
%\usepackage{geometry}
\usepackage{amsfonts}
\usepackage{amstext}
\usepackage{latexsym}
\usepackage{amssymb}
\usepackage{color}


%\include{myPreamble}
\include{qm2pi.local} 

%\ifpdf
%\usepackage[pdftex]{graphicx}
%\else
%\usepackage{graphicx}
%\fi

 % \ifpdf
%  \usepackage{pdfsync}
%  \if


%\title{Brief Article}
%\author{David F. Snyder}
%\author{L.G. Meredith}

%\address{Dept. of Math., Texas State University--San Marcos, San Marcos, TX 78666}
       
\pagestyle{empty}


\begin{document}

\lstset{language=[Objective]Caml,frame=shadowbox}

\input{qm2pi.front}

% section front matter (end)

\input{qm2pi.intro} 
 
% section introduction (end)

% \input{qm2pi.knotations} 

% section notation (end)

\input{qm2pi.process.calculi} 

% section concurrent_process_calculi_and_spatial_logics_ (end)
    
%\input{qm2pi.knots2pi} 

%\input{qm2pi.trefoil} 

%\input{qm2pi.mainthm} 

% subsection basic_interpretation (end)

%\input{qm2pi.rho.presentation} 
\subsection{The syntax and semantics of the notation system}\label{sub:the_syntax_and_semantics_of_the_notation_system} % (fold)

We now summarize a technical presentation of the calculus that
embodies our theory of dynamics. The typical presentation of such a
calculus follows the style of giving generators and relations on
them. The grammar, below, describing term constructors, freely
generates the set of processes, $\Proc$. This set is then quotiented
by a relation known as structural congruence and it is over this set
that the notion of dynamics is expressed. This presentation is
essentially that of \cite{MeredithR05} with the addition of
polyadicity and summation. For readability we have relegated some of
the technical subtleties to an appendix.

\subsubsection{Process grammar}\label{subsub:process_grammar}

\begin{mathpar}
  \inferrule* [lab=synchronization] {} {{M} \bc \pzero \;|\; x?F \;|\; x!C }
  \and
  \inferrule* [lab=abstraction] {} {{F} \bc (x)P}
  \and
  \inferrule* [lab=concretion] {} {{C} \bc \langle Q \rangle}
  \and
  \inferrule* [lab=process] {} {{P,Q} \bc M \;| \;P|Q \;|\; @{x}}
  \and
  \inferrule* [lab=name] {} {{x} \bc \quotep{P}}
\end{mathpar} 

Note that $\vec{x}$ (resp. $\vec{P}$) denotes a vector of names
(resp. processes) of length $|\vec{x}|$ (resp. $|\vec{P}|$). We adopt
the following useful abbreviations.

\begin{mathpar}
   x?(\vec{y}).P := x.(\vec{y})P \and  x\clift{\vec{P}} := x.\clift{\vec{P}}
   \and x!(y) := \lift{x}{\dropn{y}}
   \and \Pi_{i=0}^{n-1}P_i := P_0 | \ldots | P_{n-1}
\end{mathpar}

\subsubsection{Structural congruence}

\paragraph{Free and bound names and alpha-equivalence.} At the
core of structural equivalence is alpha-equivalence which identifies
process that are the same up to a change of variable. Formally, we
recognize the distinction between free and bound names. The free names
of a process, $\freenames{P}$, may be calculated recursively as
follows:

\begin{mathpar}
\freenames{\pzero} := \emptyset
  \and \\
  \freenames{x?(y).P} := \{ x \} \cup (\freenames{P} \setminus \{ y \})
  \and 
  \freenames{x!\langle P \rangle} := \{ x \} \cup \{ P \} 
  \and \\
  \freenames{P|Q} := \freenames{P} \cup \freenames{Q}
  \and \\
  \freenames{@{x}} := \{ x \}
\end{mathpar}

$\pi$
$\quotep{\pi}$

$\freenames{-} : \pi \to \mathcal{P}(\quotep{\pi})$

\begin{eqnarray*}
  \freenames{\pzero} & := & \emptyset \\
  \freenames{x?(y).P} & := & \{ x \} \cup (\freenames{P} \setminus \{ y \}) \\
  \freenames{x!\langle P \rangle} & := & \{ x \} \cup \{ P \} \\
  \freenames{P|Q} & := & \freenames{P} \cup \freenames{Q} \\
  \freenames{\dropn{x}} & := & \{ x \}
\end{eqnarray*}

The bound names of a process, $\boundnames{P}$, are those names occurring in $P$
that are not free. For example, in $x?(y).0$, the name $x$ is free, while $y$ is bound.

\begin{mathpar}
  \inferrule* [lab=monoidal-laws] {} { P|Q \equiv Q|P \and P|0 \equiv P \and P|(Q|R) \equiv (P|Q)|R }
\end{mathpar}

\begin{mathpar}
  \inferrule* [lab=alpha-equivalence] {} { (x)P \equiv (y)P\{y/x\} \and y \not\in \freenames{P} }
\end{mathpar}

\begin{definition}
Then two processes, $P,Q$, are alpha-equivalent if $P = Q\{\vec{y}/\vec{x}\}$ for
some $\vec{x} \in \boundnames{Q},\vec{y} \in \boundnames{P}$, where $Q\{\vec{y}/\vec{x}\}$
denotes the capture-avoiding substitution of $\vec{y}$ for $\vec{x}$ in $Q$.
\end{definition}

\begin{definition}
  The {\em structural congruence} \cite{SangiorgiWalker} , $\equiv$,
  between processes is the least congruence containing
  alpha-equivalence, satisfying the abelian monoid laws
  (associativity, commutativity and $\pzero$ as identity) for parallel
  composition $|$ and for summation $+$.
\end{definition}

\subsection{Name equivalence}

We take name equivalence, written $\nameeq$, to be the smallest
equivalence relation generated by the following rules.

\begin{mathpar}
\inferrule*[lab=Quote-drop]
{ }
{ \quotep{@{x}} \nameeq x }

\inferrule*[lab=Struct-equiv]
{ P \scong Q }
{ \quotep{P} \nameeq \quotep{Q} }
\end{mathpar}

The astute reader will have noticed that the mutual recursion of names
and processes imposes a mutual recursion on alpha-equivalence and
structural equivalence via name-equivalence. Fortunately, all of this
works out pleasantly and we may calculate in the natural way, free of
concern. The reader interested in the details is referred to the
appendix \ref{appendix:rho_details}.

\subsection{Substitution}

We use $\Proc$ for the set of processes, $\QProc$ for the set of
names, and $\id{\{}\vec{y} / \vec{x} \id{\}}$ to denote partial maps,
$s : \QProc \rightarrow \QProc$. A map, $s$ lifts, uniquely, to a map
on process terms, $\widehat{s} : \Proc \rightarrow \Proc$ by the
following equations.

\begin{mathpar}
  (0) \psubstp{Q}{P} := 0 \\
  (R \juxtap S) \psubstp{Q}{P}
  :=    
  (R)\psubstp{Q}{P} \juxtap (S) \psubstp{Q}{P} \\
  (x?(y).R) \psubstp{Q}{P}    
  :=    
  (x)\substp{Q}{P} (z)\concat( (R \psubstn{z}{y}) \psubstp{Q}{P} ) \\
  (\lift{x}{R}) \psubstp{Q}{P}  
  :=
  \lift{(x)\substp{Q}{P}}{ R \psubstp{Q}{P} } \\
%   (\dropn{x})  \psubstp{Q}{P}       
%   := 
%   \left\{ 
%     \begin{array}{ccc} 
%       \dropn{\quotep{Q}} & & x \nameeq \quotep{P} \\
%       \dropn{x} & & otherwise \\
%     \end{array}
%   \right. 
  (\dropn{x})  \psubstp{Q}{P}       
  := 
  \left\{ 
    \begin{array}{ccc} 
      Q & & x \nameeq \quotep{P} \\
      \dropn{x} & & otherwise \\
    \end{array}
  \right.
\end{mathpar}
 

where

\begin{eqnarray}
  (x)\id{\{} \lpquote Q \rpquote / \lpquote P \rpquote \id{\}}            = 
  \left\{ 
    \begin{array}{ccc}
      \lpquote Q \rpquote & & x \nameeq \lpquote P \rpquote \\
      x & & otherwise \\
    \end{array}
  \right. \nonumber
\end{eqnarray}

and $z$ is chosen distinct from $\quotep{P}$, $\quotep{Q}$, the free
names in $Q$, and all the names in $R$. Our $\alpha$-equivalence will
be built in the standard way from this substitution.

\begin{remark}\label{rem:no_self_referential_names}
  One consequence of these definitions is that $\forall P. \quotep{P}
  \not\in \freenames{P}$.
\end{remark}

\subsection{ Dynamic quote: an example }

Anticipating something of what's to come, consider applying the
substitution, $\widehat{\id{\{}u / z \id{\}}}$, to the following pair
of processes, $\lift{w}{y!(z)}$ and $w[ \lpquote y!(z) \rpquote ]$.

\begin{eqnarray}
	\lift{w}{y!(z)}\widehat{\id{\{}u / z \id{\}}}
		& = &
		\lift{w}{y!(u)} \nonumber\\
	w[ \lpquote y!(z) \rpquote ] \widehat{ \id{\{}u / z \id{\}} }
		& = &
		w[ \lpquote y!(z) \rpquote ] \nonumber
\end{eqnarray}

Because the body of the process between quotes is impervious to
substitution, we get radically different answers. In fact, by
examining the first process in an input context,
e.g. $x?(z).\lift{w}{y!(z)}$, we see that the process under the lift
operator may be shaped by prefixed inputs binding a name inside it. In
this sense, the lift operator will be seen as a way to dynamically
construct processes before reifying them as names.

Finally equipped with these standard features we can present the
dynamics of the calculus.

\subsubsection{Operational semantics} 

Finally, we introduce the computational dynamics. What marks these
algebras as distinct from other more traditionally studied algebraic
structures, e.g. vector spaces or polynomial rings, is the manner in
which dynamics is captured. In traditional structures, dynamics is typically
expressed through morphisms between such structures, as in linear maps
between vector spaces or morphisms between rings. In algebras
associated with the semantics of computation, the dynamics is
expressed as part of the algebraic structure itself, through a
reduction reduction relation typically denoted by $\red$. Below, we
give a recursive presentation of this relation for the calculus used
in the encoding.

$\red \subseteq \pi \times \pi$
$\red : \pi \to \mathcal{P}(\pi)$

\begin{mathpar}
  \inferrule* [lab=Comm] { \textsf{match}( x_{src}, x_{trgt} ) } { x_{trgt}?(y)P \; | \; x_{src}!\langle {Q} \rangle \red P\{\quotep{Q}/y}\} }
  \and \\
  \inferrule* [lab=Par] {{P} \red {P}'} {{{P} | {Q}} \red {{P}' | {Q}}}
  \and
  \inferrule* [lab=Equiv]{{{P} \scong {P}'} \andalso {{P}' \red {Q}'} \andalso {{Q}' \scong {Q}}}{{P} \red {Q}}
\end{mathpar}

\begin{eqnarray*}
  match_{\equiv} (\quotep{P},\quotep{Q}) & := & P \equiv Q \\
  match_{\dagger}(\quotep{P},\quotep{Q}) & := & \forall R. P|Q \red^{*} R => R \red^{*} 0 \\
  match_{K}(\quotep{P},\quotep{Q}) & := & K \mbox{ for some context } K
\end{eqnarray*}

$u?(x)P | u!\langle Q \rangle \red P\{\quotep{Q}/x\}$

%We write $\wred$ for $\red^*$, and $P\red$ if $\exists Q $ such that $ P \red Q$.
We write $P\red$ if $\exists Q $ such that $ P \red Q$ and $P\not\red$, otherwise.

\section{Replication}

As mentioned before, it is known that replication (and hence
recursion) can be implemented in a higher-order process algebra
\cite{SangiorgiWalker}. As our first example of calculation with the
machinery thus far presented we give the construction explicitly in
the {\rhoc}.

\begin{eqnarray}
	D_{x} & := & \prefix{x}{y}{(\binpar{\outputp{x}{y}}{@{y}})} \nonumber\\
	\bangp_{x}{P} & := & \binpar{{x}!\langle{\binpar{D_{x}}{P}}\rangle}{D_{x}} \nonumber
\end{eqnarray}

\begin{eqnarray}
	\bangp_{x}{P} & & \nonumber\\
	=
	& {x}!\langle{(\prefix{x}{y}{(\outputp{x}{y} | @{y})) | P}}\rangle 
	      | \prefix{x}{y}{(\outputp{x}{y} | @{y})} & \nonumber\\
	\red
	& (\outputp{x}{y} | @{y})\substn{\quotep{(\prefix{x}{y}{(@{y} | \outputp{x}{y})) | P}}}{y} & \nonumber\\
	=
	& \outputp{x}{\quotep{(\prefix{x}{y}{(\outputp{x}{y} | @{y})) | P}}}
	  | {(\prefix{x}{y}{(\outputp{x}{y} | @{y})) | P}} & \nonumber\\
	\red
	& \ldots & \nonumber\\
	\red^*
	& P | P | \ldots & \nonumber
\end{eqnarray}

Of course, this encoding, as an implementation, runs away, unfolding
$\bangp{P}$ eagerly. A lazier and more implementable replication
operator, restricted to input-guarded processes, may be obtained as follows.

\begin{eqnarray}
\bangp{\prefix{u}{v}{P}} 
	:= 
	\binpar{\lift{x}{\prefix{u}{v}{(\binpar{D(x)}{P})}}}{D(x)} \nonumber
\end{eqnarray}

\begin{remark}
  Note that the lazier definition still does not deal with summation
  or mixed summation (i.e. sums over input and output). The reader is
  invited to construct definitions of replication that deal with these
  features. 

  Further, the definitions are parameterized in a name, $x$. Can you,
  gentle reader, make a definition that eliminates this parameter and
  guarantees no accidental interaction between the replication
  machinery and the process being replicated -- i.e. no accidental
  sharing of names used by the process to get its work done and the
  name(s) used by the replication to effect copying. This latter
  revision of the definition of replication is crucial to obtaining
  the expected identity $!!P \sim !P$.
\end{remark}

\begin{remark}\label{rem:paradoxical_combinator}
  The reader familiar with the lambda calculus will have noticed the
  similarity between $D$ and the paradoxical combinator.

  [Ed. note: the existence of this seems to suggest we have to be more
  restrictive on the set of processes and names we admit if we are to
  support no-cloning.]
\end{remark}

\subsubsection{Bisimulation}

The computational dynamics gives rise to another kind of equivalence,
the equivalence of computational behavior. As previously mentioned
this is typically captured \emph{via} some form of bisimulation.

% The notion we use in this paper is weak barbed bisimulation
% \cite{milner91polyadicpi}.

The notion we use in this paper is derived from weak barbed
bisimulation \cite{milner91polyadicpi}. 

\begin{definition}
An \emph{observation relation}, $\downarrow_{\mathcal N}$, over a set
of names, $\mathcal N$, is the smallest relation satisfying the rules
below.

\infrule[Out-barb]{y \in {\mathcal N}, \; x \nameeq y}
		  {\outputp{x}{v} \downarrow_{\mathcal N} x}
\infrule[Par-barb]{\mbox{$P\downarrow_{\mathcal N} x$ or $Q\downarrow_{\mathcal N} x$}}
		  {\binpar{P}{Q} \downarrow_{\mathcal N} x}

We write $P \Downarrow_{\mathcal N} x$ if there is $Q$ such that 
$P \wred Q$ and $Q \downarrow_{\mathcal N} x$.
\end{definition}

\begin{definition}
%\label{def.bbisim}
An  ${\mathcal N}$-\emph{barbed bisimulation} over a set of names, ${\mathcal N}$, is a symmetric binary relation 
${\mathcal S}_{\mathcal N}$ between agents such that $P\rel{S}_{\mathcal N}Q$ implies:
\begin{enumerate}
\item If $P \red P'$ then $Q \wred Q'$ and $P'\rel{S}_{\mathcal N} Q'$.
\item If $P\downarrow_{\mathcal N} x$, then $Q\Downarrow_{\mathcal N} x$.
\end{enumerate}
$P$ is ${\mathcal N}$-barbed bisimilar to $Q$, written
$P \wbbisim_{\mathcal N} Q$, if $P \rel{S}_{\mathcal N} Q$ for some ${\mathcal N}$-barbed bisimulation ${\mathcal S}_{\mathcal N}$.
\end{definition}

$\mathcal{R} \subseteq \pi \times \pi$

$P \mathcal{R} Q => \forall P'. P \red P' \Rightarrow \exists Q'. Q \red Q', P' \mathcal{R} Q'$

$P \vdash x \Rightarrow Q \vdash x$

\begin{mathpar}
  \inferrule*[lab=Out-barb]{x \nameeq y}{{y}!\langle{Q}\rangle \vdash x}
  \and
  \inferrule*[lab=Par-barb]{\mbox{$P\vdash x$ or $Q\vdash x$}}{\binpar{P}{Q} \vdash x}
\end{mathpar}

\subsubsection{Contexts}

One of the principle advantages of computational calculi like the
$\pi$-calculus is a well-defined notion of context,
contextual-equivalence and a correlation between
contextual-equivalence and notions of bisimulation. The notion of
context allows the decomposition of a process into (sub-)process and
its syntactic environment, its context. Thus, a context may be
thought of as a process with a ``hole'' (written $\Box$) in it. The
application of a context $M$ to a process $P$, written $M[P]$, is
tantamount to filling the hole in $M$ with $P$. In this paper we do
not need the full weight of this theory, but do make use of the notion
of context in the proof the main theorem. 

\begin{mathpar}
  \inferrule* [lab=summation] {} {{M_{M},M_{N}} \bc \Box \;|\; x.M_{A} \;|\; M_{M}+M_{N}}
  \and
  \inferrule* [lab=agent] {} {{M_{A}} \bc (\vec{x})M_{P} \;| \; \clift{P_0,\ldots,M_{P},\ldots,P_N}}
  \and \\
  \inferrule* [lab=process] {} {{M_{P}} \bc M_{N} \;| \;P|M_{P} }
\end{mathpar} 

\begin{mathpar}
  \inferrule* [lab=sychronization] {} {M_{N} \bc \Box \;|\; x?M_{F} \;|\; x!M_{C}}
  \and
  \inferrule* [lab=abstraction] {} {{M_{F}} \bc (x)M_{P} }
  \and
  \inferrule* [lab=concretion] {} {{M_{C}} \bc \langle M_{P} \rangle }
  \and \\
  \inferrule* [lab=process] {} {{M_{P}} \bc M_{N} \;| \;P|M_{P} }
\end{mathpar}

\begin{definition}[contextual application] Given a context $M$, and
  process $P$, we define the \emph{contextual application}, $M[P] :=
  M\{P/\Box\}$. That is, the contextual application of M to P is the
  substitution of $P$ for $\Box$ in $M$.
\end{definition}

$\meaningof{-} : L \to \mathcal{P}(\pi)$

\begin{mathpar}
  \inferrule* [lab=collection] {} {\meaningof{true} = \pi, \and \meaningof{~E} = \pi \setminus \meaningof{E}, \and \meaningof{E_{1} \& E_{2}} = \meaningof{E_{1}} \cap \meaningof{E_{2}}}
\end{mathpar}

\begin{mathpar}
  \inferrule* [lab=structure] {} {\meaningof{0} = \{ P \in \pi | P \equiv 0 \}, \and \\ \meaningof{E_1 | E_2} = \{ P \in \pi | P \equiv P_{1} | P_{2}, P_{1} \in \meaningof{E_{1}}, P_{2} \in \meaningof{E_2}\} }
\end{mathpar}

\begin{mathpar}
 \inferrule* [lab=behavior] {} {\meaningof{\langle a?b \rangle E} = \{ P \in \pi | P \equiv Q | u?(y)P', \\ \and \\\\ \and \\ \;\;\; u \in \meaningof{a}, \forall z.P'\{z/y\} \in \meaningof{E\{z/b\}}\}, \and \\ \meaningof{a!E} = \{ P \in \pi | P \equiv Q | x!\langle P' \rangle, x \in \meaningof{a} P' \in \meaningof{E}\} }
\end{mathpar}

\begin{mathpar}
 \inferrule* [lab=nominal] {} {\meaningof{\quotep{E}} = \{ \quotep{P} \in \quotep{\pi} | P \in \meaningof{E} \}, \and \meaningof{\quotep{P}} = \{ \quotep{Q} \in \quotep{\pi} | P \equiv Q \} \and \\ \meaningof{@\quotep{E}} = \{ P \in \pi | P \equiv @x, x \in \meaningof{E} \}}
\end{mathpar}

\begin{eqnarray*}
  \\
  \meaningof{-} : TS \to ST
\end{eqnarray*}

\begin{eqnarray*}
  \\
  L : TS \to ST
\end{eqnarray*}

\begin{eqnarray*}
  \\
  P \models E \iff P \in \meaningof{E}
\end{eqnarray*}

\begin{eqnarray*}
  P \approx_{L} Q \iff \forall E \in L. P \models E \iff Q \models E
\end{eqnarray*}

\begin{eqnarray*}
  P \approx_{K} Q
\end{eqnarray*}

\begin{eqnarray*}
  P \approx Q
\end{eqnarray*}

$\approx_{K} = \approx = \approx_{L}$

\subsubsection{Contextual duality}

Note that contexts extend the quotation operation to a family of
operations from processes to names. Given a context, $M$, we can
define a \emph{nominal context}, $\quotep{M}$ by $\quotep{M}[P] :=
\quotep{M[P]}$. To foreshadow what is to come we observe that these
operations enjoy a duality with processes very much like the duality
between vectors and maps from vectors to scalars.

Further, because the calculus is essentially higher-order, we have a
correspondence between contexts and processes. More specifically,
given a name $x$ and a context $M$ we can construct $M^{*}_{x}$ such
that 

\begin{mathpar}
  M^{*}_{x} | \lift{x}{P} \red M[P]
\end{mathpar}

namely,

\begin{mathpar}
  M^{*}_{x} := x?(u).M[\dropn{u}]
\end{mathpar}

The dependence of $M^{*}_{x}$ on a name makes it an abstraction, 

\begin{mathpar}
  M^{*} := (x)x?(u).M[\dropn{u}]
\end{mathpar}

\subsection{Additional notation}

It will sometimes be convenient to denote the process a name
quotes. We already have the notation $x = \quotep{P}$, but it will be
convenient to introduce an alternate notation, $\procn{x}$, when we
want to emphasize the connection to the use of the name. Note that, by
virtue of name equivalence, $\quotep{\procn{x}} \nameeq x$; so, the
notation is consistent with previous definitions.

Further, because names have structure it is possible to effect
substitutions on the basis of that structure. This means we need to
upgrade our notation for substitutions, which we accomplish by
adapting comprehension notation. Thus,

\begin{mathpar}
  P\{ y / x : x \in S \}
\end{mathpar}

is interpreted to mean the process derived from P by replacing (in a
capture-avoiding manner) each occurrence of $x$ in $S$ by $y$. For example,

\begin{mathpar}
  P\{ \quotep{\procn{x}|\procn{x}} / x : x \in \freenames{P} \}
\end{mathpar}

will replace each (occurrence) of a free name $x$ in $P$ by
$\quotep{\procn{x}|\procn{x}}$.

Also, we will avail ourselves of the notation $x^{L}$ and $x^{R}$ to
denote injections of a name into disjoint copies of the name
space. There are numerous ways to accomplish this. One example can be
found in \cite{MeredithR05}. This notation overloads to vectors of
names: $\vec{x}^{\pi} := (x_{i}^{\pi} \; : \; 0 \leq i < |\vec{x}| )$ where $\pi \in \{L,R\}$.

We also use $P^{\Box} := P|\Box$.

In \cite{MeredithR05} an interpretation of the new operator is
given. It turns out that there are several possible interpretations
all enjoying the requisite algebraic properties of the operator (see
\cite{milner91polyadicpi}). We will therefore make liberal use of
$(\nu\; \vec{x})P$.

% subsection the_syntax_and_semantics_of_the_notation_system (end)   

\input{qm2pi.qmops} 

\input{qm2pi.sterngerlach} 

\input{qm2pi.metric} 

% section concurrent_process_calculi (end)

%\input{qm2pi.proofsketch}

% section proof sketch (end)

%\input{qm2pi.slviaknots} 

% section spatial logic via knots (end)

\input{qm2pi.conclusion}

% section conclusion (end)

%\input{qm2pi.dtcodes} 

% section wiring algorithm (end)

\input{qm2pi.ack} 

% section acknowledgments (end)

\newpage


\bibliographystyle{plain}   
\bibliography{../../biblios/main.bib}

\input{qm2pi.rhodetails}

\end{document}



\end{document}

 

% section concurrent_process_calculi (end)

%\documentclass[12pt]{llncs}
%\documentclass{jktr}

\usepackage[pdftex]{hyperref}                   
\usepackage {listings}
\usepackage {mathpartir}
\usepackage{bcprules}
%\usepackage{listings}
                       
\usepackage{graphicx} 
%\usepackage[margins=2.5cm,nohead,nofoot]{geometry}
%\usepackage{geometry}
\usepackage{amsfonts}
\usepackage{amstext}
\usepackage{latexsym}
\usepackage{amssymb}
\usepackage{color}


%\include{myPreamble}
\documentclass[12pt]{llncs}
%\documentclass{jktr}

\usepackage[pdftex]{hyperref}                   
\usepackage {listings}
\usepackage {mathpartir}
\usepackage{bcprules}
%\usepackage{listings}
                       
\usepackage{graphicx} 
%\usepackage[margins=2.5cm,nohead,nofoot]{geometry}
%\usepackage{geometry}
\usepackage{amsfonts}
\usepackage{amstext}
\usepackage{latexsym}
\usepackage{amssymb}
\usepackage{color}


%\include{myPreamble}
\include{qm2pi.local} 

%\ifpdf
%\usepackage[pdftex]{graphicx}
%\else
%\usepackage{graphicx}
%\fi

 % \ifpdf
%  \usepackage{pdfsync}
%  \if


%\title{Brief Article}
%\author{David F. Snyder}
%\author{L.G. Meredith}

%\address{Dept. of Math., Texas State University--San Marcos, San Marcos, TX 78666}
       
\pagestyle{empty}


\begin{document}

\lstset{language=[Objective]Caml,frame=shadowbox}

\input{qm2pi.front}

% section front matter (end)

\input{qm2pi.intro} 
 
% section introduction (end)

% \input{qm2pi.knotations} 

% section notation (end)

\input{qm2pi.process.calculi} 

% section concurrent_process_calculi_and_spatial_logics_ (end)
    
%\input{qm2pi.knots2pi} 

%\input{qm2pi.trefoil} 

%\input{qm2pi.mainthm} 

% subsection basic_interpretation (end)

%\input{qm2pi.rho.presentation} 
\subsection{The syntax and semantics of the notation system}\label{sub:the_syntax_and_semantics_of_the_notation_system} % (fold)

We now summarize a technical presentation of the calculus that
embodies our theory of dynamics. The typical presentation of such a
calculus follows the style of giving generators and relations on
them. The grammar, below, describing term constructors, freely
generates the set of processes, $\Proc$. This set is then quotiented
by a relation known as structural congruence and it is over this set
that the notion of dynamics is expressed. This presentation is
essentially that of \cite{MeredithR05} with the addition of
polyadicity and summation. For readability we have relegated some of
the technical subtleties to an appendix.

\subsubsection{Process grammar}\label{subsub:process_grammar}

\begin{mathpar}
  \inferrule* [lab=synchronization] {} {{M} \bc \pzero \;|\; x?F \;|\; x!C }
  \and
  \inferrule* [lab=abstraction] {} {{F} \bc (x)P}
  \and
  \inferrule* [lab=concretion] {} {{C} \bc \langle Q \rangle}
  \and
  \inferrule* [lab=process] {} {{P,Q} \bc M \;| \;P|Q \;|\; @{x}}
  \and
  \inferrule* [lab=name] {} {{x} \bc \quotep{P}}
\end{mathpar} 

Note that $\vec{x}$ (resp. $\vec{P}$) denotes a vector of names
(resp. processes) of length $|\vec{x}|$ (resp. $|\vec{P}|$). We adopt
the following useful abbreviations.

\begin{mathpar}
   x?(\vec{y}).P := x.(\vec{y})P \and  x\clift{\vec{P}} := x.\clift{\vec{P}}
   \and x!(y) := \lift{x}{\dropn{y}}
   \and \Pi_{i=0}^{n-1}P_i := P_0 | \ldots | P_{n-1}
\end{mathpar}

\subsubsection{Structural congruence}

\paragraph{Free and bound names and alpha-equivalence.} At the
core of structural equivalence is alpha-equivalence which identifies
process that are the same up to a change of variable. Formally, we
recognize the distinction between free and bound names. The free names
of a process, $\freenames{P}$, may be calculated recursively as
follows:

\begin{mathpar}
\freenames{\pzero} := \emptyset
  \and \\
  \freenames{x?(y).P} := \{ x \} \cup (\freenames{P} \setminus \{ y \})
  \and 
  \freenames{x!\langle P \rangle} := \{ x \} \cup \{ P \} 
  \and \\
  \freenames{P|Q} := \freenames{P} \cup \freenames{Q}
  \and \\
  \freenames{@{x}} := \{ x \}
\end{mathpar}

$\pi$
$\quotep{\pi}$

$\freenames{-} : \pi \to \mathcal{P}(\quotep{\pi})$

\begin{eqnarray*}
  \freenames{\pzero} & := & \emptyset \\
  \freenames{x?(y).P} & := & \{ x \} \cup (\freenames{P} \setminus \{ y \}) \\
  \freenames{x!\langle P \rangle} & := & \{ x \} \cup \{ P \} \\
  \freenames{P|Q} & := & \freenames{P} \cup \freenames{Q} \\
  \freenames{\dropn{x}} & := & \{ x \}
\end{eqnarray*}

The bound names of a process, $\boundnames{P}$, are those names occurring in $P$
that are not free. For example, in $x?(y).0$, the name $x$ is free, while $y$ is bound.

\begin{mathpar}
  \inferrule* [lab=monoidal-laws] {} { P|Q \equiv Q|P \and P|0 \equiv P \and P|(Q|R) \equiv (P|Q)|R }
\end{mathpar}

\begin{mathpar}
  \inferrule* [lab=alpha-equivalence] {} { (x)P \equiv (y)P\{y/x\} \and y \not\in \freenames{P} }
\end{mathpar}

\begin{definition}
Then two processes, $P,Q$, are alpha-equivalent if $P = Q\{\vec{y}/\vec{x}\}$ for
some $\vec{x} \in \boundnames{Q},\vec{y} \in \boundnames{P}$, where $Q\{\vec{y}/\vec{x}\}$
denotes the capture-avoiding substitution of $\vec{y}$ for $\vec{x}$ in $Q$.
\end{definition}

\begin{definition}
  The {\em structural congruence} \cite{SangiorgiWalker} , $\equiv$,
  between processes is the least congruence containing
  alpha-equivalence, satisfying the abelian monoid laws
  (associativity, commutativity and $\pzero$ as identity) for parallel
  composition $|$ and for summation $+$.
\end{definition}

\subsection{Name equivalence}

We take name equivalence, written $\nameeq$, to be the smallest
equivalence relation generated by the following rules.

\begin{mathpar}
\inferrule*[lab=Quote-drop]
{ }
{ \quotep{@{x}} \nameeq x }

\inferrule*[lab=Struct-equiv]
{ P \scong Q }
{ \quotep{P} \nameeq \quotep{Q} }
\end{mathpar}

The astute reader will have noticed that the mutual recursion of names
and processes imposes a mutual recursion on alpha-equivalence and
structural equivalence via name-equivalence. Fortunately, all of this
works out pleasantly and we may calculate in the natural way, free of
concern. The reader interested in the details is referred to the
appendix \ref{appendix:rho_details}.

\subsection{Substitution}

We use $\Proc$ for the set of processes, $\QProc$ for the set of
names, and $\id{\{}\vec{y} / \vec{x} \id{\}}$ to denote partial maps,
$s : \QProc \rightarrow \QProc$. A map, $s$ lifts, uniquely, to a map
on process terms, $\widehat{s} : \Proc \rightarrow \Proc$ by the
following equations.

\begin{mathpar}
  (0) \psubstp{Q}{P} := 0 \\
  (R \juxtap S) \psubstp{Q}{P}
  :=    
  (R)\psubstp{Q}{P} \juxtap (S) \psubstp{Q}{P} \\
  (x?(y).R) \psubstp{Q}{P}    
  :=    
  (x)\substp{Q}{P} (z)\concat( (R \psubstn{z}{y}) \psubstp{Q}{P} ) \\
  (\lift{x}{R}) \psubstp{Q}{P}  
  :=
  \lift{(x)\substp{Q}{P}}{ R \psubstp{Q}{P} } \\
%   (\dropn{x})  \psubstp{Q}{P}       
%   := 
%   \left\{ 
%     \begin{array}{ccc} 
%       \dropn{\quotep{Q}} & & x \nameeq \quotep{P} \\
%       \dropn{x} & & otherwise \\
%     \end{array}
%   \right. 
  (\dropn{x})  \psubstp{Q}{P}       
  := 
  \left\{ 
    \begin{array}{ccc} 
      Q & & x \nameeq \quotep{P} \\
      \dropn{x} & & otherwise \\
    \end{array}
  \right.
\end{mathpar}
 

where

\begin{eqnarray}
  (x)\id{\{} \lpquote Q \rpquote / \lpquote P \rpquote \id{\}}            = 
  \left\{ 
    \begin{array}{ccc}
      \lpquote Q \rpquote & & x \nameeq \lpquote P \rpquote \\
      x & & otherwise \\
    \end{array}
  \right. \nonumber
\end{eqnarray}

and $z$ is chosen distinct from $\quotep{P}$, $\quotep{Q}$, the free
names in $Q$, and all the names in $R$. Our $\alpha$-equivalence will
be built in the standard way from this substitution.

\begin{remark}\label{rem:no_self_referential_names}
  One consequence of these definitions is that $\forall P. \quotep{P}
  \not\in \freenames{P}$.
\end{remark}

\subsection{ Dynamic quote: an example }

Anticipating something of what's to come, consider applying the
substitution, $\widehat{\id{\{}u / z \id{\}}}$, to the following pair
of processes, $\lift{w}{y!(z)}$ and $w[ \lpquote y!(z) \rpquote ]$.

\begin{eqnarray}
	\lift{w}{y!(z)}\widehat{\id{\{}u / z \id{\}}}
		& = &
		\lift{w}{y!(u)} \nonumber\\
	w[ \lpquote y!(z) \rpquote ] \widehat{ \id{\{}u / z \id{\}} }
		& = &
		w[ \lpquote y!(z) \rpquote ] \nonumber
\end{eqnarray}

Because the body of the process between quotes is impervious to
substitution, we get radically different answers. In fact, by
examining the first process in an input context,
e.g. $x?(z).\lift{w}{y!(z)}$, we see that the process under the lift
operator may be shaped by prefixed inputs binding a name inside it. In
this sense, the lift operator will be seen as a way to dynamically
construct processes before reifying them as names.

Finally equipped with these standard features we can present the
dynamics of the calculus.

\subsubsection{Operational semantics} 

Finally, we introduce the computational dynamics. What marks these
algebras as distinct from other more traditionally studied algebraic
structures, e.g. vector spaces or polynomial rings, is the manner in
which dynamics is captured. In traditional structures, dynamics is typically
expressed through morphisms between such structures, as in linear maps
between vector spaces or morphisms between rings. In algebras
associated with the semantics of computation, the dynamics is
expressed as part of the algebraic structure itself, through a
reduction reduction relation typically denoted by $\red$. Below, we
give a recursive presentation of this relation for the calculus used
in the encoding.

$\red \subseteq \pi \times \pi$
$\red : \pi \to \mathcal{P}(\pi)$

\begin{mathpar}
  \inferrule* [lab=Comm] { \textsf{match}( x_{src}, x_{trgt} ) } { x_{trgt}?(y)P \; | \; x_{src}!\langle {Q} \rangle \red P\{\quotep{Q}/y}\} }
  \and \\
  \inferrule* [lab=Par] {{P} \red {P}'} {{{P} | {Q}} \red {{P}' | {Q}}}
  \and
  \inferrule* [lab=Equiv]{{{P} \scong {P}'} \andalso {{P}' \red {Q}'} \andalso {{Q}' \scong {Q}}}{{P} \red {Q}}
\end{mathpar}

\begin{eqnarray*}
  match_{\equiv} (\quotep{P},\quotep{Q}) & := & P \equiv Q \\
  match_{\dagger}(\quotep{P},\quotep{Q}) & := & \forall R. P|Q \red^{*} R => R \red^{*} 0 \\
  match_{K}(\quotep{P},\quotep{Q}) & := & K \mbox{ for some context } K
\end{eqnarray*}

$u?(x)P | u!\langle Q \rangle \red P\{\quotep{Q}/x\}$

%We write $\wred$ for $\red^*$, and $P\red$ if $\exists Q $ such that $ P \red Q$.
We write $P\red$ if $\exists Q $ such that $ P \red Q$ and $P\not\red$, otherwise.

\section{Replication}

As mentioned before, it is known that replication (and hence
recursion) can be implemented in a higher-order process algebra
\cite{SangiorgiWalker}. As our first example of calculation with the
machinery thus far presented we give the construction explicitly in
the {\rhoc}.

\begin{eqnarray}
	D_{x} & := & \prefix{x}{y}{(\binpar{\outputp{x}{y}}{@{y}})} \nonumber\\
	\bangp_{x}{P} & := & \binpar{{x}!\langle{\binpar{D_{x}}{P}}\rangle}{D_{x}} \nonumber
\end{eqnarray}

\begin{eqnarray}
	\bangp_{x}{P} & & \nonumber\\
	=
	& {x}!\langle{(\prefix{x}{y}{(\outputp{x}{y} | @{y})) | P}}\rangle 
	      | \prefix{x}{y}{(\outputp{x}{y} | @{y})} & \nonumber\\
	\red
	& (\outputp{x}{y} | @{y})\substn{\quotep{(\prefix{x}{y}{(@{y} | \outputp{x}{y})) | P}}}{y} & \nonumber\\
	=
	& \outputp{x}{\quotep{(\prefix{x}{y}{(\outputp{x}{y} | @{y})) | P}}}
	  | {(\prefix{x}{y}{(\outputp{x}{y} | @{y})) | P}} & \nonumber\\
	\red
	& \ldots & \nonumber\\
	\red^*
	& P | P | \ldots & \nonumber
\end{eqnarray}

Of course, this encoding, as an implementation, runs away, unfolding
$\bangp{P}$ eagerly. A lazier and more implementable replication
operator, restricted to input-guarded processes, may be obtained as follows.

\begin{eqnarray}
\bangp{\prefix{u}{v}{P}} 
	:= 
	\binpar{\lift{x}{\prefix{u}{v}{(\binpar{D(x)}{P})}}}{D(x)} \nonumber
\end{eqnarray}

\begin{remark}
  Note that the lazier definition still does not deal with summation
  or mixed summation (i.e. sums over input and output). The reader is
  invited to construct definitions of replication that deal with these
  features. 

  Further, the definitions are parameterized in a name, $x$. Can you,
  gentle reader, make a definition that eliminates this parameter and
  guarantees no accidental interaction between the replication
  machinery and the process being replicated -- i.e. no accidental
  sharing of names used by the process to get its work done and the
  name(s) used by the replication to effect copying. This latter
  revision of the definition of replication is crucial to obtaining
  the expected identity $!!P \sim !P$.
\end{remark}

\begin{remark}\label{rem:paradoxical_combinator}
  The reader familiar with the lambda calculus will have noticed the
  similarity between $D$ and the paradoxical combinator.

  [Ed. note: the existence of this seems to suggest we have to be more
  restrictive on the set of processes and names we admit if we are to
  support no-cloning.]
\end{remark}

\subsubsection{Bisimulation}

The computational dynamics gives rise to another kind of equivalence,
the equivalence of computational behavior. As previously mentioned
this is typically captured \emph{via} some form of bisimulation.

% The notion we use in this paper is weak barbed bisimulation
% \cite{milner91polyadicpi}.

The notion we use in this paper is derived from weak barbed
bisimulation \cite{milner91polyadicpi}. 

\begin{definition}
An \emph{observation relation}, $\downarrow_{\mathcal N}$, over a set
of names, $\mathcal N$, is the smallest relation satisfying the rules
below.

\infrule[Out-barb]{y \in {\mathcal N}, \; x \nameeq y}
		  {\outputp{x}{v} \downarrow_{\mathcal N} x}
\infrule[Par-barb]{\mbox{$P\downarrow_{\mathcal N} x$ or $Q\downarrow_{\mathcal N} x$}}
		  {\binpar{P}{Q} \downarrow_{\mathcal N} x}

We write $P \Downarrow_{\mathcal N} x$ if there is $Q$ such that 
$P \wred Q$ and $Q \downarrow_{\mathcal N} x$.
\end{definition}

\begin{definition}
%\label{def.bbisim}
An  ${\mathcal N}$-\emph{barbed bisimulation} over a set of names, ${\mathcal N}$, is a symmetric binary relation 
${\mathcal S}_{\mathcal N}$ between agents such that $P\rel{S}_{\mathcal N}Q$ implies:
\begin{enumerate}
\item If $P \red P'$ then $Q \wred Q'$ and $P'\rel{S}_{\mathcal N} Q'$.
\item If $P\downarrow_{\mathcal N} x$, then $Q\Downarrow_{\mathcal N} x$.
\end{enumerate}
$P$ is ${\mathcal N}$-barbed bisimilar to $Q$, written
$P \wbbisim_{\mathcal N} Q$, if $P \rel{S}_{\mathcal N} Q$ for some ${\mathcal N}$-barbed bisimulation ${\mathcal S}_{\mathcal N}$.
\end{definition}

$\mathcal{R} \subseteq \pi \times \pi$

$P \mathcal{R} Q => \forall P'. P \red P' \Rightarrow \exists Q'. Q \red Q', P' \mathcal{R} Q'$

$P \vdash x \Rightarrow Q \vdash x$

\begin{mathpar}
  \inferrule*[lab=Out-barb]{x \nameeq y}{{y}!\langle{Q}\rangle \vdash x}
  \and
  \inferrule*[lab=Par-barb]{\mbox{$P\vdash x$ or $Q\vdash x$}}{\binpar{P}{Q} \vdash x}
\end{mathpar}

\subsubsection{Contexts}

One of the principle advantages of computational calculi like the
$\pi$-calculus is a well-defined notion of context,
contextual-equivalence and a correlation between
contextual-equivalence and notions of bisimulation. The notion of
context allows the decomposition of a process into (sub-)process and
its syntactic environment, its context. Thus, a context may be
thought of as a process with a ``hole'' (written $\Box$) in it. The
application of a context $M$ to a process $P$, written $M[P]$, is
tantamount to filling the hole in $M$ with $P$. In this paper we do
not need the full weight of this theory, but do make use of the notion
of context in the proof the main theorem. 

\begin{mathpar}
  \inferrule* [lab=summation] {} {{M_{M},M_{N}} \bc \Box \;|\; x.M_{A} \;|\; M_{M}+M_{N}}
  \and
  \inferrule* [lab=agent] {} {{M_{A}} \bc (\vec{x})M_{P} \;| \; \clift{P_0,\ldots,M_{P},\ldots,P_N}}
  \and \\
  \inferrule* [lab=process] {} {{M_{P}} \bc M_{N} \;| \;P|M_{P} }
\end{mathpar} 

\begin{mathpar}
  \inferrule* [lab=sychronization] {} {M_{N} \bc \Box \;|\; x?M_{F} \;|\; x!M_{C}}
  \and
  \inferrule* [lab=abstraction] {} {{M_{F}} \bc (x)M_{P} }
  \and
  \inferrule* [lab=concretion] {} {{M_{C}} \bc \langle M_{P} \rangle }
  \and \\
  \inferrule* [lab=process] {} {{M_{P}} \bc M_{N} \;| \;P|M_{P} }
\end{mathpar}

\begin{definition}[contextual application] Given a context $M$, and
  process $P$, we define the \emph{contextual application}, $M[P] :=
  M\{P/\Box\}$. That is, the contextual application of M to P is the
  substitution of $P$ for $\Box$ in $M$.
\end{definition}

$\meaningof{-} : L \to \mathcal{P}(\pi)$

\begin{mathpar}
  \inferrule* [lab=collection] {} {\meaningof{true} = \pi, \and \meaningof{~E} = \pi \setminus \meaningof{E}, \and \meaningof{E_{1} \& E_{2}} = \meaningof{E_{1}} \cap \meaningof{E_{2}}}
\end{mathpar}

\begin{mathpar}
  \inferrule* [lab=structure] {} {\meaningof{0} = \{ P \in \pi | P \equiv 0 \}, \and \\ \meaningof{E_1 | E_2} = \{ P \in \pi | P \equiv P_{1} | P_{2}, P_{1} \in \meaningof{E_{1}}, P_{2} \in \meaningof{E_2}\} }
\end{mathpar}

\begin{mathpar}
 \inferrule* [lab=behavior] {} {\meaningof{\langle a?b \rangle E} = \{ P \in \pi | P \equiv Q | u?(y)P', \\ \and \\\\ \and \\ \;\;\; u \in \meaningof{a}, \forall z.P'\{z/y\} \in \meaningof{E\{z/b\}}\}, \and \\ \meaningof{a!E} = \{ P \in \pi | P \equiv Q | x!\langle P' \rangle, x \in \meaningof{a} P' \in \meaningof{E}\} }
\end{mathpar}

\begin{mathpar}
 \inferrule* [lab=nominal] {} {\meaningof{\quotep{E}} = \{ \quotep{P} \in \quotep{\pi} | P \in \meaningof{E} \}, \and \meaningof{\quotep{P}} = \{ \quotep{Q} \in \quotep{\pi} | P \equiv Q \} \and \\ \meaningof{@\quotep{E}} = \{ P \in \pi | P \equiv @x, x \in \meaningof{E} \}}
\end{mathpar}

\begin{eqnarray*}
  \\
  \meaningof{-} : TS \to ST
\end{eqnarray*}

\begin{eqnarray*}
  \\
  L : TS \to ST
\end{eqnarray*}

\begin{eqnarray*}
  \\
  P \models E \iff P \in \meaningof{E}
\end{eqnarray*}

\begin{eqnarray*}
  P \approx_{L} Q \iff \forall E \in L. P \models E \iff Q \models E
\end{eqnarray*}

\begin{eqnarray*}
  P \approx_{K} Q
\end{eqnarray*}

\begin{eqnarray*}
  P \approx Q
\end{eqnarray*}

$\approx_{K} = \approx = \approx_{L}$

\subsubsection{Contextual duality}

Note that contexts extend the quotation operation to a family of
operations from processes to names. Given a context, $M$, we can
define a \emph{nominal context}, $\quotep{M}$ by $\quotep{M}[P] :=
\quotep{M[P]}$. To foreshadow what is to come we observe that these
operations enjoy a duality with processes very much like the duality
between vectors and maps from vectors to scalars.

Further, because the calculus is essentially higher-order, we have a
correspondence between contexts and processes. More specifically,
given a name $x$ and a context $M$ we can construct $M^{*}_{x}$ such
that 

\begin{mathpar}
  M^{*}_{x} | \lift{x}{P} \red M[P]
\end{mathpar}

namely,

\begin{mathpar}
  M^{*}_{x} := x?(u).M[\dropn{u}]
\end{mathpar}

The dependence of $M^{*}_{x}$ on a name makes it an abstraction, 

\begin{mathpar}
  M^{*} := (x)x?(u).M[\dropn{u}]
\end{mathpar}

\subsection{Additional notation}

It will sometimes be convenient to denote the process a name
quotes. We already have the notation $x = \quotep{P}$, but it will be
convenient to introduce an alternate notation, $\procn{x}$, when we
want to emphasize the connection to the use of the name. Note that, by
virtue of name equivalence, $\quotep{\procn{x}} \nameeq x$; so, the
notation is consistent with previous definitions.

Further, because names have structure it is possible to effect
substitutions on the basis of that structure. This means we need to
upgrade our notation for substitutions, which we accomplish by
adapting comprehension notation. Thus,

\begin{mathpar}
  P\{ y / x : x \in S \}
\end{mathpar}

is interpreted to mean the process derived from P by replacing (in a
capture-avoiding manner) each occurrence of $x$ in $S$ by $y$. For example,

\begin{mathpar}
  P\{ \quotep{\procn{x}|\procn{x}} / x : x \in \freenames{P} \}
\end{mathpar}

will replace each (occurrence) of a free name $x$ in $P$ by
$\quotep{\procn{x}|\procn{x}}$.

Also, we will avail ourselves of the notation $x^{L}$ and $x^{R}$ to
denote injections of a name into disjoint copies of the name
space. There are numerous ways to accomplish this. One example can be
found in \cite{MeredithR05}. This notation overloads to vectors of
names: $\vec{x}^{\pi} := (x_{i}^{\pi} \; : \; 0 \leq i < |\vec{x}| )$ where $\pi \in \{L,R\}$.

We also use $P^{\Box} := P|\Box$.

In \cite{MeredithR05} an interpretation of the new operator is
given. It turns out that there are several possible interpretations
all enjoying the requisite algebraic properties of the operator (see
\cite{milner91polyadicpi}). We will therefore make liberal use of
$(\nu\; \vec{x})P$.

% subsection the_syntax_and_semantics_of_the_notation_system (end)   

\input{qm2pi.qmops} 

\input{qm2pi.sterngerlach} 

\input{qm2pi.metric} 

% section concurrent_process_calculi (end)

%\input{qm2pi.proofsketch}

% section proof sketch (end)

%\input{qm2pi.slviaknots} 

% section spatial logic via knots (end)

\input{qm2pi.conclusion}

% section conclusion (end)

%\input{qm2pi.dtcodes} 

% section wiring algorithm (end)

\input{qm2pi.ack} 

% section acknowledgments (end)

\newpage


\bibliographystyle{plain}   
\bibliography{../../biblios/main.bib}

\input{qm2pi.rhodetails}

\end{document}

 

%\ifpdf
%\usepackage[pdftex]{graphicx}
%\else
%\usepackage{graphicx}
%\fi

 % \ifpdf
%  \usepackage{pdfsync}
%  \if


%\title{Brief Article}
%\author{David F. Snyder}
%\author{L.G. Meredith}

%\address{Dept. of Math., Texas State University--San Marcos, San Marcos, TX 78666}
       
\pagestyle{empty}


\begin{document}

\lstset{language=[Objective]Caml,frame=shadowbox}

\documentclass[12pt]{llncs}
%\documentclass{jktr}

\usepackage[pdftex]{hyperref}                   
\usepackage {listings}
\usepackage {mathpartir}
\usepackage{bcprules}
%\usepackage{listings}
                       
\usepackage{graphicx} 
%\usepackage[margins=2.5cm,nohead,nofoot]{geometry}
%\usepackage{geometry}
\usepackage{amsfonts}
\usepackage{amstext}
\usepackage{latexsym}
\usepackage{amssymb}
\usepackage{color}


%\include{myPreamble}
\include{qm2pi.local} 

%\ifpdf
%\usepackage[pdftex]{graphicx}
%\else
%\usepackage{graphicx}
%\fi

 % \ifpdf
%  \usepackage{pdfsync}
%  \if


%\title{Brief Article}
%\author{David F. Snyder}
%\author{L.G. Meredith}

%\address{Dept. of Math., Texas State University--San Marcos, San Marcos, TX 78666}
       
\pagestyle{empty}


\begin{document}

\lstset{language=[Objective]Caml,frame=shadowbox}

\input{qm2pi.front}

% section front matter (end)

\input{qm2pi.intro} 
 
% section introduction (end)

% \input{qm2pi.knotations} 

% section notation (end)

\input{qm2pi.process.calculi} 

% section concurrent_process_calculi_and_spatial_logics_ (end)
    
%\input{qm2pi.knots2pi} 

%\input{qm2pi.trefoil} 

%\input{qm2pi.mainthm} 

% subsection basic_interpretation (end)

%\input{qm2pi.rho.presentation} 
\subsection{The syntax and semantics of the notation system}\label{sub:the_syntax_and_semantics_of_the_notation_system} % (fold)

We now summarize a technical presentation of the calculus that
embodies our theory of dynamics. The typical presentation of such a
calculus follows the style of giving generators and relations on
them. The grammar, below, describing term constructors, freely
generates the set of processes, $\Proc$. This set is then quotiented
by a relation known as structural congruence and it is over this set
that the notion of dynamics is expressed. This presentation is
essentially that of \cite{MeredithR05} with the addition of
polyadicity and summation. For readability we have relegated some of
the technical subtleties to an appendix.

\subsubsection{Process grammar}\label{subsub:process_grammar}

\begin{mathpar}
  \inferrule* [lab=synchronization] {} {{M} \bc \pzero \;|\; x?F \;|\; x!C }
  \and
  \inferrule* [lab=abstraction] {} {{F} \bc (x)P}
  \and
  \inferrule* [lab=concretion] {} {{C} \bc \langle Q \rangle}
  \and
  \inferrule* [lab=process] {} {{P,Q} \bc M \;| \;P|Q \;|\; @{x}}
  \and
  \inferrule* [lab=name] {} {{x} \bc \quotep{P}}
\end{mathpar} 

Note that $\vec{x}$ (resp. $\vec{P}$) denotes a vector of names
(resp. processes) of length $|\vec{x}|$ (resp. $|\vec{P}|$). We adopt
the following useful abbreviations.

\begin{mathpar}
   x?(\vec{y}).P := x.(\vec{y})P \and  x\clift{\vec{P}} := x.\clift{\vec{P}}
   \and x!(y) := \lift{x}{\dropn{y}}
   \and \Pi_{i=0}^{n-1}P_i := P_0 | \ldots | P_{n-1}
\end{mathpar}

\subsubsection{Structural congruence}

\paragraph{Free and bound names and alpha-equivalence.} At the
core of structural equivalence is alpha-equivalence which identifies
process that are the same up to a change of variable. Formally, we
recognize the distinction between free and bound names. The free names
of a process, $\freenames{P}$, may be calculated recursively as
follows:

\begin{mathpar}
\freenames{\pzero} := \emptyset
  \and \\
  \freenames{x?(y).P} := \{ x \} \cup (\freenames{P} \setminus \{ y \})
  \and 
  \freenames{x!\langle P \rangle} := \{ x \} \cup \{ P \} 
  \and \\
  \freenames{P|Q} := \freenames{P} \cup \freenames{Q}
  \and \\
  \freenames{@{x}} := \{ x \}
\end{mathpar}

$\pi$
$\quotep{\pi}$

$\freenames{-} : \pi \to \mathcal{P}(\quotep{\pi})$

\begin{eqnarray*}
  \freenames{\pzero} & := & \emptyset \\
  \freenames{x?(y).P} & := & \{ x \} \cup (\freenames{P} \setminus \{ y \}) \\
  \freenames{x!\langle P \rangle} & := & \{ x \} \cup \{ P \} \\
  \freenames{P|Q} & := & \freenames{P} \cup \freenames{Q} \\
  \freenames{\dropn{x}} & := & \{ x \}
\end{eqnarray*}

The bound names of a process, $\boundnames{P}$, are those names occurring in $P$
that are not free. For example, in $x?(y).0$, the name $x$ is free, while $y$ is bound.

\begin{mathpar}
  \inferrule* [lab=monoidal-laws] {} { P|Q \equiv Q|P \and P|0 \equiv P \and P|(Q|R) \equiv (P|Q)|R }
\end{mathpar}

\begin{mathpar}
  \inferrule* [lab=alpha-equivalence] {} { (x)P \equiv (y)P\{y/x\} \and y \not\in \freenames{P} }
\end{mathpar}

\begin{definition}
Then two processes, $P,Q$, are alpha-equivalent if $P = Q\{\vec{y}/\vec{x}\}$ for
some $\vec{x} \in \boundnames{Q},\vec{y} \in \boundnames{P}$, where $Q\{\vec{y}/\vec{x}\}$
denotes the capture-avoiding substitution of $\vec{y}$ for $\vec{x}$ in $Q$.
\end{definition}

\begin{definition}
  The {\em structural congruence} \cite{SangiorgiWalker} , $\equiv$,
  between processes is the least congruence containing
  alpha-equivalence, satisfying the abelian monoid laws
  (associativity, commutativity and $\pzero$ as identity) for parallel
  composition $|$ and for summation $+$.
\end{definition}

\subsection{Name equivalence}

We take name equivalence, written $\nameeq$, to be the smallest
equivalence relation generated by the following rules.

\begin{mathpar}
\inferrule*[lab=Quote-drop]
{ }
{ \quotep{@{x}} \nameeq x }

\inferrule*[lab=Struct-equiv]
{ P \scong Q }
{ \quotep{P} \nameeq \quotep{Q} }
\end{mathpar}

The astute reader will have noticed that the mutual recursion of names
and processes imposes a mutual recursion on alpha-equivalence and
structural equivalence via name-equivalence. Fortunately, all of this
works out pleasantly and we may calculate in the natural way, free of
concern. The reader interested in the details is referred to the
appendix \ref{appendix:rho_details}.

\subsection{Substitution}

We use $\Proc$ for the set of processes, $\QProc$ for the set of
names, and $\id{\{}\vec{y} / \vec{x} \id{\}}$ to denote partial maps,
$s : \QProc \rightarrow \QProc$. A map, $s$ lifts, uniquely, to a map
on process terms, $\widehat{s} : \Proc \rightarrow \Proc$ by the
following equations.

\begin{mathpar}
  (0) \psubstp{Q}{P} := 0 \\
  (R \juxtap S) \psubstp{Q}{P}
  :=    
  (R)\psubstp{Q}{P} \juxtap (S) \psubstp{Q}{P} \\
  (x?(y).R) \psubstp{Q}{P}    
  :=    
  (x)\substp{Q}{P} (z)\concat( (R \psubstn{z}{y}) \psubstp{Q}{P} ) \\
  (\lift{x}{R}) \psubstp{Q}{P}  
  :=
  \lift{(x)\substp{Q}{P}}{ R \psubstp{Q}{P} } \\
%   (\dropn{x})  \psubstp{Q}{P}       
%   := 
%   \left\{ 
%     \begin{array}{ccc} 
%       \dropn{\quotep{Q}} & & x \nameeq \quotep{P} \\
%       \dropn{x} & & otherwise \\
%     \end{array}
%   \right. 
  (\dropn{x})  \psubstp{Q}{P}       
  := 
  \left\{ 
    \begin{array}{ccc} 
      Q & & x \nameeq \quotep{P} \\
      \dropn{x} & & otherwise \\
    \end{array}
  \right.
\end{mathpar}
 

where

\begin{eqnarray}
  (x)\id{\{} \lpquote Q \rpquote / \lpquote P \rpquote \id{\}}            = 
  \left\{ 
    \begin{array}{ccc}
      \lpquote Q \rpquote & & x \nameeq \lpquote P \rpquote \\
      x & & otherwise \\
    \end{array}
  \right. \nonumber
\end{eqnarray}

and $z$ is chosen distinct from $\quotep{P}$, $\quotep{Q}$, the free
names in $Q$, and all the names in $R$. Our $\alpha$-equivalence will
be built in the standard way from this substitution.

\begin{remark}\label{rem:no_self_referential_names}
  One consequence of these definitions is that $\forall P. \quotep{P}
  \not\in \freenames{P}$.
\end{remark}

\subsection{ Dynamic quote: an example }

Anticipating something of what's to come, consider applying the
substitution, $\widehat{\id{\{}u / z \id{\}}}$, to the following pair
of processes, $\lift{w}{y!(z)}$ and $w[ \lpquote y!(z) \rpquote ]$.

\begin{eqnarray}
	\lift{w}{y!(z)}\widehat{\id{\{}u / z \id{\}}}
		& = &
		\lift{w}{y!(u)} \nonumber\\
	w[ \lpquote y!(z) \rpquote ] \widehat{ \id{\{}u / z \id{\}} }
		& = &
		w[ \lpquote y!(z) \rpquote ] \nonumber
\end{eqnarray}

Because the body of the process between quotes is impervious to
substitution, we get radically different answers. In fact, by
examining the first process in an input context,
e.g. $x?(z).\lift{w}{y!(z)}$, we see that the process under the lift
operator may be shaped by prefixed inputs binding a name inside it. In
this sense, the lift operator will be seen as a way to dynamically
construct processes before reifying them as names.

Finally equipped with these standard features we can present the
dynamics of the calculus.

\subsubsection{Operational semantics} 

Finally, we introduce the computational dynamics. What marks these
algebras as distinct from other more traditionally studied algebraic
structures, e.g. vector spaces or polynomial rings, is the manner in
which dynamics is captured. In traditional structures, dynamics is typically
expressed through morphisms between such structures, as in linear maps
between vector spaces or morphisms between rings. In algebras
associated with the semantics of computation, the dynamics is
expressed as part of the algebraic structure itself, through a
reduction reduction relation typically denoted by $\red$. Below, we
give a recursive presentation of this relation for the calculus used
in the encoding.

$\red \subseteq \pi \times \pi$
$\red : \pi \to \mathcal{P}(\pi)$

\begin{mathpar}
  \inferrule* [lab=Comm] { \textsf{match}( x_{src}, x_{trgt} ) } { x_{trgt}?(y)P \; | \; x_{src}!\langle {Q} \rangle \red P\{\quotep{Q}/y}\} }
  \and \\
  \inferrule* [lab=Par] {{P} \red {P}'} {{{P} | {Q}} \red {{P}' | {Q}}}
  \and
  \inferrule* [lab=Equiv]{{{P} \scong {P}'} \andalso {{P}' \red {Q}'} \andalso {{Q}' \scong {Q}}}{{P} \red {Q}}
\end{mathpar}

\begin{eqnarray*}
  match_{\equiv} (\quotep{P},\quotep{Q}) & := & P \equiv Q \\
  match_{\dagger}(\quotep{P},\quotep{Q}) & := & \forall R. P|Q \red^{*} R => R \red^{*} 0 \\
  match_{K}(\quotep{P},\quotep{Q}) & := & K \mbox{ for some context } K
\end{eqnarray*}

$u?(x)P | u!\langle Q \rangle \red P\{\quotep{Q}/x\}$

%We write $\wred$ for $\red^*$, and $P\red$ if $\exists Q $ such that $ P \red Q$.
We write $P\red$ if $\exists Q $ such that $ P \red Q$ and $P\not\red$, otherwise.

\section{Replication}

As mentioned before, it is known that replication (and hence
recursion) can be implemented in a higher-order process algebra
\cite{SangiorgiWalker}. As our first example of calculation with the
machinery thus far presented we give the construction explicitly in
the {\rhoc}.

\begin{eqnarray}
	D_{x} & := & \prefix{x}{y}{(\binpar{\outputp{x}{y}}{@{y}})} \nonumber\\
	\bangp_{x}{P} & := & \binpar{{x}!\langle{\binpar{D_{x}}{P}}\rangle}{D_{x}} \nonumber
\end{eqnarray}

\begin{eqnarray}
	\bangp_{x}{P} & & \nonumber\\
	=
	& {x}!\langle{(\prefix{x}{y}{(\outputp{x}{y} | @{y})) | P}}\rangle 
	      | \prefix{x}{y}{(\outputp{x}{y} | @{y})} & \nonumber\\
	\red
	& (\outputp{x}{y} | @{y})\substn{\quotep{(\prefix{x}{y}{(@{y} | \outputp{x}{y})) | P}}}{y} & \nonumber\\
	=
	& \outputp{x}{\quotep{(\prefix{x}{y}{(\outputp{x}{y} | @{y})) | P}}}
	  | {(\prefix{x}{y}{(\outputp{x}{y} | @{y})) | P}} & \nonumber\\
	\red
	& \ldots & \nonumber\\
	\red^*
	& P | P | \ldots & \nonumber
\end{eqnarray}

Of course, this encoding, as an implementation, runs away, unfolding
$\bangp{P}$ eagerly. A lazier and more implementable replication
operator, restricted to input-guarded processes, may be obtained as follows.

\begin{eqnarray}
\bangp{\prefix{u}{v}{P}} 
	:= 
	\binpar{\lift{x}{\prefix{u}{v}{(\binpar{D(x)}{P})}}}{D(x)} \nonumber
\end{eqnarray}

\begin{remark}
  Note that the lazier definition still does not deal with summation
  or mixed summation (i.e. sums over input and output). The reader is
  invited to construct definitions of replication that deal with these
  features. 

  Further, the definitions are parameterized in a name, $x$. Can you,
  gentle reader, make a definition that eliminates this parameter and
  guarantees no accidental interaction between the replication
  machinery and the process being replicated -- i.e. no accidental
  sharing of names used by the process to get its work done and the
  name(s) used by the replication to effect copying. This latter
  revision of the definition of replication is crucial to obtaining
  the expected identity $!!P \sim !P$.
\end{remark}

\begin{remark}\label{rem:paradoxical_combinator}
  The reader familiar with the lambda calculus will have noticed the
  similarity between $D$ and the paradoxical combinator.

  [Ed. note: the existence of this seems to suggest we have to be more
  restrictive on the set of processes and names we admit if we are to
  support no-cloning.]
\end{remark}

\subsubsection{Bisimulation}

The computational dynamics gives rise to another kind of equivalence,
the equivalence of computational behavior. As previously mentioned
this is typically captured \emph{via} some form of bisimulation.

% The notion we use in this paper is weak barbed bisimulation
% \cite{milner91polyadicpi}.

The notion we use in this paper is derived from weak barbed
bisimulation \cite{milner91polyadicpi}. 

\begin{definition}
An \emph{observation relation}, $\downarrow_{\mathcal N}$, over a set
of names, $\mathcal N$, is the smallest relation satisfying the rules
below.

\infrule[Out-barb]{y \in {\mathcal N}, \; x \nameeq y}
		  {\outputp{x}{v} \downarrow_{\mathcal N} x}
\infrule[Par-barb]{\mbox{$P\downarrow_{\mathcal N} x$ or $Q\downarrow_{\mathcal N} x$}}
		  {\binpar{P}{Q} \downarrow_{\mathcal N} x}

We write $P \Downarrow_{\mathcal N} x$ if there is $Q$ such that 
$P \wred Q$ and $Q \downarrow_{\mathcal N} x$.
\end{definition}

\begin{definition}
%\label{def.bbisim}
An  ${\mathcal N}$-\emph{barbed bisimulation} over a set of names, ${\mathcal N}$, is a symmetric binary relation 
${\mathcal S}_{\mathcal N}$ between agents such that $P\rel{S}_{\mathcal N}Q$ implies:
\begin{enumerate}
\item If $P \red P'$ then $Q \wred Q'$ and $P'\rel{S}_{\mathcal N} Q'$.
\item If $P\downarrow_{\mathcal N} x$, then $Q\Downarrow_{\mathcal N} x$.
\end{enumerate}
$P$ is ${\mathcal N}$-barbed bisimilar to $Q$, written
$P \wbbisim_{\mathcal N} Q$, if $P \rel{S}_{\mathcal N} Q$ for some ${\mathcal N}$-barbed bisimulation ${\mathcal S}_{\mathcal N}$.
\end{definition}

$\mathcal{R} \subseteq \pi \times \pi$

$P \mathcal{R} Q => \forall P'. P \red P' \Rightarrow \exists Q'. Q \red Q', P' \mathcal{R} Q'$

$P \vdash x \Rightarrow Q \vdash x$

\begin{mathpar}
  \inferrule*[lab=Out-barb]{x \nameeq y}{{y}!\langle{Q}\rangle \vdash x}
  \and
  \inferrule*[lab=Par-barb]{\mbox{$P\vdash x$ or $Q\vdash x$}}{\binpar{P}{Q} \vdash x}
\end{mathpar}

\subsubsection{Contexts}

One of the principle advantages of computational calculi like the
$\pi$-calculus is a well-defined notion of context,
contextual-equivalence and a correlation between
contextual-equivalence and notions of bisimulation. The notion of
context allows the decomposition of a process into (sub-)process and
its syntactic environment, its context. Thus, a context may be
thought of as a process with a ``hole'' (written $\Box$) in it. The
application of a context $M$ to a process $P$, written $M[P]$, is
tantamount to filling the hole in $M$ with $P$. In this paper we do
not need the full weight of this theory, but do make use of the notion
of context in the proof the main theorem. 

\begin{mathpar}
  \inferrule* [lab=summation] {} {{M_{M},M_{N}} \bc \Box \;|\; x.M_{A} \;|\; M_{M}+M_{N}}
  \and
  \inferrule* [lab=agent] {} {{M_{A}} \bc (\vec{x})M_{P} \;| \; \clift{P_0,\ldots,M_{P},\ldots,P_N}}
  \and \\
  \inferrule* [lab=process] {} {{M_{P}} \bc M_{N} \;| \;P|M_{P} }
\end{mathpar} 

\begin{mathpar}
  \inferrule* [lab=sychronization] {} {M_{N} \bc \Box \;|\; x?M_{F} \;|\; x!M_{C}}
  \and
  \inferrule* [lab=abstraction] {} {{M_{F}} \bc (x)M_{P} }
  \and
  \inferrule* [lab=concretion] {} {{M_{C}} \bc \langle M_{P} \rangle }
  \and \\
  \inferrule* [lab=process] {} {{M_{P}} \bc M_{N} \;| \;P|M_{P} }
\end{mathpar}

\begin{definition}[contextual application] Given a context $M$, and
  process $P$, we define the \emph{contextual application}, $M[P] :=
  M\{P/\Box\}$. That is, the contextual application of M to P is the
  substitution of $P$ for $\Box$ in $M$.
\end{definition}

$\meaningof{-} : L \to \mathcal{P}(\pi)$

\begin{mathpar}
  \inferrule* [lab=collection] {} {\meaningof{true} = \pi, \and \meaningof{~E} = \pi \setminus \meaningof{E}, \and \meaningof{E_{1} \& E_{2}} = \meaningof{E_{1}} \cap \meaningof{E_{2}}}
\end{mathpar}

\begin{mathpar}
  \inferrule* [lab=structure] {} {\meaningof{0} = \{ P \in \pi | P \equiv 0 \}, \and \\ \meaningof{E_1 | E_2} = \{ P \in \pi | P \equiv P_{1} | P_{2}, P_{1} \in \meaningof{E_{1}}, P_{2} \in \meaningof{E_2}\} }
\end{mathpar}

\begin{mathpar}
 \inferrule* [lab=behavior] {} {\meaningof{\langle a?b \rangle E} = \{ P \in \pi | P \equiv Q | u?(y)P', \\ \and \\\\ \and \\ \;\;\; u \in \meaningof{a}, \forall z.P'\{z/y\} \in \meaningof{E\{z/b\}}\}, \and \\ \meaningof{a!E} = \{ P \in \pi | P \equiv Q | x!\langle P' \rangle, x \in \meaningof{a} P' \in \meaningof{E}\} }
\end{mathpar}

\begin{mathpar}
 \inferrule* [lab=nominal] {} {\meaningof{\quotep{E}} = \{ \quotep{P} \in \quotep{\pi} | P \in \meaningof{E} \}, \and \meaningof{\quotep{P}} = \{ \quotep{Q} \in \quotep{\pi} | P \equiv Q \} \and \\ \meaningof{@\quotep{E}} = \{ P \in \pi | P \equiv @x, x \in \meaningof{E} \}}
\end{mathpar}

\begin{eqnarray*}
  \\
  \meaningof{-} : TS \to ST
\end{eqnarray*}

\begin{eqnarray*}
  \\
  L : TS \to ST
\end{eqnarray*}

\begin{eqnarray*}
  \\
  P \models E \iff P \in \meaningof{E}
\end{eqnarray*}

\begin{eqnarray*}
  P \approx_{L} Q \iff \forall E \in L. P \models E \iff Q \models E
\end{eqnarray*}

\begin{eqnarray*}
  P \approx_{K} Q
\end{eqnarray*}

\begin{eqnarray*}
  P \approx Q
\end{eqnarray*}

$\approx_{K} = \approx = \approx_{L}$

\subsubsection{Contextual duality}

Note that contexts extend the quotation operation to a family of
operations from processes to names. Given a context, $M$, we can
define a \emph{nominal context}, $\quotep{M}$ by $\quotep{M}[P] :=
\quotep{M[P]}$. To foreshadow what is to come we observe that these
operations enjoy a duality with processes very much like the duality
between vectors and maps from vectors to scalars.

Further, because the calculus is essentially higher-order, we have a
correspondence between contexts and processes. More specifically,
given a name $x$ and a context $M$ we can construct $M^{*}_{x}$ such
that 

\begin{mathpar}
  M^{*}_{x} | \lift{x}{P} \red M[P]
\end{mathpar}

namely,

\begin{mathpar}
  M^{*}_{x} := x?(u).M[\dropn{u}]
\end{mathpar}

The dependence of $M^{*}_{x}$ on a name makes it an abstraction, 

\begin{mathpar}
  M^{*} := (x)x?(u).M[\dropn{u}]
\end{mathpar}

\subsection{Additional notation}

It will sometimes be convenient to denote the process a name
quotes. We already have the notation $x = \quotep{P}$, but it will be
convenient to introduce an alternate notation, $\procn{x}$, when we
want to emphasize the connection to the use of the name. Note that, by
virtue of name equivalence, $\quotep{\procn{x}} \nameeq x$; so, the
notation is consistent with previous definitions.

Further, because names have structure it is possible to effect
substitutions on the basis of that structure. This means we need to
upgrade our notation for substitutions, which we accomplish by
adapting comprehension notation. Thus,

\begin{mathpar}
  P\{ y / x : x \in S \}
\end{mathpar}

is interpreted to mean the process derived from P by replacing (in a
capture-avoiding manner) each occurrence of $x$ in $S$ by $y$. For example,

\begin{mathpar}
  P\{ \quotep{\procn{x}|\procn{x}} / x : x \in \freenames{P} \}
\end{mathpar}

will replace each (occurrence) of a free name $x$ in $P$ by
$\quotep{\procn{x}|\procn{x}}$.

Also, we will avail ourselves of the notation $x^{L}$ and $x^{R}$ to
denote injections of a name into disjoint copies of the name
space. There are numerous ways to accomplish this. One example can be
found in \cite{MeredithR05}. This notation overloads to vectors of
names: $\vec{x}^{\pi} := (x_{i}^{\pi} \; : \; 0 \leq i < |\vec{x}| )$ where $\pi \in \{L,R\}$.

We also use $P^{\Box} := P|\Box$.

In \cite{MeredithR05} an interpretation of the new operator is
given. It turns out that there are several possible interpretations
all enjoying the requisite algebraic properties of the operator (see
\cite{milner91polyadicpi}). We will therefore make liberal use of
$(\nu\; \vec{x})P$.

% subsection the_syntax_and_semantics_of_the_notation_system (end)   

\input{qm2pi.qmops} 

\input{qm2pi.sterngerlach} 

\input{qm2pi.metric} 

% section concurrent_process_calculi (end)

%\input{qm2pi.proofsketch}

% section proof sketch (end)

%\input{qm2pi.slviaknots} 

% section spatial logic via knots (end)

\input{qm2pi.conclusion}

% section conclusion (end)

%\input{qm2pi.dtcodes} 

% section wiring algorithm (end)

\input{qm2pi.ack} 

% section acknowledgments (end)

\newpage


\bibliographystyle{plain}   
\bibliography{../../biblios/main.bib}

\input{qm2pi.rhodetails}

\end{document}



% section front matter (end)

\section{Introduction}\label{sec:introduction} % (fold)
In this draft of the material i am going to have to dispense with the
usual writing conventions adopted in papers on these topics. i'm going
to have adopt whatever tone i need at the time i'm writing up the
calculations. Sometimes this may be very conversational; others it may
be the barest mathematical grunts; others still it may be that i have
lifted text from one of my other papers because the exposition of some
point was better said there. i hope that my readers are not unduly put
out by this decision. i'm not doing this to flout convention or be
rebellious. i find these calculations very technically challenging. To
keep everything going technically, something has to give; i have to
let go of some cognitive burden. So, the academic writing style --
with all of its trade-offs in terms of facilitating technical
communication -- is what i'm letting go of. Perhaps subsequent drafts
can be tightened and polished, but for now, i'm going to speak as if
we were sitting together in a coffee shop with a laptop, wifi and a
pad of paper and a pencil.

So, here's what i have to say. We -- you and i, comfortably ensconced
in our coffee shop and well-equipped with our tools -- can realize and
carry out the calculations of quantum mechanics over a very different
formal theory of dynamics, a formal theory of dynamics that
corresponds to a theory of concurrent computation with
\emph{reflection}. It has the advantage that the underlying theory is
already `quantized', but supports analogues all of the continuuous
operations. Strikingly, this underlying theory has recently been
connected with a notion of metric that we can show, by calculating
together, coincides with the metric induced by the inner product.

There are a lot of reasons why you might be interested in seeing
calculations of this form. Here's why i'm interested. For the past
several centuries there has been no competitor to the ``Newtonian''
account of dynamics. As a result the predominant share of accounts of
dynamical systems and situations have had to be formulated in terms of
the Newtonian machinery. i view this as an intellectually dangerous
position to occupy. Everything, despite it's intrinsic shape, turns
into a nail to be hit with this hammer. Recently, however, the theory
of computation has matured to the point where we have candidates for
theories of dynamics that offer very different perspective on
reasoning about dynamical systems and situations. Testing these
candidates against very successful accounts of dynamical situations,
like quantum mechanics, is going to give us some sense of how mature
they are and some measure of the quality of these accounts of
dynamics.

\subsection{Summary of contributions and outline of paper}

So, we're going to develop an interpretation of the operations of
quantum mechanics normally interpreted by Hilbert spaces and
operators. We're going to do this over a theory of computation. Note
that this is very different than the usual quantum computation program
which develops notions of computation over quantum mechanics. Rather,
we are developing a story that aligns with Wheeler's slogan: It from
Bit. To do this we will first provide an account of the theory of
computation at play here. Then we will dive into a calculation-driven
interpretation of the operations of quantum mechanics.

The reason we take this approach is that -- until very recently --
there hasn't been an axiomatic account of quantum mechanics. As a
result there has been no sharp delineation of the mathematical theory
supporting interpretation of the physical theory and the physical
theory, itself. So, ambient features of the maths are free to be
exploited (or supressed) without a real accounting of their physical
relevance. There is no sharp statement ``here's the physical theory''
qua \emph{theory} and ``here's the mathematical interpretation''
enabling a judgment of how faithful the interpretation is -- apart
from experimental observation. When there is an axiomatic account we
can judge how well a given mathematical formalism supports an
interpretation of the axioms, independent of
experimentation. Likewise, we can judge how well we have captured our
physical evidence and experience with our axiomatics, independent of
any specific mathematical implementation, with accidental detail that
may or may not have physical significance. 

In lieu of a fully fleshed out and vetted axiomatic account of quantum
mechanics, interpreting the operational notions in service of modeling
physical systems will have to suffice. In other words, we are not in
the business of providing a model of Hilbert spaces and operators. We
are in the business of providing a model of quantum mechanics because
we are motivated by testing our notions of dynamics against physical
theory; and, the predictive calculations of the physical theory must
serve as the best formulation -- shy of a fully fleshed out axiomatic
account -- of the physical theory itself (as they have for scientific
theories since time immemorial). Put another way, despite a
whole-hearted commitment to an It-from-Bit ontology, we are firmly
aligned with the shut-up-and-calculate camp as the best way to obtain
results either from the physical perspective or as a quality assurance
measure of our fledgling theory of dynamics.

In detail, we present a reflective process calculus. Then we develop
intuitive correspondences between the notions available in this
calculus and the usual physical notions supporting quantum mechanical
calculations. Thus, 

\begin{table}[htp]
  \center{
    \fbox{
      \begin{tabular}{c|c}
        quantum mechanics & process calculus \\
        \hline
        scalar & name \\
        state vector & process \\
        dual & contextual duals \\
        matrix & formal sums of process-context-dual pairs \\
        orthogonality & process annihilation \\
        inner product & execution-formula + quoting
      \end{tabular}
    }
  }
  \caption{QM - process calculi correspondences}
\end{table}

Then we tighten up these intuitions to operational definitions. We
employ the Dirac notation as the best proxy we can find for an
abstract syntax of the quantum mechanical notions. The definitions we
develop put us in contact with equational constraints coming from the
theory that we demonstrate the definitions and calculations satisfy.

This puts us in a position to shut up and calculate for the
Stern-Gerlach experimental set up, showing how these predictive
calculations become calculations on processes in our theory of a
reflective process calculus.

Penultimately, we demonstrate that the notion of metric coming from
the inner product coincides with the notion of metric available from
the theory of bisimulation. This demonstration gives us the right to
think of space as arising from behavior. Finally, we consider where we
might go from the new vantage point we have obtained.

% section introduction (end) 
 
% section introduction (end)

% \documentclass[12pt]{llncs}
%\documentclass{jktr}

\usepackage[pdftex]{hyperref}                   
\usepackage {listings}
\usepackage {mathpartir}
\usepackage{bcprules}
%\usepackage{listings}
                       
\usepackage{graphicx} 
%\usepackage[margins=2.5cm,nohead,nofoot]{geometry}
%\usepackage{geometry}
\usepackage{amsfonts}
\usepackage{amstext}
\usepackage{latexsym}
\usepackage{amssymb}
\usepackage{color}


%\include{myPreamble}
\include{qm2pi.local} 

%\ifpdf
%\usepackage[pdftex]{graphicx}
%\else
%\usepackage{graphicx}
%\fi

 % \ifpdf
%  \usepackage{pdfsync}
%  \if


%\title{Brief Article}
%\author{David F. Snyder}
%\author{L.G. Meredith}

%\address{Dept. of Math., Texas State University--San Marcos, San Marcos, TX 78666}
       
\pagestyle{empty}


\begin{document}

\lstset{language=[Objective]Caml,frame=shadowbox}

\input{qm2pi.front}

% section front matter (end)

\input{qm2pi.intro} 
 
% section introduction (end)

% \input{qm2pi.knotations} 

% section notation (end)

\input{qm2pi.process.calculi} 

% section concurrent_process_calculi_and_spatial_logics_ (end)
    
%\input{qm2pi.knots2pi} 

%\input{qm2pi.trefoil} 

%\input{qm2pi.mainthm} 

% subsection basic_interpretation (end)

%\input{qm2pi.rho.presentation} 
\subsection{The syntax and semantics of the notation system}\label{sub:the_syntax_and_semantics_of_the_notation_system} % (fold)

We now summarize a technical presentation of the calculus that
embodies our theory of dynamics. The typical presentation of such a
calculus follows the style of giving generators and relations on
them. The grammar, below, describing term constructors, freely
generates the set of processes, $\Proc$. This set is then quotiented
by a relation known as structural congruence and it is over this set
that the notion of dynamics is expressed. This presentation is
essentially that of \cite{MeredithR05} with the addition of
polyadicity and summation. For readability we have relegated some of
the technical subtleties to an appendix.

\subsubsection{Process grammar}\label{subsub:process_grammar}

\begin{mathpar}
  \inferrule* [lab=synchronization] {} {{M} \bc \pzero \;|\; x?F \;|\; x!C }
  \and
  \inferrule* [lab=abstraction] {} {{F} \bc (x)P}
  \and
  \inferrule* [lab=concretion] {} {{C} \bc \langle Q \rangle}
  \and
  \inferrule* [lab=process] {} {{P,Q} \bc M \;| \;P|Q \;|\; @{x}}
  \and
  \inferrule* [lab=name] {} {{x} \bc \quotep{P}}
\end{mathpar} 

Note that $\vec{x}$ (resp. $\vec{P}$) denotes a vector of names
(resp. processes) of length $|\vec{x}|$ (resp. $|\vec{P}|$). We adopt
the following useful abbreviations.

\begin{mathpar}
   x?(\vec{y}).P := x.(\vec{y})P \and  x\clift{\vec{P}} := x.\clift{\vec{P}}
   \and x!(y) := \lift{x}{\dropn{y}}
   \and \Pi_{i=0}^{n-1}P_i := P_0 | \ldots | P_{n-1}
\end{mathpar}

\subsubsection{Structural congruence}

\paragraph{Free and bound names and alpha-equivalence.} At the
core of structural equivalence is alpha-equivalence which identifies
process that are the same up to a change of variable. Formally, we
recognize the distinction between free and bound names. The free names
of a process, $\freenames{P}$, may be calculated recursively as
follows:

\begin{mathpar}
\freenames{\pzero} := \emptyset
  \and \\
  \freenames{x?(y).P} := \{ x \} \cup (\freenames{P} \setminus \{ y \})
  \and 
  \freenames{x!\langle P \rangle} := \{ x \} \cup \{ P \} 
  \and \\
  \freenames{P|Q} := \freenames{P} \cup \freenames{Q}
  \and \\
  \freenames{@{x}} := \{ x \}
\end{mathpar}

$\pi$
$\quotep{\pi}$

$\freenames{-} : \pi \to \mathcal{P}(\quotep{\pi})$

\begin{eqnarray*}
  \freenames{\pzero} & := & \emptyset \\
  \freenames{x?(y).P} & := & \{ x \} \cup (\freenames{P} \setminus \{ y \}) \\
  \freenames{x!\langle P \rangle} & := & \{ x \} \cup \{ P \} \\
  \freenames{P|Q} & := & \freenames{P} \cup \freenames{Q} \\
  \freenames{\dropn{x}} & := & \{ x \}
\end{eqnarray*}

The bound names of a process, $\boundnames{P}$, are those names occurring in $P$
that are not free. For example, in $x?(y).0$, the name $x$ is free, while $y$ is bound.

\begin{mathpar}
  \inferrule* [lab=monoidal-laws] {} { P|Q \equiv Q|P \and P|0 \equiv P \and P|(Q|R) \equiv (P|Q)|R }
\end{mathpar}

\begin{mathpar}
  \inferrule* [lab=alpha-equivalence] {} { (x)P \equiv (y)P\{y/x\} \and y \not\in \freenames{P} }
\end{mathpar}

\begin{definition}
Then two processes, $P,Q$, are alpha-equivalent if $P = Q\{\vec{y}/\vec{x}\}$ for
some $\vec{x} \in \boundnames{Q},\vec{y} \in \boundnames{P}$, where $Q\{\vec{y}/\vec{x}\}$
denotes the capture-avoiding substitution of $\vec{y}$ for $\vec{x}$ in $Q$.
\end{definition}

\begin{definition}
  The {\em structural congruence} \cite{SangiorgiWalker} , $\equiv$,
  between processes is the least congruence containing
  alpha-equivalence, satisfying the abelian monoid laws
  (associativity, commutativity and $\pzero$ as identity) for parallel
  composition $|$ and for summation $+$.
\end{definition}

\subsection{Name equivalence}

We take name equivalence, written $\nameeq$, to be the smallest
equivalence relation generated by the following rules.

\begin{mathpar}
\inferrule*[lab=Quote-drop]
{ }
{ \quotep{@{x}} \nameeq x }

\inferrule*[lab=Struct-equiv]
{ P \scong Q }
{ \quotep{P} \nameeq \quotep{Q} }
\end{mathpar}

The astute reader will have noticed that the mutual recursion of names
and processes imposes a mutual recursion on alpha-equivalence and
structural equivalence via name-equivalence. Fortunately, all of this
works out pleasantly and we may calculate in the natural way, free of
concern. The reader interested in the details is referred to the
appendix \ref{appendix:rho_details}.

\subsection{Substitution}

We use $\Proc$ for the set of processes, $\QProc$ for the set of
names, and $\id{\{}\vec{y} / \vec{x} \id{\}}$ to denote partial maps,
$s : \QProc \rightarrow \QProc$. A map, $s$ lifts, uniquely, to a map
on process terms, $\widehat{s} : \Proc \rightarrow \Proc$ by the
following equations.

\begin{mathpar}
  (0) \psubstp{Q}{P} := 0 \\
  (R \juxtap S) \psubstp{Q}{P}
  :=    
  (R)\psubstp{Q}{P} \juxtap (S) \psubstp{Q}{P} \\
  (x?(y).R) \psubstp{Q}{P}    
  :=    
  (x)\substp{Q}{P} (z)\concat( (R \psubstn{z}{y}) \psubstp{Q}{P} ) \\
  (\lift{x}{R}) \psubstp{Q}{P}  
  :=
  \lift{(x)\substp{Q}{P}}{ R \psubstp{Q}{P} } \\
%   (\dropn{x})  \psubstp{Q}{P}       
%   := 
%   \left\{ 
%     \begin{array}{ccc} 
%       \dropn{\quotep{Q}} & & x \nameeq \quotep{P} \\
%       \dropn{x} & & otherwise \\
%     \end{array}
%   \right. 
  (\dropn{x})  \psubstp{Q}{P}       
  := 
  \left\{ 
    \begin{array}{ccc} 
      Q & & x \nameeq \quotep{P} \\
      \dropn{x} & & otherwise \\
    \end{array}
  \right.
\end{mathpar}
 

where

\begin{eqnarray}
  (x)\id{\{} \lpquote Q \rpquote / \lpquote P \rpquote \id{\}}            = 
  \left\{ 
    \begin{array}{ccc}
      \lpquote Q \rpquote & & x \nameeq \lpquote P \rpquote \\
      x & & otherwise \\
    \end{array}
  \right. \nonumber
\end{eqnarray}

and $z$ is chosen distinct from $\quotep{P}$, $\quotep{Q}$, the free
names in $Q$, and all the names in $R$. Our $\alpha$-equivalence will
be built in the standard way from this substitution.

\begin{remark}\label{rem:no_self_referential_names}
  One consequence of these definitions is that $\forall P. \quotep{P}
  \not\in \freenames{P}$.
\end{remark}

\subsection{ Dynamic quote: an example }

Anticipating something of what's to come, consider applying the
substitution, $\widehat{\id{\{}u / z \id{\}}}$, to the following pair
of processes, $\lift{w}{y!(z)}$ and $w[ \lpquote y!(z) \rpquote ]$.

\begin{eqnarray}
	\lift{w}{y!(z)}\widehat{\id{\{}u / z \id{\}}}
		& = &
		\lift{w}{y!(u)} \nonumber\\
	w[ \lpquote y!(z) \rpquote ] \widehat{ \id{\{}u / z \id{\}} }
		& = &
		w[ \lpquote y!(z) \rpquote ] \nonumber
\end{eqnarray}

Because the body of the process between quotes is impervious to
substitution, we get radically different answers. In fact, by
examining the first process in an input context,
e.g. $x?(z).\lift{w}{y!(z)}$, we see that the process under the lift
operator may be shaped by prefixed inputs binding a name inside it. In
this sense, the lift operator will be seen as a way to dynamically
construct processes before reifying them as names.

Finally equipped with these standard features we can present the
dynamics of the calculus.

\subsubsection{Operational semantics} 

Finally, we introduce the computational dynamics. What marks these
algebras as distinct from other more traditionally studied algebraic
structures, e.g. vector spaces or polynomial rings, is the manner in
which dynamics is captured. In traditional structures, dynamics is typically
expressed through morphisms between such structures, as in linear maps
between vector spaces or morphisms between rings. In algebras
associated with the semantics of computation, the dynamics is
expressed as part of the algebraic structure itself, through a
reduction reduction relation typically denoted by $\red$. Below, we
give a recursive presentation of this relation for the calculus used
in the encoding.

$\red \subseteq \pi \times \pi$
$\red : \pi \to \mathcal{P}(\pi)$

\begin{mathpar}
  \inferrule* [lab=Comm] { \textsf{match}( x_{src}, x_{trgt} ) } { x_{trgt}?(y)P \; | \; x_{src}!\langle {Q} \rangle \red P\{\quotep{Q}/y}\} }
  \and \\
  \inferrule* [lab=Par] {{P} \red {P}'} {{{P} | {Q}} \red {{P}' | {Q}}}
  \and
  \inferrule* [lab=Equiv]{{{P} \scong {P}'} \andalso {{P}' \red {Q}'} \andalso {{Q}' \scong {Q}}}{{P} \red {Q}}
\end{mathpar}

\begin{eqnarray*}
  match_{\equiv} (\quotep{P},\quotep{Q}) & := & P \equiv Q \\
  match_{\dagger}(\quotep{P},\quotep{Q}) & := & \forall R. P|Q \red^{*} R => R \red^{*} 0 \\
  match_{K}(\quotep{P},\quotep{Q}) & := & K \mbox{ for some context } K
\end{eqnarray*}

$u?(x)P | u!\langle Q \rangle \red P\{\quotep{Q}/x\}$

%We write $\wred$ for $\red^*$, and $P\red$ if $\exists Q $ such that $ P \red Q$.
We write $P\red$ if $\exists Q $ such that $ P \red Q$ and $P\not\red$, otherwise.

\section{Replication}

As mentioned before, it is known that replication (and hence
recursion) can be implemented in a higher-order process algebra
\cite{SangiorgiWalker}. As our first example of calculation with the
machinery thus far presented we give the construction explicitly in
the {\rhoc}.

\begin{eqnarray}
	D_{x} & := & \prefix{x}{y}{(\binpar{\outputp{x}{y}}{@{y}})} \nonumber\\
	\bangp_{x}{P} & := & \binpar{{x}!\langle{\binpar{D_{x}}{P}}\rangle}{D_{x}} \nonumber
\end{eqnarray}

\begin{eqnarray}
	\bangp_{x}{P} & & \nonumber\\
	=
	& {x}!\langle{(\prefix{x}{y}{(\outputp{x}{y} | @{y})) | P}}\rangle 
	      | \prefix{x}{y}{(\outputp{x}{y} | @{y})} & \nonumber\\
	\red
	& (\outputp{x}{y} | @{y})\substn{\quotep{(\prefix{x}{y}{(@{y} | \outputp{x}{y})) | P}}}{y} & \nonumber\\
	=
	& \outputp{x}{\quotep{(\prefix{x}{y}{(\outputp{x}{y} | @{y})) | P}}}
	  | {(\prefix{x}{y}{(\outputp{x}{y} | @{y})) | P}} & \nonumber\\
	\red
	& \ldots & \nonumber\\
	\red^*
	& P | P | \ldots & \nonumber
\end{eqnarray}

Of course, this encoding, as an implementation, runs away, unfolding
$\bangp{P}$ eagerly. A lazier and more implementable replication
operator, restricted to input-guarded processes, may be obtained as follows.

\begin{eqnarray}
\bangp{\prefix{u}{v}{P}} 
	:= 
	\binpar{\lift{x}{\prefix{u}{v}{(\binpar{D(x)}{P})}}}{D(x)} \nonumber
\end{eqnarray}

\begin{remark}
  Note that the lazier definition still does not deal with summation
  or mixed summation (i.e. sums over input and output). The reader is
  invited to construct definitions of replication that deal with these
  features. 

  Further, the definitions are parameterized in a name, $x$. Can you,
  gentle reader, make a definition that eliminates this parameter and
  guarantees no accidental interaction between the replication
  machinery and the process being replicated -- i.e. no accidental
  sharing of names used by the process to get its work done and the
  name(s) used by the replication to effect copying. This latter
  revision of the definition of replication is crucial to obtaining
  the expected identity $!!P \sim !P$.
\end{remark}

\begin{remark}\label{rem:paradoxical_combinator}
  The reader familiar with the lambda calculus will have noticed the
  similarity between $D$ and the paradoxical combinator.

  [Ed. note: the existence of this seems to suggest we have to be more
  restrictive on the set of processes and names we admit if we are to
  support no-cloning.]
\end{remark}

\subsubsection{Bisimulation}

The computational dynamics gives rise to another kind of equivalence,
the equivalence of computational behavior. As previously mentioned
this is typically captured \emph{via} some form of bisimulation.

% The notion we use in this paper is weak barbed bisimulation
% \cite{milner91polyadicpi}.

The notion we use in this paper is derived from weak barbed
bisimulation \cite{milner91polyadicpi}. 

\begin{definition}
An \emph{observation relation}, $\downarrow_{\mathcal N}$, over a set
of names, $\mathcal N$, is the smallest relation satisfying the rules
below.

\infrule[Out-barb]{y \in {\mathcal N}, \; x \nameeq y}
		  {\outputp{x}{v} \downarrow_{\mathcal N} x}
\infrule[Par-barb]{\mbox{$P\downarrow_{\mathcal N} x$ or $Q\downarrow_{\mathcal N} x$}}
		  {\binpar{P}{Q} \downarrow_{\mathcal N} x}

We write $P \Downarrow_{\mathcal N} x$ if there is $Q$ such that 
$P \wred Q$ and $Q \downarrow_{\mathcal N} x$.
\end{definition}

\begin{definition}
%\label{def.bbisim}
An  ${\mathcal N}$-\emph{barbed bisimulation} over a set of names, ${\mathcal N}$, is a symmetric binary relation 
${\mathcal S}_{\mathcal N}$ between agents such that $P\rel{S}_{\mathcal N}Q$ implies:
\begin{enumerate}
\item If $P \red P'$ then $Q \wred Q'$ and $P'\rel{S}_{\mathcal N} Q'$.
\item If $P\downarrow_{\mathcal N} x$, then $Q\Downarrow_{\mathcal N} x$.
\end{enumerate}
$P$ is ${\mathcal N}$-barbed bisimilar to $Q$, written
$P \wbbisim_{\mathcal N} Q$, if $P \rel{S}_{\mathcal N} Q$ for some ${\mathcal N}$-barbed bisimulation ${\mathcal S}_{\mathcal N}$.
\end{definition}

$\mathcal{R} \subseteq \pi \times \pi$

$P \mathcal{R} Q => \forall P'. P \red P' \Rightarrow \exists Q'. Q \red Q', P' \mathcal{R} Q'$

$P \vdash x \Rightarrow Q \vdash x$

\begin{mathpar}
  \inferrule*[lab=Out-barb]{x \nameeq y}{{y}!\langle{Q}\rangle \vdash x}
  \and
  \inferrule*[lab=Par-barb]{\mbox{$P\vdash x$ or $Q\vdash x$}}{\binpar{P}{Q} \vdash x}
\end{mathpar}

\subsubsection{Contexts}

One of the principle advantages of computational calculi like the
$\pi$-calculus is a well-defined notion of context,
contextual-equivalence and a correlation between
contextual-equivalence and notions of bisimulation. The notion of
context allows the decomposition of a process into (sub-)process and
its syntactic environment, its context. Thus, a context may be
thought of as a process with a ``hole'' (written $\Box$) in it. The
application of a context $M$ to a process $P$, written $M[P]$, is
tantamount to filling the hole in $M$ with $P$. In this paper we do
not need the full weight of this theory, but do make use of the notion
of context in the proof the main theorem. 

\begin{mathpar}
  \inferrule* [lab=summation] {} {{M_{M},M_{N}} \bc \Box \;|\; x.M_{A} \;|\; M_{M}+M_{N}}
  \and
  \inferrule* [lab=agent] {} {{M_{A}} \bc (\vec{x})M_{P} \;| \; \clift{P_0,\ldots,M_{P},\ldots,P_N}}
  \and \\
  \inferrule* [lab=process] {} {{M_{P}} \bc M_{N} \;| \;P|M_{P} }
\end{mathpar} 

\begin{mathpar}
  \inferrule* [lab=sychronization] {} {M_{N} \bc \Box \;|\; x?M_{F} \;|\; x!M_{C}}
  \and
  \inferrule* [lab=abstraction] {} {{M_{F}} \bc (x)M_{P} }
  \and
  \inferrule* [lab=concretion] {} {{M_{C}} \bc \langle M_{P} \rangle }
  \and \\
  \inferrule* [lab=process] {} {{M_{P}} \bc M_{N} \;| \;P|M_{P} }
\end{mathpar}

\begin{definition}[contextual application] Given a context $M$, and
  process $P$, we define the \emph{contextual application}, $M[P] :=
  M\{P/\Box\}$. That is, the contextual application of M to P is the
  substitution of $P$ for $\Box$ in $M$.
\end{definition}

$\meaningof{-} : L \to \mathcal{P}(\pi)$

\begin{mathpar}
  \inferrule* [lab=collection] {} {\meaningof{true} = \pi, \and \meaningof{~E} = \pi \setminus \meaningof{E}, \and \meaningof{E_{1} \& E_{2}} = \meaningof{E_{1}} \cap \meaningof{E_{2}}}
\end{mathpar}

\begin{mathpar}
  \inferrule* [lab=structure] {} {\meaningof{0} = \{ P \in \pi | P \equiv 0 \}, \and \\ \meaningof{E_1 | E_2} = \{ P \in \pi | P \equiv P_{1} | P_{2}, P_{1} \in \meaningof{E_{1}}, P_{2} \in \meaningof{E_2}\} }
\end{mathpar}

\begin{mathpar}
 \inferrule* [lab=behavior] {} {\meaningof{\langle a?b \rangle E} = \{ P \in \pi | P \equiv Q | u?(y)P', \\ \and \\\\ \and \\ \;\;\; u \in \meaningof{a}, \forall z.P'\{z/y\} \in \meaningof{E\{z/b\}}\}, \and \\ \meaningof{a!E} = \{ P \in \pi | P \equiv Q | x!\langle P' \rangle, x \in \meaningof{a} P' \in \meaningof{E}\} }
\end{mathpar}

\begin{mathpar}
 \inferrule* [lab=nominal] {} {\meaningof{\quotep{E}} = \{ \quotep{P} \in \quotep{\pi} | P \in \meaningof{E} \}, \and \meaningof{\quotep{P}} = \{ \quotep{Q} \in \quotep{\pi} | P \equiv Q \} \and \\ \meaningof{@\quotep{E}} = \{ P \in \pi | P \equiv @x, x \in \meaningof{E} \}}
\end{mathpar}

\begin{eqnarray*}
  \\
  \meaningof{-} : TS \to ST
\end{eqnarray*}

\begin{eqnarray*}
  \\
  L : TS \to ST
\end{eqnarray*}

\begin{eqnarray*}
  \\
  P \models E \iff P \in \meaningof{E}
\end{eqnarray*}

\begin{eqnarray*}
  P \approx_{L} Q \iff \forall E \in L. P \models E \iff Q \models E
\end{eqnarray*}

\begin{eqnarray*}
  P \approx_{K} Q
\end{eqnarray*}

\begin{eqnarray*}
  P \approx Q
\end{eqnarray*}

$\approx_{K} = \approx = \approx_{L}$

\subsubsection{Contextual duality}

Note that contexts extend the quotation operation to a family of
operations from processes to names. Given a context, $M$, we can
define a \emph{nominal context}, $\quotep{M}$ by $\quotep{M}[P] :=
\quotep{M[P]}$. To foreshadow what is to come we observe that these
operations enjoy a duality with processes very much like the duality
between vectors and maps from vectors to scalars.

Further, because the calculus is essentially higher-order, we have a
correspondence between contexts and processes. More specifically,
given a name $x$ and a context $M$ we can construct $M^{*}_{x}$ such
that 

\begin{mathpar}
  M^{*}_{x} | \lift{x}{P} \red M[P]
\end{mathpar}

namely,

\begin{mathpar}
  M^{*}_{x} := x?(u).M[\dropn{u}]
\end{mathpar}

The dependence of $M^{*}_{x}$ on a name makes it an abstraction, 

\begin{mathpar}
  M^{*} := (x)x?(u).M[\dropn{u}]
\end{mathpar}

\subsection{Additional notation}

It will sometimes be convenient to denote the process a name
quotes. We already have the notation $x = \quotep{P}$, but it will be
convenient to introduce an alternate notation, $\procn{x}$, when we
want to emphasize the connection to the use of the name. Note that, by
virtue of name equivalence, $\quotep{\procn{x}} \nameeq x$; so, the
notation is consistent with previous definitions.

Further, because names have structure it is possible to effect
substitutions on the basis of that structure. This means we need to
upgrade our notation for substitutions, which we accomplish by
adapting comprehension notation. Thus,

\begin{mathpar}
  P\{ y / x : x \in S \}
\end{mathpar}

is interpreted to mean the process derived from P by replacing (in a
capture-avoiding manner) each occurrence of $x$ in $S$ by $y$. For example,

\begin{mathpar}
  P\{ \quotep{\procn{x}|\procn{x}} / x : x \in \freenames{P} \}
\end{mathpar}

will replace each (occurrence) of a free name $x$ in $P$ by
$\quotep{\procn{x}|\procn{x}}$.

Also, we will avail ourselves of the notation $x^{L}$ and $x^{R}$ to
denote injections of a name into disjoint copies of the name
space. There are numerous ways to accomplish this. One example can be
found in \cite{MeredithR05}. This notation overloads to vectors of
names: $\vec{x}^{\pi} := (x_{i}^{\pi} \; : \; 0 \leq i < |\vec{x}| )$ where $\pi \in \{L,R\}$.

We also use $P^{\Box} := P|\Box$.

In \cite{MeredithR05} an interpretation of the new operator is
given. It turns out that there are several possible interpretations
all enjoying the requisite algebraic properties of the operator (see
\cite{milner91polyadicpi}). We will therefore make liberal use of
$(\nu\; \vec{x})P$.

% subsection the_syntax_and_semantics_of_the_notation_system (end)   

\input{qm2pi.qmops} 

\input{qm2pi.sterngerlach} 

\input{qm2pi.metric} 

% section concurrent_process_calculi (end)

%\input{qm2pi.proofsketch}

% section proof sketch (end)

%\input{qm2pi.slviaknots} 

% section spatial logic via knots (end)

\input{qm2pi.conclusion}

% section conclusion (end)

%\input{qm2pi.dtcodes} 

% section wiring algorithm (end)

\input{qm2pi.ack} 

% section acknowledgments (end)

\newpage


\bibliographystyle{plain}   
\bibliography{../../biblios/main.bib}

\input{qm2pi.rhodetails}

\end{document}

 

% section notation (end)

\input{qm2pi.process.calculi} 

% section concurrent_process_calculi_and_spatial_logics_ (end)
    
%\documentclass[12pt]{llncs}
%\documentclass{jktr}

\usepackage[pdftex]{hyperref}                   
\usepackage {listings}
\usepackage {mathpartir}
\usepackage{bcprules}
%\usepackage{listings}
                       
\usepackage{graphicx} 
%\usepackage[margins=2.5cm,nohead,nofoot]{geometry}
%\usepackage{geometry}
\usepackage{amsfonts}
\usepackage{amstext}
\usepackage{latexsym}
\usepackage{amssymb}
\usepackage{color}


%\include{myPreamble}
\include{qm2pi.local} 

%\ifpdf
%\usepackage[pdftex]{graphicx}
%\else
%\usepackage{graphicx}
%\fi

 % \ifpdf
%  \usepackage{pdfsync}
%  \if


%\title{Brief Article}
%\author{David F. Snyder}
%\author{L.G. Meredith}

%\address{Dept. of Math., Texas State University--San Marcos, San Marcos, TX 78666}
       
\pagestyle{empty}


\begin{document}

\lstset{language=[Objective]Caml,frame=shadowbox}

\input{qm2pi.front}

% section front matter (end)

\input{qm2pi.intro} 
 
% section introduction (end)

% \input{qm2pi.knotations} 

% section notation (end)

\input{qm2pi.process.calculi} 

% section concurrent_process_calculi_and_spatial_logics_ (end)
    
%\input{qm2pi.knots2pi} 

%\input{qm2pi.trefoil} 

%\input{qm2pi.mainthm} 

% subsection basic_interpretation (end)

%\input{qm2pi.rho.presentation} 
\subsection{The syntax and semantics of the notation system}\label{sub:the_syntax_and_semantics_of_the_notation_system} % (fold)

We now summarize a technical presentation of the calculus that
embodies our theory of dynamics. The typical presentation of such a
calculus follows the style of giving generators and relations on
them. The grammar, below, describing term constructors, freely
generates the set of processes, $\Proc$. This set is then quotiented
by a relation known as structural congruence and it is over this set
that the notion of dynamics is expressed. This presentation is
essentially that of \cite{MeredithR05} with the addition of
polyadicity and summation. For readability we have relegated some of
the technical subtleties to an appendix.

\subsubsection{Process grammar}\label{subsub:process_grammar}

\begin{mathpar}
  \inferrule* [lab=synchronization] {} {{M} \bc \pzero \;|\; x?F \;|\; x!C }
  \and
  \inferrule* [lab=abstraction] {} {{F} \bc (x)P}
  \and
  \inferrule* [lab=concretion] {} {{C} \bc \langle Q \rangle}
  \and
  \inferrule* [lab=process] {} {{P,Q} \bc M \;| \;P|Q \;|\; @{x}}
  \and
  \inferrule* [lab=name] {} {{x} \bc \quotep{P}}
\end{mathpar} 

Note that $\vec{x}$ (resp. $\vec{P}$) denotes a vector of names
(resp. processes) of length $|\vec{x}|$ (resp. $|\vec{P}|$). We adopt
the following useful abbreviations.

\begin{mathpar}
   x?(\vec{y}).P := x.(\vec{y})P \and  x\clift{\vec{P}} := x.\clift{\vec{P}}
   \and x!(y) := \lift{x}{\dropn{y}}
   \and \Pi_{i=0}^{n-1}P_i := P_0 | \ldots | P_{n-1}
\end{mathpar}

\subsubsection{Structural congruence}

\paragraph{Free and bound names and alpha-equivalence.} At the
core of structural equivalence is alpha-equivalence which identifies
process that are the same up to a change of variable. Formally, we
recognize the distinction between free and bound names. The free names
of a process, $\freenames{P}$, may be calculated recursively as
follows:

\begin{mathpar}
\freenames{\pzero} := \emptyset
  \and \\
  \freenames{x?(y).P} := \{ x \} \cup (\freenames{P} \setminus \{ y \})
  \and 
  \freenames{x!\langle P \rangle} := \{ x \} \cup \{ P \} 
  \and \\
  \freenames{P|Q} := \freenames{P} \cup \freenames{Q}
  \and \\
  \freenames{@{x}} := \{ x \}
\end{mathpar}

$\pi$
$\quotep{\pi}$

$\freenames{-} : \pi \to \mathcal{P}(\quotep{\pi})$

\begin{eqnarray*}
  \freenames{\pzero} & := & \emptyset \\
  \freenames{x?(y).P} & := & \{ x \} \cup (\freenames{P} \setminus \{ y \}) \\
  \freenames{x!\langle P \rangle} & := & \{ x \} \cup \{ P \} \\
  \freenames{P|Q} & := & \freenames{P} \cup \freenames{Q} \\
  \freenames{\dropn{x}} & := & \{ x \}
\end{eqnarray*}

The bound names of a process, $\boundnames{P}$, are those names occurring in $P$
that are not free. For example, in $x?(y).0$, the name $x$ is free, while $y$ is bound.

\begin{mathpar}
  \inferrule* [lab=monoidal-laws] {} { P|Q \equiv Q|P \and P|0 \equiv P \and P|(Q|R) \equiv (P|Q)|R }
\end{mathpar}

\begin{mathpar}
  \inferrule* [lab=alpha-equivalence] {} { (x)P \equiv (y)P\{y/x\} \and y \not\in \freenames{P} }
\end{mathpar}

\begin{definition}
Then two processes, $P,Q$, are alpha-equivalent if $P = Q\{\vec{y}/\vec{x}\}$ for
some $\vec{x} \in \boundnames{Q},\vec{y} \in \boundnames{P}$, where $Q\{\vec{y}/\vec{x}\}$
denotes the capture-avoiding substitution of $\vec{y}$ for $\vec{x}$ in $Q$.
\end{definition}

\begin{definition}
  The {\em structural congruence} \cite{SangiorgiWalker} , $\equiv$,
  between processes is the least congruence containing
  alpha-equivalence, satisfying the abelian monoid laws
  (associativity, commutativity and $\pzero$ as identity) for parallel
  composition $|$ and for summation $+$.
\end{definition}

\subsection{Name equivalence}

We take name equivalence, written $\nameeq$, to be the smallest
equivalence relation generated by the following rules.

\begin{mathpar}
\inferrule*[lab=Quote-drop]
{ }
{ \quotep{@{x}} \nameeq x }

\inferrule*[lab=Struct-equiv]
{ P \scong Q }
{ \quotep{P} \nameeq \quotep{Q} }
\end{mathpar}

The astute reader will have noticed that the mutual recursion of names
and processes imposes a mutual recursion on alpha-equivalence and
structural equivalence via name-equivalence. Fortunately, all of this
works out pleasantly and we may calculate in the natural way, free of
concern. The reader interested in the details is referred to the
appendix \ref{appendix:rho_details}.

\subsection{Substitution}

We use $\Proc$ for the set of processes, $\QProc$ for the set of
names, and $\id{\{}\vec{y} / \vec{x} \id{\}}$ to denote partial maps,
$s : \QProc \rightarrow \QProc$. A map, $s$ lifts, uniquely, to a map
on process terms, $\widehat{s} : \Proc \rightarrow \Proc$ by the
following equations.

\begin{mathpar}
  (0) \psubstp{Q}{P} := 0 \\
  (R \juxtap S) \psubstp{Q}{P}
  :=    
  (R)\psubstp{Q}{P} \juxtap (S) \psubstp{Q}{P} \\
  (x?(y).R) \psubstp{Q}{P}    
  :=    
  (x)\substp{Q}{P} (z)\concat( (R \psubstn{z}{y}) \psubstp{Q}{P} ) \\
  (\lift{x}{R}) \psubstp{Q}{P}  
  :=
  \lift{(x)\substp{Q}{P}}{ R \psubstp{Q}{P} } \\
%   (\dropn{x})  \psubstp{Q}{P}       
%   := 
%   \left\{ 
%     \begin{array}{ccc} 
%       \dropn{\quotep{Q}} & & x \nameeq \quotep{P} \\
%       \dropn{x} & & otherwise \\
%     \end{array}
%   \right. 
  (\dropn{x})  \psubstp{Q}{P}       
  := 
  \left\{ 
    \begin{array}{ccc} 
      Q & & x \nameeq \quotep{P} \\
      \dropn{x} & & otherwise \\
    \end{array}
  \right.
\end{mathpar}
 

where

\begin{eqnarray}
  (x)\id{\{} \lpquote Q \rpquote / \lpquote P \rpquote \id{\}}            = 
  \left\{ 
    \begin{array}{ccc}
      \lpquote Q \rpquote & & x \nameeq \lpquote P \rpquote \\
      x & & otherwise \\
    \end{array}
  \right. \nonumber
\end{eqnarray}

and $z$ is chosen distinct from $\quotep{P}$, $\quotep{Q}$, the free
names in $Q$, and all the names in $R$. Our $\alpha$-equivalence will
be built in the standard way from this substitution.

\begin{remark}\label{rem:no_self_referential_names}
  One consequence of these definitions is that $\forall P. \quotep{P}
  \not\in \freenames{P}$.
\end{remark}

\subsection{ Dynamic quote: an example }

Anticipating something of what's to come, consider applying the
substitution, $\widehat{\id{\{}u / z \id{\}}}$, to the following pair
of processes, $\lift{w}{y!(z)}$ and $w[ \lpquote y!(z) \rpquote ]$.

\begin{eqnarray}
	\lift{w}{y!(z)}\widehat{\id{\{}u / z \id{\}}}
		& = &
		\lift{w}{y!(u)} \nonumber\\
	w[ \lpquote y!(z) \rpquote ] \widehat{ \id{\{}u / z \id{\}} }
		& = &
		w[ \lpquote y!(z) \rpquote ] \nonumber
\end{eqnarray}

Because the body of the process between quotes is impervious to
substitution, we get radically different answers. In fact, by
examining the first process in an input context,
e.g. $x?(z).\lift{w}{y!(z)}$, we see that the process under the lift
operator may be shaped by prefixed inputs binding a name inside it. In
this sense, the lift operator will be seen as a way to dynamically
construct processes before reifying them as names.

Finally equipped with these standard features we can present the
dynamics of the calculus.

\subsubsection{Operational semantics} 

Finally, we introduce the computational dynamics. What marks these
algebras as distinct from other more traditionally studied algebraic
structures, e.g. vector spaces or polynomial rings, is the manner in
which dynamics is captured. In traditional structures, dynamics is typically
expressed through morphisms between such structures, as in linear maps
between vector spaces or morphisms between rings. In algebras
associated with the semantics of computation, the dynamics is
expressed as part of the algebraic structure itself, through a
reduction reduction relation typically denoted by $\red$. Below, we
give a recursive presentation of this relation for the calculus used
in the encoding.

$\red \subseteq \pi \times \pi$
$\red : \pi \to \mathcal{P}(\pi)$

\begin{mathpar}
  \inferrule* [lab=Comm] { \textsf{match}( x_{src}, x_{trgt} ) } { x_{trgt}?(y)P \; | \; x_{src}!\langle {Q} \rangle \red P\{\quotep{Q}/y}\} }
  \and \\
  \inferrule* [lab=Par] {{P} \red {P}'} {{{P} | {Q}} \red {{P}' | {Q}}}
  \and
  \inferrule* [lab=Equiv]{{{P} \scong {P}'} \andalso {{P}' \red {Q}'} \andalso {{Q}' \scong {Q}}}{{P} \red {Q}}
\end{mathpar}

\begin{eqnarray*}
  match_{\equiv} (\quotep{P},\quotep{Q}) & := & P \equiv Q \\
  match_{\dagger}(\quotep{P},\quotep{Q}) & := & \forall R. P|Q \red^{*} R => R \red^{*} 0 \\
  match_{K}(\quotep{P},\quotep{Q}) & := & K \mbox{ for some context } K
\end{eqnarray*}

$u?(x)P | u!\langle Q \rangle \red P\{\quotep{Q}/x\}$

%We write $\wred$ for $\red^*$, and $P\red$ if $\exists Q $ such that $ P \red Q$.
We write $P\red$ if $\exists Q $ such that $ P \red Q$ and $P\not\red$, otherwise.

\section{Replication}

As mentioned before, it is known that replication (and hence
recursion) can be implemented in a higher-order process algebra
\cite{SangiorgiWalker}. As our first example of calculation with the
machinery thus far presented we give the construction explicitly in
the {\rhoc}.

\begin{eqnarray}
	D_{x} & := & \prefix{x}{y}{(\binpar{\outputp{x}{y}}{@{y}})} \nonumber\\
	\bangp_{x}{P} & := & \binpar{{x}!\langle{\binpar{D_{x}}{P}}\rangle}{D_{x}} \nonumber
\end{eqnarray}

\begin{eqnarray}
	\bangp_{x}{P} & & \nonumber\\
	=
	& {x}!\langle{(\prefix{x}{y}{(\outputp{x}{y} | @{y})) | P}}\rangle 
	      | \prefix{x}{y}{(\outputp{x}{y} | @{y})} & \nonumber\\
	\red
	& (\outputp{x}{y} | @{y})\substn{\quotep{(\prefix{x}{y}{(@{y} | \outputp{x}{y})) | P}}}{y} & \nonumber\\
	=
	& \outputp{x}{\quotep{(\prefix{x}{y}{(\outputp{x}{y} | @{y})) | P}}}
	  | {(\prefix{x}{y}{(\outputp{x}{y} | @{y})) | P}} & \nonumber\\
	\red
	& \ldots & \nonumber\\
	\red^*
	& P | P | \ldots & \nonumber
\end{eqnarray}

Of course, this encoding, as an implementation, runs away, unfolding
$\bangp{P}$ eagerly. A lazier and more implementable replication
operator, restricted to input-guarded processes, may be obtained as follows.

\begin{eqnarray}
\bangp{\prefix{u}{v}{P}} 
	:= 
	\binpar{\lift{x}{\prefix{u}{v}{(\binpar{D(x)}{P})}}}{D(x)} \nonumber
\end{eqnarray}

\begin{remark}
  Note that the lazier definition still does not deal with summation
  or mixed summation (i.e. sums over input and output). The reader is
  invited to construct definitions of replication that deal with these
  features. 

  Further, the definitions are parameterized in a name, $x$. Can you,
  gentle reader, make a definition that eliminates this parameter and
  guarantees no accidental interaction between the replication
  machinery and the process being replicated -- i.e. no accidental
  sharing of names used by the process to get its work done and the
  name(s) used by the replication to effect copying. This latter
  revision of the definition of replication is crucial to obtaining
  the expected identity $!!P \sim !P$.
\end{remark}

\begin{remark}\label{rem:paradoxical_combinator}
  The reader familiar with the lambda calculus will have noticed the
  similarity between $D$ and the paradoxical combinator.

  [Ed. note: the existence of this seems to suggest we have to be more
  restrictive on the set of processes and names we admit if we are to
  support no-cloning.]
\end{remark}

\subsubsection{Bisimulation}

The computational dynamics gives rise to another kind of equivalence,
the equivalence of computational behavior. As previously mentioned
this is typically captured \emph{via} some form of bisimulation.

% The notion we use in this paper is weak barbed bisimulation
% \cite{milner91polyadicpi}.

The notion we use in this paper is derived from weak barbed
bisimulation \cite{milner91polyadicpi}. 

\begin{definition}
An \emph{observation relation}, $\downarrow_{\mathcal N}$, over a set
of names, $\mathcal N$, is the smallest relation satisfying the rules
below.

\infrule[Out-barb]{y \in {\mathcal N}, \; x \nameeq y}
		  {\outputp{x}{v} \downarrow_{\mathcal N} x}
\infrule[Par-barb]{\mbox{$P\downarrow_{\mathcal N} x$ or $Q\downarrow_{\mathcal N} x$}}
		  {\binpar{P}{Q} \downarrow_{\mathcal N} x}

We write $P \Downarrow_{\mathcal N} x$ if there is $Q$ such that 
$P \wred Q$ and $Q \downarrow_{\mathcal N} x$.
\end{definition}

\begin{definition}
%\label{def.bbisim}
An  ${\mathcal N}$-\emph{barbed bisimulation} over a set of names, ${\mathcal N}$, is a symmetric binary relation 
${\mathcal S}_{\mathcal N}$ between agents such that $P\rel{S}_{\mathcal N}Q$ implies:
\begin{enumerate}
\item If $P \red P'$ then $Q \wred Q'$ and $P'\rel{S}_{\mathcal N} Q'$.
\item If $P\downarrow_{\mathcal N} x$, then $Q\Downarrow_{\mathcal N} x$.
\end{enumerate}
$P$ is ${\mathcal N}$-barbed bisimilar to $Q$, written
$P \wbbisim_{\mathcal N} Q$, if $P \rel{S}_{\mathcal N} Q$ for some ${\mathcal N}$-barbed bisimulation ${\mathcal S}_{\mathcal N}$.
\end{definition}

$\mathcal{R} \subseteq \pi \times \pi$

$P \mathcal{R} Q => \forall P'. P \red P' \Rightarrow \exists Q'. Q \red Q', P' \mathcal{R} Q'$

$P \vdash x \Rightarrow Q \vdash x$

\begin{mathpar}
  \inferrule*[lab=Out-barb]{x \nameeq y}{{y}!\langle{Q}\rangle \vdash x}
  \and
  \inferrule*[lab=Par-barb]{\mbox{$P\vdash x$ or $Q\vdash x$}}{\binpar{P}{Q} \vdash x}
\end{mathpar}

\subsubsection{Contexts}

One of the principle advantages of computational calculi like the
$\pi$-calculus is a well-defined notion of context,
contextual-equivalence and a correlation between
contextual-equivalence and notions of bisimulation. The notion of
context allows the decomposition of a process into (sub-)process and
its syntactic environment, its context. Thus, a context may be
thought of as a process with a ``hole'' (written $\Box$) in it. The
application of a context $M$ to a process $P$, written $M[P]$, is
tantamount to filling the hole in $M$ with $P$. In this paper we do
not need the full weight of this theory, but do make use of the notion
of context in the proof the main theorem. 

\begin{mathpar}
  \inferrule* [lab=summation] {} {{M_{M},M_{N}} \bc \Box \;|\; x.M_{A} \;|\; M_{M}+M_{N}}
  \and
  \inferrule* [lab=agent] {} {{M_{A}} \bc (\vec{x})M_{P} \;| \; \clift{P_0,\ldots,M_{P},\ldots,P_N}}
  \and \\
  \inferrule* [lab=process] {} {{M_{P}} \bc M_{N} \;| \;P|M_{P} }
\end{mathpar} 

\begin{mathpar}
  \inferrule* [lab=sychronization] {} {M_{N} \bc \Box \;|\; x?M_{F} \;|\; x!M_{C}}
  \and
  \inferrule* [lab=abstraction] {} {{M_{F}} \bc (x)M_{P} }
  \and
  \inferrule* [lab=concretion] {} {{M_{C}} \bc \langle M_{P} \rangle }
  \and \\
  \inferrule* [lab=process] {} {{M_{P}} \bc M_{N} \;| \;P|M_{P} }
\end{mathpar}

\begin{definition}[contextual application] Given a context $M$, and
  process $P$, we define the \emph{contextual application}, $M[P] :=
  M\{P/\Box\}$. That is, the contextual application of M to P is the
  substitution of $P$ for $\Box$ in $M$.
\end{definition}

$\meaningof{-} : L \to \mathcal{P}(\pi)$

\begin{mathpar}
  \inferrule* [lab=collection] {} {\meaningof{true} = \pi, \and \meaningof{~E} = \pi \setminus \meaningof{E}, \and \meaningof{E_{1} \& E_{2}} = \meaningof{E_{1}} \cap \meaningof{E_{2}}}
\end{mathpar}

\begin{mathpar}
  \inferrule* [lab=structure] {} {\meaningof{0} = \{ P \in \pi | P \equiv 0 \}, \and \\ \meaningof{E_1 | E_2} = \{ P \in \pi | P \equiv P_{1} | P_{2}, P_{1} \in \meaningof{E_{1}}, P_{2} \in \meaningof{E_2}\} }
\end{mathpar}

\begin{mathpar}
 \inferrule* [lab=behavior] {} {\meaningof{\langle a?b \rangle E} = \{ P \in \pi | P \equiv Q | u?(y)P', \\ \and \\\\ \and \\ \;\;\; u \in \meaningof{a}, \forall z.P'\{z/y\} \in \meaningof{E\{z/b\}}\}, \and \\ \meaningof{a!E} = \{ P \in \pi | P \equiv Q | x!\langle P' \rangle, x \in \meaningof{a} P' \in \meaningof{E}\} }
\end{mathpar}

\begin{mathpar}
 \inferrule* [lab=nominal] {} {\meaningof{\quotep{E}} = \{ \quotep{P} \in \quotep{\pi} | P \in \meaningof{E} \}, \and \meaningof{\quotep{P}} = \{ \quotep{Q} \in \quotep{\pi} | P \equiv Q \} \and \\ \meaningof{@\quotep{E}} = \{ P \in \pi | P \equiv @x, x \in \meaningof{E} \}}
\end{mathpar}

\begin{eqnarray*}
  \\
  \meaningof{-} : TS \to ST
\end{eqnarray*}

\begin{eqnarray*}
  \\
  L : TS \to ST
\end{eqnarray*}

\begin{eqnarray*}
  \\
  P \models E \iff P \in \meaningof{E}
\end{eqnarray*}

\begin{eqnarray*}
  P \approx_{L} Q \iff \forall E \in L. P \models E \iff Q \models E
\end{eqnarray*}

\begin{eqnarray*}
  P \approx_{K} Q
\end{eqnarray*}

\begin{eqnarray*}
  P \approx Q
\end{eqnarray*}

$\approx_{K} = \approx = \approx_{L}$

\subsubsection{Contextual duality}

Note that contexts extend the quotation operation to a family of
operations from processes to names. Given a context, $M$, we can
define a \emph{nominal context}, $\quotep{M}$ by $\quotep{M}[P] :=
\quotep{M[P]}$. To foreshadow what is to come we observe that these
operations enjoy a duality with processes very much like the duality
between vectors and maps from vectors to scalars.

Further, because the calculus is essentially higher-order, we have a
correspondence between contexts and processes. More specifically,
given a name $x$ and a context $M$ we can construct $M^{*}_{x}$ such
that 

\begin{mathpar}
  M^{*}_{x} | \lift{x}{P} \red M[P]
\end{mathpar}

namely,

\begin{mathpar}
  M^{*}_{x} := x?(u).M[\dropn{u}]
\end{mathpar}

The dependence of $M^{*}_{x}$ on a name makes it an abstraction, 

\begin{mathpar}
  M^{*} := (x)x?(u).M[\dropn{u}]
\end{mathpar}

\subsection{Additional notation}

It will sometimes be convenient to denote the process a name
quotes. We already have the notation $x = \quotep{P}$, but it will be
convenient to introduce an alternate notation, $\procn{x}$, when we
want to emphasize the connection to the use of the name. Note that, by
virtue of name equivalence, $\quotep{\procn{x}} \nameeq x$; so, the
notation is consistent with previous definitions.

Further, because names have structure it is possible to effect
substitutions on the basis of that structure. This means we need to
upgrade our notation for substitutions, which we accomplish by
adapting comprehension notation. Thus,

\begin{mathpar}
  P\{ y / x : x \in S \}
\end{mathpar}

is interpreted to mean the process derived from P by replacing (in a
capture-avoiding manner) each occurrence of $x$ in $S$ by $y$. For example,

\begin{mathpar}
  P\{ \quotep{\procn{x}|\procn{x}} / x : x \in \freenames{P} \}
\end{mathpar}

will replace each (occurrence) of a free name $x$ in $P$ by
$\quotep{\procn{x}|\procn{x}}$.

Also, we will avail ourselves of the notation $x^{L}$ and $x^{R}$ to
denote injections of a name into disjoint copies of the name
space. There are numerous ways to accomplish this. One example can be
found in \cite{MeredithR05}. This notation overloads to vectors of
names: $\vec{x}^{\pi} := (x_{i}^{\pi} \; : \; 0 \leq i < |\vec{x}| )$ where $\pi \in \{L,R\}$.

We also use $P^{\Box} := P|\Box$.

In \cite{MeredithR05} an interpretation of the new operator is
given. It turns out that there are several possible interpretations
all enjoying the requisite algebraic properties of the operator (see
\cite{milner91polyadicpi}). We will therefore make liberal use of
$(\nu\; \vec{x})P$.

% subsection the_syntax_and_semantics_of_the_notation_system (end)   

\input{qm2pi.qmops} 

\input{qm2pi.sterngerlach} 

\input{qm2pi.metric} 

% section concurrent_process_calculi (end)

%\input{qm2pi.proofsketch}

% section proof sketch (end)

%\input{qm2pi.slviaknots} 

% section spatial logic via knots (end)

\input{qm2pi.conclusion}

% section conclusion (end)

%\input{qm2pi.dtcodes} 

% section wiring algorithm (end)

\input{qm2pi.ack} 

% section acknowledgments (end)

\newpage


\bibliographystyle{plain}   
\bibliography{../../biblios/main.bib}

\input{qm2pi.rhodetails}

\end{document}

 

%\documentclass[12pt]{llncs}
%\documentclass{jktr}

\usepackage[pdftex]{hyperref}                   
\usepackage {listings}
\usepackage {mathpartir}
\usepackage{bcprules}
%\usepackage{listings}
                       
\usepackage{graphicx} 
%\usepackage[margins=2.5cm,nohead,nofoot]{geometry}
%\usepackage{geometry}
\usepackage{amsfonts}
\usepackage{amstext}
\usepackage{latexsym}
\usepackage{amssymb}
\usepackage{color}


%\include{myPreamble}
\include{qm2pi.local} 

%\ifpdf
%\usepackage[pdftex]{graphicx}
%\else
%\usepackage{graphicx}
%\fi

 % \ifpdf
%  \usepackage{pdfsync}
%  \if


%\title{Brief Article}
%\author{David F. Snyder}
%\author{L.G. Meredith}

%\address{Dept. of Math., Texas State University--San Marcos, San Marcos, TX 78666}
       
\pagestyle{empty}


\begin{document}

\lstset{language=[Objective]Caml,frame=shadowbox}

\input{qm2pi.front}

% section front matter (end)

\input{qm2pi.intro} 
 
% section introduction (end)

% \input{qm2pi.knotations} 

% section notation (end)

\input{qm2pi.process.calculi} 

% section concurrent_process_calculi_and_spatial_logics_ (end)
    
%\input{qm2pi.knots2pi} 

%\input{qm2pi.trefoil} 

%\input{qm2pi.mainthm} 

% subsection basic_interpretation (end)

%\input{qm2pi.rho.presentation} 
\subsection{The syntax and semantics of the notation system}\label{sub:the_syntax_and_semantics_of_the_notation_system} % (fold)

We now summarize a technical presentation of the calculus that
embodies our theory of dynamics. The typical presentation of such a
calculus follows the style of giving generators and relations on
them. The grammar, below, describing term constructors, freely
generates the set of processes, $\Proc$. This set is then quotiented
by a relation known as structural congruence and it is over this set
that the notion of dynamics is expressed. This presentation is
essentially that of \cite{MeredithR05} with the addition of
polyadicity and summation. For readability we have relegated some of
the technical subtleties to an appendix.

\subsubsection{Process grammar}\label{subsub:process_grammar}

\begin{mathpar}
  \inferrule* [lab=synchronization] {} {{M} \bc \pzero \;|\; x?F \;|\; x!C }
  \and
  \inferrule* [lab=abstraction] {} {{F} \bc (x)P}
  \and
  \inferrule* [lab=concretion] {} {{C} \bc \langle Q \rangle}
  \and
  \inferrule* [lab=process] {} {{P,Q} \bc M \;| \;P|Q \;|\; @{x}}
  \and
  \inferrule* [lab=name] {} {{x} \bc \quotep{P}}
\end{mathpar} 

Note that $\vec{x}$ (resp. $\vec{P}$) denotes a vector of names
(resp. processes) of length $|\vec{x}|$ (resp. $|\vec{P}|$). We adopt
the following useful abbreviations.

\begin{mathpar}
   x?(\vec{y}).P := x.(\vec{y})P \and  x\clift{\vec{P}} := x.\clift{\vec{P}}
   \and x!(y) := \lift{x}{\dropn{y}}
   \and \Pi_{i=0}^{n-1}P_i := P_0 | \ldots | P_{n-1}
\end{mathpar}

\subsubsection{Structural congruence}

\paragraph{Free and bound names and alpha-equivalence.} At the
core of structural equivalence is alpha-equivalence which identifies
process that are the same up to a change of variable. Formally, we
recognize the distinction between free and bound names. The free names
of a process, $\freenames{P}$, may be calculated recursively as
follows:

\begin{mathpar}
\freenames{\pzero} := \emptyset
  \and \\
  \freenames{x?(y).P} := \{ x \} \cup (\freenames{P} \setminus \{ y \})
  \and 
  \freenames{x!\langle P \rangle} := \{ x \} \cup \{ P \} 
  \and \\
  \freenames{P|Q} := \freenames{P} \cup \freenames{Q}
  \and \\
  \freenames{@{x}} := \{ x \}
\end{mathpar}

$\pi$
$\quotep{\pi}$

$\freenames{-} : \pi \to \mathcal{P}(\quotep{\pi})$

\begin{eqnarray*}
  \freenames{\pzero} & := & \emptyset \\
  \freenames{x?(y).P} & := & \{ x \} \cup (\freenames{P} \setminus \{ y \}) \\
  \freenames{x!\langle P \rangle} & := & \{ x \} \cup \{ P \} \\
  \freenames{P|Q} & := & \freenames{P} \cup \freenames{Q} \\
  \freenames{\dropn{x}} & := & \{ x \}
\end{eqnarray*}

The bound names of a process, $\boundnames{P}$, are those names occurring in $P$
that are not free. For example, in $x?(y).0$, the name $x$ is free, while $y$ is bound.

\begin{mathpar}
  \inferrule* [lab=monoidal-laws] {} { P|Q \equiv Q|P \and P|0 \equiv P \and P|(Q|R) \equiv (P|Q)|R }
\end{mathpar}

\begin{mathpar}
  \inferrule* [lab=alpha-equivalence] {} { (x)P \equiv (y)P\{y/x\} \and y \not\in \freenames{P} }
\end{mathpar}

\begin{definition}
Then two processes, $P,Q$, are alpha-equivalent if $P = Q\{\vec{y}/\vec{x}\}$ for
some $\vec{x} \in \boundnames{Q},\vec{y} \in \boundnames{P}$, where $Q\{\vec{y}/\vec{x}\}$
denotes the capture-avoiding substitution of $\vec{y}$ for $\vec{x}$ in $Q$.
\end{definition}

\begin{definition}
  The {\em structural congruence} \cite{SangiorgiWalker} , $\equiv$,
  between processes is the least congruence containing
  alpha-equivalence, satisfying the abelian monoid laws
  (associativity, commutativity and $\pzero$ as identity) for parallel
  composition $|$ and for summation $+$.
\end{definition}

\subsection{Name equivalence}

We take name equivalence, written $\nameeq$, to be the smallest
equivalence relation generated by the following rules.

\begin{mathpar}
\inferrule*[lab=Quote-drop]
{ }
{ \quotep{@{x}} \nameeq x }

\inferrule*[lab=Struct-equiv]
{ P \scong Q }
{ \quotep{P} \nameeq \quotep{Q} }
\end{mathpar}

The astute reader will have noticed that the mutual recursion of names
and processes imposes a mutual recursion on alpha-equivalence and
structural equivalence via name-equivalence. Fortunately, all of this
works out pleasantly and we may calculate in the natural way, free of
concern. The reader interested in the details is referred to the
appendix \ref{appendix:rho_details}.

\subsection{Substitution}

We use $\Proc$ for the set of processes, $\QProc$ for the set of
names, and $\id{\{}\vec{y} / \vec{x} \id{\}}$ to denote partial maps,
$s : \QProc \rightarrow \QProc$. A map, $s$ lifts, uniquely, to a map
on process terms, $\widehat{s} : \Proc \rightarrow \Proc$ by the
following equations.

\begin{mathpar}
  (0) \psubstp{Q}{P} := 0 \\
  (R \juxtap S) \psubstp{Q}{P}
  :=    
  (R)\psubstp{Q}{P} \juxtap (S) \psubstp{Q}{P} \\
  (x?(y).R) \psubstp{Q}{P}    
  :=    
  (x)\substp{Q}{P} (z)\concat( (R \psubstn{z}{y}) \psubstp{Q}{P} ) \\
  (\lift{x}{R}) \psubstp{Q}{P}  
  :=
  \lift{(x)\substp{Q}{P}}{ R \psubstp{Q}{P} } \\
%   (\dropn{x})  \psubstp{Q}{P}       
%   := 
%   \left\{ 
%     \begin{array}{ccc} 
%       \dropn{\quotep{Q}} & & x \nameeq \quotep{P} \\
%       \dropn{x} & & otherwise \\
%     \end{array}
%   \right. 
  (\dropn{x})  \psubstp{Q}{P}       
  := 
  \left\{ 
    \begin{array}{ccc} 
      Q & & x \nameeq \quotep{P} \\
      \dropn{x} & & otherwise \\
    \end{array}
  \right.
\end{mathpar}
 

where

\begin{eqnarray}
  (x)\id{\{} \lpquote Q \rpquote / \lpquote P \rpquote \id{\}}            = 
  \left\{ 
    \begin{array}{ccc}
      \lpquote Q \rpquote & & x \nameeq \lpquote P \rpquote \\
      x & & otherwise \\
    \end{array}
  \right. \nonumber
\end{eqnarray}

and $z$ is chosen distinct from $\quotep{P}$, $\quotep{Q}$, the free
names in $Q$, and all the names in $R$. Our $\alpha$-equivalence will
be built in the standard way from this substitution.

\begin{remark}\label{rem:no_self_referential_names}
  One consequence of these definitions is that $\forall P. \quotep{P}
  \not\in \freenames{P}$.
\end{remark}

\subsection{ Dynamic quote: an example }

Anticipating something of what's to come, consider applying the
substitution, $\widehat{\id{\{}u / z \id{\}}}$, to the following pair
of processes, $\lift{w}{y!(z)}$ and $w[ \lpquote y!(z) \rpquote ]$.

\begin{eqnarray}
	\lift{w}{y!(z)}\widehat{\id{\{}u / z \id{\}}}
		& = &
		\lift{w}{y!(u)} \nonumber\\
	w[ \lpquote y!(z) \rpquote ] \widehat{ \id{\{}u / z \id{\}} }
		& = &
		w[ \lpquote y!(z) \rpquote ] \nonumber
\end{eqnarray}

Because the body of the process between quotes is impervious to
substitution, we get radically different answers. In fact, by
examining the first process in an input context,
e.g. $x?(z).\lift{w}{y!(z)}$, we see that the process under the lift
operator may be shaped by prefixed inputs binding a name inside it. In
this sense, the lift operator will be seen as a way to dynamically
construct processes before reifying them as names.

Finally equipped with these standard features we can present the
dynamics of the calculus.

\subsubsection{Operational semantics} 

Finally, we introduce the computational dynamics. What marks these
algebras as distinct from other more traditionally studied algebraic
structures, e.g. vector spaces or polynomial rings, is the manner in
which dynamics is captured. In traditional structures, dynamics is typically
expressed through morphisms between such structures, as in linear maps
between vector spaces or morphisms between rings. In algebras
associated with the semantics of computation, the dynamics is
expressed as part of the algebraic structure itself, through a
reduction reduction relation typically denoted by $\red$. Below, we
give a recursive presentation of this relation for the calculus used
in the encoding.

$\red \subseteq \pi \times \pi$
$\red : \pi \to \mathcal{P}(\pi)$

\begin{mathpar}
  \inferrule* [lab=Comm] { \textsf{match}( x_{src}, x_{trgt} ) } { x_{trgt}?(y)P \; | \; x_{src}!\langle {Q} \rangle \red P\{\quotep{Q}/y}\} }
  \and \\
  \inferrule* [lab=Par] {{P} \red {P}'} {{{P} | {Q}} \red {{P}' | {Q}}}
  \and
  \inferrule* [lab=Equiv]{{{P} \scong {P}'} \andalso {{P}' \red {Q}'} \andalso {{Q}' \scong {Q}}}{{P} \red {Q}}
\end{mathpar}

\begin{eqnarray*}
  match_{\equiv} (\quotep{P},\quotep{Q}) & := & P \equiv Q \\
  match_{\dagger}(\quotep{P},\quotep{Q}) & := & \forall R. P|Q \red^{*} R => R \red^{*} 0 \\
  match_{K}(\quotep{P},\quotep{Q}) & := & K \mbox{ for some context } K
\end{eqnarray*}

$u?(x)P | u!\langle Q \rangle \red P\{\quotep{Q}/x\}$

%We write $\wred$ for $\red^*$, and $P\red$ if $\exists Q $ such that $ P \red Q$.
We write $P\red$ if $\exists Q $ such that $ P \red Q$ and $P\not\red$, otherwise.

\section{Replication}

As mentioned before, it is known that replication (and hence
recursion) can be implemented in a higher-order process algebra
\cite{SangiorgiWalker}. As our first example of calculation with the
machinery thus far presented we give the construction explicitly in
the {\rhoc}.

\begin{eqnarray}
	D_{x} & := & \prefix{x}{y}{(\binpar{\outputp{x}{y}}{@{y}})} \nonumber\\
	\bangp_{x}{P} & := & \binpar{{x}!\langle{\binpar{D_{x}}{P}}\rangle}{D_{x}} \nonumber
\end{eqnarray}

\begin{eqnarray}
	\bangp_{x}{P} & & \nonumber\\
	=
	& {x}!\langle{(\prefix{x}{y}{(\outputp{x}{y} | @{y})) | P}}\rangle 
	      | \prefix{x}{y}{(\outputp{x}{y} | @{y})} & \nonumber\\
	\red
	& (\outputp{x}{y} | @{y})\substn{\quotep{(\prefix{x}{y}{(@{y} | \outputp{x}{y})) | P}}}{y} & \nonumber\\
	=
	& \outputp{x}{\quotep{(\prefix{x}{y}{(\outputp{x}{y} | @{y})) | P}}}
	  | {(\prefix{x}{y}{(\outputp{x}{y} | @{y})) | P}} & \nonumber\\
	\red
	& \ldots & \nonumber\\
	\red^*
	& P | P | \ldots & \nonumber
\end{eqnarray}

Of course, this encoding, as an implementation, runs away, unfolding
$\bangp{P}$ eagerly. A lazier and more implementable replication
operator, restricted to input-guarded processes, may be obtained as follows.

\begin{eqnarray}
\bangp{\prefix{u}{v}{P}} 
	:= 
	\binpar{\lift{x}{\prefix{u}{v}{(\binpar{D(x)}{P})}}}{D(x)} \nonumber
\end{eqnarray}

\begin{remark}
  Note that the lazier definition still does not deal with summation
  or mixed summation (i.e. sums over input and output). The reader is
  invited to construct definitions of replication that deal with these
  features. 

  Further, the definitions are parameterized in a name, $x$. Can you,
  gentle reader, make a definition that eliminates this parameter and
  guarantees no accidental interaction between the replication
  machinery and the process being replicated -- i.e. no accidental
  sharing of names used by the process to get its work done and the
  name(s) used by the replication to effect copying. This latter
  revision of the definition of replication is crucial to obtaining
  the expected identity $!!P \sim !P$.
\end{remark}

\begin{remark}\label{rem:paradoxical_combinator}
  The reader familiar with the lambda calculus will have noticed the
  similarity between $D$ and the paradoxical combinator.

  [Ed. note: the existence of this seems to suggest we have to be more
  restrictive on the set of processes and names we admit if we are to
  support no-cloning.]
\end{remark}

\subsubsection{Bisimulation}

The computational dynamics gives rise to another kind of equivalence,
the equivalence of computational behavior. As previously mentioned
this is typically captured \emph{via} some form of bisimulation.

% The notion we use in this paper is weak barbed bisimulation
% \cite{milner91polyadicpi}.

The notion we use in this paper is derived from weak barbed
bisimulation \cite{milner91polyadicpi}. 

\begin{definition}
An \emph{observation relation}, $\downarrow_{\mathcal N}$, over a set
of names, $\mathcal N$, is the smallest relation satisfying the rules
below.

\infrule[Out-barb]{y \in {\mathcal N}, \; x \nameeq y}
		  {\outputp{x}{v} \downarrow_{\mathcal N} x}
\infrule[Par-barb]{\mbox{$P\downarrow_{\mathcal N} x$ or $Q\downarrow_{\mathcal N} x$}}
		  {\binpar{P}{Q} \downarrow_{\mathcal N} x}

We write $P \Downarrow_{\mathcal N} x$ if there is $Q$ such that 
$P \wred Q$ and $Q \downarrow_{\mathcal N} x$.
\end{definition}

\begin{definition}
%\label{def.bbisim}
An  ${\mathcal N}$-\emph{barbed bisimulation} over a set of names, ${\mathcal N}$, is a symmetric binary relation 
${\mathcal S}_{\mathcal N}$ between agents such that $P\rel{S}_{\mathcal N}Q$ implies:
\begin{enumerate}
\item If $P \red P'$ then $Q \wred Q'$ and $P'\rel{S}_{\mathcal N} Q'$.
\item If $P\downarrow_{\mathcal N} x$, then $Q\Downarrow_{\mathcal N} x$.
\end{enumerate}
$P$ is ${\mathcal N}$-barbed bisimilar to $Q$, written
$P \wbbisim_{\mathcal N} Q$, if $P \rel{S}_{\mathcal N} Q$ for some ${\mathcal N}$-barbed bisimulation ${\mathcal S}_{\mathcal N}$.
\end{definition}

$\mathcal{R} \subseteq \pi \times \pi$

$P \mathcal{R} Q => \forall P'. P \red P' \Rightarrow \exists Q'. Q \red Q', P' \mathcal{R} Q'$

$P \vdash x \Rightarrow Q \vdash x$

\begin{mathpar}
  \inferrule*[lab=Out-barb]{x \nameeq y}{{y}!\langle{Q}\rangle \vdash x}
  \and
  \inferrule*[lab=Par-barb]{\mbox{$P\vdash x$ or $Q\vdash x$}}{\binpar{P}{Q} \vdash x}
\end{mathpar}

\subsubsection{Contexts}

One of the principle advantages of computational calculi like the
$\pi$-calculus is a well-defined notion of context,
contextual-equivalence and a correlation between
contextual-equivalence and notions of bisimulation. The notion of
context allows the decomposition of a process into (sub-)process and
its syntactic environment, its context. Thus, a context may be
thought of as a process with a ``hole'' (written $\Box$) in it. The
application of a context $M$ to a process $P$, written $M[P]$, is
tantamount to filling the hole in $M$ with $P$. In this paper we do
not need the full weight of this theory, but do make use of the notion
of context in the proof the main theorem. 

\begin{mathpar}
  \inferrule* [lab=summation] {} {{M_{M},M_{N}} \bc \Box \;|\; x.M_{A} \;|\; M_{M}+M_{N}}
  \and
  \inferrule* [lab=agent] {} {{M_{A}} \bc (\vec{x})M_{P} \;| \; \clift{P_0,\ldots,M_{P},\ldots,P_N}}
  \and \\
  \inferrule* [lab=process] {} {{M_{P}} \bc M_{N} \;| \;P|M_{P} }
\end{mathpar} 

\begin{mathpar}
  \inferrule* [lab=sychronization] {} {M_{N} \bc \Box \;|\; x?M_{F} \;|\; x!M_{C}}
  \and
  \inferrule* [lab=abstraction] {} {{M_{F}} \bc (x)M_{P} }
  \and
  \inferrule* [lab=concretion] {} {{M_{C}} \bc \langle M_{P} \rangle }
  \and \\
  \inferrule* [lab=process] {} {{M_{P}} \bc M_{N} \;| \;P|M_{P} }
\end{mathpar}

\begin{definition}[contextual application] Given a context $M$, and
  process $P$, we define the \emph{contextual application}, $M[P] :=
  M\{P/\Box\}$. That is, the contextual application of M to P is the
  substitution of $P$ for $\Box$ in $M$.
\end{definition}

$\meaningof{-} : L \to \mathcal{P}(\pi)$

\begin{mathpar}
  \inferrule* [lab=collection] {} {\meaningof{true} = \pi, \and \meaningof{~E} = \pi \setminus \meaningof{E}, \and \meaningof{E_{1} \& E_{2}} = \meaningof{E_{1}} \cap \meaningof{E_{2}}}
\end{mathpar}

\begin{mathpar}
  \inferrule* [lab=structure] {} {\meaningof{0} = \{ P \in \pi | P \equiv 0 \}, \and \\ \meaningof{E_1 | E_2} = \{ P \in \pi | P \equiv P_{1} | P_{2}, P_{1} \in \meaningof{E_{1}}, P_{2} \in \meaningof{E_2}\} }
\end{mathpar}

\begin{mathpar}
 \inferrule* [lab=behavior] {} {\meaningof{\langle a?b \rangle E} = \{ P \in \pi | P \equiv Q | u?(y)P', \\ \and \\\\ \and \\ \;\;\; u \in \meaningof{a}, \forall z.P'\{z/y\} \in \meaningof{E\{z/b\}}\}, \and \\ \meaningof{a!E} = \{ P \in \pi | P \equiv Q | x!\langle P' \rangle, x \in \meaningof{a} P' \in \meaningof{E}\} }
\end{mathpar}

\begin{mathpar}
 \inferrule* [lab=nominal] {} {\meaningof{\quotep{E}} = \{ \quotep{P} \in \quotep{\pi} | P \in \meaningof{E} \}, \and \meaningof{\quotep{P}} = \{ \quotep{Q} \in \quotep{\pi} | P \equiv Q \} \and \\ \meaningof{@\quotep{E}} = \{ P \in \pi | P \equiv @x, x \in \meaningof{E} \}}
\end{mathpar}

\begin{eqnarray*}
  \\
  \meaningof{-} : TS \to ST
\end{eqnarray*}

\begin{eqnarray*}
  \\
  L : TS \to ST
\end{eqnarray*}

\begin{eqnarray*}
  \\
  P \models E \iff P \in \meaningof{E}
\end{eqnarray*}

\begin{eqnarray*}
  P \approx_{L} Q \iff \forall E \in L. P \models E \iff Q \models E
\end{eqnarray*}

\begin{eqnarray*}
  P \approx_{K} Q
\end{eqnarray*}

\begin{eqnarray*}
  P \approx Q
\end{eqnarray*}

$\approx_{K} = \approx = \approx_{L}$

\subsubsection{Contextual duality}

Note that contexts extend the quotation operation to a family of
operations from processes to names. Given a context, $M$, we can
define a \emph{nominal context}, $\quotep{M}$ by $\quotep{M}[P] :=
\quotep{M[P]}$. To foreshadow what is to come we observe that these
operations enjoy a duality with processes very much like the duality
between vectors and maps from vectors to scalars.

Further, because the calculus is essentially higher-order, we have a
correspondence between contexts and processes. More specifically,
given a name $x$ and a context $M$ we can construct $M^{*}_{x}$ such
that 

\begin{mathpar}
  M^{*}_{x} | \lift{x}{P} \red M[P]
\end{mathpar}

namely,

\begin{mathpar}
  M^{*}_{x} := x?(u).M[\dropn{u}]
\end{mathpar}

The dependence of $M^{*}_{x}$ on a name makes it an abstraction, 

\begin{mathpar}
  M^{*} := (x)x?(u).M[\dropn{u}]
\end{mathpar}

\subsection{Additional notation}

It will sometimes be convenient to denote the process a name
quotes. We already have the notation $x = \quotep{P}$, but it will be
convenient to introduce an alternate notation, $\procn{x}$, when we
want to emphasize the connection to the use of the name. Note that, by
virtue of name equivalence, $\quotep{\procn{x}} \nameeq x$; so, the
notation is consistent with previous definitions.

Further, because names have structure it is possible to effect
substitutions on the basis of that structure. This means we need to
upgrade our notation for substitutions, which we accomplish by
adapting comprehension notation. Thus,

\begin{mathpar}
  P\{ y / x : x \in S \}
\end{mathpar}

is interpreted to mean the process derived from P by replacing (in a
capture-avoiding manner) each occurrence of $x$ in $S$ by $y$. For example,

\begin{mathpar}
  P\{ \quotep{\procn{x}|\procn{x}} / x : x \in \freenames{P} \}
\end{mathpar}

will replace each (occurrence) of a free name $x$ in $P$ by
$\quotep{\procn{x}|\procn{x}}$.

Also, we will avail ourselves of the notation $x^{L}$ and $x^{R}$ to
denote injections of a name into disjoint copies of the name
space. There are numerous ways to accomplish this. One example can be
found in \cite{MeredithR05}. This notation overloads to vectors of
names: $\vec{x}^{\pi} := (x_{i}^{\pi} \; : \; 0 \leq i < |\vec{x}| )$ where $\pi \in \{L,R\}$.

We also use $P^{\Box} := P|\Box$.

In \cite{MeredithR05} an interpretation of the new operator is
given. It turns out that there are several possible interpretations
all enjoying the requisite algebraic properties of the operator (see
\cite{milner91polyadicpi}). We will therefore make liberal use of
$(\nu\; \vec{x})P$.

% subsection the_syntax_and_semantics_of_the_notation_system (end)   

\input{qm2pi.qmops} 

\input{qm2pi.sterngerlach} 

\input{qm2pi.metric} 

% section concurrent_process_calculi (end)

%\input{qm2pi.proofsketch}

% section proof sketch (end)

%\input{qm2pi.slviaknots} 

% section spatial logic via knots (end)

\input{qm2pi.conclusion}

% section conclusion (end)

%\input{qm2pi.dtcodes} 

% section wiring algorithm (end)

\input{qm2pi.ack} 

% section acknowledgments (end)

\newpage


\bibliographystyle{plain}   
\bibliography{../../biblios/main.bib}

\input{qm2pi.rhodetails}

\end{document}

 

%\documentclass[12pt]{llncs}
%\documentclass{jktr}

\usepackage[pdftex]{hyperref}                   
\usepackage {listings}
\usepackage {mathpartir}
\usepackage{bcprules}
%\usepackage{listings}
                       
\usepackage{graphicx} 
%\usepackage[margins=2.5cm,nohead,nofoot]{geometry}
%\usepackage{geometry}
\usepackage{amsfonts}
\usepackage{amstext}
\usepackage{latexsym}
\usepackage{amssymb}
\usepackage{color}


%\include{myPreamble}
\include{qm2pi.local} 

%\ifpdf
%\usepackage[pdftex]{graphicx}
%\else
%\usepackage{graphicx}
%\fi

 % \ifpdf
%  \usepackage{pdfsync}
%  \if


%\title{Brief Article}
%\author{David F. Snyder}
%\author{L.G. Meredith}

%\address{Dept. of Math., Texas State University--San Marcos, San Marcos, TX 78666}
       
\pagestyle{empty}


\begin{document}

\lstset{language=[Objective]Caml,frame=shadowbox}

\input{qm2pi.front}

% section front matter (end)

\input{qm2pi.intro} 
 
% section introduction (end)

% \input{qm2pi.knotations} 

% section notation (end)

\input{qm2pi.process.calculi} 

% section concurrent_process_calculi_and_spatial_logics_ (end)
    
%\input{qm2pi.knots2pi} 

%\input{qm2pi.trefoil} 

%\input{qm2pi.mainthm} 

% subsection basic_interpretation (end)

%\input{qm2pi.rho.presentation} 
\subsection{The syntax and semantics of the notation system}\label{sub:the_syntax_and_semantics_of_the_notation_system} % (fold)

We now summarize a technical presentation of the calculus that
embodies our theory of dynamics. The typical presentation of such a
calculus follows the style of giving generators and relations on
them. The grammar, below, describing term constructors, freely
generates the set of processes, $\Proc$. This set is then quotiented
by a relation known as structural congruence and it is over this set
that the notion of dynamics is expressed. This presentation is
essentially that of \cite{MeredithR05} with the addition of
polyadicity and summation. For readability we have relegated some of
the technical subtleties to an appendix.

\subsubsection{Process grammar}\label{subsub:process_grammar}

\begin{mathpar}
  \inferrule* [lab=synchronization] {} {{M} \bc \pzero \;|\; x?F \;|\; x!C }
  \and
  \inferrule* [lab=abstraction] {} {{F} \bc (x)P}
  \and
  \inferrule* [lab=concretion] {} {{C} \bc \langle Q \rangle}
  \and
  \inferrule* [lab=process] {} {{P,Q} \bc M \;| \;P|Q \;|\; @{x}}
  \and
  \inferrule* [lab=name] {} {{x} \bc \quotep{P}}
\end{mathpar} 

Note that $\vec{x}$ (resp. $\vec{P}$) denotes a vector of names
(resp. processes) of length $|\vec{x}|$ (resp. $|\vec{P}|$). We adopt
the following useful abbreviations.

\begin{mathpar}
   x?(\vec{y}).P := x.(\vec{y})P \and  x\clift{\vec{P}} := x.\clift{\vec{P}}
   \and x!(y) := \lift{x}{\dropn{y}}
   \and \Pi_{i=0}^{n-1}P_i := P_0 | \ldots | P_{n-1}
\end{mathpar}

\subsubsection{Structural congruence}

\paragraph{Free and bound names and alpha-equivalence.} At the
core of structural equivalence is alpha-equivalence which identifies
process that are the same up to a change of variable. Formally, we
recognize the distinction between free and bound names. The free names
of a process, $\freenames{P}$, may be calculated recursively as
follows:

\begin{mathpar}
\freenames{\pzero} := \emptyset
  \and \\
  \freenames{x?(y).P} := \{ x \} \cup (\freenames{P} \setminus \{ y \})
  \and 
  \freenames{x!\langle P \rangle} := \{ x \} \cup \{ P \} 
  \and \\
  \freenames{P|Q} := \freenames{P} \cup \freenames{Q}
  \and \\
  \freenames{@{x}} := \{ x \}
\end{mathpar}

$\pi$
$\quotep{\pi}$

$\freenames{-} : \pi \to \mathcal{P}(\quotep{\pi})$

\begin{eqnarray*}
  \freenames{\pzero} & := & \emptyset \\
  \freenames{x?(y).P} & := & \{ x \} \cup (\freenames{P} \setminus \{ y \}) \\
  \freenames{x!\langle P \rangle} & := & \{ x \} \cup \{ P \} \\
  \freenames{P|Q} & := & \freenames{P} \cup \freenames{Q} \\
  \freenames{\dropn{x}} & := & \{ x \}
\end{eqnarray*}

The bound names of a process, $\boundnames{P}$, are those names occurring in $P$
that are not free. For example, in $x?(y).0$, the name $x$ is free, while $y$ is bound.

\begin{mathpar}
  \inferrule* [lab=monoidal-laws] {} { P|Q \equiv Q|P \and P|0 \equiv P \and P|(Q|R) \equiv (P|Q)|R }
\end{mathpar}

\begin{mathpar}
  \inferrule* [lab=alpha-equivalence] {} { (x)P \equiv (y)P\{y/x\} \and y \not\in \freenames{P} }
\end{mathpar}

\begin{definition}
Then two processes, $P,Q$, are alpha-equivalent if $P = Q\{\vec{y}/\vec{x}\}$ for
some $\vec{x} \in \boundnames{Q},\vec{y} \in \boundnames{P}$, where $Q\{\vec{y}/\vec{x}\}$
denotes the capture-avoiding substitution of $\vec{y}$ for $\vec{x}$ in $Q$.
\end{definition}

\begin{definition}
  The {\em structural congruence} \cite{SangiorgiWalker} , $\equiv$,
  between processes is the least congruence containing
  alpha-equivalence, satisfying the abelian monoid laws
  (associativity, commutativity and $\pzero$ as identity) for parallel
  composition $|$ and for summation $+$.
\end{definition}

\subsection{Name equivalence}

We take name equivalence, written $\nameeq$, to be the smallest
equivalence relation generated by the following rules.

\begin{mathpar}
\inferrule*[lab=Quote-drop]
{ }
{ \quotep{@{x}} \nameeq x }

\inferrule*[lab=Struct-equiv]
{ P \scong Q }
{ \quotep{P} \nameeq \quotep{Q} }
\end{mathpar}

The astute reader will have noticed that the mutual recursion of names
and processes imposes a mutual recursion on alpha-equivalence and
structural equivalence via name-equivalence. Fortunately, all of this
works out pleasantly and we may calculate in the natural way, free of
concern. The reader interested in the details is referred to the
appendix \ref{appendix:rho_details}.

\subsection{Substitution}

We use $\Proc$ for the set of processes, $\QProc$ for the set of
names, and $\id{\{}\vec{y} / \vec{x} \id{\}}$ to denote partial maps,
$s : \QProc \rightarrow \QProc$. A map, $s$ lifts, uniquely, to a map
on process terms, $\widehat{s} : \Proc \rightarrow \Proc$ by the
following equations.

\begin{mathpar}
  (0) \psubstp{Q}{P} := 0 \\
  (R \juxtap S) \psubstp{Q}{P}
  :=    
  (R)\psubstp{Q}{P} \juxtap (S) \psubstp{Q}{P} \\
  (x?(y).R) \psubstp{Q}{P}    
  :=    
  (x)\substp{Q}{P} (z)\concat( (R \psubstn{z}{y}) \psubstp{Q}{P} ) \\
  (\lift{x}{R}) \psubstp{Q}{P}  
  :=
  \lift{(x)\substp{Q}{P}}{ R \psubstp{Q}{P} } \\
%   (\dropn{x})  \psubstp{Q}{P}       
%   := 
%   \left\{ 
%     \begin{array}{ccc} 
%       \dropn{\quotep{Q}} & & x \nameeq \quotep{P} \\
%       \dropn{x} & & otherwise \\
%     \end{array}
%   \right. 
  (\dropn{x})  \psubstp{Q}{P}       
  := 
  \left\{ 
    \begin{array}{ccc} 
      Q & & x \nameeq \quotep{P} \\
      \dropn{x} & & otherwise \\
    \end{array}
  \right.
\end{mathpar}
 

where

\begin{eqnarray}
  (x)\id{\{} \lpquote Q \rpquote / \lpquote P \rpquote \id{\}}            = 
  \left\{ 
    \begin{array}{ccc}
      \lpquote Q \rpquote & & x \nameeq \lpquote P \rpquote \\
      x & & otherwise \\
    \end{array}
  \right. \nonumber
\end{eqnarray}

and $z$ is chosen distinct from $\quotep{P}$, $\quotep{Q}$, the free
names in $Q$, and all the names in $R$. Our $\alpha$-equivalence will
be built in the standard way from this substitution.

\begin{remark}\label{rem:no_self_referential_names}
  One consequence of these definitions is that $\forall P. \quotep{P}
  \not\in \freenames{P}$.
\end{remark}

\subsection{ Dynamic quote: an example }

Anticipating something of what's to come, consider applying the
substitution, $\widehat{\id{\{}u / z \id{\}}}$, to the following pair
of processes, $\lift{w}{y!(z)}$ and $w[ \lpquote y!(z) \rpquote ]$.

\begin{eqnarray}
	\lift{w}{y!(z)}\widehat{\id{\{}u / z \id{\}}}
		& = &
		\lift{w}{y!(u)} \nonumber\\
	w[ \lpquote y!(z) \rpquote ] \widehat{ \id{\{}u / z \id{\}} }
		& = &
		w[ \lpquote y!(z) \rpquote ] \nonumber
\end{eqnarray}

Because the body of the process between quotes is impervious to
substitution, we get radically different answers. In fact, by
examining the first process in an input context,
e.g. $x?(z).\lift{w}{y!(z)}$, we see that the process under the lift
operator may be shaped by prefixed inputs binding a name inside it. In
this sense, the lift operator will be seen as a way to dynamically
construct processes before reifying them as names.

Finally equipped with these standard features we can present the
dynamics of the calculus.

\subsubsection{Operational semantics} 

Finally, we introduce the computational dynamics. What marks these
algebras as distinct from other more traditionally studied algebraic
structures, e.g. vector spaces or polynomial rings, is the manner in
which dynamics is captured. In traditional structures, dynamics is typically
expressed through morphisms between such structures, as in linear maps
between vector spaces or morphisms between rings. In algebras
associated with the semantics of computation, the dynamics is
expressed as part of the algebraic structure itself, through a
reduction reduction relation typically denoted by $\red$. Below, we
give a recursive presentation of this relation for the calculus used
in the encoding.

$\red \subseteq \pi \times \pi$
$\red : \pi \to \mathcal{P}(\pi)$

\begin{mathpar}
  \inferrule* [lab=Comm] { \textsf{match}( x_{src}, x_{trgt} ) } { x_{trgt}?(y)P \; | \; x_{src}!\langle {Q} \rangle \red P\{\quotep{Q}/y}\} }
  \and \\
  \inferrule* [lab=Par] {{P} \red {P}'} {{{P} | {Q}} \red {{P}' | {Q}}}
  \and
  \inferrule* [lab=Equiv]{{{P} \scong {P}'} \andalso {{P}' \red {Q}'} \andalso {{Q}' \scong {Q}}}{{P} \red {Q}}
\end{mathpar}

\begin{eqnarray*}
  match_{\equiv} (\quotep{P},\quotep{Q}) & := & P \equiv Q \\
  match_{\dagger}(\quotep{P},\quotep{Q}) & := & \forall R. P|Q \red^{*} R => R \red^{*} 0 \\
  match_{K}(\quotep{P},\quotep{Q}) & := & K \mbox{ for some context } K
\end{eqnarray*}

$u?(x)P | u!\langle Q \rangle \red P\{\quotep{Q}/x\}$

%We write $\wred$ for $\red^*$, and $P\red$ if $\exists Q $ such that $ P \red Q$.
We write $P\red$ if $\exists Q $ such that $ P \red Q$ and $P\not\red$, otherwise.

\section{Replication}

As mentioned before, it is known that replication (and hence
recursion) can be implemented in a higher-order process algebra
\cite{SangiorgiWalker}. As our first example of calculation with the
machinery thus far presented we give the construction explicitly in
the {\rhoc}.

\begin{eqnarray}
	D_{x} & := & \prefix{x}{y}{(\binpar{\outputp{x}{y}}{@{y}})} \nonumber\\
	\bangp_{x}{P} & := & \binpar{{x}!\langle{\binpar{D_{x}}{P}}\rangle}{D_{x}} \nonumber
\end{eqnarray}

\begin{eqnarray}
	\bangp_{x}{P} & & \nonumber\\
	=
	& {x}!\langle{(\prefix{x}{y}{(\outputp{x}{y} | @{y})) | P}}\rangle 
	      | \prefix{x}{y}{(\outputp{x}{y} | @{y})} & \nonumber\\
	\red
	& (\outputp{x}{y} | @{y})\substn{\quotep{(\prefix{x}{y}{(@{y} | \outputp{x}{y})) | P}}}{y} & \nonumber\\
	=
	& \outputp{x}{\quotep{(\prefix{x}{y}{(\outputp{x}{y} | @{y})) | P}}}
	  | {(\prefix{x}{y}{(\outputp{x}{y} | @{y})) | P}} & \nonumber\\
	\red
	& \ldots & \nonumber\\
	\red^*
	& P | P | \ldots & \nonumber
\end{eqnarray}

Of course, this encoding, as an implementation, runs away, unfolding
$\bangp{P}$ eagerly. A lazier and more implementable replication
operator, restricted to input-guarded processes, may be obtained as follows.

\begin{eqnarray}
\bangp{\prefix{u}{v}{P}} 
	:= 
	\binpar{\lift{x}{\prefix{u}{v}{(\binpar{D(x)}{P})}}}{D(x)} \nonumber
\end{eqnarray}

\begin{remark}
  Note that the lazier definition still does not deal with summation
  or mixed summation (i.e. sums over input and output). The reader is
  invited to construct definitions of replication that deal with these
  features. 

  Further, the definitions are parameterized in a name, $x$. Can you,
  gentle reader, make a definition that eliminates this parameter and
  guarantees no accidental interaction between the replication
  machinery and the process being replicated -- i.e. no accidental
  sharing of names used by the process to get its work done and the
  name(s) used by the replication to effect copying. This latter
  revision of the definition of replication is crucial to obtaining
  the expected identity $!!P \sim !P$.
\end{remark}

\begin{remark}\label{rem:paradoxical_combinator}
  The reader familiar with the lambda calculus will have noticed the
  similarity between $D$ and the paradoxical combinator.

  [Ed. note: the existence of this seems to suggest we have to be more
  restrictive on the set of processes and names we admit if we are to
  support no-cloning.]
\end{remark}

\subsubsection{Bisimulation}

The computational dynamics gives rise to another kind of equivalence,
the equivalence of computational behavior. As previously mentioned
this is typically captured \emph{via} some form of bisimulation.

% The notion we use in this paper is weak barbed bisimulation
% \cite{milner91polyadicpi}.

The notion we use in this paper is derived from weak barbed
bisimulation \cite{milner91polyadicpi}. 

\begin{definition}
An \emph{observation relation}, $\downarrow_{\mathcal N}$, over a set
of names, $\mathcal N$, is the smallest relation satisfying the rules
below.

\infrule[Out-barb]{y \in {\mathcal N}, \; x \nameeq y}
		  {\outputp{x}{v} \downarrow_{\mathcal N} x}
\infrule[Par-barb]{\mbox{$P\downarrow_{\mathcal N} x$ or $Q\downarrow_{\mathcal N} x$}}
		  {\binpar{P}{Q} \downarrow_{\mathcal N} x}

We write $P \Downarrow_{\mathcal N} x$ if there is $Q$ such that 
$P \wred Q$ and $Q \downarrow_{\mathcal N} x$.
\end{definition}

\begin{definition}
%\label{def.bbisim}
An  ${\mathcal N}$-\emph{barbed bisimulation} over a set of names, ${\mathcal N}$, is a symmetric binary relation 
${\mathcal S}_{\mathcal N}$ between agents such that $P\rel{S}_{\mathcal N}Q$ implies:
\begin{enumerate}
\item If $P \red P'$ then $Q \wred Q'$ and $P'\rel{S}_{\mathcal N} Q'$.
\item If $P\downarrow_{\mathcal N} x$, then $Q\Downarrow_{\mathcal N} x$.
\end{enumerate}
$P$ is ${\mathcal N}$-barbed bisimilar to $Q$, written
$P \wbbisim_{\mathcal N} Q$, if $P \rel{S}_{\mathcal N} Q$ for some ${\mathcal N}$-barbed bisimulation ${\mathcal S}_{\mathcal N}$.
\end{definition}

$\mathcal{R} \subseteq \pi \times \pi$

$P \mathcal{R} Q => \forall P'. P \red P' \Rightarrow \exists Q'. Q \red Q', P' \mathcal{R} Q'$

$P \vdash x \Rightarrow Q \vdash x$

\begin{mathpar}
  \inferrule*[lab=Out-barb]{x \nameeq y}{{y}!\langle{Q}\rangle \vdash x}
  \and
  \inferrule*[lab=Par-barb]{\mbox{$P\vdash x$ or $Q\vdash x$}}{\binpar{P}{Q} \vdash x}
\end{mathpar}

\subsubsection{Contexts}

One of the principle advantages of computational calculi like the
$\pi$-calculus is a well-defined notion of context,
contextual-equivalence and a correlation between
contextual-equivalence and notions of bisimulation. The notion of
context allows the decomposition of a process into (sub-)process and
its syntactic environment, its context. Thus, a context may be
thought of as a process with a ``hole'' (written $\Box$) in it. The
application of a context $M$ to a process $P$, written $M[P]$, is
tantamount to filling the hole in $M$ with $P$. In this paper we do
not need the full weight of this theory, but do make use of the notion
of context in the proof the main theorem. 

\begin{mathpar}
  \inferrule* [lab=summation] {} {{M_{M},M_{N}} \bc \Box \;|\; x.M_{A} \;|\; M_{M}+M_{N}}
  \and
  \inferrule* [lab=agent] {} {{M_{A}} \bc (\vec{x})M_{P} \;| \; \clift{P_0,\ldots,M_{P},\ldots,P_N}}
  \and \\
  \inferrule* [lab=process] {} {{M_{P}} \bc M_{N} \;| \;P|M_{P} }
\end{mathpar} 

\begin{mathpar}
  \inferrule* [lab=sychronization] {} {M_{N} \bc \Box \;|\; x?M_{F} \;|\; x!M_{C}}
  \and
  \inferrule* [lab=abstraction] {} {{M_{F}} \bc (x)M_{P} }
  \and
  \inferrule* [lab=concretion] {} {{M_{C}} \bc \langle M_{P} \rangle }
  \and \\
  \inferrule* [lab=process] {} {{M_{P}} \bc M_{N} \;| \;P|M_{P} }
\end{mathpar}

\begin{definition}[contextual application] Given a context $M$, and
  process $P$, we define the \emph{contextual application}, $M[P] :=
  M\{P/\Box\}$. That is, the contextual application of M to P is the
  substitution of $P$ for $\Box$ in $M$.
\end{definition}

$\meaningof{-} : L \to \mathcal{P}(\pi)$

\begin{mathpar}
  \inferrule* [lab=collection] {} {\meaningof{true} = \pi, \and \meaningof{~E} = \pi \setminus \meaningof{E}, \and \meaningof{E_{1} \& E_{2}} = \meaningof{E_{1}} \cap \meaningof{E_{2}}}
\end{mathpar}

\begin{mathpar}
  \inferrule* [lab=structure] {} {\meaningof{0} = \{ P \in \pi | P \equiv 0 \}, \and \\ \meaningof{E_1 | E_2} = \{ P \in \pi | P \equiv P_{1} | P_{2}, P_{1} \in \meaningof{E_{1}}, P_{2} \in \meaningof{E_2}\} }
\end{mathpar}

\begin{mathpar}
 \inferrule* [lab=behavior] {} {\meaningof{\langle a?b \rangle E} = \{ P \in \pi | P \equiv Q | u?(y)P', \\ \and \\\\ \and \\ \;\;\; u \in \meaningof{a}, \forall z.P'\{z/y\} \in \meaningof{E\{z/b\}}\}, \and \\ \meaningof{a!E} = \{ P \in \pi | P \equiv Q | x!\langle P' \rangle, x \in \meaningof{a} P' \in \meaningof{E}\} }
\end{mathpar}

\begin{mathpar}
 \inferrule* [lab=nominal] {} {\meaningof{\quotep{E}} = \{ \quotep{P} \in \quotep{\pi} | P \in \meaningof{E} \}, \and \meaningof{\quotep{P}} = \{ \quotep{Q} \in \quotep{\pi} | P \equiv Q \} \and \\ \meaningof{@\quotep{E}} = \{ P \in \pi | P \equiv @x, x \in \meaningof{E} \}}
\end{mathpar}

\begin{eqnarray*}
  \\
  \meaningof{-} : TS \to ST
\end{eqnarray*}

\begin{eqnarray*}
  \\
  L : TS \to ST
\end{eqnarray*}

\begin{eqnarray*}
  \\
  P \models E \iff P \in \meaningof{E}
\end{eqnarray*}

\begin{eqnarray*}
  P \approx_{L} Q \iff \forall E \in L. P \models E \iff Q \models E
\end{eqnarray*}

\begin{eqnarray*}
  P \approx_{K} Q
\end{eqnarray*}

\begin{eqnarray*}
  P \approx Q
\end{eqnarray*}

$\approx_{K} = \approx = \approx_{L}$

\subsubsection{Contextual duality}

Note that contexts extend the quotation operation to a family of
operations from processes to names. Given a context, $M$, we can
define a \emph{nominal context}, $\quotep{M}$ by $\quotep{M}[P] :=
\quotep{M[P]}$. To foreshadow what is to come we observe that these
operations enjoy a duality with processes very much like the duality
between vectors and maps from vectors to scalars.

Further, because the calculus is essentially higher-order, we have a
correspondence between contexts and processes. More specifically,
given a name $x$ and a context $M$ we can construct $M^{*}_{x}$ such
that 

\begin{mathpar}
  M^{*}_{x} | \lift{x}{P} \red M[P]
\end{mathpar}

namely,

\begin{mathpar}
  M^{*}_{x} := x?(u).M[\dropn{u}]
\end{mathpar}

The dependence of $M^{*}_{x}$ on a name makes it an abstraction, 

\begin{mathpar}
  M^{*} := (x)x?(u).M[\dropn{u}]
\end{mathpar}

\subsection{Additional notation}

It will sometimes be convenient to denote the process a name
quotes. We already have the notation $x = \quotep{P}$, but it will be
convenient to introduce an alternate notation, $\procn{x}$, when we
want to emphasize the connection to the use of the name. Note that, by
virtue of name equivalence, $\quotep{\procn{x}} \nameeq x$; so, the
notation is consistent with previous definitions.

Further, because names have structure it is possible to effect
substitutions on the basis of that structure. This means we need to
upgrade our notation for substitutions, which we accomplish by
adapting comprehension notation. Thus,

\begin{mathpar}
  P\{ y / x : x \in S \}
\end{mathpar}

is interpreted to mean the process derived from P by replacing (in a
capture-avoiding manner) each occurrence of $x$ in $S$ by $y$. For example,

\begin{mathpar}
  P\{ \quotep{\procn{x}|\procn{x}} / x : x \in \freenames{P} \}
\end{mathpar}

will replace each (occurrence) of a free name $x$ in $P$ by
$\quotep{\procn{x}|\procn{x}}$.

Also, we will avail ourselves of the notation $x^{L}$ and $x^{R}$ to
denote injections of a name into disjoint copies of the name
space. There are numerous ways to accomplish this. One example can be
found in \cite{MeredithR05}. This notation overloads to vectors of
names: $\vec{x}^{\pi} := (x_{i}^{\pi} \; : \; 0 \leq i < |\vec{x}| )$ where $\pi \in \{L,R\}$.

We also use $P^{\Box} := P|\Box$.

In \cite{MeredithR05} an interpretation of the new operator is
given. It turns out that there are several possible interpretations
all enjoying the requisite algebraic properties of the operator (see
\cite{milner91polyadicpi}). We will therefore make liberal use of
$(\nu\; \vec{x})P$.

% subsection the_syntax_and_semantics_of_the_notation_system (end)   

\input{qm2pi.qmops} 

\input{qm2pi.sterngerlach} 

\input{qm2pi.metric} 

% section concurrent_process_calculi (end)

%\input{qm2pi.proofsketch}

% section proof sketch (end)

%\input{qm2pi.slviaknots} 

% section spatial logic via knots (end)

\input{qm2pi.conclusion}

% section conclusion (end)

%\input{qm2pi.dtcodes} 

% section wiring algorithm (end)

\input{qm2pi.ack} 

% section acknowledgments (end)

\newpage


\bibliographystyle{plain}   
\bibliography{../../biblios/main.bib}

\input{qm2pi.rhodetails}

\end{document}

 

% subsection basic_interpretation (end)

%\input{qm2pi.rho.presentation} 
\subsection{The syntax and semantics of the notation system}\label{sub:the_syntax_and_semantics_of_the_notation_system} % (fold)

We now summarize a technical presentation of the calculus that
embodies our theory of dynamics. The typical presentation of such a
calculus follows the style of giving generators and relations on
them. The grammar, below, describing term constructors, freely
generates the set of processes, $\Proc$. This set is then quotiented
by a relation known as structural congruence and it is over this set
that the notion of dynamics is expressed. This presentation is
essentially that of \cite{MeredithR05} with the addition of
polyadicity and summation. For readability we have relegated some of
the technical subtleties to an appendix.

\subsubsection{Process grammar}\label{subsub:process_grammar}

\begin{mathpar}
  \inferrule* [lab=synchronization] {} {{M} \bc \pzero \;|\; x?F \;|\; x!C }
  \and
  \inferrule* [lab=abstraction] {} {{F} \bc (x)P}
  \and
  \inferrule* [lab=concretion] {} {{C} \bc \langle Q \rangle}
  \and
  \inferrule* [lab=process] {} {{P,Q} \bc M \;| \;P|Q \;|\; @{x}}
  \and
  \inferrule* [lab=name] {} {{x} \bc \quotep{P}}
\end{mathpar} 

Note that $\vec{x}$ (resp. $\vec{P}$) denotes a vector of names
(resp. processes) of length $|\vec{x}|$ (resp. $|\vec{P}|$). We adopt
the following useful abbreviations.

\begin{mathpar}
   x?(\vec{y}).P := x.(\vec{y})P \and  x\clift{\vec{P}} := x.\clift{\vec{P}}
   \and x!(y) := \lift{x}{\dropn{y}}
   \and \Pi_{i=0}^{n-1}P_i := P_0 | \ldots | P_{n-1}
\end{mathpar}

\subsubsection{Structural congruence}

\paragraph{Free and bound names and alpha-equivalence.} At the
core of structural equivalence is alpha-equivalence which identifies
process that are the same up to a change of variable. Formally, we
recognize the distinction between free and bound names. The free names
of a process, $\freenames{P}$, may be calculated recursively as
follows:

\begin{mathpar}
\freenames{\pzero} := \emptyset
  \and \\
  \freenames{x?(y).P} := \{ x \} \cup (\freenames{P} \setminus \{ y \})
  \and 
  \freenames{x!\langle P \rangle} := \{ x \} \cup \{ P \} 
  \and \\
  \freenames{P|Q} := \freenames{P} \cup \freenames{Q}
  \and \\
  \freenames{@{x}} := \{ x \}
\end{mathpar}

$\pi$
$\quotep{\pi}$

$\freenames{-} : \pi \to \mathcal{P}(\quotep{\pi})$

\begin{eqnarray*}
  \freenames{\pzero} & := & \emptyset \\
  \freenames{x?(y).P} & := & \{ x \} \cup (\freenames{P} \setminus \{ y \}) \\
  \freenames{x!\langle P \rangle} & := & \{ x \} \cup \{ P \} \\
  \freenames{P|Q} & := & \freenames{P} \cup \freenames{Q} \\
  \freenames{\dropn{x}} & := & \{ x \}
\end{eqnarray*}

The bound names of a process, $\boundnames{P}$, are those names occurring in $P$
that are not free. For example, in $x?(y).0$, the name $x$ is free, while $y$ is bound.

\begin{mathpar}
  \inferrule* [lab=monoidal-laws] {} { P|Q \equiv Q|P \and P|0 \equiv P \and P|(Q|R) \equiv (P|Q)|R }
\end{mathpar}

\begin{mathpar}
  \inferrule* [lab=alpha-equivalence] {} { (x)P \equiv (y)P\{y/x\} \and y \not\in \freenames{P} }
\end{mathpar}

\begin{definition}
Then two processes, $P,Q$, are alpha-equivalent if $P = Q\{\vec{y}/\vec{x}\}$ for
some $\vec{x} \in \boundnames{Q},\vec{y} \in \boundnames{P}$, where $Q\{\vec{y}/\vec{x}\}$
denotes the capture-avoiding substitution of $\vec{y}$ for $\vec{x}$ in $Q$.
\end{definition}

\begin{definition}
  The {\em structural congruence} \cite{SangiorgiWalker} , $\equiv$,
  between processes is the least congruence containing
  alpha-equivalence, satisfying the abelian monoid laws
  (associativity, commutativity and $\pzero$ as identity) for parallel
  composition $|$ and for summation $+$.
\end{definition}

\subsection{Name equivalence}

We take name equivalence, written $\nameeq$, to be the smallest
equivalence relation generated by the following rules.

\begin{mathpar}
\inferrule*[lab=Quote-drop]
{ }
{ \quotep{@{x}} \nameeq x }

\inferrule*[lab=Struct-equiv]
{ P \scong Q }
{ \quotep{P} \nameeq \quotep{Q} }
\end{mathpar}

The astute reader will have noticed that the mutual recursion of names
and processes imposes a mutual recursion on alpha-equivalence and
structural equivalence via name-equivalence. Fortunately, all of this
works out pleasantly and we may calculate in the natural way, free of
concern. The reader interested in the details is referred to the
appendix \ref{appendix:rho_details}.

\subsection{Substitution}

We use $\Proc$ for the set of processes, $\QProc$ for the set of
names, and $\id{\{}\vec{y} / \vec{x} \id{\}}$ to denote partial maps,
$s : \QProc \rightarrow \QProc$. A map, $s$ lifts, uniquely, to a map
on process terms, $\widehat{s} : \Proc \rightarrow \Proc$ by the
following equations.

\begin{mathpar}
  (0) \psubstp{Q}{P} := 0 \\
  (R \juxtap S) \psubstp{Q}{P}
  :=    
  (R)\psubstp{Q}{P} \juxtap (S) \psubstp{Q}{P} \\
  (x?(y).R) \psubstp{Q}{P}    
  :=    
  (x)\substp{Q}{P} (z)\concat( (R \psubstn{z}{y}) \psubstp{Q}{P} ) \\
  (\lift{x}{R}) \psubstp{Q}{P}  
  :=
  \lift{(x)\substp{Q}{P}}{ R \psubstp{Q}{P} } \\
%   (\dropn{x})  \psubstp{Q}{P}       
%   := 
%   \left\{ 
%     \begin{array}{ccc} 
%       \dropn{\quotep{Q}} & & x \nameeq \quotep{P} \\
%       \dropn{x} & & otherwise \\
%     \end{array}
%   \right. 
  (\dropn{x})  \psubstp{Q}{P}       
  := 
  \left\{ 
    \begin{array}{ccc} 
      Q & & x \nameeq \quotep{P} \\
      \dropn{x} & & otherwise \\
    \end{array}
  \right.
\end{mathpar}
 

where

\begin{eqnarray}
  (x)\id{\{} \lpquote Q \rpquote / \lpquote P \rpquote \id{\}}            = 
  \left\{ 
    \begin{array}{ccc}
      \lpquote Q \rpquote & & x \nameeq \lpquote P \rpquote \\
      x & & otherwise \\
    \end{array}
  \right. \nonumber
\end{eqnarray}

and $z$ is chosen distinct from $\quotep{P}$, $\quotep{Q}$, the free
names in $Q$, and all the names in $R$. Our $\alpha$-equivalence will
be built in the standard way from this substitution.

\begin{remark}\label{rem:no_self_referential_names}
  One consequence of these definitions is that $\forall P. \quotep{P}
  \not\in \freenames{P}$.
\end{remark}

\subsection{ Dynamic quote: an example }

Anticipating something of what's to come, consider applying the
substitution, $\widehat{\id{\{}u / z \id{\}}}$, to the following pair
of processes, $\lift{w}{y!(z)}$ and $w[ \lpquote y!(z) \rpquote ]$.

\begin{eqnarray}
	\lift{w}{y!(z)}\widehat{\id{\{}u / z \id{\}}}
		& = &
		\lift{w}{y!(u)} \nonumber\\
	w[ \lpquote y!(z) \rpquote ] \widehat{ \id{\{}u / z \id{\}} }
		& = &
		w[ \lpquote y!(z) \rpquote ] \nonumber
\end{eqnarray}

Because the body of the process between quotes is impervious to
substitution, we get radically different answers. In fact, by
examining the first process in an input context,
e.g. $x?(z).\lift{w}{y!(z)}$, we see that the process under the lift
operator may be shaped by prefixed inputs binding a name inside it. In
this sense, the lift operator will be seen as a way to dynamically
construct processes before reifying them as names.

Finally equipped with these standard features we can present the
dynamics of the calculus.

\subsubsection{Operational semantics} 

Finally, we introduce the computational dynamics. What marks these
algebras as distinct from other more traditionally studied algebraic
structures, e.g. vector spaces or polynomial rings, is the manner in
which dynamics is captured. In traditional structures, dynamics is typically
expressed through morphisms between such structures, as in linear maps
between vector spaces or morphisms between rings. In algebras
associated with the semantics of computation, the dynamics is
expressed as part of the algebraic structure itself, through a
reduction reduction relation typically denoted by $\red$. Below, we
give a recursive presentation of this relation for the calculus used
in the encoding.

$\red \subseteq \pi \times \pi$
$\red : \pi \to \mathcal{P}(\pi)$

\begin{mathpar}
  \inferrule* [lab=Comm] { \textsf{match}( x_{src}, x_{trgt} ) } { x_{trgt}?(y)P \; | \; x_{src}!\langle {Q} \rangle \red P\{\quotep{Q}/y}\} }
  \and \\
  \inferrule* [lab=Par] {{P} \red {P}'} {{{P} | {Q}} \red {{P}' | {Q}}}
  \and
  \inferrule* [lab=Equiv]{{{P} \scong {P}'} \andalso {{P}' \red {Q}'} \andalso {{Q}' \scong {Q}}}{{P} \red {Q}}
\end{mathpar}

\begin{eqnarray*}
  match_{\equiv} (\quotep{P},\quotep{Q}) & := & P \equiv Q \\
  match_{\dagger}(\quotep{P},\quotep{Q}) & := & \forall R. P|Q \red^{*} R => R \red^{*} 0 \\
  match_{K}(\quotep{P},\quotep{Q}) & := & K \mbox{ for some context } K
\end{eqnarray*}

$u?(x)P | u!\langle Q \rangle \red P\{\quotep{Q}/x\}$

%We write $\wred$ for $\red^*$, and $P\red$ if $\exists Q $ such that $ P \red Q$.
We write $P\red$ if $\exists Q $ such that $ P \red Q$ and $P\not\red$, otherwise.

\section{Replication}

As mentioned before, it is known that replication (and hence
recursion) can be implemented in a higher-order process algebra
\cite{SangiorgiWalker}. As our first example of calculation with the
machinery thus far presented we give the construction explicitly in
the {\rhoc}.

\begin{eqnarray}
	D_{x} & := & \prefix{x}{y}{(\binpar{\outputp{x}{y}}{@{y}})} \nonumber\\
	\bangp_{x}{P} & := & \binpar{{x}!\langle{\binpar{D_{x}}{P}}\rangle}{D_{x}} \nonumber
\end{eqnarray}

\begin{eqnarray}
	\bangp_{x}{P} & & \nonumber\\
	=
	& {x}!\langle{(\prefix{x}{y}{(\outputp{x}{y} | @{y})) | P}}\rangle 
	      | \prefix{x}{y}{(\outputp{x}{y} | @{y})} & \nonumber\\
	\red
	& (\outputp{x}{y} | @{y})\substn{\quotep{(\prefix{x}{y}{(@{y} | \outputp{x}{y})) | P}}}{y} & \nonumber\\
	=
	& \outputp{x}{\quotep{(\prefix{x}{y}{(\outputp{x}{y} | @{y})) | P}}}
	  | {(\prefix{x}{y}{(\outputp{x}{y} | @{y})) | P}} & \nonumber\\
	\red
	& \ldots & \nonumber\\
	\red^*
	& P | P | \ldots & \nonumber
\end{eqnarray}

Of course, this encoding, as an implementation, runs away, unfolding
$\bangp{P}$ eagerly. A lazier and more implementable replication
operator, restricted to input-guarded processes, may be obtained as follows.

\begin{eqnarray}
\bangp{\prefix{u}{v}{P}} 
	:= 
	\binpar{\lift{x}{\prefix{u}{v}{(\binpar{D(x)}{P})}}}{D(x)} \nonumber
\end{eqnarray}

\begin{remark}
  Note that the lazier definition still does not deal with summation
  or mixed summation (i.e. sums over input and output). The reader is
  invited to construct definitions of replication that deal with these
  features. 

  Further, the definitions are parameterized in a name, $x$. Can you,
  gentle reader, make a definition that eliminates this parameter and
  guarantees no accidental interaction between the replication
  machinery and the process being replicated -- i.e. no accidental
  sharing of names used by the process to get its work done and the
  name(s) used by the replication to effect copying. This latter
  revision of the definition of replication is crucial to obtaining
  the expected identity $!!P \sim !P$.
\end{remark}

\begin{remark}\label{rem:paradoxical_combinator}
  The reader familiar with the lambda calculus will have noticed the
  similarity between $D$ and the paradoxical combinator.

  [Ed. note: the existence of this seems to suggest we have to be more
  restrictive on the set of processes and names we admit if we are to
  support no-cloning.]
\end{remark}

\subsubsection{Bisimulation}

The computational dynamics gives rise to another kind of equivalence,
the equivalence of computational behavior. As previously mentioned
this is typically captured \emph{via} some form of bisimulation.

% The notion we use in this paper is weak barbed bisimulation
% \cite{milner91polyadicpi}.

The notion we use in this paper is derived from weak barbed
bisimulation \cite{milner91polyadicpi}. 

\begin{definition}
An \emph{observation relation}, $\downarrow_{\mathcal N}$, over a set
of names, $\mathcal N$, is the smallest relation satisfying the rules
below.

\infrule[Out-barb]{y \in {\mathcal N}, \; x \nameeq y}
		  {\outputp{x}{v} \downarrow_{\mathcal N} x}
\infrule[Par-barb]{\mbox{$P\downarrow_{\mathcal N} x$ or $Q\downarrow_{\mathcal N} x$}}
		  {\binpar{P}{Q} \downarrow_{\mathcal N} x}

We write $P \Downarrow_{\mathcal N} x$ if there is $Q$ such that 
$P \wred Q$ and $Q \downarrow_{\mathcal N} x$.
\end{definition}

\begin{definition}
%\label{def.bbisim}
An  ${\mathcal N}$-\emph{barbed bisimulation} over a set of names, ${\mathcal N}$, is a symmetric binary relation 
${\mathcal S}_{\mathcal N}$ between agents such that $P\rel{S}_{\mathcal N}Q$ implies:
\begin{enumerate}
\item If $P \red P'$ then $Q \wred Q'$ and $P'\rel{S}_{\mathcal N} Q'$.
\item If $P\downarrow_{\mathcal N} x$, then $Q\Downarrow_{\mathcal N} x$.
\end{enumerate}
$P$ is ${\mathcal N}$-barbed bisimilar to $Q$, written
$P \wbbisim_{\mathcal N} Q$, if $P \rel{S}_{\mathcal N} Q$ for some ${\mathcal N}$-barbed bisimulation ${\mathcal S}_{\mathcal N}$.
\end{definition}

$\mathcal{R} \subseteq \pi \times \pi$

$P \mathcal{R} Q => \forall P'. P \red P' \Rightarrow \exists Q'. Q \red Q', P' \mathcal{R} Q'$

$P \vdash x \Rightarrow Q \vdash x$

\begin{mathpar}
  \inferrule*[lab=Out-barb]{x \nameeq y}{{y}!\langle{Q}\rangle \vdash x}
  \and
  \inferrule*[lab=Par-barb]{\mbox{$P\vdash x$ or $Q\vdash x$}}{\binpar{P}{Q} \vdash x}
\end{mathpar}

\subsubsection{Contexts}

One of the principle advantages of computational calculi like the
$\pi$-calculus is a well-defined notion of context,
contextual-equivalence and a correlation between
contextual-equivalence and notions of bisimulation. The notion of
context allows the decomposition of a process into (sub-)process and
its syntactic environment, its context. Thus, a context may be
thought of as a process with a ``hole'' (written $\Box$) in it. The
application of a context $M$ to a process $P$, written $M[P]$, is
tantamount to filling the hole in $M$ with $P$. In this paper we do
not need the full weight of this theory, but do make use of the notion
of context in the proof the main theorem. 

\begin{mathpar}
  \inferrule* [lab=summation] {} {{M_{M},M_{N}} \bc \Box \;|\; x.M_{A} \;|\; M_{M}+M_{N}}
  \and
  \inferrule* [lab=agent] {} {{M_{A}} \bc (\vec{x})M_{P} \;| \; \clift{P_0,\ldots,M_{P},\ldots,P_N}}
  \and \\
  \inferrule* [lab=process] {} {{M_{P}} \bc M_{N} \;| \;P|M_{P} }
\end{mathpar} 

\begin{mathpar}
  \inferrule* [lab=sychronization] {} {M_{N} \bc \Box \;|\; x?M_{F} \;|\; x!M_{C}}
  \and
  \inferrule* [lab=abstraction] {} {{M_{F}} \bc (x)M_{P} }
  \and
  \inferrule* [lab=concretion] {} {{M_{C}} \bc \langle M_{P} \rangle }
  \and \\
  \inferrule* [lab=process] {} {{M_{P}} \bc M_{N} \;| \;P|M_{P} }
\end{mathpar}

\begin{definition}[contextual application] Given a context $M$, and
  process $P$, we define the \emph{contextual application}, $M[P] :=
  M\{P/\Box\}$. That is, the contextual application of M to P is the
  substitution of $P$ for $\Box$ in $M$.
\end{definition}

$\meaningof{-} : L \to \mathcal{P}(\pi)$

\begin{mathpar}
  \inferrule* [lab=collection] {} {\meaningof{true} = \pi, \and \meaningof{~E} = \pi \setminus \meaningof{E}, \and \meaningof{E_{1} \& E_{2}} = \meaningof{E_{1}} \cap \meaningof{E_{2}}}
\end{mathpar}

\begin{mathpar}
  \inferrule* [lab=structure] {} {\meaningof{0} = \{ P \in \pi | P \equiv 0 \}, \and \\ \meaningof{E_1 | E_2} = \{ P \in \pi | P \equiv P_{1} | P_{2}, P_{1} \in \meaningof{E_{1}}, P_{2} \in \meaningof{E_2}\} }
\end{mathpar}

\begin{mathpar}
 \inferrule* [lab=behavior] {} {\meaningof{\langle a?b \rangle E} = \{ P \in \pi | P \equiv Q | u?(y)P', \\ \and \\\\ \and \\ \;\;\; u \in \meaningof{a}, \forall z.P'\{z/y\} \in \meaningof{E\{z/b\}}\}, \and \\ \meaningof{a!E} = \{ P \in \pi | P \equiv Q | x!\langle P' \rangle, x \in \meaningof{a} P' \in \meaningof{E}\} }
\end{mathpar}

\begin{mathpar}
 \inferrule* [lab=nominal] {} {\meaningof{\quotep{E}} = \{ \quotep{P} \in \quotep{\pi} | P \in \meaningof{E} \}, \and \meaningof{\quotep{P}} = \{ \quotep{Q} \in \quotep{\pi} | P \equiv Q \} \and \\ \meaningof{@\quotep{E}} = \{ P \in \pi | P \equiv @x, x \in \meaningof{E} \}}
\end{mathpar}

\begin{eqnarray*}
  \\
  \meaningof{-} : TS \to ST
\end{eqnarray*}

\begin{eqnarray*}
  \\
  L : TS \to ST
\end{eqnarray*}

\begin{eqnarray*}
  \\
  P \models E \iff P \in \meaningof{E}
\end{eqnarray*}

\begin{eqnarray*}
  P \approx_{L} Q \iff \forall E \in L. P \models E \iff Q \models E
\end{eqnarray*}

\begin{eqnarray*}
  P \approx_{K} Q
\end{eqnarray*}

\begin{eqnarray*}
  P \approx Q
\end{eqnarray*}

$\approx_{K} = \approx = \approx_{L}$

\subsubsection{Contextual duality}

Note that contexts extend the quotation operation to a family of
operations from processes to names. Given a context, $M$, we can
define a \emph{nominal context}, $\quotep{M}$ by $\quotep{M}[P] :=
\quotep{M[P]}$. To foreshadow what is to come we observe that these
operations enjoy a duality with processes very much like the duality
between vectors and maps from vectors to scalars.

Further, because the calculus is essentially higher-order, we have a
correspondence between contexts and processes. More specifically,
given a name $x$ and a context $M$ we can construct $M^{*}_{x}$ such
that 

\begin{mathpar}
  M^{*}_{x} | \lift{x}{P} \red M[P]
\end{mathpar}

namely,

\begin{mathpar}
  M^{*}_{x} := x?(u).M[\dropn{u}]
\end{mathpar}

The dependence of $M^{*}_{x}$ on a name makes it an abstraction, 

\begin{mathpar}
  M^{*} := (x)x?(u).M[\dropn{u}]
\end{mathpar}

\subsection{Additional notation}

It will sometimes be convenient to denote the process a name
quotes. We already have the notation $x = \quotep{P}$, but it will be
convenient to introduce an alternate notation, $\procn{x}$, when we
want to emphasize the connection to the use of the name. Note that, by
virtue of name equivalence, $\quotep{\procn{x}} \nameeq x$; so, the
notation is consistent with previous definitions.

Further, because names have structure it is possible to effect
substitutions on the basis of that structure. This means we need to
upgrade our notation for substitutions, which we accomplish by
adapting comprehension notation. Thus,

\begin{mathpar}
  P\{ y / x : x \in S \}
\end{mathpar}

is interpreted to mean the process derived from P by replacing (in a
capture-avoiding manner) each occurrence of $x$ in $S$ by $y$. For example,

\begin{mathpar}
  P\{ \quotep{\procn{x}|\procn{x}} / x : x \in \freenames{P} \}
\end{mathpar}

will replace each (occurrence) of a free name $x$ in $P$ by
$\quotep{\procn{x}|\procn{x}}$.

Also, we will avail ourselves of the notation $x^{L}$ and $x^{R}$ to
denote injections of a name into disjoint copies of the name
space. There are numerous ways to accomplish this. One example can be
found in \cite{MeredithR05}. This notation overloads to vectors of
names: $\vec{x}^{\pi} := (x_{i}^{\pi} \; : \; 0 \leq i < |\vec{x}| )$ where $\pi \in \{L,R\}$.

We also use $P^{\Box} := P|\Box$.

In \cite{MeredithR05} an interpretation of the new operator is
given. It turns out that there are several possible interpretations
all enjoying the requisite algebraic properties of the operator (see
\cite{milner91polyadicpi}). We will therefore make liberal use of
$(\nu\; \vec{x})P$.

% subsection the_syntax_and_semantics_of_the_notation_system (end)   

\section{Interpretation of QM}
\subsection{Supporting definitions}
\subsubsection{Multiplication}
\begin{mathpar}
  \quotep{Q} \cdot \quotep{R} := \quotep{Q|R}
  \and \\
  \quotep{Q} \cdot P := P\{ \quotep{Q|R} / \quotep{R} : \quotep{R} \in \freenames{P} \}
\end{mathpar}

\paragraph{Discussion}
The first line needs little explanation. The second line says that
each free name of the process is replaced with the multiplication of
that name by the scalar. Multiplication of a scalar (name) by a state
(process) results in a process all the names of which have been `moved
over' by parallel composition with the process the scalar
quotes. There is a subtlety that the bound names have to be
manipulated so that multiplied names aren't accidentally
captured. There are many ways to achieve this.

\begin{remark}\label{rem:multiplication_identities}
  The reader is invited to verify that for all $x,y,z \in \QProc$ and $P \in \Proc$
  \begin{mathpar}
    x \cdot \quotep{0} \equiv x 
    \and
    x \cdot y \equiv y \cdot x
    \and
    x \cdot (y \cdot z) \equiv (x \cdot y) \cdot z
    \and \\
    \quotep{0} \cdot P \equiv P
    \and \\
    x \cdot (y \cdot P) \equiv (x \cdot y) \cdot P
    \and \\
    x \cdot (P|Q) \equiv (x \cdot P) | (x \cdot Q)
    \and \\    
  \end{mathpar}
\end{remark}

\subsubsection{Tensor product}

We define a tensor product on processes by structural induction.

\paragraph{Tensor of sums} First note that all summations, including
$\pzero$ and sequence, can be written $\Sigma_{i} x_{i}.A_{i} +
\Sigma_{j} x_{j}.C_{j}$, where we have grouped input-guarded processes
together and output-guarded processes together.

Thus, we can define the tensor product of two summations, $N_{1}\otimes N_{2}$, where

\begin{mathpar}
  N_{1} := \Sigma_{i} x_{i}.A_{i} + \Sigma_{j} x_{j}.C_{j}
  \and
  N_{2} := \Sigma_{i'} y_{i'}.B_{i'} + \Sigma_{j'} y_{j'}.D_{j'} 
\end{mathpar}

as follows.

\begin{mathpar}
  \Sigma_{i} x_{i}.A_{i} + \Sigma_{j} x_{j}.C_{j} \otimes \Sigma_{i'}
  y_{i'}.B_{i'} + \Sigma_{j'} y_{j'}.D_{j'} 
  \and \\
  := \; \Sigma_{i} \Sigma_{i'} \quotep{\stackrel{\vee}{x_{i}}| \stackrel{\vee}{y_{i'}}}.(A_{i}\otimes B_{i'}) \; | \; \Sigma_{i'} \Sigma_{i} \quotep{\stackrel{\vee}{y_{i'}}|\stackrel{\vee}{x_{i}}}.(B_{i'}\otimes A_{i})
  \and
  \;\; | \;\; \Sigma_{j} \Sigma_{j'} \quotep{\stackrel{\vee}{x_{j}}|\stackrel{\vee}{y_{j'}}}.(A_{j}\otimes B_{j'}) \; | \; \Sigma_{j'} \Sigma_{j} \quotep{\stackrel{\vee}{y_{j'}}|\stackrel{\vee}{x_{j}}}.(B_{j'}\otimes A_{j})
\end{mathpar}

\begin{remark}
  Do we need to $x^{L}$ and $y^{R}$ for this construction as well?
\end{remark}

\paragraph{Tensor of parallel compositions} Next, we distribute tensor
over par.

\begin{mathpar}
  P_{1}|P_{2} \otimes Q_{1}|Q_{2} := (P_{1} \otimes Q_{1}) | (P_{1}
  \otimes Q_{2}) | (P_{2} \otimes Q_{1}) | (P_{2} \otimes Q_{2})
\end{mathpar}

\paragraph{Tensor with dropped names} We treat tensor of a
process with a dropped name as parallel composition.

\begin{mathpar}
  P \otimes \dropn{x} := P | \dropn{x}
\end{mathpar}

\paragraph{Tensor of agents}

Finally, we need to define tensor on agents. Note that the definition
of tensor on normal products only tensors inputs with inputs and
outputs with outputs. Thus, we only have to define the operation on
``homogeneous'' pairings.

\begin{mathpar}
  (\vec{x})P \otimes (\vec{y})Q
  \and \\
  := (x_{0}^{L}|y_{0}^{R},\ldots,x_{0}^{L}|y_{n}^{R},\ldots,x_{m}^{L}|y_{0}^{R},\ldots,x_{m}^{L}|y_{n}^R)(P\{ \vec{x}^{L}/\vec{x}\} \otimes Q \{ \vec{y}^{R}/\vec{y}\})
  \and \\
  \clift{\vec{P}} \otimes \clift{\vec{Q}}
  \and \\
  := \clift{P_{0}\otimes Q_{0},\ldots,P_{0}\otimes Q_{n},\ldots,P_{m}\otimes Q_{0},\ldots,P_{m}\otimes Q_{n}}
\end{mathpar}

\begin{remark}
  Observe that arities of tensored abstractions matches arities of
  tensored concretions if the original arities matched. Note also that
  the length of the arities corresponds to the increase in dimension
  we see in ordinary vector space tensor product.
\end{remark}

\begin{remark}
  Operationally, this definition distributes the tensor down to
  components ``linked'' by summation. Tensor over summation is
  intriguing in that it mixes names. Moreover, as a consequence of the
  way it mixes names we have the identities for all $x \in \QProc$ and
  $P,Q \in \Proc$

  \begin{mathpar}
    (x \cdot P) \otimes Q \equiv x \cdot (P \otimes Q) \equiv P \otimes (x \cdot Q)
    \and
    P \otimes \pzero \equiv P
  \end{mathpar}

  that the reader is invited to verify.
\end{remark}

\subsubsection{Annihilation}
\begin{mathpar}
  P^{\perp} := \{ Q | \forall R. P|Q \red^{*} R \Rightarrow R \red^{*} \pzero \}
  \and \\
  P^{\underline{\perp}} := \Sigma_{Q \in P^{\perp}} \quotep{Q}?(y).(\dropn{y}|Q) | \Sigma_{Q \in P^{\perp}} \quotep{Q}\clift{\Box}
\end{mathpar}

\paragraph{Discussion} The reader will note that $P^{\perp}$ is a
\emph{set} of processes, while $P^{\underline{\perp}}$ is a
\emph{context}. We call the set $P^{\perp}$ the \emph{annihilators} of
$P$. The parallel composition of a process in the annihilators of $P$
with $P$ will result in a process, the state space of which has all
paths eventually leading to $\pzero$. Execution may endure loops; but
under reasonable conditions of fairness (naturally guaranteed under
most notions of bisimulation) such a composite process cannot get
stuck in such a loop and will, eventually pop out and terminate.

The context $P^{\underline{\perp}}$ is ready and willing to ``take the
$P$ out of'' the process to which it is applied. It will effectively
transmit the code of the process to which it is applied to one of the
annihilators and run the process against it.

\subsubsection{Evaluation}
We fix $M$ a domain of fully abstract interpretation with an equality
coincident with bisimulation. We take $\meaningof{\cdot} : \Proc \to
M$ to be the map interpreting processes and $\nmeaningof{\cdot} : \M
\to Proc$ to be the map running the other way. Then we define

\begin{mathpar}
  \int P := \nmeaningof{\meaningof{P}}
\end{mathpar}

\paragraph{Discussion}
There are many fully abstract interpretations of Milner's
$\pi$-calculus. Any of them can be used as a basis for interpreting
the reflective calculus here. Equipped with such a domain it is
largely a matter of grinding through to check that the Yoneda
construction for the normalization-by-evaluation program can be
extended to this setting.

\begin{remark}
  The reader is invited to verify that $\int (P^{\underline{\perp}}[P]) = 0$.
\end{remark}

\subsection{Quantum mechanics}

Table \ref{tbl:core_qm_op_defns} gives the core operational definitions

\begin{table}[htp]\label{tbl:core_qm_op_defns}
  \center{
    \fbox{
      \begin{tabular}{c|c}
        quantum mechanics & process calculus \\
        \hline
        scalar & $x := \quotep{P}$ \\
        state vector & $\state{P} := P$ \\
        dual & $\state{P}^{*} := \event{P^{\underline{\perp}}} := \quotep{P^{\underline{\perp}}}[-]$ \\
        matrix & $ \Sigma_{\alpha} \state{P_{\alpha}}x_{\alpha}\event{Q_{\alpha}}$ \\
        vector addition & $\state{P} + \state{Q} := \state{P | Q}$ \\
        tensor product & $\state{P} \otimes \state{Q} := \state{P \otimes Q}$ \\
        inner product & $\innerprod{P}{Q} := \quotep{\int P^{\underline{\perp}}[Q]}$ \\
      \end{tabular}
    }
  }
  \caption{QM - operational definitions}
\end{table}

where

\begin{mathpar}
  \prmatrix{P}{Q} := \fprmatrix{P}{\quotep{\pzero}}{Q}
  \and
  \fprmatrix{P}{x}{Q} := (\state{P},x,\event{Q})
  \and
  (\fprmatrix{P}{x}{Q})(\state{R}) := x \cdot \innerprod{Q}{R} \cdot \state{P}
  \and
  (\fprmatrix{P}{x}{Q})(\event{R}) := x \cdot \innerprod{R}{P} \cdot \event{Q}
\end{mathpar}

\paragraph{Discussion}
As promised: vectors (aka states) are represented as processes; duals
as contextual duals; inner product definition should be compared with
standard inner product definition for ....

\begin{remark}
  Assuming $\int (P^{\underline{\perp}}[P]) = 0$, the reader is
  invited to verify that $(\fprmatrix{P}{x}{P})(\state{P}) = x \cdot \state{P}$.
\end{remark}

\begin{remark}
  The reader is invited to verify that $\innerprod{P}{Q}$ could
  equally well have been written $\quotep{\int \stackrel{\vee}{x}}$
  where $x = \event{P^{\underline{\perp}}}(Q)$.

  One of the motivations for this remark is that there is another way
  to factor these operations. We could package up evaluation in the dual:

  \begin{mathpar}
    \state{P}^{*} := \event{\int P^{\underline{\perp}}} := \quotep{\int P^{\underline{\perp}}}[-]
  \end{mathpar}

  and then have inner product defined by
  
  \begin{mathpar}
    \innerprod{P}{Q} := \event{P}(Q)
  \end{mathpar}

  Hopefully, experience with the calculations will provide guidance on
  the best factoring.
\end{remark}

\begin{remark}
  Assuming $\int (P^{\underline{\perp}}[P]) = 0$, the reader is
  invited to verify that $\forall P,Q. (\prmatrix{0}{Q})(\state{0}) =
  \state{0}$ and dually $(\prmatrix{P}{0})(\event{0}) = \event{0}$.
\end{remark}

\begin{remark}
  i'm a little worried that i don't (yet) have proper support for
  complex conjugacy. But, the observation above may give us a
  clue. According to Abramsky, it must be the case that the scalars
  are iso to the homset of the identity for the tensor -- which the
  observation above characterizes. 

  For now, we will simply bookmark the notion with $\overline{x}$.
\end{remark}

\subsubsection{Adjointness}

We need to give a definition of $(\cdot)^{\dagger}$ for matrices. The
obvious candidate definition is
\begin{mathpar}
(\Sigma_{\alpha}\fprmatrix{P_{\alpha}}{x_{\alpha}}{Q_{\alpha}})^{\dagger}
= \Sigma_{\alpha}\fprmatrix{(Q_{\alpha}^{\underline{\perp}})^{*}}{\overline{x}_{\alpha}}{P_{\alpha}^{\underline{\perp}}} 
\end{mathpar}

But, $(Q_{\alpha}^{\underline{\perp}})^{*}$ requires a name along
which to communicate the process to achieve the context application.

\subsubsection{Basis for a basis}
If processes label states and ``addition'' of states (a.k.a. vector
addition) is interpreted as parallel composition, what corresponds to
notions of linear independence and basis? Here, we recall that Yoshida
has developed a set of \emph{combinators} for an asynchronous verison
of Milner's $\pi$-calculus. These are a finite set of processes such
any process can be expressed as parallel composition of these
combinators together with liberal uses of the new operator and
replication. We can simply give a translation of these into the
present calculus and have reasonable expectation that the property
carries over. That is, that the resultant set allows to express all
processes via parallel composition. Note, however, that there is no
new operator or replication in this calculus. As a result, we expect
that the corresponding set is actually infinite. That is, we expect
that the space is actually infinite dimensional.

\begin{remark}
  The attentive reader may be a bit concerned. Certainly, the
  collection $S$, $K$ and $I$ is a finite set of
  combinators. Shouldn't we expect to see a finite set of combinators
  for an effectively equivalent system? i am very sympathetic to this
  critique and feel it warrants full attention. On the other hand, i
  also have in mind the following analogy. The natural numbers, as a
  monoid under addition, has exactly $1$ generator, while the natural
  numbers, as a monoid under multiplication, has countably many
  generators (the primes). We observe that the application of the
  lambda calculus is much less resource sensitive than the parallel
  composition of the $\pi$-calculus. Could it be the case that we have
  an analogy of the form
  
  \begin{mathpar}
    m + n : MN :: m*n : M|N
  \end{mathpar}

  giving a similar blow up in the set of ``primes''?  This is such a
  wonderful thought that, even if it's not true, i think it's worth
  writing down.
\end{remark}
 

\documentclass[12pt]{llncs}
%\documentclass{jktr}

\usepackage[pdftex]{hyperref}                   
\usepackage {listings}
\usepackage {mathpartir}
\usepackage{bcprules}
%\usepackage{listings}
                       
\usepackage{graphicx} 
%\usepackage[margins=2.5cm,nohead,nofoot]{geometry}
%\usepackage{geometry}
\usepackage{amsfonts}
\usepackage{amstext}
\usepackage{latexsym}
\usepackage{amssymb}
\usepackage{color}


%\include{myPreamble}
\include{qm2pi.local} 

%\ifpdf
%\usepackage[pdftex]{graphicx}
%\else
%\usepackage{graphicx}
%\fi

 % \ifpdf
%  \usepackage{pdfsync}
%  \if


%\title{Brief Article}
%\author{David F. Snyder}
%\author{L.G. Meredith}

%\address{Dept. of Math., Texas State University--San Marcos, San Marcos, TX 78666}
       
\pagestyle{empty}


\begin{document}

\lstset{language=[Objective]Caml,frame=shadowbox}

\input{qm2pi.front}

% section front matter (end)

\input{qm2pi.intro} 
 
% section introduction (end)

% \input{qm2pi.knotations} 

% section notation (end)

\input{qm2pi.process.calculi} 

% section concurrent_process_calculi_and_spatial_logics_ (end)
    
%\input{qm2pi.knots2pi} 

%\input{qm2pi.trefoil} 

%\input{qm2pi.mainthm} 

% subsection basic_interpretation (end)

%\input{qm2pi.rho.presentation} 
\subsection{The syntax and semantics of the notation system}\label{sub:the_syntax_and_semantics_of_the_notation_system} % (fold)

We now summarize a technical presentation of the calculus that
embodies our theory of dynamics. The typical presentation of such a
calculus follows the style of giving generators and relations on
them. The grammar, below, describing term constructors, freely
generates the set of processes, $\Proc$. This set is then quotiented
by a relation known as structural congruence and it is over this set
that the notion of dynamics is expressed. This presentation is
essentially that of \cite{MeredithR05} with the addition of
polyadicity and summation. For readability we have relegated some of
the technical subtleties to an appendix.

\subsubsection{Process grammar}\label{subsub:process_grammar}

\begin{mathpar}
  \inferrule* [lab=synchronization] {} {{M} \bc \pzero \;|\; x?F \;|\; x!C }
  \and
  \inferrule* [lab=abstraction] {} {{F} \bc (x)P}
  \and
  \inferrule* [lab=concretion] {} {{C} \bc \langle Q \rangle}
  \and
  \inferrule* [lab=process] {} {{P,Q} \bc M \;| \;P|Q \;|\; @{x}}
  \and
  \inferrule* [lab=name] {} {{x} \bc \quotep{P}}
\end{mathpar} 

Note that $\vec{x}$ (resp. $\vec{P}$) denotes a vector of names
(resp. processes) of length $|\vec{x}|$ (resp. $|\vec{P}|$). We adopt
the following useful abbreviations.

\begin{mathpar}
   x?(\vec{y}).P := x.(\vec{y})P \and  x\clift{\vec{P}} := x.\clift{\vec{P}}
   \and x!(y) := \lift{x}{\dropn{y}}
   \and \Pi_{i=0}^{n-1}P_i := P_0 | \ldots | P_{n-1}
\end{mathpar}

\subsubsection{Structural congruence}

\paragraph{Free and bound names and alpha-equivalence.} At the
core of structural equivalence is alpha-equivalence which identifies
process that are the same up to a change of variable. Formally, we
recognize the distinction between free and bound names. The free names
of a process, $\freenames{P}$, may be calculated recursively as
follows:

\begin{mathpar}
\freenames{\pzero} := \emptyset
  \and \\
  \freenames{x?(y).P} := \{ x \} \cup (\freenames{P} \setminus \{ y \})
  \and 
  \freenames{x!\langle P \rangle} := \{ x \} \cup \{ P \} 
  \and \\
  \freenames{P|Q} := \freenames{P} \cup \freenames{Q}
  \and \\
  \freenames{@{x}} := \{ x \}
\end{mathpar}

$\pi$
$\quotep{\pi}$

$\freenames{-} : \pi \to \mathcal{P}(\quotep{\pi})$

\begin{eqnarray*}
  \freenames{\pzero} & := & \emptyset \\
  \freenames{x?(y).P} & := & \{ x \} \cup (\freenames{P} \setminus \{ y \}) \\
  \freenames{x!\langle P \rangle} & := & \{ x \} \cup \{ P \} \\
  \freenames{P|Q} & := & \freenames{P} \cup \freenames{Q} \\
  \freenames{\dropn{x}} & := & \{ x \}
\end{eqnarray*}

The bound names of a process, $\boundnames{P}$, are those names occurring in $P$
that are not free. For example, in $x?(y).0$, the name $x$ is free, while $y$ is bound.

\begin{mathpar}
  \inferrule* [lab=monoidal-laws] {} { P|Q \equiv Q|P \and P|0 \equiv P \and P|(Q|R) \equiv (P|Q)|R }
\end{mathpar}

\begin{mathpar}
  \inferrule* [lab=alpha-equivalence] {} { (x)P \equiv (y)P\{y/x\} \and y \not\in \freenames{P} }
\end{mathpar}

\begin{definition}
Then two processes, $P,Q$, are alpha-equivalent if $P = Q\{\vec{y}/\vec{x}\}$ for
some $\vec{x} \in \boundnames{Q},\vec{y} \in \boundnames{P}$, where $Q\{\vec{y}/\vec{x}\}$
denotes the capture-avoiding substitution of $\vec{y}$ for $\vec{x}$ in $Q$.
\end{definition}

\begin{definition}
  The {\em structural congruence} \cite{SangiorgiWalker} , $\equiv$,
  between processes is the least congruence containing
  alpha-equivalence, satisfying the abelian monoid laws
  (associativity, commutativity and $\pzero$ as identity) for parallel
  composition $|$ and for summation $+$.
\end{definition}

\subsection{Name equivalence}

We take name equivalence, written $\nameeq$, to be the smallest
equivalence relation generated by the following rules.

\begin{mathpar}
\inferrule*[lab=Quote-drop]
{ }
{ \quotep{@{x}} \nameeq x }

\inferrule*[lab=Struct-equiv]
{ P \scong Q }
{ \quotep{P} \nameeq \quotep{Q} }
\end{mathpar}

The astute reader will have noticed that the mutual recursion of names
and processes imposes a mutual recursion on alpha-equivalence and
structural equivalence via name-equivalence. Fortunately, all of this
works out pleasantly and we may calculate in the natural way, free of
concern. The reader interested in the details is referred to the
appendix \ref{appendix:rho_details}.

\subsection{Substitution}

We use $\Proc$ for the set of processes, $\QProc$ for the set of
names, and $\id{\{}\vec{y} / \vec{x} \id{\}}$ to denote partial maps,
$s : \QProc \rightarrow \QProc$. A map, $s$ lifts, uniquely, to a map
on process terms, $\widehat{s} : \Proc \rightarrow \Proc$ by the
following equations.

\begin{mathpar}
  (0) \psubstp{Q}{P} := 0 \\
  (R \juxtap S) \psubstp{Q}{P}
  :=    
  (R)\psubstp{Q}{P} \juxtap (S) \psubstp{Q}{P} \\
  (x?(y).R) \psubstp{Q}{P}    
  :=    
  (x)\substp{Q}{P} (z)\concat( (R \psubstn{z}{y}) \psubstp{Q}{P} ) \\
  (\lift{x}{R}) \psubstp{Q}{P}  
  :=
  \lift{(x)\substp{Q}{P}}{ R \psubstp{Q}{P} } \\
%   (\dropn{x})  \psubstp{Q}{P}       
%   := 
%   \left\{ 
%     \begin{array}{ccc} 
%       \dropn{\quotep{Q}} & & x \nameeq \quotep{P} \\
%       \dropn{x} & & otherwise \\
%     \end{array}
%   \right. 
  (\dropn{x})  \psubstp{Q}{P}       
  := 
  \left\{ 
    \begin{array}{ccc} 
      Q & & x \nameeq \quotep{P} \\
      \dropn{x} & & otherwise \\
    \end{array}
  \right.
\end{mathpar}
 

where

\begin{eqnarray}
  (x)\id{\{} \lpquote Q \rpquote / \lpquote P \rpquote \id{\}}            = 
  \left\{ 
    \begin{array}{ccc}
      \lpquote Q \rpquote & & x \nameeq \lpquote P \rpquote \\
      x & & otherwise \\
    \end{array}
  \right. \nonumber
\end{eqnarray}

and $z$ is chosen distinct from $\quotep{P}$, $\quotep{Q}$, the free
names in $Q$, and all the names in $R$. Our $\alpha$-equivalence will
be built in the standard way from this substitution.

\begin{remark}\label{rem:no_self_referential_names}
  One consequence of these definitions is that $\forall P. \quotep{P}
  \not\in \freenames{P}$.
\end{remark}

\subsection{ Dynamic quote: an example }

Anticipating something of what's to come, consider applying the
substitution, $\widehat{\id{\{}u / z \id{\}}}$, to the following pair
of processes, $\lift{w}{y!(z)}$ and $w[ \lpquote y!(z) \rpquote ]$.

\begin{eqnarray}
	\lift{w}{y!(z)}\widehat{\id{\{}u / z \id{\}}}
		& = &
		\lift{w}{y!(u)} \nonumber\\
	w[ \lpquote y!(z) \rpquote ] \widehat{ \id{\{}u / z \id{\}} }
		& = &
		w[ \lpquote y!(z) \rpquote ] \nonumber
\end{eqnarray}

Because the body of the process between quotes is impervious to
substitution, we get radically different answers. In fact, by
examining the first process in an input context,
e.g. $x?(z).\lift{w}{y!(z)}$, we see that the process under the lift
operator may be shaped by prefixed inputs binding a name inside it. In
this sense, the lift operator will be seen as a way to dynamically
construct processes before reifying them as names.

Finally equipped with these standard features we can present the
dynamics of the calculus.

\subsubsection{Operational semantics} 

Finally, we introduce the computational dynamics. What marks these
algebras as distinct from other more traditionally studied algebraic
structures, e.g. vector spaces or polynomial rings, is the manner in
which dynamics is captured. In traditional structures, dynamics is typically
expressed through morphisms between such structures, as in linear maps
between vector spaces or morphisms between rings. In algebras
associated with the semantics of computation, the dynamics is
expressed as part of the algebraic structure itself, through a
reduction reduction relation typically denoted by $\red$. Below, we
give a recursive presentation of this relation for the calculus used
in the encoding.

$\red \subseteq \pi \times \pi$
$\red : \pi \to \mathcal{P}(\pi)$

\begin{mathpar}
  \inferrule* [lab=Comm] { \textsf{match}( x_{src}, x_{trgt} ) } { x_{trgt}?(y)P \; | \; x_{src}!\langle {Q} \rangle \red P\{\quotep{Q}/y}\} }
  \and \\
  \inferrule* [lab=Par] {{P} \red {P}'} {{{P} | {Q}} \red {{P}' | {Q}}}
  \and
  \inferrule* [lab=Equiv]{{{P} \scong {P}'} \andalso {{P}' \red {Q}'} \andalso {{Q}' \scong {Q}}}{{P} \red {Q}}
\end{mathpar}

\begin{eqnarray*}
  match_{\equiv} (\quotep{P},\quotep{Q}) & := & P \equiv Q \\
  match_{\dagger}(\quotep{P},\quotep{Q}) & := & \forall R. P|Q \red^{*} R => R \red^{*} 0 \\
  match_{K}(\quotep{P},\quotep{Q}) & := & K \mbox{ for some context } K
\end{eqnarray*}

$u?(x)P | u!\langle Q \rangle \red P\{\quotep{Q}/x\}$

%We write $\wred$ for $\red^*$, and $P\red$ if $\exists Q $ such that $ P \red Q$.
We write $P\red$ if $\exists Q $ such that $ P \red Q$ and $P\not\red$, otherwise.

\section{Replication}

As mentioned before, it is known that replication (and hence
recursion) can be implemented in a higher-order process algebra
\cite{SangiorgiWalker}. As our first example of calculation with the
machinery thus far presented we give the construction explicitly in
the {\rhoc}.

\begin{eqnarray}
	D_{x} & := & \prefix{x}{y}{(\binpar{\outputp{x}{y}}{@{y}})} \nonumber\\
	\bangp_{x}{P} & := & \binpar{{x}!\langle{\binpar{D_{x}}{P}}\rangle}{D_{x}} \nonumber
\end{eqnarray}

\begin{eqnarray}
	\bangp_{x}{P} & & \nonumber\\
	=
	& {x}!\langle{(\prefix{x}{y}{(\outputp{x}{y} | @{y})) | P}}\rangle 
	      | \prefix{x}{y}{(\outputp{x}{y} | @{y})} & \nonumber\\
	\red
	& (\outputp{x}{y} | @{y})\substn{\quotep{(\prefix{x}{y}{(@{y} | \outputp{x}{y})) | P}}}{y} & \nonumber\\
	=
	& \outputp{x}{\quotep{(\prefix{x}{y}{(\outputp{x}{y} | @{y})) | P}}}
	  | {(\prefix{x}{y}{(\outputp{x}{y} | @{y})) | P}} & \nonumber\\
	\red
	& \ldots & \nonumber\\
	\red^*
	& P | P | \ldots & \nonumber
\end{eqnarray}

Of course, this encoding, as an implementation, runs away, unfolding
$\bangp{P}$ eagerly. A lazier and more implementable replication
operator, restricted to input-guarded processes, may be obtained as follows.

\begin{eqnarray}
\bangp{\prefix{u}{v}{P}} 
	:= 
	\binpar{\lift{x}{\prefix{u}{v}{(\binpar{D(x)}{P})}}}{D(x)} \nonumber
\end{eqnarray}

\begin{remark}
  Note that the lazier definition still does not deal with summation
  or mixed summation (i.e. sums over input and output). The reader is
  invited to construct definitions of replication that deal with these
  features. 

  Further, the definitions are parameterized in a name, $x$. Can you,
  gentle reader, make a definition that eliminates this parameter and
  guarantees no accidental interaction between the replication
  machinery and the process being replicated -- i.e. no accidental
  sharing of names used by the process to get its work done and the
  name(s) used by the replication to effect copying. This latter
  revision of the definition of replication is crucial to obtaining
  the expected identity $!!P \sim !P$.
\end{remark}

\begin{remark}\label{rem:paradoxical_combinator}
  The reader familiar with the lambda calculus will have noticed the
  similarity between $D$ and the paradoxical combinator.

  [Ed. note: the existence of this seems to suggest we have to be more
  restrictive on the set of processes and names we admit if we are to
  support no-cloning.]
\end{remark}

\subsubsection{Bisimulation}

The computational dynamics gives rise to another kind of equivalence,
the equivalence of computational behavior. As previously mentioned
this is typically captured \emph{via} some form of bisimulation.

% The notion we use in this paper is weak barbed bisimulation
% \cite{milner91polyadicpi}.

The notion we use in this paper is derived from weak barbed
bisimulation \cite{milner91polyadicpi}. 

\begin{definition}
An \emph{observation relation}, $\downarrow_{\mathcal N}$, over a set
of names, $\mathcal N$, is the smallest relation satisfying the rules
below.

\infrule[Out-barb]{y \in {\mathcal N}, \; x \nameeq y}
		  {\outputp{x}{v} \downarrow_{\mathcal N} x}
\infrule[Par-barb]{\mbox{$P\downarrow_{\mathcal N} x$ or $Q\downarrow_{\mathcal N} x$}}
		  {\binpar{P}{Q} \downarrow_{\mathcal N} x}

We write $P \Downarrow_{\mathcal N} x$ if there is $Q$ such that 
$P \wred Q$ and $Q \downarrow_{\mathcal N} x$.
\end{definition}

\begin{definition}
%\label{def.bbisim}
An  ${\mathcal N}$-\emph{barbed bisimulation} over a set of names, ${\mathcal N}$, is a symmetric binary relation 
${\mathcal S}_{\mathcal N}$ between agents such that $P\rel{S}_{\mathcal N}Q$ implies:
\begin{enumerate}
\item If $P \red P'$ then $Q \wred Q'$ and $P'\rel{S}_{\mathcal N} Q'$.
\item If $P\downarrow_{\mathcal N} x$, then $Q\Downarrow_{\mathcal N} x$.
\end{enumerate}
$P$ is ${\mathcal N}$-barbed bisimilar to $Q$, written
$P \wbbisim_{\mathcal N} Q$, if $P \rel{S}_{\mathcal N} Q$ for some ${\mathcal N}$-barbed bisimulation ${\mathcal S}_{\mathcal N}$.
\end{definition}

$\mathcal{R} \subseteq \pi \times \pi$

$P \mathcal{R} Q => \forall P'. P \red P' \Rightarrow \exists Q'. Q \red Q', P' \mathcal{R} Q'$

$P \vdash x \Rightarrow Q \vdash x$

\begin{mathpar}
  \inferrule*[lab=Out-barb]{x \nameeq y}{{y}!\langle{Q}\rangle \vdash x}
  \and
  \inferrule*[lab=Par-barb]{\mbox{$P\vdash x$ or $Q\vdash x$}}{\binpar{P}{Q} \vdash x}
\end{mathpar}

\subsubsection{Contexts}

One of the principle advantages of computational calculi like the
$\pi$-calculus is a well-defined notion of context,
contextual-equivalence and a correlation between
contextual-equivalence and notions of bisimulation. The notion of
context allows the decomposition of a process into (sub-)process and
its syntactic environment, its context. Thus, a context may be
thought of as a process with a ``hole'' (written $\Box$) in it. The
application of a context $M$ to a process $P$, written $M[P]$, is
tantamount to filling the hole in $M$ with $P$. In this paper we do
not need the full weight of this theory, but do make use of the notion
of context in the proof the main theorem. 

\begin{mathpar}
  \inferrule* [lab=summation] {} {{M_{M},M_{N}} \bc \Box \;|\; x.M_{A} \;|\; M_{M}+M_{N}}
  \and
  \inferrule* [lab=agent] {} {{M_{A}} \bc (\vec{x})M_{P} \;| \; \clift{P_0,\ldots,M_{P},\ldots,P_N}}
  \and \\
  \inferrule* [lab=process] {} {{M_{P}} \bc M_{N} \;| \;P|M_{P} }
\end{mathpar} 

\begin{mathpar}
  \inferrule* [lab=sychronization] {} {M_{N} \bc \Box \;|\; x?M_{F} \;|\; x!M_{C}}
  \and
  \inferrule* [lab=abstraction] {} {{M_{F}} \bc (x)M_{P} }
  \and
  \inferrule* [lab=concretion] {} {{M_{C}} \bc \langle M_{P} \rangle }
  \and \\
  \inferrule* [lab=process] {} {{M_{P}} \bc M_{N} \;| \;P|M_{P} }
\end{mathpar}

\begin{definition}[contextual application] Given a context $M$, and
  process $P$, we define the \emph{contextual application}, $M[P] :=
  M\{P/\Box\}$. That is, the contextual application of M to P is the
  substitution of $P$ for $\Box$ in $M$.
\end{definition}

$\meaningof{-} : L \to \mathcal{P}(\pi)$

\begin{mathpar}
  \inferrule* [lab=collection] {} {\meaningof{true} = \pi, \and \meaningof{~E} = \pi \setminus \meaningof{E}, \and \meaningof{E_{1} \& E_{2}} = \meaningof{E_{1}} \cap \meaningof{E_{2}}}
\end{mathpar}

\begin{mathpar}
  \inferrule* [lab=structure] {} {\meaningof{0} = \{ P \in \pi | P \equiv 0 \}, \and \\ \meaningof{E_1 | E_2} = \{ P \in \pi | P \equiv P_{1} | P_{2}, P_{1} \in \meaningof{E_{1}}, P_{2} \in \meaningof{E_2}\} }
\end{mathpar}

\begin{mathpar}
 \inferrule* [lab=behavior] {} {\meaningof{\langle a?b \rangle E} = \{ P \in \pi | P \equiv Q | u?(y)P', \\ \and \\\\ \and \\ \;\;\; u \in \meaningof{a}, \forall z.P'\{z/y\} \in \meaningof{E\{z/b\}}\}, \and \\ \meaningof{a!E} = \{ P \in \pi | P \equiv Q | x!\langle P' \rangle, x \in \meaningof{a} P' \in \meaningof{E}\} }
\end{mathpar}

\begin{mathpar}
 \inferrule* [lab=nominal] {} {\meaningof{\quotep{E}} = \{ \quotep{P} \in \quotep{\pi} | P \in \meaningof{E} \}, \and \meaningof{\quotep{P}} = \{ \quotep{Q} \in \quotep{\pi} | P \equiv Q \} \and \\ \meaningof{@\quotep{E}} = \{ P \in \pi | P \equiv @x, x \in \meaningof{E} \}}
\end{mathpar}

\begin{eqnarray*}
  \\
  \meaningof{-} : TS \to ST
\end{eqnarray*}

\begin{eqnarray*}
  \\
  L : TS \to ST
\end{eqnarray*}

\begin{eqnarray*}
  \\
  P \models E \iff P \in \meaningof{E}
\end{eqnarray*}

\begin{eqnarray*}
  P \approx_{L} Q \iff \forall E \in L. P \models E \iff Q \models E
\end{eqnarray*}

\begin{eqnarray*}
  P \approx_{K} Q
\end{eqnarray*}

\begin{eqnarray*}
  P \approx Q
\end{eqnarray*}

$\approx_{K} = \approx = \approx_{L}$

\subsubsection{Contextual duality}

Note that contexts extend the quotation operation to a family of
operations from processes to names. Given a context, $M$, we can
define a \emph{nominal context}, $\quotep{M}$ by $\quotep{M}[P] :=
\quotep{M[P]}$. To foreshadow what is to come we observe that these
operations enjoy a duality with processes very much like the duality
between vectors and maps from vectors to scalars.

Further, because the calculus is essentially higher-order, we have a
correspondence between contexts and processes. More specifically,
given a name $x$ and a context $M$ we can construct $M^{*}_{x}$ such
that 

\begin{mathpar}
  M^{*}_{x} | \lift{x}{P} \red M[P]
\end{mathpar}

namely,

\begin{mathpar}
  M^{*}_{x} := x?(u).M[\dropn{u}]
\end{mathpar}

The dependence of $M^{*}_{x}$ on a name makes it an abstraction, 

\begin{mathpar}
  M^{*} := (x)x?(u).M[\dropn{u}]
\end{mathpar}

\subsection{Additional notation}

It will sometimes be convenient to denote the process a name
quotes. We already have the notation $x = \quotep{P}$, but it will be
convenient to introduce an alternate notation, $\procn{x}$, when we
want to emphasize the connection to the use of the name. Note that, by
virtue of name equivalence, $\quotep{\procn{x}} \nameeq x$; so, the
notation is consistent with previous definitions.

Further, because names have structure it is possible to effect
substitutions on the basis of that structure. This means we need to
upgrade our notation for substitutions, which we accomplish by
adapting comprehension notation. Thus,

\begin{mathpar}
  P\{ y / x : x \in S \}
\end{mathpar}

is interpreted to mean the process derived from P by replacing (in a
capture-avoiding manner) each occurrence of $x$ in $S$ by $y$. For example,

\begin{mathpar}
  P\{ \quotep{\procn{x}|\procn{x}} / x : x \in \freenames{P} \}
\end{mathpar}

will replace each (occurrence) of a free name $x$ in $P$ by
$\quotep{\procn{x}|\procn{x}}$.

Also, we will avail ourselves of the notation $x^{L}$ and $x^{R}$ to
denote injections of a name into disjoint copies of the name
space. There are numerous ways to accomplish this. One example can be
found in \cite{MeredithR05}. This notation overloads to vectors of
names: $\vec{x}^{\pi} := (x_{i}^{\pi} \; : \; 0 \leq i < |\vec{x}| )$ where $\pi \in \{L,R\}$.

We also use $P^{\Box} := P|\Box$.

In \cite{MeredithR05} an interpretation of the new operator is
given. It turns out that there are several possible interpretations
all enjoying the requisite algebraic properties of the operator (see
\cite{milner91polyadicpi}). We will therefore make liberal use of
$(\nu\; \vec{x})P$.

% subsection the_syntax_and_semantics_of_the_notation_system (end)   

\input{qm2pi.qmops} 

\input{qm2pi.sterngerlach} 

\input{qm2pi.metric} 

% section concurrent_process_calculi (end)

%\input{qm2pi.proofsketch}

% section proof sketch (end)

%\input{qm2pi.slviaknots} 

% section spatial logic via knots (end)

\input{qm2pi.conclusion}

% section conclusion (end)

%\input{qm2pi.dtcodes} 

% section wiring algorithm (end)

\input{qm2pi.ack} 

% section acknowledgments (end)

\newpage


\bibliographystyle{plain}   
\bibliography{../../biblios/main.bib}

\input{qm2pi.rhodetails}

\end{document}

 

\documentclass[12pt]{llncs}
%\documentclass{jktr}

\usepackage[pdftex]{hyperref}                   
\usepackage {listings}
\usepackage {mathpartir}
\usepackage{bcprules}
%\usepackage{listings}
                       
\usepackage{graphicx} 
%\usepackage[margins=2.5cm,nohead,nofoot]{geometry}
%\usepackage{geometry}
\usepackage{amsfonts}
\usepackage{amstext}
\usepackage{latexsym}
\usepackage{amssymb}
\usepackage{color}


%\include{myPreamble}
\include{qm2pi.local} 

%\ifpdf
%\usepackage[pdftex]{graphicx}
%\else
%\usepackage{graphicx}
%\fi

 % \ifpdf
%  \usepackage{pdfsync}
%  \if


%\title{Brief Article}
%\author{David F. Snyder}
%\author{L.G. Meredith}

%\address{Dept. of Math., Texas State University--San Marcos, San Marcos, TX 78666}
       
\pagestyle{empty}


\begin{document}

\lstset{language=[Objective]Caml,frame=shadowbox}

\input{qm2pi.front}

% section front matter (end)

\input{qm2pi.intro} 
 
% section introduction (end)

% \input{qm2pi.knotations} 

% section notation (end)

\input{qm2pi.process.calculi} 

% section concurrent_process_calculi_and_spatial_logics_ (end)
    
%\input{qm2pi.knots2pi} 

%\input{qm2pi.trefoil} 

%\input{qm2pi.mainthm} 

% subsection basic_interpretation (end)

%\input{qm2pi.rho.presentation} 
\subsection{The syntax and semantics of the notation system}\label{sub:the_syntax_and_semantics_of_the_notation_system} % (fold)

We now summarize a technical presentation of the calculus that
embodies our theory of dynamics. The typical presentation of such a
calculus follows the style of giving generators and relations on
them. The grammar, below, describing term constructors, freely
generates the set of processes, $\Proc$. This set is then quotiented
by a relation known as structural congruence and it is over this set
that the notion of dynamics is expressed. This presentation is
essentially that of \cite{MeredithR05} with the addition of
polyadicity and summation. For readability we have relegated some of
the technical subtleties to an appendix.

\subsubsection{Process grammar}\label{subsub:process_grammar}

\begin{mathpar}
  \inferrule* [lab=synchronization] {} {{M} \bc \pzero \;|\; x?F \;|\; x!C }
  \and
  \inferrule* [lab=abstraction] {} {{F} \bc (x)P}
  \and
  \inferrule* [lab=concretion] {} {{C} \bc \langle Q \rangle}
  \and
  \inferrule* [lab=process] {} {{P,Q} \bc M \;| \;P|Q \;|\; @{x}}
  \and
  \inferrule* [lab=name] {} {{x} \bc \quotep{P}}
\end{mathpar} 

Note that $\vec{x}$ (resp. $\vec{P}$) denotes a vector of names
(resp. processes) of length $|\vec{x}|$ (resp. $|\vec{P}|$). We adopt
the following useful abbreviations.

\begin{mathpar}
   x?(\vec{y}).P := x.(\vec{y})P \and  x\clift{\vec{P}} := x.\clift{\vec{P}}
   \and x!(y) := \lift{x}{\dropn{y}}
   \and \Pi_{i=0}^{n-1}P_i := P_0 | \ldots | P_{n-1}
\end{mathpar}

\subsubsection{Structural congruence}

\paragraph{Free and bound names and alpha-equivalence.} At the
core of structural equivalence is alpha-equivalence which identifies
process that are the same up to a change of variable. Formally, we
recognize the distinction between free and bound names. The free names
of a process, $\freenames{P}$, may be calculated recursively as
follows:

\begin{mathpar}
\freenames{\pzero} := \emptyset
  \and \\
  \freenames{x?(y).P} := \{ x \} \cup (\freenames{P} \setminus \{ y \})
  \and 
  \freenames{x!\langle P \rangle} := \{ x \} \cup \{ P \} 
  \and \\
  \freenames{P|Q} := \freenames{P} \cup \freenames{Q}
  \and \\
  \freenames{@{x}} := \{ x \}
\end{mathpar}

$\pi$
$\quotep{\pi}$

$\freenames{-} : \pi \to \mathcal{P}(\quotep{\pi})$

\begin{eqnarray*}
  \freenames{\pzero} & := & \emptyset \\
  \freenames{x?(y).P} & := & \{ x \} \cup (\freenames{P} \setminus \{ y \}) \\
  \freenames{x!\langle P \rangle} & := & \{ x \} \cup \{ P \} \\
  \freenames{P|Q} & := & \freenames{P} \cup \freenames{Q} \\
  \freenames{\dropn{x}} & := & \{ x \}
\end{eqnarray*}

The bound names of a process, $\boundnames{P}$, are those names occurring in $P$
that are not free. For example, in $x?(y).0$, the name $x$ is free, while $y$ is bound.

\begin{mathpar}
  \inferrule* [lab=monoidal-laws] {} { P|Q \equiv Q|P \and P|0 \equiv P \and P|(Q|R) \equiv (P|Q)|R }
\end{mathpar}

\begin{mathpar}
  \inferrule* [lab=alpha-equivalence] {} { (x)P \equiv (y)P\{y/x\} \and y \not\in \freenames{P} }
\end{mathpar}

\begin{definition}
Then two processes, $P,Q$, are alpha-equivalent if $P = Q\{\vec{y}/\vec{x}\}$ for
some $\vec{x} \in \boundnames{Q},\vec{y} \in \boundnames{P}$, where $Q\{\vec{y}/\vec{x}\}$
denotes the capture-avoiding substitution of $\vec{y}$ for $\vec{x}$ in $Q$.
\end{definition}

\begin{definition}
  The {\em structural congruence} \cite{SangiorgiWalker} , $\equiv$,
  between processes is the least congruence containing
  alpha-equivalence, satisfying the abelian monoid laws
  (associativity, commutativity and $\pzero$ as identity) for parallel
  composition $|$ and for summation $+$.
\end{definition}

\subsection{Name equivalence}

We take name equivalence, written $\nameeq$, to be the smallest
equivalence relation generated by the following rules.

\begin{mathpar}
\inferrule*[lab=Quote-drop]
{ }
{ \quotep{@{x}} \nameeq x }

\inferrule*[lab=Struct-equiv]
{ P \scong Q }
{ \quotep{P} \nameeq \quotep{Q} }
\end{mathpar}

The astute reader will have noticed that the mutual recursion of names
and processes imposes a mutual recursion on alpha-equivalence and
structural equivalence via name-equivalence. Fortunately, all of this
works out pleasantly and we may calculate in the natural way, free of
concern. The reader interested in the details is referred to the
appendix \ref{appendix:rho_details}.

\subsection{Substitution}

We use $\Proc$ for the set of processes, $\QProc$ for the set of
names, and $\id{\{}\vec{y} / \vec{x} \id{\}}$ to denote partial maps,
$s : \QProc \rightarrow \QProc$. A map, $s$ lifts, uniquely, to a map
on process terms, $\widehat{s} : \Proc \rightarrow \Proc$ by the
following equations.

\begin{mathpar}
  (0) \psubstp{Q}{P} := 0 \\
  (R \juxtap S) \psubstp{Q}{P}
  :=    
  (R)\psubstp{Q}{P} \juxtap (S) \psubstp{Q}{P} \\
  (x?(y).R) \psubstp{Q}{P}    
  :=    
  (x)\substp{Q}{P} (z)\concat( (R \psubstn{z}{y}) \psubstp{Q}{P} ) \\
  (\lift{x}{R}) \psubstp{Q}{P}  
  :=
  \lift{(x)\substp{Q}{P}}{ R \psubstp{Q}{P} } \\
%   (\dropn{x})  \psubstp{Q}{P}       
%   := 
%   \left\{ 
%     \begin{array}{ccc} 
%       \dropn{\quotep{Q}} & & x \nameeq \quotep{P} \\
%       \dropn{x} & & otherwise \\
%     \end{array}
%   \right. 
  (\dropn{x})  \psubstp{Q}{P}       
  := 
  \left\{ 
    \begin{array}{ccc} 
      Q & & x \nameeq \quotep{P} \\
      \dropn{x} & & otherwise \\
    \end{array}
  \right.
\end{mathpar}
 

where

\begin{eqnarray}
  (x)\id{\{} \lpquote Q \rpquote / \lpquote P \rpquote \id{\}}            = 
  \left\{ 
    \begin{array}{ccc}
      \lpquote Q \rpquote & & x \nameeq \lpquote P \rpquote \\
      x & & otherwise \\
    \end{array}
  \right. \nonumber
\end{eqnarray}

and $z$ is chosen distinct from $\quotep{P}$, $\quotep{Q}$, the free
names in $Q$, and all the names in $R$. Our $\alpha$-equivalence will
be built in the standard way from this substitution.

\begin{remark}\label{rem:no_self_referential_names}
  One consequence of these definitions is that $\forall P. \quotep{P}
  \not\in \freenames{P}$.
\end{remark}

\subsection{ Dynamic quote: an example }

Anticipating something of what's to come, consider applying the
substitution, $\widehat{\id{\{}u / z \id{\}}}$, to the following pair
of processes, $\lift{w}{y!(z)}$ and $w[ \lpquote y!(z) \rpquote ]$.

\begin{eqnarray}
	\lift{w}{y!(z)}\widehat{\id{\{}u / z \id{\}}}
		& = &
		\lift{w}{y!(u)} \nonumber\\
	w[ \lpquote y!(z) \rpquote ] \widehat{ \id{\{}u / z \id{\}} }
		& = &
		w[ \lpquote y!(z) \rpquote ] \nonumber
\end{eqnarray}

Because the body of the process between quotes is impervious to
substitution, we get radically different answers. In fact, by
examining the first process in an input context,
e.g. $x?(z).\lift{w}{y!(z)}$, we see that the process under the lift
operator may be shaped by prefixed inputs binding a name inside it. In
this sense, the lift operator will be seen as a way to dynamically
construct processes before reifying them as names.

Finally equipped with these standard features we can present the
dynamics of the calculus.

\subsubsection{Operational semantics} 

Finally, we introduce the computational dynamics. What marks these
algebras as distinct from other more traditionally studied algebraic
structures, e.g. vector spaces or polynomial rings, is the manner in
which dynamics is captured. In traditional structures, dynamics is typically
expressed through morphisms between such structures, as in linear maps
between vector spaces or morphisms between rings. In algebras
associated with the semantics of computation, the dynamics is
expressed as part of the algebraic structure itself, through a
reduction reduction relation typically denoted by $\red$. Below, we
give a recursive presentation of this relation for the calculus used
in the encoding.

$\red \subseteq \pi \times \pi$
$\red : \pi \to \mathcal{P}(\pi)$

\begin{mathpar}
  \inferrule* [lab=Comm] { \textsf{match}( x_{src}, x_{trgt} ) } { x_{trgt}?(y)P \; | \; x_{src}!\langle {Q} \rangle \red P\{\quotep{Q}/y}\} }
  \and \\
  \inferrule* [lab=Par] {{P} \red {P}'} {{{P} | {Q}} \red {{P}' | {Q}}}
  \and
  \inferrule* [lab=Equiv]{{{P} \scong {P}'} \andalso {{P}' \red {Q}'} \andalso {{Q}' \scong {Q}}}{{P} \red {Q}}
\end{mathpar}

\begin{eqnarray*}
  match_{\equiv} (\quotep{P},\quotep{Q}) & := & P \equiv Q \\
  match_{\dagger}(\quotep{P},\quotep{Q}) & := & \forall R. P|Q \red^{*} R => R \red^{*} 0 \\
  match_{K}(\quotep{P},\quotep{Q}) & := & K \mbox{ for some context } K
\end{eqnarray*}

$u?(x)P | u!\langle Q \rangle \red P\{\quotep{Q}/x\}$

%We write $\wred$ for $\red^*$, and $P\red$ if $\exists Q $ such that $ P \red Q$.
We write $P\red$ if $\exists Q $ such that $ P \red Q$ and $P\not\red$, otherwise.

\section{Replication}

As mentioned before, it is known that replication (and hence
recursion) can be implemented in a higher-order process algebra
\cite{SangiorgiWalker}. As our first example of calculation with the
machinery thus far presented we give the construction explicitly in
the {\rhoc}.

\begin{eqnarray}
	D_{x} & := & \prefix{x}{y}{(\binpar{\outputp{x}{y}}{@{y}})} \nonumber\\
	\bangp_{x}{P} & := & \binpar{{x}!\langle{\binpar{D_{x}}{P}}\rangle}{D_{x}} \nonumber
\end{eqnarray}

\begin{eqnarray}
	\bangp_{x}{P} & & \nonumber\\
	=
	& {x}!\langle{(\prefix{x}{y}{(\outputp{x}{y} | @{y})) | P}}\rangle 
	      | \prefix{x}{y}{(\outputp{x}{y} | @{y})} & \nonumber\\
	\red
	& (\outputp{x}{y} | @{y})\substn{\quotep{(\prefix{x}{y}{(@{y} | \outputp{x}{y})) | P}}}{y} & \nonumber\\
	=
	& \outputp{x}{\quotep{(\prefix{x}{y}{(\outputp{x}{y} | @{y})) | P}}}
	  | {(\prefix{x}{y}{(\outputp{x}{y} | @{y})) | P}} & \nonumber\\
	\red
	& \ldots & \nonumber\\
	\red^*
	& P | P | \ldots & \nonumber
\end{eqnarray}

Of course, this encoding, as an implementation, runs away, unfolding
$\bangp{P}$ eagerly. A lazier and more implementable replication
operator, restricted to input-guarded processes, may be obtained as follows.

\begin{eqnarray}
\bangp{\prefix{u}{v}{P}} 
	:= 
	\binpar{\lift{x}{\prefix{u}{v}{(\binpar{D(x)}{P})}}}{D(x)} \nonumber
\end{eqnarray}

\begin{remark}
  Note that the lazier definition still does not deal with summation
  or mixed summation (i.e. sums over input and output). The reader is
  invited to construct definitions of replication that deal with these
  features. 

  Further, the definitions are parameterized in a name, $x$. Can you,
  gentle reader, make a definition that eliminates this parameter and
  guarantees no accidental interaction between the replication
  machinery and the process being replicated -- i.e. no accidental
  sharing of names used by the process to get its work done and the
  name(s) used by the replication to effect copying. This latter
  revision of the definition of replication is crucial to obtaining
  the expected identity $!!P \sim !P$.
\end{remark}

\begin{remark}\label{rem:paradoxical_combinator}
  The reader familiar with the lambda calculus will have noticed the
  similarity between $D$ and the paradoxical combinator.

  [Ed. note: the existence of this seems to suggest we have to be more
  restrictive on the set of processes and names we admit if we are to
  support no-cloning.]
\end{remark}

\subsubsection{Bisimulation}

The computational dynamics gives rise to another kind of equivalence,
the equivalence of computational behavior. As previously mentioned
this is typically captured \emph{via} some form of bisimulation.

% The notion we use in this paper is weak barbed bisimulation
% \cite{milner91polyadicpi}.

The notion we use in this paper is derived from weak barbed
bisimulation \cite{milner91polyadicpi}. 

\begin{definition}
An \emph{observation relation}, $\downarrow_{\mathcal N}$, over a set
of names, $\mathcal N$, is the smallest relation satisfying the rules
below.

\infrule[Out-barb]{y \in {\mathcal N}, \; x \nameeq y}
		  {\outputp{x}{v} \downarrow_{\mathcal N} x}
\infrule[Par-barb]{\mbox{$P\downarrow_{\mathcal N} x$ or $Q\downarrow_{\mathcal N} x$}}
		  {\binpar{P}{Q} \downarrow_{\mathcal N} x}

We write $P \Downarrow_{\mathcal N} x$ if there is $Q$ such that 
$P \wred Q$ and $Q \downarrow_{\mathcal N} x$.
\end{definition}

\begin{definition}
%\label{def.bbisim}
An  ${\mathcal N}$-\emph{barbed bisimulation} over a set of names, ${\mathcal N}$, is a symmetric binary relation 
${\mathcal S}_{\mathcal N}$ between agents such that $P\rel{S}_{\mathcal N}Q$ implies:
\begin{enumerate}
\item If $P \red P'$ then $Q \wred Q'$ and $P'\rel{S}_{\mathcal N} Q'$.
\item If $P\downarrow_{\mathcal N} x$, then $Q\Downarrow_{\mathcal N} x$.
\end{enumerate}
$P$ is ${\mathcal N}$-barbed bisimilar to $Q$, written
$P \wbbisim_{\mathcal N} Q$, if $P \rel{S}_{\mathcal N} Q$ for some ${\mathcal N}$-barbed bisimulation ${\mathcal S}_{\mathcal N}$.
\end{definition}

$\mathcal{R} \subseteq \pi \times \pi$

$P \mathcal{R} Q => \forall P'. P \red P' \Rightarrow \exists Q'. Q \red Q', P' \mathcal{R} Q'$

$P \vdash x \Rightarrow Q \vdash x$

\begin{mathpar}
  \inferrule*[lab=Out-barb]{x \nameeq y}{{y}!\langle{Q}\rangle \vdash x}
  \and
  \inferrule*[lab=Par-barb]{\mbox{$P\vdash x$ or $Q\vdash x$}}{\binpar{P}{Q} \vdash x}
\end{mathpar}

\subsubsection{Contexts}

One of the principle advantages of computational calculi like the
$\pi$-calculus is a well-defined notion of context,
contextual-equivalence and a correlation between
contextual-equivalence and notions of bisimulation. The notion of
context allows the decomposition of a process into (sub-)process and
its syntactic environment, its context. Thus, a context may be
thought of as a process with a ``hole'' (written $\Box$) in it. The
application of a context $M$ to a process $P$, written $M[P]$, is
tantamount to filling the hole in $M$ with $P$. In this paper we do
not need the full weight of this theory, but do make use of the notion
of context in the proof the main theorem. 

\begin{mathpar}
  \inferrule* [lab=summation] {} {{M_{M},M_{N}} \bc \Box \;|\; x.M_{A} \;|\; M_{M}+M_{N}}
  \and
  \inferrule* [lab=agent] {} {{M_{A}} \bc (\vec{x})M_{P} \;| \; \clift{P_0,\ldots,M_{P},\ldots,P_N}}
  \and \\
  \inferrule* [lab=process] {} {{M_{P}} \bc M_{N} \;| \;P|M_{P} }
\end{mathpar} 

\begin{mathpar}
  \inferrule* [lab=sychronization] {} {M_{N} \bc \Box \;|\; x?M_{F} \;|\; x!M_{C}}
  \and
  \inferrule* [lab=abstraction] {} {{M_{F}} \bc (x)M_{P} }
  \and
  \inferrule* [lab=concretion] {} {{M_{C}} \bc \langle M_{P} \rangle }
  \and \\
  \inferrule* [lab=process] {} {{M_{P}} \bc M_{N} \;| \;P|M_{P} }
\end{mathpar}

\begin{definition}[contextual application] Given a context $M$, and
  process $P$, we define the \emph{contextual application}, $M[P] :=
  M\{P/\Box\}$. That is, the contextual application of M to P is the
  substitution of $P$ for $\Box$ in $M$.
\end{definition}

$\meaningof{-} : L \to \mathcal{P}(\pi)$

\begin{mathpar}
  \inferrule* [lab=collection] {} {\meaningof{true} = \pi, \and \meaningof{~E} = \pi \setminus \meaningof{E}, \and \meaningof{E_{1} \& E_{2}} = \meaningof{E_{1}} \cap \meaningof{E_{2}}}
\end{mathpar}

\begin{mathpar}
  \inferrule* [lab=structure] {} {\meaningof{0} = \{ P \in \pi | P \equiv 0 \}, \and \\ \meaningof{E_1 | E_2} = \{ P \in \pi | P \equiv P_{1} | P_{2}, P_{1} \in \meaningof{E_{1}}, P_{2} \in \meaningof{E_2}\} }
\end{mathpar}

\begin{mathpar}
 \inferrule* [lab=behavior] {} {\meaningof{\langle a?b \rangle E} = \{ P \in \pi | P \equiv Q | u?(y)P', \\ \and \\\\ \and \\ \;\;\; u \in \meaningof{a}, \forall z.P'\{z/y\} \in \meaningof{E\{z/b\}}\}, \and \\ \meaningof{a!E} = \{ P \in \pi | P \equiv Q | x!\langle P' \rangle, x \in \meaningof{a} P' \in \meaningof{E}\} }
\end{mathpar}

\begin{mathpar}
 \inferrule* [lab=nominal] {} {\meaningof{\quotep{E}} = \{ \quotep{P} \in \quotep{\pi} | P \in \meaningof{E} \}, \and \meaningof{\quotep{P}} = \{ \quotep{Q} \in \quotep{\pi} | P \equiv Q \} \and \\ \meaningof{@\quotep{E}} = \{ P \in \pi | P \equiv @x, x \in \meaningof{E} \}}
\end{mathpar}

\begin{eqnarray*}
  \\
  \meaningof{-} : TS \to ST
\end{eqnarray*}

\begin{eqnarray*}
  \\
  L : TS \to ST
\end{eqnarray*}

\begin{eqnarray*}
  \\
  P \models E \iff P \in \meaningof{E}
\end{eqnarray*}

\begin{eqnarray*}
  P \approx_{L} Q \iff \forall E \in L. P \models E \iff Q \models E
\end{eqnarray*}

\begin{eqnarray*}
  P \approx_{K} Q
\end{eqnarray*}

\begin{eqnarray*}
  P \approx Q
\end{eqnarray*}

$\approx_{K} = \approx = \approx_{L}$

\subsubsection{Contextual duality}

Note that contexts extend the quotation operation to a family of
operations from processes to names. Given a context, $M$, we can
define a \emph{nominal context}, $\quotep{M}$ by $\quotep{M}[P] :=
\quotep{M[P]}$. To foreshadow what is to come we observe that these
operations enjoy a duality with processes very much like the duality
between vectors and maps from vectors to scalars.

Further, because the calculus is essentially higher-order, we have a
correspondence between contexts and processes. More specifically,
given a name $x$ and a context $M$ we can construct $M^{*}_{x}$ such
that 

\begin{mathpar}
  M^{*}_{x} | \lift{x}{P} \red M[P]
\end{mathpar}

namely,

\begin{mathpar}
  M^{*}_{x} := x?(u).M[\dropn{u}]
\end{mathpar}

The dependence of $M^{*}_{x}$ on a name makes it an abstraction, 

\begin{mathpar}
  M^{*} := (x)x?(u).M[\dropn{u}]
\end{mathpar}

\subsection{Additional notation}

It will sometimes be convenient to denote the process a name
quotes. We already have the notation $x = \quotep{P}$, but it will be
convenient to introduce an alternate notation, $\procn{x}$, when we
want to emphasize the connection to the use of the name. Note that, by
virtue of name equivalence, $\quotep{\procn{x}} \nameeq x$; so, the
notation is consistent with previous definitions.

Further, because names have structure it is possible to effect
substitutions on the basis of that structure. This means we need to
upgrade our notation for substitutions, which we accomplish by
adapting comprehension notation. Thus,

\begin{mathpar}
  P\{ y / x : x \in S \}
\end{mathpar}

is interpreted to mean the process derived from P by replacing (in a
capture-avoiding manner) each occurrence of $x$ in $S$ by $y$. For example,

\begin{mathpar}
  P\{ \quotep{\procn{x}|\procn{x}} / x : x \in \freenames{P} \}
\end{mathpar}

will replace each (occurrence) of a free name $x$ in $P$ by
$\quotep{\procn{x}|\procn{x}}$.

Also, we will avail ourselves of the notation $x^{L}$ and $x^{R}$ to
denote injections of a name into disjoint copies of the name
space. There are numerous ways to accomplish this. One example can be
found in \cite{MeredithR05}. This notation overloads to vectors of
names: $\vec{x}^{\pi} := (x_{i}^{\pi} \; : \; 0 \leq i < |\vec{x}| )$ where $\pi \in \{L,R\}$.

We also use $P^{\Box} := P|\Box$.

In \cite{MeredithR05} an interpretation of the new operator is
given. It turns out that there are several possible interpretations
all enjoying the requisite algebraic properties of the operator (see
\cite{milner91polyadicpi}). We will therefore make liberal use of
$(\nu\; \vec{x})P$.

% subsection the_syntax_and_semantics_of_the_notation_system (end)   

\input{qm2pi.qmops} 

\input{qm2pi.sterngerlach} 

\input{qm2pi.metric} 

% section concurrent_process_calculi (end)

%\input{qm2pi.proofsketch}

% section proof sketch (end)

%\input{qm2pi.slviaknots} 

% section spatial logic via knots (end)

\input{qm2pi.conclusion}

% section conclusion (end)

%\input{qm2pi.dtcodes} 

% section wiring algorithm (end)

\input{qm2pi.ack} 

% section acknowledgments (end)

\newpage


\bibliographystyle{plain}   
\bibliography{../../biblios/main.bib}

\input{qm2pi.rhodetails}

\end{document}

 

% section concurrent_process_calculi (end)

%\documentclass[12pt]{llncs}
%\documentclass{jktr}

\usepackage[pdftex]{hyperref}                   
\usepackage {listings}
\usepackage {mathpartir}
\usepackage{bcprules}
%\usepackage{listings}
                       
\usepackage{graphicx} 
%\usepackage[margins=2.5cm,nohead,nofoot]{geometry}
%\usepackage{geometry}
\usepackage{amsfonts}
\usepackage{amstext}
\usepackage{latexsym}
\usepackage{amssymb}
\usepackage{color}


%\include{myPreamble}
\include{qm2pi.local} 

%\ifpdf
%\usepackage[pdftex]{graphicx}
%\else
%\usepackage{graphicx}
%\fi

 % \ifpdf
%  \usepackage{pdfsync}
%  \if


%\title{Brief Article}
%\author{David F. Snyder}
%\author{L.G. Meredith}

%\address{Dept. of Math., Texas State University--San Marcos, San Marcos, TX 78666}
       
\pagestyle{empty}


\begin{document}

\lstset{language=[Objective]Caml,frame=shadowbox}

\input{qm2pi.front}

% section front matter (end)

\input{qm2pi.intro} 
 
% section introduction (end)

% \input{qm2pi.knotations} 

% section notation (end)

\input{qm2pi.process.calculi} 

% section concurrent_process_calculi_and_spatial_logics_ (end)
    
%\input{qm2pi.knots2pi} 

%\input{qm2pi.trefoil} 

%\input{qm2pi.mainthm} 

% subsection basic_interpretation (end)

%\input{qm2pi.rho.presentation} 
\subsection{The syntax and semantics of the notation system}\label{sub:the_syntax_and_semantics_of_the_notation_system} % (fold)

We now summarize a technical presentation of the calculus that
embodies our theory of dynamics. The typical presentation of such a
calculus follows the style of giving generators and relations on
them. The grammar, below, describing term constructors, freely
generates the set of processes, $\Proc$. This set is then quotiented
by a relation known as structural congruence and it is over this set
that the notion of dynamics is expressed. This presentation is
essentially that of \cite{MeredithR05} with the addition of
polyadicity and summation. For readability we have relegated some of
the technical subtleties to an appendix.

\subsubsection{Process grammar}\label{subsub:process_grammar}

\begin{mathpar}
  \inferrule* [lab=synchronization] {} {{M} \bc \pzero \;|\; x?F \;|\; x!C }
  \and
  \inferrule* [lab=abstraction] {} {{F} \bc (x)P}
  \and
  \inferrule* [lab=concretion] {} {{C} \bc \langle Q \rangle}
  \and
  \inferrule* [lab=process] {} {{P,Q} \bc M \;| \;P|Q \;|\; @{x}}
  \and
  \inferrule* [lab=name] {} {{x} \bc \quotep{P}}
\end{mathpar} 

Note that $\vec{x}$ (resp. $\vec{P}$) denotes a vector of names
(resp. processes) of length $|\vec{x}|$ (resp. $|\vec{P}|$). We adopt
the following useful abbreviations.

\begin{mathpar}
   x?(\vec{y}).P := x.(\vec{y})P \and  x\clift{\vec{P}} := x.\clift{\vec{P}}
   \and x!(y) := \lift{x}{\dropn{y}}
   \and \Pi_{i=0}^{n-1}P_i := P_0 | \ldots | P_{n-1}
\end{mathpar}

\subsubsection{Structural congruence}

\paragraph{Free and bound names and alpha-equivalence.} At the
core of structural equivalence is alpha-equivalence which identifies
process that are the same up to a change of variable. Formally, we
recognize the distinction between free and bound names. The free names
of a process, $\freenames{P}$, may be calculated recursively as
follows:

\begin{mathpar}
\freenames{\pzero} := \emptyset
  \and \\
  \freenames{x?(y).P} := \{ x \} \cup (\freenames{P} \setminus \{ y \})
  \and 
  \freenames{x!\langle P \rangle} := \{ x \} \cup \{ P \} 
  \and \\
  \freenames{P|Q} := \freenames{P} \cup \freenames{Q}
  \and \\
  \freenames{@{x}} := \{ x \}
\end{mathpar}

$\pi$
$\quotep{\pi}$

$\freenames{-} : \pi \to \mathcal{P}(\quotep{\pi})$

\begin{eqnarray*}
  \freenames{\pzero} & := & \emptyset \\
  \freenames{x?(y).P} & := & \{ x \} \cup (\freenames{P} \setminus \{ y \}) \\
  \freenames{x!\langle P \rangle} & := & \{ x \} \cup \{ P \} \\
  \freenames{P|Q} & := & \freenames{P} \cup \freenames{Q} \\
  \freenames{\dropn{x}} & := & \{ x \}
\end{eqnarray*}

The bound names of a process, $\boundnames{P}$, are those names occurring in $P$
that are not free. For example, in $x?(y).0$, the name $x$ is free, while $y$ is bound.

\begin{mathpar}
  \inferrule* [lab=monoidal-laws] {} { P|Q \equiv Q|P \and P|0 \equiv P \and P|(Q|R) \equiv (P|Q)|R }
\end{mathpar}

\begin{mathpar}
  \inferrule* [lab=alpha-equivalence] {} { (x)P \equiv (y)P\{y/x\} \and y \not\in \freenames{P} }
\end{mathpar}

\begin{definition}
Then two processes, $P,Q$, are alpha-equivalent if $P = Q\{\vec{y}/\vec{x}\}$ for
some $\vec{x} \in \boundnames{Q},\vec{y} \in \boundnames{P}$, where $Q\{\vec{y}/\vec{x}\}$
denotes the capture-avoiding substitution of $\vec{y}$ for $\vec{x}$ in $Q$.
\end{definition}

\begin{definition}
  The {\em structural congruence} \cite{SangiorgiWalker} , $\equiv$,
  between processes is the least congruence containing
  alpha-equivalence, satisfying the abelian monoid laws
  (associativity, commutativity and $\pzero$ as identity) for parallel
  composition $|$ and for summation $+$.
\end{definition}

\subsection{Name equivalence}

We take name equivalence, written $\nameeq$, to be the smallest
equivalence relation generated by the following rules.

\begin{mathpar}
\inferrule*[lab=Quote-drop]
{ }
{ \quotep{@{x}} \nameeq x }

\inferrule*[lab=Struct-equiv]
{ P \scong Q }
{ \quotep{P} \nameeq \quotep{Q} }
\end{mathpar}

The astute reader will have noticed that the mutual recursion of names
and processes imposes a mutual recursion on alpha-equivalence and
structural equivalence via name-equivalence. Fortunately, all of this
works out pleasantly and we may calculate in the natural way, free of
concern. The reader interested in the details is referred to the
appendix \ref{appendix:rho_details}.

\subsection{Substitution}

We use $\Proc$ for the set of processes, $\QProc$ for the set of
names, and $\id{\{}\vec{y} / \vec{x} \id{\}}$ to denote partial maps,
$s : \QProc \rightarrow \QProc$. A map, $s$ lifts, uniquely, to a map
on process terms, $\widehat{s} : \Proc \rightarrow \Proc$ by the
following equations.

\begin{mathpar}
  (0) \psubstp{Q}{P} := 0 \\
  (R \juxtap S) \psubstp{Q}{P}
  :=    
  (R)\psubstp{Q}{P} \juxtap (S) \psubstp{Q}{P} \\
  (x?(y).R) \psubstp{Q}{P}    
  :=    
  (x)\substp{Q}{P} (z)\concat( (R \psubstn{z}{y}) \psubstp{Q}{P} ) \\
  (\lift{x}{R}) \psubstp{Q}{P}  
  :=
  \lift{(x)\substp{Q}{P}}{ R \psubstp{Q}{P} } \\
%   (\dropn{x})  \psubstp{Q}{P}       
%   := 
%   \left\{ 
%     \begin{array}{ccc} 
%       \dropn{\quotep{Q}} & & x \nameeq \quotep{P} \\
%       \dropn{x} & & otherwise \\
%     \end{array}
%   \right. 
  (\dropn{x})  \psubstp{Q}{P}       
  := 
  \left\{ 
    \begin{array}{ccc} 
      Q & & x \nameeq \quotep{P} \\
      \dropn{x} & & otherwise \\
    \end{array}
  \right.
\end{mathpar}
 

where

\begin{eqnarray}
  (x)\id{\{} \lpquote Q \rpquote / \lpquote P \rpquote \id{\}}            = 
  \left\{ 
    \begin{array}{ccc}
      \lpquote Q \rpquote & & x \nameeq \lpquote P \rpquote \\
      x & & otherwise \\
    \end{array}
  \right. \nonumber
\end{eqnarray}

and $z$ is chosen distinct from $\quotep{P}$, $\quotep{Q}$, the free
names in $Q$, and all the names in $R$. Our $\alpha$-equivalence will
be built in the standard way from this substitution.

\begin{remark}\label{rem:no_self_referential_names}
  One consequence of these definitions is that $\forall P. \quotep{P}
  \not\in \freenames{P}$.
\end{remark}

\subsection{ Dynamic quote: an example }

Anticipating something of what's to come, consider applying the
substitution, $\widehat{\id{\{}u / z \id{\}}}$, to the following pair
of processes, $\lift{w}{y!(z)}$ and $w[ \lpquote y!(z) \rpquote ]$.

\begin{eqnarray}
	\lift{w}{y!(z)}\widehat{\id{\{}u / z \id{\}}}
		& = &
		\lift{w}{y!(u)} \nonumber\\
	w[ \lpquote y!(z) \rpquote ] \widehat{ \id{\{}u / z \id{\}} }
		& = &
		w[ \lpquote y!(z) \rpquote ] \nonumber
\end{eqnarray}

Because the body of the process between quotes is impervious to
substitution, we get radically different answers. In fact, by
examining the first process in an input context,
e.g. $x?(z).\lift{w}{y!(z)}$, we see that the process under the lift
operator may be shaped by prefixed inputs binding a name inside it. In
this sense, the lift operator will be seen as a way to dynamically
construct processes before reifying them as names.

Finally equipped with these standard features we can present the
dynamics of the calculus.

\subsubsection{Operational semantics} 

Finally, we introduce the computational dynamics. What marks these
algebras as distinct from other more traditionally studied algebraic
structures, e.g. vector spaces or polynomial rings, is the manner in
which dynamics is captured. In traditional structures, dynamics is typically
expressed through morphisms between such structures, as in linear maps
between vector spaces or morphisms between rings. In algebras
associated with the semantics of computation, the dynamics is
expressed as part of the algebraic structure itself, through a
reduction reduction relation typically denoted by $\red$. Below, we
give a recursive presentation of this relation for the calculus used
in the encoding.

$\red \subseteq \pi \times \pi$
$\red : \pi \to \mathcal{P}(\pi)$

\begin{mathpar}
  \inferrule* [lab=Comm] { \textsf{match}( x_{src}, x_{trgt} ) } { x_{trgt}?(y)P \; | \; x_{src}!\langle {Q} \rangle \red P\{\quotep{Q}/y}\} }
  \and \\
  \inferrule* [lab=Par] {{P} \red {P}'} {{{P} | {Q}} \red {{P}' | {Q}}}
  \and
  \inferrule* [lab=Equiv]{{{P} \scong {P}'} \andalso {{P}' \red {Q}'} \andalso {{Q}' \scong {Q}}}{{P} \red {Q}}
\end{mathpar}

\begin{eqnarray*}
  match_{\equiv} (\quotep{P},\quotep{Q}) & := & P \equiv Q \\
  match_{\dagger}(\quotep{P},\quotep{Q}) & := & \forall R. P|Q \red^{*} R => R \red^{*} 0 \\
  match_{K}(\quotep{P},\quotep{Q}) & := & K \mbox{ for some context } K
\end{eqnarray*}

$u?(x)P | u!\langle Q \rangle \red P\{\quotep{Q}/x\}$

%We write $\wred$ for $\red^*$, and $P\red$ if $\exists Q $ such that $ P \red Q$.
We write $P\red$ if $\exists Q $ such that $ P \red Q$ and $P\not\red$, otherwise.

\section{Replication}

As mentioned before, it is known that replication (and hence
recursion) can be implemented in a higher-order process algebra
\cite{SangiorgiWalker}. As our first example of calculation with the
machinery thus far presented we give the construction explicitly in
the {\rhoc}.

\begin{eqnarray}
	D_{x} & := & \prefix{x}{y}{(\binpar{\outputp{x}{y}}{@{y}})} \nonumber\\
	\bangp_{x}{P} & := & \binpar{{x}!\langle{\binpar{D_{x}}{P}}\rangle}{D_{x}} \nonumber
\end{eqnarray}

\begin{eqnarray}
	\bangp_{x}{P} & & \nonumber\\
	=
	& {x}!\langle{(\prefix{x}{y}{(\outputp{x}{y} | @{y})) | P}}\rangle 
	      | \prefix{x}{y}{(\outputp{x}{y} | @{y})} & \nonumber\\
	\red
	& (\outputp{x}{y} | @{y})\substn{\quotep{(\prefix{x}{y}{(@{y} | \outputp{x}{y})) | P}}}{y} & \nonumber\\
	=
	& \outputp{x}{\quotep{(\prefix{x}{y}{(\outputp{x}{y} | @{y})) | P}}}
	  | {(\prefix{x}{y}{(\outputp{x}{y} | @{y})) | P}} & \nonumber\\
	\red
	& \ldots & \nonumber\\
	\red^*
	& P | P | \ldots & \nonumber
\end{eqnarray}

Of course, this encoding, as an implementation, runs away, unfolding
$\bangp{P}$ eagerly. A lazier and more implementable replication
operator, restricted to input-guarded processes, may be obtained as follows.

\begin{eqnarray}
\bangp{\prefix{u}{v}{P}} 
	:= 
	\binpar{\lift{x}{\prefix{u}{v}{(\binpar{D(x)}{P})}}}{D(x)} \nonumber
\end{eqnarray}

\begin{remark}
  Note that the lazier definition still does not deal with summation
  or mixed summation (i.e. sums over input and output). The reader is
  invited to construct definitions of replication that deal with these
  features. 

  Further, the definitions are parameterized in a name, $x$. Can you,
  gentle reader, make a definition that eliminates this parameter and
  guarantees no accidental interaction between the replication
  machinery and the process being replicated -- i.e. no accidental
  sharing of names used by the process to get its work done and the
  name(s) used by the replication to effect copying. This latter
  revision of the definition of replication is crucial to obtaining
  the expected identity $!!P \sim !P$.
\end{remark}

\begin{remark}\label{rem:paradoxical_combinator}
  The reader familiar with the lambda calculus will have noticed the
  similarity between $D$ and the paradoxical combinator.

  [Ed. note: the existence of this seems to suggest we have to be more
  restrictive on the set of processes and names we admit if we are to
  support no-cloning.]
\end{remark}

\subsubsection{Bisimulation}

The computational dynamics gives rise to another kind of equivalence,
the equivalence of computational behavior. As previously mentioned
this is typically captured \emph{via} some form of bisimulation.

% The notion we use in this paper is weak barbed bisimulation
% \cite{milner91polyadicpi}.

The notion we use in this paper is derived from weak barbed
bisimulation \cite{milner91polyadicpi}. 

\begin{definition}
An \emph{observation relation}, $\downarrow_{\mathcal N}$, over a set
of names, $\mathcal N$, is the smallest relation satisfying the rules
below.

\infrule[Out-barb]{y \in {\mathcal N}, \; x \nameeq y}
		  {\outputp{x}{v} \downarrow_{\mathcal N} x}
\infrule[Par-barb]{\mbox{$P\downarrow_{\mathcal N} x$ or $Q\downarrow_{\mathcal N} x$}}
		  {\binpar{P}{Q} \downarrow_{\mathcal N} x}

We write $P \Downarrow_{\mathcal N} x$ if there is $Q$ such that 
$P \wred Q$ and $Q \downarrow_{\mathcal N} x$.
\end{definition}

\begin{definition}
%\label{def.bbisim}
An  ${\mathcal N}$-\emph{barbed bisimulation} over a set of names, ${\mathcal N}$, is a symmetric binary relation 
${\mathcal S}_{\mathcal N}$ between agents such that $P\rel{S}_{\mathcal N}Q$ implies:
\begin{enumerate}
\item If $P \red P'$ then $Q \wred Q'$ and $P'\rel{S}_{\mathcal N} Q'$.
\item If $P\downarrow_{\mathcal N} x$, then $Q\Downarrow_{\mathcal N} x$.
\end{enumerate}
$P$ is ${\mathcal N}$-barbed bisimilar to $Q$, written
$P \wbbisim_{\mathcal N} Q$, if $P \rel{S}_{\mathcal N} Q$ for some ${\mathcal N}$-barbed bisimulation ${\mathcal S}_{\mathcal N}$.
\end{definition}

$\mathcal{R} \subseteq \pi \times \pi$

$P \mathcal{R} Q => \forall P'. P \red P' \Rightarrow \exists Q'. Q \red Q', P' \mathcal{R} Q'$

$P \vdash x \Rightarrow Q \vdash x$

\begin{mathpar}
  \inferrule*[lab=Out-barb]{x \nameeq y}{{y}!\langle{Q}\rangle \vdash x}
  \and
  \inferrule*[lab=Par-barb]{\mbox{$P\vdash x$ or $Q\vdash x$}}{\binpar{P}{Q} \vdash x}
\end{mathpar}

\subsubsection{Contexts}

One of the principle advantages of computational calculi like the
$\pi$-calculus is a well-defined notion of context,
contextual-equivalence and a correlation between
contextual-equivalence and notions of bisimulation. The notion of
context allows the decomposition of a process into (sub-)process and
its syntactic environment, its context. Thus, a context may be
thought of as a process with a ``hole'' (written $\Box$) in it. The
application of a context $M$ to a process $P$, written $M[P]$, is
tantamount to filling the hole in $M$ with $P$. In this paper we do
not need the full weight of this theory, but do make use of the notion
of context in the proof the main theorem. 

\begin{mathpar}
  \inferrule* [lab=summation] {} {{M_{M},M_{N}} \bc \Box \;|\; x.M_{A} \;|\; M_{M}+M_{N}}
  \and
  \inferrule* [lab=agent] {} {{M_{A}} \bc (\vec{x})M_{P} \;| \; \clift{P_0,\ldots,M_{P},\ldots,P_N}}
  \and \\
  \inferrule* [lab=process] {} {{M_{P}} \bc M_{N} \;| \;P|M_{P} }
\end{mathpar} 

\begin{mathpar}
  \inferrule* [lab=sychronization] {} {M_{N} \bc \Box \;|\; x?M_{F} \;|\; x!M_{C}}
  \and
  \inferrule* [lab=abstraction] {} {{M_{F}} \bc (x)M_{P} }
  \and
  \inferrule* [lab=concretion] {} {{M_{C}} \bc \langle M_{P} \rangle }
  \and \\
  \inferrule* [lab=process] {} {{M_{P}} \bc M_{N} \;| \;P|M_{P} }
\end{mathpar}

\begin{definition}[contextual application] Given a context $M$, and
  process $P$, we define the \emph{contextual application}, $M[P] :=
  M\{P/\Box\}$. That is, the contextual application of M to P is the
  substitution of $P$ for $\Box$ in $M$.
\end{definition}

$\meaningof{-} : L \to \mathcal{P}(\pi)$

\begin{mathpar}
  \inferrule* [lab=collection] {} {\meaningof{true} = \pi, \and \meaningof{~E} = \pi \setminus \meaningof{E}, \and \meaningof{E_{1} \& E_{2}} = \meaningof{E_{1}} \cap \meaningof{E_{2}}}
\end{mathpar}

\begin{mathpar}
  \inferrule* [lab=structure] {} {\meaningof{0} = \{ P \in \pi | P \equiv 0 \}, \and \\ \meaningof{E_1 | E_2} = \{ P \in \pi | P \equiv P_{1} | P_{2}, P_{1} \in \meaningof{E_{1}}, P_{2} \in \meaningof{E_2}\} }
\end{mathpar}

\begin{mathpar}
 \inferrule* [lab=behavior] {} {\meaningof{\langle a?b \rangle E} = \{ P \in \pi | P \equiv Q | u?(y)P', \\ \and \\\\ \and \\ \;\;\; u \in \meaningof{a}, \forall z.P'\{z/y\} \in \meaningof{E\{z/b\}}\}, \and \\ \meaningof{a!E} = \{ P \in \pi | P \equiv Q | x!\langle P' \rangle, x \in \meaningof{a} P' \in \meaningof{E}\} }
\end{mathpar}

\begin{mathpar}
 \inferrule* [lab=nominal] {} {\meaningof{\quotep{E}} = \{ \quotep{P} \in \quotep{\pi} | P \in \meaningof{E} \}, \and \meaningof{\quotep{P}} = \{ \quotep{Q} \in \quotep{\pi} | P \equiv Q \} \and \\ \meaningof{@\quotep{E}} = \{ P \in \pi | P \equiv @x, x \in \meaningof{E} \}}
\end{mathpar}

\begin{eqnarray*}
  \\
  \meaningof{-} : TS \to ST
\end{eqnarray*}

\begin{eqnarray*}
  \\
  L : TS \to ST
\end{eqnarray*}

\begin{eqnarray*}
  \\
  P \models E \iff P \in \meaningof{E}
\end{eqnarray*}

\begin{eqnarray*}
  P \approx_{L} Q \iff \forall E \in L. P \models E \iff Q \models E
\end{eqnarray*}

\begin{eqnarray*}
  P \approx_{K} Q
\end{eqnarray*}

\begin{eqnarray*}
  P \approx Q
\end{eqnarray*}

$\approx_{K} = \approx = \approx_{L}$

\subsubsection{Contextual duality}

Note that contexts extend the quotation operation to a family of
operations from processes to names. Given a context, $M$, we can
define a \emph{nominal context}, $\quotep{M}$ by $\quotep{M}[P] :=
\quotep{M[P]}$. To foreshadow what is to come we observe that these
operations enjoy a duality with processes very much like the duality
between vectors and maps from vectors to scalars.

Further, because the calculus is essentially higher-order, we have a
correspondence between contexts and processes. More specifically,
given a name $x$ and a context $M$ we can construct $M^{*}_{x}$ such
that 

\begin{mathpar}
  M^{*}_{x} | \lift{x}{P} \red M[P]
\end{mathpar}

namely,

\begin{mathpar}
  M^{*}_{x} := x?(u).M[\dropn{u}]
\end{mathpar}

The dependence of $M^{*}_{x}$ on a name makes it an abstraction, 

\begin{mathpar}
  M^{*} := (x)x?(u).M[\dropn{u}]
\end{mathpar}

\subsection{Additional notation}

It will sometimes be convenient to denote the process a name
quotes. We already have the notation $x = \quotep{P}$, but it will be
convenient to introduce an alternate notation, $\procn{x}$, when we
want to emphasize the connection to the use of the name. Note that, by
virtue of name equivalence, $\quotep{\procn{x}} \nameeq x$; so, the
notation is consistent with previous definitions.

Further, because names have structure it is possible to effect
substitutions on the basis of that structure. This means we need to
upgrade our notation for substitutions, which we accomplish by
adapting comprehension notation. Thus,

\begin{mathpar}
  P\{ y / x : x \in S \}
\end{mathpar}

is interpreted to mean the process derived from P by replacing (in a
capture-avoiding manner) each occurrence of $x$ in $S$ by $y$. For example,

\begin{mathpar}
  P\{ \quotep{\procn{x}|\procn{x}} / x : x \in \freenames{P} \}
\end{mathpar}

will replace each (occurrence) of a free name $x$ in $P$ by
$\quotep{\procn{x}|\procn{x}}$.

Also, we will avail ourselves of the notation $x^{L}$ and $x^{R}$ to
denote injections of a name into disjoint copies of the name
space. There are numerous ways to accomplish this. One example can be
found in \cite{MeredithR05}. This notation overloads to vectors of
names: $\vec{x}^{\pi} := (x_{i}^{\pi} \; : \; 0 \leq i < |\vec{x}| )$ where $\pi \in \{L,R\}$.

We also use $P^{\Box} := P|\Box$.

In \cite{MeredithR05} an interpretation of the new operator is
given. It turns out that there are several possible interpretations
all enjoying the requisite algebraic properties of the operator (see
\cite{milner91polyadicpi}). We will therefore make liberal use of
$(\nu\; \vec{x})P$.

% subsection the_syntax_and_semantics_of_the_notation_system (end)   

\input{qm2pi.qmops} 

\input{qm2pi.sterngerlach} 

\input{qm2pi.metric} 

% section concurrent_process_calculi (end)

%\input{qm2pi.proofsketch}

% section proof sketch (end)

%\input{qm2pi.slviaknots} 

% section spatial logic via knots (end)

\input{qm2pi.conclusion}

% section conclusion (end)

%\input{qm2pi.dtcodes} 

% section wiring algorithm (end)

\input{qm2pi.ack} 

% section acknowledgments (end)

\newpage


\bibliographystyle{plain}   
\bibliography{../../biblios/main.bib}

\input{qm2pi.rhodetails}

\end{document}



% section proof sketch (end)

%\section{Unlikely characters: spatial logic for
  knots}\label{sub:characteristic_formulae} % (fold)

Associated to the mobile process calculi are a family of logics known
as the Hennessy-Milner logics. These logics typically enjoy a
semantics interpreting formulae as sets of processes that when
factored through the encoding outlined above allows an identification
of classes of knots with logical formulae. In the context of this
encoding the sub-family known as the spatial logics \cite{CairesC03}
\cite{CairesC04} \cite{Caires04} are of particular interest providing
several important features for expressing and reasoning about
properties (i.e. classes) of knots. We hint here at how this may be done.

%\begin{description}
%\item [structural connectives] 
\subsubsection{Structural connectives} The spatial logics enjoy
structural connectives corresponding, at the logical level, to the
parallel composition ($P | Q$) and new name ($(\nu \; x)P$)
connectives for processes. As illustrated in the examples below, these
connectives are extremely expressive given the shape of our encoding.
%\item [decideable satisfaction]

\subsubsection{Decideable satisfaction}
In \cite{Caires04} the satisfaction relation is shown to be decideable
for a rich class of processes. It further turns out that the image of
the our encoding is a proper subset of that class. This result
provides the basis for an algorithm by which to search for knots
enjoying a given property.
%\item [characteristic formulae]

\subsubsection{Characteristic formulae}
In the same paper \cite{Caires04} , Caires presents a means of calculating
characteristic formulae, selecting equivalence classes of processes
up to a pre--specified depth limit on the support set of names. Composed with our
encoding, this characteristic formula can be used to select
characteristic formulae for knots.
%\end{description}

\subsubsection{Spatial logic formulae}

The grammar below (segmented for comprehension) summarizes the syntax
of spatial logic formulae. We employ illustrative examples in the
sequel to provide an intuitive understanding of their meaning
referring the reader to \cite{Caires04} for a more detailed explication
of the semantics.

\begin{mathpar}
  \inferrule* [lab=boolean] {} {{A,B} \bc T \;|\; \neg A \;|\; A \wedge B \;|\; \eta = \eta'}
  \and
  \inferrule* [lab=spatial] {} {|\; \pzero \;|\; A | B \;|\; x \text{\textregistered} A \;|\; \forall x . A \;|\;  H x . A}
  \and
  \inferrule* [lab=behavioral] {} {|\; \alpha . A}
  \and 
  \inferrule* [lab=recursion] {} {|\; X(\vec{u}) \;|\; \mu X(\vec{u}) . A}
  \and
  \inferrule* [lab=action] {} {\alpha \bc \langle x?(\vec{y}) \rangle \;|\; \langle x!(\vec{y}) \rangle \;|\; \langle \tau \rangle}
  \and 
  \inferrule* [lab=name] {} {\eta \bc x \;|\; \tau}
\end{mathpar} 

% subsection characteristic_formulae (end)   	 

\subsection{Example formulae}\label{sub:example_formulae_} % (fold)

\subsubsection{Crossing as formula.}
% 
% \begin{align*}
%   \frac{d}{dx} \sin x &= \cos x 
%   & \frac{d}{dx} e^x &= e^x \\
%   \frac{d}{dx} \cos x &= - \sin x 
%   & \frac{d}{dx} \log x &= \frac{1}{x} \\
% \end{align*} 

\begin{align*}
 \mu C(x_{0},x_{1},y_{0},y_{1},u).&(\langle x_{0}?(z) \rangle(\langle u! \rangle\langle y_{1}!z \rangle C(x_{0},x_{1},y_{0},y_{1},u)) & \\
  & \wedge \langle y_{1}?(z) \rangle (\langle u! \rangle \langle x_{0}!z \rangle C(x_{0},x_{1},y_{0},y_{1},u)) & \\
  & \wedge \langle x_{1}?(z) \rangle (\langle u? \rangle \langle y_{0}!z \rangle C(x_{0},x_{1},y_{0},y_{1},u)) & \\
  & \wedge \langle y_{0}?(z) \rangle (\langle u? \rangle \langle x_{1}!z \rangle C(x_{0},x_{1},y_{0},y_{1},u))) &
\end{align*}

The lexicographical similarity between the shape of this formulae and
the shape of definition of the process representing a crossing reveals
the intuitive meaning of this formulae. It describes the capabilities
of a process that has the right to represent a crossing. For example
it picks out processes that may perform an input on the port $x_0$ in
its initial menu of capabilities. What differentiates the formula
from the process, however, is that the crossing process is the
smallest candidate to satisfy the formula. Infinitely many other
processes -- with internal behavior hidden behind this interface, so
to speak -- also satisfy this formula. Even this simple formula,
then, can be seen to open a new view onto knots, providing a
computational interpretation of \emph{virtual} knots.

Note that this formula is derived by hand. A similar formula can be
derived by employing Caires' calculation of characteristic formula
\cite{Caires04} to the process representing a crossing. In light of
this discussion, we let
$\meaningof{C}_{\phi}(x0,x1,y0,y1,u)$ denote a formula specifying the
dynamics we wish to capture of a crossing. To guarantee we preserve
the shape of the interface and minimal semantics we demand that
$\meaningof{C}_{\phi}(x0,x1,y0,y1,u) \Rightarrow
\textbf{C}(x0,x1,y0,y1,u)$ where $\textbf{C}(x0,x1,y0,y1,u)$ denotes
the formula above.
                            
\subsubsection{Crossing number constraints.}
The moral content of the context lemma (Lemma \ref{context}) is that the notion of
``locality'' in the Reidemeister moves is effectively captured by the
parallel composition operator of the process calculus. This intuition
extends through the logic. Given a formula,
$\meaningof{C}_{\phi}(x0,x1,y0,y1,u)$, we can use the structural
connectives to specify constraints on crossing numbers, such as at
least $n$ crossings, or exactly $n$ crossings.
\begin{mathpar}
  \inferrule* [lab=at-least-n] {} { K^{\geq n}_{\phi}(\vec{xs},\vec{ys}) := \Pi_{i=0}^{n-1} Hu . \meaningof{C}_{\phi}(xs_i,ys_i,u) | T }
  \and 
  \inferrule* [lab=exactly-n] {} { K^{= n}_{\phi}(\vec{xs},\vec{ys}) := \Pi_{i=0}^{n-1} Hu . \meaningof{C}_{\phi}(xs_i,ys_i,u) | \neg (\forall x_0,y_0,x_1,y_1,u . \meaningof{C}_{\phi}(x_0,y_0,x_1,y_1,u) | T) }
\end{mathpar}

To round out this section, recall that the encoding of an $n$-crossing
knot decomposes into a parallel composition of $n$ \emph{copies} of a
crossing process together with a wiring harness. To specify different
knot classes with the same crossing number amounts to specifying
logical constraints on the wiring harness. In the interest of space,
we defer examples to a forthcoming paper. Suffice it to say that both
the conditions ``alternating knot'' and ``contains the tangle
corresponding to 5/3'' are expressible. For example, it is possible to
calculate the characteristic formula of a process corresponding to the
tangle 5/3 and conjoin it into the classifying formula via the
composition connective of the logic.

Finally, we wish to observe that it is entirely within reason to
contemplate a more domain-specific version of spatial logic tailored
to the shape of processes in the image of the encoding. Such a
domain-specific logic would have a better claim to the title formal
language of knot properties.

% subsection example_formulae_ (end)

% section knots_as_processes (end) 

% section spatial logic via knots (end)

\section{Conclusions and future work}

\paragraph{Testing physical space}
You, gentle reader, may wonder why of all the theorems to be proved
given this set up we pick the one above. In some sense it's hardly
central to quantum mechanics. We see it as central in the sense that
it firmly establishes a notion of physical space arising from a notion
of the equivalence of behavior. Relating bisimulation to a metric is a
big step forward, but one is faced with interpreting the relationship
of that metric space to something more physical. Quantum mechanical
notions of ``physical'' space are still far from intuitive, but by
relating this idea of distance as testing to calculations that predict
physical circumstances we are making a not insignificant step forward
toward an understanding of the physical space we inhabit as
essentially dynamic.

\paragraph{Effectivity and simulation}
One of the observations we have yet to make is that the entire program
spelled out here is effective. We have built various interpreters for
the reflective calculus at work in this interpretation. In principle,
then, we can simulate quantum mechanics on a computer. The place where
the simulation may lose fidelity is the infinitely branching summation
for the annihilator.

In this connection i also want to point out that the evaluation style
calculation of the inner product puts the non-determinism of the
summation right at the heart of measurement. This suggests that
Milner's original reduction-based formulation of the dynamics of his
calculi in terms of sums was not just notationally suggestive of a
notion of measure-and-continue but captured some significant part of
the physics.

\paragraph{Quantum continuations}
In light of this last observation i want to point out that the
predominant account of quantum mechanics is missing a key aspect of a
truly compositional story of the physical situation. In a real lab,
when a measurement is made the observation can be made to feed into
another device that then makes another measurement conditioned on the
results of the first. This means that after the superposition was
collapsed the entire experimental set up remained in
superposition. While QM offers a means of writing this down it doesn't
quite line up well with the well-trodden formulation of computation
and continuation that we see so succinctly expressed in Milner's
calculi. This suggests that there might be advantages to this account
of dynamics waiting to be explored.

\paragraph{Quantum logic}
In this connection, we also note that by virtue of having the
Hennessy-Milner construction, we can pull the construction through the
interpretation of QM. This gives us a natural candidate for a quantum
logic that enjoys an extremely tight connection with it's domain of
interpretation, making the construction much less ad hoc (rather it is
the image of functor!).

\paragraph{Quantum probabiity}
i have questions about the basis of the interpretation of inner
product as probability amplitude. In particular, using which
axiomatization of probability theory does the notion of probability
amplitude earn the right to be so dubbed? In other words, where is the
proof that the operation for calculating a probability amplitude (and
then squaring) satisfies the axioms of what it means to calculate a
probability? Even if such a proof exists (i have yet to find it in the
literature), i wonder if it might not be possible to turn things on
their heads. Can we view the calculation of the probability amplitude
as an axiomatization of probability? If so, then the definition we
give for calculating probability amplitude may provide the basis for
an \emph{effective} theory of probability.

\paragraph{Quantum vs ``biological'' information}
Finally, i want to conclude with a more philosophical observation. At
a recent workshop in which QM was a predominant topic i noticed
something about quantum information. The speaker was giving a riveting
discussion of axiomatic QM and showing how properties of ``no
cloning'' and ``no deleting'' emerged as consequences of the
axiomatization. Theorems of this form are necessary to give us a sense
of confidence that our axioms characterize the physical theory. What
struck me, though, was that if quantum information is neither erasable
nor replicable it is markedly different from \emph{life}. Two of the
things we know about life is that

\begin{itemize}
  \item it ends;
  \item to gain some measure of persistence, to transcend it's
    finitude it is imminently copyable.
\end{itemize}

Both of these qualities are summarized succinctly in the aphorism: all
flesh is grass. For me these two kinds of ``information'' -- call them
quantum and biological -- are end points on a spectrum of strategies
for persistence. At one end, we have those curious entities that enjoy
uniqueness and permanence; at the other, we have those who in the face
of a certain end and an uncertain present make a go of passing
something on. To me one of the more remarkable aspects of the latter
strategy is that in the presence of noise (and certain features of
copying) we get a kind of dynamism, a chance for improvement against a
given persistent condition.

% subsection other_calculi_other_bisimulations_and_geometry_as_behavior (end)




% section conclusion (end)

%\documentclass[12pt]{llncs}
%\documentclass{jktr}

\usepackage[pdftex]{hyperref}                   
\usepackage {listings}
\usepackage {mathpartir}
\usepackage{bcprules}
%\usepackage{listings}
                       
\usepackage{graphicx} 
%\usepackage[margins=2.5cm,nohead,nofoot]{geometry}
%\usepackage{geometry}
\usepackage{amsfonts}
\usepackage{amstext}
\usepackage{latexsym}
\usepackage{amssymb}
\usepackage{color}


%\include{myPreamble}
\include{qm2pi.local} 

%\ifpdf
%\usepackage[pdftex]{graphicx}
%\else
%\usepackage{graphicx}
%\fi

 % \ifpdf
%  \usepackage{pdfsync}
%  \if


%\title{Brief Article}
%\author{David F. Snyder}
%\author{L.G. Meredith}

%\address{Dept. of Math., Texas State University--San Marcos, San Marcos, TX 78666}
       
\pagestyle{empty}


\begin{document}

\lstset{language=[Objective]Caml,frame=shadowbox}

\input{qm2pi.front}

% section front matter (end)

\input{qm2pi.intro} 
 
% section introduction (end)

% \input{qm2pi.knotations} 

% section notation (end)

\input{qm2pi.process.calculi} 

% section concurrent_process_calculi_and_spatial_logics_ (end)
    
%\input{qm2pi.knots2pi} 

%\input{qm2pi.trefoil} 

%\input{qm2pi.mainthm} 

% subsection basic_interpretation (end)

%\input{qm2pi.rho.presentation} 
\subsection{The syntax and semantics of the notation system}\label{sub:the_syntax_and_semantics_of_the_notation_system} % (fold)

We now summarize a technical presentation of the calculus that
embodies our theory of dynamics. The typical presentation of such a
calculus follows the style of giving generators and relations on
them. The grammar, below, describing term constructors, freely
generates the set of processes, $\Proc$. This set is then quotiented
by a relation known as structural congruence and it is over this set
that the notion of dynamics is expressed. This presentation is
essentially that of \cite{MeredithR05} with the addition of
polyadicity and summation. For readability we have relegated some of
the technical subtleties to an appendix.

\subsubsection{Process grammar}\label{subsub:process_grammar}

\begin{mathpar}
  \inferrule* [lab=synchronization] {} {{M} \bc \pzero \;|\; x?F \;|\; x!C }
  \and
  \inferrule* [lab=abstraction] {} {{F} \bc (x)P}
  \and
  \inferrule* [lab=concretion] {} {{C} \bc \langle Q \rangle}
  \and
  \inferrule* [lab=process] {} {{P,Q} \bc M \;| \;P|Q \;|\; @{x}}
  \and
  \inferrule* [lab=name] {} {{x} \bc \quotep{P}}
\end{mathpar} 

Note that $\vec{x}$ (resp. $\vec{P}$) denotes a vector of names
(resp. processes) of length $|\vec{x}|$ (resp. $|\vec{P}|$). We adopt
the following useful abbreviations.

\begin{mathpar}
   x?(\vec{y}).P := x.(\vec{y})P \and  x\clift{\vec{P}} := x.\clift{\vec{P}}
   \and x!(y) := \lift{x}{\dropn{y}}
   \and \Pi_{i=0}^{n-1}P_i := P_0 | \ldots | P_{n-1}
\end{mathpar}

\subsubsection{Structural congruence}

\paragraph{Free and bound names and alpha-equivalence.} At the
core of structural equivalence is alpha-equivalence which identifies
process that are the same up to a change of variable. Formally, we
recognize the distinction between free and bound names. The free names
of a process, $\freenames{P}$, may be calculated recursively as
follows:

\begin{mathpar}
\freenames{\pzero} := \emptyset
  \and \\
  \freenames{x?(y).P} := \{ x \} \cup (\freenames{P} \setminus \{ y \})
  \and 
  \freenames{x!\langle P \rangle} := \{ x \} \cup \{ P \} 
  \and \\
  \freenames{P|Q} := \freenames{P} \cup \freenames{Q}
  \and \\
  \freenames{@{x}} := \{ x \}
\end{mathpar}

$\pi$
$\quotep{\pi}$

$\freenames{-} : \pi \to \mathcal{P}(\quotep{\pi})$

\begin{eqnarray*}
  \freenames{\pzero} & := & \emptyset \\
  \freenames{x?(y).P} & := & \{ x \} \cup (\freenames{P} \setminus \{ y \}) \\
  \freenames{x!\langle P \rangle} & := & \{ x \} \cup \{ P \} \\
  \freenames{P|Q} & := & \freenames{P} \cup \freenames{Q} \\
  \freenames{\dropn{x}} & := & \{ x \}
\end{eqnarray*}

The bound names of a process, $\boundnames{P}$, are those names occurring in $P$
that are not free. For example, in $x?(y).0$, the name $x$ is free, while $y$ is bound.

\begin{mathpar}
  \inferrule* [lab=monoidal-laws] {} { P|Q \equiv Q|P \and P|0 \equiv P \and P|(Q|R) \equiv (P|Q)|R }
\end{mathpar}

\begin{mathpar}
  \inferrule* [lab=alpha-equivalence] {} { (x)P \equiv (y)P\{y/x\} \and y \not\in \freenames{P} }
\end{mathpar}

\begin{definition}
Then two processes, $P,Q$, are alpha-equivalent if $P = Q\{\vec{y}/\vec{x}\}$ for
some $\vec{x} \in \boundnames{Q},\vec{y} \in \boundnames{P}$, where $Q\{\vec{y}/\vec{x}\}$
denotes the capture-avoiding substitution of $\vec{y}$ for $\vec{x}$ in $Q$.
\end{definition}

\begin{definition}
  The {\em structural congruence} \cite{SangiorgiWalker} , $\equiv$,
  between processes is the least congruence containing
  alpha-equivalence, satisfying the abelian monoid laws
  (associativity, commutativity and $\pzero$ as identity) for parallel
  composition $|$ and for summation $+$.
\end{definition}

\subsection{Name equivalence}

We take name equivalence, written $\nameeq$, to be the smallest
equivalence relation generated by the following rules.

\begin{mathpar}
\inferrule*[lab=Quote-drop]
{ }
{ \quotep{@{x}} \nameeq x }

\inferrule*[lab=Struct-equiv]
{ P \scong Q }
{ \quotep{P} \nameeq \quotep{Q} }
\end{mathpar}

The astute reader will have noticed that the mutual recursion of names
and processes imposes a mutual recursion on alpha-equivalence and
structural equivalence via name-equivalence. Fortunately, all of this
works out pleasantly and we may calculate in the natural way, free of
concern. The reader interested in the details is referred to the
appendix \ref{appendix:rho_details}.

\subsection{Substitution}

We use $\Proc$ for the set of processes, $\QProc$ for the set of
names, and $\id{\{}\vec{y} / \vec{x} \id{\}}$ to denote partial maps,
$s : \QProc \rightarrow \QProc$. A map, $s$ lifts, uniquely, to a map
on process terms, $\widehat{s} : \Proc \rightarrow \Proc$ by the
following equations.

\begin{mathpar}
  (0) \psubstp{Q}{P} := 0 \\
  (R \juxtap S) \psubstp{Q}{P}
  :=    
  (R)\psubstp{Q}{P} \juxtap (S) \psubstp{Q}{P} \\
  (x?(y).R) \psubstp{Q}{P}    
  :=    
  (x)\substp{Q}{P} (z)\concat( (R \psubstn{z}{y}) \psubstp{Q}{P} ) \\
  (\lift{x}{R}) \psubstp{Q}{P}  
  :=
  \lift{(x)\substp{Q}{P}}{ R \psubstp{Q}{P} } \\
%   (\dropn{x})  \psubstp{Q}{P}       
%   := 
%   \left\{ 
%     \begin{array}{ccc} 
%       \dropn{\quotep{Q}} & & x \nameeq \quotep{P} \\
%       \dropn{x} & & otherwise \\
%     \end{array}
%   \right. 
  (\dropn{x})  \psubstp{Q}{P}       
  := 
  \left\{ 
    \begin{array}{ccc} 
      Q & & x \nameeq \quotep{P} \\
      \dropn{x} & & otherwise \\
    \end{array}
  \right.
\end{mathpar}
 

where

\begin{eqnarray}
  (x)\id{\{} \lpquote Q \rpquote / \lpquote P \rpquote \id{\}}            = 
  \left\{ 
    \begin{array}{ccc}
      \lpquote Q \rpquote & & x \nameeq \lpquote P \rpquote \\
      x & & otherwise \\
    \end{array}
  \right. \nonumber
\end{eqnarray}

and $z$ is chosen distinct from $\quotep{P}$, $\quotep{Q}$, the free
names in $Q$, and all the names in $R$. Our $\alpha$-equivalence will
be built in the standard way from this substitution.

\begin{remark}\label{rem:no_self_referential_names}
  One consequence of these definitions is that $\forall P. \quotep{P}
  \not\in \freenames{P}$.
\end{remark}

\subsection{ Dynamic quote: an example }

Anticipating something of what's to come, consider applying the
substitution, $\widehat{\id{\{}u / z \id{\}}}$, to the following pair
of processes, $\lift{w}{y!(z)}$ and $w[ \lpquote y!(z) \rpquote ]$.

\begin{eqnarray}
	\lift{w}{y!(z)}\widehat{\id{\{}u / z \id{\}}}
		& = &
		\lift{w}{y!(u)} \nonumber\\
	w[ \lpquote y!(z) \rpquote ] \widehat{ \id{\{}u / z \id{\}} }
		& = &
		w[ \lpquote y!(z) \rpquote ] \nonumber
\end{eqnarray}

Because the body of the process between quotes is impervious to
substitution, we get radically different answers. In fact, by
examining the first process in an input context,
e.g. $x?(z).\lift{w}{y!(z)}$, we see that the process under the lift
operator may be shaped by prefixed inputs binding a name inside it. In
this sense, the lift operator will be seen as a way to dynamically
construct processes before reifying them as names.

Finally equipped with these standard features we can present the
dynamics of the calculus.

\subsubsection{Operational semantics} 

Finally, we introduce the computational dynamics. What marks these
algebras as distinct from other more traditionally studied algebraic
structures, e.g. vector spaces or polynomial rings, is the manner in
which dynamics is captured. In traditional structures, dynamics is typically
expressed through morphisms between such structures, as in linear maps
between vector spaces or morphisms between rings. In algebras
associated with the semantics of computation, the dynamics is
expressed as part of the algebraic structure itself, through a
reduction reduction relation typically denoted by $\red$. Below, we
give a recursive presentation of this relation for the calculus used
in the encoding.

$\red \subseteq \pi \times \pi$
$\red : \pi \to \mathcal{P}(\pi)$

\begin{mathpar}
  \inferrule* [lab=Comm] { \textsf{match}( x_{src}, x_{trgt} ) } { x_{trgt}?(y)P \; | \; x_{src}!\langle {Q} \rangle \red P\{\quotep{Q}/y}\} }
  \and \\
  \inferrule* [lab=Par] {{P} \red {P}'} {{{P} | {Q}} \red {{P}' | {Q}}}
  \and
  \inferrule* [lab=Equiv]{{{P} \scong {P}'} \andalso {{P}' \red {Q}'} \andalso {{Q}' \scong {Q}}}{{P} \red {Q}}
\end{mathpar}

\begin{eqnarray*}
  match_{\equiv} (\quotep{P},\quotep{Q}) & := & P \equiv Q \\
  match_{\dagger}(\quotep{P},\quotep{Q}) & := & \forall R. P|Q \red^{*} R => R \red^{*} 0 \\
  match_{K}(\quotep{P},\quotep{Q}) & := & K \mbox{ for some context } K
\end{eqnarray*}

$u?(x)P | u!\langle Q \rangle \red P\{\quotep{Q}/x\}$

%We write $\wred$ for $\red^*$, and $P\red$ if $\exists Q $ such that $ P \red Q$.
We write $P\red$ if $\exists Q $ such that $ P \red Q$ and $P\not\red$, otherwise.

\section{Replication}

As mentioned before, it is known that replication (and hence
recursion) can be implemented in a higher-order process algebra
\cite{SangiorgiWalker}. As our first example of calculation with the
machinery thus far presented we give the construction explicitly in
the {\rhoc}.

\begin{eqnarray}
	D_{x} & := & \prefix{x}{y}{(\binpar{\outputp{x}{y}}{@{y}})} \nonumber\\
	\bangp_{x}{P} & := & \binpar{{x}!\langle{\binpar{D_{x}}{P}}\rangle}{D_{x}} \nonumber
\end{eqnarray}

\begin{eqnarray}
	\bangp_{x}{P} & & \nonumber\\
	=
	& {x}!\langle{(\prefix{x}{y}{(\outputp{x}{y} | @{y})) | P}}\rangle 
	      | \prefix{x}{y}{(\outputp{x}{y} | @{y})} & \nonumber\\
	\red
	& (\outputp{x}{y} | @{y})\substn{\quotep{(\prefix{x}{y}{(@{y} | \outputp{x}{y})) | P}}}{y} & \nonumber\\
	=
	& \outputp{x}{\quotep{(\prefix{x}{y}{(\outputp{x}{y} | @{y})) | P}}}
	  | {(\prefix{x}{y}{(\outputp{x}{y} | @{y})) | P}} & \nonumber\\
	\red
	& \ldots & \nonumber\\
	\red^*
	& P | P | \ldots & \nonumber
\end{eqnarray}

Of course, this encoding, as an implementation, runs away, unfolding
$\bangp{P}$ eagerly. A lazier and more implementable replication
operator, restricted to input-guarded processes, may be obtained as follows.

\begin{eqnarray}
\bangp{\prefix{u}{v}{P}} 
	:= 
	\binpar{\lift{x}{\prefix{u}{v}{(\binpar{D(x)}{P})}}}{D(x)} \nonumber
\end{eqnarray}

\begin{remark}
  Note that the lazier definition still does not deal with summation
  or mixed summation (i.e. sums over input and output). The reader is
  invited to construct definitions of replication that deal with these
  features. 

  Further, the definitions are parameterized in a name, $x$. Can you,
  gentle reader, make a definition that eliminates this parameter and
  guarantees no accidental interaction between the replication
  machinery and the process being replicated -- i.e. no accidental
  sharing of names used by the process to get its work done and the
  name(s) used by the replication to effect copying. This latter
  revision of the definition of replication is crucial to obtaining
  the expected identity $!!P \sim !P$.
\end{remark}

\begin{remark}\label{rem:paradoxical_combinator}
  The reader familiar with the lambda calculus will have noticed the
  similarity between $D$ and the paradoxical combinator.

  [Ed. note: the existence of this seems to suggest we have to be more
  restrictive on the set of processes and names we admit if we are to
  support no-cloning.]
\end{remark}

\subsubsection{Bisimulation}

The computational dynamics gives rise to another kind of equivalence,
the equivalence of computational behavior. As previously mentioned
this is typically captured \emph{via} some form of bisimulation.

% The notion we use in this paper is weak barbed bisimulation
% \cite{milner91polyadicpi}.

The notion we use in this paper is derived from weak barbed
bisimulation \cite{milner91polyadicpi}. 

\begin{definition}
An \emph{observation relation}, $\downarrow_{\mathcal N}$, over a set
of names, $\mathcal N$, is the smallest relation satisfying the rules
below.

\infrule[Out-barb]{y \in {\mathcal N}, \; x \nameeq y}
		  {\outputp{x}{v} \downarrow_{\mathcal N} x}
\infrule[Par-barb]{\mbox{$P\downarrow_{\mathcal N} x$ or $Q\downarrow_{\mathcal N} x$}}
		  {\binpar{P}{Q} \downarrow_{\mathcal N} x}

We write $P \Downarrow_{\mathcal N} x$ if there is $Q$ such that 
$P \wred Q$ and $Q \downarrow_{\mathcal N} x$.
\end{definition}

\begin{definition}
%\label{def.bbisim}
An  ${\mathcal N}$-\emph{barbed bisimulation} over a set of names, ${\mathcal N}$, is a symmetric binary relation 
${\mathcal S}_{\mathcal N}$ between agents such that $P\rel{S}_{\mathcal N}Q$ implies:
\begin{enumerate}
\item If $P \red P'$ then $Q \wred Q'$ and $P'\rel{S}_{\mathcal N} Q'$.
\item If $P\downarrow_{\mathcal N} x$, then $Q\Downarrow_{\mathcal N} x$.
\end{enumerate}
$P$ is ${\mathcal N}$-barbed bisimilar to $Q$, written
$P \wbbisim_{\mathcal N} Q$, if $P \rel{S}_{\mathcal N} Q$ for some ${\mathcal N}$-barbed bisimulation ${\mathcal S}_{\mathcal N}$.
\end{definition}

$\mathcal{R} \subseteq \pi \times \pi$

$P \mathcal{R} Q => \forall P'. P \red P' \Rightarrow \exists Q'. Q \red Q', P' \mathcal{R} Q'$

$P \vdash x \Rightarrow Q \vdash x$

\begin{mathpar}
  \inferrule*[lab=Out-barb]{x \nameeq y}{{y}!\langle{Q}\rangle \vdash x}
  \and
  \inferrule*[lab=Par-barb]{\mbox{$P\vdash x$ or $Q\vdash x$}}{\binpar{P}{Q} \vdash x}
\end{mathpar}

\subsubsection{Contexts}

One of the principle advantages of computational calculi like the
$\pi$-calculus is a well-defined notion of context,
contextual-equivalence and a correlation between
contextual-equivalence and notions of bisimulation. The notion of
context allows the decomposition of a process into (sub-)process and
its syntactic environment, its context. Thus, a context may be
thought of as a process with a ``hole'' (written $\Box$) in it. The
application of a context $M$ to a process $P$, written $M[P]$, is
tantamount to filling the hole in $M$ with $P$. In this paper we do
not need the full weight of this theory, but do make use of the notion
of context in the proof the main theorem. 

\begin{mathpar}
  \inferrule* [lab=summation] {} {{M_{M},M_{N}} \bc \Box \;|\; x.M_{A} \;|\; M_{M}+M_{N}}
  \and
  \inferrule* [lab=agent] {} {{M_{A}} \bc (\vec{x})M_{P} \;| \; \clift{P_0,\ldots,M_{P},\ldots,P_N}}
  \and \\
  \inferrule* [lab=process] {} {{M_{P}} \bc M_{N} \;| \;P|M_{P} }
\end{mathpar} 

\begin{mathpar}
  \inferrule* [lab=sychronization] {} {M_{N} \bc \Box \;|\; x?M_{F} \;|\; x!M_{C}}
  \and
  \inferrule* [lab=abstraction] {} {{M_{F}} \bc (x)M_{P} }
  \and
  \inferrule* [lab=concretion] {} {{M_{C}} \bc \langle M_{P} \rangle }
  \and \\
  \inferrule* [lab=process] {} {{M_{P}} \bc M_{N} \;| \;P|M_{P} }
\end{mathpar}

\begin{definition}[contextual application] Given a context $M$, and
  process $P$, we define the \emph{contextual application}, $M[P] :=
  M\{P/\Box\}$. That is, the contextual application of M to P is the
  substitution of $P$ for $\Box$ in $M$.
\end{definition}

$\meaningof{-} : L \to \mathcal{P}(\pi)$

\begin{mathpar}
  \inferrule* [lab=collection] {} {\meaningof{true} = \pi, \and \meaningof{~E} = \pi \setminus \meaningof{E}, \and \meaningof{E_{1} \& E_{2}} = \meaningof{E_{1}} \cap \meaningof{E_{2}}}
\end{mathpar}

\begin{mathpar}
  \inferrule* [lab=structure] {} {\meaningof{0} = \{ P \in \pi | P \equiv 0 \}, \and \\ \meaningof{E_1 | E_2} = \{ P \in \pi | P \equiv P_{1} | P_{2}, P_{1} \in \meaningof{E_{1}}, P_{2} \in \meaningof{E_2}\} }
\end{mathpar}

\begin{mathpar}
 \inferrule* [lab=behavior] {} {\meaningof{\langle a?b \rangle E} = \{ P \in \pi | P \equiv Q | u?(y)P', \\ \and \\\\ \and \\ \;\;\; u \in \meaningof{a}, \forall z.P'\{z/y\} \in \meaningof{E\{z/b\}}\}, \and \\ \meaningof{a!E} = \{ P \in \pi | P \equiv Q | x!\langle P' \rangle, x \in \meaningof{a} P' \in \meaningof{E}\} }
\end{mathpar}

\begin{mathpar}
 \inferrule* [lab=nominal] {} {\meaningof{\quotep{E}} = \{ \quotep{P} \in \quotep{\pi} | P \in \meaningof{E} \}, \and \meaningof{\quotep{P}} = \{ \quotep{Q} \in \quotep{\pi} | P \equiv Q \} \and \\ \meaningof{@\quotep{E}} = \{ P \in \pi | P \equiv @x, x \in \meaningof{E} \}}
\end{mathpar}

\begin{eqnarray*}
  \\
  \meaningof{-} : TS \to ST
\end{eqnarray*}

\begin{eqnarray*}
  \\
  L : TS \to ST
\end{eqnarray*}

\begin{eqnarray*}
  \\
  P \models E \iff P \in \meaningof{E}
\end{eqnarray*}

\begin{eqnarray*}
  P \approx_{L} Q \iff \forall E \in L. P \models E \iff Q \models E
\end{eqnarray*}

\begin{eqnarray*}
  P \approx_{K} Q
\end{eqnarray*}

\begin{eqnarray*}
  P \approx Q
\end{eqnarray*}

$\approx_{K} = \approx = \approx_{L}$

\subsubsection{Contextual duality}

Note that contexts extend the quotation operation to a family of
operations from processes to names. Given a context, $M$, we can
define a \emph{nominal context}, $\quotep{M}$ by $\quotep{M}[P] :=
\quotep{M[P]}$. To foreshadow what is to come we observe that these
operations enjoy a duality with processes very much like the duality
between vectors and maps from vectors to scalars.

Further, because the calculus is essentially higher-order, we have a
correspondence between contexts and processes. More specifically,
given a name $x$ and a context $M$ we can construct $M^{*}_{x}$ such
that 

\begin{mathpar}
  M^{*}_{x} | \lift{x}{P} \red M[P]
\end{mathpar}

namely,

\begin{mathpar}
  M^{*}_{x} := x?(u).M[\dropn{u}]
\end{mathpar}

The dependence of $M^{*}_{x}$ on a name makes it an abstraction, 

\begin{mathpar}
  M^{*} := (x)x?(u).M[\dropn{u}]
\end{mathpar}

\subsection{Additional notation}

It will sometimes be convenient to denote the process a name
quotes. We already have the notation $x = \quotep{P}$, but it will be
convenient to introduce an alternate notation, $\procn{x}$, when we
want to emphasize the connection to the use of the name. Note that, by
virtue of name equivalence, $\quotep{\procn{x}} \nameeq x$; so, the
notation is consistent with previous definitions.

Further, because names have structure it is possible to effect
substitutions on the basis of that structure. This means we need to
upgrade our notation for substitutions, which we accomplish by
adapting comprehension notation. Thus,

\begin{mathpar}
  P\{ y / x : x \in S \}
\end{mathpar}

is interpreted to mean the process derived from P by replacing (in a
capture-avoiding manner) each occurrence of $x$ in $S$ by $y$. For example,

\begin{mathpar}
  P\{ \quotep{\procn{x}|\procn{x}} / x : x \in \freenames{P} \}
\end{mathpar}

will replace each (occurrence) of a free name $x$ in $P$ by
$\quotep{\procn{x}|\procn{x}}$.

Also, we will avail ourselves of the notation $x^{L}$ and $x^{R}$ to
denote injections of a name into disjoint copies of the name
space. There are numerous ways to accomplish this. One example can be
found in \cite{MeredithR05}. This notation overloads to vectors of
names: $\vec{x}^{\pi} := (x_{i}^{\pi} \; : \; 0 \leq i < |\vec{x}| )$ where $\pi \in \{L,R\}$.

We also use $P^{\Box} := P|\Box$.

In \cite{MeredithR05} an interpretation of the new operator is
given. It turns out that there are several possible interpretations
all enjoying the requisite algebraic properties of the operator (see
\cite{milner91polyadicpi}). We will therefore make liberal use of
$(\nu\; \vec{x})P$.

% subsection the_syntax_and_semantics_of_the_notation_system (end)   

\input{qm2pi.qmops} 

\input{qm2pi.sterngerlach} 

\input{qm2pi.metric} 

% section concurrent_process_calculi (end)

%\input{qm2pi.proofsketch}

% section proof sketch (end)

%\input{qm2pi.slviaknots} 

% section spatial logic via knots (end)

\input{qm2pi.conclusion}

% section conclusion (end)

%\input{qm2pi.dtcodes} 

% section wiring algorithm (end)

\input{qm2pi.ack} 

% section acknowledgments (end)

\newpage


\bibliographystyle{plain}   
\bibliography{../../biblios/main.bib}

\input{qm2pi.rhodetails}

\end{document}

 

% section wiring algorithm (end)

\documentclass[12pt]{llncs}
%\documentclass{jktr}

\usepackage[pdftex]{hyperref}                   
\usepackage {listings}
\usepackage {mathpartir}
\usepackage{bcprules}
%\usepackage{listings}
                       
\usepackage{graphicx} 
%\usepackage[margins=2.5cm,nohead,nofoot]{geometry}
%\usepackage{geometry}
\usepackage{amsfonts}
\usepackage{amstext}
\usepackage{latexsym}
\usepackage{amssymb}
\usepackage{color}


%\include{myPreamble}
\include{qm2pi.local} 

%\ifpdf
%\usepackage[pdftex]{graphicx}
%\else
%\usepackage{graphicx}
%\fi

 % \ifpdf
%  \usepackage{pdfsync}
%  \if


%\title{Brief Article}
%\author{David F. Snyder}
%\author{L.G. Meredith}

%\address{Dept. of Math., Texas State University--San Marcos, San Marcos, TX 78666}
       
\pagestyle{empty}


\begin{document}

\lstset{language=[Objective]Caml,frame=shadowbox}

\input{qm2pi.front}

% section front matter (end)

\input{qm2pi.intro} 
 
% section introduction (end)

% \input{qm2pi.knotations} 

% section notation (end)

\input{qm2pi.process.calculi} 

% section concurrent_process_calculi_and_spatial_logics_ (end)
    
%\input{qm2pi.knots2pi} 

%\input{qm2pi.trefoil} 

%\input{qm2pi.mainthm} 

% subsection basic_interpretation (end)

%\input{qm2pi.rho.presentation} 
\subsection{The syntax and semantics of the notation system}\label{sub:the_syntax_and_semantics_of_the_notation_system} % (fold)

We now summarize a technical presentation of the calculus that
embodies our theory of dynamics. The typical presentation of such a
calculus follows the style of giving generators and relations on
them. The grammar, below, describing term constructors, freely
generates the set of processes, $\Proc$. This set is then quotiented
by a relation known as structural congruence and it is over this set
that the notion of dynamics is expressed. This presentation is
essentially that of \cite{MeredithR05} with the addition of
polyadicity and summation. For readability we have relegated some of
the technical subtleties to an appendix.

\subsubsection{Process grammar}\label{subsub:process_grammar}

\begin{mathpar}
  \inferrule* [lab=synchronization] {} {{M} \bc \pzero \;|\; x?F \;|\; x!C }
  \and
  \inferrule* [lab=abstraction] {} {{F} \bc (x)P}
  \and
  \inferrule* [lab=concretion] {} {{C} \bc \langle Q \rangle}
  \and
  \inferrule* [lab=process] {} {{P,Q} \bc M \;| \;P|Q \;|\; @{x}}
  \and
  \inferrule* [lab=name] {} {{x} \bc \quotep{P}}
\end{mathpar} 

Note that $\vec{x}$ (resp. $\vec{P}$) denotes a vector of names
(resp. processes) of length $|\vec{x}|$ (resp. $|\vec{P}|$). We adopt
the following useful abbreviations.

\begin{mathpar}
   x?(\vec{y}).P := x.(\vec{y})P \and  x\clift{\vec{P}} := x.\clift{\vec{P}}
   \and x!(y) := \lift{x}{\dropn{y}}
   \and \Pi_{i=0}^{n-1}P_i := P_0 | \ldots | P_{n-1}
\end{mathpar}

\subsubsection{Structural congruence}

\paragraph{Free and bound names and alpha-equivalence.} At the
core of structural equivalence is alpha-equivalence which identifies
process that are the same up to a change of variable. Formally, we
recognize the distinction between free and bound names. The free names
of a process, $\freenames{P}$, may be calculated recursively as
follows:

\begin{mathpar}
\freenames{\pzero} := \emptyset
  \and \\
  \freenames{x?(y).P} := \{ x \} \cup (\freenames{P} \setminus \{ y \})
  \and 
  \freenames{x!\langle P \rangle} := \{ x \} \cup \{ P \} 
  \and \\
  \freenames{P|Q} := \freenames{P} \cup \freenames{Q}
  \and \\
  \freenames{@{x}} := \{ x \}
\end{mathpar}

$\pi$
$\quotep{\pi}$

$\freenames{-} : \pi \to \mathcal{P}(\quotep{\pi})$

\begin{eqnarray*}
  \freenames{\pzero} & := & \emptyset \\
  \freenames{x?(y).P} & := & \{ x \} \cup (\freenames{P} \setminus \{ y \}) \\
  \freenames{x!\langle P \rangle} & := & \{ x \} \cup \{ P \} \\
  \freenames{P|Q} & := & \freenames{P} \cup \freenames{Q} \\
  \freenames{\dropn{x}} & := & \{ x \}
\end{eqnarray*}

The bound names of a process, $\boundnames{P}$, are those names occurring in $P$
that are not free. For example, in $x?(y).0$, the name $x$ is free, while $y$ is bound.

\begin{mathpar}
  \inferrule* [lab=monoidal-laws] {} { P|Q \equiv Q|P \and P|0 \equiv P \and P|(Q|R) \equiv (P|Q)|R }
\end{mathpar}

\begin{mathpar}
  \inferrule* [lab=alpha-equivalence] {} { (x)P \equiv (y)P\{y/x\} \and y \not\in \freenames{P} }
\end{mathpar}

\begin{definition}
Then two processes, $P,Q$, are alpha-equivalent if $P = Q\{\vec{y}/\vec{x}\}$ for
some $\vec{x} \in \boundnames{Q},\vec{y} \in \boundnames{P}$, where $Q\{\vec{y}/\vec{x}\}$
denotes the capture-avoiding substitution of $\vec{y}$ for $\vec{x}$ in $Q$.
\end{definition}

\begin{definition}
  The {\em structural congruence} \cite{SangiorgiWalker} , $\equiv$,
  between processes is the least congruence containing
  alpha-equivalence, satisfying the abelian monoid laws
  (associativity, commutativity and $\pzero$ as identity) for parallel
  composition $|$ and for summation $+$.
\end{definition}

\subsection{Name equivalence}

We take name equivalence, written $\nameeq$, to be the smallest
equivalence relation generated by the following rules.

\begin{mathpar}
\inferrule*[lab=Quote-drop]
{ }
{ \quotep{@{x}} \nameeq x }

\inferrule*[lab=Struct-equiv]
{ P \scong Q }
{ \quotep{P} \nameeq \quotep{Q} }
\end{mathpar}

The astute reader will have noticed that the mutual recursion of names
and processes imposes a mutual recursion on alpha-equivalence and
structural equivalence via name-equivalence. Fortunately, all of this
works out pleasantly and we may calculate in the natural way, free of
concern. The reader interested in the details is referred to the
appendix \ref{appendix:rho_details}.

\subsection{Substitution}

We use $\Proc$ for the set of processes, $\QProc$ for the set of
names, and $\id{\{}\vec{y} / \vec{x} \id{\}}$ to denote partial maps,
$s : \QProc \rightarrow \QProc$. A map, $s$ lifts, uniquely, to a map
on process terms, $\widehat{s} : \Proc \rightarrow \Proc$ by the
following equations.

\begin{mathpar}
  (0) \psubstp{Q}{P} := 0 \\
  (R \juxtap S) \psubstp{Q}{P}
  :=    
  (R)\psubstp{Q}{P} \juxtap (S) \psubstp{Q}{P} \\
  (x?(y).R) \psubstp{Q}{P}    
  :=    
  (x)\substp{Q}{P} (z)\concat( (R \psubstn{z}{y}) \psubstp{Q}{P} ) \\
  (\lift{x}{R}) \psubstp{Q}{P}  
  :=
  \lift{(x)\substp{Q}{P}}{ R \psubstp{Q}{P} } \\
%   (\dropn{x})  \psubstp{Q}{P}       
%   := 
%   \left\{ 
%     \begin{array}{ccc} 
%       \dropn{\quotep{Q}} & & x \nameeq \quotep{P} \\
%       \dropn{x} & & otherwise \\
%     \end{array}
%   \right. 
  (\dropn{x})  \psubstp{Q}{P}       
  := 
  \left\{ 
    \begin{array}{ccc} 
      Q & & x \nameeq \quotep{P} \\
      \dropn{x} & & otherwise \\
    \end{array}
  \right.
\end{mathpar}
 

where

\begin{eqnarray}
  (x)\id{\{} \lpquote Q \rpquote / \lpquote P \rpquote \id{\}}            = 
  \left\{ 
    \begin{array}{ccc}
      \lpquote Q \rpquote & & x \nameeq \lpquote P \rpquote \\
      x & & otherwise \\
    \end{array}
  \right. \nonumber
\end{eqnarray}

and $z$ is chosen distinct from $\quotep{P}$, $\quotep{Q}$, the free
names in $Q$, and all the names in $R$. Our $\alpha$-equivalence will
be built in the standard way from this substitution.

\begin{remark}\label{rem:no_self_referential_names}
  One consequence of these definitions is that $\forall P. \quotep{P}
  \not\in \freenames{P}$.
\end{remark}

\subsection{ Dynamic quote: an example }

Anticipating something of what's to come, consider applying the
substitution, $\widehat{\id{\{}u / z \id{\}}}$, to the following pair
of processes, $\lift{w}{y!(z)}$ and $w[ \lpquote y!(z) \rpquote ]$.

\begin{eqnarray}
	\lift{w}{y!(z)}\widehat{\id{\{}u / z \id{\}}}
		& = &
		\lift{w}{y!(u)} \nonumber\\
	w[ \lpquote y!(z) \rpquote ] \widehat{ \id{\{}u / z \id{\}} }
		& = &
		w[ \lpquote y!(z) \rpquote ] \nonumber
\end{eqnarray}

Because the body of the process between quotes is impervious to
substitution, we get radically different answers. In fact, by
examining the first process in an input context,
e.g. $x?(z).\lift{w}{y!(z)}$, we see that the process under the lift
operator may be shaped by prefixed inputs binding a name inside it. In
this sense, the lift operator will be seen as a way to dynamically
construct processes before reifying them as names.

Finally equipped with these standard features we can present the
dynamics of the calculus.

\subsubsection{Operational semantics} 

Finally, we introduce the computational dynamics. What marks these
algebras as distinct from other more traditionally studied algebraic
structures, e.g. vector spaces or polynomial rings, is the manner in
which dynamics is captured. In traditional structures, dynamics is typically
expressed through morphisms between such structures, as in linear maps
between vector spaces or morphisms between rings. In algebras
associated with the semantics of computation, the dynamics is
expressed as part of the algebraic structure itself, through a
reduction reduction relation typically denoted by $\red$. Below, we
give a recursive presentation of this relation for the calculus used
in the encoding.

$\red \subseteq \pi \times \pi$
$\red : \pi \to \mathcal{P}(\pi)$

\begin{mathpar}
  \inferrule* [lab=Comm] { \textsf{match}( x_{src}, x_{trgt} ) } { x_{trgt}?(y)P \; | \; x_{src}!\langle {Q} \rangle \red P\{\quotep{Q}/y}\} }
  \and \\
  \inferrule* [lab=Par] {{P} \red {P}'} {{{P} | {Q}} \red {{P}' | {Q}}}
  \and
  \inferrule* [lab=Equiv]{{{P} \scong {P}'} \andalso {{P}' \red {Q}'} \andalso {{Q}' \scong {Q}}}{{P} \red {Q}}
\end{mathpar}

\begin{eqnarray*}
  match_{\equiv} (\quotep{P},\quotep{Q}) & := & P \equiv Q \\
  match_{\dagger}(\quotep{P},\quotep{Q}) & := & \forall R. P|Q \red^{*} R => R \red^{*} 0 \\
  match_{K}(\quotep{P},\quotep{Q}) & := & K \mbox{ for some context } K
\end{eqnarray*}

$u?(x)P | u!\langle Q \rangle \red P\{\quotep{Q}/x\}$

%We write $\wred$ for $\red^*$, and $P\red$ if $\exists Q $ such that $ P \red Q$.
We write $P\red$ if $\exists Q $ such that $ P \red Q$ and $P\not\red$, otherwise.

\section{Replication}

As mentioned before, it is known that replication (and hence
recursion) can be implemented in a higher-order process algebra
\cite{SangiorgiWalker}. As our first example of calculation with the
machinery thus far presented we give the construction explicitly in
the {\rhoc}.

\begin{eqnarray}
	D_{x} & := & \prefix{x}{y}{(\binpar{\outputp{x}{y}}{@{y}})} \nonumber\\
	\bangp_{x}{P} & := & \binpar{{x}!\langle{\binpar{D_{x}}{P}}\rangle}{D_{x}} \nonumber
\end{eqnarray}

\begin{eqnarray}
	\bangp_{x}{P} & & \nonumber\\
	=
	& {x}!\langle{(\prefix{x}{y}{(\outputp{x}{y} | @{y})) | P}}\rangle 
	      | \prefix{x}{y}{(\outputp{x}{y} | @{y})} & \nonumber\\
	\red
	& (\outputp{x}{y} | @{y})\substn{\quotep{(\prefix{x}{y}{(@{y} | \outputp{x}{y})) | P}}}{y} & \nonumber\\
	=
	& \outputp{x}{\quotep{(\prefix{x}{y}{(\outputp{x}{y} | @{y})) | P}}}
	  | {(\prefix{x}{y}{(\outputp{x}{y} | @{y})) | P}} & \nonumber\\
	\red
	& \ldots & \nonumber\\
	\red^*
	& P | P | \ldots & \nonumber
\end{eqnarray}

Of course, this encoding, as an implementation, runs away, unfolding
$\bangp{P}$ eagerly. A lazier and more implementable replication
operator, restricted to input-guarded processes, may be obtained as follows.

\begin{eqnarray}
\bangp{\prefix{u}{v}{P}} 
	:= 
	\binpar{\lift{x}{\prefix{u}{v}{(\binpar{D(x)}{P})}}}{D(x)} \nonumber
\end{eqnarray}

\begin{remark}
  Note that the lazier definition still does not deal with summation
  or mixed summation (i.e. sums over input and output). The reader is
  invited to construct definitions of replication that deal with these
  features. 

  Further, the definitions are parameterized in a name, $x$. Can you,
  gentle reader, make a definition that eliminates this parameter and
  guarantees no accidental interaction between the replication
  machinery and the process being replicated -- i.e. no accidental
  sharing of names used by the process to get its work done and the
  name(s) used by the replication to effect copying. This latter
  revision of the definition of replication is crucial to obtaining
  the expected identity $!!P \sim !P$.
\end{remark}

\begin{remark}\label{rem:paradoxical_combinator}
  The reader familiar with the lambda calculus will have noticed the
  similarity between $D$ and the paradoxical combinator.

  [Ed. note: the existence of this seems to suggest we have to be more
  restrictive on the set of processes and names we admit if we are to
  support no-cloning.]
\end{remark}

\subsubsection{Bisimulation}

The computational dynamics gives rise to another kind of equivalence,
the equivalence of computational behavior. As previously mentioned
this is typically captured \emph{via} some form of bisimulation.

% The notion we use in this paper is weak barbed bisimulation
% \cite{milner91polyadicpi}.

The notion we use in this paper is derived from weak barbed
bisimulation \cite{milner91polyadicpi}. 

\begin{definition}
An \emph{observation relation}, $\downarrow_{\mathcal N}$, over a set
of names, $\mathcal N$, is the smallest relation satisfying the rules
below.

\infrule[Out-barb]{y \in {\mathcal N}, \; x \nameeq y}
		  {\outputp{x}{v} \downarrow_{\mathcal N} x}
\infrule[Par-barb]{\mbox{$P\downarrow_{\mathcal N} x$ or $Q\downarrow_{\mathcal N} x$}}
		  {\binpar{P}{Q} \downarrow_{\mathcal N} x}

We write $P \Downarrow_{\mathcal N} x$ if there is $Q$ such that 
$P \wred Q$ and $Q \downarrow_{\mathcal N} x$.
\end{definition}

\begin{definition}
%\label{def.bbisim}
An  ${\mathcal N}$-\emph{barbed bisimulation} over a set of names, ${\mathcal N}$, is a symmetric binary relation 
${\mathcal S}_{\mathcal N}$ between agents such that $P\rel{S}_{\mathcal N}Q$ implies:
\begin{enumerate}
\item If $P \red P'$ then $Q \wred Q'$ and $P'\rel{S}_{\mathcal N} Q'$.
\item If $P\downarrow_{\mathcal N} x$, then $Q\Downarrow_{\mathcal N} x$.
\end{enumerate}
$P$ is ${\mathcal N}$-barbed bisimilar to $Q$, written
$P \wbbisim_{\mathcal N} Q$, if $P \rel{S}_{\mathcal N} Q$ for some ${\mathcal N}$-barbed bisimulation ${\mathcal S}_{\mathcal N}$.
\end{definition}

$\mathcal{R} \subseteq \pi \times \pi$

$P \mathcal{R} Q => \forall P'. P \red P' \Rightarrow \exists Q'. Q \red Q', P' \mathcal{R} Q'$

$P \vdash x \Rightarrow Q \vdash x$

\begin{mathpar}
  \inferrule*[lab=Out-barb]{x \nameeq y}{{y}!\langle{Q}\rangle \vdash x}
  \and
  \inferrule*[lab=Par-barb]{\mbox{$P\vdash x$ or $Q\vdash x$}}{\binpar{P}{Q} \vdash x}
\end{mathpar}

\subsubsection{Contexts}

One of the principle advantages of computational calculi like the
$\pi$-calculus is a well-defined notion of context,
contextual-equivalence and a correlation between
contextual-equivalence and notions of bisimulation. The notion of
context allows the decomposition of a process into (sub-)process and
its syntactic environment, its context. Thus, a context may be
thought of as a process with a ``hole'' (written $\Box$) in it. The
application of a context $M$ to a process $P$, written $M[P]$, is
tantamount to filling the hole in $M$ with $P$. In this paper we do
not need the full weight of this theory, but do make use of the notion
of context in the proof the main theorem. 

\begin{mathpar}
  \inferrule* [lab=summation] {} {{M_{M},M_{N}} \bc \Box \;|\; x.M_{A} \;|\; M_{M}+M_{N}}
  \and
  \inferrule* [lab=agent] {} {{M_{A}} \bc (\vec{x})M_{P} \;| \; \clift{P_0,\ldots,M_{P},\ldots,P_N}}
  \and \\
  \inferrule* [lab=process] {} {{M_{P}} \bc M_{N} \;| \;P|M_{P} }
\end{mathpar} 

\begin{mathpar}
  \inferrule* [lab=sychronization] {} {M_{N} \bc \Box \;|\; x?M_{F} \;|\; x!M_{C}}
  \and
  \inferrule* [lab=abstraction] {} {{M_{F}} \bc (x)M_{P} }
  \and
  \inferrule* [lab=concretion] {} {{M_{C}} \bc \langle M_{P} \rangle }
  \and \\
  \inferrule* [lab=process] {} {{M_{P}} \bc M_{N} \;| \;P|M_{P} }
\end{mathpar}

\begin{definition}[contextual application] Given a context $M$, and
  process $P$, we define the \emph{contextual application}, $M[P] :=
  M\{P/\Box\}$. That is, the contextual application of M to P is the
  substitution of $P$ for $\Box$ in $M$.
\end{definition}

$\meaningof{-} : L \to \mathcal{P}(\pi)$

\begin{mathpar}
  \inferrule* [lab=collection] {} {\meaningof{true} = \pi, \and \meaningof{~E} = \pi \setminus \meaningof{E}, \and \meaningof{E_{1} \& E_{2}} = \meaningof{E_{1}} \cap \meaningof{E_{2}}}
\end{mathpar}

\begin{mathpar}
  \inferrule* [lab=structure] {} {\meaningof{0} = \{ P \in \pi | P \equiv 0 \}, \and \\ \meaningof{E_1 | E_2} = \{ P \in \pi | P \equiv P_{1} | P_{2}, P_{1} \in \meaningof{E_{1}}, P_{2} \in \meaningof{E_2}\} }
\end{mathpar}

\begin{mathpar}
 \inferrule* [lab=behavior] {} {\meaningof{\langle a?b \rangle E} = \{ P \in \pi | P \equiv Q | u?(y)P', \\ \and \\\\ \and \\ \;\;\; u \in \meaningof{a}, \forall z.P'\{z/y\} \in \meaningof{E\{z/b\}}\}, \and \\ \meaningof{a!E} = \{ P \in \pi | P \equiv Q | x!\langle P' \rangle, x \in \meaningof{a} P' \in \meaningof{E}\} }
\end{mathpar}

\begin{mathpar}
 \inferrule* [lab=nominal] {} {\meaningof{\quotep{E}} = \{ \quotep{P} \in \quotep{\pi} | P \in \meaningof{E} \}, \and \meaningof{\quotep{P}} = \{ \quotep{Q} \in \quotep{\pi} | P \equiv Q \} \and \\ \meaningof{@\quotep{E}} = \{ P \in \pi | P \equiv @x, x \in \meaningof{E} \}}
\end{mathpar}

\begin{eqnarray*}
  \\
  \meaningof{-} : TS \to ST
\end{eqnarray*}

\begin{eqnarray*}
  \\
  L : TS \to ST
\end{eqnarray*}

\begin{eqnarray*}
  \\
  P \models E \iff P \in \meaningof{E}
\end{eqnarray*}

\begin{eqnarray*}
  P \approx_{L} Q \iff \forall E \in L. P \models E \iff Q \models E
\end{eqnarray*}

\begin{eqnarray*}
  P \approx_{K} Q
\end{eqnarray*}

\begin{eqnarray*}
  P \approx Q
\end{eqnarray*}

$\approx_{K} = \approx = \approx_{L}$

\subsubsection{Contextual duality}

Note that contexts extend the quotation operation to a family of
operations from processes to names. Given a context, $M$, we can
define a \emph{nominal context}, $\quotep{M}$ by $\quotep{M}[P] :=
\quotep{M[P]}$. To foreshadow what is to come we observe that these
operations enjoy a duality with processes very much like the duality
between vectors and maps from vectors to scalars.

Further, because the calculus is essentially higher-order, we have a
correspondence between contexts and processes. More specifically,
given a name $x$ and a context $M$ we can construct $M^{*}_{x}$ such
that 

\begin{mathpar}
  M^{*}_{x} | \lift{x}{P} \red M[P]
\end{mathpar}

namely,

\begin{mathpar}
  M^{*}_{x} := x?(u).M[\dropn{u}]
\end{mathpar}

The dependence of $M^{*}_{x}$ on a name makes it an abstraction, 

\begin{mathpar}
  M^{*} := (x)x?(u).M[\dropn{u}]
\end{mathpar}

\subsection{Additional notation}

It will sometimes be convenient to denote the process a name
quotes. We already have the notation $x = \quotep{P}$, but it will be
convenient to introduce an alternate notation, $\procn{x}$, when we
want to emphasize the connection to the use of the name. Note that, by
virtue of name equivalence, $\quotep{\procn{x}} \nameeq x$; so, the
notation is consistent with previous definitions.

Further, because names have structure it is possible to effect
substitutions on the basis of that structure. This means we need to
upgrade our notation for substitutions, which we accomplish by
adapting comprehension notation. Thus,

\begin{mathpar}
  P\{ y / x : x \in S \}
\end{mathpar}

is interpreted to mean the process derived from P by replacing (in a
capture-avoiding manner) each occurrence of $x$ in $S$ by $y$. For example,

\begin{mathpar}
  P\{ \quotep{\procn{x}|\procn{x}} / x : x \in \freenames{P} \}
\end{mathpar}

will replace each (occurrence) of a free name $x$ in $P$ by
$\quotep{\procn{x}|\procn{x}}$.

Also, we will avail ourselves of the notation $x^{L}$ and $x^{R}$ to
denote injections of a name into disjoint copies of the name
space. There are numerous ways to accomplish this. One example can be
found in \cite{MeredithR05}. This notation overloads to vectors of
names: $\vec{x}^{\pi} := (x_{i}^{\pi} \; : \; 0 \leq i < |\vec{x}| )$ where $\pi \in \{L,R\}$.

We also use $P^{\Box} := P|\Box$.

In \cite{MeredithR05} an interpretation of the new operator is
given. It turns out that there are several possible interpretations
all enjoying the requisite algebraic properties of the operator (see
\cite{milner91polyadicpi}). We will therefore make liberal use of
$(\nu\; \vec{x})P$.

% subsection the_syntax_and_semantics_of_the_notation_system (end)   

\input{qm2pi.qmops} 

\input{qm2pi.sterngerlach} 

\input{qm2pi.metric} 

% section concurrent_process_calculi (end)

%\input{qm2pi.proofsketch}

% section proof sketch (end)

%\input{qm2pi.slviaknots} 

% section spatial logic via knots (end)

\input{qm2pi.conclusion}

% section conclusion (end)

%\input{qm2pi.dtcodes} 

% section wiring algorithm (end)

\input{qm2pi.ack} 

% section acknowledgments (end)

\newpage


\bibliographystyle{plain}   
\bibliography{../../biblios/main.bib}

\input{qm2pi.rhodetails}

\end{document}

 

% section acknowledgments (end)

\newpage


\bibliographystyle{plain}   
\bibliography{../../biblios/main.bib}

\documentclass[12pt]{llncs}
%\documentclass{jktr}

\usepackage[pdftex]{hyperref}                   
\usepackage {listings}
\usepackage {mathpartir}
\usepackage{bcprules}
%\usepackage{listings}
                       
\usepackage{graphicx} 
%\usepackage[margins=2.5cm,nohead,nofoot]{geometry}
%\usepackage{geometry}
\usepackage{amsfonts}
\usepackage{amstext}
\usepackage{latexsym}
\usepackage{amssymb}
\usepackage{color}


%\include{myPreamble}
\include{qm2pi.local} 

%\ifpdf
%\usepackage[pdftex]{graphicx}
%\else
%\usepackage{graphicx}
%\fi

 % \ifpdf
%  \usepackage{pdfsync}
%  \if


%\title{Brief Article}
%\author{David F. Snyder}
%\author{L.G. Meredith}

%\address{Dept. of Math., Texas State University--San Marcos, San Marcos, TX 78666}
       
\pagestyle{empty}


\begin{document}

\lstset{language=[Objective]Caml,frame=shadowbox}

\input{qm2pi.front}

% section front matter (end)

\input{qm2pi.intro} 
 
% section introduction (end)

% \input{qm2pi.knotations} 

% section notation (end)

\input{qm2pi.process.calculi} 

% section concurrent_process_calculi_and_spatial_logics_ (end)
    
%\input{qm2pi.knots2pi} 

%\input{qm2pi.trefoil} 

%\input{qm2pi.mainthm} 

% subsection basic_interpretation (end)

%\input{qm2pi.rho.presentation} 
\subsection{The syntax and semantics of the notation system}\label{sub:the_syntax_and_semantics_of_the_notation_system} % (fold)

We now summarize a technical presentation of the calculus that
embodies our theory of dynamics. The typical presentation of such a
calculus follows the style of giving generators and relations on
them. The grammar, below, describing term constructors, freely
generates the set of processes, $\Proc$. This set is then quotiented
by a relation known as structural congruence and it is over this set
that the notion of dynamics is expressed. This presentation is
essentially that of \cite{MeredithR05} with the addition of
polyadicity and summation. For readability we have relegated some of
the technical subtleties to an appendix.

\subsubsection{Process grammar}\label{subsub:process_grammar}

\begin{mathpar}
  \inferrule* [lab=synchronization] {} {{M} \bc \pzero \;|\; x?F \;|\; x!C }
  \and
  \inferrule* [lab=abstraction] {} {{F} \bc (x)P}
  \and
  \inferrule* [lab=concretion] {} {{C} \bc \langle Q \rangle}
  \and
  \inferrule* [lab=process] {} {{P,Q} \bc M \;| \;P|Q \;|\; @{x}}
  \and
  \inferrule* [lab=name] {} {{x} \bc \quotep{P}}
\end{mathpar} 

Note that $\vec{x}$ (resp. $\vec{P}$) denotes a vector of names
(resp. processes) of length $|\vec{x}|$ (resp. $|\vec{P}|$). We adopt
the following useful abbreviations.

\begin{mathpar}
   x?(\vec{y}).P := x.(\vec{y})P \and  x\clift{\vec{P}} := x.\clift{\vec{P}}
   \and x!(y) := \lift{x}{\dropn{y}}
   \and \Pi_{i=0}^{n-1}P_i := P_0 | \ldots | P_{n-1}
\end{mathpar}

\subsubsection{Structural congruence}

\paragraph{Free and bound names and alpha-equivalence.} At the
core of structural equivalence is alpha-equivalence which identifies
process that are the same up to a change of variable. Formally, we
recognize the distinction between free and bound names. The free names
of a process, $\freenames{P}$, may be calculated recursively as
follows:

\begin{mathpar}
\freenames{\pzero} := \emptyset
  \and \\
  \freenames{x?(y).P} := \{ x \} \cup (\freenames{P} \setminus \{ y \})
  \and 
  \freenames{x!\langle P \rangle} := \{ x \} \cup \{ P \} 
  \and \\
  \freenames{P|Q} := \freenames{P} \cup \freenames{Q}
  \and \\
  \freenames{@{x}} := \{ x \}
\end{mathpar}

$\pi$
$\quotep{\pi}$

$\freenames{-} : \pi \to \mathcal{P}(\quotep{\pi})$

\begin{eqnarray*}
  \freenames{\pzero} & := & \emptyset \\
  \freenames{x?(y).P} & := & \{ x \} \cup (\freenames{P} \setminus \{ y \}) \\
  \freenames{x!\langle P \rangle} & := & \{ x \} \cup \{ P \} \\
  \freenames{P|Q} & := & \freenames{P} \cup \freenames{Q} \\
  \freenames{\dropn{x}} & := & \{ x \}
\end{eqnarray*}

The bound names of a process, $\boundnames{P}$, are those names occurring in $P$
that are not free. For example, in $x?(y).0$, the name $x$ is free, while $y$ is bound.

\begin{mathpar}
  \inferrule* [lab=monoidal-laws] {} { P|Q \equiv Q|P \and P|0 \equiv P \and P|(Q|R) \equiv (P|Q)|R }
\end{mathpar}

\begin{mathpar}
  \inferrule* [lab=alpha-equivalence] {} { (x)P \equiv (y)P\{y/x\} \and y \not\in \freenames{P} }
\end{mathpar}

\begin{definition}
Then two processes, $P,Q$, are alpha-equivalent if $P = Q\{\vec{y}/\vec{x}\}$ for
some $\vec{x} \in \boundnames{Q},\vec{y} \in \boundnames{P}$, where $Q\{\vec{y}/\vec{x}\}$
denotes the capture-avoiding substitution of $\vec{y}$ for $\vec{x}$ in $Q$.
\end{definition}

\begin{definition}
  The {\em structural congruence} \cite{SangiorgiWalker} , $\equiv$,
  between processes is the least congruence containing
  alpha-equivalence, satisfying the abelian monoid laws
  (associativity, commutativity and $\pzero$ as identity) for parallel
  composition $|$ and for summation $+$.
\end{definition}

\subsection{Name equivalence}

We take name equivalence, written $\nameeq$, to be the smallest
equivalence relation generated by the following rules.

\begin{mathpar}
\inferrule*[lab=Quote-drop]
{ }
{ \quotep{@{x}} \nameeq x }

\inferrule*[lab=Struct-equiv]
{ P \scong Q }
{ \quotep{P} \nameeq \quotep{Q} }
\end{mathpar}

The astute reader will have noticed that the mutual recursion of names
and processes imposes a mutual recursion on alpha-equivalence and
structural equivalence via name-equivalence. Fortunately, all of this
works out pleasantly and we may calculate in the natural way, free of
concern. The reader interested in the details is referred to the
appendix \ref{appendix:rho_details}.

\subsection{Substitution}

We use $\Proc$ for the set of processes, $\QProc$ for the set of
names, and $\id{\{}\vec{y} / \vec{x} \id{\}}$ to denote partial maps,
$s : \QProc \rightarrow \QProc$. A map, $s$ lifts, uniquely, to a map
on process terms, $\widehat{s} : \Proc \rightarrow \Proc$ by the
following equations.

\begin{mathpar}
  (0) \psubstp{Q}{P} := 0 \\
  (R \juxtap S) \psubstp{Q}{P}
  :=    
  (R)\psubstp{Q}{P} \juxtap (S) \psubstp{Q}{P} \\
  (x?(y).R) \psubstp{Q}{P}    
  :=    
  (x)\substp{Q}{P} (z)\concat( (R \psubstn{z}{y}) \psubstp{Q}{P} ) \\
  (\lift{x}{R}) \psubstp{Q}{P}  
  :=
  \lift{(x)\substp{Q}{P}}{ R \psubstp{Q}{P} } \\
%   (\dropn{x})  \psubstp{Q}{P}       
%   := 
%   \left\{ 
%     \begin{array}{ccc} 
%       \dropn{\quotep{Q}} & & x \nameeq \quotep{P} \\
%       \dropn{x} & & otherwise \\
%     \end{array}
%   \right. 
  (\dropn{x})  \psubstp{Q}{P}       
  := 
  \left\{ 
    \begin{array}{ccc} 
      Q & & x \nameeq \quotep{P} \\
      \dropn{x} & & otherwise \\
    \end{array}
  \right.
\end{mathpar}
 

where

\begin{eqnarray}
  (x)\id{\{} \lpquote Q \rpquote / \lpquote P \rpquote \id{\}}            = 
  \left\{ 
    \begin{array}{ccc}
      \lpquote Q \rpquote & & x \nameeq \lpquote P \rpquote \\
      x & & otherwise \\
    \end{array}
  \right. \nonumber
\end{eqnarray}

and $z$ is chosen distinct from $\quotep{P}$, $\quotep{Q}$, the free
names in $Q$, and all the names in $R$. Our $\alpha$-equivalence will
be built in the standard way from this substitution.

\begin{remark}\label{rem:no_self_referential_names}
  One consequence of these definitions is that $\forall P. \quotep{P}
  \not\in \freenames{P}$.
\end{remark}

\subsection{ Dynamic quote: an example }

Anticipating something of what's to come, consider applying the
substitution, $\widehat{\id{\{}u / z \id{\}}}$, to the following pair
of processes, $\lift{w}{y!(z)}$ and $w[ \lpquote y!(z) \rpquote ]$.

\begin{eqnarray}
	\lift{w}{y!(z)}\widehat{\id{\{}u / z \id{\}}}
		& = &
		\lift{w}{y!(u)} \nonumber\\
	w[ \lpquote y!(z) \rpquote ] \widehat{ \id{\{}u / z \id{\}} }
		& = &
		w[ \lpquote y!(z) \rpquote ] \nonumber
\end{eqnarray}

Because the body of the process between quotes is impervious to
substitution, we get radically different answers. In fact, by
examining the first process in an input context,
e.g. $x?(z).\lift{w}{y!(z)}$, we see that the process under the lift
operator may be shaped by prefixed inputs binding a name inside it. In
this sense, the lift operator will be seen as a way to dynamically
construct processes before reifying them as names.

Finally equipped with these standard features we can present the
dynamics of the calculus.

\subsubsection{Operational semantics} 

Finally, we introduce the computational dynamics. What marks these
algebras as distinct from other more traditionally studied algebraic
structures, e.g. vector spaces or polynomial rings, is the manner in
which dynamics is captured. In traditional structures, dynamics is typically
expressed through morphisms between such structures, as in linear maps
between vector spaces or morphisms between rings. In algebras
associated with the semantics of computation, the dynamics is
expressed as part of the algebraic structure itself, through a
reduction reduction relation typically denoted by $\red$. Below, we
give a recursive presentation of this relation for the calculus used
in the encoding.

$\red \subseteq \pi \times \pi$
$\red : \pi \to \mathcal{P}(\pi)$

\begin{mathpar}
  \inferrule* [lab=Comm] { \textsf{match}( x_{src}, x_{trgt} ) } { x_{trgt}?(y)P \; | \; x_{src}!\langle {Q} \rangle \red P\{\quotep{Q}/y}\} }
  \and \\
  \inferrule* [lab=Par] {{P} \red {P}'} {{{P} | {Q}} \red {{P}' | {Q}}}
  \and
  \inferrule* [lab=Equiv]{{{P} \scong {P}'} \andalso {{P}' \red {Q}'} \andalso {{Q}' \scong {Q}}}{{P} \red {Q}}
\end{mathpar}

\begin{eqnarray*}
  match_{\equiv} (\quotep{P},\quotep{Q}) & := & P \equiv Q \\
  match_{\dagger}(\quotep{P},\quotep{Q}) & := & \forall R. P|Q \red^{*} R => R \red^{*} 0 \\
  match_{K}(\quotep{P},\quotep{Q}) & := & K \mbox{ for some context } K
\end{eqnarray*}

$u?(x)P | u!\langle Q \rangle \red P\{\quotep{Q}/x\}$

%We write $\wred$ for $\red^*$, and $P\red$ if $\exists Q $ such that $ P \red Q$.
We write $P\red$ if $\exists Q $ such that $ P \red Q$ and $P\not\red$, otherwise.

\section{Replication}

As mentioned before, it is known that replication (and hence
recursion) can be implemented in a higher-order process algebra
\cite{SangiorgiWalker}. As our first example of calculation with the
machinery thus far presented we give the construction explicitly in
the {\rhoc}.

\begin{eqnarray}
	D_{x} & := & \prefix{x}{y}{(\binpar{\outputp{x}{y}}{@{y}})} \nonumber\\
	\bangp_{x}{P} & := & \binpar{{x}!\langle{\binpar{D_{x}}{P}}\rangle}{D_{x}} \nonumber
\end{eqnarray}

\begin{eqnarray}
	\bangp_{x}{P} & & \nonumber\\
	=
	& {x}!\langle{(\prefix{x}{y}{(\outputp{x}{y} | @{y})) | P}}\rangle 
	      | \prefix{x}{y}{(\outputp{x}{y} | @{y})} & \nonumber\\
	\red
	& (\outputp{x}{y} | @{y})\substn{\quotep{(\prefix{x}{y}{(@{y} | \outputp{x}{y})) | P}}}{y} & \nonumber\\
	=
	& \outputp{x}{\quotep{(\prefix{x}{y}{(\outputp{x}{y} | @{y})) | P}}}
	  | {(\prefix{x}{y}{(\outputp{x}{y} | @{y})) | P}} & \nonumber\\
	\red
	& \ldots & \nonumber\\
	\red^*
	& P | P | \ldots & \nonumber
\end{eqnarray}

Of course, this encoding, as an implementation, runs away, unfolding
$\bangp{P}$ eagerly. A lazier and more implementable replication
operator, restricted to input-guarded processes, may be obtained as follows.

\begin{eqnarray}
\bangp{\prefix{u}{v}{P}} 
	:= 
	\binpar{\lift{x}{\prefix{u}{v}{(\binpar{D(x)}{P})}}}{D(x)} \nonumber
\end{eqnarray}

\begin{remark}
  Note that the lazier definition still does not deal with summation
  or mixed summation (i.e. sums over input and output). The reader is
  invited to construct definitions of replication that deal with these
  features. 

  Further, the definitions are parameterized in a name, $x$. Can you,
  gentle reader, make a definition that eliminates this parameter and
  guarantees no accidental interaction between the replication
  machinery and the process being replicated -- i.e. no accidental
  sharing of names used by the process to get its work done and the
  name(s) used by the replication to effect copying. This latter
  revision of the definition of replication is crucial to obtaining
  the expected identity $!!P \sim !P$.
\end{remark}

\begin{remark}\label{rem:paradoxical_combinator}
  The reader familiar with the lambda calculus will have noticed the
  similarity between $D$ and the paradoxical combinator.

  [Ed. note: the existence of this seems to suggest we have to be more
  restrictive on the set of processes and names we admit if we are to
  support no-cloning.]
\end{remark}

\subsubsection{Bisimulation}

The computational dynamics gives rise to another kind of equivalence,
the equivalence of computational behavior. As previously mentioned
this is typically captured \emph{via} some form of bisimulation.

% The notion we use in this paper is weak barbed bisimulation
% \cite{milner91polyadicpi}.

The notion we use in this paper is derived from weak barbed
bisimulation \cite{milner91polyadicpi}. 

\begin{definition}
An \emph{observation relation}, $\downarrow_{\mathcal N}$, over a set
of names, $\mathcal N$, is the smallest relation satisfying the rules
below.

\infrule[Out-barb]{y \in {\mathcal N}, \; x \nameeq y}
		  {\outputp{x}{v} \downarrow_{\mathcal N} x}
\infrule[Par-barb]{\mbox{$P\downarrow_{\mathcal N} x$ or $Q\downarrow_{\mathcal N} x$}}
		  {\binpar{P}{Q} \downarrow_{\mathcal N} x}

We write $P \Downarrow_{\mathcal N} x$ if there is $Q$ such that 
$P \wred Q$ and $Q \downarrow_{\mathcal N} x$.
\end{definition}

\begin{definition}
%\label{def.bbisim}
An  ${\mathcal N}$-\emph{barbed bisimulation} over a set of names, ${\mathcal N}$, is a symmetric binary relation 
${\mathcal S}_{\mathcal N}$ between agents such that $P\rel{S}_{\mathcal N}Q$ implies:
\begin{enumerate}
\item If $P \red P'$ then $Q \wred Q'$ and $P'\rel{S}_{\mathcal N} Q'$.
\item If $P\downarrow_{\mathcal N} x$, then $Q\Downarrow_{\mathcal N} x$.
\end{enumerate}
$P$ is ${\mathcal N}$-barbed bisimilar to $Q$, written
$P \wbbisim_{\mathcal N} Q$, if $P \rel{S}_{\mathcal N} Q$ for some ${\mathcal N}$-barbed bisimulation ${\mathcal S}_{\mathcal N}$.
\end{definition}

$\mathcal{R} \subseteq \pi \times \pi$

$P \mathcal{R} Q => \forall P'. P \red P' \Rightarrow \exists Q'. Q \red Q', P' \mathcal{R} Q'$

$P \vdash x \Rightarrow Q \vdash x$

\begin{mathpar}
  \inferrule*[lab=Out-barb]{x \nameeq y}{{y}!\langle{Q}\rangle \vdash x}
  \and
  \inferrule*[lab=Par-barb]{\mbox{$P\vdash x$ or $Q\vdash x$}}{\binpar{P}{Q} \vdash x}
\end{mathpar}

\subsubsection{Contexts}

One of the principle advantages of computational calculi like the
$\pi$-calculus is a well-defined notion of context,
contextual-equivalence and a correlation between
contextual-equivalence and notions of bisimulation. The notion of
context allows the decomposition of a process into (sub-)process and
its syntactic environment, its context. Thus, a context may be
thought of as a process with a ``hole'' (written $\Box$) in it. The
application of a context $M$ to a process $P$, written $M[P]$, is
tantamount to filling the hole in $M$ with $P$. In this paper we do
not need the full weight of this theory, but do make use of the notion
of context in the proof the main theorem. 

\begin{mathpar}
  \inferrule* [lab=summation] {} {{M_{M},M_{N}} \bc \Box \;|\; x.M_{A} \;|\; M_{M}+M_{N}}
  \and
  \inferrule* [lab=agent] {} {{M_{A}} \bc (\vec{x})M_{P} \;| \; \clift{P_0,\ldots,M_{P},\ldots,P_N}}
  \and \\
  \inferrule* [lab=process] {} {{M_{P}} \bc M_{N} \;| \;P|M_{P} }
\end{mathpar} 

\begin{mathpar}
  \inferrule* [lab=sychronization] {} {M_{N} \bc \Box \;|\; x?M_{F} \;|\; x!M_{C}}
  \and
  \inferrule* [lab=abstraction] {} {{M_{F}} \bc (x)M_{P} }
  \and
  \inferrule* [lab=concretion] {} {{M_{C}} \bc \langle M_{P} \rangle }
  \and \\
  \inferrule* [lab=process] {} {{M_{P}} \bc M_{N} \;| \;P|M_{P} }
\end{mathpar}

\begin{definition}[contextual application] Given a context $M$, and
  process $P$, we define the \emph{contextual application}, $M[P] :=
  M\{P/\Box\}$. That is, the contextual application of M to P is the
  substitution of $P$ for $\Box$ in $M$.
\end{definition}

$\meaningof{-} : L \to \mathcal{P}(\pi)$

\begin{mathpar}
  \inferrule* [lab=collection] {} {\meaningof{true} = \pi, \and \meaningof{~E} = \pi \setminus \meaningof{E}, \and \meaningof{E_{1} \& E_{2}} = \meaningof{E_{1}} \cap \meaningof{E_{2}}}
\end{mathpar}

\begin{mathpar}
  \inferrule* [lab=structure] {} {\meaningof{0} = \{ P \in \pi | P \equiv 0 \}, \and \\ \meaningof{E_1 | E_2} = \{ P \in \pi | P \equiv P_{1} | P_{2}, P_{1} \in \meaningof{E_{1}}, P_{2} \in \meaningof{E_2}\} }
\end{mathpar}

\begin{mathpar}
 \inferrule* [lab=behavior] {} {\meaningof{\langle a?b \rangle E} = \{ P \in \pi | P \equiv Q | u?(y)P', \\ \and \\\\ \and \\ \;\;\; u \in \meaningof{a}, \forall z.P'\{z/y\} \in \meaningof{E\{z/b\}}\}, \and \\ \meaningof{a!E} = \{ P \in \pi | P \equiv Q | x!\langle P' \rangle, x \in \meaningof{a} P' \in \meaningof{E}\} }
\end{mathpar}

\begin{mathpar}
 \inferrule* [lab=nominal] {} {\meaningof{\quotep{E}} = \{ \quotep{P} \in \quotep{\pi} | P \in \meaningof{E} \}, \and \meaningof{\quotep{P}} = \{ \quotep{Q} \in \quotep{\pi} | P \equiv Q \} \and \\ \meaningof{@\quotep{E}} = \{ P \in \pi | P \equiv @x, x \in \meaningof{E} \}}
\end{mathpar}

\begin{eqnarray*}
  \\
  \meaningof{-} : TS \to ST
\end{eqnarray*}

\begin{eqnarray*}
  \\
  L : TS \to ST
\end{eqnarray*}

\begin{eqnarray*}
  \\
  P \models E \iff P \in \meaningof{E}
\end{eqnarray*}

\begin{eqnarray*}
  P \approx_{L} Q \iff \forall E \in L. P \models E \iff Q \models E
\end{eqnarray*}

\begin{eqnarray*}
  P \approx_{K} Q
\end{eqnarray*}

\begin{eqnarray*}
  P \approx Q
\end{eqnarray*}

$\approx_{K} = \approx = \approx_{L}$

\subsubsection{Contextual duality}

Note that contexts extend the quotation operation to a family of
operations from processes to names. Given a context, $M$, we can
define a \emph{nominal context}, $\quotep{M}$ by $\quotep{M}[P] :=
\quotep{M[P]}$. To foreshadow what is to come we observe that these
operations enjoy a duality with processes very much like the duality
between vectors and maps from vectors to scalars.

Further, because the calculus is essentially higher-order, we have a
correspondence between contexts and processes. More specifically,
given a name $x$ and a context $M$ we can construct $M^{*}_{x}$ such
that 

\begin{mathpar}
  M^{*}_{x} | \lift{x}{P} \red M[P]
\end{mathpar}

namely,

\begin{mathpar}
  M^{*}_{x} := x?(u).M[\dropn{u}]
\end{mathpar}

The dependence of $M^{*}_{x}$ on a name makes it an abstraction, 

\begin{mathpar}
  M^{*} := (x)x?(u).M[\dropn{u}]
\end{mathpar}

\subsection{Additional notation}

It will sometimes be convenient to denote the process a name
quotes. We already have the notation $x = \quotep{P}$, but it will be
convenient to introduce an alternate notation, $\procn{x}$, when we
want to emphasize the connection to the use of the name. Note that, by
virtue of name equivalence, $\quotep{\procn{x}} \nameeq x$; so, the
notation is consistent with previous definitions.

Further, because names have structure it is possible to effect
substitutions on the basis of that structure. This means we need to
upgrade our notation for substitutions, which we accomplish by
adapting comprehension notation. Thus,

\begin{mathpar}
  P\{ y / x : x \in S \}
\end{mathpar}

is interpreted to mean the process derived from P by replacing (in a
capture-avoiding manner) each occurrence of $x$ in $S$ by $y$. For example,

\begin{mathpar}
  P\{ \quotep{\procn{x}|\procn{x}} / x : x \in \freenames{P} \}
\end{mathpar}

will replace each (occurrence) of a free name $x$ in $P$ by
$\quotep{\procn{x}|\procn{x}}$.

Also, we will avail ourselves of the notation $x^{L}$ and $x^{R}$ to
denote injections of a name into disjoint copies of the name
space. There are numerous ways to accomplish this. One example can be
found in \cite{MeredithR05}. This notation overloads to vectors of
names: $\vec{x}^{\pi} := (x_{i}^{\pi} \; : \; 0 \leq i < |\vec{x}| )$ where $\pi \in \{L,R\}$.

We also use $P^{\Box} := P|\Box$.

In \cite{MeredithR05} an interpretation of the new operator is
given. It turns out that there are several possible interpretations
all enjoying the requisite algebraic properties of the operator (see
\cite{milner91polyadicpi}). We will therefore make liberal use of
$(\nu\; \vec{x})P$.

% subsection the_syntax_and_semantics_of_the_notation_system (end)   

\input{qm2pi.qmops} 

\input{qm2pi.sterngerlach} 

\input{qm2pi.metric} 

% section concurrent_process_calculi (end)

%\input{qm2pi.proofsketch}

% section proof sketch (end)

%\input{qm2pi.slviaknots} 

% section spatial logic via knots (end)

\input{qm2pi.conclusion}

% section conclusion (end)

%\input{qm2pi.dtcodes} 

% section wiring algorithm (end)

\input{qm2pi.ack} 

% section acknowledgments (end)

\newpage


\bibliographystyle{plain}   
\bibliography{../../biblios/main.bib}

\input{qm2pi.rhodetails}

\end{document}



\end{document}



% section proof sketch (end)

%\section{Unlikely characters: spatial logic for
  knots}\label{sub:characteristic_formulae} % (fold)

Associated to the mobile process calculi are a family of logics known
as the Hennessy-Milner logics. These logics typically enjoy a
semantics interpreting formulae as sets of processes that when
factored through the encoding outlined above allows an identification
of classes of knots with logical formulae. In the context of this
encoding the sub-family known as the spatial logics \cite{CairesC03}
\cite{CairesC04} \cite{Caires04} are of particular interest providing
several important features for expressing and reasoning about
properties (i.e. classes) of knots. We hint here at how this may be done.

%\begin{description}
%\item [structural connectives] 
\subsubsection{Structural connectives} The spatial logics enjoy
structural connectives corresponding, at the logical level, to the
parallel composition ($P | Q$) and new name ($(\nu \; x)P$)
connectives for processes. As illustrated in the examples below, these
connectives are extremely expressive given the shape of our encoding.
%\item [decideable satisfaction]

\subsubsection{Decideable satisfaction}
In \cite{Caires04} the satisfaction relation is shown to be decideable
for a rich class of processes. It further turns out that the image of
the our encoding is a proper subset of that class. This result
provides the basis for an algorithm by which to search for knots
enjoying a given property.
%\item [characteristic formulae]

\subsubsection{Characteristic formulae}
In the same paper \cite{Caires04} , Caires presents a means of calculating
characteristic formulae, selecting equivalence classes of processes
up to a pre--specified depth limit on the support set of names. Composed with our
encoding, this characteristic formula can be used to select
characteristic formulae for knots.
%\end{description}

\subsubsection{Spatial logic formulae}

The grammar below (segmented for comprehension) summarizes the syntax
of spatial logic formulae. We employ illustrative examples in the
sequel to provide an intuitive understanding of their meaning
referring the reader to \cite{Caires04} for a more detailed explication
of the semantics.

\begin{mathpar}
  \inferrule* [lab=boolean] {} {{A,B} \bc T \;|\; \neg A \;|\; A \wedge B \;|\; \eta = \eta'}
  \and
  \inferrule* [lab=spatial] {} {|\; \pzero \;|\; A | B \;|\; x \text{\textregistered} A \;|\; \forall x . A \;|\;  H x . A}
  \and
  \inferrule* [lab=behavioral] {} {|\; \alpha . A}
  \and 
  \inferrule* [lab=recursion] {} {|\; X(\vec{u}) \;|\; \mu X(\vec{u}) . A}
  \and
  \inferrule* [lab=action] {} {\alpha \bc \langle x?(\vec{y}) \rangle \;|\; \langle x!(\vec{y}) \rangle \;|\; \langle \tau \rangle}
  \and 
  \inferrule* [lab=name] {} {\eta \bc x \;|\; \tau}
\end{mathpar} 

% subsection characteristic_formulae (end)   	 

\subsection{Example formulae}\label{sub:example_formulae_} % (fold)

\subsubsection{Crossing as formula.}
% 
% \begin{align*}
%   \frac{d}{dx} \sin x &= \cos x 
%   & \frac{d}{dx} e^x &= e^x \\
%   \frac{d}{dx} \cos x &= - \sin x 
%   & \frac{d}{dx} \log x &= \frac{1}{x} \\
% \end{align*} 

\begin{align*}
 \mu C(x_{0},x_{1},y_{0},y_{1},u).&(\langle x_{0}?(z) \rangle(\langle u! \rangle\langle y_{1}!z \rangle C(x_{0},x_{1},y_{0},y_{1},u)) & \\
  & \wedge \langle y_{1}?(z) \rangle (\langle u! \rangle \langle x_{0}!z \rangle C(x_{0},x_{1},y_{0},y_{1},u)) & \\
  & \wedge \langle x_{1}?(z) \rangle (\langle u? \rangle \langle y_{0}!z \rangle C(x_{0},x_{1},y_{0},y_{1},u)) & \\
  & \wedge \langle y_{0}?(z) \rangle (\langle u? \rangle \langle x_{1}!z \rangle C(x_{0},x_{1},y_{0},y_{1},u))) &
\end{align*}

The lexicographical similarity between the shape of this formulae and
the shape of definition of the process representing a crossing reveals
the intuitive meaning of this formulae. It describes the capabilities
of a process that has the right to represent a crossing. For example
it picks out processes that may perform an input on the port $x_0$ in
its initial menu of capabilities. What differentiates the formula
from the process, however, is that the crossing process is the
smallest candidate to satisfy the formula. Infinitely many other
processes -- with internal behavior hidden behind this interface, so
to speak -- also satisfy this formula. Even this simple formula,
then, can be seen to open a new view onto knots, providing a
computational interpretation of \emph{virtual} knots.

Note that this formula is derived by hand. A similar formula can be
derived by employing Caires' calculation of characteristic formula
\cite{Caires04} to the process representing a crossing. In light of
this discussion, we let
$\meaningof{C}_{\phi}(x0,x1,y0,y1,u)$ denote a formula specifying the
dynamics we wish to capture of a crossing. To guarantee we preserve
the shape of the interface and minimal semantics we demand that
$\meaningof{C}_{\phi}(x0,x1,y0,y1,u) \Rightarrow
\textbf{C}(x0,x1,y0,y1,u)$ where $\textbf{C}(x0,x1,y0,y1,u)$ denotes
the formula above.
                            
\subsubsection{Crossing number constraints.}
The moral content of the context lemma (Lemma \ref{context}) is that the notion of
``locality'' in the Reidemeister moves is effectively captured by the
parallel composition operator of the process calculus. This intuition
extends through the logic. Given a formula,
$\meaningof{C}_{\phi}(x0,x1,y0,y1,u)$, we can use the structural
connectives to specify constraints on crossing numbers, such as at
least $n$ crossings, or exactly $n$ crossings.
\begin{mathpar}
  \inferrule* [lab=at-least-n] {} { K^{\geq n}_{\phi}(\vec{xs},\vec{ys}) := \Pi_{i=0}^{n-1} Hu . \meaningof{C}_{\phi}(xs_i,ys_i,u) | T }
  \and 
  \inferrule* [lab=exactly-n] {} { K^{= n}_{\phi}(\vec{xs},\vec{ys}) := \Pi_{i=0}^{n-1} Hu . \meaningof{C}_{\phi}(xs_i,ys_i,u) | \neg (\forall x_0,y_0,x_1,y_1,u . \meaningof{C}_{\phi}(x_0,y_0,x_1,y_1,u) | T) }
\end{mathpar}

To round out this section, recall that the encoding of an $n$-crossing
knot decomposes into a parallel composition of $n$ \emph{copies} of a
crossing process together with a wiring harness. To specify different
knot classes with the same crossing number amounts to specifying
logical constraints on the wiring harness. In the interest of space,
we defer examples to a forthcoming paper. Suffice it to say that both
the conditions ``alternating knot'' and ``contains the tangle
corresponding to 5/3'' are expressible. For example, it is possible to
calculate the characteristic formula of a process corresponding to the
tangle 5/3 and conjoin it into the classifying formula via the
composition connective of the logic.

Finally, we wish to observe that it is entirely within reason to
contemplate a more domain-specific version of spatial logic tailored
to the shape of processes in the image of the encoding. Such a
domain-specific logic would have a better claim to the title formal
language of knot properties.

% subsection example_formulae_ (end)

% section knots_as_processes (end) 

% section spatial logic via knots (end)

\section{Conclusions and future work}

\paragraph{Testing physical space}
You, gentle reader, may wonder why of all the theorems to be proved
given this set up we pick the one above. In some sense it's hardly
central to quantum mechanics. We see it as central in the sense that
it firmly establishes a notion of physical space arising from a notion
of the equivalence of behavior. Relating bisimulation to a metric is a
big step forward, but one is faced with interpreting the relationship
of that metric space to something more physical. Quantum mechanical
notions of ``physical'' space are still far from intuitive, but by
relating this idea of distance as testing to calculations that predict
physical circumstances we are making a not insignificant step forward
toward an understanding of the physical space we inhabit as
essentially dynamic.

\paragraph{Effectivity and simulation}
One of the observations we have yet to make is that the entire program
spelled out here is effective. We have built various interpreters for
the reflective calculus at work in this interpretation. In principle,
then, we can simulate quantum mechanics on a computer. The place where
the simulation may lose fidelity is the infinitely branching summation
for the annihilator.

In this connection i also want to point out that the evaluation style
calculation of the inner product puts the non-determinism of the
summation right at the heart of measurement. This suggests that
Milner's original reduction-based formulation of the dynamics of his
calculi in terms of sums was not just notationally suggestive of a
notion of measure-and-continue but captured some significant part of
the physics.

\paragraph{Quantum continuations}
In light of this last observation i want to point out that the
predominant account of quantum mechanics is missing a key aspect of a
truly compositional story of the physical situation. In a real lab,
when a measurement is made the observation can be made to feed into
another device that then makes another measurement conditioned on the
results of the first. This means that after the superposition was
collapsed the entire experimental set up remained in
superposition. While QM offers a means of writing this down it doesn't
quite line up well with the well-trodden formulation of computation
and continuation that we see so succinctly expressed in Milner's
calculi. This suggests that there might be advantages to this account
of dynamics waiting to be explored.

\paragraph{Quantum logic}
In this connection, we also note that by virtue of having the
Hennessy-Milner construction, we can pull the construction through the
interpretation of QM. This gives us a natural candidate for a quantum
logic that enjoys an extremely tight connection with it's domain of
interpretation, making the construction much less ad hoc (rather it is
the image of functor!).

\paragraph{Quantum probabiity}
i have questions about the basis of the interpretation of inner
product as probability amplitude. In particular, using which
axiomatization of probability theory does the notion of probability
amplitude earn the right to be so dubbed? In other words, where is the
proof that the operation for calculating a probability amplitude (and
then squaring) satisfies the axioms of what it means to calculate a
probability? Even if such a proof exists (i have yet to find it in the
literature), i wonder if it might not be possible to turn things on
their heads. Can we view the calculation of the probability amplitude
as an axiomatization of probability? If so, then the definition we
give for calculating probability amplitude may provide the basis for
an \emph{effective} theory of probability.

\paragraph{Quantum vs ``biological'' information}
Finally, i want to conclude with a more philosophical observation. At
a recent workshop in which QM was a predominant topic i noticed
something about quantum information. The speaker was giving a riveting
discussion of axiomatic QM and showing how properties of ``no
cloning'' and ``no deleting'' emerged as consequences of the
axiomatization. Theorems of this form are necessary to give us a sense
of confidence that our axioms characterize the physical theory. What
struck me, though, was that if quantum information is neither erasable
nor replicable it is markedly different from \emph{life}. Two of the
things we know about life is that

\begin{itemize}
  \item it ends;
  \item to gain some measure of persistence, to transcend it's
    finitude it is imminently copyable.
\end{itemize}

Both of these qualities are summarized succinctly in the aphorism: all
flesh is grass. For me these two kinds of ``information'' -- call them
quantum and biological -- are end points on a spectrum of strategies
for persistence. At one end, we have those curious entities that enjoy
uniqueness and permanence; at the other, we have those who in the face
of a certain end and an uncertain present make a go of passing
something on. To me one of the more remarkable aspects of the latter
strategy is that in the presence of noise (and certain features of
copying) we get a kind of dynamism, a chance for improvement against a
given persistent condition.

% subsection other_calculi_other_bisimulations_and_geometry_as_behavior (end)




% section conclusion (end)

%\documentclass[12pt]{llncs}
%\documentclass{jktr}

\usepackage[pdftex]{hyperref}                   
\usepackage {listings}
\usepackage {mathpartir}
\usepackage{bcprules}
%\usepackage{listings}
                       
\usepackage{graphicx} 
%\usepackage[margins=2.5cm,nohead,nofoot]{geometry}
%\usepackage{geometry}
\usepackage{amsfonts}
\usepackage{amstext}
\usepackage{latexsym}
\usepackage{amssymb}
\usepackage{color}


%\include{myPreamble}
\documentclass[12pt]{llncs}
%\documentclass{jktr}

\usepackage[pdftex]{hyperref}                   
\usepackage {listings}
\usepackage {mathpartir}
\usepackage{bcprules}
%\usepackage{listings}
                       
\usepackage{graphicx} 
%\usepackage[margins=2.5cm,nohead,nofoot]{geometry}
%\usepackage{geometry}
\usepackage{amsfonts}
\usepackage{amstext}
\usepackage{latexsym}
\usepackage{amssymb}
\usepackage{color}


%\include{myPreamble}
\include{qm2pi.local} 

%\ifpdf
%\usepackage[pdftex]{graphicx}
%\else
%\usepackage{graphicx}
%\fi

 % \ifpdf
%  \usepackage{pdfsync}
%  \if


%\title{Brief Article}
%\author{David F. Snyder}
%\author{L.G. Meredith}

%\address{Dept. of Math., Texas State University--San Marcos, San Marcos, TX 78666}
       
\pagestyle{empty}


\begin{document}

\lstset{language=[Objective]Caml,frame=shadowbox}

\input{qm2pi.front}

% section front matter (end)

\input{qm2pi.intro} 
 
% section introduction (end)

% \input{qm2pi.knotations} 

% section notation (end)

\input{qm2pi.process.calculi} 

% section concurrent_process_calculi_and_spatial_logics_ (end)
    
%\input{qm2pi.knots2pi} 

%\input{qm2pi.trefoil} 

%\input{qm2pi.mainthm} 

% subsection basic_interpretation (end)

%\input{qm2pi.rho.presentation} 
\subsection{The syntax and semantics of the notation system}\label{sub:the_syntax_and_semantics_of_the_notation_system} % (fold)

We now summarize a technical presentation of the calculus that
embodies our theory of dynamics. The typical presentation of such a
calculus follows the style of giving generators and relations on
them. The grammar, below, describing term constructors, freely
generates the set of processes, $\Proc$. This set is then quotiented
by a relation known as structural congruence and it is over this set
that the notion of dynamics is expressed. This presentation is
essentially that of \cite{MeredithR05} with the addition of
polyadicity and summation. For readability we have relegated some of
the technical subtleties to an appendix.

\subsubsection{Process grammar}\label{subsub:process_grammar}

\begin{mathpar}
  \inferrule* [lab=synchronization] {} {{M} \bc \pzero \;|\; x?F \;|\; x!C }
  \and
  \inferrule* [lab=abstraction] {} {{F} \bc (x)P}
  \and
  \inferrule* [lab=concretion] {} {{C} \bc \langle Q \rangle}
  \and
  \inferrule* [lab=process] {} {{P,Q} \bc M \;| \;P|Q \;|\; @{x}}
  \and
  \inferrule* [lab=name] {} {{x} \bc \quotep{P}}
\end{mathpar} 

Note that $\vec{x}$ (resp. $\vec{P}$) denotes a vector of names
(resp. processes) of length $|\vec{x}|$ (resp. $|\vec{P}|$). We adopt
the following useful abbreviations.

\begin{mathpar}
   x?(\vec{y}).P := x.(\vec{y})P \and  x\clift{\vec{P}} := x.\clift{\vec{P}}
   \and x!(y) := \lift{x}{\dropn{y}}
   \and \Pi_{i=0}^{n-1}P_i := P_0 | \ldots | P_{n-1}
\end{mathpar}

\subsubsection{Structural congruence}

\paragraph{Free and bound names and alpha-equivalence.} At the
core of structural equivalence is alpha-equivalence which identifies
process that are the same up to a change of variable. Formally, we
recognize the distinction between free and bound names. The free names
of a process, $\freenames{P}$, may be calculated recursively as
follows:

\begin{mathpar}
\freenames{\pzero} := \emptyset
  \and \\
  \freenames{x?(y).P} := \{ x \} \cup (\freenames{P} \setminus \{ y \})
  \and 
  \freenames{x!\langle P \rangle} := \{ x \} \cup \{ P \} 
  \and \\
  \freenames{P|Q} := \freenames{P} \cup \freenames{Q}
  \and \\
  \freenames{@{x}} := \{ x \}
\end{mathpar}

$\pi$
$\quotep{\pi}$

$\freenames{-} : \pi \to \mathcal{P}(\quotep{\pi})$

\begin{eqnarray*}
  \freenames{\pzero} & := & \emptyset \\
  \freenames{x?(y).P} & := & \{ x \} \cup (\freenames{P} \setminus \{ y \}) \\
  \freenames{x!\langle P \rangle} & := & \{ x \} \cup \{ P \} \\
  \freenames{P|Q} & := & \freenames{P} \cup \freenames{Q} \\
  \freenames{\dropn{x}} & := & \{ x \}
\end{eqnarray*}

The bound names of a process, $\boundnames{P}$, are those names occurring in $P$
that are not free. For example, in $x?(y).0$, the name $x$ is free, while $y$ is bound.

\begin{mathpar}
  \inferrule* [lab=monoidal-laws] {} { P|Q \equiv Q|P \and P|0 \equiv P \and P|(Q|R) \equiv (P|Q)|R }
\end{mathpar}

\begin{mathpar}
  \inferrule* [lab=alpha-equivalence] {} { (x)P \equiv (y)P\{y/x\} \and y \not\in \freenames{P} }
\end{mathpar}

\begin{definition}
Then two processes, $P,Q$, are alpha-equivalent if $P = Q\{\vec{y}/\vec{x}\}$ for
some $\vec{x} \in \boundnames{Q},\vec{y} \in \boundnames{P}$, where $Q\{\vec{y}/\vec{x}\}$
denotes the capture-avoiding substitution of $\vec{y}$ for $\vec{x}$ in $Q$.
\end{definition}

\begin{definition}
  The {\em structural congruence} \cite{SangiorgiWalker} , $\equiv$,
  between processes is the least congruence containing
  alpha-equivalence, satisfying the abelian monoid laws
  (associativity, commutativity and $\pzero$ as identity) for parallel
  composition $|$ and for summation $+$.
\end{definition}

\subsection{Name equivalence}

We take name equivalence, written $\nameeq$, to be the smallest
equivalence relation generated by the following rules.

\begin{mathpar}
\inferrule*[lab=Quote-drop]
{ }
{ \quotep{@{x}} \nameeq x }

\inferrule*[lab=Struct-equiv]
{ P \scong Q }
{ \quotep{P} \nameeq \quotep{Q} }
\end{mathpar}

The astute reader will have noticed that the mutual recursion of names
and processes imposes a mutual recursion on alpha-equivalence and
structural equivalence via name-equivalence. Fortunately, all of this
works out pleasantly and we may calculate in the natural way, free of
concern. The reader interested in the details is referred to the
appendix \ref{appendix:rho_details}.

\subsection{Substitution}

We use $\Proc$ for the set of processes, $\QProc$ for the set of
names, and $\id{\{}\vec{y} / \vec{x} \id{\}}$ to denote partial maps,
$s : \QProc \rightarrow \QProc$. A map, $s$ lifts, uniquely, to a map
on process terms, $\widehat{s} : \Proc \rightarrow \Proc$ by the
following equations.

\begin{mathpar}
  (0) \psubstp{Q}{P} := 0 \\
  (R \juxtap S) \psubstp{Q}{P}
  :=    
  (R)\psubstp{Q}{P} \juxtap (S) \psubstp{Q}{P} \\
  (x?(y).R) \psubstp{Q}{P}    
  :=    
  (x)\substp{Q}{P} (z)\concat( (R \psubstn{z}{y}) \psubstp{Q}{P} ) \\
  (\lift{x}{R}) \psubstp{Q}{P}  
  :=
  \lift{(x)\substp{Q}{P}}{ R \psubstp{Q}{P} } \\
%   (\dropn{x})  \psubstp{Q}{P}       
%   := 
%   \left\{ 
%     \begin{array}{ccc} 
%       \dropn{\quotep{Q}} & & x \nameeq \quotep{P} \\
%       \dropn{x} & & otherwise \\
%     \end{array}
%   \right. 
  (\dropn{x})  \psubstp{Q}{P}       
  := 
  \left\{ 
    \begin{array}{ccc} 
      Q & & x \nameeq \quotep{P} \\
      \dropn{x} & & otherwise \\
    \end{array}
  \right.
\end{mathpar}
 

where

\begin{eqnarray}
  (x)\id{\{} \lpquote Q \rpquote / \lpquote P \rpquote \id{\}}            = 
  \left\{ 
    \begin{array}{ccc}
      \lpquote Q \rpquote & & x \nameeq \lpquote P \rpquote \\
      x & & otherwise \\
    \end{array}
  \right. \nonumber
\end{eqnarray}

and $z$ is chosen distinct from $\quotep{P}$, $\quotep{Q}$, the free
names in $Q$, and all the names in $R$. Our $\alpha$-equivalence will
be built in the standard way from this substitution.

\begin{remark}\label{rem:no_self_referential_names}
  One consequence of these definitions is that $\forall P. \quotep{P}
  \not\in \freenames{P}$.
\end{remark}

\subsection{ Dynamic quote: an example }

Anticipating something of what's to come, consider applying the
substitution, $\widehat{\id{\{}u / z \id{\}}}$, to the following pair
of processes, $\lift{w}{y!(z)}$ and $w[ \lpquote y!(z) \rpquote ]$.

\begin{eqnarray}
	\lift{w}{y!(z)}\widehat{\id{\{}u / z \id{\}}}
		& = &
		\lift{w}{y!(u)} \nonumber\\
	w[ \lpquote y!(z) \rpquote ] \widehat{ \id{\{}u / z \id{\}} }
		& = &
		w[ \lpquote y!(z) \rpquote ] \nonumber
\end{eqnarray}

Because the body of the process between quotes is impervious to
substitution, we get radically different answers. In fact, by
examining the first process in an input context,
e.g. $x?(z).\lift{w}{y!(z)}$, we see that the process under the lift
operator may be shaped by prefixed inputs binding a name inside it. In
this sense, the lift operator will be seen as a way to dynamically
construct processes before reifying them as names.

Finally equipped with these standard features we can present the
dynamics of the calculus.

\subsubsection{Operational semantics} 

Finally, we introduce the computational dynamics. What marks these
algebras as distinct from other more traditionally studied algebraic
structures, e.g. vector spaces or polynomial rings, is the manner in
which dynamics is captured. In traditional structures, dynamics is typically
expressed through morphisms between such structures, as in linear maps
between vector spaces or morphisms between rings. In algebras
associated with the semantics of computation, the dynamics is
expressed as part of the algebraic structure itself, through a
reduction reduction relation typically denoted by $\red$. Below, we
give a recursive presentation of this relation for the calculus used
in the encoding.

$\red \subseteq \pi \times \pi$
$\red : \pi \to \mathcal{P}(\pi)$

\begin{mathpar}
  \inferrule* [lab=Comm] { \textsf{match}( x_{src}, x_{trgt} ) } { x_{trgt}?(y)P \; | \; x_{src}!\langle {Q} \rangle \red P\{\quotep{Q}/y}\} }
  \and \\
  \inferrule* [lab=Par] {{P} \red {P}'} {{{P} | {Q}} \red {{P}' | {Q}}}
  \and
  \inferrule* [lab=Equiv]{{{P} \scong {P}'} \andalso {{P}' \red {Q}'} \andalso {{Q}' \scong {Q}}}{{P} \red {Q}}
\end{mathpar}

\begin{eqnarray*}
  match_{\equiv} (\quotep{P},\quotep{Q}) & := & P \equiv Q \\
  match_{\dagger}(\quotep{P},\quotep{Q}) & := & \forall R. P|Q \red^{*} R => R \red^{*} 0 \\
  match_{K}(\quotep{P},\quotep{Q}) & := & K \mbox{ for some context } K
\end{eqnarray*}

$u?(x)P | u!\langle Q \rangle \red P\{\quotep{Q}/x\}$

%We write $\wred$ for $\red^*$, and $P\red$ if $\exists Q $ such that $ P \red Q$.
We write $P\red$ if $\exists Q $ such that $ P \red Q$ and $P\not\red$, otherwise.

\section{Replication}

As mentioned before, it is known that replication (and hence
recursion) can be implemented in a higher-order process algebra
\cite{SangiorgiWalker}. As our first example of calculation with the
machinery thus far presented we give the construction explicitly in
the {\rhoc}.

\begin{eqnarray}
	D_{x} & := & \prefix{x}{y}{(\binpar{\outputp{x}{y}}{@{y}})} \nonumber\\
	\bangp_{x}{P} & := & \binpar{{x}!\langle{\binpar{D_{x}}{P}}\rangle}{D_{x}} \nonumber
\end{eqnarray}

\begin{eqnarray}
	\bangp_{x}{P} & & \nonumber\\
	=
	& {x}!\langle{(\prefix{x}{y}{(\outputp{x}{y} | @{y})) | P}}\rangle 
	      | \prefix{x}{y}{(\outputp{x}{y} | @{y})} & \nonumber\\
	\red
	& (\outputp{x}{y} | @{y})\substn{\quotep{(\prefix{x}{y}{(@{y} | \outputp{x}{y})) | P}}}{y} & \nonumber\\
	=
	& \outputp{x}{\quotep{(\prefix{x}{y}{(\outputp{x}{y} | @{y})) | P}}}
	  | {(\prefix{x}{y}{(\outputp{x}{y} | @{y})) | P}} & \nonumber\\
	\red
	& \ldots & \nonumber\\
	\red^*
	& P | P | \ldots & \nonumber
\end{eqnarray}

Of course, this encoding, as an implementation, runs away, unfolding
$\bangp{P}$ eagerly. A lazier and more implementable replication
operator, restricted to input-guarded processes, may be obtained as follows.

\begin{eqnarray}
\bangp{\prefix{u}{v}{P}} 
	:= 
	\binpar{\lift{x}{\prefix{u}{v}{(\binpar{D(x)}{P})}}}{D(x)} \nonumber
\end{eqnarray}

\begin{remark}
  Note that the lazier definition still does not deal with summation
  or mixed summation (i.e. sums over input and output). The reader is
  invited to construct definitions of replication that deal with these
  features. 

  Further, the definitions are parameterized in a name, $x$. Can you,
  gentle reader, make a definition that eliminates this parameter and
  guarantees no accidental interaction between the replication
  machinery and the process being replicated -- i.e. no accidental
  sharing of names used by the process to get its work done and the
  name(s) used by the replication to effect copying. This latter
  revision of the definition of replication is crucial to obtaining
  the expected identity $!!P \sim !P$.
\end{remark}

\begin{remark}\label{rem:paradoxical_combinator}
  The reader familiar with the lambda calculus will have noticed the
  similarity between $D$ and the paradoxical combinator.

  [Ed. note: the existence of this seems to suggest we have to be more
  restrictive on the set of processes and names we admit if we are to
  support no-cloning.]
\end{remark}

\subsubsection{Bisimulation}

The computational dynamics gives rise to another kind of equivalence,
the equivalence of computational behavior. As previously mentioned
this is typically captured \emph{via} some form of bisimulation.

% The notion we use in this paper is weak barbed bisimulation
% \cite{milner91polyadicpi}.

The notion we use in this paper is derived from weak barbed
bisimulation \cite{milner91polyadicpi}. 

\begin{definition}
An \emph{observation relation}, $\downarrow_{\mathcal N}$, over a set
of names, $\mathcal N$, is the smallest relation satisfying the rules
below.

\infrule[Out-barb]{y \in {\mathcal N}, \; x \nameeq y}
		  {\outputp{x}{v} \downarrow_{\mathcal N} x}
\infrule[Par-barb]{\mbox{$P\downarrow_{\mathcal N} x$ or $Q\downarrow_{\mathcal N} x$}}
		  {\binpar{P}{Q} \downarrow_{\mathcal N} x}

We write $P \Downarrow_{\mathcal N} x$ if there is $Q$ such that 
$P \wred Q$ and $Q \downarrow_{\mathcal N} x$.
\end{definition}

\begin{definition}
%\label{def.bbisim}
An  ${\mathcal N}$-\emph{barbed bisimulation} over a set of names, ${\mathcal N}$, is a symmetric binary relation 
${\mathcal S}_{\mathcal N}$ between agents such that $P\rel{S}_{\mathcal N}Q$ implies:
\begin{enumerate}
\item If $P \red P'$ then $Q \wred Q'$ and $P'\rel{S}_{\mathcal N} Q'$.
\item If $P\downarrow_{\mathcal N} x$, then $Q\Downarrow_{\mathcal N} x$.
\end{enumerate}
$P$ is ${\mathcal N}$-barbed bisimilar to $Q$, written
$P \wbbisim_{\mathcal N} Q$, if $P \rel{S}_{\mathcal N} Q$ for some ${\mathcal N}$-barbed bisimulation ${\mathcal S}_{\mathcal N}$.
\end{definition}

$\mathcal{R} \subseteq \pi \times \pi$

$P \mathcal{R} Q => \forall P'. P \red P' \Rightarrow \exists Q'. Q \red Q', P' \mathcal{R} Q'$

$P \vdash x \Rightarrow Q \vdash x$

\begin{mathpar}
  \inferrule*[lab=Out-barb]{x \nameeq y}{{y}!\langle{Q}\rangle \vdash x}
  \and
  \inferrule*[lab=Par-barb]{\mbox{$P\vdash x$ or $Q\vdash x$}}{\binpar{P}{Q} \vdash x}
\end{mathpar}

\subsubsection{Contexts}

One of the principle advantages of computational calculi like the
$\pi$-calculus is a well-defined notion of context,
contextual-equivalence and a correlation between
contextual-equivalence and notions of bisimulation. The notion of
context allows the decomposition of a process into (sub-)process and
its syntactic environment, its context. Thus, a context may be
thought of as a process with a ``hole'' (written $\Box$) in it. The
application of a context $M$ to a process $P$, written $M[P]$, is
tantamount to filling the hole in $M$ with $P$. In this paper we do
not need the full weight of this theory, but do make use of the notion
of context in the proof the main theorem. 

\begin{mathpar}
  \inferrule* [lab=summation] {} {{M_{M},M_{N}} \bc \Box \;|\; x.M_{A} \;|\; M_{M}+M_{N}}
  \and
  \inferrule* [lab=agent] {} {{M_{A}} \bc (\vec{x})M_{P} \;| \; \clift{P_0,\ldots,M_{P},\ldots,P_N}}
  \and \\
  \inferrule* [lab=process] {} {{M_{P}} \bc M_{N} \;| \;P|M_{P} }
\end{mathpar} 

\begin{mathpar}
  \inferrule* [lab=sychronization] {} {M_{N} \bc \Box \;|\; x?M_{F} \;|\; x!M_{C}}
  \and
  \inferrule* [lab=abstraction] {} {{M_{F}} \bc (x)M_{P} }
  \and
  \inferrule* [lab=concretion] {} {{M_{C}} \bc \langle M_{P} \rangle }
  \and \\
  \inferrule* [lab=process] {} {{M_{P}} \bc M_{N} \;| \;P|M_{P} }
\end{mathpar}

\begin{definition}[contextual application] Given a context $M$, and
  process $P$, we define the \emph{contextual application}, $M[P] :=
  M\{P/\Box\}$. That is, the contextual application of M to P is the
  substitution of $P$ for $\Box$ in $M$.
\end{definition}

$\meaningof{-} : L \to \mathcal{P}(\pi)$

\begin{mathpar}
  \inferrule* [lab=collection] {} {\meaningof{true} = \pi, \and \meaningof{~E} = \pi \setminus \meaningof{E}, \and \meaningof{E_{1} \& E_{2}} = \meaningof{E_{1}} \cap \meaningof{E_{2}}}
\end{mathpar}

\begin{mathpar}
  \inferrule* [lab=structure] {} {\meaningof{0} = \{ P \in \pi | P \equiv 0 \}, \and \\ \meaningof{E_1 | E_2} = \{ P \in \pi | P \equiv P_{1} | P_{2}, P_{1} \in \meaningof{E_{1}}, P_{2} \in \meaningof{E_2}\} }
\end{mathpar}

\begin{mathpar}
 \inferrule* [lab=behavior] {} {\meaningof{\langle a?b \rangle E} = \{ P \in \pi | P \equiv Q | u?(y)P', \\ \and \\\\ \and \\ \;\;\; u \in \meaningof{a}, \forall z.P'\{z/y\} \in \meaningof{E\{z/b\}}\}, \and \\ \meaningof{a!E} = \{ P \in \pi | P \equiv Q | x!\langle P' \rangle, x \in \meaningof{a} P' \in \meaningof{E}\} }
\end{mathpar}

\begin{mathpar}
 \inferrule* [lab=nominal] {} {\meaningof{\quotep{E}} = \{ \quotep{P} \in \quotep{\pi} | P \in \meaningof{E} \}, \and \meaningof{\quotep{P}} = \{ \quotep{Q} \in \quotep{\pi} | P \equiv Q \} \and \\ \meaningof{@\quotep{E}} = \{ P \in \pi | P \equiv @x, x \in \meaningof{E} \}}
\end{mathpar}

\begin{eqnarray*}
  \\
  \meaningof{-} : TS \to ST
\end{eqnarray*}

\begin{eqnarray*}
  \\
  L : TS \to ST
\end{eqnarray*}

\begin{eqnarray*}
  \\
  P \models E \iff P \in \meaningof{E}
\end{eqnarray*}

\begin{eqnarray*}
  P \approx_{L} Q \iff \forall E \in L. P \models E \iff Q \models E
\end{eqnarray*}

\begin{eqnarray*}
  P \approx_{K} Q
\end{eqnarray*}

\begin{eqnarray*}
  P \approx Q
\end{eqnarray*}

$\approx_{K} = \approx = \approx_{L}$

\subsubsection{Contextual duality}

Note that contexts extend the quotation operation to a family of
operations from processes to names. Given a context, $M$, we can
define a \emph{nominal context}, $\quotep{M}$ by $\quotep{M}[P] :=
\quotep{M[P]}$. To foreshadow what is to come we observe that these
operations enjoy a duality with processes very much like the duality
between vectors and maps from vectors to scalars.

Further, because the calculus is essentially higher-order, we have a
correspondence between contexts and processes. More specifically,
given a name $x$ and a context $M$ we can construct $M^{*}_{x}$ such
that 

\begin{mathpar}
  M^{*}_{x} | \lift{x}{P} \red M[P]
\end{mathpar}

namely,

\begin{mathpar}
  M^{*}_{x} := x?(u).M[\dropn{u}]
\end{mathpar}

The dependence of $M^{*}_{x}$ on a name makes it an abstraction, 

\begin{mathpar}
  M^{*} := (x)x?(u).M[\dropn{u}]
\end{mathpar}

\subsection{Additional notation}

It will sometimes be convenient to denote the process a name
quotes. We already have the notation $x = \quotep{P}$, but it will be
convenient to introduce an alternate notation, $\procn{x}$, when we
want to emphasize the connection to the use of the name. Note that, by
virtue of name equivalence, $\quotep{\procn{x}} \nameeq x$; so, the
notation is consistent with previous definitions.

Further, because names have structure it is possible to effect
substitutions on the basis of that structure. This means we need to
upgrade our notation for substitutions, which we accomplish by
adapting comprehension notation. Thus,

\begin{mathpar}
  P\{ y / x : x \in S \}
\end{mathpar}

is interpreted to mean the process derived from P by replacing (in a
capture-avoiding manner) each occurrence of $x$ in $S$ by $y$. For example,

\begin{mathpar}
  P\{ \quotep{\procn{x}|\procn{x}} / x : x \in \freenames{P} \}
\end{mathpar}

will replace each (occurrence) of a free name $x$ in $P$ by
$\quotep{\procn{x}|\procn{x}}$.

Also, we will avail ourselves of the notation $x^{L}$ and $x^{R}$ to
denote injections of a name into disjoint copies of the name
space. There are numerous ways to accomplish this. One example can be
found in \cite{MeredithR05}. This notation overloads to vectors of
names: $\vec{x}^{\pi} := (x_{i}^{\pi} \; : \; 0 \leq i < |\vec{x}| )$ where $\pi \in \{L,R\}$.

We also use $P^{\Box} := P|\Box$.

In \cite{MeredithR05} an interpretation of the new operator is
given. It turns out that there are several possible interpretations
all enjoying the requisite algebraic properties of the operator (see
\cite{milner91polyadicpi}). We will therefore make liberal use of
$(\nu\; \vec{x})P$.

% subsection the_syntax_and_semantics_of_the_notation_system (end)   

\input{qm2pi.qmops} 

\input{qm2pi.sterngerlach} 

\input{qm2pi.metric} 

% section concurrent_process_calculi (end)

%\input{qm2pi.proofsketch}

% section proof sketch (end)

%\input{qm2pi.slviaknots} 

% section spatial logic via knots (end)

\input{qm2pi.conclusion}

% section conclusion (end)

%\input{qm2pi.dtcodes} 

% section wiring algorithm (end)

\input{qm2pi.ack} 

% section acknowledgments (end)

\newpage


\bibliographystyle{plain}   
\bibliography{../../biblios/main.bib}

\input{qm2pi.rhodetails}

\end{document}

 

%\ifpdf
%\usepackage[pdftex]{graphicx}
%\else
%\usepackage{graphicx}
%\fi

 % \ifpdf
%  \usepackage{pdfsync}
%  \if


%\title{Brief Article}
%\author{David F. Snyder}
%\author{L.G. Meredith}

%\address{Dept. of Math., Texas State University--San Marcos, San Marcos, TX 78666}
       
\pagestyle{empty}


\begin{document}

\lstset{language=[Objective]Caml,frame=shadowbox}

\documentclass[12pt]{llncs}
%\documentclass{jktr}

\usepackage[pdftex]{hyperref}                   
\usepackage {listings}
\usepackage {mathpartir}
\usepackage{bcprules}
%\usepackage{listings}
                       
\usepackage{graphicx} 
%\usepackage[margins=2.5cm,nohead,nofoot]{geometry}
%\usepackage{geometry}
\usepackage{amsfonts}
\usepackage{amstext}
\usepackage{latexsym}
\usepackage{amssymb}
\usepackage{color}


%\include{myPreamble}
\include{qm2pi.local} 

%\ifpdf
%\usepackage[pdftex]{graphicx}
%\else
%\usepackage{graphicx}
%\fi

 % \ifpdf
%  \usepackage{pdfsync}
%  \if


%\title{Brief Article}
%\author{David F. Snyder}
%\author{L.G. Meredith}

%\address{Dept. of Math., Texas State University--San Marcos, San Marcos, TX 78666}
       
\pagestyle{empty}


\begin{document}

\lstset{language=[Objective]Caml,frame=shadowbox}

\input{qm2pi.front}

% section front matter (end)

\input{qm2pi.intro} 
 
% section introduction (end)

% \input{qm2pi.knotations} 

% section notation (end)

\input{qm2pi.process.calculi} 

% section concurrent_process_calculi_and_spatial_logics_ (end)
    
%\input{qm2pi.knots2pi} 

%\input{qm2pi.trefoil} 

%\input{qm2pi.mainthm} 

% subsection basic_interpretation (end)

%\input{qm2pi.rho.presentation} 
\subsection{The syntax and semantics of the notation system}\label{sub:the_syntax_and_semantics_of_the_notation_system} % (fold)

We now summarize a technical presentation of the calculus that
embodies our theory of dynamics. The typical presentation of such a
calculus follows the style of giving generators and relations on
them. The grammar, below, describing term constructors, freely
generates the set of processes, $\Proc$. This set is then quotiented
by a relation known as structural congruence and it is over this set
that the notion of dynamics is expressed. This presentation is
essentially that of \cite{MeredithR05} with the addition of
polyadicity and summation. For readability we have relegated some of
the technical subtleties to an appendix.

\subsubsection{Process grammar}\label{subsub:process_grammar}

\begin{mathpar}
  \inferrule* [lab=synchronization] {} {{M} \bc \pzero \;|\; x?F \;|\; x!C }
  \and
  \inferrule* [lab=abstraction] {} {{F} \bc (x)P}
  \and
  \inferrule* [lab=concretion] {} {{C} \bc \langle Q \rangle}
  \and
  \inferrule* [lab=process] {} {{P,Q} \bc M \;| \;P|Q \;|\; @{x}}
  \and
  \inferrule* [lab=name] {} {{x} \bc \quotep{P}}
\end{mathpar} 

Note that $\vec{x}$ (resp. $\vec{P}$) denotes a vector of names
(resp. processes) of length $|\vec{x}|$ (resp. $|\vec{P}|$). We adopt
the following useful abbreviations.

\begin{mathpar}
   x?(\vec{y}).P := x.(\vec{y})P \and  x\clift{\vec{P}} := x.\clift{\vec{P}}
   \and x!(y) := \lift{x}{\dropn{y}}
   \and \Pi_{i=0}^{n-1}P_i := P_0 | \ldots | P_{n-1}
\end{mathpar}

\subsubsection{Structural congruence}

\paragraph{Free and bound names and alpha-equivalence.} At the
core of structural equivalence is alpha-equivalence which identifies
process that are the same up to a change of variable. Formally, we
recognize the distinction between free and bound names. The free names
of a process, $\freenames{P}$, may be calculated recursively as
follows:

\begin{mathpar}
\freenames{\pzero} := \emptyset
  \and \\
  \freenames{x?(y).P} := \{ x \} \cup (\freenames{P} \setminus \{ y \})
  \and 
  \freenames{x!\langle P \rangle} := \{ x \} \cup \{ P \} 
  \and \\
  \freenames{P|Q} := \freenames{P} \cup \freenames{Q}
  \and \\
  \freenames{@{x}} := \{ x \}
\end{mathpar}

$\pi$
$\quotep{\pi}$

$\freenames{-} : \pi \to \mathcal{P}(\quotep{\pi})$

\begin{eqnarray*}
  \freenames{\pzero} & := & \emptyset \\
  \freenames{x?(y).P} & := & \{ x \} \cup (\freenames{P} \setminus \{ y \}) \\
  \freenames{x!\langle P \rangle} & := & \{ x \} \cup \{ P \} \\
  \freenames{P|Q} & := & \freenames{P} \cup \freenames{Q} \\
  \freenames{\dropn{x}} & := & \{ x \}
\end{eqnarray*}

The bound names of a process, $\boundnames{P}$, are those names occurring in $P$
that are not free. For example, in $x?(y).0$, the name $x$ is free, while $y$ is bound.

\begin{mathpar}
  \inferrule* [lab=monoidal-laws] {} { P|Q \equiv Q|P \and P|0 \equiv P \and P|(Q|R) \equiv (P|Q)|R }
\end{mathpar}

\begin{mathpar}
  \inferrule* [lab=alpha-equivalence] {} { (x)P \equiv (y)P\{y/x\} \and y \not\in \freenames{P} }
\end{mathpar}

\begin{definition}
Then two processes, $P,Q$, are alpha-equivalent if $P = Q\{\vec{y}/\vec{x}\}$ for
some $\vec{x} \in \boundnames{Q},\vec{y} \in \boundnames{P}$, where $Q\{\vec{y}/\vec{x}\}$
denotes the capture-avoiding substitution of $\vec{y}$ for $\vec{x}$ in $Q$.
\end{definition}

\begin{definition}
  The {\em structural congruence} \cite{SangiorgiWalker} , $\equiv$,
  between processes is the least congruence containing
  alpha-equivalence, satisfying the abelian monoid laws
  (associativity, commutativity and $\pzero$ as identity) for parallel
  composition $|$ and for summation $+$.
\end{definition}

\subsection{Name equivalence}

We take name equivalence, written $\nameeq$, to be the smallest
equivalence relation generated by the following rules.

\begin{mathpar}
\inferrule*[lab=Quote-drop]
{ }
{ \quotep{@{x}} \nameeq x }

\inferrule*[lab=Struct-equiv]
{ P \scong Q }
{ \quotep{P} \nameeq \quotep{Q} }
\end{mathpar}

The astute reader will have noticed that the mutual recursion of names
and processes imposes a mutual recursion on alpha-equivalence and
structural equivalence via name-equivalence. Fortunately, all of this
works out pleasantly and we may calculate in the natural way, free of
concern. The reader interested in the details is referred to the
appendix \ref{appendix:rho_details}.

\subsection{Substitution}

We use $\Proc$ for the set of processes, $\QProc$ for the set of
names, and $\id{\{}\vec{y} / \vec{x} \id{\}}$ to denote partial maps,
$s : \QProc \rightarrow \QProc$. A map, $s$ lifts, uniquely, to a map
on process terms, $\widehat{s} : \Proc \rightarrow \Proc$ by the
following equations.

\begin{mathpar}
  (0) \psubstp{Q}{P} := 0 \\
  (R \juxtap S) \psubstp{Q}{P}
  :=    
  (R)\psubstp{Q}{P} \juxtap (S) \psubstp{Q}{P} \\
  (x?(y).R) \psubstp{Q}{P}    
  :=    
  (x)\substp{Q}{P} (z)\concat( (R \psubstn{z}{y}) \psubstp{Q}{P} ) \\
  (\lift{x}{R}) \psubstp{Q}{P}  
  :=
  \lift{(x)\substp{Q}{P}}{ R \psubstp{Q}{P} } \\
%   (\dropn{x})  \psubstp{Q}{P}       
%   := 
%   \left\{ 
%     \begin{array}{ccc} 
%       \dropn{\quotep{Q}} & & x \nameeq \quotep{P} \\
%       \dropn{x} & & otherwise \\
%     \end{array}
%   \right. 
  (\dropn{x})  \psubstp{Q}{P}       
  := 
  \left\{ 
    \begin{array}{ccc} 
      Q & & x \nameeq \quotep{P} \\
      \dropn{x} & & otherwise \\
    \end{array}
  \right.
\end{mathpar}
 

where

\begin{eqnarray}
  (x)\id{\{} \lpquote Q \rpquote / \lpquote P \rpquote \id{\}}            = 
  \left\{ 
    \begin{array}{ccc}
      \lpquote Q \rpquote & & x \nameeq \lpquote P \rpquote \\
      x & & otherwise \\
    \end{array}
  \right. \nonumber
\end{eqnarray}

and $z$ is chosen distinct from $\quotep{P}$, $\quotep{Q}$, the free
names in $Q$, and all the names in $R$. Our $\alpha$-equivalence will
be built in the standard way from this substitution.

\begin{remark}\label{rem:no_self_referential_names}
  One consequence of these definitions is that $\forall P. \quotep{P}
  \not\in \freenames{P}$.
\end{remark}

\subsection{ Dynamic quote: an example }

Anticipating something of what's to come, consider applying the
substitution, $\widehat{\id{\{}u / z \id{\}}}$, to the following pair
of processes, $\lift{w}{y!(z)}$ and $w[ \lpquote y!(z) \rpquote ]$.

\begin{eqnarray}
	\lift{w}{y!(z)}\widehat{\id{\{}u / z \id{\}}}
		& = &
		\lift{w}{y!(u)} \nonumber\\
	w[ \lpquote y!(z) \rpquote ] \widehat{ \id{\{}u / z \id{\}} }
		& = &
		w[ \lpquote y!(z) \rpquote ] \nonumber
\end{eqnarray}

Because the body of the process between quotes is impervious to
substitution, we get radically different answers. In fact, by
examining the first process in an input context,
e.g. $x?(z).\lift{w}{y!(z)}$, we see that the process under the lift
operator may be shaped by prefixed inputs binding a name inside it. In
this sense, the lift operator will be seen as a way to dynamically
construct processes before reifying them as names.

Finally equipped with these standard features we can present the
dynamics of the calculus.

\subsubsection{Operational semantics} 

Finally, we introduce the computational dynamics. What marks these
algebras as distinct from other more traditionally studied algebraic
structures, e.g. vector spaces or polynomial rings, is the manner in
which dynamics is captured. In traditional structures, dynamics is typically
expressed through morphisms between such structures, as in linear maps
between vector spaces or morphisms between rings. In algebras
associated with the semantics of computation, the dynamics is
expressed as part of the algebraic structure itself, through a
reduction reduction relation typically denoted by $\red$. Below, we
give a recursive presentation of this relation for the calculus used
in the encoding.

$\red \subseteq \pi \times \pi$
$\red : \pi \to \mathcal{P}(\pi)$

\begin{mathpar}
  \inferrule* [lab=Comm] { \textsf{match}( x_{src}, x_{trgt} ) } { x_{trgt}?(y)P \; | \; x_{src}!\langle {Q} \rangle \red P\{\quotep{Q}/y}\} }
  \and \\
  \inferrule* [lab=Par] {{P} \red {P}'} {{{P} | {Q}} \red {{P}' | {Q}}}
  \and
  \inferrule* [lab=Equiv]{{{P} \scong {P}'} \andalso {{P}' \red {Q}'} \andalso {{Q}' \scong {Q}}}{{P} \red {Q}}
\end{mathpar}

\begin{eqnarray*}
  match_{\equiv} (\quotep{P},\quotep{Q}) & := & P \equiv Q \\
  match_{\dagger}(\quotep{P},\quotep{Q}) & := & \forall R. P|Q \red^{*} R => R \red^{*} 0 \\
  match_{K}(\quotep{P},\quotep{Q}) & := & K \mbox{ for some context } K
\end{eqnarray*}

$u?(x)P | u!\langle Q \rangle \red P\{\quotep{Q}/x\}$

%We write $\wred$ for $\red^*$, and $P\red$ if $\exists Q $ such that $ P \red Q$.
We write $P\red$ if $\exists Q $ such that $ P \red Q$ and $P\not\red$, otherwise.

\section{Replication}

As mentioned before, it is known that replication (and hence
recursion) can be implemented in a higher-order process algebra
\cite{SangiorgiWalker}. As our first example of calculation with the
machinery thus far presented we give the construction explicitly in
the {\rhoc}.

\begin{eqnarray}
	D_{x} & := & \prefix{x}{y}{(\binpar{\outputp{x}{y}}{@{y}})} \nonumber\\
	\bangp_{x}{P} & := & \binpar{{x}!\langle{\binpar{D_{x}}{P}}\rangle}{D_{x}} \nonumber
\end{eqnarray}

\begin{eqnarray}
	\bangp_{x}{P} & & \nonumber\\
	=
	& {x}!\langle{(\prefix{x}{y}{(\outputp{x}{y} | @{y})) | P}}\rangle 
	      | \prefix{x}{y}{(\outputp{x}{y} | @{y})} & \nonumber\\
	\red
	& (\outputp{x}{y} | @{y})\substn{\quotep{(\prefix{x}{y}{(@{y} | \outputp{x}{y})) | P}}}{y} & \nonumber\\
	=
	& \outputp{x}{\quotep{(\prefix{x}{y}{(\outputp{x}{y} | @{y})) | P}}}
	  | {(\prefix{x}{y}{(\outputp{x}{y} | @{y})) | P}} & \nonumber\\
	\red
	& \ldots & \nonumber\\
	\red^*
	& P | P | \ldots & \nonumber
\end{eqnarray}

Of course, this encoding, as an implementation, runs away, unfolding
$\bangp{P}$ eagerly. A lazier and more implementable replication
operator, restricted to input-guarded processes, may be obtained as follows.

\begin{eqnarray}
\bangp{\prefix{u}{v}{P}} 
	:= 
	\binpar{\lift{x}{\prefix{u}{v}{(\binpar{D(x)}{P})}}}{D(x)} \nonumber
\end{eqnarray}

\begin{remark}
  Note that the lazier definition still does not deal with summation
  or mixed summation (i.e. sums over input and output). The reader is
  invited to construct definitions of replication that deal with these
  features. 

  Further, the definitions are parameterized in a name, $x$. Can you,
  gentle reader, make a definition that eliminates this parameter and
  guarantees no accidental interaction between the replication
  machinery and the process being replicated -- i.e. no accidental
  sharing of names used by the process to get its work done and the
  name(s) used by the replication to effect copying. This latter
  revision of the definition of replication is crucial to obtaining
  the expected identity $!!P \sim !P$.
\end{remark}

\begin{remark}\label{rem:paradoxical_combinator}
  The reader familiar with the lambda calculus will have noticed the
  similarity between $D$ and the paradoxical combinator.

  [Ed. note: the existence of this seems to suggest we have to be more
  restrictive on the set of processes and names we admit if we are to
  support no-cloning.]
\end{remark}

\subsubsection{Bisimulation}

The computational dynamics gives rise to another kind of equivalence,
the equivalence of computational behavior. As previously mentioned
this is typically captured \emph{via} some form of bisimulation.

% The notion we use in this paper is weak barbed bisimulation
% \cite{milner91polyadicpi}.

The notion we use in this paper is derived from weak barbed
bisimulation \cite{milner91polyadicpi}. 

\begin{definition}
An \emph{observation relation}, $\downarrow_{\mathcal N}$, over a set
of names, $\mathcal N$, is the smallest relation satisfying the rules
below.

\infrule[Out-barb]{y \in {\mathcal N}, \; x \nameeq y}
		  {\outputp{x}{v} \downarrow_{\mathcal N} x}
\infrule[Par-barb]{\mbox{$P\downarrow_{\mathcal N} x$ or $Q\downarrow_{\mathcal N} x$}}
		  {\binpar{P}{Q} \downarrow_{\mathcal N} x}

We write $P \Downarrow_{\mathcal N} x$ if there is $Q$ such that 
$P \wred Q$ and $Q \downarrow_{\mathcal N} x$.
\end{definition}

\begin{definition}
%\label{def.bbisim}
An  ${\mathcal N}$-\emph{barbed bisimulation} over a set of names, ${\mathcal N}$, is a symmetric binary relation 
${\mathcal S}_{\mathcal N}$ between agents such that $P\rel{S}_{\mathcal N}Q$ implies:
\begin{enumerate}
\item If $P \red P'$ then $Q \wred Q'$ and $P'\rel{S}_{\mathcal N} Q'$.
\item If $P\downarrow_{\mathcal N} x$, then $Q\Downarrow_{\mathcal N} x$.
\end{enumerate}
$P$ is ${\mathcal N}$-barbed bisimilar to $Q$, written
$P \wbbisim_{\mathcal N} Q$, if $P \rel{S}_{\mathcal N} Q$ for some ${\mathcal N}$-barbed bisimulation ${\mathcal S}_{\mathcal N}$.
\end{definition}

$\mathcal{R} \subseteq \pi \times \pi$

$P \mathcal{R} Q => \forall P'. P \red P' \Rightarrow \exists Q'. Q \red Q', P' \mathcal{R} Q'$

$P \vdash x \Rightarrow Q \vdash x$

\begin{mathpar}
  \inferrule*[lab=Out-barb]{x \nameeq y}{{y}!\langle{Q}\rangle \vdash x}
  \and
  \inferrule*[lab=Par-barb]{\mbox{$P\vdash x$ or $Q\vdash x$}}{\binpar{P}{Q} \vdash x}
\end{mathpar}

\subsubsection{Contexts}

One of the principle advantages of computational calculi like the
$\pi$-calculus is a well-defined notion of context,
contextual-equivalence and a correlation between
contextual-equivalence and notions of bisimulation. The notion of
context allows the decomposition of a process into (sub-)process and
its syntactic environment, its context. Thus, a context may be
thought of as a process with a ``hole'' (written $\Box$) in it. The
application of a context $M$ to a process $P$, written $M[P]$, is
tantamount to filling the hole in $M$ with $P$. In this paper we do
not need the full weight of this theory, but do make use of the notion
of context in the proof the main theorem. 

\begin{mathpar}
  \inferrule* [lab=summation] {} {{M_{M},M_{N}} \bc \Box \;|\; x.M_{A} \;|\; M_{M}+M_{N}}
  \and
  \inferrule* [lab=agent] {} {{M_{A}} \bc (\vec{x})M_{P} \;| \; \clift{P_0,\ldots,M_{P},\ldots,P_N}}
  \and \\
  \inferrule* [lab=process] {} {{M_{P}} \bc M_{N} \;| \;P|M_{P} }
\end{mathpar} 

\begin{mathpar}
  \inferrule* [lab=sychronization] {} {M_{N} \bc \Box \;|\; x?M_{F} \;|\; x!M_{C}}
  \and
  \inferrule* [lab=abstraction] {} {{M_{F}} \bc (x)M_{P} }
  \and
  \inferrule* [lab=concretion] {} {{M_{C}} \bc \langle M_{P} \rangle }
  \and \\
  \inferrule* [lab=process] {} {{M_{P}} \bc M_{N} \;| \;P|M_{P} }
\end{mathpar}

\begin{definition}[contextual application] Given a context $M$, and
  process $P$, we define the \emph{contextual application}, $M[P] :=
  M\{P/\Box\}$. That is, the contextual application of M to P is the
  substitution of $P$ for $\Box$ in $M$.
\end{definition}

$\meaningof{-} : L \to \mathcal{P}(\pi)$

\begin{mathpar}
  \inferrule* [lab=collection] {} {\meaningof{true} = \pi, \and \meaningof{~E} = \pi \setminus \meaningof{E}, \and \meaningof{E_{1} \& E_{2}} = \meaningof{E_{1}} \cap \meaningof{E_{2}}}
\end{mathpar}

\begin{mathpar}
  \inferrule* [lab=structure] {} {\meaningof{0} = \{ P \in \pi | P \equiv 0 \}, \and \\ \meaningof{E_1 | E_2} = \{ P \in \pi | P \equiv P_{1} | P_{2}, P_{1} \in \meaningof{E_{1}}, P_{2} \in \meaningof{E_2}\} }
\end{mathpar}

\begin{mathpar}
 \inferrule* [lab=behavior] {} {\meaningof{\langle a?b \rangle E} = \{ P \in \pi | P \equiv Q | u?(y)P', \\ \and \\\\ \and \\ \;\;\; u \in \meaningof{a}, \forall z.P'\{z/y\} \in \meaningof{E\{z/b\}}\}, \and \\ \meaningof{a!E} = \{ P \in \pi | P \equiv Q | x!\langle P' \rangle, x \in \meaningof{a} P' \in \meaningof{E}\} }
\end{mathpar}

\begin{mathpar}
 \inferrule* [lab=nominal] {} {\meaningof{\quotep{E}} = \{ \quotep{P} \in \quotep{\pi} | P \in \meaningof{E} \}, \and \meaningof{\quotep{P}} = \{ \quotep{Q} \in \quotep{\pi} | P \equiv Q \} \and \\ \meaningof{@\quotep{E}} = \{ P \in \pi | P \equiv @x, x \in \meaningof{E} \}}
\end{mathpar}

\begin{eqnarray*}
  \\
  \meaningof{-} : TS \to ST
\end{eqnarray*}

\begin{eqnarray*}
  \\
  L : TS \to ST
\end{eqnarray*}

\begin{eqnarray*}
  \\
  P \models E \iff P \in \meaningof{E}
\end{eqnarray*}

\begin{eqnarray*}
  P \approx_{L} Q \iff \forall E \in L. P \models E \iff Q \models E
\end{eqnarray*}

\begin{eqnarray*}
  P \approx_{K} Q
\end{eqnarray*}

\begin{eqnarray*}
  P \approx Q
\end{eqnarray*}

$\approx_{K} = \approx = \approx_{L}$

\subsubsection{Contextual duality}

Note that contexts extend the quotation operation to a family of
operations from processes to names. Given a context, $M$, we can
define a \emph{nominal context}, $\quotep{M}$ by $\quotep{M}[P] :=
\quotep{M[P]}$. To foreshadow what is to come we observe that these
operations enjoy a duality with processes very much like the duality
between vectors and maps from vectors to scalars.

Further, because the calculus is essentially higher-order, we have a
correspondence between contexts and processes. More specifically,
given a name $x$ and a context $M$ we can construct $M^{*}_{x}$ such
that 

\begin{mathpar}
  M^{*}_{x} | \lift{x}{P} \red M[P]
\end{mathpar}

namely,

\begin{mathpar}
  M^{*}_{x} := x?(u).M[\dropn{u}]
\end{mathpar}

The dependence of $M^{*}_{x}$ on a name makes it an abstraction, 

\begin{mathpar}
  M^{*} := (x)x?(u).M[\dropn{u}]
\end{mathpar}

\subsection{Additional notation}

It will sometimes be convenient to denote the process a name
quotes. We already have the notation $x = \quotep{P}$, but it will be
convenient to introduce an alternate notation, $\procn{x}$, when we
want to emphasize the connection to the use of the name. Note that, by
virtue of name equivalence, $\quotep{\procn{x}} \nameeq x$; so, the
notation is consistent with previous definitions.

Further, because names have structure it is possible to effect
substitutions on the basis of that structure. This means we need to
upgrade our notation for substitutions, which we accomplish by
adapting comprehension notation. Thus,

\begin{mathpar}
  P\{ y / x : x \in S \}
\end{mathpar}

is interpreted to mean the process derived from P by replacing (in a
capture-avoiding manner) each occurrence of $x$ in $S$ by $y$. For example,

\begin{mathpar}
  P\{ \quotep{\procn{x}|\procn{x}} / x : x \in \freenames{P} \}
\end{mathpar}

will replace each (occurrence) of a free name $x$ in $P$ by
$\quotep{\procn{x}|\procn{x}}$.

Also, we will avail ourselves of the notation $x^{L}$ and $x^{R}$ to
denote injections of a name into disjoint copies of the name
space. There are numerous ways to accomplish this. One example can be
found in \cite{MeredithR05}. This notation overloads to vectors of
names: $\vec{x}^{\pi} := (x_{i}^{\pi} \; : \; 0 \leq i < |\vec{x}| )$ where $\pi \in \{L,R\}$.

We also use $P^{\Box} := P|\Box$.

In \cite{MeredithR05} an interpretation of the new operator is
given. It turns out that there are several possible interpretations
all enjoying the requisite algebraic properties of the operator (see
\cite{milner91polyadicpi}). We will therefore make liberal use of
$(\nu\; \vec{x})P$.

% subsection the_syntax_and_semantics_of_the_notation_system (end)   

\input{qm2pi.qmops} 

\input{qm2pi.sterngerlach} 

\input{qm2pi.metric} 

% section concurrent_process_calculi (end)

%\input{qm2pi.proofsketch}

% section proof sketch (end)

%\input{qm2pi.slviaknots} 

% section spatial logic via knots (end)

\input{qm2pi.conclusion}

% section conclusion (end)

%\input{qm2pi.dtcodes} 

% section wiring algorithm (end)

\input{qm2pi.ack} 

% section acknowledgments (end)

\newpage


\bibliographystyle{plain}   
\bibliography{../../biblios/main.bib}

\input{qm2pi.rhodetails}

\end{document}



% section front matter (end)

\section{Introduction}\label{sec:introduction} % (fold)
In this draft of the material i am going to have to dispense with the
usual writing conventions adopted in papers on these topics. i'm going
to have adopt whatever tone i need at the time i'm writing up the
calculations. Sometimes this may be very conversational; others it may
be the barest mathematical grunts; others still it may be that i have
lifted text from one of my other papers because the exposition of some
point was better said there. i hope that my readers are not unduly put
out by this decision. i'm not doing this to flout convention or be
rebellious. i find these calculations very technically challenging. To
keep everything going technically, something has to give; i have to
let go of some cognitive burden. So, the academic writing style --
with all of its trade-offs in terms of facilitating technical
communication -- is what i'm letting go of. Perhaps subsequent drafts
can be tightened and polished, but for now, i'm going to speak as if
we were sitting together in a coffee shop with a laptop, wifi and a
pad of paper and a pencil.

So, here's what i have to say. We -- you and i, comfortably ensconced
in our coffee shop and well-equipped with our tools -- can realize and
carry out the calculations of quantum mechanics over a very different
formal theory of dynamics, a formal theory of dynamics that
corresponds to a theory of concurrent computation with
\emph{reflection}. It has the advantage that the underlying theory is
already `quantized', but supports analogues all of the continuuous
operations. Strikingly, this underlying theory has recently been
connected with a notion of metric that we can show, by calculating
together, coincides with the metric induced by the inner product.

There are a lot of reasons why you might be interested in seeing
calculations of this form. Here's why i'm interested. For the past
several centuries there has been no competitor to the ``Newtonian''
account of dynamics. As a result the predominant share of accounts of
dynamical systems and situations have had to be formulated in terms of
the Newtonian machinery. i view this as an intellectually dangerous
position to occupy. Everything, despite it's intrinsic shape, turns
into a nail to be hit with this hammer. Recently, however, the theory
of computation has matured to the point where we have candidates for
theories of dynamics that offer very different perspective on
reasoning about dynamical systems and situations. Testing these
candidates against very successful accounts of dynamical situations,
like quantum mechanics, is going to give us some sense of how mature
they are and some measure of the quality of these accounts of
dynamics.

\subsection{Summary of contributions and outline of paper}

So, we're going to develop an interpretation of the operations of
quantum mechanics normally interpreted by Hilbert spaces and
operators. We're going to do this over a theory of computation. Note
that this is very different than the usual quantum computation program
which develops notions of computation over quantum mechanics. Rather,
we are developing a story that aligns with Wheeler's slogan: It from
Bit. To do this we will first provide an account of the theory of
computation at play here. Then we will dive into a calculation-driven
interpretation of the operations of quantum mechanics.

The reason we take this approach is that -- until very recently --
there hasn't been an axiomatic account of quantum mechanics. As a
result there has been no sharp delineation of the mathematical theory
supporting interpretation of the physical theory and the physical
theory, itself. So, ambient features of the maths are free to be
exploited (or supressed) without a real accounting of their physical
relevance. There is no sharp statement ``here's the physical theory''
qua \emph{theory} and ``here's the mathematical interpretation''
enabling a judgment of how faithful the interpretation is -- apart
from experimental observation. When there is an axiomatic account we
can judge how well a given mathematical formalism supports an
interpretation of the axioms, independent of
experimentation. Likewise, we can judge how well we have captured our
physical evidence and experience with our axiomatics, independent of
any specific mathematical implementation, with accidental detail that
may or may not have physical significance. 

In lieu of a fully fleshed out and vetted axiomatic account of quantum
mechanics, interpreting the operational notions in service of modeling
physical systems will have to suffice. In other words, we are not in
the business of providing a model of Hilbert spaces and operators. We
are in the business of providing a model of quantum mechanics because
we are motivated by testing our notions of dynamics against physical
theory; and, the predictive calculations of the physical theory must
serve as the best formulation -- shy of a fully fleshed out axiomatic
account -- of the physical theory itself (as they have for scientific
theories since time immemorial). Put another way, despite a
whole-hearted commitment to an It-from-Bit ontology, we are firmly
aligned with the shut-up-and-calculate camp as the best way to obtain
results either from the physical perspective or as a quality assurance
measure of our fledgling theory of dynamics.

In detail, we present a reflective process calculus. Then we develop
intuitive correspondences between the notions available in this
calculus and the usual physical notions supporting quantum mechanical
calculations. Thus, 

\begin{table}[htp]
  \center{
    \fbox{
      \begin{tabular}{c|c}
        quantum mechanics & process calculus \\
        \hline
        scalar & name \\
        state vector & process \\
        dual & contextual duals \\
        matrix & formal sums of process-context-dual pairs \\
        orthogonality & process annihilation \\
        inner product & execution-formula + quoting
      \end{tabular}
    }
  }
  \caption{QM - process calculi correspondences}
\end{table}

Then we tighten up these intuitions to operational definitions. We
employ the Dirac notation as the best proxy we can find for an
abstract syntax of the quantum mechanical notions. The definitions we
develop put us in contact with equational constraints coming from the
theory that we demonstrate the definitions and calculations satisfy.

This puts us in a position to shut up and calculate for the
Stern-Gerlach experimental set up, showing how these predictive
calculations become calculations on processes in our theory of a
reflective process calculus.

Penultimately, we demonstrate that the notion of metric coming from
the inner product coincides with the notion of metric available from
the theory of bisimulation. This demonstration gives us the right to
think of space as arising from behavior. Finally, we consider where we
might go from the new vantage point we have obtained.

% section introduction (end) 
 
% section introduction (end)

% \documentclass[12pt]{llncs}
%\documentclass{jktr}

\usepackage[pdftex]{hyperref}                   
\usepackage {listings}
\usepackage {mathpartir}
\usepackage{bcprules}
%\usepackage{listings}
                       
\usepackage{graphicx} 
%\usepackage[margins=2.5cm,nohead,nofoot]{geometry}
%\usepackage{geometry}
\usepackage{amsfonts}
\usepackage{amstext}
\usepackage{latexsym}
\usepackage{amssymb}
\usepackage{color}


%\include{myPreamble}
\include{qm2pi.local} 

%\ifpdf
%\usepackage[pdftex]{graphicx}
%\else
%\usepackage{graphicx}
%\fi

 % \ifpdf
%  \usepackage{pdfsync}
%  \if


%\title{Brief Article}
%\author{David F. Snyder}
%\author{L.G. Meredith}

%\address{Dept. of Math., Texas State University--San Marcos, San Marcos, TX 78666}
       
\pagestyle{empty}


\begin{document}

\lstset{language=[Objective]Caml,frame=shadowbox}

\input{qm2pi.front}

% section front matter (end)

\input{qm2pi.intro} 
 
% section introduction (end)

% \input{qm2pi.knotations} 

% section notation (end)

\input{qm2pi.process.calculi} 

% section concurrent_process_calculi_and_spatial_logics_ (end)
    
%\input{qm2pi.knots2pi} 

%\input{qm2pi.trefoil} 

%\input{qm2pi.mainthm} 

% subsection basic_interpretation (end)

%\input{qm2pi.rho.presentation} 
\subsection{The syntax and semantics of the notation system}\label{sub:the_syntax_and_semantics_of_the_notation_system} % (fold)

We now summarize a technical presentation of the calculus that
embodies our theory of dynamics. The typical presentation of such a
calculus follows the style of giving generators and relations on
them. The grammar, below, describing term constructors, freely
generates the set of processes, $\Proc$. This set is then quotiented
by a relation known as structural congruence and it is over this set
that the notion of dynamics is expressed. This presentation is
essentially that of \cite{MeredithR05} with the addition of
polyadicity and summation. For readability we have relegated some of
the technical subtleties to an appendix.

\subsubsection{Process grammar}\label{subsub:process_grammar}

\begin{mathpar}
  \inferrule* [lab=synchronization] {} {{M} \bc \pzero \;|\; x?F \;|\; x!C }
  \and
  \inferrule* [lab=abstraction] {} {{F} \bc (x)P}
  \and
  \inferrule* [lab=concretion] {} {{C} \bc \langle Q \rangle}
  \and
  \inferrule* [lab=process] {} {{P,Q} \bc M \;| \;P|Q \;|\; @{x}}
  \and
  \inferrule* [lab=name] {} {{x} \bc \quotep{P}}
\end{mathpar} 

Note that $\vec{x}$ (resp. $\vec{P}$) denotes a vector of names
(resp. processes) of length $|\vec{x}|$ (resp. $|\vec{P}|$). We adopt
the following useful abbreviations.

\begin{mathpar}
   x?(\vec{y}).P := x.(\vec{y})P \and  x\clift{\vec{P}} := x.\clift{\vec{P}}
   \and x!(y) := \lift{x}{\dropn{y}}
   \and \Pi_{i=0}^{n-1}P_i := P_0 | \ldots | P_{n-1}
\end{mathpar}

\subsubsection{Structural congruence}

\paragraph{Free and bound names and alpha-equivalence.} At the
core of structural equivalence is alpha-equivalence which identifies
process that are the same up to a change of variable. Formally, we
recognize the distinction between free and bound names. The free names
of a process, $\freenames{P}$, may be calculated recursively as
follows:

\begin{mathpar}
\freenames{\pzero} := \emptyset
  \and \\
  \freenames{x?(y).P} := \{ x \} \cup (\freenames{P} \setminus \{ y \})
  \and 
  \freenames{x!\langle P \rangle} := \{ x \} \cup \{ P \} 
  \and \\
  \freenames{P|Q} := \freenames{P} \cup \freenames{Q}
  \and \\
  \freenames{@{x}} := \{ x \}
\end{mathpar}

$\pi$
$\quotep{\pi}$

$\freenames{-} : \pi \to \mathcal{P}(\quotep{\pi})$

\begin{eqnarray*}
  \freenames{\pzero} & := & \emptyset \\
  \freenames{x?(y).P} & := & \{ x \} \cup (\freenames{P} \setminus \{ y \}) \\
  \freenames{x!\langle P \rangle} & := & \{ x \} \cup \{ P \} \\
  \freenames{P|Q} & := & \freenames{P} \cup \freenames{Q} \\
  \freenames{\dropn{x}} & := & \{ x \}
\end{eqnarray*}

The bound names of a process, $\boundnames{P}$, are those names occurring in $P$
that are not free. For example, in $x?(y).0$, the name $x$ is free, while $y$ is bound.

\begin{mathpar}
  \inferrule* [lab=monoidal-laws] {} { P|Q \equiv Q|P \and P|0 \equiv P \and P|(Q|R) \equiv (P|Q)|R }
\end{mathpar}

\begin{mathpar}
  \inferrule* [lab=alpha-equivalence] {} { (x)P \equiv (y)P\{y/x\} \and y \not\in \freenames{P} }
\end{mathpar}

\begin{definition}
Then two processes, $P,Q$, are alpha-equivalent if $P = Q\{\vec{y}/\vec{x}\}$ for
some $\vec{x} \in \boundnames{Q},\vec{y} \in \boundnames{P}$, where $Q\{\vec{y}/\vec{x}\}$
denotes the capture-avoiding substitution of $\vec{y}$ for $\vec{x}$ in $Q$.
\end{definition}

\begin{definition}
  The {\em structural congruence} \cite{SangiorgiWalker} , $\equiv$,
  between processes is the least congruence containing
  alpha-equivalence, satisfying the abelian monoid laws
  (associativity, commutativity and $\pzero$ as identity) for parallel
  composition $|$ and for summation $+$.
\end{definition}

\subsection{Name equivalence}

We take name equivalence, written $\nameeq$, to be the smallest
equivalence relation generated by the following rules.

\begin{mathpar}
\inferrule*[lab=Quote-drop]
{ }
{ \quotep{@{x}} \nameeq x }

\inferrule*[lab=Struct-equiv]
{ P \scong Q }
{ \quotep{P} \nameeq \quotep{Q} }
\end{mathpar}

The astute reader will have noticed that the mutual recursion of names
and processes imposes a mutual recursion on alpha-equivalence and
structural equivalence via name-equivalence. Fortunately, all of this
works out pleasantly and we may calculate in the natural way, free of
concern. The reader interested in the details is referred to the
appendix \ref{appendix:rho_details}.

\subsection{Substitution}

We use $\Proc$ for the set of processes, $\QProc$ for the set of
names, and $\id{\{}\vec{y} / \vec{x} \id{\}}$ to denote partial maps,
$s : \QProc \rightarrow \QProc$. A map, $s$ lifts, uniquely, to a map
on process terms, $\widehat{s} : \Proc \rightarrow \Proc$ by the
following equations.

\begin{mathpar}
  (0) \psubstp{Q}{P} := 0 \\
  (R \juxtap S) \psubstp{Q}{P}
  :=    
  (R)\psubstp{Q}{P} \juxtap (S) \psubstp{Q}{P} \\
  (x?(y).R) \psubstp{Q}{P}    
  :=    
  (x)\substp{Q}{P} (z)\concat( (R \psubstn{z}{y}) \psubstp{Q}{P} ) \\
  (\lift{x}{R}) \psubstp{Q}{P}  
  :=
  \lift{(x)\substp{Q}{P}}{ R \psubstp{Q}{P} } \\
%   (\dropn{x})  \psubstp{Q}{P}       
%   := 
%   \left\{ 
%     \begin{array}{ccc} 
%       \dropn{\quotep{Q}} & & x \nameeq \quotep{P} \\
%       \dropn{x} & & otherwise \\
%     \end{array}
%   \right. 
  (\dropn{x})  \psubstp{Q}{P}       
  := 
  \left\{ 
    \begin{array}{ccc} 
      Q & & x \nameeq \quotep{P} \\
      \dropn{x} & & otherwise \\
    \end{array}
  \right.
\end{mathpar}
 

where

\begin{eqnarray}
  (x)\id{\{} \lpquote Q \rpquote / \lpquote P \rpquote \id{\}}            = 
  \left\{ 
    \begin{array}{ccc}
      \lpquote Q \rpquote & & x \nameeq \lpquote P \rpquote \\
      x & & otherwise \\
    \end{array}
  \right. \nonumber
\end{eqnarray}

and $z$ is chosen distinct from $\quotep{P}$, $\quotep{Q}$, the free
names in $Q$, and all the names in $R$. Our $\alpha$-equivalence will
be built in the standard way from this substitution.

\begin{remark}\label{rem:no_self_referential_names}
  One consequence of these definitions is that $\forall P. \quotep{P}
  \not\in \freenames{P}$.
\end{remark}

\subsection{ Dynamic quote: an example }

Anticipating something of what's to come, consider applying the
substitution, $\widehat{\id{\{}u / z \id{\}}}$, to the following pair
of processes, $\lift{w}{y!(z)}$ and $w[ \lpquote y!(z) \rpquote ]$.

\begin{eqnarray}
	\lift{w}{y!(z)}\widehat{\id{\{}u / z \id{\}}}
		& = &
		\lift{w}{y!(u)} \nonumber\\
	w[ \lpquote y!(z) \rpquote ] \widehat{ \id{\{}u / z \id{\}} }
		& = &
		w[ \lpquote y!(z) \rpquote ] \nonumber
\end{eqnarray}

Because the body of the process between quotes is impervious to
substitution, we get radically different answers. In fact, by
examining the first process in an input context,
e.g. $x?(z).\lift{w}{y!(z)}$, we see that the process under the lift
operator may be shaped by prefixed inputs binding a name inside it. In
this sense, the lift operator will be seen as a way to dynamically
construct processes before reifying them as names.

Finally equipped with these standard features we can present the
dynamics of the calculus.

\subsubsection{Operational semantics} 

Finally, we introduce the computational dynamics. What marks these
algebras as distinct from other more traditionally studied algebraic
structures, e.g. vector spaces or polynomial rings, is the manner in
which dynamics is captured. In traditional structures, dynamics is typically
expressed through morphisms between such structures, as in linear maps
between vector spaces or morphisms between rings. In algebras
associated with the semantics of computation, the dynamics is
expressed as part of the algebraic structure itself, through a
reduction reduction relation typically denoted by $\red$. Below, we
give a recursive presentation of this relation for the calculus used
in the encoding.

$\red \subseteq \pi \times \pi$
$\red : \pi \to \mathcal{P}(\pi)$

\begin{mathpar}
  \inferrule* [lab=Comm] { \textsf{match}( x_{src}, x_{trgt} ) } { x_{trgt}?(y)P \; | \; x_{src}!\langle {Q} \rangle \red P\{\quotep{Q}/y}\} }
  \and \\
  \inferrule* [lab=Par] {{P} \red {P}'} {{{P} | {Q}} \red {{P}' | {Q}}}
  \and
  \inferrule* [lab=Equiv]{{{P} \scong {P}'} \andalso {{P}' \red {Q}'} \andalso {{Q}' \scong {Q}}}{{P} \red {Q}}
\end{mathpar}

\begin{eqnarray*}
  match_{\equiv} (\quotep{P},\quotep{Q}) & := & P \equiv Q \\
  match_{\dagger}(\quotep{P},\quotep{Q}) & := & \forall R. P|Q \red^{*} R => R \red^{*} 0 \\
  match_{K}(\quotep{P},\quotep{Q}) & := & K \mbox{ for some context } K
\end{eqnarray*}

$u?(x)P | u!\langle Q \rangle \red P\{\quotep{Q}/x\}$

%We write $\wred$ for $\red^*$, and $P\red$ if $\exists Q $ such that $ P \red Q$.
We write $P\red$ if $\exists Q $ such that $ P \red Q$ and $P\not\red$, otherwise.

\section{Replication}

As mentioned before, it is known that replication (and hence
recursion) can be implemented in a higher-order process algebra
\cite{SangiorgiWalker}. As our first example of calculation with the
machinery thus far presented we give the construction explicitly in
the {\rhoc}.

\begin{eqnarray}
	D_{x} & := & \prefix{x}{y}{(\binpar{\outputp{x}{y}}{@{y}})} \nonumber\\
	\bangp_{x}{P} & := & \binpar{{x}!\langle{\binpar{D_{x}}{P}}\rangle}{D_{x}} \nonumber
\end{eqnarray}

\begin{eqnarray}
	\bangp_{x}{P} & & \nonumber\\
	=
	& {x}!\langle{(\prefix{x}{y}{(\outputp{x}{y} | @{y})) | P}}\rangle 
	      | \prefix{x}{y}{(\outputp{x}{y} | @{y})} & \nonumber\\
	\red
	& (\outputp{x}{y} | @{y})\substn{\quotep{(\prefix{x}{y}{(@{y} | \outputp{x}{y})) | P}}}{y} & \nonumber\\
	=
	& \outputp{x}{\quotep{(\prefix{x}{y}{(\outputp{x}{y} | @{y})) | P}}}
	  | {(\prefix{x}{y}{(\outputp{x}{y} | @{y})) | P}} & \nonumber\\
	\red
	& \ldots & \nonumber\\
	\red^*
	& P | P | \ldots & \nonumber
\end{eqnarray}

Of course, this encoding, as an implementation, runs away, unfolding
$\bangp{P}$ eagerly. A lazier and more implementable replication
operator, restricted to input-guarded processes, may be obtained as follows.

\begin{eqnarray}
\bangp{\prefix{u}{v}{P}} 
	:= 
	\binpar{\lift{x}{\prefix{u}{v}{(\binpar{D(x)}{P})}}}{D(x)} \nonumber
\end{eqnarray}

\begin{remark}
  Note that the lazier definition still does not deal with summation
  or mixed summation (i.e. sums over input and output). The reader is
  invited to construct definitions of replication that deal with these
  features. 

  Further, the definitions are parameterized in a name, $x$. Can you,
  gentle reader, make a definition that eliminates this parameter and
  guarantees no accidental interaction between the replication
  machinery and the process being replicated -- i.e. no accidental
  sharing of names used by the process to get its work done and the
  name(s) used by the replication to effect copying. This latter
  revision of the definition of replication is crucial to obtaining
  the expected identity $!!P \sim !P$.
\end{remark}

\begin{remark}\label{rem:paradoxical_combinator}
  The reader familiar with the lambda calculus will have noticed the
  similarity between $D$ and the paradoxical combinator.

  [Ed. note: the existence of this seems to suggest we have to be more
  restrictive on the set of processes and names we admit if we are to
  support no-cloning.]
\end{remark}

\subsubsection{Bisimulation}

The computational dynamics gives rise to another kind of equivalence,
the equivalence of computational behavior. As previously mentioned
this is typically captured \emph{via} some form of bisimulation.

% The notion we use in this paper is weak barbed bisimulation
% \cite{milner91polyadicpi}.

The notion we use in this paper is derived from weak barbed
bisimulation \cite{milner91polyadicpi}. 

\begin{definition}
An \emph{observation relation}, $\downarrow_{\mathcal N}$, over a set
of names, $\mathcal N$, is the smallest relation satisfying the rules
below.

\infrule[Out-barb]{y \in {\mathcal N}, \; x \nameeq y}
		  {\outputp{x}{v} \downarrow_{\mathcal N} x}
\infrule[Par-barb]{\mbox{$P\downarrow_{\mathcal N} x$ or $Q\downarrow_{\mathcal N} x$}}
		  {\binpar{P}{Q} \downarrow_{\mathcal N} x}

We write $P \Downarrow_{\mathcal N} x$ if there is $Q$ such that 
$P \wred Q$ and $Q \downarrow_{\mathcal N} x$.
\end{definition}

\begin{definition}
%\label{def.bbisim}
An  ${\mathcal N}$-\emph{barbed bisimulation} over a set of names, ${\mathcal N}$, is a symmetric binary relation 
${\mathcal S}_{\mathcal N}$ between agents such that $P\rel{S}_{\mathcal N}Q$ implies:
\begin{enumerate}
\item If $P \red P'$ then $Q \wred Q'$ and $P'\rel{S}_{\mathcal N} Q'$.
\item If $P\downarrow_{\mathcal N} x$, then $Q\Downarrow_{\mathcal N} x$.
\end{enumerate}
$P$ is ${\mathcal N}$-barbed bisimilar to $Q$, written
$P \wbbisim_{\mathcal N} Q$, if $P \rel{S}_{\mathcal N} Q$ for some ${\mathcal N}$-barbed bisimulation ${\mathcal S}_{\mathcal N}$.
\end{definition}

$\mathcal{R} \subseteq \pi \times \pi$

$P \mathcal{R} Q => \forall P'. P \red P' \Rightarrow \exists Q'. Q \red Q', P' \mathcal{R} Q'$

$P \vdash x \Rightarrow Q \vdash x$

\begin{mathpar}
  \inferrule*[lab=Out-barb]{x \nameeq y}{{y}!\langle{Q}\rangle \vdash x}
  \and
  \inferrule*[lab=Par-barb]{\mbox{$P\vdash x$ or $Q\vdash x$}}{\binpar{P}{Q} \vdash x}
\end{mathpar}

\subsubsection{Contexts}

One of the principle advantages of computational calculi like the
$\pi$-calculus is a well-defined notion of context,
contextual-equivalence and a correlation between
contextual-equivalence and notions of bisimulation. The notion of
context allows the decomposition of a process into (sub-)process and
its syntactic environment, its context. Thus, a context may be
thought of as a process with a ``hole'' (written $\Box$) in it. The
application of a context $M$ to a process $P$, written $M[P]$, is
tantamount to filling the hole in $M$ with $P$. In this paper we do
not need the full weight of this theory, but do make use of the notion
of context in the proof the main theorem. 

\begin{mathpar}
  \inferrule* [lab=summation] {} {{M_{M},M_{N}} \bc \Box \;|\; x.M_{A} \;|\; M_{M}+M_{N}}
  \and
  \inferrule* [lab=agent] {} {{M_{A}} \bc (\vec{x})M_{P} \;| \; \clift{P_0,\ldots,M_{P},\ldots,P_N}}
  \and \\
  \inferrule* [lab=process] {} {{M_{P}} \bc M_{N} \;| \;P|M_{P} }
\end{mathpar} 

\begin{mathpar}
  \inferrule* [lab=sychronization] {} {M_{N} \bc \Box \;|\; x?M_{F} \;|\; x!M_{C}}
  \and
  \inferrule* [lab=abstraction] {} {{M_{F}} \bc (x)M_{P} }
  \and
  \inferrule* [lab=concretion] {} {{M_{C}} \bc \langle M_{P} \rangle }
  \and \\
  \inferrule* [lab=process] {} {{M_{P}} \bc M_{N} \;| \;P|M_{P} }
\end{mathpar}

\begin{definition}[contextual application] Given a context $M$, and
  process $P$, we define the \emph{contextual application}, $M[P] :=
  M\{P/\Box\}$. That is, the contextual application of M to P is the
  substitution of $P$ for $\Box$ in $M$.
\end{definition}

$\meaningof{-} : L \to \mathcal{P}(\pi)$

\begin{mathpar}
  \inferrule* [lab=collection] {} {\meaningof{true} = \pi, \and \meaningof{~E} = \pi \setminus \meaningof{E}, \and \meaningof{E_{1} \& E_{2}} = \meaningof{E_{1}} \cap \meaningof{E_{2}}}
\end{mathpar}

\begin{mathpar}
  \inferrule* [lab=structure] {} {\meaningof{0} = \{ P \in \pi | P \equiv 0 \}, \and \\ \meaningof{E_1 | E_2} = \{ P \in \pi | P \equiv P_{1} | P_{2}, P_{1} \in \meaningof{E_{1}}, P_{2} \in \meaningof{E_2}\} }
\end{mathpar}

\begin{mathpar}
 \inferrule* [lab=behavior] {} {\meaningof{\langle a?b \rangle E} = \{ P \in \pi | P \equiv Q | u?(y)P', \\ \and \\\\ \and \\ \;\;\; u \in \meaningof{a}, \forall z.P'\{z/y\} \in \meaningof{E\{z/b\}}\}, \and \\ \meaningof{a!E} = \{ P \in \pi | P \equiv Q | x!\langle P' \rangle, x \in \meaningof{a} P' \in \meaningof{E}\} }
\end{mathpar}

\begin{mathpar}
 \inferrule* [lab=nominal] {} {\meaningof{\quotep{E}} = \{ \quotep{P} \in \quotep{\pi} | P \in \meaningof{E} \}, \and \meaningof{\quotep{P}} = \{ \quotep{Q} \in \quotep{\pi} | P \equiv Q \} \and \\ \meaningof{@\quotep{E}} = \{ P \in \pi | P \equiv @x, x \in \meaningof{E} \}}
\end{mathpar}

\begin{eqnarray*}
  \\
  \meaningof{-} : TS \to ST
\end{eqnarray*}

\begin{eqnarray*}
  \\
  L : TS \to ST
\end{eqnarray*}

\begin{eqnarray*}
  \\
  P \models E \iff P \in \meaningof{E}
\end{eqnarray*}

\begin{eqnarray*}
  P \approx_{L} Q \iff \forall E \in L. P \models E \iff Q \models E
\end{eqnarray*}

\begin{eqnarray*}
  P \approx_{K} Q
\end{eqnarray*}

\begin{eqnarray*}
  P \approx Q
\end{eqnarray*}

$\approx_{K} = \approx = \approx_{L}$

\subsubsection{Contextual duality}

Note that contexts extend the quotation operation to a family of
operations from processes to names. Given a context, $M$, we can
define a \emph{nominal context}, $\quotep{M}$ by $\quotep{M}[P] :=
\quotep{M[P]}$. To foreshadow what is to come we observe that these
operations enjoy a duality with processes very much like the duality
between vectors and maps from vectors to scalars.

Further, because the calculus is essentially higher-order, we have a
correspondence between contexts and processes. More specifically,
given a name $x$ and a context $M$ we can construct $M^{*}_{x}$ such
that 

\begin{mathpar}
  M^{*}_{x} | \lift{x}{P} \red M[P]
\end{mathpar}

namely,

\begin{mathpar}
  M^{*}_{x} := x?(u).M[\dropn{u}]
\end{mathpar}

The dependence of $M^{*}_{x}$ on a name makes it an abstraction, 

\begin{mathpar}
  M^{*} := (x)x?(u).M[\dropn{u}]
\end{mathpar}

\subsection{Additional notation}

It will sometimes be convenient to denote the process a name
quotes. We already have the notation $x = \quotep{P}$, but it will be
convenient to introduce an alternate notation, $\procn{x}$, when we
want to emphasize the connection to the use of the name. Note that, by
virtue of name equivalence, $\quotep{\procn{x}} \nameeq x$; so, the
notation is consistent with previous definitions.

Further, because names have structure it is possible to effect
substitutions on the basis of that structure. This means we need to
upgrade our notation for substitutions, which we accomplish by
adapting comprehension notation. Thus,

\begin{mathpar}
  P\{ y / x : x \in S \}
\end{mathpar}

is interpreted to mean the process derived from P by replacing (in a
capture-avoiding manner) each occurrence of $x$ in $S$ by $y$. For example,

\begin{mathpar}
  P\{ \quotep{\procn{x}|\procn{x}} / x : x \in \freenames{P} \}
\end{mathpar}

will replace each (occurrence) of a free name $x$ in $P$ by
$\quotep{\procn{x}|\procn{x}}$.

Also, we will avail ourselves of the notation $x^{L}$ and $x^{R}$ to
denote injections of a name into disjoint copies of the name
space. There are numerous ways to accomplish this. One example can be
found in \cite{MeredithR05}. This notation overloads to vectors of
names: $\vec{x}^{\pi} := (x_{i}^{\pi} \; : \; 0 \leq i < |\vec{x}| )$ where $\pi \in \{L,R\}$.

We also use $P^{\Box} := P|\Box$.

In \cite{MeredithR05} an interpretation of the new operator is
given. It turns out that there are several possible interpretations
all enjoying the requisite algebraic properties of the operator (see
\cite{milner91polyadicpi}). We will therefore make liberal use of
$(\nu\; \vec{x})P$.

% subsection the_syntax_and_semantics_of_the_notation_system (end)   

\input{qm2pi.qmops} 

\input{qm2pi.sterngerlach} 

\input{qm2pi.metric} 

% section concurrent_process_calculi (end)

%\input{qm2pi.proofsketch}

% section proof sketch (end)

%\input{qm2pi.slviaknots} 

% section spatial logic via knots (end)

\input{qm2pi.conclusion}

% section conclusion (end)

%\input{qm2pi.dtcodes} 

% section wiring algorithm (end)

\input{qm2pi.ack} 

% section acknowledgments (end)

\newpage


\bibliographystyle{plain}   
\bibliography{../../biblios/main.bib}

\input{qm2pi.rhodetails}

\end{document}

 

% section notation (end)

\input{qm2pi.process.calculi} 

% section concurrent_process_calculi_and_spatial_logics_ (end)
    
%\documentclass[12pt]{llncs}
%\documentclass{jktr}

\usepackage[pdftex]{hyperref}                   
\usepackage {listings}
\usepackage {mathpartir}
\usepackage{bcprules}
%\usepackage{listings}
                       
\usepackage{graphicx} 
%\usepackage[margins=2.5cm,nohead,nofoot]{geometry}
%\usepackage{geometry}
\usepackage{amsfonts}
\usepackage{amstext}
\usepackage{latexsym}
\usepackage{amssymb}
\usepackage{color}


%\include{myPreamble}
\include{qm2pi.local} 

%\ifpdf
%\usepackage[pdftex]{graphicx}
%\else
%\usepackage{graphicx}
%\fi

 % \ifpdf
%  \usepackage{pdfsync}
%  \if


%\title{Brief Article}
%\author{David F. Snyder}
%\author{L.G. Meredith}

%\address{Dept. of Math., Texas State University--San Marcos, San Marcos, TX 78666}
       
\pagestyle{empty}


\begin{document}

\lstset{language=[Objective]Caml,frame=shadowbox}

\input{qm2pi.front}

% section front matter (end)

\input{qm2pi.intro} 
 
% section introduction (end)

% \input{qm2pi.knotations} 

% section notation (end)

\input{qm2pi.process.calculi} 

% section concurrent_process_calculi_and_spatial_logics_ (end)
    
%\input{qm2pi.knots2pi} 

%\input{qm2pi.trefoil} 

%\input{qm2pi.mainthm} 

% subsection basic_interpretation (end)

%\input{qm2pi.rho.presentation} 
\subsection{The syntax and semantics of the notation system}\label{sub:the_syntax_and_semantics_of_the_notation_system} % (fold)

We now summarize a technical presentation of the calculus that
embodies our theory of dynamics. The typical presentation of such a
calculus follows the style of giving generators and relations on
them. The grammar, below, describing term constructors, freely
generates the set of processes, $\Proc$. This set is then quotiented
by a relation known as structural congruence and it is over this set
that the notion of dynamics is expressed. This presentation is
essentially that of \cite{MeredithR05} with the addition of
polyadicity and summation. For readability we have relegated some of
the technical subtleties to an appendix.

\subsubsection{Process grammar}\label{subsub:process_grammar}

\begin{mathpar}
  \inferrule* [lab=synchronization] {} {{M} \bc \pzero \;|\; x?F \;|\; x!C }
  \and
  \inferrule* [lab=abstraction] {} {{F} \bc (x)P}
  \and
  \inferrule* [lab=concretion] {} {{C} \bc \langle Q \rangle}
  \and
  \inferrule* [lab=process] {} {{P,Q} \bc M \;| \;P|Q \;|\; @{x}}
  \and
  \inferrule* [lab=name] {} {{x} \bc \quotep{P}}
\end{mathpar} 

Note that $\vec{x}$ (resp. $\vec{P}$) denotes a vector of names
(resp. processes) of length $|\vec{x}|$ (resp. $|\vec{P}|$). We adopt
the following useful abbreviations.

\begin{mathpar}
   x?(\vec{y}).P := x.(\vec{y})P \and  x\clift{\vec{P}} := x.\clift{\vec{P}}
   \and x!(y) := \lift{x}{\dropn{y}}
   \and \Pi_{i=0}^{n-1}P_i := P_0 | \ldots | P_{n-1}
\end{mathpar}

\subsubsection{Structural congruence}

\paragraph{Free and bound names and alpha-equivalence.} At the
core of structural equivalence is alpha-equivalence which identifies
process that are the same up to a change of variable. Formally, we
recognize the distinction between free and bound names. The free names
of a process, $\freenames{P}$, may be calculated recursively as
follows:

\begin{mathpar}
\freenames{\pzero} := \emptyset
  \and \\
  \freenames{x?(y).P} := \{ x \} \cup (\freenames{P} \setminus \{ y \})
  \and 
  \freenames{x!\langle P \rangle} := \{ x \} \cup \{ P \} 
  \and \\
  \freenames{P|Q} := \freenames{P} \cup \freenames{Q}
  \and \\
  \freenames{@{x}} := \{ x \}
\end{mathpar}

$\pi$
$\quotep{\pi}$

$\freenames{-} : \pi \to \mathcal{P}(\quotep{\pi})$

\begin{eqnarray*}
  \freenames{\pzero} & := & \emptyset \\
  \freenames{x?(y).P} & := & \{ x \} \cup (\freenames{P} \setminus \{ y \}) \\
  \freenames{x!\langle P \rangle} & := & \{ x \} \cup \{ P \} \\
  \freenames{P|Q} & := & \freenames{P} \cup \freenames{Q} \\
  \freenames{\dropn{x}} & := & \{ x \}
\end{eqnarray*}

The bound names of a process, $\boundnames{P}$, are those names occurring in $P$
that are not free. For example, in $x?(y).0$, the name $x$ is free, while $y$ is bound.

\begin{mathpar}
  \inferrule* [lab=monoidal-laws] {} { P|Q \equiv Q|P \and P|0 \equiv P \and P|(Q|R) \equiv (P|Q)|R }
\end{mathpar}

\begin{mathpar}
  \inferrule* [lab=alpha-equivalence] {} { (x)P \equiv (y)P\{y/x\} \and y \not\in \freenames{P} }
\end{mathpar}

\begin{definition}
Then two processes, $P,Q$, are alpha-equivalent if $P = Q\{\vec{y}/\vec{x}\}$ for
some $\vec{x} \in \boundnames{Q},\vec{y} \in \boundnames{P}$, where $Q\{\vec{y}/\vec{x}\}$
denotes the capture-avoiding substitution of $\vec{y}$ for $\vec{x}$ in $Q$.
\end{definition}

\begin{definition}
  The {\em structural congruence} \cite{SangiorgiWalker} , $\equiv$,
  between processes is the least congruence containing
  alpha-equivalence, satisfying the abelian monoid laws
  (associativity, commutativity and $\pzero$ as identity) for parallel
  composition $|$ and for summation $+$.
\end{definition}

\subsection{Name equivalence}

We take name equivalence, written $\nameeq$, to be the smallest
equivalence relation generated by the following rules.

\begin{mathpar}
\inferrule*[lab=Quote-drop]
{ }
{ \quotep{@{x}} \nameeq x }

\inferrule*[lab=Struct-equiv]
{ P \scong Q }
{ \quotep{P} \nameeq \quotep{Q} }
\end{mathpar}

The astute reader will have noticed that the mutual recursion of names
and processes imposes a mutual recursion on alpha-equivalence and
structural equivalence via name-equivalence. Fortunately, all of this
works out pleasantly and we may calculate in the natural way, free of
concern. The reader interested in the details is referred to the
appendix \ref{appendix:rho_details}.

\subsection{Substitution}

We use $\Proc$ for the set of processes, $\QProc$ for the set of
names, and $\id{\{}\vec{y} / \vec{x} \id{\}}$ to denote partial maps,
$s : \QProc \rightarrow \QProc$. A map, $s$ lifts, uniquely, to a map
on process terms, $\widehat{s} : \Proc \rightarrow \Proc$ by the
following equations.

\begin{mathpar}
  (0) \psubstp{Q}{P} := 0 \\
  (R \juxtap S) \psubstp{Q}{P}
  :=    
  (R)\psubstp{Q}{P} \juxtap (S) \psubstp{Q}{P} \\
  (x?(y).R) \psubstp{Q}{P}    
  :=    
  (x)\substp{Q}{P} (z)\concat( (R \psubstn{z}{y}) \psubstp{Q}{P} ) \\
  (\lift{x}{R}) \psubstp{Q}{P}  
  :=
  \lift{(x)\substp{Q}{P}}{ R \psubstp{Q}{P} } \\
%   (\dropn{x})  \psubstp{Q}{P}       
%   := 
%   \left\{ 
%     \begin{array}{ccc} 
%       \dropn{\quotep{Q}} & & x \nameeq \quotep{P} \\
%       \dropn{x} & & otherwise \\
%     \end{array}
%   \right. 
  (\dropn{x})  \psubstp{Q}{P}       
  := 
  \left\{ 
    \begin{array}{ccc} 
      Q & & x \nameeq \quotep{P} \\
      \dropn{x} & & otherwise \\
    \end{array}
  \right.
\end{mathpar}
 

where

\begin{eqnarray}
  (x)\id{\{} \lpquote Q \rpquote / \lpquote P \rpquote \id{\}}            = 
  \left\{ 
    \begin{array}{ccc}
      \lpquote Q \rpquote & & x \nameeq \lpquote P \rpquote \\
      x & & otherwise \\
    \end{array}
  \right. \nonumber
\end{eqnarray}

and $z$ is chosen distinct from $\quotep{P}$, $\quotep{Q}$, the free
names in $Q$, and all the names in $R$. Our $\alpha$-equivalence will
be built in the standard way from this substitution.

\begin{remark}\label{rem:no_self_referential_names}
  One consequence of these definitions is that $\forall P. \quotep{P}
  \not\in \freenames{P}$.
\end{remark}

\subsection{ Dynamic quote: an example }

Anticipating something of what's to come, consider applying the
substitution, $\widehat{\id{\{}u / z \id{\}}}$, to the following pair
of processes, $\lift{w}{y!(z)}$ and $w[ \lpquote y!(z) \rpquote ]$.

\begin{eqnarray}
	\lift{w}{y!(z)}\widehat{\id{\{}u / z \id{\}}}
		& = &
		\lift{w}{y!(u)} \nonumber\\
	w[ \lpquote y!(z) \rpquote ] \widehat{ \id{\{}u / z \id{\}} }
		& = &
		w[ \lpquote y!(z) \rpquote ] \nonumber
\end{eqnarray}

Because the body of the process between quotes is impervious to
substitution, we get radically different answers. In fact, by
examining the first process in an input context,
e.g. $x?(z).\lift{w}{y!(z)}$, we see that the process under the lift
operator may be shaped by prefixed inputs binding a name inside it. In
this sense, the lift operator will be seen as a way to dynamically
construct processes before reifying them as names.

Finally equipped with these standard features we can present the
dynamics of the calculus.

\subsubsection{Operational semantics} 

Finally, we introduce the computational dynamics. What marks these
algebras as distinct from other more traditionally studied algebraic
structures, e.g. vector spaces or polynomial rings, is the manner in
which dynamics is captured. In traditional structures, dynamics is typically
expressed through morphisms between such structures, as in linear maps
between vector spaces or morphisms between rings. In algebras
associated with the semantics of computation, the dynamics is
expressed as part of the algebraic structure itself, through a
reduction reduction relation typically denoted by $\red$. Below, we
give a recursive presentation of this relation for the calculus used
in the encoding.

$\red \subseteq \pi \times \pi$
$\red : \pi \to \mathcal{P}(\pi)$

\begin{mathpar}
  \inferrule* [lab=Comm] { \textsf{match}( x_{src}, x_{trgt} ) } { x_{trgt}?(y)P \; | \; x_{src}!\langle {Q} \rangle \red P\{\quotep{Q}/y}\} }
  \and \\
  \inferrule* [lab=Par] {{P} \red {P}'} {{{P} | {Q}} \red {{P}' | {Q}}}
  \and
  \inferrule* [lab=Equiv]{{{P} \scong {P}'} \andalso {{P}' \red {Q}'} \andalso {{Q}' \scong {Q}}}{{P} \red {Q}}
\end{mathpar}

\begin{eqnarray*}
  match_{\equiv} (\quotep{P},\quotep{Q}) & := & P \equiv Q \\
  match_{\dagger}(\quotep{P},\quotep{Q}) & := & \forall R. P|Q \red^{*} R => R \red^{*} 0 \\
  match_{K}(\quotep{P},\quotep{Q}) & := & K \mbox{ for some context } K
\end{eqnarray*}

$u?(x)P | u!\langle Q \rangle \red P\{\quotep{Q}/x\}$

%We write $\wred$ for $\red^*$, and $P\red$ if $\exists Q $ such that $ P \red Q$.
We write $P\red$ if $\exists Q $ such that $ P \red Q$ and $P\not\red$, otherwise.

\section{Replication}

As mentioned before, it is known that replication (and hence
recursion) can be implemented in a higher-order process algebra
\cite{SangiorgiWalker}. As our first example of calculation with the
machinery thus far presented we give the construction explicitly in
the {\rhoc}.

\begin{eqnarray}
	D_{x} & := & \prefix{x}{y}{(\binpar{\outputp{x}{y}}{@{y}})} \nonumber\\
	\bangp_{x}{P} & := & \binpar{{x}!\langle{\binpar{D_{x}}{P}}\rangle}{D_{x}} \nonumber
\end{eqnarray}

\begin{eqnarray}
	\bangp_{x}{P} & & \nonumber\\
	=
	& {x}!\langle{(\prefix{x}{y}{(\outputp{x}{y} | @{y})) | P}}\rangle 
	      | \prefix{x}{y}{(\outputp{x}{y} | @{y})} & \nonumber\\
	\red
	& (\outputp{x}{y} | @{y})\substn{\quotep{(\prefix{x}{y}{(@{y} | \outputp{x}{y})) | P}}}{y} & \nonumber\\
	=
	& \outputp{x}{\quotep{(\prefix{x}{y}{(\outputp{x}{y} | @{y})) | P}}}
	  | {(\prefix{x}{y}{(\outputp{x}{y} | @{y})) | P}} & \nonumber\\
	\red
	& \ldots & \nonumber\\
	\red^*
	& P | P | \ldots & \nonumber
\end{eqnarray}

Of course, this encoding, as an implementation, runs away, unfolding
$\bangp{P}$ eagerly. A lazier and more implementable replication
operator, restricted to input-guarded processes, may be obtained as follows.

\begin{eqnarray}
\bangp{\prefix{u}{v}{P}} 
	:= 
	\binpar{\lift{x}{\prefix{u}{v}{(\binpar{D(x)}{P})}}}{D(x)} \nonumber
\end{eqnarray}

\begin{remark}
  Note that the lazier definition still does not deal with summation
  or mixed summation (i.e. sums over input and output). The reader is
  invited to construct definitions of replication that deal with these
  features. 

  Further, the definitions are parameterized in a name, $x$. Can you,
  gentle reader, make a definition that eliminates this parameter and
  guarantees no accidental interaction between the replication
  machinery and the process being replicated -- i.e. no accidental
  sharing of names used by the process to get its work done and the
  name(s) used by the replication to effect copying. This latter
  revision of the definition of replication is crucial to obtaining
  the expected identity $!!P \sim !P$.
\end{remark}

\begin{remark}\label{rem:paradoxical_combinator}
  The reader familiar with the lambda calculus will have noticed the
  similarity between $D$ and the paradoxical combinator.

  [Ed. note: the existence of this seems to suggest we have to be more
  restrictive on the set of processes and names we admit if we are to
  support no-cloning.]
\end{remark}

\subsubsection{Bisimulation}

The computational dynamics gives rise to another kind of equivalence,
the equivalence of computational behavior. As previously mentioned
this is typically captured \emph{via} some form of bisimulation.

% The notion we use in this paper is weak barbed bisimulation
% \cite{milner91polyadicpi}.

The notion we use in this paper is derived from weak barbed
bisimulation \cite{milner91polyadicpi}. 

\begin{definition}
An \emph{observation relation}, $\downarrow_{\mathcal N}$, over a set
of names, $\mathcal N$, is the smallest relation satisfying the rules
below.

\infrule[Out-barb]{y \in {\mathcal N}, \; x \nameeq y}
		  {\outputp{x}{v} \downarrow_{\mathcal N} x}
\infrule[Par-barb]{\mbox{$P\downarrow_{\mathcal N} x$ or $Q\downarrow_{\mathcal N} x$}}
		  {\binpar{P}{Q} \downarrow_{\mathcal N} x}

We write $P \Downarrow_{\mathcal N} x$ if there is $Q$ such that 
$P \wred Q$ and $Q \downarrow_{\mathcal N} x$.
\end{definition}

\begin{definition}
%\label{def.bbisim}
An  ${\mathcal N}$-\emph{barbed bisimulation} over a set of names, ${\mathcal N}$, is a symmetric binary relation 
${\mathcal S}_{\mathcal N}$ between agents such that $P\rel{S}_{\mathcal N}Q$ implies:
\begin{enumerate}
\item If $P \red P'$ then $Q \wred Q'$ and $P'\rel{S}_{\mathcal N} Q'$.
\item If $P\downarrow_{\mathcal N} x$, then $Q\Downarrow_{\mathcal N} x$.
\end{enumerate}
$P$ is ${\mathcal N}$-barbed bisimilar to $Q$, written
$P \wbbisim_{\mathcal N} Q$, if $P \rel{S}_{\mathcal N} Q$ for some ${\mathcal N}$-barbed bisimulation ${\mathcal S}_{\mathcal N}$.
\end{definition}

$\mathcal{R} \subseteq \pi \times \pi$

$P \mathcal{R} Q => \forall P'. P \red P' \Rightarrow \exists Q'. Q \red Q', P' \mathcal{R} Q'$

$P \vdash x \Rightarrow Q \vdash x$

\begin{mathpar}
  \inferrule*[lab=Out-barb]{x \nameeq y}{{y}!\langle{Q}\rangle \vdash x}
  \and
  \inferrule*[lab=Par-barb]{\mbox{$P\vdash x$ or $Q\vdash x$}}{\binpar{P}{Q} \vdash x}
\end{mathpar}

\subsubsection{Contexts}

One of the principle advantages of computational calculi like the
$\pi$-calculus is a well-defined notion of context,
contextual-equivalence and a correlation between
contextual-equivalence and notions of bisimulation. The notion of
context allows the decomposition of a process into (sub-)process and
its syntactic environment, its context. Thus, a context may be
thought of as a process with a ``hole'' (written $\Box$) in it. The
application of a context $M$ to a process $P$, written $M[P]$, is
tantamount to filling the hole in $M$ with $P$. In this paper we do
not need the full weight of this theory, but do make use of the notion
of context in the proof the main theorem. 

\begin{mathpar}
  \inferrule* [lab=summation] {} {{M_{M},M_{N}} \bc \Box \;|\; x.M_{A} \;|\; M_{M}+M_{N}}
  \and
  \inferrule* [lab=agent] {} {{M_{A}} \bc (\vec{x})M_{P} \;| \; \clift{P_0,\ldots,M_{P},\ldots,P_N}}
  \and \\
  \inferrule* [lab=process] {} {{M_{P}} \bc M_{N} \;| \;P|M_{P} }
\end{mathpar} 

\begin{mathpar}
  \inferrule* [lab=sychronization] {} {M_{N} \bc \Box \;|\; x?M_{F} \;|\; x!M_{C}}
  \and
  \inferrule* [lab=abstraction] {} {{M_{F}} \bc (x)M_{P} }
  \and
  \inferrule* [lab=concretion] {} {{M_{C}} \bc \langle M_{P} \rangle }
  \and \\
  \inferrule* [lab=process] {} {{M_{P}} \bc M_{N} \;| \;P|M_{P} }
\end{mathpar}

\begin{definition}[contextual application] Given a context $M$, and
  process $P$, we define the \emph{contextual application}, $M[P] :=
  M\{P/\Box\}$. That is, the contextual application of M to P is the
  substitution of $P$ for $\Box$ in $M$.
\end{definition}

$\meaningof{-} : L \to \mathcal{P}(\pi)$

\begin{mathpar}
  \inferrule* [lab=collection] {} {\meaningof{true} = \pi, \and \meaningof{~E} = \pi \setminus \meaningof{E}, \and \meaningof{E_{1} \& E_{2}} = \meaningof{E_{1}} \cap \meaningof{E_{2}}}
\end{mathpar}

\begin{mathpar}
  \inferrule* [lab=structure] {} {\meaningof{0} = \{ P \in \pi | P \equiv 0 \}, \and \\ \meaningof{E_1 | E_2} = \{ P \in \pi | P \equiv P_{1} | P_{2}, P_{1} \in \meaningof{E_{1}}, P_{2} \in \meaningof{E_2}\} }
\end{mathpar}

\begin{mathpar}
 \inferrule* [lab=behavior] {} {\meaningof{\langle a?b \rangle E} = \{ P \in \pi | P \equiv Q | u?(y)P', \\ \and \\\\ \and \\ \;\;\; u \in \meaningof{a}, \forall z.P'\{z/y\} \in \meaningof{E\{z/b\}}\}, \and \\ \meaningof{a!E} = \{ P \in \pi | P \equiv Q | x!\langle P' \rangle, x \in \meaningof{a} P' \in \meaningof{E}\} }
\end{mathpar}

\begin{mathpar}
 \inferrule* [lab=nominal] {} {\meaningof{\quotep{E}} = \{ \quotep{P} \in \quotep{\pi} | P \in \meaningof{E} \}, \and \meaningof{\quotep{P}} = \{ \quotep{Q} \in \quotep{\pi} | P \equiv Q \} \and \\ \meaningof{@\quotep{E}} = \{ P \in \pi | P \equiv @x, x \in \meaningof{E} \}}
\end{mathpar}

\begin{eqnarray*}
  \\
  \meaningof{-} : TS \to ST
\end{eqnarray*}

\begin{eqnarray*}
  \\
  L : TS \to ST
\end{eqnarray*}

\begin{eqnarray*}
  \\
  P \models E \iff P \in \meaningof{E}
\end{eqnarray*}

\begin{eqnarray*}
  P \approx_{L} Q \iff \forall E \in L. P \models E \iff Q \models E
\end{eqnarray*}

\begin{eqnarray*}
  P \approx_{K} Q
\end{eqnarray*}

\begin{eqnarray*}
  P \approx Q
\end{eqnarray*}

$\approx_{K} = \approx = \approx_{L}$

\subsubsection{Contextual duality}

Note that contexts extend the quotation operation to a family of
operations from processes to names. Given a context, $M$, we can
define a \emph{nominal context}, $\quotep{M}$ by $\quotep{M}[P] :=
\quotep{M[P]}$. To foreshadow what is to come we observe that these
operations enjoy a duality with processes very much like the duality
between vectors and maps from vectors to scalars.

Further, because the calculus is essentially higher-order, we have a
correspondence between contexts and processes. More specifically,
given a name $x$ and a context $M$ we can construct $M^{*}_{x}$ such
that 

\begin{mathpar}
  M^{*}_{x} | \lift{x}{P} \red M[P]
\end{mathpar}

namely,

\begin{mathpar}
  M^{*}_{x} := x?(u).M[\dropn{u}]
\end{mathpar}

The dependence of $M^{*}_{x}$ on a name makes it an abstraction, 

\begin{mathpar}
  M^{*} := (x)x?(u).M[\dropn{u}]
\end{mathpar}

\subsection{Additional notation}

It will sometimes be convenient to denote the process a name
quotes. We already have the notation $x = \quotep{P}$, but it will be
convenient to introduce an alternate notation, $\procn{x}$, when we
want to emphasize the connection to the use of the name. Note that, by
virtue of name equivalence, $\quotep{\procn{x}} \nameeq x$; so, the
notation is consistent with previous definitions.

Further, because names have structure it is possible to effect
substitutions on the basis of that structure. This means we need to
upgrade our notation for substitutions, which we accomplish by
adapting comprehension notation. Thus,

\begin{mathpar}
  P\{ y / x : x \in S \}
\end{mathpar}

is interpreted to mean the process derived from P by replacing (in a
capture-avoiding manner) each occurrence of $x$ in $S$ by $y$. For example,

\begin{mathpar}
  P\{ \quotep{\procn{x}|\procn{x}} / x : x \in \freenames{P} \}
\end{mathpar}

will replace each (occurrence) of a free name $x$ in $P$ by
$\quotep{\procn{x}|\procn{x}}$.

Also, we will avail ourselves of the notation $x^{L}$ and $x^{R}$ to
denote injections of a name into disjoint copies of the name
space. There are numerous ways to accomplish this. One example can be
found in \cite{MeredithR05}. This notation overloads to vectors of
names: $\vec{x}^{\pi} := (x_{i}^{\pi} \; : \; 0 \leq i < |\vec{x}| )$ where $\pi \in \{L,R\}$.

We also use $P^{\Box} := P|\Box$.

In \cite{MeredithR05} an interpretation of the new operator is
given. It turns out that there are several possible interpretations
all enjoying the requisite algebraic properties of the operator (see
\cite{milner91polyadicpi}). We will therefore make liberal use of
$(\nu\; \vec{x})P$.

% subsection the_syntax_and_semantics_of_the_notation_system (end)   

\input{qm2pi.qmops} 

\input{qm2pi.sterngerlach} 

\input{qm2pi.metric} 

% section concurrent_process_calculi (end)

%\input{qm2pi.proofsketch}

% section proof sketch (end)

%\input{qm2pi.slviaknots} 

% section spatial logic via knots (end)

\input{qm2pi.conclusion}

% section conclusion (end)

%\input{qm2pi.dtcodes} 

% section wiring algorithm (end)

\input{qm2pi.ack} 

% section acknowledgments (end)

\newpage


\bibliographystyle{plain}   
\bibliography{../../biblios/main.bib}

\input{qm2pi.rhodetails}

\end{document}

 

%\documentclass[12pt]{llncs}
%\documentclass{jktr}

\usepackage[pdftex]{hyperref}                   
\usepackage {listings}
\usepackage {mathpartir}
\usepackage{bcprules}
%\usepackage{listings}
                       
\usepackage{graphicx} 
%\usepackage[margins=2.5cm,nohead,nofoot]{geometry}
%\usepackage{geometry}
\usepackage{amsfonts}
\usepackage{amstext}
\usepackage{latexsym}
\usepackage{amssymb}
\usepackage{color}


%\include{myPreamble}
\include{qm2pi.local} 

%\ifpdf
%\usepackage[pdftex]{graphicx}
%\else
%\usepackage{graphicx}
%\fi

 % \ifpdf
%  \usepackage{pdfsync}
%  \if


%\title{Brief Article}
%\author{David F. Snyder}
%\author{L.G. Meredith}

%\address{Dept. of Math., Texas State University--San Marcos, San Marcos, TX 78666}
       
\pagestyle{empty}


\begin{document}

\lstset{language=[Objective]Caml,frame=shadowbox}

\input{qm2pi.front}

% section front matter (end)

\input{qm2pi.intro} 
 
% section introduction (end)

% \input{qm2pi.knotations} 

% section notation (end)

\input{qm2pi.process.calculi} 

% section concurrent_process_calculi_and_spatial_logics_ (end)
    
%\input{qm2pi.knots2pi} 

%\input{qm2pi.trefoil} 

%\input{qm2pi.mainthm} 

% subsection basic_interpretation (end)

%\input{qm2pi.rho.presentation} 
\subsection{The syntax and semantics of the notation system}\label{sub:the_syntax_and_semantics_of_the_notation_system} % (fold)

We now summarize a technical presentation of the calculus that
embodies our theory of dynamics. The typical presentation of such a
calculus follows the style of giving generators and relations on
them. The grammar, below, describing term constructors, freely
generates the set of processes, $\Proc$. This set is then quotiented
by a relation known as structural congruence and it is over this set
that the notion of dynamics is expressed. This presentation is
essentially that of \cite{MeredithR05} with the addition of
polyadicity and summation. For readability we have relegated some of
the technical subtleties to an appendix.

\subsubsection{Process grammar}\label{subsub:process_grammar}

\begin{mathpar}
  \inferrule* [lab=synchronization] {} {{M} \bc \pzero \;|\; x?F \;|\; x!C }
  \and
  \inferrule* [lab=abstraction] {} {{F} \bc (x)P}
  \and
  \inferrule* [lab=concretion] {} {{C} \bc \langle Q \rangle}
  \and
  \inferrule* [lab=process] {} {{P,Q} \bc M \;| \;P|Q \;|\; @{x}}
  \and
  \inferrule* [lab=name] {} {{x} \bc \quotep{P}}
\end{mathpar} 

Note that $\vec{x}$ (resp. $\vec{P}$) denotes a vector of names
(resp. processes) of length $|\vec{x}|$ (resp. $|\vec{P}|$). We adopt
the following useful abbreviations.

\begin{mathpar}
   x?(\vec{y}).P := x.(\vec{y})P \and  x\clift{\vec{P}} := x.\clift{\vec{P}}
   \and x!(y) := \lift{x}{\dropn{y}}
   \and \Pi_{i=0}^{n-1}P_i := P_0 | \ldots | P_{n-1}
\end{mathpar}

\subsubsection{Structural congruence}

\paragraph{Free and bound names and alpha-equivalence.} At the
core of structural equivalence is alpha-equivalence which identifies
process that are the same up to a change of variable. Formally, we
recognize the distinction between free and bound names. The free names
of a process, $\freenames{P}$, may be calculated recursively as
follows:

\begin{mathpar}
\freenames{\pzero} := \emptyset
  \and \\
  \freenames{x?(y).P} := \{ x \} \cup (\freenames{P} \setminus \{ y \})
  \and 
  \freenames{x!\langle P \rangle} := \{ x \} \cup \{ P \} 
  \and \\
  \freenames{P|Q} := \freenames{P} \cup \freenames{Q}
  \and \\
  \freenames{@{x}} := \{ x \}
\end{mathpar}

$\pi$
$\quotep{\pi}$

$\freenames{-} : \pi \to \mathcal{P}(\quotep{\pi})$

\begin{eqnarray*}
  \freenames{\pzero} & := & \emptyset \\
  \freenames{x?(y).P} & := & \{ x \} \cup (\freenames{P} \setminus \{ y \}) \\
  \freenames{x!\langle P \rangle} & := & \{ x \} \cup \{ P \} \\
  \freenames{P|Q} & := & \freenames{P} \cup \freenames{Q} \\
  \freenames{\dropn{x}} & := & \{ x \}
\end{eqnarray*}

The bound names of a process, $\boundnames{P}$, are those names occurring in $P$
that are not free. For example, in $x?(y).0$, the name $x$ is free, while $y$ is bound.

\begin{mathpar}
  \inferrule* [lab=monoidal-laws] {} { P|Q \equiv Q|P \and P|0 \equiv P \and P|(Q|R) \equiv (P|Q)|R }
\end{mathpar}

\begin{mathpar}
  \inferrule* [lab=alpha-equivalence] {} { (x)P \equiv (y)P\{y/x\} \and y \not\in \freenames{P} }
\end{mathpar}

\begin{definition}
Then two processes, $P,Q$, are alpha-equivalent if $P = Q\{\vec{y}/\vec{x}\}$ for
some $\vec{x} \in \boundnames{Q},\vec{y} \in \boundnames{P}$, where $Q\{\vec{y}/\vec{x}\}$
denotes the capture-avoiding substitution of $\vec{y}$ for $\vec{x}$ in $Q$.
\end{definition}

\begin{definition}
  The {\em structural congruence} \cite{SangiorgiWalker} , $\equiv$,
  between processes is the least congruence containing
  alpha-equivalence, satisfying the abelian monoid laws
  (associativity, commutativity and $\pzero$ as identity) for parallel
  composition $|$ and for summation $+$.
\end{definition}

\subsection{Name equivalence}

We take name equivalence, written $\nameeq$, to be the smallest
equivalence relation generated by the following rules.

\begin{mathpar}
\inferrule*[lab=Quote-drop]
{ }
{ \quotep{@{x}} \nameeq x }

\inferrule*[lab=Struct-equiv]
{ P \scong Q }
{ \quotep{P} \nameeq \quotep{Q} }
\end{mathpar}

The astute reader will have noticed that the mutual recursion of names
and processes imposes a mutual recursion on alpha-equivalence and
structural equivalence via name-equivalence. Fortunately, all of this
works out pleasantly and we may calculate in the natural way, free of
concern. The reader interested in the details is referred to the
appendix \ref{appendix:rho_details}.

\subsection{Substitution}

We use $\Proc$ for the set of processes, $\QProc$ for the set of
names, and $\id{\{}\vec{y} / \vec{x} \id{\}}$ to denote partial maps,
$s : \QProc \rightarrow \QProc$. A map, $s$ lifts, uniquely, to a map
on process terms, $\widehat{s} : \Proc \rightarrow \Proc$ by the
following equations.

\begin{mathpar}
  (0) \psubstp{Q}{P} := 0 \\
  (R \juxtap S) \psubstp{Q}{P}
  :=    
  (R)\psubstp{Q}{P} \juxtap (S) \psubstp{Q}{P} \\
  (x?(y).R) \psubstp{Q}{P}    
  :=    
  (x)\substp{Q}{P} (z)\concat( (R \psubstn{z}{y}) \psubstp{Q}{P} ) \\
  (\lift{x}{R}) \psubstp{Q}{P}  
  :=
  \lift{(x)\substp{Q}{P}}{ R \psubstp{Q}{P} } \\
%   (\dropn{x})  \psubstp{Q}{P}       
%   := 
%   \left\{ 
%     \begin{array}{ccc} 
%       \dropn{\quotep{Q}} & & x \nameeq \quotep{P} \\
%       \dropn{x} & & otherwise \\
%     \end{array}
%   \right. 
  (\dropn{x})  \psubstp{Q}{P}       
  := 
  \left\{ 
    \begin{array}{ccc} 
      Q & & x \nameeq \quotep{P} \\
      \dropn{x} & & otherwise \\
    \end{array}
  \right.
\end{mathpar}
 

where

\begin{eqnarray}
  (x)\id{\{} \lpquote Q \rpquote / \lpquote P \rpquote \id{\}}            = 
  \left\{ 
    \begin{array}{ccc}
      \lpquote Q \rpquote & & x \nameeq \lpquote P \rpquote \\
      x & & otherwise \\
    \end{array}
  \right. \nonumber
\end{eqnarray}

and $z$ is chosen distinct from $\quotep{P}$, $\quotep{Q}$, the free
names in $Q$, and all the names in $R$. Our $\alpha$-equivalence will
be built in the standard way from this substitution.

\begin{remark}\label{rem:no_self_referential_names}
  One consequence of these definitions is that $\forall P. \quotep{P}
  \not\in \freenames{P}$.
\end{remark}

\subsection{ Dynamic quote: an example }

Anticipating something of what's to come, consider applying the
substitution, $\widehat{\id{\{}u / z \id{\}}}$, to the following pair
of processes, $\lift{w}{y!(z)}$ and $w[ \lpquote y!(z) \rpquote ]$.

\begin{eqnarray}
	\lift{w}{y!(z)}\widehat{\id{\{}u / z \id{\}}}
		& = &
		\lift{w}{y!(u)} \nonumber\\
	w[ \lpquote y!(z) \rpquote ] \widehat{ \id{\{}u / z \id{\}} }
		& = &
		w[ \lpquote y!(z) \rpquote ] \nonumber
\end{eqnarray}

Because the body of the process between quotes is impervious to
substitution, we get radically different answers. In fact, by
examining the first process in an input context,
e.g. $x?(z).\lift{w}{y!(z)}$, we see that the process under the lift
operator may be shaped by prefixed inputs binding a name inside it. In
this sense, the lift operator will be seen as a way to dynamically
construct processes before reifying them as names.

Finally equipped with these standard features we can present the
dynamics of the calculus.

\subsubsection{Operational semantics} 

Finally, we introduce the computational dynamics. What marks these
algebras as distinct from other more traditionally studied algebraic
structures, e.g. vector spaces or polynomial rings, is the manner in
which dynamics is captured. In traditional structures, dynamics is typically
expressed through morphisms between such structures, as in linear maps
between vector spaces or morphisms between rings. In algebras
associated with the semantics of computation, the dynamics is
expressed as part of the algebraic structure itself, through a
reduction reduction relation typically denoted by $\red$. Below, we
give a recursive presentation of this relation for the calculus used
in the encoding.

$\red \subseteq \pi \times \pi$
$\red : \pi \to \mathcal{P}(\pi)$

\begin{mathpar}
  \inferrule* [lab=Comm] { \textsf{match}( x_{src}, x_{trgt} ) } { x_{trgt}?(y)P \; | \; x_{src}!\langle {Q} \rangle \red P\{\quotep{Q}/y}\} }
  \and \\
  \inferrule* [lab=Par] {{P} \red {P}'} {{{P} | {Q}} \red {{P}' | {Q}}}
  \and
  \inferrule* [lab=Equiv]{{{P} \scong {P}'} \andalso {{P}' \red {Q}'} \andalso {{Q}' \scong {Q}}}{{P} \red {Q}}
\end{mathpar}

\begin{eqnarray*}
  match_{\equiv} (\quotep{P},\quotep{Q}) & := & P \equiv Q \\
  match_{\dagger}(\quotep{P},\quotep{Q}) & := & \forall R. P|Q \red^{*} R => R \red^{*} 0 \\
  match_{K}(\quotep{P},\quotep{Q}) & := & K \mbox{ for some context } K
\end{eqnarray*}

$u?(x)P | u!\langle Q \rangle \red P\{\quotep{Q}/x\}$

%We write $\wred$ for $\red^*$, and $P\red$ if $\exists Q $ such that $ P \red Q$.
We write $P\red$ if $\exists Q $ such that $ P \red Q$ and $P\not\red$, otherwise.

\section{Replication}

As mentioned before, it is known that replication (and hence
recursion) can be implemented in a higher-order process algebra
\cite{SangiorgiWalker}. As our first example of calculation with the
machinery thus far presented we give the construction explicitly in
the {\rhoc}.

\begin{eqnarray}
	D_{x} & := & \prefix{x}{y}{(\binpar{\outputp{x}{y}}{@{y}})} \nonumber\\
	\bangp_{x}{P} & := & \binpar{{x}!\langle{\binpar{D_{x}}{P}}\rangle}{D_{x}} \nonumber
\end{eqnarray}

\begin{eqnarray}
	\bangp_{x}{P} & & \nonumber\\
	=
	& {x}!\langle{(\prefix{x}{y}{(\outputp{x}{y} | @{y})) | P}}\rangle 
	      | \prefix{x}{y}{(\outputp{x}{y} | @{y})} & \nonumber\\
	\red
	& (\outputp{x}{y} | @{y})\substn{\quotep{(\prefix{x}{y}{(@{y} | \outputp{x}{y})) | P}}}{y} & \nonumber\\
	=
	& \outputp{x}{\quotep{(\prefix{x}{y}{(\outputp{x}{y} | @{y})) | P}}}
	  | {(\prefix{x}{y}{(\outputp{x}{y} | @{y})) | P}} & \nonumber\\
	\red
	& \ldots & \nonumber\\
	\red^*
	& P | P | \ldots & \nonumber
\end{eqnarray}

Of course, this encoding, as an implementation, runs away, unfolding
$\bangp{P}$ eagerly. A lazier and more implementable replication
operator, restricted to input-guarded processes, may be obtained as follows.

\begin{eqnarray}
\bangp{\prefix{u}{v}{P}} 
	:= 
	\binpar{\lift{x}{\prefix{u}{v}{(\binpar{D(x)}{P})}}}{D(x)} \nonumber
\end{eqnarray}

\begin{remark}
  Note that the lazier definition still does not deal with summation
  or mixed summation (i.e. sums over input and output). The reader is
  invited to construct definitions of replication that deal with these
  features. 

  Further, the definitions are parameterized in a name, $x$. Can you,
  gentle reader, make a definition that eliminates this parameter and
  guarantees no accidental interaction between the replication
  machinery and the process being replicated -- i.e. no accidental
  sharing of names used by the process to get its work done and the
  name(s) used by the replication to effect copying. This latter
  revision of the definition of replication is crucial to obtaining
  the expected identity $!!P \sim !P$.
\end{remark}

\begin{remark}\label{rem:paradoxical_combinator}
  The reader familiar with the lambda calculus will have noticed the
  similarity between $D$ and the paradoxical combinator.

  [Ed. note: the existence of this seems to suggest we have to be more
  restrictive on the set of processes and names we admit if we are to
  support no-cloning.]
\end{remark}

\subsubsection{Bisimulation}

The computational dynamics gives rise to another kind of equivalence,
the equivalence of computational behavior. As previously mentioned
this is typically captured \emph{via} some form of bisimulation.

% The notion we use in this paper is weak barbed bisimulation
% \cite{milner91polyadicpi}.

The notion we use in this paper is derived from weak barbed
bisimulation \cite{milner91polyadicpi}. 

\begin{definition}
An \emph{observation relation}, $\downarrow_{\mathcal N}$, over a set
of names, $\mathcal N$, is the smallest relation satisfying the rules
below.

\infrule[Out-barb]{y \in {\mathcal N}, \; x \nameeq y}
		  {\outputp{x}{v} \downarrow_{\mathcal N} x}
\infrule[Par-barb]{\mbox{$P\downarrow_{\mathcal N} x$ or $Q\downarrow_{\mathcal N} x$}}
		  {\binpar{P}{Q} \downarrow_{\mathcal N} x}

We write $P \Downarrow_{\mathcal N} x$ if there is $Q$ such that 
$P \wred Q$ and $Q \downarrow_{\mathcal N} x$.
\end{definition}

\begin{definition}
%\label{def.bbisim}
An  ${\mathcal N}$-\emph{barbed bisimulation} over a set of names, ${\mathcal N}$, is a symmetric binary relation 
${\mathcal S}_{\mathcal N}$ between agents such that $P\rel{S}_{\mathcal N}Q$ implies:
\begin{enumerate}
\item If $P \red P'$ then $Q \wred Q'$ and $P'\rel{S}_{\mathcal N} Q'$.
\item If $P\downarrow_{\mathcal N} x$, then $Q\Downarrow_{\mathcal N} x$.
\end{enumerate}
$P$ is ${\mathcal N}$-barbed bisimilar to $Q$, written
$P \wbbisim_{\mathcal N} Q$, if $P \rel{S}_{\mathcal N} Q$ for some ${\mathcal N}$-barbed bisimulation ${\mathcal S}_{\mathcal N}$.
\end{definition}

$\mathcal{R} \subseteq \pi \times \pi$

$P \mathcal{R} Q => \forall P'. P \red P' \Rightarrow \exists Q'. Q \red Q', P' \mathcal{R} Q'$

$P \vdash x \Rightarrow Q \vdash x$

\begin{mathpar}
  \inferrule*[lab=Out-barb]{x \nameeq y}{{y}!\langle{Q}\rangle \vdash x}
  \and
  \inferrule*[lab=Par-barb]{\mbox{$P\vdash x$ or $Q\vdash x$}}{\binpar{P}{Q} \vdash x}
\end{mathpar}

\subsubsection{Contexts}

One of the principle advantages of computational calculi like the
$\pi$-calculus is a well-defined notion of context,
contextual-equivalence and a correlation between
contextual-equivalence and notions of bisimulation. The notion of
context allows the decomposition of a process into (sub-)process and
its syntactic environment, its context. Thus, a context may be
thought of as a process with a ``hole'' (written $\Box$) in it. The
application of a context $M$ to a process $P$, written $M[P]$, is
tantamount to filling the hole in $M$ with $P$. In this paper we do
not need the full weight of this theory, but do make use of the notion
of context in the proof the main theorem. 

\begin{mathpar}
  \inferrule* [lab=summation] {} {{M_{M},M_{N}} \bc \Box \;|\; x.M_{A} \;|\; M_{M}+M_{N}}
  \and
  \inferrule* [lab=agent] {} {{M_{A}} \bc (\vec{x})M_{P} \;| \; \clift{P_0,\ldots,M_{P},\ldots,P_N}}
  \and \\
  \inferrule* [lab=process] {} {{M_{P}} \bc M_{N} \;| \;P|M_{P} }
\end{mathpar} 

\begin{mathpar}
  \inferrule* [lab=sychronization] {} {M_{N} \bc \Box \;|\; x?M_{F} \;|\; x!M_{C}}
  \and
  \inferrule* [lab=abstraction] {} {{M_{F}} \bc (x)M_{P} }
  \and
  \inferrule* [lab=concretion] {} {{M_{C}} \bc \langle M_{P} \rangle }
  \and \\
  \inferrule* [lab=process] {} {{M_{P}} \bc M_{N} \;| \;P|M_{P} }
\end{mathpar}

\begin{definition}[contextual application] Given a context $M$, and
  process $P$, we define the \emph{contextual application}, $M[P] :=
  M\{P/\Box\}$. That is, the contextual application of M to P is the
  substitution of $P$ for $\Box$ in $M$.
\end{definition}

$\meaningof{-} : L \to \mathcal{P}(\pi)$

\begin{mathpar}
  \inferrule* [lab=collection] {} {\meaningof{true} = \pi, \and \meaningof{~E} = \pi \setminus \meaningof{E}, \and \meaningof{E_{1} \& E_{2}} = \meaningof{E_{1}} \cap \meaningof{E_{2}}}
\end{mathpar}

\begin{mathpar}
  \inferrule* [lab=structure] {} {\meaningof{0} = \{ P \in \pi | P \equiv 0 \}, \and \\ \meaningof{E_1 | E_2} = \{ P \in \pi | P \equiv P_{1} | P_{2}, P_{1} \in \meaningof{E_{1}}, P_{2} \in \meaningof{E_2}\} }
\end{mathpar}

\begin{mathpar}
 \inferrule* [lab=behavior] {} {\meaningof{\langle a?b \rangle E} = \{ P \in \pi | P \equiv Q | u?(y)P', \\ \and \\\\ \and \\ \;\;\; u \in \meaningof{a}, \forall z.P'\{z/y\} \in \meaningof{E\{z/b\}}\}, \and \\ \meaningof{a!E} = \{ P \in \pi | P \equiv Q | x!\langle P' \rangle, x \in \meaningof{a} P' \in \meaningof{E}\} }
\end{mathpar}

\begin{mathpar}
 \inferrule* [lab=nominal] {} {\meaningof{\quotep{E}} = \{ \quotep{P} \in \quotep{\pi} | P \in \meaningof{E} \}, \and \meaningof{\quotep{P}} = \{ \quotep{Q} \in \quotep{\pi} | P \equiv Q \} \and \\ \meaningof{@\quotep{E}} = \{ P \in \pi | P \equiv @x, x \in \meaningof{E} \}}
\end{mathpar}

\begin{eqnarray*}
  \\
  \meaningof{-} : TS \to ST
\end{eqnarray*}

\begin{eqnarray*}
  \\
  L : TS \to ST
\end{eqnarray*}

\begin{eqnarray*}
  \\
  P \models E \iff P \in \meaningof{E}
\end{eqnarray*}

\begin{eqnarray*}
  P \approx_{L} Q \iff \forall E \in L. P \models E \iff Q \models E
\end{eqnarray*}

\begin{eqnarray*}
  P \approx_{K} Q
\end{eqnarray*}

\begin{eqnarray*}
  P \approx Q
\end{eqnarray*}

$\approx_{K} = \approx = \approx_{L}$

\subsubsection{Contextual duality}

Note that contexts extend the quotation operation to a family of
operations from processes to names. Given a context, $M$, we can
define a \emph{nominal context}, $\quotep{M}$ by $\quotep{M}[P] :=
\quotep{M[P]}$. To foreshadow what is to come we observe that these
operations enjoy a duality with processes very much like the duality
between vectors and maps from vectors to scalars.

Further, because the calculus is essentially higher-order, we have a
correspondence between contexts and processes. More specifically,
given a name $x$ and a context $M$ we can construct $M^{*}_{x}$ such
that 

\begin{mathpar}
  M^{*}_{x} | \lift{x}{P} \red M[P]
\end{mathpar}

namely,

\begin{mathpar}
  M^{*}_{x} := x?(u).M[\dropn{u}]
\end{mathpar}

The dependence of $M^{*}_{x}$ on a name makes it an abstraction, 

\begin{mathpar}
  M^{*} := (x)x?(u).M[\dropn{u}]
\end{mathpar}

\subsection{Additional notation}

It will sometimes be convenient to denote the process a name
quotes. We already have the notation $x = \quotep{P}$, but it will be
convenient to introduce an alternate notation, $\procn{x}$, when we
want to emphasize the connection to the use of the name. Note that, by
virtue of name equivalence, $\quotep{\procn{x}} \nameeq x$; so, the
notation is consistent with previous definitions.

Further, because names have structure it is possible to effect
substitutions on the basis of that structure. This means we need to
upgrade our notation for substitutions, which we accomplish by
adapting comprehension notation. Thus,

\begin{mathpar}
  P\{ y / x : x \in S \}
\end{mathpar}

is interpreted to mean the process derived from P by replacing (in a
capture-avoiding manner) each occurrence of $x$ in $S$ by $y$. For example,

\begin{mathpar}
  P\{ \quotep{\procn{x}|\procn{x}} / x : x \in \freenames{P} \}
\end{mathpar}

will replace each (occurrence) of a free name $x$ in $P$ by
$\quotep{\procn{x}|\procn{x}}$.

Also, we will avail ourselves of the notation $x^{L}$ and $x^{R}$ to
denote injections of a name into disjoint copies of the name
space. There are numerous ways to accomplish this. One example can be
found in \cite{MeredithR05}. This notation overloads to vectors of
names: $\vec{x}^{\pi} := (x_{i}^{\pi} \; : \; 0 \leq i < |\vec{x}| )$ where $\pi \in \{L,R\}$.

We also use $P^{\Box} := P|\Box$.

In \cite{MeredithR05} an interpretation of the new operator is
given. It turns out that there are several possible interpretations
all enjoying the requisite algebraic properties of the operator (see
\cite{milner91polyadicpi}). We will therefore make liberal use of
$(\nu\; \vec{x})P$.

% subsection the_syntax_and_semantics_of_the_notation_system (end)   

\input{qm2pi.qmops} 

\input{qm2pi.sterngerlach} 

\input{qm2pi.metric} 

% section concurrent_process_calculi (end)

%\input{qm2pi.proofsketch}

% section proof sketch (end)

%\input{qm2pi.slviaknots} 

% section spatial logic via knots (end)

\input{qm2pi.conclusion}

% section conclusion (end)

%\input{qm2pi.dtcodes} 

% section wiring algorithm (end)

\input{qm2pi.ack} 

% section acknowledgments (end)

\newpage


\bibliographystyle{plain}   
\bibliography{../../biblios/main.bib}

\input{qm2pi.rhodetails}

\end{document}

 

%\documentclass[12pt]{llncs}
%\documentclass{jktr}

\usepackage[pdftex]{hyperref}                   
\usepackage {listings}
\usepackage {mathpartir}
\usepackage{bcprules}
%\usepackage{listings}
                       
\usepackage{graphicx} 
%\usepackage[margins=2.5cm,nohead,nofoot]{geometry}
%\usepackage{geometry}
\usepackage{amsfonts}
\usepackage{amstext}
\usepackage{latexsym}
\usepackage{amssymb}
\usepackage{color}


%\include{myPreamble}
\include{qm2pi.local} 

%\ifpdf
%\usepackage[pdftex]{graphicx}
%\else
%\usepackage{graphicx}
%\fi

 % \ifpdf
%  \usepackage{pdfsync}
%  \if


%\title{Brief Article}
%\author{David F. Snyder}
%\author{L.G. Meredith}

%\address{Dept. of Math., Texas State University--San Marcos, San Marcos, TX 78666}
       
\pagestyle{empty}


\begin{document}

\lstset{language=[Objective]Caml,frame=shadowbox}

\input{qm2pi.front}

% section front matter (end)

\input{qm2pi.intro} 
 
% section introduction (end)

% \input{qm2pi.knotations} 

% section notation (end)

\input{qm2pi.process.calculi} 

% section concurrent_process_calculi_and_spatial_logics_ (end)
    
%\input{qm2pi.knots2pi} 

%\input{qm2pi.trefoil} 

%\input{qm2pi.mainthm} 

% subsection basic_interpretation (end)

%\input{qm2pi.rho.presentation} 
\subsection{The syntax and semantics of the notation system}\label{sub:the_syntax_and_semantics_of_the_notation_system} % (fold)

We now summarize a technical presentation of the calculus that
embodies our theory of dynamics. The typical presentation of such a
calculus follows the style of giving generators and relations on
them. The grammar, below, describing term constructors, freely
generates the set of processes, $\Proc$. This set is then quotiented
by a relation known as structural congruence and it is over this set
that the notion of dynamics is expressed. This presentation is
essentially that of \cite{MeredithR05} with the addition of
polyadicity and summation. For readability we have relegated some of
the technical subtleties to an appendix.

\subsubsection{Process grammar}\label{subsub:process_grammar}

\begin{mathpar}
  \inferrule* [lab=synchronization] {} {{M} \bc \pzero \;|\; x?F \;|\; x!C }
  \and
  \inferrule* [lab=abstraction] {} {{F} \bc (x)P}
  \and
  \inferrule* [lab=concretion] {} {{C} \bc \langle Q \rangle}
  \and
  \inferrule* [lab=process] {} {{P,Q} \bc M \;| \;P|Q \;|\; @{x}}
  \and
  \inferrule* [lab=name] {} {{x} \bc \quotep{P}}
\end{mathpar} 

Note that $\vec{x}$ (resp. $\vec{P}$) denotes a vector of names
(resp. processes) of length $|\vec{x}|$ (resp. $|\vec{P}|$). We adopt
the following useful abbreviations.

\begin{mathpar}
   x?(\vec{y}).P := x.(\vec{y})P \and  x\clift{\vec{P}} := x.\clift{\vec{P}}
   \and x!(y) := \lift{x}{\dropn{y}}
   \and \Pi_{i=0}^{n-1}P_i := P_0 | \ldots | P_{n-1}
\end{mathpar}

\subsubsection{Structural congruence}

\paragraph{Free and bound names and alpha-equivalence.} At the
core of structural equivalence is alpha-equivalence which identifies
process that are the same up to a change of variable. Formally, we
recognize the distinction between free and bound names. The free names
of a process, $\freenames{P}$, may be calculated recursively as
follows:

\begin{mathpar}
\freenames{\pzero} := \emptyset
  \and \\
  \freenames{x?(y).P} := \{ x \} \cup (\freenames{P} \setminus \{ y \})
  \and 
  \freenames{x!\langle P \rangle} := \{ x \} \cup \{ P \} 
  \and \\
  \freenames{P|Q} := \freenames{P} \cup \freenames{Q}
  \and \\
  \freenames{@{x}} := \{ x \}
\end{mathpar}

$\pi$
$\quotep{\pi}$

$\freenames{-} : \pi \to \mathcal{P}(\quotep{\pi})$

\begin{eqnarray*}
  \freenames{\pzero} & := & \emptyset \\
  \freenames{x?(y).P} & := & \{ x \} \cup (\freenames{P} \setminus \{ y \}) \\
  \freenames{x!\langle P \rangle} & := & \{ x \} \cup \{ P \} \\
  \freenames{P|Q} & := & \freenames{P} \cup \freenames{Q} \\
  \freenames{\dropn{x}} & := & \{ x \}
\end{eqnarray*}

The bound names of a process, $\boundnames{P}$, are those names occurring in $P$
that are not free. For example, in $x?(y).0$, the name $x$ is free, while $y$ is bound.

\begin{mathpar}
  \inferrule* [lab=monoidal-laws] {} { P|Q \equiv Q|P \and P|0 \equiv P \and P|(Q|R) \equiv (P|Q)|R }
\end{mathpar}

\begin{mathpar}
  \inferrule* [lab=alpha-equivalence] {} { (x)P \equiv (y)P\{y/x\} \and y \not\in \freenames{P} }
\end{mathpar}

\begin{definition}
Then two processes, $P,Q$, are alpha-equivalent if $P = Q\{\vec{y}/\vec{x}\}$ for
some $\vec{x} \in \boundnames{Q},\vec{y} \in \boundnames{P}$, where $Q\{\vec{y}/\vec{x}\}$
denotes the capture-avoiding substitution of $\vec{y}$ for $\vec{x}$ in $Q$.
\end{definition}

\begin{definition}
  The {\em structural congruence} \cite{SangiorgiWalker} , $\equiv$,
  between processes is the least congruence containing
  alpha-equivalence, satisfying the abelian monoid laws
  (associativity, commutativity and $\pzero$ as identity) for parallel
  composition $|$ and for summation $+$.
\end{definition}

\subsection{Name equivalence}

We take name equivalence, written $\nameeq$, to be the smallest
equivalence relation generated by the following rules.

\begin{mathpar}
\inferrule*[lab=Quote-drop]
{ }
{ \quotep{@{x}} \nameeq x }

\inferrule*[lab=Struct-equiv]
{ P \scong Q }
{ \quotep{P} \nameeq \quotep{Q} }
\end{mathpar}

The astute reader will have noticed that the mutual recursion of names
and processes imposes a mutual recursion on alpha-equivalence and
structural equivalence via name-equivalence. Fortunately, all of this
works out pleasantly and we may calculate in the natural way, free of
concern. The reader interested in the details is referred to the
appendix \ref{appendix:rho_details}.

\subsection{Substitution}

We use $\Proc$ for the set of processes, $\QProc$ for the set of
names, and $\id{\{}\vec{y} / \vec{x} \id{\}}$ to denote partial maps,
$s : \QProc \rightarrow \QProc$. A map, $s$ lifts, uniquely, to a map
on process terms, $\widehat{s} : \Proc \rightarrow \Proc$ by the
following equations.

\begin{mathpar}
  (0) \psubstp{Q}{P} := 0 \\
  (R \juxtap S) \psubstp{Q}{P}
  :=    
  (R)\psubstp{Q}{P} \juxtap (S) \psubstp{Q}{P} \\
  (x?(y).R) \psubstp{Q}{P}    
  :=    
  (x)\substp{Q}{P} (z)\concat( (R \psubstn{z}{y}) \psubstp{Q}{P} ) \\
  (\lift{x}{R}) \psubstp{Q}{P}  
  :=
  \lift{(x)\substp{Q}{P}}{ R \psubstp{Q}{P} } \\
%   (\dropn{x})  \psubstp{Q}{P}       
%   := 
%   \left\{ 
%     \begin{array}{ccc} 
%       \dropn{\quotep{Q}} & & x \nameeq \quotep{P} \\
%       \dropn{x} & & otherwise \\
%     \end{array}
%   \right. 
  (\dropn{x})  \psubstp{Q}{P}       
  := 
  \left\{ 
    \begin{array}{ccc} 
      Q & & x \nameeq \quotep{P} \\
      \dropn{x} & & otherwise \\
    \end{array}
  \right.
\end{mathpar}
 

where

\begin{eqnarray}
  (x)\id{\{} \lpquote Q \rpquote / \lpquote P \rpquote \id{\}}            = 
  \left\{ 
    \begin{array}{ccc}
      \lpquote Q \rpquote & & x \nameeq \lpquote P \rpquote \\
      x & & otherwise \\
    \end{array}
  \right. \nonumber
\end{eqnarray}

and $z$ is chosen distinct from $\quotep{P}$, $\quotep{Q}$, the free
names in $Q$, and all the names in $R$. Our $\alpha$-equivalence will
be built in the standard way from this substitution.

\begin{remark}\label{rem:no_self_referential_names}
  One consequence of these definitions is that $\forall P. \quotep{P}
  \not\in \freenames{P}$.
\end{remark}

\subsection{ Dynamic quote: an example }

Anticipating something of what's to come, consider applying the
substitution, $\widehat{\id{\{}u / z \id{\}}}$, to the following pair
of processes, $\lift{w}{y!(z)}$ and $w[ \lpquote y!(z) \rpquote ]$.

\begin{eqnarray}
	\lift{w}{y!(z)}\widehat{\id{\{}u / z \id{\}}}
		& = &
		\lift{w}{y!(u)} \nonumber\\
	w[ \lpquote y!(z) \rpquote ] \widehat{ \id{\{}u / z \id{\}} }
		& = &
		w[ \lpquote y!(z) \rpquote ] \nonumber
\end{eqnarray}

Because the body of the process between quotes is impervious to
substitution, we get radically different answers. In fact, by
examining the first process in an input context,
e.g. $x?(z).\lift{w}{y!(z)}$, we see that the process under the lift
operator may be shaped by prefixed inputs binding a name inside it. In
this sense, the lift operator will be seen as a way to dynamically
construct processes before reifying them as names.

Finally equipped with these standard features we can present the
dynamics of the calculus.

\subsubsection{Operational semantics} 

Finally, we introduce the computational dynamics. What marks these
algebras as distinct from other more traditionally studied algebraic
structures, e.g. vector spaces or polynomial rings, is the manner in
which dynamics is captured. In traditional structures, dynamics is typically
expressed through morphisms between such structures, as in linear maps
between vector spaces or morphisms between rings. In algebras
associated with the semantics of computation, the dynamics is
expressed as part of the algebraic structure itself, through a
reduction reduction relation typically denoted by $\red$. Below, we
give a recursive presentation of this relation for the calculus used
in the encoding.

$\red \subseteq \pi \times \pi$
$\red : \pi \to \mathcal{P}(\pi)$

\begin{mathpar}
  \inferrule* [lab=Comm] { \textsf{match}( x_{src}, x_{trgt} ) } { x_{trgt}?(y)P \; | \; x_{src}!\langle {Q} \rangle \red P\{\quotep{Q}/y}\} }
  \and \\
  \inferrule* [lab=Par] {{P} \red {P}'} {{{P} | {Q}} \red {{P}' | {Q}}}
  \and
  \inferrule* [lab=Equiv]{{{P} \scong {P}'} \andalso {{P}' \red {Q}'} \andalso {{Q}' \scong {Q}}}{{P} \red {Q}}
\end{mathpar}

\begin{eqnarray*}
  match_{\equiv} (\quotep{P},\quotep{Q}) & := & P \equiv Q \\
  match_{\dagger}(\quotep{P},\quotep{Q}) & := & \forall R. P|Q \red^{*} R => R \red^{*} 0 \\
  match_{K}(\quotep{P},\quotep{Q}) & := & K \mbox{ for some context } K
\end{eqnarray*}

$u?(x)P | u!\langle Q \rangle \red P\{\quotep{Q}/x\}$

%We write $\wred$ for $\red^*$, and $P\red$ if $\exists Q $ such that $ P \red Q$.
We write $P\red$ if $\exists Q $ such that $ P \red Q$ and $P\not\red$, otherwise.

\section{Replication}

As mentioned before, it is known that replication (and hence
recursion) can be implemented in a higher-order process algebra
\cite{SangiorgiWalker}. As our first example of calculation with the
machinery thus far presented we give the construction explicitly in
the {\rhoc}.

\begin{eqnarray}
	D_{x} & := & \prefix{x}{y}{(\binpar{\outputp{x}{y}}{@{y}})} \nonumber\\
	\bangp_{x}{P} & := & \binpar{{x}!\langle{\binpar{D_{x}}{P}}\rangle}{D_{x}} \nonumber
\end{eqnarray}

\begin{eqnarray}
	\bangp_{x}{P} & & \nonumber\\
	=
	& {x}!\langle{(\prefix{x}{y}{(\outputp{x}{y} | @{y})) | P}}\rangle 
	      | \prefix{x}{y}{(\outputp{x}{y} | @{y})} & \nonumber\\
	\red
	& (\outputp{x}{y} | @{y})\substn{\quotep{(\prefix{x}{y}{(@{y} | \outputp{x}{y})) | P}}}{y} & \nonumber\\
	=
	& \outputp{x}{\quotep{(\prefix{x}{y}{(\outputp{x}{y} | @{y})) | P}}}
	  | {(\prefix{x}{y}{(\outputp{x}{y} | @{y})) | P}} & \nonumber\\
	\red
	& \ldots & \nonumber\\
	\red^*
	& P | P | \ldots & \nonumber
\end{eqnarray}

Of course, this encoding, as an implementation, runs away, unfolding
$\bangp{P}$ eagerly. A lazier and more implementable replication
operator, restricted to input-guarded processes, may be obtained as follows.

\begin{eqnarray}
\bangp{\prefix{u}{v}{P}} 
	:= 
	\binpar{\lift{x}{\prefix{u}{v}{(\binpar{D(x)}{P})}}}{D(x)} \nonumber
\end{eqnarray}

\begin{remark}
  Note that the lazier definition still does not deal with summation
  or mixed summation (i.e. sums over input and output). The reader is
  invited to construct definitions of replication that deal with these
  features. 

  Further, the definitions are parameterized in a name, $x$. Can you,
  gentle reader, make a definition that eliminates this parameter and
  guarantees no accidental interaction between the replication
  machinery and the process being replicated -- i.e. no accidental
  sharing of names used by the process to get its work done and the
  name(s) used by the replication to effect copying. This latter
  revision of the definition of replication is crucial to obtaining
  the expected identity $!!P \sim !P$.
\end{remark}

\begin{remark}\label{rem:paradoxical_combinator}
  The reader familiar with the lambda calculus will have noticed the
  similarity between $D$ and the paradoxical combinator.

  [Ed. note: the existence of this seems to suggest we have to be more
  restrictive on the set of processes and names we admit if we are to
  support no-cloning.]
\end{remark}

\subsubsection{Bisimulation}

The computational dynamics gives rise to another kind of equivalence,
the equivalence of computational behavior. As previously mentioned
this is typically captured \emph{via} some form of bisimulation.

% The notion we use in this paper is weak barbed bisimulation
% \cite{milner91polyadicpi}.

The notion we use in this paper is derived from weak barbed
bisimulation \cite{milner91polyadicpi}. 

\begin{definition}
An \emph{observation relation}, $\downarrow_{\mathcal N}$, over a set
of names, $\mathcal N$, is the smallest relation satisfying the rules
below.

\infrule[Out-barb]{y \in {\mathcal N}, \; x \nameeq y}
		  {\outputp{x}{v} \downarrow_{\mathcal N} x}
\infrule[Par-barb]{\mbox{$P\downarrow_{\mathcal N} x$ or $Q\downarrow_{\mathcal N} x$}}
		  {\binpar{P}{Q} \downarrow_{\mathcal N} x}

We write $P \Downarrow_{\mathcal N} x$ if there is $Q$ such that 
$P \wred Q$ and $Q \downarrow_{\mathcal N} x$.
\end{definition}

\begin{definition}
%\label{def.bbisim}
An  ${\mathcal N}$-\emph{barbed bisimulation} over a set of names, ${\mathcal N}$, is a symmetric binary relation 
${\mathcal S}_{\mathcal N}$ between agents such that $P\rel{S}_{\mathcal N}Q$ implies:
\begin{enumerate}
\item If $P \red P'$ then $Q \wred Q'$ and $P'\rel{S}_{\mathcal N} Q'$.
\item If $P\downarrow_{\mathcal N} x$, then $Q\Downarrow_{\mathcal N} x$.
\end{enumerate}
$P$ is ${\mathcal N}$-barbed bisimilar to $Q$, written
$P \wbbisim_{\mathcal N} Q$, if $P \rel{S}_{\mathcal N} Q$ for some ${\mathcal N}$-barbed bisimulation ${\mathcal S}_{\mathcal N}$.
\end{definition}

$\mathcal{R} \subseteq \pi \times \pi$

$P \mathcal{R} Q => \forall P'. P \red P' \Rightarrow \exists Q'. Q \red Q', P' \mathcal{R} Q'$

$P \vdash x \Rightarrow Q \vdash x$

\begin{mathpar}
  \inferrule*[lab=Out-barb]{x \nameeq y}{{y}!\langle{Q}\rangle \vdash x}
  \and
  \inferrule*[lab=Par-barb]{\mbox{$P\vdash x$ or $Q\vdash x$}}{\binpar{P}{Q} \vdash x}
\end{mathpar}

\subsubsection{Contexts}

One of the principle advantages of computational calculi like the
$\pi$-calculus is a well-defined notion of context,
contextual-equivalence and a correlation between
contextual-equivalence and notions of bisimulation. The notion of
context allows the decomposition of a process into (sub-)process and
its syntactic environment, its context. Thus, a context may be
thought of as a process with a ``hole'' (written $\Box$) in it. The
application of a context $M$ to a process $P$, written $M[P]$, is
tantamount to filling the hole in $M$ with $P$. In this paper we do
not need the full weight of this theory, but do make use of the notion
of context in the proof the main theorem. 

\begin{mathpar}
  \inferrule* [lab=summation] {} {{M_{M},M_{N}} \bc \Box \;|\; x.M_{A} \;|\; M_{M}+M_{N}}
  \and
  \inferrule* [lab=agent] {} {{M_{A}} \bc (\vec{x})M_{P} \;| \; \clift{P_0,\ldots,M_{P},\ldots,P_N}}
  \and \\
  \inferrule* [lab=process] {} {{M_{P}} \bc M_{N} \;| \;P|M_{P} }
\end{mathpar} 

\begin{mathpar}
  \inferrule* [lab=sychronization] {} {M_{N} \bc \Box \;|\; x?M_{F} \;|\; x!M_{C}}
  \and
  \inferrule* [lab=abstraction] {} {{M_{F}} \bc (x)M_{P} }
  \and
  \inferrule* [lab=concretion] {} {{M_{C}} \bc \langle M_{P} \rangle }
  \and \\
  \inferrule* [lab=process] {} {{M_{P}} \bc M_{N} \;| \;P|M_{P} }
\end{mathpar}

\begin{definition}[contextual application] Given a context $M$, and
  process $P$, we define the \emph{contextual application}, $M[P] :=
  M\{P/\Box\}$. That is, the contextual application of M to P is the
  substitution of $P$ for $\Box$ in $M$.
\end{definition}

$\meaningof{-} : L \to \mathcal{P}(\pi)$

\begin{mathpar}
  \inferrule* [lab=collection] {} {\meaningof{true} = \pi, \and \meaningof{~E} = \pi \setminus \meaningof{E}, \and \meaningof{E_{1} \& E_{2}} = \meaningof{E_{1}} \cap \meaningof{E_{2}}}
\end{mathpar}

\begin{mathpar}
  \inferrule* [lab=structure] {} {\meaningof{0} = \{ P \in \pi | P \equiv 0 \}, \and \\ \meaningof{E_1 | E_2} = \{ P \in \pi | P \equiv P_{1} | P_{2}, P_{1} \in \meaningof{E_{1}}, P_{2} \in \meaningof{E_2}\} }
\end{mathpar}

\begin{mathpar}
 \inferrule* [lab=behavior] {} {\meaningof{\langle a?b \rangle E} = \{ P \in \pi | P \equiv Q | u?(y)P', \\ \and \\\\ \and \\ \;\;\; u \in \meaningof{a}, \forall z.P'\{z/y\} \in \meaningof{E\{z/b\}}\}, \and \\ \meaningof{a!E} = \{ P \in \pi | P \equiv Q | x!\langle P' \rangle, x \in \meaningof{a} P' \in \meaningof{E}\} }
\end{mathpar}

\begin{mathpar}
 \inferrule* [lab=nominal] {} {\meaningof{\quotep{E}} = \{ \quotep{P} \in \quotep{\pi} | P \in \meaningof{E} \}, \and \meaningof{\quotep{P}} = \{ \quotep{Q} \in \quotep{\pi} | P \equiv Q \} \and \\ \meaningof{@\quotep{E}} = \{ P \in \pi | P \equiv @x, x \in \meaningof{E} \}}
\end{mathpar}

\begin{eqnarray*}
  \\
  \meaningof{-} : TS \to ST
\end{eqnarray*}

\begin{eqnarray*}
  \\
  L : TS \to ST
\end{eqnarray*}

\begin{eqnarray*}
  \\
  P \models E \iff P \in \meaningof{E}
\end{eqnarray*}

\begin{eqnarray*}
  P \approx_{L} Q \iff \forall E \in L. P \models E \iff Q \models E
\end{eqnarray*}

\begin{eqnarray*}
  P \approx_{K} Q
\end{eqnarray*}

\begin{eqnarray*}
  P \approx Q
\end{eqnarray*}

$\approx_{K} = \approx = \approx_{L}$

\subsubsection{Contextual duality}

Note that contexts extend the quotation operation to a family of
operations from processes to names. Given a context, $M$, we can
define a \emph{nominal context}, $\quotep{M}$ by $\quotep{M}[P] :=
\quotep{M[P]}$. To foreshadow what is to come we observe that these
operations enjoy a duality with processes very much like the duality
between vectors and maps from vectors to scalars.

Further, because the calculus is essentially higher-order, we have a
correspondence between contexts and processes. More specifically,
given a name $x$ and a context $M$ we can construct $M^{*}_{x}$ such
that 

\begin{mathpar}
  M^{*}_{x} | \lift{x}{P} \red M[P]
\end{mathpar}

namely,

\begin{mathpar}
  M^{*}_{x} := x?(u).M[\dropn{u}]
\end{mathpar}

The dependence of $M^{*}_{x}$ on a name makes it an abstraction, 

\begin{mathpar}
  M^{*} := (x)x?(u).M[\dropn{u}]
\end{mathpar}

\subsection{Additional notation}

It will sometimes be convenient to denote the process a name
quotes. We already have the notation $x = \quotep{P}$, but it will be
convenient to introduce an alternate notation, $\procn{x}$, when we
want to emphasize the connection to the use of the name. Note that, by
virtue of name equivalence, $\quotep{\procn{x}} \nameeq x$; so, the
notation is consistent with previous definitions.

Further, because names have structure it is possible to effect
substitutions on the basis of that structure. This means we need to
upgrade our notation for substitutions, which we accomplish by
adapting comprehension notation. Thus,

\begin{mathpar}
  P\{ y / x : x \in S \}
\end{mathpar}

is interpreted to mean the process derived from P by replacing (in a
capture-avoiding manner) each occurrence of $x$ in $S$ by $y$. For example,

\begin{mathpar}
  P\{ \quotep{\procn{x}|\procn{x}} / x : x \in \freenames{P} \}
\end{mathpar}

will replace each (occurrence) of a free name $x$ in $P$ by
$\quotep{\procn{x}|\procn{x}}$.

Also, we will avail ourselves of the notation $x^{L}$ and $x^{R}$ to
denote injections of a name into disjoint copies of the name
space. There are numerous ways to accomplish this. One example can be
found in \cite{MeredithR05}. This notation overloads to vectors of
names: $\vec{x}^{\pi} := (x_{i}^{\pi} \; : \; 0 \leq i < |\vec{x}| )$ where $\pi \in \{L,R\}$.

We also use $P^{\Box} := P|\Box$.

In \cite{MeredithR05} an interpretation of the new operator is
given. It turns out that there are several possible interpretations
all enjoying the requisite algebraic properties of the operator (see
\cite{milner91polyadicpi}). We will therefore make liberal use of
$(\nu\; \vec{x})P$.

% subsection the_syntax_and_semantics_of_the_notation_system (end)   

\input{qm2pi.qmops} 

\input{qm2pi.sterngerlach} 

\input{qm2pi.metric} 

% section concurrent_process_calculi (end)

%\input{qm2pi.proofsketch}

% section proof sketch (end)

%\input{qm2pi.slviaknots} 

% section spatial logic via knots (end)

\input{qm2pi.conclusion}

% section conclusion (end)

%\input{qm2pi.dtcodes} 

% section wiring algorithm (end)

\input{qm2pi.ack} 

% section acknowledgments (end)

\newpage


\bibliographystyle{plain}   
\bibliography{../../biblios/main.bib}

\input{qm2pi.rhodetails}

\end{document}

 

% subsection basic_interpretation (end)

%\input{qm2pi.rho.presentation} 
\subsection{The syntax and semantics of the notation system}\label{sub:the_syntax_and_semantics_of_the_notation_system} % (fold)

We now summarize a technical presentation of the calculus that
embodies our theory of dynamics. The typical presentation of such a
calculus follows the style of giving generators and relations on
them. The grammar, below, describing term constructors, freely
generates the set of processes, $\Proc$. This set is then quotiented
by a relation known as structural congruence and it is over this set
that the notion of dynamics is expressed. This presentation is
essentially that of \cite{MeredithR05} with the addition of
polyadicity and summation. For readability we have relegated some of
the technical subtleties to an appendix.

\subsubsection{Process grammar}\label{subsub:process_grammar}

\begin{mathpar}
  \inferrule* [lab=synchronization] {} {{M} \bc \pzero \;|\; x?F \;|\; x!C }
  \and
  \inferrule* [lab=abstraction] {} {{F} \bc (x)P}
  \and
  \inferrule* [lab=concretion] {} {{C} \bc \langle Q \rangle}
  \and
  \inferrule* [lab=process] {} {{P,Q} \bc M \;| \;P|Q \;|\; @{x}}
  \and
  \inferrule* [lab=name] {} {{x} \bc \quotep{P}}
\end{mathpar} 

Note that $\vec{x}$ (resp. $\vec{P}$) denotes a vector of names
(resp. processes) of length $|\vec{x}|$ (resp. $|\vec{P}|$). We adopt
the following useful abbreviations.

\begin{mathpar}
   x?(\vec{y}).P := x.(\vec{y})P \and  x\clift{\vec{P}} := x.\clift{\vec{P}}
   \and x!(y) := \lift{x}{\dropn{y}}
   \and \Pi_{i=0}^{n-1}P_i := P_0 | \ldots | P_{n-1}
\end{mathpar}

\subsubsection{Structural congruence}

\paragraph{Free and bound names and alpha-equivalence.} At the
core of structural equivalence is alpha-equivalence which identifies
process that are the same up to a change of variable. Formally, we
recognize the distinction between free and bound names. The free names
of a process, $\freenames{P}$, may be calculated recursively as
follows:

\begin{mathpar}
\freenames{\pzero} := \emptyset
  \and \\
  \freenames{x?(y).P} := \{ x \} \cup (\freenames{P} \setminus \{ y \})
  \and 
  \freenames{x!\langle P \rangle} := \{ x \} \cup \{ P \} 
  \and \\
  \freenames{P|Q} := \freenames{P} \cup \freenames{Q}
  \and \\
  \freenames{@{x}} := \{ x \}
\end{mathpar}

$\pi$
$\quotep{\pi}$

$\freenames{-} : \pi \to \mathcal{P}(\quotep{\pi})$

\begin{eqnarray*}
  \freenames{\pzero} & := & \emptyset \\
  \freenames{x?(y).P} & := & \{ x \} \cup (\freenames{P} \setminus \{ y \}) \\
  \freenames{x!\langle P \rangle} & := & \{ x \} \cup \{ P \} \\
  \freenames{P|Q} & := & \freenames{P} \cup \freenames{Q} \\
  \freenames{\dropn{x}} & := & \{ x \}
\end{eqnarray*}

The bound names of a process, $\boundnames{P}$, are those names occurring in $P$
that are not free. For example, in $x?(y).0$, the name $x$ is free, while $y$ is bound.

\begin{mathpar}
  \inferrule* [lab=monoidal-laws] {} { P|Q \equiv Q|P \and P|0 \equiv P \and P|(Q|R) \equiv (P|Q)|R }
\end{mathpar}

\begin{mathpar}
  \inferrule* [lab=alpha-equivalence] {} { (x)P \equiv (y)P\{y/x\} \and y \not\in \freenames{P} }
\end{mathpar}

\begin{definition}
Then two processes, $P,Q$, are alpha-equivalent if $P = Q\{\vec{y}/\vec{x}\}$ for
some $\vec{x} \in \boundnames{Q},\vec{y} \in \boundnames{P}$, where $Q\{\vec{y}/\vec{x}\}$
denotes the capture-avoiding substitution of $\vec{y}$ for $\vec{x}$ in $Q$.
\end{definition}

\begin{definition}
  The {\em structural congruence} \cite{SangiorgiWalker} , $\equiv$,
  between processes is the least congruence containing
  alpha-equivalence, satisfying the abelian monoid laws
  (associativity, commutativity and $\pzero$ as identity) for parallel
  composition $|$ and for summation $+$.
\end{definition}

\subsection{Name equivalence}

We take name equivalence, written $\nameeq$, to be the smallest
equivalence relation generated by the following rules.

\begin{mathpar}
\inferrule*[lab=Quote-drop]
{ }
{ \quotep{@{x}} \nameeq x }

\inferrule*[lab=Struct-equiv]
{ P \scong Q }
{ \quotep{P} \nameeq \quotep{Q} }
\end{mathpar}

The astute reader will have noticed that the mutual recursion of names
and processes imposes a mutual recursion on alpha-equivalence and
structural equivalence via name-equivalence. Fortunately, all of this
works out pleasantly and we may calculate in the natural way, free of
concern. The reader interested in the details is referred to the
appendix \ref{appendix:rho_details}.

\subsection{Substitution}

We use $\Proc$ for the set of processes, $\QProc$ for the set of
names, and $\id{\{}\vec{y} / \vec{x} \id{\}}$ to denote partial maps,
$s : \QProc \rightarrow \QProc$. A map, $s$ lifts, uniquely, to a map
on process terms, $\widehat{s} : \Proc \rightarrow \Proc$ by the
following equations.

\begin{mathpar}
  (0) \psubstp{Q}{P} := 0 \\
  (R \juxtap S) \psubstp{Q}{P}
  :=    
  (R)\psubstp{Q}{P} \juxtap (S) \psubstp{Q}{P} \\
  (x?(y).R) \psubstp{Q}{P}    
  :=    
  (x)\substp{Q}{P} (z)\concat( (R \psubstn{z}{y}) \psubstp{Q}{P} ) \\
  (\lift{x}{R}) \psubstp{Q}{P}  
  :=
  \lift{(x)\substp{Q}{P}}{ R \psubstp{Q}{P} } \\
%   (\dropn{x})  \psubstp{Q}{P}       
%   := 
%   \left\{ 
%     \begin{array}{ccc} 
%       \dropn{\quotep{Q}} & & x \nameeq \quotep{P} \\
%       \dropn{x} & & otherwise \\
%     \end{array}
%   \right. 
  (\dropn{x})  \psubstp{Q}{P}       
  := 
  \left\{ 
    \begin{array}{ccc} 
      Q & & x \nameeq \quotep{P} \\
      \dropn{x} & & otherwise \\
    \end{array}
  \right.
\end{mathpar}
 

where

\begin{eqnarray}
  (x)\id{\{} \lpquote Q \rpquote / \lpquote P \rpquote \id{\}}            = 
  \left\{ 
    \begin{array}{ccc}
      \lpquote Q \rpquote & & x \nameeq \lpquote P \rpquote \\
      x & & otherwise \\
    \end{array}
  \right. \nonumber
\end{eqnarray}

and $z$ is chosen distinct from $\quotep{P}$, $\quotep{Q}$, the free
names in $Q$, and all the names in $R$. Our $\alpha$-equivalence will
be built in the standard way from this substitution.

\begin{remark}\label{rem:no_self_referential_names}
  One consequence of these definitions is that $\forall P. \quotep{P}
  \not\in \freenames{P}$.
\end{remark}

\subsection{ Dynamic quote: an example }

Anticipating something of what's to come, consider applying the
substitution, $\widehat{\id{\{}u / z \id{\}}}$, to the following pair
of processes, $\lift{w}{y!(z)}$ and $w[ \lpquote y!(z) \rpquote ]$.

\begin{eqnarray}
	\lift{w}{y!(z)}\widehat{\id{\{}u / z \id{\}}}
		& = &
		\lift{w}{y!(u)} \nonumber\\
	w[ \lpquote y!(z) \rpquote ] \widehat{ \id{\{}u / z \id{\}} }
		& = &
		w[ \lpquote y!(z) \rpquote ] \nonumber
\end{eqnarray}

Because the body of the process between quotes is impervious to
substitution, we get radically different answers. In fact, by
examining the first process in an input context,
e.g. $x?(z).\lift{w}{y!(z)}$, we see that the process under the lift
operator may be shaped by prefixed inputs binding a name inside it. In
this sense, the lift operator will be seen as a way to dynamically
construct processes before reifying them as names.

Finally equipped with these standard features we can present the
dynamics of the calculus.

\subsubsection{Operational semantics} 

Finally, we introduce the computational dynamics. What marks these
algebras as distinct from other more traditionally studied algebraic
structures, e.g. vector spaces or polynomial rings, is the manner in
which dynamics is captured. In traditional structures, dynamics is typically
expressed through morphisms between such structures, as in linear maps
between vector spaces or morphisms between rings. In algebras
associated with the semantics of computation, the dynamics is
expressed as part of the algebraic structure itself, through a
reduction reduction relation typically denoted by $\red$. Below, we
give a recursive presentation of this relation for the calculus used
in the encoding.

$\red \subseteq \pi \times \pi$
$\red : \pi \to \mathcal{P}(\pi)$

\begin{mathpar}
  \inferrule* [lab=Comm] { \textsf{match}( x_{src}, x_{trgt} ) } { x_{trgt}?(y)P \; | \; x_{src}!\langle {Q} \rangle \red P\{\quotep{Q}/y}\} }
  \and \\
  \inferrule* [lab=Par] {{P} \red {P}'} {{{P} | {Q}} \red {{P}' | {Q}}}
  \and
  \inferrule* [lab=Equiv]{{{P} \scong {P}'} \andalso {{P}' \red {Q}'} \andalso {{Q}' \scong {Q}}}{{P} \red {Q}}
\end{mathpar}

\begin{eqnarray*}
  match_{\equiv} (\quotep{P},\quotep{Q}) & := & P \equiv Q \\
  match_{\dagger}(\quotep{P},\quotep{Q}) & := & \forall R. P|Q \red^{*} R => R \red^{*} 0 \\
  match_{K}(\quotep{P},\quotep{Q}) & := & K \mbox{ for some context } K
\end{eqnarray*}

$u?(x)P | u!\langle Q \rangle \red P\{\quotep{Q}/x\}$

%We write $\wred$ for $\red^*$, and $P\red$ if $\exists Q $ such that $ P \red Q$.
We write $P\red$ if $\exists Q $ such that $ P \red Q$ and $P\not\red$, otherwise.

\section{Replication}

As mentioned before, it is known that replication (and hence
recursion) can be implemented in a higher-order process algebra
\cite{SangiorgiWalker}. As our first example of calculation with the
machinery thus far presented we give the construction explicitly in
the {\rhoc}.

\begin{eqnarray}
	D_{x} & := & \prefix{x}{y}{(\binpar{\outputp{x}{y}}{@{y}})} \nonumber\\
	\bangp_{x}{P} & := & \binpar{{x}!\langle{\binpar{D_{x}}{P}}\rangle}{D_{x}} \nonumber
\end{eqnarray}

\begin{eqnarray}
	\bangp_{x}{P} & & \nonumber\\
	=
	& {x}!\langle{(\prefix{x}{y}{(\outputp{x}{y} | @{y})) | P}}\rangle 
	      | \prefix{x}{y}{(\outputp{x}{y} | @{y})} & \nonumber\\
	\red
	& (\outputp{x}{y} | @{y})\substn{\quotep{(\prefix{x}{y}{(@{y} | \outputp{x}{y})) | P}}}{y} & \nonumber\\
	=
	& \outputp{x}{\quotep{(\prefix{x}{y}{(\outputp{x}{y} | @{y})) | P}}}
	  | {(\prefix{x}{y}{(\outputp{x}{y} | @{y})) | P}} & \nonumber\\
	\red
	& \ldots & \nonumber\\
	\red^*
	& P | P | \ldots & \nonumber
\end{eqnarray}

Of course, this encoding, as an implementation, runs away, unfolding
$\bangp{P}$ eagerly. A lazier and more implementable replication
operator, restricted to input-guarded processes, may be obtained as follows.

\begin{eqnarray}
\bangp{\prefix{u}{v}{P}} 
	:= 
	\binpar{\lift{x}{\prefix{u}{v}{(\binpar{D(x)}{P})}}}{D(x)} \nonumber
\end{eqnarray}

\begin{remark}
  Note that the lazier definition still does not deal with summation
  or mixed summation (i.e. sums over input and output). The reader is
  invited to construct definitions of replication that deal with these
  features. 

  Further, the definitions are parameterized in a name, $x$. Can you,
  gentle reader, make a definition that eliminates this parameter and
  guarantees no accidental interaction between the replication
  machinery and the process being replicated -- i.e. no accidental
  sharing of names used by the process to get its work done and the
  name(s) used by the replication to effect copying. This latter
  revision of the definition of replication is crucial to obtaining
  the expected identity $!!P \sim !P$.
\end{remark}

\begin{remark}\label{rem:paradoxical_combinator}
  The reader familiar with the lambda calculus will have noticed the
  similarity between $D$ and the paradoxical combinator.

  [Ed. note: the existence of this seems to suggest we have to be more
  restrictive on the set of processes and names we admit if we are to
  support no-cloning.]
\end{remark}

\subsubsection{Bisimulation}

The computational dynamics gives rise to another kind of equivalence,
the equivalence of computational behavior. As previously mentioned
this is typically captured \emph{via} some form of bisimulation.

% The notion we use in this paper is weak barbed bisimulation
% \cite{milner91polyadicpi}.

The notion we use in this paper is derived from weak barbed
bisimulation \cite{milner91polyadicpi}. 

\begin{definition}
An \emph{observation relation}, $\downarrow_{\mathcal N}$, over a set
of names, $\mathcal N$, is the smallest relation satisfying the rules
below.

\infrule[Out-barb]{y \in {\mathcal N}, \; x \nameeq y}
		  {\outputp{x}{v} \downarrow_{\mathcal N} x}
\infrule[Par-barb]{\mbox{$P\downarrow_{\mathcal N} x$ or $Q\downarrow_{\mathcal N} x$}}
		  {\binpar{P}{Q} \downarrow_{\mathcal N} x}

We write $P \Downarrow_{\mathcal N} x$ if there is $Q$ such that 
$P \wred Q$ and $Q \downarrow_{\mathcal N} x$.
\end{definition}

\begin{definition}
%\label{def.bbisim}
An  ${\mathcal N}$-\emph{barbed bisimulation} over a set of names, ${\mathcal N}$, is a symmetric binary relation 
${\mathcal S}_{\mathcal N}$ between agents such that $P\rel{S}_{\mathcal N}Q$ implies:
\begin{enumerate}
\item If $P \red P'$ then $Q \wred Q'$ and $P'\rel{S}_{\mathcal N} Q'$.
\item If $P\downarrow_{\mathcal N} x$, then $Q\Downarrow_{\mathcal N} x$.
\end{enumerate}
$P$ is ${\mathcal N}$-barbed bisimilar to $Q$, written
$P \wbbisim_{\mathcal N} Q$, if $P \rel{S}_{\mathcal N} Q$ for some ${\mathcal N}$-barbed bisimulation ${\mathcal S}_{\mathcal N}$.
\end{definition}

$\mathcal{R} \subseteq \pi \times \pi$

$P \mathcal{R} Q => \forall P'. P \red P' \Rightarrow \exists Q'. Q \red Q', P' \mathcal{R} Q'$

$P \vdash x \Rightarrow Q \vdash x$

\begin{mathpar}
  \inferrule*[lab=Out-barb]{x \nameeq y}{{y}!\langle{Q}\rangle \vdash x}
  \and
  \inferrule*[lab=Par-barb]{\mbox{$P\vdash x$ or $Q\vdash x$}}{\binpar{P}{Q} \vdash x}
\end{mathpar}

\subsubsection{Contexts}

One of the principle advantages of computational calculi like the
$\pi$-calculus is a well-defined notion of context,
contextual-equivalence and a correlation between
contextual-equivalence and notions of bisimulation. The notion of
context allows the decomposition of a process into (sub-)process and
its syntactic environment, its context. Thus, a context may be
thought of as a process with a ``hole'' (written $\Box$) in it. The
application of a context $M$ to a process $P$, written $M[P]$, is
tantamount to filling the hole in $M$ with $P$. In this paper we do
not need the full weight of this theory, but do make use of the notion
of context in the proof the main theorem. 

\begin{mathpar}
  \inferrule* [lab=summation] {} {{M_{M},M_{N}} \bc \Box \;|\; x.M_{A} \;|\; M_{M}+M_{N}}
  \and
  \inferrule* [lab=agent] {} {{M_{A}} \bc (\vec{x})M_{P} \;| \; \clift{P_0,\ldots,M_{P},\ldots,P_N}}
  \and \\
  \inferrule* [lab=process] {} {{M_{P}} \bc M_{N} \;| \;P|M_{P} }
\end{mathpar} 

\begin{mathpar}
  \inferrule* [lab=sychronization] {} {M_{N} \bc \Box \;|\; x?M_{F} \;|\; x!M_{C}}
  \and
  \inferrule* [lab=abstraction] {} {{M_{F}} \bc (x)M_{P} }
  \and
  \inferrule* [lab=concretion] {} {{M_{C}} \bc \langle M_{P} \rangle }
  \and \\
  \inferrule* [lab=process] {} {{M_{P}} \bc M_{N} \;| \;P|M_{P} }
\end{mathpar}

\begin{definition}[contextual application] Given a context $M$, and
  process $P$, we define the \emph{contextual application}, $M[P] :=
  M\{P/\Box\}$. That is, the contextual application of M to P is the
  substitution of $P$ for $\Box$ in $M$.
\end{definition}

$\meaningof{-} : L \to \mathcal{P}(\pi)$

\begin{mathpar}
  \inferrule* [lab=collection] {} {\meaningof{true} = \pi, \and \meaningof{~E} = \pi \setminus \meaningof{E}, \and \meaningof{E_{1} \& E_{2}} = \meaningof{E_{1}} \cap \meaningof{E_{2}}}
\end{mathpar}

\begin{mathpar}
  \inferrule* [lab=structure] {} {\meaningof{0} = \{ P \in \pi | P \equiv 0 \}, \and \\ \meaningof{E_1 | E_2} = \{ P \in \pi | P \equiv P_{1} | P_{2}, P_{1} \in \meaningof{E_{1}}, P_{2} \in \meaningof{E_2}\} }
\end{mathpar}

\begin{mathpar}
 \inferrule* [lab=behavior] {} {\meaningof{\langle a?b \rangle E} = \{ P \in \pi | P \equiv Q | u?(y)P', \\ \and \\\\ \and \\ \;\;\; u \in \meaningof{a}, \forall z.P'\{z/y\} \in \meaningof{E\{z/b\}}\}, \and \\ \meaningof{a!E} = \{ P \in \pi | P \equiv Q | x!\langle P' \rangle, x \in \meaningof{a} P' \in \meaningof{E}\} }
\end{mathpar}

\begin{mathpar}
 \inferrule* [lab=nominal] {} {\meaningof{\quotep{E}} = \{ \quotep{P} \in \quotep{\pi} | P \in \meaningof{E} \}, \and \meaningof{\quotep{P}} = \{ \quotep{Q} \in \quotep{\pi} | P \equiv Q \} \and \\ \meaningof{@\quotep{E}} = \{ P \in \pi | P \equiv @x, x \in \meaningof{E} \}}
\end{mathpar}

\begin{eqnarray*}
  \\
  \meaningof{-} : TS \to ST
\end{eqnarray*}

\begin{eqnarray*}
  \\
  L : TS \to ST
\end{eqnarray*}

\begin{eqnarray*}
  \\
  P \models E \iff P \in \meaningof{E}
\end{eqnarray*}

\begin{eqnarray*}
  P \approx_{L} Q \iff \forall E \in L. P \models E \iff Q \models E
\end{eqnarray*}

\begin{eqnarray*}
  P \approx_{K} Q
\end{eqnarray*}

\begin{eqnarray*}
  P \approx Q
\end{eqnarray*}

$\approx_{K} = \approx = \approx_{L}$

\subsubsection{Contextual duality}

Note that contexts extend the quotation operation to a family of
operations from processes to names. Given a context, $M$, we can
define a \emph{nominal context}, $\quotep{M}$ by $\quotep{M}[P] :=
\quotep{M[P]}$. To foreshadow what is to come we observe that these
operations enjoy a duality with processes very much like the duality
between vectors and maps from vectors to scalars.

Further, because the calculus is essentially higher-order, we have a
correspondence between contexts and processes. More specifically,
given a name $x$ and a context $M$ we can construct $M^{*}_{x}$ such
that 

\begin{mathpar}
  M^{*}_{x} | \lift{x}{P} \red M[P]
\end{mathpar}

namely,

\begin{mathpar}
  M^{*}_{x} := x?(u).M[\dropn{u}]
\end{mathpar}

The dependence of $M^{*}_{x}$ on a name makes it an abstraction, 

\begin{mathpar}
  M^{*} := (x)x?(u).M[\dropn{u}]
\end{mathpar}

\subsection{Additional notation}

It will sometimes be convenient to denote the process a name
quotes. We already have the notation $x = \quotep{P}$, but it will be
convenient to introduce an alternate notation, $\procn{x}$, when we
want to emphasize the connection to the use of the name. Note that, by
virtue of name equivalence, $\quotep{\procn{x}} \nameeq x$; so, the
notation is consistent with previous definitions.

Further, because names have structure it is possible to effect
substitutions on the basis of that structure. This means we need to
upgrade our notation for substitutions, which we accomplish by
adapting comprehension notation. Thus,

\begin{mathpar}
  P\{ y / x : x \in S \}
\end{mathpar}

is interpreted to mean the process derived from P by replacing (in a
capture-avoiding manner) each occurrence of $x$ in $S$ by $y$. For example,

\begin{mathpar}
  P\{ \quotep{\procn{x}|\procn{x}} / x : x \in \freenames{P} \}
\end{mathpar}

will replace each (occurrence) of a free name $x$ in $P$ by
$\quotep{\procn{x}|\procn{x}}$.

Also, we will avail ourselves of the notation $x^{L}$ and $x^{R}$ to
denote injections of a name into disjoint copies of the name
space. There are numerous ways to accomplish this. One example can be
found in \cite{MeredithR05}. This notation overloads to vectors of
names: $\vec{x}^{\pi} := (x_{i}^{\pi} \; : \; 0 \leq i < |\vec{x}| )$ where $\pi \in \{L,R\}$.

We also use $P^{\Box} := P|\Box$.

In \cite{MeredithR05} an interpretation of the new operator is
given. It turns out that there are several possible interpretations
all enjoying the requisite algebraic properties of the operator (see
\cite{milner91polyadicpi}). We will therefore make liberal use of
$(\nu\; \vec{x})P$.

% subsection the_syntax_and_semantics_of_the_notation_system (end)   

\section{Interpretation of QM}
\subsection{Supporting definitions}
\subsubsection{Multiplication}
\begin{mathpar}
  \quotep{Q} \cdot \quotep{R} := \quotep{Q|R}
  \and \\
  \quotep{Q} \cdot P := P\{ \quotep{Q|R} / \quotep{R} : \quotep{R} \in \freenames{P} \}
\end{mathpar}

\paragraph{Discussion}
The first line needs little explanation. The second line says that
each free name of the process is replaced with the multiplication of
that name by the scalar. Multiplication of a scalar (name) by a state
(process) results in a process all the names of which have been `moved
over' by parallel composition with the process the scalar
quotes. There is a subtlety that the bound names have to be
manipulated so that multiplied names aren't accidentally
captured. There are many ways to achieve this.

\begin{remark}\label{rem:multiplication_identities}
  The reader is invited to verify that for all $x,y,z \in \QProc$ and $P \in \Proc$
  \begin{mathpar}
    x \cdot \quotep{0} \equiv x 
    \and
    x \cdot y \equiv y \cdot x
    \and
    x \cdot (y \cdot z) \equiv (x \cdot y) \cdot z
    \and \\
    \quotep{0} \cdot P \equiv P
    \and \\
    x \cdot (y \cdot P) \equiv (x \cdot y) \cdot P
    \and \\
    x \cdot (P|Q) \equiv (x \cdot P) | (x \cdot Q)
    \and \\    
  \end{mathpar}
\end{remark}

\subsubsection{Tensor product}

We define a tensor product on processes by structural induction.

\paragraph{Tensor of sums} First note that all summations, including
$\pzero$ and sequence, can be written $\Sigma_{i} x_{i}.A_{i} +
\Sigma_{j} x_{j}.C_{j}$, where we have grouped input-guarded processes
together and output-guarded processes together.

Thus, we can define the tensor product of two summations, $N_{1}\otimes N_{2}$, where

\begin{mathpar}
  N_{1} := \Sigma_{i} x_{i}.A_{i} + \Sigma_{j} x_{j}.C_{j}
  \and
  N_{2} := \Sigma_{i'} y_{i'}.B_{i'} + \Sigma_{j'} y_{j'}.D_{j'} 
\end{mathpar}

as follows.

\begin{mathpar}
  \Sigma_{i} x_{i}.A_{i} + \Sigma_{j} x_{j}.C_{j} \otimes \Sigma_{i'}
  y_{i'}.B_{i'} + \Sigma_{j'} y_{j'}.D_{j'} 
  \and \\
  := \; \Sigma_{i} \Sigma_{i'} \quotep{\stackrel{\vee}{x_{i}}| \stackrel{\vee}{y_{i'}}}.(A_{i}\otimes B_{i'}) \; | \; \Sigma_{i'} \Sigma_{i} \quotep{\stackrel{\vee}{y_{i'}}|\stackrel{\vee}{x_{i}}}.(B_{i'}\otimes A_{i})
  \and
  \;\; | \;\; \Sigma_{j} \Sigma_{j'} \quotep{\stackrel{\vee}{x_{j}}|\stackrel{\vee}{y_{j'}}}.(A_{j}\otimes B_{j'}) \; | \; \Sigma_{j'} \Sigma_{j} \quotep{\stackrel{\vee}{y_{j'}}|\stackrel{\vee}{x_{j}}}.(B_{j'}\otimes A_{j})
\end{mathpar}

\begin{remark}
  Do we need to $x^{L}$ and $y^{R}$ for this construction as well?
\end{remark}

\paragraph{Tensor of parallel compositions} Next, we distribute tensor
over par.

\begin{mathpar}
  P_{1}|P_{2} \otimes Q_{1}|Q_{2} := (P_{1} \otimes Q_{1}) | (P_{1}
  \otimes Q_{2}) | (P_{2} \otimes Q_{1}) | (P_{2} \otimes Q_{2})
\end{mathpar}

\paragraph{Tensor with dropped names} We treat tensor of a
process with a dropped name as parallel composition.

\begin{mathpar}
  P \otimes \dropn{x} := P | \dropn{x}
\end{mathpar}

\paragraph{Tensor of agents}

Finally, we need to define tensor on agents. Note that the definition
of tensor on normal products only tensors inputs with inputs and
outputs with outputs. Thus, we only have to define the operation on
``homogeneous'' pairings.

\begin{mathpar}
  (\vec{x})P \otimes (\vec{y})Q
  \and \\
  := (x_{0}^{L}|y_{0}^{R},\ldots,x_{0}^{L}|y_{n}^{R},\ldots,x_{m}^{L}|y_{0}^{R},\ldots,x_{m}^{L}|y_{n}^R)(P\{ \vec{x}^{L}/\vec{x}\} \otimes Q \{ \vec{y}^{R}/\vec{y}\})
  \and \\
  \clift{\vec{P}} \otimes \clift{\vec{Q}}
  \and \\
  := \clift{P_{0}\otimes Q_{0},\ldots,P_{0}\otimes Q_{n},\ldots,P_{m}\otimes Q_{0},\ldots,P_{m}\otimes Q_{n}}
\end{mathpar}

\begin{remark}
  Observe that arities of tensored abstractions matches arities of
  tensored concretions if the original arities matched. Note also that
  the length of the arities corresponds to the increase in dimension
  we see in ordinary vector space tensor product.
\end{remark}

\begin{remark}
  Operationally, this definition distributes the tensor down to
  components ``linked'' by summation. Tensor over summation is
  intriguing in that it mixes names. Moreover, as a consequence of the
  way it mixes names we have the identities for all $x \in \QProc$ and
  $P,Q \in \Proc$

  \begin{mathpar}
    (x \cdot P) \otimes Q \equiv x \cdot (P \otimes Q) \equiv P \otimes (x \cdot Q)
    \and
    P \otimes \pzero \equiv P
  \end{mathpar}

  that the reader is invited to verify.
\end{remark}

\subsubsection{Annihilation}
\begin{mathpar}
  P^{\perp} := \{ Q | \forall R. P|Q \red^{*} R \Rightarrow R \red^{*} \pzero \}
  \and \\
  P^{\underline{\perp}} := \Sigma_{Q \in P^{\perp}} \quotep{Q}?(y).(\dropn{y}|Q) | \Sigma_{Q \in P^{\perp}} \quotep{Q}\clift{\Box}
\end{mathpar}

\paragraph{Discussion} The reader will note that $P^{\perp}$ is a
\emph{set} of processes, while $P^{\underline{\perp}}$ is a
\emph{context}. We call the set $P^{\perp}$ the \emph{annihilators} of
$P$. The parallel composition of a process in the annihilators of $P$
with $P$ will result in a process, the state space of which has all
paths eventually leading to $\pzero$. Execution may endure loops; but
under reasonable conditions of fairness (naturally guaranteed under
most notions of bisimulation) such a composite process cannot get
stuck in such a loop and will, eventually pop out and terminate.

The context $P^{\underline{\perp}}$ is ready and willing to ``take the
$P$ out of'' the process to which it is applied. It will effectively
transmit the code of the process to which it is applied to one of the
annihilators and run the process against it.

\subsubsection{Evaluation}
We fix $M$ a domain of fully abstract interpretation with an equality
coincident with bisimulation. We take $\meaningof{\cdot} : \Proc \to
M$ to be the map interpreting processes and $\nmeaningof{\cdot} : \M
\to Proc$ to be the map running the other way. Then we define

\begin{mathpar}
  \int P := \nmeaningof{\meaningof{P}}
\end{mathpar}

\paragraph{Discussion}
There are many fully abstract interpretations of Milner's
$\pi$-calculus. Any of them can be used as a basis for interpreting
the reflective calculus here. Equipped with such a domain it is
largely a matter of grinding through to check that the Yoneda
construction for the normalization-by-evaluation program can be
extended to this setting.

\begin{remark}
  The reader is invited to verify that $\int (P^{\underline{\perp}}[P]) = 0$.
\end{remark}

\subsection{Quantum mechanics}

Table \ref{tbl:core_qm_op_defns} gives the core operational definitions

\begin{table}[htp]\label{tbl:core_qm_op_defns}
  \center{
    \fbox{
      \begin{tabular}{c|c}
        quantum mechanics & process calculus \\
        \hline
        scalar & $x := \quotep{P}$ \\
        state vector & $\state{P} := P$ \\
        dual & $\state{P}^{*} := \event{P^{\underline{\perp}}} := \quotep{P^{\underline{\perp}}}[-]$ \\
        matrix & $ \Sigma_{\alpha} \state{P_{\alpha}}x_{\alpha}\event{Q_{\alpha}}$ \\
        vector addition & $\state{P} + \state{Q} := \state{P | Q}$ \\
        tensor product & $\state{P} \otimes \state{Q} := \state{P \otimes Q}$ \\
        inner product & $\innerprod{P}{Q} := \quotep{\int P^{\underline{\perp}}[Q]}$ \\
      \end{tabular}
    }
  }
  \caption{QM - operational definitions}
\end{table}

where

\begin{mathpar}
  \prmatrix{P}{Q} := \fprmatrix{P}{\quotep{\pzero}}{Q}
  \and
  \fprmatrix{P}{x}{Q} := (\state{P},x,\event{Q})
  \and
  (\fprmatrix{P}{x}{Q})(\state{R}) := x \cdot \innerprod{Q}{R} \cdot \state{P}
  \and
  (\fprmatrix{P}{x}{Q})(\event{R}) := x \cdot \innerprod{R}{P} \cdot \event{Q}
\end{mathpar}

\paragraph{Discussion}
As promised: vectors (aka states) are represented as processes; duals
as contextual duals; inner product definition should be compared with
standard inner product definition for ....

\begin{remark}
  Assuming $\int (P^{\underline{\perp}}[P]) = 0$, the reader is
  invited to verify that $(\fprmatrix{P}{x}{P})(\state{P}) = x \cdot \state{P}$.
\end{remark}

\begin{remark}
  The reader is invited to verify that $\innerprod{P}{Q}$ could
  equally well have been written $\quotep{\int \stackrel{\vee}{x}}$
  where $x = \event{P^{\underline{\perp}}}(Q)$.

  One of the motivations for this remark is that there is another way
  to factor these operations. We could package up evaluation in the dual:

  \begin{mathpar}
    \state{P}^{*} := \event{\int P^{\underline{\perp}}} := \quotep{\int P^{\underline{\perp}}}[-]
  \end{mathpar}

  and then have inner product defined by
  
  \begin{mathpar}
    \innerprod{P}{Q} := \event{P}(Q)
  \end{mathpar}

  Hopefully, experience with the calculations will provide guidance on
  the best factoring.
\end{remark}

\begin{remark}
  Assuming $\int (P^{\underline{\perp}}[P]) = 0$, the reader is
  invited to verify that $\forall P,Q. (\prmatrix{0}{Q})(\state{0}) =
  \state{0}$ and dually $(\prmatrix{P}{0})(\event{0}) = \event{0}$.
\end{remark}

\begin{remark}
  i'm a little worried that i don't (yet) have proper support for
  complex conjugacy. But, the observation above may give us a
  clue. According to Abramsky, it must be the case that the scalars
  are iso to the homset of the identity for the tensor -- which the
  observation above characterizes. 

  For now, we will simply bookmark the notion with $\overline{x}$.
\end{remark}

\subsubsection{Adjointness}

We need to give a definition of $(\cdot)^{\dagger}$ for matrices. The
obvious candidate definition is
\begin{mathpar}
(\Sigma_{\alpha}\fprmatrix{P_{\alpha}}{x_{\alpha}}{Q_{\alpha}})^{\dagger}
= \Sigma_{\alpha}\fprmatrix{(Q_{\alpha}^{\underline{\perp}})^{*}}{\overline{x}_{\alpha}}{P_{\alpha}^{\underline{\perp}}} 
\end{mathpar}

But, $(Q_{\alpha}^{\underline{\perp}})^{*}$ requires a name along
which to communicate the process to achieve the context application.

\subsubsection{Basis for a basis}
If processes label states and ``addition'' of states (a.k.a. vector
addition) is interpreted as parallel composition, what corresponds to
notions of linear independence and basis? Here, we recall that Yoshida
has developed a set of \emph{combinators} for an asynchronous verison
of Milner's $\pi$-calculus. These are a finite set of processes such
any process can be expressed as parallel composition of these
combinators together with liberal uses of the new operator and
replication. We can simply give a translation of these into the
present calculus and have reasonable expectation that the property
carries over. That is, that the resultant set allows to express all
processes via parallel composition. Note, however, that there is no
new operator or replication in this calculus. As a result, we expect
that the corresponding set is actually infinite. That is, we expect
that the space is actually infinite dimensional.

\begin{remark}
  The attentive reader may be a bit concerned. Certainly, the
  collection $S$, $K$ and $I$ is a finite set of
  combinators. Shouldn't we expect to see a finite set of combinators
  for an effectively equivalent system? i am very sympathetic to this
  critique and feel it warrants full attention. On the other hand, i
  also have in mind the following analogy. The natural numbers, as a
  monoid under addition, has exactly $1$ generator, while the natural
  numbers, as a monoid under multiplication, has countably many
  generators (the primes). We observe that the application of the
  lambda calculus is much less resource sensitive than the parallel
  composition of the $\pi$-calculus. Could it be the case that we have
  an analogy of the form
  
  \begin{mathpar}
    m + n : MN :: m*n : M|N
  \end{mathpar}

  giving a similar blow up in the set of ``primes''?  This is such a
  wonderful thought that, even if it's not true, i think it's worth
  writing down.
\end{remark}
 

\documentclass[12pt]{llncs}
%\documentclass{jktr}

\usepackage[pdftex]{hyperref}                   
\usepackage {listings}
\usepackage {mathpartir}
\usepackage{bcprules}
%\usepackage{listings}
                       
\usepackage{graphicx} 
%\usepackage[margins=2.5cm,nohead,nofoot]{geometry}
%\usepackage{geometry}
\usepackage{amsfonts}
\usepackage{amstext}
\usepackage{latexsym}
\usepackage{amssymb}
\usepackage{color}


%\include{myPreamble}
\include{qm2pi.local} 

%\ifpdf
%\usepackage[pdftex]{graphicx}
%\else
%\usepackage{graphicx}
%\fi

 % \ifpdf
%  \usepackage{pdfsync}
%  \if


%\title{Brief Article}
%\author{David F. Snyder}
%\author{L.G. Meredith}

%\address{Dept. of Math., Texas State University--San Marcos, San Marcos, TX 78666}
       
\pagestyle{empty}


\begin{document}

\lstset{language=[Objective]Caml,frame=shadowbox}

\input{qm2pi.front}

% section front matter (end)

\input{qm2pi.intro} 
 
% section introduction (end)

% \input{qm2pi.knotations} 

% section notation (end)

\input{qm2pi.process.calculi} 

% section concurrent_process_calculi_and_spatial_logics_ (end)
    
%\input{qm2pi.knots2pi} 

%\input{qm2pi.trefoil} 

%\input{qm2pi.mainthm} 

% subsection basic_interpretation (end)

%\input{qm2pi.rho.presentation} 
\subsection{The syntax and semantics of the notation system}\label{sub:the_syntax_and_semantics_of_the_notation_system} % (fold)

We now summarize a technical presentation of the calculus that
embodies our theory of dynamics. The typical presentation of such a
calculus follows the style of giving generators and relations on
them. The grammar, below, describing term constructors, freely
generates the set of processes, $\Proc$. This set is then quotiented
by a relation known as structural congruence and it is over this set
that the notion of dynamics is expressed. This presentation is
essentially that of \cite{MeredithR05} with the addition of
polyadicity and summation. For readability we have relegated some of
the technical subtleties to an appendix.

\subsubsection{Process grammar}\label{subsub:process_grammar}

\begin{mathpar}
  \inferrule* [lab=synchronization] {} {{M} \bc \pzero \;|\; x?F \;|\; x!C }
  \and
  \inferrule* [lab=abstraction] {} {{F} \bc (x)P}
  \and
  \inferrule* [lab=concretion] {} {{C} \bc \langle Q \rangle}
  \and
  \inferrule* [lab=process] {} {{P,Q} \bc M \;| \;P|Q \;|\; @{x}}
  \and
  \inferrule* [lab=name] {} {{x} \bc \quotep{P}}
\end{mathpar} 

Note that $\vec{x}$ (resp. $\vec{P}$) denotes a vector of names
(resp. processes) of length $|\vec{x}|$ (resp. $|\vec{P}|$). We adopt
the following useful abbreviations.

\begin{mathpar}
   x?(\vec{y}).P := x.(\vec{y})P \and  x\clift{\vec{P}} := x.\clift{\vec{P}}
   \and x!(y) := \lift{x}{\dropn{y}}
   \and \Pi_{i=0}^{n-1}P_i := P_0 | \ldots | P_{n-1}
\end{mathpar}

\subsubsection{Structural congruence}

\paragraph{Free and bound names and alpha-equivalence.} At the
core of structural equivalence is alpha-equivalence which identifies
process that are the same up to a change of variable. Formally, we
recognize the distinction between free and bound names. The free names
of a process, $\freenames{P}$, may be calculated recursively as
follows:

\begin{mathpar}
\freenames{\pzero} := \emptyset
  \and \\
  \freenames{x?(y).P} := \{ x \} \cup (\freenames{P} \setminus \{ y \})
  \and 
  \freenames{x!\langle P \rangle} := \{ x \} \cup \{ P \} 
  \and \\
  \freenames{P|Q} := \freenames{P} \cup \freenames{Q}
  \and \\
  \freenames{@{x}} := \{ x \}
\end{mathpar}

$\pi$
$\quotep{\pi}$

$\freenames{-} : \pi \to \mathcal{P}(\quotep{\pi})$

\begin{eqnarray*}
  \freenames{\pzero} & := & \emptyset \\
  \freenames{x?(y).P} & := & \{ x \} \cup (\freenames{P} \setminus \{ y \}) \\
  \freenames{x!\langle P \rangle} & := & \{ x \} \cup \{ P \} \\
  \freenames{P|Q} & := & \freenames{P} \cup \freenames{Q} \\
  \freenames{\dropn{x}} & := & \{ x \}
\end{eqnarray*}

The bound names of a process, $\boundnames{P}$, are those names occurring in $P$
that are not free. For example, in $x?(y).0$, the name $x$ is free, while $y$ is bound.

\begin{mathpar}
  \inferrule* [lab=monoidal-laws] {} { P|Q \equiv Q|P \and P|0 \equiv P \and P|(Q|R) \equiv (P|Q)|R }
\end{mathpar}

\begin{mathpar}
  \inferrule* [lab=alpha-equivalence] {} { (x)P \equiv (y)P\{y/x\} \and y \not\in \freenames{P} }
\end{mathpar}

\begin{definition}
Then two processes, $P,Q$, are alpha-equivalent if $P = Q\{\vec{y}/\vec{x}\}$ for
some $\vec{x} \in \boundnames{Q},\vec{y} \in \boundnames{P}$, where $Q\{\vec{y}/\vec{x}\}$
denotes the capture-avoiding substitution of $\vec{y}$ for $\vec{x}$ in $Q$.
\end{definition}

\begin{definition}
  The {\em structural congruence} \cite{SangiorgiWalker} , $\equiv$,
  between processes is the least congruence containing
  alpha-equivalence, satisfying the abelian monoid laws
  (associativity, commutativity and $\pzero$ as identity) for parallel
  composition $|$ and for summation $+$.
\end{definition}

\subsection{Name equivalence}

We take name equivalence, written $\nameeq$, to be the smallest
equivalence relation generated by the following rules.

\begin{mathpar}
\inferrule*[lab=Quote-drop]
{ }
{ \quotep{@{x}} \nameeq x }

\inferrule*[lab=Struct-equiv]
{ P \scong Q }
{ \quotep{P} \nameeq \quotep{Q} }
\end{mathpar}

The astute reader will have noticed that the mutual recursion of names
and processes imposes a mutual recursion on alpha-equivalence and
structural equivalence via name-equivalence. Fortunately, all of this
works out pleasantly and we may calculate in the natural way, free of
concern. The reader interested in the details is referred to the
appendix \ref{appendix:rho_details}.

\subsection{Substitution}

We use $\Proc$ for the set of processes, $\QProc$ for the set of
names, and $\id{\{}\vec{y} / \vec{x} \id{\}}$ to denote partial maps,
$s : \QProc \rightarrow \QProc$. A map, $s$ lifts, uniquely, to a map
on process terms, $\widehat{s} : \Proc \rightarrow \Proc$ by the
following equations.

\begin{mathpar}
  (0) \psubstp{Q}{P} := 0 \\
  (R \juxtap S) \psubstp{Q}{P}
  :=    
  (R)\psubstp{Q}{P} \juxtap (S) \psubstp{Q}{P} \\
  (x?(y).R) \psubstp{Q}{P}    
  :=    
  (x)\substp{Q}{P} (z)\concat( (R \psubstn{z}{y}) \psubstp{Q}{P} ) \\
  (\lift{x}{R}) \psubstp{Q}{P}  
  :=
  \lift{(x)\substp{Q}{P}}{ R \psubstp{Q}{P} } \\
%   (\dropn{x})  \psubstp{Q}{P}       
%   := 
%   \left\{ 
%     \begin{array}{ccc} 
%       \dropn{\quotep{Q}} & & x \nameeq \quotep{P} \\
%       \dropn{x} & & otherwise \\
%     \end{array}
%   \right. 
  (\dropn{x})  \psubstp{Q}{P}       
  := 
  \left\{ 
    \begin{array}{ccc} 
      Q & & x \nameeq \quotep{P} \\
      \dropn{x} & & otherwise \\
    \end{array}
  \right.
\end{mathpar}
 

where

\begin{eqnarray}
  (x)\id{\{} \lpquote Q \rpquote / \lpquote P \rpquote \id{\}}            = 
  \left\{ 
    \begin{array}{ccc}
      \lpquote Q \rpquote & & x \nameeq \lpquote P \rpquote \\
      x & & otherwise \\
    \end{array}
  \right. \nonumber
\end{eqnarray}

and $z$ is chosen distinct from $\quotep{P}$, $\quotep{Q}$, the free
names in $Q$, and all the names in $R$. Our $\alpha$-equivalence will
be built in the standard way from this substitution.

\begin{remark}\label{rem:no_self_referential_names}
  One consequence of these definitions is that $\forall P. \quotep{P}
  \not\in \freenames{P}$.
\end{remark}

\subsection{ Dynamic quote: an example }

Anticipating something of what's to come, consider applying the
substitution, $\widehat{\id{\{}u / z \id{\}}}$, to the following pair
of processes, $\lift{w}{y!(z)}$ and $w[ \lpquote y!(z) \rpquote ]$.

\begin{eqnarray}
	\lift{w}{y!(z)}\widehat{\id{\{}u / z \id{\}}}
		& = &
		\lift{w}{y!(u)} \nonumber\\
	w[ \lpquote y!(z) \rpquote ] \widehat{ \id{\{}u / z \id{\}} }
		& = &
		w[ \lpquote y!(z) \rpquote ] \nonumber
\end{eqnarray}

Because the body of the process between quotes is impervious to
substitution, we get radically different answers. In fact, by
examining the first process in an input context,
e.g. $x?(z).\lift{w}{y!(z)}$, we see that the process under the lift
operator may be shaped by prefixed inputs binding a name inside it. In
this sense, the lift operator will be seen as a way to dynamically
construct processes before reifying them as names.

Finally equipped with these standard features we can present the
dynamics of the calculus.

\subsubsection{Operational semantics} 

Finally, we introduce the computational dynamics. What marks these
algebras as distinct from other more traditionally studied algebraic
structures, e.g. vector spaces or polynomial rings, is the manner in
which dynamics is captured. In traditional structures, dynamics is typically
expressed through morphisms between such structures, as in linear maps
between vector spaces or morphisms between rings. In algebras
associated with the semantics of computation, the dynamics is
expressed as part of the algebraic structure itself, through a
reduction reduction relation typically denoted by $\red$. Below, we
give a recursive presentation of this relation for the calculus used
in the encoding.

$\red \subseteq \pi \times \pi$
$\red : \pi \to \mathcal{P}(\pi)$

\begin{mathpar}
  \inferrule* [lab=Comm] { \textsf{match}( x_{src}, x_{trgt} ) } { x_{trgt}?(y)P \; | \; x_{src}!\langle {Q} \rangle \red P\{\quotep{Q}/y}\} }
  \and \\
  \inferrule* [lab=Par] {{P} \red {P}'} {{{P} | {Q}} \red {{P}' | {Q}}}
  \and
  \inferrule* [lab=Equiv]{{{P} \scong {P}'} \andalso {{P}' \red {Q}'} \andalso {{Q}' \scong {Q}}}{{P} \red {Q}}
\end{mathpar}

\begin{eqnarray*}
  match_{\equiv} (\quotep{P},\quotep{Q}) & := & P \equiv Q \\
  match_{\dagger}(\quotep{P},\quotep{Q}) & := & \forall R. P|Q \red^{*} R => R \red^{*} 0 \\
  match_{K}(\quotep{P},\quotep{Q}) & := & K \mbox{ for some context } K
\end{eqnarray*}

$u?(x)P | u!\langle Q \rangle \red P\{\quotep{Q}/x\}$

%We write $\wred$ for $\red^*$, and $P\red$ if $\exists Q $ such that $ P \red Q$.
We write $P\red$ if $\exists Q $ such that $ P \red Q$ and $P\not\red$, otherwise.

\section{Replication}

As mentioned before, it is known that replication (and hence
recursion) can be implemented in a higher-order process algebra
\cite{SangiorgiWalker}. As our first example of calculation with the
machinery thus far presented we give the construction explicitly in
the {\rhoc}.

\begin{eqnarray}
	D_{x} & := & \prefix{x}{y}{(\binpar{\outputp{x}{y}}{@{y}})} \nonumber\\
	\bangp_{x}{P} & := & \binpar{{x}!\langle{\binpar{D_{x}}{P}}\rangle}{D_{x}} \nonumber
\end{eqnarray}

\begin{eqnarray}
	\bangp_{x}{P} & & \nonumber\\
	=
	& {x}!\langle{(\prefix{x}{y}{(\outputp{x}{y} | @{y})) | P}}\rangle 
	      | \prefix{x}{y}{(\outputp{x}{y} | @{y})} & \nonumber\\
	\red
	& (\outputp{x}{y} | @{y})\substn{\quotep{(\prefix{x}{y}{(@{y} | \outputp{x}{y})) | P}}}{y} & \nonumber\\
	=
	& \outputp{x}{\quotep{(\prefix{x}{y}{(\outputp{x}{y} | @{y})) | P}}}
	  | {(\prefix{x}{y}{(\outputp{x}{y} | @{y})) | P}} & \nonumber\\
	\red
	& \ldots & \nonumber\\
	\red^*
	& P | P | \ldots & \nonumber
\end{eqnarray}

Of course, this encoding, as an implementation, runs away, unfolding
$\bangp{P}$ eagerly. A lazier and more implementable replication
operator, restricted to input-guarded processes, may be obtained as follows.

\begin{eqnarray}
\bangp{\prefix{u}{v}{P}} 
	:= 
	\binpar{\lift{x}{\prefix{u}{v}{(\binpar{D(x)}{P})}}}{D(x)} \nonumber
\end{eqnarray}

\begin{remark}
  Note that the lazier definition still does not deal with summation
  or mixed summation (i.e. sums over input and output). The reader is
  invited to construct definitions of replication that deal with these
  features. 

  Further, the definitions are parameterized in a name, $x$. Can you,
  gentle reader, make a definition that eliminates this parameter and
  guarantees no accidental interaction between the replication
  machinery and the process being replicated -- i.e. no accidental
  sharing of names used by the process to get its work done and the
  name(s) used by the replication to effect copying. This latter
  revision of the definition of replication is crucial to obtaining
  the expected identity $!!P \sim !P$.
\end{remark}

\begin{remark}\label{rem:paradoxical_combinator}
  The reader familiar with the lambda calculus will have noticed the
  similarity between $D$ and the paradoxical combinator.

  [Ed. note: the existence of this seems to suggest we have to be more
  restrictive on the set of processes and names we admit if we are to
  support no-cloning.]
\end{remark}

\subsubsection{Bisimulation}

The computational dynamics gives rise to another kind of equivalence,
the equivalence of computational behavior. As previously mentioned
this is typically captured \emph{via} some form of bisimulation.

% The notion we use in this paper is weak barbed bisimulation
% \cite{milner91polyadicpi}.

The notion we use in this paper is derived from weak barbed
bisimulation \cite{milner91polyadicpi}. 

\begin{definition}
An \emph{observation relation}, $\downarrow_{\mathcal N}$, over a set
of names, $\mathcal N$, is the smallest relation satisfying the rules
below.

\infrule[Out-barb]{y \in {\mathcal N}, \; x \nameeq y}
		  {\outputp{x}{v} \downarrow_{\mathcal N} x}
\infrule[Par-barb]{\mbox{$P\downarrow_{\mathcal N} x$ or $Q\downarrow_{\mathcal N} x$}}
		  {\binpar{P}{Q} \downarrow_{\mathcal N} x}

We write $P \Downarrow_{\mathcal N} x$ if there is $Q$ such that 
$P \wred Q$ and $Q \downarrow_{\mathcal N} x$.
\end{definition}

\begin{definition}
%\label{def.bbisim}
An  ${\mathcal N}$-\emph{barbed bisimulation} over a set of names, ${\mathcal N}$, is a symmetric binary relation 
${\mathcal S}_{\mathcal N}$ between agents such that $P\rel{S}_{\mathcal N}Q$ implies:
\begin{enumerate}
\item If $P \red P'$ then $Q \wred Q'$ and $P'\rel{S}_{\mathcal N} Q'$.
\item If $P\downarrow_{\mathcal N} x$, then $Q\Downarrow_{\mathcal N} x$.
\end{enumerate}
$P$ is ${\mathcal N}$-barbed bisimilar to $Q$, written
$P \wbbisim_{\mathcal N} Q$, if $P \rel{S}_{\mathcal N} Q$ for some ${\mathcal N}$-barbed bisimulation ${\mathcal S}_{\mathcal N}$.
\end{definition}

$\mathcal{R} \subseteq \pi \times \pi$

$P \mathcal{R} Q => \forall P'. P \red P' \Rightarrow \exists Q'. Q \red Q', P' \mathcal{R} Q'$

$P \vdash x \Rightarrow Q \vdash x$

\begin{mathpar}
  \inferrule*[lab=Out-barb]{x \nameeq y}{{y}!\langle{Q}\rangle \vdash x}
  \and
  \inferrule*[lab=Par-barb]{\mbox{$P\vdash x$ or $Q\vdash x$}}{\binpar{P}{Q} \vdash x}
\end{mathpar}

\subsubsection{Contexts}

One of the principle advantages of computational calculi like the
$\pi$-calculus is a well-defined notion of context,
contextual-equivalence and a correlation between
contextual-equivalence and notions of bisimulation. The notion of
context allows the decomposition of a process into (sub-)process and
its syntactic environment, its context. Thus, a context may be
thought of as a process with a ``hole'' (written $\Box$) in it. The
application of a context $M$ to a process $P$, written $M[P]$, is
tantamount to filling the hole in $M$ with $P$. In this paper we do
not need the full weight of this theory, but do make use of the notion
of context in the proof the main theorem. 

\begin{mathpar}
  \inferrule* [lab=summation] {} {{M_{M},M_{N}} \bc \Box \;|\; x.M_{A} \;|\; M_{M}+M_{N}}
  \and
  \inferrule* [lab=agent] {} {{M_{A}} \bc (\vec{x})M_{P} \;| \; \clift{P_0,\ldots,M_{P},\ldots,P_N}}
  \and \\
  \inferrule* [lab=process] {} {{M_{P}} \bc M_{N} \;| \;P|M_{P} }
\end{mathpar} 

\begin{mathpar}
  \inferrule* [lab=sychronization] {} {M_{N} \bc \Box \;|\; x?M_{F} \;|\; x!M_{C}}
  \and
  \inferrule* [lab=abstraction] {} {{M_{F}} \bc (x)M_{P} }
  \and
  \inferrule* [lab=concretion] {} {{M_{C}} \bc \langle M_{P} \rangle }
  \and \\
  \inferrule* [lab=process] {} {{M_{P}} \bc M_{N} \;| \;P|M_{P} }
\end{mathpar}

\begin{definition}[contextual application] Given a context $M$, and
  process $P$, we define the \emph{contextual application}, $M[P] :=
  M\{P/\Box\}$. That is, the contextual application of M to P is the
  substitution of $P$ for $\Box$ in $M$.
\end{definition}

$\meaningof{-} : L \to \mathcal{P}(\pi)$

\begin{mathpar}
  \inferrule* [lab=collection] {} {\meaningof{true} = \pi, \and \meaningof{~E} = \pi \setminus \meaningof{E}, \and \meaningof{E_{1} \& E_{2}} = \meaningof{E_{1}} \cap \meaningof{E_{2}}}
\end{mathpar}

\begin{mathpar}
  \inferrule* [lab=structure] {} {\meaningof{0} = \{ P \in \pi | P \equiv 0 \}, \and \\ \meaningof{E_1 | E_2} = \{ P \in \pi | P \equiv P_{1} | P_{2}, P_{1} \in \meaningof{E_{1}}, P_{2} \in \meaningof{E_2}\} }
\end{mathpar}

\begin{mathpar}
 \inferrule* [lab=behavior] {} {\meaningof{\langle a?b \rangle E} = \{ P \in \pi | P \equiv Q | u?(y)P', \\ \and \\\\ \and \\ \;\;\; u \in \meaningof{a}, \forall z.P'\{z/y\} \in \meaningof{E\{z/b\}}\}, \and \\ \meaningof{a!E} = \{ P \in \pi | P \equiv Q | x!\langle P' \rangle, x \in \meaningof{a} P' \in \meaningof{E}\} }
\end{mathpar}

\begin{mathpar}
 \inferrule* [lab=nominal] {} {\meaningof{\quotep{E}} = \{ \quotep{P} \in \quotep{\pi} | P \in \meaningof{E} \}, \and \meaningof{\quotep{P}} = \{ \quotep{Q} \in \quotep{\pi} | P \equiv Q \} \and \\ \meaningof{@\quotep{E}} = \{ P \in \pi | P \equiv @x, x \in \meaningof{E} \}}
\end{mathpar}

\begin{eqnarray*}
  \\
  \meaningof{-} : TS \to ST
\end{eqnarray*}

\begin{eqnarray*}
  \\
  L : TS \to ST
\end{eqnarray*}

\begin{eqnarray*}
  \\
  P \models E \iff P \in \meaningof{E}
\end{eqnarray*}

\begin{eqnarray*}
  P \approx_{L} Q \iff \forall E \in L. P \models E \iff Q \models E
\end{eqnarray*}

\begin{eqnarray*}
  P \approx_{K} Q
\end{eqnarray*}

\begin{eqnarray*}
  P \approx Q
\end{eqnarray*}

$\approx_{K} = \approx = \approx_{L}$

\subsubsection{Contextual duality}

Note that contexts extend the quotation operation to a family of
operations from processes to names. Given a context, $M$, we can
define a \emph{nominal context}, $\quotep{M}$ by $\quotep{M}[P] :=
\quotep{M[P]}$. To foreshadow what is to come we observe that these
operations enjoy a duality with processes very much like the duality
between vectors and maps from vectors to scalars.

Further, because the calculus is essentially higher-order, we have a
correspondence between contexts and processes. More specifically,
given a name $x$ and a context $M$ we can construct $M^{*}_{x}$ such
that 

\begin{mathpar}
  M^{*}_{x} | \lift{x}{P} \red M[P]
\end{mathpar}

namely,

\begin{mathpar}
  M^{*}_{x} := x?(u).M[\dropn{u}]
\end{mathpar}

The dependence of $M^{*}_{x}$ on a name makes it an abstraction, 

\begin{mathpar}
  M^{*} := (x)x?(u).M[\dropn{u}]
\end{mathpar}

\subsection{Additional notation}

It will sometimes be convenient to denote the process a name
quotes. We already have the notation $x = \quotep{P}$, but it will be
convenient to introduce an alternate notation, $\procn{x}$, when we
want to emphasize the connection to the use of the name. Note that, by
virtue of name equivalence, $\quotep{\procn{x}} \nameeq x$; so, the
notation is consistent with previous definitions.

Further, because names have structure it is possible to effect
substitutions on the basis of that structure. This means we need to
upgrade our notation for substitutions, which we accomplish by
adapting comprehension notation. Thus,

\begin{mathpar}
  P\{ y / x : x \in S \}
\end{mathpar}

is interpreted to mean the process derived from P by replacing (in a
capture-avoiding manner) each occurrence of $x$ in $S$ by $y$. For example,

\begin{mathpar}
  P\{ \quotep{\procn{x}|\procn{x}} / x : x \in \freenames{P} \}
\end{mathpar}

will replace each (occurrence) of a free name $x$ in $P$ by
$\quotep{\procn{x}|\procn{x}}$.

Also, we will avail ourselves of the notation $x^{L}$ and $x^{R}$ to
denote injections of a name into disjoint copies of the name
space. There are numerous ways to accomplish this. One example can be
found in \cite{MeredithR05}. This notation overloads to vectors of
names: $\vec{x}^{\pi} := (x_{i}^{\pi} \; : \; 0 \leq i < |\vec{x}| )$ where $\pi \in \{L,R\}$.

We also use $P^{\Box} := P|\Box$.

In \cite{MeredithR05} an interpretation of the new operator is
given. It turns out that there are several possible interpretations
all enjoying the requisite algebraic properties of the operator (see
\cite{milner91polyadicpi}). We will therefore make liberal use of
$(\nu\; \vec{x})P$.

% subsection the_syntax_and_semantics_of_the_notation_system (end)   

\input{qm2pi.qmops} 

\input{qm2pi.sterngerlach} 

\input{qm2pi.metric} 

% section concurrent_process_calculi (end)

%\input{qm2pi.proofsketch}

% section proof sketch (end)

%\input{qm2pi.slviaknots} 

% section spatial logic via knots (end)

\input{qm2pi.conclusion}

% section conclusion (end)

%\input{qm2pi.dtcodes} 

% section wiring algorithm (end)

\input{qm2pi.ack} 

% section acknowledgments (end)

\newpage


\bibliographystyle{plain}   
\bibliography{../../biblios/main.bib}

\input{qm2pi.rhodetails}

\end{document}

 

\documentclass[12pt]{llncs}
%\documentclass{jktr}

\usepackage[pdftex]{hyperref}                   
\usepackage {listings}
\usepackage {mathpartir}
\usepackage{bcprules}
%\usepackage{listings}
                       
\usepackage{graphicx} 
%\usepackage[margins=2.5cm,nohead,nofoot]{geometry}
%\usepackage{geometry}
\usepackage{amsfonts}
\usepackage{amstext}
\usepackage{latexsym}
\usepackage{amssymb}
\usepackage{color}


%\include{myPreamble}
\include{qm2pi.local} 

%\ifpdf
%\usepackage[pdftex]{graphicx}
%\else
%\usepackage{graphicx}
%\fi

 % \ifpdf
%  \usepackage{pdfsync}
%  \if


%\title{Brief Article}
%\author{David F. Snyder}
%\author{L.G. Meredith}

%\address{Dept. of Math., Texas State University--San Marcos, San Marcos, TX 78666}
       
\pagestyle{empty}


\begin{document}

\lstset{language=[Objective]Caml,frame=shadowbox}

\input{qm2pi.front}

% section front matter (end)

\input{qm2pi.intro} 
 
% section introduction (end)

% \input{qm2pi.knotations} 

% section notation (end)

\input{qm2pi.process.calculi} 

% section concurrent_process_calculi_and_spatial_logics_ (end)
    
%\input{qm2pi.knots2pi} 

%\input{qm2pi.trefoil} 

%\input{qm2pi.mainthm} 

% subsection basic_interpretation (end)

%\input{qm2pi.rho.presentation} 
\subsection{The syntax and semantics of the notation system}\label{sub:the_syntax_and_semantics_of_the_notation_system} % (fold)

We now summarize a technical presentation of the calculus that
embodies our theory of dynamics. The typical presentation of such a
calculus follows the style of giving generators and relations on
them. The grammar, below, describing term constructors, freely
generates the set of processes, $\Proc$. This set is then quotiented
by a relation known as structural congruence and it is over this set
that the notion of dynamics is expressed. This presentation is
essentially that of \cite{MeredithR05} with the addition of
polyadicity and summation. For readability we have relegated some of
the technical subtleties to an appendix.

\subsubsection{Process grammar}\label{subsub:process_grammar}

\begin{mathpar}
  \inferrule* [lab=synchronization] {} {{M} \bc \pzero \;|\; x?F \;|\; x!C }
  \and
  \inferrule* [lab=abstraction] {} {{F} \bc (x)P}
  \and
  \inferrule* [lab=concretion] {} {{C} \bc \langle Q \rangle}
  \and
  \inferrule* [lab=process] {} {{P,Q} \bc M \;| \;P|Q \;|\; @{x}}
  \and
  \inferrule* [lab=name] {} {{x} \bc \quotep{P}}
\end{mathpar} 

Note that $\vec{x}$ (resp. $\vec{P}$) denotes a vector of names
(resp. processes) of length $|\vec{x}|$ (resp. $|\vec{P}|$). We adopt
the following useful abbreviations.

\begin{mathpar}
   x?(\vec{y}).P := x.(\vec{y})P \and  x\clift{\vec{P}} := x.\clift{\vec{P}}
   \and x!(y) := \lift{x}{\dropn{y}}
   \and \Pi_{i=0}^{n-1}P_i := P_0 | \ldots | P_{n-1}
\end{mathpar}

\subsubsection{Structural congruence}

\paragraph{Free and bound names and alpha-equivalence.} At the
core of structural equivalence is alpha-equivalence which identifies
process that are the same up to a change of variable. Formally, we
recognize the distinction between free and bound names. The free names
of a process, $\freenames{P}$, may be calculated recursively as
follows:

\begin{mathpar}
\freenames{\pzero} := \emptyset
  \and \\
  \freenames{x?(y).P} := \{ x \} \cup (\freenames{P} \setminus \{ y \})
  \and 
  \freenames{x!\langle P \rangle} := \{ x \} \cup \{ P \} 
  \and \\
  \freenames{P|Q} := \freenames{P} \cup \freenames{Q}
  \and \\
  \freenames{@{x}} := \{ x \}
\end{mathpar}

$\pi$
$\quotep{\pi}$

$\freenames{-} : \pi \to \mathcal{P}(\quotep{\pi})$

\begin{eqnarray*}
  \freenames{\pzero} & := & \emptyset \\
  \freenames{x?(y).P} & := & \{ x \} \cup (\freenames{P} \setminus \{ y \}) \\
  \freenames{x!\langle P \rangle} & := & \{ x \} \cup \{ P \} \\
  \freenames{P|Q} & := & \freenames{P} \cup \freenames{Q} \\
  \freenames{\dropn{x}} & := & \{ x \}
\end{eqnarray*}

The bound names of a process, $\boundnames{P}$, are those names occurring in $P$
that are not free. For example, in $x?(y).0$, the name $x$ is free, while $y$ is bound.

\begin{mathpar}
  \inferrule* [lab=monoidal-laws] {} { P|Q \equiv Q|P \and P|0 \equiv P \and P|(Q|R) \equiv (P|Q)|R }
\end{mathpar}

\begin{mathpar}
  \inferrule* [lab=alpha-equivalence] {} { (x)P \equiv (y)P\{y/x\} \and y \not\in \freenames{P} }
\end{mathpar}

\begin{definition}
Then two processes, $P,Q$, are alpha-equivalent if $P = Q\{\vec{y}/\vec{x}\}$ for
some $\vec{x} \in \boundnames{Q},\vec{y} \in \boundnames{P}$, where $Q\{\vec{y}/\vec{x}\}$
denotes the capture-avoiding substitution of $\vec{y}$ for $\vec{x}$ in $Q$.
\end{definition}

\begin{definition}
  The {\em structural congruence} \cite{SangiorgiWalker} , $\equiv$,
  between processes is the least congruence containing
  alpha-equivalence, satisfying the abelian monoid laws
  (associativity, commutativity and $\pzero$ as identity) for parallel
  composition $|$ and for summation $+$.
\end{definition}

\subsection{Name equivalence}

We take name equivalence, written $\nameeq$, to be the smallest
equivalence relation generated by the following rules.

\begin{mathpar}
\inferrule*[lab=Quote-drop]
{ }
{ \quotep{@{x}} \nameeq x }

\inferrule*[lab=Struct-equiv]
{ P \scong Q }
{ \quotep{P} \nameeq \quotep{Q} }
\end{mathpar}

The astute reader will have noticed that the mutual recursion of names
and processes imposes a mutual recursion on alpha-equivalence and
structural equivalence via name-equivalence. Fortunately, all of this
works out pleasantly and we may calculate in the natural way, free of
concern. The reader interested in the details is referred to the
appendix \ref{appendix:rho_details}.

\subsection{Substitution}

We use $\Proc$ for the set of processes, $\QProc$ for the set of
names, and $\id{\{}\vec{y} / \vec{x} \id{\}}$ to denote partial maps,
$s : \QProc \rightarrow \QProc$. A map, $s$ lifts, uniquely, to a map
on process terms, $\widehat{s} : \Proc \rightarrow \Proc$ by the
following equations.

\begin{mathpar}
  (0) \psubstp{Q}{P} := 0 \\
  (R \juxtap S) \psubstp{Q}{P}
  :=    
  (R)\psubstp{Q}{P} \juxtap (S) \psubstp{Q}{P} \\
  (x?(y).R) \psubstp{Q}{P}    
  :=    
  (x)\substp{Q}{P} (z)\concat( (R \psubstn{z}{y}) \psubstp{Q}{P} ) \\
  (\lift{x}{R}) \psubstp{Q}{P}  
  :=
  \lift{(x)\substp{Q}{P}}{ R \psubstp{Q}{P} } \\
%   (\dropn{x})  \psubstp{Q}{P}       
%   := 
%   \left\{ 
%     \begin{array}{ccc} 
%       \dropn{\quotep{Q}} & & x \nameeq \quotep{P} \\
%       \dropn{x} & & otherwise \\
%     \end{array}
%   \right. 
  (\dropn{x})  \psubstp{Q}{P}       
  := 
  \left\{ 
    \begin{array}{ccc} 
      Q & & x \nameeq \quotep{P} \\
      \dropn{x} & & otherwise \\
    \end{array}
  \right.
\end{mathpar}
 

where

\begin{eqnarray}
  (x)\id{\{} \lpquote Q \rpquote / \lpquote P \rpquote \id{\}}            = 
  \left\{ 
    \begin{array}{ccc}
      \lpquote Q \rpquote & & x \nameeq \lpquote P \rpquote \\
      x & & otherwise \\
    \end{array}
  \right. \nonumber
\end{eqnarray}

and $z$ is chosen distinct from $\quotep{P}$, $\quotep{Q}$, the free
names in $Q$, and all the names in $R$. Our $\alpha$-equivalence will
be built in the standard way from this substitution.

\begin{remark}\label{rem:no_self_referential_names}
  One consequence of these definitions is that $\forall P. \quotep{P}
  \not\in \freenames{P}$.
\end{remark}

\subsection{ Dynamic quote: an example }

Anticipating something of what's to come, consider applying the
substitution, $\widehat{\id{\{}u / z \id{\}}}$, to the following pair
of processes, $\lift{w}{y!(z)}$ and $w[ \lpquote y!(z) \rpquote ]$.

\begin{eqnarray}
	\lift{w}{y!(z)}\widehat{\id{\{}u / z \id{\}}}
		& = &
		\lift{w}{y!(u)} \nonumber\\
	w[ \lpquote y!(z) \rpquote ] \widehat{ \id{\{}u / z \id{\}} }
		& = &
		w[ \lpquote y!(z) \rpquote ] \nonumber
\end{eqnarray}

Because the body of the process between quotes is impervious to
substitution, we get radically different answers. In fact, by
examining the first process in an input context,
e.g. $x?(z).\lift{w}{y!(z)}$, we see that the process under the lift
operator may be shaped by prefixed inputs binding a name inside it. In
this sense, the lift operator will be seen as a way to dynamically
construct processes before reifying them as names.

Finally equipped with these standard features we can present the
dynamics of the calculus.

\subsubsection{Operational semantics} 

Finally, we introduce the computational dynamics. What marks these
algebras as distinct from other more traditionally studied algebraic
structures, e.g. vector spaces or polynomial rings, is the manner in
which dynamics is captured. In traditional structures, dynamics is typically
expressed through morphisms between such structures, as in linear maps
between vector spaces or morphisms between rings. In algebras
associated with the semantics of computation, the dynamics is
expressed as part of the algebraic structure itself, through a
reduction reduction relation typically denoted by $\red$. Below, we
give a recursive presentation of this relation for the calculus used
in the encoding.

$\red \subseteq \pi \times \pi$
$\red : \pi \to \mathcal{P}(\pi)$

\begin{mathpar}
  \inferrule* [lab=Comm] { \textsf{match}( x_{src}, x_{trgt} ) } { x_{trgt}?(y)P \; | \; x_{src}!\langle {Q} \rangle \red P\{\quotep{Q}/y}\} }
  \and \\
  \inferrule* [lab=Par] {{P} \red {P}'} {{{P} | {Q}} \red {{P}' | {Q}}}
  \and
  \inferrule* [lab=Equiv]{{{P} \scong {P}'} \andalso {{P}' \red {Q}'} \andalso {{Q}' \scong {Q}}}{{P} \red {Q}}
\end{mathpar}

\begin{eqnarray*}
  match_{\equiv} (\quotep{P},\quotep{Q}) & := & P \equiv Q \\
  match_{\dagger}(\quotep{P},\quotep{Q}) & := & \forall R. P|Q \red^{*} R => R \red^{*} 0 \\
  match_{K}(\quotep{P},\quotep{Q}) & := & K \mbox{ for some context } K
\end{eqnarray*}

$u?(x)P | u!\langle Q \rangle \red P\{\quotep{Q}/x\}$

%We write $\wred$ for $\red^*$, and $P\red$ if $\exists Q $ such that $ P \red Q$.
We write $P\red$ if $\exists Q $ such that $ P \red Q$ and $P\not\red$, otherwise.

\section{Replication}

As mentioned before, it is known that replication (and hence
recursion) can be implemented in a higher-order process algebra
\cite{SangiorgiWalker}. As our first example of calculation with the
machinery thus far presented we give the construction explicitly in
the {\rhoc}.

\begin{eqnarray}
	D_{x} & := & \prefix{x}{y}{(\binpar{\outputp{x}{y}}{@{y}})} \nonumber\\
	\bangp_{x}{P} & := & \binpar{{x}!\langle{\binpar{D_{x}}{P}}\rangle}{D_{x}} \nonumber
\end{eqnarray}

\begin{eqnarray}
	\bangp_{x}{P} & & \nonumber\\
	=
	& {x}!\langle{(\prefix{x}{y}{(\outputp{x}{y} | @{y})) | P}}\rangle 
	      | \prefix{x}{y}{(\outputp{x}{y} | @{y})} & \nonumber\\
	\red
	& (\outputp{x}{y} | @{y})\substn{\quotep{(\prefix{x}{y}{(@{y} | \outputp{x}{y})) | P}}}{y} & \nonumber\\
	=
	& \outputp{x}{\quotep{(\prefix{x}{y}{(\outputp{x}{y} | @{y})) | P}}}
	  | {(\prefix{x}{y}{(\outputp{x}{y} | @{y})) | P}} & \nonumber\\
	\red
	& \ldots & \nonumber\\
	\red^*
	& P | P | \ldots & \nonumber
\end{eqnarray}

Of course, this encoding, as an implementation, runs away, unfolding
$\bangp{P}$ eagerly. A lazier and more implementable replication
operator, restricted to input-guarded processes, may be obtained as follows.

\begin{eqnarray}
\bangp{\prefix{u}{v}{P}} 
	:= 
	\binpar{\lift{x}{\prefix{u}{v}{(\binpar{D(x)}{P})}}}{D(x)} \nonumber
\end{eqnarray}

\begin{remark}
  Note that the lazier definition still does not deal with summation
  or mixed summation (i.e. sums over input and output). The reader is
  invited to construct definitions of replication that deal with these
  features. 

  Further, the definitions are parameterized in a name, $x$. Can you,
  gentle reader, make a definition that eliminates this parameter and
  guarantees no accidental interaction between the replication
  machinery and the process being replicated -- i.e. no accidental
  sharing of names used by the process to get its work done and the
  name(s) used by the replication to effect copying. This latter
  revision of the definition of replication is crucial to obtaining
  the expected identity $!!P \sim !P$.
\end{remark}

\begin{remark}\label{rem:paradoxical_combinator}
  The reader familiar with the lambda calculus will have noticed the
  similarity between $D$ and the paradoxical combinator.

  [Ed. note: the existence of this seems to suggest we have to be more
  restrictive on the set of processes and names we admit if we are to
  support no-cloning.]
\end{remark}

\subsubsection{Bisimulation}

The computational dynamics gives rise to another kind of equivalence,
the equivalence of computational behavior. As previously mentioned
this is typically captured \emph{via} some form of bisimulation.

% The notion we use in this paper is weak barbed bisimulation
% \cite{milner91polyadicpi}.

The notion we use in this paper is derived from weak barbed
bisimulation \cite{milner91polyadicpi}. 

\begin{definition}
An \emph{observation relation}, $\downarrow_{\mathcal N}$, over a set
of names, $\mathcal N$, is the smallest relation satisfying the rules
below.

\infrule[Out-barb]{y \in {\mathcal N}, \; x \nameeq y}
		  {\outputp{x}{v} \downarrow_{\mathcal N} x}
\infrule[Par-barb]{\mbox{$P\downarrow_{\mathcal N} x$ or $Q\downarrow_{\mathcal N} x$}}
		  {\binpar{P}{Q} \downarrow_{\mathcal N} x}

We write $P \Downarrow_{\mathcal N} x$ if there is $Q$ such that 
$P \wred Q$ and $Q \downarrow_{\mathcal N} x$.
\end{definition}

\begin{definition}
%\label{def.bbisim}
An  ${\mathcal N}$-\emph{barbed bisimulation} over a set of names, ${\mathcal N}$, is a symmetric binary relation 
${\mathcal S}_{\mathcal N}$ between agents such that $P\rel{S}_{\mathcal N}Q$ implies:
\begin{enumerate}
\item If $P \red P'$ then $Q \wred Q'$ and $P'\rel{S}_{\mathcal N} Q'$.
\item If $P\downarrow_{\mathcal N} x$, then $Q\Downarrow_{\mathcal N} x$.
\end{enumerate}
$P$ is ${\mathcal N}$-barbed bisimilar to $Q$, written
$P \wbbisim_{\mathcal N} Q$, if $P \rel{S}_{\mathcal N} Q$ for some ${\mathcal N}$-barbed bisimulation ${\mathcal S}_{\mathcal N}$.
\end{definition}

$\mathcal{R} \subseteq \pi \times \pi$

$P \mathcal{R} Q => \forall P'. P \red P' \Rightarrow \exists Q'. Q \red Q', P' \mathcal{R} Q'$

$P \vdash x \Rightarrow Q \vdash x$

\begin{mathpar}
  \inferrule*[lab=Out-barb]{x \nameeq y}{{y}!\langle{Q}\rangle \vdash x}
  \and
  \inferrule*[lab=Par-barb]{\mbox{$P\vdash x$ or $Q\vdash x$}}{\binpar{P}{Q} \vdash x}
\end{mathpar}

\subsubsection{Contexts}

One of the principle advantages of computational calculi like the
$\pi$-calculus is a well-defined notion of context,
contextual-equivalence and a correlation between
contextual-equivalence and notions of bisimulation. The notion of
context allows the decomposition of a process into (sub-)process and
its syntactic environment, its context. Thus, a context may be
thought of as a process with a ``hole'' (written $\Box$) in it. The
application of a context $M$ to a process $P$, written $M[P]$, is
tantamount to filling the hole in $M$ with $P$. In this paper we do
not need the full weight of this theory, but do make use of the notion
of context in the proof the main theorem. 

\begin{mathpar}
  \inferrule* [lab=summation] {} {{M_{M},M_{N}} \bc \Box \;|\; x.M_{A} \;|\; M_{M}+M_{N}}
  \and
  \inferrule* [lab=agent] {} {{M_{A}} \bc (\vec{x})M_{P} \;| \; \clift{P_0,\ldots,M_{P},\ldots,P_N}}
  \and \\
  \inferrule* [lab=process] {} {{M_{P}} \bc M_{N} \;| \;P|M_{P} }
\end{mathpar} 

\begin{mathpar}
  \inferrule* [lab=sychronization] {} {M_{N} \bc \Box \;|\; x?M_{F} \;|\; x!M_{C}}
  \and
  \inferrule* [lab=abstraction] {} {{M_{F}} \bc (x)M_{P} }
  \and
  \inferrule* [lab=concretion] {} {{M_{C}} \bc \langle M_{P} \rangle }
  \and \\
  \inferrule* [lab=process] {} {{M_{P}} \bc M_{N} \;| \;P|M_{P} }
\end{mathpar}

\begin{definition}[contextual application] Given a context $M$, and
  process $P$, we define the \emph{contextual application}, $M[P] :=
  M\{P/\Box\}$. That is, the contextual application of M to P is the
  substitution of $P$ for $\Box$ in $M$.
\end{definition}

$\meaningof{-} : L \to \mathcal{P}(\pi)$

\begin{mathpar}
  \inferrule* [lab=collection] {} {\meaningof{true} = \pi, \and \meaningof{~E} = \pi \setminus \meaningof{E}, \and \meaningof{E_{1} \& E_{2}} = \meaningof{E_{1}} \cap \meaningof{E_{2}}}
\end{mathpar}

\begin{mathpar}
  \inferrule* [lab=structure] {} {\meaningof{0} = \{ P \in \pi | P \equiv 0 \}, \and \\ \meaningof{E_1 | E_2} = \{ P \in \pi | P \equiv P_{1} | P_{2}, P_{1} \in \meaningof{E_{1}}, P_{2} \in \meaningof{E_2}\} }
\end{mathpar}

\begin{mathpar}
 \inferrule* [lab=behavior] {} {\meaningof{\langle a?b \rangle E} = \{ P \in \pi | P \equiv Q | u?(y)P', \\ \and \\\\ \and \\ \;\;\; u \in \meaningof{a}, \forall z.P'\{z/y\} \in \meaningof{E\{z/b\}}\}, \and \\ \meaningof{a!E} = \{ P \in \pi | P \equiv Q | x!\langle P' \rangle, x \in \meaningof{a} P' \in \meaningof{E}\} }
\end{mathpar}

\begin{mathpar}
 \inferrule* [lab=nominal] {} {\meaningof{\quotep{E}} = \{ \quotep{P} \in \quotep{\pi} | P \in \meaningof{E} \}, \and \meaningof{\quotep{P}} = \{ \quotep{Q} \in \quotep{\pi} | P \equiv Q \} \and \\ \meaningof{@\quotep{E}} = \{ P \in \pi | P \equiv @x, x \in \meaningof{E} \}}
\end{mathpar}

\begin{eqnarray*}
  \\
  \meaningof{-} : TS \to ST
\end{eqnarray*}

\begin{eqnarray*}
  \\
  L : TS \to ST
\end{eqnarray*}

\begin{eqnarray*}
  \\
  P \models E \iff P \in \meaningof{E}
\end{eqnarray*}

\begin{eqnarray*}
  P \approx_{L} Q \iff \forall E \in L. P \models E \iff Q \models E
\end{eqnarray*}

\begin{eqnarray*}
  P \approx_{K} Q
\end{eqnarray*}

\begin{eqnarray*}
  P \approx Q
\end{eqnarray*}

$\approx_{K} = \approx = \approx_{L}$

\subsubsection{Contextual duality}

Note that contexts extend the quotation operation to a family of
operations from processes to names. Given a context, $M$, we can
define a \emph{nominal context}, $\quotep{M}$ by $\quotep{M}[P] :=
\quotep{M[P]}$. To foreshadow what is to come we observe that these
operations enjoy a duality with processes very much like the duality
between vectors and maps from vectors to scalars.

Further, because the calculus is essentially higher-order, we have a
correspondence between contexts and processes. More specifically,
given a name $x$ and a context $M$ we can construct $M^{*}_{x}$ such
that 

\begin{mathpar}
  M^{*}_{x} | \lift{x}{P} \red M[P]
\end{mathpar}

namely,

\begin{mathpar}
  M^{*}_{x} := x?(u).M[\dropn{u}]
\end{mathpar}

The dependence of $M^{*}_{x}$ on a name makes it an abstraction, 

\begin{mathpar}
  M^{*} := (x)x?(u).M[\dropn{u}]
\end{mathpar}

\subsection{Additional notation}

It will sometimes be convenient to denote the process a name
quotes. We already have the notation $x = \quotep{P}$, but it will be
convenient to introduce an alternate notation, $\procn{x}$, when we
want to emphasize the connection to the use of the name. Note that, by
virtue of name equivalence, $\quotep{\procn{x}} \nameeq x$; so, the
notation is consistent with previous definitions.

Further, because names have structure it is possible to effect
substitutions on the basis of that structure. This means we need to
upgrade our notation for substitutions, which we accomplish by
adapting comprehension notation. Thus,

\begin{mathpar}
  P\{ y / x : x \in S \}
\end{mathpar}

is interpreted to mean the process derived from P by replacing (in a
capture-avoiding manner) each occurrence of $x$ in $S$ by $y$. For example,

\begin{mathpar}
  P\{ \quotep{\procn{x}|\procn{x}} / x : x \in \freenames{P} \}
\end{mathpar}

will replace each (occurrence) of a free name $x$ in $P$ by
$\quotep{\procn{x}|\procn{x}}$.

Also, we will avail ourselves of the notation $x^{L}$ and $x^{R}$ to
denote injections of a name into disjoint copies of the name
space. There are numerous ways to accomplish this. One example can be
found in \cite{MeredithR05}. This notation overloads to vectors of
names: $\vec{x}^{\pi} := (x_{i}^{\pi} \; : \; 0 \leq i < |\vec{x}| )$ where $\pi \in \{L,R\}$.

We also use $P^{\Box} := P|\Box$.

In \cite{MeredithR05} an interpretation of the new operator is
given. It turns out that there are several possible interpretations
all enjoying the requisite algebraic properties of the operator (see
\cite{milner91polyadicpi}). We will therefore make liberal use of
$(\nu\; \vec{x})P$.

% subsection the_syntax_and_semantics_of_the_notation_system (end)   

\input{qm2pi.qmops} 

\input{qm2pi.sterngerlach} 

\input{qm2pi.metric} 

% section concurrent_process_calculi (end)

%\input{qm2pi.proofsketch}

% section proof sketch (end)

%\input{qm2pi.slviaknots} 

% section spatial logic via knots (end)

\input{qm2pi.conclusion}

% section conclusion (end)

%\input{qm2pi.dtcodes} 

% section wiring algorithm (end)

\input{qm2pi.ack} 

% section acknowledgments (end)

\newpage


\bibliographystyle{plain}   
\bibliography{../../biblios/main.bib}

\input{qm2pi.rhodetails}

\end{document}

 

% section concurrent_process_calculi (end)

%\documentclass[12pt]{llncs}
%\documentclass{jktr}

\usepackage[pdftex]{hyperref}                   
\usepackage {listings}
\usepackage {mathpartir}
\usepackage{bcprules}
%\usepackage{listings}
                       
\usepackage{graphicx} 
%\usepackage[margins=2.5cm,nohead,nofoot]{geometry}
%\usepackage{geometry}
\usepackage{amsfonts}
\usepackage{amstext}
\usepackage{latexsym}
\usepackage{amssymb}
\usepackage{color}


%\include{myPreamble}
\include{qm2pi.local} 

%\ifpdf
%\usepackage[pdftex]{graphicx}
%\else
%\usepackage{graphicx}
%\fi

 % \ifpdf
%  \usepackage{pdfsync}
%  \if


%\title{Brief Article}
%\author{David F. Snyder}
%\author{L.G. Meredith}

%\address{Dept. of Math., Texas State University--San Marcos, San Marcos, TX 78666}
       
\pagestyle{empty}


\begin{document}

\lstset{language=[Objective]Caml,frame=shadowbox}

\input{qm2pi.front}

% section front matter (end)

\input{qm2pi.intro} 
 
% section introduction (end)

% \input{qm2pi.knotations} 

% section notation (end)

\input{qm2pi.process.calculi} 

% section concurrent_process_calculi_and_spatial_logics_ (end)
    
%\input{qm2pi.knots2pi} 

%\input{qm2pi.trefoil} 

%\input{qm2pi.mainthm} 

% subsection basic_interpretation (end)

%\input{qm2pi.rho.presentation} 
\subsection{The syntax and semantics of the notation system}\label{sub:the_syntax_and_semantics_of_the_notation_system} % (fold)

We now summarize a technical presentation of the calculus that
embodies our theory of dynamics. The typical presentation of such a
calculus follows the style of giving generators and relations on
them. The grammar, below, describing term constructors, freely
generates the set of processes, $\Proc$. This set is then quotiented
by a relation known as structural congruence and it is over this set
that the notion of dynamics is expressed. This presentation is
essentially that of \cite{MeredithR05} with the addition of
polyadicity and summation. For readability we have relegated some of
the technical subtleties to an appendix.

\subsubsection{Process grammar}\label{subsub:process_grammar}

\begin{mathpar}
  \inferrule* [lab=synchronization] {} {{M} \bc \pzero \;|\; x?F \;|\; x!C }
  \and
  \inferrule* [lab=abstraction] {} {{F} \bc (x)P}
  \and
  \inferrule* [lab=concretion] {} {{C} \bc \langle Q \rangle}
  \and
  \inferrule* [lab=process] {} {{P,Q} \bc M \;| \;P|Q \;|\; @{x}}
  \and
  \inferrule* [lab=name] {} {{x} \bc \quotep{P}}
\end{mathpar} 

Note that $\vec{x}$ (resp. $\vec{P}$) denotes a vector of names
(resp. processes) of length $|\vec{x}|$ (resp. $|\vec{P}|$). We adopt
the following useful abbreviations.

\begin{mathpar}
   x?(\vec{y}).P := x.(\vec{y})P \and  x\clift{\vec{P}} := x.\clift{\vec{P}}
   \and x!(y) := \lift{x}{\dropn{y}}
   \and \Pi_{i=0}^{n-1}P_i := P_0 | \ldots | P_{n-1}
\end{mathpar}

\subsubsection{Structural congruence}

\paragraph{Free and bound names and alpha-equivalence.} At the
core of structural equivalence is alpha-equivalence which identifies
process that are the same up to a change of variable. Formally, we
recognize the distinction between free and bound names. The free names
of a process, $\freenames{P}$, may be calculated recursively as
follows:

\begin{mathpar}
\freenames{\pzero} := \emptyset
  \and \\
  \freenames{x?(y).P} := \{ x \} \cup (\freenames{P} \setminus \{ y \})
  \and 
  \freenames{x!\langle P \rangle} := \{ x \} \cup \{ P \} 
  \and \\
  \freenames{P|Q} := \freenames{P} \cup \freenames{Q}
  \and \\
  \freenames{@{x}} := \{ x \}
\end{mathpar}

$\pi$
$\quotep{\pi}$

$\freenames{-} : \pi \to \mathcal{P}(\quotep{\pi})$

\begin{eqnarray*}
  \freenames{\pzero} & := & \emptyset \\
  \freenames{x?(y).P} & := & \{ x \} \cup (\freenames{P} \setminus \{ y \}) \\
  \freenames{x!\langle P \rangle} & := & \{ x \} \cup \{ P \} \\
  \freenames{P|Q} & := & \freenames{P} \cup \freenames{Q} \\
  \freenames{\dropn{x}} & := & \{ x \}
\end{eqnarray*}

The bound names of a process, $\boundnames{P}$, are those names occurring in $P$
that are not free. For example, in $x?(y).0$, the name $x$ is free, while $y$ is bound.

\begin{mathpar}
  \inferrule* [lab=monoidal-laws] {} { P|Q \equiv Q|P \and P|0 \equiv P \and P|(Q|R) \equiv (P|Q)|R }
\end{mathpar}

\begin{mathpar}
  \inferrule* [lab=alpha-equivalence] {} { (x)P \equiv (y)P\{y/x\} \and y \not\in \freenames{P} }
\end{mathpar}

\begin{definition}
Then two processes, $P,Q$, are alpha-equivalent if $P = Q\{\vec{y}/\vec{x}\}$ for
some $\vec{x} \in \boundnames{Q},\vec{y} \in \boundnames{P}$, where $Q\{\vec{y}/\vec{x}\}$
denotes the capture-avoiding substitution of $\vec{y}$ for $\vec{x}$ in $Q$.
\end{definition}

\begin{definition}
  The {\em structural congruence} \cite{SangiorgiWalker} , $\equiv$,
  between processes is the least congruence containing
  alpha-equivalence, satisfying the abelian monoid laws
  (associativity, commutativity and $\pzero$ as identity) for parallel
  composition $|$ and for summation $+$.
\end{definition}

\subsection{Name equivalence}

We take name equivalence, written $\nameeq$, to be the smallest
equivalence relation generated by the following rules.

\begin{mathpar}
\inferrule*[lab=Quote-drop]
{ }
{ \quotep{@{x}} \nameeq x }

\inferrule*[lab=Struct-equiv]
{ P \scong Q }
{ \quotep{P} \nameeq \quotep{Q} }
\end{mathpar}

The astute reader will have noticed that the mutual recursion of names
and processes imposes a mutual recursion on alpha-equivalence and
structural equivalence via name-equivalence. Fortunately, all of this
works out pleasantly and we may calculate in the natural way, free of
concern. The reader interested in the details is referred to the
appendix \ref{appendix:rho_details}.

\subsection{Substitution}

We use $\Proc$ for the set of processes, $\QProc$ for the set of
names, and $\id{\{}\vec{y} / \vec{x} \id{\}}$ to denote partial maps,
$s : \QProc \rightarrow \QProc$. A map, $s$ lifts, uniquely, to a map
on process terms, $\widehat{s} : \Proc \rightarrow \Proc$ by the
following equations.

\begin{mathpar}
  (0) \psubstp{Q}{P} := 0 \\
  (R \juxtap S) \psubstp{Q}{P}
  :=    
  (R)\psubstp{Q}{P} \juxtap (S) \psubstp{Q}{P} \\
  (x?(y).R) \psubstp{Q}{P}    
  :=    
  (x)\substp{Q}{P} (z)\concat( (R \psubstn{z}{y}) \psubstp{Q}{P} ) \\
  (\lift{x}{R}) \psubstp{Q}{P}  
  :=
  \lift{(x)\substp{Q}{P}}{ R \psubstp{Q}{P} } \\
%   (\dropn{x})  \psubstp{Q}{P}       
%   := 
%   \left\{ 
%     \begin{array}{ccc} 
%       \dropn{\quotep{Q}} & & x \nameeq \quotep{P} \\
%       \dropn{x} & & otherwise \\
%     \end{array}
%   \right. 
  (\dropn{x})  \psubstp{Q}{P}       
  := 
  \left\{ 
    \begin{array}{ccc} 
      Q & & x \nameeq \quotep{P} \\
      \dropn{x} & & otherwise \\
    \end{array}
  \right.
\end{mathpar}
 

where

\begin{eqnarray}
  (x)\id{\{} \lpquote Q \rpquote / \lpquote P \rpquote \id{\}}            = 
  \left\{ 
    \begin{array}{ccc}
      \lpquote Q \rpquote & & x \nameeq \lpquote P \rpquote \\
      x & & otherwise \\
    \end{array}
  \right. \nonumber
\end{eqnarray}

and $z$ is chosen distinct from $\quotep{P}$, $\quotep{Q}$, the free
names in $Q$, and all the names in $R$. Our $\alpha$-equivalence will
be built in the standard way from this substitution.

\begin{remark}\label{rem:no_self_referential_names}
  One consequence of these definitions is that $\forall P. \quotep{P}
  \not\in \freenames{P}$.
\end{remark}

\subsection{ Dynamic quote: an example }

Anticipating something of what's to come, consider applying the
substitution, $\widehat{\id{\{}u / z \id{\}}}$, to the following pair
of processes, $\lift{w}{y!(z)}$ and $w[ \lpquote y!(z) \rpquote ]$.

\begin{eqnarray}
	\lift{w}{y!(z)}\widehat{\id{\{}u / z \id{\}}}
		& = &
		\lift{w}{y!(u)} \nonumber\\
	w[ \lpquote y!(z) \rpquote ] \widehat{ \id{\{}u / z \id{\}} }
		& = &
		w[ \lpquote y!(z) \rpquote ] \nonumber
\end{eqnarray}

Because the body of the process between quotes is impervious to
substitution, we get radically different answers. In fact, by
examining the first process in an input context,
e.g. $x?(z).\lift{w}{y!(z)}$, we see that the process under the lift
operator may be shaped by prefixed inputs binding a name inside it. In
this sense, the lift operator will be seen as a way to dynamically
construct processes before reifying them as names.

Finally equipped with these standard features we can present the
dynamics of the calculus.

\subsubsection{Operational semantics} 

Finally, we introduce the computational dynamics. What marks these
algebras as distinct from other more traditionally studied algebraic
structures, e.g. vector spaces or polynomial rings, is the manner in
which dynamics is captured. In traditional structures, dynamics is typically
expressed through morphisms between such structures, as in linear maps
between vector spaces or morphisms between rings. In algebras
associated with the semantics of computation, the dynamics is
expressed as part of the algebraic structure itself, through a
reduction reduction relation typically denoted by $\red$. Below, we
give a recursive presentation of this relation for the calculus used
in the encoding.

$\red \subseteq \pi \times \pi$
$\red : \pi \to \mathcal{P}(\pi)$

\begin{mathpar}
  \inferrule* [lab=Comm] { \textsf{match}( x_{src}, x_{trgt} ) } { x_{trgt}?(y)P \; | \; x_{src}!\langle {Q} \rangle \red P\{\quotep{Q}/y}\} }
  \and \\
  \inferrule* [lab=Par] {{P} \red {P}'} {{{P} | {Q}} \red {{P}' | {Q}}}
  \and
  \inferrule* [lab=Equiv]{{{P} \scong {P}'} \andalso {{P}' \red {Q}'} \andalso {{Q}' \scong {Q}}}{{P} \red {Q}}
\end{mathpar}

\begin{eqnarray*}
  match_{\equiv} (\quotep{P},\quotep{Q}) & := & P \equiv Q \\
  match_{\dagger}(\quotep{P},\quotep{Q}) & := & \forall R. P|Q \red^{*} R => R \red^{*} 0 \\
  match_{K}(\quotep{P},\quotep{Q}) & := & K \mbox{ for some context } K
\end{eqnarray*}

$u?(x)P | u!\langle Q \rangle \red P\{\quotep{Q}/x\}$

%We write $\wred$ for $\red^*$, and $P\red$ if $\exists Q $ such that $ P \red Q$.
We write $P\red$ if $\exists Q $ such that $ P \red Q$ and $P\not\red$, otherwise.

\section{Replication}

As mentioned before, it is known that replication (and hence
recursion) can be implemented in a higher-order process algebra
\cite{SangiorgiWalker}. As our first example of calculation with the
machinery thus far presented we give the construction explicitly in
the {\rhoc}.

\begin{eqnarray}
	D_{x} & := & \prefix{x}{y}{(\binpar{\outputp{x}{y}}{@{y}})} \nonumber\\
	\bangp_{x}{P} & := & \binpar{{x}!\langle{\binpar{D_{x}}{P}}\rangle}{D_{x}} \nonumber
\end{eqnarray}

\begin{eqnarray}
	\bangp_{x}{P} & & \nonumber\\
	=
	& {x}!\langle{(\prefix{x}{y}{(\outputp{x}{y} | @{y})) | P}}\rangle 
	      | \prefix{x}{y}{(\outputp{x}{y} | @{y})} & \nonumber\\
	\red
	& (\outputp{x}{y} | @{y})\substn{\quotep{(\prefix{x}{y}{(@{y} | \outputp{x}{y})) | P}}}{y} & \nonumber\\
	=
	& \outputp{x}{\quotep{(\prefix{x}{y}{(\outputp{x}{y} | @{y})) | P}}}
	  | {(\prefix{x}{y}{(\outputp{x}{y} | @{y})) | P}} & \nonumber\\
	\red
	& \ldots & \nonumber\\
	\red^*
	& P | P | \ldots & \nonumber
\end{eqnarray}

Of course, this encoding, as an implementation, runs away, unfolding
$\bangp{P}$ eagerly. A lazier and more implementable replication
operator, restricted to input-guarded processes, may be obtained as follows.

\begin{eqnarray}
\bangp{\prefix{u}{v}{P}} 
	:= 
	\binpar{\lift{x}{\prefix{u}{v}{(\binpar{D(x)}{P})}}}{D(x)} \nonumber
\end{eqnarray}

\begin{remark}
  Note that the lazier definition still does not deal with summation
  or mixed summation (i.e. sums over input and output). The reader is
  invited to construct definitions of replication that deal with these
  features. 

  Further, the definitions are parameterized in a name, $x$. Can you,
  gentle reader, make a definition that eliminates this parameter and
  guarantees no accidental interaction between the replication
  machinery and the process being replicated -- i.e. no accidental
  sharing of names used by the process to get its work done and the
  name(s) used by the replication to effect copying. This latter
  revision of the definition of replication is crucial to obtaining
  the expected identity $!!P \sim !P$.
\end{remark}

\begin{remark}\label{rem:paradoxical_combinator}
  The reader familiar with the lambda calculus will have noticed the
  similarity between $D$ and the paradoxical combinator.

  [Ed. note: the existence of this seems to suggest we have to be more
  restrictive on the set of processes and names we admit if we are to
  support no-cloning.]
\end{remark}

\subsubsection{Bisimulation}

The computational dynamics gives rise to another kind of equivalence,
the equivalence of computational behavior. As previously mentioned
this is typically captured \emph{via} some form of bisimulation.

% The notion we use in this paper is weak barbed bisimulation
% \cite{milner91polyadicpi}.

The notion we use in this paper is derived from weak barbed
bisimulation \cite{milner91polyadicpi}. 

\begin{definition}
An \emph{observation relation}, $\downarrow_{\mathcal N}$, over a set
of names, $\mathcal N$, is the smallest relation satisfying the rules
below.

\infrule[Out-barb]{y \in {\mathcal N}, \; x \nameeq y}
		  {\outputp{x}{v} \downarrow_{\mathcal N} x}
\infrule[Par-barb]{\mbox{$P\downarrow_{\mathcal N} x$ or $Q\downarrow_{\mathcal N} x$}}
		  {\binpar{P}{Q} \downarrow_{\mathcal N} x}

We write $P \Downarrow_{\mathcal N} x$ if there is $Q$ such that 
$P \wred Q$ and $Q \downarrow_{\mathcal N} x$.
\end{definition}

\begin{definition}
%\label{def.bbisim}
An  ${\mathcal N}$-\emph{barbed bisimulation} over a set of names, ${\mathcal N}$, is a symmetric binary relation 
${\mathcal S}_{\mathcal N}$ between agents such that $P\rel{S}_{\mathcal N}Q$ implies:
\begin{enumerate}
\item If $P \red P'$ then $Q \wred Q'$ and $P'\rel{S}_{\mathcal N} Q'$.
\item If $P\downarrow_{\mathcal N} x$, then $Q\Downarrow_{\mathcal N} x$.
\end{enumerate}
$P$ is ${\mathcal N}$-barbed bisimilar to $Q$, written
$P \wbbisim_{\mathcal N} Q$, if $P \rel{S}_{\mathcal N} Q$ for some ${\mathcal N}$-barbed bisimulation ${\mathcal S}_{\mathcal N}$.
\end{definition}

$\mathcal{R} \subseteq \pi \times \pi$

$P \mathcal{R} Q => \forall P'. P \red P' \Rightarrow \exists Q'. Q \red Q', P' \mathcal{R} Q'$

$P \vdash x \Rightarrow Q \vdash x$

\begin{mathpar}
  \inferrule*[lab=Out-barb]{x \nameeq y}{{y}!\langle{Q}\rangle \vdash x}
  \and
  \inferrule*[lab=Par-barb]{\mbox{$P\vdash x$ or $Q\vdash x$}}{\binpar{P}{Q} \vdash x}
\end{mathpar}

\subsubsection{Contexts}

One of the principle advantages of computational calculi like the
$\pi$-calculus is a well-defined notion of context,
contextual-equivalence and a correlation between
contextual-equivalence and notions of bisimulation. The notion of
context allows the decomposition of a process into (sub-)process and
its syntactic environment, its context. Thus, a context may be
thought of as a process with a ``hole'' (written $\Box$) in it. The
application of a context $M$ to a process $P$, written $M[P]$, is
tantamount to filling the hole in $M$ with $P$. In this paper we do
not need the full weight of this theory, but do make use of the notion
of context in the proof the main theorem. 

\begin{mathpar}
  \inferrule* [lab=summation] {} {{M_{M},M_{N}} \bc \Box \;|\; x.M_{A} \;|\; M_{M}+M_{N}}
  \and
  \inferrule* [lab=agent] {} {{M_{A}} \bc (\vec{x})M_{P} \;| \; \clift{P_0,\ldots,M_{P},\ldots,P_N}}
  \and \\
  \inferrule* [lab=process] {} {{M_{P}} \bc M_{N} \;| \;P|M_{P} }
\end{mathpar} 

\begin{mathpar}
  \inferrule* [lab=sychronization] {} {M_{N} \bc \Box \;|\; x?M_{F} \;|\; x!M_{C}}
  \and
  \inferrule* [lab=abstraction] {} {{M_{F}} \bc (x)M_{P} }
  \and
  \inferrule* [lab=concretion] {} {{M_{C}} \bc \langle M_{P} \rangle }
  \and \\
  \inferrule* [lab=process] {} {{M_{P}} \bc M_{N} \;| \;P|M_{P} }
\end{mathpar}

\begin{definition}[contextual application] Given a context $M$, and
  process $P$, we define the \emph{contextual application}, $M[P] :=
  M\{P/\Box\}$. That is, the contextual application of M to P is the
  substitution of $P$ for $\Box$ in $M$.
\end{definition}

$\meaningof{-} : L \to \mathcal{P}(\pi)$

\begin{mathpar}
  \inferrule* [lab=collection] {} {\meaningof{true} = \pi, \and \meaningof{~E} = \pi \setminus \meaningof{E}, \and \meaningof{E_{1} \& E_{2}} = \meaningof{E_{1}} \cap \meaningof{E_{2}}}
\end{mathpar}

\begin{mathpar}
  \inferrule* [lab=structure] {} {\meaningof{0} = \{ P \in \pi | P \equiv 0 \}, \and \\ \meaningof{E_1 | E_2} = \{ P \in \pi | P \equiv P_{1} | P_{2}, P_{1} \in \meaningof{E_{1}}, P_{2} \in \meaningof{E_2}\} }
\end{mathpar}

\begin{mathpar}
 \inferrule* [lab=behavior] {} {\meaningof{\langle a?b \rangle E} = \{ P \in \pi | P \equiv Q | u?(y)P', \\ \and \\\\ \and \\ \;\;\; u \in \meaningof{a}, \forall z.P'\{z/y\} \in \meaningof{E\{z/b\}}\}, \and \\ \meaningof{a!E} = \{ P \in \pi | P \equiv Q | x!\langle P' \rangle, x \in \meaningof{a} P' \in \meaningof{E}\} }
\end{mathpar}

\begin{mathpar}
 \inferrule* [lab=nominal] {} {\meaningof{\quotep{E}} = \{ \quotep{P} \in \quotep{\pi} | P \in \meaningof{E} \}, \and \meaningof{\quotep{P}} = \{ \quotep{Q} \in \quotep{\pi} | P \equiv Q \} \and \\ \meaningof{@\quotep{E}} = \{ P \in \pi | P \equiv @x, x \in \meaningof{E} \}}
\end{mathpar}

\begin{eqnarray*}
  \\
  \meaningof{-} : TS \to ST
\end{eqnarray*}

\begin{eqnarray*}
  \\
  L : TS \to ST
\end{eqnarray*}

\begin{eqnarray*}
  \\
  P \models E \iff P \in \meaningof{E}
\end{eqnarray*}

\begin{eqnarray*}
  P \approx_{L} Q \iff \forall E \in L. P \models E \iff Q \models E
\end{eqnarray*}

\begin{eqnarray*}
  P \approx_{K} Q
\end{eqnarray*}

\begin{eqnarray*}
  P \approx Q
\end{eqnarray*}

$\approx_{K} = \approx = \approx_{L}$

\subsubsection{Contextual duality}

Note that contexts extend the quotation operation to a family of
operations from processes to names. Given a context, $M$, we can
define a \emph{nominal context}, $\quotep{M}$ by $\quotep{M}[P] :=
\quotep{M[P]}$. To foreshadow what is to come we observe that these
operations enjoy a duality with processes very much like the duality
between vectors and maps from vectors to scalars.

Further, because the calculus is essentially higher-order, we have a
correspondence between contexts and processes. More specifically,
given a name $x$ and a context $M$ we can construct $M^{*}_{x}$ such
that 

\begin{mathpar}
  M^{*}_{x} | \lift{x}{P} \red M[P]
\end{mathpar}

namely,

\begin{mathpar}
  M^{*}_{x} := x?(u).M[\dropn{u}]
\end{mathpar}

The dependence of $M^{*}_{x}$ on a name makes it an abstraction, 

\begin{mathpar}
  M^{*} := (x)x?(u).M[\dropn{u}]
\end{mathpar}

\subsection{Additional notation}

It will sometimes be convenient to denote the process a name
quotes. We already have the notation $x = \quotep{P}$, but it will be
convenient to introduce an alternate notation, $\procn{x}$, when we
want to emphasize the connection to the use of the name. Note that, by
virtue of name equivalence, $\quotep{\procn{x}} \nameeq x$; so, the
notation is consistent with previous definitions.

Further, because names have structure it is possible to effect
substitutions on the basis of that structure. This means we need to
upgrade our notation for substitutions, which we accomplish by
adapting comprehension notation. Thus,

\begin{mathpar}
  P\{ y / x : x \in S \}
\end{mathpar}

is interpreted to mean the process derived from P by replacing (in a
capture-avoiding manner) each occurrence of $x$ in $S$ by $y$. For example,

\begin{mathpar}
  P\{ \quotep{\procn{x}|\procn{x}} / x : x \in \freenames{P} \}
\end{mathpar}

will replace each (occurrence) of a free name $x$ in $P$ by
$\quotep{\procn{x}|\procn{x}}$.

Also, we will avail ourselves of the notation $x^{L}$ and $x^{R}$ to
denote injections of a name into disjoint copies of the name
space. There are numerous ways to accomplish this. One example can be
found in \cite{MeredithR05}. This notation overloads to vectors of
names: $\vec{x}^{\pi} := (x_{i}^{\pi} \; : \; 0 \leq i < |\vec{x}| )$ where $\pi \in \{L,R\}$.

We also use $P^{\Box} := P|\Box$.

In \cite{MeredithR05} an interpretation of the new operator is
given. It turns out that there are several possible interpretations
all enjoying the requisite algebraic properties of the operator (see
\cite{milner91polyadicpi}). We will therefore make liberal use of
$(\nu\; \vec{x})P$.

% subsection the_syntax_and_semantics_of_the_notation_system (end)   

\input{qm2pi.qmops} 

\input{qm2pi.sterngerlach} 

\input{qm2pi.metric} 

% section concurrent_process_calculi (end)

%\input{qm2pi.proofsketch}

% section proof sketch (end)

%\input{qm2pi.slviaknots} 

% section spatial logic via knots (end)

\input{qm2pi.conclusion}

% section conclusion (end)

%\input{qm2pi.dtcodes} 

% section wiring algorithm (end)

\input{qm2pi.ack} 

% section acknowledgments (end)

\newpage


\bibliographystyle{plain}   
\bibliography{../../biblios/main.bib}

\input{qm2pi.rhodetails}

\end{document}



% section proof sketch (end)

%\section{Unlikely characters: spatial logic for
  knots}\label{sub:characteristic_formulae} % (fold)

Associated to the mobile process calculi are a family of logics known
as the Hennessy-Milner logics. These logics typically enjoy a
semantics interpreting formulae as sets of processes that when
factored through the encoding outlined above allows an identification
of classes of knots with logical formulae. In the context of this
encoding the sub-family known as the spatial logics \cite{CairesC03}
\cite{CairesC04} \cite{Caires04} are of particular interest providing
several important features for expressing and reasoning about
properties (i.e. classes) of knots. We hint here at how this may be done.

%\begin{description}
%\item [structural connectives] 
\subsubsection{Structural connectives} The spatial logics enjoy
structural connectives corresponding, at the logical level, to the
parallel composition ($P | Q$) and new name ($(\nu \; x)P$)
connectives for processes. As illustrated in the examples below, these
connectives are extremely expressive given the shape of our encoding.
%\item [decideable satisfaction]

\subsubsection{Decideable satisfaction}
In \cite{Caires04} the satisfaction relation is shown to be decideable
for a rich class of processes. It further turns out that the image of
the our encoding is a proper subset of that class. This result
provides the basis for an algorithm by which to search for knots
enjoying a given property.
%\item [characteristic formulae]

\subsubsection{Characteristic formulae}
In the same paper \cite{Caires04} , Caires presents a means of calculating
characteristic formulae, selecting equivalence classes of processes
up to a pre--specified depth limit on the support set of names. Composed with our
encoding, this characteristic formula can be used to select
characteristic formulae for knots.
%\end{description}

\subsubsection{Spatial logic formulae}

The grammar below (segmented for comprehension) summarizes the syntax
of spatial logic formulae. We employ illustrative examples in the
sequel to provide an intuitive understanding of their meaning
referring the reader to \cite{Caires04} for a more detailed explication
of the semantics.

\begin{mathpar}
  \inferrule* [lab=boolean] {} {{A,B} \bc T \;|\; \neg A \;|\; A \wedge B \;|\; \eta = \eta'}
  \and
  \inferrule* [lab=spatial] {} {|\; \pzero \;|\; A | B \;|\; x \text{\textregistered} A \;|\; \forall x . A \;|\;  H x . A}
  \and
  \inferrule* [lab=behavioral] {} {|\; \alpha . A}
  \and 
  \inferrule* [lab=recursion] {} {|\; X(\vec{u}) \;|\; \mu X(\vec{u}) . A}
  \and
  \inferrule* [lab=action] {} {\alpha \bc \langle x?(\vec{y}) \rangle \;|\; \langle x!(\vec{y}) \rangle \;|\; \langle \tau \rangle}
  \and 
  \inferrule* [lab=name] {} {\eta \bc x \;|\; \tau}
\end{mathpar} 

% subsection characteristic_formulae (end)   	 

\subsection{Example formulae}\label{sub:example_formulae_} % (fold)

\subsubsection{Crossing as formula.}
% 
% \begin{align*}
%   \frac{d}{dx} \sin x &= \cos x 
%   & \frac{d}{dx} e^x &= e^x \\
%   \frac{d}{dx} \cos x &= - \sin x 
%   & \frac{d}{dx} \log x &= \frac{1}{x} \\
% \end{align*} 

\begin{align*}
 \mu C(x_{0},x_{1},y_{0},y_{1},u).&(\langle x_{0}?(z) \rangle(\langle u! \rangle\langle y_{1}!z \rangle C(x_{0},x_{1},y_{0},y_{1},u)) & \\
  & \wedge \langle y_{1}?(z) \rangle (\langle u! \rangle \langle x_{0}!z \rangle C(x_{0},x_{1},y_{0},y_{1},u)) & \\
  & \wedge \langle x_{1}?(z) \rangle (\langle u? \rangle \langle y_{0}!z \rangle C(x_{0},x_{1},y_{0},y_{1},u)) & \\
  & \wedge \langle y_{0}?(z) \rangle (\langle u? \rangle \langle x_{1}!z \rangle C(x_{0},x_{1},y_{0},y_{1},u))) &
\end{align*}

The lexicographical similarity between the shape of this formulae and
the shape of definition of the process representing a crossing reveals
the intuitive meaning of this formulae. It describes the capabilities
of a process that has the right to represent a crossing. For example
it picks out processes that may perform an input on the port $x_0$ in
its initial menu of capabilities. What differentiates the formula
from the process, however, is that the crossing process is the
smallest candidate to satisfy the formula. Infinitely many other
processes -- with internal behavior hidden behind this interface, so
to speak -- also satisfy this formula. Even this simple formula,
then, can be seen to open a new view onto knots, providing a
computational interpretation of \emph{virtual} knots.

Note that this formula is derived by hand. A similar formula can be
derived by employing Caires' calculation of characteristic formula
\cite{Caires04} to the process representing a crossing. In light of
this discussion, we let
$\meaningof{C}_{\phi}(x0,x1,y0,y1,u)$ denote a formula specifying the
dynamics we wish to capture of a crossing. To guarantee we preserve
the shape of the interface and minimal semantics we demand that
$\meaningof{C}_{\phi}(x0,x1,y0,y1,u) \Rightarrow
\textbf{C}(x0,x1,y0,y1,u)$ where $\textbf{C}(x0,x1,y0,y1,u)$ denotes
the formula above.
                            
\subsubsection{Crossing number constraints.}
The moral content of the context lemma (Lemma \ref{context}) is that the notion of
``locality'' in the Reidemeister moves is effectively captured by the
parallel composition operator of the process calculus. This intuition
extends through the logic. Given a formula,
$\meaningof{C}_{\phi}(x0,x1,y0,y1,u)$, we can use the structural
connectives to specify constraints on crossing numbers, such as at
least $n$ crossings, or exactly $n$ crossings.
\begin{mathpar}
  \inferrule* [lab=at-least-n] {} { K^{\geq n}_{\phi}(\vec{xs},\vec{ys}) := \Pi_{i=0}^{n-1} Hu . \meaningof{C}_{\phi}(xs_i,ys_i,u) | T }
  \and 
  \inferrule* [lab=exactly-n] {} { K^{= n}_{\phi}(\vec{xs},\vec{ys}) := \Pi_{i=0}^{n-1} Hu . \meaningof{C}_{\phi}(xs_i,ys_i,u) | \neg (\forall x_0,y_0,x_1,y_1,u . \meaningof{C}_{\phi}(x_0,y_0,x_1,y_1,u) | T) }
\end{mathpar}

To round out this section, recall that the encoding of an $n$-crossing
knot decomposes into a parallel composition of $n$ \emph{copies} of a
crossing process together with a wiring harness. To specify different
knot classes with the same crossing number amounts to specifying
logical constraints on the wiring harness. In the interest of space,
we defer examples to a forthcoming paper. Suffice it to say that both
the conditions ``alternating knot'' and ``contains the tangle
corresponding to 5/3'' are expressible. For example, it is possible to
calculate the characteristic formula of a process corresponding to the
tangle 5/3 and conjoin it into the classifying formula via the
composition connective of the logic.

Finally, we wish to observe that it is entirely within reason to
contemplate a more domain-specific version of spatial logic tailored
to the shape of processes in the image of the encoding. Such a
domain-specific logic would have a better claim to the title formal
language of knot properties.

% subsection example_formulae_ (end)

% section knots_as_processes (end) 

% section spatial logic via knots (end)

\section{Conclusions and future work}

\paragraph{Testing physical space}
You, gentle reader, may wonder why of all the theorems to be proved
given this set up we pick the one above. In some sense it's hardly
central to quantum mechanics. We see it as central in the sense that
it firmly establishes a notion of physical space arising from a notion
of the equivalence of behavior. Relating bisimulation to a metric is a
big step forward, but one is faced with interpreting the relationship
of that metric space to something more physical. Quantum mechanical
notions of ``physical'' space are still far from intuitive, but by
relating this idea of distance as testing to calculations that predict
physical circumstances we are making a not insignificant step forward
toward an understanding of the physical space we inhabit as
essentially dynamic.

\paragraph{Effectivity and simulation}
One of the observations we have yet to make is that the entire program
spelled out here is effective. We have built various interpreters for
the reflective calculus at work in this interpretation. In principle,
then, we can simulate quantum mechanics on a computer. The place where
the simulation may lose fidelity is the infinitely branching summation
for the annihilator.

In this connection i also want to point out that the evaluation style
calculation of the inner product puts the non-determinism of the
summation right at the heart of measurement. This suggests that
Milner's original reduction-based formulation of the dynamics of his
calculi in terms of sums was not just notationally suggestive of a
notion of measure-and-continue but captured some significant part of
the physics.

\paragraph{Quantum continuations}
In light of this last observation i want to point out that the
predominant account of quantum mechanics is missing a key aspect of a
truly compositional story of the physical situation. In a real lab,
when a measurement is made the observation can be made to feed into
another device that then makes another measurement conditioned on the
results of the first. This means that after the superposition was
collapsed the entire experimental set up remained in
superposition. While QM offers a means of writing this down it doesn't
quite line up well with the well-trodden formulation of computation
and continuation that we see so succinctly expressed in Milner's
calculi. This suggests that there might be advantages to this account
of dynamics waiting to be explored.

\paragraph{Quantum logic}
In this connection, we also note that by virtue of having the
Hennessy-Milner construction, we can pull the construction through the
interpretation of QM. This gives us a natural candidate for a quantum
logic that enjoys an extremely tight connection with it's domain of
interpretation, making the construction much less ad hoc (rather it is
the image of functor!).

\paragraph{Quantum probabiity}
i have questions about the basis of the interpretation of inner
product as probability amplitude. In particular, using which
axiomatization of probability theory does the notion of probability
amplitude earn the right to be so dubbed? In other words, where is the
proof that the operation for calculating a probability amplitude (and
then squaring) satisfies the axioms of what it means to calculate a
probability? Even if such a proof exists (i have yet to find it in the
literature), i wonder if it might not be possible to turn things on
their heads. Can we view the calculation of the probability amplitude
as an axiomatization of probability? If so, then the definition we
give for calculating probability amplitude may provide the basis for
an \emph{effective} theory of probability.

\paragraph{Quantum vs ``biological'' information}
Finally, i want to conclude with a more philosophical observation. At
a recent workshop in which QM was a predominant topic i noticed
something about quantum information. The speaker was giving a riveting
discussion of axiomatic QM and showing how properties of ``no
cloning'' and ``no deleting'' emerged as consequences of the
axiomatization. Theorems of this form are necessary to give us a sense
of confidence that our axioms characterize the physical theory. What
struck me, though, was that if quantum information is neither erasable
nor replicable it is markedly different from \emph{life}. Two of the
things we know about life is that

\begin{itemize}
  \item it ends;
  \item to gain some measure of persistence, to transcend it's
    finitude it is imminently copyable.
\end{itemize}

Both of these qualities are summarized succinctly in the aphorism: all
flesh is grass. For me these two kinds of ``information'' -- call them
quantum and biological -- are end points on a spectrum of strategies
for persistence. At one end, we have those curious entities that enjoy
uniqueness and permanence; at the other, we have those who in the face
of a certain end and an uncertain present make a go of passing
something on. To me one of the more remarkable aspects of the latter
strategy is that in the presence of noise (and certain features of
copying) we get a kind of dynamism, a chance for improvement against a
given persistent condition.

% subsection other_calculi_other_bisimulations_and_geometry_as_behavior (end)




% section conclusion (end)

%\documentclass[12pt]{llncs}
%\documentclass{jktr}

\usepackage[pdftex]{hyperref}                   
\usepackage {listings}
\usepackage {mathpartir}
\usepackage{bcprules}
%\usepackage{listings}
                       
\usepackage{graphicx} 
%\usepackage[margins=2.5cm,nohead,nofoot]{geometry}
%\usepackage{geometry}
\usepackage{amsfonts}
\usepackage{amstext}
\usepackage{latexsym}
\usepackage{amssymb}
\usepackage{color}


%\include{myPreamble}
\include{qm2pi.local} 

%\ifpdf
%\usepackage[pdftex]{graphicx}
%\else
%\usepackage{graphicx}
%\fi

 % \ifpdf
%  \usepackage{pdfsync}
%  \if


%\title{Brief Article}
%\author{David F. Snyder}
%\author{L.G. Meredith}

%\address{Dept. of Math., Texas State University--San Marcos, San Marcos, TX 78666}
       
\pagestyle{empty}


\begin{document}

\lstset{language=[Objective]Caml,frame=shadowbox}

\input{qm2pi.front}

% section front matter (end)

\input{qm2pi.intro} 
 
% section introduction (end)

% \input{qm2pi.knotations} 

% section notation (end)

\input{qm2pi.process.calculi} 

% section concurrent_process_calculi_and_spatial_logics_ (end)
    
%\input{qm2pi.knots2pi} 

%\input{qm2pi.trefoil} 

%\input{qm2pi.mainthm} 

% subsection basic_interpretation (end)

%\input{qm2pi.rho.presentation} 
\subsection{The syntax and semantics of the notation system}\label{sub:the_syntax_and_semantics_of_the_notation_system} % (fold)

We now summarize a technical presentation of the calculus that
embodies our theory of dynamics. The typical presentation of such a
calculus follows the style of giving generators and relations on
them. The grammar, below, describing term constructors, freely
generates the set of processes, $\Proc$. This set is then quotiented
by a relation known as structural congruence and it is over this set
that the notion of dynamics is expressed. This presentation is
essentially that of \cite{MeredithR05} with the addition of
polyadicity and summation. For readability we have relegated some of
the technical subtleties to an appendix.

\subsubsection{Process grammar}\label{subsub:process_grammar}

\begin{mathpar}
  \inferrule* [lab=synchronization] {} {{M} \bc \pzero \;|\; x?F \;|\; x!C }
  \and
  \inferrule* [lab=abstraction] {} {{F} \bc (x)P}
  \and
  \inferrule* [lab=concretion] {} {{C} \bc \langle Q \rangle}
  \and
  \inferrule* [lab=process] {} {{P,Q} \bc M \;| \;P|Q \;|\; @{x}}
  \and
  \inferrule* [lab=name] {} {{x} \bc \quotep{P}}
\end{mathpar} 

Note that $\vec{x}$ (resp. $\vec{P}$) denotes a vector of names
(resp. processes) of length $|\vec{x}|$ (resp. $|\vec{P}|$). We adopt
the following useful abbreviations.

\begin{mathpar}
   x?(\vec{y}).P := x.(\vec{y})P \and  x\clift{\vec{P}} := x.\clift{\vec{P}}
   \and x!(y) := \lift{x}{\dropn{y}}
   \and \Pi_{i=0}^{n-1}P_i := P_0 | \ldots | P_{n-1}
\end{mathpar}

\subsubsection{Structural congruence}

\paragraph{Free and bound names and alpha-equivalence.} At the
core of structural equivalence is alpha-equivalence which identifies
process that are the same up to a change of variable. Formally, we
recognize the distinction between free and bound names. The free names
of a process, $\freenames{P}$, may be calculated recursively as
follows:

\begin{mathpar}
\freenames{\pzero} := \emptyset
  \and \\
  \freenames{x?(y).P} := \{ x \} \cup (\freenames{P} \setminus \{ y \})
  \and 
  \freenames{x!\langle P \rangle} := \{ x \} \cup \{ P \} 
  \and \\
  \freenames{P|Q} := \freenames{P} \cup \freenames{Q}
  \and \\
  \freenames{@{x}} := \{ x \}
\end{mathpar}

$\pi$
$\quotep{\pi}$

$\freenames{-} : \pi \to \mathcal{P}(\quotep{\pi})$

\begin{eqnarray*}
  \freenames{\pzero} & := & \emptyset \\
  \freenames{x?(y).P} & := & \{ x \} \cup (\freenames{P} \setminus \{ y \}) \\
  \freenames{x!\langle P \rangle} & := & \{ x \} \cup \{ P \} \\
  \freenames{P|Q} & := & \freenames{P} \cup \freenames{Q} \\
  \freenames{\dropn{x}} & := & \{ x \}
\end{eqnarray*}

The bound names of a process, $\boundnames{P}$, are those names occurring in $P$
that are not free. For example, in $x?(y).0$, the name $x$ is free, while $y$ is bound.

\begin{mathpar}
  \inferrule* [lab=monoidal-laws] {} { P|Q \equiv Q|P \and P|0 \equiv P \and P|(Q|R) \equiv (P|Q)|R }
\end{mathpar}

\begin{mathpar}
  \inferrule* [lab=alpha-equivalence] {} { (x)P \equiv (y)P\{y/x\} \and y \not\in \freenames{P} }
\end{mathpar}

\begin{definition}
Then two processes, $P,Q$, are alpha-equivalent if $P = Q\{\vec{y}/\vec{x}\}$ for
some $\vec{x} \in \boundnames{Q},\vec{y} \in \boundnames{P}$, where $Q\{\vec{y}/\vec{x}\}$
denotes the capture-avoiding substitution of $\vec{y}$ for $\vec{x}$ in $Q$.
\end{definition}

\begin{definition}
  The {\em structural congruence} \cite{SangiorgiWalker} , $\equiv$,
  between processes is the least congruence containing
  alpha-equivalence, satisfying the abelian monoid laws
  (associativity, commutativity and $\pzero$ as identity) for parallel
  composition $|$ and for summation $+$.
\end{definition}

\subsection{Name equivalence}

We take name equivalence, written $\nameeq$, to be the smallest
equivalence relation generated by the following rules.

\begin{mathpar}
\inferrule*[lab=Quote-drop]
{ }
{ \quotep{@{x}} \nameeq x }

\inferrule*[lab=Struct-equiv]
{ P \scong Q }
{ \quotep{P} \nameeq \quotep{Q} }
\end{mathpar}

The astute reader will have noticed that the mutual recursion of names
and processes imposes a mutual recursion on alpha-equivalence and
structural equivalence via name-equivalence. Fortunately, all of this
works out pleasantly and we may calculate in the natural way, free of
concern. The reader interested in the details is referred to the
appendix \ref{appendix:rho_details}.

\subsection{Substitution}

We use $\Proc$ for the set of processes, $\QProc$ for the set of
names, and $\id{\{}\vec{y} / \vec{x} \id{\}}$ to denote partial maps,
$s : \QProc \rightarrow \QProc$. A map, $s$ lifts, uniquely, to a map
on process terms, $\widehat{s} : \Proc \rightarrow \Proc$ by the
following equations.

\begin{mathpar}
  (0) \psubstp{Q}{P} := 0 \\
  (R \juxtap S) \psubstp{Q}{P}
  :=    
  (R)\psubstp{Q}{P} \juxtap (S) \psubstp{Q}{P} \\
  (x?(y).R) \psubstp{Q}{P}    
  :=    
  (x)\substp{Q}{P} (z)\concat( (R \psubstn{z}{y}) \psubstp{Q}{P} ) \\
  (\lift{x}{R}) \psubstp{Q}{P}  
  :=
  \lift{(x)\substp{Q}{P}}{ R \psubstp{Q}{P} } \\
%   (\dropn{x})  \psubstp{Q}{P}       
%   := 
%   \left\{ 
%     \begin{array}{ccc} 
%       \dropn{\quotep{Q}} & & x \nameeq \quotep{P} \\
%       \dropn{x} & & otherwise \\
%     \end{array}
%   \right. 
  (\dropn{x})  \psubstp{Q}{P}       
  := 
  \left\{ 
    \begin{array}{ccc} 
      Q & & x \nameeq \quotep{P} \\
      \dropn{x} & & otherwise \\
    \end{array}
  \right.
\end{mathpar}
 

where

\begin{eqnarray}
  (x)\id{\{} \lpquote Q \rpquote / \lpquote P \rpquote \id{\}}            = 
  \left\{ 
    \begin{array}{ccc}
      \lpquote Q \rpquote & & x \nameeq \lpquote P \rpquote \\
      x & & otherwise \\
    \end{array}
  \right. \nonumber
\end{eqnarray}

and $z$ is chosen distinct from $\quotep{P}$, $\quotep{Q}$, the free
names in $Q$, and all the names in $R$. Our $\alpha$-equivalence will
be built in the standard way from this substitution.

\begin{remark}\label{rem:no_self_referential_names}
  One consequence of these definitions is that $\forall P. \quotep{P}
  \not\in \freenames{P}$.
\end{remark}

\subsection{ Dynamic quote: an example }

Anticipating something of what's to come, consider applying the
substitution, $\widehat{\id{\{}u / z \id{\}}}$, to the following pair
of processes, $\lift{w}{y!(z)}$ and $w[ \lpquote y!(z) \rpquote ]$.

\begin{eqnarray}
	\lift{w}{y!(z)}\widehat{\id{\{}u / z \id{\}}}
		& = &
		\lift{w}{y!(u)} \nonumber\\
	w[ \lpquote y!(z) \rpquote ] \widehat{ \id{\{}u / z \id{\}} }
		& = &
		w[ \lpquote y!(z) \rpquote ] \nonumber
\end{eqnarray}

Because the body of the process between quotes is impervious to
substitution, we get radically different answers. In fact, by
examining the first process in an input context,
e.g. $x?(z).\lift{w}{y!(z)}$, we see that the process under the lift
operator may be shaped by prefixed inputs binding a name inside it. In
this sense, the lift operator will be seen as a way to dynamically
construct processes before reifying them as names.

Finally equipped with these standard features we can present the
dynamics of the calculus.

\subsubsection{Operational semantics} 

Finally, we introduce the computational dynamics. What marks these
algebras as distinct from other more traditionally studied algebraic
structures, e.g. vector spaces or polynomial rings, is the manner in
which dynamics is captured. In traditional structures, dynamics is typically
expressed through morphisms between such structures, as in linear maps
between vector spaces or morphisms between rings. In algebras
associated with the semantics of computation, the dynamics is
expressed as part of the algebraic structure itself, through a
reduction reduction relation typically denoted by $\red$. Below, we
give a recursive presentation of this relation for the calculus used
in the encoding.

$\red \subseteq \pi \times \pi$
$\red : \pi \to \mathcal{P}(\pi)$

\begin{mathpar}
  \inferrule* [lab=Comm] { \textsf{match}( x_{src}, x_{trgt} ) } { x_{trgt}?(y)P \; | \; x_{src}!\langle {Q} \rangle \red P\{\quotep{Q}/y}\} }
  \and \\
  \inferrule* [lab=Par] {{P} \red {P}'} {{{P} | {Q}} \red {{P}' | {Q}}}
  \and
  \inferrule* [lab=Equiv]{{{P} \scong {P}'} \andalso {{P}' \red {Q}'} \andalso {{Q}' \scong {Q}}}{{P} \red {Q}}
\end{mathpar}

\begin{eqnarray*}
  match_{\equiv} (\quotep{P},\quotep{Q}) & := & P \equiv Q \\
  match_{\dagger}(\quotep{P},\quotep{Q}) & := & \forall R. P|Q \red^{*} R => R \red^{*} 0 \\
  match_{K}(\quotep{P},\quotep{Q}) & := & K \mbox{ for some context } K
\end{eqnarray*}

$u?(x)P | u!\langle Q \rangle \red P\{\quotep{Q}/x\}$

%We write $\wred$ for $\red^*$, and $P\red$ if $\exists Q $ such that $ P \red Q$.
We write $P\red$ if $\exists Q $ such that $ P \red Q$ and $P\not\red$, otherwise.

\section{Replication}

As mentioned before, it is known that replication (and hence
recursion) can be implemented in a higher-order process algebra
\cite{SangiorgiWalker}. As our first example of calculation with the
machinery thus far presented we give the construction explicitly in
the {\rhoc}.

\begin{eqnarray}
	D_{x} & := & \prefix{x}{y}{(\binpar{\outputp{x}{y}}{@{y}})} \nonumber\\
	\bangp_{x}{P} & := & \binpar{{x}!\langle{\binpar{D_{x}}{P}}\rangle}{D_{x}} \nonumber
\end{eqnarray}

\begin{eqnarray}
	\bangp_{x}{P} & & \nonumber\\
	=
	& {x}!\langle{(\prefix{x}{y}{(\outputp{x}{y} | @{y})) | P}}\rangle 
	      | \prefix{x}{y}{(\outputp{x}{y} | @{y})} & \nonumber\\
	\red
	& (\outputp{x}{y} | @{y})\substn{\quotep{(\prefix{x}{y}{(@{y} | \outputp{x}{y})) | P}}}{y} & \nonumber\\
	=
	& \outputp{x}{\quotep{(\prefix{x}{y}{(\outputp{x}{y} | @{y})) | P}}}
	  | {(\prefix{x}{y}{(\outputp{x}{y} | @{y})) | P}} & \nonumber\\
	\red
	& \ldots & \nonumber\\
	\red^*
	& P | P | \ldots & \nonumber
\end{eqnarray}

Of course, this encoding, as an implementation, runs away, unfolding
$\bangp{P}$ eagerly. A lazier and more implementable replication
operator, restricted to input-guarded processes, may be obtained as follows.

\begin{eqnarray}
\bangp{\prefix{u}{v}{P}} 
	:= 
	\binpar{\lift{x}{\prefix{u}{v}{(\binpar{D(x)}{P})}}}{D(x)} \nonumber
\end{eqnarray}

\begin{remark}
  Note that the lazier definition still does not deal with summation
  or mixed summation (i.e. sums over input and output). The reader is
  invited to construct definitions of replication that deal with these
  features. 

  Further, the definitions are parameterized in a name, $x$. Can you,
  gentle reader, make a definition that eliminates this parameter and
  guarantees no accidental interaction between the replication
  machinery and the process being replicated -- i.e. no accidental
  sharing of names used by the process to get its work done and the
  name(s) used by the replication to effect copying. This latter
  revision of the definition of replication is crucial to obtaining
  the expected identity $!!P \sim !P$.
\end{remark}

\begin{remark}\label{rem:paradoxical_combinator}
  The reader familiar with the lambda calculus will have noticed the
  similarity between $D$ and the paradoxical combinator.

  [Ed. note: the existence of this seems to suggest we have to be more
  restrictive on the set of processes and names we admit if we are to
  support no-cloning.]
\end{remark}

\subsubsection{Bisimulation}

The computational dynamics gives rise to another kind of equivalence,
the equivalence of computational behavior. As previously mentioned
this is typically captured \emph{via} some form of bisimulation.

% The notion we use in this paper is weak barbed bisimulation
% \cite{milner91polyadicpi}.

The notion we use in this paper is derived from weak barbed
bisimulation \cite{milner91polyadicpi}. 

\begin{definition}
An \emph{observation relation}, $\downarrow_{\mathcal N}$, over a set
of names, $\mathcal N$, is the smallest relation satisfying the rules
below.

\infrule[Out-barb]{y \in {\mathcal N}, \; x \nameeq y}
		  {\outputp{x}{v} \downarrow_{\mathcal N} x}
\infrule[Par-barb]{\mbox{$P\downarrow_{\mathcal N} x$ or $Q\downarrow_{\mathcal N} x$}}
		  {\binpar{P}{Q} \downarrow_{\mathcal N} x}

We write $P \Downarrow_{\mathcal N} x$ if there is $Q$ such that 
$P \wred Q$ and $Q \downarrow_{\mathcal N} x$.
\end{definition}

\begin{definition}
%\label{def.bbisim}
An  ${\mathcal N}$-\emph{barbed bisimulation} over a set of names, ${\mathcal N}$, is a symmetric binary relation 
${\mathcal S}_{\mathcal N}$ between agents such that $P\rel{S}_{\mathcal N}Q$ implies:
\begin{enumerate}
\item If $P \red P'$ then $Q \wred Q'$ and $P'\rel{S}_{\mathcal N} Q'$.
\item If $P\downarrow_{\mathcal N} x$, then $Q\Downarrow_{\mathcal N} x$.
\end{enumerate}
$P$ is ${\mathcal N}$-barbed bisimilar to $Q$, written
$P \wbbisim_{\mathcal N} Q$, if $P \rel{S}_{\mathcal N} Q$ for some ${\mathcal N}$-barbed bisimulation ${\mathcal S}_{\mathcal N}$.
\end{definition}

$\mathcal{R} \subseteq \pi \times \pi$

$P \mathcal{R} Q => \forall P'. P \red P' \Rightarrow \exists Q'. Q \red Q', P' \mathcal{R} Q'$

$P \vdash x \Rightarrow Q \vdash x$

\begin{mathpar}
  \inferrule*[lab=Out-barb]{x \nameeq y}{{y}!\langle{Q}\rangle \vdash x}
  \and
  \inferrule*[lab=Par-barb]{\mbox{$P\vdash x$ or $Q\vdash x$}}{\binpar{P}{Q} \vdash x}
\end{mathpar}

\subsubsection{Contexts}

One of the principle advantages of computational calculi like the
$\pi$-calculus is a well-defined notion of context,
contextual-equivalence and a correlation between
contextual-equivalence and notions of bisimulation. The notion of
context allows the decomposition of a process into (sub-)process and
its syntactic environment, its context. Thus, a context may be
thought of as a process with a ``hole'' (written $\Box$) in it. The
application of a context $M$ to a process $P$, written $M[P]$, is
tantamount to filling the hole in $M$ with $P$. In this paper we do
not need the full weight of this theory, but do make use of the notion
of context in the proof the main theorem. 

\begin{mathpar}
  \inferrule* [lab=summation] {} {{M_{M},M_{N}} \bc \Box \;|\; x.M_{A} \;|\; M_{M}+M_{N}}
  \and
  \inferrule* [lab=agent] {} {{M_{A}} \bc (\vec{x})M_{P} \;| \; \clift{P_0,\ldots,M_{P},\ldots,P_N}}
  \and \\
  \inferrule* [lab=process] {} {{M_{P}} \bc M_{N} \;| \;P|M_{P} }
\end{mathpar} 

\begin{mathpar}
  \inferrule* [lab=sychronization] {} {M_{N} \bc \Box \;|\; x?M_{F} \;|\; x!M_{C}}
  \and
  \inferrule* [lab=abstraction] {} {{M_{F}} \bc (x)M_{P} }
  \and
  \inferrule* [lab=concretion] {} {{M_{C}} \bc \langle M_{P} \rangle }
  \and \\
  \inferrule* [lab=process] {} {{M_{P}} \bc M_{N} \;| \;P|M_{P} }
\end{mathpar}

\begin{definition}[contextual application] Given a context $M$, and
  process $P$, we define the \emph{contextual application}, $M[P] :=
  M\{P/\Box\}$. That is, the contextual application of M to P is the
  substitution of $P$ for $\Box$ in $M$.
\end{definition}

$\meaningof{-} : L \to \mathcal{P}(\pi)$

\begin{mathpar}
  \inferrule* [lab=collection] {} {\meaningof{true} = \pi, \and \meaningof{~E} = \pi \setminus \meaningof{E}, \and \meaningof{E_{1} \& E_{2}} = \meaningof{E_{1}} \cap \meaningof{E_{2}}}
\end{mathpar}

\begin{mathpar}
  \inferrule* [lab=structure] {} {\meaningof{0} = \{ P \in \pi | P \equiv 0 \}, \and \\ \meaningof{E_1 | E_2} = \{ P \in \pi | P \equiv P_{1} | P_{2}, P_{1} \in \meaningof{E_{1}}, P_{2} \in \meaningof{E_2}\} }
\end{mathpar}

\begin{mathpar}
 \inferrule* [lab=behavior] {} {\meaningof{\langle a?b \rangle E} = \{ P \in \pi | P \equiv Q | u?(y)P', \\ \and \\\\ \and \\ \;\;\; u \in \meaningof{a}, \forall z.P'\{z/y\} \in \meaningof{E\{z/b\}}\}, \and \\ \meaningof{a!E} = \{ P \in \pi | P \equiv Q | x!\langle P' \rangle, x \in \meaningof{a} P' \in \meaningof{E}\} }
\end{mathpar}

\begin{mathpar}
 \inferrule* [lab=nominal] {} {\meaningof{\quotep{E}} = \{ \quotep{P} \in \quotep{\pi} | P \in \meaningof{E} \}, \and \meaningof{\quotep{P}} = \{ \quotep{Q} \in \quotep{\pi} | P \equiv Q \} \and \\ \meaningof{@\quotep{E}} = \{ P \in \pi | P \equiv @x, x \in \meaningof{E} \}}
\end{mathpar}

\begin{eqnarray*}
  \\
  \meaningof{-} : TS \to ST
\end{eqnarray*}

\begin{eqnarray*}
  \\
  L : TS \to ST
\end{eqnarray*}

\begin{eqnarray*}
  \\
  P \models E \iff P \in \meaningof{E}
\end{eqnarray*}

\begin{eqnarray*}
  P \approx_{L} Q \iff \forall E \in L. P \models E \iff Q \models E
\end{eqnarray*}

\begin{eqnarray*}
  P \approx_{K} Q
\end{eqnarray*}

\begin{eqnarray*}
  P \approx Q
\end{eqnarray*}

$\approx_{K} = \approx = \approx_{L}$

\subsubsection{Contextual duality}

Note that contexts extend the quotation operation to a family of
operations from processes to names. Given a context, $M$, we can
define a \emph{nominal context}, $\quotep{M}$ by $\quotep{M}[P] :=
\quotep{M[P]}$. To foreshadow what is to come we observe that these
operations enjoy a duality with processes very much like the duality
between vectors and maps from vectors to scalars.

Further, because the calculus is essentially higher-order, we have a
correspondence between contexts and processes. More specifically,
given a name $x$ and a context $M$ we can construct $M^{*}_{x}$ such
that 

\begin{mathpar}
  M^{*}_{x} | \lift{x}{P} \red M[P]
\end{mathpar}

namely,

\begin{mathpar}
  M^{*}_{x} := x?(u).M[\dropn{u}]
\end{mathpar}

The dependence of $M^{*}_{x}$ on a name makes it an abstraction, 

\begin{mathpar}
  M^{*} := (x)x?(u).M[\dropn{u}]
\end{mathpar}

\subsection{Additional notation}

It will sometimes be convenient to denote the process a name
quotes. We already have the notation $x = \quotep{P}$, but it will be
convenient to introduce an alternate notation, $\procn{x}$, when we
want to emphasize the connection to the use of the name. Note that, by
virtue of name equivalence, $\quotep{\procn{x}} \nameeq x$; so, the
notation is consistent with previous definitions.

Further, because names have structure it is possible to effect
substitutions on the basis of that structure. This means we need to
upgrade our notation for substitutions, which we accomplish by
adapting comprehension notation. Thus,

\begin{mathpar}
  P\{ y / x : x \in S \}
\end{mathpar}

is interpreted to mean the process derived from P by replacing (in a
capture-avoiding manner) each occurrence of $x$ in $S$ by $y$. For example,

\begin{mathpar}
  P\{ \quotep{\procn{x}|\procn{x}} / x : x \in \freenames{P} \}
\end{mathpar}

will replace each (occurrence) of a free name $x$ in $P$ by
$\quotep{\procn{x}|\procn{x}}$.

Also, we will avail ourselves of the notation $x^{L}$ and $x^{R}$ to
denote injections of a name into disjoint copies of the name
space. There are numerous ways to accomplish this. One example can be
found in \cite{MeredithR05}. This notation overloads to vectors of
names: $\vec{x}^{\pi} := (x_{i}^{\pi} \; : \; 0 \leq i < |\vec{x}| )$ where $\pi \in \{L,R\}$.

We also use $P^{\Box} := P|\Box$.

In \cite{MeredithR05} an interpretation of the new operator is
given. It turns out that there are several possible interpretations
all enjoying the requisite algebraic properties of the operator (see
\cite{milner91polyadicpi}). We will therefore make liberal use of
$(\nu\; \vec{x})P$.

% subsection the_syntax_and_semantics_of_the_notation_system (end)   

\input{qm2pi.qmops} 

\input{qm2pi.sterngerlach} 

\input{qm2pi.metric} 

% section concurrent_process_calculi (end)

%\input{qm2pi.proofsketch}

% section proof sketch (end)

%\input{qm2pi.slviaknots} 

% section spatial logic via knots (end)

\input{qm2pi.conclusion}

% section conclusion (end)

%\input{qm2pi.dtcodes} 

% section wiring algorithm (end)

\input{qm2pi.ack} 

% section acknowledgments (end)

\newpage


\bibliographystyle{plain}   
\bibliography{../../biblios/main.bib}

\input{qm2pi.rhodetails}

\end{document}

 

% section wiring algorithm (end)

\documentclass[12pt]{llncs}
%\documentclass{jktr}

\usepackage[pdftex]{hyperref}                   
\usepackage {listings}
\usepackage {mathpartir}
\usepackage{bcprules}
%\usepackage{listings}
                       
\usepackage{graphicx} 
%\usepackage[margins=2.5cm,nohead,nofoot]{geometry}
%\usepackage{geometry}
\usepackage{amsfonts}
\usepackage{amstext}
\usepackage{latexsym}
\usepackage{amssymb}
\usepackage{color}


%\include{myPreamble}
\include{qm2pi.local} 

%\ifpdf
%\usepackage[pdftex]{graphicx}
%\else
%\usepackage{graphicx}
%\fi

 % \ifpdf
%  \usepackage{pdfsync}
%  \if


%\title{Brief Article}
%\author{David F. Snyder}
%\author{L.G. Meredith}

%\address{Dept. of Math., Texas State University--San Marcos, San Marcos, TX 78666}
       
\pagestyle{empty}


\begin{document}

\lstset{language=[Objective]Caml,frame=shadowbox}

\input{qm2pi.front}

% section front matter (end)

\input{qm2pi.intro} 
 
% section introduction (end)

% \input{qm2pi.knotations} 

% section notation (end)

\input{qm2pi.process.calculi} 

% section concurrent_process_calculi_and_spatial_logics_ (end)
    
%\input{qm2pi.knots2pi} 

%\input{qm2pi.trefoil} 

%\input{qm2pi.mainthm} 

% subsection basic_interpretation (end)

%\input{qm2pi.rho.presentation} 
\subsection{The syntax and semantics of the notation system}\label{sub:the_syntax_and_semantics_of_the_notation_system} % (fold)

We now summarize a technical presentation of the calculus that
embodies our theory of dynamics. The typical presentation of such a
calculus follows the style of giving generators and relations on
them. The grammar, below, describing term constructors, freely
generates the set of processes, $\Proc$. This set is then quotiented
by a relation known as structural congruence and it is over this set
that the notion of dynamics is expressed. This presentation is
essentially that of \cite{MeredithR05} with the addition of
polyadicity and summation. For readability we have relegated some of
the technical subtleties to an appendix.

\subsubsection{Process grammar}\label{subsub:process_grammar}

\begin{mathpar}
  \inferrule* [lab=synchronization] {} {{M} \bc \pzero \;|\; x?F \;|\; x!C }
  \and
  \inferrule* [lab=abstraction] {} {{F} \bc (x)P}
  \and
  \inferrule* [lab=concretion] {} {{C} \bc \langle Q \rangle}
  \and
  \inferrule* [lab=process] {} {{P,Q} \bc M \;| \;P|Q \;|\; @{x}}
  \and
  \inferrule* [lab=name] {} {{x} \bc \quotep{P}}
\end{mathpar} 

Note that $\vec{x}$ (resp. $\vec{P}$) denotes a vector of names
(resp. processes) of length $|\vec{x}|$ (resp. $|\vec{P}|$). We adopt
the following useful abbreviations.

\begin{mathpar}
   x?(\vec{y}).P := x.(\vec{y})P \and  x\clift{\vec{P}} := x.\clift{\vec{P}}
   \and x!(y) := \lift{x}{\dropn{y}}
   \and \Pi_{i=0}^{n-1}P_i := P_0 | \ldots | P_{n-1}
\end{mathpar}

\subsubsection{Structural congruence}

\paragraph{Free and bound names and alpha-equivalence.} At the
core of structural equivalence is alpha-equivalence which identifies
process that are the same up to a change of variable. Formally, we
recognize the distinction between free and bound names. The free names
of a process, $\freenames{P}$, may be calculated recursively as
follows:

\begin{mathpar}
\freenames{\pzero} := \emptyset
  \and \\
  \freenames{x?(y).P} := \{ x \} \cup (\freenames{P} \setminus \{ y \})
  \and 
  \freenames{x!\langle P \rangle} := \{ x \} \cup \{ P \} 
  \and \\
  \freenames{P|Q} := \freenames{P} \cup \freenames{Q}
  \and \\
  \freenames{@{x}} := \{ x \}
\end{mathpar}

$\pi$
$\quotep{\pi}$

$\freenames{-} : \pi \to \mathcal{P}(\quotep{\pi})$

\begin{eqnarray*}
  \freenames{\pzero} & := & \emptyset \\
  \freenames{x?(y).P} & := & \{ x \} \cup (\freenames{P} \setminus \{ y \}) \\
  \freenames{x!\langle P \rangle} & := & \{ x \} \cup \{ P \} \\
  \freenames{P|Q} & := & \freenames{P} \cup \freenames{Q} \\
  \freenames{\dropn{x}} & := & \{ x \}
\end{eqnarray*}

The bound names of a process, $\boundnames{P}$, are those names occurring in $P$
that are not free. For example, in $x?(y).0$, the name $x$ is free, while $y$ is bound.

\begin{mathpar}
  \inferrule* [lab=monoidal-laws] {} { P|Q \equiv Q|P \and P|0 \equiv P \and P|(Q|R) \equiv (P|Q)|R }
\end{mathpar}

\begin{mathpar}
  \inferrule* [lab=alpha-equivalence] {} { (x)P \equiv (y)P\{y/x\} \and y \not\in \freenames{P} }
\end{mathpar}

\begin{definition}
Then two processes, $P,Q$, are alpha-equivalent if $P = Q\{\vec{y}/\vec{x}\}$ for
some $\vec{x} \in \boundnames{Q},\vec{y} \in \boundnames{P}$, where $Q\{\vec{y}/\vec{x}\}$
denotes the capture-avoiding substitution of $\vec{y}$ for $\vec{x}$ in $Q$.
\end{definition}

\begin{definition}
  The {\em structural congruence} \cite{SangiorgiWalker} , $\equiv$,
  between processes is the least congruence containing
  alpha-equivalence, satisfying the abelian monoid laws
  (associativity, commutativity and $\pzero$ as identity) for parallel
  composition $|$ and for summation $+$.
\end{definition}

\subsection{Name equivalence}

We take name equivalence, written $\nameeq$, to be the smallest
equivalence relation generated by the following rules.

\begin{mathpar}
\inferrule*[lab=Quote-drop]
{ }
{ \quotep{@{x}} \nameeq x }

\inferrule*[lab=Struct-equiv]
{ P \scong Q }
{ \quotep{P} \nameeq \quotep{Q} }
\end{mathpar}

The astute reader will have noticed that the mutual recursion of names
and processes imposes a mutual recursion on alpha-equivalence and
structural equivalence via name-equivalence. Fortunately, all of this
works out pleasantly and we may calculate in the natural way, free of
concern. The reader interested in the details is referred to the
appendix \ref{appendix:rho_details}.

\subsection{Substitution}

We use $\Proc$ for the set of processes, $\QProc$ for the set of
names, and $\id{\{}\vec{y} / \vec{x} \id{\}}$ to denote partial maps,
$s : \QProc \rightarrow \QProc$. A map, $s$ lifts, uniquely, to a map
on process terms, $\widehat{s} : \Proc \rightarrow \Proc$ by the
following equations.

\begin{mathpar}
  (0) \psubstp{Q}{P} := 0 \\
  (R \juxtap S) \psubstp{Q}{P}
  :=    
  (R)\psubstp{Q}{P} \juxtap (S) \psubstp{Q}{P} \\
  (x?(y).R) \psubstp{Q}{P}    
  :=    
  (x)\substp{Q}{P} (z)\concat( (R \psubstn{z}{y}) \psubstp{Q}{P} ) \\
  (\lift{x}{R}) \psubstp{Q}{P}  
  :=
  \lift{(x)\substp{Q}{P}}{ R \psubstp{Q}{P} } \\
%   (\dropn{x})  \psubstp{Q}{P}       
%   := 
%   \left\{ 
%     \begin{array}{ccc} 
%       \dropn{\quotep{Q}} & & x \nameeq \quotep{P} \\
%       \dropn{x} & & otherwise \\
%     \end{array}
%   \right. 
  (\dropn{x})  \psubstp{Q}{P}       
  := 
  \left\{ 
    \begin{array}{ccc} 
      Q & & x \nameeq \quotep{P} \\
      \dropn{x} & & otherwise \\
    \end{array}
  \right.
\end{mathpar}
 

where

\begin{eqnarray}
  (x)\id{\{} \lpquote Q \rpquote / \lpquote P \rpquote \id{\}}            = 
  \left\{ 
    \begin{array}{ccc}
      \lpquote Q \rpquote & & x \nameeq \lpquote P \rpquote \\
      x & & otherwise \\
    \end{array}
  \right. \nonumber
\end{eqnarray}

and $z$ is chosen distinct from $\quotep{P}$, $\quotep{Q}$, the free
names in $Q$, and all the names in $R$. Our $\alpha$-equivalence will
be built in the standard way from this substitution.

\begin{remark}\label{rem:no_self_referential_names}
  One consequence of these definitions is that $\forall P. \quotep{P}
  \not\in \freenames{P}$.
\end{remark}

\subsection{ Dynamic quote: an example }

Anticipating something of what's to come, consider applying the
substitution, $\widehat{\id{\{}u / z \id{\}}}$, to the following pair
of processes, $\lift{w}{y!(z)}$ and $w[ \lpquote y!(z) \rpquote ]$.

\begin{eqnarray}
	\lift{w}{y!(z)}\widehat{\id{\{}u / z \id{\}}}
		& = &
		\lift{w}{y!(u)} \nonumber\\
	w[ \lpquote y!(z) \rpquote ] \widehat{ \id{\{}u / z \id{\}} }
		& = &
		w[ \lpquote y!(z) \rpquote ] \nonumber
\end{eqnarray}

Because the body of the process between quotes is impervious to
substitution, we get radically different answers. In fact, by
examining the first process in an input context,
e.g. $x?(z).\lift{w}{y!(z)}$, we see that the process under the lift
operator may be shaped by prefixed inputs binding a name inside it. In
this sense, the lift operator will be seen as a way to dynamically
construct processes before reifying them as names.

Finally equipped with these standard features we can present the
dynamics of the calculus.

\subsubsection{Operational semantics} 

Finally, we introduce the computational dynamics. What marks these
algebras as distinct from other more traditionally studied algebraic
structures, e.g. vector spaces or polynomial rings, is the manner in
which dynamics is captured. In traditional structures, dynamics is typically
expressed through morphisms between such structures, as in linear maps
between vector spaces or morphisms between rings. In algebras
associated with the semantics of computation, the dynamics is
expressed as part of the algebraic structure itself, through a
reduction reduction relation typically denoted by $\red$. Below, we
give a recursive presentation of this relation for the calculus used
in the encoding.

$\red \subseteq \pi \times \pi$
$\red : \pi \to \mathcal{P}(\pi)$

\begin{mathpar}
  \inferrule* [lab=Comm] { \textsf{match}( x_{src}, x_{trgt} ) } { x_{trgt}?(y)P \; | \; x_{src}!\langle {Q} \rangle \red P\{\quotep{Q}/y}\} }
  \and \\
  \inferrule* [lab=Par] {{P} \red {P}'} {{{P} | {Q}} \red {{P}' | {Q}}}
  \and
  \inferrule* [lab=Equiv]{{{P} \scong {P}'} \andalso {{P}' \red {Q}'} \andalso {{Q}' \scong {Q}}}{{P} \red {Q}}
\end{mathpar}

\begin{eqnarray*}
  match_{\equiv} (\quotep{P},\quotep{Q}) & := & P \equiv Q \\
  match_{\dagger}(\quotep{P},\quotep{Q}) & := & \forall R. P|Q \red^{*} R => R \red^{*} 0 \\
  match_{K}(\quotep{P},\quotep{Q}) & := & K \mbox{ for some context } K
\end{eqnarray*}

$u?(x)P | u!\langle Q \rangle \red P\{\quotep{Q}/x\}$

%We write $\wred$ for $\red^*$, and $P\red$ if $\exists Q $ such that $ P \red Q$.
We write $P\red$ if $\exists Q $ such that $ P \red Q$ and $P\not\red$, otherwise.

\section{Replication}

As mentioned before, it is known that replication (and hence
recursion) can be implemented in a higher-order process algebra
\cite{SangiorgiWalker}. As our first example of calculation with the
machinery thus far presented we give the construction explicitly in
the {\rhoc}.

\begin{eqnarray}
	D_{x} & := & \prefix{x}{y}{(\binpar{\outputp{x}{y}}{@{y}})} \nonumber\\
	\bangp_{x}{P} & := & \binpar{{x}!\langle{\binpar{D_{x}}{P}}\rangle}{D_{x}} \nonumber
\end{eqnarray}

\begin{eqnarray}
	\bangp_{x}{P} & & \nonumber\\
	=
	& {x}!\langle{(\prefix{x}{y}{(\outputp{x}{y} | @{y})) | P}}\rangle 
	      | \prefix{x}{y}{(\outputp{x}{y} | @{y})} & \nonumber\\
	\red
	& (\outputp{x}{y} | @{y})\substn{\quotep{(\prefix{x}{y}{(@{y} | \outputp{x}{y})) | P}}}{y} & \nonumber\\
	=
	& \outputp{x}{\quotep{(\prefix{x}{y}{(\outputp{x}{y} | @{y})) | P}}}
	  | {(\prefix{x}{y}{(\outputp{x}{y} | @{y})) | P}} & \nonumber\\
	\red
	& \ldots & \nonumber\\
	\red^*
	& P | P | \ldots & \nonumber
\end{eqnarray}

Of course, this encoding, as an implementation, runs away, unfolding
$\bangp{P}$ eagerly. A lazier and more implementable replication
operator, restricted to input-guarded processes, may be obtained as follows.

\begin{eqnarray}
\bangp{\prefix{u}{v}{P}} 
	:= 
	\binpar{\lift{x}{\prefix{u}{v}{(\binpar{D(x)}{P})}}}{D(x)} \nonumber
\end{eqnarray}

\begin{remark}
  Note that the lazier definition still does not deal with summation
  or mixed summation (i.e. sums over input and output). The reader is
  invited to construct definitions of replication that deal with these
  features. 

  Further, the definitions are parameterized in a name, $x$. Can you,
  gentle reader, make a definition that eliminates this parameter and
  guarantees no accidental interaction between the replication
  machinery and the process being replicated -- i.e. no accidental
  sharing of names used by the process to get its work done and the
  name(s) used by the replication to effect copying. This latter
  revision of the definition of replication is crucial to obtaining
  the expected identity $!!P \sim !P$.
\end{remark}

\begin{remark}\label{rem:paradoxical_combinator}
  The reader familiar with the lambda calculus will have noticed the
  similarity between $D$ and the paradoxical combinator.

  [Ed. note: the existence of this seems to suggest we have to be more
  restrictive on the set of processes and names we admit if we are to
  support no-cloning.]
\end{remark}

\subsubsection{Bisimulation}

The computational dynamics gives rise to another kind of equivalence,
the equivalence of computational behavior. As previously mentioned
this is typically captured \emph{via} some form of bisimulation.

% The notion we use in this paper is weak barbed bisimulation
% \cite{milner91polyadicpi}.

The notion we use in this paper is derived from weak barbed
bisimulation \cite{milner91polyadicpi}. 

\begin{definition}
An \emph{observation relation}, $\downarrow_{\mathcal N}$, over a set
of names, $\mathcal N$, is the smallest relation satisfying the rules
below.

\infrule[Out-barb]{y \in {\mathcal N}, \; x \nameeq y}
		  {\outputp{x}{v} \downarrow_{\mathcal N} x}
\infrule[Par-barb]{\mbox{$P\downarrow_{\mathcal N} x$ or $Q\downarrow_{\mathcal N} x$}}
		  {\binpar{P}{Q} \downarrow_{\mathcal N} x}

We write $P \Downarrow_{\mathcal N} x$ if there is $Q$ such that 
$P \wred Q$ and $Q \downarrow_{\mathcal N} x$.
\end{definition}

\begin{definition}
%\label{def.bbisim}
An  ${\mathcal N}$-\emph{barbed bisimulation} over a set of names, ${\mathcal N}$, is a symmetric binary relation 
${\mathcal S}_{\mathcal N}$ between agents such that $P\rel{S}_{\mathcal N}Q$ implies:
\begin{enumerate}
\item If $P \red P'$ then $Q \wred Q'$ and $P'\rel{S}_{\mathcal N} Q'$.
\item If $P\downarrow_{\mathcal N} x$, then $Q\Downarrow_{\mathcal N} x$.
\end{enumerate}
$P$ is ${\mathcal N}$-barbed bisimilar to $Q$, written
$P \wbbisim_{\mathcal N} Q$, if $P \rel{S}_{\mathcal N} Q$ for some ${\mathcal N}$-barbed bisimulation ${\mathcal S}_{\mathcal N}$.
\end{definition}

$\mathcal{R} \subseteq \pi \times \pi$

$P \mathcal{R} Q => \forall P'. P \red P' \Rightarrow \exists Q'. Q \red Q', P' \mathcal{R} Q'$

$P \vdash x \Rightarrow Q \vdash x$

\begin{mathpar}
  \inferrule*[lab=Out-barb]{x \nameeq y}{{y}!\langle{Q}\rangle \vdash x}
  \and
  \inferrule*[lab=Par-barb]{\mbox{$P\vdash x$ or $Q\vdash x$}}{\binpar{P}{Q} \vdash x}
\end{mathpar}

\subsubsection{Contexts}

One of the principle advantages of computational calculi like the
$\pi$-calculus is a well-defined notion of context,
contextual-equivalence and a correlation between
contextual-equivalence and notions of bisimulation. The notion of
context allows the decomposition of a process into (sub-)process and
its syntactic environment, its context. Thus, a context may be
thought of as a process with a ``hole'' (written $\Box$) in it. The
application of a context $M$ to a process $P$, written $M[P]$, is
tantamount to filling the hole in $M$ with $P$. In this paper we do
not need the full weight of this theory, but do make use of the notion
of context in the proof the main theorem. 

\begin{mathpar}
  \inferrule* [lab=summation] {} {{M_{M},M_{N}} \bc \Box \;|\; x.M_{A} \;|\; M_{M}+M_{N}}
  \and
  \inferrule* [lab=agent] {} {{M_{A}} \bc (\vec{x})M_{P} \;| \; \clift{P_0,\ldots,M_{P},\ldots,P_N}}
  \and \\
  \inferrule* [lab=process] {} {{M_{P}} \bc M_{N} \;| \;P|M_{P} }
\end{mathpar} 

\begin{mathpar}
  \inferrule* [lab=sychronization] {} {M_{N} \bc \Box \;|\; x?M_{F} \;|\; x!M_{C}}
  \and
  \inferrule* [lab=abstraction] {} {{M_{F}} \bc (x)M_{P} }
  \and
  \inferrule* [lab=concretion] {} {{M_{C}} \bc \langle M_{P} \rangle }
  \and \\
  \inferrule* [lab=process] {} {{M_{P}} \bc M_{N} \;| \;P|M_{P} }
\end{mathpar}

\begin{definition}[contextual application] Given a context $M$, and
  process $P$, we define the \emph{contextual application}, $M[P] :=
  M\{P/\Box\}$. That is, the contextual application of M to P is the
  substitution of $P$ for $\Box$ in $M$.
\end{definition}

$\meaningof{-} : L \to \mathcal{P}(\pi)$

\begin{mathpar}
  \inferrule* [lab=collection] {} {\meaningof{true} = \pi, \and \meaningof{~E} = \pi \setminus \meaningof{E}, \and \meaningof{E_{1} \& E_{2}} = \meaningof{E_{1}} \cap \meaningof{E_{2}}}
\end{mathpar}

\begin{mathpar}
  \inferrule* [lab=structure] {} {\meaningof{0} = \{ P \in \pi | P \equiv 0 \}, \and \\ \meaningof{E_1 | E_2} = \{ P \in \pi | P \equiv P_{1} | P_{2}, P_{1} \in \meaningof{E_{1}}, P_{2} \in \meaningof{E_2}\} }
\end{mathpar}

\begin{mathpar}
 \inferrule* [lab=behavior] {} {\meaningof{\langle a?b \rangle E} = \{ P \in \pi | P \equiv Q | u?(y)P', \\ \and \\\\ \and \\ \;\;\; u \in \meaningof{a}, \forall z.P'\{z/y\} \in \meaningof{E\{z/b\}}\}, \and \\ \meaningof{a!E} = \{ P \in \pi | P \equiv Q | x!\langle P' \rangle, x \in \meaningof{a} P' \in \meaningof{E}\} }
\end{mathpar}

\begin{mathpar}
 \inferrule* [lab=nominal] {} {\meaningof{\quotep{E}} = \{ \quotep{P} \in \quotep{\pi} | P \in \meaningof{E} \}, \and \meaningof{\quotep{P}} = \{ \quotep{Q} \in \quotep{\pi} | P \equiv Q \} \and \\ \meaningof{@\quotep{E}} = \{ P \in \pi | P \equiv @x, x \in \meaningof{E} \}}
\end{mathpar}

\begin{eqnarray*}
  \\
  \meaningof{-} : TS \to ST
\end{eqnarray*}

\begin{eqnarray*}
  \\
  L : TS \to ST
\end{eqnarray*}

\begin{eqnarray*}
  \\
  P \models E \iff P \in \meaningof{E}
\end{eqnarray*}

\begin{eqnarray*}
  P \approx_{L} Q \iff \forall E \in L. P \models E \iff Q \models E
\end{eqnarray*}

\begin{eqnarray*}
  P \approx_{K} Q
\end{eqnarray*}

\begin{eqnarray*}
  P \approx Q
\end{eqnarray*}

$\approx_{K} = \approx = \approx_{L}$

\subsubsection{Contextual duality}

Note that contexts extend the quotation operation to a family of
operations from processes to names. Given a context, $M$, we can
define a \emph{nominal context}, $\quotep{M}$ by $\quotep{M}[P] :=
\quotep{M[P]}$. To foreshadow what is to come we observe that these
operations enjoy a duality with processes very much like the duality
between vectors and maps from vectors to scalars.

Further, because the calculus is essentially higher-order, we have a
correspondence between contexts and processes. More specifically,
given a name $x$ and a context $M$ we can construct $M^{*}_{x}$ such
that 

\begin{mathpar}
  M^{*}_{x} | \lift{x}{P} \red M[P]
\end{mathpar}

namely,

\begin{mathpar}
  M^{*}_{x} := x?(u).M[\dropn{u}]
\end{mathpar}

The dependence of $M^{*}_{x}$ on a name makes it an abstraction, 

\begin{mathpar}
  M^{*} := (x)x?(u).M[\dropn{u}]
\end{mathpar}

\subsection{Additional notation}

It will sometimes be convenient to denote the process a name
quotes. We already have the notation $x = \quotep{P}$, but it will be
convenient to introduce an alternate notation, $\procn{x}$, when we
want to emphasize the connection to the use of the name. Note that, by
virtue of name equivalence, $\quotep{\procn{x}} \nameeq x$; so, the
notation is consistent with previous definitions.

Further, because names have structure it is possible to effect
substitutions on the basis of that structure. This means we need to
upgrade our notation for substitutions, which we accomplish by
adapting comprehension notation. Thus,

\begin{mathpar}
  P\{ y / x : x \in S \}
\end{mathpar}

is interpreted to mean the process derived from P by replacing (in a
capture-avoiding manner) each occurrence of $x$ in $S$ by $y$. For example,

\begin{mathpar}
  P\{ \quotep{\procn{x}|\procn{x}} / x : x \in \freenames{P} \}
\end{mathpar}

will replace each (occurrence) of a free name $x$ in $P$ by
$\quotep{\procn{x}|\procn{x}}$.

Also, we will avail ourselves of the notation $x^{L}$ and $x^{R}$ to
denote injections of a name into disjoint copies of the name
space. There are numerous ways to accomplish this. One example can be
found in \cite{MeredithR05}. This notation overloads to vectors of
names: $\vec{x}^{\pi} := (x_{i}^{\pi} \; : \; 0 \leq i < |\vec{x}| )$ where $\pi \in \{L,R\}$.

We also use $P^{\Box} := P|\Box$.

In \cite{MeredithR05} an interpretation of the new operator is
given. It turns out that there are several possible interpretations
all enjoying the requisite algebraic properties of the operator (see
\cite{milner91polyadicpi}). We will therefore make liberal use of
$(\nu\; \vec{x})P$.

% subsection the_syntax_and_semantics_of_the_notation_system (end)   

\input{qm2pi.qmops} 

\input{qm2pi.sterngerlach} 

\input{qm2pi.metric} 

% section concurrent_process_calculi (end)

%\input{qm2pi.proofsketch}

% section proof sketch (end)

%\input{qm2pi.slviaknots} 

% section spatial logic via knots (end)

\input{qm2pi.conclusion}

% section conclusion (end)

%\input{qm2pi.dtcodes} 

% section wiring algorithm (end)

\input{qm2pi.ack} 

% section acknowledgments (end)

\newpage


\bibliographystyle{plain}   
\bibliography{../../biblios/main.bib}

\input{qm2pi.rhodetails}

\end{document}

 

% section acknowledgments (end)

\newpage


\bibliographystyle{plain}   
\bibliography{../../biblios/main.bib}

\documentclass[12pt]{llncs}
%\documentclass{jktr}

\usepackage[pdftex]{hyperref}                   
\usepackage {listings}
\usepackage {mathpartir}
\usepackage{bcprules}
%\usepackage{listings}
                       
\usepackage{graphicx} 
%\usepackage[margins=2.5cm,nohead,nofoot]{geometry}
%\usepackage{geometry}
\usepackage{amsfonts}
\usepackage{amstext}
\usepackage{latexsym}
\usepackage{amssymb}
\usepackage{color}


%\include{myPreamble}
\include{qm2pi.local} 

%\ifpdf
%\usepackage[pdftex]{graphicx}
%\else
%\usepackage{graphicx}
%\fi

 % \ifpdf
%  \usepackage{pdfsync}
%  \if


%\title{Brief Article}
%\author{David F. Snyder}
%\author{L.G. Meredith}

%\address{Dept. of Math., Texas State University--San Marcos, San Marcos, TX 78666}
       
\pagestyle{empty}


\begin{document}

\lstset{language=[Objective]Caml,frame=shadowbox}

\input{qm2pi.front}

% section front matter (end)

\input{qm2pi.intro} 
 
% section introduction (end)

% \input{qm2pi.knotations} 

% section notation (end)

\input{qm2pi.process.calculi} 

% section concurrent_process_calculi_and_spatial_logics_ (end)
    
%\input{qm2pi.knots2pi} 

%\input{qm2pi.trefoil} 

%\input{qm2pi.mainthm} 

% subsection basic_interpretation (end)

%\input{qm2pi.rho.presentation} 
\subsection{The syntax and semantics of the notation system}\label{sub:the_syntax_and_semantics_of_the_notation_system} % (fold)

We now summarize a technical presentation of the calculus that
embodies our theory of dynamics. The typical presentation of such a
calculus follows the style of giving generators and relations on
them. The grammar, below, describing term constructors, freely
generates the set of processes, $\Proc$. This set is then quotiented
by a relation known as structural congruence and it is over this set
that the notion of dynamics is expressed. This presentation is
essentially that of \cite{MeredithR05} with the addition of
polyadicity and summation. For readability we have relegated some of
the technical subtleties to an appendix.

\subsubsection{Process grammar}\label{subsub:process_grammar}

\begin{mathpar}
  \inferrule* [lab=synchronization] {} {{M} \bc \pzero \;|\; x?F \;|\; x!C }
  \and
  \inferrule* [lab=abstraction] {} {{F} \bc (x)P}
  \and
  \inferrule* [lab=concretion] {} {{C} \bc \langle Q \rangle}
  \and
  \inferrule* [lab=process] {} {{P,Q} \bc M \;| \;P|Q \;|\; @{x}}
  \and
  \inferrule* [lab=name] {} {{x} \bc \quotep{P}}
\end{mathpar} 

Note that $\vec{x}$ (resp. $\vec{P}$) denotes a vector of names
(resp. processes) of length $|\vec{x}|$ (resp. $|\vec{P}|$). We adopt
the following useful abbreviations.

\begin{mathpar}
   x?(\vec{y}).P := x.(\vec{y})P \and  x\clift{\vec{P}} := x.\clift{\vec{P}}
   \and x!(y) := \lift{x}{\dropn{y}}
   \and \Pi_{i=0}^{n-1}P_i := P_0 | \ldots | P_{n-1}
\end{mathpar}

\subsubsection{Structural congruence}

\paragraph{Free and bound names and alpha-equivalence.} At the
core of structural equivalence is alpha-equivalence which identifies
process that are the same up to a change of variable. Formally, we
recognize the distinction between free and bound names. The free names
of a process, $\freenames{P}$, may be calculated recursively as
follows:

\begin{mathpar}
\freenames{\pzero} := \emptyset
  \and \\
  \freenames{x?(y).P} := \{ x \} \cup (\freenames{P} \setminus \{ y \})
  \and 
  \freenames{x!\langle P \rangle} := \{ x \} \cup \{ P \} 
  \and \\
  \freenames{P|Q} := \freenames{P} \cup \freenames{Q}
  \and \\
  \freenames{@{x}} := \{ x \}
\end{mathpar}

$\pi$
$\quotep{\pi}$

$\freenames{-} : \pi \to \mathcal{P}(\quotep{\pi})$

\begin{eqnarray*}
  \freenames{\pzero} & := & \emptyset \\
  \freenames{x?(y).P} & := & \{ x \} \cup (\freenames{P} \setminus \{ y \}) \\
  \freenames{x!\langle P \rangle} & := & \{ x \} \cup \{ P \} \\
  \freenames{P|Q} & := & \freenames{P} \cup \freenames{Q} \\
  \freenames{\dropn{x}} & := & \{ x \}
\end{eqnarray*}

The bound names of a process, $\boundnames{P}$, are those names occurring in $P$
that are not free. For example, in $x?(y).0$, the name $x$ is free, while $y$ is bound.

\begin{mathpar}
  \inferrule* [lab=monoidal-laws] {} { P|Q \equiv Q|P \and P|0 \equiv P \and P|(Q|R) \equiv (P|Q)|R }
\end{mathpar}

\begin{mathpar}
  \inferrule* [lab=alpha-equivalence] {} { (x)P \equiv (y)P\{y/x\} \and y \not\in \freenames{P} }
\end{mathpar}

\begin{definition}
Then two processes, $P,Q$, are alpha-equivalent if $P = Q\{\vec{y}/\vec{x}\}$ for
some $\vec{x} \in \boundnames{Q},\vec{y} \in \boundnames{P}$, where $Q\{\vec{y}/\vec{x}\}$
denotes the capture-avoiding substitution of $\vec{y}$ for $\vec{x}$ in $Q$.
\end{definition}

\begin{definition}
  The {\em structural congruence} \cite{SangiorgiWalker} , $\equiv$,
  between processes is the least congruence containing
  alpha-equivalence, satisfying the abelian monoid laws
  (associativity, commutativity and $\pzero$ as identity) for parallel
  composition $|$ and for summation $+$.
\end{definition}

\subsection{Name equivalence}

We take name equivalence, written $\nameeq$, to be the smallest
equivalence relation generated by the following rules.

\begin{mathpar}
\inferrule*[lab=Quote-drop]
{ }
{ \quotep{@{x}} \nameeq x }

\inferrule*[lab=Struct-equiv]
{ P \scong Q }
{ \quotep{P} \nameeq \quotep{Q} }
\end{mathpar}

The astute reader will have noticed that the mutual recursion of names
and processes imposes a mutual recursion on alpha-equivalence and
structural equivalence via name-equivalence. Fortunately, all of this
works out pleasantly and we may calculate in the natural way, free of
concern. The reader interested in the details is referred to the
appendix \ref{appendix:rho_details}.

\subsection{Substitution}

We use $\Proc$ for the set of processes, $\QProc$ for the set of
names, and $\id{\{}\vec{y} / \vec{x} \id{\}}$ to denote partial maps,
$s : \QProc \rightarrow \QProc$. A map, $s$ lifts, uniquely, to a map
on process terms, $\widehat{s} : \Proc \rightarrow \Proc$ by the
following equations.

\begin{mathpar}
  (0) \psubstp{Q}{P} := 0 \\
  (R \juxtap S) \psubstp{Q}{P}
  :=    
  (R)\psubstp{Q}{P} \juxtap (S) \psubstp{Q}{P} \\
  (x?(y).R) \psubstp{Q}{P}    
  :=    
  (x)\substp{Q}{P} (z)\concat( (R \psubstn{z}{y}) \psubstp{Q}{P} ) \\
  (\lift{x}{R}) \psubstp{Q}{P}  
  :=
  \lift{(x)\substp{Q}{P}}{ R \psubstp{Q}{P} } \\
%   (\dropn{x})  \psubstp{Q}{P}       
%   := 
%   \left\{ 
%     \begin{array}{ccc} 
%       \dropn{\quotep{Q}} & & x \nameeq \quotep{P} \\
%       \dropn{x} & & otherwise \\
%     \end{array}
%   \right. 
  (\dropn{x})  \psubstp{Q}{P}       
  := 
  \left\{ 
    \begin{array}{ccc} 
      Q & & x \nameeq \quotep{P} \\
      \dropn{x} & & otherwise \\
    \end{array}
  \right.
\end{mathpar}
 

where

\begin{eqnarray}
  (x)\id{\{} \lpquote Q \rpquote / \lpquote P \rpquote \id{\}}            = 
  \left\{ 
    \begin{array}{ccc}
      \lpquote Q \rpquote & & x \nameeq \lpquote P \rpquote \\
      x & & otherwise \\
    \end{array}
  \right. \nonumber
\end{eqnarray}

and $z$ is chosen distinct from $\quotep{P}$, $\quotep{Q}$, the free
names in $Q$, and all the names in $R$. Our $\alpha$-equivalence will
be built in the standard way from this substitution.

\begin{remark}\label{rem:no_self_referential_names}
  One consequence of these definitions is that $\forall P. \quotep{P}
  \not\in \freenames{P}$.
\end{remark}

\subsection{ Dynamic quote: an example }

Anticipating something of what's to come, consider applying the
substitution, $\widehat{\id{\{}u / z \id{\}}}$, to the following pair
of processes, $\lift{w}{y!(z)}$ and $w[ \lpquote y!(z) \rpquote ]$.

\begin{eqnarray}
	\lift{w}{y!(z)}\widehat{\id{\{}u / z \id{\}}}
		& = &
		\lift{w}{y!(u)} \nonumber\\
	w[ \lpquote y!(z) \rpquote ] \widehat{ \id{\{}u / z \id{\}} }
		& = &
		w[ \lpquote y!(z) \rpquote ] \nonumber
\end{eqnarray}

Because the body of the process between quotes is impervious to
substitution, we get radically different answers. In fact, by
examining the first process in an input context,
e.g. $x?(z).\lift{w}{y!(z)}$, we see that the process under the lift
operator may be shaped by prefixed inputs binding a name inside it. In
this sense, the lift operator will be seen as a way to dynamically
construct processes before reifying them as names.

Finally equipped with these standard features we can present the
dynamics of the calculus.

\subsubsection{Operational semantics} 

Finally, we introduce the computational dynamics. What marks these
algebras as distinct from other more traditionally studied algebraic
structures, e.g. vector spaces or polynomial rings, is the manner in
which dynamics is captured. In traditional structures, dynamics is typically
expressed through morphisms between such structures, as in linear maps
between vector spaces or morphisms between rings. In algebras
associated with the semantics of computation, the dynamics is
expressed as part of the algebraic structure itself, through a
reduction reduction relation typically denoted by $\red$. Below, we
give a recursive presentation of this relation for the calculus used
in the encoding.

$\red \subseteq \pi \times \pi$
$\red : \pi \to \mathcal{P}(\pi)$

\begin{mathpar}
  \inferrule* [lab=Comm] { \textsf{match}( x_{src}, x_{trgt} ) } { x_{trgt}?(y)P \; | \; x_{src}!\langle {Q} \rangle \red P\{\quotep{Q}/y}\} }
  \and \\
  \inferrule* [lab=Par] {{P} \red {P}'} {{{P} | {Q}} \red {{P}' | {Q}}}
  \and
  \inferrule* [lab=Equiv]{{{P} \scong {P}'} \andalso {{P}' \red {Q}'} \andalso {{Q}' \scong {Q}}}{{P} \red {Q}}
\end{mathpar}

\begin{eqnarray*}
  match_{\equiv} (\quotep{P},\quotep{Q}) & := & P \equiv Q \\
  match_{\dagger}(\quotep{P},\quotep{Q}) & := & \forall R. P|Q \red^{*} R => R \red^{*} 0 \\
  match_{K}(\quotep{P},\quotep{Q}) & := & K \mbox{ for some context } K
\end{eqnarray*}

$u?(x)P | u!\langle Q \rangle \red P\{\quotep{Q}/x\}$

%We write $\wred$ for $\red^*$, and $P\red$ if $\exists Q $ such that $ P \red Q$.
We write $P\red$ if $\exists Q $ such that $ P \red Q$ and $P\not\red$, otherwise.

\section{Replication}

As mentioned before, it is known that replication (and hence
recursion) can be implemented in a higher-order process algebra
\cite{SangiorgiWalker}. As our first example of calculation with the
machinery thus far presented we give the construction explicitly in
the {\rhoc}.

\begin{eqnarray}
	D_{x} & := & \prefix{x}{y}{(\binpar{\outputp{x}{y}}{@{y}})} \nonumber\\
	\bangp_{x}{P} & := & \binpar{{x}!\langle{\binpar{D_{x}}{P}}\rangle}{D_{x}} \nonumber
\end{eqnarray}

\begin{eqnarray}
	\bangp_{x}{P} & & \nonumber\\
	=
	& {x}!\langle{(\prefix{x}{y}{(\outputp{x}{y} | @{y})) | P}}\rangle 
	      | \prefix{x}{y}{(\outputp{x}{y} | @{y})} & \nonumber\\
	\red
	& (\outputp{x}{y} | @{y})\substn{\quotep{(\prefix{x}{y}{(@{y} | \outputp{x}{y})) | P}}}{y} & \nonumber\\
	=
	& \outputp{x}{\quotep{(\prefix{x}{y}{(\outputp{x}{y} | @{y})) | P}}}
	  | {(\prefix{x}{y}{(\outputp{x}{y} | @{y})) | P}} & \nonumber\\
	\red
	& \ldots & \nonumber\\
	\red^*
	& P | P | \ldots & \nonumber
\end{eqnarray}

Of course, this encoding, as an implementation, runs away, unfolding
$\bangp{P}$ eagerly. A lazier and more implementable replication
operator, restricted to input-guarded processes, may be obtained as follows.

\begin{eqnarray}
\bangp{\prefix{u}{v}{P}} 
	:= 
	\binpar{\lift{x}{\prefix{u}{v}{(\binpar{D(x)}{P})}}}{D(x)} \nonumber
\end{eqnarray}

\begin{remark}
  Note that the lazier definition still does not deal with summation
  or mixed summation (i.e. sums over input and output). The reader is
  invited to construct definitions of replication that deal with these
  features. 

  Further, the definitions are parameterized in a name, $x$. Can you,
  gentle reader, make a definition that eliminates this parameter and
  guarantees no accidental interaction between the replication
  machinery and the process being replicated -- i.e. no accidental
  sharing of names used by the process to get its work done and the
  name(s) used by the replication to effect copying. This latter
  revision of the definition of replication is crucial to obtaining
  the expected identity $!!P \sim !P$.
\end{remark}

\begin{remark}\label{rem:paradoxical_combinator}
  The reader familiar with the lambda calculus will have noticed the
  similarity between $D$ and the paradoxical combinator.

  [Ed. note: the existence of this seems to suggest we have to be more
  restrictive on the set of processes and names we admit if we are to
  support no-cloning.]
\end{remark}

\subsubsection{Bisimulation}

The computational dynamics gives rise to another kind of equivalence,
the equivalence of computational behavior. As previously mentioned
this is typically captured \emph{via} some form of bisimulation.

% The notion we use in this paper is weak barbed bisimulation
% \cite{milner91polyadicpi}.

The notion we use in this paper is derived from weak barbed
bisimulation \cite{milner91polyadicpi}. 

\begin{definition}
An \emph{observation relation}, $\downarrow_{\mathcal N}$, over a set
of names, $\mathcal N$, is the smallest relation satisfying the rules
below.

\infrule[Out-barb]{y \in {\mathcal N}, \; x \nameeq y}
		  {\outputp{x}{v} \downarrow_{\mathcal N} x}
\infrule[Par-barb]{\mbox{$P\downarrow_{\mathcal N} x$ or $Q\downarrow_{\mathcal N} x$}}
		  {\binpar{P}{Q} \downarrow_{\mathcal N} x}

We write $P \Downarrow_{\mathcal N} x$ if there is $Q$ such that 
$P \wred Q$ and $Q \downarrow_{\mathcal N} x$.
\end{definition}

\begin{definition}
%\label{def.bbisim}
An  ${\mathcal N}$-\emph{barbed bisimulation} over a set of names, ${\mathcal N}$, is a symmetric binary relation 
${\mathcal S}_{\mathcal N}$ between agents such that $P\rel{S}_{\mathcal N}Q$ implies:
\begin{enumerate}
\item If $P \red P'$ then $Q \wred Q'$ and $P'\rel{S}_{\mathcal N} Q'$.
\item If $P\downarrow_{\mathcal N} x$, then $Q\Downarrow_{\mathcal N} x$.
\end{enumerate}
$P$ is ${\mathcal N}$-barbed bisimilar to $Q$, written
$P \wbbisim_{\mathcal N} Q$, if $P \rel{S}_{\mathcal N} Q$ for some ${\mathcal N}$-barbed bisimulation ${\mathcal S}_{\mathcal N}$.
\end{definition}

$\mathcal{R} \subseteq \pi \times \pi$

$P \mathcal{R} Q => \forall P'. P \red P' \Rightarrow \exists Q'. Q \red Q', P' \mathcal{R} Q'$

$P \vdash x \Rightarrow Q \vdash x$

\begin{mathpar}
  \inferrule*[lab=Out-barb]{x \nameeq y}{{y}!\langle{Q}\rangle \vdash x}
  \and
  \inferrule*[lab=Par-barb]{\mbox{$P\vdash x$ or $Q\vdash x$}}{\binpar{P}{Q} \vdash x}
\end{mathpar}

\subsubsection{Contexts}

One of the principle advantages of computational calculi like the
$\pi$-calculus is a well-defined notion of context,
contextual-equivalence and a correlation between
contextual-equivalence and notions of bisimulation. The notion of
context allows the decomposition of a process into (sub-)process and
its syntactic environment, its context. Thus, a context may be
thought of as a process with a ``hole'' (written $\Box$) in it. The
application of a context $M$ to a process $P$, written $M[P]$, is
tantamount to filling the hole in $M$ with $P$. In this paper we do
not need the full weight of this theory, but do make use of the notion
of context in the proof the main theorem. 

\begin{mathpar}
  \inferrule* [lab=summation] {} {{M_{M},M_{N}} \bc \Box \;|\; x.M_{A} \;|\; M_{M}+M_{N}}
  \and
  \inferrule* [lab=agent] {} {{M_{A}} \bc (\vec{x})M_{P} \;| \; \clift{P_0,\ldots,M_{P},\ldots,P_N}}
  \and \\
  \inferrule* [lab=process] {} {{M_{P}} \bc M_{N} \;| \;P|M_{P} }
\end{mathpar} 

\begin{mathpar}
  \inferrule* [lab=sychronization] {} {M_{N} \bc \Box \;|\; x?M_{F} \;|\; x!M_{C}}
  \and
  \inferrule* [lab=abstraction] {} {{M_{F}} \bc (x)M_{P} }
  \and
  \inferrule* [lab=concretion] {} {{M_{C}} \bc \langle M_{P} \rangle }
  \and \\
  \inferrule* [lab=process] {} {{M_{P}} \bc M_{N} \;| \;P|M_{P} }
\end{mathpar}

\begin{definition}[contextual application] Given a context $M$, and
  process $P$, we define the \emph{contextual application}, $M[P] :=
  M\{P/\Box\}$. That is, the contextual application of M to P is the
  substitution of $P$ for $\Box$ in $M$.
\end{definition}

$\meaningof{-} : L \to \mathcal{P}(\pi)$

\begin{mathpar}
  \inferrule* [lab=collection] {} {\meaningof{true} = \pi, \and \meaningof{~E} = \pi \setminus \meaningof{E}, \and \meaningof{E_{1} \& E_{2}} = \meaningof{E_{1}} \cap \meaningof{E_{2}}}
\end{mathpar}

\begin{mathpar}
  \inferrule* [lab=structure] {} {\meaningof{0} = \{ P \in \pi | P \equiv 0 \}, \and \\ \meaningof{E_1 | E_2} = \{ P \in \pi | P \equiv P_{1} | P_{2}, P_{1} \in \meaningof{E_{1}}, P_{2} \in \meaningof{E_2}\} }
\end{mathpar}

\begin{mathpar}
 \inferrule* [lab=behavior] {} {\meaningof{\langle a?b \rangle E} = \{ P \in \pi | P \equiv Q | u?(y)P', \\ \and \\\\ \and \\ \;\;\; u \in \meaningof{a}, \forall z.P'\{z/y\} \in \meaningof{E\{z/b\}}\}, \and \\ \meaningof{a!E} = \{ P \in \pi | P \equiv Q | x!\langle P' \rangle, x \in \meaningof{a} P' \in \meaningof{E}\} }
\end{mathpar}

\begin{mathpar}
 \inferrule* [lab=nominal] {} {\meaningof{\quotep{E}} = \{ \quotep{P} \in \quotep{\pi} | P \in \meaningof{E} \}, \and \meaningof{\quotep{P}} = \{ \quotep{Q} \in \quotep{\pi} | P \equiv Q \} \and \\ \meaningof{@\quotep{E}} = \{ P \in \pi | P \equiv @x, x \in \meaningof{E} \}}
\end{mathpar}

\begin{eqnarray*}
  \\
  \meaningof{-} : TS \to ST
\end{eqnarray*}

\begin{eqnarray*}
  \\
  L : TS \to ST
\end{eqnarray*}

\begin{eqnarray*}
  \\
  P \models E \iff P \in \meaningof{E}
\end{eqnarray*}

\begin{eqnarray*}
  P \approx_{L} Q \iff \forall E \in L. P \models E \iff Q \models E
\end{eqnarray*}

\begin{eqnarray*}
  P \approx_{K} Q
\end{eqnarray*}

\begin{eqnarray*}
  P \approx Q
\end{eqnarray*}

$\approx_{K} = \approx = \approx_{L}$

\subsubsection{Contextual duality}

Note that contexts extend the quotation operation to a family of
operations from processes to names. Given a context, $M$, we can
define a \emph{nominal context}, $\quotep{M}$ by $\quotep{M}[P] :=
\quotep{M[P]}$. To foreshadow what is to come we observe that these
operations enjoy a duality with processes very much like the duality
between vectors and maps from vectors to scalars.

Further, because the calculus is essentially higher-order, we have a
correspondence between contexts and processes. More specifically,
given a name $x$ and a context $M$ we can construct $M^{*}_{x}$ such
that 

\begin{mathpar}
  M^{*}_{x} | \lift{x}{P} \red M[P]
\end{mathpar}

namely,

\begin{mathpar}
  M^{*}_{x} := x?(u).M[\dropn{u}]
\end{mathpar}

The dependence of $M^{*}_{x}$ on a name makes it an abstraction, 

\begin{mathpar}
  M^{*} := (x)x?(u).M[\dropn{u}]
\end{mathpar}

\subsection{Additional notation}

It will sometimes be convenient to denote the process a name
quotes. We already have the notation $x = \quotep{P}$, but it will be
convenient to introduce an alternate notation, $\procn{x}$, when we
want to emphasize the connection to the use of the name. Note that, by
virtue of name equivalence, $\quotep{\procn{x}} \nameeq x$; so, the
notation is consistent with previous definitions.

Further, because names have structure it is possible to effect
substitutions on the basis of that structure. This means we need to
upgrade our notation for substitutions, which we accomplish by
adapting comprehension notation. Thus,

\begin{mathpar}
  P\{ y / x : x \in S \}
\end{mathpar}

is interpreted to mean the process derived from P by replacing (in a
capture-avoiding manner) each occurrence of $x$ in $S$ by $y$. For example,

\begin{mathpar}
  P\{ \quotep{\procn{x}|\procn{x}} / x : x \in \freenames{P} \}
\end{mathpar}

will replace each (occurrence) of a free name $x$ in $P$ by
$\quotep{\procn{x}|\procn{x}}$.

Also, we will avail ourselves of the notation $x^{L}$ and $x^{R}$ to
denote injections of a name into disjoint copies of the name
space. There are numerous ways to accomplish this. One example can be
found in \cite{MeredithR05}. This notation overloads to vectors of
names: $\vec{x}^{\pi} := (x_{i}^{\pi} \; : \; 0 \leq i < |\vec{x}| )$ where $\pi \in \{L,R\}$.

We also use $P^{\Box} := P|\Box$.

In \cite{MeredithR05} an interpretation of the new operator is
given. It turns out that there are several possible interpretations
all enjoying the requisite algebraic properties of the operator (see
\cite{milner91polyadicpi}). We will therefore make liberal use of
$(\nu\; \vec{x})P$.

% subsection the_syntax_and_semantics_of_the_notation_system (end)   

\input{qm2pi.qmops} 

\input{qm2pi.sterngerlach} 

\input{qm2pi.metric} 

% section concurrent_process_calculi (end)

%\input{qm2pi.proofsketch}

% section proof sketch (end)

%\input{qm2pi.slviaknots} 

% section spatial logic via knots (end)

\input{qm2pi.conclusion}

% section conclusion (end)

%\input{qm2pi.dtcodes} 

% section wiring algorithm (end)

\input{qm2pi.ack} 

% section acknowledgments (end)

\newpage


\bibliographystyle{plain}   
\bibliography{../../biblios/main.bib}

\input{qm2pi.rhodetails}

\end{document}



\end{document}

 

% section wiring algorithm (end)

\documentclass[12pt]{llncs}
%\documentclass{jktr}

\usepackage[pdftex]{hyperref}                   
\usepackage {listings}
\usepackage {mathpartir}
\usepackage{bcprules}
%\usepackage{listings}
                       
\usepackage{graphicx} 
%\usepackage[margins=2.5cm,nohead,nofoot]{geometry}
%\usepackage{geometry}
\usepackage{amsfonts}
\usepackage{amstext}
\usepackage{latexsym}
\usepackage{amssymb}
\usepackage{color}


%\include{myPreamble}
\documentclass[12pt]{llncs}
%\documentclass{jktr}

\usepackage[pdftex]{hyperref}                   
\usepackage {listings}
\usepackage {mathpartir}
\usepackage{bcprules}
%\usepackage{listings}
                       
\usepackage{graphicx} 
%\usepackage[margins=2.5cm,nohead,nofoot]{geometry}
%\usepackage{geometry}
\usepackage{amsfonts}
\usepackage{amstext}
\usepackage{latexsym}
\usepackage{amssymb}
\usepackage{color}


%\include{myPreamble}
\include{qm2pi.local} 

%\ifpdf
%\usepackage[pdftex]{graphicx}
%\else
%\usepackage{graphicx}
%\fi

 % \ifpdf
%  \usepackage{pdfsync}
%  \if


%\title{Brief Article}
%\author{David F. Snyder}
%\author{L.G. Meredith}

%\address{Dept. of Math., Texas State University--San Marcos, San Marcos, TX 78666}
       
\pagestyle{empty}


\begin{document}

\lstset{language=[Objective]Caml,frame=shadowbox}

\input{qm2pi.front}

% section front matter (end)

\input{qm2pi.intro} 
 
% section introduction (end)

% \input{qm2pi.knotations} 

% section notation (end)

\input{qm2pi.process.calculi} 

% section concurrent_process_calculi_and_spatial_logics_ (end)
    
%\input{qm2pi.knots2pi} 

%\input{qm2pi.trefoil} 

%\input{qm2pi.mainthm} 

% subsection basic_interpretation (end)

%\input{qm2pi.rho.presentation} 
\subsection{The syntax and semantics of the notation system}\label{sub:the_syntax_and_semantics_of_the_notation_system} % (fold)

We now summarize a technical presentation of the calculus that
embodies our theory of dynamics. The typical presentation of such a
calculus follows the style of giving generators and relations on
them. The grammar, below, describing term constructors, freely
generates the set of processes, $\Proc$. This set is then quotiented
by a relation known as structural congruence and it is over this set
that the notion of dynamics is expressed. This presentation is
essentially that of \cite{MeredithR05} with the addition of
polyadicity and summation. For readability we have relegated some of
the technical subtleties to an appendix.

\subsubsection{Process grammar}\label{subsub:process_grammar}

\begin{mathpar}
  \inferrule* [lab=synchronization] {} {{M} \bc \pzero \;|\; x?F \;|\; x!C }
  \and
  \inferrule* [lab=abstraction] {} {{F} \bc (x)P}
  \and
  \inferrule* [lab=concretion] {} {{C} \bc \langle Q \rangle}
  \and
  \inferrule* [lab=process] {} {{P,Q} \bc M \;| \;P|Q \;|\; @{x}}
  \and
  \inferrule* [lab=name] {} {{x} \bc \quotep{P}}
\end{mathpar} 

Note that $\vec{x}$ (resp. $\vec{P}$) denotes a vector of names
(resp. processes) of length $|\vec{x}|$ (resp. $|\vec{P}|$). We adopt
the following useful abbreviations.

\begin{mathpar}
   x?(\vec{y}).P := x.(\vec{y})P \and  x\clift{\vec{P}} := x.\clift{\vec{P}}
   \and x!(y) := \lift{x}{\dropn{y}}
   \and \Pi_{i=0}^{n-1}P_i := P_0 | \ldots | P_{n-1}
\end{mathpar}

\subsubsection{Structural congruence}

\paragraph{Free and bound names and alpha-equivalence.} At the
core of structural equivalence is alpha-equivalence which identifies
process that are the same up to a change of variable. Formally, we
recognize the distinction between free and bound names. The free names
of a process, $\freenames{P}$, may be calculated recursively as
follows:

\begin{mathpar}
\freenames{\pzero} := \emptyset
  \and \\
  \freenames{x?(y).P} := \{ x \} \cup (\freenames{P} \setminus \{ y \})
  \and 
  \freenames{x!\langle P \rangle} := \{ x \} \cup \{ P \} 
  \and \\
  \freenames{P|Q} := \freenames{P} \cup \freenames{Q}
  \and \\
  \freenames{@{x}} := \{ x \}
\end{mathpar}

$\pi$
$\quotep{\pi}$

$\freenames{-} : \pi \to \mathcal{P}(\quotep{\pi})$

\begin{eqnarray*}
  \freenames{\pzero} & := & \emptyset \\
  \freenames{x?(y).P} & := & \{ x \} \cup (\freenames{P} \setminus \{ y \}) \\
  \freenames{x!\langle P \rangle} & := & \{ x \} \cup \{ P \} \\
  \freenames{P|Q} & := & \freenames{P} \cup \freenames{Q} \\
  \freenames{\dropn{x}} & := & \{ x \}
\end{eqnarray*}

The bound names of a process, $\boundnames{P}$, are those names occurring in $P$
that are not free. For example, in $x?(y).0$, the name $x$ is free, while $y$ is bound.

\begin{mathpar}
  \inferrule* [lab=monoidal-laws] {} { P|Q \equiv Q|P \and P|0 \equiv P \and P|(Q|R) \equiv (P|Q)|R }
\end{mathpar}

\begin{mathpar}
  \inferrule* [lab=alpha-equivalence] {} { (x)P \equiv (y)P\{y/x\} \and y \not\in \freenames{P} }
\end{mathpar}

\begin{definition}
Then two processes, $P,Q$, are alpha-equivalent if $P = Q\{\vec{y}/\vec{x}\}$ for
some $\vec{x} \in \boundnames{Q},\vec{y} \in \boundnames{P}$, where $Q\{\vec{y}/\vec{x}\}$
denotes the capture-avoiding substitution of $\vec{y}$ for $\vec{x}$ in $Q$.
\end{definition}

\begin{definition}
  The {\em structural congruence} \cite{SangiorgiWalker} , $\equiv$,
  between processes is the least congruence containing
  alpha-equivalence, satisfying the abelian monoid laws
  (associativity, commutativity and $\pzero$ as identity) for parallel
  composition $|$ and for summation $+$.
\end{definition}

\subsection{Name equivalence}

We take name equivalence, written $\nameeq$, to be the smallest
equivalence relation generated by the following rules.

\begin{mathpar}
\inferrule*[lab=Quote-drop]
{ }
{ \quotep{@{x}} \nameeq x }

\inferrule*[lab=Struct-equiv]
{ P \scong Q }
{ \quotep{P} \nameeq \quotep{Q} }
\end{mathpar}

The astute reader will have noticed that the mutual recursion of names
and processes imposes a mutual recursion on alpha-equivalence and
structural equivalence via name-equivalence. Fortunately, all of this
works out pleasantly and we may calculate in the natural way, free of
concern. The reader interested in the details is referred to the
appendix \ref{appendix:rho_details}.

\subsection{Substitution}

We use $\Proc$ for the set of processes, $\QProc$ for the set of
names, and $\id{\{}\vec{y} / \vec{x} \id{\}}$ to denote partial maps,
$s : \QProc \rightarrow \QProc$. A map, $s$ lifts, uniquely, to a map
on process terms, $\widehat{s} : \Proc \rightarrow \Proc$ by the
following equations.

\begin{mathpar}
  (0) \psubstp{Q}{P} := 0 \\
  (R \juxtap S) \psubstp{Q}{P}
  :=    
  (R)\psubstp{Q}{P} \juxtap (S) \psubstp{Q}{P} \\
  (x?(y).R) \psubstp{Q}{P}    
  :=    
  (x)\substp{Q}{P} (z)\concat( (R \psubstn{z}{y}) \psubstp{Q}{P} ) \\
  (\lift{x}{R}) \psubstp{Q}{P}  
  :=
  \lift{(x)\substp{Q}{P}}{ R \psubstp{Q}{P} } \\
%   (\dropn{x})  \psubstp{Q}{P}       
%   := 
%   \left\{ 
%     \begin{array}{ccc} 
%       \dropn{\quotep{Q}} & & x \nameeq \quotep{P} \\
%       \dropn{x} & & otherwise \\
%     \end{array}
%   \right. 
  (\dropn{x})  \psubstp{Q}{P}       
  := 
  \left\{ 
    \begin{array}{ccc} 
      Q & & x \nameeq \quotep{P} \\
      \dropn{x} & & otherwise \\
    \end{array}
  \right.
\end{mathpar}
 

where

\begin{eqnarray}
  (x)\id{\{} \lpquote Q \rpquote / \lpquote P \rpquote \id{\}}            = 
  \left\{ 
    \begin{array}{ccc}
      \lpquote Q \rpquote & & x \nameeq \lpquote P \rpquote \\
      x & & otherwise \\
    \end{array}
  \right. \nonumber
\end{eqnarray}

and $z$ is chosen distinct from $\quotep{P}$, $\quotep{Q}$, the free
names in $Q$, and all the names in $R$. Our $\alpha$-equivalence will
be built in the standard way from this substitution.

\begin{remark}\label{rem:no_self_referential_names}
  One consequence of these definitions is that $\forall P. \quotep{P}
  \not\in \freenames{P}$.
\end{remark}

\subsection{ Dynamic quote: an example }

Anticipating something of what's to come, consider applying the
substitution, $\widehat{\id{\{}u / z \id{\}}}$, to the following pair
of processes, $\lift{w}{y!(z)}$ and $w[ \lpquote y!(z) \rpquote ]$.

\begin{eqnarray}
	\lift{w}{y!(z)}\widehat{\id{\{}u / z \id{\}}}
		& = &
		\lift{w}{y!(u)} \nonumber\\
	w[ \lpquote y!(z) \rpquote ] \widehat{ \id{\{}u / z \id{\}} }
		& = &
		w[ \lpquote y!(z) \rpquote ] \nonumber
\end{eqnarray}

Because the body of the process between quotes is impervious to
substitution, we get radically different answers. In fact, by
examining the first process in an input context,
e.g. $x?(z).\lift{w}{y!(z)}$, we see that the process under the lift
operator may be shaped by prefixed inputs binding a name inside it. In
this sense, the lift operator will be seen as a way to dynamically
construct processes before reifying them as names.

Finally equipped with these standard features we can present the
dynamics of the calculus.

\subsubsection{Operational semantics} 

Finally, we introduce the computational dynamics. What marks these
algebras as distinct from other more traditionally studied algebraic
structures, e.g. vector spaces or polynomial rings, is the manner in
which dynamics is captured. In traditional structures, dynamics is typically
expressed through morphisms between such structures, as in linear maps
between vector spaces or morphisms between rings. In algebras
associated with the semantics of computation, the dynamics is
expressed as part of the algebraic structure itself, through a
reduction reduction relation typically denoted by $\red$. Below, we
give a recursive presentation of this relation for the calculus used
in the encoding.

$\red \subseteq \pi \times \pi$
$\red : \pi \to \mathcal{P}(\pi)$

\begin{mathpar}
  \inferrule* [lab=Comm] { \textsf{match}( x_{src}, x_{trgt} ) } { x_{trgt}?(y)P \; | \; x_{src}!\langle {Q} \rangle \red P\{\quotep{Q}/y}\} }
  \and \\
  \inferrule* [lab=Par] {{P} \red {P}'} {{{P} | {Q}} \red {{P}' | {Q}}}
  \and
  \inferrule* [lab=Equiv]{{{P} \scong {P}'} \andalso {{P}' \red {Q}'} \andalso {{Q}' \scong {Q}}}{{P} \red {Q}}
\end{mathpar}

\begin{eqnarray*}
  match_{\equiv} (\quotep{P},\quotep{Q}) & := & P \equiv Q \\
  match_{\dagger}(\quotep{P},\quotep{Q}) & := & \forall R. P|Q \red^{*} R => R \red^{*} 0 \\
  match_{K}(\quotep{P},\quotep{Q}) & := & K \mbox{ for some context } K
\end{eqnarray*}

$u?(x)P | u!\langle Q \rangle \red P\{\quotep{Q}/x\}$

%We write $\wred$ for $\red^*$, and $P\red$ if $\exists Q $ such that $ P \red Q$.
We write $P\red$ if $\exists Q $ such that $ P \red Q$ and $P\not\red$, otherwise.

\section{Replication}

As mentioned before, it is known that replication (and hence
recursion) can be implemented in a higher-order process algebra
\cite{SangiorgiWalker}. As our first example of calculation with the
machinery thus far presented we give the construction explicitly in
the {\rhoc}.

\begin{eqnarray}
	D_{x} & := & \prefix{x}{y}{(\binpar{\outputp{x}{y}}{@{y}})} \nonumber\\
	\bangp_{x}{P} & := & \binpar{{x}!\langle{\binpar{D_{x}}{P}}\rangle}{D_{x}} \nonumber
\end{eqnarray}

\begin{eqnarray}
	\bangp_{x}{P} & & \nonumber\\
	=
	& {x}!\langle{(\prefix{x}{y}{(\outputp{x}{y} | @{y})) | P}}\rangle 
	      | \prefix{x}{y}{(\outputp{x}{y} | @{y})} & \nonumber\\
	\red
	& (\outputp{x}{y} | @{y})\substn{\quotep{(\prefix{x}{y}{(@{y} | \outputp{x}{y})) | P}}}{y} & \nonumber\\
	=
	& \outputp{x}{\quotep{(\prefix{x}{y}{(\outputp{x}{y} | @{y})) | P}}}
	  | {(\prefix{x}{y}{(\outputp{x}{y} | @{y})) | P}} & \nonumber\\
	\red
	& \ldots & \nonumber\\
	\red^*
	& P | P | \ldots & \nonumber
\end{eqnarray}

Of course, this encoding, as an implementation, runs away, unfolding
$\bangp{P}$ eagerly. A lazier and more implementable replication
operator, restricted to input-guarded processes, may be obtained as follows.

\begin{eqnarray}
\bangp{\prefix{u}{v}{P}} 
	:= 
	\binpar{\lift{x}{\prefix{u}{v}{(\binpar{D(x)}{P})}}}{D(x)} \nonumber
\end{eqnarray}

\begin{remark}
  Note that the lazier definition still does not deal with summation
  or mixed summation (i.e. sums over input and output). The reader is
  invited to construct definitions of replication that deal with these
  features. 

  Further, the definitions are parameterized in a name, $x$. Can you,
  gentle reader, make a definition that eliminates this parameter and
  guarantees no accidental interaction between the replication
  machinery and the process being replicated -- i.e. no accidental
  sharing of names used by the process to get its work done and the
  name(s) used by the replication to effect copying. This latter
  revision of the definition of replication is crucial to obtaining
  the expected identity $!!P \sim !P$.
\end{remark}

\begin{remark}\label{rem:paradoxical_combinator}
  The reader familiar with the lambda calculus will have noticed the
  similarity between $D$ and the paradoxical combinator.

  [Ed. note: the existence of this seems to suggest we have to be more
  restrictive on the set of processes and names we admit if we are to
  support no-cloning.]
\end{remark}

\subsubsection{Bisimulation}

The computational dynamics gives rise to another kind of equivalence,
the equivalence of computational behavior. As previously mentioned
this is typically captured \emph{via} some form of bisimulation.

% The notion we use in this paper is weak barbed bisimulation
% \cite{milner91polyadicpi}.

The notion we use in this paper is derived from weak barbed
bisimulation \cite{milner91polyadicpi}. 

\begin{definition}
An \emph{observation relation}, $\downarrow_{\mathcal N}$, over a set
of names, $\mathcal N$, is the smallest relation satisfying the rules
below.

\infrule[Out-barb]{y \in {\mathcal N}, \; x \nameeq y}
		  {\outputp{x}{v} \downarrow_{\mathcal N} x}
\infrule[Par-barb]{\mbox{$P\downarrow_{\mathcal N} x$ or $Q\downarrow_{\mathcal N} x$}}
		  {\binpar{P}{Q} \downarrow_{\mathcal N} x}

We write $P \Downarrow_{\mathcal N} x$ if there is $Q$ such that 
$P \wred Q$ and $Q \downarrow_{\mathcal N} x$.
\end{definition}

\begin{definition}
%\label{def.bbisim}
An  ${\mathcal N}$-\emph{barbed bisimulation} over a set of names, ${\mathcal N}$, is a symmetric binary relation 
${\mathcal S}_{\mathcal N}$ between agents such that $P\rel{S}_{\mathcal N}Q$ implies:
\begin{enumerate}
\item If $P \red P'$ then $Q \wred Q'$ and $P'\rel{S}_{\mathcal N} Q'$.
\item If $P\downarrow_{\mathcal N} x$, then $Q\Downarrow_{\mathcal N} x$.
\end{enumerate}
$P$ is ${\mathcal N}$-barbed bisimilar to $Q$, written
$P \wbbisim_{\mathcal N} Q$, if $P \rel{S}_{\mathcal N} Q$ for some ${\mathcal N}$-barbed bisimulation ${\mathcal S}_{\mathcal N}$.
\end{definition}

$\mathcal{R} \subseteq \pi \times \pi$

$P \mathcal{R} Q => \forall P'. P \red P' \Rightarrow \exists Q'. Q \red Q', P' \mathcal{R} Q'$

$P \vdash x \Rightarrow Q \vdash x$

\begin{mathpar}
  \inferrule*[lab=Out-barb]{x \nameeq y}{{y}!\langle{Q}\rangle \vdash x}
  \and
  \inferrule*[lab=Par-barb]{\mbox{$P\vdash x$ or $Q\vdash x$}}{\binpar{P}{Q} \vdash x}
\end{mathpar}

\subsubsection{Contexts}

One of the principle advantages of computational calculi like the
$\pi$-calculus is a well-defined notion of context,
contextual-equivalence and a correlation between
contextual-equivalence and notions of bisimulation. The notion of
context allows the decomposition of a process into (sub-)process and
its syntactic environment, its context. Thus, a context may be
thought of as a process with a ``hole'' (written $\Box$) in it. The
application of a context $M$ to a process $P$, written $M[P]$, is
tantamount to filling the hole in $M$ with $P$. In this paper we do
not need the full weight of this theory, but do make use of the notion
of context in the proof the main theorem. 

\begin{mathpar}
  \inferrule* [lab=summation] {} {{M_{M},M_{N}} \bc \Box \;|\; x.M_{A} \;|\; M_{M}+M_{N}}
  \and
  \inferrule* [lab=agent] {} {{M_{A}} \bc (\vec{x})M_{P} \;| \; \clift{P_0,\ldots,M_{P},\ldots,P_N}}
  \and \\
  \inferrule* [lab=process] {} {{M_{P}} \bc M_{N} \;| \;P|M_{P} }
\end{mathpar} 

\begin{mathpar}
  \inferrule* [lab=sychronization] {} {M_{N} \bc \Box \;|\; x?M_{F} \;|\; x!M_{C}}
  \and
  \inferrule* [lab=abstraction] {} {{M_{F}} \bc (x)M_{P} }
  \and
  \inferrule* [lab=concretion] {} {{M_{C}} \bc \langle M_{P} \rangle }
  \and \\
  \inferrule* [lab=process] {} {{M_{P}} \bc M_{N} \;| \;P|M_{P} }
\end{mathpar}

\begin{definition}[contextual application] Given a context $M$, and
  process $P$, we define the \emph{contextual application}, $M[P] :=
  M\{P/\Box\}$. That is, the contextual application of M to P is the
  substitution of $P$ for $\Box$ in $M$.
\end{definition}

$\meaningof{-} : L \to \mathcal{P}(\pi)$

\begin{mathpar}
  \inferrule* [lab=collection] {} {\meaningof{true} = \pi, \and \meaningof{~E} = \pi \setminus \meaningof{E}, \and \meaningof{E_{1} \& E_{2}} = \meaningof{E_{1}} \cap \meaningof{E_{2}}}
\end{mathpar}

\begin{mathpar}
  \inferrule* [lab=structure] {} {\meaningof{0} = \{ P \in \pi | P \equiv 0 \}, \and \\ \meaningof{E_1 | E_2} = \{ P \in \pi | P \equiv P_{1} | P_{2}, P_{1} \in \meaningof{E_{1}}, P_{2} \in \meaningof{E_2}\} }
\end{mathpar}

\begin{mathpar}
 \inferrule* [lab=behavior] {} {\meaningof{\langle a?b \rangle E} = \{ P \in \pi | P \equiv Q | u?(y)P', \\ \and \\\\ \and \\ \;\;\; u \in \meaningof{a}, \forall z.P'\{z/y\} \in \meaningof{E\{z/b\}}\}, \and \\ \meaningof{a!E} = \{ P \in \pi | P \equiv Q | x!\langle P' \rangle, x \in \meaningof{a} P' \in \meaningof{E}\} }
\end{mathpar}

\begin{mathpar}
 \inferrule* [lab=nominal] {} {\meaningof{\quotep{E}} = \{ \quotep{P} \in \quotep{\pi} | P \in \meaningof{E} \}, \and \meaningof{\quotep{P}} = \{ \quotep{Q} \in \quotep{\pi} | P \equiv Q \} \and \\ \meaningof{@\quotep{E}} = \{ P \in \pi | P \equiv @x, x \in \meaningof{E} \}}
\end{mathpar}

\begin{eqnarray*}
  \\
  \meaningof{-} : TS \to ST
\end{eqnarray*}

\begin{eqnarray*}
  \\
  L : TS \to ST
\end{eqnarray*}

\begin{eqnarray*}
  \\
  P \models E \iff P \in \meaningof{E}
\end{eqnarray*}

\begin{eqnarray*}
  P \approx_{L} Q \iff \forall E \in L. P \models E \iff Q \models E
\end{eqnarray*}

\begin{eqnarray*}
  P \approx_{K} Q
\end{eqnarray*}

\begin{eqnarray*}
  P \approx Q
\end{eqnarray*}

$\approx_{K} = \approx = \approx_{L}$

\subsubsection{Contextual duality}

Note that contexts extend the quotation operation to a family of
operations from processes to names. Given a context, $M$, we can
define a \emph{nominal context}, $\quotep{M}$ by $\quotep{M}[P] :=
\quotep{M[P]}$. To foreshadow what is to come we observe that these
operations enjoy a duality with processes very much like the duality
between vectors and maps from vectors to scalars.

Further, because the calculus is essentially higher-order, we have a
correspondence between contexts and processes. More specifically,
given a name $x$ and a context $M$ we can construct $M^{*}_{x}$ such
that 

\begin{mathpar}
  M^{*}_{x} | \lift{x}{P} \red M[P]
\end{mathpar}

namely,

\begin{mathpar}
  M^{*}_{x} := x?(u).M[\dropn{u}]
\end{mathpar}

The dependence of $M^{*}_{x}$ on a name makes it an abstraction, 

\begin{mathpar}
  M^{*} := (x)x?(u).M[\dropn{u}]
\end{mathpar}

\subsection{Additional notation}

It will sometimes be convenient to denote the process a name
quotes. We already have the notation $x = \quotep{P}$, but it will be
convenient to introduce an alternate notation, $\procn{x}$, when we
want to emphasize the connection to the use of the name. Note that, by
virtue of name equivalence, $\quotep{\procn{x}} \nameeq x$; so, the
notation is consistent with previous definitions.

Further, because names have structure it is possible to effect
substitutions on the basis of that structure. This means we need to
upgrade our notation for substitutions, which we accomplish by
adapting comprehension notation. Thus,

\begin{mathpar}
  P\{ y / x : x \in S \}
\end{mathpar}

is interpreted to mean the process derived from P by replacing (in a
capture-avoiding manner) each occurrence of $x$ in $S$ by $y$. For example,

\begin{mathpar}
  P\{ \quotep{\procn{x}|\procn{x}} / x : x \in \freenames{P} \}
\end{mathpar}

will replace each (occurrence) of a free name $x$ in $P$ by
$\quotep{\procn{x}|\procn{x}}$.

Also, we will avail ourselves of the notation $x^{L}$ and $x^{R}$ to
denote injections of a name into disjoint copies of the name
space. There are numerous ways to accomplish this. One example can be
found in \cite{MeredithR05}. This notation overloads to vectors of
names: $\vec{x}^{\pi} := (x_{i}^{\pi} \; : \; 0 \leq i < |\vec{x}| )$ where $\pi \in \{L,R\}$.

We also use $P^{\Box} := P|\Box$.

In \cite{MeredithR05} an interpretation of the new operator is
given. It turns out that there are several possible interpretations
all enjoying the requisite algebraic properties of the operator (see
\cite{milner91polyadicpi}). We will therefore make liberal use of
$(\nu\; \vec{x})P$.

% subsection the_syntax_and_semantics_of_the_notation_system (end)   

\input{qm2pi.qmops} 

\input{qm2pi.sterngerlach} 

\input{qm2pi.metric} 

% section concurrent_process_calculi (end)

%\input{qm2pi.proofsketch}

% section proof sketch (end)

%\input{qm2pi.slviaknots} 

% section spatial logic via knots (end)

\input{qm2pi.conclusion}

% section conclusion (end)

%\input{qm2pi.dtcodes} 

% section wiring algorithm (end)

\input{qm2pi.ack} 

% section acknowledgments (end)

\newpage


\bibliographystyle{plain}   
\bibliography{../../biblios/main.bib}

\input{qm2pi.rhodetails}

\end{document}

 

%\ifpdf
%\usepackage[pdftex]{graphicx}
%\else
%\usepackage{graphicx}
%\fi

 % \ifpdf
%  \usepackage{pdfsync}
%  \if


%\title{Brief Article}
%\author{David F. Snyder}
%\author{L.G. Meredith}

%\address{Dept. of Math., Texas State University--San Marcos, San Marcos, TX 78666}
       
\pagestyle{empty}


\begin{document}

\lstset{language=[Objective]Caml,frame=shadowbox}

\documentclass[12pt]{llncs}
%\documentclass{jktr}

\usepackage[pdftex]{hyperref}                   
\usepackage {listings}
\usepackage {mathpartir}
\usepackage{bcprules}
%\usepackage{listings}
                       
\usepackage{graphicx} 
%\usepackage[margins=2.5cm,nohead,nofoot]{geometry}
%\usepackage{geometry}
\usepackage{amsfonts}
\usepackage{amstext}
\usepackage{latexsym}
\usepackage{amssymb}
\usepackage{color}


%\include{myPreamble}
\include{qm2pi.local} 

%\ifpdf
%\usepackage[pdftex]{graphicx}
%\else
%\usepackage{graphicx}
%\fi

 % \ifpdf
%  \usepackage{pdfsync}
%  \if


%\title{Brief Article}
%\author{David F. Snyder}
%\author{L.G. Meredith}

%\address{Dept. of Math., Texas State University--San Marcos, San Marcos, TX 78666}
       
\pagestyle{empty}


\begin{document}

\lstset{language=[Objective]Caml,frame=shadowbox}

\input{qm2pi.front}

% section front matter (end)

\input{qm2pi.intro} 
 
% section introduction (end)

% \input{qm2pi.knotations} 

% section notation (end)

\input{qm2pi.process.calculi} 

% section concurrent_process_calculi_and_spatial_logics_ (end)
    
%\input{qm2pi.knots2pi} 

%\input{qm2pi.trefoil} 

%\input{qm2pi.mainthm} 

% subsection basic_interpretation (end)

%\input{qm2pi.rho.presentation} 
\subsection{The syntax and semantics of the notation system}\label{sub:the_syntax_and_semantics_of_the_notation_system} % (fold)

We now summarize a technical presentation of the calculus that
embodies our theory of dynamics. The typical presentation of such a
calculus follows the style of giving generators and relations on
them. The grammar, below, describing term constructors, freely
generates the set of processes, $\Proc$. This set is then quotiented
by a relation known as structural congruence and it is over this set
that the notion of dynamics is expressed. This presentation is
essentially that of \cite{MeredithR05} with the addition of
polyadicity and summation. For readability we have relegated some of
the technical subtleties to an appendix.

\subsubsection{Process grammar}\label{subsub:process_grammar}

\begin{mathpar}
  \inferrule* [lab=synchronization] {} {{M} \bc \pzero \;|\; x?F \;|\; x!C }
  \and
  \inferrule* [lab=abstraction] {} {{F} \bc (x)P}
  \and
  \inferrule* [lab=concretion] {} {{C} \bc \langle Q \rangle}
  \and
  \inferrule* [lab=process] {} {{P,Q} \bc M \;| \;P|Q \;|\; @{x}}
  \and
  \inferrule* [lab=name] {} {{x} \bc \quotep{P}}
\end{mathpar} 

Note that $\vec{x}$ (resp. $\vec{P}$) denotes a vector of names
(resp. processes) of length $|\vec{x}|$ (resp. $|\vec{P}|$). We adopt
the following useful abbreviations.

\begin{mathpar}
   x?(\vec{y}).P := x.(\vec{y})P \and  x\clift{\vec{P}} := x.\clift{\vec{P}}
   \and x!(y) := \lift{x}{\dropn{y}}
   \and \Pi_{i=0}^{n-1}P_i := P_0 | \ldots | P_{n-1}
\end{mathpar}

\subsubsection{Structural congruence}

\paragraph{Free and bound names and alpha-equivalence.} At the
core of structural equivalence is alpha-equivalence which identifies
process that are the same up to a change of variable. Formally, we
recognize the distinction between free and bound names. The free names
of a process, $\freenames{P}$, may be calculated recursively as
follows:

\begin{mathpar}
\freenames{\pzero} := \emptyset
  \and \\
  \freenames{x?(y).P} := \{ x \} \cup (\freenames{P} \setminus \{ y \})
  \and 
  \freenames{x!\langle P \rangle} := \{ x \} \cup \{ P \} 
  \and \\
  \freenames{P|Q} := \freenames{P} \cup \freenames{Q}
  \and \\
  \freenames{@{x}} := \{ x \}
\end{mathpar}

$\pi$
$\quotep{\pi}$

$\freenames{-} : \pi \to \mathcal{P}(\quotep{\pi})$

\begin{eqnarray*}
  \freenames{\pzero} & := & \emptyset \\
  \freenames{x?(y).P} & := & \{ x \} \cup (\freenames{P} \setminus \{ y \}) \\
  \freenames{x!\langle P \rangle} & := & \{ x \} \cup \{ P \} \\
  \freenames{P|Q} & := & \freenames{P} \cup \freenames{Q} \\
  \freenames{\dropn{x}} & := & \{ x \}
\end{eqnarray*}

The bound names of a process, $\boundnames{P}$, are those names occurring in $P$
that are not free. For example, in $x?(y).0$, the name $x$ is free, while $y$ is bound.

\begin{mathpar}
  \inferrule* [lab=monoidal-laws] {} { P|Q \equiv Q|P \and P|0 \equiv P \and P|(Q|R) \equiv (P|Q)|R }
\end{mathpar}

\begin{mathpar}
  \inferrule* [lab=alpha-equivalence] {} { (x)P \equiv (y)P\{y/x\} \and y \not\in \freenames{P} }
\end{mathpar}

\begin{definition}
Then two processes, $P,Q$, are alpha-equivalent if $P = Q\{\vec{y}/\vec{x}\}$ for
some $\vec{x} \in \boundnames{Q},\vec{y} \in \boundnames{P}$, where $Q\{\vec{y}/\vec{x}\}$
denotes the capture-avoiding substitution of $\vec{y}$ for $\vec{x}$ in $Q$.
\end{definition}

\begin{definition}
  The {\em structural congruence} \cite{SangiorgiWalker} , $\equiv$,
  between processes is the least congruence containing
  alpha-equivalence, satisfying the abelian monoid laws
  (associativity, commutativity and $\pzero$ as identity) for parallel
  composition $|$ and for summation $+$.
\end{definition}

\subsection{Name equivalence}

We take name equivalence, written $\nameeq$, to be the smallest
equivalence relation generated by the following rules.

\begin{mathpar}
\inferrule*[lab=Quote-drop]
{ }
{ \quotep{@{x}} \nameeq x }

\inferrule*[lab=Struct-equiv]
{ P \scong Q }
{ \quotep{P} \nameeq \quotep{Q} }
\end{mathpar}

The astute reader will have noticed that the mutual recursion of names
and processes imposes a mutual recursion on alpha-equivalence and
structural equivalence via name-equivalence. Fortunately, all of this
works out pleasantly and we may calculate in the natural way, free of
concern. The reader interested in the details is referred to the
appendix \ref{appendix:rho_details}.

\subsection{Substitution}

We use $\Proc$ for the set of processes, $\QProc$ for the set of
names, and $\id{\{}\vec{y} / \vec{x} \id{\}}$ to denote partial maps,
$s : \QProc \rightarrow \QProc$. A map, $s$ lifts, uniquely, to a map
on process terms, $\widehat{s} : \Proc \rightarrow \Proc$ by the
following equations.

\begin{mathpar}
  (0) \psubstp{Q}{P} := 0 \\
  (R \juxtap S) \psubstp{Q}{P}
  :=    
  (R)\psubstp{Q}{P} \juxtap (S) \psubstp{Q}{P} \\
  (x?(y).R) \psubstp{Q}{P}    
  :=    
  (x)\substp{Q}{P} (z)\concat( (R \psubstn{z}{y}) \psubstp{Q}{P} ) \\
  (\lift{x}{R}) \psubstp{Q}{P}  
  :=
  \lift{(x)\substp{Q}{P}}{ R \psubstp{Q}{P} } \\
%   (\dropn{x})  \psubstp{Q}{P}       
%   := 
%   \left\{ 
%     \begin{array}{ccc} 
%       \dropn{\quotep{Q}} & & x \nameeq \quotep{P} \\
%       \dropn{x} & & otherwise \\
%     \end{array}
%   \right. 
  (\dropn{x})  \psubstp{Q}{P}       
  := 
  \left\{ 
    \begin{array}{ccc} 
      Q & & x \nameeq \quotep{P} \\
      \dropn{x} & & otherwise \\
    \end{array}
  \right.
\end{mathpar}
 

where

\begin{eqnarray}
  (x)\id{\{} \lpquote Q \rpquote / \lpquote P \rpquote \id{\}}            = 
  \left\{ 
    \begin{array}{ccc}
      \lpquote Q \rpquote & & x \nameeq \lpquote P \rpquote \\
      x & & otherwise \\
    \end{array}
  \right. \nonumber
\end{eqnarray}

and $z$ is chosen distinct from $\quotep{P}$, $\quotep{Q}$, the free
names in $Q$, and all the names in $R$. Our $\alpha$-equivalence will
be built in the standard way from this substitution.

\begin{remark}\label{rem:no_self_referential_names}
  One consequence of these definitions is that $\forall P. \quotep{P}
  \not\in \freenames{P}$.
\end{remark}

\subsection{ Dynamic quote: an example }

Anticipating something of what's to come, consider applying the
substitution, $\widehat{\id{\{}u / z \id{\}}}$, to the following pair
of processes, $\lift{w}{y!(z)}$ and $w[ \lpquote y!(z) \rpquote ]$.

\begin{eqnarray}
	\lift{w}{y!(z)}\widehat{\id{\{}u / z \id{\}}}
		& = &
		\lift{w}{y!(u)} \nonumber\\
	w[ \lpquote y!(z) \rpquote ] \widehat{ \id{\{}u / z \id{\}} }
		& = &
		w[ \lpquote y!(z) \rpquote ] \nonumber
\end{eqnarray}

Because the body of the process between quotes is impervious to
substitution, we get radically different answers. In fact, by
examining the first process in an input context,
e.g. $x?(z).\lift{w}{y!(z)}$, we see that the process under the lift
operator may be shaped by prefixed inputs binding a name inside it. In
this sense, the lift operator will be seen as a way to dynamically
construct processes before reifying them as names.

Finally equipped with these standard features we can present the
dynamics of the calculus.

\subsubsection{Operational semantics} 

Finally, we introduce the computational dynamics. What marks these
algebras as distinct from other more traditionally studied algebraic
structures, e.g. vector spaces or polynomial rings, is the manner in
which dynamics is captured. In traditional structures, dynamics is typically
expressed through morphisms between such structures, as in linear maps
between vector spaces or morphisms between rings. In algebras
associated with the semantics of computation, the dynamics is
expressed as part of the algebraic structure itself, through a
reduction reduction relation typically denoted by $\red$. Below, we
give a recursive presentation of this relation for the calculus used
in the encoding.

$\red \subseteq \pi \times \pi$
$\red : \pi \to \mathcal{P}(\pi)$

\begin{mathpar}
  \inferrule* [lab=Comm] { \textsf{match}( x_{src}, x_{trgt} ) } { x_{trgt}?(y)P \; | \; x_{src}!\langle {Q} \rangle \red P\{\quotep{Q}/y}\} }
  \and \\
  \inferrule* [lab=Par] {{P} \red {P}'} {{{P} | {Q}} \red {{P}' | {Q}}}
  \and
  \inferrule* [lab=Equiv]{{{P} \scong {P}'} \andalso {{P}' \red {Q}'} \andalso {{Q}' \scong {Q}}}{{P} \red {Q}}
\end{mathpar}

\begin{eqnarray*}
  match_{\equiv} (\quotep{P},\quotep{Q}) & := & P \equiv Q \\
  match_{\dagger}(\quotep{P},\quotep{Q}) & := & \forall R. P|Q \red^{*} R => R \red^{*} 0 \\
  match_{K}(\quotep{P},\quotep{Q}) & := & K \mbox{ for some context } K
\end{eqnarray*}

$u?(x)P | u!\langle Q \rangle \red P\{\quotep{Q}/x\}$

%We write $\wred$ for $\red^*$, and $P\red$ if $\exists Q $ such that $ P \red Q$.
We write $P\red$ if $\exists Q $ such that $ P \red Q$ and $P\not\red$, otherwise.

\section{Replication}

As mentioned before, it is known that replication (and hence
recursion) can be implemented in a higher-order process algebra
\cite{SangiorgiWalker}. As our first example of calculation with the
machinery thus far presented we give the construction explicitly in
the {\rhoc}.

\begin{eqnarray}
	D_{x} & := & \prefix{x}{y}{(\binpar{\outputp{x}{y}}{@{y}})} \nonumber\\
	\bangp_{x}{P} & := & \binpar{{x}!\langle{\binpar{D_{x}}{P}}\rangle}{D_{x}} \nonumber
\end{eqnarray}

\begin{eqnarray}
	\bangp_{x}{P} & & \nonumber\\
	=
	& {x}!\langle{(\prefix{x}{y}{(\outputp{x}{y} | @{y})) | P}}\rangle 
	      | \prefix{x}{y}{(\outputp{x}{y} | @{y})} & \nonumber\\
	\red
	& (\outputp{x}{y} | @{y})\substn{\quotep{(\prefix{x}{y}{(@{y} | \outputp{x}{y})) | P}}}{y} & \nonumber\\
	=
	& \outputp{x}{\quotep{(\prefix{x}{y}{(\outputp{x}{y} | @{y})) | P}}}
	  | {(\prefix{x}{y}{(\outputp{x}{y} | @{y})) | P}} & \nonumber\\
	\red
	& \ldots & \nonumber\\
	\red^*
	& P | P | \ldots & \nonumber
\end{eqnarray}

Of course, this encoding, as an implementation, runs away, unfolding
$\bangp{P}$ eagerly. A lazier and more implementable replication
operator, restricted to input-guarded processes, may be obtained as follows.

\begin{eqnarray}
\bangp{\prefix{u}{v}{P}} 
	:= 
	\binpar{\lift{x}{\prefix{u}{v}{(\binpar{D(x)}{P})}}}{D(x)} \nonumber
\end{eqnarray}

\begin{remark}
  Note that the lazier definition still does not deal with summation
  or mixed summation (i.e. sums over input and output). The reader is
  invited to construct definitions of replication that deal with these
  features. 

  Further, the definitions are parameterized in a name, $x$. Can you,
  gentle reader, make a definition that eliminates this parameter and
  guarantees no accidental interaction between the replication
  machinery and the process being replicated -- i.e. no accidental
  sharing of names used by the process to get its work done and the
  name(s) used by the replication to effect copying. This latter
  revision of the definition of replication is crucial to obtaining
  the expected identity $!!P \sim !P$.
\end{remark}

\begin{remark}\label{rem:paradoxical_combinator}
  The reader familiar with the lambda calculus will have noticed the
  similarity between $D$ and the paradoxical combinator.

  [Ed. note: the existence of this seems to suggest we have to be more
  restrictive on the set of processes and names we admit if we are to
  support no-cloning.]
\end{remark}

\subsubsection{Bisimulation}

The computational dynamics gives rise to another kind of equivalence,
the equivalence of computational behavior. As previously mentioned
this is typically captured \emph{via} some form of bisimulation.

% The notion we use in this paper is weak barbed bisimulation
% \cite{milner91polyadicpi}.

The notion we use in this paper is derived from weak barbed
bisimulation \cite{milner91polyadicpi}. 

\begin{definition}
An \emph{observation relation}, $\downarrow_{\mathcal N}$, over a set
of names, $\mathcal N$, is the smallest relation satisfying the rules
below.

\infrule[Out-barb]{y \in {\mathcal N}, \; x \nameeq y}
		  {\outputp{x}{v} \downarrow_{\mathcal N} x}
\infrule[Par-barb]{\mbox{$P\downarrow_{\mathcal N} x$ or $Q\downarrow_{\mathcal N} x$}}
		  {\binpar{P}{Q} \downarrow_{\mathcal N} x}

We write $P \Downarrow_{\mathcal N} x$ if there is $Q$ such that 
$P \wred Q$ and $Q \downarrow_{\mathcal N} x$.
\end{definition}

\begin{definition}
%\label{def.bbisim}
An  ${\mathcal N}$-\emph{barbed bisimulation} over a set of names, ${\mathcal N}$, is a symmetric binary relation 
${\mathcal S}_{\mathcal N}$ between agents such that $P\rel{S}_{\mathcal N}Q$ implies:
\begin{enumerate}
\item If $P \red P'$ then $Q \wred Q'$ and $P'\rel{S}_{\mathcal N} Q'$.
\item If $P\downarrow_{\mathcal N} x$, then $Q\Downarrow_{\mathcal N} x$.
\end{enumerate}
$P$ is ${\mathcal N}$-barbed bisimilar to $Q$, written
$P \wbbisim_{\mathcal N} Q$, if $P \rel{S}_{\mathcal N} Q$ for some ${\mathcal N}$-barbed bisimulation ${\mathcal S}_{\mathcal N}$.
\end{definition}

$\mathcal{R} \subseteq \pi \times \pi$

$P \mathcal{R} Q => \forall P'. P \red P' \Rightarrow \exists Q'. Q \red Q', P' \mathcal{R} Q'$

$P \vdash x \Rightarrow Q \vdash x$

\begin{mathpar}
  \inferrule*[lab=Out-barb]{x \nameeq y}{{y}!\langle{Q}\rangle \vdash x}
  \and
  \inferrule*[lab=Par-barb]{\mbox{$P\vdash x$ or $Q\vdash x$}}{\binpar{P}{Q} \vdash x}
\end{mathpar}

\subsubsection{Contexts}

One of the principle advantages of computational calculi like the
$\pi$-calculus is a well-defined notion of context,
contextual-equivalence and a correlation between
contextual-equivalence and notions of bisimulation. The notion of
context allows the decomposition of a process into (sub-)process and
its syntactic environment, its context. Thus, a context may be
thought of as a process with a ``hole'' (written $\Box$) in it. The
application of a context $M$ to a process $P$, written $M[P]$, is
tantamount to filling the hole in $M$ with $P$. In this paper we do
not need the full weight of this theory, but do make use of the notion
of context in the proof the main theorem. 

\begin{mathpar}
  \inferrule* [lab=summation] {} {{M_{M},M_{N}} \bc \Box \;|\; x.M_{A} \;|\; M_{M}+M_{N}}
  \and
  \inferrule* [lab=agent] {} {{M_{A}} \bc (\vec{x})M_{P} \;| \; \clift{P_0,\ldots,M_{P},\ldots,P_N}}
  \and \\
  \inferrule* [lab=process] {} {{M_{P}} \bc M_{N} \;| \;P|M_{P} }
\end{mathpar} 

\begin{mathpar}
  \inferrule* [lab=sychronization] {} {M_{N} \bc \Box \;|\; x?M_{F} \;|\; x!M_{C}}
  \and
  \inferrule* [lab=abstraction] {} {{M_{F}} \bc (x)M_{P} }
  \and
  \inferrule* [lab=concretion] {} {{M_{C}} \bc \langle M_{P} \rangle }
  \and \\
  \inferrule* [lab=process] {} {{M_{P}} \bc M_{N} \;| \;P|M_{P} }
\end{mathpar}

\begin{definition}[contextual application] Given a context $M$, and
  process $P$, we define the \emph{contextual application}, $M[P] :=
  M\{P/\Box\}$. That is, the contextual application of M to P is the
  substitution of $P$ for $\Box$ in $M$.
\end{definition}

$\meaningof{-} : L \to \mathcal{P}(\pi)$

\begin{mathpar}
  \inferrule* [lab=collection] {} {\meaningof{true} = \pi, \and \meaningof{~E} = \pi \setminus \meaningof{E}, \and \meaningof{E_{1} \& E_{2}} = \meaningof{E_{1}} \cap \meaningof{E_{2}}}
\end{mathpar}

\begin{mathpar}
  \inferrule* [lab=structure] {} {\meaningof{0} = \{ P \in \pi | P \equiv 0 \}, \and \\ \meaningof{E_1 | E_2} = \{ P \in \pi | P \equiv P_{1} | P_{2}, P_{1} \in \meaningof{E_{1}}, P_{2} \in \meaningof{E_2}\} }
\end{mathpar}

\begin{mathpar}
 \inferrule* [lab=behavior] {} {\meaningof{\langle a?b \rangle E} = \{ P \in \pi | P \equiv Q | u?(y)P', \\ \and \\\\ \and \\ \;\;\; u \in \meaningof{a}, \forall z.P'\{z/y\} \in \meaningof{E\{z/b\}}\}, \and \\ \meaningof{a!E} = \{ P \in \pi | P \equiv Q | x!\langle P' \rangle, x \in \meaningof{a} P' \in \meaningof{E}\} }
\end{mathpar}

\begin{mathpar}
 \inferrule* [lab=nominal] {} {\meaningof{\quotep{E}} = \{ \quotep{P} \in \quotep{\pi} | P \in \meaningof{E} \}, \and \meaningof{\quotep{P}} = \{ \quotep{Q} \in \quotep{\pi} | P \equiv Q \} \and \\ \meaningof{@\quotep{E}} = \{ P \in \pi | P \equiv @x, x \in \meaningof{E} \}}
\end{mathpar}

\begin{eqnarray*}
  \\
  \meaningof{-} : TS \to ST
\end{eqnarray*}

\begin{eqnarray*}
  \\
  L : TS \to ST
\end{eqnarray*}

\begin{eqnarray*}
  \\
  P \models E \iff P \in \meaningof{E}
\end{eqnarray*}

\begin{eqnarray*}
  P \approx_{L} Q \iff \forall E \in L. P \models E \iff Q \models E
\end{eqnarray*}

\begin{eqnarray*}
  P \approx_{K} Q
\end{eqnarray*}

\begin{eqnarray*}
  P \approx Q
\end{eqnarray*}

$\approx_{K} = \approx = \approx_{L}$

\subsubsection{Contextual duality}

Note that contexts extend the quotation operation to a family of
operations from processes to names. Given a context, $M$, we can
define a \emph{nominal context}, $\quotep{M}$ by $\quotep{M}[P] :=
\quotep{M[P]}$. To foreshadow what is to come we observe that these
operations enjoy a duality with processes very much like the duality
between vectors and maps from vectors to scalars.

Further, because the calculus is essentially higher-order, we have a
correspondence between contexts and processes. More specifically,
given a name $x$ and a context $M$ we can construct $M^{*}_{x}$ such
that 

\begin{mathpar}
  M^{*}_{x} | \lift{x}{P} \red M[P]
\end{mathpar}

namely,

\begin{mathpar}
  M^{*}_{x} := x?(u).M[\dropn{u}]
\end{mathpar}

The dependence of $M^{*}_{x}$ on a name makes it an abstraction, 

\begin{mathpar}
  M^{*} := (x)x?(u).M[\dropn{u}]
\end{mathpar}

\subsection{Additional notation}

It will sometimes be convenient to denote the process a name
quotes. We already have the notation $x = \quotep{P}$, but it will be
convenient to introduce an alternate notation, $\procn{x}$, when we
want to emphasize the connection to the use of the name. Note that, by
virtue of name equivalence, $\quotep{\procn{x}} \nameeq x$; so, the
notation is consistent with previous definitions.

Further, because names have structure it is possible to effect
substitutions on the basis of that structure. This means we need to
upgrade our notation for substitutions, which we accomplish by
adapting comprehension notation. Thus,

\begin{mathpar}
  P\{ y / x : x \in S \}
\end{mathpar}

is interpreted to mean the process derived from P by replacing (in a
capture-avoiding manner) each occurrence of $x$ in $S$ by $y$. For example,

\begin{mathpar}
  P\{ \quotep{\procn{x}|\procn{x}} / x : x \in \freenames{P} \}
\end{mathpar}

will replace each (occurrence) of a free name $x$ in $P$ by
$\quotep{\procn{x}|\procn{x}}$.

Also, we will avail ourselves of the notation $x^{L}$ and $x^{R}$ to
denote injections of a name into disjoint copies of the name
space. There are numerous ways to accomplish this. One example can be
found in \cite{MeredithR05}. This notation overloads to vectors of
names: $\vec{x}^{\pi} := (x_{i}^{\pi} \; : \; 0 \leq i < |\vec{x}| )$ where $\pi \in \{L,R\}$.

We also use $P^{\Box} := P|\Box$.

In \cite{MeredithR05} an interpretation of the new operator is
given. It turns out that there are several possible interpretations
all enjoying the requisite algebraic properties of the operator (see
\cite{milner91polyadicpi}). We will therefore make liberal use of
$(\nu\; \vec{x})P$.

% subsection the_syntax_and_semantics_of_the_notation_system (end)   

\input{qm2pi.qmops} 

\input{qm2pi.sterngerlach} 

\input{qm2pi.metric} 

% section concurrent_process_calculi (end)

%\input{qm2pi.proofsketch}

% section proof sketch (end)

%\input{qm2pi.slviaknots} 

% section spatial logic via knots (end)

\input{qm2pi.conclusion}

% section conclusion (end)

%\input{qm2pi.dtcodes} 

% section wiring algorithm (end)

\input{qm2pi.ack} 

% section acknowledgments (end)

\newpage


\bibliographystyle{plain}   
\bibliography{../../biblios/main.bib}

\input{qm2pi.rhodetails}

\end{document}



% section front matter (end)

\section{Introduction}\label{sec:introduction} % (fold)
In this draft of the material i am going to have to dispense with the
usual writing conventions adopted in papers on these topics. i'm going
to have adopt whatever tone i need at the time i'm writing up the
calculations. Sometimes this may be very conversational; others it may
be the barest mathematical grunts; others still it may be that i have
lifted text from one of my other papers because the exposition of some
point was better said there. i hope that my readers are not unduly put
out by this decision. i'm not doing this to flout convention or be
rebellious. i find these calculations very technically challenging. To
keep everything going technically, something has to give; i have to
let go of some cognitive burden. So, the academic writing style --
with all of its trade-offs in terms of facilitating technical
communication -- is what i'm letting go of. Perhaps subsequent drafts
can be tightened and polished, but for now, i'm going to speak as if
we were sitting together in a coffee shop with a laptop, wifi and a
pad of paper and a pencil.

So, here's what i have to say. We -- you and i, comfortably ensconced
in our coffee shop and well-equipped with our tools -- can realize and
carry out the calculations of quantum mechanics over a very different
formal theory of dynamics, a formal theory of dynamics that
corresponds to a theory of concurrent computation with
\emph{reflection}. It has the advantage that the underlying theory is
already `quantized', but supports analogues all of the continuuous
operations. Strikingly, this underlying theory has recently been
connected with a notion of metric that we can show, by calculating
together, coincides with the metric induced by the inner product.

There are a lot of reasons why you might be interested in seeing
calculations of this form. Here's why i'm interested. For the past
several centuries there has been no competitor to the ``Newtonian''
account of dynamics. As a result the predominant share of accounts of
dynamical systems and situations have had to be formulated in terms of
the Newtonian machinery. i view this as an intellectually dangerous
position to occupy. Everything, despite it's intrinsic shape, turns
into a nail to be hit with this hammer. Recently, however, the theory
of computation has matured to the point where we have candidates for
theories of dynamics that offer very different perspective on
reasoning about dynamical systems and situations. Testing these
candidates against very successful accounts of dynamical situations,
like quantum mechanics, is going to give us some sense of how mature
they are and some measure of the quality of these accounts of
dynamics.

\subsection{Summary of contributions and outline of paper}

So, we're going to develop an interpretation of the operations of
quantum mechanics normally interpreted by Hilbert spaces and
operators. We're going to do this over a theory of computation. Note
that this is very different than the usual quantum computation program
which develops notions of computation over quantum mechanics. Rather,
we are developing a story that aligns with Wheeler's slogan: It from
Bit. To do this we will first provide an account of the theory of
computation at play here. Then we will dive into a calculation-driven
interpretation of the operations of quantum mechanics.

The reason we take this approach is that -- until very recently --
there hasn't been an axiomatic account of quantum mechanics. As a
result there has been no sharp delineation of the mathematical theory
supporting interpretation of the physical theory and the physical
theory, itself. So, ambient features of the maths are free to be
exploited (or supressed) without a real accounting of their physical
relevance. There is no sharp statement ``here's the physical theory''
qua \emph{theory} and ``here's the mathematical interpretation''
enabling a judgment of how faithful the interpretation is -- apart
from experimental observation. When there is an axiomatic account we
can judge how well a given mathematical formalism supports an
interpretation of the axioms, independent of
experimentation. Likewise, we can judge how well we have captured our
physical evidence and experience with our axiomatics, independent of
any specific mathematical implementation, with accidental detail that
may or may not have physical significance. 

In lieu of a fully fleshed out and vetted axiomatic account of quantum
mechanics, interpreting the operational notions in service of modeling
physical systems will have to suffice. In other words, we are not in
the business of providing a model of Hilbert spaces and operators. We
are in the business of providing a model of quantum mechanics because
we are motivated by testing our notions of dynamics against physical
theory; and, the predictive calculations of the physical theory must
serve as the best formulation -- shy of a fully fleshed out axiomatic
account -- of the physical theory itself (as they have for scientific
theories since time immemorial). Put another way, despite a
whole-hearted commitment to an It-from-Bit ontology, we are firmly
aligned with the shut-up-and-calculate camp as the best way to obtain
results either from the physical perspective or as a quality assurance
measure of our fledgling theory of dynamics.

In detail, we present a reflective process calculus. Then we develop
intuitive correspondences between the notions available in this
calculus and the usual physical notions supporting quantum mechanical
calculations. Thus, 

\begin{table}[htp]
  \center{
    \fbox{
      \begin{tabular}{c|c}
        quantum mechanics & process calculus \\
        \hline
        scalar & name \\
        state vector & process \\
        dual & contextual duals \\
        matrix & formal sums of process-context-dual pairs \\
        orthogonality & process annihilation \\
        inner product & execution-formula + quoting
      \end{tabular}
    }
  }
  \caption{QM - process calculi correspondences}
\end{table}

Then we tighten up these intuitions to operational definitions. We
employ the Dirac notation as the best proxy we can find for an
abstract syntax of the quantum mechanical notions. The definitions we
develop put us in contact with equational constraints coming from the
theory that we demonstrate the definitions and calculations satisfy.

This puts us in a position to shut up and calculate for the
Stern-Gerlach experimental set up, showing how these predictive
calculations become calculations on processes in our theory of a
reflective process calculus.

Penultimately, we demonstrate that the notion of metric coming from
the inner product coincides with the notion of metric available from
the theory of bisimulation. This demonstration gives us the right to
think of space as arising from behavior. Finally, we consider where we
might go from the new vantage point we have obtained.

% section introduction (end) 
 
% section introduction (end)

% \documentclass[12pt]{llncs}
%\documentclass{jktr}

\usepackage[pdftex]{hyperref}                   
\usepackage {listings}
\usepackage {mathpartir}
\usepackage{bcprules}
%\usepackage{listings}
                       
\usepackage{graphicx} 
%\usepackage[margins=2.5cm,nohead,nofoot]{geometry}
%\usepackage{geometry}
\usepackage{amsfonts}
\usepackage{amstext}
\usepackage{latexsym}
\usepackage{amssymb}
\usepackage{color}


%\include{myPreamble}
\include{qm2pi.local} 

%\ifpdf
%\usepackage[pdftex]{graphicx}
%\else
%\usepackage{graphicx}
%\fi

 % \ifpdf
%  \usepackage{pdfsync}
%  \if


%\title{Brief Article}
%\author{David F. Snyder}
%\author{L.G. Meredith}

%\address{Dept. of Math., Texas State University--San Marcos, San Marcos, TX 78666}
       
\pagestyle{empty}


\begin{document}

\lstset{language=[Objective]Caml,frame=shadowbox}

\input{qm2pi.front}

% section front matter (end)

\input{qm2pi.intro} 
 
% section introduction (end)

% \input{qm2pi.knotations} 

% section notation (end)

\input{qm2pi.process.calculi} 

% section concurrent_process_calculi_and_spatial_logics_ (end)
    
%\input{qm2pi.knots2pi} 

%\input{qm2pi.trefoil} 

%\input{qm2pi.mainthm} 

% subsection basic_interpretation (end)

%\input{qm2pi.rho.presentation} 
\subsection{The syntax and semantics of the notation system}\label{sub:the_syntax_and_semantics_of_the_notation_system} % (fold)

We now summarize a technical presentation of the calculus that
embodies our theory of dynamics. The typical presentation of such a
calculus follows the style of giving generators and relations on
them. The grammar, below, describing term constructors, freely
generates the set of processes, $\Proc$. This set is then quotiented
by a relation known as structural congruence and it is over this set
that the notion of dynamics is expressed. This presentation is
essentially that of \cite{MeredithR05} with the addition of
polyadicity and summation. For readability we have relegated some of
the technical subtleties to an appendix.

\subsubsection{Process grammar}\label{subsub:process_grammar}

\begin{mathpar}
  \inferrule* [lab=synchronization] {} {{M} \bc \pzero \;|\; x?F \;|\; x!C }
  \and
  \inferrule* [lab=abstraction] {} {{F} \bc (x)P}
  \and
  \inferrule* [lab=concretion] {} {{C} \bc \langle Q \rangle}
  \and
  \inferrule* [lab=process] {} {{P,Q} \bc M \;| \;P|Q \;|\; @{x}}
  \and
  \inferrule* [lab=name] {} {{x} \bc \quotep{P}}
\end{mathpar} 

Note that $\vec{x}$ (resp. $\vec{P}$) denotes a vector of names
(resp. processes) of length $|\vec{x}|$ (resp. $|\vec{P}|$). We adopt
the following useful abbreviations.

\begin{mathpar}
   x?(\vec{y}).P := x.(\vec{y})P \and  x\clift{\vec{P}} := x.\clift{\vec{P}}
   \and x!(y) := \lift{x}{\dropn{y}}
   \and \Pi_{i=0}^{n-1}P_i := P_0 | \ldots | P_{n-1}
\end{mathpar}

\subsubsection{Structural congruence}

\paragraph{Free and bound names and alpha-equivalence.} At the
core of structural equivalence is alpha-equivalence which identifies
process that are the same up to a change of variable. Formally, we
recognize the distinction between free and bound names. The free names
of a process, $\freenames{P}$, may be calculated recursively as
follows:

\begin{mathpar}
\freenames{\pzero} := \emptyset
  \and \\
  \freenames{x?(y).P} := \{ x \} \cup (\freenames{P} \setminus \{ y \})
  \and 
  \freenames{x!\langle P \rangle} := \{ x \} \cup \{ P \} 
  \and \\
  \freenames{P|Q} := \freenames{P} \cup \freenames{Q}
  \and \\
  \freenames{@{x}} := \{ x \}
\end{mathpar}

$\pi$
$\quotep{\pi}$

$\freenames{-} : \pi \to \mathcal{P}(\quotep{\pi})$

\begin{eqnarray*}
  \freenames{\pzero} & := & \emptyset \\
  \freenames{x?(y).P} & := & \{ x \} \cup (\freenames{P} \setminus \{ y \}) \\
  \freenames{x!\langle P \rangle} & := & \{ x \} \cup \{ P \} \\
  \freenames{P|Q} & := & \freenames{P} \cup \freenames{Q} \\
  \freenames{\dropn{x}} & := & \{ x \}
\end{eqnarray*}

The bound names of a process, $\boundnames{P}$, are those names occurring in $P$
that are not free. For example, in $x?(y).0$, the name $x$ is free, while $y$ is bound.

\begin{mathpar}
  \inferrule* [lab=monoidal-laws] {} { P|Q \equiv Q|P \and P|0 \equiv P \and P|(Q|R) \equiv (P|Q)|R }
\end{mathpar}

\begin{mathpar}
  \inferrule* [lab=alpha-equivalence] {} { (x)P \equiv (y)P\{y/x\} \and y \not\in \freenames{P} }
\end{mathpar}

\begin{definition}
Then two processes, $P,Q$, are alpha-equivalent if $P = Q\{\vec{y}/\vec{x}\}$ for
some $\vec{x} \in \boundnames{Q},\vec{y} \in \boundnames{P}$, where $Q\{\vec{y}/\vec{x}\}$
denotes the capture-avoiding substitution of $\vec{y}$ for $\vec{x}$ in $Q$.
\end{definition}

\begin{definition}
  The {\em structural congruence} \cite{SangiorgiWalker} , $\equiv$,
  between processes is the least congruence containing
  alpha-equivalence, satisfying the abelian monoid laws
  (associativity, commutativity and $\pzero$ as identity) for parallel
  composition $|$ and for summation $+$.
\end{definition}

\subsection{Name equivalence}

We take name equivalence, written $\nameeq$, to be the smallest
equivalence relation generated by the following rules.

\begin{mathpar}
\inferrule*[lab=Quote-drop]
{ }
{ \quotep{@{x}} \nameeq x }

\inferrule*[lab=Struct-equiv]
{ P \scong Q }
{ \quotep{P} \nameeq \quotep{Q} }
\end{mathpar}

The astute reader will have noticed that the mutual recursion of names
and processes imposes a mutual recursion on alpha-equivalence and
structural equivalence via name-equivalence. Fortunately, all of this
works out pleasantly and we may calculate in the natural way, free of
concern. The reader interested in the details is referred to the
appendix \ref{appendix:rho_details}.

\subsection{Substitution}

We use $\Proc$ for the set of processes, $\QProc$ for the set of
names, and $\id{\{}\vec{y} / \vec{x} \id{\}}$ to denote partial maps,
$s : \QProc \rightarrow \QProc$. A map, $s$ lifts, uniquely, to a map
on process terms, $\widehat{s} : \Proc \rightarrow \Proc$ by the
following equations.

\begin{mathpar}
  (0) \psubstp{Q}{P} := 0 \\
  (R \juxtap S) \psubstp{Q}{P}
  :=    
  (R)\psubstp{Q}{P} \juxtap (S) \psubstp{Q}{P} \\
  (x?(y).R) \psubstp{Q}{P}    
  :=    
  (x)\substp{Q}{P} (z)\concat( (R \psubstn{z}{y}) \psubstp{Q}{P} ) \\
  (\lift{x}{R}) \psubstp{Q}{P}  
  :=
  \lift{(x)\substp{Q}{P}}{ R \psubstp{Q}{P} } \\
%   (\dropn{x})  \psubstp{Q}{P}       
%   := 
%   \left\{ 
%     \begin{array}{ccc} 
%       \dropn{\quotep{Q}} & & x \nameeq \quotep{P} \\
%       \dropn{x} & & otherwise \\
%     \end{array}
%   \right. 
  (\dropn{x})  \psubstp{Q}{P}       
  := 
  \left\{ 
    \begin{array}{ccc} 
      Q & & x \nameeq \quotep{P} \\
      \dropn{x} & & otherwise \\
    \end{array}
  \right.
\end{mathpar}
 

where

\begin{eqnarray}
  (x)\id{\{} \lpquote Q \rpquote / \lpquote P \rpquote \id{\}}            = 
  \left\{ 
    \begin{array}{ccc}
      \lpquote Q \rpquote & & x \nameeq \lpquote P \rpquote \\
      x & & otherwise \\
    \end{array}
  \right. \nonumber
\end{eqnarray}

and $z$ is chosen distinct from $\quotep{P}$, $\quotep{Q}$, the free
names in $Q$, and all the names in $R$. Our $\alpha$-equivalence will
be built in the standard way from this substitution.

\begin{remark}\label{rem:no_self_referential_names}
  One consequence of these definitions is that $\forall P. \quotep{P}
  \not\in \freenames{P}$.
\end{remark}

\subsection{ Dynamic quote: an example }

Anticipating something of what's to come, consider applying the
substitution, $\widehat{\id{\{}u / z \id{\}}}$, to the following pair
of processes, $\lift{w}{y!(z)}$ and $w[ \lpquote y!(z) \rpquote ]$.

\begin{eqnarray}
	\lift{w}{y!(z)}\widehat{\id{\{}u / z \id{\}}}
		& = &
		\lift{w}{y!(u)} \nonumber\\
	w[ \lpquote y!(z) \rpquote ] \widehat{ \id{\{}u / z \id{\}} }
		& = &
		w[ \lpquote y!(z) \rpquote ] \nonumber
\end{eqnarray}

Because the body of the process between quotes is impervious to
substitution, we get radically different answers. In fact, by
examining the first process in an input context,
e.g. $x?(z).\lift{w}{y!(z)}$, we see that the process under the lift
operator may be shaped by prefixed inputs binding a name inside it. In
this sense, the lift operator will be seen as a way to dynamically
construct processes before reifying them as names.

Finally equipped with these standard features we can present the
dynamics of the calculus.

\subsubsection{Operational semantics} 

Finally, we introduce the computational dynamics. What marks these
algebras as distinct from other more traditionally studied algebraic
structures, e.g. vector spaces or polynomial rings, is the manner in
which dynamics is captured. In traditional structures, dynamics is typically
expressed through morphisms between such structures, as in linear maps
between vector spaces or morphisms between rings. In algebras
associated with the semantics of computation, the dynamics is
expressed as part of the algebraic structure itself, through a
reduction reduction relation typically denoted by $\red$. Below, we
give a recursive presentation of this relation for the calculus used
in the encoding.

$\red \subseteq \pi \times \pi$
$\red : \pi \to \mathcal{P}(\pi)$

\begin{mathpar}
  \inferrule* [lab=Comm] { \textsf{match}( x_{src}, x_{trgt} ) } { x_{trgt}?(y)P \; | \; x_{src}!\langle {Q} \rangle \red P\{\quotep{Q}/y}\} }
  \and \\
  \inferrule* [lab=Par] {{P} \red {P}'} {{{P} | {Q}} \red {{P}' | {Q}}}
  \and
  \inferrule* [lab=Equiv]{{{P} \scong {P}'} \andalso {{P}' \red {Q}'} \andalso {{Q}' \scong {Q}}}{{P} \red {Q}}
\end{mathpar}

\begin{eqnarray*}
  match_{\equiv} (\quotep{P},\quotep{Q}) & := & P \equiv Q \\
  match_{\dagger}(\quotep{P},\quotep{Q}) & := & \forall R. P|Q \red^{*} R => R \red^{*} 0 \\
  match_{K}(\quotep{P},\quotep{Q}) & := & K \mbox{ for some context } K
\end{eqnarray*}

$u?(x)P | u!\langle Q \rangle \red P\{\quotep{Q}/x\}$

%We write $\wred$ for $\red^*$, and $P\red$ if $\exists Q $ such that $ P \red Q$.
We write $P\red$ if $\exists Q $ such that $ P \red Q$ and $P\not\red$, otherwise.

\section{Replication}

As mentioned before, it is known that replication (and hence
recursion) can be implemented in a higher-order process algebra
\cite{SangiorgiWalker}. As our first example of calculation with the
machinery thus far presented we give the construction explicitly in
the {\rhoc}.

\begin{eqnarray}
	D_{x} & := & \prefix{x}{y}{(\binpar{\outputp{x}{y}}{@{y}})} \nonumber\\
	\bangp_{x}{P} & := & \binpar{{x}!\langle{\binpar{D_{x}}{P}}\rangle}{D_{x}} \nonumber
\end{eqnarray}

\begin{eqnarray}
	\bangp_{x}{P} & & \nonumber\\
	=
	& {x}!\langle{(\prefix{x}{y}{(\outputp{x}{y} | @{y})) | P}}\rangle 
	      | \prefix{x}{y}{(\outputp{x}{y} | @{y})} & \nonumber\\
	\red
	& (\outputp{x}{y} | @{y})\substn{\quotep{(\prefix{x}{y}{(@{y} | \outputp{x}{y})) | P}}}{y} & \nonumber\\
	=
	& \outputp{x}{\quotep{(\prefix{x}{y}{(\outputp{x}{y} | @{y})) | P}}}
	  | {(\prefix{x}{y}{(\outputp{x}{y} | @{y})) | P}} & \nonumber\\
	\red
	& \ldots & \nonumber\\
	\red^*
	& P | P | \ldots & \nonumber
\end{eqnarray}

Of course, this encoding, as an implementation, runs away, unfolding
$\bangp{P}$ eagerly. A lazier and more implementable replication
operator, restricted to input-guarded processes, may be obtained as follows.

\begin{eqnarray}
\bangp{\prefix{u}{v}{P}} 
	:= 
	\binpar{\lift{x}{\prefix{u}{v}{(\binpar{D(x)}{P})}}}{D(x)} \nonumber
\end{eqnarray}

\begin{remark}
  Note that the lazier definition still does not deal with summation
  or mixed summation (i.e. sums over input and output). The reader is
  invited to construct definitions of replication that deal with these
  features. 

  Further, the definitions are parameterized in a name, $x$. Can you,
  gentle reader, make a definition that eliminates this parameter and
  guarantees no accidental interaction between the replication
  machinery and the process being replicated -- i.e. no accidental
  sharing of names used by the process to get its work done and the
  name(s) used by the replication to effect copying. This latter
  revision of the definition of replication is crucial to obtaining
  the expected identity $!!P \sim !P$.
\end{remark}

\begin{remark}\label{rem:paradoxical_combinator}
  The reader familiar with the lambda calculus will have noticed the
  similarity between $D$ and the paradoxical combinator.

  [Ed. note: the existence of this seems to suggest we have to be more
  restrictive on the set of processes and names we admit if we are to
  support no-cloning.]
\end{remark}

\subsubsection{Bisimulation}

The computational dynamics gives rise to another kind of equivalence,
the equivalence of computational behavior. As previously mentioned
this is typically captured \emph{via} some form of bisimulation.

% The notion we use in this paper is weak barbed bisimulation
% \cite{milner91polyadicpi}.

The notion we use in this paper is derived from weak barbed
bisimulation \cite{milner91polyadicpi}. 

\begin{definition}
An \emph{observation relation}, $\downarrow_{\mathcal N}$, over a set
of names, $\mathcal N$, is the smallest relation satisfying the rules
below.

\infrule[Out-barb]{y \in {\mathcal N}, \; x \nameeq y}
		  {\outputp{x}{v} \downarrow_{\mathcal N} x}
\infrule[Par-barb]{\mbox{$P\downarrow_{\mathcal N} x$ or $Q\downarrow_{\mathcal N} x$}}
		  {\binpar{P}{Q} \downarrow_{\mathcal N} x}

We write $P \Downarrow_{\mathcal N} x$ if there is $Q$ such that 
$P \wred Q$ and $Q \downarrow_{\mathcal N} x$.
\end{definition}

\begin{definition}
%\label{def.bbisim}
An  ${\mathcal N}$-\emph{barbed bisimulation} over a set of names, ${\mathcal N}$, is a symmetric binary relation 
${\mathcal S}_{\mathcal N}$ between agents such that $P\rel{S}_{\mathcal N}Q$ implies:
\begin{enumerate}
\item If $P \red P'$ then $Q \wred Q'$ and $P'\rel{S}_{\mathcal N} Q'$.
\item If $P\downarrow_{\mathcal N} x$, then $Q\Downarrow_{\mathcal N} x$.
\end{enumerate}
$P$ is ${\mathcal N}$-barbed bisimilar to $Q$, written
$P \wbbisim_{\mathcal N} Q$, if $P \rel{S}_{\mathcal N} Q$ for some ${\mathcal N}$-barbed bisimulation ${\mathcal S}_{\mathcal N}$.
\end{definition}

$\mathcal{R} \subseteq \pi \times \pi$

$P \mathcal{R} Q => \forall P'. P \red P' \Rightarrow \exists Q'. Q \red Q', P' \mathcal{R} Q'$

$P \vdash x \Rightarrow Q \vdash x$

\begin{mathpar}
  \inferrule*[lab=Out-barb]{x \nameeq y}{{y}!\langle{Q}\rangle \vdash x}
  \and
  \inferrule*[lab=Par-barb]{\mbox{$P\vdash x$ or $Q\vdash x$}}{\binpar{P}{Q} \vdash x}
\end{mathpar}

\subsubsection{Contexts}

One of the principle advantages of computational calculi like the
$\pi$-calculus is a well-defined notion of context,
contextual-equivalence and a correlation between
contextual-equivalence and notions of bisimulation. The notion of
context allows the decomposition of a process into (sub-)process and
its syntactic environment, its context. Thus, a context may be
thought of as a process with a ``hole'' (written $\Box$) in it. The
application of a context $M$ to a process $P$, written $M[P]$, is
tantamount to filling the hole in $M$ with $P$. In this paper we do
not need the full weight of this theory, but do make use of the notion
of context in the proof the main theorem. 

\begin{mathpar}
  \inferrule* [lab=summation] {} {{M_{M},M_{N}} \bc \Box \;|\; x.M_{A} \;|\; M_{M}+M_{N}}
  \and
  \inferrule* [lab=agent] {} {{M_{A}} \bc (\vec{x})M_{P} \;| \; \clift{P_0,\ldots,M_{P},\ldots,P_N}}
  \and \\
  \inferrule* [lab=process] {} {{M_{P}} \bc M_{N} \;| \;P|M_{P} }
\end{mathpar} 

\begin{mathpar}
  \inferrule* [lab=sychronization] {} {M_{N} \bc \Box \;|\; x?M_{F} \;|\; x!M_{C}}
  \and
  \inferrule* [lab=abstraction] {} {{M_{F}} \bc (x)M_{P} }
  \and
  \inferrule* [lab=concretion] {} {{M_{C}} \bc \langle M_{P} \rangle }
  \and \\
  \inferrule* [lab=process] {} {{M_{P}} \bc M_{N} \;| \;P|M_{P} }
\end{mathpar}

\begin{definition}[contextual application] Given a context $M$, and
  process $P$, we define the \emph{contextual application}, $M[P] :=
  M\{P/\Box\}$. That is, the contextual application of M to P is the
  substitution of $P$ for $\Box$ in $M$.
\end{definition}

$\meaningof{-} : L \to \mathcal{P}(\pi)$

\begin{mathpar}
  \inferrule* [lab=collection] {} {\meaningof{true} = \pi, \and \meaningof{~E} = \pi \setminus \meaningof{E}, \and \meaningof{E_{1} \& E_{2}} = \meaningof{E_{1}} \cap \meaningof{E_{2}}}
\end{mathpar}

\begin{mathpar}
  \inferrule* [lab=structure] {} {\meaningof{0} = \{ P \in \pi | P \equiv 0 \}, \and \\ \meaningof{E_1 | E_2} = \{ P \in \pi | P \equiv P_{1} | P_{2}, P_{1} \in \meaningof{E_{1}}, P_{2} \in \meaningof{E_2}\} }
\end{mathpar}

\begin{mathpar}
 \inferrule* [lab=behavior] {} {\meaningof{\langle a?b \rangle E} = \{ P \in \pi | P \equiv Q | u?(y)P', \\ \and \\\\ \and \\ \;\;\; u \in \meaningof{a}, \forall z.P'\{z/y\} \in \meaningof{E\{z/b\}}\}, \and \\ \meaningof{a!E} = \{ P \in \pi | P \equiv Q | x!\langle P' \rangle, x \in \meaningof{a} P' \in \meaningof{E}\} }
\end{mathpar}

\begin{mathpar}
 \inferrule* [lab=nominal] {} {\meaningof{\quotep{E}} = \{ \quotep{P} \in \quotep{\pi} | P \in \meaningof{E} \}, \and \meaningof{\quotep{P}} = \{ \quotep{Q} \in \quotep{\pi} | P \equiv Q \} \and \\ \meaningof{@\quotep{E}} = \{ P \in \pi | P \equiv @x, x \in \meaningof{E} \}}
\end{mathpar}

\begin{eqnarray*}
  \\
  \meaningof{-} : TS \to ST
\end{eqnarray*}

\begin{eqnarray*}
  \\
  L : TS \to ST
\end{eqnarray*}

\begin{eqnarray*}
  \\
  P \models E \iff P \in \meaningof{E}
\end{eqnarray*}

\begin{eqnarray*}
  P \approx_{L} Q \iff \forall E \in L. P \models E \iff Q \models E
\end{eqnarray*}

\begin{eqnarray*}
  P \approx_{K} Q
\end{eqnarray*}

\begin{eqnarray*}
  P \approx Q
\end{eqnarray*}

$\approx_{K} = \approx = \approx_{L}$

\subsubsection{Contextual duality}

Note that contexts extend the quotation operation to a family of
operations from processes to names. Given a context, $M$, we can
define a \emph{nominal context}, $\quotep{M}$ by $\quotep{M}[P] :=
\quotep{M[P]}$. To foreshadow what is to come we observe that these
operations enjoy a duality with processes very much like the duality
between vectors and maps from vectors to scalars.

Further, because the calculus is essentially higher-order, we have a
correspondence between contexts and processes. More specifically,
given a name $x$ and a context $M$ we can construct $M^{*}_{x}$ such
that 

\begin{mathpar}
  M^{*}_{x} | \lift{x}{P} \red M[P]
\end{mathpar}

namely,

\begin{mathpar}
  M^{*}_{x} := x?(u).M[\dropn{u}]
\end{mathpar}

The dependence of $M^{*}_{x}$ on a name makes it an abstraction, 

\begin{mathpar}
  M^{*} := (x)x?(u).M[\dropn{u}]
\end{mathpar}

\subsection{Additional notation}

It will sometimes be convenient to denote the process a name
quotes. We already have the notation $x = \quotep{P}$, but it will be
convenient to introduce an alternate notation, $\procn{x}$, when we
want to emphasize the connection to the use of the name. Note that, by
virtue of name equivalence, $\quotep{\procn{x}} \nameeq x$; so, the
notation is consistent with previous definitions.

Further, because names have structure it is possible to effect
substitutions on the basis of that structure. This means we need to
upgrade our notation for substitutions, which we accomplish by
adapting comprehension notation. Thus,

\begin{mathpar}
  P\{ y / x : x \in S \}
\end{mathpar}

is interpreted to mean the process derived from P by replacing (in a
capture-avoiding manner) each occurrence of $x$ in $S$ by $y$. For example,

\begin{mathpar}
  P\{ \quotep{\procn{x}|\procn{x}} / x : x \in \freenames{P} \}
\end{mathpar}

will replace each (occurrence) of a free name $x$ in $P$ by
$\quotep{\procn{x}|\procn{x}}$.

Also, we will avail ourselves of the notation $x^{L}$ and $x^{R}$ to
denote injections of a name into disjoint copies of the name
space. There are numerous ways to accomplish this. One example can be
found in \cite{MeredithR05}. This notation overloads to vectors of
names: $\vec{x}^{\pi} := (x_{i}^{\pi} \; : \; 0 \leq i < |\vec{x}| )$ where $\pi \in \{L,R\}$.

We also use $P^{\Box} := P|\Box$.

In \cite{MeredithR05} an interpretation of the new operator is
given. It turns out that there are several possible interpretations
all enjoying the requisite algebraic properties of the operator (see
\cite{milner91polyadicpi}). We will therefore make liberal use of
$(\nu\; \vec{x})P$.

% subsection the_syntax_and_semantics_of_the_notation_system (end)   

\input{qm2pi.qmops} 

\input{qm2pi.sterngerlach} 

\input{qm2pi.metric} 

% section concurrent_process_calculi (end)

%\input{qm2pi.proofsketch}

% section proof sketch (end)

%\input{qm2pi.slviaknots} 

% section spatial logic via knots (end)

\input{qm2pi.conclusion}

% section conclusion (end)

%\input{qm2pi.dtcodes} 

% section wiring algorithm (end)

\input{qm2pi.ack} 

% section acknowledgments (end)

\newpage


\bibliographystyle{plain}   
\bibliography{../../biblios/main.bib}

\input{qm2pi.rhodetails}

\end{document}

 

% section notation (end)

\input{qm2pi.process.calculi} 

% section concurrent_process_calculi_and_spatial_logics_ (end)
    
%\documentclass[12pt]{llncs}
%\documentclass{jktr}

\usepackage[pdftex]{hyperref}                   
\usepackage {listings}
\usepackage {mathpartir}
\usepackage{bcprules}
%\usepackage{listings}
                       
\usepackage{graphicx} 
%\usepackage[margins=2.5cm,nohead,nofoot]{geometry}
%\usepackage{geometry}
\usepackage{amsfonts}
\usepackage{amstext}
\usepackage{latexsym}
\usepackage{amssymb}
\usepackage{color}


%\include{myPreamble}
\include{qm2pi.local} 

%\ifpdf
%\usepackage[pdftex]{graphicx}
%\else
%\usepackage{graphicx}
%\fi

 % \ifpdf
%  \usepackage{pdfsync}
%  \if


%\title{Brief Article}
%\author{David F. Snyder}
%\author{L.G. Meredith}

%\address{Dept. of Math., Texas State University--San Marcos, San Marcos, TX 78666}
       
\pagestyle{empty}


\begin{document}

\lstset{language=[Objective]Caml,frame=shadowbox}

\input{qm2pi.front}

% section front matter (end)

\input{qm2pi.intro} 
 
% section introduction (end)

% \input{qm2pi.knotations} 

% section notation (end)

\input{qm2pi.process.calculi} 

% section concurrent_process_calculi_and_spatial_logics_ (end)
    
%\input{qm2pi.knots2pi} 

%\input{qm2pi.trefoil} 

%\input{qm2pi.mainthm} 

% subsection basic_interpretation (end)

%\input{qm2pi.rho.presentation} 
\subsection{The syntax and semantics of the notation system}\label{sub:the_syntax_and_semantics_of_the_notation_system} % (fold)

We now summarize a technical presentation of the calculus that
embodies our theory of dynamics. The typical presentation of such a
calculus follows the style of giving generators and relations on
them. The grammar, below, describing term constructors, freely
generates the set of processes, $\Proc$. This set is then quotiented
by a relation known as structural congruence and it is over this set
that the notion of dynamics is expressed. This presentation is
essentially that of \cite{MeredithR05} with the addition of
polyadicity and summation. For readability we have relegated some of
the technical subtleties to an appendix.

\subsubsection{Process grammar}\label{subsub:process_grammar}

\begin{mathpar}
  \inferrule* [lab=synchronization] {} {{M} \bc \pzero \;|\; x?F \;|\; x!C }
  \and
  \inferrule* [lab=abstraction] {} {{F} \bc (x)P}
  \and
  \inferrule* [lab=concretion] {} {{C} \bc \langle Q \rangle}
  \and
  \inferrule* [lab=process] {} {{P,Q} \bc M \;| \;P|Q \;|\; @{x}}
  \and
  \inferrule* [lab=name] {} {{x} \bc \quotep{P}}
\end{mathpar} 

Note that $\vec{x}$ (resp. $\vec{P}$) denotes a vector of names
(resp. processes) of length $|\vec{x}|$ (resp. $|\vec{P}|$). We adopt
the following useful abbreviations.

\begin{mathpar}
   x?(\vec{y}).P := x.(\vec{y})P \and  x\clift{\vec{P}} := x.\clift{\vec{P}}
   \and x!(y) := \lift{x}{\dropn{y}}
   \and \Pi_{i=0}^{n-1}P_i := P_0 | \ldots | P_{n-1}
\end{mathpar}

\subsubsection{Structural congruence}

\paragraph{Free and bound names and alpha-equivalence.} At the
core of structural equivalence is alpha-equivalence which identifies
process that are the same up to a change of variable. Formally, we
recognize the distinction between free and bound names. The free names
of a process, $\freenames{P}$, may be calculated recursively as
follows:

\begin{mathpar}
\freenames{\pzero} := \emptyset
  \and \\
  \freenames{x?(y).P} := \{ x \} \cup (\freenames{P} \setminus \{ y \})
  \and 
  \freenames{x!\langle P \rangle} := \{ x \} \cup \{ P \} 
  \and \\
  \freenames{P|Q} := \freenames{P} \cup \freenames{Q}
  \and \\
  \freenames{@{x}} := \{ x \}
\end{mathpar}

$\pi$
$\quotep{\pi}$

$\freenames{-} : \pi \to \mathcal{P}(\quotep{\pi})$

\begin{eqnarray*}
  \freenames{\pzero} & := & \emptyset \\
  \freenames{x?(y).P} & := & \{ x \} \cup (\freenames{P} \setminus \{ y \}) \\
  \freenames{x!\langle P \rangle} & := & \{ x \} \cup \{ P \} \\
  \freenames{P|Q} & := & \freenames{P} \cup \freenames{Q} \\
  \freenames{\dropn{x}} & := & \{ x \}
\end{eqnarray*}

The bound names of a process, $\boundnames{P}$, are those names occurring in $P$
that are not free. For example, in $x?(y).0$, the name $x$ is free, while $y$ is bound.

\begin{mathpar}
  \inferrule* [lab=monoidal-laws] {} { P|Q \equiv Q|P \and P|0 \equiv P \and P|(Q|R) \equiv (P|Q)|R }
\end{mathpar}

\begin{mathpar}
  \inferrule* [lab=alpha-equivalence] {} { (x)P \equiv (y)P\{y/x\} \and y \not\in \freenames{P} }
\end{mathpar}

\begin{definition}
Then two processes, $P,Q$, are alpha-equivalent if $P = Q\{\vec{y}/\vec{x}\}$ for
some $\vec{x} \in \boundnames{Q},\vec{y} \in \boundnames{P}$, where $Q\{\vec{y}/\vec{x}\}$
denotes the capture-avoiding substitution of $\vec{y}$ for $\vec{x}$ in $Q$.
\end{definition}

\begin{definition}
  The {\em structural congruence} \cite{SangiorgiWalker} , $\equiv$,
  between processes is the least congruence containing
  alpha-equivalence, satisfying the abelian monoid laws
  (associativity, commutativity and $\pzero$ as identity) for parallel
  composition $|$ and for summation $+$.
\end{definition}

\subsection{Name equivalence}

We take name equivalence, written $\nameeq$, to be the smallest
equivalence relation generated by the following rules.

\begin{mathpar}
\inferrule*[lab=Quote-drop]
{ }
{ \quotep{@{x}} \nameeq x }

\inferrule*[lab=Struct-equiv]
{ P \scong Q }
{ \quotep{P} \nameeq \quotep{Q} }
\end{mathpar}

The astute reader will have noticed that the mutual recursion of names
and processes imposes a mutual recursion on alpha-equivalence and
structural equivalence via name-equivalence. Fortunately, all of this
works out pleasantly and we may calculate in the natural way, free of
concern. The reader interested in the details is referred to the
appendix \ref{appendix:rho_details}.

\subsection{Substitution}

We use $\Proc$ for the set of processes, $\QProc$ for the set of
names, and $\id{\{}\vec{y} / \vec{x} \id{\}}$ to denote partial maps,
$s : \QProc \rightarrow \QProc$. A map, $s$ lifts, uniquely, to a map
on process terms, $\widehat{s} : \Proc \rightarrow \Proc$ by the
following equations.

\begin{mathpar}
  (0) \psubstp{Q}{P} := 0 \\
  (R \juxtap S) \psubstp{Q}{P}
  :=    
  (R)\psubstp{Q}{P} \juxtap (S) \psubstp{Q}{P} \\
  (x?(y).R) \psubstp{Q}{P}    
  :=    
  (x)\substp{Q}{P} (z)\concat( (R \psubstn{z}{y}) \psubstp{Q}{P} ) \\
  (\lift{x}{R}) \psubstp{Q}{P}  
  :=
  \lift{(x)\substp{Q}{P}}{ R \psubstp{Q}{P} } \\
%   (\dropn{x})  \psubstp{Q}{P}       
%   := 
%   \left\{ 
%     \begin{array}{ccc} 
%       \dropn{\quotep{Q}} & & x \nameeq \quotep{P} \\
%       \dropn{x} & & otherwise \\
%     \end{array}
%   \right. 
  (\dropn{x})  \psubstp{Q}{P}       
  := 
  \left\{ 
    \begin{array}{ccc} 
      Q & & x \nameeq \quotep{P} \\
      \dropn{x} & & otherwise \\
    \end{array}
  \right.
\end{mathpar}
 

where

\begin{eqnarray}
  (x)\id{\{} \lpquote Q \rpquote / \lpquote P \rpquote \id{\}}            = 
  \left\{ 
    \begin{array}{ccc}
      \lpquote Q \rpquote & & x \nameeq \lpquote P \rpquote \\
      x & & otherwise \\
    \end{array}
  \right. \nonumber
\end{eqnarray}

and $z$ is chosen distinct from $\quotep{P}$, $\quotep{Q}$, the free
names in $Q$, and all the names in $R$. Our $\alpha$-equivalence will
be built in the standard way from this substitution.

\begin{remark}\label{rem:no_self_referential_names}
  One consequence of these definitions is that $\forall P. \quotep{P}
  \not\in \freenames{P}$.
\end{remark}

\subsection{ Dynamic quote: an example }

Anticipating something of what's to come, consider applying the
substitution, $\widehat{\id{\{}u / z \id{\}}}$, to the following pair
of processes, $\lift{w}{y!(z)}$ and $w[ \lpquote y!(z) \rpquote ]$.

\begin{eqnarray}
	\lift{w}{y!(z)}\widehat{\id{\{}u / z \id{\}}}
		& = &
		\lift{w}{y!(u)} \nonumber\\
	w[ \lpquote y!(z) \rpquote ] \widehat{ \id{\{}u / z \id{\}} }
		& = &
		w[ \lpquote y!(z) \rpquote ] \nonumber
\end{eqnarray}

Because the body of the process between quotes is impervious to
substitution, we get radically different answers. In fact, by
examining the first process in an input context,
e.g. $x?(z).\lift{w}{y!(z)}$, we see that the process under the lift
operator may be shaped by prefixed inputs binding a name inside it. In
this sense, the lift operator will be seen as a way to dynamically
construct processes before reifying them as names.

Finally equipped with these standard features we can present the
dynamics of the calculus.

\subsubsection{Operational semantics} 

Finally, we introduce the computational dynamics. What marks these
algebras as distinct from other more traditionally studied algebraic
structures, e.g. vector spaces or polynomial rings, is the manner in
which dynamics is captured. In traditional structures, dynamics is typically
expressed through morphisms between such structures, as in linear maps
between vector spaces or morphisms between rings. In algebras
associated with the semantics of computation, the dynamics is
expressed as part of the algebraic structure itself, through a
reduction reduction relation typically denoted by $\red$. Below, we
give a recursive presentation of this relation for the calculus used
in the encoding.

$\red \subseteq \pi \times \pi$
$\red : \pi \to \mathcal{P}(\pi)$

\begin{mathpar}
  \inferrule* [lab=Comm] { \textsf{match}( x_{src}, x_{trgt} ) } { x_{trgt}?(y)P \; | \; x_{src}!\langle {Q} \rangle \red P\{\quotep{Q}/y}\} }
  \and \\
  \inferrule* [lab=Par] {{P} \red {P}'} {{{P} | {Q}} \red {{P}' | {Q}}}
  \and
  \inferrule* [lab=Equiv]{{{P} \scong {P}'} \andalso {{P}' \red {Q}'} \andalso {{Q}' \scong {Q}}}{{P} \red {Q}}
\end{mathpar}

\begin{eqnarray*}
  match_{\equiv} (\quotep{P},\quotep{Q}) & := & P \equiv Q \\
  match_{\dagger}(\quotep{P},\quotep{Q}) & := & \forall R. P|Q \red^{*} R => R \red^{*} 0 \\
  match_{K}(\quotep{P},\quotep{Q}) & := & K \mbox{ for some context } K
\end{eqnarray*}

$u?(x)P | u!\langle Q \rangle \red P\{\quotep{Q}/x\}$

%We write $\wred$ for $\red^*$, and $P\red$ if $\exists Q $ such that $ P \red Q$.
We write $P\red$ if $\exists Q $ such that $ P \red Q$ and $P\not\red$, otherwise.

\section{Replication}

As mentioned before, it is known that replication (and hence
recursion) can be implemented in a higher-order process algebra
\cite{SangiorgiWalker}. As our first example of calculation with the
machinery thus far presented we give the construction explicitly in
the {\rhoc}.

\begin{eqnarray}
	D_{x} & := & \prefix{x}{y}{(\binpar{\outputp{x}{y}}{@{y}})} \nonumber\\
	\bangp_{x}{P} & := & \binpar{{x}!\langle{\binpar{D_{x}}{P}}\rangle}{D_{x}} \nonumber
\end{eqnarray}

\begin{eqnarray}
	\bangp_{x}{P} & & \nonumber\\
	=
	& {x}!\langle{(\prefix{x}{y}{(\outputp{x}{y} | @{y})) | P}}\rangle 
	      | \prefix{x}{y}{(\outputp{x}{y} | @{y})} & \nonumber\\
	\red
	& (\outputp{x}{y} | @{y})\substn{\quotep{(\prefix{x}{y}{(@{y} | \outputp{x}{y})) | P}}}{y} & \nonumber\\
	=
	& \outputp{x}{\quotep{(\prefix{x}{y}{(\outputp{x}{y} | @{y})) | P}}}
	  | {(\prefix{x}{y}{(\outputp{x}{y} | @{y})) | P}} & \nonumber\\
	\red
	& \ldots & \nonumber\\
	\red^*
	& P | P | \ldots & \nonumber
\end{eqnarray}

Of course, this encoding, as an implementation, runs away, unfolding
$\bangp{P}$ eagerly. A lazier and more implementable replication
operator, restricted to input-guarded processes, may be obtained as follows.

\begin{eqnarray}
\bangp{\prefix{u}{v}{P}} 
	:= 
	\binpar{\lift{x}{\prefix{u}{v}{(\binpar{D(x)}{P})}}}{D(x)} \nonumber
\end{eqnarray}

\begin{remark}
  Note that the lazier definition still does not deal with summation
  or mixed summation (i.e. sums over input and output). The reader is
  invited to construct definitions of replication that deal with these
  features. 

  Further, the definitions are parameterized in a name, $x$. Can you,
  gentle reader, make a definition that eliminates this parameter and
  guarantees no accidental interaction between the replication
  machinery and the process being replicated -- i.e. no accidental
  sharing of names used by the process to get its work done and the
  name(s) used by the replication to effect copying. This latter
  revision of the definition of replication is crucial to obtaining
  the expected identity $!!P \sim !P$.
\end{remark}

\begin{remark}\label{rem:paradoxical_combinator}
  The reader familiar with the lambda calculus will have noticed the
  similarity between $D$ and the paradoxical combinator.

  [Ed. note: the existence of this seems to suggest we have to be more
  restrictive on the set of processes and names we admit if we are to
  support no-cloning.]
\end{remark}

\subsubsection{Bisimulation}

The computational dynamics gives rise to another kind of equivalence,
the equivalence of computational behavior. As previously mentioned
this is typically captured \emph{via} some form of bisimulation.

% The notion we use in this paper is weak barbed bisimulation
% \cite{milner91polyadicpi}.

The notion we use in this paper is derived from weak barbed
bisimulation \cite{milner91polyadicpi}. 

\begin{definition}
An \emph{observation relation}, $\downarrow_{\mathcal N}$, over a set
of names, $\mathcal N$, is the smallest relation satisfying the rules
below.

\infrule[Out-barb]{y \in {\mathcal N}, \; x \nameeq y}
		  {\outputp{x}{v} \downarrow_{\mathcal N} x}
\infrule[Par-barb]{\mbox{$P\downarrow_{\mathcal N} x$ or $Q\downarrow_{\mathcal N} x$}}
		  {\binpar{P}{Q} \downarrow_{\mathcal N} x}

We write $P \Downarrow_{\mathcal N} x$ if there is $Q$ such that 
$P \wred Q$ and $Q \downarrow_{\mathcal N} x$.
\end{definition}

\begin{definition}
%\label{def.bbisim}
An  ${\mathcal N}$-\emph{barbed bisimulation} over a set of names, ${\mathcal N}$, is a symmetric binary relation 
${\mathcal S}_{\mathcal N}$ between agents such that $P\rel{S}_{\mathcal N}Q$ implies:
\begin{enumerate}
\item If $P \red P'$ then $Q \wred Q'$ and $P'\rel{S}_{\mathcal N} Q'$.
\item If $P\downarrow_{\mathcal N} x$, then $Q\Downarrow_{\mathcal N} x$.
\end{enumerate}
$P$ is ${\mathcal N}$-barbed bisimilar to $Q$, written
$P \wbbisim_{\mathcal N} Q$, if $P \rel{S}_{\mathcal N} Q$ for some ${\mathcal N}$-barbed bisimulation ${\mathcal S}_{\mathcal N}$.
\end{definition}

$\mathcal{R} \subseteq \pi \times \pi$

$P \mathcal{R} Q => \forall P'. P \red P' \Rightarrow \exists Q'. Q \red Q', P' \mathcal{R} Q'$

$P \vdash x \Rightarrow Q \vdash x$

\begin{mathpar}
  \inferrule*[lab=Out-barb]{x \nameeq y}{{y}!\langle{Q}\rangle \vdash x}
  \and
  \inferrule*[lab=Par-barb]{\mbox{$P\vdash x$ or $Q\vdash x$}}{\binpar{P}{Q} \vdash x}
\end{mathpar}

\subsubsection{Contexts}

One of the principle advantages of computational calculi like the
$\pi$-calculus is a well-defined notion of context,
contextual-equivalence and a correlation between
contextual-equivalence and notions of bisimulation. The notion of
context allows the decomposition of a process into (sub-)process and
its syntactic environment, its context. Thus, a context may be
thought of as a process with a ``hole'' (written $\Box$) in it. The
application of a context $M$ to a process $P$, written $M[P]$, is
tantamount to filling the hole in $M$ with $P$. In this paper we do
not need the full weight of this theory, but do make use of the notion
of context in the proof the main theorem. 

\begin{mathpar}
  \inferrule* [lab=summation] {} {{M_{M},M_{N}} \bc \Box \;|\; x.M_{A} \;|\; M_{M}+M_{N}}
  \and
  \inferrule* [lab=agent] {} {{M_{A}} \bc (\vec{x})M_{P} \;| \; \clift{P_0,\ldots,M_{P},\ldots,P_N}}
  \and \\
  \inferrule* [lab=process] {} {{M_{P}} \bc M_{N} \;| \;P|M_{P} }
\end{mathpar} 

\begin{mathpar}
  \inferrule* [lab=sychronization] {} {M_{N} \bc \Box \;|\; x?M_{F} \;|\; x!M_{C}}
  \and
  \inferrule* [lab=abstraction] {} {{M_{F}} \bc (x)M_{P} }
  \and
  \inferrule* [lab=concretion] {} {{M_{C}} \bc \langle M_{P} \rangle }
  \and \\
  \inferrule* [lab=process] {} {{M_{P}} \bc M_{N} \;| \;P|M_{P} }
\end{mathpar}

\begin{definition}[contextual application] Given a context $M$, and
  process $P$, we define the \emph{contextual application}, $M[P] :=
  M\{P/\Box\}$. That is, the contextual application of M to P is the
  substitution of $P$ for $\Box$ in $M$.
\end{definition}

$\meaningof{-} : L \to \mathcal{P}(\pi)$

\begin{mathpar}
  \inferrule* [lab=collection] {} {\meaningof{true} = \pi, \and \meaningof{~E} = \pi \setminus \meaningof{E}, \and \meaningof{E_{1} \& E_{2}} = \meaningof{E_{1}} \cap \meaningof{E_{2}}}
\end{mathpar}

\begin{mathpar}
  \inferrule* [lab=structure] {} {\meaningof{0} = \{ P \in \pi | P \equiv 0 \}, \and \\ \meaningof{E_1 | E_2} = \{ P \in \pi | P \equiv P_{1} | P_{2}, P_{1} \in \meaningof{E_{1}}, P_{2} \in \meaningof{E_2}\} }
\end{mathpar}

\begin{mathpar}
 \inferrule* [lab=behavior] {} {\meaningof{\langle a?b \rangle E} = \{ P \in \pi | P \equiv Q | u?(y)P', \\ \and \\\\ \and \\ \;\;\; u \in \meaningof{a}, \forall z.P'\{z/y\} \in \meaningof{E\{z/b\}}\}, \and \\ \meaningof{a!E} = \{ P \in \pi | P \equiv Q | x!\langle P' \rangle, x \in \meaningof{a} P' \in \meaningof{E}\} }
\end{mathpar}

\begin{mathpar}
 \inferrule* [lab=nominal] {} {\meaningof{\quotep{E}} = \{ \quotep{P} \in \quotep{\pi} | P \in \meaningof{E} \}, \and \meaningof{\quotep{P}} = \{ \quotep{Q} \in \quotep{\pi} | P \equiv Q \} \and \\ \meaningof{@\quotep{E}} = \{ P \in \pi | P \equiv @x, x \in \meaningof{E} \}}
\end{mathpar}

\begin{eqnarray*}
  \\
  \meaningof{-} : TS \to ST
\end{eqnarray*}

\begin{eqnarray*}
  \\
  L : TS \to ST
\end{eqnarray*}

\begin{eqnarray*}
  \\
  P \models E \iff P \in \meaningof{E}
\end{eqnarray*}

\begin{eqnarray*}
  P \approx_{L} Q \iff \forall E \in L. P \models E \iff Q \models E
\end{eqnarray*}

\begin{eqnarray*}
  P \approx_{K} Q
\end{eqnarray*}

\begin{eqnarray*}
  P \approx Q
\end{eqnarray*}

$\approx_{K} = \approx = \approx_{L}$

\subsubsection{Contextual duality}

Note that contexts extend the quotation operation to a family of
operations from processes to names. Given a context, $M$, we can
define a \emph{nominal context}, $\quotep{M}$ by $\quotep{M}[P] :=
\quotep{M[P]}$. To foreshadow what is to come we observe that these
operations enjoy a duality with processes very much like the duality
between vectors and maps from vectors to scalars.

Further, because the calculus is essentially higher-order, we have a
correspondence between contexts and processes. More specifically,
given a name $x$ and a context $M$ we can construct $M^{*}_{x}$ such
that 

\begin{mathpar}
  M^{*}_{x} | \lift{x}{P} \red M[P]
\end{mathpar}

namely,

\begin{mathpar}
  M^{*}_{x} := x?(u).M[\dropn{u}]
\end{mathpar}

The dependence of $M^{*}_{x}$ on a name makes it an abstraction, 

\begin{mathpar}
  M^{*} := (x)x?(u).M[\dropn{u}]
\end{mathpar}

\subsection{Additional notation}

It will sometimes be convenient to denote the process a name
quotes. We already have the notation $x = \quotep{P}$, but it will be
convenient to introduce an alternate notation, $\procn{x}$, when we
want to emphasize the connection to the use of the name. Note that, by
virtue of name equivalence, $\quotep{\procn{x}} \nameeq x$; so, the
notation is consistent with previous definitions.

Further, because names have structure it is possible to effect
substitutions on the basis of that structure. This means we need to
upgrade our notation for substitutions, which we accomplish by
adapting comprehension notation. Thus,

\begin{mathpar}
  P\{ y / x : x \in S \}
\end{mathpar}

is interpreted to mean the process derived from P by replacing (in a
capture-avoiding manner) each occurrence of $x$ in $S$ by $y$. For example,

\begin{mathpar}
  P\{ \quotep{\procn{x}|\procn{x}} / x : x \in \freenames{P} \}
\end{mathpar}

will replace each (occurrence) of a free name $x$ in $P$ by
$\quotep{\procn{x}|\procn{x}}$.

Also, we will avail ourselves of the notation $x^{L}$ and $x^{R}$ to
denote injections of a name into disjoint copies of the name
space. There are numerous ways to accomplish this. One example can be
found in \cite{MeredithR05}. This notation overloads to vectors of
names: $\vec{x}^{\pi} := (x_{i}^{\pi} \; : \; 0 \leq i < |\vec{x}| )$ where $\pi \in \{L,R\}$.

We also use $P^{\Box} := P|\Box$.

In \cite{MeredithR05} an interpretation of the new operator is
given. It turns out that there are several possible interpretations
all enjoying the requisite algebraic properties of the operator (see
\cite{milner91polyadicpi}). We will therefore make liberal use of
$(\nu\; \vec{x})P$.

% subsection the_syntax_and_semantics_of_the_notation_system (end)   

\input{qm2pi.qmops} 

\input{qm2pi.sterngerlach} 

\input{qm2pi.metric} 

% section concurrent_process_calculi (end)

%\input{qm2pi.proofsketch}

% section proof sketch (end)

%\input{qm2pi.slviaknots} 

% section spatial logic via knots (end)

\input{qm2pi.conclusion}

% section conclusion (end)

%\input{qm2pi.dtcodes} 

% section wiring algorithm (end)

\input{qm2pi.ack} 

% section acknowledgments (end)

\newpage


\bibliographystyle{plain}   
\bibliography{../../biblios/main.bib}

\input{qm2pi.rhodetails}

\end{document}

 

%\documentclass[12pt]{llncs}
%\documentclass{jktr}

\usepackage[pdftex]{hyperref}                   
\usepackage {listings}
\usepackage {mathpartir}
\usepackage{bcprules}
%\usepackage{listings}
                       
\usepackage{graphicx} 
%\usepackage[margins=2.5cm,nohead,nofoot]{geometry}
%\usepackage{geometry}
\usepackage{amsfonts}
\usepackage{amstext}
\usepackage{latexsym}
\usepackage{amssymb}
\usepackage{color}


%\include{myPreamble}
\include{qm2pi.local} 

%\ifpdf
%\usepackage[pdftex]{graphicx}
%\else
%\usepackage{graphicx}
%\fi

 % \ifpdf
%  \usepackage{pdfsync}
%  \if


%\title{Brief Article}
%\author{David F. Snyder}
%\author{L.G. Meredith}

%\address{Dept. of Math., Texas State University--San Marcos, San Marcos, TX 78666}
       
\pagestyle{empty}


\begin{document}

\lstset{language=[Objective]Caml,frame=shadowbox}

\input{qm2pi.front}

% section front matter (end)

\input{qm2pi.intro} 
 
% section introduction (end)

% \input{qm2pi.knotations} 

% section notation (end)

\input{qm2pi.process.calculi} 

% section concurrent_process_calculi_and_spatial_logics_ (end)
    
%\input{qm2pi.knots2pi} 

%\input{qm2pi.trefoil} 

%\input{qm2pi.mainthm} 

% subsection basic_interpretation (end)

%\input{qm2pi.rho.presentation} 
\subsection{The syntax and semantics of the notation system}\label{sub:the_syntax_and_semantics_of_the_notation_system} % (fold)

We now summarize a technical presentation of the calculus that
embodies our theory of dynamics. The typical presentation of such a
calculus follows the style of giving generators and relations on
them. The grammar, below, describing term constructors, freely
generates the set of processes, $\Proc$. This set is then quotiented
by a relation known as structural congruence and it is over this set
that the notion of dynamics is expressed. This presentation is
essentially that of \cite{MeredithR05} with the addition of
polyadicity and summation. For readability we have relegated some of
the technical subtleties to an appendix.

\subsubsection{Process grammar}\label{subsub:process_grammar}

\begin{mathpar}
  \inferrule* [lab=synchronization] {} {{M} \bc \pzero \;|\; x?F \;|\; x!C }
  \and
  \inferrule* [lab=abstraction] {} {{F} \bc (x)P}
  \and
  \inferrule* [lab=concretion] {} {{C} \bc \langle Q \rangle}
  \and
  \inferrule* [lab=process] {} {{P,Q} \bc M \;| \;P|Q \;|\; @{x}}
  \and
  \inferrule* [lab=name] {} {{x} \bc \quotep{P}}
\end{mathpar} 

Note that $\vec{x}$ (resp. $\vec{P}$) denotes a vector of names
(resp. processes) of length $|\vec{x}|$ (resp. $|\vec{P}|$). We adopt
the following useful abbreviations.

\begin{mathpar}
   x?(\vec{y}).P := x.(\vec{y})P \and  x\clift{\vec{P}} := x.\clift{\vec{P}}
   \and x!(y) := \lift{x}{\dropn{y}}
   \and \Pi_{i=0}^{n-1}P_i := P_0 | \ldots | P_{n-1}
\end{mathpar}

\subsubsection{Structural congruence}

\paragraph{Free and bound names and alpha-equivalence.} At the
core of structural equivalence is alpha-equivalence which identifies
process that are the same up to a change of variable. Formally, we
recognize the distinction between free and bound names. The free names
of a process, $\freenames{P}$, may be calculated recursively as
follows:

\begin{mathpar}
\freenames{\pzero} := \emptyset
  \and \\
  \freenames{x?(y).P} := \{ x \} \cup (\freenames{P} \setminus \{ y \})
  \and 
  \freenames{x!\langle P \rangle} := \{ x \} \cup \{ P \} 
  \and \\
  \freenames{P|Q} := \freenames{P} \cup \freenames{Q}
  \and \\
  \freenames{@{x}} := \{ x \}
\end{mathpar}

$\pi$
$\quotep{\pi}$

$\freenames{-} : \pi \to \mathcal{P}(\quotep{\pi})$

\begin{eqnarray*}
  \freenames{\pzero} & := & \emptyset \\
  \freenames{x?(y).P} & := & \{ x \} \cup (\freenames{P} \setminus \{ y \}) \\
  \freenames{x!\langle P \rangle} & := & \{ x \} \cup \{ P \} \\
  \freenames{P|Q} & := & \freenames{P} \cup \freenames{Q} \\
  \freenames{\dropn{x}} & := & \{ x \}
\end{eqnarray*}

The bound names of a process, $\boundnames{P}$, are those names occurring in $P$
that are not free. For example, in $x?(y).0$, the name $x$ is free, while $y$ is bound.

\begin{mathpar}
  \inferrule* [lab=monoidal-laws] {} { P|Q \equiv Q|P \and P|0 \equiv P \and P|(Q|R) \equiv (P|Q)|R }
\end{mathpar}

\begin{mathpar}
  \inferrule* [lab=alpha-equivalence] {} { (x)P \equiv (y)P\{y/x\} \and y \not\in \freenames{P} }
\end{mathpar}

\begin{definition}
Then two processes, $P,Q$, are alpha-equivalent if $P = Q\{\vec{y}/\vec{x}\}$ for
some $\vec{x} \in \boundnames{Q},\vec{y} \in \boundnames{P}$, where $Q\{\vec{y}/\vec{x}\}$
denotes the capture-avoiding substitution of $\vec{y}$ for $\vec{x}$ in $Q$.
\end{definition}

\begin{definition}
  The {\em structural congruence} \cite{SangiorgiWalker} , $\equiv$,
  between processes is the least congruence containing
  alpha-equivalence, satisfying the abelian monoid laws
  (associativity, commutativity and $\pzero$ as identity) for parallel
  composition $|$ and for summation $+$.
\end{definition}

\subsection{Name equivalence}

We take name equivalence, written $\nameeq$, to be the smallest
equivalence relation generated by the following rules.

\begin{mathpar}
\inferrule*[lab=Quote-drop]
{ }
{ \quotep{@{x}} \nameeq x }

\inferrule*[lab=Struct-equiv]
{ P \scong Q }
{ \quotep{P} \nameeq \quotep{Q} }
\end{mathpar}

The astute reader will have noticed that the mutual recursion of names
and processes imposes a mutual recursion on alpha-equivalence and
structural equivalence via name-equivalence. Fortunately, all of this
works out pleasantly and we may calculate in the natural way, free of
concern. The reader interested in the details is referred to the
appendix \ref{appendix:rho_details}.

\subsection{Substitution}

We use $\Proc$ for the set of processes, $\QProc$ for the set of
names, and $\id{\{}\vec{y} / \vec{x} \id{\}}$ to denote partial maps,
$s : \QProc \rightarrow \QProc$. A map, $s$ lifts, uniquely, to a map
on process terms, $\widehat{s} : \Proc \rightarrow \Proc$ by the
following equations.

\begin{mathpar}
  (0) \psubstp{Q}{P} := 0 \\
  (R \juxtap S) \psubstp{Q}{P}
  :=    
  (R)\psubstp{Q}{P} \juxtap (S) \psubstp{Q}{P} \\
  (x?(y).R) \psubstp{Q}{P}    
  :=    
  (x)\substp{Q}{P} (z)\concat( (R \psubstn{z}{y}) \psubstp{Q}{P} ) \\
  (\lift{x}{R}) \psubstp{Q}{P}  
  :=
  \lift{(x)\substp{Q}{P}}{ R \psubstp{Q}{P} } \\
%   (\dropn{x})  \psubstp{Q}{P}       
%   := 
%   \left\{ 
%     \begin{array}{ccc} 
%       \dropn{\quotep{Q}} & & x \nameeq \quotep{P} \\
%       \dropn{x} & & otherwise \\
%     \end{array}
%   \right. 
  (\dropn{x})  \psubstp{Q}{P}       
  := 
  \left\{ 
    \begin{array}{ccc} 
      Q & & x \nameeq \quotep{P} \\
      \dropn{x} & & otherwise \\
    \end{array}
  \right.
\end{mathpar}
 

where

\begin{eqnarray}
  (x)\id{\{} \lpquote Q \rpquote / \lpquote P \rpquote \id{\}}            = 
  \left\{ 
    \begin{array}{ccc}
      \lpquote Q \rpquote & & x \nameeq \lpquote P \rpquote \\
      x & & otherwise \\
    \end{array}
  \right. \nonumber
\end{eqnarray}

and $z$ is chosen distinct from $\quotep{P}$, $\quotep{Q}$, the free
names in $Q$, and all the names in $R$. Our $\alpha$-equivalence will
be built in the standard way from this substitution.

\begin{remark}\label{rem:no_self_referential_names}
  One consequence of these definitions is that $\forall P. \quotep{P}
  \not\in \freenames{P}$.
\end{remark}

\subsection{ Dynamic quote: an example }

Anticipating something of what's to come, consider applying the
substitution, $\widehat{\id{\{}u / z \id{\}}}$, to the following pair
of processes, $\lift{w}{y!(z)}$ and $w[ \lpquote y!(z) \rpquote ]$.

\begin{eqnarray}
	\lift{w}{y!(z)}\widehat{\id{\{}u / z \id{\}}}
		& = &
		\lift{w}{y!(u)} \nonumber\\
	w[ \lpquote y!(z) \rpquote ] \widehat{ \id{\{}u / z \id{\}} }
		& = &
		w[ \lpquote y!(z) \rpquote ] \nonumber
\end{eqnarray}

Because the body of the process between quotes is impervious to
substitution, we get radically different answers. In fact, by
examining the first process in an input context,
e.g. $x?(z).\lift{w}{y!(z)}$, we see that the process under the lift
operator may be shaped by prefixed inputs binding a name inside it. In
this sense, the lift operator will be seen as a way to dynamically
construct processes before reifying them as names.

Finally equipped with these standard features we can present the
dynamics of the calculus.

\subsubsection{Operational semantics} 

Finally, we introduce the computational dynamics. What marks these
algebras as distinct from other more traditionally studied algebraic
structures, e.g. vector spaces or polynomial rings, is the manner in
which dynamics is captured. In traditional structures, dynamics is typically
expressed through morphisms between such structures, as in linear maps
between vector spaces or morphisms between rings. In algebras
associated with the semantics of computation, the dynamics is
expressed as part of the algebraic structure itself, through a
reduction reduction relation typically denoted by $\red$. Below, we
give a recursive presentation of this relation for the calculus used
in the encoding.

$\red \subseteq \pi \times \pi$
$\red : \pi \to \mathcal{P}(\pi)$

\begin{mathpar}
  \inferrule* [lab=Comm] { \textsf{match}( x_{src}, x_{trgt} ) } { x_{trgt}?(y)P \; | \; x_{src}!\langle {Q} \rangle \red P\{\quotep{Q}/y}\} }
  \and \\
  \inferrule* [lab=Par] {{P} \red {P}'} {{{P} | {Q}} \red {{P}' | {Q}}}
  \and
  \inferrule* [lab=Equiv]{{{P} \scong {P}'} \andalso {{P}' \red {Q}'} \andalso {{Q}' \scong {Q}}}{{P} \red {Q}}
\end{mathpar}

\begin{eqnarray*}
  match_{\equiv} (\quotep{P},\quotep{Q}) & := & P \equiv Q \\
  match_{\dagger}(\quotep{P},\quotep{Q}) & := & \forall R. P|Q \red^{*} R => R \red^{*} 0 \\
  match_{K}(\quotep{P},\quotep{Q}) & := & K \mbox{ for some context } K
\end{eqnarray*}

$u?(x)P | u!\langle Q \rangle \red P\{\quotep{Q}/x\}$

%We write $\wred$ for $\red^*$, and $P\red$ if $\exists Q $ such that $ P \red Q$.
We write $P\red$ if $\exists Q $ such that $ P \red Q$ and $P\not\red$, otherwise.

\section{Replication}

As mentioned before, it is known that replication (and hence
recursion) can be implemented in a higher-order process algebra
\cite{SangiorgiWalker}. As our first example of calculation with the
machinery thus far presented we give the construction explicitly in
the {\rhoc}.

\begin{eqnarray}
	D_{x} & := & \prefix{x}{y}{(\binpar{\outputp{x}{y}}{@{y}})} \nonumber\\
	\bangp_{x}{P} & := & \binpar{{x}!\langle{\binpar{D_{x}}{P}}\rangle}{D_{x}} \nonumber
\end{eqnarray}

\begin{eqnarray}
	\bangp_{x}{P} & & \nonumber\\
	=
	& {x}!\langle{(\prefix{x}{y}{(\outputp{x}{y} | @{y})) | P}}\rangle 
	      | \prefix{x}{y}{(\outputp{x}{y} | @{y})} & \nonumber\\
	\red
	& (\outputp{x}{y} | @{y})\substn{\quotep{(\prefix{x}{y}{(@{y} | \outputp{x}{y})) | P}}}{y} & \nonumber\\
	=
	& \outputp{x}{\quotep{(\prefix{x}{y}{(\outputp{x}{y} | @{y})) | P}}}
	  | {(\prefix{x}{y}{(\outputp{x}{y} | @{y})) | P}} & \nonumber\\
	\red
	& \ldots & \nonumber\\
	\red^*
	& P | P | \ldots & \nonumber
\end{eqnarray}

Of course, this encoding, as an implementation, runs away, unfolding
$\bangp{P}$ eagerly. A lazier and more implementable replication
operator, restricted to input-guarded processes, may be obtained as follows.

\begin{eqnarray}
\bangp{\prefix{u}{v}{P}} 
	:= 
	\binpar{\lift{x}{\prefix{u}{v}{(\binpar{D(x)}{P})}}}{D(x)} \nonumber
\end{eqnarray}

\begin{remark}
  Note that the lazier definition still does not deal with summation
  or mixed summation (i.e. sums over input and output). The reader is
  invited to construct definitions of replication that deal with these
  features. 

  Further, the definitions are parameterized in a name, $x$. Can you,
  gentle reader, make a definition that eliminates this parameter and
  guarantees no accidental interaction between the replication
  machinery and the process being replicated -- i.e. no accidental
  sharing of names used by the process to get its work done and the
  name(s) used by the replication to effect copying. This latter
  revision of the definition of replication is crucial to obtaining
  the expected identity $!!P \sim !P$.
\end{remark}

\begin{remark}\label{rem:paradoxical_combinator}
  The reader familiar with the lambda calculus will have noticed the
  similarity between $D$ and the paradoxical combinator.

  [Ed. note: the existence of this seems to suggest we have to be more
  restrictive on the set of processes and names we admit if we are to
  support no-cloning.]
\end{remark}

\subsubsection{Bisimulation}

The computational dynamics gives rise to another kind of equivalence,
the equivalence of computational behavior. As previously mentioned
this is typically captured \emph{via} some form of bisimulation.

% The notion we use in this paper is weak barbed bisimulation
% \cite{milner91polyadicpi}.

The notion we use in this paper is derived from weak barbed
bisimulation \cite{milner91polyadicpi}. 

\begin{definition}
An \emph{observation relation}, $\downarrow_{\mathcal N}$, over a set
of names, $\mathcal N$, is the smallest relation satisfying the rules
below.

\infrule[Out-barb]{y \in {\mathcal N}, \; x \nameeq y}
		  {\outputp{x}{v} \downarrow_{\mathcal N} x}
\infrule[Par-barb]{\mbox{$P\downarrow_{\mathcal N} x$ or $Q\downarrow_{\mathcal N} x$}}
		  {\binpar{P}{Q} \downarrow_{\mathcal N} x}

We write $P \Downarrow_{\mathcal N} x$ if there is $Q$ such that 
$P \wred Q$ and $Q \downarrow_{\mathcal N} x$.
\end{definition}

\begin{definition}
%\label{def.bbisim}
An  ${\mathcal N}$-\emph{barbed bisimulation} over a set of names, ${\mathcal N}$, is a symmetric binary relation 
${\mathcal S}_{\mathcal N}$ between agents such that $P\rel{S}_{\mathcal N}Q$ implies:
\begin{enumerate}
\item If $P \red P'$ then $Q \wred Q'$ and $P'\rel{S}_{\mathcal N} Q'$.
\item If $P\downarrow_{\mathcal N} x$, then $Q\Downarrow_{\mathcal N} x$.
\end{enumerate}
$P$ is ${\mathcal N}$-barbed bisimilar to $Q$, written
$P \wbbisim_{\mathcal N} Q$, if $P \rel{S}_{\mathcal N} Q$ for some ${\mathcal N}$-barbed bisimulation ${\mathcal S}_{\mathcal N}$.
\end{definition}

$\mathcal{R} \subseteq \pi \times \pi$

$P \mathcal{R} Q => \forall P'. P \red P' \Rightarrow \exists Q'. Q \red Q', P' \mathcal{R} Q'$

$P \vdash x \Rightarrow Q \vdash x$

\begin{mathpar}
  \inferrule*[lab=Out-barb]{x \nameeq y}{{y}!\langle{Q}\rangle \vdash x}
  \and
  \inferrule*[lab=Par-barb]{\mbox{$P\vdash x$ or $Q\vdash x$}}{\binpar{P}{Q} \vdash x}
\end{mathpar}

\subsubsection{Contexts}

One of the principle advantages of computational calculi like the
$\pi$-calculus is a well-defined notion of context,
contextual-equivalence and a correlation between
contextual-equivalence and notions of bisimulation. The notion of
context allows the decomposition of a process into (sub-)process and
its syntactic environment, its context. Thus, a context may be
thought of as a process with a ``hole'' (written $\Box$) in it. The
application of a context $M$ to a process $P$, written $M[P]$, is
tantamount to filling the hole in $M$ with $P$. In this paper we do
not need the full weight of this theory, but do make use of the notion
of context in the proof the main theorem. 

\begin{mathpar}
  \inferrule* [lab=summation] {} {{M_{M},M_{N}} \bc \Box \;|\; x.M_{A} \;|\; M_{M}+M_{N}}
  \and
  \inferrule* [lab=agent] {} {{M_{A}} \bc (\vec{x})M_{P} \;| \; \clift{P_0,\ldots,M_{P},\ldots,P_N}}
  \and \\
  \inferrule* [lab=process] {} {{M_{P}} \bc M_{N} \;| \;P|M_{P} }
\end{mathpar} 

\begin{mathpar}
  \inferrule* [lab=sychronization] {} {M_{N} \bc \Box \;|\; x?M_{F} \;|\; x!M_{C}}
  \and
  \inferrule* [lab=abstraction] {} {{M_{F}} \bc (x)M_{P} }
  \and
  \inferrule* [lab=concretion] {} {{M_{C}} \bc \langle M_{P} \rangle }
  \and \\
  \inferrule* [lab=process] {} {{M_{P}} \bc M_{N} \;| \;P|M_{P} }
\end{mathpar}

\begin{definition}[contextual application] Given a context $M$, and
  process $P$, we define the \emph{contextual application}, $M[P] :=
  M\{P/\Box\}$. That is, the contextual application of M to P is the
  substitution of $P$ for $\Box$ in $M$.
\end{definition}

$\meaningof{-} : L \to \mathcal{P}(\pi)$

\begin{mathpar}
  \inferrule* [lab=collection] {} {\meaningof{true} = \pi, \and \meaningof{~E} = \pi \setminus \meaningof{E}, \and \meaningof{E_{1} \& E_{2}} = \meaningof{E_{1}} \cap \meaningof{E_{2}}}
\end{mathpar}

\begin{mathpar}
  \inferrule* [lab=structure] {} {\meaningof{0} = \{ P \in \pi | P \equiv 0 \}, \and \\ \meaningof{E_1 | E_2} = \{ P \in \pi | P \equiv P_{1} | P_{2}, P_{1} \in \meaningof{E_{1}}, P_{2} \in \meaningof{E_2}\} }
\end{mathpar}

\begin{mathpar}
 \inferrule* [lab=behavior] {} {\meaningof{\langle a?b \rangle E} = \{ P \in \pi | P \equiv Q | u?(y)P', \\ \and \\\\ \and \\ \;\;\; u \in \meaningof{a}, \forall z.P'\{z/y\} \in \meaningof{E\{z/b\}}\}, \and \\ \meaningof{a!E} = \{ P \in \pi | P \equiv Q | x!\langle P' \rangle, x \in \meaningof{a} P' \in \meaningof{E}\} }
\end{mathpar}

\begin{mathpar}
 \inferrule* [lab=nominal] {} {\meaningof{\quotep{E}} = \{ \quotep{P} \in \quotep{\pi} | P \in \meaningof{E} \}, \and \meaningof{\quotep{P}} = \{ \quotep{Q} \in \quotep{\pi} | P \equiv Q \} \and \\ \meaningof{@\quotep{E}} = \{ P \in \pi | P \equiv @x, x \in \meaningof{E} \}}
\end{mathpar}

\begin{eqnarray*}
  \\
  \meaningof{-} : TS \to ST
\end{eqnarray*}

\begin{eqnarray*}
  \\
  L : TS \to ST
\end{eqnarray*}

\begin{eqnarray*}
  \\
  P \models E \iff P \in \meaningof{E}
\end{eqnarray*}

\begin{eqnarray*}
  P \approx_{L} Q \iff \forall E \in L. P \models E \iff Q \models E
\end{eqnarray*}

\begin{eqnarray*}
  P \approx_{K} Q
\end{eqnarray*}

\begin{eqnarray*}
  P \approx Q
\end{eqnarray*}

$\approx_{K} = \approx = \approx_{L}$

\subsubsection{Contextual duality}

Note that contexts extend the quotation operation to a family of
operations from processes to names. Given a context, $M$, we can
define a \emph{nominal context}, $\quotep{M}$ by $\quotep{M}[P] :=
\quotep{M[P]}$. To foreshadow what is to come we observe that these
operations enjoy a duality with processes very much like the duality
between vectors and maps from vectors to scalars.

Further, because the calculus is essentially higher-order, we have a
correspondence between contexts and processes. More specifically,
given a name $x$ and a context $M$ we can construct $M^{*}_{x}$ such
that 

\begin{mathpar}
  M^{*}_{x} | \lift{x}{P} \red M[P]
\end{mathpar}

namely,

\begin{mathpar}
  M^{*}_{x} := x?(u).M[\dropn{u}]
\end{mathpar}

The dependence of $M^{*}_{x}$ on a name makes it an abstraction, 

\begin{mathpar}
  M^{*} := (x)x?(u).M[\dropn{u}]
\end{mathpar}

\subsection{Additional notation}

It will sometimes be convenient to denote the process a name
quotes. We already have the notation $x = \quotep{P}$, but it will be
convenient to introduce an alternate notation, $\procn{x}$, when we
want to emphasize the connection to the use of the name. Note that, by
virtue of name equivalence, $\quotep{\procn{x}} \nameeq x$; so, the
notation is consistent with previous definitions.

Further, because names have structure it is possible to effect
substitutions on the basis of that structure. This means we need to
upgrade our notation for substitutions, which we accomplish by
adapting comprehension notation. Thus,

\begin{mathpar}
  P\{ y / x : x \in S \}
\end{mathpar}

is interpreted to mean the process derived from P by replacing (in a
capture-avoiding manner) each occurrence of $x$ in $S$ by $y$. For example,

\begin{mathpar}
  P\{ \quotep{\procn{x}|\procn{x}} / x : x \in \freenames{P} \}
\end{mathpar}

will replace each (occurrence) of a free name $x$ in $P$ by
$\quotep{\procn{x}|\procn{x}}$.

Also, we will avail ourselves of the notation $x^{L}$ and $x^{R}$ to
denote injections of a name into disjoint copies of the name
space. There are numerous ways to accomplish this. One example can be
found in \cite{MeredithR05}. This notation overloads to vectors of
names: $\vec{x}^{\pi} := (x_{i}^{\pi} \; : \; 0 \leq i < |\vec{x}| )$ where $\pi \in \{L,R\}$.

We also use $P^{\Box} := P|\Box$.

In \cite{MeredithR05} an interpretation of the new operator is
given. It turns out that there are several possible interpretations
all enjoying the requisite algebraic properties of the operator (see
\cite{milner91polyadicpi}). We will therefore make liberal use of
$(\nu\; \vec{x})P$.

% subsection the_syntax_and_semantics_of_the_notation_system (end)   

\input{qm2pi.qmops} 

\input{qm2pi.sterngerlach} 

\input{qm2pi.metric} 

% section concurrent_process_calculi (end)

%\input{qm2pi.proofsketch}

% section proof sketch (end)

%\input{qm2pi.slviaknots} 

% section spatial logic via knots (end)

\input{qm2pi.conclusion}

% section conclusion (end)

%\input{qm2pi.dtcodes} 

% section wiring algorithm (end)

\input{qm2pi.ack} 

% section acknowledgments (end)

\newpage


\bibliographystyle{plain}   
\bibliography{../../biblios/main.bib}

\input{qm2pi.rhodetails}

\end{document}

 

%\documentclass[12pt]{llncs}
%\documentclass{jktr}

\usepackage[pdftex]{hyperref}                   
\usepackage {listings}
\usepackage {mathpartir}
\usepackage{bcprules}
%\usepackage{listings}
                       
\usepackage{graphicx} 
%\usepackage[margins=2.5cm,nohead,nofoot]{geometry}
%\usepackage{geometry}
\usepackage{amsfonts}
\usepackage{amstext}
\usepackage{latexsym}
\usepackage{amssymb}
\usepackage{color}


%\include{myPreamble}
\include{qm2pi.local} 

%\ifpdf
%\usepackage[pdftex]{graphicx}
%\else
%\usepackage{graphicx}
%\fi

 % \ifpdf
%  \usepackage{pdfsync}
%  \if


%\title{Brief Article}
%\author{David F. Snyder}
%\author{L.G. Meredith}

%\address{Dept. of Math., Texas State University--San Marcos, San Marcos, TX 78666}
       
\pagestyle{empty}


\begin{document}

\lstset{language=[Objective]Caml,frame=shadowbox}

\input{qm2pi.front}

% section front matter (end)

\input{qm2pi.intro} 
 
% section introduction (end)

% \input{qm2pi.knotations} 

% section notation (end)

\input{qm2pi.process.calculi} 

% section concurrent_process_calculi_and_spatial_logics_ (end)
    
%\input{qm2pi.knots2pi} 

%\input{qm2pi.trefoil} 

%\input{qm2pi.mainthm} 

% subsection basic_interpretation (end)

%\input{qm2pi.rho.presentation} 
\subsection{The syntax and semantics of the notation system}\label{sub:the_syntax_and_semantics_of_the_notation_system} % (fold)

We now summarize a technical presentation of the calculus that
embodies our theory of dynamics. The typical presentation of such a
calculus follows the style of giving generators and relations on
them. The grammar, below, describing term constructors, freely
generates the set of processes, $\Proc$. This set is then quotiented
by a relation known as structural congruence and it is over this set
that the notion of dynamics is expressed. This presentation is
essentially that of \cite{MeredithR05} with the addition of
polyadicity and summation. For readability we have relegated some of
the technical subtleties to an appendix.

\subsubsection{Process grammar}\label{subsub:process_grammar}

\begin{mathpar}
  \inferrule* [lab=synchronization] {} {{M} \bc \pzero \;|\; x?F \;|\; x!C }
  \and
  \inferrule* [lab=abstraction] {} {{F} \bc (x)P}
  \and
  \inferrule* [lab=concretion] {} {{C} \bc \langle Q \rangle}
  \and
  \inferrule* [lab=process] {} {{P,Q} \bc M \;| \;P|Q \;|\; @{x}}
  \and
  \inferrule* [lab=name] {} {{x} \bc \quotep{P}}
\end{mathpar} 

Note that $\vec{x}$ (resp. $\vec{P}$) denotes a vector of names
(resp. processes) of length $|\vec{x}|$ (resp. $|\vec{P}|$). We adopt
the following useful abbreviations.

\begin{mathpar}
   x?(\vec{y}).P := x.(\vec{y})P \and  x\clift{\vec{P}} := x.\clift{\vec{P}}
   \and x!(y) := \lift{x}{\dropn{y}}
   \and \Pi_{i=0}^{n-1}P_i := P_0 | \ldots | P_{n-1}
\end{mathpar}

\subsubsection{Structural congruence}

\paragraph{Free and bound names and alpha-equivalence.} At the
core of structural equivalence is alpha-equivalence which identifies
process that are the same up to a change of variable. Formally, we
recognize the distinction between free and bound names. The free names
of a process, $\freenames{P}$, may be calculated recursively as
follows:

\begin{mathpar}
\freenames{\pzero} := \emptyset
  \and \\
  \freenames{x?(y).P} := \{ x \} \cup (\freenames{P} \setminus \{ y \})
  \and 
  \freenames{x!\langle P \rangle} := \{ x \} \cup \{ P \} 
  \and \\
  \freenames{P|Q} := \freenames{P} \cup \freenames{Q}
  \and \\
  \freenames{@{x}} := \{ x \}
\end{mathpar}

$\pi$
$\quotep{\pi}$

$\freenames{-} : \pi \to \mathcal{P}(\quotep{\pi})$

\begin{eqnarray*}
  \freenames{\pzero} & := & \emptyset \\
  \freenames{x?(y).P} & := & \{ x \} \cup (\freenames{P} \setminus \{ y \}) \\
  \freenames{x!\langle P \rangle} & := & \{ x \} \cup \{ P \} \\
  \freenames{P|Q} & := & \freenames{P} \cup \freenames{Q} \\
  \freenames{\dropn{x}} & := & \{ x \}
\end{eqnarray*}

The bound names of a process, $\boundnames{P}$, are those names occurring in $P$
that are not free. For example, in $x?(y).0$, the name $x$ is free, while $y$ is bound.

\begin{mathpar}
  \inferrule* [lab=monoidal-laws] {} { P|Q \equiv Q|P \and P|0 \equiv P \and P|(Q|R) \equiv (P|Q)|R }
\end{mathpar}

\begin{mathpar}
  \inferrule* [lab=alpha-equivalence] {} { (x)P \equiv (y)P\{y/x\} \and y \not\in \freenames{P} }
\end{mathpar}

\begin{definition}
Then two processes, $P,Q$, are alpha-equivalent if $P = Q\{\vec{y}/\vec{x}\}$ for
some $\vec{x} \in \boundnames{Q},\vec{y} \in \boundnames{P}$, where $Q\{\vec{y}/\vec{x}\}$
denotes the capture-avoiding substitution of $\vec{y}$ for $\vec{x}$ in $Q$.
\end{definition}

\begin{definition}
  The {\em structural congruence} \cite{SangiorgiWalker} , $\equiv$,
  between processes is the least congruence containing
  alpha-equivalence, satisfying the abelian monoid laws
  (associativity, commutativity and $\pzero$ as identity) for parallel
  composition $|$ and for summation $+$.
\end{definition}

\subsection{Name equivalence}

We take name equivalence, written $\nameeq$, to be the smallest
equivalence relation generated by the following rules.

\begin{mathpar}
\inferrule*[lab=Quote-drop]
{ }
{ \quotep{@{x}} \nameeq x }

\inferrule*[lab=Struct-equiv]
{ P \scong Q }
{ \quotep{P} \nameeq \quotep{Q} }
\end{mathpar}

The astute reader will have noticed that the mutual recursion of names
and processes imposes a mutual recursion on alpha-equivalence and
structural equivalence via name-equivalence. Fortunately, all of this
works out pleasantly and we may calculate in the natural way, free of
concern. The reader interested in the details is referred to the
appendix \ref{appendix:rho_details}.

\subsection{Substitution}

We use $\Proc$ for the set of processes, $\QProc$ for the set of
names, and $\id{\{}\vec{y} / \vec{x} \id{\}}$ to denote partial maps,
$s : \QProc \rightarrow \QProc$. A map, $s$ lifts, uniquely, to a map
on process terms, $\widehat{s} : \Proc \rightarrow \Proc$ by the
following equations.

\begin{mathpar}
  (0) \psubstp{Q}{P} := 0 \\
  (R \juxtap S) \psubstp{Q}{P}
  :=    
  (R)\psubstp{Q}{P} \juxtap (S) \psubstp{Q}{P} \\
  (x?(y).R) \psubstp{Q}{P}    
  :=    
  (x)\substp{Q}{P} (z)\concat( (R \psubstn{z}{y}) \psubstp{Q}{P} ) \\
  (\lift{x}{R}) \psubstp{Q}{P}  
  :=
  \lift{(x)\substp{Q}{P}}{ R \psubstp{Q}{P} } \\
%   (\dropn{x})  \psubstp{Q}{P}       
%   := 
%   \left\{ 
%     \begin{array}{ccc} 
%       \dropn{\quotep{Q}} & & x \nameeq \quotep{P} \\
%       \dropn{x} & & otherwise \\
%     \end{array}
%   \right. 
  (\dropn{x})  \psubstp{Q}{P}       
  := 
  \left\{ 
    \begin{array}{ccc} 
      Q & & x \nameeq \quotep{P} \\
      \dropn{x} & & otherwise \\
    \end{array}
  \right.
\end{mathpar}
 

where

\begin{eqnarray}
  (x)\id{\{} \lpquote Q \rpquote / \lpquote P \rpquote \id{\}}            = 
  \left\{ 
    \begin{array}{ccc}
      \lpquote Q \rpquote & & x \nameeq \lpquote P \rpquote \\
      x & & otherwise \\
    \end{array}
  \right. \nonumber
\end{eqnarray}

and $z$ is chosen distinct from $\quotep{P}$, $\quotep{Q}$, the free
names in $Q$, and all the names in $R$. Our $\alpha$-equivalence will
be built in the standard way from this substitution.

\begin{remark}\label{rem:no_self_referential_names}
  One consequence of these definitions is that $\forall P. \quotep{P}
  \not\in \freenames{P}$.
\end{remark}

\subsection{ Dynamic quote: an example }

Anticipating something of what's to come, consider applying the
substitution, $\widehat{\id{\{}u / z \id{\}}}$, to the following pair
of processes, $\lift{w}{y!(z)}$ and $w[ \lpquote y!(z) \rpquote ]$.

\begin{eqnarray}
	\lift{w}{y!(z)}\widehat{\id{\{}u / z \id{\}}}
		& = &
		\lift{w}{y!(u)} \nonumber\\
	w[ \lpquote y!(z) \rpquote ] \widehat{ \id{\{}u / z \id{\}} }
		& = &
		w[ \lpquote y!(z) \rpquote ] \nonumber
\end{eqnarray}

Because the body of the process between quotes is impervious to
substitution, we get radically different answers. In fact, by
examining the first process in an input context,
e.g. $x?(z).\lift{w}{y!(z)}$, we see that the process under the lift
operator may be shaped by prefixed inputs binding a name inside it. In
this sense, the lift operator will be seen as a way to dynamically
construct processes before reifying them as names.

Finally equipped with these standard features we can present the
dynamics of the calculus.

\subsubsection{Operational semantics} 

Finally, we introduce the computational dynamics. What marks these
algebras as distinct from other more traditionally studied algebraic
structures, e.g. vector spaces or polynomial rings, is the manner in
which dynamics is captured. In traditional structures, dynamics is typically
expressed through morphisms between such structures, as in linear maps
between vector spaces or morphisms between rings. In algebras
associated with the semantics of computation, the dynamics is
expressed as part of the algebraic structure itself, through a
reduction reduction relation typically denoted by $\red$. Below, we
give a recursive presentation of this relation for the calculus used
in the encoding.

$\red \subseteq \pi \times \pi$
$\red : \pi \to \mathcal{P}(\pi)$

\begin{mathpar}
  \inferrule* [lab=Comm] { \textsf{match}( x_{src}, x_{trgt} ) } { x_{trgt}?(y)P \; | \; x_{src}!\langle {Q} \rangle \red P\{\quotep{Q}/y}\} }
  \and \\
  \inferrule* [lab=Par] {{P} \red {P}'} {{{P} | {Q}} \red {{P}' | {Q}}}
  \and
  \inferrule* [lab=Equiv]{{{P} \scong {P}'} \andalso {{P}' \red {Q}'} \andalso {{Q}' \scong {Q}}}{{P} \red {Q}}
\end{mathpar}

\begin{eqnarray*}
  match_{\equiv} (\quotep{P},\quotep{Q}) & := & P \equiv Q \\
  match_{\dagger}(\quotep{P},\quotep{Q}) & := & \forall R. P|Q \red^{*} R => R \red^{*} 0 \\
  match_{K}(\quotep{P},\quotep{Q}) & := & K \mbox{ for some context } K
\end{eqnarray*}

$u?(x)P | u!\langle Q \rangle \red P\{\quotep{Q}/x\}$

%We write $\wred$ for $\red^*$, and $P\red$ if $\exists Q $ such that $ P \red Q$.
We write $P\red$ if $\exists Q $ such that $ P \red Q$ and $P\not\red$, otherwise.

\section{Replication}

As mentioned before, it is known that replication (and hence
recursion) can be implemented in a higher-order process algebra
\cite{SangiorgiWalker}. As our first example of calculation with the
machinery thus far presented we give the construction explicitly in
the {\rhoc}.

\begin{eqnarray}
	D_{x} & := & \prefix{x}{y}{(\binpar{\outputp{x}{y}}{@{y}})} \nonumber\\
	\bangp_{x}{P} & := & \binpar{{x}!\langle{\binpar{D_{x}}{P}}\rangle}{D_{x}} \nonumber
\end{eqnarray}

\begin{eqnarray}
	\bangp_{x}{P} & & \nonumber\\
	=
	& {x}!\langle{(\prefix{x}{y}{(\outputp{x}{y} | @{y})) | P}}\rangle 
	      | \prefix{x}{y}{(\outputp{x}{y} | @{y})} & \nonumber\\
	\red
	& (\outputp{x}{y} | @{y})\substn{\quotep{(\prefix{x}{y}{(@{y} | \outputp{x}{y})) | P}}}{y} & \nonumber\\
	=
	& \outputp{x}{\quotep{(\prefix{x}{y}{(\outputp{x}{y} | @{y})) | P}}}
	  | {(\prefix{x}{y}{(\outputp{x}{y} | @{y})) | P}} & \nonumber\\
	\red
	& \ldots & \nonumber\\
	\red^*
	& P | P | \ldots & \nonumber
\end{eqnarray}

Of course, this encoding, as an implementation, runs away, unfolding
$\bangp{P}$ eagerly. A lazier and more implementable replication
operator, restricted to input-guarded processes, may be obtained as follows.

\begin{eqnarray}
\bangp{\prefix{u}{v}{P}} 
	:= 
	\binpar{\lift{x}{\prefix{u}{v}{(\binpar{D(x)}{P})}}}{D(x)} \nonumber
\end{eqnarray}

\begin{remark}
  Note that the lazier definition still does not deal with summation
  or mixed summation (i.e. sums over input and output). The reader is
  invited to construct definitions of replication that deal with these
  features. 

  Further, the definitions are parameterized in a name, $x$. Can you,
  gentle reader, make a definition that eliminates this parameter and
  guarantees no accidental interaction between the replication
  machinery and the process being replicated -- i.e. no accidental
  sharing of names used by the process to get its work done and the
  name(s) used by the replication to effect copying. This latter
  revision of the definition of replication is crucial to obtaining
  the expected identity $!!P \sim !P$.
\end{remark}

\begin{remark}\label{rem:paradoxical_combinator}
  The reader familiar with the lambda calculus will have noticed the
  similarity between $D$ and the paradoxical combinator.

  [Ed. note: the existence of this seems to suggest we have to be more
  restrictive on the set of processes and names we admit if we are to
  support no-cloning.]
\end{remark}

\subsubsection{Bisimulation}

The computational dynamics gives rise to another kind of equivalence,
the equivalence of computational behavior. As previously mentioned
this is typically captured \emph{via} some form of bisimulation.

% The notion we use in this paper is weak barbed bisimulation
% \cite{milner91polyadicpi}.

The notion we use in this paper is derived from weak barbed
bisimulation \cite{milner91polyadicpi}. 

\begin{definition}
An \emph{observation relation}, $\downarrow_{\mathcal N}$, over a set
of names, $\mathcal N$, is the smallest relation satisfying the rules
below.

\infrule[Out-barb]{y \in {\mathcal N}, \; x \nameeq y}
		  {\outputp{x}{v} \downarrow_{\mathcal N} x}
\infrule[Par-barb]{\mbox{$P\downarrow_{\mathcal N} x$ or $Q\downarrow_{\mathcal N} x$}}
		  {\binpar{P}{Q} \downarrow_{\mathcal N} x}

We write $P \Downarrow_{\mathcal N} x$ if there is $Q$ such that 
$P \wred Q$ and $Q \downarrow_{\mathcal N} x$.
\end{definition}

\begin{definition}
%\label{def.bbisim}
An  ${\mathcal N}$-\emph{barbed bisimulation} over a set of names, ${\mathcal N}$, is a symmetric binary relation 
${\mathcal S}_{\mathcal N}$ between agents such that $P\rel{S}_{\mathcal N}Q$ implies:
\begin{enumerate}
\item If $P \red P'$ then $Q \wred Q'$ and $P'\rel{S}_{\mathcal N} Q'$.
\item If $P\downarrow_{\mathcal N} x$, then $Q\Downarrow_{\mathcal N} x$.
\end{enumerate}
$P$ is ${\mathcal N}$-barbed bisimilar to $Q$, written
$P \wbbisim_{\mathcal N} Q$, if $P \rel{S}_{\mathcal N} Q$ for some ${\mathcal N}$-barbed bisimulation ${\mathcal S}_{\mathcal N}$.
\end{definition}

$\mathcal{R} \subseteq \pi \times \pi$

$P \mathcal{R} Q => \forall P'. P \red P' \Rightarrow \exists Q'. Q \red Q', P' \mathcal{R} Q'$

$P \vdash x \Rightarrow Q \vdash x$

\begin{mathpar}
  \inferrule*[lab=Out-barb]{x \nameeq y}{{y}!\langle{Q}\rangle \vdash x}
  \and
  \inferrule*[lab=Par-barb]{\mbox{$P\vdash x$ or $Q\vdash x$}}{\binpar{P}{Q} \vdash x}
\end{mathpar}

\subsubsection{Contexts}

One of the principle advantages of computational calculi like the
$\pi$-calculus is a well-defined notion of context,
contextual-equivalence and a correlation between
contextual-equivalence and notions of bisimulation. The notion of
context allows the decomposition of a process into (sub-)process and
its syntactic environment, its context. Thus, a context may be
thought of as a process with a ``hole'' (written $\Box$) in it. The
application of a context $M$ to a process $P$, written $M[P]$, is
tantamount to filling the hole in $M$ with $P$. In this paper we do
not need the full weight of this theory, but do make use of the notion
of context in the proof the main theorem. 

\begin{mathpar}
  \inferrule* [lab=summation] {} {{M_{M},M_{N}} \bc \Box \;|\; x.M_{A} \;|\; M_{M}+M_{N}}
  \and
  \inferrule* [lab=agent] {} {{M_{A}} \bc (\vec{x})M_{P} \;| \; \clift{P_0,\ldots,M_{P},\ldots,P_N}}
  \and \\
  \inferrule* [lab=process] {} {{M_{P}} \bc M_{N} \;| \;P|M_{P} }
\end{mathpar} 

\begin{mathpar}
  \inferrule* [lab=sychronization] {} {M_{N} \bc \Box \;|\; x?M_{F} \;|\; x!M_{C}}
  \and
  \inferrule* [lab=abstraction] {} {{M_{F}} \bc (x)M_{P} }
  \and
  \inferrule* [lab=concretion] {} {{M_{C}} \bc \langle M_{P} \rangle }
  \and \\
  \inferrule* [lab=process] {} {{M_{P}} \bc M_{N} \;| \;P|M_{P} }
\end{mathpar}

\begin{definition}[contextual application] Given a context $M$, and
  process $P$, we define the \emph{contextual application}, $M[P] :=
  M\{P/\Box\}$. That is, the contextual application of M to P is the
  substitution of $P$ for $\Box$ in $M$.
\end{definition}

$\meaningof{-} : L \to \mathcal{P}(\pi)$

\begin{mathpar}
  \inferrule* [lab=collection] {} {\meaningof{true} = \pi, \and \meaningof{~E} = \pi \setminus \meaningof{E}, \and \meaningof{E_{1} \& E_{2}} = \meaningof{E_{1}} \cap \meaningof{E_{2}}}
\end{mathpar}

\begin{mathpar}
  \inferrule* [lab=structure] {} {\meaningof{0} = \{ P \in \pi | P \equiv 0 \}, \and \\ \meaningof{E_1 | E_2} = \{ P \in \pi | P \equiv P_{1} | P_{2}, P_{1} \in \meaningof{E_{1}}, P_{2} \in \meaningof{E_2}\} }
\end{mathpar}

\begin{mathpar}
 \inferrule* [lab=behavior] {} {\meaningof{\langle a?b \rangle E} = \{ P \in \pi | P \equiv Q | u?(y)P', \\ \and \\\\ \and \\ \;\;\; u \in \meaningof{a}, \forall z.P'\{z/y\} \in \meaningof{E\{z/b\}}\}, \and \\ \meaningof{a!E} = \{ P \in \pi | P \equiv Q | x!\langle P' \rangle, x \in \meaningof{a} P' \in \meaningof{E}\} }
\end{mathpar}

\begin{mathpar}
 \inferrule* [lab=nominal] {} {\meaningof{\quotep{E}} = \{ \quotep{P} \in \quotep{\pi} | P \in \meaningof{E} \}, \and \meaningof{\quotep{P}} = \{ \quotep{Q} \in \quotep{\pi} | P \equiv Q \} \and \\ \meaningof{@\quotep{E}} = \{ P \in \pi | P \equiv @x, x \in \meaningof{E} \}}
\end{mathpar}

\begin{eqnarray*}
  \\
  \meaningof{-} : TS \to ST
\end{eqnarray*}

\begin{eqnarray*}
  \\
  L : TS \to ST
\end{eqnarray*}

\begin{eqnarray*}
  \\
  P \models E \iff P \in \meaningof{E}
\end{eqnarray*}

\begin{eqnarray*}
  P \approx_{L} Q \iff \forall E \in L. P \models E \iff Q \models E
\end{eqnarray*}

\begin{eqnarray*}
  P \approx_{K} Q
\end{eqnarray*}

\begin{eqnarray*}
  P \approx Q
\end{eqnarray*}

$\approx_{K} = \approx = \approx_{L}$

\subsubsection{Contextual duality}

Note that contexts extend the quotation operation to a family of
operations from processes to names. Given a context, $M$, we can
define a \emph{nominal context}, $\quotep{M}$ by $\quotep{M}[P] :=
\quotep{M[P]}$. To foreshadow what is to come we observe that these
operations enjoy a duality with processes very much like the duality
between vectors and maps from vectors to scalars.

Further, because the calculus is essentially higher-order, we have a
correspondence between contexts and processes. More specifically,
given a name $x$ and a context $M$ we can construct $M^{*}_{x}$ such
that 

\begin{mathpar}
  M^{*}_{x} | \lift{x}{P} \red M[P]
\end{mathpar}

namely,

\begin{mathpar}
  M^{*}_{x} := x?(u).M[\dropn{u}]
\end{mathpar}

The dependence of $M^{*}_{x}$ on a name makes it an abstraction, 

\begin{mathpar}
  M^{*} := (x)x?(u).M[\dropn{u}]
\end{mathpar}

\subsection{Additional notation}

It will sometimes be convenient to denote the process a name
quotes. We already have the notation $x = \quotep{P}$, but it will be
convenient to introduce an alternate notation, $\procn{x}$, when we
want to emphasize the connection to the use of the name. Note that, by
virtue of name equivalence, $\quotep{\procn{x}} \nameeq x$; so, the
notation is consistent with previous definitions.

Further, because names have structure it is possible to effect
substitutions on the basis of that structure. This means we need to
upgrade our notation for substitutions, which we accomplish by
adapting comprehension notation. Thus,

\begin{mathpar}
  P\{ y / x : x \in S \}
\end{mathpar}

is interpreted to mean the process derived from P by replacing (in a
capture-avoiding manner) each occurrence of $x$ in $S$ by $y$. For example,

\begin{mathpar}
  P\{ \quotep{\procn{x}|\procn{x}} / x : x \in \freenames{P} \}
\end{mathpar}

will replace each (occurrence) of a free name $x$ in $P$ by
$\quotep{\procn{x}|\procn{x}}$.

Also, we will avail ourselves of the notation $x^{L}$ and $x^{R}$ to
denote injections of a name into disjoint copies of the name
space. There are numerous ways to accomplish this. One example can be
found in \cite{MeredithR05}. This notation overloads to vectors of
names: $\vec{x}^{\pi} := (x_{i}^{\pi} \; : \; 0 \leq i < |\vec{x}| )$ where $\pi \in \{L,R\}$.

We also use $P^{\Box} := P|\Box$.

In \cite{MeredithR05} an interpretation of the new operator is
given. It turns out that there are several possible interpretations
all enjoying the requisite algebraic properties of the operator (see
\cite{milner91polyadicpi}). We will therefore make liberal use of
$(\nu\; \vec{x})P$.

% subsection the_syntax_and_semantics_of_the_notation_system (end)   

\input{qm2pi.qmops} 

\input{qm2pi.sterngerlach} 

\input{qm2pi.metric} 

% section concurrent_process_calculi (end)

%\input{qm2pi.proofsketch}

% section proof sketch (end)

%\input{qm2pi.slviaknots} 

% section spatial logic via knots (end)

\input{qm2pi.conclusion}

% section conclusion (end)

%\input{qm2pi.dtcodes} 

% section wiring algorithm (end)

\input{qm2pi.ack} 

% section acknowledgments (end)

\newpage


\bibliographystyle{plain}   
\bibliography{../../biblios/main.bib}

\input{qm2pi.rhodetails}

\end{document}

 

% subsection basic_interpretation (end)

%\input{qm2pi.rho.presentation} 
\subsection{The syntax and semantics of the notation system}\label{sub:the_syntax_and_semantics_of_the_notation_system} % (fold)

We now summarize a technical presentation of the calculus that
embodies our theory of dynamics. The typical presentation of such a
calculus follows the style of giving generators and relations on
them. The grammar, below, describing term constructors, freely
generates the set of processes, $\Proc$. This set is then quotiented
by a relation known as structural congruence and it is over this set
that the notion of dynamics is expressed. This presentation is
essentially that of \cite{MeredithR05} with the addition of
polyadicity and summation. For readability we have relegated some of
the technical subtleties to an appendix.

\subsubsection{Process grammar}\label{subsub:process_grammar}

\begin{mathpar}
  \inferrule* [lab=synchronization] {} {{M} \bc \pzero \;|\; x?F \;|\; x!C }
  \and
  \inferrule* [lab=abstraction] {} {{F} \bc (x)P}
  \and
  \inferrule* [lab=concretion] {} {{C} \bc \langle Q \rangle}
  \and
  \inferrule* [lab=process] {} {{P,Q} \bc M \;| \;P|Q \;|\; @{x}}
  \and
  \inferrule* [lab=name] {} {{x} \bc \quotep{P}}
\end{mathpar} 

Note that $\vec{x}$ (resp. $\vec{P}$) denotes a vector of names
(resp. processes) of length $|\vec{x}|$ (resp. $|\vec{P}|$). We adopt
the following useful abbreviations.

\begin{mathpar}
   x?(\vec{y}).P := x.(\vec{y})P \and  x\clift{\vec{P}} := x.\clift{\vec{P}}
   \and x!(y) := \lift{x}{\dropn{y}}
   \and \Pi_{i=0}^{n-1}P_i := P_0 | \ldots | P_{n-1}
\end{mathpar}

\subsubsection{Structural congruence}

\paragraph{Free and bound names and alpha-equivalence.} At the
core of structural equivalence is alpha-equivalence which identifies
process that are the same up to a change of variable. Formally, we
recognize the distinction between free and bound names. The free names
of a process, $\freenames{P}$, may be calculated recursively as
follows:

\begin{mathpar}
\freenames{\pzero} := \emptyset
  \and \\
  \freenames{x?(y).P} := \{ x \} \cup (\freenames{P} \setminus \{ y \})
  \and 
  \freenames{x!\langle P \rangle} := \{ x \} \cup \{ P \} 
  \and \\
  \freenames{P|Q} := \freenames{P} \cup \freenames{Q}
  \and \\
  \freenames{@{x}} := \{ x \}
\end{mathpar}

$\pi$
$\quotep{\pi}$

$\freenames{-} : \pi \to \mathcal{P}(\quotep{\pi})$

\begin{eqnarray*}
  \freenames{\pzero} & := & \emptyset \\
  \freenames{x?(y).P} & := & \{ x \} \cup (\freenames{P} \setminus \{ y \}) \\
  \freenames{x!\langle P \rangle} & := & \{ x \} \cup \{ P \} \\
  \freenames{P|Q} & := & \freenames{P} \cup \freenames{Q} \\
  \freenames{\dropn{x}} & := & \{ x \}
\end{eqnarray*}

The bound names of a process, $\boundnames{P}$, are those names occurring in $P$
that are not free. For example, in $x?(y).0$, the name $x$ is free, while $y$ is bound.

\begin{mathpar}
  \inferrule* [lab=monoidal-laws] {} { P|Q \equiv Q|P \and P|0 \equiv P \and P|(Q|R) \equiv (P|Q)|R }
\end{mathpar}

\begin{mathpar}
  \inferrule* [lab=alpha-equivalence] {} { (x)P \equiv (y)P\{y/x\} \and y \not\in \freenames{P} }
\end{mathpar}

\begin{definition}
Then two processes, $P,Q$, are alpha-equivalent if $P = Q\{\vec{y}/\vec{x}\}$ for
some $\vec{x} \in \boundnames{Q},\vec{y} \in \boundnames{P}$, where $Q\{\vec{y}/\vec{x}\}$
denotes the capture-avoiding substitution of $\vec{y}$ for $\vec{x}$ in $Q$.
\end{definition}

\begin{definition}
  The {\em structural congruence} \cite{SangiorgiWalker} , $\equiv$,
  between processes is the least congruence containing
  alpha-equivalence, satisfying the abelian monoid laws
  (associativity, commutativity and $\pzero$ as identity) for parallel
  composition $|$ and for summation $+$.
\end{definition}

\subsection{Name equivalence}

We take name equivalence, written $\nameeq$, to be the smallest
equivalence relation generated by the following rules.

\begin{mathpar}
\inferrule*[lab=Quote-drop]
{ }
{ \quotep{@{x}} \nameeq x }

\inferrule*[lab=Struct-equiv]
{ P \scong Q }
{ \quotep{P} \nameeq \quotep{Q} }
\end{mathpar}

The astute reader will have noticed that the mutual recursion of names
and processes imposes a mutual recursion on alpha-equivalence and
structural equivalence via name-equivalence. Fortunately, all of this
works out pleasantly and we may calculate in the natural way, free of
concern. The reader interested in the details is referred to the
appendix \ref{appendix:rho_details}.

\subsection{Substitution}

We use $\Proc$ for the set of processes, $\QProc$ for the set of
names, and $\id{\{}\vec{y} / \vec{x} \id{\}}$ to denote partial maps,
$s : \QProc \rightarrow \QProc$. A map, $s$ lifts, uniquely, to a map
on process terms, $\widehat{s} : \Proc \rightarrow \Proc$ by the
following equations.

\begin{mathpar}
  (0) \psubstp{Q}{P} := 0 \\
  (R \juxtap S) \psubstp{Q}{P}
  :=    
  (R)\psubstp{Q}{P} \juxtap (S) \psubstp{Q}{P} \\
  (x?(y).R) \psubstp{Q}{P}    
  :=    
  (x)\substp{Q}{P} (z)\concat( (R \psubstn{z}{y}) \psubstp{Q}{P} ) \\
  (\lift{x}{R}) \psubstp{Q}{P}  
  :=
  \lift{(x)\substp{Q}{P}}{ R \psubstp{Q}{P} } \\
%   (\dropn{x})  \psubstp{Q}{P}       
%   := 
%   \left\{ 
%     \begin{array}{ccc} 
%       \dropn{\quotep{Q}} & & x \nameeq \quotep{P} \\
%       \dropn{x} & & otherwise \\
%     \end{array}
%   \right. 
  (\dropn{x})  \psubstp{Q}{P}       
  := 
  \left\{ 
    \begin{array}{ccc} 
      Q & & x \nameeq \quotep{P} \\
      \dropn{x} & & otherwise \\
    \end{array}
  \right.
\end{mathpar}
 

where

\begin{eqnarray}
  (x)\id{\{} \lpquote Q \rpquote / \lpquote P \rpquote \id{\}}            = 
  \left\{ 
    \begin{array}{ccc}
      \lpquote Q \rpquote & & x \nameeq \lpquote P \rpquote \\
      x & & otherwise \\
    \end{array}
  \right. \nonumber
\end{eqnarray}

and $z$ is chosen distinct from $\quotep{P}$, $\quotep{Q}$, the free
names in $Q$, and all the names in $R$. Our $\alpha$-equivalence will
be built in the standard way from this substitution.

\begin{remark}\label{rem:no_self_referential_names}
  One consequence of these definitions is that $\forall P. \quotep{P}
  \not\in \freenames{P}$.
\end{remark}

\subsection{ Dynamic quote: an example }

Anticipating something of what's to come, consider applying the
substitution, $\widehat{\id{\{}u / z \id{\}}}$, to the following pair
of processes, $\lift{w}{y!(z)}$ and $w[ \lpquote y!(z) \rpquote ]$.

\begin{eqnarray}
	\lift{w}{y!(z)}\widehat{\id{\{}u / z \id{\}}}
		& = &
		\lift{w}{y!(u)} \nonumber\\
	w[ \lpquote y!(z) \rpquote ] \widehat{ \id{\{}u / z \id{\}} }
		& = &
		w[ \lpquote y!(z) \rpquote ] \nonumber
\end{eqnarray}

Because the body of the process between quotes is impervious to
substitution, we get radically different answers. In fact, by
examining the first process in an input context,
e.g. $x?(z).\lift{w}{y!(z)}$, we see that the process under the lift
operator may be shaped by prefixed inputs binding a name inside it. In
this sense, the lift operator will be seen as a way to dynamically
construct processes before reifying them as names.

Finally equipped with these standard features we can present the
dynamics of the calculus.

\subsubsection{Operational semantics} 

Finally, we introduce the computational dynamics. What marks these
algebras as distinct from other more traditionally studied algebraic
structures, e.g. vector spaces or polynomial rings, is the manner in
which dynamics is captured. In traditional structures, dynamics is typically
expressed through morphisms between such structures, as in linear maps
between vector spaces or morphisms between rings. In algebras
associated with the semantics of computation, the dynamics is
expressed as part of the algebraic structure itself, through a
reduction reduction relation typically denoted by $\red$. Below, we
give a recursive presentation of this relation for the calculus used
in the encoding.

$\red \subseteq \pi \times \pi$
$\red : \pi \to \mathcal{P}(\pi)$

\begin{mathpar}
  \inferrule* [lab=Comm] { \textsf{match}( x_{src}, x_{trgt} ) } { x_{trgt}?(y)P \; | \; x_{src}!\langle {Q} \rangle \red P\{\quotep{Q}/y}\} }
  \and \\
  \inferrule* [lab=Par] {{P} \red {P}'} {{{P} | {Q}} \red {{P}' | {Q}}}
  \and
  \inferrule* [lab=Equiv]{{{P} \scong {P}'} \andalso {{P}' \red {Q}'} \andalso {{Q}' \scong {Q}}}{{P} \red {Q}}
\end{mathpar}

\begin{eqnarray*}
  match_{\equiv} (\quotep{P},\quotep{Q}) & := & P \equiv Q \\
  match_{\dagger}(\quotep{P},\quotep{Q}) & := & \forall R. P|Q \red^{*} R => R \red^{*} 0 \\
  match_{K}(\quotep{P},\quotep{Q}) & := & K \mbox{ for some context } K
\end{eqnarray*}

$u?(x)P | u!\langle Q \rangle \red P\{\quotep{Q}/x\}$

%We write $\wred$ for $\red^*$, and $P\red$ if $\exists Q $ such that $ P \red Q$.
We write $P\red$ if $\exists Q $ such that $ P \red Q$ and $P\not\red$, otherwise.

\section{Replication}

As mentioned before, it is known that replication (and hence
recursion) can be implemented in a higher-order process algebra
\cite{SangiorgiWalker}. As our first example of calculation with the
machinery thus far presented we give the construction explicitly in
the {\rhoc}.

\begin{eqnarray}
	D_{x} & := & \prefix{x}{y}{(\binpar{\outputp{x}{y}}{@{y}})} \nonumber\\
	\bangp_{x}{P} & := & \binpar{{x}!\langle{\binpar{D_{x}}{P}}\rangle}{D_{x}} \nonumber
\end{eqnarray}

\begin{eqnarray}
	\bangp_{x}{P} & & \nonumber\\
	=
	& {x}!\langle{(\prefix{x}{y}{(\outputp{x}{y} | @{y})) | P}}\rangle 
	      | \prefix{x}{y}{(\outputp{x}{y} | @{y})} & \nonumber\\
	\red
	& (\outputp{x}{y} | @{y})\substn{\quotep{(\prefix{x}{y}{(@{y} | \outputp{x}{y})) | P}}}{y} & \nonumber\\
	=
	& \outputp{x}{\quotep{(\prefix{x}{y}{(\outputp{x}{y} | @{y})) | P}}}
	  | {(\prefix{x}{y}{(\outputp{x}{y} | @{y})) | P}} & \nonumber\\
	\red
	& \ldots & \nonumber\\
	\red^*
	& P | P | \ldots & \nonumber
\end{eqnarray}

Of course, this encoding, as an implementation, runs away, unfolding
$\bangp{P}$ eagerly. A lazier and more implementable replication
operator, restricted to input-guarded processes, may be obtained as follows.

\begin{eqnarray}
\bangp{\prefix{u}{v}{P}} 
	:= 
	\binpar{\lift{x}{\prefix{u}{v}{(\binpar{D(x)}{P})}}}{D(x)} \nonumber
\end{eqnarray}

\begin{remark}
  Note that the lazier definition still does not deal with summation
  or mixed summation (i.e. sums over input and output). The reader is
  invited to construct definitions of replication that deal with these
  features. 

  Further, the definitions are parameterized in a name, $x$. Can you,
  gentle reader, make a definition that eliminates this parameter and
  guarantees no accidental interaction between the replication
  machinery and the process being replicated -- i.e. no accidental
  sharing of names used by the process to get its work done and the
  name(s) used by the replication to effect copying. This latter
  revision of the definition of replication is crucial to obtaining
  the expected identity $!!P \sim !P$.
\end{remark}

\begin{remark}\label{rem:paradoxical_combinator}
  The reader familiar with the lambda calculus will have noticed the
  similarity between $D$ and the paradoxical combinator.

  [Ed. note: the existence of this seems to suggest we have to be more
  restrictive on the set of processes and names we admit if we are to
  support no-cloning.]
\end{remark}

\subsubsection{Bisimulation}

The computational dynamics gives rise to another kind of equivalence,
the equivalence of computational behavior. As previously mentioned
this is typically captured \emph{via} some form of bisimulation.

% The notion we use in this paper is weak barbed bisimulation
% \cite{milner91polyadicpi}.

The notion we use in this paper is derived from weak barbed
bisimulation \cite{milner91polyadicpi}. 

\begin{definition}
An \emph{observation relation}, $\downarrow_{\mathcal N}$, over a set
of names, $\mathcal N$, is the smallest relation satisfying the rules
below.

\infrule[Out-barb]{y \in {\mathcal N}, \; x \nameeq y}
		  {\outputp{x}{v} \downarrow_{\mathcal N} x}
\infrule[Par-barb]{\mbox{$P\downarrow_{\mathcal N} x$ or $Q\downarrow_{\mathcal N} x$}}
		  {\binpar{P}{Q} \downarrow_{\mathcal N} x}

We write $P \Downarrow_{\mathcal N} x$ if there is $Q$ such that 
$P \wred Q$ and $Q \downarrow_{\mathcal N} x$.
\end{definition}

\begin{definition}
%\label{def.bbisim}
An  ${\mathcal N}$-\emph{barbed bisimulation} over a set of names, ${\mathcal N}$, is a symmetric binary relation 
${\mathcal S}_{\mathcal N}$ between agents such that $P\rel{S}_{\mathcal N}Q$ implies:
\begin{enumerate}
\item If $P \red P'$ then $Q \wred Q'$ and $P'\rel{S}_{\mathcal N} Q'$.
\item If $P\downarrow_{\mathcal N} x$, then $Q\Downarrow_{\mathcal N} x$.
\end{enumerate}
$P$ is ${\mathcal N}$-barbed bisimilar to $Q$, written
$P \wbbisim_{\mathcal N} Q$, if $P \rel{S}_{\mathcal N} Q$ for some ${\mathcal N}$-barbed bisimulation ${\mathcal S}_{\mathcal N}$.
\end{definition}

$\mathcal{R} \subseteq \pi \times \pi$

$P \mathcal{R} Q => \forall P'. P \red P' \Rightarrow \exists Q'. Q \red Q', P' \mathcal{R} Q'$

$P \vdash x \Rightarrow Q \vdash x$

\begin{mathpar}
  \inferrule*[lab=Out-barb]{x \nameeq y}{{y}!\langle{Q}\rangle \vdash x}
  \and
  \inferrule*[lab=Par-barb]{\mbox{$P\vdash x$ or $Q\vdash x$}}{\binpar{P}{Q} \vdash x}
\end{mathpar}

\subsubsection{Contexts}

One of the principle advantages of computational calculi like the
$\pi$-calculus is a well-defined notion of context,
contextual-equivalence and a correlation between
contextual-equivalence and notions of bisimulation. The notion of
context allows the decomposition of a process into (sub-)process and
its syntactic environment, its context. Thus, a context may be
thought of as a process with a ``hole'' (written $\Box$) in it. The
application of a context $M$ to a process $P$, written $M[P]$, is
tantamount to filling the hole in $M$ with $P$. In this paper we do
not need the full weight of this theory, but do make use of the notion
of context in the proof the main theorem. 

\begin{mathpar}
  \inferrule* [lab=summation] {} {{M_{M},M_{N}} \bc \Box \;|\; x.M_{A} \;|\; M_{M}+M_{N}}
  \and
  \inferrule* [lab=agent] {} {{M_{A}} \bc (\vec{x})M_{P} \;| \; \clift{P_0,\ldots,M_{P},\ldots,P_N}}
  \and \\
  \inferrule* [lab=process] {} {{M_{P}} \bc M_{N} \;| \;P|M_{P} }
\end{mathpar} 

\begin{mathpar}
  \inferrule* [lab=sychronization] {} {M_{N} \bc \Box \;|\; x?M_{F} \;|\; x!M_{C}}
  \and
  \inferrule* [lab=abstraction] {} {{M_{F}} \bc (x)M_{P} }
  \and
  \inferrule* [lab=concretion] {} {{M_{C}} \bc \langle M_{P} \rangle }
  \and \\
  \inferrule* [lab=process] {} {{M_{P}} \bc M_{N} \;| \;P|M_{P} }
\end{mathpar}

\begin{definition}[contextual application] Given a context $M$, and
  process $P$, we define the \emph{contextual application}, $M[P] :=
  M\{P/\Box\}$. That is, the contextual application of M to P is the
  substitution of $P$ for $\Box$ in $M$.
\end{definition}

$\meaningof{-} : L \to \mathcal{P}(\pi)$

\begin{mathpar}
  \inferrule* [lab=collection] {} {\meaningof{true} = \pi, \and \meaningof{~E} = \pi \setminus \meaningof{E}, \and \meaningof{E_{1} \& E_{2}} = \meaningof{E_{1}} \cap \meaningof{E_{2}}}
\end{mathpar}

\begin{mathpar}
  \inferrule* [lab=structure] {} {\meaningof{0} = \{ P \in \pi | P \equiv 0 \}, \and \\ \meaningof{E_1 | E_2} = \{ P \in \pi | P \equiv P_{1} | P_{2}, P_{1} \in \meaningof{E_{1}}, P_{2} \in \meaningof{E_2}\} }
\end{mathpar}

\begin{mathpar}
 \inferrule* [lab=behavior] {} {\meaningof{\langle a?b \rangle E} = \{ P \in \pi | P \equiv Q | u?(y)P', \\ \and \\\\ \and \\ \;\;\; u \in \meaningof{a}, \forall z.P'\{z/y\} \in \meaningof{E\{z/b\}}\}, \and \\ \meaningof{a!E} = \{ P \in \pi | P \equiv Q | x!\langle P' \rangle, x \in \meaningof{a} P' \in \meaningof{E}\} }
\end{mathpar}

\begin{mathpar}
 \inferrule* [lab=nominal] {} {\meaningof{\quotep{E}} = \{ \quotep{P} \in \quotep{\pi} | P \in \meaningof{E} \}, \and \meaningof{\quotep{P}} = \{ \quotep{Q} \in \quotep{\pi} | P \equiv Q \} \and \\ \meaningof{@\quotep{E}} = \{ P \in \pi | P \equiv @x, x \in \meaningof{E} \}}
\end{mathpar}

\begin{eqnarray*}
  \\
  \meaningof{-} : TS \to ST
\end{eqnarray*}

\begin{eqnarray*}
  \\
  L : TS \to ST
\end{eqnarray*}

\begin{eqnarray*}
  \\
  P \models E \iff P \in \meaningof{E}
\end{eqnarray*}

\begin{eqnarray*}
  P \approx_{L} Q \iff \forall E \in L. P \models E \iff Q \models E
\end{eqnarray*}

\begin{eqnarray*}
  P \approx_{K} Q
\end{eqnarray*}

\begin{eqnarray*}
  P \approx Q
\end{eqnarray*}

$\approx_{K} = \approx = \approx_{L}$

\subsubsection{Contextual duality}

Note that contexts extend the quotation operation to a family of
operations from processes to names. Given a context, $M$, we can
define a \emph{nominal context}, $\quotep{M}$ by $\quotep{M}[P] :=
\quotep{M[P]}$. To foreshadow what is to come we observe that these
operations enjoy a duality with processes very much like the duality
between vectors and maps from vectors to scalars.

Further, because the calculus is essentially higher-order, we have a
correspondence between contexts and processes. More specifically,
given a name $x$ and a context $M$ we can construct $M^{*}_{x}$ such
that 

\begin{mathpar}
  M^{*}_{x} | \lift{x}{P} \red M[P]
\end{mathpar}

namely,

\begin{mathpar}
  M^{*}_{x} := x?(u).M[\dropn{u}]
\end{mathpar}

The dependence of $M^{*}_{x}$ on a name makes it an abstraction, 

\begin{mathpar}
  M^{*} := (x)x?(u).M[\dropn{u}]
\end{mathpar}

\subsection{Additional notation}

It will sometimes be convenient to denote the process a name
quotes. We already have the notation $x = \quotep{P}$, but it will be
convenient to introduce an alternate notation, $\procn{x}$, when we
want to emphasize the connection to the use of the name. Note that, by
virtue of name equivalence, $\quotep{\procn{x}} \nameeq x$; so, the
notation is consistent with previous definitions.

Further, because names have structure it is possible to effect
substitutions on the basis of that structure. This means we need to
upgrade our notation for substitutions, which we accomplish by
adapting comprehension notation. Thus,

\begin{mathpar}
  P\{ y / x : x \in S \}
\end{mathpar}

is interpreted to mean the process derived from P by replacing (in a
capture-avoiding manner) each occurrence of $x$ in $S$ by $y$. For example,

\begin{mathpar}
  P\{ \quotep{\procn{x}|\procn{x}} / x : x \in \freenames{P} \}
\end{mathpar}

will replace each (occurrence) of a free name $x$ in $P$ by
$\quotep{\procn{x}|\procn{x}}$.

Also, we will avail ourselves of the notation $x^{L}$ and $x^{R}$ to
denote injections of a name into disjoint copies of the name
space. There are numerous ways to accomplish this. One example can be
found in \cite{MeredithR05}. This notation overloads to vectors of
names: $\vec{x}^{\pi} := (x_{i}^{\pi} \; : \; 0 \leq i < |\vec{x}| )$ where $\pi \in \{L,R\}$.

We also use $P^{\Box} := P|\Box$.

In \cite{MeredithR05} an interpretation of the new operator is
given. It turns out that there are several possible interpretations
all enjoying the requisite algebraic properties of the operator (see
\cite{milner91polyadicpi}). We will therefore make liberal use of
$(\nu\; \vec{x})P$.

% subsection the_syntax_and_semantics_of_the_notation_system (end)   

\section{Interpretation of QM}
\subsection{Supporting definitions}
\subsubsection{Multiplication}
\begin{mathpar}
  \quotep{Q} \cdot \quotep{R} := \quotep{Q|R}
  \and \\
  \quotep{Q} \cdot P := P\{ \quotep{Q|R} / \quotep{R} : \quotep{R} \in \freenames{P} \}
\end{mathpar}

\paragraph{Discussion}
The first line needs little explanation. The second line says that
each free name of the process is replaced with the multiplication of
that name by the scalar. Multiplication of a scalar (name) by a state
(process) results in a process all the names of which have been `moved
over' by parallel composition with the process the scalar
quotes. There is a subtlety that the bound names have to be
manipulated so that multiplied names aren't accidentally
captured. There are many ways to achieve this.

\begin{remark}\label{rem:multiplication_identities}
  The reader is invited to verify that for all $x,y,z \in \QProc$ and $P \in \Proc$
  \begin{mathpar}
    x \cdot \quotep{0} \equiv x 
    \and
    x \cdot y \equiv y \cdot x
    \and
    x \cdot (y \cdot z) \equiv (x \cdot y) \cdot z
    \and \\
    \quotep{0} \cdot P \equiv P
    \and \\
    x \cdot (y \cdot P) \equiv (x \cdot y) \cdot P
    \and \\
    x \cdot (P|Q) \equiv (x \cdot P) | (x \cdot Q)
    \and \\    
  \end{mathpar}
\end{remark}

\subsubsection{Tensor product}

We define a tensor product on processes by structural induction.

\paragraph{Tensor of sums} First note that all summations, including
$\pzero$ and sequence, can be written $\Sigma_{i} x_{i}.A_{i} +
\Sigma_{j} x_{j}.C_{j}$, where we have grouped input-guarded processes
together and output-guarded processes together.

Thus, we can define the tensor product of two summations, $N_{1}\otimes N_{2}$, where

\begin{mathpar}
  N_{1} := \Sigma_{i} x_{i}.A_{i} + \Sigma_{j} x_{j}.C_{j}
  \and
  N_{2} := \Sigma_{i'} y_{i'}.B_{i'} + \Sigma_{j'} y_{j'}.D_{j'} 
\end{mathpar}

as follows.

\begin{mathpar}
  \Sigma_{i} x_{i}.A_{i} + \Sigma_{j} x_{j}.C_{j} \otimes \Sigma_{i'}
  y_{i'}.B_{i'} + \Sigma_{j'} y_{j'}.D_{j'} 
  \and \\
  := \; \Sigma_{i} \Sigma_{i'} \quotep{\stackrel{\vee}{x_{i}}| \stackrel{\vee}{y_{i'}}}.(A_{i}\otimes B_{i'}) \; | \; \Sigma_{i'} \Sigma_{i} \quotep{\stackrel{\vee}{y_{i'}}|\stackrel{\vee}{x_{i}}}.(B_{i'}\otimes A_{i})
  \and
  \;\; | \;\; \Sigma_{j} \Sigma_{j'} \quotep{\stackrel{\vee}{x_{j}}|\stackrel{\vee}{y_{j'}}}.(A_{j}\otimes B_{j'}) \; | \; \Sigma_{j'} \Sigma_{j} \quotep{\stackrel{\vee}{y_{j'}}|\stackrel{\vee}{x_{j}}}.(B_{j'}\otimes A_{j})
\end{mathpar}

\begin{remark}
  Do we need to $x^{L}$ and $y^{R}$ for this construction as well?
\end{remark}

\paragraph{Tensor of parallel compositions} Next, we distribute tensor
over par.

\begin{mathpar}
  P_{1}|P_{2} \otimes Q_{1}|Q_{2} := (P_{1} \otimes Q_{1}) | (P_{1}
  \otimes Q_{2}) | (P_{2} \otimes Q_{1}) | (P_{2} \otimes Q_{2})
\end{mathpar}

\paragraph{Tensor with dropped names} We treat tensor of a
process with a dropped name as parallel composition.

\begin{mathpar}
  P \otimes \dropn{x} := P | \dropn{x}
\end{mathpar}

\paragraph{Tensor of agents}

Finally, we need to define tensor on agents. Note that the definition
of tensor on normal products only tensors inputs with inputs and
outputs with outputs. Thus, we only have to define the operation on
``homogeneous'' pairings.

\begin{mathpar}
  (\vec{x})P \otimes (\vec{y})Q
  \and \\
  := (x_{0}^{L}|y_{0}^{R},\ldots,x_{0}^{L}|y_{n}^{R},\ldots,x_{m}^{L}|y_{0}^{R},\ldots,x_{m}^{L}|y_{n}^R)(P\{ \vec{x}^{L}/\vec{x}\} \otimes Q \{ \vec{y}^{R}/\vec{y}\})
  \and \\
  \clift{\vec{P}} \otimes \clift{\vec{Q}}
  \and \\
  := \clift{P_{0}\otimes Q_{0},\ldots,P_{0}\otimes Q_{n},\ldots,P_{m}\otimes Q_{0},\ldots,P_{m}\otimes Q_{n}}
\end{mathpar}

\begin{remark}
  Observe that arities of tensored abstractions matches arities of
  tensored concretions if the original arities matched. Note also that
  the length of the arities corresponds to the increase in dimension
  we see in ordinary vector space tensor product.
\end{remark}

\begin{remark}
  Operationally, this definition distributes the tensor down to
  components ``linked'' by summation. Tensor over summation is
  intriguing in that it mixes names. Moreover, as a consequence of the
  way it mixes names we have the identities for all $x \in \QProc$ and
  $P,Q \in \Proc$

  \begin{mathpar}
    (x \cdot P) \otimes Q \equiv x \cdot (P \otimes Q) \equiv P \otimes (x \cdot Q)
    \and
    P \otimes \pzero \equiv P
  \end{mathpar}

  that the reader is invited to verify.
\end{remark}

\subsubsection{Annihilation}
\begin{mathpar}
  P^{\perp} := \{ Q | \forall R. P|Q \red^{*} R \Rightarrow R \red^{*} \pzero \}
  \and \\
  P^{\underline{\perp}} := \Sigma_{Q \in P^{\perp}} \quotep{Q}?(y).(\dropn{y}|Q) | \Sigma_{Q \in P^{\perp}} \quotep{Q}\clift{\Box}
\end{mathpar}

\paragraph{Discussion} The reader will note that $P^{\perp}$ is a
\emph{set} of processes, while $P^{\underline{\perp}}$ is a
\emph{context}. We call the set $P^{\perp}$ the \emph{annihilators} of
$P$. The parallel composition of a process in the annihilators of $P$
with $P$ will result in a process, the state space of which has all
paths eventually leading to $\pzero$. Execution may endure loops; but
under reasonable conditions of fairness (naturally guaranteed under
most notions of bisimulation) such a composite process cannot get
stuck in such a loop and will, eventually pop out and terminate.

The context $P^{\underline{\perp}}$ is ready and willing to ``take the
$P$ out of'' the process to which it is applied. It will effectively
transmit the code of the process to which it is applied to one of the
annihilators and run the process against it.

\subsubsection{Evaluation}
We fix $M$ a domain of fully abstract interpretation with an equality
coincident with bisimulation. We take $\meaningof{\cdot} : \Proc \to
M$ to be the map interpreting processes and $\nmeaningof{\cdot} : \M
\to Proc$ to be the map running the other way. Then we define

\begin{mathpar}
  \int P := \nmeaningof{\meaningof{P}}
\end{mathpar}

\paragraph{Discussion}
There are many fully abstract interpretations of Milner's
$\pi$-calculus. Any of them can be used as a basis for interpreting
the reflective calculus here. Equipped with such a domain it is
largely a matter of grinding through to check that the Yoneda
construction for the normalization-by-evaluation program can be
extended to this setting.

\begin{remark}
  The reader is invited to verify that $\int (P^{\underline{\perp}}[P]) = 0$.
\end{remark}

\subsection{Quantum mechanics}

Table \ref{tbl:core_qm_op_defns} gives the core operational definitions

\begin{table}[htp]\label{tbl:core_qm_op_defns}
  \center{
    \fbox{
      \begin{tabular}{c|c}
        quantum mechanics & process calculus \\
        \hline
        scalar & $x := \quotep{P}$ \\
        state vector & $\state{P} := P$ \\
        dual & $\state{P}^{*} := \event{P^{\underline{\perp}}} := \quotep{P^{\underline{\perp}}}[-]$ \\
        matrix & $ \Sigma_{\alpha} \state{P_{\alpha}}x_{\alpha}\event{Q_{\alpha}}$ \\
        vector addition & $\state{P} + \state{Q} := \state{P | Q}$ \\
        tensor product & $\state{P} \otimes \state{Q} := \state{P \otimes Q}$ \\
        inner product & $\innerprod{P}{Q} := \quotep{\int P^{\underline{\perp}}[Q]}$ \\
      \end{tabular}
    }
  }
  \caption{QM - operational definitions}
\end{table}

where

\begin{mathpar}
  \prmatrix{P}{Q} := \fprmatrix{P}{\quotep{\pzero}}{Q}
  \and
  \fprmatrix{P}{x}{Q} := (\state{P},x,\event{Q})
  \and
  (\fprmatrix{P}{x}{Q})(\state{R}) := x \cdot \innerprod{Q}{R} \cdot \state{P}
  \and
  (\fprmatrix{P}{x}{Q})(\event{R}) := x \cdot \innerprod{R}{P} \cdot \event{Q}
\end{mathpar}

\paragraph{Discussion}
As promised: vectors (aka states) are represented as processes; duals
as contextual duals; inner product definition should be compared with
standard inner product definition for ....

\begin{remark}
  Assuming $\int (P^{\underline{\perp}}[P]) = 0$, the reader is
  invited to verify that $(\fprmatrix{P}{x}{P})(\state{P}) = x \cdot \state{P}$.
\end{remark}

\begin{remark}
  The reader is invited to verify that $\innerprod{P}{Q}$ could
  equally well have been written $\quotep{\int \stackrel{\vee}{x}}$
  where $x = \event{P^{\underline{\perp}}}(Q)$.

  One of the motivations for this remark is that there is another way
  to factor these operations. We could package up evaluation in the dual:

  \begin{mathpar}
    \state{P}^{*} := \event{\int P^{\underline{\perp}}} := \quotep{\int P^{\underline{\perp}}}[-]
  \end{mathpar}

  and then have inner product defined by
  
  \begin{mathpar}
    \innerprod{P}{Q} := \event{P}(Q)
  \end{mathpar}

  Hopefully, experience with the calculations will provide guidance on
  the best factoring.
\end{remark}

\begin{remark}
  Assuming $\int (P^{\underline{\perp}}[P]) = 0$, the reader is
  invited to verify that $\forall P,Q. (\prmatrix{0}{Q})(\state{0}) =
  \state{0}$ and dually $(\prmatrix{P}{0})(\event{0}) = \event{0}$.
\end{remark}

\begin{remark}
  i'm a little worried that i don't (yet) have proper support for
  complex conjugacy. But, the observation above may give us a
  clue. According to Abramsky, it must be the case that the scalars
  are iso to the homset of the identity for the tensor -- which the
  observation above characterizes. 

  For now, we will simply bookmark the notion with $\overline{x}$.
\end{remark}

\subsubsection{Adjointness}

We need to give a definition of $(\cdot)^{\dagger}$ for matrices. The
obvious candidate definition is
\begin{mathpar}
(\Sigma_{\alpha}\fprmatrix{P_{\alpha}}{x_{\alpha}}{Q_{\alpha}})^{\dagger}
= \Sigma_{\alpha}\fprmatrix{(Q_{\alpha}^{\underline{\perp}})^{*}}{\overline{x}_{\alpha}}{P_{\alpha}^{\underline{\perp}}} 
\end{mathpar}

But, $(Q_{\alpha}^{\underline{\perp}})^{*}$ requires a name along
which to communicate the process to achieve the context application.

\subsubsection{Basis for a basis}
If processes label states and ``addition'' of states (a.k.a. vector
addition) is interpreted as parallel composition, what corresponds to
notions of linear independence and basis? Here, we recall that Yoshida
has developed a set of \emph{combinators} for an asynchronous verison
of Milner's $\pi$-calculus. These are a finite set of processes such
any process can be expressed as parallel composition of these
combinators together with liberal uses of the new operator and
replication. We can simply give a translation of these into the
present calculus and have reasonable expectation that the property
carries over. That is, that the resultant set allows to express all
processes via parallel composition. Note, however, that there is no
new operator or replication in this calculus. As a result, we expect
that the corresponding set is actually infinite. That is, we expect
that the space is actually infinite dimensional.

\begin{remark}
  The attentive reader may be a bit concerned. Certainly, the
  collection $S$, $K$ and $I$ is a finite set of
  combinators. Shouldn't we expect to see a finite set of combinators
  for an effectively equivalent system? i am very sympathetic to this
  critique and feel it warrants full attention. On the other hand, i
  also have in mind the following analogy. The natural numbers, as a
  monoid under addition, has exactly $1$ generator, while the natural
  numbers, as a monoid under multiplication, has countably many
  generators (the primes). We observe that the application of the
  lambda calculus is much less resource sensitive than the parallel
  composition of the $\pi$-calculus. Could it be the case that we have
  an analogy of the form
  
  \begin{mathpar}
    m + n : MN :: m*n : M|N
  \end{mathpar}

  giving a similar blow up in the set of ``primes''?  This is such a
  wonderful thought that, even if it's not true, i think it's worth
  writing down.
\end{remark}
 

\documentclass[12pt]{llncs}
%\documentclass{jktr}

\usepackage[pdftex]{hyperref}                   
\usepackage {listings}
\usepackage {mathpartir}
\usepackage{bcprules}
%\usepackage{listings}
                       
\usepackage{graphicx} 
%\usepackage[margins=2.5cm,nohead,nofoot]{geometry}
%\usepackage{geometry}
\usepackage{amsfonts}
\usepackage{amstext}
\usepackage{latexsym}
\usepackage{amssymb}
\usepackage{color}


%\include{myPreamble}
\include{qm2pi.local} 

%\ifpdf
%\usepackage[pdftex]{graphicx}
%\else
%\usepackage{graphicx}
%\fi

 % \ifpdf
%  \usepackage{pdfsync}
%  \if


%\title{Brief Article}
%\author{David F. Snyder}
%\author{L.G. Meredith}

%\address{Dept. of Math., Texas State University--San Marcos, San Marcos, TX 78666}
       
\pagestyle{empty}


\begin{document}

\lstset{language=[Objective]Caml,frame=shadowbox}

\input{qm2pi.front}

% section front matter (end)

\input{qm2pi.intro} 
 
% section introduction (end)

% \input{qm2pi.knotations} 

% section notation (end)

\input{qm2pi.process.calculi} 

% section concurrent_process_calculi_and_spatial_logics_ (end)
    
%\input{qm2pi.knots2pi} 

%\input{qm2pi.trefoil} 

%\input{qm2pi.mainthm} 

% subsection basic_interpretation (end)

%\input{qm2pi.rho.presentation} 
\subsection{The syntax and semantics of the notation system}\label{sub:the_syntax_and_semantics_of_the_notation_system} % (fold)

We now summarize a technical presentation of the calculus that
embodies our theory of dynamics. The typical presentation of such a
calculus follows the style of giving generators and relations on
them. The grammar, below, describing term constructors, freely
generates the set of processes, $\Proc$. This set is then quotiented
by a relation known as structural congruence and it is over this set
that the notion of dynamics is expressed. This presentation is
essentially that of \cite{MeredithR05} with the addition of
polyadicity and summation. For readability we have relegated some of
the technical subtleties to an appendix.

\subsubsection{Process grammar}\label{subsub:process_grammar}

\begin{mathpar}
  \inferrule* [lab=synchronization] {} {{M} \bc \pzero \;|\; x?F \;|\; x!C }
  \and
  \inferrule* [lab=abstraction] {} {{F} \bc (x)P}
  \and
  \inferrule* [lab=concretion] {} {{C} \bc \langle Q \rangle}
  \and
  \inferrule* [lab=process] {} {{P,Q} \bc M \;| \;P|Q \;|\; @{x}}
  \and
  \inferrule* [lab=name] {} {{x} \bc \quotep{P}}
\end{mathpar} 

Note that $\vec{x}$ (resp. $\vec{P}$) denotes a vector of names
(resp. processes) of length $|\vec{x}|$ (resp. $|\vec{P}|$). We adopt
the following useful abbreviations.

\begin{mathpar}
   x?(\vec{y}).P := x.(\vec{y})P \and  x\clift{\vec{P}} := x.\clift{\vec{P}}
   \and x!(y) := \lift{x}{\dropn{y}}
   \and \Pi_{i=0}^{n-1}P_i := P_0 | \ldots | P_{n-1}
\end{mathpar}

\subsubsection{Structural congruence}

\paragraph{Free and bound names and alpha-equivalence.} At the
core of structural equivalence is alpha-equivalence which identifies
process that are the same up to a change of variable. Formally, we
recognize the distinction between free and bound names. The free names
of a process, $\freenames{P}$, may be calculated recursively as
follows:

\begin{mathpar}
\freenames{\pzero} := \emptyset
  \and \\
  \freenames{x?(y).P} := \{ x \} \cup (\freenames{P} \setminus \{ y \})
  \and 
  \freenames{x!\langle P \rangle} := \{ x \} \cup \{ P \} 
  \and \\
  \freenames{P|Q} := \freenames{P} \cup \freenames{Q}
  \and \\
  \freenames{@{x}} := \{ x \}
\end{mathpar}

$\pi$
$\quotep{\pi}$

$\freenames{-} : \pi \to \mathcal{P}(\quotep{\pi})$

\begin{eqnarray*}
  \freenames{\pzero} & := & \emptyset \\
  \freenames{x?(y).P} & := & \{ x \} \cup (\freenames{P} \setminus \{ y \}) \\
  \freenames{x!\langle P \rangle} & := & \{ x \} \cup \{ P \} \\
  \freenames{P|Q} & := & \freenames{P} \cup \freenames{Q} \\
  \freenames{\dropn{x}} & := & \{ x \}
\end{eqnarray*}

The bound names of a process, $\boundnames{P}$, are those names occurring in $P$
that are not free. For example, in $x?(y).0$, the name $x$ is free, while $y$ is bound.

\begin{mathpar}
  \inferrule* [lab=monoidal-laws] {} { P|Q \equiv Q|P \and P|0 \equiv P \and P|(Q|R) \equiv (P|Q)|R }
\end{mathpar}

\begin{mathpar}
  \inferrule* [lab=alpha-equivalence] {} { (x)P \equiv (y)P\{y/x\} \and y \not\in \freenames{P} }
\end{mathpar}

\begin{definition}
Then two processes, $P,Q$, are alpha-equivalent if $P = Q\{\vec{y}/\vec{x}\}$ for
some $\vec{x} \in \boundnames{Q},\vec{y} \in \boundnames{P}$, where $Q\{\vec{y}/\vec{x}\}$
denotes the capture-avoiding substitution of $\vec{y}$ for $\vec{x}$ in $Q$.
\end{definition}

\begin{definition}
  The {\em structural congruence} \cite{SangiorgiWalker} , $\equiv$,
  between processes is the least congruence containing
  alpha-equivalence, satisfying the abelian monoid laws
  (associativity, commutativity and $\pzero$ as identity) for parallel
  composition $|$ and for summation $+$.
\end{definition}

\subsection{Name equivalence}

We take name equivalence, written $\nameeq$, to be the smallest
equivalence relation generated by the following rules.

\begin{mathpar}
\inferrule*[lab=Quote-drop]
{ }
{ \quotep{@{x}} \nameeq x }

\inferrule*[lab=Struct-equiv]
{ P \scong Q }
{ \quotep{P} \nameeq \quotep{Q} }
\end{mathpar}

The astute reader will have noticed that the mutual recursion of names
and processes imposes a mutual recursion on alpha-equivalence and
structural equivalence via name-equivalence. Fortunately, all of this
works out pleasantly and we may calculate in the natural way, free of
concern. The reader interested in the details is referred to the
appendix \ref{appendix:rho_details}.

\subsection{Substitution}

We use $\Proc$ for the set of processes, $\QProc$ for the set of
names, and $\id{\{}\vec{y} / \vec{x} \id{\}}$ to denote partial maps,
$s : \QProc \rightarrow \QProc$. A map, $s$ lifts, uniquely, to a map
on process terms, $\widehat{s} : \Proc \rightarrow \Proc$ by the
following equations.

\begin{mathpar}
  (0) \psubstp{Q}{P} := 0 \\
  (R \juxtap S) \psubstp{Q}{P}
  :=    
  (R)\psubstp{Q}{P} \juxtap (S) \psubstp{Q}{P} \\
  (x?(y).R) \psubstp{Q}{P}    
  :=    
  (x)\substp{Q}{P} (z)\concat( (R \psubstn{z}{y}) \psubstp{Q}{P} ) \\
  (\lift{x}{R}) \psubstp{Q}{P}  
  :=
  \lift{(x)\substp{Q}{P}}{ R \psubstp{Q}{P} } \\
%   (\dropn{x})  \psubstp{Q}{P}       
%   := 
%   \left\{ 
%     \begin{array}{ccc} 
%       \dropn{\quotep{Q}} & & x \nameeq \quotep{P} \\
%       \dropn{x} & & otherwise \\
%     \end{array}
%   \right. 
  (\dropn{x})  \psubstp{Q}{P}       
  := 
  \left\{ 
    \begin{array}{ccc} 
      Q & & x \nameeq \quotep{P} \\
      \dropn{x} & & otherwise \\
    \end{array}
  \right.
\end{mathpar}
 

where

\begin{eqnarray}
  (x)\id{\{} \lpquote Q \rpquote / \lpquote P \rpquote \id{\}}            = 
  \left\{ 
    \begin{array}{ccc}
      \lpquote Q \rpquote & & x \nameeq \lpquote P \rpquote \\
      x & & otherwise \\
    \end{array}
  \right. \nonumber
\end{eqnarray}

and $z$ is chosen distinct from $\quotep{P}$, $\quotep{Q}$, the free
names in $Q$, and all the names in $R$. Our $\alpha$-equivalence will
be built in the standard way from this substitution.

\begin{remark}\label{rem:no_self_referential_names}
  One consequence of these definitions is that $\forall P. \quotep{P}
  \not\in \freenames{P}$.
\end{remark}

\subsection{ Dynamic quote: an example }

Anticipating something of what's to come, consider applying the
substitution, $\widehat{\id{\{}u / z \id{\}}}$, to the following pair
of processes, $\lift{w}{y!(z)}$ and $w[ \lpquote y!(z) \rpquote ]$.

\begin{eqnarray}
	\lift{w}{y!(z)}\widehat{\id{\{}u / z \id{\}}}
		& = &
		\lift{w}{y!(u)} \nonumber\\
	w[ \lpquote y!(z) \rpquote ] \widehat{ \id{\{}u / z \id{\}} }
		& = &
		w[ \lpquote y!(z) \rpquote ] \nonumber
\end{eqnarray}

Because the body of the process between quotes is impervious to
substitution, we get radically different answers. In fact, by
examining the first process in an input context,
e.g. $x?(z).\lift{w}{y!(z)}$, we see that the process under the lift
operator may be shaped by prefixed inputs binding a name inside it. In
this sense, the lift operator will be seen as a way to dynamically
construct processes before reifying them as names.

Finally equipped with these standard features we can present the
dynamics of the calculus.

\subsubsection{Operational semantics} 

Finally, we introduce the computational dynamics. What marks these
algebras as distinct from other more traditionally studied algebraic
structures, e.g. vector spaces or polynomial rings, is the manner in
which dynamics is captured. In traditional structures, dynamics is typically
expressed through morphisms between such structures, as in linear maps
between vector spaces or morphisms between rings. In algebras
associated with the semantics of computation, the dynamics is
expressed as part of the algebraic structure itself, through a
reduction reduction relation typically denoted by $\red$. Below, we
give a recursive presentation of this relation for the calculus used
in the encoding.

$\red \subseteq \pi \times \pi$
$\red : \pi \to \mathcal{P}(\pi)$

\begin{mathpar}
  \inferrule* [lab=Comm] { \textsf{match}( x_{src}, x_{trgt} ) } { x_{trgt}?(y)P \; | \; x_{src}!\langle {Q} \rangle \red P\{\quotep{Q}/y}\} }
  \and \\
  \inferrule* [lab=Par] {{P} \red {P}'} {{{P} | {Q}} \red {{P}' | {Q}}}
  \and
  \inferrule* [lab=Equiv]{{{P} \scong {P}'} \andalso {{P}' \red {Q}'} \andalso {{Q}' \scong {Q}}}{{P} \red {Q}}
\end{mathpar}

\begin{eqnarray*}
  match_{\equiv} (\quotep{P},\quotep{Q}) & := & P \equiv Q \\
  match_{\dagger}(\quotep{P},\quotep{Q}) & := & \forall R. P|Q \red^{*} R => R \red^{*} 0 \\
  match_{K}(\quotep{P},\quotep{Q}) & := & K \mbox{ for some context } K
\end{eqnarray*}

$u?(x)P | u!\langle Q \rangle \red P\{\quotep{Q}/x\}$

%We write $\wred$ for $\red^*$, and $P\red$ if $\exists Q $ such that $ P \red Q$.
We write $P\red$ if $\exists Q $ such that $ P \red Q$ and $P\not\red$, otherwise.

\section{Replication}

As mentioned before, it is known that replication (and hence
recursion) can be implemented in a higher-order process algebra
\cite{SangiorgiWalker}. As our first example of calculation with the
machinery thus far presented we give the construction explicitly in
the {\rhoc}.

\begin{eqnarray}
	D_{x} & := & \prefix{x}{y}{(\binpar{\outputp{x}{y}}{@{y}})} \nonumber\\
	\bangp_{x}{P} & := & \binpar{{x}!\langle{\binpar{D_{x}}{P}}\rangle}{D_{x}} \nonumber
\end{eqnarray}

\begin{eqnarray}
	\bangp_{x}{P} & & \nonumber\\
	=
	& {x}!\langle{(\prefix{x}{y}{(\outputp{x}{y} | @{y})) | P}}\rangle 
	      | \prefix{x}{y}{(\outputp{x}{y} | @{y})} & \nonumber\\
	\red
	& (\outputp{x}{y} | @{y})\substn{\quotep{(\prefix{x}{y}{(@{y} | \outputp{x}{y})) | P}}}{y} & \nonumber\\
	=
	& \outputp{x}{\quotep{(\prefix{x}{y}{(\outputp{x}{y} | @{y})) | P}}}
	  | {(\prefix{x}{y}{(\outputp{x}{y} | @{y})) | P}} & \nonumber\\
	\red
	& \ldots & \nonumber\\
	\red^*
	& P | P | \ldots & \nonumber
\end{eqnarray}

Of course, this encoding, as an implementation, runs away, unfolding
$\bangp{P}$ eagerly. A lazier and more implementable replication
operator, restricted to input-guarded processes, may be obtained as follows.

\begin{eqnarray}
\bangp{\prefix{u}{v}{P}} 
	:= 
	\binpar{\lift{x}{\prefix{u}{v}{(\binpar{D(x)}{P})}}}{D(x)} \nonumber
\end{eqnarray}

\begin{remark}
  Note that the lazier definition still does not deal with summation
  or mixed summation (i.e. sums over input and output). The reader is
  invited to construct definitions of replication that deal with these
  features. 

  Further, the definitions are parameterized in a name, $x$. Can you,
  gentle reader, make a definition that eliminates this parameter and
  guarantees no accidental interaction between the replication
  machinery and the process being replicated -- i.e. no accidental
  sharing of names used by the process to get its work done and the
  name(s) used by the replication to effect copying. This latter
  revision of the definition of replication is crucial to obtaining
  the expected identity $!!P \sim !P$.
\end{remark}

\begin{remark}\label{rem:paradoxical_combinator}
  The reader familiar with the lambda calculus will have noticed the
  similarity between $D$ and the paradoxical combinator.

  [Ed. note: the existence of this seems to suggest we have to be more
  restrictive on the set of processes and names we admit if we are to
  support no-cloning.]
\end{remark}

\subsubsection{Bisimulation}

The computational dynamics gives rise to another kind of equivalence,
the equivalence of computational behavior. As previously mentioned
this is typically captured \emph{via} some form of bisimulation.

% The notion we use in this paper is weak barbed bisimulation
% \cite{milner91polyadicpi}.

The notion we use in this paper is derived from weak barbed
bisimulation \cite{milner91polyadicpi}. 

\begin{definition}
An \emph{observation relation}, $\downarrow_{\mathcal N}$, over a set
of names, $\mathcal N$, is the smallest relation satisfying the rules
below.

\infrule[Out-barb]{y \in {\mathcal N}, \; x \nameeq y}
		  {\outputp{x}{v} \downarrow_{\mathcal N} x}
\infrule[Par-barb]{\mbox{$P\downarrow_{\mathcal N} x$ or $Q\downarrow_{\mathcal N} x$}}
		  {\binpar{P}{Q} \downarrow_{\mathcal N} x}

We write $P \Downarrow_{\mathcal N} x$ if there is $Q$ such that 
$P \wred Q$ and $Q \downarrow_{\mathcal N} x$.
\end{definition}

\begin{definition}
%\label{def.bbisim}
An  ${\mathcal N}$-\emph{barbed bisimulation} over a set of names, ${\mathcal N}$, is a symmetric binary relation 
${\mathcal S}_{\mathcal N}$ between agents such that $P\rel{S}_{\mathcal N}Q$ implies:
\begin{enumerate}
\item If $P \red P'$ then $Q \wred Q'$ and $P'\rel{S}_{\mathcal N} Q'$.
\item If $P\downarrow_{\mathcal N} x$, then $Q\Downarrow_{\mathcal N} x$.
\end{enumerate}
$P$ is ${\mathcal N}$-barbed bisimilar to $Q$, written
$P \wbbisim_{\mathcal N} Q$, if $P \rel{S}_{\mathcal N} Q$ for some ${\mathcal N}$-barbed bisimulation ${\mathcal S}_{\mathcal N}$.
\end{definition}

$\mathcal{R} \subseteq \pi \times \pi$

$P \mathcal{R} Q => \forall P'. P \red P' \Rightarrow \exists Q'. Q \red Q', P' \mathcal{R} Q'$

$P \vdash x \Rightarrow Q \vdash x$

\begin{mathpar}
  \inferrule*[lab=Out-barb]{x \nameeq y}{{y}!\langle{Q}\rangle \vdash x}
  \and
  \inferrule*[lab=Par-barb]{\mbox{$P\vdash x$ or $Q\vdash x$}}{\binpar{P}{Q} \vdash x}
\end{mathpar}

\subsubsection{Contexts}

One of the principle advantages of computational calculi like the
$\pi$-calculus is a well-defined notion of context,
contextual-equivalence and a correlation between
contextual-equivalence and notions of bisimulation. The notion of
context allows the decomposition of a process into (sub-)process and
its syntactic environment, its context. Thus, a context may be
thought of as a process with a ``hole'' (written $\Box$) in it. The
application of a context $M$ to a process $P$, written $M[P]$, is
tantamount to filling the hole in $M$ with $P$. In this paper we do
not need the full weight of this theory, but do make use of the notion
of context in the proof the main theorem. 

\begin{mathpar}
  \inferrule* [lab=summation] {} {{M_{M},M_{N}} \bc \Box \;|\; x.M_{A} \;|\; M_{M}+M_{N}}
  \and
  \inferrule* [lab=agent] {} {{M_{A}} \bc (\vec{x})M_{P} \;| \; \clift{P_0,\ldots,M_{P},\ldots,P_N}}
  \and \\
  \inferrule* [lab=process] {} {{M_{P}} \bc M_{N} \;| \;P|M_{P} }
\end{mathpar} 

\begin{mathpar}
  \inferrule* [lab=sychronization] {} {M_{N} \bc \Box \;|\; x?M_{F} \;|\; x!M_{C}}
  \and
  \inferrule* [lab=abstraction] {} {{M_{F}} \bc (x)M_{P} }
  \and
  \inferrule* [lab=concretion] {} {{M_{C}} \bc \langle M_{P} \rangle }
  \and \\
  \inferrule* [lab=process] {} {{M_{P}} \bc M_{N} \;| \;P|M_{P} }
\end{mathpar}

\begin{definition}[contextual application] Given a context $M$, and
  process $P$, we define the \emph{contextual application}, $M[P] :=
  M\{P/\Box\}$. That is, the contextual application of M to P is the
  substitution of $P$ for $\Box$ in $M$.
\end{definition}

$\meaningof{-} : L \to \mathcal{P}(\pi)$

\begin{mathpar}
  \inferrule* [lab=collection] {} {\meaningof{true} = \pi, \and \meaningof{~E} = \pi \setminus \meaningof{E}, \and \meaningof{E_{1} \& E_{2}} = \meaningof{E_{1}} \cap \meaningof{E_{2}}}
\end{mathpar}

\begin{mathpar}
  \inferrule* [lab=structure] {} {\meaningof{0} = \{ P \in \pi | P \equiv 0 \}, \and \\ \meaningof{E_1 | E_2} = \{ P \in \pi | P \equiv P_{1} | P_{2}, P_{1} \in \meaningof{E_{1}}, P_{2} \in \meaningof{E_2}\} }
\end{mathpar}

\begin{mathpar}
 \inferrule* [lab=behavior] {} {\meaningof{\langle a?b \rangle E} = \{ P \in \pi | P \equiv Q | u?(y)P', \\ \and \\\\ \and \\ \;\;\; u \in \meaningof{a}, \forall z.P'\{z/y\} \in \meaningof{E\{z/b\}}\}, \and \\ \meaningof{a!E} = \{ P \in \pi | P \equiv Q | x!\langle P' \rangle, x \in \meaningof{a} P' \in \meaningof{E}\} }
\end{mathpar}

\begin{mathpar}
 \inferrule* [lab=nominal] {} {\meaningof{\quotep{E}} = \{ \quotep{P} \in \quotep{\pi} | P \in \meaningof{E} \}, \and \meaningof{\quotep{P}} = \{ \quotep{Q} \in \quotep{\pi} | P \equiv Q \} \and \\ \meaningof{@\quotep{E}} = \{ P \in \pi | P \equiv @x, x \in \meaningof{E} \}}
\end{mathpar}

\begin{eqnarray*}
  \\
  \meaningof{-} : TS \to ST
\end{eqnarray*}

\begin{eqnarray*}
  \\
  L : TS \to ST
\end{eqnarray*}

\begin{eqnarray*}
  \\
  P \models E \iff P \in \meaningof{E}
\end{eqnarray*}

\begin{eqnarray*}
  P \approx_{L} Q \iff \forall E \in L. P \models E \iff Q \models E
\end{eqnarray*}

\begin{eqnarray*}
  P \approx_{K} Q
\end{eqnarray*}

\begin{eqnarray*}
  P \approx Q
\end{eqnarray*}

$\approx_{K} = \approx = \approx_{L}$

\subsubsection{Contextual duality}

Note that contexts extend the quotation operation to a family of
operations from processes to names. Given a context, $M$, we can
define a \emph{nominal context}, $\quotep{M}$ by $\quotep{M}[P] :=
\quotep{M[P]}$. To foreshadow what is to come we observe that these
operations enjoy a duality with processes very much like the duality
between vectors and maps from vectors to scalars.

Further, because the calculus is essentially higher-order, we have a
correspondence between contexts and processes. More specifically,
given a name $x$ and a context $M$ we can construct $M^{*}_{x}$ such
that 

\begin{mathpar}
  M^{*}_{x} | \lift{x}{P} \red M[P]
\end{mathpar}

namely,

\begin{mathpar}
  M^{*}_{x} := x?(u).M[\dropn{u}]
\end{mathpar}

The dependence of $M^{*}_{x}$ on a name makes it an abstraction, 

\begin{mathpar}
  M^{*} := (x)x?(u).M[\dropn{u}]
\end{mathpar}

\subsection{Additional notation}

It will sometimes be convenient to denote the process a name
quotes. We already have the notation $x = \quotep{P}$, but it will be
convenient to introduce an alternate notation, $\procn{x}$, when we
want to emphasize the connection to the use of the name. Note that, by
virtue of name equivalence, $\quotep{\procn{x}} \nameeq x$; so, the
notation is consistent with previous definitions.

Further, because names have structure it is possible to effect
substitutions on the basis of that structure. This means we need to
upgrade our notation for substitutions, which we accomplish by
adapting comprehension notation. Thus,

\begin{mathpar}
  P\{ y / x : x \in S \}
\end{mathpar}

is interpreted to mean the process derived from P by replacing (in a
capture-avoiding manner) each occurrence of $x$ in $S$ by $y$. For example,

\begin{mathpar}
  P\{ \quotep{\procn{x}|\procn{x}} / x : x \in \freenames{P} \}
\end{mathpar}

will replace each (occurrence) of a free name $x$ in $P$ by
$\quotep{\procn{x}|\procn{x}}$.

Also, we will avail ourselves of the notation $x^{L}$ and $x^{R}$ to
denote injections of a name into disjoint copies of the name
space. There are numerous ways to accomplish this. One example can be
found in \cite{MeredithR05}. This notation overloads to vectors of
names: $\vec{x}^{\pi} := (x_{i}^{\pi} \; : \; 0 \leq i < |\vec{x}| )$ where $\pi \in \{L,R\}$.

We also use $P^{\Box} := P|\Box$.

In \cite{MeredithR05} an interpretation of the new operator is
given. It turns out that there are several possible interpretations
all enjoying the requisite algebraic properties of the operator (see
\cite{milner91polyadicpi}). We will therefore make liberal use of
$(\nu\; \vec{x})P$.

% subsection the_syntax_and_semantics_of_the_notation_system (end)   

\input{qm2pi.qmops} 

\input{qm2pi.sterngerlach} 

\input{qm2pi.metric} 

% section concurrent_process_calculi (end)

%\input{qm2pi.proofsketch}

% section proof sketch (end)

%\input{qm2pi.slviaknots} 

% section spatial logic via knots (end)

\input{qm2pi.conclusion}

% section conclusion (end)

%\input{qm2pi.dtcodes} 

% section wiring algorithm (end)

\input{qm2pi.ack} 

% section acknowledgments (end)

\newpage


\bibliographystyle{plain}   
\bibliography{../../biblios/main.bib}

\input{qm2pi.rhodetails}

\end{document}

 

\documentclass[12pt]{llncs}
%\documentclass{jktr}

\usepackage[pdftex]{hyperref}                   
\usepackage {listings}
\usepackage {mathpartir}
\usepackage{bcprules}
%\usepackage{listings}
                       
\usepackage{graphicx} 
%\usepackage[margins=2.5cm,nohead,nofoot]{geometry}
%\usepackage{geometry}
\usepackage{amsfonts}
\usepackage{amstext}
\usepackage{latexsym}
\usepackage{amssymb}
\usepackage{color}


%\include{myPreamble}
\include{qm2pi.local} 

%\ifpdf
%\usepackage[pdftex]{graphicx}
%\else
%\usepackage{graphicx}
%\fi

 % \ifpdf
%  \usepackage{pdfsync}
%  \if


%\title{Brief Article}
%\author{David F. Snyder}
%\author{L.G. Meredith}

%\address{Dept. of Math., Texas State University--San Marcos, San Marcos, TX 78666}
       
\pagestyle{empty}


\begin{document}

\lstset{language=[Objective]Caml,frame=shadowbox}

\input{qm2pi.front}

% section front matter (end)

\input{qm2pi.intro} 
 
% section introduction (end)

% \input{qm2pi.knotations} 

% section notation (end)

\input{qm2pi.process.calculi} 

% section concurrent_process_calculi_and_spatial_logics_ (end)
    
%\input{qm2pi.knots2pi} 

%\input{qm2pi.trefoil} 

%\input{qm2pi.mainthm} 

% subsection basic_interpretation (end)

%\input{qm2pi.rho.presentation} 
\subsection{The syntax and semantics of the notation system}\label{sub:the_syntax_and_semantics_of_the_notation_system} % (fold)

We now summarize a technical presentation of the calculus that
embodies our theory of dynamics. The typical presentation of such a
calculus follows the style of giving generators and relations on
them. The grammar, below, describing term constructors, freely
generates the set of processes, $\Proc$. This set is then quotiented
by a relation known as structural congruence and it is over this set
that the notion of dynamics is expressed. This presentation is
essentially that of \cite{MeredithR05} with the addition of
polyadicity and summation. For readability we have relegated some of
the technical subtleties to an appendix.

\subsubsection{Process grammar}\label{subsub:process_grammar}

\begin{mathpar}
  \inferrule* [lab=synchronization] {} {{M} \bc \pzero \;|\; x?F \;|\; x!C }
  \and
  \inferrule* [lab=abstraction] {} {{F} \bc (x)P}
  \and
  \inferrule* [lab=concretion] {} {{C} \bc \langle Q \rangle}
  \and
  \inferrule* [lab=process] {} {{P,Q} \bc M \;| \;P|Q \;|\; @{x}}
  \and
  \inferrule* [lab=name] {} {{x} \bc \quotep{P}}
\end{mathpar} 

Note that $\vec{x}$ (resp. $\vec{P}$) denotes a vector of names
(resp. processes) of length $|\vec{x}|$ (resp. $|\vec{P}|$). We adopt
the following useful abbreviations.

\begin{mathpar}
   x?(\vec{y}).P := x.(\vec{y})P \and  x\clift{\vec{P}} := x.\clift{\vec{P}}
   \and x!(y) := \lift{x}{\dropn{y}}
   \and \Pi_{i=0}^{n-1}P_i := P_0 | \ldots | P_{n-1}
\end{mathpar}

\subsubsection{Structural congruence}

\paragraph{Free and bound names and alpha-equivalence.} At the
core of structural equivalence is alpha-equivalence which identifies
process that are the same up to a change of variable. Formally, we
recognize the distinction between free and bound names. The free names
of a process, $\freenames{P}$, may be calculated recursively as
follows:

\begin{mathpar}
\freenames{\pzero} := \emptyset
  \and \\
  \freenames{x?(y).P} := \{ x \} \cup (\freenames{P} \setminus \{ y \})
  \and 
  \freenames{x!\langle P \rangle} := \{ x \} \cup \{ P \} 
  \and \\
  \freenames{P|Q} := \freenames{P} \cup \freenames{Q}
  \and \\
  \freenames{@{x}} := \{ x \}
\end{mathpar}

$\pi$
$\quotep{\pi}$

$\freenames{-} : \pi \to \mathcal{P}(\quotep{\pi})$

\begin{eqnarray*}
  \freenames{\pzero} & := & \emptyset \\
  \freenames{x?(y).P} & := & \{ x \} \cup (\freenames{P} \setminus \{ y \}) \\
  \freenames{x!\langle P \rangle} & := & \{ x \} \cup \{ P \} \\
  \freenames{P|Q} & := & \freenames{P} \cup \freenames{Q} \\
  \freenames{\dropn{x}} & := & \{ x \}
\end{eqnarray*}

The bound names of a process, $\boundnames{P}$, are those names occurring in $P$
that are not free. For example, in $x?(y).0$, the name $x$ is free, while $y$ is bound.

\begin{mathpar}
  \inferrule* [lab=monoidal-laws] {} { P|Q \equiv Q|P \and P|0 \equiv P \and P|(Q|R) \equiv (P|Q)|R }
\end{mathpar}

\begin{mathpar}
  \inferrule* [lab=alpha-equivalence] {} { (x)P \equiv (y)P\{y/x\} \and y \not\in \freenames{P} }
\end{mathpar}

\begin{definition}
Then two processes, $P,Q$, are alpha-equivalent if $P = Q\{\vec{y}/\vec{x}\}$ for
some $\vec{x} \in \boundnames{Q},\vec{y} \in \boundnames{P}$, where $Q\{\vec{y}/\vec{x}\}$
denotes the capture-avoiding substitution of $\vec{y}$ for $\vec{x}$ in $Q$.
\end{definition}

\begin{definition}
  The {\em structural congruence} \cite{SangiorgiWalker} , $\equiv$,
  between processes is the least congruence containing
  alpha-equivalence, satisfying the abelian monoid laws
  (associativity, commutativity and $\pzero$ as identity) for parallel
  composition $|$ and for summation $+$.
\end{definition}

\subsection{Name equivalence}

We take name equivalence, written $\nameeq$, to be the smallest
equivalence relation generated by the following rules.

\begin{mathpar}
\inferrule*[lab=Quote-drop]
{ }
{ \quotep{@{x}} \nameeq x }

\inferrule*[lab=Struct-equiv]
{ P \scong Q }
{ \quotep{P} \nameeq \quotep{Q} }
\end{mathpar}

The astute reader will have noticed that the mutual recursion of names
and processes imposes a mutual recursion on alpha-equivalence and
structural equivalence via name-equivalence. Fortunately, all of this
works out pleasantly and we may calculate in the natural way, free of
concern. The reader interested in the details is referred to the
appendix \ref{appendix:rho_details}.

\subsection{Substitution}

We use $\Proc$ for the set of processes, $\QProc$ for the set of
names, and $\id{\{}\vec{y} / \vec{x} \id{\}}$ to denote partial maps,
$s : \QProc \rightarrow \QProc$. A map, $s$ lifts, uniquely, to a map
on process terms, $\widehat{s} : \Proc \rightarrow \Proc$ by the
following equations.

\begin{mathpar}
  (0) \psubstp{Q}{P} := 0 \\
  (R \juxtap S) \psubstp{Q}{P}
  :=    
  (R)\psubstp{Q}{P} \juxtap (S) \psubstp{Q}{P} \\
  (x?(y).R) \psubstp{Q}{P}    
  :=    
  (x)\substp{Q}{P} (z)\concat( (R \psubstn{z}{y}) \psubstp{Q}{P} ) \\
  (\lift{x}{R}) \psubstp{Q}{P}  
  :=
  \lift{(x)\substp{Q}{P}}{ R \psubstp{Q}{P} } \\
%   (\dropn{x})  \psubstp{Q}{P}       
%   := 
%   \left\{ 
%     \begin{array}{ccc} 
%       \dropn{\quotep{Q}} & & x \nameeq \quotep{P} \\
%       \dropn{x} & & otherwise \\
%     \end{array}
%   \right. 
  (\dropn{x})  \psubstp{Q}{P}       
  := 
  \left\{ 
    \begin{array}{ccc} 
      Q & & x \nameeq \quotep{P} \\
      \dropn{x} & & otherwise \\
    \end{array}
  \right.
\end{mathpar}
 

where

\begin{eqnarray}
  (x)\id{\{} \lpquote Q \rpquote / \lpquote P \rpquote \id{\}}            = 
  \left\{ 
    \begin{array}{ccc}
      \lpquote Q \rpquote & & x \nameeq \lpquote P \rpquote \\
      x & & otherwise \\
    \end{array}
  \right. \nonumber
\end{eqnarray}

and $z$ is chosen distinct from $\quotep{P}$, $\quotep{Q}$, the free
names in $Q$, and all the names in $R$. Our $\alpha$-equivalence will
be built in the standard way from this substitution.

\begin{remark}\label{rem:no_self_referential_names}
  One consequence of these definitions is that $\forall P. \quotep{P}
  \not\in \freenames{P}$.
\end{remark}

\subsection{ Dynamic quote: an example }

Anticipating something of what's to come, consider applying the
substitution, $\widehat{\id{\{}u / z \id{\}}}$, to the following pair
of processes, $\lift{w}{y!(z)}$ and $w[ \lpquote y!(z) \rpquote ]$.

\begin{eqnarray}
	\lift{w}{y!(z)}\widehat{\id{\{}u / z \id{\}}}
		& = &
		\lift{w}{y!(u)} \nonumber\\
	w[ \lpquote y!(z) \rpquote ] \widehat{ \id{\{}u / z \id{\}} }
		& = &
		w[ \lpquote y!(z) \rpquote ] \nonumber
\end{eqnarray}

Because the body of the process between quotes is impervious to
substitution, we get radically different answers. In fact, by
examining the first process in an input context,
e.g. $x?(z).\lift{w}{y!(z)}$, we see that the process under the lift
operator may be shaped by prefixed inputs binding a name inside it. In
this sense, the lift operator will be seen as a way to dynamically
construct processes before reifying them as names.

Finally equipped with these standard features we can present the
dynamics of the calculus.

\subsubsection{Operational semantics} 

Finally, we introduce the computational dynamics. What marks these
algebras as distinct from other more traditionally studied algebraic
structures, e.g. vector spaces or polynomial rings, is the manner in
which dynamics is captured. In traditional structures, dynamics is typically
expressed through morphisms between such structures, as in linear maps
between vector spaces or morphisms between rings. In algebras
associated with the semantics of computation, the dynamics is
expressed as part of the algebraic structure itself, through a
reduction reduction relation typically denoted by $\red$. Below, we
give a recursive presentation of this relation for the calculus used
in the encoding.

$\red \subseteq \pi \times \pi$
$\red : \pi \to \mathcal{P}(\pi)$

\begin{mathpar}
  \inferrule* [lab=Comm] { \textsf{match}( x_{src}, x_{trgt} ) } { x_{trgt}?(y)P \; | \; x_{src}!\langle {Q} \rangle \red P\{\quotep{Q}/y}\} }
  \and \\
  \inferrule* [lab=Par] {{P} \red {P}'} {{{P} | {Q}} \red {{P}' | {Q}}}
  \and
  \inferrule* [lab=Equiv]{{{P} \scong {P}'} \andalso {{P}' \red {Q}'} \andalso {{Q}' \scong {Q}}}{{P} \red {Q}}
\end{mathpar}

\begin{eqnarray*}
  match_{\equiv} (\quotep{P},\quotep{Q}) & := & P \equiv Q \\
  match_{\dagger}(\quotep{P},\quotep{Q}) & := & \forall R. P|Q \red^{*} R => R \red^{*} 0 \\
  match_{K}(\quotep{P},\quotep{Q}) & := & K \mbox{ for some context } K
\end{eqnarray*}

$u?(x)P | u!\langle Q \rangle \red P\{\quotep{Q}/x\}$

%We write $\wred$ for $\red^*$, and $P\red$ if $\exists Q $ such that $ P \red Q$.
We write $P\red$ if $\exists Q $ such that $ P \red Q$ and $P\not\red$, otherwise.

\section{Replication}

As mentioned before, it is known that replication (and hence
recursion) can be implemented in a higher-order process algebra
\cite{SangiorgiWalker}. As our first example of calculation with the
machinery thus far presented we give the construction explicitly in
the {\rhoc}.

\begin{eqnarray}
	D_{x} & := & \prefix{x}{y}{(\binpar{\outputp{x}{y}}{@{y}})} \nonumber\\
	\bangp_{x}{P} & := & \binpar{{x}!\langle{\binpar{D_{x}}{P}}\rangle}{D_{x}} \nonumber
\end{eqnarray}

\begin{eqnarray}
	\bangp_{x}{P} & & \nonumber\\
	=
	& {x}!\langle{(\prefix{x}{y}{(\outputp{x}{y} | @{y})) | P}}\rangle 
	      | \prefix{x}{y}{(\outputp{x}{y} | @{y})} & \nonumber\\
	\red
	& (\outputp{x}{y} | @{y})\substn{\quotep{(\prefix{x}{y}{(@{y} | \outputp{x}{y})) | P}}}{y} & \nonumber\\
	=
	& \outputp{x}{\quotep{(\prefix{x}{y}{(\outputp{x}{y} | @{y})) | P}}}
	  | {(\prefix{x}{y}{(\outputp{x}{y} | @{y})) | P}} & \nonumber\\
	\red
	& \ldots & \nonumber\\
	\red^*
	& P | P | \ldots & \nonumber
\end{eqnarray}

Of course, this encoding, as an implementation, runs away, unfolding
$\bangp{P}$ eagerly. A lazier and more implementable replication
operator, restricted to input-guarded processes, may be obtained as follows.

\begin{eqnarray}
\bangp{\prefix{u}{v}{P}} 
	:= 
	\binpar{\lift{x}{\prefix{u}{v}{(\binpar{D(x)}{P})}}}{D(x)} \nonumber
\end{eqnarray}

\begin{remark}
  Note that the lazier definition still does not deal with summation
  or mixed summation (i.e. sums over input and output). The reader is
  invited to construct definitions of replication that deal with these
  features. 

  Further, the definitions are parameterized in a name, $x$. Can you,
  gentle reader, make a definition that eliminates this parameter and
  guarantees no accidental interaction between the replication
  machinery and the process being replicated -- i.e. no accidental
  sharing of names used by the process to get its work done and the
  name(s) used by the replication to effect copying. This latter
  revision of the definition of replication is crucial to obtaining
  the expected identity $!!P \sim !P$.
\end{remark}

\begin{remark}\label{rem:paradoxical_combinator}
  The reader familiar with the lambda calculus will have noticed the
  similarity between $D$ and the paradoxical combinator.

  [Ed. note: the existence of this seems to suggest we have to be more
  restrictive on the set of processes and names we admit if we are to
  support no-cloning.]
\end{remark}

\subsubsection{Bisimulation}

The computational dynamics gives rise to another kind of equivalence,
the equivalence of computational behavior. As previously mentioned
this is typically captured \emph{via} some form of bisimulation.

% The notion we use in this paper is weak barbed bisimulation
% \cite{milner91polyadicpi}.

The notion we use in this paper is derived from weak barbed
bisimulation \cite{milner91polyadicpi}. 

\begin{definition}
An \emph{observation relation}, $\downarrow_{\mathcal N}$, over a set
of names, $\mathcal N$, is the smallest relation satisfying the rules
below.

\infrule[Out-barb]{y \in {\mathcal N}, \; x \nameeq y}
		  {\outputp{x}{v} \downarrow_{\mathcal N} x}
\infrule[Par-barb]{\mbox{$P\downarrow_{\mathcal N} x$ or $Q\downarrow_{\mathcal N} x$}}
		  {\binpar{P}{Q} \downarrow_{\mathcal N} x}

We write $P \Downarrow_{\mathcal N} x$ if there is $Q$ such that 
$P \wred Q$ and $Q \downarrow_{\mathcal N} x$.
\end{definition}

\begin{definition}
%\label{def.bbisim}
An  ${\mathcal N}$-\emph{barbed bisimulation} over a set of names, ${\mathcal N}$, is a symmetric binary relation 
${\mathcal S}_{\mathcal N}$ between agents such that $P\rel{S}_{\mathcal N}Q$ implies:
\begin{enumerate}
\item If $P \red P'$ then $Q \wred Q'$ and $P'\rel{S}_{\mathcal N} Q'$.
\item If $P\downarrow_{\mathcal N} x$, then $Q\Downarrow_{\mathcal N} x$.
\end{enumerate}
$P$ is ${\mathcal N}$-barbed bisimilar to $Q$, written
$P \wbbisim_{\mathcal N} Q$, if $P \rel{S}_{\mathcal N} Q$ for some ${\mathcal N}$-barbed bisimulation ${\mathcal S}_{\mathcal N}$.
\end{definition}

$\mathcal{R} \subseteq \pi \times \pi$

$P \mathcal{R} Q => \forall P'. P \red P' \Rightarrow \exists Q'. Q \red Q', P' \mathcal{R} Q'$

$P \vdash x \Rightarrow Q \vdash x$

\begin{mathpar}
  \inferrule*[lab=Out-barb]{x \nameeq y}{{y}!\langle{Q}\rangle \vdash x}
  \and
  \inferrule*[lab=Par-barb]{\mbox{$P\vdash x$ or $Q\vdash x$}}{\binpar{P}{Q} \vdash x}
\end{mathpar}

\subsubsection{Contexts}

One of the principle advantages of computational calculi like the
$\pi$-calculus is a well-defined notion of context,
contextual-equivalence and a correlation between
contextual-equivalence and notions of bisimulation. The notion of
context allows the decomposition of a process into (sub-)process and
its syntactic environment, its context. Thus, a context may be
thought of as a process with a ``hole'' (written $\Box$) in it. The
application of a context $M$ to a process $P$, written $M[P]$, is
tantamount to filling the hole in $M$ with $P$. In this paper we do
not need the full weight of this theory, but do make use of the notion
of context in the proof the main theorem. 

\begin{mathpar}
  \inferrule* [lab=summation] {} {{M_{M},M_{N}} \bc \Box \;|\; x.M_{A} \;|\; M_{M}+M_{N}}
  \and
  \inferrule* [lab=agent] {} {{M_{A}} \bc (\vec{x})M_{P} \;| \; \clift{P_0,\ldots,M_{P},\ldots,P_N}}
  \and \\
  \inferrule* [lab=process] {} {{M_{P}} \bc M_{N} \;| \;P|M_{P} }
\end{mathpar} 

\begin{mathpar}
  \inferrule* [lab=sychronization] {} {M_{N} \bc \Box \;|\; x?M_{F} \;|\; x!M_{C}}
  \and
  \inferrule* [lab=abstraction] {} {{M_{F}} \bc (x)M_{P} }
  \and
  \inferrule* [lab=concretion] {} {{M_{C}} \bc \langle M_{P} \rangle }
  \and \\
  \inferrule* [lab=process] {} {{M_{P}} \bc M_{N} \;| \;P|M_{P} }
\end{mathpar}

\begin{definition}[contextual application] Given a context $M$, and
  process $P$, we define the \emph{contextual application}, $M[P] :=
  M\{P/\Box\}$. That is, the contextual application of M to P is the
  substitution of $P$ for $\Box$ in $M$.
\end{definition}

$\meaningof{-} : L \to \mathcal{P}(\pi)$

\begin{mathpar}
  \inferrule* [lab=collection] {} {\meaningof{true} = \pi, \and \meaningof{~E} = \pi \setminus \meaningof{E}, \and \meaningof{E_{1} \& E_{2}} = \meaningof{E_{1}} \cap \meaningof{E_{2}}}
\end{mathpar}

\begin{mathpar}
  \inferrule* [lab=structure] {} {\meaningof{0} = \{ P \in \pi | P \equiv 0 \}, \and \\ \meaningof{E_1 | E_2} = \{ P \in \pi | P \equiv P_{1} | P_{2}, P_{1} \in \meaningof{E_{1}}, P_{2} \in \meaningof{E_2}\} }
\end{mathpar}

\begin{mathpar}
 \inferrule* [lab=behavior] {} {\meaningof{\langle a?b \rangle E} = \{ P \in \pi | P \equiv Q | u?(y)P', \\ \and \\\\ \and \\ \;\;\; u \in \meaningof{a}, \forall z.P'\{z/y\} \in \meaningof{E\{z/b\}}\}, \and \\ \meaningof{a!E} = \{ P \in \pi | P \equiv Q | x!\langle P' \rangle, x \in \meaningof{a} P' \in \meaningof{E}\} }
\end{mathpar}

\begin{mathpar}
 \inferrule* [lab=nominal] {} {\meaningof{\quotep{E}} = \{ \quotep{P} \in \quotep{\pi} | P \in \meaningof{E} \}, \and \meaningof{\quotep{P}} = \{ \quotep{Q} \in \quotep{\pi} | P \equiv Q \} \and \\ \meaningof{@\quotep{E}} = \{ P \in \pi | P \equiv @x, x \in \meaningof{E} \}}
\end{mathpar}

\begin{eqnarray*}
  \\
  \meaningof{-} : TS \to ST
\end{eqnarray*}

\begin{eqnarray*}
  \\
  L : TS \to ST
\end{eqnarray*}

\begin{eqnarray*}
  \\
  P \models E \iff P \in \meaningof{E}
\end{eqnarray*}

\begin{eqnarray*}
  P \approx_{L} Q \iff \forall E \in L. P \models E \iff Q \models E
\end{eqnarray*}

\begin{eqnarray*}
  P \approx_{K} Q
\end{eqnarray*}

\begin{eqnarray*}
  P \approx Q
\end{eqnarray*}

$\approx_{K} = \approx = \approx_{L}$

\subsubsection{Contextual duality}

Note that contexts extend the quotation operation to a family of
operations from processes to names. Given a context, $M$, we can
define a \emph{nominal context}, $\quotep{M}$ by $\quotep{M}[P] :=
\quotep{M[P]}$. To foreshadow what is to come we observe that these
operations enjoy a duality with processes very much like the duality
between vectors and maps from vectors to scalars.

Further, because the calculus is essentially higher-order, we have a
correspondence between contexts and processes. More specifically,
given a name $x$ and a context $M$ we can construct $M^{*}_{x}$ such
that 

\begin{mathpar}
  M^{*}_{x} | \lift{x}{P} \red M[P]
\end{mathpar}

namely,

\begin{mathpar}
  M^{*}_{x} := x?(u).M[\dropn{u}]
\end{mathpar}

The dependence of $M^{*}_{x}$ on a name makes it an abstraction, 

\begin{mathpar}
  M^{*} := (x)x?(u).M[\dropn{u}]
\end{mathpar}

\subsection{Additional notation}

It will sometimes be convenient to denote the process a name
quotes. We already have the notation $x = \quotep{P}$, but it will be
convenient to introduce an alternate notation, $\procn{x}$, when we
want to emphasize the connection to the use of the name. Note that, by
virtue of name equivalence, $\quotep{\procn{x}} \nameeq x$; so, the
notation is consistent with previous definitions.

Further, because names have structure it is possible to effect
substitutions on the basis of that structure. This means we need to
upgrade our notation for substitutions, which we accomplish by
adapting comprehension notation. Thus,

\begin{mathpar}
  P\{ y / x : x \in S \}
\end{mathpar}

is interpreted to mean the process derived from P by replacing (in a
capture-avoiding manner) each occurrence of $x$ in $S$ by $y$. For example,

\begin{mathpar}
  P\{ \quotep{\procn{x}|\procn{x}} / x : x \in \freenames{P} \}
\end{mathpar}

will replace each (occurrence) of a free name $x$ in $P$ by
$\quotep{\procn{x}|\procn{x}}$.

Also, we will avail ourselves of the notation $x^{L}$ and $x^{R}$ to
denote injections of a name into disjoint copies of the name
space. There are numerous ways to accomplish this. One example can be
found in \cite{MeredithR05}. This notation overloads to vectors of
names: $\vec{x}^{\pi} := (x_{i}^{\pi} \; : \; 0 \leq i < |\vec{x}| )$ where $\pi \in \{L,R\}$.

We also use $P^{\Box} := P|\Box$.

In \cite{MeredithR05} an interpretation of the new operator is
given. It turns out that there are several possible interpretations
all enjoying the requisite algebraic properties of the operator (see
\cite{milner91polyadicpi}). We will therefore make liberal use of
$(\nu\; \vec{x})P$.

% subsection the_syntax_and_semantics_of_the_notation_system (end)   

\input{qm2pi.qmops} 

\input{qm2pi.sterngerlach} 

\input{qm2pi.metric} 

% section concurrent_process_calculi (end)

%\input{qm2pi.proofsketch}

% section proof sketch (end)

%\input{qm2pi.slviaknots} 

% section spatial logic via knots (end)

\input{qm2pi.conclusion}

% section conclusion (end)

%\input{qm2pi.dtcodes} 

% section wiring algorithm (end)

\input{qm2pi.ack} 

% section acknowledgments (end)

\newpage


\bibliographystyle{plain}   
\bibliography{../../biblios/main.bib}

\input{qm2pi.rhodetails}

\end{document}

 

% section concurrent_process_calculi (end)

%\documentclass[12pt]{llncs}
%\documentclass{jktr}

\usepackage[pdftex]{hyperref}                   
\usepackage {listings}
\usepackage {mathpartir}
\usepackage{bcprules}
%\usepackage{listings}
                       
\usepackage{graphicx} 
%\usepackage[margins=2.5cm,nohead,nofoot]{geometry}
%\usepackage{geometry}
\usepackage{amsfonts}
\usepackage{amstext}
\usepackage{latexsym}
\usepackage{amssymb}
\usepackage{color}


%\include{myPreamble}
\include{qm2pi.local} 

%\ifpdf
%\usepackage[pdftex]{graphicx}
%\else
%\usepackage{graphicx}
%\fi

 % \ifpdf
%  \usepackage{pdfsync}
%  \if


%\title{Brief Article}
%\author{David F. Snyder}
%\author{L.G. Meredith}

%\address{Dept. of Math., Texas State University--San Marcos, San Marcos, TX 78666}
       
\pagestyle{empty}


\begin{document}

\lstset{language=[Objective]Caml,frame=shadowbox}

\input{qm2pi.front}

% section front matter (end)

\input{qm2pi.intro} 
 
% section introduction (end)

% \input{qm2pi.knotations} 

% section notation (end)

\input{qm2pi.process.calculi} 

% section concurrent_process_calculi_and_spatial_logics_ (end)
    
%\input{qm2pi.knots2pi} 

%\input{qm2pi.trefoil} 

%\input{qm2pi.mainthm} 

% subsection basic_interpretation (end)

%\input{qm2pi.rho.presentation} 
\subsection{The syntax and semantics of the notation system}\label{sub:the_syntax_and_semantics_of_the_notation_system} % (fold)

We now summarize a technical presentation of the calculus that
embodies our theory of dynamics. The typical presentation of such a
calculus follows the style of giving generators and relations on
them. The grammar, below, describing term constructors, freely
generates the set of processes, $\Proc$. This set is then quotiented
by a relation known as structural congruence and it is over this set
that the notion of dynamics is expressed. This presentation is
essentially that of \cite{MeredithR05} with the addition of
polyadicity and summation. For readability we have relegated some of
the technical subtleties to an appendix.

\subsubsection{Process grammar}\label{subsub:process_grammar}

\begin{mathpar}
  \inferrule* [lab=synchronization] {} {{M} \bc \pzero \;|\; x?F \;|\; x!C }
  \and
  \inferrule* [lab=abstraction] {} {{F} \bc (x)P}
  \and
  \inferrule* [lab=concretion] {} {{C} \bc \langle Q \rangle}
  \and
  \inferrule* [lab=process] {} {{P,Q} \bc M \;| \;P|Q \;|\; @{x}}
  \and
  \inferrule* [lab=name] {} {{x} \bc \quotep{P}}
\end{mathpar} 

Note that $\vec{x}$ (resp. $\vec{P}$) denotes a vector of names
(resp. processes) of length $|\vec{x}|$ (resp. $|\vec{P}|$). We adopt
the following useful abbreviations.

\begin{mathpar}
   x?(\vec{y}).P := x.(\vec{y})P \and  x\clift{\vec{P}} := x.\clift{\vec{P}}
   \and x!(y) := \lift{x}{\dropn{y}}
   \and \Pi_{i=0}^{n-1}P_i := P_0 | \ldots | P_{n-1}
\end{mathpar}

\subsubsection{Structural congruence}

\paragraph{Free and bound names and alpha-equivalence.} At the
core of structural equivalence is alpha-equivalence which identifies
process that are the same up to a change of variable. Formally, we
recognize the distinction between free and bound names. The free names
of a process, $\freenames{P}$, may be calculated recursively as
follows:

\begin{mathpar}
\freenames{\pzero} := \emptyset
  \and \\
  \freenames{x?(y).P} := \{ x \} \cup (\freenames{P} \setminus \{ y \})
  \and 
  \freenames{x!\langle P \rangle} := \{ x \} \cup \{ P \} 
  \and \\
  \freenames{P|Q} := \freenames{P} \cup \freenames{Q}
  \and \\
  \freenames{@{x}} := \{ x \}
\end{mathpar}

$\pi$
$\quotep{\pi}$

$\freenames{-} : \pi \to \mathcal{P}(\quotep{\pi})$

\begin{eqnarray*}
  \freenames{\pzero} & := & \emptyset \\
  \freenames{x?(y).P} & := & \{ x \} \cup (\freenames{P} \setminus \{ y \}) \\
  \freenames{x!\langle P \rangle} & := & \{ x \} \cup \{ P \} \\
  \freenames{P|Q} & := & \freenames{P} \cup \freenames{Q} \\
  \freenames{\dropn{x}} & := & \{ x \}
\end{eqnarray*}

The bound names of a process, $\boundnames{P}$, are those names occurring in $P$
that are not free. For example, in $x?(y).0$, the name $x$ is free, while $y$ is bound.

\begin{mathpar}
  \inferrule* [lab=monoidal-laws] {} { P|Q \equiv Q|P \and P|0 \equiv P \and P|(Q|R) \equiv (P|Q)|R }
\end{mathpar}

\begin{mathpar}
  \inferrule* [lab=alpha-equivalence] {} { (x)P \equiv (y)P\{y/x\} \and y \not\in \freenames{P} }
\end{mathpar}

\begin{definition}
Then two processes, $P,Q$, are alpha-equivalent if $P = Q\{\vec{y}/\vec{x}\}$ for
some $\vec{x} \in \boundnames{Q},\vec{y} \in \boundnames{P}$, where $Q\{\vec{y}/\vec{x}\}$
denotes the capture-avoiding substitution of $\vec{y}$ for $\vec{x}$ in $Q$.
\end{definition}

\begin{definition}
  The {\em structural congruence} \cite{SangiorgiWalker} , $\equiv$,
  between processes is the least congruence containing
  alpha-equivalence, satisfying the abelian monoid laws
  (associativity, commutativity and $\pzero$ as identity) for parallel
  composition $|$ and for summation $+$.
\end{definition}

\subsection{Name equivalence}

We take name equivalence, written $\nameeq$, to be the smallest
equivalence relation generated by the following rules.

\begin{mathpar}
\inferrule*[lab=Quote-drop]
{ }
{ \quotep{@{x}} \nameeq x }

\inferrule*[lab=Struct-equiv]
{ P \scong Q }
{ \quotep{P} \nameeq \quotep{Q} }
\end{mathpar}

The astute reader will have noticed that the mutual recursion of names
and processes imposes a mutual recursion on alpha-equivalence and
structural equivalence via name-equivalence. Fortunately, all of this
works out pleasantly and we may calculate in the natural way, free of
concern. The reader interested in the details is referred to the
appendix \ref{appendix:rho_details}.

\subsection{Substitution}

We use $\Proc$ for the set of processes, $\QProc$ for the set of
names, and $\id{\{}\vec{y} / \vec{x} \id{\}}$ to denote partial maps,
$s : \QProc \rightarrow \QProc$. A map, $s$ lifts, uniquely, to a map
on process terms, $\widehat{s} : \Proc \rightarrow \Proc$ by the
following equations.

\begin{mathpar}
  (0) \psubstp{Q}{P} := 0 \\
  (R \juxtap S) \psubstp{Q}{P}
  :=    
  (R)\psubstp{Q}{P} \juxtap (S) \psubstp{Q}{P} \\
  (x?(y).R) \psubstp{Q}{P}    
  :=    
  (x)\substp{Q}{P} (z)\concat( (R \psubstn{z}{y}) \psubstp{Q}{P} ) \\
  (\lift{x}{R}) \psubstp{Q}{P}  
  :=
  \lift{(x)\substp{Q}{P}}{ R \psubstp{Q}{P} } \\
%   (\dropn{x})  \psubstp{Q}{P}       
%   := 
%   \left\{ 
%     \begin{array}{ccc} 
%       \dropn{\quotep{Q}} & & x \nameeq \quotep{P} \\
%       \dropn{x} & & otherwise \\
%     \end{array}
%   \right. 
  (\dropn{x})  \psubstp{Q}{P}       
  := 
  \left\{ 
    \begin{array}{ccc} 
      Q & & x \nameeq \quotep{P} \\
      \dropn{x} & & otherwise \\
    \end{array}
  \right.
\end{mathpar}
 

where

\begin{eqnarray}
  (x)\id{\{} \lpquote Q \rpquote / \lpquote P \rpquote \id{\}}            = 
  \left\{ 
    \begin{array}{ccc}
      \lpquote Q \rpquote & & x \nameeq \lpquote P \rpquote \\
      x & & otherwise \\
    \end{array}
  \right. \nonumber
\end{eqnarray}

and $z$ is chosen distinct from $\quotep{P}$, $\quotep{Q}$, the free
names in $Q$, and all the names in $R$. Our $\alpha$-equivalence will
be built in the standard way from this substitution.

\begin{remark}\label{rem:no_self_referential_names}
  One consequence of these definitions is that $\forall P. \quotep{P}
  \not\in \freenames{P}$.
\end{remark}

\subsection{ Dynamic quote: an example }

Anticipating something of what's to come, consider applying the
substitution, $\widehat{\id{\{}u / z \id{\}}}$, to the following pair
of processes, $\lift{w}{y!(z)}$ and $w[ \lpquote y!(z) \rpquote ]$.

\begin{eqnarray}
	\lift{w}{y!(z)}\widehat{\id{\{}u / z \id{\}}}
		& = &
		\lift{w}{y!(u)} \nonumber\\
	w[ \lpquote y!(z) \rpquote ] \widehat{ \id{\{}u / z \id{\}} }
		& = &
		w[ \lpquote y!(z) \rpquote ] \nonumber
\end{eqnarray}

Because the body of the process between quotes is impervious to
substitution, we get radically different answers. In fact, by
examining the first process in an input context,
e.g. $x?(z).\lift{w}{y!(z)}$, we see that the process under the lift
operator may be shaped by prefixed inputs binding a name inside it. In
this sense, the lift operator will be seen as a way to dynamically
construct processes before reifying them as names.

Finally equipped with these standard features we can present the
dynamics of the calculus.

\subsubsection{Operational semantics} 

Finally, we introduce the computational dynamics. What marks these
algebras as distinct from other more traditionally studied algebraic
structures, e.g. vector spaces or polynomial rings, is the manner in
which dynamics is captured. In traditional structures, dynamics is typically
expressed through morphisms between such structures, as in linear maps
between vector spaces or morphisms between rings. In algebras
associated with the semantics of computation, the dynamics is
expressed as part of the algebraic structure itself, through a
reduction reduction relation typically denoted by $\red$. Below, we
give a recursive presentation of this relation for the calculus used
in the encoding.

$\red \subseteq \pi \times \pi$
$\red : \pi \to \mathcal{P}(\pi)$

\begin{mathpar}
  \inferrule* [lab=Comm] { \textsf{match}( x_{src}, x_{trgt} ) } { x_{trgt}?(y)P \; | \; x_{src}!\langle {Q} \rangle \red P\{\quotep{Q}/y}\} }
  \and \\
  \inferrule* [lab=Par] {{P} \red {P}'} {{{P} | {Q}} \red {{P}' | {Q}}}
  \and
  \inferrule* [lab=Equiv]{{{P} \scong {P}'} \andalso {{P}' \red {Q}'} \andalso {{Q}' \scong {Q}}}{{P} \red {Q}}
\end{mathpar}

\begin{eqnarray*}
  match_{\equiv} (\quotep{P},\quotep{Q}) & := & P \equiv Q \\
  match_{\dagger}(\quotep{P},\quotep{Q}) & := & \forall R. P|Q \red^{*} R => R \red^{*} 0 \\
  match_{K}(\quotep{P},\quotep{Q}) & := & K \mbox{ for some context } K
\end{eqnarray*}

$u?(x)P | u!\langle Q \rangle \red P\{\quotep{Q}/x\}$

%We write $\wred$ for $\red^*$, and $P\red$ if $\exists Q $ such that $ P \red Q$.
We write $P\red$ if $\exists Q $ such that $ P \red Q$ and $P\not\red$, otherwise.

\section{Replication}

As mentioned before, it is known that replication (and hence
recursion) can be implemented in a higher-order process algebra
\cite{SangiorgiWalker}. As our first example of calculation with the
machinery thus far presented we give the construction explicitly in
the {\rhoc}.

\begin{eqnarray}
	D_{x} & := & \prefix{x}{y}{(\binpar{\outputp{x}{y}}{@{y}})} \nonumber\\
	\bangp_{x}{P} & := & \binpar{{x}!\langle{\binpar{D_{x}}{P}}\rangle}{D_{x}} \nonumber
\end{eqnarray}

\begin{eqnarray}
	\bangp_{x}{P} & & \nonumber\\
	=
	& {x}!\langle{(\prefix{x}{y}{(\outputp{x}{y} | @{y})) | P}}\rangle 
	      | \prefix{x}{y}{(\outputp{x}{y} | @{y})} & \nonumber\\
	\red
	& (\outputp{x}{y} | @{y})\substn{\quotep{(\prefix{x}{y}{(@{y} | \outputp{x}{y})) | P}}}{y} & \nonumber\\
	=
	& \outputp{x}{\quotep{(\prefix{x}{y}{(\outputp{x}{y} | @{y})) | P}}}
	  | {(\prefix{x}{y}{(\outputp{x}{y} | @{y})) | P}} & \nonumber\\
	\red
	& \ldots & \nonumber\\
	\red^*
	& P | P | \ldots & \nonumber
\end{eqnarray}

Of course, this encoding, as an implementation, runs away, unfolding
$\bangp{P}$ eagerly. A lazier and more implementable replication
operator, restricted to input-guarded processes, may be obtained as follows.

\begin{eqnarray}
\bangp{\prefix{u}{v}{P}} 
	:= 
	\binpar{\lift{x}{\prefix{u}{v}{(\binpar{D(x)}{P})}}}{D(x)} \nonumber
\end{eqnarray}

\begin{remark}
  Note that the lazier definition still does not deal with summation
  or mixed summation (i.e. sums over input and output). The reader is
  invited to construct definitions of replication that deal with these
  features. 

  Further, the definitions are parameterized in a name, $x$. Can you,
  gentle reader, make a definition that eliminates this parameter and
  guarantees no accidental interaction between the replication
  machinery and the process being replicated -- i.e. no accidental
  sharing of names used by the process to get its work done and the
  name(s) used by the replication to effect copying. This latter
  revision of the definition of replication is crucial to obtaining
  the expected identity $!!P \sim !P$.
\end{remark}

\begin{remark}\label{rem:paradoxical_combinator}
  The reader familiar with the lambda calculus will have noticed the
  similarity between $D$ and the paradoxical combinator.

  [Ed. note: the existence of this seems to suggest we have to be more
  restrictive on the set of processes and names we admit if we are to
  support no-cloning.]
\end{remark}

\subsubsection{Bisimulation}

The computational dynamics gives rise to another kind of equivalence,
the equivalence of computational behavior. As previously mentioned
this is typically captured \emph{via} some form of bisimulation.

% The notion we use in this paper is weak barbed bisimulation
% \cite{milner91polyadicpi}.

The notion we use in this paper is derived from weak barbed
bisimulation \cite{milner91polyadicpi}. 

\begin{definition}
An \emph{observation relation}, $\downarrow_{\mathcal N}$, over a set
of names, $\mathcal N$, is the smallest relation satisfying the rules
below.

\infrule[Out-barb]{y \in {\mathcal N}, \; x \nameeq y}
		  {\outputp{x}{v} \downarrow_{\mathcal N} x}
\infrule[Par-barb]{\mbox{$P\downarrow_{\mathcal N} x$ or $Q\downarrow_{\mathcal N} x$}}
		  {\binpar{P}{Q} \downarrow_{\mathcal N} x}

We write $P \Downarrow_{\mathcal N} x$ if there is $Q$ such that 
$P \wred Q$ and $Q \downarrow_{\mathcal N} x$.
\end{definition}

\begin{definition}
%\label{def.bbisim}
An  ${\mathcal N}$-\emph{barbed bisimulation} over a set of names, ${\mathcal N}$, is a symmetric binary relation 
${\mathcal S}_{\mathcal N}$ between agents such that $P\rel{S}_{\mathcal N}Q$ implies:
\begin{enumerate}
\item If $P \red P'$ then $Q \wred Q'$ and $P'\rel{S}_{\mathcal N} Q'$.
\item If $P\downarrow_{\mathcal N} x$, then $Q\Downarrow_{\mathcal N} x$.
\end{enumerate}
$P$ is ${\mathcal N}$-barbed bisimilar to $Q$, written
$P \wbbisim_{\mathcal N} Q$, if $P \rel{S}_{\mathcal N} Q$ for some ${\mathcal N}$-barbed bisimulation ${\mathcal S}_{\mathcal N}$.
\end{definition}

$\mathcal{R} \subseteq \pi \times \pi$

$P \mathcal{R} Q => \forall P'. P \red P' \Rightarrow \exists Q'. Q \red Q', P' \mathcal{R} Q'$

$P \vdash x \Rightarrow Q \vdash x$

\begin{mathpar}
  \inferrule*[lab=Out-barb]{x \nameeq y}{{y}!\langle{Q}\rangle \vdash x}
  \and
  \inferrule*[lab=Par-barb]{\mbox{$P\vdash x$ or $Q\vdash x$}}{\binpar{P}{Q} \vdash x}
\end{mathpar}

\subsubsection{Contexts}

One of the principle advantages of computational calculi like the
$\pi$-calculus is a well-defined notion of context,
contextual-equivalence and a correlation between
contextual-equivalence and notions of bisimulation. The notion of
context allows the decomposition of a process into (sub-)process and
its syntactic environment, its context. Thus, a context may be
thought of as a process with a ``hole'' (written $\Box$) in it. The
application of a context $M$ to a process $P$, written $M[P]$, is
tantamount to filling the hole in $M$ with $P$. In this paper we do
not need the full weight of this theory, but do make use of the notion
of context in the proof the main theorem. 

\begin{mathpar}
  \inferrule* [lab=summation] {} {{M_{M},M_{N}} \bc \Box \;|\; x.M_{A} \;|\; M_{M}+M_{N}}
  \and
  \inferrule* [lab=agent] {} {{M_{A}} \bc (\vec{x})M_{P} \;| \; \clift{P_0,\ldots,M_{P},\ldots,P_N}}
  \and \\
  \inferrule* [lab=process] {} {{M_{P}} \bc M_{N} \;| \;P|M_{P} }
\end{mathpar} 

\begin{mathpar}
  \inferrule* [lab=sychronization] {} {M_{N} \bc \Box \;|\; x?M_{F} \;|\; x!M_{C}}
  \and
  \inferrule* [lab=abstraction] {} {{M_{F}} \bc (x)M_{P} }
  \and
  \inferrule* [lab=concretion] {} {{M_{C}} \bc \langle M_{P} \rangle }
  \and \\
  \inferrule* [lab=process] {} {{M_{P}} \bc M_{N} \;| \;P|M_{P} }
\end{mathpar}

\begin{definition}[contextual application] Given a context $M$, and
  process $P$, we define the \emph{contextual application}, $M[P] :=
  M\{P/\Box\}$. That is, the contextual application of M to P is the
  substitution of $P$ for $\Box$ in $M$.
\end{definition}

$\meaningof{-} : L \to \mathcal{P}(\pi)$

\begin{mathpar}
  \inferrule* [lab=collection] {} {\meaningof{true} = \pi, \and \meaningof{~E} = \pi \setminus \meaningof{E}, \and \meaningof{E_{1} \& E_{2}} = \meaningof{E_{1}} \cap \meaningof{E_{2}}}
\end{mathpar}

\begin{mathpar}
  \inferrule* [lab=structure] {} {\meaningof{0} = \{ P \in \pi | P \equiv 0 \}, \and \\ \meaningof{E_1 | E_2} = \{ P \in \pi | P \equiv P_{1} | P_{2}, P_{1} \in \meaningof{E_{1}}, P_{2} \in \meaningof{E_2}\} }
\end{mathpar}

\begin{mathpar}
 \inferrule* [lab=behavior] {} {\meaningof{\langle a?b \rangle E} = \{ P \in \pi | P \equiv Q | u?(y)P', \\ \and \\\\ \and \\ \;\;\; u \in \meaningof{a}, \forall z.P'\{z/y\} \in \meaningof{E\{z/b\}}\}, \and \\ \meaningof{a!E} = \{ P \in \pi | P \equiv Q | x!\langle P' \rangle, x \in \meaningof{a} P' \in \meaningof{E}\} }
\end{mathpar}

\begin{mathpar}
 \inferrule* [lab=nominal] {} {\meaningof{\quotep{E}} = \{ \quotep{P} \in \quotep{\pi} | P \in \meaningof{E} \}, \and \meaningof{\quotep{P}} = \{ \quotep{Q} \in \quotep{\pi} | P \equiv Q \} \and \\ \meaningof{@\quotep{E}} = \{ P \in \pi | P \equiv @x, x \in \meaningof{E} \}}
\end{mathpar}

\begin{eqnarray*}
  \\
  \meaningof{-} : TS \to ST
\end{eqnarray*}

\begin{eqnarray*}
  \\
  L : TS \to ST
\end{eqnarray*}

\begin{eqnarray*}
  \\
  P \models E \iff P \in \meaningof{E}
\end{eqnarray*}

\begin{eqnarray*}
  P \approx_{L} Q \iff \forall E \in L. P \models E \iff Q \models E
\end{eqnarray*}

\begin{eqnarray*}
  P \approx_{K} Q
\end{eqnarray*}

\begin{eqnarray*}
  P \approx Q
\end{eqnarray*}

$\approx_{K} = \approx = \approx_{L}$

\subsubsection{Contextual duality}

Note that contexts extend the quotation operation to a family of
operations from processes to names. Given a context, $M$, we can
define a \emph{nominal context}, $\quotep{M}$ by $\quotep{M}[P] :=
\quotep{M[P]}$. To foreshadow what is to come we observe that these
operations enjoy a duality with processes very much like the duality
between vectors and maps from vectors to scalars.

Further, because the calculus is essentially higher-order, we have a
correspondence between contexts and processes. More specifically,
given a name $x$ and a context $M$ we can construct $M^{*}_{x}$ such
that 

\begin{mathpar}
  M^{*}_{x} | \lift{x}{P} \red M[P]
\end{mathpar}

namely,

\begin{mathpar}
  M^{*}_{x} := x?(u).M[\dropn{u}]
\end{mathpar}

The dependence of $M^{*}_{x}$ on a name makes it an abstraction, 

\begin{mathpar}
  M^{*} := (x)x?(u).M[\dropn{u}]
\end{mathpar}

\subsection{Additional notation}

It will sometimes be convenient to denote the process a name
quotes. We already have the notation $x = \quotep{P}$, but it will be
convenient to introduce an alternate notation, $\procn{x}$, when we
want to emphasize the connection to the use of the name. Note that, by
virtue of name equivalence, $\quotep{\procn{x}} \nameeq x$; so, the
notation is consistent with previous definitions.

Further, because names have structure it is possible to effect
substitutions on the basis of that structure. This means we need to
upgrade our notation for substitutions, which we accomplish by
adapting comprehension notation. Thus,

\begin{mathpar}
  P\{ y / x : x \in S \}
\end{mathpar}

is interpreted to mean the process derived from P by replacing (in a
capture-avoiding manner) each occurrence of $x$ in $S$ by $y$. For example,

\begin{mathpar}
  P\{ \quotep{\procn{x}|\procn{x}} / x : x \in \freenames{P} \}
\end{mathpar}

will replace each (occurrence) of a free name $x$ in $P$ by
$\quotep{\procn{x}|\procn{x}}$.

Also, we will avail ourselves of the notation $x^{L}$ and $x^{R}$ to
denote injections of a name into disjoint copies of the name
space. There are numerous ways to accomplish this. One example can be
found in \cite{MeredithR05}. This notation overloads to vectors of
names: $\vec{x}^{\pi} := (x_{i}^{\pi} \; : \; 0 \leq i < |\vec{x}| )$ where $\pi \in \{L,R\}$.

We also use $P^{\Box} := P|\Box$.

In \cite{MeredithR05} an interpretation of the new operator is
given. It turns out that there are several possible interpretations
all enjoying the requisite algebraic properties of the operator (see
\cite{milner91polyadicpi}). We will therefore make liberal use of
$(\nu\; \vec{x})P$.

% subsection the_syntax_and_semantics_of_the_notation_system (end)   

\input{qm2pi.qmops} 

\input{qm2pi.sterngerlach} 

\input{qm2pi.metric} 

% section concurrent_process_calculi (end)

%\input{qm2pi.proofsketch}

% section proof sketch (end)

%\input{qm2pi.slviaknots} 

% section spatial logic via knots (end)

\input{qm2pi.conclusion}

% section conclusion (end)

%\input{qm2pi.dtcodes} 

% section wiring algorithm (end)

\input{qm2pi.ack} 

% section acknowledgments (end)

\newpage


\bibliographystyle{plain}   
\bibliography{../../biblios/main.bib}

\input{qm2pi.rhodetails}

\end{document}



% section proof sketch (end)

%\section{Unlikely characters: spatial logic for
  knots}\label{sub:characteristic_formulae} % (fold)

Associated to the mobile process calculi are a family of logics known
as the Hennessy-Milner logics. These logics typically enjoy a
semantics interpreting formulae as sets of processes that when
factored through the encoding outlined above allows an identification
of classes of knots with logical formulae. In the context of this
encoding the sub-family known as the spatial logics \cite{CairesC03}
\cite{CairesC04} \cite{Caires04} are of particular interest providing
several important features for expressing and reasoning about
properties (i.e. classes) of knots. We hint here at how this may be done.

%\begin{description}
%\item [structural connectives] 
\subsubsection{Structural connectives} The spatial logics enjoy
structural connectives corresponding, at the logical level, to the
parallel composition ($P | Q$) and new name ($(\nu \; x)P$)
connectives for processes. As illustrated in the examples below, these
connectives are extremely expressive given the shape of our encoding.
%\item [decideable satisfaction]

\subsubsection{Decideable satisfaction}
In \cite{Caires04} the satisfaction relation is shown to be decideable
for a rich class of processes. It further turns out that the image of
the our encoding is a proper subset of that class. This result
provides the basis for an algorithm by which to search for knots
enjoying a given property.
%\item [characteristic formulae]

\subsubsection{Characteristic formulae}
In the same paper \cite{Caires04} , Caires presents a means of calculating
characteristic formulae, selecting equivalence classes of processes
up to a pre--specified depth limit on the support set of names. Composed with our
encoding, this characteristic formula can be used to select
characteristic formulae for knots.
%\end{description}

\subsubsection{Spatial logic formulae}

The grammar below (segmented for comprehension) summarizes the syntax
of spatial logic formulae. We employ illustrative examples in the
sequel to provide an intuitive understanding of their meaning
referring the reader to \cite{Caires04} for a more detailed explication
of the semantics.

\begin{mathpar}
  \inferrule* [lab=boolean] {} {{A,B} \bc T \;|\; \neg A \;|\; A \wedge B \;|\; \eta = \eta'}
  \and
  \inferrule* [lab=spatial] {} {|\; \pzero \;|\; A | B \;|\; x \text{\textregistered} A \;|\; \forall x . A \;|\;  H x . A}
  \and
  \inferrule* [lab=behavioral] {} {|\; \alpha . A}
  \and 
  \inferrule* [lab=recursion] {} {|\; X(\vec{u}) \;|\; \mu X(\vec{u}) . A}
  \and
  \inferrule* [lab=action] {} {\alpha \bc \langle x?(\vec{y}) \rangle \;|\; \langle x!(\vec{y}) \rangle \;|\; \langle \tau \rangle}
  \and 
  \inferrule* [lab=name] {} {\eta \bc x \;|\; \tau}
\end{mathpar} 

% subsection characteristic_formulae (end)   	 

\subsection{Example formulae}\label{sub:example_formulae_} % (fold)

\subsubsection{Crossing as formula.}
% 
% \begin{align*}
%   \frac{d}{dx} \sin x &= \cos x 
%   & \frac{d}{dx} e^x &= e^x \\
%   \frac{d}{dx} \cos x &= - \sin x 
%   & \frac{d}{dx} \log x &= \frac{1}{x} \\
% \end{align*} 

\begin{align*}
 \mu C(x_{0},x_{1},y_{0},y_{1},u).&(\langle x_{0}?(z) \rangle(\langle u! \rangle\langle y_{1}!z \rangle C(x_{0},x_{1},y_{0},y_{1},u)) & \\
  & \wedge \langle y_{1}?(z) \rangle (\langle u! \rangle \langle x_{0}!z \rangle C(x_{0},x_{1},y_{0},y_{1},u)) & \\
  & \wedge \langle x_{1}?(z) \rangle (\langle u? \rangle \langle y_{0}!z \rangle C(x_{0},x_{1},y_{0},y_{1},u)) & \\
  & \wedge \langle y_{0}?(z) \rangle (\langle u? \rangle \langle x_{1}!z \rangle C(x_{0},x_{1},y_{0},y_{1},u))) &
\end{align*}

The lexicographical similarity between the shape of this formulae and
the shape of definition of the process representing a crossing reveals
the intuitive meaning of this formulae. It describes the capabilities
of a process that has the right to represent a crossing. For example
it picks out processes that may perform an input on the port $x_0$ in
its initial menu of capabilities. What differentiates the formula
from the process, however, is that the crossing process is the
smallest candidate to satisfy the formula. Infinitely many other
processes -- with internal behavior hidden behind this interface, so
to speak -- also satisfy this formula. Even this simple formula,
then, can be seen to open a new view onto knots, providing a
computational interpretation of \emph{virtual} knots.

Note that this formula is derived by hand. A similar formula can be
derived by employing Caires' calculation of characteristic formula
\cite{Caires04} to the process representing a crossing. In light of
this discussion, we let
$\meaningof{C}_{\phi}(x0,x1,y0,y1,u)$ denote a formula specifying the
dynamics we wish to capture of a crossing. To guarantee we preserve
the shape of the interface and minimal semantics we demand that
$\meaningof{C}_{\phi}(x0,x1,y0,y1,u) \Rightarrow
\textbf{C}(x0,x1,y0,y1,u)$ where $\textbf{C}(x0,x1,y0,y1,u)$ denotes
the formula above.
                            
\subsubsection{Crossing number constraints.}
The moral content of the context lemma (Lemma \ref{context}) is that the notion of
``locality'' in the Reidemeister moves is effectively captured by the
parallel composition operator of the process calculus. This intuition
extends through the logic. Given a formula,
$\meaningof{C}_{\phi}(x0,x1,y0,y1,u)$, we can use the structural
connectives to specify constraints on crossing numbers, such as at
least $n$ crossings, or exactly $n$ crossings.
\begin{mathpar}
  \inferrule* [lab=at-least-n] {} { K^{\geq n}_{\phi}(\vec{xs},\vec{ys}) := \Pi_{i=0}^{n-1} Hu . \meaningof{C}_{\phi}(xs_i,ys_i,u) | T }
  \and 
  \inferrule* [lab=exactly-n] {} { K^{= n}_{\phi}(\vec{xs},\vec{ys}) := \Pi_{i=0}^{n-1} Hu . \meaningof{C}_{\phi}(xs_i,ys_i,u) | \neg (\forall x_0,y_0,x_1,y_1,u . \meaningof{C}_{\phi}(x_0,y_0,x_1,y_1,u) | T) }
\end{mathpar}

To round out this section, recall that the encoding of an $n$-crossing
knot decomposes into a parallel composition of $n$ \emph{copies} of a
crossing process together with a wiring harness. To specify different
knot classes with the same crossing number amounts to specifying
logical constraints on the wiring harness. In the interest of space,
we defer examples to a forthcoming paper. Suffice it to say that both
the conditions ``alternating knot'' and ``contains the tangle
corresponding to 5/3'' are expressible. For example, it is possible to
calculate the characteristic formula of a process corresponding to the
tangle 5/3 and conjoin it into the classifying formula via the
composition connective of the logic.

Finally, we wish to observe that it is entirely within reason to
contemplate a more domain-specific version of spatial logic tailored
to the shape of processes in the image of the encoding. Such a
domain-specific logic would have a better claim to the title formal
language of knot properties.

% subsection example_formulae_ (end)

% section knots_as_processes (end) 

% section spatial logic via knots (end)

\section{Conclusions and future work}

\paragraph{Testing physical space}
You, gentle reader, may wonder why of all the theorems to be proved
given this set up we pick the one above. In some sense it's hardly
central to quantum mechanics. We see it as central in the sense that
it firmly establishes a notion of physical space arising from a notion
of the equivalence of behavior. Relating bisimulation to a metric is a
big step forward, but one is faced with interpreting the relationship
of that metric space to something more physical. Quantum mechanical
notions of ``physical'' space are still far from intuitive, but by
relating this idea of distance as testing to calculations that predict
physical circumstances we are making a not insignificant step forward
toward an understanding of the physical space we inhabit as
essentially dynamic.

\paragraph{Effectivity and simulation}
One of the observations we have yet to make is that the entire program
spelled out here is effective. We have built various interpreters for
the reflective calculus at work in this interpretation. In principle,
then, we can simulate quantum mechanics on a computer. The place where
the simulation may lose fidelity is the infinitely branching summation
for the annihilator.

In this connection i also want to point out that the evaluation style
calculation of the inner product puts the non-determinism of the
summation right at the heart of measurement. This suggests that
Milner's original reduction-based formulation of the dynamics of his
calculi in terms of sums was not just notationally suggestive of a
notion of measure-and-continue but captured some significant part of
the physics.

\paragraph{Quantum continuations}
In light of this last observation i want to point out that the
predominant account of quantum mechanics is missing a key aspect of a
truly compositional story of the physical situation. In a real lab,
when a measurement is made the observation can be made to feed into
another device that then makes another measurement conditioned on the
results of the first. This means that after the superposition was
collapsed the entire experimental set up remained in
superposition. While QM offers a means of writing this down it doesn't
quite line up well with the well-trodden formulation of computation
and continuation that we see so succinctly expressed in Milner's
calculi. This suggests that there might be advantages to this account
of dynamics waiting to be explored.

\paragraph{Quantum logic}
In this connection, we also note that by virtue of having the
Hennessy-Milner construction, we can pull the construction through the
interpretation of QM. This gives us a natural candidate for a quantum
logic that enjoys an extremely tight connection with it's domain of
interpretation, making the construction much less ad hoc (rather it is
the image of functor!).

\paragraph{Quantum probabiity}
i have questions about the basis of the interpretation of inner
product as probability amplitude. In particular, using which
axiomatization of probability theory does the notion of probability
amplitude earn the right to be so dubbed? In other words, where is the
proof that the operation for calculating a probability amplitude (and
then squaring) satisfies the axioms of what it means to calculate a
probability? Even if such a proof exists (i have yet to find it in the
literature), i wonder if it might not be possible to turn things on
their heads. Can we view the calculation of the probability amplitude
as an axiomatization of probability? If so, then the definition we
give for calculating probability amplitude may provide the basis for
an \emph{effective} theory of probability.

\paragraph{Quantum vs ``biological'' information}
Finally, i want to conclude with a more philosophical observation. At
a recent workshop in which QM was a predominant topic i noticed
something about quantum information. The speaker was giving a riveting
discussion of axiomatic QM and showing how properties of ``no
cloning'' and ``no deleting'' emerged as consequences of the
axiomatization. Theorems of this form are necessary to give us a sense
of confidence that our axioms characterize the physical theory. What
struck me, though, was that if quantum information is neither erasable
nor replicable it is markedly different from \emph{life}. Two of the
things we know about life is that

\begin{itemize}
  \item it ends;
  \item to gain some measure of persistence, to transcend it's
    finitude it is imminently copyable.
\end{itemize}

Both of these qualities are summarized succinctly in the aphorism: all
flesh is grass. For me these two kinds of ``information'' -- call them
quantum and biological -- are end points on a spectrum of strategies
for persistence. At one end, we have those curious entities that enjoy
uniqueness and permanence; at the other, we have those who in the face
of a certain end and an uncertain present make a go of passing
something on. To me one of the more remarkable aspects of the latter
strategy is that in the presence of noise (and certain features of
copying) we get a kind of dynamism, a chance for improvement against a
given persistent condition.

% subsection other_calculi_other_bisimulations_and_geometry_as_behavior (end)




% section conclusion (end)

%\documentclass[12pt]{llncs}
%\documentclass{jktr}

\usepackage[pdftex]{hyperref}                   
\usepackage {listings}
\usepackage {mathpartir}
\usepackage{bcprules}
%\usepackage{listings}
                       
\usepackage{graphicx} 
%\usepackage[margins=2.5cm,nohead,nofoot]{geometry}
%\usepackage{geometry}
\usepackage{amsfonts}
\usepackage{amstext}
\usepackage{latexsym}
\usepackage{amssymb}
\usepackage{color}


%\include{myPreamble}
\include{qm2pi.local} 

%\ifpdf
%\usepackage[pdftex]{graphicx}
%\else
%\usepackage{graphicx}
%\fi

 % \ifpdf
%  \usepackage{pdfsync}
%  \if


%\title{Brief Article}
%\author{David F. Snyder}
%\author{L.G. Meredith}

%\address{Dept. of Math., Texas State University--San Marcos, San Marcos, TX 78666}
       
\pagestyle{empty}


\begin{document}

\lstset{language=[Objective]Caml,frame=shadowbox}

\input{qm2pi.front}

% section front matter (end)

\input{qm2pi.intro} 
 
% section introduction (end)

% \input{qm2pi.knotations} 

% section notation (end)

\input{qm2pi.process.calculi} 

% section concurrent_process_calculi_and_spatial_logics_ (end)
    
%\input{qm2pi.knots2pi} 

%\input{qm2pi.trefoil} 

%\input{qm2pi.mainthm} 

% subsection basic_interpretation (end)

%\input{qm2pi.rho.presentation} 
\subsection{The syntax and semantics of the notation system}\label{sub:the_syntax_and_semantics_of_the_notation_system} % (fold)

We now summarize a technical presentation of the calculus that
embodies our theory of dynamics. The typical presentation of such a
calculus follows the style of giving generators and relations on
them. The grammar, below, describing term constructors, freely
generates the set of processes, $\Proc$. This set is then quotiented
by a relation known as structural congruence and it is over this set
that the notion of dynamics is expressed. This presentation is
essentially that of \cite{MeredithR05} with the addition of
polyadicity and summation. For readability we have relegated some of
the technical subtleties to an appendix.

\subsubsection{Process grammar}\label{subsub:process_grammar}

\begin{mathpar}
  \inferrule* [lab=synchronization] {} {{M} \bc \pzero \;|\; x?F \;|\; x!C }
  \and
  \inferrule* [lab=abstraction] {} {{F} \bc (x)P}
  \and
  \inferrule* [lab=concretion] {} {{C} \bc \langle Q \rangle}
  \and
  \inferrule* [lab=process] {} {{P,Q} \bc M \;| \;P|Q \;|\; @{x}}
  \and
  \inferrule* [lab=name] {} {{x} \bc \quotep{P}}
\end{mathpar} 

Note that $\vec{x}$ (resp. $\vec{P}$) denotes a vector of names
(resp. processes) of length $|\vec{x}|$ (resp. $|\vec{P}|$). We adopt
the following useful abbreviations.

\begin{mathpar}
   x?(\vec{y}).P := x.(\vec{y})P \and  x\clift{\vec{P}} := x.\clift{\vec{P}}
   \and x!(y) := \lift{x}{\dropn{y}}
   \and \Pi_{i=0}^{n-1}P_i := P_0 | \ldots | P_{n-1}
\end{mathpar}

\subsubsection{Structural congruence}

\paragraph{Free and bound names and alpha-equivalence.} At the
core of structural equivalence is alpha-equivalence which identifies
process that are the same up to a change of variable. Formally, we
recognize the distinction between free and bound names. The free names
of a process, $\freenames{P}$, may be calculated recursively as
follows:

\begin{mathpar}
\freenames{\pzero} := \emptyset
  \and \\
  \freenames{x?(y).P} := \{ x \} \cup (\freenames{P} \setminus \{ y \})
  \and 
  \freenames{x!\langle P \rangle} := \{ x \} \cup \{ P \} 
  \and \\
  \freenames{P|Q} := \freenames{P} \cup \freenames{Q}
  \and \\
  \freenames{@{x}} := \{ x \}
\end{mathpar}

$\pi$
$\quotep{\pi}$

$\freenames{-} : \pi \to \mathcal{P}(\quotep{\pi})$

\begin{eqnarray*}
  \freenames{\pzero} & := & \emptyset \\
  \freenames{x?(y).P} & := & \{ x \} \cup (\freenames{P} \setminus \{ y \}) \\
  \freenames{x!\langle P \rangle} & := & \{ x \} \cup \{ P \} \\
  \freenames{P|Q} & := & \freenames{P} \cup \freenames{Q} \\
  \freenames{\dropn{x}} & := & \{ x \}
\end{eqnarray*}

The bound names of a process, $\boundnames{P}$, are those names occurring in $P$
that are not free. For example, in $x?(y).0$, the name $x$ is free, while $y$ is bound.

\begin{mathpar}
  \inferrule* [lab=monoidal-laws] {} { P|Q \equiv Q|P \and P|0 \equiv P \and P|(Q|R) \equiv (P|Q)|R }
\end{mathpar}

\begin{mathpar}
  \inferrule* [lab=alpha-equivalence] {} { (x)P \equiv (y)P\{y/x\} \and y \not\in \freenames{P} }
\end{mathpar}

\begin{definition}
Then two processes, $P,Q$, are alpha-equivalent if $P = Q\{\vec{y}/\vec{x}\}$ for
some $\vec{x} \in \boundnames{Q},\vec{y} \in \boundnames{P}$, where $Q\{\vec{y}/\vec{x}\}$
denotes the capture-avoiding substitution of $\vec{y}$ for $\vec{x}$ in $Q$.
\end{definition}

\begin{definition}
  The {\em structural congruence} \cite{SangiorgiWalker} , $\equiv$,
  between processes is the least congruence containing
  alpha-equivalence, satisfying the abelian monoid laws
  (associativity, commutativity and $\pzero$ as identity) for parallel
  composition $|$ and for summation $+$.
\end{definition}

\subsection{Name equivalence}

We take name equivalence, written $\nameeq$, to be the smallest
equivalence relation generated by the following rules.

\begin{mathpar}
\inferrule*[lab=Quote-drop]
{ }
{ \quotep{@{x}} \nameeq x }

\inferrule*[lab=Struct-equiv]
{ P \scong Q }
{ \quotep{P} \nameeq \quotep{Q} }
\end{mathpar}

The astute reader will have noticed that the mutual recursion of names
and processes imposes a mutual recursion on alpha-equivalence and
structural equivalence via name-equivalence. Fortunately, all of this
works out pleasantly and we may calculate in the natural way, free of
concern. The reader interested in the details is referred to the
appendix \ref{appendix:rho_details}.

\subsection{Substitution}

We use $\Proc$ for the set of processes, $\QProc$ for the set of
names, and $\id{\{}\vec{y} / \vec{x} \id{\}}$ to denote partial maps,
$s : \QProc \rightarrow \QProc$. A map, $s$ lifts, uniquely, to a map
on process terms, $\widehat{s} : \Proc \rightarrow \Proc$ by the
following equations.

\begin{mathpar}
  (0) \psubstp{Q}{P} := 0 \\
  (R \juxtap S) \psubstp{Q}{P}
  :=    
  (R)\psubstp{Q}{P} \juxtap (S) \psubstp{Q}{P} \\
  (x?(y).R) \psubstp{Q}{P}    
  :=    
  (x)\substp{Q}{P} (z)\concat( (R \psubstn{z}{y}) \psubstp{Q}{P} ) \\
  (\lift{x}{R}) \psubstp{Q}{P}  
  :=
  \lift{(x)\substp{Q}{P}}{ R \psubstp{Q}{P} } \\
%   (\dropn{x})  \psubstp{Q}{P}       
%   := 
%   \left\{ 
%     \begin{array}{ccc} 
%       \dropn{\quotep{Q}} & & x \nameeq \quotep{P} \\
%       \dropn{x} & & otherwise \\
%     \end{array}
%   \right. 
  (\dropn{x})  \psubstp{Q}{P}       
  := 
  \left\{ 
    \begin{array}{ccc} 
      Q & & x \nameeq \quotep{P} \\
      \dropn{x} & & otherwise \\
    \end{array}
  \right.
\end{mathpar}
 

where

\begin{eqnarray}
  (x)\id{\{} \lpquote Q \rpquote / \lpquote P \rpquote \id{\}}            = 
  \left\{ 
    \begin{array}{ccc}
      \lpquote Q \rpquote & & x \nameeq \lpquote P \rpquote \\
      x & & otherwise \\
    \end{array}
  \right. \nonumber
\end{eqnarray}

and $z$ is chosen distinct from $\quotep{P}$, $\quotep{Q}$, the free
names in $Q$, and all the names in $R$. Our $\alpha$-equivalence will
be built in the standard way from this substitution.

\begin{remark}\label{rem:no_self_referential_names}
  One consequence of these definitions is that $\forall P. \quotep{P}
  \not\in \freenames{P}$.
\end{remark}

\subsection{ Dynamic quote: an example }

Anticipating something of what's to come, consider applying the
substitution, $\widehat{\id{\{}u / z \id{\}}}$, to the following pair
of processes, $\lift{w}{y!(z)}$ and $w[ \lpquote y!(z) \rpquote ]$.

\begin{eqnarray}
	\lift{w}{y!(z)}\widehat{\id{\{}u / z \id{\}}}
		& = &
		\lift{w}{y!(u)} \nonumber\\
	w[ \lpquote y!(z) \rpquote ] \widehat{ \id{\{}u / z \id{\}} }
		& = &
		w[ \lpquote y!(z) \rpquote ] \nonumber
\end{eqnarray}

Because the body of the process between quotes is impervious to
substitution, we get radically different answers. In fact, by
examining the first process in an input context,
e.g. $x?(z).\lift{w}{y!(z)}$, we see that the process under the lift
operator may be shaped by prefixed inputs binding a name inside it. In
this sense, the lift operator will be seen as a way to dynamically
construct processes before reifying them as names.

Finally equipped with these standard features we can present the
dynamics of the calculus.

\subsubsection{Operational semantics} 

Finally, we introduce the computational dynamics. What marks these
algebras as distinct from other more traditionally studied algebraic
structures, e.g. vector spaces or polynomial rings, is the manner in
which dynamics is captured. In traditional structures, dynamics is typically
expressed through morphisms between such structures, as in linear maps
between vector spaces or morphisms between rings. In algebras
associated with the semantics of computation, the dynamics is
expressed as part of the algebraic structure itself, through a
reduction reduction relation typically denoted by $\red$. Below, we
give a recursive presentation of this relation for the calculus used
in the encoding.

$\red \subseteq \pi \times \pi$
$\red : \pi \to \mathcal{P}(\pi)$

\begin{mathpar}
  \inferrule* [lab=Comm] { \textsf{match}( x_{src}, x_{trgt} ) } { x_{trgt}?(y)P \; | \; x_{src}!\langle {Q} \rangle \red P\{\quotep{Q}/y}\} }
  \and \\
  \inferrule* [lab=Par] {{P} \red {P}'} {{{P} | {Q}} \red {{P}' | {Q}}}
  \and
  \inferrule* [lab=Equiv]{{{P} \scong {P}'} \andalso {{P}' \red {Q}'} \andalso {{Q}' \scong {Q}}}{{P} \red {Q}}
\end{mathpar}

\begin{eqnarray*}
  match_{\equiv} (\quotep{P},\quotep{Q}) & := & P \equiv Q \\
  match_{\dagger}(\quotep{P},\quotep{Q}) & := & \forall R. P|Q \red^{*} R => R \red^{*} 0 \\
  match_{K}(\quotep{P},\quotep{Q}) & := & K \mbox{ for some context } K
\end{eqnarray*}

$u?(x)P | u!\langle Q \rangle \red P\{\quotep{Q}/x\}$

%We write $\wred$ for $\red^*$, and $P\red$ if $\exists Q $ such that $ P \red Q$.
We write $P\red$ if $\exists Q $ such that $ P \red Q$ and $P\not\red$, otherwise.

\section{Replication}

As mentioned before, it is known that replication (and hence
recursion) can be implemented in a higher-order process algebra
\cite{SangiorgiWalker}. As our first example of calculation with the
machinery thus far presented we give the construction explicitly in
the {\rhoc}.

\begin{eqnarray}
	D_{x} & := & \prefix{x}{y}{(\binpar{\outputp{x}{y}}{@{y}})} \nonumber\\
	\bangp_{x}{P} & := & \binpar{{x}!\langle{\binpar{D_{x}}{P}}\rangle}{D_{x}} \nonumber
\end{eqnarray}

\begin{eqnarray}
	\bangp_{x}{P} & & \nonumber\\
	=
	& {x}!\langle{(\prefix{x}{y}{(\outputp{x}{y} | @{y})) | P}}\rangle 
	      | \prefix{x}{y}{(\outputp{x}{y} | @{y})} & \nonumber\\
	\red
	& (\outputp{x}{y} | @{y})\substn{\quotep{(\prefix{x}{y}{(@{y} | \outputp{x}{y})) | P}}}{y} & \nonumber\\
	=
	& \outputp{x}{\quotep{(\prefix{x}{y}{(\outputp{x}{y} | @{y})) | P}}}
	  | {(\prefix{x}{y}{(\outputp{x}{y} | @{y})) | P}} & \nonumber\\
	\red
	& \ldots & \nonumber\\
	\red^*
	& P | P | \ldots & \nonumber
\end{eqnarray}

Of course, this encoding, as an implementation, runs away, unfolding
$\bangp{P}$ eagerly. A lazier and more implementable replication
operator, restricted to input-guarded processes, may be obtained as follows.

\begin{eqnarray}
\bangp{\prefix{u}{v}{P}} 
	:= 
	\binpar{\lift{x}{\prefix{u}{v}{(\binpar{D(x)}{P})}}}{D(x)} \nonumber
\end{eqnarray}

\begin{remark}
  Note that the lazier definition still does not deal with summation
  or mixed summation (i.e. sums over input and output). The reader is
  invited to construct definitions of replication that deal with these
  features. 

  Further, the definitions are parameterized in a name, $x$. Can you,
  gentle reader, make a definition that eliminates this parameter and
  guarantees no accidental interaction between the replication
  machinery and the process being replicated -- i.e. no accidental
  sharing of names used by the process to get its work done and the
  name(s) used by the replication to effect copying. This latter
  revision of the definition of replication is crucial to obtaining
  the expected identity $!!P \sim !P$.
\end{remark}

\begin{remark}\label{rem:paradoxical_combinator}
  The reader familiar with the lambda calculus will have noticed the
  similarity between $D$ and the paradoxical combinator.

  [Ed. note: the existence of this seems to suggest we have to be more
  restrictive on the set of processes and names we admit if we are to
  support no-cloning.]
\end{remark}

\subsubsection{Bisimulation}

The computational dynamics gives rise to another kind of equivalence,
the equivalence of computational behavior. As previously mentioned
this is typically captured \emph{via} some form of bisimulation.

% The notion we use in this paper is weak barbed bisimulation
% \cite{milner91polyadicpi}.

The notion we use in this paper is derived from weak barbed
bisimulation \cite{milner91polyadicpi}. 

\begin{definition}
An \emph{observation relation}, $\downarrow_{\mathcal N}$, over a set
of names, $\mathcal N$, is the smallest relation satisfying the rules
below.

\infrule[Out-barb]{y \in {\mathcal N}, \; x \nameeq y}
		  {\outputp{x}{v} \downarrow_{\mathcal N} x}
\infrule[Par-barb]{\mbox{$P\downarrow_{\mathcal N} x$ or $Q\downarrow_{\mathcal N} x$}}
		  {\binpar{P}{Q} \downarrow_{\mathcal N} x}

We write $P \Downarrow_{\mathcal N} x$ if there is $Q$ such that 
$P \wred Q$ and $Q \downarrow_{\mathcal N} x$.
\end{definition}

\begin{definition}
%\label{def.bbisim}
An  ${\mathcal N}$-\emph{barbed bisimulation} over a set of names, ${\mathcal N}$, is a symmetric binary relation 
${\mathcal S}_{\mathcal N}$ between agents such that $P\rel{S}_{\mathcal N}Q$ implies:
\begin{enumerate}
\item If $P \red P'$ then $Q \wred Q'$ and $P'\rel{S}_{\mathcal N} Q'$.
\item If $P\downarrow_{\mathcal N} x$, then $Q\Downarrow_{\mathcal N} x$.
\end{enumerate}
$P$ is ${\mathcal N}$-barbed bisimilar to $Q$, written
$P \wbbisim_{\mathcal N} Q$, if $P \rel{S}_{\mathcal N} Q$ for some ${\mathcal N}$-barbed bisimulation ${\mathcal S}_{\mathcal N}$.
\end{definition}

$\mathcal{R} \subseteq \pi \times \pi$

$P \mathcal{R} Q => \forall P'. P \red P' \Rightarrow \exists Q'. Q \red Q', P' \mathcal{R} Q'$

$P \vdash x \Rightarrow Q \vdash x$

\begin{mathpar}
  \inferrule*[lab=Out-barb]{x \nameeq y}{{y}!\langle{Q}\rangle \vdash x}
  \and
  \inferrule*[lab=Par-barb]{\mbox{$P\vdash x$ or $Q\vdash x$}}{\binpar{P}{Q} \vdash x}
\end{mathpar}

\subsubsection{Contexts}

One of the principle advantages of computational calculi like the
$\pi$-calculus is a well-defined notion of context,
contextual-equivalence and a correlation between
contextual-equivalence and notions of bisimulation. The notion of
context allows the decomposition of a process into (sub-)process and
its syntactic environment, its context. Thus, a context may be
thought of as a process with a ``hole'' (written $\Box$) in it. The
application of a context $M$ to a process $P$, written $M[P]$, is
tantamount to filling the hole in $M$ with $P$. In this paper we do
not need the full weight of this theory, but do make use of the notion
of context in the proof the main theorem. 

\begin{mathpar}
  \inferrule* [lab=summation] {} {{M_{M},M_{N}} \bc \Box \;|\; x.M_{A} \;|\; M_{M}+M_{N}}
  \and
  \inferrule* [lab=agent] {} {{M_{A}} \bc (\vec{x})M_{P} \;| \; \clift{P_0,\ldots,M_{P},\ldots,P_N}}
  \and \\
  \inferrule* [lab=process] {} {{M_{P}} \bc M_{N} \;| \;P|M_{P} }
\end{mathpar} 

\begin{mathpar}
  \inferrule* [lab=sychronization] {} {M_{N} \bc \Box \;|\; x?M_{F} \;|\; x!M_{C}}
  \and
  \inferrule* [lab=abstraction] {} {{M_{F}} \bc (x)M_{P} }
  \and
  \inferrule* [lab=concretion] {} {{M_{C}} \bc \langle M_{P} \rangle }
  \and \\
  \inferrule* [lab=process] {} {{M_{P}} \bc M_{N} \;| \;P|M_{P} }
\end{mathpar}

\begin{definition}[contextual application] Given a context $M$, and
  process $P$, we define the \emph{contextual application}, $M[P] :=
  M\{P/\Box\}$. That is, the contextual application of M to P is the
  substitution of $P$ for $\Box$ in $M$.
\end{definition}

$\meaningof{-} : L \to \mathcal{P}(\pi)$

\begin{mathpar}
  \inferrule* [lab=collection] {} {\meaningof{true} = \pi, \and \meaningof{~E} = \pi \setminus \meaningof{E}, \and \meaningof{E_{1} \& E_{2}} = \meaningof{E_{1}} \cap \meaningof{E_{2}}}
\end{mathpar}

\begin{mathpar}
  \inferrule* [lab=structure] {} {\meaningof{0} = \{ P \in \pi | P \equiv 0 \}, \and \\ \meaningof{E_1 | E_2} = \{ P \in \pi | P \equiv P_{1} | P_{2}, P_{1} \in \meaningof{E_{1}}, P_{2} \in \meaningof{E_2}\} }
\end{mathpar}

\begin{mathpar}
 \inferrule* [lab=behavior] {} {\meaningof{\langle a?b \rangle E} = \{ P \in \pi | P \equiv Q | u?(y)P', \\ \and \\\\ \and \\ \;\;\; u \in \meaningof{a}, \forall z.P'\{z/y\} \in \meaningof{E\{z/b\}}\}, \and \\ \meaningof{a!E} = \{ P \in \pi | P \equiv Q | x!\langle P' \rangle, x \in \meaningof{a} P' \in \meaningof{E}\} }
\end{mathpar}

\begin{mathpar}
 \inferrule* [lab=nominal] {} {\meaningof{\quotep{E}} = \{ \quotep{P} \in \quotep{\pi} | P \in \meaningof{E} \}, \and \meaningof{\quotep{P}} = \{ \quotep{Q} \in \quotep{\pi} | P \equiv Q \} \and \\ \meaningof{@\quotep{E}} = \{ P \in \pi | P \equiv @x, x \in \meaningof{E} \}}
\end{mathpar}

\begin{eqnarray*}
  \\
  \meaningof{-} : TS \to ST
\end{eqnarray*}

\begin{eqnarray*}
  \\
  L : TS \to ST
\end{eqnarray*}

\begin{eqnarray*}
  \\
  P \models E \iff P \in \meaningof{E}
\end{eqnarray*}

\begin{eqnarray*}
  P \approx_{L} Q \iff \forall E \in L. P \models E \iff Q \models E
\end{eqnarray*}

\begin{eqnarray*}
  P \approx_{K} Q
\end{eqnarray*}

\begin{eqnarray*}
  P \approx Q
\end{eqnarray*}

$\approx_{K} = \approx = \approx_{L}$

\subsubsection{Contextual duality}

Note that contexts extend the quotation operation to a family of
operations from processes to names. Given a context, $M$, we can
define a \emph{nominal context}, $\quotep{M}$ by $\quotep{M}[P] :=
\quotep{M[P]}$. To foreshadow what is to come we observe that these
operations enjoy a duality with processes very much like the duality
between vectors and maps from vectors to scalars.

Further, because the calculus is essentially higher-order, we have a
correspondence between contexts and processes. More specifically,
given a name $x$ and a context $M$ we can construct $M^{*}_{x}$ such
that 

\begin{mathpar}
  M^{*}_{x} | \lift{x}{P} \red M[P]
\end{mathpar}

namely,

\begin{mathpar}
  M^{*}_{x} := x?(u).M[\dropn{u}]
\end{mathpar}

The dependence of $M^{*}_{x}$ on a name makes it an abstraction, 

\begin{mathpar}
  M^{*} := (x)x?(u).M[\dropn{u}]
\end{mathpar}

\subsection{Additional notation}

It will sometimes be convenient to denote the process a name
quotes. We already have the notation $x = \quotep{P}$, but it will be
convenient to introduce an alternate notation, $\procn{x}$, when we
want to emphasize the connection to the use of the name. Note that, by
virtue of name equivalence, $\quotep{\procn{x}} \nameeq x$; so, the
notation is consistent with previous definitions.

Further, because names have structure it is possible to effect
substitutions on the basis of that structure. This means we need to
upgrade our notation for substitutions, which we accomplish by
adapting comprehension notation. Thus,

\begin{mathpar}
  P\{ y / x : x \in S \}
\end{mathpar}

is interpreted to mean the process derived from P by replacing (in a
capture-avoiding manner) each occurrence of $x$ in $S$ by $y$. For example,

\begin{mathpar}
  P\{ \quotep{\procn{x}|\procn{x}} / x : x \in \freenames{P} \}
\end{mathpar}

will replace each (occurrence) of a free name $x$ in $P$ by
$\quotep{\procn{x}|\procn{x}}$.

Also, we will avail ourselves of the notation $x^{L}$ and $x^{R}$ to
denote injections of a name into disjoint copies of the name
space. There are numerous ways to accomplish this. One example can be
found in \cite{MeredithR05}. This notation overloads to vectors of
names: $\vec{x}^{\pi} := (x_{i}^{\pi} \; : \; 0 \leq i < |\vec{x}| )$ where $\pi \in \{L,R\}$.

We also use $P^{\Box} := P|\Box$.

In \cite{MeredithR05} an interpretation of the new operator is
given. It turns out that there are several possible interpretations
all enjoying the requisite algebraic properties of the operator (see
\cite{milner91polyadicpi}). We will therefore make liberal use of
$(\nu\; \vec{x})P$.

% subsection the_syntax_and_semantics_of_the_notation_system (end)   

\input{qm2pi.qmops} 

\input{qm2pi.sterngerlach} 

\input{qm2pi.metric} 

% section concurrent_process_calculi (end)

%\input{qm2pi.proofsketch}

% section proof sketch (end)

%\input{qm2pi.slviaknots} 

% section spatial logic via knots (end)

\input{qm2pi.conclusion}

% section conclusion (end)

%\input{qm2pi.dtcodes} 

% section wiring algorithm (end)

\input{qm2pi.ack} 

% section acknowledgments (end)

\newpage


\bibliographystyle{plain}   
\bibliography{../../biblios/main.bib}

\input{qm2pi.rhodetails}

\end{document}

 

% section wiring algorithm (end)

\documentclass[12pt]{llncs}
%\documentclass{jktr}

\usepackage[pdftex]{hyperref}                   
\usepackage {listings}
\usepackage {mathpartir}
\usepackage{bcprules}
%\usepackage{listings}
                       
\usepackage{graphicx} 
%\usepackage[margins=2.5cm,nohead,nofoot]{geometry}
%\usepackage{geometry}
\usepackage{amsfonts}
\usepackage{amstext}
\usepackage{latexsym}
\usepackage{amssymb}
\usepackage{color}


%\include{myPreamble}
\include{qm2pi.local} 

%\ifpdf
%\usepackage[pdftex]{graphicx}
%\else
%\usepackage{graphicx}
%\fi

 % \ifpdf
%  \usepackage{pdfsync}
%  \if


%\title{Brief Article}
%\author{David F. Snyder}
%\author{L.G. Meredith}

%\address{Dept. of Math., Texas State University--San Marcos, San Marcos, TX 78666}
       
\pagestyle{empty}


\begin{document}

\lstset{language=[Objective]Caml,frame=shadowbox}

\input{qm2pi.front}

% section front matter (end)

\input{qm2pi.intro} 
 
% section introduction (end)

% \input{qm2pi.knotations} 

% section notation (end)

\input{qm2pi.process.calculi} 

% section concurrent_process_calculi_and_spatial_logics_ (end)
    
%\input{qm2pi.knots2pi} 

%\input{qm2pi.trefoil} 

%\input{qm2pi.mainthm} 

% subsection basic_interpretation (end)

%\input{qm2pi.rho.presentation} 
\subsection{The syntax and semantics of the notation system}\label{sub:the_syntax_and_semantics_of_the_notation_system} % (fold)

We now summarize a technical presentation of the calculus that
embodies our theory of dynamics. The typical presentation of such a
calculus follows the style of giving generators and relations on
them. The grammar, below, describing term constructors, freely
generates the set of processes, $\Proc$. This set is then quotiented
by a relation known as structural congruence and it is over this set
that the notion of dynamics is expressed. This presentation is
essentially that of \cite{MeredithR05} with the addition of
polyadicity and summation. For readability we have relegated some of
the technical subtleties to an appendix.

\subsubsection{Process grammar}\label{subsub:process_grammar}

\begin{mathpar}
  \inferrule* [lab=synchronization] {} {{M} \bc \pzero \;|\; x?F \;|\; x!C }
  \and
  \inferrule* [lab=abstraction] {} {{F} \bc (x)P}
  \and
  \inferrule* [lab=concretion] {} {{C} \bc \langle Q \rangle}
  \and
  \inferrule* [lab=process] {} {{P,Q} \bc M \;| \;P|Q \;|\; @{x}}
  \and
  \inferrule* [lab=name] {} {{x} \bc \quotep{P}}
\end{mathpar} 

Note that $\vec{x}$ (resp. $\vec{P}$) denotes a vector of names
(resp. processes) of length $|\vec{x}|$ (resp. $|\vec{P}|$). We adopt
the following useful abbreviations.

\begin{mathpar}
   x?(\vec{y}).P := x.(\vec{y})P \and  x\clift{\vec{P}} := x.\clift{\vec{P}}
   \and x!(y) := \lift{x}{\dropn{y}}
   \and \Pi_{i=0}^{n-1}P_i := P_0 | \ldots | P_{n-1}
\end{mathpar}

\subsubsection{Structural congruence}

\paragraph{Free and bound names and alpha-equivalence.} At the
core of structural equivalence is alpha-equivalence which identifies
process that are the same up to a change of variable. Formally, we
recognize the distinction between free and bound names. The free names
of a process, $\freenames{P}$, may be calculated recursively as
follows:

\begin{mathpar}
\freenames{\pzero} := \emptyset
  \and \\
  \freenames{x?(y).P} := \{ x \} \cup (\freenames{P} \setminus \{ y \})
  \and 
  \freenames{x!\langle P \rangle} := \{ x \} \cup \{ P \} 
  \and \\
  \freenames{P|Q} := \freenames{P} \cup \freenames{Q}
  \and \\
  \freenames{@{x}} := \{ x \}
\end{mathpar}

$\pi$
$\quotep{\pi}$

$\freenames{-} : \pi \to \mathcal{P}(\quotep{\pi})$

\begin{eqnarray*}
  \freenames{\pzero} & := & \emptyset \\
  \freenames{x?(y).P} & := & \{ x \} \cup (\freenames{P} \setminus \{ y \}) \\
  \freenames{x!\langle P \rangle} & := & \{ x \} \cup \{ P \} \\
  \freenames{P|Q} & := & \freenames{P} \cup \freenames{Q} \\
  \freenames{\dropn{x}} & := & \{ x \}
\end{eqnarray*}

The bound names of a process, $\boundnames{P}$, are those names occurring in $P$
that are not free. For example, in $x?(y).0$, the name $x$ is free, while $y$ is bound.

\begin{mathpar}
  \inferrule* [lab=monoidal-laws] {} { P|Q \equiv Q|P \and P|0 \equiv P \and P|(Q|R) \equiv (P|Q)|R }
\end{mathpar}

\begin{mathpar}
  \inferrule* [lab=alpha-equivalence] {} { (x)P \equiv (y)P\{y/x\} \and y \not\in \freenames{P} }
\end{mathpar}

\begin{definition}
Then two processes, $P,Q$, are alpha-equivalent if $P = Q\{\vec{y}/\vec{x}\}$ for
some $\vec{x} \in \boundnames{Q},\vec{y} \in \boundnames{P}$, where $Q\{\vec{y}/\vec{x}\}$
denotes the capture-avoiding substitution of $\vec{y}$ for $\vec{x}$ in $Q$.
\end{definition}

\begin{definition}
  The {\em structural congruence} \cite{SangiorgiWalker} , $\equiv$,
  between processes is the least congruence containing
  alpha-equivalence, satisfying the abelian monoid laws
  (associativity, commutativity and $\pzero$ as identity) for parallel
  composition $|$ and for summation $+$.
\end{definition}

\subsection{Name equivalence}

We take name equivalence, written $\nameeq$, to be the smallest
equivalence relation generated by the following rules.

\begin{mathpar}
\inferrule*[lab=Quote-drop]
{ }
{ \quotep{@{x}} \nameeq x }

\inferrule*[lab=Struct-equiv]
{ P \scong Q }
{ \quotep{P} \nameeq \quotep{Q} }
\end{mathpar}

The astute reader will have noticed that the mutual recursion of names
and processes imposes a mutual recursion on alpha-equivalence and
structural equivalence via name-equivalence. Fortunately, all of this
works out pleasantly and we may calculate in the natural way, free of
concern. The reader interested in the details is referred to the
appendix \ref{appendix:rho_details}.

\subsection{Substitution}

We use $\Proc$ for the set of processes, $\QProc$ for the set of
names, and $\id{\{}\vec{y} / \vec{x} \id{\}}$ to denote partial maps,
$s : \QProc \rightarrow \QProc$. A map, $s$ lifts, uniquely, to a map
on process terms, $\widehat{s} : \Proc \rightarrow \Proc$ by the
following equations.

\begin{mathpar}
  (0) \psubstp{Q}{P} := 0 \\
  (R \juxtap S) \psubstp{Q}{P}
  :=    
  (R)\psubstp{Q}{P} \juxtap (S) \psubstp{Q}{P} \\
  (x?(y).R) \psubstp{Q}{P}    
  :=    
  (x)\substp{Q}{P} (z)\concat( (R \psubstn{z}{y}) \psubstp{Q}{P} ) \\
  (\lift{x}{R}) \psubstp{Q}{P}  
  :=
  \lift{(x)\substp{Q}{P}}{ R \psubstp{Q}{P} } \\
%   (\dropn{x})  \psubstp{Q}{P}       
%   := 
%   \left\{ 
%     \begin{array}{ccc} 
%       \dropn{\quotep{Q}} & & x \nameeq \quotep{P} \\
%       \dropn{x} & & otherwise \\
%     \end{array}
%   \right. 
  (\dropn{x})  \psubstp{Q}{P}       
  := 
  \left\{ 
    \begin{array}{ccc} 
      Q & & x \nameeq \quotep{P} \\
      \dropn{x} & & otherwise \\
    \end{array}
  \right.
\end{mathpar}
 

where

\begin{eqnarray}
  (x)\id{\{} \lpquote Q \rpquote / \lpquote P \rpquote \id{\}}            = 
  \left\{ 
    \begin{array}{ccc}
      \lpquote Q \rpquote & & x \nameeq \lpquote P \rpquote \\
      x & & otherwise \\
    \end{array}
  \right. \nonumber
\end{eqnarray}

and $z$ is chosen distinct from $\quotep{P}$, $\quotep{Q}$, the free
names in $Q$, and all the names in $R$. Our $\alpha$-equivalence will
be built in the standard way from this substitution.

\begin{remark}\label{rem:no_self_referential_names}
  One consequence of these definitions is that $\forall P. \quotep{P}
  \not\in \freenames{P}$.
\end{remark}

\subsection{ Dynamic quote: an example }

Anticipating something of what's to come, consider applying the
substitution, $\widehat{\id{\{}u / z \id{\}}}$, to the following pair
of processes, $\lift{w}{y!(z)}$ and $w[ \lpquote y!(z) \rpquote ]$.

\begin{eqnarray}
	\lift{w}{y!(z)}\widehat{\id{\{}u / z \id{\}}}
		& = &
		\lift{w}{y!(u)} \nonumber\\
	w[ \lpquote y!(z) \rpquote ] \widehat{ \id{\{}u / z \id{\}} }
		& = &
		w[ \lpquote y!(z) \rpquote ] \nonumber
\end{eqnarray}

Because the body of the process between quotes is impervious to
substitution, we get radically different answers. In fact, by
examining the first process in an input context,
e.g. $x?(z).\lift{w}{y!(z)}$, we see that the process under the lift
operator may be shaped by prefixed inputs binding a name inside it. In
this sense, the lift operator will be seen as a way to dynamically
construct processes before reifying them as names.

Finally equipped with these standard features we can present the
dynamics of the calculus.

\subsubsection{Operational semantics} 

Finally, we introduce the computational dynamics. What marks these
algebras as distinct from other more traditionally studied algebraic
structures, e.g. vector spaces or polynomial rings, is the manner in
which dynamics is captured. In traditional structures, dynamics is typically
expressed through morphisms between such structures, as in linear maps
between vector spaces or morphisms between rings. In algebras
associated with the semantics of computation, the dynamics is
expressed as part of the algebraic structure itself, through a
reduction reduction relation typically denoted by $\red$. Below, we
give a recursive presentation of this relation for the calculus used
in the encoding.

$\red \subseteq \pi \times \pi$
$\red : \pi \to \mathcal{P}(\pi)$

\begin{mathpar}
  \inferrule* [lab=Comm] { \textsf{match}( x_{src}, x_{trgt} ) } { x_{trgt}?(y)P \; | \; x_{src}!\langle {Q} \rangle \red P\{\quotep{Q}/y}\} }
  \and \\
  \inferrule* [lab=Par] {{P} \red {P}'} {{{P} | {Q}} \red {{P}' | {Q}}}
  \and
  \inferrule* [lab=Equiv]{{{P} \scong {P}'} \andalso {{P}' \red {Q}'} \andalso {{Q}' \scong {Q}}}{{P} \red {Q}}
\end{mathpar}

\begin{eqnarray*}
  match_{\equiv} (\quotep{P},\quotep{Q}) & := & P \equiv Q \\
  match_{\dagger}(\quotep{P},\quotep{Q}) & := & \forall R. P|Q \red^{*} R => R \red^{*} 0 \\
  match_{K}(\quotep{P},\quotep{Q}) & := & K \mbox{ for some context } K
\end{eqnarray*}

$u?(x)P | u!\langle Q \rangle \red P\{\quotep{Q}/x\}$

%We write $\wred$ for $\red^*$, and $P\red$ if $\exists Q $ such that $ P \red Q$.
We write $P\red$ if $\exists Q $ such that $ P \red Q$ and $P\not\red$, otherwise.

\section{Replication}

As mentioned before, it is known that replication (and hence
recursion) can be implemented in a higher-order process algebra
\cite{SangiorgiWalker}. As our first example of calculation with the
machinery thus far presented we give the construction explicitly in
the {\rhoc}.

\begin{eqnarray}
	D_{x} & := & \prefix{x}{y}{(\binpar{\outputp{x}{y}}{@{y}})} \nonumber\\
	\bangp_{x}{P} & := & \binpar{{x}!\langle{\binpar{D_{x}}{P}}\rangle}{D_{x}} \nonumber
\end{eqnarray}

\begin{eqnarray}
	\bangp_{x}{P} & & \nonumber\\
	=
	& {x}!\langle{(\prefix{x}{y}{(\outputp{x}{y} | @{y})) | P}}\rangle 
	      | \prefix{x}{y}{(\outputp{x}{y} | @{y})} & \nonumber\\
	\red
	& (\outputp{x}{y} | @{y})\substn{\quotep{(\prefix{x}{y}{(@{y} | \outputp{x}{y})) | P}}}{y} & \nonumber\\
	=
	& \outputp{x}{\quotep{(\prefix{x}{y}{(\outputp{x}{y} | @{y})) | P}}}
	  | {(\prefix{x}{y}{(\outputp{x}{y} | @{y})) | P}} & \nonumber\\
	\red
	& \ldots & \nonumber\\
	\red^*
	& P | P | \ldots & \nonumber
\end{eqnarray}

Of course, this encoding, as an implementation, runs away, unfolding
$\bangp{P}$ eagerly. A lazier and more implementable replication
operator, restricted to input-guarded processes, may be obtained as follows.

\begin{eqnarray}
\bangp{\prefix{u}{v}{P}} 
	:= 
	\binpar{\lift{x}{\prefix{u}{v}{(\binpar{D(x)}{P})}}}{D(x)} \nonumber
\end{eqnarray}

\begin{remark}
  Note that the lazier definition still does not deal with summation
  or mixed summation (i.e. sums over input and output). The reader is
  invited to construct definitions of replication that deal with these
  features. 

  Further, the definitions are parameterized in a name, $x$. Can you,
  gentle reader, make a definition that eliminates this parameter and
  guarantees no accidental interaction between the replication
  machinery and the process being replicated -- i.e. no accidental
  sharing of names used by the process to get its work done and the
  name(s) used by the replication to effect copying. This latter
  revision of the definition of replication is crucial to obtaining
  the expected identity $!!P \sim !P$.
\end{remark}

\begin{remark}\label{rem:paradoxical_combinator}
  The reader familiar with the lambda calculus will have noticed the
  similarity between $D$ and the paradoxical combinator.

  [Ed. note: the existence of this seems to suggest we have to be more
  restrictive on the set of processes and names we admit if we are to
  support no-cloning.]
\end{remark}

\subsubsection{Bisimulation}

The computational dynamics gives rise to another kind of equivalence,
the equivalence of computational behavior. As previously mentioned
this is typically captured \emph{via} some form of bisimulation.

% The notion we use in this paper is weak barbed bisimulation
% \cite{milner91polyadicpi}.

The notion we use in this paper is derived from weak barbed
bisimulation \cite{milner91polyadicpi}. 

\begin{definition}
An \emph{observation relation}, $\downarrow_{\mathcal N}$, over a set
of names, $\mathcal N$, is the smallest relation satisfying the rules
below.

\infrule[Out-barb]{y \in {\mathcal N}, \; x \nameeq y}
		  {\outputp{x}{v} \downarrow_{\mathcal N} x}
\infrule[Par-barb]{\mbox{$P\downarrow_{\mathcal N} x$ or $Q\downarrow_{\mathcal N} x$}}
		  {\binpar{P}{Q} \downarrow_{\mathcal N} x}

We write $P \Downarrow_{\mathcal N} x$ if there is $Q$ such that 
$P \wred Q$ and $Q \downarrow_{\mathcal N} x$.
\end{definition}

\begin{definition}
%\label{def.bbisim}
An  ${\mathcal N}$-\emph{barbed bisimulation} over a set of names, ${\mathcal N}$, is a symmetric binary relation 
${\mathcal S}_{\mathcal N}$ between agents such that $P\rel{S}_{\mathcal N}Q$ implies:
\begin{enumerate}
\item If $P \red P'$ then $Q \wred Q'$ and $P'\rel{S}_{\mathcal N} Q'$.
\item If $P\downarrow_{\mathcal N} x$, then $Q\Downarrow_{\mathcal N} x$.
\end{enumerate}
$P$ is ${\mathcal N}$-barbed bisimilar to $Q$, written
$P \wbbisim_{\mathcal N} Q$, if $P \rel{S}_{\mathcal N} Q$ for some ${\mathcal N}$-barbed bisimulation ${\mathcal S}_{\mathcal N}$.
\end{definition}

$\mathcal{R} \subseteq \pi \times \pi$

$P \mathcal{R} Q => \forall P'. P \red P' \Rightarrow \exists Q'. Q \red Q', P' \mathcal{R} Q'$

$P \vdash x \Rightarrow Q \vdash x$

\begin{mathpar}
  \inferrule*[lab=Out-barb]{x \nameeq y}{{y}!\langle{Q}\rangle \vdash x}
  \and
  \inferrule*[lab=Par-barb]{\mbox{$P\vdash x$ or $Q\vdash x$}}{\binpar{P}{Q} \vdash x}
\end{mathpar}

\subsubsection{Contexts}

One of the principle advantages of computational calculi like the
$\pi$-calculus is a well-defined notion of context,
contextual-equivalence and a correlation between
contextual-equivalence and notions of bisimulation. The notion of
context allows the decomposition of a process into (sub-)process and
its syntactic environment, its context. Thus, a context may be
thought of as a process with a ``hole'' (written $\Box$) in it. The
application of a context $M$ to a process $P$, written $M[P]$, is
tantamount to filling the hole in $M$ with $P$. In this paper we do
not need the full weight of this theory, but do make use of the notion
of context in the proof the main theorem. 

\begin{mathpar}
  \inferrule* [lab=summation] {} {{M_{M},M_{N}} \bc \Box \;|\; x.M_{A} \;|\; M_{M}+M_{N}}
  \and
  \inferrule* [lab=agent] {} {{M_{A}} \bc (\vec{x})M_{P} \;| \; \clift{P_0,\ldots,M_{P},\ldots,P_N}}
  \and \\
  \inferrule* [lab=process] {} {{M_{P}} \bc M_{N} \;| \;P|M_{P} }
\end{mathpar} 

\begin{mathpar}
  \inferrule* [lab=sychronization] {} {M_{N} \bc \Box \;|\; x?M_{F} \;|\; x!M_{C}}
  \and
  \inferrule* [lab=abstraction] {} {{M_{F}} \bc (x)M_{P} }
  \and
  \inferrule* [lab=concretion] {} {{M_{C}} \bc \langle M_{P} \rangle }
  \and \\
  \inferrule* [lab=process] {} {{M_{P}} \bc M_{N} \;| \;P|M_{P} }
\end{mathpar}

\begin{definition}[contextual application] Given a context $M$, and
  process $P$, we define the \emph{contextual application}, $M[P] :=
  M\{P/\Box\}$. That is, the contextual application of M to P is the
  substitution of $P$ for $\Box$ in $M$.
\end{definition}

$\meaningof{-} : L \to \mathcal{P}(\pi)$

\begin{mathpar}
  \inferrule* [lab=collection] {} {\meaningof{true} = \pi, \and \meaningof{~E} = \pi \setminus \meaningof{E}, \and \meaningof{E_{1} \& E_{2}} = \meaningof{E_{1}} \cap \meaningof{E_{2}}}
\end{mathpar}

\begin{mathpar}
  \inferrule* [lab=structure] {} {\meaningof{0} = \{ P \in \pi | P \equiv 0 \}, \and \\ \meaningof{E_1 | E_2} = \{ P \in \pi | P \equiv P_{1} | P_{2}, P_{1} \in \meaningof{E_{1}}, P_{2} \in \meaningof{E_2}\} }
\end{mathpar}

\begin{mathpar}
 \inferrule* [lab=behavior] {} {\meaningof{\langle a?b \rangle E} = \{ P \in \pi | P \equiv Q | u?(y)P', \\ \and \\\\ \and \\ \;\;\; u \in \meaningof{a}, \forall z.P'\{z/y\} \in \meaningof{E\{z/b\}}\}, \and \\ \meaningof{a!E} = \{ P \in \pi | P \equiv Q | x!\langle P' \rangle, x \in \meaningof{a} P' \in \meaningof{E}\} }
\end{mathpar}

\begin{mathpar}
 \inferrule* [lab=nominal] {} {\meaningof{\quotep{E}} = \{ \quotep{P} \in \quotep{\pi} | P \in \meaningof{E} \}, \and \meaningof{\quotep{P}} = \{ \quotep{Q} \in \quotep{\pi} | P \equiv Q \} \and \\ \meaningof{@\quotep{E}} = \{ P \in \pi | P \equiv @x, x \in \meaningof{E} \}}
\end{mathpar}

\begin{eqnarray*}
  \\
  \meaningof{-} : TS \to ST
\end{eqnarray*}

\begin{eqnarray*}
  \\
  L : TS \to ST
\end{eqnarray*}

\begin{eqnarray*}
  \\
  P \models E \iff P \in \meaningof{E}
\end{eqnarray*}

\begin{eqnarray*}
  P \approx_{L} Q \iff \forall E \in L. P \models E \iff Q \models E
\end{eqnarray*}

\begin{eqnarray*}
  P \approx_{K} Q
\end{eqnarray*}

\begin{eqnarray*}
  P \approx Q
\end{eqnarray*}

$\approx_{K} = \approx = \approx_{L}$

\subsubsection{Contextual duality}

Note that contexts extend the quotation operation to a family of
operations from processes to names. Given a context, $M$, we can
define a \emph{nominal context}, $\quotep{M}$ by $\quotep{M}[P] :=
\quotep{M[P]}$. To foreshadow what is to come we observe that these
operations enjoy a duality with processes very much like the duality
between vectors and maps from vectors to scalars.

Further, because the calculus is essentially higher-order, we have a
correspondence between contexts and processes. More specifically,
given a name $x$ and a context $M$ we can construct $M^{*}_{x}$ such
that 

\begin{mathpar}
  M^{*}_{x} | \lift{x}{P} \red M[P]
\end{mathpar}

namely,

\begin{mathpar}
  M^{*}_{x} := x?(u).M[\dropn{u}]
\end{mathpar}

The dependence of $M^{*}_{x}$ on a name makes it an abstraction, 

\begin{mathpar}
  M^{*} := (x)x?(u).M[\dropn{u}]
\end{mathpar}

\subsection{Additional notation}

It will sometimes be convenient to denote the process a name
quotes. We already have the notation $x = \quotep{P}$, but it will be
convenient to introduce an alternate notation, $\procn{x}$, when we
want to emphasize the connection to the use of the name. Note that, by
virtue of name equivalence, $\quotep{\procn{x}} \nameeq x$; so, the
notation is consistent with previous definitions.

Further, because names have structure it is possible to effect
substitutions on the basis of that structure. This means we need to
upgrade our notation for substitutions, which we accomplish by
adapting comprehension notation. Thus,

\begin{mathpar}
  P\{ y / x : x \in S \}
\end{mathpar}

is interpreted to mean the process derived from P by replacing (in a
capture-avoiding manner) each occurrence of $x$ in $S$ by $y$. For example,

\begin{mathpar}
  P\{ \quotep{\procn{x}|\procn{x}} / x : x \in \freenames{P} \}
\end{mathpar}

will replace each (occurrence) of a free name $x$ in $P$ by
$\quotep{\procn{x}|\procn{x}}$.

Also, we will avail ourselves of the notation $x^{L}$ and $x^{R}$ to
denote injections of a name into disjoint copies of the name
space. There are numerous ways to accomplish this. One example can be
found in \cite{MeredithR05}. This notation overloads to vectors of
names: $\vec{x}^{\pi} := (x_{i}^{\pi} \; : \; 0 \leq i < |\vec{x}| )$ where $\pi \in \{L,R\}$.

We also use $P^{\Box} := P|\Box$.

In \cite{MeredithR05} an interpretation of the new operator is
given. It turns out that there are several possible interpretations
all enjoying the requisite algebraic properties of the operator (see
\cite{milner91polyadicpi}). We will therefore make liberal use of
$(\nu\; \vec{x})P$.

% subsection the_syntax_and_semantics_of_the_notation_system (end)   

\input{qm2pi.qmops} 

\input{qm2pi.sterngerlach} 

\input{qm2pi.metric} 

% section concurrent_process_calculi (end)

%\input{qm2pi.proofsketch}

% section proof sketch (end)

%\input{qm2pi.slviaknots} 

% section spatial logic via knots (end)

\input{qm2pi.conclusion}

% section conclusion (end)

%\input{qm2pi.dtcodes} 

% section wiring algorithm (end)

\input{qm2pi.ack} 

% section acknowledgments (end)

\newpage


\bibliographystyle{plain}   
\bibliography{../../biblios/main.bib}

\input{qm2pi.rhodetails}

\end{document}

 

% section acknowledgments (end)

\newpage


\bibliographystyle{plain}   
\bibliography{../../biblios/main.bib}

\documentclass[12pt]{llncs}
%\documentclass{jktr}

\usepackage[pdftex]{hyperref}                   
\usepackage {listings}
\usepackage {mathpartir}
\usepackage{bcprules}
%\usepackage{listings}
                       
\usepackage{graphicx} 
%\usepackage[margins=2.5cm,nohead,nofoot]{geometry}
%\usepackage{geometry}
\usepackage{amsfonts}
\usepackage{amstext}
\usepackage{latexsym}
\usepackage{amssymb}
\usepackage{color}


%\include{myPreamble}
\include{qm2pi.local} 

%\ifpdf
%\usepackage[pdftex]{graphicx}
%\else
%\usepackage{graphicx}
%\fi

 % \ifpdf
%  \usepackage{pdfsync}
%  \if


%\title{Brief Article}
%\author{David F. Snyder}
%\author{L.G. Meredith}

%\address{Dept. of Math., Texas State University--San Marcos, San Marcos, TX 78666}
       
\pagestyle{empty}


\begin{document}

\lstset{language=[Objective]Caml,frame=shadowbox}

\input{qm2pi.front}

% section front matter (end)

\input{qm2pi.intro} 
 
% section introduction (end)

% \input{qm2pi.knotations} 

% section notation (end)

\input{qm2pi.process.calculi} 

% section concurrent_process_calculi_and_spatial_logics_ (end)
    
%\input{qm2pi.knots2pi} 

%\input{qm2pi.trefoil} 

%\input{qm2pi.mainthm} 

% subsection basic_interpretation (end)

%\input{qm2pi.rho.presentation} 
\subsection{The syntax and semantics of the notation system}\label{sub:the_syntax_and_semantics_of_the_notation_system} % (fold)

We now summarize a technical presentation of the calculus that
embodies our theory of dynamics. The typical presentation of such a
calculus follows the style of giving generators and relations on
them. The grammar, below, describing term constructors, freely
generates the set of processes, $\Proc$. This set is then quotiented
by a relation known as structural congruence and it is over this set
that the notion of dynamics is expressed. This presentation is
essentially that of \cite{MeredithR05} with the addition of
polyadicity and summation. For readability we have relegated some of
the technical subtleties to an appendix.

\subsubsection{Process grammar}\label{subsub:process_grammar}

\begin{mathpar}
  \inferrule* [lab=synchronization] {} {{M} \bc \pzero \;|\; x?F \;|\; x!C }
  \and
  \inferrule* [lab=abstraction] {} {{F} \bc (x)P}
  \and
  \inferrule* [lab=concretion] {} {{C} \bc \langle Q \rangle}
  \and
  \inferrule* [lab=process] {} {{P,Q} \bc M \;| \;P|Q \;|\; @{x}}
  \and
  \inferrule* [lab=name] {} {{x} \bc \quotep{P}}
\end{mathpar} 

Note that $\vec{x}$ (resp. $\vec{P}$) denotes a vector of names
(resp. processes) of length $|\vec{x}|$ (resp. $|\vec{P}|$). We adopt
the following useful abbreviations.

\begin{mathpar}
   x?(\vec{y}).P := x.(\vec{y})P \and  x\clift{\vec{P}} := x.\clift{\vec{P}}
   \and x!(y) := \lift{x}{\dropn{y}}
   \and \Pi_{i=0}^{n-1}P_i := P_0 | \ldots | P_{n-1}
\end{mathpar}

\subsubsection{Structural congruence}

\paragraph{Free and bound names and alpha-equivalence.} At the
core of structural equivalence is alpha-equivalence which identifies
process that are the same up to a change of variable. Formally, we
recognize the distinction between free and bound names. The free names
of a process, $\freenames{P}$, may be calculated recursively as
follows:

\begin{mathpar}
\freenames{\pzero} := \emptyset
  \and \\
  \freenames{x?(y).P} := \{ x \} \cup (\freenames{P} \setminus \{ y \})
  \and 
  \freenames{x!\langle P \rangle} := \{ x \} \cup \{ P \} 
  \and \\
  \freenames{P|Q} := \freenames{P} \cup \freenames{Q}
  \and \\
  \freenames{@{x}} := \{ x \}
\end{mathpar}

$\pi$
$\quotep{\pi}$

$\freenames{-} : \pi \to \mathcal{P}(\quotep{\pi})$

\begin{eqnarray*}
  \freenames{\pzero} & := & \emptyset \\
  \freenames{x?(y).P} & := & \{ x \} \cup (\freenames{P} \setminus \{ y \}) \\
  \freenames{x!\langle P \rangle} & := & \{ x \} \cup \{ P \} \\
  \freenames{P|Q} & := & \freenames{P} \cup \freenames{Q} \\
  \freenames{\dropn{x}} & := & \{ x \}
\end{eqnarray*}

The bound names of a process, $\boundnames{P}$, are those names occurring in $P$
that are not free. For example, in $x?(y).0$, the name $x$ is free, while $y$ is bound.

\begin{mathpar}
  \inferrule* [lab=monoidal-laws] {} { P|Q \equiv Q|P \and P|0 \equiv P \and P|(Q|R) \equiv (P|Q)|R }
\end{mathpar}

\begin{mathpar}
  \inferrule* [lab=alpha-equivalence] {} { (x)P \equiv (y)P\{y/x\} \and y \not\in \freenames{P} }
\end{mathpar}

\begin{definition}
Then two processes, $P,Q$, are alpha-equivalent if $P = Q\{\vec{y}/\vec{x}\}$ for
some $\vec{x} \in \boundnames{Q},\vec{y} \in \boundnames{P}$, where $Q\{\vec{y}/\vec{x}\}$
denotes the capture-avoiding substitution of $\vec{y}$ for $\vec{x}$ in $Q$.
\end{definition}

\begin{definition}
  The {\em structural congruence} \cite{SangiorgiWalker} , $\equiv$,
  between processes is the least congruence containing
  alpha-equivalence, satisfying the abelian monoid laws
  (associativity, commutativity and $\pzero$ as identity) for parallel
  composition $|$ and for summation $+$.
\end{definition}

\subsection{Name equivalence}

We take name equivalence, written $\nameeq$, to be the smallest
equivalence relation generated by the following rules.

\begin{mathpar}
\inferrule*[lab=Quote-drop]
{ }
{ \quotep{@{x}} \nameeq x }

\inferrule*[lab=Struct-equiv]
{ P \scong Q }
{ \quotep{P} \nameeq \quotep{Q} }
\end{mathpar}

The astute reader will have noticed that the mutual recursion of names
and processes imposes a mutual recursion on alpha-equivalence and
structural equivalence via name-equivalence. Fortunately, all of this
works out pleasantly and we may calculate in the natural way, free of
concern. The reader interested in the details is referred to the
appendix \ref{appendix:rho_details}.

\subsection{Substitution}

We use $\Proc$ for the set of processes, $\QProc$ for the set of
names, and $\id{\{}\vec{y} / \vec{x} \id{\}}$ to denote partial maps,
$s : \QProc \rightarrow \QProc$. A map, $s$ lifts, uniquely, to a map
on process terms, $\widehat{s} : \Proc \rightarrow \Proc$ by the
following equations.

\begin{mathpar}
  (0) \psubstp{Q}{P} := 0 \\
  (R \juxtap S) \psubstp{Q}{P}
  :=    
  (R)\psubstp{Q}{P} \juxtap (S) \psubstp{Q}{P} \\
  (x?(y).R) \psubstp{Q}{P}    
  :=    
  (x)\substp{Q}{P} (z)\concat( (R \psubstn{z}{y}) \psubstp{Q}{P} ) \\
  (\lift{x}{R}) \psubstp{Q}{P}  
  :=
  \lift{(x)\substp{Q}{P}}{ R \psubstp{Q}{P} } \\
%   (\dropn{x})  \psubstp{Q}{P}       
%   := 
%   \left\{ 
%     \begin{array}{ccc} 
%       \dropn{\quotep{Q}} & & x \nameeq \quotep{P} \\
%       \dropn{x} & & otherwise \\
%     \end{array}
%   \right. 
  (\dropn{x})  \psubstp{Q}{P}       
  := 
  \left\{ 
    \begin{array}{ccc} 
      Q & & x \nameeq \quotep{P} \\
      \dropn{x} & & otherwise \\
    \end{array}
  \right.
\end{mathpar}
 

where

\begin{eqnarray}
  (x)\id{\{} \lpquote Q \rpquote / \lpquote P \rpquote \id{\}}            = 
  \left\{ 
    \begin{array}{ccc}
      \lpquote Q \rpquote & & x \nameeq \lpquote P \rpquote \\
      x & & otherwise \\
    \end{array}
  \right. \nonumber
\end{eqnarray}

and $z$ is chosen distinct from $\quotep{P}$, $\quotep{Q}$, the free
names in $Q$, and all the names in $R$. Our $\alpha$-equivalence will
be built in the standard way from this substitution.

\begin{remark}\label{rem:no_self_referential_names}
  One consequence of these definitions is that $\forall P. \quotep{P}
  \not\in \freenames{P}$.
\end{remark}

\subsection{ Dynamic quote: an example }

Anticipating something of what's to come, consider applying the
substitution, $\widehat{\id{\{}u / z \id{\}}}$, to the following pair
of processes, $\lift{w}{y!(z)}$ and $w[ \lpquote y!(z) \rpquote ]$.

\begin{eqnarray}
	\lift{w}{y!(z)}\widehat{\id{\{}u / z \id{\}}}
		& = &
		\lift{w}{y!(u)} \nonumber\\
	w[ \lpquote y!(z) \rpquote ] \widehat{ \id{\{}u / z \id{\}} }
		& = &
		w[ \lpquote y!(z) \rpquote ] \nonumber
\end{eqnarray}

Because the body of the process between quotes is impervious to
substitution, we get radically different answers. In fact, by
examining the first process in an input context,
e.g. $x?(z).\lift{w}{y!(z)}$, we see that the process under the lift
operator may be shaped by prefixed inputs binding a name inside it. In
this sense, the lift operator will be seen as a way to dynamically
construct processes before reifying them as names.

Finally equipped with these standard features we can present the
dynamics of the calculus.

\subsubsection{Operational semantics} 

Finally, we introduce the computational dynamics. What marks these
algebras as distinct from other more traditionally studied algebraic
structures, e.g. vector spaces or polynomial rings, is the manner in
which dynamics is captured. In traditional structures, dynamics is typically
expressed through morphisms between such structures, as in linear maps
between vector spaces or morphisms between rings. In algebras
associated with the semantics of computation, the dynamics is
expressed as part of the algebraic structure itself, through a
reduction reduction relation typically denoted by $\red$. Below, we
give a recursive presentation of this relation for the calculus used
in the encoding.

$\red \subseteq \pi \times \pi$
$\red : \pi \to \mathcal{P}(\pi)$

\begin{mathpar}
  \inferrule* [lab=Comm] { \textsf{match}( x_{src}, x_{trgt} ) } { x_{trgt}?(y)P \; | \; x_{src}!\langle {Q} \rangle \red P\{\quotep{Q}/y}\} }
  \and \\
  \inferrule* [lab=Par] {{P} \red {P}'} {{{P} | {Q}} \red {{P}' | {Q}}}
  \and
  \inferrule* [lab=Equiv]{{{P} \scong {P}'} \andalso {{P}' \red {Q}'} \andalso {{Q}' \scong {Q}}}{{P} \red {Q}}
\end{mathpar}

\begin{eqnarray*}
  match_{\equiv} (\quotep{P},\quotep{Q}) & := & P \equiv Q \\
  match_{\dagger}(\quotep{P},\quotep{Q}) & := & \forall R. P|Q \red^{*} R => R \red^{*} 0 \\
  match_{K}(\quotep{P},\quotep{Q}) & := & K \mbox{ for some context } K
\end{eqnarray*}

$u?(x)P | u!\langle Q \rangle \red P\{\quotep{Q}/x\}$

%We write $\wred$ for $\red^*$, and $P\red$ if $\exists Q $ such that $ P \red Q$.
We write $P\red$ if $\exists Q $ such that $ P \red Q$ and $P\not\red$, otherwise.

\section{Replication}

As mentioned before, it is known that replication (and hence
recursion) can be implemented in a higher-order process algebra
\cite{SangiorgiWalker}. As our first example of calculation with the
machinery thus far presented we give the construction explicitly in
the {\rhoc}.

\begin{eqnarray}
	D_{x} & := & \prefix{x}{y}{(\binpar{\outputp{x}{y}}{@{y}})} \nonumber\\
	\bangp_{x}{P} & := & \binpar{{x}!\langle{\binpar{D_{x}}{P}}\rangle}{D_{x}} \nonumber
\end{eqnarray}

\begin{eqnarray}
	\bangp_{x}{P} & & \nonumber\\
	=
	& {x}!\langle{(\prefix{x}{y}{(\outputp{x}{y} | @{y})) | P}}\rangle 
	      | \prefix{x}{y}{(\outputp{x}{y} | @{y})} & \nonumber\\
	\red
	& (\outputp{x}{y} | @{y})\substn{\quotep{(\prefix{x}{y}{(@{y} | \outputp{x}{y})) | P}}}{y} & \nonumber\\
	=
	& \outputp{x}{\quotep{(\prefix{x}{y}{(\outputp{x}{y} | @{y})) | P}}}
	  | {(\prefix{x}{y}{(\outputp{x}{y} | @{y})) | P}} & \nonumber\\
	\red
	& \ldots & \nonumber\\
	\red^*
	& P | P | \ldots & \nonumber
\end{eqnarray}

Of course, this encoding, as an implementation, runs away, unfolding
$\bangp{P}$ eagerly. A lazier and more implementable replication
operator, restricted to input-guarded processes, may be obtained as follows.

\begin{eqnarray}
\bangp{\prefix{u}{v}{P}} 
	:= 
	\binpar{\lift{x}{\prefix{u}{v}{(\binpar{D(x)}{P})}}}{D(x)} \nonumber
\end{eqnarray}

\begin{remark}
  Note that the lazier definition still does not deal with summation
  or mixed summation (i.e. sums over input and output). The reader is
  invited to construct definitions of replication that deal with these
  features. 

  Further, the definitions are parameterized in a name, $x$. Can you,
  gentle reader, make a definition that eliminates this parameter and
  guarantees no accidental interaction between the replication
  machinery and the process being replicated -- i.e. no accidental
  sharing of names used by the process to get its work done and the
  name(s) used by the replication to effect copying. This latter
  revision of the definition of replication is crucial to obtaining
  the expected identity $!!P \sim !P$.
\end{remark}

\begin{remark}\label{rem:paradoxical_combinator}
  The reader familiar with the lambda calculus will have noticed the
  similarity between $D$ and the paradoxical combinator.

  [Ed. note: the existence of this seems to suggest we have to be more
  restrictive on the set of processes and names we admit if we are to
  support no-cloning.]
\end{remark}

\subsubsection{Bisimulation}

The computational dynamics gives rise to another kind of equivalence,
the equivalence of computational behavior. As previously mentioned
this is typically captured \emph{via} some form of bisimulation.

% The notion we use in this paper is weak barbed bisimulation
% \cite{milner91polyadicpi}.

The notion we use in this paper is derived from weak barbed
bisimulation \cite{milner91polyadicpi}. 

\begin{definition}
An \emph{observation relation}, $\downarrow_{\mathcal N}$, over a set
of names, $\mathcal N$, is the smallest relation satisfying the rules
below.

\infrule[Out-barb]{y \in {\mathcal N}, \; x \nameeq y}
		  {\outputp{x}{v} \downarrow_{\mathcal N} x}
\infrule[Par-barb]{\mbox{$P\downarrow_{\mathcal N} x$ or $Q\downarrow_{\mathcal N} x$}}
		  {\binpar{P}{Q} \downarrow_{\mathcal N} x}

We write $P \Downarrow_{\mathcal N} x$ if there is $Q$ such that 
$P \wred Q$ and $Q \downarrow_{\mathcal N} x$.
\end{definition}

\begin{definition}
%\label{def.bbisim}
An  ${\mathcal N}$-\emph{barbed bisimulation} over a set of names, ${\mathcal N}$, is a symmetric binary relation 
${\mathcal S}_{\mathcal N}$ between agents such that $P\rel{S}_{\mathcal N}Q$ implies:
\begin{enumerate}
\item If $P \red P'$ then $Q \wred Q'$ and $P'\rel{S}_{\mathcal N} Q'$.
\item If $P\downarrow_{\mathcal N} x$, then $Q\Downarrow_{\mathcal N} x$.
\end{enumerate}
$P$ is ${\mathcal N}$-barbed bisimilar to $Q$, written
$P \wbbisim_{\mathcal N} Q$, if $P \rel{S}_{\mathcal N} Q$ for some ${\mathcal N}$-barbed bisimulation ${\mathcal S}_{\mathcal N}$.
\end{definition}

$\mathcal{R} \subseteq \pi \times \pi$

$P \mathcal{R} Q => \forall P'. P \red P' \Rightarrow \exists Q'. Q \red Q', P' \mathcal{R} Q'$

$P \vdash x \Rightarrow Q \vdash x$

\begin{mathpar}
  \inferrule*[lab=Out-barb]{x \nameeq y}{{y}!\langle{Q}\rangle \vdash x}
  \and
  \inferrule*[lab=Par-barb]{\mbox{$P\vdash x$ or $Q\vdash x$}}{\binpar{P}{Q} \vdash x}
\end{mathpar}

\subsubsection{Contexts}

One of the principle advantages of computational calculi like the
$\pi$-calculus is a well-defined notion of context,
contextual-equivalence and a correlation between
contextual-equivalence and notions of bisimulation. The notion of
context allows the decomposition of a process into (sub-)process and
its syntactic environment, its context. Thus, a context may be
thought of as a process with a ``hole'' (written $\Box$) in it. The
application of a context $M$ to a process $P$, written $M[P]$, is
tantamount to filling the hole in $M$ with $P$. In this paper we do
not need the full weight of this theory, but do make use of the notion
of context in the proof the main theorem. 

\begin{mathpar}
  \inferrule* [lab=summation] {} {{M_{M},M_{N}} \bc \Box \;|\; x.M_{A} \;|\; M_{M}+M_{N}}
  \and
  \inferrule* [lab=agent] {} {{M_{A}} \bc (\vec{x})M_{P} \;| \; \clift{P_0,\ldots,M_{P},\ldots,P_N}}
  \and \\
  \inferrule* [lab=process] {} {{M_{P}} \bc M_{N} \;| \;P|M_{P} }
\end{mathpar} 

\begin{mathpar}
  \inferrule* [lab=sychronization] {} {M_{N} \bc \Box \;|\; x?M_{F} \;|\; x!M_{C}}
  \and
  \inferrule* [lab=abstraction] {} {{M_{F}} \bc (x)M_{P} }
  \and
  \inferrule* [lab=concretion] {} {{M_{C}} \bc \langle M_{P} \rangle }
  \and \\
  \inferrule* [lab=process] {} {{M_{P}} \bc M_{N} \;| \;P|M_{P} }
\end{mathpar}

\begin{definition}[contextual application] Given a context $M$, and
  process $P$, we define the \emph{contextual application}, $M[P] :=
  M\{P/\Box\}$. That is, the contextual application of M to P is the
  substitution of $P$ for $\Box$ in $M$.
\end{definition}

$\meaningof{-} : L \to \mathcal{P}(\pi)$

\begin{mathpar}
  \inferrule* [lab=collection] {} {\meaningof{true} = \pi, \and \meaningof{~E} = \pi \setminus \meaningof{E}, \and \meaningof{E_{1} \& E_{2}} = \meaningof{E_{1}} \cap \meaningof{E_{2}}}
\end{mathpar}

\begin{mathpar}
  \inferrule* [lab=structure] {} {\meaningof{0} = \{ P \in \pi | P \equiv 0 \}, \and \\ \meaningof{E_1 | E_2} = \{ P \in \pi | P \equiv P_{1} | P_{2}, P_{1} \in \meaningof{E_{1}}, P_{2} \in \meaningof{E_2}\} }
\end{mathpar}

\begin{mathpar}
 \inferrule* [lab=behavior] {} {\meaningof{\langle a?b \rangle E} = \{ P \in \pi | P \equiv Q | u?(y)P', \\ \and \\\\ \and \\ \;\;\; u \in \meaningof{a}, \forall z.P'\{z/y\} \in \meaningof{E\{z/b\}}\}, \and \\ \meaningof{a!E} = \{ P \in \pi | P \equiv Q | x!\langle P' \rangle, x \in \meaningof{a} P' \in \meaningof{E}\} }
\end{mathpar}

\begin{mathpar}
 \inferrule* [lab=nominal] {} {\meaningof{\quotep{E}} = \{ \quotep{P} \in \quotep{\pi} | P \in \meaningof{E} \}, \and \meaningof{\quotep{P}} = \{ \quotep{Q} \in \quotep{\pi} | P \equiv Q \} \and \\ \meaningof{@\quotep{E}} = \{ P \in \pi | P \equiv @x, x \in \meaningof{E} \}}
\end{mathpar}

\begin{eqnarray*}
  \\
  \meaningof{-} : TS \to ST
\end{eqnarray*}

\begin{eqnarray*}
  \\
  L : TS \to ST
\end{eqnarray*}

\begin{eqnarray*}
  \\
  P \models E \iff P \in \meaningof{E}
\end{eqnarray*}

\begin{eqnarray*}
  P \approx_{L} Q \iff \forall E \in L. P \models E \iff Q \models E
\end{eqnarray*}

\begin{eqnarray*}
  P \approx_{K} Q
\end{eqnarray*}

\begin{eqnarray*}
  P \approx Q
\end{eqnarray*}

$\approx_{K} = \approx = \approx_{L}$

\subsubsection{Contextual duality}

Note that contexts extend the quotation operation to a family of
operations from processes to names. Given a context, $M$, we can
define a \emph{nominal context}, $\quotep{M}$ by $\quotep{M}[P] :=
\quotep{M[P]}$. To foreshadow what is to come we observe that these
operations enjoy a duality with processes very much like the duality
between vectors and maps from vectors to scalars.

Further, because the calculus is essentially higher-order, we have a
correspondence between contexts and processes. More specifically,
given a name $x$ and a context $M$ we can construct $M^{*}_{x}$ such
that 

\begin{mathpar}
  M^{*}_{x} | \lift{x}{P} \red M[P]
\end{mathpar}

namely,

\begin{mathpar}
  M^{*}_{x} := x?(u).M[\dropn{u}]
\end{mathpar}

The dependence of $M^{*}_{x}$ on a name makes it an abstraction, 

\begin{mathpar}
  M^{*} := (x)x?(u).M[\dropn{u}]
\end{mathpar}

\subsection{Additional notation}

It will sometimes be convenient to denote the process a name
quotes. We already have the notation $x = \quotep{P}$, but it will be
convenient to introduce an alternate notation, $\procn{x}$, when we
want to emphasize the connection to the use of the name. Note that, by
virtue of name equivalence, $\quotep{\procn{x}} \nameeq x$; so, the
notation is consistent with previous definitions.

Further, because names have structure it is possible to effect
substitutions on the basis of that structure. This means we need to
upgrade our notation for substitutions, which we accomplish by
adapting comprehension notation. Thus,

\begin{mathpar}
  P\{ y / x : x \in S \}
\end{mathpar}

is interpreted to mean the process derived from P by replacing (in a
capture-avoiding manner) each occurrence of $x$ in $S$ by $y$. For example,

\begin{mathpar}
  P\{ \quotep{\procn{x}|\procn{x}} / x : x \in \freenames{P} \}
\end{mathpar}

will replace each (occurrence) of a free name $x$ in $P$ by
$\quotep{\procn{x}|\procn{x}}$.

Also, we will avail ourselves of the notation $x^{L}$ and $x^{R}$ to
denote injections of a name into disjoint copies of the name
space. There are numerous ways to accomplish this. One example can be
found in \cite{MeredithR05}. This notation overloads to vectors of
names: $\vec{x}^{\pi} := (x_{i}^{\pi} \; : \; 0 \leq i < |\vec{x}| )$ where $\pi \in \{L,R\}$.

We also use $P^{\Box} := P|\Box$.

In \cite{MeredithR05} an interpretation of the new operator is
given. It turns out that there are several possible interpretations
all enjoying the requisite algebraic properties of the operator (see
\cite{milner91polyadicpi}). We will therefore make liberal use of
$(\nu\; \vec{x})P$.

% subsection the_syntax_and_semantics_of_the_notation_system (end)   

\input{qm2pi.qmops} 

\input{qm2pi.sterngerlach} 

\input{qm2pi.metric} 

% section concurrent_process_calculi (end)

%\input{qm2pi.proofsketch}

% section proof sketch (end)

%\input{qm2pi.slviaknots} 

% section spatial logic via knots (end)

\input{qm2pi.conclusion}

% section conclusion (end)

%\input{qm2pi.dtcodes} 

% section wiring algorithm (end)

\input{qm2pi.ack} 

% section acknowledgments (end)

\newpage


\bibliographystyle{plain}   
\bibliography{../../biblios/main.bib}

\input{qm2pi.rhodetails}

\end{document}



\end{document}

 

% section acknowledgments (end)

\newpage


\bibliographystyle{plain}   
\bibliography{../../biblios/main.bib}

\documentclass[12pt]{llncs}
%\documentclass{jktr}

\usepackage[pdftex]{hyperref}                   
\usepackage {listings}
\usepackage {mathpartir}
\usepackage{bcprules}
%\usepackage{listings}
                       
\usepackage{graphicx} 
%\usepackage[margins=2.5cm,nohead,nofoot]{geometry}
%\usepackage{geometry}
\usepackage{amsfonts}
\usepackage{amstext}
\usepackage{latexsym}
\usepackage{amssymb}
\usepackage{color}


%\include{myPreamble}
\documentclass[12pt]{llncs}
%\documentclass{jktr}

\usepackage[pdftex]{hyperref}                   
\usepackage {listings}
\usepackage {mathpartir}
\usepackage{bcprules}
%\usepackage{listings}
                       
\usepackage{graphicx} 
%\usepackage[margins=2.5cm,nohead,nofoot]{geometry}
%\usepackage{geometry}
\usepackage{amsfonts}
\usepackage{amstext}
\usepackage{latexsym}
\usepackage{amssymb}
\usepackage{color}


%\include{myPreamble}
\include{qm2pi.local} 

%\ifpdf
%\usepackage[pdftex]{graphicx}
%\else
%\usepackage{graphicx}
%\fi

 % \ifpdf
%  \usepackage{pdfsync}
%  \if


%\title{Brief Article}
%\author{David F. Snyder}
%\author{L.G. Meredith}

%\address{Dept. of Math., Texas State University--San Marcos, San Marcos, TX 78666}
       
\pagestyle{empty}


\begin{document}

\lstset{language=[Objective]Caml,frame=shadowbox}

\input{qm2pi.front}

% section front matter (end)

\input{qm2pi.intro} 
 
% section introduction (end)

% \input{qm2pi.knotations} 

% section notation (end)

\input{qm2pi.process.calculi} 

% section concurrent_process_calculi_and_spatial_logics_ (end)
    
%\input{qm2pi.knots2pi} 

%\input{qm2pi.trefoil} 

%\input{qm2pi.mainthm} 

% subsection basic_interpretation (end)

%\input{qm2pi.rho.presentation} 
\subsection{The syntax and semantics of the notation system}\label{sub:the_syntax_and_semantics_of_the_notation_system} % (fold)

We now summarize a technical presentation of the calculus that
embodies our theory of dynamics. The typical presentation of such a
calculus follows the style of giving generators and relations on
them. The grammar, below, describing term constructors, freely
generates the set of processes, $\Proc$. This set is then quotiented
by a relation known as structural congruence and it is over this set
that the notion of dynamics is expressed. This presentation is
essentially that of \cite{MeredithR05} with the addition of
polyadicity and summation. For readability we have relegated some of
the technical subtleties to an appendix.

\subsubsection{Process grammar}\label{subsub:process_grammar}

\begin{mathpar}
  \inferrule* [lab=synchronization] {} {{M} \bc \pzero \;|\; x?F \;|\; x!C }
  \and
  \inferrule* [lab=abstraction] {} {{F} \bc (x)P}
  \and
  \inferrule* [lab=concretion] {} {{C} \bc \langle Q \rangle}
  \and
  \inferrule* [lab=process] {} {{P,Q} \bc M \;| \;P|Q \;|\; @{x}}
  \and
  \inferrule* [lab=name] {} {{x} \bc \quotep{P}}
\end{mathpar} 

Note that $\vec{x}$ (resp. $\vec{P}$) denotes a vector of names
(resp. processes) of length $|\vec{x}|$ (resp. $|\vec{P}|$). We adopt
the following useful abbreviations.

\begin{mathpar}
   x?(\vec{y}).P := x.(\vec{y})P \and  x\clift{\vec{P}} := x.\clift{\vec{P}}
   \and x!(y) := \lift{x}{\dropn{y}}
   \and \Pi_{i=0}^{n-1}P_i := P_0 | \ldots | P_{n-1}
\end{mathpar}

\subsubsection{Structural congruence}

\paragraph{Free and bound names and alpha-equivalence.} At the
core of structural equivalence is alpha-equivalence which identifies
process that are the same up to a change of variable. Formally, we
recognize the distinction between free and bound names. The free names
of a process, $\freenames{P}$, may be calculated recursively as
follows:

\begin{mathpar}
\freenames{\pzero} := \emptyset
  \and \\
  \freenames{x?(y).P} := \{ x \} \cup (\freenames{P} \setminus \{ y \})
  \and 
  \freenames{x!\langle P \rangle} := \{ x \} \cup \{ P \} 
  \and \\
  \freenames{P|Q} := \freenames{P} \cup \freenames{Q}
  \and \\
  \freenames{@{x}} := \{ x \}
\end{mathpar}

$\pi$
$\quotep{\pi}$

$\freenames{-} : \pi \to \mathcal{P}(\quotep{\pi})$

\begin{eqnarray*}
  \freenames{\pzero} & := & \emptyset \\
  \freenames{x?(y).P} & := & \{ x \} \cup (\freenames{P} \setminus \{ y \}) \\
  \freenames{x!\langle P \rangle} & := & \{ x \} \cup \{ P \} \\
  \freenames{P|Q} & := & \freenames{P} \cup \freenames{Q} \\
  \freenames{\dropn{x}} & := & \{ x \}
\end{eqnarray*}

The bound names of a process, $\boundnames{P}$, are those names occurring in $P$
that are not free. For example, in $x?(y).0$, the name $x$ is free, while $y$ is bound.

\begin{mathpar}
  \inferrule* [lab=monoidal-laws] {} { P|Q \equiv Q|P \and P|0 \equiv P \and P|(Q|R) \equiv (P|Q)|R }
\end{mathpar}

\begin{mathpar}
  \inferrule* [lab=alpha-equivalence] {} { (x)P \equiv (y)P\{y/x\} \and y \not\in \freenames{P} }
\end{mathpar}

\begin{definition}
Then two processes, $P,Q$, are alpha-equivalent if $P = Q\{\vec{y}/\vec{x}\}$ for
some $\vec{x} \in \boundnames{Q},\vec{y} \in \boundnames{P}$, where $Q\{\vec{y}/\vec{x}\}$
denotes the capture-avoiding substitution of $\vec{y}$ for $\vec{x}$ in $Q$.
\end{definition}

\begin{definition}
  The {\em structural congruence} \cite{SangiorgiWalker} , $\equiv$,
  between processes is the least congruence containing
  alpha-equivalence, satisfying the abelian monoid laws
  (associativity, commutativity and $\pzero$ as identity) for parallel
  composition $|$ and for summation $+$.
\end{definition}

\subsection{Name equivalence}

We take name equivalence, written $\nameeq$, to be the smallest
equivalence relation generated by the following rules.

\begin{mathpar}
\inferrule*[lab=Quote-drop]
{ }
{ \quotep{@{x}} \nameeq x }

\inferrule*[lab=Struct-equiv]
{ P \scong Q }
{ \quotep{P} \nameeq \quotep{Q} }
\end{mathpar}

The astute reader will have noticed that the mutual recursion of names
and processes imposes a mutual recursion on alpha-equivalence and
structural equivalence via name-equivalence. Fortunately, all of this
works out pleasantly and we may calculate in the natural way, free of
concern. The reader interested in the details is referred to the
appendix \ref{appendix:rho_details}.

\subsection{Substitution}

We use $\Proc$ for the set of processes, $\QProc$ for the set of
names, and $\id{\{}\vec{y} / \vec{x} \id{\}}$ to denote partial maps,
$s : \QProc \rightarrow \QProc$. A map, $s$ lifts, uniquely, to a map
on process terms, $\widehat{s} : \Proc \rightarrow \Proc$ by the
following equations.

\begin{mathpar}
  (0) \psubstp{Q}{P} := 0 \\
  (R \juxtap S) \psubstp{Q}{P}
  :=    
  (R)\psubstp{Q}{P} \juxtap (S) \psubstp{Q}{P} \\
  (x?(y).R) \psubstp{Q}{P}    
  :=    
  (x)\substp{Q}{P} (z)\concat( (R \psubstn{z}{y}) \psubstp{Q}{P} ) \\
  (\lift{x}{R}) \psubstp{Q}{P}  
  :=
  \lift{(x)\substp{Q}{P}}{ R \psubstp{Q}{P} } \\
%   (\dropn{x})  \psubstp{Q}{P}       
%   := 
%   \left\{ 
%     \begin{array}{ccc} 
%       \dropn{\quotep{Q}} & & x \nameeq \quotep{P} \\
%       \dropn{x} & & otherwise \\
%     \end{array}
%   \right. 
  (\dropn{x})  \psubstp{Q}{P}       
  := 
  \left\{ 
    \begin{array}{ccc} 
      Q & & x \nameeq \quotep{P} \\
      \dropn{x} & & otherwise \\
    \end{array}
  \right.
\end{mathpar}
 

where

\begin{eqnarray}
  (x)\id{\{} \lpquote Q \rpquote / \lpquote P \rpquote \id{\}}            = 
  \left\{ 
    \begin{array}{ccc}
      \lpquote Q \rpquote & & x \nameeq \lpquote P \rpquote \\
      x & & otherwise \\
    \end{array}
  \right. \nonumber
\end{eqnarray}

and $z$ is chosen distinct from $\quotep{P}$, $\quotep{Q}$, the free
names in $Q$, and all the names in $R$. Our $\alpha$-equivalence will
be built in the standard way from this substitution.

\begin{remark}\label{rem:no_self_referential_names}
  One consequence of these definitions is that $\forall P. \quotep{P}
  \not\in \freenames{P}$.
\end{remark}

\subsection{ Dynamic quote: an example }

Anticipating something of what's to come, consider applying the
substitution, $\widehat{\id{\{}u / z \id{\}}}$, to the following pair
of processes, $\lift{w}{y!(z)}$ and $w[ \lpquote y!(z) \rpquote ]$.

\begin{eqnarray}
	\lift{w}{y!(z)}\widehat{\id{\{}u / z \id{\}}}
		& = &
		\lift{w}{y!(u)} \nonumber\\
	w[ \lpquote y!(z) \rpquote ] \widehat{ \id{\{}u / z \id{\}} }
		& = &
		w[ \lpquote y!(z) \rpquote ] \nonumber
\end{eqnarray}

Because the body of the process between quotes is impervious to
substitution, we get radically different answers. In fact, by
examining the first process in an input context,
e.g. $x?(z).\lift{w}{y!(z)}$, we see that the process under the lift
operator may be shaped by prefixed inputs binding a name inside it. In
this sense, the lift operator will be seen as a way to dynamically
construct processes before reifying them as names.

Finally equipped with these standard features we can present the
dynamics of the calculus.

\subsubsection{Operational semantics} 

Finally, we introduce the computational dynamics. What marks these
algebras as distinct from other more traditionally studied algebraic
structures, e.g. vector spaces or polynomial rings, is the manner in
which dynamics is captured. In traditional structures, dynamics is typically
expressed through morphisms between such structures, as in linear maps
between vector spaces or morphisms between rings. In algebras
associated with the semantics of computation, the dynamics is
expressed as part of the algebraic structure itself, through a
reduction reduction relation typically denoted by $\red$. Below, we
give a recursive presentation of this relation for the calculus used
in the encoding.

$\red \subseteq \pi \times \pi$
$\red : \pi \to \mathcal{P}(\pi)$

\begin{mathpar}
  \inferrule* [lab=Comm] { \textsf{match}( x_{src}, x_{trgt} ) } { x_{trgt}?(y)P \; | \; x_{src}!\langle {Q} \rangle \red P\{\quotep{Q}/y}\} }
  \and \\
  \inferrule* [lab=Par] {{P} \red {P}'} {{{P} | {Q}} \red {{P}' | {Q}}}
  \and
  \inferrule* [lab=Equiv]{{{P} \scong {P}'} \andalso {{P}' \red {Q}'} \andalso {{Q}' \scong {Q}}}{{P} \red {Q}}
\end{mathpar}

\begin{eqnarray*}
  match_{\equiv} (\quotep{P},\quotep{Q}) & := & P \equiv Q \\
  match_{\dagger}(\quotep{P},\quotep{Q}) & := & \forall R. P|Q \red^{*} R => R \red^{*} 0 \\
  match_{K}(\quotep{P},\quotep{Q}) & := & K \mbox{ for some context } K
\end{eqnarray*}

$u?(x)P | u!\langle Q \rangle \red P\{\quotep{Q}/x\}$

%We write $\wred$ for $\red^*$, and $P\red$ if $\exists Q $ such that $ P \red Q$.
We write $P\red$ if $\exists Q $ such that $ P \red Q$ and $P\not\red$, otherwise.

\section{Replication}

As mentioned before, it is known that replication (and hence
recursion) can be implemented in a higher-order process algebra
\cite{SangiorgiWalker}. As our first example of calculation with the
machinery thus far presented we give the construction explicitly in
the {\rhoc}.

\begin{eqnarray}
	D_{x} & := & \prefix{x}{y}{(\binpar{\outputp{x}{y}}{@{y}})} \nonumber\\
	\bangp_{x}{P} & := & \binpar{{x}!\langle{\binpar{D_{x}}{P}}\rangle}{D_{x}} \nonumber
\end{eqnarray}

\begin{eqnarray}
	\bangp_{x}{P} & & \nonumber\\
	=
	& {x}!\langle{(\prefix{x}{y}{(\outputp{x}{y} | @{y})) | P}}\rangle 
	      | \prefix{x}{y}{(\outputp{x}{y} | @{y})} & \nonumber\\
	\red
	& (\outputp{x}{y} | @{y})\substn{\quotep{(\prefix{x}{y}{(@{y} | \outputp{x}{y})) | P}}}{y} & \nonumber\\
	=
	& \outputp{x}{\quotep{(\prefix{x}{y}{(\outputp{x}{y} | @{y})) | P}}}
	  | {(\prefix{x}{y}{(\outputp{x}{y} | @{y})) | P}} & \nonumber\\
	\red
	& \ldots & \nonumber\\
	\red^*
	& P | P | \ldots & \nonumber
\end{eqnarray}

Of course, this encoding, as an implementation, runs away, unfolding
$\bangp{P}$ eagerly. A lazier and more implementable replication
operator, restricted to input-guarded processes, may be obtained as follows.

\begin{eqnarray}
\bangp{\prefix{u}{v}{P}} 
	:= 
	\binpar{\lift{x}{\prefix{u}{v}{(\binpar{D(x)}{P})}}}{D(x)} \nonumber
\end{eqnarray}

\begin{remark}
  Note that the lazier definition still does not deal with summation
  or mixed summation (i.e. sums over input and output). The reader is
  invited to construct definitions of replication that deal with these
  features. 

  Further, the definitions are parameterized in a name, $x$. Can you,
  gentle reader, make a definition that eliminates this parameter and
  guarantees no accidental interaction between the replication
  machinery and the process being replicated -- i.e. no accidental
  sharing of names used by the process to get its work done and the
  name(s) used by the replication to effect copying. This latter
  revision of the definition of replication is crucial to obtaining
  the expected identity $!!P \sim !P$.
\end{remark}

\begin{remark}\label{rem:paradoxical_combinator}
  The reader familiar with the lambda calculus will have noticed the
  similarity between $D$ and the paradoxical combinator.

  [Ed. note: the existence of this seems to suggest we have to be more
  restrictive on the set of processes and names we admit if we are to
  support no-cloning.]
\end{remark}

\subsubsection{Bisimulation}

The computational dynamics gives rise to another kind of equivalence,
the equivalence of computational behavior. As previously mentioned
this is typically captured \emph{via} some form of bisimulation.

% The notion we use in this paper is weak barbed bisimulation
% \cite{milner91polyadicpi}.

The notion we use in this paper is derived from weak barbed
bisimulation \cite{milner91polyadicpi}. 

\begin{definition}
An \emph{observation relation}, $\downarrow_{\mathcal N}$, over a set
of names, $\mathcal N$, is the smallest relation satisfying the rules
below.

\infrule[Out-barb]{y \in {\mathcal N}, \; x \nameeq y}
		  {\outputp{x}{v} \downarrow_{\mathcal N} x}
\infrule[Par-barb]{\mbox{$P\downarrow_{\mathcal N} x$ or $Q\downarrow_{\mathcal N} x$}}
		  {\binpar{P}{Q} \downarrow_{\mathcal N} x}

We write $P \Downarrow_{\mathcal N} x$ if there is $Q$ such that 
$P \wred Q$ and $Q \downarrow_{\mathcal N} x$.
\end{definition}

\begin{definition}
%\label{def.bbisim}
An  ${\mathcal N}$-\emph{barbed bisimulation} over a set of names, ${\mathcal N}$, is a symmetric binary relation 
${\mathcal S}_{\mathcal N}$ between agents such that $P\rel{S}_{\mathcal N}Q$ implies:
\begin{enumerate}
\item If $P \red P'$ then $Q \wred Q'$ and $P'\rel{S}_{\mathcal N} Q'$.
\item If $P\downarrow_{\mathcal N} x$, then $Q\Downarrow_{\mathcal N} x$.
\end{enumerate}
$P$ is ${\mathcal N}$-barbed bisimilar to $Q$, written
$P \wbbisim_{\mathcal N} Q$, if $P \rel{S}_{\mathcal N} Q$ for some ${\mathcal N}$-barbed bisimulation ${\mathcal S}_{\mathcal N}$.
\end{definition}

$\mathcal{R} \subseteq \pi \times \pi$

$P \mathcal{R} Q => \forall P'. P \red P' \Rightarrow \exists Q'. Q \red Q', P' \mathcal{R} Q'$

$P \vdash x \Rightarrow Q \vdash x$

\begin{mathpar}
  \inferrule*[lab=Out-barb]{x \nameeq y}{{y}!\langle{Q}\rangle \vdash x}
  \and
  \inferrule*[lab=Par-barb]{\mbox{$P\vdash x$ or $Q\vdash x$}}{\binpar{P}{Q} \vdash x}
\end{mathpar}

\subsubsection{Contexts}

One of the principle advantages of computational calculi like the
$\pi$-calculus is a well-defined notion of context,
contextual-equivalence and a correlation between
contextual-equivalence and notions of bisimulation. The notion of
context allows the decomposition of a process into (sub-)process and
its syntactic environment, its context. Thus, a context may be
thought of as a process with a ``hole'' (written $\Box$) in it. The
application of a context $M$ to a process $P$, written $M[P]$, is
tantamount to filling the hole in $M$ with $P$. In this paper we do
not need the full weight of this theory, but do make use of the notion
of context in the proof the main theorem. 

\begin{mathpar}
  \inferrule* [lab=summation] {} {{M_{M},M_{N}} \bc \Box \;|\; x.M_{A} \;|\; M_{M}+M_{N}}
  \and
  \inferrule* [lab=agent] {} {{M_{A}} \bc (\vec{x})M_{P} \;| \; \clift{P_0,\ldots,M_{P},\ldots,P_N}}
  \and \\
  \inferrule* [lab=process] {} {{M_{P}} \bc M_{N} \;| \;P|M_{P} }
\end{mathpar} 

\begin{mathpar}
  \inferrule* [lab=sychronization] {} {M_{N} \bc \Box \;|\; x?M_{F} \;|\; x!M_{C}}
  \and
  \inferrule* [lab=abstraction] {} {{M_{F}} \bc (x)M_{P} }
  \and
  \inferrule* [lab=concretion] {} {{M_{C}} \bc \langle M_{P} \rangle }
  \and \\
  \inferrule* [lab=process] {} {{M_{P}} \bc M_{N} \;| \;P|M_{P} }
\end{mathpar}

\begin{definition}[contextual application] Given a context $M$, and
  process $P$, we define the \emph{contextual application}, $M[P] :=
  M\{P/\Box\}$. That is, the contextual application of M to P is the
  substitution of $P$ for $\Box$ in $M$.
\end{definition}

$\meaningof{-} : L \to \mathcal{P}(\pi)$

\begin{mathpar}
  \inferrule* [lab=collection] {} {\meaningof{true} = \pi, \and \meaningof{~E} = \pi \setminus \meaningof{E}, \and \meaningof{E_{1} \& E_{2}} = \meaningof{E_{1}} \cap \meaningof{E_{2}}}
\end{mathpar}

\begin{mathpar}
  \inferrule* [lab=structure] {} {\meaningof{0} = \{ P \in \pi | P \equiv 0 \}, \and \\ \meaningof{E_1 | E_2} = \{ P \in \pi | P \equiv P_{1} | P_{2}, P_{1} \in \meaningof{E_{1}}, P_{2} \in \meaningof{E_2}\} }
\end{mathpar}

\begin{mathpar}
 \inferrule* [lab=behavior] {} {\meaningof{\langle a?b \rangle E} = \{ P \in \pi | P \equiv Q | u?(y)P', \\ \and \\\\ \and \\ \;\;\; u \in \meaningof{a}, \forall z.P'\{z/y\} \in \meaningof{E\{z/b\}}\}, \and \\ \meaningof{a!E} = \{ P \in \pi | P \equiv Q | x!\langle P' \rangle, x \in \meaningof{a} P' \in \meaningof{E}\} }
\end{mathpar}

\begin{mathpar}
 \inferrule* [lab=nominal] {} {\meaningof{\quotep{E}} = \{ \quotep{P} \in \quotep{\pi} | P \in \meaningof{E} \}, \and \meaningof{\quotep{P}} = \{ \quotep{Q} \in \quotep{\pi} | P \equiv Q \} \and \\ \meaningof{@\quotep{E}} = \{ P \in \pi | P \equiv @x, x \in \meaningof{E} \}}
\end{mathpar}

\begin{eqnarray*}
  \\
  \meaningof{-} : TS \to ST
\end{eqnarray*}

\begin{eqnarray*}
  \\
  L : TS \to ST
\end{eqnarray*}

\begin{eqnarray*}
  \\
  P \models E \iff P \in \meaningof{E}
\end{eqnarray*}

\begin{eqnarray*}
  P \approx_{L} Q \iff \forall E \in L. P \models E \iff Q \models E
\end{eqnarray*}

\begin{eqnarray*}
  P \approx_{K} Q
\end{eqnarray*}

\begin{eqnarray*}
  P \approx Q
\end{eqnarray*}

$\approx_{K} = \approx = \approx_{L}$

\subsubsection{Contextual duality}

Note that contexts extend the quotation operation to a family of
operations from processes to names. Given a context, $M$, we can
define a \emph{nominal context}, $\quotep{M}$ by $\quotep{M}[P] :=
\quotep{M[P]}$. To foreshadow what is to come we observe that these
operations enjoy a duality with processes very much like the duality
between vectors and maps from vectors to scalars.

Further, because the calculus is essentially higher-order, we have a
correspondence between contexts and processes. More specifically,
given a name $x$ and a context $M$ we can construct $M^{*}_{x}$ such
that 

\begin{mathpar}
  M^{*}_{x} | \lift{x}{P} \red M[P]
\end{mathpar}

namely,

\begin{mathpar}
  M^{*}_{x} := x?(u).M[\dropn{u}]
\end{mathpar}

The dependence of $M^{*}_{x}$ on a name makes it an abstraction, 

\begin{mathpar}
  M^{*} := (x)x?(u).M[\dropn{u}]
\end{mathpar}

\subsection{Additional notation}

It will sometimes be convenient to denote the process a name
quotes. We already have the notation $x = \quotep{P}$, but it will be
convenient to introduce an alternate notation, $\procn{x}$, when we
want to emphasize the connection to the use of the name. Note that, by
virtue of name equivalence, $\quotep{\procn{x}} \nameeq x$; so, the
notation is consistent with previous definitions.

Further, because names have structure it is possible to effect
substitutions on the basis of that structure. This means we need to
upgrade our notation for substitutions, which we accomplish by
adapting comprehension notation. Thus,

\begin{mathpar}
  P\{ y / x : x \in S \}
\end{mathpar}

is interpreted to mean the process derived from P by replacing (in a
capture-avoiding manner) each occurrence of $x$ in $S$ by $y$. For example,

\begin{mathpar}
  P\{ \quotep{\procn{x}|\procn{x}} / x : x \in \freenames{P} \}
\end{mathpar}

will replace each (occurrence) of a free name $x$ in $P$ by
$\quotep{\procn{x}|\procn{x}}$.

Also, we will avail ourselves of the notation $x^{L}$ and $x^{R}$ to
denote injections of a name into disjoint copies of the name
space. There are numerous ways to accomplish this. One example can be
found in \cite{MeredithR05}. This notation overloads to vectors of
names: $\vec{x}^{\pi} := (x_{i}^{\pi} \; : \; 0 \leq i < |\vec{x}| )$ where $\pi \in \{L,R\}$.

We also use $P^{\Box} := P|\Box$.

In \cite{MeredithR05} an interpretation of the new operator is
given. It turns out that there are several possible interpretations
all enjoying the requisite algebraic properties of the operator (see
\cite{milner91polyadicpi}). We will therefore make liberal use of
$(\nu\; \vec{x})P$.

% subsection the_syntax_and_semantics_of_the_notation_system (end)   

\input{qm2pi.qmops} 

\input{qm2pi.sterngerlach} 

\input{qm2pi.metric} 

% section concurrent_process_calculi (end)

%\input{qm2pi.proofsketch}

% section proof sketch (end)

%\input{qm2pi.slviaknots} 

% section spatial logic via knots (end)

\input{qm2pi.conclusion}

% section conclusion (end)

%\input{qm2pi.dtcodes} 

% section wiring algorithm (end)

\input{qm2pi.ack} 

% section acknowledgments (end)

\newpage


\bibliographystyle{plain}   
\bibliography{../../biblios/main.bib}

\input{qm2pi.rhodetails}

\end{document}

 

%\ifpdf
%\usepackage[pdftex]{graphicx}
%\else
%\usepackage{graphicx}
%\fi

 % \ifpdf
%  \usepackage{pdfsync}
%  \if


%\title{Brief Article}
%\author{David F. Snyder}
%\author{L.G. Meredith}

%\address{Dept. of Math., Texas State University--San Marcos, San Marcos, TX 78666}
       
\pagestyle{empty}


\begin{document}

\lstset{language=[Objective]Caml,frame=shadowbox}

\documentclass[12pt]{llncs}
%\documentclass{jktr}

\usepackage[pdftex]{hyperref}                   
\usepackage {listings}
\usepackage {mathpartir}
\usepackage{bcprules}
%\usepackage{listings}
                       
\usepackage{graphicx} 
%\usepackage[margins=2.5cm,nohead,nofoot]{geometry}
%\usepackage{geometry}
\usepackage{amsfonts}
\usepackage{amstext}
\usepackage{latexsym}
\usepackage{amssymb}
\usepackage{color}


%\include{myPreamble}
\include{qm2pi.local} 

%\ifpdf
%\usepackage[pdftex]{graphicx}
%\else
%\usepackage{graphicx}
%\fi

 % \ifpdf
%  \usepackage{pdfsync}
%  \if


%\title{Brief Article}
%\author{David F. Snyder}
%\author{L.G. Meredith}

%\address{Dept. of Math., Texas State University--San Marcos, San Marcos, TX 78666}
       
\pagestyle{empty}


\begin{document}

\lstset{language=[Objective]Caml,frame=shadowbox}

\input{qm2pi.front}

% section front matter (end)

\input{qm2pi.intro} 
 
% section introduction (end)

% \input{qm2pi.knotations} 

% section notation (end)

\input{qm2pi.process.calculi} 

% section concurrent_process_calculi_and_spatial_logics_ (end)
    
%\input{qm2pi.knots2pi} 

%\input{qm2pi.trefoil} 

%\input{qm2pi.mainthm} 

% subsection basic_interpretation (end)

%\input{qm2pi.rho.presentation} 
\subsection{The syntax and semantics of the notation system}\label{sub:the_syntax_and_semantics_of_the_notation_system} % (fold)

We now summarize a technical presentation of the calculus that
embodies our theory of dynamics. The typical presentation of such a
calculus follows the style of giving generators and relations on
them. The grammar, below, describing term constructors, freely
generates the set of processes, $\Proc$. This set is then quotiented
by a relation known as structural congruence and it is over this set
that the notion of dynamics is expressed. This presentation is
essentially that of \cite{MeredithR05} with the addition of
polyadicity and summation. For readability we have relegated some of
the technical subtleties to an appendix.

\subsubsection{Process grammar}\label{subsub:process_grammar}

\begin{mathpar}
  \inferrule* [lab=synchronization] {} {{M} \bc \pzero \;|\; x?F \;|\; x!C }
  \and
  \inferrule* [lab=abstraction] {} {{F} \bc (x)P}
  \and
  \inferrule* [lab=concretion] {} {{C} \bc \langle Q \rangle}
  \and
  \inferrule* [lab=process] {} {{P,Q} \bc M \;| \;P|Q \;|\; @{x}}
  \and
  \inferrule* [lab=name] {} {{x} \bc \quotep{P}}
\end{mathpar} 

Note that $\vec{x}$ (resp. $\vec{P}$) denotes a vector of names
(resp. processes) of length $|\vec{x}|$ (resp. $|\vec{P}|$). We adopt
the following useful abbreviations.

\begin{mathpar}
   x?(\vec{y}).P := x.(\vec{y})P \and  x\clift{\vec{P}} := x.\clift{\vec{P}}
   \and x!(y) := \lift{x}{\dropn{y}}
   \and \Pi_{i=0}^{n-1}P_i := P_0 | \ldots | P_{n-1}
\end{mathpar}

\subsubsection{Structural congruence}

\paragraph{Free and bound names and alpha-equivalence.} At the
core of structural equivalence is alpha-equivalence which identifies
process that are the same up to a change of variable. Formally, we
recognize the distinction between free and bound names. The free names
of a process, $\freenames{P}$, may be calculated recursively as
follows:

\begin{mathpar}
\freenames{\pzero} := \emptyset
  \and \\
  \freenames{x?(y).P} := \{ x \} \cup (\freenames{P} \setminus \{ y \})
  \and 
  \freenames{x!\langle P \rangle} := \{ x \} \cup \{ P \} 
  \and \\
  \freenames{P|Q} := \freenames{P} \cup \freenames{Q}
  \and \\
  \freenames{@{x}} := \{ x \}
\end{mathpar}

$\pi$
$\quotep{\pi}$

$\freenames{-} : \pi \to \mathcal{P}(\quotep{\pi})$

\begin{eqnarray*}
  \freenames{\pzero} & := & \emptyset \\
  \freenames{x?(y).P} & := & \{ x \} \cup (\freenames{P} \setminus \{ y \}) \\
  \freenames{x!\langle P \rangle} & := & \{ x \} \cup \{ P \} \\
  \freenames{P|Q} & := & \freenames{P} \cup \freenames{Q} \\
  \freenames{\dropn{x}} & := & \{ x \}
\end{eqnarray*}

The bound names of a process, $\boundnames{P}$, are those names occurring in $P$
that are not free. For example, in $x?(y).0$, the name $x$ is free, while $y$ is bound.

\begin{mathpar}
  \inferrule* [lab=monoidal-laws] {} { P|Q \equiv Q|P \and P|0 \equiv P \and P|(Q|R) \equiv (P|Q)|R }
\end{mathpar}

\begin{mathpar}
  \inferrule* [lab=alpha-equivalence] {} { (x)P \equiv (y)P\{y/x\} \and y \not\in \freenames{P} }
\end{mathpar}

\begin{definition}
Then two processes, $P,Q$, are alpha-equivalent if $P = Q\{\vec{y}/\vec{x}\}$ for
some $\vec{x} \in \boundnames{Q},\vec{y} \in \boundnames{P}$, where $Q\{\vec{y}/\vec{x}\}$
denotes the capture-avoiding substitution of $\vec{y}$ for $\vec{x}$ in $Q$.
\end{definition}

\begin{definition}
  The {\em structural congruence} \cite{SangiorgiWalker} , $\equiv$,
  between processes is the least congruence containing
  alpha-equivalence, satisfying the abelian monoid laws
  (associativity, commutativity and $\pzero$ as identity) for parallel
  composition $|$ and for summation $+$.
\end{definition}

\subsection{Name equivalence}

We take name equivalence, written $\nameeq$, to be the smallest
equivalence relation generated by the following rules.

\begin{mathpar}
\inferrule*[lab=Quote-drop]
{ }
{ \quotep{@{x}} \nameeq x }

\inferrule*[lab=Struct-equiv]
{ P \scong Q }
{ \quotep{P} \nameeq \quotep{Q} }
\end{mathpar}

The astute reader will have noticed that the mutual recursion of names
and processes imposes a mutual recursion on alpha-equivalence and
structural equivalence via name-equivalence. Fortunately, all of this
works out pleasantly and we may calculate in the natural way, free of
concern. The reader interested in the details is referred to the
appendix \ref{appendix:rho_details}.

\subsection{Substitution}

We use $\Proc$ for the set of processes, $\QProc$ for the set of
names, and $\id{\{}\vec{y} / \vec{x} \id{\}}$ to denote partial maps,
$s : \QProc \rightarrow \QProc$. A map, $s$ lifts, uniquely, to a map
on process terms, $\widehat{s} : \Proc \rightarrow \Proc$ by the
following equations.

\begin{mathpar}
  (0) \psubstp{Q}{P} := 0 \\
  (R \juxtap S) \psubstp{Q}{P}
  :=    
  (R)\psubstp{Q}{P} \juxtap (S) \psubstp{Q}{P} \\
  (x?(y).R) \psubstp{Q}{P}    
  :=    
  (x)\substp{Q}{P} (z)\concat( (R \psubstn{z}{y}) \psubstp{Q}{P} ) \\
  (\lift{x}{R}) \psubstp{Q}{P}  
  :=
  \lift{(x)\substp{Q}{P}}{ R \psubstp{Q}{P} } \\
%   (\dropn{x})  \psubstp{Q}{P}       
%   := 
%   \left\{ 
%     \begin{array}{ccc} 
%       \dropn{\quotep{Q}} & & x \nameeq \quotep{P} \\
%       \dropn{x} & & otherwise \\
%     \end{array}
%   \right. 
  (\dropn{x})  \psubstp{Q}{P}       
  := 
  \left\{ 
    \begin{array}{ccc} 
      Q & & x \nameeq \quotep{P} \\
      \dropn{x} & & otherwise \\
    \end{array}
  \right.
\end{mathpar}
 

where

\begin{eqnarray}
  (x)\id{\{} \lpquote Q \rpquote / \lpquote P \rpquote \id{\}}            = 
  \left\{ 
    \begin{array}{ccc}
      \lpquote Q \rpquote & & x \nameeq \lpquote P \rpquote \\
      x & & otherwise \\
    \end{array}
  \right. \nonumber
\end{eqnarray}

and $z$ is chosen distinct from $\quotep{P}$, $\quotep{Q}$, the free
names in $Q$, and all the names in $R$. Our $\alpha$-equivalence will
be built in the standard way from this substitution.

\begin{remark}\label{rem:no_self_referential_names}
  One consequence of these definitions is that $\forall P. \quotep{P}
  \not\in \freenames{P}$.
\end{remark}

\subsection{ Dynamic quote: an example }

Anticipating something of what's to come, consider applying the
substitution, $\widehat{\id{\{}u / z \id{\}}}$, to the following pair
of processes, $\lift{w}{y!(z)}$ and $w[ \lpquote y!(z) \rpquote ]$.

\begin{eqnarray}
	\lift{w}{y!(z)}\widehat{\id{\{}u / z \id{\}}}
		& = &
		\lift{w}{y!(u)} \nonumber\\
	w[ \lpquote y!(z) \rpquote ] \widehat{ \id{\{}u / z \id{\}} }
		& = &
		w[ \lpquote y!(z) \rpquote ] \nonumber
\end{eqnarray}

Because the body of the process between quotes is impervious to
substitution, we get radically different answers. In fact, by
examining the first process in an input context,
e.g. $x?(z).\lift{w}{y!(z)}$, we see that the process under the lift
operator may be shaped by prefixed inputs binding a name inside it. In
this sense, the lift operator will be seen as a way to dynamically
construct processes before reifying them as names.

Finally equipped with these standard features we can present the
dynamics of the calculus.

\subsubsection{Operational semantics} 

Finally, we introduce the computational dynamics. What marks these
algebras as distinct from other more traditionally studied algebraic
structures, e.g. vector spaces or polynomial rings, is the manner in
which dynamics is captured. In traditional structures, dynamics is typically
expressed through morphisms between such structures, as in linear maps
between vector spaces or morphisms between rings. In algebras
associated with the semantics of computation, the dynamics is
expressed as part of the algebraic structure itself, through a
reduction reduction relation typically denoted by $\red$. Below, we
give a recursive presentation of this relation for the calculus used
in the encoding.

$\red \subseteq \pi \times \pi$
$\red : \pi \to \mathcal{P}(\pi)$

\begin{mathpar}
  \inferrule* [lab=Comm] { \textsf{match}( x_{src}, x_{trgt} ) } { x_{trgt}?(y)P \; | \; x_{src}!\langle {Q} \rangle \red P\{\quotep{Q}/y}\} }
  \and \\
  \inferrule* [lab=Par] {{P} \red {P}'} {{{P} | {Q}} \red {{P}' | {Q}}}
  \and
  \inferrule* [lab=Equiv]{{{P} \scong {P}'} \andalso {{P}' \red {Q}'} \andalso {{Q}' \scong {Q}}}{{P} \red {Q}}
\end{mathpar}

\begin{eqnarray*}
  match_{\equiv} (\quotep{P},\quotep{Q}) & := & P \equiv Q \\
  match_{\dagger}(\quotep{P},\quotep{Q}) & := & \forall R. P|Q \red^{*} R => R \red^{*} 0 \\
  match_{K}(\quotep{P},\quotep{Q}) & := & K \mbox{ for some context } K
\end{eqnarray*}

$u?(x)P | u!\langle Q \rangle \red P\{\quotep{Q}/x\}$

%We write $\wred$ for $\red^*$, and $P\red$ if $\exists Q $ such that $ P \red Q$.
We write $P\red$ if $\exists Q $ such that $ P \red Q$ and $P\not\red$, otherwise.

\section{Replication}

As mentioned before, it is known that replication (and hence
recursion) can be implemented in a higher-order process algebra
\cite{SangiorgiWalker}. As our first example of calculation with the
machinery thus far presented we give the construction explicitly in
the {\rhoc}.

\begin{eqnarray}
	D_{x} & := & \prefix{x}{y}{(\binpar{\outputp{x}{y}}{@{y}})} \nonumber\\
	\bangp_{x}{P} & := & \binpar{{x}!\langle{\binpar{D_{x}}{P}}\rangle}{D_{x}} \nonumber
\end{eqnarray}

\begin{eqnarray}
	\bangp_{x}{P} & & \nonumber\\
	=
	& {x}!\langle{(\prefix{x}{y}{(\outputp{x}{y} | @{y})) | P}}\rangle 
	      | \prefix{x}{y}{(\outputp{x}{y} | @{y})} & \nonumber\\
	\red
	& (\outputp{x}{y} | @{y})\substn{\quotep{(\prefix{x}{y}{(@{y} | \outputp{x}{y})) | P}}}{y} & \nonumber\\
	=
	& \outputp{x}{\quotep{(\prefix{x}{y}{(\outputp{x}{y} | @{y})) | P}}}
	  | {(\prefix{x}{y}{(\outputp{x}{y} | @{y})) | P}} & \nonumber\\
	\red
	& \ldots & \nonumber\\
	\red^*
	& P | P | \ldots & \nonumber
\end{eqnarray}

Of course, this encoding, as an implementation, runs away, unfolding
$\bangp{P}$ eagerly. A lazier and more implementable replication
operator, restricted to input-guarded processes, may be obtained as follows.

\begin{eqnarray}
\bangp{\prefix{u}{v}{P}} 
	:= 
	\binpar{\lift{x}{\prefix{u}{v}{(\binpar{D(x)}{P})}}}{D(x)} \nonumber
\end{eqnarray}

\begin{remark}
  Note that the lazier definition still does not deal with summation
  or mixed summation (i.e. sums over input and output). The reader is
  invited to construct definitions of replication that deal with these
  features. 

  Further, the definitions are parameterized in a name, $x$. Can you,
  gentle reader, make a definition that eliminates this parameter and
  guarantees no accidental interaction between the replication
  machinery and the process being replicated -- i.e. no accidental
  sharing of names used by the process to get its work done and the
  name(s) used by the replication to effect copying. This latter
  revision of the definition of replication is crucial to obtaining
  the expected identity $!!P \sim !P$.
\end{remark}

\begin{remark}\label{rem:paradoxical_combinator}
  The reader familiar with the lambda calculus will have noticed the
  similarity between $D$ and the paradoxical combinator.

  [Ed. note: the existence of this seems to suggest we have to be more
  restrictive on the set of processes and names we admit if we are to
  support no-cloning.]
\end{remark}

\subsubsection{Bisimulation}

The computational dynamics gives rise to another kind of equivalence,
the equivalence of computational behavior. As previously mentioned
this is typically captured \emph{via} some form of bisimulation.

% The notion we use in this paper is weak barbed bisimulation
% \cite{milner91polyadicpi}.

The notion we use in this paper is derived from weak barbed
bisimulation \cite{milner91polyadicpi}. 

\begin{definition}
An \emph{observation relation}, $\downarrow_{\mathcal N}$, over a set
of names, $\mathcal N$, is the smallest relation satisfying the rules
below.

\infrule[Out-barb]{y \in {\mathcal N}, \; x \nameeq y}
		  {\outputp{x}{v} \downarrow_{\mathcal N} x}
\infrule[Par-barb]{\mbox{$P\downarrow_{\mathcal N} x$ or $Q\downarrow_{\mathcal N} x$}}
		  {\binpar{P}{Q} \downarrow_{\mathcal N} x}

We write $P \Downarrow_{\mathcal N} x$ if there is $Q$ such that 
$P \wred Q$ and $Q \downarrow_{\mathcal N} x$.
\end{definition}

\begin{definition}
%\label{def.bbisim}
An  ${\mathcal N}$-\emph{barbed bisimulation} over a set of names, ${\mathcal N}$, is a symmetric binary relation 
${\mathcal S}_{\mathcal N}$ between agents such that $P\rel{S}_{\mathcal N}Q$ implies:
\begin{enumerate}
\item If $P \red P'$ then $Q \wred Q'$ and $P'\rel{S}_{\mathcal N} Q'$.
\item If $P\downarrow_{\mathcal N} x$, then $Q\Downarrow_{\mathcal N} x$.
\end{enumerate}
$P$ is ${\mathcal N}$-barbed bisimilar to $Q$, written
$P \wbbisim_{\mathcal N} Q$, if $P \rel{S}_{\mathcal N} Q$ for some ${\mathcal N}$-barbed bisimulation ${\mathcal S}_{\mathcal N}$.
\end{definition}

$\mathcal{R} \subseteq \pi \times \pi$

$P \mathcal{R} Q => \forall P'. P \red P' \Rightarrow \exists Q'. Q \red Q', P' \mathcal{R} Q'$

$P \vdash x \Rightarrow Q \vdash x$

\begin{mathpar}
  \inferrule*[lab=Out-barb]{x \nameeq y}{{y}!\langle{Q}\rangle \vdash x}
  \and
  \inferrule*[lab=Par-barb]{\mbox{$P\vdash x$ or $Q\vdash x$}}{\binpar{P}{Q} \vdash x}
\end{mathpar}

\subsubsection{Contexts}

One of the principle advantages of computational calculi like the
$\pi$-calculus is a well-defined notion of context,
contextual-equivalence and a correlation between
contextual-equivalence and notions of bisimulation. The notion of
context allows the decomposition of a process into (sub-)process and
its syntactic environment, its context. Thus, a context may be
thought of as a process with a ``hole'' (written $\Box$) in it. The
application of a context $M$ to a process $P$, written $M[P]$, is
tantamount to filling the hole in $M$ with $P$. In this paper we do
not need the full weight of this theory, but do make use of the notion
of context in the proof the main theorem. 

\begin{mathpar}
  \inferrule* [lab=summation] {} {{M_{M},M_{N}} \bc \Box \;|\; x.M_{A} \;|\; M_{M}+M_{N}}
  \and
  \inferrule* [lab=agent] {} {{M_{A}} \bc (\vec{x})M_{P} \;| \; \clift{P_0,\ldots,M_{P},\ldots,P_N}}
  \and \\
  \inferrule* [lab=process] {} {{M_{P}} \bc M_{N} \;| \;P|M_{P} }
\end{mathpar} 

\begin{mathpar}
  \inferrule* [lab=sychronization] {} {M_{N} \bc \Box \;|\; x?M_{F} \;|\; x!M_{C}}
  \and
  \inferrule* [lab=abstraction] {} {{M_{F}} \bc (x)M_{P} }
  \and
  \inferrule* [lab=concretion] {} {{M_{C}} \bc \langle M_{P} \rangle }
  \and \\
  \inferrule* [lab=process] {} {{M_{P}} \bc M_{N} \;| \;P|M_{P} }
\end{mathpar}

\begin{definition}[contextual application] Given a context $M$, and
  process $P$, we define the \emph{contextual application}, $M[P] :=
  M\{P/\Box\}$. That is, the contextual application of M to P is the
  substitution of $P$ for $\Box$ in $M$.
\end{definition}

$\meaningof{-} : L \to \mathcal{P}(\pi)$

\begin{mathpar}
  \inferrule* [lab=collection] {} {\meaningof{true} = \pi, \and \meaningof{~E} = \pi \setminus \meaningof{E}, \and \meaningof{E_{1} \& E_{2}} = \meaningof{E_{1}} \cap \meaningof{E_{2}}}
\end{mathpar}

\begin{mathpar}
  \inferrule* [lab=structure] {} {\meaningof{0} = \{ P \in \pi | P \equiv 0 \}, \and \\ \meaningof{E_1 | E_2} = \{ P \in \pi | P \equiv P_{1} | P_{2}, P_{1} \in \meaningof{E_{1}}, P_{2} \in \meaningof{E_2}\} }
\end{mathpar}

\begin{mathpar}
 \inferrule* [lab=behavior] {} {\meaningof{\langle a?b \rangle E} = \{ P \in \pi | P \equiv Q | u?(y)P', \\ \and \\\\ \and \\ \;\;\; u \in \meaningof{a}, \forall z.P'\{z/y\} \in \meaningof{E\{z/b\}}\}, \and \\ \meaningof{a!E} = \{ P \in \pi | P \equiv Q | x!\langle P' \rangle, x \in \meaningof{a} P' \in \meaningof{E}\} }
\end{mathpar}

\begin{mathpar}
 \inferrule* [lab=nominal] {} {\meaningof{\quotep{E}} = \{ \quotep{P} \in \quotep{\pi} | P \in \meaningof{E} \}, \and \meaningof{\quotep{P}} = \{ \quotep{Q} \in \quotep{\pi} | P \equiv Q \} \and \\ \meaningof{@\quotep{E}} = \{ P \in \pi | P \equiv @x, x \in \meaningof{E} \}}
\end{mathpar}

\begin{eqnarray*}
  \\
  \meaningof{-} : TS \to ST
\end{eqnarray*}

\begin{eqnarray*}
  \\
  L : TS \to ST
\end{eqnarray*}

\begin{eqnarray*}
  \\
  P \models E \iff P \in \meaningof{E}
\end{eqnarray*}

\begin{eqnarray*}
  P \approx_{L} Q \iff \forall E \in L. P \models E \iff Q \models E
\end{eqnarray*}

\begin{eqnarray*}
  P \approx_{K} Q
\end{eqnarray*}

\begin{eqnarray*}
  P \approx Q
\end{eqnarray*}

$\approx_{K} = \approx = \approx_{L}$

\subsubsection{Contextual duality}

Note that contexts extend the quotation operation to a family of
operations from processes to names. Given a context, $M$, we can
define a \emph{nominal context}, $\quotep{M}$ by $\quotep{M}[P] :=
\quotep{M[P]}$. To foreshadow what is to come we observe that these
operations enjoy a duality with processes very much like the duality
between vectors and maps from vectors to scalars.

Further, because the calculus is essentially higher-order, we have a
correspondence between contexts and processes. More specifically,
given a name $x$ and a context $M$ we can construct $M^{*}_{x}$ such
that 

\begin{mathpar}
  M^{*}_{x} | \lift{x}{P} \red M[P]
\end{mathpar}

namely,

\begin{mathpar}
  M^{*}_{x} := x?(u).M[\dropn{u}]
\end{mathpar}

The dependence of $M^{*}_{x}$ on a name makes it an abstraction, 

\begin{mathpar}
  M^{*} := (x)x?(u).M[\dropn{u}]
\end{mathpar}

\subsection{Additional notation}

It will sometimes be convenient to denote the process a name
quotes. We already have the notation $x = \quotep{P}$, but it will be
convenient to introduce an alternate notation, $\procn{x}$, when we
want to emphasize the connection to the use of the name. Note that, by
virtue of name equivalence, $\quotep{\procn{x}} \nameeq x$; so, the
notation is consistent with previous definitions.

Further, because names have structure it is possible to effect
substitutions on the basis of that structure. This means we need to
upgrade our notation for substitutions, which we accomplish by
adapting comprehension notation. Thus,

\begin{mathpar}
  P\{ y / x : x \in S \}
\end{mathpar}

is interpreted to mean the process derived from P by replacing (in a
capture-avoiding manner) each occurrence of $x$ in $S$ by $y$. For example,

\begin{mathpar}
  P\{ \quotep{\procn{x}|\procn{x}} / x : x \in \freenames{P} \}
\end{mathpar}

will replace each (occurrence) of a free name $x$ in $P$ by
$\quotep{\procn{x}|\procn{x}}$.

Also, we will avail ourselves of the notation $x^{L}$ and $x^{R}$ to
denote injections of a name into disjoint copies of the name
space. There are numerous ways to accomplish this. One example can be
found in \cite{MeredithR05}. This notation overloads to vectors of
names: $\vec{x}^{\pi} := (x_{i}^{\pi} \; : \; 0 \leq i < |\vec{x}| )$ where $\pi \in \{L,R\}$.

We also use $P^{\Box} := P|\Box$.

In \cite{MeredithR05} an interpretation of the new operator is
given. It turns out that there are several possible interpretations
all enjoying the requisite algebraic properties of the operator (see
\cite{milner91polyadicpi}). We will therefore make liberal use of
$(\nu\; \vec{x})P$.

% subsection the_syntax_and_semantics_of_the_notation_system (end)   

\input{qm2pi.qmops} 

\input{qm2pi.sterngerlach} 

\input{qm2pi.metric} 

% section concurrent_process_calculi (end)

%\input{qm2pi.proofsketch}

% section proof sketch (end)

%\input{qm2pi.slviaknots} 

% section spatial logic via knots (end)

\input{qm2pi.conclusion}

% section conclusion (end)

%\input{qm2pi.dtcodes} 

% section wiring algorithm (end)

\input{qm2pi.ack} 

% section acknowledgments (end)

\newpage


\bibliographystyle{plain}   
\bibliography{../../biblios/main.bib}

\input{qm2pi.rhodetails}

\end{document}



% section front matter (end)

\section{Introduction}\label{sec:introduction} % (fold)
In this draft of the material i am going to have to dispense with the
usual writing conventions adopted in papers on these topics. i'm going
to have adopt whatever tone i need at the time i'm writing up the
calculations. Sometimes this may be very conversational; others it may
be the barest mathematical grunts; others still it may be that i have
lifted text from one of my other papers because the exposition of some
point was better said there. i hope that my readers are not unduly put
out by this decision. i'm not doing this to flout convention or be
rebellious. i find these calculations very technically challenging. To
keep everything going technically, something has to give; i have to
let go of some cognitive burden. So, the academic writing style --
with all of its trade-offs in terms of facilitating technical
communication -- is what i'm letting go of. Perhaps subsequent drafts
can be tightened and polished, but for now, i'm going to speak as if
we were sitting together in a coffee shop with a laptop, wifi and a
pad of paper and a pencil.

So, here's what i have to say. We -- you and i, comfortably ensconced
in our coffee shop and well-equipped with our tools -- can realize and
carry out the calculations of quantum mechanics over a very different
formal theory of dynamics, a formal theory of dynamics that
corresponds to a theory of concurrent computation with
\emph{reflection}. It has the advantage that the underlying theory is
already `quantized', but supports analogues all of the continuuous
operations. Strikingly, this underlying theory has recently been
connected with a notion of metric that we can show, by calculating
together, coincides with the metric induced by the inner product.

There are a lot of reasons why you might be interested in seeing
calculations of this form. Here's why i'm interested. For the past
several centuries there has been no competitor to the ``Newtonian''
account of dynamics. As a result the predominant share of accounts of
dynamical systems and situations have had to be formulated in terms of
the Newtonian machinery. i view this as an intellectually dangerous
position to occupy. Everything, despite it's intrinsic shape, turns
into a nail to be hit with this hammer. Recently, however, the theory
of computation has matured to the point where we have candidates for
theories of dynamics that offer very different perspective on
reasoning about dynamical systems and situations. Testing these
candidates against very successful accounts of dynamical situations,
like quantum mechanics, is going to give us some sense of how mature
they are and some measure of the quality of these accounts of
dynamics.

\subsection{Summary of contributions and outline of paper}

So, we're going to develop an interpretation of the operations of
quantum mechanics normally interpreted by Hilbert spaces and
operators. We're going to do this over a theory of computation. Note
that this is very different than the usual quantum computation program
which develops notions of computation over quantum mechanics. Rather,
we are developing a story that aligns with Wheeler's slogan: It from
Bit. To do this we will first provide an account of the theory of
computation at play here. Then we will dive into a calculation-driven
interpretation of the operations of quantum mechanics.

The reason we take this approach is that -- until very recently --
there hasn't been an axiomatic account of quantum mechanics. As a
result there has been no sharp delineation of the mathematical theory
supporting interpretation of the physical theory and the physical
theory, itself. So, ambient features of the maths are free to be
exploited (or supressed) without a real accounting of their physical
relevance. There is no sharp statement ``here's the physical theory''
qua \emph{theory} and ``here's the mathematical interpretation''
enabling a judgment of how faithful the interpretation is -- apart
from experimental observation. When there is an axiomatic account we
can judge how well a given mathematical formalism supports an
interpretation of the axioms, independent of
experimentation. Likewise, we can judge how well we have captured our
physical evidence and experience with our axiomatics, independent of
any specific mathematical implementation, with accidental detail that
may or may not have physical significance. 

In lieu of a fully fleshed out and vetted axiomatic account of quantum
mechanics, interpreting the operational notions in service of modeling
physical systems will have to suffice. In other words, we are not in
the business of providing a model of Hilbert spaces and operators. We
are in the business of providing a model of quantum mechanics because
we are motivated by testing our notions of dynamics against physical
theory; and, the predictive calculations of the physical theory must
serve as the best formulation -- shy of a fully fleshed out axiomatic
account -- of the physical theory itself (as they have for scientific
theories since time immemorial). Put another way, despite a
whole-hearted commitment to an It-from-Bit ontology, we are firmly
aligned with the shut-up-and-calculate camp as the best way to obtain
results either from the physical perspective or as a quality assurance
measure of our fledgling theory of dynamics.

In detail, we present a reflective process calculus. Then we develop
intuitive correspondences between the notions available in this
calculus and the usual physical notions supporting quantum mechanical
calculations. Thus, 

\begin{table}[htp]
  \center{
    \fbox{
      \begin{tabular}{c|c}
        quantum mechanics & process calculus \\
        \hline
        scalar & name \\
        state vector & process \\
        dual & contextual duals \\
        matrix & formal sums of process-context-dual pairs \\
        orthogonality & process annihilation \\
        inner product & execution-formula + quoting
      \end{tabular}
    }
  }
  \caption{QM - process calculi correspondences}
\end{table}

Then we tighten up these intuitions to operational definitions. We
employ the Dirac notation as the best proxy we can find for an
abstract syntax of the quantum mechanical notions. The definitions we
develop put us in contact with equational constraints coming from the
theory that we demonstrate the definitions and calculations satisfy.

This puts us in a position to shut up and calculate for the
Stern-Gerlach experimental set up, showing how these predictive
calculations become calculations on processes in our theory of a
reflective process calculus.

Penultimately, we demonstrate that the notion of metric coming from
the inner product coincides with the notion of metric available from
the theory of bisimulation. This demonstration gives us the right to
think of space as arising from behavior. Finally, we consider where we
might go from the new vantage point we have obtained.

% section introduction (end) 
 
% section introduction (end)

% \documentclass[12pt]{llncs}
%\documentclass{jktr}

\usepackage[pdftex]{hyperref}                   
\usepackage {listings}
\usepackage {mathpartir}
\usepackage{bcprules}
%\usepackage{listings}
                       
\usepackage{graphicx} 
%\usepackage[margins=2.5cm,nohead,nofoot]{geometry}
%\usepackage{geometry}
\usepackage{amsfonts}
\usepackage{amstext}
\usepackage{latexsym}
\usepackage{amssymb}
\usepackage{color}


%\include{myPreamble}
\include{qm2pi.local} 

%\ifpdf
%\usepackage[pdftex]{graphicx}
%\else
%\usepackage{graphicx}
%\fi

 % \ifpdf
%  \usepackage{pdfsync}
%  \if


%\title{Brief Article}
%\author{David F. Snyder}
%\author{L.G. Meredith}

%\address{Dept. of Math., Texas State University--San Marcos, San Marcos, TX 78666}
       
\pagestyle{empty}


\begin{document}

\lstset{language=[Objective]Caml,frame=shadowbox}

\input{qm2pi.front}

% section front matter (end)

\input{qm2pi.intro} 
 
% section introduction (end)

% \input{qm2pi.knotations} 

% section notation (end)

\input{qm2pi.process.calculi} 

% section concurrent_process_calculi_and_spatial_logics_ (end)
    
%\input{qm2pi.knots2pi} 

%\input{qm2pi.trefoil} 

%\input{qm2pi.mainthm} 

% subsection basic_interpretation (end)

%\input{qm2pi.rho.presentation} 
\subsection{The syntax and semantics of the notation system}\label{sub:the_syntax_and_semantics_of_the_notation_system} % (fold)

We now summarize a technical presentation of the calculus that
embodies our theory of dynamics. The typical presentation of such a
calculus follows the style of giving generators and relations on
them. The grammar, below, describing term constructors, freely
generates the set of processes, $\Proc$. This set is then quotiented
by a relation known as structural congruence and it is over this set
that the notion of dynamics is expressed. This presentation is
essentially that of \cite{MeredithR05} with the addition of
polyadicity and summation. For readability we have relegated some of
the technical subtleties to an appendix.

\subsubsection{Process grammar}\label{subsub:process_grammar}

\begin{mathpar}
  \inferrule* [lab=synchronization] {} {{M} \bc \pzero \;|\; x?F \;|\; x!C }
  \and
  \inferrule* [lab=abstraction] {} {{F} \bc (x)P}
  \and
  \inferrule* [lab=concretion] {} {{C} \bc \langle Q \rangle}
  \and
  \inferrule* [lab=process] {} {{P,Q} \bc M \;| \;P|Q \;|\; @{x}}
  \and
  \inferrule* [lab=name] {} {{x} \bc \quotep{P}}
\end{mathpar} 

Note that $\vec{x}$ (resp. $\vec{P}$) denotes a vector of names
(resp. processes) of length $|\vec{x}|$ (resp. $|\vec{P}|$). We adopt
the following useful abbreviations.

\begin{mathpar}
   x?(\vec{y}).P := x.(\vec{y})P \and  x\clift{\vec{P}} := x.\clift{\vec{P}}
   \and x!(y) := \lift{x}{\dropn{y}}
   \and \Pi_{i=0}^{n-1}P_i := P_0 | \ldots | P_{n-1}
\end{mathpar}

\subsubsection{Structural congruence}

\paragraph{Free and bound names and alpha-equivalence.} At the
core of structural equivalence is alpha-equivalence which identifies
process that are the same up to a change of variable. Formally, we
recognize the distinction between free and bound names. The free names
of a process, $\freenames{P}$, may be calculated recursively as
follows:

\begin{mathpar}
\freenames{\pzero} := \emptyset
  \and \\
  \freenames{x?(y).P} := \{ x \} \cup (\freenames{P} \setminus \{ y \})
  \and 
  \freenames{x!\langle P \rangle} := \{ x \} \cup \{ P \} 
  \and \\
  \freenames{P|Q} := \freenames{P} \cup \freenames{Q}
  \and \\
  \freenames{@{x}} := \{ x \}
\end{mathpar}

$\pi$
$\quotep{\pi}$

$\freenames{-} : \pi \to \mathcal{P}(\quotep{\pi})$

\begin{eqnarray*}
  \freenames{\pzero} & := & \emptyset \\
  \freenames{x?(y).P} & := & \{ x \} \cup (\freenames{P} \setminus \{ y \}) \\
  \freenames{x!\langle P \rangle} & := & \{ x \} \cup \{ P \} \\
  \freenames{P|Q} & := & \freenames{P} \cup \freenames{Q} \\
  \freenames{\dropn{x}} & := & \{ x \}
\end{eqnarray*}

The bound names of a process, $\boundnames{P}$, are those names occurring in $P$
that are not free. For example, in $x?(y).0$, the name $x$ is free, while $y$ is bound.

\begin{mathpar}
  \inferrule* [lab=monoidal-laws] {} { P|Q \equiv Q|P \and P|0 \equiv P \and P|(Q|R) \equiv (P|Q)|R }
\end{mathpar}

\begin{mathpar}
  \inferrule* [lab=alpha-equivalence] {} { (x)P \equiv (y)P\{y/x\} \and y \not\in \freenames{P} }
\end{mathpar}

\begin{definition}
Then two processes, $P,Q$, are alpha-equivalent if $P = Q\{\vec{y}/\vec{x}\}$ for
some $\vec{x} \in \boundnames{Q},\vec{y} \in \boundnames{P}$, where $Q\{\vec{y}/\vec{x}\}$
denotes the capture-avoiding substitution of $\vec{y}$ for $\vec{x}$ in $Q$.
\end{definition}

\begin{definition}
  The {\em structural congruence} \cite{SangiorgiWalker} , $\equiv$,
  between processes is the least congruence containing
  alpha-equivalence, satisfying the abelian monoid laws
  (associativity, commutativity and $\pzero$ as identity) for parallel
  composition $|$ and for summation $+$.
\end{definition}

\subsection{Name equivalence}

We take name equivalence, written $\nameeq$, to be the smallest
equivalence relation generated by the following rules.

\begin{mathpar}
\inferrule*[lab=Quote-drop]
{ }
{ \quotep{@{x}} \nameeq x }

\inferrule*[lab=Struct-equiv]
{ P \scong Q }
{ \quotep{P} \nameeq \quotep{Q} }
\end{mathpar}

The astute reader will have noticed that the mutual recursion of names
and processes imposes a mutual recursion on alpha-equivalence and
structural equivalence via name-equivalence. Fortunately, all of this
works out pleasantly and we may calculate in the natural way, free of
concern. The reader interested in the details is referred to the
appendix \ref{appendix:rho_details}.

\subsection{Substitution}

We use $\Proc$ for the set of processes, $\QProc$ for the set of
names, and $\id{\{}\vec{y} / \vec{x} \id{\}}$ to denote partial maps,
$s : \QProc \rightarrow \QProc$. A map, $s$ lifts, uniquely, to a map
on process terms, $\widehat{s} : \Proc \rightarrow \Proc$ by the
following equations.

\begin{mathpar}
  (0) \psubstp{Q}{P} := 0 \\
  (R \juxtap S) \psubstp{Q}{P}
  :=    
  (R)\psubstp{Q}{P} \juxtap (S) \psubstp{Q}{P} \\
  (x?(y).R) \psubstp{Q}{P}    
  :=    
  (x)\substp{Q}{P} (z)\concat( (R \psubstn{z}{y}) \psubstp{Q}{P} ) \\
  (\lift{x}{R}) \psubstp{Q}{P}  
  :=
  \lift{(x)\substp{Q}{P}}{ R \psubstp{Q}{P} } \\
%   (\dropn{x})  \psubstp{Q}{P}       
%   := 
%   \left\{ 
%     \begin{array}{ccc} 
%       \dropn{\quotep{Q}} & & x \nameeq \quotep{P} \\
%       \dropn{x} & & otherwise \\
%     \end{array}
%   \right. 
  (\dropn{x})  \psubstp{Q}{P}       
  := 
  \left\{ 
    \begin{array}{ccc} 
      Q & & x \nameeq \quotep{P} \\
      \dropn{x} & & otherwise \\
    \end{array}
  \right.
\end{mathpar}
 

where

\begin{eqnarray}
  (x)\id{\{} \lpquote Q \rpquote / \lpquote P \rpquote \id{\}}            = 
  \left\{ 
    \begin{array}{ccc}
      \lpquote Q \rpquote & & x \nameeq \lpquote P \rpquote \\
      x & & otherwise \\
    \end{array}
  \right. \nonumber
\end{eqnarray}

and $z$ is chosen distinct from $\quotep{P}$, $\quotep{Q}$, the free
names in $Q$, and all the names in $R$. Our $\alpha$-equivalence will
be built in the standard way from this substitution.

\begin{remark}\label{rem:no_self_referential_names}
  One consequence of these definitions is that $\forall P. \quotep{P}
  \not\in \freenames{P}$.
\end{remark}

\subsection{ Dynamic quote: an example }

Anticipating something of what's to come, consider applying the
substitution, $\widehat{\id{\{}u / z \id{\}}}$, to the following pair
of processes, $\lift{w}{y!(z)}$ and $w[ \lpquote y!(z) \rpquote ]$.

\begin{eqnarray}
	\lift{w}{y!(z)}\widehat{\id{\{}u / z \id{\}}}
		& = &
		\lift{w}{y!(u)} \nonumber\\
	w[ \lpquote y!(z) \rpquote ] \widehat{ \id{\{}u / z \id{\}} }
		& = &
		w[ \lpquote y!(z) \rpquote ] \nonumber
\end{eqnarray}

Because the body of the process between quotes is impervious to
substitution, we get radically different answers. In fact, by
examining the first process in an input context,
e.g. $x?(z).\lift{w}{y!(z)}$, we see that the process under the lift
operator may be shaped by prefixed inputs binding a name inside it. In
this sense, the lift operator will be seen as a way to dynamically
construct processes before reifying them as names.

Finally equipped with these standard features we can present the
dynamics of the calculus.

\subsubsection{Operational semantics} 

Finally, we introduce the computational dynamics. What marks these
algebras as distinct from other more traditionally studied algebraic
structures, e.g. vector spaces or polynomial rings, is the manner in
which dynamics is captured. In traditional structures, dynamics is typically
expressed through morphisms between such structures, as in linear maps
between vector spaces or morphisms between rings. In algebras
associated with the semantics of computation, the dynamics is
expressed as part of the algebraic structure itself, through a
reduction reduction relation typically denoted by $\red$. Below, we
give a recursive presentation of this relation for the calculus used
in the encoding.

$\red \subseteq \pi \times \pi$
$\red : \pi \to \mathcal{P}(\pi)$

\begin{mathpar}
  \inferrule* [lab=Comm] { \textsf{match}( x_{src}, x_{trgt} ) } { x_{trgt}?(y)P \; | \; x_{src}!\langle {Q} \rangle \red P\{\quotep{Q}/y}\} }
  \and \\
  \inferrule* [lab=Par] {{P} \red {P}'} {{{P} | {Q}} \red {{P}' | {Q}}}
  \and
  \inferrule* [lab=Equiv]{{{P} \scong {P}'} \andalso {{P}' \red {Q}'} \andalso {{Q}' \scong {Q}}}{{P} \red {Q}}
\end{mathpar}

\begin{eqnarray*}
  match_{\equiv} (\quotep{P},\quotep{Q}) & := & P \equiv Q \\
  match_{\dagger}(\quotep{P},\quotep{Q}) & := & \forall R. P|Q \red^{*} R => R \red^{*} 0 \\
  match_{K}(\quotep{P},\quotep{Q}) & := & K \mbox{ for some context } K
\end{eqnarray*}

$u?(x)P | u!\langle Q \rangle \red P\{\quotep{Q}/x\}$

%We write $\wred$ for $\red^*$, and $P\red$ if $\exists Q $ such that $ P \red Q$.
We write $P\red$ if $\exists Q $ such that $ P \red Q$ and $P\not\red$, otherwise.

\section{Replication}

As mentioned before, it is known that replication (and hence
recursion) can be implemented in a higher-order process algebra
\cite{SangiorgiWalker}. As our first example of calculation with the
machinery thus far presented we give the construction explicitly in
the {\rhoc}.

\begin{eqnarray}
	D_{x} & := & \prefix{x}{y}{(\binpar{\outputp{x}{y}}{@{y}})} \nonumber\\
	\bangp_{x}{P} & := & \binpar{{x}!\langle{\binpar{D_{x}}{P}}\rangle}{D_{x}} \nonumber
\end{eqnarray}

\begin{eqnarray}
	\bangp_{x}{P} & & \nonumber\\
	=
	& {x}!\langle{(\prefix{x}{y}{(\outputp{x}{y} | @{y})) | P}}\rangle 
	      | \prefix{x}{y}{(\outputp{x}{y} | @{y})} & \nonumber\\
	\red
	& (\outputp{x}{y} | @{y})\substn{\quotep{(\prefix{x}{y}{(@{y} | \outputp{x}{y})) | P}}}{y} & \nonumber\\
	=
	& \outputp{x}{\quotep{(\prefix{x}{y}{(\outputp{x}{y} | @{y})) | P}}}
	  | {(\prefix{x}{y}{(\outputp{x}{y} | @{y})) | P}} & \nonumber\\
	\red
	& \ldots & \nonumber\\
	\red^*
	& P | P | \ldots & \nonumber
\end{eqnarray}

Of course, this encoding, as an implementation, runs away, unfolding
$\bangp{P}$ eagerly. A lazier and more implementable replication
operator, restricted to input-guarded processes, may be obtained as follows.

\begin{eqnarray}
\bangp{\prefix{u}{v}{P}} 
	:= 
	\binpar{\lift{x}{\prefix{u}{v}{(\binpar{D(x)}{P})}}}{D(x)} \nonumber
\end{eqnarray}

\begin{remark}
  Note that the lazier definition still does not deal with summation
  or mixed summation (i.e. sums over input and output). The reader is
  invited to construct definitions of replication that deal with these
  features. 

  Further, the definitions are parameterized in a name, $x$. Can you,
  gentle reader, make a definition that eliminates this parameter and
  guarantees no accidental interaction between the replication
  machinery and the process being replicated -- i.e. no accidental
  sharing of names used by the process to get its work done and the
  name(s) used by the replication to effect copying. This latter
  revision of the definition of replication is crucial to obtaining
  the expected identity $!!P \sim !P$.
\end{remark}

\begin{remark}\label{rem:paradoxical_combinator}
  The reader familiar with the lambda calculus will have noticed the
  similarity between $D$ and the paradoxical combinator.

  [Ed. note: the existence of this seems to suggest we have to be more
  restrictive on the set of processes and names we admit if we are to
  support no-cloning.]
\end{remark}

\subsubsection{Bisimulation}

The computational dynamics gives rise to another kind of equivalence,
the equivalence of computational behavior. As previously mentioned
this is typically captured \emph{via} some form of bisimulation.

% The notion we use in this paper is weak barbed bisimulation
% \cite{milner91polyadicpi}.

The notion we use in this paper is derived from weak barbed
bisimulation \cite{milner91polyadicpi}. 

\begin{definition}
An \emph{observation relation}, $\downarrow_{\mathcal N}$, over a set
of names, $\mathcal N$, is the smallest relation satisfying the rules
below.

\infrule[Out-barb]{y \in {\mathcal N}, \; x \nameeq y}
		  {\outputp{x}{v} \downarrow_{\mathcal N} x}
\infrule[Par-barb]{\mbox{$P\downarrow_{\mathcal N} x$ or $Q\downarrow_{\mathcal N} x$}}
		  {\binpar{P}{Q} \downarrow_{\mathcal N} x}

We write $P \Downarrow_{\mathcal N} x$ if there is $Q$ such that 
$P \wred Q$ and $Q \downarrow_{\mathcal N} x$.
\end{definition}

\begin{definition}
%\label{def.bbisim}
An  ${\mathcal N}$-\emph{barbed bisimulation} over a set of names, ${\mathcal N}$, is a symmetric binary relation 
${\mathcal S}_{\mathcal N}$ between agents such that $P\rel{S}_{\mathcal N}Q$ implies:
\begin{enumerate}
\item If $P \red P'$ then $Q \wred Q'$ and $P'\rel{S}_{\mathcal N} Q'$.
\item If $P\downarrow_{\mathcal N} x$, then $Q\Downarrow_{\mathcal N} x$.
\end{enumerate}
$P$ is ${\mathcal N}$-barbed bisimilar to $Q$, written
$P \wbbisim_{\mathcal N} Q$, if $P \rel{S}_{\mathcal N} Q$ for some ${\mathcal N}$-barbed bisimulation ${\mathcal S}_{\mathcal N}$.
\end{definition}

$\mathcal{R} \subseteq \pi \times \pi$

$P \mathcal{R} Q => \forall P'. P \red P' \Rightarrow \exists Q'. Q \red Q', P' \mathcal{R} Q'$

$P \vdash x \Rightarrow Q \vdash x$

\begin{mathpar}
  \inferrule*[lab=Out-barb]{x \nameeq y}{{y}!\langle{Q}\rangle \vdash x}
  \and
  \inferrule*[lab=Par-barb]{\mbox{$P\vdash x$ or $Q\vdash x$}}{\binpar{P}{Q} \vdash x}
\end{mathpar}

\subsubsection{Contexts}

One of the principle advantages of computational calculi like the
$\pi$-calculus is a well-defined notion of context,
contextual-equivalence and a correlation between
contextual-equivalence and notions of bisimulation. The notion of
context allows the decomposition of a process into (sub-)process and
its syntactic environment, its context. Thus, a context may be
thought of as a process with a ``hole'' (written $\Box$) in it. The
application of a context $M$ to a process $P$, written $M[P]$, is
tantamount to filling the hole in $M$ with $P$. In this paper we do
not need the full weight of this theory, but do make use of the notion
of context in the proof the main theorem. 

\begin{mathpar}
  \inferrule* [lab=summation] {} {{M_{M},M_{N}} \bc \Box \;|\; x.M_{A} \;|\; M_{M}+M_{N}}
  \and
  \inferrule* [lab=agent] {} {{M_{A}} \bc (\vec{x})M_{P} \;| \; \clift{P_0,\ldots,M_{P},\ldots,P_N}}
  \and \\
  \inferrule* [lab=process] {} {{M_{P}} \bc M_{N} \;| \;P|M_{P} }
\end{mathpar} 

\begin{mathpar}
  \inferrule* [lab=sychronization] {} {M_{N} \bc \Box \;|\; x?M_{F} \;|\; x!M_{C}}
  \and
  \inferrule* [lab=abstraction] {} {{M_{F}} \bc (x)M_{P} }
  \and
  \inferrule* [lab=concretion] {} {{M_{C}} \bc \langle M_{P} \rangle }
  \and \\
  \inferrule* [lab=process] {} {{M_{P}} \bc M_{N} \;| \;P|M_{P} }
\end{mathpar}

\begin{definition}[contextual application] Given a context $M$, and
  process $P$, we define the \emph{contextual application}, $M[P] :=
  M\{P/\Box\}$. That is, the contextual application of M to P is the
  substitution of $P$ for $\Box$ in $M$.
\end{definition}

$\meaningof{-} : L \to \mathcal{P}(\pi)$

\begin{mathpar}
  \inferrule* [lab=collection] {} {\meaningof{true} = \pi, \and \meaningof{~E} = \pi \setminus \meaningof{E}, \and \meaningof{E_{1} \& E_{2}} = \meaningof{E_{1}} \cap \meaningof{E_{2}}}
\end{mathpar}

\begin{mathpar}
  \inferrule* [lab=structure] {} {\meaningof{0} = \{ P \in \pi | P \equiv 0 \}, \and \\ \meaningof{E_1 | E_2} = \{ P \in \pi | P \equiv P_{1} | P_{2}, P_{1} \in \meaningof{E_{1}}, P_{2} \in \meaningof{E_2}\} }
\end{mathpar}

\begin{mathpar}
 \inferrule* [lab=behavior] {} {\meaningof{\langle a?b \rangle E} = \{ P \in \pi | P \equiv Q | u?(y)P', \\ \and \\\\ \and \\ \;\;\; u \in \meaningof{a}, \forall z.P'\{z/y\} \in \meaningof{E\{z/b\}}\}, \and \\ \meaningof{a!E} = \{ P \in \pi | P \equiv Q | x!\langle P' \rangle, x \in \meaningof{a} P' \in \meaningof{E}\} }
\end{mathpar}

\begin{mathpar}
 \inferrule* [lab=nominal] {} {\meaningof{\quotep{E}} = \{ \quotep{P} \in \quotep{\pi} | P \in \meaningof{E} \}, \and \meaningof{\quotep{P}} = \{ \quotep{Q} \in \quotep{\pi} | P \equiv Q \} \and \\ \meaningof{@\quotep{E}} = \{ P \in \pi | P \equiv @x, x \in \meaningof{E} \}}
\end{mathpar}

\begin{eqnarray*}
  \\
  \meaningof{-} : TS \to ST
\end{eqnarray*}

\begin{eqnarray*}
  \\
  L : TS \to ST
\end{eqnarray*}

\begin{eqnarray*}
  \\
  P \models E \iff P \in \meaningof{E}
\end{eqnarray*}

\begin{eqnarray*}
  P \approx_{L} Q \iff \forall E \in L. P \models E \iff Q \models E
\end{eqnarray*}

\begin{eqnarray*}
  P \approx_{K} Q
\end{eqnarray*}

\begin{eqnarray*}
  P \approx Q
\end{eqnarray*}

$\approx_{K} = \approx = \approx_{L}$

\subsubsection{Contextual duality}

Note that contexts extend the quotation operation to a family of
operations from processes to names. Given a context, $M$, we can
define a \emph{nominal context}, $\quotep{M}$ by $\quotep{M}[P] :=
\quotep{M[P]}$. To foreshadow what is to come we observe that these
operations enjoy a duality with processes very much like the duality
between vectors and maps from vectors to scalars.

Further, because the calculus is essentially higher-order, we have a
correspondence between contexts and processes. More specifically,
given a name $x$ and a context $M$ we can construct $M^{*}_{x}$ such
that 

\begin{mathpar}
  M^{*}_{x} | \lift{x}{P} \red M[P]
\end{mathpar}

namely,

\begin{mathpar}
  M^{*}_{x} := x?(u).M[\dropn{u}]
\end{mathpar}

The dependence of $M^{*}_{x}$ on a name makes it an abstraction, 

\begin{mathpar}
  M^{*} := (x)x?(u).M[\dropn{u}]
\end{mathpar}

\subsection{Additional notation}

It will sometimes be convenient to denote the process a name
quotes. We already have the notation $x = \quotep{P}$, but it will be
convenient to introduce an alternate notation, $\procn{x}$, when we
want to emphasize the connection to the use of the name. Note that, by
virtue of name equivalence, $\quotep{\procn{x}} \nameeq x$; so, the
notation is consistent with previous definitions.

Further, because names have structure it is possible to effect
substitutions on the basis of that structure. This means we need to
upgrade our notation for substitutions, which we accomplish by
adapting comprehension notation. Thus,

\begin{mathpar}
  P\{ y / x : x \in S \}
\end{mathpar}

is interpreted to mean the process derived from P by replacing (in a
capture-avoiding manner) each occurrence of $x$ in $S$ by $y$. For example,

\begin{mathpar}
  P\{ \quotep{\procn{x}|\procn{x}} / x : x \in \freenames{P} \}
\end{mathpar}

will replace each (occurrence) of a free name $x$ in $P$ by
$\quotep{\procn{x}|\procn{x}}$.

Also, we will avail ourselves of the notation $x^{L}$ and $x^{R}$ to
denote injections of a name into disjoint copies of the name
space. There are numerous ways to accomplish this. One example can be
found in \cite{MeredithR05}. This notation overloads to vectors of
names: $\vec{x}^{\pi} := (x_{i}^{\pi} \; : \; 0 \leq i < |\vec{x}| )$ where $\pi \in \{L,R\}$.

We also use $P^{\Box} := P|\Box$.

In \cite{MeredithR05} an interpretation of the new operator is
given. It turns out that there are several possible interpretations
all enjoying the requisite algebraic properties of the operator (see
\cite{milner91polyadicpi}). We will therefore make liberal use of
$(\nu\; \vec{x})P$.

% subsection the_syntax_and_semantics_of_the_notation_system (end)   

\input{qm2pi.qmops} 

\input{qm2pi.sterngerlach} 

\input{qm2pi.metric} 

% section concurrent_process_calculi (end)

%\input{qm2pi.proofsketch}

% section proof sketch (end)

%\input{qm2pi.slviaknots} 

% section spatial logic via knots (end)

\input{qm2pi.conclusion}

% section conclusion (end)

%\input{qm2pi.dtcodes} 

% section wiring algorithm (end)

\input{qm2pi.ack} 

% section acknowledgments (end)

\newpage


\bibliographystyle{plain}   
\bibliography{../../biblios/main.bib}

\input{qm2pi.rhodetails}

\end{document}

 

% section notation (end)

\input{qm2pi.process.calculi} 

% section concurrent_process_calculi_and_spatial_logics_ (end)
    
%\documentclass[12pt]{llncs}
%\documentclass{jktr}

\usepackage[pdftex]{hyperref}                   
\usepackage {listings}
\usepackage {mathpartir}
\usepackage{bcprules}
%\usepackage{listings}
                       
\usepackage{graphicx} 
%\usepackage[margins=2.5cm,nohead,nofoot]{geometry}
%\usepackage{geometry}
\usepackage{amsfonts}
\usepackage{amstext}
\usepackage{latexsym}
\usepackage{amssymb}
\usepackage{color}


%\include{myPreamble}
\include{qm2pi.local} 

%\ifpdf
%\usepackage[pdftex]{graphicx}
%\else
%\usepackage{graphicx}
%\fi

 % \ifpdf
%  \usepackage{pdfsync}
%  \if


%\title{Brief Article}
%\author{David F. Snyder}
%\author{L.G. Meredith}

%\address{Dept. of Math., Texas State University--San Marcos, San Marcos, TX 78666}
       
\pagestyle{empty}


\begin{document}

\lstset{language=[Objective]Caml,frame=shadowbox}

\input{qm2pi.front}

% section front matter (end)

\input{qm2pi.intro} 
 
% section introduction (end)

% \input{qm2pi.knotations} 

% section notation (end)

\input{qm2pi.process.calculi} 

% section concurrent_process_calculi_and_spatial_logics_ (end)
    
%\input{qm2pi.knots2pi} 

%\input{qm2pi.trefoil} 

%\input{qm2pi.mainthm} 

% subsection basic_interpretation (end)

%\input{qm2pi.rho.presentation} 
\subsection{The syntax and semantics of the notation system}\label{sub:the_syntax_and_semantics_of_the_notation_system} % (fold)

We now summarize a technical presentation of the calculus that
embodies our theory of dynamics. The typical presentation of such a
calculus follows the style of giving generators and relations on
them. The grammar, below, describing term constructors, freely
generates the set of processes, $\Proc$. This set is then quotiented
by a relation known as structural congruence and it is over this set
that the notion of dynamics is expressed. This presentation is
essentially that of \cite{MeredithR05} with the addition of
polyadicity and summation. For readability we have relegated some of
the technical subtleties to an appendix.

\subsubsection{Process grammar}\label{subsub:process_grammar}

\begin{mathpar}
  \inferrule* [lab=synchronization] {} {{M} \bc \pzero \;|\; x?F \;|\; x!C }
  \and
  \inferrule* [lab=abstraction] {} {{F} \bc (x)P}
  \and
  \inferrule* [lab=concretion] {} {{C} \bc \langle Q \rangle}
  \and
  \inferrule* [lab=process] {} {{P,Q} \bc M \;| \;P|Q \;|\; @{x}}
  \and
  \inferrule* [lab=name] {} {{x} \bc \quotep{P}}
\end{mathpar} 

Note that $\vec{x}$ (resp. $\vec{P}$) denotes a vector of names
(resp. processes) of length $|\vec{x}|$ (resp. $|\vec{P}|$). We adopt
the following useful abbreviations.

\begin{mathpar}
   x?(\vec{y}).P := x.(\vec{y})P \and  x\clift{\vec{P}} := x.\clift{\vec{P}}
   \and x!(y) := \lift{x}{\dropn{y}}
   \and \Pi_{i=0}^{n-1}P_i := P_0 | \ldots | P_{n-1}
\end{mathpar}

\subsubsection{Structural congruence}

\paragraph{Free and bound names and alpha-equivalence.} At the
core of structural equivalence is alpha-equivalence which identifies
process that are the same up to a change of variable. Formally, we
recognize the distinction between free and bound names. The free names
of a process, $\freenames{P}$, may be calculated recursively as
follows:

\begin{mathpar}
\freenames{\pzero} := \emptyset
  \and \\
  \freenames{x?(y).P} := \{ x \} \cup (\freenames{P} \setminus \{ y \})
  \and 
  \freenames{x!\langle P \rangle} := \{ x \} \cup \{ P \} 
  \and \\
  \freenames{P|Q} := \freenames{P} \cup \freenames{Q}
  \and \\
  \freenames{@{x}} := \{ x \}
\end{mathpar}

$\pi$
$\quotep{\pi}$

$\freenames{-} : \pi \to \mathcal{P}(\quotep{\pi})$

\begin{eqnarray*}
  \freenames{\pzero} & := & \emptyset \\
  \freenames{x?(y).P} & := & \{ x \} \cup (\freenames{P} \setminus \{ y \}) \\
  \freenames{x!\langle P \rangle} & := & \{ x \} \cup \{ P \} \\
  \freenames{P|Q} & := & \freenames{P} \cup \freenames{Q} \\
  \freenames{\dropn{x}} & := & \{ x \}
\end{eqnarray*}

The bound names of a process, $\boundnames{P}$, are those names occurring in $P$
that are not free. For example, in $x?(y).0$, the name $x$ is free, while $y$ is bound.

\begin{mathpar}
  \inferrule* [lab=monoidal-laws] {} { P|Q \equiv Q|P \and P|0 \equiv P \and P|(Q|R) \equiv (P|Q)|R }
\end{mathpar}

\begin{mathpar}
  \inferrule* [lab=alpha-equivalence] {} { (x)P \equiv (y)P\{y/x\} \and y \not\in \freenames{P} }
\end{mathpar}

\begin{definition}
Then two processes, $P,Q$, are alpha-equivalent if $P = Q\{\vec{y}/\vec{x}\}$ for
some $\vec{x} \in \boundnames{Q},\vec{y} \in \boundnames{P}$, where $Q\{\vec{y}/\vec{x}\}$
denotes the capture-avoiding substitution of $\vec{y}$ for $\vec{x}$ in $Q$.
\end{definition}

\begin{definition}
  The {\em structural congruence} \cite{SangiorgiWalker} , $\equiv$,
  between processes is the least congruence containing
  alpha-equivalence, satisfying the abelian monoid laws
  (associativity, commutativity and $\pzero$ as identity) for parallel
  composition $|$ and for summation $+$.
\end{definition}

\subsection{Name equivalence}

We take name equivalence, written $\nameeq$, to be the smallest
equivalence relation generated by the following rules.

\begin{mathpar}
\inferrule*[lab=Quote-drop]
{ }
{ \quotep{@{x}} \nameeq x }

\inferrule*[lab=Struct-equiv]
{ P \scong Q }
{ \quotep{P} \nameeq \quotep{Q} }
\end{mathpar}

The astute reader will have noticed that the mutual recursion of names
and processes imposes a mutual recursion on alpha-equivalence and
structural equivalence via name-equivalence. Fortunately, all of this
works out pleasantly and we may calculate in the natural way, free of
concern. The reader interested in the details is referred to the
appendix \ref{appendix:rho_details}.

\subsection{Substitution}

We use $\Proc$ for the set of processes, $\QProc$ for the set of
names, and $\id{\{}\vec{y} / \vec{x} \id{\}}$ to denote partial maps,
$s : \QProc \rightarrow \QProc$. A map, $s$ lifts, uniquely, to a map
on process terms, $\widehat{s} : \Proc \rightarrow \Proc$ by the
following equations.

\begin{mathpar}
  (0) \psubstp{Q}{P} := 0 \\
  (R \juxtap S) \psubstp{Q}{P}
  :=    
  (R)\psubstp{Q}{P} \juxtap (S) \psubstp{Q}{P} \\
  (x?(y).R) \psubstp{Q}{P}    
  :=    
  (x)\substp{Q}{P} (z)\concat( (R \psubstn{z}{y}) \psubstp{Q}{P} ) \\
  (\lift{x}{R}) \psubstp{Q}{P}  
  :=
  \lift{(x)\substp{Q}{P}}{ R \psubstp{Q}{P} } \\
%   (\dropn{x})  \psubstp{Q}{P}       
%   := 
%   \left\{ 
%     \begin{array}{ccc} 
%       \dropn{\quotep{Q}} & & x \nameeq \quotep{P} \\
%       \dropn{x} & & otherwise \\
%     \end{array}
%   \right. 
  (\dropn{x})  \psubstp{Q}{P}       
  := 
  \left\{ 
    \begin{array}{ccc} 
      Q & & x \nameeq \quotep{P} \\
      \dropn{x} & & otherwise \\
    \end{array}
  \right.
\end{mathpar}
 

where

\begin{eqnarray}
  (x)\id{\{} \lpquote Q \rpquote / \lpquote P \rpquote \id{\}}            = 
  \left\{ 
    \begin{array}{ccc}
      \lpquote Q \rpquote & & x \nameeq \lpquote P \rpquote \\
      x & & otherwise \\
    \end{array}
  \right. \nonumber
\end{eqnarray}

and $z$ is chosen distinct from $\quotep{P}$, $\quotep{Q}$, the free
names in $Q$, and all the names in $R$. Our $\alpha$-equivalence will
be built in the standard way from this substitution.

\begin{remark}\label{rem:no_self_referential_names}
  One consequence of these definitions is that $\forall P. \quotep{P}
  \not\in \freenames{P}$.
\end{remark}

\subsection{ Dynamic quote: an example }

Anticipating something of what's to come, consider applying the
substitution, $\widehat{\id{\{}u / z \id{\}}}$, to the following pair
of processes, $\lift{w}{y!(z)}$ and $w[ \lpquote y!(z) \rpquote ]$.

\begin{eqnarray}
	\lift{w}{y!(z)}\widehat{\id{\{}u / z \id{\}}}
		& = &
		\lift{w}{y!(u)} \nonumber\\
	w[ \lpquote y!(z) \rpquote ] \widehat{ \id{\{}u / z \id{\}} }
		& = &
		w[ \lpquote y!(z) \rpquote ] \nonumber
\end{eqnarray}

Because the body of the process between quotes is impervious to
substitution, we get radically different answers. In fact, by
examining the first process in an input context,
e.g. $x?(z).\lift{w}{y!(z)}$, we see that the process under the lift
operator may be shaped by prefixed inputs binding a name inside it. In
this sense, the lift operator will be seen as a way to dynamically
construct processes before reifying them as names.

Finally equipped with these standard features we can present the
dynamics of the calculus.

\subsubsection{Operational semantics} 

Finally, we introduce the computational dynamics. What marks these
algebras as distinct from other more traditionally studied algebraic
structures, e.g. vector spaces or polynomial rings, is the manner in
which dynamics is captured. In traditional structures, dynamics is typically
expressed through morphisms between such structures, as in linear maps
between vector spaces or morphisms between rings. In algebras
associated with the semantics of computation, the dynamics is
expressed as part of the algebraic structure itself, through a
reduction reduction relation typically denoted by $\red$. Below, we
give a recursive presentation of this relation for the calculus used
in the encoding.

$\red \subseteq \pi \times \pi$
$\red : \pi \to \mathcal{P}(\pi)$

\begin{mathpar}
  \inferrule* [lab=Comm] { \textsf{match}( x_{src}, x_{trgt} ) } { x_{trgt}?(y)P \; | \; x_{src}!\langle {Q} \rangle \red P\{\quotep{Q}/y}\} }
  \and \\
  \inferrule* [lab=Par] {{P} \red {P}'} {{{P} | {Q}} \red {{P}' | {Q}}}
  \and
  \inferrule* [lab=Equiv]{{{P} \scong {P}'} \andalso {{P}' \red {Q}'} \andalso {{Q}' \scong {Q}}}{{P} \red {Q}}
\end{mathpar}

\begin{eqnarray*}
  match_{\equiv} (\quotep{P},\quotep{Q}) & := & P \equiv Q \\
  match_{\dagger}(\quotep{P},\quotep{Q}) & := & \forall R. P|Q \red^{*} R => R \red^{*} 0 \\
  match_{K}(\quotep{P},\quotep{Q}) & := & K \mbox{ for some context } K
\end{eqnarray*}

$u?(x)P | u!\langle Q \rangle \red P\{\quotep{Q}/x\}$

%We write $\wred$ for $\red^*$, and $P\red$ if $\exists Q $ such that $ P \red Q$.
We write $P\red$ if $\exists Q $ such that $ P \red Q$ and $P\not\red$, otherwise.

\section{Replication}

As mentioned before, it is known that replication (and hence
recursion) can be implemented in a higher-order process algebra
\cite{SangiorgiWalker}. As our first example of calculation with the
machinery thus far presented we give the construction explicitly in
the {\rhoc}.

\begin{eqnarray}
	D_{x} & := & \prefix{x}{y}{(\binpar{\outputp{x}{y}}{@{y}})} \nonumber\\
	\bangp_{x}{P} & := & \binpar{{x}!\langle{\binpar{D_{x}}{P}}\rangle}{D_{x}} \nonumber
\end{eqnarray}

\begin{eqnarray}
	\bangp_{x}{P} & & \nonumber\\
	=
	& {x}!\langle{(\prefix{x}{y}{(\outputp{x}{y} | @{y})) | P}}\rangle 
	      | \prefix{x}{y}{(\outputp{x}{y} | @{y})} & \nonumber\\
	\red
	& (\outputp{x}{y} | @{y})\substn{\quotep{(\prefix{x}{y}{(@{y} | \outputp{x}{y})) | P}}}{y} & \nonumber\\
	=
	& \outputp{x}{\quotep{(\prefix{x}{y}{(\outputp{x}{y} | @{y})) | P}}}
	  | {(\prefix{x}{y}{(\outputp{x}{y} | @{y})) | P}} & \nonumber\\
	\red
	& \ldots & \nonumber\\
	\red^*
	& P | P | \ldots & \nonumber
\end{eqnarray}

Of course, this encoding, as an implementation, runs away, unfolding
$\bangp{P}$ eagerly. A lazier and more implementable replication
operator, restricted to input-guarded processes, may be obtained as follows.

\begin{eqnarray}
\bangp{\prefix{u}{v}{P}} 
	:= 
	\binpar{\lift{x}{\prefix{u}{v}{(\binpar{D(x)}{P})}}}{D(x)} \nonumber
\end{eqnarray}

\begin{remark}
  Note that the lazier definition still does not deal with summation
  or mixed summation (i.e. sums over input and output). The reader is
  invited to construct definitions of replication that deal with these
  features. 

  Further, the definitions are parameterized in a name, $x$. Can you,
  gentle reader, make a definition that eliminates this parameter and
  guarantees no accidental interaction between the replication
  machinery and the process being replicated -- i.e. no accidental
  sharing of names used by the process to get its work done and the
  name(s) used by the replication to effect copying. This latter
  revision of the definition of replication is crucial to obtaining
  the expected identity $!!P \sim !P$.
\end{remark}

\begin{remark}\label{rem:paradoxical_combinator}
  The reader familiar with the lambda calculus will have noticed the
  similarity between $D$ and the paradoxical combinator.

  [Ed. note: the existence of this seems to suggest we have to be more
  restrictive on the set of processes and names we admit if we are to
  support no-cloning.]
\end{remark}

\subsubsection{Bisimulation}

The computational dynamics gives rise to another kind of equivalence,
the equivalence of computational behavior. As previously mentioned
this is typically captured \emph{via} some form of bisimulation.

% The notion we use in this paper is weak barbed bisimulation
% \cite{milner91polyadicpi}.

The notion we use in this paper is derived from weak barbed
bisimulation \cite{milner91polyadicpi}. 

\begin{definition}
An \emph{observation relation}, $\downarrow_{\mathcal N}$, over a set
of names, $\mathcal N$, is the smallest relation satisfying the rules
below.

\infrule[Out-barb]{y \in {\mathcal N}, \; x \nameeq y}
		  {\outputp{x}{v} \downarrow_{\mathcal N} x}
\infrule[Par-barb]{\mbox{$P\downarrow_{\mathcal N} x$ or $Q\downarrow_{\mathcal N} x$}}
		  {\binpar{P}{Q} \downarrow_{\mathcal N} x}

We write $P \Downarrow_{\mathcal N} x$ if there is $Q$ such that 
$P \wred Q$ and $Q \downarrow_{\mathcal N} x$.
\end{definition}

\begin{definition}
%\label{def.bbisim}
An  ${\mathcal N}$-\emph{barbed bisimulation} over a set of names, ${\mathcal N}$, is a symmetric binary relation 
${\mathcal S}_{\mathcal N}$ between agents such that $P\rel{S}_{\mathcal N}Q$ implies:
\begin{enumerate}
\item If $P \red P'$ then $Q \wred Q'$ and $P'\rel{S}_{\mathcal N} Q'$.
\item If $P\downarrow_{\mathcal N} x$, then $Q\Downarrow_{\mathcal N} x$.
\end{enumerate}
$P$ is ${\mathcal N}$-barbed bisimilar to $Q$, written
$P \wbbisim_{\mathcal N} Q$, if $P \rel{S}_{\mathcal N} Q$ for some ${\mathcal N}$-barbed bisimulation ${\mathcal S}_{\mathcal N}$.
\end{definition}

$\mathcal{R} \subseteq \pi \times \pi$

$P \mathcal{R} Q => \forall P'. P \red P' \Rightarrow \exists Q'. Q \red Q', P' \mathcal{R} Q'$

$P \vdash x \Rightarrow Q \vdash x$

\begin{mathpar}
  \inferrule*[lab=Out-barb]{x \nameeq y}{{y}!\langle{Q}\rangle \vdash x}
  \and
  \inferrule*[lab=Par-barb]{\mbox{$P\vdash x$ or $Q\vdash x$}}{\binpar{P}{Q} \vdash x}
\end{mathpar}

\subsubsection{Contexts}

One of the principle advantages of computational calculi like the
$\pi$-calculus is a well-defined notion of context,
contextual-equivalence and a correlation between
contextual-equivalence and notions of bisimulation. The notion of
context allows the decomposition of a process into (sub-)process and
its syntactic environment, its context. Thus, a context may be
thought of as a process with a ``hole'' (written $\Box$) in it. The
application of a context $M$ to a process $P$, written $M[P]$, is
tantamount to filling the hole in $M$ with $P$. In this paper we do
not need the full weight of this theory, but do make use of the notion
of context in the proof the main theorem. 

\begin{mathpar}
  \inferrule* [lab=summation] {} {{M_{M},M_{N}} \bc \Box \;|\; x.M_{A} \;|\; M_{M}+M_{N}}
  \and
  \inferrule* [lab=agent] {} {{M_{A}} \bc (\vec{x})M_{P} \;| \; \clift{P_0,\ldots,M_{P},\ldots,P_N}}
  \and \\
  \inferrule* [lab=process] {} {{M_{P}} \bc M_{N} \;| \;P|M_{P} }
\end{mathpar} 

\begin{mathpar}
  \inferrule* [lab=sychronization] {} {M_{N} \bc \Box \;|\; x?M_{F} \;|\; x!M_{C}}
  \and
  \inferrule* [lab=abstraction] {} {{M_{F}} \bc (x)M_{P} }
  \and
  \inferrule* [lab=concretion] {} {{M_{C}} \bc \langle M_{P} \rangle }
  \and \\
  \inferrule* [lab=process] {} {{M_{P}} \bc M_{N} \;| \;P|M_{P} }
\end{mathpar}

\begin{definition}[contextual application] Given a context $M$, and
  process $P$, we define the \emph{contextual application}, $M[P] :=
  M\{P/\Box\}$. That is, the contextual application of M to P is the
  substitution of $P$ for $\Box$ in $M$.
\end{definition}

$\meaningof{-} : L \to \mathcal{P}(\pi)$

\begin{mathpar}
  \inferrule* [lab=collection] {} {\meaningof{true} = \pi, \and \meaningof{~E} = \pi \setminus \meaningof{E}, \and \meaningof{E_{1} \& E_{2}} = \meaningof{E_{1}} \cap \meaningof{E_{2}}}
\end{mathpar}

\begin{mathpar}
  \inferrule* [lab=structure] {} {\meaningof{0} = \{ P \in \pi | P \equiv 0 \}, \and \\ \meaningof{E_1 | E_2} = \{ P \in \pi | P \equiv P_{1} | P_{2}, P_{1} \in \meaningof{E_{1}}, P_{2} \in \meaningof{E_2}\} }
\end{mathpar}

\begin{mathpar}
 \inferrule* [lab=behavior] {} {\meaningof{\langle a?b \rangle E} = \{ P \in \pi | P \equiv Q | u?(y)P', \\ \and \\\\ \and \\ \;\;\; u \in \meaningof{a}, \forall z.P'\{z/y\} \in \meaningof{E\{z/b\}}\}, \and \\ \meaningof{a!E} = \{ P \in \pi | P \equiv Q | x!\langle P' \rangle, x \in \meaningof{a} P' \in \meaningof{E}\} }
\end{mathpar}

\begin{mathpar}
 \inferrule* [lab=nominal] {} {\meaningof{\quotep{E}} = \{ \quotep{P} \in \quotep{\pi} | P \in \meaningof{E} \}, \and \meaningof{\quotep{P}} = \{ \quotep{Q} \in \quotep{\pi} | P \equiv Q \} \and \\ \meaningof{@\quotep{E}} = \{ P \in \pi | P \equiv @x, x \in \meaningof{E} \}}
\end{mathpar}

\begin{eqnarray*}
  \\
  \meaningof{-} : TS \to ST
\end{eqnarray*}

\begin{eqnarray*}
  \\
  L : TS \to ST
\end{eqnarray*}

\begin{eqnarray*}
  \\
  P \models E \iff P \in \meaningof{E}
\end{eqnarray*}

\begin{eqnarray*}
  P \approx_{L} Q \iff \forall E \in L. P \models E \iff Q \models E
\end{eqnarray*}

\begin{eqnarray*}
  P \approx_{K} Q
\end{eqnarray*}

\begin{eqnarray*}
  P \approx Q
\end{eqnarray*}

$\approx_{K} = \approx = \approx_{L}$

\subsubsection{Contextual duality}

Note that contexts extend the quotation operation to a family of
operations from processes to names. Given a context, $M$, we can
define a \emph{nominal context}, $\quotep{M}$ by $\quotep{M}[P] :=
\quotep{M[P]}$. To foreshadow what is to come we observe that these
operations enjoy a duality with processes very much like the duality
between vectors and maps from vectors to scalars.

Further, because the calculus is essentially higher-order, we have a
correspondence between contexts and processes. More specifically,
given a name $x$ and a context $M$ we can construct $M^{*}_{x}$ such
that 

\begin{mathpar}
  M^{*}_{x} | \lift{x}{P} \red M[P]
\end{mathpar}

namely,

\begin{mathpar}
  M^{*}_{x} := x?(u).M[\dropn{u}]
\end{mathpar}

The dependence of $M^{*}_{x}$ on a name makes it an abstraction, 

\begin{mathpar}
  M^{*} := (x)x?(u).M[\dropn{u}]
\end{mathpar}

\subsection{Additional notation}

It will sometimes be convenient to denote the process a name
quotes. We already have the notation $x = \quotep{P}$, but it will be
convenient to introduce an alternate notation, $\procn{x}$, when we
want to emphasize the connection to the use of the name. Note that, by
virtue of name equivalence, $\quotep{\procn{x}} \nameeq x$; so, the
notation is consistent with previous definitions.

Further, because names have structure it is possible to effect
substitutions on the basis of that structure. This means we need to
upgrade our notation for substitutions, which we accomplish by
adapting comprehension notation. Thus,

\begin{mathpar}
  P\{ y / x : x \in S \}
\end{mathpar}

is interpreted to mean the process derived from P by replacing (in a
capture-avoiding manner) each occurrence of $x$ in $S$ by $y$. For example,

\begin{mathpar}
  P\{ \quotep{\procn{x}|\procn{x}} / x : x \in \freenames{P} \}
\end{mathpar}

will replace each (occurrence) of a free name $x$ in $P$ by
$\quotep{\procn{x}|\procn{x}}$.

Also, we will avail ourselves of the notation $x^{L}$ and $x^{R}$ to
denote injections of a name into disjoint copies of the name
space. There are numerous ways to accomplish this. One example can be
found in \cite{MeredithR05}. This notation overloads to vectors of
names: $\vec{x}^{\pi} := (x_{i}^{\pi} \; : \; 0 \leq i < |\vec{x}| )$ where $\pi \in \{L,R\}$.

We also use $P^{\Box} := P|\Box$.

In \cite{MeredithR05} an interpretation of the new operator is
given. It turns out that there are several possible interpretations
all enjoying the requisite algebraic properties of the operator (see
\cite{milner91polyadicpi}). We will therefore make liberal use of
$(\nu\; \vec{x})P$.

% subsection the_syntax_and_semantics_of_the_notation_system (end)   

\input{qm2pi.qmops} 

\input{qm2pi.sterngerlach} 

\input{qm2pi.metric} 

% section concurrent_process_calculi (end)

%\input{qm2pi.proofsketch}

% section proof sketch (end)

%\input{qm2pi.slviaknots} 

% section spatial logic via knots (end)

\input{qm2pi.conclusion}

% section conclusion (end)

%\input{qm2pi.dtcodes} 

% section wiring algorithm (end)

\input{qm2pi.ack} 

% section acknowledgments (end)

\newpage


\bibliographystyle{plain}   
\bibliography{../../biblios/main.bib}

\input{qm2pi.rhodetails}

\end{document}

 

%\documentclass[12pt]{llncs}
%\documentclass{jktr}

\usepackage[pdftex]{hyperref}                   
\usepackage {listings}
\usepackage {mathpartir}
\usepackage{bcprules}
%\usepackage{listings}
                       
\usepackage{graphicx} 
%\usepackage[margins=2.5cm,nohead,nofoot]{geometry}
%\usepackage{geometry}
\usepackage{amsfonts}
\usepackage{amstext}
\usepackage{latexsym}
\usepackage{amssymb}
\usepackage{color}


%\include{myPreamble}
\include{qm2pi.local} 

%\ifpdf
%\usepackage[pdftex]{graphicx}
%\else
%\usepackage{graphicx}
%\fi

 % \ifpdf
%  \usepackage{pdfsync}
%  \if


%\title{Brief Article}
%\author{David F. Snyder}
%\author{L.G. Meredith}

%\address{Dept. of Math., Texas State University--San Marcos, San Marcos, TX 78666}
       
\pagestyle{empty}


\begin{document}

\lstset{language=[Objective]Caml,frame=shadowbox}

\input{qm2pi.front}

% section front matter (end)

\input{qm2pi.intro} 
 
% section introduction (end)

% \input{qm2pi.knotations} 

% section notation (end)

\input{qm2pi.process.calculi} 

% section concurrent_process_calculi_and_spatial_logics_ (end)
    
%\input{qm2pi.knots2pi} 

%\input{qm2pi.trefoil} 

%\input{qm2pi.mainthm} 

% subsection basic_interpretation (end)

%\input{qm2pi.rho.presentation} 
\subsection{The syntax and semantics of the notation system}\label{sub:the_syntax_and_semantics_of_the_notation_system} % (fold)

We now summarize a technical presentation of the calculus that
embodies our theory of dynamics. The typical presentation of such a
calculus follows the style of giving generators and relations on
them. The grammar, below, describing term constructors, freely
generates the set of processes, $\Proc$. This set is then quotiented
by a relation known as structural congruence and it is over this set
that the notion of dynamics is expressed. This presentation is
essentially that of \cite{MeredithR05} with the addition of
polyadicity and summation. For readability we have relegated some of
the technical subtleties to an appendix.

\subsubsection{Process grammar}\label{subsub:process_grammar}

\begin{mathpar}
  \inferrule* [lab=synchronization] {} {{M} \bc \pzero \;|\; x?F \;|\; x!C }
  \and
  \inferrule* [lab=abstraction] {} {{F} \bc (x)P}
  \and
  \inferrule* [lab=concretion] {} {{C} \bc \langle Q \rangle}
  \and
  \inferrule* [lab=process] {} {{P,Q} \bc M \;| \;P|Q \;|\; @{x}}
  \and
  \inferrule* [lab=name] {} {{x} \bc \quotep{P}}
\end{mathpar} 

Note that $\vec{x}$ (resp. $\vec{P}$) denotes a vector of names
(resp. processes) of length $|\vec{x}|$ (resp. $|\vec{P}|$). We adopt
the following useful abbreviations.

\begin{mathpar}
   x?(\vec{y}).P := x.(\vec{y})P \and  x\clift{\vec{P}} := x.\clift{\vec{P}}
   \and x!(y) := \lift{x}{\dropn{y}}
   \and \Pi_{i=0}^{n-1}P_i := P_0 | \ldots | P_{n-1}
\end{mathpar}

\subsubsection{Structural congruence}

\paragraph{Free and bound names and alpha-equivalence.} At the
core of structural equivalence is alpha-equivalence which identifies
process that are the same up to a change of variable. Formally, we
recognize the distinction between free and bound names. The free names
of a process, $\freenames{P}$, may be calculated recursively as
follows:

\begin{mathpar}
\freenames{\pzero} := \emptyset
  \and \\
  \freenames{x?(y).P} := \{ x \} \cup (\freenames{P} \setminus \{ y \})
  \and 
  \freenames{x!\langle P \rangle} := \{ x \} \cup \{ P \} 
  \and \\
  \freenames{P|Q} := \freenames{P} \cup \freenames{Q}
  \and \\
  \freenames{@{x}} := \{ x \}
\end{mathpar}

$\pi$
$\quotep{\pi}$

$\freenames{-} : \pi \to \mathcal{P}(\quotep{\pi})$

\begin{eqnarray*}
  \freenames{\pzero} & := & \emptyset \\
  \freenames{x?(y).P} & := & \{ x \} \cup (\freenames{P} \setminus \{ y \}) \\
  \freenames{x!\langle P \rangle} & := & \{ x \} \cup \{ P \} \\
  \freenames{P|Q} & := & \freenames{P} \cup \freenames{Q} \\
  \freenames{\dropn{x}} & := & \{ x \}
\end{eqnarray*}

The bound names of a process, $\boundnames{P}$, are those names occurring in $P$
that are not free. For example, in $x?(y).0$, the name $x$ is free, while $y$ is bound.

\begin{mathpar}
  \inferrule* [lab=monoidal-laws] {} { P|Q \equiv Q|P \and P|0 \equiv P \and P|(Q|R) \equiv (P|Q)|R }
\end{mathpar}

\begin{mathpar}
  \inferrule* [lab=alpha-equivalence] {} { (x)P \equiv (y)P\{y/x\} \and y \not\in \freenames{P} }
\end{mathpar}

\begin{definition}
Then two processes, $P,Q$, are alpha-equivalent if $P = Q\{\vec{y}/\vec{x}\}$ for
some $\vec{x} \in \boundnames{Q},\vec{y} \in \boundnames{P}$, where $Q\{\vec{y}/\vec{x}\}$
denotes the capture-avoiding substitution of $\vec{y}$ for $\vec{x}$ in $Q$.
\end{definition}

\begin{definition}
  The {\em structural congruence} \cite{SangiorgiWalker} , $\equiv$,
  between processes is the least congruence containing
  alpha-equivalence, satisfying the abelian monoid laws
  (associativity, commutativity and $\pzero$ as identity) for parallel
  composition $|$ and for summation $+$.
\end{definition}

\subsection{Name equivalence}

We take name equivalence, written $\nameeq$, to be the smallest
equivalence relation generated by the following rules.

\begin{mathpar}
\inferrule*[lab=Quote-drop]
{ }
{ \quotep{@{x}} \nameeq x }

\inferrule*[lab=Struct-equiv]
{ P \scong Q }
{ \quotep{P} \nameeq \quotep{Q} }
\end{mathpar}

The astute reader will have noticed that the mutual recursion of names
and processes imposes a mutual recursion on alpha-equivalence and
structural equivalence via name-equivalence. Fortunately, all of this
works out pleasantly and we may calculate in the natural way, free of
concern. The reader interested in the details is referred to the
appendix \ref{appendix:rho_details}.

\subsection{Substitution}

We use $\Proc$ for the set of processes, $\QProc$ for the set of
names, and $\id{\{}\vec{y} / \vec{x} \id{\}}$ to denote partial maps,
$s : \QProc \rightarrow \QProc$. A map, $s$ lifts, uniquely, to a map
on process terms, $\widehat{s} : \Proc \rightarrow \Proc$ by the
following equations.

\begin{mathpar}
  (0) \psubstp{Q}{P} := 0 \\
  (R \juxtap S) \psubstp{Q}{P}
  :=    
  (R)\psubstp{Q}{P} \juxtap (S) \psubstp{Q}{P} \\
  (x?(y).R) \psubstp{Q}{P}    
  :=    
  (x)\substp{Q}{P} (z)\concat( (R \psubstn{z}{y}) \psubstp{Q}{P} ) \\
  (\lift{x}{R}) \psubstp{Q}{P}  
  :=
  \lift{(x)\substp{Q}{P}}{ R \psubstp{Q}{P} } \\
%   (\dropn{x})  \psubstp{Q}{P}       
%   := 
%   \left\{ 
%     \begin{array}{ccc} 
%       \dropn{\quotep{Q}} & & x \nameeq \quotep{P} \\
%       \dropn{x} & & otherwise \\
%     \end{array}
%   \right. 
  (\dropn{x})  \psubstp{Q}{P}       
  := 
  \left\{ 
    \begin{array}{ccc} 
      Q & & x \nameeq \quotep{P} \\
      \dropn{x} & & otherwise \\
    \end{array}
  \right.
\end{mathpar}
 

where

\begin{eqnarray}
  (x)\id{\{} \lpquote Q \rpquote / \lpquote P \rpquote \id{\}}            = 
  \left\{ 
    \begin{array}{ccc}
      \lpquote Q \rpquote & & x \nameeq \lpquote P \rpquote \\
      x & & otherwise \\
    \end{array}
  \right. \nonumber
\end{eqnarray}

and $z$ is chosen distinct from $\quotep{P}$, $\quotep{Q}$, the free
names in $Q$, and all the names in $R$. Our $\alpha$-equivalence will
be built in the standard way from this substitution.

\begin{remark}\label{rem:no_self_referential_names}
  One consequence of these definitions is that $\forall P. \quotep{P}
  \not\in \freenames{P}$.
\end{remark}

\subsection{ Dynamic quote: an example }

Anticipating something of what's to come, consider applying the
substitution, $\widehat{\id{\{}u / z \id{\}}}$, to the following pair
of processes, $\lift{w}{y!(z)}$ and $w[ \lpquote y!(z) \rpquote ]$.

\begin{eqnarray}
	\lift{w}{y!(z)}\widehat{\id{\{}u / z \id{\}}}
		& = &
		\lift{w}{y!(u)} \nonumber\\
	w[ \lpquote y!(z) \rpquote ] \widehat{ \id{\{}u / z \id{\}} }
		& = &
		w[ \lpquote y!(z) \rpquote ] \nonumber
\end{eqnarray}

Because the body of the process between quotes is impervious to
substitution, we get radically different answers. In fact, by
examining the first process in an input context,
e.g. $x?(z).\lift{w}{y!(z)}$, we see that the process under the lift
operator may be shaped by prefixed inputs binding a name inside it. In
this sense, the lift operator will be seen as a way to dynamically
construct processes before reifying them as names.

Finally equipped with these standard features we can present the
dynamics of the calculus.

\subsubsection{Operational semantics} 

Finally, we introduce the computational dynamics. What marks these
algebras as distinct from other more traditionally studied algebraic
structures, e.g. vector spaces or polynomial rings, is the manner in
which dynamics is captured. In traditional structures, dynamics is typically
expressed through morphisms between such structures, as in linear maps
between vector spaces or morphisms between rings. In algebras
associated with the semantics of computation, the dynamics is
expressed as part of the algebraic structure itself, through a
reduction reduction relation typically denoted by $\red$. Below, we
give a recursive presentation of this relation for the calculus used
in the encoding.

$\red \subseteq \pi \times \pi$
$\red : \pi \to \mathcal{P}(\pi)$

\begin{mathpar}
  \inferrule* [lab=Comm] { \textsf{match}( x_{src}, x_{trgt} ) } { x_{trgt}?(y)P \; | \; x_{src}!\langle {Q} \rangle \red P\{\quotep{Q}/y}\} }
  \and \\
  \inferrule* [lab=Par] {{P} \red {P}'} {{{P} | {Q}} \red {{P}' | {Q}}}
  \and
  \inferrule* [lab=Equiv]{{{P} \scong {P}'} \andalso {{P}' \red {Q}'} \andalso {{Q}' \scong {Q}}}{{P} \red {Q}}
\end{mathpar}

\begin{eqnarray*}
  match_{\equiv} (\quotep{P},\quotep{Q}) & := & P \equiv Q \\
  match_{\dagger}(\quotep{P},\quotep{Q}) & := & \forall R. P|Q \red^{*} R => R \red^{*} 0 \\
  match_{K}(\quotep{P},\quotep{Q}) & := & K \mbox{ for some context } K
\end{eqnarray*}

$u?(x)P | u!\langle Q \rangle \red P\{\quotep{Q}/x\}$

%We write $\wred$ for $\red^*$, and $P\red$ if $\exists Q $ such that $ P \red Q$.
We write $P\red$ if $\exists Q $ such that $ P \red Q$ and $P\not\red$, otherwise.

\section{Replication}

As mentioned before, it is known that replication (and hence
recursion) can be implemented in a higher-order process algebra
\cite{SangiorgiWalker}. As our first example of calculation with the
machinery thus far presented we give the construction explicitly in
the {\rhoc}.

\begin{eqnarray}
	D_{x} & := & \prefix{x}{y}{(\binpar{\outputp{x}{y}}{@{y}})} \nonumber\\
	\bangp_{x}{P} & := & \binpar{{x}!\langle{\binpar{D_{x}}{P}}\rangle}{D_{x}} \nonumber
\end{eqnarray}

\begin{eqnarray}
	\bangp_{x}{P} & & \nonumber\\
	=
	& {x}!\langle{(\prefix{x}{y}{(\outputp{x}{y} | @{y})) | P}}\rangle 
	      | \prefix{x}{y}{(\outputp{x}{y} | @{y})} & \nonumber\\
	\red
	& (\outputp{x}{y} | @{y})\substn{\quotep{(\prefix{x}{y}{(@{y} | \outputp{x}{y})) | P}}}{y} & \nonumber\\
	=
	& \outputp{x}{\quotep{(\prefix{x}{y}{(\outputp{x}{y} | @{y})) | P}}}
	  | {(\prefix{x}{y}{(\outputp{x}{y} | @{y})) | P}} & \nonumber\\
	\red
	& \ldots & \nonumber\\
	\red^*
	& P | P | \ldots & \nonumber
\end{eqnarray}

Of course, this encoding, as an implementation, runs away, unfolding
$\bangp{P}$ eagerly. A lazier and more implementable replication
operator, restricted to input-guarded processes, may be obtained as follows.

\begin{eqnarray}
\bangp{\prefix{u}{v}{P}} 
	:= 
	\binpar{\lift{x}{\prefix{u}{v}{(\binpar{D(x)}{P})}}}{D(x)} \nonumber
\end{eqnarray}

\begin{remark}
  Note that the lazier definition still does not deal with summation
  or mixed summation (i.e. sums over input and output). The reader is
  invited to construct definitions of replication that deal with these
  features. 

  Further, the definitions are parameterized in a name, $x$. Can you,
  gentle reader, make a definition that eliminates this parameter and
  guarantees no accidental interaction between the replication
  machinery and the process being replicated -- i.e. no accidental
  sharing of names used by the process to get its work done and the
  name(s) used by the replication to effect copying. This latter
  revision of the definition of replication is crucial to obtaining
  the expected identity $!!P \sim !P$.
\end{remark}

\begin{remark}\label{rem:paradoxical_combinator}
  The reader familiar with the lambda calculus will have noticed the
  similarity between $D$ and the paradoxical combinator.

  [Ed. note: the existence of this seems to suggest we have to be more
  restrictive on the set of processes and names we admit if we are to
  support no-cloning.]
\end{remark}

\subsubsection{Bisimulation}

The computational dynamics gives rise to another kind of equivalence,
the equivalence of computational behavior. As previously mentioned
this is typically captured \emph{via} some form of bisimulation.

% The notion we use in this paper is weak barbed bisimulation
% \cite{milner91polyadicpi}.

The notion we use in this paper is derived from weak barbed
bisimulation \cite{milner91polyadicpi}. 

\begin{definition}
An \emph{observation relation}, $\downarrow_{\mathcal N}$, over a set
of names, $\mathcal N$, is the smallest relation satisfying the rules
below.

\infrule[Out-barb]{y \in {\mathcal N}, \; x \nameeq y}
		  {\outputp{x}{v} \downarrow_{\mathcal N} x}
\infrule[Par-barb]{\mbox{$P\downarrow_{\mathcal N} x$ or $Q\downarrow_{\mathcal N} x$}}
		  {\binpar{P}{Q} \downarrow_{\mathcal N} x}

We write $P \Downarrow_{\mathcal N} x$ if there is $Q$ such that 
$P \wred Q$ and $Q \downarrow_{\mathcal N} x$.
\end{definition}

\begin{definition}
%\label{def.bbisim}
An  ${\mathcal N}$-\emph{barbed bisimulation} over a set of names, ${\mathcal N}$, is a symmetric binary relation 
${\mathcal S}_{\mathcal N}$ between agents such that $P\rel{S}_{\mathcal N}Q$ implies:
\begin{enumerate}
\item If $P \red P'$ then $Q \wred Q'$ and $P'\rel{S}_{\mathcal N} Q'$.
\item If $P\downarrow_{\mathcal N} x$, then $Q\Downarrow_{\mathcal N} x$.
\end{enumerate}
$P$ is ${\mathcal N}$-barbed bisimilar to $Q$, written
$P \wbbisim_{\mathcal N} Q$, if $P \rel{S}_{\mathcal N} Q$ for some ${\mathcal N}$-barbed bisimulation ${\mathcal S}_{\mathcal N}$.
\end{definition}

$\mathcal{R} \subseteq \pi \times \pi$

$P \mathcal{R} Q => \forall P'. P \red P' \Rightarrow \exists Q'. Q \red Q', P' \mathcal{R} Q'$

$P \vdash x \Rightarrow Q \vdash x$

\begin{mathpar}
  \inferrule*[lab=Out-barb]{x \nameeq y}{{y}!\langle{Q}\rangle \vdash x}
  \and
  \inferrule*[lab=Par-barb]{\mbox{$P\vdash x$ or $Q\vdash x$}}{\binpar{P}{Q} \vdash x}
\end{mathpar}

\subsubsection{Contexts}

One of the principle advantages of computational calculi like the
$\pi$-calculus is a well-defined notion of context,
contextual-equivalence and a correlation between
contextual-equivalence and notions of bisimulation. The notion of
context allows the decomposition of a process into (sub-)process and
its syntactic environment, its context. Thus, a context may be
thought of as a process with a ``hole'' (written $\Box$) in it. The
application of a context $M$ to a process $P$, written $M[P]$, is
tantamount to filling the hole in $M$ with $P$. In this paper we do
not need the full weight of this theory, but do make use of the notion
of context in the proof the main theorem. 

\begin{mathpar}
  \inferrule* [lab=summation] {} {{M_{M},M_{N}} \bc \Box \;|\; x.M_{A} \;|\; M_{M}+M_{N}}
  \and
  \inferrule* [lab=agent] {} {{M_{A}} \bc (\vec{x})M_{P} \;| \; \clift{P_0,\ldots,M_{P},\ldots,P_N}}
  \and \\
  \inferrule* [lab=process] {} {{M_{P}} \bc M_{N} \;| \;P|M_{P} }
\end{mathpar} 

\begin{mathpar}
  \inferrule* [lab=sychronization] {} {M_{N} \bc \Box \;|\; x?M_{F} \;|\; x!M_{C}}
  \and
  \inferrule* [lab=abstraction] {} {{M_{F}} \bc (x)M_{P} }
  \and
  \inferrule* [lab=concretion] {} {{M_{C}} \bc \langle M_{P} \rangle }
  \and \\
  \inferrule* [lab=process] {} {{M_{P}} \bc M_{N} \;| \;P|M_{P} }
\end{mathpar}

\begin{definition}[contextual application] Given a context $M$, and
  process $P$, we define the \emph{contextual application}, $M[P] :=
  M\{P/\Box\}$. That is, the contextual application of M to P is the
  substitution of $P$ for $\Box$ in $M$.
\end{definition}

$\meaningof{-} : L \to \mathcal{P}(\pi)$

\begin{mathpar}
  \inferrule* [lab=collection] {} {\meaningof{true} = \pi, \and \meaningof{~E} = \pi \setminus \meaningof{E}, \and \meaningof{E_{1} \& E_{2}} = \meaningof{E_{1}} \cap \meaningof{E_{2}}}
\end{mathpar}

\begin{mathpar}
  \inferrule* [lab=structure] {} {\meaningof{0} = \{ P \in \pi | P \equiv 0 \}, \and \\ \meaningof{E_1 | E_2} = \{ P \in \pi | P \equiv P_{1} | P_{2}, P_{1} \in \meaningof{E_{1}}, P_{2} \in \meaningof{E_2}\} }
\end{mathpar}

\begin{mathpar}
 \inferrule* [lab=behavior] {} {\meaningof{\langle a?b \rangle E} = \{ P \in \pi | P \equiv Q | u?(y)P', \\ \and \\\\ \and \\ \;\;\; u \in \meaningof{a}, \forall z.P'\{z/y\} \in \meaningof{E\{z/b\}}\}, \and \\ \meaningof{a!E} = \{ P \in \pi | P \equiv Q | x!\langle P' \rangle, x \in \meaningof{a} P' \in \meaningof{E}\} }
\end{mathpar}

\begin{mathpar}
 \inferrule* [lab=nominal] {} {\meaningof{\quotep{E}} = \{ \quotep{P} \in \quotep{\pi} | P \in \meaningof{E} \}, \and \meaningof{\quotep{P}} = \{ \quotep{Q} \in \quotep{\pi} | P \equiv Q \} \and \\ \meaningof{@\quotep{E}} = \{ P \in \pi | P \equiv @x, x \in \meaningof{E} \}}
\end{mathpar}

\begin{eqnarray*}
  \\
  \meaningof{-} : TS \to ST
\end{eqnarray*}

\begin{eqnarray*}
  \\
  L : TS \to ST
\end{eqnarray*}

\begin{eqnarray*}
  \\
  P \models E \iff P \in \meaningof{E}
\end{eqnarray*}

\begin{eqnarray*}
  P \approx_{L} Q \iff \forall E \in L. P \models E \iff Q \models E
\end{eqnarray*}

\begin{eqnarray*}
  P \approx_{K} Q
\end{eqnarray*}

\begin{eqnarray*}
  P \approx Q
\end{eqnarray*}

$\approx_{K} = \approx = \approx_{L}$

\subsubsection{Contextual duality}

Note that contexts extend the quotation operation to a family of
operations from processes to names. Given a context, $M$, we can
define a \emph{nominal context}, $\quotep{M}$ by $\quotep{M}[P] :=
\quotep{M[P]}$. To foreshadow what is to come we observe that these
operations enjoy a duality with processes very much like the duality
between vectors and maps from vectors to scalars.

Further, because the calculus is essentially higher-order, we have a
correspondence between contexts and processes. More specifically,
given a name $x$ and a context $M$ we can construct $M^{*}_{x}$ such
that 

\begin{mathpar}
  M^{*}_{x} | \lift{x}{P} \red M[P]
\end{mathpar}

namely,

\begin{mathpar}
  M^{*}_{x} := x?(u).M[\dropn{u}]
\end{mathpar}

The dependence of $M^{*}_{x}$ on a name makes it an abstraction, 

\begin{mathpar}
  M^{*} := (x)x?(u).M[\dropn{u}]
\end{mathpar}

\subsection{Additional notation}

It will sometimes be convenient to denote the process a name
quotes. We already have the notation $x = \quotep{P}$, but it will be
convenient to introduce an alternate notation, $\procn{x}$, when we
want to emphasize the connection to the use of the name. Note that, by
virtue of name equivalence, $\quotep{\procn{x}} \nameeq x$; so, the
notation is consistent with previous definitions.

Further, because names have structure it is possible to effect
substitutions on the basis of that structure. This means we need to
upgrade our notation for substitutions, which we accomplish by
adapting comprehension notation. Thus,

\begin{mathpar}
  P\{ y / x : x \in S \}
\end{mathpar}

is interpreted to mean the process derived from P by replacing (in a
capture-avoiding manner) each occurrence of $x$ in $S$ by $y$. For example,

\begin{mathpar}
  P\{ \quotep{\procn{x}|\procn{x}} / x : x \in \freenames{P} \}
\end{mathpar}

will replace each (occurrence) of a free name $x$ in $P$ by
$\quotep{\procn{x}|\procn{x}}$.

Also, we will avail ourselves of the notation $x^{L}$ and $x^{R}$ to
denote injections of a name into disjoint copies of the name
space. There are numerous ways to accomplish this. One example can be
found in \cite{MeredithR05}. This notation overloads to vectors of
names: $\vec{x}^{\pi} := (x_{i}^{\pi} \; : \; 0 \leq i < |\vec{x}| )$ where $\pi \in \{L,R\}$.

We also use $P^{\Box} := P|\Box$.

In \cite{MeredithR05} an interpretation of the new operator is
given. It turns out that there are several possible interpretations
all enjoying the requisite algebraic properties of the operator (see
\cite{milner91polyadicpi}). We will therefore make liberal use of
$(\nu\; \vec{x})P$.

% subsection the_syntax_and_semantics_of_the_notation_system (end)   

\input{qm2pi.qmops} 

\input{qm2pi.sterngerlach} 

\input{qm2pi.metric} 

% section concurrent_process_calculi (end)

%\input{qm2pi.proofsketch}

% section proof sketch (end)

%\input{qm2pi.slviaknots} 

% section spatial logic via knots (end)

\input{qm2pi.conclusion}

% section conclusion (end)

%\input{qm2pi.dtcodes} 

% section wiring algorithm (end)

\input{qm2pi.ack} 

% section acknowledgments (end)

\newpage


\bibliographystyle{plain}   
\bibliography{../../biblios/main.bib}

\input{qm2pi.rhodetails}

\end{document}

 

%\documentclass[12pt]{llncs}
%\documentclass{jktr}

\usepackage[pdftex]{hyperref}                   
\usepackage {listings}
\usepackage {mathpartir}
\usepackage{bcprules}
%\usepackage{listings}
                       
\usepackage{graphicx} 
%\usepackage[margins=2.5cm,nohead,nofoot]{geometry}
%\usepackage{geometry}
\usepackage{amsfonts}
\usepackage{amstext}
\usepackage{latexsym}
\usepackage{amssymb}
\usepackage{color}


%\include{myPreamble}
\include{qm2pi.local} 

%\ifpdf
%\usepackage[pdftex]{graphicx}
%\else
%\usepackage{graphicx}
%\fi

 % \ifpdf
%  \usepackage{pdfsync}
%  \if


%\title{Brief Article}
%\author{David F. Snyder}
%\author{L.G. Meredith}

%\address{Dept. of Math., Texas State University--San Marcos, San Marcos, TX 78666}
       
\pagestyle{empty}


\begin{document}

\lstset{language=[Objective]Caml,frame=shadowbox}

\input{qm2pi.front}

% section front matter (end)

\input{qm2pi.intro} 
 
% section introduction (end)

% \input{qm2pi.knotations} 

% section notation (end)

\input{qm2pi.process.calculi} 

% section concurrent_process_calculi_and_spatial_logics_ (end)
    
%\input{qm2pi.knots2pi} 

%\input{qm2pi.trefoil} 

%\input{qm2pi.mainthm} 

% subsection basic_interpretation (end)

%\input{qm2pi.rho.presentation} 
\subsection{The syntax and semantics of the notation system}\label{sub:the_syntax_and_semantics_of_the_notation_system} % (fold)

We now summarize a technical presentation of the calculus that
embodies our theory of dynamics. The typical presentation of such a
calculus follows the style of giving generators and relations on
them. The grammar, below, describing term constructors, freely
generates the set of processes, $\Proc$. This set is then quotiented
by a relation known as structural congruence and it is over this set
that the notion of dynamics is expressed. This presentation is
essentially that of \cite{MeredithR05} with the addition of
polyadicity and summation. For readability we have relegated some of
the technical subtleties to an appendix.

\subsubsection{Process grammar}\label{subsub:process_grammar}

\begin{mathpar}
  \inferrule* [lab=synchronization] {} {{M} \bc \pzero \;|\; x?F \;|\; x!C }
  \and
  \inferrule* [lab=abstraction] {} {{F} \bc (x)P}
  \and
  \inferrule* [lab=concretion] {} {{C} \bc \langle Q \rangle}
  \and
  \inferrule* [lab=process] {} {{P,Q} \bc M \;| \;P|Q \;|\; @{x}}
  \and
  \inferrule* [lab=name] {} {{x} \bc \quotep{P}}
\end{mathpar} 

Note that $\vec{x}$ (resp. $\vec{P}$) denotes a vector of names
(resp. processes) of length $|\vec{x}|$ (resp. $|\vec{P}|$). We adopt
the following useful abbreviations.

\begin{mathpar}
   x?(\vec{y}).P := x.(\vec{y})P \and  x\clift{\vec{P}} := x.\clift{\vec{P}}
   \and x!(y) := \lift{x}{\dropn{y}}
   \and \Pi_{i=0}^{n-1}P_i := P_0 | \ldots | P_{n-1}
\end{mathpar}

\subsubsection{Structural congruence}

\paragraph{Free and bound names and alpha-equivalence.} At the
core of structural equivalence is alpha-equivalence which identifies
process that are the same up to a change of variable. Formally, we
recognize the distinction between free and bound names. The free names
of a process, $\freenames{P}$, may be calculated recursively as
follows:

\begin{mathpar}
\freenames{\pzero} := \emptyset
  \and \\
  \freenames{x?(y).P} := \{ x \} \cup (\freenames{P} \setminus \{ y \})
  \and 
  \freenames{x!\langle P \rangle} := \{ x \} \cup \{ P \} 
  \and \\
  \freenames{P|Q} := \freenames{P} \cup \freenames{Q}
  \and \\
  \freenames{@{x}} := \{ x \}
\end{mathpar}

$\pi$
$\quotep{\pi}$

$\freenames{-} : \pi \to \mathcal{P}(\quotep{\pi})$

\begin{eqnarray*}
  \freenames{\pzero} & := & \emptyset \\
  \freenames{x?(y).P} & := & \{ x \} \cup (\freenames{P} \setminus \{ y \}) \\
  \freenames{x!\langle P \rangle} & := & \{ x \} \cup \{ P \} \\
  \freenames{P|Q} & := & \freenames{P} \cup \freenames{Q} \\
  \freenames{\dropn{x}} & := & \{ x \}
\end{eqnarray*}

The bound names of a process, $\boundnames{P}$, are those names occurring in $P$
that are not free. For example, in $x?(y).0$, the name $x$ is free, while $y$ is bound.

\begin{mathpar}
  \inferrule* [lab=monoidal-laws] {} { P|Q \equiv Q|P \and P|0 \equiv P \and P|(Q|R) \equiv (P|Q)|R }
\end{mathpar}

\begin{mathpar}
  \inferrule* [lab=alpha-equivalence] {} { (x)P \equiv (y)P\{y/x\} \and y \not\in \freenames{P} }
\end{mathpar}

\begin{definition}
Then two processes, $P,Q$, are alpha-equivalent if $P = Q\{\vec{y}/\vec{x}\}$ for
some $\vec{x} \in \boundnames{Q},\vec{y} \in \boundnames{P}$, where $Q\{\vec{y}/\vec{x}\}$
denotes the capture-avoiding substitution of $\vec{y}$ for $\vec{x}$ in $Q$.
\end{definition}

\begin{definition}
  The {\em structural congruence} \cite{SangiorgiWalker} , $\equiv$,
  between processes is the least congruence containing
  alpha-equivalence, satisfying the abelian monoid laws
  (associativity, commutativity and $\pzero$ as identity) for parallel
  composition $|$ and for summation $+$.
\end{definition}

\subsection{Name equivalence}

We take name equivalence, written $\nameeq$, to be the smallest
equivalence relation generated by the following rules.

\begin{mathpar}
\inferrule*[lab=Quote-drop]
{ }
{ \quotep{@{x}} \nameeq x }

\inferrule*[lab=Struct-equiv]
{ P \scong Q }
{ \quotep{P} \nameeq \quotep{Q} }
\end{mathpar}

The astute reader will have noticed that the mutual recursion of names
and processes imposes a mutual recursion on alpha-equivalence and
structural equivalence via name-equivalence. Fortunately, all of this
works out pleasantly and we may calculate in the natural way, free of
concern. The reader interested in the details is referred to the
appendix \ref{appendix:rho_details}.

\subsection{Substitution}

We use $\Proc$ for the set of processes, $\QProc$ for the set of
names, and $\id{\{}\vec{y} / \vec{x} \id{\}}$ to denote partial maps,
$s : \QProc \rightarrow \QProc$. A map, $s$ lifts, uniquely, to a map
on process terms, $\widehat{s} : \Proc \rightarrow \Proc$ by the
following equations.

\begin{mathpar}
  (0) \psubstp{Q}{P} := 0 \\
  (R \juxtap S) \psubstp{Q}{P}
  :=    
  (R)\psubstp{Q}{P} \juxtap (S) \psubstp{Q}{P} \\
  (x?(y).R) \psubstp{Q}{P}    
  :=    
  (x)\substp{Q}{P} (z)\concat( (R \psubstn{z}{y}) \psubstp{Q}{P} ) \\
  (\lift{x}{R}) \psubstp{Q}{P}  
  :=
  \lift{(x)\substp{Q}{P}}{ R \psubstp{Q}{P} } \\
%   (\dropn{x})  \psubstp{Q}{P}       
%   := 
%   \left\{ 
%     \begin{array}{ccc} 
%       \dropn{\quotep{Q}} & & x \nameeq \quotep{P} \\
%       \dropn{x} & & otherwise \\
%     \end{array}
%   \right. 
  (\dropn{x})  \psubstp{Q}{P}       
  := 
  \left\{ 
    \begin{array}{ccc} 
      Q & & x \nameeq \quotep{P} \\
      \dropn{x} & & otherwise \\
    \end{array}
  \right.
\end{mathpar}
 

where

\begin{eqnarray}
  (x)\id{\{} \lpquote Q \rpquote / \lpquote P \rpquote \id{\}}            = 
  \left\{ 
    \begin{array}{ccc}
      \lpquote Q \rpquote & & x \nameeq \lpquote P \rpquote \\
      x & & otherwise \\
    \end{array}
  \right. \nonumber
\end{eqnarray}

and $z$ is chosen distinct from $\quotep{P}$, $\quotep{Q}$, the free
names in $Q$, and all the names in $R$. Our $\alpha$-equivalence will
be built in the standard way from this substitution.

\begin{remark}\label{rem:no_self_referential_names}
  One consequence of these definitions is that $\forall P. \quotep{P}
  \not\in \freenames{P}$.
\end{remark}

\subsection{ Dynamic quote: an example }

Anticipating something of what's to come, consider applying the
substitution, $\widehat{\id{\{}u / z \id{\}}}$, to the following pair
of processes, $\lift{w}{y!(z)}$ and $w[ \lpquote y!(z) \rpquote ]$.

\begin{eqnarray}
	\lift{w}{y!(z)}\widehat{\id{\{}u / z \id{\}}}
		& = &
		\lift{w}{y!(u)} \nonumber\\
	w[ \lpquote y!(z) \rpquote ] \widehat{ \id{\{}u / z \id{\}} }
		& = &
		w[ \lpquote y!(z) \rpquote ] \nonumber
\end{eqnarray}

Because the body of the process between quotes is impervious to
substitution, we get radically different answers. In fact, by
examining the first process in an input context,
e.g. $x?(z).\lift{w}{y!(z)}$, we see that the process under the lift
operator may be shaped by prefixed inputs binding a name inside it. In
this sense, the lift operator will be seen as a way to dynamically
construct processes before reifying them as names.

Finally equipped with these standard features we can present the
dynamics of the calculus.

\subsubsection{Operational semantics} 

Finally, we introduce the computational dynamics. What marks these
algebras as distinct from other more traditionally studied algebraic
structures, e.g. vector spaces or polynomial rings, is the manner in
which dynamics is captured. In traditional structures, dynamics is typically
expressed through morphisms between such structures, as in linear maps
between vector spaces or morphisms between rings. In algebras
associated with the semantics of computation, the dynamics is
expressed as part of the algebraic structure itself, through a
reduction reduction relation typically denoted by $\red$. Below, we
give a recursive presentation of this relation for the calculus used
in the encoding.

$\red \subseteq \pi \times \pi$
$\red : \pi \to \mathcal{P}(\pi)$

\begin{mathpar}
  \inferrule* [lab=Comm] { \textsf{match}( x_{src}, x_{trgt} ) } { x_{trgt}?(y)P \; | \; x_{src}!\langle {Q} \rangle \red P\{\quotep{Q}/y}\} }
  \and \\
  \inferrule* [lab=Par] {{P} \red {P}'} {{{P} | {Q}} \red {{P}' | {Q}}}
  \and
  \inferrule* [lab=Equiv]{{{P} \scong {P}'} \andalso {{P}' \red {Q}'} \andalso {{Q}' \scong {Q}}}{{P} \red {Q}}
\end{mathpar}

\begin{eqnarray*}
  match_{\equiv} (\quotep{P},\quotep{Q}) & := & P \equiv Q \\
  match_{\dagger}(\quotep{P},\quotep{Q}) & := & \forall R. P|Q \red^{*} R => R \red^{*} 0 \\
  match_{K}(\quotep{P},\quotep{Q}) & := & K \mbox{ for some context } K
\end{eqnarray*}

$u?(x)P | u!\langle Q \rangle \red P\{\quotep{Q}/x\}$

%We write $\wred$ for $\red^*$, and $P\red$ if $\exists Q $ such that $ P \red Q$.
We write $P\red$ if $\exists Q $ such that $ P \red Q$ and $P\not\red$, otherwise.

\section{Replication}

As mentioned before, it is known that replication (and hence
recursion) can be implemented in a higher-order process algebra
\cite{SangiorgiWalker}. As our first example of calculation with the
machinery thus far presented we give the construction explicitly in
the {\rhoc}.

\begin{eqnarray}
	D_{x} & := & \prefix{x}{y}{(\binpar{\outputp{x}{y}}{@{y}})} \nonumber\\
	\bangp_{x}{P} & := & \binpar{{x}!\langle{\binpar{D_{x}}{P}}\rangle}{D_{x}} \nonumber
\end{eqnarray}

\begin{eqnarray}
	\bangp_{x}{P} & & \nonumber\\
	=
	& {x}!\langle{(\prefix{x}{y}{(\outputp{x}{y} | @{y})) | P}}\rangle 
	      | \prefix{x}{y}{(\outputp{x}{y} | @{y})} & \nonumber\\
	\red
	& (\outputp{x}{y} | @{y})\substn{\quotep{(\prefix{x}{y}{(@{y} | \outputp{x}{y})) | P}}}{y} & \nonumber\\
	=
	& \outputp{x}{\quotep{(\prefix{x}{y}{(\outputp{x}{y} | @{y})) | P}}}
	  | {(\prefix{x}{y}{(\outputp{x}{y} | @{y})) | P}} & \nonumber\\
	\red
	& \ldots & \nonumber\\
	\red^*
	& P | P | \ldots & \nonumber
\end{eqnarray}

Of course, this encoding, as an implementation, runs away, unfolding
$\bangp{P}$ eagerly. A lazier and more implementable replication
operator, restricted to input-guarded processes, may be obtained as follows.

\begin{eqnarray}
\bangp{\prefix{u}{v}{P}} 
	:= 
	\binpar{\lift{x}{\prefix{u}{v}{(\binpar{D(x)}{P})}}}{D(x)} \nonumber
\end{eqnarray}

\begin{remark}
  Note that the lazier definition still does not deal with summation
  or mixed summation (i.e. sums over input and output). The reader is
  invited to construct definitions of replication that deal with these
  features. 

  Further, the definitions are parameterized in a name, $x$. Can you,
  gentle reader, make a definition that eliminates this parameter and
  guarantees no accidental interaction between the replication
  machinery and the process being replicated -- i.e. no accidental
  sharing of names used by the process to get its work done and the
  name(s) used by the replication to effect copying. This latter
  revision of the definition of replication is crucial to obtaining
  the expected identity $!!P \sim !P$.
\end{remark}

\begin{remark}\label{rem:paradoxical_combinator}
  The reader familiar with the lambda calculus will have noticed the
  similarity between $D$ and the paradoxical combinator.

  [Ed. note: the existence of this seems to suggest we have to be more
  restrictive on the set of processes and names we admit if we are to
  support no-cloning.]
\end{remark}

\subsubsection{Bisimulation}

The computational dynamics gives rise to another kind of equivalence,
the equivalence of computational behavior. As previously mentioned
this is typically captured \emph{via} some form of bisimulation.

% The notion we use in this paper is weak barbed bisimulation
% \cite{milner91polyadicpi}.

The notion we use in this paper is derived from weak barbed
bisimulation \cite{milner91polyadicpi}. 

\begin{definition}
An \emph{observation relation}, $\downarrow_{\mathcal N}$, over a set
of names, $\mathcal N$, is the smallest relation satisfying the rules
below.

\infrule[Out-barb]{y \in {\mathcal N}, \; x \nameeq y}
		  {\outputp{x}{v} \downarrow_{\mathcal N} x}
\infrule[Par-barb]{\mbox{$P\downarrow_{\mathcal N} x$ or $Q\downarrow_{\mathcal N} x$}}
		  {\binpar{P}{Q} \downarrow_{\mathcal N} x}

We write $P \Downarrow_{\mathcal N} x$ if there is $Q$ such that 
$P \wred Q$ and $Q \downarrow_{\mathcal N} x$.
\end{definition}

\begin{definition}
%\label{def.bbisim}
An  ${\mathcal N}$-\emph{barbed bisimulation} over a set of names, ${\mathcal N}$, is a symmetric binary relation 
${\mathcal S}_{\mathcal N}$ between agents such that $P\rel{S}_{\mathcal N}Q$ implies:
\begin{enumerate}
\item If $P \red P'$ then $Q \wred Q'$ and $P'\rel{S}_{\mathcal N} Q'$.
\item If $P\downarrow_{\mathcal N} x$, then $Q\Downarrow_{\mathcal N} x$.
\end{enumerate}
$P$ is ${\mathcal N}$-barbed bisimilar to $Q$, written
$P \wbbisim_{\mathcal N} Q$, if $P \rel{S}_{\mathcal N} Q$ for some ${\mathcal N}$-barbed bisimulation ${\mathcal S}_{\mathcal N}$.
\end{definition}

$\mathcal{R} \subseteq \pi \times \pi$

$P \mathcal{R} Q => \forall P'. P \red P' \Rightarrow \exists Q'. Q \red Q', P' \mathcal{R} Q'$

$P \vdash x \Rightarrow Q \vdash x$

\begin{mathpar}
  \inferrule*[lab=Out-barb]{x \nameeq y}{{y}!\langle{Q}\rangle \vdash x}
  \and
  \inferrule*[lab=Par-barb]{\mbox{$P\vdash x$ or $Q\vdash x$}}{\binpar{P}{Q} \vdash x}
\end{mathpar}

\subsubsection{Contexts}

One of the principle advantages of computational calculi like the
$\pi$-calculus is a well-defined notion of context,
contextual-equivalence and a correlation between
contextual-equivalence and notions of bisimulation. The notion of
context allows the decomposition of a process into (sub-)process and
its syntactic environment, its context. Thus, a context may be
thought of as a process with a ``hole'' (written $\Box$) in it. The
application of a context $M$ to a process $P$, written $M[P]$, is
tantamount to filling the hole in $M$ with $P$. In this paper we do
not need the full weight of this theory, but do make use of the notion
of context in the proof the main theorem. 

\begin{mathpar}
  \inferrule* [lab=summation] {} {{M_{M},M_{N}} \bc \Box \;|\; x.M_{A} \;|\; M_{M}+M_{N}}
  \and
  \inferrule* [lab=agent] {} {{M_{A}} \bc (\vec{x})M_{P} \;| \; \clift{P_0,\ldots,M_{P},\ldots,P_N}}
  \and \\
  \inferrule* [lab=process] {} {{M_{P}} \bc M_{N} \;| \;P|M_{P} }
\end{mathpar} 

\begin{mathpar}
  \inferrule* [lab=sychronization] {} {M_{N} \bc \Box \;|\; x?M_{F} \;|\; x!M_{C}}
  \and
  \inferrule* [lab=abstraction] {} {{M_{F}} \bc (x)M_{P} }
  \and
  \inferrule* [lab=concretion] {} {{M_{C}} \bc \langle M_{P} \rangle }
  \and \\
  \inferrule* [lab=process] {} {{M_{P}} \bc M_{N} \;| \;P|M_{P} }
\end{mathpar}

\begin{definition}[contextual application] Given a context $M$, and
  process $P$, we define the \emph{contextual application}, $M[P] :=
  M\{P/\Box\}$. That is, the contextual application of M to P is the
  substitution of $P$ for $\Box$ in $M$.
\end{definition}

$\meaningof{-} : L \to \mathcal{P}(\pi)$

\begin{mathpar}
  \inferrule* [lab=collection] {} {\meaningof{true} = \pi, \and \meaningof{~E} = \pi \setminus \meaningof{E}, \and \meaningof{E_{1} \& E_{2}} = \meaningof{E_{1}} \cap \meaningof{E_{2}}}
\end{mathpar}

\begin{mathpar}
  \inferrule* [lab=structure] {} {\meaningof{0} = \{ P \in \pi | P \equiv 0 \}, \and \\ \meaningof{E_1 | E_2} = \{ P \in \pi | P \equiv P_{1} | P_{2}, P_{1} \in \meaningof{E_{1}}, P_{2} \in \meaningof{E_2}\} }
\end{mathpar}

\begin{mathpar}
 \inferrule* [lab=behavior] {} {\meaningof{\langle a?b \rangle E} = \{ P \in \pi | P \equiv Q | u?(y)P', \\ \and \\\\ \and \\ \;\;\; u \in \meaningof{a}, \forall z.P'\{z/y\} \in \meaningof{E\{z/b\}}\}, \and \\ \meaningof{a!E} = \{ P \in \pi | P \equiv Q | x!\langle P' \rangle, x \in \meaningof{a} P' \in \meaningof{E}\} }
\end{mathpar}

\begin{mathpar}
 \inferrule* [lab=nominal] {} {\meaningof{\quotep{E}} = \{ \quotep{P} \in \quotep{\pi} | P \in \meaningof{E} \}, \and \meaningof{\quotep{P}} = \{ \quotep{Q} \in \quotep{\pi} | P \equiv Q \} \and \\ \meaningof{@\quotep{E}} = \{ P \in \pi | P \equiv @x, x \in \meaningof{E} \}}
\end{mathpar}

\begin{eqnarray*}
  \\
  \meaningof{-} : TS \to ST
\end{eqnarray*}

\begin{eqnarray*}
  \\
  L : TS \to ST
\end{eqnarray*}

\begin{eqnarray*}
  \\
  P \models E \iff P \in \meaningof{E}
\end{eqnarray*}

\begin{eqnarray*}
  P \approx_{L} Q \iff \forall E \in L. P \models E \iff Q \models E
\end{eqnarray*}

\begin{eqnarray*}
  P \approx_{K} Q
\end{eqnarray*}

\begin{eqnarray*}
  P \approx Q
\end{eqnarray*}

$\approx_{K} = \approx = \approx_{L}$

\subsubsection{Contextual duality}

Note that contexts extend the quotation operation to a family of
operations from processes to names. Given a context, $M$, we can
define a \emph{nominal context}, $\quotep{M}$ by $\quotep{M}[P] :=
\quotep{M[P]}$. To foreshadow what is to come we observe that these
operations enjoy a duality with processes very much like the duality
between vectors and maps from vectors to scalars.

Further, because the calculus is essentially higher-order, we have a
correspondence between contexts and processes. More specifically,
given a name $x$ and a context $M$ we can construct $M^{*}_{x}$ such
that 

\begin{mathpar}
  M^{*}_{x} | \lift{x}{P} \red M[P]
\end{mathpar}

namely,

\begin{mathpar}
  M^{*}_{x} := x?(u).M[\dropn{u}]
\end{mathpar}

The dependence of $M^{*}_{x}$ on a name makes it an abstraction, 

\begin{mathpar}
  M^{*} := (x)x?(u).M[\dropn{u}]
\end{mathpar}

\subsection{Additional notation}

It will sometimes be convenient to denote the process a name
quotes. We already have the notation $x = \quotep{P}$, but it will be
convenient to introduce an alternate notation, $\procn{x}$, when we
want to emphasize the connection to the use of the name. Note that, by
virtue of name equivalence, $\quotep{\procn{x}} \nameeq x$; so, the
notation is consistent with previous definitions.

Further, because names have structure it is possible to effect
substitutions on the basis of that structure. This means we need to
upgrade our notation for substitutions, which we accomplish by
adapting comprehension notation. Thus,

\begin{mathpar}
  P\{ y / x : x \in S \}
\end{mathpar}

is interpreted to mean the process derived from P by replacing (in a
capture-avoiding manner) each occurrence of $x$ in $S$ by $y$. For example,

\begin{mathpar}
  P\{ \quotep{\procn{x}|\procn{x}} / x : x \in \freenames{P} \}
\end{mathpar}

will replace each (occurrence) of a free name $x$ in $P$ by
$\quotep{\procn{x}|\procn{x}}$.

Also, we will avail ourselves of the notation $x^{L}$ and $x^{R}$ to
denote injections of a name into disjoint copies of the name
space. There are numerous ways to accomplish this. One example can be
found in \cite{MeredithR05}. This notation overloads to vectors of
names: $\vec{x}^{\pi} := (x_{i}^{\pi} \; : \; 0 \leq i < |\vec{x}| )$ where $\pi \in \{L,R\}$.

We also use $P^{\Box} := P|\Box$.

In \cite{MeredithR05} an interpretation of the new operator is
given. It turns out that there are several possible interpretations
all enjoying the requisite algebraic properties of the operator (see
\cite{milner91polyadicpi}). We will therefore make liberal use of
$(\nu\; \vec{x})P$.

% subsection the_syntax_and_semantics_of_the_notation_system (end)   

\input{qm2pi.qmops} 

\input{qm2pi.sterngerlach} 

\input{qm2pi.metric} 

% section concurrent_process_calculi (end)

%\input{qm2pi.proofsketch}

% section proof sketch (end)

%\input{qm2pi.slviaknots} 

% section spatial logic via knots (end)

\input{qm2pi.conclusion}

% section conclusion (end)

%\input{qm2pi.dtcodes} 

% section wiring algorithm (end)

\input{qm2pi.ack} 

% section acknowledgments (end)

\newpage


\bibliographystyle{plain}   
\bibliography{../../biblios/main.bib}

\input{qm2pi.rhodetails}

\end{document}

 

% subsection basic_interpretation (end)

%\input{qm2pi.rho.presentation} 
\subsection{The syntax and semantics of the notation system}\label{sub:the_syntax_and_semantics_of_the_notation_system} % (fold)

We now summarize a technical presentation of the calculus that
embodies our theory of dynamics. The typical presentation of such a
calculus follows the style of giving generators and relations on
them. The grammar, below, describing term constructors, freely
generates the set of processes, $\Proc$. This set is then quotiented
by a relation known as structural congruence and it is over this set
that the notion of dynamics is expressed. This presentation is
essentially that of \cite{MeredithR05} with the addition of
polyadicity and summation. For readability we have relegated some of
the technical subtleties to an appendix.

\subsubsection{Process grammar}\label{subsub:process_grammar}

\begin{mathpar}
  \inferrule* [lab=synchronization] {} {{M} \bc \pzero \;|\; x?F \;|\; x!C }
  \and
  \inferrule* [lab=abstraction] {} {{F} \bc (x)P}
  \and
  \inferrule* [lab=concretion] {} {{C} \bc \langle Q \rangle}
  \and
  \inferrule* [lab=process] {} {{P,Q} \bc M \;| \;P|Q \;|\; @{x}}
  \and
  \inferrule* [lab=name] {} {{x} \bc \quotep{P}}
\end{mathpar} 

Note that $\vec{x}$ (resp. $\vec{P}$) denotes a vector of names
(resp. processes) of length $|\vec{x}|$ (resp. $|\vec{P}|$). We adopt
the following useful abbreviations.

\begin{mathpar}
   x?(\vec{y}).P := x.(\vec{y})P \and  x\clift{\vec{P}} := x.\clift{\vec{P}}
   \and x!(y) := \lift{x}{\dropn{y}}
   \and \Pi_{i=0}^{n-1}P_i := P_0 | \ldots | P_{n-1}
\end{mathpar}

\subsubsection{Structural congruence}

\paragraph{Free and bound names and alpha-equivalence.} At the
core of structural equivalence is alpha-equivalence which identifies
process that are the same up to a change of variable. Formally, we
recognize the distinction between free and bound names. The free names
of a process, $\freenames{P}$, may be calculated recursively as
follows:

\begin{mathpar}
\freenames{\pzero} := \emptyset
  \and \\
  \freenames{x?(y).P} := \{ x \} \cup (\freenames{P} \setminus \{ y \})
  \and 
  \freenames{x!\langle P \rangle} := \{ x \} \cup \{ P \} 
  \and \\
  \freenames{P|Q} := \freenames{P} \cup \freenames{Q}
  \and \\
  \freenames{@{x}} := \{ x \}
\end{mathpar}

$\pi$
$\quotep{\pi}$

$\freenames{-} : \pi \to \mathcal{P}(\quotep{\pi})$

\begin{eqnarray*}
  \freenames{\pzero} & := & \emptyset \\
  \freenames{x?(y).P} & := & \{ x \} \cup (\freenames{P} \setminus \{ y \}) \\
  \freenames{x!\langle P \rangle} & := & \{ x \} \cup \{ P \} \\
  \freenames{P|Q} & := & \freenames{P} \cup \freenames{Q} \\
  \freenames{\dropn{x}} & := & \{ x \}
\end{eqnarray*}

The bound names of a process, $\boundnames{P}$, are those names occurring in $P$
that are not free. For example, in $x?(y).0$, the name $x$ is free, while $y$ is bound.

\begin{mathpar}
  \inferrule* [lab=monoidal-laws] {} { P|Q \equiv Q|P \and P|0 \equiv P \and P|(Q|R) \equiv (P|Q)|R }
\end{mathpar}

\begin{mathpar}
  \inferrule* [lab=alpha-equivalence] {} { (x)P \equiv (y)P\{y/x\} \and y \not\in \freenames{P} }
\end{mathpar}

\begin{definition}
Then two processes, $P,Q$, are alpha-equivalent if $P = Q\{\vec{y}/\vec{x}\}$ for
some $\vec{x} \in \boundnames{Q},\vec{y} \in \boundnames{P}$, where $Q\{\vec{y}/\vec{x}\}$
denotes the capture-avoiding substitution of $\vec{y}$ for $\vec{x}$ in $Q$.
\end{definition}

\begin{definition}
  The {\em structural congruence} \cite{SangiorgiWalker} , $\equiv$,
  between processes is the least congruence containing
  alpha-equivalence, satisfying the abelian monoid laws
  (associativity, commutativity and $\pzero$ as identity) for parallel
  composition $|$ and for summation $+$.
\end{definition}

\subsection{Name equivalence}

We take name equivalence, written $\nameeq$, to be the smallest
equivalence relation generated by the following rules.

\begin{mathpar}
\inferrule*[lab=Quote-drop]
{ }
{ \quotep{@{x}} \nameeq x }

\inferrule*[lab=Struct-equiv]
{ P \scong Q }
{ \quotep{P} \nameeq \quotep{Q} }
\end{mathpar}

The astute reader will have noticed that the mutual recursion of names
and processes imposes a mutual recursion on alpha-equivalence and
structural equivalence via name-equivalence. Fortunately, all of this
works out pleasantly and we may calculate in the natural way, free of
concern. The reader interested in the details is referred to the
appendix \ref{appendix:rho_details}.

\subsection{Substitution}

We use $\Proc$ for the set of processes, $\QProc$ for the set of
names, and $\id{\{}\vec{y} / \vec{x} \id{\}}$ to denote partial maps,
$s : \QProc \rightarrow \QProc$. A map, $s$ lifts, uniquely, to a map
on process terms, $\widehat{s} : \Proc \rightarrow \Proc$ by the
following equations.

\begin{mathpar}
  (0) \psubstp{Q}{P} := 0 \\
  (R \juxtap S) \psubstp{Q}{P}
  :=    
  (R)\psubstp{Q}{P} \juxtap (S) \psubstp{Q}{P} \\
  (x?(y).R) \psubstp{Q}{P}    
  :=    
  (x)\substp{Q}{P} (z)\concat( (R \psubstn{z}{y}) \psubstp{Q}{P} ) \\
  (\lift{x}{R}) \psubstp{Q}{P}  
  :=
  \lift{(x)\substp{Q}{P}}{ R \psubstp{Q}{P} } \\
%   (\dropn{x})  \psubstp{Q}{P}       
%   := 
%   \left\{ 
%     \begin{array}{ccc} 
%       \dropn{\quotep{Q}} & & x \nameeq \quotep{P} \\
%       \dropn{x} & & otherwise \\
%     \end{array}
%   \right. 
  (\dropn{x})  \psubstp{Q}{P}       
  := 
  \left\{ 
    \begin{array}{ccc} 
      Q & & x \nameeq \quotep{P} \\
      \dropn{x} & & otherwise \\
    \end{array}
  \right.
\end{mathpar}
 

where

\begin{eqnarray}
  (x)\id{\{} \lpquote Q \rpquote / \lpquote P \rpquote \id{\}}            = 
  \left\{ 
    \begin{array}{ccc}
      \lpquote Q \rpquote & & x \nameeq \lpquote P \rpquote \\
      x & & otherwise \\
    \end{array}
  \right. \nonumber
\end{eqnarray}

and $z$ is chosen distinct from $\quotep{P}$, $\quotep{Q}$, the free
names in $Q$, and all the names in $R$. Our $\alpha$-equivalence will
be built in the standard way from this substitution.

\begin{remark}\label{rem:no_self_referential_names}
  One consequence of these definitions is that $\forall P. \quotep{P}
  \not\in \freenames{P}$.
\end{remark}

\subsection{ Dynamic quote: an example }

Anticipating something of what's to come, consider applying the
substitution, $\widehat{\id{\{}u / z \id{\}}}$, to the following pair
of processes, $\lift{w}{y!(z)}$ and $w[ \lpquote y!(z) \rpquote ]$.

\begin{eqnarray}
	\lift{w}{y!(z)}\widehat{\id{\{}u / z \id{\}}}
		& = &
		\lift{w}{y!(u)} \nonumber\\
	w[ \lpquote y!(z) \rpquote ] \widehat{ \id{\{}u / z \id{\}} }
		& = &
		w[ \lpquote y!(z) \rpquote ] \nonumber
\end{eqnarray}

Because the body of the process between quotes is impervious to
substitution, we get radically different answers. In fact, by
examining the first process in an input context,
e.g. $x?(z).\lift{w}{y!(z)}$, we see that the process under the lift
operator may be shaped by prefixed inputs binding a name inside it. In
this sense, the lift operator will be seen as a way to dynamically
construct processes before reifying them as names.

Finally equipped with these standard features we can present the
dynamics of the calculus.

\subsubsection{Operational semantics} 

Finally, we introduce the computational dynamics. What marks these
algebras as distinct from other more traditionally studied algebraic
structures, e.g. vector spaces or polynomial rings, is the manner in
which dynamics is captured. In traditional structures, dynamics is typically
expressed through morphisms between such structures, as in linear maps
between vector spaces or morphisms between rings. In algebras
associated with the semantics of computation, the dynamics is
expressed as part of the algebraic structure itself, through a
reduction reduction relation typically denoted by $\red$. Below, we
give a recursive presentation of this relation for the calculus used
in the encoding.

$\red \subseteq \pi \times \pi$
$\red : \pi \to \mathcal{P}(\pi)$

\begin{mathpar}
  \inferrule* [lab=Comm] { \textsf{match}( x_{src}, x_{trgt} ) } { x_{trgt}?(y)P \; | \; x_{src}!\langle {Q} \rangle \red P\{\quotep{Q}/y}\} }
  \and \\
  \inferrule* [lab=Par] {{P} \red {P}'} {{{P} | {Q}} \red {{P}' | {Q}}}
  \and
  \inferrule* [lab=Equiv]{{{P} \scong {P}'} \andalso {{P}' \red {Q}'} \andalso {{Q}' \scong {Q}}}{{P} \red {Q}}
\end{mathpar}

\begin{eqnarray*}
  match_{\equiv} (\quotep{P},\quotep{Q}) & := & P \equiv Q \\
  match_{\dagger}(\quotep{P},\quotep{Q}) & := & \forall R. P|Q \red^{*} R => R \red^{*} 0 \\
  match_{K}(\quotep{P},\quotep{Q}) & := & K \mbox{ for some context } K
\end{eqnarray*}

$u?(x)P | u!\langle Q \rangle \red P\{\quotep{Q}/x\}$

%We write $\wred$ for $\red^*$, and $P\red$ if $\exists Q $ such that $ P \red Q$.
We write $P\red$ if $\exists Q $ such that $ P \red Q$ and $P\not\red$, otherwise.

\section{Replication}

As mentioned before, it is known that replication (and hence
recursion) can be implemented in a higher-order process algebra
\cite{SangiorgiWalker}. As our first example of calculation with the
machinery thus far presented we give the construction explicitly in
the {\rhoc}.

\begin{eqnarray}
	D_{x} & := & \prefix{x}{y}{(\binpar{\outputp{x}{y}}{@{y}})} \nonumber\\
	\bangp_{x}{P} & := & \binpar{{x}!\langle{\binpar{D_{x}}{P}}\rangle}{D_{x}} \nonumber
\end{eqnarray}

\begin{eqnarray}
	\bangp_{x}{P} & & \nonumber\\
	=
	& {x}!\langle{(\prefix{x}{y}{(\outputp{x}{y} | @{y})) | P}}\rangle 
	      | \prefix{x}{y}{(\outputp{x}{y} | @{y})} & \nonumber\\
	\red
	& (\outputp{x}{y} | @{y})\substn{\quotep{(\prefix{x}{y}{(@{y} | \outputp{x}{y})) | P}}}{y} & \nonumber\\
	=
	& \outputp{x}{\quotep{(\prefix{x}{y}{(\outputp{x}{y} | @{y})) | P}}}
	  | {(\prefix{x}{y}{(\outputp{x}{y} | @{y})) | P}} & \nonumber\\
	\red
	& \ldots & \nonumber\\
	\red^*
	& P | P | \ldots & \nonumber
\end{eqnarray}

Of course, this encoding, as an implementation, runs away, unfolding
$\bangp{P}$ eagerly. A lazier and more implementable replication
operator, restricted to input-guarded processes, may be obtained as follows.

\begin{eqnarray}
\bangp{\prefix{u}{v}{P}} 
	:= 
	\binpar{\lift{x}{\prefix{u}{v}{(\binpar{D(x)}{P})}}}{D(x)} \nonumber
\end{eqnarray}

\begin{remark}
  Note that the lazier definition still does not deal with summation
  or mixed summation (i.e. sums over input and output). The reader is
  invited to construct definitions of replication that deal with these
  features. 

  Further, the definitions are parameterized in a name, $x$. Can you,
  gentle reader, make a definition that eliminates this parameter and
  guarantees no accidental interaction between the replication
  machinery and the process being replicated -- i.e. no accidental
  sharing of names used by the process to get its work done and the
  name(s) used by the replication to effect copying. This latter
  revision of the definition of replication is crucial to obtaining
  the expected identity $!!P \sim !P$.
\end{remark}

\begin{remark}\label{rem:paradoxical_combinator}
  The reader familiar with the lambda calculus will have noticed the
  similarity between $D$ and the paradoxical combinator.

  [Ed. note: the existence of this seems to suggest we have to be more
  restrictive on the set of processes and names we admit if we are to
  support no-cloning.]
\end{remark}

\subsubsection{Bisimulation}

The computational dynamics gives rise to another kind of equivalence,
the equivalence of computational behavior. As previously mentioned
this is typically captured \emph{via} some form of bisimulation.

% The notion we use in this paper is weak barbed bisimulation
% \cite{milner91polyadicpi}.

The notion we use in this paper is derived from weak barbed
bisimulation \cite{milner91polyadicpi}. 

\begin{definition}
An \emph{observation relation}, $\downarrow_{\mathcal N}$, over a set
of names, $\mathcal N$, is the smallest relation satisfying the rules
below.

\infrule[Out-barb]{y \in {\mathcal N}, \; x \nameeq y}
		  {\outputp{x}{v} \downarrow_{\mathcal N} x}
\infrule[Par-barb]{\mbox{$P\downarrow_{\mathcal N} x$ or $Q\downarrow_{\mathcal N} x$}}
		  {\binpar{P}{Q} \downarrow_{\mathcal N} x}

We write $P \Downarrow_{\mathcal N} x$ if there is $Q$ such that 
$P \wred Q$ and $Q \downarrow_{\mathcal N} x$.
\end{definition}

\begin{definition}
%\label{def.bbisim}
An  ${\mathcal N}$-\emph{barbed bisimulation} over a set of names, ${\mathcal N}$, is a symmetric binary relation 
${\mathcal S}_{\mathcal N}$ between agents such that $P\rel{S}_{\mathcal N}Q$ implies:
\begin{enumerate}
\item If $P \red P'$ then $Q \wred Q'$ and $P'\rel{S}_{\mathcal N} Q'$.
\item If $P\downarrow_{\mathcal N} x$, then $Q\Downarrow_{\mathcal N} x$.
\end{enumerate}
$P$ is ${\mathcal N}$-barbed bisimilar to $Q$, written
$P \wbbisim_{\mathcal N} Q$, if $P \rel{S}_{\mathcal N} Q$ for some ${\mathcal N}$-barbed bisimulation ${\mathcal S}_{\mathcal N}$.
\end{definition}

$\mathcal{R} \subseteq \pi \times \pi$

$P \mathcal{R} Q => \forall P'. P \red P' \Rightarrow \exists Q'. Q \red Q', P' \mathcal{R} Q'$

$P \vdash x \Rightarrow Q \vdash x$

\begin{mathpar}
  \inferrule*[lab=Out-barb]{x \nameeq y}{{y}!\langle{Q}\rangle \vdash x}
  \and
  \inferrule*[lab=Par-barb]{\mbox{$P\vdash x$ or $Q\vdash x$}}{\binpar{P}{Q} \vdash x}
\end{mathpar}

\subsubsection{Contexts}

One of the principle advantages of computational calculi like the
$\pi$-calculus is a well-defined notion of context,
contextual-equivalence and a correlation between
contextual-equivalence and notions of bisimulation. The notion of
context allows the decomposition of a process into (sub-)process and
its syntactic environment, its context. Thus, a context may be
thought of as a process with a ``hole'' (written $\Box$) in it. The
application of a context $M$ to a process $P$, written $M[P]$, is
tantamount to filling the hole in $M$ with $P$. In this paper we do
not need the full weight of this theory, but do make use of the notion
of context in the proof the main theorem. 

\begin{mathpar}
  \inferrule* [lab=summation] {} {{M_{M},M_{N}} \bc \Box \;|\; x.M_{A} \;|\; M_{M}+M_{N}}
  \and
  \inferrule* [lab=agent] {} {{M_{A}} \bc (\vec{x})M_{P} \;| \; \clift{P_0,\ldots,M_{P},\ldots,P_N}}
  \and \\
  \inferrule* [lab=process] {} {{M_{P}} \bc M_{N} \;| \;P|M_{P} }
\end{mathpar} 

\begin{mathpar}
  \inferrule* [lab=sychronization] {} {M_{N} \bc \Box \;|\; x?M_{F} \;|\; x!M_{C}}
  \and
  \inferrule* [lab=abstraction] {} {{M_{F}} \bc (x)M_{P} }
  \and
  \inferrule* [lab=concretion] {} {{M_{C}} \bc \langle M_{P} \rangle }
  \and \\
  \inferrule* [lab=process] {} {{M_{P}} \bc M_{N} \;| \;P|M_{P} }
\end{mathpar}

\begin{definition}[contextual application] Given a context $M$, and
  process $P$, we define the \emph{contextual application}, $M[P] :=
  M\{P/\Box\}$. That is, the contextual application of M to P is the
  substitution of $P$ for $\Box$ in $M$.
\end{definition}

$\meaningof{-} : L \to \mathcal{P}(\pi)$

\begin{mathpar}
  \inferrule* [lab=collection] {} {\meaningof{true} = \pi, \and \meaningof{~E} = \pi \setminus \meaningof{E}, \and \meaningof{E_{1} \& E_{2}} = \meaningof{E_{1}} \cap \meaningof{E_{2}}}
\end{mathpar}

\begin{mathpar}
  \inferrule* [lab=structure] {} {\meaningof{0} = \{ P \in \pi | P \equiv 0 \}, \and \\ \meaningof{E_1 | E_2} = \{ P \in \pi | P \equiv P_{1} | P_{2}, P_{1} \in \meaningof{E_{1}}, P_{2} \in \meaningof{E_2}\} }
\end{mathpar}

\begin{mathpar}
 \inferrule* [lab=behavior] {} {\meaningof{\langle a?b \rangle E} = \{ P \in \pi | P \equiv Q | u?(y)P', \\ \and \\\\ \and \\ \;\;\; u \in \meaningof{a}, \forall z.P'\{z/y\} \in \meaningof{E\{z/b\}}\}, \and \\ \meaningof{a!E} = \{ P \in \pi | P \equiv Q | x!\langle P' \rangle, x \in \meaningof{a} P' \in \meaningof{E}\} }
\end{mathpar}

\begin{mathpar}
 \inferrule* [lab=nominal] {} {\meaningof{\quotep{E}} = \{ \quotep{P} \in \quotep{\pi} | P \in \meaningof{E} \}, \and \meaningof{\quotep{P}} = \{ \quotep{Q} \in \quotep{\pi} | P \equiv Q \} \and \\ \meaningof{@\quotep{E}} = \{ P \in \pi | P \equiv @x, x \in \meaningof{E} \}}
\end{mathpar}

\begin{eqnarray*}
  \\
  \meaningof{-} : TS \to ST
\end{eqnarray*}

\begin{eqnarray*}
  \\
  L : TS \to ST
\end{eqnarray*}

\begin{eqnarray*}
  \\
  P \models E \iff P \in \meaningof{E}
\end{eqnarray*}

\begin{eqnarray*}
  P \approx_{L} Q \iff \forall E \in L. P \models E \iff Q \models E
\end{eqnarray*}

\begin{eqnarray*}
  P \approx_{K} Q
\end{eqnarray*}

\begin{eqnarray*}
  P \approx Q
\end{eqnarray*}

$\approx_{K} = \approx = \approx_{L}$

\subsubsection{Contextual duality}

Note that contexts extend the quotation operation to a family of
operations from processes to names. Given a context, $M$, we can
define a \emph{nominal context}, $\quotep{M}$ by $\quotep{M}[P] :=
\quotep{M[P]}$. To foreshadow what is to come we observe that these
operations enjoy a duality with processes very much like the duality
between vectors and maps from vectors to scalars.

Further, because the calculus is essentially higher-order, we have a
correspondence between contexts and processes. More specifically,
given a name $x$ and a context $M$ we can construct $M^{*}_{x}$ such
that 

\begin{mathpar}
  M^{*}_{x} | \lift{x}{P} \red M[P]
\end{mathpar}

namely,

\begin{mathpar}
  M^{*}_{x} := x?(u).M[\dropn{u}]
\end{mathpar}

The dependence of $M^{*}_{x}$ on a name makes it an abstraction, 

\begin{mathpar}
  M^{*} := (x)x?(u).M[\dropn{u}]
\end{mathpar}

\subsection{Additional notation}

It will sometimes be convenient to denote the process a name
quotes. We already have the notation $x = \quotep{P}$, but it will be
convenient to introduce an alternate notation, $\procn{x}$, when we
want to emphasize the connection to the use of the name. Note that, by
virtue of name equivalence, $\quotep{\procn{x}} \nameeq x$; so, the
notation is consistent with previous definitions.

Further, because names have structure it is possible to effect
substitutions on the basis of that structure. This means we need to
upgrade our notation for substitutions, which we accomplish by
adapting comprehension notation. Thus,

\begin{mathpar}
  P\{ y / x : x \in S \}
\end{mathpar}

is interpreted to mean the process derived from P by replacing (in a
capture-avoiding manner) each occurrence of $x$ in $S$ by $y$. For example,

\begin{mathpar}
  P\{ \quotep{\procn{x}|\procn{x}} / x : x \in \freenames{P} \}
\end{mathpar}

will replace each (occurrence) of a free name $x$ in $P$ by
$\quotep{\procn{x}|\procn{x}}$.

Also, we will avail ourselves of the notation $x^{L}$ and $x^{R}$ to
denote injections of a name into disjoint copies of the name
space. There are numerous ways to accomplish this. One example can be
found in \cite{MeredithR05}. This notation overloads to vectors of
names: $\vec{x}^{\pi} := (x_{i}^{\pi} \; : \; 0 \leq i < |\vec{x}| )$ where $\pi \in \{L,R\}$.

We also use $P^{\Box} := P|\Box$.

In \cite{MeredithR05} an interpretation of the new operator is
given. It turns out that there are several possible interpretations
all enjoying the requisite algebraic properties of the operator (see
\cite{milner91polyadicpi}). We will therefore make liberal use of
$(\nu\; \vec{x})P$.

% subsection the_syntax_and_semantics_of_the_notation_system (end)   

\section{Interpretation of QM}
\subsection{Supporting definitions}
\subsubsection{Multiplication}
\begin{mathpar}
  \quotep{Q} \cdot \quotep{R} := \quotep{Q|R}
  \and \\
  \quotep{Q} \cdot P := P\{ \quotep{Q|R} / \quotep{R} : \quotep{R} \in \freenames{P} \}
\end{mathpar}

\paragraph{Discussion}
The first line needs little explanation. The second line says that
each free name of the process is replaced with the multiplication of
that name by the scalar. Multiplication of a scalar (name) by a state
(process) results in a process all the names of which have been `moved
over' by parallel composition with the process the scalar
quotes. There is a subtlety that the bound names have to be
manipulated so that multiplied names aren't accidentally
captured. There are many ways to achieve this.

\begin{remark}\label{rem:multiplication_identities}
  The reader is invited to verify that for all $x,y,z \in \QProc$ and $P \in \Proc$
  \begin{mathpar}
    x \cdot \quotep{0} \equiv x 
    \and
    x \cdot y \equiv y \cdot x
    \and
    x \cdot (y \cdot z) \equiv (x \cdot y) \cdot z
    \and \\
    \quotep{0} \cdot P \equiv P
    \and \\
    x \cdot (y \cdot P) \equiv (x \cdot y) \cdot P
    \and \\
    x \cdot (P|Q) \equiv (x \cdot P) | (x \cdot Q)
    \and \\    
  \end{mathpar}
\end{remark}

\subsubsection{Tensor product}

We define a tensor product on processes by structural induction.

\paragraph{Tensor of sums} First note that all summations, including
$\pzero$ and sequence, can be written $\Sigma_{i} x_{i}.A_{i} +
\Sigma_{j} x_{j}.C_{j}$, where we have grouped input-guarded processes
together and output-guarded processes together.

Thus, we can define the tensor product of two summations, $N_{1}\otimes N_{2}$, where

\begin{mathpar}
  N_{1} := \Sigma_{i} x_{i}.A_{i} + \Sigma_{j} x_{j}.C_{j}
  \and
  N_{2} := \Sigma_{i'} y_{i'}.B_{i'} + \Sigma_{j'} y_{j'}.D_{j'} 
\end{mathpar}

as follows.

\begin{mathpar}
  \Sigma_{i} x_{i}.A_{i} + \Sigma_{j} x_{j}.C_{j} \otimes \Sigma_{i'}
  y_{i'}.B_{i'} + \Sigma_{j'} y_{j'}.D_{j'} 
  \and \\
  := \; \Sigma_{i} \Sigma_{i'} \quotep{\stackrel{\vee}{x_{i}}| \stackrel{\vee}{y_{i'}}}.(A_{i}\otimes B_{i'}) \; | \; \Sigma_{i'} \Sigma_{i} \quotep{\stackrel{\vee}{y_{i'}}|\stackrel{\vee}{x_{i}}}.(B_{i'}\otimes A_{i})
  \and
  \;\; | \;\; \Sigma_{j} \Sigma_{j'} \quotep{\stackrel{\vee}{x_{j}}|\stackrel{\vee}{y_{j'}}}.(A_{j}\otimes B_{j'}) \; | \; \Sigma_{j'} \Sigma_{j} \quotep{\stackrel{\vee}{y_{j'}}|\stackrel{\vee}{x_{j}}}.(B_{j'}\otimes A_{j})
\end{mathpar}

\begin{remark}
  Do we need to $x^{L}$ and $y^{R}$ for this construction as well?
\end{remark}

\paragraph{Tensor of parallel compositions} Next, we distribute tensor
over par.

\begin{mathpar}
  P_{1}|P_{2} \otimes Q_{1}|Q_{2} := (P_{1} \otimes Q_{1}) | (P_{1}
  \otimes Q_{2}) | (P_{2} \otimes Q_{1}) | (P_{2} \otimes Q_{2})
\end{mathpar}

\paragraph{Tensor with dropped names} We treat tensor of a
process with a dropped name as parallel composition.

\begin{mathpar}
  P \otimes \dropn{x} := P | \dropn{x}
\end{mathpar}

\paragraph{Tensor of agents}

Finally, we need to define tensor on agents. Note that the definition
of tensor on normal products only tensors inputs with inputs and
outputs with outputs. Thus, we only have to define the operation on
``homogeneous'' pairings.

\begin{mathpar}
  (\vec{x})P \otimes (\vec{y})Q
  \and \\
  := (x_{0}^{L}|y_{0}^{R},\ldots,x_{0}^{L}|y_{n}^{R},\ldots,x_{m}^{L}|y_{0}^{R},\ldots,x_{m}^{L}|y_{n}^R)(P\{ \vec{x}^{L}/\vec{x}\} \otimes Q \{ \vec{y}^{R}/\vec{y}\})
  \and \\
  \clift{\vec{P}} \otimes \clift{\vec{Q}}
  \and \\
  := \clift{P_{0}\otimes Q_{0},\ldots,P_{0}\otimes Q_{n},\ldots,P_{m}\otimes Q_{0},\ldots,P_{m}\otimes Q_{n}}
\end{mathpar}

\begin{remark}
  Observe that arities of tensored abstractions matches arities of
  tensored concretions if the original arities matched. Note also that
  the length of the arities corresponds to the increase in dimension
  we see in ordinary vector space tensor product.
\end{remark}

\begin{remark}
  Operationally, this definition distributes the tensor down to
  components ``linked'' by summation. Tensor over summation is
  intriguing in that it mixes names. Moreover, as a consequence of the
  way it mixes names we have the identities for all $x \in \QProc$ and
  $P,Q \in \Proc$

  \begin{mathpar}
    (x \cdot P) \otimes Q \equiv x \cdot (P \otimes Q) \equiv P \otimes (x \cdot Q)
    \and
    P \otimes \pzero \equiv P
  \end{mathpar}

  that the reader is invited to verify.
\end{remark}

\subsubsection{Annihilation}
\begin{mathpar}
  P^{\perp} := \{ Q | \forall R. P|Q \red^{*} R \Rightarrow R \red^{*} \pzero \}
  \and \\
  P^{\underline{\perp}} := \Sigma_{Q \in P^{\perp}} \quotep{Q}?(y).(\dropn{y}|Q) | \Sigma_{Q \in P^{\perp}} \quotep{Q}\clift{\Box}
\end{mathpar}

\paragraph{Discussion} The reader will note that $P^{\perp}$ is a
\emph{set} of processes, while $P^{\underline{\perp}}$ is a
\emph{context}. We call the set $P^{\perp}$ the \emph{annihilators} of
$P$. The parallel composition of a process in the annihilators of $P$
with $P$ will result in a process, the state space of which has all
paths eventually leading to $\pzero$. Execution may endure loops; but
under reasonable conditions of fairness (naturally guaranteed under
most notions of bisimulation) such a composite process cannot get
stuck in such a loop and will, eventually pop out and terminate.

The context $P^{\underline{\perp}}$ is ready and willing to ``take the
$P$ out of'' the process to which it is applied. It will effectively
transmit the code of the process to which it is applied to one of the
annihilators and run the process against it.

\subsubsection{Evaluation}
We fix $M$ a domain of fully abstract interpretation with an equality
coincident with bisimulation. We take $\meaningof{\cdot} : \Proc \to
M$ to be the map interpreting processes and $\nmeaningof{\cdot} : \M
\to Proc$ to be the map running the other way. Then we define

\begin{mathpar}
  \int P := \nmeaningof{\meaningof{P}}
\end{mathpar}

\paragraph{Discussion}
There are many fully abstract interpretations of Milner's
$\pi$-calculus. Any of them can be used as a basis for interpreting
the reflective calculus here. Equipped with such a domain it is
largely a matter of grinding through to check that the Yoneda
construction for the normalization-by-evaluation program can be
extended to this setting.

\begin{remark}
  The reader is invited to verify that $\int (P^{\underline{\perp}}[P]) = 0$.
\end{remark}

\subsection{Quantum mechanics}

Table \ref{tbl:core_qm_op_defns} gives the core operational definitions

\begin{table}[htp]\label{tbl:core_qm_op_defns}
  \center{
    \fbox{
      \begin{tabular}{c|c}
        quantum mechanics & process calculus \\
        \hline
        scalar & $x := \quotep{P}$ \\
        state vector & $\state{P} := P$ \\
        dual & $\state{P}^{*} := \event{P^{\underline{\perp}}} := \quotep{P^{\underline{\perp}}}[-]$ \\
        matrix & $ \Sigma_{\alpha} \state{P_{\alpha}}x_{\alpha}\event{Q_{\alpha}}$ \\
        vector addition & $\state{P} + \state{Q} := \state{P | Q}$ \\
        tensor product & $\state{P} \otimes \state{Q} := \state{P \otimes Q}$ \\
        inner product & $\innerprod{P}{Q} := \quotep{\int P^{\underline{\perp}}[Q]}$ \\
      \end{tabular}
    }
  }
  \caption{QM - operational definitions}
\end{table}

where

\begin{mathpar}
  \prmatrix{P}{Q} := \fprmatrix{P}{\quotep{\pzero}}{Q}
  \and
  \fprmatrix{P}{x}{Q} := (\state{P},x,\event{Q})
  \and
  (\fprmatrix{P}{x}{Q})(\state{R}) := x \cdot \innerprod{Q}{R} \cdot \state{P}
  \and
  (\fprmatrix{P}{x}{Q})(\event{R}) := x \cdot \innerprod{R}{P} \cdot \event{Q}
\end{mathpar}

\paragraph{Discussion}
As promised: vectors (aka states) are represented as processes; duals
as contextual duals; inner product definition should be compared with
standard inner product definition for ....

\begin{remark}
  Assuming $\int (P^{\underline{\perp}}[P]) = 0$, the reader is
  invited to verify that $(\fprmatrix{P}{x}{P})(\state{P}) = x \cdot \state{P}$.
\end{remark}

\begin{remark}
  The reader is invited to verify that $\innerprod{P}{Q}$ could
  equally well have been written $\quotep{\int \stackrel{\vee}{x}}$
  where $x = \event{P^{\underline{\perp}}}(Q)$.

  One of the motivations for this remark is that there is another way
  to factor these operations. We could package up evaluation in the dual:

  \begin{mathpar}
    \state{P}^{*} := \event{\int P^{\underline{\perp}}} := \quotep{\int P^{\underline{\perp}}}[-]
  \end{mathpar}

  and then have inner product defined by
  
  \begin{mathpar}
    \innerprod{P}{Q} := \event{P}(Q)
  \end{mathpar}

  Hopefully, experience with the calculations will provide guidance on
  the best factoring.
\end{remark}

\begin{remark}
  Assuming $\int (P^{\underline{\perp}}[P]) = 0$, the reader is
  invited to verify that $\forall P,Q. (\prmatrix{0}{Q})(\state{0}) =
  \state{0}$ and dually $(\prmatrix{P}{0})(\event{0}) = \event{0}$.
\end{remark}

\begin{remark}
  i'm a little worried that i don't (yet) have proper support for
  complex conjugacy. But, the observation above may give us a
  clue. According to Abramsky, it must be the case that the scalars
  are iso to the homset of the identity for the tensor -- which the
  observation above characterizes. 

  For now, we will simply bookmark the notion with $\overline{x}$.
\end{remark}

\subsubsection{Adjointness}

We need to give a definition of $(\cdot)^{\dagger}$ for matrices. The
obvious candidate definition is
\begin{mathpar}
(\Sigma_{\alpha}\fprmatrix{P_{\alpha}}{x_{\alpha}}{Q_{\alpha}})^{\dagger}
= \Sigma_{\alpha}\fprmatrix{(Q_{\alpha}^{\underline{\perp}})^{*}}{\overline{x}_{\alpha}}{P_{\alpha}^{\underline{\perp}}} 
\end{mathpar}

But, $(Q_{\alpha}^{\underline{\perp}})^{*}$ requires a name along
which to communicate the process to achieve the context application.

\subsubsection{Basis for a basis}
If processes label states and ``addition'' of states (a.k.a. vector
addition) is interpreted as parallel composition, what corresponds to
notions of linear independence and basis? Here, we recall that Yoshida
has developed a set of \emph{combinators} for an asynchronous verison
of Milner's $\pi$-calculus. These are a finite set of processes such
any process can be expressed as parallel composition of these
combinators together with liberal uses of the new operator and
replication. We can simply give a translation of these into the
present calculus and have reasonable expectation that the property
carries over. That is, that the resultant set allows to express all
processes via parallel composition. Note, however, that there is no
new operator or replication in this calculus. As a result, we expect
that the corresponding set is actually infinite. That is, we expect
that the space is actually infinite dimensional.

\begin{remark}
  The attentive reader may be a bit concerned. Certainly, the
  collection $S$, $K$ and $I$ is a finite set of
  combinators. Shouldn't we expect to see a finite set of combinators
  for an effectively equivalent system? i am very sympathetic to this
  critique and feel it warrants full attention. On the other hand, i
  also have in mind the following analogy. The natural numbers, as a
  monoid under addition, has exactly $1$ generator, while the natural
  numbers, as a monoid under multiplication, has countably many
  generators (the primes). We observe that the application of the
  lambda calculus is much less resource sensitive than the parallel
  composition of the $\pi$-calculus. Could it be the case that we have
  an analogy of the form
  
  \begin{mathpar}
    m + n : MN :: m*n : M|N
  \end{mathpar}

  giving a similar blow up in the set of ``primes''?  This is such a
  wonderful thought that, even if it's not true, i think it's worth
  writing down.
\end{remark}
 

\documentclass[12pt]{llncs}
%\documentclass{jktr}

\usepackage[pdftex]{hyperref}                   
\usepackage {listings}
\usepackage {mathpartir}
\usepackage{bcprules}
%\usepackage{listings}
                       
\usepackage{graphicx} 
%\usepackage[margins=2.5cm,nohead,nofoot]{geometry}
%\usepackage{geometry}
\usepackage{amsfonts}
\usepackage{amstext}
\usepackage{latexsym}
\usepackage{amssymb}
\usepackage{color}


%\include{myPreamble}
\include{qm2pi.local} 

%\ifpdf
%\usepackage[pdftex]{graphicx}
%\else
%\usepackage{graphicx}
%\fi

 % \ifpdf
%  \usepackage{pdfsync}
%  \if


%\title{Brief Article}
%\author{David F. Snyder}
%\author{L.G. Meredith}

%\address{Dept. of Math., Texas State University--San Marcos, San Marcos, TX 78666}
       
\pagestyle{empty}


\begin{document}

\lstset{language=[Objective]Caml,frame=shadowbox}

\input{qm2pi.front}

% section front matter (end)

\input{qm2pi.intro} 
 
% section introduction (end)

% \input{qm2pi.knotations} 

% section notation (end)

\input{qm2pi.process.calculi} 

% section concurrent_process_calculi_and_spatial_logics_ (end)
    
%\input{qm2pi.knots2pi} 

%\input{qm2pi.trefoil} 

%\input{qm2pi.mainthm} 

% subsection basic_interpretation (end)

%\input{qm2pi.rho.presentation} 
\subsection{The syntax and semantics of the notation system}\label{sub:the_syntax_and_semantics_of_the_notation_system} % (fold)

We now summarize a technical presentation of the calculus that
embodies our theory of dynamics. The typical presentation of such a
calculus follows the style of giving generators and relations on
them. The grammar, below, describing term constructors, freely
generates the set of processes, $\Proc$. This set is then quotiented
by a relation known as structural congruence and it is over this set
that the notion of dynamics is expressed. This presentation is
essentially that of \cite{MeredithR05} with the addition of
polyadicity and summation. For readability we have relegated some of
the technical subtleties to an appendix.

\subsubsection{Process grammar}\label{subsub:process_grammar}

\begin{mathpar}
  \inferrule* [lab=synchronization] {} {{M} \bc \pzero \;|\; x?F \;|\; x!C }
  \and
  \inferrule* [lab=abstraction] {} {{F} \bc (x)P}
  \and
  \inferrule* [lab=concretion] {} {{C} \bc \langle Q \rangle}
  \and
  \inferrule* [lab=process] {} {{P,Q} \bc M \;| \;P|Q \;|\; @{x}}
  \and
  \inferrule* [lab=name] {} {{x} \bc \quotep{P}}
\end{mathpar} 

Note that $\vec{x}$ (resp. $\vec{P}$) denotes a vector of names
(resp. processes) of length $|\vec{x}|$ (resp. $|\vec{P}|$). We adopt
the following useful abbreviations.

\begin{mathpar}
   x?(\vec{y}).P := x.(\vec{y})P \and  x\clift{\vec{P}} := x.\clift{\vec{P}}
   \and x!(y) := \lift{x}{\dropn{y}}
   \and \Pi_{i=0}^{n-1}P_i := P_0 | \ldots | P_{n-1}
\end{mathpar}

\subsubsection{Structural congruence}

\paragraph{Free and bound names and alpha-equivalence.} At the
core of structural equivalence is alpha-equivalence which identifies
process that are the same up to a change of variable. Formally, we
recognize the distinction between free and bound names. The free names
of a process, $\freenames{P}$, may be calculated recursively as
follows:

\begin{mathpar}
\freenames{\pzero} := \emptyset
  \and \\
  \freenames{x?(y).P} := \{ x \} \cup (\freenames{P} \setminus \{ y \})
  \and 
  \freenames{x!\langle P \rangle} := \{ x \} \cup \{ P \} 
  \and \\
  \freenames{P|Q} := \freenames{P} \cup \freenames{Q}
  \and \\
  \freenames{@{x}} := \{ x \}
\end{mathpar}

$\pi$
$\quotep{\pi}$

$\freenames{-} : \pi \to \mathcal{P}(\quotep{\pi})$

\begin{eqnarray*}
  \freenames{\pzero} & := & \emptyset \\
  \freenames{x?(y).P} & := & \{ x \} \cup (\freenames{P} \setminus \{ y \}) \\
  \freenames{x!\langle P \rangle} & := & \{ x \} \cup \{ P \} \\
  \freenames{P|Q} & := & \freenames{P} \cup \freenames{Q} \\
  \freenames{\dropn{x}} & := & \{ x \}
\end{eqnarray*}

The bound names of a process, $\boundnames{P}$, are those names occurring in $P$
that are not free. For example, in $x?(y).0$, the name $x$ is free, while $y$ is bound.

\begin{mathpar}
  \inferrule* [lab=monoidal-laws] {} { P|Q \equiv Q|P \and P|0 \equiv P \and P|(Q|R) \equiv (P|Q)|R }
\end{mathpar}

\begin{mathpar}
  \inferrule* [lab=alpha-equivalence] {} { (x)P \equiv (y)P\{y/x\} \and y \not\in \freenames{P} }
\end{mathpar}

\begin{definition}
Then two processes, $P,Q$, are alpha-equivalent if $P = Q\{\vec{y}/\vec{x}\}$ for
some $\vec{x} \in \boundnames{Q},\vec{y} \in \boundnames{P}$, where $Q\{\vec{y}/\vec{x}\}$
denotes the capture-avoiding substitution of $\vec{y}$ for $\vec{x}$ in $Q$.
\end{definition}

\begin{definition}
  The {\em structural congruence} \cite{SangiorgiWalker} , $\equiv$,
  between processes is the least congruence containing
  alpha-equivalence, satisfying the abelian monoid laws
  (associativity, commutativity and $\pzero$ as identity) for parallel
  composition $|$ and for summation $+$.
\end{definition}

\subsection{Name equivalence}

We take name equivalence, written $\nameeq$, to be the smallest
equivalence relation generated by the following rules.

\begin{mathpar}
\inferrule*[lab=Quote-drop]
{ }
{ \quotep{@{x}} \nameeq x }

\inferrule*[lab=Struct-equiv]
{ P \scong Q }
{ \quotep{P} \nameeq \quotep{Q} }
\end{mathpar}

The astute reader will have noticed that the mutual recursion of names
and processes imposes a mutual recursion on alpha-equivalence and
structural equivalence via name-equivalence. Fortunately, all of this
works out pleasantly and we may calculate in the natural way, free of
concern. The reader interested in the details is referred to the
appendix \ref{appendix:rho_details}.

\subsection{Substitution}

We use $\Proc$ for the set of processes, $\QProc$ for the set of
names, and $\id{\{}\vec{y} / \vec{x} \id{\}}$ to denote partial maps,
$s : \QProc \rightarrow \QProc$. A map, $s$ lifts, uniquely, to a map
on process terms, $\widehat{s} : \Proc \rightarrow \Proc$ by the
following equations.

\begin{mathpar}
  (0) \psubstp{Q}{P} := 0 \\
  (R \juxtap S) \psubstp{Q}{P}
  :=    
  (R)\psubstp{Q}{P} \juxtap (S) \psubstp{Q}{P} \\
  (x?(y).R) \psubstp{Q}{P}    
  :=    
  (x)\substp{Q}{P} (z)\concat( (R \psubstn{z}{y}) \psubstp{Q}{P} ) \\
  (\lift{x}{R}) \psubstp{Q}{P}  
  :=
  \lift{(x)\substp{Q}{P}}{ R \psubstp{Q}{P} } \\
%   (\dropn{x})  \psubstp{Q}{P}       
%   := 
%   \left\{ 
%     \begin{array}{ccc} 
%       \dropn{\quotep{Q}} & & x \nameeq \quotep{P} \\
%       \dropn{x} & & otherwise \\
%     \end{array}
%   \right. 
  (\dropn{x})  \psubstp{Q}{P}       
  := 
  \left\{ 
    \begin{array}{ccc} 
      Q & & x \nameeq \quotep{P} \\
      \dropn{x} & & otherwise \\
    \end{array}
  \right.
\end{mathpar}
 

where

\begin{eqnarray}
  (x)\id{\{} \lpquote Q \rpquote / \lpquote P \rpquote \id{\}}            = 
  \left\{ 
    \begin{array}{ccc}
      \lpquote Q \rpquote & & x \nameeq \lpquote P \rpquote \\
      x & & otherwise \\
    \end{array}
  \right. \nonumber
\end{eqnarray}

and $z$ is chosen distinct from $\quotep{P}$, $\quotep{Q}$, the free
names in $Q$, and all the names in $R$. Our $\alpha$-equivalence will
be built in the standard way from this substitution.

\begin{remark}\label{rem:no_self_referential_names}
  One consequence of these definitions is that $\forall P. \quotep{P}
  \not\in \freenames{P}$.
\end{remark}

\subsection{ Dynamic quote: an example }

Anticipating something of what's to come, consider applying the
substitution, $\widehat{\id{\{}u / z \id{\}}}$, to the following pair
of processes, $\lift{w}{y!(z)}$ and $w[ \lpquote y!(z) \rpquote ]$.

\begin{eqnarray}
	\lift{w}{y!(z)}\widehat{\id{\{}u / z \id{\}}}
		& = &
		\lift{w}{y!(u)} \nonumber\\
	w[ \lpquote y!(z) \rpquote ] \widehat{ \id{\{}u / z \id{\}} }
		& = &
		w[ \lpquote y!(z) \rpquote ] \nonumber
\end{eqnarray}

Because the body of the process between quotes is impervious to
substitution, we get radically different answers. In fact, by
examining the first process in an input context,
e.g. $x?(z).\lift{w}{y!(z)}$, we see that the process under the lift
operator may be shaped by prefixed inputs binding a name inside it. In
this sense, the lift operator will be seen as a way to dynamically
construct processes before reifying them as names.

Finally equipped with these standard features we can present the
dynamics of the calculus.

\subsubsection{Operational semantics} 

Finally, we introduce the computational dynamics. What marks these
algebras as distinct from other more traditionally studied algebraic
structures, e.g. vector spaces or polynomial rings, is the manner in
which dynamics is captured. In traditional structures, dynamics is typically
expressed through morphisms between such structures, as in linear maps
between vector spaces or morphisms between rings. In algebras
associated with the semantics of computation, the dynamics is
expressed as part of the algebraic structure itself, through a
reduction reduction relation typically denoted by $\red$. Below, we
give a recursive presentation of this relation for the calculus used
in the encoding.

$\red \subseteq \pi \times \pi$
$\red : \pi \to \mathcal{P}(\pi)$

\begin{mathpar}
  \inferrule* [lab=Comm] { \textsf{match}( x_{src}, x_{trgt} ) } { x_{trgt}?(y)P \; | \; x_{src}!\langle {Q} \rangle \red P\{\quotep{Q}/y}\} }
  \and \\
  \inferrule* [lab=Par] {{P} \red {P}'} {{{P} | {Q}} \red {{P}' | {Q}}}
  \and
  \inferrule* [lab=Equiv]{{{P} \scong {P}'} \andalso {{P}' \red {Q}'} \andalso {{Q}' \scong {Q}}}{{P} \red {Q}}
\end{mathpar}

\begin{eqnarray*}
  match_{\equiv} (\quotep{P},\quotep{Q}) & := & P \equiv Q \\
  match_{\dagger}(\quotep{P},\quotep{Q}) & := & \forall R. P|Q \red^{*} R => R \red^{*} 0 \\
  match_{K}(\quotep{P},\quotep{Q}) & := & K \mbox{ for some context } K
\end{eqnarray*}

$u?(x)P | u!\langle Q \rangle \red P\{\quotep{Q}/x\}$

%We write $\wred$ for $\red^*$, and $P\red$ if $\exists Q $ such that $ P \red Q$.
We write $P\red$ if $\exists Q $ such that $ P \red Q$ and $P\not\red$, otherwise.

\section{Replication}

As mentioned before, it is known that replication (and hence
recursion) can be implemented in a higher-order process algebra
\cite{SangiorgiWalker}. As our first example of calculation with the
machinery thus far presented we give the construction explicitly in
the {\rhoc}.

\begin{eqnarray}
	D_{x} & := & \prefix{x}{y}{(\binpar{\outputp{x}{y}}{@{y}})} \nonumber\\
	\bangp_{x}{P} & := & \binpar{{x}!\langle{\binpar{D_{x}}{P}}\rangle}{D_{x}} \nonumber
\end{eqnarray}

\begin{eqnarray}
	\bangp_{x}{P} & & \nonumber\\
	=
	& {x}!\langle{(\prefix{x}{y}{(\outputp{x}{y} | @{y})) | P}}\rangle 
	      | \prefix{x}{y}{(\outputp{x}{y} | @{y})} & \nonumber\\
	\red
	& (\outputp{x}{y} | @{y})\substn{\quotep{(\prefix{x}{y}{(@{y} | \outputp{x}{y})) | P}}}{y} & \nonumber\\
	=
	& \outputp{x}{\quotep{(\prefix{x}{y}{(\outputp{x}{y} | @{y})) | P}}}
	  | {(\prefix{x}{y}{(\outputp{x}{y} | @{y})) | P}} & \nonumber\\
	\red
	& \ldots & \nonumber\\
	\red^*
	& P | P | \ldots & \nonumber
\end{eqnarray}

Of course, this encoding, as an implementation, runs away, unfolding
$\bangp{P}$ eagerly. A lazier and more implementable replication
operator, restricted to input-guarded processes, may be obtained as follows.

\begin{eqnarray}
\bangp{\prefix{u}{v}{P}} 
	:= 
	\binpar{\lift{x}{\prefix{u}{v}{(\binpar{D(x)}{P})}}}{D(x)} \nonumber
\end{eqnarray}

\begin{remark}
  Note that the lazier definition still does not deal with summation
  or mixed summation (i.e. sums over input and output). The reader is
  invited to construct definitions of replication that deal with these
  features. 

  Further, the definitions are parameterized in a name, $x$. Can you,
  gentle reader, make a definition that eliminates this parameter and
  guarantees no accidental interaction between the replication
  machinery and the process being replicated -- i.e. no accidental
  sharing of names used by the process to get its work done and the
  name(s) used by the replication to effect copying. This latter
  revision of the definition of replication is crucial to obtaining
  the expected identity $!!P \sim !P$.
\end{remark}

\begin{remark}\label{rem:paradoxical_combinator}
  The reader familiar with the lambda calculus will have noticed the
  similarity between $D$ and the paradoxical combinator.

  [Ed. note: the existence of this seems to suggest we have to be more
  restrictive on the set of processes and names we admit if we are to
  support no-cloning.]
\end{remark}

\subsubsection{Bisimulation}

The computational dynamics gives rise to another kind of equivalence,
the equivalence of computational behavior. As previously mentioned
this is typically captured \emph{via} some form of bisimulation.

% The notion we use in this paper is weak barbed bisimulation
% \cite{milner91polyadicpi}.

The notion we use in this paper is derived from weak barbed
bisimulation \cite{milner91polyadicpi}. 

\begin{definition}
An \emph{observation relation}, $\downarrow_{\mathcal N}$, over a set
of names, $\mathcal N$, is the smallest relation satisfying the rules
below.

\infrule[Out-barb]{y \in {\mathcal N}, \; x \nameeq y}
		  {\outputp{x}{v} \downarrow_{\mathcal N} x}
\infrule[Par-barb]{\mbox{$P\downarrow_{\mathcal N} x$ or $Q\downarrow_{\mathcal N} x$}}
		  {\binpar{P}{Q} \downarrow_{\mathcal N} x}

We write $P \Downarrow_{\mathcal N} x$ if there is $Q$ such that 
$P \wred Q$ and $Q \downarrow_{\mathcal N} x$.
\end{definition}

\begin{definition}
%\label{def.bbisim}
An  ${\mathcal N}$-\emph{barbed bisimulation} over a set of names, ${\mathcal N}$, is a symmetric binary relation 
${\mathcal S}_{\mathcal N}$ between agents such that $P\rel{S}_{\mathcal N}Q$ implies:
\begin{enumerate}
\item If $P \red P'$ then $Q \wred Q'$ and $P'\rel{S}_{\mathcal N} Q'$.
\item If $P\downarrow_{\mathcal N} x$, then $Q\Downarrow_{\mathcal N} x$.
\end{enumerate}
$P$ is ${\mathcal N}$-barbed bisimilar to $Q$, written
$P \wbbisim_{\mathcal N} Q$, if $P \rel{S}_{\mathcal N} Q$ for some ${\mathcal N}$-barbed bisimulation ${\mathcal S}_{\mathcal N}$.
\end{definition}

$\mathcal{R} \subseteq \pi \times \pi$

$P \mathcal{R} Q => \forall P'. P \red P' \Rightarrow \exists Q'. Q \red Q', P' \mathcal{R} Q'$

$P \vdash x \Rightarrow Q \vdash x$

\begin{mathpar}
  \inferrule*[lab=Out-barb]{x \nameeq y}{{y}!\langle{Q}\rangle \vdash x}
  \and
  \inferrule*[lab=Par-barb]{\mbox{$P\vdash x$ or $Q\vdash x$}}{\binpar{P}{Q} \vdash x}
\end{mathpar}

\subsubsection{Contexts}

One of the principle advantages of computational calculi like the
$\pi$-calculus is a well-defined notion of context,
contextual-equivalence and a correlation between
contextual-equivalence and notions of bisimulation. The notion of
context allows the decomposition of a process into (sub-)process and
its syntactic environment, its context. Thus, a context may be
thought of as a process with a ``hole'' (written $\Box$) in it. The
application of a context $M$ to a process $P$, written $M[P]$, is
tantamount to filling the hole in $M$ with $P$. In this paper we do
not need the full weight of this theory, but do make use of the notion
of context in the proof the main theorem. 

\begin{mathpar}
  \inferrule* [lab=summation] {} {{M_{M},M_{N}} \bc \Box \;|\; x.M_{A} \;|\; M_{M}+M_{N}}
  \and
  \inferrule* [lab=agent] {} {{M_{A}} \bc (\vec{x})M_{P} \;| \; \clift{P_0,\ldots,M_{P},\ldots,P_N}}
  \and \\
  \inferrule* [lab=process] {} {{M_{P}} \bc M_{N} \;| \;P|M_{P} }
\end{mathpar} 

\begin{mathpar}
  \inferrule* [lab=sychronization] {} {M_{N} \bc \Box \;|\; x?M_{F} \;|\; x!M_{C}}
  \and
  \inferrule* [lab=abstraction] {} {{M_{F}} \bc (x)M_{P} }
  \and
  \inferrule* [lab=concretion] {} {{M_{C}} \bc \langle M_{P} \rangle }
  \and \\
  \inferrule* [lab=process] {} {{M_{P}} \bc M_{N} \;| \;P|M_{P} }
\end{mathpar}

\begin{definition}[contextual application] Given a context $M$, and
  process $P$, we define the \emph{contextual application}, $M[P] :=
  M\{P/\Box\}$. That is, the contextual application of M to P is the
  substitution of $P$ for $\Box$ in $M$.
\end{definition}

$\meaningof{-} : L \to \mathcal{P}(\pi)$

\begin{mathpar}
  \inferrule* [lab=collection] {} {\meaningof{true} = \pi, \and \meaningof{~E} = \pi \setminus \meaningof{E}, \and \meaningof{E_{1} \& E_{2}} = \meaningof{E_{1}} \cap \meaningof{E_{2}}}
\end{mathpar}

\begin{mathpar}
  \inferrule* [lab=structure] {} {\meaningof{0} = \{ P \in \pi | P \equiv 0 \}, \and \\ \meaningof{E_1 | E_2} = \{ P \in \pi | P \equiv P_{1} | P_{2}, P_{1} \in \meaningof{E_{1}}, P_{2} \in \meaningof{E_2}\} }
\end{mathpar}

\begin{mathpar}
 \inferrule* [lab=behavior] {} {\meaningof{\langle a?b \rangle E} = \{ P \in \pi | P \equiv Q | u?(y)P', \\ \and \\\\ \and \\ \;\;\; u \in \meaningof{a}, \forall z.P'\{z/y\} \in \meaningof{E\{z/b\}}\}, \and \\ \meaningof{a!E} = \{ P \in \pi | P \equiv Q | x!\langle P' \rangle, x \in \meaningof{a} P' \in \meaningof{E}\} }
\end{mathpar}

\begin{mathpar}
 \inferrule* [lab=nominal] {} {\meaningof{\quotep{E}} = \{ \quotep{P} \in \quotep{\pi} | P \in \meaningof{E} \}, \and \meaningof{\quotep{P}} = \{ \quotep{Q} \in \quotep{\pi} | P \equiv Q \} \and \\ \meaningof{@\quotep{E}} = \{ P \in \pi | P \equiv @x, x \in \meaningof{E} \}}
\end{mathpar}

\begin{eqnarray*}
  \\
  \meaningof{-} : TS \to ST
\end{eqnarray*}

\begin{eqnarray*}
  \\
  L : TS \to ST
\end{eqnarray*}

\begin{eqnarray*}
  \\
  P \models E \iff P \in \meaningof{E}
\end{eqnarray*}

\begin{eqnarray*}
  P \approx_{L} Q \iff \forall E \in L. P \models E \iff Q \models E
\end{eqnarray*}

\begin{eqnarray*}
  P \approx_{K} Q
\end{eqnarray*}

\begin{eqnarray*}
  P \approx Q
\end{eqnarray*}

$\approx_{K} = \approx = \approx_{L}$

\subsubsection{Contextual duality}

Note that contexts extend the quotation operation to a family of
operations from processes to names. Given a context, $M$, we can
define a \emph{nominal context}, $\quotep{M}$ by $\quotep{M}[P] :=
\quotep{M[P]}$. To foreshadow what is to come we observe that these
operations enjoy a duality with processes very much like the duality
between vectors and maps from vectors to scalars.

Further, because the calculus is essentially higher-order, we have a
correspondence between contexts and processes. More specifically,
given a name $x$ and a context $M$ we can construct $M^{*}_{x}$ such
that 

\begin{mathpar}
  M^{*}_{x} | \lift{x}{P} \red M[P]
\end{mathpar}

namely,

\begin{mathpar}
  M^{*}_{x} := x?(u).M[\dropn{u}]
\end{mathpar}

The dependence of $M^{*}_{x}$ on a name makes it an abstraction, 

\begin{mathpar}
  M^{*} := (x)x?(u).M[\dropn{u}]
\end{mathpar}

\subsection{Additional notation}

It will sometimes be convenient to denote the process a name
quotes. We already have the notation $x = \quotep{P}$, but it will be
convenient to introduce an alternate notation, $\procn{x}$, when we
want to emphasize the connection to the use of the name. Note that, by
virtue of name equivalence, $\quotep{\procn{x}} \nameeq x$; so, the
notation is consistent with previous definitions.

Further, because names have structure it is possible to effect
substitutions on the basis of that structure. This means we need to
upgrade our notation for substitutions, which we accomplish by
adapting comprehension notation. Thus,

\begin{mathpar}
  P\{ y / x : x \in S \}
\end{mathpar}

is interpreted to mean the process derived from P by replacing (in a
capture-avoiding manner) each occurrence of $x$ in $S$ by $y$. For example,

\begin{mathpar}
  P\{ \quotep{\procn{x}|\procn{x}} / x : x \in \freenames{P} \}
\end{mathpar}

will replace each (occurrence) of a free name $x$ in $P$ by
$\quotep{\procn{x}|\procn{x}}$.

Also, we will avail ourselves of the notation $x^{L}$ and $x^{R}$ to
denote injections of a name into disjoint copies of the name
space. There are numerous ways to accomplish this. One example can be
found in \cite{MeredithR05}. This notation overloads to vectors of
names: $\vec{x}^{\pi} := (x_{i}^{\pi} \; : \; 0 \leq i < |\vec{x}| )$ where $\pi \in \{L,R\}$.

We also use $P^{\Box} := P|\Box$.

In \cite{MeredithR05} an interpretation of the new operator is
given. It turns out that there are several possible interpretations
all enjoying the requisite algebraic properties of the operator (see
\cite{milner91polyadicpi}). We will therefore make liberal use of
$(\nu\; \vec{x})P$.

% subsection the_syntax_and_semantics_of_the_notation_system (end)   

\input{qm2pi.qmops} 

\input{qm2pi.sterngerlach} 

\input{qm2pi.metric} 

% section concurrent_process_calculi (end)

%\input{qm2pi.proofsketch}

% section proof sketch (end)

%\input{qm2pi.slviaknots} 

% section spatial logic via knots (end)

\input{qm2pi.conclusion}

% section conclusion (end)

%\input{qm2pi.dtcodes} 

% section wiring algorithm (end)

\input{qm2pi.ack} 

% section acknowledgments (end)

\newpage


\bibliographystyle{plain}   
\bibliography{../../biblios/main.bib}

\input{qm2pi.rhodetails}

\end{document}

 

\documentclass[12pt]{llncs}
%\documentclass{jktr}

\usepackage[pdftex]{hyperref}                   
\usepackage {listings}
\usepackage {mathpartir}
\usepackage{bcprules}
%\usepackage{listings}
                       
\usepackage{graphicx} 
%\usepackage[margins=2.5cm,nohead,nofoot]{geometry}
%\usepackage{geometry}
\usepackage{amsfonts}
\usepackage{amstext}
\usepackage{latexsym}
\usepackage{amssymb}
\usepackage{color}


%\include{myPreamble}
\include{qm2pi.local} 

%\ifpdf
%\usepackage[pdftex]{graphicx}
%\else
%\usepackage{graphicx}
%\fi

 % \ifpdf
%  \usepackage{pdfsync}
%  \if


%\title{Brief Article}
%\author{David F. Snyder}
%\author{L.G. Meredith}

%\address{Dept. of Math., Texas State University--San Marcos, San Marcos, TX 78666}
       
\pagestyle{empty}


\begin{document}

\lstset{language=[Objective]Caml,frame=shadowbox}

\input{qm2pi.front}

% section front matter (end)

\input{qm2pi.intro} 
 
% section introduction (end)

% \input{qm2pi.knotations} 

% section notation (end)

\input{qm2pi.process.calculi} 

% section concurrent_process_calculi_and_spatial_logics_ (end)
    
%\input{qm2pi.knots2pi} 

%\input{qm2pi.trefoil} 

%\input{qm2pi.mainthm} 

% subsection basic_interpretation (end)

%\input{qm2pi.rho.presentation} 
\subsection{The syntax and semantics of the notation system}\label{sub:the_syntax_and_semantics_of_the_notation_system} % (fold)

We now summarize a technical presentation of the calculus that
embodies our theory of dynamics. The typical presentation of such a
calculus follows the style of giving generators and relations on
them. The grammar, below, describing term constructors, freely
generates the set of processes, $\Proc$. This set is then quotiented
by a relation known as structural congruence and it is over this set
that the notion of dynamics is expressed. This presentation is
essentially that of \cite{MeredithR05} with the addition of
polyadicity and summation. For readability we have relegated some of
the technical subtleties to an appendix.

\subsubsection{Process grammar}\label{subsub:process_grammar}

\begin{mathpar}
  \inferrule* [lab=synchronization] {} {{M} \bc \pzero \;|\; x?F \;|\; x!C }
  \and
  \inferrule* [lab=abstraction] {} {{F} \bc (x)P}
  \and
  \inferrule* [lab=concretion] {} {{C} \bc \langle Q \rangle}
  \and
  \inferrule* [lab=process] {} {{P,Q} \bc M \;| \;P|Q \;|\; @{x}}
  \and
  \inferrule* [lab=name] {} {{x} \bc \quotep{P}}
\end{mathpar} 

Note that $\vec{x}$ (resp. $\vec{P}$) denotes a vector of names
(resp. processes) of length $|\vec{x}|$ (resp. $|\vec{P}|$). We adopt
the following useful abbreviations.

\begin{mathpar}
   x?(\vec{y}).P := x.(\vec{y})P \and  x\clift{\vec{P}} := x.\clift{\vec{P}}
   \and x!(y) := \lift{x}{\dropn{y}}
   \and \Pi_{i=0}^{n-1}P_i := P_0 | \ldots | P_{n-1}
\end{mathpar}

\subsubsection{Structural congruence}

\paragraph{Free and bound names and alpha-equivalence.} At the
core of structural equivalence is alpha-equivalence which identifies
process that are the same up to a change of variable. Formally, we
recognize the distinction between free and bound names. The free names
of a process, $\freenames{P}$, may be calculated recursively as
follows:

\begin{mathpar}
\freenames{\pzero} := \emptyset
  \and \\
  \freenames{x?(y).P} := \{ x \} \cup (\freenames{P} \setminus \{ y \})
  \and 
  \freenames{x!\langle P \rangle} := \{ x \} \cup \{ P \} 
  \and \\
  \freenames{P|Q} := \freenames{P} \cup \freenames{Q}
  \and \\
  \freenames{@{x}} := \{ x \}
\end{mathpar}

$\pi$
$\quotep{\pi}$

$\freenames{-} : \pi \to \mathcal{P}(\quotep{\pi})$

\begin{eqnarray*}
  \freenames{\pzero} & := & \emptyset \\
  \freenames{x?(y).P} & := & \{ x \} \cup (\freenames{P} \setminus \{ y \}) \\
  \freenames{x!\langle P \rangle} & := & \{ x \} \cup \{ P \} \\
  \freenames{P|Q} & := & \freenames{P} \cup \freenames{Q} \\
  \freenames{\dropn{x}} & := & \{ x \}
\end{eqnarray*}

The bound names of a process, $\boundnames{P}$, are those names occurring in $P$
that are not free. For example, in $x?(y).0$, the name $x$ is free, while $y$ is bound.

\begin{mathpar}
  \inferrule* [lab=monoidal-laws] {} { P|Q \equiv Q|P \and P|0 \equiv P \and P|(Q|R) \equiv (P|Q)|R }
\end{mathpar}

\begin{mathpar}
  \inferrule* [lab=alpha-equivalence] {} { (x)P \equiv (y)P\{y/x\} \and y \not\in \freenames{P} }
\end{mathpar}

\begin{definition}
Then two processes, $P,Q$, are alpha-equivalent if $P = Q\{\vec{y}/\vec{x}\}$ for
some $\vec{x} \in \boundnames{Q},\vec{y} \in \boundnames{P}$, where $Q\{\vec{y}/\vec{x}\}$
denotes the capture-avoiding substitution of $\vec{y}$ for $\vec{x}$ in $Q$.
\end{definition}

\begin{definition}
  The {\em structural congruence} \cite{SangiorgiWalker} , $\equiv$,
  between processes is the least congruence containing
  alpha-equivalence, satisfying the abelian monoid laws
  (associativity, commutativity and $\pzero$ as identity) for parallel
  composition $|$ and for summation $+$.
\end{definition}

\subsection{Name equivalence}

We take name equivalence, written $\nameeq$, to be the smallest
equivalence relation generated by the following rules.

\begin{mathpar}
\inferrule*[lab=Quote-drop]
{ }
{ \quotep{@{x}} \nameeq x }

\inferrule*[lab=Struct-equiv]
{ P \scong Q }
{ \quotep{P} \nameeq \quotep{Q} }
\end{mathpar}

The astute reader will have noticed that the mutual recursion of names
and processes imposes a mutual recursion on alpha-equivalence and
structural equivalence via name-equivalence. Fortunately, all of this
works out pleasantly and we may calculate in the natural way, free of
concern. The reader interested in the details is referred to the
appendix \ref{appendix:rho_details}.

\subsection{Substitution}

We use $\Proc$ for the set of processes, $\QProc$ for the set of
names, and $\id{\{}\vec{y} / \vec{x} \id{\}}$ to denote partial maps,
$s : \QProc \rightarrow \QProc$. A map, $s$ lifts, uniquely, to a map
on process terms, $\widehat{s} : \Proc \rightarrow \Proc$ by the
following equations.

\begin{mathpar}
  (0) \psubstp{Q}{P} := 0 \\
  (R \juxtap S) \psubstp{Q}{P}
  :=    
  (R)\psubstp{Q}{P} \juxtap (S) \psubstp{Q}{P} \\
  (x?(y).R) \psubstp{Q}{P}    
  :=    
  (x)\substp{Q}{P} (z)\concat( (R \psubstn{z}{y}) \psubstp{Q}{P} ) \\
  (\lift{x}{R}) \psubstp{Q}{P}  
  :=
  \lift{(x)\substp{Q}{P}}{ R \psubstp{Q}{P} } \\
%   (\dropn{x})  \psubstp{Q}{P}       
%   := 
%   \left\{ 
%     \begin{array}{ccc} 
%       \dropn{\quotep{Q}} & & x \nameeq \quotep{P} \\
%       \dropn{x} & & otherwise \\
%     \end{array}
%   \right. 
  (\dropn{x})  \psubstp{Q}{P}       
  := 
  \left\{ 
    \begin{array}{ccc} 
      Q & & x \nameeq \quotep{P} \\
      \dropn{x} & & otherwise \\
    \end{array}
  \right.
\end{mathpar}
 

where

\begin{eqnarray}
  (x)\id{\{} \lpquote Q \rpquote / \lpquote P \rpquote \id{\}}            = 
  \left\{ 
    \begin{array}{ccc}
      \lpquote Q \rpquote & & x \nameeq \lpquote P \rpquote \\
      x & & otherwise \\
    \end{array}
  \right. \nonumber
\end{eqnarray}

and $z$ is chosen distinct from $\quotep{P}$, $\quotep{Q}$, the free
names in $Q$, and all the names in $R$. Our $\alpha$-equivalence will
be built in the standard way from this substitution.

\begin{remark}\label{rem:no_self_referential_names}
  One consequence of these definitions is that $\forall P. \quotep{P}
  \not\in \freenames{P}$.
\end{remark}

\subsection{ Dynamic quote: an example }

Anticipating something of what's to come, consider applying the
substitution, $\widehat{\id{\{}u / z \id{\}}}$, to the following pair
of processes, $\lift{w}{y!(z)}$ and $w[ \lpquote y!(z) \rpquote ]$.

\begin{eqnarray}
	\lift{w}{y!(z)}\widehat{\id{\{}u / z \id{\}}}
		& = &
		\lift{w}{y!(u)} \nonumber\\
	w[ \lpquote y!(z) \rpquote ] \widehat{ \id{\{}u / z \id{\}} }
		& = &
		w[ \lpquote y!(z) \rpquote ] \nonumber
\end{eqnarray}

Because the body of the process between quotes is impervious to
substitution, we get radically different answers. In fact, by
examining the first process in an input context,
e.g. $x?(z).\lift{w}{y!(z)}$, we see that the process under the lift
operator may be shaped by prefixed inputs binding a name inside it. In
this sense, the lift operator will be seen as a way to dynamically
construct processes before reifying them as names.

Finally equipped with these standard features we can present the
dynamics of the calculus.

\subsubsection{Operational semantics} 

Finally, we introduce the computational dynamics. What marks these
algebras as distinct from other more traditionally studied algebraic
structures, e.g. vector spaces or polynomial rings, is the manner in
which dynamics is captured. In traditional structures, dynamics is typically
expressed through morphisms between such structures, as in linear maps
between vector spaces or morphisms between rings. In algebras
associated with the semantics of computation, the dynamics is
expressed as part of the algebraic structure itself, through a
reduction reduction relation typically denoted by $\red$. Below, we
give a recursive presentation of this relation for the calculus used
in the encoding.

$\red \subseteq \pi \times \pi$
$\red : \pi \to \mathcal{P}(\pi)$

\begin{mathpar}
  \inferrule* [lab=Comm] { \textsf{match}( x_{src}, x_{trgt} ) } { x_{trgt}?(y)P \; | \; x_{src}!\langle {Q} \rangle \red P\{\quotep{Q}/y}\} }
  \and \\
  \inferrule* [lab=Par] {{P} \red {P}'} {{{P} | {Q}} \red {{P}' | {Q}}}
  \and
  \inferrule* [lab=Equiv]{{{P} \scong {P}'} \andalso {{P}' \red {Q}'} \andalso {{Q}' \scong {Q}}}{{P} \red {Q}}
\end{mathpar}

\begin{eqnarray*}
  match_{\equiv} (\quotep{P},\quotep{Q}) & := & P \equiv Q \\
  match_{\dagger}(\quotep{P},\quotep{Q}) & := & \forall R. P|Q \red^{*} R => R \red^{*} 0 \\
  match_{K}(\quotep{P},\quotep{Q}) & := & K \mbox{ for some context } K
\end{eqnarray*}

$u?(x)P | u!\langle Q \rangle \red P\{\quotep{Q}/x\}$

%We write $\wred$ for $\red^*$, and $P\red$ if $\exists Q $ such that $ P \red Q$.
We write $P\red$ if $\exists Q $ such that $ P \red Q$ and $P\not\red$, otherwise.

\section{Replication}

As mentioned before, it is known that replication (and hence
recursion) can be implemented in a higher-order process algebra
\cite{SangiorgiWalker}. As our first example of calculation with the
machinery thus far presented we give the construction explicitly in
the {\rhoc}.

\begin{eqnarray}
	D_{x} & := & \prefix{x}{y}{(\binpar{\outputp{x}{y}}{@{y}})} \nonumber\\
	\bangp_{x}{P} & := & \binpar{{x}!\langle{\binpar{D_{x}}{P}}\rangle}{D_{x}} \nonumber
\end{eqnarray}

\begin{eqnarray}
	\bangp_{x}{P} & & \nonumber\\
	=
	& {x}!\langle{(\prefix{x}{y}{(\outputp{x}{y} | @{y})) | P}}\rangle 
	      | \prefix{x}{y}{(\outputp{x}{y} | @{y})} & \nonumber\\
	\red
	& (\outputp{x}{y} | @{y})\substn{\quotep{(\prefix{x}{y}{(@{y} | \outputp{x}{y})) | P}}}{y} & \nonumber\\
	=
	& \outputp{x}{\quotep{(\prefix{x}{y}{(\outputp{x}{y} | @{y})) | P}}}
	  | {(\prefix{x}{y}{(\outputp{x}{y} | @{y})) | P}} & \nonumber\\
	\red
	& \ldots & \nonumber\\
	\red^*
	& P | P | \ldots & \nonumber
\end{eqnarray}

Of course, this encoding, as an implementation, runs away, unfolding
$\bangp{P}$ eagerly. A lazier and more implementable replication
operator, restricted to input-guarded processes, may be obtained as follows.

\begin{eqnarray}
\bangp{\prefix{u}{v}{P}} 
	:= 
	\binpar{\lift{x}{\prefix{u}{v}{(\binpar{D(x)}{P})}}}{D(x)} \nonumber
\end{eqnarray}

\begin{remark}
  Note that the lazier definition still does not deal with summation
  or mixed summation (i.e. sums over input and output). The reader is
  invited to construct definitions of replication that deal with these
  features. 

  Further, the definitions are parameterized in a name, $x$. Can you,
  gentle reader, make a definition that eliminates this parameter and
  guarantees no accidental interaction between the replication
  machinery and the process being replicated -- i.e. no accidental
  sharing of names used by the process to get its work done and the
  name(s) used by the replication to effect copying. This latter
  revision of the definition of replication is crucial to obtaining
  the expected identity $!!P \sim !P$.
\end{remark}

\begin{remark}\label{rem:paradoxical_combinator}
  The reader familiar with the lambda calculus will have noticed the
  similarity between $D$ and the paradoxical combinator.

  [Ed. note: the existence of this seems to suggest we have to be more
  restrictive on the set of processes and names we admit if we are to
  support no-cloning.]
\end{remark}

\subsubsection{Bisimulation}

The computational dynamics gives rise to another kind of equivalence,
the equivalence of computational behavior. As previously mentioned
this is typically captured \emph{via} some form of bisimulation.

% The notion we use in this paper is weak barbed bisimulation
% \cite{milner91polyadicpi}.

The notion we use in this paper is derived from weak barbed
bisimulation \cite{milner91polyadicpi}. 

\begin{definition}
An \emph{observation relation}, $\downarrow_{\mathcal N}$, over a set
of names, $\mathcal N$, is the smallest relation satisfying the rules
below.

\infrule[Out-barb]{y \in {\mathcal N}, \; x \nameeq y}
		  {\outputp{x}{v} \downarrow_{\mathcal N} x}
\infrule[Par-barb]{\mbox{$P\downarrow_{\mathcal N} x$ or $Q\downarrow_{\mathcal N} x$}}
		  {\binpar{P}{Q} \downarrow_{\mathcal N} x}

We write $P \Downarrow_{\mathcal N} x$ if there is $Q$ such that 
$P \wred Q$ and $Q \downarrow_{\mathcal N} x$.
\end{definition}

\begin{definition}
%\label{def.bbisim}
An  ${\mathcal N}$-\emph{barbed bisimulation} over a set of names, ${\mathcal N}$, is a symmetric binary relation 
${\mathcal S}_{\mathcal N}$ between agents such that $P\rel{S}_{\mathcal N}Q$ implies:
\begin{enumerate}
\item If $P \red P'$ then $Q \wred Q'$ and $P'\rel{S}_{\mathcal N} Q'$.
\item If $P\downarrow_{\mathcal N} x$, then $Q\Downarrow_{\mathcal N} x$.
\end{enumerate}
$P$ is ${\mathcal N}$-barbed bisimilar to $Q$, written
$P \wbbisim_{\mathcal N} Q$, if $P \rel{S}_{\mathcal N} Q$ for some ${\mathcal N}$-barbed bisimulation ${\mathcal S}_{\mathcal N}$.
\end{definition}

$\mathcal{R} \subseteq \pi \times \pi$

$P \mathcal{R} Q => \forall P'. P \red P' \Rightarrow \exists Q'. Q \red Q', P' \mathcal{R} Q'$

$P \vdash x \Rightarrow Q \vdash x$

\begin{mathpar}
  \inferrule*[lab=Out-barb]{x \nameeq y}{{y}!\langle{Q}\rangle \vdash x}
  \and
  \inferrule*[lab=Par-barb]{\mbox{$P\vdash x$ or $Q\vdash x$}}{\binpar{P}{Q} \vdash x}
\end{mathpar}

\subsubsection{Contexts}

One of the principle advantages of computational calculi like the
$\pi$-calculus is a well-defined notion of context,
contextual-equivalence and a correlation between
contextual-equivalence and notions of bisimulation. The notion of
context allows the decomposition of a process into (sub-)process and
its syntactic environment, its context. Thus, a context may be
thought of as a process with a ``hole'' (written $\Box$) in it. The
application of a context $M$ to a process $P$, written $M[P]$, is
tantamount to filling the hole in $M$ with $P$. In this paper we do
not need the full weight of this theory, but do make use of the notion
of context in the proof the main theorem. 

\begin{mathpar}
  \inferrule* [lab=summation] {} {{M_{M},M_{N}} \bc \Box \;|\; x.M_{A} \;|\; M_{M}+M_{N}}
  \and
  \inferrule* [lab=agent] {} {{M_{A}} \bc (\vec{x})M_{P} \;| \; \clift{P_0,\ldots,M_{P},\ldots,P_N}}
  \and \\
  \inferrule* [lab=process] {} {{M_{P}} \bc M_{N} \;| \;P|M_{P} }
\end{mathpar} 

\begin{mathpar}
  \inferrule* [lab=sychronization] {} {M_{N} \bc \Box \;|\; x?M_{F} \;|\; x!M_{C}}
  \and
  \inferrule* [lab=abstraction] {} {{M_{F}} \bc (x)M_{P} }
  \and
  \inferrule* [lab=concretion] {} {{M_{C}} \bc \langle M_{P} \rangle }
  \and \\
  \inferrule* [lab=process] {} {{M_{P}} \bc M_{N} \;| \;P|M_{P} }
\end{mathpar}

\begin{definition}[contextual application] Given a context $M$, and
  process $P$, we define the \emph{contextual application}, $M[P] :=
  M\{P/\Box\}$. That is, the contextual application of M to P is the
  substitution of $P$ for $\Box$ in $M$.
\end{definition}

$\meaningof{-} : L \to \mathcal{P}(\pi)$

\begin{mathpar}
  \inferrule* [lab=collection] {} {\meaningof{true} = \pi, \and \meaningof{~E} = \pi \setminus \meaningof{E}, \and \meaningof{E_{1} \& E_{2}} = \meaningof{E_{1}} \cap \meaningof{E_{2}}}
\end{mathpar}

\begin{mathpar}
  \inferrule* [lab=structure] {} {\meaningof{0} = \{ P \in \pi | P \equiv 0 \}, \and \\ \meaningof{E_1 | E_2} = \{ P \in \pi | P \equiv P_{1} | P_{2}, P_{1} \in \meaningof{E_{1}}, P_{2} \in \meaningof{E_2}\} }
\end{mathpar}

\begin{mathpar}
 \inferrule* [lab=behavior] {} {\meaningof{\langle a?b \rangle E} = \{ P \in \pi | P \equiv Q | u?(y)P', \\ \and \\\\ \and \\ \;\;\; u \in \meaningof{a}, \forall z.P'\{z/y\} \in \meaningof{E\{z/b\}}\}, \and \\ \meaningof{a!E} = \{ P \in \pi | P \equiv Q | x!\langle P' \rangle, x \in \meaningof{a} P' \in \meaningof{E}\} }
\end{mathpar}

\begin{mathpar}
 \inferrule* [lab=nominal] {} {\meaningof{\quotep{E}} = \{ \quotep{P} \in \quotep{\pi} | P \in \meaningof{E} \}, \and \meaningof{\quotep{P}} = \{ \quotep{Q} \in \quotep{\pi} | P \equiv Q \} \and \\ \meaningof{@\quotep{E}} = \{ P \in \pi | P \equiv @x, x \in \meaningof{E} \}}
\end{mathpar}

\begin{eqnarray*}
  \\
  \meaningof{-} : TS \to ST
\end{eqnarray*}

\begin{eqnarray*}
  \\
  L : TS \to ST
\end{eqnarray*}

\begin{eqnarray*}
  \\
  P \models E \iff P \in \meaningof{E}
\end{eqnarray*}

\begin{eqnarray*}
  P \approx_{L} Q \iff \forall E \in L. P \models E \iff Q \models E
\end{eqnarray*}

\begin{eqnarray*}
  P \approx_{K} Q
\end{eqnarray*}

\begin{eqnarray*}
  P \approx Q
\end{eqnarray*}

$\approx_{K} = \approx = \approx_{L}$

\subsubsection{Contextual duality}

Note that contexts extend the quotation operation to a family of
operations from processes to names. Given a context, $M$, we can
define a \emph{nominal context}, $\quotep{M}$ by $\quotep{M}[P] :=
\quotep{M[P]}$. To foreshadow what is to come we observe that these
operations enjoy a duality with processes very much like the duality
between vectors and maps from vectors to scalars.

Further, because the calculus is essentially higher-order, we have a
correspondence between contexts and processes. More specifically,
given a name $x$ and a context $M$ we can construct $M^{*}_{x}$ such
that 

\begin{mathpar}
  M^{*}_{x} | \lift{x}{P} \red M[P]
\end{mathpar}

namely,

\begin{mathpar}
  M^{*}_{x} := x?(u).M[\dropn{u}]
\end{mathpar}

The dependence of $M^{*}_{x}$ on a name makes it an abstraction, 

\begin{mathpar}
  M^{*} := (x)x?(u).M[\dropn{u}]
\end{mathpar}

\subsection{Additional notation}

It will sometimes be convenient to denote the process a name
quotes. We already have the notation $x = \quotep{P}$, but it will be
convenient to introduce an alternate notation, $\procn{x}$, when we
want to emphasize the connection to the use of the name. Note that, by
virtue of name equivalence, $\quotep{\procn{x}} \nameeq x$; so, the
notation is consistent with previous definitions.

Further, because names have structure it is possible to effect
substitutions on the basis of that structure. This means we need to
upgrade our notation for substitutions, which we accomplish by
adapting comprehension notation. Thus,

\begin{mathpar}
  P\{ y / x : x \in S \}
\end{mathpar}

is interpreted to mean the process derived from P by replacing (in a
capture-avoiding manner) each occurrence of $x$ in $S$ by $y$. For example,

\begin{mathpar}
  P\{ \quotep{\procn{x}|\procn{x}} / x : x \in \freenames{P} \}
\end{mathpar}

will replace each (occurrence) of a free name $x$ in $P$ by
$\quotep{\procn{x}|\procn{x}}$.

Also, we will avail ourselves of the notation $x^{L}$ and $x^{R}$ to
denote injections of a name into disjoint copies of the name
space. There are numerous ways to accomplish this. One example can be
found in \cite{MeredithR05}. This notation overloads to vectors of
names: $\vec{x}^{\pi} := (x_{i}^{\pi} \; : \; 0 \leq i < |\vec{x}| )$ where $\pi \in \{L,R\}$.

We also use $P^{\Box} := P|\Box$.

In \cite{MeredithR05} an interpretation of the new operator is
given. It turns out that there are several possible interpretations
all enjoying the requisite algebraic properties of the operator (see
\cite{milner91polyadicpi}). We will therefore make liberal use of
$(\nu\; \vec{x})P$.

% subsection the_syntax_and_semantics_of_the_notation_system (end)   

\input{qm2pi.qmops} 

\input{qm2pi.sterngerlach} 

\input{qm2pi.metric} 

% section concurrent_process_calculi (end)

%\input{qm2pi.proofsketch}

% section proof sketch (end)

%\input{qm2pi.slviaknots} 

% section spatial logic via knots (end)

\input{qm2pi.conclusion}

% section conclusion (end)

%\input{qm2pi.dtcodes} 

% section wiring algorithm (end)

\input{qm2pi.ack} 

% section acknowledgments (end)

\newpage


\bibliographystyle{plain}   
\bibliography{../../biblios/main.bib}

\input{qm2pi.rhodetails}

\end{document}

 

% section concurrent_process_calculi (end)

%\documentclass[12pt]{llncs}
%\documentclass{jktr}

\usepackage[pdftex]{hyperref}                   
\usepackage {listings}
\usepackage {mathpartir}
\usepackage{bcprules}
%\usepackage{listings}
                       
\usepackage{graphicx} 
%\usepackage[margins=2.5cm,nohead,nofoot]{geometry}
%\usepackage{geometry}
\usepackage{amsfonts}
\usepackage{amstext}
\usepackage{latexsym}
\usepackage{amssymb}
\usepackage{color}


%\include{myPreamble}
\include{qm2pi.local} 

%\ifpdf
%\usepackage[pdftex]{graphicx}
%\else
%\usepackage{graphicx}
%\fi

 % \ifpdf
%  \usepackage{pdfsync}
%  \if


%\title{Brief Article}
%\author{David F. Snyder}
%\author{L.G. Meredith}

%\address{Dept. of Math., Texas State University--San Marcos, San Marcos, TX 78666}
       
\pagestyle{empty}


\begin{document}

\lstset{language=[Objective]Caml,frame=shadowbox}

\input{qm2pi.front}

% section front matter (end)

\input{qm2pi.intro} 
 
% section introduction (end)

% \input{qm2pi.knotations} 

% section notation (end)

\input{qm2pi.process.calculi} 

% section concurrent_process_calculi_and_spatial_logics_ (end)
    
%\input{qm2pi.knots2pi} 

%\input{qm2pi.trefoil} 

%\input{qm2pi.mainthm} 

% subsection basic_interpretation (end)

%\input{qm2pi.rho.presentation} 
\subsection{The syntax and semantics of the notation system}\label{sub:the_syntax_and_semantics_of_the_notation_system} % (fold)

We now summarize a technical presentation of the calculus that
embodies our theory of dynamics. The typical presentation of such a
calculus follows the style of giving generators and relations on
them. The grammar, below, describing term constructors, freely
generates the set of processes, $\Proc$. This set is then quotiented
by a relation known as structural congruence and it is over this set
that the notion of dynamics is expressed. This presentation is
essentially that of \cite{MeredithR05} with the addition of
polyadicity and summation. For readability we have relegated some of
the technical subtleties to an appendix.

\subsubsection{Process grammar}\label{subsub:process_grammar}

\begin{mathpar}
  \inferrule* [lab=synchronization] {} {{M} \bc \pzero \;|\; x?F \;|\; x!C }
  \and
  \inferrule* [lab=abstraction] {} {{F} \bc (x)P}
  \and
  \inferrule* [lab=concretion] {} {{C} \bc \langle Q \rangle}
  \and
  \inferrule* [lab=process] {} {{P,Q} \bc M \;| \;P|Q \;|\; @{x}}
  \and
  \inferrule* [lab=name] {} {{x} \bc \quotep{P}}
\end{mathpar} 

Note that $\vec{x}$ (resp. $\vec{P}$) denotes a vector of names
(resp. processes) of length $|\vec{x}|$ (resp. $|\vec{P}|$). We adopt
the following useful abbreviations.

\begin{mathpar}
   x?(\vec{y}).P := x.(\vec{y})P \and  x\clift{\vec{P}} := x.\clift{\vec{P}}
   \and x!(y) := \lift{x}{\dropn{y}}
   \and \Pi_{i=0}^{n-1}P_i := P_0 | \ldots | P_{n-1}
\end{mathpar}

\subsubsection{Structural congruence}

\paragraph{Free and bound names and alpha-equivalence.} At the
core of structural equivalence is alpha-equivalence which identifies
process that are the same up to a change of variable. Formally, we
recognize the distinction between free and bound names. The free names
of a process, $\freenames{P}$, may be calculated recursively as
follows:

\begin{mathpar}
\freenames{\pzero} := \emptyset
  \and \\
  \freenames{x?(y).P} := \{ x \} \cup (\freenames{P} \setminus \{ y \})
  \and 
  \freenames{x!\langle P \rangle} := \{ x \} \cup \{ P \} 
  \and \\
  \freenames{P|Q} := \freenames{P} \cup \freenames{Q}
  \and \\
  \freenames{@{x}} := \{ x \}
\end{mathpar}

$\pi$
$\quotep{\pi}$

$\freenames{-} : \pi \to \mathcal{P}(\quotep{\pi})$

\begin{eqnarray*}
  \freenames{\pzero} & := & \emptyset \\
  \freenames{x?(y).P} & := & \{ x \} \cup (\freenames{P} \setminus \{ y \}) \\
  \freenames{x!\langle P \rangle} & := & \{ x \} \cup \{ P \} \\
  \freenames{P|Q} & := & \freenames{P} \cup \freenames{Q} \\
  \freenames{\dropn{x}} & := & \{ x \}
\end{eqnarray*}

The bound names of a process, $\boundnames{P}$, are those names occurring in $P$
that are not free. For example, in $x?(y).0$, the name $x$ is free, while $y$ is bound.

\begin{mathpar}
  \inferrule* [lab=monoidal-laws] {} { P|Q \equiv Q|P \and P|0 \equiv P \and P|(Q|R) \equiv (P|Q)|R }
\end{mathpar}

\begin{mathpar}
  \inferrule* [lab=alpha-equivalence] {} { (x)P \equiv (y)P\{y/x\} \and y \not\in \freenames{P} }
\end{mathpar}

\begin{definition}
Then two processes, $P,Q$, are alpha-equivalent if $P = Q\{\vec{y}/\vec{x}\}$ for
some $\vec{x} \in \boundnames{Q},\vec{y} \in \boundnames{P}$, where $Q\{\vec{y}/\vec{x}\}$
denotes the capture-avoiding substitution of $\vec{y}$ for $\vec{x}$ in $Q$.
\end{definition}

\begin{definition}
  The {\em structural congruence} \cite{SangiorgiWalker} , $\equiv$,
  between processes is the least congruence containing
  alpha-equivalence, satisfying the abelian monoid laws
  (associativity, commutativity and $\pzero$ as identity) for parallel
  composition $|$ and for summation $+$.
\end{definition}

\subsection{Name equivalence}

We take name equivalence, written $\nameeq$, to be the smallest
equivalence relation generated by the following rules.

\begin{mathpar}
\inferrule*[lab=Quote-drop]
{ }
{ \quotep{@{x}} \nameeq x }

\inferrule*[lab=Struct-equiv]
{ P \scong Q }
{ \quotep{P} \nameeq \quotep{Q} }
\end{mathpar}

The astute reader will have noticed that the mutual recursion of names
and processes imposes a mutual recursion on alpha-equivalence and
structural equivalence via name-equivalence. Fortunately, all of this
works out pleasantly and we may calculate in the natural way, free of
concern. The reader interested in the details is referred to the
appendix \ref{appendix:rho_details}.

\subsection{Substitution}

We use $\Proc$ for the set of processes, $\QProc$ for the set of
names, and $\id{\{}\vec{y} / \vec{x} \id{\}}$ to denote partial maps,
$s : \QProc \rightarrow \QProc$. A map, $s$ lifts, uniquely, to a map
on process terms, $\widehat{s} : \Proc \rightarrow \Proc$ by the
following equations.

\begin{mathpar}
  (0) \psubstp{Q}{P} := 0 \\
  (R \juxtap S) \psubstp{Q}{P}
  :=    
  (R)\psubstp{Q}{P} \juxtap (S) \psubstp{Q}{P} \\
  (x?(y).R) \psubstp{Q}{P}    
  :=    
  (x)\substp{Q}{P} (z)\concat( (R \psubstn{z}{y}) \psubstp{Q}{P} ) \\
  (\lift{x}{R}) \psubstp{Q}{P}  
  :=
  \lift{(x)\substp{Q}{P}}{ R \psubstp{Q}{P} } \\
%   (\dropn{x})  \psubstp{Q}{P}       
%   := 
%   \left\{ 
%     \begin{array}{ccc} 
%       \dropn{\quotep{Q}} & & x \nameeq \quotep{P} \\
%       \dropn{x} & & otherwise \\
%     \end{array}
%   \right. 
  (\dropn{x})  \psubstp{Q}{P}       
  := 
  \left\{ 
    \begin{array}{ccc} 
      Q & & x \nameeq \quotep{P} \\
      \dropn{x} & & otherwise \\
    \end{array}
  \right.
\end{mathpar}
 

where

\begin{eqnarray}
  (x)\id{\{} \lpquote Q \rpquote / \lpquote P \rpquote \id{\}}            = 
  \left\{ 
    \begin{array}{ccc}
      \lpquote Q \rpquote & & x \nameeq \lpquote P \rpquote \\
      x & & otherwise \\
    \end{array}
  \right. \nonumber
\end{eqnarray}

and $z$ is chosen distinct from $\quotep{P}$, $\quotep{Q}$, the free
names in $Q$, and all the names in $R$. Our $\alpha$-equivalence will
be built in the standard way from this substitution.

\begin{remark}\label{rem:no_self_referential_names}
  One consequence of these definitions is that $\forall P. \quotep{P}
  \not\in \freenames{P}$.
\end{remark}

\subsection{ Dynamic quote: an example }

Anticipating something of what's to come, consider applying the
substitution, $\widehat{\id{\{}u / z \id{\}}}$, to the following pair
of processes, $\lift{w}{y!(z)}$ and $w[ \lpquote y!(z) \rpquote ]$.

\begin{eqnarray}
	\lift{w}{y!(z)}\widehat{\id{\{}u / z \id{\}}}
		& = &
		\lift{w}{y!(u)} \nonumber\\
	w[ \lpquote y!(z) \rpquote ] \widehat{ \id{\{}u / z \id{\}} }
		& = &
		w[ \lpquote y!(z) \rpquote ] \nonumber
\end{eqnarray}

Because the body of the process between quotes is impervious to
substitution, we get radically different answers. In fact, by
examining the first process in an input context,
e.g. $x?(z).\lift{w}{y!(z)}$, we see that the process under the lift
operator may be shaped by prefixed inputs binding a name inside it. In
this sense, the lift operator will be seen as a way to dynamically
construct processes before reifying them as names.

Finally equipped with these standard features we can present the
dynamics of the calculus.

\subsubsection{Operational semantics} 

Finally, we introduce the computational dynamics. What marks these
algebras as distinct from other more traditionally studied algebraic
structures, e.g. vector spaces or polynomial rings, is the manner in
which dynamics is captured. In traditional structures, dynamics is typically
expressed through morphisms between such structures, as in linear maps
between vector spaces or morphisms between rings. In algebras
associated with the semantics of computation, the dynamics is
expressed as part of the algebraic structure itself, through a
reduction reduction relation typically denoted by $\red$. Below, we
give a recursive presentation of this relation for the calculus used
in the encoding.

$\red \subseteq \pi \times \pi$
$\red : \pi \to \mathcal{P}(\pi)$

\begin{mathpar}
  \inferrule* [lab=Comm] { \textsf{match}( x_{src}, x_{trgt} ) } { x_{trgt}?(y)P \; | \; x_{src}!\langle {Q} \rangle \red P\{\quotep{Q}/y}\} }
  \and \\
  \inferrule* [lab=Par] {{P} \red {P}'} {{{P} | {Q}} \red {{P}' | {Q}}}
  \and
  \inferrule* [lab=Equiv]{{{P} \scong {P}'} \andalso {{P}' \red {Q}'} \andalso {{Q}' \scong {Q}}}{{P} \red {Q}}
\end{mathpar}

\begin{eqnarray*}
  match_{\equiv} (\quotep{P},\quotep{Q}) & := & P \equiv Q \\
  match_{\dagger}(\quotep{P},\quotep{Q}) & := & \forall R. P|Q \red^{*} R => R \red^{*} 0 \\
  match_{K}(\quotep{P},\quotep{Q}) & := & K \mbox{ for some context } K
\end{eqnarray*}

$u?(x)P | u!\langle Q \rangle \red P\{\quotep{Q}/x\}$

%We write $\wred$ for $\red^*$, and $P\red$ if $\exists Q $ such that $ P \red Q$.
We write $P\red$ if $\exists Q $ such that $ P \red Q$ and $P\not\red$, otherwise.

\section{Replication}

As mentioned before, it is known that replication (and hence
recursion) can be implemented in a higher-order process algebra
\cite{SangiorgiWalker}. As our first example of calculation with the
machinery thus far presented we give the construction explicitly in
the {\rhoc}.

\begin{eqnarray}
	D_{x} & := & \prefix{x}{y}{(\binpar{\outputp{x}{y}}{@{y}})} \nonumber\\
	\bangp_{x}{P} & := & \binpar{{x}!\langle{\binpar{D_{x}}{P}}\rangle}{D_{x}} \nonumber
\end{eqnarray}

\begin{eqnarray}
	\bangp_{x}{P} & & \nonumber\\
	=
	& {x}!\langle{(\prefix{x}{y}{(\outputp{x}{y} | @{y})) | P}}\rangle 
	      | \prefix{x}{y}{(\outputp{x}{y} | @{y})} & \nonumber\\
	\red
	& (\outputp{x}{y} | @{y})\substn{\quotep{(\prefix{x}{y}{(@{y} | \outputp{x}{y})) | P}}}{y} & \nonumber\\
	=
	& \outputp{x}{\quotep{(\prefix{x}{y}{(\outputp{x}{y} | @{y})) | P}}}
	  | {(\prefix{x}{y}{(\outputp{x}{y} | @{y})) | P}} & \nonumber\\
	\red
	& \ldots & \nonumber\\
	\red^*
	& P | P | \ldots & \nonumber
\end{eqnarray}

Of course, this encoding, as an implementation, runs away, unfolding
$\bangp{P}$ eagerly. A lazier and more implementable replication
operator, restricted to input-guarded processes, may be obtained as follows.

\begin{eqnarray}
\bangp{\prefix{u}{v}{P}} 
	:= 
	\binpar{\lift{x}{\prefix{u}{v}{(\binpar{D(x)}{P})}}}{D(x)} \nonumber
\end{eqnarray}

\begin{remark}
  Note that the lazier definition still does not deal with summation
  or mixed summation (i.e. sums over input and output). The reader is
  invited to construct definitions of replication that deal with these
  features. 

  Further, the definitions are parameterized in a name, $x$. Can you,
  gentle reader, make a definition that eliminates this parameter and
  guarantees no accidental interaction between the replication
  machinery and the process being replicated -- i.e. no accidental
  sharing of names used by the process to get its work done and the
  name(s) used by the replication to effect copying. This latter
  revision of the definition of replication is crucial to obtaining
  the expected identity $!!P \sim !P$.
\end{remark}

\begin{remark}\label{rem:paradoxical_combinator}
  The reader familiar with the lambda calculus will have noticed the
  similarity between $D$ and the paradoxical combinator.

  [Ed. note: the existence of this seems to suggest we have to be more
  restrictive on the set of processes and names we admit if we are to
  support no-cloning.]
\end{remark}

\subsubsection{Bisimulation}

The computational dynamics gives rise to another kind of equivalence,
the equivalence of computational behavior. As previously mentioned
this is typically captured \emph{via} some form of bisimulation.

% The notion we use in this paper is weak barbed bisimulation
% \cite{milner91polyadicpi}.

The notion we use in this paper is derived from weak barbed
bisimulation \cite{milner91polyadicpi}. 

\begin{definition}
An \emph{observation relation}, $\downarrow_{\mathcal N}$, over a set
of names, $\mathcal N$, is the smallest relation satisfying the rules
below.

\infrule[Out-barb]{y \in {\mathcal N}, \; x \nameeq y}
		  {\outputp{x}{v} \downarrow_{\mathcal N} x}
\infrule[Par-barb]{\mbox{$P\downarrow_{\mathcal N} x$ or $Q\downarrow_{\mathcal N} x$}}
		  {\binpar{P}{Q} \downarrow_{\mathcal N} x}

We write $P \Downarrow_{\mathcal N} x$ if there is $Q$ such that 
$P \wred Q$ and $Q \downarrow_{\mathcal N} x$.
\end{definition}

\begin{definition}
%\label{def.bbisim}
An  ${\mathcal N}$-\emph{barbed bisimulation} over a set of names, ${\mathcal N}$, is a symmetric binary relation 
${\mathcal S}_{\mathcal N}$ between agents such that $P\rel{S}_{\mathcal N}Q$ implies:
\begin{enumerate}
\item If $P \red P'$ then $Q \wred Q'$ and $P'\rel{S}_{\mathcal N} Q'$.
\item If $P\downarrow_{\mathcal N} x$, then $Q\Downarrow_{\mathcal N} x$.
\end{enumerate}
$P$ is ${\mathcal N}$-barbed bisimilar to $Q$, written
$P \wbbisim_{\mathcal N} Q$, if $P \rel{S}_{\mathcal N} Q$ for some ${\mathcal N}$-barbed bisimulation ${\mathcal S}_{\mathcal N}$.
\end{definition}

$\mathcal{R} \subseteq \pi \times \pi$

$P \mathcal{R} Q => \forall P'. P \red P' \Rightarrow \exists Q'. Q \red Q', P' \mathcal{R} Q'$

$P \vdash x \Rightarrow Q \vdash x$

\begin{mathpar}
  \inferrule*[lab=Out-barb]{x \nameeq y}{{y}!\langle{Q}\rangle \vdash x}
  \and
  \inferrule*[lab=Par-barb]{\mbox{$P\vdash x$ or $Q\vdash x$}}{\binpar{P}{Q} \vdash x}
\end{mathpar}

\subsubsection{Contexts}

One of the principle advantages of computational calculi like the
$\pi$-calculus is a well-defined notion of context,
contextual-equivalence and a correlation between
contextual-equivalence and notions of bisimulation. The notion of
context allows the decomposition of a process into (sub-)process and
its syntactic environment, its context. Thus, a context may be
thought of as a process with a ``hole'' (written $\Box$) in it. The
application of a context $M$ to a process $P$, written $M[P]$, is
tantamount to filling the hole in $M$ with $P$. In this paper we do
not need the full weight of this theory, but do make use of the notion
of context in the proof the main theorem. 

\begin{mathpar}
  \inferrule* [lab=summation] {} {{M_{M},M_{N}} \bc \Box \;|\; x.M_{A} \;|\; M_{M}+M_{N}}
  \and
  \inferrule* [lab=agent] {} {{M_{A}} \bc (\vec{x})M_{P} \;| \; \clift{P_0,\ldots,M_{P},\ldots,P_N}}
  \and \\
  \inferrule* [lab=process] {} {{M_{P}} \bc M_{N} \;| \;P|M_{P} }
\end{mathpar} 

\begin{mathpar}
  \inferrule* [lab=sychronization] {} {M_{N} \bc \Box \;|\; x?M_{F} \;|\; x!M_{C}}
  \and
  \inferrule* [lab=abstraction] {} {{M_{F}} \bc (x)M_{P} }
  \and
  \inferrule* [lab=concretion] {} {{M_{C}} \bc \langle M_{P} \rangle }
  \and \\
  \inferrule* [lab=process] {} {{M_{P}} \bc M_{N} \;| \;P|M_{P} }
\end{mathpar}

\begin{definition}[contextual application] Given a context $M$, and
  process $P$, we define the \emph{contextual application}, $M[P] :=
  M\{P/\Box\}$. That is, the contextual application of M to P is the
  substitution of $P$ for $\Box$ in $M$.
\end{definition}

$\meaningof{-} : L \to \mathcal{P}(\pi)$

\begin{mathpar}
  \inferrule* [lab=collection] {} {\meaningof{true} = \pi, \and \meaningof{~E} = \pi \setminus \meaningof{E}, \and \meaningof{E_{1} \& E_{2}} = \meaningof{E_{1}} \cap \meaningof{E_{2}}}
\end{mathpar}

\begin{mathpar}
  \inferrule* [lab=structure] {} {\meaningof{0} = \{ P \in \pi | P \equiv 0 \}, \and \\ \meaningof{E_1 | E_2} = \{ P \in \pi | P \equiv P_{1} | P_{2}, P_{1} \in \meaningof{E_{1}}, P_{2} \in \meaningof{E_2}\} }
\end{mathpar}

\begin{mathpar}
 \inferrule* [lab=behavior] {} {\meaningof{\langle a?b \rangle E} = \{ P \in \pi | P \equiv Q | u?(y)P', \\ \and \\\\ \and \\ \;\;\; u \in \meaningof{a}, \forall z.P'\{z/y\} \in \meaningof{E\{z/b\}}\}, \and \\ \meaningof{a!E} = \{ P \in \pi | P \equiv Q | x!\langle P' \rangle, x \in \meaningof{a} P' \in \meaningof{E}\} }
\end{mathpar}

\begin{mathpar}
 \inferrule* [lab=nominal] {} {\meaningof{\quotep{E}} = \{ \quotep{P} \in \quotep{\pi} | P \in \meaningof{E} \}, \and \meaningof{\quotep{P}} = \{ \quotep{Q} \in \quotep{\pi} | P \equiv Q \} \and \\ \meaningof{@\quotep{E}} = \{ P \in \pi | P \equiv @x, x \in \meaningof{E} \}}
\end{mathpar}

\begin{eqnarray*}
  \\
  \meaningof{-} : TS \to ST
\end{eqnarray*}

\begin{eqnarray*}
  \\
  L : TS \to ST
\end{eqnarray*}

\begin{eqnarray*}
  \\
  P \models E \iff P \in \meaningof{E}
\end{eqnarray*}

\begin{eqnarray*}
  P \approx_{L} Q \iff \forall E \in L. P \models E \iff Q \models E
\end{eqnarray*}

\begin{eqnarray*}
  P \approx_{K} Q
\end{eqnarray*}

\begin{eqnarray*}
  P \approx Q
\end{eqnarray*}

$\approx_{K} = \approx = \approx_{L}$

\subsubsection{Contextual duality}

Note that contexts extend the quotation operation to a family of
operations from processes to names. Given a context, $M$, we can
define a \emph{nominal context}, $\quotep{M}$ by $\quotep{M}[P] :=
\quotep{M[P]}$. To foreshadow what is to come we observe that these
operations enjoy a duality with processes very much like the duality
between vectors and maps from vectors to scalars.

Further, because the calculus is essentially higher-order, we have a
correspondence between contexts and processes. More specifically,
given a name $x$ and a context $M$ we can construct $M^{*}_{x}$ such
that 

\begin{mathpar}
  M^{*}_{x} | \lift{x}{P} \red M[P]
\end{mathpar}

namely,

\begin{mathpar}
  M^{*}_{x} := x?(u).M[\dropn{u}]
\end{mathpar}

The dependence of $M^{*}_{x}$ on a name makes it an abstraction, 

\begin{mathpar}
  M^{*} := (x)x?(u).M[\dropn{u}]
\end{mathpar}

\subsection{Additional notation}

It will sometimes be convenient to denote the process a name
quotes. We already have the notation $x = \quotep{P}$, but it will be
convenient to introduce an alternate notation, $\procn{x}$, when we
want to emphasize the connection to the use of the name. Note that, by
virtue of name equivalence, $\quotep{\procn{x}} \nameeq x$; so, the
notation is consistent with previous definitions.

Further, because names have structure it is possible to effect
substitutions on the basis of that structure. This means we need to
upgrade our notation for substitutions, which we accomplish by
adapting comprehension notation. Thus,

\begin{mathpar}
  P\{ y / x : x \in S \}
\end{mathpar}

is interpreted to mean the process derived from P by replacing (in a
capture-avoiding manner) each occurrence of $x$ in $S$ by $y$. For example,

\begin{mathpar}
  P\{ \quotep{\procn{x}|\procn{x}} / x : x \in \freenames{P} \}
\end{mathpar}

will replace each (occurrence) of a free name $x$ in $P$ by
$\quotep{\procn{x}|\procn{x}}$.

Also, we will avail ourselves of the notation $x^{L}$ and $x^{R}$ to
denote injections of a name into disjoint copies of the name
space. There are numerous ways to accomplish this. One example can be
found in \cite{MeredithR05}. This notation overloads to vectors of
names: $\vec{x}^{\pi} := (x_{i}^{\pi} \; : \; 0 \leq i < |\vec{x}| )$ where $\pi \in \{L,R\}$.

We also use $P^{\Box} := P|\Box$.

In \cite{MeredithR05} an interpretation of the new operator is
given. It turns out that there are several possible interpretations
all enjoying the requisite algebraic properties of the operator (see
\cite{milner91polyadicpi}). We will therefore make liberal use of
$(\nu\; \vec{x})P$.

% subsection the_syntax_and_semantics_of_the_notation_system (end)   

\input{qm2pi.qmops} 

\input{qm2pi.sterngerlach} 

\input{qm2pi.metric} 

% section concurrent_process_calculi (end)

%\input{qm2pi.proofsketch}

% section proof sketch (end)

%\input{qm2pi.slviaknots} 

% section spatial logic via knots (end)

\input{qm2pi.conclusion}

% section conclusion (end)

%\input{qm2pi.dtcodes} 

% section wiring algorithm (end)

\input{qm2pi.ack} 

% section acknowledgments (end)

\newpage


\bibliographystyle{plain}   
\bibliography{../../biblios/main.bib}

\input{qm2pi.rhodetails}

\end{document}



% section proof sketch (end)

%\section{Unlikely characters: spatial logic for
  knots}\label{sub:characteristic_formulae} % (fold)

Associated to the mobile process calculi are a family of logics known
as the Hennessy-Milner logics. These logics typically enjoy a
semantics interpreting formulae as sets of processes that when
factored through the encoding outlined above allows an identification
of classes of knots with logical formulae. In the context of this
encoding the sub-family known as the spatial logics \cite{CairesC03}
\cite{CairesC04} \cite{Caires04} are of particular interest providing
several important features for expressing and reasoning about
properties (i.e. classes) of knots. We hint here at how this may be done.

%\begin{description}
%\item [structural connectives] 
\subsubsection{Structural connectives} The spatial logics enjoy
structural connectives corresponding, at the logical level, to the
parallel composition ($P | Q$) and new name ($(\nu \; x)P$)
connectives for processes. As illustrated in the examples below, these
connectives are extremely expressive given the shape of our encoding.
%\item [decideable satisfaction]

\subsubsection{Decideable satisfaction}
In \cite{Caires04} the satisfaction relation is shown to be decideable
for a rich class of processes. It further turns out that the image of
the our encoding is a proper subset of that class. This result
provides the basis for an algorithm by which to search for knots
enjoying a given property.
%\item [characteristic formulae]

\subsubsection{Characteristic formulae}
In the same paper \cite{Caires04} , Caires presents a means of calculating
characteristic formulae, selecting equivalence classes of processes
up to a pre--specified depth limit on the support set of names. Composed with our
encoding, this characteristic formula can be used to select
characteristic formulae for knots.
%\end{description}

\subsubsection{Spatial logic formulae}

The grammar below (segmented for comprehension) summarizes the syntax
of spatial logic formulae. We employ illustrative examples in the
sequel to provide an intuitive understanding of their meaning
referring the reader to \cite{Caires04} for a more detailed explication
of the semantics.

\begin{mathpar}
  \inferrule* [lab=boolean] {} {{A,B} \bc T \;|\; \neg A \;|\; A \wedge B \;|\; \eta = \eta'}
  \and
  \inferrule* [lab=spatial] {} {|\; \pzero \;|\; A | B \;|\; x \text{\textregistered} A \;|\; \forall x . A \;|\;  H x . A}
  \and
  \inferrule* [lab=behavioral] {} {|\; \alpha . A}
  \and 
  \inferrule* [lab=recursion] {} {|\; X(\vec{u}) \;|\; \mu X(\vec{u}) . A}
  \and
  \inferrule* [lab=action] {} {\alpha \bc \langle x?(\vec{y}) \rangle \;|\; \langle x!(\vec{y}) \rangle \;|\; \langle \tau \rangle}
  \and 
  \inferrule* [lab=name] {} {\eta \bc x \;|\; \tau}
\end{mathpar} 

% subsection characteristic_formulae (end)   	 

\subsection{Example formulae}\label{sub:example_formulae_} % (fold)

\subsubsection{Crossing as formula.}
% 
% \begin{align*}
%   \frac{d}{dx} \sin x &= \cos x 
%   & \frac{d}{dx} e^x &= e^x \\
%   \frac{d}{dx} \cos x &= - \sin x 
%   & \frac{d}{dx} \log x &= \frac{1}{x} \\
% \end{align*} 

\begin{align*}
 \mu C(x_{0},x_{1},y_{0},y_{1},u).&(\langle x_{0}?(z) \rangle(\langle u! \rangle\langle y_{1}!z \rangle C(x_{0},x_{1},y_{0},y_{1},u)) & \\
  & \wedge \langle y_{1}?(z) \rangle (\langle u! \rangle \langle x_{0}!z \rangle C(x_{0},x_{1},y_{0},y_{1},u)) & \\
  & \wedge \langle x_{1}?(z) \rangle (\langle u? \rangle \langle y_{0}!z \rangle C(x_{0},x_{1},y_{0},y_{1},u)) & \\
  & \wedge \langle y_{0}?(z) \rangle (\langle u? \rangle \langle x_{1}!z \rangle C(x_{0},x_{1},y_{0},y_{1},u))) &
\end{align*}

The lexicographical similarity between the shape of this formulae and
the shape of definition of the process representing a crossing reveals
the intuitive meaning of this formulae. It describes the capabilities
of a process that has the right to represent a crossing. For example
it picks out processes that may perform an input on the port $x_0$ in
its initial menu of capabilities. What differentiates the formula
from the process, however, is that the crossing process is the
smallest candidate to satisfy the formula. Infinitely many other
processes -- with internal behavior hidden behind this interface, so
to speak -- also satisfy this formula. Even this simple formula,
then, can be seen to open a new view onto knots, providing a
computational interpretation of \emph{virtual} knots.

Note that this formula is derived by hand. A similar formula can be
derived by employing Caires' calculation of characteristic formula
\cite{Caires04} to the process representing a crossing. In light of
this discussion, we let
$\meaningof{C}_{\phi}(x0,x1,y0,y1,u)$ denote a formula specifying the
dynamics we wish to capture of a crossing. To guarantee we preserve
the shape of the interface and minimal semantics we demand that
$\meaningof{C}_{\phi}(x0,x1,y0,y1,u) \Rightarrow
\textbf{C}(x0,x1,y0,y1,u)$ where $\textbf{C}(x0,x1,y0,y1,u)$ denotes
the formula above.
                            
\subsubsection{Crossing number constraints.}
The moral content of the context lemma (Lemma \ref{context}) is that the notion of
``locality'' in the Reidemeister moves is effectively captured by the
parallel composition operator of the process calculus. This intuition
extends through the logic. Given a formula,
$\meaningof{C}_{\phi}(x0,x1,y0,y1,u)$, we can use the structural
connectives to specify constraints on crossing numbers, such as at
least $n$ crossings, or exactly $n$ crossings.
\begin{mathpar}
  \inferrule* [lab=at-least-n] {} { K^{\geq n}_{\phi}(\vec{xs},\vec{ys}) := \Pi_{i=0}^{n-1} Hu . \meaningof{C}_{\phi}(xs_i,ys_i,u) | T }
  \and 
  \inferrule* [lab=exactly-n] {} { K^{= n}_{\phi}(\vec{xs},\vec{ys}) := \Pi_{i=0}^{n-1} Hu . \meaningof{C}_{\phi}(xs_i,ys_i,u) | \neg (\forall x_0,y_0,x_1,y_1,u . \meaningof{C}_{\phi}(x_0,y_0,x_1,y_1,u) | T) }
\end{mathpar}

To round out this section, recall that the encoding of an $n$-crossing
knot decomposes into a parallel composition of $n$ \emph{copies} of a
crossing process together with a wiring harness. To specify different
knot classes with the same crossing number amounts to specifying
logical constraints on the wiring harness. In the interest of space,
we defer examples to a forthcoming paper. Suffice it to say that both
the conditions ``alternating knot'' and ``contains the tangle
corresponding to 5/3'' are expressible. For example, it is possible to
calculate the characteristic formula of a process corresponding to the
tangle 5/3 and conjoin it into the classifying formula via the
composition connective of the logic.

Finally, we wish to observe that it is entirely within reason to
contemplate a more domain-specific version of spatial logic tailored
to the shape of processes in the image of the encoding. Such a
domain-specific logic would have a better claim to the title formal
language of knot properties.

% subsection example_formulae_ (end)

% section knots_as_processes (end) 

% section spatial logic via knots (end)

\section{Conclusions and future work}

\paragraph{Testing physical space}
You, gentle reader, may wonder why of all the theorems to be proved
given this set up we pick the one above. In some sense it's hardly
central to quantum mechanics. We see it as central in the sense that
it firmly establishes a notion of physical space arising from a notion
of the equivalence of behavior. Relating bisimulation to a metric is a
big step forward, but one is faced with interpreting the relationship
of that metric space to something more physical. Quantum mechanical
notions of ``physical'' space are still far from intuitive, but by
relating this idea of distance as testing to calculations that predict
physical circumstances we are making a not insignificant step forward
toward an understanding of the physical space we inhabit as
essentially dynamic.

\paragraph{Effectivity and simulation}
One of the observations we have yet to make is that the entire program
spelled out here is effective. We have built various interpreters for
the reflective calculus at work in this interpretation. In principle,
then, we can simulate quantum mechanics on a computer. The place where
the simulation may lose fidelity is the infinitely branching summation
for the annihilator.

In this connection i also want to point out that the evaluation style
calculation of the inner product puts the non-determinism of the
summation right at the heart of measurement. This suggests that
Milner's original reduction-based formulation of the dynamics of his
calculi in terms of sums was not just notationally suggestive of a
notion of measure-and-continue but captured some significant part of
the physics.

\paragraph{Quantum continuations}
In light of this last observation i want to point out that the
predominant account of quantum mechanics is missing a key aspect of a
truly compositional story of the physical situation. In a real lab,
when a measurement is made the observation can be made to feed into
another device that then makes another measurement conditioned on the
results of the first. This means that after the superposition was
collapsed the entire experimental set up remained in
superposition. While QM offers a means of writing this down it doesn't
quite line up well with the well-trodden formulation of computation
and continuation that we see so succinctly expressed in Milner's
calculi. This suggests that there might be advantages to this account
of dynamics waiting to be explored.

\paragraph{Quantum logic}
In this connection, we also note that by virtue of having the
Hennessy-Milner construction, we can pull the construction through the
interpretation of QM. This gives us a natural candidate for a quantum
logic that enjoys an extremely tight connection with it's domain of
interpretation, making the construction much less ad hoc (rather it is
the image of functor!).

\paragraph{Quantum probabiity}
i have questions about the basis of the interpretation of inner
product as probability amplitude. In particular, using which
axiomatization of probability theory does the notion of probability
amplitude earn the right to be so dubbed? In other words, where is the
proof that the operation for calculating a probability amplitude (and
then squaring) satisfies the axioms of what it means to calculate a
probability? Even if such a proof exists (i have yet to find it in the
literature), i wonder if it might not be possible to turn things on
their heads. Can we view the calculation of the probability amplitude
as an axiomatization of probability? If so, then the definition we
give for calculating probability amplitude may provide the basis for
an \emph{effective} theory of probability.

\paragraph{Quantum vs ``biological'' information}
Finally, i want to conclude with a more philosophical observation. At
a recent workshop in which QM was a predominant topic i noticed
something about quantum information. The speaker was giving a riveting
discussion of axiomatic QM and showing how properties of ``no
cloning'' and ``no deleting'' emerged as consequences of the
axiomatization. Theorems of this form are necessary to give us a sense
of confidence that our axioms characterize the physical theory. What
struck me, though, was that if quantum information is neither erasable
nor replicable it is markedly different from \emph{life}. Two of the
things we know about life is that

\begin{itemize}
  \item it ends;
  \item to gain some measure of persistence, to transcend it's
    finitude it is imminently copyable.
\end{itemize}

Both of these qualities are summarized succinctly in the aphorism: all
flesh is grass. For me these two kinds of ``information'' -- call them
quantum and biological -- are end points on a spectrum of strategies
for persistence. At one end, we have those curious entities that enjoy
uniqueness and permanence; at the other, we have those who in the face
of a certain end and an uncertain present make a go of passing
something on. To me one of the more remarkable aspects of the latter
strategy is that in the presence of noise (and certain features of
copying) we get a kind of dynamism, a chance for improvement against a
given persistent condition.

% subsection other_calculi_other_bisimulations_and_geometry_as_behavior (end)




% section conclusion (end)

%\documentclass[12pt]{llncs}
%\documentclass{jktr}

\usepackage[pdftex]{hyperref}                   
\usepackage {listings}
\usepackage {mathpartir}
\usepackage{bcprules}
%\usepackage{listings}
                       
\usepackage{graphicx} 
%\usepackage[margins=2.5cm,nohead,nofoot]{geometry}
%\usepackage{geometry}
\usepackage{amsfonts}
\usepackage{amstext}
\usepackage{latexsym}
\usepackage{amssymb}
\usepackage{color}


%\include{myPreamble}
\include{qm2pi.local} 

%\ifpdf
%\usepackage[pdftex]{graphicx}
%\else
%\usepackage{graphicx}
%\fi

 % \ifpdf
%  \usepackage{pdfsync}
%  \if


%\title{Brief Article}
%\author{David F. Snyder}
%\author{L.G. Meredith}

%\address{Dept. of Math., Texas State University--San Marcos, San Marcos, TX 78666}
       
\pagestyle{empty}


\begin{document}

\lstset{language=[Objective]Caml,frame=shadowbox}

\input{qm2pi.front}

% section front matter (end)

\input{qm2pi.intro} 
 
% section introduction (end)

% \input{qm2pi.knotations} 

% section notation (end)

\input{qm2pi.process.calculi} 

% section concurrent_process_calculi_and_spatial_logics_ (end)
    
%\input{qm2pi.knots2pi} 

%\input{qm2pi.trefoil} 

%\input{qm2pi.mainthm} 

% subsection basic_interpretation (end)

%\input{qm2pi.rho.presentation} 
\subsection{The syntax and semantics of the notation system}\label{sub:the_syntax_and_semantics_of_the_notation_system} % (fold)

We now summarize a technical presentation of the calculus that
embodies our theory of dynamics. The typical presentation of such a
calculus follows the style of giving generators and relations on
them. The grammar, below, describing term constructors, freely
generates the set of processes, $\Proc$. This set is then quotiented
by a relation known as structural congruence and it is over this set
that the notion of dynamics is expressed. This presentation is
essentially that of \cite{MeredithR05} with the addition of
polyadicity and summation. For readability we have relegated some of
the technical subtleties to an appendix.

\subsubsection{Process grammar}\label{subsub:process_grammar}

\begin{mathpar}
  \inferrule* [lab=synchronization] {} {{M} \bc \pzero \;|\; x?F \;|\; x!C }
  \and
  \inferrule* [lab=abstraction] {} {{F} \bc (x)P}
  \and
  \inferrule* [lab=concretion] {} {{C} \bc \langle Q \rangle}
  \and
  \inferrule* [lab=process] {} {{P,Q} \bc M \;| \;P|Q \;|\; @{x}}
  \and
  \inferrule* [lab=name] {} {{x} \bc \quotep{P}}
\end{mathpar} 

Note that $\vec{x}$ (resp. $\vec{P}$) denotes a vector of names
(resp. processes) of length $|\vec{x}|$ (resp. $|\vec{P}|$). We adopt
the following useful abbreviations.

\begin{mathpar}
   x?(\vec{y}).P := x.(\vec{y})P \and  x\clift{\vec{P}} := x.\clift{\vec{P}}
   \and x!(y) := \lift{x}{\dropn{y}}
   \and \Pi_{i=0}^{n-1}P_i := P_0 | \ldots | P_{n-1}
\end{mathpar}

\subsubsection{Structural congruence}

\paragraph{Free and bound names and alpha-equivalence.} At the
core of structural equivalence is alpha-equivalence which identifies
process that are the same up to a change of variable. Formally, we
recognize the distinction between free and bound names. The free names
of a process, $\freenames{P}$, may be calculated recursively as
follows:

\begin{mathpar}
\freenames{\pzero} := \emptyset
  \and \\
  \freenames{x?(y).P} := \{ x \} \cup (\freenames{P} \setminus \{ y \})
  \and 
  \freenames{x!\langle P \rangle} := \{ x \} \cup \{ P \} 
  \and \\
  \freenames{P|Q} := \freenames{P} \cup \freenames{Q}
  \and \\
  \freenames{@{x}} := \{ x \}
\end{mathpar}

$\pi$
$\quotep{\pi}$

$\freenames{-} : \pi \to \mathcal{P}(\quotep{\pi})$

\begin{eqnarray*}
  \freenames{\pzero} & := & \emptyset \\
  \freenames{x?(y).P} & := & \{ x \} \cup (\freenames{P} \setminus \{ y \}) \\
  \freenames{x!\langle P \rangle} & := & \{ x \} \cup \{ P \} \\
  \freenames{P|Q} & := & \freenames{P} \cup \freenames{Q} \\
  \freenames{\dropn{x}} & := & \{ x \}
\end{eqnarray*}

The bound names of a process, $\boundnames{P}$, are those names occurring in $P$
that are not free. For example, in $x?(y).0$, the name $x$ is free, while $y$ is bound.

\begin{mathpar}
  \inferrule* [lab=monoidal-laws] {} { P|Q \equiv Q|P \and P|0 \equiv P \and P|(Q|R) \equiv (P|Q)|R }
\end{mathpar}

\begin{mathpar}
  \inferrule* [lab=alpha-equivalence] {} { (x)P \equiv (y)P\{y/x\} \and y \not\in \freenames{P} }
\end{mathpar}

\begin{definition}
Then two processes, $P,Q$, are alpha-equivalent if $P = Q\{\vec{y}/\vec{x}\}$ for
some $\vec{x} \in \boundnames{Q},\vec{y} \in \boundnames{P}$, where $Q\{\vec{y}/\vec{x}\}$
denotes the capture-avoiding substitution of $\vec{y}$ for $\vec{x}$ in $Q$.
\end{definition}

\begin{definition}
  The {\em structural congruence} \cite{SangiorgiWalker} , $\equiv$,
  between processes is the least congruence containing
  alpha-equivalence, satisfying the abelian monoid laws
  (associativity, commutativity and $\pzero$ as identity) for parallel
  composition $|$ and for summation $+$.
\end{definition}

\subsection{Name equivalence}

We take name equivalence, written $\nameeq$, to be the smallest
equivalence relation generated by the following rules.

\begin{mathpar}
\inferrule*[lab=Quote-drop]
{ }
{ \quotep{@{x}} \nameeq x }

\inferrule*[lab=Struct-equiv]
{ P \scong Q }
{ \quotep{P} \nameeq \quotep{Q} }
\end{mathpar}

The astute reader will have noticed that the mutual recursion of names
and processes imposes a mutual recursion on alpha-equivalence and
structural equivalence via name-equivalence. Fortunately, all of this
works out pleasantly and we may calculate in the natural way, free of
concern. The reader interested in the details is referred to the
appendix \ref{appendix:rho_details}.

\subsection{Substitution}

We use $\Proc$ for the set of processes, $\QProc$ for the set of
names, and $\id{\{}\vec{y} / \vec{x} \id{\}}$ to denote partial maps,
$s : \QProc \rightarrow \QProc$. A map, $s$ lifts, uniquely, to a map
on process terms, $\widehat{s} : \Proc \rightarrow \Proc$ by the
following equations.

\begin{mathpar}
  (0) \psubstp{Q}{P} := 0 \\
  (R \juxtap S) \psubstp{Q}{P}
  :=    
  (R)\psubstp{Q}{P} \juxtap (S) \psubstp{Q}{P} \\
  (x?(y).R) \psubstp{Q}{P}    
  :=    
  (x)\substp{Q}{P} (z)\concat( (R \psubstn{z}{y}) \psubstp{Q}{P} ) \\
  (\lift{x}{R}) \psubstp{Q}{P}  
  :=
  \lift{(x)\substp{Q}{P}}{ R \psubstp{Q}{P} } \\
%   (\dropn{x})  \psubstp{Q}{P}       
%   := 
%   \left\{ 
%     \begin{array}{ccc} 
%       \dropn{\quotep{Q}} & & x \nameeq \quotep{P} \\
%       \dropn{x} & & otherwise \\
%     \end{array}
%   \right. 
  (\dropn{x})  \psubstp{Q}{P}       
  := 
  \left\{ 
    \begin{array}{ccc} 
      Q & & x \nameeq \quotep{P} \\
      \dropn{x} & & otherwise \\
    \end{array}
  \right.
\end{mathpar}
 

where

\begin{eqnarray}
  (x)\id{\{} \lpquote Q \rpquote / \lpquote P \rpquote \id{\}}            = 
  \left\{ 
    \begin{array}{ccc}
      \lpquote Q \rpquote & & x \nameeq \lpquote P \rpquote \\
      x & & otherwise \\
    \end{array}
  \right. \nonumber
\end{eqnarray}

and $z$ is chosen distinct from $\quotep{P}$, $\quotep{Q}$, the free
names in $Q$, and all the names in $R$. Our $\alpha$-equivalence will
be built in the standard way from this substitution.

\begin{remark}\label{rem:no_self_referential_names}
  One consequence of these definitions is that $\forall P. \quotep{P}
  \not\in \freenames{P}$.
\end{remark}

\subsection{ Dynamic quote: an example }

Anticipating something of what's to come, consider applying the
substitution, $\widehat{\id{\{}u / z \id{\}}}$, to the following pair
of processes, $\lift{w}{y!(z)}$ and $w[ \lpquote y!(z) \rpquote ]$.

\begin{eqnarray}
	\lift{w}{y!(z)}\widehat{\id{\{}u / z \id{\}}}
		& = &
		\lift{w}{y!(u)} \nonumber\\
	w[ \lpquote y!(z) \rpquote ] \widehat{ \id{\{}u / z \id{\}} }
		& = &
		w[ \lpquote y!(z) \rpquote ] \nonumber
\end{eqnarray}

Because the body of the process between quotes is impervious to
substitution, we get radically different answers. In fact, by
examining the first process in an input context,
e.g. $x?(z).\lift{w}{y!(z)}$, we see that the process under the lift
operator may be shaped by prefixed inputs binding a name inside it. In
this sense, the lift operator will be seen as a way to dynamically
construct processes before reifying them as names.

Finally equipped with these standard features we can present the
dynamics of the calculus.

\subsubsection{Operational semantics} 

Finally, we introduce the computational dynamics. What marks these
algebras as distinct from other more traditionally studied algebraic
structures, e.g. vector spaces or polynomial rings, is the manner in
which dynamics is captured. In traditional structures, dynamics is typically
expressed through morphisms between such structures, as in linear maps
between vector spaces or morphisms between rings. In algebras
associated with the semantics of computation, the dynamics is
expressed as part of the algebraic structure itself, through a
reduction reduction relation typically denoted by $\red$. Below, we
give a recursive presentation of this relation for the calculus used
in the encoding.

$\red \subseteq \pi \times \pi$
$\red : \pi \to \mathcal{P}(\pi)$

\begin{mathpar}
  \inferrule* [lab=Comm] { \textsf{match}( x_{src}, x_{trgt} ) } { x_{trgt}?(y)P \; | \; x_{src}!\langle {Q} \rangle \red P\{\quotep{Q}/y}\} }
  \and \\
  \inferrule* [lab=Par] {{P} \red {P}'} {{{P} | {Q}} \red {{P}' | {Q}}}
  \and
  \inferrule* [lab=Equiv]{{{P} \scong {P}'} \andalso {{P}' \red {Q}'} \andalso {{Q}' \scong {Q}}}{{P} \red {Q}}
\end{mathpar}

\begin{eqnarray*}
  match_{\equiv} (\quotep{P},\quotep{Q}) & := & P \equiv Q \\
  match_{\dagger}(\quotep{P},\quotep{Q}) & := & \forall R. P|Q \red^{*} R => R \red^{*} 0 \\
  match_{K}(\quotep{P},\quotep{Q}) & := & K \mbox{ for some context } K
\end{eqnarray*}

$u?(x)P | u!\langle Q \rangle \red P\{\quotep{Q}/x\}$

%We write $\wred$ for $\red^*$, and $P\red$ if $\exists Q $ such that $ P \red Q$.
We write $P\red$ if $\exists Q $ such that $ P \red Q$ and $P\not\red$, otherwise.

\section{Replication}

As mentioned before, it is known that replication (and hence
recursion) can be implemented in a higher-order process algebra
\cite{SangiorgiWalker}. As our first example of calculation with the
machinery thus far presented we give the construction explicitly in
the {\rhoc}.

\begin{eqnarray}
	D_{x} & := & \prefix{x}{y}{(\binpar{\outputp{x}{y}}{@{y}})} \nonumber\\
	\bangp_{x}{P} & := & \binpar{{x}!\langle{\binpar{D_{x}}{P}}\rangle}{D_{x}} \nonumber
\end{eqnarray}

\begin{eqnarray}
	\bangp_{x}{P} & & \nonumber\\
	=
	& {x}!\langle{(\prefix{x}{y}{(\outputp{x}{y} | @{y})) | P}}\rangle 
	      | \prefix{x}{y}{(\outputp{x}{y} | @{y})} & \nonumber\\
	\red
	& (\outputp{x}{y} | @{y})\substn{\quotep{(\prefix{x}{y}{(@{y} | \outputp{x}{y})) | P}}}{y} & \nonumber\\
	=
	& \outputp{x}{\quotep{(\prefix{x}{y}{(\outputp{x}{y} | @{y})) | P}}}
	  | {(\prefix{x}{y}{(\outputp{x}{y} | @{y})) | P}} & \nonumber\\
	\red
	& \ldots & \nonumber\\
	\red^*
	& P | P | \ldots & \nonumber
\end{eqnarray}

Of course, this encoding, as an implementation, runs away, unfolding
$\bangp{P}$ eagerly. A lazier and more implementable replication
operator, restricted to input-guarded processes, may be obtained as follows.

\begin{eqnarray}
\bangp{\prefix{u}{v}{P}} 
	:= 
	\binpar{\lift{x}{\prefix{u}{v}{(\binpar{D(x)}{P})}}}{D(x)} \nonumber
\end{eqnarray}

\begin{remark}
  Note that the lazier definition still does not deal with summation
  or mixed summation (i.e. sums over input and output). The reader is
  invited to construct definitions of replication that deal with these
  features. 

  Further, the definitions are parameterized in a name, $x$. Can you,
  gentle reader, make a definition that eliminates this parameter and
  guarantees no accidental interaction between the replication
  machinery and the process being replicated -- i.e. no accidental
  sharing of names used by the process to get its work done and the
  name(s) used by the replication to effect copying. This latter
  revision of the definition of replication is crucial to obtaining
  the expected identity $!!P \sim !P$.
\end{remark}

\begin{remark}\label{rem:paradoxical_combinator}
  The reader familiar with the lambda calculus will have noticed the
  similarity between $D$ and the paradoxical combinator.

  [Ed. note: the existence of this seems to suggest we have to be more
  restrictive on the set of processes and names we admit if we are to
  support no-cloning.]
\end{remark}

\subsubsection{Bisimulation}

The computational dynamics gives rise to another kind of equivalence,
the equivalence of computational behavior. As previously mentioned
this is typically captured \emph{via} some form of bisimulation.

% The notion we use in this paper is weak barbed bisimulation
% \cite{milner91polyadicpi}.

The notion we use in this paper is derived from weak barbed
bisimulation \cite{milner91polyadicpi}. 

\begin{definition}
An \emph{observation relation}, $\downarrow_{\mathcal N}$, over a set
of names, $\mathcal N$, is the smallest relation satisfying the rules
below.

\infrule[Out-barb]{y \in {\mathcal N}, \; x \nameeq y}
		  {\outputp{x}{v} \downarrow_{\mathcal N} x}
\infrule[Par-barb]{\mbox{$P\downarrow_{\mathcal N} x$ or $Q\downarrow_{\mathcal N} x$}}
		  {\binpar{P}{Q} \downarrow_{\mathcal N} x}

We write $P \Downarrow_{\mathcal N} x$ if there is $Q$ such that 
$P \wred Q$ and $Q \downarrow_{\mathcal N} x$.
\end{definition}

\begin{definition}
%\label{def.bbisim}
An  ${\mathcal N}$-\emph{barbed bisimulation} over a set of names, ${\mathcal N}$, is a symmetric binary relation 
${\mathcal S}_{\mathcal N}$ between agents such that $P\rel{S}_{\mathcal N}Q$ implies:
\begin{enumerate}
\item If $P \red P'$ then $Q \wred Q'$ and $P'\rel{S}_{\mathcal N} Q'$.
\item If $P\downarrow_{\mathcal N} x$, then $Q\Downarrow_{\mathcal N} x$.
\end{enumerate}
$P$ is ${\mathcal N}$-barbed bisimilar to $Q$, written
$P \wbbisim_{\mathcal N} Q$, if $P \rel{S}_{\mathcal N} Q$ for some ${\mathcal N}$-barbed bisimulation ${\mathcal S}_{\mathcal N}$.
\end{definition}

$\mathcal{R} \subseteq \pi \times \pi$

$P \mathcal{R} Q => \forall P'. P \red P' \Rightarrow \exists Q'. Q \red Q', P' \mathcal{R} Q'$

$P \vdash x \Rightarrow Q \vdash x$

\begin{mathpar}
  \inferrule*[lab=Out-barb]{x \nameeq y}{{y}!\langle{Q}\rangle \vdash x}
  \and
  \inferrule*[lab=Par-barb]{\mbox{$P\vdash x$ or $Q\vdash x$}}{\binpar{P}{Q} \vdash x}
\end{mathpar}

\subsubsection{Contexts}

One of the principle advantages of computational calculi like the
$\pi$-calculus is a well-defined notion of context,
contextual-equivalence and a correlation between
contextual-equivalence and notions of bisimulation. The notion of
context allows the decomposition of a process into (sub-)process and
its syntactic environment, its context. Thus, a context may be
thought of as a process with a ``hole'' (written $\Box$) in it. The
application of a context $M$ to a process $P$, written $M[P]$, is
tantamount to filling the hole in $M$ with $P$. In this paper we do
not need the full weight of this theory, but do make use of the notion
of context in the proof the main theorem. 

\begin{mathpar}
  \inferrule* [lab=summation] {} {{M_{M},M_{N}} \bc \Box \;|\; x.M_{A} \;|\; M_{M}+M_{N}}
  \and
  \inferrule* [lab=agent] {} {{M_{A}} \bc (\vec{x})M_{P} \;| \; \clift{P_0,\ldots,M_{P},\ldots,P_N}}
  \and \\
  \inferrule* [lab=process] {} {{M_{P}} \bc M_{N} \;| \;P|M_{P} }
\end{mathpar} 

\begin{mathpar}
  \inferrule* [lab=sychronization] {} {M_{N} \bc \Box \;|\; x?M_{F} \;|\; x!M_{C}}
  \and
  \inferrule* [lab=abstraction] {} {{M_{F}} \bc (x)M_{P} }
  \and
  \inferrule* [lab=concretion] {} {{M_{C}} \bc \langle M_{P} \rangle }
  \and \\
  \inferrule* [lab=process] {} {{M_{P}} \bc M_{N} \;| \;P|M_{P} }
\end{mathpar}

\begin{definition}[contextual application] Given a context $M$, and
  process $P$, we define the \emph{contextual application}, $M[P] :=
  M\{P/\Box\}$. That is, the contextual application of M to P is the
  substitution of $P$ for $\Box$ in $M$.
\end{definition}

$\meaningof{-} : L \to \mathcal{P}(\pi)$

\begin{mathpar}
  \inferrule* [lab=collection] {} {\meaningof{true} = \pi, \and \meaningof{~E} = \pi \setminus \meaningof{E}, \and \meaningof{E_{1} \& E_{2}} = \meaningof{E_{1}} \cap \meaningof{E_{2}}}
\end{mathpar}

\begin{mathpar}
  \inferrule* [lab=structure] {} {\meaningof{0} = \{ P \in \pi | P \equiv 0 \}, \and \\ \meaningof{E_1 | E_2} = \{ P \in \pi | P \equiv P_{1} | P_{2}, P_{1} \in \meaningof{E_{1}}, P_{2} \in \meaningof{E_2}\} }
\end{mathpar}

\begin{mathpar}
 \inferrule* [lab=behavior] {} {\meaningof{\langle a?b \rangle E} = \{ P \in \pi | P \equiv Q | u?(y)P', \\ \and \\\\ \and \\ \;\;\; u \in \meaningof{a}, \forall z.P'\{z/y\} \in \meaningof{E\{z/b\}}\}, \and \\ \meaningof{a!E} = \{ P \in \pi | P \equiv Q | x!\langle P' \rangle, x \in \meaningof{a} P' \in \meaningof{E}\} }
\end{mathpar}

\begin{mathpar}
 \inferrule* [lab=nominal] {} {\meaningof{\quotep{E}} = \{ \quotep{P} \in \quotep{\pi} | P \in \meaningof{E} \}, \and \meaningof{\quotep{P}} = \{ \quotep{Q} \in \quotep{\pi} | P \equiv Q \} \and \\ \meaningof{@\quotep{E}} = \{ P \in \pi | P \equiv @x, x \in \meaningof{E} \}}
\end{mathpar}

\begin{eqnarray*}
  \\
  \meaningof{-} : TS \to ST
\end{eqnarray*}

\begin{eqnarray*}
  \\
  L : TS \to ST
\end{eqnarray*}

\begin{eqnarray*}
  \\
  P \models E \iff P \in \meaningof{E}
\end{eqnarray*}

\begin{eqnarray*}
  P \approx_{L} Q \iff \forall E \in L. P \models E \iff Q \models E
\end{eqnarray*}

\begin{eqnarray*}
  P \approx_{K} Q
\end{eqnarray*}

\begin{eqnarray*}
  P \approx Q
\end{eqnarray*}

$\approx_{K} = \approx = \approx_{L}$

\subsubsection{Contextual duality}

Note that contexts extend the quotation operation to a family of
operations from processes to names. Given a context, $M$, we can
define a \emph{nominal context}, $\quotep{M}$ by $\quotep{M}[P] :=
\quotep{M[P]}$. To foreshadow what is to come we observe that these
operations enjoy a duality with processes very much like the duality
between vectors and maps from vectors to scalars.

Further, because the calculus is essentially higher-order, we have a
correspondence between contexts and processes. More specifically,
given a name $x$ and a context $M$ we can construct $M^{*}_{x}$ such
that 

\begin{mathpar}
  M^{*}_{x} | \lift{x}{P} \red M[P]
\end{mathpar}

namely,

\begin{mathpar}
  M^{*}_{x} := x?(u).M[\dropn{u}]
\end{mathpar}

The dependence of $M^{*}_{x}$ on a name makes it an abstraction, 

\begin{mathpar}
  M^{*} := (x)x?(u).M[\dropn{u}]
\end{mathpar}

\subsection{Additional notation}

It will sometimes be convenient to denote the process a name
quotes. We already have the notation $x = \quotep{P}$, but it will be
convenient to introduce an alternate notation, $\procn{x}$, when we
want to emphasize the connection to the use of the name. Note that, by
virtue of name equivalence, $\quotep{\procn{x}} \nameeq x$; so, the
notation is consistent with previous definitions.

Further, because names have structure it is possible to effect
substitutions on the basis of that structure. This means we need to
upgrade our notation for substitutions, which we accomplish by
adapting comprehension notation. Thus,

\begin{mathpar}
  P\{ y / x : x \in S \}
\end{mathpar}

is interpreted to mean the process derived from P by replacing (in a
capture-avoiding manner) each occurrence of $x$ in $S$ by $y$. For example,

\begin{mathpar}
  P\{ \quotep{\procn{x}|\procn{x}} / x : x \in \freenames{P} \}
\end{mathpar}

will replace each (occurrence) of a free name $x$ in $P$ by
$\quotep{\procn{x}|\procn{x}}$.

Also, we will avail ourselves of the notation $x^{L}$ and $x^{R}$ to
denote injections of a name into disjoint copies of the name
space. There are numerous ways to accomplish this. One example can be
found in \cite{MeredithR05}. This notation overloads to vectors of
names: $\vec{x}^{\pi} := (x_{i}^{\pi} \; : \; 0 \leq i < |\vec{x}| )$ where $\pi \in \{L,R\}$.

We also use $P^{\Box} := P|\Box$.

In \cite{MeredithR05} an interpretation of the new operator is
given. It turns out that there are several possible interpretations
all enjoying the requisite algebraic properties of the operator (see
\cite{milner91polyadicpi}). We will therefore make liberal use of
$(\nu\; \vec{x})P$.

% subsection the_syntax_and_semantics_of_the_notation_system (end)   

\input{qm2pi.qmops} 

\input{qm2pi.sterngerlach} 

\input{qm2pi.metric} 

% section concurrent_process_calculi (end)

%\input{qm2pi.proofsketch}

% section proof sketch (end)

%\input{qm2pi.slviaknots} 

% section spatial logic via knots (end)

\input{qm2pi.conclusion}

% section conclusion (end)

%\input{qm2pi.dtcodes} 

% section wiring algorithm (end)

\input{qm2pi.ack} 

% section acknowledgments (end)

\newpage


\bibliographystyle{plain}   
\bibliography{../../biblios/main.bib}

\input{qm2pi.rhodetails}

\end{document}

 

% section wiring algorithm (end)

\documentclass[12pt]{llncs}
%\documentclass{jktr}

\usepackage[pdftex]{hyperref}                   
\usepackage {listings}
\usepackage {mathpartir}
\usepackage{bcprules}
%\usepackage{listings}
                       
\usepackage{graphicx} 
%\usepackage[margins=2.5cm,nohead,nofoot]{geometry}
%\usepackage{geometry}
\usepackage{amsfonts}
\usepackage{amstext}
\usepackage{latexsym}
\usepackage{amssymb}
\usepackage{color}


%\include{myPreamble}
\include{qm2pi.local} 

%\ifpdf
%\usepackage[pdftex]{graphicx}
%\else
%\usepackage{graphicx}
%\fi

 % \ifpdf
%  \usepackage{pdfsync}
%  \if


%\title{Brief Article}
%\author{David F. Snyder}
%\author{L.G. Meredith}

%\address{Dept. of Math., Texas State University--San Marcos, San Marcos, TX 78666}
       
\pagestyle{empty}


\begin{document}

\lstset{language=[Objective]Caml,frame=shadowbox}

\input{qm2pi.front}

% section front matter (end)

\input{qm2pi.intro} 
 
% section introduction (end)

% \input{qm2pi.knotations} 

% section notation (end)

\input{qm2pi.process.calculi} 

% section concurrent_process_calculi_and_spatial_logics_ (end)
    
%\input{qm2pi.knots2pi} 

%\input{qm2pi.trefoil} 

%\input{qm2pi.mainthm} 

% subsection basic_interpretation (end)

%\input{qm2pi.rho.presentation} 
\subsection{The syntax and semantics of the notation system}\label{sub:the_syntax_and_semantics_of_the_notation_system} % (fold)

We now summarize a technical presentation of the calculus that
embodies our theory of dynamics. The typical presentation of such a
calculus follows the style of giving generators and relations on
them. The grammar, below, describing term constructors, freely
generates the set of processes, $\Proc$. This set is then quotiented
by a relation known as structural congruence and it is over this set
that the notion of dynamics is expressed. This presentation is
essentially that of \cite{MeredithR05} with the addition of
polyadicity and summation. For readability we have relegated some of
the technical subtleties to an appendix.

\subsubsection{Process grammar}\label{subsub:process_grammar}

\begin{mathpar}
  \inferrule* [lab=synchronization] {} {{M} \bc \pzero \;|\; x?F \;|\; x!C }
  \and
  \inferrule* [lab=abstraction] {} {{F} \bc (x)P}
  \and
  \inferrule* [lab=concretion] {} {{C} \bc \langle Q \rangle}
  \and
  \inferrule* [lab=process] {} {{P,Q} \bc M \;| \;P|Q \;|\; @{x}}
  \and
  \inferrule* [lab=name] {} {{x} \bc \quotep{P}}
\end{mathpar} 

Note that $\vec{x}$ (resp. $\vec{P}$) denotes a vector of names
(resp. processes) of length $|\vec{x}|$ (resp. $|\vec{P}|$). We adopt
the following useful abbreviations.

\begin{mathpar}
   x?(\vec{y}).P := x.(\vec{y})P \and  x\clift{\vec{P}} := x.\clift{\vec{P}}
   \and x!(y) := \lift{x}{\dropn{y}}
   \and \Pi_{i=0}^{n-1}P_i := P_0 | \ldots | P_{n-1}
\end{mathpar}

\subsubsection{Structural congruence}

\paragraph{Free and bound names and alpha-equivalence.} At the
core of structural equivalence is alpha-equivalence which identifies
process that are the same up to a change of variable. Formally, we
recognize the distinction between free and bound names. The free names
of a process, $\freenames{P}$, may be calculated recursively as
follows:

\begin{mathpar}
\freenames{\pzero} := \emptyset
  \and \\
  \freenames{x?(y).P} := \{ x \} \cup (\freenames{P} \setminus \{ y \})
  \and 
  \freenames{x!\langle P \rangle} := \{ x \} \cup \{ P \} 
  \and \\
  \freenames{P|Q} := \freenames{P} \cup \freenames{Q}
  \and \\
  \freenames{@{x}} := \{ x \}
\end{mathpar}

$\pi$
$\quotep{\pi}$

$\freenames{-} : \pi \to \mathcal{P}(\quotep{\pi})$

\begin{eqnarray*}
  \freenames{\pzero} & := & \emptyset \\
  \freenames{x?(y).P} & := & \{ x \} \cup (\freenames{P} \setminus \{ y \}) \\
  \freenames{x!\langle P \rangle} & := & \{ x \} \cup \{ P \} \\
  \freenames{P|Q} & := & \freenames{P} \cup \freenames{Q} \\
  \freenames{\dropn{x}} & := & \{ x \}
\end{eqnarray*}

The bound names of a process, $\boundnames{P}$, are those names occurring in $P$
that are not free. For example, in $x?(y).0$, the name $x$ is free, while $y$ is bound.

\begin{mathpar}
  \inferrule* [lab=monoidal-laws] {} { P|Q \equiv Q|P \and P|0 \equiv P \and P|(Q|R) \equiv (P|Q)|R }
\end{mathpar}

\begin{mathpar}
  \inferrule* [lab=alpha-equivalence] {} { (x)P \equiv (y)P\{y/x\} \and y \not\in \freenames{P} }
\end{mathpar}

\begin{definition}
Then two processes, $P,Q$, are alpha-equivalent if $P = Q\{\vec{y}/\vec{x}\}$ for
some $\vec{x} \in \boundnames{Q},\vec{y} \in \boundnames{P}$, where $Q\{\vec{y}/\vec{x}\}$
denotes the capture-avoiding substitution of $\vec{y}$ for $\vec{x}$ in $Q$.
\end{definition}

\begin{definition}
  The {\em structural congruence} \cite{SangiorgiWalker} , $\equiv$,
  between processes is the least congruence containing
  alpha-equivalence, satisfying the abelian monoid laws
  (associativity, commutativity and $\pzero$ as identity) for parallel
  composition $|$ and for summation $+$.
\end{definition}

\subsection{Name equivalence}

We take name equivalence, written $\nameeq$, to be the smallest
equivalence relation generated by the following rules.

\begin{mathpar}
\inferrule*[lab=Quote-drop]
{ }
{ \quotep{@{x}} \nameeq x }

\inferrule*[lab=Struct-equiv]
{ P \scong Q }
{ \quotep{P} \nameeq \quotep{Q} }
\end{mathpar}

The astute reader will have noticed that the mutual recursion of names
and processes imposes a mutual recursion on alpha-equivalence and
structural equivalence via name-equivalence. Fortunately, all of this
works out pleasantly and we may calculate in the natural way, free of
concern. The reader interested in the details is referred to the
appendix \ref{appendix:rho_details}.

\subsection{Substitution}

We use $\Proc$ for the set of processes, $\QProc$ for the set of
names, and $\id{\{}\vec{y} / \vec{x} \id{\}}$ to denote partial maps,
$s : \QProc \rightarrow \QProc$. A map, $s$ lifts, uniquely, to a map
on process terms, $\widehat{s} : \Proc \rightarrow \Proc$ by the
following equations.

\begin{mathpar}
  (0) \psubstp{Q}{P} := 0 \\
  (R \juxtap S) \psubstp{Q}{P}
  :=    
  (R)\psubstp{Q}{P} \juxtap (S) \psubstp{Q}{P} \\
  (x?(y).R) \psubstp{Q}{P}    
  :=    
  (x)\substp{Q}{P} (z)\concat( (R \psubstn{z}{y}) \psubstp{Q}{P} ) \\
  (\lift{x}{R}) \psubstp{Q}{P}  
  :=
  \lift{(x)\substp{Q}{P}}{ R \psubstp{Q}{P} } \\
%   (\dropn{x})  \psubstp{Q}{P}       
%   := 
%   \left\{ 
%     \begin{array}{ccc} 
%       \dropn{\quotep{Q}} & & x \nameeq \quotep{P} \\
%       \dropn{x} & & otherwise \\
%     \end{array}
%   \right. 
  (\dropn{x})  \psubstp{Q}{P}       
  := 
  \left\{ 
    \begin{array}{ccc} 
      Q & & x \nameeq \quotep{P} \\
      \dropn{x} & & otherwise \\
    \end{array}
  \right.
\end{mathpar}
 

where

\begin{eqnarray}
  (x)\id{\{} \lpquote Q \rpquote / \lpquote P \rpquote \id{\}}            = 
  \left\{ 
    \begin{array}{ccc}
      \lpquote Q \rpquote & & x \nameeq \lpquote P \rpquote \\
      x & & otherwise \\
    \end{array}
  \right. \nonumber
\end{eqnarray}

and $z$ is chosen distinct from $\quotep{P}$, $\quotep{Q}$, the free
names in $Q$, and all the names in $R$. Our $\alpha$-equivalence will
be built in the standard way from this substitution.

\begin{remark}\label{rem:no_self_referential_names}
  One consequence of these definitions is that $\forall P. \quotep{P}
  \not\in \freenames{P}$.
\end{remark}

\subsection{ Dynamic quote: an example }

Anticipating something of what's to come, consider applying the
substitution, $\widehat{\id{\{}u / z \id{\}}}$, to the following pair
of processes, $\lift{w}{y!(z)}$ and $w[ \lpquote y!(z) \rpquote ]$.

\begin{eqnarray}
	\lift{w}{y!(z)}\widehat{\id{\{}u / z \id{\}}}
		& = &
		\lift{w}{y!(u)} \nonumber\\
	w[ \lpquote y!(z) \rpquote ] \widehat{ \id{\{}u / z \id{\}} }
		& = &
		w[ \lpquote y!(z) \rpquote ] \nonumber
\end{eqnarray}

Because the body of the process between quotes is impervious to
substitution, we get radically different answers. In fact, by
examining the first process in an input context,
e.g. $x?(z).\lift{w}{y!(z)}$, we see that the process under the lift
operator may be shaped by prefixed inputs binding a name inside it. In
this sense, the lift operator will be seen as a way to dynamically
construct processes before reifying them as names.

Finally equipped with these standard features we can present the
dynamics of the calculus.

\subsubsection{Operational semantics} 

Finally, we introduce the computational dynamics. What marks these
algebras as distinct from other more traditionally studied algebraic
structures, e.g. vector spaces or polynomial rings, is the manner in
which dynamics is captured. In traditional structures, dynamics is typically
expressed through morphisms between such structures, as in linear maps
between vector spaces or morphisms between rings. In algebras
associated with the semantics of computation, the dynamics is
expressed as part of the algebraic structure itself, through a
reduction reduction relation typically denoted by $\red$. Below, we
give a recursive presentation of this relation for the calculus used
in the encoding.

$\red \subseteq \pi \times \pi$
$\red : \pi \to \mathcal{P}(\pi)$

\begin{mathpar}
  \inferrule* [lab=Comm] { \textsf{match}( x_{src}, x_{trgt} ) } { x_{trgt}?(y)P \; | \; x_{src}!\langle {Q} \rangle \red P\{\quotep{Q}/y}\} }
  \and \\
  \inferrule* [lab=Par] {{P} \red {P}'} {{{P} | {Q}} \red {{P}' | {Q}}}
  \and
  \inferrule* [lab=Equiv]{{{P} \scong {P}'} \andalso {{P}' \red {Q}'} \andalso {{Q}' \scong {Q}}}{{P} \red {Q}}
\end{mathpar}

\begin{eqnarray*}
  match_{\equiv} (\quotep{P},\quotep{Q}) & := & P \equiv Q \\
  match_{\dagger}(\quotep{P},\quotep{Q}) & := & \forall R. P|Q \red^{*} R => R \red^{*} 0 \\
  match_{K}(\quotep{P},\quotep{Q}) & := & K \mbox{ for some context } K
\end{eqnarray*}

$u?(x)P | u!\langle Q \rangle \red P\{\quotep{Q}/x\}$

%We write $\wred$ for $\red^*$, and $P\red$ if $\exists Q $ such that $ P \red Q$.
We write $P\red$ if $\exists Q $ such that $ P \red Q$ and $P\not\red$, otherwise.

\section{Replication}

As mentioned before, it is known that replication (and hence
recursion) can be implemented in a higher-order process algebra
\cite{SangiorgiWalker}. As our first example of calculation with the
machinery thus far presented we give the construction explicitly in
the {\rhoc}.

\begin{eqnarray}
	D_{x} & := & \prefix{x}{y}{(\binpar{\outputp{x}{y}}{@{y}})} \nonumber\\
	\bangp_{x}{P} & := & \binpar{{x}!\langle{\binpar{D_{x}}{P}}\rangle}{D_{x}} \nonumber
\end{eqnarray}

\begin{eqnarray}
	\bangp_{x}{P} & & \nonumber\\
	=
	& {x}!\langle{(\prefix{x}{y}{(\outputp{x}{y} | @{y})) | P}}\rangle 
	      | \prefix{x}{y}{(\outputp{x}{y} | @{y})} & \nonumber\\
	\red
	& (\outputp{x}{y} | @{y})\substn{\quotep{(\prefix{x}{y}{(@{y} | \outputp{x}{y})) | P}}}{y} & \nonumber\\
	=
	& \outputp{x}{\quotep{(\prefix{x}{y}{(\outputp{x}{y} | @{y})) | P}}}
	  | {(\prefix{x}{y}{(\outputp{x}{y} | @{y})) | P}} & \nonumber\\
	\red
	& \ldots & \nonumber\\
	\red^*
	& P | P | \ldots & \nonumber
\end{eqnarray}

Of course, this encoding, as an implementation, runs away, unfolding
$\bangp{P}$ eagerly. A lazier and more implementable replication
operator, restricted to input-guarded processes, may be obtained as follows.

\begin{eqnarray}
\bangp{\prefix{u}{v}{P}} 
	:= 
	\binpar{\lift{x}{\prefix{u}{v}{(\binpar{D(x)}{P})}}}{D(x)} \nonumber
\end{eqnarray}

\begin{remark}
  Note that the lazier definition still does not deal with summation
  or mixed summation (i.e. sums over input and output). The reader is
  invited to construct definitions of replication that deal with these
  features. 

  Further, the definitions are parameterized in a name, $x$. Can you,
  gentle reader, make a definition that eliminates this parameter and
  guarantees no accidental interaction between the replication
  machinery and the process being replicated -- i.e. no accidental
  sharing of names used by the process to get its work done and the
  name(s) used by the replication to effect copying. This latter
  revision of the definition of replication is crucial to obtaining
  the expected identity $!!P \sim !P$.
\end{remark}

\begin{remark}\label{rem:paradoxical_combinator}
  The reader familiar with the lambda calculus will have noticed the
  similarity between $D$ and the paradoxical combinator.

  [Ed. note: the existence of this seems to suggest we have to be more
  restrictive on the set of processes and names we admit if we are to
  support no-cloning.]
\end{remark}

\subsubsection{Bisimulation}

The computational dynamics gives rise to another kind of equivalence,
the equivalence of computational behavior. As previously mentioned
this is typically captured \emph{via} some form of bisimulation.

% The notion we use in this paper is weak barbed bisimulation
% \cite{milner91polyadicpi}.

The notion we use in this paper is derived from weak barbed
bisimulation \cite{milner91polyadicpi}. 

\begin{definition}
An \emph{observation relation}, $\downarrow_{\mathcal N}$, over a set
of names, $\mathcal N$, is the smallest relation satisfying the rules
below.

\infrule[Out-barb]{y \in {\mathcal N}, \; x \nameeq y}
		  {\outputp{x}{v} \downarrow_{\mathcal N} x}
\infrule[Par-barb]{\mbox{$P\downarrow_{\mathcal N} x$ or $Q\downarrow_{\mathcal N} x$}}
		  {\binpar{P}{Q} \downarrow_{\mathcal N} x}

We write $P \Downarrow_{\mathcal N} x$ if there is $Q$ such that 
$P \wred Q$ and $Q \downarrow_{\mathcal N} x$.
\end{definition}

\begin{definition}
%\label{def.bbisim}
An  ${\mathcal N}$-\emph{barbed bisimulation} over a set of names, ${\mathcal N}$, is a symmetric binary relation 
${\mathcal S}_{\mathcal N}$ between agents such that $P\rel{S}_{\mathcal N}Q$ implies:
\begin{enumerate}
\item If $P \red P'$ then $Q \wred Q'$ and $P'\rel{S}_{\mathcal N} Q'$.
\item If $P\downarrow_{\mathcal N} x$, then $Q\Downarrow_{\mathcal N} x$.
\end{enumerate}
$P$ is ${\mathcal N}$-barbed bisimilar to $Q$, written
$P \wbbisim_{\mathcal N} Q$, if $P \rel{S}_{\mathcal N} Q$ for some ${\mathcal N}$-barbed bisimulation ${\mathcal S}_{\mathcal N}$.
\end{definition}

$\mathcal{R} \subseteq \pi \times \pi$

$P \mathcal{R} Q => \forall P'. P \red P' \Rightarrow \exists Q'. Q \red Q', P' \mathcal{R} Q'$

$P \vdash x \Rightarrow Q \vdash x$

\begin{mathpar}
  \inferrule*[lab=Out-barb]{x \nameeq y}{{y}!\langle{Q}\rangle \vdash x}
  \and
  \inferrule*[lab=Par-barb]{\mbox{$P\vdash x$ or $Q\vdash x$}}{\binpar{P}{Q} \vdash x}
\end{mathpar}

\subsubsection{Contexts}

One of the principle advantages of computational calculi like the
$\pi$-calculus is a well-defined notion of context,
contextual-equivalence and a correlation between
contextual-equivalence and notions of bisimulation. The notion of
context allows the decomposition of a process into (sub-)process and
its syntactic environment, its context. Thus, a context may be
thought of as a process with a ``hole'' (written $\Box$) in it. The
application of a context $M$ to a process $P$, written $M[P]$, is
tantamount to filling the hole in $M$ with $P$. In this paper we do
not need the full weight of this theory, but do make use of the notion
of context in the proof the main theorem. 

\begin{mathpar}
  \inferrule* [lab=summation] {} {{M_{M},M_{N}} \bc \Box \;|\; x.M_{A} \;|\; M_{M}+M_{N}}
  \and
  \inferrule* [lab=agent] {} {{M_{A}} \bc (\vec{x})M_{P} \;| \; \clift{P_0,\ldots,M_{P},\ldots,P_N}}
  \and \\
  \inferrule* [lab=process] {} {{M_{P}} \bc M_{N} \;| \;P|M_{P} }
\end{mathpar} 

\begin{mathpar}
  \inferrule* [lab=sychronization] {} {M_{N} \bc \Box \;|\; x?M_{F} \;|\; x!M_{C}}
  \and
  \inferrule* [lab=abstraction] {} {{M_{F}} \bc (x)M_{P} }
  \and
  \inferrule* [lab=concretion] {} {{M_{C}} \bc \langle M_{P} \rangle }
  \and \\
  \inferrule* [lab=process] {} {{M_{P}} \bc M_{N} \;| \;P|M_{P} }
\end{mathpar}

\begin{definition}[contextual application] Given a context $M$, and
  process $P$, we define the \emph{contextual application}, $M[P] :=
  M\{P/\Box\}$. That is, the contextual application of M to P is the
  substitution of $P$ for $\Box$ in $M$.
\end{definition}

$\meaningof{-} : L \to \mathcal{P}(\pi)$

\begin{mathpar}
  \inferrule* [lab=collection] {} {\meaningof{true} = \pi, \and \meaningof{~E} = \pi \setminus \meaningof{E}, \and \meaningof{E_{1} \& E_{2}} = \meaningof{E_{1}} \cap \meaningof{E_{2}}}
\end{mathpar}

\begin{mathpar}
  \inferrule* [lab=structure] {} {\meaningof{0} = \{ P \in \pi | P \equiv 0 \}, \and \\ \meaningof{E_1 | E_2} = \{ P \in \pi | P \equiv P_{1} | P_{2}, P_{1} \in \meaningof{E_{1}}, P_{2} \in \meaningof{E_2}\} }
\end{mathpar}

\begin{mathpar}
 \inferrule* [lab=behavior] {} {\meaningof{\langle a?b \rangle E} = \{ P \in \pi | P \equiv Q | u?(y)P', \\ \and \\\\ \and \\ \;\;\; u \in \meaningof{a}, \forall z.P'\{z/y\} \in \meaningof{E\{z/b\}}\}, \and \\ \meaningof{a!E} = \{ P \in \pi | P \equiv Q | x!\langle P' \rangle, x \in \meaningof{a} P' \in \meaningof{E}\} }
\end{mathpar}

\begin{mathpar}
 \inferrule* [lab=nominal] {} {\meaningof{\quotep{E}} = \{ \quotep{P} \in \quotep{\pi} | P \in \meaningof{E} \}, \and \meaningof{\quotep{P}} = \{ \quotep{Q} \in \quotep{\pi} | P \equiv Q \} \and \\ \meaningof{@\quotep{E}} = \{ P \in \pi | P \equiv @x, x \in \meaningof{E} \}}
\end{mathpar}

\begin{eqnarray*}
  \\
  \meaningof{-} : TS \to ST
\end{eqnarray*}

\begin{eqnarray*}
  \\
  L : TS \to ST
\end{eqnarray*}

\begin{eqnarray*}
  \\
  P \models E \iff P \in \meaningof{E}
\end{eqnarray*}

\begin{eqnarray*}
  P \approx_{L} Q \iff \forall E \in L. P \models E \iff Q \models E
\end{eqnarray*}

\begin{eqnarray*}
  P \approx_{K} Q
\end{eqnarray*}

\begin{eqnarray*}
  P \approx Q
\end{eqnarray*}

$\approx_{K} = \approx = \approx_{L}$

\subsubsection{Contextual duality}

Note that contexts extend the quotation operation to a family of
operations from processes to names. Given a context, $M$, we can
define a \emph{nominal context}, $\quotep{M}$ by $\quotep{M}[P] :=
\quotep{M[P]}$. To foreshadow what is to come we observe that these
operations enjoy a duality with processes very much like the duality
between vectors and maps from vectors to scalars.

Further, because the calculus is essentially higher-order, we have a
correspondence between contexts and processes. More specifically,
given a name $x$ and a context $M$ we can construct $M^{*}_{x}$ such
that 

\begin{mathpar}
  M^{*}_{x} | \lift{x}{P} \red M[P]
\end{mathpar}

namely,

\begin{mathpar}
  M^{*}_{x} := x?(u).M[\dropn{u}]
\end{mathpar}

The dependence of $M^{*}_{x}$ on a name makes it an abstraction, 

\begin{mathpar}
  M^{*} := (x)x?(u).M[\dropn{u}]
\end{mathpar}

\subsection{Additional notation}

It will sometimes be convenient to denote the process a name
quotes. We already have the notation $x = \quotep{P}$, but it will be
convenient to introduce an alternate notation, $\procn{x}$, when we
want to emphasize the connection to the use of the name. Note that, by
virtue of name equivalence, $\quotep{\procn{x}} \nameeq x$; so, the
notation is consistent with previous definitions.

Further, because names have structure it is possible to effect
substitutions on the basis of that structure. This means we need to
upgrade our notation for substitutions, which we accomplish by
adapting comprehension notation. Thus,

\begin{mathpar}
  P\{ y / x : x \in S \}
\end{mathpar}

is interpreted to mean the process derived from P by replacing (in a
capture-avoiding manner) each occurrence of $x$ in $S$ by $y$. For example,

\begin{mathpar}
  P\{ \quotep{\procn{x}|\procn{x}} / x : x \in \freenames{P} \}
\end{mathpar}

will replace each (occurrence) of a free name $x$ in $P$ by
$\quotep{\procn{x}|\procn{x}}$.

Also, we will avail ourselves of the notation $x^{L}$ and $x^{R}$ to
denote injections of a name into disjoint copies of the name
space. There are numerous ways to accomplish this. One example can be
found in \cite{MeredithR05}. This notation overloads to vectors of
names: $\vec{x}^{\pi} := (x_{i}^{\pi} \; : \; 0 \leq i < |\vec{x}| )$ where $\pi \in \{L,R\}$.

We also use $P^{\Box} := P|\Box$.

In \cite{MeredithR05} an interpretation of the new operator is
given. It turns out that there are several possible interpretations
all enjoying the requisite algebraic properties of the operator (see
\cite{milner91polyadicpi}). We will therefore make liberal use of
$(\nu\; \vec{x})P$.

% subsection the_syntax_and_semantics_of_the_notation_system (end)   

\input{qm2pi.qmops} 

\input{qm2pi.sterngerlach} 

\input{qm2pi.metric} 

% section concurrent_process_calculi (end)

%\input{qm2pi.proofsketch}

% section proof sketch (end)

%\input{qm2pi.slviaknots} 

% section spatial logic via knots (end)

\input{qm2pi.conclusion}

% section conclusion (end)

%\input{qm2pi.dtcodes} 

% section wiring algorithm (end)

\input{qm2pi.ack} 

% section acknowledgments (end)

\newpage


\bibliographystyle{plain}   
\bibliography{../../biblios/main.bib}

\input{qm2pi.rhodetails}

\end{document}

 

% section acknowledgments (end)

\newpage


\bibliographystyle{plain}   
\bibliography{../../biblios/main.bib}

\documentclass[12pt]{llncs}
%\documentclass{jktr}

\usepackage[pdftex]{hyperref}                   
\usepackage {listings}
\usepackage {mathpartir}
\usepackage{bcprules}
%\usepackage{listings}
                       
\usepackage{graphicx} 
%\usepackage[margins=2.5cm,nohead,nofoot]{geometry}
%\usepackage{geometry}
\usepackage{amsfonts}
\usepackage{amstext}
\usepackage{latexsym}
\usepackage{amssymb}
\usepackage{color}


%\include{myPreamble}
\include{qm2pi.local} 

%\ifpdf
%\usepackage[pdftex]{graphicx}
%\else
%\usepackage{graphicx}
%\fi

 % \ifpdf
%  \usepackage{pdfsync}
%  \if


%\title{Brief Article}
%\author{David F. Snyder}
%\author{L.G. Meredith}

%\address{Dept. of Math., Texas State University--San Marcos, San Marcos, TX 78666}
       
\pagestyle{empty}


\begin{document}

\lstset{language=[Objective]Caml,frame=shadowbox}

\input{qm2pi.front}

% section front matter (end)

\input{qm2pi.intro} 
 
% section introduction (end)

% \input{qm2pi.knotations} 

% section notation (end)

\input{qm2pi.process.calculi} 

% section concurrent_process_calculi_and_spatial_logics_ (end)
    
%\input{qm2pi.knots2pi} 

%\input{qm2pi.trefoil} 

%\input{qm2pi.mainthm} 

% subsection basic_interpretation (end)

%\input{qm2pi.rho.presentation} 
\subsection{The syntax and semantics of the notation system}\label{sub:the_syntax_and_semantics_of_the_notation_system} % (fold)

We now summarize a technical presentation of the calculus that
embodies our theory of dynamics. The typical presentation of such a
calculus follows the style of giving generators and relations on
them. The grammar, below, describing term constructors, freely
generates the set of processes, $\Proc$. This set is then quotiented
by a relation known as structural congruence and it is over this set
that the notion of dynamics is expressed. This presentation is
essentially that of \cite{MeredithR05} with the addition of
polyadicity and summation. For readability we have relegated some of
the technical subtleties to an appendix.

\subsubsection{Process grammar}\label{subsub:process_grammar}

\begin{mathpar}
  \inferrule* [lab=synchronization] {} {{M} \bc \pzero \;|\; x?F \;|\; x!C }
  \and
  \inferrule* [lab=abstraction] {} {{F} \bc (x)P}
  \and
  \inferrule* [lab=concretion] {} {{C} \bc \langle Q \rangle}
  \and
  \inferrule* [lab=process] {} {{P,Q} \bc M \;| \;P|Q \;|\; @{x}}
  \and
  \inferrule* [lab=name] {} {{x} \bc \quotep{P}}
\end{mathpar} 

Note that $\vec{x}$ (resp. $\vec{P}$) denotes a vector of names
(resp. processes) of length $|\vec{x}|$ (resp. $|\vec{P}|$). We adopt
the following useful abbreviations.

\begin{mathpar}
   x?(\vec{y}).P := x.(\vec{y})P \and  x\clift{\vec{P}} := x.\clift{\vec{P}}
   \and x!(y) := \lift{x}{\dropn{y}}
   \and \Pi_{i=0}^{n-1}P_i := P_0 | \ldots | P_{n-1}
\end{mathpar}

\subsubsection{Structural congruence}

\paragraph{Free and bound names and alpha-equivalence.} At the
core of structural equivalence is alpha-equivalence which identifies
process that are the same up to a change of variable. Formally, we
recognize the distinction between free and bound names. The free names
of a process, $\freenames{P}$, may be calculated recursively as
follows:

\begin{mathpar}
\freenames{\pzero} := \emptyset
  \and \\
  \freenames{x?(y).P} := \{ x \} \cup (\freenames{P} \setminus \{ y \})
  \and 
  \freenames{x!\langle P \rangle} := \{ x \} \cup \{ P \} 
  \and \\
  \freenames{P|Q} := \freenames{P} \cup \freenames{Q}
  \and \\
  \freenames{@{x}} := \{ x \}
\end{mathpar}

$\pi$
$\quotep{\pi}$

$\freenames{-} : \pi \to \mathcal{P}(\quotep{\pi})$

\begin{eqnarray*}
  \freenames{\pzero} & := & \emptyset \\
  \freenames{x?(y).P} & := & \{ x \} \cup (\freenames{P} \setminus \{ y \}) \\
  \freenames{x!\langle P \rangle} & := & \{ x \} \cup \{ P \} \\
  \freenames{P|Q} & := & \freenames{P} \cup \freenames{Q} \\
  \freenames{\dropn{x}} & := & \{ x \}
\end{eqnarray*}

The bound names of a process, $\boundnames{P}$, are those names occurring in $P$
that are not free. For example, in $x?(y).0$, the name $x$ is free, while $y$ is bound.

\begin{mathpar}
  \inferrule* [lab=monoidal-laws] {} { P|Q \equiv Q|P \and P|0 \equiv P \and P|(Q|R) \equiv (P|Q)|R }
\end{mathpar}

\begin{mathpar}
  \inferrule* [lab=alpha-equivalence] {} { (x)P \equiv (y)P\{y/x\} \and y \not\in \freenames{P} }
\end{mathpar}

\begin{definition}
Then two processes, $P,Q$, are alpha-equivalent if $P = Q\{\vec{y}/\vec{x}\}$ for
some $\vec{x} \in \boundnames{Q},\vec{y} \in \boundnames{P}$, where $Q\{\vec{y}/\vec{x}\}$
denotes the capture-avoiding substitution of $\vec{y}$ for $\vec{x}$ in $Q$.
\end{definition}

\begin{definition}
  The {\em structural congruence} \cite{SangiorgiWalker} , $\equiv$,
  between processes is the least congruence containing
  alpha-equivalence, satisfying the abelian monoid laws
  (associativity, commutativity and $\pzero$ as identity) for parallel
  composition $|$ and for summation $+$.
\end{definition}

\subsection{Name equivalence}

We take name equivalence, written $\nameeq$, to be the smallest
equivalence relation generated by the following rules.

\begin{mathpar}
\inferrule*[lab=Quote-drop]
{ }
{ \quotep{@{x}} \nameeq x }

\inferrule*[lab=Struct-equiv]
{ P \scong Q }
{ \quotep{P} \nameeq \quotep{Q} }
\end{mathpar}

The astute reader will have noticed that the mutual recursion of names
and processes imposes a mutual recursion on alpha-equivalence and
structural equivalence via name-equivalence. Fortunately, all of this
works out pleasantly and we may calculate in the natural way, free of
concern. The reader interested in the details is referred to the
appendix \ref{appendix:rho_details}.

\subsection{Substitution}

We use $\Proc$ for the set of processes, $\QProc$ for the set of
names, and $\id{\{}\vec{y} / \vec{x} \id{\}}$ to denote partial maps,
$s : \QProc \rightarrow \QProc$. A map, $s$ lifts, uniquely, to a map
on process terms, $\widehat{s} : \Proc \rightarrow \Proc$ by the
following equations.

\begin{mathpar}
  (0) \psubstp{Q}{P} := 0 \\
  (R \juxtap S) \psubstp{Q}{P}
  :=    
  (R)\psubstp{Q}{P} \juxtap (S) \psubstp{Q}{P} \\
  (x?(y).R) \psubstp{Q}{P}    
  :=    
  (x)\substp{Q}{P} (z)\concat( (R \psubstn{z}{y}) \psubstp{Q}{P} ) \\
  (\lift{x}{R}) \psubstp{Q}{P}  
  :=
  \lift{(x)\substp{Q}{P}}{ R \psubstp{Q}{P} } \\
%   (\dropn{x})  \psubstp{Q}{P}       
%   := 
%   \left\{ 
%     \begin{array}{ccc} 
%       \dropn{\quotep{Q}} & & x \nameeq \quotep{P} \\
%       \dropn{x} & & otherwise \\
%     \end{array}
%   \right. 
  (\dropn{x})  \psubstp{Q}{P}       
  := 
  \left\{ 
    \begin{array}{ccc} 
      Q & & x \nameeq \quotep{P} \\
      \dropn{x} & & otherwise \\
    \end{array}
  \right.
\end{mathpar}
 

where

\begin{eqnarray}
  (x)\id{\{} \lpquote Q \rpquote / \lpquote P \rpquote \id{\}}            = 
  \left\{ 
    \begin{array}{ccc}
      \lpquote Q \rpquote & & x \nameeq \lpquote P \rpquote \\
      x & & otherwise \\
    \end{array}
  \right. \nonumber
\end{eqnarray}

and $z$ is chosen distinct from $\quotep{P}$, $\quotep{Q}$, the free
names in $Q$, and all the names in $R$. Our $\alpha$-equivalence will
be built in the standard way from this substitution.

\begin{remark}\label{rem:no_self_referential_names}
  One consequence of these definitions is that $\forall P. \quotep{P}
  \not\in \freenames{P}$.
\end{remark}

\subsection{ Dynamic quote: an example }

Anticipating something of what's to come, consider applying the
substitution, $\widehat{\id{\{}u / z \id{\}}}$, to the following pair
of processes, $\lift{w}{y!(z)}$ and $w[ \lpquote y!(z) \rpquote ]$.

\begin{eqnarray}
	\lift{w}{y!(z)}\widehat{\id{\{}u / z \id{\}}}
		& = &
		\lift{w}{y!(u)} \nonumber\\
	w[ \lpquote y!(z) \rpquote ] \widehat{ \id{\{}u / z \id{\}} }
		& = &
		w[ \lpquote y!(z) \rpquote ] \nonumber
\end{eqnarray}

Because the body of the process between quotes is impervious to
substitution, we get radically different answers. In fact, by
examining the first process in an input context,
e.g. $x?(z).\lift{w}{y!(z)}$, we see that the process under the lift
operator may be shaped by prefixed inputs binding a name inside it. In
this sense, the lift operator will be seen as a way to dynamically
construct processes before reifying them as names.

Finally equipped with these standard features we can present the
dynamics of the calculus.

\subsubsection{Operational semantics} 

Finally, we introduce the computational dynamics. What marks these
algebras as distinct from other more traditionally studied algebraic
structures, e.g. vector spaces or polynomial rings, is the manner in
which dynamics is captured. In traditional structures, dynamics is typically
expressed through morphisms between such structures, as in linear maps
between vector spaces or morphisms between rings. In algebras
associated with the semantics of computation, the dynamics is
expressed as part of the algebraic structure itself, through a
reduction reduction relation typically denoted by $\red$. Below, we
give a recursive presentation of this relation for the calculus used
in the encoding.

$\red \subseteq \pi \times \pi$
$\red : \pi \to \mathcal{P}(\pi)$

\begin{mathpar}
  \inferrule* [lab=Comm] { \textsf{match}( x_{src}, x_{trgt} ) } { x_{trgt}?(y)P \; | \; x_{src}!\langle {Q} \rangle \red P\{\quotep{Q}/y}\} }
  \and \\
  \inferrule* [lab=Par] {{P} \red {P}'} {{{P} | {Q}} \red {{P}' | {Q}}}
  \and
  \inferrule* [lab=Equiv]{{{P} \scong {P}'} \andalso {{P}' \red {Q}'} \andalso {{Q}' \scong {Q}}}{{P} \red {Q}}
\end{mathpar}

\begin{eqnarray*}
  match_{\equiv} (\quotep{P},\quotep{Q}) & := & P \equiv Q \\
  match_{\dagger}(\quotep{P},\quotep{Q}) & := & \forall R. P|Q \red^{*} R => R \red^{*} 0 \\
  match_{K}(\quotep{P},\quotep{Q}) & := & K \mbox{ for some context } K
\end{eqnarray*}

$u?(x)P | u!\langle Q \rangle \red P\{\quotep{Q}/x\}$

%We write $\wred$ for $\red^*$, and $P\red$ if $\exists Q $ such that $ P \red Q$.
We write $P\red$ if $\exists Q $ such that $ P \red Q$ and $P\not\red$, otherwise.

\section{Replication}

As mentioned before, it is known that replication (and hence
recursion) can be implemented in a higher-order process algebra
\cite{SangiorgiWalker}. As our first example of calculation with the
machinery thus far presented we give the construction explicitly in
the {\rhoc}.

\begin{eqnarray}
	D_{x} & := & \prefix{x}{y}{(\binpar{\outputp{x}{y}}{@{y}})} \nonumber\\
	\bangp_{x}{P} & := & \binpar{{x}!\langle{\binpar{D_{x}}{P}}\rangle}{D_{x}} \nonumber
\end{eqnarray}

\begin{eqnarray}
	\bangp_{x}{P} & & \nonumber\\
	=
	& {x}!\langle{(\prefix{x}{y}{(\outputp{x}{y} | @{y})) | P}}\rangle 
	      | \prefix{x}{y}{(\outputp{x}{y} | @{y})} & \nonumber\\
	\red
	& (\outputp{x}{y} | @{y})\substn{\quotep{(\prefix{x}{y}{(@{y} | \outputp{x}{y})) | P}}}{y} & \nonumber\\
	=
	& \outputp{x}{\quotep{(\prefix{x}{y}{(\outputp{x}{y} | @{y})) | P}}}
	  | {(\prefix{x}{y}{(\outputp{x}{y} | @{y})) | P}} & \nonumber\\
	\red
	& \ldots & \nonumber\\
	\red^*
	& P | P | \ldots & \nonumber
\end{eqnarray}

Of course, this encoding, as an implementation, runs away, unfolding
$\bangp{P}$ eagerly. A lazier and more implementable replication
operator, restricted to input-guarded processes, may be obtained as follows.

\begin{eqnarray}
\bangp{\prefix{u}{v}{P}} 
	:= 
	\binpar{\lift{x}{\prefix{u}{v}{(\binpar{D(x)}{P})}}}{D(x)} \nonumber
\end{eqnarray}

\begin{remark}
  Note that the lazier definition still does not deal with summation
  or mixed summation (i.e. sums over input and output). The reader is
  invited to construct definitions of replication that deal with these
  features. 

  Further, the definitions are parameterized in a name, $x$. Can you,
  gentle reader, make a definition that eliminates this parameter and
  guarantees no accidental interaction between the replication
  machinery and the process being replicated -- i.e. no accidental
  sharing of names used by the process to get its work done and the
  name(s) used by the replication to effect copying. This latter
  revision of the definition of replication is crucial to obtaining
  the expected identity $!!P \sim !P$.
\end{remark}

\begin{remark}\label{rem:paradoxical_combinator}
  The reader familiar with the lambda calculus will have noticed the
  similarity between $D$ and the paradoxical combinator.

  [Ed. note: the existence of this seems to suggest we have to be more
  restrictive on the set of processes and names we admit if we are to
  support no-cloning.]
\end{remark}

\subsubsection{Bisimulation}

The computational dynamics gives rise to another kind of equivalence,
the equivalence of computational behavior. As previously mentioned
this is typically captured \emph{via} some form of bisimulation.

% The notion we use in this paper is weak barbed bisimulation
% \cite{milner91polyadicpi}.

The notion we use in this paper is derived from weak barbed
bisimulation \cite{milner91polyadicpi}. 

\begin{definition}
An \emph{observation relation}, $\downarrow_{\mathcal N}$, over a set
of names, $\mathcal N$, is the smallest relation satisfying the rules
below.

\infrule[Out-barb]{y \in {\mathcal N}, \; x \nameeq y}
		  {\outputp{x}{v} \downarrow_{\mathcal N} x}
\infrule[Par-barb]{\mbox{$P\downarrow_{\mathcal N} x$ or $Q\downarrow_{\mathcal N} x$}}
		  {\binpar{P}{Q} \downarrow_{\mathcal N} x}

We write $P \Downarrow_{\mathcal N} x$ if there is $Q$ such that 
$P \wred Q$ and $Q \downarrow_{\mathcal N} x$.
\end{definition}

\begin{definition}
%\label{def.bbisim}
An  ${\mathcal N}$-\emph{barbed bisimulation} over a set of names, ${\mathcal N}$, is a symmetric binary relation 
${\mathcal S}_{\mathcal N}$ between agents such that $P\rel{S}_{\mathcal N}Q$ implies:
\begin{enumerate}
\item If $P \red P'$ then $Q \wred Q'$ and $P'\rel{S}_{\mathcal N} Q'$.
\item If $P\downarrow_{\mathcal N} x$, then $Q\Downarrow_{\mathcal N} x$.
\end{enumerate}
$P$ is ${\mathcal N}$-barbed bisimilar to $Q$, written
$P \wbbisim_{\mathcal N} Q$, if $P \rel{S}_{\mathcal N} Q$ for some ${\mathcal N}$-barbed bisimulation ${\mathcal S}_{\mathcal N}$.
\end{definition}

$\mathcal{R} \subseteq \pi \times \pi$

$P \mathcal{R} Q => \forall P'. P \red P' \Rightarrow \exists Q'. Q \red Q', P' \mathcal{R} Q'$

$P \vdash x \Rightarrow Q \vdash x$

\begin{mathpar}
  \inferrule*[lab=Out-barb]{x \nameeq y}{{y}!\langle{Q}\rangle \vdash x}
  \and
  \inferrule*[lab=Par-barb]{\mbox{$P\vdash x$ or $Q\vdash x$}}{\binpar{P}{Q} \vdash x}
\end{mathpar}

\subsubsection{Contexts}

One of the principle advantages of computational calculi like the
$\pi$-calculus is a well-defined notion of context,
contextual-equivalence and a correlation between
contextual-equivalence and notions of bisimulation. The notion of
context allows the decomposition of a process into (sub-)process and
its syntactic environment, its context. Thus, a context may be
thought of as a process with a ``hole'' (written $\Box$) in it. The
application of a context $M$ to a process $P$, written $M[P]$, is
tantamount to filling the hole in $M$ with $P$. In this paper we do
not need the full weight of this theory, but do make use of the notion
of context in the proof the main theorem. 

\begin{mathpar}
  \inferrule* [lab=summation] {} {{M_{M},M_{N}} \bc \Box \;|\; x.M_{A} \;|\; M_{M}+M_{N}}
  \and
  \inferrule* [lab=agent] {} {{M_{A}} \bc (\vec{x})M_{P} \;| \; \clift{P_0,\ldots,M_{P},\ldots,P_N}}
  \and \\
  \inferrule* [lab=process] {} {{M_{P}} \bc M_{N} \;| \;P|M_{P} }
\end{mathpar} 

\begin{mathpar}
  \inferrule* [lab=sychronization] {} {M_{N} \bc \Box \;|\; x?M_{F} \;|\; x!M_{C}}
  \and
  \inferrule* [lab=abstraction] {} {{M_{F}} \bc (x)M_{P} }
  \and
  \inferrule* [lab=concretion] {} {{M_{C}} \bc \langle M_{P} \rangle }
  \and \\
  \inferrule* [lab=process] {} {{M_{P}} \bc M_{N} \;| \;P|M_{P} }
\end{mathpar}

\begin{definition}[contextual application] Given a context $M$, and
  process $P$, we define the \emph{contextual application}, $M[P] :=
  M\{P/\Box\}$. That is, the contextual application of M to P is the
  substitution of $P$ for $\Box$ in $M$.
\end{definition}

$\meaningof{-} : L \to \mathcal{P}(\pi)$

\begin{mathpar}
  \inferrule* [lab=collection] {} {\meaningof{true} = \pi, \and \meaningof{~E} = \pi \setminus \meaningof{E}, \and \meaningof{E_{1} \& E_{2}} = \meaningof{E_{1}} \cap \meaningof{E_{2}}}
\end{mathpar}

\begin{mathpar}
  \inferrule* [lab=structure] {} {\meaningof{0} = \{ P \in \pi | P \equiv 0 \}, \and \\ \meaningof{E_1 | E_2} = \{ P \in \pi | P \equiv P_{1} | P_{2}, P_{1} \in \meaningof{E_{1}}, P_{2} \in \meaningof{E_2}\} }
\end{mathpar}

\begin{mathpar}
 \inferrule* [lab=behavior] {} {\meaningof{\langle a?b \rangle E} = \{ P \in \pi | P \equiv Q | u?(y)P', \\ \and \\\\ \and \\ \;\;\; u \in \meaningof{a}, \forall z.P'\{z/y\} \in \meaningof{E\{z/b\}}\}, \and \\ \meaningof{a!E} = \{ P \in \pi | P \equiv Q | x!\langle P' \rangle, x \in \meaningof{a} P' \in \meaningof{E}\} }
\end{mathpar}

\begin{mathpar}
 \inferrule* [lab=nominal] {} {\meaningof{\quotep{E}} = \{ \quotep{P} \in \quotep{\pi} | P \in \meaningof{E} \}, \and \meaningof{\quotep{P}} = \{ \quotep{Q} \in \quotep{\pi} | P \equiv Q \} \and \\ \meaningof{@\quotep{E}} = \{ P \in \pi | P \equiv @x, x \in \meaningof{E} \}}
\end{mathpar}

\begin{eqnarray*}
  \\
  \meaningof{-} : TS \to ST
\end{eqnarray*}

\begin{eqnarray*}
  \\
  L : TS \to ST
\end{eqnarray*}

\begin{eqnarray*}
  \\
  P \models E \iff P \in \meaningof{E}
\end{eqnarray*}

\begin{eqnarray*}
  P \approx_{L} Q \iff \forall E \in L. P \models E \iff Q \models E
\end{eqnarray*}

\begin{eqnarray*}
  P \approx_{K} Q
\end{eqnarray*}

\begin{eqnarray*}
  P \approx Q
\end{eqnarray*}

$\approx_{K} = \approx = \approx_{L}$

\subsubsection{Contextual duality}

Note that contexts extend the quotation operation to a family of
operations from processes to names. Given a context, $M$, we can
define a \emph{nominal context}, $\quotep{M}$ by $\quotep{M}[P] :=
\quotep{M[P]}$. To foreshadow what is to come we observe that these
operations enjoy a duality with processes very much like the duality
between vectors and maps from vectors to scalars.

Further, because the calculus is essentially higher-order, we have a
correspondence between contexts and processes. More specifically,
given a name $x$ and a context $M$ we can construct $M^{*}_{x}$ such
that 

\begin{mathpar}
  M^{*}_{x} | \lift{x}{P} \red M[P]
\end{mathpar}

namely,

\begin{mathpar}
  M^{*}_{x} := x?(u).M[\dropn{u}]
\end{mathpar}

The dependence of $M^{*}_{x}$ on a name makes it an abstraction, 

\begin{mathpar}
  M^{*} := (x)x?(u).M[\dropn{u}]
\end{mathpar}

\subsection{Additional notation}

It will sometimes be convenient to denote the process a name
quotes. We already have the notation $x = \quotep{P}$, but it will be
convenient to introduce an alternate notation, $\procn{x}$, when we
want to emphasize the connection to the use of the name. Note that, by
virtue of name equivalence, $\quotep{\procn{x}} \nameeq x$; so, the
notation is consistent with previous definitions.

Further, because names have structure it is possible to effect
substitutions on the basis of that structure. This means we need to
upgrade our notation for substitutions, which we accomplish by
adapting comprehension notation. Thus,

\begin{mathpar}
  P\{ y / x : x \in S \}
\end{mathpar}

is interpreted to mean the process derived from P by replacing (in a
capture-avoiding manner) each occurrence of $x$ in $S$ by $y$. For example,

\begin{mathpar}
  P\{ \quotep{\procn{x}|\procn{x}} / x : x \in \freenames{P} \}
\end{mathpar}

will replace each (occurrence) of a free name $x$ in $P$ by
$\quotep{\procn{x}|\procn{x}}$.

Also, we will avail ourselves of the notation $x^{L}$ and $x^{R}$ to
denote injections of a name into disjoint copies of the name
space. There are numerous ways to accomplish this. One example can be
found in \cite{MeredithR05}. This notation overloads to vectors of
names: $\vec{x}^{\pi} := (x_{i}^{\pi} \; : \; 0 \leq i < |\vec{x}| )$ where $\pi \in \{L,R\}$.

We also use $P^{\Box} := P|\Box$.

In \cite{MeredithR05} an interpretation of the new operator is
given. It turns out that there are several possible interpretations
all enjoying the requisite algebraic properties of the operator (see
\cite{milner91polyadicpi}). We will therefore make liberal use of
$(\nu\; \vec{x})P$.

% subsection the_syntax_and_semantics_of_the_notation_system (end)   

\input{qm2pi.qmops} 

\input{qm2pi.sterngerlach} 

\input{qm2pi.metric} 

% section concurrent_process_calculi (end)

%\input{qm2pi.proofsketch}

% section proof sketch (end)

%\input{qm2pi.slviaknots} 

% section spatial logic via knots (end)

\input{qm2pi.conclusion}

% section conclusion (end)

%\input{qm2pi.dtcodes} 

% section wiring algorithm (end)

\input{qm2pi.ack} 

% section acknowledgments (end)

\newpage


\bibliographystyle{plain}   
\bibliography{../../biblios/main.bib}

\input{qm2pi.rhodetails}

\end{document}



\end{document}



\end{document}

 

% section notation (end)

\input{qm2pi.process.calculi} 

% section concurrent_process_calculi_and_spatial_logics_ (end)
    
%\documentclass[12pt]{llncs}
%\documentclass{jktr}

\usepackage[pdftex]{hyperref}                   
\usepackage {listings}
\usepackage {mathpartir}
\usepackage{bcprules}
%\usepackage{listings}
                       
\usepackage{graphicx} 
%\usepackage[margins=2.5cm,nohead,nofoot]{geometry}
%\usepackage{geometry}
\usepackage{amsfonts}
\usepackage{amstext}
\usepackage{latexsym}
\usepackage{amssymb}
\usepackage{color}


%\include{myPreamble}
\documentclass[12pt]{llncs}
%\documentclass{jktr}

\usepackage[pdftex]{hyperref}                   
\usepackage {listings}
\usepackage {mathpartir}
\usepackage{bcprules}
%\usepackage{listings}
                       
\usepackage{graphicx} 
%\usepackage[margins=2.5cm,nohead,nofoot]{geometry}
%\usepackage{geometry}
\usepackage{amsfonts}
\usepackage{amstext}
\usepackage{latexsym}
\usepackage{amssymb}
\usepackage{color}


%\include{myPreamble}
\documentclass[12pt]{llncs}
%\documentclass{jktr}

\usepackage[pdftex]{hyperref}                   
\usepackage {listings}
\usepackage {mathpartir}
\usepackage{bcprules}
%\usepackage{listings}
                       
\usepackage{graphicx} 
%\usepackage[margins=2.5cm,nohead,nofoot]{geometry}
%\usepackage{geometry}
\usepackage{amsfonts}
\usepackage{amstext}
\usepackage{latexsym}
\usepackage{amssymb}
\usepackage{color}


%\include{myPreamble}
\include{qm2pi.local} 

%\ifpdf
%\usepackage[pdftex]{graphicx}
%\else
%\usepackage{graphicx}
%\fi

 % \ifpdf
%  \usepackage{pdfsync}
%  \if


%\title{Brief Article}
%\author{David F. Snyder}
%\author{L.G. Meredith}

%\address{Dept. of Math., Texas State University--San Marcos, San Marcos, TX 78666}
       
\pagestyle{empty}


\begin{document}

\lstset{language=[Objective]Caml,frame=shadowbox}

\input{qm2pi.front}

% section front matter (end)

\input{qm2pi.intro} 
 
% section introduction (end)

% \input{qm2pi.knotations} 

% section notation (end)

\input{qm2pi.process.calculi} 

% section concurrent_process_calculi_and_spatial_logics_ (end)
    
%\input{qm2pi.knots2pi} 

%\input{qm2pi.trefoil} 

%\input{qm2pi.mainthm} 

% subsection basic_interpretation (end)

%\input{qm2pi.rho.presentation} 
\subsection{The syntax and semantics of the notation system}\label{sub:the_syntax_and_semantics_of_the_notation_system} % (fold)

We now summarize a technical presentation of the calculus that
embodies our theory of dynamics. The typical presentation of such a
calculus follows the style of giving generators and relations on
them. The grammar, below, describing term constructors, freely
generates the set of processes, $\Proc$. This set is then quotiented
by a relation known as structural congruence and it is over this set
that the notion of dynamics is expressed. This presentation is
essentially that of \cite{MeredithR05} with the addition of
polyadicity and summation. For readability we have relegated some of
the technical subtleties to an appendix.

\subsubsection{Process grammar}\label{subsub:process_grammar}

\begin{mathpar}
  \inferrule* [lab=synchronization] {} {{M} \bc \pzero \;|\; x?F \;|\; x!C }
  \and
  \inferrule* [lab=abstraction] {} {{F} \bc (x)P}
  \and
  \inferrule* [lab=concretion] {} {{C} \bc \langle Q \rangle}
  \and
  \inferrule* [lab=process] {} {{P,Q} \bc M \;| \;P|Q \;|\; @{x}}
  \and
  \inferrule* [lab=name] {} {{x} \bc \quotep{P}}
\end{mathpar} 

Note that $\vec{x}$ (resp. $\vec{P}$) denotes a vector of names
(resp. processes) of length $|\vec{x}|$ (resp. $|\vec{P}|$). We adopt
the following useful abbreviations.

\begin{mathpar}
   x?(\vec{y}).P := x.(\vec{y})P \and  x\clift{\vec{P}} := x.\clift{\vec{P}}
   \and x!(y) := \lift{x}{\dropn{y}}
   \and \Pi_{i=0}^{n-1}P_i := P_0 | \ldots | P_{n-1}
\end{mathpar}

\subsubsection{Structural congruence}

\paragraph{Free and bound names and alpha-equivalence.} At the
core of structural equivalence is alpha-equivalence which identifies
process that are the same up to a change of variable. Formally, we
recognize the distinction between free and bound names. The free names
of a process, $\freenames{P}$, may be calculated recursively as
follows:

\begin{mathpar}
\freenames{\pzero} := \emptyset
  \and \\
  \freenames{x?(y).P} := \{ x \} \cup (\freenames{P} \setminus \{ y \})
  \and 
  \freenames{x!\langle P \rangle} := \{ x \} \cup \{ P \} 
  \and \\
  \freenames{P|Q} := \freenames{P} \cup \freenames{Q}
  \and \\
  \freenames{@{x}} := \{ x \}
\end{mathpar}

$\pi$
$\quotep{\pi}$

$\freenames{-} : \pi \to \mathcal{P}(\quotep{\pi})$

\begin{eqnarray*}
  \freenames{\pzero} & := & \emptyset \\
  \freenames{x?(y).P} & := & \{ x \} \cup (\freenames{P} \setminus \{ y \}) \\
  \freenames{x!\langle P \rangle} & := & \{ x \} \cup \{ P \} \\
  \freenames{P|Q} & := & \freenames{P} \cup \freenames{Q} \\
  \freenames{\dropn{x}} & := & \{ x \}
\end{eqnarray*}

The bound names of a process, $\boundnames{P}$, are those names occurring in $P$
that are not free. For example, in $x?(y).0$, the name $x$ is free, while $y$ is bound.

\begin{mathpar}
  \inferrule* [lab=monoidal-laws] {} { P|Q \equiv Q|P \and P|0 \equiv P \and P|(Q|R) \equiv (P|Q)|R }
\end{mathpar}

\begin{mathpar}
  \inferrule* [lab=alpha-equivalence] {} { (x)P \equiv (y)P\{y/x\} \and y \not\in \freenames{P} }
\end{mathpar}

\begin{definition}
Then two processes, $P,Q$, are alpha-equivalent if $P = Q\{\vec{y}/\vec{x}\}$ for
some $\vec{x} \in \boundnames{Q},\vec{y} \in \boundnames{P}$, where $Q\{\vec{y}/\vec{x}\}$
denotes the capture-avoiding substitution of $\vec{y}$ for $\vec{x}$ in $Q$.
\end{definition}

\begin{definition}
  The {\em structural congruence} \cite{SangiorgiWalker} , $\equiv$,
  between processes is the least congruence containing
  alpha-equivalence, satisfying the abelian monoid laws
  (associativity, commutativity and $\pzero$ as identity) for parallel
  composition $|$ and for summation $+$.
\end{definition}

\subsection{Name equivalence}

We take name equivalence, written $\nameeq$, to be the smallest
equivalence relation generated by the following rules.

\begin{mathpar}
\inferrule*[lab=Quote-drop]
{ }
{ \quotep{@{x}} \nameeq x }

\inferrule*[lab=Struct-equiv]
{ P \scong Q }
{ \quotep{P} \nameeq \quotep{Q} }
\end{mathpar}

The astute reader will have noticed that the mutual recursion of names
and processes imposes a mutual recursion on alpha-equivalence and
structural equivalence via name-equivalence. Fortunately, all of this
works out pleasantly and we may calculate in the natural way, free of
concern. The reader interested in the details is referred to the
appendix \ref{appendix:rho_details}.

\subsection{Substitution}

We use $\Proc$ for the set of processes, $\QProc$ for the set of
names, and $\id{\{}\vec{y} / \vec{x} \id{\}}$ to denote partial maps,
$s : \QProc \rightarrow \QProc$. A map, $s$ lifts, uniquely, to a map
on process terms, $\widehat{s} : \Proc \rightarrow \Proc$ by the
following equations.

\begin{mathpar}
  (0) \psubstp{Q}{P} := 0 \\
  (R \juxtap S) \psubstp{Q}{P}
  :=    
  (R)\psubstp{Q}{P} \juxtap (S) \psubstp{Q}{P} \\
  (x?(y).R) \psubstp{Q}{P}    
  :=    
  (x)\substp{Q}{P} (z)\concat( (R \psubstn{z}{y}) \psubstp{Q}{P} ) \\
  (\lift{x}{R}) \psubstp{Q}{P}  
  :=
  \lift{(x)\substp{Q}{P}}{ R \psubstp{Q}{P} } \\
%   (\dropn{x})  \psubstp{Q}{P}       
%   := 
%   \left\{ 
%     \begin{array}{ccc} 
%       \dropn{\quotep{Q}} & & x \nameeq \quotep{P} \\
%       \dropn{x} & & otherwise \\
%     \end{array}
%   \right. 
  (\dropn{x})  \psubstp{Q}{P}       
  := 
  \left\{ 
    \begin{array}{ccc} 
      Q & & x \nameeq \quotep{P} \\
      \dropn{x} & & otherwise \\
    \end{array}
  \right.
\end{mathpar}
 

where

\begin{eqnarray}
  (x)\id{\{} \lpquote Q \rpquote / \lpquote P \rpquote \id{\}}            = 
  \left\{ 
    \begin{array}{ccc}
      \lpquote Q \rpquote & & x \nameeq \lpquote P \rpquote \\
      x & & otherwise \\
    \end{array}
  \right. \nonumber
\end{eqnarray}

and $z$ is chosen distinct from $\quotep{P}$, $\quotep{Q}$, the free
names in $Q$, and all the names in $R$. Our $\alpha$-equivalence will
be built in the standard way from this substitution.

\begin{remark}\label{rem:no_self_referential_names}
  One consequence of these definitions is that $\forall P. \quotep{P}
  \not\in \freenames{P}$.
\end{remark}

\subsection{ Dynamic quote: an example }

Anticipating something of what's to come, consider applying the
substitution, $\widehat{\id{\{}u / z \id{\}}}$, to the following pair
of processes, $\lift{w}{y!(z)}$ and $w[ \lpquote y!(z) \rpquote ]$.

\begin{eqnarray}
	\lift{w}{y!(z)}\widehat{\id{\{}u / z \id{\}}}
		& = &
		\lift{w}{y!(u)} \nonumber\\
	w[ \lpquote y!(z) \rpquote ] \widehat{ \id{\{}u / z \id{\}} }
		& = &
		w[ \lpquote y!(z) \rpquote ] \nonumber
\end{eqnarray}

Because the body of the process between quotes is impervious to
substitution, we get radically different answers. In fact, by
examining the first process in an input context,
e.g. $x?(z).\lift{w}{y!(z)}$, we see that the process under the lift
operator may be shaped by prefixed inputs binding a name inside it. In
this sense, the lift operator will be seen as a way to dynamically
construct processes before reifying them as names.

Finally equipped with these standard features we can present the
dynamics of the calculus.

\subsubsection{Operational semantics} 

Finally, we introduce the computational dynamics. What marks these
algebras as distinct from other more traditionally studied algebraic
structures, e.g. vector spaces or polynomial rings, is the manner in
which dynamics is captured. In traditional structures, dynamics is typically
expressed through morphisms between such structures, as in linear maps
between vector spaces or morphisms between rings. In algebras
associated with the semantics of computation, the dynamics is
expressed as part of the algebraic structure itself, through a
reduction reduction relation typically denoted by $\red$. Below, we
give a recursive presentation of this relation for the calculus used
in the encoding.

$\red \subseteq \pi \times \pi$
$\red : \pi \to \mathcal{P}(\pi)$

\begin{mathpar}
  \inferrule* [lab=Comm] { \textsf{match}( x_{src}, x_{trgt} ) } { x_{trgt}?(y)P \; | \; x_{src}!\langle {Q} \rangle \red P\{\quotep{Q}/y}\} }
  \and \\
  \inferrule* [lab=Par] {{P} \red {P}'} {{{P} | {Q}} \red {{P}' | {Q}}}
  \and
  \inferrule* [lab=Equiv]{{{P} \scong {P}'} \andalso {{P}' \red {Q}'} \andalso {{Q}' \scong {Q}}}{{P} \red {Q}}
\end{mathpar}

\begin{eqnarray*}
  match_{\equiv} (\quotep{P},\quotep{Q}) & := & P \equiv Q \\
  match_{\dagger}(\quotep{P},\quotep{Q}) & := & \forall R. P|Q \red^{*} R => R \red^{*} 0 \\
  match_{K}(\quotep{P},\quotep{Q}) & := & K \mbox{ for some context } K
\end{eqnarray*}

$u?(x)P | u!\langle Q \rangle \red P\{\quotep{Q}/x\}$

%We write $\wred$ for $\red^*$, and $P\red$ if $\exists Q $ such that $ P \red Q$.
We write $P\red$ if $\exists Q $ such that $ P \red Q$ and $P\not\red$, otherwise.

\section{Replication}

As mentioned before, it is known that replication (and hence
recursion) can be implemented in a higher-order process algebra
\cite{SangiorgiWalker}. As our first example of calculation with the
machinery thus far presented we give the construction explicitly in
the {\rhoc}.

\begin{eqnarray}
	D_{x} & := & \prefix{x}{y}{(\binpar{\outputp{x}{y}}{@{y}})} \nonumber\\
	\bangp_{x}{P} & := & \binpar{{x}!\langle{\binpar{D_{x}}{P}}\rangle}{D_{x}} \nonumber
\end{eqnarray}

\begin{eqnarray}
	\bangp_{x}{P} & & \nonumber\\
	=
	& {x}!\langle{(\prefix{x}{y}{(\outputp{x}{y} | @{y})) | P}}\rangle 
	      | \prefix{x}{y}{(\outputp{x}{y} | @{y})} & \nonumber\\
	\red
	& (\outputp{x}{y} | @{y})\substn{\quotep{(\prefix{x}{y}{(@{y} | \outputp{x}{y})) | P}}}{y} & \nonumber\\
	=
	& \outputp{x}{\quotep{(\prefix{x}{y}{(\outputp{x}{y} | @{y})) | P}}}
	  | {(\prefix{x}{y}{(\outputp{x}{y} | @{y})) | P}} & \nonumber\\
	\red
	& \ldots & \nonumber\\
	\red^*
	& P | P | \ldots & \nonumber
\end{eqnarray}

Of course, this encoding, as an implementation, runs away, unfolding
$\bangp{P}$ eagerly. A lazier and more implementable replication
operator, restricted to input-guarded processes, may be obtained as follows.

\begin{eqnarray}
\bangp{\prefix{u}{v}{P}} 
	:= 
	\binpar{\lift{x}{\prefix{u}{v}{(\binpar{D(x)}{P})}}}{D(x)} \nonumber
\end{eqnarray}

\begin{remark}
  Note that the lazier definition still does not deal with summation
  or mixed summation (i.e. sums over input and output). The reader is
  invited to construct definitions of replication that deal with these
  features. 

  Further, the definitions are parameterized in a name, $x$. Can you,
  gentle reader, make a definition that eliminates this parameter and
  guarantees no accidental interaction between the replication
  machinery and the process being replicated -- i.e. no accidental
  sharing of names used by the process to get its work done and the
  name(s) used by the replication to effect copying. This latter
  revision of the definition of replication is crucial to obtaining
  the expected identity $!!P \sim !P$.
\end{remark}

\begin{remark}\label{rem:paradoxical_combinator}
  The reader familiar with the lambda calculus will have noticed the
  similarity between $D$ and the paradoxical combinator.

  [Ed. note: the existence of this seems to suggest we have to be more
  restrictive on the set of processes and names we admit if we are to
  support no-cloning.]
\end{remark}

\subsubsection{Bisimulation}

The computational dynamics gives rise to another kind of equivalence,
the equivalence of computational behavior. As previously mentioned
this is typically captured \emph{via} some form of bisimulation.

% The notion we use in this paper is weak barbed bisimulation
% \cite{milner91polyadicpi}.

The notion we use in this paper is derived from weak barbed
bisimulation \cite{milner91polyadicpi}. 

\begin{definition}
An \emph{observation relation}, $\downarrow_{\mathcal N}$, over a set
of names, $\mathcal N$, is the smallest relation satisfying the rules
below.

\infrule[Out-barb]{y \in {\mathcal N}, \; x \nameeq y}
		  {\outputp{x}{v} \downarrow_{\mathcal N} x}
\infrule[Par-barb]{\mbox{$P\downarrow_{\mathcal N} x$ or $Q\downarrow_{\mathcal N} x$}}
		  {\binpar{P}{Q} \downarrow_{\mathcal N} x}

We write $P \Downarrow_{\mathcal N} x$ if there is $Q$ such that 
$P \wred Q$ and $Q \downarrow_{\mathcal N} x$.
\end{definition}

\begin{definition}
%\label{def.bbisim}
An  ${\mathcal N}$-\emph{barbed bisimulation} over a set of names, ${\mathcal N}$, is a symmetric binary relation 
${\mathcal S}_{\mathcal N}$ between agents such that $P\rel{S}_{\mathcal N}Q$ implies:
\begin{enumerate}
\item If $P \red P'$ then $Q \wred Q'$ and $P'\rel{S}_{\mathcal N} Q'$.
\item If $P\downarrow_{\mathcal N} x$, then $Q\Downarrow_{\mathcal N} x$.
\end{enumerate}
$P$ is ${\mathcal N}$-barbed bisimilar to $Q$, written
$P \wbbisim_{\mathcal N} Q$, if $P \rel{S}_{\mathcal N} Q$ for some ${\mathcal N}$-barbed bisimulation ${\mathcal S}_{\mathcal N}$.
\end{definition}

$\mathcal{R} \subseteq \pi \times \pi$

$P \mathcal{R} Q => \forall P'. P \red P' \Rightarrow \exists Q'. Q \red Q', P' \mathcal{R} Q'$

$P \vdash x \Rightarrow Q \vdash x$

\begin{mathpar}
  \inferrule*[lab=Out-barb]{x \nameeq y}{{y}!\langle{Q}\rangle \vdash x}
  \and
  \inferrule*[lab=Par-barb]{\mbox{$P\vdash x$ or $Q\vdash x$}}{\binpar{P}{Q} \vdash x}
\end{mathpar}

\subsubsection{Contexts}

One of the principle advantages of computational calculi like the
$\pi$-calculus is a well-defined notion of context,
contextual-equivalence and a correlation between
contextual-equivalence and notions of bisimulation. The notion of
context allows the decomposition of a process into (sub-)process and
its syntactic environment, its context. Thus, a context may be
thought of as a process with a ``hole'' (written $\Box$) in it. The
application of a context $M$ to a process $P$, written $M[P]$, is
tantamount to filling the hole in $M$ with $P$. In this paper we do
not need the full weight of this theory, but do make use of the notion
of context in the proof the main theorem. 

\begin{mathpar}
  \inferrule* [lab=summation] {} {{M_{M},M_{N}} \bc \Box \;|\; x.M_{A} \;|\; M_{M}+M_{N}}
  \and
  \inferrule* [lab=agent] {} {{M_{A}} \bc (\vec{x})M_{P} \;| \; \clift{P_0,\ldots,M_{P},\ldots,P_N}}
  \and \\
  \inferrule* [lab=process] {} {{M_{P}} \bc M_{N} \;| \;P|M_{P} }
\end{mathpar} 

\begin{mathpar}
  \inferrule* [lab=sychronization] {} {M_{N} \bc \Box \;|\; x?M_{F} \;|\; x!M_{C}}
  \and
  \inferrule* [lab=abstraction] {} {{M_{F}} \bc (x)M_{P} }
  \and
  \inferrule* [lab=concretion] {} {{M_{C}} \bc \langle M_{P} \rangle }
  \and \\
  \inferrule* [lab=process] {} {{M_{P}} \bc M_{N} \;| \;P|M_{P} }
\end{mathpar}

\begin{definition}[contextual application] Given a context $M$, and
  process $P$, we define the \emph{contextual application}, $M[P] :=
  M\{P/\Box\}$. That is, the contextual application of M to P is the
  substitution of $P$ for $\Box$ in $M$.
\end{definition}

$\meaningof{-} : L \to \mathcal{P}(\pi)$

\begin{mathpar}
  \inferrule* [lab=collection] {} {\meaningof{true} = \pi, \and \meaningof{~E} = \pi \setminus \meaningof{E}, \and \meaningof{E_{1} \& E_{2}} = \meaningof{E_{1}} \cap \meaningof{E_{2}}}
\end{mathpar}

\begin{mathpar}
  \inferrule* [lab=structure] {} {\meaningof{0} = \{ P \in \pi | P \equiv 0 \}, \and \\ \meaningof{E_1 | E_2} = \{ P \in \pi | P \equiv P_{1} | P_{2}, P_{1} \in \meaningof{E_{1}}, P_{2} \in \meaningof{E_2}\} }
\end{mathpar}

\begin{mathpar}
 \inferrule* [lab=behavior] {} {\meaningof{\langle a?b \rangle E} = \{ P \in \pi | P \equiv Q | u?(y)P', \\ \and \\\\ \and \\ \;\;\; u \in \meaningof{a}, \forall z.P'\{z/y\} \in \meaningof{E\{z/b\}}\}, \and \\ \meaningof{a!E} = \{ P \in \pi | P \equiv Q | x!\langle P' \rangle, x \in \meaningof{a} P' \in \meaningof{E}\} }
\end{mathpar}

\begin{mathpar}
 \inferrule* [lab=nominal] {} {\meaningof{\quotep{E}} = \{ \quotep{P} \in \quotep{\pi} | P \in \meaningof{E} \}, \and \meaningof{\quotep{P}} = \{ \quotep{Q} \in \quotep{\pi} | P \equiv Q \} \and \\ \meaningof{@\quotep{E}} = \{ P \in \pi | P \equiv @x, x \in \meaningof{E} \}}
\end{mathpar}

\begin{eqnarray*}
  \\
  \meaningof{-} : TS \to ST
\end{eqnarray*}

\begin{eqnarray*}
  \\
  L : TS \to ST
\end{eqnarray*}

\begin{eqnarray*}
  \\
  P \models E \iff P \in \meaningof{E}
\end{eqnarray*}

\begin{eqnarray*}
  P \approx_{L} Q \iff \forall E \in L. P \models E \iff Q \models E
\end{eqnarray*}

\begin{eqnarray*}
  P \approx_{K} Q
\end{eqnarray*}

\begin{eqnarray*}
  P \approx Q
\end{eqnarray*}

$\approx_{K} = \approx = \approx_{L}$

\subsubsection{Contextual duality}

Note that contexts extend the quotation operation to a family of
operations from processes to names. Given a context, $M$, we can
define a \emph{nominal context}, $\quotep{M}$ by $\quotep{M}[P] :=
\quotep{M[P]}$. To foreshadow what is to come we observe that these
operations enjoy a duality with processes very much like the duality
between vectors and maps from vectors to scalars.

Further, because the calculus is essentially higher-order, we have a
correspondence between contexts and processes. More specifically,
given a name $x$ and a context $M$ we can construct $M^{*}_{x}$ such
that 

\begin{mathpar}
  M^{*}_{x} | \lift{x}{P} \red M[P]
\end{mathpar}

namely,

\begin{mathpar}
  M^{*}_{x} := x?(u).M[\dropn{u}]
\end{mathpar}

The dependence of $M^{*}_{x}$ on a name makes it an abstraction, 

\begin{mathpar}
  M^{*} := (x)x?(u).M[\dropn{u}]
\end{mathpar}

\subsection{Additional notation}

It will sometimes be convenient to denote the process a name
quotes. We already have the notation $x = \quotep{P}$, but it will be
convenient to introduce an alternate notation, $\procn{x}$, when we
want to emphasize the connection to the use of the name. Note that, by
virtue of name equivalence, $\quotep{\procn{x}} \nameeq x$; so, the
notation is consistent with previous definitions.

Further, because names have structure it is possible to effect
substitutions on the basis of that structure. This means we need to
upgrade our notation for substitutions, which we accomplish by
adapting comprehension notation. Thus,

\begin{mathpar}
  P\{ y / x : x \in S \}
\end{mathpar}

is interpreted to mean the process derived from P by replacing (in a
capture-avoiding manner) each occurrence of $x$ in $S$ by $y$. For example,

\begin{mathpar}
  P\{ \quotep{\procn{x}|\procn{x}} / x : x \in \freenames{P} \}
\end{mathpar}

will replace each (occurrence) of a free name $x$ in $P$ by
$\quotep{\procn{x}|\procn{x}}$.

Also, we will avail ourselves of the notation $x^{L}$ and $x^{R}$ to
denote injections of a name into disjoint copies of the name
space. There are numerous ways to accomplish this. One example can be
found in \cite{MeredithR05}. This notation overloads to vectors of
names: $\vec{x}^{\pi} := (x_{i}^{\pi} \; : \; 0 \leq i < |\vec{x}| )$ where $\pi \in \{L,R\}$.

We also use $P^{\Box} := P|\Box$.

In \cite{MeredithR05} an interpretation of the new operator is
given. It turns out that there are several possible interpretations
all enjoying the requisite algebraic properties of the operator (see
\cite{milner91polyadicpi}). We will therefore make liberal use of
$(\nu\; \vec{x})P$.

% subsection the_syntax_and_semantics_of_the_notation_system (end)   

\input{qm2pi.qmops} 

\input{qm2pi.sterngerlach} 

\input{qm2pi.metric} 

% section concurrent_process_calculi (end)

%\input{qm2pi.proofsketch}

% section proof sketch (end)

%\input{qm2pi.slviaknots} 

% section spatial logic via knots (end)

\input{qm2pi.conclusion}

% section conclusion (end)

%\input{qm2pi.dtcodes} 

% section wiring algorithm (end)

\input{qm2pi.ack} 

% section acknowledgments (end)

\newpage


\bibliographystyle{plain}   
\bibliography{../../biblios/main.bib}

\input{qm2pi.rhodetails}

\end{document}

 

%\ifpdf
%\usepackage[pdftex]{graphicx}
%\else
%\usepackage{graphicx}
%\fi

 % \ifpdf
%  \usepackage{pdfsync}
%  \if


%\title{Brief Article}
%\author{David F. Snyder}
%\author{L.G. Meredith}

%\address{Dept. of Math., Texas State University--San Marcos, San Marcos, TX 78666}
       
\pagestyle{empty}


\begin{document}

\lstset{language=[Objective]Caml,frame=shadowbox}

\documentclass[12pt]{llncs}
%\documentclass{jktr}

\usepackage[pdftex]{hyperref}                   
\usepackage {listings}
\usepackage {mathpartir}
\usepackage{bcprules}
%\usepackage{listings}
                       
\usepackage{graphicx} 
%\usepackage[margins=2.5cm,nohead,nofoot]{geometry}
%\usepackage{geometry}
\usepackage{amsfonts}
\usepackage{amstext}
\usepackage{latexsym}
\usepackage{amssymb}
\usepackage{color}


%\include{myPreamble}
\include{qm2pi.local} 

%\ifpdf
%\usepackage[pdftex]{graphicx}
%\else
%\usepackage{graphicx}
%\fi

 % \ifpdf
%  \usepackage{pdfsync}
%  \if


%\title{Brief Article}
%\author{David F. Snyder}
%\author{L.G. Meredith}

%\address{Dept. of Math., Texas State University--San Marcos, San Marcos, TX 78666}
       
\pagestyle{empty}


\begin{document}

\lstset{language=[Objective]Caml,frame=shadowbox}

\input{qm2pi.front}

% section front matter (end)

\input{qm2pi.intro} 
 
% section introduction (end)

% \input{qm2pi.knotations} 

% section notation (end)

\input{qm2pi.process.calculi} 

% section concurrent_process_calculi_and_spatial_logics_ (end)
    
%\input{qm2pi.knots2pi} 

%\input{qm2pi.trefoil} 

%\input{qm2pi.mainthm} 

% subsection basic_interpretation (end)

%\input{qm2pi.rho.presentation} 
\subsection{The syntax and semantics of the notation system}\label{sub:the_syntax_and_semantics_of_the_notation_system} % (fold)

We now summarize a technical presentation of the calculus that
embodies our theory of dynamics. The typical presentation of such a
calculus follows the style of giving generators and relations on
them. The grammar, below, describing term constructors, freely
generates the set of processes, $\Proc$. This set is then quotiented
by a relation known as structural congruence and it is over this set
that the notion of dynamics is expressed. This presentation is
essentially that of \cite{MeredithR05} with the addition of
polyadicity and summation. For readability we have relegated some of
the technical subtleties to an appendix.

\subsubsection{Process grammar}\label{subsub:process_grammar}

\begin{mathpar}
  \inferrule* [lab=synchronization] {} {{M} \bc \pzero \;|\; x?F \;|\; x!C }
  \and
  \inferrule* [lab=abstraction] {} {{F} \bc (x)P}
  \and
  \inferrule* [lab=concretion] {} {{C} \bc \langle Q \rangle}
  \and
  \inferrule* [lab=process] {} {{P,Q} \bc M \;| \;P|Q \;|\; @{x}}
  \and
  \inferrule* [lab=name] {} {{x} \bc \quotep{P}}
\end{mathpar} 

Note that $\vec{x}$ (resp. $\vec{P}$) denotes a vector of names
(resp. processes) of length $|\vec{x}|$ (resp. $|\vec{P}|$). We adopt
the following useful abbreviations.

\begin{mathpar}
   x?(\vec{y}).P := x.(\vec{y})P \and  x\clift{\vec{P}} := x.\clift{\vec{P}}
   \and x!(y) := \lift{x}{\dropn{y}}
   \and \Pi_{i=0}^{n-1}P_i := P_0 | \ldots | P_{n-1}
\end{mathpar}

\subsubsection{Structural congruence}

\paragraph{Free and bound names and alpha-equivalence.} At the
core of structural equivalence is alpha-equivalence which identifies
process that are the same up to a change of variable. Formally, we
recognize the distinction between free and bound names. The free names
of a process, $\freenames{P}$, may be calculated recursively as
follows:

\begin{mathpar}
\freenames{\pzero} := \emptyset
  \and \\
  \freenames{x?(y).P} := \{ x \} \cup (\freenames{P} \setminus \{ y \})
  \and 
  \freenames{x!\langle P \rangle} := \{ x \} \cup \{ P \} 
  \and \\
  \freenames{P|Q} := \freenames{P} \cup \freenames{Q}
  \and \\
  \freenames{@{x}} := \{ x \}
\end{mathpar}

$\pi$
$\quotep{\pi}$

$\freenames{-} : \pi \to \mathcal{P}(\quotep{\pi})$

\begin{eqnarray*}
  \freenames{\pzero} & := & \emptyset \\
  \freenames{x?(y).P} & := & \{ x \} \cup (\freenames{P} \setminus \{ y \}) \\
  \freenames{x!\langle P \rangle} & := & \{ x \} \cup \{ P \} \\
  \freenames{P|Q} & := & \freenames{P} \cup \freenames{Q} \\
  \freenames{\dropn{x}} & := & \{ x \}
\end{eqnarray*}

The bound names of a process, $\boundnames{P}$, are those names occurring in $P$
that are not free. For example, in $x?(y).0$, the name $x$ is free, while $y$ is bound.

\begin{mathpar}
  \inferrule* [lab=monoidal-laws] {} { P|Q \equiv Q|P \and P|0 \equiv P \and P|(Q|R) \equiv (P|Q)|R }
\end{mathpar}

\begin{mathpar}
  \inferrule* [lab=alpha-equivalence] {} { (x)P \equiv (y)P\{y/x\} \and y \not\in \freenames{P} }
\end{mathpar}

\begin{definition}
Then two processes, $P,Q$, are alpha-equivalent if $P = Q\{\vec{y}/\vec{x}\}$ for
some $\vec{x} \in \boundnames{Q},\vec{y} \in \boundnames{P}$, where $Q\{\vec{y}/\vec{x}\}$
denotes the capture-avoiding substitution of $\vec{y}$ for $\vec{x}$ in $Q$.
\end{definition}

\begin{definition}
  The {\em structural congruence} \cite{SangiorgiWalker} , $\equiv$,
  between processes is the least congruence containing
  alpha-equivalence, satisfying the abelian monoid laws
  (associativity, commutativity and $\pzero$ as identity) for parallel
  composition $|$ and for summation $+$.
\end{definition}

\subsection{Name equivalence}

We take name equivalence, written $\nameeq$, to be the smallest
equivalence relation generated by the following rules.

\begin{mathpar}
\inferrule*[lab=Quote-drop]
{ }
{ \quotep{@{x}} \nameeq x }

\inferrule*[lab=Struct-equiv]
{ P \scong Q }
{ \quotep{P} \nameeq \quotep{Q} }
\end{mathpar}

The astute reader will have noticed that the mutual recursion of names
and processes imposes a mutual recursion on alpha-equivalence and
structural equivalence via name-equivalence. Fortunately, all of this
works out pleasantly and we may calculate in the natural way, free of
concern. The reader interested in the details is referred to the
appendix \ref{appendix:rho_details}.

\subsection{Substitution}

We use $\Proc$ for the set of processes, $\QProc$ for the set of
names, and $\id{\{}\vec{y} / \vec{x} \id{\}}$ to denote partial maps,
$s : \QProc \rightarrow \QProc$. A map, $s$ lifts, uniquely, to a map
on process terms, $\widehat{s} : \Proc \rightarrow \Proc$ by the
following equations.

\begin{mathpar}
  (0) \psubstp{Q}{P} := 0 \\
  (R \juxtap S) \psubstp{Q}{P}
  :=    
  (R)\psubstp{Q}{P} \juxtap (S) \psubstp{Q}{P} \\
  (x?(y).R) \psubstp{Q}{P}    
  :=    
  (x)\substp{Q}{P} (z)\concat( (R \psubstn{z}{y}) \psubstp{Q}{P} ) \\
  (\lift{x}{R}) \psubstp{Q}{P}  
  :=
  \lift{(x)\substp{Q}{P}}{ R \psubstp{Q}{P} } \\
%   (\dropn{x})  \psubstp{Q}{P}       
%   := 
%   \left\{ 
%     \begin{array}{ccc} 
%       \dropn{\quotep{Q}} & & x \nameeq \quotep{P} \\
%       \dropn{x} & & otherwise \\
%     \end{array}
%   \right. 
  (\dropn{x})  \psubstp{Q}{P}       
  := 
  \left\{ 
    \begin{array}{ccc} 
      Q & & x \nameeq \quotep{P} \\
      \dropn{x} & & otherwise \\
    \end{array}
  \right.
\end{mathpar}
 

where

\begin{eqnarray}
  (x)\id{\{} \lpquote Q \rpquote / \lpquote P \rpquote \id{\}}            = 
  \left\{ 
    \begin{array}{ccc}
      \lpquote Q \rpquote & & x \nameeq \lpquote P \rpquote \\
      x & & otherwise \\
    \end{array}
  \right. \nonumber
\end{eqnarray}

and $z$ is chosen distinct from $\quotep{P}$, $\quotep{Q}$, the free
names in $Q$, and all the names in $R$. Our $\alpha$-equivalence will
be built in the standard way from this substitution.

\begin{remark}\label{rem:no_self_referential_names}
  One consequence of these definitions is that $\forall P. \quotep{P}
  \not\in \freenames{P}$.
\end{remark}

\subsection{ Dynamic quote: an example }

Anticipating something of what's to come, consider applying the
substitution, $\widehat{\id{\{}u / z \id{\}}}$, to the following pair
of processes, $\lift{w}{y!(z)}$ and $w[ \lpquote y!(z) \rpquote ]$.

\begin{eqnarray}
	\lift{w}{y!(z)}\widehat{\id{\{}u / z \id{\}}}
		& = &
		\lift{w}{y!(u)} \nonumber\\
	w[ \lpquote y!(z) \rpquote ] \widehat{ \id{\{}u / z \id{\}} }
		& = &
		w[ \lpquote y!(z) \rpquote ] \nonumber
\end{eqnarray}

Because the body of the process between quotes is impervious to
substitution, we get radically different answers. In fact, by
examining the first process in an input context,
e.g. $x?(z).\lift{w}{y!(z)}$, we see that the process under the lift
operator may be shaped by prefixed inputs binding a name inside it. In
this sense, the lift operator will be seen as a way to dynamically
construct processes before reifying them as names.

Finally equipped with these standard features we can present the
dynamics of the calculus.

\subsubsection{Operational semantics} 

Finally, we introduce the computational dynamics. What marks these
algebras as distinct from other more traditionally studied algebraic
structures, e.g. vector spaces or polynomial rings, is the manner in
which dynamics is captured. In traditional structures, dynamics is typically
expressed through morphisms between such structures, as in linear maps
between vector spaces or morphisms between rings. In algebras
associated with the semantics of computation, the dynamics is
expressed as part of the algebraic structure itself, through a
reduction reduction relation typically denoted by $\red$. Below, we
give a recursive presentation of this relation for the calculus used
in the encoding.

$\red \subseteq \pi \times \pi$
$\red : \pi \to \mathcal{P}(\pi)$

\begin{mathpar}
  \inferrule* [lab=Comm] { \textsf{match}( x_{src}, x_{trgt} ) } { x_{trgt}?(y)P \; | \; x_{src}!\langle {Q} \rangle \red P\{\quotep{Q}/y}\} }
  \and \\
  \inferrule* [lab=Par] {{P} \red {P}'} {{{P} | {Q}} \red {{P}' | {Q}}}
  \and
  \inferrule* [lab=Equiv]{{{P} \scong {P}'} \andalso {{P}' \red {Q}'} \andalso {{Q}' \scong {Q}}}{{P} \red {Q}}
\end{mathpar}

\begin{eqnarray*}
  match_{\equiv} (\quotep{P},\quotep{Q}) & := & P \equiv Q \\
  match_{\dagger}(\quotep{P},\quotep{Q}) & := & \forall R. P|Q \red^{*} R => R \red^{*} 0 \\
  match_{K}(\quotep{P},\quotep{Q}) & := & K \mbox{ for some context } K
\end{eqnarray*}

$u?(x)P | u!\langle Q \rangle \red P\{\quotep{Q}/x\}$

%We write $\wred$ for $\red^*$, and $P\red$ if $\exists Q $ such that $ P \red Q$.
We write $P\red$ if $\exists Q $ such that $ P \red Q$ and $P\not\red$, otherwise.

\section{Replication}

As mentioned before, it is known that replication (and hence
recursion) can be implemented in a higher-order process algebra
\cite{SangiorgiWalker}. As our first example of calculation with the
machinery thus far presented we give the construction explicitly in
the {\rhoc}.

\begin{eqnarray}
	D_{x} & := & \prefix{x}{y}{(\binpar{\outputp{x}{y}}{@{y}})} \nonumber\\
	\bangp_{x}{P} & := & \binpar{{x}!\langle{\binpar{D_{x}}{P}}\rangle}{D_{x}} \nonumber
\end{eqnarray}

\begin{eqnarray}
	\bangp_{x}{P} & & \nonumber\\
	=
	& {x}!\langle{(\prefix{x}{y}{(\outputp{x}{y} | @{y})) | P}}\rangle 
	      | \prefix{x}{y}{(\outputp{x}{y} | @{y})} & \nonumber\\
	\red
	& (\outputp{x}{y} | @{y})\substn{\quotep{(\prefix{x}{y}{(@{y} | \outputp{x}{y})) | P}}}{y} & \nonumber\\
	=
	& \outputp{x}{\quotep{(\prefix{x}{y}{(\outputp{x}{y} | @{y})) | P}}}
	  | {(\prefix{x}{y}{(\outputp{x}{y} | @{y})) | P}} & \nonumber\\
	\red
	& \ldots & \nonumber\\
	\red^*
	& P | P | \ldots & \nonumber
\end{eqnarray}

Of course, this encoding, as an implementation, runs away, unfolding
$\bangp{P}$ eagerly. A lazier and more implementable replication
operator, restricted to input-guarded processes, may be obtained as follows.

\begin{eqnarray}
\bangp{\prefix{u}{v}{P}} 
	:= 
	\binpar{\lift{x}{\prefix{u}{v}{(\binpar{D(x)}{P})}}}{D(x)} \nonumber
\end{eqnarray}

\begin{remark}
  Note that the lazier definition still does not deal with summation
  or mixed summation (i.e. sums over input and output). The reader is
  invited to construct definitions of replication that deal with these
  features. 

  Further, the definitions are parameterized in a name, $x$. Can you,
  gentle reader, make a definition that eliminates this parameter and
  guarantees no accidental interaction between the replication
  machinery and the process being replicated -- i.e. no accidental
  sharing of names used by the process to get its work done and the
  name(s) used by the replication to effect copying. This latter
  revision of the definition of replication is crucial to obtaining
  the expected identity $!!P \sim !P$.
\end{remark}

\begin{remark}\label{rem:paradoxical_combinator}
  The reader familiar with the lambda calculus will have noticed the
  similarity between $D$ and the paradoxical combinator.

  [Ed. note: the existence of this seems to suggest we have to be more
  restrictive on the set of processes and names we admit if we are to
  support no-cloning.]
\end{remark}

\subsubsection{Bisimulation}

The computational dynamics gives rise to another kind of equivalence,
the equivalence of computational behavior. As previously mentioned
this is typically captured \emph{via} some form of bisimulation.

% The notion we use in this paper is weak barbed bisimulation
% \cite{milner91polyadicpi}.

The notion we use in this paper is derived from weak barbed
bisimulation \cite{milner91polyadicpi}. 

\begin{definition}
An \emph{observation relation}, $\downarrow_{\mathcal N}$, over a set
of names, $\mathcal N$, is the smallest relation satisfying the rules
below.

\infrule[Out-barb]{y \in {\mathcal N}, \; x \nameeq y}
		  {\outputp{x}{v} \downarrow_{\mathcal N} x}
\infrule[Par-barb]{\mbox{$P\downarrow_{\mathcal N} x$ or $Q\downarrow_{\mathcal N} x$}}
		  {\binpar{P}{Q} \downarrow_{\mathcal N} x}

We write $P \Downarrow_{\mathcal N} x$ if there is $Q$ such that 
$P \wred Q$ and $Q \downarrow_{\mathcal N} x$.
\end{definition}

\begin{definition}
%\label{def.bbisim}
An  ${\mathcal N}$-\emph{barbed bisimulation} over a set of names, ${\mathcal N}$, is a symmetric binary relation 
${\mathcal S}_{\mathcal N}$ between agents such that $P\rel{S}_{\mathcal N}Q$ implies:
\begin{enumerate}
\item If $P \red P'$ then $Q \wred Q'$ and $P'\rel{S}_{\mathcal N} Q'$.
\item If $P\downarrow_{\mathcal N} x$, then $Q\Downarrow_{\mathcal N} x$.
\end{enumerate}
$P$ is ${\mathcal N}$-barbed bisimilar to $Q$, written
$P \wbbisim_{\mathcal N} Q$, if $P \rel{S}_{\mathcal N} Q$ for some ${\mathcal N}$-barbed bisimulation ${\mathcal S}_{\mathcal N}$.
\end{definition}

$\mathcal{R} \subseteq \pi \times \pi$

$P \mathcal{R} Q => \forall P'. P \red P' \Rightarrow \exists Q'. Q \red Q', P' \mathcal{R} Q'$

$P \vdash x \Rightarrow Q \vdash x$

\begin{mathpar}
  \inferrule*[lab=Out-barb]{x \nameeq y}{{y}!\langle{Q}\rangle \vdash x}
  \and
  \inferrule*[lab=Par-barb]{\mbox{$P\vdash x$ or $Q\vdash x$}}{\binpar{P}{Q} \vdash x}
\end{mathpar}

\subsubsection{Contexts}

One of the principle advantages of computational calculi like the
$\pi$-calculus is a well-defined notion of context,
contextual-equivalence and a correlation between
contextual-equivalence and notions of bisimulation. The notion of
context allows the decomposition of a process into (sub-)process and
its syntactic environment, its context. Thus, a context may be
thought of as a process with a ``hole'' (written $\Box$) in it. The
application of a context $M$ to a process $P$, written $M[P]$, is
tantamount to filling the hole in $M$ with $P$. In this paper we do
not need the full weight of this theory, but do make use of the notion
of context in the proof the main theorem. 

\begin{mathpar}
  \inferrule* [lab=summation] {} {{M_{M},M_{N}} \bc \Box \;|\; x.M_{A} \;|\; M_{M}+M_{N}}
  \and
  \inferrule* [lab=agent] {} {{M_{A}} \bc (\vec{x})M_{P} \;| \; \clift{P_0,\ldots,M_{P},\ldots,P_N}}
  \and \\
  \inferrule* [lab=process] {} {{M_{P}} \bc M_{N} \;| \;P|M_{P} }
\end{mathpar} 

\begin{mathpar}
  \inferrule* [lab=sychronization] {} {M_{N} \bc \Box \;|\; x?M_{F} \;|\; x!M_{C}}
  \and
  \inferrule* [lab=abstraction] {} {{M_{F}} \bc (x)M_{P} }
  \and
  \inferrule* [lab=concretion] {} {{M_{C}} \bc \langle M_{P} \rangle }
  \and \\
  \inferrule* [lab=process] {} {{M_{P}} \bc M_{N} \;| \;P|M_{P} }
\end{mathpar}

\begin{definition}[contextual application] Given a context $M$, and
  process $P$, we define the \emph{contextual application}, $M[P] :=
  M\{P/\Box\}$. That is, the contextual application of M to P is the
  substitution of $P$ for $\Box$ in $M$.
\end{definition}

$\meaningof{-} : L \to \mathcal{P}(\pi)$

\begin{mathpar}
  \inferrule* [lab=collection] {} {\meaningof{true} = \pi, \and \meaningof{~E} = \pi \setminus \meaningof{E}, \and \meaningof{E_{1} \& E_{2}} = \meaningof{E_{1}} \cap \meaningof{E_{2}}}
\end{mathpar}

\begin{mathpar}
  \inferrule* [lab=structure] {} {\meaningof{0} = \{ P \in \pi | P \equiv 0 \}, \and \\ \meaningof{E_1 | E_2} = \{ P \in \pi | P \equiv P_{1} | P_{2}, P_{1} \in \meaningof{E_{1}}, P_{2} \in \meaningof{E_2}\} }
\end{mathpar}

\begin{mathpar}
 \inferrule* [lab=behavior] {} {\meaningof{\langle a?b \rangle E} = \{ P \in \pi | P \equiv Q | u?(y)P', \\ \and \\\\ \and \\ \;\;\; u \in \meaningof{a}, \forall z.P'\{z/y\} \in \meaningof{E\{z/b\}}\}, \and \\ \meaningof{a!E} = \{ P \in \pi | P \equiv Q | x!\langle P' \rangle, x \in \meaningof{a} P' \in \meaningof{E}\} }
\end{mathpar}

\begin{mathpar}
 \inferrule* [lab=nominal] {} {\meaningof{\quotep{E}} = \{ \quotep{P} \in \quotep{\pi} | P \in \meaningof{E} \}, \and \meaningof{\quotep{P}} = \{ \quotep{Q} \in \quotep{\pi} | P \equiv Q \} \and \\ \meaningof{@\quotep{E}} = \{ P \in \pi | P \equiv @x, x \in \meaningof{E} \}}
\end{mathpar}

\begin{eqnarray*}
  \\
  \meaningof{-} : TS \to ST
\end{eqnarray*}

\begin{eqnarray*}
  \\
  L : TS \to ST
\end{eqnarray*}

\begin{eqnarray*}
  \\
  P \models E \iff P \in \meaningof{E}
\end{eqnarray*}

\begin{eqnarray*}
  P \approx_{L} Q \iff \forall E \in L. P \models E \iff Q \models E
\end{eqnarray*}

\begin{eqnarray*}
  P \approx_{K} Q
\end{eqnarray*}

\begin{eqnarray*}
  P \approx Q
\end{eqnarray*}

$\approx_{K} = \approx = \approx_{L}$

\subsubsection{Contextual duality}

Note that contexts extend the quotation operation to a family of
operations from processes to names. Given a context, $M$, we can
define a \emph{nominal context}, $\quotep{M}$ by $\quotep{M}[P] :=
\quotep{M[P]}$. To foreshadow what is to come we observe that these
operations enjoy a duality with processes very much like the duality
between vectors and maps from vectors to scalars.

Further, because the calculus is essentially higher-order, we have a
correspondence between contexts and processes. More specifically,
given a name $x$ and a context $M$ we can construct $M^{*}_{x}$ such
that 

\begin{mathpar}
  M^{*}_{x} | \lift{x}{P} \red M[P]
\end{mathpar}

namely,

\begin{mathpar}
  M^{*}_{x} := x?(u).M[\dropn{u}]
\end{mathpar}

The dependence of $M^{*}_{x}$ on a name makes it an abstraction, 

\begin{mathpar}
  M^{*} := (x)x?(u).M[\dropn{u}]
\end{mathpar}

\subsection{Additional notation}

It will sometimes be convenient to denote the process a name
quotes. We already have the notation $x = \quotep{P}$, but it will be
convenient to introduce an alternate notation, $\procn{x}$, when we
want to emphasize the connection to the use of the name. Note that, by
virtue of name equivalence, $\quotep{\procn{x}} \nameeq x$; so, the
notation is consistent with previous definitions.

Further, because names have structure it is possible to effect
substitutions on the basis of that structure. This means we need to
upgrade our notation for substitutions, which we accomplish by
adapting comprehension notation. Thus,

\begin{mathpar}
  P\{ y / x : x \in S \}
\end{mathpar}

is interpreted to mean the process derived from P by replacing (in a
capture-avoiding manner) each occurrence of $x$ in $S$ by $y$. For example,

\begin{mathpar}
  P\{ \quotep{\procn{x}|\procn{x}} / x : x \in \freenames{P} \}
\end{mathpar}

will replace each (occurrence) of a free name $x$ in $P$ by
$\quotep{\procn{x}|\procn{x}}$.

Also, we will avail ourselves of the notation $x^{L}$ and $x^{R}$ to
denote injections of a name into disjoint copies of the name
space. There are numerous ways to accomplish this. One example can be
found in \cite{MeredithR05}. This notation overloads to vectors of
names: $\vec{x}^{\pi} := (x_{i}^{\pi} \; : \; 0 \leq i < |\vec{x}| )$ where $\pi \in \{L,R\}$.

We also use $P^{\Box} := P|\Box$.

In \cite{MeredithR05} an interpretation of the new operator is
given. It turns out that there are several possible interpretations
all enjoying the requisite algebraic properties of the operator (see
\cite{milner91polyadicpi}). We will therefore make liberal use of
$(\nu\; \vec{x})P$.

% subsection the_syntax_and_semantics_of_the_notation_system (end)   

\input{qm2pi.qmops} 

\input{qm2pi.sterngerlach} 

\input{qm2pi.metric} 

% section concurrent_process_calculi (end)

%\input{qm2pi.proofsketch}

% section proof sketch (end)

%\input{qm2pi.slviaknots} 

% section spatial logic via knots (end)

\input{qm2pi.conclusion}

% section conclusion (end)

%\input{qm2pi.dtcodes} 

% section wiring algorithm (end)

\input{qm2pi.ack} 

% section acknowledgments (end)

\newpage


\bibliographystyle{plain}   
\bibliography{../../biblios/main.bib}

\input{qm2pi.rhodetails}

\end{document}



% section front matter (end)

\section{Introduction}\label{sec:introduction} % (fold)
In this draft of the material i am going to have to dispense with the
usual writing conventions adopted in papers on these topics. i'm going
to have adopt whatever tone i need at the time i'm writing up the
calculations. Sometimes this may be very conversational; others it may
be the barest mathematical grunts; others still it may be that i have
lifted text from one of my other papers because the exposition of some
point was better said there. i hope that my readers are not unduly put
out by this decision. i'm not doing this to flout convention or be
rebellious. i find these calculations very technically challenging. To
keep everything going technically, something has to give; i have to
let go of some cognitive burden. So, the academic writing style --
with all of its trade-offs in terms of facilitating technical
communication -- is what i'm letting go of. Perhaps subsequent drafts
can be tightened and polished, but for now, i'm going to speak as if
we were sitting together in a coffee shop with a laptop, wifi and a
pad of paper and a pencil.

So, here's what i have to say. We -- you and i, comfortably ensconced
in our coffee shop and well-equipped with our tools -- can realize and
carry out the calculations of quantum mechanics over a very different
formal theory of dynamics, a formal theory of dynamics that
corresponds to a theory of concurrent computation with
\emph{reflection}. It has the advantage that the underlying theory is
already `quantized', but supports analogues all of the continuuous
operations. Strikingly, this underlying theory has recently been
connected with a notion of metric that we can show, by calculating
together, coincides with the metric induced by the inner product.

There are a lot of reasons why you might be interested in seeing
calculations of this form. Here's why i'm interested. For the past
several centuries there has been no competitor to the ``Newtonian''
account of dynamics. As a result the predominant share of accounts of
dynamical systems and situations have had to be formulated in terms of
the Newtonian machinery. i view this as an intellectually dangerous
position to occupy. Everything, despite it's intrinsic shape, turns
into a nail to be hit with this hammer. Recently, however, the theory
of computation has matured to the point where we have candidates for
theories of dynamics that offer very different perspective on
reasoning about dynamical systems and situations. Testing these
candidates against very successful accounts of dynamical situations,
like quantum mechanics, is going to give us some sense of how mature
they are and some measure of the quality of these accounts of
dynamics.

\subsection{Summary of contributions and outline of paper}

So, we're going to develop an interpretation of the operations of
quantum mechanics normally interpreted by Hilbert spaces and
operators. We're going to do this over a theory of computation. Note
that this is very different than the usual quantum computation program
which develops notions of computation over quantum mechanics. Rather,
we are developing a story that aligns with Wheeler's slogan: It from
Bit. To do this we will first provide an account of the theory of
computation at play here. Then we will dive into a calculation-driven
interpretation of the operations of quantum mechanics.

The reason we take this approach is that -- until very recently --
there hasn't been an axiomatic account of quantum mechanics. As a
result there has been no sharp delineation of the mathematical theory
supporting interpretation of the physical theory and the physical
theory, itself. So, ambient features of the maths are free to be
exploited (or supressed) without a real accounting of their physical
relevance. There is no sharp statement ``here's the physical theory''
qua \emph{theory} and ``here's the mathematical interpretation''
enabling a judgment of how faithful the interpretation is -- apart
from experimental observation. When there is an axiomatic account we
can judge how well a given mathematical formalism supports an
interpretation of the axioms, independent of
experimentation. Likewise, we can judge how well we have captured our
physical evidence and experience with our axiomatics, independent of
any specific mathematical implementation, with accidental detail that
may or may not have physical significance. 

In lieu of a fully fleshed out and vetted axiomatic account of quantum
mechanics, interpreting the operational notions in service of modeling
physical systems will have to suffice. In other words, we are not in
the business of providing a model of Hilbert spaces and operators. We
are in the business of providing a model of quantum mechanics because
we are motivated by testing our notions of dynamics against physical
theory; and, the predictive calculations of the physical theory must
serve as the best formulation -- shy of a fully fleshed out axiomatic
account -- of the physical theory itself (as they have for scientific
theories since time immemorial). Put another way, despite a
whole-hearted commitment to an It-from-Bit ontology, we are firmly
aligned with the shut-up-and-calculate camp as the best way to obtain
results either from the physical perspective or as a quality assurance
measure of our fledgling theory of dynamics.

In detail, we present a reflective process calculus. Then we develop
intuitive correspondences between the notions available in this
calculus and the usual physical notions supporting quantum mechanical
calculations. Thus, 

\begin{table}[htp]
  \center{
    \fbox{
      \begin{tabular}{c|c}
        quantum mechanics & process calculus \\
        \hline
        scalar & name \\
        state vector & process \\
        dual & contextual duals \\
        matrix & formal sums of process-context-dual pairs \\
        orthogonality & process annihilation \\
        inner product & execution-formula + quoting
      \end{tabular}
    }
  }
  \caption{QM - process calculi correspondences}
\end{table}

Then we tighten up these intuitions to operational definitions. We
employ the Dirac notation as the best proxy we can find for an
abstract syntax of the quantum mechanical notions. The definitions we
develop put us in contact with equational constraints coming from the
theory that we demonstrate the definitions and calculations satisfy.

This puts us in a position to shut up and calculate for the
Stern-Gerlach experimental set up, showing how these predictive
calculations become calculations on processes in our theory of a
reflective process calculus.

Penultimately, we demonstrate that the notion of metric coming from
the inner product coincides with the notion of metric available from
the theory of bisimulation. This demonstration gives us the right to
think of space as arising from behavior. Finally, we consider where we
might go from the new vantage point we have obtained.

% section introduction (end) 
 
% section introduction (end)

% \documentclass[12pt]{llncs}
%\documentclass{jktr}

\usepackage[pdftex]{hyperref}                   
\usepackage {listings}
\usepackage {mathpartir}
\usepackage{bcprules}
%\usepackage{listings}
                       
\usepackage{graphicx} 
%\usepackage[margins=2.5cm,nohead,nofoot]{geometry}
%\usepackage{geometry}
\usepackage{amsfonts}
\usepackage{amstext}
\usepackage{latexsym}
\usepackage{amssymb}
\usepackage{color}


%\include{myPreamble}
\include{qm2pi.local} 

%\ifpdf
%\usepackage[pdftex]{graphicx}
%\else
%\usepackage{graphicx}
%\fi

 % \ifpdf
%  \usepackage{pdfsync}
%  \if


%\title{Brief Article}
%\author{David F. Snyder}
%\author{L.G. Meredith}

%\address{Dept. of Math., Texas State University--San Marcos, San Marcos, TX 78666}
       
\pagestyle{empty}


\begin{document}

\lstset{language=[Objective]Caml,frame=shadowbox}

\input{qm2pi.front}

% section front matter (end)

\input{qm2pi.intro} 
 
% section introduction (end)

% \input{qm2pi.knotations} 

% section notation (end)

\input{qm2pi.process.calculi} 

% section concurrent_process_calculi_and_spatial_logics_ (end)
    
%\input{qm2pi.knots2pi} 

%\input{qm2pi.trefoil} 

%\input{qm2pi.mainthm} 

% subsection basic_interpretation (end)

%\input{qm2pi.rho.presentation} 
\subsection{The syntax and semantics of the notation system}\label{sub:the_syntax_and_semantics_of_the_notation_system} % (fold)

We now summarize a technical presentation of the calculus that
embodies our theory of dynamics. The typical presentation of such a
calculus follows the style of giving generators and relations on
them. The grammar, below, describing term constructors, freely
generates the set of processes, $\Proc$. This set is then quotiented
by a relation known as structural congruence and it is over this set
that the notion of dynamics is expressed. This presentation is
essentially that of \cite{MeredithR05} with the addition of
polyadicity and summation. For readability we have relegated some of
the technical subtleties to an appendix.

\subsubsection{Process grammar}\label{subsub:process_grammar}

\begin{mathpar}
  \inferrule* [lab=synchronization] {} {{M} \bc \pzero \;|\; x?F \;|\; x!C }
  \and
  \inferrule* [lab=abstraction] {} {{F} \bc (x)P}
  \and
  \inferrule* [lab=concretion] {} {{C} \bc \langle Q \rangle}
  \and
  \inferrule* [lab=process] {} {{P,Q} \bc M \;| \;P|Q \;|\; @{x}}
  \and
  \inferrule* [lab=name] {} {{x} \bc \quotep{P}}
\end{mathpar} 

Note that $\vec{x}$ (resp. $\vec{P}$) denotes a vector of names
(resp. processes) of length $|\vec{x}|$ (resp. $|\vec{P}|$). We adopt
the following useful abbreviations.

\begin{mathpar}
   x?(\vec{y}).P := x.(\vec{y})P \and  x\clift{\vec{P}} := x.\clift{\vec{P}}
   \and x!(y) := \lift{x}{\dropn{y}}
   \and \Pi_{i=0}^{n-1}P_i := P_0 | \ldots | P_{n-1}
\end{mathpar}

\subsubsection{Structural congruence}

\paragraph{Free and bound names and alpha-equivalence.} At the
core of structural equivalence is alpha-equivalence which identifies
process that are the same up to a change of variable. Formally, we
recognize the distinction between free and bound names. The free names
of a process, $\freenames{P}$, may be calculated recursively as
follows:

\begin{mathpar}
\freenames{\pzero} := \emptyset
  \and \\
  \freenames{x?(y).P} := \{ x \} \cup (\freenames{P} \setminus \{ y \})
  \and 
  \freenames{x!\langle P \rangle} := \{ x \} \cup \{ P \} 
  \and \\
  \freenames{P|Q} := \freenames{P} \cup \freenames{Q}
  \and \\
  \freenames{@{x}} := \{ x \}
\end{mathpar}

$\pi$
$\quotep{\pi}$

$\freenames{-} : \pi \to \mathcal{P}(\quotep{\pi})$

\begin{eqnarray*}
  \freenames{\pzero} & := & \emptyset \\
  \freenames{x?(y).P} & := & \{ x \} \cup (\freenames{P} \setminus \{ y \}) \\
  \freenames{x!\langle P \rangle} & := & \{ x \} \cup \{ P \} \\
  \freenames{P|Q} & := & \freenames{P} \cup \freenames{Q} \\
  \freenames{\dropn{x}} & := & \{ x \}
\end{eqnarray*}

The bound names of a process, $\boundnames{P}$, are those names occurring in $P$
that are not free. For example, in $x?(y).0$, the name $x$ is free, while $y$ is bound.

\begin{mathpar}
  \inferrule* [lab=monoidal-laws] {} { P|Q \equiv Q|P \and P|0 \equiv P \and P|(Q|R) \equiv (P|Q)|R }
\end{mathpar}

\begin{mathpar}
  \inferrule* [lab=alpha-equivalence] {} { (x)P \equiv (y)P\{y/x\} \and y \not\in \freenames{P} }
\end{mathpar}

\begin{definition}
Then two processes, $P,Q$, are alpha-equivalent if $P = Q\{\vec{y}/\vec{x}\}$ for
some $\vec{x} \in \boundnames{Q},\vec{y} \in \boundnames{P}$, where $Q\{\vec{y}/\vec{x}\}$
denotes the capture-avoiding substitution of $\vec{y}$ for $\vec{x}$ in $Q$.
\end{definition}

\begin{definition}
  The {\em structural congruence} \cite{SangiorgiWalker} , $\equiv$,
  between processes is the least congruence containing
  alpha-equivalence, satisfying the abelian monoid laws
  (associativity, commutativity and $\pzero$ as identity) for parallel
  composition $|$ and for summation $+$.
\end{definition}

\subsection{Name equivalence}

We take name equivalence, written $\nameeq$, to be the smallest
equivalence relation generated by the following rules.

\begin{mathpar}
\inferrule*[lab=Quote-drop]
{ }
{ \quotep{@{x}} \nameeq x }

\inferrule*[lab=Struct-equiv]
{ P \scong Q }
{ \quotep{P} \nameeq \quotep{Q} }
\end{mathpar}

The astute reader will have noticed that the mutual recursion of names
and processes imposes a mutual recursion on alpha-equivalence and
structural equivalence via name-equivalence. Fortunately, all of this
works out pleasantly and we may calculate in the natural way, free of
concern. The reader interested in the details is referred to the
appendix \ref{appendix:rho_details}.

\subsection{Substitution}

We use $\Proc$ for the set of processes, $\QProc$ for the set of
names, and $\id{\{}\vec{y} / \vec{x} \id{\}}$ to denote partial maps,
$s : \QProc \rightarrow \QProc$. A map, $s$ lifts, uniquely, to a map
on process terms, $\widehat{s} : \Proc \rightarrow \Proc$ by the
following equations.

\begin{mathpar}
  (0) \psubstp{Q}{P} := 0 \\
  (R \juxtap S) \psubstp{Q}{P}
  :=    
  (R)\psubstp{Q}{P} \juxtap (S) \psubstp{Q}{P} \\
  (x?(y).R) \psubstp{Q}{P}    
  :=    
  (x)\substp{Q}{P} (z)\concat( (R \psubstn{z}{y}) \psubstp{Q}{P} ) \\
  (\lift{x}{R}) \psubstp{Q}{P}  
  :=
  \lift{(x)\substp{Q}{P}}{ R \psubstp{Q}{P} } \\
%   (\dropn{x})  \psubstp{Q}{P}       
%   := 
%   \left\{ 
%     \begin{array}{ccc} 
%       \dropn{\quotep{Q}} & & x \nameeq \quotep{P} \\
%       \dropn{x} & & otherwise \\
%     \end{array}
%   \right. 
  (\dropn{x})  \psubstp{Q}{P}       
  := 
  \left\{ 
    \begin{array}{ccc} 
      Q & & x \nameeq \quotep{P} \\
      \dropn{x} & & otherwise \\
    \end{array}
  \right.
\end{mathpar}
 

where

\begin{eqnarray}
  (x)\id{\{} \lpquote Q \rpquote / \lpquote P \rpquote \id{\}}            = 
  \left\{ 
    \begin{array}{ccc}
      \lpquote Q \rpquote & & x \nameeq \lpquote P \rpquote \\
      x & & otherwise \\
    \end{array}
  \right. \nonumber
\end{eqnarray}

and $z$ is chosen distinct from $\quotep{P}$, $\quotep{Q}$, the free
names in $Q$, and all the names in $R$. Our $\alpha$-equivalence will
be built in the standard way from this substitution.

\begin{remark}\label{rem:no_self_referential_names}
  One consequence of these definitions is that $\forall P. \quotep{P}
  \not\in \freenames{P}$.
\end{remark}

\subsection{ Dynamic quote: an example }

Anticipating something of what's to come, consider applying the
substitution, $\widehat{\id{\{}u / z \id{\}}}$, to the following pair
of processes, $\lift{w}{y!(z)}$ and $w[ \lpquote y!(z) \rpquote ]$.

\begin{eqnarray}
	\lift{w}{y!(z)}\widehat{\id{\{}u / z \id{\}}}
		& = &
		\lift{w}{y!(u)} \nonumber\\
	w[ \lpquote y!(z) \rpquote ] \widehat{ \id{\{}u / z \id{\}} }
		& = &
		w[ \lpquote y!(z) \rpquote ] \nonumber
\end{eqnarray}

Because the body of the process between quotes is impervious to
substitution, we get radically different answers. In fact, by
examining the first process in an input context,
e.g. $x?(z).\lift{w}{y!(z)}$, we see that the process under the lift
operator may be shaped by prefixed inputs binding a name inside it. In
this sense, the lift operator will be seen as a way to dynamically
construct processes before reifying them as names.

Finally equipped with these standard features we can present the
dynamics of the calculus.

\subsubsection{Operational semantics} 

Finally, we introduce the computational dynamics. What marks these
algebras as distinct from other more traditionally studied algebraic
structures, e.g. vector spaces or polynomial rings, is the manner in
which dynamics is captured. In traditional structures, dynamics is typically
expressed through morphisms between such structures, as in linear maps
between vector spaces or morphisms between rings. In algebras
associated with the semantics of computation, the dynamics is
expressed as part of the algebraic structure itself, through a
reduction reduction relation typically denoted by $\red$. Below, we
give a recursive presentation of this relation for the calculus used
in the encoding.

$\red \subseteq \pi \times \pi$
$\red : \pi \to \mathcal{P}(\pi)$

\begin{mathpar}
  \inferrule* [lab=Comm] { \textsf{match}( x_{src}, x_{trgt} ) } { x_{trgt}?(y)P \; | \; x_{src}!\langle {Q} \rangle \red P\{\quotep{Q}/y}\} }
  \and \\
  \inferrule* [lab=Par] {{P} \red {P}'} {{{P} | {Q}} \red {{P}' | {Q}}}
  \and
  \inferrule* [lab=Equiv]{{{P} \scong {P}'} \andalso {{P}' \red {Q}'} \andalso {{Q}' \scong {Q}}}{{P} \red {Q}}
\end{mathpar}

\begin{eqnarray*}
  match_{\equiv} (\quotep{P},\quotep{Q}) & := & P \equiv Q \\
  match_{\dagger}(\quotep{P},\quotep{Q}) & := & \forall R. P|Q \red^{*} R => R \red^{*} 0 \\
  match_{K}(\quotep{P},\quotep{Q}) & := & K \mbox{ for some context } K
\end{eqnarray*}

$u?(x)P | u!\langle Q \rangle \red P\{\quotep{Q}/x\}$

%We write $\wred$ for $\red^*$, and $P\red$ if $\exists Q $ such that $ P \red Q$.
We write $P\red$ if $\exists Q $ such that $ P \red Q$ and $P\not\red$, otherwise.

\section{Replication}

As mentioned before, it is known that replication (and hence
recursion) can be implemented in a higher-order process algebra
\cite{SangiorgiWalker}. As our first example of calculation with the
machinery thus far presented we give the construction explicitly in
the {\rhoc}.

\begin{eqnarray}
	D_{x} & := & \prefix{x}{y}{(\binpar{\outputp{x}{y}}{@{y}})} \nonumber\\
	\bangp_{x}{P} & := & \binpar{{x}!\langle{\binpar{D_{x}}{P}}\rangle}{D_{x}} \nonumber
\end{eqnarray}

\begin{eqnarray}
	\bangp_{x}{P} & & \nonumber\\
	=
	& {x}!\langle{(\prefix{x}{y}{(\outputp{x}{y} | @{y})) | P}}\rangle 
	      | \prefix{x}{y}{(\outputp{x}{y} | @{y})} & \nonumber\\
	\red
	& (\outputp{x}{y} | @{y})\substn{\quotep{(\prefix{x}{y}{(@{y} | \outputp{x}{y})) | P}}}{y} & \nonumber\\
	=
	& \outputp{x}{\quotep{(\prefix{x}{y}{(\outputp{x}{y} | @{y})) | P}}}
	  | {(\prefix{x}{y}{(\outputp{x}{y} | @{y})) | P}} & \nonumber\\
	\red
	& \ldots & \nonumber\\
	\red^*
	& P | P | \ldots & \nonumber
\end{eqnarray}

Of course, this encoding, as an implementation, runs away, unfolding
$\bangp{P}$ eagerly. A lazier and more implementable replication
operator, restricted to input-guarded processes, may be obtained as follows.

\begin{eqnarray}
\bangp{\prefix{u}{v}{P}} 
	:= 
	\binpar{\lift{x}{\prefix{u}{v}{(\binpar{D(x)}{P})}}}{D(x)} \nonumber
\end{eqnarray}

\begin{remark}
  Note that the lazier definition still does not deal with summation
  or mixed summation (i.e. sums over input and output). The reader is
  invited to construct definitions of replication that deal with these
  features. 

  Further, the definitions are parameterized in a name, $x$. Can you,
  gentle reader, make a definition that eliminates this parameter and
  guarantees no accidental interaction between the replication
  machinery and the process being replicated -- i.e. no accidental
  sharing of names used by the process to get its work done and the
  name(s) used by the replication to effect copying. This latter
  revision of the definition of replication is crucial to obtaining
  the expected identity $!!P \sim !P$.
\end{remark}

\begin{remark}\label{rem:paradoxical_combinator}
  The reader familiar with the lambda calculus will have noticed the
  similarity between $D$ and the paradoxical combinator.

  [Ed. note: the existence of this seems to suggest we have to be more
  restrictive on the set of processes and names we admit if we are to
  support no-cloning.]
\end{remark}

\subsubsection{Bisimulation}

The computational dynamics gives rise to another kind of equivalence,
the equivalence of computational behavior. As previously mentioned
this is typically captured \emph{via} some form of bisimulation.

% The notion we use in this paper is weak barbed bisimulation
% \cite{milner91polyadicpi}.

The notion we use in this paper is derived from weak barbed
bisimulation \cite{milner91polyadicpi}. 

\begin{definition}
An \emph{observation relation}, $\downarrow_{\mathcal N}$, over a set
of names, $\mathcal N$, is the smallest relation satisfying the rules
below.

\infrule[Out-barb]{y \in {\mathcal N}, \; x \nameeq y}
		  {\outputp{x}{v} \downarrow_{\mathcal N} x}
\infrule[Par-barb]{\mbox{$P\downarrow_{\mathcal N} x$ or $Q\downarrow_{\mathcal N} x$}}
		  {\binpar{P}{Q} \downarrow_{\mathcal N} x}

We write $P \Downarrow_{\mathcal N} x$ if there is $Q$ such that 
$P \wred Q$ and $Q \downarrow_{\mathcal N} x$.
\end{definition}

\begin{definition}
%\label{def.bbisim}
An  ${\mathcal N}$-\emph{barbed bisimulation} over a set of names, ${\mathcal N}$, is a symmetric binary relation 
${\mathcal S}_{\mathcal N}$ between agents such that $P\rel{S}_{\mathcal N}Q$ implies:
\begin{enumerate}
\item If $P \red P'$ then $Q \wred Q'$ and $P'\rel{S}_{\mathcal N} Q'$.
\item If $P\downarrow_{\mathcal N} x$, then $Q\Downarrow_{\mathcal N} x$.
\end{enumerate}
$P$ is ${\mathcal N}$-barbed bisimilar to $Q$, written
$P \wbbisim_{\mathcal N} Q$, if $P \rel{S}_{\mathcal N} Q$ for some ${\mathcal N}$-barbed bisimulation ${\mathcal S}_{\mathcal N}$.
\end{definition}

$\mathcal{R} \subseteq \pi \times \pi$

$P \mathcal{R} Q => \forall P'. P \red P' \Rightarrow \exists Q'. Q \red Q', P' \mathcal{R} Q'$

$P \vdash x \Rightarrow Q \vdash x$

\begin{mathpar}
  \inferrule*[lab=Out-barb]{x \nameeq y}{{y}!\langle{Q}\rangle \vdash x}
  \and
  \inferrule*[lab=Par-barb]{\mbox{$P\vdash x$ or $Q\vdash x$}}{\binpar{P}{Q} \vdash x}
\end{mathpar}

\subsubsection{Contexts}

One of the principle advantages of computational calculi like the
$\pi$-calculus is a well-defined notion of context,
contextual-equivalence and a correlation between
contextual-equivalence and notions of bisimulation. The notion of
context allows the decomposition of a process into (sub-)process and
its syntactic environment, its context. Thus, a context may be
thought of as a process with a ``hole'' (written $\Box$) in it. The
application of a context $M$ to a process $P$, written $M[P]$, is
tantamount to filling the hole in $M$ with $P$. In this paper we do
not need the full weight of this theory, but do make use of the notion
of context in the proof the main theorem. 

\begin{mathpar}
  \inferrule* [lab=summation] {} {{M_{M},M_{N}} \bc \Box \;|\; x.M_{A} \;|\; M_{M}+M_{N}}
  \and
  \inferrule* [lab=agent] {} {{M_{A}} \bc (\vec{x})M_{P} \;| \; \clift{P_0,\ldots,M_{P},\ldots,P_N}}
  \and \\
  \inferrule* [lab=process] {} {{M_{P}} \bc M_{N} \;| \;P|M_{P} }
\end{mathpar} 

\begin{mathpar}
  \inferrule* [lab=sychronization] {} {M_{N} \bc \Box \;|\; x?M_{F} \;|\; x!M_{C}}
  \and
  \inferrule* [lab=abstraction] {} {{M_{F}} \bc (x)M_{P} }
  \and
  \inferrule* [lab=concretion] {} {{M_{C}} \bc \langle M_{P} \rangle }
  \and \\
  \inferrule* [lab=process] {} {{M_{P}} \bc M_{N} \;| \;P|M_{P} }
\end{mathpar}

\begin{definition}[contextual application] Given a context $M$, and
  process $P$, we define the \emph{contextual application}, $M[P] :=
  M\{P/\Box\}$. That is, the contextual application of M to P is the
  substitution of $P$ for $\Box$ in $M$.
\end{definition}

$\meaningof{-} : L \to \mathcal{P}(\pi)$

\begin{mathpar}
  \inferrule* [lab=collection] {} {\meaningof{true} = \pi, \and \meaningof{~E} = \pi \setminus \meaningof{E}, \and \meaningof{E_{1} \& E_{2}} = \meaningof{E_{1}} \cap \meaningof{E_{2}}}
\end{mathpar}

\begin{mathpar}
  \inferrule* [lab=structure] {} {\meaningof{0} = \{ P \in \pi | P \equiv 0 \}, \and \\ \meaningof{E_1 | E_2} = \{ P \in \pi | P \equiv P_{1} | P_{2}, P_{1} \in \meaningof{E_{1}}, P_{2} \in \meaningof{E_2}\} }
\end{mathpar}

\begin{mathpar}
 \inferrule* [lab=behavior] {} {\meaningof{\langle a?b \rangle E} = \{ P \in \pi | P \equiv Q | u?(y)P', \\ \and \\\\ \and \\ \;\;\; u \in \meaningof{a}, \forall z.P'\{z/y\} \in \meaningof{E\{z/b\}}\}, \and \\ \meaningof{a!E} = \{ P \in \pi | P \equiv Q | x!\langle P' \rangle, x \in \meaningof{a} P' \in \meaningof{E}\} }
\end{mathpar}

\begin{mathpar}
 \inferrule* [lab=nominal] {} {\meaningof{\quotep{E}} = \{ \quotep{P} \in \quotep{\pi} | P \in \meaningof{E} \}, \and \meaningof{\quotep{P}} = \{ \quotep{Q} \in \quotep{\pi} | P \equiv Q \} \and \\ \meaningof{@\quotep{E}} = \{ P \in \pi | P \equiv @x, x \in \meaningof{E} \}}
\end{mathpar}

\begin{eqnarray*}
  \\
  \meaningof{-} : TS \to ST
\end{eqnarray*}

\begin{eqnarray*}
  \\
  L : TS \to ST
\end{eqnarray*}

\begin{eqnarray*}
  \\
  P \models E \iff P \in \meaningof{E}
\end{eqnarray*}

\begin{eqnarray*}
  P \approx_{L} Q \iff \forall E \in L. P \models E \iff Q \models E
\end{eqnarray*}

\begin{eqnarray*}
  P \approx_{K} Q
\end{eqnarray*}

\begin{eqnarray*}
  P \approx Q
\end{eqnarray*}

$\approx_{K} = \approx = \approx_{L}$

\subsubsection{Contextual duality}

Note that contexts extend the quotation operation to a family of
operations from processes to names. Given a context, $M$, we can
define a \emph{nominal context}, $\quotep{M}$ by $\quotep{M}[P] :=
\quotep{M[P]}$. To foreshadow what is to come we observe that these
operations enjoy a duality with processes very much like the duality
between vectors and maps from vectors to scalars.

Further, because the calculus is essentially higher-order, we have a
correspondence between contexts and processes. More specifically,
given a name $x$ and a context $M$ we can construct $M^{*}_{x}$ such
that 

\begin{mathpar}
  M^{*}_{x} | \lift{x}{P} \red M[P]
\end{mathpar}

namely,

\begin{mathpar}
  M^{*}_{x} := x?(u).M[\dropn{u}]
\end{mathpar}

The dependence of $M^{*}_{x}$ on a name makes it an abstraction, 

\begin{mathpar}
  M^{*} := (x)x?(u).M[\dropn{u}]
\end{mathpar}

\subsection{Additional notation}

It will sometimes be convenient to denote the process a name
quotes. We already have the notation $x = \quotep{P}$, but it will be
convenient to introduce an alternate notation, $\procn{x}$, when we
want to emphasize the connection to the use of the name. Note that, by
virtue of name equivalence, $\quotep{\procn{x}} \nameeq x$; so, the
notation is consistent with previous definitions.

Further, because names have structure it is possible to effect
substitutions on the basis of that structure. This means we need to
upgrade our notation for substitutions, which we accomplish by
adapting comprehension notation. Thus,

\begin{mathpar}
  P\{ y / x : x \in S \}
\end{mathpar}

is interpreted to mean the process derived from P by replacing (in a
capture-avoiding manner) each occurrence of $x$ in $S$ by $y$. For example,

\begin{mathpar}
  P\{ \quotep{\procn{x}|\procn{x}} / x : x \in \freenames{P} \}
\end{mathpar}

will replace each (occurrence) of a free name $x$ in $P$ by
$\quotep{\procn{x}|\procn{x}}$.

Also, we will avail ourselves of the notation $x^{L}$ and $x^{R}$ to
denote injections of a name into disjoint copies of the name
space. There are numerous ways to accomplish this. One example can be
found in \cite{MeredithR05}. This notation overloads to vectors of
names: $\vec{x}^{\pi} := (x_{i}^{\pi} \; : \; 0 \leq i < |\vec{x}| )$ where $\pi \in \{L,R\}$.

We also use $P^{\Box} := P|\Box$.

In \cite{MeredithR05} an interpretation of the new operator is
given. It turns out that there are several possible interpretations
all enjoying the requisite algebraic properties of the operator (see
\cite{milner91polyadicpi}). We will therefore make liberal use of
$(\nu\; \vec{x})P$.

% subsection the_syntax_and_semantics_of_the_notation_system (end)   

\input{qm2pi.qmops} 

\input{qm2pi.sterngerlach} 

\input{qm2pi.metric} 

% section concurrent_process_calculi (end)

%\input{qm2pi.proofsketch}

% section proof sketch (end)

%\input{qm2pi.slviaknots} 

% section spatial logic via knots (end)

\input{qm2pi.conclusion}

% section conclusion (end)

%\input{qm2pi.dtcodes} 

% section wiring algorithm (end)

\input{qm2pi.ack} 

% section acknowledgments (end)

\newpage


\bibliographystyle{plain}   
\bibliography{../../biblios/main.bib}

\input{qm2pi.rhodetails}

\end{document}

 

% section notation (end)

\input{qm2pi.process.calculi} 

% section concurrent_process_calculi_and_spatial_logics_ (end)
    
%\documentclass[12pt]{llncs}
%\documentclass{jktr}

\usepackage[pdftex]{hyperref}                   
\usepackage {listings}
\usepackage {mathpartir}
\usepackage{bcprules}
%\usepackage{listings}
                       
\usepackage{graphicx} 
%\usepackage[margins=2.5cm,nohead,nofoot]{geometry}
%\usepackage{geometry}
\usepackage{amsfonts}
\usepackage{amstext}
\usepackage{latexsym}
\usepackage{amssymb}
\usepackage{color}


%\include{myPreamble}
\include{qm2pi.local} 

%\ifpdf
%\usepackage[pdftex]{graphicx}
%\else
%\usepackage{graphicx}
%\fi

 % \ifpdf
%  \usepackage{pdfsync}
%  \if


%\title{Brief Article}
%\author{David F. Snyder}
%\author{L.G. Meredith}

%\address{Dept. of Math., Texas State University--San Marcos, San Marcos, TX 78666}
       
\pagestyle{empty}


\begin{document}

\lstset{language=[Objective]Caml,frame=shadowbox}

\input{qm2pi.front}

% section front matter (end)

\input{qm2pi.intro} 
 
% section introduction (end)

% \input{qm2pi.knotations} 

% section notation (end)

\input{qm2pi.process.calculi} 

% section concurrent_process_calculi_and_spatial_logics_ (end)
    
%\input{qm2pi.knots2pi} 

%\input{qm2pi.trefoil} 

%\input{qm2pi.mainthm} 

% subsection basic_interpretation (end)

%\input{qm2pi.rho.presentation} 
\subsection{The syntax and semantics of the notation system}\label{sub:the_syntax_and_semantics_of_the_notation_system} % (fold)

We now summarize a technical presentation of the calculus that
embodies our theory of dynamics. The typical presentation of such a
calculus follows the style of giving generators and relations on
them. The grammar, below, describing term constructors, freely
generates the set of processes, $\Proc$. This set is then quotiented
by a relation known as structural congruence and it is over this set
that the notion of dynamics is expressed. This presentation is
essentially that of \cite{MeredithR05} with the addition of
polyadicity and summation. For readability we have relegated some of
the technical subtleties to an appendix.

\subsubsection{Process grammar}\label{subsub:process_grammar}

\begin{mathpar}
  \inferrule* [lab=synchronization] {} {{M} \bc \pzero \;|\; x?F \;|\; x!C }
  \and
  \inferrule* [lab=abstraction] {} {{F} \bc (x)P}
  \and
  \inferrule* [lab=concretion] {} {{C} \bc \langle Q \rangle}
  \and
  \inferrule* [lab=process] {} {{P,Q} \bc M \;| \;P|Q \;|\; @{x}}
  \and
  \inferrule* [lab=name] {} {{x} \bc \quotep{P}}
\end{mathpar} 

Note that $\vec{x}$ (resp. $\vec{P}$) denotes a vector of names
(resp. processes) of length $|\vec{x}|$ (resp. $|\vec{P}|$). We adopt
the following useful abbreviations.

\begin{mathpar}
   x?(\vec{y}).P := x.(\vec{y})P \and  x\clift{\vec{P}} := x.\clift{\vec{P}}
   \and x!(y) := \lift{x}{\dropn{y}}
   \and \Pi_{i=0}^{n-1}P_i := P_0 | \ldots | P_{n-1}
\end{mathpar}

\subsubsection{Structural congruence}

\paragraph{Free and bound names and alpha-equivalence.} At the
core of structural equivalence is alpha-equivalence which identifies
process that are the same up to a change of variable. Formally, we
recognize the distinction between free and bound names. The free names
of a process, $\freenames{P}$, may be calculated recursively as
follows:

\begin{mathpar}
\freenames{\pzero} := \emptyset
  \and \\
  \freenames{x?(y).P} := \{ x \} \cup (\freenames{P} \setminus \{ y \})
  \and 
  \freenames{x!\langle P \rangle} := \{ x \} \cup \{ P \} 
  \and \\
  \freenames{P|Q} := \freenames{P} \cup \freenames{Q}
  \and \\
  \freenames{@{x}} := \{ x \}
\end{mathpar}

$\pi$
$\quotep{\pi}$

$\freenames{-} : \pi \to \mathcal{P}(\quotep{\pi})$

\begin{eqnarray*}
  \freenames{\pzero} & := & \emptyset \\
  \freenames{x?(y).P} & := & \{ x \} \cup (\freenames{P} \setminus \{ y \}) \\
  \freenames{x!\langle P \rangle} & := & \{ x \} \cup \{ P \} \\
  \freenames{P|Q} & := & \freenames{P} \cup \freenames{Q} \\
  \freenames{\dropn{x}} & := & \{ x \}
\end{eqnarray*}

The bound names of a process, $\boundnames{P}$, are those names occurring in $P$
that are not free. For example, in $x?(y).0$, the name $x$ is free, while $y$ is bound.

\begin{mathpar}
  \inferrule* [lab=monoidal-laws] {} { P|Q \equiv Q|P \and P|0 \equiv P \and P|(Q|R) \equiv (P|Q)|R }
\end{mathpar}

\begin{mathpar}
  \inferrule* [lab=alpha-equivalence] {} { (x)P \equiv (y)P\{y/x\} \and y \not\in \freenames{P} }
\end{mathpar}

\begin{definition}
Then two processes, $P,Q$, are alpha-equivalent if $P = Q\{\vec{y}/\vec{x}\}$ for
some $\vec{x} \in \boundnames{Q},\vec{y} \in \boundnames{P}$, where $Q\{\vec{y}/\vec{x}\}$
denotes the capture-avoiding substitution of $\vec{y}$ for $\vec{x}$ in $Q$.
\end{definition}

\begin{definition}
  The {\em structural congruence} \cite{SangiorgiWalker} , $\equiv$,
  between processes is the least congruence containing
  alpha-equivalence, satisfying the abelian monoid laws
  (associativity, commutativity and $\pzero$ as identity) for parallel
  composition $|$ and for summation $+$.
\end{definition}

\subsection{Name equivalence}

We take name equivalence, written $\nameeq$, to be the smallest
equivalence relation generated by the following rules.

\begin{mathpar}
\inferrule*[lab=Quote-drop]
{ }
{ \quotep{@{x}} \nameeq x }

\inferrule*[lab=Struct-equiv]
{ P \scong Q }
{ \quotep{P} \nameeq \quotep{Q} }
\end{mathpar}

The astute reader will have noticed that the mutual recursion of names
and processes imposes a mutual recursion on alpha-equivalence and
structural equivalence via name-equivalence. Fortunately, all of this
works out pleasantly and we may calculate in the natural way, free of
concern. The reader interested in the details is referred to the
appendix \ref{appendix:rho_details}.

\subsection{Substitution}

We use $\Proc$ for the set of processes, $\QProc$ for the set of
names, and $\id{\{}\vec{y} / \vec{x} \id{\}}$ to denote partial maps,
$s : \QProc \rightarrow \QProc$. A map, $s$ lifts, uniquely, to a map
on process terms, $\widehat{s} : \Proc \rightarrow \Proc$ by the
following equations.

\begin{mathpar}
  (0) \psubstp{Q}{P} := 0 \\
  (R \juxtap S) \psubstp{Q}{P}
  :=    
  (R)\psubstp{Q}{P} \juxtap (S) \psubstp{Q}{P} \\
  (x?(y).R) \psubstp{Q}{P}    
  :=    
  (x)\substp{Q}{P} (z)\concat( (R \psubstn{z}{y}) \psubstp{Q}{P} ) \\
  (\lift{x}{R}) \psubstp{Q}{P}  
  :=
  \lift{(x)\substp{Q}{P}}{ R \psubstp{Q}{P} } \\
%   (\dropn{x})  \psubstp{Q}{P}       
%   := 
%   \left\{ 
%     \begin{array}{ccc} 
%       \dropn{\quotep{Q}} & & x \nameeq \quotep{P} \\
%       \dropn{x} & & otherwise \\
%     \end{array}
%   \right. 
  (\dropn{x})  \psubstp{Q}{P}       
  := 
  \left\{ 
    \begin{array}{ccc} 
      Q & & x \nameeq \quotep{P} \\
      \dropn{x} & & otherwise \\
    \end{array}
  \right.
\end{mathpar}
 

where

\begin{eqnarray}
  (x)\id{\{} \lpquote Q \rpquote / \lpquote P \rpquote \id{\}}            = 
  \left\{ 
    \begin{array}{ccc}
      \lpquote Q \rpquote & & x \nameeq \lpquote P \rpquote \\
      x & & otherwise \\
    \end{array}
  \right. \nonumber
\end{eqnarray}

and $z$ is chosen distinct from $\quotep{P}$, $\quotep{Q}$, the free
names in $Q$, and all the names in $R$. Our $\alpha$-equivalence will
be built in the standard way from this substitution.

\begin{remark}\label{rem:no_self_referential_names}
  One consequence of these definitions is that $\forall P. \quotep{P}
  \not\in \freenames{P}$.
\end{remark}

\subsection{ Dynamic quote: an example }

Anticipating something of what's to come, consider applying the
substitution, $\widehat{\id{\{}u / z \id{\}}}$, to the following pair
of processes, $\lift{w}{y!(z)}$ and $w[ \lpquote y!(z) \rpquote ]$.

\begin{eqnarray}
	\lift{w}{y!(z)}\widehat{\id{\{}u / z \id{\}}}
		& = &
		\lift{w}{y!(u)} \nonumber\\
	w[ \lpquote y!(z) \rpquote ] \widehat{ \id{\{}u / z \id{\}} }
		& = &
		w[ \lpquote y!(z) \rpquote ] \nonumber
\end{eqnarray}

Because the body of the process between quotes is impervious to
substitution, we get radically different answers. In fact, by
examining the first process in an input context,
e.g. $x?(z).\lift{w}{y!(z)}$, we see that the process under the lift
operator may be shaped by prefixed inputs binding a name inside it. In
this sense, the lift operator will be seen as a way to dynamically
construct processes before reifying them as names.

Finally equipped with these standard features we can present the
dynamics of the calculus.

\subsubsection{Operational semantics} 

Finally, we introduce the computational dynamics. What marks these
algebras as distinct from other more traditionally studied algebraic
structures, e.g. vector spaces or polynomial rings, is the manner in
which dynamics is captured. In traditional structures, dynamics is typically
expressed through morphisms between such structures, as in linear maps
between vector spaces or morphisms between rings. In algebras
associated with the semantics of computation, the dynamics is
expressed as part of the algebraic structure itself, through a
reduction reduction relation typically denoted by $\red$. Below, we
give a recursive presentation of this relation for the calculus used
in the encoding.

$\red \subseteq \pi \times \pi$
$\red : \pi \to \mathcal{P}(\pi)$

\begin{mathpar}
  \inferrule* [lab=Comm] { \textsf{match}( x_{src}, x_{trgt} ) } { x_{trgt}?(y)P \; | \; x_{src}!\langle {Q} \rangle \red P\{\quotep{Q}/y}\} }
  \and \\
  \inferrule* [lab=Par] {{P} \red {P}'} {{{P} | {Q}} \red {{P}' | {Q}}}
  \and
  \inferrule* [lab=Equiv]{{{P} \scong {P}'} \andalso {{P}' \red {Q}'} \andalso {{Q}' \scong {Q}}}{{P} \red {Q}}
\end{mathpar}

\begin{eqnarray*}
  match_{\equiv} (\quotep{P},\quotep{Q}) & := & P \equiv Q \\
  match_{\dagger}(\quotep{P},\quotep{Q}) & := & \forall R. P|Q \red^{*} R => R \red^{*} 0 \\
  match_{K}(\quotep{P},\quotep{Q}) & := & K \mbox{ for some context } K
\end{eqnarray*}

$u?(x)P | u!\langle Q \rangle \red P\{\quotep{Q}/x\}$

%We write $\wred$ for $\red^*$, and $P\red$ if $\exists Q $ such that $ P \red Q$.
We write $P\red$ if $\exists Q $ such that $ P \red Q$ and $P\not\red$, otherwise.

\section{Replication}

As mentioned before, it is known that replication (and hence
recursion) can be implemented in a higher-order process algebra
\cite{SangiorgiWalker}. As our first example of calculation with the
machinery thus far presented we give the construction explicitly in
the {\rhoc}.

\begin{eqnarray}
	D_{x} & := & \prefix{x}{y}{(\binpar{\outputp{x}{y}}{@{y}})} \nonumber\\
	\bangp_{x}{P} & := & \binpar{{x}!\langle{\binpar{D_{x}}{P}}\rangle}{D_{x}} \nonumber
\end{eqnarray}

\begin{eqnarray}
	\bangp_{x}{P} & & \nonumber\\
	=
	& {x}!\langle{(\prefix{x}{y}{(\outputp{x}{y} | @{y})) | P}}\rangle 
	      | \prefix{x}{y}{(\outputp{x}{y} | @{y})} & \nonumber\\
	\red
	& (\outputp{x}{y} | @{y})\substn{\quotep{(\prefix{x}{y}{(@{y} | \outputp{x}{y})) | P}}}{y} & \nonumber\\
	=
	& \outputp{x}{\quotep{(\prefix{x}{y}{(\outputp{x}{y} | @{y})) | P}}}
	  | {(\prefix{x}{y}{(\outputp{x}{y} | @{y})) | P}} & \nonumber\\
	\red
	& \ldots & \nonumber\\
	\red^*
	& P | P | \ldots & \nonumber
\end{eqnarray}

Of course, this encoding, as an implementation, runs away, unfolding
$\bangp{P}$ eagerly. A lazier and more implementable replication
operator, restricted to input-guarded processes, may be obtained as follows.

\begin{eqnarray}
\bangp{\prefix{u}{v}{P}} 
	:= 
	\binpar{\lift{x}{\prefix{u}{v}{(\binpar{D(x)}{P})}}}{D(x)} \nonumber
\end{eqnarray}

\begin{remark}
  Note that the lazier definition still does not deal with summation
  or mixed summation (i.e. sums over input and output). The reader is
  invited to construct definitions of replication that deal with these
  features. 

  Further, the definitions are parameterized in a name, $x$. Can you,
  gentle reader, make a definition that eliminates this parameter and
  guarantees no accidental interaction between the replication
  machinery and the process being replicated -- i.e. no accidental
  sharing of names used by the process to get its work done and the
  name(s) used by the replication to effect copying. This latter
  revision of the definition of replication is crucial to obtaining
  the expected identity $!!P \sim !P$.
\end{remark}

\begin{remark}\label{rem:paradoxical_combinator}
  The reader familiar with the lambda calculus will have noticed the
  similarity between $D$ and the paradoxical combinator.

  [Ed. note: the existence of this seems to suggest we have to be more
  restrictive on the set of processes and names we admit if we are to
  support no-cloning.]
\end{remark}

\subsubsection{Bisimulation}

The computational dynamics gives rise to another kind of equivalence,
the equivalence of computational behavior. As previously mentioned
this is typically captured \emph{via} some form of bisimulation.

% The notion we use in this paper is weak barbed bisimulation
% \cite{milner91polyadicpi}.

The notion we use in this paper is derived from weak barbed
bisimulation \cite{milner91polyadicpi}. 

\begin{definition}
An \emph{observation relation}, $\downarrow_{\mathcal N}$, over a set
of names, $\mathcal N$, is the smallest relation satisfying the rules
below.

\infrule[Out-barb]{y \in {\mathcal N}, \; x \nameeq y}
		  {\outputp{x}{v} \downarrow_{\mathcal N} x}
\infrule[Par-barb]{\mbox{$P\downarrow_{\mathcal N} x$ or $Q\downarrow_{\mathcal N} x$}}
		  {\binpar{P}{Q} \downarrow_{\mathcal N} x}

We write $P \Downarrow_{\mathcal N} x$ if there is $Q$ such that 
$P \wred Q$ and $Q \downarrow_{\mathcal N} x$.
\end{definition}

\begin{definition}
%\label{def.bbisim}
An  ${\mathcal N}$-\emph{barbed bisimulation} over a set of names, ${\mathcal N}$, is a symmetric binary relation 
${\mathcal S}_{\mathcal N}$ between agents such that $P\rel{S}_{\mathcal N}Q$ implies:
\begin{enumerate}
\item If $P \red P'$ then $Q \wred Q'$ and $P'\rel{S}_{\mathcal N} Q'$.
\item If $P\downarrow_{\mathcal N} x$, then $Q\Downarrow_{\mathcal N} x$.
\end{enumerate}
$P$ is ${\mathcal N}$-barbed bisimilar to $Q$, written
$P \wbbisim_{\mathcal N} Q$, if $P \rel{S}_{\mathcal N} Q$ for some ${\mathcal N}$-barbed bisimulation ${\mathcal S}_{\mathcal N}$.
\end{definition}

$\mathcal{R} \subseteq \pi \times \pi$

$P \mathcal{R} Q => \forall P'. P \red P' \Rightarrow \exists Q'. Q \red Q', P' \mathcal{R} Q'$

$P \vdash x \Rightarrow Q \vdash x$

\begin{mathpar}
  \inferrule*[lab=Out-barb]{x \nameeq y}{{y}!\langle{Q}\rangle \vdash x}
  \and
  \inferrule*[lab=Par-barb]{\mbox{$P\vdash x$ or $Q\vdash x$}}{\binpar{P}{Q} \vdash x}
\end{mathpar}

\subsubsection{Contexts}

One of the principle advantages of computational calculi like the
$\pi$-calculus is a well-defined notion of context,
contextual-equivalence and a correlation between
contextual-equivalence and notions of bisimulation. The notion of
context allows the decomposition of a process into (sub-)process and
its syntactic environment, its context. Thus, a context may be
thought of as a process with a ``hole'' (written $\Box$) in it. The
application of a context $M$ to a process $P$, written $M[P]$, is
tantamount to filling the hole in $M$ with $P$. In this paper we do
not need the full weight of this theory, but do make use of the notion
of context in the proof the main theorem. 

\begin{mathpar}
  \inferrule* [lab=summation] {} {{M_{M},M_{N}} \bc \Box \;|\; x.M_{A} \;|\; M_{M}+M_{N}}
  \and
  \inferrule* [lab=agent] {} {{M_{A}} \bc (\vec{x})M_{P} \;| \; \clift{P_0,\ldots,M_{P},\ldots,P_N}}
  \and \\
  \inferrule* [lab=process] {} {{M_{P}} \bc M_{N} \;| \;P|M_{P} }
\end{mathpar} 

\begin{mathpar}
  \inferrule* [lab=sychronization] {} {M_{N} \bc \Box \;|\; x?M_{F} \;|\; x!M_{C}}
  \and
  \inferrule* [lab=abstraction] {} {{M_{F}} \bc (x)M_{P} }
  \and
  \inferrule* [lab=concretion] {} {{M_{C}} \bc \langle M_{P} \rangle }
  \and \\
  \inferrule* [lab=process] {} {{M_{P}} \bc M_{N} \;| \;P|M_{P} }
\end{mathpar}

\begin{definition}[contextual application] Given a context $M$, and
  process $P$, we define the \emph{contextual application}, $M[P] :=
  M\{P/\Box\}$. That is, the contextual application of M to P is the
  substitution of $P$ for $\Box$ in $M$.
\end{definition}

$\meaningof{-} : L \to \mathcal{P}(\pi)$

\begin{mathpar}
  \inferrule* [lab=collection] {} {\meaningof{true} = \pi, \and \meaningof{~E} = \pi \setminus \meaningof{E}, \and \meaningof{E_{1} \& E_{2}} = \meaningof{E_{1}} \cap \meaningof{E_{2}}}
\end{mathpar}

\begin{mathpar}
  \inferrule* [lab=structure] {} {\meaningof{0} = \{ P \in \pi | P \equiv 0 \}, \and \\ \meaningof{E_1 | E_2} = \{ P \in \pi | P \equiv P_{1} | P_{2}, P_{1} \in \meaningof{E_{1}}, P_{2} \in \meaningof{E_2}\} }
\end{mathpar}

\begin{mathpar}
 \inferrule* [lab=behavior] {} {\meaningof{\langle a?b \rangle E} = \{ P \in \pi | P \equiv Q | u?(y)P', \\ \and \\\\ \and \\ \;\;\; u \in \meaningof{a}, \forall z.P'\{z/y\} \in \meaningof{E\{z/b\}}\}, \and \\ \meaningof{a!E} = \{ P \in \pi | P \equiv Q | x!\langle P' \rangle, x \in \meaningof{a} P' \in \meaningof{E}\} }
\end{mathpar}

\begin{mathpar}
 \inferrule* [lab=nominal] {} {\meaningof{\quotep{E}} = \{ \quotep{P} \in \quotep{\pi} | P \in \meaningof{E} \}, \and \meaningof{\quotep{P}} = \{ \quotep{Q} \in \quotep{\pi} | P \equiv Q \} \and \\ \meaningof{@\quotep{E}} = \{ P \in \pi | P \equiv @x, x \in \meaningof{E} \}}
\end{mathpar}

\begin{eqnarray*}
  \\
  \meaningof{-} : TS \to ST
\end{eqnarray*}

\begin{eqnarray*}
  \\
  L : TS \to ST
\end{eqnarray*}

\begin{eqnarray*}
  \\
  P \models E \iff P \in \meaningof{E}
\end{eqnarray*}

\begin{eqnarray*}
  P \approx_{L} Q \iff \forall E \in L. P \models E \iff Q \models E
\end{eqnarray*}

\begin{eqnarray*}
  P \approx_{K} Q
\end{eqnarray*}

\begin{eqnarray*}
  P \approx Q
\end{eqnarray*}

$\approx_{K} = \approx = \approx_{L}$

\subsubsection{Contextual duality}

Note that contexts extend the quotation operation to a family of
operations from processes to names. Given a context, $M$, we can
define a \emph{nominal context}, $\quotep{M}$ by $\quotep{M}[P] :=
\quotep{M[P]}$. To foreshadow what is to come we observe that these
operations enjoy a duality with processes very much like the duality
between vectors and maps from vectors to scalars.

Further, because the calculus is essentially higher-order, we have a
correspondence between contexts and processes. More specifically,
given a name $x$ and a context $M$ we can construct $M^{*}_{x}$ such
that 

\begin{mathpar}
  M^{*}_{x} | \lift{x}{P} \red M[P]
\end{mathpar}

namely,

\begin{mathpar}
  M^{*}_{x} := x?(u).M[\dropn{u}]
\end{mathpar}

The dependence of $M^{*}_{x}$ on a name makes it an abstraction, 

\begin{mathpar}
  M^{*} := (x)x?(u).M[\dropn{u}]
\end{mathpar}

\subsection{Additional notation}

It will sometimes be convenient to denote the process a name
quotes. We already have the notation $x = \quotep{P}$, but it will be
convenient to introduce an alternate notation, $\procn{x}$, when we
want to emphasize the connection to the use of the name. Note that, by
virtue of name equivalence, $\quotep{\procn{x}} \nameeq x$; so, the
notation is consistent with previous definitions.

Further, because names have structure it is possible to effect
substitutions on the basis of that structure. This means we need to
upgrade our notation for substitutions, which we accomplish by
adapting comprehension notation. Thus,

\begin{mathpar}
  P\{ y / x : x \in S \}
\end{mathpar}

is interpreted to mean the process derived from P by replacing (in a
capture-avoiding manner) each occurrence of $x$ in $S$ by $y$. For example,

\begin{mathpar}
  P\{ \quotep{\procn{x}|\procn{x}} / x : x \in \freenames{P} \}
\end{mathpar}

will replace each (occurrence) of a free name $x$ in $P$ by
$\quotep{\procn{x}|\procn{x}}$.

Also, we will avail ourselves of the notation $x^{L}$ and $x^{R}$ to
denote injections of a name into disjoint copies of the name
space. There are numerous ways to accomplish this. One example can be
found in \cite{MeredithR05}. This notation overloads to vectors of
names: $\vec{x}^{\pi} := (x_{i}^{\pi} \; : \; 0 \leq i < |\vec{x}| )$ where $\pi \in \{L,R\}$.

We also use $P^{\Box} := P|\Box$.

In \cite{MeredithR05} an interpretation of the new operator is
given. It turns out that there are several possible interpretations
all enjoying the requisite algebraic properties of the operator (see
\cite{milner91polyadicpi}). We will therefore make liberal use of
$(\nu\; \vec{x})P$.

% subsection the_syntax_and_semantics_of_the_notation_system (end)   

\input{qm2pi.qmops} 

\input{qm2pi.sterngerlach} 

\input{qm2pi.metric} 

% section concurrent_process_calculi (end)

%\input{qm2pi.proofsketch}

% section proof sketch (end)

%\input{qm2pi.slviaknots} 

% section spatial logic via knots (end)

\input{qm2pi.conclusion}

% section conclusion (end)

%\input{qm2pi.dtcodes} 

% section wiring algorithm (end)

\input{qm2pi.ack} 

% section acknowledgments (end)

\newpage


\bibliographystyle{plain}   
\bibliography{../../biblios/main.bib}

\input{qm2pi.rhodetails}

\end{document}

 

%\documentclass[12pt]{llncs}
%\documentclass{jktr}

\usepackage[pdftex]{hyperref}                   
\usepackage {listings}
\usepackage {mathpartir}
\usepackage{bcprules}
%\usepackage{listings}
                       
\usepackage{graphicx} 
%\usepackage[margins=2.5cm,nohead,nofoot]{geometry}
%\usepackage{geometry}
\usepackage{amsfonts}
\usepackage{amstext}
\usepackage{latexsym}
\usepackage{amssymb}
\usepackage{color}


%\include{myPreamble}
\include{qm2pi.local} 

%\ifpdf
%\usepackage[pdftex]{graphicx}
%\else
%\usepackage{graphicx}
%\fi

 % \ifpdf
%  \usepackage{pdfsync}
%  \if


%\title{Brief Article}
%\author{David F. Snyder}
%\author{L.G. Meredith}

%\address{Dept. of Math., Texas State University--San Marcos, San Marcos, TX 78666}
       
\pagestyle{empty}


\begin{document}

\lstset{language=[Objective]Caml,frame=shadowbox}

\input{qm2pi.front}

% section front matter (end)

\input{qm2pi.intro} 
 
% section introduction (end)

% \input{qm2pi.knotations} 

% section notation (end)

\input{qm2pi.process.calculi} 

% section concurrent_process_calculi_and_spatial_logics_ (end)
    
%\input{qm2pi.knots2pi} 

%\input{qm2pi.trefoil} 

%\input{qm2pi.mainthm} 

% subsection basic_interpretation (end)

%\input{qm2pi.rho.presentation} 
\subsection{The syntax and semantics of the notation system}\label{sub:the_syntax_and_semantics_of_the_notation_system} % (fold)

We now summarize a technical presentation of the calculus that
embodies our theory of dynamics. The typical presentation of such a
calculus follows the style of giving generators and relations on
them. The grammar, below, describing term constructors, freely
generates the set of processes, $\Proc$. This set is then quotiented
by a relation known as structural congruence and it is over this set
that the notion of dynamics is expressed. This presentation is
essentially that of \cite{MeredithR05} with the addition of
polyadicity and summation. For readability we have relegated some of
the technical subtleties to an appendix.

\subsubsection{Process grammar}\label{subsub:process_grammar}

\begin{mathpar}
  \inferrule* [lab=synchronization] {} {{M} \bc \pzero \;|\; x?F \;|\; x!C }
  \and
  \inferrule* [lab=abstraction] {} {{F} \bc (x)P}
  \and
  \inferrule* [lab=concretion] {} {{C} \bc \langle Q \rangle}
  \and
  \inferrule* [lab=process] {} {{P,Q} \bc M \;| \;P|Q \;|\; @{x}}
  \and
  \inferrule* [lab=name] {} {{x} \bc \quotep{P}}
\end{mathpar} 

Note that $\vec{x}$ (resp. $\vec{P}$) denotes a vector of names
(resp. processes) of length $|\vec{x}|$ (resp. $|\vec{P}|$). We adopt
the following useful abbreviations.

\begin{mathpar}
   x?(\vec{y}).P := x.(\vec{y})P \and  x\clift{\vec{P}} := x.\clift{\vec{P}}
   \and x!(y) := \lift{x}{\dropn{y}}
   \and \Pi_{i=0}^{n-1}P_i := P_0 | \ldots | P_{n-1}
\end{mathpar}

\subsubsection{Structural congruence}

\paragraph{Free and bound names and alpha-equivalence.} At the
core of structural equivalence is alpha-equivalence which identifies
process that are the same up to a change of variable. Formally, we
recognize the distinction between free and bound names. The free names
of a process, $\freenames{P}$, may be calculated recursively as
follows:

\begin{mathpar}
\freenames{\pzero} := \emptyset
  \and \\
  \freenames{x?(y).P} := \{ x \} \cup (\freenames{P} \setminus \{ y \})
  \and 
  \freenames{x!\langle P \rangle} := \{ x \} \cup \{ P \} 
  \and \\
  \freenames{P|Q} := \freenames{P} \cup \freenames{Q}
  \and \\
  \freenames{@{x}} := \{ x \}
\end{mathpar}

$\pi$
$\quotep{\pi}$

$\freenames{-} : \pi \to \mathcal{P}(\quotep{\pi})$

\begin{eqnarray*}
  \freenames{\pzero} & := & \emptyset \\
  \freenames{x?(y).P} & := & \{ x \} \cup (\freenames{P} \setminus \{ y \}) \\
  \freenames{x!\langle P \rangle} & := & \{ x \} \cup \{ P \} \\
  \freenames{P|Q} & := & \freenames{P} \cup \freenames{Q} \\
  \freenames{\dropn{x}} & := & \{ x \}
\end{eqnarray*}

The bound names of a process, $\boundnames{P}$, are those names occurring in $P$
that are not free. For example, in $x?(y).0$, the name $x$ is free, while $y$ is bound.

\begin{mathpar}
  \inferrule* [lab=monoidal-laws] {} { P|Q \equiv Q|P \and P|0 \equiv P \and P|(Q|R) \equiv (P|Q)|R }
\end{mathpar}

\begin{mathpar}
  \inferrule* [lab=alpha-equivalence] {} { (x)P \equiv (y)P\{y/x\} \and y \not\in \freenames{P} }
\end{mathpar}

\begin{definition}
Then two processes, $P,Q$, are alpha-equivalent if $P = Q\{\vec{y}/\vec{x}\}$ for
some $\vec{x} \in \boundnames{Q},\vec{y} \in \boundnames{P}$, where $Q\{\vec{y}/\vec{x}\}$
denotes the capture-avoiding substitution of $\vec{y}$ for $\vec{x}$ in $Q$.
\end{definition}

\begin{definition}
  The {\em structural congruence} \cite{SangiorgiWalker} , $\equiv$,
  between processes is the least congruence containing
  alpha-equivalence, satisfying the abelian monoid laws
  (associativity, commutativity and $\pzero$ as identity) for parallel
  composition $|$ and for summation $+$.
\end{definition}

\subsection{Name equivalence}

We take name equivalence, written $\nameeq$, to be the smallest
equivalence relation generated by the following rules.

\begin{mathpar}
\inferrule*[lab=Quote-drop]
{ }
{ \quotep{@{x}} \nameeq x }

\inferrule*[lab=Struct-equiv]
{ P \scong Q }
{ \quotep{P} \nameeq \quotep{Q} }
\end{mathpar}

The astute reader will have noticed that the mutual recursion of names
and processes imposes a mutual recursion on alpha-equivalence and
structural equivalence via name-equivalence. Fortunately, all of this
works out pleasantly and we may calculate in the natural way, free of
concern. The reader interested in the details is referred to the
appendix \ref{appendix:rho_details}.

\subsection{Substitution}

We use $\Proc$ for the set of processes, $\QProc$ for the set of
names, and $\id{\{}\vec{y} / \vec{x} \id{\}}$ to denote partial maps,
$s : \QProc \rightarrow \QProc$. A map, $s$ lifts, uniquely, to a map
on process terms, $\widehat{s} : \Proc \rightarrow \Proc$ by the
following equations.

\begin{mathpar}
  (0) \psubstp{Q}{P} := 0 \\
  (R \juxtap S) \psubstp{Q}{P}
  :=    
  (R)\psubstp{Q}{P} \juxtap (S) \psubstp{Q}{P} \\
  (x?(y).R) \psubstp{Q}{P}    
  :=    
  (x)\substp{Q}{P} (z)\concat( (R \psubstn{z}{y}) \psubstp{Q}{P} ) \\
  (\lift{x}{R}) \psubstp{Q}{P}  
  :=
  \lift{(x)\substp{Q}{P}}{ R \psubstp{Q}{P} } \\
%   (\dropn{x})  \psubstp{Q}{P}       
%   := 
%   \left\{ 
%     \begin{array}{ccc} 
%       \dropn{\quotep{Q}} & & x \nameeq \quotep{P} \\
%       \dropn{x} & & otherwise \\
%     \end{array}
%   \right. 
  (\dropn{x})  \psubstp{Q}{P}       
  := 
  \left\{ 
    \begin{array}{ccc} 
      Q & & x \nameeq \quotep{P} \\
      \dropn{x} & & otherwise \\
    \end{array}
  \right.
\end{mathpar}
 

where

\begin{eqnarray}
  (x)\id{\{} \lpquote Q \rpquote / \lpquote P \rpquote \id{\}}            = 
  \left\{ 
    \begin{array}{ccc}
      \lpquote Q \rpquote & & x \nameeq \lpquote P \rpquote \\
      x & & otherwise \\
    \end{array}
  \right. \nonumber
\end{eqnarray}

and $z$ is chosen distinct from $\quotep{P}$, $\quotep{Q}$, the free
names in $Q$, and all the names in $R$. Our $\alpha$-equivalence will
be built in the standard way from this substitution.

\begin{remark}\label{rem:no_self_referential_names}
  One consequence of these definitions is that $\forall P. \quotep{P}
  \not\in \freenames{P}$.
\end{remark}

\subsection{ Dynamic quote: an example }

Anticipating something of what's to come, consider applying the
substitution, $\widehat{\id{\{}u / z \id{\}}}$, to the following pair
of processes, $\lift{w}{y!(z)}$ and $w[ \lpquote y!(z) \rpquote ]$.

\begin{eqnarray}
	\lift{w}{y!(z)}\widehat{\id{\{}u / z \id{\}}}
		& = &
		\lift{w}{y!(u)} \nonumber\\
	w[ \lpquote y!(z) \rpquote ] \widehat{ \id{\{}u / z \id{\}} }
		& = &
		w[ \lpquote y!(z) \rpquote ] \nonumber
\end{eqnarray}

Because the body of the process between quotes is impervious to
substitution, we get radically different answers. In fact, by
examining the first process in an input context,
e.g. $x?(z).\lift{w}{y!(z)}$, we see that the process under the lift
operator may be shaped by prefixed inputs binding a name inside it. In
this sense, the lift operator will be seen as a way to dynamically
construct processes before reifying them as names.

Finally equipped with these standard features we can present the
dynamics of the calculus.

\subsubsection{Operational semantics} 

Finally, we introduce the computational dynamics. What marks these
algebras as distinct from other more traditionally studied algebraic
structures, e.g. vector spaces or polynomial rings, is the manner in
which dynamics is captured. In traditional structures, dynamics is typically
expressed through morphisms between such structures, as in linear maps
between vector spaces or morphisms between rings. In algebras
associated with the semantics of computation, the dynamics is
expressed as part of the algebraic structure itself, through a
reduction reduction relation typically denoted by $\red$. Below, we
give a recursive presentation of this relation for the calculus used
in the encoding.

$\red \subseteq \pi \times \pi$
$\red : \pi \to \mathcal{P}(\pi)$

\begin{mathpar}
  \inferrule* [lab=Comm] { \textsf{match}( x_{src}, x_{trgt} ) } { x_{trgt}?(y)P \; | \; x_{src}!\langle {Q} \rangle \red P\{\quotep{Q}/y}\} }
  \and \\
  \inferrule* [lab=Par] {{P} \red {P}'} {{{P} | {Q}} \red {{P}' | {Q}}}
  \and
  \inferrule* [lab=Equiv]{{{P} \scong {P}'} \andalso {{P}' \red {Q}'} \andalso {{Q}' \scong {Q}}}{{P} \red {Q}}
\end{mathpar}

\begin{eqnarray*}
  match_{\equiv} (\quotep{P},\quotep{Q}) & := & P \equiv Q \\
  match_{\dagger}(\quotep{P},\quotep{Q}) & := & \forall R. P|Q \red^{*} R => R \red^{*} 0 \\
  match_{K}(\quotep{P},\quotep{Q}) & := & K \mbox{ for some context } K
\end{eqnarray*}

$u?(x)P | u!\langle Q \rangle \red P\{\quotep{Q}/x\}$

%We write $\wred$ for $\red^*$, and $P\red$ if $\exists Q $ such that $ P \red Q$.
We write $P\red$ if $\exists Q $ such that $ P \red Q$ and $P\not\red$, otherwise.

\section{Replication}

As mentioned before, it is known that replication (and hence
recursion) can be implemented in a higher-order process algebra
\cite{SangiorgiWalker}. As our first example of calculation with the
machinery thus far presented we give the construction explicitly in
the {\rhoc}.

\begin{eqnarray}
	D_{x} & := & \prefix{x}{y}{(\binpar{\outputp{x}{y}}{@{y}})} \nonumber\\
	\bangp_{x}{P} & := & \binpar{{x}!\langle{\binpar{D_{x}}{P}}\rangle}{D_{x}} \nonumber
\end{eqnarray}

\begin{eqnarray}
	\bangp_{x}{P} & & \nonumber\\
	=
	& {x}!\langle{(\prefix{x}{y}{(\outputp{x}{y} | @{y})) | P}}\rangle 
	      | \prefix{x}{y}{(\outputp{x}{y} | @{y})} & \nonumber\\
	\red
	& (\outputp{x}{y} | @{y})\substn{\quotep{(\prefix{x}{y}{(@{y} | \outputp{x}{y})) | P}}}{y} & \nonumber\\
	=
	& \outputp{x}{\quotep{(\prefix{x}{y}{(\outputp{x}{y} | @{y})) | P}}}
	  | {(\prefix{x}{y}{(\outputp{x}{y} | @{y})) | P}} & \nonumber\\
	\red
	& \ldots & \nonumber\\
	\red^*
	& P | P | \ldots & \nonumber
\end{eqnarray}

Of course, this encoding, as an implementation, runs away, unfolding
$\bangp{P}$ eagerly. A lazier and more implementable replication
operator, restricted to input-guarded processes, may be obtained as follows.

\begin{eqnarray}
\bangp{\prefix{u}{v}{P}} 
	:= 
	\binpar{\lift{x}{\prefix{u}{v}{(\binpar{D(x)}{P})}}}{D(x)} \nonumber
\end{eqnarray}

\begin{remark}
  Note that the lazier definition still does not deal with summation
  or mixed summation (i.e. sums over input and output). The reader is
  invited to construct definitions of replication that deal with these
  features. 

  Further, the definitions are parameterized in a name, $x$. Can you,
  gentle reader, make a definition that eliminates this parameter and
  guarantees no accidental interaction between the replication
  machinery and the process being replicated -- i.e. no accidental
  sharing of names used by the process to get its work done and the
  name(s) used by the replication to effect copying. This latter
  revision of the definition of replication is crucial to obtaining
  the expected identity $!!P \sim !P$.
\end{remark}

\begin{remark}\label{rem:paradoxical_combinator}
  The reader familiar with the lambda calculus will have noticed the
  similarity between $D$ and the paradoxical combinator.

  [Ed. note: the existence of this seems to suggest we have to be more
  restrictive on the set of processes and names we admit if we are to
  support no-cloning.]
\end{remark}

\subsubsection{Bisimulation}

The computational dynamics gives rise to another kind of equivalence,
the equivalence of computational behavior. As previously mentioned
this is typically captured \emph{via} some form of bisimulation.

% The notion we use in this paper is weak barbed bisimulation
% \cite{milner91polyadicpi}.

The notion we use in this paper is derived from weak barbed
bisimulation \cite{milner91polyadicpi}. 

\begin{definition}
An \emph{observation relation}, $\downarrow_{\mathcal N}$, over a set
of names, $\mathcal N$, is the smallest relation satisfying the rules
below.

\infrule[Out-barb]{y \in {\mathcal N}, \; x \nameeq y}
		  {\outputp{x}{v} \downarrow_{\mathcal N} x}
\infrule[Par-barb]{\mbox{$P\downarrow_{\mathcal N} x$ or $Q\downarrow_{\mathcal N} x$}}
		  {\binpar{P}{Q} \downarrow_{\mathcal N} x}

We write $P \Downarrow_{\mathcal N} x$ if there is $Q$ such that 
$P \wred Q$ and $Q \downarrow_{\mathcal N} x$.
\end{definition}

\begin{definition}
%\label{def.bbisim}
An  ${\mathcal N}$-\emph{barbed bisimulation} over a set of names, ${\mathcal N}$, is a symmetric binary relation 
${\mathcal S}_{\mathcal N}$ between agents such that $P\rel{S}_{\mathcal N}Q$ implies:
\begin{enumerate}
\item If $P \red P'$ then $Q \wred Q'$ and $P'\rel{S}_{\mathcal N} Q'$.
\item If $P\downarrow_{\mathcal N} x$, then $Q\Downarrow_{\mathcal N} x$.
\end{enumerate}
$P$ is ${\mathcal N}$-barbed bisimilar to $Q$, written
$P \wbbisim_{\mathcal N} Q$, if $P \rel{S}_{\mathcal N} Q$ for some ${\mathcal N}$-barbed bisimulation ${\mathcal S}_{\mathcal N}$.
\end{definition}

$\mathcal{R} \subseteq \pi \times \pi$

$P \mathcal{R} Q => \forall P'. P \red P' \Rightarrow \exists Q'. Q \red Q', P' \mathcal{R} Q'$

$P \vdash x \Rightarrow Q \vdash x$

\begin{mathpar}
  \inferrule*[lab=Out-barb]{x \nameeq y}{{y}!\langle{Q}\rangle \vdash x}
  \and
  \inferrule*[lab=Par-barb]{\mbox{$P\vdash x$ or $Q\vdash x$}}{\binpar{P}{Q} \vdash x}
\end{mathpar}

\subsubsection{Contexts}

One of the principle advantages of computational calculi like the
$\pi$-calculus is a well-defined notion of context,
contextual-equivalence and a correlation between
contextual-equivalence and notions of bisimulation. The notion of
context allows the decomposition of a process into (sub-)process and
its syntactic environment, its context. Thus, a context may be
thought of as a process with a ``hole'' (written $\Box$) in it. The
application of a context $M$ to a process $P$, written $M[P]$, is
tantamount to filling the hole in $M$ with $P$. In this paper we do
not need the full weight of this theory, but do make use of the notion
of context in the proof the main theorem. 

\begin{mathpar}
  \inferrule* [lab=summation] {} {{M_{M},M_{N}} \bc \Box \;|\; x.M_{A} \;|\; M_{M}+M_{N}}
  \and
  \inferrule* [lab=agent] {} {{M_{A}} \bc (\vec{x})M_{P} \;| \; \clift{P_0,\ldots,M_{P},\ldots,P_N}}
  \and \\
  \inferrule* [lab=process] {} {{M_{P}} \bc M_{N} \;| \;P|M_{P} }
\end{mathpar} 

\begin{mathpar}
  \inferrule* [lab=sychronization] {} {M_{N} \bc \Box \;|\; x?M_{F} \;|\; x!M_{C}}
  \and
  \inferrule* [lab=abstraction] {} {{M_{F}} \bc (x)M_{P} }
  \and
  \inferrule* [lab=concretion] {} {{M_{C}} \bc \langle M_{P} \rangle }
  \and \\
  \inferrule* [lab=process] {} {{M_{P}} \bc M_{N} \;| \;P|M_{P} }
\end{mathpar}

\begin{definition}[contextual application] Given a context $M$, and
  process $P$, we define the \emph{contextual application}, $M[P] :=
  M\{P/\Box\}$. That is, the contextual application of M to P is the
  substitution of $P$ for $\Box$ in $M$.
\end{definition}

$\meaningof{-} : L \to \mathcal{P}(\pi)$

\begin{mathpar}
  \inferrule* [lab=collection] {} {\meaningof{true} = \pi, \and \meaningof{~E} = \pi \setminus \meaningof{E}, \and \meaningof{E_{1} \& E_{2}} = \meaningof{E_{1}} \cap \meaningof{E_{2}}}
\end{mathpar}

\begin{mathpar}
  \inferrule* [lab=structure] {} {\meaningof{0} = \{ P \in \pi | P \equiv 0 \}, \and \\ \meaningof{E_1 | E_2} = \{ P \in \pi | P \equiv P_{1} | P_{2}, P_{1} \in \meaningof{E_{1}}, P_{2} \in \meaningof{E_2}\} }
\end{mathpar}

\begin{mathpar}
 \inferrule* [lab=behavior] {} {\meaningof{\langle a?b \rangle E} = \{ P \in \pi | P \equiv Q | u?(y)P', \\ \and \\\\ \and \\ \;\;\; u \in \meaningof{a}, \forall z.P'\{z/y\} \in \meaningof{E\{z/b\}}\}, \and \\ \meaningof{a!E} = \{ P \in \pi | P \equiv Q | x!\langle P' \rangle, x \in \meaningof{a} P' \in \meaningof{E}\} }
\end{mathpar}

\begin{mathpar}
 \inferrule* [lab=nominal] {} {\meaningof{\quotep{E}} = \{ \quotep{P} \in \quotep{\pi} | P \in \meaningof{E} \}, \and \meaningof{\quotep{P}} = \{ \quotep{Q} \in \quotep{\pi} | P \equiv Q \} \and \\ \meaningof{@\quotep{E}} = \{ P \in \pi | P \equiv @x, x \in \meaningof{E} \}}
\end{mathpar}

\begin{eqnarray*}
  \\
  \meaningof{-} : TS \to ST
\end{eqnarray*}

\begin{eqnarray*}
  \\
  L : TS \to ST
\end{eqnarray*}

\begin{eqnarray*}
  \\
  P \models E \iff P \in \meaningof{E}
\end{eqnarray*}

\begin{eqnarray*}
  P \approx_{L} Q \iff \forall E \in L. P \models E \iff Q \models E
\end{eqnarray*}

\begin{eqnarray*}
  P \approx_{K} Q
\end{eqnarray*}

\begin{eqnarray*}
  P \approx Q
\end{eqnarray*}

$\approx_{K} = \approx = \approx_{L}$

\subsubsection{Contextual duality}

Note that contexts extend the quotation operation to a family of
operations from processes to names. Given a context, $M$, we can
define a \emph{nominal context}, $\quotep{M}$ by $\quotep{M}[P] :=
\quotep{M[P]}$. To foreshadow what is to come we observe that these
operations enjoy a duality with processes very much like the duality
between vectors and maps from vectors to scalars.

Further, because the calculus is essentially higher-order, we have a
correspondence between contexts and processes. More specifically,
given a name $x$ and a context $M$ we can construct $M^{*}_{x}$ such
that 

\begin{mathpar}
  M^{*}_{x} | \lift{x}{P} \red M[P]
\end{mathpar}

namely,

\begin{mathpar}
  M^{*}_{x} := x?(u).M[\dropn{u}]
\end{mathpar}

The dependence of $M^{*}_{x}$ on a name makes it an abstraction, 

\begin{mathpar}
  M^{*} := (x)x?(u).M[\dropn{u}]
\end{mathpar}

\subsection{Additional notation}

It will sometimes be convenient to denote the process a name
quotes. We already have the notation $x = \quotep{P}$, but it will be
convenient to introduce an alternate notation, $\procn{x}$, when we
want to emphasize the connection to the use of the name. Note that, by
virtue of name equivalence, $\quotep{\procn{x}} \nameeq x$; so, the
notation is consistent with previous definitions.

Further, because names have structure it is possible to effect
substitutions on the basis of that structure. This means we need to
upgrade our notation for substitutions, which we accomplish by
adapting comprehension notation. Thus,

\begin{mathpar}
  P\{ y / x : x \in S \}
\end{mathpar}

is interpreted to mean the process derived from P by replacing (in a
capture-avoiding manner) each occurrence of $x$ in $S$ by $y$. For example,

\begin{mathpar}
  P\{ \quotep{\procn{x}|\procn{x}} / x : x \in \freenames{P} \}
\end{mathpar}

will replace each (occurrence) of a free name $x$ in $P$ by
$\quotep{\procn{x}|\procn{x}}$.

Also, we will avail ourselves of the notation $x^{L}$ and $x^{R}$ to
denote injections of a name into disjoint copies of the name
space. There are numerous ways to accomplish this. One example can be
found in \cite{MeredithR05}. This notation overloads to vectors of
names: $\vec{x}^{\pi} := (x_{i}^{\pi} \; : \; 0 \leq i < |\vec{x}| )$ where $\pi \in \{L,R\}$.

We also use $P^{\Box} := P|\Box$.

In \cite{MeredithR05} an interpretation of the new operator is
given. It turns out that there are several possible interpretations
all enjoying the requisite algebraic properties of the operator (see
\cite{milner91polyadicpi}). We will therefore make liberal use of
$(\nu\; \vec{x})P$.

% subsection the_syntax_and_semantics_of_the_notation_system (end)   

\input{qm2pi.qmops} 

\input{qm2pi.sterngerlach} 

\input{qm2pi.metric} 

% section concurrent_process_calculi (end)

%\input{qm2pi.proofsketch}

% section proof sketch (end)

%\input{qm2pi.slviaknots} 

% section spatial logic via knots (end)

\input{qm2pi.conclusion}

% section conclusion (end)

%\input{qm2pi.dtcodes} 

% section wiring algorithm (end)

\input{qm2pi.ack} 

% section acknowledgments (end)

\newpage


\bibliographystyle{plain}   
\bibliography{../../biblios/main.bib}

\input{qm2pi.rhodetails}

\end{document}

 

%\documentclass[12pt]{llncs}
%\documentclass{jktr}

\usepackage[pdftex]{hyperref}                   
\usepackage {listings}
\usepackage {mathpartir}
\usepackage{bcprules}
%\usepackage{listings}
                       
\usepackage{graphicx} 
%\usepackage[margins=2.5cm,nohead,nofoot]{geometry}
%\usepackage{geometry}
\usepackage{amsfonts}
\usepackage{amstext}
\usepackage{latexsym}
\usepackage{amssymb}
\usepackage{color}


%\include{myPreamble}
\include{qm2pi.local} 

%\ifpdf
%\usepackage[pdftex]{graphicx}
%\else
%\usepackage{graphicx}
%\fi

 % \ifpdf
%  \usepackage{pdfsync}
%  \if


%\title{Brief Article}
%\author{David F. Snyder}
%\author{L.G. Meredith}

%\address{Dept. of Math., Texas State University--San Marcos, San Marcos, TX 78666}
       
\pagestyle{empty}


\begin{document}

\lstset{language=[Objective]Caml,frame=shadowbox}

\input{qm2pi.front}

% section front matter (end)

\input{qm2pi.intro} 
 
% section introduction (end)

% \input{qm2pi.knotations} 

% section notation (end)

\input{qm2pi.process.calculi} 

% section concurrent_process_calculi_and_spatial_logics_ (end)
    
%\input{qm2pi.knots2pi} 

%\input{qm2pi.trefoil} 

%\input{qm2pi.mainthm} 

% subsection basic_interpretation (end)

%\input{qm2pi.rho.presentation} 
\subsection{The syntax and semantics of the notation system}\label{sub:the_syntax_and_semantics_of_the_notation_system} % (fold)

We now summarize a technical presentation of the calculus that
embodies our theory of dynamics. The typical presentation of such a
calculus follows the style of giving generators and relations on
them. The grammar, below, describing term constructors, freely
generates the set of processes, $\Proc$. This set is then quotiented
by a relation known as structural congruence and it is over this set
that the notion of dynamics is expressed. This presentation is
essentially that of \cite{MeredithR05} with the addition of
polyadicity and summation. For readability we have relegated some of
the technical subtleties to an appendix.

\subsubsection{Process grammar}\label{subsub:process_grammar}

\begin{mathpar}
  \inferrule* [lab=synchronization] {} {{M} \bc \pzero \;|\; x?F \;|\; x!C }
  \and
  \inferrule* [lab=abstraction] {} {{F} \bc (x)P}
  \and
  \inferrule* [lab=concretion] {} {{C} \bc \langle Q \rangle}
  \and
  \inferrule* [lab=process] {} {{P,Q} \bc M \;| \;P|Q \;|\; @{x}}
  \and
  \inferrule* [lab=name] {} {{x} \bc \quotep{P}}
\end{mathpar} 

Note that $\vec{x}$ (resp. $\vec{P}$) denotes a vector of names
(resp. processes) of length $|\vec{x}|$ (resp. $|\vec{P}|$). We adopt
the following useful abbreviations.

\begin{mathpar}
   x?(\vec{y}).P := x.(\vec{y})P \and  x\clift{\vec{P}} := x.\clift{\vec{P}}
   \and x!(y) := \lift{x}{\dropn{y}}
   \and \Pi_{i=0}^{n-1}P_i := P_0 | \ldots | P_{n-1}
\end{mathpar}

\subsubsection{Structural congruence}

\paragraph{Free and bound names and alpha-equivalence.} At the
core of structural equivalence is alpha-equivalence which identifies
process that are the same up to a change of variable. Formally, we
recognize the distinction between free and bound names. The free names
of a process, $\freenames{P}$, may be calculated recursively as
follows:

\begin{mathpar}
\freenames{\pzero} := \emptyset
  \and \\
  \freenames{x?(y).P} := \{ x \} \cup (\freenames{P} \setminus \{ y \})
  \and 
  \freenames{x!\langle P \rangle} := \{ x \} \cup \{ P \} 
  \and \\
  \freenames{P|Q} := \freenames{P} \cup \freenames{Q}
  \and \\
  \freenames{@{x}} := \{ x \}
\end{mathpar}

$\pi$
$\quotep{\pi}$

$\freenames{-} : \pi \to \mathcal{P}(\quotep{\pi})$

\begin{eqnarray*}
  \freenames{\pzero} & := & \emptyset \\
  \freenames{x?(y).P} & := & \{ x \} \cup (\freenames{P} \setminus \{ y \}) \\
  \freenames{x!\langle P \rangle} & := & \{ x \} \cup \{ P \} \\
  \freenames{P|Q} & := & \freenames{P} \cup \freenames{Q} \\
  \freenames{\dropn{x}} & := & \{ x \}
\end{eqnarray*}

The bound names of a process, $\boundnames{P}$, are those names occurring in $P$
that are not free. For example, in $x?(y).0$, the name $x$ is free, while $y$ is bound.

\begin{mathpar}
  \inferrule* [lab=monoidal-laws] {} { P|Q \equiv Q|P \and P|0 \equiv P \and P|(Q|R) \equiv (P|Q)|R }
\end{mathpar}

\begin{mathpar}
  \inferrule* [lab=alpha-equivalence] {} { (x)P \equiv (y)P\{y/x\} \and y \not\in \freenames{P} }
\end{mathpar}

\begin{definition}
Then two processes, $P,Q$, are alpha-equivalent if $P = Q\{\vec{y}/\vec{x}\}$ for
some $\vec{x} \in \boundnames{Q},\vec{y} \in \boundnames{P}$, where $Q\{\vec{y}/\vec{x}\}$
denotes the capture-avoiding substitution of $\vec{y}$ for $\vec{x}$ in $Q$.
\end{definition}

\begin{definition}
  The {\em structural congruence} \cite{SangiorgiWalker} , $\equiv$,
  between processes is the least congruence containing
  alpha-equivalence, satisfying the abelian monoid laws
  (associativity, commutativity and $\pzero$ as identity) for parallel
  composition $|$ and for summation $+$.
\end{definition}

\subsection{Name equivalence}

We take name equivalence, written $\nameeq$, to be the smallest
equivalence relation generated by the following rules.

\begin{mathpar}
\inferrule*[lab=Quote-drop]
{ }
{ \quotep{@{x}} \nameeq x }

\inferrule*[lab=Struct-equiv]
{ P \scong Q }
{ \quotep{P} \nameeq \quotep{Q} }
\end{mathpar}

The astute reader will have noticed that the mutual recursion of names
and processes imposes a mutual recursion on alpha-equivalence and
structural equivalence via name-equivalence. Fortunately, all of this
works out pleasantly and we may calculate in the natural way, free of
concern. The reader interested in the details is referred to the
appendix \ref{appendix:rho_details}.

\subsection{Substitution}

We use $\Proc$ for the set of processes, $\QProc$ for the set of
names, and $\id{\{}\vec{y} / \vec{x} \id{\}}$ to denote partial maps,
$s : \QProc \rightarrow \QProc$. A map, $s$ lifts, uniquely, to a map
on process terms, $\widehat{s} : \Proc \rightarrow \Proc$ by the
following equations.

\begin{mathpar}
  (0) \psubstp{Q}{P} := 0 \\
  (R \juxtap S) \psubstp{Q}{P}
  :=    
  (R)\psubstp{Q}{P} \juxtap (S) \psubstp{Q}{P} \\
  (x?(y).R) \psubstp{Q}{P}    
  :=    
  (x)\substp{Q}{P} (z)\concat( (R \psubstn{z}{y}) \psubstp{Q}{P} ) \\
  (\lift{x}{R}) \psubstp{Q}{P}  
  :=
  \lift{(x)\substp{Q}{P}}{ R \psubstp{Q}{P} } \\
%   (\dropn{x})  \psubstp{Q}{P}       
%   := 
%   \left\{ 
%     \begin{array}{ccc} 
%       \dropn{\quotep{Q}} & & x \nameeq \quotep{P} \\
%       \dropn{x} & & otherwise \\
%     \end{array}
%   \right. 
  (\dropn{x})  \psubstp{Q}{P}       
  := 
  \left\{ 
    \begin{array}{ccc} 
      Q & & x \nameeq \quotep{P} \\
      \dropn{x} & & otherwise \\
    \end{array}
  \right.
\end{mathpar}
 

where

\begin{eqnarray}
  (x)\id{\{} \lpquote Q \rpquote / \lpquote P \rpquote \id{\}}            = 
  \left\{ 
    \begin{array}{ccc}
      \lpquote Q \rpquote & & x \nameeq \lpquote P \rpquote \\
      x & & otherwise \\
    \end{array}
  \right. \nonumber
\end{eqnarray}

and $z$ is chosen distinct from $\quotep{P}$, $\quotep{Q}$, the free
names in $Q$, and all the names in $R$. Our $\alpha$-equivalence will
be built in the standard way from this substitution.

\begin{remark}\label{rem:no_self_referential_names}
  One consequence of these definitions is that $\forall P. \quotep{P}
  \not\in \freenames{P}$.
\end{remark}

\subsection{ Dynamic quote: an example }

Anticipating something of what's to come, consider applying the
substitution, $\widehat{\id{\{}u / z \id{\}}}$, to the following pair
of processes, $\lift{w}{y!(z)}$ and $w[ \lpquote y!(z) \rpquote ]$.

\begin{eqnarray}
	\lift{w}{y!(z)}\widehat{\id{\{}u / z \id{\}}}
		& = &
		\lift{w}{y!(u)} \nonumber\\
	w[ \lpquote y!(z) \rpquote ] \widehat{ \id{\{}u / z \id{\}} }
		& = &
		w[ \lpquote y!(z) \rpquote ] \nonumber
\end{eqnarray}

Because the body of the process between quotes is impervious to
substitution, we get radically different answers. In fact, by
examining the first process in an input context,
e.g. $x?(z).\lift{w}{y!(z)}$, we see that the process under the lift
operator may be shaped by prefixed inputs binding a name inside it. In
this sense, the lift operator will be seen as a way to dynamically
construct processes before reifying them as names.

Finally equipped with these standard features we can present the
dynamics of the calculus.

\subsubsection{Operational semantics} 

Finally, we introduce the computational dynamics. What marks these
algebras as distinct from other more traditionally studied algebraic
structures, e.g. vector spaces or polynomial rings, is the manner in
which dynamics is captured. In traditional structures, dynamics is typically
expressed through morphisms between such structures, as in linear maps
between vector spaces or morphisms between rings. In algebras
associated with the semantics of computation, the dynamics is
expressed as part of the algebraic structure itself, through a
reduction reduction relation typically denoted by $\red$. Below, we
give a recursive presentation of this relation for the calculus used
in the encoding.

$\red \subseteq \pi \times \pi$
$\red : \pi \to \mathcal{P}(\pi)$

\begin{mathpar}
  \inferrule* [lab=Comm] { \textsf{match}( x_{src}, x_{trgt} ) } { x_{trgt}?(y)P \; | \; x_{src}!\langle {Q} \rangle \red P\{\quotep{Q}/y}\} }
  \and \\
  \inferrule* [lab=Par] {{P} \red {P}'} {{{P} | {Q}} \red {{P}' | {Q}}}
  \and
  \inferrule* [lab=Equiv]{{{P} \scong {P}'} \andalso {{P}' \red {Q}'} \andalso {{Q}' \scong {Q}}}{{P} \red {Q}}
\end{mathpar}

\begin{eqnarray*}
  match_{\equiv} (\quotep{P},\quotep{Q}) & := & P \equiv Q \\
  match_{\dagger}(\quotep{P},\quotep{Q}) & := & \forall R. P|Q \red^{*} R => R \red^{*} 0 \\
  match_{K}(\quotep{P},\quotep{Q}) & := & K \mbox{ for some context } K
\end{eqnarray*}

$u?(x)P | u!\langle Q \rangle \red P\{\quotep{Q}/x\}$

%We write $\wred$ for $\red^*$, and $P\red$ if $\exists Q $ such that $ P \red Q$.
We write $P\red$ if $\exists Q $ such that $ P \red Q$ and $P\not\red$, otherwise.

\section{Replication}

As mentioned before, it is known that replication (and hence
recursion) can be implemented in a higher-order process algebra
\cite{SangiorgiWalker}. As our first example of calculation with the
machinery thus far presented we give the construction explicitly in
the {\rhoc}.

\begin{eqnarray}
	D_{x} & := & \prefix{x}{y}{(\binpar{\outputp{x}{y}}{@{y}})} \nonumber\\
	\bangp_{x}{P} & := & \binpar{{x}!\langle{\binpar{D_{x}}{P}}\rangle}{D_{x}} \nonumber
\end{eqnarray}

\begin{eqnarray}
	\bangp_{x}{P} & & \nonumber\\
	=
	& {x}!\langle{(\prefix{x}{y}{(\outputp{x}{y} | @{y})) | P}}\rangle 
	      | \prefix{x}{y}{(\outputp{x}{y} | @{y})} & \nonumber\\
	\red
	& (\outputp{x}{y} | @{y})\substn{\quotep{(\prefix{x}{y}{(@{y} | \outputp{x}{y})) | P}}}{y} & \nonumber\\
	=
	& \outputp{x}{\quotep{(\prefix{x}{y}{(\outputp{x}{y} | @{y})) | P}}}
	  | {(\prefix{x}{y}{(\outputp{x}{y} | @{y})) | P}} & \nonumber\\
	\red
	& \ldots & \nonumber\\
	\red^*
	& P | P | \ldots & \nonumber
\end{eqnarray}

Of course, this encoding, as an implementation, runs away, unfolding
$\bangp{P}$ eagerly. A lazier and more implementable replication
operator, restricted to input-guarded processes, may be obtained as follows.

\begin{eqnarray}
\bangp{\prefix{u}{v}{P}} 
	:= 
	\binpar{\lift{x}{\prefix{u}{v}{(\binpar{D(x)}{P})}}}{D(x)} \nonumber
\end{eqnarray}

\begin{remark}
  Note that the lazier definition still does not deal with summation
  or mixed summation (i.e. sums over input and output). The reader is
  invited to construct definitions of replication that deal with these
  features. 

  Further, the definitions are parameterized in a name, $x$. Can you,
  gentle reader, make a definition that eliminates this parameter and
  guarantees no accidental interaction between the replication
  machinery and the process being replicated -- i.e. no accidental
  sharing of names used by the process to get its work done and the
  name(s) used by the replication to effect copying. This latter
  revision of the definition of replication is crucial to obtaining
  the expected identity $!!P \sim !P$.
\end{remark}

\begin{remark}\label{rem:paradoxical_combinator}
  The reader familiar with the lambda calculus will have noticed the
  similarity between $D$ and the paradoxical combinator.

  [Ed. note: the existence of this seems to suggest we have to be more
  restrictive on the set of processes and names we admit if we are to
  support no-cloning.]
\end{remark}

\subsubsection{Bisimulation}

The computational dynamics gives rise to another kind of equivalence,
the equivalence of computational behavior. As previously mentioned
this is typically captured \emph{via} some form of bisimulation.

% The notion we use in this paper is weak barbed bisimulation
% \cite{milner91polyadicpi}.

The notion we use in this paper is derived from weak barbed
bisimulation \cite{milner91polyadicpi}. 

\begin{definition}
An \emph{observation relation}, $\downarrow_{\mathcal N}$, over a set
of names, $\mathcal N$, is the smallest relation satisfying the rules
below.

\infrule[Out-barb]{y \in {\mathcal N}, \; x \nameeq y}
		  {\outputp{x}{v} \downarrow_{\mathcal N} x}
\infrule[Par-barb]{\mbox{$P\downarrow_{\mathcal N} x$ or $Q\downarrow_{\mathcal N} x$}}
		  {\binpar{P}{Q} \downarrow_{\mathcal N} x}

We write $P \Downarrow_{\mathcal N} x$ if there is $Q$ such that 
$P \wred Q$ and $Q \downarrow_{\mathcal N} x$.
\end{definition}

\begin{definition}
%\label{def.bbisim}
An  ${\mathcal N}$-\emph{barbed bisimulation} over a set of names, ${\mathcal N}$, is a symmetric binary relation 
${\mathcal S}_{\mathcal N}$ between agents such that $P\rel{S}_{\mathcal N}Q$ implies:
\begin{enumerate}
\item If $P \red P'$ then $Q \wred Q'$ and $P'\rel{S}_{\mathcal N} Q'$.
\item If $P\downarrow_{\mathcal N} x$, then $Q\Downarrow_{\mathcal N} x$.
\end{enumerate}
$P$ is ${\mathcal N}$-barbed bisimilar to $Q$, written
$P \wbbisim_{\mathcal N} Q$, if $P \rel{S}_{\mathcal N} Q$ for some ${\mathcal N}$-barbed bisimulation ${\mathcal S}_{\mathcal N}$.
\end{definition}

$\mathcal{R} \subseteq \pi \times \pi$

$P \mathcal{R} Q => \forall P'. P \red P' \Rightarrow \exists Q'. Q \red Q', P' \mathcal{R} Q'$

$P \vdash x \Rightarrow Q \vdash x$

\begin{mathpar}
  \inferrule*[lab=Out-barb]{x \nameeq y}{{y}!\langle{Q}\rangle \vdash x}
  \and
  \inferrule*[lab=Par-barb]{\mbox{$P\vdash x$ or $Q\vdash x$}}{\binpar{P}{Q} \vdash x}
\end{mathpar}

\subsubsection{Contexts}

One of the principle advantages of computational calculi like the
$\pi$-calculus is a well-defined notion of context,
contextual-equivalence and a correlation between
contextual-equivalence and notions of bisimulation. The notion of
context allows the decomposition of a process into (sub-)process and
its syntactic environment, its context. Thus, a context may be
thought of as a process with a ``hole'' (written $\Box$) in it. The
application of a context $M$ to a process $P$, written $M[P]$, is
tantamount to filling the hole in $M$ with $P$. In this paper we do
not need the full weight of this theory, but do make use of the notion
of context in the proof the main theorem. 

\begin{mathpar}
  \inferrule* [lab=summation] {} {{M_{M},M_{N}} \bc \Box \;|\; x.M_{A} \;|\; M_{M}+M_{N}}
  \and
  \inferrule* [lab=agent] {} {{M_{A}} \bc (\vec{x})M_{P} \;| \; \clift{P_0,\ldots,M_{P},\ldots,P_N}}
  \and \\
  \inferrule* [lab=process] {} {{M_{P}} \bc M_{N} \;| \;P|M_{P} }
\end{mathpar} 

\begin{mathpar}
  \inferrule* [lab=sychronization] {} {M_{N} \bc \Box \;|\; x?M_{F} \;|\; x!M_{C}}
  \and
  \inferrule* [lab=abstraction] {} {{M_{F}} \bc (x)M_{P} }
  \and
  \inferrule* [lab=concretion] {} {{M_{C}} \bc \langle M_{P} \rangle }
  \and \\
  \inferrule* [lab=process] {} {{M_{P}} \bc M_{N} \;| \;P|M_{P} }
\end{mathpar}

\begin{definition}[contextual application] Given a context $M$, and
  process $P$, we define the \emph{contextual application}, $M[P] :=
  M\{P/\Box\}$. That is, the contextual application of M to P is the
  substitution of $P$ for $\Box$ in $M$.
\end{definition}

$\meaningof{-} : L \to \mathcal{P}(\pi)$

\begin{mathpar}
  \inferrule* [lab=collection] {} {\meaningof{true} = \pi, \and \meaningof{~E} = \pi \setminus \meaningof{E}, \and \meaningof{E_{1} \& E_{2}} = \meaningof{E_{1}} \cap \meaningof{E_{2}}}
\end{mathpar}

\begin{mathpar}
  \inferrule* [lab=structure] {} {\meaningof{0} = \{ P \in \pi | P \equiv 0 \}, \and \\ \meaningof{E_1 | E_2} = \{ P \in \pi | P \equiv P_{1} | P_{2}, P_{1} \in \meaningof{E_{1}}, P_{2} \in \meaningof{E_2}\} }
\end{mathpar}

\begin{mathpar}
 \inferrule* [lab=behavior] {} {\meaningof{\langle a?b \rangle E} = \{ P \in \pi | P \equiv Q | u?(y)P', \\ \and \\\\ \and \\ \;\;\; u \in \meaningof{a}, \forall z.P'\{z/y\} \in \meaningof{E\{z/b\}}\}, \and \\ \meaningof{a!E} = \{ P \in \pi | P \equiv Q | x!\langle P' \rangle, x \in \meaningof{a} P' \in \meaningof{E}\} }
\end{mathpar}

\begin{mathpar}
 \inferrule* [lab=nominal] {} {\meaningof{\quotep{E}} = \{ \quotep{P} \in \quotep{\pi} | P \in \meaningof{E} \}, \and \meaningof{\quotep{P}} = \{ \quotep{Q} \in \quotep{\pi} | P \equiv Q \} \and \\ \meaningof{@\quotep{E}} = \{ P \in \pi | P \equiv @x, x \in \meaningof{E} \}}
\end{mathpar}

\begin{eqnarray*}
  \\
  \meaningof{-} : TS \to ST
\end{eqnarray*}

\begin{eqnarray*}
  \\
  L : TS \to ST
\end{eqnarray*}

\begin{eqnarray*}
  \\
  P \models E \iff P \in \meaningof{E}
\end{eqnarray*}

\begin{eqnarray*}
  P \approx_{L} Q \iff \forall E \in L. P \models E \iff Q \models E
\end{eqnarray*}

\begin{eqnarray*}
  P \approx_{K} Q
\end{eqnarray*}

\begin{eqnarray*}
  P \approx Q
\end{eqnarray*}

$\approx_{K} = \approx = \approx_{L}$

\subsubsection{Contextual duality}

Note that contexts extend the quotation operation to a family of
operations from processes to names. Given a context, $M$, we can
define a \emph{nominal context}, $\quotep{M}$ by $\quotep{M}[P] :=
\quotep{M[P]}$. To foreshadow what is to come we observe that these
operations enjoy a duality with processes very much like the duality
between vectors and maps from vectors to scalars.

Further, because the calculus is essentially higher-order, we have a
correspondence between contexts and processes. More specifically,
given a name $x$ and a context $M$ we can construct $M^{*}_{x}$ such
that 

\begin{mathpar}
  M^{*}_{x} | \lift{x}{P} \red M[P]
\end{mathpar}

namely,

\begin{mathpar}
  M^{*}_{x} := x?(u).M[\dropn{u}]
\end{mathpar}

The dependence of $M^{*}_{x}$ on a name makes it an abstraction, 

\begin{mathpar}
  M^{*} := (x)x?(u).M[\dropn{u}]
\end{mathpar}

\subsection{Additional notation}

It will sometimes be convenient to denote the process a name
quotes. We already have the notation $x = \quotep{P}$, but it will be
convenient to introduce an alternate notation, $\procn{x}$, when we
want to emphasize the connection to the use of the name. Note that, by
virtue of name equivalence, $\quotep{\procn{x}} \nameeq x$; so, the
notation is consistent with previous definitions.

Further, because names have structure it is possible to effect
substitutions on the basis of that structure. This means we need to
upgrade our notation for substitutions, which we accomplish by
adapting comprehension notation. Thus,

\begin{mathpar}
  P\{ y / x : x \in S \}
\end{mathpar}

is interpreted to mean the process derived from P by replacing (in a
capture-avoiding manner) each occurrence of $x$ in $S$ by $y$. For example,

\begin{mathpar}
  P\{ \quotep{\procn{x}|\procn{x}} / x : x \in \freenames{P} \}
\end{mathpar}

will replace each (occurrence) of a free name $x$ in $P$ by
$\quotep{\procn{x}|\procn{x}}$.

Also, we will avail ourselves of the notation $x^{L}$ and $x^{R}$ to
denote injections of a name into disjoint copies of the name
space. There are numerous ways to accomplish this. One example can be
found in \cite{MeredithR05}. This notation overloads to vectors of
names: $\vec{x}^{\pi} := (x_{i}^{\pi} \; : \; 0 \leq i < |\vec{x}| )$ where $\pi \in \{L,R\}$.

We also use $P^{\Box} := P|\Box$.

In \cite{MeredithR05} an interpretation of the new operator is
given. It turns out that there are several possible interpretations
all enjoying the requisite algebraic properties of the operator (see
\cite{milner91polyadicpi}). We will therefore make liberal use of
$(\nu\; \vec{x})P$.

% subsection the_syntax_and_semantics_of_the_notation_system (end)   

\input{qm2pi.qmops} 

\input{qm2pi.sterngerlach} 

\input{qm2pi.metric} 

% section concurrent_process_calculi (end)

%\input{qm2pi.proofsketch}

% section proof sketch (end)

%\input{qm2pi.slviaknots} 

% section spatial logic via knots (end)

\input{qm2pi.conclusion}

% section conclusion (end)

%\input{qm2pi.dtcodes} 

% section wiring algorithm (end)

\input{qm2pi.ack} 

% section acknowledgments (end)

\newpage


\bibliographystyle{plain}   
\bibliography{../../biblios/main.bib}

\input{qm2pi.rhodetails}

\end{document}

 

% subsection basic_interpretation (end)

%\input{qm2pi.rho.presentation} 
\subsection{The syntax and semantics of the notation system}\label{sub:the_syntax_and_semantics_of_the_notation_system} % (fold)

We now summarize a technical presentation of the calculus that
embodies our theory of dynamics. The typical presentation of such a
calculus follows the style of giving generators and relations on
them. The grammar, below, describing term constructors, freely
generates the set of processes, $\Proc$. This set is then quotiented
by a relation known as structural congruence and it is over this set
that the notion of dynamics is expressed. This presentation is
essentially that of \cite{MeredithR05} with the addition of
polyadicity and summation. For readability we have relegated some of
the technical subtleties to an appendix.

\subsubsection{Process grammar}\label{subsub:process_grammar}

\begin{mathpar}
  \inferrule* [lab=synchronization] {} {{M} \bc \pzero \;|\; x?F \;|\; x!C }
  \and
  \inferrule* [lab=abstraction] {} {{F} \bc (x)P}
  \and
  \inferrule* [lab=concretion] {} {{C} \bc \langle Q \rangle}
  \and
  \inferrule* [lab=process] {} {{P,Q} \bc M \;| \;P|Q \;|\; @{x}}
  \and
  \inferrule* [lab=name] {} {{x} \bc \quotep{P}}
\end{mathpar} 

Note that $\vec{x}$ (resp. $\vec{P}$) denotes a vector of names
(resp. processes) of length $|\vec{x}|$ (resp. $|\vec{P}|$). We adopt
the following useful abbreviations.

\begin{mathpar}
   x?(\vec{y}).P := x.(\vec{y})P \and  x\clift{\vec{P}} := x.\clift{\vec{P}}
   \and x!(y) := \lift{x}{\dropn{y}}
   \and \Pi_{i=0}^{n-1}P_i := P_0 | \ldots | P_{n-1}
\end{mathpar}

\subsubsection{Structural congruence}

\paragraph{Free and bound names and alpha-equivalence.} At the
core of structural equivalence is alpha-equivalence which identifies
process that are the same up to a change of variable. Formally, we
recognize the distinction between free and bound names. The free names
of a process, $\freenames{P}$, may be calculated recursively as
follows:

\begin{mathpar}
\freenames{\pzero} := \emptyset
  \and \\
  \freenames{x?(y).P} := \{ x \} \cup (\freenames{P} \setminus \{ y \})
  \and 
  \freenames{x!\langle P \rangle} := \{ x \} \cup \{ P \} 
  \and \\
  \freenames{P|Q} := \freenames{P} \cup \freenames{Q}
  \and \\
  \freenames{@{x}} := \{ x \}
\end{mathpar}

$\pi$
$\quotep{\pi}$

$\freenames{-} : \pi \to \mathcal{P}(\quotep{\pi})$

\begin{eqnarray*}
  \freenames{\pzero} & := & \emptyset \\
  \freenames{x?(y).P} & := & \{ x \} \cup (\freenames{P} \setminus \{ y \}) \\
  \freenames{x!\langle P \rangle} & := & \{ x \} \cup \{ P \} \\
  \freenames{P|Q} & := & \freenames{P} \cup \freenames{Q} \\
  \freenames{\dropn{x}} & := & \{ x \}
\end{eqnarray*}

The bound names of a process, $\boundnames{P}$, are those names occurring in $P$
that are not free. For example, in $x?(y).0$, the name $x$ is free, while $y$ is bound.

\begin{mathpar}
  \inferrule* [lab=monoidal-laws] {} { P|Q \equiv Q|P \and P|0 \equiv P \and P|(Q|R) \equiv (P|Q)|R }
\end{mathpar}

\begin{mathpar}
  \inferrule* [lab=alpha-equivalence] {} { (x)P \equiv (y)P\{y/x\} \and y \not\in \freenames{P} }
\end{mathpar}

\begin{definition}
Then two processes, $P,Q$, are alpha-equivalent if $P = Q\{\vec{y}/\vec{x}\}$ for
some $\vec{x} \in \boundnames{Q},\vec{y} \in \boundnames{P}$, where $Q\{\vec{y}/\vec{x}\}$
denotes the capture-avoiding substitution of $\vec{y}$ for $\vec{x}$ in $Q$.
\end{definition}

\begin{definition}
  The {\em structural congruence} \cite{SangiorgiWalker} , $\equiv$,
  between processes is the least congruence containing
  alpha-equivalence, satisfying the abelian monoid laws
  (associativity, commutativity and $\pzero$ as identity) for parallel
  composition $|$ and for summation $+$.
\end{definition}

\subsection{Name equivalence}

We take name equivalence, written $\nameeq$, to be the smallest
equivalence relation generated by the following rules.

\begin{mathpar}
\inferrule*[lab=Quote-drop]
{ }
{ \quotep{@{x}} \nameeq x }

\inferrule*[lab=Struct-equiv]
{ P \scong Q }
{ \quotep{P} \nameeq \quotep{Q} }
\end{mathpar}

The astute reader will have noticed that the mutual recursion of names
and processes imposes a mutual recursion on alpha-equivalence and
structural equivalence via name-equivalence. Fortunately, all of this
works out pleasantly and we may calculate in the natural way, free of
concern. The reader interested in the details is referred to the
appendix \ref{appendix:rho_details}.

\subsection{Substitution}

We use $\Proc$ for the set of processes, $\QProc$ for the set of
names, and $\id{\{}\vec{y} / \vec{x} \id{\}}$ to denote partial maps,
$s : \QProc \rightarrow \QProc$. A map, $s$ lifts, uniquely, to a map
on process terms, $\widehat{s} : \Proc \rightarrow \Proc$ by the
following equations.

\begin{mathpar}
  (0) \psubstp{Q}{P} := 0 \\
  (R \juxtap S) \psubstp{Q}{P}
  :=    
  (R)\psubstp{Q}{P} \juxtap (S) \psubstp{Q}{P} \\
  (x?(y).R) \psubstp{Q}{P}    
  :=    
  (x)\substp{Q}{P} (z)\concat( (R \psubstn{z}{y}) \psubstp{Q}{P} ) \\
  (\lift{x}{R}) \psubstp{Q}{P}  
  :=
  \lift{(x)\substp{Q}{P}}{ R \psubstp{Q}{P} } \\
%   (\dropn{x})  \psubstp{Q}{P}       
%   := 
%   \left\{ 
%     \begin{array}{ccc} 
%       \dropn{\quotep{Q}} & & x \nameeq \quotep{P} \\
%       \dropn{x} & & otherwise \\
%     \end{array}
%   \right. 
  (\dropn{x})  \psubstp{Q}{P}       
  := 
  \left\{ 
    \begin{array}{ccc} 
      Q & & x \nameeq \quotep{P} \\
      \dropn{x} & & otherwise \\
    \end{array}
  \right.
\end{mathpar}
 

where

\begin{eqnarray}
  (x)\id{\{} \lpquote Q \rpquote / \lpquote P \rpquote \id{\}}            = 
  \left\{ 
    \begin{array}{ccc}
      \lpquote Q \rpquote & & x \nameeq \lpquote P \rpquote \\
      x & & otherwise \\
    \end{array}
  \right. \nonumber
\end{eqnarray}

and $z$ is chosen distinct from $\quotep{P}$, $\quotep{Q}$, the free
names in $Q$, and all the names in $R$. Our $\alpha$-equivalence will
be built in the standard way from this substitution.

\begin{remark}\label{rem:no_self_referential_names}
  One consequence of these definitions is that $\forall P. \quotep{P}
  \not\in \freenames{P}$.
\end{remark}

\subsection{ Dynamic quote: an example }

Anticipating something of what's to come, consider applying the
substitution, $\widehat{\id{\{}u / z \id{\}}}$, to the following pair
of processes, $\lift{w}{y!(z)}$ and $w[ \lpquote y!(z) \rpquote ]$.

\begin{eqnarray}
	\lift{w}{y!(z)}\widehat{\id{\{}u / z \id{\}}}
		& = &
		\lift{w}{y!(u)} \nonumber\\
	w[ \lpquote y!(z) \rpquote ] \widehat{ \id{\{}u / z \id{\}} }
		& = &
		w[ \lpquote y!(z) \rpquote ] \nonumber
\end{eqnarray}

Because the body of the process between quotes is impervious to
substitution, we get radically different answers. In fact, by
examining the first process in an input context,
e.g. $x?(z).\lift{w}{y!(z)}$, we see that the process under the lift
operator may be shaped by prefixed inputs binding a name inside it. In
this sense, the lift operator will be seen as a way to dynamically
construct processes before reifying them as names.

Finally equipped with these standard features we can present the
dynamics of the calculus.

\subsubsection{Operational semantics} 

Finally, we introduce the computational dynamics. What marks these
algebras as distinct from other more traditionally studied algebraic
structures, e.g. vector spaces or polynomial rings, is the manner in
which dynamics is captured. In traditional structures, dynamics is typically
expressed through morphisms between such structures, as in linear maps
between vector spaces or morphisms between rings. In algebras
associated with the semantics of computation, the dynamics is
expressed as part of the algebraic structure itself, through a
reduction reduction relation typically denoted by $\red$. Below, we
give a recursive presentation of this relation for the calculus used
in the encoding.

$\red \subseteq \pi \times \pi$
$\red : \pi \to \mathcal{P}(\pi)$

\begin{mathpar}
  \inferrule* [lab=Comm] { \textsf{match}( x_{src}, x_{trgt} ) } { x_{trgt}?(y)P \; | \; x_{src}!\langle {Q} \rangle \red P\{\quotep{Q}/y}\} }
  \and \\
  \inferrule* [lab=Par] {{P} \red {P}'} {{{P} | {Q}} \red {{P}' | {Q}}}
  \and
  \inferrule* [lab=Equiv]{{{P} \scong {P}'} \andalso {{P}' \red {Q}'} \andalso {{Q}' \scong {Q}}}{{P} \red {Q}}
\end{mathpar}

\begin{eqnarray*}
  match_{\equiv} (\quotep{P},\quotep{Q}) & := & P \equiv Q \\
  match_{\dagger}(\quotep{P},\quotep{Q}) & := & \forall R. P|Q \red^{*} R => R \red^{*} 0 \\
  match_{K}(\quotep{P},\quotep{Q}) & := & K \mbox{ for some context } K
\end{eqnarray*}

$u?(x)P | u!\langle Q \rangle \red P\{\quotep{Q}/x\}$

%We write $\wred$ for $\red^*$, and $P\red$ if $\exists Q $ such that $ P \red Q$.
We write $P\red$ if $\exists Q $ such that $ P \red Q$ and $P\not\red$, otherwise.

\section{Replication}

As mentioned before, it is known that replication (and hence
recursion) can be implemented in a higher-order process algebra
\cite{SangiorgiWalker}. As our first example of calculation with the
machinery thus far presented we give the construction explicitly in
the {\rhoc}.

\begin{eqnarray}
	D_{x} & := & \prefix{x}{y}{(\binpar{\outputp{x}{y}}{@{y}})} \nonumber\\
	\bangp_{x}{P} & := & \binpar{{x}!\langle{\binpar{D_{x}}{P}}\rangle}{D_{x}} \nonumber
\end{eqnarray}

\begin{eqnarray}
	\bangp_{x}{P} & & \nonumber\\
	=
	& {x}!\langle{(\prefix{x}{y}{(\outputp{x}{y} | @{y})) | P}}\rangle 
	      | \prefix{x}{y}{(\outputp{x}{y} | @{y})} & \nonumber\\
	\red
	& (\outputp{x}{y} | @{y})\substn{\quotep{(\prefix{x}{y}{(@{y} | \outputp{x}{y})) | P}}}{y} & \nonumber\\
	=
	& \outputp{x}{\quotep{(\prefix{x}{y}{(\outputp{x}{y} | @{y})) | P}}}
	  | {(\prefix{x}{y}{(\outputp{x}{y} | @{y})) | P}} & \nonumber\\
	\red
	& \ldots & \nonumber\\
	\red^*
	& P | P | \ldots & \nonumber
\end{eqnarray}

Of course, this encoding, as an implementation, runs away, unfolding
$\bangp{P}$ eagerly. A lazier and more implementable replication
operator, restricted to input-guarded processes, may be obtained as follows.

\begin{eqnarray}
\bangp{\prefix{u}{v}{P}} 
	:= 
	\binpar{\lift{x}{\prefix{u}{v}{(\binpar{D(x)}{P})}}}{D(x)} \nonumber
\end{eqnarray}

\begin{remark}
  Note that the lazier definition still does not deal with summation
  or mixed summation (i.e. sums over input and output). The reader is
  invited to construct definitions of replication that deal with these
  features. 

  Further, the definitions are parameterized in a name, $x$. Can you,
  gentle reader, make a definition that eliminates this parameter and
  guarantees no accidental interaction between the replication
  machinery and the process being replicated -- i.e. no accidental
  sharing of names used by the process to get its work done and the
  name(s) used by the replication to effect copying. This latter
  revision of the definition of replication is crucial to obtaining
  the expected identity $!!P \sim !P$.
\end{remark}

\begin{remark}\label{rem:paradoxical_combinator}
  The reader familiar with the lambda calculus will have noticed the
  similarity between $D$ and the paradoxical combinator.

  [Ed. note: the existence of this seems to suggest we have to be more
  restrictive on the set of processes and names we admit if we are to
  support no-cloning.]
\end{remark}

\subsubsection{Bisimulation}

The computational dynamics gives rise to another kind of equivalence,
the equivalence of computational behavior. As previously mentioned
this is typically captured \emph{via} some form of bisimulation.

% The notion we use in this paper is weak barbed bisimulation
% \cite{milner91polyadicpi}.

The notion we use in this paper is derived from weak barbed
bisimulation \cite{milner91polyadicpi}. 

\begin{definition}
An \emph{observation relation}, $\downarrow_{\mathcal N}$, over a set
of names, $\mathcal N$, is the smallest relation satisfying the rules
below.

\infrule[Out-barb]{y \in {\mathcal N}, \; x \nameeq y}
		  {\outputp{x}{v} \downarrow_{\mathcal N} x}
\infrule[Par-barb]{\mbox{$P\downarrow_{\mathcal N} x$ or $Q\downarrow_{\mathcal N} x$}}
		  {\binpar{P}{Q} \downarrow_{\mathcal N} x}

We write $P \Downarrow_{\mathcal N} x$ if there is $Q$ such that 
$P \wred Q$ and $Q \downarrow_{\mathcal N} x$.
\end{definition}

\begin{definition}
%\label{def.bbisim}
An  ${\mathcal N}$-\emph{barbed bisimulation} over a set of names, ${\mathcal N}$, is a symmetric binary relation 
${\mathcal S}_{\mathcal N}$ between agents such that $P\rel{S}_{\mathcal N}Q$ implies:
\begin{enumerate}
\item If $P \red P'$ then $Q \wred Q'$ and $P'\rel{S}_{\mathcal N} Q'$.
\item If $P\downarrow_{\mathcal N} x$, then $Q\Downarrow_{\mathcal N} x$.
\end{enumerate}
$P$ is ${\mathcal N}$-barbed bisimilar to $Q$, written
$P \wbbisim_{\mathcal N} Q$, if $P \rel{S}_{\mathcal N} Q$ for some ${\mathcal N}$-barbed bisimulation ${\mathcal S}_{\mathcal N}$.
\end{definition}

$\mathcal{R} \subseteq \pi \times \pi$

$P \mathcal{R} Q => \forall P'. P \red P' \Rightarrow \exists Q'. Q \red Q', P' \mathcal{R} Q'$

$P \vdash x \Rightarrow Q \vdash x$

\begin{mathpar}
  \inferrule*[lab=Out-barb]{x \nameeq y}{{y}!\langle{Q}\rangle \vdash x}
  \and
  \inferrule*[lab=Par-barb]{\mbox{$P\vdash x$ or $Q\vdash x$}}{\binpar{P}{Q} \vdash x}
\end{mathpar}

\subsubsection{Contexts}

One of the principle advantages of computational calculi like the
$\pi$-calculus is a well-defined notion of context,
contextual-equivalence and a correlation between
contextual-equivalence and notions of bisimulation. The notion of
context allows the decomposition of a process into (sub-)process and
its syntactic environment, its context. Thus, a context may be
thought of as a process with a ``hole'' (written $\Box$) in it. The
application of a context $M$ to a process $P$, written $M[P]$, is
tantamount to filling the hole in $M$ with $P$. In this paper we do
not need the full weight of this theory, but do make use of the notion
of context in the proof the main theorem. 

\begin{mathpar}
  \inferrule* [lab=summation] {} {{M_{M},M_{N}} \bc \Box \;|\; x.M_{A} \;|\; M_{M}+M_{N}}
  \and
  \inferrule* [lab=agent] {} {{M_{A}} \bc (\vec{x})M_{P} \;| \; \clift{P_0,\ldots,M_{P},\ldots,P_N}}
  \and \\
  \inferrule* [lab=process] {} {{M_{P}} \bc M_{N} \;| \;P|M_{P} }
\end{mathpar} 

\begin{mathpar}
  \inferrule* [lab=sychronization] {} {M_{N} \bc \Box \;|\; x?M_{F} \;|\; x!M_{C}}
  \and
  \inferrule* [lab=abstraction] {} {{M_{F}} \bc (x)M_{P} }
  \and
  \inferrule* [lab=concretion] {} {{M_{C}} \bc \langle M_{P} \rangle }
  \and \\
  \inferrule* [lab=process] {} {{M_{P}} \bc M_{N} \;| \;P|M_{P} }
\end{mathpar}

\begin{definition}[contextual application] Given a context $M$, and
  process $P$, we define the \emph{contextual application}, $M[P] :=
  M\{P/\Box\}$. That is, the contextual application of M to P is the
  substitution of $P$ for $\Box$ in $M$.
\end{definition}

$\meaningof{-} : L \to \mathcal{P}(\pi)$

\begin{mathpar}
  \inferrule* [lab=collection] {} {\meaningof{true} = \pi, \and \meaningof{~E} = \pi \setminus \meaningof{E}, \and \meaningof{E_{1} \& E_{2}} = \meaningof{E_{1}} \cap \meaningof{E_{2}}}
\end{mathpar}

\begin{mathpar}
  \inferrule* [lab=structure] {} {\meaningof{0} = \{ P \in \pi | P \equiv 0 \}, \and \\ \meaningof{E_1 | E_2} = \{ P \in \pi | P \equiv P_{1} | P_{2}, P_{1} \in \meaningof{E_{1}}, P_{2} \in \meaningof{E_2}\} }
\end{mathpar}

\begin{mathpar}
 \inferrule* [lab=behavior] {} {\meaningof{\langle a?b \rangle E} = \{ P \in \pi | P \equiv Q | u?(y)P', \\ \and \\\\ \and \\ \;\;\; u \in \meaningof{a}, \forall z.P'\{z/y\} \in \meaningof{E\{z/b\}}\}, \and \\ \meaningof{a!E} = \{ P \in \pi | P \equiv Q | x!\langle P' \rangle, x \in \meaningof{a} P' \in \meaningof{E}\} }
\end{mathpar}

\begin{mathpar}
 \inferrule* [lab=nominal] {} {\meaningof{\quotep{E}} = \{ \quotep{P} \in \quotep{\pi} | P \in \meaningof{E} \}, \and \meaningof{\quotep{P}} = \{ \quotep{Q} \in \quotep{\pi} | P \equiv Q \} \and \\ \meaningof{@\quotep{E}} = \{ P \in \pi | P \equiv @x, x \in \meaningof{E} \}}
\end{mathpar}

\begin{eqnarray*}
  \\
  \meaningof{-} : TS \to ST
\end{eqnarray*}

\begin{eqnarray*}
  \\
  L : TS \to ST
\end{eqnarray*}

\begin{eqnarray*}
  \\
  P \models E \iff P \in \meaningof{E}
\end{eqnarray*}

\begin{eqnarray*}
  P \approx_{L} Q \iff \forall E \in L. P \models E \iff Q \models E
\end{eqnarray*}

\begin{eqnarray*}
  P \approx_{K} Q
\end{eqnarray*}

\begin{eqnarray*}
  P \approx Q
\end{eqnarray*}

$\approx_{K} = \approx = \approx_{L}$

\subsubsection{Contextual duality}

Note that contexts extend the quotation operation to a family of
operations from processes to names. Given a context, $M$, we can
define a \emph{nominal context}, $\quotep{M}$ by $\quotep{M}[P] :=
\quotep{M[P]}$. To foreshadow what is to come we observe that these
operations enjoy a duality with processes very much like the duality
between vectors and maps from vectors to scalars.

Further, because the calculus is essentially higher-order, we have a
correspondence between contexts and processes. More specifically,
given a name $x$ and a context $M$ we can construct $M^{*}_{x}$ such
that 

\begin{mathpar}
  M^{*}_{x} | \lift{x}{P} \red M[P]
\end{mathpar}

namely,

\begin{mathpar}
  M^{*}_{x} := x?(u).M[\dropn{u}]
\end{mathpar}

The dependence of $M^{*}_{x}$ on a name makes it an abstraction, 

\begin{mathpar}
  M^{*} := (x)x?(u).M[\dropn{u}]
\end{mathpar}

\subsection{Additional notation}

It will sometimes be convenient to denote the process a name
quotes. We already have the notation $x = \quotep{P}$, but it will be
convenient to introduce an alternate notation, $\procn{x}$, when we
want to emphasize the connection to the use of the name. Note that, by
virtue of name equivalence, $\quotep{\procn{x}} \nameeq x$; so, the
notation is consistent with previous definitions.

Further, because names have structure it is possible to effect
substitutions on the basis of that structure. This means we need to
upgrade our notation for substitutions, which we accomplish by
adapting comprehension notation. Thus,

\begin{mathpar}
  P\{ y / x : x \in S \}
\end{mathpar}

is interpreted to mean the process derived from P by replacing (in a
capture-avoiding manner) each occurrence of $x$ in $S$ by $y$. For example,

\begin{mathpar}
  P\{ \quotep{\procn{x}|\procn{x}} / x : x \in \freenames{P} \}
\end{mathpar}

will replace each (occurrence) of a free name $x$ in $P$ by
$\quotep{\procn{x}|\procn{x}}$.

Also, we will avail ourselves of the notation $x^{L}$ and $x^{R}$ to
denote injections of a name into disjoint copies of the name
space. There are numerous ways to accomplish this. One example can be
found in \cite{MeredithR05}. This notation overloads to vectors of
names: $\vec{x}^{\pi} := (x_{i}^{\pi} \; : \; 0 \leq i < |\vec{x}| )$ where $\pi \in \{L,R\}$.

We also use $P^{\Box} := P|\Box$.

In \cite{MeredithR05} an interpretation of the new operator is
given. It turns out that there are several possible interpretations
all enjoying the requisite algebraic properties of the operator (see
\cite{milner91polyadicpi}). We will therefore make liberal use of
$(\nu\; \vec{x})P$.

% subsection the_syntax_and_semantics_of_the_notation_system (end)   

\section{Interpretation of QM}
\subsection{Supporting definitions}
\subsubsection{Multiplication}
\begin{mathpar}
  \quotep{Q} \cdot \quotep{R} := \quotep{Q|R}
  \and \\
  \quotep{Q} \cdot P := P\{ \quotep{Q|R} / \quotep{R} : \quotep{R} \in \freenames{P} \}
\end{mathpar}

\paragraph{Discussion}
The first line needs little explanation. The second line says that
each free name of the process is replaced with the multiplication of
that name by the scalar. Multiplication of a scalar (name) by a state
(process) results in a process all the names of which have been `moved
over' by parallel composition with the process the scalar
quotes. There is a subtlety that the bound names have to be
manipulated so that multiplied names aren't accidentally
captured. There are many ways to achieve this.

\begin{remark}\label{rem:multiplication_identities}
  The reader is invited to verify that for all $x,y,z \in \QProc$ and $P \in \Proc$
  \begin{mathpar}
    x \cdot \quotep{0} \equiv x 
    \and
    x \cdot y \equiv y \cdot x
    \and
    x \cdot (y \cdot z) \equiv (x \cdot y) \cdot z
    \and \\
    \quotep{0} \cdot P \equiv P
    \and \\
    x \cdot (y \cdot P) \equiv (x \cdot y) \cdot P
    \and \\
    x \cdot (P|Q) \equiv (x \cdot P) | (x \cdot Q)
    \and \\    
  \end{mathpar}
\end{remark}

\subsubsection{Tensor product}

We define a tensor product on processes by structural induction.

\paragraph{Tensor of sums} First note that all summations, including
$\pzero$ and sequence, can be written $\Sigma_{i} x_{i}.A_{i} +
\Sigma_{j} x_{j}.C_{j}$, where we have grouped input-guarded processes
together and output-guarded processes together.

Thus, we can define the tensor product of two summations, $N_{1}\otimes N_{2}$, where

\begin{mathpar}
  N_{1} := \Sigma_{i} x_{i}.A_{i} + \Sigma_{j} x_{j}.C_{j}
  \and
  N_{2} := \Sigma_{i'} y_{i'}.B_{i'} + \Sigma_{j'} y_{j'}.D_{j'} 
\end{mathpar}

as follows.

\begin{mathpar}
  \Sigma_{i} x_{i}.A_{i} + \Sigma_{j} x_{j}.C_{j} \otimes \Sigma_{i'}
  y_{i'}.B_{i'} + \Sigma_{j'} y_{j'}.D_{j'} 
  \and \\
  := \; \Sigma_{i} \Sigma_{i'} \quotep{\stackrel{\vee}{x_{i}}| \stackrel{\vee}{y_{i'}}}.(A_{i}\otimes B_{i'}) \; | \; \Sigma_{i'} \Sigma_{i} \quotep{\stackrel{\vee}{y_{i'}}|\stackrel{\vee}{x_{i}}}.(B_{i'}\otimes A_{i})
  \and
  \;\; | \;\; \Sigma_{j} \Sigma_{j'} \quotep{\stackrel{\vee}{x_{j}}|\stackrel{\vee}{y_{j'}}}.(A_{j}\otimes B_{j'}) \; | \; \Sigma_{j'} \Sigma_{j} \quotep{\stackrel{\vee}{y_{j'}}|\stackrel{\vee}{x_{j}}}.(B_{j'}\otimes A_{j})
\end{mathpar}

\begin{remark}
  Do we need to $x^{L}$ and $y^{R}$ for this construction as well?
\end{remark}

\paragraph{Tensor of parallel compositions} Next, we distribute tensor
over par.

\begin{mathpar}
  P_{1}|P_{2} \otimes Q_{1}|Q_{2} := (P_{1} \otimes Q_{1}) | (P_{1}
  \otimes Q_{2}) | (P_{2} \otimes Q_{1}) | (P_{2} \otimes Q_{2})
\end{mathpar}

\paragraph{Tensor with dropped names} We treat tensor of a
process with a dropped name as parallel composition.

\begin{mathpar}
  P \otimes \dropn{x} := P | \dropn{x}
\end{mathpar}

\paragraph{Tensor of agents}

Finally, we need to define tensor on agents. Note that the definition
of tensor on normal products only tensors inputs with inputs and
outputs with outputs. Thus, we only have to define the operation on
``homogeneous'' pairings.

\begin{mathpar}
  (\vec{x})P \otimes (\vec{y})Q
  \and \\
  := (x_{0}^{L}|y_{0}^{R},\ldots,x_{0}^{L}|y_{n}^{R},\ldots,x_{m}^{L}|y_{0}^{R},\ldots,x_{m}^{L}|y_{n}^R)(P\{ \vec{x}^{L}/\vec{x}\} \otimes Q \{ \vec{y}^{R}/\vec{y}\})
  \and \\
  \clift{\vec{P}} \otimes \clift{\vec{Q}}
  \and \\
  := \clift{P_{0}\otimes Q_{0},\ldots,P_{0}\otimes Q_{n},\ldots,P_{m}\otimes Q_{0},\ldots,P_{m}\otimes Q_{n}}
\end{mathpar}

\begin{remark}
  Observe that arities of tensored abstractions matches arities of
  tensored concretions if the original arities matched. Note also that
  the length of the arities corresponds to the increase in dimension
  we see in ordinary vector space tensor product.
\end{remark}

\begin{remark}
  Operationally, this definition distributes the tensor down to
  components ``linked'' by summation. Tensor over summation is
  intriguing in that it mixes names. Moreover, as a consequence of the
  way it mixes names we have the identities for all $x \in \QProc$ and
  $P,Q \in \Proc$

  \begin{mathpar}
    (x \cdot P) \otimes Q \equiv x \cdot (P \otimes Q) \equiv P \otimes (x \cdot Q)
    \and
    P \otimes \pzero \equiv P
  \end{mathpar}

  that the reader is invited to verify.
\end{remark}

\subsubsection{Annihilation}
\begin{mathpar}
  P^{\perp} := \{ Q | \forall R. P|Q \red^{*} R \Rightarrow R \red^{*} \pzero \}
  \and \\
  P^{\underline{\perp}} := \Sigma_{Q \in P^{\perp}} \quotep{Q}?(y).(\dropn{y}|Q) | \Sigma_{Q \in P^{\perp}} \quotep{Q}\clift{\Box}
\end{mathpar}

\paragraph{Discussion} The reader will note that $P^{\perp}$ is a
\emph{set} of processes, while $P^{\underline{\perp}}$ is a
\emph{context}. We call the set $P^{\perp}$ the \emph{annihilators} of
$P$. The parallel composition of a process in the annihilators of $P$
with $P$ will result in a process, the state space of which has all
paths eventually leading to $\pzero$. Execution may endure loops; but
under reasonable conditions of fairness (naturally guaranteed under
most notions of bisimulation) such a composite process cannot get
stuck in such a loop and will, eventually pop out and terminate.

The context $P^{\underline{\perp}}$ is ready and willing to ``take the
$P$ out of'' the process to which it is applied. It will effectively
transmit the code of the process to which it is applied to one of the
annihilators and run the process against it.

\subsubsection{Evaluation}
We fix $M$ a domain of fully abstract interpretation with an equality
coincident with bisimulation. We take $\meaningof{\cdot} : \Proc \to
M$ to be the map interpreting processes and $\nmeaningof{\cdot} : \M
\to Proc$ to be the map running the other way. Then we define

\begin{mathpar}
  \int P := \nmeaningof{\meaningof{P}}
\end{mathpar}

\paragraph{Discussion}
There are many fully abstract interpretations of Milner's
$\pi$-calculus. Any of them can be used as a basis for interpreting
the reflective calculus here. Equipped with such a domain it is
largely a matter of grinding through to check that the Yoneda
construction for the normalization-by-evaluation program can be
extended to this setting.

\begin{remark}
  The reader is invited to verify that $\int (P^{\underline{\perp}}[P]) = 0$.
\end{remark}

\subsection{Quantum mechanics}

Table \ref{tbl:core_qm_op_defns} gives the core operational definitions

\begin{table}[htp]\label{tbl:core_qm_op_defns}
  \center{
    \fbox{
      \begin{tabular}{c|c}
        quantum mechanics & process calculus \\
        \hline
        scalar & $x := \quotep{P}$ \\
        state vector & $\state{P} := P$ \\
        dual & $\state{P}^{*} := \event{P^{\underline{\perp}}} := \quotep{P^{\underline{\perp}}}[-]$ \\
        matrix & $ \Sigma_{\alpha} \state{P_{\alpha}}x_{\alpha}\event{Q_{\alpha}}$ \\
        vector addition & $\state{P} + \state{Q} := \state{P | Q}$ \\
        tensor product & $\state{P} \otimes \state{Q} := \state{P \otimes Q}$ \\
        inner product & $\innerprod{P}{Q} := \quotep{\int P^{\underline{\perp}}[Q]}$ \\
      \end{tabular}
    }
  }
  \caption{QM - operational definitions}
\end{table}

where

\begin{mathpar}
  \prmatrix{P}{Q} := \fprmatrix{P}{\quotep{\pzero}}{Q}
  \and
  \fprmatrix{P}{x}{Q} := (\state{P},x,\event{Q})
  \and
  (\fprmatrix{P}{x}{Q})(\state{R}) := x \cdot \innerprod{Q}{R} \cdot \state{P}
  \and
  (\fprmatrix{P}{x}{Q})(\event{R}) := x \cdot \innerprod{R}{P} \cdot \event{Q}
\end{mathpar}

\paragraph{Discussion}
As promised: vectors (aka states) are represented as processes; duals
as contextual duals; inner product definition should be compared with
standard inner product definition for ....

\begin{remark}
  Assuming $\int (P^{\underline{\perp}}[P]) = 0$, the reader is
  invited to verify that $(\fprmatrix{P}{x}{P})(\state{P}) = x \cdot \state{P}$.
\end{remark}

\begin{remark}
  The reader is invited to verify that $\innerprod{P}{Q}$ could
  equally well have been written $\quotep{\int \stackrel{\vee}{x}}$
  where $x = \event{P^{\underline{\perp}}}(Q)$.

  One of the motivations for this remark is that there is another way
  to factor these operations. We could package up evaluation in the dual:

  \begin{mathpar}
    \state{P}^{*} := \event{\int P^{\underline{\perp}}} := \quotep{\int P^{\underline{\perp}}}[-]
  \end{mathpar}

  and then have inner product defined by
  
  \begin{mathpar}
    \innerprod{P}{Q} := \event{P}(Q)
  \end{mathpar}

  Hopefully, experience with the calculations will provide guidance on
  the best factoring.
\end{remark}

\begin{remark}
  Assuming $\int (P^{\underline{\perp}}[P]) = 0$, the reader is
  invited to verify that $\forall P,Q. (\prmatrix{0}{Q})(\state{0}) =
  \state{0}$ and dually $(\prmatrix{P}{0})(\event{0}) = \event{0}$.
\end{remark}

\begin{remark}
  i'm a little worried that i don't (yet) have proper support for
  complex conjugacy. But, the observation above may give us a
  clue. According to Abramsky, it must be the case that the scalars
  are iso to the homset of the identity for the tensor -- which the
  observation above characterizes. 

  For now, we will simply bookmark the notion with $\overline{x}$.
\end{remark}

\subsubsection{Adjointness}

We need to give a definition of $(\cdot)^{\dagger}$ for matrices. The
obvious candidate definition is
\begin{mathpar}
(\Sigma_{\alpha}\fprmatrix{P_{\alpha}}{x_{\alpha}}{Q_{\alpha}})^{\dagger}
= \Sigma_{\alpha}\fprmatrix{(Q_{\alpha}^{\underline{\perp}})^{*}}{\overline{x}_{\alpha}}{P_{\alpha}^{\underline{\perp}}} 
\end{mathpar}

But, $(Q_{\alpha}^{\underline{\perp}})^{*}$ requires a name along
which to communicate the process to achieve the context application.

\subsubsection{Basis for a basis}
If processes label states and ``addition'' of states (a.k.a. vector
addition) is interpreted as parallel composition, what corresponds to
notions of linear independence and basis? Here, we recall that Yoshida
has developed a set of \emph{combinators} for an asynchronous verison
of Milner's $\pi$-calculus. These are a finite set of processes such
any process can be expressed as parallel composition of these
combinators together with liberal uses of the new operator and
replication. We can simply give a translation of these into the
present calculus and have reasonable expectation that the property
carries over. That is, that the resultant set allows to express all
processes via parallel composition. Note, however, that there is no
new operator or replication in this calculus. As a result, we expect
that the corresponding set is actually infinite. That is, we expect
that the space is actually infinite dimensional.

\begin{remark}
  The attentive reader may be a bit concerned. Certainly, the
  collection $S$, $K$ and $I$ is a finite set of
  combinators. Shouldn't we expect to see a finite set of combinators
  for an effectively equivalent system? i am very sympathetic to this
  critique and feel it warrants full attention. On the other hand, i
  also have in mind the following analogy. The natural numbers, as a
  monoid under addition, has exactly $1$ generator, while the natural
  numbers, as a monoid under multiplication, has countably many
  generators (the primes). We observe that the application of the
  lambda calculus is much less resource sensitive than the parallel
  composition of the $\pi$-calculus. Could it be the case that we have
  an analogy of the form
  
  \begin{mathpar}
    m + n : MN :: m*n : M|N
  \end{mathpar}

  giving a similar blow up in the set of ``primes''?  This is such a
  wonderful thought that, even if it's not true, i think it's worth
  writing down.
\end{remark}
 

\documentclass[12pt]{llncs}
%\documentclass{jktr}

\usepackage[pdftex]{hyperref}                   
\usepackage {listings}
\usepackage {mathpartir}
\usepackage{bcprules}
%\usepackage{listings}
                       
\usepackage{graphicx} 
%\usepackage[margins=2.5cm,nohead,nofoot]{geometry}
%\usepackage{geometry}
\usepackage{amsfonts}
\usepackage{amstext}
\usepackage{latexsym}
\usepackage{amssymb}
\usepackage{color}


%\include{myPreamble}
\include{qm2pi.local} 

%\ifpdf
%\usepackage[pdftex]{graphicx}
%\else
%\usepackage{graphicx}
%\fi

 % \ifpdf
%  \usepackage{pdfsync}
%  \if


%\title{Brief Article}
%\author{David F. Snyder}
%\author{L.G. Meredith}

%\address{Dept. of Math., Texas State University--San Marcos, San Marcos, TX 78666}
       
\pagestyle{empty}


\begin{document}

\lstset{language=[Objective]Caml,frame=shadowbox}

\input{qm2pi.front}

% section front matter (end)

\input{qm2pi.intro} 
 
% section introduction (end)

% \input{qm2pi.knotations} 

% section notation (end)

\input{qm2pi.process.calculi} 

% section concurrent_process_calculi_and_spatial_logics_ (end)
    
%\input{qm2pi.knots2pi} 

%\input{qm2pi.trefoil} 

%\input{qm2pi.mainthm} 

% subsection basic_interpretation (end)

%\input{qm2pi.rho.presentation} 
\subsection{The syntax and semantics of the notation system}\label{sub:the_syntax_and_semantics_of_the_notation_system} % (fold)

We now summarize a technical presentation of the calculus that
embodies our theory of dynamics. The typical presentation of such a
calculus follows the style of giving generators and relations on
them. The grammar, below, describing term constructors, freely
generates the set of processes, $\Proc$. This set is then quotiented
by a relation known as structural congruence and it is over this set
that the notion of dynamics is expressed. This presentation is
essentially that of \cite{MeredithR05} with the addition of
polyadicity and summation. For readability we have relegated some of
the technical subtleties to an appendix.

\subsubsection{Process grammar}\label{subsub:process_grammar}

\begin{mathpar}
  \inferrule* [lab=synchronization] {} {{M} \bc \pzero \;|\; x?F \;|\; x!C }
  \and
  \inferrule* [lab=abstraction] {} {{F} \bc (x)P}
  \and
  \inferrule* [lab=concretion] {} {{C} \bc \langle Q \rangle}
  \and
  \inferrule* [lab=process] {} {{P,Q} \bc M \;| \;P|Q \;|\; @{x}}
  \and
  \inferrule* [lab=name] {} {{x} \bc \quotep{P}}
\end{mathpar} 

Note that $\vec{x}$ (resp. $\vec{P}$) denotes a vector of names
(resp. processes) of length $|\vec{x}|$ (resp. $|\vec{P}|$). We adopt
the following useful abbreviations.

\begin{mathpar}
   x?(\vec{y}).P := x.(\vec{y})P \and  x\clift{\vec{P}} := x.\clift{\vec{P}}
   \and x!(y) := \lift{x}{\dropn{y}}
   \and \Pi_{i=0}^{n-1}P_i := P_0 | \ldots | P_{n-1}
\end{mathpar}

\subsubsection{Structural congruence}

\paragraph{Free and bound names and alpha-equivalence.} At the
core of structural equivalence is alpha-equivalence which identifies
process that are the same up to a change of variable. Formally, we
recognize the distinction between free and bound names. The free names
of a process, $\freenames{P}$, may be calculated recursively as
follows:

\begin{mathpar}
\freenames{\pzero} := \emptyset
  \and \\
  \freenames{x?(y).P} := \{ x \} \cup (\freenames{P} \setminus \{ y \})
  \and 
  \freenames{x!\langle P \rangle} := \{ x \} \cup \{ P \} 
  \and \\
  \freenames{P|Q} := \freenames{P} \cup \freenames{Q}
  \and \\
  \freenames{@{x}} := \{ x \}
\end{mathpar}

$\pi$
$\quotep{\pi}$

$\freenames{-} : \pi \to \mathcal{P}(\quotep{\pi})$

\begin{eqnarray*}
  \freenames{\pzero} & := & \emptyset \\
  \freenames{x?(y).P} & := & \{ x \} \cup (\freenames{P} \setminus \{ y \}) \\
  \freenames{x!\langle P \rangle} & := & \{ x \} \cup \{ P \} \\
  \freenames{P|Q} & := & \freenames{P} \cup \freenames{Q} \\
  \freenames{\dropn{x}} & := & \{ x \}
\end{eqnarray*}

The bound names of a process, $\boundnames{P}$, are those names occurring in $P$
that are not free. For example, in $x?(y).0$, the name $x$ is free, while $y$ is bound.

\begin{mathpar}
  \inferrule* [lab=monoidal-laws] {} { P|Q \equiv Q|P \and P|0 \equiv P \and P|(Q|R) \equiv (P|Q)|R }
\end{mathpar}

\begin{mathpar}
  \inferrule* [lab=alpha-equivalence] {} { (x)P \equiv (y)P\{y/x\} \and y \not\in \freenames{P} }
\end{mathpar}

\begin{definition}
Then two processes, $P,Q$, are alpha-equivalent if $P = Q\{\vec{y}/\vec{x}\}$ for
some $\vec{x} \in \boundnames{Q},\vec{y} \in \boundnames{P}$, where $Q\{\vec{y}/\vec{x}\}$
denotes the capture-avoiding substitution of $\vec{y}$ for $\vec{x}$ in $Q$.
\end{definition}

\begin{definition}
  The {\em structural congruence} \cite{SangiorgiWalker} , $\equiv$,
  between processes is the least congruence containing
  alpha-equivalence, satisfying the abelian monoid laws
  (associativity, commutativity and $\pzero$ as identity) for parallel
  composition $|$ and for summation $+$.
\end{definition}

\subsection{Name equivalence}

We take name equivalence, written $\nameeq$, to be the smallest
equivalence relation generated by the following rules.

\begin{mathpar}
\inferrule*[lab=Quote-drop]
{ }
{ \quotep{@{x}} \nameeq x }

\inferrule*[lab=Struct-equiv]
{ P \scong Q }
{ \quotep{P} \nameeq \quotep{Q} }
\end{mathpar}

The astute reader will have noticed that the mutual recursion of names
and processes imposes a mutual recursion on alpha-equivalence and
structural equivalence via name-equivalence. Fortunately, all of this
works out pleasantly and we may calculate in the natural way, free of
concern. The reader interested in the details is referred to the
appendix \ref{appendix:rho_details}.

\subsection{Substitution}

We use $\Proc$ for the set of processes, $\QProc$ for the set of
names, and $\id{\{}\vec{y} / \vec{x} \id{\}}$ to denote partial maps,
$s : \QProc \rightarrow \QProc$. A map, $s$ lifts, uniquely, to a map
on process terms, $\widehat{s} : \Proc \rightarrow \Proc$ by the
following equations.

\begin{mathpar}
  (0) \psubstp{Q}{P} := 0 \\
  (R \juxtap S) \psubstp{Q}{P}
  :=    
  (R)\psubstp{Q}{P} \juxtap (S) \psubstp{Q}{P} \\
  (x?(y).R) \psubstp{Q}{P}    
  :=    
  (x)\substp{Q}{P} (z)\concat( (R \psubstn{z}{y}) \psubstp{Q}{P} ) \\
  (\lift{x}{R}) \psubstp{Q}{P}  
  :=
  \lift{(x)\substp{Q}{P}}{ R \psubstp{Q}{P} } \\
%   (\dropn{x})  \psubstp{Q}{P}       
%   := 
%   \left\{ 
%     \begin{array}{ccc} 
%       \dropn{\quotep{Q}} & & x \nameeq \quotep{P} \\
%       \dropn{x} & & otherwise \\
%     \end{array}
%   \right. 
  (\dropn{x})  \psubstp{Q}{P}       
  := 
  \left\{ 
    \begin{array}{ccc} 
      Q & & x \nameeq \quotep{P} \\
      \dropn{x} & & otherwise \\
    \end{array}
  \right.
\end{mathpar}
 

where

\begin{eqnarray}
  (x)\id{\{} \lpquote Q \rpquote / \lpquote P \rpquote \id{\}}            = 
  \left\{ 
    \begin{array}{ccc}
      \lpquote Q \rpquote & & x \nameeq \lpquote P \rpquote \\
      x & & otherwise \\
    \end{array}
  \right. \nonumber
\end{eqnarray}

and $z$ is chosen distinct from $\quotep{P}$, $\quotep{Q}$, the free
names in $Q$, and all the names in $R$. Our $\alpha$-equivalence will
be built in the standard way from this substitution.

\begin{remark}\label{rem:no_self_referential_names}
  One consequence of these definitions is that $\forall P. \quotep{P}
  \not\in \freenames{P}$.
\end{remark}

\subsection{ Dynamic quote: an example }

Anticipating something of what's to come, consider applying the
substitution, $\widehat{\id{\{}u / z \id{\}}}$, to the following pair
of processes, $\lift{w}{y!(z)}$ and $w[ \lpquote y!(z) \rpquote ]$.

\begin{eqnarray}
	\lift{w}{y!(z)}\widehat{\id{\{}u / z \id{\}}}
		& = &
		\lift{w}{y!(u)} \nonumber\\
	w[ \lpquote y!(z) \rpquote ] \widehat{ \id{\{}u / z \id{\}} }
		& = &
		w[ \lpquote y!(z) \rpquote ] \nonumber
\end{eqnarray}

Because the body of the process between quotes is impervious to
substitution, we get radically different answers. In fact, by
examining the first process in an input context,
e.g. $x?(z).\lift{w}{y!(z)}$, we see that the process under the lift
operator may be shaped by prefixed inputs binding a name inside it. In
this sense, the lift operator will be seen as a way to dynamically
construct processes before reifying them as names.

Finally equipped with these standard features we can present the
dynamics of the calculus.

\subsubsection{Operational semantics} 

Finally, we introduce the computational dynamics. What marks these
algebras as distinct from other more traditionally studied algebraic
structures, e.g. vector spaces or polynomial rings, is the manner in
which dynamics is captured. In traditional structures, dynamics is typically
expressed through morphisms between such structures, as in linear maps
between vector spaces or morphisms between rings. In algebras
associated with the semantics of computation, the dynamics is
expressed as part of the algebraic structure itself, through a
reduction reduction relation typically denoted by $\red$. Below, we
give a recursive presentation of this relation for the calculus used
in the encoding.

$\red \subseteq \pi \times \pi$
$\red : \pi \to \mathcal{P}(\pi)$

\begin{mathpar}
  \inferrule* [lab=Comm] { \textsf{match}( x_{src}, x_{trgt} ) } { x_{trgt}?(y)P \; | \; x_{src}!\langle {Q} \rangle \red P\{\quotep{Q}/y}\} }
  \and \\
  \inferrule* [lab=Par] {{P} \red {P}'} {{{P} | {Q}} \red {{P}' | {Q}}}
  \and
  \inferrule* [lab=Equiv]{{{P} \scong {P}'} \andalso {{P}' \red {Q}'} \andalso {{Q}' \scong {Q}}}{{P} \red {Q}}
\end{mathpar}

\begin{eqnarray*}
  match_{\equiv} (\quotep{P},\quotep{Q}) & := & P \equiv Q \\
  match_{\dagger}(\quotep{P},\quotep{Q}) & := & \forall R. P|Q \red^{*} R => R \red^{*} 0 \\
  match_{K}(\quotep{P},\quotep{Q}) & := & K \mbox{ for some context } K
\end{eqnarray*}

$u?(x)P | u!\langle Q \rangle \red P\{\quotep{Q}/x\}$

%We write $\wred$ for $\red^*$, and $P\red$ if $\exists Q $ such that $ P \red Q$.
We write $P\red$ if $\exists Q $ such that $ P \red Q$ and $P\not\red$, otherwise.

\section{Replication}

As mentioned before, it is known that replication (and hence
recursion) can be implemented in a higher-order process algebra
\cite{SangiorgiWalker}. As our first example of calculation with the
machinery thus far presented we give the construction explicitly in
the {\rhoc}.

\begin{eqnarray}
	D_{x} & := & \prefix{x}{y}{(\binpar{\outputp{x}{y}}{@{y}})} \nonumber\\
	\bangp_{x}{P} & := & \binpar{{x}!\langle{\binpar{D_{x}}{P}}\rangle}{D_{x}} \nonumber
\end{eqnarray}

\begin{eqnarray}
	\bangp_{x}{P} & & \nonumber\\
	=
	& {x}!\langle{(\prefix{x}{y}{(\outputp{x}{y} | @{y})) | P}}\rangle 
	      | \prefix{x}{y}{(\outputp{x}{y} | @{y})} & \nonumber\\
	\red
	& (\outputp{x}{y} | @{y})\substn{\quotep{(\prefix{x}{y}{(@{y} | \outputp{x}{y})) | P}}}{y} & \nonumber\\
	=
	& \outputp{x}{\quotep{(\prefix{x}{y}{(\outputp{x}{y} | @{y})) | P}}}
	  | {(\prefix{x}{y}{(\outputp{x}{y} | @{y})) | P}} & \nonumber\\
	\red
	& \ldots & \nonumber\\
	\red^*
	& P | P | \ldots & \nonumber
\end{eqnarray}

Of course, this encoding, as an implementation, runs away, unfolding
$\bangp{P}$ eagerly. A lazier and more implementable replication
operator, restricted to input-guarded processes, may be obtained as follows.

\begin{eqnarray}
\bangp{\prefix{u}{v}{P}} 
	:= 
	\binpar{\lift{x}{\prefix{u}{v}{(\binpar{D(x)}{P})}}}{D(x)} \nonumber
\end{eqnarray}

\begin{remark}
  Note that the lazier definition still does not deal with summation
  or mixed summation (i.e. sums over input and output). The reader is
  invited to construct definitions of replication that deal with these
  features. 

  Further, the definitions are parameterized in a name, $x$. Can you,
  gentle reader, make a definition that eliminates this parameter and
  guarantees no accidental interaction between the replication
  machinery and the process being replicated -- i.e. no accidental
  sharing of names used by the process to get its work done and the
  name(s) used by the replication to effect copying. This latter
  revision of the definition of replication is crucial to obtaining
  the expected identity $!!P \sim !P$.
\end{remark}

\begin{remark}\label{rem:paradoxical_combinator}
  The reader familiar with the lambda calculus will have noticed the
  similarity between $D$ and the paradoxical combinator.

  [Ed. note: the existence of this seems to suggest we have to be more
  restrictive on the set of processes and names we admit if we are to
  support no-cloning.]
\end{remark}

\subsubsection{Bisimulation}

The computational dynamics gives rise to another kind of equivalence,
the equivalence of computational behavior. As previously mentioned
this is typically captured \emph{via} some form of bisimulation.

% The notion we use in this paper is weak barbed bisimulation
% \cite{milner91polyadicpi}.

The notion we use in this paper is derived from weak barbed
bisimulation \cite{milner91polyadicpi}. 

\begin{definition}
An \emph{observation relation}, $\downarrow_{\mathcal N}$, over a set
of names, $\mathcal N$, is the smallest relation satisfying the rules
below.

\infrule[Out-barb]{y \in {\mathcal N}, \; x \nameeq y}
		  {\outputp{x}{v} \downarrow_{\mathcal N} x}
\infrule[Par-barb]{\mbox{$P\downarrow_{\mathcal N} x$ or $Q\downarrow_{\mathcal N} x$}}
		  {\binpar{P}{Q} \downarrow_{\mathcal N} x}

We write $P \Downarrow_{\mathcal N} x$ if there is $Q$ such that 
$P \wred Q$ and $Q \downarrow_{\mathcal N} x$.
\end{definition}

\begin{definition}
%\label{def.bbisim}
An  ${\mathcal N}$-\emph{barbed bisimulation} over a set of names, ${\mathcal N}$, is a symmetric binary relation 
${\mathcal S}_{\mathcal N}$ between agents such that $P\rel{S}_{\mathcal N}Q$ implies:
\begin{enumerate}
\item If $P \red P'$ then $Q \wred Q'$ and $P'\rel{S}_{\mathcal N} Q'$.
\item If $P\downarrow_{\mathcal N} x$, then $Q\Downarrow_{\mathcal N} x$.
\end{enumerate}
$P$ is ${\mathcal N}$-barbed bisimilar to $Q$, written
$P \wbbisim_{\mathcal N} Q$, if $P \rel{S}_{\mathcal N} Q$ for some ${\mathcal N}$-barbed bisimulation ${\mathcal S}_{\mathcal N}$.
\end{definition}

$\mathcal{R} \subseteq \pi \times \pi$

$P \mathcal{R} Q => \forall P'. P \red P' \Rightarrow \exists Q'. Q \red Q', P' \mathcal{R} Q'$

$P \vdash x \Rightarrow Q \vdash x$

\begin{mathpar}
  \inferrule*[lab=Out-barb]{x \nameeq y}{{y}!\langle{Q}\rangle \vdash x}
  \and
  \inferrule*[lab=Par-barb]{\mbox{$P\vdash x$ or $Q\vdash x$}}{\binpar{P}{Q} \vdash x}
\end{mathpar}

\subsubsection{Contexts}

One of the principle advantages of computational calculi like the
$\pi$-calculus is a well-defined notion of context,
contextual-equivalence and a correlation between
contextual-equivalence and notions of bisimulation. The notion of
context allows the decomposition of a process into (sub-)process and
its syntactic environment, its context. Thus, a context may be
thought of as a process with a ``hole'' (written $\Box$) in it. The
application of a context $M$ to a process $P$, written $M[P]$, is
tantamount to filling the hole in $M$ with $P$. In this paper we do
not need the full weight of this theory, but do make use of the notion
of context in the proof the main theorem. 

\begin{mathpar}
  \inferrule* [lab=summation] {} {{M_{M},M_{N}} \bc \Box \;|\; x.M_{A} \;|\; M_{M}+M_{N}}
  \and
  \inferrule* [lab=agent] {} {{M_{A}} \bc (\vec{x})M_{P} \;| \; \clift{P_0,\ldots,M_{P},\ldots,P_N}}
  \and \\
  \inferrule* [lab=process] {} {{M_{P}} \bc M_{N} \;| \;P|M_{P} }
\end{mathpar} 

\begin{mathpar}
  \inferrule* [lab=sychronization] {} {M_{N} \bc \Box \;|\; x?M_{F} \;|\; x!M_{C}}
  \and
  \inferrule* [lab=abstraction] {} {{M_{F}} \bc (x)M_{P} }
  \and
  \inferrule* [lab=concretion] {} {{M_{C}} \bc \langle M_{P} \rangle }
  \and \\
  \inferrule* [lab=process] {} {{M_{P}} \bc M_{N} \;| \;P|M_{P} }
\end{mathpar}

\begin{definition}[contextual application] Given a context $M$, and
  process $P$, we define the \emph{contextual application}, $M[P] :=
  M\{P/\Box\}$. That is, the contextual application of M to P is the
  substitution of $P$ for $\Box$ in $M$.
\end{definition}

$\meaningof{-} : L \to \mathcal{P}(\pi)$

\begin{mathpar}
  \inferrule* [lab=collection] {} {\meaningof{true} = \pi, \and \meaningof{~E} = \pi \setminus \meaningof{E}, \and \meaningof{E_{1} \& E_{2}} = \meaningof{E_{1}} \cap \meaningof{E_{2}}}
\end{mathpar}

\begin{mathpar}
  \inferrule* [lab=structure] {} {\meaningof{0} = \{ P \in \pi | P \equiv 0 \}, \and \\ \meaningof{E_1 | E_2} = \{ P \in \pi | P \equiv P_{1} | P_{2}, P_{1} \in \meaningof{E_{1}}, P_{2} \in \meaningof{E_2}\} }
\end{mathpar}

\begin{mathpar}
 \inferrule* [lab=behavior] {} {\meaningof{\langle a?b \rangle E} = \{ P \in \pi | P \equiv Q | u?(y)P', \\ \and \\\\ \and \\ \;\;\; u \in \meaningof{a}, \forall z.P'\{z/y\} \in \meaningof{E\{z/b\}}\}, \and \\ \meaningof{a!E} = \{ P \in \pi | P \equiv Q | x!\langle P' \rangle, x \in \meaningof{a} P' \in \meaningof{E}\} }
\end{mathpar}

\begin{mathpar}
 \inferrule* [lab=nominal] {} {\meaningof{\quotep{E}} = \{ \quotep{P} \in \quotep{\pi} | P \in \meaningof{E} \}, \and \meaningof{\quotep{P}} = \{ \quotep{Q} \in \quotep{\pi} | P \equiv Q \} \and \\ \meaningof{@\quotep{E}} = \{ P \in \pi | P \equiv @x, x \in \meaningof{E} \}}
\end{mathpar}

\begin{eqnarray*}
  \\
  \meaningof{-} : TS \to ST
\end{eqnarray*}

\begin{eqnarray*}
  \\
  L : TS \to ST
\end{eqnarray*}

\begin{eqnarray*}
  \\
  P \models E \iff P \in \meaningof{E}
\end{eqnarray*}

\begin{eqnarray*}
  P \approx_{L} Q \iff \forall E \in L. P \models E \iff Q \models E
\end{eqnarray*}

\begin{eqnarray*}
  P \approx_{K} Q
\end{eqnarray*}

\begin{eqnarray*}
  P \approx Q
\end{eqnarray*}

$\approx_{K} = \approx = \approx_{L}$

\subsubsection{Contextual duality}

Note that contexts extend the quotation operation to a family of
operations from processes to names. Given a context, $M$, we can
define a \emph{nominal context}, $\quotep{M}$ by $\quotep{M}[P] :=
\quotep{M[P]}$. To foreshadow what is to come we observe that these
operations enjoy a duality with processes very much like the duality
between vectors and maps from vectors to scalars.

Further, because the calculus is essentially higher-order, we have a
correspondence between contexts and processes. More specifically,
given a name $x$ and a context $M$ we can construct $M^{*}_{x}$ such
that 

\begin{mathpar}
  M^{*}_{x} | \lift{x}{P} \red M[P]
\end{mathpar}

namely,

\begin{mathpar}
  M^{*}_{x} := x?(u).M[\dropn{u}]
\end{mathpar}

The dependence of $M^{*}_{x}$ on a name makes it an abstraction, 

\begin{mathpar}
  M^{*} := (x)x?(u).M[\dropn{u}]
\end{mathpar}

\subsection{Additional notation}

It will sometimes be convenient to denote the process a name
quotes. We already have the notation $x = \quotep{P}$, but it will be
convenient to introduce an alternate notation, $\procn{x}$, when we
want to emphasize the connection to the use of the name. Note that, by
virtue of name equivalence, $\quotep{\procn{x}} \nameeq x$; so, the
notation is consistent with previous definitions.

Further, because names have structure it is possible to effect
substitutions on the basis of that structure. This means we need to
upgrade our notation for substitutions, which we accomplish by
adapting comprehension notation. Thus,

\begin{mathpar}
  P\{ y / x : x \in S \}
\end{mathpar}

is interpreted to mean the process derived from P by replacing (in a
capture-avoiding manner) each occurrence of $x$ in $S$ by $y$. For example,

\begin{mathpar}
  P\{ \quotep{\procn{x}|\procn{x}} / x : x \in \freenames{P} \}
\end{mathpar}

will replace each (occurrence) of a free name $x$ in $P$ by
$\quotep{\procn{x}|\procn{x}}$.

Also, we will avail ourselves of the notation $x^{L}$ and $x^{R}$ to
denote injections of a name into disjoint copies of the name
space. There are numerous ways to accomplish this. One example can be
found in \cite{MeredithR05}. This notation overloads to vectors of
names: $\vec{x}^{\pi} := (x_{i}^{\pi} \; : \; 0 \leq i < |\vec{x}| )$ where $\pi \in \{L,R\}$.

We also use $P^{\Box} := P|\Box$.

In \cite{MeredithR05} an interpretation of the new operator is
given. It turns out that there are several possible interpretations
all enjoying the requisite algebraic properties of the operator (see
\cite{milner91polyadicpi}). We will therefore make liberal use of
$(\nu\; \vec{x})P$.

% subsection the_syntax_and_semantics_of_the_notation_system (end)   

\input{qm2pi.qmops} 

\input{qm2pi.sterngerlach} 

\input{qm2pi.metric} 

% section concurrent_process_calculi (end)

%\input{qm2pi.proofsketch}

% section proof sketch (end)

%\input{qm2pi.slviaknots} 

% section spatial logic via knots (end)

\input{qm2pi.conclusion}

% section conclusion (end)

%\input{qm2pi.dtcodes} 

% section wiring algorithm (end)

\input{qm2pi.ack} 

% section acknowledgments (end)

\newpage


\bibliographystyle{plain}   
\bibliography{../../biblios/main.bib}

\input{qm2pi.rhodetails}

\end{document}

 

\documentclass[12pt]{llncs}
%\documentclass{jktr}

\usepackage[pdftex]{hyperref}                   
\usepackage {listings}
\usepackage {mathpartir}
\usepackage{bcprules}
%\usepackage{listings}
                       
\usepackage{graphicx} 
%\usepackage[margins=2.5cm,nohead,nofoot]{geometry}
%\usepackage{geometry}
\usepackage{amsfonts}
\usepackage{amstext}
\usepackage{latexsym}
\usepackage{amssymb}
\usepackage{color}


%\include{myPreamble}
\include{qm2pi.local} 

%\ifpdf
%\usepackage[pdftex]{graphicx}
%\else
%\usepackage{graphicx}
%\fi

 % \ifpdf
%  \usepackage{pdfsync}
%  \if


%\title{Brief Article}
%\author{David F. Snyder}
%\author{L.G. Meredith}

%\address{Dept. of Math., Texas State University--San Marcos, San Marcos, TX 78666}
       
\pagestyle{empty}


\begin{document}

\lstset{language=[Objective]Caml,frame=shadowbox}

\input{qm2pi.front}

% section front matter (end)

\input{qm2pi.intro} 
 
% section introduction (end)

% \input{qm2pi.knotations} 

% section notation (end)

\input{qm2pi.process.calculi} 

% section concurrent_process_calculi_and_spatial_logics_ (end)
    
%\input{qm2pi.knots2pi} 

%\input{qm2pi.trefoil} 

%\input{qm2pi.mainthm} 

% subsection basic_interpretation (end)

%\input{qm2pi.rho.presentation} 
\subsection{The syntax and semantics of the notation system}\label{sub:the_syntax_and_semantics_of_the_notation_system} % (fold)

We now summarize a technical presentation of the calculus that
embodies our theory of dynamics. The typical presentation of such a
calculus follows the style of giving generators and relations on
them. The grammar, below, describing term constructors, freely
generates the set of processes, $\Proc$. This set is then quotiented
by a relation known as structural congruence and it is over this set
that the notion of dynamics is expressed. This presentation is
essentially that of \cite{MeredithR05} with the addition of
polyadicity and summation. For readability we have relegated some of
the technical subtleties to an appendix.

\subsubsection{Process grammar}\label{subsub:process_grammar}

\begin{mathpar}
  \inferrule* [lab=synchronization] {} {{M} \bc \pzero \;|\; x?F \;|\; x!C }
  \and
  \inferrule* [lab=abstraction] {} {{F} \bc (x)P}
  \and
  \inferrule* [lab=concretion] {} {{C} \bc \langle Q \rangle}
  \and
  \inferrule* [lab=process] {} {{P,Q} \bc M \;| \;P|Q \;|\; @{x}}
  \and
  \inferrule* [lab=name] {} {{x} \bc \quotep{P}}
\end{mathpar} 

Note that $\vec{x}$ (resp. $\vec{P}$) denotes a vector of names
(resp. processes) of length $|\vec{x}|$ (resp. $|\vec{P}|$). We adopt
the following useful abbreviations.

\begin{mathpar}
   x?(\vec{y}).P := x.(\vec{y})P \and  x\clift{\vec{P}} := x.\clift{\vec{P}}
   \and x!(y) := \lift{x}{\dropn{y}}
   \and \Pi_{i=0}^{n-1}P_i := P_0 | \ldots | P_{n-1}
\end{mathpar}

\subsubsection{Structural congruence}

\paragraph{Free and bound names and alpha-equivalence.} At the
core of structural equivalence is alpha-equivalence which identifies
process that are the same up to a change of variable. Formally, we
recognize the distinction between free and bound names. The free names
of a process, $\freenames{P}$, may be calculated recursively as
follows:

\begin{mathpar}
\freenames{\pzero} := \emptyset
  \and \\
  \freenames{x?(y).P} := \{ x \} \cup (\freenames{P} \setminus \{ y \})
  \and 
  \freenames{x!\langle P \rangle} := \{ x \} \cup \{ P \} 
  \and \\
  \freenames{P|Q} := \freenames{P} \cup \freenames{Q}
  \and \\
  \freenames{@{x}} := \{ x \}
\end{mathpar}

$\pi$
$\quotep{\pi}$

$\freenames{-} : \pi \to \mathcal{P}(\quotep{\pi})$

\begin{eqnarray*}
  \freenames{\pzero} & := & \emptyset \\
  \freenames{x?(y).P} & := & \{ x \} \cup (\freenames{P} \setminus \{ y \}) \\
  \freenames{x!\langle P \rangle} & := & \{ x \} \cup \{ P \} \\
  \freenames{P|Q} & := & \freenames{P} \cup \freenames{Q} \\
  \freenames{\dropn{x}} & := & \{ x \}
\end{eqnarray*}

The bound names of a process, $\boundnames{P}$, are those names occurring in $P$
that are not free. For example, in $x?(y).0$, the name $x$ is free, while $y$ is bound.

\begin{mathpar}
  \inferrule* [lab=monoidal-laws] {} { P|Q \equiv Q|P \and P|0 \equiv P \and P|(Q|R) \equiv (P|Q)|R }
\end{mathpar}

\begin{mathpar}
  \inferrule* [lab=alpha-equivalence] {} { (x)P \equiv (y)P\{y/x\} \and y \not\in \freenames{P} }
\end{mathpar}

\begin{definition}
Then two processes, $P,Q$, are alpha-equivalent if $P = Q\{\vec{y}/\vec{x}\}$ for
some $\vec{x} \in \boundnames{Q},\vec{y} \in \boundnames{P}$, where $Q\{\vec{y}/\vec{x}\}$
denotes the capture-avoiding substitution of $\vec{y}$ for $\vec{x}$ in $Q$.
\end{definition}

\begin{definition}
  The {\em structural congruence} \cite{SangiorgiWalker} , $\equiv$,
  between processes is the least congruence containing
  alpha-equivalence, satisfying the abelian monoid laws
  (associativity, commutativity and $\pzero$ as identity) for parallel
  composition $|$ and for summation $+$.
\end{definition}

\subsection{Name equivalence}

We take name equivalence, written $\nameeq$, to be the smallest
equivalence relation generated by the following rules.

\begin{mathpar}
\inferrule*[lab=Quote-drop]
{ }
{ \quotep{@{x}} \nameeq x }

\inferrule*[lab=Struct-equiv]
{ P \scong Q }
{ \quotep{P} \nameeq \quotep{Q} }
\end{mathpar}

The astute reader will have noticed that the mutual recursion of names
and processes imposes a mutual recursion on alpha-equivalence and
structural equivalence via name-equivalence. Fortunately, all of this
works out pleasantly and we may calculate in the natural way, free of
concern. The reader interested in the details is referred to the
appendix \ref{appendix:rho_details}.

\subsection{Substitution}

We use $\Proc$ for the set of processes, $\QProc$ for the set of
names, and $\id{\{}\vec{y} / \vec{x} \id{\}}$ to denote partial maps,
$s : \QProc \rightarrow \QProc$. A map, $s$ lifts, uniquely, to a map
on process terms, $\widehat{s} : \Proc \rightarrow \Proc$ by the
following equations.

\begin{mathpar}
  (0) \psubstp{Q}{P} := 0 \\
  (R \juxtap S) \psubstp{Q}{P}
  :=    
  (R)\psubstp{Q}{P} \juxtap (S) \psubstp{Q}{P} \\
  (x?(y).R) \psubstp{Q}{P}    
  :=    
  (x)\substp{Q}{P} (z)\concat( (R \psubstn{z}{y}) \psubstp{Q}{P} ) \\
  (\lift{x}{R}) \psubstp{Q}{P}  
  :=
  \lift{(x)\substp{Q}{P}}{ R \psubstp{Q}{P} } \\
%   (\dropn{x})  \psubstp{Q}{P}       
%   := 
%   \left\{ 
%     \begin{array}{ccc} 
%       \dropn{\quotep{Q}} & & x \nameeq \quotep{P} \\
%       \dropn{x} & & otherwise \\
%     \end{array}
%   \right. 
  (\dropn{x})  \psubstp{Q}{P}       
  := 
  \left\{ 
    \begin{array}{ccc} 
      Q & & x \nameeq \quotep{P} \\
      \dropn{x} & & otherwise \\
    \end{array}
  \right.
\end{mathpar}
 

where

\begin{eqnarray}
  (x)\id{\{} \lpquote Q \rpquote / \lpquote P \rpquote \id{\}}            = 
  \left\{ 
    \begin{array}{ccc}
      \lpquote Q \rpquote & & x \nameeq \lpquote P \rpquote \\
      x & & otherwise \\
    \end{array}
  \right. \nonumber
\end{eqnarray}

and $z$ is chosen distinct from $\quotep{P}$, $\quotep{Q}$, the free
names in $Q$, and all the names in $R$. Our $\alpha$-equivalence will
be built in the standard way from this substitution.

\begin{remark}\label{rem:no_self_referential_names}
  One consequence of these definitions is that $\forall P. \quotep{P}
  \not\in \freenames{P}$.
\end{remark}

\subsection{ Dynamic quote: an example }

Anticipating something of what's to come, consider applying the
substitution, $\widehat{\id{\{}u / z \id{\}}}$, to the following pair
of processes, $\lift{w}{y!(z)}$ and $w[ \lpquote y!(z) \rpquote ]$.

\begin{eqnarray}
	\lift{w}{y!(z)}\widehat{\id{\{}u / z \id{\}}}
		& = &
		\lift{w}{y!(u)} \nonumber\\
	w[ \lpquote y!(z) \rpquote ] \widehat{ \id{\{}u / z \id{\}} }
		& = &
		w[ \lpquote y!(z) \rpquote ] \nonumber
\end{eqnarray}

Because the body of the process between quotes is impervious to
substitution, we get radically different answers. In fact, by
examining the first process in an input context,
e.g. $x?(z).\lift{w}{y!(z)}$, we see that the process under the lift
operator may be shaped by prefixed inputs binding a name inside it. In
this sense, the lift operator will be seen as a way to dynamically
construct processes before reifying them as names.

Finally equipped with these standard features we can present the
dynamics of the calculus.

\subsubsection{Operational semantics} 

Finally, we introduce the computational dynamics. What marks these
algebras as distinct from other more traditionally studied algebraic
structures, e.g. vector spaces or polynomial rings, is the manner in
which dynamics is captured. In traditional structures, dynamics is typically
expressed through morphisms between such structures, as in linear maps
between vector spaces or morphisms between rings. In algebras
associated with the semantics of computation, the dynamics is
expressed as part of the algebraic structure itself, through a
reduction reduction relation typically denoted by $\red$. Below, we
give a recursive presentation of this relation for the calculus used
in the encoding.

$\red \subseteq \pi \times \pi$
$\red : \pi \to \mathcal{P}(\pi)$

\begin{mathpar}
  \inferrule* [lab=Comm] { \textsf{match}( x_{src}, x_{trgt} ) } { x_{trgt}?(y)P \; | \; x_{src}!\langle {Q} \rangle \red P\{\quotep{Q}/y}\} }
  \and \\
  \inferrule* [lab=Par] {{P} \red {P}'} {{{P} | {Q}} \red {{P}' | {Q}}}
  \and
  \inferrule* [lab=Equiv]{{{P} \scong {P}'} \andalso {{P}' \red {Q}'} \andalso {{Q}' \scong {Q}}}{{P} \red {Q}}
\end{mathpar}

\begin{eqnarray*}
  match_{\equiv} (\quotep{P},\quotep{Q}) & := & P \equiv Q \\
  match_{\dagger}(\quotep{P},\quotep{Q}) & := & \forall R. P|Q \red^{*} R => R \red^{*} 0 \\
  match_{K}(\quotep{P},\quotep{Q}) & := & K \mbox{ for some context } K
\end{eqnarray*}

$u?(x)P | u!\langle Q \rangle \red P\{\quotep{Q}/x\}$

%We write $\wred$ for $\red^*$, and $P\red$ if $\exists Q $ such that $ P \red Q$.
We write $P\red$ if $\exists Q $ such that $ P \red Q$ and $P\not\red$, otherwise.

\section{Replication}

As mentioned before, it is known that replication (and hence
recursion) can be implemented in a higher-order process algebra
\cite{SangiorgiWalker}. As our first example of calculation with the
machinery thus far presented we give the construction explicitly in
the {\rhoc}.

\begin{eqnarray}
	D_{x} & := & \prefix{x}{y}{(\binpar{\outputp{x}{y}}{@{y}})} \nonumber\\
	\bangp_{x}{P} & := & \binpar{{x}!\langle{\binpar{D_{x}}{P}}\rangle}{D_{x}} \nonumber
\end{eqnarray}

\begin{eqnarray}
	\bangp_{x}{P} & & \nonumber\\
	=
	& {x}!\langle{(\prefix{x}{y}{(\outputp{x}{y} | @{y})) | P}}\rangle 
	      | \prefix{x}{y}{(\outputp{x}{y} | @{y})} & \nonumber\\
	\red
	& (\outputp{x}{y} | @{y})\substn{\quotep{(\prefix{x}{y}{(@{y} | \outputp{x}{y})) | P}}}{y} & \nonumber\\
	=
	& \outputp{x}{\quotep{(\prefix{x}{y}{(\outputp{x}{y} | @{y})) | P}}}
	  | {(\prefix{x}{y}{(\outputp{x}{y} | @{y})) | P}} & \nonumber\\
	\red
	& \ldots & \nonumber\\
	\red^*
	& P | P | \ldots & \nonumber
\end{eqnarray}

Of course, this encoding, as an implementation, runs away, unfolding
$\bangp{P}$ eagerly. A lazier and more implementable replication
operator, restricted to input-guarded processes, may be obtained as follows.

\begin{eqnarray}
\bangp{\prefix{u}{v}{P}} 
	:= 
	\binpar{\lift{x}{\prefix{u}{v}{(\binpar{D(x)}{P})}}}{D(x)} \nonumber
\end{eqnarray}

\begin{remark}
  Note that the lazier definition still does not deal with summation
  or mixed summation (i.e. sums over input and output). The reader is
  invited to construct definitions of replication that deal with these
  features. 

  Further, the definitions are parameterized in a name, $x$. Can you,
  gentle reader, make a definition that eliminates this parameter and
  guarantees no accidental interaction between the replication
  machinery and the process being replicated -- i.e. no accidental
  sharing of names used by the process to get its work done and the
  name(s) used by the replication to effect copying. This latter
  revision of the definition of replication is crucial to obtaining
  the expected identity $!!P \sim !P$.
\end{remark}

\begin{remark}\label{rem:paradoxical_combinator}
  The reader familiar with the lambda calculus will have noticed the
  similarity between $D$ and the paradoxical combinator.

  [Ed. note: the existence of this seems to suggest we have to be more
  restrictive on the set of processes and names we admit if we are to
  support no-cloning.]
\end{remark}

\subsubsection{Bisimulation}

The computational dynamics gives rise to another kind of equivalence,
the equivalence of computational behavior. As previously mentioned
this is typically captured \emph{via} some form of bisimulation.

% The notion we use in this paper is weak barbed bisimulation
% \cite{milner91polyadicpi}.

The notion we use in this paper is derived from weak barbed
bisimulation \cite{milner91polyadicpi}. 

\begin{definition}
An \emph{observation relation}, $\downarrow_{\mathcal N}$, over a set
of names, $\mathcal N$, is the smallest relation satisfying the rules
below.

\infrule[Out-barb]{y \in {\mathcal N}, \; x \nameeq y}
		  {\outputp{x}{v} \downarrow_{\mathcal N} x}
\infrule[Par-barb]{\mbox{$P\downarrow_{\mathcal N} x$ or $Q\downarrow_{\mathcal N} x$}}
		  {\binpar{P}{Q} \downarrow_{\mathcal N} x}

We write $P \Downarrow_{\mathcal N} x$ if there is $Q$ such that 
$P \wred Q$ and $Q \downarrow_{\mathcal N} x$.
\end{definition}

\begin{definition}
%\label{def.bbisim}
An  ${\mathcal N}$-\emph{barbed bisimulation} over a set of names, ${\mathcal N}$, is a symmetric binary relation 
${\mathcal S}_{\mathcal N}$ between agents such that $P\rel{S}_{\mathcal N}Q$ implies:
\begin{enumerate}
\item If $P \red P'$ then $Q \wred Q'$ and $P'\rel{S}_{\mathcal N} Q'$.
\item If $P\downarrow_{\mathcal N} x$, then $Q\Downarrow_{\mathcal N} x$.
\end{enumerate}
$P$ is ${\mathcal N}$-barbed bisimilar to $Q$, written
$P \wbbisim_{\mathcal N} Q$, if $P \rel{S}_{\mathcal N} Q$ for some ${\mathcal N}$-barbed bisimulation ${\mathcal S}_{\mathcal N}$.
\end{definition}

$\mathcal{R} \subseteq \pi \times \pi$

$P \mathcal{R} Q => \forall P'. P \red P' \Rightarrow \exists Q'. Q \red Q', P' \mathcal{R} Q'$

$P \vdash x \Rightarrow Q \vdash x$

\begin{mathpar}
  \inferrule*[lab=Out-barb]{x \nameeq y}{{y}!\langle{Q}\rangle \vdash x}
  \and
  \inferrule*[lab=Par-barb]{\mbox{$P\vdash x$ or $Q\vdash x$}}{\binpar{P}{Q} \vdash x}
\end{mathpar}

\subsubsection{Contexts}

One of the principle advantages of computational calculi like the
$\pi$-calculus is a well-defined notion of context,
contextual-equivalence and a correlation between
contextual-equivalence and notions of bisimulation. The notion of
context allows the decomposition of a process into (sub-)process and
its syntactic environment, its context. Thus, a context may be
thought of as a process with a ``hole'' (written $\Box$) in it. The
application of a context $M$ to a process $P$, written $M[P]$, is
tantamount to filling the hole in $M$ with $P$. In this paper we do
not need the full weight of this theory, but do make use of the notion
of context in the proof the main theorem. 

\begin{mathpar}
  \inferrule* [lab=summation] {} {{M_{M},M_{N}} \bc \Box \;|\; x.M_{A} \;|\; M_{M}+M_{N}}
  \and
  \inferrule* [lab=agent] {} {{M_{A}} \bc (\vec{x})M_{P} \;| \; \clift{P_0,\ldots,M_{P},\ldots,P_N}}
  \and \\
  \inferrule* [lab=process] {} {{M_{P}} \bc M_{N} \;| \;P|M_{P} }
\end{mathpar} 

\begin{mathpar}
  \inferrule* [lab=sychronization] {} {M_{N} \bc \Box \;|\; x?M_{F} \;|\; x!M_{C}}
  \and
  \inferrule* [lab=abstraction] {} {{M_{F}} \bc (x)M_{P} }
  \and
  \inferrule* [lab=concretion] {} {{M_{C}} \bc \langle M_{P} \rangle }
  \and \\
  \inferrule* [lab=process] {} {{M_{P}} \bc M_{N} \;| \;P|M_{P} }
\end{mathpar}

\begin{definition}[contextual application] Given a context $M$, and
  process $P$, we define the \emph{contextual application}, $M[P] :=
  M\{P/\Box\}$. That is, the contextual application of M to P is the
  substitution of $P$ for $\Box$ in $M$.
\end{definition}

$\meaningof{-} : L \to \mathcal{P}(\pi)$

\begin{mathpar}
  \inferrule* [lab=collection] {} {\meaningof{true} = \pi, \and \meaningof{~E} = \pi \setminus \meaningof{E}, \and \meaningof{E_{1} \& E_{2}} = \meaningof{E_{1}} \cap \meaningof{E_{2}}}
\end{mathpar}

\begin{mathpar}
  \inferrule* [lab=structure] {} {\meaningof{0} = \{ P \in \pi | P \equiv 0 \}, \and \\ \meaningof{E_1 | E_2} = \{ P \in \pi | P \equiv P_{1} | P_{2}, P_{1} \in \meaningof{E_{1}}, P_{2} \in \meaningof{E_2}\} }
\end{mathpar}

\begin{mathpar}
 \inferrule* [lab=behavior] {} {\meaningof{\langle a?b \rangle E} = \{ P \in \pi | P \equiv Q | u?(y)P', \\ \and \\\\ \and \\ \;\;\; u \in \meaningof{a}, \forall z.P'\{z/y\} \in \meaningof{E\{z/b\}}\}, \and \\ \meaningof{a!E} = \{ P \in \pi | P \equiv Q | x!\langle P' \rangle, x \in \meaningof{a} P' \in \meaningof{E}\} }
\end{mathpar}

\begin{mathpar}
 \inferrule* [lab=nominal] {} {\meaningof{\quotep{E}} = \{ \quotep{P} \in \quotep{\pi} | P \in \meaningof{E} \}, \and \meaningof{\quotep{P}} = \{ \quotep{Q} \in \quotep{\pi} | P \equiv Q \} \and \\ \meaningof{@\quotep{E}} = \{ P \in \pi | P \equiv @x, x \in \meaningof{E} \}}
\end{mathpar}

\begin{eqnarray*}
  \\
  \meaningof{-} : TS \to ST
\end{eqnarray*}

\begin{eqnarray*}
  \\
  L : TS \to ST
\end{eqnarray*}

\begin{eqnarray*}
  \\
  P \models E \iff P \in \meaningof{E}
\end{eqnarray*}

\begin{eqnarray*}
  P \approx_{L} Q \iff \forall E \in L. P \models E \iff Q \models E
\end{eqnarray*}

\begin{eqnarray*}
  P \approx_{K} Q
\end{eqnarray*}

\begin{eqnarray*}
  P \approx Q
\end{eqnarray*}

$\approx_{K} = \approx = \approx_{L}$

\subsubsection{Contextual duality}

Note that contexts extend the quotation operation to a family of
operations from processes to names. Given a context, $M$, we can
define a \emph{nominal context}, $\quotep{M}$ by $\quotep{M}[P] :=
\quotep{M[P]}$. To foreshadow what is to come we observe that these
operations enjoy a duality with processes very much like the duality
between vectors and maps from vectors to scalars.

Further, because the calculus is essentially higher-order, we have a
correspondence between contexts and processes. More specifically,
given a name $x$ and a context $M$ we can construct $M^{*}_{x}$ such
that 

\begin{mathpar}
  M^{*}_{x} | \lift{x}{P} \red M[P]
\end{mathpar}

namely,

\begin{mathpar}
  M^{*}_{x} := x?(u).M[\dropn{u}]
\end{mathpar}

The dependence of $M^{*}_{x}$ on a name makes it an abstraction, 

\begin{mathpar}
  M^{*} := (x)x?(u).M[\dropn{u}]
\end{mathpar}

\subsection{Additional notation}

It will sometimes be convenient to denote the process a name
quotes. We already have the notation $x = \quotep{P}$, but it will be
convenient to introduce an alternate notation, $\procn{x}$, when we
want to emphasize the connection to the use of the name. Note that, by
virtue of name equivalence, $\quotep{\procn{x}} \nameeq x$; so, the
notation is consistent with previous definitions.

Further, because names have structure it is possible to effect
substitutions on the basis of that structure. This means we need to
upgrade our notation for substitutions, which we accomplish by
adapting comprehension notation. Thus,

\begin{mathpar}
  P\{ y / x : x \in S \}
\end{mathpar}

is interpreted to mean the process derived from P by replacing (in a
capture-avoiding manner) each occurrence of $x$ in $S$ by $y$. For example,

\begin{mathpar}
  P\{ \quotep{\procn{x}|\procn{x}} / x : x \in \freenames{P} \}
\end{mathpar}

will replace each (occurrence) of a free name $x$ in $P$ by
$\quotep{\procn{x}|\procn{x}}$.

Also, we will avail ourselves of the notation $x^{L}$ and $x^{R}$ to
denote injections of a name into disjoint copies of the name
space. There are numerous ways to accomplish this. One example can be
found in \cite{MeredithR05}. This notation overloads to vectors of
names: $\vec{x}^{\pi} := (x_{i}^{\pi} \; : \; 0 \leq i < |\vec{x}| )$ where $\pi \in \{L,R\}$.

We also use $P^{\Box} := P|\Box$.

In \cite{MeredithR05} an interpretation of the new operator is
given. It turns out that there are several possible interpretations
all enjoying the requisite algebraic properties of the operator (see
\cite{milner91polyadicpi}). We will therefore make liberal use of
$(\nu\; \vec{x})P$.

% subsection the_syntax_and_semantics_of_the_notation_system (end)   

\input{qm2pi.qmops} 

\input{qm2pi.sterngerlach} 

\input{qm2pi.metric} 

% section concurrent_process_calculi (end)

%\input{qm2pi.proofsketch}

% section proof sketch (end)

%\input{qm2pi.slviaknots} 

% section spatial logic via knots (end)

\input{qm2pi.conclusion}

% section conclusion (end)

%\input{qm2pi.dtcodes} 

% section wiring algorithm (end)

\input{qm2pi.ack} 

% section acknowledgments (end)

\newpage


\bibliographystyle{plain}   
\bibliography{../../biblios/main.bib}

\input{qm2pi.rhodetails}

\end{document}

 

% section concurrent_process_calculi (end)

%\documentclass[12pt]{llncs}
%\documentclass{jktr}

\usepackage[pdftex]{hyperref}                   
\usepackage {listings}
\usepackage {mathpartir}
\usepackage{bcprules}
%\usepackage{listings}
                       
\usepackage{graphicx} 
%\usepackage[margins=2.5cm,nohead,nofoot]{geometry}
%\usepackage{geometry}
\usepackage{amsfonts}
\usepackage{amstext}
\usepackage{latexsym}
\usepackage{amssymb}
\usepackage{color}


%\include{myPreamble}
\include{qm2pi.local} 

%\ifpdf
%\usepackage[pdftex]{graphicx}
%\else
%\usepackage{graphicx}
%\fi

 % \ifpdf
%  \usepackage{pdfsync}
%  \if


%\title{Brief Article}
%\author{David F. Snyder}
%\author{L.G. Meredith}

%\address{Dept. of Math., Texas State University--San Marcos, San Marcos, TX 78666}
       
\pagestyle{empty}


\begin{document}

\lstset{language=[Objective]Caml,frame=shadowbox}

\input{qm2pi.front}

% section front matter (end)

\input{qm2pi.intro} 
 
% section introduction (end)

% \input{qm2pi.knotations} 

% section notation (end)

\input{qm2pi.process.calculi} 

% section concurrent_process_calculi_and_spatial_logics_ (end)
    
%\input{qm2pi.knots2pi} 

%\input{qm2pi.trefoil} 

%\input{qm2pi.mainthm} 

% subsection basic_interpretation (end)

%\input{qm2pi.rho.presentation} 
\subsection{The syntax and semantics of the notation system}\label{sub:the_syntax_and_semantics_of_the_notation_system} % (fold)

We now summarize a technical presentation of the calculus that
embodies our theory of dynamics. The typical presentation of such a
calculus follows the style of giving generators and relations on
them. The grammar, below, describing term constructors, freely
generates the set of processes, $\Proc$. This set is then quotiented
by a relation known as structural congruence and it is over this set
that the notion of dynamics is expressed. This presentation is
essentially that of \cite{MeredithR05} with the addition of
polyadicity and summation. For readability we have relegated some of
the technical subtleties to an appendix.

\subsubsection{Process grammar}\label{subsub:process_grammar}

\begin{mathpar}
  \inferrule* [lab=synchronization] {} {{M} \bc \pzero \;|\; x?F \;|\; x!C }
  \and
  \inferrule* [lab=abstraction] {} {{F} \bc (x)P}
  \and
  \inferrule* [lab=concretion] {} {{C} \bc \langle Q \rangle}
  \and
  \inferrule* [lab=process] {} {{P,Q} \bc M \;| \;P|Q \;|\; @{x}}
  \and
  \inferrule* [lab=name] {} {{x} \bc \quotep{P}}
\end{mathpar} 

Note that $\vec{x}$ (resp. $\vec{P}$) denotes a vector of names
(resp. processes) of length $|\vec{x}|$ (resp. $|\vec{P}|$). We adopt
the following useful abbreviations.

\begin{mathpar}
   x?(\vec{y}).P := x.(\vec{y})P \and  x\clift{\vec{P}} := x.\clift{\vec{P}}
   \and x!(y) := \lift{x}{\dropn{y}}
   \and \Pi_{i=0}^{n-1}P_i := P_0 | \ldots | P_{n-1}
\end{mathpar}

\subsubsection{Structural congruence}

\paragraph{Free and bound names and alpha-equivalence.} At the
core of structural equivalence is alpha-equivalence which identifies
process that are the same up to a change of variable. Formally, we
recognize the distinction between free and bound names. The free names
of a process, $\freenames{P}$, may be calculated recursively as
follows:

\begin{mathpar}
\freenames{\pzero} := \emptyset
  \and \\
  \freenames{x?(y).P} := \{ x \} \cup (\freenames{P} \setminus \{ y \})
  \and 
  \freenames{x!\langle P \rangle} := \{ x \} \cup \{ P \} 
  \and \\
  \freenames{P|Q} := \freenames{P} \cup \freenames{Q}
  \and \\
  \freenames{@{x}} := \{ x \}
\end{mathpar}

$\pi$
$\quotep{\pi}$

$\freenames{-} : \pi \to \mathcal{P}(\quotep{\pi})$

\begin{eqnarray*}
  \freenames{\pzero} & := & \emptyset \\
  \freenames{x?(y).P} & := & \{ x \} \cup (\freenames{P} \setminus \{ y \}) \\
  \freenames{x!\langle P \rangle} & := & \{ x \} \cup \{ P \} \\
  \freenames{P|Q} & := & \freenames{P} \cup \freenames{Q} \\
  \freenames{\dropn{x}} & := & \{ x \}
\end{eqnarray*}

The bound names of a process, $\boundnames{P}$, are those names occurring in $P$
that are not free. For example, in $x?(y).0$, the name $x$ is free, while $y$ is bound.

\begin{mathpar}
  \inferrule* [lab=monoidal-laws] {} { P|Q \equiv Q|P \and P|0 \equiv P \and P|(Q|R) \equiv (P|Q)|R }
\end{mathpar}

\begin{mathpar}
  \inferrule* [lab=alpha-equivalence] {} { (x)P \equiv (y)P\{y/x\} \and y \not\in \freenames{P} }
\end{mathpar}

\begin{definition}
Then two processes, $P,Q$, are alpha-equivalent if $P = Q\{\vec{y}/\vec{x}\}$ for
some $\vec{x} \in \boundnames{Q},\vec{y} \in \boundnames{P}$, where $Q\{\vec{y}/\vec{x}\}$
denotes the capture-avoiding substitution of $\vec{y}$ for $\vec{x}$ in $Q$.
\end{definition}

\begin{definition}
  The {\em structural congruence} \cite{SangiorgiWalker} , $\equiv$,
  between processes is the least congruence containing
  alpha-equivalence, satisfying the abelian monoid laws
  (associativity, commutativity and $\pzero$ as identity) for parallel
  composition $|$ and for summation $+$.
\end{definition}

\subsection{Name equivalence}

We take name equivalence, written $\nameeq$, to be the smallest
equivalence relation generated by the following rules.

\begin{mathpar}
\inferrule*[lab=Quote-drop]
{ }
{ \quotep{@{x}} \nameeq x }

\inferrule*[lab=Struct-equiv]
{ P \scong Q }
{ \quotep{P} \nameeq \quotep{Q} }
\end{mathpar}

The astute reader will have noticed that the mutual recursion of names
and processes imposes a mutual recursion on alpha-equivalence and
structural equivalence via name-equivalence. Fortunately, all of this
works out pleasantly and we may calculate in the natural way, free of
concern. The reader interested in the details is referred to the
appendix \ref{appendix:rho_details}.

\subsection{Substitution}

We use $\Proc$ for the set of processes, $\QProc$ for the set of
names, and $\id{\{}\vec{y} / \vec{x} \id{\}}$ to denote partial maps,
$s : \QProc \rightarrow \QProc$. A map, $s$ lifts, uniquely, to a map
on process terms, $\widehat{s} : \Proc \rightarrow \Proc$ by the
following equations.

\begin{mathpar}
  (0) \psubstp{Q}{P} := 0 \\
  (R \juxtap S) \psubstp{Q}{P}
  :=    
  (R)\psubstp{Q}{P} \juxtap (S) \psubstp{Q}{P} \\
  (x?(y).R) \psubstp{Q}{P}    
  :=    
  (x)\substp{Q}{P} (z)\concat( (R \psubstn{z}{y}) \psubstp{Q}{P} ) \\
  (\lift{x}{R}) \psubstp{Q}{P}  
  :=
  \lift{(x)\substp{Q}{P}}{ R \psubstp{Q}{P} } \\
%   (\dropn{x})  \psubstp{Q}{P}       
%   := 
%   \left\{ 
%     \begin{array}{ccc} 
%       \dropn{\quotep{Q}} & & x \nameeq \quotep{P} \\
%       \dropn{x} & & otherwise \\
%     \end{array}
%   \right. 
  (\dropn{x})  \psubstp{Q}{P}       
  := 
  \left\{ 
    \begin{array}{ccc} 
      Q & & x \nameeq \quotep{P} \\
      \dropn{x} & & otherwise \\
    \end{array}
  \right.
\end{mathpar}
 

where

\begin{eqnarray}
  (x)\id{\{} \lpquote Q \rpquote / \lpquote P \rpquote \id{\}}            = 
  \left\{ 
    \begin{array}{ccc}
      \lpquote Q \rpquote & & x \nameeq \lpquote P \rpquote \\
      x & & otherwise \\
    \end{array}
  \right. \nonumber
\end{eqnarray}

and $z$ is chosen distinct from $\quotep{P}$, $\quotep{Q}$, the free
names in $Q$, and all the names in $R$. Our $\alpha$-equivalence will
be built in the standard way from this substitution.

\begin{remark}\label{rem:no_self_referential_names}
  One consequence of these definitions is that $\forall P. \quotep{P}
  \not\in \freenames{P}$.
\end{remark}

\subsection{ Dynamic quote: an example }

Anticipating something of what's to come, consider applying the
substitution, $\widehat{\id{\{}u / z \id{\}}}$, to the following pair
of processes, $\lift{w}{y!(z)}$ and $w[ \lpquote y!(z) \rpquote ]$.

\begin{eqnarray}
	\lift{w}{y!(z)}\widehat{\id{\{}u / z \id{\}}}
		& = &
		\lift{w}{y!(u)} \nonumber\\
	w[ \lpquote y!(z) \rpquote ] \widehat{ \id{\{}u / z \id{\}} }
		& = &
		w[ \lpquote y!(z) \rpquote ] \nonumber
\end{eqnarray}

Because the body of the process between quotes is impervious to
substitution, we get radically different answers. In fact, by
examining the first process in an input context,
e.g. $x?(z).\lift{w}{y!(z)}$, we see that the process under the lift
operator may be shaped by prefixed inputs binding a name inside it. In
this sense, the lift operator will be seen as a way to dynamically
construct processes before reifying them as names.

Finally equipped with these standard features we can present the
dynamics of the calculus.

\subsubsection{Operational semantics} 

Finally, we introduce the computational dynamics. What marks these
algebras as distinct from other more traditionally studied algebraic
structures, e.g. vector spaces or polynomial rings, is the manner in
which dynamics is captured. In traditional structures, dynamics is typically
expressed through morphisms between such structures, as in linear maps
between vector spaces or morphisms between rings. In algebras
associated with the semantics of computation, the dynamics is
expressed as part of the algebraic structure itself, through a
reduction reduction relation typically denoted by $\red$. Below, we
give a recursive presentation of this relation for the calculus used
in the encoding.

$\red \subseteq \pi \times \pi$
$\red : \pi \to \mathcal{P}(\pi)$

\begin{mathpar}
  \inferrule* [lab=Comm] { \textsf{match}( x_{src}, x_{trgt} ) } { x_{trgt}?(y)P \; | \; x_{src}!\langle {Q} \rangle \red P\{\quotep{Q}/y}\} }
  \and \\
  \inferrule* [lab=Par] {{P} \red {P}'} {{{P} | {Q}} \red {{P}' | {Q}}}
  \and
  \inferrule* [lab=Equiv]{{{P} \scong {P}'} \andalso {{P}' \red {Q}'} \andalso {{Q}' \scong {Q}}}{{P} \red {Q}}
\end{mathpar}

\begin{eqnarray*}
  match_{\equiv} (\quotep{P},\quotep{Q}) & := & P \equiv Q \\
  match_{\dagger}(\quotep{P},\quotep{Q}) & := & \forall R. P|Q \red^{*} R => R \red^{*} 0 \\
  match_{K}(\quotep{P},\quotep{Q}) & := & K \mbox{ for some context } K
\end{eqnarray*}

$u?(x)P | u!\langle Q \rangle \red P\{\quotep{Q}/x\}$

%We write $\wred$ for $\red^*$, and $P\red$ if $\exists Q $ such that $ P \red Q$.
We write $P\red$ if $\exists Q $ such that $ P \red Q$ and $P\not\red$, otherwise.

\section{Replication}

As mentioned before, it is known that replication (and hence
recursion) can be implemented in a higher-order process algebra
\cite{SangiorgiWalker}. As our first example of calculation with the
machinery thus far presented we give the construction explicitly in
the {\rhoc}.

\begin{eqnarray}
	D_{x} & := & \prefix{x}{y}{(\binpar{\outputp{x}{y}}{@{y}})} \nonumber\\
	\bangp_{x}{P} & := & \binpar{{x}!\langle{\binpar{D_{x}}{P}}\rangle}{D_{x}} \nonumber
\end{eqnarray}

\begin{eqnarray}
	\bangp_{x}{P} & & \nonumber\\
	=
	& {x}!\langle{(\prefix{x}{y}{(\outputp{x}{y} | @{y})) | P}}\rangle 
	      | \prefix{x}{y}{(\outputp{x}{y} | @{y})} & \nonumber\\
	\red
	& (\outputp{x}{y} | @{y})\substn{\quotep{(\prefix{x}{y}{(@{y} | \outputp{x}{y})) | P}}}{y} & \nonumber\\
	=
	& \outputp{x}{\quotep{(\prefix{x}{y}{(\outputp{x}{y} | @{y})) | P}}}
	  | {(\prefix{x}{y}{(\outputp{x}{y} | @{y})) | P}} & \nonumber\\
	\red
	& \ldots & \nonumber\\
	\red^*
	& P | P | \ldots & \nonumber
\end{eqnarray}

Of course, this encoding, as an implementation, runs away, unfolding
$\bangp{P}$ eagerly. A lazier and more implementable replication
operator, restricted to input-guarded processes, may be obtained as follows.

\begin{eqnarray}
\bangp{\prefix{u}{v}{P}} 
	:= 
	\binpar{\lift{x}{\prefix{u}{v}{(\binpar{D(x)}{P})}}}{D(x)} \nonumber
\end{eqnarray}

\begin{remark}
  Note that the lazier definition still does not deal with summation
  or mixed summation (i.e. sums over input and output). The reader is
  invited to construct definitions of replication that deal with these
  features. 

  Further, the definitions are parameterized in a name, $x$. Can you,
  gentle reader, make a definition that eliminates this parameter and
  guarantees no accidental interaction between the replication
  machinery and the process being replicated -- i.e. no accidental
  sharing of names used by the process to get its work done and the
  name(s) used by the replication to effect copying. This latter
  revision of the definition of replication is crucial to obtaining
  the expected identity $!!P \sim !P$.
\end{remark}

\begin{remark}\label{rem:paradoxical_combinator}
  The reader familiar with the lambda calculus will have noticed the
  similarity between $D$ and the paradoxical combinator.

  [Ed. note: the existence of this seems to suggest we have to be more
  restrictive on the set of processes and names we admit if we are to
  support no-cloning.]
\end{remark}

\subsubsection{Bisimulation}

The computational dynamics gives rise to another kind of equivalence,
the equivalence of computational behavior. As previously mentioned
this is typically captured \emph{via} some form of bisimulation.

% The notion we use in this paper is weak barbed bisimulation
% \cite{milner91polyadicpi}.

The notion we use in this paper is derived from weak barbed
bisimulation \cite{milner91polyadicpi}. 

\begin{definition}
An \emph{observation relation}, $\downarrow_{\mathcal N}$, over a set
of names, $\mathcal N$, is the smallest relation satisfying the rules
below.

\infrule[Out-barb]{y \in {\mathcal N}, \; x \nameeq y}
		  {\outputp{x}{v} \downarrow_{\mathcal N} x}
\infrule[Par-barb]{\mbox{$P\downarrow_{\mathcal N} x$ or $Q\downarrow_{\mathcal N} x$}}
		  {\binpar{P}{Q} \downarrow_{\mathcal N} x}

We write $P \Downarrow_{\mathcal N} x$ if there is $Q$ such that 
$P \wred Q$ and $Q \downarrow_{\mathcal N} x$.
\end{definition}

\begin{definition}
%\label{def.bbisim}
An  ${\mathcal N}$-\emph{barbed bisimulation} over a set of names, ${\mathcal N}$, is a symmetric binary relation 
${\mathcal S}_{\mathcal N}$ between agents such that $P\rel{S}_{\mathcal N}Q$ implies:
\begin{enumerate}
\item If $P \red P'$ then $Q \wred Q'$ and $P'\rel{S}_{\mathcal N} Q'$.
\item If $P\downarrow_{\mathcal N} x$, then $Q\Downarrow_{\mathcal N} x$.
\end{enumerate}
$P$ is ${\mathcal N}$-barbed bisimilar to $Q$, written
$P \wbbisim_{\mathcal N} Q$, if $P \rel{S}_{\mathcal N} Q$ for some ${\mathcal N}$-barbed bisimulation ${\mathcal S}_{\mathcal N}$.
\end{definition}

$\mathcal{R} \subseteq \pi \times \pi$

$P \mathcal{R} Q => \forall P'. P \red P' \Rightarrow \exists Q'. Q \red Q', P' \mathcal{R} Q'$

$P \vdash x \Rightarrow Q \vdash x$

\begin{mathpar}
  \inferrule*[lab=Out-barb]{x \nameeq y}{{y}!\langle{Q}\rangle \vdash x}
  \and
  \inferrule*[lab=Par-barb]{\mbox{$P\vdash x$ or $Q\vdash x$}}{\binpar{P}{Q} \vdash x}
\end{mathpar}

\subsubsection{Contexts}

One of the principle advantages of computational calculi like the
$\pi$-calculus is a well-defined notion of context,
contextual-equivalence and a correlation between
contextual-equivalence and notions of bisimulation. The notion of
context allows the decomposition of a process into (sub-)process and
its syntactic environment, its context. Thus, a context may be
thought of as a process with a ``hole'' (written $\Box$) in it. The
application of a context $M$ to a process $P$, written $M[P]$, is
tantamount to filling the hole in $M$ with $P$. In this paper we do
not need the full weight of this theory, but do make use of the notion
of context in the proof the main theorem. 

\begin{mathpar}
  \inferrule* [lab=summation] {} {{M_{M},M_{N}} \bc \Box \;|\; x.M_{A} \;|\; M_{M}+M_{N}}
  \and
  \inferrule* [lab=agent] {} {{M_{A}} \bc (\vec{x})M_{P} \;| \; \clift{P_0,\ldots,M_{P},\ldots,P_N}}
  \and \\
  \inferrule* [lab=process] {} {{M_{P}} \bc M_{N} \;| \;P|M_{P} }
\end{mathpar} 

\begin{mathpar}
  \inferrule* [lab=sychronization] {} {M_{N} \bc \Box \;|\; x?M_{F} \;|\; x!M_{C}}
  \and
  \inferrule* [lab=abstraction] {} {{M_{F}} \bc (x)M_{P} }
  \and
  \inferrule* [lab=concretion] {} {{M_{C}} \bc \langle M_{P} \rangle }
  \and \\
  \inferrule* [lab=process] {} {{M_{P}} \bc M_{N} \;| \;P|M_{P} }
\end{mathpar}

\begin{definition}[contextual application] Given a context $M$, and
  process $P$, we define the \emph{contextual application}, $M[P] :=
  M\{P/\Box\}$. That is, the contextual application of M to P is the
  substitution of $P$ for $\Box$ in $M$.
\end{definition}

$\meaningof{-} : L \to \mathcal{P}(\pi)$

\begin{mathpar}
  \inferrule* [lab=collection] {} {\meaningof{true} = \pi, \and \meaningof{~E} = \pi \setminus \meaningof{E}, \and \meaningof{E_{1} \& E_{2}} = \meaningof{E_{1}} \cap \meaningof{E_{2}}}
\end{mathpar}

\begin{mathpar}
  \inferrule* [lab=structure] {} {\meaningof{0} = \{ P \in \pi | P \equiv 0 \}, \and \\ \meaningof{E_1 | E_2} = \{ P \in \pi | P \equiv P_{1} | P_{2}, P_{1} \in \meaningof{E_{1}}, P_{2} \in \meaningof{E_2}\} }
\end{mathpar}

\begin{mathpar}
 \inferrule* [lab=behavior] {} {\meaningof{\langle a?b \rangle E} = \{ P \in \pi | P \equiv Q | u?(y)P', \\ \and \\\\ \and \\ \;\;\; u \in \meaningof{a}, \forall z.P'\{z/y\} \in \meaningof{E\{z/b\}}\}, \and \\ \meaningof{a!E} = \{ P \in \pi | P \equiv Q | x!\langle P' \rangle, x \in \meaningof{a} P' \in \meaningof{E}\} }
\end{mathpar}

\begin{mathpar}
 \inferrule* [lab=nominal] {} {\meaningof{\quotep{E}} = \{ \quotep{P} \in \quotep{\pi} | P \in \meaningof{E} \}, \and \meaningof{\quotep{P}} = \{ \quotep{Q} \in \quotep{\pi} | P \equiv Q \} \and \\ \meaningof{@\quotep{E}} = \{ P \in \pi | P \equiv @x, x \in \meaningof{E} \}}
\end{mathpar}

\begin{eqnarray*}
  \\
  \meaningof{-} : TS \to ST
\end{eqnarray*}

\begin{eqnarray*}
  \\
  L : TS \to ST
\end{eqnarray*}

\begin{eqnarray*}
  \\
  P \models E \iff P \in \meaningof{E}
\end{eqnarray*}

\begin{eqnarray*}
  P \approx_{L} Q \iff \forall E \in L. P \models E \iff Q \models E
\end{eqnarray*}

\begin{eqnarray*}
  P \approx_{K} Q
\end{eqnarray*}

\begin{eqnarray*}
  P \approx Q
\end{eqnarray*}

$\approx_{K} = \approx = \approx_{L}$

\subsubsection{Contextual duality}

Note that contexts extend the quotation operation to a family of
operations from processes to names. Given a context, $M$, we can
define a \emph{nominal context}, $\quotep{M}$ by $\quotep{M}[P] :=
\quotep{M[P]}$. To foreshadow what is to come we observe that these
operations enjoy a duality with processes very much like the duality
between vectors and maps from vectors to scalars.

Further, because the calculus is essentially higher-order, we have a
correspondence between contexts and processes. More specifically,
given a name $x$ and a context $M$ we can construct $M^{*}_{x}$ such
that 

\begin{mathpar}
  M^{*}_{x} | \lift{x}{P} \red M[P]
\end{mathpar}

namely,

\begin{mathpar}
  M^{*}_{x} := x?(u).M[\dropn{u}]
\end{mathpar}

The dependence of $M^{*}_{x}$ on a name makes it an abstraction, 

\begin{mathpar}
  M^{*} := (x)x?(u).M[\dropn{u}]
\end{mathpar}

\subsection{Additional notation}

It will sometimes be convenient to denote the process a name
quotes. We already have the notation $x = \quotep{P}$, but it will be
convenient to introduce an alternate notation, $\procn{x}$, when we
want to emphasize the connection to the use of the name. Note that, by
virtue of name equivalence, $\quotep{\procn{x}} \nameeq x$; so, the
notation is consistent with previous definitions.

Further, because names have structure it is possible to effect
substitutions on the basis of that structure. This means we need to
upgrade our notation for substitutions, which we accomplish by
adapting comprehension notation. Thus,

\begin{mathpar}
  P\{ y / x : x \in S \}
\end{mathpar}

is interpreted to mean the process derived from P by replacing (in a
capture-avoiding manner) each occurrence of $x$ in $S$ by $y$. For example,

\begin{mathpar}
  P\{ \quotep{\procn{x}|\procn{x}} / x : x \in \freenames{P} \}
\end{mathpar}

will replace each (occurrence) of a free name $x$ in $P$ by
$\quotep{\procn{x}|\procn{x}}$.

Also, we will avail ourselves of the notation $x^{L}$ and $x^{R}$ to
denote injections of a name into disjoint copies of the name
space. There are numerous ways to accomplish this. One example can be
found in \cite{MeredithR05}. This notation overloads to vectors of
names: $\vec{x}^{\pi} := (x_{i}^{\pi} \; : \; 0 \leq i < |\vec{x}| )$ where $\pi \in \{L,R\}$.

We also use $P^{\Box} := P|\Box$.

In \cite{MeredithR05} an interpretation of the new operator is
given. It turns out that there are several possible interpretations
all enjoying the requisite algebraic properties of the operator (see
\cite{milner91polyadicpi}). We will therefore make liberal use of
$(\nu\; \vec{x})P$.

% subsection the_syntax_and_semantics_of_the_notation_system (end)   

\input{qm2pi.qmops} 

\input{qm2pi.sterngerlach} 

\input{qm2pi.metric} 

% section concurrent_process_calculi (end)

%\input{qm2pi.proofsketch}

% section proof sketch (end)

%\input{qm2pi.slviaknots} 

% section spatial logic via knots (end)

\input{qm2pi.conclusion}

% section conclusion (end)

%\input{qm2pi.dtcodes} 

% section wiring algorithm (end)

\input{qm2pi.ack} 

% section acknowledgments (end)

\newpage


\bibliographystyle{plain}   
\bibliography{../../biblios/main.bib}

\input{qm2pi.rhodetails}

\end{document}



% section proof sketch (end)

%\section{Unlikely characters: spatial logic for
  knots}\label{sub:characteristic_formulae} % (fold)

Associated to the mobile process calculi are a family of logics known
as the Hennessy-Milner logics. These logics typically enjoy a
semantics interpreting formulae as sets of processes that when
factored through the encoding outlined above allows an identification
of classes of knots with logical formulae. In the context of this
encoding the sub-family known as the spatial logics \cite{CairesC03}
\cite{CairesC04} \cite{Caires04} are of particular interest providing
several important features for expressing and reasoning about
properties (i.e. classes) of knots. We hint here at how this may be done.

%\begin{description}
%\item [structural connectives] 
\subsubsection{Structural connectives} The spatial logics enjoy
structural connectives corresponding, at the logical level, to the
parallel composition ($P | Q$) and new name ($(\nu \; x)P$)
connectives for processes. As illustrated in the examples below, these
connectives are extremely expressive given the shape of our encoding.
%\item [decideable satisfaction]

\subsubsection{Decideable satisfaction}
In \cite{Caires04} the satisfaction relation is shown to be decideable
for a rich class of processes. It further turns out that the image of
the our encoding is a proper subset of that class. This result
provides the basis for an algorithm by which to search for knots
enjoying a given property.
%\item [characteristic formulae]

\subsubsection{Characteristic formulae}
In the same paper \cite{Caires04} , Caires presents a means of calculating
characteristic formulae, selecting equivalence classes of processes
up to a pre--specified depth limit on the support set of names. Composed with our
encoding, this characteristic formula can be used to select
characteristic formulae for knots.
%\end{description}

\subsubsection{Spatial logic formulae}

The grammar below (segmented for comprehension) summarizes the syntax
of spatial logic formulae. We employ illustrative examples in the
sequel to provide an intuitive understanding of their meaning
referring the reader to \cite{Caires04} for a more detailed explication
of the semantics.

\begin{mathpar}
  \inferrule* [lab=boolean] {} {{A,B} \bc T \;|\; \neg A \;|\; A \wedge B \;|\; \eta = \eta'}
  \and
  \inferrule* [lab=spatial] {} {|\; \pzero \;|\; A | B \;|\; x \text{\textregistered} A \;|\; \forall x . A \;|\;  H x . A}
  \and
  \inferrule* [lab=behavioral] {} {|\; \alpha . A}
  \and 
  \inferrule* [lab=recursion] {} {|\; X(\vec{u}) \;|\; \mu X(\vec{u}) . A}
  \and
  \inferrule* [lab=action] {} {\alpha \bc \langle x?(\vec{y}) \rangle \;|\; \langle x!(\vec{y}) \rangle \;|\; \langle \tau \rangle}
  \and 
  \inferrule* [lab=name] {} {\eta \bc x \;|\; \tau}
\end{mathpar} 

% subsection characteristic_formulae (end)   	 

\subsection{Example formulae}\label{sub:example_formulae_} % (fold)

\subsubsection{Crossing as formula.}
% 
% \begin{align*}
%   \frac{d}{dx} \sin x &= \cos x 
%   & \frac{d}{dx} e^x &= e^x \\
%   \frac{d}{dx} \cos x &= - \sin x 
%   & \frac{d}{dx} \log x &= \frac{1}{x} \\
% \end{align*} 

\begin{align*}
 \mu C(x_{0},x_{1},y_{0},y_{1},u).&(\langle x_{0}?(z) \rangle(\langle u! \rangle\langle y_{1}!z \rangle C(x_{0},x_{1},y_{0},y_{1},u)) & \\
  & \wedge \langle y_{1}?(z) \rangle (\langle u! \rangle \langle x_{0}!z \rangle C(x_{0},x_{1},y_{0},y_{1},u)) & \\
  & \wedge \langle x_{1}?(z) \rangle (\langle u? \rangle \langle y_{0}!z \rangle C(x_{0},x_{1},y_{0},y_{1},u)) & \\
  & \wedge \langle y_{0}?(z) \rangle (\langle u? \rangle \langle x_{1}!z \rangle C(x_{0},x_{1},y_{0},y_{1},u))) &
\end{align*}

The lexicographical similarity between the shape of this formulae and
the shape of definition of the process representing a crossing reveals
the intuitive meaning of this formulae. It describes the capabilities
of a process that has the right to represent a crossing. For example
it picks out processes that may perform an input on the port $x_0$ in
its initial menu of capabilities. What differentiates the formula
from the process, however, is that the crossing process is the
smallest candidate to satisfy the formula. Infinitely many other
processes -- with internal behavior hidden behind this interface, so
to speak -- also satisfy this formula. Even this simple formula,
then, can be seen to open a new view onto knots, providing a
computational interpretation of \emph{virtual} knots.

Note that this formula is derived by hand. A similar formula can be
derived by employing Caires' calculation of characteristic formula
\cite{Caires04} to the process representing a crossing. In light of
this discussion, we let
$\meaningof{C}_{\phi}(x0,x1,y0,y1,u)$ denote a formula specifying the
dynamics we wish to capture of a crossing. To guarantee we preserve
the shape of the interface and minimal semantics we demand that
$\meaningof{C}_{\phi}(x0,x1,y0,y1,u) \Rightarrow
\textbf{C}(x0,x1,y0,y1,u)$ where $\textbf{C}(x0,x1,y0,y1,u)$ denotes
the formula above.
                            
\subsubsection{Crossing number constraints.}
The moral content of the context lemma (Lemma \ref{context}) is that the notion of
``locality'' in the Reidemeister moves is effectively captured by the
parallel composition operator of the process calculus. This intuition
extends through the logic. Given a formula,
$\meaningof{C}_{\phi}(x0,x1,y0,y1,u)$, we can use the structural
connectives to specify constraints on crossing numbers, such as at
least $n$ crossings, or exactly $n$ crossings.
\begin{mathpar}
  \inferrule* [lab=at-least-n] {} { K^{\geq n}_{\phi}(\vec{xs},\vec{ys}) := \Pi_{i=0}^{n-1} Hu . \meaningof{C}_{\phi}(xs_i,ys_i,u) | T }
  \and 
  \inferrule* [lab=exactly-n] {} { K^{= n}_{\phi}(\vec{xs},\vec{ys}) := \Pi_{i=0}^{n-1} Hu . \meaningof{C}_{\phi}(xs_i,ys_i,u) | \neg (\forall x_0,y_0,x_1,y_1,u . \meaningof{C}_{\phi}(x_0,y_0,x_1,y_1,u) | T) }
\end{mathpar}

To round out this section, recall that the encoding of an $n$-crossing
knot decomposes into a parallel composition of $n$ \emph{copies} of a
crossing process together with a wiring harness. To specify different
knot classes with the same crossing number amounts to specifying
logical constraints on the wiring harness. In the interest of space,
we defer examples to a forthcoming paper. Suffice it to say that both
the conditions ``alternating knot'' and ``contains the tangle
corresponding to 5/3'' are expressible. For example, it is possible to
calculate the characteristic formula of a process corresponding to the
tangle 5/3 and conjoin it into the classifying formula via the
composition connective of the logic.

Finally, we wish to observe that it is entirely within reason to
contemplate a more domain-specific version of spatial logic tailored
to the shape of processes in the image of the encoding. Such a
domain-specific logic would have a better claim to the title formal
language of knot properties.

% subsection example_formulae_ (end)

% section knots_as_processes (end) 

% section spatial logic via knots (end)

\section{Conclusions and future work}

\paragraph{Testing physical space}
You, gentle reader, may wonder why of all the theorems to be proved
given this set up we pick the one above. In some sense it's hardly
central to quantum mechanics. We see it as central in the sense that
it firmly establishes a notion of physical space arising from a notion
of the equivalence of behavior. Relating bisimulation to a metric is a
big step forward, but one is faced with interpreting the relationship
of that metric space to something more physical. Quantum mechanical
notions of ``physical'' space are still far from intuitive, but by
relating this idea of distance as testing to calculations that predict
physical circumstances we are making a not insignificant step forward
toward an understanding of the physical space we inhabit as
essentially dynamic.

\paragraph{Effectivity and simulation}
One of the observations we have yet to make is that the entire program
spelled out here is effective. We have built various interpreters for
the reflective calculus at work in this interpretation. In principle,
then, we can simulate quantum mechanics on a computer. The place where
the simulation may lose fidelity is the infinitely branching summation
for the annihilator.

In this connection i also want to point out that the evaluation style
calculation of the inner product puts the non-determinism of the
summation right at the heart of measurement. This suggests that
Milner's original reduction-based formulation of the dynamics of his
calculi in terms of sums was not just notationally suggestive of a
notion of measure-and-continue but captured some significant part of
the physics.

\paragraph{Quantum continuations}
In light of this last observation i want to point out that the
predominant account of quantum mechanics is missing a key aspect of a
truly compositional story of the physical situation. In a real lab,
when a measurement is made the observation can be made to feed into
another device that then makes another measurement conditioned on the
results of the first. This means that after the superposition was
collapsed the entire experimental set up remained in
superposition. While QM offers a means of writing this down it doesn't
quite line up well with the well-trodden formulation of computation
and continuation that we see so succinctly expressed in Milner's
calculi. This suggests that there might be advantages to this account
of dynamics waiting to be explored.

\paragraph{Quantum logic}
In this connection, we also note that by virtue of having the
Hennessy-Milner construction, we can pull the construction through the
interpretation of QM. This gives us a natural candidate for a quantum
logic that enjoys an extremely tight connection with it's domain of
interpretation, making the construction much less ad hoc (rather it is
the image of functor!).

\paragraph{Quantum probabiity}
i have questions about the basis of the interpretation of inner
product as probability amplitude. In particular, using which
axiomatization of probability theory does the notion of probability
amplitude earn the right to be so dubbed? In other words, where is the
proof that the operation for calculating a probability amplitude (and
then squaring) satisfies the axioms of what it means to calculate a
probability? Even if such a proof exists (i have yet to find it in the
literature), i wonder if it might not be possible to turn things on
their heads. Can we view the calculation of the probability amplitude
as an axiomatization of probability? If so, then the definition we
give for calculating probability amplitude may provide the basis for
an \emph{effective} theory of probability.

\paragraph{Quantum vs ``biological'' information}
Finally, i want to conclude with a more philosophical observation. At
a recent workshop in which QM was a predominant topic i noticed
something about quantum information. The speaker was giving a riveting
discussion of axiomatic QM and showing how properties of ``no
cloning'' and ``no deleting'' emerged as consequences of the
axiomatization. Theorems of this form are necessary to give us a sense
of confidence that our axioms characterize the physical theory. What
struck me, though, was that if quantum information is neither erasable
nor replicable it is markedly different from \emph{life}. Two of the
things we know about life is that

\begin{itemize}
  \item it ends;
  \item to gain some measure of persistence, to transcend it's
    finitude it is imminently copyable.
\end{itemize}

Both of these qualities are summarized succinctly in the aphorism: all
flesh is grass. For me these two kinds of ``information'' -- call them
quantum and biological -- are end points on a spectrum of strategies
for persistence. At one end, we have those curious entities that enjoy
uniqueness and permanence; at the other, we have those who in the face
of a certain end and an uncertain present make a go of passing
something on. To me one of the more remarkable aspects of the latter
strategy is that in the presence of noise (and certain features of
copying) we get a kind of dynamism, a chance for improvement against a
given persistent condition.

% subsection other_calculi_other_bisimulations_and_geometry_as_behavior (end)




% section conclusion (end)

%\documentclass[12pt]{llncs}
%\documentclass{jktr}

\usepackage[pdftex]{hyperref}                   
\usepackage {listings}
\usepackage {mathpartir}
\usepackage{bcprules}
%\usepackage{listings}
                       
\usepackage{graphicx} 
%\usepackage[margins=2.5cm,nohead,nofoot]{geometry}
%\usepackage{geometry}
\usepackage{amsfonts}
\usepackage{amstext}
\usepackage{latexsym}
\usepackage{amssymb}
\usepackage{color}


%\include{myPreamble}
\include{qm2pi.local} 

%\ifpdf
%\usepackage[pdftex]{graphicx}
%\else
%\usepackage{graphicx}
%\fi

 % \ifpdf
%  \usepackage{pdfsync}
%  \if


%\title{Brief Article}
%\author{David F. Snyder}
%\author{L.G. Meredith}

%\address{Dept. of Math., Texas State University--San Marcos, San Marcos, TX 78666}
       
\pagestyle{empty}


\begin{document}

\lstset{language=[Objective]Caml,frame=shadowbox}

\input{qm2pi.front}

% section front matter (end)

\input{qm2pi.intro} 
 
% section introduction (end)

% \input{qm2pi.knotations} 

% section notation (end)

\input{qm2pi.process.calculi} 

% section concurrent_process_calculi_and_spatial_logics_ (end)
    
%\input{qm2pi.knots2pi} 

%\input{qm2pi.trefoil} 

%\input{qm2pi.mainthm} 

% subsection basic_interpretation (end)

%\input{qm2pi.rho.presentation} 
\subsection{The syntax and semantics of the notation system}\label{sub:the_syntax_and_semantics_of_the_notation_system} % (fold)

We now summarize a technical presentation of the calculus that
embodies our theory of dynamics. The typical presentation of such a
calculus follows the style of giving generators and relations on
them. The grammar, below, describing term constructors, freely
generates the set of processes, $\Proc$. This set is then quotiented
by a relation known as structural congruence and it is over this set
that the notion of dynamics is expressed. This presentation is
essentially that of \cite{MeredithR05} with the addition of
polyadicity and summation. For readability we have relegated some of
the technical subtleties to an appendix.

\subsubsection{Process grammar}\label{subsub:process_grammar}

\begin{mathpar}
  \inferrule* [lab=synchronization] {} {{M} \bc \pzero \;|\; x?F \;|\; x!C }
  \and
  \inferrule* [lab=abstraction] {} {{F} \bc (x)P}
  \and
  \inferrule* [lab=concretion] {} {{C} \bc \langle Q \rangle}
  \and
  \inferrule* [lab=process] {} {{P,Q} \bc M \;| \;P|Q \;|\; @{x}}
  \and
  \inferrule* [lab=name] {} {{x} \bc \quotep{P}}
\end{mathpar} 

Note that $\vec{x}$ (resp. $\vec{P}$) denotes a vector of names
(resp. processes) of length $|\vec{x}|$ (resp. $|\vec{P}|$). We adopt
the following useful abbreviations.

\begin{mathpar}
   x?(\vec{y}).P := x.(\vec{y})P \and  x\clift{\vec{P}} := x.\clift{\vec{P}}
   \and x!(y) := \lift{x}{\dropn{y}}
   \and \Pi_{i=0}^{n-1}P_i := P_0 | \ldots | P_{n-1}
\end{mathpar}

\subsubsection{Structural congruence}

\paragraph{Free and bound names and alpha-equivalence.} At the
core of structural equivalence is alpha-equivalence which identifies
process that are the same up to a change of variable. Formally, we
recognize the distinction between free and bound names. The free names
of a process, $\freenames{P}$, may be calculated recursively as
follows:

\begin{mathpar}
\freenames{\pzero} := \emptyset
  \and \\
  \freenames{x?(y).P} := \{ x \} \cup (\freenames{P} \setminus \{ y \})
  \and 
  \freenames{x!\langle P \rangle} := \{ x \} \cup \{ P \} 
  \and \\
  \freenames{P|Q} := \freenames{P} \cup \freenames{Q}
  \and \\
  \freenames{@{x}} := \{ x \}
\end{mathpar}

$\pi$
$\quotep{\pi}$

$\freenames{-} : \pi \to \mathcal{P}(\quotep{\pi})$

\begin{eqnarray*}
  \freenames{\pzero} & := & \emptyset \\
  \freenames{x?(y).P} & := & \{ x \} \cup (\freenames{P} \setminus \{ y \}) \\
  \freenames{x!\langle P \rangle} & := & \{ x \} \cup \{ P \} \\
  \freenames{P|Q} & := & \freenames{P} \cup \freenames{Q} \\
  \freenames{\dropn{x}} & := & \{ x \}
\end{eqnarray*}

The bound names of a process, $\boundnames{P}$, are those names occurring in $P$
that are not free. For example, in $x?(y).0$, the name $x$ is free, while $y$ is bound.

\begin{mathpar}
  \inferrule* [lab=monoidal-laws] {} { P|Q \equiv Q|P \and P|0 \equiv P \and P|(Q|R) \equiv (P|Q)|R }
\end{mathpar}

\begin{mathpar}
  \inferrule* [lab=alpha-equivalence] {} { (x)P \equiv (y)P\{y/x\} \and y \not\in \freenames{P} }
\end{mathpar}

\begin{definition}
Then two processes, $P,Q$, are alpha-equivalent if $P = Q\{\vec{y}/\vec{x}\}$ for
some $\vec{x} \in \boundnames{Q},\vec{y} \in \boundnames{P}$, where $Q\{\vec{y}/\vec{x}\}$
denotes the capture-avoiding substitution of $\vec{y}$ for $\vec{x}$ in $Q$.
\end{definition}

\begin{definition}
  The {\em structural congruence} \cite{SangiorgiWalker} , $\equiv$,
  between processes is the least congruence containing
  alpha-equivalence, satisfying the abelian monoid laws
  (associativity, commutativity and $\pzero$ as identity) for parallel
  composition $|$ and for summation $+$.
\end{definition}

\subsection{Name equivalence}

We take name equivalence, written $\nameeq$, to be the smallest
equivalence relation generated by the following rules.

\begin{mathpar}
\inferrule*[lab=Quote-drop]
{ }
{ \quotep{@{x}} \nameeq x }

\inferrule*[lab=Struct-equiv]
{ P \scong Q }
{ \quotep{P} \nameeq \quotep{Q} }
\end{mathpar}

The astute reader will have noticed that the mutual recursion of names
and processes imposes a mutual recursion on alpha-equivalence and
structural equivalence via name-equivalence. Fortunately, all of this
works out pleasantly and we may calculate in the natural way, free of
concern. The reader interested in the details is referred to the
appendix \ref{appendix:rho_details}.

\subsection{Substitution}

We use $\Proc$ for the set of processes, $\QProc$ for the set of
names, and $\id{\{}\vec{y} / \vec{x} \id{\}}$ to denote partial maps,
$s : \QProc \rightarrow \QProc$. A map, $s$ lifts, uniquely, to a map
on process terms, $\widehat{s} : \Proc \rightarrow \Proc$ by the
following equations.

\begin{mathpar}
  (0) \psubstp{Q}{P} := 0 \\
  (R \juxtap S) \psubstp{Q}{P}
  :=    
  (R)\psubstp{Q}{P} \juxtap (S) \psubstp{Q}{P} \\
  (x?(y).R) \psubstp{Q}{P}    
  :=    
  (x)\substp{Q}{P} (z)\concat( (R \psubstn{z}{y}) \psubstp{Q}{P} ) \\
  (\lift{x}{R}) \psubstp{Q}{P}  
  :=
  \lift{(x)\substp{Q}{P}}{ R \psubstp{Q}{P} } \\
%   (\dropn{x})  \psubstp{Q}{P}       
%   := 
%   \left\{ 
%     \begin{array}{ccc} 
%       \dropn{\quotep{Q}} & & x \nameeq \quotep{P} \\
%       \dropn{x} & & otherwise \\
%     \end{array}
%   \right. 
  (\dropn{x})  \psubstp{Q}{P}       
  := 
  \left\{ 
    \begin{array}{ccc} 
      Q & & x \nameeq \quotep{P} \\
      \dropn{x} & & otherwise \\
    \end{array}
  \right.
\end{mathpar}
 

where

\begin{eqnarray}
  (x)\id{\{} \lpquote Q \rpquote / \lpquote P \rpquote \id{\}}            = 
  \left\{ 
    \begin{array}{ccc}
      \lpquote Q \rpquote & & x \nameeq \lpquote P \rpquote \\
      x & & otherwise \\
    \end{array}
  \right. \nonumber
\end{eqnarray}

and $z$ is chosen distinct from $\quotep{P}$, $\quotep{Q}$, the free
names in $Q$, and all the names in $R$. Our $\alpha$-equivalence will
be built in the standard way from this substitution.

\begin{remark}\label{rem:no_self_referential_names}
  One consequence of these definitions is that $\forall P. \quotep{P}
  \not\in \freenames{P}$.
\end{remark}

\subsection{ Dynamic quote: an example }

Anticipating something of what's to come, consider applying the
substitution, $\widehat{\id{\{}u / z \id{\}}}$, to the following pair
of processes, $\lift{w}{y!(z)}$ and $w[ \lpquote y!(z) \rpquote ]$.

\begin{eqnarray}
	\lift{w}{y!(z)}\widehat{\id{\{}u / z \id{\}}}
		& = &
		\lift{w}{y!(u)} \nonumber\\
	w[ \lpquote y!(z) \rpquote ] \widehat{ \id{\{}u / z \id{\}} }
		& = &
		w[ \lpquote y!(z) \rpquote ] \nonumber
\end{eqnarray}

Because the body of the process between quotes is impervious to
substitution, we get radically different answers. In fact, by
examining the first process in an input context,
e.g. $x?(z).\lift{w}{y!(z)}$, we see that the process under the lift
operator may be shaped by prefixed inputs binding a name inside it. In
this sense, the lift operator will be seen as a way to dynamically
construct processes before reifying them as names.

Finally equipped with these standard features we can present the
dynamics of the calculus.

\subsubsection{Operational semantics} 

Finally, we introduce the computational dynamics. What marks these
algebras as distinct from other more traditionally studied algebraic
structures, e.g. vector spaces or polynomial rings, is the manner in
which dynamics is captured. In traditional structures, dynamics is typically
expressed through morphisms between such structures, as in linear maps
between vector spaces or morphisms between rings. In algebras
associated with the semantics of computation, the dynamics is
expressed as part of the algebraic structure itself, through a
reduction reduction relation typically denoted by $\red$. Below, we
give a recursive presentation of this relation for the calculus used
in the encoding.

$\red \subseteq \pi \times \pi$
$\red : \pi \to \mathcal{P}(\pi)$

\begin{mathpar}
  \inferrule* [lab=Comm] { \textsf{match}( x_{src}, x_{trgt} ) } { x_{trgt}?(y)P \; | \; x_{src}!\langle {Q} \rangle \red P\{\quotep{Q}/y}\} }
  \and \\
  \inferrule* [lab=Par] {{P} \red {P}'} {{{P} | {Q}} \red {{P}' | {Q}}}
  \and
  \inferrule* [lab=Equiv]{{{P} \scong {P}'} \andalso {{P}' \red {Q}'} \andalso {{Q}' \scong {Q}}}{{P} \red {Q}}
\end{mathpar}

\begin{eqnarray*}
  match_{\equiv} (\quotep{P},\quotep{Q}) & := & P \equiv Q \\
  match_{\dagger}(\quotep{P},\quotep{Q}) & := & \forall R. P|Q \red^{*} R => R \red^{*} 0 \\
  match_{K}(\quotep{P},\quotep{Q}) & := & K \mbox{ for some context } K
\end{eqnarray*}

$u?(x)P | u!\langle Q \rangle \red P\{\quotep{Q}/x\}$

%We write $\wred$ for $\red^*$, and $P\red$ if $\exists Q $ such that $ P \red Q$.
We write $P\red$ if $\exists Q $ such that $ P \red Q$ and $P\not\red$, otherwise.

\section{Replication}

As mentioned before, it is known that replication (and hence
recursion) can be implemented in a higher-order process algebra
\cite{SangiorgiWalker}. As our first example of calculation with the
machinery thus far presented we give the construction explicitly in
the {\rhoc}.

\begin{eqnarray}
	D_{x} & := & \prefix{x}{y}{(\binpar{\outputp{x}{y}}{@{y}})} \nonumber\\
	\bangp_{x}{P} & := & \binpar{{x}!\langle{\binpar{D_{x}}{P}}\rangle}{D_{x}} \nonumber
\end{eqnarray}

\begin{eqnarray}
	\bangp_{x}{P} & & \nonumber\\
	=
	& {x}!\langle{(\prefix{x}{y}{(\outputp{x}{y} | @{y})) | P}}\rangle 
	      | \prefix{x}{y}{(\outputp{x}{y} | @{y})} & \nonumber\\
	\red
	& (\outputp{x}{y} | @{y})\substn{\quotep{(\prefix{x}{y}{(@{y} | \outputp{x}{y})) | P}}}{y} & \nonumber\\
	=
	& \outputp{x}{\quotep{(\prefix{x}{y}{(\outputp{x}{y} | @{y})) | P}}}
	  | {(\prefix{x}{y}{(\outputp{x}{y} | @{y})) | P}} & \nonumber\\
	\red
	& \ldots & \nonumber\\
	\red^*
	& P | P | \ldots & \nonumber
\end{eqnarray}

Of course, this encoding, as an implementation, runs away, unfolding
$\bangp{P}$ eagerly. A lazier and more implementable replication
operator, restricted to input-guarded processes, may be obtained as follows.

\begin{eqnarray}
\bangp{\prefix{u}{v}{P}} 
	:= 
	\binpar{\lift{x}{\prefix{u}{v}{(\binpar{D(x)}{P})}}}{D(x)} \nonumber
\end{eqnarray}

\begin{remark}
  Note that the lazier definition still does not deal with summation
  or mixed summation (i.e. sums over input and output). The reader is
  invited to construct definitions of replication that deal with these
  features. 

  Further, the definitions are parameterized in a name, $x$. Can you,
  gentle reader, make a definition that eliminates this parameter and
  guarantees no accidental interaction between the replication
  machinery and the process being replicated -- i.e. no accidental
  sharing of names used by the process to get its work done and the
  name(s) used by the replication to effect copying. This latter
  revision of the definition of replication is crucial to obtaining
  the expected identity $!!P \sim !P$.
\end{remark}

\begin{remark}\label{rem:paradoxical_combinator}
  The reader familiar with the lambda calculus will have noticed the
  similarity between $D$ and the paradoxical combinator.

  [Ed. note: the existence of this seems to suggest we have to be more
  restrictive on the set of processes and names we admit if we are to
  support no-cloning.]
\end{remark}

\subsubsection{Bisimulation}

The computational dynamics gives rise to another kind of equivalence,
the equivalence of computational behavior. As previously mentioned
this is typically captured \emph{via} some form of bisimulation.

% The notion we use in this paper is weak barbed bisimulation
% \cite{milner91polyadicpi}.

The notion we use in this paper is derived from weak barbed
bisimulation \cite{milner91polyadicpi}. 

\begin{definition}
An \emph{observation relation}, $\downarrow_{\mathcal N}$, over a set
of names, $\mathcal N$, is the smallest relation satisfying the rules
below.

\infrule[Out-barb]{y \in {\mathcal N}, \; x \nameeq y}
		  {\outputp{x}{v} \downarrow_{\mathcal N} x}
\infrule[Par-barb]{\mbox{$P\downarrow_{\mathcal N} x$ or $Q\downarrow_{\mathcal N} x$}}
		  {\binpar{P}{Q} \downarrow_{\mathcal N} x}

We write $P \Downarrow_{\mathcal N} x$ if there is $Q$ such that 
$P \wred Q$ and $Q \downarrow_{\mathcal N} x$.
\end{definition}

\begin{definition}
%\label{def.bbisim}
An  ${\mathcal N}$-\emph{barbed bisimulation} over a set of names, ${\mathcal N}$, is a symmetric binary relation 
${\mathcal S}_{\mathcal N}$ between agents such that $P\rel{S}_{\mathcal N}Q$ implies:
\begin{enumerate}
\item If $P \red P'$ then $Q \wred Q'$ and $P'\rel{S}_{\mathcal N} Q'$.
\item If $P\downarrow_{\mathcal N} x$, then $Q\Downarrow_{\mathcal N} x$.
\end{enumerate}
$P$ is ${\mathcal N}$-barbed bisimilar to $Q$, written
$P \wbbisim_{\mathcal N} Q$, if $P \rel{S}_{\mathcal N} Q$ for some ${\mathcal N}$-barbed bisimulation ${\mathcal S}_{\mathcal N}$.
\end{definition}

$\mathcal{R} \subseteq \pi \times \pi$

$P \mathcal{R} Q => \forall P'. P \red P' \Rightarrow \exists Q'. Q \red Q', P' \mathcal{R} Q'$

$P \vdash x \Rightarrow Q \vdash x$

\begin{mathpar}
  \inferrule*[lab=Out-barb]{x \nameeq y}{{y}!\langle{Q}\rangle \vdash x}
  \and
  \inferrule*[lab=Par-barb]{\mbox{$P\vdash x$ or $Q\vdash x$}}{\binpar{P}{Q} \vdash x}
\end{mathpar}

\subsubsection{Contexts}

One of the principle advantages of computational calculi like the
$\pi$-calculus is a well-defined notion of context,
contextual-equivalence and a correlation between
contextual-equivalence and notions of bisimulation. The notion of
context allows the decomposition of a process into (sub-)process and
its syntactic environment, its context. Thus, a context may be
thought of as a process with a ``hole'' (written $\Box$) in it. The
application of a context $M$ to a process $P$, written $M[P]$, is
tantamount to filling the hole in $M$ with $P$. In this paper we do
not need the full weight of this theory, but do make use of the notion
of context in the proof the main theorem. 

\begin{mathpar}
  \inferrule* [lab=summation] {} {{M_{M},M_{N}} \bc \Box \;|\; x.M_{A} \;|\; M_{M}+M_{N}}
  \and
  \inferrule* [lab=agent] {} {{M_{A}} \bc (\vec{x})M_{P} \;| \; \clift{P_0,\ldots,M_{P},\ldots,P_N}}
  \and \\
  \inferrule* [lab=process] {} {{M_{P}} \bc M_{N} \;| \;P|M_{P} }
\end{mathpar} 

\begin{mathpar}
  \inferrule* [lab=sychronization] {} {M_{N} \bc \Box \;|\; x?M_{F} \;|\; x!M_{C}}
  \and
  \inferrule* [lab=abstraction] {} {{M_{F}} \bc (x)M_{P} }
  \and
  \inferrule* [lab=concretion] {} {{M_{C}} \bc \langle M_{P} \rangle }
  \and \\
  \inferrule* [lab=process] {} {{M_{P}} \bc M_{N} \;| \;P|M_{P} }
\end{mathpar}

\begin{definition}[contextual application] Given a context $M$, and
  process $P$, we define the \emph{contextual application}, $M[P] :=
  M\{P/\Box\}$. That is, the contextual application of M to P is the
  substitution of $P$ for $\Box$ in $M$.
\end{definition}

$\meaningof{-} : L \to \mathcal{P}(\pi)$

\begin{mathpar}
  \inferrule* [lab=collection] {} {\meaningof{true} = \pi, \and \meaningof{~E} = \pi \setminus \meaningof{E}, \and \meaningof{E_{1} \& E_{2}} = \meaningof{E_{1}} \cap \meaningof{E_{2}}}
\end{mathpar}

\begin{mathpar}
  \inferrule* [lab=structure] {} {\meaningof{0} = \{ P \in \pi | P \equiv 0 \}, \and \\ \meaningof{E_1 | E_2} = \{ P \in \pi | P \equiv P_{1} | P_{2}, P_{1} \in \meaningof{E_{1}}, P_{2} \in \meaningof{E_2}\} }
\end{mathpar}

\begin{mathpar}
 \inferrule* [lab=behavior] {} {\meaningof{\langle a?b \rangle E} = \{ P \in \pi | P \equiv Q | u?(y)P', \\ \and \\\\ \and \\ \;\;\; u \in \meaningof{a}, \forall z.P'\{z/y\} \in \meaningof{E\{z/b\}}\}, \and \\ \meaningof{a!E} = \{ P \in \pi | P \equiv Q | x!\langle P' \rangle, x \in \meaningof{a} P' \in \meaningof{E}\} }
\end{mathpar}

\begin{mathpar}
 \inferrule* [lab=nominal] {} {\meaningof{\quotep{E}} = \{ \quotep{P} \in \quotep{\pi} | P \in \meaningof{E} \}, \and \meaningof{\quotep{P}} = \{ \quotep{Q} \in \quotep{\pi} | P \equiv Q \} \and \\ \meaningof{@\quotep{E}} = \{ P \in \pi | P \equiv @x, x \in \meaningof{E} \}}
\end{mathpar}

\begin{eqnarray*}
  \\
  \meaningof{-} : TS \to ST
\end{eqnarray*}

\begin{eqnarray*}
  \\
  L : TS \to ST
\end{eqnarray*}

\begin{eqnarray*}
  \\
  P \models E \iff P \in \meaningof{E}
\end{eqnarray*}

\begin{eqnarray*}
  P \approx_{L} Q \iff \forall E \in L. P \models E \iff Q \models E
\end{eqnarray*}

\begin{eqnarray*}
  P \approx_{K} Q
\end{eqnarray*}

\begin{eqnarray*}
  P \approx Q
\end{eqnarray*}

$\approx_{K} = \approx = \approx_{L}$

\subsubsection{Contextual duality}

Note that contexts extend the quotation operation to a family of
operations from processes to names. Given a context, $M$, we can
define a \emph{nominal context}, $\quotep{M}$ by $\quotep{M}[P] :=
\quotep{M[P]}$. To foreshadow what is to come we observe that these
operations enjoy a duality with processes very much like the duality
between vectors and maps from vectors to scalars.

Further, because the calculus is essentially higher-order, we have a
correspondence between contexts and processes. More specifically,
given a name $x$ and a context $M$ we can construct $M^{*}_{x}$ such
that 

\begin{mathpar}
  M^{*}_{x} | \lift{x}{P} \red M[P]
\end{mathpar}

namely,

\begin{mathpar}
  M^{*}_{x} := x?(u).M[\dropn{u}]
\end{mathpar}

The dependence of $M^{*}_{x}$ on a name makes it an abstraction, 

\begin{mathpar}
  M^{*} := (x)x?(u).M[\dropn{u}]
\end{mathpar}

\subsection{Additional notation}

It will sometimes be convenient to denote the process a name
quotes. We already have the notation $x = \quotep{P}$, but it will be
convenient to introduce an alternate notation, $\procn{x}$, when we
want to emphasize the connection to the use of the name. Note that, by
virtue of name equivalence, $\quotep{\procn{x}} \nameeq x$; so, the
notation is consistent with previous definitions.

Further, because names have structure it is possible to effect
substitutions on the basis of that structure. This means we need to
upgrade our notation for substitutions, which we accomplish by
adapting comprehension notation. Thus,

\begin{mathpar}
  P\{ y / x : x \in S \}
\end{mathpar}

is interpreted to mean the process derived from P by replacing (in a
capture-avoiding manner) each occurrence of $x$ in $S$ by $y$. For example,

\begin{mathpar}
  P\{ \quotep{\procn{x}|\procn{x}} / x : x \in \freenames{P} \}
\end{mathpar}

will replace each (occurrence) of a free name $x$ in $P$ by
$\quotep{\procn{x}|\procn{x}}$.

Also, we will avail ourselves of the notation $x^{L}$ and $x^{R}$ to
denote injections of a name into disjoint copies of the name
space. There are numerous ways to accomplish this. One example can be
found in \cite{MeredithR05}. This notation overloads to vectors of
names: $\vec{x}^{\pi} := (x_{i}^{\pi} \; : \; 0 \leq i < |\vec{x}| )$ where $\pi \in \{L,R\}$.

We also use $P^{\Box} := P|\Box$.

In \cite{MeredithR05} an interpretation of the new operator is
given. It turns out that there are several possible interpretations
all enjoying the requisite algebraic properties of the operator (see
\cite{milner91polyadicpi}). We will therefore make liberal use of
$(\nu\; \vec{x})P$.

% subsection the_syntax_and_semantics_of_the_notation_system (end)   

\input{qm2pi.qmops} 

\input{qm2pi.sterngerlach} 

\input{qm2pi.metric} 

% section concurrent_process_calculi (end)

%\input{qm2pi.proofsketch}

% section proof sketch (end)

%\input{qm2pi.slviaknots} 

% section spatial logic via knots (end)

\input{qm2pi.conclusion}

% section conclusion (end)

%\input{qm2pi.dtcodes} 

% section wiring algorithm (end)

\input{qm2pi.ack} 

% section acknowledgments (end)

\newpage


\bibliographystyle{plain}   
\bibliography{../../biblios/main.bib}

\input{qm2pi.rhodetails}

\end{document}

 

% section wiring algorithm (end)

\documentclass[12pt]{llncs}
%\documentclass{jktr}

\usepackage[pdftex]{hyperref}                   
\usepackage {listings}
\usepackage {mathpartir}
\usepackage{bcprules}
%\usepackage{listings}
                       
\usepackage{graphicx} 
%\usepackage[margins=2.5cm,nohead,nofoot]{geometry}
%\usepackage{geometry}
\usepackage{amsfonts}
\usepackage{amstext}
\usepackage{latexsym}
\usepackage{amssymb}
\usepackage{color}


%\include{myPreamble}
\include{qm2pi.local} 

%\ifpdf
%\usepackage[pdftex]{graphicx}
%\else
%\usepackage{graphicx}
%\fi

 % \ifpdf
%  \usepackage{pdfsync}
%  \if


%\title{Brief Article}
%\author{David F. Snyder}
%\author{L.G. Meredith}

%\address{Dept. of Math., Texas State University--San Marcos, San Marcos, TX 78666}
       
\pagestyle{empty}


\begin{document}

\lstset{language=[Objective]Caml,frame=shadowbox}

\input{qm2pi.front}

% section front matter (end)

\input{qm2pi.intro} 
 
% section introduction (end)

% \input{qm2pi.knotations} 

% section notation (end)

\input{qm2pi.process.calculi} 

% section concurrent_process_calculi_and_spatial_logics_ (end)
    
%\input{qm2pi.knots2pi} 

%\input{qm2pi.trefoil} 

%\input{qm2pi.mainthm} 

% subsection basic_interpretation (end)

%\input{qm2pi.rho.presentation} 
\subsection{The syntax and semantics of the notation system}\label{sub:the_syntax_and_semantics_of_the_notation_system} % (fold)

We now summarize a technical presentation of the calculus that
embodies our theory of dynamics. The typical presentation of such a
calculus follows the style of giving generators and relations on
them. The grammar, below, describing term constructors, freely
generates the set of processes, $\Proc$. This set is then quotiented
by a relation known as structural congruence and it is over this set
that the notion of dynamics is expressed. This presentation is
essentially that of \cite{MeredithR05} with the addition of
polyadicity and summation. For readability we have relegated some of
the technical subtleties to an appendix.

\subsubsection{Process grammar}\label{subsub:process_grammar}

\begin{mathpar}
  \inferrule* [lab=synchronization] {} {{M} \bc \pzero \;|\; x?F \;|\; x!C }
  \and
  \inferrule* [lab=abstraction] {} {{F} \bc (x)P}
  \and
  \inferrule* [lab=concretion] {} {{C} \bc \langle Q \rangle}
  \and
  \inferrule* [lab=process] {} {{P,Q} \bc M \;| \;P|Q \;|\; @{x}}
  \and
  \inferrule* [lab=name] {} {{x} \bc \quotep{P}}
\end{mathpar} 

Note that $\vec{x}$ (resp. $\vec{P}$) denotes a vector of names
(resp. processes) of length $|\vec{x}|$ (resp. $|\vec{P}|$). We adopt
the following useful abbreviations.

\begin{mathpar}
   x?(\vec{y}).P := x.(\vec{y})P \and  x\clift{\vec{P}} := x.\clift{\vec{P}}
   \and x!(y) := \lift{x}{\dropn{y}}
   \and \Pi_{i=0}^{n-1}P_i := P_0 | \ldots | P_{n-1}
\end{mathpar}

\subsubsection{Structural congruence}

\paragraph{Free and bound names and alpha-equivalence.} At the
core of structural equivalence is alpha-equivalence which identifies
process that are the same up to a change of variable. Formally, we
recognize the distinction between free and bound names. The free names
of a process, $\freenames{P}$, may be calculated recursively as
follows:

\begin{mathpar}
\freenames{\pzero} := \emptyset
  \and \\
  \freenames{x?(y).P} := \{ x \} \cup (\freenames{P} \setminus \{ y \})
  \and 
  \freenames{x!\langle P \rangle} := \{ x \} \cup \{ P \} 
  \and \\
  \freenames{P|Q} := \freenames{P} \cup \freenames{Q}
  \and \\
  \freenames{@{x}} := \{ x \}
\end{mathpar}

$\pi$
$\quotep{\pi}$

$\freenames{-} : \pi \to \mathcal{P}(\quotep{\pi})$

\begin{eqnarray*}
  \freenames{\pzero} & := & \emptyset \\
  \freenames{x?(y).P} & := & \{ x \} \cup (\freenames{P} \setminus \{ y \}) \\
  \freenames{x!\langle P \rangle} & := & \{ x \} \cup \{ P \} \\
  \freenames{P|Q} & := & \freenames{P} \cup \freenames{Q} \\
  \freenames{\dropn{x}} & := & \{ x \}
\end{eqnarray*}

The bound names of a process, $\boundnames{P}$, are those names occurring in $P$
that are not free. For example, in $x?(y).0$, the name $x$ is free, while $y$ is bound.

\begin{mathpar}
  \inferrule* [lab=monoidal-laws] {} { P|Q \equiv Q|P \and P|0 \equiv P \and P|(Q|R) \equiv (P|Q)|R }
\end{mathpar}

\begin{mathpar}
  \inferrule* [lab=alpha-equivalence] {} { (x)P \equiv (y)P\{y/x\} \and y \not\in \freenames{P} }
\end{mathpar}

\begin{definition}
Then two processes, $P,Q$, are alpha-equivalent if $P = Q\{\vec{y}/\vec{x}\}$ for
some $\vec{x} \in \boundnames{Q},\vec{y} \in \boundnames{P}$, where $Q\{\vec{y}/\vec{x}\}$
denotes the capture-avoiding substitution of $\vec{y}$ for $\vec{x}$ in $Q$.
\end{definition}

\begin{definition}
  The {\em structural congruence} \cite{SangiorgiWalker} , $\equiv$,
  between processes is the least congruence containing
  alpha-equivalence, satisfying the abelian monoid laws
  (associativity, commutativity and $\pzero$ as identity) for parallel
  composition $|$ and for summation $+$.
\end{definition}

\subsection{Name equivalence}

We take name equivalence, written $\nameeq$, to be the smallest
equivalence relation generated by the following rules.

\begin{mathpar}
\inferrule*[lab=Quote-drop]
{ }
{ \quotep{@{x}} \nameeq x }

\inferrule*[lab=Struct-equiv]
{ P \scong Q }
{ \quotep{P} \nameeq \quotep{Q} }
\end{mathpar}

The astute reader will have noticed that the mutual recursion of names
and processes imposes a mutual recursion on alpha-equivalence and
structural equivalence via name-equivalence. Fortunately, all of this
works out pleasantly and we may calculate in the natural way, free of
concern. The reader interested in the details is referred to the
appendix \ref{appendix:rho_details}.

\subsection{Substitution}

We use $\Proc$ for the set of processes, $\QProc$ for the set of
names, and $\id{\{}\vec{y} / \vec{x} \id{\}}$ to denote partial maps,
$s : \QProc \rightarrow \QProc$. A map, $s$ lifts, uniquely, to a map
on process terms, $\widehat{s} : \Proc \rightarrow \Proc$ by the
following equations.

\begin{mathpar}
  (0) \psubstp{Q}{P} := 0 \\
  (R \juxtap S) \psubstp{Q}{P}
  :=    
  (R)\psubstp{Q}{P} \juxtap (S) \psubstp{Q}{P} \\
  (x?(y).R) \psubstp{Q}{P}    
  :=    
  (x)\substp{Q}{P} (z)\concat( (R \psubstn{z}{y}) \psubstp{Q}{P} ) \\
  (\lift{x}{R}) \psubstp{Q}{P}  
  :=
  \lift{(x)\substp{Q}{P}}{ R \psubstp{Q}{P} } \\
%   (\dropn{x})  \psubstp{Q}{P}       
%   := 
%   \left\{ 
%     \begin{array}{ccc} 
%       \dropn{\quotep{Q}} & & x \nameeq \quotep{P} \\
%       \dropn{x} & & otherwise \\
%     \end{array}
%   \right. 
  (\dropn{x})  \psubstp{Q}{P}       
  := 
  \left\{ 
    \begin{array}{ccc} 
      Q & & x \nameeq \quotep{P} \\
      \dropn{x} & & otherwise \\
    \end{array}
  \right.
\end{mathpar}
 

where

\begin{eqnarray}
  (x)\id{\{} \lpquote Q \rpquote / \lpquote P \rpquote \id{\}}            = 
  \left\{ 
    \begin{array}{ccc}
      \lpquote Q \rpquote & & x \nameeq \lpquote P \rpquote \\
      x & & otherwise \\
    \end{array}
  \right. \nonumber
\end{eqnarray}

and $z$ is chosen distinct from $\quotep{P}$, $\quotep{Q}$, the free
names in $Q$, and all the names in $R$. Our $\alpha$-equivalence will
be built in the standard way from this substitution.

\begin{remark}\label{rem:no_self_referential_names}
  One consequence of these definitions is that $\forall P. \quotep{P}
  \not\in \freenames{P}$.
\end{remark}

\subsection{ Dynamic quote: an example }

Anticipating something of what's to come, consider applying the
substitution, $\widehat{\id{\{}u / z \id{\}}}$, to the following pair
of processes, $\lift{w}{y!(z)}$ and $w[ \lpquote y!(z) \rpquote ]$.

\begin{eqnarray}
	\lift{w}{y!(z)}\widehat{\id{\{}u / z \id{\}}}
		& = &
		\lift{w}{y!(u)} \nonumber\\
	w[ \lpquote y!(z) \rpquote ] \widehat{ \id{\{}u / z \id{\}} }
		& = &
		w[ \lpquote y!(z) \rpquote ] \nonumber
\end{eqnarray}

Because the body of the process between quotes is impervious to
substitution, we get radically different answers. In fact, by
examining the first process in an input context,
e.g. $x?(z).\lift{w}{y!(z)}$, we see that the process under the lift
operator may be shaped by prefixed inputs binding a name inside it. In
this sense, the lift operator will be seen as a way to dynamically
construct processes before reifying them as names.

Finally equipped with these standard features we can present the
dynamics of the calculus.

\subsubsection{Operational semantics} 

Finally, we introduce the computational dynamics. What marks these
algebras as distinct from other more traditionally studied algebraic
structures, e.g. vector spaces or polynomial rings, is the manner in
which dynamics is captured. In traditional structures, dynamics is typically
expressed through morphisms between such structures, as in linear maps
between vector spaces or morphisms between rings. In algebras
associated with the semantics of computation, the dynamics is
expressed as part of the algebraic structure itself, through a
reduction reduction relation typically denoted by $\red$. Below, we
give a recursive presentation of this relation for the calculus used
in the encoding.

$\red \subseteq \pi \times \pi$
$\red : \pi \to \mathcal{P}(\pi)$

\begin{mathpar}
  \inferrule* [lab=Comm] { \textsf{match}( x_{src}, x_{trgt} ) } { x_{trgt}?(y)P \; | \; x_{src}!\langle {Q} \rangle \red P\{\quotep{Q}/y}\} }
  \and \\
  \inferrule* [lab=Par] {{P} \red {P}'} {{{P} | {Q}} \red {{P}' | {Q}}}
  \and
  \inferrule* [lab=Equiv]{{{P} \scong {P}'} \andalso {{P}' \red {Q}'} \andalso {{Q}' \scong {Q}}}{{P} \red {Q}}
\end{mathpar}

\begin{eqnarray*}
  match_{\equiv} (\quotep{P},\quotep{Q}) & := & P \equiv Q \\
  match_{\dagger}(\quotep{P},\quotep{Q}) & := & \forall R. P|Q \red^{*} R => R \red^{*} 0 \\
  match_{K}(\quotep{P},\quotep{Q}) & := & K \mbox{ for some context } K
\end{eqnarray*}

$u?(x)P | u!\langle Q \rangle \red P\{\quotep{Q}/x\}$

%We write $\wred$ for $\red^*$, and $P\red$ if $\exists Q $ such that $ P \red Q$.
We write $P\red$ if $\exists Q $ such that $ P \red Q$ and $P\not\red$, otherwise.

\section{Replication}

As mentioned before, it is known that replication (and hence
recursion) can be implemented in a higher-order process algebra
\cite{SangiorgiWalker}. As our first example of calculation with the
machinery thus far presented we give the construction explicitly in
the {\rhoc}.

\begin{eqnarray}
	D_{x} & := & \prefix{x}{y}{(\binpar{\outputp{x}{y}}{@{y}})} \nonumber\\
	\bangp_{x}{P} & := & \binpar{{x}!\langle{\binpar{D_{x}}{P}}\rangle}{D_{x}} \nonumber
\end{eqnarray}

\begin{eqnarray}
	\bangp_{x}{P} & & \nonumber\\
	=
	& {x}!\langle{(\prefix{x}{y}{(\outputp{x}{y} | @{y})) | P}}\rangle 
	      | \prefix{x}{y}{(\outputp{x}{y} | @{y})} & \nonumber\\
	\red
	& (\outputp{x}{y} | @{y})\substn{\quotep{(\prefix{x}{y}{(@{y} | \outputp{x}{y})) | P}}}{y} & \nonumber\\
	=
	& \outputp{x}{\quotep{(\prefix{x}{y}{(\outputp{x}{y} | @{y})) | P}}}
	  | {(\prefix{x}{y}{(\outputp{x}{y} | @{y})) | P}} & \nonumber\\
	\red
	& \ldots & \nonumber\\
	\red^*
	& P | P | \ldots & \nonumber
\end{eqnarray}

Of course, this encoding, as an implementation, runs away, unfolding
$\bangp{P}$ eagerly. A lazier and more implementable replication
operator, restricted to input-guarded processes, may be obtained as follows.

\begin{eqnarray}
\bangp{\prefix{u}{v}{P}} 
	:= 
	\binpar{\lift{x}{\prefix{u}{v}{(\binpar{D(x)}{P})}}}{D(x)} \nonumber
\end{eqnarray}

\begin{remark}
  Note that the lazier definition still does not deal with summation
  or mixed summation (i.e. sums over input and output). The reader is
  invited to construct definitions of replication that deal with these
  features. 

  Further, the definitions are parameterized in a name, $x$. Can you,
  gentle reader, make a definition that eliminates this parameter and
  guarantees no accidental interaction between the replication
  machinery and the process being replicated -- i.e. no accidental
  sharing of names used by the process to get its work done and the
  name(s) used by the replication to effect copying. This latter
  revision of the definition of replication is crucial to obtaining
  the expected identity $!!P \sim !P$.
\end{remark}

\begin{remark}\label{rem:paradoxical_combinator}
  The reader familiar with the lambda calculus will have noticed the
  similarity between $D$ and the paradoxical combinator.

  [Ed. note: the existence of this seems to suggest we have to be more
  restrictive on the set of processes and names we admit if we are to
  support no-cloning.]
\end{remark}

\subsubsection{Bisimulation}

The computational dynamics gives rise to another kind of equivalence,
the equivalence of computational behavior. As previously mentioned
this is typically captured \emph{via} some form of bisimulation.

% The notion we use in this paper is weak barbed bisimulation
% \cite{milner91polyadicpi}.

The notion we use in this paper is derived from weak barbed
bisimulation \cite{milner91polyadicpi}. 

\begin{definition}
An \emph{observation relation}, $\downarrow_{\mathcal N}$, over a set
of names, $\mathcal N$, is the smallest relation satisfying the rules
below.

\infrule[Out-barb]{y \in {\mathcal N}, \; x \nameeq y}
		  {\outputp{x}{v} \downarrow_{\mathcal N} x}
\infrule[Par-barb]{\mbox{$P\downarrow_{\mathcal N} x$ or $Q\downarrow_{\mathcal N} x$}}
		  {\binpar{P}{Q} \downarrow_{\mathcal N} x}

We write $P \Downarrow_{\mathcal N} x$ if there is $Q$ such that 
$P \wred Q$ and $Q \downarrow_{\mathcal N} x$.
\end{definition}

\begin{definition}
%\label{def.bbisim}
An  ${\mathcal N}$-\emph{barbed bisimulation} over a set of names, ${\mathcal N}$, is a symmetric binary relation 
${\mathcal S}_{\mathcal N}$ between agents such that $P\rel{S}_{\mathcal N}Q$ implies:
\begin{enumerate}
\item If $P \red P'$ then $Q \wred Q'$ and $P'\rel{S}_{\mathcal N} Q'$.
\item If $P\downarrow_{\mathcal N} x$, then $Q\Downarrow_{\mathcal N} x$.
\end{enumerate}
$P$ is ${\mathcal N}$-barbed bisimilar to $Q$, written
$P \wbbisim_{\mathcal N} Q$, if $P \rel{S}_{\mathcal N} Q$ for some ${\mathcal N}$-barbed bisimulation ${\mathcal S}_{\mathcal N}$.
\end{definition}

$\mathcal{R} \subseteq \pi \times \pi$

$P \mathcal{R} Q => \forall P'. P \red P' \Rightarrow \exists Q'. Q \red Q', P' \mathcal{R} Q'$

$P \vdash x \Rightarrow Q \vdash x$

\begin{mathpar}
  \inferrule*[lab=Out-barb]{x \nameeq y}{{y}!\langle{Q}\rangle \vdash x}
  \and
  \inferrule*[lab=Par-barb]{\mbox{$P\vdash x$ or $Q\vdash x$}}{\binpar{P}{Q} \vdash x}
\end{mathpar}

\subsubsection{Contexts}

One of the principle advantages of computational calculi like the
$\pi$-calculus is a well-defined notion of context,
contextual-equivalence and a correlation between
contextual-equivalence and notions of bisimulation. The notion of
context allows the decomposition of a process into (sub-)process and
its syntactic environment, its context. Thus, a context may be
thought of as a process with a ``hole'' (written $\Box$) in it. The
application of a context $M$ to a process $P$, written $M[P]$, is
tantamount to filling the hole in $M$ with $P$. In this paper we do
not need the full weight of this theory, but do make use of the notion
of context in the proof the main theorem. 

\begin{mathpar}
  \inferrule* [lab=summation] {} {{M_{M},M_{N}} \bc \Box \;|\; x.M_{A} \;|\; M_{M}+M_{N}}
  \and
  \inferrule* [lab=agent] {} {{M_{A}} \bc (\vec{x})M_{P} \;| \; \clift{P_0,\ldots,M_{P},\ldots,P_N}}
  \and \\
  \inferrule* [lab=process] {} {{M_{P}} \bc M_{N} \;| \;P|M_{P} }
\end{mathpar} 

\begin{mathpar}
  \inferrule* [lab=sychronization] {} {M_{N} \bc \Box \;|\; x?M_{F} \;|\; x!M_{C}}
  \and
  \inferrule* [lab=abstraction] {} {{M_{F}} \bc (x)M_{P} }
  \and
  \inferrule* [lab=concretion] {} {{M_{C}} \bc \langle M_{P} \rangle }
  \and \\
  \inferrule* [lab=process] {} {{M_{P}} \bc M_{N} \;| \;P|M_{P} }
\end{mathpar}

\begin{definition}[contextual application] Given a context $M$, and
  process $P$, we define the \emph{contextual application}, $M[P] :=
  M\{P/\Box\}$. That is, the contextual application of M to P is the
  substitution of $P$ for $\Box$ in $M$.
\end{definition}

$\meaningof{-} : L \to \mathcal{P}(\pi)$

\begin{mathpar}
  \inferrule* [lab=collection] {} {\meaningof{true} = \pi, \and \meaningof{~E} = \pi \setminus \meaningof{E}, \and \meaningof{E_{1} \& E_{2}} = \meaningof{E_{1}} \cap \meaningof{E_{2}}}
\end{mathpar}

\begin{mathpar}
  \inferrule* [lab=structure] {} {\meaningof{0} = \{ P \in \pi | P \equiv 0 \}, \and \\ \meaningof{E_1 | E_2} = \{ P \in \pi | P \equiv P_{1} | P_{2}, P_{1} \in \meaningof{E_{1}}, P_{2} \in \meaningof{E_2}\} }
\end{mathpar}

\begin{mathpar}
 \inferrule* [lab=behavior] {} {\meaningof{\langle a?b \rangle E} = \{ P \in \pi | P \equiv Q | u?(y)P', \\ \and \\\\ \and \\ \;\;\; u \in \meaningof{a}, \forall z.P'\{z/y\} \in \meaningof{E\{z/b\}}\}, \and \\ \meaningof{a!E} = \{ P \in \pi | P \equiv Q | x!\langle P' \rangle, x \in \meaningof{a} P' \in \meaningof{E}\} }
\end{mathpar}

\begin{mathpar}
 \inferrule* [lab=nominal] {} {\meaningof{\quotep{E}} = \{ \quotep{P} \in \quotep{\pi} | P \in \meaningof{E} \}, \and \meaningof{\quotep{P}} = \{ \quotep{Q} \in \quotep{\pi} | P \equiv Q \} \and \\ \meaningof{@\quotep{E}} = \{ P \in \pi | P \equiv @x, x \in \meaningof{E} \}}
\end{mathpar}

\begin{eqnarray*}
  \\
  \meaningof{-} : TS \to ST
\end{eqnarray*}

\begin{eqnarray*}
  \\
  L : TS \to ST
\end{eqnarray*}

\begin{eqnarray*}
  \\
  P \models E \iff P \in \meaningof{E}
\end{eqnarray*}

\begin{eqnarray*}
  P \approx_{L} Q \iff \forall E \in L. P \models E \iff Q \models E
\end{eqnarray*}

\begin{eqnarray*}
  P \approx_{K} Q
\end{eqnarray*}

\begin{eqnarray*}
  P \approx Q
\end{eqnarray*}

$\approx_{K} = \approx = \approx_{L}$

\subsubsection{Contextual duality}

Note that contexts extend the quotation operation to a family of
operations from processes to names. Given a context, $M$, we can
define a \emph{nominal context}, $\quotep{M}$ by $\quotep{M}[P] :=
\quotep{M[P]}$. To foreshadow what is to come we observe that these
operations enjoy a duality with processes very much like the duality
between vectors and maps from vectors to scalars.

Further, because the calculus is essentially higher-order, we have a
correspondence between contexts and processes. More specifically,
given a name $x$ and a context $M$ we can construct $M^{*}_{x}$ such
that 

\begin{mathpar}
  M^{*}_{x} | \lift{x}{P} \red M[P]
\end{mathpar}

namely,

\begin{mathpar}
  M^{*}_{x} := x?(u).M[\dropn{u}]
\end{mathpar}

The dependence of $M^{*}_{x}$ on a name makes it an abstraction, 

\begin{mathpar}
  M^{*} := (x)x?(u).M[\dropn{u}]
\end{mathpar}

\subsection{Additional notation}

It will sometimes be convenient to denote the process a name
quotes. We already have the notation $x = \quotep{P}$, but it will be
convenient to introduce an alternate notation, $\procn{x}$, when we
want to emphasize the connection to the use of the name. Note that, by
virtue of name equivalence, $\quotep{\procn{x}} \nameeq x$; so, the
notation is consistent with previous definitions.

Further, because names have structure it is possible to effect
substitutions on the basis of that structure. This means we need to
upgrade our notation for substitutions, which we accomplish by
adapting comprehension notation. Thus,

\begin{mathpar}
  P\{ y / x : x \in S \}
\end{mathpar}

is interpreted to mean the process derived from P by replacing (in a
capture-avoiding manner) each occurrence of $x$ in $S$ by $y$. For example,

\begin{mathpar}
  P\{ \quotep{\procn{x}|\procn{x}} / x : x \in \freenames{P} \}
\end{mathpar}

will replace each (occurrence) of a free name $x$ in $P$ by
$\quotep{\procn{x}|\procn{x}}$.

Also, we will avail ourselves of the notation $x^{L}$ and $x^{R}$ to
denote injections of a name into disjoint copies of the name
space. There are numerous ways to accomplish this. One example can be
found in \cite{MeredithR05}. This notation overloads to vectors of
names: $\vec{x}^{\pi} := (x_{i}^{\pi} \; : \; 0 \leq i < |\vec{x}| )$ where $\pi \in \{L,R\}$.

We also use $P^{\Box} := P|\Box$.

In \cite{MeredithR05} an interpretation of the new operator is
given. It turns out that there are several possible interpretations
all enjoying the requisite algebraic properties of the operator (see
\cite{milner91polyadicpi}). We will therefore make liberal use of
$(\nu\; \vec{x})P$.

% subsection the_syntax_and_semantics_of_the_notation_system (end)   

\input{qm2pi.qmops} 

\input{qm2pi.sterngerlach} 

\input{qm2pi.metric} 

% section concurrent_process_calculi (end)

%\input{qm2pi.proofsketch}

% section proof sketch (end)

%\input{qm2pi.slviaknots} 

% section spatial logic via knots (end)

\input{qm2pi.conclusion}

% section conclusion (end)

%\input{qm2pi.dtcodes} 

% section wiring algorithm (end)

\input{qm2pi.ack} 

% section acknowledgments (end)

\newpage


\bibliographystyle{plain}   
\bibliography{../../biblios/main.bib}

\input{qm2pi.rhodetails}

\end{document}

 

% section acknowledgments (end)

\newpage


\bibliographystyle{plain}   
\bibliography{../../biblios/main.bib}

\documentclass[12pt]{llncs}
%\documentclass{jktr}

\usepackage[pdftex]{hyperref}                   
\usepackage {listings}
\usepackage {mathpartir}
\usepackage{bcprules}
%\usepackage{listings}
                       
\usepackage{graphicx} 
%\usepackage[margins=2.5cm,nohead,nofoot]{geometry}
%\usepackage{geometry}
\usepackage{amsfonts}
\usepackage{amstext}
\usepackage{latexsym}
\usepackage{amssymb}
\usepackage{color}


%\include{myPreamble}
\include{qm2pi.local} 

%\ifpdf
%\usepackage[pdftex]{graphicx}
%\else
%\usepackage{graphicx}
%\fi

 % \ifpdf
%  \usepackage{pdfsync}
%  \if


%\title{Brief Article}
%\author{David F. Snyder}
%\author{L.G. Meredith}

%\address{Dept. of Math., Texas State University--San Marcos, San Marcos, TX 78666}
       
\pagestyle{empty}


\begin{document}

\lstset{language=[Objective]Caml,frame=shadowbox}

\input{qm2pi.front}

% section front matter (end)

\input{qm2pi.intro} 
 
% section introduction (end)

% \input{qm2pi.knotations} 

% section notation (end)

\input{qm2pi.process.calculi} 

% section concurrent_process_calculi_and_spatial_logics_ (end)
    
%\input{qm2pi.knots2pi} 

%\input{qm2pi.trefoil} 

%\input{qm2pi.mainthm} 

% subsection basic_interpretation (end)

%\input{qm2pi.rho.presentation} 
\subsection{The syntax and semantics of the notation system}\label{sub:the_syntax_and_semantics_of_the_notation_system} % (fold)

We now summarize a technical presentation of the calculus that
embodies our theory of dynamics. The typical presentation of such a
calculus follows the style of giving generators and relations on
them. The grammar, below, describing term constructors, freely
generates the set of processes, $\Proc$. This set is then quotiented
by a relation known as structural congruence and it is over this set
that the notion of dynamics is expressed. This presentation is
essentially that of \cite{MeredithR05} with the addition of
polyadicity and summation. For readability we have relegated some of
the technical subtleties to an appendix.

\subsubsection{Process grammar}\label{subsub:process_grammar}

\begin{mathpar}
  \inferrule* [lab=synchronization] {} {{M} \bc \pzero \;|\; x?F \;|\; x!C }
  \and
  \inferrule* [lab=abstraction] {} {{F} \bc (x)P}
  \and
  \inferrule* [lab=concretion] {} {{C} \bc \langle Q \rangle}
  \and
  \inferrule* [lab=process] {} {{P,Q} \bc M \;| \;P|Q \;|\; @{x}}
  \and
  \inferrule* [lab=name] {} {{x} \bc \quotep{P}}
\end{mathpar} 

Note that $\vec{x}$ (resp. $\vec{P}$) denotes a vector of names
(resp. processes) of length $|\vec{x}|$ (resp. $|\vec{P}|$). We adopt
the following useful abbreviations.

\begin{mathpar}
   x?(\vec{y}).P := x.(\vec{y})P \and  x\clift{\vec{P}} := x.\clift{\vec{P}}
   \and x!(y) := \lift{x}{\dropn{y}}
   \and \Pi_{i=0}^{n-1}P_i := P_0 | \ldots | P_{n-1}
\end{mathpar}

\subsubsection{Structural congruence}

\paragraph{Free and bound names and alpha-equivalence.} At the
core of structural equivalence is alpha-equivalence which identifies
process that are the same up to a change of variable. Formally, we
recognize the distinction between free and bound names. The free names
of a process, $\freenames{P}$, may be calculated recursively as
follows:

\begin{mathpar}
\freenames{\pzero} := \emptyset
  \and \\
  \freenames{x?(y).P} := \{ x \} \cup (\freenames{P} \setminus \{ y \})
  \and 
  \freenames{x!\langle P \rangle} := \{ x \} \cup \{ P \} 
  \and \\
  \freenames{P|Q} := \freenames{P} \cup \freenames{Q}
  \and \\
  \freenames{@{x}} := \{ x \}
\end{mathpar}

$\pi$
$\quotep{\pi}$

$\freenames{-} : \pi \to \mathcal{P}(\quotep{\pi})$

\begin{eqnarray*}
  \freenames{\pzero} & := & \emptyset \\
  \freenames{x?(y).P} & := & \{ x \} \cup (\freenames{P} \setminus \{ y \}) \\
  \freenames{x!\langle P \rangle} & := & \{ x \} \cup \{ P \} \\
  \freenames{P|Q} & := & \freenames{P} \cup \freenames{Q} \\
  \freenames{\dropn{x}} & := & \{ x \}
\end{eqnarray*}

The bound names of a process, $\boundnames{P}$, are those names occurring in $P$
that are not free. For example, in $x?(y).0$, the name $x$ is free, while $y$ is bound.

\begin{mathpar}
  \inferrule* [lab=monoidal-laws] {} { P|Q \equiv Q|P \and P|0 \equiv P \and P|(Q|R) \equiv (P|Q)|R }
\end{mathpar}

\begin{mathpar}
  \inferrule* [lab=alpha-equivalence] {} { (x)P \equiv (y)P\{y/x\} \and y \not\in \freenames{P} }
\end{mathpar}

\begin{definition}
Then two processes, $P,Q$, are alpha-equivalent if $P = Q\{\vec{y}/\vec{x}\}$ for
some $\vec{x} \in \boundnames{Q},\vec{y} \in \boundnames{P}$, where $Q\{\vec{y}/\vec{x}\}$
denotes the capture-avoiding substitution of $\vec{y}$ for $\vec{x}$ in $Q$.
\end{definition}

\begin{definition}
  The {\em structural congruence} \cite{SangiorgiWalker} , $\equiv$,
  between processes is the least congruence containing
  alpha-equivalence, satisfying the abelian monoid laws
  (associativity, commutativity and $\pzero$ as identity) for parallel
  composition $|$ and for summation $+$.
\end{definition}

\subsection{Name equivalence}

We take name equivalence, written $\nameeq$, to be the smallest
equivalence relation generated by the following rules.

\begin{mathpar}
\inferrule*[lab=Quote-drop]
{ }
{ \quotep{@{x}} \nameeq x }

\inferrule*[lab=Struct-equiv]
{ P \scong Q }
{ \quotep{P} \nameeq \quotep{Q} }
\end{mathpar}

The astute reader will have noticed that the mutual recursion of names
and processes imposes a mutual recursion on alpha-equivalence and
structural equivalence via name-equivalence. Fortunately, all of this
works out pleasantly and we may calculate in the natural way, free of
concern. The reader interested in the details is referred to the
appendix \ref{appendix:rho_details}.

\subsection{Substitution}

We use $\Proc$ for the set of processes, $\QProc$ for the set of
names, and $\id{\{}\vec{y} / \vec{x} \id{\}}$ to denote partial maps,
$s : \QProc \rightarrow \QProc$. A map, $s$ lifts, uniquely, to a map
on process terms, $\widehat{s} : \Proc \rightarrow \Proc$ by the
following equations.

\begin{mathpar}
  (0) \psubstp{Q}{P} := 0 \\
  (R \juxtap S) \psubstp{Q}{P}
  :=    
  (R)\psubstp{Q}{P} \juxtap (S) \psubstp{Q}{P} \\
  (x?(y).R) \psubstp{Q}{P}    
  :=    
  (x)\substp{Q}{P} (z)\concat( (R \psubstn{z}{y}) \psubstp{Q}{P} ) \\
  (\lift{x}{R}) \psubstp{Q}{P}  
  :=
  \lift{(x)\substp{Q}{P}}{ R \psubstp{Q}{P} } \\
%   (\dropn{x})  \psubstp{Q}{P}       
%   := 
%   \left\{ 
%     \begin{array}{ccc} 
%       \dropn{\quotep{Q}} & & x \nameeq \quotep{P} \\
%       \dropn{x} & & otherwise \\
%     \end{array}
%   \right. 
  (\dropn{x})  \psubstp{Q}{P}       
  := 
  \left\{ 
    \begin{array}{ccc} 
      Q & & x \nameeq \quotep{P} \\
      \dropn{x} & & otherwise \\
    \end{array}
  \right.
\end{mathpar}
 

where

\begin{eqnarray}
  (x)\id{\{} \lpquote Q \rpquote / \lpquote P \rpquote \id{\}}            = 
  \left\{ 
    \begin{array}{ccc}
      \lpquote Q \rpquote & & x \nameeq \lpquote P \rpquote \\
      x & & otherwise \\
    \end{array}
  \right. \nonumber
\end{eqnarray}

and $z$ is chosen distinct from $\quotep{P}$, $\quotep{Q}$, the free
names in $Q$, and all the names in $R$. Our $\alpha$-equivalence will
be built in the standard way from this substitution.

\begin{remark}\label{rem:no_self_referential_names}
  One consequence of these definitions is that $\forall P. \quotep{P}
  \not\in \freenames{P}$.
\end{remark}

\subsection{ Dynamic quote: an example }

Anticipating something of what's to come, consider applying the
substitution, $\widehat{\id{\{}u / z \id{\}}}$, to the following pair
of processes, $\lift{w}{y!(z)}$ and $w[ \lpquote y!(z) \rpquote ]$.

\begin{eqnarray}
	\lift{w}{y!(z)}\widehat{\id{\{}u / z \id{\}}}
		& = &
		\lift{w}{y!(u)} \nonumber\\
	w[ \lpquote y!(z) \rpquote ] \widehat{ \id{\{}u / z \id{\}} }
		& = &
		w[ \lpquote y!(z) \rpquote ] \nonumber
\end{eqnarray}

Because the body of the process between quotes is impervious to
substitution, we get radically different answers. In fact, by
examining the first process in an input context,
e.g. $x?(z).\lift{w}{y!(z)}$, we see that the process under the lift
operator may be shaped by prefixed inputs binding a name inside it. In
this sense, the lift operator will be seen as a way to dynamically
construct processes before reifying them as names.

Finally equipped with these standard features we can present the
dynamics of the calculus.

\subsubsection{Operational semantics} 

Finally, we introduce the computational dynamics. What marks these
algebras as distinct from other more traditionally studied algebraic
structures, e.g. vector spaces or polynomial rings, is the manner in
which dynamics is captured. In traditional structures, dynamics is typically
expressed through morphisms between such structures, as in linear maps
between vector spaces or morphisms between rings. In algebras
associated with the semantics of computation, the dynamics is
expressed as part of the algebraic structure itself, through a
reduction reduction relation typically denoted by $\red$. Below, we
give a recursive presentation of this relation for the calculus used
in the encoding.

$\red \subseteq \pi \times \pi$
$\red : \pi \to \mathcal{P}(\pi)$

\begin{mathpar}
  \inferrule* [lab=Comm] { \textsf{match}( x_{src}, x_{trgt} ) } { x_{trgt}?(y)P \; | \; x_{src}!\langle {Q} \rangle \red P\{\quotep{Q}/y}\} }
  \and \\
  \inferrule* [lab=Par] {{P} \red {P}'} {{{P} | {Q}} \red {{P}' | {Q}}}
  \and
  \inferrule* [lab=Equiv]{{{P} \scong {P}'} \andalso {{P}' \red {Q}'} \andalso {{Q}' \scong {Q}}}{{P} \red {Q}}
\end{mathpar}

\begin{eqnarray*}
  match_{\equiv} (\quotep{P},\quotep{Q}) & := & P \equiv Q \\
  match_{\dagger}(\quotep{P},\quotep{Q}) & := & \forall R. P|Q \red^{*} R => R \red^{*} 0 \\
  match_{K}(\quotep{P},\quotep{Q}) & := & K \mbox{ for some context } K
\end{eqnarray*}

$u?(x)P | u!\langle Q \rangle \red P\{\quotep{Q}/x\}$

%We write $\wred$ for $\red^*$, and $P\red$ if $\exists Q $ such that $ P \red Q$.
We write $P\red$ if $\exists Q $ such that $ P \red Q$ and $P\not\red$, otherwise.

\section{Replication}

As mentioned before, it is known that replication (and hence
recursion) can be implemented in a higher-order process algebra
\cite{SangiorgiWalker}. As our first example of calculation with the
machinery thus far presented we give the construction explicitly in
the {\rhoc}.

\begin{eqnarray}
	D_{x} & := & \prefix{x}{y}{(\binpar{\outputp{x}{y}}{@{y}})} \nonumber\\
	\bangp_{x}{P} & := & \binpar{{x}!\langle{\binpar{D_{x}}{P}}\rangle}{D_{x}} \nonumber
\end{eqnarray}

\begin{eqnarray}
	\bangp_{x}{P} & & \nonumber\\
	=
	& {x}!\langle{(\prefix{x}{y}{(\outputp{x}{y} | @{y})) | P}}\rangle 
	      | \prefix{x}{y}{(\outputp{x}{y} | @{y})} & \nonumber\\
	\red
	& (\outputp{x}{y} | @{y})\substn{\quotep{(\prefix{x}{y}{(@{y} | \outputp{x}{y})) | P}}}{y} & \nonumber\\
	=
	& \outputp{x}{\quotep{(\prefix{x}{y}{(\outputp{x}{y} | @{y})) | P}}}
	  | {(\prefix{x}{y}{(\outputp{x}{y} | @{y})) | P}} & \nonumber\\
	\red
	& \ldots & \nonumber\\
	\red^*
	& P | P | \ldots & \nonumber
\end{eqnarray}

Of course, this encoding, as an implementation, runs away, unfolding
$\bangp{P}$ eagerly. A lazier and more implementable replication
operator, restricted to input-guarded processes, may be obtained as follows.

\begin{eqnarray}
\bangp{\prefix{u}{v}{P}} 
	:= 
	\binpar{\lift{x}{\prefix{u}{v}{(\binpar{D(x)}{P})}}}{D(x)} \nonumber
\end{eqnarray}

\begin{remark}
  Note that the lazier definition still does not deal with summation
  or mixed summation (i.e. sums over input and output). The reader is
  invited to construct definitions of replication that deal with these
  features. 

  Further, the definitions are parameterized in a name, $x$. Can you,
  gentle reader, make a definition that eliminates this parameter and
  guarantees no accidental interaction between the replication
  machinery and the process being replicated -- i.e. no accidental
  sharing of names used by the process to get its work done and the
  name(s) used by the replication to effect copying. This latter
  revision of the definition of replication is crucial to obtaining
  the expected identity $!!P \sim !P$.
\end{remark}

\begin{remark}\label{rem:paradoxical_combinator}
  The reader familiar with the lambda calculus will have noticed the
  similarity between $D$ and the paradoxical combinator.

  [Ed. note: the existence of this seems to suggest we have to be more
  restrictive on the set of processes and names we admit if we are to
  support no-cloning.]
\end{remark}

\subsubsection{Bisimulation}

The computational dynamics gives rise to another kind of equivalence,
the equivalence of computational behavior. As previously mentioned
this is typically captured \emph{via} some form of bisimulation.

% The notion we use in this paper is weak barbed bisimulation
% \cite{milner91polyadicpi}.

The notion we use in this paper is derived from weak barbed
bisimulation \cite{milner91polyadicpi}. 

\begin{definition}
An \emph{observation relation}, $\downarrow_{\mathcal N}$, over a set
of names, $\mathcal N$, is the smallest relation satisfying the rules
below.

\infrule[Out-barb]{y \in {\mathcal N}, \; x \nameeq y}
		  {\outputp{x}{v} \downarrow_{\mathcal N} x}
\infrule[Par-barb]{\mbox{$P\downarrow_{\mathcal N} x$ or $Q\downarrow_{\mathcal N} x$}}
		  {\binpar{P}{Q} \downarrow_{\mathcal N} x}

We write $P \Downarrow_{\mathcal N} x$ if there is $Q$ such that 
$P \wred Q$ and $Q \downarrow_{\mathcal N} x$.
\end{definition}

\begin{definition}
%\label{def.bbisim}
An  ${\mathcal N}$-\emph{barbed bisimulation} over a set of names, ${\mathcal N}$, is a symmetric binary relation 
${\mathcal S}_{\mathcal N}$ between agents such that $P\rel{S}_{\mathcal N}Q$ implies:
\begin{enumerate}
\item If $P \red P'$ then $Q \wred Q'$ and $P'\rel{S}_{\mathcal N} Q'$.
\item If $P\downarrow_{\mathcal N} x$, then $Q\Downarrow_{\mathcal N} x$.
\end{enumerate}
$P$ is ${\mathcal N}$-barbed bisimilar to $Q$, written
$P \wbbisim_{\mathcal N} Q$, if $P \rel{S}_{\mathcal N} Q$ for some ${\mathcal N}$-barbed bisimulation ${\mathcal S}_{\mathcal N}$.
\end{definition}

$\mathcal{R} \subseteq \pi \times \pi$

$P \mathcal{R} Q => \forall P'. P \red P' \Rightarrow \exists Q'. Q \red Q', P' \mathcal{R} Q'$

$P \vdash x \Rightarrow Q \vdash x$

\begin{mathpar}
  \inferrule*[lab=Out-barb]{x \nameeq y}{{y}!\langle{Q}\rangle \vdash x}
  \and
  \inferrule*[lab=Par-barb]{\mbox{$P\vdash x$ or $Q\vdash x$}}{\binpar{P}{Q} \vdash x}
\end{mathpar}

\subsubsection{Contexts}

One of the principle advantages of computational calculi like the
$\pi$-calculus is a well-defined notion of context,
contextual-equivalence and a correlation between
contextual-equivalence and notions of bisimulation. The notion of
context allows the decomposition of a process into (sub-)process and
its syntactic environment, its context. Thus, a context may be
thought of as a process with a ``hole'' (written $\Box$) in it. The
application of a context $M$ to a process $P$, written $M[P]$, is
tantamount to filling the hole in $M$ with $P$. In this paper we do
not need the full weight of this theory, but do make use of the notion
of context in the proof the main theorem. 

\begin{mathpar}
  \inferrule* [lab=summation] {} {{M_{M},M_{N}} \bc \Box \;|\; x.M_{A} \;|\; M_{M}+M_{N}}
  \and
  \inferrule* [lab=agent] {} {{M_{A}} \bc (\vec{x})M_{P} \;| \; \clift{P_0,\ldots,M_{P},\ldots,P_N}}
  \and \\
  \inferrule* [lab=process] {} {{M_{P}} \bc M_{N} \;| \;P|M_{P} }
\end{mathpar} 

\begin{mathpar}
  \inferrule* [lab=sychronization] {} {M_{N} \bc \Box \;|\; x?M_{F} \;|\; x!M_{C}}
  \and
  \inferrule* [lab=abstraction] {} {{M_{F}} \bc (x)M_{P} }
  \and
  \inferrule* [lab=concretion] {} {{M_{C}} \bc \langle M_{P} \rangle }
  \and \\
  \inferrule* [lab=process] {} {{M_{P}} \bc M_{N} \;| \;P|M_{P} }
\end{mathpar}

\begin{definition}[contextual application] Given a context $M$, and
  process $P$, we define the \emph{contextual application}, $M[P] :=
  M\{P/\Box\}$. That is, the contextual application of M to P is the
  substitution of $P$ for $\Box$ in $M$.
\end{definition}

$\meaningof{-} : L \to \mathcal{P}(\pi)$

\begin{mathpar}
  \inferrule* [lab=collection] {} {\meaningof{true} = \pi, \and \meaningof{~E} = \pi \setminus \meaningof{E}, \and \meaningof{E_{1} \& E_{2}} = \meaningof{E_{1}} \cap \meaningof{E_{2}}}
\end{mathpar}

\begin{mathpar}
  \inferrule* [lab=structure] {} {\meaningof{0} = \{ P \in \pi | P \equiv 0 \}, \and \\ \meaningof{E_1 | E_2} = \{ P \in \pi | P \equiv P_{1} | P_{2}, P_{1} \in \meaningof{E_{1}}, P_{2} \in \meaningof{E_2}\} }
\end{mathpar}

\begin{mathpar}
 \inferrule* [lab=behavior] {} {\meaningof{\langle a?b \rangle E} = \{ P \in \pi | P \equiv Q | u?(y)P', \\ \and \\\\ \and \\ \;\;\; u \in \meaningof{a}, \forall z.P'\{z/y\} \in \meaningof{E\{z/b\}}\}, \and \\ \meaningof{a!E} = \{ P \in \pi | P \equiv Q | x!\langle P' \rangle, x \in \meaningof{a} P' \in \meaningof{E}\} }
\end{mathpar}

\begin{mathpar}
 \inferrule* [lab=nominal] {} {\meaningof{\quotep{E}} = \{ \quotep{P} \in \quotep{\pi} | P \in \meaningof{E} \}, \and \meaningof{\quotep{P}} = \{ \quotep{Q} \in \quotep{\pi} | P \equiv Q \} \and \\ \meaningof{@\quotep{E}} = \{ P \in \pi | P \equiv @x, x \in \meaningof{E} \}}
\end{mathpar}

\begin{eqnarray*}
  \\
  \meaningof{-} : TS \to ST
\end{eqnarray*}

\begin{eqnarray*}
  \\
  L : TS \to ST
\end{eqnarray*}

\begin{eqnarray*}
  \\
  P \models E \iff P \in \meaningof{E}
\end{eqnarray*}

\begin{eqnarray*}
  P \approx_{L} Q \iff \forall E \in L. P \models E \iff Q \models E
\end{eqnarray*}

\begin{eqnarray*}
  P \approx_{K} Q
\end{eqnarray*}

\begin{eqnarray*}
  P \approx Q
\end{eqnarray*}

$\approx_{K} = \approx = \approx_{L}$

\subsubsection{Contextual duality}

Note that contexts extend the quotation operation to a family of
operations from processes to names. Given a context, $M$, we can
define a \emph{nominal context}, $\quotep{M}$ by $\quotep{M}[P] :=
\quotep{M[P]}$. To foreshadow what is to come we observe that these
operations enjoy a duality with processes very much like the duality
between vectors and maps from vectors to scalars.

Further, because the calculus is essentially higher-order, we have a
correspondence between contexts and processes. More specifically,
given a name $x$ and a context $M$ we can construct $M^{*}_{x}$ such
that 

\begin{mathpar}
  M^{*}_{x} | \lift{x}{P} \red M[P]
\end{mathpar}

namely,

\begin{mathpar}
  M^{*}_{x} := x?(u).M[\dropn{u}]
\end{mathpar}

The dependence of $M^{*}_{x}$ on a name makes it an abstraction, 

\begin{mathpar}
  M^{*} := (x)x?(u).M[\dropn{u}]
\end{mathpar}

\subsection{Additional notation}

It will sometimes be convenient to denote the process a name
quotes. We already have the notation $x = \quotep{P}$, but it will be
convenient to introduce an alternate notation, $\procn{x}$, when we
want to emphasize the connection to the use of the name. Note that, by
virtue of name equivalence, $\quotep{\procn{x}} \nameeq x$; so, the
notation is consistent with previous definitions.

Further, because names have structure it is possible to effect
substitutions on the basis of that structure. This means we need to
upgrade our notation for substitutions, which we accomplish by
adapting comprehension notation. Thus,

\begin{mathpar}
  P\{ y / x : x \in S \}
\end{mathpar}

is interpreted to mean the process derived from P by replacing (in a
capture-avoiding manner) each occurrence of $x$ in $S$ by $y$. For example,

\begin{mathpar}
  P\{ \quotep{\procn{x}|\procn{x}} / x : x \in \freenames{P} \}
\end{mathpar}

will replace each (occurrence) of a free name $x$ in $P$ by
$\quotep{\procn{x}|\procn{x}}$.

Also, we will avail ourselves of the notation $x^{L}$ and $x^{R}$ to
denote injections of a name into disjoint copies of the name
space. There are numerous ways to accomplish this. One example can be
found in \cite{MeredithR05}. This notation overloads to vectors of
names: $\vec{x}^{\pi} := (x_{i}^{\pi} \; : \; 0 \leq i < |\vec{x}| )$ where $\pi \in \{L,R\}$.

We also use $P^{\Box} := P|\Box$.

In \cite{MeredithR05} an interpretation of the new operator is
given. It turns out that there are several possible interpretations
all enjoying the requisite algebraic properties of the operator (see
\cite{milner91polyadicpi}). We will therefore make liberal use of
$(\nu\; \vec{x})P$.

% subsection the_syntax_and_semantics_of_the_notation_system (end)   

\input{qm2pi.qmops} 

\input{qm2pi.sterngerlach} 

\input{qm2pi.metric} 

% section concurrent_process_calculi (end)

%\input{qm2pi.proofsketch}

% section proof sketch (end)

%\input{qm2pi.slviaknots} 

% section spatial logic via knots (end)

\input{qm2pi.conclusion}

% section conclusion (end)

%\input{qm2pi.dtcodes} 

% section wiring algorithm (end)

\input{qm2pi.ack} 

% section acknowledgments (end)

\newpage


\bibliographystyle{plain}   
\bibliography{../../biblios/main.bib}

\input{qm2pi.rhodetails}

\end{document}



\end{document}

 

%\ifpdf
%\usepackage[pdftex]{graphicx}
%\else
%\usepackage{graphicx}
%\fi

 % \ifpdf
%  \usepackage{pdfsync}
%  \if


%\title{Brief Article}
%\author{David F. Snyder}
%\author{L.G. Meredith}

%\address{Dept. of Math., Texas State University--San Marcos, San Marcos, TX 78666}
       
\pagestyle{empty}


\begin{document}

\lstset{language=[Objective]Caml,frame=shadowbox}

\documentclass[12pt]{llncs}
%\documentclass{jktr}

\usepackage[pdftex]{hyperref}                   
\usepackage {listings}
\usepackage {mathpartir}
\usepackage{bcprules}
%\usepackage{listings}
                       
\usepackage{graphicx} 
%\usepackage[margins=2.5cm,nohead,nofoot]{geometry}
%\usepackage{geometry}
\usepackage{amsfonts}
\usepackage{amstext}
\usepackage{latexsym}
\usepackage{amssymb}
\usepackage{color}


%\include{myPreamble}
\documentclass[12pt]{llncs}
%\documentclass{jktr}

\usepackage[pdftex]{hyperref}                   
\usepackage {listings}
\usepackage {mathpartir}
\usepackage{bcprules}
%\usepackage{listings}
                       
\usepackage{graphicx} 
%\usepackage[margins=2.5cm,nohead,nofoot]{geometry}
%\usepackage{geometry}
\usepackage{amsfonts}
\usepackage{amstext}
\usepackage{latexsym}
\usepackage{amssymb}
\usepackage{color}


%\include{myPreamble}
\include{qm2pi.local} 

%\ifpdf
%\usepackage[pdftex]{graphicx}
%\else
%\usepackage{graphicx}
%\fi

 % \ifpdf
%  \usepackage{pdfsync}
%  \if


%\title{Brief Article}
%\author{David F. Snyder}
%\author{L.G. Meredith}

%\address{Dept. of Math., Texas State University--San Marcos, San Marcos, TX 78666}
       
\pagestyle{empty}


\begin{document}

\lstset{language=[Objective]Caml,frame=shadowbox}

\input{qm2pi.front}

% section front matter (end)

\input{qm2pi.intro} 
 
% section introduction (end)

% \input{qm2pi.knotations} 

% section notation (end)

\input{qm2pi.process.calculi} 

% section concurrent_process_calculi_and_spatial_logics_ (end)
    
%\input{qm2pi.knots2pi} 

%\input{qm2pi.trefoil} 

%\input{qm2pi.mainthm} 

% subsection basic_interpretation (end)

%\input{qm2pi.rho.presentation} 
\subsection{The syntax and semantics of the notation system}\label{sub:the_syntax_and_semantics_of_the_notation_system} % (fold)

We now summarize a technical presentation of the calculus that
embodies our theory of dynamics. The typical presentation of such a
calculus follows the style of giving generators and relations on
them. The grammar, below, describing term constructors, freely
generates the set of processes, $\Proc$. This set is then quotiented
by a relation known as structural congruence and it is over this set
that the notion of dynamics is expressed. This presentation is
essentially that of \cite{MeredithR05} with the addition of
polyadicity and summation. For readability we have relegated some of
the technical subtleties to an appendix.

\subsubsection{Process grammar}\label{subsub:process_grammar}

\begin{mathpar}
  \inferrule* [lab=synchronization] {} {{M} \bc \pzero \;|\; x?F \;|\; x!C }
  \and
  \inferrule* [lab=abstraction] {} {{F} \bc (x)P}
  \and
  \inferrule* [lab=concretion] {} {{C} \bc \langle Q \rangle}
  \and
  \inferrule* [lab=process] {} {{P,Q} \bc M \;| \;P|Q \;|\; @{x}}
  \and
  \inferrule* [lab=name] {} {{x} \bc \quotep{P}}
\end{mathpar} 

Note that $\vec{x}$ (resp. $\vec{P}$) denotes a vector of names
(resp. processes) of length $|\vec{x}|$ (resp. $|\vec{P}|$). We adopt
the following useful abbreviations.

\begin{mathpar}
   x?(\vec{y}).P := x.(\vec{y})P \and  x\clift{\vec{P}} := x.\clift{\vec{P}}
   \and x!(y) := \lift{x}{\dropn{y}}
   \and \Pi_{i=0}^{n-1}P_i := P_0 | \ldots | P_{n-1}
\end{mathpar}

\subsubsection{Structural congruence}

\paragraph{Free and bound names and alpha-equivalence.} At the
core of structural equivalence is alpha-equivalence which identifies
process that are the same up to a change of variable. Formally, we
recognize the distinction between free and bound names. The free names
of a process, $\freenames{P}$, may be calculated recursively as
follows:

\begin{mathpar}
\freenames{\pzero} := \emptyset
  \and \\
  \freenames{x?(y).P} := \{ x \} \cup (\freenames{P} \setminus \{ y \})
  \and 
  \freenames{x!\langle P \rangle} := \{ x \} \cup \{ P \} 
  \and \\
  \freenames{P|Q} := \freenames{P} \cup \freenames{Q}
  \and \\
  \freenames{@{x}} := \{ x \}
\end{mathpar}

$\pi$
$\quotep{\pi}$

$\freenames{-} : \pi \to \mathcal{P}(\quotep{\pi})$

\begin{eqnarray*}
  \freenames{\pzero} & := & \emptyset \\
  \freenames{x?(y).P} & := & \{ x \} \cup (\freenames{P} \setminus \{ y \}) \\
  \freenames{x!\langle P \rangle} & := & \{ x \} \cup \{ P \} \\
  \freenames{P|Q} & := & \freenames{P} \cup \freenames{Q} \\
  \freenames{\dropn{x}} & := & \{ x \}
\end{eqnarray*}

The bound names of a process, $\boundnames{P}$, are those names occurring in $P$
that are not free. For example, in $x?(y).0$, the name $x$ is free, while $y$ is bound.

\begin{mathpar}
  \inferrule* [lab=monoidal-laws] {} { P|Q \equiv Q|P \and P|0 \equiv P \and P|(Q|R) \equiv (P|Q)|R }
\end{mathpar}

\begin{mathpar}
  \inferrule* [lab=alpha-equivalence] {} { (x)P \equiv (y)P\{y/x\} \and y \not\in \freenames{P} }
\end{mathpar}

\begin{definition}
Then two processes, $P,Q$, are alpha-equivalent if $P = Q\{\vec{y}/\vec{x}\}$ for
some $\vec{x} \in \boundnames{Q},\vec{y} \in \boundnames{P}$, where $Q\{\vec{y}/\vec{x}\}$
denotes the capture-avoiding substitution of $\vec{y}$ for $\vec{x}$ in $Q$.
\end{definition}

\begin{definition}
  The {\em structural congruence} \cite{SangiorgiWalker} , $\equiv$,
  between processes is the least congruence containing
  alpha-equivalence, satisfying the abelian monoid laws
  (associativity, commutativity and $\pzero$ as identity) for parallel
  composition $|$ and for summation $+$.
\end{definition}

\subsection{Name equivalence}

We take name equivalence, written $\nameeq$, to be the smallest
equivalence relation generated by the following rules.

\begin{mathpar}
\inferrule*[lab=Quote-drop]
{ }
{ \quotep{@{x}} \nameeq x }

\inferrule*[lab=Struct-equiv]
{ P \scong Q }
{ \quotep{P} \nameeq \quotep{Q} }
\end{mathpar}

The astute reader will have noticed that the mutual recursion of names
and processes imposes a mutual recursion on alpha-equivalence and
structural equivalence via name-equivalence. Fortunately, all of this
works out pleasantly and we may calculate in the natural way, free of
concern. The reader interested in the details is referred to the
appendix \ref{appendix:rho_details}.

\subsection{Substitution}

We use $\Proc$ for the set of processes, $\QProc$ for the set of
names, and $\id{\{}\vec{y} / \vec{x} \id{\}}$ to denote partial maps,
$s : \QProc \rightarrow \QProc$. A map, $s$ lifts, uniquely, to a map
on process terms, $\widehat{s} : \Proc \rightarrow \Proc$ by the
following equations.

\begin{mathpar}
  (0) \psubstp{Q}{P} := 0 \\
  (R \juxtap S) \psubstp{Q}{P}
  :=    
  (R)\psubstp{Q}{P} \juxtap (S) \psubstp{Q}{P} \\
  (x?(y).R) \psubstp{Q}{P}    
  :=    
  (x)\substp{Q}{P} (z)\concat( (R \psubstn{z}{y}) \psubstp{Q}{P} ) \\
  (\lift{x}{R}) \psubstp{Q}{P}  
  :=
  \lift{(x)\substp{Q}{P}}{ R \psubstp{Q}{P} } \\
%   (\dropn{x})  \psubstp{Q}{P}       
%   := 
%   \left\{ 
%     \begin{array}{ccc} 
%       \dropn{\quotep{Q}} & & x \nameeq \quotep{P} \\
%       \dropn{x} & & otherwise \\
%     \end{array}
%   \right. 
  (\dropn{x})  \psubstp{Q}{P}       
  := 
  \left\{ 
    \begin{array}{ccc} 
      Q & & x \nameeq \quotep{P} \\
      \dropn{x} & & otherwise \\
    \end{array}
  \right.
\end{mathpar}
 

where

\begin{eqnarray}
  (x)\id{\{} \lpquote Q \rpquote / \lpquote P \rpquote \id{\}}            = 
  \left\{ 
    \begin{array}{ccc}
      \lpquote Q \rpquote & & x \nameeq \lpquote P \rpquote \\
      x & & otherwise \\
    \end{array}
  \right. \nonumber
\end{eqnarray}

and $z$ is chosen distinct from $\quotep{P}$, $\quotep{Q}$, the free
names in $Q$, and all the names in $R$. Our $\alpha$-equivalence will
be built in the standard way from this substitution.

\begin{remark}\label{rem:no_self_referential_names}
  One consequence of these definitions is that $\forall P. \quotep{P}
  \not\in \freenames{P}$.
\end{remark}

\subsection{ Dynamic quote: an example }

Anticipating something of what's to come, consider applying the
substitution, $\widehat{\id{\{}u / z \id{\}}}$, to the following pair
of processes, $\lift{w}{y!(z)}$ and $w[ \lpquote y!(z) \rpquote ]$.

\begin{eqnarray}
	\lift{w}{y!(z)}\widehat{\id{\{}u / z \id{\}}}
		& = &
		\lift{w}{y!(u)} \nonumber\\
	w[ \lpquote y!(z) \rpquote ] \widehat{ \id{\{}u / z \id{\}} }
		& = &
		w[ \lpquote y!(z) \rpquote ] \nonumber
\end{eqnarray}

Because the body of the process between quotes is impervious to
substitution, we get radically different answers. In fact, by
examining the first process in an input context,
e.g. $x?(z).\lift{w}{y!(z)}$, we see that the process under the lift
operator may be shaped by prefixed inputs binding a name inside it. In
this sense, the lift operator will be seen as a way to dynamically
construct processes before reifying them as names.

Finally equipped with these standard features we can present the
dynamics of the calculus.

\subsubsection{Operational semantics} 

Finally, we introduce the computational dynamics. What marks these
algebras as distinct from other more traditionally studied algebraic
structures, e.g. vector spaces or polynomial rings, is the manner in
which dynamics is captured. In traditional structures, dynamics is typically
expressed through morphisms between such structures, as in linear maps
between vector spaces or morphisms between rings. In algebras
associated with the semantics of computation, the dynamics is
expressed as part of the algebraic structure itself, through a
reduction reduction relation typically denoted by $\red$. Below, we
give a recursive presentation of this relation for the calculus used
in the encoding.

$\red \subseteq \pi \times \pi$
$\red : \pi \to \mathcal{P}(\pi)$

\begin{mathpar}
  \inferrule* [lab=Comm] { \textsf{match}( x_{src}, x_{trgt} ) } { x_{trgt}?(y)P \; | \; x_{src}!\langle {Q} \rangle \red P\{\quotep{Q}/y}\} }
  \and \\
  \inferrule* [lab=Par] {{P} \red {P}'} {{{P} | {Q}} \red {{P}' | {Q}}}
  \and
  \inferrule* [lab=Equiv]{{{P} \scong {P}'} \andalso {{P}' \red {Q}'} \andalso {{Q}' \scong {Q}}}{{P} \red {Q}}
\end{mathpar}

\begin{eqnarray*}
  match_{\equiv} (\quotep{P},\quotep{Q}) & := & P \equiv Q \\
  match_{\dagger}(\quotep{P},\quotep{Q}) & := & \forall R. P|Q \red^{*} R => R \red^{*} 0 \\
  match_{K}(\quotep{P},\quotep{Q}) & := & K \mbox{ for some context } K
\end{eqnarray*}

$u?(x)P | u!\langle Q \rangle \red P\{\quotep{Q}/x\}$

%We write $\wred$ for $\red^*$, and $P\red$ if $\exists Q $ such that $ P \red Q$.
We write $P\red$ if $\exists Q $ such that $ P \red Q$ and $P\not\red$, otherwise.

\section{Replication}

As mentioned before, it is known that replication (and hence
recursion) can be implemented in a higher-order process algebra
\cite{SangiorgiWalker}. As our first example of calculation with the
machinery thus far presented we give the construction explicitly in
the {\rhoc}.

\begin{eqnarray}
	D_{x} & := & \prefix{x}{y}{(\binpar{\outputp{x}{y}}{@{y}})} \nonumber\\
	\bangp_{x}{P} & := & \binpar{{x}!\langle{\binpar{D_{x}}{P}}\rangle}{D_{x}} \nonumber
\end{eqnarray}

\begin{eqnarray}
	\bangp_{x}{P} & & \nonumber\\
	=
	& {x}!\langle{(\prefix{x}{y}{(\outputp{x}{y} | @{y})) | P}}\rangle 
	      | \prefix{x}{y}{(\outputp{x}{y} | @{y})} & \nonumber\\
	\red
	& (\outputp{x}{y} | @{y})\substn{\quotep{(\prefix{x}{y}{(@{y} | \outputp{x}{y})) | P}}}{y} & \nonumber\\
	=
	& \outputp{x}{\quotep{(\prefix{x}{y}{(\outputp{x}{y} | @{y})) | P}}}
	  | {(\prefix{x}{y}{(\outputp{x}{y} | @{y})) | P}} & \nonumber\\
	\red
	& \ldots & \nonumber\\
	\red^*
	& P | P | \ldots & \nonumber
\end{eqnarray}

Of course, this encoding, as an implementation, runs away, unfolding
$\bangp{P}$ eagerly. A lazier and more implementable replication
operator, restricted to input-guarded processes, may be obtained as follows.

\begin{eqnarray}
\bangp{\prefix{u}{v}{P}} 
	:= 
	\binpar{\lift{x}{\prefix{u}{v}{(\binpar{D(x)}{P})}}}{D(x)} \nonumber
\end{eqnarray}

\begin{remark}
  Note that the lazier definition still does not deal with summation
  or mixed summation (i.e. sums over input and output). The reader is
  invited to construct definitions of replication that deal with these
  features. 

  Further, the definitions are parameterized in a name, $x$. Can you,
  gentle reader, make a definition that eliminates this parameter and
  guarantees no accidental interaction between the replication
  machinery and the process being replicated -- i.e. no accidental
  sharing of names used by the process to get its work done and the
  name(s) used by the replication to effect copying. This latter
  revision of the definition of replication is crucial to obtaining
  the expected identity $!!P \sim !P$.
\end{remark}

\begin{remark}\label{rem:paradoxical_combinator}
  The reader familiar with the lambda calculus will have noticed the
  similarity between $D$ and the paradoxical combinator.

  [Ed. note: the existence of this seems to suggest we have to be more
  restrictive on the set of processes and names we admit if we are to
  support no-cloning.]
\end{remark}

\subsubsection{Bisimulation}

The computational dynamics gives rise to another kind of equivalence,
the equivalence of computational behavior. As previously mentioned
this is typically captured \emph{via} some form of bisimulation.

% The notion we use in this paper is weak barbed bisimulation
% \cite{milner91polyadicpi}.

The notion we use in this paper is derived from weak barbed
bisimulation \cite{milner91polyadicpi}. 

\begin{definition}
An \emph{observation relation}, $\downarrow_{\mathcal N}$, over a set
of names, $\mathcal N$, is the smallest relation satisfying the rules
below.

\infrule[Out-barb]{y \in {\mathcal N}, \; x \nameeq y}
		  {\outputp{x}{v} \downarrow_{\mathcal N} x}
\infrule[Par-barb]{\mbox{$P\downarrow_{\mathcal N} x$ or $Q\downarrow_{\mathcal N} x$}}
		  {\binpar{P}{Q} \downarrow_{\mathcal N} x}

We write $P \Downarrow_{\mathcal N} x$ if there is $Q$ such that 
$P \wred Q$ and $Q \downarrow_{\mathcal N} x$.
\end{definition}

\begin{definition}
%\label{def.bbisim}
An  ${\mathcal N}$-\emph{barbed bisimulation} over a set of names, ${\mathcal N}$, is a symmetric binary relation 
${\mathcal S}_{\mathcal N}$ between agents such that $P\rel{S}_{\mathcal N}Q$ implies:
\begin{enumerate}
\item If $P \red P'$ then $Q \wred Q'$ and $P'\rel{S}_{\mathcal N} Q'$.
\item If $P\downarrow_{\mathcal N} x$, then $Q\Downarrow_{\mathcal N} x$.
\end{enumerate}
$P$ is ${\mathcal N}$-barbed bisimilar to $Q$, written
$P \wbbisim_{\mathcal N} Q$, if $P \rel{S}_{\mathcal N} Q$ for some ${\mathcal N}$-barbed bisimulation ${\mathcal S}_{\mathcal N}$.
\end{definition}

$\mathcal{R} \subseteq \pi \times \pi$

$P \mathcal{R} Q => \forall P'. P \red P' \Rightarrow \exists Q'. Q \red Q', P' \mathcal{R} Q'$

$P \vdash x \Rightarrow Q \vdash x$

\begin{mathpar}
  \inferrule*[lab=Out-barb]{x \nameeq y}{{y}!\langle{Q}\rangle \vdash x}
  \and
  \inferrule*[lab=Par-barb]{\mbox{$P\vdash x$ or $Q\vdash x$}}{\binpar{P}{Q} \vdash x}
\end{mathpar}

\subsubsection{Contexts}

One of the principle advantages of computational calculi like the
$\pi$-calculus is a well-defined notion of context,
contextual-equivalence and a correlation between
contextual-equivalence and notions of bisimulation. The notion of
context allows the decomposition of a process into (sub-)process and
its syntactic environment, its context. Thus, a context may be
thought of as a process with a ``hole'' (written $\Box$) in it. The
application of a context $M$ to a process $P$, written $M[P]$, is
tantamount to filling the hole in $M$ with $P$. In this paper we do
not need the full weight of this theory, but do make use of the notion
of context in the proof the main theorem. 

\begin{mathpar}
  \inferrule* [lab=summation] {} {{M_{M},M_{N}} \bc \Box \;|\; x.M_{A} \;|\; M_{M}+M_{N}}
  \and
  \inferrule* [lab=agent] {} {{M_{A}} \bc (\vec{x})M_{P} \;| \; \clift{P_0,\ldots,M_{P},\ldots,P_N}}
  \and \\
  \inferrule* [lab=process] {} {{M_{P}} \bc M_{N} \;| \;P|M_{P} }
\end{mathpar} 

\begin{mathpar}
  \inferrule* [lab=sychronization] {} {M_{N} \bc \Box \;|\; x?M_{F} \;|\; x!M_{C}}
  \and
  \inferrule* [lab=abstraction] {} {{M_{F}} \bc (x)M_{P} }
  \and
  \inferrule* [lab=concretion] {} {{M_{C}} \bc \langle M_{P} \rangle }
  \and \\
  \inferrule* [lab=process] {} {{M_{P}} \bc M_{N} \;| \;P|M_{P} }
\end{mathpar}

\begin{definition}[contextual application] Given a context $M$, and
  process $P$, we define the \emph{contextual application}, $M[P] :=
  M\{P/\Box\}$. That is, the contextual application of M to P is the
  substitution of $P$ for $\Box$ in $M$.
\end{definition}

$\meaningof{-} : L \to \mathcal{P}(\pi)$

\begin{mathpar}
  \inferrule* [lab=collection] {} {\meaningof{true} = \pi, \and \meaningof{~E} = \pi \setminus \meaningof{E}, \and \meaningof{E_{1} \& E_{2}} = \meaningof{E_{1}} \cap \meaningof{E_{2}}}
\end{mathpar}

\begin{mathpar}
  \inferrule* [lab=structure] {} {\meaningof{0} = \{ P \in \pi | P \equiv 0 \}, \and \\ \meaningof{E_1 | E_2} = \{ P \in \pi | P \equiv P_{1} | P_{2}, P_{1} \in \meaningof{E_{1}}, P_{2} \in \meaningof{E_2}\} }
\end{mathpar}

\begin{mathpar}
 \inferrule* [lab=behavior] {} {\meaningof{\langle a?b \rangle E} = \{ P \in \pi | P \equiv Q | u?(y)P', \\ \and \\\\ \and \\ \;\;\; u \in \meaningof{a}, \forall z.P'\{z/y\} \in \meaningof{E\{z/b\}}\}, \and \\ \meaningof{a!E} = \{ P \in \pi | P \equiv Q | x!\langle P' \rangle, x \in \meaningof{a} P' \in \meaningof{E}\} }
\end{mathpar}

\begin{mathpar}
 \inferrule* [lab=nominal] {} {\meaningof{\quotep{E}} = \{ \quotep{P} \in \quotep{\pi} | P \in \meaningof{E} \}, \and \meaningof{\quotep{P}} = \{ \quotep{Q} \in \quotep{\pi} | P \equiv Q \} \and \\ \meaningof{@\quotep{E}} = \{ P \in \pi | P \equiv @x, x \in \meaningof{E} \}}
\end{mathpar}

\begin{eqnarray*}
  \\
  \meaningof{-} : TS \to ST
\end{eqnarray*}

\begin{eqnarray*}
  \\
  L : TS \to ST
\end{eqnarray*}

\begin{eqnarray*}
  \\
  P \models E \iff P \in \meaningof{E}
\end{eqnarray*}

\begin{eqnarray*}
  P \approx_{L} Q \iff \forall E \in L. P \models E \iff Q \models E
\end{eqnarray*}

\begin{eqnarray*}
  P \approx_{K} Q
\end{eqnarray*}

\begin{eqnarray*}
  P \approx Q
\end{eqnarray*}

$\approx_{K} = \approx = \approx_{L}$

\subsubsection{Contextual duality}

Note that contexts extend the quotation operation to a family of
operations from processes to names. Given a context, $M$, we can
define a \emph{nominal context}, $\quotep{M}$ by $\quotep{M}[P] :=
\quotep{M[P]}$. To foreshadow what is to come we observe that these
operations enjoy a duality with processes very much like the duality
between vectors and maps from vectors to scalars.

Further, because the calculus is essentially higher-order, we have a
correspondence between contexts and processes. More specifically,
given a name $x$ and a context $M$ we can construct $M^{*}_{x}$ such
that 

\begin{mathpar}
  M^{*}_{x} | \lift{x}{P} \red M[P]
\end{mathpar}

namely,

\begin{mathpar}
  M^{*}_{x} := x?(u).M[\dropn{u}]
\end{mathpar}

The dependence of $M^{*}_{x}$ on a name makes it an abstraction, 

\begin{mathpar}
  M^{*} := (x)x?(u).M[\dropn{u}]
\end{mathpar}

\subsection{Additional notation}

It will sometimes be convenient to denote the process a name
quotes. We already have the notation $x = \quotep{P}$, but it will be
convenient to introduce an alternate notation, $\procn{x}$, when we
want to emphasize the connection to the use of the name. Note that, by
virtue of name equivalence, $\quotep{\procn{x}} \nameeq x$; so, the
notation is consistent with previous definitions.

Further, because names have structure it is possible to effect
substitutions on the basis of that structure. This means we need to
upgrade our notation for substitutions, which we accomplish by
adapting comprehension notation. Thus,

\begin{mathpar}
  P\{ y / x : x \in S \}
\end{mathpar}

is interpreted to mean the process derived from P by replacing (in a
capture-avoiding manner) each occurrence of $x$ in $S$ by $y$. For example,

\begin{mathpar}
  P\{ \quotep{\procn{x}|\procn{x}} / x : x \in \freenames{P} \}
\end{mathpar}

will replace each (occurrence) of a free name $x$ in $P$ by
$\quotep{\procn{x}|\procn{x}}$.

Also, we will avail ourselves of the notation $x^{L}$ and $x^{R}$ to
denote injections of a name into disjoint copies of the name
space. There are numerous ways to accomplish this. One example can be
found in \cite{MeredithR05}. This notation overloads to vectors of
names: $\vec{x}^{\pi} := (x_{i}^{\pi} \; : \; 0 \leq i < |\vec{x}| )$ where $\pi \in \{L,R\}$.

We also use $P^{\Box} := P|\Box$.

In \cite{MeredithR05} an interpretation of the new operator is
given. It turns out that there are several possible interpretations
all enjoying the requisite algebraic properties of the operator (see
\cite{milner91polyadicpi}). We will therefore make liberal use of
$(\nu\; \vec{x})P$.

% subsection the_syntax_and_semantics_of_the_notation_system (end)   

\input{qm2pi.qmops} 

\input{qm2pi.sterngerlach} 

\input{qm2pi.metric} 

% section concurrent_process_calculi (end)

%\input{qm2pi.proofsketch}

% section proof sketch (end)

%\input{qm2pi.slviaknots} 

% section spatial logic via knots (end)

\input{qm2pi.conclusion}

% section conclusion (end)

%\input{qm2pi.dtcodes} 

% section wiring algorithm (end)

\input{qm2pi.ack} 

% section acknowledgments (end)

\newpage


\bibliographystyle{plain}   
\bibliography{../../biblios/main.bib}

\input{qm2pi.rhodetails}

\end{document}

 

%\ifpdf
%\usepackage[pdftex]{graphicx}
%\else
%\usepackage{graphicx}
%\fi

 % \ifpdf
%  \usepackage{pdfsync}
%  \if


%\title{Brief Article}
%\author{David F. Snyder}
%\author{L.G. Meredith}

%\address{Dept. of Math., Texas State University--San Marcos, San Marcos, TX 78666}
       
\pagestyle{empty}


\begin{document}

\lstset{language=[Objective]Caml,frame=shadowbox}

\documentclass[12pt]{llncs}
%\documentclass{jktr}

\usepackage[pdftex]{hyperref}                   
\usepackage {listings}
\usepackage {mathpartir}
\usepackage{bcprules}
%\usepackage{listings}
                       
\usepackage{graphicx} 
%\usepackage[margins=2.5cm,nohead,nofoot]{geometry}
%\usepackage{geometry}
\usepackage{amsfonts}
\usepackage{amstext}
\usepackage{latexsym}
\usepackage{amssymb}
\usepackage{color}


%\include{myPreamble}
\include{qm2pi.local} 

%\ifpdf
%\usepackage[pdftex]{graphicx}
%\else
%\usepackage{graphicx}
%\fi

 % \ifpdf
%  \usepackage{pdfsync}
%  \if


%\title{Brief Article}
%\author{David F. Snyder}
%\author{L.G. Meredith}

%\address{Dept. of Math., Texas State University--San Marcos, San Marcos, TX 78666}
       
\pagestyle{empty}


\begin{document}

\lstset{language=[Objective]Caml,frame=shadowbox}

\input{qm2pi.front}

% section front matter (end)

\input{qm2pi.intro} 
 
% section introduction (end)

% \input{qm2pi.knotations} 

% section notation (end)

\input{qm2pi.process.calculi} 

% section concurrent_process_calculi_and_spatial_logics_ (end)
    
%\input{qm2pi.knots2pi} 

%\input{qm2pi.trefoil} 

%\input{qm2pi.mainthm} 

% subsection basic_interpretation (end)

%\input{qm2pi.rho.presentation} 
\subsection{The syntax and semantics of the notation system}\label{sub:the_syntax_and_semantics_of_the_notation_system} % (fold)

We now summarize a technical presentation of the calculus that
embodies our theory of dynamics. The typical presentation of such a
calculus follows the style of giving generators and relations on
them. The grammar, below, describing term constructors, freely
generates the set of processes, $\Proc$. This set is then quotiented
by a relation known as structural congruence and it is over this set
that the notion of dynamics is expressed. This presentation is
essentially that of \cite{MeredithR05} with the addition of
polyadicity and summation. For readability we have relegated some of
the technical subtleties to an appendix.

\subsubsection{Process grammar}\label{subsub:process_grammar}

\begin{mathpar}
  \inferrule* [lab=synchronization] {} {{M} \bc \pzero \;|\; x?F \;|\; x!C }
  \and
  \inferrule* [lab=abstraction] {} {{F} \bc (x)P}
  \and
  \inferrule* [lab=concretion] {} {{C} \bc \langle Q \rangle}
  \and
  \inferrule* [lab=process] {} {{P,Q} \bc M \;| \;P|Q \;|\; @{x}}
  \and
  \inferrule* [lab=name] {} {{x} \bc \quotep{P}}
\end{mathpar} 

Note that $\vec{x}$ (resp. $\vec{P}$) denotes a vector of names
(resp. processes) of length $|\vec{x}|$ (resp. $|\vec{P}|$). We adopt
the following useful abbreviations.

\begin{mathpar}
   x?(\vec{y}).P := x.(\vec{y})P \and  x\clift{\vec{P}} := x.\clift{\vec{P}}
   \and x!(y) := \lift{x}{\dropn{y}}
   \and \Pi_{i=0}^{n-1}P_i := P_0 | \ldots | P_{n-1}
\end{mathpar}

\subsubsection{Structural congruence}

\paragraph{Free and bound names and alpha-equivalence.} At the
core of structural equivalence is alpha-equivalence which identifies
process that are the same up to a change of variable. Formally, we
recognize the distinction between free and bound names. The free names
of a process, $\freenames{P}$, may be calculated recursively as
follows:

\begin{mathpar}
\freenames{\pzero} := \emptyset
  \and \\
  \freenames{x?(y).P} := \{ x \} \cup (\freenames{P} \setminus \{ y \})
  \and 
  \freenames{x!\langle P \rangle} := \{ x \} \cup \{ P \} 
  \and \\
  \freenames{P|Q} := \freenames{P} \cup \freenames{Q}
  \and \\
  \freenames{@{x}} := \{ x \}
\end{mathpar}

$\pi$
$\quotep{\pi}$

$\freenames{-} : \pi \to \mathcal{P}(\quotep{\pi})$

\begin{eqnarray*}
  \freenames{\pzero} & := & \emptyset \\
  \freenames{x?(y).P} & := & \{ x \} \cup (\freenames{P} \setminus \{ y \}) \\
  \freenames{x!\langle P \rangle} & := & \{ x \} \cup \{ P \} \\
  \freenames{P|Q} & := & \freenames{P} \cup \freenames{Q} \\
  \freenames{\dropn{x}} & := & \{ x \}
\end{eqnarray*}

The bound names of a process, $\boundnames{P}$, are those names occurring in $P$
that are not free. For example, in $x?(y).0$, the name $x$ is free, while $y$ is bound.

\begin{mathpar}
  \inferrule* [lab=monoidal-laws] {} { P|Q \equiv Q|P \and P|0 \equiv P \and P|(Q|R) \equiv (P|Q)|R }
\end{mathpar}

\begin{mathpar}
  \inferrule* [lab=alpha-equivalence] {} { (x)P \equiv (y)P\{y/x\} \and y \not\in \freenames{P} }
\end{mathpar}

\begin{definition}
Then two processes, $P,Q$, are alpha-equivalent if $P = Q\{\vec{y}/\vec{x}\}$ for
some $\vec{x} \in \boundnames{Q},\vec{y} \in \boundnames{P}$, where $Q\{\vec{y}/\vec{x}\}$
denotes the capture-avoiding substitution of $\vec{y}$ for $\vec{x}$ in $Q$.
\end{definition}

\begin{definition}
  The {\em structural congruence} \cite{SangiorgiWalker} , $\equiv$,
  between processes is the least congruence containing
  alpha-equivalence, satisfying the abelian monoid laws
  (associativity, commutativity and $\pzero$ as identity) for parallel
  composition $|$ and for summation $+$.
\end{definition}

\subsection{Name equivalence}

We take name equivalence, written $\nameeq$, to be the smallest
equivalence relation generated by the following rules.

\begin{mathpar}
\inferrule*[lab=Quote-drop]
{ }
{ \quotep{@{x}} \nameeq x }

\inferrule*[lab=Struct-equiv]
{ P \scong Q }
{ \quotep{P} \nameeq \quotep{Q} }
\end{mathpar}

The astute reader will have noticed that the mutual recursion of names
and processes imposes a mutual recursion on alpha-equivalence and
structural equivalence via name-equivalence. Fortunately, all of this
works out pleasantly and we may calculate in the natural way, free of
concern. The reader interested in the details is referred to the
appendix \ref{appendix:rho_details}.

\subsection{Substitution}

We use $\Proc$ for the set of processes, $\QProc$ for the set of
names, and $\id{\{}\vec{y} / \vec{x} \id{\}}$ to denote partial maps,
$s : \QProc \rightarrow \QProc$. A map, $s$ lifts, uniquely, to a map
on process terms, $\widehat{s} : \Proc \rightarrow \Proc$ by the
following equations.

\begin{mathpar}
  (0) \psubstp{Q}{P} := 0 \\
  (R \juxtap S) \psubstp{Q}{P}
  :=    
  (R)\psubstp{Q}{P} \juxtap (S) \psubstp{Q}{P} \\
  (x?(y).R) \psubstp{Q}{P}    
  :=    
  (x)\substp{Q}{P} (z)\concat( (R \psubstn{z}{y}) \psubstp{Q}{P} ) \\
  (\lift{x}{R}) \psubstp{Q}{P}  
  :=
  \lift{(x)\substp{Q}{P}}{ R \psubstp{Q}{P} } \\
%   (\dropn{x})  \psubstp{Q}{P}       
%   := 
%   \left\{ 
%     \begin{array}{ccc} 
%       \dropn{\quotep{Q}} & & x \nameeq \quotep{P} \\
%       \dropn{x} & & otherwise \\
%     \end{array}
%   \right. 
  (\dropn{x})  \psubstp{Q}{P}       
  := 
  \left\{ 
    \begin{array}{ccc} 
      Q & & x \nameeq \quotep{P} \\
      \dropn{x} & & otherwise \\
    \end{array}
  \right.
\end{mathpar}
 

where

\begin{eqnarray}
  (x)\id{\{} \lpquote Q \rpquote / \lpquote P \rpquote \id{\}}            = 
  \left\{ 
    \begin{array}{ccc}
      \lpquote Q \rpquote & & x \nameeq \lpquote P \rpquote \\
      x & & otherwise \\
    \end{array}
  \right. \nonumber
\end{eqnarray}

and $z$ is chosen distinct from $\quotep{P}$, $\quotep{Q}$, the free
names in $Q$, and all the names in $R$. Our $\alpha$-equivalence will
be built in the standard way from this substitution.

\begin{remark}\label{rem:no_self_referential_names}
  One consequence of these definitions is that $\forall P. \quotep{P}
  \not\in \freenames{P}$.
\end{remark}

\subsection{ Dynamic quote: an example }

Anticipating something of what's to come, consider applying the
substitution, $\widehat{\id{\{}u / z \id{\}}}$, to the following pair
of processes, $\lift{w}{y!(z)}$ and $w[ \lpquote y!(z) \rpquote ]$.

\begin{eqnarray}
	\lift{w}{y!(z)}\widehat{\id{\{}u / z \id{\}}}
		& = &
		\lift{w}{y!(u)} \nonumber\\
	w[ \lpquote y!(z) \rpquote ] \widehat{ \id{\{}u / z \id{\}} }
		& = &
		w[ \lpquote y!(z) \rpquote ] \nonumber
\end{eqnarray}

Because the body of the process between quotes is impervious to
substitution, we get radically different answers. In fact, by
examining the first process in an input context,
e.g. $x?(z).\lift{w}{y!(z)}$, we see that the process under the lift
operator may be shaped by prefixed inputs binding a name inside it. In
this sense, the lift operator will be seen as a way to dynamically
construct processes before reifying them as names.

Finally equipped with these standard features we can present the
dynamics of the calculus.

\subsubsection{Operational semantics} 

Finally, we introduce the computational dynamics. What marks these
algebras as distinct from other more traditionally studied algebraic
structures, e.g. vector spaces or polynomial rings, is the manner in
which dynamics is captured. In traditional structures, dynamics is typically
expressed through morphisms between such structures, as in linear maps
between vector spaces or morphisms between rings. In algebras
associated with the semantics of computation, the dynamics is
expressed as part of the algebraic structure itself, through a
reduction reduction relation typically denoted by $\red$. Below, we
give a recursive presentation of this relation for the calculus used
in the encoding.

$\red \subseteq \pi \times \pi$
$\red : \pi \to \mathcal{P}(\pi)$

\begin{mathpar}
  \inferrule* [lab=Comm] { \textsf{match}( x_{src}, x_{trgt} ) } { x_{trgt}?(y)P \; | \; x_{src}!\langle {Q} \rangle \red P\{\quotep{Q}/y}\} }
  \and \\
  \inferrule* [lab=Par] {{P} \red {P}'} {{{P} | {Q}} \red {{P}' | {Q}}}
  \and
  \inferrule* [lab=Equiv]{{{P} \scong {P}'} \andalso {{P}' \red {Q}'} \andalso {{Q}' \scong {Q}}}{{P} \red {Q}}
\end{mathpar}

\begin{eqnarray*}
  match_{\equiv} (\quotep{P},\quotep{Q}) & := & P \equiv Q \\
  match_{\dagger}(\quotep{P},\quotep{Q}) & := & \forall R. P|Q \red^{*} R => R \red^{*} 0 \\
  match_{K}(\quotep{P},\quotep{Q}) & := & K \mbox{ for some context } K
\end{eqnarray*}

$u?(x)P | u!\langle Q \rangle \red P\{\quotep{Q}/x\}$

%We write $\wred$ for $\red^*$, and $P\red$ if $\exists Q $ such that $ P \red Q$.
We write $P\red$ if $\exists Q $ such that $ P \red Q$ and $P\not\red$, otherwise.

\section{Replication}

As mentioned before, it is known that replication (and hence
recursion) can be implemented in a higher-order process algebra
\cite{SangiorgiWalker}. As our first example of calculation with the
machinery thus far presented we give the construction explicitly in
the {\rhoc}.

\begin{eqnarray}
	D_{x} & := & \prefix{x}{y}{(\binpar{\outputp{x}{y}}{@{y}})} \nonumber\\
	\bangp_{x}{P} & := & \binpar{{x}!\langle{\binpar{D_{x}}{P}}\rangle}{D_{x}} \nonumber
\end{eqnarray}

\begin{eqnarray}
	\bangp_{x}{P} & & \nonumber\\
	=
	& {x}!\langle{(\prefix{x}{y}{(\outputp{x}{y} | @{y})) | P}}\rangle 
	      | \prefix{x}{y}{(\outputp{x}{y} | @{y})} & \nonumber\\
	\red
	& (\outputp{x}{y} | @{y})\substn{\quotep{(\prefix{x}{y}{(@{y} | \outputp{x}{y})) | P}}}{y} & \nonumber\\
	=
	& \outputp{x}{\quotep{(\prefix{x}{y}{(\outputp{x}{y} | @{y})) | P}}}
	  | {(\prefix{x}{y}{(\outputp{x}{y} | @{y})) | P}} & \nonumber\\
	\red
	& \ldots & \nonumber\\
	\red^*
	& P | P | \ldots & \nonumber
\end{eqnarray}

Of course, this encoding, as an implementation, runs away, unfolding
$\bangp{P}$ eagerly. A lazier and more implementable replication
operator, restricted to input-guarded processes, may be obtained as follows.

\begin{eqnarray}
\bangp{\prefix{u}{v}{P}} 
	:= 
	\binpar{\lift{x}{\prefix{u}{v}{(\binpar{D(x)}{P})}}}{D(x)} \nonumber
\end{eqnarray}

\begin{remark}
  Note that the lazier definition still does not deal with summation
  or mixed summation (i.e. sums over input and output). The reader is
  invited to construct definitions of replication that deal with these
  features. 

  Further, the definitions are parameterized in a name, $x$. Can you,
  gentle reader, make a definition that eliminates this parameter and
  guarantees no accidental interaction between the replication
  machinery and the process being replicated -- i.e. no accidental
  sharing of names used by the process to get its work done and the
  name(s) used by the replication to effect copying. This latter
  revision of the definition of replication is crucial to obtaining
  the expected identity $!!P \sim !P$.
\end{remark}

\begin{remark}\label{rem:paradoxical_combinator}
  The reader familiar with the lambda calculus will have noticed the
  similarity between $D$ and the paradoxical combinator.

  [Ed. note: the existence of this seems to suggest we have to be more
  restrictive on the set of processes and names we admit if we are to
  support no-cloning.]
\end{remark}

\subsubsection{Bisimulation}

The computational dynamics gives rise to another kind of equivalence,
the equivalence of computational behavior. As previously mentioned
this is typically captured \emph{via} some form of bisimulation.

% The notion we use in this paper is weak barbed bisimulation
% \cite{milner91polyadicpi}.

The notion we use in this paper is derived from weak barbed
bisimulation \cite{milner91polyadicpi}. 

\begin{definition}
An \emph{observation relation}, $\downarrow_{\mathcal N}$, over a set
of names, $\mathcal N$, is the smallest relation satisfying the rules
below.

\infrule[Out-barb]{y \in {\mathcal N}, \; x \nameeq y}
		  {\outputp{x}{v} \downarrow_{\mathcal N} x}
\infrule[Par-barb]{\mbox{$P\downarrow_{\mathcal N} x$ or $Q\downarrow_{\mathcal N} x$}}
		  {\binpar{P}{Q} \downarrow_{\mathcal N} x}

We write $P \Downarrow_{\mathcal N} x$ if there is $Q$ such that 
$P \wred Q$ and $Q \downarrow_{\mathcal N} x$.
\end{definition}

\begin{definition}
%\label{def.bbisim}
An  ${\mathcal N}$-\emph{barbed bisimulation} over a set of names, ${\mathcal N}$, is a symmetric binary relation 
${\mathcal S}_{\mathcal N}$ between agents such that $P\rel{S}_{\mathcal N}Q$ implies:
\begin{enumerate}
\item If $P \red P'$ then $Q \wred Q'$ and $P'\rel{S}_{\mathcal N} Q'$.
\item If $P\downarrow_{\mathcal N} x$, then $Q\Downarrow_{\mathcal N} x$.
\end{enumerate}
$P$ is ${\mathcal N}$-barbed bisimilar to $Q$, written
$P \wbbisim_{\mathcal N} Q$, if $P \rel{S}_{\mathcal N} Q$ for some ${\mathcal N}$-barbed bisimulation ${\mathcal S}_{\mathcal N}$.
\end{definition}

$\mathcal{R} \subseteq \pi \times \pi$

$P \mathcal{R} Q => \forall P'. P \red P' \Rightarrow \exists Q'. Q \red Q', P' \mathcal{R} Q'$

$P \vdash x \Rightarrow Q \vdash x$

\begin{mathpar}
  \inferrule*[lab=Out-barb]{x \nameeq y}{{y}!\langle{Q}\rangle \vdash x}
  \and
  \inferrule*[lab=Par-barb]{\mbox{$P\vdash x$ or $Q\vdash x$}}{\binpar{P}{Q} \vdash x}
\end{mathpar}

\subsubsection{Contexts}

One of the principle advantages of computational calculi like the
$\pi$-calculus is a well-defined notion of context,
contextual-equivalence and a correlation between
contextual-equivalence and notions of bisimulation. The notion of
context allows the decomposition of a process into (sub-)process and
its syntactic environment, its context. Thus, a context may be
thought of as a process with a ``hole'' (written $\Box$) in it. The
application of a context $M$ to a process $P$, written $M[P]$, is
tantamount to filling the hole in $M$ with $P$. In this paper we do
not need the full weight of this theory, but do make use of the notion
of context in the proof the main theorem. 

\begin{mathpar}
  \inferrule* [lab=summation] {} {{M_{M},M_{N}} \bc \Box \;|\; x.M_{A} \;|\; M_{M}+M_{N}}
  \and
  \inferrule* [lab=agent] {} {{M_{A}} \bc (\vec{x})M_{P} \;| \; \clift{P_0,\ldots,M_{P},\ldots,P_N}}
  \and \\
  \inferrule* [lab=process] {} {{M_{P}} \bc M_{N} \;| \;P|M_{P} }
\end{mathpar} 

\begin{mathpar}
  \inferrule* [lab=sychronization] {} {M_{N} \bc \Box \;|\; x?M_{F} \;|\; x!M_{C}}
  \and
  \inferrule* [lab=abstraction] {} {{M_{F}} \bc (x)M_{P} }
  \and
  \inferrule* [lab=concretion] {} {{M_{C}} \bc \langle M_{P} \rangle }
  \and \\
  \inferrule* [lab=process] {} {{M_{P}} \bc M_{N} \;| \;P|M_{P} }
\end{mathpar}

\begin{definition}[contextual application] Given a context $M$, and
  process $P$, we define the \emph{contextual application}, $M[P] :=
  M\{P/\Box\}$. That is, the contextual application of M to P is the
  substitution of $P$ for $\Box$ in $M$.
\end{definition}

$\meaningof{-} : L \to \mathcal{P}(\pi)$

\begin{mathpar}
  \inferrule* [lab=collection] {} {\meaningof{true} = \pi, \and \meaningof{~E} = \pi \setminus \meaningof{E}, \and \meaningof{E_{1} \& E_{2}} = \meaningof{E_{1}} \cap \meaningof{E_{2}}}
\end{mathpar}

\begin{mathpar}
  \inferrule* [lab=structure] {} {\meaningof{0} = \{ P \in \pi | P \equiv 0 \}, \and \\ \meaningof{E_1 | E_2} = \{ P \in \pi | P \equiv P_{1} | P_{2}, P_{1} \in \meaningof{E_{1}}, P_{2} \in \meaningof{E_2}\} }
\end{mathpar}

\begin{mathpar}
 \inferrule* [lab=behavior] {} {\meaningof{\langle a?b \rangle E} = \{ P \in \pi | P \equiv Q | u?(y)P', \\ \and \\\\ \and \\ \;\;\; u \in \meaningof{a}, \forall z.P'\{z/y\} \in \meaningof{E\{z/b\}}\}, \and \\ \meaningof{a!E} = \{ P \in \pi | P \equiv Q | x!\langle P' \rangle, x \in \meaningof{a} P' \in \meaningof{E}\} }
\end{mathpar}

\begin{mathpar}
 \inferrule* [lab=nominal] {} {\meaningof{\quotep{E}} = \{ \quotep{P} \in \quotep{\pi} | P \in \meaningof{E} \}, \and \meaningof{\quotep{P}} = \{ \quotep{Q} \in \quotep{\pi} | P \equiv Q \} \and \\ \meaningof{@\quotep{E}} = \{ P \in \pi | P \equiv @x, x \in \meaningof{E} \}}
\end{mathpar}

\begin{eqnarray*}
  \\
  \meaningof{-} : TS \to ST
\end{eqnarray*}

\begin{eqnarray*}
  \\
  L : TS \to ST
\end{eqnarray*}

\begin{eqnarray*}
  \\
  P \models E \iff P \in \meaningof{E}
\end{eqnarray*}

\begin{eqnarray*}
  P \approx_{L} Q \iff \forall E \in L. P \models E \iff Q \models E
\end{eqnarray*}

\begin{eqnarray*}
  P \approx_{K} Q
\end{eqnarray*}

\begin{eqnarray*}
  P \approx Q
\end{eqnarray*}

$\approx_{K} = \approx = \approx_{L}$

\subsubsection{Contextual duality}

Note that contexts extend the quotation operation to a family of
operations from processes to names. Given a context, $M$, we can
define a \emph{nominal context}, $\quotep{M}$ by $\quotep{M}[P] :=
\quotep{M[P]}$. To foreshadow what is to come we observe that these
operations enjoy a duality with processes very much like the duality
between vectors and maps from vectors to scalars.

Further, because the calculus is essentially higher-order, we have a
correspondence between contexts and processes. More specifically,
given a name $x$ and a context $M$ we can construct $M^{*}_{x}$ such
that 

\begin{mathpar}
  M^{*}_{x} | \lift{x}{P} \red M[P]
\end{mathpar}

namely,

\begin{mathpar}
  M^{*}_{x} := x?(u).M[\dropn{u}]
\end{mathpar}

The dependence of $M^{*}_{x}$ on a name makes it an abstraction, 

\begin{mathpar}
  M^{*} := (x)x?(u).M[\dropn{u}]
\end{mathpar}

\subsection{Additional notation}

It will sometimes be convenient to denote the process a name
quotes. We already have the notation $x = \quotep{P}$, but it will be
convenient to introduce an alternate notation, $\procn{x}$, when we
want to emphasize the connection to the use of the name. Note that, by
virtue of name equivalence, $\quotep{\procn{x}} \nameeq x$; so, the
notation is consistent with previous definitions.

Further, because names have structure it is possible to effect
substitutions on the basis of that structure. This means we need to
upgrade our notation for substitutions, which we accomplish by
adapting comprehension notation. Thus,

\begin{mathpar}
  P\{ y / x : x \in S \}
\end{mathpar}

is interpreted to mean the process derived from P by replacing (in a
capture-avoiding manner) each occurrence of $x$ in $S$ by $y$. For example,

\begin{mathpar}
  P\{ \quotep{\procn{x}|\procn{x}} / x : x \in \freenames{P} \}
\end{mathpar}

will replace each (occurrence) of a free name $x$ in $P$ by
$\quotep{\procn{x}|\procn{x}}$.

Also, we will avail ourselves of the notation $x^{L}$ and $x^{R}$ to
denote injections of a name into disjoint copies of the name
space. There are numerous ways to accomplish this. One example can be
found in \cite{MeredithR05}. This notation overloads to vectors of
names: $\vec{x}^{\pi} := (x_{i}^{\pi} \; : \; 0 \leq i < |\vec{x}| )$ where $\pi \in \{L,R\}$.

We also use $P^{\Box} := P|\Box$.

In \cite{MeredithR05} an interpretation of the new operator is
given. It turns out that there are several possible interpretations
all enjoying the requisite algebraic properties of the operator (see
\cite{milner91polyadicpi}). We will therefore make liberal use of
$(\nu\; \vec{x})P$.

% subsection the_syntax_and_semantics_of_the_notation_system (end)   

\input{qm2pi.qmops} 

\input{qm2pi.sterngerlach} 

\input{qm2pi.metric} 

% section concurrent_process_calculi (end)

%\input{qm2pi.proofsketch}

% section proof sketch (end)

%\input{qm2pi.slviaknots} 

% section spatial logic via knots (end)

\input{qm2pi.conclusion}

% section conclusion (end)

%\input{qm2pi.dtcodes} 

% section wiring algorithm (end)

\input{qm2pi.ack} 

% section acknowledgments (end)

\newpage


\bibliographystyle{plain}   
\bibliography{../../biblios/main.bib}

\input{qm2pi.rhodetails}

\end{document}



% section front matter (end)

\section{Introduction}\label{sec:introduction} % (fold)
In this draft of the material i am going to have to dispense with the
usual writing conventions adopted in papers on these topics. i'm going
to have adopt whatever tone i need at the time i'm writing up the
calculations. Sometimes this may be very conversational; others it may
be the barest mathematical grunts; others still it may be that i have
lifted text from one of my other papers because the exposition of some
point was better said there. i hope that my readers are not unduly put
out by this decision. i'm not doing this to flout convention or be
rebellious. i find these calculations very technically challenging. To
keep everything going technically, something has to give; i have to
let go of some cognitive burden. So, the academic writing style --
with all of its trade-offs in terms of facilitating technical
communication -- is what i'm letting go of. Perhaps subsequent drafts
can be tightened and polished, but for now, i'm going to speak as if
we were sitting together in a coffee shop with a laptop, wifi and a
pad of paper and a pencil.

So, here's what i have to say. We -- you and i, comfortably ensconced
in our coffee shop and well-equipped with our tools -- can realize and
carry out the calculations of quantum mechanics over a very different
formal theory of dynamics, a formal theory of dynamics that
corresponds to a theory of concurrent computation with
\emph{reflection}. It has the advantage that the underlying theory is
already `quantized', but supports analogues all of the continuuous
operations. Strikingly, this underlying theory has recently been
connected with a notion of metric that we can show, by calculating
together, coincides with the metric induced by the inner product.

There are a lot of reasons why you might be interested in seeing
calculations of this form. Here's why i'm interested. For the past
several centuries there has been no competitor to the ``Newtonian''
account of dynamics. As a result the predominant share of accounts of
dynamical systems and situations have had to be formulated in terms of
the Newtonian machinery. i view this as an intellectually dangerous
position to occupy. Everything, despite it's intrinsic shape, turns
into a nail to be hit with this hammer. Recently, however, the theory
of computation has matured to the point where we have candidates for
theories of dynamics that offer very different perspective on
reasoning about dynamical systems and situations. Testing these
candidates against very successful accounts of dynamical situations,
like quantum mechanics, is going to give us some sense of how mature
they are and some measure of the quality of these accounts of
dynamics.

\subsection{Summary of contributions and outline of paper}

So, we're going to develop an interpretation of the operations of
quantum mechanics normally interpreted by Hilbert spaces and
operators. We're going to do this over a theory of computation. Note
that this is very different than the usual quantum computation program
which develops notions of computation over quantum mechanics. Rather,
we are developing a story that aligns with Wheeler's slogan: It from
Bit. To do this we will first provide an account of the theory of
computation at play here. Then we will dive into a calculation-driven
interpretation of the operations of quantum mechanics.

The reason we take this approach is that -- until very recently --
there hasn't been an axiomatic account of quantum mechanics. As a
result there has been no sharp delineation of the mathematical theory
supporting interpretation of the physical theory and the physical
theory, itself. So, ambient features of the maths are free to be
exploited (or supressed) without a real accounting of their physical
relevance. There is no sharp statement ``here's the physical theory''
qua \emph{theory} and ``here's the mathematical interpretation''
enabling a judgment of how faithful the interpretation is -- apart
from experimental observation. When there is an axiomatic account we
can judge how well a given mathematical formalism supports an
interpretation of the axioms, independent of
experimentation. Likewise, we can judge how well we have captured our
physical evidence and experience with our axiomatics, independent of
any specific mathematical implementation, with accidental detail that
may or may not have physical significance. 

In lieu of a fully fleshed out and vetted axiomatic account of quantum
mechanics, interpreting the operational notions in service of modeling
physical systems will have to suffice. In other words, we are not in
the business of providing a model of Hilbert spaces and operators. We
are in the business of providing a model of quantum mechanics because
we are motivated by testing our notions of dynamics against physical
theory; and, the predictive calculations of the physical theory must
serve as the best formulation -- shy of a fully fleshed out axiomatic
account -- of the physical theory itself (as they have for scientific
theories since time immemorial). Put another way, despite a
whole-hearted commitment to an It-from-Bit ontology, we are firmly
aligned with the shut-up-and-calculate camp as the best way to obtain
results either from the physical perspective or as a quality assurance
measure of our fledgling theory of dynamics.

In detail, we present a reflective process calculus. Then we develop
intuitive correspondences between the notions available in this
calculus and the usual physical notions supporting quantum mechanical
calculations. Thus, 

\begin{table}[htp]
  \center{
    \fbox{
      \begin{tabular}{c|c}
        quantum mechanics & process calculus \\
        \hline
        scalar & name \\
        state vector & process \\
        dual & contextual duals \\
        matrix & formal sums of process-context-dual pairs \\
        orthogonality & process annihilation \\
        inner product & execution-formula + quoting
      \end{tabular}
    }
  }
  \caption{QM - process calculi correspondences}
\end{table}

Then we tighten up these intuitions to operational definitions. We
employ the Dirac notation as the best proxy we can find for an
abstract syntax of the quantum mechanical notions. The definitions we
develop put us in contact with equational constraints coming from the
theory that we demonstrate the definitions and calculations satisfy.

This puts us in a position to shut up and calculate for the
Stern-Gerlach experimental set up, showing how these predictive
calculations become calculations on processes in our theory of a
reflective process calculus.

Penultimately, we demonstrate that the notion of metric coming from
the inner product coincides with the notion of metric available from
the theory of bisimulation. This demonstration gives us the right to
think of space as arising from behavior. Finally, we consider where we
might go from the new vantage point we have obtained.

% section introduction (end) 
 
% section introduction (end)

% \documentclass[12pt]{llncs}
%\documentclass{jktr}

\usepackage[pdftex]{hyperref}                   
\usepackage {listings}
\usepackage {mathpartir}
\usepackage{bcprules}
%\usepackage{listings}
                       
\usepackage{graphicx} 
%\usepackage[margins=2.5cm,nohead,nofoot]{geometry}
%\usepackage{geometry}
\usepackage{amsfonts}
\usepackage{amstext}
\usepackage{latexsym}
\usepackage{amssymb}
\usepackage{color}


%\include{myPreamble}
\include{qm2pi.local} 

%\ifpdf
%\usepackage[pdftex]{graphicx}
%\else
%\usepackage{graphicx}
%\fi

 % \ifpdf
%  \usepackage{pdfsync}
%  \if


%\title{Brief Article}
%\author{David F. Snyder}
%\author{L.G. Meredith}

%\address{Dept. of Math., Texas State University--San Marcos, San Marcos, TX 78666}
       
\pagestyle{empty}


\begin{document}

\lstset{language=[Objective]Caml,frame=shadowbox}

\input{qm2pi.front}

% section front matter (end)

\input{qm2pi.intro} 
 
% section introduction (end)

% \input{qm2pi.knotations} 

% section notation (end)

\input{qm2pi.process.calculi} 

% section concurrent_process_calculi_and_spatial_logics_ (end)
    
%\input{qm2pi.knots2pi} 

%\input{qm2pi.trefoil} 

%\input{qm2pi.mainthm} 

% subsection basic_interpretation (end)

%\input{qm2pi.rho.presentation} 
\subsection{The syntax and semantics of the notation system}\label{sub:the_syntax_and_semantics_of_the_notation_system} % (fold)

We now summarize a technical presentation of the calculus that
embodies our theory of dynamics. The typical presentation of such a
calculus follows the style of giving generators and relations on
them. The grammar, below, describing term constructors, freely
generates the set of processes, $\Proc$. This set is then quotiented
by a relation known as structural congruence and it is over this set
that the notion of dynamics is expressed. This presentation is
essentially that of \cite{MeredithR05} with the addition of
polyadicity and summation. For readability we have relegated some of
the technical subtleties to an appendix.

\subsubsection{Process grammar}\label{subsub:process_grammar}

\begin{mathpar}
  \inferrule* [lab=synchronization] {} {{M} \bc \pzero \;|\; x?F \;|\; x!C }
  \and
  \inferrule* [lab=abstraction] {} {{F} \bc (x)P}
  \and
  \inferrule* [lab=concretion] {} {{C} \bc \langle Q \rangle}
  \and
  \inferrule* [lab=process] {} {{P,Q} \bc M \;| \;P|Q \;|\; @{x}}
  \and
  \inferrule* [lab=name] {} {{x} \bc \quotep{P}}
\end{mathpar} 

Note that $\vec{x}$ (resp. $\vec{P}$) denotes a vector of names
(resp. processes) of length $|\vec{x}|$ (resp. $|\vec{P}|$). We adopt
the following useful abbreviations.

\begin{mathpar}
   x?(\vec{y}).P := x.(\vec{y})P \and  x\clift{\vec{P}} := x.\clift{\vec{P}}
   \and x!(y) := \lift{x}{\dropn{y}}
   \and \Pi_{i=0}^{n-1}P_i := P_0 | \ldots | P_{n-1}
\end{mathpar}

\subsubsection{Structural congruence}

\paragraph{Free and bound names and alpha-equivalence.} At the
core of structural equivalence is alpha-equivalence which identifies
process that are the same up to a change of variable. Formally, we
recognize the distinction between free and bound names. The free names
of a process, $\freenames{P}$, may be calculated recursively as
follows:

\begin{mathpar}
\freenames{\pzero} := \emptyset
  \and \\
  \freenames{x?(y).P} := \{ x \} \cup (\freenames{P} \setminus \{ y \})
  \and 
  \freenames{x!\langle P \rangle} := \{ x \} \cup \{ P \} 
  \and \\
  \freenames{P|Q} := \freenames{P} \cup \freenames{Q}
  \and \\
  \freenames{@{x}} := \{ x \}
\end{mathpar}

$\pi$
$\quotep{\pi}$

$\freenames{-} : \pi \to \mathcal{P}(\quotep{\pi})$

\begin{eqnarray*}
  \freenames{\pzero} & := & \emptyset \\
  \freenames{x?(y).P} & := & \{ x \} \cup (\freenames{P} \setminus \{ y \}) \\
  \freenames{x!\langle P \rangle} & := & \{ x \} \cup \{ P \} \\
  \freenames{P|Q} & := & \freenames{P} \cup \freenames{Q} \\
  \freenames{\dropn{x}} & := & \{ x \}
\end{eqnarray*}

The bound names of a process, $\boundnames{P}$, are those names occurring in $P$
that are not free. For example, in $x?(y).0$, the name $x$ is free, while $y$ is bound.

\begin{mathpar}
  \inferrule* [lab=monoidal-laws] {} { P|Q \equiv Q|P \and P|0 \equiv P \and P|(Q|R) \equiv (P|Q)|R }
\end{mathpar}

\begin{mathpar}
  \inferrule* [lab=alpha-equivalence] {} { (x)P \equiv (y)P\{y/x\} \and y \not\in \freenames{P} }
\end{mathpar}

\begin{definition}
Then two processes, $P,Q$, are alpha-equivalent if $P = Q\{\vec{y}/\vec{x}\}$ for
some $\vec{x} \in \boundnames{Q},\vec{y} \in \boundnames{P}$, where $Q\{\vec{y}/\vec{x}\}$
denotes the capture-avoiding substitution of $\vec{y}$ for $\vec{x}$ in $Q$.
\end{definition}

\begin{definition}
  The {\em structural congruence} \cite{SangiorgiWalker} , $\equiv$,
  between processes is the least congruence containing
  alpha-equivalence, satisfying the abelian monoid laws
  (associativity, commutativity and $\pzero$ as identity) for parallel
  composition $|$ and for summation $+$.
\end{definition}

\subsection{Name equivalence}

We take name equivalence, written $\nameeq$, to be the smallest
equivalence relation generated by the following rules.

\begin{mathpar}
\inferrule*[lab=Quote-drop]
{ }
{ \quotep{@{x}} \nameeq x }

\inferrule*[lab=Struct-equiv]
{ P \scong Q }
{ \quotep{P} \nameeq \quotep{Q} }
\end{mathpar}

The astute reader will have noticed that the mutual recursion of names
and processes imposes a mutual recursion on alpha-equivalence and
structural equivalence via name-equivalence. Fortunately, all of this
works out pleasantly and we may calculate in the natural way, free of
concern. The reader interested in the details is referred to the
appendix \ref{appendix:rho_details}.

\subsection{Substitution}

We use $\Proc$ for the set of processes, $\QProc$ for the set of
names, and $\id{\{}\vec{y} / \vec{x} \id{\}}$ to denote partial maps,
$s : \QProc \rightarrow \QProc$. A map, $s$ lifts, uniquely, to a map
on process terms, $\widehat{s} : \Proc \rightarrow \Proc$ by the
following equations.

\begin{mathpar}
  (0) \psubstp{Q}{P} := 0 \\
  (R \juxtap S) \psubstp{Q}{P}
  :=    
  (R)\psubstp{Q}{P} \juxtap (S) \psubstp{Q}{P} \\
  (x?(y).R) \psubstp{Q}{P}    
  :=    
  (x)\substp{Q}{P} (z)\concat( (R \psubstn{z}{y}) \psubstp{Q}{P} ) \\
  (\lift{x}{R}) \psubstp{Q}{P}  
  :=
  \lift{(x)\substp{Q}{P}}{ R \psubstp{Q}{P} } \\
%   (\dropn{x})  \psubstp{Q}{P}       
%   := 
%   \left\{ 
%     \begin{array}{ccc} 
%       \dropn{\quotep{Q}} & & x \nameeq \quotep{P} \\
%       \dropn{x} & & otherwise \\
%     \end{array}
%   \right. 
  (\dropn{x})  \psubstp{Q}{P}       
  := 
  \left\{ 
    \begin{array}{ccc} 
      Q & & x \nameeq \quotep{P} \\
      \dropn{x} & & otherwise \\
    \end{array}
  \right.
\end{mathpar}
 

where

\begin{eqnarray}
  (x)\id{\{} \lpquote Q \rpquote / \lpquote P \rpquote \id{\}}            = 
  \left\{ 
    \begin{array}{ccc}
      \lpquote Q \rpquote & & x \nameeq \lpquote P \rpquote \\
      x & & otherwise \\
    \end{array}
  \right. \nonumber
\end{eqnarray}

and $z$ is chosen distinct from $\quotep{P}$, $\quotep{Q}$, the free
names in $Q$, and all the names in $R$. Our $\alpha$-equivalence will
be built in the standard way from this substitution.

\begin{remark}\label{rem:no_self_referential_names}
  One consequence of these definitions is that $\forall P. \quotep{P}
  \not\in \freenames{P}$.
\end{remark}

\subsection{ Dynamic quote: an example }

Anticipating something of what's to come, consider applying the
substitution, $\widehat{\id{\{}u / z \id{\}}}$, to the following pair
of processes, $\lift{w}{y!(z)}$ and $w[ \lpquote y!(z) \rpquote ]$.

\begin{eqnarray}
	\lift{w}{y!(z)}\widehat{\id{\{}u / z \id{\}}}
		& = &
		\lift{w}{y!(u)} \nonumber\\
	w[ \lpquote y!(z) \rpquote ] \widehat{ \id{\{}u / z \id{\}} }
		& = &
		w[ \lpquote y!(z) \rpquote ] \nonumber
\end{eqnarray}

Because the body of the process between quotes is impervious to
substitution, we get radically different answers. In fact, by
examining the first process in an input context,
e.g. $x?(z).\lift{w}{y!(z)}$, we see that the process under the lift
operator may be shaped by prefixed inputs binding a name inside it. In
this sense, the lift operator will be seen as a way to dynamically
construct processes before reifying them as names.

Finally equipped with these standard features we can present the
dynamics of the calculus.

\subsubsection{Operational semantics} 

Finally, we introduce the computational dynamics. What marks these
algebras as distinct from other more traditionally studied algebraic
structures, e.g. vector spaces or polynomial rings, is the manner in
which dynamics is captured. In traditional structures, dynamics is typically
expressed through morphisms between such structures, as in linear maps
between vector spaces or morphisms between rings. In algebras
associated with the semantics of computation, the dynamics is
expressed as part of the algebraic structure itself, through a
reduction reduction relation typically denoted by $\red$. Below, we
give a recursive presentation of this relation for the calculus used
in the encoding.

$\red \subseteq \pi \times \pi$
$\red : \pi \to \mathcal{P}(\pi)$

\begin{mathpar}
  \inferrule* [lab=Comm] { \textsf{match}( x_{src}, x_{trgt} ) } { x_{trgt}?(y)P \; | \; x_{src}!\langle {Q} \rangle \red P\{\quotep{Q}/y}\} }
  \and \\
  \inferrule* [lab=Par] {{P} \red {P}'} {{{P} | {Q}} \red {{P}' | {Q}}}
  \and
  \inferrule* [lab=Equiv]{{{P} \scong {P}'} \andalso {{P}' \red {Q}'} \andalso {{Q}' \scong {Q}}}{{P} \red {Q}}
\end{mathpar}

\begin{eqnarray*}
  match_{\equiv} (\quotep{P},\quotep{Q}) & := & P \equiv Q \\
  match_{\dagger}(\quotep{P},\quotep{Q}) & := & \forall R. P|Q \red^{*} R => R \red^{*} 0 \\
  match_{K}(\quotep{P},\quotep{Q}) & := & K \mbox{ for some context } K
\end{eqnarray*}

$u?(x)P | u!\langle Q \rangle \red P\{\quotep{Q}/x\}$

%We write $\wred$ for $\red^*$, and $P\red$ if $\exists Q $ such that $ P \red Q$.
We write $P\red$ if $\exists Q $ such that $ P \red Q$ and $P\not\red$, otherwise.

\section{Replication}

As mentioned before, it is known that replication (and hence
recursion) can be implemented in a higher-order process algebra
\cite{SangiorgiWalker}. As our first example of calculation with the
machinery thus far presented we give the construction explicitly in
the {\rhoc}.

\begin{eqnarray}
	D_{x} & := & \prefix{x}{y}{(\binpar{\outputp{x}{y}}{@{y}})} \nonumber\\
	\bangp_{x}{P} & := & \binpar{{x}!\langle{\binpar{D_{x}}{P}}\rangle}{D_{x}} \nonumber
\end{eqnarray}

\begin{eqnarray}
	\bangp_{x}{P} & & \nonumber\\
	=
	& {x}!\langle{(\prefix{x}{y}{(\outputp{x}{y} | @{y})) | P}}\rangle 
	      | \prefix{x}{y}{(\outputp{x}{y} | @{y})} & \nonumber\\
	\red
	& (\outputp{x}{y} | @{y})\substn{\quotep{(\prefix{x}{y}{(@{y} | \outputp{x}{y})) | P}}}{y} & \nonumber\\
	=
	& \outputp{x}{\quotep{(\prefix{x}{y}{(\outputp{x}{y} | @{y})) | P}}}
	  | {(\prefix{x}{y}{(\outputp{x}{y} | @{y})) | P}} & \nonumber\\
	\red
	& \ldots & \nonumber\\
	\red^*
	& P | P | \ldots & \nonumber
\end{eqnarray}

Of course, this encoding, as an implementation, runs away, unfolding
$\bangp{P}$ eagerly. A lazier and more implementable replication
operator, restricted to input-guarded processes, may be obtained as follows.

\begin{eqnarray}
\bangp{\prefix{u}{v}{P}} 
	:= 
	\binpar{\lift{x}{\prefix{u}{v}{(\binpar{D(x)}{P})}}}{D(x)} \nonumber
\end{eqnarray}

\begin{remark}
  Note that the lazier definition still does not deal with summation
  or mixed summation (i.e. sums over input and output). The reader is
  invited to construct definitions of replication that deal with these
  features. 

  Further, the definitions are parameterized in a name, $x$. Can you,
  gentle reader, make a definition that eliminates this parameter and
  guarantees no accidental interaction between the replication
  machinery and the process being replicated -- i.e. no accidental
  sharing of names used by the process to get its work done and the
  name(s) used by the replication to effect copying. This latter
  revision of the definition of replication is crucial to obtaining
  the expected identity $!!P \sim !P$.
\end{remark}

\begin{remark}\label{rem:paradoxical_combinator}
  The reader familiar with the lambda calculus will have noticed the
  similarity between $D$ and the paradoxical combinator.

  [Ed. note: the existence of this seems to suggest we have to be more
  restrictive on the set of processes and names we admit if we are to
  support no-cloning.]
\end{remark}

\subsubsection{Bisimulation}

The computational dynamics gives rise to another kind of equivalence,
the equivalence of computational behavior. As previously mentioned
this is typically captured \emph{via} some form of bisimulation.

% The notion we use in this paper is weak barbed bisimulation
% \cite{milner91polyadicpi}.

The notion we use in this paper is derived from weak barbed
bisimulation \cite{milner91polyadicpi}. 

\begin{definition}
An \emph{observation relation}, $\downarrow_{\mathcal N}$, over a set
of names, $\mathcal N$, is the smallest relation satisfying the rules
below.

\infrule[Out-barb]{y \in {\mathcal N}, \; x \nameeq y}
		  {\outputp{x}{v} \downarrow_{\mathcal N} x}
\infrule[Par-barb]{\mbox{$P\downarrow_{\mathcal N} x$ or $Q\downarrow_{\mathcal N} x$}}
		  {\binpar{P}{Q} \downarrow_{\mathcal N} x}

We write $P \Downarrow_{\mathcal N} x$ if there is $Q$ such that 
$P \wred Q$ and $Q \downarrow_{\mathcal N} x$.
\end{definition}

\begin{definition}
%\label{def.bbisim}
An  ${\mathcal N}$-\emph{barbed bisimulation} over a set of names, ${\mathcal N}$, is a symmetric binary relation 
${\mathcal S}_{\mathcal N}$ between agents such that $P\rel{S}_{\mathcal N}Q$ implies:
\begin{enumerate}
\item If $P \red P'$ then $Q \wred Q'$ and $P'\rel{S}_{\mathcal N} Q'$.
\item If $P\downarrow_{\mathcal N} x$, then $Q\Downarrow_{\mathcal N} x$.
\end{enumerate}
$P$ is ${\mathcal N}$-barbed bisimilar to $Q$, written
$P \wbbisim_{\mathcal N} Q$, if $P \rel{S}_{\mathcal N} Q$ for some ${\mathcal N}$-barbed bisimulation ${\mathcal S}_{\mathcal N}$.
\end{definition}

$\mathcal{R} \subseteq \pi \times \pi$

$P \mathcal{R} Q => \forall P'. P \red P' \Rightarrow \exists Q'. Q \red Q', P' \mathcal{R} Q'$

$P \vdash x \Rightarrow Q \vdash x$

\begin{mathpar}
  \inferrule*[lab=Out-barb]{x \nameeq y}{{y}!\langle{Q}\rangle \vdash x}
  \and
  \inferrule*[lab=Par-barb]{\mbox{$P\vdash x$ or $Q\vdash x$}}{\binpar{P}{Q} \vdash x}
\end{mathpar}

\subsubsection{Contexts}

One of the principle advantages of computational calculi like the
$\pi$-calculus is a well-defined notion of context,
contextual-equivalence and a correlation between
contextual-equivalence and notions of bisimulation. The notion of
context allows the decomposition of a process into (sub-)process and
its syntactic environment, its context. Thus, a context may be
thought of as a process with a ``hole'' (written $\Box$) in it. The
application of a context $M$ to a process $P$, written $M[P]$, is
tantamount to filling the hole in $M$ with $P$. In this paper we do
not need the full weight of this theory, but do make use of the notion
of context in the proof the main theorem. 

\begin{mathpar}
  \inferrule* [lab=summation] {} {{M_{M},M_{N}} \bc \Box \;|\; x.M_{A} \;|\; M_{M}+M_{N}}
  \and
  \inferrule* [lab=agent] {} {{M_{A}} \bc (\vec{x})M_{P} \;| \; \clift{P_0,\ldots,M_{P},\ldots,P_N}}
  \and \\
  \inferrule* [lab=process] {} {{M_{P}} \bc M_{N} \;| \;P|M_{P} }
\end{mathpar} 

\begin{mathpar}
  \inferrule* [lab=sychronization] {} {M_{N} \bc \Box \;|\; x?M_{F} \;|\; x!M_{C}}
  \and
  \inferrule* [lab=abstraction] {} {{M_{F}} \bc (x)M_{P} }
  \and
  \inferrule* [lab=concretion] {} {{M_{C}} \bc \langle M_{P} \rangle }
  \and \\
  \inferrule* [lab=process] {} {{M_{P}} \bc M_{N} \;| \;P|M_{P} }
\end{mathpar}

\begin{definition}[contextual application] Given a context $M$, and
  process $P$, we define the \emph{contextual application}, $M[P] :=
  M\{P/\Box\}$. That is, the contextual application of M to P is the
  substitution of $P$ for $\Box$ in $M$.
\end{definition}

$\meaningof{-} : L \to \mathcal{P}(\pi)$

\begin{mathpar}
  \inferrule* [lab=collection] {} {\meaningof{true} = \pi, \and \meaningof{~E} = \pi \setminus \meaningof{E}, \and \meaningof{E_{1} \& E_{2}} = \meaningof{E_{1}} \cap \meaningof{E_{2}}}
\end{mathpar}

\begin{mathpar}
  \inferrule* [lab=structure] {} {\meaningof{0} = \{ P \in \pi | P \equiv 0 \}, \and \\ \meaningof{E_1 | E_2} = \{ P \in \pi | P \equiv P_{1} | P_{2}, P_{1} \in \meaningof{E_{1}}, P_{2} \in \meaningof{E_2}\} }
\end{mathpar}

\begin{mathpar}
 \inferrule* [lab=behavior] {} {\meaningof{\langle a?b \rangle E} = \{ P \in \pi | P \equiv Q | u?(y)P', \\ \and \\\\ \and \\ \;\;\; u \in \meaningof{a}, \forall z.P'\{z/y\} \in \meaningof{E\{z/b\}}\}, \and \\ \meaningof{a!E} = \{ P \in \pi | P \equiv Q | x!\langle P' \rangle, x \in \meaningof{a} P' \in \meaningof{E}\} }
\end{mathpar}

\begin{mathpar}
 \inferrule* [lab=nominal] {} {\meaningof{\quotep{E}} = \{ \quotep{P} \in \quotep{\pi} | P \in \meaningof{E} \}, \and \meaningof{\quotep{P}} = \{ \quotep{Q} \in \quotep{\pi} | P \equiv Q \} \and \\ \meaningof{@\quotep{E}} = \{ P \in \pi | P \equiv @x, x \in \meaningof{E} \}}
\end{mathpar}

\begin{eqnarray*}
  \\
  \meaningof{-} : TS \to ST
\end{eqnarray*}

\begin{eqnarray*}
  \\
  L : TS \to ST
\end{eqnarray*}

\begin{eqnarray*}
  \\
  P \models E \iff P \in \meaningof{E}
\end{eqnarray*}

\begin{eqnarray*}
  P \approx_{L} Q \iff \forall E \in L. P \models E \iff Q \models E
\end{eqnarray*}

\begin{eqnarray*}
  P \approx_{K} Q
\end{eqnarray*}

\begin{eqnarray*}
  P \approx Q
\end{eqnarray*}

$\approx_{K} = \approx = \approx_{L}$

\subsubsection{Contextual duality}

Note that contexts extend the quotation operation to a family of
operations from processes to names. Given a context, $M$, we can
define a \emph{nominal context}, $\quotep{M}$ by $\quotep{M}[P] :=
\quotep{M[P]}$. To foreshadow what is to come we observe that these
operations enjoy a duality with processes very much like the duality
between vectors and maps from vectors to scalars.

Further, because the calculus is essentially higher-order, we have a
correspondence between contexts and processes. More specifically,
given a name $x$ and a context $M$ we can construct $M^{*}_{x}$ such
that 

\begin{mathpar}
  M^{*}_{x} | \lift{x}{P} \red M[P]
\end{mathpar}

namely,

\begin{mathpar}
  M^{*}_{x} := x?(u).M[\dropn{u}]
\end{mathpar}

The dependence of $M^{*}_{x}$ on a name makes it an abstraction, 

\begin{mathpar}
  M^{*} := (x)x?(u).M[\dropn{u}]
\end{mathpar}

\subsection{Additional notation}

It will sometimes be convenient to denote the process a name
quotes. We already have the notation $x = \quotep{P}$, but it will be
convenient to introduce an alternate notation, $\procn{x}$, when we
want to emphasize the connection to the use of the name. Note that, by
virtue of name equivalence, $\quotep{\procn{x}} \nameeq x$; so, the
notation is consistent with previous definitions.

Further, because names have structure it is possible to effect
substitutions on the basis of that structure. This means we need to
upgrade our notation for substitutions, which we accomplish by
adapting comprehension notation. Thus,

\begin{mathpar}
  P\{ y / x : x \in S \}
\end{mathpar}

is interpreted to mean the process derived from P by replacing (in a
capture-avoiding manner) each occurrence of $x$ in $S$ by $y$. For example,

\begin{mathpar}
  P\{ \quotep{\procn{x}|\procn{x}} / x : x \in \freenames{P} \}
\end{mathpar}

will replace each (occurrence) of a free name $x$ in $P$ by
$\quotep{\procn{x}|\procn{x}}$.

Also, we will avail ourselves of the notation $x^{L}$ and $x^{R}$ to
denote injections of a name into disjoint copies of the name
space. There are numerous ways to accomplish this. One example can be
found in \cite{MeredithR05}. This notation overloads to vectors of
names: $\vec{x}^{\pi} := (x_{i}^{\pi} \; : \; 0 \leq i < |\vec{x}| )$ where $\pi \in \{L,R\}$.

We also use $P^{\Box} := P|\Box$.

In \cite{MeredithR05} an interpretation of the new operator is
given. It turns out that there are several possible interpretations
all enjoying the requisite algebraic properties of the operator (see
\cite{milner91polyadicpi}). We will therefore make liberal use of
$(\nu\; \vec{x})P$.

% subsection the_syntax_and_semantics_of_the_notation_system (end)   

\input{qm2pi.qmops} 

\input{qm2pi.sterngerlach} 

\input{qm2pi.metric} 

% section concurrent_process_calculi (end)

%\input{qm2pi.proofsketch}

% section proof sketch (end)

%\input{qm2pi.slviaknots} 

% section spatial logic via knots (end)

\input{qm2pi.conclusion}

% section conclusion (end)

%\input{qm2pi.dtcodes} 

% section wiring algorithm (end)

\input{qm2pi.ack} 

% section acknowledgments (end)

\newpage


\bibliographystyle{plain}   
\bibliography{../../biblios/main.bib}

\input{qm2pi.rhodetails}

\end{document}

 

% section notation (end)

\input{qm2pi.process.calculi} 

% section concurrent_process_calculi_and_spatial_logics_ (end)
    
%\documentclass[12pt]{llncs}
%\documentclass{jktr}

\usepackage[pdftex]{hyperref}                   
\usepackage {listings}
\usepackage {mathpartir}
\usepackage{bcprules}
%\usepackage{listings}
                       
\usepackage{graphicx} 
%\usepackage[margins=2.5cm,nohead,nofoot]{geometry}
%\usepackage{geometry}
\usepackage{amsfonts}
\usepackage{amstext}
\usepackage{latexsym}
\usepackage{amssymb}
\usepackage{color}


%\include{myPreamble}
\include{qm2pi.local} 

%\ifpdf
%\usepackage[pdftex]{graphicx}
%\else
%\usepackage{graphicx}
%\fi

 % \ifpdf
%  \usepackage{pdfsync}
%  \if


%\title{Brief Article}
%\author{David F. Snyder}
%\author{L.G. Meredith}

%\address{Dept. of Math., Texas State University--San Marcos, San Marcos, TX 78666}
       
\pagestyle{empty}


\begin{document}

\lstset{language=[Objective]Caml,frame=shadowbox}

\input{qm2pi.front}

% section front matter (end)

\input{qm2pi.intro} 
 
% section introduction (end)

% \input{qm2pi.knotations} 

% section notation (end)

\input{qm2pi.process.calculi} 

% section concurrent_process_calculi_and_spatial_logics_ (end)
    
%\input{qm2pi.knots2pi} 

%\input{qm2pi.trefoil} 

%\input{qm2pi.mainthm} 

% subsection basic_interpretation (end)

%\input{qm2pi.rho.presentation} 
\subsection{The syntax and semantics of the notation system}\label{sub:the_syntax_and_semantics_of_the_notation_system} % (fold)

We now summarize a technical presentation of the calculus that
embodies our theory of dynamics. The typical presentation of such a
calculus follows the style of giving generators and relations on
them. The grammar, below, describing term constructors, freely
generates the set of processes, $\Proc$. This set is then quotiented
by a relation known as structural congruence and it is over this set
that the notion of dynamics is expressed. This presentation is
essentially that of \cite{MeredithR05} with the addition of
polyadicity and summation. For readability we have relegated some of
the technical subtleties to an appendix.

\subsubsection{Process grammar}\label{subsub:process_grammar}

\begin{mathpar}
  \inferrule* [lab=synchronization] {} {{M} \bc \pzero \;|\; x?F \;|\; x!C }
  \and
  \inferrule* [lab=abstraction] {} {{F} \bc (x)P}
  \and
  \inferrule* [lab=concretion] {} {{C} \bc \langle Q \rangle}
  \and
  \inferrule* [lab=process] {} {{P,Q} \bc M \;| \;P|Q \;|\; @{x}}
  \and
  \inferrule* [lab=name] {} {{x} \bc \quotep{P}}
\end{mathpar} 

Note that $\vec{x}$ (resp. $\vec{P}$) denotes a vector of names
(resp. processes) of length $|\vec{x}|$ (resp. $|\vec{P}|$). We adopt
the following useful abbreviations.

\begin{mathpar}
   x?(\vec{y}).P := x.(\vec{y})P \and  x\clift{\vec{P}} := x.\clift{\vec{P}}
   \and x!(y) := \lift{x}{\dropn{y}}
   \and \Pi_{i=0}^{n-1}P_i := P_0 | \ldots | P_{n-1}
\end{mathpar}

\subsubsection{Structural congruence}

\paragraph{Free and bound names and alpha-equivalence.} At the
core of structural equivalence is alpha-equivalence which identifies
process that are the same up to a change of variable. Formally, we
recognize the distinction between free and bound names. The free names
of a process, $\freenames{P}$, may be calculated recursively as
follows:

\begin{mathpar}
\freenames{\pzero} := \emptyset
  \and \\
  \freenames{x?(y).P} := \{ x \} \cup (\freenames{P} \setminus \{ y \})
  \and 
  \freenames{x!\langle P \rangle} := \{ x \} \cup \{ P \} 
  \and \\
  \freenames{P|Q} := \freenames{P} \cup \freenames{Q}
  \and \\
  \freenames{@{x}} := \{ x \}
\end{mathpar}

$\pi$
$\quotep{\pi}$

$\freenames{-} : \pi \to \mathcal{P}(\quotep{\pi})$

\begin{eqnarray*}
  \freenames{\pzero} & := & \emptyset \\
  \freenames{x?(y).P} & := & \{ x \} \cup (\freenames{P} \setminus \{ y \}) \\
  \freenames{x!\langle P \rangle} & := & \{ x \} \cup \{ P \} \\
  \freenames{P|Q} & := & \freenames{P} \cup \freenames{Q} \\
  \freenames{\dropn{x}} & := & \{ x \}
\end{eqnarray*}

The bound names of a process, $\boundnames{P}$, are those names occurring in $P$
that are not free. For example, in $x?(y).0$, the name $x$ is free, while $y$ is bound.

\begin{mathpar}
  \inferrule* [lab=monoidal-laws] {} { P|Q \equiv Q|P \and P|0 \equiv P \and P|(Q|R) \equiv (P|Q)|R }
\end{mathpar}

\begin{mathpar}
  \inferrule* [lab=alpha-equivalence] {} { (x)P \equiv (y)P\{y/x\} \and y \not\in \freenames{P} }
\end{mathpar}

\begin{definition}
Then two processes, $P,Q$, are alpha-equivalent if $P = Q\{\vec{y}/\vec{x}\}$ for
some $\vec{x} \in \boundnames{Q},\vec{y} \in \boundnames{P}$, where $Q\{\vec{y}/\vec{x}\}$
denotes the capture-avoiding substitution of $\vec{y}$ for $\vec{x}$ in $Q$.
\end{definition}

\begin{definition}
  The {\em structural congruence} \cite{SangiorgiWalker} , $\equiv$,
  between processes is the least congruence containing
  alpha-equivalence, satisfying the abelian monoid laws
  (associativity, commutativity and $\pzero$ as identity) for parallel
  composition $|$ and for summation $+$.
\end{definition}

\subsection{Name equivalence}

We take name equivalence, written $\nameeq$, to be the smallest
equivalence relation generated by the following rules.

\begin{mathpar}
\inferrule*[lab=Quote-drop]
{ }
{ \quotep{@{x}} \nameeq x }

\inferrule*[lab=Struct-equiv]
{ P \scong Q }
{ \quotep{P} \nameeq \quotep{Q} }
\end{mathpar}

The astute reader will have noticed that the mutual recursion of names
and processes imposes a mutual recursion on alpha-equivalence and
structural equivalence via name-equivalence. Fortunately, all of this
works out pleasantly and we may calculate in the natural way, free of
concern. The reader interested in the details is referred to the
appendix \ref{appendix:rho_details}.

\subsection{Substitution}

We use $\Proc$ for the set of processes, $\QProc$ for the set of
names, and $\id{\{}\vec{y} / \vec{x} \id{\}}$ to denote partial maps,
$s : \QProc \rightarrow \QProc$. A map, $s$ lifts, uniquely, to a map
on process terms, $\widehat{s} : \Proc \rightarrow \Proc$ by the
following equations.

\begin{mathpar}
  (0) \psubstp{Q}{P} := 0 \\
  (R \juxtap S) \psubstp{Q}{P}
  :=    
  (R)\psubstp{Q}{P} \juxtap (S) \psubstp{Q}{P} \\
  (x?(y).R) \psubstp{Q}{P}    
  :=    
  (x)\substp{Q}{P} (z)\concat( (R \psubstn{z}{y}) \psubstp{Q}{P} ) \\
  (\lift{x}{R}) \psubstp{Q}{P}  
  :=
  \lift{(x)\substp{Q}{P}}{ R \psubstp{Q}{P} } \\
%   (\dropn{x})  \psubstp{Q}{P}       
%   := 
%   \left\{ 
%     \begin{array}{ccc} 
%       \dropn{\quotep{Q}} & & x \nameeq \quotep{P} \\
%       \dropn{x} & & otherwise \\
%     \end{array}
%   \right. 
  (\dropn{x})  \psubstp{Q}{P}       
  := 
  \left\{ 
    \begin{array}{ccc} 
      Q & & x \nameeq \quotep{P} \\
      \dropn{x} & & otherwise \\
    \end{array}
  \right.
\end{mathpar}
 

where

\begin{eqnarray}
  (x)\id{\{} \lpquote Q \rpquote / \lpquote P \rpquote \id{\}}            = 
  \left\{ 
    \begin{array}{ccc}
      \lpquote Q \rpquote & & x \nameeq \lpquote P \rpquote \\
      x & & otherwise \\
    \end{array}
  \right. \nonumber
\end{eqnarray}

and $z$ is chosen distinct from $\quotep{P}$, $\quotep{Q}$, the free
names in $Q$, and all the names in $R$. Our $\alpha$-equivalence will
be built in the standard way from this substitution.

\begin{remark}\label{rem:no_self_referential_names}
  One consequence of these definitions is that $\forall P. \quotep{P}
  \not\in \freenames{P}$.
\end{remark}

\subsection{ Dynamic quote: an example }

Anticipating something of what's to come, consider applying the
substitution, $\widehat{\id{\{}u / z \id{\}}}$, to the following pair
of processes, $\lift{w}{y!(z)}$ and $w[ \lpquote y!(z) \rpquote ]$.

\begin{eqnarray}
	\lift{w}{y!(z)}\widehat{\id{\{}u / z \id{\}}}
		& = &
		\lift{w}{y!(u)} \nonumber\\
	w[ \lpquote y!(z) \rpquote ] \widehat{ \id{\{}u / z \id{\}} }
		& = &
		w[ \lpquote y!(z) \rpquote ] \nonumber
\end{eqnarray}

Because the body of the process between quotes is impervious to
substitution, we get radically different answers. In fact, by
examining the first process in an input context,
e.g. $x?(z).\lift{w}{y!(z)}$, we see that the process under the lift
operator may be shaped by prefixed inputs binding a name inside it. In
this sense, the lift operator will be seen as a way to dynamically
construct processes before reifying them as names.

Finally equipped with these standard features we can present the
dynamics of the calculus.

\subsubsection{Operational semantics} 

Finally, we introduce the computational dynamics. What marks these
algebras as distinct from other more traditionally studied algebraic
structures, e.g. vector spaces or polynomial rings, is the manner in
which dynamics is captured. In traditional structures, dynamics is typically
expressed through morphisms between such structures, as in linear maps
between vector spaces or morphisms between rings. In algebras
associated with the semantics of computation, the dynamics is
expressed as part of the algebraic structure itself, through a
reduction reduction relation typically denoted by $\red$. Below, we
give a recursive presentation of this relation for the calculus used
in the encoding.

$\red \subseteq \pi \times \pi$
$\red : \pi \to \mathcal{P}(\pi)$

\begin{mathpar}
  \inferrule* [lab=Comm] { \textsf{match}( x_{src}, x_{trgt} ) } { x_{trgt}?(y)P \; | \; x_{src}!\langle {Q} \rangle \red P\{\quotep{Q}/y}\} }
  \and \\
  \inferrule* [lab=Par] {{P} \red {P}'} {{{P} | {Q}} \red {{P}' | {Q}}}
  \and
  \inferrule* [lab=Equiv]{{{P} \scong {P}'} \andalso {{P}' \red {Q}'} \andalso {{Q}' \scong {Q}}}{{P} \red {Q}}
\end{mathpar}

\begin{eqnarray*}
  match_{\equiv} (\quotep{P},\quotep{Q}) & := & P \equiv Q \\
  match_{\dagger}(\quotep{P},\quotep{Q}) & := & \forall R. P|Q \red^{*} R => R \red^{*} 0 \\
  match_{K}(\quotep{P},\quotep{Q}) & := & K \mbox{ for some context } K
\end{eqnarray*}

$u?(x)P | u!\langle Q \rangle \red P\{\quotep{Q}/x\}$

%We write $\wred$ for $\red^*$, and $P\red$ if $\exists Q $ such that $ P \red Q$.
We write $P\red$ if $\exists Q $ such that $ P \red Q$ and $P\not\red$, otherwise.

\section{Replication}

As mentioned before, it is known that replication (and hence
recursion) can be implemented in a higher-order process algebra
\cite{SangiorgiWalker}. As our first example of calculation with the
machinery thus far presented we give the construction explicitly in
the {\rhoc}.

\begin{eqnarray}
	D_{x} & := & \prefix{x}{y}{(\binpar{\outputp{x}{y}}{@{y}})} \nonumber\\
	\bangp_{x}{P} & := & \binpar{{x}!\langle{\binpar{D_{x}}{P}}\rangle}{D_{x}} \nonumber
\end{eqnarray}

\begin{eqnarray}
	\bangp_{x}{P} & & \nonumber\\
	=
	& {x}!\langle{(\prefix{x}{y}{(\outputp{x}{y} | @{y})) | P}}\rangle 
	      | \prefix{x}{y}{(\outputp{x}{y} | @{y})} & \nonumber\\
	\red
	& (\outputp{x}{y} | @{y})\substn{\quotep{(\prefix{x}{y}{(@{y} | \outputp{x}{y})) | P}}}{y} & \nonumber\\
	=
	& \outputp{x}{\quotep{(\prefix{x}{y}{(\outputp{x}{y} | @{y})) | P}}}
	  | {(\prefix{x}{y}{(\outputp{x}{y} | @{y})) | P}} & \nonumber\\
	\red
	& \ldots & \nonumber\\
	\red^*
	& P | P | \ldots & \nonumber
\end{eqnarray}

Of course, this encoding, as an implementation, runs away, unfolding
$\bangp{P}$ eagerly. A lazier and more implementable replication
operator, restricted to input-guarded processes, may be obtained as follows.

\begin{eqnarray}
\bangp{\prefix{u}{v}{P}} 
	:= 
	\binpar{\lift{x}{\prefix{u}{v}{(\binpar{D(x)}{P})}}}{D(x)} \nonumber
\end{eqnarray}

\begin{remark}
  Note that the lazier definition still does not deal with summation
  or mixed summation (i.e. sums over input and output). The reader is
  invited to construct definitions of replication that deal with these
  features. 

  Further, the definitions are parameterized in a name, $x$. Can you,
  gentle reader, make a definition that eliminates this parameter and
  guarantees no accidental interaction between the replication
  machinery and the process being replicated -- i.e. no accidental
  sharing of names used by the process to get its work done and the
  name(s) used by the replication to effect copying. This latter
  revision of the definition of replication is crucial to obtaining
  the expected identity $!!P \sim !P$.
\end{remark}

\begin{remark}\label{rem:paradoxical_combinator}
  The reader familiar with the lambda calculus will have noticed the
  similarity between $D$ and the paradoxical combinator.

  [Ed. note: the existence of this seems to suggest we have to be more
  restrictive on the set of processes and names we admit if we are to
  support no-cloning.]
\end{remark}

\subsubsection{Bisimulation}

The computational dynamics gives rise to another kind of equivalence,
the equivalence of computational behavior. As previously mentioned
this is typically captured \emph{via} some form of bisimulation.

% The notion we use in this paper is weak barbed bisimulation
% \cite{milner91polyadicpi}.

The notion we use in this paper is derived from weak barbed
bisimulation \cite{milner91polyadicpi}. 

\begin{definition}
An \emph{observation relation}, $\downarrow_{\mathcal N}$, over a set
of names, $\mathcal N$, is the smallest relation satisfying the rules
below.

\infrule[Out-barb]{y \in {\mathcal N}, \; x \nameeq y}
		  {\outputp{x}{v} \downarrow_{\mathcal N} x}
\infrule[Par-barb]{\mbox{$P\downarrow_{\mathcal N} x$ or $Q\downarrow_{\mathcal N} x$}}
		  {\binpar{P}{Q} \downarrow_{\mathcal N} x}

We write $P \Downarrow_{\mathcal N} x$ if there is $Q$ such that 
$P \wred Q$ and $Q \downarrow_{\mathcal N} x$.
\end{definition}

\begin{definition}
%\label{def.bbisim}
An  ${\mathcal N}$-\emph{barbed bisimulation} over a set of names, ${\mathcal N}$, is a symmetric binary relation 
${\mathcal S}_{\mathcal N}$ between agents such that $P\rel{S}_{\mathcal N}Q$ implies:
\begin{enumerate}
\item If $P \red P'$ then $Q \wred Q'$ and $P'\rel{S}_{\mathcal N} Q'$.
\item If $P\downarrow_{\mathcal N} x$, then $Q\Downarrow_{\mathcal N} x$.
\end{enumerate}
$P$ is ${\mathcal N}$-barbed bisimilar to $Q$, written
$P \wbbisim_{\mathcal N} Q$, if $P \rel{S}_{\mathcal N} Q$ for some ${\mathcal N}$-barbed bisimulation ${\mathcal S}_{\mathcal N}$.
\end{definition}

$\mathcal{R} \subseteq \pi \times \pi$

$P \mathcal{R} Q => \forall P'. P \red P' \Rightarrow \exists Q'. Q \red Q', P' \mathcal{R} Q'$

$P \vdash x \Rightarrow Q \vdash x$

\begin{mathpar}
  \inferrule*[lab=Out-barb]{x \nameeq y}{{y}!\langle{Q}\rangle \vdash x}
  \and
  \inferrule*[lab=Par-barb]{\mbox{$P\vdash x$ or $Q\vdash x$}}{\binpar{P}{Q} \vdash x}
\end{mathpar}

\subsubsection{Contexts}

One of the principle advantages of computational calculi like the
$\pi$-calculus is a well-defined notion of context,
contextual-equivalence and a correlation between
contextual-equivalence and notions of bisimulation. The notion of
context allows the decomposition of a process into (sub-)process and
its syntactic environment, its context. Thus, a context may be
thought of as a process with a ``hole'' (written $\Box$) in it. The
application of a context $M$ to a process $P$, written $M[P]$, is
tantamount to filling the hole in $M$ with $P$. In this paper we do
not need the full weight of this theory, but do make use of the notion
of context in the proof the main theorem. 

\begin{mathpar}
  \inferrule* [lab=summation] {} {{M_{M},M_{N}} \bc \Box \;|\; x.M_{A} \;|\; M_{M}+M_{N}}
  \and
  \inferrule* [lab=agent] {} {{M_{A}} \bc (\vec{x})M_{P} \;| \; \clift{P_0,\ldots,M_{P},\ldots,P_N}}
  \and \\
  \inferrule* [lab=process] {} {{M_{P}} \bc M_{N} \;| \;P|M_{P} }
\end{mathpar} 

\begin{mathpar}
  \inferrule* [lab=sychronization] {} {M_{N} \bc \Box \;|\; x?M_{F} \;|\; x!M_{C}}
  \and
  \inferrule* [lab=abstraction] {} {{M_{F}} \bc (x)M_{P} }
  \and
  \inferrule* [lab=concretion] {} {{M_{C}} \bc \langle M_{P} \rangle }
  \and \\
  \inferrule* [lab=process] {} {{M_{P}} \bc M_{N} \;| \;P|M_{P} }
\end{mathpar}

\begin{definition}[contextual application] Given a context $M$, and
  process $P$, we define the \emph{contextual application}, $M[P] :=
  M\{P/\Box\}$. That is, the contextual application of M to P is the
  substitution of $P$ for $\Box$ in $M$.
\end{definition}

$\meaningof{-} : L \to \mathcal{P}(\pi)$

\begin{mathpar}
  \inferrule* [lab=collection] {} {\meaningof{true} = \pi, \and \meaningof{~E} = \pi \setminus \meaningof{E}, \and \meaningof{E_{1} \& E_{2}} = \meaningof{E_{1}} \cap \meaningof{E_{2}}}
\end{mathpar}

\begin{mathpar}
  \inferrule* [lab=structure] {} {\meaningof{0} = \{ P \in \pi | P \equiv 0 \}, \and \\ \meaningof{E_1 | E_2} = \{ P \in \pi | P \equiv P_{1} | P_{2}, P_{1} \in \meaningof{E_{1}}, P_{2} \in \meaningof{E_2}\} }
\end{mathpar}

\begin{mathpar}
 \inferrule* [lab=behavior] {} {\meaningof{\langle a?b \rangle E} = \{ P \in \pi | P \equiv Q | u?(y)P', \\ \and \\\\ \and \\ \;\;\; u \in \meaningof{a}, \forall z.P'\{z/y\} \in \meaningof{E\{z/b\}}\}, \and \\ \meaningof{a!E} = \{ P \in \pi | P \equiv Q | x!\langle P' \rangle, x \in \meaningof{a} P' \in \meaningof{E}\} }
\end{mathpar}

\begin{mathpar}
 \inferrule* [lab=nominal] {} {\meaningof{\quotep{E}} = \{ \quotep{P} \in \quotep{\pi} | P \in \meaningof{E} \}, \and \meaningof{\quotep{P}} = \{ \quotep{Q} \in \quotep{\pi} | P \equiv Q \} \and \\ \meaningof{@\quotep{E}} = \{ P \in \pi | P \equiv @x, x \in \meaningof{E} \}}
\end{mathpar}

\begin{eqnarray*}
  \\
  \meaningof{-} : TS \to ST
\end{eqnarray*}

\begin{eqnarray*}
  \\
  L : TS \to ST
\end{eqnarray*}

\begin{eqnarray*}
  \\
  P \models E \iff P \in \meaningof{E}
\end{eqnarray*}

\begin{eqnarray*}
  P \approx_{L} Q \iff \forall E \in L. P \models E \iff Q \models E
\end{eqnarray*}

\begin{eqnarray*}
  P \approx_{K} Q
\end{eqnarray*}

\begin{eqnarray*}
  P \approx Q
\end{eqnarray*}

$\approx_{K} = \approx = \approx_{L}$

\subsubsection{Contextual duality}

Note that contexts extend the quotation operation to a family of
operations from processes to names. Given a context, $M$, we can
define a \emph{nominal context}, $\quotep{M}$ by $\quotep{M}[P] :=
\quotep{M[P]}$. To foreshadow what is to come we observe that these
operations enjoy a duality with processes very much like the duality
between vectors and maps from vectors to scalars.

Further, because the calculus is essentially higher-order, we have a
correspondence between contexts and processes. More specifically,
given a name $x$ and a context $M$ we can construct $M^{*}_{x}$ such
that 

\begin{mathpar}
  M^{*}_{x} | \lift{x}{P} \red M[P]
\end{mathpar}

namely,

\begin{mathpar}
  M^{*}_{x} := x?(u).M[\dropn{u}]
\end{mathpar}

The dependence of $M^{*}_{x}$ on a name makes it an abstraction, 

\begin{mathpar}
  M^{*} := (x)x?(u).M[\dropn{u}]
\end{mathpar}

\subsection{Additional notation}

It will sometimes be convenient to denote the process a name
quotes. We already have the notation $x = \quotep{P}$, but it will be
convenient to introduce an alternate notation, $\procn{x}$, when we
want to emphasize the connection to the use of the name. Note that, by
virtue of name equivalence, $\quotep{\procn{x}} \nameeq x$; so, the
notation is consistent with previous definitions.

Further, because names have structure it is possible to effect
substitutions on the basis of that structure. This means we need to
upgrade our notation for substitutions, which we accomplish by
adapting comprehension notation. Thus,

\begin{mathpar}
  P\{ y / x : x \in S \}
\end{mathpar}

is interpreted to mean the process derived from P by replacing (in a
capture-avoiding manner) each occurrence of $x$ in $S$ by $y$. For example,

\begin{mathpar}
  P\{ \quotep{\procn{x}|\procn{x}} / x : x \in \freenames{P} \}
\end{mathpar}

will replace each (occurrence) of a free name $x$ in $P$ by
$\quotep{\procn{x}|\procn{x}}$.

Also, we will avail ourselves of the notation $x^{L}$ and $x^{R}$ to
denote injections of a name into disjoint copies of the name
space. There are numerous ways to accomplish this. One example can be
found in \cite{MeredithR05}. This notation overloads to vectors of
names: $\vec{x}^{\pi} := (x_{i}^{\pi} \; : \; 0 \leq i < |\vec{x}| )$ where $\pi \in \{L,R\}$.

We also use $P^{\Box} := P|\Box$.

In \cite{MeredithR05} an interpretation of the new operator is
given. It turns out that there are several possible interpretations
all enjoying the requisite algebraic properties of the operator (see
\cite{milner91polyadicpi}). We will therefore make liberal use of
$(\nu\; \vec{x})P$.

% subsection the_syntax_and_semantics_of_the_notation_system (end)   

\input{qm2pi.qmops} 

\input{qm2pi.sterngerlach} 

\input{qm2pi.metric} 

% section concurrent_process_calculi (end)

%\input{qm2pi.proofsketch}

% section proof sketch (end)

%\input{qm2pi.slviaknots} 

% section spatial logic via knots (end)

\input{qm2pi.conclusion}

% section conclusion (end)

%\input{qm2pi.dtcodes} 

% section wiring algorithm (end)

\input{qm2pi.ack} 

% section acknowledgments (end)

\newpage


\bibliographystyle{plain}   
\bibliography{../../biblios/main.bib}

\input{qm2pi.rhodetails}

\end{document}

 

%\documentclass[12pt]{llncs}
%\documentclass{jktr}

\usepackage[pdftex]{hyperref}                   
\usepackage {listings}
\usepackage {mathpartir}
\usepackage{bcprules}
%\usepackage{listings}
                       
\usepackage{graphicx} 
%\usepackage[margins=2.5cm,nohead,nofoot]{geometry}
%\usepackage{geometry}
\usepackage{amsfonts}
\usepackage{amstext}
\usepackage{latexsym}
\usepackage{amssymb}
\usepackage{color}


%\include{myPreamble}
\include{qm2pi.local} 

%\ifpdf
%\usepackage[pdftex]{graphicx}
%\else
%\usepackage{graphicx}
%\fi

 % \ifpdf
%  \usepackage{pdfsync}
%  \if


%\title{Brief Article}
%\author{David F. Snyder}
%\author{L.G. Meredith}

%\address{Dept. of Math., Texas State University--San Marcos, San Marcos, TX 78666}
       
\pagestyle{empty}


\begin{document}

\lstset{language=[Objective]Caml,frame=shadowbox}

\input{qm2pi.front}

% section front matter (end)

\input{qm2pi.intro} 
 
% section introduction (end)

% \input{qm2pi.knotations} 

% section notation (end)

\input{qm2pi.process.calculi} 

% section concurrent_process_calculi_and_spatial_logics_ (end)
    
%\input{qm2pi.knots2pi} 

%\input{qm2pi.trefoil} 

%\input{qm2pi.mainthm} 

% subsection basic_interpretation (end)

%\input{qm2pi.rho.presentation} 
\subsection{The syntax and semantics of the notation system}\label{sub:the_syntax_and_semantics_of_the_notation_system} % (fold)

We now summarize a technical presentation of the calculus that
embodies our theory of dynamics. The typical presentation of such a
calculus follows the style of giving generators and relations on
them. The grammar, below, describing term constructors, freely
generates the set of processes, $\Proc$. This set is then quotiented
by a relation known as structural congruence and it is over this set
that the notion of dynamics is expressed. This presentation is
essentially that of \cite{MeredithR05} with the addition of
polyadicity and summation. For readability we have relegated some of
the technical subtleties to an appendix.

\subsubsection{Process grammar}\label{subsub:process_grammar}

\begin{mathpar}
  \inferrule* [lab=synchronization] {} {{M} \bc \pzero \;|\; x?F \;|\; x!C }
  \and
  \inferrule* [lab=abstraction] {} {{F} \bc (x)P}
  \and
  \inferrule* [lab=concretion] {} {{C} \bc \langle Q \rangle}
  \and
  \inferrule* [lab=process] {} {{P,Q} \bc M \;| \;P|Q \;|\; @{x}}
  \and
  \inferrule* [lab=name] {} {{x} \bc \quotep{P}}
\end{mathpar} 

Note that $\vec{x}$ (resp. $\vec{P}$) denotes a vector of names
(resp. processes) of length $|\vec{x}|$ (resp. $|\vec{P}|$). We adopt
the following useful abbreviations.

\begin{mathpar}
   x?(\vec{y}).P := x.(\vec{y})P \and  x\clift{\vec{P}} := x.\clift{\vec{P}}
   \and x!(y) := \lift{x}{\dropn{y}}
   \and \Pi_{i=0}^{n-1}P_i := P_0 | \ldots | P_{n-1}
\end{mathpar}

\subsubsection{Structural congruence}

\paragraph{Free and bound names and alpha-equivalence.} At the
core of structural equivalence is alpha-equivalence which identifies
process that are the same up to a change of variable. Formally, we
recognize the distinction between free and bound names. The free names
of a process, $\freenames{P}$, may be calculated recursively as
follows:

\begin{mathpar}
\freenames{\pzero} := \emptyset
  \and \\
  \freenames{x?(y).P} := \{ x \} \cup (\freenames{P} \setminus \{ y \})
  \and 
  \freenames{x!\langle P \rangle} := \{ x \} \cup \{ P \} 
  \and \\
  \freenames{P|Q} := \freenames{P} \cup \freenames{Q}
  \and \\
  \freenames{@{x}} := \{ x \}
\end{mathpar}

$\pi$
$\quotep{\pi}$

$\freenames{-} : \pi \to \mathcal{P}(\quotep{\pi})$

\begin{eqnarray*}
  \freenames{\pzero} & := & \emptyset \\
  \freenames{x?(y).P} & := & \{ x \} \cup (\freenames{P} \setminus \{ y \}) \\
  \freenames{x!\langle P \rangle} & := & \{ x \} \cup \{ P \} \\
  \freenames{P|Q} & := & \freenames{P} \cup \freenames{Q} \\
  \freenames{\dropn{x}} & := & \{ x \}
\end{eqnarray*}

The bound names of a process, $\boundnames{P}$, are those names occurring in $P$
that are not free. For example, in $x?(y).0$, the name $x$ is free, while $y$ is bound.

\begin{mathpar}
  \inferrule* [lab=monoidal-laws] {} { P|Q \equiv Q|P \and P|0 \equiv P \and P|(Q|R) \equiv (P|Q)|R }
\end{mathpar}

\begin{mathpar}
  \inferrule* [lab=alpha-equivalence] {} { (x)P \equiv (y)P\{y/x\} \and y \not\in \freenames{P} }
\end{mathpar}

\begin{definition}
Then two processes, $P,Q$, are alpha-equivalent if $P = Q\{\vec{y}/\vec{x}\}$ for
some $\vec{x} \in \boundnames{Q},\vec{y} \in \boundnames{P}$, where $Q\{\vec{y}/\vec{x}\}$
denotes the capture-avoiding substitution of $\vec{y}$ for $\vec{x}$ in $Q$.
\end{definition}

\begin{definition}
  The {\em structural congruence} \cite{SangiorgiWalker} , $\equiv$,
  between processes is the least congruence containing
  alpha-equivalence, satisfying the abelian monoid laws
  (associativity, commutativity and $\pzero$ as identity) for parallel
  composition $|$ and for summation $+$.
\end{definition}

\subsection{Name equivalence}

We take name equivalence, written $\nameeq$, to be the smallest
equivalence relation generated by the following rules.

\begin{mathpar}
\inferrule*[lab=Quote-drop]
{ }
{ \quotep{@{x}} \nameeq x }

\inferrule*[lab=Struct-equiv]
{ P \scong Q }
{ \quotep{P} \nameeq \quotep{Q} }
\end{mathpar}

The astute reader will have noticed that the mutual recursion of names
and processes imposes a mutual recursion on alpha-equivalence and
structural equivalence via name-equivalence. Fortunately, all of this
works out pleasantly and we may calculate in the natural way, free of
concern. The reader interested in the details is referred to the
appendix \ref{appendix:rho_details}.

\subsection{Substitution}

We use $\Proc$ for the set of processes, $\QProc$ for the set of
names, and $\id{\{}\vec{y} / \vec{x} \id{\}}$ to denote partial maps,
$s : \QProc \rightarrow \QProc$. A map, $s$ lifts, uniquely, to a map
on process terms, $\widehat{s} : \Proc \rightarrow \Proc$ by the
following equations.

\begin{mathpar}
  (0) \psubstp{Q}{P} := 0 \\
  (R \juxtap S) \psubstp{Q}{P}
  :=    
  (R)\psubstp{Q}{P} \juxtap (S) \psubstp{Q}{P} \\
  (x?(y).R) \psubstp{Q}{P}    
  :=    
  (x)\substp{Q}{P} (z)\concat( (R \psubstn{z}{y}) \psubstp{Q}{P} ) \\
  (\lift{x}{R}) \psubstp{Q}{P}  
  :=
  \lift{(x)\substp{Q}{P}}{ R \psubstp{Q}{P} } \\
%   (\dropn{x})  \psubstp{Q}{P}       
%   := 
%   \left\{ 
%     \begin{array}{ccc} 
%       \dropn{\quotep{Q}} & & x \nameeq \quotep{P} \\
%       \dropn{x} & & otherwise \\
%     \end{array}
%   \right. 
  (\dropn{x})  \psubstp{Q}{P}       
  := 
  \left\{ 
    \begin{array}{ccc} 
      Q & & x \nameeq \quotep{P} \\
      \dropn{x} & & otherwise \\
    \end{array}
  \right.
\end{mathpar}
 

where

\begin{eqnarray}
  (x)\id{\{} \lpquote Q \rpquote / \lpquote P \rpquote \id{\}}            = 
  \left\{ 
    \begin{array}{ccc}
      \lpquote Q \rpquote & & x \nameeq \lpquote P \rpquote \\
      x & & otherwise \\
    \end{array}
  \right. \nonumber
\end{eqnarray}

and $z$ is chosen distinct from $\quotep{P}$, $\quotep{Q}$, the free
names in $Q$, and all the names in $R$. Our $\alpha$-equivalence will
be built in the standard way from this substitution.

\begin{remark}\label{rem:no_self_referential_names}
  One consequence of these definitions is that $\forall P. \quotep{P}
  \not\in \freenames{P}$.
\end{remark}

\subsection{ Dynamic quote: an example }

Anticipating something of what's to come, consider applying the
substitution, $\widehat{\id{\{}u / z \id{\}}}$, to the following pair
of processes, $\lift{w}{y!(z)}$ and $w[ \lpquote y!(z) \rpquote ]$.

\begin{eqnarray}
	\lift{w}{y!(z)}\widehat{\id{\{}u / z \id{\}}}
		& = &
		\lift{w}{y!(u)} \nonumber\\
	w[ \lpquote y!(z) \rpquote ] \widehat{ \id{\{}u / z \id{\}} }
		& = &
		w[ \lpquote y!(z) \rpquote ] \nonumber
\end{eqnarray}

Because the body of the process between quotes is impervious to
substitution, we get radically different answers. In fact, by
examining the first process in an input context,
e.g. $x?(z).\lift{w}{y!(z)}$, we see that the process under the lift
operator may be shaped by prefixed inputs binding a name inside it. In
this sense, the lift operator will be seen as a way to dynamically
construct processes before reifying them as names.

Finally equipped with these standard features we can present the
dynamics of the calculus.

\subsubsection{Operational semantics} 

Finally, we introduce the computational dynamics. What marks these
algebras as distinct from other more traditionally studied algebraic
structures, e.g. vector spaces or polynomial rings, is the manner in
which dynamics is captured. In traditional structures, dynamics is typically
expressed through morphisms between such structures, as in linear maps
between vector spaces or morphisms between rings. In algebras
associated with the semantics of computation, the dynamics is
expressed as part of the algebraic structure itself, through a
reduction reduction relation typically denoted by $\red$. Below, we
give a recursive presentation of this relation for the calculus used
in the encoding.

$\red \subseteq \pi \times \pi$
$\red : \pi \to \mathcal{P}(\pi)$

\begin{mathpar}
  \inferrule* [lab=Comm] { \textsf{match}( x_{src}, x_{trgt} ) } { x_{trgt}?(y)P \; | \; x_{src}!\langle {Q} \rangle \red P\{\quotep{Q}/y}\} }
  \and \\
  \inferrule* [lab=Par] {{P} \red {P}'} {{{P} | {Q}} \red {{P}' | {Q}}}
  \and
  \inferrule* [lab=Equiv]{{{P} \scong {P}'} \andalso {{P}' \red {Q}'} \andalso {{Q}' \scong {Q}}}{{P} \red {Q}}
\end{mathpar}

\begin{eqnarray*}
  match_{\equiv} (\quotep{P},\quotep{Q}) & := & P \equiv Q \\
  match_{\dagger}(\quotep{P},\quotep{Q}) & := & \forall R. P|Q \red^{*} R => R \red^{*} 0 \\
  match_{K}(\quotep{P},\quotep{Q}) & := & K \mbox{ for some context } K
\end{eqnarray*}

$u?(x)P | u!\langle Q \rangle \red P\{\quotep{Q}/x\}$

%We write $\wred$ for $\red^*$, and $P\red$ if $\exists Q $ such that $ P \red Q$.
We write $P\red$ if $\exists Q $ such that $ P \red Q$ and $P\not\red$, otherwise.

\section{Replication}

As mentioned before, it is known that replication (and hence
recursion) can be implemented in a higher-order process algebra
\cite{SangiorgiWalker}. As our first example of calculation with the
machinery thus far presented we give the construction explicitly in
the {\rhoc}.

\begin{eqnarray}
	D_{x} & := & \prefix{x}{y}{(\binpar{\outputp{x}{y}}{@{y}})} \nonumber\\
	\bangp_{x}{P} & := & \binpar{{x}!\langle{\binpar{D_{x}}{P}}\rangle}{D_{x}} \nonumber
\end{eqnarray}

\begin{eqnarray}
	\bangp_{x}{P} & & \nonumber\\
	=
	& {x}!\langle{(\prefix{x}{y}{(\outputp{x}{y} | @{y})) | P}}\rangle 
	      | \prefix{x}{y}{(\outputp{x}{y} | @{y})} & \nonumber\\
	\red
	& (\outputp{x}{y} | @{y})\substn{\quotep{(\prefix{x}{y}{(@{y} | \outputp{x}{y})) | P}}}{y} & \nonumber\\
	=
	& \outputp{x}{\quotep{(\prefix{x}{y}{(\outputp{x}{y} | @{y})) | P}}}
	  | {(\prefix{x}{y}{(\outputp{x}{y} | @{y})) | P}} & \nonumber\\
	\red
	& \ldots & \nonumber\\
	\red^*
	& P | P | \ldots & \nonumber
\end{eqnarray}

Of course, this encoding, as an implementation, runs away, unfolding
$\bangp{P}$ eagerly. A lazier and more implementable replication
operator, restricted to input-guarded processes, may be obtained as follows.

\begin{eqnarray}
\bangp{\prefix{u}{v}{P}} 
	:= 
	\binpar{\lift{x}{\prefix{u}{v}{(\binpar{D(x)}{P})}}}{D(x)} \nonumber
\end{eqnarray}

\begin{remark}
  Note that the lazier definition still does not deal with summation
  or mixed summation (i.e. sums over input and output). The reader is
  invited to construct definitions of replication that deal with these
  features. 

  Further, the definitions are parameterized in a name, $x$. Can you,
  gentle reader, make a definition that eliminates this parameter and
  guarantees no accidental interaction between the replication
  machinery and the process being replicated -- i.e. no accidental
  sharing of names used by the process to get its work done and the
  name(s) used by the replication to effect copying. This latter
  revision of the definition of replication is crucial to obtaining
  the expected identity $!!P \sim !P$.
\end{remark}

\begin{remark}\label{rem:paradoxical_combinator}
  The reader familiar with the lambda calculus will have noticed the
  similarity between $D$ and the paradoxical combinator.

  [Ed. note: the existence of this seems to suggest we have to be more
  restrictive on the set of processes and names we admit if we are to
  support no-cloning.]
\end{remark}

\subsubsection{Bisimulation}

The computational dynamics gives rise to another kind of equivalence,
the equivalence of computational behavior. As previously mentioned
this is typically captured \emph{via} some form of bisimulation.

% The notion we use in this paper is weak barbed bisimulation
% \cite{milner91polyadicpi}.

The notion we use in this paper is derived from weak barbed
bisimulation \cite{milner91polyadicpi}. 

\begin{definition}
An \emph{observation relation}, $\downarrow_{\mathcal N}$, over a set
of names, $\mathcal N$, is the smallest relation satisfying the rules
below.

\infrule[Out-barb]{y \in {\mathcal N}, \; x \nameeq y}
		  {\outputp{x}{v} \downarrow_{\mathcal N} x}
\infrule[Par-barb]{\mbox{$P\downarrow_{\mathcal N} x$ or $Q\downarrow_{\mathcal N} x$}}
		  {\binpar{P}{Q} \downarrow_{\mathcal N} x}

We write $P \Downarrow_{\mathcal N} x$ if there is $Q$ such that 
$P \wred Q$ and $Q \downarrow_{\mathcal N} x$.
\end{definition}

\begin{definition}
%\label{def.bbisim}
An  ${\mathcal N}$-\emph{barbed bisimulation} over a set of names, ${\mathcal N}$, is a symmetric binary relation 
${\mathcal S}_{\mathcal N}$ between agents such that $P\rel{S}_{\mathcal N}Q$ implies:
\begin{enumerate}
\item If $P \red P'$ then $Q \wred Q'$ and $P'\rel{S}_{\mathcal N} Q'$.
\item If $P\downarrow_{\mathcal N} x$, then $Q\Downarrow_{\mathcal N} x$.
\end{enumerate}
$P$ is ${\mathcal N}$-barbed bisimilar to $Q$, written
$P \wbbisim_{\mathcal N} Q$, if $P \rel{S}_{\mathcal N} Q$ for some ${\mathcal N}$-barbed bisimulation ${\mathcal S}_{\mathcal N}$.
\end{definition}

$\mathcal{R} \subseteq \pi \times \pi$

$P \mathcal{R} Q => \forall P'. P \red P' \Rightarrow \exists Q'. Q \red Q', P' \mathcal{R} Q'$

$P \vdash x \Rightarrow Q \vdash x$

\begin{mathpar}
  \inferrule*[lab=Out-barb]{x \nameeq y}{{y}!\langle{Q}\rangle \vdash x}
  \and
  \inferrule*[lab=Par-barb]{\mbox{$P\vdash x$ or $Q\vdash x$}}{\binpar{P}{Q} \vdash x}
\end{mathpar}

\subsubsection{Contexts}

One of the principle advantages of computational calculi like the
$\pi$-calculus is a well-defined notion of context,
contextual-equivalence and a correlation between
contextual-equivalence and notions of bisimulation. The notion of
context allows the decomposition of a process into (sub-)process and
its syntactic environment, its context. Thus, a context may be
thought of as a process with a ``hole'' (written $\Box$) in it. The
application of a context $M$ to a process $P$, written $M[P]$, is
tantamount to filling the hole in $M$ with $P$. In this paper we do
not need the full weight of this theory, but do make use of the notion
of context in the proof the main theorem. 

\begin{mathpar}
  \inferrule* [lab=summation] {} {{M_{M},M_{N}} \bc \Box \;|\; x.M_{A} \;|\; M_{M}+M_{N}}
  \and
  \inferrule* [lab=agent] {} {{M_{A}} \bc (\vec{x})M_{P} \;| \; \clift{P_0,\ldots,M_{P},\ldots,P_N}}
  \and \\
  \inferrule* [lab=process] {} {{M_{P}} \bc M_{N} \;| \;P|M_{P} }
\end{mathpar} 

\begin{mathpar}
  \inferrule* [lab=sychronization] {} {M_{N} \bc \Box \;|\; x?M_{F} \;|\; x!M_{C}}
  \and
  \inferrule* [lab=abstraction] {} {{M_{F}} \bc (x)M_{P} }
  \and
  \inferrule* [lab=concretion] {} {{M_{C}} \bc \langle M_{P} \rangle }
  \and \\
  \inferrule* [lab=process] {} {{M_{P}} \bc M_{N} \;| \;P|M_{P} }
\end{mathpar}

\begin{definition}[contextual application] Given a context $M$, and
  process $P$, we define the \emph{contextual application}, $M[P] :=
  M\{P/\Box\}$. That is, the contextual application of M to P is the
  substitution of $P$ for $\Box$ in $M$.
\end{definition}

$\meaningof{-} : L \to \mathcal{P}(\pi)$

\begin{mathpar}
  \inferrule* [lab=collection] {} {\meaningof{true} = \pi, \and \meaningof{~E} = \pi \setminus \meaningof{E}, \and \meaningof{E_{1} \& E_{2}} = \meaningof{E_{1}} \cap \meaningof{E_{2}}}
\end{mathpar}

\begin{mathpar}
  \inferrule* [lab=structure] {} {\meaningof{0} = \{ P \in \pi | P \equiv 0 \}, \and \\ \meaningof{E_1 | E_2} = \{ P \in \pi | P \equiv P_{1} | P_{2}, P_{1} \in \meaningof{E_{1}}, P_{2} \in \meaningof{E_2}\} }
\end{mathpar}

\begin{mathpar}
 \inferrule* [lab=behavior] {} {\meaningof{\langle a?b \rangle E} = \{ P \in \pi | P \equiv Q | u?(y)P', \\ \and \\\\ \and \\ \;\;\; u \in \meaningof{a}, \forall z.P'\{z/y\} \in \meaningof{E\{z/b\}}\}, \and \\ \meaningof{a!E} = \{ P \in \pi | P \equiv Q | x!\langle P' \rangle, x \in \meaningof{a} P' \in \meaningof{E}\} }
\end{mathpar}

\begin{mathpar}
 \inferrule* [lab=nominal] {} {\meaningof{\quotep{E}} = \{ \quotep{P} \in \quotep{\pi} | P \in \meaningof{E} \}, \and \meaningof{\quotep{P}} = \{ \quotep{Q} \in \quotep{\pi} | P \equiv Q \} \and \\ \meaningof{@\quotep{E}} = \{ P \in \pi | P \equiv @x, x \in \meaningof{E} \}}
\end{mathpar}

\begin{eqnarray*}
  \\
  \meaningof{-} : TS \to ST
\end{eqnarray*}

\begin{eqnarray*}
  \\
  L : TS \to ST
\end{eqnarray*}

\begin{eqnarray*}
  \\
  P \models E \iff P \in \meaningof{E}
\end{eqnarray*}

\begin{eqnarray*}
  P \approx_{L} Q \iff \forall E \in L. P \models E \iff Q \models E
\end{eqnarray*}

\begin{eqnarray*}
  P \approx_{K} Q
\end{eqnarray*}

\begin{eqnarray*}
  P \approx Q
\end{eqnarray*}

$\approx_{K} = \approx = \approx_{L}$

\subsubsection{Contextual duality}

Note that contexts extend the quotation operation to a family of
operations from processes to names. Given a context, $M$, we can
define a \emph{nominal context}, $\quotep{M}$ by $\quotep{M}[P] :=
\quotep{M[P]}$. To foreshadow what is to come we observe that these
operations enjoy a duality with processes very much like the duality
between vectors and maps from vectors to scalars.

Further, because the calculus is essentially higher-order, we have a
correspondence between contexts and processes. More specifically,
given a name $x$ and a context $M$ we can construct $M^{*}_{x}$ such
that 

\begin{mathpar}
  M^{*}_{x} | \lift{x}{P} \red M[P]
\end{mathpar}

namely,

\begin{mathpar}
  M^{*}_{x} := x?(u).M[\dropn{u}]
\end{mathpar}

The dependence of $M^{*}_{x}$ on a name makes it an abstraction, 

\begin{mathpar}
  M^{*} := (x)x?(u).M[\dropn{u}]
\end{mathpar}

\subsection{Additional notation}

It will sometimes be convenient to denote the process a name
quotes. We already have the notation $x = \quotep{P}$, but it will be
convenient to introduce an alternate notation, $\procn{x}$, when we
want to emphasize the connection to the use of the name. Note that, by
virtue of name equivalence, $\quotep{\procn{x}} \nameeq x$; so, the
notation is consistent with previous definitions.

Further, because names have structure it is possible to effect
substitutions on the basis of that structure. This means we need to
upgrade our notation for substitutions, which we accomplish by
adapting comprehension notation. Thus,

\begin{mathpar}
  P\{ y / x : x \in S \}
\end{mathpar}

is interpreted to mean the process derived from P by replacing (in a
capture-avoiding manner) each occurrence of $x$ in $S$ by $y$. For example,

\begin{mathpar}
  P\{ \quotep{\procn{x}|\procn{x}} / x : x \in \freenames{P} \}
\end{mathpar}

will replace each (occurrence) of a free name $x$ in $P$ by
$\quotep{\procn{x}|\procn{x}}$.

Also, we will avail ourselves of the notation $x^{L}$ and $x^{R}$ to
denote injections of a name into disjoint copies of the name
space. There are numerous ways to accomplish this. One example can be
found in \cite{MeredithR05}. This notation overloads to vectors of
names: $\vec{x}^{\pi} := (x_{i}^{\pi} \; : \; 0 \leq i < |\vec{x}| )$ where $\pi \in \{L,R\}$.

We also use $P^{\Box} := P|\Box$.

In \cite{MeredithR05} an interpretation of the new operator is
given. It turns out that there are several possible interpretations
all enjoying the requisite algebraic properties of the operator (see
\cite{milner91polyadicpi}). We will therefore make liberal use of
$(\nu\; \vec{x})P$.

% subsection the_syntax_and_semantics_of_the_notation_system (end)   

\input{qm2pi.qmops} 

\input{qm2pi.sterngerlach} 

\input{qm2pi.metric} 

% section concurrent_process_calculi (end)

%\input{qm2pi.proofsketch}

% section proof sketch (end)

%\input{qm2pi.slviaknots} 

% section spatial logic via knots (end)

\input{qm2pi.conclusion}

% section conclusion (end)

%\input{qm2pi.dtcodes} 

% section wiring algorithm (end)

\input{qm2pi.ack} 

% section acknowledgments (end)

\newpage


\bibliographystyle{plain}   
\bibliography{../../biblios/main.bib}

\input{qm2pi.rhodetails}

\end{document}

 

%\documentclass[12pt]{llncs}
%\documentclass{jktr}

\usepackage[pdftex]{hyperref}                   
\usepackage {listings}
\usepackage {mathpartir}
\usepackage{bcprules}
%\usepackage{listings}
                       
\usepackage{graphicx} 
%\usepackage[margins=2.5cm,nohead,nofoot]{geometry}
%\usepackage{geometry}
\usepackage{amsfonts}
\usepackage{amstext}
\usepackage{latexsym}
\usepackage{amssymb}
\usepackage{color}


%\include{myPreamble}
\include{qm2pi.local} 

%\ifpdf
%\usepackage[pdftex]{graphicx}
%\else
%\usepackage{graphicx}
%\fi

 % \ifpdf
%  \usepackage{pdfsync}
%  \if


%\title{Brief Article}
%\author{David F. Snyder}
%\author{L.G. Meredith}

%\address{Dept. of Math., Texas State University--San Marcos, San Marcos, TX 78666}
       
\pagestyle{empty}


\begin{document}

\lstset{language=[Objective]Caml,frame=shadowbox}

\input{qm2pi.front}

% section front matter (end)

\input{qm2pi.intro} 
 
% section introduction (end)

% \input{qm2pi.knotations} 

% section notation (end)

\input{qm2pi.process.calculi} 

% section concurrent_process_calculi_and_spatial_logics_ (end)
    
%\input{qm2pi.knots2pi} 

%\input{qm2pi.trefoil} 

%\input{qm2pi.mainthm} 

% subsection basic_interpretation (end)

%\input{qm2pi.rho.presentation} 
\subsection{The syntax and semantics of the notation system}\label{sub:the_syntax_and_semantics_of_the_notation_system} % (fold)

We now summarize a technical presentation of the calculus that
embodies our theory of dynamics. The typical presentation of such a
calculus follows the style of giving generators and relations on
them. The grammar, below, describing term constructors, freely
generates the set of processes, $\Proc$. This set is then quotiented
by a relation known as structural congruence and it is over this set
that the notion of dynamics is expressed. This presentation is
essentially that of \cite{MeredithR05} with the addition of
polyadicity and summation. For readability we have relegated some of
the technical subtleties to an appendix.

\subsubsection{Process grammar}\label{subsub:process_grammar}

\begin{mathpar}
  \inferrule* [lab=synchronization] {} {{M} \bc \pzero \;|\; x?F \;|\; x!C }
  \and
  \inferrule* [lab=abstraction] {} {{F} \bc (x)P}
  \and
  \inferrule* [lab=concretion] {} {{C} \bc \langle Q \rangle}
  \and
  \inferrule* [lab=process] {} {{P,Q} \bc M \;| \;P|Q \;|\; @{x}}
  \and
  \inferrule* [lab=name] {} {{x} \bc \quotep{P}}
\end{mathpar} 

Note that $\vec{x}$ (resp. $\vec{P}$) denotes a vector of names
(resp. processes) of length $|\vec{x}|$ (resp. $|\vec{P}|$). We adopt
the following useful abbreviations.

\begin{mathpar}
   x?(\vec{y}).P := x.(\vec{y})P \and  x\clift{\vec{P}} := x.\clift{\vec{P}}
   \and x!(y) := \lift{x}{\dropn{y}}
   \and \Pi_{i=0}^{n-1}P_i := P_0 | \ldots | P_{n-1}
\end{mathpar}

\subsubsection{Structural congruence}

\paragraph{Free and bound names and alpha-equivalence.} At the
core of structural equivalence is alpha-equivalence which identifies
process that are the same up to a change of variable. Formally, we
recognize the distinction between free and bound names. The free names
of a process, $\freenames{P}$, may be calculated recursively as
follows:

\begin{mathpar}
\freenames{\pzero} := \emptyset
  \and \\
  \freenames{x?(y).P} := \{ x \} \cup (\freenames{P} \setminus \{ y \})
  \and 
  \freenames{x!\langle P \rangle} := \{ x \} \cup \{ P \} 
  \and \\
  \freenames{P|Q} := \freenames{P} \cup \freenames{Q}
  \and \\
  \freenames{@{x}} := \{ x \}
\end{mathpar}

$\pi$
$\quotep{\pi}$

$\freenames{-} : \pi \to \mathcal{P}(\quotep{\pi})$

\begin{eqnarray*}
  \freenames{\pzero} & := & \emptyset \\
  \freenames{x?(y).P} & := & \{ x \} \cup (\freenames{P} \setminus \{ y \}) \\
  \freenames{x!\langle P \rangle} & := & \{ x \} \cup \{ P \} \\
  \freenames{P|Q} & := & \freenames{P} \cup \freenames{Q} \\
  \freenames{\dropn{x}} & := & \{ x \}
\end{eqnarray*}

The bound names of a process, $\boundnames{P}$, are those names occurring in $P$
that are not free. For example, in $x?(y).0$, the name $x$ is free, while $y$ is bound.

\begin{mathpar}
  \inferrule* [lab=monoidal-laws] {} { P|Q \equiv Q|P \and P|0 \equiv P \and P|(Q|R) \equiv (P|Q)|R }
\end{mathpar}

\begin{mathpar}
  \inferrule* [lab=alpha-equivalence] {} { (x)P \equiv (y)P\{y/x\} \and y \not\in \freenames{P} }
\end{mathpar}

\begin{definition}
Then two processes, $P,Q$, are alpha-equivalent if $P = Q\{\vec{y}/\vec{x}\}$ for
some $\vec{x} \in \boundnames{Q},\vec{y} \in \boundnames{P}$, where $Q\{\vec{y}/\vec{x}\}$
denotes the capture-avoiding substitution of $\vec{y}$ for $\vec{x}$ in $Q$.
\end{definition}

\begin{definition}
  The {\em structural congruence} \cite{SangiorgiWalker} , $\equiv$,
  between processes is the least congruence containing
  alpha-equivalence, satisfying the abelian monoid laws
  (associativity, commutativity and $\pzero$ as identity) for parallel
  composition $|$ and for summation $+$.
\end{definition}

\subsection{Name equivalence}

We take name equivalence, written $\nameeq$, to be the smallest
equivalence relation generated by the following rules.

\begin{mathpar}
\inferrule*[lab=Quote-drop]
{ }
{ \quotep{@{x}} \nameeq x }

\inferrule*[lab=Struct-equiv]
{ P \scong Q }
{ \quotep{P} \nameeq \quotep{Q} }
\end{mathpar}

The astute reader will have noticed that the mutual recursion of names
and processes imposes a mutual recursion on alpha-equivalence and
structural equivalence via name-equivalence. Fortunately, all of this
works out pleasantly and we may calculate in the natural way, free of
concern. The reader interested in the details is referred to the
appendix \ref{appendix:rho_details}.

\subsection{Substitution}

We use $\Proc$ for the set of processes, $\QProc$ for the set of
names, and $\id{\{}\vec{y} / \vec{x} \id{\}}$ to denote partial maps,
$s : \QProc \rightarrow \QProc$. A map, $s$ lifts, uniquely, to a map
on process terms, $\widehat{s} : \Proc \rightarrow \Proc$ by the
following equations.

\begin{mathpar}
  (0) \psubstp{Q}{P} := 0 \\
  (R \juxtap S) \psubstp{Q}{P}
  :=    
  (R)\psubstp{Q}{P} \juxtap (S) \psubstp{Q}{P} \\
  (x?(y).R) \psubstp{Q}{P}    
  :=    
  (x)\substp{Q}{P} (z)\concat( (R \psubstn{z}{y}) \psubstp{Q}{P} ) \\
  (\lift{x}{R}) \psubstp{Q}{P}  
  :=
  \lift{(x)\substp{Q}{P}}{ R \psubstp{Q}{P} } \\
%   (\dropn{x})  \psubstp{Q}{P}       
%   := 
%   \left\{ 
%     \begin{array}{ccc} 
%       \dropn{\quotep{Q}} & & x \nameeq \quotep{P} \\
%       \dropn{x} & & otherwise \\
%     \end{array}
%   \right. 
  (\dropn{x})  \psubstp{Q}{P}       
  := 
  \left\{ 
    \begin{array}{ccc} 
      Q & & x \nameeq \quotep{P} \\
      \dropn{x} & & otherwise \\
    \end{array}
  \right.
\end{mathpar}
 

where

\begin{eqnarray}
  (x)\id{\{} \lpquote Q \rpquote / \lpquote P \rpquote \id{\}}            = 
  \left\{ 
    \begin{array}{ccc}
      \lpquote Q \rpquote & & x \nameeq \lpquote P \rpquote \\
      x & & otherwise \\
    \end{array}
  \right. \nonumber
\end{eqnarray}

and $z$ is chosen distinct from $\quotep{P}$, $\quotep{Q}$, the free
names in $Q$, and all the names in $R$. Our $\alpha$-equivalence will
be built in the standard way from this substitution.

\begin{remark}\label{rem:no_self_referential_names}
  One consequence of these definitions is that $\forall P. \quotep{P}
  \not\in \freenames{P}$.
\end{remark}

\subsection{ Dynamic quote: an example }

Anticipating something of what's to come, consider applying the
substitution, $\widehat{\id{\{}u / z \id{\}}}$, to the following pair
of processes, $\lift{w}{y!(z)}$ and $w[ \lpquote y!(z) \rpquote ]$.

\begin{eqnarray}
	\lift{w}{y!(z)}\widehat{\id{\{}u / z \id{\}}}
		& = &
		\lift{w}{y!(u)} \nonumber\\
	w[ \lpquote y!(z) \rpquote ] \widehat{ \id{\{}u / z \id{\}} }
		& = &
		w[ \lpquote y!(z) \rpquote ] \nonumber
\end{eqnarray}

Because the body of the process between quotes is impervious to
substitution, we get radically different answers. In fact, by
examining the first process in an input context,
e.g. $x?(z).\lift{w}{y!(z)}$, we see that the process under the lift
operator may be shaped by prefixed inputs binding a name inside it. In
this sense, the lift operator will be seen as a way to dynamically
construct processes before reifying them as names.

Finally equipped with these standard features we can present the
dynamics of the calculus.

\subsubsection{Operational semantics} 

Finally, we introduce the computational dynamics. What marks these
algebras as distinct from other more traditionally studied algebraic
structures, e.g. vector spaces or polynomial rings, is the manner in
which dynamics is captured. In traditional structures, dynamics is typically
expressed through morphisms between such structures, as in linear maps
between vector spaces or morphisms between rings. In algebras
associated with the semantics of computation, the dynamics is
expressed as part of the algebraic structure itself, through a
reduction reduction relation typically denoted by $\red$. Below, we
give a recursive presentation of this relation for the calculus used
in the encoding.

$\red \subseteq \pi \times \pi$
$\red : \pi \to \mathcal{P}(\pi)$

\begin{mathpar}
  \inferrule* [lab=Comm] { \textsf{match}( x_{src}, x_{trgt} ) } { x_{trgt}?(y)P \; | \; x_{src}!\langle {Q} \rangle \red P\{\quotep{Q}/y}\} }
  \and \\
  \inferrule* [lab=Par] {{P} \red {P}'} {{{P} | {Q}} \red {{P}' | {Q}}}
  \and
  \inferrule* [lab=Equiv]{{{P} \scong {P}'} \andalso {{P}' \red {Q}'} \andalso {{Q}' \scong {Q}}}{{P} \red {Q}}
\end{mathpar}

\begin{eqnarray*}
  match_{\equiv} (\quotep{P},\quotep{Q}) & := & P \equiv Q \\
  match_{\dagger}(\quotep{P},\quotep{Q}) & := & \forall R. P|Q \red^{*} R => R \red^{*} 0 \\
  match_{K}(\quotep{P},\quotep{Q}) & := & K \mbox{ for some context } K
\end{eqnarray*}

$u?(x)P | u!\langle Q \rangle \red P\{\quotep{Q}/x\}$

%We write $\wred$ for $\red^*$, and $P\red$ if $\exists Q $ such that $ P \red Q$.
We write $P\red$ if $\exists Q $ such that $ P \red Q$ and $P\not\red$, otherwise.

\section{Replication}

As mentioned before, it is known that replication (and hence
recursion) can be implemented in a higher-order process algebra
\cite{SangiorgiWalker}. As our first example of calculation with the
machinery thus far presented we give the construction explicitly in
the {\rhoc}.

\begin{eqnarray}
	D_{x} & := & \prefix{x}{y}{(\binpar{\outputp{x}{y}}{@{y}})} \nonumber\\
	\bangp_{x}{P} & := & \binpar{{x}!\langle{\binpar{D_{x}}{P}}\rangle}{D_{x}} \nonumber
\end{eqnarray}

\begin{eqnarray}
	\bangp_{x}{P} & & \nonumber\\
	=
	& {x}!\langle{(\prefix{x}{y}{(\outputp{x}{y} | @{y})) | P}}\rangle 
	      | \prefix{x}{y}{(\outputp{x}{y} | @{y})} & \nonumber\\
	\red
	& (\outputp{x}{y} | @{y})\substn{\quotep{(\prefix{x}{y}{(@{y} | \outputp{x}{y})) | P}}}{y} & \nonumber\\
	=
	& \outputp{x}{\quotep{(\prefix{x}{y}{(\outputp{x}{y} | @{y})) | P}}}
	  | {(\prefix{x}{y}{(\outputp{x}{y} | @{y})) | P}} & \nonumber\\
	\red
	& \ldots & \nonumber\\
	\red^*
	& P | P | \ldots & \nonumber
\end{eqnarray}

Of course, this encoding, as an implementation, runs away, unfolding
$\bangp{P}$ eagerly. A lazier and more implementable replication
operator, restricted to input-guarded processes, may be obtained as follows.

\begin{eqnarray}
\bangp{\prefix{u}{v}{P}} 
	:= 
	\binpar{\lift{x}{\prefix{u}{v}{(\binpar{D(x)}{P})}}}{D(x)} \nonumber
\end{eqnarray}

\begin{remark}
  Note that the lazier definition still does not deal with summation
  or mixed summation (i.e. sums over input and output). The reader is
  invited to construct definitions of replication that deal with these
  features. 

  Further, the definitions are parameterized in a name, $x$. Can you,
  gentle reader, make a definition that eliminates this parameter and
  guarantees no accidental interaction between the replication
  machinery and the process being replicated -- i.e. no accidental
  sharing of names used by the process to get its work done and the
  name(s) used by the replication to effect copying. This latter
  revision of the definition of replication is crucial to obtaining
  the expected identity $!!P \sim !P$.
\end{remark}

\begin{remark}\label{rem:paradoxical_combinator}
  The reader familiar with the lambda calculus will have noticed the
  similarity between $D$ and the paradoxical combinator.

  [Ed. note: the existence of this seems to suggest we have to be more
  restrictive on the set of processes and names we admit if we are to
  support no-cloning.]
\end{remark}

\subsubsection{Bisimulation}

The computational dynamics gives rise to another kind of equivalence,
the equivalence of computational behavior. As previously mentioned
this is typically captured \emph{via} some form of bisimulation.

% The notion we use in this paper is weak barbed bisimulation
% \cite{milner91polyadicpi}.

The notion we use in this paper is derived from weak barbed
bisimulation \cite{milner91polyadicpi}. 

\begin{definition}
An \emph{observation relation}, $\downarrow_{\mathcal N}$, over a set
of names, $\mathcal N$, is the smallest relation satisfying the rules
below.

\infrule[Out-barb]{y \in {\mathcal N}, \; x \nameeq y}
		  {\outputp{x}{v} \downarrow_{\mathcal N} x}
\infrule[Par-barb]{\mbox{$P\downarrow_{\mathcal N} x$ or $Q\downarrow_{\mathcal N} x$}}
		  {\binpar{P}{Q} \downarrow_{\mathcal N} x}

We write $P \Downarrow_{\mathcal N} x$ if there is $Q$ such that 
$P \wred Q$ and $Q \downarrow_{\mathcal N} x$.
\end{definition}

\begin{definition}
%\label{def.bbisim}
An  ${\mathcal N}$-\emph{barbed bisimulation} over a set of names, ${\mathcal N}$, is a symmetric binary relation 
${\mathcal S}_{\mathcal N}$ between agents such that $P\rel{S}_{\mathcal N}Q$ implies:
\begin{enumerate}
\item If $P \red P'$ then $Q \wred Q'$ and $P'\rel{S}_{\mathcal N} Q'$.
\item If $P\downarrow_{\mathcal N} x$, then $Q\Downarrow_{\mathcal N} x$.
\end{enumerate}
$P$ is ${\mathcal N}$-barbed bisimilar to $Q$, written
$P \wbbisim_{\mathcal N} Q$, if $P \rel{S}_{\mathcal N} Q$ for some ${\mathcal N}$-barbed bisimulation ${\mathcal S}_{\mathcal N}$.
\end{definition}

$\mathcal{R} \subseteq \pi \times \pi$

$P \mathcal{R} Q => \forall P'. P \red P' \Rightarrow \exists Q'. Q \red Q', P' \mathcal{R} Q'$

$P \vdash x \Rightarrow Q \vdash x$

\begin{mathpar}
  \inferrule*[lab=Out-barb]{x \nameeq y}{{y}!\langle{Q}\rangle \vdash x}
  \and
  \inferrule*[lab=Par-barb]{\mbox{$P\vdash x$ or $Q\vdash x$}}{\binpar{P}{Q} \vdash x}
\end{mathpar}

\subsubsection{Contexts}

One of the principle advantages of computational calculi like the
$\pi$-calculus is a well-defined notion of context,
contextual-equivalence and a correlation between
contextual-equivalence and notions of bisimulation. The notion of
context allows the decomposition of a process into (sub-)process and
its syntactic environment, its context. Thus, a context may be
thought of as a process with a ``hole'' (written $\Box$) in it. The
application of a context $M$ to a process $P$, written $M[P]$, is
tantamount to filling the hole in $M$ with $P$. In this paper we do
not need the full weight of this theory, but do make use of the notion
of context in the proof the main theorem. 

\begin{mathpar}
  \inferrule* [lab=summation] {} {{M_{M},M_{N}} \bc \Box \;|\; x.M_{A} \;|\; M_{M}+M_{N}}
  \and
  \inferrule* [lab=agent] {} {{M_{A}} \bc (\vec{x})M_{P} \;| \; \clift{P_0,\ldots,M_{P},\ldots,P_N}}
  \and \\
  \inferrule* [lab=process] {} {{M_{P}} \bc M_{N} \;| \;P|M_{P} }
\end{mathpar} 

\begin{mathpar}
  \inferrule* [lab=sychronization] {} {M_{N} \bc \Box \;|\; x?M_{F} \;|\; x!M_{C}}
  \and
  \inferrule* [lab=abstraction] {} {{M_{F}} \bc (x)M_{P} }
  \and
  \inferrule* [lab=concretion] {} {{M_{C}} \bc \langle M_{P} \rangle }
  \and \\
  \inferrule* [lab=process] {} {{M_{P}} \bc M_{N} \;| \;P|M_{P} }
\end{mathpar}

\begin{definition}[contextual application] Given a context $M$, and
  process $P$, we define the \emph{contextual application}, $M[P] :=
  M\{P/\Box\}$. That is, the contextual application of M to P is the
  substitution of $P$ for $\Box$ in $M$.
\end{definition}

$\meaningof{-} : L \to \mathcal{P}(\pi)$

\begin{mathpar}
  \inferrule* [lab=collection] {} {\meaningof{true} = \pi, \and \meaningof{~E} = \pi \setminus \meaningof{E}, \and \meaningof{E_{1} \& E_{2}} = \meaningof{E_{1}} \cap \meaningof{E_{2}}}
\end{mathpar}

\begin{mathpar}
  \inferrule* [lab=structure] {} {\meaningof{0} = \{ P \in \pi | P \equiv 0 \}, \and \\ \meaningof{E_1 | E_2} = \{ P \in \pi | P \equiv P_{1} | P_{2}, P_{1} \in \meaningof{E_{1}}, P_{2} \in \meaningof{E_2}\} }
\end{mathpar}

\begin{mathpar}
 \inferrule* [lab=behavior] {} {\meaningof{\langle a?b \rangle E} = \{ P \in \pi | P \equiv Q | u?(y)P', \\ \and \\\\ \and \\ \;\;\; u \in \meaningof{a}, \forall z.P'\{z/y\} \in \meaningof{E\{z/b\}}\}, \and \\ \meaningof{a!E} = \{ P \in \pi | P \equiv Q | x!\langle P' \rangle, x \in \meaningof{a} P' \in \meaningof{E}\} }
\end{mathpar}

\begin{mathpar}
 \inferrule* [lab=nominal] {} {\meaningof{\quotep{E}} = \{ \quotep{P} \in \quotep{\pi} | P \in \meaningof{E} \}, \and \meaningof{\quotep{P}} = \{ \quotep{Q} \in \quotep{\pi} | P \equiv Q \} \and \\ \meaningof{@\quotep{E}} = \{ P \in \pi | P \equiv @x, x \in \meaningof{E} \}}
\end{mathpar}

\begin{eqnarray*}
  \\
  \meaningof{-} : TS \to ST
\end{eqnarray*}

\begin{eqnarray*}
  \\
  L : TS \to ST
\end{eqnarray*}

\begin{eqnarray*}
  \\
  P \models E \iff P \in \meaningof{E}
\end{eqnarray*}

\begin{eqnarray*}
  P \approx_{L} Q \iff \forall E \in L. P \models E \iff Q \models E
\end{eqnarray*}

\begin{eqnarray*}
  P \approx_{K} Q
\end{eqnarray*}

\begin{eqnarray*}
  P \approx Q
\end{eqnarray*}

$\approx_{K} = \approx = \approx_{L}$

\subsubsection{Contextual duality}

Note that contexts extend the quotation operation to a family of
operations from processes to names. Given a context, $M$, we can
define a \emph{nominal context}, $\quotep{M}$ by $\quotep{M}[P] :=
\quotep{M[P]}$. To foreshadow what is to come we observe that these
operations enjoy a duality with processes very much like the duality
between vectors and maps from vectors to scalars.

Further, because the calculus is essentially higher-order, we have a
correspondence between contexts and processes. More specifically,
given a name $x$ and a context $M$ we can construct $M^{*}_{x}$ such
that 

\begin{mathpar}
  M^{*}_{x} | \lift{x}{P} \red M[P]
\end{mathpar}

namely,

\begin{mathpar}
  M^{*}_{x} := x?(u).M[\dropn{u}]
\end{mathpar}

The dependence of $M^{*}_{x}$ on a name makes it an abstraction, 

\begin{mathpar}
  M^{*} := (x)x?(u).M[\dropn{u}]
\end{mathpar}

\subsection{Additional notation}

It will sometimes be convenient to denote the process a name
quotes. We already have the notation $x = \quotep{P}$, but it will be
convenient to introduce an alternate notation, $\procn{x}$, when we
want to emphasize the connection to the use of the name. Note that, by
virtue of name equivalence, $\quotep{\procn{x}} \nameeq x$; so, the
notation is consistent with previous definitions.

Further, because names have structure it is possible to effect
substitutions on the basis of that structure. This means we need to
upgrade our notation for substitutions, which we accomplish by
adapting comprehension notation. Thus,

\begin{mathpar}
  P\{ y / x : x \in S \}
\end{mathpar}

is interpreted to mean the process derived from P by replacing (in a
capture-avoiding manner) each occurrence of $x$ in $S$ by $y$. For example,

\begin{mathpar}
  P\{ \quotep{\procn{x}|\procn{x}} / x : x \in \freenames{P} \}
\end{mathpar}

will replace each (occurrence) of a free name $x$ in $P$ by
$\quotep{\procn{x}|\procn{x}}$.

Also, we will avail ourselves of the notation $x^{L}$ and $x^{R}$ to
denote injections of a name into disjoint copies of the name
space. There are numerous ways to accomplish this. One example can be
found in \cite{MeredithR05}. This notation overloads to vectors of
names: $\vec{x}^{\pi} := (x_{i}^{\pi} \; : \; 0 \leq i < |\vec{x}| )$ where $\pi \in \{L,R\}$.

We also use $P^{\Box} := P|\Box$.

In \cite{MeredithR05} an interpretation of the new operator is
given. It turns out that there are several possible interpretations
all enjoying the requisite algebraic properties of the operator (see
\cite{milner91polyadicpi}). We will therefore make liberal use of
$(\nu\; \vec{x})P$.

% subsection the_syntax_and_semantics_of_the_notation_system (end)   

\input{qm2pi.qmops} 

\input{qm2pi.sterngerlach} 

\input{qm2pi.metric} 

% section concurrent_process_calculi (end)

%\input{qm2pi.proofsketch}

% section proof sketch (end)

%\input{qm2pi.slviaknots} 

% section spatial logic via knots (end)

\input{qm2pi.conclusion}

% section conclusion (end)

%\input{qm2pi.dtcodes} 

% section wiring algorithm (end)

\input{qm2pi.ack} 

% section acknowledgments (end)

\newpage


\bibliographystyle{plain}   
\bibliography{../../biblios/main.bib}

\input{qm2pi.rhodetails}

\end{document}

 

% subsection basic_interpretation (end)

%\input{qm2pi.rho.presentation} 
\subsection{The syntax and semantics of the notation system}\label{sub:the_syntax_and_semantics_of_the_notation_system} % (fold)

We now summarize a technical presentation of the calculus that
embodies our theory of dynamics. The typical presentation of such a
calculus follows the style of giving generators and relations on
them. The grammar, below, describing term constructors, freely
generates the set of processes, $\Proc$. This set is then quotiented
by a relation known as structural congruence and it is over this set
that the notion of dynamics is expressed. This presentation is
essentially that of \cite{MeredithR05} with the addition of
polyadicity and summation. For readability we have relegated some of
the technical subtleties to an appendix.

\subsubsection{Process grammar}\label{subsub:process_grammar}

\begin{mathpar}
  \inferrule* [lab=synchronization] {} {{M} \bc \pzero \;|\; x?F \;|\; x!C }
  \and
  \inferrule* [lab=abstraction] {} {{F} \bc (x)P}
  \and
  \inferrule* [lab=concretion] {} {{C} \bc \langle Q \rangle}
  \and
  \inferrule* [lab=process] {} {{P,Q} \bc M \;| \;P|Q \;|\; @{x}}
  \and
  \inferrule* [lab=name] {} {{x} \bc \quotep{P}}
\end{mathpar} 

Note that $\vec{x}$ (resp. $\vec{P}$) denotes a vector of names
(resp. processes) of length $|\vec{x}|$ (resp. $|\vec{P}|$). We adopt
the following useful abbreviations.

\begin{mathpar}
   x?(\vec{y}).P := x.(\vec{y})P \and  x\clift{\vec{P}} := x.\clift{\vec{P}}
   \and x!(y) := \lift{x}{\dropn{y}}
   \and \Pi_{i=0}^{n-1}P_i := P_0 | \ldots | P_{n-1}
\end{mathpar}

\subsubsection{Structural congruence}

\paragraph{Free and bound names and alpha-equivalence.} At the
core of structural equivalence is alpha-equivalence which identifies
process that are the same up to a change of variable. Formally, we
recognize the distinction between free and bound names. The free names
of a process, $\freenames{P}$, may be calculated recursively as
follows:

\begin{mathpar}
\freenames{\pzero} := \emptyset
  \and \\
  \freenames{x?(y).P} := \{ x \} \cup (\freenames{P} \setminus \{ y \})
  \and 
  \freenames{x!\langle P \rangle} := \{ x \} \cup \{ P \} 
  \and \\
  \freenames{P|Q} := \freenames{P} \cup \freenames{Q}
  \and \\
  \freenames{@{x}} := \{ x \}
\end{mathpar}

$\pi$
$\quotep{\pi}$

$\freenames{-} : \pi \to \mathcal{P}(\quotep{\pi})$

\begin{eqnarray*}
  \freenames{\pzero} & := & \emptyset \\
  \freenames{x?(y).P} & := & \{ x \} \cup (\freenames{P} \setminus \{ y \}) \\
  \freenames{x!\langle P \rangle} & := & \{ x \} \cup \{ P \} \\
  \freenames{P|Q} & := & \freenames{P} \cup \freenames{Q} \\
  \freenames{\dropn{x}} & := & \{ x \}
\end{eqnarray*}

The bound names of a process, $\boundnames{P}$, are those names occurring in $P$
that are not free. For example, in $x?(y).0$, the name $x$ is free, while $y$ is bound.

\begin{mathpar}
  \inferrule* [lab=monoidal-laws] {} { P|Q \equiv Q|P \and P|0 \equiv P \and P|(Q|R) \equiv (P|Q)|R }
\end{mathpar}

\begin{mathpar}
  \inferrule* [lab=alpha-equivalence] {} { (x)P \equiv (y)P\{y/x\} \and y \not\in \freenames{P} }
\end{mathpar}

\begin{definition}
Then two processes, $P,Q$, are alpha-equivalent if $P = Q\{\vec{y}/\vec{x}\}$ for
some $\vec{x} \in \boundnames{Q},\vec{y} \in \boundnames{P}$, where $Q\{\vec{y}/\vec{x}\}$
denotes the capture-avoiding substitution of $\vec{y}$ for $\vec{x}$ in $Q$.
\end{definition}

\begin{definition}
  The {\em structural congruence} \cite{SangiorgiWalker} , $\equiv$,
  between processes is the least congruence containing
  alpha-equivalence, satisfying the abelian monoid laws
  (associativity, commutativity and $\pzero$ as identity) for parallel
  composition $|$ and for summation $+$.
\end{definition}

\subsection{Name equivalence}

We take name equivalence, written $\nameeq$, to be the smallest
equivalence relation generated by the following rules.

\begin{mathpar}
\inferrule*[lab=Quote-drop]
{ }
{ \quotep{@{x}} \nameeq x }

\inferrule*[lab=Struct-equiv]
{ P \scong Q }
{ \quotep{P} \nameeq \quotep{Q} }
\end{mathpar}

The astute reader will have noticed that the mutual recursion of names
and processes imposes a mutual recursion on alpha-equivalence and
structural equivalence via name-equivalence. Fortunately, all of this
works out pleasantly and we may calculate in the natural way, free of
concern. The reader interested in the details is referred to the
appendix \ref{appendix:rho_details}.

\subsection{Substitution}

We use $\Proc$ for the set of processes, $\QProc$ for the set of
names, and $\id{\{}\vec{y} / \vec{x} \id{\}}$ to denote partial maps,
$s : \QProc \rightarrow \QProc$. A map, $s$ lifts, uniquely, to a map
on process terms, $\widehat{s} : \Proc \rightarrow \Proc$ by the
following equations.

\begin{mathpar}
  (0) \psubstp{Q}{P} := 0 \\
  (R \juxtap S) \psubstp{Q}{P}
  :=    
  (R)\psubstp{Q}{P} \juxtap (S) \psubstp{Q}{P} \\
  (x?(y).R) \psubstp{Q}{P}    
  :=    
  (x)\substp{Q}{P} (z)\concat( (R \psubstn{z}{y}) \psubstp{Q}{P} ) \\
  (\lift{x}{R}) \psubstp{Q}{P}  
  :=
  \lift{(x)\substp{Q}{P}}{ R \psubstp{Q}{P} } \\
%   (\dropn{x})  \psubstp{Q}{P}       
%   := 
%   \left\{ 
%     \begin{array}{ccc} 
%       \dropn{\quotep{Q}} & & x \nameeq \quotep{P} \\
%       \dropn{x} & & otherwise \\
%     \end{array}
%   \right. 
  (\dropn{x})  \psubstp{Q}{P}       
  := 
  \left\{ 
    \begin{array}{ccc} 
      Q & & x \nameeq \quotep{P} \\
      \dropn{x} & & otherwise \\
    \end{array}
  \right.
\end{mathpar}
 

where

\begin{eqnarray}
  (x)\id{\{} \lpquote Q \rpquote / \lpquote P \rpquote \id{\}}            = 
  \left\{ 
    \begin{array}{ccc}
      \lpquote Q \rpquote & & x \nameeq \lpquote P \rpquote \\
      x & & otherwise \\
    \end{array}
  \right. \nonumber
\end{eqnarray}

and $z$ is chosen distinct from $\quotep{P}$, $\quotep{Q}$, the free
names in $Q$, and all the names in $R$. Our $\alpha$-equivalence will
be built in the standard way from this substitution.

\begin{remark}\label{rem:no_self_referential_names}
  One consequence of these definitions is that $\forall P. \quotep{P}
  \not\in \freenames{P}$.
\end{remark}

\subsection{ Dynamic quote: an example }

Anticipating something of what's to come, consider applying the
substitution, $\widehat{\id{\{}u / z \id{\}}}$, to the following pair
of processes, $\lift{w}{y!(z)}$ and $w[ \lpquote y!(z) \rpquote ]$.

\begin{eqnarray}
	\lift{w}{y!(z)}\widehat{\id{\{}u / z \id{\}}}
		& = &
		\lift{w}{y!(u)} \nonumber\\
	w[ \lpquote y!(z) \rpquote ] \widehat{ \id{\{}u / z \id{\}} }
		& = &
		w[ \lpquote y!(z) \rpquote ] \nonumber
\end{eqnarray}

Because the body of the process between quotes is impervious to
substitution, we get radically different answers. In fact, by
examining the first process in an input context,
e.g. $x?(z).\lift{w}{y!(z)}$, we see that the process under the lift
operator may be shaped by prefixed inputs binding a name inside it. In
this sense, the lift operator will be seen as a way to dynamically
construct processes before reifying them as names.

Finally equipped with these standard features we can present the
dynamics of the calculus.

\subsubsection{Operational semantics} 

Finally, we introduce the computational dynamics. What marks these
algebras as distinct from other more traditionally studied algebraic
structures, e.g. vector spaces or polynomial rings, is the manner in
which dynamics is captured. In traditional structures, dynamics is typically
expressed through morphisms between such structures, as in linear maps
between vector spaces or morphisms between rings. In algebras
associated with the semantics of computation, the dynamics is
expressed as part of the algebraic structure itself, through a
reduction reduction relation typically denoted by $\red$. Below, we
give a recursive presentation of this relation for the calculus used
in the encoding.

$\red \subseteq \pi \times \pi$
$\red : \pi \to \mathcal{P}(\pi)$

\begin{mathpar}
  \inferrule* [lab=Comm] { \textsf{match}( x_{src}, x_{trgt} ) } { x_{trgt}?(y)P \; | \; x_{src}!\langle {Q} \rangle \red P\{\quotep{Q}/y}\} }
  \and \\
  \inferrule* [lab=Par] {{P} \red {P}'} {{{P} | {Q}} \red {{P}' | {Q}}}
  \and
  \inferrule* [lab=Equiv]{{{P} \scong {P}'} \andalso {{P}' \red {Q}'} \andalso {{Q}' \scong {Q}}}{{P} \red {Q}}
\end{mathpar}

\begin{eqnarray*}
  match_{\equiv} (\quotep{P},\quotep{Q}) & := & P \equiv Q \\
  match_{\dagger}(\quotep{P},\quotep{Q}) & := & \forall R. P|Q \red^{*} R => R \red^{*} 0 \\
  match_{K}(\quotep{P},\quotep{Q}) & := & K \mbox{ for some context } K
\end{eqnarray*}

$u?(x)P | u!\langle Q \rangle \red P\{\quotep{Q}/x\}$

%We write $\wred$ for $\red^*$, and $P\red$ if $\exists Q $ such that $ P \red Q$.
We write $P\red$ if $\exists Q $ such that $ P \red Q$ and $P\not\red$, otherwise.

\section{Replication}

As mentioned before, it is known that replication (and hence
recursion) can be implemented in a higher-order process algebra
\cite{SangiorgiWalker}. As our first example of calculation with the
machinery thus far presented we give the construction explicitly in
the {\rhoc}.

\begin{eqnarray}
	D_{x} & := & \prefix{x}{y}{(\binpar{\outputp{x}{y}}{@{y}})} \nonumber\\
	\bangp_{x}{P} & := & \binpar{{x}!\langle{\binpar{D_{x}}{P}}\rangle}{D_{x}} \nonumber
\end{eqnarray}

\begin{eqnarray}
	\bangp_{x}{P} & & \nonumber\\
	=
	& {x}!\langle{(\prefix{x}{y}{(\outputp{x}{y} | @{y})) | P}}\rangle 
	      | \prefix{x}{y}{(\outputp{x}{y} | @{y})} & \nonumber\\
	\red
	& (\outputp{x}{y} | @{y})\substn{\quotep{(\prefix{x}{y}{(@{y} | \outputp{x}{y})) | P}}}{y} & \nonumber\\
	=
	& \outputp{x}{\quotep{(\prefix{x}{y}{(\outputp{x}{y} | @{y})) | P}}}
	  | {(\prefix{x}{y}{(\outputp{x}{y} | @{y})) | P}} & \nonumber\\
	\red
	& \ldots & \nonumber\\
	\red^*
	& P | P | \ldots & \nonumber
\end{eqnarray}

Of course, this encoding, as an implementation, runs away, unfolding
$\bangp{P}$ eagerly. A lazier and more implementable replication
operator, restricted to input-guarded processes, may be obtained as follows.

\begin{eqnarray}
\bangp{\prefix{u}{v}{P}} 
	:= 
	\binpar{\lift{x}{\prefix{u}{v}{(\binpar{D(x)}{P})}}}{D(x)} \nonumber
\end{eqnarray}

\begin{remark}
  Note that the lazier definition still does not deal with summation
  or mixed summation (i.e. sums over input and output). The reader is
  invited to construct definitions of replication that deal with these
  features. 

  Further, the definitions are parameterized in a name, $x$. Can you,
  gentle reader, make a definition that eliminates this parameter and
  guarantees no accidental interaction between the replication
  machinery and the process being replicated -- i.e. no accidental
  sharing of names used by the process to get its work done and the
  name(s) used by the replication to effect copying. This latter
  revision of the definition of replication is crucial to obtaining
  the expected identity $!!P \sim !P$.
\end{remark}

\begin{remark}\label{rem:paradoxical_combinator}
  The reader familiar with the lambda calculus will have noticed the
  similarity between $D$ and the paradoxical combinator.

  [Ed. note: the existence of this seems to suggest we have to be more
  restrictive on the set of processes and names we admit if we are to
  support no-cloning.]
\end{remark}

\subsubsection{Bisimulation}

The computational dynamics gives rise to another kind of equivalence,
the equivalence of computational behavior. As previously mentioned
this is typically captured \emph{via} some form of bisimulation.

% The notion we use in this paper is weak barbed bisimulation
% \cite{milner91polyadicpi}.

The notion we use in this paper is derived from weak barbed
bisimulation \cite{milner91polyadicpi}. 

\begin{definition}
An \emph{observation relation}, $\downarrow_{\mathcal N}$, over a set
of names, $\mathcal N$, is the smallest relation satisfying the rules
below.

\infrule[Out-barb]{y \in {\mathcal N}, \; x \nameeq y}
		  {\outputp{x}{v} \downarrow_{\mathcal N} x}
\infrule[Par-barb]{\mbox{$P\downarrow_{\mathcal N} x$ or $Q\downarrow_{\mathcal N} x$}}
		  {\binpar{P}{Q} \downarrow_{\mathcal N} x}

We write $P \Downarrow_{\mathcal N} x$ if there is $Q$ such that 
$P \wred Q$ and $Q \downarrow_{\mathcal N} x$.
\end{definition}

\begin{definition}
%\label{def.bbisim}
An  ${\mathcal N}$-\emph{barbed bisimulation} over a set of names, ${\mathcal N}$, is a symmetric binary relation 
${\mathcal S}_{\mathcal N}$ between agents such that $P\rel{S}_{\mathcal N}Q$ implies:
\begin{enumerate}
\item If $P \red P'$ then $Q \wred Q'$ and $P'\rel{S}_{\mathcal N} Q'$.
\item If $P\downarrow_{\mathcal N} x$, then $Q\Downarrow_{\mathcal N} x$.
\end{enumerate}
$P$ is ${\mathcal N}$-barbed bisimilar to $Q$, written
$P \wbbisim_{\mathcal N} Q$, if $P \rel{S}_{\mathcal N} Q$ for some ${\mathcal N}$-barbed bisimulation ${\mathcal S}_{\mathcal N}$.
\end{definition}

$\mathcal{R} \subseteq \pi \times \pi$

$P \mathcal{R} Q => \forall P'. P \red P' \Rightarrow \exists Q'. Q \red Q', P' \mathcal{R} Q'$

$P \vdash x \Rightarrow Q \vdash x$

\begin{mathpar}
  \inferrule*[lab=Out-barb]{x \nameeq y}{{y}!\langle{Q}\rangle \vdash x}
  \and
  \inferrule*[lab=Par-barb]{\mbox{$P\vdash x$ or $Q\vdash x$}}{\binpar{P}{Q} \vdash x}
\end{mathpar}

\subsubsection{Contexts}

One of the principle advantages of computational calculi like the
$\pi$-calculus is a well-defined notion of context,
contextual-equivalence and a correlation between
contextual-equivalence and notions of bisimulation. The notion of
context allows the decomposition of a process into (sub-)process and
its syntactic environment, its context. Thus, a context may be
thought of as a process with a ``hole'' (written $\Box$) in it. The
application of a context $M$ to a process $P$, written $M[P]$, is
tantamount to filling the hole in $M$ with $P$. In this paper we do
not need the full weight of this theory, but do make use of the notion
of context in the proof the main theorem. 

\begin{mathpar}
  \inferrule* [lab=summation] {} {{M_{M},M_{N}} \bc \Box \;|\; x.M_{A} \;|\; M_{M}+M_{N}}
  \and
  \inferrule* [lab=agent] {} {{M_{A}} \bc (\vec{x})M_{P} \;| \; \clift{P_0,\ldots,M_{P},\ldots,P_N}}
  \and \\
  \inferrule* [lab=process] {} {{M_{P}} \bc M_{N} \;| \;P|M_{P} }
\end{mathpar} 

\begin{mathpar}
  \inferrule* [lab=sychronization] {} {M_{N} \bc \Box \;|\; x?M_{F} \;|\; x!M_{C}}
  \and
  \inferrule* [lab=abstraction] {} {{M_{F}} \bc (x)M_{P} }
  \and
  \inferrule* [lab=concretion] {} {{M_{C}} \bc \langle M_{P} \rangle }
  \and \\
  \inferrule* [lab=process] {} {{M_{P}} \bc M_{N} \;| \;P|M_{P} }
\end{mathpar}

\begin{definition}[contextual application] Given a context $M$, and
  process $P$, we define the \emph{contextual application}, $M[P] :=
  M\{P/\Box\}$. That is, the contextual application of M to P is the
  substitution of $P$ for $\Box$ in $M$.
\end{definition}

$\meaningof{-} : L \to \mathcal{P}(\pi)$

\begin{mathpar}
  \inferrule* [lab=collection] {} {\meaningof{true} = \pi, \and \meaningof{~E} = \pi \setminus \meaningof{E}, \and \meaningof{E_{1} \& E_{2}} = \meaningof{E_{1}} \cap \meaningof{E_{2}}}
\end{mathpar}

\begin{mathpar}
  \inferrule* [lab=structure] {} {\meaningof{0} = \{ P \in \pi | P \equiv 0 \}, \and \\ \meaningof{E_1 | E_2} = \{ P \in \pi | P \equiv P_{1} | P_{2}, P_{1} \in \meaningof{E_{1}}, P_{2} \in \meaningof{E_2}\} }
\end{mathpar}

\begin{mathpar}
 \inferrule* [lab=behavior] {} {\meaningof{\langle a?b \rangle E} = \{ P \in \pi | P \equiv Q | u?(y)P', \\ \and \\\\ \and \\ \;\;\; u \in \meaningof{a}, \forall z.P'\{z/y\} \in \meaningof{E\{z/b\}}\}, \and \\ \meaningof{a!E} = \{ P \in \pi | P \equiv Q | x!\langle P' \rangle, x \in \meaningof{a} P' \in \meaningof{E}\} }
\end{mathpar}

\begin{mathpar}
 \inferrule* [lab=nominal] {} {\meaningof{\quotep{E}} = \{ \quotep{P} \in \quotep{\pi} | P \in \meaningof{E} \}, \and \meaningof{\quotep{P}} = \{ \quotep{Q} \in \quotep{\pi} | P \equiv Q \} \and \\ \meaningof{@\quotep{E}} = \{ P \in \pi | P \equiv @x, x \in \meaningof{E} \}}
\end{mathpar}

\begin{eqnarray*}
  \\
  \meaningof{-} : TS \to ST
\end{eqnarray*}

\begin{eqnarray*}
  \\
  L : TS \to ST
\end{eqnarray*}

\begin{eqnarray*}
  \\
  P \models E \iff P \in \meaningof{E}
\end{eqnarray*}

\begin{eqnarray*}
  P \approx_{L} Q \iff \forall E \in L. P \models E \iff Q \models E
\end{eqnarray*}

\begin{eqnarray*}
  P \approx_{K} Q
\end{eqnarray*}

\begin{eqnarray*}
  P \approx Q
\end{eqnarray*}

$\approx_{K} = \approx = \approx_{L}$

\subsubsection{Contextual duality}

Note that contexts extend the quotation operation to a family of
operations from processes to names. Given a context, $M$, we can
define a \emph{nominal context}, $\quotep{M}$ by $\quotep{M}[P] :=
\quotep{M[P]}$. To foreshadow what is to come we observe that these
operations enjoy a duality with processes very much like the duality
between vectors and maps from vectors to scalars.

Further, because the calculus is essentially higher-order, we have a
correspondence between contexts and processes. More specifically,
given a name $x$ and a context $M$ we can construct $M^{*}_{x}$ such
that 

\begin{mathpar}
  M^{*}_{x} | \lift{x}{P} \red M[P]
\end{mathpar}

namely,

\begin{mathpar}
  M^{*}_{x} := x?(u).M[\dropn{u}]
\end{mathpar}

The dependence of $M^{*}_{x}$ on a name makes it an abstraction, 

\begin{mathpar}
  M^{*} := (x)x?(u).M[\dropn{u}]
\end{mathpar}

\subsection{Additional notation}

It will sometimes be convenient to denote the process a name
quotes. We already have the notation $x = \quotep{P}$, but it will be
convenient to introduce an alternate notation, $\procn{x}$, when we
want to emphasize the connection to the use of the name. Note that, by
virtue of name equivalence, $\quotep{\procn{x}} \nameeq x$; so, the
notation is consistent with previous definitions.

Further, because names have structure it is possible to effect
substitutions on the basis of that structure. This means we need to
upgrade our notation for substitutions, which we accomplish by
adapting comprehension notation. Thus,

\begin{mathpar}
  P\{ y / x : x \in S \}
\end{mathpar}

is interpreted to mean the process derived from P by replacing (in a
capture-avoiding manner) each occurrence of $x$ in $S$ by $y$. For example,

\begin{mathpar}
  P\{ \quotep{\procn{x}|\procn{x}} / x : x \in \freenames{P} \}
\end{mathpar}

will replace each (occurrence) of a free name $x$ in $P$ by
$\quotep{\procn{x}|\procn{x}}$.

Also, we will avail ourselves of the notation $x^{L}$ and $x^{R}$ to
denote injections of a name into disjoint copies of the name
space. There are numerous ways to accomplish this. One example can be
found in \cite{MeredithR05}. This notation overloads to vectors of
names: $\vec{x}^{\pi} := (x_{i}^{\pi} \; : \; 0 \leq i < |\vec{x}| )$ where $\pi \in \{L,R\}$.

We also use $P^{\Box} := P|\Box$.

In \cite{MeredithR05} an interpretation of the new operator is
given. It turns out that there are several possible interpretations
all enjoying the requisite algebraic properties of the operator (see
\cite{milner91polyadicpi}). We will therefore make liberal use of
$(\nu\; \vec{x})P$.

% subsection the_syntax_and_semantics_of_the_notation_system (end)   

\section{Interpretation of QM}
\subsection{Supporting definitions}
\subsubsection{Multiplication}
\begin{mathpar}
  \quotep{Q} \cdot \quotep{R} := \quotep{Q|R}
  \and \\
  \quotep{Q} \cdot P := P\{ \quotep{Q|R} / \quotep{R} : \quotep{R} \in \freenames{P} \}
\end{mathpar}

\paragraph{Discussion}
The first line needs little explanation. The second line says that
each free name of the process is replaced with the multiplication of
that name by the scalar. Multiplication of a scalar (name) by a state
(process) results in a process all the names of which have been `moved
over' by parallel composition with the process the scalar
quotes. There is a subtlety that the bound names have to be
manipulated so that multiplied names aren't accidentally
captured. There are many ways to achieve this.

\begin{remark}\label{rem:multiplication_identities}
  The reader is invited to verify that for all $x,y,z \in \QProc$ and $P \in \Proc$
  \begin{mathpar}
    x \cdot \quotep{0} \equiv x 
    \and
    x \cdot y \equiv y \cdot x
    \and
    x \cdot (y \cdot z) \equiv (x \cdot y) \cdot z
    \and \\
    \quotep{0} \cdot P \equiv P
    \and \\
    x \cdot (y \cdot P) \equiv (x \cdot y) \cdot P
    \and \\
    x \cdot (P|Q) \equiv (x \cdot P) | (x \cdot Q)
    \and \\    
  \end{mathpar}
\end{remark}

\subsubsection{Tensor product}

We define a tensor product on processes by structural induction.

\paragraph{Tensor of sums} First note that all summations, including
$\pzero$ and sequence, can be written $\Sigma_{i} x_{i}.A_{i} +
\Sigma_{j} x_{j}.C_{j}$, where we have grouped input-guarded processes
together and output-guarded processes together.

Thus, we can define the tensor product of two summations, $N_{1}\otimes N_{2}$, where

\begin{mathpar}
  N_{1} := \Sigma_{i} x_{i}.A_{i} + \Sigma_{j} x_{j}.C_{j}
  \and
  N_{2} := \Sigma_{i'} y_{i'}.B_{i'} + \Sigma_{j'} y_{j'}.D_{j'} 
\end{mathpar}

as follows.

\begin{mathpar}
  \Sigma_{i} x_{i}.A_{i} + \Sigma_{j} x_{j}.C_{j} \otimes \Sigma_{i'}
  y_{i'}.B_{i'} + \Sigma_{j'} y_{j'}.D_{j'} 
  \and \\
  := \; \Sigma_{i} \Sigma_{i'} \quotep{\stackrel{\vee}{x_{i}}| \stackrel{\vee}{y_{i'}}}.(A_{i}\otimes B_{i'}) \; | \; \Sigma_{i'} \Sigma_{i} \quotep{\stackrel{\vee}{y_{i'}}|\stackrel{\vee}{x_{i}}}.(B_{i'}\otimes A_{i})
  \and
  \;\; | \;\; \Sigma_{j} \Sigma_{j'} \quotep{\stackrel{\vee}{x_{j}}|\stackrel{\vee}{y_{j'}}}.(A_{j}\otimes B_{j'}) \; | \; \Sigma_{j'} \Sigma_{j} \quotep{\stackrel{\vee}{y_{j'}}|\stackrel{\vee}{x_{j}}}.(B_{j'}\otimes A_{j})
\end{mathpar}

\begin{remark}
  Do we need to $x^{L}$ and $y^{R}$ for this construction as well?
\end{remark}

\paragraph{Tensor of parallel compositions} Next, we distribute tensor
over par.

\begin{mathpar}
  P_{1}|P_{2} \otimes Q_{1}|Q_{2} := (P_{1} \otimes Q_{1}) | (P_{1}
  \otimes Q_{2}) | (P_{2} \otimes Q_{1}) | (P_{2} \otimes Q_{2})
\end{mathpar}

\paragraph{Tensor with dropped names} We treat tensor of a
process with a dropped name as parallel composition.

\begin{mathpar}
  P \otimes \dropn{x} := P | \dropn{x}
\end{mathpar}

\paragraph{Tensor of agents}

Finally, we need to define tensor on agents. Note that the definition
of tensor on normal products only tensors inputs with inputs and
outputs with outputs. Thus, we only have to define the operation on
``homogeneous'' pairings.

\begin{mathpar}
  (\vec{x})P \otimes (\vec{y})Q
  \and \\
  := (x_{0}^{L}|y_{0}^{R},\ldots,x_{0}^{L}|y_{n}^{R},\ldots,x_{m}^{L}|y_{0}^{R},\ldots,x_{m}^{L}|y_{n}^R)(P\{ \vec{x}^{L}/\vec{x}\} \otimes Q \{ \vec{y}^{R}/\vec{y}\})
  \and \\
  \clift{\vec{P}} \otimes \clift{\vec{Q}}
  \and \\
  := \clift{P_{0}\otimes Q_{0},\ldots,P_{0}\otimes Q_{n},\ldots,P_{m}\otimes Q_{0},\ldots,P_{m}\otimes Q_{n}}
\end{mathpar}

\begin{remark}
  Observe that arities of tensored abstractions matches arities of
  tensored concretions if the original arities matched. Note also that
  the length of the arities corresponds to the increase in dimension
  we see in ordinary vector space tensor product.
\end{remark}

\begin{remark}
  Operationally, this definition distributes the tensor down to
  components ``linked'' by summation. Tensor over summation is
  intriguing in that it mixes names. Moreover, as a consequence of the
  way it mixes names we have the identities for all $x \in \QProc$ and
  $P,Q \in \Proc$

  \begin{mathpar}
    (x \cdot P) \otimes Q \equiv x \cdot (P \otimes Q) \equiv P \otimes (x \cdot Q)
    \and
    P \otimes \pzero \equiv P
  \end{mathpar}

  that the reader is invited to verify.
\end{remark}

\subsubsection{Annihilation}
\begin{mathpar}
  P^{\perp} := \{ Q | \forall R. P|Q \red^{*} R \Rightarrow R \red^{*} \pzero \}
  \and \\
  P^{\underline{\perp}} := \Sigma_{Q \in P^{\perp}} \quotep{Q}?(y).(\dropn{y}|Q) | \Sigma_{Q \in P^{\perp}} \quotep{Q}\clift{\Box}
\end{mathpar}

\paragraph{Discussion} The reader will note that $P^{\perp}$ is a
\emph{set} of processes, while $P^{\underline{\perp}}$ is a
\emph{context}. We call the set $P^{\perp}$ the \emph{annihilators} of
$P$. The parallel composition of a process in the annihilators of $P$
with $P$ will result in a process, the state space of which has all
paths eventually leading to $\pzero$. Execution may endure loops; but
under reasonable conditions of fairness (naturally guaranteed under
most notions of bisimulation) such a composite process cannot get
stuck in such a loop and will, eventually pop out and terminate.

The context $P^{\underline{\perp}}$ is ready and willing to ``take the
$P$ out of'' the process to which it is applied. It will effectively
transmit the code of the process to which it is applied to one of the
annihilators and run the process against it.

\subsubsection{Evaluation}
We fix $M$ a domain of fully abstract interpretation with an equality
coincident with bisimulation. We take $\meaningof{\cdot} : \Proc \to
M$ to be the map interpreting processes and $\nmeaningof{\cdot} : \M
\to Proc$ to be the map running the other way. Then we define

\begin{mathpar}
  \int P := \nmeaningof{\meaningof{P}}
\end{mathpar}

\paragraph{Discussion}
There are many fully abstract interpretations of Milner's
$\pi$-calculus. Any of them can be used as a basis for interpreting
the reflective calculus here. Equipped with such a domain it is
largely a matter of grinding through to check that the Yoneda
construction for the normalization-by-evaluation program can be
extended to this setting.

\begin{remark}
  The reader is invited to verify that $\int (P^{\underline{\perp}}[P]) = 0$.
\end{remark}

\subsection{Quantum mechanics}

Table \ref{tbl:core_qm_op_defns} gives the core operational definitions

\begin{table}[htp]\label{tbl:core_qm_op_defns}
  \center{
    \fbox{
      \begin{tabular}{c|c}
        quantum mechanics & process calculus \\
        \hline
        scalar & $x := \quotep{P}$ \\
        state vector & $\state{P} := P$ \\
        dual & $\state{P}^{*} := \event{P^{\underline{\perp}}} := \quotep{P^{\underline{\perp}}}[-]$ \\
        matrix & $ \Sigma_{\alpha} \state{P_{\alpha}}x_{\alpha}\event{Q_{\alpha}}$ \\
        vector addition & $\state{P} + \state{Q} := \state{P | Q}$ \\
        tensor product & $\state{P} \otimes \state{Q} := \state{P \otimes Q}$ \\
        inner product & $\innerprod{P}{Q} := \quotep{\int P^{\underline{\perp}}[Q]}$ \\
      \end{tabular}
    }
  }
  \caption{QM - operational definitions}
\end{table}

where

\begin{mathpar}
  \prmatrix{P}{Q} := \fprmatrix{P}{\quotep{\pzero}}{Q}
  \and
  \fprmatrix{P}{x}{Q} := (\state{P},x,\event{Q})
  \and
  (\fprmatrix{P}{x}{Q})(\state{R}) := x \cdot \innerprod{Q}{R} \cdot \state{P}
  \and
  (\fprmatrix{P}{x}{Q})(\event{R}) := x \cdot \innerprod{R}{P} \cdot \event{Q}
\end{mathpar}

\paragraph{Discussion}
As promised: vectors (aka states) are represented as processes; duals
as contextual duals; inner product definition should be compared with
standard inner product definition for ....

\begin{remark}
  Assuming $\int (P^{\underline{\perp}}[P]) = 0$, the reader is
  invited to verify that $(\fprmatrix{P}{x}{P})(\state{P}) = x \cdot \state{P}$.
\end{remark}

\begin{remark}
  The reader is invited to verify that $\innerprod{P}{Q}$ could
  equally well have been written $\quotep{\int \stackrel{\vee}{x}}$
  where $x = \event{P^{\underline{\perp}}}(Q)$.

  One of the motivations for this remark is that there is another way
  to factor these operations. We could package up evaluation in the dual:

  \begin{mathpar}
    \state{P}^{*} := \event{\int P^{\underline{\perp}}} := \quotep{\int P^{\underline{\perp}}}[-]
  \end{mathpar}

  and then have inner product defined by
  
  \begin{mathpar}
    \innerprod{P}{Q} := \event{P}(Q)
  \end{mathpar}

  Hopefully, experience with the calculations will provide guidance on
  the best factoring.
\end{remark}

\begin{remark}
  Assuming $\int (P^{\underline{\perp}}[P]) = 0$, the reader is
  invited to verify that $\forall P,Q. (\prmatrix{0}{Q})(\state{0}) =
  \state{0}$ and dually $(\prmatrix{P}{0})(\event{0}) = \event{0}$.
\end{remark}

\begin{remark}
  i'm a little worried that i don't (yet) have proper support for
  complex conjugacy. But, the observation above may give us a
  clue. According to Abramsky, it must be the case that the scalars
  are iso to the homset of the identity for the tensor -- which the
  observation above characterizes. 

  For now, we will simply bookmark the notion with $\overline{x}$.
\end{remark}

\subsubsection{Adjointness}

We need to give a definition of $(\cdot)^{\dagger}$ for matrices. The
obvious candidate definition is
\begin{mathpar}
(\Sigma_{\alpha}\fprmatrix{P_{\alpha}}{x_{\alpha}}{Q_{\alpha}})^{\dagger}
= \Sigma_{\alpha}\fprmatrix{(Q_{\alpha}^{\underline{\perp}})^{*}}{\overline{x}_{\alpha}}{P_{\alpha}^{\underline{\perp}}} 
\end{mathpar}

But, $(Q_{\alpha}^{\underline{\perp}})^{*}$ requires a name along
which to communicate the process to achieve the context application.

\subsubsection{Basis for a basis}
If processes label states and ``addition'' of states (a.k.a. vector
addition) is interpreted as parallel composition, what corresponds to
notions of linear independence and basis? Here, we recall that Yoshida
has developed a set of \emph{combinators} for an asynchronous verison
of Milner's $\pi$-calculus. These are a finite set of processes such
any process can be expressed as parallel composition of these
combinators together with liberal uses of the new operator and
replication. We can simply give a translation of these into the
present calculus and have reasonable expectation that the property
carries over. That is, that the resultant set allows to express all
processes via parallel composition. Note, however, that there is no
new operator or replication in this calculus. As a result, we expect
that the corresponding set is actually infinite. That is, we expect
that the space is actually infinite dimensional.

\begin{remark}
  The attentive reader may be a bit concerned. Certainly, the
  collection $S$, $K$ and $I$ is a finite set of
  combinators. Shouldn't we expect to see a finite set of combinators
  for an effectively equivalent system? i am very sympathetic to this
  critique and feel it warrants full attention. On the other hand, i
  also have in mind the following analogy. The natural numbers, as a
  monoid under addition, has exactly $1$ generator, while the natural
  numbers, as a monoid under multiplication, has countably many
  generators (the primes). We observe that the application of the
  lambda calculus is much less resource sensitive than the parallel
  composition of the $\pi$-calculus. Could it be the case that we have
  an analogy of the form
  
  \begin{mathpar}
    m + n : MN :: m*n : M|N
  \end{mathpar}

  giving a similar blow up in the set of ``primes''?  This is such a
  wonderful thought that, even if it's not true, i think it's worth
  writing down.
\end{remark}
 

\documentclass[12pt]{llncs}
%\documentclass{jktr}

\usepackage[pdftex]{hyperref}                   
\usepackage {listings}
\usepackage {mathpartir}
\usepackage{bcprules}
%\usepackage{listings}
                       
\usepackage{graphicx} 
%\usepackage[margins=2.5cm,nohead,nofoot]{geometry}
%\usepackage{geometry}
\usepackage{amsfonts}
\usepackage{amstext}
\usepackage{latexsym}
\usepackage{amssymb}
\usepackage{color}


%\include{myPreamble}
\include{qm2pi.local} 

%\ifpdf
%\usepackage[pdftex]{graphicx}
%\else
%\usepackage{graphicx}
%\fi

 % \ifpdf
%  \usepackage{pdfsync}
%  \if


%\title{Brief Article}
%\author{David F. Snyder}
%\author{L.G. Meredith}

%\address{Dept. of Math., Texas State University--San Marcos, San Marcos, TX 78666}
       
\pagestyle{empty}


\begin{document}

\lstset{language=[Objective]Caml,frame=shadowbox}

\input{qm2pi.front}

% section front matter (end)

\input{qm2pi.intro} 
 
% section introduction (end)

% \input{qm2pi.knotations} 

% section notation (end)

\input{qm2pi.process.calculi} 

% section concurrent_process_calculi_and_spatial_logics_ (end)
    
%\input{qm2pi.knots2pi} 

%\input{qm2pi.trefoil} 

%\input{qm2pi.mainthm} 

% subsection basic_interpretation (end)

%\input{qm2pi.rho.presentation} 
\subsection{The syntax and semantics of the notation system}\label{sub:the_syntax_and_semantics_of_the_notation_system} % (fold)

We now summarize a technical presentation of the calculus that
embodies our theory of dynamics. The typical presentation of such a
calculus follows the style of giving generators and relations on
them. The grammar, below, describing term constructors, freely
generates the set of processes, $\Proc$. This set is then quotiented
by a relation known as structural congruence and it is over this set
that the notion of dynamics is expressed. This presentation is
essentially that of \cite{MeredithR05} with the addition of
polyadicity and summation. For readability we have relegated some of
the technical subtleties to an appendix.

\subsubsection{Process grammar}\label{subsub:process_grammar}

\begin{mathpar}
  \inferrule* [lab=synchronization] {} {{M} \bc \pzero \;|\; x?F \;|\; x!C }
  \and
  \inferrule* [lab=abstraction] {} {{F} \bc (x)P}
  \and
  \inferrule* [lab=concretion] {} {{C} \bc \langle Q \rangle}
  \and
  \inferrule* [lab=process] {} {{P,Q} \bc M \;| \;P|Q \;|\; @{x}}
  \and
  \inferrule* [lab=name] {} {{x} \bc \quotep{P}}
\end{mathpar} 

Note that $\vec{x}$ (resp. $\vec{P}$) denotes a vector of names
(resp. processes) of length $|\vec{x}|$ (resp. $|\vec{P}|$). We adopt
the following useful abbreviations.

\begin{mathpar}
   x?(\vec{y}).P := x.(\vec{y})P \and  x\clift{\vec{P}} := x.\clift{\vec{P}}
   \and x!(y) := \lift{x}{\dropn{y}}
   \and \Pi_{i=0}^{n-1}P_i := P_0 | \ldots | P_{n-1}
\end{mathpar}

\subsubsection{Structural congruence}

\paragraph{Free and bound names and alpha-equivalence.} At the
core of structural equivalence is alpha-equivalence which identifies
process that are the same up to a change of variable. Formally, we
recognize the distinction between free and bound names. The free names
of a process, $\freenames{P}$, may be calculated recursively as
follows:

\begin{mathpar}
\freenames{\pzero} := \emptyset
  \and \\
  \freenames{x?(y).P} := \{ x \} \cup (\freenames{P} \setminus \{ y \})
  \and 
  \freenames{x!\langle P \rangle} := \{ x \} \cup \{ P \} 
  \and \\
  \freenames{P|Q} := \freenames{P} \cup \freenames{Q}
  \and \\
  \freenames{@{x}} := \{ x \}
\end{mathpar}

$\pi$
$\quotep{\pi}$

$\freenames{-} : \pi \to \mathcal{P}(\quotep{\pi})$

\begin{eqnarray*}
  \freenames{\pzero} & := & \emptyset \\
  \freenames{x?(y).P} & := & \{ x \} \cup (\freenames{P} \setminus \{ y \}) \\
  \freenames{x!\langle P \rangle} & := & \{ x \} \cup \{ P \} \\
  \freenames{P|Q} & := & \freenames{P} \cup \freenames{Q} \\
  \freenames{\dropn{x}} & := & \{ x \}
\end{eqnarray*}

The bound names of a process, $\boundnames{P}$, are those names occurring in $P$
that are not free. For example, in $x?(y).0$, the name $x$ is free, while $y$ is bound.

\begin{mathpar}
  \inferrule* [lab=monoidal-laws] {} { P|Q \equiv Q|P \and P|0 \equiv P \and P|(Q|R) \equiv (P|Q)|R }
\end{mathpar}

\begin{mathpar}
  \inferrule* [lab=alpha-equivalence] {} { (x)P \equiv (y)P\{y/x\} \and y \not\in \freenames{P} }
\end{mathpar}

\begin{definition}
Then two processes, $P,Q$, are alpha-equivalent if $P = Q\{\vec{y}/\vec{x}\}$ for
some $\vec{x} \in \boundnames{Q},\vec{y} \in \boundnames{P}$, where $Q\{\vec{y}/\vec{x}\}$
denotes the capture-avoiding substitution of $\vec{y}$ for $\vec{x}$ in $Q$.
\end{definition}

\begin{definition}
  The {\em structural congruence} \cite{SangiorgiWalker} , $\equiv$,
  between processes is the least congruence containing
  alpha-equivalence, satisfying the abelian monoid laws
  (associativity, commutativity and $\pzero$ as identity) for parallel
  composition $|$ and for summation $+$.
\end{definition}

\subsection{Name equivalence}

We take name equivalence, written $\nameeq$, to be the smallest
equivalence relation generated by the following rules.

\begin{mathpar}
\inferrule*[lab=Quote-drop]
{ }
{ \quotep{@{x}} \nameeq x }

\inferrule*[lab=Struct-equiv]
{ P \scong Q }
{ \quotep{P} \nameeq \quotep{Q} }
\end{mathpar}

The astute reader will have noticed that the mutual recursion of names
and processes imposes a mutual recursion on alpha-equivalence and
structural equivalence via name-equivalence. Fortunately, all of this
works out pleasantly and we may calculate in the natural way, free of
concern. The reader interested in the details is referred to the
appendix \ref{appendix:rho_details}.

\subsection{Substitution}

We use $\Proc$ for the set of processes, $\QProc$ for the set of
names, and $\id{\{}\vec{y} / \vec{x} \id{\}}$ to denote partial maps,
$s : \QProc \rightarrow \QProc$. A map, $s$ lifts, uniquely, to a map
on process terms, $\widehat{s} : \Proc \rightarrow \Proc$ by the
following equations.

\begin{mathpar}
  (0) \psubstp{Q}{P} := 0 \\
  (R \juxtap S) \psubstp{Q}{P}
  :=    
  (R)\psubstp{Q}{P} \juxtap (S) \psubstp{Q}{P} \\
  (x?(y).R) \psubstp{Q}{P}    
  :=    
  (x)\substp{Q}{P} (z)\concat( (R \psubstn{z}{y}) \psubstp{Q}{P} ) \\
  (\lift{x}{R}) \psubstp{Q}{P}  
  :=
  \lift{(x)\substp{Q}{P}}{ R \psubstp{Q}{P} } \\
%   (\dropn{x})  \psubstp{Q}{P}       
%   := 
%   \left\{ 
%     \begin{array}{ccc} 
%       \dropn{\quotep{Q}} & & x \nameeq \quotep{P} \\
%       \dropn{x} & & otherwise \\
%     \end{array}
%   \right. 
  (\dropn{x})  \psubstp{Q}{P}       
  := 
  \left\{ 
    \begin{array}{ccc} 
      Q & & x \nameeq \quotep{P} \\
      \dropn{x} & & otherwise \\
    \end{array}
  \right.
\end{mathpar}
 

where

\begin{eqnarray}
  (x)\id{\{} \lpquote Q \rpquote / \lpquote P \rpquote \id{\}}            = 
  \left\{ 
    \begin{array}{ccc}
      \lpquote Q \rpquote & & x \nameeq \lpquote P \rpquote \\
      x & & otherwise \\
    \end{array}
  \right. \nonumber
\end{eqnarray}

and $z$ is chosen distinct from $\quotep{P}$, $\quotep{Q}$, the free
names in $Q$, and all the names in $R$. Our $\alpha$-equivalence will
be built in the standard way from this substitution.

\begin{remark}\label{rem:no_self_referential_names}
  One consequence of these definitions is that $\forall P. \quotep{P}
  \not\in \freenames{P}$.
\end{remark}

\subsection{ Dynamic quote: an example }

Anticipating something of what's to come, consider applying the
substitution, $\widehat{\id{\{}u / z \id{\}}}$, to the following pair
of processes, $\lift{w}{y!(z)}$ and $w[ \lpquote y!(z) \rpquote ]$.

\begin{eqnarray}
	\lift{w}{y!(z)}\widehat{\id{\{}u / z \id{\}}}
		& = &
		\lift{w}{y!(u)} \nonumber\\
	w[ \lpquote y!(z) \rpquote ] \widehat{ \id{\{}u / z \id{\}} }
		& = &
		w[ \lpquote y!(z) \rpquote ] \nonumber
\end{eqnarray}

Because the body of the process between quotes is impervious to
substitution, we get radically different answers. In fact, by
examining the first process in an input context,
e.g. $x?(z).\lift{w}{y!(z)}$, we see that the process under the lift
operator may be shaped by prefixed inputs binding a name inside it. In
this sense, the lift operator will be seen as a way to dynamically
construct processes before reifying them as names.

Finally equipped with these standard features we can present the
dynamics of the calculus.

\subsubsection{Operational semantics} 

Finally, we introduce the computational dynamics. What marks these
algebras as distinct from other more traditionally studied algebraic
structures, e.g. vector spaces or polynomial rings, is the manner in
which dynamics is captured. In traditional structures, dynamics is typically
expressed through morphisms between such structures, as in linear maps
between vector spaces or morphisms between rings. In algebras
associated with the semantics of computation, the dynamics is
expressed as part of the algebraic structure itself, through a
reduction reduction relation typically denoted by $\red$. Below, we
give a recursive presentation of this relation for the calculus used
in the encoding.

$\red \subseteq \pi \times \pi$
$\red : \pi \to \mathcal{P}(\pi)$

\begin{mathpar}
  \inferrule* [lab=Comm] { \textsf{match}( x_{src}, x_{trgt} ) } { x_{trgt}?(y)P \; | \; x_{src}!\langle {Q} \rangle \red P\{\quotep{Q}/y}\} }
  \and \\
  \inferrule* [lab=Par] {{P} \red {P}'} {{{P} | {Q}} \red {{P}' | {Q}}}
  \and
  \inferrule* [lab=Equiv]{{{P} \scong {P}'} \andalso {{P}' \red {Q}'} \andalso {{Q}' \scong {Q}}}{{P} \red {Q}}
\end{mathpar}

\begin{eqnarray*}
  match_{\equiv} (\quotep{P},\quotep{Q}) & := & P \equiv Q \\
  match_{\dagger}(\quotep{P},\quotep{Q}) & := & \forall R. P|Q \red^{*} R => R \red^{*} 0 \\
  match_{K}(\quotep{P},\quotep{Q}) & := & K \mbox{ for some context } K
\end{eqnarray*}

$u?(x)P | u!\langle Q \rangle \red P\{\quotep{Q}/x\}$

%We write $\wred$ for $\red^*$, and $P\red$ if $\exists Q $ such that $ P \red Q$.
We write $P\red$ if $\exists Q $ such that $ P \red Q$ and $P\not\red$, otherwise.

\section{Replication}

As mentioned before, it is known that replication (and hence
recursion) can be implemented in a higher-order process algebra
\cite{SangiorgiWalker}. As our first example of calculation with the
machinery thus far presented we give the construction explicitly in
the {\rhoc}.

\begin{eqnarray}
	D_{x} & := & \prefix{x}{y}{(\binpar{\outputp{x}{y}}{@{y}})} \nonumber\\
	\bangp_{x}{P} & := & \binpar{{x}!\langle{\binpar{D_{x}}{P}}\rangle}{D_{x}} \nonumber
\end{eqnarray}

\begin{eqnarray}
	\bangp_{x}{P} & & \nonumber\\
	=
	& {x}!\langle{(\prefix{x}{y}{(\outputp{x}{y} | @{y})) | P}}\rangle 
	      | \prefix{x}{y}{(\outputp{x}{y} | @{y})} & \nonumber\\
	\red
	& (\outputp{x}{y} | @{y})\substn{\quotep{(\prefix{x}{y}{(@{y} | \outputp{x}{y})) | P}}}{y} & \nonumber\\
	=
	& \outputp{x}{\quotep{(\prefix{x}{y}{(\outputp{x}{y} | @{y})) | P}}}
	  | {(\prefix{x}{y}{(\outputp{x}{y} | @{y})) | P}} & \nonumber\\
	\red
	& \ldots & \nonumber\\
	\red^*
	& P | P | \ldots & \nonumber
\end{eqnarray}

Of course, this encoding, as an implementation, runs away, unfolding
$\bangp{P}$ eagerly. A lazier and more implementable replication
operator, restricted to input-guarded processes, may be obtained as follows.

\begin{eqnarray}
\bangp{\prefix{u}{v}{P}} 
	:= 
	\binpar{\lift{x}{\prefix{u}{v}{(\binpar{D(x)}{P})}}}{D(x)} \nonumber
\end{eqnarray}

\begin{remark}
  Note that the lazier definition still does not deal with summation
  or mixed summation (i.e. sums over input and output). The reader is
  invited to construct definitions of replication that deal with these
  features. 

  Further, the definitions are parameterized in a name, $x$. Can you,
  gentle reader, make a definition that eliminates this parameter and
  guarantees no accidental interaction between the replication
  machinery and the process being replicated -- i.e. no accidental
  sharing of names used by the process to get its work done and the
  name(s) used by the replication to effect copying. This latter
  revision of the definition of replication is crucial to obtaining
  the expected identity $!!P \sim !P$.
\end{remark}

\begin{remark}\label{rem:paradoxical_combinator}
  The reader familiar with the lambda calculus will have noticed the
  similarity between $D$ and the paradoxical combinator.

  [Ed. note: the existence of this seems to suggest we have to be more
  restrictive on the set of processes and names we admit if we are to
  support no-cloning.]
\end{remark}

\subsubsection{Bisimulation}

The computational dynamics gives rise to another kind of equivalence,
the equivalence of computational behavior. As previously mentioned
this is typically captured \emph{via} some form of bisimulation.

% The notion we use in this paper is weak barbed bisimulation
% \cite{milner91polyadicpi}.

The notion we use in this paper is derived from weak barbed
bisimulation \cite{milner91polyadicpi}. 

\begin{definition}
An \emph{observation relation}, $\downarrow_{\mathcal N}$, over a set
of names, $\mathcal N$, is the smallest relation satisfying the rules
below.

\infrule[Out-barb]{y \in {\mathcal N}, \; x \nameeq y}
		  {\outputp{x}{v} \downarrow_{\mathcal N} x}
\infrule[Par-barb]{\mbox{$P\downarrow_{\mathcal N} x$ or $Q\downarrow_{\mathcal N} x$}}
		  {\binpar{P}{Q} \downarrow_{\mathcal N} x}

We write $P \Downarrow_{\mathcal N} x$ if there is $Q$ such that 
$P \wred Q$ and $Q \downarrow_{\mathcal N} x$.
\end{definition}

\begin{definition}
%\label{def.bbisim}
An  ${\mathcal N}$-\emph{barbed bisimulation} over a set of names, ${\mathcal N}$, is a symmetric binary relation 
${\mathcal S}_{\mathcal N}$ between agents such that $P\rel{S}_{\mathcal N}Q$ implies:
\begin{enumerate}
\item If $P \red P'$ then $Q \wred Q'$ and $P'\rel{S}_{\mathcal N} Q'$.
\item If $P\downarrow_{\mathcal N} x$, then $Q\Downarrow_{\mathcal N} x$.
\end{enumerate}
$P$ is ${\mathcal N}$-barbed bisimilar to $Q$, written
$P \wbbisim_{\mathcal N} Q$, if $P \rel{S}_{\mathcal N} Q$ for some ${\mathcal N}$-barbed bisimulation ${\mathcal S}_{\mathcal N}$.
\end{definition}

$\mathcal{R} \subseteq \pi \times \pi$

$P \mathcal{R} Q => \forall P'. P \red P' \Rightarrow \exists Q'. Q \red Q', P' \mathcal{R} Q'$

$P \vdash x \Rightarrow Q \vdash x$

\begin{mathpar}
  \inferrule*[lab=Out-barb]{x \nameeq y}{{y}!\langle{Q}\rangle \vdash x}
  \and
  \inferrule*[lab=Par-barb]{\mbox{$P\vdash x$ or $Q\vdash x$}}{\binpar{P}{Q} \vdash x}
\end{mathpar}

\subsubsection{Contexts}

One of the principle advantages of computational calculi like the
$\pi$-calculus is a well-defined notion of context,
contextual-equivalence and a correlation between
contextual-equivalence and notions of bisimulation. The notion of
context allows the decomposition of a process into (sub-)process and
its syntactic environment, its context. Thus, a context may be
thought of as a process with a ``hole'' (written $\Box$) in it. The
application of a context $M$ to a process $P$, written $M[P]$, is
tantamount to filling the hole in $M$ with $P$. In this paper we do
not need the full weight of this theory, but do make use of the notion
of context in the proof the main theorem. 

\begin{mathpar}
  \inferrule* [lab=summation] {} {{M_{M},M_{N}} \bc \Box \;|\; x.M_{A} \;|\; M_{M}+M_{N}}
  \and
  \inferrule* [lab=agent] {} {{M_{A}} \bc (\vec{x})M_{P} \;| \; \clift{P_0,\ldots,M_{P},\ldots,P_N}}
  \and \\
  \inferrule* [lab=process] {} {{M_{P}} \bc M_{N} \;| \;P|M_{P} }
\end{mathpar} 

\begin{mathpar}
  \inferrule* [lab=sychronization] {} {M_{N} \bc \Box \;|\; x?M_{F} \;|\; x!M_{C}}
  \and
  \inferrule* [lab=abstraction] {} {{M_{F}} \bc (x)M_{P} }
  \and
  \inferrule* [lab=concretion] {} {{M_{C}} \bc \langle M_{P} \rangle }
  \and \\
  \inferrule* [lab=process] {} {{M_{P}} \bc M_{N} \;| \;P|M_{P} }
\end{mathpar}

\begin{definition}[contextual application] Given a context $M$, and
  process $P$, we define the \emph{contextual application}, $M[P] :=
  M\{P/\Box\}$. That is, the contextual application of M to P is the
  substitution of $P$ for $\Box$ in $M$.
\end{definition}

$\meaningof{-} : L \to \mathcal{P}(\pi)$

\begin{mathpar}
  \inferrule* [lab=collection] {} {\meaningof{true} = \pi, \and \meaningof{~E} = \pi \setminus \meaningof{E}, \and \meaningof{E_{1} \& E_{2}} = \meaningof{E_{1}} \cap \meaningof{E_{2}}}
\end{mathpar}

\begin{mathpar}
  \inferrule* [lab=structure] {} {\meaningof{0} = \{ P \in \pi | P \equiv 0 \}, \and \\ \meaningof{E_1 | E_2} = \{ P \in \pi | P \equiv P_{1} | P_{2}, P_{1} \in \meaningof{E_{1}}, P_{2} \in \meaningof{E_2}\} }
\end{mathpar}

\begin{mathpar}
 \inferrule* [lab=behavior] {} {\meaningof{\langle a?b \rangle E} = \{ P \in \pi | P \equiv Q | u?(y)P', \\ \and \\\\ \and \\ \;\;\; u \in \meaningof{a}, \forall z.P'\{z/y\} \in \meaningof{E\{z/b\}}\}, \and \\ \meaningof{a!E} = \{ P \in \pi | P \equiv Q | x!\langle P' \rangle, x \in \meaningof{a} P' \in \meaningof{E}\} }
\end{mathpar}

\begin{mathpar}
 \inferrule* [lab=nominal] {} {\meaningof{\quotep{E}} = \{ \quotep{P} \in \quotep{\pi} | P \in \meaningof{E} \}, \and \meaningof{\quotep{P}} = \{ \quotep{Q} \in \quotep{\pi} | P \equiv Q \} \and \\ \meaningof{@\quotep{E}} = \{ P \in \pi | P \equiv @x, x \in \meaningof{E} \}}
\end{mathpar}

\begin{eqnarray*}
  \\
  \meaningof{-} : TS \to ST
\end{eqnarray*}

\begin{eqnarray*}
  \\
  L : TS \to ST
\end{eqnarray*}

\begin{eqnarray*}
  \\
  P \models E \iff P \in \meaningof{E}
\end{eqnarray*}

\begin{eqnarray*}
  P \approx_{L} Q \iff \forall E \in L. P \models E \iff Q \models E
\end{eqnarray*}

\begin{eqnarray*}
  P \approx_{K} Q
\end{eqnarray*}

\begin{eqnarray*}
  P \approx Q
\end{eqnarray*}

$\approx_{K} = \approx = \approx_{L}$

\subsubsection{Contextual duality}

Note that contexts extend the quotation operation to a family of
operations from processes to names. Given a context, $M$, we can
define a \emph{nominal context}, $\quotep{M}$ by $\quotep{M}[P] :=
\quotep{M[P]}$. To foreshadow what is to come we observe that these
operations enjoy a duality with processes very much like the duality
between vectors and maps from vectors to scalars.

Further, because the calculus is essentially higher-order, we have a
correspondence between contexts and processes. More specifically,
given a name $x$ and a context $M$ we can construct $M^{*}_{x}$ such
that 

\begin{mathpar}
  M^{*}_{x} | \lift{x}{P} \red M[P]
\end{mathpar}

namely,

\begin{mathpar}
  M^{*}_{x} := x?(u).M[\dropn{u}]
\end{mathpar}

The dependence of $M^{*}_{x}$ on a name makes it an abstraction, 

\begin{mathpar}
  M^{*} := (x)x?(u).M[\dropn{u}]
\end{mathpar}

\subsection{Additional notation}

It will sometimes be convenient to denote the process a name
quotes. We already have the notation $x = \quotep{P}$, but it will be
convenient to introduce an alternate notation, $\procn{x}$, when we
want to emphasize the connection to the use of the name. Note that, by
virtue of name equivalence, $\quotep{\procn{x}} \nameeq x$; so, the
notation is consistent with previous definitions.

Further, because names have structure it is possible to effect
substitutions on the basis of that structure. This means we need to
upgrade our notation for substitutions, which we accomplish by
adapting comprehension notation. Thus,

\begin{mathpar}
  P\{ y / x : x \in S \}
\end{mathpar}

is interpreted to mean the process derived from P by replacing (in a
capture-avoiding manner) each occurrence of $x$ in $S$ by $y$. For example,

\begin{mathpar}
  P\{ \quotep{\procn{x}|\procn{x}} / x : x \in \freenames{P} \}
\end{mathpar}

will replace each (occurrence) of a free name $x$ in $P$ by
$\quotep{\procn{x}|\procn{x}}$.

Also, we will avail ourselves of the notation $x^{L}$ and $x^{R}$ to
denote injections of a name into disjoint copies of the name
space. There are numerous ways to accomplish this. One example can be
found in \cite{MeredithR05}. This notation overloads to vectors of
names: $\vec{x}^{\pi} := (x_{i}^{\pi} \; : \; 0 \leq i < |\vec{x}| )$ where $\pi \in \{L,R\}$.

We also use $P^{\Box} := P|\Box$.

In \cite{MeredithR05} an interpretation of the new operator is
given. It turns out that there are several possible interpretations
all enjoying the requisite algebraic properties of the operator (see
\cite{milner91polyadicpi}). We will therefore make liberal use of
$(\nu\; \vec{x})P$.

% subsection the_syntax_and_semantics_of_the_notation_system (end)   

\input{qm2pi.qmops} 

\input{qm2pi.sterngerlach} 

\input{qm2pi.metric} 

% section concurrent_process_calculi (end)

%\input{qm2pi.proofsketch}

% section proof sketch (end)

%\input{qm2pi.slviaknots} 

% section spatial logic via knots (end)

\input{qm2pi.conclusion}

% section conclusion (end)

%\input{qm2pi.dtcodes} 

% section wiring algorithm (end)

\input{qm2pi.ack} 

% section acknowledgments (end)

\newpage


\bibliographystyle{plain}   
\bibliography{../../biblios/main.bib}

\input{qm2pi.rhodetails}

\end{document}

 

\documentclass[12pt]{llncs}
%\documentclass{jktr}

\usepackage[pdftex]{hyperref}                   
\usepackage {listings}
\usepackage {mathpartir}
\usepackage{bcprules}
%\usepackage{listings}
                       
\usepackage{graphicx} 
%\usepackage[margins=2.5cm,nohead,nofoot]{geometry}
%\usepackage{geometry}
\usepackage{amsfonts}
\usepackage{amstext}
\usepackage{latexsym}
\usepackage{amssymb}
\usepackage{color}


%\include{myPreamble}
\include{qm2pi.local} 

%\ifpdf
%\usepackage[pdftex]{graphicx}
%\else
%\usepackage{graphicx}
%\fi

 % \ifpdf
%  \usepackage{pdfsync}
%  \if


%\title{Brief Article}
%\author{David F. Snyder}
%\author{L.G. Meredith}

%\address{Dept. of Math., Texas State University--San Marcos, San Marcos, TX 78666}
       
\pagestyle{empty}


\begin{document}

\lstset{language=[Objective]Caml,frame=shadowbox}

\input{qm2pi.front}

% section front matter (end)

\input{qm2pi.intro} 
 
% section introduction (end)

% \input{qm2pi.knotations} 

% section notation (end)

\input{qm2pi.process.calculi} 

% section concurrent_process_calculi_and_spatial_logics_ (end)
    
%\input{qm2pi.knots2pi} 

%\input{qm2pi.trefoil} 

%\input{qm2pi.mainthm} 

% subsection basic_interpretation (end)

%\input{qm2pi.rho.presentation} 
\subsection{The syntax and semantics of the notation system}\label{sub:the_syntax_and_semantics_of_the_notation_system} % (fold)

We now summarize a technical presentation of the calculus that
embodies our theory of dynamics. The typical presentation of such a
calculus follows the style of giving generators and relations on
them. The grammar, below, describing term constructors, freely
generates the set of processes, $\Proc$. This set is then quotiented
by a relation known as structural congruence and it is over this set
that the notion of dynamics is expressed. This presentation is
essentially that of \cite{MeredithR05} with the addition of
polyadicity and summation. For readability we have relegated some of
the technical subtleties to an appendix.

\subsubsection{Process grammar}\label{subsub:process_grammar}

\begin{mathpar}
  \inferrule* [lab=synchronization] {} {{M} \bc \pzero \;|\; x?F \;|\; x!C }
  \and
  \inferrule* [lab=abstraction] {} {{F} \bc (x)P}
  \and
  \inferrule* [lab=concretion] {} {{C} \bc \langle Q \rangle}
  \and
  \inferrule* [lab=process] {} {{P,Q} \bc M \;| \;P|Q \;|\; @{x}}
  \and
  \inferrule* [lab=name] {} {{x} \bc \quotep{P}}
\end{mathpar} 

Note that $\vec{x}$ (resp. $\vec{P}$) denotes a vector of names
(resp. processes) of length $|\vec{x}|$ (resp. $|\vec{P}|$). We adopt
the following useful abbreviations.

\begin{mathpar}
   x?(\vec{y}).P := x.(\vec{y})P \and  x\clift{\vec{P}} := x.\clift{\vec{P}}
   \and x!(y) := \lift{x}{\dropn{y}}
   \and \Pi_{i=0}^{n-1}P_i := P_0 | \ldots | P_{n-1}
\end{mathpar}

\subsubsection{Structural congruence}

\paragraph{Free and bound names and alpha-equivalence.} At the
core of structural equivalence is alpha-equivalence which identifies
process that are the same up to a change of variable. Formally, we
recognize the distinction between free and bound names. The free names
of a process, $\freenames{P}$, may be calculated recursively as
follows:

\begin{mathpar}
\freenames{\pzero} := \emptyset
  \and \\
  \freenames{x?(y).P} := \{ x \} \cup (\freenames{P} \setminus \{ y \})
  \and 
  \freenames{x!\langle P \rangle} := \{ x \} \cup \{ P \} 
  \and \\
  \freenames{P|Q} := \freenames{P} \cup \freenames{Q}
  \and \\
  \freenames{@{x}} := \{ x \}
\end{mathpar}

$\pi$
$\quotep{\pi}$

$\freenames{-} : \pi \to \mathcal{P}(\quotep{\pi})$

\begin{eqnarray*}
  \freenames{\pzero} & := & \emptyset \\
  \freenames{x?(y).P} & := & \{ x \} \cup (\freenames{P} \setminus \{ y \}) \\
  \freenames{x!\langle P \rangle} & := & \{ x \} \cup \{ P \} \\
  \freenames{P|Q} & := & \freenames{P} \cup \freenames{Q} \\
  \freenames{\dropn{x}} & := & \{ x \}
\end{eqnarray*}

The bound names of a process, $\boundnames{P}$, are those names occurring in $P$
that are not free. For example, in $x?(y).0$, the name $x$ is free, while $y$ is bound.

\begin{mathpar}
  \inferrule* [lab=monoidal-laws] {} { P|Q \equiv Q|P \and P|0 \equiv P \and P|(Q|R) \equiv (P|Q)|R }
\end{mathpar}

\begin{mathpar}
  \inferrule* [lab=alpha-equivalence] {} { (x)P \equiv (y)P\{y/x\} \and y \not\in \freenames{P} }
\end{mathpar}

\begin{definition}
Then two processes, $P,Q$, are alpha-equivalent if $P = Q\{\vec{y}/\vec{x}\}$ for
some $\vec{x} \in \boundnames{Q},\vec{y} \in \boundnames{P}$, where $Q\{\vec{y}/\vec{x}\}$
denotes the capture-avoiding substitution of $\vec{y}$ for $\vec{x}$ in $Q$.
\end{definition}

\begin{definition}
  The {\em structural congruence} \cite{SangiorgiWalker} , $\equiv$,
  between processes is the least congruence containing
  alpha-equivalence, satisfying the abelian monoid laws
  (associativity, commutativity and $\pzero$ as identity) for parallel
  composition $|$ and for summation $+$.
\end{definition}

\subsection{Name equivalence}

We take name equivalence, written $\nameeq$, to be the smallest
equivalence relation generated by the following rules.

\begin{mathpar}
\inferrule*[lab=Quote-drop]
{ }
{ \quotep{@{x}} \nameeq x }

\inferrule*[lab=Struct-equiv]
{ P \scong Q }
{ \quotep{P} \nameeq \quotep{Q} }
\end{mathpar}

The astute reader will have noticed that the mutual recursion of names
and processes imposes a mutual recursion on alpha-equivalence and
structural equivalence via name-equivalence. Fortunately, all of this
works out pleasantly and we may calculate in the natural way, free of
concern. The reader interested in the details is referred to the
appendix \ref{appendix:rho_details}.

\subsection{Substitution}

We use $\Proc$ for the set of processes, $\QProc$ for the set of
names, and $\id{\{}\vec{y} / \vec{x} \id{\}}$ to denote partial maps,
$s : \QProc \rightarrow \QProc$. A map, $s$ lifts, uniquely, to a map
on process terms, $\widehat{s} : \Proc \rightarrow \Proc$ by the
following equations.

\begin{mathpar}
  (0) \psubstp{Q}{P} := 0 \\
  (R \juxtap S) \psubstp{Q}{P}
  :=    
  (R)\psubstp{Q}{P} \juxtap (S) \psubstp{Q}{P} \\
  (x?(y).R) \psubstp{Q}{P}    
  :=    
  (x)\substp{Q}{P} (z)\concat( (R \psubstn{z}{y}) \psubstp{Q}{P} ) \\
  (\lift{x}{R}) \psubstp{Q}{P}  
  :=
  \lift{(x)\substp{Q}{P}}{ R \psubstp{Q}{P} } \\
%   (\dropn{x})  \psubstp{Q}{P}       
%   := 
%   \left\{ 
%     \begin{array}{ccc} 
%       \dropn{\quotep{Q}} & & x \nameeq \quotep{P} \\
%       \dropn{x} & & otherwise \\
%     \end{array}
%   \right. 
  (\dropn{x})  \psubstp{Q}{P}       
  := 
  \left\{ 
    \begin{array}{ccc} 
      Q & & x \nameeq \quotep{P} \\
      \dropn{x} & & otherwise \\
    \end{array}
  \right.
\end{mathpar}
 

where

\begin{eqnarray}
  (x)\id{\{} \lpquote Q \rpquote / \lpquote P \rpquote \id{\}}            = 
  \left\{ 
    \begin{array}{ccc}
      \lpquote Q \rpquote & & x \nameeq \lpquote P \rpquote \\
      x & & otherwise \\
    \end{array}
  \right. \nonumber
\end{eqnarray}

and $z$ is chosen distinct from $\quotep{P}$, $\quotep{Q}$, the free
names in $Q$, and all the names in $R$. Our $\alpha$-equivalence will
be built in the standard way from this substitution.

\begin{remark}\label{rem:no_self_referential_names}
  One consequence of these definitions is that $\forall P. \quotep{P}
  \not\in \freenames{P}$.
\end{remark}

\subsection{ Dynamic quote: an example }

Anticipating something of what's to come, consider applying the
substitution, $\widehat{\id{\{}u / z \id{\}}}$, to the following pair
of processes, $\lift{w}{y!(z)}$ and $w[ \lpquote y!(z) \rpquote ]$.

\begin{eqnarray}
	\lift{w}{y!(z)}\widehat{\id{\{}u / z \id{\}}}
		& = &
		\lift{w}{y!(u)} \nonumber\\
	w[ \lpquote y!(z) \rpquote ] \widehat{ \id{\{}u / z \id{\}} }
		& = &
		w[ \lpquote y!(z) \rpquote ] \nonumber
\end{eqnarray}

Because the body of the process between quotes is impervious to
substitution, we get radically different answers. In fact, by
examining the first process in an input context,
e.g. $x?(z).\lift{w}{y!(z)}$, we see that the process under the lift
operator may be shaped by prefixed inputs binding a name inside it. In
this sense, the lift operator will be seen as a way to dynamically
construct processes before reifying them as names.

Finally equipped with these standard features we can present the
dynamics of the calculus.

\subsubsection{Operational semantics} 

Finally, we introduce the computational dynamics. What marks these
algebras as distinct from other more traditionally studied algebraic
structures, e.g. vector spaces or polynomial rings, is the manner in
which dynamics is captured. In traditional structures, dynamics is typically
expressed through morphisms between such structures, as in linear maps
between vector spaces or morphisms between rings. In algebras
associated with the semantics of computation, the dynamics is
expressed as part of the algebraic structure itself, through a
reduction reduction relation typically denoted by $\red$. Below, we
give a recursive presentation of this relation for the calculus used
in the encoding.

$\red \subseteq \pi \times \pi$
$\red : \pi \to \mathcal{P}(\pi)$

\begin{mathpar}
  \inferrule* [lab=Comm] { \textsf{match}( x_{src}, x_{trgt} ) } { x_{trgt}?(y)P \; | \; x_{src}!\langle {Q} \rangle \red P\{\quotep{Q}/y}\} }
  \and \\
  \inferrule* [lab=Par] {{P} \red {P}'} {{{P} | {Q}} \red {{P}' | {Q}}}
  \and
  \inferrule* [lab=Equiv]{{{P} \scong {P}'} \andalso {{P}' \red {Q}'} \andalso {{Q}' \scong {Q}}}{{P} \red {Q}}
\end{mathpar}

\begin{eqnarray*}
  match_{\equiv} (\quotep{P},\quotep{Q}) & := & P \equiv Q \\
  match_{\dagger}(\quotep{P},\quotep{Q}) & := & \forall R. P|Q \red^{*} R => R \red^{*} 0 \\
  match_{K}(\quotep{P},\quotep{Q}) & := & K \mbox{ for some context } K
\end{eqnarray*}

$u?(x)P | u!\langle Q \rangle \red P\{\quotep{Q}/x\}$

%We write $\wred$ for $\red^*$, and $P\red$ if $\exists Q $ such that $ P \red Q$.
We write $P\red$ if $\exists Q $ such that $ P \red Q$ and $P\not\red$, otherwise.

\section{Replication}

As mentioned before, it is known that replication (and hence
recursion) can be implemented in a higher-order process algebra
\cite{SangiorgiWalker}. As our first example of calculation with the
machinery thus far presented we give the construction explicitly in
the {\rhoc}.

\begin{eqnarray}
	D_{x} & := & \prefix{x}{y}{(\binpar{\outputp{x}{y}}{@{y}})} \nonumber\\
	\bangp_{x}{P} & := & \binpar{{x}!\langle{\binpar{D_{x}}{P}}\rangle}{D_{x}} \nonumber
\end{eqnarray}

\begin{eqnarray}
	\bangp_{x}{P} & & \nonumber\\
	=
	& {x}!\langle{(\prefix{x}{y}{(\outputp{x}{y} | @{y})) | P}}\rangle 
	      | \prefix{x}{y}{(\outputp{x}{y} | @{y})} & \nonumber\\
	\red
	& (\outputp{x}{y} | @{y})\substn{\quotep{(\prefix{x}{y}{(@{y} | \outputp{x}{y})) | P}}}{y} & \nonumber\\
	=
	& \outputp{x}{\quotep{(\prefix{x}{y}{(\outputp{x}{y} | @{y})) | P}}}
	  | {(\prefix{x}{y}{(\outputp{x}{y} | @{y})) | P}} & \nonumber\\
	\red
	& \ldots & \nonumber\\
	\red^*
	& P | P | \ldots & \nonumber
\end{eqnarray}

Of course, this encoding, as an implementation, runs away, unfolding
$\bangp{P}$ eagerly. A lazier and more implementable replication
operator, restricted to input-guarded processes, may be obtained as follows.

\begin{eqnarray}
\bangp{\prefix{u}{v}{P}} 
	:= 
	\binpar{\lift{x}{\prefix{u}{v}{(\binpar{D(x)}{P})}}}{D(x)} \nonumber
\end{eqnarray}

\begin{remark}
  Note that the lazier definition still does not deal with summation
  or mixed summation (i.e. sums over input and output). The reader is
  invited to construct definitions of replication that deal with these
  features. 

  Further, the definitions are parameterized in a name, $x$. Can you,
  gentle reader, make a definition that eliminates this parameter and
  guarantees no accidental interaction between the replication
  machinery and the process being replicated -- i.e. no accidental
  sharing of names used by the process to get its work done and the
  name(s) used by the replication to effect copying. This latter
  revision of the definition of replication is crucial to obtaining
  the expected identity $!!P \sim !P$.
\end{remark}

\begin{remark}\label{rem:paradoxical_combinator}
  The reader familiar with the lambda calculus will have noticed the
  similarity between $D$ and the paradoxical combinator.

  [Ed. note: the existence of this seems to suggest we have to be more
  restrictive on the set of processes and names we admit if we are to
  support no-cloning.]
\end{remark}

\subsubsection{Bisimulation}

The computational dynamics gives rise to another kind of equivalence,
the equivalence of computational behavior. As previously mentioned
this is typically captured \emph{via} some form of bisimulation.

% The notion we use in this paper is weak barbed bisimulation
% \cite{milner91polyadicpi}.

The notion we use in this paper is derived from weak barbed
bisimulation \cite{milner91polyadicpi}. 

\begin{definition}
An \emph{observation relation}, $\downarrow_{\mathcal N}$, over a set
of names, $\mathcal N$, is the smallest relation satisfying the rules
below.

\infrule[Out-barb]{y \in {\mathcal N}, \; x \nameeq y}
		  {\outputp{x}{v} \downarrow_{\mathcal N} x}
\infrule[Par-barb]{\mbox{$P\downarrow_{\mathcal N} x$ or $Q\downarrow_{\mathcal N} x$}}
		  {\binpar{P}{Q} \downarrow_{\mathcal N} x}

We write $P \Downarrow_{\mathcal N} x$ if there is $Q$ such that 
$P \wred Q$ and $Q \downarrow_{\mathcal N} x$.
\end{definition}

\begin{definition}
%\label{def.bbisim}
An  ${\mathcal N}$-\emph{barbed bisimulation} over a set of names, ${\mathcal N}$, is a symmetric binary relation 
${\mathcal S}_{\mathcal N}$ between agents such that $P\rel{S}_{\mathcal N}Q$ implies:
\begin{enumerate}
\item If $P \red P'$ then $Q \wred Q'$ and $P'\rel{S}_{\mathcal N} Q'$.
\item If $P\downarrow_{\mathcal N} x$, then $Q\Downarrow_{\mathcal N} x$.
\end{enumerate}
$P$ is ${\mathcal N}$-barbed bisimilar to $Q$, written
$P \wbbisim_{\mathcal N} Q$, if $P \rel{S}_{\mathcal N} Q$ for some ${\mathcal N}$-barbed bisimulation ${\mathcal S}_{\mathcal N}$.
\end{definition}

$\mathcal{R} \subseteq \pi \times \pi$

$P \mathcal{R} Q => \forall P'. P \red P' \Rightarrow \exists Q'. Q \red Q', P' \mathcal{R} Q'$

$P \vdash x \Rightarrow Q \vdash x$

\begin{mathpar}
  \inferrule*[lab=Out-barb]{x \nameeq y}{{y}!\langle{Q}\rangle \vdash x}
  \and
  \inferrule*[lab=Par-barb]{\mbox{$P\vdash x$ or $Q\vdash x$}}{\binpar{P}{Q} \vdash x}
\end{mathpar}

\subsubsection{Contexts}

One of the principle advantages of computational calculi like the
$\pi$-calculus is a well-defined notion of context,
contextual-equivalence and a correlation between
contextual-equivalence and notions of bisimulation. The notion of
context allows the decomposition of a process into (sub-)process and
its syntactic environment, its context. Thus, a context may be
thought of as a process with a ``hole'' (written $\Box$) in it. The
application of a context $M$ to a process $P$, written $M[P]$, is
tantamount to filling the hole in $M$ with $P$. In this paper we do
not need the full weight of this theory, but do make use of the notion
of context in the proof the main theorem. 

\begin{mathpar}
  \inferrule* [lab=summation] {} {{M_{M},M_{N}} \bc \Box \;|\; x.M_{A} \;|\; M_{M}+M_{N}}
  \and
  \inferrule* [lab=agent] {} {{M_{A}} \bc (\vec{x})M_{P} \;| \; \clift{P_0,\ldots,M_{P},\ldots,P_N}}
  \and \\
  \inferrule* [lab=process] {} {{M_{P}} \bc M_{N} \;| \;P|M_{P} }
\end{mathpar} 

\begin{mathpar}
  \inferrule* [lab=sychronization] {} {M_{N} \bc \Box \;|\; x?M_{F} \;|\; x!M_{C}}
  \and
  \inferrule* [lab=abstraction] {} {{M_{F}} \bc (x)M_{P} }
  \and
  \inferrule* [lab=concretion] {} {{M_{C}} \bc \langle M_{P} \rangle }
  \and \\
  \inferrule* [lab=process] {} {{M_{P}} \bc M_{N} \;| \;P|M_{P} }
\end{mathpar}

\begin{definition}[contextual application] Given a context $M$, and
  process $P$, we define the \emph{contextual application}, $M[P] :=
  M\{P/\Box\}$. That is, the contextual application of M to P is the
  substitution of $P$ for $\Box$ in $M$.
\end{definition}

$\meaningof{-} : L \to \mathcal{P}(\pi)$

\begin{mathpar}
  \inferrule* [lab=collection] {} {\meaningof{true} = \pi, \and \meaningof{~E} = \pi \setminus \meaningof{E}, \and \meaningof{E_{1} \& E_{2}} = \meaningof{E_{1}} \cap \meaningof{E_{2}}}
\end{mathpar}

\begin{mathpar}
  \inferrule* [lab=structure] {} {\meaningof{0} = \{ P \in \pi | P \equiv 0 \}, \and \\ \meaningof{E_1 | E_2} = \{ P \in \pi | P \equiv P_{1} | P_{2}, P_{1} \in \meaningof{E_{1}}, P_{2} \in \meaningof{E_2}\} }
\end{mathpar}

\begin{mathpar}
 \inferrule* [lab=behavior] {} {\meaningof{\langle a?b \rangle E} = \{ P \in \pi | P \equiv Q | u?(y)P', \\ \and \\\\ \and \\ \;\;\; u \in \meaningof{a}, \forall z.P'\{z/y\} \in \meaningof{E\{z/b\}}\}, \and \\ \meaningof{a!E} = \{ P \in \pi | P \equiv Q | x!\langle P' \rangle, x \in \meaningof{a} P' \in \meaningof{E}\} }
\end{mathpar}

\begin{mathpar}
 \inferrule* [lab=nominal] {} {\meaningof{\quotep{E}} = \{ \quotep{P} \in \quotep{\pi} | P \in \meaningof{E} \}, \and \meaningof{\quotep{P}} = \{ \quotep{Q} \in \quotep{\pi} | P \equiv Q \} \and \\ \meaningof{@\quotep{E}} = \{ P \in \pi | P \equiv @x, x \in \meaningof{E} \}}
\end{mathpar}

\begin{eqnarray*}
  \\
  \meaningof{-} : TS \to ST
\end{eqnarray*}

\begin{eqnarray*}
  \\
  L : TS \to ST
\end{eqnarray*}

\begin{eqnarray*}
  \\
  P \models E \iff P \in \meaningof{E}
\end{eqnarray*}

\begin{eqnarray*}
  P \approx_{L} Q \iff \forall E \in L. P \models E \iff Q \models E
\end{eqnarray*}

\begin{eqnarray*}
  P \approx_{K} Q
\end{eqnarray*}

\begin{eqnarray*}
  P \approx Q
\end{eqnarray*}

$\approx_{K} = \approx = \approx_{L}$

\subsubsection{Contextual duality}

Note that contexts extend the quotation operation to a family of
operations from processes to names. Given a context, $M$, we can
define a \emph{nominal context}, $\quotep{M}$ by $\quotep{M}[P] :=
\quotep{M[P]}$. To foreshadow what is to come we observe that these
operations enjoy a duality with processes very much like the duality
between vectors and maps from vectors to scalars.

Further, because the calculus is essentially higher-order, we have a
correspondence between contexts and processes. More specifically,
given a name $x$ and a context $M$ we can construct $M^{*}_{x}$ such
that 

\begin{mathpar}
  M^{*}_{x} | \lift{x}{P} \red M[P]
\end{mathpar}

namely,

\begin{mathpar}
  M^{*}_{x} := x?(u).M[\dropn{u}]
\end{mathpar}

The dependence of $M^{*}_{x}$ on a name makes it an abstraction, 

\begin{mathpar}
  M^{*} := (x)x?(u).M[\dropn{u}]
\end{mathpar}

\subsection{Additional notation}

It will sometimes be convenient to denote the process a name
quotes. We already have the notation $x = \quotep{P}$, but it will be
convenient to introduce an alternate notation, $\procn{x}$, when we
want to emphasize the connection to the use of the name. Note that, by
virtue of name equivalence, $\quotep{\procn{x}} \nameeq x$; so, the
notation is consistent with previous definitions.

Further, because names have structure it is possible to effect
substitutions on the basis of that structure. This means we need to
upgrade our notation for substitutions, which we accomplish by
adapting comprehension notation. Thus,

\begin{mathpar}
  P\{ y / x : x \in S \}
\end{mathpar}

is interpreted to mean the process derived from P by replacing (in a
capture-avoiding manner) each occurrence of $x$ in $S$ by $y$. For example,

\begin{mathpar}
  P\{ \quotep{\procn{x}|\procn{x}} / x : x \in \freenames{P} \}
\end{mathpar}

will replace each (occurrence) of a free name $x$ in $P$ by
$\quotep{\procn{x}|\procn{x}}$.

Also, we will avail ourselves of the notation $x^{L}$ and $x^{R}$ to
denote injections of a name into disjoint copies of the name
space. There are numerous ways to accomplish this. One example can be
found in \cite{MeredithR05}. This notation overloads to vectors of
names: $\vec{x}^{\pi} := (x_{i}^{\pi} \; : \; 0 \leq i < |\vec{x}| )$ where $\pi \in \{L,R\}$.

We also use $P^{\Box} := P|\Box$.

In \cite{MeredithR05} an interpretation of the new operator is
given. It turns out that there are several possible interpretations
all enjoying the requisite algebraic properties of the operator (see
\cite{milner91polyadicpi}). We will therefore make liberal use of
$(\nu\; \vec{x})P$.

% subsection the_syntax_and_semantics_of_the_notation_system (end)   

\input{qm2pi.qmops} 

\input{qm2pi.sterngerlach} 

\input{qm2pi.metric} 

% section concurrent_process_calculi (end)

%\input{qm2pi.proofsketch}

% section proof sketch (end)

%\input{qm2pi.slviaknots} 

% section spatial logic via knots (end)

\input{qm2pi.conclusion}

% section conclusion (end)

%\input{qm2pi.dtcodes} 

% section wiring algorithm (end)

\input{qm2pi.ack} 

% section acknowledgments (end)

\newpage


\bibliographystyle{plain}   
\bibliography{../../biblios/main.bib}

\input{qm2pi.rhodetails}

\end{document}

 

% section concurrent_process_calculi (end)

%\documentclass[12pt]{llncs}
%\documentclass{jktr}

\usepackage[pdftex]{hyperref}                   
\usepackage {listings}
\usepackage {mathpartir}
\usepackage{bcprules}
%\usepackage{listings}
                       
\usepackage{graphicx} 
%\usepackage[margins=2.5cm,nohead,nofoot]{geometry}
%\usepackage{geometry}
\usepackage{amsfonts}
\usepackage{amstext}
\usepackage{latexsym}
\usepackage{amssymb}
\usepackage{color}


%\include{myPreamble}
\include{qm2pi.local} 

%\ifpdf
%\usepackage[pdftex]{graphicx}
%\else
%\usepackage{graphicx}
%\fi

 % \ifpdf
%  \usepackage{pdfsync}
%  \if


%\title{Brief Article}
%\author{David F. Snyder}
%\author{L.G. Meredith}

%\address{Dept. of Math., Texas State University--San Marcos, San Marcos, TX 78666}
       
\pagestyle{empty}


\begin{document}

\lstset{language=[Objective]Caml,frame=shadowbox}

\input{qm2pi.front}

% section front matter (end)

\input{qm2pi.intro} 
 
% section introduction (end)

% \input{qm2pi.knotations} 

% section notation (end)

\input{qm2pi.process.calculi} 

% section concurrent_process_calculi_and_spatial_logics_ (end)
    
%\input{qm2pi.knots2pi} 

%\input{qm2pi.trefoil} 

%\input{qm2pi.mainthm} 

% subsection basic_interpretation (end)

%\input{qm2pi.rho.presentation} 
\subsection{The syntax and semantics of the notation system}\label{sub:the_syntax_and_semantics_of_the_notation_system} % (fold)

We now summarize a technical presentation of the calculus that
embodies our theory of dynamics. The typical presentation of such a
calculus follows the style of giving generators and relations on
them. The grammar, below, describing term constructors, freely
generates the set of processes, $\Proc$. This set is then quotiented
by a relation known as structural congruence and it is over this set
that the notion of dynamics is expressed. This presentation is
essentially that of \cite{MeredithR05} with the addition of
polyadicity and summation. For readability we have relegated some of
the technical subtleties to an appendix.

\subsubsection{Process grammar}\label{subsub:process_grammar}

\begin{mathpar}
  \inferrule* [lab=synchronization] {} {{M} \bc \pzero \;|\; x?F \;|\; x!C }
  \and
  \inferrule* [lab=abstraction] {} {{F} \bc (x)P}
  \and
  \inferrule* [lab=concretion] {} {{C} \bc \langle Q \rangle}
  \and
  \inferrule* [lab=process] {} {{P,Q} \bc M \;| \;P|Q \;|\; @{x}}
  \and
  \inferrule* [lab=name] {} {{x} \bc \quotep{P}}
\end{mathpar} 

Note that $\vec{x}$ (resp. $\vec{P}$) denotes a vector of names
(resp. processes) of length $|\vec{x}|$ (resp. $|\vec{P}|$). We adopt
the following useful abbreviations.

\begin{mathpar}
   x?(\vec{y}).P := x.(\vec{y})P \and  x\clift{\vec{P}} := x.\clift{\vec{P}}
   \and x!(y) := \lift{x}{\dropn{y}}
   \and \Pi_{i=0}^{n-1}P_i := P_0 | \ldots | P_{n-1}
\end{mathpar}

\subsubsection{Structural congruence}

\paragraph{Free and bound names and alpha-equivalence.} At the
core of structural equivalence is alpha-equivalence which identifies
process that are the same up to a change of variable. Formally, we
recognize the distinction between free and bound names. The free names
of a process, $\freenames{P}$, may be calculated recursively as
follows:

\begin{mathpar}
\freenames{\pzero} := \emptyset
  \and \\
  \freenames{x?(y).P} := \{ x \} \cup (\freenames{P} \setminus \{ y \})
  \and 
  \freenames{x!\langle P \rangle} := \{ x \} \cup \{ P \} 
  \and \\
  \freenames{P|Q} := \freenames{P} \cup \freenames{Q}
  \and \\
  \freenames{@{x}} := \{ x \}
\end{mathpar}

$\pi$
$\quotep{\pi}$

$\freenames{-} : \pi \to \mathcal{P}(\quotep{\pi})$

\begin{eqnarray*}
  \freenames{\pzero} & := & \emptyset \\
  \freenames{x?(y).P} & := & \{ x \} \cup (\freenames{P} \setminus \{ y \}) \\
  \freenames{x!\langle P \rangle} & := & \{ x \} \cup \{ P \} \\
  \freenames{P|Q} & := & \freenames{P} \cup \freenames{Q} \\
  \freenames{\dropn{x}} & := & \{ x \}
\end{eqnarray*}

The bound names of a process, $\boundnames{P}$, are those names occurring in $P$
that are not free. For example, in $x?(y).0$, the name $x$ is free, while $y$ is bound.

\begin{mathpar}
  \inferrule* [lab=monoidal-laws] {} { P|Q \equiv Q|P \and P|0 \equiv P \and P|(Q|R) \equiv (P|Q)|R }
\end{mathpar}

\begin{mathpar}
  \inferrule* [lab=alpha-equivalence] {} { (x)P \equiv (y)P\{y/x\} \and y \not\in \freenames{P} }
\end{mathpar}

\begin{definition}
Then two processes, $P,Q$, are alpha-equivalent if $P = Q\{\vec{y}/\vec{x}\}$ for
some $\vec{x} \in \boundnames{Q},\vec{y} \in \boundnames{P}$, where $Q\{\vec{y}/\vec{x}\}$
denotes the capture-avoiding substitution of $\vec{y}$ for $\vec{x}$ in $Q$.
\end{definition}

\begin{definition}
  The {\em structural congruence} \cite{SangiorgiWalker} , $\equiv$,
  between processes is the least congruence containing
  alpha-equivalence, satisfying the abelian monoid laws
  (associativity, commutativity and $\pzero$ as identity) for parallel
  composition $|$ and for summation $+$.
\end{definition}

\subsection{Name equivalence}

We take name equivalence, written $\nameeq$, to be the smallest
equivalence relation generated by the following rules.

\begin{mathpar}
\inferrule*[lab=Quote-drop]
{ }
{ \quotep{@{x}} \nameeq x }

\inferrule*[lab=Struct-equiv]
{ P \scong Q }
{ \quotep{P} \nameeq \quotep{Q} }
\end{mathpar}

The astute reader will have noticed that the mutual recursion of names
and processes imposes a mutual recursion on alpha-equivalence and
structural equivalence via name-equivalence. Fortunately, all of this
works out pleasantly and we may calculate in the natural way, free of
concern. The reader interested in the details is referred to the
appendix \ref{appendix:rho_details}.

\subsection{Substitution}

We use $\Proc$ for the set of processes, $\QProc$ for the set of
names, and $\id{\{}\vec{y} / \vec{x} \id{\}}$ to denote partial maps,
$s : \QProc \rightarrow \QProc$. A map, $s$ lifts, uniquely, to a map
on process terms, $\widehat{s} : \Proc \rightarrow \Proc$ by the
following equations.

\begin{mathpar}
  (0) \psubstp{Q}{P} := 0 \\
  (R \juxtap S) \psubstp{Q}{P}
  :=    
  (R)\psubstp{Q}{P} \juxtap (S) \psubstp{Q}{P} \\
  (x?(y).R) \psubstp{Q}{P}    
  :=    
  (x)\substp{Q}{P} (z)\concat( (R \psubstn{z}{y}) \psubstp{Q}{P} ) \\
  (\lift{x}{R}) \psubstp{Q}{P}  
  :=
  \lift{(x)\substp{Q}{P}}{ R \psubstp{Q}{P} } \\
%   (\dropn{x})  \psubstp{Q}{P}       
%   := 
%   \left\{ 
%     \begin{array}{ccc} 
%       \dropn{\quotep{Q}} & & x \nameeq \quotep{P} \\
%       \dropn{x} & & otherwise \\
%     \end{array}
%   \right. 
  (\dropn{x})  \psubstp{Q}{P}       
  := 
  \left\{ 
    \begin{array}{ccc} 
      Q & & x \nameeq \quotep{P} \\
      \dropn{x} & & otherwise \\
    \end{array}
  \right.
\end{mathpar}
 

where

\begin{eqnarray}
  (x)\id{\{} \lpquote Q \rpquote / \lpquote P \rpquote \id{\}}            = 
  \left\{ 
    \begin{array}{ccc}
      \lpquote Q \rpquote & & x \nameeq \lpquote P \rpquote \\
      x & & otherwise \\
    \end{array}
  \right. \nonumber
\end{eqnarray}

and $z$ is chosen distinct from $\quotep{P}$, $\quotep{Q}$, the free
names in $Q$, and all the names in $R$. Our $\alpha$-equivalence will
be built in the standard way from this substitution.

\begin{remark}\label{rem:no_self_referential_names}
  One consequence of these definitions is that $\forall P. \quotep{P}
  \not\in \freenames{P}$.
\end{remark}

\subsection{ Dynamic quote: an example }

Anticipating something of what's to come, consider applying the
substitution, $\widehat{\id{\{}u / z \id{\}}}$, to the following pair
of processes, $\lift{w}{y!(z)}$ and $w[ \lpquote y!(z) \rpquote ]$.

\begin{eqnarray}
	\lift{w}{y!(z)}\widehat{\id{\{}u / z \id{\}}}
		& = &
		\lift{w}{y!(u)} \nonumber\\
	w[ \lpquote y!(z) \rpquote ] \widehat{ \id{\{}u / z \id{\}} }
		& = &
		w[ \lpquote y!(z) \rpquote ] \nonumber
\end{eqnarray}

Because the body of the process between quotes is impervious to
substitution, we get radically different answers. In fact, by
examining the first process in an input context,
e.g. $x?(z).\lift{w}{y!(z)}$, we see that the process under the lift
operator may be shaped by prefixed inputs binding a name inside it. In
this sense, the lift operator will be seen as a way to dynamically
construct processes before reifying them as names.

Finally equipped with these standard features we can present the
dynamics of the calculus.

\subsubsection{Operational semantics} 

Finally, we introduce the computational dynamics. What marks these
algebras as distinct from other more traditionally studied algebraic
structures, e.g. vector spaces or polynomial rings, is the manner in
which dynamics is captured. In traditional structures, dynamics is typically
expressed through morphisms between such structures, as in linear maps
between vector spaces or morphisms between rings. In algebras
associated with the semantics of computation, the dynamics is
expressed as part of the algebraic structure itself, through a
reduction reduction relation typically denoted by $\red$. Below, we
give a recursive presentation of this relation for the calculus used
in the encoding.

$\red \subseteq \pi \times \pi$
$\red : \pi \to \mathcal{P}(\pi)$

\begin{mathpar}
  \inferrule* [lab=Comm] { \textsf{match}( x_{src}, x_{trgt} ) } { x_{trgt}?(y)P \; | \; x_{src}!\langle {Q} \rangle \red P\{\quotep{Q}/y}\} }
  \and \\
  \inferrule* [lab=Par] {{P} \red {P}'} {{{P} | {Q}} \red {{P}' | {Q}}}
  \and
  \inferrule* [lab=Equiv]{{{P} \scong {P}'} \andalso {{P}' \red {Q}'} \andalso {{Q}' \scong {Q}}}{{P} \red {Q}}
\end{mathpar}

\begin{eqnarray*}
  match_{\equiv} (\quotep{P},\quotep{Q}) & := & P \equiv Q \\
  match_{\dagger}(\quotep{P},\quotep{Q}) & := & \forall R. P|Q \red^{*} R => R \red^{*} 0 \\
  match_{K}(\quotep{P},\quotep{Q}) & := & K \mbox{ for some context } K
\end{eqnarray*}

$u?(x)P | u!\langle Q \rangle \red P\{\quotep{Q}/x\}$

%We write $\wred$ for $\red^*$, and $P\red$ if $\exists Q $ such that $ P \red Q$.
We write $P\red$ if $\exists Q $ such that $ P \red Q$ and $P\not\red$, otherwise.

\section{Replication}

As mentioned before, it is known that replication (and hence
recursion) can be implemented in a higher-order process algebra
\cite{SangiorgiWalker}. As our first example of calculation with the
machinery thus far presented we give the construction explicitly in
the {\rhoc}.

\begin{eqnarray}
	D_{x} & := & \prefix{x}{y}{(\binpar{\outputp{x}{y}}{@{y}})} \nonumber\\
	\bangp_{x}{P} & := & \binpar{{x}!\langle{\binpar{D_{x}}{P}}\rangle}{D_{x}} \nonumber
\end{eqnarray}

\begin{eqnarray}
	\bangp_{x}{P} & & \nonumber\\
	=
	& {x}!\langle{(\prefix{x}{y}{(\outputp{x}{y} | @{y})) | P}}\rangle 
	      | \prefix{x}{y}{(\outputp{x}{y} | @{y})} & \nonumber\\
	\red
	& (\outputp{x}{y} | @{y})\substn{\quotep{(\prefix{x}{y}{(@{y} | \outputp{x}{y})) | P}}}{y} & \nonumber\\
	=
	& \outputp{x}{\quotep{(\prefix{x}{y}{(\outputp{x}{y} | @{y})) | P}}}
	  | {(\prefix{x}{y}{(\outputp{x}{y} | @{y})) | P}} & \nonumber\\
	\red
	& \ldots & \nonumber\\
	\red^*
	& P | P | \ldots & \nonumber
\end{eqnarray}

Of course, this encoding, as an implementation, runs away, unfolding
$\bangp{P}$ eagerly. A lazier and more implementable replication
operator, restricted to input-guarded processes, may be obtained as follows.

\begin{eqnarray}
\bangp{\prefix{u}{v}{P}} 
	:= 
	\binpar{\lift{x}{\prefix{u}{v}{(\binpar{D(x)}{P})}}}{D(x)} \nonumber
\end{eqnarray}

\begin{remark}
  Note that the lazier definition still does not deal with summation
  or mixed summation (i.e. sums over input and output). The reader is
  invited to construct definitions of replication that deal with these
  features. 

  Further, the definitions are parameterized in a name, $x$. Can you,
  gentle reader, make a definition that eliminates this parameter and
  guarantees no accidental interaction between the replication
  machinery and the process being replicated -- i.e. no accidental
  sharing of names used by the process to get its work done and the
  name(s) used by the replication to effect copying. This latter
  revision of the definition of replication is crucial to obtaining
  the expected identity $!!P \sim !P$.
\end{remark}

\begin{remark}\label{rem:paradoxical_combinator}
  The reader familiar with the lambda calculus will have noticed the
  similarity between $D$ and the paradoxical combinator.

  [Ed. note: the existence of this seems to suggest we have to be more
  restrictive on the set of processes and names we admit if we are to
  support no-cloning.]
\end{remark}

\subsubsection{Bisimulation}

The computational dynamics gives rise to another kind of equivalence,
the equivalence of computational behavior. As previously mentioned
this is typically captured \emph{via} some form of bisimulation.

% The notion we use in this paper is weak barbed bisimulation
% \cite{milner91polyadicpi}.

The notion we use in this paper is derived from weak barbed
bisimulation \cite{milner91polyadicpi}. 

\begin{definition}
An \emph{observation relation}, $\downarrow_{\mathcal N}$, over a set
of names, $\mathcal N$, is the smallest relation satisfying the rules
below.

\infrule[Out-barb]{y \in {\mathcal N}, \; x \nameeq y}
		  {\outputp{x}{v} \downarrow_{\mathcal N} x}
\infrule[Par-barb]{\mbox{$P\downarrow_{\mathcal N} x$ or $Q\downarrow_{\mathcal N} x$}}
		  {\binpar{P}{Q} \downarrow_{\mathcal N} x}

We write $P \Downarrow_{\mathcal N} x$ if there is $Q$ such that 
$P \wred Q$ and $Q \downarrow_{\mathcal N} x$.
\end{definition}

\begin{definition}
%\label{def.bbisim}
An  ${\mathcal N}$-\emph{barbed bisimulation} over a set of names, ${\mathcal N}$, is a symmetric binary relation 
${\mathcal S}_{\mathcal N}$ between agents such that $P\rel{S}_{\mathcal N}Q$ implies:
\begin{enumerate}
\item If $P \red P'$ then $Q \wred Q'$ and $P'\rel{S}_{\mathcal N} Q'$.
\item If $P\downarrow_{\mathcal N} x$, then $Q\Downarrow_{\mathcal N} x$.
\end{enumerate}
$P$ is ${\mathcal N}$-barbed bisimilar to $Q$, written
$P \wbbisim_{\mathcal N} Q$, if $P \rel{S}_{\mathcal N} Q$ for some ${\mathcal N}$-barbed bisimulation ${\mathcal S}_{\mathcal N}$.
\end{definition}

$\mathcal{R} \subseteq \pi \times \pi$

$P \mathcal{R} Q => \forall P'. P \red P' \Rightarrow \exists Q'. Q \red Q', P' \mathcal{R} Q'$

$P \vdash x \Rightarrow Q \vdash x$

\begin{mathpar}
  \inferrule*[lab=Out-barb]{x \nameeq y}{{y}!\langle{Q}\rangle \vdash x}
  \and
  \inferrule*[lab=Par-barb]{\mbox{$P\vdash x$ or $Q\vdash x$}}{\binpar{P}{Q} \vdash x}
\end{mathpar}

\subsubsection{Contexts}

One of the principle advantages of computational calculi like the
$\pi$-calculus is a well-defined notion of context,
contextual-equivalence and a correlation between
contextual-equivalence and notions of bisimulation. The notion of
context allows the decomposition of a process into (sub-)process and
its syntactic environment, its context. Thus, a context may be
thought of as a process with a ``hole'' (written $\Box$) in it. The
application of a context $M$ to a process $P$, written $M[P]$, is
tantamount to filling the hole in $M$ with $P$. In this paper we do
not need the full weight of this theory, but do make use of the notion
of context in the proof the main theorem. 

\begin{mathpar}
  \inferrule* [lab=summation] {} {{M_{M},M_{N}} \bc \Box \;|\; x.M_{A} \;|\; M_{M}+M_{N}}
  \and
  \inferrule* [lab=agent] {} {{M_{A}} \bc (\vec{x})M_{P} \;| \; \clift{P_0,\ldots,M_{P},\ldots,P_N}}
  \and \\
  \inferrule* [lab=process] {} {{M_{P}} \bc M_{N} \;| \;P|M_{P} }
\end{mathpar} 

\begin{mathpar}
  \inferrule* [lab=sychronization] {} {M_{N} \bc \Box \;|\; x?M_{F} \;|\; x!M_{C}}
  \and
  \inferrule* [lab=abstraction] {} {{M_{F}} \bc (x)M_{P} }
  \and
  \inferrule* [lab=concretion] {} {{M_{C}} \bc \langle M_{P} \rangle }
  \and \\
  \inferrule* [lab=process] {} {{M_{P}} \bc M_{N} \;| \;P|M_{P} }
\end{mathpar}

\begin{definition}[contextual application] Given a context $M$, and
  process $P$, we define the \emph{contextual application}, $M[P] :=
  M\{P/\Box\}$. That is, the contextual application of M to P is the
  substitution of $P$ for $\Box$ in $M$.
\end{definition}

$\meaningof{-} : L \to \mathcal{P}(\pi)$

\begin{mathpar}
  \inferrule* [lab=collection] {} {\meaningof{true} = \pi, \and \meaningof{~E} = \pi \setminus \meaningof{E}, \and \meaningof{E_{1} \& E_{2}} = \meaningof{E_{1}} \cap \meaningof{E_{2}}}
\end{mathpar}

\begin{mathpar}
  \inferrule* [lab=structure] {} {\meaningof{0} = \{ P \in \pi | P \equiv 0 \}, \and \\ \meaningof{E_1 | E_2} = \{ P \in \pi | P \equiv P_{1} | P_{2}, P_{1} \in \meaningof{E_{1}}, P_{2} \in \meaningof{E_2}\} }
\end{mathpar}

\begin{mathpar}
 \inferrule* [lab=behavior] {} {\meaningof{\langle a?b \rangle E} = \{ P \in \pi | P \equiv Q | u?(y)P', \\ \and \\\\ \and \\ \;\;\; u \in \meaningof{a}, \forall z.P'\{z/y\} \in \meaningof{E\{z/b\}}\}, \and \\ \meaningof{a!E} = \{ P \in \pi | P \equiv Q | x!\langle P' \rangle, x \in \meaningof{a} P' \in \meaningof{E}\} }
\end{mathpar}

\begin{mathpar}
 \inferrule* [lab=nominal] {} {\meaningof{\quotep{E}} = \{ \quotep{P} \in \quotep{\pi} | P \in \meaningof{E} \}, \and \meaningof{\quotep{P}} = \{ \quotep{Q} \in \quotep{\pi} | P \equiv Q \} \and \\ \meaningof{@\quotep{E}} = \{ P \in \pi | P \equiv @x, x \in \meaningof{E} \}}
\end{mathpar}

\begin{eqnarray*}
  \\
  \meaningof{-} : TS \to ST
\end{eqnarray*}

\begin{eqnarray*}
  \\
  L : TS \to ST
\end{eqnarray*}

\begin{eqnarray*}
  \\
  P \models E \iff P \in \meaningof{E}
\end{eqnarray*}

\begin{eqnarray*}
  P \approx_{L} Q \iff \forall E \in L. P \models E \iff Q \models E
\end{eqnarray*}

\begin{eqnarray*}
  P \approx_{K} Q
\end{eqnarray*}

\begin{eqnarray*}
  P \approx Q
\end{eqnarray*}

$\approx_{K} = \approx = \approx_{L}$

\subsubsection{Contextual duality}

Note that contexts extend the quotation operation to a family of
operations from processes to names. Given a context, $M$, we can
define a \emph{nominal context}, $\quotep{M}$ by $\quotep{M}[P] :=
\quotep{M[P]}$. To foreshadow what is to come we observe that these
operations enjoy a duality with processes very much like the duality
between vectors and maps from vectors to scalars.

Further, because the calculus is essentially higher-order, we have a
correspondence between contexts and processes. More specifically,
given a name $x$ and a context $M$ we can construct $M^{*}_{x}$ such
that 

\begin{mathpar}
  M^{*}_{x} | \lift{x}{P} \red M[P]
\end{mathpar}

namely,

\begin{mathpar}
  M^{*}_{x} := x?(u).M[\dropn{u}]
\end{mathpar}

The dependence of $M^{*}_{x}$ on a name makes it an abstraction, 

\begin{mathpar}
  M^{*} := (x)x?(u).M[\dropn{u}]
\end{mathpar}

\subsection{Additional notation}

It will sometimes be convenient to denote the process a name
quotes. We already have the notation $x = \quotep{P}$, but it will be
convenient to introduce an alternate notation, $\procn{x}$, when we
want to emphasize the connection to the use of the name. Note that, by
virtue of name equivalence, $\quotep{\procn{x}} \nameeq x$; so, the
notation is consistent with previous definitions.

Further, because names have structure it is possible to effect
substitutions on the basis of that structure. This means we need to
upgrade our notation for substitutions, which we accomplish by
adapting comprehension notation. Thus,

\begin{mathpar}
  P\{ y / x : x \in S \}
\end{mathpar}

is interpreted to mean the process derived from P by replacing (in a
capture-avoiding manner) each occurrence of $x$ in $S$ by $y$. For example,

\begin{mathpar}
  P\{ \quotep{\procn{x}|\procn{x}} / x : x \in \freenames{P} \}
\end{mathpar}

will replace each (occurrence) of a free name $x$ in $P$ by
$\quotep{\procn{x}|\procn{x}}$.

Also, we will avail ourselves of the notation $x^{L}$ and $x^{R}$ to
denote injections of a name into disjoint copies of the name
space. There are numerous ways to accomplish this. One example can be
found in \cite{MeredithR05}. This notation overloads to vectors of
names: $\vec{x}^{\pi} := (x_{i}^{\pi} \; : \; 0 \leq i < |\vec{x}| )$ where $\pi \in \{L,R\}$.

We also use $P^{\Box} := P|\Box$.

In \cite{MeredithR05} an interpretation of the new operator is
given. It turns out that there are several possible interpretations
all enjoying the requisite algebraic properties of the operator (see
\cite{milner91polyadicpi}). We will therefore make liberal use of
$(\nu\; \vec{x})P$.

% subsection the_syntax_and_semantics_of_the_notation_system (end)   

\input{qm2pi.qmops} 

\input{qm2pi.sterngerlach} 

\input{qm2pi.metric} 

% section concurrent_process_calculi (end)

%\input{qm2pi.proofsketch}

% section proof sketch (end)

%\input{qm2pi.slviaknots} 

% section spatial logic via knots (end)

\input{qm2pi.conclusion}

% section conclusion (end)

%\input{qm2pi.dtcodes} 

% section wiring algorithm (end)

\input{qm2pi.ack} 

% section acknowledgments (end)

\newpage


\bibliographystyle{plain}   
\bibliography{../../biblios/main.bib}

\input{qm2pi.rhodetails}

\end{document}



% section proof sketch (end)

%\section{Unlikely characters: spatial logic for
  knots}\label{sub:characteristic_formulae} % (fold)

Associated to the mobile process calculi are a family of logics known
as the Hennessy-Milner logics. These logics typically enjoy a
semantics interpreting formulae as sets of processes that when
factored through the encoding outlined above allows an identification
of classes of knots with logical formulae. In the context of this
encoding the sub-family known as the spatial logics \cite{CairesC03}
\cite{CairesC04} \cite{Caires04} are of particular interest providing
several important features for expressing and reasoning about
properties (i.e. classes) of knots. We hint here at how this may be done.

%\begin{description}
%\item [structural connectives] 
\subsubsection{Structural connectives} The spatial logics enjoy
structural connectives corresponding, at the logical level, to the
parallel composition ($P | Q$) and new name ($(\nu \; x)P$)
connectives for processes. As illustrated in the examples below, these
connectives are extremely expressive given the shape of our encoding.
%\item [decideable satisfaction]

\subsubsection{Decideable satisfaction}
In \cite{Caires04} the satisfaction relation is shown to be decideable
for a rich class of processes. It further turns out that the image of
the our encoding is a proper subset of that class. This result
provides the basis for an algorithm by which to search for knots
enjoying a given property.
%\item [characteristic formulae]

\subsubsection{Characteristic formulae}
In the same paper \cite{Caires04} , Caires presents a means of calculating
characteristic formulae, selecting equivalence classes of processes
up to a pre--specified depth limit on the support set of names. Composed with our
encoding, this characteristic formula can be used to select
characteristic formulae for knots.
%\end{description}

\subsubsection{Spatial logic formulae}

The grammar below (segmented for comprehension) summarizes the syntax
of spatial logic formulae. We employ illustrative examples in the
sequel to provide an intuitive understanding of their meaning
referring the reader to \cite{Caires04} for a more detailed explication
of the semantics.

\begin{mathpar}
  \inferrule* [lab=boolean] {} {{A,B} \bc T \;|\; \neg A \;|\; A \wedge B \;|\; \eta = \eta'}
  \and
  \inferrule* [lab=spatial] {} {|\; \pzero \;|\; A | B \;|\; x \text{\textregistered} A \;|\; \forall x . A \;|\;  H x . A}
  \and
  \inferrule* [lab=behavioral] {} {|\; \alpha . A}
  \and 
  \inferrule* [lab=recursion] {} {|\; X(\vec{u}) \;|\; \mu X(\vec{u}) . A}
  \and
  \inferrule* [lab=action] {} {\alpha \bc \langle x?(\vec{y}) \rangle \;|\; \langle x!(\vec{y}) \rangle \;|\; \langle \tau \rangle}
  \and 
  \inferrule* [lab=name] {} {\eta \bc x \;|\; \tau}
\end{mathpar} 

% subsection characteristic_formulae (end)   	 

\subsection{Example formulae}\label{sub:example_formulae_} % (fold)

\subsubsection{Crossing as formula.}
% 
% \begin{align*}
%   \frac{d}{dx} \sin x &= \cos x 
%   & \frac{d}{dx} e^x &= e^x \\
%   \frac{d}{dx} \cos x &= - \sin x 
%   & \frac{d}{dx} \log x &= \frac{1}{x} \\
% \end{align*} 

\begin{align*}
 \mu C(x_{0},x_{1},y_{0},y_{1},u).&(\langle x_{0}?(z) \rangle(\langle u! \rangle\langle y_{1}!z \rangle C(x_{0},x_{1},y_{0},y_{1},u)) & \\
  & \wedge \langle y_{1}?(z) \rangle (\langle u! \rangle \langle x_{0}!z \rangle C(x_{0},x_{1},y_{0},y_{1},u)) & \\
  & \wedge \langle x_{1}?(z) \rangle (\langle u? \rangle \langle y_{0}!z \rangle C(x_{0},x_{1},y_{0},y_{1},u)) & \\
  & \wedge \langle y_{0}?(z) \rangle (\langle u? \rangle \langle x_{1}!z \rangle C(x_{0},x_{1},y_{0},y_{1},u))) &
\end{align*}

The lexicographical similarity between the shape of this formulae and
the shape of definition of the process representing a crossing reveals
the intuitive meaning of this formulae. It describes the capabilities
of a process that has the right to represent a crossing. For example
it picks out processes that may perform an input on the port $x_0$ in
its initial menu of capabilities. What differentiates the formula
from the process, however, is that the crossing process is the
smallest candidate to satisfy the formula. Infinitely many other
processes -- with internal behavior hidden behind this interface, so
to speak -- also satisfy this formula. Even this simple formula,
then, can be seen to open a new view onto knots, providing a
computational interpretation of \emph{virtual} knots.

Note that this formula is derived by hand. A similar formula can be
derived by employing Caires' calculation of characteristic formula
\cite{Caires04} to the process representing a crossing. In light of
this discussion, we let
$\meaningof{C}_{\phi}(x0,x1,y0,y1,u)$ denote a formula specifying the
dynamics we wish to capture of a crossing. To guarantee we preserve
the shape of the interface and minimal semantics we demand that
$\meaningof{C}_{\phi}(x0,x1,y0,y1,u) \Rightarrow
\textbf{C}(x0,x1,y0,y1,u)$ where $\textbf{C}(x0,x1,y0,y1,u)$ denotes
the formula above.
                            
\subsubsection{Crossing number constraints.}
The moral content of the context lemma (Lemma \ref{context}) is that the notion of
``locality'' in the Reidemeister moves is effectively captured by the
parallel composition operator of the process calculus. This intuition
extends through the logic. Given a formula,
$\meaningof{C}_{\phi}(x0,x1,y0,y1,u)$, we can use the structural
connectives to specify constraints on crossing numbers, such as at
least $n$ crossings, or exactly $n$ crossings.
\begin{mathpar}
  \inferrule* [lab=at-least-n] {} { K^{\geq n}_{\phi}(\vec{xs},\vec{ys}) := \Pi_{i=0}^{n-1} Hu . \meaningof{C}_{\phi}(xs_i,ys_i,u) | T }
  \and 
  \inferrule* [lab=exactly-n] {} { K^{= n}_{\phi}(\vec{xs},\vec{ys}) := \Pi_{i=0}^{n-1} Hu . \meaningof{C}_{\phi}(xs_i,ys_i,u) | \neg (\forall x_0,y_0,x_1,y_1,u . \meaningof{C}_{\phi}(x_0,y_0,x_1,y_1,u) | T) }
\end{mathpar}

To round out this section, recall that the encoding of an $n$-crossing
knot decomposes into a parallel composition of $n$ \emph{copies} of a
crossing process together with a wiring harness. To specify different
knot classes with the same crossing number amounts to specifying
logical constraints on the wiring harness. In the interest of space,
we defer examples to a forthcoming paper. Suffice it to say that both
the conditions ``alternating knot'' and ``contains the tangle
corresponding to 5/3'' are expressible. For example, it is possible to
calculate the characteristic formula of a process corresponding to the
tangle 5/3 and conjoin it into the classifying formula via the
composition connective of the logic.

Finally, we wish to observe that it is entirely within reason to
contemplate a more domain-specific version of spatial logic tailored
to the shape of processes in the image of the encoding. Such a
domain-specific logic would have a better claim to the title formal
language of knot properties.

% subsection example_formulae_ (end)

% section knots_as_processes (end) 

% section spatial logic via knots (end)

\section{Conclusions and future work}

\paragraph{Testing physical space}
You, gentle reader, may wonder why of all the theorems to be proved
given this set up we pick the one above. In some sense it's hardly
central to quantum mechanics. We see it as central in the sense that
it firmly establishes a notion of physical space arising from a notion
of the equivalence of behavior. Relating bisimulation to a metric is a
big step forward, but one is faced with interpreting the relationship
of that metric space to something more physical. Quantum mechanical
notions of ``physical'' space are still far from intuitive, but by
relating this idea of distance as testing to calculations that predict
physical circumstances we are making a not insignificant step forward
toward an understanding of the physical space we inhabit as
essentially dynamic.

\paragraph{Effectivity and simulation}
One of the observations we have yet to make is that the entire program
spelled out here is effective. We have built various interpreters for
the reflective calculus at work in this interpretation. In principle,
then, we can simulate quantum mechanics on a computer. The place where
the simulation may lose fidelity is the infinitely branching summation
for the annihilator.

In this connection i also want to point out that the evaluation style
calculation of the inner product puts the non-determinism of the
summation right at the heart of measurement. This suggests that
Milner's original reduction-based formulation of the dynamics of his
calculi in terms of sums was not just notationally suggestive of a
notion of measure-and-continue but captured some significant part of
the physics.

\paragraph{Quantum continuations}
In light of this last observation i want to point out that the
predominant account of quantum mechanics is missing a key aspect of a
truly compositional story of the physical situation. In a real lab,
when a measurement is made the observation can be made to feed into
another device that then makes another measurement conditioned on the
results of the first. This means that after the superposition was
collapsed the entire experimental set up remained in
superposition. While QM offers a means of writing this down it doesn't
quite line up well with the well-trodden formulation of computation
and continuation that we see so succinctly expressed in Milner's
calculi. This suggests that there might be advantages to this account
of dynamics waiting to be explored.

\paragraph{Quantum logic}
In this connection, we also note that by virtue of having the
Hennessy-Milner construction, we can pull the construction through the
interpretation of QM. This gives us a natural candidate for a quantum
logic that enjoys an extremely tight connection with it's domain of
interpretation, making the construction much less ad hoc (rather it is
the image of functor!).

\paragraph{Quantum probabiity}
i have questions about the basis of the interpretation of inner
product as probability amplitude. In particular, using which
axiomatization of probability theory does the notion of probability
amplitude earn the right to be so dubbed? In other words, where is the
proof that the operation for calculating a probability amplitude (and
then squaring) satisfies the axioms of what it means to calculate a
probability? Even if such a proof exists (i have yet to find it in the
literature), i wonder if it might not be possible to turn things on
their heads. Can we view the calculation of the probability amplitude
as an axiomatization of probability? If so, then the definition we
give for calculating probability amplitude may provide the basis for
an \emph{effective} theory of probability.

\paragraph{Quantum vs ``biological'' information}
Finally, i want to conclude with a more philosophical observation. At
a recent workshop in which QM was a predominant topic i noticed
something about quantum information. The speaker was giving a riveting
discussion of axiomatic QM and showing how properties of ``no
cloning'' and ``no deleting'' emerged as consequences of the
axiomatization. Theorems of this form are necessary to give us a sense
of confidence that our axioms characterize the physical theory. What
struck me, though, was that if quantum information is neither erasable
nor replicable it is markedly different from \emph{life}. Two of the
things we know about life is that

\begin{itemize}
  \item it ends;
  \item to gain some measure of persistence, to transcend it's
    finitude it is imminently copyable.
\end{itemize}

Both of these qualities are summarized succinctly in the aphorism: all
flesh is grass. For me these two kinds of ``information'' -- call them
quantum and biological -- are end points on a spectrum of strategies
for persistence. At one end, we have those curious entities that enjoy
uniqueness and permanence; at the other, we have those who in the face
of a certain end and an uncertain present make a go of passing
something on. To me one of the more remarkable aspects of the latter
strategy is that in the presence of noise (and certain features of
copying) we get a kind of dynamism, a chance for improvement against a
given persistent condition.

% subsection other_calculi_other_bisimulations_and_geometry_as_behavior (end)




% section conclusion (end)

%\documentclass[12pt]{llncs}
%\documentclass{jktr}

\usepackage[pdftex]{hyperref}                   
\usepackage {listings}
\usepackage {mathpartir}
\usepackage{bcprules}
%\usepackage{listings}
                       
\usepackage{graphicx} 
%\usepackage[margins=2.5cm,nohead,nofoot]{geometry}
%\usepackage{geometry}
\usepackage{amsfonts}
\usepackage{amstext}
\usepackage{latexsym}
\usepackage{amssymb}
\usepackage{color}


%\include{myPreamble}
\include{qm2pi.local} 

%\ifpdf
%\usepackage[pdftex]{graphicx}
%\else
%\usepackage{graphicx}
%\fi

 % \ifpdf
%  \usepackage{pdfsync}
%  \if


%\title{Brief Article}
%\author{David F. Snyder}
%\author{L.G. Meredith}

%\address{Dept. of Math., Texas State University--San Marcos, San Marcos, TX 78666}
       
\pagestyle{empty}


\begin{document}

\lstset{language=[Objective]Caml,frame=shadowbox}

\input{qm2pi.front}

% section front matter (end)

\input{qm2pi.intro} 
 
% section introduction (end)

% \input{qm2pi.knotations} 

% section notation (end)

\input{qm2pi.process.calculi} 

% section concurrent_process_calculi_and_spatial_logics_ (end)
    
%\input{qm2pi.knots2pi} 

%\input{qm2pi.trefoil} 

%\input{qm2pi.mainthm} 

% subsection basic_interpretation (end)

%\input{qm2pi.rho.presentation} 
\subsection{The syntax and semantics of the notation system}\label{sub:the_syntax_and_semantics_of_the_notation_system} % (fold)

We now summarize a technical presentation of the calculus that
embodies our theory of dynamics. The typical presentation of such a
calculus follows the style of giving generators and relations on
them. The grammar, below, describing term constructors, freely
generates the set of processes, $\Proc$. This set is then quotiented
by a relation known as structural congruence and it is over this set
that the notion of dynamics is expressed. This presentation is
essentially that of \cite{MeredithR05} with the addition of
polyadicity and summation. For readability we have relegated some of
the technical subtleties to an appendix.

\subsubsection{Process grammar}\label{subsub:process_grammar}

\begin{mathpar}
  \inferrule* [lab=synchronization] {} {{M} \bc \pzero \;|\; x?F \;|\; x!C }
  \and
  \inferrule* [lab=abstraction] {} {{F} \bc (x)P}
  \and
  \inferrule* [lab=concretion] {} {{C} \bc \langle Q \rangle}
  \and
  \inferrule* [lab=process] {} {{P,Q} \bc M \;| \;P|Q \;|\; @{x}}
  \and
  \inferrule* [lab=name] {} {{x} \bc \quotep{P}}
\end{mathpar} 

Note that $\vec{x}$ (resp. $\vec{P}$) denotes a vector of names
(resp. processes) of length $|\vec{x}|$ (resp. $|\vec{P}|$). We adopt
the following useful abbreviations.

\begin{mathpar}
   x?(\vec{y}).P := x.(\vec{y})P \and  x\clift{\vec{P}} := x.\clift{\vec{P}}
   \and x!(y) := \lift{x}{\dropn{y}}
   \and \Pi_{i=0}^{n-1}P_i := P_0 | \ldots | P_{n-1}
\end{mathpar}

\subsubsection{Structural congruence}

\paragraph{Free and bound names and alpha-equivalence.} At the
core of structural equivalence is alpha-equivalence which identifies
process that are the same up to a change of variable. Formally, we
recognize the distinction between free and bound names. The free names
of a process, $\freenames{P}$, may be calculated recursively as
follows:

\begin{mathpar}
\freenames{\pzero} := \emptyset
  \and \\
  \freenames{x?(y).P} := \{ x \} \cup (\freenames{P} \setminus \{ y \})
  \and 
  \freenames{x!\langle P \rangle} := \{ x \} \cup \{ P \} 
  \and \\
  \freenames{P|Q} := \freenames{P} \cup \freenames{Q}
  \and \\
  \freenames{@{x}} := \{ x \}
\end{mathpar}

$\pi$
$\quotep{\pi}$

$\freenames{-} : \pi \to \mathcal{P}(\quotep{\pi})$

\begin{eqnarray*}
  \freenames{\pzero} & := & \emptyset \\
  \freenames{x?(y).P} & := & \{ x \} \cup (\freenames{P} \setminus \{ y \}) \\
  \freenames{x!\langle P \rangle} & := & \{ x \} \cup \{ P \} \\
  \freenames{P|Q} & := & \freenames{P} \cup \freenames{Q} \\
  \freenames{\dropn{x}} & := & \{ x \}
\end{eqnarray*}

The bound names of a process, $\boundnames{P}$, are those names occurring in $P$
that are not free. For example, in $x?(y).0$, the name $x$ is free, while $y$ is bound.

\begin{mathpar}
  \inferrule* [lab=monoidal-laws] {} { P|Q \equiv Q|P \and P|0 \equiv P \and P|(Q|R) \equiv (P|Q)|R }
\end{mathpar}

\begin{mathpar}
  \inferrule* [lab=alpha-equivalence] {} { (x)P \equiv (y)P\{y/x\} \and y \not\in \freenames{P} }
\end{mathpar}

\begin{definition}
Then two processes, $P,Q$, are alpha-equivalent if $P = Q\{\vec{y}/\vec{x}\}$ for
some $\vec{x} \in \boundnames{Q},\vec{y} \in \boundnames{P}$, where $Q\{\vec{y}/\vec{x}\}$
denotes the capture-avoiding substitution of $\vec{y}$ for $\vec{x}$ in $Q$.
\end{definition}

\begin{definition}
  The {\em structural congruence} \cite{SangiorgiWalker} , $\equiv$,
  between processes is the least congruence containing
  alpha-equivalence, satisfying the abelian monoid laws
  (associativity, commutativity and $\pzero$ as identity) for parallel
  composition $|$ and for summation $+$.
\end{definition}

\subsection{Name equivalence}

We take name equivalence, written $\nameeq$, to be the smallest
equivalence relation generated by the following rules.

\begin{mathpar}
\inferrule*[lab=Quote-drop]
{ }
{ \quotep{@{x}} \nameeq x }

\inferrule*[lab=Struct-equiv]
{ P \scong Q }
{ \quotep{P} \nameeq \quotep{Q} }
\end{mathpar}

The astute reader will have noticed that the mutual recursion of names
and processes imposes a mutual recursion on alpha-equivalence and
structural equivalence via name-equivalence. Fortunately, all of this
works out pleasantly and we may calculate in the natural way, free of
concern. The reader interested in the details is referred to the
appendix \ref{appendix:rho_details}.

\subsection{Substitution}

We use $\Proc$ for the set of processes, $\QProc$ for the set of
names, and $\id{\{}\vec{y} / \vec{x} \id{\}}$ to denote partial maps,
$s : \QProc \rightarrow \QProc$. A map, $s$ lifts, uniquely, to a map
on process terms, $\widehat{s} : \Proc \rightarrow \Proc$ by the
following equations.

\begin{mathpar}
  (0) \psubstp{Q}{P} := 0 \\
  (R \juxtap S) \psubstp{Q}{P}
  :=    
  (R)\psubstp{Q}{P} \juxtap (S) \psubstp{Q}{P} \\
  (x?(y).R) \psubstp{Q}{P}    
  :=    
  (x)\substp{Q}{P} (z)\concat( (R \psubstn{z}{y}) \psubstp{Q}{P} ) \\
  (\lift{x}{R}) \psubstp{Q}{P}  
  :=
  \lift{(x)\substp{Q}{P}}{ R \psubstp{Q}{P} } \\
%   (\dropn{x})  \psubstp{Q}{P}       
%   := 
%   \left\{ 
%     \begin{array}{ccc} 
%       \dropn{\quotep{Q}} & & x \nameeq \quotep{P} \\
%       \dropn{x} & & otherwise \\
%     \end{array}
%   \right. 
  (\dropn{x})  \psubstp{Q}{P}       
  := 
  \left\{ 
    \begin{array}{ccc} 
      Q & & x \nameeq \quotep{P} \\
      \dropn{x} & & otherwise \\
    \end{array}
  \right.
\end{mathpar}
 

where

\begin{eqnarray}
  (x)\id{\{} \lpquote Q \rpquote / \lpquote P \rpquote \id{\}}            = 
  \left\{ 
    \begin{array}{ccc}
      \lpquote Q \rpquote & & x \nameeq \lpquote P \rpquote \\
      x & & otherwise \\
    \end{array}
  \right. \nonumber
\end{eqnarray}

and $z$ is chosen distinct from $\quotep{P}$, $\quotep{Q}$, the free
names in $Q$, and all the names in $R$. Our $\alpha$-equivalence will
be built in the standard way from this substitution.

\begin{remark}\label{rem:no_self_referential_names}
  One consequence of these definitions is that $\forall P. \quotep{P}
  \not\in \freenames{P}$.
\end{remark}

\subsection{ Dynamic quote: an example }

Anticipating something of what's to come, consider applying the
substitution, $\widehat{\id{\{}u / z \id{\}}}$, to the following pair
of processes, $\lift{w}{y!(z)}$ and $w[ \lpquote y!(z) \rpquote ]$.

\begin{eqnarray}
	\lift{w}{y!(z)}\widehat{\id{\{}u / z \id{\}}}
		& = &
		\lift{w}{y!(u)} \nonumber\\
	w[ \lpquote y!(z) \rpquote ] \widehat{ \id{\{}u / z \id{\}} }
		& = &
		w[ \lpquote y!(z) \rpquote ] \nonumber
\end{eqnarray}

Because the body of the process between quotes is impervious to
substitution, we get radically different answers. In fact, by
examining the first process in an input context,
e.g. $x?(z).\lift{w}{y!(z)}$, we see that the process under the lift
operator may be shaped by prefixed inputs binding a name inside it. In
this sense, the lift operator will be seen as a way to dynamically
construct processes before reifying them as names.

Finally equipped with these standard features we can present the
dynamics of the calculus.

\subsubsection{Operational semantics} 

Finally, we introduce the computational dynamics. What marks these
algebras as distinct from other more traditionally studied algebraic
structures, e.g. vector spaces or polynomial rings, is the manner in
which dynamics is captured. In traditional structures, dynamics is typically
expressed through morphisms between such structures, as in linear maps
between vector spaces or morphisms between rings. In algebras
associated with the semantics of computation, the dynamics is
expressed as part of the algebraic structure itself, through a
reduction reduction relation typically denoted by $\red$. Below, we
give a recursive presentation of this relation for the calculus used
in the encoding.

$\red \subseteq \pi \times \pi$
$\red : \pi \to \mathcal{P}(\pi)$

\begin{mathpar}
  \inferrule* [lab=Comm] { \textsf{match}( x_{src}, x_{trgt} ) } { x_{trgt}?(y)P \; | \; x_{src}!\langle {Q} \rangle \red P\{\quotep{Q}/y}\} }
  \and \\
  \inferrule* [lab=Par] {{P} \red {P}'} {{{P} | {Q}} \red {{P}' | {Q}}}
  \and
  \inferrule* [lab=Equiv]{{{P} \scong {P}'} \andalso {{P}' \red {Q}'} \andalso {{Q}' \scong {Q}}}{{P} \red {Q}}
\end{mathpar}

\begin{eqnarray*}
  match_{\equiv} (\quotep{P},\quotep{Q}) & := & P \equiv Q \\
  match_{\dagger}(\quotep{P},\quotep{Q}) & := & \forall R. P|Q \red^{*} R => R \red^{*} 0 \\
  match_{K}(\quotep{P},\quotep{Q}) & := & K \mbox{ for some context } K
\end{eqnarray*}

$u?(x)P | u!\langle Q \rangle \red P\{\quotep{Q}/x\}$

%We write $\wred$ for $\red^*$, and $P\red$ if $\exists Q $ such that $ P \red Q$.
We write $P\red$ if $\exists Q $ such that $ P \red Q$ and $P\not\red$, otherwise.

\section{Replication}

As mentioned before, it is known that replication (and hence
recursion) can be implemented in a higher-order process algebra
\cite{SangiorgiWalker}. As our first example of calculation with the
machinery thus far presented we give the construction explicitly in
the {\rhoc}.

\begin{eqnarray}
	D_{x} & := & \prefix{x}{y}{(\binpar{\outputp{x}{y}}{@{y}})} \nonumber\\
	\bangp_{x}{P} & := & \binpar{{x}!\langle{\binpar{D_{x}}{P}}\rangle}{D_{x}} \nonumber
\end{eqnarray}

\begin{eqnarray}
	\bangp_{x}{P} & & \nonumber\\
	=
	& {x}!\langle{(\prefix{x}{y}{(\outputp{x}{y} | @{y})) | P}}\rangle 
	      | \prefix{x}{y}{(\outputp{x}{y} | @{y})} & \nonumber\\
	\red
	& (\outputp{x}{y} | @{y})\substn{\quotep{(\prefix{x}{y}{(@{y} | \outputp{x}{y})) | P}}}{y} & \nonumber\\
	=
	& \outputp{x}{\quotep{(\prefix{x}{y}{(\outputp{x}{y} | @{y})) | P}}}
	  | {(\prefix{x}{y}{(\outputp{x}{y} | @{y})) | P}} & \nonumber\\
	\red
	& \ldots & \nonumber\\
	\red^*
	& P | P | \ldots & \nonumber
\end{eqnarray}

Of course, this encoding, as an implementation, runs away, unfolding
$\bangp{P}$ eagerly. A lazier and more implementable replication
operator, restricted to input-guarded processes, may be obtained as follows.

\begin{eqnarray}
\bangp{\prefix{u}{v}{P}} 
	:= 
	\binpar{\lift{x}{\prefix{u}{v}{(\binpar{D(x)}{P})}}}{D(x)} \nonumber
\end{eqnarray}

\begin{remark}
  Note that the lazier definition still does not deal with summation
  or mixed summation (i.e. sums over input and output). The reader is
  invited to construct definitions of replication that deal with these
  features. 

  Further, the definitions are parameterized in a name, $x$. Can you,
  gentle reader, make a definition that eliminates this parameter and
  guarantees no accidental interaction between the replication
  machinery and the process being replicated -- i.e. no accidental
  sharing of names used by the process to get its work done and the
  name(s) used by the replication to effect copying. This latter
  revision of the definition of replication is crucial to obtaining
  the expected identity $!!P \sim !P$.
\end{remark}

\begin{remark}\label{rem:paradoxical_combinator}
  The reader familiar with the lambda calculus will have noticed the
  similarity between $D$ and the paradoxical combinator.

  [Ed. note: the existence of this seems to suggest we have to be more
  restrictive on the set of processes and names we admit if we are to
  support no-cloning.]
\end{remark}

\subsubsection{Bisimulation}

The computational dynamics gives rise to another kind of equivalence,
the equivalence of computational behavior. As previously mentioned
this is typically captured \emph{via} some form of bisimulation.

% The notion we use in this paper is weak barbed bisimulation
% \cite{milner91polyadicpi}.

The notion we use in this paper is derived from weak barbed
bisimulation \cite{milner91polyadicpi}. 

\begin{definition}
An \emph{observation relation}, $\downarrow_{\mathcal N}$, over a set
of names, $\mathcal N$, is the smallest relation satisfying the rules
below.

\infrule[Out-barb]{y \in {\mathcal N}, \; x \nameeq y}
		  {\outputp{x}{v} \downarrow_{\mathcal N} x}
\infrule[Par-barb]{\mbox{$P\downarrow_{\mathcal N} x$ or $Q\downarrow_{\mathcal N} x$}}
		  {\binpar{P}{Q} \downarrow_{\mathcal N} x}

We write $P \Downarrow_{\mathcal N} x$ if there is $Q$ such that 
$P \wred Q$ and $Q \downarrow_{\mathcal N} x$.
\end{definition}

\begin{definition}
%\label{def.bbisim}
An  ${\mathcal N}$-\emph{barbed bisimulation} over a set of names, ${\mathcal N}$, is a symmetric binary relation 
${\mathcal S}_{\mathcal N}$ between agents such that $P\rel{S}_{\mathcal N}Q$ implies:
\begin{enumerate}
\item If $P \red P'$ then $Q \wred Q'$ and $P'\rel{S}_{\mathcal N} Q'$.
\item If $P\downarrow_{\mathcal N} x$, then $Q\Downarrow_{\mathcal N} x$.
\end{enumerate}
$P$ is ${\mathcal N}$-barbed bisimilar to $Q$, written
$P \wbbisim_{\mathcal N} Q$, if $P \rel{S}_{\mathcal N} Q$ for some ${\mathcal N}$-barbed bisimulation ${\mathcal S}_{\mathcal N}$.
\end{definition}

$\mathcal{R} \subseteq \pi \times \pi$

$P \mathcal{R} Q => \forall P'. P \red P' \Rightarrow \exists Q'. Q \red Q', P' \mathcal{R} Q'$

$P \vdash x \Rightarrow Q \vdash x$

\begin{mathpar}
  \inferrule*[lab=Out-barb]{x \nameeq y}{{y}!\langle{Q}\rangle \vdash x}
  \and
  \inferrule*[lab=Par-barb]{\mbox{$P\vdash x$ or $Q\vdash x$}}{\binpar{P}{Q} \vdash x}
\end{mathpar}

\subsubsection{Contexts}

One of the principle advantages of computational calculi like the
$\pi$-calculus is a well-defined notion of context,
contextual-equivalence and a correlation between
contextual-equivalence and notions of bisimulation. The notion of
context allows the decomposition of a process into (sub-)process and
its syntactic environment, its context. Thus, a context may be
thought of as a process with a ``hole'' (written $\Box$) in it. The
application of a context $M$ to a process $P$, written $M[P]$, is
tantamount to filling the hole in $M$ with $P$. In this paper we do
not need the full weight of this theory, but do make use of the notion
of context in the proof the main theorem. 

\begin{mathpar}
  \inferrule* [lab=summation] {} {{M_{M},M_{N}} \bc \Box \;|\; x.M_{A} \;|\; M_{M}+M_{N}}
  \and
  \inferrule* [lab=agent] {} {{M_{A}} \bc (\vec{x})M_{P} \;| \; \clift{P_0,\ldots,M_{P},\ldots,P_N}}
  \and \\
  \inferrule* [lab=process] {} {{M_{P}} \bc M_{N} \;| \;P|M_{P} }
\end{mathpar} 

\begin{mathpar}
  \inferrule* [lab=sychronization] {} {M_{N} \bc \Box \;|\; x?M_{F} \;|\; x!M_{C}}
  \and
  \inferrule* [lab=abstraction] {} {{M_{F}} \bc (x)M_{P} }
  \and
  \inferrule* [lab=concretion] {} {{M_{C}} \bc \langle M_{P} \rangle }
  \and \\
  \inferrule* [lab=process] {} {{M_{P}} \bc M_{N} \;| \;P|M_{P} }
\end{mathpar}

\begin{definition}[contextual application] Given a context $M$, and
  process $P$, we define the \emph{contextual application}, $M[P] :=
  M\{P/\Box\}$. That is, the contextual application of M to P is the
  substitution of $P$ for $\Box$ in $M$.
\end{definition}

$\meaningof{-} : L \to \mathcal{P}(\pi)$

\begin{mathpar}
  \inferrule* [lab=collection] {} {\meaningof{true} = \pi, \and \meaningof{~E} = \pi \setminus \meaningof{E}, \and \meaningof{E_{1} \& E_{2}} = \meaningof{E_{1}} \cap \meaningof{E_{2}}}
\end{mathpar}

\begin{mathpar}
  \inferrule* [lab=structure] {} {\meaningof{0} = \{ P \in \pi | P \equiv 0 \}, \and \\ \meaningof{E_1 | E_2} = \{ P \in \pi | P \equiv P_{1} | P_{2}, P_{1} \in \meaningof{E_{1}}, P_{2} \in \meaningof{E_2}\} }
\end{mathpar}

\begin{mathpar}
 \inferrule* [lab=behavior] {} {\meaningof{\langle a?b \rangle E} = \{ P \in \pi | P \equiv Q | u?(y)P', \\ \and \\\\ \and \\ \;\;\; u \in \meaningof{a}, \forall z.P'\{z/y\} \in \meaningof{E\{z/b\}}\}, \and \\ \meaningof{a!E} = \{ P \in \pi | P \equiv Q | x!\langle P' \rangle, x \in \meaningof{a} P' \in \meaningof{E}\} }
\end{mathpar}

\begin{mathpar}
 \inferrule* [lab=nominal] {} {\meaningof{\quotep{E}} = \{ \quotep{P} \in \quotep{\pi} | P \in \meaningof{E} \}, \and \meaningof{\quotep{P}} = \{ \quotep{Q} \in \quotep{\pi} | P \equiv Q \} \and \\ \meaningof{@\quotep{E}} = \{ P \in \pi | P \equiv @x, x \in \meaningof{E} \}}
\end{mathpar}

\begin{eqnarray*}
  \\
  \meaningof{-} : TS \to ST
\end{eqnarray*}

\begin{eqnarray*}
  \\
  L : TS \to ST
\end{eqnarray*}

\begin{eqnarray*}
  \\
  P \models E \iff P \in \meaningof{E}
\end{eqnarray*}

\begin{eqnarray*}
  P \approx_{L} Q \iff \forall E \in L. P \models E \iff Q \models E
\end{eqnarray*}

\begin{eqnarray*}
  P \approx_{K} Q
\end{eqnarray*}

\begin{eqnarray*}
  P \approx Q
\end{eqnarray*}

$\approx_{K} = \approx = \approx_{L}$

\subsubsection{Contextual duality}

Note that contexts extend the quotation operation to a family of
operations from processes to names. Given a context, $M$, we can
define a \emph{nominal context}, $\quotep{M}$ by $\quotep{M}[P] :=
\quotep{M[P]}$. To foreshadow what is to come we observe that these
operations enjoy a duality with processes very much like the duality
between vectors and maps from vectors to scalars.

Further, because the calculus is essentially higher-order, we have a
correspondence between contexts and processes. More specifically,
given a name $x$ and a context $M$ we can construct $M^{*}_{x}$ such
that 

\begin{mathpar}
  M^{*}_{x} | \lift{x}{P} \red M[P]
\end{mathpar}

namely,

\begin{mathpar}
  M^{*}_{x} := x?(u).M[\dropn{u}]
\end{mathpar}

The dependence of $M^{*}_{x}$ on a name makes it an abstraction, 

\begin{mathpar}
  M^{*} := (x)x?(u).M[\dropn{u}]
\end{mathpar}

\subsection{Additional notation}

It will sometimes be convenient to denote the process a name
quotes. We already have the notation $x = \quotep{P}$, but it will be
convenient to introduce an alternate notation, $\procn{x}$, when we
want to emphasize the connection to the use of the name. Note that, by
virtue of name equivalence, $\quotep{\procn{x}} \nameeq x$; so, the
notation is consistent with previous definitions.

Further, because names have structure it is possible to effect
substitutions on the basis of that structure. This means we need to
upgrade our notation for substitutions, which we accomplish by
adapting comprehension notation. Thus,

\begin{mathpar}
  P\{ y / x : x \in S \}
\end{mathpar}

is interpreted to mean the process derived from P by replacing (in a
capture-avoiding manner) each occurrence of $x$ in $S$ by $y$. For example,

\begin{mathpar}
  P\{ \quotep{\procn{x}|\procn{x}} / x : x \in \freenames{P} \}
\end{mathpar}

will replace each (occurrence) of a free name $x$ in $P$ by
$\quotep{\procn{x}|\procn{x}}$.

Also, we will avail ourselves of the notation $x^{L}$ and $x^{R}$ to
denote injections of a name into disjoint copies of the name
space. There are numerous ways to accomplish this. One example can be
found in \cite{MeredithR05}. This notation overloads to vectors of
names: $\vec{x}^{\pi} := (x_{i}^{\pi} \; : \; 0 \leq i < |\vec{x}| )$ where $\pi \in \{L,R\}$.

We also use $P^{\Box} := P|\Box$.

In \cite{MeredithR05} an interpretation of the new operator is
given. It turns out that there are several possible interpretations
all enjoying the requisite algebraic properties of the operator (see
\cite{milner91polyadicpi}). We will therefore make liberal use of
$(\nu\; \vec{x})P$.

% subsection the_syntax_and_semantics_of_the_notation_system (end)   

\input{qm2pi.qmops} 

\input{qm2pi.sterngerlach} 

\input{qm2pi.metric} 

% section concurrent_process_calculi (end)

%\input{qm2pi.proofsketch}

% section proof sketch (end)

%\input{qm2pi.slviaknots} 

% section spatial logic via knots (end)

\input{qm2pi.conclusion}

% section conclusion (end)

%\input{qm2pi.dtcodes} 

% section wiring algorithm (end)

\input{qm2pi.ack} 

% section acknowledgments (end)

\newpage


\bibliographystyle{plain}   
\bibliography{../../biblios/main.bib}

\input{qm2pi.rhodetails}

\end{document}

 

% section wiring algorithm (end)

\documentclass[12pt]{llncs}
%\documentclass{jktr}

\usepackage[pdftex]{hyperref}                   
\usepackage {listings}
\usepackage {mathpartir}
\usepackage{bcprules}
%\usepackage{listings}
                       
\usepackage{graphicx} 
%\usepackage[margins=2.5cm,nohead,nofoot]{geometry}
%\usepackage{geometry}
\usepackage{amsfonts}
\usepackage{amstext}
\usepackage{latexsym}
\usepackage{amssymb}
\usepackage{color}


%\include{myPreamble}
\include{qm2pi.local} 

%\ifpdf
%\usepackage[pdftex]{graphicx}
%\else
%\usepackage{graphicx}
%\fi

 % \ifpdf
%  \usepackage{pdfsync}
%  \if


%\title{Brief Article}
%\author{David F. Snyder}
%\author{L.G. Meredith}

%\address{Dept. of Math., Texas State University--San Marcos, San Marcos, TX 78666}
       
\pagestyle{empty}


\begin{document}

\lstset{language=[Objective]Caml,frame=shadowbox}

\input{qm2pi.front}

% section front matter (end)

\input{qm2pi.intro} 
 
% section introduction (end)

% \input{qm2pi.knotations} 

% section notation (end)

\input{qm2pi.process.calculi} 

% section concurrent_process_calculi_and_spatial_logics_ (end)
    
%\input{qm2pi.knots2pi} 

%\input{qm2pi.trefoil} 

%\input{qm2pi.mainthm} 

% subsection basic_interpretation (end)

%\input{qm2pi.rho.presentation} 
\subsection{The syntax and semantics of the notation system}\label{sub:the_syntax_and_semantics_of_the_notation_system} % (fold)

We now summarize a technical presentation of the calculus that
embodies our theory of dynamics. The typical presentation of such a
calculus follows the style of giving generators and relations on
them. The grammar, below, describing term constructors, freely
generates the set of processes, $\Proc$. This set is then quotiented
by a relation known as structural congruence and it is over this set
that the notion of dynamics is expressed. This presentation is
essentially that of \cite{MeredithR05} with the addition of
polyadicity and summation. For readability we have relegated some of
the technical subtleties to an appendix.

\subsubsection{Process grammar}\label{subsub:process_grammar}

\begin{mathpar}
  \inferrule* [lab=synchronization] {} {{M} \bc \pzero \;|\; x?F \;|\; x!C }
  \and
  \inferrule* [lab=abstraction] {} {{F} \bc (x)P}
  \and
  \inferrule* [lab=concretion] {} {{C} \bc \langle Q \rangle}
  \and
  \inferrule* [lab=process] {} {{P,Q} \bc M \;| \;P|Q \;|\; @{x}}
  \and
  \inferrule* [lab=name] {} {{x} \bc \quotep{P}}
\end{mathpar} 

Note that $\vec{x}$ (resp. $\vec{P}$) denotes a vector of names
(resp. processes) of length $|\vec{x}|$ (resp. $|\vec{P}|$). We adopt
the following useful abbreviations.

\begin{mathpar}
   x?(\vec{y}).P := x.(\vec{y})P \and  x\clift{\vec{P}} := x.\clift{\vec{P}}
   \and x!(y) := \lift{x}{\dropn{y}}
   \and \Pi_{i=0}^{n-1}P_i := P_0 | \ldots | P_{n-1}
\end{mathpar}

\subsubsection{Structural congruence}

\paragraph{Free and bound names and alpha-equivalence.} At the
core of structural equivalence is alpha-equivalence which identifies
process that are the same up to a change of variable. Formally, we
recognize the distinction between free and bound names. The free names
of a process, $\freenames{P}$, may be calculated recursively as
follows:

\begin{mathpar}
\freenames{\pzero} := \emptyset
  \and \\
  \freenames{x?(y).P} := \{ x \} \cup (\freenames{P} \setminus \{ y \})
  \and 
  \freenames{x!\langle P \rangle} := \{ x \} \cup \{ P \} 
  \and \\
  \freenames{P|Q} := \freenames{P} \cup \freenames{Q}
  \and \\
  \freenames{@{x}} := \{ x \}
\end{mathpar}

$\pi$
$\quotep{\pi}$

$\freenames{-} : \pi \to \mathcal{P}(\quotep{\pi})$

\begin{eqnarray*}
  \freenames{\pzero} & := & \emptyset \\
  \freenames{x?(y).P} & := & \{ x \} \cup (\freenames{P} \setminus \{ y \}) \\
  \freenames{x!\langle P \rangle} & := & \{ x \} \cup \{ P \} \\
  \freenames{P|Q} & := & \freenames{P} \cup \freenames{Q} \\
  \freenames{\dropn{x}} & := & \{ x \}
\end{eqnarray*}

The bound names of a process, $\boundnames{P}$, are those names occurring in $P$
that are not free. For example, in $x?(y).0$, the name $x$ is free, while $y$ is bound.

\begin{mathpar}
  \inferrule* [lab=monoidal-laws] {} { P|Q \equiv Q|P \and P|0 \equiv P \and P|(Q|R) \equiv (P|Q)|R }
\end{mathpar}

\begin{mathpar}
  \inferrule* [lab=alpha-equivalence] {} { (x)P \equiv (y)P\{y/x\} \and y \not\in \freenames{P} }
\end{mathpar}

\begin{definition}
Then two processes, $P,Q$, are alpha-equivalent if $P = Q\{\vec{y}/\vec{x}\}$ for
some $\vec{x} \in \boundnames{Q},\vec{y} \in \boundnames{P}$, where $Q\{\vec{y}/\vec{x}\}$
denotes the capture-avoiding substitution of $\vec{y}$ for $\vec{x}$ in $Q$.
\end{definition}

\begin{definition}
  The {\em structural congruence} \cite{SangiorgiWalker} , $\equiv$,
  between processes is the least congruence containing
  alpha-equivalence, satisfying the abelian monoid laws
  (associativity, commutativity and $\pzero$ as identity) for parallel
  composition $|$ and for summation $+$.
\end{definition}

\subsection{Name equivalence}

We take name equivalence, written $\nameeq$, to be the smallest
equivalence relation generated by the following rules.

\begin{mathpar}
\inferrule*[lab=Quote-drop]
{ }
{ \quotep{@{x}} \nameeq x }

\inferrule*[lab=Struct-equiv]
{ P \scong Q }
{ \quotep{P} \nameeq \quotep{Q} }
\end{mathpar}

The astute reader will have noticed that the mutual recursion of names
and processes imposes a mutual recursion on alpha-equivalence and
structural equivalence via name-equivalence. Fortunately, all of this
works out pleasantly and we may calculate in the natural way, free of
concern. The reader interested in the details is referred to the
appendix \ref{appendix:rho_details}.

\subsection{Substitution}

We use $\Proc$ for the set of processes, $\QProc$ for the set of
names, and $\id{\{}\vec{y} / \vec{x} \id{\}}$ to denote partial maps,
$s : \QProc \rightarrow \QProc$. A map, $s$ lifts, uniquely, to a map
on process terms, $\widehat{s} : \Proc \rightarrow \Proc$ by the
following equations.

\begin{mathpar}
  (0) \psubstp{Q}{P} := 0 \\
  (R \juxtap S) \psubstp{Q}{P}
  :=    
  (R)\psubstp{Q}{P} \juxtap (S) \psubstp{Q}{P} \\
  (x?(y).R) \psubstp{Q}{P}    
  :=    
  (x)\substp{Q}{P} (z)\concat( (R \psubstn{z}{y}) \psubstp{Q}{P} ) \\
  (\lift{x}{R}) \psubstp{Q}{P}  
  :=
  \lift{(x)\substp{Q}{P}}{ R \psubstp{Q}{P} } \\
%   (\dropn{x})  \psubstp{Q}{P}       
%   := 
%   \left\{ 
%     \begin{array}{ccc} 
%       \dropn{\quotep{Q}} & & x \nameeq \quotep{P} \\
%       \dropn{x} & & otherwise \\
%     \end{array}
%   \right. 
  (\dropn{x})  \psubstp{Q}{P}       
  := 
  \left\{ 
    \begin{array}{ccc} 
      Q & & x \nameeq \quotep{P} \\
      \dropn{x} & & otherwise \\
    \end{array}
  \right.
\end{mathpar}
 

where

\begin{eqnarray}
  (x)\id{\{} \lpquote Q \rpquote / \lpquote P \rpquote \id{\}}            = 
  \left\{ 
    \begin{array}{ccc}
      \lpquote Q \rpquote & & x \nameeq \lpquote P \rpquote \\
      x & & otherwise \\
    \end{array}
  \right. \nonumber
\end{eqnarray}

and $z$ is chosen distinct from $\quotep{P}$, $\quotep{Q}$, the free
names in $Q$, and all the names in $R$. Our $\alpha$-equivalence will
be built in the standard way from this substitution.

\begin{remark}\label{rem:no_self_referential_names}
  One consequence of these definitions is that $\forall P. \quotep{P}
  \not\in \freenames{P}$.
\end{remark}

\subsection{ Dynamic quote: an example }

Anticipating something of what's to come, consider applying the
substitution, $\widehat{\id{\{}u / z \id{\}}}$, to the following pair
of processes, $\lift{w}{y!(z)}$ and $w[ \lpquote y!(z) \rpquote ]$.

\begin{eqnarray}
	\lift{w}{y!(z)}\widehat{\id{\{}u / z \id{\}}}
		& = &
		\lift{w}{y!(u)} \nonumber\\
	w[ \lpquote y!(z) \rpquote ] \widehat{ \id{\{}u / z \id{\}} }
		& = &
		w[ \lpquote y!(z) \rpquote ] \nonumber
\end{eqnarray}

Because the body of the process between quotes is impervious to
substitution, we get radically different answers. In fact, by
examining the first process in an input context,
e.g. $x?(z).\lift{w}{y!(z)}$, we see that the process under the lift
operator may be shaped by prefixed inputs binding a name inside it. In
this sense, the lift operator will be seen as a way to dynamically
construct processes before reifying them as names.

Finally equipped with these standard features we can present the
dynamics of the calculus.

\subsubsection{Operational semantics} 

Finally, we introduce the computational dynamics. What marks these
algebras as distinct from other more traditionally studied algebraic
structures, e.g. vector spaces or polynomial rings, is the manner in
which dynamics is captured. In traditional structures, dynamics is typically
expressed through morphisms between such structures, as in linear maps
between vector spaces or morphisms between rings. In algebras
associated with the semantics of computation, the dynamics is
expressed as part of the algebraic structure itself, through a
reduction reduction relation typically denoted by $\red$. Below, we
give a recursive presentation of this relation for the calculus used
in the encoding.

$\red \subseteq \pi \times \pi$
$\red : \pi \to \mathcal{P}(\pi)$

\begin{mathpar}
  \inferrule* [lab=Comm] { \textsf{match}( x_{src}, x_{trgt} ) } { x_{trgt}?(y)P \; | \; x_{src}!\langle {Q} \rangle \red P\{\quotep{Q}/y}\} }
  \and \\
  \inferrule* [lab=Par] {{P} \red {P}'} {{{P} | {Q}} \red {{P}' | {Q}}}
  \and
  \inferrule* [lab=Equiv]{{{P} \scong {P}'} \andalso {{P}' \red {Q}'} \andalso {{Q}' \scong {Q}}}{{P} \red {Q}}
\end{mathpar}

\begin{eqnarray*}
  match_{\equiv} (\quotep{P},\quotep{Q}) & := & P \equiv Q \\
  match_{\dagger}(\quotep{P},\quotep{Q}) & := & \forall R. P|Q \red^{*} R => R \red^{*} 0 \\
  match_{K}(\quotep{P},\quotep{Q}) & := & K \mbox{ for some context } K
\end{eqnarray*}

$u?(x)P | u!\langle Q \rangle \red P\{\quotep{Q}/x\}$

%We write $\wred$ for $\red^*$, and $P\red$ if $\exists Q $ such that $ P \red Q$.
We write $P\red$ if $\exists Q $ such that $ P \red Q$ and $P\not\red$, otherwise.

\section{Replication}

As mentioned before, it is known that replication (and hence
recursion) can be implemented in a higher-order process algebra
\cite{SangiorgiWalker}. As our first example of calculation with the
machinery thus far presented we give the construction explicitly in
the {\rhoc}.

\begin{eqnarray}
	D_{x} & := & \prefix{x}{y}{(\binpar{\outputp{x}{y}}{@{y}})} \nonumber\\
	\bangp_{x}{P} & := & \binpar{{x}!\langle{\binpar{D_{x}}{P}}\rangle}{D_{x}} \nonumber
\end{eqnarray}

\begin{eqnarray}
	\bangp_{x}{P} & & \nonumber\\
	=
	& {x}!\langle{(\prefix{x}{y}{(\outputp{x}{y} | @{y})) | P}}\rangle 
	      | \prefix{x}{y}{(\outputp{x}{y} | @{y})} & \nonumber\\
	\red
	& (\outputp{x}{y} | @{y})\substn{\quotep{(\prefix{x}{y}{(@{y} | \outputp{x}{y})) | P}}}{y} & \nonumber\\
	=
	& \outputp{x}{\quotep{(\prefix{x}{y}{(\outputp{x}{y} | @{y})) | P}}}
	  | {(\prefix{x}{y}{(\outputp{x}{y} | @{y})) | P}} & \nonumber\\
	\red
	& \ldots & \nonumber\\
	\red^*
	& P | P | \ldots & \nonumber
\end{eqnarray}

Of course, this encoding, as an implementation, runs away, unfolding
$\bangp{P}$ eagerly. A lazier and more implementable replication
operator, restricted to input-guarded processes, may be obtained as follows.

\begin{eqnarray}
\bangp{\prefix{u}{v}{P}} 
	:= 
	\binpar{\lift{x}{\prefix{u}{v}{(\binpar{D(x)}{P})}}}{D(x)} \nonumber
\end{eqnarray}

\begin{remark}
  Note that the lazier definition still does not deal with summation
  or mixed summation (i.e. sums over input and output). The reader is
  invited to construct definitions of replication that deal with these
  features. 

  Further, the definitions are parameterized in a name, $x$. Can you,
  gentle reader, make a definition that eliminates this parameter and
  guarantees no accidental interaction between the replication
  machinery and the process being replicated -- i.e. no accidental
  sharing of names used by the process to get its work done and the
  name(s) used by the replication to effect copying. This latter
  revision of the definition of replication is crucial to obtaining
  the expected identity $!!P \sim !P$.
\end{remark}

\begin{remark}\label{rem:paradoxical_combinator}
  The reader familiar with the lambda calculus will have noticed the
  similarity between $D$ and the paradoxical combinator.

  [Ed. note: the existence of this seems to suggest we have to be more
  restrictive on the set of processes and names we admit if we are to
  support no-cloning.]
\end{remark}

\subsubsection{Bisimulation}

The computational dynamics gives rise to another kind of equivalence,
the equivalence of computational behavior. As previously mentioned
this is typically captured \emph{via} some form of bisimulation.

% The notion we use in this paper is weak barbed bisimulation
% \cite{milner91polyadicpi}.

The notion we use in this paper is derived from weak barbed
bisimulation \cite{milner91polyadicpi}. 

\begin{definition}
An \emph{observation relation}, $\downarrow_{\mathcal N}$, over a set
of names, $\mathcal N$, is the smallest relation satisfying the rules
below.

\infrule[Out-barb]{y \in {\mathcal N}, \; x \nameeq y}
		  {\outputp{x}{v} \downarrow_{\mathcal N} x}
\infrule[Par-barb]{\mbox{$P\downarrow_{\mathcal N} x$ or $Q\downarrow_{\mathcal N} x$}}
		  {\binpar{P}{Q} \downarrow_{\mathcal N} x}

We write $P \Downarrow_{\mathcal N} x$ if there is $Q$ such that 
$P \wred Q$ and $Q \downarrow_{\mathcal N} x$.
\end{definition}

\begin{definition}
%\label{def.bbisim}
An  ${\mathcal N}$-\emph{barbed bisimulation} over a set of names, ${\mathcal N}$, is a symmetric binary relation 
${\mathcal S}_{\mathcal N}$ between agents such that $P\rel{S}_{\mathcal N}Q$ implies:
\begin{enumerate}
\item If $P \red P'$ then $Q \wred Q'$ and $P'\rel{S}_{\mathcal N} Q'$.
\item If $P\downarrow_{\mathcal N} x$, then $Q\Downarrow_{\mathcal N} x$.
\end{enumerate}
$P$ is ${\mathcal N}$-barbed bisimilar to $Q$, written
$P \wbbisim_{\mathcal N} Q$, if $P \rel{S}_{\mathcal N} Q$ for some ${\mathcal N}$-barbed bisimulation ${\mathcal S}_{\mathcal N}$.
\end{definition}

$\mathcal{R} \subseteq \pi \times \pi$

$P \mathcal{R} Q => \forall P'. P \red P' \Rightarrow \exists Q'. Q \red Q', P' \mathcal{R} Q'$

$P \vdash x \Rightarrow Q \vdash x$

\begin{mathpar}
  \inferrule*[lab=Out-barb]{x \nameeq y}{{y}!\langle{Q}\rangle \vdash x}
  \and
  \inferrule*[lab=Par-barb]{\mbox{$P\vdash x$ or $Q\vdash x$}}{\binpar{P}{Q} \vdash x}
\end{mathpar}

\subsubsection{Contexts}

One of the principle advantages of computational calculi like the
$\pi$-calculus is a well-defined notion of context,
contextual-equivalence and a correlation between
contextual-equivalence and notions of bisimulation. The notion of
context allows the decomposition of a process into (sub-)process and
its syntactic environment, its context. Thus, a context may be
thought of as a process with a ``hole'' (written $\Box$) in it. The
application of a context $M$ to a process $P$, written $M[P]$, is
tantamount to filling the hole in $M$ with $P$. In this paper we do
not need the full weight of this theory, but do make use of the notion
of context in the proof the main theorem. 

\begin{mathpar}
  \inferrule* [lab=summation] {} {{M_{M},M_{N}} \bc \Box \;|\; x.M_{A} \;|\; M_{M}+M_{N}}
  \and
  \inferrule* [lab=agent] {} {{M_{A}} \bc (\vec{x})M_{P} \;| \; \clift{P_0,\ldots,M_{P},\ldots,P_N}}
  \and \\
  \inferrule* [lab=process] {} {{M_{P}} \bc M_{N} \;| \;P|M_{P} }
\end{mathpar} 

\begin{mathpar}
  \inferrule* [lab=sychronization] {} {M_{N} \bc \Box \;|\; x?M_{F} \;|\; x!M_{C}}
  \and
  \inferrule* [lab=abstraction] {} {{M_{F}} \bc (x)M_{P} }
  \and
  \inferrule* [lab=concretion] {} {{M_{C}} \bc \langle M_{P} \rangle }
  \and \\
  \inferrule* [lab=process] {} {{M_{P}} \bc M_{N} \;| \;P|M_{P} }
\end{mathpar}

\begin{definition}[contextual application] Given a context $M$, and
  process $P$, we define the \emph{contextual application}, $M[P] :=
  M\{P/\Box\}$. That is, the contextual application of M to P is the
  substitution of $P$ for $\Box$ in $M$.
\end{definition}

$\meaningof{-} : L \to \mathcal{P}(\pi)$

\begin{mathpar}
  \inferrule* [lab=collection] {} {\meaningof{true} = \pi, \and \meaningof{~E} = \pi \setminus \meaningof{E}, \and \meaningof{E_{1} \& E_{2}} = \meaningof{E_{1}} \cap \meaningof{E_{2}}}
\end{mathpar}

\begin{mathpar}
  \inferrule* [lab=structure] {} {\meaningof{0} = \{ P \in \pi | P \equiv 0 \}, \and \\ \meaningof{E_1 | E_2} = \{ P \in \pi | P \equiv P_{1} | P_{2}, P_{1} \in \meaningof{E_{1}}, P_{2} \in \meaningof{E_2}\} }
\end{mathpar}

\begin{mathpar}
 \inferrule* [lab=behavior] {} {\meaningof{\langle a?b \rangle E} = \{ P \in \pi | P \equiv Q | u?(y)P', \\ \and \\\\ \and \\ \;\;\; u \in \meaningof{a}, \forall z.P'\{z/y\} \in \meaningof{E\{z/b\}}\}, \and \\ \meaningof{a!E} = \{ P \in \pi | P \equiv Q | x!\langle P' \rangle, x \in \meaningof{a} P' \in \meaningof{E}\} }
\end{mathpar}

\begin{mathpar}
 \inferrule* [lab=nominal] {} {\meaningof{\quotep{E}} = \{ \quotep{P} \in \quotep{\pi} | P \in \meaningof{E} \}, \and \meaningof{\quotep{P}} = \{ \quotep{Q} \in \quotep{\pi} | P \equiv Q \} \and \\ \meaningof{@\quotep{E}} = \{ P \in \pi | P \equiv @x, x \in \meaningof{E} \}}
\end{mathpar}

\begin{eqnarray*}
  \\
  \meaningof{-} : TS \to ST
\end{eqnarray*}

\begin{eqnarray*}
  \\
  L : TS \to ST
\end{eqnarray*}

\begin{eqnarray*}
  \\
  P \models E \iff P \in \meaningof{E}
\end{eqnarray*}

\begin{eqnarray*}
  P \approx_{L} Q \iff \forall E \in L. P \models E \iff Q \models E
\end{eqnarray*}

\begin{eqnarray*}
  P \approx_{K} Q
\end{eqnarray*}

\begin{eqnarray*}
  P \approx Q
\end{eqnarray*}

$\approx_{K} = \approx = \approx_{L}$

\subsubsection{Contextual duality}

Note that contexts extend the quotation operation to a family of
operations from processes to names. Given a context, $M$, we can
define a \emph{nominal context}, $\quotep{M}$ by $\quotep{M}[P] :=
\quotep{M[P]}$. To foreshadow what is to come we observe that these
operations enjoy a duality with processes very much like the duality
between vectors and maps from vectors to scalars.

Further, because the calculus is essentially higher-order, we have a
correspondence between contexts and processes. More specifically,
given a name $x$ and a context $M$ we can construct $M^{*}_{x}$ such
that 

\begin{mathpar}
  M^{*}_{x} | \lift{x}{P} \red M[P]
\end{mathpar}

namely,

\begin{mathpar}
  M^{*}_{x} := x?(u).M[\dropn{u}]
\end{mathpar}

The dependence of $M^{*}_{x}$ on a name makes it an abstraction, 

\begin{mathpar}
  M^{*} := (x)x?(u).M[\dropn{u}]
\end{mathpar}

\subsection{Additional notation}

It will sometimes be convenient to denote the process a name
quotes. We already have the notation $x = \quotep{P}$, but it will be
convenient to introduce an alternate notation, $\procn{x}$, when we
want to emphasize the connection to the use of the name. Note that, by
virtue of name equivalence, $\quotep{\procn{x}} \nameeq x$; so, the
notation is consistent with previous definitions.

Further, because names have structure it is possible to effect
substitutions on the basis of that structure. This means we need to
upgrade our notation for substitutions, which we accomplish by
adapting comprehension notation. Thus,

\begin{mathpar}
  P\{ y / x : x \in S \}
\end{mathpar}

is interpreted to mean the process derived from P by replacing (in a
capture-avoiding manner) each occurrence of $x$ in $S$ by $y$. For example,

\begin{mathpar}
  P\{ \quotep{\procn{x}|\procn{x}} / x : x \in \freenames{P} \}
\end{mathpar}

will replace each (occurrence) of a free name $x$ in $P$ by
$\quotep{\procn{x}|\procn{x}}$.

Also, we will avail ourselves of the notation $x^{L}$ and $x^{R}$ to
denote injections of a name into disjoint copies of the name
space. There are numerous ways to accomplish this. One example can be
found in \cite{MeredithR05}. This notation overloads to vectors of
names: $\vec{x}^{\pi} := (x_{i}^{\pi} \; : \; 0 \leq i < |\vec{x}| )$ where $\pi \in \{L,R\}$.

We also use $P^{\Box} := P|\Box$.

In \cite{MeredithR05} an interpretation of the new operator is
given. It turns out that there are several possible interpretations
all enjoying the requisite algebraic properties of the operator (see
\cite{milner91polyadicpi}). We will therefore make liberal use of
$(\nu\; \vec{x})P$.

% subsection the_syntax_and_semantics_of_the_notation_system (end)   

\input{qm2pi.qmops} 

\input{qm2pi.sterngerlach} 

\input{qm2pi.metric} 

% section concurrent_process_calculi (end)

%\input{qm2pi.proofsketch}

% section proof sketch (end)

%\input{qm2pi.slviaknots} 

% section spatial logic via knots (end)

\input{qm2pi.conclusion}

% section conclusion (end)

%\input{qm2pi.dtcodes} 

% section wiring algorithm (end)

\input{qm2pi.ack} 

% section acknowledgments (end)

\newpage


\bibliographystyle{plain}   
\bibliography{../../biblios/main.bib}

\input{qm2pi.rhodetails}

\end{document}

 

% section acknowledgments (end)

\newpage


\bibliographystyle{plain}   
\bibliography{../../biblios/main.bib}

\documentclass[12pt]{llncs}
%\documentclass{jktr}

\usepackage[pdftex]{hyperref}                   
\usepackage {listings}
\usepackage {mathpartir}
\usepackage{bcprules}
%\usepackage{listings}
                       
\usepackage{graphicx} 
%\usepackage[margins=2.5cm,nohead,nofoot]{geometry}
%\usepackage{geometry}
\usepackage{amsfonts}
\usepackage{amstext}
\usepackage{latexsym}
\usepackage{amssymb}
\usepackage{color}


%\include{myPreamble}
\include{qm2pi.local} 

%\ifpdf
%\usepackage[pdftex]{graphicx}
%\else
%\usepackage{graphicx}
%\fi

 % \ifpdf
%  \usepackage{pdfsync}
%  \if


%\title{Brief Article}
%\author{David F. Snyder}
%\author{L.G. Meredith}

%\address{Dept. of Math., Texas State University--San Marcos, San Marcos, TX 78666}
       
\pagestyle{empty}


\begin{document}

\lstset{language=[Objective]Caml,frame=shadowbox}

\input{qm2pi.front}

% section front matter (end)

\input{qm2pi.intro} 
 
% section introduction (end)

% \input{qm2pi.knotations} 

% section notation (end)

\input{qm2pi.process.calculi} 

% section concurrent_process_calculi_and_spatial_logics_ (end)
    
%\input{qm2pi.knots2pi} 

%\input{qm2pi.trefoil} 

%\input{qm2pi.mainthm} 

% subsection basic_interpretation (end)

%\input{qm2pi.rho.presentation} 
\subsection{The syntax and semantics of the notation system}\label{sub:the_syntax_and_semantics_of_the_notation_system} % (fold)

We now summarize a technical presentation of the calculus that
embodies our theory of dynamics. The typical presentation of such a
calculus follows the style of giving generators and relations on
them. The grammar, below, describing term constructors, freely
generates the set of processes, $\Proc$. This set is then quotiented
by a relation known as structural congruence and it is over this set
that the notion of dynamics is expressed. This presentation is
essentially that of \cite{MeredithR05} with the addition of
polyadicity and summation. For readability we have relegated some of
the technical subtleties to an appendix.

\subsubsection{Process grammar}\label{subsub:process_grammar}

\begin{mathpar}
  \inferrule* [lab=synchronization] {} {{M} \bc \pzero \;|\; x?F \;|\; x!C }
  \and
  \inferrule* [lab=abstraction] {} {{F} \bc (x)P}
  \and
  \inferrule* [lab=concretion] {} {{C} \bc \langle Q \rangle}
  \and
  \inferrule* [lab=process] {} {{P,Q} \bc M \;| \;P|Q \;|\; @{x}}
  \and
  \inferrule* [lab=name] {} {{x} \bc \quotep{P}}
\end{mathpar} 

Note that $\vec{x}$ (resp. $\vec{P}$) denotes a vector of names
(resp. processes) of length $|\vec{x}|$ (resp. $|\vec{P}|$). We adopt
the following useful abbreviations.

\begin{mathpar}
   x?(\vec{y}).P := x.(\vec{y})P \and  x\clift{\vec{P}} := x.\clift{\vec{P}}
   \and x!(y) := \lift{x}{\dropn{y}}
   \and \Pi_{i=0}^{n-1}P_i := P_0 | \ldots | P_{n-1}
\end{mathpar}

\subsubsection{Structural congruence}

\paragraph{Free and bound names and alpha-equivalence.} At the
core of structural equivalence is alpha-equivalence which identifies
process that are the same up to a change of variable. Formally, we
recognize the distinction between free and bound names. The free names
of a process, $\freenames{P}$, may be calculated recursively as
follows:

\begin{mathpar}
\freenames{\pzero} := \emptyset
  \and \\
  \freenames{x?(y).P} := \{ x \} \cup (\freenames{P} \setminus \{ y \})
  \and 
  \freenames{x!\langle P \rangle} := \{ x \} \cup \{ P \} 
  \and \\
  \freenames{P|Q} := \freenames{P} \cup \freenames{Q}
  \and \\
  \freenames{@{x}} := \{ x \}
\end{mathpar}

$\pi$
$\quotep{\pi}$

$\freenames{-} : \pi \to \mathcal{P}(\quotep{\pi})$

\begin{eqnarray*}
  \freenames{\pzero} & := & \emptyset \\
  \freenames{x?(y).P} & := & \{ x \} \cup (\freenames{P} \setminus \{ y \}) \\
  \freenames{x!\langle P \rangle} & := & \{ x \} \cup \{ P \} \\
  \freenames{P|Q} & := & \freenames{P} \cup \freenames{Q} \\
  \freenames{\dropn{x}} & := & \{ x \}
\end{eqnarray*}

The bound names of a process, $\boundnames{P}$, are those names occurring in $P$
that are not free. For example, in $x?(y).0$, the name $x$ is free, while $y$ is bound.

\begin{mathpar}
  \inferrule* [lab=monoidal-laws] {} { P|Q \equiv Q|P \and P|0 \equiv P \and P|(Q|R) \equiv (P|Q)|R }
\end{mathpar}

\begin{mathpar}
  \inferrule* [lab=alpha-equivalence] {} { (x)P \equiv (y)P\{y/x\} \and y \not\in \freenames{P} }
\end{mathpar}

\begin{definition}
Then two processes, $P,Q$, are alpha-equivalent if $P = Q\{\vec{y}/\vec{x}\}$ for
some $\vec{x} \in \boundnames{Q},\vec{y} \in \boundnames{P}$, where $Q\{\vec{y}/\vec{x}\}$
denotes the capture-avoiding substitution of $\vec{y}$ for $\vec{x}$ in $Q$.
\end{definition}

\begin{definition}
  The {\em structural congruence} \cite{SangiorgiWalker} , $\equiv$,
  between processes is the least congruence containing
  alpha-equivalence, satisfying the abelian monoid laws
  (associativity, commutativity and $\pzero$ as identity) for parallel
  composition $|$ and for summation $+$.
\end{definition}

\subsection{Name equivalence}

We take name equivalence, written $\nameeq$, to be the smallest
equivalence relation generated by the following rules.

\begin{mathpar}
\inferrule*[lab=Quote-drop]
{ }
{ \quotep{@{x}} \nameeq x }

\inferrule*[lab=Struct-equiv]
{ P \scong Q }
{ \quotep{P} \nameeq \quotep{Q} }
\end{mathpar}

The astute reader will have noticed that the mutual recursion of names
and processes imposes a mutual recursion on alpha-equivalence and
structural equivalence via name-equivalence. Fortunately, all of this
works out pleasantly and we may calculate in the natural way, free of
concern. The reader interested in the details is referred to the
appendix \ref{appendix:rho_details}.

\subsection{Substitution}

We use $\Proc$ for the set of processes, $\QProc$ for the set of
names, and $\id{\{}\vec{y} / \vec{x} \id{\}}$ to denote partial maps,
$s : \QProc \rightarrow \QProc$. A map, $s$ lifts, uniquely, to a map
on process terms, $\widehat{s} : \Proc \rightarrow \Proc$ by the
following equations.

\begin{mathpar}
  (0) \psubstp{Q}{P} := 0 \\
  (R \juxtap S) \psubstp{Q}{P}
  :=    
  (R)\psubstp{Q}{P} \juxtap (S) \psubstp{Q}{P} \\
  (x?(y).R) \psubstp{Q}{P}    
  :=    
  (x)\substp{Q}{P} (z)\concat( (R \psubstn{z}{y}) \psubstp{Q}{P} ) \\
  (\lift{x}{R}) \psubstp{Q}{P}  
  :=
  \lift{(x)\substp{Q}{P}}{ R \psubstp{Q}{P} } \\
%   (\dropn{x})  \psubstp{Q}{P}       
%   := 
%   \left\{ 
%     \begin{array}{ccc} 
%       \dropn{\quotep{Q}} & & x \nameeq \quotep{P} \\
%       \dropn{x} & & otherwise \\
%     \end{array}
%   \right. 
  (\dropn{x})  \psubstp{Q}{P}       
  := 
  \left\{ 
    \begin{array}{ccc} 
      Q & & x \nameeq \quotep{P} \\
      \dropn{x} & & otherwise \\
    \end{array}
  \right.
\end{mathpar}
 

where

\begin{eqnarray}
  (x)\id{\{} \lpquote Q \rpquote / \lpquote P \rpquote \id{\}}            = 
  \left\{ 
    \begin{array}{ccc}
      \lpquote Q \rpquote & & x \nameeq \lpquote P \rpquote \\
      x & & otherwise \\
    \end{array}
  \right. \nonumber
\end{eqnarray}

and $z$ is chosen distinct from $\quotep{P}$, $\quotep{Q}$, the free
names in $Q$, and all the names in $R$. Our $\alpha$-equivalence will
be built in the standard way from this substitution.

\begin{remark}\label{rem:no_self_referential_names}
  One consequence of these definitions is that $\forall P. \quotep{P}
  \not\in \freenames{P}$.
\end{remark}

\subsection{ Dynamic quote: an example }

Anticipating something of what's to come, consider applying the
substitution, $\widehat{\id{\{}u / z \id{\}}}$, to the following pair
of processes, $\lift{w}{y!(z)}$ and $w[ \lpquote y!(z) \rpquote ]$.

\begin{eqnarray}
	\lift{w}{y!(z)}\widehat{\id{\{}u / z \id{\}}}
		& = &
		\lift{w}{y!(u)} \nonumber\\
	w[ \lpquote y!(z) \rpquote ] \widehat{ \id{\{}u / z \id{\}} }
		& = &
		w[ \lpquote y!(z) \rpquote ] \nonumber
\end{eqnarray}

Because the body of the process between quotes is impervious to
substitution, we get radically different answers. In fact, by
examining the first process in an input context,
e.g. $x?(z).\lift{w}{y!(z)}$, we see that the process under the lift
operator may be shaped by prefixed inputs binding a name inside it. In
this sense, the lift operator will be seen as a way to dynamically
construct processes before reifying them as names.

Finally equipped with these standard features we can present the
dynamics of the calculus.

\subsubsection{Operational semantics} 

Finally, we introduce the computational dynamics. What marks these
algebras as distinct from other more traditionally studied algebraic
structures, e.g. vector spaces or polynomial rings, is the manner in
which dynamics is captured. In traditional structures, dynamics is typically
expressed through morphisms between such structures, as in linear maps
between vector spaces or morphisms between rings. In algebras
associated with the semantics of computation, the dynamics is
expressed as part of the algebraic structure itself, through a
reduction reduction relation typically denoted by $\red$. Below, we
give a recursive presentation of this relation for the calculus used
in the encoding.

$\red \subseteq \pi \times \pi$
$\red : \pi \to \mathcal{P}(\pi)$

\begin{mathpar}
  \inferrule* [lab=Comm] { \textsf{match}( x_{src}, x_{trgt} ) } { x_{trgt}?(y)P \; | \; x_{src}!\langle {Q} \rangle \red P\{\quotep{Q}/y}\} }
  \and \\
  \inferrule* [lab=Par] {{P} \red {P}'} {{{P} | {Q}} \red {{P}' | {Q}}}
  \and
  \inferrule* [lab=Equiv]{{{P} \scong {P}'} \andalso {{P}' \red {Q}'} \andalso {{Q}' \scong {Q}}}{{P} \red {Q}}
\end{mathpar}

\begin{eqnarray*}
  match_{\equiv} (\quotep{P},\quotep{Q}) & := & P \equiv Q \\
  match_{\dagger}(\quotep{P},\quotep{Q}) & := & \forall R. P|Q \red^{*} R => R \red^{*} 0 \\
  match_{K}(\quotep{P},\quotep{Q}) & := & K \mbox{ for some context } K
\end{eqnarray*}

$u?(x)P | u!\langle Q \rangle \red P\{\quotep{Q}/x\}$

%We write $\wred$ for $\red^*$, and $P\red$ if $\exists Q $ such that $ P \red Q$.
We write $P\red$ if $\exists Q $ such that $ P \red Q$ and $P\not\red$, otherwise.

\section{Replication}

As mentioned before, it is known that replication (and hence
recursion) can be implemented in a higher-order process algebra
\cite{SangiorgiWalker}. As our first example of calculation with the
machinery thus far presented we give the construction explicitly in
the {\rhoc}.

\begin{eqnarray}
	D_{x} & := & \prefix{x}{y}{(\binpar{\outputp{x}{y}}{@{y}})} \nonumber\\
	\bangp_{x}{P} & := & \binpar{{x}!\langle{\binpar{D_{x}}{P}}\rangle}{D_{x}} \nonumber
\end{eqnarray}

\begin{eqnarray}
	\bangp_{x}{P} & & \nonumber\\
	=
	& {x}!\langle{(\prefix{x}{y}{(\outputp{x}{y} | @{y})) | P}}\rangle 
	      | \prefix{x}{y}{(\outputp{x}{y} | @{y})} & \nonumber\\
	\red
	& (\outputp{x}{y} | @{y})\substn{\quotep{(\prefix{x}{y}{(@{y} | \outputp{x}{y})) | P}}}{y} & \nonumber\\
	=
	& \outputp{x}{\quotep{(\prefix{x}{y}{(\outputp{x}{y} | @{y})) | P}}}
	  | {(\prefix{x}{y}{(\outputp{x}{y} | @{y})) | P}} & \nonumber\\
	\red
	& \ldots & \nonumber\\
	\red^*
	& P | P | \ldots & \nonumber
\end{eqnarray}

Of course, this encoding, as an implementation, runs away, unfolding
$\bangp{P}$ eagerly. A lazier and more implementable replication
operator, restricted to input-guarded processes, may be obtained as follows.

\begin{eqnarray}
\bangp{\prefix{u}{v}{P}} 
	:= 
	\binpar{\lift{x}{\prefix{u}{v}{(\binpar{D(x)}{P})}}}{D(x)} \nonumber
\end{eqnarray}

\begin{remark}
  Note that the lazier definition still does not deal with summation
  or mixed summation (i.e. sums over input and output). The reader is
  invited to construct definitions of replication that deal with these
  features. 

  Further, the definitions are parameterized in a name, $x$. Can you,
  gentle reader, make a definition that eliminates this parameter and
  guarantees no accidental interaction between the replication
  machinery and the process being replicated -- i.e. no accidental
  sharing of names used by the process to get its work done and the
  name(s) used by the replication to effect copying. This latter
  revision of the definition of replication is crucial to obtaining
  the expected identity $!!P \sim !P$.
\end{remark}

\begin{remark}\label{rem:paradoxical_combinator}
  The reader familiar with the lambda calculus will have noticed the
  similarity between $D$ and the paradoxical combinator.

  [Ed. note: the existence of this seems to suggest we have to be more
  restrictive on the set of processes and names we admit if we are to
  support no-cloning.]
\end{remark}

\subsubsection{Bisimulation}

The computational dynamics gives rise to another kind of equivalence,
the equivalence of computational behavior. As previously mentioned
this is typically captured \emph{via} some form of bisimulation.

% The notion we use in this paper is weak barbed bisimulation
% \cite{milner91polyadicpi}.

The notion we use in this paper is derived from weak barbed
bisimulation \cite{milner91polyadicpi}. 

\begin{definition}
An \emph{observation relation}, $\downarrow_{\mathcal N}$, over a set
of names, $\mathcal N$, is the smallest relation satisfying the rules
below.

\infrule[Out-barb]{y \in {\mathcal N}, \; x \nameeq y}
		  {\outputp{x}{v} \downarrow_{\mathcal N} x}
\infrule[Par-barb]{\mbox{$P\downarrow_{\mathcal N} x$ or $Q\downarrow_{\mathcal N} x$}}
		  {\binpar{P}{Q} \downarrow_{\mathcal N} x}

We write $P \Downarrow_{\mathcal N} x$ if there is $Q$ such that 
$P \wred Q$ and $Q \downarrow_{\mathcal N} x$.
\end{definition}

\begin{definition}
%\label{def.bbisim}
An  ${\mathcal N}$-\emph{barbed bisimulation} over a set of names, ${\mathcal N}$, is a symmetric binary relation 
${\mathcal S}_{\mathcal N}$ between agents such that $P\rel{S}_{\mathcal N}Q$ implies:
\begin{enumerate}
\item If $P \red P'$ then $Q \wred Q'$ and $P'\rel{S}_{\mathcal N} Q'$.
\item If $P\downarrow_{\mathcal N} x$, then $Q\Downarrow_{\mathcal N} x$.
\end{enumerate}
$P$ is ${\mathcal N}$-barbed bisimilar to $Q$, written
$P \wbbisim_{\mathcal N} Q$, if $P \rel{S}_{\mathcal N} Q$ for some ${\mathcal N}$-barbed bisimulation ${\mathcal S}_{\mathcal N}$.
\end{definition}

$\mathcal{R} \subseteq \pi \times \pi$

$P \mathcal{R} Q => \forall P'. P \red P' \Rightarrow \exists Q'. Q \red Q', P' \mathcal{R} Q'$

$P \vdash x \Rightarrow Q \vdash x$

\begin{mathpar}
  \inferrule*[lab=Out-barb]{x \nameeq y}{{y}!\langle{Q}\rangle \vdash x}
  \and
  \inferrule*[lab=Par-barb]{\mbox{$P\vdash x$ or $Q\vdash x$}}{\binpar{P}{Q} \vdash x}
\end{mathpar}

\subsubsection{Contexts}

One of the principle advantages of computational calculi like the
$\pi$-calculus is a well-defined notion of context,
contextual-equivalence and a correlation between
contextual-equivalence and notions of bisimulation. The notion of
context allows the decomposition of a process into (sub-)process and
its syntactic environment, its context. Thus, a context may be
thought of as a process with a ``hole'' (written $\Box$) in it. The
application of a context $M$ to a process $P$, written $M[P]$, is
tantamount to filling the hole in $M$ with $P$. In this paper we do
not need the full weight of this theory, but do make use of the notion
of context in the proof the main theorem. 

\begin{mathpar}
  \inferrule* [lab=summation] {} {{M_{M},M_{N}} \bc \Box \;|\; x.M_{A} \;|\; M_{M}+M_{N}}
  \and
  \inferrule* [lab=agent] {} {{M_{A}} \bc (\vec{x})M_{P} \;| \; \clift{P_0,\ldots,M_{P},\ldots,P_N}}
  \and \\
  \inferrule* [lab=process] {} {{M_{P}} \bc M_{N} \;| \;P|M_{P} }
\end{mathpar} 

\begin{mathpar}
  \inferrule* [lab=sychronization] {} {M_{N} \bc \Box \;|\; x?M_{F} \;|\; x!M_{C}}
  \and
  \inferrule* [lab=abstraction] {} {{M_{F}} \bc (x)M_{P} }
  \and
  \inferrule* [lab=concretion] {} {{M_{C}} \bc \langle M_{P} \rangle }
  \and \\
  \inferrule* [lab=process] {} {{M_{P}} \bc M_{N} \;| \;P|M_{P} }
\end{mathpar}

\begin{definition}[contextual application] Given a context $M$, and
  process $P$, we define the \emph{contextual application}, $M[P] :=
  M\{P/\Box\}$. That is, the contextual application of M to P is the
  substitution of $P$ for $\Box$ in $M$.
\end{definition}

$\meaningof{-} : L \to \mathcal{P}(\pi)$

\begin{mathpar}
  \inferrule* [lab=collection] {} {\meaningof{true} = \pi, \and \meaningof{~E} = \pi \setminus \meaningof{E}, \and \meaningof{E_{1} \& E_{2}} = \meaningof{E_{1}} \cap \meaningof{E_{2}}}
\end{mathpar}

\begin{mathpar}
  \inferrule* [lab=structure] {} {\meaningof{0} = \{ P \in \pi | P \equiv 0 \}, \and \\ \meaningof{E_1 | E_2} = \{ P \in \pi | P \equiv P_{1} | P_{2}, P_{1} \in \meaningof{E_{1}}, P_{2} \in \meaningof{E_2}\} }
\end{mathpar}

\begin{mathpar}
 \inferrule* [lab=behavior] {} {\meaningof{\langle a?b \rangle E} = \{ P \in \pi | P \equiv Q | u?(y)P', \\ \and \\\\ \and \\ \;\;\; u \in \meaningof{a}, \forall z.P'\{z/y\} \in \meaningof{E\{z/b\}}\}, \and \\ \meaningof{a!E} = \{ P \in \pi | P \equiv Q | x!\langle P' \rangle, x \in \meaningof{a} P' \in \meaningof{E}\} }
\end{mathpar}

\begin{mathpar}
 \inferrule* [lab=nominal] {} {\meaningof{\quotep{E}} = \{ \quotep{P} \in \quotep{\pi} | P \in \meaningof{E} \}, \and \meaningof{\quotep{P}} = \{ \quotep{Q} \in \quotep{\pi} | P \equiv Q \} \and \\ \meaningof{@\quotep{E}} = \{ P \in \pi | P \equiv @x, x \in \meaningof{E} \}}
\end{mathpar}

\begin{eqnarray*}
  \\
  \meaningof{-} : TS \to ST
\end{eqnarray*}

\begin{eqnarray*}
  \\
  L : TS \to ST
\end{eqnarray*}

\begin{eqnarray*}
  \\
  P \models E \iff P \in \meaningof{E}
\end{eqnarray*}

\begin{eqnarray*}
  P \approx_{L} Q \iff \forall E \in L. P \models E \iff Q \models E
\end{eqnarray*}

\begin{eqnarray*}
  P \approx_{K} Q
\end{eqnarray*}

\begin{eqnarray*}
  P \approx Q
\end{eqnarray*}

$\approx_{K} = \approx = \approx_{L}$

\subsubsection{Contextual duality}

Note that contexts extend the quotation operation to a family of
operations from processes to names. Given a context, $M$, we can
define a \emph{nominal context}, $\quotep{M}$ by $\quotep{M}[P] :=
\quotep{M[P]}$. To foreshadow what is to come we observe that these
operations enjoy a duality with processes very much like the duality
between vectors and maps from vectors to scalars.

Further, because the calculus is essentially higher-order, we have a
correspondence between contexts and processes. More specifically,
given a name $x$ and a context $M$ we can construct $M^{*}_{x}$ such
that 

\begin{mathpar}
  M^{*}_{x} | \lift{x}{P} \red M[P]
\end{mathpar}

namely,

\begin{mathpar}
  M^{*}_{x} := x?(u).M[\dropn{u}]
\end{mathpar}

The dependence of $M^{*}_{x}$ on a name makes it an abstraction, 

\begin{mathpar}
  M^{*} := (x)x?(u).M[\dropn{u}]
\end{mathpar}

\subsection{Additional notation}

It will sometimes be convenient to denote the process a name
quotes. We already have the notation $x = \quotep{P}$, but it will be
convenient to introduce an alternate notation, $\procn{x}$, when we
want to emphasize the connection to the use of the name. Note that, by
virtue of name equivalence, $\quotep{\procn{x}} \nameeq x$; so, the
notation is consistent with previous definitions.

Further, because names have structure it is possible to effect
substitutions on the basis of that structure. This means we need to
upgrade our notation for substitutions, which we accomplish by
adapting comprehension notation. Thus,

\begin{mathpar}
  P\{ y / x : x \in S \}
\end{mathpar}

is interpreted to mean the process derived from P by replacing (in a
capture-avoiding manner) each occurrence of $x$ in $S$ by $y$. For example,

\begin{mathpar}
  P\{ \quotep{\procn{x}|\procn{x}} / x : x \in \freenames{P} \}
\end{mathpar}

will replace each (occurrence) of a free name $x$ in $P$ by
$\quotep{\procn{x}|\procn{x}}$.

Also, we will avail ourselves of the notation $x^{L}$ and $x^{R}$ to
denote injections of a name into disjoint copies of the name
space. There are numerous ways to accomplish this. One example can be
found in \cite{MeredithR05}. This notation overloads to vectors of
names: $\vec{x}^{\pi} := (x_{i}^{\pi} \; : \; 0 \leq i < |\vec{x}| )$ where $\pi \in \{L,R\}$.

We also use $P^{\Box} := P|\Box$.

In \cite{MeredithR05} an interpretation of the new operator is
given. It turns out that there are several possible interpretations
all enjoying the requisite algebraic properties of the operator (see
\cite{milner91polyadicpi}). We will therefore make liberal use of
$(\nu\; \vec{x})P$.

% subsection the_syntax_and_semantics_of_the_notation_system (end)   

\input{qm2pi.qmops} 

\input{qm2pi.sterngerlach} 

\input{qm2pi.metric} 

% section concurrent_process_calculi (end)

%\input{qm2pi.proofsketch}

% section proof sketch (end)

%\input{qm2pi.slviaknots} 

% section spatial logic via knots (end)

\input{qm2pi.conclusion}

% section conclusion (end)

%\input{qm2pi.dtcodes} 

% section wiring algorithm (end)

\input{qm2pi.ack} 

% section acknowledgments (end)

\newpage


\bibliographystyle{plain}   
\bibliography{../../biblios/main.bib}

\input{qm2pi.rhodetails}

\end{document}



\end{document}



% section front matter (end)

\section{Introduction}\label{sec:introduction} % (fold)
In this draft of the material i am going to have to dispense with the
usual writing conventions adopted in papers on these topics. i'm going
to have adopt whatever tone i need at the time i'm writing up the
calculations. Sometimes this may be very conversational; others it may
be the barest mathematical grunts; others still it may be that i have
lifted text from one of my other papers because the exposition of some
point was better said there. i hope that my readers are not unduly put
out by this decision. i'm not doing this to flout convention or be
rebellious. i find these calculations very technically challenging. To
keep everything going technically, something has to give; i have to
let go of some cognitive burden. So, the academic writing style --
with all of its trade-offs in terms of facilitating technical
communication -- is what i'm letting go of. Perhaps subsequent drafts
can be tightened and polished, but for now, i'm going to speak as if
we were sitting together in a coffee shop with a laptop, wifi and a
pad of paper and a pencil.

So, here's what i have to say. We -- you and i, comfortably ensconced
in our coffee shop and well-equipped with our tools -- can realize and
carry out the calculations of quantum mechanics over a very different
formal theory of dynamics, a formal theory of dynamics that
corresponds to a theory of concurrent computation with
\emph{reflection}. It has the advantage that the underlying theory is
already `quantized', but supports analogues all of the continuuous
operations. Strikingly, this underlying theory has recently been
connected with a notion of metric that we can show, by calculating
together, coincides with the metric induced by the inner product.

There are a lot of reasons why you might be interested in seeing
calculations of this form. Here's why i'm interested. For the past
several centuries there has been no competitor to the ``Newtonian''
account of dynamics. As a result the predominant share of accounts of
dynamical systems and situations have had to be formulated in terms of
the Newtonian machinery. i view this as an intellectually dangerous
position to occupy. Everything, despite it's intrinsic shape, turns
into a nail to be hit with this hammer. Recently, however, the theory
of computation has matured to the point where we have candidates for
theories of dynamics that offer very different perspective on
reasoning about dynamical systems and situations. Testing these
candidates against very successful accounts of dynamical situations,
like quantum mechanics, is going to give us some sense of how mature
they are and some measure of the quality of these accounts of
dynamics.

\subsection{Summary of contributions and outline of paper}

So, we're going to develop an interpretation of the operations of
quantum mechanics normally interpreted by Hilbert spaces and
operators. We're going to do this over a theory of computation. Note
that this is very different than the usual quantum computation program
which develops notions of computation over quantum mechanics. Rather,
we are developing a story that aligns with Wheeler's slogan: It from
Bit. To do this we will first provide an account of the theory of
computation at play here. Then we will dive into a calculation-driven
interpretation of the operations of quantum mechanics.

The reason we take this approach is that -- until very recently --
there hasn't been an axiomatic account of quantum mechanics. As a
result there has been no sharp delineation of the mathematical theory
supporting interpretation of the physical theory and the physical
theory, itself. So, ambient features of the maths are free to be
exploited (or supressed) without a real accounting of their physical
relevance. There is no sharp statement ``here's the physical theory''
qua \emph{theory} and ``here's the mathematical interpretation''
enabling a judgment of how faithful the interpretation is -- apart
from experimental observation. When there is an axiomatic account we
can judge how well a given mathematical formalism supports an
interpretation of the axioms, independent of
experimentation. Likewise, we can judge how well we have captured our
physical evidence and experience with our axiomatics, independent of
any specific mathematical implementation, with accidental detail that
may or may not have physical significance. 

In lieu of a fully fleshed out and vetted axiomatic account of quantum
mechanics, interpreting the operational notions in service of modeling
physical systems will have to suffice. In other words, we are not in
the business of providing a model of Hilbert spaces and operators. We
are in the business of providing a model of quantum mechanics because
we are motivated by testing our notions of dynamics against physical
theory; and, the predictive calculations of the physical theory must
serve as the best formulation -- shy of a fully fleshed out axiomatic
account -- of the physical theory itself (as they have for scientific
theories since time immemorial). Put another way, despite a
whole-hearted commitment to an It-from-Bit ontology, we are firmly
aligned with the shut-up-and-calculate camp as the best way to obtain
results either from the physical perspective or as a quality assurance
measure of our fledgling theory of dynamics.

In detail, we present a reflective process calculus. Then we develop
intuitive correspondences between the notions available in this
calculus and the usual physical notions supporting quantum mechanical
calculations. Thus, 

\begin{table}[htp]
  \center{
    \fbox{
      \begin{tabular}{c|c}
        quantum mechanics & process calculus \\
        \hline
        scalar & name \\
        state vector & process \\
        dual & contextual duals \\
        matrix & formal sums of process-context-dual pairs \\
        orthogonality & process annihilation \\
        inner product & execution-formula + quoting
      \end{tabular}
    }
  }
  \caption{QM - process calculi correspondences}
\end{table}

Then we tighten up these intuitions to operational definitions. We
employ the Dirac notation as the best proxy we can find for an
abstract syntax of the quantum mechanical notions. The definitions we
develop put us in contact with equational constraints coming from the
theory that we demonstrate the definitions and calculations satisfy.

This puts us in a position to shut up and calculate for the
Stern-Gerlach experimental set up, showing how these predictive
calculations become calculations on processes in our theory of a
reflective process calculus.

Penultimately, we demonstrate that the notion of metric coming from
the inner product coincides with the notion of metric available from
the theory of bisimulation. This demonstration gives us the right to
think of space as arising from behavior. Finally, we consider where we
might go from the new vantage point we have obtained.

% section introduction (end) 
 
% section introduction (end)

% \documentclass[12pt]{llncs}
%\documentclass{jktr}

\usepackage[pdftex]{hyperref}                   
\usepackage {listings}
\usepackage {mathpartir}
\usepackage{bcprules}
%\usepackage{listings}
                       
\usepackage{graphicx} 
%\usepackage[margins=2.5cm,nohead,nofoot]{geometry}
%\usepackage{geometry}
\usepackage{amsfonts}
\usepackage{amstext}
\usepackage{latexsym}
\usepackage{amssymb}
\usepackage{color}


%\include{myPreamble}
\documentclass[12pt]{llncs}
%\documentclass{jktr}

\usepackage[pdftex]{hyperref}                   
\usepackage {listings}
\usepackage {mathpartir}
\usepackage{bcprules}
%\usepackage{listings}
                       
\usepackage{graphicx} 
%\usepackage[margins=2.5cm,nohead,nofoot]{geometry}
%\usepackage{geometry}
\usepackage{amsfonts}
\usepackage{amstext}
\usepackage{latexsym}
\usepackage{amssymb}
\usepackage{color}


%\include{myPreamble}
\include{qm2pi.local} 

%\ifpdf
%\usepackage[pdftex]{graphicx}
%\else
%\usepackage{graphicx}
%\fi

 % \ifpdf
%  \usepackage{pdfsync}
%  \if


%\title{Brief Article}
%\author{David F. Snyder}
%\author{L.G. Meredith}

%\address{Dept. of Math., Texas State University--San Marcos, San Marcos, TX 78666}
       
\pagestyle{empty}


\begin{document}

\lstset{language=[Objective]Caml,frame=shadowbox}

\input{qm2pi.front}

% section front matter (end)

\input{qm2pi.intro} 
 
% section introduction (end)

% \input{qm2pi.knotations} 

% section notation (end)

\input{qm2pi.process.calculi} 

% section concurrent_process_calculi_and_spatial_logics_ (end)
    
%\input{qm2pi.knots2pi} 

%\input{qm2pi.trefoil} 

%\input{qm2pi.mainthm} 

% subsection basic_interpretation (end)

%\input{qm2pi.rho.presentation} 
\subsection{The syntax and semantics of the notation system}\label{sub:the_syntax_and_semantics_of_the_notation_system} % (fold)

We now summarize a technical presentation of the calculus that
embodies our theory of dynamics. The typical presentation of such a
calculus follows the style of giving generators and relations on
them. The grammar, below, describing term constructors, freely
generates the set of processes, $\Proc$. This set is then quotiented
by a relation known as structural congruence and it is over this set
that the notion of dynamics is expressed. This presentation is
essentially that of \cite{MeredithR05} with the addition of
polyadicity and summation. For readability we have relegated some of
the technical subtleties to an appendix.

\subsubsection{Process grammar}\label{subsub:process_grammar}

\begin{mathpar}
  \inferrule* [lab=synchronization] {} {{M} \bc \pzero \;|\; x?F \;|\; x!C }
  \and
  \inferrule* [lab=abstraction] {} {{F} \bc (x)P}
  \and
  \inferrule* [lab=concretion] {} {{C} \bc \langle Q \rangle}
  \and
  \inferrule* [lab=process] {} {{P,Q} \bc M \;| \;P|Q \;|\; @{x}}
  \and
  \inferrule* [lab=name] {} {{x} \bc \quotep{P}}
\end{mathpar} 

Note that $\vec{x}$ (resp. $\vec{P}$) denotes a vector of names
(resp. processes) of length $|\vec{x}|$ (resp. $|\vec{P}|$). We adopt
the following useful abbreviations.

\begin{mathpar}
   x?(\vec{y}).P := x.(\vec{y})P \and  x\clift{\vec{P}} := x.\clift{\vec{P}}
   \and x!(y) := \lift{x}{\dropn{y}}
   \and \Pi_{i=0}^{n-1}P_i := P_0 | \ldots | P_{n-1}
\end{mathpar}

\subsubsection{Structural congruence}

\paragraph{Free and bound names and alpha-equivalence.} At the
core of structural equivalence is alpha-equivalence which identifies
process that are the same up to a change of variable. Formally, we
recognize the distinction between free and bound names. The free names
of a process, $\freenames{P}$, may be calculated recursively as
follows:

\begin{mathpar}
\freenames{\pzero} := \emptyset
  \and \\
  \freenames{x?(y).P} := \{ x \} \cup (\freenames{P} \setminus \{ y \})
  \and 
  \freenames{x!\langle P \rangle} := \{ x \} \cup \{ P \} 
  \and \\
  \freenames{P|Q} := \freenames{P} \cup \freenames{Q}
  \and \\
  \freenames{@{x}} := \{ x \}
\end{mathpar}

$\pi$
$\quotep{\pi}$

$\freenames{-} : \pi \to \mathcal{P}(\quotep{\pi})$

\begin{eqnarray*}
  \freenames{\pzero} & := & \emptyset \\
  \freenames{x?(y).P} & := & \{ x \} \cup (\freenames{P} \setminus \{ y \}) \\
  \freenames{x!\langle P \rangle} & := & \{ x \} \cup \{ P \} \\
  \freenames{P|Q} & := & \freenames{P} \cup \freenames{Q} \\
  \freenames{\dropn{x}} & := & \{ x \}
\end{eqnarray*}

The bound names of a process, $\boundnames{P}$, are those names occurring in $P$
that are not free. For example, in $x?(y).0$, the name $x$ is free, while $y$ is bound.

\begin{mathpar}
  \inferrule* [lab=monoidal-laws] {} { P|Q \equiv Q|P \and P|0 \equiv P \and P|(Q|R) \equiv (P|Q)|R }
\end{mathpar}

\begin{mathpar}
  \inferrule* [lab=alpha-equivalence] {} { (x)P \equiv (y)P\{y/x\} \and y \not\in \freenames{P} }
\end{mathpar}

\begin{definition}
Then two processes, $P,Q$, are alpha-equivalent if $P = Q\{\vec{y}/\vec{x}\}$ for
some $\vec{x} \in \boundnames{Q},\vec{y} \in \boundnames{P}$, where $Q\{\vec{y}/\vec{x}\}$
denotes the capture-avoiding substitution of $\vec{y}$ for $\vec{x}$ in $Q$.
\end{definition}

\begin{definition}
  The {\em structural congruence} \cite{SangiorgiWalker} , $\equiv$,
  between processes is the least congruence containing
  alpha-equivalence, satisfying the abelian monoid laws
  (associativity, commutativity and $\pzero$ as identity) for parallel
  composition $|$ and for summation $+$.
\end{definition}

\subsection{Name equivalence}

We take name equivalence, written $\nameeq$, to be the smallest
equivalence relation generated by the following rules.

\begin{mathpar}
\inferrule*[lab=Quote-drop]
{ }
{ \quotep{@{x}} \nameeq x }

\inferrule*[lab=Struct-equiv]
{ P \scong Q }
{ \quotep{P} \nameeq \quotep{Q} }
\end{mathpar}

The astute reader will have noticed that the mutual recursion of names
and processes imposes a mutual recursion on alpha-equivalence and
structural equivalence via name-equivalence. Fortunately, all of this
works out pleasantly and we may calculate in the natural way, free of
concern. The reader interested in the details is referred to the
appendix \ref{appendix:rho_details}.

\subsection{Substitution}

We use $\Proc$ for the set of processes, $\QProc$ for the set of
names, and $\id{\{}\vec{y} / \vec{x} \id{\}}$ to denote partial maps,
$s : \QProc \rightarrow \QProc$. A map, $s$ lifts, uniquely, to a map
on process terms, $\widehat{s} : \Proc \rightarrow \Proc$ by the
following equations.

\begin{mathpar}
  (0) \psubstp{Q}{P} := 0 \\
  (R \juxtap S) \psubstp{Q}{P}
  :=    
  (R)\psubstp{Q}{P} \juxtap (S) \psubstp{Q}{P} \\
  (x?(y).R) \psubstp{Q}{P}    
  :=    
  (x)\substp{Q}{P} (z)\concat( (R \psubstn{z}{y}) \psubstp{Q}{P} ) \\
  (\lift{x}{R}) \psubstp{Q}{P}  
  :=
  \lift{(x)\substp{Q}{P}}{ R \psubstp{Q}{P} } \\
%   (\dropn{x})  \psubstp{Q}{P}       
%   := 
%   \left\{ 
%     \begin{array}{ccc} 
%       \dropn{\quotep{Q}} & & x \nameeq \quotep{P} \\
%       \dropn{x} & & otherwise \\
%     \end{array}
%   \right. 
  (\dropn{x})  \psubstp{Q}{P}       
  := 
  \left\{ 
    \begin{array}{ccc} 
      Q & & x \nameeq \quotep{P} \\
      \dropn{x} & & otherwise \\
    \end{array}
  \right.
\end{mathpar}
 

where

\begin{eqnarray}
  (x)\id{\{} \lpquote Q \rpquote / \lpquote P \rpquote \id{\}}            = 
  \left\{ 
    \begin{array}{ccc}
      \lpquote Q \rpquote & & x \nameeq \lpquote P \rpquote \\
      x & & otherwise \\
    \end{array}
  \right. \nonumber
\end{eqnarray}

and $z$ is chosen distinct from $\quotep{P}$, $\quotep{Q}$, the free
names in $Q$, and all the names in $R$. Our $\alpha$-equivalence will
be built in the standard way from this substitution.

\begin{remark}\label{rem:no_self_referential_names}
  One consequence of these definitions is that $\forall P. \quotep{P}
  \not\in \freenames{P}$.
\end{remark}

\subsection{ Dynamic quote: an example }

Anticipating something of what's to come, consider applying the
substitution, $\widehat{\id{\{}u / z \id{\}}}$, to the following pair
of processes, $\lift{w}{y!(z)}$ and $w[ \lpquote y!(z) \rpquote ]$.

\begin{eqnarray}
	\lift{w}{y!(z)}\widehat{\id{\{}u / z \id{\}}}
		& = &
		\lift{w}{y!(u)} \nonumber\\
	w[ \lpquote y!(z) \rpquote ] \widehat{ \id{\{}u / z \id{\}} }
		& = &
		w[ \lpquote y!(z) \rpquote ] \nonumber
\end{eqnarray}

Because the body of the process between quotes is impervious to
substitution, we get radically different answers. In fact, by
examining the first process in an input context,
e.g. $x?(z).\lift{w}{y!(z)}$, we see that the process under the lift
operator may be shaped by prefixed inputs binding a name inside it. In
this sense, the lift operator will be seen as a way to dynamically
construct processes before reifying them as names.

Finally equipped with these standard features we can present the
dynamics of the calculus.

\subsubsection{Operational semantics} 

Finally, we introduce the computational dynamics. What marks these
algebras as distinct from other more traditionally studied algebraic
structures, e.g. vector spaces or polynomial rings, is the manner in
which dynamics is captured. In traditional structures, dynamics is typically
expressed through morphisms between such structures, as in linear maps
between vector spaces or morphisms between rings. In algebras
associated with the semantics of computation, the dynamics is
expressed as part of the algebraic structure itself, through a
reduction reduction relation typically denoted by $\red$. Below, we
give a recursive presentation of this relation for the calculus used
in the encoding.

$\red \subseteq \pi \times \pi$
$\red : \pi \to \mathcal{P}(\pi)$

\begin{mathpar}
  \inferrule* [lab=Comm] { \textsf{match}( x_{src}, x_{trgt} ) } { x_{trgt}?(y)P \; | \; x_{src}!\langle {Q} \rangle \red P\{\quotep{Q}/y}\} }
  \and \\
  \inferrule* [lab=Par] {{P} \red {P}'} {{{P} | {Q}} \red {{P}' | {Q}}}
  \and
  \inferrule* [lab=Equiv]{{{P} \scong {P}'} \andalso {{P}' \red {Q}'} \andalso {{Q}' \scong {Q}}}{{P} \red {Q}}
\end{mathpar}

\begin{eqnarray*}
  match_{\equiv} (\quotep{P},\quotep{Q}) & := & P \equiv Q \\
  match_{\dagger}(\quotep{P},\quotep{Q}) & := & \forall R. P|Q \red^{*} R => R \red^{*} 0 \\
  match_{K}(\quotep{P},\quotep{Q}) & := & K \mbox{ for some context } K
\end{eqnarray*}

$u?(x)P | u!\langle Q \rangle \red P\{\quotep{Q}/x\}$

%We write $\wred$ for $\red^*$, and $P\red$ if $\exists Q $ such that $ P \red Q$.
We write $P\red$ if $\exists Q $ such that $ P \red Q$ and $P\not\red$, otherwise.

\section{Replication}

As mentioned before, it is known that replication (and hence
recursion) can be implemented in a higher-order process algebra
\cite{SangiorgiWalker}. As our first example of calculation with the
machinery thus far presented we give the construction explicitly in
the {\rhoc}.

\begin{eqnarray}
	D_{x} & := & \prefix{x}{y}{(\binpar{\outputp{x}{y}}{@{y}})} \nonumber\\
	\bangp_{x}{P} & := & \binpar{{x}!\langle{\binpar{D_{x}}{P}}\rangle}{D_{x}} \nonumber
\end{eqnarray}

\begin{eqnarray}
	\bangp_{x}{P} & & \nonumber\\
	=
	& {x}!\langle{(\prefix{x}{y}{(\outputp{x}{y} | @{y})) | P}}\rangle 
	      | \prefix{x}{y}{(\outputp{x}{y} | @{y})} & \nonumber\\
	\red
	& (\outputp{x}{y} | @{y})\substn{\quotep{(\prefix{x}{y}{(@{y} | \outputp{x}{y})) | P}}}{y} & \nonumber\\
	=
	& \outputp{x}{\quotep{(\prefix{x}{y}{(\outputp{x}{y} | @{y})) | P}}}
	  | {(\prefix{x}{y}{(\outputp{x}{y} | @{y})) | P}} & \nonumber\\
	\red
	& \ldots & \nonumber\\
	\red^*
	& P | P | \ldots & \nonumber
\end{eqnarray}

Of course, this encoding, as an implementation, runs away, unfolding
$\bangp{P}$ eagerly. A lazier and more implementable replication
operator, restricted to input-guarded processes, may be obtained as follows.

\begin{eqnarray}
\bangp{\prefix{u}{v}{P}} 
	:= 
	\binpar{\lift{x}{\prefix{u}{v}{(\binpar{D(x)}{P})}}}{D(x)} \nonumber
\end{eqnarray}

\begin{remark}
  Note that the lazier definition still does not deal with summation
  or mixed summation (i.e. sums over input and output). The reader is
  invited to construct definitions of replication that deal with these
  features. 

  Further, the definitions are parameterized in a name, $x$. Can you,
  gentle reader, make a definition that eliminates this parameter and
  guarantees no accidental interaction between the replication
  machinery and the process being replicated -- i.e. no accidental
  sharing of names used by the process to get its work done and the
  name(s) used by the replication to effect copying. This latter
  revision of the definition of replication is crucial to obtaining
  the expected identity $!!P \sim !P$.
\end{remark}

\begin{remark}\label{rem:paradoxical_combinator}
  The reader familiar with the lambda calculus will have noticed the
  similarity between $D$ and the paradoxical combinator.

  [Ed. note: the existence of this seems to suggest we have to be more
  restrictive on the set of processes and names we admit if we are to
  support no-cloning.]
\end{remark}

\subsubsection{Bisimulation}

The computational dynamics gives rise to another kind of equivalence,
the equivalence of computational behavior. As previously mentioned
this is typically captured \emph{via} some form of bisimulation.

% The notion we use in this paper is weak barbed bisimulation
% \cite{milner91polyadicpi}.

The notion we use in this paper is derived from weak barbed
bisimulation \cite{milner91polyadicpi}. 

\begin{definition}
An \emph{observation relation}, $\downarrow_{\mathcal N}$, over a set
of names, $\mathcal N$, is the smallest relation satisfying the rules
below.

\infrule[Out-barb]{y \in {\mathcal N}, \; x \nameeq y}
		  {\outputp{x}{v} \downarrow_{\mathcal N} x}
\infrule[Par-barb]{\mbox{$P\downarrow_{\mathcal N} x$ or $Q\downarrow_{\mathcal N} x$}}
		  {\binpar{P}{Q} \downarrow_{\mathcal N} x}

We write $P \Downarrow_{\mathcal N} x$ if there is $Q$ such that 
$P \wred Q$ and $Q \downarrow_{\mathcal N} x$.
\end{definition}

\begin{definition}
%\label{def.bbisim}
An  ${\mathcal N}$-\emph{barbed bisimulation} over a set of names, ${\mathcal N}$, is a symmetric binary relation 
${\mathcal S}_{\mathcal N}$ between agents such that $P\rel{S}_{\mathcal N}Q$ implies:
\begin{enumerate}
\item If $P \red P'$ then $Q \wred Q'$ and $P'\rel{S}_{\mathcal N} Q'$.
\item If $P\downarrow_{\mathcal N} x$, then $Q\Downarrow_{\mathcal N} x$.
\end{enumerate}
$P$ is ${\mathcal N}$-barbed bisimilar to $Q$, written
$P \wbbisim_{\mathcal N} Q$, if $P \rel{S}_{\mathcal N} Q$ for some ${\mathcal N}$-barbed bisimulation ${\mathcal S}_{\mathcal N}$.
\end{definition}

$\mathcal{R} \subseteq \pi \times \pi$

$P \mathcal{R} Q => \forall P'. P \red P' \Rightarrow \exists Q'. Q \red Q', P' \mathcal{R} Q'$

$P \vdash x \Rightarrow Q \vdash x$

\begin{mathpar}
  \inferrule*[lab=Out-barb]{x \nameeq y}{{y}!\langle{Q}\rangle \vdash x}
  \and
  \inferrule*[lab=Par-barb]{\mbox{$P\vdash x$ or $Q\vdash x$}}{\binpar{P}{Q} \vdash x}
\end{mathpar}

\subsubsection{Contexts}

One of the principle advantages of computational calculi like the
$\pi$-calculus is a well-defined notion of context,
contextual-equivalence and a correlation between
contextual-equivalence and notions of bisimulation. The notion of
context allows the decomposition of a process into (sub-)process and
its syntactic environment, its context. Thus, a context may be
thought of as a process with a ``hole'' (written $\Box$) in it. The
application of a context $M$ to a process $P$, written $M[P]$, is
tantamount to filling the hole in $M$ with $P$. In this paper we do
not need the full weight of this theory, but do make use of the notion
of context in the proof the main theorem. 

\begin{mathpar}
  \inferrule* [lab=summation] {} {{M_{M},M_{N}} \bc \Box \;|\; x.M_{A} \;|\; M_{M}+M_{N}}
  \and
  \inferrule* [lab=agent] {} {{M_{A}} \bc (\vec{x})M_{P} \;| \; \clift{P_0,\ldots,M_{P},\ldots,P_N}}
  \and \\
  \inferrule* [lab=process] {} {{M_{P}} \bc M_{N} \;| \;P|M_{P} }
\end{mathpar} 

\begin{mathpar}
  \inferrule* [lab=sychronization] {} {M_{N} \bc \Box \;|\; x?M_{F} \;|\; x!M_{C}}
  \and
  \inferrule* [lab=abstraction] {} {{M_{F}} \bc (x)M_{P} }
  \and
  \inferrule* [lab=concretion] {} {{M_{C}} \bc \langle M_{P} \rangle }
  \and \\
  \inferrule* [lab=process] {} {{M_{P}} \bc M_{N} \;| \;P|M_{P} }
\end{mathpar}

\begin{definition}[contextual application] Given a context $M$, and
  process $P$, we define the \emph{contextual application}, $M[P] :=
  M\{P/\Box\}$. That is, the contextual application of M to P is the
  substitution of $P$ for $\Box$ in $M$.
\end{definition}

$\meaningof{-} : L \to \mathcal{P}(\pi)$

\begin{mathpar}
  \inferrule* [lab=collection] {} {\meaningof{true} = \pi, \and \meaningof{~E} = \pi \setminus \meaningof{E}, \and \meaningof{E_{1} \& E_{2}} = \meaningof{E_{1}} \cap \meaningof{E_{2}}}
\end{mathpar}

\begin{mathpar}
  \inferrule* [lab=structure] {} {\meaningof{0} = \{ P \in \pi | P \equiv 0 \}, \and \\ \meaningof{E_1 | E_2} = \{ P \in \pi | P \equiv P_{1} | P_{2}, P_{1} \in \meaningof{E_{1}}, P_{2} \in \meaningof{E_2}\} }
\end{mathpar}

\begin{mathpar}
 \inferrule* [lab=behavior] {} {\meaningof{\langle a?b \rangle E} = \{ P \in \pi | P \equiv Q | u?(y)P', \\ \and \\\\ \and \\ \;\;\; u \in \meaningof{a}, \forall z.P'\{z/y\} \in \meaningof{E\{z/b\}}\}, \and \\ \meaningof{a!E} = \{ P \in \pi | P \equiv Q | x!\langle P' \rangle, x \in \meaningof{a} P' \in \meaningof{E}\} }
\end{mathpar}

\begin{mathpar}
 \inferrule* [lab=nominal] {} {\meaningof{\quotep{E}} = \{ \quotep{P} \in \quotep{\pi} | P \in \meaningof{E} \}, \and \meaningof{\quotep{P}} = \{ \quotep{Q} \in \quotep{\pi} | P \equiv Q \} \and \\ \meaningof{@\quotep{E}} = \{ P \in \pi | P \equiv @x, x \in \meaningof{E} \}}
\end{mathpar}

\begin{eqnarray*}
  \\
  \meaningof{-} : TS \to ST
\end{eqnarray*}

\begin{eqnarray*}
  \\
  L : TS \to ST
\end{eqnarray*}

\begin{eqnarray*}
  \\
  P \models E \iff P \in \meaningof{E}
\end{eqnarray*}

\begin{eqnarray*}
  P \approx_{L} Q \iff \forall E \in L. P \models E \iff Q \models E
\end{eqnarray*}

\begin{eqnarray*}
  P \approx_{K} Q
\end{eqnarray*}

\begin{eqnarray*}
  P \approx Q
\end{eqnarray*}

$\approx_{K} = \approx = \approx_{L}$

\subsubsection{Contextual duality}

Note that contexts extend the quotation operation to a family of
operations from processes to names. Given a context, $M$, we can
define a \emph{nominal context}, $\quotep{M}$ by $\quotep{M}[P] :=
\quotep{M[P]}$. To foreshadow what is to come we observe that these
operations enjoy a duality with processes very much like the duality
between vectors and maps from vectors to scalars.

Further, because the calculus is essentially higher-order, we have a
correspondence between contexts and processes. More specifically,
given a name $x$ and a context $M$ we can construct $M^{*}_{x}$ such
that 

\begin{mathpar}
  M^{*}_{x} | \lift{x}{P} \red M[P]
\end{mathpar}

namely,

\begin{mathpar}
  M^{*}_{x} := x?(u).M[\dropn{u}]
\end{mathpar}

The dependence of $M^{*}_{x}$ on a name makes it an abstraction, 

\begin{mathpar}
  M^{*} := (x)x?(u).M[\dropn{u}]
\end{mathpar}

\subsection{Additional notation}

It will sometimes be convenient to denote the process a name
quotes. We already have the notation $x = \quotep{P}$, but it will be
convenient to introduce an alternate notation, $\procn{x}$, when we
want to emphasize the connection to the use of the name. Note that, by
virtue of name equivalence, $\quotep{\procn{x}} \nameeq x$; so, the
notation is consistent with previous definitions.

Further, because names have structure it is possible to effect
substitutions on the basis of that structure. This means we need to
upgrade our notation for substitutions, which we accomplish by
adapting comprehension notation. Thus,

\begin{mathpar}
  P\{ y / x : x \in S \}
\end{mathpar}

is interpreted to mean the process derived from P by replacing (in a
capture-avoiding manner) each occurrence of $x$ in $S$ by $y$. For example,

\begin{mathpar}
  P\{ \quotep{\procn{x}|\procn{x}} / x : x \in \freenames{P} \}
\end{mathpar}

will replace each (occurrence) of a free name $x$ in $P$ by
$\quotep{\procn{x}|\procn{x}}$.

Also, we will avail ourselves of the notation $x^{L}$ and $x^{R}$ to
denote injections of a name into disjoint copies of the name
space. There are numerous ways to accomplish this. One example can be
found in \cite{MeredithR05}. This notation overloads to vectors of
names: $\vec{x}^{\pi} := (x_{i}^{\pi} \; : \; 0 \leq i < |\vec{x}| )$ where $\pi \in \{L,R\}$.

We also use $P^{\Box} := P|\Box$.

In \cite{MeredithR05} an interpretation of the new operator is
given. It turns out that there are several possible interpretations
all enjoying the requisite algebraic properties of the operator (see
\cite{milner91polyadicpi}). We will therefore make liberal use of
$(\nu\; \vec{x})P$.

% subsection the_syntax_and_semantics_of_the_notation_system (end)   

\input{qm2pi.qmops} 

\input{qm2pi.sterngerlach} 

\input{qm2pi.metric} 

% section concurrent_process_calculi (end)

%\input{qm2pi.proofsketch}

% section proof sketch (end)

%\input{qm2pi.slviaknots} 

% section spatial logic via knots (end)

\input{qm2pi.conclusion}

% section conclusion (end)

%\input{qm2pi.dtcodes} 

% section wiring algorithm (end)

\input{qm2pi.ack} 

% section acknowledgments (end)

\newpage


\bibliographystyle{plain}   
\bibliography{../../biblios/main.bib}

\input{qm2pi.rhodetails}

\end{document}

 

%\ifpdf
%\usepackage[pdftex]{graphicx}
%\else
%\usepackage{graphicx}
%\fi

 % \ifpdf
%  \usepackage{pdfsync}
%  \if


%\title{Brief Article}
%\author{David F. Snyder}
%\author{L.G. Meredith}

%\address{Dept. of Math., Texas State University--San Marcos, San Marcos, TX 78666}
       
\pagestyle{empty}


\begin{document}

\lstset{language=[Objective]Caml,frame=shadowbox}

\documentclass[12pt]{llncs}
%\documentclass{jktr}

\usepackage[pdftex]{hyperref}                   
\usepackage {listings}
\usepackage {mathpartir}
\usepackage{bcprules}
%\usepackage{listings}
                       
\usepackage{graphicx} 
%\usepackage[margins=2.5cm,nohead,nofoot]{geometry}
%\usepackage{geometry}
\usepackage{amsfonts}
\usepackage{amstext}
\usepackage{latexsym}
\usepackage{amssymb}
\usepackage{color}


%\include{myPreamble}
\include{qm2pi.local} 

%\ifpdf
%\usepackage[pdftex]{graphicx}
%\else
%\usepackage{graphicx}
%\fi

 % \ifpdf
%  \usepackage{pdfsync}
%  \if


%\title{Brief Article}
%\author{David F. Snyder}
%\author{L.G. Meredith}

%\address{Dept. of Math., Texas State University--San Marcos, San Marcos, TX 78666}
       
\pagestyle{empty}


\begin{document}

\lstset{language=[Objective]Caml,frame=shadowbox}

\input{qm2pi.front}

% section front matter (end)

\input{qm2pi.intro} 
 
% section introduction (end)

% \input{qm2pi.knotations} 

% section notation (end)

\input{qm2pi.process.calculi} 

% section concurrent_process_calculi_and_spatial_logics_ (end)
    
%\input{qm2pi.knots2pi} 

%\input{qm2pi.trefoil} 

%\input{qm2pi.mainthm} 

% subsection basic_interpretation (end)

%\input{qm2pi.rho.presentation} 
\subsection{The syntax and semantics of the notation system}\label{sub:the_syntax_and_semantics_of_the_notation_system} % (fold)

We now summarize a technical presentation of the calculus that
embodies our theory of dynamics. The typical presentation of such a
calculus follows the style of giving generators and relations on
them. The grammar, below, describing term constructors, freely
generates the set of processes, $\Proc$. This set is then quotiented
by a relation known as structural congruence and it is over this set
that the notion of dynamics is expressed. This presentation is
essentially that of \cite{MeredithR05} with the addition of
polyadicity and summation. For readability we have relegated some of
the technical subtleties to an appendix.

\subsubsection{Process grammar}\label{subsub:process_grammar}

\begin{mathpar}
  \inferrule* [lab=synchronization] {} {{M} \bc \pzero \;|\; x?F \;|\; x!C }
  \and
  \inferrule* [lab=abstraction] {} {{F} \bc (x)P}
  \and
  \inferrule* [lab=concretion] {} {{C} \bc \langle Q \rangle}
  \and
  \inferrule* [lab=process] {} {{P,Q} \bc M \;| \;P|Q \;|\; @{x}}
  \and
  \inferrule* [lab=name] {} {{x} \bc \quotep{P}}
\end{mathpar} 

Note that $\vec{x}$ (resp. $\vec{P}$) denotes a vector of names
(resp. processes) of length $|\vec{x}|$ (resp. $|\vec{P}|$). We adopt
the following useful abbreviations.

\begin{mathpar}
   x?(\vec{y}).P := x.(\vec{y})P \and  x\clift{\vec{P}} := x.\clift{\vec{P}}
   \and x!(y) := \lift{x}{\dropn{y}}
   \and \Pi_{i=0}^{n-1}P_i := P_0 | \ldots | P_{n-1}
\end{mathpar}

\subsubsection{Structural congruence}

\paragraph{Free and bound names and alpha-equivalence.} At the
core of structural equivalence is alpha-equivalence which identifies
process that are the same up to a change of variable. Formally, we
recognize the distinction between free and bound names. The free names
of a process, $\freenames{P}$, may be calculated recursively as
follows:

\begin{mathpar}
\freenames{\pzero} := \emptyset
  \and \\
  \freenames{x?(y).P} := \{ x \} \cup (\freenames{P} \setminus \{ y \})
  \and 
  \freenames{x!\langle P \rangle} := \{ x \} \cup \{ P \} 
  \and \\
  \freenames{P|Q} := \freenames{P} \cup \freenames{Q}
  \and \\
  \freenames{@{x}} := \{ x \}
\end{mathpar}

$\pi$
$\quotep{\pi}$

$\freenames{-} : \pi \to \mathcal{P}(\quotep{\pi})$

\begin{eqnarray*}
  \freenames{\pzero} & := & \emptyset \\
  \freenames{x?(y).P} & := & \{ x \} \cup (\freenames{P} \setminus \{ y \}) \\
  \freenames{x!\langle P \rangle} & := & \{ x \} \cup \{ P \} \\
  \freenames{P|Q} & := & \freenames{P} \cup \freenames{Q} \\
  \freenames{\dropn{x}} & := & \{ x \}
\end{eqnarray*}

The bound names of a process, $\boundnames{P}$, are those names occurring in $P$
that are not free. For example, in $x?(y).0$, the name $x$ is free, while $y$ is bound.

\begin{mathpar}
  \inferrule* [lab=monoidal-laws] {} { P|Q \equiv Q|P \and P|0 \equiv P \and P|(Q|R) \equiv (P|Q)|R }
\end{mathpar}

\begin{mathpar}
  \inferrule* [lab=alpha-equivalence] {} { (x)P \equiv (y)P\{y/x\} \and y \not\in \freenames{P} }
\end{mathpar}

\begin{definition}
Then two processes, $P,Q$, are alpha-equivalent if $P = Q\{\vec{y}/\vec{x}\}$ for
some $\vec{x} \in \boundnames{Q},\vec{y} \in \boundnames{P}$, where $Q\{\vec{y}/\vec{x}\}$
denotes the capture-avoiding substitution of $\vec{y}$ for $\vec{x}$ in $Q$.
\end{definition}

\begin{definition}
  The {\em structural congruence} \cite{SangiorgiWalker} , $\equiv$,
  between processes is the least congruence containing
  alpha-equivalence, satisfying the abelian monoid laws
  (associativity, commutativity and $\pzero$ as identity) for parallel
  composition $|$ and for summation $+$.
\end{definition}

\subsection{Name equivalence}

We take name equivalence, written $\nameeq$, to be the smallest
equivalence relation generated by the following rules.

\begin{mathpar}
\inferrule*[lab=Quote-drop]
{ }
{ \quotep{@{x}} \nameeq x }

\inferrule*[lab=Struct-equiv]
{ P \scong Q }
{ \quotep{P} \nameeq \quotep{Q} }
\end{mathpar}

The astute reader will have noticed that the mutual recursion of names
and processes imposes a mutual recursion on alpha-equivalence and
structural equivalence via name-equivalence. Fortunately, all of this
works out pleasantly and we may calculate in the natural way, free of
concern. The reader interested in the details is referred to the
appendix \ref{appendix:rho_details}.

\subsection{Substitution}

We use $\Proc$ for the set of processes, $\QProc$ for the set of
names, and $\id{\{}\vec{y} / \vec{x} \id{\}}$ to denote partial maps,
$s : \QProc \rightarrow \QProc$. A map, $s$ lifts, uniquely, to a map
on process terms, $\widehat{s} : \Proc \rightarrow \Proc$ by the
following equations.

\begin{mathpar}
  (0) \psubstp{Q}{P} := 0 \\
  (R \juxtap S) \psubstp{Q}{P}
  :=    
  (R)\psubstp{Q}{P} \juxtap (S) \psubstp{Q}{P} \\
  (x?(y).R) \psubstp{Q}{P}    
  :=    
  (x)\substp{Q}{P} (z)\concat( (R \psubstn{z}{y}) \psubstp{Q}{P} ) \\
  (\lift{x}{R}) \psubstp{Q}{P}  
  :=
  \lift{(x)\substp{Q}{P}}{ R \psubstp{Q}{P} } \\
%   (\dropn{x})  \psubstp{Q}{P}       
%   := 
%   \left\{ 
%     \begin{array}{ccc} 
%       \dropn{\quotep{Q}} & & x \nameeq \quotep{P} \\
%       \dropn{x} & & otherwise \\
%     \end{array}
%   \right. 
  (\dropn{x})  \psubstp{Q}{P}       
  := 
  \left\{ 
    \begin{array}{ccc} 
      Q & & x \nameeq \quotep{P} \\
      \dropn{x} & & otherwise \\
    \end{array}
  \right.
\end{mathpar}
 

where

\begin{eqnarray}
  (x)\id{\{} \lpquote Q \rpquote / \lpquote P \rpquote \id{\}}            = 
  \left\{ 
    \begin{array}{ccc}
      \lpquote Q \rpquote & & x \nameeq \lpquote P \rpquote \\
      x & & otherwise \\
    \end{array}
  \right. \nonumber
\end{eqnarray}

and $z$ is chosen distinct from $\quotep{P}$, $\quotep{Q}$, the free
names in $Q$, and all the names in $R$. Our $\alpha$-equivalence will
be built in the standard way from this substitution.

\begin{remark}\label{rem:no_self_referential_names}
  One consequence of these definitions is that $\forall P. \quotep{P}
  \not\in \freenames{P}$.
\end{remark}

\subsection{ Dynamic quote: an example }

Anticipating something of what's to come, consider applying the
substitution, $\widehat{\id{\{}u / z \id{\}}}$, to the following pair
of processes, $\lift{w}{y!(z)}$ and $w[ \lpquote y!(z) \rpquote ]$.

\begin{eqnarray}
	\lift{w}{y!(z)}\widehat{\id{\{}u / z \id{\}}}
		& = &
		\lift{w}{y!(u)} \nonumber\\
	w[ \lpquote y!(z) \rpquote ] \widehat{ \id{\{}u / z \id{\}} }
		& = &
		w[ \lpquote y!(z) \rpquote ] \nonumber
\end{eqnarray}

Because the body of the process between quotes is impervious to
substitution, we get radically different answers. In fact, by
examining the first process in an input context,
e.g. $x?(z).\lift{w}{y!(z)}$, we see that the process under the lift
operator may be shaped by prefixed inputs binding a name inside it. In
this sense, the lift operator will be seen as a way to dynamically
construct processes before reifying them as names.

Finally equipped with these standard features we can present the
dynamics of the calculus.

\subsubsection{Operational semantics} 

Finally, we introduce the computational dynamics. What marks these
algebras as distinct from other more traditionally studied algebraic
structures, e.g. vector spaces or polynomial rings, is the manner in
which dynamics is captured. In traditional structures, dynamics is typically
expressed through morphisms between such structures, as in linear maps
between vector spaces or morphisms between rings. In algebras
associated with the semantics of computation, the dynamics is
expressed as part of the algebraic structure itself, through a
reduction reduction relation typically denoted by $\red$. Below, we
give a recursive presentation of this relation for the calculus used
in the encoding.

$\red \subseteq \pi \times \pi$
$\red : \pi \to \mathcal{P}(\pi)$

\begin{mathpar}
  \inferrule* [lab=Comm] { \textsf{match}( x_{src}, x_{trgt} ) } { x_{trgt}?(y)P \; | \; x_{src}!\langle {Q} \rangle \red P\{\quotep{Q}/y}\} }
  \and \\
  \inferrule* [lab=Par] {{P} \red {P}'} {{{P} | {Q}} \red {{P}' | {Q}}}
  \and
  \inferrule* [lab=Equiv]{{{P} \scong {P}'} \andalso {{P}' \red {Q}'} \andalso {{Q}' \scong {Q}}}{{P} \red {Q}}
\end{mathpar}

\begin{eqnarray*}
  match_{\equiv} (\quotep{P},\quotep{Q}) & := & P \equiv Q \\
  match_{\dagger}(\quotep{P},\quotep{Q}) & := & \forall R. P|Q \red^{*} R => R \red^{*} 0 \\
  match_{K}(\quotep{P},\quotep{Q}) & := & K \mbox{ for some context } K
\end{eqnarray*}

$u?(x)P | u!\langle Q \rangle \red P\{\quotep{Q}/x\}$

%We write $\wred$ for $\red^*$, and $P\red$ if $\exists Q $ such that $ P \red Q$.
We write $P\red$ if $\exists Q $ such that $ P \red Q$ and $P\not\red$, otherwise.

\section{Replication}

As mentioned before, it is known that replication (and hence
recursion) can be implemented in a higher-order process algebra
\cite{SangiorgiWalker}. As our first example of calculation with the
machinery thus far presented we give the construction explicitly in
the {\rhoc}.

\begin{eqnarray}
	D_{x} & := & \prefix{x}{y}{(\binpar{\outputp{x}{y}}{@{y}})} \nonumber\\
	\bangp_{x}{P} & := & \binpar{{x}!\langle{\binpar{D_{x}}{P}}\rangle}{D_{x}} \nonumber
\end{eqnarray}

\begin{eqnarray}
	\bangp_{x}{P} & & \nonumber\\
	=
	& {x}!\langle{(\prefix{x}{y}{(\outputp{x}{y} | @{y})) | P}}\rangle 
	      | \prefix{x}{y}{(\outputp{x}{y} | @{y})} & \nonumber\\
	\red
	& (\outputp{x}{y} | @{y})\substn{\quotep{(\prefix{x}{y}{(@{y} | \outputp{x}{y})) | P}}}{y} & \nonumber\\
	=
	& \outputp{x}{\quotep{(\prefix{x}{y}{(\outputp{x}{y} | @{y})) | P}}}
	  | {(\prefix{x}{y}{(\outputp{x}{y} | @{y})) | P}} & \nonumber\\
	\red
	& \ldots & \nonumber\\
	\red^*
	& P | P | \ldots & \nonumber
\end{eqnarray}

Of course, this encoding, as an implementation, runs away, unfolding
$\bangp{P}$ eagerly. A lazier and more implementable replication
operator, restricted to input-guarded processes, may be obtained as follows.

\begin{eqnarray}
\bangp{\prefix{u}{v}{P}} 
	:= 
	\binpar{\lift{x}{\prefix{u}{v}{(\binpar{D(x)}{P})}}}{D(x)} \nonumber
\end{eqnarray}

\begin{remark}
  Note that the lazier definition still does not deal with summation
  or mixed summation (i.e. sums over input and output). The reader is
  invited to construct definitions of replication that deal with these
  features. 

  Further, the definitions are parameterized in a name, $x$. Can you,
  gentle reader, make a definition that eliminates this parameter and
  guarantees no accidental interaction between the replication
  machinery and the process being replicated -- i.e. no accidental
  sharing of names used by the process to get its work done and the
  name(s) used by the replication to effect copying. This latter
  revision of the definition of replication is crucial to obtaining
  the expected identity $!!P \sim !P$.
\end{remark}

\begin{remark}\label{rem:paradoxical_combinator}
  The reader familiar with the lambda calculus will have noticed the
  similarity between $D$ and the paradoxical combinator.

  [Ed. note: the existence of this seems to suggest we have to be more
  restrictive on the set of processes and names we admit if we are to
  support no-cloning.]
\end{remark}

\subsubsection{Bisimulation}

The computational dynamics gives rise to another kind of equivalence,
the equivalence of computational behavior. As previously mentioned
this is typically captured \emph{via} some form of bisimulation.

% The notion we use in this paper is weak barbed bisimulation
% \cite{milner91polyadicpi}.

The notion we use in this paper is derived from weak barbed
bisimulation \cite{milner91polyadicpi}. 

\begin{definition}
An \emph{observation relation}, $\downarrow_{\mathcal N}$, over a set
of names, $\mathcal N$, is the smallest relation satisfying the rules
below.

\infrule[Out-barb]{y \in {\mathcal N}, \; x \nameeq y}
		  {\outputp{x}{v} \downarrow_{\mathcal N} x}
\infrule[Par-barb]{\mbox{$P\downarrow_{\mathcal N} x$ or $Q\downarrow_{\mathcal N} x$}}
		  {\binpar{P}{Q} \downarrow_{\mathcal N} x}

We write $P \Downarrow_{\mathcal N} x$ if there is $Q$ such that 
$P \wred Q$ and $Q \downarrow_{\mathcal N} x$.
\end{definition}

\begin{definition}
%\label{def.bbisim}
An  ${\mathcal N}$-\emph{barbed bisimulation} over a set of names, ${\mathcal N}$, is a symmetric binary relation 
${\mathcal S}_{\mathcal N}$ between agents such that $P\rel{S}_{\mathcal N}Q$ implies:
\begin{enumerate}
\item If $P \red P'$ then $Q \wred Q'$ and $P'\rel{S}_{\mathcal N} Q'$.
\item If $P\downarrow_{\mathcal N} x$, then $Q\Downarrow_{\mathcal N} x$.
\end{enumerate}
$P$ is ${\mathcal N}$-barbed bisimilar to $Q$, written
$P \wbbisim_{\mathcal N} Q$, if $P \rel{S}_{\mathcal N} Q$ for some ${\mathcal N}$-barbed bisimulation ${\mathcal S}_{\mathcal N}$.
\end{definition}

$\mathcal{R} \subseteq \pi \times \pi$

$P \mathcal{R} Q => \forall P'. P \red P' \Rightarrow \exists Q'. Q \red Q', P' \mathcal{R} Q'$

$P \vdash x \Rightarrow Q \vdash x$

\begin{mathpar}
  \inferrule*[lab=Out-barb]{x \nameeq y}{{y}!\langle{Q}\rangle \vdash x}
  \and
  \inferrule*[lab=Par-barb]{\mbox{$P\vdash x$ or $Q\vdash x$}}{\binpar{P}{Q} \vdash x}
\end{mathpar}

\subsubsection{Contexts}

One of the principle advantages of computational calculi like the
$\pi$-calculus is a well-defined notion of context,
contextual-equivalence and a correlation between
contextual-equivalence and notions of bisimulation. The notion of
context allows the decomposition of a process into (sub-)process and
its syntactic environment, its context. Thus, a context may be
thought of as a process with a ``hole'' (written $\Box$) in it. The
application of a context $M$ to a process $P$, written $M[P]$, is
tantamount to filling the hole in $M$ with $P$. In this paper we do
not need the full weight of this theory, but do make use of the notion
of context in the proof the main theorem. 

\begin{mathpar}
  \inferrule* [lab=summation] {} {{M_{M},M_{N}} \bc \Box \;|\; x.M_{A} \;|\; M_{M}+M_{N}}
  \and
  \inferrule* [lab=agent] {} {{M_{A}} \bc (\vec{x})M_{P} \;| \; \clift{P_0,\ldots,M_{P},\ldots,P_N}}
  \and \\
  \inferrule* [lab=process] {} {{M_{P}} \bc M_{N} \;| \;P|M_{P} }
\end{mathpar} 

\begin{mathpar}
  \inferrule* [lab=sychronization] {} {M_{N} \bc \Box \;|\; x?M_{F} \;|\; x!M_{C}}
  \and
  \inferrule* [lab=abstraction] {} {{M_{F}} \bc (x)M_{P} }
  \and
  \inferrule* [lab=concretion] {} {{M_{C}} \bc \langle M_{P} \rangle }
  \and \\
  \inferrule* [lab=process] {} {{M_{P}} \bc M_{N} \;| \;P|M_{P} }
\end{mathpar}

\begin{definition}[contextual application] Given a context $M$, and
  process $P$, we define the \emph{contextual application}, $M[P] :=
  M\{P/\Box\}$. That is, the contextual application of M to P is the
  substitution of $P$ for $\Box$ in $M$.
\end{definition}

$\meaningof{-} : L \to \mathcal{P}(\pi)$

\begin{mathpar}
  \inferrule* [lab=collection] {} {\meaningof{true} = \pi, \and \meaningof{~E} = \pi \setminus \meaningof{E}, \and \meaningof{E_{1} \& E_{2}} = \meaningof{E_{1}} \cap \meaningof{E_{2}}}
\end{mathpar}

\begin{mathpar}
  \inferrule* [lab=structure] {} {\meaningof{0} = \{ P \in \pi | P \equiv 0 \}, \and \\ \meaningof{E_1 | E_2} = \{ P \in \pi | P \equiv P_{1} | P_{2}, P_{1} \in \meaningof{E_{1}}, P_{2} \in \meaningof{E_2}\} }
\end{mathpar}

\begin{mathpar}
 \inferrule* [lab=behavior] {} {\meaningof{\langle a?b \rangle E} = \{ P \in \pi | P \equiv Q | u?(y)P', \\ \and \\\\ \and \\ \;\;\; u \in \meaningof{a}, \forall z.P'\{z/y\} \in \meaningof{E\{z/b\}}\}, \and \\ \meaningof{a!E} = \{ P \in \pi | P \equiv Q | x!\langle P' \rangle, x \in \meaningof{a} P' \in \meaningof{E}\} }
\end{mathpar}

\begin{mathpar}
 \inferrule* [lab=nominal] {} {\meaningof{\quotep{E}} = \{ \quotep{P} \in \quotep{\pi} | P \in \meaningof{E} \}, \and \meaningof{\quotep{P}} = \{ \quotep{Q} \in \quotep{\pi} | P \equiv Q \} \and \\ \meaningof{@\quotep{E}} = \{ P \in \pi | P \equiv @x, x \in \meaningof{E} \}}
\end{mathpar}

\begin{eqnarray*}
  \\
  \meaningof{-} : TS \to ST
\end{eqnarray*}

\begin{eqnarray*}
  \\
  L : TS \to ST
\end{eqnarray*}

\begin{eqnarray*}
  \\
  P \models E \iff P \in \meaningof{E}
\end{eqnarray*}

\begin{eqnarray*}
  P \approx_{L} Q \iff \forall E \in L. P \models E \iff Q \models E
\end{eqnarray*}

\begin{eqnarray*}
  P \approx_{K} Q
\end{eqnarray*}

\begin{eqnarray*}
  P \approx Q
\end{eqnarray*}

$\approx_{K} = \approx = \approx_{L}$

\subsubsection{Contextual duality}

Note that contexts extend the quotation operation to a family of
operations from processes to names. Given a context, $M$, we can
define a \emph{nominal context}, $\quotep{M}$ by $\quotep{M}[P] :=
\quotep{M[P]}$. To foreshadow what is to come we observe that these
operations enjoy a duality with processes very much like the duality
between vectors and maps from vectors to scalars.

Further, because the calculus is essentially higher-order, we have a
correspondence between contexts and processes. More specifically,
given a name $x$ and a context $M$ we can construct $M^{*}_{x}$ such
that 

\begin{mathpar}
  M^{*}_{x} | \lift{x}{P} \red M[P]
\end{mathpar}

namely,

\begin{mathpar}
  M^{*}_{x} := x?(u).M[\dropn{u}]
\end{mathpar}

The dependence of $M^{*}_{x}$ on a name makes it an abstraction, 

\begin{mathpar}
  M^{*} := (x)x?(u).M[\dropn{u}]
\end{mathpar}

\subsection{Additional notation}

It will sometimes be convenient to denote the process a name
quotes. We already have the notation $x = \quotep{P}$, but it will be
convenient to introduce an alternate notation, $\procn{x}$, when we
want to emphasize the connection to the use of the name. Note that, by
virtue of name equivalence, $\quotep{\procn{x}} \nameeq x$; so, the
notation is consistent with previous definitions.

Further, because names have structure it is possible to effect
substitutions on the basis of that structure. This means we need to
upgrade our notation for substitutions, which we accomplish by
adapting comprehension notation. Thus,

\begin{mathpar}
  P\{ y / x : x \in S \}
\end{mathpar}

is interpreted to mean the process derived from P by replacing (in a
capture-avoiding manner) each occurrence of $x$ in $S$ by $y$. For example,

\begin{mathpar}
  P\{ \quotep{\procn{x}|\procn{x}} / x : x \in \freenames{P} \}
\end{mathpar}

will replace each (occurrence) of a free name $x$ in $P$ by
$\quotep{\procn{x}|\procn{x}}$.

Also, we will avail ourselves of the notation $x^{L}$ and $x^{R}$ to
denote injections of a name into disjoint copies of the name
space. There are numerous ways to accomplish this. One example can be
found in \cite{MeredithR05}. This notation overloads to vectors of
names: $\vec{x}^{\pi} := (x_{i}^{\pi} \; : \; 0 \leq i < |\vec{x}| )$ where $\pi \in \{L,R\}$.

We also use $P^{\Box} := P|\Box$.

In \cite{MeredithR05} an interpretation of the new operator is
given. It turns out that there are several possible interpretations
all enjoying the requisite algebraic properties of the operator (see
\cite{milner91polyadicpi}). We will therefore make liberal use of
$(\nu\; \vec{x})P$.

% subsection the_syntax_and_semantics_of_the_notation_system (end)   

\input{qm2pi.qmops} 

\input{qm2pi.sterngerlach} 

\input{qm2pi.metric} 

% section concurrent_process_calculi (end)

%\input{qm2pi.proofsketch}

% section proof sketch (end)

%\input{qm2pi.slviaknots} 

% section spatial logic via knots (end)

\input{qm2pi.conclusion}

% section conclusion (end)

%\input{qm2pi.dtcodes} 

% section wiring algorithm (end)

\input{qm2pi.ack} 

% section acknowledgments (end)

\newpage


\bibliographystyle{plain}   
\bibliography{../../biblios/main.bib}

\input{qm2pi.rhodetails}

\end{document}



% section front matter (end)

\section{Introduction}\label{sec:introduction} % (fold)
In this draft of the material i am going to have to dispense with the
usual writing conventions adopted in papers on these topics. i'm going
to have adopt whatever tone i need at the time i'm writing up the
calculations. Sometimes this may be very conversational; others it may
be the barest mathematical grunts; others still it may be that i have
lifted text from one of my other papers because the exposition of some
point was better said there. i hope that my readers are not unduly put
out by this decision. i'm not doing this to flout convention or be
rebellious. i find these calculations very technically challenging. To
keep everything going technically, something has to give; i have to
let go of some cognitive burden. So, the academic writing style --
with all of its trade-offs in terms of facilitating technical
communication -- is what i'm letting go of. Perhaps subsequent drafts
can be tightened and polished, but for now, i'm going to speak as if
we were sitting together in a coffee shop with a laptop, wifi and a
pad of paper and a pencil.

So, here's what i have to say. We -- you and i, comfortably ensconced
in our coffee shop and well-equipped with our tools -- can realize and
carry out the calculations of quantum mechanics over a very different
formal theory of dynamics, a formal theory of dynamics that
corresponds to a theory of concurrent computation with
\emph{reflection}. It has the advantage that the underlying theory is
already `quantized', but supports analogues all of the continuuous
operations. Strikingly, this underlying theory has recently been
connected with a notion of metric that we can show, by calculating
together, coincides with the metric induced by the inner product.

There are a lot of reasons why you might be interested in seeing
calculations of this form. Here's why i'm interested. For the past
several centuries there has been no competitor to the ``Newtonian''
account of dynamics. As a result the predominant share of accounts of
dynamical systems and situations have had to be formulated in terms of
the Newtonian machinery. i view this as an intellectually dangerous
position to occupy. Everything, despite it's intrinsic shape, turns
into a nail to be hit with this hammer. Recently, however, the theory
of computation has matured to the point where we have candidates for
theories of dynamics that offer very different perspective on
reasoning about dynamical systems and situations. Testing these
candidates against very successful accounts of dynamical situations,
like quantum mechanics, is going to give us some sense of how mature
they are and some measure of the quality of these accounts of
dynamics.

\subsection{Summary of contributions and outline of paper}

So, we're going to develop an interpretation of the operations of
quantum mechanics normally interpreted by Hilbert spaces and
operators. We're going to do this over a theory of computation. Note
that this is very different than the usual quantum computation program
which develops notions of computation over quantum mechanics. Rather,
we are developing a story that aligns with Wheeler's slogan: It from
Bit. To do this we will first provide an account of the theory of
computation at play here. Then we will dive into a calculation-driven
interpretation of the operations of quantum mechanics.

The reason we take this approach is that -- until very recently --
there hasn't been an axiomatic account of quantum mechanics. As a
result there has been no sharp delineation of the mathematical theory
supporting interpretation of the physical theory and the physical
theory, itself. So, ambient features of the maths are free to be
exploited (or supressed) without a real accounting of their physical
relevance. There is no sharp statement ``here's the physical theory''
qua \emph{theory} and ``here's the mathematical interpretation''
enabling a judgment of how faithful the interpretation is -- apart
from experimental observation. When there is an axiomatic account we
can judge how well a given mathematical formalism supports an
interpretation of the axioms, independent of
experimentation. Likewise, we can judge how well we have captured our
physical evidence and experience with our axiomatics, independent of
any specific mathematical implementation, with accidental detail that
may or may not have physical significance. 

In lieu of a fully fleshed out and vetted axiomatic account of quantum
mechanics, interpreting the operational notions in service of modeling
physical systems will have to suffice. In other words, we are not in
the business of providing a model of Hilbert spaces and operators. We
are in the business of providing a model of quantum mechanics because
we are motivated by testing our notions of dynamics against physical
theory; and, the predictive calculations of the physical theory must
serve as the best formulation -- shy of a fully fleshed out axiomatic
account -- of the physical theory itself (as they have for scientific
theories since time immemorial). Put another way, despite a
whole-hearted commitment to an It-from-Bit ontology, we are firmly
aligned with the shut-up-and-calculate camp as the best way to obtain
results either from the physical perspective or as a quality assurance
measure of our fledgling theory of dynamics.

In detail, we present a reflective process calculus. Then we develop
intuitive correspondences between the notions available in this
calculus and the usual physical notions supporting quantum mechanical
calculations. Thus, 

\begin{table}[htp]
  \center{
    \fbox{
      \begin{tabular}{c|c}
        quantum mechanics & process calculus \\
        \hline
        scalar & name \\
        state vector & process \\
        dual & contextual duals \\
        matrix & formal sums of process-context-dual pairs \\
        orthogonality & process annihilation \\
        inner product & execution-formula + quoting
      \end{tabular}
    }
  }
  \caption{QM - process calculi correspondences}
\end{table}

Then we tighten up these intuitions to operational definitions. We
employ the Dirac notation as the best proxy we can find for an
abstract syntax of the quantum mechanical notions. The definitions we
develop put us in contact with equational constraints coming from the
theory that we demonstrate the definitions and calculations satisfy.

This puts us in a position to shut up and calculate for the
Stern-Gerlach experimental set up, showing how these predictive
calculations become calculations on processes in our theory of a
reflective process calculus.

Penultimately, we demonstrate that the notion of metric coming from
the inner product coincides with the notion of metric available from
the theory of bisimulation. This demonstration gives us the right to
think of space as arising from behavior. Finally, we consider where we
might go from the new vantage point we have obtained.

% section introduction (end) 
 
% section introduction (end)

% \documentclass[12pt]{llncs}
%\documentclass{jktr}

\usepackage[pdftex]{hyperref}                   
\usepackage {listings}
\usepackage {mathpartir}
\usepackage{bcprules}
%\usepackage{listings}
                       
\usepackage{graphicx} 
%\usepackage[margins=2.5cm,nohead,nofoot]{geometry}
%\usepackage{geometry}
\usepackage{amsfonts}
\usepackage{amstext}
\usepackage{latexsym}
\usepackage{amssymb}
\usepackage{color}


%\include{myPreamble}
\include{qm2pi.local} 

%\ifpdf
%\usepackage[pdftex]{graphicx}
%\else
%\usepackage{graphicx}
%\fi

 % \ifpdf
%  \usepackage{pdfsync}
%  \if


%\title{Brief Article}
%\author{David F. Snyder}
%\author{L.G. Meredith}

%\address{Dept. of Math., Texas State University--San Marcos, San Marcos, TX 78666}
       
\pagestyle{empty}


\begin{document}

\lstset{language=[Objective]Caml,frame=shadowbox}

\input{qm2pi.front}

% section front matter (end)

\input{qm2pi.intro} 
 
% section introduction (end)

% \input{qm2pi.knotations} 

% section notation (end)

\input{qm2pi.process.calculi} 

% section concurrent_process_calculi_and_spatial_logics_ (end)
    
%\input{qm2pi.knots2pi} 

%\input{qm2pi.trefoil} 

%\input{qm2pi.mainthm} 

% subsection basic_interpretation (end)

%\input{qm2pi.rho.presentation} 
\subsection{The syntax and semantics of the notation system}\label{sub:the_syntax_and_semantics_of_the_notation_system} % (fold)

We now summarize a technical presentation of the calculus that
embodies our theory of dynamics. The typical presentation of such a
calculus follows the style of giving generators and relations on
them. The grammar, below, describing term constructors, freely
generates the set of processes, $\Proc$. This set is then quotiented
by a relation known as structural congruence and it is over this set
that the notion of dynamics is expressed. This presentation is
essentially that of \cite{MeredithR05} with the addition of
polyadicity and summation. For readability we have relegated some of
the technical subtleties to an appendix.

\subsubsection{Process grammar}\label{subsub:process_grammar}

\begin{mathpar}
  \inferrule* [lab=synchronization] {} {{M} \bc \pzero \;|\; x?F \;|\; x!C }
  \and
  \inferrule* [lab=abstraction] {} {{F} \bc (x)P}
  \and
  \inferrule* [lab=concretion] {} {{C} \bc \langle Q \rangle}
  \and
  \inferrule* [lab=process] {} {{P,Q} \bc M \;| \;P|Q \;|\; @{x}}
  \and
  \inferrule* [lab=name] {} {{x} \bc \quotep{P}}
\end{mathpar} 

Note that $\vec{x}$ (resp. $\vec{P}$) denotes a vector of names
(resp. processes) of length $|\vec{x}|$ (resp. $|\vec{P}|$). We adopt
the following useful abbreviations.

\begin{mathpar}
   x?(\vec{y}).P := x.(\vec{y})P \and  x\clift{\vec{P}} := x.\clift{\vec{P}}
   \and x!(y) := \lift{x}{\dropn{y}}
   \and \Pi_{i=0}^{n-1}P_i := P_0 | \ldots | P_{n-1}
\end{mathpar}

\subsubsection{Structural congruence}

\paragraph{Free and bound names and alpha-equivalence.} At the
core of structural equivalence is alpha-equivalence which identifies
process that are the same up to a change of variable. Formally, we
recognize the distinction between free and bound names. The free names
of a process, $\freenames{P}$, may be calculated recursively as
follows:

\begin{mathpar}
\freenames{\pzero} := \emptyset
  \and \\
  \freenames{x?(y).P} := \{ x \} \cup (\freenames{P} \setminus \{ y \})
  \and 
  \freenames{x!\langle P \rangle} := \{ x \} \cup \{ P \} 
  \and \\
  \freenames{P|Q} := \freenames{P} \cup \freenames{Q}
  \and \\
  \freenames{@{x}} := \{ x \}
\end{mathpar}

$\pi$
$\quotep{\pi}$

$\freenames{-} : \pi \to \mathcal{P}(\quotep{\pi})$

\begin{eqnarray*}
  \freenames{\pzero} & := & \emptyset \\
  \freenames{x?(y).P} & := & \{ x \} \cup (\freenames{P} \setminus \{ y \}) \\
  \freenames{x!\langle P \rangle} & := & \{ x \} \cup \{ P \} \\
  \freenames{P|Q} & := & \freenames{P} \cup \freenames{Q} \\
  \freenames{\dropn{x}} & := & \{ x \}
\end{eqnarray*}

The bound names of a process, $\boundnames{P}$, are those names occurring in $P$
that are not free. For example, in $x?(y).0$, the name $x$ is free, while $y$ is bound.

\begin{mathpar}
  \inferrule* [lab=monoidal-laws] {} { P|Q \equiv Q|P \and P|0 \equiv P \and P|(Q|R) \equiv (P|Q)|R }
\end{mathpar}

\begin{mathpar}
  \inferrule* [lab=alpha-equivalence] {} { (x)P \equiv (y)P\{y/x\} \and y \not\in \freenames{P} }
\end{mathpar}

\begin{definition}
Then two processes, $P,Q$, are alpha-equivalent if $P = Q\{\vec{y}/\vec{x}\}$ for
some $\vec{x} \in \boundnames{Q},\vec{y} \in \boundnames{P}$, where $Q\{\vec{y}/\vec{x}\}$
denotes the capture-avoiding substitution of $\vec{y}$ for $\vec{x}$ in $Q$.
\end{definition}

\begin{definition}
  The {\em structural congruence} \cite{SangiorgiWalker} , $\equiv$,
  between processes is the least congruence containing
  alpha-equivalence, satisfying the abelian monoid laws
  (associativity, commutativity and $\pzero$ as identity) for parallel
  composition $|$ and for summation $+$.
\end{definition}

\subsection{Name equivalence}

We take name equivalence, written $\nameeq$, to be the smallest
equivalence relation generated by the following rules.

\begin{mathpar}
\inferrule*[lab=Quote-drop]
{ }
{ \quotep{@{x}} \nameeq x }

\inferrule*[lab=Struct-equiv]
{ P \scong Q }
{ \quotep{P} \nameeq \quotep{Q} }
\end{mathpar}

The astute reader will have noticed that the mutual recursion of names
and processes imposes a mutual recursion on alpha-equivalence and
structural equivalence via name-equivalence. Fortunately, all of this
works out pleasantly and we may calculate in the natural way, free of
concern. The reader interested in the details is referred to the
appendix \ref{appendix:rho_details}.

\subsection{Substitution}

We use $\Proc$ for the set of processes, $\QProc$ for the set of
names, and $\id{\{}\vec{y} / \vec{x} \id{\}}$ to denote partial maps,
$s : \QProc \rightarrow \QProc$. A map, $s$ lifts, uniquely, to a map
on process terms, $\widehat{s} : \Proc \rightarrow \Proc$ by the
following equations.

\begin{mathpar}
  (0) \psubstp{Q}{P} := 0 \\
  (R \juxtap S) \psubstp{Q}{P}
  :=    
  (R)\psubstp{Q}{P} \juxtap (S) \psubstp{Q}{P} \\
  (x?(y).R) \psubstp{Q}{P}    
  :=    
  (x)\substp{Q}{P} (z)\concat( (R \psubstn{z}{y}) \psubstp{Q}{P} ) \\
  (\lift{x}{R}) \psubstp{Q}{P}  
  :=
  \lift{(x)\substp{Q}{P}}{ R \psubstp{Q}{P} } \\
%   (\dropn{x})  \psubstp{Q}{P}       
%   := 
%   \left\{ 
%     \begin{array}{ccc} 
%       \dropn{\quotep{Q}} & & x \nameeq \quotep{P} \\
%       \dropn{x} & & otherwise \\
%     \end{array}
%   \right. 
  (\dropn{x})  \psubstp{Q}{P}       
  := 
  \left\{ 
    \begin{array}{ccc} 
      Q & & x \nameeq \quotep{P} \\
      \dropn{x} & & otherwise \\
    \end{array}
  \right.
\end{mathpar}
 

where

\begin{eqnarray}
  (x)\id{\{} \lpquote Q \rpquote / \lpquote P \rpquote \id{\}}            = 
  \left\{ 
    \begin{array}{ccc}
      \lpquote Q \rpquote & & x \nameeq \lpquote P \rpquote \\
      x & & otherwise \\
    \end{array}
  \right. \nonumber
\end{eqnarray}

and $z$ is chosen distinct from $\quotep{P}$, $\quotep{Q}$, the free
names in $Q$, and all the names in $R$. Our $\alpha$-equivalence will
be built in the standard way from this substitution.

\begin{remark}\label{rem:no_self_referential_names}
  One consequence of these definitions is that $\forall P. \quotep{P}
  \not\in \freenames{P}$.
\end{remark}

\subsection{ Dynamic quote: an example }

Anticipating something of what's to come, consider applying the
substitution, $\widehat{\id{\{}u / z \id{\}}}$, to the following pair
of processes, $\lift{w}{y!(z)}$ and $w[ \lpquote y!(z) \rpquote ]$.

\begin{eqnarray}
	\lift{w}{y!(z)}\widehat{\id{\{}u / z \id{\}}}
		& = &
		\lift{w}{y!(u)} \nonumber\\
	w[ \lpquote y!(z) \rpquote ] \widehat{ \id{\{}u / z \id{\}} }
		& = &
		w[ \lpquote y!(z) \rpquote ] \nonumber
\end{eqnarray}

Because the body of the process between quotes is impervious to
substitution, we get radically different answers. In fact, by
examining the first process in an input context,
e.g. $x?(z).\lift{w}{y!(z)}$, we see that the process under the lift
operator may be shaped by prefixed inputs binding a name inside it. In
this sense, the lift operator will be seen as a way to dynamically
construct processes before reifying them as names.

Finally equipped with these standard features we can present the
dynamics of the calculus.

\subsubsection{Operational semantics} 

Finally, we introduce the computational dynamics. What marks these
algebras as distinct from other more traditionally studied algebraic
structures, e.g. vector spaces or polynomial rings, is the manner in
which dynamics is captured. In traditional structures, dynamics is typically
expressed through morphisms between such structures, as in linear maps
between vector spaces or morphisms between rings. In algebras
associated with the semantics of computation, the dynamics is
expressed as part of the algebraic structure itself, through a
reduction reduction relation typically denoted by $\red$. Below, we
give a recursive presentation of this relation for the calculus used
in the encoding.

$\red \subseteq \pi \times \pi$
$\red : \pi \to \mathcal{P}(\pi)$

\begin{mathpar}
  \inferrule* [lab=Comm] { \textsf{match}( x_{src}, x_{trgt} ) } { x_{trgt}?(y)P \; | \; x_{src}!\langle {Q} \rangle \red P\{\quotep{Q}/y}\} }
  \and \\
  \inferrule* [lab=Par] {{P} \red {P}'} {{{P} | {Q}} \red {{P}' | {Q}}}
  \and
  \inferrule* [lab=Equiv]{{{P} \scong {P}'} \andalso {{P}' \red {Q}'} \andalso {{Q}' \scong {Q}}}{{P} \red {Q}}
\end{mathpar}

\begin{eqnarray*}
  match_{\equiv} (\quotep{P},\quotep{Q}) & := & P \equiv Q \\
  match_{\dagger}(\quotep{P},\quotep{Q}) & := & \forall R. P|Q \red^{*} R => R \red^{*} 0 \\
  match_{K}(\quotep{P},\quotep{Q}) & := & K \mbox{ for some context } K
\end{eqnarray*}

$u?(x)P | u!\langle Q \rangle \red P\{\quotep{Q}/x\}$

%We write $\wred$ for $\red^*$, and $P\red$ if $\exists Q $ such that $ P \red Q$.
We write $P\red$ if $\exists Q $ such that $ P \red Q$ and $P\not\red$, otherwise.

\section{Replication}

As mentioned before, it is known that replication (and hence
recursion) can be implemented in a higher-order process algebra
\cite{SangiorgiWalker}. As our first example of calculation with the
machinery thus far presented we give the construction explicitly in
the {\rhoc}.

\begin{eqnarray}
	D_{x} & := & \prefix{x}{y}{(\binpar{\outputp{x}{y}}{@{y}})} \nonumber\\
	\bangp_{x}{P} & := & \binpar{{x}!\langle{\binpar{D_{x}}{P}}\rangle}{D_{x}} \nonumber
\end{eqnarray}

\begin{eqnarray}
	\bangp_{x}{P} & & \nonumber\\
	=
	& {x}!\langle{(\prefix{x}{y}{(\outputp{x}{y} | @{y})) | P}}\rangle 
	      | \prefix{x}{y}{(\outputp{x}{y} | @{y})} & \nonumber\\
	\red
	& (\outputp{x}{y} | @{y})\substn{\quotep{(\prefix{x}{y}{(@{y} | \outputp{x}{y})) | P}}}{y} & \nonumber\\
	=
	& \outputp{x}{\quotep{(\prefix{x}{y}{(\outputp{x}{y} | @{y})) | P}}}
	  | {(\prefix{x}{y}{(\outputp{x}{y} | @{y})) | P}} & \nonumber\\
	\red
	& \ldots & \nonumber\\
	\red^*
	& P | P | \ldots & \nonumber
\end{eqnarray}

Of course, this encoding, as an implementation, runs away, unfolding
$\bangp{P}$ eagerly. A lazier and more implementable replication
operator, restricted to input-guarded processes, may be obtained as follows.

\begin{eqnarray}
\bangp{\prefix{u}{v}{P}} 
	:= 
	\binpar{\lift{x}{\prefix{u}{v}{(\binpar{D(x)}{P})}}}{D(x)} \nonumber
\end{eqnarray}

\begin{remark}
  Note that the lazier definition still does not deal with summation
  or mixed summation (i.e. sums over input and output). The reader is
  invited to construct definitions of replication that deal with these
  features. 

  Further, the definitions are parameterized in a name, $x$. Can you,
  gentle reader, make a definition that eliminates this parameter and
  guarantees no accidental interaction between the replication
  machinery and the process being replicated -- i.e. no accidental
  sharing of names used by the process to get its work done and the
  name(s) used by the replication to effect copying. This latter
  revision of the definition of replication is crucial to obtaining
  the expected identity $!!P \sim !P$.
\end{remark}

\begin{remark}\label{rem:paradoxical_combinator}
  The reader familiar with the lambda calculus will have noticed the
  similarity between $D$ and the paradoxical combinator.

  [Ed. note: the existence of this seems to suggest we have to be more
  restrictive on the set of processes and names we admit if we are to
  support no-cloning.]
\end{remark}

\subsubsection{Bisimulation}

The computational dynamics gives rise to another kind of equivalence,
the equivalence of computational behavior. As previously mentioned
this is typically captured \emph{via} some form of bisimulation.

% The notion we use in this paper is weak barbed bisimulation
% \cite{milner91polyadicpi}.

The notion we use in this paper is derived from weak barbed
bisimulation \cite{milner91polyadicpi}. 

\begin{definition}
An \emph{observation relation}, $\downarrow_{\mathcal N}$, over a set
of names, $\mathcal N$, is the smallest relation satisfying the rules
below.

\infrule[Out-barb]{y \in {\mathcal N}, \; x \nameeq y}
		  {\outputp{x}{v} \downarrow_{\mathcal N} x}
\infrule[Par-barb]{\mbox{$P\downarrow_{\mathcal N} x$ or $Q\downarrow_{\mathcal N} x$}}
		  {\binpar{P}{Q} \downarrow_{\mathcal N} x}

We write $P \Downarrow_{\mathcal N} x$ if there is $Q$ such that 
$P \wred Q$ and $Q \downarrow_{\mathcal N} x$.
\end{definition}

\begin{definition}
%\label{def.bbisim}
An  ${\mathcal N}$-\emph{barbed bisimulation} over a set of names, ${\mathcal N}$, is a symmetric binary relation 
${\mathcal S}_{\mathcal N}$ between agents such that $P\rel{S}_{\mathcal N}Q$ implies:
\begin{enumerate}
\item If $P \red P'$ then $Q \wred Q'$ and $P'\rel{S}_{\mathcal N} Q'$.
\item If $P\downarrow_{\mathcal N} x$, then $Q\Downarrow_{\mathcal N} x$.
\end{enumerate}
$P$ is ${\mathcal N}$-barbed bisimilar to $Q$, written
$P \wbbisim_{\mathcal N} Q$, if $P \rel{S}_{\mathcal N} Q$ for some ${\mathcal N}$-barbed bisimulation ${\mathcal S}_{\mathcal N}$.
\end{definition}

$\mathcal{R} \subseteq \pi \times \pi$

$P \mathcal{R} Q => \forall P'. P \red P' \Rightarrow \exists Q'. Q \red Q', P' \mathcal{R} Q'$

$P \vdash x \Rightarrow Q \vdash x$

\begin{mathpar}
  \inferrule*[lab=Out-barb]{x \nameeq y}{{y}!\langle{Q}\rangle \vdash x}
  \and
  \inferrule*[lab=Par-barb]{\mbox{$P\vdash x$ or $Q\vdash x$}}{\binpar{P}{Q} \vdash x}
\end{mathpar}

\subsubsection{Contexts}

One of the principle advantages of computational calculi like the
$\pi$-calculus is a well-defined notion of context,
contextual-equivalence and a correlation between
contextual-equivalence and notions of bisimulation. The notion of
context allows the decomposition of a process into (sub-)process and
its syntactic environment, its context. Thus, a context may be
thought of as a process with a ``hole'' (written $\Box$) in it. The
application of a context $M$ to a process $P$, written $M[P]$, is
tantamount to filling the hole in $M$ with $P$. In this paper we do
not need the full weight of this theory, but do make use of the notion
of context in the proof the main theorem. 

\begin{mathpar}
  \inferrule* [lab=summation] {} {{M_{M},M_{N}} \bc \Box \;|\; x.M_{A} \;|\; M_{M}+M_{N}}
  \and
  \inferrule* [lab=agent] {} {{M_{A}} \bc (\vec{x})M_{P} \;| \; \clift{P_0,\ldots,M_{P},\ldots,P_N}}
  \and \\
  \inferrule* [lab=process] {} {{M_{P}} \bc M_{N} \;| \;P|M_{P} }
\end{mathpar} 

\begin{mathpar}
  \inferrule* [lab=sychronization] {} {M_{N} \bc \Box \;|\; x?M_{F} \;|\; x!M_{C}}
  \and
  \inferrule* [lab=abstraction] {} {{M_{F}} \bc (x)M_{P} }
  \and
  \inferrule* [lab=concretion] {} {{M_{C}} \bc \langle M_{P} \rangle }
  \and \\
  \inferrule* [lab=process] {} {{M_{P}} \bc M_{N} \;| \;P|M_{P} }
\end{mathpar}

\begin{definition}[contextual application] Given a context $M$, and
  process $P$, we define the \emph{contextual application}, $M[P] :=
  M\{P/\Box\}$. That is, the contextual application of M to P is the
  substitution of $P$ for $\Box$ in $M$.
\end{definition}

$\meaningof{-} : L \to \mathcal{P}(\pi)$

\begin{mathpar}
  \inferrule* [lab=collection] {} {\meaningof{true} = \pi, \and \meaningof{~E} = \pi \setminus \meaningof{E}, \and \meaningof{E_{1} \& E_{2}} = \meaningof{E_{1}} \cap \meaningof{E_{2}}}
\end{mathpar}

\begin{mathpar}
  \inferrule* [lab=structure] {} {\meaningof{0} = \{ P \in \pi | P \equiv 0 \}, \and \\ \meaningof{E_1 | E_2} = \{ P \in \pi | P \equiv P_{1} | P_{2}, P_{1} \in \meaningof{E_{1}}, P_{2} \in \meaningof{E_2}\} }
\end{mathpar}

\begin{mathpar}
 \inferrule* [lab=behavior] {} {\meaningof{\langle a?b \rangle E} = \{ P \in \pi | P \equiv Q | u?(y)P', \\ \and \\\\ \and \\ \;\;\; u \in \meaningof{a}, \forall z.P'\{z/y\} \in \meaningof{E\{z/b\}}\}, \and \\ \meaningof{a!E} = \{ P \in \pi | P \equiv Q | x!\langle P' \rangle, x \in \meaningof{a} P' \in \meaningof{E}\} }
\end{mathpar}

\begin{mathpar}
 \inferrule* [lab=nominal] {} {\meaningof{\quotep{E}} = \{ \quotep{P} \in \quotep{\pi} | P \in \meaningof{E} \}, \and \meaningof{\quotep{P}} = \{ \quotep{Q} \in \quotep{\pi} | P \equiv Q \} \and \\ \meaningof{@\quotep{E}} = \{ P \in \pi | P \equiv @x, x \in \meaningof{E} \}}
\end{mathpar}

\begin{eqnarray*}
  \\
  \meaningof{-} : TS \to ST
\end{eqnarray*}

\begin{eqnarray*}
  \\
  L : TS \to ST
\end{eqnarray*}

\begin{eqnarray*}
  \\
  P \models E \iff P \in \meaningof{E}
\end{eqnarray*}

\begin{eqnarray*}
  P \approx_{L} Q \iff \forall E \in L. P \models E \iff Q \models E
\end{eqnarray*}

\begin{eqnarray*}
  P \approx_{K} Q
\end{eqnarray*}

\begin{eqnarray*}
  P \approx Q
\end{eqnarray*}

$\approx_{K} = \approx = \approx_{L}$

\subsubsection{Contextual duality}

Note that contexts extend the quotation operation to a family of
operations from processes to names. Given a context, $M$, we can
define a \emph{nominal context}, $\quotep{M}$ by $\quotep{M}[P] :=
\quotep{M[P]}$. To foreshadow what is to come we observe that these
operations enjoy a duality with processes very much like the duality
between vectors and maps from vectors to scalars.

Further, because the calculus is essentially higher-order, we have a
correspondence between contexts and processes. More specifically,
given a name $x$ and a context $M$ we can construct $M^{*}_{x}$ such
that 

\begin{mathpar}
  M^{*}_{x} | \lift{x}{P} \red M[P]
\end{mathpar}

namely,

\begin{mathpar}
  M^{*}_{x} := x?(u).M[\dropn{u}]
\end{mathpar}

The dependence of $M^{*}_{x}$ on a name makes it an abstraction, 

\begin{mathpar}
  M^{*} := (x)x?(u).M[\dropn{u}]
\end{mathpar}

\subsection{Additional notation}

It will sometimes be convenient to denote the process a name
quotes. We already have the notation $x = \quotep{P}$, but it will be
convenient to introduce an alternate notation, $\procn{x}$, when we
want to emphasize the connection to the use of the name. Note that, by
virtue of name equivalence, $\quotep{\procn{x}} \nameeq x$; so, the
notation is consistent with previous definitions.

Further, because names have structure it is possible to effect
substitutions on the basis of that structure. This means we need to
upgrade our notation for substitutions, which we accomplish by
adapting comprehension notation. Thus,

\begin{mathpar}
  P\{ y / x : x \in S \}
\end{mathpar}

is interpreted to mean the process derived from P by replacing (in a
capture-avoiding manner) each occurrence of $x$ in $S$ by $y$. For example,

\begin{mathpar}
  P\{ \quotep{\procn{x}|\procn{x}} / x : x \in \freenames{P} \}
\end{mathpar}

will replace each (occurrence) of a free name $x$ in $P$ by
$\quotep{\procn{x}|\procn{x}}$.

Also, we will avail ourselves of the notation $x^{L}$ and $x^{R}$ to
denote injections of a name into disjoint copies of the name
space. There are numerous ways to accomplish this. One example can be
found in \cite{MeredithR05}. This notation overloads to vectors of
names: $\vec{x}^{\pi} := (x_{i}^{\pi} \; : \; 0 \leq i < |\vec{x}| )$ where $\pi \in \{L,R\}$.

We also use $P^{\Box} := P|\Box$.

In \cite{MeredithR05} an interpretation of the new operator is
given. It turns out that there are several possible interpretations
all enjoying the requisite algebraic properties of the operator (see
\cite{milner91polyadicpi}). We will therefore make liberal use of
$(\nu\; \vec{x})P$.

% subsection the_syntax_and_semantics_of_the_notation_system (end)   

\input{qm2pi.qmops} 

\input{qm2pi.sterngerlach} 

\input{qm2pi.metric} 

% section concurrent_process_calculi (end)

%\input{qm2pi.proofsketch}

% section proof sketch (end)

%\input{qm2pi.slviaknots} 

% section spatial logic via knots (end)

\input{qm2pi.conclusion}

% section conclusion (end)

%\input{qm2pi.dtcodes} 

% section wiring algorithm (end)

\input{qm2pi.ack} 

% section acknowledgments (end)

\newpage


\bibliographystyle{plain}   
\bibliography{../../biblios/main.bib}

\input{qm2pi.rhodetails}

\end{document}

 

% section notation (end)

\input{qm2pi.process.calculi} 

% section concurrent_process_calculi_and_spatial_logics_ (end)
    
%\documentclass[12pt]{llncs}
%\documentclass{jktr}

\usepackage[pdftex]{hyperref}                   
\usepackage {listings}
\usepackage {mathpartir}
\usepackage{bcprules}
%\usepackage{listings}
                       
\usepackage{graphicx} 
%\usepackage[margins=2.5cm,nohead,nofoot]{geometry}
%\usepackage{geometry}
\usepackage{amsfonts}
\usepackage{amstext}
\usepackage{latexsym}
\usepackage{amssymb}
\usepackage{color}


%\include{myPreamble}
\include{qm2pi.local} 

%\ifpdf
%\usepackage[pdftex]{graphicx}
%\else
%\usepackage{graphicx}
%\fi

 % \ifpdf
%  \usepackage{pdfsync}
%  \if


%\title{Brief Article}
%\author{David F. Snyder}
%\author{L.G. Meredith}

%\address{Dept. of Math., Texas State University--San Marcos, San Marcos, TX 78666}
       
\pagestyle{empty}


\begin{document}

\lstset{language=[Objective]Caml,frame=shadowbox}

\input{qm2pi.front}

% section front matter (end)

\input{qm2pi.intro} 
 
% section introduction (end)

% \input{qm2pi.knotations} 

% section notation (end)

\input{qm2pi.process.calculi} 

% section concurrent_process_calculi_and_spatial_logics_ (end)
    
%\input{qm2pi.knots2pi} 

%\input{qm2pi.trefoil} 

%\input{qm2pi.mainthm} 

% subsection basic_interpretation (end)

%\input{qm2pi.rho.presentation} 
\subsection{The syntax and semantics of the notation system}\label{sub:the_syntax_and_semantics_of_the_notation_system} % (fold)

We now summarize a technical presentation of the calculus that
embodies our theory of dynamics. The typical presentation of such a
calculus follows the style of giving generators and relations on
them. The grammar, below, describing term constructors, freely
generates the set of processes, $\Proc$. This set is then quotiented
by a relation known as structural congruence and it is over this set
that the notion of dynamics is expressed. This presentation is
essentially that of \cite{MeredithR05} with the addition of
polyadicity and summation. For readability we have relegated some of
the technical subtleties to an appendix.

\subsubsection{Process grammar}\label{subsub:process_grammar}

\begin{mathpar}
  \inferrule* [lab=synchronization] {} {{M} \bc \pzero \;|\; x?F \;|\; x!C }
  \and
  \inferrule* [lab=abstraction] {} {{F} \bc (x)P}
  \and
  \inferrule* [lab=concretion] {} {{C} \bc \langle Q \rangle}
  \and
  \inferrule* [lab=process] {} {{P,Q} \bc M \;| \;P|Q \;|\; @{x}}
  \and
  \inferrule* [lab=name] {} {{x} \bc \quotep{P}}
\end{mathpar} 

Note that $\vec{x}$ (resp. $\vec{P}$) denotes a vector of names
(resp. processes) of length $|\vec{x}|$ (resp. $|\vec{P}|$). We adopt
the following useful abbreviations.

\begin{mathpar}
   x?(\vec{y}).P := x.(\vec{y})P \and  x\clift{\vec{P}} := x.\clift{\vec{P}}
   \and x!(y) := \lift{x}{\dropn{y}}
   \and \Pi_{i=0}^{n-1}P_i := P_0 | \ldots | P_{n-1}
\end{mathpar}

\subsubsection{Structural congruence}

\paragraph{Free and bound names and alpha-equivalence.} At the
core of structural equivalence is alpha-equivalence which identifies
process that are the same up to a change of variable. Formally, we
recognize the distinction between free and bound names. The free names
of a process, $\freenames{P}$, may be calculated recursively as
follows:

\begin{mathpar}
\freenames{\pzero} := \emptyset
  \and \\
  \freenames{x?(y).P} := \{ x \} \cup (\freenames{P} \setminus \{ y \})
  \and 
  \freenames{x!\langle P \rangle} := \{ x \} \cup \{ P \} 
  \and \\
  \freenames{P|Q} := \freenames{P} \cup \freenames{Q}
  \and \\
  \freenames{@{x}} := \{ x \}
\end{mathpar}

$\pi$
$\quotep{\pi}$

$\freenames{-} : \pi \to \mathcal{P}(\quotep{\pi})$

\begin{eqnarray*}
  \freenames{\pzero} & := & \emptyset \\
  \freenames{x?(y).P} & := & \{ x \} \cup (\freenames{P} \setminus \{ y \}) \\
  \freenames{x!\langle P \rangle} & := & \{ x \} \cup \{ P \} \\
  \freenames{P|Q} & := & \freenames{P} \cup \freenames{Q} \\
  \freenames{\dropn{x}} & := & \{ x \}
\end{eqnarray*}

The bound names of a process, $\boundnames{P}$, are those names occurring in $P$
that are not free. For example, in $x?(y).0$, the name $x$ is free, while $y$ is bound.

\begin{mathpar}
  \inferrule* [lab=monoidal-laws] {} { P|Q \equiv Q|P \and P|0 \equiv P \and P|(Q|R) \equiv (P|Q)|R }
\end{mathpar}

\begin{mathpar}
  \inferrule* [lab=alpha-equivalence] {} { (x)P \equiv (y)P\{y/x\} \and y \not\in \freenames{P} }
\end{mathpar}

\begin{definition}
Then two processes, $P,Q$, are alpha-equivalent if $P = Q\{\vec{y}/\vec{x}\}$ for
some $\vec{x} \in \boundnames{Q},\vec{y} \in \boundnames{P}$, where $Q\{\vec{y}/\vec{x}\}$
denotes the capture-avoiding substitution of $\vec{y}$ for $\vec{x}$ in $Q$.
\end{definition}

\begin{definition}
  The {\em structural congruence} \cite{SangiorgiWalker} , $\equiv$,
  between processes is the least congruence containing
  alpha-equivalence, satisfying the abelian monoid laws
  (associativity, commutativity and $\pzero$ as identity) for parallel
  composition $|$ and for summation $+$.
\end{definition}

\subsection{Name equivalence}

We take name equivalence, written $\nameeq$, to be the smallest
equivalence relation generated by the following rules.

\begin{mathpar}
\inferrule*[lab=Quote-drop]
{ }
{ \quotep{@{x}} \nameeq x }

\inferrule*[lab=Struct-equiv]
{ P \scong Q }
{ \quotep{P} \nameeq \quotep{Q} }
\end{mathpar}

The astute reader will have noticed that the mutual recursion of names
and processes imposes a mutual recursion on alpha-equivalence and
structural equivalence via name-equivalence. Fortunately, all of this
works out pleasantly and we may calculate in the natural way, free of
concern. The reader interested in the details is referred to the
appendix \ref{appendix:rho_details}.

\subsection{Substitution}

We use $\Proc$ for the set of processes, $\QProc$ for the set of
names, and $\id{\{}\vec{y} / \vec{x} \id{\}}$ to denote partial maps,
$s : \QProc \rightarrow \QProc$. A map, $s$ lifts, uniquely, to a map
on process terms, $\widehat{s} : \Proc \rightarrow \Proc$ by the
following equations.

\begin{mathpar}
  (0) \psubstp{Q}{P} := 0 \\
  (R \juxtap S) \psubstp{Q}{P}
  :=    
  (R)\psubstp{Q}{P} \juxtap (S) \psubstp{Q}{P} \\
  (x?(y).R) \psubstp{Q}{P}    
  :=    
  (x)\substp{Q}{P} (z)\concat( (R \psubstn{z}{y}) \psubstp{Q}{P} ) \\
  (\lift{x}{R}) \psubstp{Q}{P}  
  :=
  \lift{(x)\substp{Q}{P}}{ R \psubstp{Q}{P} } \\
%   (\dropn{x})  \psubstp{Q}{P}       
%   := 
%   \left\{ 
%     \begin{array}{ccc} 
%       \dropn{\quotep{Q}} & & x \nameeq \quotep{P} \\
%       \dropn{x} & & otherwise \\
%     \end{array}
%   \right. 
  (\dropn{x})  \psubstp{Q}{P}       
  := 
  \left\{ 
    \begin{array}{ccc} 
      Q & & x \nameeq \quotep{P} \\
      \dropn{x} & & otherwise \\
    \end{array}
  \right.
\end{mathpar}
 

where

\begin{eqnarray}
  (x)\id{\{} \lpquote Q \rpquote / \lpquote P \rpquote \id{\}}            = 
  \left\{ 
    \begin{array}{ccc}
      \lpquote Q \rpquote & & x \nameeq \lpquote P \rpquote \\
      x & & otherwise \\
    \end{array}
  \right. \nonumber
\end{eqnarray}

and $z$ is chosen distinct from $\quotep{P}$, $\quotep{Q}$, the free
names in $Q$, and all the names in $R$. Our $\alpha$-equivalence will
be built in the standard way from this substitution.

\begin{remark}\label{rem:no_self_referential_names}
  One consequence of these definitions is that $\forall P. \quotep{P}
  \not\in \freenames{P}$.
\end{remark}

\subsection{ Dynamic quote: an example }

Anticipating something of what's to come, consider applying the
substitution, $\widehat{\id{\{}u / z \id{\}}}$, to the following pair
of processes, $\lift{w}{y!(z)}$ and $w[ \lpquote y!(z) \rpquote ]$.

\begin{eqnarray}
	\lift{w}{y!(z)}\widehat{\id{\{}u / z \id{\}}}
		& = &
		\lift{w}{y!(u)} \nonumber\\
	w[ \lpquote y!(z) \rpquote ] \widehat{ \id{\{}u / z \id{\}} }
		& = &
		w[ \lpquote y!(z) \rpquote ] \nonumber
\end{eqnarray}

Because the body of the process between quotes is impervious to
substitution, we get radically different answers. In fact, by
examining the first process in an input context,
e.g. $x?(z).\lift{w}{y!(z)}$, we see that the process under the lift
operator may be shaped by prefixed inputs binding a name inside it. In
this sense, the lift operator will be seen as a way to dynamically
construct processes before reifying them as names.

Finally equipped with these standard features we can present the
dynamics of the calculus.

\subsubsection{Operational semantics} 

Finally, we introduce the computational dynamics. What marks these
algebras as distinct from other more traditionally studied algebraic
structures, e.g. vector spaces or polynomial rings, is the manner in
which dynamics is captured. In traditional structures, dynamics is typically
expressed through morphisms between such structures, as in linear maps
between vector spaces or morphisms between rings. In algebras
associated with the semantics of computation, the dynamics is
expressed as part of the algebraic structure itself, through a
reduction reduction relation typically denoted by $\red$. Below, we
give a recursive presentation of this relation for the calculus used
in the encoding.

$\red \subseteq \pi \times \pi$
$\red : \pi \to \mathcal{P}(\pi)$

\begin{mathpar}
  \inferrule* [lab=Comm] { \textsf{match}( x_{src}, x_{trgt} ) } { x_{trgt}?(y)P \; | \; x_{src}!\langle {Q} \rangle \red P\{\quotep{Q}/y}\} }
  \and \\
  \inferrule* [lab=Par] {{P} \red {P}'} {{{P} | {Q}} \red {{P}' | {Q}}}
  \and
  \inferrule* [lab=Equiv]{{{P} \scong {P}'} \andalso {{P}' \red {Q}'} \andalso {{Q}' \scong {Q}}}{{P} \red {Q}}
\end{mathpar}

\begin{eqnarray*}
  match_{\equiv} (\quotep{P},\quotep{Q}) & := & P \equiv Q \\
  match_{\dagger}(\quotep{P},\quotep{Q}) & := & \forall R. P|Q \red^{*} R => R \red^{*} 0 \\
  match_{K}(\quotep{P},\quotep{Q}) & := & K \mbox{ for some context } K
\end{eqnarray*}

$u?(x)P | u!\langle Q \rangle \red P\{\quotep{Q}/x\}$

%We write $\wred$ for $\red^*$, and $P\red$ if $\exists Q $ such that $ P \red Q$.
We write $P\red$ if $\exists Q $ such that $ P \red Q$ and $P\not\red$, otherwise.

\section{Replication}

As mentioned before, it is known that replication (and hence
recursion) can be implemented in a higher-order process algebra
\cite{SangiorgiWalker}. As our first example of calculation with the
machinery thus far presented we give the construction explicitly in
the {\rhoc}.

\begin{eqnarray}
	D_{x} & := & \prefix{x}{y}{(\binpar{\outputp{x}{y}}{@{y}})} \nonumber\\
	\bangp_{x}{P} & := & \binpar{{x}!\langle{\binpar{D_{x}}{P}}\rangle}{D_{x}} \nonumber
\end{eqnarray}

\begin{eqnarray}
	\bangp_{x}{P} & & \nonumber\\
	=
	& {x}!\langle{(\prefix{x}{y}{(\outputp{x}{y} | @{y})) | P}}\rangle 
	      | \prefix{x}{y}{(\outputp{x}{y} | @{y})} & \nonumber\\
	\red
	& (\outputp{x}{y} | @{y})\substn{\quotep{(\prefix{x}{y}{(@{y} | \outputp{x}{y})) | P}}}{y} & \nonumber\\
	=
	& \outputp{x}{\quotep{(\prefix{x}{y}{(\outputp{x}{y} | @{y})) | P}}}
	  | {(\prefix{x}{y}{(\outputp{x}{y} | @{y})) | P}} & \nonumber\\
	\red
	& \ldots & \nonumber\\
	\red^*
	& P | P | \ldots & \nonumber
\end{eqnarray}

Of course, this encoding, as an implementation, runs away, unfolding
$\bangp{P}$ eagerly. A lazier and more implementable replication
operator, restricted to input-guarded processes, may be obtained as follows.

\begin{eqnarray}
\bangp{\prefix{u}{v}{P}} 
	:= 
	\binpar{\lift{x}{\prefix{u}{v}{(\binpar{D(x)}{P})}}}{D(x)} \nonumber
\end{eqnarray}

\begin{remark}
  Note that the lazier definition still does not deal with summation
  or mixed summation (i.e. sums over input and output). The reader is
  invited to construct definitions of replication that deal with these
  features. 

  Further, the definitions are parameterized in a name, $x$. Can you,
  gentle reader, make a definition that eliminates this parameter and
  guarantees no accidental interaction between the replication
  machinery and the process being replicated -- i.e. no accidental
  sharing of names used by the process to get its work done and the
  name(s) used by the replication to effect copying. This latter
  revision of the definition of replication is crucial to obtaining
  the expected identity $!!P \sim !P$.
\end{remark}

\begin{remark}\label{rem:paradoxical_combinator}
  The reader familiar with the lambda calculus will have noticed the
  similarity between $D$ and the paradoxical combinator.

  [Ed. note: the existence of this seems to suggest we have to be more
  restrictive on the set of processes and names we admit if we are to
  support no-cloning.]
\end{remark}

\subsubsection{Bisimulation}

The computational dynamics gives rise to another kind of equivalence,
the equivalence of computational behavior. As previously mentioned
this is typically captured \emph{via} some form of bisimulation.

% The notion we use in this paper is weak barbed bisimulation
% \cite{milner91polyadicpi}.

The notion we use in this paper is derived from weak barbed
bisimulation \cite{milner91polyadicpi}. 

\begin{definition}
An \emph{observation relation}, $\downarrow_{\mathcal N}$, over a set
of names, $\mathcal N$, is the smallest relation satisfying the rules
below.

\infrule[Out-barb]{y \in {\mathcal N}, \; x \nameeq y}
		  {\outputp{x}{v} \downarrow_{\mathcal N} x}
\infrule[Par-barb]{\mbox{$P\downarrow_{\mathcal N} x$ or $Q\downarrow_{\mathcal N} x$}}
		  {\binpar{P}{Q} \downarrow_{\mathcal N} x}

We write $P \Downarrow_{\mathcal N} x$ if there is $Q$ such that 
$P \wred Q$ and $Q \downarrow_{\mathcal N} x$.
\end{definition}

\begin{definition}
%\label{def.bbisim}
An  ${\mathcal N}$-\emph{barbed bisimulation} over a set of names, ${\mathcal N}$, is a symmetric binary relation 
${\mathcal S}_{\mathcal N}$ between agents such that $P\rel{S}_{\mathcal N}Q$ implies:
\begin{enumerate}
\item If $P \red P'$ then $Q \wred Q'$ and $P'\rel{S}_{\mathcal N} Q'$.
\item If $P\downarrow_{\mathcal N} x$, then $Q\Downarrow_{\mathcal N} x$.
\end{enumerate}
$P$ is ${\mathcal N}$-barbed bisimilar to $Q$, written
$P \wbbisim_{\mathcal N} Q$, if $P \rel{S}_{\mathcal N} Q$ for some ${\mathcal N}$-barbed bisimulation ${\mathcal S}_{\mathcal N}$.
\end{definition}

$\mathcal{R} \subseteq \pi \times \pi$

$P \mathcal{R} Q => \forall P'. P \red P' \Rightarrow \exists Q'. Q \red Q', P' \mathcal{R} Q'$

$P \vdash x \Rightarrow Q \vdash x$

\begin{mathpar}
  \inferrule*[lab=Out-barb]{x \nameeq y}{{y}!\langle{Q}\rangle \vdash x}
  \and
  \inferrule*[lab=Par-barb]{\mbox{$P\vdash x$ or $Q\vdash x$}}{\binpar{P}{Q} \vdash x}
\end{mathpar}

\subsubsection{Contexts}

One of the principle advantages of computational calculi like the
$\pi$-calculus is a well-defined notion of context,
contextual-equivalence and a correlation between
contextual-equivalence and notions of bisimulation. The notion of
context allows the decomposition of a process into (sub-)process and
its syntactic environment, its context. Thus, a context may be
thought of as a process with a ``hole'' (written $\Box$) in it. The
application of a context $M$ to a process $P$, written $M[P]$, is
tantamount to filling the hole in $M$ with $P$. In this paper we do
not need the full weight of this theory, but do make use of the notion
of context in the proof the main theorem. 

\begin{mathpar}
  \inferrule* [lab=summation] {} {{M_{M},M_{N}} \bc \Box \;|\; x.M_{A} \;|\; M_{M}+M_{N}}
  \and
  \inferrule* [lab=agent] {} {{M_{A}} \bc (\vec{x})M_{P} \;| \; \clift{P_0,\ldots,M_{P},\ldots,P_N}}
  \and \\
  \inferrule* [lab=process] {} {{M_{P}} \bc M_{N} \;| \;P|M_{P} }
\end{mathpar} 

\begin{mathpar}
  \inferrule* [lab=sychronization] {} {M_{N} \bc \Box \;|\; x?M_{F} \;|\; x!M_{C}}
  \and
  \inferrule* [lab=abstraction] {} {{M_{F}} \bc (x)M_{P} }
  \and
  \inferrule* [lab=concretion] {} {{M_{C}} \bc \langle M_{P} \rangle }
  \and \\
  \inferrule* [lab=process] {} {{M_{P}} \bc M_{N} \;| \;P|M_{P} }
\end{mathpar}

\begin{definition}[contextual application] Given a context $M$, and
  process $P$, we define the \emph{contextual application}, $M[P] :=
  M\{P/\Box\}$. That is, the contextual application of M to P is the
  substitution of $P$ for $\Box$ in $M$.
\end{definition}

$\meaningof{-} : L \to \mathcal{P}(\pi)$

\begin{mathpar}
  \inferrule* [lab=collection] {} {\meaningof{true} = \pi, \and \meaningof{~E} = \pi \setminus \meaningof{E}, \and \meaningof{E_{1} \& E_{2}} = \meaningof{E_{1}} \cap \meaningof{E_{2}}}
\end{mathpar}

\begin{mathpar}
  \inferrule* [lab=structure] {} {\meaningof{0} = \{ P \in \pi | P \equiv 0 \}, \and \\ \meaningof{E_1 | E_2} = \{ P \in \pi | P \equiv P_{1} | P_{2}, P_{1} \in \meaningof{E_{1}}, P_{2} \in \meaningof{E_2}\} }
\end{mathpar}

\begin{mathpar}
 \inferrule* [lab=behavior] {} {\meaningof{\langle a?b \rangle E} = \{ P \in \pi | P \equiv Q | u?(y)P', \\ \and \\\\ \and \\ \;\;\; u \in \meaningof{a}, \forall z.P'\{z/y\} \in \meaningof{E\{z/b\}}\}, \and \\ \meaningof{a!E} = \{ P \in \pi | P \equiv Q | x!\langle P' \rangle, x \in \meaningof{a} P' \in \meaningof{E}\} }
\end{mathpar}

\begin{mathpar}
 \inferrule* [lab=nominal] {} {\meaningof{\quotep{E}} = \{ \quotep{P} \in \quotep{\pi} | P \in \meaningof{E} \}, \and \meaningof{\quotep{P}} = \{ \quotep{Q} \in \quotep{\pi} | P \equiv Q \} \and \\ \meaningof{@\quotep{E}} = \{ P \in \pi | P \equiv @x, x \in \meaningof{E} \}}
\end{mathpar}

\begin{eqnarray*}
  \\
  \meaningof{-} : TS \to ST
\end{eqnarray*}

\begin{eqnarray*}
  \\
  L : TS \to ST
\end{eqnarray*}

\begin{eqnarray*}
  \\
  P \models E \iff P \in \meaningof{E}
\end{eqnarray*}

\begin{eqnarray*}
  P \approx_{L} Q \iff \forall E \in L. P \models E \iff Q \models E
\end{eqnarray*}

\begin{eqnarray*}
  P \approx_{K} Q
\end{eqnarray*}

\begin{eqnarray*}
  P \approx Q
\end{eqnarray*}

$\approx_{K} = \approx = \approx_{L}$

\subsubsection{Contextual duality}

Note that contexts extend the quotation operation to a family of
operations from processes to names. Given a context, $M$, we can
define a \emph{nominal context}, $\quotep{M}$ by $\quotep{M}[P] :=
\quotep{M[P]}$. To foreshadow what is to come we observe that these
operations enjoy a duality with processes very much like the duality
between vectors and maps from vectors to scalars.

Further, because the calculus is essentially higher-order, we have a
correspondence between contexts and processes. More specifically,
given a name $x$ and a context $M$ we can construct $M^{*}_{x}$ such
that 

\begin{mathpar}
  M^{*}_{x} | \lift{x}{P} \red M[P]
\end{mathpar}

namely,

\begin{mathpar}
  M^{*}_{x} := x?(u).M[\dropn{u}]
\end{mathpar}

The dependence of $M^{*}_{x}$ on a name makes it an abstraction, 

\begin{mathpar}
  M^{*} := (x)x?(u).M[\dropn{u}]
\end{mathpar}

\subsection{Additional notation}

It will sometimes be convenient to denote the process a name
quotes. We already have the notation $x = \quotep{P}$, but it will be
convenient to introduce an alternate notation, $\procn{x}$, when we
want to emphasize the connection to the use of the name. Note that, by
virtue of name equivalence, $\quotep{\procn{x}} \nameeq x$; so, the
notation is consistent with previous definitions.

Further, because names have structure it is possible to effect
substitutions on the basis of that structure. This means we need to
upgrade our notation for substitutions, which we accomplish by
adapting comprehension notation. Thus,

\begin{mathpar}
  P\{ y / x : x \in S \}
\end{mathpar}

is interpreted to mean the process derived from P by replacing (in a
capture-avoiding manner) each occurrence of $x$ in $S$ by $y$. For example,

\begin{mathpar}
  P\{ \quotep{\procn{x}|\procn{x}} / x : x \in \freenames{P} \}
\end{mathpar}

will replace each (occurrence) of a free name $x$ in $P$ by
$\quotep{\procn{x}|\procn{x}}$.

Also, we will avail ourselves of the notation $x^{L}$ and $x^{R}$ to
denote injections of a name into disjoint copies of the name
space. There are numerous ways to accomplish this. One example can be
found in \cite{MeredithR05}. This notation overloads to vectors of
names: $\vec{x}^{\pi} := (x_{i}^{\pi} \; : \; 0 \leq i < |\vec{x}| )$ where $\pi \in \{L,R\}$.

We also use $P^{\Box} := P|\Box$.

In \cite{MeredithR05} an interpretation of the new operator is
given. It turns out that there are several possible interpretations
all enjoying the requisite algebraic properties of the operator (see
\cite{milner91polyadicpi}). We will therefore make liberal use of
$(\nu\; \vec{x})P$.

% subsection the_syntax_and_semantics_of_the_notation_system (end)   

\input{qm2pi.qmops} 

\input{qm2pi.sterngerlach} 

\input{qm2pi.metric} 

% section concurrent_process_calculi (end)

%\input{qm2pi.proofsketch}

% section proof sketch (end)

%\input{qm2pi.slviaknots} 

% section spatial logic via knots (end)

\input{qm2pi.conclusion}

% section conclusion (end)

%\input{qm2pi.dtcodes} 

% section wiring algorithm (end)

\input{qm2pi.ack} 

% section acknowledgments (end)

\newpage


\bibliographystyle{plain}   
\bibliography{../../biblios/main.bib}

\input{qm2pi.rhodetails}

\end{document}

 

%\documentclass[12pt]{llncs}
%\documentclass{jktr}

\usepackage[pdftex]{hyperref}                   
\usepackage {listings}
\usepackage {mathpartir}
\usepackage{bcprules}
%\usepackage{listings}
                       
\usepackage{graphicx} 
%\usepackage[margins=2.5cm,nohead,nofoot]{geometry}
%\usepackage{geometry}
\usepackage{amsfonts}
\usepackage{amstext}
\usepackage{latexsym}
\usepackage{amssymb}
\usepackage{color}


%\include{myPreamble}
\include{qm2pi.local} 

%\ifpdf
%\usepackage[pdftex]{graphicx}
%\else
%\usepackage{graphicx}
%\fi

 % \ifpdf
%  \usepackage{pdfsync}
%  \if


%\title{Brief Article}
%\author{David F. Snyder}
%\author{L.G. Meredith}

%\address{Dept. of Math., Texas State University--San Marcos, San Marcos, TX 78666}
       
\pagestyle{empty}


\begin{document}

\lstset{language=[Objective]Caml,frame=shadowbox}

\input{qm2pi.front}

% section front matter (end)

\input{qm2pi.intro} 
 
% section introduction (end)

% \input{qm2pi.knotations} 

% section notation (end)

\input{qm2pi.process.calculi} 

% section concurrent_process_calculi_and_spatial_logics_ (end)
    
%\input{qm2pi.knots2pi} 

%\input{qm2pi.trefoil} 

%\input{qm2pi.mainthm} 

% subsection basic_interpretation (end)

%\input{qm2pi.rho.presentation} 
\subsection{The syntax and semantics of the notation system}\label{sub:the_syntax_and_semantics_of_the_notation_system} % (fold)

We now summarize a technical presentation of the calculus that
embodies our theory of dynamics. The typical presentation of such a
calculus follows the style of giving generators and relations on
them. The grammar, below, describing term constructors, freely
generates the set of processes, $\Proc$. This set is then quotiented
by a relation known as structural congruence and it is over this set
that the notion of dynamics is expressed. This presentation is
essentially that of \cite{MeredithR05} with the addition of
polyadicity and summation. For readability we have relegated some of
the technical subtleties to an appendix.

\subsubsection{Process grammar}\label{subsub:process_grammar}

\begin{mathpar}
  \inferrule* [lab=synchronization] {} {{M} \bc \pzero \;|\; x?F \;|\; x!C }
  \and
  \inferrule* [lab=abstraction] {} {{F} \bc (x)P}
  \and
  \inferrule* [lab=concretion] {} {{C} \bc \langle Q \rangle}
  \and
  \inferrule* [lab=process] {} {{P,Q} \bc M \;| \;P|Q \;|\; @{x}}
  \and
  \inferrule* [lab=name] {} {{x} \bc \quotep{P}}
\end{mathpar} 

Note that $\vec{x}$ (resp. $\vec{P}$) denotes a vector of names
(resp. processes) of length $|\vec{x}|$ (resp. $|\vec{P}|$). We adopt
the following useful abbreviations.

\begin{mathpar}
   x?(\vec{y}).P := x.(\vec{y})P \and  x\clift{\vec{P}} := x.\clift{\vec{P}}
   \and x!(y) := \lift{x}{\dropn{y}}
   \and \Pi_{i=0}^{n-1}P_i := P_0 | \ldots | P_{n-1}
\end{mathpar}

\subsubsection{Structural congruence}

\paragraph{Free and bound names and alpha-equivalence.} At the
core of structural equivalence is alpha-equivalence which identifies
process that are the same up to a change of variable. Formally, we
recognize the distinction between free and bound names. The free names
of a process, $\freenames{P}$, may be calculated recursively as
follows:

\begin{mathpar}
\freenames{\pzero} := \emptyset
  \and \\
  \freenames{x?(y).P} := \{ x \} \cup (\freenames{P} \setminus \{ y \})
  \and 
  \freenames{x!\langle P \rangle} := \{ x \} \cup \{ P \} 
  \and \\
  \freenames{P|Q} := \freenames{P} \cup \freenames{Q}
  \and \\
  \freenames{@{x}} := \{ x \}
\end{mathpar}

$\pi$
$\quotep{\pi}$

$\freenames{-} : \pi \to \mathcal{P}(\quotep{\pi})$

\begin{eqnarray*}
  \freenames{\pzero} & := & \emptyset \\
  \freenames{x?(y).P} & := & \{ x \} \cup (\freenames{P} \setminus \{ y \}) \\
  \freenames{x!\langle P \rangle} & := & \{ x \} \cup \{ P \} \\
  \freenames{P|Q} & := & \freenames{P} \cup \freenames{Q} \\
  \freenames{\dropn{x}} & := & \{ x \}
\end{eqnarray*}

The bound names of a process, $\boundnames{P}$, are those names occurring in $P$
that are not free. For example, in $x?(y).0$, the name $x$ is free, while $y$ is bound.

\begin{mathpar}
  \inferrule* [lab=monoidal-laws] {} { P|Q \equiv Q|P \and P|0 \equiv P \and P|(Q|R) \equiv (P|Q)|R }
\end{mathpar}

\begin{mathpar}
  \inferrule* [lab=alpha-equivalence] {} { (x)P \equiv (y)P\{y/x\} \and y \not\in \freenames{P} }
\end{mathpar}

\begin{definition}
Then two processes, $P,Q$, are alpha-equivalent if $P = Q\{\vec{y}/\vec{x}\}$ for
some $\vec{x} \in \boundnames{Q},\vec{y} \in \boundnames{P}$, where $Q\{\vec{y}/\vec{x}\}$
denotes the capture-avoiding substitution of $\vec{y}$ for $\vec{x}$ in $Q$.
\end{definition}

\begin{definition}
  The {\em structural congruence} \cite{SangiorgiWalker} , $\equiv$,
  between processes is the least congruence containing
  alpha-equivalence, satisfying the abelian monoid laws
  (associativity, commutativity and $\pzero$ as identity) for parallel
  composition $|$ and for summation $+$.
\end{definition}

\subsection{Name equivalence}

We take name equivalence, written $\nameeq$, to be the smallest
equivalence relation generated by the following rules.

\begin{mathpar}
\inferrule*[lab=Quote-drop]
{ }
{ \quotep{@{x}} \nameeq x }

\inferrule*[lab=Struct-equiv]
{ P \scong Q }
{ \quotep{P} \nameeq \quotep{Q} }
\end{mathpar}

The astute reader will have noticed that the mutual recursion of names
and processes imposes a mutual recursion on alpha-equivalence and
structural equivalence via name-equivalence. Fortunately, all of this
works out pleasantly and we may calculate in the natural way, free of
concern. The reader interested in the details is referred to the
appendix \ref{appendix:rho_details}.

\subsection{Substitution}

We use $\Proc$ for the set of processes, $\QProc$ for the set of
names, and $\id{\{}\vec{y} / \vec{x} \id{\}}$ to denote partial maps,
$s : \QProc \rightarrow \QProc$. A map, $s$ lifts, uniquely, to a map
on process terms, $\widehat{s} : \Proc \rightarrow \Proc$ by the
following equations.

\begin{mathpar}
  (0) \psubstp{Q}{P} := 0 \\
  (R \juxtap S) \psubstp{Q}{P}
  :=    
  (R)\psubstp{Q}{P} \juxtap (S) \psubstp{Q}{P} \\
  (x?(y).R) \psubstp{Q}{P}    
  :=    
  (x)\substp{Q}{P} (z)\concat( (R \psubstn{z}{y}) \psubstp{Q}{P} ) \\
  (\lift{x}{R}) \psubstp{Q}{P}  
  :=
  \lift{(x)\substp{Q}{P}}{ R \psubstp{Q}{P} } \\
%   (\dropn{x})  \psubstp{Q}{P}       
%   := 
%   \left\{ 
%     \begin{array}{ccc} 
%       \dropn{\quotep{Q}} & & x \nameeq \quotep{P} \\
%       \dropn{x} & & otherwise \\
%     \end{array}
%   \right. 
  (\dropn{x})  \psubstp{Q}{P}       
  := 
  \left\{ 
    \begin{array}{ccc} 
      Q & & x \nameeq \quotep{P} \\
      \dropn{x} & & otherwise \\
    \end{array}
  \right.
\end{mathpar}
 

where

\begin{eqnarray}
  (x)\id{\{} \lpquote Q \rpquote / \lpquote P \rpquote \id{\}}            = 
  \left\{ 
    \begin{array}{ccc}
      \lpquote Q \rpquote & & x \nameeq \lpquote P \rpquote \\
      x & & otherwise \\
    \end{array}
  \right. \nonumber
\end{eqnarray}

and $z$ is chosen distinct from $\quotep{P}$, $\quotep{Q}$, the free
names in $Q$, and all the names in $R$. Our $\alpha$-equivalence will
be built in the standard way from this substitution.

\begin{remark}\label{rem:no_self_referential_names}
  One consequence of these definitions is that $\forall P. \quotep{P}
  \not\in \freenames{P}$.
\end{remark}

\subsection{ Dynamic quote: an example }

Anticipating something of what's to come, consider applying the
substitution, $\widehat{\id{\{}u / z \id{\}}}$, to the following pair
of processes, $\lift{w}{y!(z)}$ and $w[ \lpquote y!(z) \rpquote ]$.

\begin{eqnarray}
	\lift{w}{y!(z)}\widehat{\id{\{}u / z \id{\}}}
		& = &
		\lift{w}{y!(u)} \nonumber\\
	w[ \lpquote y!(z) \rpquote ] \widehat{ \id{\{}u / z \id{\}} }
		& = &
		w[ \lpquote y!(z) \rpquote ] \nonumber
\end{eqnarray}

Because the body of the process between quotes is impervious to
substitution, we get radically different answers. In fact, by
examining the first process in an input context,
e.g. $x?(z).\lift{w}{y!(z)}$, we see that the process under the lift
operator may be shaped by prefixed inputs binding a name inside it. In
this sense, the lift operator will be seen as a way to dynamically
construct processes before reifying them as names.

Finally equipped with these standard features we can present the
dynamics of the calculus.

\subsubsection{Operational semantics} 

Finally, we introduce the computational dynamics. What marks these
algebras as distinct from other more traditionally studied algebraic
structures, e.g. vector spaces or polynomial rings, is the manner in
which dynamics is captured. In traditional structures, dynamics is typically
expressed through morphisms between such structures, as in linear maps
between vector spaces or morphisms between rings. In algebras
associated with the semantics of computation, the dynamics is
expressed as part of the algebraic structure itself, through a
reduction reduction relation typically denoted by $\red$. Below, we
give a recursive presentation of this relation for the calculus used
in the encoding.

$\red \subseteq \pi \times \pi$
$\red : \pi \to \mathcal{P}(\pi)$

\begin{mathpar}
  \inferrule* [lab=Comm] { \textsf{match}( x_{src}, x_{trgt} ) } { x_{trgt}?(y)P \; | \; x_{src}!\langle {Q} \rangle \red P\{\quotep{Q}/y}\} }
  \and \\
  \inferrule* [lab=Par] {{P} \red {P}'} {{{P} | {Q}} \red {{P}' | {Q}}}
  \and
  \inferrule* [lab=Equiv]{{{P} \scong {P}'} \andalso {{P}' \red {Q}'} \andalso {{Q}' \scong {Q}}}{{P} \red {Q}}
\end{mathpar}

\begin{eqnarray*}
  match_{\equiv} (\quotep{P},\quotep{Q}) & := & P \equiv Q \\
  match_{\dagger}(\quotep{P},\quotep{Q}) & := & \forall R. P|Q \red^{*} R => R \red^{*} 0 \\
  match_{K}(\quotep{P},\quotep{Q}) & := & K \mbox{ for some context } K
\end{eqnarray*}

$u?(x)P | u!\langle Q \rangle \red P\{\quotep{Q}/x\}$

%We write $\wred$ for $\red^*$, and $P\red$ if $\exists Q $ such that $ P \red Q$.
We write $P\red$ if $\exists Q $ such that $ P \red Q$ and $P\not\red$, otherwise.

\section{Replication}

As mentioned before, it is known that replication (and hence
recursion) can be implemented in a higher-order process algebra
\cite{SangiorgiWalker}. As our first example of calculation with the
machinery thus far presented we give the construction explicitly in
the {\rhoc}.

\begin{eqnarray}
	D_{x} & := & \prefix{x}{y}{(\binpar{\outputp{x}{y}}{@{y}})} \nonumber\\
	\bangp_{x}{P} & := & \binpar{{x}!\langle{\binpar{D_{x}}{P}}\rangle}{D_{x}} \nonumber
\end{eqnarray}

\begin{eqnarray}
	\bangp_{x}{P} & & \nonumber\\
	=
	& {x}!\langle{(\prefix{x}{y}{(\outputp{x}{y} | @{y})) | P}}\rangle 
	      | \prefix{x}{y}{(\outputp{x}{y} | @{y})} & \nonumber\\
	\red
	& (\outputp{x}{y} | @{y})\substn{\quotep{(\prefix{x}{y}{(@{y} | \outputp{x}{y})) | P}}}{y} & \nonumber\\
	=
	& \outputp{x}{\quotep{(\prefix{x}{y}{(\outputp{x}{y} | @{y})) | P}}}
	  | {(\prefix{x}{y}{(\outputp{x}{y} | @{y})) | P}} & \nonumber\\
	\red
	& \ldots & \nonumber\\
	\red^*
	& P | P | \ldots & \nonumber
\end{eqnarray}

Of course, this encoding, as an implementation, runs away, unfolding
$\bangp{P}$ eagerly. A lazier and more implementable replication
operator, restricted to input-guarded processes, may be obtained as follows.

\begin{eqnarray}
\bangp{\prefix{u}{v}{P}} 
	:= 
	\binpar{\lift{x}{\prefix{u}{v}{(\binpar{D(x)}{P})}}}{D(x)} \nonumber
\end{eqnarray}

\begin{remark}
  Note that the lazier definition still does not deal with summation
  or mixed summation (i.e. sums over input and output). The reader is
  invited to construct definitions of replication that deal with these
  features. 

  Further, the definitions are parameterized in a name, $x$. Can you,
  gentle reader, make a definition that eliminates this parameter and
  guarantees no accidental interaction between the replication
  machinery and the process being replicated -- i.e. no accidental
  sharing of names used by the process to get its work done and the
  name(s) used by the replication to effect copying. This latter
  revision of the definition of replication is crucial to obtaining
  the expected identity $!!P \sim !P$.
\end{remark}

\begin{remark}\label{rem:paradoxical_combinator}
  The reader familiar with the lambda calculus will have noticed the
  similarity between $D$ and the paradoxical combinator.

  [Ed. note: the existence of this seems to suggest we have to be more
  restrictive on the set of processes and names we admit if we are to
  support no-cloning.]
\end{remark}

\subsubsection{Bisimulation}

The computational dynamics gives rise to another kind of equivalence,
the equivalence of computational behavior. As previously mentioned
this is typically captured \emph{via} some form of bisimulation.

% The notion we use in this paper is weak barbed bisimulation
% \cite{milner91polyadicpi}.

The notion we use in this paper is derived from weak barbed
bisimulation \cite{milner91polyadicpi}. 

\begin{definition}
An \emph{observation relation}, $\downarrow_{\mathcal N}$, over a set
of names, $\mathcal N$, is the smallest relation satisfying the rules
below.

\infrule[Out-barb]{y \in {\mathcal N}, \; x \nameeq y}
		  {\outputp{x}{v} \downarrow_{\mathcal N} x}
\infrule[Par-barb]{\mbox{$P\downarrow_{\mathcal N} x$ or $Q\downarrow_{\mathcal N} x$}}
		  {\binpar{P}{Q} \downarrow_{\mathcal N} x}

We write $P \Downarrow_{\mathcal N} x$ if there is $Q$ such that 
$P \wred Q$ and $Q \downarrow_{\mathcal N} x$.
\end{definition}

\begin{definition}
%\label{def.bbisim}
An  ${\mathcal N}$-\emph{barbed bisimulation} over a set of names, ${\mathcal N}$, is a symmetric binary relation 
${\mathcal S}_{\mathcal N}$ between agents such that $P\rel{S}_{\mathcal N}Q$ implies:
\begin{enumerate}
\item If $P \red P'$ then $Q \wred Q'$ and $P'\rel{S}_{\mathcal N} Q'$.
\item If $P\downarrow_{\mathcal N} x$, then $Q\Downarrow_{\mathcal N} x$.
\end{enumerate}
$P$ is ${\mathcal N}$-barbed bisimilar to $Q$, written
$P \wbbisim_{\mathcal N} Q$, if $P \rel{S}_{\mathcal N} Q$ for some ${\mathcal N}$-barbed bisimulation ${\mathcal S}_{\mathcal N}$.
\end{definition}

$\mathcal{R} \subseteq \pi \times \pi$

$P \mathcal{R} Q => \forall P'. P \red P' \Rightarrow \exists Q'. Q \red Q', P' \mathcal{R} Q'$

$P \vdash x \Rightarrow Q \vdash x$

\begin{mathpar}
  \inferrule*[lab=Out-barb]{x \nameeq y}{{y}!\langle{Q}\rangle \vdash x}
  \and
  \inferrule*[lab=Par-barb]{\mbox{$P\vdash x$ or $Q\vdash x$}}{\binpar{P}{Q} \vdash x}
\end{mathpar}

\subsubsection{Contexts}

One of the principle advantages of computational calculi like the
$\pi$-calculus is a well-defined notion of context,
contextual-equivalence and a correlation between
contextual-equivalence and notions of bisimulation. The notion of
context allows the decomposition of a process into (sub-)process and
its syntactic environment, its context. Thus, a context may be
thought of as a process with a ``hole'' (written $\Box$) in it. The
application of a context $M$ to a process $P$, written $M[P]$, is
tantamount to filling the hole in $M$ with $P$. In this paper we do
not need the full weight of this theory, but do make use of the notion
of context in the proof the main theorem. 

\begin{mathpar}
  \inferrule* [lab=summation] {} {{M_{M},M_{N}} \bc \Box \;|\; x.M_{A} \;|\; M_{M}+M_{N}}
  \and
  \inferrule* [lab=agent] {} {{M_{A}} \bc (\vec{x})M_{P} \;| \; \clift{P_0,\ldots,M_{P},\ldots,P_N}}
  \and \\
  \inferrule* [lab=process] {} {{M_{P}} \bc M_{N} \;| \;P|M_{P} }
\end{mathpar} 

\begin{mathpar}
  \inferrule* [lab=sychronization] {} {M_{N} \bc \Box \;|\; x?M_{F} \;|\; x!M_{C}}
  \and
  \inferrule* [lab=abstraction] {} {{M_{F}} \bc (x)M_{P} }
  \and
  \inferrule* [lab=concretion] {} {{M_{C}} \bc \langle M_{P} \rangle }
  \and \\
  \inferrule* [lab=process] {} {{M_{P}} \bc M_{N} \;| \;P|M_{P} }
\end{mathpar}

\begin{definition}[contextual application] Given a context $M$, and
  process $P$, we define the \emph{contextual application}, $M[P] :=
  M\{P/\Box\}$. That is, the contextual application of M to P is the
  substitution of $P$ for $\Box$ in $M$.
\end{definition}

$\meaningof{-} : L \to \mathcal{P}(\pi)$

\begin{mathpar}
  \inferrule* [lab=collection] {} {\meaningof{true} = \pi, \and \meaningof{~E} = \pi \setminus \meaningof{E}, \and \meaningof{E_{1} \& E_{2}} = \meaningof{E_{1}} \cap \meaningof{E_{2}}}
\end{mathpar}

\begin{mathpar}
  \inferrule* [lab=structure] {} {\meaningof{0} = \{ P \in \pi | P \equiv 0 \}, \and \\ \meaningof{E_1 | E_2} = \{ P \in \pi | P \equiv P_{1} | P_{2}, P_{1} \in \meaningof{E_{1}}, P_{2} \in \meaningof{E_2}\} }
\end{mathpar}

\begin{mathpar}
 \inferrule* [lab=behavior] {} {\meaningof{\langle a?b \rangle E} = \{ P \in \pi | P \equiv Q | u?(y)P', \\ \and \\\\ \and \\ \;\;\; u \in \meaningof{a}, \forall z.P'\{z/y\} \in \meaningof{E\{z/b\}}\}, \and \\ \meaningof{a!E} = \{ P \in \pi | P \equiv Q | x!\langle P' \rangle, x \in \meaningof{a} P' \in \meaningof{E}\} }
\end{mathpar}

\begin{mathpar}
 \inferrule* [lab=nominal] {} {\meaningof{\quotep{E}} = \{ \quotep{P} \in \quotep{\pi} | P \in \meaningof{E} \}, \and \meaningof{\quotep{P}} = \{ \quotep{Q} \in \quotep{\pi} | P \equiv Q \} \and \\ \meaningof{@\quotep{E}} = \{ P \in \pi | P \equiv @x, x \in \meaningof{E} \}}
\end{mathpar}

\begin{eqnarray*}
  \\
  \meaningof{-} : TS \to ST
\end{eqnarray*}

\begin{eqnarray*}
  \\
  L : TS \to ST
\end{eqnarray*}

\begin{eqnarray*}
  \\
  P \models E \iff P \in \meaningof{E}
\end{eqnarray*}

\begin{eqnarray*}
  P \approx_{L} Q \iff \forall E \in L. P \models E \iff Q \models E
\end{eqnarray*}

\begin{eqnarray*}
  P \approx_{K} Q
\end{eqnarray*}

\begin{eqnarray*}
  P \approx Q
\end{eqnarray*}

$\approx_{K} = \approx = \approx_{L}$

\subsubsection{Contextual duality}

Note that contexts extend the quotation operation to a family of
operations from processes to names. Given a context, $M$, we can
define a \emph{nominal context}, $\quotep{M}$ by $\quotep{M}[P] :=
\quotep{M[P]}$. To foreshadow what is to come we observe that these
operations enjoy a duality with processes very much like the duality
between vectors and maps from vectors to scalars.

Further, because the calculus is essentially higher-order, we have a
correspondence between contexts and processes. More specifically,
given a name $x$ and a context $M$ we can construct $M^{*}_{x}$ such
that 

\begin{mathpar}
  M^{*}_{x} | \lift{x}{P} \red M[P]
\end{mathpar}

namely,

\begin{mathpar}
  M^{*}_{x} := x?(u).M[\dropn{u}]
\end{mathpar}

The dependence of $M^{*}_{x}$ on a name makes it an abstraction, 

\begin{mathpar}
  M^{*} := (x)x?(u).M[\dropn{u}]
\end{mathpar}

\subsection{Additional notation}

It will sometimes be convenient to denote the process a name
quotes. We already have the notation $x = \quotep{P}$, but it will be
convenient to introduce an alternate notation, $\procn{x}$, when we
want to emphasize the connection to the use of the name. Note that, by
virtue of name equivalence, $\quotep{\procn{x}} \nameeq x$; so, the
notation is consistent with previous definitions.

Further, because names have structure it is possible to effect
substitutions on the basis of that structure. This means we need to
upgrade our notation for substitutions, which we accomplish by
adapting comprehension notation. Thus,

\begin{mathpar}
  P\{ y / x : x \in S \}
\end{mathpar}

is interpreted to mean the process derived from P by replacing (in a
capture-avoiding manner) each occurrence of $x$ in $S$ by $y$. For example,

\begin{mathpar}
  P\{ \quotep{\procn{x}|\procn{x}} / x : x \in \freenames{P} \}
\end{mathpar}

will replace each (occurrence) of a free name $x$ in $P$ by
$\quotep{\procn{x}|\procn{x}}$.

Also, we will avail ourselves of the notation $x^{L}$ and $x^{R}$ to
denote injections of a name into disjoint copies of the name
space. There are numerous ways to accomplish this. One example can be
found in \cite{MeredithR05}. This notation overloads to vectors of
names: $\vec{x}^{\pi} := (x_{i}^{\pi} \; : \; 0 \leq i < |\vec{x}| )$ where $\pi \in \{L,R\}$.

We also use $P^{\Box} := P|\Box$.

In \cite{MeredithR05} an interpretation of the new operator is
given. It turns out that there are several possible interpretations
all enjoying the requisite algebraic properties of the operator (see
\cite{milner91polyadicpi}). We will therefore make liberal use of
$(\nu\; \vec{x})P$.

% subsection the_syntax_and_semantics_of_the_notation_system (end)   

\input{qm2pi.qmops} 

\input{qm2pi.sterngerlach} 

\input{qm2pi.metric} 

% section concurrent_process_calculi (end)

%\input{qm2pi.proofsketch}

% section proof sketch (end)

%\input{qm2pi.slviaknots} 

% section spatial logic via knots (end)

\input{qm2pi.conclusion}

% section conclusion (end)

%\input{qm2pi.dtcodes} 

% section wiring algorithm (end)

\input{qm2pi.ack} 

% section acknowledgments (end)

\newpage


\bibliographystyle{plain}   
\bibliography{../../biblios/main.bib}

\input{qm2pi.rhodetails}

\end{document}

 

%\documentclass[12pt]{llncs}
%\documentclass{jktr}

\usepackage[pdftex]{hyperref}                   
\usepackage {listings}
\usepackage {mathpartir}
\usepackage{bcprules}
%\usepackage{listings}
                       
\usepackage{graphicx} 
%\usepackage[margins=2.5cm,nohead,nofoot]{geometry}
%\usepackage{geometry}
\usepackage{amsfonts}
\usepackage{amstext}
\usepackage{latexsym}
\usepackage{amssymb}
\usepackage{color}


%\include{myPreamble}
\include{qm2pi.local} 

%\ifpdf
%\usepackage[pdftex]{graphicx}
%\else
%\usepackage{graphicx}
%\fi

 % \ifpdf
%  \usepackage{pdfsync}
%  \if


%\title{Brief Article}
%\author{David F. Snyder}
%\author{L.G. Meredith}

%\address{Dept. of Math., Texas State University--San Marcos, San Marcos, TX 78666}
       
\pagestyle{empty}


\begin{document}

\lstset{language=[Objective]Caml,frame=shadowbox}

\input{qm2pi.front}

% section front matter (end)

\input{qm2pi.intro} 
 
% section introduction (end)

% \input{qm2pi.knotations} 

% section notation (end)

\input{qm2pi.process.calculi} 

% section concurrent_process_calculi_and_spatial_logics_ (end)
    
%\input{qm2pi.knots2pi} 

%\input{qm2pi.trefoil} 

%\input{qm2pi.mainthm} 

% subsection basic_interpretation (end)

%\input{qm2pi.rho.presentation} 
\subsection{The syntax and semantics of the notation system}\label{sub:the_syntax_and_semantics_of_the_notation_system} % (fold)

We now summarize a technical presentation of the calculus that
embodies our theory of dynamics. The typical presentation of such a
calculus follows the style of giving generators and relations on
them. The grammar, below, describing term constructors, freely
generates the set of processes, $\Proc$. This set is then quotiented
by a relation known as structural congruence and it is over this set
that the notion of dynamics is expressed. This presentation is
essentially that of \cite{MeredithR05} with the addition of
polyadicity and summation. For readability we have relegated some of
the technical subtleties to an appendix.

\subsubsection{Process grammar}\label{subsub:process_grammar}

\begin{mathpar}
  \inferrule* [lab=synchronization] {} {{M} \bc \pzero \;|\; x?F \;|\; x!C }
  \and
  \inferrule* [lab=abstraction] {} {{F} \bc (x)P}
  \and
  \inferrule* [lab=concretion] {} {{C} \bc \langle Q \rangle}
  \and
  \inferrule* [lab=process] {} {{P,Q} \bc M \;| \;P|Q \;|\; @{x}}
  \and
  \inferrule* [lab=name] {} {{x} \bc \quotep{P}}
\end{mathpar} 

Note that $\vec{x}$ (resp. $\vec{P}$) denotes a vector of names
(resp. processes) of length $|\vec{x}|$ (resp. $|\vec{P}|$). We adopt
the following useful abbreviations.

\begin{mathpar}
   x?(\vec{y}).P := x.(\vec{y})P \and  x\clift{\vec{P}} := x.\clift{\vec{P}}
   \and x!(y) := \lift{x}{\dropn{y}}
   \and \Pi_{i=0}^{n-1}P_i := P_0 | \ldots | P_{n-1}
\end{mathpar}

\subsubsection{Structural congruence}

\paragraph{Free and bound names and alpha-equivalence.} At the
core of structural equivalence is alpha-equivalence which identifies
process that are the same up to a change of variable. Formally, we
recognize the distinction between free and bound names. The free names
of a process, $\freenames{P}$, may be calculated recursively as
follows:

\begin{mathpar}
\freenames{\pzero} := \emptyset
  \and \\
  \freenames{x?(y).P} := \{ x \} \cup (\freenames{P} \setminus \{ y \})
  \and 
  \freenames{x!\langle P \rangle} := \{ x \} \cup \{ P \} 
  \and \\
  \freenames{P|Q} := \freenames{P} \cup \freenames{Q}
  \and \\
  \freenames{@{x}} := \{ x \}
\end{mathpar}

$\pi$
$\quotep{\pi}$

$\freenames{-} : \pi \to \mathcal{P}(\quotep{\pi})$

\begin{eqnarray*}
  \freenames{\pzero} & := & \emptyset \\
  \freenames{x?(y).P} & := & \{ x \} \cup (\freenames{P} \setminus \{ y \}) \\
  \freenames{x!\langle P \rangle} & := & \{ x \} \cup \{ P \} \\
  \freenames{P|Q} & := & \freenames{P} \cup \freenames{Q} \\
  \freenames{\dropn{x}} & := & \{ x \}
\end{eqnarray*}

The bound names of a process, $\boundnames{P}$, are those names occurring in $P$
that are not free. For example, in $x?(y).0$, the name $x$ is free, while $y$ is bound.

\begin{mathpar}
  \inferrule* [lab=monoidal-laws] {} { P|Q \equiv Q|P \and P|0 \equiv P \and P|(Q|R) \equiv (P|Q)|R }
\end{mathpar}

\begin{mathpar}
  \inferrule* [lab=alpha-equivalence] {} { (x)P \equiv (y)P\{y/x\} \and y \not\in \freenames{P} }
\end{mathpar}

\begin{definition}
Then two processes, $P,Q$, are alpha-equivalent if $P = Q\{\vec{y}/\vec{x}\}$ for
some $\vec{x} \in \boundnames{Q},\vec{y} \in \boundnames{P}$, where $Q\{\vec{y}/\vec{x}\}$
denotes the capture-avoiding substitution of $\vec{y}$ for $\vec{x}$ in $Q$.
\end{definition}

\begin{definition}
  The {\em structural congruence} \cite{SangiorgiWalker} , $\equiv$,
  between processes is the least congruence containing
  alpha-equivalence, satisfying the abelian monoid laws
  (associativity, commutativity and $\pzero$ as identity) for parallel
  composition $|$ and for summation $+$.
\end{definition}

\subsection{Name equivalence}

We take name equivalence, written $\nameeq$, to be the smallest
equivalence relation generated by the following rules.

\begin{mathpar}
\inferrule*[lab=Quote-drop]
{ }
{ \quotep{@{x}} \nameeq x }

\inferrule*[lab=Struct-equiv]
{ P \scong Q }
{ \quotep{P} \nameeq \quotep{Q} }
\end{mathpar}

The astute reader will have noticed that the mutual recursion of names
and processes imposes a mutual recursion on alpha-equivalence and
structural equivalence via name-equivalence. Fortunately, all of this
works out pleasantly and we may calculate in the natural way, free of
concern. The reader interested in the details is referred to the
appendix \ref{appendix:rho_details}.

\subsection{Substitution}

We use $\Proc$ for the set of processes, $\QProc$ for the set of
names, and $\id{\{}\vec{y} / \vec{x} \id{\}}$ to denote partial maps,
$s : \QProc \rightarrow \QProc$. A map, $s$ lifts, uniquely, to a map
on process terms, $\widehat{s} : \Proc \rightarrow \Proc$ by the
following equations.

\begin{mathpar}
  (0) \psubstp{Q}{P} := 0 \\
  (R \juxtap S) \psubstp{Q}{P}
  :=    
  (R)\psubstp{Q}{P} \juxtap (S) \psubstp{Q}{P} \\
  (x?(y).R) \psubstp{Q}{P}    
  :=    
  (x)\substp{Q}{P} (z)\concat( (R \psubstn{z}{y}) \psubstp{Q}{P} ) \\
  (\lift{x}{R}) \psubstp{Q}{P}  
  :=
  \lift{(x)\substp{Q}{P}}{ R \psubstp{Q}{P} } \\
%   (\dropn{x})  \psubstp{Q}{P}       
%   := 
%   \left\{ 
%     \begin{array}{ccc} 
%       \dropn{\quotep{Q}} & & x \nameeq \quotep{P} \\
%       \dropn{x} & & otherwise \\
%     \end{array}
%   \right. 
  (\dropn{x})  \psubstp{Q}{P}       
  := 
  \left\{ 
    \begin{array}{ccc} 
      Q & & x \nameeq \quotep{P} \\
      \dropn{x} & & otherwise \\
    \end{array}
  \right.
\end{mathpar}
 

where

\begin{eqnarray}
  (x)\id{\{} \lpquote Q \rpquote / \lpquote P \rpquote \id{\}}            = 
  \left\{ 
    \begin{array}{ccc}
      \lpquote Q \rpquote & & x \nameeq \lpquote P \rpquote \\
      x & & otherwise \\
    \end{array}
  \right. \nonumber
\end{eqnarray}

and $z$ is chosen distinct from $\quotep{P}$, $\quotep{Q}$, the free
names in $Q$, and all the names in $R$. Our $\alpha$-equivalence will
be built in the standard way from this substitution.

\begin{remark}\label{rem:no_self_referential_names}
  One consequence of these definitions is that $\forall P. \quotep{P}
  \not\in \freenames{P}$.
\end{remark}

\subsection{ Dynamic quote: an example }

Anticipating something of what's to come, consider applying the
substitution, $\widehat{\id{\{}u / z \id{\}}}$, to the following pair
of processes, $\lift{w}{y!(z)}$ and $w[ \lpquote y!(z) \rpquote ]$.

\begin{eqnarray}
	\lift{w}{y!(z)}\widehat{\id{\{}u / z \id{\}}}
		& = &
		\lift{w}{y!(u)} \nonumber\\
	w[ \lpquote y!(z) \rpquote ] \widehat{ \id{\{}u / z \id{\}} }
		& = &
		w[ \lpquote y!(z) \rpquote ] \nonumber
\end{eqnarray}

Because the body of the process between quotes is impervious to
substitution, we get radically different answers. In fact, by
examining the first process in an input context,
e.g. $x?(z).\lift{w}{y!(z)}$, we see that the process under the lift
operator may be shaped by prefixed inputs binding a name inside it. In
this sense, the lift operator will be seen as a way to dynamically
construct processes before reifying them as names.

Finally equipped with these standard features we can present the
dynamics of the calculus.

\subsubsection{Operational semantics} 

Finally, we introduce the computational dynamics. What marks these
algebras as distinct from other more traditionally studied algebraic
structures, e.g. vector spaces or polynomial rings, is the manner in
which dynamics is captured. In traditional structures, dynamics is typically
expressed through morphisms between such structures, as in linear maps
between vector spaces or morphisms between rings. In algebras
associated with the semantics of computation, the dynamics is
expressed as part of the algebraic structure itself, through a
reduction reduction relation typically denoted by $\red$. Below, we
give a recursive presentation of this relation for the calculus used
in the encoding.

$\red \subseteq \pi \times \pi$
$\red : \pi \to \mathcal{P}(\pi)$

\begin{mathpar}
  \inferrule* [lab=Comm] { \textsf{match}( x_{src}, x_{trgt} ) } { x_{trgt}?(y)P \; | \; x_{src}!\langle {Q} \rangle \red P\{\quotep{Q}/y}\} }
  \and \\
  \inferrule* [lab=Par] {{P} \red {P}'} {{{P} | {Q}} \red {{P}' | {Q}}}
  \and
  \inferrule* [lab=Equiv]{{{P} \scong {P}'} \andalso {{P}' \red {Q}'} \andalso {{Q}' \scong {Q}}}{{P} \red {Q}}
\end{mathpar}

\begin{eqnarray*}
  match_{\equiv} (\quotep{P},\quotep{Q}) & := & P \equiv Q \\
  match_{\dagger}(\quotep{P},\quotep{Q}) & := & \forall R. P|Q \red^{*} R => R \red^{*} 0 \\
  match_{K}(\quotep{P},\quotep{Q}) & := & K \mbox{ for some context } K
\end{eqnarray*}

$u?(x)P | u!\langle Q \rangle \red P\{\quotep{Q}/x\}$

%We write $\wred$ for $\red^*$, and $P\red$ if $\exists Q $ such that $ P \red Q$.
We write $P\red$ if $\exists Q $ such that $ P \red Q$ and $P\not\red$, otherwise.

\section{Replication}

As mentioned before, it is known that replication (and hence
recursion) can be implemented in a higher-order process algebra
\cite{SangiorgiWalker}. As our first example of calculation with the
machinery thus far presented we give the construction explicitly in
the {\rhoc}.

\begin{eqnarray}
	D_{x} & := & \prefix{x}{y}{(\binpar{\outputp{x}{y}}{@{y}})} \nonumber\\
	\bangp_{x}{P} & := & \binpar{{x}!\langle{\binpar{D_{x}}{P}}\rangle}{D_{x}} \nonumber
\end{eqnarray}

\begin{eqnarray}
	\bangp_{x}{P} & & \nonumber\\
	=
	& {x}!\langle{(\prefix{x}{y}{(\outputp{x}{y} | @{y})) | P}}\rangle 
	      | \prefix{x}{y}{(\outputp{x}{y} | @{y})} & \nonumber\\
	\red
	& (\outputp{x}{y} | @{y})\substn{\quotep{(\prefix{x}{y}{(@{y} | \outputp{x}{y})) | P}}}{y} & \nonumber\\
	=
	& \outputp{x}{\quotep{(\prefix{x}{y}{(\outputp{x}{y} | @{y})) | P}}}
	  | {(\prefix{x}{y}{(\outputp{x}{y} | @{y})) | P}} & \nonumber\\
	\red
	& \ldots & \nonumber\\
	\red^*
	& P | P | \ldots & \nonumber
\end{eqnarray}

Of course, this encoding, as an implementation, runs away, unfolding
$\bangp{P}$ eagerly. A lazier and more implementable replication
operator, restricted to input-guarded processes, may be obtained as follows.

\begin{eqnarray}
\bangp{\prefix{u}{v}{P}} 
	:= 
	\binpar{\lift{x}{\prefix{u}{v}{(\binpar{D(x)}{P})}}}{D(x)} \nonumber
\end{eqnarray}

\begin{remark}
  Note that the lazier definition still does not deal with summation
  or mixed summation (i.e. sums over input and output). The reader is
  invited to construct definitions of replication that deal with these
  features. 

  Further, the definitions are parameterized in a name, $x$. Can you,
  gentle reader, make a definition that eliminates this parameter and
  guarantees no accidental interaction between the replication
  machinery and the process being replicated -- i.e. no accidental
  sharing of names used by the process to get its work done and the
  name(s) used by the replication to effect copying. This latter
  revision of the definition of replication is crucial to obtaining
  the expected identity $!!P \sim !P$.
\end{remark}

\begin{remark}\label{rem:paradoxical_combinator}
  The reader familiar with the lambda calculus will have noticed the
  similarity between $D$ and the paradoxical combinator.

  [Ed. note: the existence of this seems to suggest we have to be more
  restrictive on the set of processes and names we admit if we are to
  support no-cloning.]
\end{remark}

\subsubsection{Bisimulation}

The computational dynamics gives rise to another kind of equivalence,
the equivalence of computational behavior. As previously mentioned
this is typically captured \emph{via} some form of bisimulation.

% The notion we use in this paper is weak barbed bisimulation
% \cite{milner91polyadicpi}.

The notion we use in this paper is derived from weak barbed
bisimulation \cite{milner91polyadicpi}. 

\begin{definition}
An \emph{observation relation}, $\downarrow_{\mathcal N}$, over a set
of names, $\mathcal N$, is the smallest relation satisfying the rules
below.

\infrule[Out-barb]{y \in {\mathcal N}, \; x \nameeq y}
		  {\outputp{x}{v} \downarrow_{\mathcal N} x}
\infrule[Par-barb]{\mbox{$P\downarrow_{\mathcal N} x$ or $Q\downarrow_{\mathcal N} x$}}
		  {\binpar{P}{Q} \downarrow_{\mathcal N} x}

We write $P \Downarrow_{\mathcal N} x$ if there is $Q$ such that 
$P \wred Q$ and $Q \downarrow_{\mathcal N} x$.
\end{definition}

\begin{definition}
%\label{def.bbisim}
An  ${\mathcal N}$-\emph{barbed bisimulation} over a set of names, ${\mathcal N}$, is a symmetric binary relation 
${\mathcal S}_{\mathcal N}$ between agents such that $P\rel{S}_{\mathcal N}Q$ implies:
\begin{enumerate}
\item If $P \red P'$ then $Q \wred Q'$ and $P'\rel{S}_{\mathcal N} Q'$.
\item If $P\downarrow_{\mathcal N} x$, then $Q\Downarrow_{\mathcal N} x$.
\end{enumerate}
$P$ is ${\mathcal N}$-barbed bisimilar to $Q$, written
$P \wbbisim_{\mathcal N} Q$, if $P \rel{S}_{\mathcal N} Q$ for some ${\mathcal N}$-barbed bisimulation ${\mathcal S}_{\mathcal N}$.
\end{definition}

$\mathcal{R} \subseteq \pi \times \pi$

$P \mathcal{R} Q => \forall P'. P \red P' \Rightarrow \exists Q'. Q \red Q', P' \mathcal{R} Q'$

$P \vdash x \Rightarrow Q \vdash x$

\begin{mathpar}
  \inferrule*[lab=Out-barb]{x \nameeq y}{{y}!\langle{Q}\rangle \vdash x}
  \and
  \inferrule*[lab=Par-barb]{\mbox{$P\vdash x$ or $Q\vdash x$}}{\binpar{P}{Q} \vdash x}
\end{mathpar}

\subsubsection{Contexts}

One of the principle advantages of computational calculi like the
$\pi$-calculus is a well-defined notion of context,
contextual-equivalence and a correlation between
contextual-equivalence and notions of bisimulation. The notion of
context allows the decomposition of a process into (sub-)process and
its syntactic environment, its context. Thus, a context may be
thought of as a process with a ``hole'' (written $\Box$) in it. The
application of a context $M$ to a process $P$, written $M[P]$, is
tantamount to filling the hole in $M$ with $P$. In this paper we do
not need the full weight of this theory, but do make use of the notion
of context in the proof the main theorem. 

\begin{mathpar}
  \inferrule* [lab=summation] {} {{M_{M},M_{N}} \bc \Box \;|\; x.M_{A} \;|\; M_{M}+M_{N}}
  \and
  \inferrule* [lab=agent] {} {{M_{A}} \bc (\vec{x})M_{P} \;| \; \clift{P_0,\ldots,M_{P},\ldots,P_N}}
  \and \\
  \inferrule* [lab=process] {} {{M_{P}} \bc M_{N} \;| \;P|M_{P} }
\end{mathpar} 

\begin{mathpar}
  \inferrule* [lab=sychronization] {} {M_{N} \bc \Box \;|\; x?M_{F} \;|\; x!M_{C}}
  \and
  \inferrule* [lab=abstraction] {} {{M_{F}} \bc (x)M_{P} }
  \and
  \inferrule* [lab=concretion] {} {{M_{C}} \bc \langle M_{P} \rangle }
  \and \\
  \inferrule* [lab=process] {} {{M_{P}} \bc M_{N} \;| \;P|M_{P} }
\end{mathpar}

\begin{definition}[contextual application] Given a context $M$, and
  process $P$, we define the \emph{contextual application}, $M[P] :=
  M\{P/\Box\}$. That is, the contextual application of M to P is the
  substitution of $P$ for $\Box$ in $M$.
\end{definition}

$\meaningof{-} : L \to \mathcal{P}(\pi)$

\begin{mathpar}
  \inferrule* [lab=collection] {} {\meaningof{true} = \pi, \and \meaningof{~E} = \pi \setminus \meaningof{E}, \and \meaningof{E_{1} \& E_{2}} = \meaningof{E_{1}} \cap \meaningof{E_{2}}}
\end{mathpar}

\begin{mathpar}
  \inferrule* [lab=structure] {} {\meaningof{0} = \{ P \in \pi | P \equiv 0 \}, \and \\ \meaningof{E_1 | E_2} = \{ P \in \pi | P \equiv P_{1} | P_{2}, P_{1} \in \meaningof{E_{1}}, P_{2} \in \meaningof{E_2}\} }
\end{mathpar}

\begin{mathpar}
 \inferrule* [lab=behavior] {} {\meaningof{\langle a?b \rangle E} = \{ P \in \pi | P \equiv Q | u?(y)P', \\ \and \\\\ \and \\ \;\;\; u \in \meaningof{a}, \forall z.P'\{z/y\} \in \meaningof{E\{z/b\}}\}, \and \\ \meaningof{a!E} = \{ P \in \pi | P \equiv Q | x!\langle P' \rangle, x \in \meaningof{a} P' \in \meaningof{E}\} }
\end{mathpar}

\begin{mathpar}
 \inferrule* [lab=nominal] {} {\meaningof{\quotep{E}} = \{ \quotep{P} \in \quotep{\pi} | P \in \meaningof{E} \}, \and \meaningof{\quotep{P}} = \{ \quotep{Q} \in \quotep{\pi} | P \equiv Q \} \and \\ \meaningof{@\quotep{E}} = \{ P \in \pi | P \equiv @x, x \in \meaningof{E} \}}
\end{mathpar}

\begin{eqnarray*}
  \\
  \meaningof{-} : TS \to ST
\end{eqnarray*}

\begin{eqnarray*}
  \\
  L : TS \to ST
\end{eqnarray*}

\begin{eqnarray*}
  \\
  P \models E \iff P \in \meaningof{E}
\end{eqnarray*}

\begin{eqnarray*}
  P \approx_{L} Q \iff \forall E \in L. P \models E \iff Q \models E
\end{eqnarray*}

\begin{eqnarray*}
  P \approx_{K} Q
\end{eqnarray*}

\begin{eqnarray*}
  P \approx Q
\end{eqnarray*}

$\approx_{K} = \approx = \approx_{L}$

\subsubsection{Contextual duality}

Note that contexts extend the quotation operation to a family of
operations from processes to names. Given a context, $M$, we can
define a \emph{nominal context}, $\quotep{M}$ by $\quotep{M}[P] :=
\quotep{M[P]}$. To foreshadow what is to come we observe that these
operations enjoy a duality with processes very much like the duality
between vectors and maps from vectors to scalars.

Further, because the calculus is essentially higher-order, we have a
correspondence between contexts and processes. More specifically,
given a name $x$ and a context $M$ we can construct $M^{*}_{x}$ such
that 

\begin{mathpar}
  M^{*}_{x} | \lift{x}{P} \red M[P]
\end{mathpar}

namely,

\begin{mathpar}
  M^{*}_{x} := x?(u).M[\dropn{u}]
\end{mathpar}

The dependence of $M^{*}_{x}$ on a name makes it an abstraction, 

\begin{mathpar}
  M^{*} := (x)x?(u).M[\dropn{u}]
\end{mathpar}

\subsection{Additional notation}

It will sometimes be convenient to denote the process a name
quotes. We already have the notation $x = \quotep{P}$, but it will be
convenient to introduce an alternate notation, $\procn{x}$, when we
want to emphasize the connection to the use of the name. Note that, by
virtue of name equivalence, $\quotep{\procn{x}} \nameeq x$; so, the
notation is consistent with previous definitions.

Further, because names have structure it is possible to effect
substitutions on the basis of that structure. This means we need to
upgrade our notation for substitutions, which we accomplish by
adapting comprehension notation. Thus,

\begin{mathpar}
  P\{ y / x : x \in S \}
\end{mathpar}

is interpreted to mean the process derived from P by replacing (in a
capture-avoiding manner) each occurrence of $x$ in $S$ by $y$. For example,

\begin{mathpar}
  P\{ \quotep{\procn{x}|\procn{x}} / x : x \in \freenames{P} \}
\end{mathpar}

will replace each (occurrence) of a free name $x$ in $P$ by
$\quotep{\procn{x}|\procn{x}}$.

Also, we will avail ourselves of the notation $x^{L}$ and $x^{R}$ to
denote injections of a name into disjoint copies of the name
space. There are numerous ways to accomplish this. One example can be
found in \cite{MeredithR05}. This notation overloads to vectors of
names: $\vec{x}^{\pi} := (x_{i}^{\pi} \; : \; 0 \leq i < |\vec{x}| )$ where $\pi \in \{L,R\}$.

We also use $P^{\Box} := P|\Box$.

In \cite{MeredithR05} an interpretation of the new operator is
given. It turns out that there are several possible interpretations
all enjoying the requisite algebraic properties of the operator (see
\cite{milner91polyadicpi}). We will therefore make liberal use of
$(\nu\; \vec{x})P$.

% subsection the_syntax_and_semantics_of_the_notation_system (end)   

\input{qm2pi.qmops} 

\input{qm2pi.sterngerlach} 

\input{qm2pi.metric} 

% section concurrent_process_calculi (end)

%\input{qm2pi.proofsketch}

% section proof sketch (end)

%\input{qm2pi.slviaknots} 

% section spatial logic via knots (end)

\input{qm2pi.conclusion}

% section conclusion (end)

%\input{qm2pi.dtcodes} 

% section wiring algorithm (end)

\input{qm2pi.ack} 

% section acknowledgments (end)

\newpage


\bibliographystyle{plain}   
\bibliography{../../biblios/main.bib}

\input{qm2pi.rhodetails}

\end{document}

 

% subsection basic_interpretation (end)

%\input{qm2pi.rho.presentation} 
\subsection{The syntax and semantics of the notation system}\label{sub:the_syntax_and_semantics_of_the_notation_system} % (fold)

We now summarize a technical presentation of the calculus that
embodies our theory of dynamics. The typical presentation of such a
calculus follows the style of giving generators and relations on
them. The grammar, below, describing term constructors, freely
generates the set of processes, $\Proc$. This set is then quotiented
by a relation known as structural congruence and it is over this set
that the notion of dynamics is expressed. This presentation is
essentially that of \cite{MeredithR05} with the addition of
polyadicity and summation. For readability we have relegated some of
the technical subtleties to an appendix.

\subsubsection{Process grammar}\label{subsub:process_grammar}

\begin{mathpar}
  \inferrule* [lab=synchronization] {} {{M} \bc \pzero \;|\; x?F \;|\; x!C }
  \and
  \inferrule* [lab=abstraction] {} {{F} \bc (x)P}
  \and
  \inferrule* [lab=concretion] {} {{C} \bc \langle Q \rangle}
  \and
  \inferrule* [lab=process] {} {{P,Q} \bc M \;| \;P|Q \;|\; @{x}}
  \and
  \inferrule* [lab=name] {} {{x} \bc \quotep{P}}
\end{mathpar} 

Note that $\vec{x}$ (resp. $\vec{P}$) denotes a vector of names
(resp. processes) of length $|\vec{x}|$ (resp. $|\vec{P}|$). We adopt
the following useful abbreviations.

\begin{mathpar}
   x?(\vec{y}).P := x.(\vec{y})P \and  x\clift{\vec{P}} := x.\clift{\vec{P}}
   \and x!(y) := \lift{x}{\dropn{y}}
   \and \Pi_{i=0}^{n-1}P_i := P_0 | \ldots | P_{n-1}
\end{mathpar}

\subsubsection{Structural congruence}

\paragraph{Free and bound names and alpha-equivalence.} At the
core of structural equivalence is alpha-equivalence which identifies
process that are the same up to a change of variable. Formally, we
recognize the distinction between free and bound names. The free names
of a process, $\freenames{P}$, may be calculated recursively as
follows:

\begin{mathpar}
\freenames{\pzero} := \emptyset
  \and \\
  \freenames{x?(y).P} := \{ x \} \cup (\freenames{P} \setminus \{ y \})
  \and 
  \freenames{x!\langle P \rangle} := \{ x \} \cup \{ P \} 
  \and \\
  \freenames{P|Q} := \freenames{P} \cup \freenames{Q}
  \and \\
  \freenames{@{x}} := \{ x \}
\end{mathpar}

$\pi$
$\quotep{\pi}$

$\freenames{-} : \pi \to \mathcal{P}(\quotep{\pi})$

\begin{eqnarray*}
  \freenames{\pzero} & := & \emptyset \\
  \freenames{x?(y).P} & := & \{ x \} \cup (\freenames{P} \setminus \{ y \}) \\
  \freenames{x!\langle P \rangle} & := & \{ x \} \cup \{ P \} \\
  \freenames{P|Q} & := & \freenames{P} \cup \freenames{Q} \\
  \freenames{\dropn{x}} & := & \{ x \}
\end{eqnarray*}

The bound names of a process, $\boundnames{P}$, are those names occurring in $P$
that are not free. For example, in $x?(y).0$, the name $x$ is free, while $y$ is bound.

\begin{mathpar}
  \inferrule* [lab=monoidal-laws] {} { P|Q \equiv Q|P \and P|0 \equiv P \and P|(Q|R) \equiv (P|Q)|R }
\end{mathpar}

\begin{mathpar}
  \inferrule* [lab=alpha-equivalence] {} { (x)P \equiv (y)P\{y/x\} \and y \not\in \freenames{P} }
\end{mathpar}

\begin{definition}
Then two processes, $P,Q$, are alpha-equivalent if $P = Q\{\vec{y}/\vec{x}\}$ for
some $\vec{x} \in \boundnames{Q},\vec{y} \in \boundnames{P}$, where $Q\{\vec{y}/\vec{x}\}$
denotes the capture-avoiding substitution of $\vec{y}$ for $\vec{x}$ in $Q$.
\end{definition}

\begin{definition}
  The {\em structural congruence} \cite{SangiorgiWalker} , $\equiv$,
  between processes is the least congruence containing
  alpha-equivalence, satisfying the abelian monoid laws
  (associativity, commutativity and $\pzero$ as identity) for parallel
  composition $|$ and for summation $+$.
\end{definition}

\subsection{Name equivalence}

We take name equivalence, written $\nameeq$, to be the smallest
equivalence relation generated by the following rules.

\begin{mathpar}
\inferrule*[lab=Quote-drop]
{ }
{ \quotep{@{x}} \nameeq x }

\inferrule*[lab=Struct-equiv]
{ P \scong Q }
{ \quotep{P} \nameeq \quotep{Q} }
\end{mathpar}

The astute reader will have noticed that the mutual recursion of names
and processes imposes a mutual recursion on alpha-equivalence and
structural equivalence via name-equivalence. Fortunately, all of this
works out pleasantly and we may calculate in the natural way, free of
concern. The reader interested in the details is referred to the
appendix \ref{appendix:rho_details}.

\subsection{Substitution}

We use $\Proc$ for the set of processes, $\QProc$ for the set of
names, and $\id{\{}\vec{y} / \vec{x} \id{\}}$ to denote partial maps,
$s : \QProc \rightarrow \QProc$. A map, $s$ lifts, uniquely, to a map
on process terms, $\widehat{s} : \Proc \rightarrow \Proc$ by the
following equations.

\begin{mathpar}
  (0) \psubstp{Q}{P} := 0 \\
  (R \juxtap S) \psubstp{Q}{P}
  :=    
  (R)\psubstp{Q}{P} \juxtap (S) \psubstp{Q}{P} \\
  (x?(y).R) \psubstp{Q}{P}    
  :=    
  (x)\substp{Q}{P} (z)\concat( (R \psubstn{z}{y}) \psubstp{Q}{P} ) \\
  (\lift{x}{R}) \psubstp{Q}{P}  
  :=
  \lift{(x)\substp{Q}{P}}{ R \psubstp{Q}{P} } \\
%   (\dropn{x})  \psubstp{Q}{P}       
%   := 
%   \left\{ 
%     \begin{array}{ccc} 
%       \dropn{\quotep{Q}} & & x \nameeq \quotep{P} \\
%       \dropn{x} & & otherwise \\
%     \end{array}
%   \right. 
  (\dropn{x})  \psubstp{Q}{P}       
  := 
  \left\{ 
    \begin{array}{ccc} 
      Q & & x \nameeq \quotep{P} \\
      \dropn{x} & & otherwise \\
    \end{array}
  \right.
\end{mathpar}
 

where

\begin{eqnarray}
  (x)\id{\{} \lpquote Q \rpquote / \lpquote P \rpquote \id{\}}            = 
  \left\{ 
    \begin{array}{ccc}
      \lpquote Q \rpquote & & x \nameeq \lpquote P \rpquote \\
      x & & otherwise \\
    \end{array}
  \right. \nonumber
\end{eqnarray}

and $z$ is chosen distinct from $\quotep{P}$, $\quotep{Q}$, the free
names in $Q$, and all the names in $R$. Our $\alpha$-equivalence will
be built in the standard way from this substitution.

\begin{remark}\label{rem:no_self_referential_names}
  One consequence of these definitions is that $\forall P. \quotep{P}
  \not\in \freenames{P}$.
\end{remark}

\subsection{ Dynamic quote: an example }

Anticipating something of what's to come, consider applying the
substitution, $\widehat{\id{\{}u / z \id{\}}}$, to the following pair
of processes, $\lift{w}{y!(z)}$ and $w[ \lpquote y!(z) \rpquote ]$.

\begin{eqnarray}
	\lift{w}{y!(z)}\widehat{\id{\{}u / z \id{\}}}
		& = &
		\lift{w}{y!(u)} \nonumber\\
	w[ \lpquote y!(z) \rpquote ] \widehat{ \id{\{}u / z \id{\}} }
		& = &
		w[ \lpquote y!(z) \rpquote ] \nonumber
\end{eqnarray}

Because the body of the process between quotes is impervious to
substitution, we get radically different answers. In fact, by
examining the first process in an input context,
e.g. $x?(z).\lift{w}{y!(z)}$, we see that the process under the lift
operator may be shaped by prefixed inputs binding a name inside it. In
this sense, the lift operator will be seen as a way to dynamically
construct processes before reifying them as names.

Finally equipped with these standard features we can present the
dynamics of the calculus.

\subsubsection{Operational semantics} 

Finally, we introduce the computational dynamics. What marks these
algebras as distinct from other more traditionally studied algebraic
structures, e.g. vector spaces or polynomial rings, is the manner in
which dynamics is captured. In traditional structures, dynamics is typically
expressed through morphisms between such structures, as in linear maps
between vector spaces or morphisms between rings. In algebras
associated with the semantics of computation, the dynamics is
expressed as part of the algebraic structure itself, through a
reduction reduction relation typically denoted by $\red$. Below, we
give a recursive presentation of this relation for the calculus used
in the encoding.

$\red \subseteq \pi \times \pi$
$\red : \pi \to \mathcal{P}(\pi)$

\begin{mathpar}
  \inferrule* [lab=Comm] { \textsf{match}( x_{src}, x_{trgt} ) } { x_{trgt}?(y)P \; | \; x_{src}!\langle {Q} \rangle \red P\{\quotep{Q}/y}\} }
  \and \\
  \inferrule* [lab=Par] {{P} \red {P}'} {{{P} | {Q}} \red {{P}' | {Q}}}
  \and
  \inferrule* [lab=Equiv]{{{P} \scong {P}'} \andalso {{P}' \red {Q}'} \andalso {{Q}' \scong {Q}}}{{P} \red {Q}}
\end{mathpar}

\begin{eqnarray*}
  match_{\equiv} (\quotep{P},\quotep{Q}) & := & P \equiv Q \\
  match_{\dagger}(\quotep{P},\quotep{Q}) & := & \forall R. P|Q \red^{*} R => R \red^{*} 0 \\
  match_{K}(\quotep{P},\quotep{Q}) & := & K \mbox{ for some context } K
\end{eqnarray*}

$u?(x)P | u!\langle Q \rangle \red P\{\quotep{Q}/x\}$

%We write $\wred$ for $\red^*$, and $P\red$ if $\exists Q $ such that $ P \red Q$.
We write $P\red$ if $\exists Q $ such that $ P \red Q$ and $P\not\red$, otherwise.

\section{Replication}

As mentioned before, it is known that replication (and hence
recursion) can be implemented in a higher-order process algebra
\cite{SangiorgiWalker}. As our first example of calculation with the
machinery thus far presented we give the construction explicitly in
the {\rhoc}.

\begin{eqnarray}
	D_{x} & := & \prefix{x}{y}{(\binpar{\outputp{x}{y}}{@{y}})} \nonumber\\
	\bangp_{x}{P} & := & \binpar{{x}!\langle{\binpar{D_{x}}{P}}\rangle}{D_{x}} \nonumber
\end{eqnarray}

\begin{eqnarray}
	\bangp_{x}{P} & & \nonumber\\
	=
	& {x}!\langle{(\prefix{x}{y}{(\outputp{x}{y} | @{y})) | P}}\rangle 
	      | \prefix{x}{y}{(\outputp{x}{y} | @{y})} & \nonumber\\
	\red
	& (\outputp{x}{y} | @{y})\substn{\quotep{(\prefix{x}{y}{(@{y} | \outputp{x}{y})) | P}}}{y} & \nonumber\\
	=
	& \outputp{x}{\quotep{(\prefix{x}{y}{(\outputp{x}{y} | @{y})) | P}}}
	  | {(\prefix{x}{y}{(\outputp{x}{y} | @{y})) | P}} & \nonumber\\
	\red
	& \ldots & \nonumber\\
	\red^*
	& P | P | \ldots & \nonumber
\end{eqnarray}

Of course, this encoding, as an implementation, runs away, unfolding
$\bangp{P}$ eagerly. A lazier and more implementable replication
operator, restricted to input-guarded processes, may be obtained as follows.

\begin{eqnarray}
\bangp{\prefix{u}{v}{P}} 
	:= 
	\binpar{\lift{x}{\prefix{u}{v}{(\binpar{D(x)}{P})}}}{D(x)} \nonumber
\end{eqnarray}

\begin{remark}
  Note that the lazier definition still does not deal with summation
  or mixed summation (i.e. sums over input and output). The reader is
  invited to construct definitions of replication that deal with these
  features. 

  Further, the definitions are parameterized in a name, $x$. Can you,
  gentle reader, make a definition that eliminates this parameter and
  guarantees no accidental interaction between the replication
  machinery and the process being replicated -- i.e. no accidental
  sharing of names used by the process to get its work done and the
  name(s) used by the replication to effect copying. This latter
  revision of the definition of replication is crucial to obtaining
  the expected identity $!!P \sim !P$.
\end{remark}

\begin{remark}\label{rem:paradoxical_combinator}
  The reader familiar with the lambda calculus will have noticed the
  similarity between $D$ and the paradoxical combinator.

  [Ed. note: the existence of this seems to suggest we have to be more
  restrictive on the set of processes and names we admit if we are to
  support no-cloning.]
\end{remark}

\subsubsection{Bisimulation}

The computational dynamics gives rise to another kind of equivalence,
the equivalence of computational behavior. As previously mentioned
this is typically captured \emph{via} some form of bisimulation.

% The notion we use in this paper is weak barbed bisimulation
% \cite{milner91polyadicpi}.

The notion we use in this paper is derived from weak barbed
bisimulation \cite{milner91polyadicpi}. 

\begin{definition}
An \emph{observation relation}, $\downarrow_{\mathcal N}$, over a set
of names, $\mathcal N$, is the smallest relation satisfying the rules
below.

\infrule[Out-barb]{y \in {\mathcal N}, \; x \nameeq y}
		  {\outputp{x}{v} \downarrow_{\mathcal N} x}
\infrule[Par-barb]{\mbox{$P\downarrow_{\mathcal N} x$ or $Q\downarrow_{\mathcal N} x$}}
		  {\binpar{P}{Q} \downarrow_{\mathcal N} x}

We write $P \Downarrow_{\mathcal N} x$ if there is $Q$ such that 
$P \wred Q$ and $Q \downarrow_{\mathcal N} x$.
\end{definition}

\begin{definition}
%\label{def.bbisim}
An  ${\mathcal N}$-\emph{barbed bisimulation} over a set of names, ${\mathcal N}$, is a symmetric binary relation 
${\mathcal S}_{\mathcal N}$ between agents such that $P\rel{S}_{\mathcal N}Q$ implies:
\begin{enumerate}
\item If $P \red P'$ then $Q \wred Q'$ and $P'\rel{S}_{\mathcal N} Q'$.
\item If $P\downarrow_{\mathcal N} x$, then $Q\Downarrow_{\mathcal N} x$.
\end{enumerate}
$P$ is ${\mathcal N}$-barbed bisimilar to $Q$, written
$P \wbbisim_{\mathcal N} Q$, if $P \rel{S}_{\mathcal N} Q$ for some ${\mathcal N}$-barbed bisimulation ${\mathcal S}_{\mathcal N}$.
\end{definition}

$\mathcal{R} \subseteq \pi \times \pi$

$P \mathcal{R} Q => \forall P'. P \red P' \Rightarrow \exists Q'. Q \red Q', P' \mathcal{R} Q'$

$P \vdash x \Rightarrow Q \vdash x$

\begin{mathpar}
  \inferrule*[lab=Out-barb]{x \nameeq y}{{y}!\langle{Q}\rangle \vdash x}
  \and
  \inferrule*[lab=Par-barb]{\mbox{$P\vdash x$ or $Q\vdash x$}}{\binpar{P}{Q} \vdash x}
\end{mathpar}

\subsubsection{Contexts}

One of the principle advantages of computational calculi like the
$\pi$-calculus is a well-defined notion of context,
contextual-equivalence and a correlation between
contextual-equivalence and notions of bisimulation. The notion of
context allows the decomposition of a process into (sub-)process and
its syntactic environment, its context. Thus, a context may be
thought of as a process with a ``hole'' (written $\Box$) in it. The
application of a context $M$ to a process $P$, written $M[P]$, is
tantamount to filling the hole in $M$ with $P$. In this paper we do
not need the full weight of this theory, but do make use of the notion
of context in the proof the main theorem. 

\begin{mathpar}
  \inferrule* [lab=summation] {} {{M_{M},M_{N}} \bc \Box \;|\; x.M_{A} \;|\; M_{M}+M_{N}}
  \and
  \inferrule* [lab=agent] {} {{M_{A}} \bc (\vec{x})M_{P} \;| \; \clift{P_0,\ldots,M_{P},\ldots,P_N}}
  \and \\
  \inferrule* [lab=process] {} {{M_{P}} \bc M_{N} \;| \;P|M_{P} }
\end{mathpar} 

\begin{mathpar}
  \inferrule* [lab=sychronization] {} {M_{N} \bc \Box \;|\; x?M_{F} \;|\; x!M_{C}}
  \and
  \inferrule* [lab=abstraction] {} {{M_{F}} \bc (x)M_{P} }
  \and
  \inferrule* [lab=concretion] {} {{M_{C}} \bc \langle M_{P} \rangle }
  \and \\
  \inferrule* [lab=process] {} {{M_{P}} \bc M_{N} \;| \;P|M_{P} }
\end{mathpar}

\begin{definition}[contextual application] Given a context $M$, and
  process $P$, we define the \emph{contextual application}, $M[P] :=
  M\{P/\Box\}$. That is, the contextual application of M to P is the
  substitution of $P$ for $\Box$ in $M$.
\end{definition}

$\meaningof{-} : L \to \mathcal{P}(\pi)$

\begin{mathpar}
  \inferrule* [lab=collection] {} {\meaningof{true} = \pi, \and \meaningof{~E} = \pi \setminus \meaningof{E}, \and \meaningof{E_{1} \& E_{2}} = \meaningof{E_{1}} \cap \meaningof{E_{2}}}
\end{mathpar}

\begin{mathpar}
  \inferrule* [lab=structure] {} {\meaningof{0} = \{ P \in \pi | P \equiv 0 \}, \and \\ \meaningof{E_1 | E_2} = \{ P \in \pi | P \equiv P_{1} | P_{2}, P_{1} \in \meaningof{E_{1}}, P_{2} \in \meaningof{E_2}\} }
\end{mathpar}

\begin{mathpar}
 \inferrule* [lab=behavior] {} {\meaningof{\langle a?b \rangle E} = \{ P \in \pi | P \equiv Q | u?(y)P', \\ \and \\\\ \and \\ \;\;\; u \in \meaningof{a}, \forall z.P'\{z/y\} \in \meaningof{E\{z/b\}}\}, \and \\ \meaningof{a!E} = \{ P \in \pi | P \equiv Q | x!\langle P' \rangle, x \in \meaningof{a} P' \in \meaningof{E}\} }
\end{mathpar}

\begin{mathpar}
 \inferrule* [lab=nominal] {} {\meaningof{\quotep{E}} = \{ \quotep{P} \in \quotep{\pi} | P \in \meaningof{E} \}, \and \meaningof{\quotep{P}} = \{ \quotep{Q} \in \quotep{\pi} | P \equiv Q \} \and \\ \meaningof{@\quotep{E}} = \{ P \in \pi | P \equiv @x, x \in \meaningof{E} \}}
\end{mathpar}

\begin{eqnarray*}
  \\
  \meaningof{-} : TS \to ST
\end{eqnarray*}

\begin{eqnarray*}
  \\
  L : TS \to ST
\end{eqnarray*}

\begin{eqnarray*}
  \\
  P \models E \iff P \in \meaningof{E}
\end{eqnarray*}

\begin{eqnarray*}
  P \approx_{L} Q \iff \forall E \in L. P \models E \iff Q \models E
\end{eqnarray*}

\begin{eqnarray*}
  P \approx_{K} Q
\end{eqnarray*}

\begin{eqnarray*}
  P \approx Q
\end{eqnarray*}

$\approx_{K} = \approx = \approx_{L}$

\subsubsection{Contextual duality}

Note that contexts extend the quotation operation to a family of
operations from processes to names. Given a context, $M$, we can
define a \emph{nominal context}, $\quotep{M}$ by $\quotep{M}[P] :=
\quotep{M[P]}$. To foreshadow what is to come we observe that these
operations enjoy a duality with processes very much like the duality
between vectors and maps from vectors to scalars.

Further, because the calculus is essentially higher-order, we have a
correspondence between contexts and processes. More specifically,
given a name $x$ and a context $M$ we can construct $M^{*}_{x}$ such
that 

\begin{mathpar}
  M^{*}_{x} | \lift{x}{P} \red M[P]
\end{mathpar}

namely,

\begin{mathpar}
  M^{*}_{x} := x?(u).M[\dropn{u}]
\end{mathpar}

The dependence of $M^{*}_{x}$ on a name makes it an abstraction, 

\begin{mathpar}
  M^{*} := (x)x?(u).M[\dropn{u}]
\end{mathpar}

\subsection{Additional notation}

It will sometimes be convenient to denote the process a name
quotes. We already have the notation $x = \quotep{P}$, but it will be
convenient to introduce an alternate notation, $\procn{x}$, when we
want to emphasize the connection to the use of the name. Note that, by
virtue of name equivalence, $\quotep{\procn{x}} \nameeq x$; so, the
notation is consistent with previous definitions.

Further, because names have structure it is possible to effect
substitutions on the basis of that structure. This means we need to
upgrade our notation for substitutions, which we accomplish by
adapting comprehension notation. Thus,

\begin{mathpar}
  P\{ y / x : x \in S \}
\end{mathpar}

is interpreted to mean the process derived from P by replacing (in a
capture-avoiding manner) each occurrence of $x$ in $S$ by $y$. For example,

\begin{mathpar}
  P\{ \quotep{\procn{x}|\procn{x}} / x : x \in \freenames{P} \}
\end{mathpar}

will replace each (occurrence) of a free name $x$ in $P$ by
$\quotep{\procn{x}|\procn{x}}$.

Also, we will avail ourselves of the notation $x^{L}$ and $x^{R}$ to
denote injections of a name into disjoint copies of the name
space. There are numerous ways to accomplish this. One example can be
found in \cite{MeredithR05}. This notation overloads to vectors of
names: $\vec{x}^{\pi} := (x_{i}^{\pi} \; : \; 0 \leq i < |\vec{x}| )$ where $\pi \in \{L,R\}$.

We also use $P^{\Box} := P|\Box$.

In \cite{MeredithR05} an interpretation of the new operator is
given. It turns out that there are several possible interpretations
all enjoying the requisite algebraic properties of the operator (see
\cite{milner91polyadicpi}). We will therefore make liberal use of
$(\nu\; \vec{x})P$.

% subsection the_syntax_and_semantics_of_the_notation_system (end)   

\section{Interpretation of QM}
\subsection{Supporting definitions}
\subsubsection{Multiplication}
\begin{mathpar}
  \quotep{Q} \cdot \quotep{R} := \quotep{Q|R}
  \and \\
  \quotep{Q} \cdot P := P\{ \quotep{Q|R} / \quotep{R} : \quotep{R} \in \freenames{P} \}
\end{mathpar}

\paragraph{Discussion}
The first line needs little explanation. The second line says that
each free name of the process is replaced with the multiplication of
that name by the scalar. Multiplication of a scalar (name) by a state
(process) results in a process all the names of which have been `moved
over' by parallel composition with the process the scalar
quotes. There is a subtlety that the bound names have to be
manipulated so that multiplied names aren't accidentally
captured. There are many ways to achieve this.

\begin{remark}\label{rem:multiplication_identities}
  The reader is invited to verify that for all $x,y,z \in \QProc$ and $P \in \Proc$
  \begin{mathpar}
    x \cdot \quotep{0} \equiv x 
    \and
    x \cdot y \equiv y \cdot x
    \and
    x \cdot (y \cdot z) \equiv (x \cdot y) \cdot z
    \and \\
    \quotep{0} \cdot P \equiv P
    \and \\
    x \cdot (y \cdot P) \equiv (x \cdot y) \cdot P
    \and \\
    x \cdot (P|Q) \equiv (x \cdot P) | (x \cdot Q)
    \and \\    
  \end{mathpar}
\end{remark}

\subsubsection{Tensor product}

We define a tensor product on processes by structural induction.

\paragraph{Tensor of sums} First note that all summations, including
$\pzero$ and sequence, can be written $\Sigma_{i} x_{i}.A_{i} +
\Sigma_{j} x_{j}.C_{j}$, where we have grouped input-guarded processes
together and output-guarded processes together.

Thus, we can define the tensor product of two summations, $N_{1}\otimes N_{2}$, where

\begin{mathpar}
  N_{1} := \Sigma_{i} x_{i}.A_{i} + \Sigma_{j} x_{j}.C_{j}
  \and
  N_{2} := \Sigma_{i'} y_{i'}.B_{i'} + \Sigma_{j'} y_{j'}.D_{j'} 
\end{mathpar}

as follows.

\begin{mathpar}
  \Sigma_{i} x_{i}.A_{i} + \Sigma_{j} x_{j}.C_{j} \otimes \Sigma_{i'}
  y_{i'}.B_{i'} + \Sigma_{j'} y_{j'}.D_{j'} 
  \and \\
  := \; \Sigma_{i} \Sigma_{i'} \quotep{\stackrel{\vee}{x_{i}}| \stackrel{\vee}{y_{i'}}}.(A_{i}\otimes B_{i'}) \; | \; \Sigma_{i'} \Sigma_{i} \quotep{\stackrel{\vee}{y_{i'}}|\stackrel{\vee}{x_{i}}}.(B_{i'}\otimes A_{i})
  \and
  \;\; | \;\; \Sigma_{j} \Sigma_{j'} \quotep{\stackrel{\vee}{x_{j}}|\stackrel{\vee}{y_{j'}}}.(A_{j}\otimes B_{j'}) \; | \; \Sigma_{j'} \Sigma_{j} \quotep{\stackrel{\vee}{y_{j'}}|\stackrel{\vee}{x_{j}}}.(B_{j'}\otimes A_{j})
\end{mathpar}

\begin{remark}
  Do we need to $x^{L}$ and $y^{R}$ for this construction as well?
\end{remark}

\paragraph{Tensor of parallel compositions} Next, we distribute tensor
over par.

\begin{mathpar}
  P_{1}|P_{2} \otimes Q_{1}|Q_{2} := (P_{1} \otimes Q_{1}) | (P_{1}
  \otimes Q_{2}) | (P_{2} \otimes Q_{1}) | (P_{2} \otimes Q_{2})
\end{mathpar}

\paragraph{Tensor with dropped names} We treat tensor of a
process with a dropped name as parallel composition.

\begin{mathpar}
  P \otimes \dropn{x} := P | \dropn{x}
\end{mathpar}

\paragraph{Tensor of agents}

Finally, we need to define tensor on agents. Note that the definition
of tensor on normal products only tensors inputs with inputs and
outputs with outputs. Thus, we only have to define the operation on
``homogeneous'' pairings.

\begin{mathpar}
  (\vec{x})P \otimes (\vec{y})Q
  \and \\
  := (x_{0}^{L}|y_{0}^{R},\ldots,x_{0}^{L}|y_{n}^{R},\ldots,x_{m}^{L}|y_{0}^{R},\ldots,x_{m}^{L}|y_{n}^R)(P\{ \vec{x}^{L}/\vec{x}\} \otimes Q \{ \vec{y}^{R}/\vec{y}\})
  \and \\
  \clift{\vec{P}} \otimes \clift{\vec{Q}}
  \and \\
  := \clift{P_{0}\otimes Q_{0},\ldots,P_{0}\otimes Q_{n},\ldots,P_{m}\otimes Q_{0},\ldots,P_{m}\otimes Q_{n}}
\end{mathpar}

\begin{remark}
  Observe that arities of tensored abstractions matches arities of
  tensored concretions if the original arities matched. Note also that
  the length of the arities corresponds to the increase in dimension
  we see in ordinary vector space tensor product.
\end{remark}

\begin{remark}
  Operationally, this definition distributes the tensor down to
  components ``linked'' by summation. Tensor over summation is
  intriguing in that it mixes names. Moreover, as a consequence of the
  way it mixes names we have the identities for all $x \in \QProc$ and
  $P,Q \in \Proc$

  \begin{mathpar}
    (x \cdot P) \otimes Q \equiv x \cdot (P \otimes Q) \equiv P \otimes (x \cdot Q)
    \and
    P \otimes \pzero \equiv P
  \end{mathpar}

  that the reader is invited to verify.
\end{remark}

\subsubsection{Annihilation}
\begin{mathpar}
  P^{\perp} := \{ Q | \forall R. P|Q \red^{*} R \Rightarrow R \red^{*} \pzero \}
  \and \\
  P^{\underline{\perp}} := \Sigma_{Q \in P^{\perp}} \quotep{Q}?(y).(\dropn{y}|Q) | \Sigma_{Q \in P^{\perp}} \quotep{Q}\clift{\Box}
\end{mathpar}

\paragraph{Discussion} The reader will note that $P^{\perp}$ is a
\emph{set} of processes, while $P^{\underline{\perp}}$ is a
\emph{context}. We call the set $P^{\perp}$ the \emph{annihilators} of
$P$. The parallel composition of a process in the annihilators of $P$
with $P$ will result in a process, the state space of which has all
paths eventually leading to $\pzero$. Execution may endure loops; but
under reasonable conditions of fairness (naturally guaranteed under
most notions of bisimulation) such a composite process cannot get
stuck in such a loop and will, eventually pop out and terminate.

The context $P^{\underline{\perp}}$ is ready and willing to ``take the
$P$ out of'' the process to which it is applied. It will effectively
transmit the code of the process to which it is applied to one of the
annihilators and run the process against it.

\subsubsection{Evaluation}
We fix $M$ a domain of fully abstract interpretation with an equality
coincident with bisimulation. We take $\meaningof{\cdot} : \Proc \to
M$ to be the map interpreting processes and $\nmeaningof{\cdot} : \M
\to Proc$ to be the map running the other way. Then we define

\begin{mathpar}
  \int P := \nmeaningof{\meaningof{P}}
\end{mathpar}

\paragraph{Discussion}
There are many fully abstract interpretations of Milner's
$\pi$-calculus. Any of them can be used as a basis for interpreting
the reflective calculus here. Equipped with such a domain it is
largely a matter of grinding through to check that the Yoneda
construction for the normalization-by-evaluation program can be
extended to this setting.

\begin{remark}
  The reader is invited to verify that $\int (P^{\underline{\perp}}[P]) = 0$.
\end{remark}

\subsection{Quantum mechanics}

Table \ref{tbl:core_qm_op_defns} gives the core operational definitions

\begin{table}[htp]\label{tbl:core_qm_op_defns}
  \center{
    \fbox{
      \begin{tabular}{c|c}
        quantum mechanics & process calculus \\
        \hline
        scalar & $x := \quotep{P}$ \\
        state vector & $\state{P} := P$ \\
        dual & $\state{P}^{*} := \event{P^{\underline{\perp}}} := \quotep{P^{\underline{\perp}}}[-]$ \\
        matrix & $ \Sigma_{\alpha} \state{P_{\alpha}}x_{\alpha}\event{Q_{\alpha}}$ \\
        vector addition & $\state{P} + \state{Q} := \state{P | Q}$ \\
        tensor product & $\state{P} \otimes \state{Q} := \state{P \otimes Q}$ \\
        inner product & $\innerprod{P}{Q} := \quotep{\int P^{\underline{\perp}}[Q]}$ \\
      \end{tabular}
    }
  }
  \caption{QM - operational definitions}
\end{table}

where

\begin{mathpar}
  \prmatrix{P}{Q} := \fprmatrix{P}{\quotep{\pzero}}{Q}
  \and
  \fprmatrix{P}{x}{Q} := (\state{P},x,\event{Q})
  \and
  (\fprmatrix{P}{x}{Q})(\state{R}) := x \cdot \innerprod{Q}{R} \cdot \state{P}
  \and
  (\fprmatrix{P}{x}{Q})(\event{R}) := x \cdot \innerprod{R}{P} \cdot \event{Q}
\end{mathpar}

\paragraph{Discussion}
As promised: vectors (aka states) are represented as processes; duals
as contextual duals; inner product definition should be compared with
standard inner product definition for ....

\begin{remark}
  Assuming $\int (P^{\underline{\perp}}[P]) = 0$, the reader is
  invited to verify that $(\fprmatrix{P}{x}{P})(\state{P}) = x \cdot \state{P}$.
\end{remark}

\begin{remark}
  The reader is invited to verify that $\innerprod{P}{Q}$ could
  equally well have been written $\quotep{\int \stackrel{\vee}{x}}$
  where $x = \event{P^{\underline{\perp}}}(Q)$.

  One of the motivations for this remark is that there is another way
  to factor these operations. We could package up evaluation in the dual:

  \begin{mathpar}
    \state{P}^{*} := \event{\int P^{\underline{\perp}}} := \quotep{\int P^{\underline{\perp}}}[-]
  \end{mathpar}

  and then have inner product defined by
  
  \begin{mathpar}
    \innerprod{P}{Q} := \event{P}(Q)
  \end{mathpar}

  Hopefully, experience with the calculations will provide guidance on
  the best factoring.
\end{remark}

\begin{remark}
  Assuming $\int (P^{\underline{\perp}}[P]) = 0$, the reader is
  invited to verify that $\forall P,Q. (\prmatrix{0}{Q})(\state{0}) =
  \state{0}$ and dually $(\prmatrix{P}{0})(\event{0}) = \event{0}$.
\end{remark}

\begin{remark}
  i'm a little worried that i don't (yet) have proper support for
  complex conjugacy. But, the observation above may give us a
  clue. According to Abramsky, it must be the case that the scalars
  are iso to the homset of the identity for the tensor -- which the
  observation above characterizes. 

  For now, we will simply bookmark the notion with $\overline{x}$.
\end{remark}

\subsubsection{Adjointness}

We need to give a definition of $(\cdot)^{\dagger}$ for matrices. The
obvious candidate definition is
\begin{mathpar}
(\Sigma_{\alpha}\fprmatrix{P_{\alpha}}{x_{\alpha}}{Q_{\alpha}})^{\dagger}
= \Sigma_{\alpha}\fprmatrix{(Q_{\alpha}^{\underline{\perp}})^{*}}{\overline{x}_{\alpha}}{P_{\alpha}^{\underline{\perp}}} 
\end{mathpar}

But, $(Q_{\alpha}^{\underline{\perp}})^{*}$ requires a name along
which to communicate the process to achieve the context application.

\subsubsection{Basis for a basis}
If processes label states and ``addition'' of states (a.k.a. vector
addition) is interpreted as parallel composition, what corresponds to
notions of linear independence and basis? Here, we recall that Yoshida
has developed a set of \emph{combinators} for an asynchronous verison
of Milner's $\pi$-calculus. These are a finite set of processes such
any process can be expressed as parallel composition of these
combinators together with liberal uses of the new operator and
replication. We can simply give a translation of these into the
present calculus and have reasonable expectation that the property
carries over. That is, that the resultant set allows to express all
processes via parallel composition. Note, however, that there is no
new operator or replication in this calculus. As a result, we expect
that the corresponding set is actually infinite. That is, we expect
that the space is actually infinite dimensional.

\begin{remark}
  The attentive reader may be a bit concerned. Certainly, the
  collection $S$, $K$ and $I$ is a finite set of
  combinators. Shouldn't we expect to see a finite set of combinators
  for an effectively equivalent system? i am very sympathetic to this
  critique and feel it warrants full attention. On the other hand, i
  also have in mind the following analogy. The natural numbers, as a
  monoid under addition, has exactly $1$ generator, while the natural
  numbers, as a monoid under multiplication, has countably many
  generators (the primes). We observe that the application of the
  lambda calculus is much less resource sensitive than the parallel
  composition of the $\pi$-calculus. Could it be the case that we have
  an analogy of the form
  
  \begin{mathpar}
    m + n : MN :: m*n : M|N
  \end{mathpar}

  giving a similar blow up in the set of ``primes''?  This is such a
  wonderful thought that, even if it's not true, i think it's worth
  writing down.
\end{remark}
 

\documentclass[12pt]{llncs}
%\documentclass{jktr}

\usepackage[pdftex]{hyperref}                   
\usepackage {listings}
\usepackage {mathpartir}
\usepackage{bcprules}
%\usepackage{listings}
                       
\usepackage{graphicx} 
%\usepackage[margins=2.5cm,nohead,nofoot]{geometry}
%\usepackage{geometry}
\usepackage{amsfonts}
\usepackage{amstext}
\usepackage{latexsym}
\usepackage{amssymb}
\usepackage{color}


%\include{myPreamble}
\include{qm2pi.local} 

%\ifpdf
%\usepackage[pdftex]{graphicx}
%\else
%\usepackage{graphicx}
%\fi

 % \ifpdf
%  \usepackage{pdfsync}
%  \if


%\title{Brief Article}
%\author{David F. Snyder}
%\author{L.G. Meredith}

%\address{Dept. of Math., Texas State University--San Marcos, San Marcos, TX 78666}
       
\pagestyle{empty}


\begin{document}

\lstset{language=[Objective]Caml,frame=shadowbox}

\input{qm2pi.front}

% section front matter (end)

\input{qm2pi.intro} 
 
% section introduction (end)

% \input{qm2pi.knotations} 

% section notation (end)

\input{qm2pi.process.calculi} 

% section concurrent_process_calculi_and_spatial_logics_ (end)
    
%\input{qm2pi.knots2pi} 

%\input{qm2pi.trefoil} 

%\input{qm2pi.mainthm} 

% subsection basic_interpretation (end)

%\input{qm2pi.rho.presentation} 
\subsection{The syntax and semantics of the notation system}\label{sub:the_syntax_and_semantics_of_the_notation_system} % (fold)

We now summarize a technical presentation of the calculus that
embodies our theory of dynamics. The typical presentation of such a
calculus follows the style of giving generators and relations on
them. The grammar, below, describing term constructors, freely
generates the set of processes, $\Proc$. This set is then quotiented
by a relation known as structural congruence and it is over this set
that the notion of dynamics is expressed. This presentation is
essentially that of \cite{MeredithR05} with the addition of
polyadicity and summation. For readability we have relegated some of
the technical subtleties to an appendix.

\subsubsection{Process grammar}\label{subsub:process_grammar}

\begin{mathpar}
  \inferrule* [lab=synchronization] {} {{M} \bc \pzero \;|\; x?F \;|\; x!C }
  \and
  \inferrule* [lab=abstraction] {} {{F} \bc (x)P}
  \and
  \inferrule* [lab=concretion] {} {{C} \bc \langle Q \rangle}
  \and
  \inferrule* [lab=process] {} {{P,Q} \bc M \;| \;P|Q \;|\; @{x}}
  \and
  \inferrule* [lab=name] {} {{x} \bc \quotep{P}}
\end{mathpar} 

Note that $\vec{x}$ (resp. $\vec{P}$) denotes a vector of names
(resp. processes) of length $|\vec{x}|$ (resp. $|\vec{P}|$). We adopt
the following useful abbreviations.

\begin{mathpar}
   x?(\vec{y}).P := x.(\vec{y})P \and  x\clift{\vec{P}} := x.\clift{\vec{P}}
   \and x!(y) := \lift{x}{\dropn{y}}
   \and \Pi_{i=0}^{n-1}P_i := P_0 | \ldots | P_{n-1}
\end{mathpar}

\subsubsection{Structural congruence}

\paragraph{Free and bound names and alpha-equivalence.} At the
core of structural equivalence is alpha-equivalence which identifies
process that are the same up to a change of variable. Formally, we
recognize the distinction between free and bound names. The free names
of a process, $\freenames{P}$, may be calculated recursively as
follows:

\begin{mathpar}
\freenames{\pzero} := \emptyset
  \and \\
  \freenames{x?(y).P} := \{ x \} \cup (\freenames{P} \setminus \{ y \})
  \and 
  \freenames{x!\langle P \rangle} := \{ x \} \cup \{ P \} 
  \and \\
  \freenames{P|Q} := \freenames{P} \cup \freenames{Q}
  \and \\
  \freenames{@{x}} := \{ x \}
\end{mathpar}

$\pi$
$\quotep{\pi}$

$\freenames{-} : \pi \to \mathcal{P}(\quotep{\pi})$

\begin{eqnarray*}
  \freenames{\pzero} & := & \emptyset \\
  \freenames{x?(y).P} & := & \{ x \} \cup (\freenames{P} \setminus \{ y \}) \\
  \freenames{x!\langle P \rangle} & := & \{ x \} \cup \{ P \} \\
  \freenames{P|Q} & := & \freenames{P} \cup \freenames{Q} \\
  \freenames{\dropn{x}} & := & \{ x \}
\end{eqnarray*}

The bound names of a process, $\boundnames{P}$, are those names occurring in $P$
that are not free. For example, in $x?(y).0$, the name $x$ is free, while $y$ is bound.

\begin{mathpar}
  \inferrule* [lab=monoidal-laws] {} { P|Q \equiv Q|P \and P|0 \equiv P \and P|(Q|R) \equiv (P|Q)|R }
\end{mathpar}

\begin{mathpar}
  \inferrule* [lab=alpha-equivalence] {} { (x)P \equiv (y)P\{y/x\} \and y \not\in \freenames{P} }
\end{mathpar}

\begin{definition}
Then two processes, $P,Q$, are alpha-equivalent if $P = Q\{\vec{y}/\vec{x}\}$ for
some $\vec{x} \in \boundnames{Q},\vec{y} \in \boundnames{P}$, where $Q\{\vec{y}/\vec{x}\}$
denotes the capture-avoiding substitution of $\vec{y}$ for $\vec{x}$ in $Q$.
\end{definition}

\begin{definition}
  The {\em structural congruence} \cite{SangiorgiWalker} , $\equiv$,
  between processes is the least congruence containing
  alpha-equivalence, satisfying the abelian monoid laws
  (associativity, commutativity and $\pzero$ as identity) for parallel
  composition $|$ and for summation $+$.
\end{definition}

\subsection{Name equivalence}

We take name equivalence, written $\nameeq$, to be the smallest
equivalence relation generated by the following rules.

\begin{mathpar}
\inferrule*[lab=Quote-drop]
{ }
{ \quotep{@{x}} \nameeq x }

\inferrule*[lab=Struct-equiv]
{ P \scong Q }
{ \quotep{P} \nameeq \quotep{Q} }
\end{mathpar}

The astute reader will have noticed that the mutual recursion of names
and processes imposes a mutual recursion on alpha-equivalence and
structural equivalence via name-equivalence. Fortunately, all of this
works out pleasantly and we may calculate in the natural way, free of
concern. The reader interested in the details is referred to the
appendix \ref{appendix:rho_details}.

\subsection{Substitution}

We use $\Proc$ for the set of processes, $\QProc$ for the set of
names, and $\id{\{}\vec{y} / \vec{x} \id{\}}$ to denote partial maps,
$s : \QProc \rightarrow \QProc$. A map, $s$ lifts, uniquely, to a map
on process terms, $\widehat{s} : \Proc \rightarrow \Proc$ by the
following equations.

\begin{mathpar}
  (0) \psubstp{Q}{P} := 0 \\
  (R \juxtap S) \psubstp{Q}{P}
  :=    
  (R)\psubstp{Q}{P} \juxtap (S) \psubstp{Q}{P} \\
  (x?(y).R) \psubstp{Q}{P}    
  :=    
  (x)\substp{Q}{P} (z)\concat( (R \psubstn{z}{y}) \psubstp{Q}{P} ) \\
  (\lift{x}{R}) \psubstp{Q}{P}  
  :=
  \lift{(x)\substp{Q}{P}}{ R \psubstp{Q}{P} } \\
%   (\dropn{x})  \psubstp{Q}{P}       
%   := 
%   \left\{ 
%     \begin{array}{ccc} 
%       \dropn{\quotep{Q}} & & x \nameeq \quotep{P} \\
%       \dropn{x} & & otherwise \\
%     \end{array}
%   \right. 
  (\dropn{x})  \psubstp{Q}{P}       
  := 
  \left\{ 
    \begin{array}{ccc} 
      Q & & x \nameeq \quotep{P} \\
      \dropn{x} & & otherwise \\
    \end{array}
  \right.
\end{mathpar}
 

where

\begin{eqnarray}
  (x)\id{\{} \lpquote Q \rpquote / \lpquote P \rpquote \id{\}}            = 
  \left\{ 
    \begin{array}{ccc}
      \lpquote Q \rpquote & & x \nameeq \lpquote P \rpquote \\
      x & & otherwise \\
    \end{array}
  \right. \nonumber
\end{eqnarray}

and $z$ is chosen distinct from $\quotep{P}$, $\quotep{Q}$, the free
names in $Q$, and all the names in $R$. Our $\alpha$-equivalence will
be built in the standard way from this substitution.

\begin{remark}\label{rem:no_self_referential_names}
  One consequence of these definitions is that $\forall P. \quotep{P}
  \not\in \freenames{P}$.
\end{remark}

\subsection{ Dynamic quote: an example }

Anticipating something of what's to come, consider applying the
substitution, $\widehat{\id{\{}u / z \id{\}}}$, to the following pair
of processes, $\lift{w}{y!(z)}$ and $w[ \lpquote y!(z) \rpquote ]$.

\begin{eqnarray}
	\lift{w}{y!(z)}\widehat{\id{\{}u / z \id{\}}}
		& = &
		\lift{w}{y!(u)} \nonumber\\
	w[ \lpquote y!(z) \rpquote ] \widehat{ \id{\{}u / z \id{\}} }
		& = &
		w[ \lpquote y!(z) \rpquote ] \nonumber
\end{eqnarray}

Because the body of the process between quotes is impervious to
substitution, we get radically different answers. In fact, by
examining the first process in an input context,
e.g. $x?(z).\lift{w}{y!(z)}$, we see that the process under the lift
operator may be shaped by prefixed inputs binding a name inside it. In
this sense, the lift operator will be seen as a way to dynamically
construct processes before reifying them as names.

Finally equipped with these standard features we can present the
dynamics of the calculus.

\subsubsection{Operational semantics} 

Finally, we introduce the computational dynamics. What marks these
algebras as distinct from other more traditionally studied algebraic
structures, e.g. vector spaces or polynomial rings, is the manner in
which dynamics is captured. In traditional structures, dynamics is typically
expressed through morphisms between such structures, as in linear maps
between vector spaces or morphisms between rings. In algebras
associated with the semantics of computation, the dynamics is
expressed as part of the algebraic structure itself, through a
reduction reduction relation typically denoted by $\red$. Below, we
give a recursive presentation of this relation for the calculus used
in the encoding.

$\red \subseteq \pi \times \pi$
$\red : \pi \to \mathcal{P}(\pi)$

\begin{mathpar}
  \inferrule* [lab=Comm] { \textsf{match}( x_{src}, x_{trgt} ) } { x_{trgt}?(y)P \; | \; x_{src}!\langle {Q} \rangle \red P\{\quotep{Q}/y}\} }
  \and \\
  \inferrule* [lab=Par] {{P} \red {P}'} {{{P} | {Q}} \red {{P}' | {Q}}}
  \and
  \inferrule* [lab=Equiv]{{{P} \scong {P}'} \andalso {{P}' \red {Q}'} \andalso {{Q}' \scong {Q}}}{{P} \red {Q}}
\end{mathpar}

\begin{eqnarray*}
  match_{\equiv} (\quotep{P},\quotep{Q}) & := & P \equiv Q \\
  match_{\dagger}(\quotep{P},\quotep{Q}) & := & \forall R. P|Q \red^{*} R => R \red^{*} 0 \\
  match_{K}(\quotep{P},\quotep{Q}) & := & K \mbox{ for some context } K
\end{eqnarray*}

$u?(x)P | u!\langle Q \rangle \red P\{\quotep{Q}/x\}$

%We write $\wred$ for $\red^*$, and $P\red$ if $\exists Q $ such that $ P \red Q$.
We write $P\red$ if $\exists Q $ such that $ P \red Q$ and $P\not\red$, otherwise.

\section{Replication}

As mentioned before, it is known that replication (and hence
recursion) can be implemented in a higher-order process algebra
\cite{SangiorgiWalker}. As our first example of calculation with the
machinery thus far presented we give the construction explicitly in
the {\rhoc}.

\begin{eqnarray}
	D_{x} & := & \prefix{x}{y}{(\binpar{\outputp{x}{y}}{@{y}})} \nonumber\\
	\bangp_{x}{P} & := & \binpar{{x}!\langle{\binpar{D_{x}}{P}}\rangle}{D_{x}} \nonumber
\end{eqnarray}

\begin{eqnarray}
	\bangp_{x}{P} & & \nonumber\\
	=
	& {x}!\langle{(\prefix{x}{y}{(\outputp{x}{y} | @{y})) | P}}\rangle 
	      | \prefix{x}{y}{(\outputp{x}{y} | @{y})} & \nonumber\\
	\red
	& (\outputp{x}{y} | @{y})\substn{\quotep{(\prefix{x}{y}{(@{y} | \outputp{x}{y})) | P}}}{y} & \nonumber\\
	=
	& \outputp{x}{\quotep{(\prefix{x}{y}{(\outputp{x}{y} | @{y})) | P}}}
	  | {(\prefix{x}{y}{(\outputp{x}{y} | @{y})) | P}} & \nonumber\\
	\red
	& \ldots & \nonumber\\
	\red^*
	& P | P | \ldots & \nonumber
\end{eqnarray}

Of course, this encoding, as an implementation, runs away, unfolding
$\bangp{P}$ eagerly. A lazier and more implementable replication
operator, restricted to input-guarded processes, may be obtained as follows.

\begin{eqnarray}
\bangp{\prefix{u}{v}{P}} 
	:= 
	\binpar{\lift{x}{\prefix{u}{v}{(\binpar{D(x)}{P})}}}{D(x)} \nonumber
\end{eqnarray}

\begin{remark}
  Note that the lazier definition still does not deal with summation
  or mixed summation (i.e. sums over input and output). The reader is
  invited to construct definitions of replication that deal with these
  features. 

  Further, the definitions are parameterized in a name, $x$. Can you,
  gentle reader, make a definition that eliminates this parameter and
  guarantees no accidental interaction between the replication
  machinery and the process being replicated -- i.e. no accidental
  sharing of names used by the process to get its work done and the
  name(s) used by the replication to effect copying. This latter
  revision of the definition of replication is crucial to obtaining
  the expected identity $!!P \sim !P$.
\end{remark}

\begin{remark}\label{rem:paradoxical_combinator}
  The reader familiar with the lambda calculus will have noticed the
  similarity between $D$ and the paradoxical combinator.

  [Ed. note: the existence of this seems to suggest we have to be more
  restrictive on the set of processes and names we admit if we are to
  support no-cloning.]
\end{remark}

\subsubsection{Bisimulation}

The computational dynamics gives rise to another kind of equivalence,
the equivalence of computational behavior. As previously mentioned
this is typically captured \emph{via} some form of bisimulation.

% The notion we use in this paper is weak barbed bisimulation
% \cite{milner91polyadicpi}.

The notion we use in this paper is derived from weak barbed
bisimulation \cite{milner91polyadicpi}. 

\begin{definition}
An \emph{observation relation}, $\downarrow_{\mathcal N}$, over a set
of names, $\mathcal N$, is the smallest relation satisfying the rules
below.

\infrule[Out-barb]{y \in {\mathcal N}, \; x \nameeq y}
		  {\outputp{x}{v} \downarrow_{\mathcal N} x}
\infrule[Par-barb]{\mbox{$P\downarrow_{\mathcal N} x$ or $Q\downarrow_{\mathcal N} x$}}
		  {\binpar{P}{Q} \downarrow_{\mathcal N} x}

We write $P \Downarrow_{\mathcal N} x$ if there is $Q$ such that 
$P \wred Q$ and $Q \downarrow_{\mathcal N} x$.
\end{definition}

\begin{definition}
%\label{def.bbisim}
An  ${\mathcal N}$-\emph{barbed bisimulation} over a set of names, ${\mathcal N}$, is a symmetric binary relation 
${\mathcal S}_{\mathcal N}$ between agents such that $P\rel{S}_{\mathcal N}Q$ implies:
\begin{enumerate}
\item If $P \red P'$ then $Q \wred Q'$ and $P'\rel{S}_{\mathcal N} Q'$.
\item If $P\downarrow_{\mathcal N} x$, then $Q\Downarrow_{\mathcal N} x$.
\end{enumerate}
$P$ is ${\mathcal N}$-barbed bisimilar to $Q$, written
$P \wbbisim_{\mathcal N} Q$, if $P \rel{S}_{\mathcal N} Q$ for some ${\mathcal N}$-barbed bisimulation ${\mathcal S}_{\mathcal N}$.
\end{definition}

$\mathcal{R} \subseteq \pi \times \pi$

$P \mathcal{R} Q => \forall P'. P \red P' \Rightarrow \exists Q'. Q \red Q', P' \mathcal{R} Q'$

$P \vdash x \Rightarrow Q \vdash x$

\begin{mathpar}
  \inferrule*[lab=Out-barb]{x \nameeq y}{{y}!\langle{Q}\rangle \vdash x}
  \and
  \inferrule*[lab=Par-barb]{\mbox{$P\vdash x$ or $Q\vdash x$}}{\binpar{P}{Q} \vdash x}
\end{mathpar}

\subsubsection{Contexts}

One of the principle advantages of computational calculi like the
$\pi$-calculus is a well-defined notion of context,
contextual-equivalence and a correlation between
contextual-equivalence and notions of bisimulation. The notion of
context allows the decomposition of a process into (sub-)process and
its syntactic environment, its context. Thus, a context may be
thought of as a process with a ``hole'' (written $\Box$) in it. The
application of a context $M$ to a process $P$, written $M[P]$, is
tantamount to filling the hole in $M$ with $P$. In this paper we do
not need the full weight of this theory, but do make use of the notion
of context in the proof the main theorem. 

\begin{mathpar}
  \inferrule* [lab=summation] {} {{M_{M},M_{N}} \bc \Box \;|\; x.M_{A} \;|\; M_{M}+M_{N}}
  \and
  \inferrule* [lab=agent] {} {{M_{A}} \bc (\vec{x})M_{P} \;| \; \clift{P_0,\ldots,M_{P},\ldots,P_N}}
  \and \\
  \inferrule* [lab=process] {} {{M_{P}} \bc M_{N} \;| \;P|M_{P} }
\end{mathpar} 

\begin{mathpar}
  \inferrule* [lab=sychronization] {} {M_{N} \bc \Box \;|\; x?M_{F} \;|\; x!M_{C}}
  \and
  \inferrule* [lab=abstraction] {} {{M_{F}} \bc (x)M_{P} }
  \and
  \inferrule* [lab=concretion] {} {{M_{C}} \bc \langle M_{P} \rangle }
  \and \\
  \inferrule* [lab=process] {} {{M_{P}} \bc M_{N} \;| \;P|M_{P} }
\end{mathpar}

\begin{definition}[contextual application] Given a context $M$, and
  process $P$, we define the \emph{contextual application}, $M[P] :=
  M\{P/\Box\}$. That is, the contextual application of M to P is the
  substitution of $P$ for $\Box$ in $M$.
\end{definition}

$\meaningof{-} : L \to \mathcal{P}(\pi)$

\begin{mathpar}
  \inferrule* [lab=collection] {} {\meaningof{true} = \pi, \and \meaningof{~E} = \pi \setminus \meaningof{E}, \and \meaningof{E_{1} \& E_{2}} = \meaningof{E_{1}} \cap \meaningof{E_{2}}}
\end{mathpar}

\begin{mathpar}
  \inferrule* [lab=structure] {} {\meaningof{0} = \{ P \in \pi | P \equiv 0 \}, \and \\ \meaningof{E_1 | E_2} = \{ P \in \pi | P \equiv P_{1} | P_{2}, P_{1} \in \meaningof{E_{1}}, P_{2} \in \meaningof{E_2}\} }
\end{mathpar}

\begin{mathpar}
 \inferrule* [lab=behavior] {} {\meaningof{\langle a?b \rangle E} = \{ P \in \pi | P \equiv Q | u?(y)P', \\ \and \\\\ \and \\ \;\;\; u \in \meaningof{a}, \forall z.P'\{z/y\} \in \meaningof{E\{z/b\}}\}, \and \\ \meaningof{a!E} = \{ P \in \pi | P \equiv Q | x!\langle P' \rangle, x \in \meaningof{a} P' \in \meaningof{E}\} }
\end{mathpar}

\begin{mathpar}
 \inferrule* [lab=nominal] {} {\meaningof{\quotep{E}} = \{ \quotep{P} \in \quotep{\pi} | P \in \meaningof{E} \}, \and \meaningof{\quotep{P}} = \{ \quotep{Q} \in \quotep{\pi} | P \equiv Q \} \and \\ \meaningof{@\quotep{E}} = \{ P \in \pi | P \equiv @x, x \in \meaningof{E} \}}
\end{mathpar}

\begin{eqnarray*}
  \\
  \meaningof{-} : TS \to ST
\end{eqnarray*}

\begin{eqnarray*}
  \\
  L : TS \to ST
\end{eqnarray*}

\begin{eqnarray*}
  \\
  P \models E \iff P \in \meaningof{E}
\end{eqnarray*}

\begin{eqnarray*}
  P \approx_{L} Q \iff \forall E \in L. P \models E \iff Q \models E
\end{eqnarray*}

\begin{eqnarray*}
  P \approx_{K} Q
\end{eqnarray*}

\begin{eqnarray*}
  P \approx Q
\end{eqnarray*}

$\approx_{K} = \approx = \approx_{L}$

\subsubsection{Contextual duality}

Note that contexts extend the quotation operation to a family of
operations from processes to names. Given a context, $M$, we can
define a \emph{nominal context}, $\quotep{M}$ by $\quotep{M}[P] :=
\quotep{M[P]}$. To foreshadow what is to come we observe that these
operations enjoy a duality with processes very much like the duality
between vectors and maps from vectors to scalars.

Further, because the calculus is essentially higher-order, we have a
correspondence between contexts and processes. More specifically,
given a name $x$ and a context $M$ we can construct $M^{*}_{x}$ such
that 

\begin{mathpar}
  M^{*}_{x} | \lift{x}{P} \red M[P]
\end{mathpar}

namely,

\begin{mathpar}
  M^{*}_{x} := x?(u).M[\dropn{u}]
\end{mathpar}

The dependence of $M^{*}_{x}$ on a name makes it an abstraction, 

\begin{mathpar}
  M^{*} := (x)x?(u).M[\dropn{u}]
\end{mathpar}

\subsection{Additional notation}

It will sometimes be convenient to denote the process a name
quotes. We already have the notation $x = \quotep{P}$, but it will be
convenient to introduce an alternate notation, $\procn{x}$, when we
want to emphasize the connection to the use of the name. Note that, by
virtue of name equivalence, $\quotep{\procn{x}} \nameeq x$; so, the
notation is consistent with previous definitions.

Further, because names have structure it is possible to effect
substitutions on the basis of that structure. This means we need to
upgrade our notation for substitutions, which we accomplish by
adapting comprehension notation. Thus,

\begin{mathpar}
  P\{ y / x : x \in S \}
\end{mathpar}

is interpreted to mean the process derived from P by replacing (in a
capture-avoiding manner) each occurrence of $x$ in $S$ by $y$. For example,

\begin{mathpar}
  P\{ \quotep{\procn{x}|\procn{x}} / x : x \in \freenames{P} \}
\end{mathpar}

will replace each (occurrence) of a free name $x$ in $P$ by
$\quotep{\procn{x}|\procn{x}}$.

Also, we will avail ourselves of the notation $x^{L}$ and $x^{R}$ to
denote injections of a name into disjoint copies of the name
space. There are numerous ways to accomplish this. One example can be
found in \cite{MeredithR05}. This notation overloads to vectors of
names: $\vec{x}^{\pi} := (x_{i}^{\pi} \; : \; 0 \leq i < |\vec{x}| )$ where $\pi \in \{L,R\}$.

We also use $P^{\Box} := P|\Box$.

In \cite{MeredithR05} an interpretation of the new operator is
given. It turns out that there are several possible interpretations
all enjoying the requisite algebraic properties of the operator (see
\cite{milner91polyadicpi}). We will therefore make liberal use of
$(\nu\; \vec{x})P$.

% subsection the_syntax_and_semantics_of_the_notation_system (end)   

\input{qm2pi.qmops} 

\input{qm2pi.sterngerlach} 

\input{qm2pi.metric} 

% section concurrent_process_calculi (end)

%\input{qm2pi.proofsketch}

% section proof sketch (end)

%\input{qm2pi.slviaknots} 

% section spatial logic via knots (end)

\input{qm2pi.conclusion}

% section conclusion (end)

%\input{qm2pi.dtcodes} 

% section wiring algorithm (end)

\input{qm2pi.ack} 

% section acknowledgments (end)

\newpage


\bibliographystyle{plain}   
\bibliography{../../biblios/main.bib}

\input{qm2pi.rhodetails}

\end{document}

 

\documentclass[12pt]{llncs}
%\documentclass{jktr}

\usepackage[pdftex]{hyperref}                   
\usepackage {listings}
\usepackage {mathpartir}
\usepackage{bcprules}
%\usepackage{listings}
                       
\usepackage{graphicx} 
%\usepackage[margins=2.5cm,nohead,nofoot]{geometry}
%\usepackage{geometry}
\usepackage{amsfonts}
\usepackage{amstext}
\usepackage{latexsym}
\usepackage{amssymb}
\usepackage{color}


%\include{myPreamble}
\include{qm2pi.local} 

%\ifpdf
%\usepackage[pdftex]{graphicx}
%\else
%\usepackage{graphicx}
%\fi

 % \ifpdf
%  \usepackage{pdfsync}
%  \if


%\title{Brief Article}
%\author{David F. Snyder}
%\author{L.G. Meredith}

%\address{Dept. of Math., Texas State University--San Marcos, San Marcos, TX 78666}
       
\pagestyle{empty}


\begin{document}

\lstset{language=[Objective]Caml,frame=shadowbox}

\input{qm2pi.front}

% section front matter (end)

\input{qm2pi.intro} 
 
% section introduction (end)

% \input{qm2pi.knotations} 

% section notation (end)

\input{qm2pi.process.calculi} 

% section concurrent_process_calculi_and_spatial_logics_ (end)
    
%\input{qm2pi.knots2pi} 

%\input{qm2pi.trefoil} 

%\input{qm2pi.mainthm} 

% subsection basic_interpretation (end)

%\input{qm2pi.rho.presentation} 
\subsection{The syntax and semantics of the notation system}\label{sub:the_syntax_and_semantics_of_the_notation_system} % (fold)

We now summarize a technical presentation of the calculus that
embodies our theory of dynamics. The typical presentation of such a
calculus follows the style of giving generators and relations on
them. The grammar, below, describing term constructors, freely
generates the set of processes, $\Proc$. This set is then quotiented
by a relation known as structural congruence and it is over this set
that the notion of dynamics is expressed. This presentation is
essentially that of \cite{MeredithR05} with the addition of
polyadicity and summation. For readability we have relegated some of
the technical subtleties to an appendix.

\subsubsection{Process grammar}\label{subsub:process_grammar}

\begin{mathpar}
  \inferrule* [lab=synchronization] {} {{M} \bc \pzero \;|\; x?F \;|\; x!C }
  \and
  \inferrule* [lab=abstraction] {} {{F} \bc (x)P}
  \and
  \inferrule* [lab=concretion] {} {{C} \bc \langle Q \rangle}
  \and
  \inferrule* [lab=process] {} {{P,Q} \bc M \;| \;P|Q \;|\; @{x}}
  \and
  \inferrule* [lab=name] {} {{x} \bc \quotep{P}}
\end{mathpar} 

Note that $\vec{x}$ (resp. $\vec{P}$) denotes a vector of names
(resp. processes) of length $|\vec{x}|$ (resp. $|\vec{P}|$). We adopt
the following useful abbreviations.

\begin{mathpar}
   x?(\vec{y}).P := x.(\vec{y})P \and  x\clift{\vec{P}} := x.\clift{\vec{P}}
   \and x!(y) := \lift{x}{\dropn{y}}
   \and \Pi_{i=0}^{n-1}P_i := P_0 | \ldots | P_{n-1}
\end{mathpar}

\subsubsection{Structural congruence}

\paragraph{Free and bound names and alpha-equivalence.} At the
core of structural equivalence is alpha-equivalence which identifies
process that are the same up to a change of variable. Formally, we
recognize the distinction between free and bound names. The free names
of a process, $\freenames{P}$, may be calculated recursively as
follows:

\begin{mathpar}
\freenames{\pzero} := \emptyset
  \and \\
  \freenames{x?(y).P} := \{ x \} \cup (\freenames{P} \setminus \{ y \})
  \and 
  \freenames{x!\langle P \rangle} := \{ x \} \cup \{ P \} 
  \and \\
  \freenames{P|Q} := \freenames{P} \cup \freenames{Q}
  \and \\
  \freenames{@{x}} := \{ x \}
\end{mathpar}

$\pi$
$\quotep{\pi}$

$\freenames{-} : \pi \to \mathcal{P}(\quotep{\pi})$

\begin{eqnarray*}
  \freenames{\pzero} & := & \emptyset \\
  \freenames{x?(y).P} & := & \{ x \} \cup (\freenames{P} \setminus \{ y \}) \\
  \freenames{x!\langle P \rangle} & := & \{ x \} \cup \{ P \} \\
  \freenames{P|Q} & := & \freenames{P} \cup \freenames{Q} \\
  \freenames{\dropn{x}} & := & \{ x \}
\end{eqnarray*}

The bound names of a process, $\boundnames{P}$, are those names occurring in $P$
that are not free. For example, in $x?(y).0$, the name $x$ is free, while $y$ is bound.

\begin{mathpar}
  \inferrule* [lab=monoidal-laws] {} { P|Q \equiv Q|P \and P|0 \equiv P \and P|(Q|R) \equiv (P|Q)|R }
\end{mathpar}

\begin{mathpar}
  \inferrule* [lab=alpha-equivalence] {} { (x)P \equiv (y)P\{y/x\} \and y \not\in \freenames{P} }
\end{mathpar}

\begin{definition}
Then two processes, $P,Q$, are alpha-equivalent if $P = Q\{\vec{y}/\vec{x}\}$ for
some $\vec{x} \in \boundnames{Q},\vec{y} \in \boundnames{P}$, where $Q\{\vec{y}/\vec{x}\}$
denotes the capture-avoiding substitution of $\vec{y}$ for $\vec{x}$ in $Q$.
\end{definition}

\begin{definition}
  The {\em structural congruence} \cite{SangiorgiWalker} , $\equiv$,
  between processes is the least congruence containing
  alpha-equivalence, satisfying the abelian monoid laws
  (associativity, commutativity and $\pzero$ as identity) for parallel
  composition $|$ and for summation $+$.
\end{definition}

\subsection{Name equivalence}

We take name equivalence, written $\nameeq$, to be the smallest
equivalence relation generated by the following rules.

\begin{mathpar}
\inferrule*[lab=Quote-drop]
{ }
{ \quotep{@{x}} \nameeq x }

\inferrule*[lab=Struct-equiv]
{ P \scong Q }
{ \quotep{P} \nameeq \quotep{Q} }
\end{mathpar}

The astute reader will have noticed that the mutual recursion of names
and processes imposes a mutual recursion on alpha-equivalence and
structural equivalence via name-equivalence. Fortunately, all of this
works out pleasantly and we may calculate in the natural way, free of
concern. The reader interested in the details is referred to the
appendix \ref{appendix:rho_details}.

\subsection{Substitution}

We use $\Proc$ for the set of processes, $\QProc$ for the set of
names, and $\id{\{}\vec{y} / \vec{x} \id{\}}$ to denote partial maps,
$s : \QProc \rightarrow \QProc$. A map, $s$ lifts, uniquely, to a map
on process terms, $\widehat{s} : \Proc \rightarrow \Proc$ by the
following equations.

\begin{mathpar}
  (0) \psubstp{Q}{P} := 0 \\
  (R \juxtap S) \psubstp{Q}{P}
  :=    
  (R)\psubstp{Q}{P} \juxtap (S) \psubstp{Q}{P} \\
  (x?(y).R) \psubstp{Q}{P}    
  :=    
  (x)\substp{Q}{P} (z)\concat( (R \psubstn{z}{y}) \psubstp{Q}{P} ) \\
  (\lift{x}{R}) \psubstp{Q}{P}  
  :=
  \lift{(x)\substp{Q}{P}}{ R \psubstp{Q}{P} } \\
%   (\dropn{x})  \psubstp{Q}{P}       
%   := 
%   \left\{ 
%     \begin{array}{ccc} 
%       \dropn{\quotep{Q}} & & x \nameeq \quotep{P} \\
%       \dropn{x} & & otherwise \\
%     \end{array}
%   \right. 
  (\dropn{x})  \psubstp{Q}{P}       
  := 
  \left\{ 
    \begin{array}{ccc} 
      Q & & x \nameeq \quotep{P} \\
      \dropn{x} & & otherwise \\
    \end{array}
  \right.
\end{mathpar}
 

where

\begin{eqnarray}
  (x)\id{\{} \lpquote Q \rpquote / \lpquote P \rpquote \id{\}}            = 
  \left\{ 
    \begin{array}{ccc}
      \lpquote Q \rpquote & & x \nameeq \lpquote P \rpquote \\
      x & & otherwise \\
    \end{array}
  \right. \nonumber
\end{eqnarray}

and $z$ is chosen distinct from $\quotep{P}$, $\quotep{Q}$, the free
names in $Q$, and all the names in $R$. Our $\alpha$-equivalence will
be built in the standard way from this substitution.

\begin{remark}\label{rem:no_self_referential_names}
  One consequence of these definitions is that $\forall P. \quotep{P}
  \not\in \freenames{P}$.
\end{remark}

\subsection{ Dynamic quote: an example }

Anticipating something of what's to come, consider applying the
substitution, $\widehat{\id{\{}u / z \id{\}}}$, to the following pair
of processes, $\lift{w}{y!(z)}$ and $w[ \lpquote y!(z) \rpquote ]$.

\begin{eqnarray}
	\lift{w}{y!(z)}\widehat{\id{\{}u / z \id{\}}}
		& = &
		\lift{w}{y!(u)} \nonumber\\
	w[ \lpquote y!(z) \rpquote ] \widehat{ \id{\{}u / z \id{\}} }
		& = &
		w[ \lpquote y!(z) \rpquote ] \nonumber
\end{eqnarray}

Because the body of the process between quotes is impervious to
substitution, we get radically different answers. In fact, by
examining the first process in an input context,
e.g. $x?(z).\lift{w}{y!(z)}$, we see that the process under the lift
operator may be shaped by prefixed inputs binding a name inside it. In
this sense, the lift operator will be seen as a way to dynamically
construct processes before reifying them as names.

Finally equipped with these standard features we can present the
dynamics of the calculus.

\subsubsection{Operational semantics} 

Finally, we introduce the computational dynamics. What marks these
algebras as distinct from other more traditionally studied algebraic
structures, e.g. vector spaces or polynomial rings, is the manner in
which dynamics is captured. In traditional structures, dynamics is typically
expressed through morphisms between such structures, as in linear maps
between vector spaces or morphisms between rings. In algebras
associated with the semantics of computation, the dynamics is
expressed as part of the algebraic structure itself, through a
reduction reduction relation typically denoted by $\red$. Below, we
give a recursive presentation of this relation for the calculus used
in the encoding.

$\red \subseteq \pi \times \pi$
$\red : \pi \to \mathcal{P}(\pi)$

\begin{mathpar}
  \inferrule* [lab=Comm] { \textsf{match}( x_{src}, x_{trgt} ) } { x_{trgt}?(y)P \; | \; x_{src}!\langle {Q} \rangle \red P\{\quotep{Q}/y}\} }
  \and \\
  \inferrule* [lab=Par] {{P} \red {P}'} {{{P} | {Q}} \red {{P}' | {Q}}}
  \and
  \inferrule* [lab=Equiv]{{{P} \scong {P}'} \andalso {{P}' \red {Q}'} \andalso {{Q}' \scong {Q}}}{{P} \red {Q}}
\end{mathpar}

\begin{eqnarray*}
  match_{\equiv} (\quotep{P},\quotep{Q}) & := & P \equiv Q \\
  match_{\dagger}(\quotep{P},\quotep{Q}) & := & \forall R. P|Q \red^{*} R => R \red^{*} 0 \\
  match_{K}(\quotep{P},\quotep{Q}) & := & K \mbox{ for some context } K
\end{eqnarray*}

$u?(x)P | u!\langle Q \rangle \red P\{\quotep{Q}/x\}$

%We write $\wred$ for $\red^*$, and $P\red$ if $\exists Q $ such that $ P \red Q$.
We write $P\red$ if $\exists Q $ such that $ P \red Q$ and $P\not\red$, otherwise.

\section{Replication}

As mentioned before, it is known that replication (and hence
recursion) can be implemented in a higher-order process algebra
\cite{SangiorgiWalker}. As our first example of calculation with the
machinery thus far presented we give the construction explicitly in
the {\rhoc}.

\begin{eqnarray}
	D_{x} & := & \prefix{x}{y}{(\binpar{\outputp{x}{y}}{@{y}})} \nonumber\\
	\bangp_{x}{P} & := & \binpar{{x}!\langle{\binpar{D_{x}}{P}}\rangle}{D_{x}} \nonumber
\end{eqnarray}

\begin{eqnarray}
	\bangp_{x}{P} & & \nonumber\\
	=
	& {x}!\langle{(\prefix{x}{y}{(\outputp{x}{y} | @{y})) | P}}\rangle 
	      | \prefix{x}{y}{(\outputp{x}{y} | @{y})} & \nonumber\\
	\red
	& (\outputp{x}{y} | @{y})\substn{\quotep{(\prefix{x}{y}{(@{y} | \outputp{x}{y})) | P}}}{y} & \nonumber\\
	=
	& \outputp{x}{\quotep{(\prefix{x}{y}{(\outputp{x}{y} | @{y})) | P}}}
	  | {(\prefix{x}{y}{(\outputp{x}{y} | @{y})) | P}} & \nonumber\\
	\red
	& \ldots & \nonumber\\
	\red^*
	& P | P | \ldots & \nonumber
\end{eqnarray}

Of course, this encoding, as an implementation, runs away, unfolding
$\bangp{P}$ eagerly. A lazier and more implementable replication
operator, restricted to input-guarded processes, may be obtained as follows.

\begin{eqnarray}
\bangp{\prefix{u}{v}{P}} 
	:= 
	\binpar{\lift{x}{\prefix{u}{v}{(\binpar{D(x)}{P})}}}{D(x)} \nonumber
\end{eqnarray}

\begin{remark}
  Note that the lazier definition still does not deal with summation
  or mixed summation (i.e. sums over input and output). The reader is
  invited to construct definitions of replication that deal with these
  features. 

  Further, the definitions are parameterized in a name, $x$. Can you,
  gentle reader, make a definition that eliminates this parameter and
  guarantees no accidental interaction between the replication
  machinery and the process being replicated -- i.e. no accidental
  sharing of names used by the process to get its work done and the
  name(s) used by the replication to effect copying. This latter
  revision of the definition of replication is crucial to obtaining
  the expected identity $!!P \sim !P$.
\end{remark}

\begin{remark}\label{rem:paradoxical_combinator}
  The reader familiar with the lambda calculus will have noticed the
  similarity between $D$ and the paradoxical combinator.

  [Ed. note: the existence of this seems to suggest we have to be more
  restrictive on the set of processes and names we admit if we are to
  support no-cloning.]
\end{remark}

\subsubsection{Bisimulation}

The computational dynamics gives rise to another kind of equivalence,
the equivalence of computational behavior. As previously mentioned
this is typically captured \emph{via} some form of bisimulation.

% The notion we use in this paper is weak barbed bisimulation
% \cite{milner91polyadicpi}.

The notion we use in this paper is derived from weak barbed
bisimulation \cite{milner91polyadicpi}. 

\begin{definition}
An \emph{observation relation}, $\downarrow_{\mathcal N}$, over a set
of names, $\mathcal N$, is the smallest relation satisfying the rules
below.

\infrule[Out-barb]{y \in {\mathcal N}, \; x \nameeq y}
		  {\outputp{x}{v} \downarrow_{\mathcal N} x}
\infrule[Par-barb]{\mbox{$P\downarrow_{\mathcal N} x$ or $Q\downarrow_{\mathcal N} x$}}
		  {\binpar{P}{Q} \downarrow_{\mathcal N} x}

We write $P \Downarrow_{\mathcal N} x$ if there is $Q$ such that 
$P \wred Q$ and $Q \downarrow_{\mathcal N} x$.
\end{definition}

\begin{definition}
%\label{def.bbisim}
An  ${\mathcal N}$-\emph{barbed bisimulation} over a set of names, ${\mathcal N}$, is a symmetric binary relation 
${\mathcal S}_{\mathcal N}$ between agents such that $P\rel{S}_{\mathcal N}Q$ implies:
\begin{enumerate}
\item If $P \red P'$ then $Q \wred Q'$ and $P'\rel{S}_{\mathcal N} Q'$.
\item If $P\downarrow_{\mathcal N} x$, then $Q\Downarrow_{\mathcal N} x$.
\end{enumerate}
$P$ is ${\mathcal N}$-barbed bisimilar to $Q$, written
$P \wbbisim_{\mathcal N} Q$, if $P \rel{S}_{\mathcal N} Q$ for some ${\mathcal N}$-barbed bisimulation ${\mathcal S}_{\mathcal N}$.
\end{definition}

$\mathcal{R} \subseteq \pi \times \pi$

$P \mathcal{R} Q => \forall P'. P \red P' \Rightarrow \exists Q'. Q \red Q', P' \mathcal{R} Q'$

$P \vdash x \Rightarrow Q \vdash x$

\begin{mathpar}
  \inferrule*[lab=Out-barb]{x \nameeq y}{{y}!\langle{Q}\rangle \vdash x}
  \and
  \inferrule*[lab=Par-barb]{\mbox{$P\vdash x$ or $Q\vdash x$}}{\binpar{P}{Q} \vdash x}
\end{mathpar}

\subsubsection{Contexts}

One of the principle advantages of computational calculi like the
$\pi$-calculus is a well-defined notion of context,
contextual-equivalence and a correlation between
contextual-equivalence and notions of bisimulation. The notion of
context allows the decomposition of a process into (sub-)process and
its syntactic environment, its context. Thus, a context may be
thought of as a process with a ``hole'' (written $\Box$) in it. The
application of a context $M$ to a process $P$, written $M[P]$, is
tantamount to filling the hole in $M$ with $P$. In this paper we do
not need the full weight of this theory, but do make use of the notion
of context in the proof the main theorem. 

\begin{mathpar}
  \inferrule* [lab=summation] {} {{M_{M},M_{N}} \bc \Box \;|\; x.M_{A} \;|\; M_{M}+M_{N}}
  \and
  \inferrule* [lab=agent] {} {{M_{A}} \bc (\vec{x})M_{P} \;| \; \clift{P_0,\ldots,M_{P},\ldots,P_N}}
  \and \\
  \inferrule* [lab=process] {} {{M_{P}} \bc M_{N} \;| \;P|M_{P} }
\end{mathpar} 

\begin{mathpar}
  \inferrule* [lab=sychronization] {} {M_{N} \bc \Box \;|\; x?M_{F} \;|\; x!M_{C}}
  \and
  \inferrule* [lab=abstraction] {} {{M_{F}} \bc (x)M_{P} }
  \and
  \inferrule* [lab=concretion] {} {{M_{C}} \bc \langle M_{P} \rangle }
  \and \\
  \inferrule* [lab=process] {} {{M_{P}} \bc M_{N} \;| \;P|M_{P} }
\end{mathpar}

\begin{definition}[contextual application] Given a context $M$, and
  process $P$, we define the \emph{contextual application}, $M[P] :=
  M\{P/\Box\}$. That is, the contextual application of M to P is the
  substitution of $P$ for $\Box$ in $M$.
\end{definition}

$\meaningof{-} : L \to \mathcal{P}(\pi)$

\begin{mathpar}
  \inferrule* [lab=collection] {} {\meaningof{true} = \pi, \and \meaningof{~E} = \pi \setminus \meaningof{E}, \and \meaningof{E_{1} \& E_{2}} = \meaningof{E_{1}} \cap \meaningof{E_{2}}}
\end{mathpar}

\begin{mathpar}
  \inferrule* [lab=structure] {} {\meaningof{0} = \{ P \in \pi | P \equiv 0 \}, \and \\ \meaningof{E_1 | E_2} = \{ P \in \pi | P \equiv P_{1} | P_{2}, P_{1} \in \meaningof{E_{1}}, P_{2} \in \meaningof{E_2}\} }
\end{mathpar}

\begin{mathpar}
 \inferrule* [lab=behavior] {} {\meaningof{\langle a?b \rangle E} = \{ P \in \pi | P \equiv Q | u?(y)P', \\ \and \\\\ \and \\ \;\;\; u \in \meaningof{a}, \forall z.P'\{z/y\} \in \meaningof{E\{z/b\}}\}, \and \\ \meaningof{a!E} = \{ P \in \pi | P \equiv Q | x!\langle P' \rangle, x \in \meaningof{a} P' \in \meaningof{E}\} }
\end{mathpar}

\begin{mathpar}
 \inferrule* [lab=nominal] {} {\meaningof{\quotep{E}} = \{ \quotep{P} \in \quotep{\pi} | P \in \meaningof{E} \}, \and \meaningof{\quotep{P}} = \{ \quotep{Q} \in \quotep{\pi} | P \equiv Q \} \and \\ \meaningof{@\quotep{E}} = \{ P \in \pi | P \equiv @x, x \in \meaningof{E} \}}
\end{mathpar}

\begin{eqnarray*}
  \\
  \meaningof{-} : TS \to ST
\end{eqnarray*}

\begin{eqnarray*}
  \\
  L : TS \to ST
\end{eqnarray*}

\begin{eqnarray*}
  \\
  P \models E \iff P \in \meaningof{E}
\end{eqnarray*}

\begin{eqnarray*}
  P \approx_{L} Q \iff \forall E \in L. P \models E \iff Q \models E
\end{eqnarray*}

\begin{eqnarray*}
  P \approx_{K} Q
\end{eqnarray*}

\begin{eqnarray*}
  P \approx Q
\end{eqnarray*}

$\approx_{K} = \approx = \approx_{L}$

\subsubsection{Contextual duality}

Note that contexts extend the quotation operation to a family of
operations from processes to names. Given a context, $M$, we can
define a \emph{nominal context}, $\quotep{M}$ by $\quotep{M}[P] :=
\quotep{M[P]}$. To foreshadow what is to come we observe that these
operations enjoy a duality with processes very much like the duality
between vectors and maps from vectors to scalars.

Further, because the calculus is essentially higher-order, we have a
correspondence between contexts and processes. More specifically,
given a name $x$ and a context $M$ we can construct $M^{*}_{x}$ such
that 

\begin{mathpar}
  M^{*}_{x} | \lift{x}{P} \red M[P]
\end{mathpar}

namely,

\begin{mathpar}
  M^{*}_{x} := x?(u).M[\dropn{u}]
\end{mathpar}

The dependence of $M^{*}_{x}$ on a name makes it an abstraction, 

\begin{mathpar}
  M^{*} := (x)x?(u).M[\dropn{u}]
\end{mathpar}

\subsection{Additional notation}

It will sometimes be convenient to denote the process a name
quotes. We already have the notation $x = \quotep{P}$, but it will be
convenient to introduce an alternate notation, $\procn{x}$, when we
want to emphasize the connection to the use of the name. Note that, by
virtue of name equivalence, $\quotep{\procn{x}} \nameeq x$; so, the
notation is consistent with previous definitions.

Further, because names have structure it is possible to effect
substitutions on the basis of that structure. This means we need to
upgrade our notation for substitutions, which we accomplish by
adapting comprehension notation. Thus,

\begin{mathpar}
  P\{ y / x : x \in S \}
\end{mathpar}

is interpreted to mean the process derived from P by replacing (in a
capture-avoiding manner) each occurrence of $x$ in $S$ by $y$. For example,

\begin{mathpar}
  P\{ \quotep{\procn{x}|\procn{x}} / x : x \in \freenames{P} \}
\end{mathpar}

will replace each (occurrence) of a free name $x$ in $P$ by
$\quotep{\procn{x}|\procn{x}}$.

Also, we will avail ourselves of the notation $x^{L}$ and $x^{R}$ to
denote injections of a name into disjoint copies of the name
space. There are numerous ways to accomplish this. One example can be
found in \cite{MeredithR05}. This notation overloads to vectors of
names: $\vec{x}^{\pi} := (x_{i}^{\pi} \; : \; 0 \leq i < |\vec{x}| )$ where $\pi \in \{L,R\}$.

We also use $P^{\Box} := P|\Box$.

In \cite{MeredithR05} an interpretation of the new operator is
given. It turns out that there are several possible interpretations
all enjoying the requisite algebraic properties of the operator (see
\cite{milner91polyadicpi}). We will therefore make liberal use of
$(\nu\; \vec{x})P$.

% subsection the_syntax_and_semantics_of_the_notation_system (end)   

\input{qm2pi.qmops} 

\input{qm2pi.sterngerlach} 

\input{qm2pi.metric} 

% section concurrent_process_calculi (end)

%\input{qm2pi.proofsketch}

% section proof sketch (end)

%\input{qm2pi.slviaknots} 

% section spatial logic via knots (end)

\input{qm2pi.conclusion}

% section conclusion (end)

%\input{qm2pi.dtcodes} 

% section wiring algorithm (end)

\input{qm2pi.ack} 

% section acknowledgments (end)

\newpage


\bibliographystyle{plain}   
\bibliography{../../biblios/main.bib}

\input{qm2pi.rhodetails}

\end{document}

 

% section concurrent_process_calculi (end)

%\documentclass[12pt]{llncs}
%\documentclass{jktr}

\usepackage[pdftex]{hyperref}                   
\usepackage {listings}
\usepackage {mathpartir}
\usepackage{bcprules}
%\usepackage{listings}
                       
\usepackage{graphicx} 
%\usepackage[margins=2.5cm,nohead,nofoot]{geometry}
%\usepackage{geometry}
\usepackage{amsfonts}
\usepackage{amstext}
\usepackage{latexsym}
\usepackage{amssymb}
\usepackage{color}


%\include{myPreamble}
\include{qm2pi.local} 

%\ifpdf
%\usepackage[pdftex]{graphicx}
%\else
%\usepackage{graphicx}
%\fi

 % \ifpdf
%  \usepackage{pdfsync}
%  \if


%\title{Brief Article}
%\author{David F. Snyder}
%\author{L.G. Meredith}

%\address{Dept. of Math., Texas State University--San Marcos, San Marcos, TX 78666}
       
\pagestyle{empty}


\begin{document}

\lstset{language=[Objective]Caml,frame=shadowbox}

\input{qm2pi.front}

% section front matter (end)

\input{qm2pi.intro} 
 
% section introduction (end)

% \input{qm2pi.knotations} 

% section notation (end)

\input{qm2pi.process.calculi} 

% section concurrent_process_calculi_and_spatial_logics_ (end)
    
%\input{qm2pi.knots2pi} 

%\input{qm2pi.trefoil} 

%\input{qm2pi.mainthm} 

% subsection basic_interpretation (end)

%\input{qm2pi.rho.presentation} 
\subsection{The syntax and semantics of the notation system}\label{sub:the_syntax_and_semantics_of_the_notation_system} % (fold)

We now summarize a technical presentation of the calculus that
embodies our theory of dynamics. The typical presentation of such a
calculus follows the style of giving generators and relations on
them. The grammar, below, describing term constructors, freely
generates the set of processes, $\Proc$. This set is then quotiented
by a relation known as structural congruence and it is over this set
that the notion of dynamics is expressed. This presentation is
essentially that of \cite{MeredithR05} with the addition of
polyadicity and summation. For readability we have relegated some of
the technical subtleties to an appendix.

\subsubsection{Process grammar}\label{subsub:process_grammar}

\begin{mathpar}
  \inferrule* [lab=synchronization] {} {{M} \bc \pzero \;|\; x?F \;|\; x!C }
  \and
  \inferrule* [lab=abstraction] {} {{F} \bc (x)P}
  \and
  \inferrule* [lab=concretion] {} {{C} \bc \langle Q \rangle}
  \and
  \inferrule* [lab=process] {} {{P,Q} \bc M \;| \;P|Q \;|\; @{x}}
  \and
  \inferrule* [lab=name] {} {{x} \bc \quotep{P}}
\end{mathpar} 

Note that $\vec{x}$ (resp. $\vec{P}$) denotes a vector of names
(resp. processes) of length $|\vec{x}|$ (resp. $|\vec{P}|$). We adopt
the following useful abbreviations.

\begin{mathpar}
   x?(\vec{y}).P := x.(\vec{y})P \and  x\clift{\vec{P}} := x.\clift{\vec{P}}
   \and x!(y) := \lift{x}{\dropn{y}}
   \and \Pi_{i=0}^{n-1}P_i := P_0 | \ldots | P_{n-1}
\end{mathpar}

\subsubsection{Structural congruence}

\paragraph{Free and bound names and alpha-equivalence.} At the
core of structural equivalence is alpha-equivalence which identifies
process that are the same up to a change of variable. Formally, we
recognize the distinction between free and bound names. The free names
of a process, $\freenames{P}$, may be calculated recursively as
follows:

\begin{mathpar}
\freenames{\pzero} := \emptyset
  \and \\
  \freenames{x?(y).P} := \{ x \} \cup (\freenames{P} \setminus \{ y \})
  \and 
  \freenames{x!\langle P \rangle} := \{ x \} \cup \{ P \} 
  \and \\
  \freenames{P|Q} := \freenames{P} \cup \freenames{Q}
  \and \\
  \freenames{@{x}} := \{ x \}
\end{mathpar}

$\pi$
$\quotep{\pi}$

$\freenames{-} : \pi \to \mathcal{P}(\quotep{\pi})$

\begin{eqnarray*}
  \freenames{\pzero} & := & \emptyset \\
  \freenames{x?(y).P} & := & \{ x \} \cup (\freenames{P} \setminus \{ y \}) \\
  \freenames{x!\langle P \rangle} & := & \{ x \} \cup \{ P \} \\
  \freenames{P|Q} & := & \freenames{P} \cup \freenames{Q} \\
  \freenames{\dropn{x}} & := & \{ x \}
\end{eqnarray*}

The bound names of a process, $\boundnames{P}$, are those names occurring in $P$
that are not free. For example, in $x?(y).0$, the name $x$ is free, while $y$ is bound.

\begin{mathpar}
  \inferrule* [lab=monoidal-laws] {} { P|Q \equiv Q|P \and P|0 \equiv P \and P|(Q|R) \equiv (P|Q)|R }
\end{mathpar}

\begin{mathpar}
  \inferrule* [lab=alpha-equivalence] {} { (x)P \equiv (y)P\{y/x\} \and y \not\in \freenames{P} }
\end{mathpar}

\begin{definition}
Then two processes, $P,Q$, are alpha-equivalent if $P = Q\{\vec{y}/\vec{x}\}$ for
some $\vec{x} \in \boundnames{Q},\vec{y} \in \boundnames{P}$, where $Q\{\vec{y}/\vec{x}\}$
denotes the capture-avoiding substitution of $\vec{y}$ for $\vec{x}$ in $Q$.
\end{definition}

\begin{definition}
  The {\em structural congruence} \cite{SangiorgiWalker} , $\equiv$,
  between processes is the least congruence containing
  alpha-equivalence, satisfying the abelian monoid laws
  (associativity, commutativity and $\pzero$ as identity) for parallel
  composition $|$ and for summation $+$.
\end{definition}

\subsection{Name equivalence}

We take name equivalence, written $\nameeq$, to be the smallest
equivalence relation generated by the following rules.

\begin{mathpar}
\inferrule*[lab=Quote-drop]
{ }
{ \quotep{@{x}} \nameeq x }

\inferrule*[lab=Struct-equiv]
{ P \scong Q }
{ \quotep{P} \nameeq \quotep{Q} }
\end{mathpar}

The astute reader will have noticed that the mutual recursion of names
and processes imposes a mutual recursion on alpha-equivalence and
structural equivalence via name-equivalence. Fortunately, all of this
works out pleasantly and we may calculate in the natural way, free of
concern. The reader interested in the details is referred to the
appendix \ref{appendix:rho_details}.

\subsection{Substitution}

We use $\Proc$ for the set of processes, $\QProc$ for the set of
names, and $\id{\{}\vec{y} / \vec{x} \id{\}}$ to denote partial maps,
$s : \QProc \rightarrow \QProc$. A map, $s$ lifts, uniquely, to a map
on process terms, $\widehat{s} : \Proc \rightarrow \Proc$ by the
following equations.

\begin{mathpar}
  (0) \psubstp{Q}{P} := 0 \\
  (R \juxtap S) \psubstp{Q}{P}
  :=    
  (R)\psubstp{Q}{P} \juxtap (S) \psubstp{Q}{P} \\
  (x?(y).R) \psubstp{Q}{P}    
  :=    
  (x)\substp{Q}{P} (z)\concat( (R \psubstn{z}{y}) \psubstp{Q}{P} ) \\
  (\lift{x}{R}) \psubstp{Q}{P}  
  :=
  \lift{(x)\substp{Q}{P}}{ R \psubstp{Q}{P} } \\
%   (\dropn{x})  \psubstp{Q}{P}       
%   := 
%   \left\{ 
%     \begin{array}{ccc} 
%       \dropn{\quotep{Q}} & & x \nameeq \quotep{P} \\
%       \dropn{x} & & otherwise \\
%     \end{array}
%   \right. 
  (\dropn{x})  \psubstp{Q}{P}       
  := 
  \left\{ 
    \begin{array}{ccc} 
      Q & & x \nameeq \quotep{P} \\
      \dropn{x} & & otherwise \\
    \end{array}
  \right.
\end{mathpar}
 

where

\begin{eqnarray}
  (x)\id{\{} \lpquote Q \rpquote / \lpquote P \rpquote \id{\}}            = 
  \left\{ 
    \begin{array}{ccc}
      \lpquote Q \rpquote & & x \nameeq \lpquote P \rpquote \\
      x & & otherwise \\
    \end{array}
  \right. \nonumber
\end{eqnarray}

and $z$ is chosen distinct from $\quotep{P}$, $\quotep{Q}$, the free
names in $Q$, and all the names in $R$. Our $\alpha$-equivalence will
be built in the standard way from this substitution.

\begin{remark}\label{rem:no_self_referential_names}
  One consequence of these definitions is that $\forall P. \quotep{P}
  \not\in \freenames{P}$.
\end{remark}

\subsection{ Dynamic quote: an example }

Anticipating something of what's to come, consider applying the
substitution, $\widehat{\id{\{}u / z \id{\}}}$, to the following pair
of processes, $\lift{w}{y!(z)}$ and $w[ \lpquote y!(z) \rpquote ]$.

\begin{eqnarray}
	\lift{w}{y!(z)}\widehat{\id{\{}u / z \id{\}}}
		& = &
		\lift{w}{y!(u)} \nonumber\\
	w[ \lpquote y!(z) \rpquote ] \widehat{ \id{\{}u / z \id{\}} }
		& = &
		w[ \lpquote y!(z) \rpquote ] \nonumber
\end{eqnarray}

Because the body of the process between quotes is impervious to
substitution, we get radically different answers. In fact, by
examining the first process in an input context,
e.g. $x?(z).\lift{w}{y!(z)}$, we see that the process under the lift
operator may be shaped by prefixed inputs binding a name inside it. In
this sense, the lift operator will be seen as a way to dynamically
construct processes before reifying them as names.

Finally equipped with these standard features we can present the
dynamics of the calculus.

\subsubsection{Operational semantics} 

Finally, we introduce the computational dynamics. What marks these
algebras as distinct from other more traditionally studied algebraic
structures, e.g. vector spaces or polynomial rings, is the manner in
which dynamics is captured. In traditional structures, dynamics is typically
expressed through morphisms between such structures, as in linear maps
between vector spaces or morphisms between rings. In algebras
associated with the semantics of computation, the dynamics is
expressed as part of the algebraic structure itself, through a
reduction reduction relation typically denoted by $\red$. Below, we
give a recursive presentation of this relation for the calculus used
in the encoding.

$\red \subseteq \pi \times \pi$
$\red : \pi \to \mathcal{P}(\pi)$

\begin{mathpar}
  \inferrule* [lab=Comm] { \textsf{match}( x_{src}, x_{trgt} ) } { x_{trgt}?(y)P \; | \; x_{src}!\langle {Q} \rangle \red P\{\quotep{Q}/y}\} }
  \and \\
  \inferrule* [lab=Par] {{P} \red {P}'} {{{P} | {Q}} \red {{P}' | {Q}}}
  \and
  \inferrule* [lab=Equiv]{{{P} \scong {P}'} \andalso {{P}' \red {Q}'} \andalso {{Q}' \scong {Q}}}{{P} \red {Q}}
\end{mathpar}

\begin{eqnarray*}
  match_{\equiv} (\quotep{P},\quotep{Q}) & := & P \equiv Q \\
  match_{\dagger}(\quotep{P},\quotep{Q}) & := & \forall R. P|Q \red^{*} R => R \red^{*} 0 \\
  match_{K}(\quotep{P},\quotep{Q}) & := & K \mbox{ for some context } K
\end{eqnarray*}

$u?(x)P | u!\langle Q \rangle \red P\{\quotep{Q}/x\}$

%We write $\wred$ for $\red^*$, and $P\red$ if $\exists Q $ such that $ P \red Q$.
We write $P\red$ if $\exists Q $ such that $ P \red Q$ and $P\not\red$, otherwise.

\section{Replication}

As mentioned before, it is known that replication (and hence
recursion) can be implemented in a higher-order process algebra
\cite{SangiorgiWalker}. As our first example of calculation with the
machinery thus far presented we give the construction explicitly in
the {\rhoc}.

\begin{eqnarray}
	D_{x} & := & \prefix{x}{y}{(\binpar{\outputp{x}{y}}{@{y}})} \nonumber\\
	\bangp_{x}{P} & := & \binpar{{x}!\langle{\binpar{D_{x}}{P}}\rangle}{D_{x}} \nonumber
\end{eqnarray}

\begin{eqnarray}
	\bangp_{x}{P} & & \nonumber\\
	=
	& {x}!\langle{(\prefix{x}{y}{(\outputp{x}{y} | @{y})) | P}}\rangle 
	      | \prefix{x}{y}{(\outputp{x}{y} | @{y})} & \nonumber\\
	\red
	& (\outputp{x}{y} | @{y})\substn{\quotep{(\prefix{x}{y}{(@{y} | \outputp{x}{y})) | P}}}{y} & \nonumber\\
	=
	& \outputp{x}{\quotep{(\prefix{x}{y}{(\outputp{x}{y} | @{y})) | P}}}
	  | {(\prefix{x}{y}{(\outputp{x}{y} | @{y})) | P}} & \nonumber\\
	\red
	& \ldots & \nonumber\\
	\red^*
	& P | P | \ldots & \nonumber
\end{eqnarray}

Of course, this encoding, as an implementation, runs away, unfolding
$\bangp{P}$ eagerly. A lazier and more implementable replication
operator, restricted to input-guarded processes, may be obtained as follows.

\begin{eqnarray}
\bangp{\prefix{u}{v}{P}} 
	:= 
	\binpar{\lift{x}{\prefix{u}{v}{(\binpar{D(x)}{P})}}}{D(x)} \nonumber
\end{eqnarray}

\begin{remark}
  Note that the lazier definition still does not deal with summation
  or mixed summation (i.e. sums over input and output). The reader is
  invited to construct definitions of replication that deal with these
  features. 

  Further, the definitions are parameterized in a name, $x$. Can you,
  gentle reader, make a definition that eliminates this parameter and
  guarantees no accidental interaction between the replication
  machinery and the process being replicated -- i.e. no accidental
  sharing of names used by the process to get its work done and the
  name(s) used by the replication to effect copying. This latter
  revision of the definition of replication is crucial to obtaining
  the expected identity $!!P \sim !P$.
\end{remark}

\begin{remark}\label{rem:paradoxical_combinator}
  The reader familiar with the lambda calculus will have noticed the
  similarity between $D$ and the paradoxical combinator.

  [Ed. note: the existence of this seems to suggest we have to be more
  restrictive on the set of processes and names we admit if we are to
  support no-cloning.]
\end{remark}

\subsubsection{Bisimulation}

The computational dynamics gives rise to another kind of equivalence,
the equivalence of computational behavior. As previously mentioned
this is typically captured \emph{via} some form of bisimulation.

% The notion we use in this paper is weak barbed bisimulation
% \cite{milner91polyadicpi}.

The notion we use in this paper is derived from weak barbed
bisimulation \cite{milner91polyadicpi}. 

\begin{definition}
An \emph{observation relation}, $\downarrow_{\mathcal N}$, over a set
of names, $\mathcal N$, is the smallest relation satisfying the rules
below.

\infrule[Out-barb]{y \in {\mathcal N}, \; x \nameeq y}
		  {\outputp{x}{v} \downarrow_{\mathcal N} x}
\infrule[Par-barb]{\mbox{$P\downarrow_{\mathcal N} x$ or $Q\downarrow_{\mathcal N} x$}}
		  {\binpar{P}{Q} \downarrow_{\mathcal N} x}

We write $P \Downarrow_{\mathcal N} x$ if there is $Q$ such that 
$P \wred Q$ and $Q \downarrow_{\mathcal N} x$.
\end{definition}

\begin{definition}
%\label{def.bbisim}
An  ${\mathcal N}$-\emph{barbed bisimulation} over a set of names, ${\mathcal N}$, is a symmetric binary relation 
${\mathcal S}_{\mathcal N}$ between agents such that $P\rel{S}_{\mathcal N}Q$ implies:
\begin{enumerate}
\item If $P \red P'$ then $Q \wred Q'$ and $P'\rel{S}_{\mathcal N} Q'$.
\item If $P\downarrow_{\mathcal N} x$, then $Q\Downarrow_{\mathcal N} x$.
\end{enumerate}
$P$ is ${\mathcal N}$-barbed bisimilar to $Q$, written
$P \wbbisim_{\mathcal N} Q$, if $P \rel{S}_{\mathcal N} Q$ for some ${\mathcal N}$-barbed bisimulation ${\mathcal S}_{\mathcal N}$.
\end{definition}

$\mathcal{R} \subseteq \pi \times \pi$

$P \mathcal{R} Q => \forall P'. P \red P' \Rightarrow \exists Q'. Q \red Q', P' \mathcal{R} Q'$

$P \vdash x \Rightarrow Q \vdash x$

\begin{mathpar}
  \inferrule*[lab=Out-barb]{x \nameeq y}{{y}!\langle{Q}\rangle \vdash x}
  \and
  \inferrule*[lab=Par-barb]{\mbox{$P\vdash x$ or $Q\vdash x$}}{\binpar{P}{Q} \vdash x}
\end{mathpar}

\subsubsection{Contexts}

One of the principle advantages of computational calculi like the
$\pi$-calculus is a well-defined notion of context,
contextual-equivalence and a correlation between
contextual-equivalence and notions of bisimulation. The notion of
context allows the decomposition of a process into (sub-)process and
its syntactic environment, its context. Thus, a context may be
thought of as a process with a ``hole'' (written $\Box$) in it. The
application of a context $M$ to a process $P$, written $M[P]$, is
tantamount to filling the hole in $M$ with $P$. In this paper we do
not need the full weight of this theory, but do make use of the notion
of context in the proof the main theorem. 

\begin{mathpar}
  \inferrule* [lab=summation] {} {{M_{M},M_{N}} \bc \Box \;|\; x.M_{A} \;|\; M_{M}+M_{N}}
  \and
  \inferrule* [lab=agent] {} {{M_{A}} \bc (\vec{x})M_{P} \;| \; \clift{P_0,\ldots,M_{P},\ldots,P_N}}
  \and \\
  \inferrule* [lab=process] {} {{M_{P}} \bc M_{N} \;| \;P|M_{P} }
\end{mathpar} 

\begin{mathpar}
  \inferrule* [lab=sychronization] {} {M_{N} \bc \Box \;|\; x?M_{F} \;|\; x!M_{C}}
  \and
  \inferrule* [lab=abstraction] {} {{M_{F}} \bc (x)M_{P} }
  \and
  \inferrule* [lab=concretion] {} {{M_{C}} \bc \langle M_{P} \rangle }
  \and \\
  \inferrule* [lab=process] {} {{M_{P}} \bc M_{N} \;| \;P|M_{P} }
\end{mathpar}

\begin{definition}[contextual application] Given a context $M$, and
  process $P$, we define the \emph{contextual application}, $M[P] :=
  M\{P/\Box\}$. That is, the contextual application of M to P is the
  substitution of $P$ for $\Box$ in $M$.
\end{definition}

$\meaningof{-} : L \to \mathcal{P}(\pi)$

\begin{mathpar}
  \inferrule* [lab=collection] {} {\meaningof{true} = \pi, \and \meaningof{~E} = \pi \setminus \meaningof{E}, \and \meaningof{E_{1} \& E_{2}} = \meaningof{E_{1}} \cap \meaningof{E_{2}}}
\end{mathpar}

\begin{mathpar}
  \inferrule* [lab=structure] {} {\meaningof{0} = \{ P \in \pi | P \equiv 0 \}, \and \\ \meaningof{E_1 | E_2} = \{ P \in \pi | P \equiv P_{1} | P_{2}, P_{1} \in \meaningof{E_{1}}, P_{2} \in \meaningof{E_2}\} }
\end{mathpar}

\begin{mathpar}
 \inferrule* [lab=behavior] {} {\meaningof{\langle a?b \rangle E} = \{ P \in \pi | P \equiv Q | u?(y)P', \\ \and \\\\ \and \\ \;\;\; u \in \meaningof{a}, \forall z.P'\{z/y\} \in \meaningof{E\{z/b\}}\}, \and \\ \meaningof{a!E} = \{ P \in \pi | P \equiv Q | x!\langle P' \rangle, x \in \meaningof{a} P' \in \meaningof{E}\} }
\end{mathpar}

\begin{mathpar}
 \inferrule* [lab=nominal] {} {\meaningof{\quotep{E}} = \{ \quotep{P} \in \quotep{\pi} | P \in \meaningof{E} \}, \and \meaningof{\quotep{P}} = \{ \quotep{Q} \in \quotep{\pi} | P \equiv Q \} \and \\ \meaningof{@\quotep{E}} = \{ P \in \pi | P \equiv @x, x \in \meaningof{E} \}}
\end{mathpar}

\begin{eqnarray*}
  \\
  \meaningof{-} : TS \to ST
\end{eqnarray*}

\begin{eqnarray*}
  \\
  L : TS \to ST
\end{eqnarray*}

\begin{eqnarray*}
  \\
  P \models E \iff P \in \meaningof{E}
\end{eqnarray*}

\begin{eqnarray*}
  P \approx_{L} Q \iff \forall E \in L. P \models E \iff Q \models E
\end{eqnarray*}

\begin{eqnarray*}
  P \approx_{K} Q
\end{eqnarray*}

\begin{eqnarray*}
  P \approx Q
\end{eqnarray*}

$\approx_{K} = \approx = \approx_{L}$

\subsubsection{Contextual duality}

Note that contexts extend the quotation operation to a family of
operations from processes to names. Given a context, $M$, we can
define a \emph{nominal context}, $\quotep{M}$ by $\quotep{M}[P] :=
\quotep{M[P]}$. To foreshadow what is to come we observe that these
operations enjoy a duality with processes very much like the duality
between vectors and maps from vectors to scalars.

Further, because the calculus is essentially higher-order, we have a
correspondence between contexts and processes. More specifically,
given a name $x$ and a context $M$ we can construct $M^{*}_{x}$ such
that 

\begin{mathpar}
  M^{*}_{x} | \lift{x}{P} \red M[P]
\end{mathpar}

namely,

\begin{mathpar}
  M^{*}_{x} := x?(u).M[\dropn{u}]
\end{mathpar}

The dependence of $M^{*}_{x}$ on a name makes it an abstraction, 

\begin{mathpar}
  M^{*} := (x)x?(u).M[\dropn{u}]
\end{mathpar}

\subsection{Additional notation}

It will sometimes be convenient to denote the process a name
quotes. We already have the notation $x = \quotep{P}$, but it will be
convenient to introduce an alternate notation, $\procn{x}$, when we
want to emphasize the connection to the use of the name. Note that, by
virtue of name equivalence, $\quotep{\procn{x}} \nameeq x$; so, the
notation is consistent with previous definitions.

Further, because names have structure it is possible to effect
substitutions on the basis of that structure. This means we need to
upgrade our notation for substitutions, which we accomplish by
adapting comprehension notation. Thus,

\begin{mathpar}
  P\{ y / x : x \in S \}
\end{mathpar}

is interpreted to mean the process derived from P by replacing (in a
capture-avoiding manner) each occurrence of $x$ in $S$ by $y$. For example,

\begin{mathpar}
  P\{ \quotep{\procn{x}|\procn{x}} / x : x \in \freenames{P} \}
\end{mathpar}

will replace each (occurrence) of a free name $x$ in $P$ by
$\quotep{\procn{x}|\procn{x}}$.

Also, we will avail ourselves of the notation $x^{L}$ and $x^{R}$ to
denote injections of a name into disjoint copies of the name
space. There are numerous ways to accomplish this. One example can be
found in \cite{MeredithR05}. This notation overloads to vectors of
names: $\vec{x}^{\pi} := (x_{i}^{\pi} \; : \; 0 \leq i < |\vec{x}| )$ where $\pi \in \{L,R\}$.

We also use $P^{\Box} := P|\Box$.

In \cite{MeredithR05} an interpretation of the new operator is
given. It turns out that there are several possible interpretations
all enjoying the requisite algebraic properties of the operator (see
\cite{milner91polyadicpi}). We will therefore make liberal use of
$(\nu\; \vec{x})P$.

% subsection the_syntax_and_semantics_of_the_notation_system (end)   

\input{qm2pi.qmops} 

\input{qm2pi.sterngerlach} 

\input{qm2pi.metric} 

% section concurrent_process_calculi (end)

%\input{qm2pi.proofsketch}

% section proof sketch (end)

%\input{qm2pi.slviaknots} 

% section spatial logic via knots (end)

\input{qm2pi.conclusion}

% section conclusion (end)

%\input{qm2pi.dtcodes} 

% section wiring algorithm (end)

\input{qm2pi.ack} 

% section acknowledgments (end)

\newpage


\bibliographystyle{plain}   
\bibliography{../../biblios/main.bib}

\input{qm2pi.rhodetails}

\end{document}



% section proof sketch (end)

%\section{Unlikely characters: spatial logic for
  knots}\label{sub:characteristic_formulae} % (fold)

Associated to the mobile process calculi are a family of logics known
as the Hennessy-Milner logics. These logics typically enjoy a
semantics interpreting formulae as sets of processes that when
factored through the encoding outlined above allows an identification
of classes of knots with logical formulae. In the context of this
encoding the sub-family known as the spatial logics \cite{CairesC03}
\cite{CairesC04} \cite{Caires04} are of particular interest providing
several important features for expressing and reasoning about
properties (i.e. classes) of knots. We hint here at how this may be done.

%\begin{description}
%\item [structural connectives] 
\subsubsection{Structural connectives} The spatial logics enjoy
structural connectives corresponding, at the logical level, to the
parallel composition ($P | Q$) and new name ($(\nu \; x)P$)
connectives for processes. As illustrated in the examples below, these
connectives are extremely expressive given the shape of our encoding.
%\item [decideable satisfaction]

\subsubsection{Decideable satisfaction}
In \cite{Caires04} the satisfaction relation is shown to be decideable
for a rich class of processes. It further turns out that the image of
the our encoding is a proper subset of that class. This result
provides the basis for an algorithm by which to search for knots
enjoying a given property.
%\item [characteristic formulae]

\subsubsection{Characteristic formulae}
In the same paper \cite{Caires04} , Caires presents a means of calculating
characteristic formulae, selecting equivalence classes of processes
up to a pre--specified depth limit on the support set of names. Composed with our
encoding, this characteristic formula can be used to select
characteristic formulae for knots.
%\end{description}

\subsubsection{Spatial logic formulae}

The grammar below (segmented for comprehension) summarizes the syntax
of spatial logic formulae. We employ illustrative examples in the
sequel to provide an intuitive understanding of their meaning
referring the reader to \cite{Caires04} for a more detailed explication
of the semantics.

\begin{mathpar}
  \inferrule* [lab=boolean] {} {{A,B} \bc T \;|\; \neg A \;|\; A \wedge B \;|\; \eta = \eta'}
  \and
  \inferrule* [lab=spatial] {} {|\; \pzero \;|\; A | B \;|\; x \text{\textregistered} A \;|\; \forall x . A \;|\;  H x . A}
  \and
  \inferrule* [lab=behavioral] {} {|\; \alpha . A}
  \and 
  \inferrule* [lab=recursion] {} {|\; X(\vec{u}) \;|\; \mu X(\vec{u}) . A}
  \and
  \inferrule* [lab=action] {} {\alpha \bc \langle x?(\vec{y}) \rangle \;|\; \langle x!(\vec{y}) \rangle \;|\; \langle \tau \rangle}
  \and 
  \inferrule* [lab=name] {} {\eta \bc x \;|\; \tau}
\end{mathpar} 

% subsection characteristic_formulae (end)   	 

\subsection{Example formulae}\label{sub:example_formulae_} % (fold)

\subsubsection{Crossing as formula.}
% 
% \begin{align*}
%   \frac{d}{dx} \sin x &= \cos x 
%   & \frac{d}{dx} e^x &= e^x \\
%   \frac{d}{dx} \cos x &= - \sin x 
%   & \frac{d}{dx} \log x &= \frac{1}{x} \\
% \end{align*} 

\begin{align*}
 \mu C(x_{0},x_{1},y_{0},y_{1},u).&(\langle x_{0}?(z) \rangle(\langle u! \rangle\langle y_{1}!z \rangle C(x_{0},x_{1},y_{0},y_{1},u)) & \\
  & \wedge \langle y_{1}?(z) \rangle (\langle u! \rangle \langle x_{0}!z \rangle C(x_{0},x_{1},y_{0},y_{1},u)) & \\
  & \wedge \langle x_{1}?(z) \rangle (\langle u? \rangle \langle y_{0}!z \rangle C(x_{0},x_{1},y_{0},y_{1},u)) & \\
  & \wedge \langle y_{0}?(z) \rangle (\langle u? \rangle \langle x_{1}!z \rangle C(x_{0},x_{1},y_{0},y_{1},u))) &
\end{align*}

The lexicographical similarity between the shape of this formulae and
the shape of definition of the process representing a crossing reveals
the intuitive meaning of this formulae. It describes the capabilities
of a process that has the right to represent a crossing. For example
it picks out processes that may perform an input on the port $x_0$ in
its initial menu of capabilities. What differentiates the formula
from the process, however, is that the crossing process is the
smallest candidate to satisfy the formula. Infinitely many other
processes -- with internal behavior hidden behind this interface, so
to speak -- also satisfy this formula. Even this simple formula,
then, can be seen to open a new view onto knots, providing a
computational interpretation of \emph{virtual} knots.

Note that this formula is derived by hand. A similar formula can be
derived by employing Caires' calculation of characteristic formula
\cite{Caires04} to the process representing a crossing. In light of
this discussion, we let
$\meaningof{C}_{\phi}(x0,x1,y0,y1,u)$ denote a formula specifying the
dynamics we wish to capture of a crossing. To guarantee we preserve
the shape of the interface and minimal semantics we demand that
$\meaningof{C}_{\phi}(x0,x1,y0,y1,u) \Rightarrow
\textbf{C}(x0,x1,y0,y1,u)$ where $\textbf{C}(x0,x1,y0,y1,u)$ denotes
the formula above.
                            
\subsubsection{Crossing number constraints.}
The moral content of the context lemma (Lemma \ref{context}) is that the notion of
``locality'' in the Reidemeister moves is effectively captured by the
parallel composition operator of the process calculus. This intuition
extends through the logic. Given a formula,
$\meaningof{C}_{\phi}(x0,x1,y0,y1,u)$, we can use the structural
connectives to specify constraints on crossing numbers, such as at
least $n$ crossings, or exactly $n$ crossings.
\begin{mathpar}
  \inferrule* [lab=at-least-n] {} { K^{\geq n}_{\phi}(\vec{xs},\vec{ys}) := \Pi_{i=0}^{n-1} Hu . \meaningof{C}_{\phi}(xs_i,ys_i,u) | T }
  \and 
  \inferrule* [lab=exactly-n] {} { K^{= n}_{\phi}(\vec{xs},\vec{ys}) := \Pi_{i=0}^{n-1} Hu . \meaningof{C}_{\phi}(xs_i,ys_i,u) | \neg (\forall x_0,y_0,x_1,y_1,u . \meaningof{C}_{\phi}(x_0,y_0,x_1,y_1,u) | T) }
\end{mathpar}

To round out this section, recall that the encoding of an $n$-crossing
knot decomposes into a parallel composition of $n$ \emph{copies} of a
crossing process together with a wiring harness. To specify different
knot classes with the same crossing number amounts to specifying
logical constraints on the wiring harness. In the interest of space,
we defer examples to a forthcoming paper. Suffice it to say that both
the conditions ``alternating knot'' and ``contains the tangle
corresponding to 5/3'' are expressible. For example, it is possible to
calculate the characteristic formula of a process corresponding to the
tangle 5/3 and conjoin it into the classifying formula via the
composition connective of the logic.

Finally, we wish to observe that it is entirely within reason to
contemplate a more domain-specific version of spatial logic tailored
to the shape of processes in the image of the encoding. Such a
domain-specific logic would have a better claim to the title formal
language of knot properties.

% subsection example_formulae_ (end)

% section knots_as_processes (end) 

% section spatial logic via knots (end)

\section{Conclusions and future work}

\paragraph{Testing physical space}
You, gentle reader, may wonder why of all the theorems to be proved
given this set up we pick the one above. In some sense it's hardly
central to quantum mechanics. We see it as central in the sense that
it firmly establishes a notion of physical space arising from a notion
of the equivalence of behavior. Relating bisimulation to a metric is a
big step forward, but one is faced with interpreting the relationship
of that metric space to something more physical. Quantum mechanical
notions of ``physical'' space are still far from intuitive, but by
relating this idea of distance as testing to calculations that predict
physical circumstances we are making a not insignificant step forward
toward an understanding of the physical space we inhabit as
essentially dynamic.

\paragraph{Effectivity and simulation}
One of the observations we have yet to make is that the entire program
spelled out here is effective. We have built various interpreters for
the reflective calculus at work in this interpretation. In principle,
then, we can simulate quantum mechanics on a computer. The place where
the simulation may lose fidelity is the infinitely branching summation
for the annihilator.

In this connection i also want to point out that the evaluation style
calculation of the inner product puts the non-determinism of the
summation right at the heart of measurement. This suggests that
Milner's original reduction-based formulation of the dynamics of his
calculi in terms of sums was not just notationally suggestive of a
notion of measure-and-continue but captured some significant part of
the physics.

\paragraph{Quantum continuations}
In light of this last observation i want to point out that the
predominant account of quantum mechanics is missing a key aspect of a
truly compositional story of the physical situation. In a real lab,
when a measurement is made the observation can be made to feed into
another device that then makes another measurement conditioned on the
results of the first. This means that after the superposition was
collapsed the entire experimental set up remained in
superposition. While QM offers a means of writing this down it doesn't
quite line up well with the well-trodden formulation of computation
and continuation that we see so succinctly expressed in Milner's
calculi. This suggests that there might be advantages to this account
of dynamics waiting to be explored.

\paragraph{Quantum logic}
In this connection, we also note that by virtue of having the
Hennessy-Milner construction, we can pull the construction through the
interpretation of QM. This gives us a natural candidate for a quantum
logic that enjoys an extremely tight connection with it's domain of
interpretation, making the construction much less ad hoc (rather it is
the image of functor!).

\paragraph{Quantum probabiity}
i have questions about the basis of the interpretation of inner
product as probability amplitude. In particular, using which
axiomatization of probability theory does the notion of probability
amplitude earn the right to be so dubbed? In other words, where is the
proof that the operation for calculating a probability amplitude (and
then squaring) satisfies the axioms of what it means to calculate a
probability? Even if such a proof exists (i have yet to find it in the
literature), i wonder if it might not be possible to turn things on
their heads. Can we view the calculation of the probability amplitude
as an axiomatization of probability? If so, then the definition we
give for calculating probability amplitude may provide the basis for
an \emph{effective} theory of probability.

\paragraph{Quantum vs ``biological'' information}
Finally, i want to conclude with a more philosophical observation. At
a recent workshop in which QM was a predominant topic i noticed
something about quantum information. The speaker was giving a riveting
discussion of axiomatic QM and showing how properties of ``no
cloning'' and ``no deleting'' emerged as consequences of the
axiomatization. Theorems of this form are necessary to give us a sense
of confidence that our axioms characterize the physical theory. What
struck me, though, was that if quantum information is neither erasable
nor replicable it is markedly different from \emph{life}. Two of the
things we know about life is that

\begin{itemize}
  \item it ends;
  \item to gain some measure of persistence, to transcend it's
    finitude it is imminently copyable.
\end{itemize}

Both of these qualities are summarized succinctly in the aphorism: all
flesh is grass. For me these two kinds of ``information'' -- call them
quantum and biological -- are end points on a spectrum of strategies
for persistence. At one end, we have those curious entities that enjoy
uniqueness and permanence; at the other, we have those who in the face
of a certain end and an uncertain present make a go of passing
something on. To me one of the more remarkable aspects of the latter
strategy is that in the presence of noise (and certain features of
copying) we get a kind of dynamism, a chance for improvement against a
given persistent condition.

% subsection other_calculi_other_bisimulations_and_geometry_as_behavior (end)




% section conclusion (end)

%\documentclass[12pt]{llncs}
%\documentclass{jktr}

\usepackage[pdftex]{hyperref}                   
\usepackage {listings}
\usepackage {mathpartir}
\usepackage{bcprules}
%\usepackage{listings}
                       
\usepackage{graphicx} 
%\usepackage[margins=2.5cm,nohead,nofoot]{geometry}
%\usepackage{geometry}
\usepackage{amsfonts}
\usepackage{amstext}
\usepackage{latexsym}
\usepackage{amssymb}
\usepackage{color}


%\include{myPreamble}
\include{qm2pi.local} 

%\ifpdf
%\usepackage[pdftex]{graphicx}
%\else
%\usepackage{graphicx}
%\fi

 % \ifpdf
%  \usepackage{pdfsync}
%  \if


%\title{Brief Article}
%\author{David F. Snyder}
%\author{L.G. Meredith}

%\address{Dept. of Math., Texas State University--San Marcos, San Marcos, TX 78666}
       
\pagestyle{empty}


\begin{document}

\lstset{language=[Objective]Caml,frame=shadowbox}

\input{qm2pi.front}

% section front matter (end)

\input{qm2pi.intro} 
 
% section introduction (end)

% \input{qm2pi.knotations} 

% section notation (end)

\input{qm2pi.process.calculi} 

% section concurrent_process_calculi_and_spatial_logics_ (end)
    
%\input{qm2pi.knots2pi} 

%\input{qm2pi.trefoil} 

%\input{qm2pi.mainthm} 

% subsection basic_interpretation (end)

%\input{qm2pi.rho.presentation} 
\subsection{The syntax and semantics of the notation system}\label{sub:the_syntax_and_semantics_of_the_notation_system} % (fold)

We now summarize a technical presentation of the calculus that
embodies our theory of dynamics. The typical presentation of such a
calculus follows the style of giving generators and relations on
them. The grammar, below, describing term constructors, freely
generates the set of processes, $\Proc$. This set is then quotiented
by a relation known as structural congruence and it is over this set
that the notion of dynamics is expressed. This presentation is
essentially that of \cite{MeredithR05} with the addition of
polyadicity and summation. For readability we have relegated some of
the technical subtleties to an appendix.

\subsubsection{Process grammar}\label{subsub:process_grammar}

\begin{mathpar}
  \inferrule* [lab=synchronization] {} {{M} \bc \pzero \;|\; x?F \;|\; x!C }
  \and
  \inferrule* [lab=abstraction] {} {{F} \bc (x)P}
  \and
  \inferrule* [lab=concretion] {} {{C} \bc \langle Q \rangle}
  \and
  \inferrule* [lab=process] {} {{P,Q} \bc M \;| \;P|Q \;|\; @{x}}
  \and
  \inferrule* [lab=name] {} {{x} \bc \quotep{P}}
\end{mathpar} 

Note that $\vec{x}$ (resp. $\vec{P}$) denotes a vector of names
(resp. processes) of length $|\vec{x}|$ (resp. $|\vec{P}|$). We adopt
the following useful abbreviations.

\begin{mathpar}
   x?(\vec{y}).P := x.(\vec{y})P \and  x\clift{\vec{P}} := x.\clift{\vec{P}}
   \and x!(y) := \lift{x}{\dropn{y}}
   \and \Pi_{i=0}^{n-1}P_i := P_0 | \ldots | P_{n-1}
\end{mathpar}

\subsubsection{Structural congruence}

\paragraph{Free and bound names and alpha-equivalence.} At the
core of structural equivalence is alpha-equivalence which identifies
process that are the same up to a change of variable. Formally, we
recognize the distinction between free and bound names. The free names
of a process, $\freenames{P}$, may be calculated recursively as
follows:

\begin{mathpar}
\freenames{\pzero} := \emptyset
  \and \\
  \freenames{x?(y).P} := \{ x \} \cup (\freenames{P} \setminus \{ y \})
  \and 
  \freenames{x!\langle P \rangle} := \{ x \} \cup \{ P \} 
  \and \\
  \freenames{P|Q} := \freenames{P} \cup \freenames{Q}
  \and \\
  \freenames{@{x}} := \{ x \}
\end{mathpar}

$\pi$
$\quotep{\pi}$

$\freenames{-} : \pi \to \mathcal{P}(\quotep{\pi})$

\begin{eqnarray*}
  \freenames{\pzero} & := & \emptyset \\
  \freenames{x?(y).P} & := & \{ x \} \cup (\freenames{P} \setminus \{ y \}) \\
  \freenames{x!\langle P \rangle} & := & \{ x \} \cup \{ P \} \\
  \freenames{P|Q} & := & \freenames{P} \cup \freenames{Q} \\
  \freenames{\dropn{x}} & := & \{ x \}
\end{eqnarray*}

The bound names of a process, $\boundnames{P}$, are those names occurring in $P$
that are not free. For example, in $x?(y).0$, the name $x$ is free, while $y$ is bound.

\begin{mathpar}
  \inferrule* [lab=monoidal-laws] {} { P|Q \equiv Q|P \and P|0 \equiv P \and P|(Q|R) \equiv (P|Q)|R }
\end{mathpar}

\begin{mathpar}
  \inferrule* [lab=alpha-equivalence] {} { (x)P \equiv (y)P\{y/x\} \and y \not\in \freenames{P} }
\end{mathpar}

\begin{definition}
Then two processes, $P,Q$, are alpha-equivalent if $P = Q\{\vec{y}/\vec{x}\}$ for
some $\vec{x} \in \boundnames{Q},\vec{y} \in \boundnames{P}$, where $Q\{\vec{y}/\vec{x}\}$
denotes the capture-avoiding substitution of $\vec{y}$ for $\vec{x}$ in $Q$.
\end{definition}

\begin{definition}
  The {\em structural congruence} \cite{SangiorgiWalker} , $\equiv$,
  between processes is the least congruence containing
  alpha-equivalence, satisfying the abelian monoid laws
  (associativity, commutativity and $\pzero$ as identity) for parallel
  composition $|$ and for summation $+$.
\end{definition}

\subsection{Name equivalence}

We take name equivalence, written $\nameeq$, to be the smallest
equivalence relation generated by the following rules.

\begin{mathpar}
\inferrule*[lab=Quote-drop]
{ }
{ \quotep{@{x}} \nameeq x }

\inferrule*[lab=Struct-equiv]
{ P \scong Q }
{ \quotep{P} \nameeq \quotep{Q} }
\end{mathpar}

The astute reader will have noticed that the mutual recursion of names
and processes imposes a mutual recursion on alpha-equivalence and
structural equivalence via name-equivalence. Fortunately, all of this
works out pleasantly and we may calculate in the natural way, free of
concern. The reader interested in the details is referred to the
appendix \ref{appendix:rho_details}.

\subsection{Substitution}

We use $\Proc$ for the set of processes, $\QProc$ for the set of
names, and $\id{\{}\vec{y} / \vec{x} \id{\}}$ to denote partial maps,
$s : \QProc \rightarrow \QProc$. A map, $s$ lifts, uniquely, to a map
on process terms, $\widehat{s} : \Proc \rightarrow \Proc$ by the
following equations.

\begin{mathpar}
  (0) \psubstp{Q}{P} := 0 \\
  (R \juxtap S) \psubstp{Q}{P}
  :=    
  (R)\psubstp{Q}{P} \juxtap (S) \psubstp{Q}{P} \\
  (x?(y).R) \psubstp{Q}{P}    
  :=    
  (x)\substp{Q}{P} (z)\concat( (R \psubstn{z}{y}) \psubstp{Q}{P} ) \\
  (\lift{x}{R}) \psubstp{Q}{P}  
  :=
  \lift{(x)\substp{Q}{P}}{ R \psubstp{Q}{P} } \\
%   (\dropn{x})  \psubstp{Q}{P}       
%   := 
%   \left\{ 
%     \begin{array}{ccc} 
%       \dropn{\quotep{Q}} & & x \nameeq \quotep{P} \\
%       \dropn{x} & & otherwise \\
%     \end{array}
%   \right. 
  (\dropn{x})  \psubstp{Q}{P}       
  := 
  \left\{ 
    \begin{array}{ccc} 
      Q & & x \nameeq \quotep{P} \\
      \dropn{x} & & otherwise \\
    \end{array}
  \right.
\end{mathpar}
 

where

\begin{eqnarray}
  (x)\id{\{} \lpquote Q \rpquote / \lpquote P \rpquote \id{\}}            = 
  \left\{ 
    \begin{array}{ccc}
      \lpquote Q \rpquote & & x \nameeq \lpquote P \rpquote \\
      x & & otherwise \\
    \end{array}
  \right. \nonumber
\end{eqnarray}

and $z$ is chosen distinct from $\quotep{P}$, $\quotep{Q}$, the free
names in $Q$, and all the names in $R$. Our $\alpha$-equivalence will
be built in the standard way from this substitution.

\begin{remark}\label{rem:no_self_referential_names}
  One consequence of these definitions is that $\forall P. \quotep{P}
  \not\in \freenames{P}$.
\end{remark}

\subsection{ Dynamic quote: an example }

Anticipating something of what's to come, consider applying the
substitution, $\widehat{\id{\{}u / z \id{\}}}$, to the following pair
of processes, $\lift{w}{y!(z)}$ and $w[ \lpquote y!(z) \rpquote ]$.

\begin{eqnarray}
	\lift{w}{y!(z)}\widehat{\id{\{}u / z \id{\}}}
		& = &
		\lift{w}{y!(u)} \nonumber\\
	w[ \lpquote y!(z) \rpquote ] \widehat{ \id{\{}u / z \id{\}} }
		& = &
		w[ \lpquote y!(z) \rpquote ] \nonumber
\end{eqnarray}

Because the body of the process between quotes is impervious to
substitution, we get radically different answers. In fact, by
examining the first process in an input context,
e.g. $x?(z).\lift{w}{y!(z)}$, we see that the process under the lift
operator may be shaped by prefixed inputs binding a name inside it. In
this sense, the lift operator will be seen as a way to dynamically
construct processes before reifying them as names.

Finally equipped with these standard features we can present the
dynamics of the calculus.

\subsubsection{Operational semantics} 

Finally, we introduce the computational dynamics. What marks these
algebras as distinct from other more traditionally studied algebraic
structures, e.g. vector spaces or polynomial rings, is the manner in
which dynamics is captured. In traditional structures, dynamics is typically
expressed through morphisms between such structures, as in linear maps
between vector spaces or morphisms between rings. In algebras
associated with the semantics of computation, the dynamics is
expressed as part of the algebraic structure itself, through a
reduction reduction relation typically denoted by $\red$. Below, we
give a recursive presentation of this relation for the calculus used
in the encoding.

$\red \subseteq \pi \times \pi$
$\red : \pi \to \mathcal{P}(\pi)$

\begin{mathpar}
  \inferrule* [lab=Comm] { \textsf{match}( x_{src}, x_{trgt} ) } { x_{trgt}?(y)P \; | \; x_{src}!\langle {Q} \rangle \red P\{\quotep{Q}/y}\} }
  \and \\
  \inferrule* [lab=Par] {{P} \red {P}'} {{{P} | {Q}} \red {{P}' | {Q}}}
  \and
  \inferrule* [lab=Equiv]{{{P} \scong {P}'} \andalso {{P}' \red {Q}'} \andalso {{Q}' \scong {Q}}}{{P} \red {Q}}
\end{mathpar}

\begin{eqnarray*}
  match_{\equiv} (\quotep{P},\quotep{Q}) & := & P \equiv Q \\
  match_{\dagger}(\quotep{P},\quotep{Q}) & := & \forall R. P|Q \red^{*} R => R \red^{*} 0 \\
  match_{K}(\quotep{P},\quotep{Q}) & := & K \mbox{ for some context } K
\end{eqnarray*}

$u?(x)P | u!\langle Q \rangle \red P\{\quotep{Q}/x\}$

%We write $\wred$ for $\red^*$, and $P\red$ if $\exists Q $ such that $ P \red Q$.
We write $P\red$ if $\exists Q $ such that $ P \red Q$ and $P\not\red$, otherwise.

\section{Replication}

As mentioned before, it is known that replication (and hence
recursion) can be implemented in a higher-order process algebra
\cite{SangiorgiWalker}. As our first example of calculation with the
machinery thus far presented we give the construction explicitly in
the {\rhoc}.

\begin{eqnarray}
	D_{x} & := & \prefix{x}{y}{(\binpar{\outputp{x}{y}}{@{y}})} \nonumber\\
	\bangp_{x}{P} & := & \binpar{{x}!\langle{\binpar{D_{x}}{P}}\rangle}{D_{x}} \nonumber
\end{eqnarray}

\begin{eqnarray}
	\bangp_{x}{P} & & \nonumber\\
	=
	& {x}!\langle{(\prefix{x}{y}{(\outputp{x}{y} | @{y})) | P}}\rangle 
	      | \prefix{x}{y}{(\outputp{x}{y} | @{y})} & \nonumber\\
	\red
	& (\outputp{x}{y} | @{y})\substn{\quotep{(\prefix{x}{y}{(@{y} | \outputp{x}{y})) | P}}}{y} & \nonumber\\
	=
	& \outputp{x}{\quotep{(\prefix{x}{y}{(\outputp{x}{y} | @{y})) | P}}}
	  | {(\prefix{x}{y}{(\outputp{x}{y} | @{y})) | P}} & \nonumber\\
	\red
	& \ldots & \nonumber\\
	\red^*
	& P | P | \ldots & \nonumber
\end{eqnarray}

Of course, this encoding, as an implementation, runs away, unfolding
$\bangp{P}$ eagerly. A lazier and more implementable replication
operator, restricted to input-guarded processes, may be obtained as follows.

\begin{eqnarray}
\bangp{\prefix{u}{v}{P}} 
	:= 
	\binpar{\lift{x}{\prefix{u}{v}{(\binpar{D(x)}{P})}}}{D(x)} \nonumber
\end{eqnarray}

\begin{remark}
  Note that the lazier definition still does not deal with summation
  or mixed summation (i.e. sums over input and output). The reader is
  invited to construct definitions of replication that deal with these
  features. 

  Further, the definitions are parameterized in a name, $x$. Can you,
  gentle reader, make a definition that eliminates this parameter and
  guarantees no accidental interaction between the replication
  machinery and the process being replicated -- i.e. no accidental
  sharing of names used by the process to get its work done and the
  name(s) used by the replication to effect copying. This latter
  revision of the definition of replication is crucial to obtaining
  the expected identity $!!P \sim !P$.
\end{remark}

\begin{remark}\label{rem:paradoxical_combinator}
  The reader familiar with the lambda calculus will have noticed the
  similarity between $D$ and the paradoxical combinator.

  [Ed. note: the existence of this seems to suggest we have to be more
  restrictive on the set of processes and names we admit if we are to
  support no-cloning.]
\end{remark}

\subsubsection{Bisimulation}

The computational dynamics gives rise to another kind of equivalence,
the equivalence of computational behavior. As previously mentioned
this is typically captured \emph{via} some form of bisimulation.

% The notion we use in this paper is weak barbed bisimulation
% \cite{milner91polyadicpi}.

The notion we use in this paper is derived from weak barbed
bisimulation \cite{milner91polyadicpi}. 

\begin{definition}
An \emph{observation relation}, $\downarrow_{\mathcal N}$, over a set
of names, $\mathcal N$, is the smallest relation satisfying the rules
below.

\infrule[Out-barb]{y \in {\mathcal N}, \; x \nameeq y}
		  {\outputp{x}{v} \downarrow_{\mathcal N} x}
\infrule[Par-barb]{\mbox{$P\downarrow_{\mathcal N} x$ or $Q\downarrow_{\mathcal N} x$}}
		  {\binpar{P}{Q} \downarrow_{\mathcal N} x}

We write $P \Downarrow_{\mathcal N} x$ if there is $Q$ such that 
$P \wred Q$ and $Q \downarrow_{\mathcal N} x$.
\end{definition}

\begin{definition}
%\label{def.bbisim}
An  ${\mathcal N}$-\emph{barbed bisimulation} over a set of names, ${\mathcal N}$, is a symmetric binary relation 
${\mathcal S}_{\mathcal N}$ between agents such that $P\rel{S}_{\mathcal N}Q$ implies:
\begin{enumerate}
\item If $P \red P'$ then $Q \wred Q'$ and $P'\rel{S}_{\mathcal N} Q'$.
\item If $P\downarrow_{\mathcal N} x$, then $Q\Downarrow_{\mathcal N} x$.
\end{enumerate}
$P$ is ${\mathcal N}$-barbed bisimilar to $Q$, written
$P \wbbisim_{\mathcal N} Q$, if $P \rel{S}_{\mathcal N} Q$ for some ${\mathcal N}$-barbed bisimulation ${\mathcal S}_{\mathcal N}$.
\end{definition}

$\mathcal{R} \subseteq \pi \times \pi$

$P \mathcal{R} Q => \forall P'. P \red P' \Rightarrow \exists Q'. Q \red Q', P' \mathcal{R} Q'$

$P \vdash x \Rightarrow Q \vdash x$

\begin{mathpar}
  \inferrule*[lab=Out-barb]{x \nameeq y}{{y}!\langle{Q}\rangle \vdash x}
  \and
  \inferrule*[lab=Par-barb]{\mbox{$P\vdash x$ or $Q\vdash x$}}{\binpar{P}{Q} \vdash x}
\end{mathpar}

\subsubsection{Contexts}

One of the principle advantages of computational calculi like the
$\pi$-calculus is a well-defined notion of context,
contextual-equivalence and a correlation between
contextual-equivalence and notions of bisimulation. The notion of
context allows the decomposition of a process into (sub-)process and
its syntactic environment, its context. Thus, a context may be
thought of as a process with a ``hole'' (written $\Box$) in it. The
application of a context $M$ to a process $P$, written $M[P]$, is
tantamount to filling the hole in $M$ with $P$. In this paper we do
not need the full weight of this theory, but do make use of the notion
of context in the proof the main theorem. 

\begin{mathpar}
  \inferrule* [lab=summation] {} {{M_{M},M_{N}} \bc \Box \;|\; x.M_{A} \;|\; M_{M}+M_{N}}
  \and
  \inferrule* [lab=agent] {} {{M_{A}} \bc (\vec{x})M_{P} \;| \; \clift{P_0,\ldots,M_{P},\ldots,P_N}}
  \and \\
  \inferrule* [lab=process] {} {{M_{P}} \bc M_{N} \;| \;P|M_{P} }
\end{mathpar} 

\begin{mathpar}
  \inferrule* [lab=sychronization] {} {M_{N} \bc \Box \;|\; x?M_{F} \;|\; x!M_{C}}
  \and
  \inferrule* [lab=abstraction] {} {{M_{F}} \bc (x)M_{P} }
  \and
  \inferrule* [lab=concretion] {} {{M_{C}} \bc \langle M_{P} \rangle }
  \and \\
  \inferrule* [lab=process] {} {{M_{P}} \bc M_{N} \;| \;P|M_{P} }
\end{mathpar}

\begin{definition}[contextual application] Given a context $M$, and
  process $P$, we define the \emph{contextual application}, $M[P] :=
  M\{P/\Box\}$. That is, the contextual application of M to P is the
  substitution of $P$ for $\Box$ in $M$.
\end{definition}

$\meaningof{-} : L \to \mathcal{P}(\pi)$

\begin{mathpar}
  \inferrule* [lab=collection] {} {\meaningof{true} = \pi, \and \meaningof{~E} = \pi \setminus \meaningof{E}, \and \meaningof{E_{1} \& E_{2}} = \meaningof{E_{1}} \cap \meaningof{E_{2}}}
\end{mathpar}

\begin{mathpar}
  \inferrule* [lab=structure] {} {\meaningof{0} = \{ P \in \pi | P \equiv 0 \}, \and \\ \meaningof{E_1 | E_2} = \{ P \in \pi | P \equiv P_{1} | P_{2}, P_{1} \in \meaningof{E_{1}}, P_{2} \in \meaningof{E_2}\} }
\end{mathpar}

\begin{mathpar}
 \inferrule* [lab=behavior] {} {\meaningof{\langle a?b \rangle E} = \{ P \in \pi | P \equiv Q | u?(y)P', \\ \and \\\\ \and \\ \;\;\; u \in \meaningof{a}, \forall z.P'\{z/y\} \in \meaningof{E\{z/b\}}\}, \and \\ \meaningof{a!E} = \{ P \in \pi | P \equiv Q | x!\langle P' \rangle, x \in \meaningof{a} P' \in \meaningof{E}\} }
\end{mathpar}

\begin{mathpar}
 \inferrule* [lab=nominal] {} {\meaningof{\quotep{E}} = \{ \quotep{P} \in \quotep{\pi} | P \in \meaningof{E} \}, \and \meaningof{\quotep{P}} = \{ \quotep{Q} \in \quotep{\pi} | P \equiv Q \} \and \\ \meaningof{@\quotep{E}} = \{ P \in \pi | P \equiv @x, x \in \meaningof{E} \}}
\end{mathpar}

\begin{eqnarray*}
  \\
  \meaningof{-} : TS \to ST
\end{eqnarray*}

\begin{eqnarray*}
  \\
  L : TS \to ST
\end{eqnarray*}

\begin{eqnarray*}
  \\
  P \models E \iff P \in \meaningof{E}
\end{eqnarray*}

\begin{eqnarray*}
  P \approx_{L} Q \iff \forall E \in L. P \models E \iff Q \models E
\end{eqnarray*}

\begin{eqnarray*}
  P \approx_{K} Q
\end{eqnarray*}

\begin{eqnarray*}
  P \approx Q
\end{eqnarray*}

$\approx_{K} = \approx = \approx_{L}$

\subsubsection{Contextual duality}

Note that contexts extend the quotation operation to a family of
operations from processes to names. Given a context, $M$, we can
define a \emph{nominal context}, $\quotep{M}$ by $\quotep{M}[P] :=
\quotep{M[P]}$. To foreshadow what is to come we observe that these
operations enjoy a duality with processes very much like the duality
between vectors and maps from vectors to scalars.

Further, because the calculus is essentially higher-order, we have a
correspondence between contexts and processes. More specifically,
given a name $x$ and a context $M$ we can construct $M^{*}_{x}$ such
that 

\begin{mathpar}
  M^{*}_{x} | \lift{x}{P} \red M[P]
\end{mathpar}

namely,

\begin{mathpar}
  M^{*}_{x} := x?(u).M[\dropn{u}]
\end{mathpar}

The dependence of $M^{*}_{x}$ on a name makes it an abstraction, 

\begin{mathpar}
  M^{*} := (x)x?(u).M[\dropn{u}]
\end{mathpar}

\subsection{Additional notation}

It will sometimes be convenient to denote the process a name
quotes. We already have the notation $x = \quotep{P}$, but it will be
convenient to introduce an alternate notation, $\procn{x}$, when we
want to emphasize the connection to the use of the name. Note that, by
virtue of name equivalence, $\quotep{\procn{x}} \nameeq x$; so, the
notation is consistent with previous definitions.

Further, because names have structure it is possible to effect
substitutions on the basis of that structure. This means we need to
upgrade our notation for substitutions, which we accomplish by
adapting comprehension notation. Thus,

\begin{mathpar}
  P\{ y / x : x \in S \}
\end{mathpar}

is interpreted to mean the process derived from P by replacing (in a
capture-avoiding manner) each occurrence of $x$ in $S$ by $y$. For example,

\begin{mathpar}
  P\{ \quotep{\procn{x}|\procn{x}} / x : x \in \freenames{P} \}
\end{mathpar}

will replace each (occurrence) of a free name $x$ in $P$ by
$\quotep{\procn{x}|\procn{x}}$.

Also, we will avail ourselves of the notation $x^{L}$ and $x^{R}$ to
denote injections of a name into disjoint copies of the name
space. There are numerous ways to accomplish this. One example can be
found in \cite{MeredithR05}. This notation overloads to vectors of
names: $\vec{x}^{\pi} := (x_{i}^{\pi} \; : \; 0 \leq i < |\vec{x}| )$ where $\pi \in \{L,R\}$.

We also use $P^{\Box} := P|\Box$.

In \cite{MeredithR05} an interpretation of the new operator is
given. It turns out that there are several possible interpretations
all enjoying the requisite algebraic properties of the operator (see
\cite{milner91polyadicpi}). We will therefore make liberal use of
$(\nu\; \vec{x})P$.

% subsection the_syntax_and_semantics_of_the_notation_system (end)   

\input{qm2pi.qmops} 

\input{qm2pi.sterngerlach} 

\input{qm2pi.metric} 

% section concurrent_process_calculi (end)

%\input{qm2pi.proofsketch}

% section proof sketch (end)

%\input{qm2pi.slviaknots} 

% section spatial logic via knots (end)

\input{qm2pi.conclusion}

% section conclusion (end)

%\input{qm2pi.dtcodes} 

% section wiring algorithm (end)

\input{qm2pi.ack} 

% section acknowledgments (end)

\newpage


\bibliographystyle{plain}   
\bibliography{../../biblios/main.bib}

\input{qm2pi.rhodetails}

\end{document}

 

% section wiring algorithm (end)

\documentclass[12pt]{llncs}
%\documentclass{jktr}

\usepackage[pdftex]{hyperref}                   
\usepackage {listings}
\usepackage {mathpartir}
\usepackage{bcprules}
%\usepackage{listings}
                       
\usepackage{graphicx} 
%\usepackage[margins=2.5cm,nohead,nofoot]{geometry}
%\usepackage{geometry}
\usepackage{amsfonts}
\usepackage{amstext}
\usepackage{latexsym}
\usepackage{amssymb}
\usepackage{color}


%\include{myPreamble}
\include{qm2pi.local} 

%\ifpdf
%\usepackage[pdftex]{graphicx}
%\else
%\usepackage{graphicx}
%\fi

 % \ifpdf
%  \usepackage{pdfsync}
%  \if


%\title{Brief Article}
%\author{David F. Snyder}
%\author{L.G. Meredith}

%\address{Dept. of Math., Texas State University--San Marcos, San Marcos, TX 78666}
       
\pagestyle{empty}


\begin{document}

\lstset{language=[Objective]Caml,frame=shadowbox}

\input{qm2pi.front}

% section front matter (end)

\input{qm2pi.intro} 
 
% section introduction (end)

% \input{qm2pi.knotations} 

% section notation (end)

\input{qm2pi.process.calculi} 

% section concurrent_process_calculi_and_spatial_logics_ (end)
    
%\input{qm2pi.knots2pi} 

%\input{qm2pi.trefoil} 

%\input{qm2pi.mainthm} 

% subsection basic_interpretation (end)

%\input{qm2pi.rho.presentation} 
\subsection{The syntax and semantics of the notation system}\label{sub:the_syntax_and_semantics_of_the_notation_system} % (fold)

We now summarize a technical presentation of the calculus that
embodies our theory of dynamics. The typical presentation of such a
calculus follows the style of giving generators and relations on
them. The grammar, below, describing term constructors, freely
generates the set of processes, $\Proc$. This set is then quotiented
by a relation known as structural congruence and it is over this set
that the notion of dynamics is expressed. This presentation is
essentially that of \cite{MeredithR05} with the addition of
polyadicity and summation. For readability we have relegated some of
the technical subtleties to an appendix.

\subsubsection{Process grammar}\label{subsub:process_grammar}

\begin{mathpar}
  \inferrule* [lab=synchronization] {} {{M} \bc \pzero \;|\; x?F \;|\; x!C }
  \and
  \inferrule* [lab=abstraction] {} {{F} \bc (x)P}
  \and
  \inferrule* [lab=concretion] {} {{C} \bc \langle Q \rangle}
  \and
  \inferrule* [lab=process] {} {{P,Q} \bc M \;| \;P|Q \;|\; @{x}}
  \and
  \inferrule* [lab=name] {} {{x} \bc \quotep{P}}
\end{mathpar} 

Note that $\vec{x}$ (resp. $\vec{P}$) denotes a vector of names
(resp. processes) of length $|\vec{x}|$ (resp. $|\vec{P}|$). We adopt
the following useful abbreviations.

\begin{mathpar}
   x?(\vec{y}).P := x.(\vec{y})P \and  x\clift{\vec{P}} := x.\clift{\vec{P}}
   \and x!(y) := \lift{x}{\dropn{y}}
   \and \Pi_{i=0}^{n-1}P_i := P_0 | \ldots | P_{n-1}
\end{mathpar}

\subsubsection{Structural congruence}

\paragraph{Free and bound names and alpha-equivalence.} At the
core of structural equivalence is alpha-equivalence which identifies
process that are the same up to a change of variable. Formally, we
recognize the distinction between free and bound names. The free names
of a process, $\freenames{P}$, may be calculated recursively as
follows:

\begin{mathpar}
\freenames{\pzero} := \emptyset
  \and \\
  \freenames{x?(y).P} := \{ x \} \cup (\freenames{P} \setminus \{ y \})
  \and 
  \freenames{x!\langle P \rangle} := \{ x \} \cup \{ P \} 
  \and \\
  \freenames{P|Q} := \freenames{P} \cup \freenames{Q}
  \and \\
  \freenames{@{x}} := \{ x \}
\end{mathpar}

$\pi$
$\quotep{\pi}$

$\freenames{-} : \pi \to \mathcal{P}(\quotep{\pi})$

\begin{eqnarray*}
  \freenames{\pzero} & := & \emptyset \\
  \freenames{x?(y).P} & := & \{ x \} \cup (\freenames{P} \setminus \{ y \}) \\
  \freenames{x!\langle P \rangle} & := & \{ x \} \cup \{ P \} \\
  \freenames{P|Q} & := & \freenames{P} \cup \freenames{Q} \\
  \freenames{\dropn{x}} & := & \{ x \}
\end{eqnarray*}

The bound names of a process, $\boundnames{P}$, are those names occurring in $P$
that are not free. For example, in $x?(y).0$, the name $x$ is free, while $y$ is bound.

\begin{mathpar}
  \inferrule* [lab=monoidal-laws] {} { P|Q \equiv Q|P \and P|0 \equiv P \and P|(Q|R) \equiv (P|Q)|R }
\end{mathpar}

\begin{mathpar}
  \inferrule* [lab=alpha-equivalence] {} { (x)P \equiv (y)P\{y/x\} \and y \not\in \freenames{P} }
\end{mathpar}

\begin{definition}
Then two processes, $P,Q$, are alpha-equivalent if $P = Q\{\vec{y}/\vec{x}\}$ for
some $\vec{x} \in \boundnames{Q},\vec{y} \in \boundnames{P}$, where $Q\{\vec{y}/\vec{x}\}$
denotes the capture-avoiding substitution of $\vec{y}$ for $\vec{x}$ in $Q$.
\end{definition}

\begin{definition}
  The {\em structural congruence} \cite{SangiorgiWalker} , $\equiv$,
  between processes is the least congruence containing
  alpha-equivalence, satisfying the abelian monoid laws
  (associativity, commutativity and $\pzero$ as identity) for parallel
  composition $|$ and for summation $+$.
\end{definition}

\subsection{Name equivalence}

We take name equivalence, written $\nameeq$, to be the smallest
equivalence relation generated by the following rules.

\begin{mathpar}
\inferrule*[lab=Quote-drop]
{ }
{ \quotep{@{x}} \nameeq x }

\inferrule*[lab=Struct-equiv]
{ P \scong Q }
{ \quotep{P} \nameeq \quotep{Q} }
\end{mathpar}

The astute reader will have noticed that the mutual recursion of names
and processes imposes a mutual recursion on alpha-equivalence and
structural equivalence via name-equivalence. Fortunately, all of this
works out pleasantly and we may calculate in the natural way, free of
concern. The reader interested in the details is referred to the
appendix \ref{appendix:rho_details}.

\subsection{Substitution}

We use $\Proc$ for the set of processes, $\QProc$ for the set of
names, and $\id{\{}\vec{y} / \vec{x} \id{\}}$ to denote partial maps,
$s : \QProc \rightarrow \QProc$. A map, $s$ lifts, uniquely, to a map
on process terms, $\widehat{s} : \Proc \rightarrow \Proc$ by the
following equations.

\begin{mathpar}
  (0) \psubstp{Q}{P} := 0 \\
  (R \juxtap S) \psubstp{Q}{P}
  :=    
  (R)\psubstp{Q}{P} \juxtap (S) \psubstp{Q}{P} \\
  (x?(y).R) \psubstp{Q}{P}    
  :=    
  (x)\substp{Q}{P} (z)\concat( (R \psubstn{z}{y}) \psubstp{Q}{P} ) \\
  (\lift{x}{R}) \psubstp{Q}{P}  
  :=
  \lift{(x)\substp{Q}{P}}{ R \psubstp{Q}{P} } \\
%   (\dropn{x})  \psubstp{Q}{P}       
%   := 
%   \left\{ 
%     \begin{array}{ccc} 
%       \dropn{\quotep{Q}} & & x \nameeq \quotep{P} \\
%       \dropn{x} & & otherwise \\
%     \end{array}
%   \right. 
  (\dropn{x})  \psubstp{Q}{P}       
  := 
  \left\{ 
    \begin{array}{ccc} 
      Q & & x \nameeq \quotep{P} \\
      \dropn{x} & & otherwise \\
    \end{array}
  \right.
\end{mathpar}
 

where

\begin{eqnarray}
  (x)\id{\{} \lpquote Q \rpquote / \lpquote P \rpquote \id{\}}            = 
  \left\{ 
    \begin{array}{ccc}
      \lpquote Q \rpquote & & x \nameeq \lpquote P \rpquote \\
      x & & otherwise \\
    \end{array}
  \right. \nonumber
\end{eqnarray}

and $z$ is chosen distinct from $\quotep{P}$, $\quotep{Q}$, the free
names in $Q$, and all the names in $R$. Our $\alpha$-equivalence will
be built in the standard way from this substitution.

\begin{remark}\label{rem:no_self_referential_names}
  One consequence of these definitions is that $\forall P. \quotep{P}
  \not\in \freenames{P}$.
\end{remark}

\subsection{ Dynamic quote: an example }

Anticipating something of what's to come, consider applying the
substitution, $\widehat{\id{\{}u / z \id{\}}}$, to the following pair
of processes, $\lift{w}{y!(z)}$ and $w[ \lpquote y!(z) \rpquote ]$.

\begin{eqnarray}
	\lift{w}{y!(z)}\widehat{\id{\{}u / z \id{\}}}
		& = &
		\lift{w}{y!(u)} \nonumber\\
	w[ \lpquote y!(z) \rpquote ] \widehat{ \id{\{}u / z \id{\}} }
		& = &
		w[ \lpquote y!(z) \rpquote ] \nonumber
\end{eqnarray}

Because the body of the process between quotes is impervious to
substitution, we get radically different answers. In fact, by
examining the first process in an input context,
e.g. $x?(z).\lift{w}{y!(z)}$, we see that the process under the lift
operator may be shaped by prefixed inputs binding a name inside it. In
this sense, the lift operator will be seen as a way to dynamically
construct processes before reifying them as names.

Finally equipped with these standard features we can present the
dynamics of the calculus.

\subsubsection{Operational semantics} 

Finally, we introduce the computational dynamics. What marks these
algebras as distinct from other more traditionally studied algebraic
structures, e.g. vector spaces or polynomial rings, is the manner in
which dynamics is captured. In traditional structures, dynamics is typically
expressed through morphisms between such structures, as in linear maps
between vector spaces or morphisms between rings. In algebras
associated with the semantics of computation, the dynamics is
expressed as part of the algebraic structure itself, through a
reduction reduction relation typically denoted by $\red$. Below, we
give a recursive presentation of this relation for the calculus used
in the encoding.

$\red \subseteq \pi \times \pi$
$\red : \pi \to \mathcal{P}(\pi)$

\begin{mathpar}
  \inferrule* [lab=Comm] { \textsf{match}( x_{src}, x_{trgt} ) } { x_{trgt}?(y)P \; | \; x_{src}!\langle {Q} \rangle \red P\{\quotep{Q}/y}\} }
  \and \\
  \inferrule* [lab=Par] {{P} \red {P}'} {{{P} | {Q}} \red {{P}' | {Q}}}
  \and
  \inferrule* [lab=Equiv]{{{P} \scong {P}'} \andalso {{P}' \red {Q}'} \andalso {{Q}' \scong {Q}}}{{P} \red {Q}}
\end{mathpar}

\begin{eqnarray*}
  match_{\equiv} (\quotep{P},\quotep{Q}) & := & P \equiv Q \\
  match_{\dagger}(\quotep{P},\quotep{Q}) & := & \forall R. P|Q \red^{*} R => R \red^{*} 0 \\
  match_{K}(\quotep{P},\quotep{Q}) & := & K \mbox{ for some context } K
\end{eqnarray*}

$u?(x)P | u!\langle Q \rangle \red P\{\quotep{Q}/x\}$

%We write $\wred$ for $\red^*$, and $P\red$ if $\exists Q $ such that $ P \red Q$.
We write $P\red$ if $\exists Q $ such that $ P \red Q$ and $P\not\red$, otherwise.

\section{Replication}

As mentioned before, it is known that replication (and hence
recursion) can be implemented in a higher-order process algebra
\cite{SangiorgiWalker}. As our first example of calculation with the
machinery thus far presented we give the construction explicitly in
the {\rhoc}.

\begin{eqnarray}
	D_{x} & := & \prefix{x}{y}{(\binpar{\outputp{x}{y}}{@{y}})} \nonumber\\
	\bangp_{x}{P} & := & \binpar{{x}!\langle{\binpar{D_{x}}{P}}\rangle}{D_{x}} \nonumber
\end{eqnarray}

\begin{eqnarray}
	\bangp_{x}{P} & & \nonumber\\
	=
	& {x}!\langle{(\prefix{x}{y}{(\outputp{x}{y} | @{y})) | P}}\rangle 
	      | \prefix{x}{y}{(\outputp{x}{y} | @{y})} & \nonumber\\
	\red
	& (\outputp{x}{y} | @{y})\substn{\quotep{(\prefix{x}{y}{(@{y} | \outputp{x}{y})) | P}}}{y} & \nonumber\\
	=
	& \outputp{x}{\quotep{(\prefix{x}{y}{(\outputp{x}{y} | @{y})) | P}}}
	  | {(\prefix{x}{y}{(\outputp{x}{y} | @{y})) | P}} & \nonumber\\
	\red
	& \ldots & \nonumber\\
	\red^*
	& P | P | \ldots & \nonumber
\end{eqnarray}

Of course, this encoding, as an implementation, runs away, unfolding
$\bangp{P}$ eagerly. A lazier and more implementable replication
operator, restricted to input-guarded processes, may be obtained as follows.

\begin{eqnarray}
\bangp{\prefix{u}{v}{P}} 
	:= 
	\binpar{\lift{x}{\prefix{u}{v}{(\binpar{D(x)}{P})}}}{D(x)} \nonumber
\end{eqnarray}

\begin{remark}
  Note that the lazier definition still does not deal with summation
  or mixed summation (i.e. sums over input and output). The reader is
  invited to construct definitions of replication that deal with these
  features. 

  Further, the definitions are parameterized in a name, $x$. Can you,
  gentle reader, make a definition that eliminates this parameter and
  guarantees no accidental interaction between the replication
  machinery and the process being replicated -- i.e. no accidental
  sharing of names used by the process to get its work done and the
  name(s) used by the replication to effect copying. This latter
  revision of the definition of replication is crucial to obtaining
  the expected identity $!!P \sim !P$.
\end{remark}

\begin{remark}\label{rem:paradoxical_combinator}
  The reader familiar with the lambda calculus will have noticed the
  similarity between $D$ and the paradoxical combinator.

  [Ed. note: the existence of this seems to suggest we have to be more
  restrictive on the set of processes and names we admit if we are to
  support no-cloning.]
\end{remark}

\subsubsection{Bisimulation}

The computational dynamics gives rise to another kind of equivalence,
the equivalence of computational behavior. As previously mentioned
this is typically captured \emph{via} some form of bisimulation.

% The notion we use in this paper is weak barbed bisimulation
% \cite{milner91polyadicpi}.

The notion we use in this paper is derived from weak barbed
bisimulation \cite{milner91polyadicpi}. 

\begin{definition}
An \emph{observation relation}, $\downarrow_{\mathcal N}$, over a set
of names, $\mathcal N$, is the smallest relation satisfying the rules
below.

\infrule[Out-barb]{y \in {\mathcal N}, \; x \nameeq y}
		  {\outputp{x}{v} \downarrow_{\mathcal N} x}
\infrule[Par-barb]{\mbox{$P\downarrow_{\mathcal N} x$ or $Q\downarrow_{\mathcal N} x$}}
		  {\binpar{P}{Q} \downarrow_{\mathcal N} x}

We write $P \Downarrow_{\mathcal N} x$ if there is $Q$ such that 
$P \wred Q$ and $Q \downarrow_{\mathcal N} x$.
\end{definition}

\begin{definition}
%\label{def.bbisim}
An  ${\mathcal N}$-\emph{barbed bisimulation} over a set of names, ${\mathcal N}$, is a symmetric binary relation 
${\mathcal S}_{\mathcal N}$ between agents such that $P\rel{S}_{\mathcal N}Q$ implies:
\begin{enumerate}
\item If $P \red P'$ then $Q \wred Q'$ and $P'\rel{S}_{\mathcal N} Q'$.
\item If $P\downarrow_{\mathcal N} x$, then $Q\Downarrow_{\mathcal N} x$.
\end{enumerate}
$P$ is ${\mathcal N}$-barbed bisimilar to $Q$, written
$P \wbbisim_{\mathcal N} Q$, if $P \rel{S}_{\mathcal N} Q$ for some ${\mathcal N}$-barbed bisimulation ${\mathcal S}_{\mathcal N}$.
\end{definition}

$\mathcal{R} \subseteq \pi \times \pi$

$P \mathcal{R} Q => \forall P'. P \red P' \Rightarrow \exists Q'. Q \red Q', P' \mathcal{R} Q'$

$P \vdash x \Rightarrow Q \vdash x$

\begin{mathpar}
  \inferrule*[lab=Out-barb]{x \nameeq y}{{y}!\langle{Q}\rangle \vdash x}
  \and
  \inferrule*[lab=Par-barb]{\mbox{$P\vdash x$ or $Q\vdash x$}}{\binpar{P}{Q} \vdash x}
\end{mathpar}

\subsubsection{Contexts}

One of the principle advantages of computational calculi like the
$\pi$-calculus is a well-defined notion of context,
contextual-equivalence and a correlation between
contextual-equivalence and notions of bisimulation. The notion of
context allows the decomposition of a process into (sub-)process and
its syntactic environment, its context. Thus, a context may be
thought of as a process with a ``hole'' (written $\Box$) in it. The
application of a context $M$ to a process $P$, written $M[P]$, is
tantamount to filling the hole in $M$ with $P$. In this paper we do
not need the full weight of this theory, but do make use of the notion
of context in the proof the main theorem. 

\begin{mathpar}
  \inferrule* [lab=summation] {} {{M_{M},M_{N}} \bc \Box \;|\; x.M_{A} \;|\; M_{M}+M_{N}}
  \and
  \inferrule* [lab=agent] {} {{M_{A}} \bc (\vec{x})M_{P} \;| \; \clift{P_0,\ldots,M_{P},\ldots,P_N}}
  \and \\
  \inferrule* [lab=process] {} {{M_{P}} \bc M_{N} \;| \;P|M_{P} }
\end{mathpar} 

\begin{mathpar}
  \inferrule* [lab=sychronization] {} {M_{N} \bc \Box \;|\; x?M_{F} \;|\; x!M_{C}}
  \and
  \inferrule* [lab=abstraction] {} {{M_{F}} \bc (x)M_{P} }
  \and
  \inferrule* [lab=concretion] {} {{M_{C}} \bc \langle M_{P} \rangle }
  \and \\
  \inferrule* [lab=process] {} {{M_{P}} \bc M_{N} \;| \;P|M_{P} }
\end{mathpar}

\begin{definition}[contextual application] Given a context $M$, and
  process $P$, we define the \emph{contextual application}, $M[P] :=
  M\{P/\Box\}$. That is, the contextual application of M to P is the
  substitution of $P$ for $\Box$ in $M$.
\end{definition}

$\meaningof{-} : L \to \mathcal{P}(\pi)$

\begin{mathpar}
  \inferrule* [lab=collection] {} {\meaningof{true} = \pi, \and \meaningof{~E} = \pi \setminus \meaningof{E}, \and \meaningof{E_{1} \& E_{2}} = \meaningof{E_{1}} \cap \meaningof{E_{2}}}
\end{mathpar}

\begin{mathpar}
  \inferrule* [lab=structure] {} {\meaningof{0} = \{ P \in \pi | P \equiv 0 \}, \and \\ \meaningof{E_1 | E_2} = \{ P \in \pi | P \equiv P_{1} | P_{2}, P_{1} \in \meaningof{E_{1}}, P_{2} \in \meaningof{E_2}\} }
\end{mathpar}

\begin{mathpar}
 \inferrule* [lab=behavior] {} {\meaningof{\langle a?b \rangle E} = \{ P \in \pi | P \equiv Q | u?(y)P', \\ \and \\\\ \and \\ \;\;\; u \in \meaningof{a}, \forall z.P'\{z/y\} \in \meaningof{E\{z/b\}}\}, \and \\ \meaningof{a!E} = \{ P \in \pi | P \equiv Q | x!\langle P' \rangle, x \in \meaningof{a} P' \in \meaningof{E}\} }
\end{mathpar}

\begin{mathpar}
 \inferrule* [lab=nominal] {} {\meaningof{\quotep{E}} = \{ \quotep{P} \in \quotep{\pi} | P \in \meaningof{E} \}, \and \meaningof{\quotep{P}} = \{ \quotep{Q} \in \quotep{\pi} | P \equiv Q \} \and \\ \meaningof{@\quotep{E}} = \{ P \in \pi | P \equiv @x, x \in \meaningof{E} \}}
\end{mathpar}

\begin{eqnarray*}
  \\
  \meaningof{-} : TS \to ST
\end{eqnarray*}

\begin{eqnarray*}
  \\
  L : TS \to ST
\end{eqnarray*}

\begin{eqnarray*}
  \\
  P \models E \iff P \in \meaningof{E}
\end{eqnarray*}

\begin{eqnarray*}
  P \approx_{L} Q \iff \forall E \in L. P \models E \iff Q \models E
\end{eqnarray*}

\begin{eqnarray*}
  P \approx_{K} Q
\end{eqnarray*}

\begin{eqnarray*}
  P \approx Q
\end{eqnarray*}

$\approx_{K} = \approx = \approx_{L}$

\subsubsection{Contextual duality}

Note that contexts extend the quotation operation to a family of
operations from processes to names. Given a context, $M$, we can
define a \emph{nominal context}, $\quotep{M}$ by $\quotep{M}[P] :=
\quotep{M[P]}$. To foreshadow what is to come we observe that these
operations enjoy a duality with processes very much like the duality
between vectors and maps from vectors to scalars.

Further, because the calculus is essentially higher-order, we have a
correspondence between contexts and processes. More specifically,
given a name $x$ and a context $M$ we can construct $M^{*}_{x}$ such
that 

\begin{mathpar}
  M^{*}_{x} | \lift{x}{P} \red M[P]
\end{mathpar}

namely,

\begin{mathpar}
  M^{*}_{x} := x?(u).M[\dropn{u}]
\end{mathpar}

The dependence of $M^{*}_{x}$ on a name makes it an abstraction, 

\begin{mathpar}
  M^{*} := (x)x?(u).M[\dropn{u}]
\end{mathpar}

\subsection{Additional notation}

It will sometimes be convenient to denote the process a name
quotes. We already have the notation $x = \quotep{P}$, but it will be
convenient to introduce an alternate notation, $\procn{x}$, when we
want to emphasize the connection to the use of the name. Note that, by
virtue of name equivalence, $\quotep{\procn{x}} \nameeq x$; so, the
notation is consistent with previous definitions.

Further, because names have structure it is possible to effect
substitutions on the basis of that structure. This means we need to
upgrade our notation for substitutions, which we accomplish by
adapting comprehension notation. Thus,

\begin{mathpar}
  P\{ y / x : x \in S \}
\end{mathpar}

is interpreted to mean the process derived from P by replacing (in a
capture-avoiding manner) each occurrence of $x$ in $S$ by $y$. For example,

\begin{mathpar}
  P\{ \quotep{\procn{x}|\procn{x}} / x : x \in \freenames{P} \}
\end{mathpar}

will replace each (occurrence) of a free name $x$ in $P$ by
$\quotep{\procn{x}|\procn{x}}$.

Also, we will avail ourselves of the notation $x^{L}$ and $x^{R}$ to
denote injections of a name into disjoint copies of the name
space. There are numerous ways to accomplish this. One example can be
found in \cite{MeredithR05}. This notation overloads to vectors of
names: $\vec{x}^{\pi} := (x_{i}^{\pi} \; : \; 0 \leq i < |\vec{x}| )$ where $\pi \in \{L,R\}$.

We also use $P^{\Box} := P|\Box$.

In \cite{MeredithR05} an interpretation of the new operator is
given. It turns out that there are several possible interpretations
all enjoying the requisite algebraic properties of the operator (see
\cite{milner91polyadicpi}). We will therefore make liberal use of
$(\nu\; \vec{x})P$.

% subsection the_syntax_and_semantics_of_the_notation_system (end)   

\input{qm2pi.qmops} 

\input{qm2pi.sterngerlach} 

\input{qm2pi.metric} 

% section concurrent_process_calculi (end)

%\input{qm2pi.proofsketch}

% section proof sketch (end)

%\input{qm2pi.slviaknots} 

% section spatial logic via knots (end)

\input{qm2pi.conclusion}

% section conclusion (end)

%\input{qm2pi.dtcodes} 

% section wiring algorithm (end)

\input{qm2pi.ack} 

% section acknowledgments (end)

\newpage


\bibliographystyle{plain}   
\bibliography{../../biblios/main.bib}

\input{qm2pi.rhodetails}

\end{document}

 

% section acknowledgments (end)

\newpage


\bibliographystyle{plain}   
\bibliography{../../biblios/main.bib}

\documentclass[12pt]{llncs}
%\documentclass{jktr}

\usepackage[pdftex]{hyperref}                   
\usepackage {listings}
\usepackage {mathpartir}
\usepackage{bcprules}
%\usepackage{listings}
                       
\usepackage{graphicx} 
%\usepackage[margins=2.5cm,nohead,nofoot]{geometry}
%\usepackage{geometry}
\usepackage{amsfonts}
\usepackage{amstext}
\usepackage{latexsym}
\usepackage{amssymb}
\usepackage{color}


%\include{myPreamble}
\include{qm2pi.local} 

%\ifpdf
%\usepackage[pdftex]{graphicx}
%\else
%\usepackage{graphicx}
%\fi

 % \ifpdf
%  \usepackage{pdfsync}
%  \if


%\title{Brief Article}
%\author{David F. Snyder}
%\author{L.G. Meredith}

%\address{Dept. of Math., Texas State University--San Marcos, San Marcos, TX 78666}
       
\pagestyle{empty}


\begin{document}

\lstset{language=[Objective]Caml,frame=shadowbox}

\input{qm2pi.front}

% section front matter (end)

\input{qm2pi.intro} 
 
% section introduction (end)

% \input{qm2pi.knotations} 

% section notation (end)

\input{qm2pi.process.calculi} 

% section concurrent_process_calculi_and_spatial_logics_ (end)
    
%\input{qm2pi.knots2pi} 

%\input{qm2pi.trefoil} 

%\input{qm2pi.mainthm} 

% subsection basic_interpretation (end)

%\input{qm2pi.rho.presentation} 
\subsection{The syntax and semantics of the notation system}\label{sub:the_syntax_and_semantics_of_the_notation_system} % (fold)

We now summarize a technical presentation of the calculus that
embodies our theory of dynamics. The typical presentation of such a
calculus follows the style of giving generators and relations on
them. The grammar, below, describing term constructors, freely
generates the set of processes, $\Proc$. This set is then quotiented
by a relation known as structural congruence and it is over this set
that the notion of dynamics is expressed. This presentation is
essentially that of \cite{MeredithR05} with the addition of
polyadicity and summation. For readability we have relegated some of
the technical subtleties to an appendix.

\subsubsection{Process grammar}\label{subsub:process_grammar}

\begin{mathpar}
  \inferrule* [lab=synchronization] {} {{M} \bc \pzero \;|\; x?F \;|\; x!C }
  \and
  \inferrule* [lab=abstraction] {} {{F} \bc (x)P}
  \and
  \inferrule* [lab=concretion] {} {{C} \bc \langle Q \rangle}
  \and
  \inferrule* [lab=process] {} {{P,Q} \bc M \;| \;P|Q \;|\; @{x}}
  \and
  \inferrule* [lab=name] {} {{x} \bc \quotep{P}}
\end{mathpar} 

Note that $\vec{x}$ (resp. $\vec{P}$) denotes a vector of names
(resp. processes) of length $|\vec{x}|$ (resp. $|\vec{P}|$). We adopt
the following useful abbreviations.

\begin{mathpar}
   x?(\vec{y}).P := x.(\vec{y})P \and  x\clift{\vec{P}} := x.\clift{\vec{P}}
   \and x!(y) := \lift{x}{\dropn{y}}
   \and \Pi_{i=0}^{n-1}P_i := P_0 | \ldots | P_{n-1}
\end{mathpar}

\subsubsection{Structural congruence}

\paragraph{Free and bound names and alpha-equivalence.} At the
core of structural equivalence is alpha-equivalence which identifies
process that are the same up to a change of variable. Formally, we
recognize the distinction between free and bound names. The free names
of a process, $\freenames{P}$, may be calculated recursively as
follows:

\begin{mathpar}
\freenames{\pzero} := \emptyset
  \and \\
  \freenames{x?(y).P} := \{ x \} \cup (\freenames{P} \setminus \{ y \})
  \and 
  \freenames{x!\langle P \rangle} := \{ x \} \cup \{ P \} 
  \and \\
  \freenames{P|Q} := \freenames{P} \cup \freenames{Q}
  \and \\
  \freenames{@{x}} := \{ x \}
\end{mathpar}

$\pi$
$\quotep{\pi}$

$\freenames{-} : \pi \to \mathcal{P}(\quotep{\pi})$

\begin{eqnarray*}
  \freenames{\pzero} & := & \emptyset \\
  \freenames{x?(y).P} & := & \{ x \} \cup (\freenames{P} \setminus \{ y \}) \\
  \freenames{x!\langle P \rangle} & := & \{ x \} \cup \{ P \} \\
  \freenames{P|Q} & := & \freenames{P} \cup \freenames{Q} \\
  \freenames{\dropn{x}} & := & \{ x \}
\end{eqnarray*}

The bound names of a process, $\boundnames{P}$, are those names occurring in $P$
that are not free. For example, in $x?(y).0$, the name $x$ is free, while $y$ is bound.

\begin{mathpar}
  \inferrule* [lab=monoidal-laws] {} { P|Q \equiv Q|P \and P|0 \equiv P \and P|(Q|R) \equiv (P|Q)|R }
\end{mathpar}

\begin{mathpar}
  \inferrule* [lab=alpha-equivalence] {} { (x)P \equiv (y)P\{y/x\} \and y \not\in \freenames{P} }
\end{mathpar}

\begin{definition}
Then two processes, $P,Q$, are alpha-equivalent if $P = Q\{\vec{y}/\vec{x}\}$ for
some $\vec{x} \in \boundnames{Q},\vec{y} \in \boundnames{P}$, where $Q\{\vec{y}/\vec{x}\}$
denotes the capture-avoiding substitution of $\vec{y}$ for $\vec{x}$ in $Q$.
\end{definition}

\begin{definition}
  The {\em structural congruence} \cite{SangiorgiWalker} , $\equiv$,
  between processes is the least congruence containing
  alpha-equivalence, satisfying the abelian monoid laws
  (associativity, commutativity and $\pzero$ as identity) for parallel
  composition $|$ and for summation $+$.
\end{definition}

\subsection{Name equivalence}

We take name equivalence, written $\nameeq$, to be the smallest
equivalence relation generated by the following rules.

\begin{mathpar}
\inferrule*[lab=Quote-drop]
{ }
{ \quotep{@{x}} \nameeq x }

\inferrule*[lab=Struct-equiv]
{ P \scong Q }
{ \quotep{P} \nameeq \quotep{Q} }
\end{mathpar}

The astute reader will have noticed that the mutual recursion of names
and processes imposes a mutual recursion on alpha-equivalence and
structural equivalence via name-equivalence. Fortunately, all of this
works out pleasantly and we may calculate in the natural way, free of
concern. The reader interested in the details is referred to the
appendix \ref{appendix:rho_details}.

\subsection{Substitution}

We use $\Proc$ for the set of processes, $\QProc$ for the set of
names, and $\id{\{}\vec{y} / \vec{x} \id{\}}$ to denote partial maps,
$s : \QProc \rightarrow \QProc$. A map, $s$ lifts, uniquely, to a map
on process terms, $\widehat{s} : \Proc \rightarrow \Proc$ by the
following equations.

\begin{mathpar}
  (0) \psubstp{Q}{P} := 0 \\
  (R \juxtap S) \psubstp{Q}{P}
  :=    
  (R)\psubstp{Q}{P} \juxtap (S) \psubstp{Q}{P} \\
  (x?(y).R) \psubstp{Q}{P}    
  :=    
  (x)\substp{Q}{P} (z)\concat( (R \psubstn{z}{y}) \psubstp{Q}{P} ) \\
  (\lift{x}{R}) \psubstp{Q}{P}  
  :=
  \lift{(x)\substp{Q}{P}}{ R \psubstp{Q}{P} } \\
%   (\dropn{x})  \psubstp{Q}{P}       
%   := 
%   \left\{ 
%     \begin{array}{ccc} 
%       \dropn{\quotep{Q}} & & x \nameeq \quotep{P} \\
%       \dropn{x} & & otherwise \\
%     \end{array}
%   \right. 
  (\dropn{x})  \psubstp{Q}{P}       
  := 
  \left\{ 
    \begin{array}{ccc} 
      Q & & x \nameeq \quotep{P} \\
      \dropn{x} & & otherwise \\
    \end{array}
  \right.
\end{mathpar}
 

where

\begin{eqnarray}
  (x)\id{\{} \lpquote Q \rpquote / \lpquote P \rpquote \id{\}}            = 
  \left\{ 
    \begin{array}{ccc}
      \lpquote Q \rpquote & & x \nameeq \lpquote P \rpquote \\
      x & & otherwise \\
    \end{array}
  \right. \nonumber
\end{eqnarray}

and $z$ is chosen distinct from $\quotep{P}$, $\quotep{Q}$, the free
names in $Q$, and all the names in $R$. Our $\alpha$-equivalence will
be built in the standard way from this substitution.

\begin{remark}\label{rem:no_self_referential_names}
  One consequence of these definitions is that $\forall P. \quotep{P}
  \not\in \freenames{P}$.
\end{remark}

\subsection{ Dynamic quote: an example }

Anticipating something of what's to come, consider applying the
substitution, $\widehat{\id{\{}u / z \id{\}}}$, to the following pair
of processes, $\lift{w}{y!(z)}$ and $w[ \lpquote y!(z) \rpquote ]$.

\begin{eqnarray}
	\lift{w}{y!(z)}\widehat{\id{\{}u / z \id{\}}}
		& = &
		\lift{w}{y!(u)} \nonumber\\
	w[ \lpquote y!(z) \rpquote ] \widehat{ \id{\{}u / z \id{\}} }
		& = &
		w[ \lpquote y!(z) \rpquote ] \nonumber
\end{eqnarray}

Because the body of the process between quotes is impervious to
substitution, we get radically different answers. In fact, by
examining the first process in an input context,
e.g. $x?(z).\lift{w}{y!(z)}$, we see that the process under the lift
operator may be shaped by prefixed inputs binding a name inside it. In
this sense, the lift operator will be seen as a way to dynamically
construct processes before reifying them as names.

Finally equipped with these standard features we can present the
dynamics of the calculus.

\subsubsection{Operational semantics} 

Finally, we introduce the computational dynamics. What marks these
algebras as distinct from other more traditionally studied algebraic
structures, e.g. vector spaces or polynomial rings, is the manner in
which dynamics is captured. In traditional structures, dynamics is typically
expressed through morphisms between such structures, as in linear maps
between vector spaces or morphisms between rings. In algebras
associated with the semantics of computation, the dynamics is
expressed as part of the algebraic structure itself, through a
reduction reduction relation typically denoted by $\red$. Below, we
give a recursive presentation of this relation for the calculus used
in the encoding.

$\red \subseteq \pi \times \pi$
$\red : \pi \to \mathcal{P}(\pi)$

\begin{mathpar}
  \inferrule* [lab=Comm] { \textsf{match}( x_{src}, x_{trgt} ) } { x_{trgt}?(y)P \; | \; x_{src}!\langle {Q} \rangle \red P\{\quotep{Q}/y}\} }
  \and \\
  \inferrule* [lab=Par] {{P} \red {P}'} {{{P} | {Q}} \red {{P}' | {Q}}}
  \and
  \inferrule* [lab=Equiv]{{{P} \scong {P}'} \andalso {{P}' \red {Q}'} \andalso {{Q}' \scong {Q}}}{{P} \red {Q}}
\end{mathpar}

\begin{eqnarray*}
  match_{\equiv} (\quotep{P},\quotep{Q}) & := & P \equiv Q \\
  match_{\dagger}(\quotep{P},\quotep{Q}) & := & \forall R. P|Q \red^{*} R => R \red^{*} 0 \\
  match_{K}(\quotep{P},\quotep{Q}) & := & K \mbox{ for some context } K
\end{eqnarray*}

$u?(x)P | u!\langle Q \rangle \red P\{\quotep{Q}/x\}$

%We write $\wred$ for $\red^*$, and $P\red$ if $\exists Q $ such that $ P \red Q$.
We write $P\red$ if $\exists Q $ such that $ P \red Q$ and $P\not\red$, otherwise.

\section{Replication}

As mentioned before, it is known that replication (and hence
recursion) can be implemented in a higher-order process algebra
\cite{SangiorgiWalker}. As our first example of calculation with the
machinery thus far presented we give the construction explicitly in
the {\rhoc}.

\begin{eqnarray}
	D_{x} & := & \prefix{x}{y}{(\binpar{\outputp{x}{y}}{@{y}})} \nonumber\\
	\bangp_{x}{P} & := & \binpar{{x}!\langle{\binpar{D_{x}}{P}}\rangle}{D_{x}} \nonumber
\end{eqnarray}

\begin{eqnarray}
	\bangp_{x}{P} & & \nonumber\\
	=
	& {x}!\langle{(\prefix{x}{y}{(\outputp{x}{y} | @{y})) | P}}\rangle 
	      | \prefix{x}{y}{(\outputp{x}{y} | @{y})} & \nonumber\\
	\red
	& (\outputp{x}{y} | @{y})\substn{\quotep{(\prefix{x}{y}{(@{y} | \outputp{x}{y})) | P}}}{y} & \nonumber\\
	=
	& \outputp{x}{\quotep{(\prefix{x}{y}{(\outputp{x}{y} | @{y})) | P}}}
	  | {(\prefix{x}{y}{(\outputp{x}{y} | @{y})) | P}} & \nonumber\\
	\red
	& \ldots & \nonumber\\
	\red^*
	& P | P | \ldots & \nonumber
\end{eqnarray}

Of course, this encoding, as an implementation, runs away, unfolding
$\bangp{P}$ eagerly. A lazier and more implementable replication
operator, restricted to input-guarded processes, may be obtained as follows.

\begin{eqnarray}
\bangp{\prefix{u}{v}{P}} 
	:= 
	\binpar{\lift{x}{\prefix{u}{v}{(\binpar{D(x)}{P})}}}{D(x)} \nonumber
\end{eqnarray}

\begin{remark}
  Note that the lazier definition still does not deal with summation
  or mixed summation (i.e. sums over input and output). The reader is
  invited to construct definitions of replication that deal with these
  features. 

  Further, the definitions are parameterized in a name, $x$. Can you,
  gentle reader, make a definition that eliminates this parameter and
  guarantees no accidental interaction between the replication
  machinery and the process being replicated -- i.e. no accidental
  sharing of names used by the process to get its work done and the
  name(s) used by the replication to effect copying. This latter
  revision of the definition of replication is crucial to obtaining
  the expected identity $!!P \sim !P$.
\end{remark}

\begin{remark}\label{rem:paradoxical_combinator}
  The reader familiar with the lambda calculus will have noticed the
  similarity between $D$ and the paradoxical combinator.

  [Ed. note: the existence of this seems to suggest we have to be more
  restrictive on the set of processes and names we admit if we are to
  support no-cloning.]
\end{remark}

\subsubsection{Bisimulation}

The computational dynamics gives rise to another kind of equivalence,
the equivalence of computational behavior. As previously mentioned
this is typically captured \emph{via} some form of bisimulation.

% The notion we use in this paper is weak barbed bisimulation
% \cite{milner91polyadicpi}.

The notion we use in this paper is derived from weak barbed
bisimulation \cite{milner91polyadicpi}. 

\begin{definition}
An \emph{observation relation}, $\downarrow_{\mathcal N}$, over a set
of names, $\mathcal N$, is the smallest relation satisfying the rules
below.

\infrule[Out-barb]{y \in {\mathcal N}, \; x \nameeq y}
		  {\outputp{x}{v} \downarrow_{\mathcal N} x}
\infrule[Par-barb]{\mbox{$P\downarrow_{\mathcal N} x$ or $Q\downarrow_{\mathcal N} x$}}
		  {\binpar{P}{Q} \downarrow_{\mathcal N} x}

We write $P \Downarrow_{\mathcal N} x$ if there is $Q$ such that 
$P \wred Q$ and $Q \downarrow_{\mathcal N} x$.
\end{definition}

\begin{definition}
%\label{def.bbisim}
An  ${\mathcal N}$-\emph{barbed bisimulation} over a set of names, ${\mathcal N}$, is a symmetric binary relation 
${\mathcal S}_{\mathcal N}$ between agents such that $P\rel{S}_{\mathcal N}Q$ implies:
\begin{enumerate}
\item If $P \red P'$ then $Q \wred Q'$ and $P'\rel{S}_{\mathcal N} Q'$.
\item If $P\downarrow_{\mathcal N} x$, then $Q\Downarrow_{\mathcal N} x$.
\end{enumerate}
$P$ is ${\mathcal N}$-barbed bisimilar to $Q$, written
$P \wbbisim_{\mathcal N} Q$, if $P \rel{S}_{\mathcal N} Q$ for some ${\mathcal N}$-barbed bisimulation ${\mathcal S}_{\mathcal N}$.
\end{definition}

$\mathcal{R} \subseteq \pi \times \pi$

$P \mathcal{R} Q => \forall P'. P \red P' \Rightarrow \exists Q'. Q \red Q', P' \mathcal{R} Q'$

$P \vdash x \Rightarrow Q \vdash x$

\begin{mathpar}
  \inferrule*[lab=Out-barb]{x \nameeq y}{{y}!\langle{Q}\rangle \vdash x}
  \and
  \inferrule*[lab=Par-barb]{\mbox{$P\vdash x$ or $Q\vdash x$}}{\binpar{P}{Q} \vdash x}
\end{mathpar}

\subsubsection{Contexts}

One of the principle advantages of computational calculi like the
$\pi$-calculus is a well-defined notion of context,
contextual-equivalence and a correlation between
contextual-equivalence and notions of bisimulation. The notion of
context allows the decomposition of a process into (sub-)process and
its syntactic environment, its context. Thus, a context may be
thought of as a process with a ``hole'' (written $\Box$) in it. The
application of a context $M$ to a process $P$, written $M[P]$, is
tantamount to filling the hole in $M$ with $P$. In this paper we do
not need the full weight of this theory, but do make use of the notion
of context in the proof the main theorem. 

\begin{mathpar}
  \inferrule* [lab=summation] {} {{M_{M},M_{N}} \bc \Box \;|\; x.M_{A} \;|\; M_{M}+M_{N}}
  \and
  \inferrule* [lab=agent] {} {{M_{A}} \bc (\vec{x})M_{P} \;| \; \clift{P_0,\ldots,M_{P},\ldots,P_N}}
  \and \\
  \inferrule* [lab=process] {} {{M_{P}} \bc M_{N} \;| \;P|M_{P} }
\end{mathpar} 

\begin{mathpar}
  \inferrule* [lab=sychronization] {} {M_{N} \bc \Box \;|\; x?M_{F} \;|\; x!M_{C}}
  \and
  \inferrule* [lab=abstraction] {} {{M_{F}} \bc (x)M_{P} }
  \and
  \inferrule* [lab=concretion] {} {{M_{C}} \bc \langle M_{P} \rangle }
  \and \\
  \inferrule* [lab=process] {} {{M_{P}} \bc M_{N} \;| \;P|M_{P} }
\end{mathpar}

\begin{definition}[contextual application] Given a context $M$, and
  process $P$, we define the \emph{contextual application}, $M[P] :=
  M\{P/\Box\}$. That is, the contextual application of M to P is the
  substitution of $P$ for $\Box$ in $M$.
\end{definition}

$\meaningof{-} : L \to \mathcal{P}(\pi)$

\begin{mathpar}
  \inferrule* [lab=collection] {} {\meaningof{true} = \pi, \and \meaningof{~E} = \pi \setminus \meaningof{E}, \and \meaningof{E_{1} \& E_{2}} = \meaningof{E_{1}} \cap \meaningof{E_{2}}}
\end{mathpar}

\begin{mathpar}
  \inferrule* [lab=structure] {} {\meaningof{0} = \{ P \in \pi | P \equiv 0 \}, \and \\ \meaningof{E_1 | E_2} = \{ P \in \pi | P \equiv P_{1} | P_{2}, P_{1} \in \meaningof{E_{1}}, P_{2} \in \meaningof{E_2}\} }
\end{mathpar}

\begin{mathpar}
 \inferrule* [lab=behavior] {} {\meaningof{\langle a?b \rangle E} = \{ P \in \pi | P \equiv Q | u?(y)P', \\ \and \\\\ \and \\ \;\;\; u \in \meaningof{a}, \forall z.P'\{z/y\} \in \meaningof{E\{z/b\}}\}, \and \\ \meaningof{a!E} = \{ P \in \pi | P \equiv Q | x!\langle P' \rangle, x \in \meaningof{a} P' \in \meaningof{E}\} }
\end{mathpar}

\begin{mathpar}
 \inferrule* [lab=nominal] {} {\meaningof{\quotep{E}} = \{ \quotep{P} \in \quotep{\pi} | P \in \meaningof{E} \}, \and \meaningof{\quotep{P}} = \{ \quotep{Q} \in \quotep{\pi} | P \equiv Q \} \and \\ \meaningof{@\quotep{E}} = \{ P \in \pi | P \equiv @x, x \in \meaningof{E} \}}
\end{mathpar}

\begin{eqnarray*}
  \\
  \meaningof{-} : TS \to ST
\end{eqnarray*}

\begin{eqnarray*}
  \\
  L : TS \to ST
\end{eqnarray*}

\begin{eqnarray*}
  \\
  P \models E \iff P \in \meaningof{E}
\end{eqnarray*}

\begin{eqnarray*}
  P \approx_{L} Q \iff \forall E \in L. P \models E \iff Q \models E
\end{eqnarray*}

\begin{eqnarray*}
  P \approx_{K} Q
\end{eqnarray*}

\begin{eqnarray*}
  P \approx Q
\end{eqnarray*}

$\approx_{K} = \approx = \approx_{L}$

\subsubsection{Contextual duality}

Note that contexts extend the quotation operation to a family of
operations from processes to names. Given a context, $M$, we can
define a \emph{nominal context}, $\quotep{M}$ by $\quotep{M}[P] :=
\quotep{M[P]}$. To foreshadow what is to come we observe that these
operations enjoy a duality with processes very much like the duality
between vectors and maps from vectors to scalars.

Further, because the calculus is essentially higher-order, we have a
correspondence between contexts and processes. More specifically,
given a name $x$ and a context $M$ we can construct $M^{*}_{x}$ such
that 

\begin{mathpar}
  M^{*}_{x} | \lift{x}{P} \red M[P]
\end{mathpar}

namely,

\begin{mathpar}
  M^{*}_{x} := x?(u).M[\dropn{u}]
\end{mathpar}

The dependence of $M^{*}_{x}$ on a name makes it an abstraction, 

\begin{mathpar}
  M^{*} := (x)x?(u).M[\dropn{u}]
\end{mathpar}

\subsection{Additional notation}

It will sometimes be convenient to denote the process a name
quotes. We already have the notation $x = \quotep{P}$, but it will be
convenient to introduce an alternate notation, $\procn{x}$, when we
want to emphasize the connection to the use of the name. Note that, by
virtue of name equivalence, $\quotep{\procn{x}} \nameeq x$; so, the
notation is consistent with previous definitions.

Further, because names have structure it is possible to effect
substitutions on the basis of that structure. This means we need to
upgrade our notation for substitutions, which we accomplish by
adapting comprehension notation. Thus,

\begin{mathpar}
  P\{ y / x : x \in S \}
\end{mathpar}

is interpreted to mean the process derived from P by replacing (in a
capture-avoiding manner) each occurrence of $x$ in $S$ by $y$. For example,

\begin{mathpar}
  P\{ \quotep{\procn{x}|\procn{x}} / x : x \in \freenames{P} \}
\end{mathpar}

will replace each (occurrence) of a free name $x$ in $P$ by
$\quotep{\procn{x}|\procn{x}}$.

Also, we will avail ourselves of the notation $x^{L}$ and $x^{R}$ to
denote injections of a name into disjoint copies of the name
space. There are numerous ways to accomplish this. One example can be
found in \cite{MeredithR05}. This notation overloads to vectors of
names: $\vec{x}^{\pi} := (x_{i}^{\pi} \; : \; 0 \leq i < |\vec{x}| )$ where $\pi \in \{L,R\}$.

We also use $P^{\Box} := P|\Box$.

In \cite{MeredithR05} an interpretation of the new operator is
given. It turns out that there are several possible interpretations
all enjoying the requisite algebraic properties of the operator (see
\cite{milner91polyadicpi}). We will therefore make liberal use of
$(\nu\; \vec{x})P$.

% subsection the_syntax_and_semantics_of_the_notation_system (end)   

\input{qm2pi.qmops} 

\input{qm2pi.sterngerlach} 

\input{qm2pi.metric} 

% section concurrent_process_calculi (end)

%\input{qm2pi.proofsketch}

% section proof sketch (end)

%\input{qm2pi.slviaknots} 

% section spatial logic via knots (end)

\input{qm2pi.conclusion}

% section conclusion (end)

%\input{qm2pi.dtcodes} 

% section wiring algorithm (end)

\input{qm2pi.ack} 

% section acknowledgments (end)

\newpage


\bibliographystyle{plain}   
\bibliography{../../biblios/main.bib}

\input{qm2pi.rhodetails}

\end{document}



\end{document}

 

% section notation (end)

\input{qm2pi.process.calculi} 

% section concurrent_process_calculi_and_spatial_logics_ (end)
    
%\documentclass[12pt]{llncs}
%\documentclass{jktr}

\usepackage[pdftex]{hyperref}                   
\usepackage {listings}
\usepackage {mathpartir}
\usepackage{bcprules}
%\usepackage{listings}
                       
\usepackage{graphicx} 
%\usepackage[margins=2.5cm,nohead,nofoot]{geometry}
%\usepackage{geometry}
\usepackage{amsfonts}
\usepackage{amstext}
\usepackage{latexsym}
\usepackage{amssymb}
\usepackage{color}


%\include{myPreamble}
\documentclass[12pt]{llncs}
%\documentclass{jktr}

\usepackage[pdftex]{hyperref}                   
\usepackage {listings}
\usepackage {mathpartir}
\usepackage{bcprules}
%\usepackage{listings}
                       
\usepackage{graphicx} 
%\usepackage[margins=2.5cm,nohead,nofoot]{geometry}
%\usepackage{geometry}
\usepackage{amsfonts}
\usepackage{amstext}
\usepackage{latexsym}
\usepackage{amssymb}
\usepackage{color}


%\include{myPreamble}
\include{qm2pi.local} 

%\ifpdf
%\usepackage[pdftex]{graphicx}
%\else
%\usepackage{graphicx}
%\fi

 % \ifpdf
%  \usepackage{pdfsync}
%  \if


%\title{Brief Article}
%\author{David F. Snyder}
%\author{L.G. Meredith}

%\address{Dept. of Math., Texas State University--San Marcos, San Marcos, TX 78666}
       
\pagestyle{empty}


\begin{document}

\lstset{language=[Objective]Caml,frame=shadowbox}

\input{qm2pi.front}

% section front matter (end)

\input{qm2pi.intro} 
 
% section introduction (end)

% \input{qm2pi.knotations} 

% section notation (end)

\input{qm2pi.process.calculi} 

% section concurrent_process_calculi_and_spatial_logics_ (end)
    
%\input{qm2pi.knots2pi} 

%\input{qm2pi.trefoil} 

%\input{qm2pi.mainthm} 

% subsection basic_interpretation (end)

%\input{qm2pi.rho.presentation} 
\subsection{The syntax and semantics of the notation system}\label{sub:the_syntax_and_semantics_of_the_notation_system} % (fold)

We now summarize a technical presentation of the calculus that
embodies our theory of dynamics. The typical presentation of such a
calculus follows the style of giving generators and relations on
them. The grammar, below, describing term constructors, freely
generates the set of processes, $\Proc$. This set is then quotiented
by a relation known as structural congruence and it is over this set
that the notion of dynamics is expressed. This presentation is
essentially that of \cite{MeredithR05} with the addition of
polyadicity and summation. For readability we have relegated some of
the technical subtleties to an appendix.

\subsubsection{Process grammar}\label{subsub:process_grammar}

\begin{mathpar}
  \inferrule* [lab=synchronization] {} {{M} \bc \pzero \;|\; x?F \;|\; x!C }
  \and
  \inferrule* [lab=abstraction] {} {{F} \bc (x)P}
  \and
  \inferrule* [lab=concretion] {} {{C} \bc \langle Q \rangle}
  \and
  \inferrule* [lab=process] {} {{P,Q} \bc M \;| \;P|Q \;|\; @{x}}
  \and
  \inferrule* [lab=name] {} {{x} \bc \quotep{P}}
\end{mathpar} 

Note that $\vec{x}$ (resp. $\vec{P}$) denotes a vector of names
(resp. processes) of length $|\vec{x}|$ (resp. $|\vec{P}|$). We adopt
the following useful abbreviations.

\begin{mathpar}
   x?(\vec{y}).P := x.(\vec{y})P \and  x\clift{\vec{P}} := x.\clift{\vec{P}}
   \and x!(y) := \lift{x}{\dropn{y}}
   \and \Pi_{i=0}^{n-1}P_i := P_0 | \ldots | P_{n-1}
\end{mathpar}

\subsubsection{Structural congruence}

\paragraph{Free and bound names and alpha-equivalence.} At the
core of structural equivalence is alpha-equivalence which identifies
process that are the same up to a change of variable. Formally, we
recognize the distinction between free and bound names. The free names
of a process, $\freenames{P}$, may be calculated recursively as
follows:

\begin{mathpar}
\freenames{\pzero} := \emptyset
  \and \\
  \freenames{x?(y).P} := \{ x \} \cup (\freenames{P} \setminus \{ y \})
  \and 
  \freenames{x!\langle P \rangle} := \{ x \} \cup \{ P \} 
  \and \\
  \freenames{P|Q} := \freenames{P} \cup \freenames{Q}
  \and \\
  \freenames{@{x}} := \{ x \}
\end{mathpar}

$\pi$
$\quotep{\pi}$

$\freenames{-} : \pi \to \mathcal{P}(\quotep{\pi})$

\begin{eqnarray*}
  \freenames{\pzero} & := & \emptyset \\
  \freenames{x?(y).P} & := & \{ x \} \cup (\freenames{P} \setminus \{ y \}) \\
  \freenames{x!\langle P \rangle} & := & \{ x \} \cup \{ P \} \\
  \freenames{P|Q} & := & \freenames{P} \cup \freenames{Q} \\
  \freenames{\dropn{x}} & := & \{ x \}
\end{eqnarray*}

The bound names of a process, $\boundnames{P}$, are those names occurring in $P$
that are not free. For example, in $x?(y).0$, the name $x$ is free, while $y$ is bound.

\begin{mathpar}
  \inferrule* [lab=monoidal-laws] {} { P|Q \equiv Q|P \and P|0 \equiv P \and P|(Q|R) \equiv (P|Q)|R }
\end{mathpar}

\begin{mathpar}
  \inferrule* [lab=alpha-equivalence] {} { (x)P \equiv (y)P\{y/x\} \and y \not\in \freenames{P} }
\end{mathpar}

\begin{definition}
Then two processes, $P,Q$, are alpha-equivalent if $P = Q\{\vec{y}/\vec{x}\}$ for
some $\vec{x} \in \boundnames{Q},\vec{y} \in \boundnames{P}$, where $Q\{\vec{y}/\vec{x}\}$
denotes the capture-avoiding substitution of $\vec{y}$ for $\vec{x}$ in $Q$.
\end{definition}

\begin{definition}
  The {\em structural congruence} \cite{SangiorgiWalker} , $\equiv$,
  between processes is the least congruence containing
  alpha-equivalence, satisfying the abelian monoid laws
  (associativity, commutativity and $\pzero$ as identity) for parallel
  composition $|$ and for summation $+$.
\end{definition}

\subsection{Name equivalence}

We take name equivalence, written $\nameeq$, to be the smallest
equivalence relation generated by the following rules.

\begin{mathpar}
\inferrule*[lab=Quote-drop]
{ }
{ \quotep{@{x}} \nameeq x }

\inferrule*[lab=Struct-equiv]
{ P \scong Q }
{ \quotep{P} \nameeq \quotep{Q} }
\end{mathpar}

The astute reader will have noticed that the mutual recursion of names
and processes imposes a mutual recursion on alpha-equivalence and
structural equivalence via name-equivalence. Fortunately, all of this
works out pleasantly and we may calculate in the natural way, free of
concern. The reader interested in the details is referred to the
appendix \ref{appendix:rho_details}.

\subsection{Substitution}

We use $\Proc$ for the set of processes, $\QProc$ for the set of
names, and $\id{\{}\vec{y} / \vec{x} \id{\}}$ to denote partial maps,
$s : \QProc \rightarrow \QProc$. A map, $s$ lifts, uniquely, to a map
on process terms, $\widehat{s} : \Proc \rightarrow \Proc$ by the
following equations.

\begin{mathpar}
  (0) \psubstp{Q}{P} := 0 \\
  (R \juxtap S) \psubstp{Q}{P}
  :=    
  (R)\psubstp{Q}{P} \juxtap (S) \psubstp{Q}{P} \\
  (x?(y).R) \psubstp{Q}{P}    
  :=    
  (x)\substp{Q}{P} (z)\concat( (R \psubstn{z}{y}) \psubstp{Q}{P} ) \\
  (\lift{x}{R}) \psubstp{Q}{P}  
  :=
  \lift{(x)\substp{Q}{P}}{ R \psubstp{Q}{P} } \\
%   (\dropn{x})  \psubstp{Q}{P}       
%   := 
%   \left\{ 
%     \begin{array}{ccc} 
%       \dropn{\quotep{Q}} & & x \nameeq \quotep{P} \\
%       \dropn{x} & & otherwise \\
%     \end{array}
%   \right. 
  (\dropn{x})  \psubstp{Q}{P}       
  := 
  \left\{ 
    \begin{array}{ccc} 
      Q & & x \nameeq \quotep{P} \\
      \dropn{x} & & otherwise \\
    \end{array}
  \right.
\end{mathpar}
 

where

\begin{eqnarray}
  (x)\id{\{} \lpquote Q \rpquote / \lpquote P \rpquote \id{\}}            = 
  \left\{ 
    \begin{array}{ccc}
      \lpquote Q \rpquote & & x \nameeq \lpquote P \rpquote \\
      x & & otherwise \\
    \end{array}
  \right. \nonumber
\end{eqnarray}

and $z$ is chosen distinct from $\quotep{P}$, $\quotep{Q}$, the free
names in $Q$, and all the names in $R$. Our $\alpha$-equivalence will
be built in the standard way from this substitution.

\begin{remark}\label{rem:no_self_referential_names}
  One consequence of these definitions is that $\forall P. \quotep{P}
  \not\in \freenames{P}$.
\end{remark}

\subsection{ Dynamic quote: an example }

Anticipating something of what's to come, consider applying the
substitution, $\widehat{\id{\{}u / z \id{\}}}$, to the following pair
of processes, $\lift{w}{y!(z)}$ and $w[ \lpquote y!(z) \rpquote ]$.

\begin{eqnarray}
	\lift{w}{y!(z)}\widehat{\id{\{}u / z \id{\}}}
		& = &
		\lift{w}{y!(u)} \nonumber\\
	w[ \lpquote y!(z) \rpquote ] \widehat{ \id{\{}u / z \id{\}} }
		& = &
		w[ \lpquote y!(z) \rpquote ] \nonumber
\end{eqnarray}

Because the body of the process between quotes is impervious to
substitution, we get radically different answers. In fact, by
examining the first process in an input context,
e.g. $x?(z).\lift{w}{y!(z)}$, we see that the process under the lift
operator may be shaped by prefixed inputs binding a name inside it. In
this sense, the lift operator will be seen as a way to dynamically
construct processes before reifying them as names.

Finally equipped with these standard features we can present the
dynamics of the calculus.

\subsubsection{Operational semantics} 

Finally, we introduce the computational dynamics. What marks these
algebras as distinct from other more traditionally studied algebraic
structures, e.g. vector spaces or polynomial rings, is the manner in
which dynamics is captured. In traditional structures, dynamics is typically
expressed through morphisms between such structures, as in linear maps
between vector spaces or morphisms between rings. In algebras
associated with the semantics of computation, the dynamics is
expressed as part of the algebraic structure itself, through a
reduction reduction relation typically denoted by $\red$. Below, we
give a recursive presentation of this relation for the calculus used
in the encoding.

$\red \subseteq \pi \times \pi$
$\red : \pi \to \mathcal{P}(\pi)$

\begin{mathpar}
  \inferrule* [lab=Comm] { \textsf{match}( x_{src}, x_{trgt} ) } { x_{trgt}?(y)P \; | \; x_{src}!\langle {Q} \rangle \red P\{\quotep{Q}/y}\} }
  \and \\
  \inferrule* [lab=Par] {{P} \red {P}'} {{{P} | {Q}} \red {{P}' | {Q}}}
  \and
  \inferrule* [lab=Equiv]{{{P} \scong {P}'} \andalso {{P}' \red {Q}'} \andalso {{Q}' \scong {Q}}}{{P} \red {Q}}
\end{mathpar}

\begin{eqnarray*}
  match_{\equiv} (\quotep{P},\quotep{Q}) & := & P \equiv Q \\
  match_{\dagger}(\quotep{P},\quotep{Q}) & := & \forall R. P|Q \red^{*} R => R \red^{*} 0 \\
  match_{K}(\quotep{P},\quotep{Q}) & := & K \mbox{ for some context } K
\end{eqnarray*}

$u?(x)P | u!\langle Q \rangle \red P\{\quotep{Q}/x\}$

%We write $\wred$ for $\red^*$, and $P\red$ if $\exists Q $ such that $ P \red Q$.
We write $P\red$ if $\exists Q $ such that $ P \red Q$ and $P\not\red$, otherwise.

\section{Replication}

As mentioned before, it is known that replication (and hence
recursion) can be implemented in a higher-order process algebra
\cite{SangiorgiWalker}. As our first example of calculation with the
machinery thus far presented we give the construction explicitly in
the {\rhoc}.

\begin{eqnarray}
	D_{x} & := & \prefix{x}{y}{(\binpar{\outputp{x}{y}}{@{y}})} \nonumber\\
	\bangp_{x}{P} & := & \binpar{{x}!\langle{\binpar{D_{x}}{P}}\rangle}{D_{x}} \nonumber
\end{eqnarray}

\begin{eqnarray}
	\bangp_{x}{P} & & \nonumber\\
	=
	& {x}!\langle{(\prefix{x}{y}{(\outputp{x}{y} | @{y})) | P}}\rangle 
	      | \prefix{x}{y}{(\outputp{x}{y} | @{y})} & \nonumber\\
	\red
	& (\outputp{x}{y} | @{y})\substn{\quotep{(\prefix{x}{y}{(@{y} | \outputp{x}{y})) | P}}}{y} & \nonumber\\
	=
	& \outputp{x}{\quotep{(\prefix{x}{y}{(\outputp{x}{y} | @{y})) | P}}}
	  | {(\prefix{x}{y}{(\outputp{x}{y} | @{y})) | P}} & \nonumber\\
	\red
	& \ldots & \nonumber\\
	\red^*
	& P | P | \ldots & \nonumber
\end{eqnarray}

Of course, this encoding, as an implementation, runs away, unfolding
$\bangp{P}$ eagerly. A lazier and more implementable replication
operator, restricted to input-guarded processes, may be obtained as follows.

\begin{eqnarray}
\bangp{\prefix{u}{v}{P}} 
	:= 
	\binpar{\lift{x}{\prefix{u}{v}{(\binpar{D(x)}{P})}}}{D(x)} \nonumber
\end{eqnarray}

\begin{remark}
  Note that the lazier definition still does not deal with summation
  or mixed summation (i.e. sums over input and output). The reader is
  invited to construct definitions of replication that deal with these
  features. 

  Further, the definitions are parameterized in a name, $x$. Can you,
  gentle reader, make a definition that eliminates this parameter and
  guarantees no accidental interaction between the replication
  machinery and the process being replicated -- i.e. no accidental
  sharing of names used by the process to get its work done and the
  name(s) used by the replication to effect copying. This latter
  revision of the definition of replication is crucial to obtaining
  the expected identity $!!P \sim !P$.
\end{remark}

\begin{remark}\label{rem:paradoxical_combinator}
  The reader familiar with the lambda calculus will have noticed the
  similarity between $D$ and the paradoxical combinator.

  [Ed. note: the existence of this seems to suggest we have to be more
  restrictive on the set of processes and names we admit if we are to
  support no-cloning.]
\end{remark}

\subsubsection{Bisimulation}

The computational dynamics gives rise to another kind of equivalence,
the equivalence of computational behavior. As previously mentioned
this is typically captured \emph{via} some form of bisimulation.

% The notion we use in this paper is weak barbed bisimulation
% \cite{milner91polyadicpi}.

The notion we use in this paper is derived from weak barbed
bisimulation \cite{milner91polyadicpi}. 

\begin{definition}
An \emph{observation relation}, $\downarrow_{\mathcal N}$, over a set
of names, $\mathcal N$, is the smallest relation satisfying the rules
below.

\infrule[Out-barb]{y \in {\mathcal N}, \; x \nameeq y}
		  {\outputp{x}{v} \downarrow_{\mathcal N} x}
\infrule[Par-barb]{\mbox{$P\downarrow_{\mathcal N} x$ or $Q\downarrow_{\mathcal N} x$}}
		  {\binpar{P}{Q} \downarrow_{\mathcal N} x}

We write $P \Downarrow_{\mathcal N} x$ if there is $Q$ such that 
$P \wred Q$ and $Q \downarrow_{\mathcal N} x$.
\end{definition}

\begin{definition}
%\label{def.bbisim}
An  ${\mathcal N}$-\emph{barbed bisimulation} over a set of names, ${\mathcal N}$, is a symmetric binary relation 
${\mathcal S}_{\mathcal N}$ between agents such that $P\rel{S}_{\mathcal N}Q$ implies:
\begin{enumerate}
\item If $P \red P'$ then $Q \wred Q'$ and $P'\rel{S}_{\mathcal N} Q'$.
\item If $P\downarrow_{\mathcal N} x$, then $Q\Downarrow_{\mathcal N} x$.
\end{enumerate}
$P$ is ${\mathcal N}$-barbed bisimilar to $Q$, written
$P \wbbisim_{\mathcal N} Q$, if $P \rel{S}_{\mathcal N} Q$ for some ${\mathcal N}$-barbed bisimulation ${\mathcal S}_{\mathcal N}$.
\end{definition}

$\mathcal{R} \subseteq \pi \times \pi$

$P \mathcal{R} Q => \forall P'. P \red P' \Rightarrow \exists Q'. Q \red Q', P' \mathcal{R} Q'$

$P \vdash x \Rightarrow Q \vdash x$

\begin{mathpar}
  \inferrule*[lab=Out-barb]{x \nameeq y}{{y}!\langle{Q}\rangle \vdash x}
  \and
  \inferrule*[lab=Par-barb]{\mbox{$P\vdash x$ or $Q\vdash x$}}{\binpar{P}{Q} \vdash x}
\end{mathpar}

\subsubsection{Contexts}

One of the principle advantages of computational calculi like the
$\pi$-calculus is a well-defined notion of context,
contextual-equivalence and a correlation between
contextual-equivalence and notions of bisimulation. The notion of
context allows the decomposition of a process into (sub-)process and
its syntactic environment, its context. Thus, a context may be
thought of as a process with a ``hole'' (written $\Box$) in it. The
application of a context $M$ to a process $P$, written $M[P]$, is
tantamount to filling the hole in $M$ with $P$. In this paper we do
not need the full weight of this theory, but do make use of the notion
of context in the proof the main theorem. 

\begin{mathpar}
  \inferrule* [lab=summation] {} {{M_{M},M_{N}} \bc \Box \;|\; x.M_{A} \;|\; M_{M}+M_{N}}
  \and
  \inferrule* [lab=agent] {} {{M_{A}} \bc (\vec{x})M_{P} \;| \; \clift{P_0,\ldots,M_{P},\ldots,P_N}}
  \and \\
  \inferrule* [lab=process] {} {{M_{P}} \bc M_{N} \;| \;P|M_{P} }
\end{mathpar} 

\begin{mathpar}
  \inferrule* [lab=sychronization] {} {M_{N} \bc \Box \;|\; x?M_{F} \;|\; x!M_{C}}
  \and
  \inferrule* [lab=abstraction] {} {{M_{F}} \bc (x)M_{P} }
  \and
  \inferrule* [lab=concretion] {} {{M_{C}} \bc \langle M_{P} \rangle }
  \and \\
  \inferrule* [lab=process] {} {{M_{P}} \bc M_{N} \;| \;P|M_{P} }
\end{mathpar}

\begin{definition}[contextual application] Given a context $M$, and
  process $P$, we define the \emph{contextual application}, $M[P] :=
  M\{P/\Box\}$. That is, the contextual application of M to P is the
  substitution of $P$ for $\Box$ in $M$.
\end{definition}

$\meaningof{-} : L \to \mathcal{P}(\pi)$

\begin{mathpar}
  \inferrule* [lab=collection] {} {\meaningof{true} = \pi, \and \meaningof{~E} = \pi \setminus \meaningof{E}, \and \meaningof{E_{1} \& E_{2}} = \meaningof{E_{1}} \cap \meaningof{E_{2}}}
\end{mathpar}

\begin{mathpar}
  \inferrule* [lab=structure] {} {\meaningof{0} = \{ P \in \pi | P \equiv 0 \}, \and \\ \meaningof{E_1 | E_2} = \{ P \in \pi | P \equiv P_{1} | P_{2}, P_{1} \in \meaningof{E_{1}}, P_{2} \in \meaningof{E_2}\} }
\end{mathpar}

\begin{mathpar}
 \inferrule* [lab=behavior] {} {\meaningof{\langle a?b \rangle E} = \{ P \in \pi | P \equiv Q | u?(y)P', \\ \and \\\\ \and \\ \;\;\; u \in \meaningof{a}, \forall z.P'\{z/y\} \in \meaningof{E\{z/b\}}\}, \and \\ \meaningof{a!E} = \{ P \in \pi | P \equiv Q | x!\langle P' \rangle, x \in \meaningof{a} P' \in \meaningof{E}\} }
\end{mathpar}

\begin{mathpar}
 \inferrule* [lab=nominal] {} {\meaningof{\quotep{E}} = \{ \quotep{P} \in \quotep{\pi} | P \in \meaningof{E} \}, \and \meaningof{\quotep{P}} = \{ \quotep{Q} \in \quotep{\pi} | P \equiv Q \} \and \\ \meaningof{@\quotep{E}} = \{ P \in \pi | P \equiv @x, x \in \meaningof{E} \}}
\end{mathpar}

\begin{eqnarray*}
  \\
  \meaningof{-} : TS \to ST
\end{eqnarray*}

\begin{eqnarray*}
  \\
  L : TS \to ST
\end{eqnarray*}

\begin{eqnarray*}
  \\
  P \models E \iff P \in \meaningof{E}
\end{eqnarray*}

\begin{eqnarray*}
  P \approx_{L} Q \iff \forall E \in L. P \models E \iff Q \models E
\end{eqnarray*}

\begin{eqnarray*}
  P \approx_{K} Q
\end{eqnarray*}

\begin{eqnarray*}
  P \approx Q
\end{eqnarray*}

$\approx_{K} = \approx = \approx_{L}$

\subsubsection{Contextual duality}

Note that contexts extend the quotation operation to a family of
operations from processes to names. Given a context, $M$, we can
define a \emph{nominal context}, $\quotep{M}$ by $\quotep{M}[P] :=
\quotep{M[P]}$. To foreshadow what is to come we observe that these
operations enjoy a duality with processes very much like the duality
between vectors and maps from vectors to scalars.

Further, because the calculus is essentially higher-order, we have a
correspondence between contexts and processes. More specifically,
given a name $x$ and a context $M$ we can construct $M^{*}_{x}$ such
that 

\begin{mathpar}
  M^{*}_{x} | \lift{x}{P} \red M[P]
\end{mathpar}

namely,

\begin{mathpar}
  M^{*}_{x} := x?(u).M[\dropn{u}]
\end{mathpar}

The dependence of $M^{*}_{x}$ on a name makes it an abstraction, 

\begin{mathpar}
  M^{*} := (x)x?(u).M[\dropn{u}]
\end{mathpar}

\subsection{Additional notation}

It will sometimes be convenient to denote the process a name
quotes. We already have the notation $x = \quotep{P}$, but it will be
convenient to introduce an alternate notation, $\procn{x}$, when we
want to emphasize the connection to the use of the name. Note that, by
virtue of name equivalence, $\quotep{\procn{x}} \nameeq x$; so, the
notation is consistent with previous definitions.

Further, because names have structure it is possible to effect
substitutions on the basis of that structure. This means we need to
upgrade our notation for substitutions, which we accomplish by
adapting comprehension notation. Thus,

\begin{mathpar}
  P\{ y / x : x \in S \}
\end{mathpar}

is interpreted to mean the process derived from P by replacing (in a
capture-avoiding manner) each occurrence of $x$ in $S$ by $y$. For example,

\begin{mathpar}
  P\{ \quotep{\procn{x}|\procn{x}} / x : x \in \freenames{P} \}
\end{mathpar}

will replace each (occurrence) of a free name $x$ in $P$ by
$\quotep{\procn{x}|\procn{x}}$.

Also, we will avail ourselves of the notation $x^{L}$ and $x^{R}$ to
denote injections of a name into disjoint copies of the name
space. There are numerous ways to accomplish this. One example can be
found in \cite{MeredithR05}. This notation overloads to vectors of
names: $\vec{x}^{\pi} := (x_{i}^{\pi} \; : \; 0 \leq i < |\vec{x}| )$ where $\pi \in \{L,R\}$.

We also use $P^{\Box} := P|\Box$.

In \cite{MeredithR05} an interpretation of the new operator is
given. It turns out that there are several possible interpretations
all enjoying the requisite algebraic properties of the operator (see
\cite{milner91polyadicpi}). We will therefore make liberal use of
$(\nu\; \vec{x})P$.

% subsection the_syntax_and_semantics_of_the_notation_system (end)   

\input{qm2pi.qmops} 

\input{qm2pi.sterngerlach} 

\input{qm2pi.metric} 

% section concurrent_process_calculi (end)

%\input{qm2pi.proofsketch}

% section proof sketch (end)

%\input{qm2pi.slviaknots} 

% section spatial logic via knots (end)

\input{qm2pi.conclusion}

% section conclusion (end)

%\input{qm2pi.dtcodes} 

% section wiring algorithm (end)

\input{qm2pi.ack} 

% section acknowledgments (end)

\newpage


\bibliographystyle{plain}   
\bibliography{../../biblios/main.bib}

\input{qm2pi.rhodetails}

\end{document}

 

%\ifpdf
%\usepackage[pdftex]{graphicx}
%\else
%\usepackage{graphicx}
%\fi

 % \ifpdf
%  \usepackage{pdfsync}
%  \if


%\title{Brief Article}
%\author{David F. Snyder}
%\author{L.G. Meredith}

%\address{Dept. of Math., Texas State University--San Marcos, San Marcos, TX 78666}
       
\pagestyle{empty}


\begin{document}

\lstset{language=[Objective]Caml,frame=shadowbox}

\documentclass[12pt]{llncs}
%\documentclass{jktr}

\usepackage[pdftex]{hyperref}                   
\usepackage {listings}
\usepackage {mathpartir}
\usepackage{bcprules}
%\usepackage{listings}
                       
\usepackage{graphicx} 
%\usepackage[margins=2.5cm,nohead,nofoot]{geometry}
%\usepackage{geometry}
\usepackage{amsfonts}
\usepackage{amstext}
\usepackage{latexsym}
\usepackage{amssymb}
\usepackage{color}


%\include{myPreamble}
\include{qm2pi.local} 

%\ifpdf
%\usepackage[pdftex]{graphicx}
%\else
%\usepackage{graphicx}
%\fi

 % \ifpdf
%  \usepackage{pdfsync}
%  \if


%\title{Brief Article}
%\author{David F. Snyder}
%\author{L.G. Meredith}

%\address{Dept. of Math., Texas State University--San Marcos, San Marcos, TX 78666}
       
\pagestyle{empty}


\begin{document}

\lstset{language=[Objective]Caml,frame=shadowbox}

\input{qm2pi.front}

% section front matter (end)

\input{qm2pi.intro} 
 
% section introduction (end)

% \input{qm2pi.knotations} 

% section notation (end)

\input{qm2pi.process.calculi} 

% section concurrent_process_calculi_and_spatial_logics_ (end)
    
%\input{qm2pi.knots2pi} 

%\input{qm2pi.trefoil} 

%\input{qm2pi.mainthm} 

% subsection basic_interpretation (end)

%\input{qm2pi.rho.presentation} 
\subsection{The syntax and semantics of the notation system}\label{sub:the_syntax_and_semantics_of_the_notation_system} % (fold)

We now summarize a technical presentation of the calculus that
embodies our theory of dynamics. The typical presentation of such a
calculus follows the style of giving generators and relations on
them. The grammar, below, describing term constructors, freely
generates the set of processes, $\Proc$. This set is then quotiented
by a relation known as structural congruence and it is over this set
that the notion of dynamics is expressed. This presentation is
essentially that of \cite{MeredithR05} with the addition of
polyadicity and summation. For readability we have relegated some of
the technical subtleties to an appendix.

\subsubsection{Process grammar}\label{subsub:process_grammar}

\begin{mathpar}
  \inferrule* [lab=synchronization] {} {{M} \bc \pzero \;|\; x?F \;|\; x!C }
  \and
  \inferrule* [lab=abstraction] {} {{F} \bc (x)P}
  \and
  \inferrule* [lab=concretion] {} {{C} \bc \langle Q \rangle}
  \and
  \inferrule* [lab=process] {} {{P,Q} \bc M \;| \;P|Q \;|\; @{x}}
  \and
  \inferrule* [lab=name] {} {{x} \bc \quotep{P}}
\end{mathpar} 

Note that $\vec{x}$ (resp. $\vec{P}$) denotes a vector of names
(resp. processes) of length $|\vec{x}|$ (resp. $|\vec{P}|$). We adopt
the following useful abbreviations.

\begin{mathpar}
   x?(\vec{y}).P := x.(\vec{y})P \and  x\clift{\vec{P}} := x.\clift{\vec{P}}
   \and x!(y) := \lift{x}{\dropn{y}}
   \and \Pi_{i=0}^{n-1}P_i := P_0 | \ldots | P_{n-1}
\end{mathpar}

\subsubsection{Structural congruence}

\paragraph{Free and bound names and alpha-equivalence.} At the
core of structural equivalence is alpha-equivalence which identifies
process that are the same up to a change of variable. Formally, we
recognize the distinction between free and bound names. The free names
of a process, $\freenames{P}$, may be calculated recursively as
follows:

\begin{mathpar}
\freenames{\pzero} := \emptyset
  \and \\
  \freenames{x?(y).P} := \{ x \} \cup (\freenames{P} \setminus \{ y \})
  \and 
  \freenames{x!\langle P \rangle} := \{ x \} \cup \{ P \} 
  \and \\
  \freenames{P|Q} := \freenames{P} \cup \freenames{Q}
  \and \\
  \freenames{@{x}} := \{ x \}
\end{mathpar}

$\pi$
$\quotep{\pi}$

$\freenames{-} : \pi \to \mathcal{P}(\quotep{\pi})$

\begin{eqnarray*}
  \freenames{\pzero} & := & \emptyset \\
  \freenames{x?(y).P} & := & \{ x \} \cup (\freenames{P} \setminus \{ y \}) \\
  \freenames{x!\langle P \rangle} & := & \{ x \} \cup \{ P \} \\
  \freenames{P|Q} & := & \freenames{P} \cup \freenames{Q} \\
  \freenames{\dropn{x}} & := & \{ x \}
\end{eqnarray*}

The bound names of a process, $\boundnames{P}$, are those names occurring in $P$
that are not free. For example, in $x?(y).0$, the name $x$ is free, while $y$ is bound.

\begin{mathpar}
  \inferrule* [lab=monoidal-laws] {} { P|Q \equiv Q|P \and P|0 \equiv P \and P|(Q|R) \equiv (P|Q)|R }
\end{mathpar}

\begin{mathpar}
  \inferrule* [lab=alpha-equivalence] {} { (x)P \equiv (y)P\{y/x\} \and y \not\in \freenames{P} }
\end{mathpar}

\begin{definition}
Then two processes, $P,Q$, are alpha-equivalent if $P = Q\{\vec{y}/\vec{x}\}$ for
some $\vec{x} \in \boundnames{Q},\vec{y} \in \boundnames{P}$, where $Q\{\vec{y}/\vec{x}\}$
denotes the capture-avoiding substitution of $\vec{y}$ for $\vec{x}$ in $Q$.
\end{definition}

\begin{definition}
  The {\em structural congruence} \cite{SangiorgiWalker} , $\equiv$,
  between processes is the least congruence containing
  alpha-equivalence, satisfying the abelian monoid laws
  (associativity, commutativity and $\pzero$ as identity) for parallel
  composition $|$ and for summation $+$.
\end{definition}

\subsection{Name equivalence}

We take name equivalence, written $\nameeq$, to be the smallest
equivalence relation generated by the following rules.

\begin{mathpar}
\inferrule*[lab=Quote-drop]
{ }
{ \quotep{@{x}} \nameeq x }

\inferrule*[lab=Struct-equiv]
{ P \scong Q }
{ \quotep{P} \nameeq \quotep{Q} }
\end{mathpar}

The astute reader will have noticed that the mutual recursion of names
and processes imposes a mutual recursion on alpha-equivalence and
structural equivalence via name-equivalence. Fortunately, all of this
works out pleasantly and we may calculate in the natural way, free of
concern. The reader interested in the details is referred to the
appendix \ref{appendix:rho_details}.

\subsection{Substitution}

We use $\Proc$ for the set of processes, $\QProc$ for the set of
names, and $\id{\{}\vec{y} / \vec{x} \id{\}}$ to denote partial maps,
$s : \QProc \rightarrow \QProc$. A map, $s$ lifts, uniquely, to a map
on process terms, $\widehat{s} : \Proc \rightarrow \Proc$ by the
following equations.

\begin{mathpar}
  (0) \psubstp{Q}{P} := 0 \\
  (R \juxtap S) \psubstp{Q}{P}
  :=    
  (R)\psubstp{Q}{P} \juxtap (S) \psubstp{Q}{P} \\
  (x?(y).R) \psubstp{Q}{P}    
  :=    
  (x)\substp{Q}{P} (z)\concat( (R \psubstn{z}{y}) \psubstp{Q}{P} ) \\
  (\lift{x}{R}) \psubstp{Q}{P}  
  :=
  \lift{(x)\substp{Q}{P}}{ R \psubstp{Q}{P} } \\
%   (\dropn{x})  \psubstp{Q}{P}       
%   := 
%   \left\{ 
%     \begin{array}{ccc} 
%       \dropn{\quotep{Q}} & & x \nameeq \quotep{P} \\
%       \dropn{x} & & otherwise \\
%     \end{array}
%   \right. 
  (\dropn{x})  \psubstp{Q}{P}       
  := 
  \left\{ 
    \begin{array}{ccc} 
      Q & & x \nameeq \quotep{P} \\
      \dropn{x} & & otherwise \\
    \end{array}
  \right.
\end{mathpar}
 

where

\begin{eqnarray}
  (x)\id{\{} \lpquote Q \rpquote / \lpquote P \rpquote \id{\}}            = 
  \left\{ 
    \begin{array}{ccc}
      \lpquote Q \rpquote & & x \nameeq \lpquote P \rpquote \\
      x & & otherwise \\
    \end{array}
  \right. \nonumber
\end{eqnarray}

and $z$ is chosen distinct from $\quotep{P}$, $\quotep{Q}$, the free
names in $Q$, and all the names in $R$. Our $\alpha$-equivalence will
be built in the standard way from this substitution.

\begin{remark}\label{rem:no_self_referential_names}
  One consequence of these definitions is that $\forall P. \quotep{P}
  \not\in \freenames{P}$.
\end{remark}

\subsection{ Dynamic quote: an example }

Anticipating something of what's to come, consider applying the
substitution, $\widehat{\id{\{}u / z \id{\}}}$, to the following pair
of processes, $\lift{w}{y!(z)}$ and $w[ \lpquote y!(z) \rpquote ]$.

\begin{eqnarray}
	\lift{w}{y!(z)}\widehat{\id{\{}u / z \id{\}}}
		& = &
		\lift{w}{y!(u)} \nonumber\\
	w[ \lpquote y!(z) \rpquote ] \widehat{ \id{\{}u / z \id{\}} }
		& = &
		w[ \lpquote y!(z) \rpquote ] \nonumber
\end{eqnarray}

Because the body of the process between quotes is impervious to
substitution, we get radically different answers. In fact, by
examining the first process in an input context,
e.g. $x?(z).\lift{w}{y!(z)}$, we see that the process under the lift
operator may be shaped by prefixed inputs binding a name inside it. In
this sense, the lift operator will be seen as a way to dynamically
construct processes before reifying them as names.

Finally equipped with these standard features we can present the
dynamics of the calculus.

\subsubsection{Operational semantics} 

Finally, we introduce the computational dynamics. What marks these
algebras as distinct from other more traditionally studied algebraic
structures, e.g. vector spaces or polynomial rings, is the manner in
which dynamics is captured. In traditional structures, dynamics is typically
expressed through morphisms between such structures, as in linear maps
between vector spaces or morphisms between rings. In algebras
associated with the semantics of computation, the dynamics is
expressed as part of the algebraic structure itself, through a
reduction reduction relation typically denoted by $\red$. Below, we
give a recursive presentation of this relation for the calculus used
in the encoding.

$\red \subseteq \pi \times \pi$
$\red : \pi \to \mathcal{P}(\pi)$

\begin{mathpar}
  \inferrule* [lab=Comm] { \textsf{match}( x_{src}, x_{trgt} ) } { x_{trgt}?(y)P \; | \; x_{src}!\langle {Q} \rangle \red P\{\quotep{Q}/y}\} }
  \and \\
  \inferrule* [lab=Par] {{P} \red {P}'} {{{P} | {Q}} \red {{P}' | {Q}}}
  \and
  \inferrule* [lab=Equiv]{{{P} \scong {P}'} \andalso {{P}' \red {Q}'} \andalso {{Q}' \scong {Q}}}{{P} \red {Q}}
\end{mathpar}

\begin{eqnarray*}
  match_{\equiv} (\quotep{P},\quotep{Q}) & := & P \equiv Q \\
  match_{\dagger}(\quotep{P},\quotep{Q}) & := & \forall R. P|Q \red^{*} R => R \red^{*} 0 \\
  match_{K}(\quotep{P},\quotep{Q}) & := & K \mbox{ for some context } K
\end{eqnarray*}

$u?(x)P | u!\langle Q \rangle \red P\{\quotep{Q}/x\}$

%We write $\wred$ for $\red^*$, and $P\red$ if $\exists Q $ such that $ P \red Q$.
We write $P\red$ if $\exists Q $ such that $ P \red Q$ and $P\not\red$, otherwise.

\section{Replication}

As mentioned before, it is known that replication (and hence
recursion) can be implemented in a higher-order process algebra
\cite{SangiorgiWalker}. As our first example of calculation with the
machinery thus far presented we give the construction explicitly in
the {\rhoc}.

\begin{eqnarray}
	D_{x} & := & \prefix{x}{y}{(\binpar{\outputp{x}{y}}{@{y}})} \nonumber\\
	\bangp_{x}{P} & := & \binpar{{x}!\langle{\binpar{D_{x}}{P}}\rangle}{D_{x}} \nonumber
\end{eqnarray}

\begin{eqnarray}
	\bangp_{x}{P} & & \nonumber\\
	=
	& {x}!\langle{(\prefix{x}{y}{(\outputp{x}{y} | @{y})) | P}}\rangle 
	      | \prefix{x}{y}{(\outputp{x}{y} | @{y})} & \nonumber\\
	\red
	& (\outputp{x}{y} | @{y})\substn{\quotep{(\prefix{x}{y}{(@{y} | \outputp{x}{y})) | P}}}{y} & \nonumber\\
	=
	& \outputp{x}{\quotep{(\prefix{x}{y}{(\outputp{x}{y} | @{y})) | P}}}
	  | {(\prefix{x}{y}{(\outputp{x}{y} | @{y})) | P}} & \nonumber\\
	\red
	& \ldots & \nonumber\\
	\red^*
	& P | P | \ldots & \nonumber
\end{eqnarray}

Of course, this encoding, as an implementation, runs away, unfolding
$\bangp{P}$ eagerly. A lazier and more implementable replication
operator, restricted to input-guarded processes, may be obtained as follows.

\begin{eqnarray}
\bangp{\prefix{u}{v}{P}} 
	:= 
	\binpar{\lift{x}{\prefix{u}{v}{(\binpar{D(x)}{P})}}}{D(x)} \nonumber
\end{eqnarray}

\begin{remark}
  Note that the lazier definition still does not deal with summation
  or mixed summation (i.e. sums over input and output). The reader is
  invited to construct definitions of replication that deal with these
  features. 

  Further, the definitions are parameterized in a name, $x$. Can you,
  gentle reader, make a definition that eliminates this parameter and
  guarantees no accidental interaction between the replication
  machinery and the process being replicated -- i.e. no accidental
  sharing of names used by the process to get its work done and the
  name(s) used by the replication to effect copying. This latter
  revision of the definition of replication is crucial to obtaining
  the expected identity $!!P \sim !P$.
\end{remark}

\begin{remark}\label{rem:paradoxical_combinator}
  The reader familiar with the lambda calculus will have noticed the
  similarity between $D$ and the paradoxical combinator.

  [Ed. note: the existence of this seems to suggest we have to be more
  restrictive on the set of processes and names we admit if we are to
  support no-cloning.]
\end{remark}

\subsubsection{Bisimulation}

The computational dynamics gives rise to another kind of equivalence,
the equivalence of computational behavior. As previously mentioned
this is typically captured \emph{via} some form of bisimulation.

% The notion we use in this paper is weak barbed bisimulation
% \cite{milner91polyadicpi}.

The notion we use in this paper is derived from weak barbed
bisimulation \cite{milner91polyadicpi}. 

\begin{definition}
An \emph{observation relation}, $\downarrow_{\mathcal N}$, over a set
of names, $\mathcal N$, is the smallest relation satisfying the rules
below.

\infrule[Out-barb]{y \in {\mathcal N}, \; x \nameeq y}
		  {\outputp{x}{v} \downarrow_{\mathcal N} x}
\infrule[Par-barb]{\mbox{$P\downarrow_{\mathcal N} x$ or $Q\downarrow_{\mathcal N} x$}}
		  {\binpar{P}{Q} \downarrow_{\mathcal N} x}

We write $P \Downarrow_{\mathcal N} x$ if there is $Q$ such that 
$P \wred Q$ and $Q \downarrow_{\mathcal N} x$.
\end{definition}

\begin{definition}
%\label{def.bbisim}
An  ${\mathcal N}$-\emph{barbed bisimulation} over a set of names, ${\mathcal N}$, is a symmetric binary relation 
${\mathcal S}_{\mathcal N}$ between agents such that $P\rel{S}_{\mathcal N}Q$ implies:
\begin{enumerate}
\item If $P \red P'$ then $Q \wred Q'$ and $P'\rel{S}_{\mathcal N} Q'$.
\item If $P\downarrow_{\mathcal N} x$, then $Q\Downarrow_{\mathcal N} x$.
\end{enumerate}
$P$ is ${\mathcal N}$-barbed bisimilar to $Q$, written
$P \wbbisim_{\mathcal N} Q$, if $P \rel{S}_{\mathcal N} Q$ for some ${\mathcal N}$-barbed bisimulation ${\mathcal S}_{\mathcal N}$.
\end{definition}

$\mathcal{R} \subseteq \pi \times \pi$

$P \mathcal{R} Q => \forall P'. P \red P' \Rightarrow \exists Q'. Q \red Q', P' \mathcal{R} Q'$

$P \vdash x \Rightarrow Q \vdash x$

\begin{mathpar}
  \inferrule*[lab=Out-barb]{x \nameeq y}{{y}!\langle{Q}\rangle \vdash x}
  \and
  \inferrule*[lab=Par-barb]{\mbox{$P\vdash x$ or $Q\vdash x$}}{\binpar{P}{Q} \vdash x}
\end{mathpar}

\subsubsection{Contexts}

One of the principle advantages of computational calculi like the
$\pi$-calculus is a well-defined notion of context,
contextual-equivalence and a correlation between
contextual-equivalence and notions of bisimulation. The notion of
context allows the decomposition of a process into (sub-)process and
its syntactic environment, its context. Thus, a context may be
thought of as a process with a ``hole'' (written $\Box$) in it. The
application of a context $M$ to a process $P$, written $M[P]$, is
tantamount to filling the hole in $M$ with $P$. In this paper we do
not need the full weight of this theory, but do make use of the notion
of context in the proof the main theorem. 

\begin{mathpar}
  \inferrule* [lab=summation] {} {{M_{M},M_{N}} \bc \Box \;|\; x.M_{A} \;|\; M_{M}+M_{N}}
  \and
  \inferrule* [lab=agent] {} {{M_{A}} \bc (\vec{x})M_{P} \;| \; \clift{P_0,\ldots,M_{P},\ldots,P_N}}
  \and \\
  \inferrule* [lab=process] {} {{M_{P}} \bc M_{N} \;| \;P|M_{P} }
\end{mathpar} 

\begin{mathpar}
  \inferrule* [lab=sychronization] {} {M_{N} \bc \Box \;|\; x?M_{F} \;|\; x!M_{C}}
  \and
  \inferrule* [lab=abstraction] {} {{M_{F}} \bc (x)M_{P} }
  \and
  \inferrule* [lab=concretion] {} {{M_{C}} \bc \langle M_{P} \rangle }
  \and \\
  \inferrule* [lab=process] {} {{M_{P}} \bc M_{N} \;| \;P|M_{P} }
\end{mathpar}

\begin{definition}[contextual application] Given a context $M$, and
  process $P$, we define the \emph{contextual application}, $M[P] :=
  M\{P/\Box\}$. That is, the contextual application of M to P is the
  substitution of $P$ for $\Box$ in $M$.
\end{definition}

$\meaningof{-} : L \to \mathcal{P}(\pi)$

\begin{mathpar}
  \inferrule* [lab=collection] {} {\meaningof{true} = \pi, \and \meaningof{~E} = \pi \setminus \meaningof{E}, \and \meaningof{E_{1} \& E_{2}} = \meaningof{E_{1}} \cap \meaningof{E_{2}}}
\end{mathpar}

\begin{mathpar}
  \inferrule* [lab=structure] {} {\meaningof{0} = \{ P \in \pi | P \equiv 0 \}, \and \\ \meaningof{E_1 | E_2} = \{ P \in \pi | P \equiv P_{1} | P_{2}, P_{1} \in \meaningof{E_{1}}, P_{2} \in \meaningof{E_2}\} }
\end{mathpar}

\begin{mathpar}
 \inferrule* [lab=behavior] {} {\meaningof{\langle a?b \rangle E} = \{ P \in \pi | P \equiv Q | u?(y)P', \\ \and \\\\ \and \\ \;\;\; u \in \meaningof{a}, \forall z.P'\{z/y\} \in \meaningof{E\{z/b\}}\}, \and \\ \meaningof{a!E} = \{ P \in \pi | P \equiv Q | x!\langle P' \rangle, x \in \meaningof{a} P' \in \meaningof{E}\} }
\end{mathpar}

\begin{mathpar}
 \inferrule* [lab=nominal] {} {\meaningof{\quotep{E}} = \{ \quotep{P} \in \quotep{\pi} | P \in \meaningof{E} \}, \and \meaningof{\quotep{P}} = \{ \quotep{Q} \in \quotep{\pi} | P \equiv Q \} \and \\ \meaningof{@\quotep{E}} = \{ P \in \pi | P \equiv @x, x \in \meaningof{E} \}}
\end{mathpar}

\begin{eqnarray*}
  \\
  \meaningof{-} : TS \to ST
\end{eqnarray*}

\begin{eqnarray*}
  \\
  L : TS \to ST
\end{eqnarray*}

\begin{eqnarray*}
  \\
  P \models E \iff P \in \meaningof{E}
\end{eqnarray*}

\begin{eqnarray*}
  P \approx_{L} Q \iff \forall E \in L. P \models E \iff Q \models E
\end{eqnarray*}

\begin{eqnarray*}
  P \approx_{K} Q
\end{eqnarray*}

\begin{eqnarray*}
  P \approx Q
\end{eqnarray*}

$\approx_{K} = \approx = \approx_{L}$

\subsubsection{Contextual duality}

Note that contexts extend the quotation operation to a family of
operations from processes to names. Given a context, $M$, we can
define a \emph{nominal context}, $\quotep{M}$ by $\quotep{M}[P] :=
\quotep{M[P]}$. To foreshadow what is to come we observe that these
operations enjoy a duality with processes very much like the duality
between vectors and maps from vectors to scalars.

Further, because the calculus is essentially higher-order, we have a
correspondence between contexts and processes. More specifically,
given a name $x$ and a context $M$ we can construct $M^{*}_{x}$ such
that 

\begin{mathpar}
  M^{*}_{x} | \lift{x}{P} \red M[P]
\end{mathpar}

namely,

\begin{mathpar}
  M^{*}_{x} := x?(u).M[\dropn{u}]
\end{mathpar}

The dependence of $M^{*}_{x}$ on a name makes it an abstraction, 

\begin{mathpar}
  M^{*} := (x)x?(u).M[\dropn{u}]
\end{mathpar}

\subsection{Additional notation}

It will sometimes be convenient to denote the process a name
quotes. We already have the notation $x = \quotep{P}$, but it will be
convenient to introduce an alternate notation, $\procn{x}$, when we
want to emphasize the connection to the use of the name. Note that, by
virtue of name equivalence, $\quotep{\procn{x}} \nameeq x$; so, the
notation is consistent with previous definitions.

Further, because names have structure it is possible to effect
substitutions on the basis of that structure. This means we need to
upgrade our notation for substitutions, which we accomplish by
adapting comprehension notation. Thus,

\begin{mathpar}
  P\{ y / x : x \in S \}
\end{mathpar}

is interpreted to mean the process derived from P by replacing (in a
capture-avoiding manner) each occurrence of $x$ in $S$ by $y$. For example,

\begin{mathpar}
  P\{ \quotep{\procn{x}|\procn{x}} / x : x \in \freenames{P} \}
\end{mathpar}

will replace each (occurrence) of a free name $x$ in $P$ by
$\quotep{\procn{x}|\procn{x}}$.

Also, we will avail ourselves of the notation $x^{L}$ and $x^{R}$ to
denote injections of a name into disjoint copies of the name
space. There are numerous ways to accomplish this. One example can be
found in \cite{MeredithR05}. This notation overloads to vectors of
names: $\vec{x}^{\pi} := (x_{i}^{\pi} \; : \; 0 \leq i < |\vec{x}| )$ where $\pi \in \{L,R\}$.

We also use $P^{\Box} := P|\Box$.

In \cite{MeredithR05} an interpretation of the new operator is
given. It turns out that there are several possible interpretations
all enjoying the requisite algebraic properties of the operator (see
\cite{milner91polyadicpi}). We will therefore make liberal use of
$(\nu\; \vec{x})P$.

% subsection the_syntax_and_semantics_of_the_notation_system (end)   

\input{qm2pi.qmops} 

\input{qm2pi.sterngerlach} 

\input{qm2pi.metric} 

% section concurrent_process_calculi (end)

%\input{qm2pi.proofsketch}

% section proof sketch (end)

%\input{qm2pi.slviaknots} 

% section spatial logic via knots (end)

\input{qm2pi.conclusion}

% section conclusion (end)

%\input{qm2pi.dtcodes} 

% section wiring algorithm (end)

\input{qm2pi.ack} 

% section acknowledgments (end)

\newpage


\bibliographystyle{plain}   
\bibliography{../../biblios/main.bib}

\input{qm2pi.rhodetails}

\end{document}



% section front matter (end)

\section{Introduction}\label{sec:introduction} % (fold)
In this draft of the material i am going to have to dispense with the
usual writing conventions adopted in papers on these topics. i'm going
to have adopt whatever tone i need at the time i'm writing up the
calculations. Sometimes this may be very conversational; others it may
be the barest mathematical grunts; others still it may be that i have
lifted text from one of my other papers because the exposition of some
point was better said there. i hope that my readers are not unduly put
out by this decision. i'm not doing this to flout convention or be
rebellious. i find these calculations very technically challenging. To
keep everything going technically, something has to give; i have to
let go of some cognitive burden. So, the academic writing style --
with all of its trade-offs in terms of facilitating technical
communication -- is what i'm letting go of. Perhaps subsequent drafts
can be tightened and polished, but for now, i'm going to speak as if
we were sitting together in a coffee shop with a laptop, wifi and a
pad of paper and a pencil.

So, here's what i have to say. We -- you and i, comfortably ensconced
in our coffee shop and well-equipped with our tools -- can realize and
carry out the calculations of quantum mechanics over a very different
formal theory of dynamics, a formal theory of dynamics that
corresponds to a theory of concurrent computation with
\emph{reflection}. It has the advantage that the underlying theory is
already `quantized', but supports analogues all of the continuuous
operations. Strikingly, this underlying theory has recently been
connected with a notion of metric that we can show, by calculating
together, coincides with the metric induced by the inner product.

There are a lot of reasons why you might be interested in seeing
calculations of this form. Here's why i'm interested. For the past
several centuries there has been no competitor to the ``Newtonian''
account of dynamics. As a result the predominant share of accounts of
dynamical systems and situations have had to be formulated in terms of
the Newtonian machinery. i view this as an intellectually dangerous
position to occupy. Everything, despite it's intrinsic shape, turns
into a nail to be hit with this hammer. Recently, however, the theory
of computation has matured to the point where we have candidates for
theories of dynamics that offer very different perspective on
reasoning about dynamical systems and situations. Testing these
candidates against very successful accounts of dynamical situations,
like quantum mechanics, is going to give us some sense of how mature
they are and some measure of the quality of these accounts of
dynamics.

\subsection{Summary of contributions and outline of paper}

So, we're going to develop an interpretation of the operations of
quantum mechanics normally interpreted by Hilbert spaces and
operators. We're going to do this over a theory of computation. Note
that this is very different than the usual quantum computation program
which develops notions of computation over quantum mechanics. Rather,
we are developing a story that aligns with Wheeler's slogan: It from
Bit. To do this we will first provide an account of the theory of
computation at play here. Then we will dive into a calculation-driven
interpretation of the operations of quantum mechanics.

The reason we take this approach is that -- until very recently --
there hasn't been an axiomatic account of quantum mechanics. As a
result there has been no sharp delineation of the mathematical theory
supporting interpretation of the physical theory and the physical
theory, itself. So, ambient features of the maths are free to be
exploited (or supressed) without a real accounting of their physical
relevance. There is no sharp statement ``here's the physical theory''
qua \emph{theory} and ``here's the mathematical interpretation''
enabling a judgment of how faithful the interpretation is -- apart
from experimental observation. When there is an axiomatic account we
can judge how well a given mathematical formalism supports an
interpretation of the axioms, independent of
experimentation. Likewise, we can judge how well we have captured our
physical evidence and experience with our axiomatics, independent of
any specific mathematical implementation, with accidental detail that
may or may not have physical significance. 

In lieu of a fully fleshed out and vetted axiomatic account of quantum
mechanics, interpreting the operational notions in service of modeling
physical systems will have to suffice. In other words, we are not in
the business of providing a model of Hilbert spaces and operators. We
are in the business of providing a model of quantum mechanics because
we are motivated by testing our notions of dynamics against physical
theory; and, the predictive calculations of the physical theory must
serve as the best formulation -- shy of a fully fleshed out axiomatic
account -- of the physical theory itself (as they have for scientific
theories since time immemorial). Put another way, despite a
whole-hearted commitment to an It-from-Bit ontology, we are firmly
aligned with the shut-up-and-calculate camp as the best way to obtain
results either from the physical perspective or as a quality assurance
measure of our fledgling theory of dynamics.

In detail, we present a reflective process calculus. Then we develop
intuitive correspondences between the notions available in this
calculus and the usual physical notions supporting quantum mechanical
calculations. Thus, 

\begin{table}[htp]
  \center{
    \fbox{
      \begin{tabular}{c|c}
        quantum mechanics & process calculus \\
        \hline
        scalar & name \\
        state vector & process \\
        dual & contextual duals \\
        matrix & formal sums of process-context-dual pairs \\
        orthogonality & process annihilation \\
        inner product & execution-formula + quoting
      \end{tabular}
    }
  }
  \caption{QM - process calculi correspondences}
\end{table}

Then we tighten up these intuitions to operational definitions. We
employ the Dirac notation as the best proxy we can find for an
abstract syntax of the quantum mechanical notions. The definitions we
develop put us in contact with equational constraints coming from the
theory that we demonstrate the definitions and calculations satisfy.

This puts us in a position to shut up and calculate for the
Stern-Gerlach experimental set up, showing how these predictive
calculations become calculations on processes in our theory of a
reflective process calculus.

Penultimately, we demonstrate that the notion of metric coming from
the inner product coincides with the notion of metric available from
the theory of bisimulation. This demonstration gives us the right to
think of space as arising from behavior. Finally, we consider where we
might go from the new vantage point we have obtained.

% section introduction (end) 
 
% section introduction (end)

% \documentclass[12pt]{llncs}
%\documentclass{jktr}

\usepackage[pdftex]{hyperref}                   
\usepackage {listings}
\usepackage {mathpartir}
\usepackage{bcprules}
%\usepackage{listings}
                       
\usepackage{graphicx} 
%\usepackage[margins=2.5cm,nohead,nofoot]{geometry}
%\usepackage{geometry}
\usepackage{amsfonts}
\usepackage{amstext}
\usepackage{latexsym}
\usepackage{amssymb}
\usepackage{color}


%\include{myPreamble}
\include{qm2pi.local} 

%\ifpdf
%\usepackage[pdftex]{graphicx}
%\else
%\usepackage{graphicx}
%\fi

 % \ifpdf
%  \usepackage{pdfsync}
%  \if


%\title{Brief Article}
%\author{David F. Snyder}
%\author{L.G. Meredith}

%\address{Dept. of Math., Texas State University--San Marcos, San Marcos, TX 78666}
       
\pagestyle{empty}


\begin{document}

\lstset{language=[Objective]Caml,frame=shadowbox}

\input{qm2pi.front}

% section front matter (end)

\input{qm2pi.intro} 
 
% section introduction (end)

% \input{qm2pi.knotations} 

% section notation (end)

\input{qm2pi.process.calculi} 

% section concurrent_process_calculi_and_spatial_logics_ (end)
    
%\input{qm2pi.knots2pi} 

%\input{qm2pi.trefoil} 

%\input{qm2pi.mainthm} 

% subsection basic_interpretation (end)

%\input{qm2pi.rho.presentation} 
\subsection{The syntax and semantics of the notation system}\label{sub:the_syntax_and_semantics_of_the_notation_system} % (fold)

We now summarize a technical presentation of the calculus that
embodies our theory of dynamics. The typical presentation of such a
calculus follows the style of giving generators and relations on
them. The grammar, below, describing term constructors, freely
generates the set of processes, $\Proc$. This set is then quotiented
by a relation known as structural congruence and it is over this set
that the notion of dynamics is expressed. This presentation is
essentially that of \cite{MeredithR05} with the addition of
polyadicity and summation. For readability we have relegated some of
the technical subtleties to an appendix.

\subsubsection{Process grammar}\label{subsub:process_grammar}

\begin{mathpar}
  \inferrule* [lab=synchronization] {} {{M} \bc \pzero \;|\; x?F \;|\; x!C }
  \and
  \inferrule* [lab=abstraction] {} {{F} \bc (x)P}
  \and
  \inferrule* [lab=concretion] {} {{C} \bc \langle Q \rangle}
  \and
  \inferrule* [lab=process] {} {{P,Q} \bc M \;| \;P|Q \;|\; @{x}}
  \and
  \inferrule* [lab=name] {} {{x} \bc \quotep{P}}
\end{mathpar} 

Note that $\vec{x}$ (resp. $\vec{P}$) denotes a vector of names
(resp. processes) of length $|\vec{x}|$ (resp. $|\vec{P}|$). We adopt
the following useful abbreviations.

\begin{mathpar}
   x?(\vec{y}).P := x.(\vec{y})P \and  x\clift{\vec{P}} := x.\clift{\vec{P}}
   \and x!(y) := \lift{x}{\dropn{y}}
   \and \Pi_{i=0}^{n-1}P_i := P_0 | \ldots | P_{n-1}
\end{mathpar}

\subsubsection{Structural congruence}

\paragraph{Free and bound names and alpha-equivalence.} At the
core of structural equivalence is alpha-equivalence which identifies
process that are the same up to a change of variable. Formally, we
recognize the distinction between free and bound names. The free names
of a process, $\freenames{P}$, may be calculated recursively as
follows:

\begin{mathpar}
\freenames{\pzero} := \emptyset
  \and \\
  \freenames{x?(y).P} := \{ x \} \cup (\freenames{P} \setminus \{ y \})
  \and 
  \freenames{x!\langle P \rangle} := \{ x \} \cup \{ P \} 
  \and \\
  \freenames{P|Q} := \freenames{P} \cup \freenames{Q}
  \and \\
  \freenames{@{x}} := \{ x \}
\end{mathpar}

$\pi$
$\quotep{\pi}$

$\freenames{-} : \pi \to \mathcal{P}(\quotep{\pi})$

\begin{eqnarray*}
  \freenames{\pzero} & := & \emptyset \\
  \freenames{x?(y).P} & := & \{ x \} \cup (\freenames{P} \setminus \{ y \}) \\
  \freenames{x!\langle P \rangle} & := & \{ x \} \cup \{ P \} \\
  \freenames{P|Q} & := & \freenames{P} \cup \freenames{Q} \\
  \freenames{\dropn{x}} & := & \{ x \}
\end{eqnarray*}

The bound names of a process, $\boundnames{P}$, are those names occurring in $P$
that are not free. For example, in $x?(y).0$, the name $x$ is free, while $y$ is bound.

\begin{mathpar}
  \inferrule* [lab=monoidal-laws] {} { P|Q \equiv Q|P \and P|0 \equiv P \and P|(Q|R) \equiv (P|Q)|R }
\end{mathpar}

\begin{mathpar}
  \inferrule* [lab=alpha-equivalence] {} { (x)P \equiv (y)P\{y/x\} \and y \not\in \freenames{P} }
\end{mathpar}

\begin{definition}
Then two processes, $P,Q$, are alpha-equivalent if $P = Q\{\vec{y}/\vec{x}\}$ for
some $\vec{x} \in \boundnames{Q},\vec{y} \in \boundnames{P}$, where $Q\{\vec{y}/\vec{x}\}$
denotes the capture-avoiding substitution of $\vec{y}$ for $\vec{x}$ in $Q$.
\end{definition}

\begin{definition}
  The {\em structural congruence} \cite{SangiorgiWalker} , $\equiv$,
  between processes is the least congruence containing
  alpha-equivalence, satisfying the abelian monoid laws
  (associativity, commutativity and $\pzero$ as identity) for parallel
  composition $|$ and for summation $+$.
\end{definition}

\subsection{Name equivalence}

We take name equivalence, written $\nameeq$, to be the smallest
equivalence relation generated by the following rules.

\begin{mathpar}
\inferrule*[lab=Quote-drop]
{ }
{ \quotep{@{x}} \nameeq x }

\inferrule*[lab=Struct-equiv]
{ P \scong Q }
{ \quotep{P} \nameeq \quotep{Q} }
\end{mathpar}

The astute reader will have noticed that the mutual recursion of names
and processes imposes a mutual recursion on alpha-equivalence and
structural equivalence via name-equivalence. Fortunately, all of this
works out pleasantly and we may calculate in the natural way, free of
concern. The reader interested in the details is referred to the
appendix \ref{appendix:rho_details}.

\subsection{Substitution}

We use $\Proc$ for the set of processes, $\QProc$ for the set of
names, and $\id{\{}\vec{y} / \vec{x} \id{\}}$ to denote partial maps,
$s : \QProc \rightarrow \QProc$. A map, $s$ lifts, uniquely, to a map
on process terms, $\widehat{s} : \Proc \rightarrow \Proc$ by the
following equations.

\begin{mathpar}
  (0) \psubstp{Q}{P} := 0 \\
  (R \juxtap S) \psubstp{Q}{P}
  :=    
  (R)\psubstp{Q}{P} \juxtap (S) \psubstp{Q}{P} \\
  (x?(y).R) \psubstp{Q}{P}    
  :=    
  (x)\substp{Q}{P} (z)\concat( (R \psubstn{z}{y}) \psubstp{Q}{P} ) \\
  (\lift{x}{R}) \psubstp{Q}{P}  
  :=
  \lift{(x)\substp{Q}{P}}{ R \psubstp{Q}{P} } \\
%   (\dropn{x})  \psubstp{Q}{P}       
%   := 
%   \left\{ 
%     \begin{array}{ccc} 
%       \dropn{\quotep{Q}} & & x \nameeq \quotep{P} \\
%       \dropn{x} & & otherwise \\
%     \end{array}
%   \right. 
  (\dropn{x})  \psubstp{Q}{P}       
  := 
  \left\{ 
    \begin{array}{ccc} 
      Q & & x \nameeq \quotep{P} \\
      \dropn{x} & & otherwise \\
    \end{array}
  \right.
\end{mathpar}
 

where

\begin{eqnarray}
  (x)\id{\{} \lpquote Q \rpquote / \lpquote P \rpquote \id{\}}            = 
  \left\{ 
    \begin{array}{ccc}
      \lpquote Q \rpquote & & x \nameeq \lpquote P \rpquote \\
      x & & otherwise \\
    \end{array}
  \right. \nonumber
\end{eqnarray}

and $z$ is chosen distinct from $\quotep{P}$, $\quotep{Q}$, the free
names in $Q$, and all the names in $R$. Our $\alpha$-equivalence will
be built in the standard way from this substitution.

\begin{remark}\label{rem:no_self_referential_names}
  One consequence of these definitions is that $\forall P. \quotep{P}
  \not\in \freenames{P}$.
\end{remark}

\subsection{ Dynamic quote: an example }

Anticipating something of what's to come, consider applying the
substitution, $\widehat{\id{\{}u / z \id{\}}}$, to the following pair
of processes, $\lift{w}{y!(z)}$ and $w[ \lpquote y!(z) \rpquote ]$.

\begin{eqnarray}
	\lift{w}{y!(z)}\widehat{\id{\{}u / z \id{\}}}
		& = &
		\lift{w}{y!(u)} \nonumber\\
	w[ \lpquote y!(z) \rpquote ] \widehat{ \id{\{}u / z \id{\}} }
		& = &
		w[ \lpquote y!(z) \rpquote ] \nonumber
\end{eqnarray}

Because the body of the process between quotes is impervious to
substitution, we get radically different answers. In fact, by
examining the first process in an input context,
e.g. $x?(z).\lift{w}{y!(z)}$, we see that the process under the lift
operator may be shaped by prefixed inputs binding a name inside it. In
this sense, the lift operator will be seen as a way to dynamically
construct processes before reifying them as names.

Finally equipped with these standard features we can present the
dynamics of the calculus.

\subsubsection{Operational semantics} 

Finally, we introduce the computational dynamics. What marks these
algebras as distinct from other more traditionally studied algebraic
structures, e.g. vector spaces or polynomial rings, is the manner in
which dynamics is captured. In traditional structures, dynamics is typically
expressed through morphisms between such structures, as in linear maps
between vector spaces or morphisms between rings. In algebras
associated with the semantics of computation, the dynamics is
expressed as part of the algebraic structure itself, through a
reduction reduction relation typically denoted by $\red$. Below, we
give a recursive presentation of this relation for the calculus used
in the encoding.

$\red \subseteq \pi \times \pi$
$\red : \pi \to \mathcal{P}(\pi)$

\begin{mathpar}
  \inferrule* [lab=Comm] { \textsf{match}( x_{src}, x_{trgt} ) } { x_{trgt}?(y)P \; | \; x_{src}!\langle {Q} \rangle \red P\{\quotep{Q}/y}\} }
  \and \\
  \inferrule* [lab=Par] {{P} \red {P}'} {{{P} | {Q}} \red {{P}' | {Q}}}
  \and
  \inferrule* [lab=Equiv]{{{P} \scong {P}'} \andalso {{P}' \red {Q}'} \andalso {{Q}' \scong {Q}}}{{P} \red {Q}}
\end{mathpar}

\begin{eqnarray*}
  match_{\equiv} (\quotep{P},\quotep{Q}) & := & P \equiv Q \\
  match_{\dagger}(\quotep{P},\quotep{Q}) & := & \forall R. P|Q \red^{*} R => R \red^{*} 0 \\
  match_{K}(\quotep{P},\quotep{Q}) & := & K \mbox{ for some context } K
\end{eqnarray*}

$u?(x)P | u!\langle Q \rangle \red P\{\quotep{Q}/x\}$

%We write $\wred$ for $\red^*$, and $P\red$ if $\exists Q $ such that $ P \red Q$.
We write $P\red$ if $\exists Q $ such that $ P \red Q$ and $P\not\red$, otherwise.

\section{Replication}

As mentioned before, it is known that replication (and hence
recursion) can be implemented in a higher-order process algebra
\cite{SangiorgiWalker}. As our first example of calculation with the
machinery thus far presented we give the construction explicitly in
the {\rhoc}.

\begin{eqnarray}
	D_{x} & := & \prefix{x}{y}{(\binpar{\outputp{x}{y}}{@{y}})} \nonumber\\
	\bangp_{x}{P} & := & \binpar{{x}!\langle{\binpar{D_{x}}{P}}\rangle}{D_{x}} \nonumber
\end{eqnarray}

\begin{eqnarray}
	\bangp_{x}{P} & & \nonumber\\
	=
	& {x}!\langle{(\prefix{x}{y}{(\outputp{x}{y} | @{y})) | P}}\rangle 
	      | \prefix{x}{y}{(\outputp{x}{y} | @{y})} & \nonumber\\
	\red
	& (\outputp{x}{y} | @{y})\substn{\quotep{(\prefix{x}{y}{(@{y} | \outputp{x}{y})) | P}}}{y} & \nonumber\\
	=
	& \outputp{x}{\quotep{(\prefix{x}{y}{(\outputp{x}{y} | @{y})) | P}}}
	  | {(\prefix{x}{y}{(\outputp{x}{y} | @{y})) | P}} & \nonumber\\
	\red
	& \ldots & \nonumber\\
	\red^*
	& P | P | \ldots & \nonumber
\end{eqnarray}

Of course, this encoding, as an implementation, runs away, unfolding
$\bangp{P}$ eagerly. A lazier and more implementable replication
operator, restricted to input-guarded processes, may be obtained as follows.

\begin{eqnarray}
\bangp{\prefix{u}{v}{P}} 
	:= 
	\binpar{\lift{x}{\prefix{u}{v}{(\binpar{D(x)}{P})}}}{D(x)} \nonumber
\end{eqnarray}

\begin{remark}
  Note that the lazier definition still does not deal with summation
  or mixed summation (i.e. sums over input and output). The reader is
  invited to construct definitions of replication that deal with these
  features. 

  Further, the definitions are parameterized in a name, $x$. Can you,
  gentle reader, make a definition that eliminates this parameter and
  guarantees no accidental interaction between the replication
  machinery and the process being replicated -- i.e. no accidental
  sharing of names used by the process to get its work done and the
  name(s) used by the replication to effect copying. This latter
  revision of the definition of replication is crucial to obtaining
  the expected identity $!!P \sim !P$.
\end{remark}

\begin{remark}\label{rem:paradoxical_combinator}
  The reader familiar with the lambda calculus will have noticed the
  similarity between $D$ and the paradoxical combinator.

  [Ed. note: the existence of this seems to suggest we have to be more
  restrictive on the set of processes and names we admit if we are to
  support no-cloning.]
\end{remark}

\subsubsection{Bisimulation}

The computational dynamics gives rise to another kind of equivalence,
the equivalence of computational behavior. As previously mentioned
this is typically captured \emph{via} some form of bisimulation.

% The notion we use in this paper is weak barbed bisimulation
% \cite{milner91polyadicpi}.

The notion we use in this paper is derived from weak barbed
bisimulation \cite{milner91polyadicpi}. 

\begin{definition}
An \emph{observation relation}, $\downarrow_{\mathcal N}$, over a set
of names, $\mathcal N$, is the smallest relation satisfying the rules
below.

\infrule[Out-barb]{y \in {\mathcal N}, \; x \nameeq y}
		  {\outputp{x}{v} \downarrow_{\mathcal N} x}
\infrule[Par-barb]{\mbox{$P\downarrow_{\mathcal N} x$ or $Q\downarrow_{\mathcal N} x$}}
		  {\binpar{P}{Q} \downarrow_{\mathcal N} x}

We write $P \Downarrow_{\mathcal N} x$ if there is $Q$ such that 
$P \wred Q$ and $Q \downarrow_{\mathcal N} x$.
\end{definition}

\begin{definition}
%\label{def.bbisim}
An  ${\mathcal N}$-\emph{barbed bisimulation} over a set of names, ${\mathcal N}$, is a symmetric binary relation 
${\mathcal S}_{\mathcal N}$ between agents such that $P\rel{S}_{\mathcal N}Q$ implies:
\begin{enumerate}
\item If $P \red P'$ then $Q \wred Q'$ and $P'\rel{S}_{\mathcal N} Q'$.
\item If $P\downarrow_{\mathcal N} x$, then $Q\Downarrow_{\mathcal N} x$.
\end{enumerate}
$P$ is ${\mathcal N}$-barbed bisimilar to $Q$, written
$P \wbbisim_{\mathcal N} Q$, if $P \rel{S}_{\mathcal N} Q$ for some ${\mathcal N}$-barbed bisimulation ${\mathcal S}_{\mathcal N}$.
\end{definition}

$\mathcal{R} \subseteq \pi \times \pi$

$P \mathcal{R} Q => \forall P'. P \red P' \Rightarrow \exists Q'. Q \red Q', P' \mathcal{R} Q'$

$P \vdash x \Rightarrow Q \vdash x$

\begin{mathpar}
  \inferrule*[lab=Out-barb]{x \nameeq y}{{y}!\langle{Q}\rangle \vdash x}
  \and
  \inferrule*[lab=Par-barb]{\mbox{$P\vdash x$ or $Q\vdash x$}}{\binpar{P}{Q} \vdash x}
\end{mathpar}

\subsubsection{Contexts}

One of the principle advantages of computational calculi like the
$\pi$-calculus is a well-defined notion of context,
contextual-equivalence and a correlation between
contextual-equivalence and notions of bisimulation. The notion of
context allows the decomposition of a process into (sub-)process and
its syntactic environment, its context. Thus, a context may be
thought of as a process with a ``hole'' (written $\Box$) in it. The
application of a context $M$ to a process $P$, written $M[P]$, is
tantamount to filling the hole in $M$ with $P$. In this paper we do
not need the full weight of this theory, but do make use of the notion
of context in the proof the main theorem. 

\begin{mathpar}
  \inferrule* [lab=summation] {} {{M_{M},M_{N}} \bc \Box \;|\; x.M_{A} \;|\; M_{M}+M_{N}}
  \and
  \inferrule* [lab=agent] {} {{M_{A}} \bc (\vec{x})M_{P} \;| \; \clift{P_0,\ldots,M_{P},\ldots,P_N}}
  \and \\
  \inferrule* [lab=process] {} {{M_{P}} \bc M_{N} \;| \;P|M_{P} }
\end{mathpar} 

\begin{mathpar}
  \inferrule* [lab=sychronization] {} {M_{N} \bc \Box \;|\; x?M_{F} \;|\; x!M_{C}}
  \and
  \inferrule* [lab=abstraction] {} {{M_{F}} \bc (x)M_{P} }
  \and
  \inferrule* [lab=concretion] {} {{M_{C}} \bc \langle M_{P} \rangle }
  \and \\
  \inferrule* [lab=process] {} {{M_{P}} \bc M_{N} \;| \;P|M_{P} }
\end{mathpar}

\begin{definition}[contextual application] Given a context $M$, and
  process $P$, we define the \emph{contextual application}, $M[P] :=
  M\{P/\Box\}$. That is, the contextual application of M to P is the
  substitution of $P$ for $\Box$ in $M$.
\end{definition}

$\meaningof{-} : L \to \mathcal{P}(\pi)$

\begin{mathpar}
  \inferrule* [lab=collection] {} {\meaningof{true} = \pi, \and \meaningof{~E} = \pi \setminus \meaningof{E}, \and \meaningof{E_{1} \& E_{2}} = \meaningof{E_{1}} \cap \meaningof{E_{2}}}
\end{mathpar}

\begin{mathpar}
  \inferrule* [lab=structure] {} {\meaningof{0} = \{ P \in \pi | P \equiv 0 \}, \and \\ \meaningof{E_1 | E_2} = \{ P \in \pi | P \equiv P_{1} | P_{2}, P_{1} \in \meaningof{E_{1}}, P_{2} \in \meaningof{E_2}\} }
\end{mathpar}

\begin{mathpar}
 \inferrule* [lab=behavior] {} {\meaningof{\langle a?b \rangle E} = \{ P \in \pi | P \equiv Q | u?(y)P', \\ \and \\\\ \and \\ \;\;\; u \in \meaningof{a}, \forall z.P'\{z/y\} \in \meaningof{E\{z/b\}}\}, \and \\ \meaningof{a!E} = \{ P \in \pi | P \equiv Q | x!\langle P' \rangle, x \in \meaningof{a} P' \in \meaningof{E}\} }
\end{mathpar}

\begin{mathpar}
 \inferrule* [lab=nominal] {} {\meaningof{\quotep{E}} = \{ \quotep{P} \in \quotep{\pi} | P \in \meaningof{E} \}, \and \meaningof{\quotep{P}} = \{ \quotep{Q} \in \quotep{\pi} | P \equiv Q \} \and \\ \meaningof{@\quotep{E}} = \{ P \in \pi | P \equiv @x, x \in \meaningof{E} \}}
\end{mathpar}

\begin{eqnarray*}
  \\
  \meaningof{-} : TS \to ST
\end{eqnarray*}

\begin{eqnarray*}
  \\
  L : TS \to ST
\end{eqnarray*}

\begin{eqnarray*}
  \\
  P \models E \iff P \in \meaningof{E}
\end{eqnarray*}

\begin{eqnarray*}
  P \approx_{L} Q \iff \forall E \in L. P \models E \iff Q \models E
\end{eqnarray*}

\begin{eqnarray*}
  P \approx_{K} Q
\end{eqnarray*}

\begin{eqnarray*}
  P \approx Q
\end{eqnarray*}

$\approx_{K} = \approx = \approx_{L}$

\subsubsection{Contextual duality}

Note that contexts extend the quotation operation to a family of
operations from processes to names. Given a context, $M$, we can
define a \emph{nominal context}, $\quotep{M}$ by $\quotep{M}[P] :=
\quotep{M[P]}$. To foreshadow what is to come we observe that these
operations enjoy a duality with processes very much like the duality
between vectors and maps from vectors to scalars.

Further, because the calculus is essentially higher-order, we have a
correspondence between contexts and processes. More specifically,
given a name $x$ and a context $M$ we can construct $M^{*}_{x}$ such
that 

\begin{mathpar}
  M^{*}_{x} | \lift{x}{P} \red M[P]
\end{mathpar}

namely,

\begin{mathpar}
  M^{*}_{x} := x?(u).M[\dropn{u}]
\end{mathpar}

The dependence of $M^{*}_{x}$ on a name makes it an abstraction, 

\begin{mathpar}
  M^{*} := (x)x?(u).M[\dropn{u}]
\end{mathpar}

\subsection{Additional notation}

It will sometimes be convenient to denote the process a name
quotes. We already have the notation $x = \quotep{P}$, but it will be
convenient to introduce an alternate notation, $\procn{x}$, when we
want to emphasize the connection to the use of the name. Note that, by
virtue of name equivalence, $\quotep{\procn{x}} \nameeq x$; so, the
notation is consistent with previous definitions.

Further, because names have structure it is possible to effect
substitutions on the basis of that structure. This means we need to
upgrade our notation for substitutions, which we accomplish by
adapting comprehension notation. Thus,

\begin{mathpar}
  P\{ y / x : x \in S \}
\end{mathpar}

is interpreted to mean the process derived from P by replacing (in a
capture-avoiding manner) each occurrence of $x$ in $S$ by $y$. For example,

\begin{mathpar}
  P\{ \quotep{\procn{x}|\procn{x}} / x : x \in \freenames{P} \}
\end{mathpar}

will replace each (occurrence) of a free name $x$ in $P$ by
$\quotep{\procn{x}|\procn{x}}$.

Also, we will avail ourselves of the notation $x^{L}$ and $x^{R}$ to
denote injections of a name into disjoint copies of the name
space. There are numerous ways to accomplish this. One example can be
found in \cite{MeredithR05}. This notation overloads to vectors of
names: $\vec{x}^{\pi} := (x_{i}^{\pi} \; : \; 0 \leq i < |\vec{x}| )$ where $\pi \in \{L,R\}$.

We also use $P^{\Box} := P|\Box$.

In \cite{MeredithR05} an interpretation of the new operator is
given. It turns out that there are several possible interpretations
all enjoying the requisite algebraic properties of the operator (see
\cite{milner91polyadicpi}). We will therefore make liberal use of
$(\nu\; \vec{x})P$.

% subsection the_syntax_and_semantics_of_the_notation_system (end)   

\input{qm2pi.qmops} 

\input{qm2pi.sterngerlach} 

\input{qm2pi.metric} 

% section concurrent_process_calculi (end)

%\input{qm2pi.proofsketch}

% section proof sketch (end)

%\input{qm2pi.slviaknots} 

% section spatial logic via knots (end)

\input{qm2pi.conclusion}

% section conclusion (end)

%\input{qm2pi.dtcodes} 

% section wiring algorithm (end)

\input{qm2pi.ack} 

% section acknowledgments (end)

\newpage


\bibliographystyle{plain}   
\bibliography{../../biblios/main.bib}

\input{qm2pi.rhodetails}

\end{document}

 

% section notation (end)

\input{qm2pi.process.calculi} 

% section concurrent_process_calculi_and_spatial_logics_ (end)
    
%\documentclass[12pt]{llncs}
%\documentclass{jktr}

\usepackage[pdftex]{hyperref}                   
\usepackage {listings}
\usepackage {mathpartir}
\usepackage{bcprules}
%\usepackage{listings}
                       
\usepackage{graphicx} 
%\usepackage[margins=2.5cm,nohead,nofoot]{geometry}
%\usepackage{geometry}
\usepackage{amsfonts}
\usepackage{amstext}
\usepackage{latexsym}
\usepackage{amssymb}
\usepackage{color}


%\include{myPreamble}
\include{qm2pi.local} 

%\ifpdf
%\usepackage[pdftex]{graphicx}
%\else
%\usepackage{graphicx}
%\fi

 % \ifpdf
%  \usepackage{pdfsync}
%  \if


%\title{Brief Article}
%\author{David F. Snyder}
%\author{L.G. Meredith}

%\address{Dept. of Math., Texas State University--San Marcos, San Marcos, TX 78666}
       
\pagestyle{empty}


\begin{document}

\lstset{language=[Objective]Caml,frame=shadowbox}

\input{qm2pi.front}

% section front matter (end)

\input{qm2pi.intro} 
 
% section introduction (end)

% \input{qm2pi.knotations} 

% section notation (end)

\input{qm2pi.process.calculi} 

% section concurrent_process_calculi_and_spatial_logics_ (end)
    
%\input{qm2pi.knots2pi} 

%\input{qm2pi.trefoil} 

%\input{qm2pi.mainthm} 

% subsection basic_interpretation (end)

%\input{qm2pi.rho.presentation} 
\subsection{The syntax and semantics of the notation system}\label{sub:the_syntax_and_semantics_of_the_notation_system} % (fold)

We now summarize a technical presentation of the calculus that
embodies our theory of dynamics. The typical presentation of such a
calculus follows the style of giving generators and relations on
them. The grammar, below, describing term constructors, freely
generates the set of processes, $\Proc$. This set is then quotiented
by a relation known as structural congruence and it is over this set
that the notion of dynamics is expressed. This presentation is
essentially that of \cite{MeredithR05} with the addition of
polyadicity and summation. For readability we have relegated some of
the technical subtleties to an appendix.

\subsubsection{Process grammar}\label{subsub:process_grammar}

\begin{mathpar}
  \inferrule* [lab=synchronization] {} {{M} \bc \pzero \;|\; x?F \;|\; x!C }
  \and
  \inferrule* [lab=abstraction] {} {{F} \bc (x)P}
  \and
  \inferrule* [lab=concretion] {} {{C} \bc \langle Q \rangle}
  \and
  \inferrule* [lab=process] {} {{P,Q} \bc M \;| \;P|Q \;|\; @{x}}
  \and
  \inferrule* [lab=name] {} {{x} \bc \quotep{P}}
\end{mathpar} 

Note that $\vec{x}$ (resp. $\vec{P}$) denotes a vector of names
(resp. processes) of length $|\vec{x}|$ (resp. $|\vec{P}|$). We adopt
the following useful abbreviations.

\begin{mathpar}
   x?(\vec{y}).P := x.(\vec{y})P \and  x\clift{\vec{P}} := x.\clift{\vec{P}}
   \and x!(y) := \lift{x}{\dropn{y}}
   \and \Pi_{i=0}^{n-1}P_i := P_0 | \ldots | P_{n-1}
\end{mathpar}

\subsubsection{Structural congruence}

\paragraph{Free and bound names and alpha-equivalence.} At the
core of structural equivalence is alpha-equivalence which identifies
process that are the same up to a change of variable. Formally, we
recognize the distinction between free and bound names. The free names
of a process, $\freenames{P}$, may be calculated recursively as
follows:

\begin{mathpar}
\freenames{\pzero} := \emptyset
  \and \\
  \freenames{x?(y).P} := \{ x \} \cup (\freenames{P} \setminus \{ y \})
  \and 
  \freenames{x!\langle P \rangle} := \{ x \} \cup \{ P \} 
  \and \\
  \freenames{P|Q} := \freenames{P} \cup \freenames{Q}
  \and \\
  \freenames{@{x}} := \{ x \}
\end{mathpar}

$\pi$
$\quotep{\pi}$

$\freenames{-} : \pi \to \mathcal{P}(\quotep{\pi})$

\begin{eqnarray*}
  \freenames{\pzero} & := & \emptyset \\
  \freenames{x?(y).P} & := & \{ x \} \cup (\freenames{P} \setminus \{ y \}) \\
  \freenames{x!\langle P \rangle} & := & \{ x \} \cup \{ P \} \\
  \freenames{P|Q} & := & \freenames{P} \cup \freenames{Q} \\
  \freenames{\dropn{x}} & := & \{ x \}
\end{eqnarray*}

The bound names of a process, $\boundnames{P}$, are those names occurring in $P$
that are not free. For example, in $x?(y).0$, the name $x$ is free, while $y$ is bound.

\begin{mathpar}
  \inferrule* [lab=monoidal-laws] {} { P|Q \equiv Q|P \and P|0 \equiv P \and P|(Q|R) \equiv (P|Q)|R }
\end{mathpar}

\begin{mathpar}
  \inferrule* [lab=alpha-equivalence] {} { (x)P \equiv (y)P\{y/x\} \and y \not\in \freenames{P} }
\end{mathpar}

\begin{definition}
Then two processes, $P,Q$, are alpha-equivalent if $P = Q\{\vec{y}/\vec{x}\}$ for
some $\vec{x} \in \boundnames{Q},\vec{y} \in \boundnames{P}$, where $Q\{\vec{y}/\vec{x}\}$
denotes the capture-avoiding substitution of $\vec{y}$ for $\vec{x}$ in $Q$.
\end{definition}

\begin{definition}
  The {\em structural congruence} \cite{SangiorgiWalker} , $\equiv$,
  between processes is the least congruence containing
  alpha-equivalence, satisfying the abelian monoid laws
  (associativity, commutativity and $\pzero$ as identity) for parallel
  composition $|$ and for summation $+$.
\end{definition}

\subsection{Name equivalence}

We take name equivalence, written $\nameeq$, to be the smallest
equivalence relation generated by the following rules.

\begin{mathpar}
\inferrule*[lab=Quote-drop]
{ }
{ \quotep{@{x}} \nameeq x }

\inferrule*[lab=Struct-equiv]
{ P \scong Q }
{ \quotep{P} \nameeq \quotep{Q} }
\end{mathpar}

The astute reader will have noticed that the mutual recursion of names
and processes imposes a mutual recursion on alpha-equivalence and
structural equivalence via name-equivalence. Fortunately, all of this
works out pleasantly and we may calculate in the natural way, free of
concern. The reader interested in the details is referred to the
appendix \ref{appendix:rho_details}.

\subsection{Substitution}

We use $\Proc$ for the set of processes, $\QProc$ for the set of
names, and $\id{\{}\vec{y} / \vec{x} \id{\}}$ to denote partial maps,
$s : \QProc \rightarrow \QProc$. A map, $s$ lifts, uniquely, to a map
on process terms, $\widehat{s} : \Proc \rightarrow \Proc$ by the
following equations.

\begin{mathpar}
  (0) \psubstp{Q}{P} := 0 \\
  (R \juxtap S) \psubstp{Q}{P}
  :=    
  (R)\psubstp{Q}{P} \juxtap (S) \psubstp{Q}{P} \\
  (x?(y).R) \psubstp{Q}{P}    
  :=    
  (x)\substp{Q}{P} (z)\concat( (R \psubstn{z}{y}) \psubstp{Q}{P} ) \\
  (\lift{x}{R}) \psubstp{Q}{P}  
  :=
  \lift{(x)\substp{Q}{P}}{ R \psubstp{Q}{P} } \\
%   (\dropn{x})  \psubstp{Q}{P}       
%   := 
%   \left\{ 
%     \begin{array}{ccc} 
%       \dropn{\quotep{Q}} & & x \nameeq \quotep{P} \\
%       \dropn{x} & & otherwise \\
%     \end{array}
%   \right. 
  (\dropn{x})  \psubstp{Q}{P}       
  := 
  \left\{ 
    \begin{array}{ccc} 
      Q & & x \nameeq \quotep{P} \\
      \dropn{x} & & otherwise \\
    \end{array}
  \right.
\end{mathpar}
 

where

\begin{eqnarray}
  (x)\id{\{} \lpquote Q \rpquote / \lpquote P \rpquote \id{\}}            = 
  \left\{ 
    \begin{array}{ccc}
      \lpquote Q \rpquote & & x \nameeq \lpquote P \rpquote \\
      x & & otherwise \\
    \end{array}
  \right. \nonumber
\end{eqnarray}

and $z$ is chosen distinct from $\quotep{P}$, $\quotep{Q}$, the free
names in $Q$, and all the names in $R$. Our $\alpha$-equivalence will
be built in the standard way from this substitution.

\begin{remark}\label{rem:no_self_referential_names}
  One consequence of these definitions is that $\forall P. \quotep{P}
  \not\in \freenames{P}$.
\end{remark}

\subsection{ Dynamic quote: an example }

Anticipating something of what's to come, consider applying the
substitution, $\widehat{\id{\{}u / z \id{\}}}$, to the following pair
of processes, $\lift{w}{y!(z)}$ and $w[ \lpquote y!(z) \rpquote ]$.

\begin{eqnarray}
	\lift{w}{y!(z)}\widehat{\id{\{}u / z \id{\}}}
		& = &
		\lift{w}{y!(u)} \nonumber\\
	w[ \lpquote y!(z) \rpquote ] \widehat{ \id{\{}u / z \id{\}} }
		& = &
		w[ \lpquote y!(z) \rpquote ] \nonumber
\end{eqnarray}

Because the body of the process between quotes is impervious to
substitution, we get radically different answers. In fact, by
examining the first process in an input context,
e.g. $x?(z).\lift{w}{y!(z)}$, we see that the process under the lift
operator may be shaped by prefixed inputs binding a name inside it. In
this sense, the lift operator will be seen as a way to dynamically
construct processes before reifying them as names.

Finally equipped with these standard features we can present the
dynamics of the calculus.

\subsubsection{Operational semantics} 

Finally, we introduce the computational dynamics. What marks these
algebras as distinct from other more traditionally studied algebraic
structures, e.g. vector spaces or polynomial rings, is the manner in
which dynamics is captured. In traditional structures, dynamics is typically
expressed through morphisms between such structures, as in linear maps
between vector spaces or morphisms between rings. In algebras
associated with the semantics of computation, the dynamics is
expressed as part of the algebraic structure itself, through a
reduction reduction relation typically denoted by $\red$. Below, we
give a recursive presentation of this relation for the calculus used
in the encoding.

$\red \subseteq \pi \times \pi$
$\red : \pi \to \mathcal{P}(\pi)$

\begin{mathpar}
  \inferrule* [lab=Comm] { \textsf{match}( x_{src}, x_{trgt} ) } { x_{trgt}?(y)P \; | \; x_{src}!\langle {Q} \rangle \red P\{\quotep{Q}/y}\} }
  \and \\
  \inferrule* [lab=Par] {{P} \red {P}'} {{{P} | {Q}} \red {{P}' | {Q}}}
  \and
  \inferrule* [lab=Equiv]{{{P} \scong {P}'} \andalso {{P}' \red {Q}'} \andalso {{Q}' \scong {Q}}}{{P} \red {Q}}
\end{mathpar}

\begin{eqnarray*}
  match_{\equiv} (\quotep{P},\quotep{Q}) & := & P \equiv Q \\
  match_{\dagger}(\quotep{P},\quotep{Q}) & := & \forall R. P|Q \red^{*} R => R \red^{*} 0 \\
  match_{K}(\quotep{P},\quotep{Q}) & := & K \mbox{ for some context } K
\end{eqnarray*}

$u?(x)P | u!\langle Q \rangle \red P\{\quotep{Q}/x\}$

%We write $\wred$ for $\red^*$, and $P\red$ if $\exists Q $ such that $ P \red Q$.
We write $P\red$ if $\exists Q $ such that $ P \red Q$ and $P\not\red$, otherwise.

\section{Replication}

As mentioned before, it is known that replication (and hence
recursion) can be implemented in a higher-order process algebra
\cite{SangiorgiWalker}. As our first example of calculation with the
machinery thus far presented we give the construction explicitly in
the {\rhoc}.

\begin{eqnarray}
	D_{x} & := & \prefix{x}{y}{(\binpar{\outputp{x}{y}}{@{y}})} \nonumber\\
	\bangp_{x}{P} & := & \binpar{{x}!\langle{\binpar{D_{x}}{P}}\rangle}{D_{x}} \nonumber
\end{eqnarray}

\begin{eqnarray}
	\bangp_{x}{P} & & \nonumber\\
	=
	& {x}!\langle{(\prefix{x}{y}{(\outputp{x}{y} | @{y})) | P}}\rangle 
	      | \prefix{x}{y}{(\outputp{x}{y} | @{y})} & \nonumber\\
	\red
	& (\outputp{x}{y} | @{y})\substn{\quotep{(\prefix{x}{y}{(@{y} | \outputp{x}{y})) | P}}}{y} & \nonumber\\
	=
	& \outputp{x}{\quotep{(\prefix{x}{y}{(\outputp{x}{y} | @{y})) | P}}}
	  | {(\prefix{x}{y}{(\outputp{x}{y} | @{y})) | P}} & \nonumber\\
	\red
	& \ldots & \nonumber\\
	\red^*
	& P | P | \ldots & \nonumber
\end{eqnarray}

Of course, this encoding, as an implementation, runs away, unfolding
$\bangp{P}$ eagerly. A lazier and more implementable replication
operator, restricted to input-guarded processes, may be obtained as follows.

\begin{eqnarray}
\bangp{\prefix{u}{v}{P}} 
	:= 
	\binpar{\lift{x}{\prefix{u}{v}{(\binpar{D(x)}{P})}}}{D(x)} \nonumber
\end{eqnarray}

\begin{remark}
  Note that the lazier definition still does not deal with summation
  or mixed summation (i.e. sums over input and output). The reader is
  invited to construct definitions of replication that deal with these
  features. 

  Further, the definitions are parameterized in a name, $x$. Can you,
  gentle reader, make a definition that eliminates this parameter and
  guarantees no accidental interaction between the replication
  machinery and the process being replicated -- i.e. no accidental
  sharing of names used by the process to get its work done and the
  name(s) used by the replication to effect copying. This latter
  revision of the definition of replication is crucial to obtaining
  the expected identity $!!P \sim !P$.
\end{remark}

\begin{remark}\label{rem:paradoxical_combinator}
  The reader familiar with the lambda calculus will have noticed the
  similarity between $D$ and the paradoxical combinator.

  [Ed. note: the existence of this seems to suggest we have to be more
  restrictive on the set of processes and names we admit if we are to
  support no-cloning.]
\end{remark}

\subsubsection{Bisimulation}

The computational dynamics gives rise to another kind of equivalence,
the equivalence of computational behavior. As previously mentioned
this is typically captured \emph{via} some form of bisimulation.

% The notion we use in this paper is weak barbed bisimulation
% \cite{milner91polyadicpi}.

The notion we use in this paper is derived from weak barbed
bisimulation \cite{milner91polyadicpi}. 

\begin{definition}
An \emph{observation relation}, $\downarrow_{\mathcal N}$, over a set
of names, $\mathcal N$, is the smallest relation satisfying the rules
below.

\infrule[Out-barb]{y \in {\mathcal N}, \; x \nameeq y}
		  {\outputp{x}{v} \downarrow_{\mathcal N} x}
\infrule[Par-barb]{\mbox{$P\downarrow_{\mathcal N} x$ or $Q\downarrow_{\mathcal N} x$}}
		  {\binpar{P}{Q} \downarrow_{\mathcal N} x}

We write $P \Downarrow_{\mathcal N} x$ if there is $Q$ such that 
$P \wred Q$ and $Q \downarrow_{\mathcal N} x$.
\end{definition}

\begin{definition}
%\label{def.bbisim}
An  ${\mathcal N}$-\emph{barbed bisimulation} over a set of names, ${\mathcal N}$, is a symmetric binary relation 
${\mathcal S}_{\mathcal N}$ between agents such that $P\rel{S}_{\mathcal N}Q$ implies:
\begin{enumerate}
\item If $P \red P'$ then $Q \wred Q'$ and $P'\rel{S}_{\mathcal N} Q'$.
\item If $P\downarrow_{\mathcal N} x$, then $Q\Downarrow_{\mathcal N} x$.
\end{enumerate}
$P$ is ${\mathcal N}$-barbed bisimilar to $Q$, written
$P \wbbisim_{\mathcal N} Q$, if $P \rel{S}_{\mathcal N} Q$ for some ${\mathcal N}$-barbed bisimulation ${\mathcal S}_{\mathcal N}$.
\end{definition}

$\mathcal{R} \subseteq \pi \times \pi$

$P \mathcal{R} Q => \forall P'. P \red P' \Rightarrow \exists Q'. Q \red Q', P' \mathcal{R} Q'$

$P \vdash x \Rightarrow Q \vdash x$

\begin{mathpar}
  \inferrule*[lab=Out-barb]{x \nameeq y}{{y}!\langle{Q}\rangle \vdash x}
  \and
  \inferrule*[lab=Par-barb]{\mbox{$P\vdash x$ or $Q\vdash x$}}{\binpar{P}{Q} \vdash x}
\end{mathpar}

\subsubsection{Contexts}

One of the principle advantages of computational calculi like the
$\pi$-calculus is a well-defined notion of context,
contextual-equivalence and a correlation between
contextual-equivalence and notions of bisimulation. The notion of
context allows the decomposition of a process into (sub-)process and
its syntactic environment, its context. Thus, a context may be
thought of as a process with a ``hole'' (written $\Box$) in it. The
application of a context $M$ to a process $P$, written $M[P]$, is
tantamount to filling the hole in $M$ with $P$. In this paper we do
not need the full weight of this theory, but do make use of the notion
of context in the proof the main theorem. 

\begin{mathpar}
  \inferrule* [lab=summation] {} {{M_{M},M_{N}} \bc \Box \;|\; x.M_{A} \;|\; M_{M}+M_{N}}
  \and
  \inferrule* [lab=agent] {} {{M_{A}} \bc (\vec{x})M_{P} \;| \; \clift{P_0,\ldots,M_{P},\ldots,P_N}}
  \and \\
  \inferrule* [lab=process] {} {{M_{P}} \bc M_{N} \;| \;P|M_{P} }
\end{mathpar} 

\begin{mathpar}
  \inferrule* [lab=sychronization] {} {M_{N} \bc \Box \;|\; x?M_{F} \;|\; x!M_{C}}
  \and
  \inferrule* [lab=abstraction] {} {{M_{F}} \bc (x)M_{P} }
  \and
  \inferrule* [lab=concretion] {} {{M_{C}} \bc \langle M_{P} \rangle }
  \and \\
  \inferrule* [lab=process] {} {{M_{P}} \bc M_{N} \;| \;P|M_{P} }
\end{mathpar}

\begin{definition}[contextual application] Given a context $M$, and
  process $P$, we define the \emph{contextual application}, $M[P] :=
  M\{P/\Box\}$. That is, the contextual application of M to P is the
  substitution of $P$ for $\Box$ in $M$.
\end{definition}

$\meaningof{-} : L \to \mathcal{P}(\pi)$

\begin{mathpar}
  \inferrule* [lab=collection] {} {\meaningof{true} = \pi, \and \meaningof{~E} = \pi \setminus \meaningof{E}, \and \meaningof{E_{1} \& E_{2}} = \meaningof{E_{1}} \cap \meaningof{E_{2}}}
\end{mathpar}

\begin{mathpar}
  \inferrule* [lab=structure] {} {\meaningof{0} = \{ P \in \pi | P \equiv 0 \}, \and \\ \meaningof{E_1 | E_2} = \{ P \in \pi | P \equiv P_{1} | P_{2}, P_{1} \in \meaningof{E_{1}}, P_{2} \in \meaningof{E_2}\} }
\end{mathpar}

\begin{mathpar}
 \inferrule* [lab=behavior] {} {\meaningof{\langle a?b \rangle E} = \{ P \in \pi | P \equiv Q | u?(y)P', \\ \and \\\\ \and \\ \;\;\; u \in \meaningof{a}, \forall z.P'\{z/y\} \in \meaningof{E\{z/b\}}\}, \and \\ \meaningof{a!E} = \{ P \in \pi | P \equiv Q | x!\langle P' \rangle, x \in \meaningof{a} P' \in \meaningof{E}\} }
\end{mathpar}

\begin{mathpar}
 \inferrule* [lab=nominal] {} {\meaningof{\quotep{E}} = \{ \quotep{P} \in \quotep{\pi} | P \in \meaningof{E} \}, \and \meaningof{\quotep{P}} = \{ \quotep{Q} \in \quotep{\pi} | P \equiv Q \} \and \\ \meaningof{@\quotep{E}} = \{ P \in \pi | P \equiv @x, x \in \meaningof{E} \}}
\end{mathpar}

\begin{eqnarray*}
  \\
  \meaningof{-} : TS \to ST
\end{eqnarray*}

\begin{eqnarray*}
  \\
  L : TS \to ST
\end{eqnarray*}

\begin{eqnarray*}
  \\
  P \models E \iff P \in \meaningof{E}
\end{eqnarray*}

\begin{eqnarray*}
  P \approx_{L} Q \iff \forall E \in L. P \models E \iff Q \models E
\end{eqnarray*}

\begin{eqnarray*}
  P \approx_{K} Q
\end{eqnarray*}

\begin{eqnarray*}
  P \approx Q
\end{eqnarray*}

$\approx_{K} = \approx = \approx_{L}$

\subsubsection{Contextual duality}

Note that contexts extend the quotation operation to a family of
operations from processes to names. Given a context, $M$, we can
define a \emph{nominal context}, $\quotep{M}$ by $\quotep{M}[P] :=
\quotep{M[P]}$. To foreshadow what is to come we observe that these
operations enjoy a duality with processes very much like the duality
between vectors and maps from vectors to scalars.

Further, because the calculus is essentially higher-order, we have a
correspondence between contexts and processes. More specifically,
given a name $x$ and a context $M$ we can construct $M^{*}_{x}$ such
that 

\begin{mathpar}
  M^{*}_{x} | \lift{x}{P} \red M[P]
\end{mathpar}

namely,

\begin{mathpar}
  M^{*}_{x} := x?(u).M[\dropn{u}]
\end{mathpar}

The dependence of $M^{*}_{x}$ on a name makes it an abstraction, 

\begin{mathpar}
  M^{*} := (x)x?(u).M[\dropn{u}]
\end{mathpar}

\subsection{Additional notation}

It will sometimes be convenient to denote the process a name
quotes. We already have the notation $x = \quotep{P}$, but it will be
convenient to introduce an alternate notation, $\procn{x}$, when we
want to emphasize the connection to the use of the name. Note that, by
virtue of name equivalence, $\quotep{\procn{x}} \nameeq x$; so, the
notation is consistent with previous definitions.

Further, because names have structure it is possible to effect
substitutions on the basis of that structure. This means we need to
upgrade our notation for substitutions, which we accomplish by
adapting comprehension notation. Thus,

\begin{mathpar}
  P\{ y / x : x \in S \}
\end{mathpar}

is interpreted to mean the process derived from P by replacing (in a
capture-avoiding manner) each occurrence of $x$ in $S$ by $y$. For example,

\begin{mathpar}
  P\{ \quotep{\procn{x}|\procn{x}} / x : x \in \freenames{P} \}
\end{mathpar}

will replace each (occurrence) of a free name $x$ in $P$ by
$\quotep{\procn{x}|\procn{x}}$.

Also, we will avail ourselves of the notation $x^{L}$ and $x^{R}$ to
denote injections of a name into disjoint copies of the name
space. There are numerous ways to accomplish this. One example can be
found in \cite{MeredithR05}. This notation overloads to vectors of
names: $\vec{x}^{\pi} := (x_{i}^{\pi} \; : \; 0 \leq i < |\vec{x}| )$ where $\pi \in \{L,R\}$.

We also use $P^{\Box} := P|\Box$.

In \cite{MeredithR05} an interpretation of the new operator is
given. It turns out that there are several possible interpretations
all enjoying the requisite algebraic properties of the operator (see
\cite{milner91polyadicpi}). We will therefore make liberal use of
$(\nu\; \vec{x})P$.

% subsection the_syntax_and_semantics_of_the_notation_system (end)   

\input{qm2pi.qmops} 

\input{qm2pi.sterngerlach} 

\input{qm2pi.metric} 

% section concurrent_process_calculi (end)

%\input{qm2pi.proofsketch}

% section proof sketch (end)

%\input{qm2pi.slviaknots} 

% section spatial logic via knots (end)

\input{qm2pi.conclusion}

% section conclusion (end)

%\input{qm2pi.dtcodes} 

% section wiring algorithm (end)

\input{qm2pi.ack} 

% section acknowledgments (end)

\newpage


\bibliographystyle{plain}   
\bibliography{../../biblios/main.bib}

\input{qm2pi.rhodetails}

\end{document}

 

%\documentclass[12pt]{llncs}
%\documentclass{jktr}

\usepackage[pdftex]{hyperref}                   
\usepackage {listings}
\usepackage {mathpartir}
\usepackage{bcprules}
%\usepackage{listings}
                       
\usepackage{graphicx} 
%\usepackage[margins=2.5cm,nohead,nofoot]{geometry}
%\usepackage{geometry}
\usepackage{amsfonts}
\usepackage{amstext}
\usepackage{latexsym}
\usepackage{amssymb}
\usepackage{color}


%\include{myPreamble}
\include{qm2pi.local} 

%\ifpdf
%\usepackage[pdftex]{graphicx}
%\else
%\usepackage{graphicx}
%\fi

 % \ifpdf
%  \usepackage{pdfsync}
%  \if


%\title{Brief Article}
%\author{David F. Snyder}
%\author{L.G. Meredith}

%\address{Dept. of Math., Texas State University--San Marcos, San Marcos, TX 78666}
       
\pagestyle{empty}


\begin{document}

\lstset{language=[Objective]Caml,frame=shadowbox}

\input{qm2pi.front}

% section front matter (end)

\input{qm2pi.intro} 
 
% section introduction (end)

% \input{qm2pi.knotations} 

% section notation (end)

\input{qm2pi.process.calculi} 

% section concurrent_process_calculi_and_spatial_logics_ (end)
    
%\input{qm2pi.knots2pi} 

%\input{qm2pi.trefoil} 

%\input{qm2pi.mainthm} 

% subsection basic_interpretation (end)

%\input{qm2pi.rho.presentation} 
\subsection{The syntax and semantics of the notation system}\label{sub:the_syntax_and_semantics_of_the_notation_system} % (fold)

We now summarize a technical presentation of the calculus that
embodies our theory of dynamics. The typical presentation of such a
calculus follows the style of giving generators and relations on
them. The grammar, below, describing term constructors, freely
generates the set of processes, $\Proc$. This set is then quotiented
by a relation known as structural congruence and it is over this set
that the notion of dynamics is expressed. This presentation is
essentially that of \cite{MeredithR05} with the addition of
polyadicity and summation. For readability we have relegated some of
the technical subtleties to an appendix.

\subsubsection{Process grammar}\label{subsub:process_grammar}

\begin{mathpar}
  \inferrule* [lab=synchronization] {} {{M} \bc \pzero \;|\; x?F \;|\; x!C }
  \and
  \inferrule* [lab=abstraction] {} {{F} \bc (x)P}
  \and
  \inferrule* [lab=concretion] {} {{C} \bc \langle Q \rangle}
  \and
  \inferrule* [lab=process] {} {{P,Q} \bc M \;| \;P|Q \;|\; @{x}}
  \and
  \inferrule* [lab=name] {} {{x} \bc \quotep{P}}
\end{mathpar} 

Note that $\vec{x}$ (resp. $\vec{P}$) denotes a vector of names
(resp. processes) of length $|\vec{x}|$ (resp. $|\vec{P}|$). We adopt
the following useful abbreviations.

\begin{mathpar}
   x?(\vec{y}).P := x.(\vec{y})P \and  x\clift{\vec{P}} := x.\clift{\vec{P}}
   \and x!(y) := \lift{x}{\dropn{y}}
   \and \Pi_{i=0}^{n-1}P_i := P_0 | \ldots | P_{n-1}
\end{mathpar}

\subsubsection{Structural congruence}

\paragraph{Free and bound names and alpha-equivalence.} At the
core of structural equivalence is alpha-equivalence which identifies
process that are the same up to a change of variable. Formally, we
recognize the distinction between free and bound names. The free names
of a process, $\freenames{P}$, may be calculated recursively as
follows:

\begin{mathpar}
\freenames{\pzero} := \emptyset
  \and \\
  \freenames{x?(y).P} := \{ x \} \cup (\freenames{P} \setminus \{ y \})
  \and 
  \freenames{x!\langle P \rangle} := \{ x \} \cup \{ P \} 
  \and \\
  \freenames{P|Q} := \freenames{P} \cup \freenames{Q}
  \and \\
  \freenames{@{x}} := \{ x \}
\end{mathpar}

$\pi$
$\quotep{\pi}$

$\freenames{-} : \pi \to \mathcal{P}(\quotep{\pi})$

\begin{eqnarray*}
  \freenames{\pzero} & := & \emptyset \\
  \freenames{x?(y).P} & := & \{ x \} \cup (\freenames{P} \setminus \{ y \}) \\
  \freenames{x!\langle P \rangle} & := & \{ x \} \cup \{ P \} \\
  \freenames{P|Q} & := & \freenames{P} \cup \freenames{Q} \\
  \freenames{\dropn{x}} & := & \{ x \}
\end{eqnarray*}

The bound names of a process, $\boundnames{P}$, are those names occurring in $P$
that are not free. For example, in $x?(y).0$, the name $x$ is free, while $y$ is bound.

\begin{mathpar}
  \inferrule* [lab=monoidal-laws] {} { P|Q \equiv Q|P \and P|0 \equiv P \and P|(Q|R) \equiv (P|Q)|R }
\end{mathpar}

\begin{mathpar}
  \inferrule* [lab=alpha-equivalence] {} { (x)P \equiv (y)P\{y/x\} \and y \not\in \freenames{P} }
\end{mathpar}

\begin{definition}
Then two processes, $P,Q$, are alpha-equivalent if $P = Q\{\vec{y}/\vec{x}\}$ for
some $\vec{x} \in \boundnames{Q},\vec{y} \in \boundnames{P}$, where $Q\{\vec{y}/\vec{x}\}$
denotes the capture-avoiding substitution of $\vec{y}$ for $\vec{x}$ in $Q$.
\end{definition}

\begin{definition}
  The {\em structural congruence} \cite{SangiorgiWalker} , $\equiv$,
  between processes is the least congruence containing
  alpha-equivalence, satisfying the abelian monoid laws
  (associativity, commutativity and $\pzero$ as identity) for parallel
  composition $|$ and for summation $+$.
\end{definition}

\subsection{Name equivalence}

We take name equivalence, written $\nameeq$, to be the smallest
equivalence relation generated by the following rules.

\begin{mathpar}
\inferrule*[lab=Quote-drop]
{ }
{ \quotep{@{x}} \nameeq x }

\inferrule*[lab=Struct-equiv]
{ P \scong Q }
{ \quotep{P} \nameeq \quotep{Q} }
\end{mathpar}

The astute reader will have noticed that the mutual recursion of names
and processes imposes a mutual recursion on alpha-equivalence and
structural equivalence via name-equivalence. Fortunately, all of this
works out pleasantly and we may calculate in the natural way, free of
concern. The reader interested in the details is referred to the
appendix \ref{appendix:rho_details}.

\subsection{Substitution}

We use $\Proc$ for the set of processes, $\QProc$ for the set of
names, and $\id{\{}\vec{y} / \vec{x} \id{\}}$ to denote partial maps,
$s : \QProc \rightarrow \QProc$. A map, $s$ lifts, uniquely, to a map
on process terms, $\widehat{s} : \Proc \rightarrow \Proc$ by the
following equations.

\begin{mathpar}
  (0) \psubstp{Q}{P} := 0 \\
  (R \juxtap S) \psubstp{Q}{P}
  :=    
  (R)\psubstp{Q}{P} \juxtap (S) \psubstp{Q}{P} \\
  (x?(y).R) \psubstp{Q}{P}    
  :=    
  (x)\substp{Q}{P} (z)\concat( (R \psubstn{z}{y}) \psubstp{Q}{P} ) \\
  (\lift{x}{R}) \psubstp{Q}{P}  
  :=
  \lift{(x)\substp{Q}{P}}{ R \psubstp{Q}{P} } \\
%   (\dropn{x})  \psubstp{Q}{P}       
%   := 
%   \left\{ 
%     \begin{array}{ccc} 
%       \dropn{\quotep{Q}} & & x \nameeq \quotep{P} \\
%       \dropn{x} & & otherwise \\
%     \end{array}
%   \right. 
  (\dropn{x})  \psubstp{Q}{P}       
  := 
  \left\{ 
    \begin{array}{ccc} 
      Q & & x \nameeq \quotep{P} \\
      \dropn{x} & & otherwise \\
    \end{array}
  \right.
\end{mathpar}
 

where

\begin{eqnarray}
  (x)\id{\{} \lpquote Q \rpquote / \lpquote P \rpquote \id{\}}            = 
  \left\{ 
    \begin{array}{ccc}
      \lpquote Q \rpquote & & x \nameeq \lpquote P \rpquote \\
      x & & otherwise \\
    \end{array}
  \right. \nonumber
\end{eqnarray}

and $z$ is chosen distinct from $\quotep{P}$, $\quotep{Q}$, the free
names in $Q$, and all the names in $R$. Our $\alpha$-equivalence will
be built in the standard way from this substitution.

\begin{remark}\label{rem:no_self_referential_names}
  One consequence of these definitions is that $\forall P. \quotep{P}
  \not\in \freenames{P}$.
\end{remark}

\subsection{ Dynamic quote: an example }

Anticipating something of what's to come, consider applying the
substitution, $\widehat{\id{\{}u / z \id{\}}}$, to the following pair
of processes, $\lift{w}{y!(z)}$ and $w[ \lpquote y!(z) \rpquote ]$.

\begin{eqnarray}
	\lift{w}{y!(z)}\widehat{\id{\{}u / z \id{\}}}
		& = &
		\lift{w}{y!(u)} \nonumber\\
	w[ \lpquote y!(z) \rpquote ] \widehat{ \id{\{}u / z \id{\}} }
		& = &
		w[ \lpquote y!(z) \rpquote ] \nonumber
\end{eqnarray}

Because the body of the process between quotes is impervious to
substitution, we get radically different answers. In fact, by
examining the first process in an input context,
e.g. $x?(z).\lift{w}{y!(z)}$, we see that the process under the lift
operator may be shaped by prefixed inputs binding a name inside it. In
this sense, the lift operator will be seen as a way to dynamically
construct processes before reifying them as names.

Finally equipped with these standard features we can present the
dynamics of the calculus.

\subsubsection{Operational semantics} 

Finally, we introduce the computational dynamics. What marks these
algebras as distinct from other more traditionally studied algebraic
structures, e.g. vector spaces or polynomial rings, is the manner in
which dynamics is captured. In traditional structures, dynamics is typically
expressed through morphisms between such structures, as in linear maps
between vector spaces or morphisms between rings. In algebras
associated with the semantics of computation, the dynamics is
expressed as part of the algebraic structure itself, through a
reduction reduction relation typically denoted by $\red$. Below, we
give a recursive presentation of this relation for the calculus used
in the encoding.

$\red \subseteq \pi \times \pi$
$\red : \pi \to \mathcal{P}(\pi)$

\begin{mathpar}
  \inferrule* [lab=Comm] { \textsf{match}( x_{src}, x_{trgt} ) } { x_{trgt}?(y)P \; | \; x_{src}!\langle {Q} \rangle \red P\{\quotep{Q}/y}\} }
  \and \\
  \inferrule* [lab=Par] {{P} \red {P}'} {{{P} | {Q}} \red {{P}' | {Q}}}
  \and
  \inferrule* [lab=Equiv]{{{P} \scong {P}'} \andalso {{P}' \red {Q}'} \andalso {{Q}' \scong {Q}}}{{P} \red {Q}}
\end{mathpar}

\begin{eqnarray*}
  match_{\equiv} (\quotep{P},\quotep{Q}) & := & P \equiv Q \\
  match_{\dagger}(\quotep{P},\quotep{Q}) & := & \forall R. P|Q \red^{*} R => R \red^{*} 0 \\
  match_{K}(\quotep{P},\quotep{Q}) & := & K \mbox{ for some context } K
\end{eqnarray*}

$u?(x)P | u!\langle Q \rangle \red P\{\quotep{Q}/x\}$

%We write $\wred$ for $\red^*$, and $P\red$ if $\exists Q $ such that $ P \red Q$.
We write $P\red$ if $\exists Q $ such that $ P \red Q$ and $P\not\red$, otherwise.

\section{Replication}

As mentioned before, it is known that replication (and hence
recursion) can be implemented in a higher-order process algebra
\cite{SangiorgiWalker}. As our first example of calculation with the
machinery thus far presented we give the construction explicitly in
the {\rhoc}.

\begin{eqnarray}
	D_{x} & := & \prefix{x}{y}{(\binpar{\outputp{x}{y}}{@{y}})} \nonumber\\
	\bangp_{x}{P} & := & \binpar{{x}!\langle{\binpar{D_{x}}{P}}\rangle}{D_{x}} \nonumber
\end{eqnarray}

\begin{eqnarray}
	\bangp_{x}{P} & & \nonumber\\
	=
	& {x}!\langle{(\prefix{x}{y}{(\outputp{x}{y} | @{y})) | P}}\rangle 
	      | \prefix{x}{y}{(\outputp{x}{y} | @{y})} & \nonumber\\
	\red
	& (\outputp{x}{y} | @{y})\substn{\quotep{(\prefix{x}{y}{(@{y} | \outputp{x}{y})) | P}}}{y} & \nonumber\\
	=
	& \outputp{x}{\quotep{(\prefix{x}{y}{(\outputp{x}{y} | @{y})) | P}}}
	  | {(\prefix{x}{y}{(\outputp{x}{y} | @{y})) | P}} & \nonumber\\
	\red
	& \ldots & \nonumber\\
	\red^*
	& P | P | \ldots & \nonumber
\end{eqnarray}

Of course, this encoding, as an implementation, runs away, unfolding
$\bangp{P}$ eagerly. A lazier and more implementable replication
operator, restricted to input-guarded processes, may be obtained as follows.

\begin{eqnarray}
\bangp{\prefix{u}{v}{P}} 
	:= 
	\binpar{\lift{x}{\prefix{u}{v}{(\binpar{D(x)}{P})}}}{D(x)} \nonumber
\end{eqnarray}

\begin{remark}
  Note that the lazier definition still does not deal with summation
  or mixed summation (i.e. sums over input and output). The reader is
  invited to construct definitions of replication that deal with these
  features. 

  Further, the definitions are parameterized in a name, $x$. Can you,
  gentle reader, make a definition that eliminates this parameter and
  guarantees no accidental interaction between the replication
  machinery and the process being replicated -- i.e. no accidental
  sharing of names used by the process to get its work done and the
  name(s) used by the replication to effect copying. This latter
  revision of the definition of replication is crucial to obtaining
  the expected identity $!!P \sim !P$.
\end{remark}

\begin{remark}\label{rem:paradoxical_combinator}
  The reader familiar with the lambda calculus will have noticed the
  similarity between $D$ and the paradoxical combinator.

  [Ed. note: the existence of this seems to suggest we have to be more
  restrictive on the set of processes and names we admit if we are to
  support no-cloning.]
\end{remark}

\subsubsection{Bisimulation}

The computational dynamics gives rise to another kind of equivalence,
the equivalence of computational behavior. As previously mentioned
this is typically captured \emph{via} some form of bisimulation.

% The notion we use in this paper is weak barbed bisimulation
% \cite{milner91polyadicpi}.

The notion we use in this paper is derived from weak barbed
bisimulation \cite{milner91polyadicpi}. 

\begin{definition}
An \emph{observation relation}, $\downarrow_{\mathcal N}$, over a set
of names, $\mathcal N$, is the smallest relation satisfying the rules
below.

\infrule[Out-barb]{y \in {\mathcal N}, \; x \nameeq y}
		  {\outputp{x}{v} \downarrow_{\mathcal N} x}
\infrule[Par-barb]{\mbox{$P\downarrow_{\mathcal N} x$ or $Q\downarrow_{\mathcal N} x$}}
		  {\binpar{P}{Q} \downarrow_{\mathcal N} x}

We write $P \Downarrow_{\mathcal N} x$ if there is $Q$ such that 
$P \wred Q$ and $Q \downarrow_{\mathcal N} x$.
\end{definition}

\begin{definition}
%\label{def.bbisim}
An  ${\mathcal N}$-\emph{barbed bisimulation} over a set of names, ${\mathcal N}$, is a symmetric binary relation 
${\mathcal S}_{\mathcal N}$ between agents such that $P\rel{S}_{\mathcal N}Q$ implies:
\begin{enumerate}
\item If $P \red P'$ then $Q \wred Q'$ and $P'\rel{S}_{\mathcal N} Q'$.
\item If $P\downarrow_{\mathcal N} x$, then $Q\Downarrow_{\mathcal N} x$.
\end{enumerate}
$P$ is ${\mathcal N}$-barbed bisimilar to $Q$, written
$P \wbbisim_{\mathcal N} Q$, if $P \rel{S}_{\mathcal N} Q$ for some ${\mathcal N}$-barbed bisimulation ${\mathcal S}_{\mathcal N}$.
\end{definition}

$\mathcal{R} \subseteq \pi \times \pi$

$P \mathcal{R} Q => \forall P'. P \red P' \Rightarrow \exists Q'. Q \red Q', P' \mathcal{R} Q'$

$P \vdash x \Rightarrow Q \vdash x$

\begin{mathpar}
  \inferrule*[lab=Out-barb]{x \nameeq y}{{y}!\langle{Q}\rangle \vdash x}
  \and
  \inferrule*[lab=Par-barb]{\mbox{$P\vdash x$ or $Q\vdash x$}}{\binpar{P}{Q} \vdash x}
\end{mathpar}

\subsubsection{Contexts}

One of the principle advantages of computational calculi like the
$\pi$-calculus is a well-defined notion of context,
contextual-equivalence and a correlation between
contextual-equivalence and notions of bisimulation. The notion of
context allows the decomposition of a process into (sub-)process and
its syntactic environment, its context. Thus, a context may be
thought of as a process with a ``hole'' (written $\Box$) in it. The
application of a context $M$ to a process $P$, written $M[P]$, is
tantamount to filling the hole in $M$ with $P$. In this paper we do
not need the full weight of this theory, but do make use of the notion
of context in the proof the main theorem. 

\begin{mathpar}
  \inferrule* [lab=summation] {} {{M_{M},M_{N}} \bc \Box \;|\; x.M_{A} \;|\; M_{M}+M_{N}}
  \and
  \inferrule* [lab=agent] {} {{M_{A}} \bc (\vec{x})M_{P} \;| \; \clift{P_0,\ldots,M_{P},\ldots,P_N}}
  \and \\
  \inferrule* [lab=process] {} {{M_{P}} \bc M_{N} \;| \;P|M_{P} }
\end{mathpar} 

\begin{mathpar}
  \inferrule* [lab=sychronization] {} {M_{N} \bc \Box \;|\; x?M_{F} \;|\; x!M_{C}}
  \and
  \inferrule* [lab=abstraction] {} {{M_{F}} \bc (x)M_{P} }
  \and
  \inferrule* [lab=concretion] {} {{M_{C}} \bc \langle M_{P} \rangle }
  \and \\
  \inferrule* [lab=process] {} {{M_{P}} \bc M_{N} \;| \;P|M_{P} }
\end{mathpar}

\begin{definition}[contextual application] Given a context $M$, and
  process $P$, we define the \emph{contextual application}, $M[P] :=
  M\{P/\Box\}$. That is, the contextual application of M to P is the
  substitution of $P$ for $\Box$ in $M$.
\end{definition}

$\meaningof{-} : L \to \mathcal{P}(\pi)$

\begin{mathpar}
  \inferrule* [lab=collection] {} {\meaningof{true} = \pi, \and \meaningof{~E} = \pi \setminus \meaningof{E}, \and \meaningof{E_{1} \& E_{2}} = \meaningof{E_{1}} \cap \meaningof{E_{2}}}
\end{mathpar}

\begin{mathpar}
  \inferrule* [lab=structure] {} {\meaningof{0} = \{ P \in \pi | P \equiv 0 \}, \and \\ \meaningof{E_1 | E_2} = \{ P \in \pi | P \equiv P_{1} | P_{2}, P_{1} \in \meaningof{E_{1}}, P_{2} \in \meaningof{E_2}\} }
\end{mathpar}

\begin{mathpar}
 \inferrule* [lab=behavior] {} {\meaningof{\langle a?b \rangle E} = \{ P \in \pi | P \equiv Q | u?(y)P', \\ \and \\\\ \and \\ \;\;\; u \in \meaningof{a}, \forall z.P'\{z/y\} \in \meaningof{E\{z/b\}}\}, \and \\ \meaningof{a!E} = \{ P \in \pi | P \equiv Q | x!\langle P' \rangle, x \in \meaningof{a} P' \in \meaningof{E}\} }
\end{mathpar}

\begin{mathpar}
 \inferrule* [lab=nominal] {} {\meaningof{\quotep{E}} = \{ \quotep{P} \in \quotep{\pi} | P \in \meaningof{E} \}, \and \meaningof{\quotep{P}} = \{ \quotep{Q} \in \quotep{\pi} | P \equiv Q \} \and \\ \meaningof{@\quotep{E}} = \{ P \in \pi | P \equiv @x, x \in \meaningof{E} \}}
\end{mathpar}

\begin{eqnarray*}
  \\
  \meaningof{-} : TS \to ST
\end{eqnarray*}

\begin{eqnarray*}
  \\
  L : TS \to ST
\end{eqnarray*}

\begin{eqnarray*}
  \\
  P \models E \iff P \in \meaningof{E}
\end{eqnarray*}

\begin{eqnarray*}
  P \approx_{L} Q \iff \forall E \in L. P \models E \iff Q \models E
\end{eqnarray*}

\begin{eqnarray*}
  P \approx_{K} Q
\end{eqnarray*}

\begin{eqnarray*}
  P \approx Q
\end{eqnarray*}

$\approx_{K} = \approx = \approx_{L}$

\subsubsection{Contextual duality}

Note that contexts extend the quotation operation to a family of
operations from processes to names. Given a context, $M$, we can
define a \emph{nominal context}, $\quotep{M}$ by $\quotep{M}[P] :=
\quotep{M[P]}$. To foreshadow what is to come we observe that these
operations enjoy a duality with processes very much like the duality
between vectors and maps from vectors to scalars.

Further, because the calculus is essentially higher-order, we have a
correspondence between contexts and processes. More specifically,
given a name $x$ and a context $M$ we can construct $M^{*}_{x}$ such
that 

\begin{mathpar}
  M^{*}_{x} | \lift{x}{P} \red M[P]
\end{mathpar}

namely,

\begin{mathpar}
  M^{*}_{x} := x?(u).M[\dropn{u}]
\end{mathpar}

The dependence of $M^{*}_{x}$ on a name makes it an abstraction, 

\begin{mathpar}
  M^{*} := (x)x?(u).M[\dropn{u}]
\end{mathpar}

\subsection{Additional notation}

It will sometimes be convenient to denote the process a name
quotes. We already have the notation $x = \quotep{P}$, but it will be
convenient to introduce an alternate notation, $\procn{x}$, when we
want to emphasize the connection to the use of the name. Note that, by
virtue of name equivalence, $\quotep{\procn{x}} \nameeq x$; so, the
notation is consistent with previous definitions.

Further, because names have structure it is possible to effect
substitutions on the basis of that structure. This means we need to
upgrade our notation for substitutions, which we accomplish by
adapting comprehension notation. Thus,

\begin{mathpar}
  P\{ y / x : x \in S \}
\end{mathpar}

is interpreted to mean the process derived from P by replacing (in a
capture-avoiding manner) each occurrence of $x$ in $S$ by $y$. For example,

\begin{mathpar}
  P\{ \quotep{\procn{x}|\procn{x}} / x : x \in \freenames{P} \}
\end{mathpar}

will replace each (occurrence) of a free name $x$ in $P$ by
$\quotep{\procn{x}|\procn{x}}$.

Also, we will avail ourselves of the notation $x^{L}$ and $x^{R}$ to
denote injections of a name into disjoint copies of the name
space. There are numerous ways to accomplish this. One example can be
found in \cite{MeredithR05}. This notation overloads to vectors of
names: $\vec{x}^{\pi} := (x_{i}^{\pi} \; : \; 0 \leq i < |\vec{x}| )$ where $\pi \in \{L,R\}$.

We also use $P^{\Box} := P|\Box$.

In \cite{MeredithR05} an interpretation of the new operator is
given. It turns out that there are several possible interpretations
all enjoying the requisite algebraic properties of the operator (see
\cite{milner91polyadicpi}). We will therefore make liberal use of
$(\nu\; \vec{x})P$.

% subsection the_syntax_and_semantics_of_the_notation_system (end)   

\input{qm2pi.qmops} 

\input{qm2pi.sterngerlach} 

\input{qm2pi.metric} 

% section concurrent_process_calculi (end)

%\input{qm2pi.proofsketch}

% section proof sketch (end)

%\input{qm2pi.slviaknots} 

% section spatial logic via knots (end)

\input{qm2pi.conclusion}

% section conclusion (end)

%\input{qm2pi.dtcodes} 

% section wiring algorithm (end)

\input{qm2pi.ack} 

% section acknowledgments (end)

\newpage


\bibliographystyle{plain}   
\bibliography{../../biblios/main.bib}

\input{qm2pi.rhodetails}

\end{document}

 

%\documentclass[12pt]{llncs}
%\documentclass{jktr}

\usepackage[pdftex]{hyperref}                   
\usepackage {listings}
\usepackage {mathpartir}
\usepackage{bcprules}
%\usepackage{listings}
                       
\usepackage{graphicx} 
%\usepackage[margins=2.5cm,nohead,nofoot]{geometry}
%\usepackage{geometry}
\usepackage{amsfonts}
\usepackage{amstext}
\usepackage{latexsym}
\usepackage{amssymb}
\usepackage{color}


%\include{myPreamble}
\include{qm2pi.local} 

%\ifpdf
%\usepackage[pdftex]{graphicx}
%\else
%\usepackage{graphicx}
%\fi

 % \ifpdf
%  \usepackage{pdfsync}
%  \if


%\title{Brief Article}
%\author{David F. Snyder}
%\author{L.G. Meredith}

%\address{Dept. of Math., Texas State University--San Marcos, San Marcos, TX 78666}
       
\pagestyle{empty}


\begin{document}

\lstset{language=[Objective]Caml,frame=shadowbox}

\input{qm2pi.front}

% section front matter (end)

\input{qm2pi.intro} 
 
% section introduction (end)

% \input{qm2pi.knotations} 

% section notation (end)

\input{qm2pi.process.calculi} 

% section concurrent_process_calculi_and_spatial_logics_ (end)
    
%\input{qm2pi.knots2pi} 

%\input{qm2pi.trefoil} 

%\input{qm2pi.mainthm} 

% subsection basic_interpretation (end)

%\input{qm2pi.rho.presentation} 
\subsection{The syntax and semantics of the notation system}\label{sub:the_syntax_and_semantics_of_the_notation_system} % (fold)

We now summarize a technical presentation of the calculus that
embodies our theory of dynamics. The typical presentation of such a
calculus follows the style of giving generators and relations on
them. The grammar, below, describing term constructors, freely
generates the set of processes, $\Proc$. This set is then quotiented
by a relation known as structural congruence and it is over this set
that the notion of dynamics is expressed. This presentation is
essentially that of \cite{MeredithR05} with the addition of
polyadicity and summation. For readability we have relegated some of
the technical subtleties to an appendix.

\subsubsection{Process grammar}\label{subsub:process_grammar}

\begin{mathpar}
  \inferrule* [lab=synchronization] {} {{M} \bc \pzero \;|\; x?F \;|\; x!C }
  \and
  \inferrule* [lab=abstraction] {} {{F} \bc (x)P}
  \and
  \inferrule* [lab=concretion] {} {{C} \bc \langle Q \rangle}
  \and
  \inferrule* [lab=process] {} {{P,Q} \bc M \;| \;P|Q \;|\; @{x}}
  \and
  \inferrule* [lab=name] {} {{x} \bc \quotep{P}}
\end{mathpar} 

Note that $\vec{x}$ (resp. $\vec{P}$) denotes a vector of names
(resp. processes) of length $|\vec{x}|$ (resp. $|\vec{P}|$). We adopt
the following useful abbreviations.

\begin{mathpar}
   x?(\vec{y}).P := x.(\vec{y})P \and  x\clift{\vec{P}} := x.\clift{\vec{P}}
   \and x!(y) := \lift{x}{\dropn{y}}
   \and \Pi_{i=0}^{n-1}P_i := P_0 | \ldots | P_{n-1}
\end{mathpar}

\subsubsection{Structural congruence}

\paragraph{Free and bound names and alpha-equivalence.} At the
core of structural equivalence is alpha-equivalence which identifies
process that are the same up to a change of variable. Formally, we
recognize the distinction between free and bound names. The free names
of a process, $\freenames{P}$, may be calculated recursively as
follows:

\begin{mathpar}
\freenames{\pzero} := \emptyset
  \and \\
  \freenames{x?(y).P} := \{ x \} \cup (\freenames{P} \setminus \{ y \})
  \and 
  \freenames{x!\langle P \rangle} := \{ x \} \cup \{ P \} 
  \and \\
  \freenames{P|Q} := \freenames{P} \cup \freenames{Q}
  \and \\
  \freenames{@{x}} := \{ x \}
\end{mathpar}

$\pi$
$\quotep{\pi}$

$\freenames{-} : \pi \to \mathcal{P}(\quotep{\pi})$

\begin{eqnarray*}
  \freenames{\pzero} & := & \emptyset \\
  \freenames{x?(y).P} & := & \{ x \} \cup (\freenames{P} \setminus \{ y \}) \\
  \freenames{x!\langle P \rangle} & := & \{ x \} \cup \{ P \} \\
  \freenames{P|Q} & := & \freenames{P} \cup \freenames{Q} \\
  \freenames{\dropn{x}} & := & \{ x \}
\end{eqnarray*}

The bound names of a process, $\boundnames{P}$, are those names occurring in $P$
that are not free. For example, in $x?(y).0$, the name $x$ is free, while $y$ is bound.

\begin{mathpar}
  \inferrule* [lab=monoidal-laws] {} { P|Q \equiv Q|P \and P|0 \equiv P \and P|(Q|R) \equiv (P|Q)|R }
\end{mathpar}

\begin{mathpar}
  \inferrule* [lab=alpha-equivalence] {} { (x)P \equiv (y)P\{y/x\} \and y \not\in \freenames{P} }
\end{mathpar}

\begin{definition}
Then two processes, $P,Q$, are alpha-equivalent if $P = Q\{\vec{y}/\vec{x}\}$ for
some $\vec{x} \in \boundnames{Q},\vec{y} \in \boundnames{P}$, where $Q\{\vec{y}/\vec{x}\}$
denotes the capture-avoiding substitution of $\vec{y}$ for $\vec{x}$ in $Q$.
\end{definition}

\begin{definition}
  The {\em structural congruence} \cite{SangiorgiWalker} , $\equiv$,
  between processes is the least congruence containing
  alpha-equivalence, satisfying the abelian monoid laws
  (associativity, commutativity and $\pzero$ as identity) for parallel
  composition $|$ and for summation $+$.
\end{definition}

\subsection{Name equivalence}

We take name equivalence, written $\nameeq$, to be the smallest
equivalence relation generated by the following rules.

\begin{mathpar}
\inferrule*[lab=Quote-drop]
{ }
{ \quotep{@{x}} \nameeq x }

\inferrule*[lab=Struct-equiv]
{ P \scong Q }
{ \quotep{P} \nameeq \quotep{Q} }
\end{mathpar}

The astute reader will have noticed that the mutual recursion of names
and processes imposes a mutual recursion on alpha-equivalence and
structural equivalence via name-equivalence. Fortunately, all of this
works out pleasantly and we may calculate in the natural way, free of
concern. The reader interested in the details is referred to the
appendix \ref{appendix:rho_details}.

\subsection{Substitution}

We use $\Proc$ for the set of processes, $\QProc$ for the set of
names, and $\id{\{}\vec{y} / \vec{x} \id{\}}$ to denote partial maps,
$s : \QProc \rightarrow \QProc$. A map, $s$ lifts, uniquely, to a map
on process terms, $\widehat{s} : \Proc \rightarrow \Proc$ by the
following equations.

\begin{mathpar}
  (0) \psubstp{Q}{P} := 0 \\
  (R \juxtap S) \psubstp{Q}{P}
  :=    
  (R)\psubstp{Q}{P} \juxtap (S) \psubstp{Q}{P} \\
  (x?(y).R) \psubstp{Q}{P}    
  :=    
  (x)\substp{Q}{P} (z)\concat( (R \psubstn{z}{y}) \psubstp{Q}{P} ) \\
  (\lift{x}{R}) \psubstp{Q}{P}  
  :=
  \lift{(x)\substp{Q}{P}}{ R \psubstp{Q}{P} } \\
%   (\dropn{x})  \psubstp{Q}{P}       
%   := 
%   \left\{ 
%     \begin{array}{ccc} 
%       \dropn{\quotep{Q}} & & x \nameeq \quotep{P} \\
%       \dropn{x} & & otherwise \\
%     \end{array}
%   \right. 
  (\dropn{x})  \psubstp{Q}{P}       
  := 
  \left\{ 
    \begin{array}{ccc} 
      Q & & x \nameeq \quotep{P} \\
      \dropn{x} & & otherwise \\
    \end{array}
  \right.
\end{mathpar}
 

where

\begin{eqnarray}
  (x)\id{\{} \lpquote Q \rpquote / \lpquote P \rpquote \id{\}}            = 
  \left\{ 
    \begin{array}{ccc}
      \lpquote Q \rpquote & & x \nameeq \lpquote P \rpquote \\
      x & & otherwise \\
    \end{array}
  \right. \nonumber
\end{eqnarray}

and $z$ is chosen distinct from $\quotep{P}$, $\quotep{Q}$, the free
names in $Q$, and all the names in $R$. Our $\alpha$-equivalence will
be built in the standard way from this substitution.

\begin{remark}\label{rem:no_self_referential_names}
  One consequence of these definitions is that $\forall P. \quotep{P}
  \not\in \freenames{P}$.
\end{remark}

\subsection{ Dynamic quote: an example }

Anticipating something of what's to come, consider applying the
substitution, $\widehat{\id{\{}u / z \id{\}}}$, to the following pair
of processes, $\lift{w}{y!(z)}$ and $w[ \lpquote y!(z) \rpquote ]$.

\begin{eqnarray}
	\lift{w}{y!(z)}\widehat{\id{\{}u / z \id{\}}}
		& = &
		\lift{w}{y!(u)} \nonumber\\
	w[ \lpquote y!(z) \rpquote ] \widehat{ \id{\{}u / z \id{\}} }
		& = &
		w[ \lpquote y!(z) \rpquote ] \nonumber
\end{eqnarray}

Because the body of the process between quotes is impervious to
substitution, we get radically different answers. In fact, by
examining the first process in an input context,
e.g. $x?(z).\lift{w}{y!(z)}$, we see that the process under the lift
operator may be shaped by prefixed inputs binding a name inside it. In
this sense, the lift operator will be seen as a way to dynamically
construct processes before reifying them as names.

Finally equipped with these standard features we can present the
dynamics of the calculus.

\subsubsection{Operational semantics} 

Finally, we introduce the computational dynamics. What marks these
algebras as distinct from other more traditionally studied algebraic
structures, e.g. vector spaces or polynomial rings, is the manner in
which dynamics is captured. In traditional structures, dynamics is typically
expressed through morphisms between such structures, as in linear maps
between vector spaces or morphisms between rings. In algebras
associated with the semantics of computation, the dynamics is
expressed as part of the algebraic structure itself, through a
reduction reduction relation typically denoted by $\red$. Below, we
give a recursive presentation of this relation for the calculus used
in the encoding.

$\red \subseteq \pi \times \pi$
$\red : \pi \to \mathcal{P}(\pi)$

\begin{mathpar}
  \inferrule* [lab=Comm] { \textsf{match}( x_{src}, x_{trgt} ) } { x_{trgt}?(y)P \; | \; x_{src}!\langle {Q} \rangle \red P\{\quotep{Q}/y}\} }
  \and \\
  \inferrule* [lab=Par] {{P} \red {P}'} {{{P} | {Q}} \red {{P}' | {Q}}}
  \and
  \inferrule* [lab=Equiv]{{{P} \scong {P}'} \andalso {{P}' \red {Q}'} \andalso {{Q}' \scong {Q}}}{{P} \red {Q}}
\end{mathpar}

\begin{eqnarray*}
  match_{\equiv} (\quotep{P},\quotep{Q}) & := & P \equiv Q \\
  match_{\dagger}(\quotep{P},\quotep{Q}) & := & \forall R. P|Q \red^{*} R => R \red^{*} 0 \\
  match_{K}(\quotep{P},\quotep{Q}) & := & K \mbox{ for some context } K
\end{eqnarray*}

$u?(x)P | u!\langle Q \rangle \red P\{\quotep{Q}/x\}$

%We write $\wred$ for $\red^*$, and $P\red$ if $\exists Q $ such that $ P \red Q$.
We write $P\red$ if $\exists Q $ such that $ P \red Q$ and $P\not\red$, otherwise.

\section{Replication}

As mentioned before, it is known that replication (and hence
recursion) can be implemented in a higher-order process algebra
\cite{SangiorgiWalker}. As our first example of calculation with the
machinery thus far presented we give the construction explicitly in
the {\rhoc}.

\begin{eqnarray}
	D_{x} & := & \prefix{x}{y}{(\binpar{\outputp{x}{y}}{@{y}})} \nonumber\\
	\bangp_{x}{P} & := & \binpar{{x}!\langle{\binpar{D_{x}}{P}}\rangle}{D_{x}} \nonumber
\end{eqnarray}

\begin{eqnarray}
	\bangp_{x}{P} & & \nonumber\\
	=
	& {x}!\langle{(\prefix{x}{y}{(\outputp{x}{y} | @{y})) | P}}\rangle 
	      | \prefix{x}{y}{(\outputp{x}{y} | @{y})} & \nonumber\\
	\red
	& (\outputp{x}{y} | @{y})\substn{\quotep{(\prefix{x}{y}{(@{y} | \outputp{x}{y})) | P}}}{y} & \nonumber\\
	=
	& \outputp{x}{\quotep{(\prefix{x}{y}{(\outputp{x}{y} | @{y})) | P}}}
	  | {(\prefix{x}{y}{(\outputp{x}{y} | @{y})) | P}} & \nonumber\\
	\red
	& \ldots & \nonumber\\
	\red^*
	& P | P | \ldots & \nonumber
\end{eqnarray}

Of course, this encoding, as an implementation, runs away, unfolding
$\bangp{P}$ eagerly. A lazier and more implementable replication
operator, restricted to input-guarded processes, may be obtained as follows.

\begin{eqnarray}
\bangp{\prefix{u}{v}{P}} 
	:= 
	\binpar{\lift{x}{\prefix{u}{v}{(\binpar{D(x)}{P})}}}{D(x)} \nonumber
\end{eqnarray}

\begin{remark}
  Note that the lazier definition still does not deal with summation
  or mixed summation (i.e. sums over input and output). The reader is
  invited to construct definitions of replication that deal with these
  features. 

  Further, the definitions are parameterized in a name, $x$. Can you,
  gentle reader, make a definition that eliminates this parameter and
  guarantees no accidental interaction between the replication
  machinery and the process being replicated -- i.e. no accidental
  sharing of names used by the process to get its work done and the
  name(s) used by the replication to effect copying. This latter
  revision of the definition of replication is crucial to obtaining
  the expected identity $!!P \sim !P$.
\end{remark}

\begin{remark}\label{rem:paradoxical_combinator}
  The reader familiar with the lambda calculus will have noticed the
  similarity between $D$ and the paradoxical combinator.

  [Ed. note: the existence of this seems to suggest we have to be more
  restrictive on the set of processes and names we admit if we are to
  support no-cloning.]
\end{remark}

\subsubsection{Bisimulation}

The computational dynamics gives rise to another kind of equivalence,
the equivalence of computational behavior. As previously mentioned
this is typically captured \emph{via} some form of bisimulation.

% The notion we use in this paper is weak barbed bisimulation
% \cite{milner91polyadicpi}.

The notion we use in this paper is derived from weak barbed
bisimulation \cite{milner91polyadicpi}. 

\begin{definition}
An \emph{observation relation}, $\downarrow_{\mathcal N}$, over a set
of names, $\mathcal N$, is the smallest relation satisfying the rules
below.

\infrule[Out-barb]{y \in {\mathcal N}, \; x \nameeq y}
		  {\outputp{x}{v} \downarrow_{\mathcal N} x}
\infrule[Par-barb]{\mbox{$P\downarrow_{\mathcal N} x$ or $Q\downarrow_{\mathcal N} x$}}
		  {\binpar{P}{Q} \downarrow_{\mathcal N} x}

We write $P \Downarrow_{\mathcal N} x$ if there is $Q$ such that 
$P \wred Q$ and $Q \downarrow_{\mathcal N} x$.
\end{definition}

\begin{definition}
%\label{def.bbisim}
An  ${\mathcal N}$-\emph{barbed bisimulation} over a set of names, ${\mathcal N}$, is a symmetric binary relation 
${\mathcal S}_{\mathcal N}$ between agents such that $P\rel{S}_{\mathcal N}Q$ implies:
\begin{enumerate}
\item If $P \red P'$ then $Q \wred Q'$ and $P'\rel{S}_{\mathcal N} Q'$.
\item If $P\downarrow_{\mathcal N} x$, then $Q\Downarrow_{\mathcal N} x$.
\end{enumerate}
$P$ is ${\mathcal N}$-barbed bisimilar to $Q$, written
$P \wbbisim_{\mathcal N} Q$, if $P \rel{S}_{\mathcal N} Q$ for some ${\mathcal N}$-barbed bisimulation ${\mathcal S}_{\mathcal N}$.
\end{definition}

$\mathcal{R} \subseteq \pi \times \pi$

$P \mathcal{R} Q => \forall P'. P \red P' \Rightarrow \exists Q'. Q \red Q', P' \mathcal{R} Q'$

$P \vdash x \Rightarrow Q \vdash x$

\begin{mathpar}
  \inferrule*[lab=Out-barb]{x \nameeq y}{{y}!\langle{Q}\rangle \vdash x}
  \and
  \inferrule*[lab=Par-barb]{\mbox{$P\vdash x$ or $Q\vdash x$}}{\binpar{P}{Q} \vdash x}
\end{mathpar}

\subsubsection{Contexts}

One of the principle advantages of computational calculi like the
$\pi$-calculus is a well-defined notion of context,
contextual-equivalence and a correlation between
contextual-equivalence and notions of bisimulation. The notion of
context allows the decomposition of a process into (sub-)process and
its syntactic environment, its context. Thus, a context may be
thought of as a process with a ``hole'' (written $\Box$) in it. The
application of a context $M$ to a process $P$, written $M[P]$, is
tantamount to filling the hole in $M$ with $P$. In this paper we do
not need the full weight of this theory, but do make use of the notion
of context in the proof the main theorem. 

\begin{mathpar}
  \inferrule* [lab=summation] {} {{M_{M},M_{N}} \bc \Box \;|\; x.M_{A} \;|\; M_{M}+M_{N}}
  \and
  \inferrule* [lab=agent] {} {{M_{A}} \bc (\vec{x})M_{P} \;| \; \clift{P_0,\ldots,M_{P},\ldots,P_N}}
  \and \\
  \inferrule* [lab=process] {} {{M_{P}} \bc M_{N} \;| \;P|M_{P} }
\end{mathpar} 

\begin{mathpar}
  \inferrule* [lab=sychronization] {} {M_{N} \bc \Box \;|\; x?M_{F} \;|\; x!M_{C}}
  \and
  \inferrule* [lab=abstraction] {} {{M_{F}} \bc (x)M_{P} }
  \and
  \inferrule* [lab=concretion] {} {{M_{C}} \bc \langle M_{P} \rangle }
  \and \\
  \inferrule* [lab=process] {} {{M_{P}} \bc M_{N} \;| \;P|M_{P} }
\end{mathpar}

\begin{definition}[contextual application] Given a context $M$, and
  process $P$, we define the \emph{contextual application}, $M[P] :=
  M\{P/\Box\}$. That is, the contextual application of M to P is the
  substitution of $P$ for $\Box$ in $M$.
\end{definition}

$\meaningof{-} : L \to \mathcal{P}(\pi)$

\begin{mathpar}
  \inferrule* [lab=collection] {} {\meaningof{true} = \pi, \and \meaningof{~E} = \pi \setminus \meaningof{E}, \and \meaningof{E_{1} \& E_{2}} = \meaningof{E_{1}} \cap \meaningof{E_{2}}}
\end{mathpar}

\begin{mathpar}
  \inferrule* [lab=structure] {} {\meaningof{0} = \{ P \in \pi | P \equiv 0 \}, \and \\ \meaningof{E_1 | E_2} = \{ P \in \pi | P \equiv P_{1} | P_{2}, P_{1} \in \meaningof{E_{1}}, P_{2} \in \meaningof{E_2}\} }
\end{mathpar}

\begin{mathpar}
 \inferrule* [lab=behavior] {} {\meaningof{\langle a?b \rangle E} = \{ P \in \pi | P \equiv Q | u?(y)P', \\ \and \\\\ \and \\ \;\;\; u \in \meaningof{a}, \forall z.P'\{z/y\} \in \meaningof{E\{z/b\}}\}, \and \\ \meaningof{a!E} = \{ P \in \pi | P \equiv Q | x!\langle P' \rangle, x \in \meaningof{a} P' \in \meaningof{E}\} }
\end{mathpar}

\begin{mathpar}
 \inferrule* [lab=nominal] {} {\meaningof{\quotep{E}} = \{ \quotep{P} \in \quotep{\pi} | P \in \meaningof{E} \}, \and \meaningof{\quotep{P}} = \{ \quotep{Q} \in \quotep{\pi} | P \equiv Q \} \and \\ \meaningof{@\quotep{E}} = \{ P \in \pi | P \equiv @x, x \in \meaningof{E} \}}
\end{mathpar}

\begin{eqnarray*}
  \\
  \meaningof{-} : TS \to ST
\end{eqnarray*}

\begin{eqnarray*}
  \\
  L : TS \to ST
\end{eqnarray*}

\begin{eqnarray*}
  \\
  P \models E \iff P \in \meaningof{E}
\end{eqnarray*}

\begin{eqnarray*}
  P \approx_{L} Q \iff \forall E \in L. P \models E \iff Q \models E
\end{eqnarray*}

\begin{eqnarray*}
  P \approx_{K} Q
\end{eqnarray*}

\begin{eqnarray*}
  P \approx Q
\end{eqnarray*}

$\approx_{K} = \approx = \approx_{L}$

\subsubsection{Contextual duality}

Note that contexts extend the quotation operation to a family of
operations from processes to names. Given a context, $M$, we can
define a \emph{nominal context}, $\quotep{M}$ by $\quotep{M}[P] :=
\quotep{M[P]}$. To foreshadow what is to come we observe that these
operations enjoy a duality with processes very much like the duality
between vectors and maps from vectors to scalars.

Further, because the calculus is essentially higher-order, we have a
correspondence between contexts and processes. More specifically,
given a name $x$ and a context $M$ we can construct $M^{*}_{x}$ such
that 

\begin{mathpar}
  M^{*}_{x} | \lift{x}{P} \red M[P]
\end{mathpar}

namely,

\begin{mathpar}
  M^{*}_{x} := x?(u).M[\dropn{u}]
\end{mathpar}

The dependence of $M^{*}_{x}$ on a name makes it an abstraction, 

\begin{mathpar}
  M^{*} := (x)x?(u).M[\dropn{u}]
\end{mathpar}

\subsection{Additional notation}

It will sometimes be convenient to denote the process a name
quotes. We already have the notation $x = \quotep{P}$, but it will be
convenient to introduce an alternate notation, $\procn{x}$, when we
want to emphasize the connection to the use of the name. Note that, by
virtue of name equivalence, $\quotep{\procn{x}} \nameeq x$; so, the
notation is consistent with previous definitions.

Further, because names have structure it is possible to effect
substitutions on the basis of that structure. This means we need to
upgrade our notation for substitutions, which we accomplish by
adapting comprehension notation. Thus,

\begin{mathpar}
  P\{ y / x : x \in S \}
\end{mathpar}

is interpreted to mean the process derived from P by replacing (in a
capture-avoiding manner) each occurrence of $x$ in $S$ by $y$. For example,

\begin{mathpar}
  P\{ \quotep{\procn{x}|\procn{x}} / x : x \in \freenames{P} \}
\end{mathpar}

will replace each (occurrence) of a free name $x$ in $P$ by
$\quotep{\procn{x}|\procn{x}}$.

Also, we will avail ourselves of the notation $x^{L}$ and $x^{R}$ to
denote injections of a name into disjoint copies of the name
space. There are numerous ways to accomplish this. One example can be
found in \cite{MeredithR05}. This notation overloads to vectors of
names: $\vec{x}^{\pi} := (x_{i}^{\pi} \; : \; 0 \leq i < |\vec{x}| )$ where $\pi \in \{L,R\}$.

We also use $P^{\Box} := P|\Box$.

In \cite{MeredithR05} an interpretation of the new operator is
given. It turns out that there are several possible interpretations
all enjoying the requisite algebraic properties of the operator (see
\cite{milner91polyadicpi}). We will therefore make liberal use of
$(\nu\; \vec{x})P$.

% subsection the_syntax_and_semantics_of_the_notation_system (end)   

\input{qm2pi.qmops} 

\input{qm2pi.sterngerlach} 

\input{qm2pi.metric} 

% section concurrent_process_calculi (end)

%\input{qm2pi.proofsketch}

% section proof sketch (end)

%\input{qm2pi.slviaknots} 

% section spatial logic via knots (end)

\input{qm2pi.conclusion}

% section conclusion (end)

%\input{qm2pi.dtcodes} 

% section wiring algorithm (end)

\input{qm2pi.ack} 

% section acknowledgments (end)

\newpage


\bibliographystyle{plain}   
\bibliography{../../biblios/main.bib}

\input{qm2pi.rhodetails}

\end{document}

 

% subsection basic_interpretation (end)

%\input{qm2pi.rho.presentation} 
\subsection{The syntax and semantics of the notation system}\label{sub:the_syntax_and_semantics_of_the_notation_system} % (fold)

We now summarize a technical presentation of the calculus that
embodies our theory of dynamics. The typical presentation of such a
calculus follows the style of giving generators and relations on
them. The grammar, below, describing term constructors, freely
generates the set of processes, $\Proc$. This set is then quotiented
by a relation known as structural congruence and it is over this set
that the notion of dynamics is expressed. This presentation is
essentially that of \cite{MeredithR05} with the addition of
polyadicity and summation. For readability we have relegated some of
the technical subtleties to an appendix.

\subsubsection{Process grammar}\label{subsub:process_grammar}

\begin{mathpar}
  \inferrule* [lab=synchronization] {} {{M} \bc \pzero \;|\; x?F \;|\; x!C }
  \and
  \inferrule* [lab=abstraction] {} {{F} \bc (x)P}
  \and
  \inferrule* [lab=concretion] {} {{C} \bc \langle Q \rangle}
  \and
  \inferrule* [lab=process] {} {{P,Q} \bc M \;| \;P|Q \;|\; @{x}}
  \and
  \inferrule* [lab=name] {} {{x} \bc \quotep{P}}
\end{mathpar} 

Note that $\vec{x}$ (resp. $\vec{P}$) denotes a vector of names
(resp. processes) of length $|\vec{x}|$ (resp. $|\vec{P}|$). We adopt
the following useful abbreviations.

\begin{mathpar}
   x?(\vec{y}).P := x.(\vec{y})P \and  x\clift{\vec{P}} := x.\clift{\vec{P}}
   \and x!(y) := \lift{x}{\dropn{y}}
   \and \Pi_{i=0}^{n-1}P_i := P_0 | \ldots | P_{n-1}
\end{mathpar}

\subsubsection{Structural congruence}

\paragraph{Free and bound names and alpha-equivalence.} At the
core of structural equivalence is alpha-equivalence which identifies
process that are the same up to a change of variable. Formally, we
recognize the distinction between free and bound names. The free names
of a process, $\freenames{P}$, may be calculated recursively as
follows:

\begin{mathpar}
\freenames{\pzero} := \emptyset
  \and \\
  \freenames{x?(y).P} := \{ x \} \cup (\freenames{P} \setminus \{ y \})
  \and 
  \freenames{x!\langle P \rangle} := \{ x \} \cup \{ P \} 
  \and \\
  \freenames{P|Q} := \freenames{P} \cup \freenames{Q}
  \and \\
  \freenames{@{x}} := \{ x \}
\end{mathpar}

$\pi$
$\quotep{\pi}$

$\freenames{-} : \pi \to \mathcal{P}(\quotep{\pi})$

\begin{eqnarray*}
  \freenames{\pzero} & := & \emptyset \\
  \freenames{x?(y).P} & := & \{ x \} \cup (\freenames{P} \setminus \{ y \}) \\
  \freenames{x!\langle P \rangle} & := & \{ x \} \cup \{ P \} \\
  \freenames{P|Q} & := & \freenames{P} \cup \freenames{Q} \\
  \freenames{\dropn{x}} & := & \{ x \}
\end{eqnarray*}

The bound names of a process, $\boundnames{P}$, are those names occurring in $P$
that are not free. For example, in $x?(y).0$, the name $x$ is free, while $y$ is bound.

\begin{mathpar}
  \inferrule* [lab=monoidal-laws] {} { P|Q \equiv Q|P \and P|0 \equiv P \and P|(Q|R) \equiv (P|Q)|R }
\end{mathpar}

\begin{mathpar}
  \inferrule* [lab=alpha-equivalence] {} { (x)P \equiv (y)P\{y/x\} \and y \not\in \freenames{P} }
\end{mathpar}

\begin{definition}
Then two processes, $P,Q$, are alpha-equivalent if $P = Q\{\vec{y}/\vec{x}\}$ for
some $\vec{x} \in \boundnames{Q},\vec{y} \in \boundnames{P}$, where $Q\{\vec{y}/\vec{x}\}$
denotes the capture-avoiding substitution of $\vec{y}$ for $\vec{x}$ in $Q$.
\end{definition}

\begin{definition}
  The {\em structural congruence} \cite{SangiorgiWalker} , $\equiv$,
  between processes is the least congruence containing
  alpha-equivalence, satisfying the abelian monoid laws
  (associativity, commutativity and $\pzero$ as identity) for parallel
  composition $|$ and for summation $+$.
\end{definition}

\subsection{Name equivalence}

We take name equivalence, written $\nameeq$, to be the smallest
equivalence relation generated by the following rules.

\begin{mathpar}
\inferrule*[lab=Quote-drop]
{ }
{ \quotep{@{x}} \nameeq x }

\inferrule*[lab=Struct-equiv]
{ P \scong Q }
{ \quotep{P} \nameeq \quotep{Q} }
\end{mathpar}

The astute reader will have noticed that the mutual recursion of names
and processes imposes a mutual recursion on alpha-equivalence and
structural equivalence via name-equivalence. Fortunately, all of this
works out pleasantly and we may calculate in the natural way, free of
concern. The reader interested in the details is referred to the
appendix \ref{appendix:rho_details}.

\subsection{Substitution}

We use $\Proc$ for the set of processes, $\QProc$ for the set of
names, and $\id{\{}\vec{y} / \vec{x} \id{\}}$ to denote partial maps,
$s : \QProc \rightarrow \QProc$. A map, $s$ lifts, uniquely, to a map
on process terms, $\widehat{s} : \Proc \rightarrow \Proc$ by the
following equations.

\begin{mathpar}
  (0) \psubstp{Q}{P} := 0 \\
  (R \juxtap S) \psubstp{Q}{P}
  :=    
  (R)\psubstp{Q}{P} \juxtap (S) \psubstp{Q}{P} \\
  (x?(y).R) \psubstp{Q}{P}    
  :=    
  (x)\substp{Q}{P} (z)\concat( (R \psubstn{z}{y}) \psubstp{Q}{P} ) \\
  (\lift{x}{R}) \psubstp{Q}{P}  
  :=
  \lift{(x)\substp{Q}{P}}{ R \psubstp{Q}{P} } \\
%   (\dropn{x})  \psubstp{Q}{P}       
%   := 
%   \left\{ 
%     \begin{array}{ccc} 
%       \dropn{\quotep{Q}} & & x \nameeq \quotep{P} \\
%       \dropn{x} & & otherwise \\
%     \end{array}
%   \right. 
  (\dropn{x})  \psubstp{Q}{P}       
  := 
  \left\{ 
    \begin{array}{ccc} 
      Q & & x \nameeq \quotep{P} \\
      \dropn{x} & & otherwise \\
    \end{array}
  \right.
\end{mathpar}
 

where

\begin{eqnarray}
  (x)\id{\{} \lpquote Q \rpquote / \lpquote P \rpquote \id{\}}            = 
  \left\{ 
    \begin{array}{ccc}
      \lpquote Q \rpquote & & x \nameeq \lpquote P \rpquote \\
      x & & otherwise \\
    \end{array}
  \right. \nonumber
\end{eqnarray}

and $z$ is chosen distinct from $\quotep{P}$, $\quotep{Q}$, the free
names in $Q$, and all the names in $R$. Our $\alpha$-equivalence will
be built in the standard way from this substitution.

\begin{remark}\label{rem:no_self_referential_names}
  One consequence of these definitions is that $\forall P. \quotep{P}
  \not\in \freenames{P}$.
\end{remark}

\subsection{ Dynamic quote: an example }

Anticipating something of what's to come, consider applying the
substitution, $\widehat{\id{\{}u / z \id{\}}}$, to the following pair
of processes, $\lift{w}{y!(z)}$ and $w[ \lpquote y!(z) \rpquote ]$.

\begin{eqnarray}
	\lift{w}{y!(z)}\widehat{\id{\{}u / z \id{\}}}
		& = &
		\lift{w}{y!(u)} \nonumber\\
	w[ \lpquote y!(z) \rpquote ] \widehat{ \id{\{}u / z \id{\}} }
		& = &
		w[ \lpquote y!(z) \rpquote ] \nonumber
\end{eqnarray}

Because the body of the process between quotes is impervious to
substitution, we get radically different answers. In fact, by
examining the first process in an input context,
e.g. $x?(z).\lift{w}{y!(z)}$, we see that the process under the lift
operator may be shaped by prefixed inputs binding a name inside it. In
this sense, the lift operator will be seen as a way to dynamically
construct processes before reifying them as names.

Finally equipped with these standard features we can present the
dynamics of the calculus.

\subsubsection{Operational semantics} 

Finally, we introduce the computational dynamics. What marks these
algebras as distinct from other more traditionally studied algebraic
structures, e.g. vector spaces or polynomial rings, is the manner in
which dynamics is captured. In traditional structures, dynamics is typically
expressed through morphisms between such structures, as in linear maps
between vector spaces or morphisms between rings. In algebras
associated with the semantics of computation, the dynamics is
expressed as part of the algebraic structure itself, through a
reduction reduction relation typically denoted by $\red$. Below, we
give a recursive presentation of this relation for the calculus used
in the encoding.

$\red \subseteq \pi \times \pi$
$\red : \pi \to \mathcal{P}(\pi)$

\begin{mathpar}
  \inferrule* [lab=Comm] { \textsf{match}( x_{src}, x_{trgt} ) } { x_{trgt}?(y)P \; | \; x_{src}!\langle {Q} \rangle \red P\{\quotep{Q}/y}\} }
  \and \\
  \inferrule* [lab=Par] {{P} \red {P}'} {{{P} | {Q}} \red {{P}' | {Q}}}
  \and
  \inferrule* [lab=Equiv]{{{P} \scong {P}'} \andalso {{P}' \red {Q}'} \andalso {{Q}' \scong {Q}}}{{P} \red {Q}}
\end{mathpar}

\begin{eqnarray*}
  match_{\equiv} (\quotep{P},\quotep{Q}) & := & P \equiv Q \\
  match_{\dagger}(\quotep{P},\quotep{Q}) & := & \forall R. P|Q \red^{*} R => R \red^{*} 0 \\
  match_{K}(\quotep{P},\quotep{Q}) & := & K \mbox{ for some context } K
\end{eqnarray*}

$u?(x)P | u!\langle Q \rangle \red P\{\quotep{Q}/x\}$

%We write $\wred$ for $\red^*$, and $P\red$ if $\exists Q $ such that $ P \red Q$.
We write $P\red$ if $\exists Q $ such that $ P \red Q$ and $P\not\red$, otherwise.

\section{Replication}

As mentioned before, it is known that replication (and hence
recursion) can be implemented in a higher-order process algebra
\cite{SangiorgiWalker}. As our first example of calculation with the
machinery thus far presented we give the construction explicitly in
the {\rhoc}.

\begin{eqnarray}
	D_{x} & := & \prefix{x}{y}{(\binpar{\outputp{x}{y}}{@{y}})} \nonumber\\
	\bangp_{x}{P} & := & \binpar{{x}!\langle{\binpar{D_{x}}{P}}\rangle}{D_{x}} \nonumber
\end{eqnarray}

\begin{eqnarray}
	\bangp_{x}{P} & & \nonumber\\
	=
	& {x}!\langle{(\prefix{x}{y}{(\outputp{x}{y} | @{y})) | P}}\rangle 
	      | \prefix{x}{y}{(\outputp{x}{y} | @{y})} & \nonumber\\
	\red
	& (\outputp{x}{y} | @{y})\substn{\quotep{(\prefix{x}{y}{(@{y} | \outputp{x}{y})) | P}}}{y} & \nonumber\\
	=
	& \outputp{x}{\quotep{(\prefix{x}{y}{(\outputp{x}{y} | @{y})) | P}}}
	  | {(\prefix{x}{y}{(\outputp{x}{y} | @{y})) | P}} & \nonumber\\
	\red
	& \ldots & \nonumber\\
	\red^*
	& P | P | \ldots & \nonumber
\end{eqnarray}

Of course, this encoding, as an implementation, runs away, unfolding
$\bangp{P}$ eagerly. A lazier and more implementable replication
operator, restricted to input-guarded processes, may be obtained as follows.

\begin{eqnarray}
\bangp{\prefix{u}{v}{P}} 
	:= 
	\binpar{\lift{x}{\prefix{u}{v}{(\binpar{D(x)}{P})}}}{D(x)} \nonumber
\end{eqnarray}

\begin{remark}
  Note that the lazier definition still does not deal with summation
  or mixed summation (i.e. sums over input and output). The reader is
  invited to construct definitions of replication that deal with these
  features. 

  Further, the definitions are parameterized in a name, $x$. Can you,
  gentle reader, make a definition that eliminates this parameter and
  guarantees no accidental interaction between the replication
  machinery and the process being replicated -- i.e. no accidental
  sharing of names used by the process to get its work done and the
  name(s) used by the replication to effect copying. This latter
  revision of the definition of replication is crucial to obtaining
  the expected identity $!!P \sim !P$.
\end{remark}

\begin{remark}\label{rem:paradoxical_combinator}
  The reader familiar with the lambda calculus will have noticed the
  similarity between $D$ and the paradoxical combinator.

  [Ed. note: the existence of this seems to suggest we have to be more
  restrictive on the set of processes and names we admit if we are to
  support no-cloning.]
\end{remark}

\subsubsection{Bisimulation}

The computational dynamics gives rise to another kind of equivalence,
the equivalence of computational behavior. As previously mentioned
this is typically captured \emph{via} some form of bisimulation.

% The notion we use in this paper is weak barbed bisimulation
% \cite{milner91polyadicpi}.

The notion we use in this paper is derived from weak barbed
bisimulation \cite{milner91polyadicpi}. 

\begin{definition}
An \emph{observation relation}, $\downarrow_{\mathcal N}$, over a set
of names, $\mathcal N$, is the smallest relation satisfying the rules
below.

\infrule[Out-barb]{y \in {\mathcal N}, \; x \nameeq y}
		  {\outputp{x}{v} \downarrow_{\mathcal N} x}
\infrule[Par-barb]{\mbox{$P\downarrow_{\mathcal N} x$ or $Q\downarrow_{\mathcal N} x$}}
		  {\binpar{P}{Q} \downarrow_{\mathcal N} x}

We write $P \Downarrow_{\mathcal N} x$ if there is $Q$ such that 
$P \wred Q$ and $Q \downarrow_{\mathcal N} x$.
\end{definition}

\begin{definition}
%\label{def.bbisim}
An  ${\mathcal N}$-\emph{barbed bisimulation} over a set of names, ${\mathcal N}$, is a symmetric binary relation 
${\mathcal S}_{\mathcal N}$ between agents such that $P\rel{S}_{\mathcal N}Q$ implies:
\begin{enumerate}
\item If $P \red P'$ then $Q \wred Q'$ and $P'\rel{S}_{\mathcal N} Q'$.
\item If $P\downarrow_{\mathcal N} x$, then $Q\Downarrow_{\mathcal N} x$.
\end{enumerate}
$P$ is ${\mathcal N}$-barbed bisimilar to $Q$, written
$P \wbbisim_{\mathcal N} Q$, if $P \rel{S}_{\mathcal N} Q$ for some ${\mathcal N}$-barbed bisimulation ${\mathcal S}_{\mathcal N}$.
\end{definition}

$\mathcal{R} \subseteq \pi \times \pi$

$P \mathcal{R} Q => \forall P'. P \red P' \Rightarrow \exists Q'. Q \red Q', P' \mathcal{R} Q'$

$P \vdash x \Rightarrow Q \vdash x$

\begin{mathpar}
  \inferrule*[lab=Out-barb]{x \nameeq y}{{y}!\langle{Q}\rangle \vdash x}
  \and
  \inferrule*[lab=Par-barb]{\mbox{$P\vdash x$ or $Q\vdash x$}}{\binpar{P}{Q} \vdash x}
\end{mathpar}

\subsubsection{Contexts}

One of the principle advantages of computational calculi like the
$\pi$-calculus is a well-defined notion of context,
contextual-equivalence and a correlation between
contextual-equivalence and notions of bisimulation. The notion of
context allows the decomposition of a process into (sub-)process and
its syntactic environment, its context. Thus, a context may be
thought of as a process with a ``hole'' (written $\Box$) in it. The
application of a context $M$ to a process $P$, written $M[P]$, is
tantamount to filling the hole in $M$ with $P$. In this paper we do
not need the full weight of this theory, but do make use of the notion
of context in the proof the main theorem. 

\begin{mathpar}
  \inferrule* [lab=summation] {} {{M_{M},M_{N}} \bc \Box \;|\; x.M_{A} \;|\; M_{M}+M_{N}}
  \and
  \inferrule* [lab=agent] {} {{M_{A}} \bc (\vec{x})M_{P} \;| \; \clift{P_0,\ldots,M_{P},\ldots,P_N}}
  \and \\
  \inferrule* [lab=process] {} {{M_{P}} \bc M_{N} \;| \;P|M_{P} }
\end{mathpar} 

\begin{mathpar}
  \inferrule* [lab=sychronization] {} {M_{N} \bc \Box \;|\; x?M_{F} \;|\; x!M_{C}}
  \and
  \inferrule* [lab=abstraction] {} {{M_{F}} \bc (x)M_{P} }
  \and
  \inferrule* [lab=concretion] {} {{M_{C}} \bc \langle M_{P} \rangle }
  \and \\
  \inferrule* [lab=process] {} {{M_{P}} \bc M_{N} \;| \;P|M_{P} }
\end{mathpar}

\begin{definition}[contextual application] Given a context $M$, and
  process $P$, we define the \emph{contextual application}, $M[P] :=
  M\{P/\Box\}$. That is, the contextual application of M to P is the
  substitution of $P$ for $\Box$ in $M$.
\end{definition}

$\meaningof{-} : L \to \mathcal{P}(\pi)$

\begin{mathpar}
  \inferrule* [lab=collection] {} {\meaningof{true} = \pi, \and \meaningof{~E} = \pi \setminus \meaningof{E}, \and \meaningof{E_{1} \& E_{2}} = \meaningof{E_{1}} \cap \meaningof{E_{2}}}
\end{mathpar}

\begin{mathpar}
  \inferrule* [lab=structure] {} {\meaningof{0} = \{ P \in \pi | P \equiv 0 \}, \and \\ \meaningof{E_1 | E_2} = \{ P \in \pi | P \equiv P_{1} | P_{2}, P_{1} \in \meaningof{E_{1}}, P_{2} \in \meaningof{E_2}\} }
\end{mathpar}

\begin{mathpar}
 \inferrule* [lab=behavior] {} {\meaningof{\langle a?b \rangle E} = \{ P \in \pi | P \equiv Q | u?(y)P', \\ \and \\\\ \and \\ \;\;\; u \in \meaningof{a}, \forall z.P'\{z/y\} \in \meaningof{E\{z/b\}}\}, \and \\ \meaningof{a!E} = \{ P \in \pi | P \equiv Q | x!\langle P' \rangle, x \in \meaningof{a} P' \in \meaningof{E}\} }
\end{mathpar}

\begin{mathpar}
 \inferrule* [lab=nominal] {} {\meaningof{\quotep{E}} = \{ \quotep{P} \in \quotep{\pi} | P \in \meaningof{E} \}, \and \meaningof{\quotep{P}} = \{ \quotep{Q} \in \quotep{\pi} | P \equiv Q \} \and \\ \meaningof{@\quotep{E}} = \{ P \in \pi | P \equiv @x, x \in \meaningof{E} \}}
\end{mathpar}

\begin{eqnarray*}
  \\
  \meaningof{-} : TS \to ST
\end{eqnarray*}

\begin{eqnarray*}
  \\
  L : TS \to ST
\end{eqnarray*}

\begin{eqnarray*}
  \\
  P \models E \iff P \in \meaningof{E}
\end{eqnarray*}

\begin{eqnarray*}
  P \approx_{L} Q \iff \forall E \in L. P \models E \iff Q \models E
\end{eqnarray*}

\begin{eqnarray*}
  P \approx_{K} Q
\end{eqnarray*}

\begin{eqnarray*}
  P \approx Q
\end{eqnarray*}

$\approx_{K} = \approx = \approx_{L}$

\subsubsection{Contextual duality}

Note that contexts extend the quotation operation to a family of
operations from processes to names. Given a context, $M$, we can
define a \emph{nominal context}, $\quotep{M}$ by $\quotep{M}[P] :=
\quotep{M[P]}$. To foreshadow what is to come we observe that these
operations enjoy a duality with processes very much like the duality
between vectors and maps from vectors to scalars.

Further, because the calculus is essentially higher-order, we have a
correspondence between contexts and processes. More specifically,
given a name $x$ and a context $M$ we can construct $M^{*}_{x}$ such
that 

\begin{mathpar}
  M^{*}_{x} | \lift{x}{P} \red M[P]
\end{mathpar}

namely,

\begin{mathpar}
  M^{*}_{x} := x?(u).M[\dropn{u}]
\end{mathpar}

The dependence of $M^{*}_{x}$ on a name makes it an abstraction, 

\begin{mathpar}
  M^{*} := (x)x?(u).M[\dropn{u}]
\end{mathpar}

\subsection{Additional notation}

It will sometimes be convenient to denote the process a name
quotes. We already have the notation $x = \quotep{P}$, but it will be
convenient to introduce an alternate notation, $\procn{x}$, when we
want to emphasize the connection to the use of the name. Note that, by
virtue of name equivalence, $\quotep{\procn{x}} \nameeq x$; so, the
notation is consistent with previous definitions.

Further, because names have structure it is possible to effect
substitutions on the basis of that structure. This means we need to
upgrade our notation for substitutions, which we accomplish by
adapting comprehension notation. Thus,

\begin{mathpar}
  P\{ y / x : x \in S \}
\end{mathpar}

is interpreted to mean the process derived from P by replacing (in a
capture-avoiding manner) each occurrence of $x$ in $S$ by $y$. For example,

\begin{mathpar}
  P\{ \quotep{\procn{x}|\procn{x}} / x : x \in \freenames{P} \}
\end{mathpar}

will replace each (occurrence) of a free name $x$ in $P$ by
$\quotep{\procn{x}|\procn{x}}$.

Also, we will avail ourselves of the notation $x^{L}$ and $x^{R}$ to
denote injections of a name into disjoint copies of the name
space. There are numerous ways to accomplish this. One example can be
found in \cite{MeredithR05}. This notation overloads to vectors of
names: $\vec{x}^{\pi} := (x_{i}^{\pi} \; : \; 0 \leq i < |\vec{x}| )$ where $\pi \in \{L,R\}$.

We also use $P^{\Box} := P|\Box$.

In \cite{MeredithR05} an interpretation of the new operator is
given. It turns out that there are several possible interpretations
all enjoying the requisite algebraic properties of the operator (see
\cite{milner91polyadicpi}). We will therefore make liberal use of
$(\nu\; \vec{x})P$.

% subsection the_syntax_and_semantics_of_the_notation_system (end)   

\section{Interpretation of QM}
\subsection{Supporting definitions}
\subsubsection{Multiplication}
\begin{mathpar}
  \quotep{Q} \cdot \quotep{R} := \quotep{Q|R}
  \and \\
  \quotep{Q} \cdot P := P\{ \quotep{Q|R} / \quotep{R} : \quotep{R} \in \freenames{P} \}
\end{mathpar}

\paragraph{Discussion}
The first line needs little explanation. The second line says that
each free name of the process is replaced with the multiplication of
that name by the scalar. Multiplication of a scalar (name) by a state
(process) results in a process all the names of which have been `moved
over' by parallel composition with the process the scalar
quotes. There is a subtlety that the bound names have to be
manipulated so that multiplied names aren't accidentally
captured. There are many ways to achieve this.

\begin{remark}\label{rem:multiplication_identities}
  The reader is invited to verify that for all $x,y,z \in \QProc$ and $P \in \Proc$
  \begin{mathpar}
    x \cdot \quotep{0} \equiv x 
    \and
    x \cdot y \equiv y \cdot x
    \and
    x \cdot (y \cdot z) \equiv (x \cdot y) \cdot z
    \and \\
    \quotep{0} \cdot P \equiv P
    \and \\
    x \cdot (y \cdot P) \equiv (x \cdot y) \cdot P
    \and \\
    x \cdot (P|Q) \equiv (x \cdot P) | (x \cdot Q)
    \and \\    
  \end{mathpar}
\end{remark}

\subsubsection{Tensor product}

We define a tensor product on processes by structural induction.

\paragraph{Tensor of sums} First note that all summations, including
$\pzero$ and sequence, can be written $\Sigma_{i} x_{i}.A_{i} +
\Sigma_{j} x_{j}.C_{j}$, where we have grouped input-guarded processes
together and output-guarded processes together.

Thus, we can define the tensor product of two summations, $N_{1}\otimes N_{2}$, where

\begin{mathpar}
  N_{1} := \Sigma_{i} x_{i}.A_{i} + \Sigma_{j} x_{j}.C_{j}
  \and
  N_{2} := \Sigma_{i'} y_{i'}.B_{i'} + \Sigma_{j'} y_{j'}.D_{j'} 
\end{mathpar}

as follows.

\begin{mathpar}
  \Sigma_{i} x_{i}.A_{i} + \Sigma_{j} x_{j}.C_{j} \otimes \Sigma_{i'}
  y_{i'}.B_{i'} + \Sigma_{j'} y_{j'}.D_{j'} 
  \and \\
  := \; \Sigma_{i} \Sigma_{i'} \quotep{\stackrel{\vee}{x_{i}}| \stackrel{\vee}{y_{i'}}}.(A_{i}\otimes B_{i'}) \; | \; \Sigma_{i'} \Sigma_{i} \quotep{\stackrel{\vee}{y_{i'}}|\stackrel{\vee}{x_{i}}}.(B_{i'}\otimes A_{i})
  \and
  \;\; | \;\; \Sigma_{j} \Sigma_{j'} \quotep{\stackrel{\vee}{x_{j}}|\stackrel{\vee}{y_{j'}}}.(A_{j}\otimes B_{j'}) \; | \; \Sigma_{j'} \Sigma_{j} \quotep{\stackrel{\vee}{y_{j'}}|\stackrel{\vee}{x_{j}}}.(B_{j'}\otimes A_{j})
\end{mathpar}

\begin{remark}
  Do we need to $x^{L}$ and $y^{R}$ for this construction as well?
\end{remark}

\paragraph{Tensor of parallel compositions} Next, we distribute tensor
over par.

\begin{mathpar}
  P_{1}|P_{2} \otimes Q_{1}|Q_{2} := (P_{1} \otimes Q_{1}) | (P_{1}
  \otimes Q_{2}) | (P_{2} \otimes Q_{1}) | (P_{2} \otimes Q_{2})
\end{mathpar}

\paragraph{Tensor with dropped names} We treat tensor of a
process with a dropped name as parallel composition.

\begin{mathpar}
  P \otimes \dropn{x} := P | \dropn{x}
\end{mathpar}

\paragraph{Tensor of agents}

Finally, we need to define tensor on agents. Note that the definition
of tensor on normal products only tensors inputs with inputs and
outputs with outputs. Thus, we only have to define the operation on
``homogeneous'' pairings.

\begin{mathpar}
  (\vec{x})P \otimes (\vec{y})Q
  \and \\
  := (x_{0}^{L}|y_{0}^{R},\ldots,x_{0}^{L}|y_{n}^{R},\ldots,x_{m}^{L}|y_{0}^{R},\ldots,x_{m}^{L}|y_{n}^R)(P\{ \vec{x}^{L}/\vec{x}\} \otimes Q \{ \vec{y}^{R}/\vec{y}\})
  \and \\
  \clift{\vec{P}} \otimes \clift{\vec{Q}}
  \and \\
  := \clift{P_{0}\otimes Q_{0},\ldots,P_{0}\otimes Q_{n},\ldots,P_{m}\otimes Q_{0},\ldots,P_{m}\otimes Q_{n}}
\end{mathpar}

\begin{remark}
  Observe that arities of tensored abstractions matches arities of
  tensored concretions if the original arities matched. Note also that
  the length of the arities corresponds to the increase in dimension
  we see in ordinary vector space tensor product.
\end{remark}

\begin{remark}
  Operationally, this definition distributes the tensor down to
  components ``linked'' by summation. Tensor over summation is
  intriguing in that it mixes names. Moreover, as a consequence of the
  way it mixes names we have the identities for all $x \in \QProc$ and
  $P,Q \in \Proc$

  \begin{mathpar}
    (x \cdot P) \otimes Q \equiv x \cdot (P \otimes Q) \equiv P \otimes (x \cdot Q)
    \and
    P \otimes \pzero \equiv P
  \end{mathpar}

  that the reader is invited to verify.
\end{remark}

\subsubsection{Annihilation}
\begin{mathpar}
  P^{\perp} := \{ Q | \forall R. P|Q \red^{*} R \Rightarrow R \red^{*} \pzero \}
  \and \\
  P^{\underline{\perp}} := \Sigma_{Q \in P^{\perp}} \quotep{Q}?(y).(\dropn{y}|Q) | \Sigma_{Q \in P^{\perp}} \quotep{Q}\clift{\Box}
\end{mathpar}

\paragraph{Discussion} The reader will note that $P^{\perp}$ is a
\emph{set} of processes, while $P^{\underline{\perp}}$ is a
\emph{context}. We call the set $P^{\perp}$ the \emph{annihilators} of
$P$. The parallel composition of a process in the annihilators of $P$
with $P$ will result in a process, the state space of which has all
paths eventually leading to $\pzero$. Execution may endure loops; but
under reasonable conditions of fairness (naturally guaranteed under
most notions of bisimulation) such a composite process cannot get
stuck in such a loop and will, eventually pop out and terminate.

The context $P^{\underline{\perp}}$ is ready and willing to ``take the
$P$ out of'' the process to which it is applied. It will effectively
transmit the code of the process to which it is applied to one of the
annihilators and run the process against it.

\subsubsection{Evaluation}
We fix $M$ a domain of fully abstract interpretation with an equality
coincident with bisimulation. We take $\meaningof{\cdot} : \Proc \to
M$ to be the map interpreting processes and $\nmeaningof{\cdot} : \M
\to Proc$ to be the map running the other way. Then we define

\begin{mathpar}
  \int P := \nmeaningof{\meaningof{P}}
\end{mathpar}

\paragraph{Discussion}
There are many fully abstract interpretations of Milner's
$\pi$-calculus. Any of them can be used as a basis for interpreting
the reflective calculus here. Equipped with such a domain it is
largely a matter of grinding through to check that the Yoneda
construction for the normalization-by-evaluation program can be
extended to this setting.

\begin{remark}
  The reader is invited to verify that $\int (P^{\underline{\perp}}[P]) = 0$.
\end{remark}

\subsection{Quantum mechanics}

Table \ref{tbl:core_qm_op_defns} gives the core operational definitions

\begin{table}[htp]\label{tbl:core_qm_op_defns}
  \center{
    \fbox{
      \begin{tabular}{c|c}
        quantum mechanics & process calculus \\
        \hline
        scalar & $x := \quotep{P}$ \\
        state vector & $\state{P} := P$ \\
        dual & $\state{P}^{*} := \event{P^{\underline{\perp}}} := \quotep{P^{\underline{\perp}}}[-]$ \\
        matrix & $ \Sigma_{\alpha} \state{P_{\alpha}}x_{\alpha}\event{Q_{\alpha}}$ \\
        vector addition & $\state{P} + \state{Q} := \state{P | Q}$ \\
        tensor product & $\state{P} \otimes \state{Q} := \state{P \otimes Q}$ \\
        inner product & $\innerprod{P}{Q} := \quotep{\int P^{\underline{\perp}}[Q]}$ \\
      \end{tabular}
    }
  }
  \caption{QM - operational definitions}
\end{table}

where

\begin{mathpar}
  \prmatrix{P}{Q} := \fprmatrix{P}{\quotep{\pzero}}{Q}
  \and
  \fprmatrix{P}{x}{Q} := (\state{P},x,\event{Q})
  \and
  (\fprmatrix{P}{x}{Q})(\state{R}) := x \cdot \innerprod{Q}{R} \cdot \state{P}
  \and
  (\fprmatrix{P}{x}{Q})(\event{R}) := x \cdot \innerprod{R}{P} \cdot \event{Q}
\end{mathpar}

\paragraph{Discussion}
As promised: vectors (aka states) are represented as processes; duals
as contextual duals; inner product definition should be compared with
standard inner product definition for ....

\begin{remark}
  Assuming $\int (P^{\underline{\perp}}[P]) = 0$, the reader is
  invited to verify that $(\fprmatrix{P}{x}{P})(\state{P}) = x \cdot \state{P}$.
\end{remark}

\begin{remark}
  The reader is invited to verify that $\innerprod{P}{Q}$ could
  equally well have been written $\quotep{\int \stackrel{\vee}{x}}$
  where $x = \event{P^{\underline{\perp}}}(Q)$.

  One of the motivations for this remark is that there is another way
  to factor these operations. We could package up evaluation in the dual:

  \begin{mathpar}
    \state{P}^{*} := \event{\int P^{\underline{\perp}}} := \quotep{\int P^{\underline{\perp}}}[-]
  \end{mathpar}

  and then have inner product defined by
  
  \begin{mathpar}
    \innerprod{P}{Q} := \event{P}(Q)
  \end{mathpar}

  Hopefully, experience with the calculations will provide guidance on
  the best factoring.
\end{remark}

\begin{remark}
  Assuming $\int (P^{\underline{\perp}}[P]) = 0$, the reader is
  invited to verify that $\forall P,Q. (\prmatrix{0}{Q})(\state{0}) =
  \state{0}$ and dually $(\prmatrix{P}{0})(\event{0}) = \event{0}$.
\end{remark}

\begin{remark}
  i'm a little worried that i don't (yet) have proper support for
  complex conjugacy. But, the observation above may give us a
  clue. According to Abramsky, it must be the case that the scalars
  are iso to the homset of the identity for the tensor -- which the
  observation above characterizes. 

  For now, we will simply bookmark the notion with $\overline{x}$.
\end{remark}

\subsubsection{Adjointness}

We need to give a definition of $(\cdot)^{\dagger}$ for matrices. The
obvious candidate definition is
\begin{mathpar}
(\Sigma_{\alpha}\fprmatrix{P_{\alpha}}{x_{\alpha}}{Q_{\alpha}})^{\dagger}
= \Sigma_{\alpha}\fprmatrix{(Q_{\alpha}^{\underline{\perp}})^{*}}{\overline{x}_{\alpha}}{P_{\alpha}^{\underline{\perp}}} 
\end{mathpar}

But, $(Q_{\alpha}^{\underline{\perp}})^{*}$ requires a name along
which to communicate the process to achieve the context application.

\subsubsection{Basis for a basis}
If processes label states and ``addition'' of states (a.k.a. vector
addition) is interpreted as parallel composition, what corresponds to
notions of linear independence and basis? Here, we recall that Yoshida
has developed a set of \emph{combinators} for an asynchronous verison
of Milner's $\pi$-calculus. These are a finite set of processes such
any process can be expressed as parallel composition of these
combinators together with liberal uses of the new operator and
replication. We can simply give a translation of these into the
present calculus and have reasonable expectation that the property
carries over. That is, that the resultant set allows to express all
processes via parallel composition. Note, however, that there is no
new operator or replication in this calculus. As a result, we expect
that the corresponding set is actually infinite. That is, we expect
that the space is actually infinite dimensional.

\begin{remark}
  The attentive reader may be a bit concerned. Certainly, the
  collection $S$, $K$ and $I$ is a finite set of
  combinators. Shouldn't we expect to see a finite set of combinators
  for an effectively equivalent system? i am very sympathetic to this
  critique and feel it warrants full attention. On the other hand, i
  also have in mind the following analogy. The natural numbers, as a
  monoid under addition, has exactly $1$ generator, while the natural
  numbers, as a monoid under multiplication, has countably many
  generators (the primes). We observe that the application of the
  lambda calculus is much less resource sensitive than the parallel
  composition of the $\pi$-calculus. Could it be the case that we have
  an analogy of the form
  
  \begin{mathpar}
    m + n : MN :: m*n : M|N
  \end{mathpar}

  giving a similar blow up in the set of ``primes''?  This is such a
  wonderful thought that, even if it's not true, i think it's worth
  writing down.
\end{remark}
 

\documentclass[12pt]{llncs}
%\documentclass{jktr}

\usepackage[pdftex]{hyperref}                   
\usepackage {listings}
\usepackage {mathpartir}
\usepackage{bcprules}
%\usepackage{listings}
                       
\usepackage{graphicx} 
%\usepackage[margins=2.5cm,nohead,nofoot]{geometry}
%\usepackage{geometry}
\usepackage{amsfonts}
\usepackage{amstext}
\usepackage{latexsym}
\usepackage{amssymb}
\usepackage{color}


%\include{myPreamble}
\include{qm2pi.local} 

%\ifpdf
%\usepackage[pdftex]{graphicx}
%\else
%\usepackage{graphicx}
%\fi

 % \ifpdf
%  \usepackage{pdfsync}
%  \if


%\title{Brief Article}
%\author{David F. Snyder}
%\author{L.G. Meredith}

%\address{Dept. of Math., Texas State University--San Marcos, San Marcos, TX 78666}
       
\pagestyle{empty}


\begin{document}

\lstset{language=[Objective]Caml,frame=shadowbox}

\input{qm2pi.front}

% section front matter (end)

\input{qm2pi.intro} 
 
% section introduction (end)

% \input{qm2pi.knotations} 

% section notation (end)

\input{qm2pi.process.calculi} 

% section concurrent_process_calculi_and_spatial_logics_ (end)
    
%\input{qm2pi.knots2pi} 

%\input{qm2pi.trefoil} 

%\input{qm2pi.mainthm} 

% subsection basic_interpretation (end)

%\input{qm2pi.rho.presentation} 
\subsection{The syntax and semantics of the notation system}\label{sub:the_syntax_and_semantics_of_the_notation_system} % (fold)

We now summarize a technical presentation of the calculus that
embodies our theory of dynamics. The typical presentation of such a
calculus follows the style of giving generators and relations on
them. The grammar, below, describing term constructors, freely
generates the set of processes, $\Proc$. This set is then quotiented
by a relation known as structural congruence and it is over this set
that the notion of dynamics is expressed. This presentation is
essentially that of \cite{MeredithR05} with the addition of
polyadicity and summation. For readability we have relegated some of
the technical subtleties to an appendix.

\subsubsection{Process grammar}\label{subsub:process_grammar}

\begin{mathpar}
  \inferrule* [lab=synchronization] {} {{M} \bc \pzero \;|\; x?F \;|\; x!C }
  \and
  \inferrule* [lab=abstraction] {} {{F} \bc (x)P}
  \and
  \inferrule* [lab=concretion] {} {{C} \bc \langle Q \rangle}
  \and
  \inferrule* [lab=process] {} {{P,Q} \bc M \;| \;P|Q \;|\; @{x}}
  \and
  \inferrule* [lab=name] {} {{x} \bc \quotep{P}}
\end{mathpar} 

Note that $\vec{x}$ (resp. $\vec{P}$) denotes a vector of names
(resp. processes) of length $|\vec{x}|$ (resp. $|\vec{P}|$). We adopt
the following useful abbreviations.

\begin{mathpar}
   x?(\vec{y}).P := x.(\vec{y})P \and  x\clift{\vec{P}} := x.\clift{\vec{P}}
   \and x!(y) := \lift{x}{\dropn{y}}
   \and \Pi_{i=0}^{n-1}P_i := P_0 | \ldots | P_{n-1}
\end{mathpar}

\subsubsection{Structural congruence}

\paragraph{Free and bound names and alpha-equivalence.} At the
core of structural equivalence is alpha-equivalence which identifies
process that are the same up to a change of variable. Formally, we
recognize the distinction between free and bound names. The free names
of a process, $\freenames{P}$, may be calculated recursively as
follows:

\begin{mathpar}
\freenames{\pzero} := \emptyset
  \and \\
  \freenames{x?(y).P} := \{ x \} \cup (\freenames{P} \setminus \{ y \})
  \and 
  \freenames{x!\langle P \rangle} := \{ x \} \cup \{ P \} 
  \and \\
  \freenames{P|Q} := \freenames{P} \cup \freenames{Q}
  \and \\
  \freenames{@{x}} := \{ x \}
\end{mathpar}

$\pi$
$\quotep{\pi}$

$\freenames{-} : \pi \to \mathcal{P}(\quotep{\pi})$

\begin{eqnarray*}
  \freenames{\pzero} & := & \emptyset \\
  \freenames{x?(y).P} & := & \{ x \} \cup (\freenames{P} \setminus \{ y \}) \\
  \freenames{x!\langle P \rangle} & := & \{ x \} \cup \{ P \} \\
  \freenames{P|Q} & := & \freenames{P} \cup \freenames{Q} \\
  \freenames{\dropn{x}} & := & \{ x \}
\end{eqnarray*}

The bound names of a process, $\boundnames{P}$, are those names occurring in $P$
that are not free. For example, in $x?(y).0$, the name $x$ is free, while $y$ is bound.

\begin{mathpar}
  \inferrule* [lab=monoidal-laws] {} { P|Q \equiv Q|P \and P|0 \equiv P \and P|(Q|R) \equiv (P|Q)|R }
\end{mathpar}

\begin{mathpar}
  \inferrule* [lab=alpha-equivalence] {} { (x)P \equiv (y)P\{y/x\} \and y \not\in \freenames{P} }
\end{mathpar}

\begin{definition}
Then two processes, $P,Q$, are alpha-equivalent if $P = Q\{\vec{y}/\vec{x}\}$ for
some $\vec{x} \in \boundnames{Q},\vec{y} \in \boundnames{P}$, where $Q\{\vec{y}/\vec{x}\}$
denotes the capture-avoiding substitution of $\vec{y}$ for $\vec{x}$ in $Q$.
\end{definition}

\begin{definition}
  The {\em structural congruence} \cite{SangiorgiWalker} , $\equiv$,
  between processes is the least congruence containing
  alpha-equivalence, satisfying the abelian monoid laws
  (associativity, commutativity and $\pzero$ as identity) for parallel
  composition $|$ and for summation $+$.
\end{definition}

\subsection{Name equivalence}

We take name equivalence, written $\nameeq$, to be the smallest
equivalence relation generated by the following rules.

\begin{mathpar}
\inferrule*[lab=Quote-drop]
{ }
{ \quotep{@{x}} \nameeq x }

\inferrule*[lab=Struct-equiv]
{ P \scong Q }
{ \quotep{P} \nameeq \quotep{Q} }
\end{mathpar}

The astute reader will have noticed that the mutual recursion of names
and processes imposes a mutual recursion on alpha-equivalence and
structural equivalence via name-equivalence. Fortunately, all of this
works out pleasantly and we may calculate in the natural way, free of
concern. The reader interested in the details is referred to the
appendix \ref{appendix:rho_details}.

\subsection{Substitution}

We use $\Proc$ for the set of processes, $\QProc$ for the set of
names, and $\id{\{}\vec{y} / \vec{x} \id{\}}$ to denote partial maps,
$s : \QProc \rightarrow \QProc$. A map, $s$ lifts, uniquely, to a map
on process terms, $\widehat{s} : \Proc \rightarrow \Proc$ by the
following equations.

\begin{mathpar}
  (0) \psubstp{Q}{P} := 0 \\
  (R \juxtap S) \psubstp{Q}{P}
  :=    
  (R)\psubstp{Q}{P} \juxtap (S) \psubstp{Q}{P} \\
  (x?(y).R) \psubstp{Q}{P}    
  :=    
  (x)\substp{Q}{P} (z)\concat( (R \psubstn{z}{y}) \psubstp{Q}{P} ) \\
  (\lift{x}{R}) \psubstp{Q}{P}  
  :=
  \lift{(x)\substp{Q}{P}}{ R \psubstp{Q}{P} } \\
%   (\dropn{x})  \psubstp{Q}{P}       
%   := 
%   \left\{ 
%     \begin{array}{ccc} 
%       \dropn{\quotep{Q}} & & x \nameeq \quotep{P} \\
%       \dropn{x} & & otherwise \\
%     \end{array}
%   \right. 
  (\dropn{x})  \psubstp{Q}{P}       
  := 
  \left\{ 
    \begin{array}{ccc} 
      Q & & x \nameeq \quotep{P} \\
      \dropn{x} & & otherwise \\
    \end{array}
  \right.
\end{mathpar}
 

where

\begin{eqnarray}
  (x)\id{\{} \lpquote Q \rpquote / \lpquote P \rpquote \id{\}}            = 
  \left\{ 
    \begin{array}{ccc}
      \lpquote Q \rpquote & & x \nameeq \lpquote P \rpquote \\
      x & & otherwise \\
    \end{array}
  \right. \nonumber
\end{eqnarray}

and $z$ is chosen distinct from $\quotep{P}$, $\quotep{Q}$, the free
names in $Q$, and all the names in $R$. Our $\alpha$-equivalence will
be built in the standard way from this substitution.

\begin{remark}\label{rem:no_self_referential_names}
  One consequence of these definitions is that $\forall P. \quotep{P}
  \not\in \freenames{P}$.
\end{remark}

\subsection{ Dynamic quote: an example }

Anticipating something of what's to come, consider applying the
substitution, $\widehat{\id{\{}u / z \id{\}}}$, to the following pair
of processes, $\lift{w}{y!(z)}$ and $w[ \lpquote y!(z) \rpquote ]$.

\begin{eqnarray}
	\lift{w}{y!(z)}\widehat{\id{\{}u / z \id{\}}}
		& = &
		\lift{w}{y!(u)} \nonumber\\
	w[ \lpquote y!(z) \rpquote ] \widehat{ \id{\{}u / z \id{\}} }
		& = &
		w[ \lpquote y!(z) \rpquote ] \nonumber
\end{eqnarray}

Because the body of the process between quotes is impervious to
substitution, we get radically different answers. In fact, by
examining the first process in an input context,
e.g. $x?(z).\lift{w}{y!(z)}$, we see that the process under the lift
operator may be shaped by prefixed inputs binding a name inside it. In
this sense, the lift operator will be seen as a way to dynamically
construct processes before reifying them as names.

Finally equipped with these standard features we can present the
dynamics of the calculus.

\subsubsection{Operational semantics} 

Finally, we introduce the computational dynamics. What marks these
algebras as distinct from other more traditionally studied algebraic
structures, e.g. vector spaces or polynomial rings, is the manner in
which dynamics is captured. In traditional structures, dynamics is typically
expressed through morphisms between such structures, as in linear maps
between vector spaces or morphisms between rings. In algebras
associated with the semantics of computation, the dynamics is
expressed as part of the algebraic structure itself, through a
reduction reduction relation typically denoted by $\red$. Below, we
give a recursive presentation of this relation for the calculus used
in the encoding.

$\red \subseteq \pi \times \pi$
$\red : \pi \to \mathcal{P}(\pi)$

\begin{mathpar}
  \inferrule* [lab=Comm] { \textsf{match}( x_{src}, x_{trgt} ) } { x_{trgt}?(y)P \; | \; x_{src}!\langle {Q} \rangle \red P\{\quotep{Q}/y}\} }
  \and \\
  \inferrule* [lab=Par] {{P} \red {P}'} {{{P} | {Q}} \red {{P}' | {Q}}}
  \and
  \inferrule* [lab=Equiv]{{{P} \scong {P}'} \andalso {{P}' \red {Q}'} \andalso {{Q}' \scong {Q}}}{{P} \red {Q}}
\end{mathpar}

\begin{eqnarray*}
  match_{\equiv} (\quotep{P},\quotep{Q}) & := & P \equiv Q \\
  match_{\dagger}(\quotep{P},\quotep{Q}) & := & \forall R. P|Q \red^{*} R => R \red^{*} 0 \\
  match_{K}(\quotep{P},\quotep{Q}) & := & K \mbox{ for some context } K
\end{eqnarray*}

$u?(x)P | u!\langle Q \rangle \red P\{\quotep{Q}/x\}$

%We write $\wred$ for $\red^*$, and $P\red$ if $\exists Q $ such that $ P \red Q$.
We write $P\red$ if $\exists Q $ such that $ P \red Q$ and $P\not\red$, otherwise.

\section{Replication}

As mentioned before, it is known that replication (and hence
recursion) can be implemented in a higher-order process algebra
\cite{SangiorgiWalker}. As our first example of calculation with the
machinery thus far presented we give the construction explicitly in
the {\rhoc}.

\begin{eqnarray}
	D_{x} & := & \prefix{x}{y}{(\binpar{\outputp{x}{y}}{@{y}})} \nonumber\\
	\bangp_{x}{P} & := & \binpar{{x}!\langle{\binpar{D_{x}}{P}}\rangle}{D_{x}} \nonumber
\end{eqnarray}

\begin{eqnarray}
	\bangp_{x}{P} & & \nonumber\\
	=
	& {x}!\langle{(\prefix{x}{y}{(\outputp{x}{y} | @{y})) | P}}\rangle 
	      | \prefix{x}{y}{(\outputp{x}{y} | @{y})} & \nonumber\\
	\red
	& (\outputp{x}{y} | @{y})\substn{\quotep{(\prefix{x}{y}{(@{y} | \outputp{x}{y})) | P}}}{y} & \nonumber\\
	=
	& \outputp{x}{\quotep{(\prefix{x}{y}{(\outputp{x}{y} | @{y})) | P}}}
	  | {(\prefix{x}{y}{(\outputp{x}{y} | @{y})) | P}} & \nonumber\\
	\red
	& \ldots & \nonumber\\
	\red^*
	& P | P | \ldots & \nonumber
\end{eqnarray}

Of course, this encoding, as an implementation, runs away, unfolding
$\bangp{P}$ eagerly. A lazier and more implementable replication
operator, restricted to input-guarded processes, may be obtained as follows.

\begin{eqnarray}
\bangp{\prefix{u}{v}{P}} 
	:= 
	\binpar{\lift{x}{\prefix{u}{v}{(\binpar{D(x)}{P})}}}{D(x)} \nonumber
\end{eqnarray}

\begin{remark}
  Note that the lazier definition still does not deal with summation
  or mixed summation (i.e. sums over input and output). The reader is
  invited to construct definitions of replication that deal with these
  features. 

  Further, the definitions are parameterized in a name, $x$. Can you,
  gentle reader, make a definition that eliminates this parameter and
  guarantees no accidental interaction between the replication
  machinery and the process being replicated -- i.e. no accidental
  sharing of names used by the process to get its work done and the
  name(s) used by the replication to effect copying. This latter
  revision of the definition of replication is crucial to obtaining
  the expected identity $!!P \sim !P$.
\end{remark}

\begin{remark}\label{rem:paradoxical_combinator}
  The reader familiar with the lambda calculus will have noticed the
  similarity between $D$ and the paradoxical combinator.

  [Ed. note: the existence of this seems to suggest we have to be more
  restrictive on the set of processes and names we admit if we are to
  support no-cloning.]
\end{remark}

\subsubsection{Bisimulation}

The computational dynamics gives rise to another kind of equivalence,
the equivalence of computational behavior. As previously mentioned
this is typically captured \emph{via} some form of bisimulation.

% The notion we use in this paper is weak barbed bisimulation
% \cite{milner91polyadicpi}.

The notion we use in this paper is derived from weak barbed
bisimulation \cite{milner91polyadicpi}. 

\begin{definition}
An \emph{observation relation}, $\downarrow_{\mathcal N}$, over a set
of names, $\mathcal N$, is the smallest relation satisfying the rules
below.

\infrule[Out-barb]{y \in {\mathcal N}, \; x \nameeq y}
		  {\outputp{x}{v} \downarrow_{\mathcal N} x}
\infrule[Par-barb]{\mbox{$P\downarrow_{\mathcal N} x$ or $Q\downarrow_{\mathcal N} x$}}
		  {\binpar{P}{Q} \downarrow_{\mathcal N} x}

We write $P \Downarrow_{\mathcal N} x$ if there is $Q$ such that 
$P \wred Q$ and $Q \downarrow_{\mathcal N} x$.
\end{definition}

\begin{definition}
%\label{def.bbisim}
An  ${\mathcal N}$-\emph{barbed bisimulation} over a set of names, ${\mathcal N}$, is a symmetric binary relation 
${\mathcal S}_{\mathcal N}$ between agents such that $P\rel{S}_{\mathcal N}Q$ implies:
\begin{enumerate}
\item If $P \red P'$ then $Q \wred Q'$ and $P'\rel{S}_{\mathcal N} Q'$.
\item If $P\downarrow_{\mathcal N} x$, then $Q\Downarrow_{\mathcal N} x$.
\end{enumerate}
$P$ is ${\mathcal N}$-barbed bisimilar to $Q$, written
$P \wbbisim_{\mathcal N} Q$, if $P \rel{S}_{\mathcal N} Q$ for some ${\mathcal N}$-barbed bisimulation ${\mathcal S}_{\mathcal N}$.
\end{definition}

$\mathcal{R} \subseteq \pi \times \pi$

$P \mathcal{R} Q => \forall P'. P \red P' \Rightarrow \exists Q'. Q \red Q', P' \mathcal{R} Q'$

$P \vdash x \Rightarrow Q \vdash x$

\begin{mathpar}
  \inferrule*[lab=Out-barb]{x \nameeq y}{{y}!\langle{Q}\rangle \vdash x}
  \and
  \inferrule*[lab=Par-barb]{\mbox{$P\vdash x$ or $Q\vdash x$}}{\binpar{P}{Q} \vdash x}
\end{mathpar}

\subsubsection{Contexts}

One of the principle advantages of computational calculi like the
$\pi$-calculus is a well-defined notion of context,
contextual-equivalence and a correlation between
contextual-equivalence and notions of bisimulation. The notion of
context allows the decomposition of a process into (sub-)process and
its syntactic environment, its context. Thus, a context may be
thought of as a process with a ``hole'' (written $\Box$) in it. The
application of a context $M$ to a process $P$, written $M[P]$, is
tantamount to filling the hole in $M$ with $P$. In this paper we do
not need the full weight of this theory, but do make use of the notion
of context in the proof the main theorem. 

\begin{mathpar}
  \inferrule* [lab=summation] {} {{M_{M},M_{N}} \bc \Box \;|\; x.M_{A} \;|\; M_{M}+M_{N}}
  \and
  \inferrule* [lab=agent] {} {{M_{A}} \bc (\vec{x})M_{P} \;| \; \clift{P_0,\ldots,M_{P},\ldots,P_N}}
  \and \\
  \inferrule* [lab=process] {} {{M_{P}} \bc M_{N} \;| \;P|M_{P} }
\end{mathpar} 

\begin{mathpar}
  \inferrule* [lab=sychronization] {} {M_{N} \bc \Box \;|\; x?M_{F} \;|\; x!M_{C}}
  \and
  \inferrule* [lab=abstraction] {} {{M_{F}} \bc (x)M_{P} }
  \and
  \inferrule* [lab=concretion] {} {{M_{C}} \bc \langle M_{P} \rangle }
  \and \\
  \inferrule* [lab=process] {} {{M_{P}} \bc M_{N} \;| \;P|M_{P} }
\end{mathpar}

\begin{definition}[contextual application] Given a context $M$, and
  process $P$, we define the \emph{contextual application}, $M[P] :=
  M\{P/\Box\}$. That is, the contextual application of M to P is the
  substitution of $P$ for $\Box$ in $M$.
\end{definition}

$\meaningof{-} : L \to \mathcal{P}(\pi)$

\begin{mathpar}
  \inferrule* [lab=collection] {} {\meaningof{true} = \pi, \and \meaningof{~E} = \pi \setminus \meaningof{E}, \and \meaningof{E_{1} \& E_{2}} = \meaningof{E_{1}} \cap \meaningof{E_{2}}}
\end{mathpar}

\begin{mathpar}
  \inferrule* [lab=structure] {} {\meaningof{0} = \{ P \in \pi | P \equiv 0 \}, \and \\ \meaningof{E_1 | E_2} = \{ P \in \pi | P \equiv P_{1} | P_{2}, P_{1} \in \meaningof{E_{1}}, P_{2} \in \meaningof{E_2}\} }
\end{mathpar}

\begin{mathpar}
 \inferrule* [lab=behavior] {} {\meaningof{\langle a?b \rangle E} = \{ P \in \pi | P \equiv Q | u?(y)P', \\ \and \\\\ \and \\ \;\;\; u \in \meaningof{a}, \forall z.P'\{z/y\} \in \meaningof{E\{z/b\}}\}, \and \\ \meaningof{a!E} = \{ P \in \pi | P \equiv Q | x!\langle P' \rangle, x \in \meaningof{a} P' \in \meaningof{E}\} }
\end{mathpar}

\begin{mathpar}
 \inferrule* [lab=nominal] {} {\meaningof{\quotep{E}} = \{ \quotep{P} \in \quotep{\pi} | P \in \meaningof{E} \}, \and \meaningof{\quotep{P}} = \{ \quotep{Q} \in \quotep{\pi} | P \equiv Q \} \and \\ \meaningof{@\quotep{E}} = \{ P \in \pi | P \equiv @x, x \in \meaningof{E} \}}
\end{mathpar}

\begin{eqnarray*}
  \\
  \meaningof{-} : TS \to ST
\end{eqnarray*}

\begin{eqnarray*}
  \\
  L : TS \to ST
\end{eqnarray*}

\begin{eqnarray*}
  \\
  P \models E \iff P \in \meaningof{E}
\end{eqnarray*}

\begin{eqnarray*}
  P \approx_{L} Q \iff \forall E \in L. P \models E \iff Q \models E
\end{eqnarray*}

\begin{eqnarray*}
  P \approx_{K} Q
\end{eqnarray*}

\begin{eqnarray*}
  P \approx Q
\end{eqnarray*}

$\approx_{K} = \approx = \approx_{L}$

\subsubsection{Contextual duality}

Note that contexts extend the quotation operation to a family of
operations from processes to names. Given a context, $M$, we can
define a \emph{nominal context}, $\quotep{M}$ by $\quotep{M}[P] :=
\quotep{M[P]}$. To foreshadow what is to come we observe that these
operations enjoy a duality with processes very much like the duality
between vectors and maps from vectors to scalars.

Further, because the calculus is essentially higher-order, we have a
correspondence between contexts and processes. More specifically,
given a name $x$ and a context $M$ we can construct $M^{*}_{x}$ such
that 

\begin{mathpar}
  M^{*}_{x} | \lift{x}{P} \red M[P]
\end{mathpar}

namely,

\begin{mathpar}
  M^{*}_{x} := x?(u).M[\dropn{u}]
\end{mathpar}

The dependence of $M^{*}_{x}$ on a name makes it an abstraction, 

\begin{mathpar}
  M^{*} := (x)x?(u).M[\dropn{u}]
\end{mathpar}

\subsection{Additional notation}

It will sometimes be convenient to denote the process a name
quotes. We already have the notation $x = \quotep{P}$, but it will be
convenient to introduce an alternate notation, $\procn{x}$, when we
want to emphasize the connection to the use of the name. Note that, by
virtue of name equivalence, $\quotep{\procn{x}} \nameeq x$; so, the
notation is consistent with previous definitions.

Further, because names have structure it is possible to effect
substitutions on the basis of that structure. This means we need to
upgrade our notation for substitutions, which we accomplish by
adapting comprehension notation. Thus,

\begin{mathpar}
  P\{ y / x : x \in S \}
\end{mathpar}

is interpreted to mean the process derived from P by replacing (in a
capture-avoiding manner) each occurrence of $x$ in $S$ by $y$. For example,

\begin{mathpar}
  P\{ \quotep{\procn{x}|\procn{x}} / x : x \in \freenames{P} \}
\end{mathpar}

will replace each (occurrence) of a free name $x$ in $P$ by
$\quotep{\procn{x}|\procn{x}}$.

Also, we will avail ourselves of the notation $x^{L}$ and $x^{R}$ to
denote injections of a name into disjoint copies of the name
space. There are numerous ways to accomplish this. One example can be
found in \cite{MeredithR05}. This notation overloads to vectors of
names: $\vec{x}^{\pi} := (x_{i}^{\pi} \; : \; 0 \leq i < |\vec{x}| )$ where $\pi \in \{L,R\}$.

We also use $P^{\Box} := P|\Box$.

In \cite{MeredithR05} an interpretation of the new operator is
given. It turns out that there are several possible interpretations
all enjoying the requisite algebraic properties of the operator (see
\cite{milner91polyadicpi}). We will therefore make liberal use of
$(\nu\; \vec{x})P$.

% subsection the_syntax_and_semantics_of_the_notation_system (end)   

\input{qm2pi.qmops} 

\input{qm2pi.sterngerlach} 

\input{qm2pi.metric} 

% section concurrent_process_calculi (end)

%\input{qm2pi.proofsketch}

% section proof sketch (end)

%\input{qm2pi.slviaknots} 

% section spatial logic via knots (end)

\input{qm2pi.conclusion}

% section conclusion (end)

%\input{qm2pi.dtcodes} 

% section wiring algorithm (end)

\input{qm2pi.ack} 

% section acknowledgments (end)

\newpage


\bibliographystyle{plain}   
\bibliography{../../biblios/main.bib}

\input{qm2pi.rhodetails}

\end{document}

 

\documentclass[12pt]{llncs}
%\documentclass{jktr}

\usepackage[pdftex]{hyperref}                   
\usepackage {listings}
\usepackage {mathpartir}
\usepackage{bcprules}
%\usepackage{listings}
                       
\usepackage{graphicx} 
%\usepackage[margins=2.5cm,nohead,nofoot]{geometry}
%\usepackage{geometry}
\usepackage{amsfonts}
\usepackage{amstext}
\usepackage{latexsym}
\usepackage{amssymb}
\usepackage{color}


%\include{myPreamble}
\include{qm2pi.local} 

%\ifpdf
%\usepackage[pdftex]{graphicx}
%\else
%\usepackage{graphicx}
%\fi

 % \ifpdf
%  \usepackage{pdfsync}
%  \if


%\title{Brief Article}
%\author{David F. Snyder}
%\author{L.G. Meredith}

%\address{Dept. of Math., Texas State University--San Marcos, San Marcos, TX 78666}
       
\pagestyle{empty}


\begin{document}

\lstset{language=[Objective]Caml,frame=shadowbox}

\input{qm2pi.front}

% section front matter (end)

\input{qm2pi.intro} 
 
% section introduction (end)

% \input{qm2pi.knotations} 

% section notation (end)

\input{qm2pi.process.calculi} 

% section concurrent_process_calculi_and_spatial_logics_ (end)
    
%\input{qm2pi.knots2pi} 

%\input{qm2pi.trefoil} 

%\input{qm2pi.mainthm} 

% subsection basic_interpretation (end)

%\input{qm2pi.rho.presentation} 
\subsection{The syntax and semantics of the notation system}\label{sub:the_syntax_and_semantics_of_the_notation_system} % (fold)

We now summarize a technical presentation of the calculus that
embodies our theory of dynamics. The typical presentation of such a
calculus follows the style of giving generators and relations on
them. The grammar, below, describing term constructors, freely
generates the set of processes, $\Proc$. This set is then quotiented
by a relation known as structural congruence and it is over this set
that the notion of dynamics is expressed. This presentation is
essentially that of \cite{MeredithR05} with the addition of
polyadicity and summation. For readability we have relegated some of
the technical subtleties to an appendix.

\subsubsection{Process grammar}\label{subsub:process_grammar}

\begin{mathpar}
  \inferrule* [lab=synchronization] {} {{M} \bc \pzero \;|\; x?F \;|\; x!C }
  \and
  \inferrule* [lab=abstraction] {} {{F} \bc (x)P}
  \and
  \inferrule* [lab=concretion] {} {{C} \bc \langle Q \rangle}
  \and
  \inferrule* [lab=process] {} {{P,Q} \bc M \;| \;P|Q \;|\; @{x}}
  \and
  \inferrule* [lab=name] {} {{x} \bc \quotep{P}}
\end{mathpar} 

Note that $\vec{x}$ (resp. $\vec{P}$) denotes a vector of names
(resp. processes) of length $|\vec{x}|$ (resp. $|\vec{P}|$). We adopt
the following useful abbreviations.

\begin{mathpar}
   x?(\vec{y}).P := x.(\vec{y})P \and  x\clift{\vec{P}} := x.\clift{\vec{P}}
   \and x!(y) := \lift{x}{\dropn{y}}
   \and \Pi_{i=0}^{n-1}P_i := P_0 | \ldots | P_{n-1}
\end{mathpar}

\subsubsection{Structural congruence}

\paragraph{Free and bound names and alpha-equivalence.} At the
core of structural equivalence is alpha-equivalence which identifies
process that are the same up to a change of variable. Formally, we
recognize the distinction between free and bound names. The free names
of a process, $\freenames{P}$, may be calculated recursively as
follows:

\begin{mathpar}
\freenames{\pzero} := \emptyset
  \and \\
  \freenames{x?(y).P} := \{ x \} \cup (\freenames{P} \setminus \{ y \})
  \and 
  \freenames{x!\langle P \rangle} := \{ x \} \cup \{ P \} 
  \and \\
  \freenames{P|Q} := \freenames{P} \cup \freenames{Q}
  \and \\
  \freenames{@{x}} := \{ x \}
\end{mathpar}

$\pi$
$\quotep{\pi}$

$\freenames{-} : \pi \to \mathcal{P}(\quotep{\pi})$

\begin{eqnarray*}
  \freenames{\pzero} & := & \emptyset \\
  \freenames{x?(y).P} & := & \{ x \} \cup (\freenames{P} \setminus \{ y \}) \\
  \freenames{x!\langle P \rangle} & := & \{ x \} \cup \{ P \} \\
  \freenames{P|Q} & := & \freenames{P} \cup \freenames{Q} \\
  \freenames{\dropn{x}} & := & \{ x \}
\end{eqnarray*}

The bound names of a process, $\boundnames{P}$, are those names occurring in $P$
that are not free. For example, in $x?(y).0$, the name $x$ is free, while $y$ is bound.

\begin{mathpar}
  \inferrule* [lab=monoidal-laws] {} { P|Q \equiv Q|P \and P|0 \equiv P \and P|(Q|R) \equiv (P|Q)|R }
\end{mathpar}

\begin{mathpar}
  \inferrule* [lab=alpha-equivalence] {} { (x)P \equiv (y)P\{y/x\} \and y \not\in \freenames{P} }
\end{mathpar}

\begin{definition}
Then two processes, $P,Q$, are alpha-equivalent if $P = Q\{\vec{y}/\vec{x}\}$ for
some $\vec{x} \in \boundnames{Q},\vec{y} \in \boundnames{P}$, where $Q\{\vec{y}/\vec{x}\}$
denotes the capture-avoiding substitution of $\vec{y}$ for $\vec{x}$ in $Q$.
\end{definition}

\begin{definition}
  The {\em structural congruence} \cite{SangiorgiWalker} , $\equiv$,
  between processes is the least congruence containing
  alpha-equivalence, satisfying the abelian monoid laws
  (associativity, commutativity and $\pzero$ as identity) for parallel
  composition $|$ and for summation $+$.
\end{definition}

\subsection{Name equivalence}

We take name equivalence, written $\nameeq$, to be the smallest
equivalence relation generated by the following rules.

\begin{mathpar}
\inferrule*[lab=Quote-drop]
{ }
{ \quotep{@{x}} \nameeq x }

\inferrule*[lab=Struct-equiv]
{ P \scong Q }
{ \quotep{P} \nameeq \quotep{Q} }
\end{mathpar}

The astute reader will have noticed that the mutual recursion of names
and processes imposes a mutual recursion on alpha-equivalence and
structural equivalence via name-equivalence. Fortunately, all of this
works out pleasantly and we may calculate in the natural way, free of
concern. The reader interested in the details is referred to the
appendix \ref{appendix:rho_details}.

\subsection{Substitution}

We use $\Proc$ for the set of processes, $\QProc$ for the set of
names, and $\id{\{}\vec{y} / \vec{x} \id{\}}$ to denote partial maps,
$s : \QProc \rightarrow \QProc$. A map, $s$ lifts, uniquely, to a map
on process terms, $\widehat{s} : \Proc \rightarrow \Proc$ by the
following equations.

\begin{mathpar}
  (0) \psubstp{Q}{P} := 0 \\
  (R \juxtap S) \psubstp{Q}{P}
  :=    
  (R)\psubstp{Q}{P} \juxtap (S) \psubstp{Q}{P} \\
  (x?(y).R) \psubstp{Q}{P}    
  :=    
  (x)\substp{Q}{P} (z)\concat( (R \psubstn{z}{y}) \psubstp{Q}{P} ) \\
  (\lift{x}{R}) \psubstp{Q}{P}  
  :=
  \lift{(x)\substp{Q}{P}}{ R \psubstp{Q}{P} } \\
%   (\dropn{x})  \psubstp{Q}{P}       
%   := 
%   \left\{ 
%     \begin{array}{ccc} 
%       \dropn{\quotep{Q}} & & x \nameeq \quotep{P} \\
%       \dropn{x} & & otherwise \\
%     \end{array}
%   \right. 
  (\dropn{x})  \psubstp{Q}{P}       
  := 
  \left\{ 
    \begin{array}{ccc} 
      Q & & x \nameeq \quotep{P} \\
      \dropn{x} & & otherwise \\
    \end{array}
  \right.
\end{mathpar}
 

where

\begin{eqnarray}
  (x)\id{\{} \lpquote Q \rpquote / \lpquote P \rpquote \id{\}}            = 
  \left\{ 
    \begin{array}{ccc}
      \lpquote Q \rpquote & & x \nameeq \lpquote P \rpquote \\
      x & & otherwise \\
    \end{array}
  \right. \nonumber
\end{eqnarray}

and $z$ is chosen distinct from $\quotep{P}$, $\quotep{Q}$, the free
names in $Q$, and all the names in $R$. Our $\alpha$-equivalence will
be built in the standard way from this substitution.

\begin{remark}\label{rem:no_self_referential_names}
  One consequence of these definitions is that $\forall P. \quotep{P}
  \not\in \freenames{P}$.
\end{remark}

\subsection{ Dynamic quote: an example }

Anticipating something of what's to come, consider applying the
substitution, $\widehat{\id{\{}u / z \id{\}}}$, to the following pair
of processes, $\lift{w}{y!(z)}$ and $w[ \lpquote y!(z) \rpquote ]$.

\begin{eqnarray}
	\lift{w}{y!(z)}\widehat{\id{\{}u / z \id{\}}}
		& = &
		\lift{w}{y!(u)} \nonumber\\
	w[ \lpquote y!(z) \rpquote ] \widehat{ \id{\{}u / z \id{\}} }
		& = &
		w[ \lpquote y!(z) \rpquote ] \nonumber
\end{eqnarray}

Because the body of the process between quotes is impervious to
substitution, we get radically different answers. In fact, by
examining the first process in an input context,
e.g. $x?(z).\lift{w}{y!(z)}$, we see that the process under the lift
operator may be shaped by prefixed inputs binding a name inside it. In
this sense, the lift operator will be seen as a way to dynamically
construct processes before reifying them as names.

Finally equipped with these standard features we can present the
dynamics of the calculus.

\subsubsection{Operational semantics} 

Finally, we introduce the computational dynamics. What marks these
algebras as distinct from other more traditionally studied algebraic
structures, e.g. vector spaces or polynomial rings, is the manner in
which dynamics is captured. In traditional structures, dynamics is typically
expressed through morphisms between such structures, as in linear maps
between vector spaces or morphisms between rings. In algebras
associated with the semantics of computation, the dynamics is
expressed as part of the algebraic structure itself, through a
reduction reduction relation typically denoted by $\red$. Below, we
give a recursive presentation of this relation for the calculus used
in the encoding.

$\red \subseteq \pi \times \pi$
$\red : \pi \to \mathcal{P}(\pi)$

\begin{mathpar}
  \inferrule* [lab=Comm] { \textsf{match}( x_{src}, x_{trgt} ) } { x_{trgt}?(y)P \; | \; x_{src}!\langle {Q} \rangle \red P\{\quotep{Q}/y}\} }
  \and \\
  \inferrule* [lab=Par] {{P} \red {P}'} {{{P} | {Q}} \red {{P}' | {Q}}}
  \and
  \inferrule* [lab=Equiv]{{{P} \scong {P}'} \andalso {{P}' \red {Q}'} \andalso {{Q}' \scong {Q}}}{{P} \red {Q}}
\end{mathpar}

\begin{eqnarray*}
  match_{\equiv} (\quotep{P},\quotep{Q}) & := & P \equiv Q \\
  match_{\dagger}(\quotep{P},\quotep{Q}) & := & \forall R. P|Q \red^{*} R => R \red^{*} 0 \\
  match_{K}(\quotep{P},\quotep{Q}) & := & K \mbox{ for some context } K
\end{eqnarray*}

$u?(x)P | u!\langle Q \rangle \red P\{\quotep{Q}/x\}$

%We write $\wred$ for $\red^*$, and $P\red$ if $\exists Q $ such that $ P \red Q$.
We write $P\red$ if $\exists Q $ such that $ P \red Q$ and $P\not\red$, otherwise.

\section{Replication}

As mentioned before, it is known that replication (and hence
recursion) can be implemented in a higher-order process algebra
\cite{SangiorgiWalker}. As our first example of calculation with the
machinery thus far presented we give the construction explicitly in
the {\rhoc}.

\begin{eqnarray}
	D_{x} & := & \prefix{x}{y}{(\binpar{\outputp{x}{y}}{@{y}})} \nonumber\\
	\bangp_{x}{P} & := & \binpar{{x}!\langle{\binpar{D_{x}}{P}}\rangle}{D_{x}} \nonumber
\end{eqnarray}

\begin{eqnarray}
	\bangp_{x}{P} & & \nonumber\\
	=
	& {x}!\langle{(\prefix{x}{y}{(\outputp{x}{y} | @{y})) | P}}\rangle 
	      | \prefix{x}{y}{(\outputp{x}{y} | @{y})} & \nonumber\\
	\red
	& (\outputp{x}{y} | @{y})\substn{\quotep{(\prefix{x}{y}{(@{y} | \outputp{x}{y})) | P}}}{y} & \nonumber\\
	=
	& \outputp{x}{\quotep{(\prefix{x}{y}{(\outputp{x}{y} | @{y})) | P}}}
	  | {(\prefix{x}{y}{(\outputp{x}{y} | @{y})) | P}} & \nonumber\\
	\red
	& \ldots & \nonumber\\
	\red^*
	& P | P | \ldots & \nonumber
\end{eqnarray}

Of course, this encoding, as an implementation, runs away, unfolding
$\bangp{P}$ eagerly. A lazier and more implementable replication
operator, restricted to input-guarded processes, may be obtained as follows.

\begin{eqnarray}
\bangp{\prefix{u}{v}{P}} 
	:= 
	\binpar{\lift{x}{\prefix{u}{v}{(\binpar{D(x)}{P})}}}{D(x)} \nonumber
\end{eqnarray}

\begin{remark}
  Note that the lazier definition still does not deal with summation
  or mixed summation (i.e. sums over input and output). The reader is
  invited to construct definitions of replication that deal with these
  features. 

  Further, the definitions are parameterized in a name, $x$. Can you,
  gentle reader, make a definition that eliminates this parameter and
  guarantees no accidental interaction between the replication
  machinery and the process being replicated -- i.e. no accidental
  sharing of names used by the process to get its work done and the
  name(s) used by the replication to effect copying. This latter
  revision of the definition of replication is crucial to obtaining
  the expected identity $!!P \sim !P$.
\end{remark}

\begin{remark}\label{rem:paradoxical_combinator}
  The reader familiar with the lambda calculus will have noticed the
  similarity between $D$ and the paradoxical combinator.

  [Ed. note: the existence of this seems to suggest we have to be more
  restrictive on the set of processes and names we admit if we are to
  support no-cloning.]
\end{remark}

\subsubsection{Bisimulation}

The computational dynamics gives rise to another kind of equivalence,
the equivalence of computational behavior. As previously mentioned
this is typically captured \emph{via} some form of bisimulation.

% The notion we use in this paper is weak barbed bisimulation
% \cite{milner91polyadicpi}.

The notion we use in this paper is derived from weak barbed
bisimulation \cite{milner91polyadicpi}. 

\begin{definition}
An \emph{observation relation}, $\downarrow_{\mathcal N}$, over a set
of names, $\mathcal N$, is the smallest relation satisfying the rules
below.

\infrule[Out-barb]{y \in {\mathcal N}, \; x \nameeq y}
		  {\outputp{x}{v} \downarrow_{\mathcal N} x}
\infrule[Par-barb]{\mbox{$P\downarrow_{\mathcal N} x$ or $Q\downarrow_{\mathcal N} x$}}
		  {\binpar{P}{Q} \downarrow_{\mathcal N} x}

We write $P \Downarrow_{\mathcal N} x$ if there is $Q$ such that 
$P \wred Q$ and $Q \downarrow_{\mathcal N} x$.
\end{definition}

\begin{definition}
%\label{def.bbisim}
An  ${\mathcal N}$-\emph{barbed bisimulation} over a set of names, ${\mathcal N}$, is a symmetric binary relation 
${\mathcal S}_{\mathcal N}$ between agents such that $P\rel{S}_{\mathcal N}Q$ implies:
\begin{enumerate}
\item If $P \red P'$ then $Q \wred Q'$ and $P'\rel{S}_{\mathcal N} Q'$.
\item If $P\downarrow_{\mathcal N} x$, then $Q\Downarrow_{\mathcal N} x$.
\end{enumerate}
$P$ is ${\mathcal N}$-barbed bisimilar to $Q$, written
$P \wbbisim_{\mathcal N} Q$, if $P \rel{S}_{\mathcal N} Q$ for some ${\mathcal N}$-barbed bisimulation ${\mathcal S}_{\mathcal N}$.
\end{definition}

$\mathcal{R} \subseteq \pi \times \pi$

$P \mathcal{R} Q => \forall P'. P \red P' \Rightarrow \exists Q'. Q \red Q', P' \mathcal{R} Q'$

$P \vdash x \Rightarrow Q \vdash x$

\begin{mathpar}
  \inferrule*[lab=Out-barb]{x \nameeq y}{{y}!\langle{Q}\rangle \vdash x}
  \and
  \inferrule*[lab=Par-barb]{\mbox{$P\vdash x$ or $Q\vdash x$}}{\binpar{P}{Q} \vdash x}
\end{mathpar}

\subsubsection{Contexts}

One of the principle advantages of computational calculi like the
$\pi$-calculus is a well-defined notion of context,
contextual-equivalence and a correlation between
contextual-equivalence and notions of bisimulation. The notion of
context allows the decomposition of a process into (sub-)process and
its syntactic environment, its context. Thus, a context may be
thought of as a process with a ``hole'' (written $\Box$) in it. The
application of a context $M$ to a process $P$, written $M[P]$, is
tantamount to filling the hole in $M$ with $P$. In this paper we do
not need the full weight of this theory, but do make use of the notion
of context in the proof the main theorem. 

\begin{mathpar}
  \inferrule* [lab=summation] {} {{M_{M},M_{N}} \bc \Box \;|\; x.M_{A} \;|\; M_{M}+M_{N}}
  \and
  \inferrule* [lab=agent] {} {{M_{A}} \bc (\vec{x})M_{P} \;| \; \clift{P_0,\ldots,M_{P},\ldots,P_N}}
  \and \\
  \inferrule* [lab=process] {} {{M_{P}} \bc M_{N} \;| \;P|M_{P} }
\end{mathpar} 

\begin{mathpar}
  \inferrule* [lab=sychronization] {} {M_{N} \bc \Box \;|\; x?M_{F} \;|\; x!M_{C}}
  \and
  \inferrule* [lab=abstraction] {} {{M_{F}} \bc (x)M_{P} }
  \and
  \inferrule* [lab=concretion] {} {{M_{C}} \bc \langle M_{P} \rangle }
  \and \\
  \inferrule* [lab=process] {} {{M_{P}} \bc M_{N} \;| \;P|M_{P} }
\end{mathpar}

\begin{definition}[contextual application] Given a context $M$, and
  process $P$, we define the \emph{contextual application}, $M[P] :=
  M\{P/\Box\}$. That is, the contextual application of M to P is the
  substitution of $P$ for $\Box$ in $M$.
\end{definition}

$\meaningof{-} : L \to \mathcal{P}(\pi)$

\begin{mathpar}
  \inferrule* [lab=collection] {} {\meaningof{true} = \pi, \and \meaningof{~E} = \pi \setminus \meaningof{E}, \and \meaningof{E_{1} \& E_{2}} = \meaningof{E_{1}} \cap \meaningof{E_{2}}}
\end{mathpar}

\begin{mathpar}
  \inferrule* [lab=structure] {} {\meaningof{0} = \{ P \in \pi | P \equiv 0 \}, \and \\ \meaningof{E_1 | E_2} = \{ P \in \pi | P \equiv P_{1} | P_{2}, P_{1} \in \meaningof{E_{1}}, P_{2} \in \meaningof{E_2}\} }
\end{mathpar}

\begin{mathpar}
 \inferrule* [lab=behavior] {} {\meaningof{\langle a?b \rangle E} = \{ P \in \pi | P \equiv Q | u?(y)P', \\ \and \\\\ \and \\ \;\;\; u \in \meaningof{a}, \forall z.P'\{z/y\} \in \meaningof{E\{z/b\}}\}, \and \\ \meaningof{a!E} = \{ P \in \pi | P \equiv Q | x!\langle P' \rangle, x \in \meaningof{a} P' \in \meaningof{E}\} }
\end{mathpar}

\begin{mathpar}
 \inferrule* [lab=nominal] {} {\meaningof{\quotep{E}} = \{ \quotep{P} \in \quotep{\pi} | P \in \meaningof{E} \}, \and \meaningof{\quotep{P}} = \{ \quotep{Q} \in \quotep{\pi} | P \equiv Q \} \and \\ \meaningof{@\quotep{E}} = \{ P \in \pi | P \equiv @x, x \in \meaningof{E} \}}
\end{mathpar}

\begin{eqnarray*}
  \\
  \meaningof{-} : TS \to ST
\end{eqnarray*}

\begin{eqnarray*}
  \\
  L : TS \to ST
\end{eqnarray*}

\begin{eqnarray*}
  \\
  P \models E \iff P \in \meaningof{E}
\end{eqnarray*}

\begin{eqnarray*}
  P \approx_{L} Q \iff \forall E \in L. P \models E \iff Q \models E
\end{eqnarray*}

\begin{eqnarray*}
  P \approx_{K} Q
\end{eqnarray*}

\begin{eqnarray*}
  P \approx Q
\end{eqnarray*}

$\approx_{K} = \approx = \approx_{L}$

\subsubsection{Contextual duality}

Note that contexts extend the quotation operation to a family of
operations from processes to names. Given a context, $M$, we can
define a \emph{nominal context}, $\quotep{M}$ by $\quotep{M}[P] :=
\quotep{M[P]}$. To foreshadow what is to come we observe that these
operations enjoy a duality with processes very much like the duality
between vectors and maps from vectors to scalars.

Further, because the calculus is essentially higher-order, we have a
correspondence between contexts and processes. More specifically,
given a name $x$ and a context $M$ we can construct $M^{*}_{x}$ such
that 

\begin{mathpar}
  M^{*}_{x} | \lift{x}{P} \red M[P]
\end{mathpar}

namely,

\begin{mathpar}
  M^{*}_{x} := x?(u).M[\dropn{u}]
\end{mathpar}

The dependence of $M^{*}_{x}$ on a name makes it an abstraction, 

\begin{mathpar}
  M^{*} := (x)x?(u).M[\dropn{u}]
\end{mathpar}

\subsection{Additional notation}

It will sometimes be convenient to denote the process a name
quotes. We already have the notation $x = \quotep{P}$, but it will be
convenient to introduce an alternate notation, $\procn{x}$, when we
want to emphasize the connection to the use of the name. Note that, by
virtue of name equivalence, $\quotep{\procn{x}} \nameeq x$; so, the
notation is consistent with previous definitions.

Further, because names have structure it is possible to effect
substitutions on the basis of that structure. This means we need to
upgrade our notation for substitutions, which we accomplish by
adapting comprehension notation. Thus,

\begin{mathpar}
  P\{ y / x : x \in S \}
\end{mathpar}

is interpreted to mean the process derived from P by replacing (in a
capture-avoiding manner) each occurrence of $x$ in $S$ by $y$. For example,

\begin{mathpar}
  P\{ \quotep{\procn{x}|\procn{x}} / x : x \in \freenames{P} \}
\end{mathpar}

will replace each (occurrence) of a free name $x$ in $P$ by
$\quotep{\procn{x}|\procn{x}}$.

Also, we will avail ourselves of the notation $x^{L}$ and $x^{R}$ to
denote injections of a name into disjoint copies of the name
space. There are numerous ways to accomplish this. One example can be
found in \cite{MeredithR05}. This notation overloads to vectors of
names: $\vec{x}^{\pi} := (x_{i}^{\pi} \; : \; 0 \leq i < |\vec{x}| )$ where $\pi \in \{L,R\}$.

We also use $P^{\Box} := P|\Box$.

In \cite{MeredithR05} an interpretation of the new operator is
given. It turns out that there are several possible interpretations
all enjoying the requisite algebraic properties of the operator (see
\cite{milner91polyadicpi}). We will therefore make liberal use of
$(\nu\; \vec{x})P$.

% subsection the_syntax_and_semantics_of_the_notation_system (end)   

\input{qm2pi.qmops} 

\input{qm2pi.sterngerlach} 

\input{qm2pi.metric} 

% section concurrent_process_calculi (end)

%\input{qm2pi.proofsketch}

% section proof sketch (end)

%\input{qm2pi.slviaknots} 

% section spatial logic via knots (end)

\input{qm2pi.conclusion}

% section conclusion (end)

%\input{qm2pi.dtcodes} 

% section wiring algorithm (end)

\input{qm2pi.ack} 

% section acknowledgments (end)

\newpage


\bibliographystyle{plain}   
\bibliography{../../biblios/main.bib}

\input{qm2pi.rhodetails}

\end{document}

 

% section concurrent_process_calculi (end)

%\documentclass[12pt]{llncs}
%\documentclass{jktr}

\usepackage[pdftex]{hyperref}                   
\usepackage {listings}
\usepackage {mathpartir}
\usepackage{bcprules}
%\usepackage{listings}
                       
\usepackage{graphicx} 
%\usepackage[margins=2.5cm,nohead,nofoot]{geometry}
%\usepackage{geometry}
\usepackage{amsfonts}
\usepackage{amstext}
\usepackage{latexsym}
\usepackage{amssymb}
\usepackage{color}


%\include{myPreamble}
\include{qm2pi.local} 

%\ifpdf
%\usepackage[pdftex]{graphicx}
%\else
%\usepackage{graphicx}
%\fi

 % \ifpdf
%  \usepackage{pdfsync}
%  \if


%\title{Brief Article}
%\author{David F. Snyder}
%\author{L.G. Meredith}

%\address{Dept. of Math., Texas State University--San Marcos, San Marcos, TX 78666}
       
\pagestyle{empty}


\begin{document}

\lstset{language=[Objective]Caml,frame=shadowbox}

\input{qm2pi.front}

% section front matter (end)

\input{qm2pi.intro} 
 
% section introduction (end)

% \input{qm2pi.knotations} 

% section notation (end)

\input{qm2pi.process.calculi} 

% section concurrent_process_calculi_and_spatial_logics_ (end)
    
%\input{qm2pi.knots2pi} 

%\input{qm2pi.trefoil} 

%\input{qm2pi.mainthm} 

% subsection basic_interpretation (end)

%\input{qm2pi.rho.presentation} 
\subsection{The syntax and semantics of the notation system}\label{sub:the_syntax_and_semantics_of_the_notation_system} % (fold)

We now summarize a technical presentation of the calculus that
embodies our theory of dynamics. The typical presentation of such a
calculus follows the style of giving generators and relations on
them. The grammar, below, describing term constructors, freely
generates the set of processes, $\Proc$. This set is then quotiented
by a relation known as structural congruence and it is over this set
that the notion of dynamics is expressed. This presentation is
essentially that of \cite{MeredithR05} with the addition of
polyadicity and summation. For readability we have relegated some of
the technical subtleties to an appendix.

\subsubsection{Process grammar}\label{subsub:process_grammar}

\begin{mathpar}
  \inferrule* [lab=synchronization] {} {{M} \bc \pzero \;|\; x?F \;|\; x!C }
  \and
  \inferrule* [lab=abstraction] {} {{F} \bc (x)P}
  \and
  \inferrule* [lab=concretion] {} {{C} \bc \langle Q \rangle}
  \and
  \inferrule* [lab=process] {} {{P,Q} \bc M \;| \;P|Q \;|\; @{x}}
  \and
  \inferrule* [lab=name] {} {{x} \bc \quotep{P}}
\end{mathpar} 

Note that $\vec{x}$ (resp. $\vec{P}$) denotes a vector of names
(resp. processes) of length $|\vec{x}|$ (resp. $|\vec{P}|$). We adopt
the following useful abbreviations.

\begin{mathpar}
   x?(\vec{y}).P := x.(\vec{y})P \and  x\clift{\vec{P}} := x.\clift{\vec{P}}
   \and x!(y) := \lift{x}{\dropn{y}}
   \and \Pi_{i=0}^{n-1}P_i := P_0 | \ldots | P_{n-1}
\end{mathpar}

\subsubsection{Structural congruence}

\paragraph{Free and bound names and alpha-equivalence.} At the
core of structural equivalence is alpha-equivalence which identifies
process that are the same up to a change of variable. Formally, we
recognize the distinction between free and bound names. The free names
of a process, $\freenames{P}$, may be calculated recursively as
follows:

\begin{mathpar}
\freenames{\pzero} := \emptyset
  \and \\
  \freenames{x?(y).P} := \{ x \} \cup (\freenames{P} \setminus \{ y \})
  \and 
  \freenames{x!\langle P \rangle} := \{ x \} \cup \{ P \} 
  \and \\
  \freenames{P|Q} := \freenames{P} \cup \freenames{Q}
  \and \\
  \freenames{@{x}} := \{ x \}
\end{mathpar}

$\pi$
$\quotep{\pi}$

$\freenames{-} : \pi \to \mathcal{P}(\quotep{\pi})$

\begin{eqnarray*}
  \freenames{\pzero} & := & \emptyset \\
  \freenames{x?(y).P} & := & \{ x \} \cup (\freenames{P} \setminus \{ y \}) \\
  \freenames{x!\langle P \rangle} & := & \{ x \} \cup \{ P \} \\
  \freenames{P|Q} & := & \freenames{P} \cup \freenames{Q} \\
  \freenames{\dropn{x}} & := & \{ x \}
\end{eqnarray*}

The bound names of a process, $\boundnames{P}$, are those names occurring in $P$
that are not free. For example, in $x?(y).0$, the name $x$ is free, while $y$ is bound.

\begin{mathpar}
  \inferrule* [lab=monoidal-laws] {} { P|Q \equiv Q|P \and P|0 \equiv P \and P|(Q|R) \equiv (P|Q)|R }
\end{mathpar}

\begin{mathpar}
  \inferrule* [lab=alpha-equivalence] {} { (x)P \equiv (y)P\{y/x\} \and y \not\in \freenames{P} }
\end{mathpar}

\begin{definition}
Then two processes, $P,Q$, are alpha-equivalent if $P = Q\{\vec{y}/\vec{x}\}$ for
some $\vec{x} \in \boundnames{Q},\vec{y} \in \boundnames{P}$, where $Q\{\vec{y}/\vec{x}\}$
denotes the capture-avoiding substitution of $\vec{y}$ for $\vec{x}$ in $Q$.
\end{definition}

\begin{definition}
  The {\em structural congruence} \cite{SangiorgiWalker} , $\equiv$,
  between processes is the least congruence containing
  alpha-equivalence, satisfying the abelian monoid laws
  (associativity, commutativity and $\pzero$ as identity) for parallel
  composition $|$ and for summation $+$.
\end{definition}

\subsection{Name equivalence}

We take name equivalence, written $\nameeq$, to be the smallest
equivalence relation generated by the following rules.

\begin{mathpar}
\inferrule*[lab=Quote-drop]
{ }
{ \quotep{@{x}} \nameeq x }

\inferrule*[lab=Struct-equiv]
{ P \scong Q }
{ \quotep{P} \nameeq \quotep{Q} }
\end{mathpar}

The astute reader will have noticed that the mutual recursion of names
and processes imposes a mutual recursion on alpha-equivalence and
structural equivalence via name-equivalence. Fortunately, all of this
works out pleasantly and we may calculate in the natural way, free of
concern. The reader interested in the details is referred to the
appendix \ref{appendix:rho_details}.

\subsection{Substitution}

We use $\Proc$ for the set of processes, $\QProc$ for the set of
names, and $\id{\{}\vec{y} / \vec{x} \id{\}}$ to denote partial maps,
$s : \QProc \rightarrow \QProc$. A map, $s$ lifts, uniquely, to a map
on process terms, $\widehat{s} : \Proc \rightarrow \Proc$ by the
following equations.

\begin{mathpar}
  (0) \psubstp{Q}{P} := 0 \\
  (R \juxtap S) \psubstp{Q}{P}
  :=    
  (R)\psubstp{Q}{P} \juxtap (S) \psubstp{Q}{P} \\
  (x?(y).R) \psubstp{Q}{P}    
  :=    
  (x)\substp{Q}{P} (z)\concat( (R \psubstn{z}{y}) \psubstp{Q}{P} ) \\
  (\lift{x}{R}) \psubstp{Q}{P}  
  :=
  \lift{(x)\substp{Q}{P}}{ R \psubstp{Q}{P} } \\
%   (\dropn{x})  \psubstp{Q}{P}       
%   := 
%   \left\{ 
%     \begin{array}{ccc} 
%       \dropn{\quotep{Q}} & & x \nameeq \quotep{P} \\
%       \dropn{x} & & otherwise \\
%     \end{array}
%   \right. 
  (\dropn{x})  \psubstp{Q}{P}       
  := 
  \left\{ 
    \begin{array}{ccc} 
      Q & & x \nameeq \quotep{P} \\
      \dropn{x} & & otherwise \\
    \end{array}
  \right.
\end{mathpar}
 

where

\begin{eqnarray}
  (x)\id{\{} \lpquote Q \rpquote / \lpquote P \rpquote \id{\}}            = 
  \left\{ 
    \begin{array}{ccc}
      \lpquote Q \rpquote & & x \nameeq \lpquote P \rpquote \\
      x & & otherwise \\
    \end{array}
  \right. \nonumber
\end{eqnarray}

and $z$ is chosen distinct from $\quotep{P}$, $\quotep{Q}$, the free
names in $Q$, and all the names in $R$. Our $\alpha$-equivalence will
be built in the standard way from this substitution.

\begin{remark}\label{rem:no_self_referential_names}
  One consequence of these definitions is that $\forall P. \quotep{P}
  \not\in \freenames{P}$.
\end{remark}

\subsection{ Dynamic quote: an example }

Anticipating something of what's to come, consider applying the
substitution, $\widehat{\id{\{}u / z \id{\}}}$, to the following pair
of processes, $\lift{w}{y!(z)}$ and $w[ \lpquote y!(z) \rpquote ]$.

\begin{eqnarray}
	\lift{w}{y!(z)}\widehat{\id{\{}u / z \id{\}}}
		& = &
		\lift{w}{y!(u)} \nonumber\\
	w[ \lpquote y!(z) \rpquote ] \widehat{ \id{\{}u / z \id{\}} }
		& = &
		w[ \lpquote y!(z) \rpquote ] \nonumber
\end{eqnarray}

Because the body of the process between quotes is impervious to
substitution, we get radically different answers. In fact, by
examining the first process in an input context,
e.g. $x?(z).\lift{w}{y!(z)}$, we see that the process under the lift
operator may be shaped by prefixed inputs binding a name inside it. In
this sense, the lift operator will be seen as a way to dynamically
construct processes before reifying them as names.

Finally equipped with these standard features we can present the
dynamics of the calculus.

\subsubsection{Operational semantics} 

Finally, we introduce the computational dynamics. What marks these
algebras as distinct from other more traditionally studied algebraic
structures, e.g. vector spaces or polynomial rings, is the manner in
which dynamics is captured. In traditional structures, dynamics is typically
expressed through morphisms between such structures, as in linear maps
between vector spaces or morphisms between rings. In algebras
associated with the semantics of computation, the dynamics is
expressed as part of the algebraic structure itself, through a
reduction reduction relation typically denoted by $\red$. Below, we
give a recursive presentation of this relation for the calculus used
in the encoding.

$\red \subseteq \pi \times \pi$
$\red : \pi \to \mathcal{P}(\pi)$

\begin{mathpar}
  \inferrule* [lab=Comm] { \textsf{match}( x_{src}, x_{trgt} ) } { x_{trgt}?(y)P \; | \; x_{src}!\langle {Q} \rangle \red P\{\quotep{Q}/y}\} }
  \and \\
  \inferrule* [lab=Par] {{P} \red {P}'} {{{P} | {Q}} \red {{P}' | {Q}}}
  \and
  \inferrule* [lab=Equiv]{{{P} \scong {P}'} \andalso {{P}' \red {Q}'} \andalso {{Q}' \scong {Q}}}{{P} \red {Q}}
\end{mathpar}

\begin{eqnarray*}
  match_{\equiv} (\quotep{P},\quotep{Q}) & := & P \equiv Q \\
  match_{\dagger}(\quotep{P},\quotep{Q}) & := & \forall R. P|Q \red^{*} R => R \red^{*} 0 \\
  match_{K}(\quotep{P},\quotep{Q}) & := & K \mbox{ for some context } K
\end{eqnarray*}

$u?(x)P | u!\langle Q \rangle \red P\{\quotep{Q}/x\}$

%We write $\wred$ for $\red^*$, and $P\red$ if $\exists Q $ such that $ P \red Q$.
We write $P\red$ if $\exists Q $ such that $ P \red Q$ and $P\not\red$, otherwise.

\section{Replication}

As mentioned before, it is known that replication (and hence
recursion) can be implemented in a higher-order process algebra
\cite{SangiorgiWalker}. As our first example of calculation with the
machinery thus far presented we give the construction explicitly in
the {\rhoc}.

\begin{eqnarray}
	D_{x} & := & \prefix{x}{y}{(\binpar{\outputp{x}{y}}{@{y}})} \nonumber\\
	\bangp_{x}{P} & := & \binpar{{x}!\langle{\binpar{D_{x}}{P}}\rangle}{D_{x}} \nonumber
\end{eqnarray}

\begin{eqnarray}
	\bangp_{x}{P} & & \nonumber\\
	=
	& {x}!\langle{(\prefix{x}{y}{(\outputp{x}{y} | @{y})) | P}}\rangle 
	      | \prefix{x}{y}{(\outputp{x}{y} | @{y})} & \nonumber\\
	\red
	& (\outputp{x}{y} | @{y})\substn{\quotep{(\prefix{x}{y}{(@{y} | \outputp{x}{y})) | P}}}{y} & \nonumber\\
	=
	& \outputp{x}{\quotep{(\prefix{x}{y}{(\outputp{x}{y} | @{y})) | P}}}
	  | {(\prefix{x}{y}{(\outputp{x}{y} | @{y})) | P}} & \nonumber\\
	\red
	& \ldots & \nonumber\\
	\red^*
	& P | P | \ldots & \nonumber
\end{eqnarray}

Of course, this encoding, as an implementation, runs away, unfolding
$\bangp{P}$ eagerly. A lazier and more implementable replication
operator, restricted to input-guarded processes, may be obtained as follows.

\begin{eqnarray}
\bangp{\prefix{u}{v}{P}} 
	:= 
	\binpar{\lift{x}{\prefix{u}{v}{(\binpar{D(x)}{P})}}}{D(x)} \nonumber
\end{eqnarray}

\begin{remark}
  Note that the lazier definition still does not deal with summation
  or mixed summation (i.e. sums over input and output). The reader is
  invited to construct definitions of replication that deal with these
  features. 

  Further, the definitions are parameterized in a name, $x$. Can you,
  gentle reader, make a definition that eliminates this parameter and
  guarantees no accidental interaction between the replication
  machinery and the process being replicated -- i.e. no accidental
  sharing of names used by the process to get its work done and the
  name(s) used by the replication to effect copying. This latter
  revision of the definition of replication is crucial to obtaining
  the expected identity $!!P \sim !P$.
\end{remark}

\begin{remark}\label{rem:paradoxical_combinator}
  The reader familiar with the lambda calculus will have noticed the
  similarity between $D$ and the paradoxical combinator.

  [Ed. note: the existence of this seems to suggest we have to be more
  restrictive on the set of processes and names we admit if we are to
  support no-cloning.]
\end{remark}

\subsubsection{Bisimulation}

The computational dynamics gives rise to another kind of equivalence,
the equivalence of computational behavior. As previously mentioned
this is typically captured \emph{via} some form of bisimulation.

% The notion we use in this paper is weak barbed bisimulation
% \cite{milner91polyadicpi}.

The notion we use in this paper is derived from weak barbed
bisimulation \cite{milner91polyadicpi}. 

\begin{definition}
An \emph{observation relation}, $\downarrow_{\mathcal N}$, over a set
of names, $\mathcal N$, is the smallest relation satisfying the rules
below.

\infrule[Out-barb]{y \in {\mathcal N}, \; x \nameeq y}
		  {\outputp{x}{v} \downarrow_{\mathcal N} x}
\infrule[Par-barb]{\mbox{$P\downarrow_{\mathcal N} x$ or $Q\downarrow_{\mathcal N} x$}}
		  {\binpar{P}{Q} \downarrow_{\mathcal N} x}

We write $P \Downarrow_{\mathcal N} x$ if there is $Q$ such that 
$P \wred Q$ and $Q \downarrow_{\mathcal N} x$.
\end{definition}

\begin{definition}
%\label{def.bbisim}
An  ${\mathcal N}$-\emph{barbed bisimulation} over a set of names, ${\mathcal N}$, is a symmetric binary relation 
${\mathcal S}_{\mathcal N}$ between agents such that $P\rel{S}_{\mathcal N}Q$ implies:
\begin{enumerate}
\item If $P \red P'$ then $Q \wred Q'$ and $P'\rel{S}_{\mathcal N} Q'$.
\item If $P\downarrow_{\mathcal N} x$, then $Q\Downarrow_{\mathcal N} x$.
\end{enumerate}
$P$ is ${\mathcal N}$-barbed bisimilar to $Q$, written
$P \wbbisim_{\mathcal N} Q$, if $P \rel{S}_{\mathcal N} Q$ for some ${\mathcal N}$-barbed bisimulation ${\mathcal S}_{\mathcal N}$.
\end{definition}

$\mathcal{R} \subseteq \pi \times \pi$

$P \mathcal{R} Q => \forall P'. P \red P' \Rightarrow \exists Q'. Q \red Q', P' \mathcal{R} Q'$

$P \vdash x \Rightarrow Q \vdash x$

\begin{mathpar}
  \inferrule*[lab=Out-barb]{x \nameeq y}{{y}!\langle{Q}\rangle \vdash x}
  \and
  \inferrule*[lab=Par-barb]{\mbox{$P\vdash x$ or $Q\vdash x$}}{\binpar{P}{Q} \vdash x}
\end{mathpar}

\subsubsection{Contexts}

One of the principle advantages of computational calculi like the
$\pi$-calculus is a well-defined notion of context,
contextual-equivalence and a correlation between
contextual-equivalence and notions of bisimulation. The notion of
context allows the decomposition of a process into (sub-)process and
its syntactic environment, its context. Thus, a context may be
thought of as a process with a ``hole'' (written $\Box$) in it. The
application of a context $M$ to a process $P$, written $M[P]$, is
tantamount to filling the hole in $M$ with $P$. In this paper we do
not need the full weight of this theory, but do make use of the notion
of context in the proof the main theorem. 

\begin{mathpar}
  \inferrule* [lab=summation] {} {{M_{M},M_{N}} \bc \Box \;|\; x.M_{A} \;|\; M_{M}+M_{N}}
  \and
  \inferrule* [lab=agent] {} {{M_{A}} \bc (\vec{x})M_{P} \;| \; \clift{P_0,\ldots,M_{P},\ldots,P_N}}
  \and \\
  \inferrule* [lab=process] {} {{M_{P}} \bc M_{N} \;| \;P|M_{P} }
\end{mathpar} 

\begin{mathpar}
  \inferrule* [lab=sychronization] {} {M_{N} \bc \Box \;|\; x?M_{F} \;|\; x!M_{C}}
  \and
  \inferrule* [lab=abstraction] {} {{M_{F}} \bc (x)M_{P} }
  \and
  \inferrule* [lab=concretion] {} {{M_{C}} \bc \langle M_{P} \rangle }
  \and \\
  \inferrule* [lab=process] {} {{M_{P}} \bc M_{N} \;| \;P|M_{P} }
\end{mathpar}

\begin{definition}[contextual application] Given a context $M$, and
  process $P$, we define the \emph{contextual application}, $M[P] :=
  M\{P/\Box\}$. That is, the contextual application of M to P is the
  substitution of $P$ for $\Box$ in $M$.
\end{definition}

$\meaningof{-} : L \to \mathcal{P}(\pi)$

\begin{mathpar}
  \inferrule* [lab=collection] {} {\meaningof{true} = \pi, \and \meaningof{~E} = \pi \setminus \meaningof{E}, \and \meaningof{E_{1} \& E_{2}} = \meaningof{E_{1}} \cap \meaningof{E_{2}}}
\end{mathpar}

\begin{mathpar}
  \inferrule* [lab=structure] {} {\meaningof{0} = \{ P \in \pi | P \equiv 0 \}, \and \\ \meaningof{E_1 | E_2} = \{ P \in \pi | P \equiv P_{1} | P_{2}, P_{1} \in \meaningof{E_{1}}, P_{2} \in \meaningof{E_2}\} }
\end{mathpar}

\begin{mathpar}
 \inferrule* [lab=behavior] {} {\meaningof{\langle a?b \rangle E} = \{ P \in \pi | P \equiv Q | u?(y)P', \\ \and \\\\ \and \\ \;\;\; u \in \meaningof{a}, \forall z.P'\{z/y\} \in \meaningof{E\{z/b\}}\}, \and \\ \meaningof{a!E} = \{ P \in \pi | P \equiv Q | x!\langle P' \rangle, x \in \meaningof{a} P' \in \meaningof{E}\} }
\end{mathpar}

\begin{mathpar}
 \inferrule* [lab=nominal] {} {\meaningof{\quotep{E}} = \{ \quotep{P} \in \quotep{\pi} | P \in \meaningof{E} \}, \and \meaningof{\quotep{P}} = \{ \quotep{Q} \in \quotep{\pi} | P \equiv Q \} \and \\ \meaningof{@\quotep{E}} = \{ P \in \pi | P \equiv @x, x \in \meaningof{E} \}}
\end{mathpar}

\begin{eqnarray*}
  \\
  \meaningof{-} : TS \to ST
\end{eqnarray*}

\begin{eqnarray*}
  \\
  L : TS \to ST
\end{eqnarray*}

\begin{eqnarray*}
  \\
  P \models E \iff P \in \meaningof{E}
\end{eqnarray*}

\begin{eqnarray*}
  P \approx_{L} Q \iff \forall E \in L. P \models E \iff Q \models E
\end{eqnarray*}

\begin{eqnarray*}
  P \approx_{K} Q
\end{eqnarray*}

\begin{eqnarray*}
  P \approx Q
\end{eqnarray*}

$\approx_{K} = \approx = \approx_{L}$

\subsubsection{Contextual duality}

Note that contexts extend the quotation operation to a family of
operations from processes to names. Given a context, $M$, we can
define a \emph{nominal context}, $\quotep{M}$ by $\quotep{M}[P] :=
\quotep{M[P]}$. To foreshadow what is to come we observe that these
operations enjoy a duality with processes very much like the duality
between vectors and maps from vectors to scalars.

Further, because the calculus is essentially higher-order, we have a
correspondence between contexts and processes. More specifically,
given a name $x$ and a context $M$ we can construct $M^{*}_{x}$ such
that 

\begin{mathpar}
  M^{*}_{x} | \lift{x}{P} \red M[P]
\end{mathpar}

namely,

\begin{mathpar}
  M^{*}_{x} := x?(u).M[\dropn{u}]
\end{mathpar}

The dependence of $M^{*}_{x}$ on a name makes it an abstraction, 

\begin{mathpar}
  M^{*} := (x)x?(u).M[\dropn{u}]
\end{mathpar}

\subsection{Additional notation}

It will sometimes be convenient to denote the process a name
quotes. We already have the notation $x = \quotep{P}$, but it will be
convenient to introduce an alternate notation, $\procn{x}$, when we
want to emphasize the connection to the use of the name. Note that, by
virtue of name equivalence, $\quotep{\procn{x}} \nameeq x$; so, the
notation is consistent with previous definitions.

Further, because names have structure it is possible to effect
substitutions on the basis of that structure. This means we need to
upgrade our notation for substitutions, which we accomplish by
adapting comprehension notation. Thus,

\begin{mathpar}
  P\{ y / x : x \in S \}
\end{mathpar}

is interpreted to mean the process derived from P by replacing (in a
capture-avoiding manner) each occurrence of $x$ in $S$ by $y$. For example,

\begin{mathpar}
  P\{ \quotep{\procn{x}|\procn{x}} / x : x \in \freenames{P} \}
\end{mathpar}

will replace each (occurrence) of a free name $x$ in $P$ by
$\quotep{\procn{x}|\procn{x}}$.

Also, we will avail ourselves of the notation $x^{L}$ and $x^{R}$ to
denote injections of a name into disjoint copies of the name
space. There are numerous ways to accomplish this. One example can be
found in \cite{MeredithR05}. This notation overloads to vectors of
names: $\vec{x}^{\pi} := (x_{i}^{\pi} \; : \; 0 \leq i < |\vec{x}| )$ where $\pi \in \{L,R\}$.

We also use $P^{\Box} := P|\Box$.

In \cite{MeredithR05} an interpretation of the new operator is
given. It turns out that there are several possible interpretations
all enjoying the requisite algebraic properties of the operator (see
\cite{milner91polyadicpi}). We will therefore make liberal use of
$(\nu\; \vec{x})P$.

% subsection the_syntax_and_semantics_of_the_notation_system (end)   

\input{qm2pi.qmops} 

\input{qm2pi.sterngerlach} 

\input{qm2pi.metric} 

% section concurrent_process_calculi (end)

%\input{qm2pi.proofsketch}

% section proof sketch (end)

%\input{qm2pi.slviaknots} 

% section spatial logic via knots (end)

\input{qm2pi.conclusion}

% section conclusion (end)

%\input{qm2pi.dtcodes} 

% section wiring algorithm (end)

\input{qm2pi.ack} 

% section acknowledgments (end)

\newpage


\bibliographystyle{plain}   
\bibliography{../../biblios/main.bib}

\input{qm2pi.rhodetails}

\end{document}



% section proof sketch (end)

%\section{Unlikely characters: spatial logic for
  knots}\label{sub:characteristic_formulae} % (fold)

Associated to the mobile process calculi are a family of logics known
as the Hennessy-Milner logics. These logics typically enjoy a
semantics interpreting formulae as sets of processes that when
factored through the encoding outlined above allows an identification
of classes of knots with logical formulae. In the context of this
encoding the sub-family known as the spatial logics \cite{CairesC03}
\cite{CairesC04} \cite{Caires04} are of particular interest providing
several important features for expressing and reasoning about
properties (i.e. classes) of knots. We hint here at how this may be done.

%\begin{description}
%\item [structural connectives] 
\subsubsection{Structural connectives} The spatial logics enjoy
structural connectives corresponding, at the logical level, to the
parallel composition ($P | Q$) and new name ($(\nu \; x)P$)
connectives for processes. As illustrated in the examples below, these
connectives are extremely expressive given the shape of our encoding.
%\item [decideable satisfaction]

\subsubsection{Decideable satisfaction}
In \cite{Caires04} the satisfaction relation is shown to be decideable
for a rich class of processes. It further turns out that the image of
the our encoding is a proper subset of that class. This result
provides the basis for an algorithm by which to search for knots
enjoying a given property.
%\item [characteristic formulae]

\subsubsection{Characteristic formulae}
In the same paper \cite{Caires04} , Caires presents a means of calculating
characteristic formulae, selecting equivalence classes of processes
up to a pre--specified depth limit on the support set of names. Composed with our
encoding, this characteristic formula can be used to select
characteristic formulae for knots.
%\end{description}

\subsubsection{Spatial logic formulae}

The grammar below (segmented for comprehension) summarizes the syntax
of spatial logic formulae. We employ illustrative examples in the
sequel to provide an intuitive understanding of their meaning
referring the reader to \cite{Caires04} for a more detailed explication
of the semantics.

\begin{mathpar}
  \inferrule* [lab=boolean] {} {{A,B} \bc T \;|\; \neg A \;|\; A \wedge B \;|\; \eta = \eta'}
  \and
  \inferrule* [lab=spatial] {} {|\; \pzero \;|\; A | B \;|\; x \text{\textregistered} A \;|\; \forall x . A \;|\;  H x . A}
  \and
  \inferrule* [lab=behavioral] {} {|\; \alpha . A}
  \and 
  \inferrule* [lab=recursion] {} {|\; X(\vec{u}) \;|\; \mu X(\vec{u}) . A}
  \and
  \inferrule* [lab=action] {} {\alpha \bc \langle x?(\vec{y}) \rangle \;|\; \langle x!(\vec{y}) \rangle \;|\; \langle \tau \rangle}
  \and 
  \inferrule* [lab=name] {} {\eta \bc x \;|\; \tau}
\end{mathpar} 

% subsection characteristic_formulae (end)   	 

\subsection{Example formulae}\label{sub:example_formulae_} % (fold)

\subsubsection{Crossing as formula.}
% 
% \begin{align*}
%   \frac{d}{dx} \sin x &= \cos x 
%   & \frac{d}{dx} e^x &= e^x \\
%   \frac{d}{dx} \cos x &= - \sin x 
%   & \frac{d}{dx} \log x &= \frac{1}{x} \\
% \end{align*} 

\begin{align*}
 \mu C(x_{0},x_{1},y_{0},y_{1},u).&(\langle x_{0}?(z) \rangle(\langle u! \rangle\langle y_{1}!z \rangle C(x_{0},x_{1},y_{0},y_{1},u)) & \\
  & \wedge \langle y_{1}?(z) \rangle (\langle u! \rangle \langle x_{0}!z \rangle C(x_{0},x_{1},y_{0},y_{1},u)) & \\
  & \wedge \langle x_{1}?(z) \rangle (\langle u? \rangle \langle y_{0}!z \rangle C(x_{0},x_{1},y_{0},y_{1},u)) & \\
  & \wedge \langle y_{0}?(z) \rangle (\langle u? \rangle \langle x_{1}!z \rangle C(x_{0},x_{1},y_{0},y_{1},u))) &
\end{align*}

The lexicographical similarity between the shape of this formulae and
the shape of definition of the process representing a crossing reveals
the intuitive meaning of this formulae. It describes the capabilities
of a process that has the right to represent a crossing. For example
it picks out processes that may perform an input on the port $x_0$ in
its initial menu of capabilities. What differentiates the formula
from the process, however, is that the crossing process is the
smallest candidate to satisfy the formula. Infinitely many other
processes -- with internal behavior hidden behind this interface, so
to speak -- also satisfy this formula. Even this simple formula,
then, can be seen to open a new view onto knots, providing a
computational interpretation of \emph{virtual} knots.

Note that this formula is derived by hand. A similar formula can be
derived by employing Caires' calculation of characteristic formula
\cite{Caires04} to the process representing a crossing. In light of
this discussion, we let
$\meaningof{C}_{\phi}(x0,x1,y0,y1,u)$ denote a formula specifying the
dynamics we wish to capture of a crossing. To guarantee we preserve
the shape of the interface and minimal semantics we demand that
$\meaningof{C}_{\phi}(x0,x1,y0,y1,u) \Rightarrow
\textbf{C}(x0,x1,y0,y1,u)$ where $\textbf{C}(x0,x1,y0,y1,u)$ denotes
the formula above.
                            
\subsubsection{Crossing number constraints.}
The moral content of the context lemma (Lemma \ref{context}) is that the notion of
``locality'' in the Reidemeister moves is effectively captured by the
parallel composition operator of the process calculus. This intuition
extends through the logic. Given a formula,
$\meaningof{C}_{\phi}(x0,x1,y0,y1,u)$, we can use the structural
connectives to specify constraints on crossing numbers, such as at
least $n$ crossings, or exactly $n$ crossings.
\begin{mathpar}
  \inferrule* [lab=at-least-n] {} { K^{\geq n}_{\phi}(\vec{xs},\vec{ys}) := \Pi_{i=0}^{n-1} Hu . \meaningof{C}_{\phi}(xs_i,ys_i,u) | T }
  \and 
  \inferrule* [lab=exactly-n] {} { K^{= n}_{\phi}(\vec{xs},\vec{ys}) := \Pi_{i=0}^{n-1} Hu . \meaningof{C}_{\phi}(xs_i,ys_i,u) | \neg (\forall x_0,y_0,x_1,y_1,u . \meaningof{C}_{\phi}(x_0,y_0,x_1,y_1,u) | T) }
\end{mathpar}

To round out this section, recall that the encoding of an $n$-crossing
knot decomposes into a parallel composition of $n$ \emph{copies} of a
crossing process together with a wiring harness. To specify different
knot classes with the same crossing number amounts to specifying
logical constraints on the wiring harness. In the interest of space,
we defer examples to a forthcoming paper. Suffice it to say that both
the conditions ``alternating knot'' and ``contains the tangle
corresponding to 5/3'' are expressible. For example, it is possible to
calculate the characteristic formula of a process corresponding to the
tangle 5/3 and conjoin it into the classifying formula via the
composition connective of the logic.

Finally, we wish to observe that it is entirely within reason to
contemplate a more domain-specific version of spatial logic tailored
to the shape of processes in the image of the encoding. Such a
domain-specific logic would have a better claim to the title formal
language of knot properties.

% subsection example_formulae_ (end)

% section knots_as_processes (end) 

% section spatial logic via knots (end)

\section{Conclusions and future work}

\paragraph{Testing physical space}
You, gentle reader, may wonder why of all the theorems to be proved
given this set up we pick the one above. In some sense it's hardly
central to quantum mechanics. We see it as central in the sense that
it firmly establishes a notion of physical space arising from a notion
of the equivalence of behavior. Relating bisimulation to a metric is a
big step forward, but one is faced with interpreting the relationship
of that metric space to something more physical. Quantum mechanical
notions of ``physical'' space are still far from intuitive, but by
relating this idea of distance as testing to calculations that predict
physical circumstances we are making a not insignificant step forward
toward an understanding of the physical space we inhabit as
essentially dynamic.

\paragraph{Effectivity and simulation}
One of the observations we have yet to make is that the entire program
spelled out here is effective. We have built various interpreters for
the reflective calculus at work in this interpretation. In principle,
then, we can simulate quantum mechanics on a computer. The place where
the simulation may lose fidelity is the infinitely branching summation
for the annihilator.

In this connection i also want to point out that the evaluation style
calculation of the inner product puts the non-determinism of the
summation right at the heart of measurement. This suggests that
Milner's original reduction-based formulation of the dynamics of his
calculi in terms of sums was not just notationally suggestive of a
notion of measure-and-continue but captured some significant part of
the physics.

\paragraph{Quantum continuations}
In light of this last observation i want to point out that the
predominant account of quantum mechanics is missing a key aspect of a
truly compositional story of the physical situation. In a real lab,
when a measurement is made the observation can be made to feed into
another device that then makes another measurement conditioned on the
results of the first. This means that after the superposition was
collapsed the entire experimental set up remained in
superposition. While QM offers a means of writing this down it doesn't
quite line up well with the well-trodden formulation of computation
and continuation that we see so succinctly expressed in Milner's
calculi. This suggests that there might be advantages to this account
of dynamics waiting to be explored.

\paragraph{Quantum logic}
In this connection, we also note that by virtue of having the
Hennessy-Milner construction, we can pull the construction through the
interpretation of QM. This gives us a natural candidate for a quantum
logic that enjoys an extremely tight connection with it's domain of
interpretation, making the construction much less ad hoc (rather it is
the image of functor!).

\paragraph{Quantum probabiity}
i have questions about the basis of the interpretation of inner
product as probability amplitude. In particular, using which
axiomatization of probability theory does the notion of probability
amplitude earn the right to be so dubbed? In other words, where is the
proof that the operation for calculating a probability amplitude (and
then squaring) satisfies the axioms of what it means to calculate a
probability? Even if such a proof exists (i have yet to find it in the
literature), i wonder if it might not be possible to turn things on
their heads. Can we view the calculation of the probability amplitude
as an axiomatization of probability? If so, then the definition we
give for calculating probability amplitude may provide the basis for
an \emph{effective} theory of probability.

\paragraph{Quantum vs ``biological'' information}
Finally, i want to conclude with a more philosophical observation. At
a recent workshop in which QM was a predominant topic i noticed
something about quantum information. The speaker was giving a riveting
discussion of axiomatic QM and showing how properties of ``no
cloning'' and ``no deleting'' emerged as consequences of the
axiomatization. Theorems of this form are necessary to give us a sense
of confidence that our axioms characterize the physical theory. What
struck me, though, was that if quantum information is neither erasable
nor replicable it is markedly different from \emph{life}. Two of the
things we know about life is that

\begin{itemize}
  \item it ends;
  \item to gain some measure of persistence, to transcend it's
    finitude it is imminently copyable.
\end{itemize}

Both of these qualities are summarized succinctly in the aphorism: all
flesh is grass. For me these two kinds of ``information'' -- call them
quantum and biological -- are end points on a spectrum of strategies
for persistence. At one end, we have those curious entities that enjoy
uniqueness and permanence; at the other, we have those who in the face
of a certain end and an uncertain present make a go of passing
something on. To me one of the more remarkable aspects of the latter
strategy is that in the presence of noise (and certain features of
copying) we get a kind of dynamism, a chance for improvement against a
given persistent condition.

% subsection other_calculi_other_bisimulations_and_geometry_as_behavior (end)




% section conclusion (end)

%\documentclass[12pt]{llncs}
%\documentclass{jktr}

\usepackage[pdftex]{hyperref}                   
\usepackage {listings}
\usepackage {mathpartir}
\usepackage{bcprules}
%\usepackage{listings}
                       
\usepackage{graphicx} 
%\usepackage[margins=2.5cm,nohead,nofoot]{geometry}
%\usepackage{geometry}
\usepackage{amsfonts}
\usepackage{amstext}
\usepackage{latexsym}
\usepackage{amssymb}
\usepackage{color}


%\include{myPreamble}
\include{qm2pi.local} 

%\ifpdf
%\usepackage[pdftex]{graphicx}
%\else
%\usepackage{graphicx}
%\fi

 % \ifpdf
%  \usepackage{pdfsync}
%  \if


%\title{Brief Article}
%\author{David F. Snyder}
%\author{L.G. Meredith}

%\address{Dept. of Math., Texas State University--San Marcos, San Marcos, TX 78666}
       
\pagestyle{empty}


\begin{document}

\lstset{language=[Objective]Caml,frame=shadowbox}

\input{qm2pi.front}

% section front matter (end)

\input{qm2pi.intro} 
 
% section introduction (end)

% \input{qm2pi.knotations} 

% section notation (end)

\input{qm2pi.process.calculi} 

% section concurrent_process_calculi_and_spatial_logics_ (end)
    
%\input{qm2pi.knots2pi} 

%\input{qm2pi.trefoil} 

%\input{qm2pi.mainthm} 

% subsection basic_interpretation (end)

%\input{qm2pi.rho.presentation} 
\subsection{The syntax and semantics of the notation system}\label{sub:the_syntax_and_semantics_of_the_notation_system} % (fold)

We now summarize a technical presentation of the calculus that
embodies our theory of dynamics. The typical presentation of such a
calculus follows the style of giving generators and relations on
them. The grammar, below, describing term constructors, freely
generates the set of processes, $\Proc$. This set is then quotiented
by a relation known as structural congruence and it is over this set
that the notion of dynamics is expressed. This presentation is
essentially that of \cite{MeredithR05} with the addition of
polyadicity and summation. For readability we have relegated some of
the technical subtleties to an appendix.

\subsubsection{Process grammar}\label{subsub:process_grammar}

\begin{mathpar}
  \inferrule* [lab=synchronization] {} {{M} \bc \pzero \;|\; x?F \;|\; x!C }
  \and
  \inferrule* [lab=abstraction] {} {{F} \bc (x)P}
  \and
  \inferrule* [lab=concretion] {} {{C} \bc \langle Q \rangle}
  \and
  \inferrule* [lab=process] {} {{P,Q} \bc M \;| \;P|Q \;|\; @{x}}
  \and
  \inferrule* [lab=name] {} {{x} \bc \quotep{P}}
\end{mathpar} 

Note that $\vec{x}$ (resp. $\vec{P}$) denotes a vector of names
(resp. processes) of length $|\vec{x}|$ (resp. $|\vec{P}|$). We adopt
the following useful abbreviations.

\begin{mathpar}
   x?(\vec{y}).P := x.(\vec{y})P \and  x\clift{\vec{P}} := x.\clift{\vec{P}}
   \and x!(y) := \lift{x}{\dropn{y}}
   \and \Pi_{i=0}^{n-1}P_i := P_0 | \ldots | P_{n-1}
\end{mathpar}

\subsubsection{Structural congruence}

\paragraph{Free and bound names and alpha-equivalence.} At the
core of structural equivalence is alpha-equivalence which identifies
process that are the same up to a change of variable. Formally, we
recognize the distinction between free and bound names. The free names
of a process, $\freenames{P}$, may be calculated recursively as
follows:

\begin{mathpar}
\freenames{\pzero} := \emptyset
  \and \\
  \freenames{x?(y).P} := \{ x \} \cup (\freenames{P} \setminus \{ y \})
  \and 
  \freenames{x!\langle P \rangle} := \{ x \} \cup \{ P \} 
  \and \\
  \freenames{P|Q} := \freenames{P} \cup \freenames{Q}
  \and \\
  \freenames{@{x}} := \{ x \}
\end{mathpar}

$\pi$
$\quotep{\pi}$

$\freenames{-} : \pi \to \mathcal{P}(\quotep{\pi})$

\begin{eqnarray*}
  \freenames{\pzero} & := & \emptyset \\
  \freenames{x?(y).P} & := & \{ x \} \cup (\freenames{P} \setminus \{ y \}) \\
  \freenames{x!\langle P \rangle} & := & \{ x \} \cup \{ P \} \\
  \freenames{P|Q} & := & \freenames{P} \cup \freenames{Q} \\
  \freenames{\dropn{x}} & := & \{ x \}
\end{eqnarray*}

The bound names of a process, $\boundnames{P}$, are those names occurring in $P$
that are not free. For example, in $x?(y).0$, the name $x$ is free, while $y$ is bound.

\begin{mathpar}
  \inferrule* [lab=monoidal-laws] {} { P|Q \equiv Q|P \and P|0 \equiv P \and P|(Q|R) \equiv (P|Q)|R }
\end{mathpar}

\begin{mathpar}
  \inferrule* [lab=alpha-equivalence] {} { (x)P \equiv (y)P\{y/x\} \and y \not\in \freenames{P} }
\end{mathpar}

\begin{definition}
Then two processes, $P,Q$, are alpha-equivalent if $P = Q\{\vec{y}/\vec{x}\}$ for
some $\vec{x} \in \boundnames{Q},\vec{y} \in \boundnames{P}$, where $Q\{\vec{y}/\vec{x}\}$
denotes the capture-avoiding substitution of $\vec{y}$ for $\vec{x}$ in $Q$.
\end{definition}

\begin{definition}
  The {\em structural congruence} \cite{SangiorgiWalker} , $\equiv$,
  between processes is the least congruence containing
  alpha-equivalence, satisfying the abelian monoid laws
  (associativity, commutativity and $\pzero$ as identity) for parallel
  composition $|$ and for summation $+$.
\end{definition}

\subsection{Name equivalence}

We take name equivalence, written $\nameeq$, to be the smallest
equivalence relation generated by the following rules.

\begin{mathpar}
\inferrule*[lab=Quote-drop]
{ }
{ \quotep{@{x}} \nameeq x }

\inferrule*[lab=Struct-equiv]
{ P \scong Q }
{ \quotep{P} \nameeq \quotep{Q} }
\end{mathpar}

The astute reader will have noticed that the mutual recursion of names
and processes imposes a mutual recursion on alpha-equivalence and
structural equivalence via name-equivalence. Fortunately, all of this
works out pleasantly and we may calculate in the natural way, free of
concern. The reader interested in the details is referred to the
appendix \ref{appendix:rho_details}.

\subsection{Substitution}

We use $\Proc$ for the set of processes, $\QProc$ for the set of
names, and $\id{\{}\vec{y} / \vec{x} \id{\}}$ to denote partial maps,
$s : \QProc \rightarrow \QProc$. A map, $s$ lifts, uniquely, to a map
on process terms, $\widehat{s} : \Proc \rightarrow \Proc$ by the
following equations.

\begin{mathpar}
  (0) \psubstp{Q}{P} := 0 \\
  (R \juxtap S) \psubstp{Q}{P}
  :=    
  (R)\psubstp{Q}{P} \juxtap (S) \psubstp{Q}{P} \\
  (x?(y).R) \psubstp{Q}{P}    
  :=    
  (x)\substp{Q}{P} (z)\concat( (R \psubstn{z}{y}) \psubstp{Q}{P} ) \\
  (\lift{x}{R}) \psubstp{Q}{P}  
  :=
  \lift{(x)\substp{Q}{P}}{ R \psubstp{Q}{P} } \\
%   (\dropn{x})  \psubstp{Q}{P}       
%   := 
%   \left\{ 
%     \begin{array}{ccc} 
%       \dropn{\quotep{Q}} & & x \nameeq \quotep{P} \\
%       \dropn{x} & & otherwise \\
%     \end{array}
%   \right. 
  (\dropn{x})  \psubstp{Q}{P}       
  := 
  \left\{ 
    \begin{array}{ccc} 
      Q & & x \nameeq \quotep{P} \\
      \dropn{x} & & otherwise \\
    \end{array}
  \right.
\end{mathpar}
 

where

\begin{eqnarray}
  (x)\id{\{} \lpquote Q \rpquote / \lpquote P \rpquote \id{\}}            = 
  \left\{ 
    \begin{array}{ccc}
      \lpquote Q \rpquote & & x \nameeq \lpquote P \rpquote \\
      x & & otherwise \\
    \end{array}
  \right. \nonumber
\end{eqnarray}

and $z$ is chosen distinct from $\quotep{P}$, $\quotep{Q}$, the free
names in $Q$, and all the names in $R$. Our $\alpha$-equivalence will
be built in the standard way from this substitution.

\begin{remark}\label{rem:no_self_referential_names}
  One consequence of these definitions is that $\forall P. \quotep{P}
  \not\in \freenames{P}$.
\end{remark}

\subsection{ Dynamic quote: an example }

Anticipating something of what's to come, consider applying the
substitution, $\widehat{\id{\{}u / z \id{\}}}$, to the following pair
of processes, $\lift{w}{y!(z)}$ and $w[ \lpquote y!(z) \rpquote ]$.

\begin{eqnarray}
	\lift{w}{y!(z)}\widehat{\id{\{}u / z \id{\}}}
		& = &
		\lift{w}{y!(u)} \nonumber\\
	w[ \lpquote y!(z) \rpquote ] \widehat{ \id{\{}u / z \id{\}} }
		& = &
		w[ \lpquote y!(z) \rpquote ] \nonumber
\end{eqnarray}

Because the body of the process between quotes is impervious to
substitution, we get radically different answers. In fact, by
examining the first process in an input context,
e.g. $x?(z).\lift{w}{y!(z)}$, we see that the process under the lift
operator may be shaped by prefixed inputs binding a name inside it. In
this sense, the lift operator will be seen as a way to dynamically
construct processes before reifying them as names.

Finally equipped with these standard features we can present the
dynamics of the calculus.

\subsubsection{Operational semantics} 

Finally, we introduce the computational dynamics. What marks these
algebras as distinct from other more traditionally studied algebraic
structures, e.g. vector spaces or polynomial rings, is the manner in
which dynamics is captured. In traditional structures, dynamics is typically
expressed through morphisms between such structures, as in linear maps
between vector spaces or morphisms between rings. In algebras
associated with the semantics of computation, the dynamics is
expressed as part of the algebraic structure itself, through a
reduction reduction relation typically denoted by $\red$. Below, we
give a recursive presentation of this relation for the calculus used
in the encoding.

$\red \subseteq \pi \times \pi$
$\red : \pi \to \mathcal{P}(\pi)$

\begin{mathpar}
  \inferrule* [lab=Comm] { \textsf{match}( x_{src}, x_{trgt} ) } { x_{trgt}?(y)P \; | \; x_{src}!\langle {Q} \rangle \red P\{\quotep{Q}/y}\} }
  \and \\
  \inferrule* [lab=Par] {{P} \red {P}'} {{{P} | {Q}} \red {{P}' | {Q}}}
  \and
  \inferrule* [lab=Equiv]{{{P} \scong {P}'} \andalso {{P}' \red {Q}'} \andalso {{Q}' \scong {Q}}}{{P} \red {Q}}
\end{mathpar}

\begin{eqnarray*}
  match_{\equiv} (\quotep{P},\quotep{Q}) & := & P \equiv Q \\
  match_{\dagger}(\quotep{P},\quotep{Q}) & := & \forall R. P|Q \red^{*} R => R \red^{*} 0 \\
  match_{K}(\quotep{P},\quotep{Q}) & := & K \mbox{ for some context } K
\end{eqnarray*}

$u?(x)P | u!\langle Q \rangle \red P\{\quotep{Q}/x\}$

%We write $\wred$ for $\red^*$, and $P\red$ if $\exists Q $ such that $ P \red Q$.
We write $P\red$ if $\exists Q $ such that $ P \red Q$ and $P\not\red$, otherwise.

\section{Replication}

As mentioned before, it is known that replication (and hence
recursion) can be implemented in a higher-order process algebra
\cite{SangiorgiWalker}. As our first example of calculation with the
machinery thus far presented we give the construction explicitly in
the {\rhoc}.

\begin{eqnarray}
	D_{x} & := & \prefix{x}{y}{(\binpar{\outputp{x}{y}}{@{y}})} \nonumber\\
	\bangp_{x}{P} & := & \binpar{{x}!\langle{\binpar{D_{x}}{P}}\rangle}{D_{x}} \nonumber
\end{eqnarray}

\begin{eqnarray}
	\bangp_{x}{P} & & \nonumber\\
	=
	& {x}!\langle{(\prefix{x}{y}{(\outputp{x}{y} | @{y})) | P}}\rangle 
	      | \prefix{x}{y}{(\outputp{x}{y} | @{y})} & \nonumber\\
	\red
	& (\outputp{x}{y} | @{y})\substn{\quotep{(\prefix{x}{y}{(@{y} | \outputp{x}{y})) | P}}}{y} & \nonumber\\
	=
	& \outputp{x}{\quotep{(\prefix{x}{y}{(\outputp{x}{y} | @{y})) | P}}}
	  | {(\prefix{x}{y}{(\outputp{x}{y} | @{y})) | P}} & \nonumber\\
	\red
	& \ldots & \nonumber\\
	\red^*
	& P | P | \ldots & \nonumber
\end{eqnarray}

Of course, this encoding, as an implementation, runs away, unfolding
$\bangp{P}$ eagerly. A lazier and more implementable replication
operator, restricted to input-guarded processes, may be obtained as follows.

\begin{eqnarray}
\bangp{\prefix{u}{v}{P}} 
	:= 
	\binpar{\lift{x}{\prefix{u}{v}{(\binpar{D(x)}{P})}}}{D(x)} \nonumber
\end{eqnarray}

\begin{remark}
  Note that the lazier definition still does not deal with summation
  or mixed summation (i.e. sums over input and output). The reader is
  invited to construct definitions of replication that deal with these
  features. 

  Further, the definitions are parameterized in a name, $x$. Can you,
  gentle reader, make a definition that eliminates this parameter and
  guarantees no accidental interaction between the replication
  machinery and the process being replicated -- i.e. no accidental
  sharing of names used by the process to get its work done and the
  name(s) used by the replication to effect copying. This latter
  revision of the definition of replication is crucial to obtaining
  the expected identity $!!P \sim !P$.
\end{remark}

\begin{remark}\label{rem:paradoxical_combinator}
  The reader familiar with the lambda calculus will have noticed the
  similarity between $D$ and the paradoxical combinator.

  [Ed. note: the existence of this seems to suggest we have to be more
  restrictive on the set of processes and names we admit if we are to
  support no-cloning.]
\end{remark}

\subsubsection{Bisimulation}

The computational dynamics gives rise to another kind of equivalence,
the equivalence of computational behavior. As previously mentioned
this is typically captured \emph{via} some form of bisimulation.

% The notion we use in this paper is weak barbed bisimulation
% \cite{milner91polyadicpi}.

The notion we use in this paper is derived from weak barbed
bisimulation \cite{milner91polyadicpi}. 

\begin{definition}
An \emph{observation relation}, $\downarrow_{\mathcal N}$, over a set
of names, $\mathcal N$, is the smallest relation satisfying the rules
below.

\infrule[Out-barb]{y \in {\mathcal N}, \; x \nameeq y}
		  {\outputp{x}{v} \downarrow_{\mathcal N} x}
\infrule[Par-barb]{\mbox{$P\downarrow_{\mathcal N} x$ or $Q\downarrow_{\mathcal N} x$}}
		  {\binpar{P}{Q} \downarrow_{\mathcal N} x}

We write $P \Downarrow_{\mathcal N} x$ if there is $Q$ such that 
$P \wred Q$ and $Q \downarrow_{\mathcal N} x$.
\end{definition}

\begin{definition}
%\label{def.bbisim}
An  ${\mathcal N}$-\emph{barbed bisimulation} over a set of names, ${\mathcal N}$, is a symmetric binary relation 
${\mathcal S}_{\mathcal N}$ between agents such that $P\rel{S}_{\mathcal N}Q$ implies:
\begin{enumerate}
\item If $P \red P'$ then $Q \wred Q'$ and $P'\rel{S}_{\mathcal N} Q'$.
\item If $P\downarrow_{\mathcal N} x$, then $Q\Downarrow_{\mathcal N} x$.
\end{enumerate}
$P$ is ${\mathcal N}$-barbed bisimilar to $Q$, written
$P \wbbisim_{\mathcal N} Q$, if $P \rel{S}_{\mathcal N} Q$ for some ${\mathcal N}$-barbed bisimulation ${\mathcal S}_{\mathcal N}$.
\end{definition}

$\mathcal{R} \subseteq \pi \times \pi$

$P \mathcal{R} Q => \forall P'. P \red P' \Rightarrow \exists Q'. Q \red Q', P' \mathcal{R} Q'$

$P \vdash x \Rightarrow Q \vdash x$

\begin{mathpar}
  \inferrule*[lab=Out-barb]{x \nameeq y}{{y}!\langle{Q}\rangle \vdash x}
  \and
  \inferrule*[lab=Par-barb]{\mbox{$P\vdash x$ or $Q\vdash x$}}{\binpar{P}{Q} \vdash x}
\end{mathpar}

\subsubsection{Contexts}

One of the principle advantages of computational calculi like the
$\pi$-calculus is a well-defined notion of context,
contextual-equivalence and a correlation between
contextual-equivalence and notions of bisimulation. The notion of
context allows the decomposition of a process into (sub-)process and
its syntactic environment, its context. Thus, a context may be
thought of as a process with a ``hole'' (written $\Box$) in it. The
application of a context $M$ to a process $P$, written $M[P]$, is
tantamount to filling the hole in $M$ with $P$. In this paper we do
not need the full weight of this theory, but do make use of the notion
of context in the proof the main theorem. 

\begin{mathpar}
  \inferrule* [lab=summation] {} {{M_{M},M_{N}} \bc \Box \;|\; x.M_{A} \;|\; M_{M}+M_{N}}
  \and
  \inferrule* [lab=agent] {} {{M_{A}} \bc (\vec{x})M_{P} \;| \; \clift{P_0,\ldots,M_{P},\ldots,P_N}}
  \and \\
  \inferrule* [lab=process] {} {{M_{P}} \bc M_{N} \;| \;P|M_{P} }
\end{mathpar} 

\begin{mathpar}
  \inferrule* [lab=sychronization] {} {M_{N} \bc \Box \;|\; x?M_{F} \;|\; x!M_{C}}
  \and
  \inferrule* [lab=abstraction] {} {{M_{F}} \bc (x)M_{P} }
  \and
  \inferrule* [lab=concretion] {} {{M_{C}} \bc \langle M_{P} \rangle }
  \and \\
  \inferrule* [lab=process] {} {{M_{P}} \bc M_{N} \;| \;P|M_{P} }
\end{mathpar}

\begin{definition}[contextual application] Given a context $M$, and
  process $P$, we define the \emph{contextual application}, $M[P] :=
  M\{P/\Box\}$. That is, the contextual application of M to P is the
  substitution of $P$ for $\Box$ in $M$.
\end{definition}

$\meaningof{-} : L \to \mathcal{P}(\pi)$

\begin{mathpar}
  \inferrule* [lab=collection] {} {\meaningof{true} = \pi, \and \meaningof{~E} = \pi \setminus \meaningof{E}, \and \meaningof{E_{1} \& E_{2}} = \meaningof{E_{1}} \cap \meaningof{E_{2}}}
\end{mathpar}

\begin{mathpar}
  \inferrule* [lab=structure] {} {\meaningof{0} = \{ P \in \pi | P \equiv 0 \}, \and \\ \meaningof{E_1 | E_2} = \{ P \in \pi | P \equiv P_{1} | P_{2}, P_{1} \in \meaningof{E_{1}}, P_{2} \in \meaningof{E_2}\} }
\end{mathpar}

\begin{mathpar}
 \inferrule* [lab=behavior] {} {\meaningof{\langle a?b \rangle E} = \{ P \in \pi | P \equiv Q | u?(y)P', \\ \and \\\\ \and \\ \;\;\; u \in \meaningof{a}, \forall z.P'\{z/y\} \in \meaningof{E\{z/b\}}\}, \and \\ \meaningof{a!E} = \{ P \in \pi | P \equiv Q | x!\langle P' \rangle, x \in \meaningof{a} P' \in \meaningof{E}\} }
\end{mathpar}

\begin{mathpar}
 \inferrule* [lab=nominal] {} {\meaningof{\quotep{E}} = \{ \quotep{P} \in \quotep{\pi} | P \in \meaningof{E} \}, \and \meaningof{\quotep{P}} = \{ \quotep{Q} \in \quotep{\pi} | P \equiv Q \} \and \\ \meaningof{@\quotep{E}} = \{ P \in \pi | P \equiv @x, x \in \meaningof{E} \}}
\end{mathpar}

\begin{eqnarray*}
  \\
  \meaningof{-} : TS \to ST
\end{eqnarray*}

\begin{eqnarray*}
  \\
  L : TS \to ST
\end{eqnarray*}

\begin{eqnarray*}
  \\
  P \models E \iff P \in \meaningof{E}
\end{eqnarray*}

\begin{eqnarray*}
  P \approx_{L} Q \iff \forall E \in L. P \models E \iff Q \models E
\end{eqnarray*}

\begin{eqnarray*}
  P \approx_{K} Q
\end{eqnarray*}

\begin{eqnarray*}
  P \approx Q
\end{eqnarray*}

$\approx_{K} = \approx = \approx_{L}$

\subsubsection{Contextual duality}

Note that contexts extend the quotation operation to a family of
operations from processes to names. Given a context, $M$, we can
define a \emph{nominal context}, $\quotep{M}$ by $\quotep{M}[P] :=
\quotep{M[P]}$. To foreshadow what is to come we observe that these
operations enjoy a duality with processes very much like the duality
between vectors and maps from vectors to scalars.

Further, because the calculus is essentially higher-order, we have a
correspondence between contexts and processes. More specifically,
given a name $x$ and a context $M$ we can construct $M^{*}_{x}$ such
that 

\begin{mathpar}
  M^{*}_{x} | \lift{x}{P} \red M[P]
\end{mathpar}

namely,

\begin{mathpar}
  M^{*}_{x} := x?(u).M[\dropn{u}]
\end{mathpar}

The dependence of $M^{*}_{x}$ on a name makes it an abstraction, 

\begin{mathpar}
  M^{*} := (x)x?(u).M[\dropn{u}]
\end{mathpar}

\subsection{Additional notation}

It will sometimes be convenient to denote the process a name
quotes. We already have the notation $x = \quotep{P}$, but it will be
convenient to introduce an alternate notation, $\procn{x}$, when we
want to emphasize the connection to the use of the name. Note that, by
virtue of name equivalence, $\quotep{\procn{x}} \nameeq x$; so, the
notation is consistent with previous definitions.

Further, because names have structure it is possible to effect
substitutions on the basis of that structure. This means we need to
upgrade our notation for substitutions, which we accomplish by
adapting comprehension notation. Thus,

\begin{mathpar}
  P\{ y / x : x \in S \}
\end{mathpar}

is interpreted to mean the process derived from P by replacing (in a
capture-avoiding manner) each occurrence of $x$ in $S$ by $y$. For example,

\begin{mathpar}
  P\{ \quotep{\procn{x}|\procn{x}} / x : x \in \freenames{P} \}
\end{mathpar}

will replace each (occurrence) of a free name $x$ in $P$ by
$\quotep{\procn{x}|\procn{x}}$.

Also, we will avail ourselves of the notation $x^{L}$ and $x^{R}$ to
denote injections of a name into disjoint copies of the name
space. There are numerous ways to accomplish this. One example can be
found in \cite{MeredithR05}. This notation overloads to vectors of
names: $\vec{x}^{\pi} := (x_{i}^{\pi} \; : \; 0 \leq i < |\vec{x}| )$ where $\pi \in \{L,R\}$.

We also use $P^{\Box} := P|\Box$.

In \cite{MeredithR05} an interpretation of the new operator is
given. It turns out that there are several possible interpretations
all enjoying the requisite algebraic properties of the operator (see
\cite{milner91polyadicpi}). We will therefore make liberal use of
$(\nu\; \vec{x})P$.

% subsection the_syntax_and_semantics_of_the_notation_system (end)   

\input{qm2pi.qmops} 

\input{qm2pi.sterngerlach} 

\input{qm2pi.metric} 

% section concurrent_process_calculi (end)

%\input{qm2pi.proofsketch}

% section proof sketch (end)

%\input{qm2pi.slviaknots} 

% section spatial logic via knots (end)

\input{qm2pi.conclusion}

% section conclusion (end)

%\input{qm2pi.dtcodes} 

% section wiring algorithm (end)

\input{qm2pi.ack} 

% section acknowledgments (end)

\newpage


\bibliographystyle{plain}   
\bibliography{../../biblios/main.bib}

\input{qm2pi.rhodetails}

\end{document}

 

% section wiring algorithm (end)

\documentclass[12pt]{llncs}
%\documentclass{jktr}

\usepackage[pdftex]{hyperref}                   
\usepackage {listings}
\usepackage {mathpartir}
\usepackage{bcprules}
%\usepackage{listings}
                       
\usepackage{graphicx} 
%\usepackage[margins=2.5cm,nohead,nofoot]{geometry}
%\usepackage{geometry}
\usepackage{amsfonts}
\usepackage{amstext}
\usepackage{latexsym}
\usepackage{amssymb}
\usepackage{color}


%\include{myPreamble}
\include{qm2pi.local} 

%\ifpdf
%\usepackage[pdftex]{graphicx}
%\else
%\usepackage{graphicx}
%\fi

 % \ifpdf
%  \usepackage{pdfsync}
%  \if


%\title{Brief Article}
%\author{David F. Snyder}
%\author{L.G. Meredith}

%\address{Dept. of Math., Texas State University--San Marcos, San Marcos, TX 78666}
       
\pagestyle{empty}


\begin{document}

\lstset{language=[Objective]Caml,frame=shadowbox}

\input{qm2pi.front}

% section front matter (end)

\input{qm2pi.intro} 
 
% section introduction (end)

% \input{qm2pi.knotations} 

% section notation (end)

\input{qm2pi.process.calculi} 

% section concurrent_process_calculi_and_spatial_logics_ (end)
    
%\input{qm2pi.knots2pi} 

%\input{qm2pi.trefoil} 

%\input{qm2pi.mainthm} 

% subsection basic_interpretation (end)

%\input{qm2pi.rho.presentation} 
\subsection{The syntax and semantics of the notation system}\label{sub:the_syntax_and_semantics_of_the_notation_system} % (fold)

We now summarize a technical presentation of the calculus that
embodies our theory of dynamics. The typical presentation of such a
calculus follows the style of giving generators and relations on
them. The grammar, below, describing term constructors, freely
generates the set of processes, $\Proc$. This set is then quotiented
by a relation known as structural congruence and it is over this set
that the notion of dynamics is expressed. This presentation is
essentially that of \cite{MeredithR05} with the addition of
polyadicity and summation. For readability we have relegated some of
the technical subtleties to an appendix.

\subsubsection{Process grammar}\label{subsub:process_grammar}

\begin{mathpar}
  \inferrule* [lab=synchronization] {} {{M} \bc \pzero \;|\; x?F \;|\; x!C }
  \and
  \inferrule* [lab=abstraction] {} {{F} \bc (x)P}
  \and
  \inferrule* [lab=concretion] {} {{C} \bc \langle Q \rangle}
  \and
  \inferrule* [lab=process] {} {{P,Q} \bc M \;| \;P|Q \;|\; @{x}}
  \and
  \inferrule* [lab=name] {} {{x} \bc \quotep{P}}
\end{mathpar} 

Note that $\vec{x}$ (resp. $\vec{P}$) denotes a vector of names
(resp. processes) of length $|\vec{x}|$ (resp. $|\vec{P}|$). We adopt
the following useful abbreviations.

\begin{mathpar}
   x?(\vec{y}).P := x.(\vec{y})P \and  x\clift{\vec{P}} := x.\clift{\vec{P}}
   \and x!(y) := \lift{x}{\dropn{y}}
   \and \Pi_{i=0}^{n-1}P_i := P_0 | \ldots | P_{n-1}
\end{mathpar}

\subsubsection{Structural congruence}

\paragraph{Free and bound names and alpha-equivalence.} At the
core of structural equivalence is alpha-equivalence which identifies
process that are the same up to a change of variable. Formally, we
recognize the distinction between free and bound names. The free names
of a process, $\freenames{P}$, may be calculated recursively as
follows:

\begin{mathpar}
\freenames{\pzero} := \emptyset
  \and \\
  \freenames{x?(y).P} := \{ x \} \cup (\freenames{P} \setminus \{ y \})
  \and 
  \freenames{x!\langle P \rangle} := \{ x \} \cup \{ P \} 
  \and \\
  \freenames{P|Q} := \freenames{P} \cup \freenames{Q}
  \and \\
  \freenames{@{x}} := \{ x \}
\end{mathpar}

$\pi$
$\quotep{\pi}$

$\freenames{-} : \pi \to \mathcal{P}(\quotep{\pi})$

\begin{eqnarray*}
  \freenames{\pzero} & := & \emptyset \\
  \freenames{x?(y).P} & := & \{ x \} \cup (\freenames{P} \setminus \{ y \}) \\
  \freenames{x!\langle P \rangle} & := & \{ x \} \cup \{ P \} \\
  \freenames{P|Q} & := & \freenames{P} \cup \freenames{Q} \\
  \freenames{\dropn{x}} & := & \{ x \}
\end{eqnarray*}

The bound names of a process, $\boundnames{P}$, are those names occurring in $P$
that are not free. For example, in $x?(y).0$, the name $x$ is free, while $y$ is bound.

\begin{mathpar}
  \inferrule* [lab=monoidal-laws] {} { P|Q \equiv Q|P \and P|0 \equiv P \and P|(Q|R) \equiv (P|Q)|R }
\end{mathpar}

\begin{mathpar}
  \inferrule* [lab=alpha-equivalence] {} { (x)P \equiv (y)P\{y/x\} \and y \not\in \freenames{P} }
\end{mathpar}

\begin{definition}
Then two processes, $P,Q$, are alpha-equivalent if $P = Q\{\vec{y}/\vec{x}\}$ for
some $\vec{x} \in \boundnames{Q},\vec{y} \in \boundnames{P}$, where $Q\{\vec{y}/\vec{x}\}$
denotes the capture-avoiding substitution of $\vec{y}$ for $\vec{x}$ in $Q$.
\end{definition}

\begin{definition}
  The {\em structural congruence} \cite{SangiorgiWalker} , $\equiv$,
  between processes is the least congruence containing
  alpha-equivalence, satisfying the abelian monoid laws
  (associativity, commutativity and $\pzero$ as identity) for parallel
  composition $|$ and for summation $+$.
\end{definition}

\subsection{Name equivalence}

We take name equivalence, written $\nameeq$, to be the smallest
equivalence relation generated by the following rules.

\begin{mathpar}
\inferrule*[lab=Quote-drop]
{ }
{ \quotep{@{x}} \nameeq x }

\inferrule*[lab=Struct-equiv]
{ P \scong Q }
{ \quotep{P} \nameeq \quotep{Q} }
\end{mathpar}

The astute reader will have noticed that the mutual recursion of names
and processes imposes a mutual recursion on alpha-equivalence and
structural equivalence via name-equivalence. Fortunately, all of this
works out pleasantly and we may calculate in the natural way, free of
concern. The reader interested in the details is referred to the
appendix \ref{appendix:rho_details}.

\subsection{Substitution}

We use $\Proc$ for the set of processes, $\QProc$ for the set of
names, and $\id{\{}\vec{y} / \vec{x} \id{\}}$ to denote partial maps,
$s : \QProc \rightarrow \QProc$. A map, $s$ lifts, uniquely, to a map
on process terms, $\widehat{s} : \Proc \rightarrow \Proc$ by the
following equations.

\begin{mathpar}
  (0) \psubstp{Q}{P} := 0 \\
  (R \juxtap S) \psubstp{Q}{P}
  :=    
  (R)\psubstp{Q}{P} \juxtap (S) \psubstp{Q}{P} \\
  (x?(y).R) \psubstp{Q}{P}    
  :=    
  (x)\substp{Q}{P} (z)\concat( (R \psubstn{z}{y}) \psubstp{Q}{P} ) \\
  (\lift{x}{R}) \psubstp{Q}{P}  
  :=
  \lift{(x)\substp{Q}{P}}{ R \psubstp{Q}{P} } \\
%   (\dropn{x})  \psubstp{Q}{P}       
%   := 
%   \left\{ 
%     \begin{array}{ccc} 
%       \dropn{\quotep{Q}} & & x \nameeq \quotep{P} \\
%       \dropn{x} & & otherwise \\
%     \end{array}
%   \right. 
  (\dropn{x})  \psubstp{Q}{P}       
  := 
  \left\{ 
    \begin{array}{ccc} 
      Q & & x \nameeq \quotep{P} \\
      \dropn{x} & & otherwise \\
    \end{array}
  \right.
\end{mathpar}
 

where

\begin{eqnarray}
  (x)\id{\{} \lpquote Q \rpquote / \lpquote P \rpquote \id{\}}            = 
  \left\{ 
    \begin{array}{ccc}
      \lpquote Q \rpquote & & x \nameeq \lpquote P \rpquote \\
      x & & otherwise \\
    \end{array}
  \right. \nonumber
\end{eqnarray}

and $z$ is chosen distinct from $\quotep{P}$, $\quotep{Q}$, the free
names in $Q$, and all the names in $R$. Our $\alpha$-equivalence will
be built in the standard way from this substitution.

\begin{remark}\label{rem:no_self_referential_names}
  One consequence of these definitions is that $\forall P. \quotep{P}
  \not\in \freenames{P}$.
\end{remark}

\subsection{ Dynamic quote: an example }

Anticipating something of what's to come, consider applying the
substitution, $\widehat{\id{\{}u / z \id{\}}}$, to the following pair
of processes, $\lift{w}{y!(z)}$ and $w[ \lpquote y!(z) \rpquote ]$.

\begin{eqnarray}
	\lift{w}{y!(z)}\widehat{\id{\{}u / z \id{\}}}
		& = &
		\lift{w}{y!(u)} \nonumber\\
	w[ \lpquote y!(z) \rpquote ] \widehat{ \id{\{}u / z \id{\}} }
		& = &
		w[ \lpquote y!(z) \rpquote ] \nonumber
\end{eqnarray}

Because the body of the process between quotes is impervious to
substitution, we get radically different answers. In fact, by
examining the first process in an input context,
e.g. $x?(z).\lift{w}{y!(z)}$, we see that the process under the lift
operator may be shaped by prefixed inputs binding a name inside it. In
this sense, the lift operator will be seen as a way to dynamically
construct processes before reifying them as names.

Finally equipped with these standard features we can present the
dynamics of the calculus.

\subsubsection{Operational semantics} 

Finally, we introduce the computational dynamics. What marks these
algebras as distinct from other more traditionally studied algebraic
structures, e.g. vector spaces or polynomial rings, is the manner in
which dynamics is captured. In traditional structures, dynamics is typically
expressed through morphisms between such structures, as in linear maps
between vector spaces or morphisms between rings. In algebras
associated with the semantics of computation, the dynamics is
expressed as part of the algebraic structure itself, through a
reduction reduction relation typically denoted by $\red$. Below, we
give a recursive presentation of this relation for the calculus used
in the encoding.

$\red \subseteq \pi \times \pi$
$\red : \pi \to \mathcal{P}(\pi)$

\begin{mathpar}
  \inferrule* [lab=Comm] { \textsf{match}( x_{src}, x_{trgt} ) } { x_{trgt}?(y)P \; | \; x_{src}!\langle {Q} \rangle \red P\{\quotep{Q}/y}\} }
  \and \\
  \inferrule* [lab=Par] {{P} \red {P}'} {{{P} | {Q}} \red {{P}' | {Q}}}
  \and
  \inferrule* [lab=Equiv]{{{P} \scong {P}'} \andalso {{P}' \red {Q}'} \andalso {{Q}' \scong {Q}}}{{P} \red {Q}}
\end{mathpar}

\begin{eqnarray*}
  match_{\equiv} (\quotep{P},\quotep{Q}) & := & P \equiv Q \\
  match_{\dagger}(\quotep{P},\quotep{Q}) & := & \forall R. P|Q \red^{*} R => R \red^{*} 0 \\
  match_{K}(\quotep{P},\quotep{Q}) & := & K \mbox{ for some context } K
\end{eqnarray*}

$u?(x)P | u!\langle Q \rangle \red P\{\quotep{Q}/x\}$

%We write $\wred$ for $\red^*$, and $P\red$ if $\exists Q $ such that $ P \red Q$.
We write $P\red$ if $\exists Q $ such that $ P \red Q$ and $P\not\red$, otherwise.

\section{Replication}

As mentioned before, it is known that replication (and hence
recursion) can be implemented in a higher-order process algebra
\cite{SangiorgiWalker}. As our first example of calculation with the
machinery thus far presented we give the construction explicitly in
the {\rhoc}.

\begin{eqnarray}
	D_{x} & := & \prefix{x}{y}{(\binpar{\outputp{x}{y}}{@{y}})} \nonumber\\
	\bangp_{x}{P} & := & \binpar{{x}!\langle{\binpar{D_{x}}{P}}\rangle}{D_{x}} \nonumber
\end{eqnarray}

\begin{eqnarray}
	\bangp_{x}{P} & & \nonumber\\
	=
	& {x}!\langle{(\prefix{x}{y}{(\outputp{x}{y} | @{y})) | P}}\rangle 
	      | \prefix{x}{y}{(\outputp{x}{y} | @{y})} & \nonumber\\
	\red
	& (\outputp{x}{y} | @{y})\substn{\quotep{(\prefix{x}{y}{(@{y} | \outputp{x}{y})) | P}}}{y} & \nonumber\\
	=
	& \outputp{x}{\quotep{(\prefix{x}{y}{(\outputp{x}{y} | @{y})) | P}}}
	  | {(\prefix{x}{y}{(\outputp{x}{y} | @{y})) | P}} & \nonumber\\
	\red
	& \ldots & \nonumber\\
	\red^*
	& P | P | \ldots & \nonumber
\end{eqnarray}

Of course, this encoding, as an implementation, runs away, unfolding
$\bangp{P}$ eagerly. A lazier and more implementable replication
operator, restricted to input-guarded processes, may be obtained as follows.

\begin{eqnarray}
\bangp{\prefix{u}{v}{P}} 
	:= 
	\binpar{\lift{x}{\prefix{u}{v}{(\binpar{D(x)}{P})}}}{D(x)} \nonumber
\end{eqnarray}

\begin{remark}
  Note that the lazier definition still does not deal with summation
  or mixed summation (i.e. sums over input and output). The reader is
  invited to construct definitions of replication that deal with these
  features. 

  Further, the definitions are parameterized in a name, $x$. Can you,
  gentle reader, make a definition that eliminates this parameter and
  guarantees no accidental interaction between the replication
  machinery and the process being replicated -- i.e. no accidental
  sharing of names used by the process to get its work done and the
  name(s) used by the replication to effect copying. This latter
  revision of the definition of replication is crucial to obtaining
  the expected identity $!!P \sim !P$.
\end{remark}

\begin{remark}\label{rem:paradoxical_combinator}
  The reader familiar with the lambda calculus will have noticed the
  similarity between $D$ and the paradoxical combinator.

  [Ed. note: the existence of this seems to suggest we have to be more
  restrictive on the set of processes and names we admit if we are to
  support no-cloning.]
\end{remark}

\subsubsection{Bisimulation}

The computational dynamics gives rise to another kind of equivalence,
the equivalence of computational behavior. As previously mentioned
this is typically captured \emph{via} some form of bisimulation.

% The notion we use in this paper is weak barbed bisimulation
% \cite{milner91polyadicpi}.

The notion we use in this paper is derived from weak barbed
bisimulation \cite{milner91polyadicpi}. 

\begin{definition}
An \emph{observation relation}, $\downarrow_{\mathcal N}$, over a set
of names, $\mathcal N$, is the smallest relation satisfying the rules
below.

\infrule[Out-barb]{y \in {\mathcal N}, \; x \nameeq y}
		  {\outputp{x}{v} \downarrow_{\mathcal N} x}
\infrule[Par-barb]{\mbox{$P\downarrow_{\mathcal N} x$ or $Q\downarrow_{\mathcal N} x$}}
		  {\binpar{P}{Q} \downarrow_{\mathcal N} x}

We write $P \Downarrow_{\mathcal N} x$ if there is $Q$ such that 
$P \wred Q$ and $Q \downarrow_{\mathcal N} x$.
\end{definition}

\begin{definition}
%\label{def.bbisim}
An  ${\mathcal N}$-\emph{barbed bisimulation} over a set of names, ${\mathcal N}$, is a symmetric binary relation 
${\mathcal S}_{\mathcal N}$ between agents such that $P\rel{S}_{\mathcal N}Q$ implies:
\begin{enumerate}
\item If $P \red P'$ then $Q \wred Q'$ and $P'\rel{S}_{\mathcal N} Q'$.
\item If $P\downarrow_{\mathcal N} x$, then $Q\Downarrow_{\mathcal N} x$.
\end{enumerate}
$P$ is ${\mathcal N}$-barbed bisimilar to $Q$, written
$P \wbbisim_{\mathcal N} Q$, if $P \rel{S}_{\mathcal N} Q$ for some ${\mathcal N}$-barbed bisimulation ${\mathcal S}_{\mathcal N}$.
\end{definition}

$\mathcal{R} \subseteq \pi \times \pi$

$P \mathcal{R} Q => \forall P'. P \red P' \Rightarrow \exists Q'. Q \red Q', P' \mathcal{R} Q'$

$P \vdash x \Rightarrow Q \vdash x$

\begin{mathpar}
  \inferrule*[lab=Out-barb]{x \nameeq y}{{y}!\langle{Q}\rangle \vdash x}
  \and
  \inferrule*[lab=Par-barb]{\mbox{$P\vdash x$ or $Q\vdash x$}}{\binpar{P}{Q} \vdash x}
\end{mathpar}

\subsubsection{Contexts}

One of the principle advantages of computational calculi like the
$\pi$-calculus is a well-defined notion of context,
contextual-equivalence and a correlation between
contextual-equivalence and notions of bisimulation. The notion of
context allows the decomposition of a process into (sub-)process and
its syntactic environment, its context. Thus, a context may be
thought of as a process with a ``hole'' (written $\Box$) in it. The
application of a context $M$ to a process $P$, written $M[P]$, is
tantamount to filling the hole in $M$ with $P$. In this paper we do
not need the full weight of this theory, but do make use of the notion
of context in the proof the main theorem. 

\begin{mathpar}
  \inferrule* [lab=summation] {} {{M_{M},M_{N}} \bc \Box \;|\; x.M_{A} \;|\; M_{M}+M_{N}}
  \and
  \inferrule* [lab=agent] {} {{M_{A}} \bc (\vec{x})M_{P} \;| \; \clift{P_0,\ldots,M_{P},\ldots,P_N}}
  \and \\
  \inferrule* [lab=process] {} {{M_{P}} \bc M_{N} \;| \;P|M_{P} }
\end{mathpar} 

\begin{mathpar}
  \inferrule* [lab=sychronization] {} {M_{N} \bc \Box \;|\; x?M_{F} \;|\; x!M_{C}}
  \and
  \inferrule* [lab=abstraction] {} {{M_{F}} \bc (x)M_{P} }
  \and
  \inferrule* [lab=concretion] {} {{M_{C}} \bc \langle M_{P} \rangle }
  \and \\
  \inferrule* [lab=process] {} {{M_{P}} \bc M_{N} \;| \;P|M_{P} }
\end{mathpar}

\begin{definition}[contextual application] Given a context $M$, and
  process $P$, we define the \emph{contextual application}, $M[P] :=
  M\{P/\Box\}$. That is, the contextual application of M to P is the
  substitution of $P$ for $\Box$ in $M$.
\end{definition}

$\meaningof{-} : L \to \mathcal{P}(\pi)$

\begin{mathpar}
  \inferrule* [lab=collection] {} {\meaningof{true} = \pi, \and \meaningof{~E} = \pi \setminus \meaningof{E}, \and \meaningof{E_{1} \& E_{2}} = \meaningof{E_{1}} \cap \meaningof{E_{2}}}
\end{mathpar}

\begin{mathpar}
  \inferrule* [lab=structure] {} {\meaningof{0} = \{ P \in \pi | P \equiv 0 \}, \and \\ \meaningof{E_1 | E_2} = \{ P \in \pi | P \equiv P_{1} | P_{2}, P_{1} \in \meaningof{E_{1}}, P_{2} \in \meaningof{E_2}\} }
\end{mathpar}

\begin{mathpar}
 \inferrule* [lab=behavior] {} {\meaningof{\langle a?b \rangle E} = \{ P \in \pi | P \equiv Q | u?(y)P', \\ \and \\\\ \and \\ \;\;\; u \in \meaningof{a}, \forall z.P'\{z/y\} \in \meaningof{E\{z/b\}}\}, \and \\ \meaningof{a!E} = \{ P \in \pi | P \equiv Q | x!\langle P' \rangle, x \in \meaningof{a} P' \in \meaningof{E}\} }
\end{mathpar}

\begin{mathpar}
 \inferrule* [lab=nominal] {} {\meaningof{\quotep{E}} = \{ \quotep{P} \in \quotep{\pi} | P \in \meaningof{E} \}, \and \meaningof{\quotep{P}} = \{ \quotep{Q} \in \quotep{\pi} | P \equiv Q \} \and \\ \meaningof{@\quotep{E}} = \{ P \in \pi | P \equiv @x, x \in \meaningof{E} \}}
\end{mathpar}

\begin{eqnarray*}
  \\
  \meaningof{-} : TS \to ST
\end{eqnarray*}

\begin{eqnarray*}
  \\
  L : TS \to ST
\end{eqnarray*}

\begin{eqnarray*}
  \\
  P \models E \iff P \in \meaningof{E}
\end{eqnarray*}

\begin{eqnarray*}
  P \approx_{L} Q \iff \forall E \in L. P \models E \iff Q \models E
\end{eqnarray*}

\begin{eqnarray*}
  P \approx_{K} Q
\end{eqnarray*}

\begin{eqnarray*}
  P \approx Q
\end{eqnarray*}

$\approx_{K} = \approx = \approx_{L}$

\subsubsection{Contextual duality}

Note that contexts extend the quotation operation to a family of
operations from processes to names. Given a context, $M$, we can
define a \emph{nominal context}, $\quotep{M}$ by $\quotep{M}[P] :=
\quotep{M[P]}$. To foreshadow what is to come we observe that these
operations enjoy a duality with processes very much like the duality
between vectors and maps from vectors to scalars.

Further, because the calculus is essentially higher-order, we have a
correspondence between contexts and processes. More specifically,
given a name $x$ and a context $M$ we can construct $M^{*}_{x}$ such
that 

\begin{mathpar}
  M^{*}_{x} | \lift{x}{P} \red M[P]
\end{mathpar}

namely,

\begin{mathpar}
  M^{*}_{x} := x?(u).M[\dropn{u}]
\end{mathpar}

The dependence of $M^{*}_{x}$ on a name makes it an abstraction, 

\begin{mathpar}
  M^{*} := (x)x?(u).M[\dropn{u}]
\end{mathpar}

\subsection{Additional notation}

It will sometimes be convenient to denote the process a name
quotes. We already have the notation $x = \quotep{P}$, but it will be
convenient to introduce an alternate notation, $\procn{x}$, when we
want to emphasize the connection to the use of the name. Note that, by
virtue of name equivalence, $\quotep{\procn{x}} \nameeq x$; so, the
notation is consistent with previous definitions.

Further, because names have structure it is possible to effect
substitutions on the basis of that structure. This means we need to
upgrade our notation for substitutions, which we accomplish by
adapting comprehension notation. Thus,

\begin{mathpar}
  P\{ y / x : x \in S \}
\end{mathpar}

is interpreted to mean the process derived from P by replacing (in a
capture-avoiding manner) each occurrence of $x$ in $S$ by $y$. For example,

\begin{mathpar}
  P\{ \quotep{\procn{x}|\procn{x}} / x : x \in \freenames{P} \}
\end{mathpar}

will replace each (occurrence) of a free name $x$ in $P$ by
$\quotep{\procn{x}|\procn{x}}$.

Also, we will avail ourselves of the notation $x^{L}$ and $x^{R}$ to
denote injections of a name into disjoint copies of the name
space. There are numerous ways to accomplish this. One example can be
found in \cite{MeredithR05}. This notation overloads to vectors of
names: $\vec{x}^{\pi} := (x_{i}^{\pi} \; : \; 0 \leq i < |\vec{x}| )$ where $\pi \in \{L,R\}$.

We also use $P^{\Box} := P|\Box$.

In \cite{MeredithR05} an interpretation of the new operator is
given. It turns out that there are several possible interpretations
all enjoying the requisite algebraic properties of the operator (see
\cite{milner91polyadicpi}). We will therefore make liberal use of
$(\nu\; \vec{x})P$.

% subsection the_syntax_and_semantics_of_the_notation_system (end)   

\input{qm2pi.qmops} 

\input{qm2pi.sterngerlach} 

\input{qm2pi.metric} 

% section concurrent_process_calculi (end)

%\input{qm2pi.proofsketch}

% section proof sketch (end)

%\input{qm2pi.slviaknots} 

% section spatial logic via knots (end)

\input{qm2pi.conclusion}

% section conclusion (end)

%\input{qm2pi.dtcodes} 

% section wiring algorithm (end)

\input{qm2pi.ack} 

% section acknowledgments (end)

\newpage


\bibliographystyle{plain}   
\bibliography{../../biblios/main.bib}

\input{qm2pi.rhodetails}

\end{document}

 

% section acknowledgments (end)

\newpage


\bibliographystyle{plain}   
\bibliography{../../biblios/main.bib}

\documentclass[12pt]{llncs}
%\documentclass{jktr}

\usepackage[pdftex]{hyperref}                   
\usepackage {listings}
\usepackage {mathpartir}
\usepackage{bcprules}
%\usepackage{listings}
                       
\usepackage{graphicx} 
%\usepackage[margins=2.5cm,nohead,nofoot]{geometry}
%\usepackage{geometry}
\usepackage{amsfonts}
\usepackage{amstext}
\usepackage{latexsym}
\usepackage{amssymb}
\usepackage{color}


%\include{myPreamble}
\include{qm2pi.local} 

%\ifpdf
%\usepackage[pdftex]{graphicx}
%\else
%\usepackage{graphicx}
%\fi

 % \ifpdf
%  \usepackage{pdfsync}
%  \if


%\title{Brief Article}
%\author{David F. Snyder}
%\author{L.G. Meredith}

%\address{Dept. of Math., Texas State University--San Marcos, San Marcos, TX 78666}
       
\pagestyle{empty}


\begin{document}

\lstset{language=[Objective]Caml,frame=shadowbox}

\input{qm2pi.front}

% section front matter (end)

\input{qm2pi.intro} 
 
% section introduction (end)

% \input{qm2pi.knotations} 

% section notation (end)

\input{qm2pi.process.calculi} 

% section concurrent_process_calculi_and_spatial_logics_ (end)
    
%\input{qm2pi.knots2pi} 

%\input{qm2pi.trefoil} 

%\input{qm2pi.mainthm} 

% subsection basic_interpretation (end)

%\input{qm2pi.rho.presentation} 
\subsection{The syntax and semantics of the notation system}\label{sub:the_syntax_and_semantics_of_the_notation_system} % (fold)

We now summarize a technical presentation of the calculus that
embodies our theory of dynamics. The typical presentation of such a
calculus follows the style of giving generators and relations on
them. The grammar, below, describing term constructors, freely
generates the set of processes, $\Proc$. This set is then quotiented
by a relation known as structural congruence and it is over this set
that the notion of dynamics is expressed. This presentation is
essentially that of \cite{MeredithR05} with the addition of
polyadicity and summation. For readability we have relegated some of
the technical subtleties to an appendix.

\subsubsection{Process grammar}\label{subsub:process_grammar}

\begin{mathpar}
  \inferrule* [lab=synchronization] {} {{M} \bc \pzero \;|\; x?F \;|\; x!C }
  \and
  \inferrule* [lab=abstraction] {} {{F} \bc (x)P}
  \and
  \inferrule* [lab=concretion] {} {{C} \bc \langle Q \rangle}
  \and
  \inferrule* [lab=process] {} {{P,Q} \bc M \;| \;P|Q \;|\; @{x}}
  \and
  \inferrule* [lab=name] {} {{x} \bc \quotep{P}}
\end{mathpar} 

Note that $\vec{x}$ (resp. $\vec{P}$) denotes a vector of names
(resp. processes) of length $|\vec{x}|$ (resp. $|\vec{P}|$). We adopt
the following useful abbreviations.

\begin{mathpar}
   x?(\vec{y}).P := x.(\vec{y})P \and  x\clift{\vec{P}} := x.\clift{\vec{P}}
   \and x!(y) := \lift{x}{\dropn{y}}
   \and \Pi_{i=0}^{n-1}P_i := P_0 | \ldots | P_{n-1}
\end{mathpar}

\subsubsection{Structural congruence}

\paragraph{Free and bound names and alpha-equivalence.} At the
core of structural equivalence is alpha-equivalence which identifies
process that are the same up to a change of variable. Formally, we
recognize the distinction between free and bound names. The free names
of a process, $\freenames{P}$, may be calculated recursively as
follows:

\begin{mathpar}
\freenames{\pzero} := \emptyset
  \and \\
  \freenames{x?(y).P} := \{ x \} \cup (\freenames{P} \setminus \{ y \})
  \and 
  \freenames{x!\langle P \rangle} := \{ x \} \cup \{ P \} 
  \and \\
  \freenames{P|Q} := \freenames{P} \cup \freenames{Q}
  \and \\
  \freenames{@{x}} := \{ x \}
\end{mathpar}

$\pi$
$\quotep{\pi}$

$\freenames{-} : \pi \to \mathcal{P}(\quotep{\pi})$

\begin{eqnarray*}
  \freenames{\pzero} & := & \emptyset \\
  \freenames{x?(y).P} & := & \{ x \} \cup (\freenames{P} \setminus \{ y \}) \\
  \freenames{x!\langle P \rangle} & := & \{ x \} \cup \{ P \} \\
  \freenames{P|Q} & := & \freenames{P} \cup \freenames{Q} \\
  \freenames{\dropn{x}} & := & \{ x \}
\end{eqnarray*}

The bound names of a process, $\boundnames{P}$, are those names occurring in $P$
that are not free. For example, in $x?(y).0$, the name $x$ is free, while $y$ is bound.

\begin{mathpar}
  \inferrule* [lab=monoidal-laws] {} { P|Q \equiv Q|P \and P|0 \equiv P \and P|(Q|R) \equiv (P|Q)|R }
\end{mathpar}

\begin{mathpar}
  \inferrule* [lab=alpha-equivalence] {} { (x)P \equiv (y)P\{y/x\} \and y \not\in \freenames{P} }
\end{mathpar}

\begin{definition}
Then two processes, $P,Q$, are alpha-equivalent if $P = Q\{\vec{y}/\vec{x}\}$ for
some $\vec{x} \in \boundnames{Q},\vec{y} \in \boundnames{P}$, where $Q\{\vec{y}/\vec{x}\}$
denotes the capture-avoiding substitution of $\vec{y}$ for $\vec{x}$ in $Q$.
\end{definition}

\begin{definition}
  The {\em structural congruence} \cite{SangiorgiWalker} , $\equiv$,
  between processes is the least congruence containing
  alpha-equivalence, satisfying the abelian monoid laws
  (associativity, commutativity and $\pzero$ as identity) for parallel
  composition $|$ and for summation $+$.
\end{definition}

\subsection{Name equivalence}

We take name equivalence, written $\nameeq$, to be the smallest
equivalence relation generated by the following rules.

\begin{mathpar}
\inferrule*[lab=Quote-drop]
{ }
{ \quotep{@{x}} \nameeq x }

\inferrule*[lab=Struct-equiv]
{ P \scong Q }
{ \quotep{P} \nameeq \quotep{Q} }
\end{mathpar}

The astute reader will have noticed that the mutual recursion of names
and processes imposes a mutual recursion on alpha-equivalence and
structural equivalence via name-equivalence. Fortunately, all of this
works out pleasantly and we may calculate in the natural way, free of
concern. The reader interested in the details is referred to the
appendix \ref{appendix:rho_details}.

\subsection{Substitution}

We use $\Proc$ for the set of processes, $\QProc$ for the set of
names, and $\id{\{}\vec{y} / \vec{x} \id{\}}$ to denote partial maps,
$s : \QProc \rightarrow \QProc$. A map, $s$ lifts, uniquely, to a map
on process terms, $\widehat{s} : \Proc \rightarrow \Proc$ by the
following equations.

\begin{mathpar}
  (0) \psubstp{Q}{P} := 0 \\
  (R \juxtap S) \psubstp{Q}{P}
  :=    
  (R)\psubstp{Q}{P} \juxtap (S) \psubstp{Q}{P} \\
  (x?(y).R) \psubstp{Q}{P}    
  :=    
  (x)\substp{Q}{P} (z)\concat( (R \psubstn{z}{y}) \psubstp{Q}{P} ) \\
  (\lift{x}{R}) \psubstp{Q}{P}  
  :=
  \lift{(x)\substp{Q}{P}}{ R \psubstp{Q}{P} } \\
%   (\dropn{x})  \psubstp{Q}{P}       
%   := 
%   \left\{ 
%     \begin{array}{ccc} 
%       \dropn{\quotep{Q}} & & x \nameeq \quotep{P} \\
%       \dropn{x} & & otherwise \\
%     \end{array}
%   \right. 
  (\dropn{x})  \psubstp{Q}{P}       
  := 
  \left\{ 
    \begin{array}{ccc} 
      Q & & x \nameeq \quotep{P} \\
      \dropn{x} & & otherwise \\
    \end{array}
  \right.
\end{mathpar}
 

where

\begin{eqnarray}
  (x)\id{\{} \lpquote Q \rpquote / \lpquote P \rpquote \id{\}}            = 
  \left\{ 
    \begin{array}{ccc}
      \lpquote Q \rpquote & & x \nameeq \lpquote P \rpquote \\
      x & & otherwise \\
    \end{array}
  \right. \nonumber
\end{eqnarray}

and $z$ is chosen distinct from $\quotep{P}$, $\quotep{Q}$, the free
names in $Q$, and all the names in $R$. Our $\alpha$-equivalence will
be built in the standard way from this substitution.

\begin{remark}\label{rem:no_self_referential_names}
  One consequence of these definitions is that $\forall P. \quotep{P}
  \not\in \freenames{P}$.
\end{remark}

\subsection{ Dynamic quote: an example }

Anticipating something of what's to come, consider applying the
substitution, $\widehat{\id{\{}u / z \id{\}}}$, to the following pair
of processes, $\lift{w}{y!(z)}$ and $w[ \lpquote y!(z) \rpquote ]$.

\begin{eqnarray}
	\lift{w}{y!(z)}\widehat{\id{\{}u / z \id{\}}}
		& = &
		\lift{w}{y!(u)} \nonumber\\
	w[ \lpquote y!(z) \rpquote ] \widehat{ \id{\{}u / z \id{\}} }
		& = &
		w[ \lpquote y!(z) \rpquote ] \nonumber
\end{eqnarray}

Because the body of the process between quotes is impervious to
substitution, we get radically different answers. In fact, by
examining the first process in an input context,
e.g. $x?(z).\lift{w}{y!(z)}$, we see that the process under the lift
operator may be shaped by prefixed inputs binding a name inside it. In
this sense, the lift operator will be seen as a way to dynamically
construct processes before reifying them as names.

Finally equipped with these standard features we can present the
dynamics of the calculus.

\subsubsection{Operational semantics} 

Finally, we introduce the computational dynamics. What marks these
algebras as distinct from other more traditionally studied algebraic
structures, e.g. vector spaces or polynomial rings, is the manner in
which dynamics is captured. In traditional structures, dynamics is typically
expressed through morphisms between such structures, as in linear maps
between vector spaces or morphisms between rings. In algebras
associated with the semantics of computation, the dynamics is
expressed as part of the algebraic structure itself, through a
reduction reduction relation typically denoted by $\red$. Below, we
give a recursive presentation of this relation for the calculus used
in the encoding.

$\red \subseteq \pi \times \pi$
$\red : \pi \to \mathcal{P}(\pi)$

\begin{mathpar}
  \inferrule* [lab=Comm] { \textsf{match}( x_{src}, x_{trgt} ) } { x_{trgt}?(y)P \; | \; x_{src}!\langle {Q} \rangle \red P\{\quotep{Q}/y}\} }
  \and \\
  \inferrule* [lab=Par] {{P} \red {P}'} {{{P} | {Q}} \red {{P}' | {Q}}}
  \and
  \inferrule* [lab=Equiv]{{{P} \scong {P}'} \andalso {{P}' \red {Q}'} \andalso {{Q}' \scong {Q}}}{{P} \red {Q}}
\end{mathpar}

\begin{eqnarray*}
  match_{\equiv} (\quotep{P},\quotep{Q}) & := & P \equiv Q \\
  match_{\dagger}(\quotep{P},\quotep{Q}) & := & \forall R. P|Q \red^{*} R => R \red^{*} 0 \\
  match_{K}(\quotep{P},\quotep{Q}) & := & K \mbox{ for some context } K
\end{eqnarray*}

$u?(x)P | u!\langle Q \rangle \red P\{\quotep{Q}/x\}$

%We write $\wred$ for $\red^*$, and $P\red$ if $\exists Q $ such that $ P \red Q$.
We write $P\red$ if $\exists Q $ such that $ P \red Q$ and $P\not\red$, otherwise.

\section{Replication}

As mentioned before, it is known that replication (and hence
recursion) can be implemented in a higher-order process algebra
\cite{SangiorgiWalker}. As our first example of calculation with the
machinery thus far presented we give the construction explicitly in
the {\rhoc}.

\begin{eqnarray}
	D_{x} & := & \prefix{x}{y}{(\binpar{\outputp{x}{y}}{@{y}})} \nonumber\\
	\bangp_{x}{P} & := & \binpar{{x}!\langle{\binpar{D_{x}}{P}}\rangle}{D_{x}} \nonumber
\end{eqnarray}

\begin{eqnarray}
	\bangp_{x}{P} & & \nonumber\\
	=
	& {x}!\langle{(\prefix{x}{y}{(\outputp{x}{y} | @{y})) | P}}\rangle 
	      | \prefix{x}{y}{(\outputp{x}{y} | @{y})} & \nonumber\\
	\red
	& (\outputp{x}{y} | @{y})\substn{\quotep{(\prefix{x}{y}{(@{y} | \outputp{x}{y})) | P}}}{y} & \nonumber\\
	=
	& \outputp{x}{\quotep{(\prefix{x}{y}{(\outputp{x}{y} | @{y})) | P}}}
	  | {(\prefix{x}{y}{(\outputp{x}{y} | @{y})) | P}} & \nonumber\\
	\red
	& \ldots & \nonumber\\
	\red^*
	& P | P | \ldots & \nonumber
\end{eqnarray}

Of course, this encoding, as an implementation, runs away, unfolding
$\bangp{P}$ eagerly. A lazier and more implementable replication
operator, restricted to input-guarded processes, may be obtained as follows.

\begin{eqnarray}
\bangp{\prefix{u}{v}{P}} 
	:= 
	\binpar{\lift{x}{\prefix{u}{v}{(\binpar{D(x)}{P})}}}{D(x)} \nonumber
\end{eqnarray}

\begin{remark}
  Note that the lazier definition still does not deal with summation
  or mixed summation (i.e. sums over input and output). The reader is
  invited to construct definitions of replication that deal with these
  features. 

  Further, the definitions are parameterized in a name, $x$. Can you,
  gentle reader, make a definition that eliminates this parameter and
  guarantees no accidental interaction between the replication
  machinery and the process being replicated -- i.e. no accidental
  sharing of names used by the process to get its work done and the
  name(s) used by the replication to effect copying. This latter
  revision of the definition of replication is crucial to obtaining
  the expected identity $!!P \sim !P$.
\end{remark}

\begin{remark}\label{rem:paradoxical_combinator}
  The reader familiar with the lambda calculus will have noticed the
  similarity between $D$ and the paradoxical combinator.

  [Ed. note: the existence of this seems to suggest we have to be more
  restrictive on the set of processes and names we admit if we are to
  support no-cloning.]
\end{remark}

\subsubsection{Bisimulation}

The computational dynamics gives rise to another kind of equivalence,
the equivalence of computational behavior. As previously mentioned
this is typically captured \emph{via} some form of bisimulation.

% The notion we use in this paper is weak barbed bisimulation
% \cite{milner91polyadicpi}.

The notion we use in this paper is derived from weak barbed
bisimulation \cite{milner91polyadicpi}. 

\begin{definition}
An \emph{observation relation}, $\downarrow_{\mathcal N}$, over a set
of names, $\mathcal N$, is the smallest relation satisfying the rules
below.

\infrule[Out-barb]{y \in {\mathcal N}, \; x \nameeq y}
		  {\outputp{x}{v} \downarrow_{\mathcal N} x}
\infrule[Par-barb]{\mbox{$P\downarrow_{\mathcal N} x$ or $Q\downarrow_{\mathcal N} x$}}
		  {\binpar{P}{Q} \downarrow_{\mathcal N} x}

We write $P \Downarrow_{\mathcal N} x$ if there is $Q$ such that 
$P \wred Q$ and $Q \downarrow_{\mathcal N} x$.
\end{definition}

\begin{definition}
%\label{def.bbisim}
An  ${\mathcal N}$-\emph{barbed bisimulation} over a set of names, ${\mathcal N}$, is a symmetric binary relation 
${\mathcal S}_{\mathcal N}$ between agents such that $P\rel{S}_{\mathcal N}Q$ implies:
\begin{enumerate}
\item If $P \red P'$ then $Q \wred Q'$ and $P'\rel{S}_{\mathcal N} Q'$.
\item If $P\downarrow_{\mathcal N} x$, then $Q\Downarrow_{\mathcal N} x$.
\end{enumerate}
$P$ is ${\mathcal N}$-barbed bisimilar to $Q$, written
$P \wbbisim_{\mathcal N} Q$, if $P \rel{S}_{\mathcal N} Q$ for some ${\mathcal N}$-barbed bisimulation ${\mathcal S}_{\mathcal N}$.
\end{definition}

$\mathcal{R} \subseteq \pi \times \pi$

$P \mathcal{R} Q => \forall P'. P \red P' \Rightarrow \exists Q'. Q \red Q', P' \mathcal{R} Q'$

$P \vdash x \Rightarrow Q \vdash x$

\begin{mathpar}
  \inferrule*[lab=Out-barb]{x \nameeq y}{{y}!\langle{Q}\rangle \vdash x}
  \and
  \inferrule*[lab=Par-barb]{\mbox{$P\vdash x$ or $Q\vdash x$}}{\binpar{P}{Q} \vdash x}
\end{mathpar}

\subsubsection{Contexts}

One of the principle advantages of computational calculi like the
$\pi$-calculus is a well-defined notion of context,
contextual-equivalence and a correlation between
contextual-equivalence and notions of bisimulation. The notion of
context allows the decomposition of a process into (sub-)process and
its syntactic environment, its context. Thus, a context may be
thought of as a process with a ``hole'' (written $\Box$) in it. The
application of a context $M$ to a process $P$, written $M[P]$, is
tantamount to filling the hole in $M$ with $P$. In this paper we do
not need the full weight of this theory, but do make use of the notion
of context in the proof the main theorem. 

\begin{mathpar}
  \inferrule* [lab=summation] {} {{M_{M},M_{N}} \bc \Box \;|\; x.M_{A} \;|\; M_{M}+M_{N}}
  \and
  \inferrule* [lab=agent] {} {{M_{A}} \bc (\vec{x})M_{P} \;| \; \clift{P_0,\ldots,M_{P},\ldots,P_N}}
  \and \\
  \inferrule* [lab=process] {} {{M_{P}} \bc M_{N} \;| \;P|M_{P} }
\end{mathpar} 

\begin{mathpar}
  \inferrule* [lab=sychronization] {} {M_{N} \bc \Box \;|\; x?M_{F} \;|\; x!M_{C}}
  \and
  \inferrule* [lab=abstraction] {} {{M_{F}} \bc (x)M_{P} }
  \and
  \inferrule* [lab=concretion] {} {{M_{C}} \bc \langle M_{P} \rangle }
  \and \\
  \inferrule* [lab=process] {} {{M_{P}} \bc M_{N} \;| \;P|M_{P} }
\end{mathpar}

\begin{definition}[contextual application] Given a context $M$, and
  process $P$, we define the \emph{contextual application}, $M[P] :=
  M\{P/\Box\}$. That is, the contextual application of M to P is the
  substitution of $P$ for $\Box$ in $M$.
\end{definition}

$\meaningof{-} : L \to \mathcal{P}(\pi)$

\begin{mathpar}
  \inferrule* [lab=collection] {} {\meaningof{true} = \pi, \and \meaningof{~E} = \pi \setminus \meaningof{E}, \and \meaningof{E_{1} \& E_{2}} = \meaningof{E_{1}} \cap \meaningof{E_{2}}}
\end{mathpar}

\begin{mathpar}
  \inferrule* [lab=structure] {} {\meaningof{0} = \{ P \in \pi | P \equiv 0 \}, \and \\ \meaningof{E_1 | E_2} = \{ P \in \pi | P \equiv P_{1} | P_{2}, P_{1} \in \meaningof{E_{1}}, P_{2} \in \meaningof{E_2}\} }
\end{mathpar}

\begin{mathpar}
 \inferrule* [lab=behavior] {} {\meaningof{\langle a?b \rangle E} = \{ P \in \pi | P \equiv Q | u?(y)P', \\ \and \\\\ \and \\ \;\;\; u \in \meaningof{a}, \forall z.P'\{z/y\} \in \meaningof{E\{z/b\}}\}, \and \\ \meaningof{a!E} = \{ P \in \pi | P \equiv Q | x!\langle P' \rangle, x \in \meaningof{a} P' \in \meaningof{E}\} }
\end{mathpar}

\begin{mathpar}
 \inferrule* [lab=nominal] {} {\meaningof{\quotep{E}} = \{ \quotep{P} \in \quotep{\pi} | P \in \meaningof{E} \}, \and \meaningof{\quotep{P}} = \{ \quotep{Q} \in \quotep{\pi} | P \equiv Q \} \and \\ \meaningof{@\quotep{E}} = \{ P \in \pi | P \equiv @x, x \in \meaningof{E} \}}
\end{mathpar}

\begin{eqnarray*}
  \\
  \meaningof{-} : TS \to ST
\end{eqnarray*}

\begin{eqnarray*}
  \\
  L : TS \to ST
\end{eqnarray*}

\begin{eqnarray*}
  \\
  P \models E \iff P \in \meaningof{E}
\end{eqnarray*}

\begin{eqnarray*}
  P \approx_{L} Q \iff \forall E \in L. P \models E \iff Q \models E
\end{eqnarray*}

\begin{eqnarray*}
  P \approx_{K} Q
\end{eqnarray*}

\begin{eqnarray*}
  P \approx Q
\end{eqnarray*}

$\approx_{K} = \approx = \approx_{L}$

\subsubsection{Contextual duality}

Note that contexts extend the quotation operation to a family of
operations from processes to names. Given a context, $M$, we can
define a \emph{nominal context}, $\quotep{M}$ by $\quotep{M}[P] :=
\quotep{M[P]}$. To foreshadow what is to come we observe that these
operations enjoy a duality with processes very much like the duality
between vectors and maps from vectors to scalars.

Further, because the calculus is essentially higher-order, we have a
correspondence between contexts and processes. More specifically,
given a name $x$ and a context $M$ we can construct $M^{*}_{x}$ such
that 

\begin{mathpar}
  M^{*}_{x} | \lift{x}{P} \red M[P]
\end{mathpar}

namely,

\begin{mathpar}
  M^{*}_{x} := x?(u).M[\dropn{u}]
\end{mathpar}

The dependence of $M^{*}_{x}$ on a name makes it an abstraction, 

\begin{mathpar}
  M^{*} := (x)x?(u).M[\dropn{u}]
\end{mathpar}

\subsection{Additional notation}

It will sometimes be convenient to denote the process a name
quotes. We already have the notation $x = \quotep{P}$, but it will be
convenient to introduce an alternate notation, $\procn{x}$, when we
want to emphasize the connection to the use of the name. Note that, by
virtue of name equivalence, $\quotep{\procn{x}} \nameeq x$; so, the
notation is consistent with previous definitions.

Further, because names have structure it is possible to effect
substitutions on the basis of that structure. This means we need to
upgrade our notation for substitutions, which we accomplish by
adapting comprehension notation. Thus,

\begin{mathpar}
  P\{ y / x : x \in S \}
\end{mathpar}

is interpreted to mean the process derived from P by replacing (in a
capture-avoiding manner) each occurrence of $x$ in $S$ by $y$. For example,

\begin{mathpar}
  P\{ \quotep{\procn{x}|\procn{x}} / x : x \in \freenames{P} \}
\end{mathpar}

will replace each (occurrence) of a free name $x$ in $P$ by
$\quotep{\procn{x}|\procn{x}}$.

Also, we will avail ourselves of the notation $x^{L}$ and $x^{R}$ to
denote injections of a name into disjoint copies of the name
space. There are numerous ways to accomplish this. One example can be
found in \cite{MeredithR05}. This notation overloads to vectors of
names: $\vec{x}^{\pi} := (x_{i}^{\pi} \; : \; 0 \leq i < |\vec{x}| )$ where $\pi \in \{L,R\}$.

We also use $P^{\Box} := P|\Box$.

In \cite{MeredithR05} an interpretation of the new operator is
given. It turns out that there are several possible interpretations
all enjoying the requisite algebraic properties of the operator (see
\cite{milner91polyadicpi}). We will therefore make liberal use of
$(\nu\; \vec{x})P$.

% subsection the_syntax_and_semantics_of_the_notation_system (end)   

\input{qm2pi.qmops} 

\input{qm2pi.sterngerlach} 

\input{qm2pi.metric} 

% section concurrent_process_calculi (end)

%\input{qm2pi.proofsketch}

% section proof sketch (end)

%\input{qm2pi.slviaknots} 

% section spatial logic via knots (end)

\input{qm2pi.conclusion}

% section conclusion (end)

%\input{qm2pi.dtcodes} 

% section wiring algorithm (end)

\input{qm2pi.ack} 

% section acknowledgments (end)

\newpage


\bibliographystyle{plain}   
\bibliography{../../biblios/main.bib}

\input{qm2pi.rhodetails}

\end{document}



\end{document}

 

%\documentclass[12pt]{llncs}
%\documentclass{jktr}

\usepackage[pdftex]{hyperref}                   
\usepackage {listings}
\usepackage {mathpartir}
\usepackage{bcprules}
%\usepackage{listings}
                       
\usepackage{graphicx} 
%\usepackage[margins=2.5cm,nohead,nofoot]{geometry}
%\usepackage{geometry}
\usepackage{amsfonts}
\usepackage{amstext}
\usepackage{latexsym}
\usepackage{amssymb}
\usepackage{color}


%\include{myPreamble}
\documentclass[12pt]{llncs}
%\documentclass{jktr}

\usepackage[pdftex]{hyperref}                   
\usepackage {listings}
\usepackage {mathpartir}
\usepackage{bcprules}
%\usepackage{listings}
                       
\usepackage{graphicx} 
%\usepackage[margins=2.5cm,nohead,nofoot]{geometry}
%\usepackage{geometry}
\usepackage{amsfonts}
\usepackage{amstext}
\usepackage{latexsym}
\usepackage{amssymb}
\usepackage{color}


%\include{myPreamble}
\include{qm2pi.local} 

%\ifpdf
%\usepackage[pdftex]{graphicx}
%\else
%\usepackage{graphicx}
%\fi

 % \ifpdf
%  \usepackage{pdfsync}
%  \if


%\title{Brief Article}
%\author{David F. Snyder}
%\author{L.G. Meredith}

%\address{Dept. of Math., Texas State University--San Marcos, San Marcos, TX 78666}
       
\pagestyle{empty}


\begin{document}

\lstset{language=[Objective]Caml,frame=shadowbox}

\input{qm2pi.front}

% section front matter (end)

\input{qm2pi.intro} 
 
% section introduction (end)

% \input{qm2pi.knotations} 

% section notation (end)

\input{qm2pi.process.calculi} 

% section concurrent_process_calculi_and_spatial_logics_ (end)
    
%\input{qm2pi.knots2pi} 

%\input{qm2pi.trefoil} 

%\input{qm2pi.mainthm} 

% subsection basic_interpretation (end)

%\input{qm2pi.rho.presentation} 
\subsection{The syntax and semantics of the notation system}\label{sub:the_syntax_and_semantics_of_the_notation_system} % (fold)

We now summarize a technical presentation of the calculus that
embodies our theory of dynamics. The typical presentation of such a
calculus follows the style of giving generators and relations on
them. The grammar, below, describing term constructors, freely
generates the set of processes, $\Proc$. This set is then quotiented
by a relation known as structural congruence and it is over this set
that the notion of dynamics is expressed. This presentation is
essentially that of \cite{MeredithR05} with the addition of
polyadicity and summation. For readability we have relegated some of
the technical subtleties to an appendix.

\subsubsection{Process grammar}\label{subsub:process_grammar}

\begin{mathpar}
  \inferrule* [lab=synchronization] {} {{M} \bc \pzero \;|\; x?F \;|\; x!C }
  \and
  \inferrule* [lab=abstraction] {} {{F} \bc (x)P}
  \and
  \inferrule* [lab=concretion] {} {{C} \bc \langle Q \rangle}
  \and
  \inferrule* [lab=process] {} {{P,Q} \bc M \;| \;P|Q \;|\; @{x}}
  \and
  \inferrule* [lab=name] {} {{x} \bc \quotep{P}}
\end{mathpar} 

Note that $\vec{x}$ (resp. $\vec{P}$) denotes a vector of names
(resp. processes) of length $|\vec{x}|$ (resp. $|\vec{P}|$). We adopt
the following useful abbreviations.

\begin{mathpar}
   x?(\vec{y}).P := x.(\vec{y})P \and  x\clift{\vec{P}} := x.\clift{\vec{P}}
   \and x!(y) := \lift{x}{\dropn{y}}
   \and \Pi_{i=0}^{n-1}P_i := P_0 | \ldots | P_{n-1}
\end{mathpar}

\subsubsection{Structural congruence}

\paragraph{Free and bound names and alpha-equivalence.} At the
core of structural equivalence is alpha-equivalence which identifies
process that are the same up to a change of variable. Formally, we
recognize the distinction between free and bound names. The free names
of a process, $\freenames{P}$, may be calculated recursively as
follows:

\begin{mathpar}
\freenames{\pzero} := \emptyset
  \and \\
  \freenames{x?(y).P} := \{ x \} \cup (\freenames{P} \setminus \{ y \})
  \and 
  \freenames{x!\langle P \rangle} := \{ x \} \cup \{ P \} 
  \and \\
  \freenames{P|Q} := \freenames{P} \cup \freenames{Q}
  \and \\
  \freenames{@{x}} := \{ x \}
\end{mathpar}

$\pi$
$\quotep{\pi}$

$\freenames{-} : \pi \to \mathcal{P}(\quotep{\pi})$

\begin{eqnarray*}
  \freenames{\pzero} & := & \emptyset \\
  \freenames{x?(y).P} & := & \{ x \} \cup (\freenames{P} \setminus \{ y \}) \\
  \freenames{x!\langle P \rangle} & := & \{ x \} \cup \{ P \} \\
  \freenames{P|Q} & := & \freenames{P} \cup \freenames{Q} \\
  \freenames{\dropn{x}} & := & \{ x \}
\end{eqnarray*}

The bound names of a process, $\boundnames{P}$, are those names occurring in $P$
that are not free. For example, in $x?(y).0$, the name $x$ is free, while $y$ is bound.

\begin{mathpar}
  \inferrule* [lab=monoidal-laws] {} { P|Q \equiv Q|P \and P|0 \equiv P \and P|(Q|R) \equiv (P|Q)|R }
\end{mathpar}

\begin{mathpar}
  \inferrule* [lab=alpha-equivalence] {} { (x)P \equiv (y)P\{y/x\} \and y \not\in \freenames{P} }
\end{mathpar}

\begin{definition}
Then two processes, $P,Q$, are alpha-equivalent if $P = Q\{\vec{y}/\vec{x}\}$ for
some $\vec{x} \in \boundnames{Q},\vec{y} \in \boundnames{P}$, where $Q\{\vec{y}/\vec{x}\}$
denotes the capture-avoiding substitution of $\vec{y}$ for $\vec{x}$ in $Q$.
\end{definition}

\begin{definition}
  The {\em structural congruence} \cite{SangiorgiWalker} , $\equiv$,
  between processes is the least congruence containing
  alpha-equivalence, satisfying the abelian monoid laws
  (associativity, commutativity and $\pzero$ as identity) for parallel
  composition $|$ and for summation $+$.
\end{definition}

\subsection{Name equivalence}

We take name equivalence, written $\nameeq$, to be the smallest
equivalence relation generated by the following rules.

\begin{mathpar}
\inferrule*[lab=Quote-drop]
{ }
{ \quotep{@{x}} \nameeq x }

\inferrule*[lab=Struct-equiv]
{ P \scong Q }
{ \quotep{P} \nameeq \quotep{Q} }
\end{mathpar}

The astute reader will have noticed that the mutual recursion of names
and processes imposes a mutual recursion on alpha-equivalence and
structural equivalence via name-equivalence. Fortunately, all of this
works out pleasantly and we may calculate in the natural way, free of
concern. The reader interested in the details is referred to the
appendix \ref{appendix:rho_details}.

\subsection{Substitution}

We use $\Proc$ for the set of processes, $\QProc$ for the set of
names, and $\id{\{}\vec{y} / \vec{x} \id{\}}$ to denote partial maps,
$s : \QProc \rightarrow \QProc$. A map, $s$ lifts, uniquely, to a map
on process terms, $\widehat{s} : \Proc \rightarrow \Proc$ by the
following equations.

\begin{mathpar}
  (0) \psubstp{Q}{P} := 0 \\
  (R \juxtap S) \psubstp{Q}{P}
  :=    
  (R)\psubstp{Q}{P} \juxtap (S) \psubstp{Q}{P} \\
  (x?(y).R) \psubstp{Q}{P}    
  :=    
  (x)\substp{Q}{P} (z)\concat( (R \psubstn{z}{y}) \psubstp{Q}{P} ) \\
  (\lift{x}{R}) \psubstp{Q}{P}  
  :=
  \lift{(x)\substp{Q}{P}}{ R \psubstp{Q}{P} } \\
%   (\dropn{x})  \psubstp{Q}{P}       
%   := 
%   \left\{ 
%     \begin{array}{ccc} 
%       \dropn{\quotep{Q}} & & x \nameeq \quotep{P} \\
%       \dropn{x} & & otherwise \\
%     \end{array}
%   \right. 
  (\dropn{x})  \psubstp{Q}{P}       
  := 
  \left\{ 
    \begin{array}{ccc} 
      Q & & x \nameeq \quotep{P} \\
      \dropn{x} & & otherwise \\
    \end{array}
  \right.
\end{mathpar}
 

where

\begin{eqnarray}
  (x)\id{\{} \lpquote Q \rpquote / \lpquote P \rpquote \id{\}}            = 
  \left\{ 
    \begin{array}{ccc}
      \lpquote Q \rpquote & & x \nameeq \lpquote P \rpquote \\
      x & & otherwise \\
    \end{array}
  \right. \nonumber
\end{eqnarray}

and $z$ is chosen distinct from $\quotep{P}$, $\quotep{Q}$, the free
names in $Q$, and all the names in $R$. Our $\alpha$-equivalence will
be built in the standard way from this substitution.

\begin{remark}\label{rem:no_self_referential_names}
  One consequence of these definitions is that $\forall P. \quotep{P}
  \not\in \freenames{P}$.
\end{remark}

\subsection{ Dynamic quote: an example }

Anticipating something of what's to come, consider applying the
substitution, $\widehat{\id{\{}u / z \id{\}}}$, to the following pair
of processes, $\lift{w}{y!(z)}$ and $w[ \lpquote y!(z) \rpquote ]$.

\begin{eqnarray}
	\lift{w}{y!(z)}\widehat{\id{\{}u / z \id{\}}}
		& = &
		\lift{w}{y!(u)} \nonumber\\
	w[ \lpquote y!(z) \rpquote ] \widehat{ \id{\{}u / z \id{\}} }
		& = &
		w[ \lpquote y!(z) \rpquote ] \nonumber
\end{eqnarray}

Because the body of the process between quotes is impervious to
substitution, we get radically different answers. In fact, by
examining the first process in an input context,
e.g. $x?(z).\lift{w}{y!(z)}$, we see that the process under the lift
operator may be shaped by prefixed inputs binding a name inside it. In
this sense, the lift operator will be seen as a way to dynamically
construct processes before reifying them as names.

Finally equipped with these standard features we can present the
dynamics of the calculus.

\subsubsection{Operational semantics} 

Finally, we introduce the computational dynamics. What marks these
algebras as distinct from other more traditionally studied algebraic
structures, e.g. vector spaces or polynomial rings, is the manner in
which dynamics is captured. In traditional structures, dynamics is typically
expressed through morphisms between such structures, as in linear maps
between vector spaces or morphisms between rings. In algebras
associated with the semantics of computation, the dynamics is
expressed as part of the algebraic structure itself, through a
reduction reduction relation typically denoted by $\red$. Below, we
give a recursive presentation of this relation for the calculus used
in the encoding.

$\red \subseteq \pi \times \pi$
$\red : \pi \to \mathcal{P}(\pi)$

\begin{mathpar}
  \inferrule* [lab=Comm] { \textsf{match}( x_{src}, x_{trgt} ) } { x_{trgt}?(y)P \; | \; x_{src}!\langle {Q} \rangle \red P\{\quotep{Q}/y}\} }
  \and \\
  \inferrule* [lab=Par] {{P} \red {P}'} {{{P} | {Q}} \red {{P}' | {Q}}}
  \and
  \inferrule* [lab=Equiv]{{{P} \scong {P}'} \andalso {{P}' \red {Q}'} \andalso {{Q}' \scong {Q}}}{{P} \red {Q}}
\end{mathpar}

\begin{eqnarray*}
  match_{\equiv} (\quotep{P},\quotep{Q}) & := & P \equiv Q \\
  match_{\dagger}(\quotep{P},\quotep{Q}) & := & \forall R. P|Q \red^{*} R => R \red^{*} 0 \\
  match_{K}(\quotep{P},\quotep{Q}) & := & K \mbox{ for some context } K
\end{eqnarray*}

$u?(x)P | u!\langle Q \rangle \red P\{\quotep{Q}/x\}$

%We write $\wred$ for $\red^*$, and $P\red$ if $\exists Q $ such that $ P \red Q$.
We write $P\red$ if $\exists Q $ such that $ P \red Q$ and $P\not\red$, otherwise.

\section{Replication}

As mentioned before, it is known that replication (and hence
recursion) can be implemented in a higher-order process algebra
\cite{SangiorgiWalker}. As our first example of calculation with the
machinery thus far presented we give the construction explicitly in
the {\rhoc}.

\begin{eqnarray}
	D_{x} & := & \prefix{x}{y}{(\binpar{\outputp{x}{y}}{@{y}})} \nonumber\\
	\bangp_{x}{P} & := & \binpar{{x}!\langle{\binpar{D_{x}}{P}}\rangle}{D_{x}} \nonumber
\end{eqnarray}

\begin{eqnarray}
	\bangp_{x}{P} & & \nonumber\\
	=
	& {x}!\langle{(\prefix{x}{y}{(\outputp{x}{y} | @{y})) | P}}\rangle 
	      | \prefix{x}{y}{(\outputp{x}{y} | @{y})} & \nonumber\\
	\red
	& (\outputp{x}{y} | @{y})\substn{\quotep{(\prefix{x}{y}{(@{y} | \outputp{x}{y})) | P}}}{y} & \nonumber\\
	=
	& \outputp{x}{\quotep{(\prefix{x}{y}{(\outputp{x}{y} | @{y})) | P}}}
	  | {(\prefix{x}{y}{(\outputp{x}{y} | @{y})) | P}} & \nonumber\\
	\red
	& \ldots & \nonumber\\
	\red^*
	& P | P | \ldots & \nonumber
\end{eqnarray}

Of course, this encoding, as an implementation, runs away, unfolding
$\bangp{P}$ eagerly. A lazier and more implementable replication
operator, restricted to input-guarded processes, may be obtained as follows.

\begin{eqnarray}
\bangp{\prefix{u}{v}{P}} 
	:= 
	\binpar{\lift{x}{\prefix{u}{v}{(\binpar{D(x)}{P})}}}{D(x)} \nonumber
\end{eqnarray}

\begin{remark}
  Note that the lazier definition still does not deal with summation
  or mixed summation (i.e. sums over input and output). The reader is
  invited to construct definitions of replication that deal with these
  features. 

  Further, the definitions are parameterized in a name, $x$. Can you,
  gentle reader, make a definition that eliminates this parameter and
  guarantees no accidental interaction between the replication
  machinery and the process being replicated -- i.e. no accidental
  sharing of names used by the process to get its work done and the
  name(s) used by the replication to effect copying. This latter
  revision of the definition of replication is crucial to obtaining
  the expected identity $!!P \sim !P$.
\end{remark}

\begin{remark}\label{rem:paradoxical_combinator}
  The reader familiar with the lambda calculus will have noticed the
  similarity between $D$ and the paradoxical combinator.

  [Ed. note: the existence of this seems to suggest we have to be more
  restrictive on the set of processes and names we admit if we are to
  support no-cloning.]
\end{remark}

\subsubsection{Bisimulation}

The computational dynamics gives rise to another kind of equivalence,
the equivalence of computational behavior. As previously mentioned
this is typically captured \emph{via} some form of bisimulation.

% The notion we use in this paper is weak barbed bisimulation
% \cite{milner91polyadicpi}.

The notion we use in this paper is derived from weak barbed
bisimulation \cite{milner91polyadicpi}. 

\begin{definition}
An \emph{observation relation}, $\downarrow_{\mathcal N}$, over a set
of names, $\mathcal N$, is the smallest relation satisfying the rules
below.

\infrule[Out-barb]{y \in {\mathcal N}, \; x \nameeq y}
		  {\outputp{x}{v} \downarrow_{\mathcal N} x}
\infrule[Par-barb]{\mbox{$P\downarrow_{\mathcal N} x$ or $Q\downarrow_{\mathcal N} x$}}
		  {\binpar{P}{Q} \downarrow_{\mathcal N} x}

We write $P \Downarrow_{\mathcal N} x$ if there is $Q$ such that 
$P \wred Q$ and $Q \downarrow_{\mathcal N} x$.
\end{definition}

\begin{definition}
%\label{def.bbisim}
An  ${\mathcal N}$-\emph{barbed bisimulation} over a set of names, ${\mathcal N}$, is a symmetric binary relation 
${\mathcal S}_{\mathcal N}$ between agents such that $P\rel{S}_{\mathcal N}Q$ implies:
\begin{enumerate}
\item If $P \red P'$ then $Q \wred Q'$ and $P'\rel{S}_{\mathcal N} Q'$.
\item If $P\downarrow_{\mathcal N} x$, then $Q\Downarrow_{\mathcal N} x$.
\end{enumerate}
$P$ is ${\mathcal N}$-barbed bisimilar to $Q$, written
$P \wbbisim_{\mathcal N} Q$, if $P \rel{S}_{\mathcal N} Q$ for some ${\mathcal N}$-barbed bisimulation ${\mathcal S}_{\mathcal N}$.
\end{definition}

$\mathcal{R} \subseteq \pi \times \pi$

$P \mathcal{R} Q => \forall P'. P \red P' \Rightarrow \exists Q'. Q \red Q', P' \mathcal{R} Q'$

$P \vdash x \Rightarrow Q \vdash x$

\begin{mathpar}
  \inferrule*[lab=Out-barb]{x \nameeq y}{{y}!\langle{Q}\rangle \vdash x}
  \and
  \inferrule*[lab=Par-barb]{\mbox{$P\vdash x$ or $Q\vdash x$}}{\binpar{P}{Q} \vdash x}
\end{mathpar}

\subsubsection{Contexts}

One of the principle advantages of computational calculi like the
$\pi$-calculus is a well-defined notion of context,
contextual-equivalence and a correlation between
contextual-equivalence and notions of bisimulation. The notion of
context allows the decomposition of a process into (sub-)process and
its syntactic environment, its context. Thus, a context may be
thought of as a process with a ``hole'' (written $\Box$) in it. The
application of a context $M$ to a process $P$, written $M[P]$, is
tantamount to filling the hole in $M$ with $P$. In this paper we do
not need the full weight of this theory, but do make use of the notion
of context in the proof the main theorem. 

\begin{mathpar}
  \inferrule* [lab=summation] {} {{M_{M},M_{N}} \bc \Box \;|\; x.M_{A} \;|\; M_{M}+M_{N}}
  \and
  \inferrule* [lab=agent] {} {{M_{A}} \bc (\vec{x})M_{P} \;| \; \clift{P_0,\ldots,M_{P},\ldots,P_N}}
  \and \\
  \inferrule* [lab=process] {} {{M_{P}} \bc M_{N} \;| \;P|M_{P} }
\end{mathpar} 

\begin{mathpar}
  \inferrule* [lab=sychronization] {} {M_{N} \bc \Box \;|\; x?M_{F} \;|\; x!M_{C}}
  \and
  \inferrule* [lab=abstraction] {} {{M_{F}} \bc (x)M_{P} }
  \and
  \inferrule* [lab=concretion] {} {{M_{C}} \bc \langle M_{P} \rangle }
  \and \\
  \inferrule* [lab=process] {} {{M_{P}} \bc M_{N} \;| \;P|M_{P} }
\end{mathpar}

\begin{definition}[contextual application] Given a context $M$, and
  process $P$, we define the \emph{contextual application}, $M[P] :=
  M\{P/\Box\}$. That is, the contextual application of M to P is the
  substitution of $P$ for $\Box$ in $M$.
\end{definition}

$\meaningof{-} : L \to \mathcal{P}(\pi)$

\begin{mathpar}
  \inferrule* [lab=collection] {} {\meaningof{true} = \pi, \and \meaningof{~E} = \pi \setminus \meaningof{E}, \and \meaningof{E_{1} \& E_{2}} = \meaningof{E_{1}} \cap \meaningof{E_{2}}}
\end{mathpar}

\begin{mathpar}
  \inferrule* [lab=structure] {} {\meaningof{0} = \{ P \in \pi | P \equiv 0 \}, \and \\ \meaningof{E_1 | E_2} = \{ P \in \pi | P \equiv P_{1} | P_{2}, P_{1} \in \meaningof{E_{1}}, P_{2} \in \meaningof{E_2}\} }
\end{mathpar}

\begin{mathpar}
 \inferrule* [lab=behavior] {} {\meaningof{\langle a?b \rangle E} = \{ P \in \pi | P \equiv Q | u?(y)P', \\ \and \\\\ \and \\ \;\;\; u \in \meaningof{a}, \forall z.P'\{z/y\} \in \meaningof{E\{z/b\}}\}, \and \\ \meaningof{a!E} = \{ P \in \pi | P \equiv Q | x!\langle P' \rangle, x \in \meaningof{a} P' \in \meaningof{E}\} }
\end{mathpar}

\begin{mathpar}
 \inferrule* [lab=nominal] {} {\meaningof{\quotep{E}} = \{ \quotep{P} \in \quotep{\pi} | P \in \meaningof{E} \}, \and \meaningof{\quotep{P}} = \{ \quotep{Q} \in \quotep{\pi} | P \equiv Q \} \and \\ \meaningof{@\quotep{E}} = \{ P \in \pi | P \equiv @x, x \in \meaningof{E} \}}
\end{mathpar}

\begin{eqnarray*}
  \\
  \meaningof{-} : TS \to ST
\end{eqnarray*}

\begin{eqnarray*}
  \\
  L : TS \to ST
\end{eqnarray*}

\begin{eqnarray*}
  \\
  P \models E \iff P \in \meaningof{E}
\end{eqnarray*}

\begin{eqnarray*}
  P \approx_{L} Q \iff \forall E \in L. P \models E \iff Q \models E
\end{eqnarray*}

\begin{eqnarray*}
  P \approx_{K} Q
\end{eqnarray*}

\begin{eqnarray*}
  P \approx Q
\end{eqnarray*}

$\approx_{K} = \approx = \approx_{L}$

\subsubsection{Contextual duality}

Note that contexts extend the quotation operation to a family of
operations from processes to names. Given a context, $M$, we can
define a \emph{nominal context}, $\quotep{M}$ by $\quotep{M}[P] :=
\quotep{M[P]}$. To foreshadow what is to come we observe that these
operations enjoy a duality with processes very much like the duality
between vectors and maps from vectors to scalars.

Further, because the calculus is essentially higher-order, we have a
correspondence between contexts and processes. More specifically,
given a name $x$ and a context $M$ we can construct $M^{*}_{x}$ such
that 

\begin{mathpar}
  M^{*}_{x} | \lift{x}{P} \red M[P]
\end{mathpar}

namely,

\begin{mathpar}
  M^{*}_{x} := x?(u).M[\dropn{u}]
\end{mathpar}

The dependence of $M^{*}_{x}$ on a name makes it an abstraction, 

\begin{mathpar}
  M^{*} := (x)x?(u).M[\dropn{u}]
\end{mathpar}

\subsection{Additional notation}

It will sometimes be convenient to denote the process a name
quotes. We already have the notation $x = \quotep{P}$, but it will be
convenient to introduce an alternate notation, $\procn{x}$, when we
want to emphasize the connection to the use of the name. Note that, by
virtue of name equivalence, $\quotep{\procn{x}} \nameeq x$; so, the
notation is consistent with previous definitions.

Further, because names have structure it is possible to effect
substitutions on the basis of that structure. This means we need to
upgrade our notation for substitutions, which we accomplish by
adapting comprehension notation. Thus,

\begin{mathpar}
  P\{ y / x : x \in S \}
\end{mathpar}

is interpreted to mean the process derived from P by replacing (in a
capture-avoiding manner) each occurrence of $x$ in $S$ by $y$. For example,

\begin{mathpar}
  P\{ \quotep{\procn{x}|\procn{x}} / x : x \in \freenames{P} \}
\end{mathpar}

will replace each (occurrence) of a free name $x$ in $P$ by
$\quotep{\procn{x}|\procn{x}}$.

Also, we will avail ourselves of the notation $x^{L}$ and $x^{R}$ to
denote injections of a name into disjoint copies of the name
space. There are numerous ways to accomplish this. One example can be
found in \cite{MeredithR05}. This notation overloads to vectors of
names: $\vec{x}^{\pi} := (x_{i}^{\pi} \; : \; 0 \leq i < |\vec{x}| )$ where $\pi \in \{L,R\}$.

We also use $P^{\Box} := P|\Box$.

In \cite{MeredithR05} an interpretation of the new operator is
given. It turns out that there are several possible interpretations
all enjoying the requisite algebraic properties of the operator (see
\cite{milner91polyadicpi}). We will therefore make liberal use of
$(\nu\; \vec{x})P$.

% subsection the_syntax_and_semantics_of_the_notation_system (end)   

\input{qm2pi.qmops} 

\input{qm2pi.sterngerlach} 

\input{qm2pi.metric} 

% section concurrent_process_calculi (end)

%\input{qm2pi.proofsketch}

% section proof sketch (end)

%\input{qm2pi.slviaknots} 

% section spatial logic via knots (end)

\input{qm2pi.conclusion}

% section conclusion (end)

%\input{qm2pi.dtcodes} 

% section wiring algorithm (end)

\input{qm2pi.ack} 

% section acknowledgments (end)

\newpage


\bibliographystyle{plain}   
\bibliography{../../biblios/main.bib}

\input{qm2pi.rhodetails}

\end{document}

 

%\ifpdf
%\usepackage[pdftex]{graphicx}
%\else
%\usepackage{graphicx}
%\fi

 % \ifpdf
%  \usepackage{pdfsync}
%  \if


%\title{Brief Article}
%\author{David F. Snyder}
%\author{L.G. Meredith}

%\address{Dept. of Math., Texas State University--San Marcos, San Marcos, TX 78666}
       
\pagestyle{empty}


\begin{document}

\lstset{language=[Objective]Caml,frame=shadowbox}

\documentclass[12pt]{llncs}
%\documentclass{jktr}

\usepackage[pdftex]{hyperref}                   
\usepackage {listings}
\usepackage {mathpartir}
\usepackage{bcprules}
%\usepackage{listings}
                       
\usepackage{graphicx} 
%\usepackage[margins=2.5cm,nohead,nofoot]{geometry}
%\usepackage{geometry}
\usepackage{amsfonts}
\usepackage{amstext}
\usepackage{latexsym}
\usepackage{amssymb}
\usepackage{color}


%\include{myPreamble}
\include{qm2pi.local} 

%\ifpdf
%\usepackage[pdftex]{graphicx}
%\else
%\usepackage{graphicx}
%\fi

 % \ifpdf
%  \usepackage{pdfsync}
%  \if


%\title{Brief Article}
%\author{David F. Snyder}
%\author{L.G. Meredith}

%\address{Dept. of Math., Texas State University--San Marcos, San Marcos, TX 78666}
       
\pagestyle{empty}


\begin{document}

\lstset{language=[Objective]Caml,frame=shadowbox}

\input{qm2pi.front}

% section front matter (end)

\input{qm2pi.intro} 
 
% section introduction (end)

% \input{qm2pi.knotations} 

% section notation (end)

\input{qm2pi.process.calculi} 

% section concurrent_process_calculi_and_spatial_logics_ (end)
    
%\input{qm2pi.knots2pi} 

%\input{qm2pi.trefoil} 

%\input{qm2pi.mainthm} 

% subsection basic_interpretation (end)

%\input{qm2pi.rho.presentation} 
\subsection{The syntax and semantics of the notation system}\label{sub:the_syntax_and_semantics_of_the_notation_system} % (fold)

We now summarize a technical presentation of the calculus that
embodies our theory of dynamics. The typical presentation of such a
calculus follows the style of giving generators and relations on
them. The grammar, below, describing term constructors, freely
generates the set of processes, $\Proc$. This set is then quotiented
by a relation known as structural congruence and it is over this set
that the notion of dynamics is expressed. This presentation is
essentially that of \cite{MeredithR05} with the addition of
polyadicity and summation. For readability we have relegated some of
the technical subtleties to an appendix.

\subsubsection{Process grammar}\label{subsub:process_grammar}

\begin{mathpar}
  \inferrule* [lab=synchronization] {} {{M} \bc \pzero \;|\; x?F \;|\; x!C }
  \and
  \inferrule* [lab=abstraction] {} {{F} \bc (x)P}
  \and
  \inferrule* [lab=concretion] {} {{C} \bc \langle Q \rangle}
  \and
  \inferrule* [lab=process] {} {{P,Q} \bc M \;| \;P|Q \;|\; @{x}}
  \and
  \inferrule* [lab=name] {} {{x} \bc \quotep{P}}
\end{mathpar} 

Note that $\vec{x}$ (resp. $\vec{P}$) denotes a vector of names
(resp. processes) of length $|\vec{x}|$ (resp. $|\vec{P}|$). We adopt
the following useful abbreviations.

\begin{mathpar}
   x?(\vec{y}).P := x.(\vec{y})P \and  x\clift{\vec{P}} := x.\clift{\vec{P}}
   \and x!(y) := \lift{x}{\dropn{y}}
   \and \Pi_{i=0}^{n-1}P_i := P_0 | \ldots | P_{n-1}
\end{mathpar}

\subsubsection{Structural congruence}

\paragraph{Free and bound names and alpha-equivalence.} At the
core of structural equivalence is alpha-equivalence which identifies
process that are the same up to a change of variable. Formally, we
recognize the distinction between free and bound names. The free names
of a process, $\freenames{P}$, may be calculated recursively as
follows:

\begin{mathpar}
\freenames{\pzero} := \emptyset
  \and \\
  \freenames{x?(y).P} := \{ x \} \cup (\freenames{P} \setminus \{ y \})
  \and 
  \freenames{x!\langle P \rangle} := \{ x \} \cup \{ P \} 
  \and \\
  \freenames{P|Q} := \freenames{P} \cup \freenames{Q}
  \and \\
  \freenames{@{x}} := \{ x \}
\end{mathpar}

$\pi$
$\quotep{\pi}$

$\freenames{-} : \pi \to \mathcal{P}(\quotep{\pi})$

\begin{eqnarray*}
  \freenames{\pzero} & := & \emptyset \\
  \freenames{x?(y).P} & := & \{ x \} \cup (\freenames{P} \setminus \{ y \}) \\
  \freenames{x!\langle P \rangle} & := & \{ x \} \cup \{ P \} \\
  \freenames{P|Q} & := & \freenames{P} \cup \freenames{Q} \\
  \freenames{\dropn{x}} & := & \{ x \}
\end{eqnarray*}

The bound names of a process, $\boundnames{P}$, are those names occurring in $P$
that are not free. For example, in $x?(y).0$, the name $x$ is free, while $y$ is bound.

\begin{mathpar}
  \inferrule* [lab=monoidal-laws] {} { P|Q \equiv Q|P \and P|0 \equiv P \and P|(Q|R) \equiv (P|Q)|R }
\end{mathpar}

\begin{mathpar}
  \inferrule* [lab=alpha-equivalence] {} { (x)P \equiv (y)P\{y/x\} \and y \not\in \freenames{P} }
\end{mathpar}

\begin{definition}
Then two processes, $P,Q$, are alpha-equivalent if $P = Q\{\vec{y}/\vec{x}\}$ for
some $\vec{x} \in \boundnames{Q},\vec{y} \in \boundnames{P}$, where $Q\{\vec{y}/\vec{x}\}$
denotes the capture-avoiding substitution of $\vec{y}$ for $\vec{x}$ in $Q$.
\end{definition}

\begin{definition}
  The {\em structural congruence} \cite{SangiorgiWalker} , $\equiv$,
  between processes is the least congruence containing
  alpha-equivalence, satisfying the abelian monoid laws
  (associativity, commutativity and $\pzero$ as identity) for parallel
  composition $|$ and for summation $+$.
\end{definition}

\subsection{Name equivalence}

We take name equivalence, written $\nameeq$, to be the smallest
equivalence relation generated by the following rules.

\begin{mathpar}
\inferrule*[lab=Quote-drop]
{ }
{ \quotep{@{x}} \nameeq x }

\inferrule*[lab=Struct-equiv]
{ P \scong Q }
{ \quotep{P} \nameeq \quotep{Q} }
\end{mathpar}

The astute reader will have noticed that the mutual recursion of names
and processes imposes a mutual recursion on alpha-equivalence and
structural equivalence via name-equivalence. Fortunately, all of this
works out pleasantly and we may calculate in the natural way, free of
concern. The reader interested in the details is referred to the
appendix \ref{appendix:rho_details}.

\subsection{Substitution}

We use $\Proc$ for the set of processes, $\QProc$ for the set of
names, and $\id{\{}\vec{y} / \vec{x} \id{\}}$ to denote partial maps,
$s : \QProc \rightarrow \QProc$. A map, $s$ lifts, uniquely, to a map
on process terms, $\widehat{s} : \Proc \rightarrow \Proc$ by the
following equations.

\begin{mathpar}
  (0) \psubstp{Q}{P} := 0 \\
  (R \juxtap S) \psubstp{Q}{P}
  :=    
  (R)\psubstp{Q}{P} \juxtap (S) \psubstp{Q}{P} \\
  (x?(y).R) \psubstp{Q}{P}    
  :=    
  (x)\substp{Q}{P} (z)\concat( (R \psubstn{z}{y}) \psubstp{Q}{P} ) \\
  (\lift{x}{R}) \psubstp{Q}{P}  
  :=
  \lift{(x)\substp{Q}{P}}{ R \psubstp{Q}{P} } \\
%   (\dropn{x})  \psubstp{Q}{P}       
%   := 
%   \left\{ 
%     \begin{array}{ccc} 
%       \dropn{\quotep{Q}} & & x \nameeq \quotep{P} \\
%       \dropn{x} & & otherwise \\
%     \end{array}
%   \right. 
  (\dropn{x})  \psubstp{Q}{P}       
  := 
  \left\{ 
    \begin{array}{ccc} 
      Q & & x \nameeq \quotep{P} \\
      \dropn{x} & & otherwise \\
    \end{array}
  \right.
\end{mathpar}
 

where

\begin{eqnarray}
  (x)\id{\{} \lpquote Q \rpquote / \lpquote P \rpquote \id{\}}            = 
  \left\{ 
    \begin{array}{ccc}
      \lpquote Q \rpquote & & x \nameeq \lpquote P \rpquote \\
      x & & otherwise \\
    \end{array}
  \right. \nonumber
\end{eqnarray}

and $z$ is chosen distinct from $\quotep{P}$, $\quotep{Q}$, the free
names in $Q$, and all the names in $R$. Our $\alpha$-equivalence will
be built in the standard way from this substitution.

\begin{remark}\label{rem:no_self_referential_names}
  One consequence of these definitions is that $\forall P. \quotep{P}
  \not\in \freenames{P}$.
\end{remark}

\subsection{ Dynamic quote: an example }

Anticipating something of what's to come, consider applying the
substitution, $\widehat{\id{\{}u / z \id{\}}}$, to the following pair
of processes, $\lift{w}{y!(z)}$ and $w[ \lpquote y!(z) \rpquote ]$.

\begin{eqnarray}
	\lift{w}{y!(z)}\widehat{\id{\{}u / z \id{\}}}
		& = &
		\lift{w}{y!(u)} \nonumber\\
	w[ \lpquote y!(z) \rpquote ] \widehat{ \id{\{}u / z \id{\}} }
		& = &
		w[ \lpquote y!(z) \rpquote ] \nonumber
\end{eqnarray}

Because the body of the process between quotes is impervious to
substitution, we get radically different answers. In fact, by
examining the first process in an input context,
e.g. $x?(z).\lift{w}{y!(z)}$, we see that the process under the lift
operator may be shaped by prefixed inputs binding a name inside it. In
this sense, the lift operator will be seen as a way to dynamically
construct processes before reifying them as names.

Finally equipped with these standard features we can present the
dynamics of the calculus.

\subsubsection{Operational semantics} 

Finally, we introduce the computational dynamics. What marks these
algebras as distinct from other more traditionally studied algebraic
structures, e.g. vector spaces or polynomial rings, is the manner in
which dynamics is captured. In traditional structures, dynamics is typically
expressed through morphisms between such structures, as in linear maps
between vector spaces or morphisms between rings. In algebras
associated with the semantics of computation, the dynamics is
expressed as part of the algebraic structure itself, through a
reduction reduction relation typically denoted by $\red$. Below, we
give a recursive presentation of this relation for the calculus used
in the encoding.

$\red \subseteq \pi \times \pi$
$\red : \pi \to \mathcal{P}(\pi)$

\begin{mathpar}
  \inferrule* [lab=Comm] { \textsf{match}( x_{src}, x_{trgt} ) } { x_{trgt}?(y)P \; | \; x_{src}!\langle {Q} \rangle \red P\{\quotep{Q}/y}\} }
  \and \\
  \inferrule* [lab=Par] {{P} \red {P}'} {{{P} | {Q}} \red {{P}' | {Q}}}
  \and
  \inferrule* [lab=Equiv]{{{P} \scong {P}'} \andalso {{P}' \red {Q}'} \andalso {{Q}' \scong {Q}}}{{P} \red {Q}}
\end{mathpar}

\begin{eqnarray*}
  match_{\equiv} (\quotep{P},\quotep{Q}) & := & P \equiv Q \\
  match_{\dagger}(\quotep{P},\quotep{Q}) & := & \forall R. P|Q \red^{*} R => R \red^{*} 0 \\
  match_{K}(\quotep{P},\quotep{Q}) & := & K \mbox{ for some context } K
\end{eqnarray*}

$u?(x)P | u!\langle Q \rangle \red P\{\quotep{Q}/x\}$

%We write $\wred$ for $\red^*$, and $P\red$ if $\exists Q $ such that $ P \red Q$.
We write $P\red$ if $\exists Q $ such that $ P \red Q$ and $P\not\red$, otherwise.

\section{Replication}

As mentioned before, it is known that replication (and hence
recursion) can be implemented in a higher-order process algebra
\cite{SangiorgiWalker}. As our first example of calculation with the
machinery thus far presented we give the construction explicitly in
the {\rhoc}.

\begin{eqnarray}
	D_{x} & := & \prefix{x}{y}{(\binpar{\outputp{x}{y}}{@{y}})} \nonumber\\
	\bangp_{x}{P} & := & \binpar{{x}!\langle{\binpar{D_{x}}{P}}\rangle}{D_{x}} \nonumber
\end{eqnarray}

\begin{eqnarray}
	\bangp_{x}{P} & & \nonumber\\
	=
	& {x}!\langle{(\prefix{x}{y}{(\outputp{x}{y} | @{y})) | P}}\rangle 
	      | \prefix{x}{y}{(\outputp{x}{y} | @{y})} & \nonumber\\
	\red
	& (\outputp{x}{y} | @{y})\substn{\quotep{(\prefix{x}{y}{(@{y} | \outputp{x}{y})) | P}}}{y} & \nonumber\\
	=
	& \outputp{x}{\quotep{(\prefix{x}{y}{(\outputp{x}{y} | @{y})) | P}}}
	  | {(\prefix{x}{y}{(\outputp{x}{y} | @{y})) | P}} & \nonumber\\
	\red
	& \ldots & \nonumber\\
	\red^*
	& P | P | \ldots & \nonumber
\end{eqnarray}

Of course, this encoding, as an implementation, runs away, unfolding
$\bangp{P}$ eagerly. A lazier and more implementable replication
operator, restricted to input-guarded processes, may be obtained as follows.

\begin{eqnarray}
\bangp{\prefix{u}{v}{P}} 
	:= 
	\binpar{\lift{x}{\prefix{u}{v}{(\binpar{D(x)}{P})}}}{D(x)} \nonumber
\end{eqnarray}

\begin{remark}
  Note that the lazier definition still does not deal with summation
  or mixed summation (i.e. sums over input and output). The reader is
  invited to construct definitions of replication that deal with these
  features. 

  Further, the definitions are parameterized in a name, $x$. Can you,
  gentle reader, make a definition that eliminates this parameter and
  guarantees no accidental interaction between the replication
  machinery and the process being replicated -- i.e. no accidental
  sharing of names used by the process to get its work done and the
  name(s) used by the replication to effect copying. This latter
  revision of the definition of replication is crucial to obtaining
  the expected identity $!!P \sim !P$.
\end{remark}

\begin{remark}\label{rem:paradoxical_combinator}
  The reader familiar with the lambda calculus will have noticed the
  similarity between $D$ and the paradoxical combinator.

  [Ed. note: the existence of this seems to suggest we have to be more
  restrictive on the set of processes and names we admit if we are to
  support no-cloning.]
\end{remark}

\subsubsection{Bisimulation}

The computational dynamics gives rise to another kind of equivalence,
the equivalence of computational behavior. As previously mentioned
this is typically captured \emph{via} some form of bisimulation.

% The notion we use in this paper is weak barbed bisimulation
% \cite{milner91polyadicpi}.

The notion we use in this paper is derived from weak barbed
bisimulation \cite{milner91polyadicpi}. 

\begin{definition}
An \emph{observation relation}, $\downarrow_{\mathcal N}$, over a set
of names, $\mathcal N$, is the smallest relation satisfying the rules
below.

\infrule[Out-barb]{y \in {\mathcal N}, \; x \nameeq y}
		  {\outputp{x}{v} \downarrow_{\mathcal N} x}
\infrule[Par-barb]{\mbox{$P\downarrow_{\mathcal N} x$ or $Q\downarrow_{\mathcal N} x$}}
		  {\binpar{P}{Q} \downarrow_{\mathcal N} x}

We write $P \Downarrow_{\mathcal N} x$ if there is $Q$ such that 
$P \wred Q$ and $Q \downarrow_{\mathcal N} x$.
\end{definition}

\begin{definition}
%\label{def.bbisim}
An  ${\mathcal N}$-\emph{barbed bisimulation} over a set of names, ${\mathcal N}$, is a symmetric binary relation 
${\mathcal S}_{\mathcal N}$ between agents such that $P\rel{S}_{\mathcal N}Q$ implies:
\begin{enumerate}
\item If $P \red P'$ then $Q \wred Q'$ and $P'\rel{S}_{\mathcal N} Q'$.
\item If $P\downarrow_{\mathcal N} x$, then $Q\Downarrow_{\mathcal N} x$.
\end{enumerate}
$P$ is ${\mathcal N}$-barbed bisimilar to $Q$, written
$P \wbbisim_{\mathcal N} Q$, if $P \rel{S}_{\mathcal N} Q$ for some ${\mathcal N}$-barbed bisimulation ${\mathcal S}_{\mathcal N}$.
\end{definition}

$\mathcal{R} \subseteq \pi \times \pi$

$P \mathcal{R} Q => \forall P'. P \red P' \Rightarrow \exists Q'. Q \red Q', P' \mathcal{R} Q'$

$P \vdash x \Rightarrow Q \vdash x$

\begin{mathpar}
  \inferrule*[lab=Out-barb]{x \nameeq y}{{y}!\langle{Q}\rangle \vdash x}
  \and
  \inferrule*[lab=Par-barb]{\mbox{$P\vdash x$ or $Q\vdash x$}}{\binpar{P}{Q} \vdash x}
\end{mathpar}

\subsubsection{Contexts}

One of the principle advantages of computational calculi like the
$\pi$-calculus is a well-defined notion of context,
contextual-equivalence and a correlation between
contextual-equivalence and notions of bisimulation. The notion of
context allows the decomposition of a process into (sub-)process and
its syntactic environment, its context. Thus, a context may be
thought of as a process with a ``hole'' (written $\Box$) in it. The
application of a context $M$ to a process $P$, written $M[P]$, is
tantamount to filling the hole in $M$ with $P$. In this paper we do
not need the full weight of this theory, but do make use of the notion
of context in the proof the main theorem. 

\begin{mathpar}
  \inferrule* [lab=summation] {} {{M_{M},M_{N}} \bc \Box \;|\; x.M_{A} \;|\; M_{M}+M_{N}}
  \and
  \inferrule* [lab=agent] {} {{M_{A}} \bc (\vec{x})M_{P} \;| \; \clift{P_0,\ldots,M_{P},\ldots,P_N}}
  \and \\
  \inferrule* [lab=process] {} {{M_{P}} \bc M_{N} \;| \;P|M_{P} }
\end{mathpar} 

\begin{mathpar}
  \inferrule* [lab=sychronization] {} {M_{N} \bc \Box \;|\; x?M_{F} \;|\; x!M_{C}}
  \and
  \inferrule* [lab=abstraction] {} {{M_{F}} \bc (x)M_{P} }
  \and
  \inferrule* [lab=concretion] {} {{M_{C}} \bc \langle M_{P} \rangle }
  \and \\
  \inferrule* [lab=process] {} {{M_{P}} \bc M_{N} \;| \;P|M_{P} }
\end{mathpar}

\begin{definition}[contextual application] Given a context $M$, and
  process $P$, we define the \emph{contextual application}, $M[P] :=
  M\{P/\Box\}$. That is, the contextual application of M to P is the
  substitution of $P$ for $\Box$ in $M$.
\end{definition}

$\meaningof{-} : L \to \mathcal{P}(\pi)$

\begin{mathpar}
  \inferrule* [lab=collection] {} {\meaningof{true} = \pi, \and \meaningof{~E} = \pi \setminus \meaningof{E}, \and \meaningof{E_{1} \& E_{2}} = \meaningof{E_{1}} \cap \meaningof{E_{2}}}
\end{mathpar}

\begin{mathpar}
  \inferrule* [lab=structure] {} {\meaningof{0} = \{ P \in \pi | P \equiv 0 \}, \and \\ \meaningof{E_1 | E_2} = \{ P \in \pi | P \equiv P_{1} | P_{2}, P_{1} \in \meaningof{E_{1}}, P_{2} \in \meaningof{E_2}\} }
\end{mathpar}

\begin{mathpar}
 \inferrule* [lab=behavior] {} {\meaningof{\langle a?b \rangle E} = \{ P \in \pi | P \equiv Q | u?(y)P', \\ \and \\\\ \and \\ \;\;\; u \in \meaningof{a}, \forall z.P'\{z/y\} \in \meaningof{E\{z/b\}}\}, \and \\ \meaningof{a!E} = \{ P \in \pi | P \equiv Q | x!\langle P' \rangle, x \in \meaningof{a} P' \in \meaningof{E}\} }
\end{mathpar}

\begin{mathpar}
 \inferrule* [lab=nominal] {} {\meaningof{\quotep{E}} = \{ \quotep{P} \in \quotep{\pi} | P \in \meaningof{E} \}, \and \meaningof{\quotep{P}} = \{ \quotep{Q} \in \quotep{\pi} | P \equiv Q \} \and \\ \meaningof{@\quotep{E}} = \{ P \in \pi | P \equiv @x, x \in \meaningof{E} \}}
\end{mathpar}

\begin{eqnarray*}
  \\
  \meaningof{-} : TS \to ST
\end{eqnarray*}

\begin{eqnarray*}
  \\
  L : TS \to ST
\end{eqnarray*}

\begin{eqnarray*}
  \\
  P \models E \iff P \in \meaningof{E}
\end{eqnarray*}

\begin{eqnarray*}
  P \approx_{L} Q \iff \forall E \in L. P \models E \iff Q \models E
\end{eqnarray*}

\begin{eqnarray*}
  P \approx_{K} Q
\end{eqnarray*}

\begin{eqnarray*}
  P \approx Q
\end{eqnarray*}

$\approx_{K} = \approx = \approx_{L}$

\subsubsection{Contextual duality}

Note that contexts extend the quotation operation to a family of
operations from processes to names. Given a context, $M$, we can
define a \emph{nominal context}, $\quotep{M}$ by $\quotep{M}[P] :=
\quotep{M[P]}$. To foreshadow what is to come we observe that these
operations enjoy a duality with processes very much like the duality
between vectors and maps from vectors to scalars.

Further, because the calculus is essentially higher-order, we have a
correspondence between contexts and processes. More specifically,
given a name $x$ and a context $M$ we can construct $M^{*}_{x}$ such
that 

\begin{mathpar}
  M^{*}_{x} | \lift{x}{P} \red M[P]
\end{mathpar}

namely,

\begin{mathpar}
  M^{*}_{x} := x?(u).M[\dropn{u}]
\end{mathpar}

The dependence of $M^{*}_{x}$ on a name makes it an abstraction, 

\begin{mathpar}
  M^{*} := (x)x?(u).M[\dropn{u}]
\end{mathpar}

\subsection{Additional notation}

It will sometimes be convenient to denote the process a name
quotes. We already have the notation $x = \quotep{P}$, but it will be
convenient to introduce an alternate notation, $\procn{x}$, when we
want to emphasize the connection to the use of the name. Note that, by
virtue of name equivalence, $\quotep{\procn{x}} \nameeq x$; so, the
notation is consistent with previous definitions.

Further, because names have structure it is possible to effect
substitutions on the basis of that structure. This means we need to
upgrade our notation for substitutions, which we accomplish by
adapting comprehension notation. Thus,

\begin{mathpar}
  P\{ y / x : x \in S \}
\end{mathpar}

is interpreted to mean the process derived from P by replacing (in a
capture-avoiding manner) each occurrence of $x$ in $S$ by $y$. For example,

\begin{mathpar}
  P\{ \quotep{\procn{x}|\procn{x}} / x : x \in \freenames{P} \}
\end{mathpar}

will replace each (occurrence) of a free name $x$ in $P$ by
$\quotep{\procn{x}|\procn{x}}$.

Also, we will avail ourselves of the notation $x^{L}$ and $x^{R}$ to
denote injections of a name into disjoint copies of the name
space. There are numerous ways to accomplish this. One example can be
found in \cite{MeredithR05}. This notation overloads to vectors of
names: $\vec{x}^{\pi} := (x_{i}^{\pi} \; : \; 0 \leq i < |\vec{x}| )$ where $\pi \in \{L,R\}$.

We also use $P^{\Box} := P|\Box$.

In \cite{MeredithR05} an interpretation of the new operator is
given. It turns out that there are several possible interpretations
all enjoying the requisite algebraic properties of the operator (see
\cite{milner91polyadicpi}). We will therefore make liberal use of
$(\nu\; \vec{x})P$.

% subsection the_syntax_and_semantics_of_the_notation_system (end)   

\input{qm2pi.qmops} 

\input{qm2pi.sterngerlach} 

\input{qm2pi.metric} 

% section concurrent_process_calculi (end)

%\input{qm2pi.proofsketch}

% section proof sketch (end)

%\input{qm2pi.slviaknots} 

% section spatial logic via knots (end)

\input{qm2pi.conclusion}

% section conclusion (end)

%\input{qm2pi.dtcodes} 

% section wiring algorithm (end)

\input{qm2pi.ack} 

% section acknowledgments (end)

\newpage


\bibliographystyle{plain}   
\bibliography{../../biblios/main.bib}

\input{qm2pi.rhodetails}

\end{document}



% section front matter (end)

\section{Introduction}\label{sec:introduction} % (fold)
In this draft of the material i am going to have to dispense with the
usual writing conventions adopted in papers on these topics. i'm going
to have adopt whatever tone i need at the time i'm writing up the
calculations. Sometimes this may be very conversational; others it may
be the barest mathematical grunts; others still it may be that i have
lifted text from one of my other papers because the exposition of some
point was better said there. i hope that my readers are not unduly put
out by this decision. i'm not doing this to flout convention or be
rebellious. i find these calculations very technically challenging. To
keep everything going technically, something has to give; i have to
let go of some cognitive burden. So, the academic writing style --
with all of its trade-offs in terms of facilitating technical
communication -- is what i'm letting go of. Perhaps subsequent drafts
can be tightened and polished, but for now, i'm going to speak as if
we were sitting together in a coffee shop with a laptop, wifi and a
pad of paper and a pencil.

So, here's what i have to say. We -- you and i, comfortably ensconced
in our coffee shop and well-equipped with our tools -- can realize and
carry out the calculations of quantum mechanics over a very different
formal theory of dynamics, a formal theory of dynamics that
corresponds to a theory of concurrent computation with
\emph{reflection}. It has the advantage that the underlying theory is
already `quantized', but supports analogues all of the continuuous
operations. Strikingly, this underlying theory has recently been
connected with a notion of metric that we can show, by calculating
together, coincides with the metric induced by the inner product.

There are a lot of reasons why you might be interested in seeing
calculations of this form. Here's why i'm interested. For the past
several centuries there has been no competitor to the ``Newtonian''
account of dynamics. As a result the predominant share of accounts of
dynamical systems and situations have had to be formulated in terms of
the Newtonian machinery. i view this as an intellectually dangerous
position to occupy. Everything, despite it's intrinsic shape, turns
into a nail to be hit with this hammer. Recently, however, the theory
of computation has matured to the point where we have candidates for
theories of dynamics that offer very different perspective on
reasoning about dynamical systems and situations. Testing these
candidates against very successful accounts of dynamical situations,
like quantum mechanics, is going to give us some sense of how mature
they are and some measure of the quality of these accounts of
dynamics.

\subsection{Summary of contributions and outline of paper}

So, we're going to develop an interpretation of the operations of
quantum mechanics normally interpreted by Hilbert spaces and
operators. We're going to do this over a theory of computation. Note
that this is very different than the usual quantum computation program
which develops notions of computation over quantum mechanics. Rather,
we are developing a story that aligns with Wheeler's slogan: It from
Bit. To do this we will first provide an account of the theory of
computation at play here. Then we will dive into a calculation-driven
interpretation of the operations of quantum mechanics.

The reason we take this approach is that -- until very recently --
there hasn't been an axiomatic account of quantum mechanics. As a
result there has been no sharp delineation of the mathematical theory
supporting interpretation of the physical theory and the physical
theory, itself. So, ambient features of the maths are free to be
exploited (or supressed) without a real accounting of their physical
relevance. There is no sharp statement ``here's the physical theory''
qua \emph{theory} and ``here's the mathematical interpretation''
enabling a judgment of how faithful the interpretation is -- apart
from experimental observation. When there is an axiomatic account we
can judge how well a given mathematical formalism supports an
interpretation of the axioms, independent of
experimentation. Likewise, we can judge how well we have captured our
physical evidence and experience with our axiomatics, independent of
any specific mathematical implementation, with accidental detail that
may or may not have physical significance. 

In lieu of a fully fleshed out and vetted axiomatic account of quantum
mechanics, interpreting the operational notions in service of modeling
physical systems will have to suffice. In other words, we are not in
the business of providing a model of Hilbert spaces and operators. We
are in the business of providing a model of quantum mechanics because
we are motivated by testing our notions of dynamics against physical
theory; and, the predictive calculations of the physical theory must
serve as the best formulation -- shy of a fully fleshed out axiomatic
account -- of the physical theory itself (as they have for scientific
theories since time immemorial). Put another way, despite a
whole-hearted commitment to an It-from-Bit ontology, we are firmly
aligned with the shut-up-and-calculate camp as the best way to obtain
results either from the physical perspective or as a quality assurance
measure of our fledgling theory of dynamics.

In detail, we present a reflective process calculus. Then we develop
intuitive correspondences between the notions available in this
calculus and the usual physical notions supporting quantum mechanical
calculations. Thus, 

\begin{table}[htp]
  \center{
    \fbox{
      \begin{tabular}{c|c}
        quantum mechanics & process calculus \\
        \hline
        scalar & name \\
        state vector & process \\
        dual & contextual duals \\
        matrix & formal sums of process-context-dual pairs \\
        orthogonality & process annihilation \\
        inner product & execution-formula + quoting
      \end{tabular}
    }
  }
  \caption{QM - process calculi correspondences}
\end{table}

Then we tighten up these intuitions to operational definitions. We
employ the Dirac notation as the best proxy we can find for an
abstract syntax of the quantum mechanical notions. The definitions we
develop put us in contact with equational constraints coming from the
theory that we demonstrate the definitions and calculations satisfy.

This puts us in a position to shut up and calculate for the
Stern-Gerlach experimental set up, showing how these predictive
calculations become calculations on processes in our theory of a
reflective process calculus.

Penultimately, we demonstrate that the notion of metric coming from
the inner product coincides with the notion of metric available from
the theory of bisimulation. This demonstration gives us the right to
think of space as arising from behavior. Finally, we consider where we
might go from the new vantage point we have obtained.

% section introduction (end) 
 
% section introduction (end)

% \documentclass[12pt]{llncs}
%\documentclass{jktr}

\usepackage[pdftex]{hyperref}                   
\usepackage {listings}
\usepackage {mathpartir}
\usepackage{bcprules}
%\usepackage{listings}
                       
\usepackage{graphicx} 
%\usepackage[margins=2.5cm,nohead,nofoot]{geometry}
%\usepackage{geometry}
\usepackage{amsfonts}
\usepackage{amstext}
\usepackage{latexsym}
\usepackage{amssymb}
\usepackage{color}


%\include{myPreamble}
\include{qm2pi.local} 

%\ifpdf
%\usepackage[pdftex]{graphicx}
%\else
%\usepackage{graphicx}
%\fi

 % \ifpdf
%  \usepackage{pdfsync}
%  \if


%\title{Brief Article}
%\author{David F. Snyder}
%\author{L.G. Meredith}

%\address{Dept. of Math., Texas State University--San Marcos, San Marcos, TX 78666}
       
\pagestyle{empty}


\begin{document}

\lstset{language=[Objective]Caml,frame=shadowbox}

\input{qm2pi.front}

% section front matter (end)

\input{qm2pi.intro} 
 
% section introduction (end)

% \input{qm2pi.knotations} 

% section notation (end)

\input{qm2pi.process.calculi} 

% section concurrent_process_calculi_and_spatial_logics_ (end)
    
%\input{qm2pi.knots2pi} 

%\input{qm2pi.trefoil} 

%\input{qm2pi.mainthm} 

% subsection basic_interpretation (end)

%\input{qm2pi.rho.presentation} 
\subsection{The syntax and semantics of the notation system}\label{sub:the_syntax_and_semantics_of_the_notation_system} % (fold)

We now summarize a technical presentation of the calculus that
embodies our theory of dynamics. The typical presentation of such a
calculus follows the style of giving generators and relations on
them. The grammar, below, describing term constructors, freely
generates the set of processes, $\Proc$. This set is then quotiented
by a relation known as structural congruence and it is over this set
that the notion of dynamics is expressed. This presentation is
essentially that of \cite{MeredithR05} with the addition of
polyadicity and summation. For readability we have relegated some of
the technical subtleties to an appendix.

\subsubsection{Process grammar}\label{subsub:process_grammar}

\begin{mathpar}
  \inferrule* [lab=synchronization] {} {{M} \bc \pzero \;|\; x?F \;|\; x!C }
  \and
  \inferrule* [lab=abstraction] {} {{F} \bc (x)P}
  \and
  \inferrule* [lab=concretion] {} {{C} \bc \langle Q \rangle}
  \and
  \inferrule* [lab=process] {} {{P,Q} \bc M \;| \;P|Q \;|\; @{x}}
  \and
  \inferrule* [lab=name] {} {{x} \bc \quotep{P}}
\end{mathpar} 

Note that $\vec{x}$ (resp. $\vec{P}$) denotes a vector of names
(resp. processes) of length $|\vec{x}|$ (resp. $|\vec{P}|$). We adopt
the following useful abbreviations.

\begin{mathpar}
   x?(\vec{y}).P := x.(\vec{y})P \and  x\clift{\vec{P}} := x.\clift{\vec{P}}
   \and x!(y) := \lift{x}{\dropn{y}}
   \and \Pi_{i=0}^{n-1}P_i := P_0 | \ldots | P_{n-1}
\end{mathpar}

\subsubsection{Structural congruence}

\paragraph{Free and bound names and alpha-equivalence.} At the
core of structural equivalence is alpha-equivalence which identifies
process that are the same up to a change of variable. Formally, we
recognize the distinction between free and bound names. The free names
of a process, $\freenames{P}$, may be calculated recursively as
follows:

\begin{mathpar}
\freenames{\pzero} := \emptyset
  \and \\
  \freenames{x?(y).P} := \{ x \} \cup (\freenames{P} \setminus \{ y \})
  \and 
  \freenames{x!\langle P \rangle} := \{ x \} \cup \{ P \} 
  \and \\
  \freenames{P|Q} := \freenames{P} \cup \freenames{Q}
  \and \\
  \freenames{@{x}} := \{ x \}
\end{mathpar}

$\pi$
$\quotep{\pi}$

$\freenames{-} : \pi \to \mathcal{P}(\quotep{\pi})$

\begin{eqnarray*}
  \freenames{\pzero} & := & \emptyset \\
  \freenames{x?(y).P} & := & \{ x \} \cup (\freenames{P} \setminus \{ y \}) \\
  \freenames{x!\langle P \rangle} & := & \{ x \} \cup \{ P \} \\
  \freenames{P|Q} & := & \freenames{P} \cup \freenames{Q} \\
  \freenames{\dropn{x}} & := & \{ x \}
\end{eqnarray*}

The bound names of a process, $\boundnames{P}$, are those names occurring in $P$
that are not free. For example, in $x?(y).0$, the name $x$ is free, while $y$ is bound.

\begin{mathpar}
  \inferrule* [lab=monoidal-laws] {} { P|Q \equiv Q|P \and P|0 \equiv P \and P|(Q|R) \equiv (P|Q)|R }
\end{mathpar}

\begin{mathpar}
  \inferrule* [lab=alpha-equivalence] {} { (x)P \equiv (y)P\{y/x\} \and y \not\in \freenames{P} }
\end{mathpar}

\begin{definition}
Then two processes, $P,Q$, are alpha-equivalent if $P = Q\{\vec{y}/\vec{x}\}$ for
some $\vec{x} \in \boundnames{Q},\vec{y} \in \boundnames{P}$, where $Q\{\vec{y}/\vec{x}\}$
denotes the capture-avoiding substitution of $\vec{y}$ for $\vec{x}$ in $Q$.
\end{definition}

\begin{definition}
  The {\em structural congruence} \cite{SangiorgiWalker} , $\equiv$,
  between processes is the least congruence containing
  alpha-equivalence, satisfying the abelian monoid laws
  (associativity, commutativity and $\pzero$ as identity) for parallel
  composition $|$ and for summation $+$.
\end{definition}

\subsection{Name equivalence}

We take name equivalence, written $\nameeq$, to be the smallest
equivalence relation generated by the following rules.

\begin{mathpar}
\inferrule*[lab=Quote-drop]
{ }
{ \quotep{@{x}} \nameeq x }

\inferrule*[lab=Struct-equiv]
{ P \scong Q }
{ \quotep{P} \nameeq \quotep{Q} }
\end{mathpar}

The astute reader will have noticed that the mutual recursion of names
and processes imposes a mutual recursion on alpha-equivalence and
structural equivalence via name-equivalence. Fortunately, all of this
works out pleasantly and we may calculate in the natural way, free of
concern. The reader interested in the details is referred to the
appendix \ref{appendix:rho_details}.

\subsection{Substitution}

We use $\Proc$ for the set of processes, $\QProc$ for the set of
names, and $\id{\{}\vec{y} / \vec{x} \id{\}}$ to denote partial maps,
$s : \QProc \rightarrow \QProc$. A map, $s$ lifts, uniquely, to a map
on process terms, $\widehat{s} : \Proc \rightarrow \Proc$ by the
following equations.

\begin{mathpar}
  (0) \psubstp{Q}{P} := 0 \\
  (R \juxtap S) \psubstp{Q}{P}
  :=    
  (R)\psubstp{Q}{P} \juxtap (S) \psubstp{Q}{P} \\
  (x?(y).R) \psubstp{Q}{P}    
  :=    
  (x)\substp{Q}{P} (z)\concat( (R \psubstn{z}{y}) \psubstp{Q}{P} ) \\
  (\lift{x}{R}) \psubstp{Q}{P}  
  :=
  \lift{(x)\substp{Q}{P}}{ R \psubstp{Q}{P} } \\
%   (\dropn{x})  \psubstp{Q}{P}       
%   := 
%   \left\{ 
%     \begin{array}{ccc} 
%       \dropn{\quotep{Q}} & & x \nameeq \quotep{P} \\
%       \dropn{x} & & otherwise \\
%     \end{array}
%   \right. 
  (\dropn{x})  \psubstp{Q}{P}       
  := 
  \left\{ 
    \begin{array}{ccc} 
      Q & & x \nameeq \quotep{P} \\
      \dropn{x} & & otherwise \\
    \end{array}
  \right.
\end{mathpar}
 

where

\begin{eqnarray}
  (x)\id{\{} \lpquote Q \rpquote / \lpquote P \rpquote \id{\}}            = 
  \left\{ 
    \begin{array}{ccc}
      \lpquote Q \rpquote & & x \nameeq \lpquote P \rpquote \\
      x & & otherwise \\
    \end{array}
  \right. \nonumber
\end{eqnarray}

and $z$ is chosen distinct from $\quotep{P}$, $\quotep{Q}$, the free
names in $Q$, and all the names in $R$. Our $\alpha$-equivalence will
be built in the standard way from this substitution.

\begin{remark}\label{rem:no_self_referential_names}
  One consequence of these definitions is that $\forall P. \quotep{P}
  \not\in \freenames{P}$.
\end{remark}

\subsection{ Dynamic quote: an example }

Anticipating something of what's to come, consider applying the
substitution, $\widehat{\id{\{}u / z \id{\}}}$, to the following pair
of processes, $\lift{w}{y!(z)}$ and $w[ \lpquote y!(z) \rpquote ]$.

\begin{eqnarray}
	\lift{w}{y!(z)}\widehat{\id{\{}u / z \id{\}}}
		& = &
		\lift{w}{y!(u)} \nonumber\\
	w[ \lpquote y!(z) \rpquote ] \widehat{ \id{\{}u / z \id{\}} }
		& = &
		w[ \lpquote y!(z) \rpquote ] \nonumber
\end{eqnarray}

Because the body of the process between quotes is impervious to
substitution, we get radically different answers. In fact, by
examining the first process in an input context,
e.g. $x?(z).\lift{w}{y!(z)}$, we see that the process under the lift
operator may be shaped by prefixed inputs binding a name inside it. In
this sense, the lift operator will be seen as a way to dynamically
construct processes before reifying them as names.

Finally equipped with these standard features we can present the
dynamics of the calculus.

\subsubsection{Operational semantics} 

Finally, we introduce the computational dynamics. What marks these
algebras as distinct from other more traditionally studied algebraic
structures, e.g. vector spaces or polynomial rings, is the manner in
which dynamics is captured. In traditional structures, dynamics is typically
expressed through morphisms between such structures, as in linear maps
between vector spaces or morphisms between rings. In algebras
associated with the semantics of computation, the dynamics is
expressed as part of the algebraic structure itself, through a
reduction reduction relation typically denoted by $\red$. Below, we
give a recursive presentation of this relation for the calculus used
in the encoding.

$\red \subseteq \pi \times \pi$
$\red : \pi \to \mathcal{P}(\pi)$

\begin{mathpar}
  \inferrule* [lab=Comm] { \textsf{match}( x_{src}, x_{trgt} ) } { x_{trgt}?(y)P \; | \; x_{src}!\langle {Q} \rangle \red P\{\quotep{Q}/y}\} }
  \and \\
  \inferrule* [lab=Par] {{P} \red {P}'} {{{P} | {Q}} \red {{P}' | {Q}}}
  \and
  \inferrule* [lab=Equiv]{{{P} \scong {P}'} \andalso {{P}' \red {Q}'} \andalso {{Q}' \scong {Q}}}{{P} \red {Q}}
\end{mathpar}

\begin{eqnarray*}
  match_{\equiv} (\quotep{P},\quotep{Q}) & := & P \equiv Q \\
  match_{\dagger}(\quotep{P},\quotep{Q}) & := & \forall R. P|Q \red^{*} R => R \red^{*} 0 \\
  match_{K}(\quotep{P},\quotep{Q}) & := & K \mbox{ for some context } K
\end{eqnarray*}

$u?(x)P | u!\langle Q \rangle \red P\{\quotep{Q}/x\}$

%We write $\wred$ for $\red^*$, and $P\red$ if $\exists Q $ such that $ P \red Q$.
We write $P\red$ if $\exists Q $ such that $ P \red Q$ and $P\not\red$, otherwise.

\section{Replication}

As mentioned before, it is known that replication (and hence
recursion) can be implemented in a higher-order process algebra
\cite{SangiorgiWalker}. As our first example of calculation with the
machinery thus far presented we give the construction explicitly in
the {\rhoc}.

\begin{eqnarray}
	D_{x} & := & \prefix{x}{y}{(\binpar{\outputp{x}{y}}{@{y}})} \nonumber\\
	\bangp_{x}{P} & := & \binpar{{x}!\langle{\binpar{D_{x}}{P}}\rangle}{D_{x}} \nonumber
\end{eqnarray}

\begin{eqnarray}
	\bangp_{x}{P} & & \nonumber\\
	=
	& {x}!\langle{(\prefix{x}{y}{(\outputp{x}{y} | @{y})) | P}}\rangle 
	      | \prefix{x}{y}{(\outputp{x}{y} | @{y})} & \nonumber\\
	\red
	& (\outputp{x}{y} | @{y})\substn{\quotep{(\prefix{x}{y}{(@{y} | \outputp{x}{y})) | P}}}{y} & \nonumber\\
	=
	& \outputp{x}{\quotep{(\prefix{x}{y}{(\outputp{x}{y} | @{y})) | P}}}
	  | {(\prefix{x}{y}{(\outputp{x}{y} | @{y})) | P}} & \nonumber\\
	\red
	& \ldots & \nonumber\\
	\red^*
	& P | P | \ldots & \nonumber
\end{eqnarray}

Of course, this encoding, as an implementation, runs away, unfolding
$\bangp{P}$ eagerly. A lazier and more implementable replication
operator, restricted to input-guarded processes, may be obtained as follows.

\begin{eqnarray}
\bangp{\prefix{u}{v}{P}} 
	:= 
	\binpar{\lift{x}{\prefix{u}{v}{(\binpar{D(x)}{P})}}}{D(x)} \nonumber
\end{eqnarray}

\begin{remark}
  Note that the lazier definition still does not deal with summation
  or mixed summation (i.e. sums over input and output). The reader is
  invited to construct definitions of replication that deal with these
  features. 

  Further, the definitions are parameterized in a name, $x$. Can you,
  gentle reader, make a definition that eliminates this parameter and
  guarantees no accidental interaction between the replication
  machinery and the process being replicated -- i.e. no accidental
  sharing of names used by the process to get its work done and the
  name(s) used by the replication to effect copying. This latter
  revision of the definition of replication is crucial to obtaining
  the expected identity $!!P \sim !P$.
\end{remark}

\begin{remark}\label{rem:paradoxical_combinator}
  The reader familiar with the lambda calculus will have noticed the
  similarity between $D$ and the paradoxical combinator.

  [Ed. note: the existence of this seems to suggest we have to be more
  restrictive on the set of processes and names we admit if we are to
  support no-cloning.]
\end{remark}

\subsubsection{Bisimulation}

The computational dynamics gives rise to another kind of equivalence,
the equivalence of computational behavior. As previously mentioned
this is typically captured \emph{via} some form of bisimulation.

% The notion we use in this paper is weak barbed bisimulation
% \cite{milner91polyadicpi}.

The notion we use in this paper is derived from weak barbed
bisimulation \cite{milner91polyadicpi}. 

\begin{definition}
An \emph{observation relation}, $\downarrow_{\mathcal N}$, over a set
of names, $\mathcal N$, is the smallest relation satisfying the rules
below.

\infrule[Out-barb]{y \in {\mathcal N}, \; x \nameeq y}
		  {\outputp{x}{v} \downarrow_{\mathcal N} x}
\infrule[Par-barb]{\mbox{$P\downarrow_{\mathcal N} x$ or $Q\downarrow_{\mathcal N} x$}}
		  {\binpar{P}{Q} \downarrow_{\mathcal N} x}

We write $P \Downarrow_{\mathcal N} x$ if there is $Q$ such that 
$P \wred Q$ and $Q \downarrow_{\mathcal N} x$.
\end{definition}

\begin{definition}
%\label{def.bbisim}
An  ${\mathcal N}$-\emph{barbed bisimulation} over a set of names, ${\mathcal N}$, is a symmetric binary relation 
${\mathcal S}_{\mathcal N}$ between agents such that $P\rel{S}_{\mathcal N}Q$ implies:
\begin{enumerate}
\item If $P \red P'$ then $Q \wred Q'$ and $P'\rel{S}_{\mathcal N} Q'$.
\item If $P\downarrow_{\mathcal N} x$, then $Q\Downarrow_{\mathcal N} x$.
\end{enumerate}
$P$ is ${\mathcal N}$-barbed bisimilar to $Q$, written
$P \wbbisim_{\mathcal N} Q$, if $P \rel{S}_{\mathcal N} Q$ for some ${\mathcal N}$-barbed bisimulation ${\mathcal S}_{\mathcal N}$.
\end{definition}

$\mathcal{R} \subseteq \pi \times \pi$

$P \mathcal{R} Q => \forall P'. P \red P' \Rightarrow \exists Q'. Q \red Q', P' \mathcal{R} Q'$

$P \vdash x \Rightarrow Q \vdash x$

\begin{mathpar}
  \inferrule*[lab=Out-barb]{x \nameeq y}{{y}!\langle{Q}\rangle \vdash x}
  \and
  \inferrule*[lab=Par-barb]{\mbox{$P\vdash x$ or $Q\vdash x$}}{\binpar{P}{Q} \vdash x}
\end{mathpar}

\subsubsection{Contexts}

One of the principle advantages of computational calculi like the
$\pi$-calculus is a well-defined notion of context,
contextual-equivalence and a correlation between
contextual-equivalence and notions of bisimulation. The notion of
context allows the decomposition of a process into (sub-)process and
its syntactic environment, its context. Thus, a context may be
thought of as a process with a ``hole'' (written $\Box$) in it. The
application of a context $M$ to a process $P$, written $M[P]$, is
tantamount to filling the hole in $M$ with $P$. In this paper we do
not need the full weight of this theory, but do make use of the notion
of context in the proof the main theorem. 

\begin{mathpar}
  \inferrule* [lab=summation] {} {{M_{M},M_{N}} \bc \Box \;|\; x.M_{A} \;|\; M_{M}+M_{N}}
  \and
  \inferrule* [lab=agent] {} {{M_{A}} \bc (\vec{x})M_{P} \;| \; \clift{P_0,\ldots,M_{P},\ldots,P_N}}
  \and \\
  \inferrule* [lab=process] {} {{M_{P}} \bc M_{N} \;| \;P|M_{P} }
\end{mathpar} 

\begin{mathpar}
  \inferrule* [lab=sychronization] {} {M_{N} \bc \Box \;|\; x?M_{F} \;|\; x!M_{C}}
  \and
  \inferrule* [lab=abstraction] {} {{M_{F}} \bc (x)M_{P} }
  \and
  \inferrule* [lab=concretion] {} {{M_{C}} \bc \langle M_{P} \rangle }
  \and \\
  \inferrule* [lab=process] {} {{M_{P}} \bc M_{N} \;| \;P|M_{P} }
\end{mathpar}

\begin{definition}[contextual application] Given a context $M$, and
  process $P$, we define the \emph{contextual application}, $M[P] :=
  M\{P/\Box\}$. That is, the contextual application of M to P is the
  substitution of $P$ for $\Box$ in $M$.
\end{definition}

$\meaningof{-} : L \to \mathcal{P}(\pi)$

\begin{mathpar}
  \inferrule* [lab=collection] {} {\meaningof{true} = \pi, \and \meaningof{~E} = \pi \setminus \meaningof{E}, \and \meaningof{E_{1} \& E_{2}} = \meaningof{E_{1}} \cap \meaningof{E_{2}}}
\end{mathpar}

\begin{mathpar}
  \inferrule* [lab=structure] {} {\meaningof{0} = \{ P \in \pi | P \equiv 0 \}, \and \\ \meaningof{E_1 | E_2} = \{ P \in \pi | P \equiv P_{1} | P_{2}, P_{1} \in \meaningof{E_{1}}, P_{2} \in \meaningof{E_2}\} }
\end{mathpar}

\begin{mathpar}
 \inferrule* [lab=behavior] {} {\meaningof{\langle a?b \rangle E} = \{ P \in \pi | P \equiv Q | u?(y)P', \\ \and \\\\ \and \\ \;\;\; u \in \meaningof{a}, \forall z.P'\{z/y\} \in \meaningof{E\{z/b\}}\}, \and \\ \meaningof{a!E} = \{ P \in \pi | P \equiv Q | x!\langle P' \rangle, x \in \meaningof{a} P' \in \meaningof{E}\} }
\end{mathpar}

\begin{mathpar}
 \inferrule* [lab=nominal] {} {\meaningof{\quotep{E}} = \{ \quotep{P} \in \quotep{\pi} | P \in \meaningof{E} \}, \and \meaningof{\quotep{P}} = \{ \quotep{Q} \in \quotep{\pi} | P \equiv Q \} \and \\ \meaningof{@\quotep{E}} = \{ P \in \pi | P \equiv @x, x \in \meaningof{E} \}}
\end{mathpar}

\begin{eqnarray*}
  \\
  \meaningof{-} : TS \to ST
\end{eqnarray*}

\begin{eqnarray*}
  \\
  L : TS \to ST
\end{eqnarray*}

\begin{eqnarray*}
  \\
  P \models E \iff P \in \meaningof{E}
\end{eqnarray*}

\begin{eqnarray*}
  P \approx_{L} Q \iff \forall E \in L. P \models E \iff Q \models E
\end{eqnarray*}

\begin{eqnarray*}
  P \approx_{K} Q
\end{eqnarray*}

\begin{eqnarray*}
  P \approx Q
\end{eqnarray*}

$\approx_{K} = \approx = \approx_{L}$

\subsubsection{Contextual duality}

Note that contexts extend the quotation operation to a family of
operations from processes to names. Given a context, $M$, we can
define a \emph{nominal context}, $\quotep{M}$ by $\quotep{M}[P] :=
\quotep{M[P]}$. To foreshadow what is to come we observe that these
operations enjoy a duality with processes very much like the duality
between vectors and maps from vectors to scalars.

Further, because the calculus is essentially higher-order, we have a
correspondence between contexts and processes. More specifically,
given a name $x$ and a context $M$ we can construct $M^{*}_{x}$ such
that 

\begin{mathpar}
  M^{*}_{x} | \lift{x}{P} \red M[P]
\end{mathpar}

namely,

\begin{mathpar}
  M^{*}_{x} := x?(u).M[\dropn{u}]
\end{mathpar}

The dependence of $M^{*}_{x}$ on a name makes it an abstraction, 

\begin{mathpar}
  M^{*} := (x)x?(u).M[\dropn{u}]
\end{mathpar}

\subsection{Additional notation}

It will sometimes be convenient to denote the process a name
quotes. We already have the notation $x = \quotep{P}$, but it will be
convenient to introduce an alternate notation, $\procn{x}$, when we
want to emphasize the connection to the use of the name. Note that, by
virtue of name equivalence, $\quotep{\procn{x}} \nameeq x$; so, the
notation is consistent with previous definitions.

Further, because names have structure it is possible to effect
substitutions on the basis of that structure. This means we need to
upgrade our notation for substitutions, which we accomplish by
adapting comprehension notation. Thus,

\begin{mathpar}
  P\{ y / x : x \in S \}
\end{mathpar}

is interpreted to mean the process derived from P by replacing (in a
capture-avoiding manner) each occurrence of $x$ in $S$ by $y$. For example,

\begin{mathpar}
  P\{ \quotep{\procn{x}|\procn{x}} / x : x \in \freenames{P} \}
\end{mathpar}

will replace each (occurrence) of a free name $x$ in $P$ by
$\quotep{\procn{x}|\procn{x}}$.

Also, we will avail ourselves of the notation $x^{L}$ and $x^{R}$ to
denote injections of a name into disjoint copies of the name
space. There are numerous ways to accomplish this. One example can be
found in \cite{MeredithR05}. This notation overloads to vectors of
names: $\vec{x}^{\pi} := (x_{i}^{\pi} \; : \; 0 \leq i < |\vec{x}| )$ where $\pi \in \{L,R\}$.

We also use $P^{\Box} := P|\Box$.

In \cite{MeredithR05} an interpretation of the new operator is
given. It turns out that there are several possible interpretations
all enjoying the requisite algebraic properties of the operator (see
\cite{milner91polyadicpi}). We will therefore make liberal use of
$(\nu\; \vec{x})P$.

% subsection the_syntax_and_semantics_of_the_notation_system (end)   

\input{qm2pi.qmops} 

\input{qm2pi.sterngerlach} 

\input{qm2pi.metric} 

% section concurrent_process_calculi (end)

%\input{qm2pi.proofsketch}

% section proof sketch (end)

%\input{qm2pi.slviaknots} 

% section spatial logic via knots (end)

\input{qm2pi.conclusion}

% section conclusion (end)

%\input{qm2pi.dtcodes} 

% section wiring algorithm (end)

\input{qm2pi.ack} 

% section acknowledgments (end)

\newpage


\bibliographystyle{plain}   
\bibliography{../../biblios/main.bib}

\input{qm2pi.rhodetails}

\end{document}

 

% section notation (end)

\input{qm2pi.process.calculi} 

% section concurrent_process_calculi_and_spatial_logics_ (end)
    
%\documentclass[12pt]{llncs}
%\documentclass{jktr}

\usepackage[pdftex]{hyperref}                   
\usepackage {listings}
\usepackage {mathpartir}
\usepackage{bcprules}
%\usepackage{listings}
                       
\usepackage{graphicx} 
%\usepackage[margins=2.5cm,nohead,nofoot]{geometry}
%\usepackage{geometry}
\usepackage{amsfonts}
\usepackage{amstext}
\usepackage{latexsym}
\usepackage{amssymb}
\usepackage{color}


%\include{myPreamble}
\include{qm2pi.local} 

%\ifpdf
%\usepackage[pdftex]{graphicx}
%\else
%\usepackage{graphicx}
%\fi

 % \ifpdf
%  \usepackage{pdfsync}
%  \if


%\title{Brief Article}
%\author{David F. Snyder}
%\author{L.G. Meredith}

%\address{Dept. of Math., Texas State University--San Marcos, San Marcos, TX 78666}
       
\pagestyle{empty}


\begin{document}

\lstset{language=[Objective]Caml,frame=shadowbox}

\input{qm2pi.front}

% section front matter (end)

\input{qm2pi.intro} 
 
% section introduction (end)

% \input{qm2pi.knotations} 

% section notation (end)

\input{qm2pi.process.calculi} 

% section concurrent_process_calculi_and_spatial_logics_ (end)
    
%\input{qm2pi.knots2pi} 

%\input{qm2pi.trefoil} 

%\input{qm2pi.mainthm} 

% subsection basic_interpretation (end)

%\input{qm2pi.rho.presentation} 
\subsection{The syntax and semantics of the notation system}\label{sub:the_syntax_and_semantics_of_the_notation_system} % (fold)

We now summarize a technical presentation of the calculus that
embodies our theory of dynamics. The typical presentation of such a
calculus follows the style of giving generators and relations on
them. The grammar, below, describing term constructors, freely
generates the set of processes, $\Proc$. This set is then quotiented
by a relation known as structural congruence and it is over this set
that the notion of dynamics is expressed. This presentation is
essentially that of \cite{MeredithR05} with the addition of
polyadicity and summation. For readability we have relegated some of
the technical subtleties to an appendix.

\subsubsection{Process grammar}\label{subsub:process_grammar}

\begin{mathpar}
  \inferrule* [lab=synchronization] {} {{M} \bc \pzero \;|\; x?F \;|\; x!C }
  \and
  \inferrule* [lab=abstraction] {} {{F} \bc (x)P}
  \and
  \inferrule* [lab=concretion] {} {{C} \bc \langle Q \rangle}
  \and
  \inferrule* [lab=process] {} {{P,Q} \bc M \;| \;P|Q \;|\; @{x}}
  \and
  \inferrule* [lab=name] {} {{x} \bc \quotep{P}}
\end{mathpar} 

Note that $\vec{x}$ (resp. $\vec{P}$) denotes a vector of names
(resp. processes) of length $|\vec{x}|$ (resp. $|\vec{P}|$). We adopt
the following useful abbreviations.

\begin{mathpar}
   x?(\vec{y}).P := x.(\vec{y})P \and  x\clift{\vec{P}} := x.\clift{\vec{P}}
   \and x!(y) := \lift{x}{\dropn{y}}
   \and \Pi_{i=0}^{n-1}P_i := P_0 | \ldots | P_{n-1}
\end{mathpar}

\subsubsection{Structural congruence}

\paragraph{Free and bound names and alpha-equivalence.} At the
core of structural equivalence is alpha-equivalence which identifies
process that are the same up to a change of variable. Formally, we
recognize the distinction between free and bound names. The free names
of a process, $\freenames{P}$, may be calculated recursively as
follows:

\begin{mathpar}
\freenames{\pzero} := \emptyset
  \and \\
  \freenames{x?(y).P} := \{ x \} \cup (\freenames{P} \setminus \{ y \})
  \and 
  \freenames{x!\langle P \rangle} := \{ x \} \cup \{ P \} 
  \and \\
  \freenames{P|Q} := \freenames{P} \cup \freenames{Q}
  \and \\
  \freenames{@{x}} := \{ x \}
\end{mathpar}

$\pi$
$\quotep{\pi}$

$\freenames{-} : \pi \to \mathcal{P}(\quotep{\pi})$

\begin{eqnarray*}
  \freenames{\pzero} & := & \emptyset \\
  \freenames{x?(y).P} & := & \{ x \} \cup (\freenames{P} \setminus \{ y \}) \\
  \freenames{x!\langle P \rangle} & := & \{ x \} \cup \{ P \} \\
  \freenames{P|Q} & := & \freenames{P} \cup \freenames{Q} \\
  \freenames{\dropn{x}} & := & \{ x \}
\end{eqnarray*}

The bound names of a process, $\boundnames{P}$, are those names occurring in $P$
that are not free. For example, in $x?(y).0$, the name $x$ is free, while $y$ is bound.

\begin{mathpar}
  \inferrule* [lab=monoidal-laws] {} { P|Q \equiv Q|P \and P|0 \equiv P \and P|(Q|R) \equiv (P|Q)|R }
\end{mathpar}

\begin{mathpar}
  \inferrule* [lab=alpha-equivalence] {} { (x)P \equiv (y)P\{y/x\} \and y \not\in \freenames{P} }
\end{mathpar}

\begin{definition}
Then two processes, $P,Q$, are alpha-equivalent if $P = Q\{\vec{y}/\vec{x}\}$ for
some $\vec{x} \in \boundnames{Q},\vec{y} \in \boundnames{P}$, where $Q\{\vec{y}/\vec{x}\}$
denotes the capture-avoiding substitution of $\vec{y}$ for $\vec{x}$ in $Q$.
\end{definition}

\begin{definition}
  The {\em structural congruence} \cite{SangiorgiWalker} , $\equiv$,
  between processes is the least congruence containing
  alpha-equivalence, satisfying the abelian monoid laws
  (associativity, commutativity and $\pzero$ as identity) for parallel
  composition $|$ and for summation $+$.
\end{definition}

\subsection{Name equivalence}

We take name equivalence, written $\nameeq$, to be the smallest
equivalence relation generated by the following rules.

\begin{mathpar}
\inferrule*[lab=Quote-drop]
{ }
{ \quotep{@{x}} \nameeq x }

\inferrule*[lab=Struct-equiv]
{ P \scong Q }
{ \quotep{P} \nameeq \quotep{Q} }
\end{mathpar}

The astute reader will have noticed that the mutual recursion of names
and processes imposes a mutual recursion on alpha-equivalence and
structural equivalence via name-equivalence. Fortunately, all of this
works out pleasantly and we may calculate in the natural way, free of
concern. The reader interested in the details is referred to the
appendix \ref{appendix:rho_details}.

\subsection{Substitution}

We use $\Proc$ for the set of processes, $\QProc$ for the set of
names, and $\id{\{}\vec{y} / \vec{x} \id{\}}$ to denote partial maps,
$s : \QProc \rightarrow \QProc$. A map, $s$ lifts, uniquely, to a map
on process terms, $\widehat{s} : \Proc \rightarrow \Proc$ by the
following equations.

\begin{mathpar}
  (0) \psubstp{Q}{P} := 0 \\
  (R \juxtap S) \psubstp{Q}{P}
  :=    
  (R)\psubstp{Q}{P} \juxtap (S) \psubstp{Q}{P} \\
  (x?(y).R) \psubstp{Q}{P}    
  :=    
  (x)\substp{Q}{P} (z)\concat( (R \psubstn{z}{y}) \psubstp{Q}{P} ) \\
  (\lift{x}{R}) \psubstp{Q}{P}  
  :=
  \lift{(x)\substp{Q}{P}}{ R \psubstp{Q}{P} } \\
%   (\dropn{x})  \psubstp{Q}{P}       
%   := 
%   \left\{ 
%     \begin{array}{ccc} 
%       \dropn{\quotep{Q}} & & x \nameeq \quotep{P} \\
%       \dropn{x} & & otherwise \\
%     \end{array}
%   \right. 
  (\dropn{x})  \psubstp{Q}{P}       
  := 
  \left\{ 
    \begin{array}{ccc} 
      Q & & x \nameeq \quotep{P} \\
      \dropn{x} & & otherwise \\
    \end{array}
  \right.
\end{mathpar}
 

where

\begin{eqnarray}
  (x)\id{\{} \lpquote Q \rpquote / \lpquote P \rpquote \id{\}}            = 
  \left\{ 
    \begin{array}{ccc}
      \lpquote Q \rpquote & & x \nameeq \lpquote P \rpquote \\
      x & & otherwise \\
    \end{array}
  \right. \nonumber
\end{eqnarray}

and $z$ is chosen distinct from $\quotep{P}$, $\quotep{Q}$, the free
names in $Q$, and all the names in $R$. Our $\alpha$-equivalence will
be built in the standard way from this substitution.

\begin{remark}\label{rem:no_self_referential_names}
  One consequence of these definitions is that $\forall P. \quotep{P}
  \not\in \freenames{P}$.
\end{remark}

\subsection{ Dynamic quote: an example }

Anticipating something of what's to come, consider applying the
substitution, $\widehat{\id{\{}u / z \id{\}}}$, to the following pair
of processes, $\lift{w}{y!(z)}$ and $w[ \lpquote y!(z) \rpquote ]$.

\begin{eqnarray}
	\lift{w}{y!(z)}\widehat{\id{\{}u / z \id{\}}}
		& = &
		\lift{w}{y!(u)} \nonumber\\
	w[ \lpquote y!(z) \rpquote ] \widehat{ \id{\{}u / z \id{\}} }
		& = &
		w[ \lpquote y!(z) \rpquote ] \nonumber
\end{eqnarray}

Because the body of the process between quotes is impervious to
substitution, we get radically different answers. In fact, by
examining the first process in an input context,
e.g. $x?(z).\lift{w}{y!(z)}$, we see that the process under the lift
operator may be shaped by prefixed inputs binding a name inside it. In
this sense, the lift operator will be seen as a way to dynamically
construct processes before reifying them as names.

Finally equipped with these standard features we can present the
dynamics of the calculus.

\subsubsection{Operational semantics} 

Finally, we introduce the computational dynamics. What marks these
algebras as distinct from other more traditionally studied algebraic
structures, e.g. vector spaces or polynomial rings, is the manner in
which dynamics is captured. In traditional structures, dynamics is typically
expressed through morphisms between such structures, as in linear maps
between vector spaces or morphisms between rings. In algebras
associated with the semantics of computation, the dynamics is
expressed as part of the algebraic structure itself, through a
reduction reduction relation typically denoted by $\red$. Below, we
give a recursive presentation of this relation for the calculus used
in the encoding.

$\red \subseteq \pi \times \pi$
$\red : \pi \to \mathcal{P}(\pi)$

\begin{mathpar}
  \inferrule* [lab=Comm] { \textsf{match}( x_{src}, x_{trgt} ) } { x_{trgt}?(y)P \; | \; x_{src}!\langle {Q} \rangle \red P\{\quotep{Q}/y}\} }
  \and \\
  \inferrule* [lab=Par] {{P} \red {P}'} {{{P} | {Q}} \red {{P}' | {Q}}}
  \and
  \inferrule* [lab=Equiv]{{{P} \scong {P}'} \andalso {{P}' \red {Q}'} \andalso {{Q}' \scong {Q}}}{{P} \red {Q}}
\end{mathpar}

\begin{eqnarray*}
  match_{\equiv} (\quotep{P},\quotep{Q}) & := & P \equiv Q \\
  match_{\dagger}(\quotep{P},\quotep{Q}) & := & \forall R. P|Q \red^{*} R => R \red^{*} 0 \\
  match_{K}(\quotep{P},\quotep{Q}) & := & K \mbox{ for some context } K
\end{eqnarray*}

$u?(x)P | u!\langle Q \rangle \red P\{\quotep{Q}/x\}$

%We write $\wred$ for $\red^*$, and $P\red$ if $\exists Q $ such that $ P \red Q$.
We write $P\red$ if $\exists Q $ such that $ P \red Q$ and $P\not\red$, otherwise.

\section{Replication}

As mentioned before, it is known that replication (and hence
recursion) can be implemented in a higher-order process algebra
\cite{SangiorgiWalker}. As our first example of calculation with the
machinery thus far presented we give the construction explicitly in
the {\rhoc}.

\begin{eqnarray}
	D_{x} & := & \prefix{x}{y}{(\binpar{\outputp{x}{y}}{@{y}})} \nonumber\\
	\bangp_{x}{P} & := & \binpar{{x}!\langle{\binpar{D_{x}}{P}}\rangle}{D_{x}} \nonumber
\end{eqnarray}

\begin{eqnarray}
	\bangp_{x}{P} & & \nonumber\\
	=
	& {x}!\langle{(\prefix{x}{y}{(\outputp{x}{y} | @{y})) | P}}\rangle 
	      | \prefix{x}{y}{(\outputp{x}{y} | @{y})} & \nonumber\\
	\red
	& (\outputp{x}{y} | @{y})\substn{\quotep{(\prefix{x}{y}{(@{y} | \outputp{x}{y})) | P}}}{y} & \nonumber\\
	=
	& \outputp{x}{\quotep{(\prefix{x}{y}{(\outputp{x}{y} | @{y})) | P}}}
	  | {(\prefix{x}{y}{(\outputp{x}{y} | @{y})) | P}} & \nonumber\\
	\red
	& \ldots & \nonumber\\
	\red^*
	& P | P | \ldots & \nonumber
\end{eqnarray}

Of course, this encoding, as an implementation, runs away, unfolding
$\bangp{P}$ eagerly. A lazier and more implementable replication
operator, restricted to input-guarded processes, may be obtained as follows.

\begin{eqnarray}
\bangp{\prefix{u}{v}{P}} 
	:= 
	\binpar{\lift{x}{\prefix{u}{v}{(\binpar{D(x)}{P})}}}{D(x)} \nonumber
\end{eqnarray}

\begin{remark}
  Note that the lazier definition still does not deal with summation
  or mixed summation (i.e. sums over input and output). The reader is
  invited to construct definitions of replication that deal with these
  features. 

  Further, the definitions are parameterized in a name, $x$. Can you,
  gentle reader, make a definition that eliminates this parameter and
  guarantees no accidental interaction between the replication
  machinery and the process being replicated -- i.e. no accidental
  sharing of names used by the process to get its work done and the
  name(s) used by the replication to effect copying. This latter
  revision of the definition of replication is crucial to obtaining
  the expected identity $!!P \sim !P$.
\end{remark}

\begin{remark}\label{rem:paradoxical_combinator}
  The reader familiar with the lambda calculus will have noticed the
  similarity between $D$ and the paradoxical combinator.

  [Ed. note: the existence of this seems to suggest we have to be more
  restrictive on the set of processes and names we admit if we are to
  support no-cloning.]
\end{remark}

\subsubsection{Bisimulation}

The computational dynamics gives rise to another kind of equivalence,
the equivalence of computational behavior. As previously mentioned
this is typically captured \emph{via} some form of bisimulation.

% The notion we use in this paper is weak barbed bisimulation
% \cite{milner91polyadicpi}.

The notion we use in this paper is derived from weak barbed
bisimulation \cite{milner91polyadicpi}. 

\begin{definition}
An \emph{observation relation}, $\downarrow_{\mathcal N}$, over a set
of names, $\mathcal N$, is the smallest relation satisfying the rules
below.

\infrule[Out-barb]{y \in {\mathcal N}, \; x \nameeq y}
		  {\outputp{x}{v} \downarrow_{\mathcal N} x}
\infrule[Par-barb]{\mbox{$P\downarrow_{\mathcal N} x$ or $Q\downarrow_{\mathcal N} x$}}
		  {\binpar{P}{Q} \downarrow_{\mathcal N} x}

We write $P \Downarrow_{\mathcal N} x$ if there is $Q$ such that 
$P \wred Q$ and $Q \downarrow_{\mathcal N} x$.
\end{definition}

\begin{definition}
%\label{def.bbisim}
An  ${\mathcal N}$-\emph{barbed bisimulation} over a set of names, ${\mathcal N}$, is a symmetric binary relation 
${\mathcal S}_{\mathcal N}$ between agents such that $P\rel{S}_{\mathcal N}Q$ implies:
\begin{enumerate}
\item If $P \red P'$ then $Q \wred Q'$ and $P'\rel{S}_{\mathcal N} Q'$.
\item If $P\downarrow_{\mathcal N} x$, then $Q\Downarrow_{\mathcal N} x$.
\end{enumerate}
$P$ is ${\mathcal N}$-barbed bisimilar to $Q$, written
$P \wbbisim_{\mathcal N} Q$, if $P \rel{S}_{\mathcal N} Q$ for some ${\mathcal N}$-barbed bisimulation ${\mathcal S}_{\mathcal N}$.
\end{definition}

$\mathcal{R} \subseteq \pi \times \pi$

$P \mathcal{R} Q => \forall P'. P \red P' \Rightarrow \exists Q'. Q \red Q', P' \mathcal{R} Q'$

$P \vdash x \Rightarrow Q \vdash x$

\begin{mathpar}
  \inferrule*[lab=Out-barb]{x \nameeq y}{{y}!\langle{Q}\rangle \vdash x}
  \and
  \inferrule*[lab=Par-barb]{\mbox{$P\vdash x$ or $Q\vdash x$}}{\binpar{P}{Q} \vdash x}
\end{mathpar}

\subsubsection{Contexts}

One of the principle advantages of computational calculi like the
$\pi$-calculus is a well-defined notion of context,
contextual-equivalence and a correlation between
contextual-equivalence and notions of bisimulation. The notion of
context allows the decomposition of a process into (sub-)process and
its syntactic environment, its context. Thus, a context may be
thought of as a process with a ``hole'' (written $\Box$) in it. The
application of a context $M$ to a process $P$, written $M[P]$, is
tantamount to filling the hole in $M$ with $P$. In this paper we do
not need the full weight of this theory, but do make use of the notion
of context in the proof the main theorem. 

\begin{mathpar}
  \inferrule* [lab=summation] {} {{M_{M},M_{N}} \bc \Box \;|\; x.M_{A} \;|\; M_{M}+M_{N}}
  \and
  \inferrule* [lab=agent] {} {{M_{A}} \bc (\vec{x})M_{P} \;| \; \clift{P_0,\ldots,M_{P},\ldots,P_N}}
  \and \\
  \inferrule* [lab=process] {} {{M_{P}} \bc M_{N} \;| \;P|M_{P} }
\end{mathpar} 

\begin{mathpar}
  \inferrule* [lab=sychronization] {} {M_{N} \bc \Box \;|\; x?M_{F} \;|\; x!M_{C}}
  \and
  \inferrule* [lab=abstraction] {} {{M_{F}} \bc (x)M_{P} }
  \and
  \inferrule* [lab=concretion] {} {{M_{C}} \bc \langle M_{P} \rangle }
  \and \\
  \inferrule* [lab=process] {} {{M_{P}} \bc M_{N} \;| \;P|M_{P} }
\end{mathpar}

\begin{definition}[contextual application] Given a context $M$, and
  process $P$, we define the \emph{contextual application}, $M[P] :=
  M\{P/\Box\}$. That is, the contextual application of M to P is the
  substitution of $P$ for $\Box$ in $M$.
\end{definition}

$\meaningof{-} : L \to \mathcal{P}(\pi)$

\begin{mathpar}
  \inferrule* [lab=collection] {} {\meaningof{true} = \pi, \and \meaningof{~E} = \pi \setminus \meaningof{E}, \and \meaningof{E_{1} \& E_{2}} = \meaningof{E_{1}} \cap \meaningof{E_{2}}}
\end{mathpar}

\begin{mathpar}
  \inferrule* [lab=structure] {} {\meaningof{0} = \{ P \in \pi | P \equiv 0 \}, \and \\ \meaningof{E_1 | E_2} = \{ P \in \pi | P \equiv P_{1} | P_{2}, P_{1} \in \meaningof{E_{1}}, P_{2} \in \meaningof{E_2}\} }
\end{mathpar}

\begin{mathpar}
 \inferrule* [lab=behavior] {} {\meaningof{\langle a?b \rangle E} = \{ P \in \pi | P \equiv Q | u?(y)P', \\ \and \\\\ \and \\ \;\;\; u \in \meaningof{a}, \forall z.P'\{z/y\} \in \meaningof{E\{z/b\}}\}, \and \\ \meaningof{a!E} = \{ P \in \pi | P \equiv Q | x!\langle P' \rangle, x \in \meaningof{a} P' \in \meaningof{E}\} }
\end{mathpar}

\begin{mathpar}
 \inferrule* [lab=nominal] {} {\meaningof{\quotep{E}} = \{ \quotep{P} \in \quotep{\pi} | P \in \meaningof{E} \}, \and \meaningof{\quotep{P}} = \{ \quotep{Q} \in \quotep{\pi} | P \equiv Q \} \and \\ \meaningof{@\quotep{E}} = \{ P \in \pi | P \equiv @x, x \in \meaningof{E} \}}
\end{mathpar}

\begin{eqnarray*}
  \\
  \meaningof{-} : TS \to ST
\end{eqnarray*}

\begin{eqnarray*}
  \\
  L : TS \to ST
\end{eqnarray*}

\begin{eqnarray*}
  \\
  P \models E \iff P \in \meaningof{E}
\end{eqnarray*}

\begin{eqnarray*}
  P \approx_{L} Q \iff \forall E \in L. P \models E \iff Q \models E
\end{eqnarray*}

\begin{eqnarray*}
  P \approx_{K} Q
\end{eqnarray*}

\begin{eqnarray*}
  P \approx Q
\end{eqnarray*}

$\approx_{K} = \approx = \approx_{L}$

\subsubsection{Contextual duality}

Note that contexts extend the quotation operation to a family of
operations from processes to names. Given a context, $M$, we can
define a \emph{nominal context}, $\quotep{M}$ by $\quotep{M}[P] :=
\quotep{M[P]}$. To foreshadow what is to come we observe that these
operations enjoy a duality with processes very much like the duality
between vectors and maps from vectors to scalars.

Further, because the calculus is essentially higher-order, we have a
correspondence between contexts and processes. More specifically,
given a name $x$ and a context $M$ we can construct $M^{*}_{x}$ such
that 

\begin{mathpar}
  M^{*}_{x} | \lift{x}{P} \red M[P]
\end{mathpar}

namely,

\begin{mathpar}
  M^{*}_{x} := x?(u).M[\dropn{u}]
\end{mathpar}

The dependence of $M^{*}_{x}$ on a name makes it an abstraction, 

\begin{mathpar}
  M^{*} := (x)x?(u).M[\dropn{u}]
\end{mathpar}

\subsection{Additional notation}

It will sometimes be convenient to denote the process a name
quotes. We already have the notation $x = \quotep{P}$, but it will be
convenient to introduce an alternate notation, $\procn{x}$, when we
want to emphasize the connection to the use of the name. Note that, by
virtue of name equivalence, $\quotep{\procn{x}} \nameeq x$; so, the
notation is consistent with previous definitions.

Further, because names have structure it is possible to effect
substitutions on the basis of that structure. This means we need to
upgrade our notation for substitutions, which we accomplish by
adapting comprehension notation. Thus,

\begin{mathpar}
  P\{ y / x : x \in S \}
\end{mathpar}

is interpreted to mean the process derived from P by replacing (in a
capture-avoiding manner) each occurrence of $x$ in $S$ by $y$. For example,

\begin{mathpar}
  P\{ \quotep{\procn{x}|\procn{x}} / x : x \in \freenames{P} \}
\end{mathpar}

will replace each (occurrence) of a free name $x$ in $P$ by
$\quotep{\procn{x}|\procn{x}}$.

Also, we will avail ourselves of the notation $x^{L}$ and $x^{R}$ to
denote injections of a name into disjoint copies of the name
space. There are numerous ways to accomplish this. One example can be
found in \cite{MeredithR05}. This notation overloads to vectors of
names: $\vec{x}^{\pi} := (x_{i}^{\pi} \; : \; 0 \leq i < |\vec{x}| )$ where $\pi \in \{L,R\}$.

We also use $P^{\Box} := P|\Box$.

In \cite{MeredithR05} an interpretation of the new operator is
given. It turns out that there are several possible interpretations
all enjoying the requisite algebraic properties of the operator (see
\cite{milner91polyadicpi}). We will therefore make liberal use of
$(\nu\; \vec{x})P$.

% subsection the_syntax_and_semantics_of_the_notation_system (end)   

\input{qm2pi.qmops} 

\input{qm2pi.sterngerlach} 

\input{qm2pi.metric} 

% section concurrent_process_calculi (end)

%\input{qm2pi.proofsketch}

% section proof sketch (end)

%\input{qm2pi.slviaknots} 

% section spatial logic via knots (end)

\input{qm2pi.conclusion}

% section conclusion (end)

%\input{qm2pi.dtcodes} 

% section wiring algorithm (end)

\input{qm2pi.ack} 

% section acknowledgments (end)

\newpage


\bibliographystyle{plain}   
\bibliography{../../biblios/main.bib}

\input{qm2pi.rhodetails}

\end{document}

 

%\documentclass[12pt]{llncs}
%\documentclass{jktr}

\usepackage[pdftex]{hyperref}                   
\usepackage {listings}
\usepackage {mathpartir}
\usepackage{bcprules}
%\usepackage{listings}
                       
\usepackage{graphicx} 
%\usepackage[margins=2.5cm,nohead,nofoot]{geometry}
%\usepackage{geometry}
\usepackage{amsfonts}
\usepackage{amstext}
\usepackage{latexsym}
\usepackage{amssymb}
\usepackage{color}


%\include{myPreamble}
\include{qm2pi.local} 

%\ifpdf
%\usepackage[pdftex]{graphicx}
%\else
%\usepackage{graphicx}
%\fi

 % \ifpdf
%  \usepackage{pdfsync}
%  \if


%\title{Brief Article}
%\author{David F. Snyder}
%\author{L.G. Meredith}

%\address{Dept. of Math., Texas State University--San Marcos, San Marcos, TX 78666}
       
\pagestyle{empty}


\begin{document}

\lstset{language=[Objective]Caml,frame=shadowbox}

\input{qm2pi.front}

% section front matter (end)

\input{qm2pi.intro} 
 
% section introduction (end)

% \input{qm2pi.knotations} 

% section notation (end)

\input{qm2pi.process.calculi} 

% section concurrent_process_calculi_and_spatial_logics_ (end)
    
%\input{qm2pi.knots2pi} 

%\input{qm2pi.trefoil} 

%\input{qm2pi.mainthm} 

% subsection basic_interpretation (end)

%\input{qm2pi.rho.presentation} 
\subsection{The syntax and semantics of the notation system}\label{sub:the_syntax_and_semantics_of_the_notation_system} % (fold)

We now summarize a technical presentation of the calculus that
embodies our theory of dynamics. The typical presentation of such a
calculus follows the style of giving generators and relations on
them. The grammar, below, describing term constructors, freely
generates the set of processes, $\Proc$. This set is then quotiented
by a relation known as structural congruence and it is over this set
that the notion of dynamics is expressed. This presentation is
essentially that of \cite{MeredithR05} with the addition of
polyadicity and summation. For readability we have relegated some of
the technical subtleties to an appendix.

\subsubsection{Process grammar}\label{subsub:process_grammar}

\begin{mathpar}
  \inferrule* [lab=synchronization] {} {{M} \bc \pzero \;|\; x?F \;|\; x!C }
  \and
  \inferrule* [lab=abstraction] {} {{F} \bc (x)P}
  \and
  \inferrule* [lab=concretion] {} {{C} \bc \langle Q \rangle}
  \and
  \inferrule* [lab=process] {} {{P,Q} \bc M \;| \;P|Q \;|\; @{x}}
  \and
  \inferrule* [lab=name] {} {{x} \bc \quotep{P}}
\end{mathpar} 

Note that $\vec{x}$ (resp. $\vec{P}$) denotes a vector of names
(resp. processes) of length $|\vec{x}|$ (resp. $|\vec{P}|$). We adopt
the following useful abbreviations.

\begin{mathpar}
   x?(\vec{y}).P := x.(\vec{y})P \and  x\clift{\vec{P}} := x.\clift{\vec{P}}
   \and x!(y) := \lift{x}{\dropn{y}}
   \and \Pi_{i=0}^{n-1}P_i := P_0 | \ldots | P_{n-1}
\end{mathpar}

\subsubsection{Structural congruence}

\paragraph{Free and bound names and alpha-equivalence.} At the
core of structural equivalence is alpha-equivalence which identifies
process that are the same up to a change of variable. Formally, we
recognize the distinction between free and bound names. The free names
of a process, $\freenames{P}$, may be calculated recursively as
follows:

\begin{mathpar}
\freenames{\pzero} := \emptyset
  \and \\
  \freenames{x?(y).P} := \{ x \} \cup (\freenames{P} \setminus \{ y \})
  \and 
  \freenames{x!\langle P \rangle} := \{ x \} \cup \{ P \} 
  \and \\
  \freenames{P|Q} := \freenames{P} \cup \freenames{Q}
  \and \\
  \freenames{@{x}} := \{ x \}
\end{mathpar}

$\pi$
$\quotep{\pi}$

$\freenames{-} : \pi \to \mathcal{P}(\quotep{\pi})$

\begin{eqnarray*}
  \freenames{\pzero} & := & \emptyset \\
  \freenames{x?(y).P} & := & \{ x \} \cup (\freenames{P} \setminus \{ y \}) \\
  \freenames{x!\langle P \rangle} & := & \{ x \} \cup \{ P \} \\
  \freenames{P|Q} & := & \freenames{P} \cup \freenames{Q} \\
  \freenames{\dropn{x}} & := & \{ x \}
\end{eqnarray*}

The bound names of a process, $\boundnames{P}$, are those names occurring in $P$
that are not free. For example, in $x?(y).0$, the name $x$ is free, while $y$ is bound.

\begin{mathpar}
  \inferrule* [lab=monoidal-laws] {} { P|Q \equiv Q|P \and P|0 \equiv P \and P|(Q|R) \equiv (P|Q)|R }
\end{mathpar}

\begin{mathpar}
  \inferrule* [lab=alpha-equivalence] {} { (x)P \equiv (y)P\{y/x\} \and y \not\in \freenames{P} }
\end{mathpar}

\begin{definition}
Then two processes, $P,Q$, are alpha-equivalent if $P = Q\{\vec{y}/\vec{x}\}$ for
some $\vec{x} \in \boundnames{Q},\vec{y} \in \boundnames{P}$, where $Q\{\vec{y}/\vec{x}\}$
denotes the capture-avoiding substitution of $\vec{y}$ for $\vec{x}$ in $Q$.
\end{definition}

\begin{definition}
  The {\em structural congruence} \cite{SangiorgiWalker} , $\equiv$,
  between processes is the least congruence containing
  alpha-equivalence, satisfying the abelian monoid laws
  (associativity, commutativity and $\pzero$ as identity) for parallel
  composition $|$ and for summation $+$.
\end{definition}

\subsection{Name equivalence}

We take name equivalence, written $\nameeq$, to be the smallest
equivalence relation generated by the following rules.

\begin{mathpar}
\inferrule*[lab=Quote-drop]
{ }
{ \quotep{@{x}} \nameeq x }

\inferrule*[lab=Struct-equiv]
{ P \scong Q }
{ \quotep{P} \nameeq \quotep{Q} }
\end{mathpar}

The astute reader will have noticed that the mutual recursion of names
and processes imposes a mutual recursion on alpha-equivalence and
structural equivalence via name-equivalence. Fortunately, all of this
works out pleasantly and we may calculate in the natural way, free of
concern. The reader interested in the details is referred to the
appendix \ref{appendix:rho_details}.

\subsection{Substitution}

We use $\Proc$ for the set of processes, $\QProc$ for the set of
names, and $\id{\{}\vec{y} / \vec{x} \id{\}}$ to denote partial maps,
$s : \QProc \rightarrow \QProc$. A map, $s$ lifts, uniquely, to a map
on process terms, $\widehat{s} : \Proc \rightarrow \Proc$ by the
following equations.

\begin{mathpar}
  (0) \psubstp{Q}{P} := 0 \\
  (R \juxtap S) \psubstp{Q}{P}
  :=    
  (R)\psubstp{Q}{P} \juxtap (S) \psubstp{Q}{P} \\
  (x?(y).R) \psubstp{Q}{P}    
  :=    
  (x)\substp{Q}{P} (z)\concat( (R \psubstn{z}{y}) \psubstp{Q}{P} ) \\
  (\lift{x}{R}) \psubstp{Q}{P}  
  :=
  \lift{(x)\substp{Q}{P}}{ R \psubstp{Q}{P} } \\
%   (\dropn{x})  \psubstp{Q}{P}       
%   := 
%   \left\{ 
%     \begin{array}{ccc} 
%       \dropn{\quotep{Q}} & & x \nameeq \quotep{P} \\
%       \dropn{x} & & otherwise \\
%     \end{array}
%   \right. 
  (\dropn{x})  \psubstp{Q}{P}       
  := 
  \left\{ 
    \begin{array}{ccc} 
      Q & & x \nameeq \quotep{P} \\
      \dropn{x} & & otherwise \\
    \end{array}
  \right.
\end{mathpar}
 

where

\begin{eqnarray}
  (x)\id{\{} \lpquote Q \rpquote / \lpquote P \rpquote \id{\}}            = 
  \left\{ 
    \begin{array}{ccc}
      \lpquote Q \rpquote & & x \nameeq \lpquote P \rpquote \\
      x & & otherwise \\
    \end{array}
  \right. \nonumber
\end{eqnarray}

and $z$ is chosen distinct from $\quotep{P}$, $\quotep{Q}$, the free
names in $Q$, and all the names in $R$. Our $\alpha$-equivalence will
be built in the standard way from this substitution.

\begin{remark}\label{rem:no_self_referential_names}
  One consequence of these definitions is that $\forall P. \quotep{P}
  \not\in \freenames{P}$.
\end{remark}

\subsection{ Dynamic quote: an example }

Anticipating something of what's to come, consider applying the
substitution, $\widehat{\id{\{}u / z \id{\}}}$, to the following pair
of processes, $\lift{w}{y!(z)}$ and $w[ \lpquote y!(z) \rpquote ]$.

\begin{eqnarray}
	\lift{w}{y!(z)}\widehat{\id{\{}u / z \id{\}}}
		& = &
		\lift{w}{y!(u)} \nonumber\\
	w[ \lpquote y!(z) \rpquote ] \widehat{ \id{\{}u / z \id{\}} }
		& = &
		w[ \lpquote y!(z) \rpquote ] \nonumber
\end{eqnarray}

Because the body of the process between quotes is impervious to
substitution, we get radically different answers. In fact, by
examining the first process in an input context,
e.g. $x?(z).\lift{w}{y!(z)}$, we see that the process under the lift
operator may be shaped by prefixed inputs binding a name inside it. In
this sense, the lift operator will be seen as a way to dynamically
construct processes before reifying them as names.

Finally equipped with these standard features we can present the
dynamics of the calculus.

\subsubsection{Operational semantics} 

Finally, we introduce the computational dynamics. What marks these
algebras as distinct from other more traditionally studied algebraic
structures, e.g. vector spaces or polynomial rings, is the manner in
which dynamics is captured. In traditional structures, dynamics is typically
expressed through morphisms between such structures, as in linear maps
between vector spaces or morphisms between rings. In algebras
associated with the semantics of computation, the dynamics is
expressed as part of the algebraic structure itself, through a
reduction reduction relation typically denoted by $\red$. Below, we
give a recursive presentation of this relation for the calculus used
in the encoding.

$\red \subseteq \pi \times \pi$
$\red : \pi \to \mathcal{P}(\pi)$

\begin{mathpar}
  \inferrule* [lab=Comm] { \textsf{match}( x_{src}, x_{trgt} ) } { x_{trgt}?(y)P \; | \; x_{src}!\langle {Q} \rangle \red P\{\quotep{Q}/y}\} }
  \and \\
  \inferrule* [lab=Par] {{P} \red {P}'} {{{P} | {Q}} \red {{P}' | {Q}}}
  \and
  \inferrule* [lab=Equiv]{{{P} \scong {P}'} \andalso {{P}' \red {Q}'} \andalso {{Q}' \scong {Q}}}{{P} \red {Q}}
\end{mathpar}

\begin{eqnarray*}
  match_{\equiv} (\quotep{P},\quotep{Q}) & := & P \equiv Q \\
  match_{\dagger}(\quotep{P},\quotep{Q}) & := & \forall R. P|Q \red^{*} R => R \red^{*} 0 \\
  match_{K}(\quotep{P},\quotep{Q}) & := & K \mbox{ for some context } K
\end{eqnarray*}

$u?(x)P | u!\langle Q \rangle \red P\{\quotep{Q}/x\}$

%We write $\wred$ for $\red^*$, and $P\red$ if $\exists Q $ such that $ P \red Q$.
We write $P\red$ if $\exists Q $ such that $ P \red Q$ and $P\not\red$, otherwise.

\section{Replication}

As mentioned before, it is known that replication (and hence
recursion) can be implemented in a higher-order process algebra
\cite{SangiorgiWalker}. As our first example of calculation with the
machinery thus far presented we give the construction explicitly in
the {\rhoc}.

\begin{eqnarray}
	D_{x} & := & \prefix{x}{y}{(\binpar{\outputp{x}{y}}{@{y}})} \nonumber\\
	\bangp_{x}{P} & := & \binpar{{x}!\langle{\binpar{D_{x}}{P}}\rangle}{D_{x}} \nonumber
\end{eqnarray}

\begin{eqnarray}
	\bangp_{x}{P} & & \nonumber\\
	=
	& {x}!\langle{(\prefix{x}{y}{(\outputp{x}{y} | @{y})) | P}}\rangle 
	      | \prefix{x}{y}{(\outputp{x}{y} | @{y})} & \nonumber\\
	\red
	& (\outputp{x}{y} | @{y})\substn{\quotep{(\prefix{x}{y}{(@{y} | \outputp{x}{y})) | P}}}{y} & \nonumber\\
	=
	& \outputp{x}{\quotep{(\prefix{x}{y}{(\outputp{x}{y} | @{y})) | P}}}
	  | {(\prefix{x}{y}{(\outputp{x}{y} | @{y})) | P}} & \nonumber\\
	\red
	& \ldots & \nonumber\\
	\red^*
	& P | P | \ldots & \nonumber
\end{eqnarray}

Of course, this encoding, as an implementation, runs away, unfolding
$\bangp{P}$ eagerly. A lazier and more implementable replication
operator, restricted to input-guarded processes, may be obtained as follows.

\begin{eqnarray}
\bangp{\prefix{u}{v}{P}} 
	:= 
	\binpar{\lift{x}{\prefix{u}{v}{(\binpar{D(x)}{P})}}}{D(x)} \nonumber
\end{eqnarray}

\begin{remark}
  Note that the lazier definition still does not deal with summation
  or mixed summation (i.e. sums over input and output). The reader is
  invited to construct definitions of replication that deal with these
  features. 

  Further, the definitions are parameterized in a name, $x$. Can you,
  gentle reader, make a definition that eliminates this parameter and
  guarantees no accidental interaction between the replication
  machinery and the process being replicated -- i.e. no accidental
  sharing of names used by the process to get its work done and the
  name(s) used by the replication to effect copying. This latter
  revision of the definition of replication is crucial to obtaining
  the expected identity $!!P \sim !P$.
\end{remark}

\begin{remark}\label{rem:paradoxical_combinator}
  The reader familiar with the lambda calculus will have noticed the
  similarity between $D$ and the paradoxical combinator.

  [Ed. note: the existence of this seems to suggest we have to be more
  restrictive on the set of processes and names we admit if we are to
  support no-cloning.]
\end{remark}

\subsubsection{Bisimulation}

The computational dynamics gives rise to another kind of equivalence,
the equivalence of computational behavior. As previously mentioned
this is typically captured \emph{via} some form of bisimulation.

% The notion we use in this paper is weak barbed bisimulation
% \cite{milner91polyadicpi}.

The notion we use in this paper is derived from weak barbed
bisimulation \cite{milner91polyadicpi}. 

\begin{definition}
An \emph{observation relation}, $\downarrow_{\mathcal N}$, over a set
of names, $\mathcal N$, is the smallest relation satisfying the rules
below.

\infrule[Out-barb]{y \in {\mathcal N}, \; x \nameeq y}
		  {\outputp{x}{v} \downarrow_{\mathcal N} x}
\infrule[Par-barb]{\mbox{$P\downarrow_{\mathcal N} x$ or $Q\downarrow_{\mathcal N} x$}}
		  {\binpar{P}{Q} \downarrow_{\mathcal N} x}

We write $P \Downarrow_{\mathcal N} x$ if there is $Q$ such that 
$P \wred Q$ and $Q \downarrow_{\mathcal N} x$.
\end{definition}

\begin{definition}
%\label{def.bbisim}
An  ${\mathcal N}$-\emph{barbed bisimulation} over a set of names, ${\mathcal N}$, is a symmetric binary relation 
${\mathcal S}_{\mathcal N}$ between agents such that $P\rel{S}_{\mathcal N}Q$ implies:
\begin{enumerate}
\item If $P \red P'$ then $Q \wred Q'$ and $P'\rel{S}_{\mathcal N} Q'$.
\item If $P\downarrow_{\mathcal N} x$, then $Q\Downarrow_{\mathcal N} x$.
\end{enumerate}
$P$ is ${\mathcal N}$-barbed bisimilar to $Q$, written
$P \wbbisim_{\mathcal N} Q$, if $P \rel{S}_{\mathcal N} Q$ for some ${\mathcal N}$-barbed bisimulation ${\mathcal S}_{\mathcal N}$.
\end{definition}

$\mathcal{R} \subseteq \pi \times \pi$

$P \mathcal{R} Q => \forall P'. P \red P' \Rightarrow \exists Q'. Q \red Q', P' \mathcal{R} Q'$

$P \vdash x \Rightarrow Q \vdash x$

\begin{mathpar}
  \inferrule*[lab=Out-barb]{x \nameeq y}{{y}!\langle{Q}\rangle \vdash x}
  \and
  \inferrule*[lab=Par-barb]{\mbox{$P\vdash x$ or $Q\vdash x$}}{\binpar{P}{Q} \vdash x}
\end{mathpar}

\subsubsection{Contexts}

One of the principle advantages of computational calculi like the
$\pi$-calculus is a well-defined notion of context,
contextual-equivalence and a correlation between
contextual-equivalence and notions of bisimulation. The notion of
context allows the decomposition of a process into (sub-)process and
its syntactic environment, its context. Thus, a context may be
thought of as a process with a ``hole'' (written $\Box$) in it. The
application of a context $M$ to a process $P$, written $M[P]$, is
tantamount to filling the hole in $M$ with $P$. In this paper we do
not need the full weight of this theory, but do make use of the notion
of context in the proof the main theorem. 

\begin{mathpar}
  \inferrule* [lab=summation] {} {{M_{M},M_{N}} \bc \Box \;|\; x.M_{A} \;|\; M_{M}+M_{N}}
  \and
  \inferrule* [lab=agent] {} {{M_{A}} \bc (\vec{x})M_{P} \;| \; \clift{P_0,\ldots,M_{P},\ldots,P_N}}
  \and \\
  \inferrule* [lab=process] {} {{M_{P}} \bc M_{N} \;| \;P|M_{P} }
\end{mathpar} 

\begin{mathpar}
  \inferrule* [lab=sychronization] {} {M_{N} \bc \Box \;|\; x?M_{F} \;|\; x!M_{C}}
  \and
  \inferrule* [lab=abstraction] {} {{M_{F}} \bc (x)M_{P} }
  \and
  \inferrule* [lab=concretion] {} {{M_{C}} \bc \langle M_{P} \rangle }
  \and \\
  \inferrule* [lab=process] {} {{M_{P}} \bc M_{N} \;| \;P|M_{P} }
\end{mathpar}

\begin{definition}[contextual application] Given a context $M$, and
  process $P$, we define the \emph{contextual application}, $M[P] :=
  M\{P/\Box\}$. That is, the contextual application of M to P is the
  substitution of $P$ for $\Box$ in $M$.
\end{definition}

$\meaningof{-} : L \to \mathcal{P}(\pi)$

\begin{mathpar}
  \inferrule* [lab=collection] {} {\meaningof{true} = \pi, \and \meaningof{~E} = \pi \setminus \meaningof{E}, \and \meaningof{E_{1} \& E_{2}} = \meaningof{E_{1}} \cap \meaningof{E_{2}}}
\end{mathpar}

\begin{mathpar}
  \inferrule* [lab=structure] {} {\meaningof{0} = \{ P \in \pi | P \equiv 0 \}, \and \\ \meaningof{E_1 | E_2} = \{ P \in \pi | P \equiv P_{1} | P_{2}, P_{1} \in \meaningof{E_{1}}, P_{2} \in \meaningof{E_2}\} }
\end{mathpar}

\begin{mathpar}
 \inferrule* [lab=behavior] {} {\meaningof{\langle a?b \rangle E} = \{ P \in \pi | P \equiv Q | u?(y)P', \\ \and \\\\ \and \\ \;\;\; u \in \meaningof{a}, \forall z.P'\{z/y\} \in \meaningof{E\{z/b\}}\}, \and \\ \meaningof{a!E} = \{ P \in \pi | P \equiv Q | x!\langle P' \rangle, x \in \meaningof{a} P' \in \meaningof{E}\} }
\end{mathpar}

\begin{mathpar}
 \inferrule* [lab=nominal] {} {\meaningof{\quotep{E}} = \{ \quotep{P} \in \quotep{\pi} | P \in \meaningof{E} \}, \and \meaningof{\quotep{P}} = \{ \quotep{Q} \in \quotep{\pi} | P \equiv Q \} \and \\ \meaningof{@\quotep{E}} = \{ P \in \pi | P \equiv @x, x \in \meaningof{E} \}}
\end{mathpar}

\begin{eqnarray*}
  \\
  \meaningof{-} : TS \to ST
\end{eqnarray*}

\begin{eqnarray*}
  \\
  L : TS \to ST
\end{eqnarray*}

\begin{eqnarray*}
  \\
  P \models E \iff P \in \meaningof{E}
\end{eqnarray*}

\begin{eqnarray*}
  P \approx_{L} Q \iff \forall E \in L. P \models E \iff Q \models E
\end{eqnarray*}

\begin{eqnarray*}
  P \approx_{K} Q
\end{eqnarray*}

\begin{eqnarray*}
  P \approx Q
\end{eqnarray*}

$\approx_{K} = \approx = \approx_{L}$

\subsubsection{Contextual duality}

Note that contexts extend the quotation operation to a family of
operations from processes to names. Given a context, $M$, we can
define a \emph{nominal context}, $\quotep{M}$ by $\quotep{M}[P] :=
\quotep{M[P]}$. To foreshadow what is to come we observe that these
operations enjoy a duality with processes very much like the duality
between vectors and maps from vectors to scalars.

Further, because the calculus is essentially higher-order, we have a
correspondence between contexts and processes. More specifically,
given a name $x$ and a context $M$ we can construct $M^{*}_{x}$ such
that 

\begin{mathpar}
  M^{*}_{x} | \lift{x}{P} \red M[P]
\end{mathpar}

namely,

\begin{mathpar}
  M^{*}_{x} := x?(u).M[\dropn{u}]
\end{mathpar}

The dependence of $M^{*}_{x}$ on a name makes it an abstraction, 

\begin{mathpar}
  M^{*} := (x)x?(u).M[\dropn{u}]
\end{mathpar}

\subsection{Additional notation}

It will sometimes be convenient to denote the process a name
quotes. We already have the notation $x = \quotep{P}$, but it will be
convenient to introduce an alternate notation, $\procn{x}$, when we
want to emphasize the connection to the use of the name. Note that, by
virtue of name equivalence, $\quotep{\procn{x}} \nameeq x$; so, the
notation is consistent with previous definitions.

Further, because names have structure it is possible to effect
substitutions on the basis of that structure. This means we need to
upgrade our notation for substitutions, which we accomplish by
adapting comprehension notation. Thus,

\begin{mathpar}
  P\{ y / x : x \in S \}
\end{mathpar}

is interpreted to mean the process derived from P by replacing (in a
capture-avoiding manner) each occurrence of $x$ in $S$ by $y$. For example,

\begin{mathpar}
  P\{ \quotep{\procn{x}|\procn{x}} / x : x \in \freenames{P} \}
\end{mathpar}

will replace each (occurrence) of a free name $x$ in $P$ by
$\quotep{\procn{x}|\procn{x}}$.

Also, we will avail ourselves of the notation $x^{L}$ and $x^{R}$ to
denote injections of a name into disjoint copies of the name
space. There are numerous ways to accomplish this. One example can be
found in \cite{MeredithR05}. This notation overloads to vectors of
names: $\vec{x}^{\pi} := (x_{i}^{\pi} \; : \; 0 \leq i < |\vec{x}| )$ where $\pi \in \{L,R\}$.

We also use $P^{\Box} := P|\Box$.

In \cite{MeredithR05} an interpretation of the new operator is
given. It turns out that there are several possible interpretations
all enjoying the requisite algebraic properties of the operator (see
\cite{milner91polyadicpi}). We will therefore make liberal use of
$(\nu\; \vec{x})P$.

% subsection the_syntax_and_semantics_of_the_notation_system (end)   

\input{qm2pi.qmops} 

\input{qm2pi.sterngerlach} 

\input{qm2pi.metric} 

% section concurrent_process_calculi (end)

%\input{qm2pi.proofsketch}

% section proof sketch (end)

%\input{qm2pi.slviaknots} 

% section spatial logic via knots (end)

\input{qm2pi.conclusion}

% section conclusion (end)

%\input{qm2pi.dtcodes} 

% section wiring algorithm (end)

\input{qm2pi.ack} 

% section acknowledgments (end)

\newpage


\bibliographystyle{plain}   
\bibliography{../../biblios/main.bib}

\input{qm2pi.rhodetails}

\end{document}

 

%\documentclass[12pt]{llncs}
%\documentclass{jktr}

\usepackage[pdftex]{hyperref}                   
\usepackage {listings}
\usepackage {mathpartir}
\usepackage{bcprules}
%\usepackage{listings}
                       
\usepackage{graphicx} 
%\usepackage[margins=2.5cm,nohead,nofoot]{geometry}
%\usepackage{geometry}
\usepackage{amsfonts}
\usepackage{amstext}
\usepackage{latexsym}
\usepackage{amssymb}
\usepackage{color}


%\include{myPreamble}
\include{qm2pi.local} 

%\ifpdf
%\usepackage[pdftex]{graphicx}
%\else
%\usepackage{graphicx}
%\fi

 % \ifpdf
%  \usepackage{pdfsync}
%  \if


%\title{Brief Article}
%\author{David F. Snyder}
%\author{L.G. Meredith}

%\address{Dept. of Math., Texas State University--San Marcos, San Marcos, TX 78666}
       
\pagestyle{empty}


\begin{document}

\lstset{language=[Objective]Caml,frame=shadowbox}

\input{qm2pi.front}

% section front matter (end)

\input{qm2pi.intro} 
 
% section introduction (end)

% \input{qm2pi.knotations} 

% section notation (end)

\input{qm2pi.process.calculi} 

% section concurrent_process_calculi_and_spatial_logics_ (end)
    
%\input{qm2pi.knots2pi} 

%\input{qm2pi.trefoil} 

%\input{qm2pi.mainthm} 

% subsection basic_interpretation (end)

%\input{qm2pi.rho.presentation} 
\subsection{The syntax and semantics of the notation system}\label{sub:the_syntax_and_semantics_of_the_notation_system} % (fold)

We now summarize a technical presentation of the calculus that
embodies our theory of dynamics. The typical presentation of such a
calculus follows the style of giving generators and relations on
them. The grammar, below, describing term constructors, freely
generates the set of processes, $\Proc$. This set is then quotiented
by a relation known as structural congruence and it is over this set
that the notion of dynamics is expressed. This presentation is
essentially that of \cite{MeredithR05} with the addition of
polyadicity and summation. For readability we have relegated some of
the technical subtleties to an appendix.

\subsubsection{Process grammar}\label{subsub:process_grammar}

\begin{mathpar}
  \inferrule* [lab=synchronization] {} {{M} \bc \pzero \;|\; x?F \;|\; x!C }
  \and
  \inferrule* [lab=abstraction] {} {{F} \bc (x)P}
  \and
  \inferrule* [lab=concretion] {} {{C} \bc \langle Q \rangle}
  \and
  \inferrule* [lab=process] {} {{P,Q} \bc M \;| \;P|Q \;|\; @{x}}
  \and
  \inferrule* [lab=name] {} {{x} \bc \quotep{P}}
\end{mathpar} 

Note that $\vec{x}$ (resp. $\vec{P}$) denotes a vector of names
(resp. processes) of length $|\vec{x}|$ (resp. $|\vec{P}|$). We adopt
the following useful abbreviations.

\begin{mathpar}
   x?(\vec{y}).P := x.(\vec{y})P \and  x\clift{\vec{P}} := x.\clift{\vec{P}}
   \and x!(y) := \lift{x}{\dropn{y}}
   \and \Pi_{i=0}^{n-1}P_i := P_0 | \ldots | P_{n-1}
\end{mathpar}

\subsubsection{Structural congruence}

\paragraph{Free and bound names and alpha-equivalence.} At the
core of structural equivalence is alpha-equivalence which identifies
process that are the same up to a change of variable. Formally, we
recognize the distinction between free and bound names. The free names
of a process, $\freenames{P}$, may be calculated recursively as
follows:

\begin{mathpar}
\freenames{\pzero} := \emptyset
  \and \\
  \freenames{x?(y).P} := \{ x \} \cup (\freenames{P} \setminus \{ y \})
  \and 
  \freenames{x!\langle P \rangle} := \{ x \} \cup \{ P \} 
  \and \\
  \freenames{P|Q} := \freenames{P} \cup \freenames{Q}
  \and \\
  \freenames{@{x}} := \{ x \}
\end{mathpar}

$\pi$
$\quotep{\pi}$

$\freenames{-} : \pi \to \mathcal{P}(\quotep{\pi})$

\begin{eqnarray*}
  \freenames{\pzero} & := & \emptyset \\
  \freenames{x?(y).P} & := & \{ x \} \cup (\freenames{P} \setminus \{ y \}) \\
  \freenames{x!\langle P \rangle} & := & \{ x \} \cup \{ P \} \\
  \freenames{P|Q} & := & \freenames{P} \cup \freenames{Q} \\
  \freenames{\dropn{x}} & := & \{ x \}
\end{eqnarray*}

The bound names of a process, $\boundnames{P}$, are those names occurring in $P$
that are not free. For example, in $x?(y).0$, the name $x$ is free, while $y$ is bound.

\begin{mathpar}
  \inferrule* [lab=monoidal-laws] {} { P|Q \equiv Q|P \and P|0 \equiv P \and P|(Q|R) \equiv (P|Q)|R }
\end{mathpar}

\begin{mathpar}
  \inferrule* [lab=alpha-equivalence] {} { (x)P \equiv (y)P\{y/x\} \and y \not\in \freenames{P} }
\end{mathpar}

\begin{definition}
Then two processes, $P,Q$, are alpha-equivalent if $P = Q\{\vec{y}/\vec{x}\}$ for
some $\vec{x} \in \boundnames{Q},\vec{y} \in \boundnames{P}$, where $Q\{\vec{y}/\vec{x}\}$
denotes the capture-avoiding substitution of $\vec{y}$ for $\vec{x}$ in $Q$.
\end{definition}

\begin{definition}
  The {\em structural congruence} \cite{SangiorgiWalker} , $\equiv$,
  between processes is the least congruence containing
  alpha-equivalence, satisfying the abelian monoid laws
  (associativity, commutativity and $\pzero$ as identity) for parallel
  composition $|$ and for summation $+$.
\end{definition}

\subsection{Name equivalence}

We take name equivalence, written $\nameeq$, to be the smallest
equivalence relation generated by the following rules.

\begin{mathpar}
\inferrule*[lab=Quote-drop]
{ }
{ \quotep{@{x}} \nameeq x }

\inferrule*[lab=Struct-equiv]
{ P \scong Q }
{ \quotep{P} \nameeq \quotep{Q} }
\end{mathpar}

The astute reader will have noticed that the mutual recursion of names
and processes imposes a mutual recursion on alpha-equivalence and
structural equivalence via name-equivalence. Fortunately, all of this
works out pleasantly and we may calculate in the natural way, free of
concern. The reader interested in the details is referred to the
appendix \ref{appendix:rho_details}.

\subsection{Substitution}

We use $\Proc$ for the set of processes, $\QProc$ for the set of
names, and $\id{\{}\vec{y} / \vec{x} \id{\}}$ to denote partial maps,
$s : \QProc \rightarrow \QProc$. A map, $s$ lifts, uniquely, to a map
on process terms, $\widehat{s} : \Proc \rightarrow \Proc$ by the
following equations.

\begin{mathpar}
  (0) \psubstp{Q}{P} := 0 \\
  (R \juxtap S) \psubstp{Q}{P}
  :=    
  (R)\psubstp{Q}{P} \juxtap (S) \psubstp{Q}{P} \\
  (x?(y).R) \psubstp{Q}{P}    
  :=    
  (x)\substp{Q}{P} (z)\concat( (R \psubstn{z}{y}) \psubstp{Q}{P} ) \\
  (\lift{x}{R}) \psubstp{Q}{P}  
  :=
  \lift{(x)\substp{Q}{P}}{ R \psubstp{Q}{P} } \\
%   (\dropn{x})  \psubstp{Q}{P}       
%   := 
%   \left\{ 
%     \begin{array}{ccc} 
%       \dropn{\quotep{Q}} & & x \nameeq \quotep{P} \\
%       \dropn{x} & & otherwise \\
%     \end{array}
%   \right. 
  (\dropn{x})  \psubstp{Q}{P}       
  := 
  \left\{ 
    \begin{array}{ccc} 
      Q & & x \nameeq \quotep{P} \\
      \dropn{x} & & otherwise \\
    \end{array}
  \right.
\end{mathpar}
 

where

\begin{eqnarray}
  (x)\id{\{} \lpquote Q \rpquote / \lpquote P \rpquote \id{\}}            = 
  \left\{ 
    \begin{array}{ccc}
      \lpquote Q \rpquote & & x \nameeq \lpquote P \rpquote \\
      x & & otherwise \\
    \end{array}
  \right. \nonumber
\end{eqnarray}

and $z$ is chosen distinct from $\quotep{P}$, $\quotep{Q}$, the free
names in $Q$, and all the names in $R$. Our $\alpha$-equivalence will
be built in the standard way from this substitution.

\begin{remark}\label{rem:no_self_referential_names}
  One consequence of these definitions is that $\forall P. \quotep{P}
  \not\in \freenames{P}$.
\end{remark}

\subsection{ Dynamic quote: an example }

Anticipating something of what's to come, consider applying the
substitution, $\widehat{\id{\{}u / z \id{\}}}$, to the following pair
of processes, $\lift{w}{y!(z)}$ and $w[ \lpquote y!(z) \rpquote ]$.

\begin{eqnarray}
	\lift{w}{y!(z)}\widehat{\id{\{}u / z \id{\}}}
		& = &
		\lift{w}{y!(u)} \nonumber\\
	w[ \lpquote y!(z) \rpquote ] \widehat{ \id{\{}u / z \id{\}} }
		& = &
		w[ \lpquote y!(z) \rpquote ] \nonumber
\end{eqnarray}

Because the body of the process between quotes is impervious to
substitution, we get radically different answers. In fact, by
examining the first process in an input context,
e.g. $x?(z).\lift{w}{y!(z)}$, we see that the process under the lift
operator may be shaped by prefixed inputs binding a name inside it. In
this sense, the lift operator will be seen as a way to dynamically
construct processes before reifying them as names.

Finally equipped with these standard features we can present the
dynamics of the calculus.

\subsubsection{Operational semantics} 

Finally, we introduce the computational dynamics. What marks these
algebras as distinct from other more traditionally studied algebraic
structures, e.g. vector spaces or polynomial rings, is the manner in
which dynamics is captured. In traditional structures, dynamics is typically
expressed through morphisms between such structures, as in linear maps
between vector spaces or morphisms between rings. In algebras
associated with the semantics of computation, the dynamics is
expressed as part of the algebraic structure itself, through a
reduction reduction relation typically denoted by $\red$. Below, we
give a recursive presentation of this relation for the calculus used
in the encoding.

$\red \subseteq \pi \times \pi$
$\red : \pi \to \mathcal{P}(\pi)$

\begin{mathpar}
  \inferrule* [lab=Comm] { \textsf{match}( x_{src}, x_{trgt} ) } { x_{trgt}?(y)P \; | \; x_{src}!\langle {Q} \rangle \red P\{\quotep{Q}/y}\} }
  \and \\
  \inferrule* [lab=Par] {{P} \red {P}'} {{{P} | {Q}} \red {{P}' | {Q}}}
  \and
  \inferrule* [lab=Equiv]{{{P} \scong {P}'} \andalso {{P}' \red {Q}'} \andalso {{Q}' \scong {Q}}}{{P} \red {Q}}
\end{mathpar}

\begin{eqnarray*}
  match_{\equiv} (\quotep{P},\quotep{Q}) & := & P \equiv Q \\
  match_{\dagger}(\quotep{P},\quotep{Q}) & := & \forall R. P|Q \red^{*} R => R \red^{*} 0 \\
  match_{K}(\quotep{P},\quotep{Q}) & := & K \mbox{ for some context } K
\end{eqnarray*}

$u?(x)P | u!\langle Q \rangle \red P\{\quotep{Q}/x\}$

%We write $\wred$ for $\red^*$, and $P\red$ if $\exists Q $ such that $ P \red Q$.
We write $P\red$ if $\exists Q $ such that $ P \red Q$ and $P\not\red$, otherwise.

\section{Replication}

As mentioned before, it is known that replication (and hence
recursion) can be implemented in a higher-order process algebra
\cite{SangiorgiWalker}. As our first example of calculation with the
machinery thus far presented we give the construction explicitly in
the {\rhoc}.

\begin{eqnarray}
	D_{x} & := & \prefix{x}{y}{(\binpar{\outputp{x}{y}}{@{y}})} \nonumber\\
	\bangp_{x}{P} & := & \binpar{{x}!\langle{\binpar{D_{x}}{P}}\rangle}{D_{x}} \nonumber
\end{eqnarray}

\begin{eqnarray}
	\bangp_{x}{P} & & \nonumber\\
	=
	& {x}!\langle{(\prefix{x}{y}{(\outputp{x}{y} | @{y})) | P}}\rangle 
	      | \prefix{x}{y}{(\outputp{x}{y} | @{y})} & \nonumber\\
	\red
	& (\outputp{x}{y} | @{y})\substn{\quotep{(\prefix{x}{y}{(@{y} | \outputp{x}{y})) | P}}}{y} & \nonumber\\
	=
	& \outputp{x}{\quotep{(\prefix{x}{y}{(\outputp{x}{y} | @{y})) | P}}}
	  | {(\prefix{x}{y}{(\outputp{x}{y} | @{y})) | P}} & \nonumber\\
	\red
	& \ldots & \nonumber\\
	\red^*
	& P | P | \ldots & \nonumber
\end{eqnarray}

Of course, this encoding, as an implementation, runs away, unfolding
$\bangp{P}$ eagerly. A lazier and more implementable replication
operator, restricted to input-guarded processes, may be obtained as follows.

\begin{eqnarray}
\bangp{\prefix{u}{v}{P}} 
	:= 
	\binpar{\lift{x}{\prefix{u}{v}{(\binpar{D(x)}{P})}}}{D(x)} \nonumber
\end{eqnarray}

\begin{remark}
  Note that the lazier definition still does not deal with summation
  or mixed summation (i.e. sums over input and output). The reader is
  invited to construct definitions of replication that deal with these
  features. 

  Further, the definitions are parameterized in a name, $x$. Can you,
  gentle reader, make a definition that eliminates this parameter and
  guarantees no accidental interaction between the replication
  machinery and the process being replicated -- i.e. no accidental
  sharing of names used by the process to get its work done and the
  name(s) used by the replication to effect copying. This latter
  revision of the definition of replication is crucial to obtaining
  the expected identity $!!P \sim !P$.
\end{remark}

\begin{remark}\label{rem:paradoxical_combinator}
  The reader familiar with the lambda calculus will have noticed the
  similarity between $D$ and the paradoxical combinator.

  [Ed. note: the existence of this seems to suggest we have to be more
  restrictive on the set of processes and names we admit if we are to
  support no-cloning.]
\end{remark}

\subsubsection{Bisimulation}

The computational dynamics gives rise to another kind of equivalence,
the equivalence of computational behavior. As previously mentioned
this is typically captured \emph{via} some form of bisimulation.

% The notion we use in this paper is weak barbed bisimulation
% \cite{milner91polyadicpi}.

The notion we use in this paper is derived from weak barbed
bisimulation \cite{milner91polyadicpi}. 

\begin{definition}
An \emph{observation relation}, $\downarrow_{\mathcal N}$, over a set
of names, $\mathcal N$, is the smallest relation satisfying the rules
below.

\infrule[Out-barb]{y \in {\mathcal N}, \; x \nameeq y}
		  {\outputp{x}{v} \downarrow_{\mathcal N} x}
\infrule[Par-barb]{\mbox{$P\downarrow_{\mathcal N} x$ or $Q\downarrow_{\mathcal N} x$}}
		  {\binpar{P}{Q} \downarrow_{\mathcal N} x}

We write $P \Downarrow_{\mathcal N} x$ if there is $Q$ such that 
$P \wred Q$ and $Q \downarrow_{\mathcal N} x$.
\end{definition}

\begin{definition}
%\label{def.bbisim}
An  ${\mathcal N}$-\emph{barbed bisimulation} over a set of names, ${\mathcal N}$, is a symmetric binary relation 
${\mathcal S}_{\mathcal N}$ between agents such that $P\rel{S}_{\mathcal N}Q$ implies:
\begin{enumerate}
\item If $P \red P'$ then $Q \wred Q'$ and $P'\rel{S}_{\mathcal N} Q'$.
\item If $P\downarrow_{\mathcal N} x$, then $Q\Downarrow_{\mathcal N} x$.
\end{enumerate}
$P$ is ${\mathcal N}$-barbed bisimilar to $Q$, written
$P \wbbisim_{\mathcal N} Q$, if $P \rel{S}_{\mathcal N} Q$ for some ${\mathcal N}$-barbed bisimulation ${\mathcal S}_{\mathcal N}$.
\end{definition}

$\mathcal{R} \subseteq \pi \times \pi$

$P \mathcal{R} Q => \forall P'. P \red P' \Rightarrow \exists Q'. Q \red Q', P' \mathcal{R} Q'$

$P \vdash x \Rightarrow Q \vdash x$

\begin{mathpar}
  \inferrule*[lab=Out-barb]{x \nameeq y}{{y}!\langle{Q}\rangle \vdash x}
  \and
  \inferrule*[lab=Par-barb]{\mbox{$P\vdash x$ or $Q\vdash x$}}{\binpar{P}{Q} \vdash x}
\end{mathpar}

\subsubsection{Contexts}

One of the principle advantages of computational calculi like the
$\pi$-calculus is a well-defined notion of context,
contextual-equivalence and a correlation between
contextual-equivalence and notions of bisimulation. The notion of
context allows the decomposition of a process into (sub-)process and
its syntactic environment, its context. Thus, a context may be
thought of as a process with a ``hole'' (written $\Box$) in it. The
application of a context $M$ to a process $P$, written $M[P]$, is
tantamount to filling the hole in $M$ with $P$. In this paper we do
not need the full weight of this theory, but do make use of the notion
of context in the proof the main theorem. 

\begin{mathpar}
  \inferrule* [lab=summation] {} {{M_{M},M_{N}} \bc \Box \;|\; x.M_{A} \;|\; M_{M}+M_{N}}
  \and
  \inferrule* [lab=agent] {} {{M_{A}} \bc (\vec{x})M_{P} \;| \; \clift{P_0,\ldots,M_{P},\ldots,P_N}}
  \and \\
  \inferrule* [lab=process] {} {{M_{P}} \bc M_{N} \;| \;P|M_{P} }
\end{mathpar} 

\begin{mathpar}
  \inferrule* [lab=sychronization] {} {M_{N} \bc \Box \;|\; x?M_{F} \;|\; x!M_{C}}
  \and
  \inferrule* [lab=abstraction] {} {{M_{F}} \bc (x)M_{P} }
  \and
  \inferrule* [lab=concretion] {} {{M_{C}} \bc \langle M_{P} \rangle }
  \and \\
  \inferrule* [lab=process] {} {{M_{P}} \bc M_{N} \;| \;P|M_{P} }
\end{mathpar}

\begin{definition}[contextual application] Given a context $M$, and
  process $P$, we define the \emph{contextual application}, $M[P] :=
  M\{P/\Box\}$. That is, the contextual application of M to P is the
  substitution of $P$ for $\Box$ in $M$.
\end{definition}

$\meaningof{-} : L \to \mathcal{P}(\pi)$

\begin{mathpar}
  \inferrule* [lab=collection] {} {\meaningof{true} = \pi, \and \meaningof{~E} = \pi \setminus \meaningof{E}, \and \meaningof{E_{1} \& E_{2}} = \meaningof{E_{1}} \cap \meaningof{E_{2}}}
\end{mathpar}

\begin{mathpar}
  \inferrule* [lab=structure] {} {\meaningof{0} = \{ P \in \pi | P \equiv 0 \}, \and \\ \meaningof{E_1 | E_2} = \{ P \in \pi | P \equiv P_{1} | P_{2}, P_{1} \in \meaningof{E_{1}}, P_{2} \in \meaningof{E_2}\} }
\end{mathpar}

\begin{mathpar}
 \inferrule* [lab=behavior] {} {\meaningof{\langle a?b \rangle E} = \{ P \in \pi | P \equiv Q | u?(y)P', \\ \and \\\\ \and \\ \;\;\; u \in \meaningof{a}, \forall z.P'\{z/y\} \in \meaningof{E\{z/b\}}\}, \and \\ \meaningof{a!E} = \{ P \in \pi | P \equiv Q | x!\langle P' \rangle, x \in \meaningof{a} P' \in \meaningof{E}\} }
\end{mathpar}

\begin{mathpar}
 \inferrule* [lab=nominal] {} {\meaningof{\quotep{E}} = \{ \quotep{P} \in \quotep{\pi} | P \in \meaningof{E} \}, \and \meaningof{\quotep{P}} = \{ \quotep{Q} \in \quotep{\pi} | P \equiv Q \} \and \\ \meaningof{@\quotep{E}} = \{ P \in \pi | P \equiv @x, x \in \meaningof{E} \}}
\end{mathpar}

\begin{eqnarray*}
  \\
  \meaningof{-} : TS \to ST
\end{eqnarray*}

\begin{eqnarray*}
  \\
  L : TS \to ST
\end{eqnarray*}

\begin{eqnarray*}
  \\
  P \models E \iff P \in \meaningof{E}
\end{eqnarray*}

\begin{eqnarray*}
  P \approx_{L} Q \iff \forall E \in L. P \models E \iff Q \models E
\end{eqnarray*}

\begin{eqnarray*}
  P \approx_{K} Q
\end{eqnarray*}

\begin{eqnarray*}
  P \approx Q
\end{eqnarray*}

$\approx_{K} = \approx = \approx_{L}$

\subsubsection{Contextual duality}

Note that contexts extend the quotation operation to a family of
operations from processes to names. Given a context, $M$, we can
define a \emph{nominal context}, $\quotep{M}$ by $\quotep{M}[P] :=
\quotep{M[P]}$. To foreshadow what is to come we observe that these
operations enjoy a duality with processes very much like the duality
between vectors and maps from vectors to scalars.

Further, because the calculus is essentially higher-order, we have a
correspondence between contexts and processes. More specifically,
given a name $x$ and a context $M$ we can construct $M^{*}_{x}$ such
that 

\begin{mathpar}
  M^{*}_{x} | \lift{x}{P} \red M[P]
\end{mathpar}

namely,

\begin{mathpar}
  M^{*}_{x} := x?(u).M[\dropn{u}]
\end{mathpar}

The dependence of $M^{*}_{x}$ on a name makes it an abstraction, 

\begin{mathpar}
  M^{*} := (x)x?(u).M[\dropn{u}]
\end{mathpar}

\subsection{Additional notation}

It will sometimes be convenient to denote the process a name
quotes. We already have the notation $x = \quotep{P}$, but it will be
convenient to introduce an alternate notation, $\procn{x}$, when we
want to emphasize the connection to the use of the name. Note that, by
virtue of name equivalence, $\quotep{\procn{x}} \nameeq x$; so, the
notation is consistent with previous definitions.

Further, because names have structure it is possible to effect
substitutions on the basis of that structure. This means we need to
upgrade our notation for substitutions, which we accomplish by
adapting comprehension notation. Thus,

\begin{mathpar}
  P\{ y / x : x \in S \}
\end{mathpar}

is interpreted to mean the process derived from P by replacing (in a
capture-avoiding manner) each occurrence of $x$ in $S$ by $y$. For example,

\begin{mathpar}
  P\{ \quotep{\procn{x}|\procn{x}} / x : x \in \freenames{P} \}
\end{mathpar}

will replace each (occurrence) of a free name $x$ in $P$ by
$\quotep{\procn{x}|\procn{x}}$.

Also, we will avail ourselves of the notation $x^{L}$ and $x^{R}$ to
denote injections of a name into disjoint copies of the name
space. There are numerous ways to accomplish this. One example can be
found in \cite{MeredithR05}. This notation overloads to vectors of
names: $\vec{x}^{\pi} := (x_{i}^{\pi} \; : \; 0 \leq i < |\vec{x}| )$ where $\pi \in \{L,R\}$.

We also use $P^{\Box} := P|\Box$.

In \cite{MeredithR05} an interpretation of the new operator is
given. It turns out that there are several possible interpretations
all enjoying the requisite algebraic properties of the operator (see
\cite{milner91polyadicpi}). We will therefore make liberal use of
$(\nu\; \vec{x})P$.

% subsection the_syntax_and_semantics_of_the_notation_system (end)   

\input{qm2pi.qmops} 

\input{qm2pi.sterngerlach} 

\input{qm2pi.metric} 

% section concurrent_process_calculi (end)

%\input{qm2pi.proofsketch}

% section proof sketch (end)

%\input{qm2pi.slviaknots} 

% section spatial logic via knots (end)

\input{qm2pi.conclusion}

% section conclusion (end)

%\input{qm2pi.dtcodes} 

% section wiring algorithm (end)

\input{qm2pi.ack} 

% section acknowledgments (end)

\newpage


\bibliographystyle{plain}   
\bibliography{../../biblios/main.bib}

\input{qm2pi.rhodetails}

\end{document}

 

% subsection basic_interpretation (end)

%\input{qm2pi.rho.presentation} 
\subsection{The syntax and semantics of the notation system}\label{sub:the_syntax_and_semantics_of_the_notation_system} % (fold)

We now summarize a technical presentation of the calculus that
embodies our theory of dynamics. The typical presentation of such a
calculus follows the style of giving generators and relations on
them. The grammar, below, describing term constructors, freely
generates the set of processes, $\Proc$. This set is then quotiented
by a relation known as structural congruence and it is over this set
that the notion of dynamics is expressed. This presentation is
essentially that of \cite{MeredithR05} with the addition of
polyadicity and summation. For readability we have relegated some of
the technical subtleties to an appendix.

\subsubsection{Process grammar}\label{subsub:process_grammar}

\begin{mathpar}
  \inferrule* [lab=synchronization] {} {{M} \bc \pzero \;|\; x?F \;|\; x!C }
  \and
  \inferrule* [lab=abstraction] {} {{F} \bc (x)P}
  \and
  \inferrule* [lab=concretion] {} {{C} \bc \langle Q \rangle}
  \and
  \inferrule* [lab=process] {} {{P,Q} \bc M \;| \;P|Q \;|\; @{x}}
  \and
  \inferrule* [lab=name] {} {{x} \bc \quotep{P}}
\end{mathpar} 

Note that $\vec{x}$ (resp. $\vec{P}$) denotes a vector of names
(resp. processes) of length $|\vec{x}|$ (resp. $|\vec{P}|$). We adopt
the following useful abbreviations.

\begin{mathpar}
   x?(\vec{y}).P := x.(\vec{y})P \and  x\clift{\vec{P}} := x.\clift{\vec{P}}
   \and x!(y) := \lift{x}{\dropn{y}}
   \and \Pi_{i=0}^{n-1}P_i := P_0 | \ldots | P_{n-1}
\end{mathpar}

\subsubsection{Structural congruence}

\paragraph{Free and bound names and alpha-equivalence.} At the
core of structural equivalence is alpha-equivalence which identifies
process that are the same up to a change of variable. Formally, we
recognize the distinction between free and bound names. The free names
of a process, $\freenames{P}$, may be calculated recursively as
follows:

\begin{mathpar}
\freenames{\pzero} := \emptyset
  \and \\
  \freenames{x?(y).P} := \{ x \} \cup (\freenames{P} \setminus \{ y \})
  \and 
  \freenames{x!\langle P \rangle} := \{ x \} \cup \{ P \} 
  \and \\
  \freenames{P|Q} := \freenames{P} \cup \freenames{Q}
  \and \\
  \freenames{@{x}} := \{ x \}
\end{mathpar}

$\pi$
$\quotep{\pi}$

$\freenames{-} : \pi \to \mathcal{P}(\quotep{\pi})$

\begin{eqnarray*}
  \freenames{\pzero} & := & \emptyset \\
  \freenames{x?(y).P} & := & \{ x \} \cup (\freenames{P} \setminus \{ y \}) \\
  \freenames{x!\langle P \rangle} & := & \{ x \} \cup \{ P \} \\
  \freenames{P|Q} & := & \freenames{P} \cup \freenames{Q} \\
  \freenames{\dropn{x}} & := & \{ x \}
\end{eqnarray*}

The bound names of a process, $\boundnames{P}$, are those names occurring in $P$
that are not free. For example, in $x?(y).0$, the name $x$ is free, while $y$ is bound.

\begin{mathpar}
  \inferrule* [lab=monoidal-laws] {} { P|Q \equiv Q|P \and P|0 \equiv P \and P|(Q|R) \equiv (P|Q)|R }
\end{mathpar}

\begin{mathpar}
  \inferrule* [lab=alpha-equivalence] {} { (x)P \equiv (y)P\{y/x\} \and y \not\in \freenames{P} }
\end{mathpar}

\begin{definition}
Then two processes, $P,Q$, are alpha-equivalent if $P = Q\{\vec{y}/\vec{x}\}$ for
some $\vec{x} \in \boundnames{Q},\vec{y} \in \boundnames{P}$, where $Q\{\vec{y}/\vec{x}\}$
denotes the capture-avoiding substitution of $\vec{y}$ for $\vec{x}$ in $Q$.
\end{definition}

\begin{definition}
  The {\em structural congruence} \cite{SangiorgiWalker} , $\equiv$,
  between processes is the least congruence containing
  alpha-equivalence, satisfying the abelian monoid laws
  (associativity, commutativity and $\pzero$ as identity) for parallel
  composition $|$ and for summation $+$.
\end{definition}

\subsection{Name equivalence}

We take name equivalence, written $\nameeq$, to be the smallest
equivalence relation generated by the following rules.

\begin{mathpar}
\inferrule*[lab=Quote-drop]
{ }
{ \quotep{@{x}} \nameeq x }

\inferrule*[lab=Struct-equiv]
{ P \scong Q }
{ \quotep{P} \nameeq \quotep{Q} }
\end{mathpar}

The astute reader will have noticed that the mutual recursion of names
and processes imposes a mutual recursion on alpha-equivalence and
structural equivalence via name-equivalence. Fortunately, all of this
works out pleasantly and we may calculate in the natural way, free of
concern. The reader interested in the details is referred to the
appendix \ref{appendix:rho_details}.

\subsection{Substitution}

We use $\Proc$ for the set of processes, $\QProc$ for the set of
names, and $\id{\{}\vec{y} / \vec{x} \id{\}}$ to denote partial maps,
$s : \QProc \rightarrow \QProc$. A map, $s$ lifts, uniquely, to a map
on process terms, $\widehat{s} : \Proc \rightarrow \Proc$ by the
following equations.

\begin{mathpar}
  (0) \psubstp{Q}{P} := 0 \\
  (R \juxtap S) \psubstp{Q}{P}
  :=    
  (R)\psubstp{Q}{P} \juxtap (S) \psubstp{Q}{P} \\
  (x?(y).R) \psubstp{Q}{P}    
  :=    
  (x)\substp{Q}{P} (z)\concat( (R \psubstn{z}{y}) \psubstp{Q}{P} ) \\
  (\lift{x}{R}) \psubstp{Q}{P}  
  :=
  \lift{(x)\substp{Q}{P}}{ R \psubstp{Q}{P} } \\
%   (\dropn{x})  \psubstp{Q}{P}       
%   := 
%   \left\{ 
%     \begin{array}{ccc} 
%       \dropn{\quotep{Q}} & & x \nameeq \quotep{P} \\
%       \dropn{x} & & otherwise \\
%     \end{array}
%   \right. 
  (\dropn{x})  \psubstp{Q}{P}       
  := 
  \left\{ 
    \begin{array}{ccc} 
      Q & & x \nameeq \quotep{P} \\
      \dropn{x} & & otherwise \\
    \end{array}
  \right.
\end{mathpar}
 

where

\begin{eqnarray}
  (x)\id{\{} \lpquote Q \rpquote / \lpquote P \rpquote \id{\}}            = 
  \left\{ 
    \begin{array}{ccc}
      \lpquote Q \rpquote & & x \nameeq \lpquote P \rpquote \\
      x & & otherwise \\
    \end{array}
  \right. \nonumber
\end{eqnarray}

and $z$ is chosen distinct from $\quotep{P}$, $\quotep{Q}$, the free
names in $Q$, and all the names in $R$. Our $\alpha$-equivalence will
be built in the standard way from this substitution.

\begin{remark}\label{rem:no_self_referential_names}
  One consequence of these definitions is that $\forall P. \quotep{P}
  \not\in \freenames{P}$.
\end{remark}

\subsection{ Dynamic quote: an example }

Anticipating something of what's to come, consider applying the
substitution, $\widehat{\id{\{}u / z \id{\}}}$, to the following pair
of processes, $\lift{w}{y!(z)}$ and $w[ \lpquote y!(z) \rpquote ]$.

\begin{eqnarray}
	\lift{w}{y!(z)}\widehat{\id{\{}u / z \id{\}}}
		& = &
		\lift{w}{y!(u)} \nonumber\\
	w[ \lpquote y!(z) \rpquote ] \widehat{ \id{\{}u / z \id{\}} }
		& = &
		w[ \lpquote y!(z) \rpquote ] \nonumber
\end{eqnarray}

Because the body of the process between quotes is impervious to
substitution, we get radically different answers. In fact, by
examining the first process in an input context,
e.g. $x?(z).\lift{w}{y!(z)}$, we see that the process under the lift
operator may be shaped by prefixed inputs binding a name inside it. In
this sense, the lift operator will be seen as a way to dynamically
construct processes before reifying them as names.

Finally equipped with these standard features we can present the
dynamics of the calculus.

\subsubsection{Operational semantics} 

Finally, we introduce the computational dynamics. What marks these
algebras as distinct from other more traditionally studied algebraic
structures, e.g. vector spaces or polynomial rings, is the manner in
which dynamics is captured. In traditional structures, dynamics is typically
expressed through morphisms between such structures, as in linear maps
between vector spaces or morphisms between rings. In algebras
associated with the semantics of computation, the dynamics is
expressed as part of the algebraic structure itself, through a
reduction reduction relation typically denoted by $\red$. Below, we
give a recursive presentation of this relation for the calculus used
in the encoding.

$\red \subseteq \pi \times \pi$
$\red : \pi \to \mathcal{P}(\pi)$

\begin{mathpar}
  \inferrule* [lab=Comm] { \textsf{match}( x_{src}, x_{trgt} ) } { x_{trgt}?(y)P \; | \; x_{src}!\langle {Q} \rangle \red P\{\quotep{Q}/y}\} }
  \and \\
  \inferrule* [lab=Par] {{P} \red {P}'} {{{P} | {Q}} \red {{P}' | {Q}}}
  \and
  \inferrule* [lab=Equiv]{{{P} \scong {P}'} \andalso {{P}' \red {Q}'} \andalso {{Q}' \scong {Q}}}{{P} \red {Q}}
\end{mathpar}

\begin{eqnarray*}
  match_{\equiv} (\quotep{P},\quotep{Q}) & := & P \equiv Q \\
  match_{\dagger}(\quotep{P},\quotep{Q}) & := & \forall R. P|Q \red^{*} R => R \red^{*} 0 \\
  match_{K}(\quotep{P},\quotep{Q}) & := & K \mbox{ for some context } K
\end{eqnarray*}

$u?(x)P | u!\langle Q \rangle \red P\{\quotep{Q}/x\}$

%We write $\wred$ for $\red^*$, and $P\red$ if $\exists Q $ such that $ P \red Q$.
We write $P\red$ if $\exists Q $ such that $ P \red Q$ and $P\not\red$, otherwise.

\section{Replication}

As mentioned before, it is known that replication (and hence
recursion) can be implemented in a higher-order process algebra
\cite{SangiorgiWalker}. As our first example of calculation with the
machinery thus far presented we give the construction explicitly in
the {\rhoc}.

\begin{eqnarray}
	D_{x} & := & \prefix{x}{y}{(\binpar{\outputp{x}{y}}{@{y}})} \nonumber\\
	\bangp_{x}{P} & := & \binpar{{x}!\langle{\binpar{D_{x}}{P}}\rangle}{D_{x}} \nonumber
\end{eqnarray}

\begin{eqnarray}
	\bangp_{x}{P} & & \nonumber\\
	=
	& {x}!\langle{(\prefix{x}{y}{(\outputp{x}{y} | @{y})) | P}}\rangle 
	      | \prefix{x}{y}{(\outputp{x}{y} | @{y})} & \nonumber\\
	\red
	& (\outputp{x}{y} | @{y})\substn{\quotep{(\prefix{x}{y}{(@{y} | \outputp{x}{y})) | P}}}{y} & \nonumber\\
	=
	& \outputp{x}{\quotep{(\prefix{x}{y}{(\outputp{x}{y} | @{y})) | P}}}
	  | {(\prefix{x}{y}{(\outputp{x}{y} | @{y})) | P}} & \nonumber\\
	\red
	& \ldots & \nonumber\\
	\red^*
	& P | P | \ldots & \nonumber
\end{eqnarray}

Of course, this encoding, as an implementation, runs away, unfolding
$\bangp{P}$ eagerly. A lazier and more implementable replication
operator, restricted to input-guarded processes, may be obtained as follows.

\begin{eqnarray}
\bangp{\prefix{u}{v}{P}} 
	:= 
	\binpar{\lift{x}{\prefix{u}{v}{(\binpar{D(x)}{P})}}}{D(x)} \nonumber
\end{eqnarray}

\begin{remark}
  Note that the lazier definition still does not deal with summation
  or mixed summation (i.e. sums over input and output). The reader is
  invited to construct definitions of replication that deal with these
  features. 

  Further, the definitions are parameterized in a name, $x$. Can you,
  gentle reader, make a definition that eliminates this parameter and
  guarantees no accidental interaction between the replication
  machinery and the process being replicated -- i.e. no accidental
  sharing of names used by the process to get its work done and the
  name(s) used by the replication to effect copying. This latter
  revision of the definition of replication is crucial to obtaining
  the expected identity $!!P \sim !P$.
\end{remark}

\begin{remark}\label{rem:paradoxical_combinator}
  The reader familiar with the lambda calculus will have noticed the
  similarity between $D$ and the paradoxical combinator.

  [Ed. note: the existence of this seems to suggest we have to be more
  restrictive on the set of processes and names we admit if we are to
  support no-cloning.]
\end{remark}

\subsubsection{Bisimulation}

The computational dynamics gives rise to another kind of equivalence,
the equivalence of computational behavior. As previously mentioned
this is typically captured \emph{via} some form of bisimulation.

% The notion we use in this paper is weak barbed bisimulation
% \cite{milner91polyadicpi}.

The notion we use in this paper is derived from weak barbed
bisimulation \cite{milner91polyadicpi}. 

\begin{definition}
An \emph{observation relation}, $\downarrow_{\mathcal N}$, over a set
of names, $\mathcal N$, is the smallest relation satisfying the rules
below.

\infrule[Out-barb]{y \in {\mathcal N}, \; x \nameeq y}
		  {\outputp{x}{v} \downarrow_{\mathcal N} x}
\infrule[Par-barb]{\mbox{$P\downarrow_{\mathcal N} x$ or $Q\downarrow_{\mathcal N} x$}}
		  {\binpar{P}{Q} \downarrow_{\mathcal N} x}

We write $P \Downarrow_{\mathcal N} x$ if there is $Q$ such that 
$P \wred Q$ and $Q \downarrow_{\mathcal N} x$.
\end{definition}

\begin{definition}
%\label{def.bbisim}
An  ${\mathcal N}$-\emph{barbed bisimulation} over a set of names, ${\mathcal N}$, is a symmetric binary relation 
${\mathcal S}_{\mathcal N}$ between agents such that $P\rel{S}_{\mathcal N}Q$ implies:
\begin{enumerate}
\item If $P \red P'$ then $Q \wred Q'$ and $P'\rel{S}_{\mathcal N} Q'$.
\item If $P\downarrow_{\mathcal N} x$, then $Q\Downarrow_{\mathcal N} x$.
\end{enumerate}
$P$ is ${\mathcal N}$-barbed bisimilar to $Q$, written
$P \wbbisim_{\mathcal N} Q$, if $P \rel{S}_{\mathcal N} Q$ for some ${\mathcal N}$-barbed bisimulation ${\mathcal S}_{\mathcal N}$.
\end{definition}

$\mathcal{R} \subseteq \pi \times \pi$

$P \mathcal{R} Q => \forall P'. P \red P' \Rightarrow \exists Q'. Q \red Q', P' \mathcal{R} Q'$

$P \vdash x \Rightarrow Q \vdash x$

\begin{mathpar}
  \inferrule*[lab=Out-barb]{x \nameeq y}{{y}!\langle{Q}\rangle \vdash x}
  \and
  \inferrule*[lab=Par-barb]{\mbox{$P\vdash x$ or $Q\vdash x$}}{\binpar{P}{Q} \vdash x}
\end{mathpar}

\subsubsection{Contexts}

One of the principle advantages of computational calculi like the
$\pi$-calculus is a well-defined notion of context,
contextual-equivalence and a correlation between
contextual-equivalence and notions of bisimulation. The notion of
context allows the decomposition of a process into (sub-)process and
its syntactic environment, its context. Thus, a context may be
thought of as a process with a ``hole'' (written $\Box$) in it. The
application of a context $M$ to a process $P$, written $M[P]$, is
tantamount to filling the hole in $M$ with $P$. In this paper we do
not need the full weight of this theory, but do make use of the notion
of context in the proof the main theorem. 

\begin{mathpar}
  \inferrule* [lab=summation] {} {{M_{M},M_{N}} \bc \Box \;|\; x.M_{A} \;|\; M_{M}+M_{N}}
  \and
  \inferrule* [lab=agent] {} {{M_{A}} \bc (\vec{x})M_{P} \;| \; \clift{P_0,\ldots,M_{P},\ldots,P_N}}
  \and \\
  \inferrule* [lab=process] {} {{M_{P}} \bc M_{N} \;| \;P|M_{P} }
\end{mathpar} 

\begin{mathpar}
  \inferrule* [lab=sychronization] {} {M_{N} \bc \Box \;|\; x?M_{F} \;|\; x!M_{C}}
  \and
  \inferrule* [lab=abstraction] {} {{M_{F}} \bc (x)M_{P} }
  \and
  \inferrule* [lab=concretion] {} {{M_{C}} \bc \langle M_{P} \rangle }
  \and \\
  \inferrule* [lab=process] {} {{M_{P}} \bc M_{N} \;| \;P|M_{P} }
\end{mathpar}

\begin{definition}[contextual application] Given a context $M$, and
  process $P$, we define the \emph{contextual application}, $M[P] :=
  M\{P/\Box\}$. That is, the contextual application of M to P is the
  substitution of $P$ for $\Box$ in $M$.
\end{definition}

$\meaningof{-} : L \to \mathcal{P}(\pi)$

\begin{mathpar}
  \inferrule* [lab=collection] {} {\meaningof{true} = \pi, \and \meaningof{~E} = \pi \setminus \meaningof{E}, \and \meaningof{E_{1} \& E_{2}} = \meaningof{E_{1}} \cap \meaningof{E_{2}}}
\end{mathpar}

\begin{mathpar}
  \inferrule* [lab=structure] {} {\meaningof{0} = \{ P \in \pi | P \equiv 0 \}, \and \\ \meaningof{E_1 | E_2} = \{ P \in \pi | P \equiv P_{1} | P_{2}, P_{1} \in \meaningof{E_{1}}, P_{2} \in \meaningof{E_2}\} }
\end{mathpar}

\begin{mathpar}
 \inferrule* [lab=behavior] {} {\meaningof{\langle a?b \rangle E} = \{ P \in \pi | P \equiv Q | u?(y)P', \\ \and \\\\ \and \\ \;\;\; u \in \meaningof{a}, \forall z.P'\{z/y\} \in \meaningof{E\{z/b\}}\}, \and \\ \meaningof{a!E} = \{ P \in \pi | P \equiv Q | x!\langle P' \rangle, x \in \meaningof{a} P' \in \meaningof{E}\} }
\end{mathpar}

\begin{mathpar}
 \inferrule* [lab=nominal] {} {\meaningof{\quotep{E}} = \{ \quotep{P} \in \quotep{\pi} | P \in \meaningof{E} \}, \and \meaningof{\quotep{P}} = \{ \quotep{Q} \in \quotep{\pi} | P \equiv Q \} \and \\ \meaningof{@\quotep{E}} = \{ P \in \pi | P \equiv @x, x \in \meaningof{E} \}}
\end{mathpar}

\begin{eqnarray*}
  \\
  \meaningof{-} : TS \to ST
\end{eqnarray*}

\begin{eqnarray*}
  \\
  L : TS \to ST
\end{eqnarray*}

\begin{eqnarray*}
  \\
  P \models E \iff P \in \meaningof{E}
\end{eqnarray*}

\begin{eqnarray*}
  P \approx_{L} Q \iff \forall E \in L. P \models E \iff Q \models E
\end{eqnarray*}

\begin{eqnarray*}
  P \approx_{K} Q
\end{eqnarray*}

\begin{eqnarray*}
  P \approx Q
\end{eqnarray*}

$\approx_{K} = \approx = \approx_{L}$

\subsubsection{Contextual duality}

Note that contexts extend the quotation operation to a family of
operations from processes to names. Given a context, $M$, we can
define a \emph{nominal context}, $\quotep{M}$ by $\quotep{M}[P] :=
\quotep{M[P]}$. To foreshadow what is to come we observe that these
operations enjoy a duality with processes very much like the duality
between vectors and maps from vectors to scalars.

Further, because the calculus is essentially higher-order, we have a
correspondence between contexts and processes. More specifically,
given a name $x$ and a context $M$ we can construct $M^{*}_{x}$ such
that 

\begin{mathpar}
  M^{*}_{x} | \lift{x}{P} \red M[P]
\end{mathpar}

namely,

\begin{mathpar}
  M^{*}_{x} := x?(u).M[\dropn{u}]
\end{mathpar}

The dependence of $M^{*}_{x}$ on a name makes it an abstraction, 

\begin{mathpar}
  M^{*} := (x)x?(u).M[\dropn{u}]
\end{mathpar}

\subsection{Additional notation}

It will sometimes be convenient to denote the process a name
quotes. We already have the notation $x = \quotep{P}$, but it will be
convenient to introduce an alternate notation, $\procn{x}$, when we
want to emphasize the connection to the use of the name. Note that, by
virtue of name equivalence, $\quotep{\procn{x}} \nameeq x$; so, the
notation is consistent with previous definitions.

Further, because names have structure it is possible to effect
substitutions on the basis of that structure. This means we need to
upgrade our notation for substitutions, which we accomplish by
adapting comprehension notation. Thus,

\begin{mathpar}
  P\{ y / x : x \in S \}
\end{mathpar}

is interpreted to mean the process derived from P by replacing (in a
capture-avoiding manner) each occurrence of $x$ in $S$ by $y$. For example,

\begin{mathpar}
  P\{ \quotep{\procn{x}|\procn{x}} / x : x \in \freenames{P} \}
\end{mathpar}

will replace each (occurrence) of a free name $x$ in $P$ by
$\quotep{\procn{x}|\procn{x}}$.

Also, we will avail ourselves of the notation $x^{L}$ and $x^{R}$ to
denote injections of a name into disjoint copies of the name
space. There are numerous ways to accomplish this. One example can be
found in \cite{MeredithR05}. This notation overloads to vectors of
names: $\vec{x}^{\pi} := (x_{i}^{\pi} \; : \; 0 \leq i < |\vec{x}| )$ where $\pi \in \{L,R\}$.

We also use $P^{\Box} := P|\Box$.

In \cite{MeredithR05} an interpretation of the new operator is
given. It turns out that there are several possible interpretations
all enjoying the requisite algebraic properties of the operator (see
\cite{milner91polyadicpi}). We will therefore make liberal use of
$(\nu\; \vec{x})P$.

% subsection the_syntax_and_semantics_of_the_notation_system (end)   

\section{Interpretation of QM}
\subsection{Supporting definitions}
\subsubsection{Multiplication}
\begin{mathpar}
  \quotep{Q} \cdot \quotep{R} := \quotep{Q|R}
  \and \\
  \quotep{Q} \cdot P := P\{ \quotep{Q|R} / \quotep{R} : \quotep{R} \in \freenames{P} \}
\end{mathpar}

\paragraph{Discussion}
The first line needs little explanation. The second line says that
each free name of the process is replaced with the multiplication of
that name by the scalar. Multiplication of a scalar (name) by a state
(process) results in a process all the names of which have been `moved
over' by parallel composition with the process the scalar
quotes. There is a subtlety that the bound names have to be
manipulated so that multiplied names aren't accidentally
captured. There are many ways to achieve this.

\begin{remark}\label{rem:multiplication_identities}
  The reader is invited to verify that for all $x,y,z \in \QProc$ and $P \in \Proc$
  \begin{mathpar}
    x \cdot \quotep{0} \equiv x 
    \and
    x \cdot y \equiv y \cdot x
    \and
    x \cdot (y \cdot z) \equiv (x \cdot y) \cdot z
    \and \\
    \quotep{0} \cdot P \equiv P
    \and \\
    x \cdot (y \cdot P) \equiv (x \cdot y) \cdot P
    \and \\
    x \cdot (P|Q) \equiv (x \cdot P) | (x \cdot Q)
    \and \\    
  \end{mathpar}
\end{remark}

\subsubsection{Tensor product}

We define a tensor product on processes by structural induction.

\paragraph{Tensor of sums} First note that all summations, including
$\pzero$ and sequence, can be written $\Sigma_{i} x_{i}.A_{i} +
\Sigma_{j} x_{j}.C_{j}$, where we have grouped input-guarded processes
together and output-guarded processes together.

Thus, we can define the tensor product of two summations, $N_{1}\otimes N_{2}$, where

\begin{mathpar}
  N_{1} := \Sigma_{i} x_{i}.A_{i} + \Sigma_{j} x_{j}.C_{j}
  \and
  N_{2} := \Sigma_{i'} y_{i'}.B_{i'} + \Sigma_{j'} y_{j'}.D_{j'} 
\end{mathpar}

as follows.

\begin{mathpar}
  \Sigma_{i} x_{i}.A_{i} + \Sigma_{j} x_{j}.C_{j} \otimes \Sigma_{i'}
  y_{i'}.B_{i'} + \Sigma_{j'} y_{j'}.D_{j'} 
  \and \\
  := \; \Sigma_{i} \Sigma_{i'} \quotep{\stackrel{\vee}{x_{i}}| \stackrel{\vee}{y_{i'}}}.(A_{i}\otimes B_{i'}) \; | \; \Sigma_{i'} \Sigma_{i} \quotep{\stackrel{\vee}{y_{i'}}|\stackrel{\vee}{x_{i}}}.(B_{i'}\otimes A_{i})
  \and
  \;\; | \;\; \Sigma_{j} \Sigma_{j'} \quotep{\stackrel{\vee}{x_{j}}|\stackrel{\vee}{y_{j'}}}.(A_{j}\otimes B_{j'}) \; | \; \Sigma_{j'} \Sigma_{j} \quotep{\stackrel{\vee}{y_{j'}}|\stackrel{\vee}{x_{j}}}.(B_{j'}\otimes A_{j})
\end{mathpar}

\begin{remark}
  Do we need to $x^{L}$ and $y^{R}$ for this construction as well?
\end{remark}

\paragraph{Tensor of parallel compositions} Next, we distribute tensor
over par.

\begin{mathpar}
  P_{1}|P_{2} \otimes Q_{1}|Q_{2} := (P_{1} \otimes Q_{1}) | (P_{1}
  \otimes Q_{2}) | (P_{2} \otimes Q_{1}) | (P_{2} \otimes Q_{2})
\end{mathpar}

\paragraph{Tensor with dropped names} We treat tensor of a
process with a dropped name as parallel composition.

\begin{mathpar}
  P \otimes \dropn{x} := P | \dropn{x}
\end{mathpar}

\paragraph{Tensor of agents}

Finally, we need to define tensor on agents. Note that the definition
of tensor on normal products only tensors inputs with inputs and
outputs with outputs. Thus, we only have to define the operation on
``homogeneous'' pairings.

\begin{mathpar}
  (\vec{x})P \otimes (\vec{y})Q
  \and \\
  := (x_{0}^{L}|y_{0}^{R},\ldots,x_{0}^{L}|y_{n}^{R},\ldots,x_{m}^{L}|y_{0}^{R},\ldots,x_{m}^{L}|y_{n}^R)(P\{ \vec{x}^{L}/\vec{x}\} \otimes Q \{ \vec{y}^{R}/\vec{y}\})
  \and \\
  \clift{\vec{P}} \otimes \clift{\vec{Q}}
  \and \\
  := \clift{P_{0}\otimes Q_{0},\ldots,P_{0}\otimes Q_{n},\ldots,P_{m}\otimes Q_{0},\ldots,P_{m}\otimes Q_{n}}
\end{mathpar}

\begin{remark}
  Observe that arities of tensored abstractions matches arities of
  tensored concretions if the original arities matched. Note also that
  the length of the arities corresponds to the increase in dimension
  we see in ordinary vector space tensor product.
\end{remark}

\begin{remark}
  Operationally, this definition distributes the tensor down to
  components ``linked'' by summation. Tensor over summation is
  intriguing in that it mixes names. Moreover, as a consequence of the
  way it mixes names we have the identities for all $x \in \QProc$ and
  $P,Q \in \Proc$

  \begin{mathpar}
    (x \cdot P) \otimes Q \equiv x \cdot (P \otimes Q) \equiv P \otimes (x \cdot Q)
    \and
    P \otimes \pzero \equiv P
  \end{mathpar}

  that the reader is invited to verify.
\end{remark}

\subsubsection{Annihilation}
\begin{mathpar}
  P^{\perp} := \{ Q | \forall R. P|Q \red^{*} R \Rightarrow R \red^{*} \pzero \}
  \and \\
  P^{\underline{\perp}} := \Sigma_{Q \in P^{\perp}} \quotep{Q}?(y).(\dropn{y}|Q) | \Sigma_{Q \in P^{\perp}} \quotep{Q}\clift{\Box}
\end{mathpar}

\paragraph{Discussion} The reader will note that $P^{\perp}$ is a
\emph{set} of processes, while $P^{\underline{\perp}}$ is a
\emph{context}. We call the set $P^{\perp}$ the \emph{annihilators} of
$P$. The parallel composition of a process in the annihilators of $P$
with $P$ will result in a process, the state space of which has all
paths eventually leading to $\pzero$. Execution may endure loops; but
under reasonable conditions of fairness (naturally guaranteed under
most notions of bisimulation) such a composite process cannot get
stuck in such a loop and will, eventually pop out and terminate.

The context $P^{\underline{\perp}}$ is ready and willing to ``take the
$P$ out of'' the process to which it is applied. It will effectively
transmit the code of the process to which it is applied to one of the
annihilators and run the process against it.

\subsubsection{Evaluation}
We fix $M$ a domain of fully abstract interpretation with an equality
coincident with bisimulation. We take $\meaningof{\cdot} : \Proc \to
M$ to be the map interpreting processes and $\nmeaningof{\cdot} : \M
\to Proc$ to be the map running the other way. Then we define

\begin{mathpar}
  \int P := \nmeaningof{\meaningof{P}}
\end{mathpar}

\paragraph{Discussion}
There are many fully abstract interpretations of Milner's
$\pi$-calculus. Any of them can be used as a basis for interpreting
the reflective calculus here. Equipped with such a domain it is
largely a matter of grinding through to check that the Yoneda
construction for the normalization-by-evaluation program can be
extended to this setting.

\begin{remark}
  The reader is invited to verify that $\int (P^{\underline{\perp}}[P]) = 0$.
\end{remark}

\subsection{Quantum mechanics}

Table \ref{tbl:core_qm_op_defns} gives the core operational definitions

\begin{table}[htp]\label{tbl:core_qm_op_defns}
  \center{
    \fbox{
      \begin{tabular}{c|c}
        quantum mechanics & process calculus \\
        \hline
        scalar & $x := \quotep{P}$ \\
        state vector & $\state{P} := P$ \\
        dual & $\state{P}^{*} := \event{P^{\underline{\perp}}} := \quotep{P^{\underline{\perp}}}[-]$ \\
        matrix & $ \Sigma_{\alpha} \state{P_{\alpha}}x_{\alpha}\event{Q_{\alpha}}$ \\
        vector addition & $\state{P} + \state{Q} := \state{P | Q}$ \\
        tensor product & $\state{P} \otimes \state{Q} := \state{P \otimes Q}$ \\
        inner product & $\innerprod{P}{Q} := \quotep{\int P^{\underline{\perp}}[Q]}$ \\
      \end{tabular}
    }
  }
  \caption{QM - operational definitions}
\end{table}

where

\begin{mathpar}
  \prmatrix{P}{Q} := \fprmatrix{P}{\quotep{\pzero}}{Q}
  \and
  \fprmatrix{P}{x}{Q} := (\state{P},x,\event{Q})
  \and
  (\fprmatrix{P}{x}{Q})(\state{R}) := x \cdot \innerprod{Q}{R} \cdot \state{P}
  \and
  (\fprmatrix{P}{x}{Q})(\event{R}) := x \cdot \innerprod{R}{P} \cdot \event{Q}
\end{mathpar}

\paragraph{Discussion}
As promised: vectors (aka states) are represented as processes; duals
as contextual duals; inner product definition should be compared with
standard inner product definition for ....

\begin{remark}
  Assuming $\int (P^{\underline{\perp}}[P]) = 0$, the reader is
  invited to verify that $(\fprmatrix{P}{x}{P})(\state{P}) = x \cdot \state{P}$.
\end{remark}

\begin{remark}
  The reader is invited to verify that $\innerprod{P}{Q}$ could
  equally well have been written $\quotep{\int \stackrel{\vee}{x}}$
  where $x = \event{P^{\underline{\perp}}}(Q)$.

  One of the motivations for this remark is that there is another way
  to factor these operations. We could package up evaluation in the dual:

  \begin{mathpar}
    \state{P}^{*} := \event{\int P^{\underline{\perp}}} := \quotep{\int P^{\underline{\perp}}}[-]
  \end{mathpar}

  and then have inner product defined by
  
  \begin{mathpar}
    \innerprod{P}{Q} := \event{P}(Q)
  \end{mathpar}

  Hopefully, experience with the calculations will provide guidance on
  the best factoring.
\end{remark}

\begin{remark}
  Assuming $\int (P^{\underline{\perp}}[P]) = 0$, the reader is
  invited to verify that $\forall P,Q. (\prmatrix{0}{Q})(\state{0}) =
  \state{0}$ and dually $(\prmatrix{P}{0})(\event{0}) = \event{0}$.
\end{remark}

\begin{remark}
  i'm a little worried that i don't (yet) have proper support for
  complex conjugacy. But, the observation above may give us a
  clue. According to Abramsky, it must be the case that the scalars
  are iso to the homset of the identity for the tensor -- which the
  observation above characterizes. 

  For now, we will simply bookmark the notion with $\overline{x}$.
\end{remark}

\subsubsection{Adjointness}

We need to give a definition of $(\cdot)^{\dagger}$ for matrices. The
obvious candidate definition is
\begin{mathpar}
(\Sigma_{\alpha}\fprmatrix{P_{\alpha}}{x_{\alpha}}{Q_{\alpha}})^{\dagger}
= \Sigma_{\alpha}\fprmatrix{(Q_{\alpha}^{\underline{\perp}})^{*}}{\overline{x}_{\alpha}}{P_{\alpha}^{\underline{\perp}}} 
\end{mathpar}

But, $(Q_{\alpha}^{\underline{\perp}})^{*}$ requires a name along
which to communicate the process to achieve the context application.

\subsubsection{Basis for a basis}
If processes label states and ``addition'' of states (a.k.a. vector
addition) is interpreted as parallel composition, what corresponds to
notions of linear independence and basis? Here, we recall that Yoshida
has developed a set of \emph{combinators} for an asynchronous verison
of Milner's $\pi$-calculus. These are a finite set of processes such
any process can be expressed as parallel composition of these
combinators together with liberal uses of the new operator and
replication. We can simply give a translation of these into the
present calculus and have reasonable expectation that the property
carries over. That is, that the resultant set allows to express all
processes via parallel composition. Note, however, that there is no
new operator or replication in this calculus. As a result, we expect
that the corresponding set is actually infinite. That is, we expect
that the space is actually infinite dimensional.

\begin{remark}
  The attentive reader may be a bit concerned. Certainly, the
  collection $S$, $K$ and $I$ is a finite set of
  combinators. Shouldn't we expect to see a finite set of combinators
  for an effectively equivalent system? i am very sympathetic to this
  critique and feel it warrants full attention. On the other hand, i
  also have in mind the following analogy. The natural numbers, as a
  monoid under addition, has exactly $1$ generator, while the natural
  numbers, as a monoid under multiplication, has countably many
  generators (the primes). We observe that the application of the
  lambda calculus is much less resource sensitive than the parallel
  composition of the $\pi$-calculus. Could it be the case that we have
  an analogy of the form
  
  \begin{mathpar}
    m + n : MN :: m*n : M|N
  \end{mathpar}

  giving a similar blow up in the set of ``primes''?  This is such a
  wonderful thought that, even if it's not true, i think it's worth
  writing down.
\end{remark}
 

\documentclass[12pt]{llncs}
%\documentclass{jktr}

\usepackage[pdftex]{hyperref}                   
\usepackage {listings}
\usepackage {mathpartir}
\usepackage{bcprules}
%\usepackage{listings}
                       
\usepackage{graphicx} 
%\usepackage[margins=2.5cm,nohead,nofoot]{geometry}
%\usepackage{geometry}
\usepackage{amsfonts}
\usepackage{amstext}
\usepackage{latexsym}
\usepackage{amssymb}
\usepackage{color}


%\include{myPreamble}
\include{qm2pi.local} 

%\ifpdf
%\usepackage[pdftex]{graphicx}
%\else
%\usepackage{graphicx}
%\fi

 % \ifpdf
%  \usepackage{pdfsync}
%  \if


%\title{Brief Article}
%\author{David F. Snyder}
%\author{L.G. Meredith}

%\address{Dept. of Math., Texas State University--San Marcos, San Marcos, TX 78666}
       
\pagestyle{empty}


\begin{document}

\lstset{language=[Objective]Caml,frame=shadowbox}

\input{qm2pi.front}

% section front matter (end)

\input{qm2pi.intro} 
 
% section introduction (end)

% \input{qm2pi.knotations} 

% section notation (end)

\input{qm2pi.process.calculi} 

% section concurrent_process_calculi_and_spatial_logics_ (end)
    
%\input{qm2pi.knots2pi} 

%\input{qm2pi.trefoil} 

%\input{qm2pi.mainthm} 

% subsection basic_interpretation (end)

%\input{qm2pi.rho.presentation} 
\subsection{The syntax and semantics of the notation system}\label{sub:the_syntax_and_semantics_of_the_notation_system} % (fold)

We now summarize a technical presentation of the calculus that
embodies our theory of dynamics. The typical presentation of such a
calculus follows the style of giving generators and relations on
them. The grammar, below, describing term constructors, freely
generates the set of processes, $\Proc$. This set is then quotiented
by a relation known as structural congruence and it is over this set
that the notion of dynamics is expressed. This presentation is
essentially that of \cite{MeredithR05} with the addition of
polyadicity and summation. For readability we have relegated some of
the technical subtleties to an appendix.

\subsubsection{Process grammar}\label{subsub:process_grammar}

\begin{mathpar}
  \inferrule* [lab=synchronization] {} {{M} \bc \pzero \;|\; x?F \;|\; x!C }
  \and
  \inferrule* [lab=abstraction] {} {{F} \bc (x)P}
  \and
  \inferrule* [lab=concretion] {} {{C} \bc \langle Q \rangle}
  \and
  \inferrule* [lab=process] {} {{P,Q} \bc M \;| \;P|Q \;|\; @{x}}
  \and
  \inferrule* [lab=name] {} {{x} \bc \quotep{P}}
\end{mathpar} 

Note that $\vec{x}$ (resp. $\vec{P}$) denotes a vector of names
(resp. processes) of length $|\vec{x}|$ (resp. $|\vec{P}|$). We adopt
the following useful abbreviations.

\begin{mathpar}
   x?(\vec{y}).P := x.(\vec{y})P \and  x\clift{\vec{P}} := x.\clift{\vec{P}}
   \and x!(y) := \lift{x}{\dropn{y}}
   \and \Pi_{i=0}^{n-1}P_i := P_0 | \ldots | P_{n-1}
\end{mathpar}

\subsubsection{Structural congruence}

\paragraph{Free and bound names and alpha-equivalence.} At the
core of structural equivalence is alpha-equivalence which identifies
process that are the same up to a change of variable. Formally, we
recognize the distinction between free and bound names. The free names
of a process, $\freenames{P}$, may be calculated recursively as
follows:

\begin{mathpar}
\freenames{\pzero} := \emptyset
  \and \\
  \freenames{x?(y).P} := \{ x \} \cup (\freenames{P} \setminus \{ y \})
  \and 
  \freenames{x!\langle P \rangle} := \{ x \} \cup \{ P \} 
  \and \\
  \freenames{P|Q} := \freenames{P} \cup \freenames{Q}
  \and \\
  \freenames{@{x}} := \{ x \}
\end{mathpar}

$\pi$
$\quotep{\pi}$

$\freenames{-} : \pi \to \mathcal{P}(\quotep{\pi})$

\begin{eqnarray*}
  \freenames{\pzero} & := & \emptyset \\
  \freenames{x?(y).P} & := & \{ x \} \cup (\freenames{P} \setminus \{ y \}) \\
  \freenames{x!\langle P \rangle} & := & \{ x \} \cup \{ P \} \\
  \freenames{P|Q} & := & \freenames{P} \cup \freenames{Q} \\
  \freenames{\dropn{x}} & := & \{ x \}
\end{eqnarray*}

The bound names of a process, $\boundnames{P}$, are those names occurring in $P$
that are not free. For example, in $x?(y).0$, the name $x$ is free, while $y$ is bound.

\begin{mathpar}
  \inferrule* [lab=monoidal-laws] {} { P|Q \equiv Q|P \and P|0 \equiv P \and P|(Q|R) \equiv (P|Q)|R }
\end{mathpar}

\begin{mathpar}
  \inferrule* [lab=alpha-equivalence] {} { (x)P \equiv (y)P\{y/x\} \and y \not\in \freenames{P} }
\end{mathpar}

\begin{definition}
Then two processes, $P,Q$, are alpha-equivalent if $P = Q\{\vec{y}/\vec{x}\}$ for
some $\vec{x} \in \boundnames{Q},\vec{y} \in \boundnames{P}$, where $Q\{\vec{y}/\vec{x}\}$
denotes the capture-avoiding substitution of $\vec{y}$ for $\vec{x}$ in $Q$.
\end{definition}

\begin{definition}
  The {\em structural congruence} \cite{SangiorgiWalker} , $\equiv$,
  between processes is the least congruence containing
  alpha-equivalence, satisfying the abelian monoid laws
  (associativity, commutativity and $\pzero$ as identity) for parallel
  composition $|$ and for summation $+$.
\end{definition}

\subsection{Name equivalence}

We take name equivalence, written $\nameeq$, to be the smallest
equivalence relation generated by the following rules.

\begin{mathpar}
\inferrule*[lab=Quote-drop]
{ }
{ \quotep{@{x}} \nameeq x }

\inferrule*[lab=Struct-equiv]
{ P \scong Q }
{ \quotep{P} \nameeq \quotep{Q} }
\end{mathpar}

The astute reader will have noticed that the mutual recursion of names
and processes imposes a mutual recursion on alpha-equivalence and
structural equivalence via name-equivalence. Fortunately, all of this
works out pleasantly and we may calculate in the natural way, free of
concern. The reader interested in the details is referred to the
appendix \ref{appendix:rho_details}.

\subsection{Substitution}

We use $\Proc$ for the set of processes, $\QProc$ for the set of
names, and $\id{\{}\vec{y} / \vec{x} \id{\}}$ to denote partial maps,
$s : \QProc \rightarrow \QProc$. A map, $s$ lifts, uniquely, to a map
on process terms, $\widehat{s} : \Proc \rightarrow \Proc$ by the
following equations.

\begin{mathpar}
  (0) \psubstp{Q}{P} := 0 \\
  (R \juxtap S) \psubstp{Q}{P}
  :=    
  (R)\psubstp{Q}{P} \juxtap (S) \psubstp{Q}{P} \\
  (x?(y).R) \psubstp{Q}{P}    
  :=    
  (x)\substp{Q}{P} (z)\concat( (R \psubstn{z}{y}) \psubstp{Q}{P} ) \\
  (\lift{x}{R}) \psubstp{Q}{P}  
  :=
  \lift{(x)\substp{Q}{P}}{ R \psubstp{Q}{P} } \\
%   (\dropn{x})  \psubstp{Q}{P}       
%   := 
%   \left\{ 
%     \begin{array}{ccc} 
%       \dropn{\quotep{Q}} & & x \nameeq \quotep{P} \\
%       \dropn{x} & & otherwise \\
%     \end{array}
%   \right. 
  (\dropn{x})  \psubstp{Q}{P}       
  := 
  \left\{ 
    \begin{array}{ccc} 
      Q & & x \nameeq \quotep{P} \\
      \dropn{x} & & otherwise \\
    \end{array}
  \right.
\end{mathpar}
 

where

\begin{eqnarray}
  (x)\id{\{} \lpquote Q \rpquote / \lpquote P \rpquote \id{\}}            = 
  \left\{ 
    \begin{array}{ccc}
      \lpquote Q \rpquote & & x \nameeq \lpquote P \rpquote \\
      x & & otherwise \\
    \end{array}
  \right. \nonumber
\end{eqnarray}

and $z$ is chosen distinct from $\quotep{P}$, $\quotep{Q}$, the free
names in $Q$, and all the names in $R$. Our $\alpha$-equivalence will
be built in the standard way from this substitution.

\begin{remark}\label{rem:no_self_referential_names}
  One consequence of these definitions is that $\forall P. \quotep{P}
  \not\in \freenames{P}$.
\end{remark}

\subsection{ Dynamic quote: an example }

Anticipating something of what's to come, consider applying the
substitution, $\widehat{\id{\{}u / z \id{\}}}$, to the following pair
of processes, $\lift{w}{y!(z)}$ and $w[ \lpquote y!(z) \rpquote ]$.

\begin{eqnarray}
	\lift{w}{y!(z)}\widehat{\id{\{}u / z \id{\}}}
		& = &
		\lift{w}{y!(u)} \nonumber\\
	w[ \lpquote y!(z) \rpquote ] \widehat{ \id{\{}u / z \id{\}} }
		& = &
		w[ \lpquote y!(z) \rpquote ] \nonumber
\end{eqnarray}

Because the body of the process between quotes is impervious to
substitution, we get radically different answers. In fact, by
examining the first process in an input context,
e.g. $x?(z).\lift{w}{y!(z)}$, we see that the process under the lift
operator may be shaped by prefixed inputs binding a name inside it. In
this sense, the lift operator will be seen as a way to dynamically
construct processes before reifying them as names.

Finally equipped with these standard features we can present the
dynamics of the calculus.

\subsubsection{Operational semantics} 

Finally, we introduce the computational dynamics. What marks these
algebras as distinct from other more traditionally studied algebraic
structures, e.g. vector spaces or polynomial rings, is the manner in
which dynamics is captured. In traditional structures, dynamics is typically
expressed through morphisms between such structures, as in linear maps
between vector spaces or morphisms between rings. In algebras
associated with the semantics of computation, the dynamics is
expressed as part of the algebraic structure itself, through a
reduction reduction relation typically denoted by $\red$. Below, we
give a recursive presentation of this relation for the calculus used
in the encoding.

$\red \subseteq \pi \times \pi$
$\red : \pi \to \mathcal{P}(\pi)$

\begin{mathpar}
  \inferrule* [lab=Comm] { \textsf{match}( x_{src}, x_{trgt} ) } { x_{trgt}?(y)P \; | \; x_{src}!\langle {Q} \rangle \red P\{\quotep{Q}/y}\} }
  \and \\
  \inferrule* [lab=Par] {{P} \red {P}'} {{{P} | {Q}} \red {{P}' | {Q}}}
  \and
  \inferrule* [lab=Equiv]{{{P} \scong {P}'} \andalso {{P}' \red {Q}'} \andalso {{Q}' \scong {Q}}}{{P} \red {Q}}
\end{mathpar}

\begin{eqnarray*}
  match_{\equiv} (\quotep{P},\quotep{Q}) & := & P \equiv Q \\
  match_{\dagger}(\quotep{P},\quotep{Q}) & := & \forall R. P|Q \red^{*} R => R \red^{*} 0 \\
  match_{K}(\quotep{P},\quotep{Q}) & := & K \mbox{ for some context } K
\end{eqnarray*}

$u?(x)P | u!\langle Q \rangle \red P\{\quotep{Q}/x\}$

%We write $\wred$ for $\red^*$, and $P\red$ if $\exists Q $ such that $ P \red Q$.
We write $P\red$ if $\exists Q $ such that $ P \red Q$ and $P\not\red$, otherwise.

\section{Replication}

As mentioned before, it is known that replication (and hence
recursion) can be implemented in a higher-order process algebra
\cite{SangiorgiWalker}. As our first example of calculation with the
machinery thus far presented we give the construction explicitly in
the {\rhoc}.

\begin{eqnarray}
	D_{x} & := & \prefix{x}{y}{(\binpar{\outputp{x}{y}}{@{y}})} \nonumber\\
	\bangp_{x}{P} & := & \binpar{{x}!\langle{\binpar{D_{x}}{P}}\rangle}{D_{x}} \nonumber
\end{eqnarray}

\begin{eqnarray}
	\bangp_{x}{P} & & \nonumber\\
	=
	& {x}!\langle{(\prefix{x}{y}{(\outputp{x}{y} | @{y})) | P}}\rangle 
	      | \prefix{x}{y}{(\outputp{x}{y} | @{y})} & \nonumber\\
	\red
	& (\outputp{x}{y} | @{y})\substn{\quotep{(\prefix{x}{y}{(@{y} | \outputp{x}{y})) | P}}}{y} & \nonumber\\
	=
	& \outputp{x}{\quotep{(\prefix{x}{y}{(\outputp{x}{y} | @{y})) | P}}}
	  | {(\prefix{x}{y}{(\outputp{x}{y} | @{y})) | P}} & \nonumber\\
	\red
	& \ldots & \nonumber\\
	\red^*
	& P | P | \ldots & \nonumber
\end{eqnarray}

Of course, this encoding, as an implementation, runs away, unfolding
$\bangp{P}$ eagerly. A lazier and more implementable replication
operator, restricted to input-guarded processes, may be obtained as follows.

\begin{eqnarray}
\bangp{\prefix{u}{v}{P}} 
	:= 
	\binpar{\lift{x}{\prefix{u}{v}{(\binpar{D(x)}{P})}}}{D(x)} \nonumber
\end{eqnarray}

\begin{remark}
  Note that the lazier definition still does not deal with summation
  or mixed summation (i.e. sums over input and output). The reader is
  invited to construct definitions of replication that deal with these
  features. 

  Further, the definitions are parameterized in a name, $x$. Can you,
  gentle reader, make a definition that eliminates this parameter and
  guarantees no accidental interaction between the replication
  machinery and the process being replicated -- i.e. no accidental
  sharing of names used by the process to get its work done and the
  name(s) used by the replication to effect copying. This latter
  revision of the definition of replication is crucial to obtaining
  the expected identity $!!P \sim !P$.
\end{remark}

\begin{remark}\label{rem:paradoxical_combinator}
  The reader familiar with the lambda calculus will have noticed the
  similarity between $D$ and the paradoxical combinator.

  [Ed. note: the existence of this seems to suggest we have to be more
  restrictive on the set of processes and names we admit if we are to
  support no-cloning.]
\end{remark}

\subsubsection{Bisimulation}

The computational dynamics gives rise to another kind of equivalence,
the equivalence of computational behavior. As previously mentioned
this is typically captured \emph{via} some form of bisimulation.

% The notion we use in this paper is weak barbed bisimulation
% \cite{milner91polyadicpi}.

The notion we use in this paper is derived from weak barbed
bisimulation \cite{milner91polyadicpi}. 

\begin{definition}
An \emph{observation relation}, $\downarrow_{\mathcal N}$, over a set
of names, $\mathcal N$, is the smallest relation satisfying the rules
below.

\infrule[Out-barb]{y \in {\mathcal N}, \; x \nameeq y}
		  {\outputp{x}{v} \downarrow_{\mathcal N} x}
\infrule[Par-barb]{\mbox{$P\downarrow_{\mathcal N} x$ or $Q\downarrow_{\mathcal N} x$}}
		  {\binpar{P}{Q} \downarrow_{\mathcal N} x}

We write $P \Downarrow_{\mathcal N} x$ if there is $Q$ such that 
$P \wred Q$ and $Q \downarrow_{\mathcal N} x$.
\end{definition}

\begin{definition}
%\label{def.bbisim}
An  ${\mathcal N}$-\emph{barbed bisimulation} over a set of names, ${\mathcal N}$, is a symmetric binary relation 
${\mathcal S}_{\mathcal N}$ between agents such that $P\rel{S}_{\mathcal N}Q$ implies:
\begin{enumerate}
\item If $P \red P'$ then $Q \wred Q'$ and $P'\rel{S}_{\mathcal N} Q'$.
\item If $P\downarrow_{\mathcal N} x$, then $Q\Downarrow_{\mathcal N} x$.
\end{enumerate}
$P$ is ${\mathcal N}$-barbed bisimilar to $Q$, written
$P \wbbisim_{\mathcal N} Q$, if $P \rel{S}_{\mathcal N} Q$ for some ${\mathcal N}$-barbed bisimulation ${\mathcal S}_{\mathcal N}$.
\end{definition}

$\mathcal{R} \subseteq \pi \times \pi$

$P \mathcal{R} Q => \forall P'. P \red P' \Rightarrow \exists Q'. Q \red Q', P' \mathcal{R} Q'$

$P \vdash x \Rightarrow Q \vdash x$

\begin{mathpar}
  \inferrule*[lab=Out-barb]{x \nameeq y}{{y}!\langle{Q}\rangle \vdash x}
  \and
  \inferrule*[lab=Par-barb]{\mbox{$P\vdash x$ or $Q\vdash x$}}{\binpar{P}{Q} \vdash x}
\end{mathpar}

\subsubsection{Contexts}

One of the principle advantages of computational calculi like the
$\pi$-calculus is a well-defined notion of context,
contextual-equivalence and a correlation between
contextual-equivalence and notions of bisimulation. The notion of
context allows the decomposition of a process into (sub-)process and
its syntactic environment, its context. Thus, a context may be
thought of as a process with a ``hole'' (written $\Box$) in it. The
application of a context $M$ to a process $P$, written $M[P]$, is
tantamount to filling the hole in $M$ with $P$. In this paper we do
not need the full weight of this theory, but do make use of the notion
of context in the proof the main theorem. 

\begin{mathpar}
  \inferrule* [lab=summation] {} {{M_{M},M_{N}} \bc \Box \;|\; x.M_{A} \;|\; M_{M}+M_{N}}
  \and
  \inferrule* [lab=agent] {} {{M_{A}} \bc (\vec{x})M_{P} \;| \; \clift{P_0,\ldots,M_{P},\ldots,P_N}}
  \and \\
  \inferrule* [lab=process] {} {{M_{P}} \bc M_{N} \;| \;P|M_{P} }
\end{mathpar} 

\begin{mathpar}
  \inferrule* [lab=sychronization] {} {M_{N} \bc \Box \;|\; x?M_{F} \;|\; x!M_{C}}
  \and
  \inferrule* [lab=abstraction] {} {{M_{F}} \bc (x)M_{P} }
  \and
  \inferrule* [lab=concretion] {} {{M_{C}} \bc \langle M_{P} \rangle }
  \and \\
  \inferrule* [lab=process] {} {{M_{P}} \bc M_{N} \;| \;P|M_{P} }
\end{mathpar}

\begin{definition}[contextual application] Given a context $M$, and
  process $P$, we define the \emph{contextual application}, $M[P] :=
  M\{P/\Box\}$. That is, the contextual application of M to P is the
  substitution of $P$ for $\Box$ in $M$.
\end{definition}

$\meaningof{-} : L \to \mathcal{P}(\pi)$

\begin{mathpar}
  \inferrule* [lab=collection] {} {\meaningof{true} = \pi, \and \meaningof{~E} = \pi \setminus \meaningof{E}, \and \meaningof{E_{1} \& E_{2}} = \meaningof{E_{1}} \cap \meaningof{E_{2}}}
\end{mathpar}

\begin{mathpar}
  \inferrule* [lab=structure] {} {\meaningof{0} = \{ P \in \pi | P \equiv 0 \}, \and \\ \meaningof{E_1 | E_2} = \{ P \in \pi | P \equiv P_{1} | P_{2}, P_{1} \in \meaningof{E_{1}}, P_{2} \in \meaningof{E_2}\} }
\end{mathpar}

\begin{mathpar}
 \inferrule* [lab=behavior] {} {\meaningof{\langle a?b \rangle E} = \{ P \in \pi | P \equiv Q | u?(y)P', \\ \and \\\\ \and \\ \;\;\; u \in \meaningof{a}, \forall z.P'\{z/y\} \in \meaningof{E\{z/b\}}\}, \and \\ \meaningof{a!E} = \{ P \in \pi | P \equiv Q | x!\langle P' \rangle, x \in \meaningof{a} P' \in \meaningof{E}\} }
\end{mathpar}

\begin{mathpar}
 \inferrule* [lab=nominal] {} {\meaningof{\quotep{E}} = \{ \quotep{P} \in \quotep{\pi} | P \in \meaningof{E} \}, \and \meaningof{\quotep{P}} = \{ \quotep{Q} \in \quotep{\pi} | P \equiv Q \} \and \\ \meaningof{@\quotep{E}} = \{ P \in \pi | P \equiv @x, x \in \meaningof{E} \}}
\end{mathpar}

\begin{eqnarray*}
  \\
  \meaningof{-} : TS \to ST
\end{eqnarray*}

\begin{eqnarray*}
  \\
  L : TS \to ST
\end{eqnarray*}

\begin{eqnarray*}
  \\
  P \models E \iff P \in \meaningof{E}
\end{eqnarray*}

\begin{eqnarray*}
  P \approx_{L} Q \iff \forall E \in L. P \models E \iff Q \models E
\end{eqnarray*}

\begin{eqnarray*}
  P \approx_{K} Q
\end{eqnarray*}

\begin{eqnarray*}
  P \approx Q
\end{eqnarray*}

$\approx_{K} = \approx = \approx_{L}$

\subsubsection{Contextual duality}

Note that contexts extend the quotation operation to a family of
operations from processes to names. Given a context, $M$, we can
define a \emph{nominal context}, $\quotep{M}$ by $\quotep{M}[P] :=
\quotep{M[P]}$. To foreshadow what is to come we observe that these
operations enjoy a duality with processes very much like the duality
between vectors and maps from vectors to scalars.

Further, because the calculus is essentially higher-order, we have a
correspondence between contexts and processes. More specifically,
given a name $x$ and a context $M$ we can construct $M^{*}_{x}$ such
that 

\begin{mathpar}
  M^{*}_{x} | \lift{x}{P} \red M[P]
\end{mathpar}

namely,

\begin{mathpar}
  M^{*}_{x} := x?(u).M[\dropn{u}]
\end{mathpar}

The dependence of $M^{*}_{x}$ on a name makes it an abstraction, 

\begin{mathpar}
  M^{*} := (x)x?(u).M[\dropn{u}]
\end{mathpar}

\subsection{Additional notation}

It will sometimes be convenient to denote the process a name
quotes. We already have the notation $x = \quotep{P}$, but it will be
convenient to introduce an alternate notation, $\procn{x}$, when we
want to emphasize the connection to the use of the name. Note that, by
virtue of name equivalence, $\quotep{\procn{x}} \nameeq x$; so, the
notation is consistent with previous definitions.

Further, because names have structure it is possible to effect
substitutions on the basis of that structure. This means we need to
upgrade our notation for substitutions, which we accomplish by
adapting comprehension notation. Thus,

\begin{mathpar}
  P\{ y / x : x \in S \}
\end{mathpar}

is interpreted to mean the process derived from P by replacing (in a
capture-avoiding manner) each occurrence of $x$ in $S$ by $y$. For example,

\begin{mathpar}
  P\{ \quotep{\procn{x}|\procn{x}} / x : x \in \freenames{P} \}
\end{mathpar}

will replace each (occurrence) of a free name $x$ in $P$ by
$\quotep{\procn{x}|\procn{x}}$.

Also, we will avail ourselves of the notation $x^{L}$ and $x^{R}$ to
denote injections of a name into disjoint copies of the name
space. There are numerous ways to accomplish this. One example can be
found in \cite{MeredithR05}. This notation overloads to vectors of
names: $\vec{x}^{\pi} := (x_{i}^{\pi} \; : \; 0 \leq i < |\vec{x}| )$ where $\pi \in \{L,R\}$.

We also use $P^{\Box} := P|\Box$.

In \cite{MeredithR05} an interpretation of the new operator is
given. It turns out that there are several possible interpretations
all enjoying the requisite algebraic properties of the operator (see
\cite{milner91polyadicpi}). We will therefore make liberal use of
$(\nu\; \vec{x})P$.

% subsection the_syntax_and_semantics_of_the_notation_system (end)   

\input{qm2pi.qmops} 

\input{qm2pi.sterngerlach} 

\input{qm2pi.metric} 

% section concurrent_process_calculi (end)

%\input{qm2pi.proofsketch}

% section proof sketch (end)

%\input{qm2pi.slviaknots} 

% section spatial logic via knots (end)

\input{qm2pi.conclusion}

% section conclusion (end)

%\input{qm2pi.dtcodes} 

% section wiring algorithm (end)

\input{qm2pi.ack} 

% section acknowledgments (end)

\newpage


\bibliographystyle{plain}   
\bibliography{../../biblios/main.bib}

\input{qm2pi.rhodetails}

\end{document}

 

\documentclass[12pt]{llncs}
%\documentclass{jktr}

\usepackage[pdftex]{hyperref}                   
\usepackage {listings}
\usepackage {mathpartir}
\usepackage{bcprules}
%\usepackage{listings}
                       
\usepackage{graphicx} 
%\usepackage[margins=2.5cm,nohead,nofoot]{geometry}
%\usepackage{geometry}
\usepackage{amsfonts}
\usepackage{amstext}
\usepackage{latexsym}
\usepackage{amssymb}
\usepackage{color}


%\include{myPreamble}
\include{qm2pi.local} 

%\ifpdf
%\usepackage[pdftex]{graphicx}
%\else
%\usepackage{graphicx}
%\fi

 % \ifpdf
%  \usepackage{pdfsync}
%  \if


%\title{Brief Article}
%\author{David F. Snyder}
%\author{L.G. Meredith}

%\address{Dept. of Math., Texas State University--San Marcos, San Marcos, TX 78666}
       
\pagestyle{empty}


\begin{document}

\lstset{language=[Objective]Caml,frame=shadowbox}

\input{qm2pi.front}

% section front matter (end)

\input{qm2pi.intro} 
 
% section introduction (end)

% \input{qm2pi.knotations} 

% section notation (end)

\input{qm2pi.process.calculi} 

% section concurrent_process_calculi_and_spatial_logics_ (end)
    
%\input{qm2pi.knots2pi} 

%\input{qm2pi.trefoil} 

%\input{qm2pi.mainthm} 

% subsection basic_interpretation (end)

%\input{qm2pi.rho.presentation} 
\subsection{The syntax and semantics of the notation system}\label{sub:the_syntax_and_semantics_of_the_notation_system} % (fold)

We now summarize a technical presentation of the calculus that
embodies our theory of dynamics. The typical presentation of such a
calculus follows the style of giving generators and relations on
them. The grammar, below, describing term constructors, freely
generates the set of processes, $\Proc$. This set is then quotiented
by a relation known as structural congruence and it is over this set
that the notion of dynamics is expressed. This presentation is
essentially that of \cite{MeredithR05} with the addition of
polyadicity and summation. For readability we have relegated some of
the technical subtleties to an appendix.

\subsubsection{Process grammar}\label{subsub:process_grammar}

\begin{mathpar}
  \inferrule* [lab=synchronization] {} {{M} \bc \pzero \;|\; x?F \;|\; x!C }
  \and
  \inferrule* [lab=abstraction] {} {{F} \bc (x)P}
  \and
  \inferrule* [lab=concretion] {} {{C} \bc \langle Q \rangle}
  \and
  \inferrule* [lab=process] {} {{P,Q} \bc M \;| \;P|Q \;|\; @{x}}
  \and
  \inferrule* [lab=name] {} {{x} \bc \quotep{P}}
\end{mathpar} 

Note that $\vec{x}$ (resp. $\vec{P}$) denotes a vector of names
(resp. processes) of length $|\vec{x}|$ (resp. $|\vec{P}|$). We adopt
the following useful abbreviations.

\begin{mathpar}
   x?(\vec{y}).P := x.(\vec{y})P \and  x\clift{\vec{P}} := x.\clift{\vec{P}}
   \and x!(y) := \lift{x}{\dropn{y}}
   \and \Pi_{i=0}^{n-1}P_i := P_0 | \ldots | P_{n-1}
\end{mathpar}

\subsubsection{Structural congruence}

\paragraph{Free and bound names and alpha-equivalence.} At the
core of structural equivalence is alpha-equivalence which identifies
process that are the same up to a change of variable. Formally, we
recognize the distinction between free and bound names. The free names
of a process, $\freenames{P}$, may be calculated recursively as
follows:

\begin{mathpar}
\freenames{\pzero} := \emptyset
  \and \\
  \freenames{x?(y).P} := \{ x \} \cup (\freenames{P} \setminus \{ y \})
  \and 
  \freenames{x!\langle P \rangle} := \{ x \} \cup \{ P \} 
  \and \\
  \freenames{P|Q} := \freenames{P} \cup \freenames{Q}
  \and \\
  \freenames{@{x}} := \{ x \}
\end{mathpar}

$\pi$
$\quotep{\pi}$

$\freenames{-} : \pi \to \mathcal{P}(\quotep{\pi})$

\begin{eqnarray*}
  \freenames{\pzero} & := & \emptyset \\
  \freenames{x?(y).P} & := & \{ x \} \cup (\freenames{P} \setminus \{ y \}) \\
  \freenames{x!\langle P \rangle} & := & \{ x \} \cup \{ P \} \\
  \freenames{P|Q} & := & \freenames{P} \cup \freenames{Q} \\
  \freenames{\dropn{x}} & := & \{ x \}
\end{eqnarray*}

The bound names of a process, $\boundnames{P}$, are those names occurring in $P$
that are not free. For example, in $x?(y).0$, the name $x$ is free, while $y$ is bound.

\begin{mathpar}
  \inferrule* [lab=monoidal-laws] {} { P|Q \equiv Q|P \and P|0 \equiv P \and P|(Q|R) \equiv (P|Q)|R }
\end{mathpar}

\begin{mathpar}
  \inferrule* [lab=alpha-equivalence] {} { (x)P \equiv (y)P\{y/x\} \and y \not\in \freenames{P} }
\end{mathpar}

\begin{definition}
Then two processes, $P,Q$, are alpha-equivalent if $P = Q\{\vec{y}/\vec{x}\}$ for
some $\vec{x} \in \boundnames{Q},\vec{y} \in \boundnames{P}$, where $Q\{\vec{y}/\vec{x}\}$
denotes the capture-avoiding substitution of $\vec{y}$ for $\vec{x}$ in $Q$.
\end{definition}

\begin{definition}
  The {\em structural congruence} \cite{SangiorgiWalker} , $\equiv$,
  between processes is the least congruence containing
  alpha-equivalence, satisfying the abelian monoid laws
  (associativity, commutativity and $\pzero$ as identity) for parallel
  composition $|$ and for summation $+$.
\end{definition}

\subsection{Name equivalence}

We take name equivalence, written $\nameeq$, to be the smallest
equivalence relation generated by the following rules.

\begin{mathpar}
\inferrule*[lab=Quote-drop]
{ }
{ \quotep{@{x}} \nameeq x }

\inferrule*[lab=Struct-equiv]
{ P \scong Q }
{ \quotep{P} \nameeq \quotep{Q} }
\end{mathpar}

The astute reader will have noticed that the mutual recursion of names
and processes imposes a mutual recursion on alpha-equivalence and
structural equivalence via name-equivalence. Fortunately, all of this
works out pleasantly and we may calculate in the natural way, free of
concern. The reader interested in the details is referred to the
appendix \ref{appendix:rho_details}.

\subsection{Substitution}

We use $\Proc$ for the set of processes, $\QProc$ for the set of
names, and $\id{\{}\vec{y} / \vec{x} \id{\}}$ to denote partial maps,
$s : \QProc \rightarrow \QProc$. A map, $s$ lifts, uniquely, to a map
on process terms, $\widehat{s} : \Proc \rightarrow \Proc$ by the
following equations.

\begin{mathpar}
  (0) \psubstp{Q}{P} := 0 \\
  (R \juxtap S) \psubstp{Q}{P}
  :=    
  (R)\psubstp{Q}{P} \juxtap (S) \psubstp{Q}{P} \\
  (x?(y).R) \psubstp{Q}{P}    
  :=    
  (x)\substp{Q}{P} (z)\concat( (R \psubstn{z}{y}) \psubstp{Q}{P} ) \\
  (\lift{x}{R}) \psubstp{Q}{P}  
  :=
  \lift{(x)\substp{Q}{P}}{ R \psubstp{Q}{P} } \\
%   (\dropn{x})  \psubstp{Q}{P}       
%   := 
%   \left\{ 
%     \begin{array}{ccc} 
%       \dropn{\quotep{Q}} & & x \nameeq \quotep{P} \\
%       \dropn{x} & & otherwise \\
%     \end{array}
%   \right. 
  (\dropn{x})  \psubstp{Q}{P}       
  := 
  \left\{ 
    \begin{array}{ccc} 
      Q & & x \nameeq \quotep{P} \\
      \dropn{x} & & otherwise \\
    \end{array}
  \right.
\end{mathpar}
 

where

\begin{eqnarray}
  (x)\id{\{} \lpquote Q \rpquote / \lpquote P \rpquote \id{\}}            = 
  \left\{ 
    \begin{array}{ccc}
      \lpquote Q \rpquote & & x \nameeq \lpquote P \rpquote \\
      x & & otherwise \\
    \end{array}
  \right. \nonumber
\end{eqnarray}

and $z$ is chosen distinct from $\quotep{P}$, $\quotep{Q}$, the free
names in $Q$, and all the names in $R$. Our $\alpha$-equivalence will
be built in the standard way from this substitution.

\begin{remark}\label{rem:no_self_referential_names}
  One consequence of these definitions is that $\forall P. \quotep{P}
  \not\in \freenames{P}$.
\end{remark}

\subsection{ Dynamic quote: an example }

Anticipating something of what's to come, consider applying the
substitution, $\widehat{\id{\{}u / z \id{\}}}$, to the following pair
of processes, $\lift{w}{y!(z)}$ and $w[ \lpquote y!(z) \rpquote ]$.

\begin{eqnarray}
	\lift{w}{y!(z)}\widehat{\id{\{}u / z \id{\}}}
		& = &
		\lift{w}{y!(u)} \nonumber\\
	w[ \lpquote y!(z) \rpquote ] \widehat{ \id{\{}u / z \id{\}} }
		& = &
		w[ \lpquote y!(z) \rpquote ] \nonumber
\end{eqnarray}

Because the body of the process between quotes is impervious to
substitution, we get radically different answers. In fact, by
examining the first process in an input context,
e.g. $x?(z).\lift{w}{y!(z)}$, we see that the process under the lift
operator may be shaped by prefixed inputs binding a name inside it. In
this sense, the lift operator will be seen as a way to dynamically
construct processes before reifying them as names.

Finally equipped with these standard features we can present the
dynamics of the calculus.

\subsubsection{Operational semantics} 

Finally, we introduce the computational dynamics. What marks these
algebras as distinct from other more traditionally studied algebraic
structures, e.g. vector spaces or polynomial rings, is the manner in
which dynamics is captured. In traditional structures, dynamics is typically
expressed through morphisms between such structures, as in linear maps
between vector spaces or morphisms between rings. In algebras
associated with the semantics of computation, the dynamics is
expressed as part of the algebraic structure itself, through a
reduction reduction relation typically denoted by $\red$. Below, we
give a recursive presentation of this relation for the calculus used
in the encoding.

$\red \subseteq \pi \times \pi$
$\red : \pi \to \mathcal{P}(\pi)$

\begin{mathpar}
  \inferrule* [lab=Comm] { \textsf{match}( x_{src}, x_{trgt} ) } { x_{trgt}?(y)P \; | \; x_{src}!\langle {Q} \rangle \red P\{\quotep{Q}/y}\} }
  \and \\
  \inferrule* [lab=Par] {{P} \red {P}'} {{{P} | {Q}} \red {{P}' | {Q}}}
  \and
  \inferrule* [lab=Equiv]{{{P} \scong {P}'} \andalso {{P}' \red {Q}'} \andalso {{Q}' \scong {Q}}}{{P} \red {Q}}
\end{mathpar}

\begin{eqnarray*}
  match_{\equiv} (\quotep{P},\quotep{Q}) & := & P \equiv Q \\
  match_{\dagger}(\quotep{P},\quotep{Q}) & := & \forall R. P|Q \red^{*} R => R \red^{*} 0 \\
  match_{K}(\quotep{P},\quotep{Q}) & := & K \mbox{ for some context } K
\end{eqnarray*}

$u?(x)P | u!\langle Q \rangle \red P\{\quotep{Q}/x\}$

%We write $\wred$ for $\red^*$, and $P\red$ if $\exists Q $ such that $ P \red Q$.
We write $P\red$ if $\exists Q $ such that $ P \red Q$ and $P\not\red$, otherwise.

\section{Replication}

As mentioned before, it is known that replication (and hence
recursion) can be implemented in a higher-order process algebra
\cite{SangiorgiWalker}. As our first example of calculation with the
machinery thus far presented we give the construction explicitly in
the {\rhoc}.

\begin{eqnarray}
	D_{x} & := & \prefix{x}{y}{(\binpar{\outputp{x}{y}}{@{y}})} \nonumber\\
	\bangp_{x}{P} & := & \binpar{{x}!\langle{\binpar{D_{x}}{P}}\rangle}{D_{x}} \nonumber
\end{eqnarray}

\begin{eqnarray}
	\bangp_{x}{P} & & \nonumber\\
	=
	& {x}!\langle{(\prefix{x}{y}{(\outputp{x}{y} | @{y})) | P}}\rangle 
	      | \prefix{x}{y}{(\outputp{x}{y} | @{y})} & \nonumber\\
	\red
	& (\outputp{x}{y} | @{y})\substn{\quotep{(\prefix{x}{y}{(@{y} | \outputp{x}{y})) | P}}}{y} & \nonumber\\
	=
	& \outputp{x}{\quotep{(\prefix{x}{y}{(\outputp{x}{y} | @{y})) | P}}}
	  | {(\prefix{x}{y}{(\outputp{x}{y} | @{y})) | P}} & \nonumber\\
	\red
	& \ldots & \nonumber\\
	\red^*
	& P | P | \ldots & \nonumber
\end{eqnarray}

Of course, this encoding, as an implementation, runs away, unfolding
$\bangp{P}$ eagerly. A lazier and more implementable replication
operator, restricted to input-guarded processes, may be obtained as follows.

\begin{eqnarray}
\bangp{\prefix{u}{v}{P}} 
	:= 
	\binpar{\lift{x}{\prefix{u}{v}{(\binpar{D(x)}{P})}}}{D(x)} \nonumber
\end{eqnarray}

\begin{remark}
  Note that the lazier definition still does not deal with summation
  or mixed summation (i.e. sums over input and output). The reader is
  invited to construct definitions of replication that deal with these
  features. 

  Further, the definitions are parameterized in a name, $x$. Can you,
  gentle reader, make a definition that eliminates this parameter and
  guarantees no accidental interaction between the replication
  machinery and the process being replicated -- i.e. no accidental
  sharing of names used by the process to get its work done and the
  name(s) used by the replication to effect copying. This latter
  revision of the definition of replication is crucial to obtaining
  the expected identity $!!P \sim !P$.
\end{remark}

\begin{remark}\label{rem:paradoxical_combinator}
  The reader familiar with the lambda calculus will have noticed the
  similarity between $D$ and the paradoxical combinator.

  [Ed. note: the existence of this seems to suggest we have to be more
  restrictive on the set of processes and names we admit if we are to
  support no-cloning.]
\end{remark}

\subsubsection{Bisimulation}

The computational dynamics gives rise to another kind of equivalence,
the equivalence of computational behavior. As previously mentioned
this is typically captured \emph{via} some form of bisimulation.

% The notion we use in this paper is weak barbed bisimulation
% \cite{milner91polyadicpi}.

The notion we use in this paper is derived from weak barbed
bisimulation \cite{milner91polyadicpi}. 

\begin{definition}
An \emph{observation relation}, $\downarrow_{\mathcal N}$, over a set
of names, $\mathcal N$, is the smallest relation satisfying the rules
below.

\infrule[Out-barb]{y \in {\mathcal N}, \; x \nameeq y}
		  {\outputp{x}{v} \downarrow_{\mathcal N} x}
\infrule[Par-barb]{\mbox{$P\downarrow_{\mathcal N} x$ or $Q\downarrow_{\mathcal N} x$}}
		  {\binpar{P}{Q} \downarrow_{\mathcal N} x}

We write $P \Downarrow_{\mathcal N} x$ if there is $Q$ such that 
$P \wred Q$ and $Q \downarrow_{\mathcal N} x$.
\end{definition}

\begin{definition}
%\label{def.bbisim}
An  ${\mathcal N}$-\emph{barbed bisimulation} over a set of names, ${\mathcal N}$, is a symmetric binary relation 
${\mathcal S}_{\mathcal N}$ between agents such that $P\rel{S}_{\mathcal N}Q$ implies:
\begin{enumerate}
\item If $P \red P'$ then $Q \wred Q'$ and $P'\rel{S}_{\mathcal N} Q'$.
\item If $P\downarrow_{\mathcal N} x$, then $Q\Downarrow_{\mathcal N} x$.
\end{enumerate}
$P$ is ${\mathcal N}$-barbed bisimilar to $Q$, written
$P \wbbisim_{\mathcal N} Q$, if $P \rel{S}_{\mathcal N} Q$ for some ${\mathcal N}$-barbed bisimulation ${\mathcal S}_{\mathcal N}$.
\end{definition}

$\mathcal{R} \subseteq \pi \times \pi$

$P \mathcal{R} Q => \forall P'. P \red P' \Rightarrow \exists Q'. Q \red Q', P' \mathcal{R} Q'$

$P \vdash x \Rightarrow Q \vdash x$

\begin{mathpar}
  \inferrule*[lab=Out-barb]{x \nameeq y}{{y}!\langle{Q}\rangle \vdash x}
  \and
  \inferrule*[lab=Par-barb]{\mbox{$P\vdash x$ or $Q\vdash x$}}{\binpar{P}{Q} \vdash x}
\end{mathpar}

\subsubsection{Contexts}

One of the principle advantages of computational calculi like the
$\pi$-calculus is a well-defined notion of context,
contextual-equivalence and a correlation between
contextual-equivalence and notions of bisimulation. The notion of
context allows the decomposition of a process into (sub-)process and
its syntactic environment, its context. Thus, a context may be
thought of as a process with a ``hole'' (written $\Box$) in it. The
application of a context $M$ to a process $P$, written $M[P]$, is
tantamount to filling the hole in $M$ with $P$. In this paper we do
not need the full weight of this theory, but do make use of the notion
of context in the proof the main theorem. 

\begin{mathpar}
  \inferrule* [lab=summation] {} {{M_{M},M_{N}} \bc \Box \;|\; x.M_{A} \;|\; M_{M}+M_{N}}
  \and
  \inferrule* [lab=agent] {} {{M_{A}} \bc (\vec{x})M_{P} \;| \; \clift{P_0,\ldots,M_{P},\ldots,P_N}}
  \and \\
  \inferrule* [lab=process] {} {{M_{P}} \bc M_{N} \;| \;P|M_{P} }
\end{mathpar} 

\begin{mathpar}
  \inferrule* [lab=sychronization] {} {M_{N} \bc \Box \;|\; x?M_{F} \;|\; x!M_{C}}
  \and
  \inferrule* [lab=abstraction] {} {{M_{F}} \bc (x)M_{P} }
  \and
  \inferrule* [lab=concretion] {} {{M_{C}} \bc \langle M_{P} \rangle }
  \and \\
  \inferrule* [lab=process] {} {{M_{P}} \bc M_{N} \;| \;P|M_{P} }
\end{mathpar}

\begin{definition}[contextual application] Given a context $M$, and
  process $P$, we define the \emph{contextual application}, $M[P] :=
  M\{P/\Box\}$. That is, the contextual application of M to P is the
  substitution of $P$ for $\Box$ in $M$.
\end{definition}

$\meaningof{-} : L \to \mathcal{P}(\pi)$

\begin{mathpar}
  \inferrule* [lab=collection] {} {\meaningof{true} = \pi, \and \meaningof{~E} = \pi \setminus \meaningof{E}, \and \meaningof{E_{1} \& E_{2}} = \meaningof{E_{1}} \cap \meaningof{E_{2}}}
\end{mathpar}

\begin{mathpar}
  \inferrule* [lab=structure] {} {\meaningof{0} = \{ P \in \pi | P \equiv 0 \}, \and \\ \meaningof{E_1 | E_2} = \{ P \in \pi | P \equiv P_{1} | P_{2}, P_{1} \in \meaningof{E_{1}}, P_{2} \in \meaningof{E_2}\} }
\end{mathpar}

\begin{mathpar}
 \inferrule* [lab=behavior] {} {\meaningof{\langle a?b \rangle E} = \{ P \in \pi | P \equiv Q | u?(y)P', \\ \and \\\\ \and \\ \;\;\; u \in \meaningof{a}, \forall z.P'\{z/y\} \in \meaningof{E\{z/b\}}\}, \and \\ \meaningof{a!E} = \{ P \in \pi | P \equiv Q | x!\langle P' \rangle, x \in \meaningof{a} P' \in \meaningof{E}\} }
\end{mathpar}

\begin{mathpar}
 \inferrule* [lab=nominal] {} {\meaningof{\quotep{E}} = \{ \quotep{P} \in \quotep{\pi} | P \in \meaningof{E} \}, \and \meaningof{\quotep{P}} = \{ \quotep{Q} \in \quotep{\pi} | P \equiv Q \} \and \\ \meaningof{@\quotep{E}} = \{ P \in \pi | P \equiv @x, x \in \meaningof{E} \}}
\end{mathpar}

\begin{eqnarray*}
  \\
  \meaningof{-} : TS \to ST
\end{eqnarray*}

\begin{eqnarray*}
  \\
  L : TS \to ST
\end{eqnarray*}

\begin{eqnarray*}
  \\
  P \models E \iff P \in \meaningof{E}
\end{eqnarray*}

\begin{eqnarray*}
  P \approx_{L} Q \iff \forall E \in L. P \models E \iff Q \models E
\end{eqnarray*}

\begin{eqnarray*}
  P \approx_{K} Q
\end{eqnarray*}

\begin{eqnarray*}
  P \approx Q
\end{eqnarray*}

$\approx_{K} = \approx = \approx_{L}$

\subsubsection{Contextual duality}

Note that contexts extend the quotation operation to a family of
operations from processes to names. Given a context, $M$, we can
define a \emph{nominal context}, $\quotep{M}$ by $\quotep{M}[P] :=
\quotep{M[P]}$. To foreshadow what is to come we observe that these
operations enjoy a duality with processes very much like the duality
between vectors and maps from vectors to scalars.

Further, because the calculus is essentially higher-order, we have a
correspondence between contexts and processes. More specifically,
given a name $x$ and a context $M$ we can construct $M^{*}_{x}$ such
that 

\begin{mathpar}
  M^{*}_{x} | \lift{x}{P} \red M[P]
\end{mathpar}

namely,

\begin{mathpar}
  M^{*}_{x} := x?(u).M[\dropn{u}]
\end{mathpar}

The dependence of $M^{*}_{x}$ on a name makes it an abstraction, 

\begin{mathpar}
  M^{*} := (x)x?(u).M[\dropn{u}]
\end{mathpar}

\subsection{Additional notation}

It will sometimes be convenient to denote the process a name
quotes. We already have the notation $x = \quotep{P}$, but it will be
convenient to introduce an alternate notation, $\procn{x}$, when we
want to emphasize the connection to the use of the name. Note that, by
virtue of name equivalence, $\quotep{\procn{x}} \nameeq x$; so, the
notation is consistent with previous definitions.

Further, because names have structure it is possible to effect
substitutions on the basis of that structure. This means we need to
upgrade our notation for substitutions, which we accomplish by
adapting comprehension notation. Thus,

\begin{mathpar}
  P\{ y / x : x \in S \}
\end{mathpar}

is interpreted to mean the process derived from P by replacing (in a
capture-avoiding manner) each occurrence of $x$ in $S$ by $y$. For example,

\begin{mathpar}
  P\{ \quotep{\procn{x}|\procn{x}} / x : x \in \freenames{P} \}
\end{mathpar}

will replace each (occurrence) of a free name $x$ in $P$ by
$\quotep{\procn{x}|\procn{x}}$.

Also, we will avail ourselves of the notation $x^{L}$ and $x^{R}$ to
denote injections of a name into disjoint copies of the name
space. There are numerous ways to accomplish this. One example can be
found in \cite{MeredithR05}. This notation overloads to vectors of
names: $\vec{x}^{\pi} := (x_{i}^{\pi} \; : \; 0 \leq i < |\vec{x}| )$ where $\pi \in \{L,R\}$.

We also use $P^{\Box} := P|\Box$.

In \cite{MeredithR05} an interpretation of the new operator is
given. It turns out that there are several possible interpretations
all enjoying the requisite algebraic properties of the operator (see
\cite{milner91polyadicpi}). We will therefore make liberal use of
$(\nu\; \vec{x})P$.

% subsection the_syntax_and_semantics_of_the_notation_system (end)   

\input{qm2pi.qmops} 

\input{qm2pi.sterngerlach} 

\input{qm2pi.metric} 

% section concurrent_process_calculi (end)

%\input{qm2pi.proofsketch}

% section proof sketch (end)

%\input{qm2pi.slviaknots} 

% section spatial logic via knots (end)

\input{qm2pi.conclusion}

% section conclusion (end)

%\input{qm2pi.dtcodes} 

% section wiring algorithm (end)

\input{qm2pi.ack} 

% section acknowledgments (end)

\newpage


\bibliographystyle{plain}   
\bibliography{../../biblios/main.bib}

\input{qm2pi.rhodetails}

\end{document}

 

% section concurrent_process_calculi (end)

%\documentclass[12pt]{llncs}
%\documentclass{jktr}

\usepackage[pdftex]{hyperref}                   
\usepackage {listings}
\usepackage {mathpartir}
\usepackage{bcprules}
%\usepackage{listings}
                       
\usepackage{graphicx} 
%\usepackage[margins=2.5cm,nohead,nofoot]{geometry}
%\usepackage{geometry}
\usepackage{amsfonts}
\usepackage{amstext}
\usepackage{latexsym}
\usepackage{amssymb}
\usepackage{color}


%\include{myPreamble}
\include{qm2pi.local} 

%\ifpdf
%\usepackage[pdftex]{graphicx}
%\else
%\usepackage{graphicx}
%\fi

 % \ifpdf
%  \usepackage{pdfsync}
%  \if


%\title{Brief Article}
%\author{David F. Snyder}
%\author{L.G. Meredith}

%\address{Dept. of Math., Texas State University--San Marcos, San Marcos, TX 78666}
       
\pagestyle{empty}


\begin{document}

\lstset{language=[Objective]Caml,frame=shadowbox}

\input{qm2pi.front}

% section front matter (end)

\input{qm2pi.intro} 
 
% section introduction (end)

% \input{qm2pi.knotations} 

% section notation (end)

\input{qm2pi.process.calculi} 

% section concurrent_process_calculi_and_spatial_logics_ (end)
    
%\input{qm2pi.knots2pi} 

%\input{qm2pi.trefoil} 

%\input{qm2pi.mainthm} 

% subsection basic_interpretation (end)

%\input{qm2pi.rho.presentation} 
\subsection{The syntax and semantics of the notation system}\label{sub:the_syntax_and_semantics_of_the_notation_system} % (fold)

We now summarize a technical presentation of the calculus that
embodies our theory of dynamics. The typical presentation of such a
calculus follows the style of giving generators and relations on
them. The grammar, below, describing term constructors, freely
generates the set of processes, $\Proc$. This set is then quotiented
by a relation known as structural congruence and it is over this set
that the notion of dynamics is expressed. This presentation is
essentially that of \cite{MeredithR05} with the addition of
polyadicity and summation. For readability we have relegated some of
the technical subtleties to an appendix.

\subsubsection{Process grammar}\label{subsub:process_grammar}

\begin{mathpar}
  \inferrule* [lab=synchronization] {} {{M} \bc \pzero \;|\; x?F \;|\; x!C }
  \and
  \inferrule* [lab=abstraction] {} {{F} \bc (x)P}
  \and
  \inferrule* [lab=concretion] {} {{C} \bc \langle Q \rangle}
  \and
  \inferrule* [lab=process] {} {{P,Q} \bc M \;| \;P|Q \;|\; @{x}}
  \and
  \inferrule* [lab=name] {} {{x} \bc \quotep{P}}
\end{mathpar} 

Note that $\vec{x}$ (resp. $\vec{P}$) denotes a vector of names
(resp. processes) of length $|\vec{x}|$ (resp. $|\vec{P}|$). We adopt
the following useful abbreviations.

\begin{mathpar}
   x?(\vec{y}).P := x.(\vec{y})P \and  x\clift{\vec{P}} := x.\clift{\vec{P}}
   \and x!(y) := \lift{x}{\dropn{y}}
   \and \Pi_{i=0}^{n-1}P_i := P_0 | \ldots | P_{n-1}
\end{mathpar}

\subsubsection{Structural congruence}

\paragraph{Free and bound names and alpha-equivalence.} At the
core of structural equivalence is alpha-equivalence which identifies
process that are the same up to a change of variable. Formally, we
recognize the distinction between free and bound names. The free names
of a process, $\freenames{P}$, may be calculated recursively as
follows:

\begin{mathpar}
\freenames{\pzero} := \emptyset
  \and \\
  \freenames{x?(y).P} := \{ x \} \cup (\freenames{P} \setminus \{ y \})
  \and 
  \freenames{x!\langle P \rangle} := \{ x \} \cup \{ P \} 
  \and \\
  \freenames{P|Q} := \freenames{P} \cup \freenames{Q}
  \and \\
  \freenames{@{x}} := \{ x \}
\end{mathpar}

$\pi$
$\quotep{\pi}$

$\freenames{-} : \pi \to \mathcal{P}(\quotep{\pi})$

\begin{eqnarray*}
  \freenames{\pzero} & := & \emptyset \\
  \freenames{x?(y).P} & := & \{ x \} \cup (\freenames{P} \setminus \{ y \}) \\
  \freenames{x!\langle P \rangle} & := & \{ x \} \cup \{ P \} \\
  \freenames{P|Q} & := & \freenames{P} \cup \freenames{Q} \\
  \freenames{\dropn{x}} & := & \{ x \}
\end{eqnarray*}

The bound names of a process, $\boundnames{P}$, are those names occurring in $P$
that are not free. For example, in $x?(y).0$, the name $x$ is free, while $y$ is bound.

\begin{mathpar}
  \inferrule* [lab=monoidal-laws] {} { P|Q \equiv Q|P \and P|0 \equiv P \and P|(Q|R) \equiv (P|Q)|R }
\end{mathpar}

\begin{mathpar}
  \inferrule* [lab=alpha-equivalence] {} { (x)P \equiv (y)P\{y/x\} \and y \not\in \freenames{P} }
\end{mathpar}

\begin{definition}
Then two processes, $P,Q$, are alpha-equivalent if $P = Q\{\vec{y}/\vec{x}\}$ for
some $\vec{x} \in \boundnames{Q},\vec{y} \in \boundnames{P}$, where $Q\{\vec{y}/\vec{x}\}$
denotes the capture-avoiding substitution of $\vec{y}$ for $\vec{x}$ in $Q$.
\end{definition}

\begin{definition}
  The {\em structural congruence} \cite{SangiorgiWalker} , $\equiv$,
  between processes is the least congruence containing
  alpha-equivalence, satisfying the abelian monoid laws
  (associativity, commutativity and $\pzero$ as identity) for parallel
  composition $|$ and for summation $+$.
\end{definition}

\subsection{Name equivalence}

We take name equivalence, written $\nameeq$, to be the smallest
equivalence relation generated by the following rules.

\begin{mathpar}
\inferrule*[lab=Quote-drop]
{ }
{ \quotep{@{x}} \nameeq x }

\inferrule*[lab=Struct-equiv]
{ P \scong Q }
{ \quotep{P} \nameeq \quotep{Q} }
\end{mathpar}

The astute reader will have noticed that the mutual recursion of names
and processes imposes a mutual recursion on alpha-equivalence and
structural equivalence via name-equivalence. Fortunately, all of this
works out pleasantly and we may calculate in the natural way, free of
concern. The reader interested in the details is referred to the
appendix \ref{appendix:rho_details}.

\subsection{Substitution}

We use $\Proc$ for the set of processes, $\QProc$ for the set of
names, and $\id{\{}\vec{y} / \vec{x} \id{\}}$ to denote partial maps,
$s : \QProc \rightarrow \QProc$. A map, $s$ lifts, uniquely, to a map
on process terms, $\widehat{s} : \Proc \rightarrow \Proc$ by the
following equations.

\begin{mathpar}
  (0) \psubstp{Q}{P} := 0 \\
  (R \juxtap S) \psubstp{Q}{P}
  :=    
  (R)\psubstp{Q}{P} \juxtap (S) \psubstp{Q}{P} \\
  (x?(y).R) \psubstp{Q}{P}    
  :=    
  (x)\substp{Q}{P} (z)\concat( (R \psubstn{z}{y}) \psubstp{Q}{P} ) \\
  (\lift{x}{R}) \psubstp{Q}{P}  
  :=
  \lift{(x)\substp{Q}{P}}{ R \psubstp{Q}{P} } \\
%   (\dropn{x})  \psubstp{Q}{P}       
%   := 
%   \left\{ 
%     \begin{array}{ccc} 
%       \dropn{\quotep{Q}} & & x \nameeq \quotep{P} \\
%       \dropn{x} & & otherwise \\
%     \end{array}
%   \right. 
  (\dropn{x})  \psubstp{Q}{P}       
  := 
  \left\{ 
    \begin{array}{ccc} 
      Q & & x \nameeq \quotep{P} \\
      \dropn{x} & & otherwise \\
    \end{array}
  \right.
\end{mathpar}
 

where

\begin{eqnarray}
  (x)\id{\{} \lpquote Q \rpquote / \lpquote P \rpquote \id{\}}            = 
  \left\{ 
    \begin{array}{ccc}
      \lpquote Q \rpquote & & x \nameeq \lpquote P \rpquote \\
      x & & otherwise \\
    \end{array}
  \right. \nonumber
\end{eqnarray}

and $z$ is chosen distinct from $\quotep{P}$, $\quotep{Q}$, the free
names in $Q$, and all the names in $R$. Our $\alpha$-equivalence will
be built in the standard way from this substitution.

\begin{remark}\label{rem:no_self_referential_names}
  One consequence of these definitions is that $\forall P. \quotep{P}
  \not\in \freenames{P}$.
\end{remark}

\subsection{ Dynamic quote: an example }

Anticipating something of what's to come, consider applying the
substitution, $\widehat{\id{\{}u / z \id{\}}}$, to the following pair
of processes, $\lift{w}{y!(z)}$ and $w[ \lpquote y!(z) \rpquote ]$.

\begin{eqnarray}
	\lift{w}{y!(z)}\widehat{\id{\{}u / z \id{\}}}
		& = &
		\lift{w}{y!(u)} \nonumber\\
	w[ \lpquote y!(z) \rpquote ] \widehat{ \id{\{}u / z \id{\}} }
		& = &
		w[ \lpquote y!(z) \rpquote ] \nonumber
\end{eqnarray}

Because the body of the process between quotes is impervious to
substitution, we get radically different answers. In fact, by
examining the first process in an input context,
e.g. $x?(z).\lift{w}{y!(z)}$, we see that the process under the lift
operator may be shaped by prefixed inputs binding a name inside it. In
this sense, the lift operator will be seen as a way to dynamically
construct processes before reifying them as names.

Finally equipped with these standard features we can present the
dynamics of the calculus.

\subsubsection{Operational semantics} 

Finally, we introduce the computational dynamics. What marks these
algebras as distinct from other more traditionally studied algebraic
structures, e.g. vector spaces or polynomial rings, is the manner in
which dynamics is captured. In traditional structures, dynamics is typically
expressed through morphisms between such structures, as in linear maps
between vector spaces or morphisms between rings. In algebras
associated with the semantics of computation, the dynamics is
expressed as part of the algebraic structure itself, through a
reduction reduction relation typically denoted by $\red$. Below, we
give a recursive presentation of this relation for the calculus used
in the encoding.

$\red \subseteq \pi \times \pi$
$\red : \pi \to \mathcal{P}(\pi)$

\begin{mathpar}
  \inferrule* [lab=Comm] { \textsf{match}( x_{src}, x_{trgt} ) } { x_{trgt}?(y)P \; | \; x_{src}!\langle {Q} \rangle \red P\{\quotep{Q}/y}\} }
  \and \\
  \inferrule* [lab=Par] {{P} \red {P}'} {{{P} | {Q}} \red {{P}' | {Q}}}
  \and
  \inferrule* [lab=Equiv]{{{P} \scong {P}'} \andalso {{P}' \red {Q}'} \andalso {{Q}' \scong {Q}}}{{P} \red {Q}}
\end{mathpar}

\begin{eqnarray*}
  match_{\equiv} (\quotep{P},\quotep{Q}) & := & P \equiv Q \\
  match_{\dagger}(\quotep{P},\quotep{Q}) & := & \forall R. P|Q \red^{*} R => R \red^{*} 0 \\
  match_{K}(\quotep{P},\quotep{Q}) & := & K \mbox{ for some context } K
\end{eqnarray*}

$u?(x)P | u!\langle Q \rangle \red P\{\quotep{Q}/x\}$

%We write $\wred$ for $\red^*$, and $P\red$ if $\exists Q $ such that $ P \red Q$.
We write $P\red$ if $\exists Q $ such that $ P \red Q$ and $P\not\red$, otherwise.

\section{Replication}

As mentioned before, it is known that replication (and hence
recursion) can be implemented in a higher-order process algebra
\cite{SangiorgiWalker}. As our first example of calculation with the
machinery thus far presented we give the construction explicitly in
the {\rhoc}.

\begin{eqnarray}
	D_{x} & := & \prefix{x}{y}{(\binpar{\outputp{x}{y}}{@{y}})} \nonumber\\
	\bangp_{x}{P} & := & \binpar{{x}!\langle{\binpar{D_{x}}{P}}\rangle}{D_{x}} \nonumber
\end{eqnarray}

\begin{eqnarray}
	\bangp_{x}{P} & & \nonumber\\
	=
	& {x}!\langle{(\prefix{x}{y}{(\outputp{x}{y} | @{y})) | P}}\rangle 
	      | \prefix{x}{y}{(\outputp{x}{y} | @{y})} & \nonumber\\
	\red
	& (\outputp{x}{y} | @{y})\substn{\quotep{(\prefix{x}{y}{(@{y} | \outputp{x}{y})) | P}}}{y} & \nonumber\\
	=
	& \outputp{x}{\quotep{(\prefix{x}{y}{(\outputp{x}{y} | @{y})) | P}}}
	  | {(\prefix{x}{y}{(\outputp{x}{y} | @{y})) | P}} & \nonumber\\
	\red
	& \ldots & \nonumber\\
	\red^*
	& P | P | \ldots & \nonumber
\end{eqnarray}

Of course, this encoding, as an implementation, runs away, unfolding
$\bangp{P}$ eagerly. A lazier and more implementable replication
operator, restricted to input-guarded processes, may be obtained as follows.

\begin{eqnarray}
\bangp{\prefix{u}{v}{P}} 
	:= 
	\binpar{\lift{x}{\prefix{u}{v}{(\binpar{D(x)}{P})}}}{D(x)} \nonumber
\end{eqnarray}

\begin{remark}
  Note that the lazier definition still does not deal with summation
  or mixed summation (i.e. sums over input and output). The reader is
  invited to construct definitions of replication that deal with these
  features. 

  Further, the definitions are parameterized in a name, $x$. Can you,
  gentle reader, make a definition that eliminates this parameter and
  guarantees no accidental interaction between the replication
  machinery and the process being replicated -- i.e. no accidental
  sharing of names used by the process to get its work done and the
  name(s) used by the replication to effect copying. This latter
  revision of the definition of replication is crucial to obtaining
  the expected identity $!!P \sim !P$.
\end{remark}

\begin{remark}\label{rem:paradoxical_combinator}
  The reader familiar with the lambda calculus will have noticed the
  similarity between $D$ and the paradoxical combinator.

  [Ed. note: the existence of this seems to suggest we have to be more
  restrictive on the set of processes and names we admit if we are to
  support no-cloning.]
\end{remark}

\subsubsection{Bisimulation}

The computational dynamics gives rise to another kind of equivalence,
the equivalence of computational behavior. As previously mentioned
this is typically captured \emph{via} some form of bisimulation.

% The notion we use in this paper is weak barbed bisimulation
% \cite{milner91polyadicpi}.

The notion we use in this paper is derived from weak barbed
bisimulation \cite{milner91polyadicpi}. 

\begin{definition}
An \emph{observation relation}, $\downarrow_{\mathcal N}$, over a set
of names, $\mathcal N$, is the smallest relation satisfying the rules
below.

\infrule[Out-barb]{y \in {\mathcal N}, \; x \nameeq y}
		  {\outputp{x}{v} \downarrow_{\mathcal N} x}
\infrule[Par-barb]{\mbox{$P\downarrow_{\mathcal N} x$ or $Q\downarrow_{\mathcal N} x$}}
		  {\binpar{P}{Q} \downarrow_{\mathcal N} x}

We write $P \Downarrow_{\mathcal N} x$ if there is $Q$ such that 
$P \wred Q$ and $Q \downarrow_{\mathcal N} x$.
\end{definition}

\begin{definition}
%\label{def.bbisim}
An  ${\mathcal N}$-\emph{barbed bisimulation} over a set of names, ${\mathcal N}$, is a symmetric binary relation 
${\mathcal S}_{\mathcal N}$ between agents such that $P\rel{S}_{\mathcal N}Q$ implies:
\begin{enumerate}
\item If $P \red P'$ then $Q \wred Q'$ and $P'\rel{S}_{\mathcal N} Q'$.
\item If $P\downarrow_{\mathcal N} x$, then $Q\Downarrow_{\mathcal N} x$.
\end{enumerate}
$P$ is ${\mathcal N}$-barbed bisimilar to $Q$, written
$P \wbbisim_{\mathcal N} Q$, if $P \rel{S}_{\mathcal N} Q$ for some ${\mathcal N}$-barbed bisimulation ${\mathcal S}_{\mathcal N}$.
\end{definition}

$\mathcal{R} \subseteq \pi \times \pi$

$P \mathcal{R} Q => \forall P'. P \red P' \Rightarrow \exists Q'. Q \red Q', P' \mathcal{R} Q'$

$P \vdash x \Rightarrow Q \vdash x$

\begin{mathpar}
  \inferrule*[lab=Out-barb]{x \nameeq y}{{y}!\langle{Q}\rangle \vdash x}
  \and
  \inferrule*[lab=Par-barb]{\mbox{$P\vdash x$ or $Q\vdash x$}}{\binpar{P}{Q} \vdash x}
\end{mathpar}

\subsubsection{Contexts}

One of the principle advantages of computational calculi like the
$\pi$-calculus is a well-defined notion of context,
contextual-equivalence and a correlation between
contextual-equivalence and notions of bisimulation. The notion of
context allows the decomposition of a process into (sub-)process and
its syntactic environment, its context. Thus, a context may be
thought of as a process with a ``hole'' (written $\Box$) in it. The
application of a context $M$ to a process $P$, written $M[P]$, is
tantamount to filling the hole in $M$ with $P$. In this paper we do
not need the full weight of this theory, but do make use of the notion
of context in the proof the main theorem. 

\begin{mathpar}
  \inferrule* [lab=summation] {} {{M_{M},M_{N}} \bc \Box \;|\; x.M_{A} \;|\; M_{M}+M_{N}}
  \and
  \inferrule* [lab=agent] {} {{M_{A}} \bc (\vec{x})M_{P} \;| \; \clift{P_0,\ldots,M_{P},\ldots,P_N}}
  \and \\
  \inferrule* [lab=process] {} {{M_{P}} \bc M_{N} \;| \;P|M_{P} }
\end{mathpar} 

\begin{mathpar}
  \inferrule* [lab=sychronization] {} {M_{N} \bc \Box \;|\; x?M_{F} \;|\; x!M_{C}}
  \and
  \inferrule* [lab=abstraction] {} {{M_{F}} \bc (x)M_{P} }
  \and
  \inferrule* [lab=concretion] {} {{M_{C}} \bc \langle M_{P} \rangle }
  \and \\
  \inferrule* [lab=process] {} {{M_{P}} \bc M_{N} \;| \;P|M_{P} }
\end{mathpar}

\begin{definition}[contextual application] Given a context $M$, and
  process $P$, we define the \emph{contextual application}, $M[P] :=
  M\{P/\Box\}$. That is, the contextual application of M to P is the
  substitution of $P$ for $\Box$ in $M$.
\end{definition}

$\meaningof{-} : L \to \mathcal{P}(\pi)$

\begin{mathpar}
  \inferrule* [lab=collection] {} {\meaningof{true} = \pi, \and \meaningof{~E} = \pi \setminus \meaningof{E}, \and \meaningof{E_{1} \& E_{2}} = \meaningof{E_{1}} \cap \meaningof{E_{2}}}
\end{mathpar}

\begin{mathpar}
  \inferrule* [lab=structure] {} {\meaningof{0} = \{ P \in \pi | P \equiv 0 \}, \and \\ \meaningof{E_1 | E_2} = \{ P \in \pi | P \equiv P_{1} | P_{2}, P_{1} \in \meaningof{E_{1}}, P_{2} \in \meaningof{E_2}\} }
\end{mathpar}

\begin{mathpar}
 \inferrule* [lab=behavior] {} {\meaningof{\langle a?b \rangle E} = \{ P \in \pi | P \equiv Q | u?(y)P', \\ \and \\\\ \and \\ \;\;\; u \in \meaningof{a}, \forall z.P'\{z/y\} \in \meaningof{E\{z/b\}}\}, \and \\ \meaningof{a!E} = \{ P \in \pi | P \equiv Q | x!\langle P' \rangle, x \in \meaningof{a} P' \in \meaningof{E}\} }
\end{mathpar}

\begin{mathpar}
 \inferrule* [lab=nominal] {} {\meaningof{\quotep{E}} = \{ \quotep{P} \in \quotep{\pi} | P \in \meaningof{E} \}, \and \meaningof{\quotep{P}} = \{ \quotep{Q} \in \quotep{\pi} | P \equiv Q \} \and \\ \meaningof{@\quotep{E}} = \{ P \in \pi | P \equiv @x, x \in \meaningof{E} \}}
\end{mathpar}

\begin{eqnarray*}
  \\
  \meaningof{-} : TS \to ST
\end{eqnarray*}

\begin{eqnarray*}
  \\
  L : TS \to ST
\end{eqnarray*}

\begin{eqnarray*}
  \\
  P \models E \iff P \in \meaningof{E}
\end{eqnarray*}

\begin{eqnarray*}
  P \approx_{L} Q \iff \forall E \in L. P \models E \iff Q \models E
\end{eqnarray*}

\begin{eqnarray*}
  P \approx_{K} Q
\end{eqnarray*}

\begin{eqnarray*}
  P \approx Q
\end{eqnarray*}

$\approx_{K} = \approx = \approx_{L}$

\subsubsection{Contextual duality}

Note that contexts extend the quotation operation to a family of
operations from processes to names. Given a context, $M$, we can
define a \emph{nominal context}, $\quotep{M}$ by $\quotep{M}[P] :=
\quotep{M[P]}$. To foreshadow what is to come we observe that these
operations enjoy a duality with processes very much like the duality
between vectors and maps from vectors to scalars.

Further, because the calculus is essentially higher-order, we have a
correspondence between contexts and processes. More specifically,
given a name $x$ and a context $M$ we can construct $M^{*}_{x}$ such
that 

\begin{mathpar}
  M^{*}_{x} | \lift{x}{P} \red M[P]
\end{mathpar}

namely,

\begin{mathpar}
  M^{*}_{x} := x?(u).M[\dropn{u}]
\end{mathpar}

The dependence of $M^{*}_{x}$ on a name makes it an abstraction, 

\begin{mathpar}
  M^{*} := (x)x?(u).M[\dropn{u}]
\end{mathpar}

\subsection{Additional notation}

It will sometimes be convenient to denote the process a name
quotes. We already have the notation $x = \quotep{P}$, but it will be
convenient to introduce an alternate notation, $\procn{x}$, when we
want to emphasize the connection to the use of the name. Note that, by
virtue of name equivalence, $\quotep{\procn{x}} \nameeq x$; so, the
notation is consistent with previous definitions.

Further, because names have structure it is possible to effect
substitutions on the basis of that structure. This means we need to
upgrade our notation for substitutions, which we accomplish by
adapting comprehension notation. Thus,

\begin{mathpar}
  P\{ y / x : x \in S \}
\end{mathpar}

is interpreted to mean the process derived from P by replacing (in a
capture-avoiding manner) each occurrence of $x$ in $S$ by $y$. For example,

\begin{mathpar}
  P\{ \quotep{\procn{x}|\procn{x}} / x : x \in \freenames{P} \}
\end{mathpar}

will replace each (occurrence) of a free name $x$ in $P$ by
$\quotep{\procn{x}|\procn{x}}$.

Also, we will avail ourselves of the notation $x^{L}$ and $x^{R}$ to
denote injections of a name into disjoint copies of the name
space. There are numerous ways to accomplish this. One example can be
found in \cite{MeredithR05}. This notation overloads to vectors of
names: $\vec{x}^{\pi} := (x_{i}^{\pi} \; : \; 0 \leq i < |\vec{x}| )$ where $\pi \in \{L,R\}$.

We also use $P^{\Box} := P|\Box$.

In \cite{MeredithR05} an interpretation of the new operator is
given. It turns out that there are several possible interpretations
all enjoying the requisite algebraic properties of the operator (see
\cite{milner91polyadicpi}). We will therefore make liberal use of
$(\nu\; \vec{x})P$.

% subsection the_syntax_and_semantics_of_the_notation_system (end)   

\input{qm2pi.qmops} 

\input{qm2pi.sterngerlach} 

\input{qm2pi.metric} 

% section concurrent_process_calculi (end)

%\input{qm2pi.proofsketch}

% section proof sketch (end)

%\input{qm2pi.slviaknots} 

% section spatial logic via knots (end)

\input{qm2pi.conclusion}

% section conclusion (end)

%\input{qm2pi.dtcodes} 

% section wiring algorithm (end)

\input{qm2pi.ack} 

% section acknowledgments (end)

\newpage


\bibliographystyle{plain}   
\bibliography{../../biblios/main.bib}

\input{qm2pi.rhodetails}

\end{document}



% section proof sketch (end)

%\section{Unlikely characters: spatial logic for
  knots}\label{sub:characteristic_formulae} % (fold)

Associated to the mobile process calculi are a family of logics known
as the Hennessy-Milner logics. These logics typically enjoy a
semantics interpreting formulae as sets of processes that when
factored through the encoding outlined above allows an identification
of classes of knots with logical formulae. In the context of this
encoding the sub-family known as the spatial logics \cite{CairesC03}
\cite{CairesC04} \cite{Caires04} are of particular interest providing
several important features for expressing and reasoning about
properties (i.e. classes) of knots. We hint here at how this may be done.

%\begin{description}
%\item [structural connectives] 
\subsubsection{Structural connectives} The spatial logics enjoy
structural connectives corresponding, at the logical level, to the
parallel composition ($P | Q$) and new name ($(\nu \; x)P$)
connectives for processes. As illustrated in the examples below, these
connectives are extremely expressive given the shape of our encoding.
%\item [decideable satisfaction]

\subsubsection{Decideable satisfaction}
In \cite{Caires04} the satisfaction relation is shown to be decideable
for a rich class of processes. It further turns out that the image of
the our encoding is a proper subset of that class. This result
provides the basis for an algorithm by which to search for knots
enjoying a given property.
%\item [characteristic formulae]

\subsubsection{Characteristic formulae}
In the same paper \cite{Caires04} , Caires presents a means of calculating
characteristic formulae, selecting equivalence classes of processes
up to a pre--specified depth limit on the support set of names. Composed with our
encoding, this characteristic formula can be used to select
characteristic formulae for knots.
%\end{description}

\subsubsection{Spatial logic formulae}

The grammar below (segmented for comprehension) summarizes the syntax
of spatial logic formulae. We employ illustrative examples in the
sequel to provide an intuitive understanding of their meaning
referring the reader to \cite{Caires04} for a more detailed explication
of the semantics.

\begin{mathpar}
  \inferrule* [lab=boolean] {} {{A,B} \bc T \;|\; \neg A \;|\; A \wedge B \;|\; \eta = \eta'}
  \and
  \inferrule* [lab=spatial] {} {|\; \pzero \;|\; A | B \;|\; x \text{\textregistered} A \;|\; \forall x . A \;|\;  H x . A}
  \and
  \inferrule* [lab=behavioral] {} {|\; \alpha . A}
  \and 
  \inferrule* [lab=recursion] {} {|\; X(\vec{u}) \;|\; \mu X(\vec{u}) . A}
  \and
  \inferrule* [lab=action] {} {\alpha \bc \langle x?(\vec{y}) \rangle \;|\; \langle x!(\vec{y}) \rangle \;|\; \langle \tau \rangle}
  \and 
  \inferrule* [lab=name] {} {\eta \bc x \;|\; \tau}
\end{mathpar} 

% subsection characteristic_formulae (end)   	 

\subsection{Example formulae}\label{sub:example_formulae_} % (fold)

\subsubsection{Crossing as formula.}
% 
% \begin{align*}
%   \frac{d}{dx} \sin x &= \cos x 
%   & \frac{d}{dx} e^x &= e^x \\
%   \frac{d}{dx} \cos x &= - \sin x 
%   & \frac{d}{dx} \log x &= \frac{1}{x} \\
% \end{align*} 

\begin{align*}
 \mu C(x_{0},x_{1},y_{0},y_{1},u).&(\langle x_{0}?(z) \rangle(\langle u! \rangle\langle y_{1}!z \rangle C(x_{0},x_{1},y_{0},y_{1},u)) & \\
  & \wedge \langle y_{1}?(z) \rangle (\langle u! \rangle \langle x_{0}!z \rangle C(x_{0},x_{1},y_{0},y_{1},u)) & \\
  & \wedge \langle x_{1}?(z) \rangle (\langle u? \rangle \langle y_{0}!z \rangle C(x_{0},x_{1},y_{0},y_{1},u)) & \\
  & \wedge \langle y_{0}?(z) \rangle (\langle u? \rangle \langle x_{1}!z \rangle C(x_{0},x_{1},y_{0},y_{1},u))) &
\end{align*}

The lexicographical similarity between the shape of this formulae and
the shape of definition of the process representing a crossing reveals
the intuitive meaning of this formulae. It describes the capabilities
of a process that has the right to represent a crossing. For example
it picks out processes that may perform an input on the port $x_0$ in
its initial menu of capabilities. What differentiates the formula
from the process, however, is that the crossing process is the
smallest candidate to satisfy the formula. Infinitely many other
processes -- with internal behavior hidden behind this interface, so
to speak -- also satisfy this formula. Even this simple formula,
then, can be seen to open a new view onto knots, providing a
computational interpretation of \emph{virtual} knots.

Note that this formula is derived by hand. A similar formula can be
derived by employing Caires' calculation of characteristic formula
\cite{Caires04} to the process representing a crossing. In light of
this discussion, we let
$\meaningof{C}_{\phi}(x0,x1,y0,y1,u)$ denote a formula specifying the
dynamics we wish to capture of a crossing. To guarantee we preserve
the shape of the interface and minimal semantics we demand that
$\meaningof{C}_{\phi}(x0,x1,y0,y1,u) \Rightarrow
\textbf{C}(x0,x1,y0,y1,u)$ where $\textbf{C}(x0,x1,y0,y1,u)$ denotes
the formula above.
                            
\subsubsection{Crossing number constraints.}
The moral content of the context lemma (Lemma \ref{context}) is that the notion of
``locality'' in the Reidemeister moves is effectively captured by the
parallel composition operator of the process calculus. This intuition
extends through the logic. Given a formula,
$\meaningof{C}_{\phi}(x0,x1,y0,y1,u)$, we can use the structural
connectives to specify constraints on crossing numbers, such as at
least $n$ crossings, or exactly $n$ crossings.
\begin{mathpar}
  \inferrule* [lab=at-least-n] {} { K^{\geq n}_{\phi}(\vec{xs},\vec{ys}) := \Pi_{i=0}^{n-1} Hu . \meaningof{C}_{\phi}(xs_i,ys_i,u) | T }
  \and 
  \inferrule* [lab=exactly-n] {} { K^{= n}_{\phi}(\vec{xs},\vec{ys}) := \Pi_{i=0}^{n-1} Hu . \meaningof{C}_{\phi}(xs_i,ys_i,u) | \neg (\forall x_0,y_0,x_1,y_1,u . \meaningof{C}_{\phi}(x_0,y_0,x_1,y_1,u) | T) }
\end{mathpar}

To round out this section, recall that the encoding of an $n$-crossing
knot decomposes into a parallel composition of $n$ \emph{copies} of a
crossing process together with a wiring harness. To specify different
knot classes with the same crossing number amounts to specifying
logical constraints on the wiring harness. In the interest of space,
we defer examples to a forthcoming paper. Suffice it to say that both
the conditions ``alternating knot'' and ``contains the tangle
corresponding to 5/3'' are expressible. For example, it is possible to
calculate the characteristic formula of a process corresponding to the
tangle 5/3 and conjoin it into the classifying formula via the
composition connective of the logic.

Finally, we wish to observe that it is entirely within reason to
contemplate a more domain-specific version of spatial logic tailored
to the shape of processes in the image of the encoding. Such a
domain-specific logic would have a better claim to the title formal
language of knot properties.

% subsection example_formulae_ (end)

% section knots_as_processes (end) 

% section spatial logic via knots (end)

\section{Conclusions and future work}

\paragraph{Testing physical space}
You, gentle reader, may wonder why of all the theorems to be proved
given this set up we pick the one above. In some sense it's hardly
central to quantum mechanics. We see it as central in the sense that
it firmly establishes a notion of physical space arising from a notion
of the equivalence of behavior. Relating bisimulation to a metric is a
big step forward, but one is faced with interpreting the relationship
of that metric space to something more physical. Quantum mechanical
notions of ``physical'' space are still far from intuitive, but by
relating this idea of distance as testing to calculations that predict
physical circumstances we are making a not insignificant step forward
toward an understanding of the physical space we inhabit as
essentially dynamic.

\paragraph{Effectivity and simulation}
One of the observations we have yet to make is that the entire program
spelled out here is effective. We have built various interpreters for
the reflective calculus at work in this interpretation. In principle,
then, we can simulate quantum mechanics on a computer. The place where
the simulation may lose fidelity is the infinitely branching summation
for the annihilator.

In this connection i also want to point out that the evaluation style
calculation of the inner product puts the non-determinism of the
summation right at the heart of measurement. This suggests that
Milner's original reduction-based formulation of the dynamics of his
calculi in terms of sums was not just notationally suggestive of a
notion of measure-and-continue but captured some significant part of
the physics.

\paragraph{Quantum continuations}
In light of this last observation i want to point out that the
predominant account of quantum mechanics is missing a key aspect of a
truly compositional story of the physical situation. In a real lab,
when a measurement is made the observation can be made to feed into
another device that then makes another measurement conditioned on the
results of the first. This means that after the superposition was
collapsed the entire experimental set up remained in
superposition. While QM offers a means of writing this down it doesn't
quite line up well with the well-trodden formulation of computation
and continuation that we see so succinctly expressed in Milner's
calculi. This suggests that there might be advantages to this account
of dynamics waiting to be explored.

\paragraph{Quantum logic}
In this connection, we also note that by virtue of having the
Hennessy-Milner construction, we can pull the construction through the
interpretation of QM. This gives us a natural candidate for a quantum
logic that enjoys an extremely tight connection with it's domain of
interpretation, making the construction much less ad hoc (rather it is
the image of functor!).

\paragraph{Quantum probabiity}
i have questions about the basis of the interpretation of inner
product as probability amplitude. In particular, using which
axiomatization of probability theory does the notion of probability
amplitude earn the right to be so dubbed? In other words, where is the
proof that the operation for calculating a probability amplitude (and
then squaring) satisfies the axioms of what it means to calculate a
probability? Even if such a proof exists (i have yet to find it in the
literature), i wonder if it might not be possible to turn things on
their heads. Can we view the calculation of the probability amplitude
as an axiomatization of probability? If so, then the definition we
give for calculating probability amplitude may provide the basis for
an \emph{effective} theory of probability.

\paragraph{Quantum vs ``biological'' information}
Finally, i want to conclude with a more philosophical observation. At
a recent workshop in which QM was a predominant topic i noticed
something about quantum information. The speaker was giving a riveting
discussion of axiomatic QM and showing how properties of ``no
cloning'' and ``no deleting'' emerged as consequences of the
axiomatization. Theorems of this form are necessary to give us a sense
of confidence that our axioms characterize the physical theory. What
struck me, though, was that if quantum information is neither erasable
nor replicable it is markedly different from \emph{life}. Two of the
things we know about life is that

\begin{itemize}
  \item it ends;
  \item to gain some measure of persistence, to transcend it's
    finitude it is imminently copyable.
\end{itemize}

Both of these qualities are summarized succinctly in the aphorism: all
flesh is grass. For me these two kinds of ``information'' -- call them
quantum and biological -- are end points on a spectrum of strategies
for persistence. At one end, we have those curious entities that enjoy
uniqueness and permanence; at the other, we have those who in the face
of a certain end and an uncertain present make a go of passing
something on. To me one of the more remarkable aspects of the latter
strategy is that in the presence of noise (and certain features of
copying) we get a kind of dynamism, a chance for improvement against a
given persistent condition.

% subsection other_calculi_other_bisimulations_and_geometry_as_behavior (end)




% section conclusion (end)

%\documentclass[12pt]{llncs}
%\documentclass{jktr}

\usepackage[pdftex]{hyperref}                   
\usepackage {listings}
\usepackage {mathpartir}
\usepackage{bcprules}
%\usepackage{listings}
                       
\usepackage{graphicx} 
%\usepackage[margins=2.5cm,nohead,nofoot]{geometry}
%\usepackage{geometry}
\usepackage{amsfonts}
\usepackage{amstext}
\usepackage{latexsym}
\usepackage{amssymb}
\usepackage{color}


%\include{myPreamble}
\include{qm2pi.local} 

%\ifpdf
%\usepackage[pdftex]{graphicx}
%\else
%\usepackage{graphicx}
%\fi

 % \ifpdf
%  \usepackage{pdfsync}
%  \if


%\title{Brief Article}
%\author{David F. Snyder}
%\author{L.G. Meredith}

%\address{Dept. of Math., Texas State University--San Marcos, San Marcos, TX 78666}
       
\pagestyle{empty}


\begin{document}

\lstset{language=[Objective]Caml,frame=shadowbox}

\input{qm2pi.front}

% section front matter (end)

\input{qm2pi.intro} 
 
% section introduction (end)

% \input{qm2pi.knotations} 

% section notation (end)

\input{qm2pi.process.calculi} 

% section concurrent_process_calculi_and_spatial_logics_ (end)
    
%\input{qm2pi.knots2pi} 

%\input{qm2pi.trefoil} 

%\input{qm2pi.mainthm} 

% subsection basic_interpretation (end)

%\input{qm2pi.rho.presentation} 
\subsection{The syntax and semantics of the notation system}\label{sub:the_syntax_and_semantics_of_the_notation_system} % (fold)

We now summarize a technical presentation of the calculus that
embodies our theory of dynamics. The typical presentation of such a
calculus follows the style of giving generators and relations on
them. The grammar, below, describing term constructors, freely
generates the set of processes, $\Proc$. This set is then quotiented
by a relation known as structural congruence and it is over this set
that the notion of dynamics is expressed. This presentation is
essentially that of \cite{MeredithR05} with the addition of
polyadicity and summation. For readability we have relegated some of
the technical subtleties to an appendix.

\subsubsection{Process grammar}\label{subsub:process_grammar}

\begin{mathpar}
  \inferrule* [lab=synchronization] {} {{M} \bc \pzero \;|\; x?F \;|\; x!C }
  \and
  \inferrule* [lab=abstraction] {} {{F} \bc (x)P}
  \and
  \inferrule* [lab=concretion] {} {{C} \bc \langle Q \rangle}
  \and
  \inferrule* [lab=process] {} {{P,Q} \bc M \;| \;P|Q \;|\; @{x}}
  \and
  \inferrule* [lab=name] {} {{x} \bc \quotep{P}}
\end{mathpar} 

Note that $\vec{x}$ (resp. $\vec{P}$) denotes a vector of names
(resp. processes) of length $|\vec{x}|$ (resp. $|\vec{P}|$). We adopt
the following useful abbreviations.

\begin{mathpar}
   x?(\vec{y}).P := x.(\vec{y})P \and  x\clift{\vec{P}} := x.\clift{\vec{P}}
   \and x!(y) := \lift{x}{\dropn{y}}
   \and \Pi_{i=0}^{n-1}P_i := P_0 | \ldots | P_{n-1}
\end{mathpar}

\subsubsection{Structural congruence}

\paragraph{Free and bound names and alpha-equivalence.} At the
core of structural equivalence is alpha-equivalence which identifies
process that are the same up to a change of variable. Formally, we
recognize the distinction between free and bound names. The free names
of a process, $\freenames{P}$, may be calculated recursively as
follows:

\begin{mathpar}
\freenames{\pzero} := \emptyset
  \and \\
  \freenames{x?(y).P} := \{ x \} \cup (\freenames{P} \setminus \{ y \})
  \and 
  \freenames{x!\langle P \rangle} := \{ x \} \cup \{ P \} 
  \and \\
  \freenames{P|Q} := \freenames{P} \cup \freenames{Q}
  \and \\
  \freenames{@{x}} := \{ x \}
\end{mathpar}

$\pi$
$\quotep{\pi}$

$\freenames{-} : \pi \to \mathcal{P}(\quotep{\pi})$

\begin{eqnarray*}
  \freenames{\pzero} & := & \emptyset \\
  \freenames{x?(y).P} & := & \{ x \} \cup (\freenames{P} \setminus \{ y \}) \\
  \freenames{x!\langle P \rangle} & := & \{ x \} \cup \{ P \} \\
  \freenames{P|Q} & := & \freenames{P} \cup \freenames{Q} \\
  \freenames{\dropn{x}} & := & \{ x \}
\end{eqnarray*}

The bound names of a process, $\boundnames{P}$, are those names occurring in $P$
that are not free. For example, in $x?(y).0$, the name $x$ is free, while $y$ is bound.

\begin{mathpar}
  \inferrule* [lab=monoidal-laws] {} { P|Q \equiv Q|P \and P|0 \equiv P \and P|(Q|R) \equiv (P|Q)|R }
\end{mathpar}

\begin{mathpar}
  \inferrule* [lab=alpha-equivalence] {} { (x)P \equiv (y)P\{y/x\} \and y \not\in \freenames{P} }
\end{mathpar}

\begin{definition}
Then two processes, $P,Q$, are alpha-equivalent if $P = Q\{\vec{y}/\vec{x}\}$ for
some $\vec{x} \in \boundnames{Q},\vec{y} \in \boundnames{P}$, where $Q\{\vec{y}/\vec{x}\}$
denotes the capture-avoiding substitution of $\vec{y}$ for $\vec{x}$ in $Q$.
\end{definition}

\begin{definition}
  The {\em structural congruence} \cite{SangiorgiWalker} , $\equiv$,
  between processes is the least congruence containing
  alpha-equivalence, satisfying the abelian monoid laws
  (associativity, commutativity and $\pzero$ as identity) for parallel
  composition $|$ and for summation $+$.
\end{definition}

\subsection{Name equivalence}

We take name equivalence, written $\nameeq$, to be the smallest
equivalence relation generated by the following rules.

\begin{mathpar}
\inferrule*[lab=Quote-drop]
{ }
{ \quotep{@{x}} \nameeq x }

\inferrule*[lab=Struct-equiv]
{ P \scong Q }
{ \quotep{P} \nameeq \quotep{Q} }
\end{mathpar}

The astute reader will have noticed that the mutual recursion of names
and processes imposes a mutual recursion on alpha-equivalence and
structural equivalence via name-equivalence. Fortunately, all of this
works out pleasantly and we may calculate in the natural way, free of
concern. The reader interested in the details is referred to the
appendix \ref{appendix:rho_details}.

\subsection{Substitution}

We use $\Proc$ for the set of processes, $\QProc$ for the set of
names, and $\id{\{}\vec{y} / \vec{x} \id{\}}$ to denote partial maps,
$s : \QProc \rightarrow \QProc$. A map, $s$ lifts, uniquely, to a map
on process terms, $\widehat{s} : \Proc \rightarrow \Proc$ by the
following equations.

\begin{mathpar}
  (0) \psubstp{Q}{P} := 0 \\
  (R \juxtap S) \psubstp{Q}{P}
  :=    
  (R)\psubstp{Q}{P} \juxtap (S) \psubstp{Q}{P} \\
  (x?(y).R) \psubstp{Q}{P}    
  :=    
  (x)\substp{Q}{P} (z)\concat( (R \psubstn{z}{y}) \psubstp{Q}{P} ) \\
  (\lift{x}{R}) \psubstp{Q}{P}  
  :=
  \lift{(x)\substp{Q}{P}}{ R \psubstp{Q}{P} } \\
%   (\dropn{x})  \psubstp{Q}{P}       
%   := 
%   \left\{ 
%     \begin{array}{ccc} 
%       \dropn{\quotep{Q}} & & x \nameeq \quotep{P} \\
%       \dropn{x} & & otherwise \\
%     \end{array}
%   \right. 
  (\dropn{x})  \psubstp{Q}{P}       
  := 
  \left\{ 
    \begin{array}{ccc} 
      Q & & x \nameeq \quotep{P} \\
      \dropn{x} & & otherwise \\
    \end{array}
  \right.
\end{mathpar}
 

where

\begin{eqnarray}
  (x)\id{\{} \lpquote Q \rpquote / \lpquote P \rpquote \id{\}}            = 
  \left\{ 
    \begin{array}{ccc}
      \lpquote Q \rpquote & & x \nameeq \lpquote P \rpquote \\
      x & & otherwise \\
    \end{array}
  \right. \nonumber
\end{eqnarray}

and $z$ is chosen distinct from $\quotep{P}$, $\quotep{Q}$, the free
names in $Q$, and all the names in $R$. Our $\alpha$-equivalence will
be built in the standard way from this substitution.

\begin{remark}\label{rem:no_self_referential_names}
  One consequence of these definitions is that $\forall P. \quotep{P}
  \not\in \freenames{P}$.
\end{remark}

\subsection{ Dynamic quote: an example }

Anticipating something of what's to come, consider applying the
substitution, $\widehat{\id{\{}u / z \id{\}}}$, to the following pair
of processes, $\lift{w}{y!(z)}$ and $w[ \lpquote y!(z) \rpquote ]$.

\begin{eqnarray}
	\lift{w}{y!(z)}\widehat{\id{\{}u / z \id{\}}}
		& = &
		\lift{w}{y!(u)} \nonumber\\
	w[ \lpquote y!(z) \rpquote ] \widehat{ \id{\{}u / z \id{\}} }
		& = &
		w[ \lpquote y!(z) \rpquote ] \nonumber
\end{eqnarray}

Because the body of the process between quotes is impervious to
substitution, we get radically different answers. In fact, by
examining the first process in an input context,
e.g. $x?(z).\lift{w}{y!(z)}$, we see that the process under the lift
operator may be shaped by prefixed inputs binding a name inside it. In
this sense, the lift operator will be seen as a way to dynamically
construct processes before reifying them as names.

Finally equipped with these standard features we can present the
dynamics of the calculus.

\subsubsection{Operational semantics} 

Finally, we introduce the computational dynamics. What marks these
algebras as distinct from other more traditionally studied algebraic
structures, e.g. vector spaces or polynomial rings, is the manner in
which dynamics is captured. In traditional structures, dynamics is typically
expressed through morphisms between such structures, as in linear maps
between vector spaces or morphisms between rings. In algebras
associated with the semantics of computation, the dynamics is
expressed as part of the algebraic structure itself, through a
reduction reduction relation typically denoted by $\red$. Below, we
give a recursive presentation of this relation for the calculus used
in the encoding.

$\red \subseteq \pi \times \pi$
$\red : \pi \to \mathcal{P}(\pi)$

\begin{mathpar}
  \inferrule* [lab=Comm] { \textsf{match}( x_{src}, x_{trgt} ) } { x_{trgt}?(y)P \; | \; x_{src}!\langle {Q} \rangle \red P\{\quotep{Q}/y}\} }
  \and \\
  \inferrule* [lab=Par] {{P} \red {P}'} {{{P} | {Q}} \red {{P}' | {Q}}}
  \and
  \inferrule* [lab=Equiv]{{{P} \scong {P}'} \andalso {{P}' \red {Q}'} \andalso {{Q}' \scong {Q}}}{{P} \red {Q}}
\end{mathpar}

\begin{eqnarray*}
  match_{\equiv} (\quotep{P},\quotep{Q}) & := & P \equiv Q \\
  match_{\dagger}(\quotep{P},\quotep{Q}) & := & \forall R. P|Q \red^{*} R => R \red^{*} 0 \\
  match_{K}(\quotep{P},\quotep{Q}) & := & K \mbox{ for some context } K
\end{eqnarray*}

$u?(x)P | u!\langle Q \rangle \red P\{\quotep{Q}/x\}$

%We write $\wred$ for $\red^*$, and $P\red$ if $\exists Q $ such that $ P \red Q$.
We write $P\red$ if $\exists Q $ such that $ P \red Q$ and $P\not\red$, otherwise.

\section{Replication}

As mentioned before, it is known that replication (and hence
recursion) can be implemented in a higher-order process algebra
\cite{SangiorgiWalker}. As our first example of calculation with the
machinery thus far presented we give the construction explicitly in
the {\rhoc}.

\begin{eqnarray}
	D_{x} & := & \prefix{x}{y}{(\binpar{\outputp{x}{y}}{@{y}})} \nonumber\\
	\bangp_{x}{P} & := & \binpar{{x}!\langle{\binpar{D_{x}}{P}}\rangle}{D_{x}} \nonumber
\end{eqnarray}

\begin{eqnarray}
	\bangp_{x}{P} & & \nonumber\\
	=
	& {x}!\langle{(\prefix{x}{y}{(\outputp{x}{y} | @{y})) | P}}\rangle 
	      | \prefix{x}{y}{(\outputp{x}{y} | @{y})} & \nonumber\\
	\red
	& (\outputp{x}{y} | @{y})\substn{\quotep{(\prefix{x}{y}{(@{y} | \outputp{x}{y})) | P}}}{y} & \nonumber\\
	=
	& \outputp{x}{\quotep{(\prefix{x}{y}{(\outputp{x}{y} | @{y})) | P}}}
	  | {(\prefix{x}{y}{(\outputp{x}{y} | @{y})) | P}} & \nonumber\\
	\red
	& \ldots & \nonumber\\
	\red^*
	& P | P | \ldots & \nonumber
\end{eqnarray}

Of course, this encoding, as an implementation, runs away, unfolding
$\bangp{P}$ eagerly. A lazier and more implementable replication
operator, restricted to input-guarded processes, may be obtained as follows.

\begin{eqnarray}
\bangp{\prefix{u}{v}{P}} 
	:= 
	\binpar{\lift{x}{\prefix{u}{v}{(\binpar{D(x)}{P})}}}{D(x)} \nonumber
\end{eqnarray}

\begin{remark}
  Note that the lazier definition still does not deal with summation
  or mixed summation (i.e. sums over input and output). The reader is
  invited to construct definitions of replication that deal with these
  features. 

  Further, the definitions are parameterized in a name, $x$. Can you,
  gentle reader, make a definition that eliminates this parameter and
  guarantees no accidental interaction between the replication
  machinery and the process being replicated -- i.e. no accidental
  sharing of names used by the process to get its work done and the
  name(s) used by the replication to effect copying. This latter
  revision of the definition of replication is crucial to obtaining
  the expected identity $!!P \sim !P$.
\end{remark}

\begin{remark}\label{rem:paradoxical_combinator}
  The reader familiar with the lambda calculus will have noticed the
  similarity between $D$ and the paradoxical combinator.

  [Ed. note: the existence of this seems to suggest we have to be more
  restrictive on the set of processes and names we admit if we are to
  support no-cloning.]
\end{remark}

\subsubsection{Bisimulation}

The computational dynamics gives rise to another kind of equivalence,
the equivalence of computational behavior. As previously mentioned
this is typically captured \emph{via} some form of bisimulation.

% The notion we use in this paper is weak barbed bisimulation
% \cite{milner91polyadicpi}.

The notion we use in this paper is derived from weak barbed
bisimulation \cite{milner91polyadicpi}. 

\begin{definition}
An \emph{observation relation}, $\downarrow_{\mathcal N}$, over a set
of names, $\mathcal N$, is the smallest relation satisfying the rules
below.

\infrule[Out-barb]{y \in {\mathcal N}, \; x \nameeq y}
		  {\outputp{x}{v} \downarrow_{\mathcal N} x}
\infrule[Par-barb]{\mbox{$P\downarrow_{\mathcal N} x$ or $Q\downarrow_{\mathcal N} x$}}
		  {\binpar{P}{Q} \downarrow_{\mathcal N} x}

We write $P \Downarrow_{\mathcal N} x$ if there is $Q$ such that 
$P \wred Q$ and $Q \downarrow_{\mathcal N} x$.
\end{definition}

\begin{definition}
%\label{def.bbisim}
An  ${\mathcal N}$-\emph{barbed bisimulation} over a set of names, ${\mathcal N}$, is a symmetric binary relation 
${\mathcal S}_{\mathcal N}$ between agents such that $P\rel{S}_{\mathcal N}Q$ implies:
\begin{enumerate}
\item If $P \red P'$ then $Q \wred Q'$ and $P'\rel{S}_{\mathcal N} Q'$.
\item If $P\downarrow_{\mathcal N} x$, then $Q\Downarrow_{\mathcal N} x$.
\end{enumerate}
$P$ is ${\mathcal N}$-barbed bisimilar to $Q$, written
$P \wbbisim_{\mathcal N} Q$, if $P \rel{S}_{\mathcal N} Q$ for some ${\mathcal N}$-barbed bisimulation ${\mathcal S}_{\mathcal N}$.
\end{definition}

$\mathcal{R} \subseteq \pi \times \pi$

$P \mathcal{R} Q => \forall P'. P \red P' \Rightarrow \exists Q'. Q \red Q', P' \mathcal{R} Q'$

$P \vdash x \Rightarrow Q \vdash x$

\begin{mathpar}
  \inferrule*[lab=Out-barb]{x \nameeq y}{{y}!\langle{Q}\rangle \vdash x}
  \and
  \inferrule*[lab=Par-barb]{\mbox{$P\vdash x$ or $Q\vdash x$}}{\binpar{P}{Q} \vdash x}
\end{mathpar}

\subsubsection{Contexts}

One of the principle advantages of computational calculi like the
$\pi$-calculus is a well-defined notion of context,
contextual-equivalence and a correlation between
contextual-equivalence and notions of bisimulation. The notion of
context allows the decomposition of a process into (sub-)process and
its syntactic environment, its context. Thus, a context may be
thought of as a process with a ``hole'' (written $\Box$) in it. The
application of a context $M$ to a process $P$, written $M[P]$, is
tantamount to filling the hole in $M$ with $P$. In this paper we do
not need the full weight of this theory, but do make use of the notion
of context in the proof the main theorem. 

\begin{mathpar}
  \inferrule* [lab=summation] {} {{M_{M},M_{N}} \bc \Box \;|\; x.M_{A} \;|\; M_{M}+M_{N}}
  \and
  \inferrule* [lab=agent] {} {{M_{A}} \bc (\vec{x})M_{P} \;| \; \clift{P_0,\ldots,M_{P},\ldots,P_N}}
  \and \\
  \inferrule* [lab=process] {} {{M_{P}} \bc M_{N} \;| \;P|M_{P} }
\end{mathpar} 

\begin{mathpar}
  \inferrule* [lab=sychronization] {} {M_{N} \bc \Box \;|\; x?M_{F} \;|\; x!M_{C}}
  \and
  \inferrule* [lab=abstraction] {} {{M_{F}} \bc (x)M_{P} }
  \and
  \inferrule* [lab=concretion] {} {{M_{C}} \bc \langle M_{P} \rangle }
  \and \\
  \inferrule* [lab=process] {} {{M_{P}} \bc M_{N} \;| \;P|M_{P} }
\end{mathpar}

\begin{definition}[contextual application] Given a context $M$, and
  process $P$, we define the \emph{contextual application}, $M[P] :=
  M\{P/\Box\}$. That is, the contextual application of M to P is the
  substitution of $P$ for $\Box$ in $M$.
\end{definition}

$\meaningof{-} : L \to \mathcal{P}(\pi)$

\begin{mathpar}
  \inferrule* [lab=collection] {} {\meaningof{true} = \pi, \and \meaningof{~E} = \pi \setminus \meaningof{E}, \and \meaningof{E_{1} \& E_{2}} = \meaningof{E_{1}} \cap \meaningof{E_{2}}}
\end{mathpar}

\begin{mathpar}
  \inferrule* [lab=structure] {} {\meaningof{0} = \{ P \in \pi | P \equiv 0 \}, \and \\ \meaningof{E_1 | E_2} = \{ P \in \pi | P \equiv P_{1} | P_{2}, P_{1} \in \meaningof{E_{1}}, P_{2} \in \meaningof{E_2}\} }
\end{mathpar}

\begin{mathpar}
 \inferrule* [lab=behavior] {} {\meaningof{\langle a?b \rangle E} = \{ P \in \pi | P \equiv Q | u?(y)P', \\ \and \\\\ \and \\ \;\;\; u \in \meaningof{a}, \forall z.P'\{z/y\} \in \meaningof{E\{z/b\}}\}, \and \\ \meaningof{a!E} = \{ P \in \pi | P \equiv Q | x!\langle P' \rangle, x \in \meaningof{a} P' \in \meaningof{E}\} }
\end{mathpar}

\begin{mathpar}
 \inferrule* [lab=nominal] {} {\meaningof{\quotep{E}} = \{ \quotep{P} \in \quotep{\pi} | P \in \meaningof{E} \}, \and \meaningof{\quotep{P}} = \{ \quotep{Q} \in \quotep{\pi} | P \equiv Q \} \and \\ \meaningof{@\quotep{E}} = \{ P \in \pi | P \equiv @x, x \in \meaningof{E} \}}
\end{mathpar}

\begin{eqnarray*}
  \\
  \meaningof{-} : TS \to ST
\end{eqnarray*}

\begin{eqnarray*}
  \\
  L : TS \to ST
\end{eqnarray*}

\begin{eqnarray*}
  \\
  P \models E \iff P \in \meaningof{E}
\end{eqnarray*}

\begin{eqnarray*}
  P \approx_{L} Q \iff \forall E \in L. P \models E \iff Q \models E
\end{eqnarray*}

\begin{eqnarray*}
  P \approx_{K} Q
\end{eqnarray*}

\begin{eqnarray*}
  P \approx Q
\end{eqnarray*}

$\approx_{K} = \approx = \approx_{L}$

\subsubsection{Contextual duality}

Note that contexts extend the quotation operation to a family of
operations from processes to names. Given a context, $M$, we can
define a \emph{nominal context}, $\quotep{M}$ by $\quotep{M}[P] :=
\quotep{M[P]}$. To foreshadow what is to come we observe that these
operations enjoy a duality with processes very much like the duality
between vectors and maps from vectors to scalars.

Further, because the calculus is essentially higher-order, we have a
correspondence between contexts and processes. More specifically,
given a name $x$ and a context $M$ we can construct $M^{*}_{x}$ such
that 

\begin{mathpar}
  M^{*}_{x} | \lift{x}{P} \red M[P]
\end{mathpar}

namely,

\begin{mathpar}
  M^{*}_{x} := x?(u).M[\dropn{u}]
\end{mathpar}

The dependence of $M^{*}_{x}$ on a name makes it an abstraction, 

\begin{mathpar}
  M^{*} := (x)x?(u).M[\dropn{u}]
\end{mathpar}

\subsection{Additional notation}

It will sometimes be convenient to denote the process a name
quotes. We already have the notation $x = \quotep{P}$, but it will be
convenient to introduce an alternate notation, $\procn{x}$, when we
want to emphasize the connection to the use of the name. Note that, by
virtue of name equivalence, $\quotep{\procn{x}} \nameeq x$; so, the
notation is consistent with previous definitions.

Further, because names have structure it is possible to effect
substitutions on the basis of that structure. This means we need to
upgrade our notation for substitutions, which we accomplish by
adapting comprehension notation. Thus,

\begin{mathpar}
  P\{ y / x : x \in S \}
\end{mathpar}

is interpreted to mean the process derived from P by replacing (in a
capture-avoiding manner) each occurrence of $x$ in $S$ by $y$. For example,

\begin{mathpar}
  P\{ \quotep{\procn{x}|\procn{x}} / x : x \in \freenames{P} \}
\end{mathpar}

will replace each (occurrence) of a free name $x$ in $P$ by
$\quotep{\procn{x}|\procn{x}}$.

Also, we will avail ourselves of the notation $x^{L}$ and $x^{R}$ to
denote injections of a name into disjoint copies of the name
space. There are numerous ways to accomplish this. One example can be
found in \cite{MeredithR05}. This notation overloads to vectors of
names: $\vec{x}^{\pi} := (x_{i}^{\pi} \; : \; 0 \leq i < |\vec{x}| )$ where $\pi \in \{L,R\}$.

We also use $P^{\Box} := P|\Box$.

In \cite{MeredithR05} an interpretation of the new operator is
given. It turns out that there are several possible interpretations
all enjoying the requisite algebraic properties of the operator (see
\cite{milner91polyadicpi}). We will therefore make liberal use of
$(\nu\; \vec{x})P$.

% subsection the_syntax_and_semantics_of_the_notation_system (end)   

\input{qm2pi.qmops} 

\input{qm2pi.sterngerlach} 

\input{qm2pi.metric} 

% section concurrent_process_calculi (end)

%\input{qm2pi.proofsketch}

% section proof sketch (end)

%\input{qm2pi.slviaknots} 

% section spatial logic via knots (end)

\input{qm2pi.conclusion}

% section conclusion (end)

%\input{qm2pi.dtcodes} 

% section wiring algorithm (end)

\input{qm2pi.ack} 

% section acknowledgments (end)

\newpage


\bibliographystyle{plain}   
\bibliography{../../biblios/main.bib}

\input{qm2pi.rhodetails}

\end{document}

 

% section wiring algorithm (end)

\documentclass[12pt]{llncs}
%\documentclass{jktr}

\usepackage[pdftex]{hyperref}                   
\usepackage {listings}
\usepackage {mathpartir}
\usepackage{bcprules}
%\usepackage{listings}
                       
\usepackage{graphicx} 
%\usepackage[margins=2.5cm,nohead,nofoot]{geometry}
%\usepackage{geometry}
\usepackage{amsfonts}
\usepackage{amstext}
\usepackage{latexsym}
\usepackage{amssymb}
\usepackage{color}


%\include{myPreamble}
\include{qm2pi.local} 

%\ifpdf
%\usepackage[pdftex]{graphicx}
%\else
%\usepackage{graphicx}
%\fi

 % \ifpdf
%  \usepackage{pdfsync}
%  \if


%\title{Brief Article}
%\author{David F. Snyder}
%\author{L.G. Meredith}

%\address{Dept. of Math., Texas State University--San Marcos, San Marcos, TX 78666}
       
\pagestyle{empty}


\begin{document}

\lstset{language=[Objective]Caml,frame=shadowbox}

\input{qm2pi.front}

% section front matter (end)

\input{qm2pi.intro} 
 
% section introduction (end)

% \input{qm2pi.knotations} 

% section notation (end)

\input{qm2pi.process.calculi} 

% section concurrent_process_calculi_and_spatial_logics_ (end)
    
%\input{qm2pi.knots2pi} 

%\input{qm2pi.trefoil} 

%\input{qm2pi.mainthm} 

% subsection basic_interpretation (end)

%\input{qm2pi.rho.presentation} 
\subsection{The syntax and semantics of the notation system}\label{sub:the_syntax_and_semantics_of_the_notation_system} % (fold)

We now summarize a technical presentation of the calculus that
embodies our theory of dynamics. The typical presentation of such a
calculus follows the style of giving generators and relations on
them. The grammar, below, describing term constructors, freely
generates the set of processes, $\Proc$. This set is then quotiented
by a relation known as structural congruence and it is over this set
that the notion of dynamics is expressed. This presentation is
essentially that of \cite{MeredithR05} with the addition of
polyadicity and summation. For readability we have relegated some of
the technical subtleties to an appendix.

\subsubsection{Process grammar}\label{subsub:process_grammar}

\begin{mathpar}
  \inferrule* [lab=synchronization] {} {{M} \bc \pzero \;|\; x?F \;|\; x!C }
  \and
  \inferrule* [lab=abstraction] {} {{F} \bc (x)P}
  \and
  \inferrule* [lab=concretion] {} {{C} \bc \langle Q \rangle}
  \and
  \inferrule* [lab=process] {} {{P,Q} \bc M \;| \;P|Q \;|\; @{x}}
  \and
  \inferrule* [lab=name] {} {{x} \bc \quotep{P}}
\end{mathpar} 

Note that $\vec{x}$ (resp. $\vec{P}$) denotes a vector of names
(resp. processes) of length $|\vec{x}|$ (resp. $|\vec{P}|$). We adopt
the following useful abbreviations.

\begin{mathpar}
   x?(\vec{y}).P := x.(\vec{y})P \and  x\clift{\vec{P}} := x.\clift{\vec{P}}
   \and x!(y) := \lift{x}{\dropn{y}}
   \and \Pi_{i=0}^{n-1}P_i := P_0 | \ldots | P_{n-1}
\end{mathpar}

\subsubsection{Structural congruence}

\paragraph{Free and bound names and alpha-equivalence.} At the
core of structural equivalence is alpha-equivalence which identifies
process that are the same up to a change of variable. Formally, we
recognize the distinction between free and bound names. The free names
of a process, $\freenames{P}$, may be calculated recursively as
follows:

\begin{mathpar}
\freenames{\pzero} := \emptyset
  \and \\
  \freenames{x?(y).P} := \{ x \} \cup (\freenames{P} \setminus \{ y \})
  \and 
  \freenames{x!\langle P \rangle} := \{ x \} \cup \{ P \} 
  \and \\
  \freenames{P|Q} := \freenames{P} \cup \freenames{Q}
  \and \\
  \freenames{@{x}} := \{ x \}
\end{mathpar}

$\pi$
$\quotep{\pi}$

$\freenames{-} : \pi \to \mathcal{P}(\quotep{\pi})$

\begin{eqnarray*}
  \freenames{\pzero} & := & \emptyset \\
  \freenames{x?(y).P} & := & \{ x \} \cup (\freenames{P} \setminus \{ y \}) \\
  \freenames{x!\langle P \rangle} & := & \{ x \} \cup \{ P \} \\
  \freenames{P|Q} & := & \freenames{P} \cup \freenames{Q} \\
  \freenames{\dropn{x}} & := & \{ x \}
\end{eqnarray*}

The bound names of a process, $\boundnames{P}$, are those names occurring in $P$
that are not free. For example, in $x?(y).0$, the name $x$ is free, while $y$ is bound.

\begin{mathpar}
  \inferrule* [lab=monoidal-laws] {} { P|Q \equiv Q|P \and P|0 \equiv P \and P|(Q|R) \equiv (P|Q)|R }
\end{mathpar}

\begin{mathpar}
  \inferrule* [lab=alpha-equivalence] {} { (x)P \equiv (y)P\{y/x\} \and y \not\in \freenames{P} }
\end{mathpar}

\begin{definition}
Then two processes, $P,Q$, are alpha-equivalent if $P = Q\{\vec{y}/\vec{x}\}$ for
some $\vec{x} \in \boundnames{Q},\vec{y} \in \boundnames{P}$, where $Q\{\vec{y}/\vec{x}\}$
denotes the capture-avoiding substitution of $\vec{y}$ for $\vec{x}$ in $Q$.
\end{definition}

\begin{definition}
  The {\em structural congruence} \cite{SangiorgiWalker} , $\equiv$,
  between processes is the least congruence containing
  alpha-equivalence, satisfying the abelian monoid laws
  (associativity, commutativity and $\pzero$ as identity) for parallel
  composition $|$ and for summation $+$.
\end{definition}

\subsection{Name equivalence}

We take name equivalence, written $\nameeq$, to be the smallest
equivalence relation generated by the following rules.

\begin{mathpar}
\inferrule*[lab=Quote-drop]
{ }
{ \quotep{@{x}} \nameeq x }

\inferrule*[lab=Struct-equiv]
{ P \scong Q }
{ \quotep{P} \nameeq \quotep{Q} }
\end{mathpar}

The astute reader will have noticed that the mutual recursion of names
and processes imposes a mutual recursion on alpha-equivalence and
structural equivalence via name-equivalence. Fortunately, all of this
works out pleasantly and we may calculate in the natural way, free of
concern. The reader interested in the details is referred to the
appendix \ref{appendix:rho_details}.

\subsection{Substitution}

We use $\Proc$ for the set of processes, $\QProc$ for the set of
names, and $\id{\{}\vec{y} / \vec{x} \id{\}}$ to denote partial maps,
$s : \QProc \rightarrow \QProc$. A map, $s$ lifts, uniquely, to a map
on process terms, $\widehat{s} : \Proc \rightarrow \Proc$ by the
following equations.

\begin{mathpar}
  (0) \psubstp{Q}{P} := 0 \\
  (R \juxtap S) \psubstp{Q}{P}
  :=    
  (R)\psubstp{Q}{P} \juxtap (S) \psubstp{Q}{P} \\
  (x?(y).R) \psubstp{Q}{P}    
  :=    
  (x)\substp{Q}{P} (z)\concat( (R \psubstn{z}{y}) \psubstp{Q}{P} ) \\
  (\lift{x}{R}) \psubstp{Q}{P}  
  :=
  \lift{(x)\substp{Q}{P}}{ R \psubstp{Q}{P} } \\
%   (\dropn{x})  \psubstp{Q}{P}       
%   := 
%   \left\{ 
%     \begin{array}{ccc} 
%       \dropn{\quotep{Q}} & & x \nameeq \quotep{P} \\
%       \dropn{x} & & otherwise \\
%     \end{array}
%   \right. 
  (\dropn{x})  \psubstp{Q}{P}       
  := 
  \left\{ 
    \begin{array}{ccc} 
      Q & & x \nameeq \quotep{P} \\
      \dropn{x} & & otherwise \\
    \end{array}
  \right.
\end{mathpar}
 

where

\begin{eqnarray}
  (x)\id{\{} \lpquote Q \rpquote / \lpquote P \rpquote \id{\}}            = 
  \left\{ 
    \begin{array}{ccc}
      \lpquote Q \rpquote & & x \nameeq \lpquote P \rpquote \\
      x & & otherwise \\
    \end{array}
  \right. \nonumber
\end{eqnarray}

and $z$ is chosen distinct from $\quotep{P}$, $\quotep{Q}$, the free
names in $Q$, and all the names in $R$. Our $\alpha$-equivalence will
be built in the standard way from this substitution.

\begin{remark}\label{rem:no_self_referential_names}
  One consequence of these definitions is that $\forall P. \quotep{P}
  \not\in \freenames{P}$.
\end{remark}

\subsection{ Dynamic quote: an example }

Anticipating something of what's to come, consider applying the
substitution, $\widehat{\id{\{}u / z \id{\}}}$, to the following pair
of processes, $\lift{w}{y!(z)}$ and $w[ \lpquote y!(z) \rpquote ]$.

\begin{eqnarray}
	\lift{w}{y!(z)}\widehat{\id{\{}u / z \id{\}}}
		& = &
		\lift{w}{y!(u)} \nonumber\\
	w[ \lpquote y!(z) \rpquote ] \widehat{ \id{\{}u / z \id{\}} }
		& = &
		w[ \lpquote y!(z) \rpquote ] \nonumber
\end{eqnarray}

Because the body of the process between quotes is impervious to
substitution, we get radically different answers. In fact, by
examining the first process in an input context,
e.g. $x?(z).\lift{w}{y!(z)}$, we see that the process under the lift
operator may be shaped by prefixed inputs binding a name inside it. In
this sense, the lift operator will be seen as a way to dynamically
construct processes before reifying them as names.

Finally equipped with these standard features we can present the
dynamics of the calculus.

\subsubsection{Operational semantics} 

Finally, we introduce the computational dynamics. What marks these
algebras as distinct from other more traditionally studied algebraic
structures, e.g. vector spaces or polynomial rings, is the manner in
which dynamics is captured. In traditional structures, dynamics is typically
expressed through morphisms between such structures, as in linear maps
between vector spaces or morphisms between rings. In algebras
associated with the semantics of computation, the dynamics is
expressed as part of the algebraic structure itself, through a
reduction reduction relation typically denoted by $\red$. Below, we
give a recursive presentation of this relation for the calculus used
in the encoding.

$\red \subseteq \pi \times \pi$
$\red : \pi \to \mathcal{P}(\pi)$

\begin{mathpar}
  \inferrule* [lab=Comm] { \textsf{match}( x_{src}, x_{trgt} ) } { x_{trgt}?(y)P \; | \; x_{src}!\langle {Q} \rangle \red P\{\quotep{Q}/y}\} }
  \and \\
  \inferrule* [lab=Par] {{P} \red {P}'} {{{P} | {Q}} \red {{P}' | {Q}}}
  \and
  \inferrule* [lab=Equiv]{{{P} \scong {P}'} \andalso {{P}' \red {Q}'} \andalso {{Q}' \scong {Q}}}{{P} \red {Q}}
\end{mathpar}

\begin{eqnarray*}
  match_{\equiv} (\quotep{P},\quotep{Q}) & := & P \equiv Q \\
  match_{\dagger}(\quotep{P},\quotep{Q}) & := & \forall R. P|Q \red^{*} R => R \red^{*} 0 \\
  match_{K}(\quotep{P},\quotep{Q}) & := & K \mbox{ for some context } K
\end{eqnarray*}

$u?(x)P | u!\langle Q \rangle \red P\{\quotep{Q}/x\}$

%We write $\wred$ for $\red^*$, and $P\red$ if $\exists Q $ such that $ P \red Q$.
We write $P\red$ if $\exists Q $ such that $ P \red Q$ and $P\not\red$, otherwise.

\section{Replication}

As mentioned before, it is known that replication (and hence
recursion) can be implemented in a higher-order process algebra
\cite{SangiorgiWalker}. As our first example of calculation with the
machinery thus far presented we give the construction explicitly in
the {\rhoc}.

\begin{eqnarray}
	D_{x} & := & \prefix{x}{y}{(\binpar{\outputp{x}{y}}{@{y}})} \nonumber\\
	\bangp_{x}{P} & := & \binpar{{x}!\langle{\binpar{D_{x}}{P}}\rangle}{D_{x}} \nonumber
\end{eqnarray}

\begin{eqnarray}
	\bangp_{x}{P} & & \nonumber\\
	=
	& {x}!\langle{(\prefix{x}{y}{(\outputp{x}{y} | @{y})) | P}}\rangle 
	      | \prefix{x}{y}{(\outputp{x}{y} | @{y})} & \nonumber\\
	\red
	& (\outputp{x}{y} | @{y})\substn{\quotep{(\prefix{x}{y}{(@{y} | \outputp{x}{y})) | P}}}{y} & \nonumber\\
	=
	& \outputp{x}{\quotep{(\prefix{x}{y}{(\outputp{x}{y} | @{y})) | P}}}
	  | {(\prefix{x}{y}{(\outputp{x}{y} | @{y})) | P}} & \nonumber\\
	\red
	& \ldots & \nonumber\\
	\red^*
	& P | P | \ldots & \nonumber
\end{eqnarray}

Of course, this encoding, as an implementation, runs away, unfolding
$\bangp{P}$ eagerly. A lazier and more implementable replication
operator, restricted to input-guarded processes, may be obtained as follows.

\begin{eqnarray}
\bangp{\prefix{u}{v}{P}} 
	:= 
	\binpar{\lift{x}{\prefix{u}{v}{(\binpar{D(x)}{P})}}}{D(x)} \nonumber
\end{eqnarray}

\begin{remark}
  Note that the lazier definition still does not deal with summation
  or mixed summation (i.e. sums over input and output). The reader is
  invited to construct definitions of replication that deal with these
  features. 

  Further, the definitions are parameterized in a name, $x$. Can you,
  gentle reader, make a definition that eliminates this parameter and
  guarantees no accidental interaction between the replication
  machinery and the process being replicated -- i.e. no accidental
  sharing of names used by the process to get its work done and the
  name(s) used by the replication to effect copying. This latter
  revision of the definition of replication is crucial to obtaining
  the expected identity $!!P \sim !P$.
\end{remark}

\begin{remark}\label{rem:paradoxical_combinator}
  The reader familiar with the lambda calculus will have noticed the
  similarity between $D$ and the paradoxical combinator.

  [Ed. note: the existence of this seems to suggest we have to be more
  restrictive on the set of processes and names we admit if we are to
  support no-cloning.]
\end{remark}

\subsubsection{Bisimulation}

The computational dynamics gives rise to another kind of equivalence,
the equivalence of computational behavior. As previously mentioned
this is typically captured \emph{via} some form of bisimulation.

% The notion we use in this paper is weak barbed bisimulation
% \cite{milner91polyadicpi}.

The notion we use in this paper is derived from weak barbed
bisimulation \cite{milner91polyadicpi}. 

\begin{definition}
An \emph{observation relation}, $\downarrow_{\mathcal N}$, over a set
of names, $\mathcal N$, is the smallest relation satisfying the rules
below.

\infrule[Out-barb]{y \in {\mathcal N}, \; x \nameeq y}
		  {\outputp{x}{v} \downarrow_{\mathcal N} x}
\infrule[Par-barb]{\mbox{$P\downarrow_{\mathcal N} x$ or $Q\downarrow_{\mathcal N} x$}}
		  {\binpar{P}{Q} \downarrow_{\mathcal N} x}

We write $P \Downarrow_{\mathcal N} x$ if there is $Q$ such that 
$P \wred Q$ and $Q \downarrow_{\mathcal N} x$.
\end{definition}

\begin{definition}
%\label{def.bbisim}
An  ${\mathcal N}$-\emph{barbed bisimulation} over a set of names, ${\mathcal N}$, is a symmetric binary relation 
${\mathcal S}_{\mathcal N}$ between agents such that $P\rel{S}_{\mathcal N}Q$ implies:
\begin{enumerate}
\item If $P \red P'$ then $Q \wred Q'$ and $P'\rel{S}_{\mathcal N} Q'$.
\item If $P\downarrow_{\mathcal N} x$, then $Q\Downarrow_{\mathcal N} x$.
\end{enumerate}
$P$ is ${\mathcal N}$-barbed bisimilar to $Q$, written
$P \wbbisim_{\mathcal N} Q$, if $P \rel{S}_{\mathcal N} Q$ for some ${\mathcal N}$-barbed bisimulation ${\mathcal S}_{\mathcal N}$.
\end{definition}

$\mathcal{R} \subseteq \pi \times \pi$

$P \mathcal{R} Q => \forall P'. P \red P' \Rightarrow \exists Q'. Q \red Q', P' \mathcal{R} Q'$

$P \vdash x \Rightarrow Q \vdash x$

\begin{mathpar}
  \inferrule*[lab=Out-barb]{x \nameeq y}{{y}!\langle{Q}\rangle \vdash x}
  \and
  \inferrule*[lab=Par-barb]{\mbox{$P\vdash x$ or $Q\vdash x$}}{\binpar{P}{Q} \vdash x}
\end{mathpar}

\subsubsection{Contexts}

One of the principle advantages of computational calculi like the
$\pi$-calculus is a well-defined notion of context,
contextual-equivalence and a correlation between
contextual-equivalence and notions of bisimulation. The notion of
context allows the decomposition of a process into (sub-)process and
its syntactic environment, its context. Thus, a context may be
thought of as a process with a ``hole'' (written $\Box$) in it. The
application of a context $M$ to a process $P$, written $M[P]$, is
tantamount to filling the hole in $M$ with $P$. In this paper we do
not need the full weight of this theory, but do make use of the notion
of context in the proof the main theorem. 

\begin{mathpar}
  \inferrule* [lab=summation] {} {{M_{M},M_{N}} \bc \Box \;|\; x.M_{A} \;|\; M_{M}+M_{N}}
  \and
  \inferrule* [lab=agent] {} {{M_{A}} \bc (\vec{x})M_{P} \;| \; \clift{P_0,\ldots,M_{P},\ldots,P_N}}
  \and \\
  \inferrule* [lab=process] {} {{M_{P}} \bc M_{N} \;| \;P|M_{P} }
\end{mathpar} 

\begin{mathpar}
  \inferrule* [lab=sychronization] {} {M_{N} \bc \Box \;|\; x?M_{F} \;|\; x!M_{C}}
  \and
  \inferrule* [lab=abstraction] {} {{M_{F}} \bc (x)M_{P} }
  \and
  \inferrule* [lab=concretion] {} {{M_{C}} \bc \langle M_{P} \rangle }
  \and \\
  \inferrule* [lab=process] {} {{M_{P}} \bc M_{N} \;| \;P|M_{P} }
\end{mathpar}

\begin{definition}[contextual application] Given a context $M$, and
  process $P$, we define the \emph{contextual application}, $M[P] :=
  M\{P/\Box\}$. That is, the contextual application of M to P is the
  substitution of $P$ for $\Box$ in $M$.
\end{definition}

$\meaningof{-} : L \to \mathcal{P}(\pi)$

\begin{mathpar}
  \inferrule* [lab=collection] {} {\meaningof{true} = \pi, \and \meaningof{~E} = \pi \setminus \meaningof{E}, \and \meaningof{E_{1} \& E_{2}} = \meaningof{E_{1}} \cap \meaningof{E_{2}}}
\end{mathpar}

\begin{mathpar}
  \inferrule* [lab=structure] {} {\meaningof{0} = \{ P \in \pi | P \equiv 0 \}, \and \\ \meaningof{E_1 | E_2} = \{ P \in \pi | P \equiv P_{1} | P_{2}, P_{1} \in \meaningof{E_{1}}, P_{2} \in \meaningof{E_2}\} }
\end{mathpar}

\begin{mathpar}
 \inferrule* [lab=behavior] {} {\meaningof{\langle a?b \rangle E} = \{ P \in \pi | P \equiv Q | u?(y)P', \\ \and \\\\ \and \\ \;\;\; u \in \meaningof{a}, \forall z.P'\{z/y\} \in \meaningof{E\{z/b\}}\}, \and \\ \meaningof{a!E} = \{ P \in \pi | P \equiv Q | x!\langle P' \rangle, x \in \meaningof{a} P' \in \meaningof{E}\} }
\end{mathpar}

\begin{mathpar}
 \inferrule* [lab=nominal] {} {\meaningof{\quotep{E}} = \{ \quotep{P} \in \quotep{\pi} | P \in \meaningof{E} \}, \and \meaningof{\quotep{P}} = \{ \quotep{Q} \in \quotep{\pi} | P \equiv Q \} \and \\ \meaningof{@\quotep{E}} = \{ P \in \pi | P \equiv @x, x \in \meaningof{E} \}}
\end{mathpar}

\begin{eqnarray*}
  \\
  \meaningof{-} : TS \to ST
\end{eqnarray*}

\begin{eqnarray*}
  \\
  L : TS \to ST
\end{eqnarray*}

\begin{eqnarray*}
  \\
  P \models E \iff P \in \meaningof{E}
\end{eqnarray*}

\begin{eqnarray*}
  P \approx_{L} Q \iff \forall E \in L. P \models E \iff Q \models E
\end{eqnarray*}

\begin{eqnarray*}
  P \approx_{K} Q
\end{eqnarray*}

\begin{eqnarray*}
  P \approx Q
\end{eqnarray*}

$\approx_{K} = \approx = \approx_{L}$

\subsubsection{Contextual duality}

Note that contexts extend the quotation operation to a family of
operations from processes to names. Given a context, $M$, we can
define a \emph{nominal context}, $\quotep{M}$ by $\quotep{M}[P] :=
\quotep{M[P]}$. To foreshadow what is to come we observe that these
operations enjoy a duality with processes very much like the duality
between vectors and maps from vectors to scalars.

Further, because the calculus is essentially higher-order, we have a
correspondence between contexts and processes. More specifically,
given a name $x$ and a context $M$ we can construct $M^{*}_{x}$ such
that 

\begin{mathpar}
  M^{*}_{x} | \lift{x}{P} \red M[P]
\end{mathpar}

namely,

\begin{mathpar}
  M^{*}_{x} := x?(u).M[\dropn{u}]
\end{mathpar}

The dependence of $M^{*}_{x}$ on a name makes it an abstraction, 

\begin{mathpar}
  M^{*} := (x)x?(u).M[\dropn{u}]
\end{mathpar}

\subsection{Additional notation}

It will sometimes be convenient to denote the process a name
quotes. We already have the notation $x = \quotep{P}$, but it will be
convenient to introduce an alternate notation, $\procn{x}$, when we
want to emphasize the connection to the use of the name. Note that, by
virtue of name equivalence, $\quotep{\procn{x}} \nameeq x$; so, the
notation is consistent with previous definitions.

Further, because names have structure it is possible to effect
substitutions on the basis of that structure. This means we need to
upgrade our notation for substitutions, which we accomplish by
adapting comprehension notation. Thus,

\begin{mathpar}
  P\{ y / x : x \in S \}
\end{mathpar}

is interpreted to mean the process derived from P by replacing (in a
capture-avoiding manner) each occurrence of $x$ in $S$ by $y$. For example,

\begin{mathpar}
  P\{ \quotep{\procn{x}|\procn{x}} / x : x \in \freenames{P} \}
\end{mathpar}

will replace each (occurrence) of a free name $x$ in $P$ by
$\quotep{\procn{x}|\procn{x}}$.

Also, we will avail ourselves of the notation $x^{L}$ and $x^{R}$ to
denote injections of a name into disjoint copies of the name
space. There are numerous ways to accomplish this. One example can be
found in \cite{MeredithR05}. This notation overloads to vectors of
names: $\vec{x}^{\pi} := (x_{i}^{\pi} \; : \; 0 \leq i < |\vec{x}| )$ where $\pi \in \{L,R\}$.

We also use $P^{\Box} := P|\Box$.

In \cite{MeredithR05} an interpretation of the new operator is
given. It turns out that there are several possible interpretations
all enjoying the requisite algebraic properties of the operator (see
\cite{milner91polyadicpi}). We will therefore make liberal use of
$(\nu\; \vec{x})P$.

% subsection the_syntax_and_semantics_of_the_notation_system (end)   

\input{qm2pi.qmops} 

\input{qm2pi.sterngerlach} 

\input{qm2pi.metric} 

% section concurrent_process_calculi (end)

%\input{qm2pi.proofsketch}

% section proof sketch (end)

%\input{qm2pi.slviaknots} 

% section spatial logic via knots (end)

\input{qm2pi.conclusion}

% section conclusion (end)

%\input{qm2pi.dtcodes} 

% section wiring algorithm (end)

\input{qm2pi.ack} 

% section acknowledgments (end)

\newpage


\bibliographystyle{plain}   
\bibliography{../../biblios/main.bib}

\input{qm2pi.rhodetails}

\end{document}

 

% section acknowledgments (end)

\newpage


\bibliographystyle{plain}   
\bibliography{../../biblios/main.bib}

\documentclass[12pt]{llncs}
%\documentclass{jktr}

\usepackage[pdftex]{hyperref}                   
\usepackage {listings}
\usepackage {mathpartir}
\usepackage{bcprules}
%\usepackage{listings}
                       
\usepackage{graphicx} 
%\usepackage[margins=2.5cm,nohead,nofoot]{geometry}
%\usepackage{geometry}
\usepackage{amsfonts}
\usepackage{amstext}
\usepackage{latexsym}
\usepackage{amssymb}
\usepackage{color}


%\include{myPreamble}
\include{qm2pi.local} 

%\ifpdf
%\usepackage[pdftex]{graphicx}
%\else
%\usepackage{graphicx}
%\fi

 % \ifpdf
%  \usepackage{pdfsync}
%  \if


%\title{Brief Article}
%\author{David F. Snyder}
%\author{L.G. Meredith}

%\address{Dept. of Math., Texas State University--San Marcos, San Marcos, TX 78666}
       
\pagestyle{empty}


\begin{document}

\lstset{language=[Objective]Caml,frame=shadowbox}

\input{qm2pi.front}

% section front matter (end)

\input{qm2pi.intro} 
 
% section introduction (end)

% \input{qm2pi.knotations} 

% section notation (end)

\input{qm2pi.process.calculi} 

% section concurrent_process_calculi_and_spatial_logics_ (end)
    
%\input{qm2pi.knots2pi} 

%\input{qm2pi.trefoil} 

%\input{qm2pi.mainthm} 

% subsection basic_interpretation (end)

%\input{qm2pi.rho.presentation} 
\subsection{The syntax and semantics of the notation system}\label{sub:the_syntax_and_semantics_of_the_notation_system} % (fold)

We now summarize a technical presentation of the calculus that
embodies our theory of dynamics. The typical presentation of such a
calculus follows the style of giving generators and relations on
them. The grammar, below, describing term constructors, freely
generates the set of processes, $\Proc$. This set is then quotiented
by a relation known as structural congruence and it is over this set
that the notion of dynamics is expressed. This presentation is
essentially that of \cite{MeredithR05} with the addition of
polyadicity and summation. For readability we have relegated some of
the technical subtleties to an appendix.

\subsubsection{Process grammar}\label{subsub:process_grammar}

\begin{mathpar}
  \inferrule* [lab=synchronization] {} {{M} \bc \pzero \;|\; x?F \;|\; x!C }
  \and
  \inferrule* [lab=abstraction] {} {{F} \bc (x)P}
  \and
  \inferrule* [lab=concretion] {} {{C} \bc \langle Q \rangle}
  \and
  \inferrule* [lab=process] {} {{P,Q} \bc M \;| \;P|Q \;|\; @{x}}
  \and
  \inferrule* [lab=name] {} {{x} \bc \quotep{P}}
\end{mathpar} 

Note that $\vec{x}$ (resp. $\vec{P}$) denotes a vector of names
(resp. processes) of length $|\vec{x}|$ (resp. $|\vec{P}|$). We adopt
the following useful abbreviations.

\begin{mathpar}
   x?(\vec{y}).P := x.(\vec{y})P \and  x\clift{\vec{P}} := x.\clift{\vec{P}}
   \and x!(y) := \lift{x}{\dropn{y}}
   \and \Pi_{i=0}^{n-1}P_i := P_0 | \ldots | P_{n-1}
\end{mathpar}

\subsubsection{Structural congruence}

\paragraph{Free and bound names and alpha-equivalence.} At the
core of structural equivalence is alpha-equivalence which identifies
process that are the same up to a change of variable. Formally, we
recognize the distinction between free and bound names. The free names
of a process, $\freenames{P}$, may be calculated recursively as
follows:

\begin{mathpar}
\freenames{\pzero} := \emptyset
  \and \\
  \freenames{x?(y).P} := \{ x \} \cup (\freenames{P} \setminus \{ y \})
  \and 
  \freenames{x!\langle P \rangle} := \{ x \} \cup \{ P \} 
  \and \\
  \freenames{P|Q} := \freenames{P} \cup \freenames{Q}
  \and \\
  \freenames{@{x}} := \{ x \}
\end{mathpar}

$\pi$
$\quotep{\pi}$

$\freenames{-} : \pi \to \mathcal{P}(\quotep{\pi})$

\begin{eqnarray*}
  \freenames{\pzero} & := & \emptyset \\
  \freenames{x?(y).P} & := & \{ x \} \cup (\freenames{P} \setminus \{ y \}) \\
  \freenames{x!\langle P \rangle} & := & \{ x \} \cup \{ P \} \\
  \freenames{P|Q} & := & \freenames{P} \cup \freenames{Q} \\
  \freenames{\dropn{x}} & := & \{ x \}
\end{eqnarray*}

The bound names of a process, $\boundnames{P}$, are those names occurring in $P$
that are not free. For example, in $x?(y).0$, the name $x$ is free, while $y$ is bound.

\begin{mathpar}
  \inferrule* [lab=monoidal-laws] {} { P|Q \equiv Q|P \and P|0 \equiv P \and P|(Q|R) \equiv (P|Q)|R }
\end{mathpar}

\begin{mathpar}
  \inferrule* [lab=alpha-equivalence] {} { (x)P \equiv (y)P\{y/x\} \and y \not\in \freenames{P} }
\end{mathpar}

\begin{definition}
Then two processes, $P,Q$, are alpha-equivalent if $P = Q\{\vec{y}/\vec{x}\}$ for
some $\vec{x} \in \boundnames{Q},\vec{y} \in \boundnames{P}$, where $Q\{\vec{y}/\vec{x}\}$
denotes the capture-avoiding substitution of $\vec{y}$ for $\vec{x}$ in $Q$.
\end{definition}

\begin{definition}
  The {\em structural congruence} \cite{SangiorgiWalker} , $\equiv$,
  between processes is the least congruence containing
  alpha-equivalence, satisfying the abelian monoid laws
  (associativity, commutativity and $\pzero$ as identity) for parallel
  composition $|$ and for summation $+$.
\end{definition}

\subsection{Name equivalence}

We take name equivalence, written $\nameeq$, to be the smallest
equivalence relation generated by the following rules.

\begin{mathpar}
\inferrule*[lab=Quote-drop]
{ }
{ \quotep{@{x}} \nameeq x }

\inferrule*[lab=Struct-equiv]
{ P \scong Q }
{ \quotep{P} \nameeq \quotep{Q} }
\end{mathpar}

The astute reader will have noticed that the mutual recursion of names
and processes imposes a mutual recursion on alpha-equivalence and
structural equivalence via name-equivalence. Fortunately, all of this
works out pleasantly and we may calculate in the natural way, free of
concern. The reader interested in the details is referred to the
appendix \ref{appendix:rho_details}.

\subsection{Substitution}

We use $\Proc$ for the set of processes, $\QProc$ for the set of
names, and $\id{\{}\vec{y} / \vec{x} \id{\}}$ to denote partial maps,
$s : \QProc \rightarrow \QProc$. A map, $s$ lifts, uniquely, to a map
on process terms, $\widehat{s} : \Proc \rightarrow \Proc$ by the
following equations.

\begin{mathpar}
  (0) \psubstp{Q}{P} := 0 \\
  (R \juxtap S) \psubstp{Q}{P}
  :=    
  (R)\psubstp{Q}{P} \juxtap (S) \psubstp{Q}{P} \\
  (x?(y).R) \psubstp{Q}{P}    
  :=    
  (x)\substp{Q}{P} (z)\concat( (R \psubstn{z}{y}) \psubstp{Q}{P} ) \\
  (\lift{x}{R}) \psubstp{Q}{P}  
  :=
  \lift{(x)\substp{Q}{P}}{ R \psubstp{Q}{P} } \\
%   (\dropn{x})  \psubstp{Q}{P}       
%   := 
%   \left\{ 
%     \begin{array}{ccc} 
%       \dropn{\quotep{Q}} & & x \nameeq \quotep{P} \\
%       \dropn{x} & & otherwise \\
%     \end{array}
%   \right. 
  (\dropn{x})  \psubstp{Q}{P}       
  := 
  \left\{ 
    \begin{array}{ccc} 
      Q & & x \nameeq \quotep{P} \\
      \dropn{x} & & otherwise \\
    \end{array}
  \right.
\end{mathpar}
 

where

\begin{eqnarray}
  (x)\id{\{} \lpquote Q \rpquote / \lpquote P \rpquote \id{\}}            = 
  \left\{ 
    \begin{array}{ccc}
      \lpquote Q \rpquote & & x \nameeq \lpquote P \rpquote \\
      x & & otherwise \\
    \end{array}
  \right. \nonumber
\end{eqnarray}

and $z$ is chosen distinct from $\quotep{P}$, $\quotep{Q}$, the free
names in $Q$, and all the names in $R$. Our $\alpha$-equivalence will
be built in the standard way from this substitution.

\begin{remark}\label{rem:no_self_referential_names}
  One consequence of these definitions is that $\forall P. \quotep{P}
  \not\in \freenames{P}$.
\end{remark}

\subsection{ Dynamic quote: an example }

Anticipating something of what's to come, consider applying the
substitution, $\widehat{\id{\{}u / z \id{\}}}$, to the following pair
of processes, $\lift{w}{y!(z)}$ and $w[ \lpquote y!(z) \rpquote ]$.

\begin{eqnarray}
	\lift{w}{y!(z)}\widehat{\id{\{}u / z \id{\}}}
		& = &
		\lift{w}{y!(u)} \nonumber\\
	w[ \lpquote y!(z) \rpquote ] \widehat{ \id{\{}u / z \id{\}} }
		& = &
		w[ \lpquote y!(z) \rpquote ] \nonumber
\end{eqnarray}

Because the body of the process between quotes is impervious to
substitution, we get radically different answers. In fact, by
examining the first process in an input context,
e.g. $x?(z).\lift{w}{y!(z)}$, we see that the process under the lift
operator may be shaped by prefixed inputs binding a name inside it. In
this sense, the lift operator will be seen as a way to dynamically
construct processes before reifying them as names.

Finally equipped with these standard features we can present the
dynamics of the calculus.

\subsubsection{Operational semantics} 

Finally, we introduce the computational dynamics. What marks these
algebras as distinct from other more traditionally studied algebraic
structures, e.g. vector spaces or polynomial rings, is the manner in
which dynamics is captured. In traditional structures, dynamics is typically
expressed through morphisms between such structures, as in linear maps
between vector spaces or morphisms between rings. In algebras
associated with the semantics of computation, the dynamics is
expressed as part of the algebraic structure itself, through a
reduction reduction relation typically denoted by $\red$. Below, we
give a recursive presentation of this relation for the calculus used
in the encoding.

$\red \subseteq \pi \times \pi$
$\red : \pi \to \mathcal{P}(\pi)$

\begin{mathpar}
  \inferrule* [lab=Comm] { \textsf{match}( x_{src}, x_{trgt} ) } { x_{trgt}?(y)P \; | \; x_{src}!\langle {Q} \rangle \red P\{\quotep{Q}/y}\} }
  \and \\
  \inferrule* [lab=Par] {{P} \red {P}'} {{{P} | {Q}} \red {{P}' | {Q}}}
  \and
  \inferrule* [lab=Equiv]{{{P} \scong {P}'} \andalso {{P}' \red {Q}'} \andalso {{Q}' \scong {Q}}}{{P} \red {Q}}
\end{mathpar}

\begin{eqnarray*}
  match_{\equiv} (\quotep{P},\quotep{Q}) & := & P \equiv Q \\
  match_{\dagger}(\quotep{P},\quotep{Q}) & := & \forall R. P|Q \red^{*} R => R \red^{*} 0 \\
  match_{K}(\quotep{P},\quotep{Q}) & := & K \mbox{ for some context } K
\end{eqnarray*}

$u?(x)P | u!\langle Q \rangle \red P\{\quotep{Q}/x\}$

%We write $\wred$ for $\red^*$, and $P\red$ if $\exists Q $ such that $ P \red Q$.
We write $P\red$ if $\exists Q $ such that $ P \red Q$ and $P\not\red$, otherwise.

\section{Replication}

As mentioned before, it is known that replication (and hence
recursion) can be implemented in a higher-order process algebra
\cite{SangiorgiWalker}. As our first example of calculation with the
machinery thus far presented we give the construction explicitly in
the {\rhoc}.

\begin{eqnarray}
	D_{x} & := & \prefix{x}{y}{(\binpar{\outputp{x}{y}}{@{y}})} \nonumber\\
	\bangp_{x}{P} & := & \binpar{{x}!\langle{\binpar{D_{x}}{P}}\rangle}{D_{x}} \nonumber
\end{eqnarray}

\begin{eqnarray}
	\bangp_{x}{P} & & \nonumber\\
	=
	& {x}!\langle{(\prefix{x}{y}{(\outputp{x}{y} | @{y})) | P}}\rangle 
	      | \prefix{x}{y}{(\outputp{x}{y} | @{y})} & \nonumber\\
	\red
	& (\outputp{x}{y} | @{y})\substn{\quotep{(\prefix{x}{y}{(@{y} | \outputp{x}{y})) | P}}}{y} & \nonumber\\
	=
	& \outputp{x}{\quotep{(\prefix{x}{y}{(\outputp{x}{y} | @{y})) | P}}}
	  | {(\prefix{x}{y}{(\outputp{x}{y} | @{y})) | P}} & \nonumber\\
	\red
	& \ldots & \nonumber\\
	\red^*
	& P | P | \ldots & \nonumber
\end{eqnarray}

Of course, this encoding, as an implementation, runs away, unfolding
$\bangp{P}$ eagerly. A lazier and more implementable replication
operator, restricted to input-guarded processes, may be obtained as follows.

\begin{eqnarray}
\bangp{\prefix{u}{v}{P}} 
	:= 
	\binpar{\lift{x}{\prefix{u}{v}{(\binpar{D(x)}{P})}}}{D(x)} \nonumber
\end{eqnarray}

\begin{remark}
  Note that the lazier definition still does not deal with summation
  or mixed summation (i.e. sums over input and output). The reader is
  invited to construct definitions of replication that deal with these
  features. 

  Further, the definitions are parameterized in a name, $x$. Can you,
  gentle reader, make a definition that eliminates this parameter and
  guarantees no accidental interaction between the replication
  machinery and the process being replicated -- i.e. no accidental
  sharing of names used by the process to get its work done and the
  name(s) used by the replication to effect copying. This latter
  revision of the definition of replication is crucial to obtaining
  the expected identity $!!P \sim !P$.
\end{remark}

\begin{remark}\label{rem:paradoxical_combinator}
  The reader familiar with the lambda calculus will have noticed the
  similarity between $D$ and the paradoxical combinator.

  [Ed. note: the existence of this seems to suggest we have to be more
  restrictive on the set of processes and names we admit if we are to
  support no-cloning.]
\end{remark}

\subsubsection{Bisimulation}

The computational dynamics gives rise to another kind of equivalence,
the equivalence of computational behavior. As previously mentioned
this is typically captured \emph{via} some form of bisimulation.

% The notion we use in this paper is weak barbed bisimulation
% \cite{milner91polyadicpi}.

The notion we use in this paper is derived from weak barbed
bisimulation \cite{milner91polyadicpi}. 

\begin{definition}
An \emph{observation relation}, $\downarrow_{\mathcal N}$, over a set
of names, $\mathcal N$, is the smallest relation satisfying the rules
below.

\infrule[Out-barb]{y \in {\mathcal N}, \; x \nameeq y}
		  {\outputp{x}{v} \downarrow_{\mathcal N} x}
\infrule[Par-barb]{\mbox{$P\downarrow_{\mathcal N} x$ or $Q\downarrow_{\mathcal N} x$}}
		  {\binpar{P}{Q} \downarrow_{\mathcal N} x}

We write $P \Downarrow_{\mathcal N} x$ if there is $Q$ such that 
$P \wred Q$ and $Q \downarrow_{\mathcal N} x$.
\end{definition}

\begin{definition}
%\label{def.bbisim}
An  ${\mathcal N}$-\emph{barbed bisimulation} over a set of names, ${\mathcal N}$, is a symmetric binary relation 
${\mathcal S}_{\mathcal N}$ between agents such that $P\rel{S}_{\mathcal N}Q$ implies:
\begin{enumerate}
\item If $P \red P'$ then $Q \wred Q'$ and $P'\rel{S}_{\mathcal N} Q'$.
\item If $P\downarrow_{\mathcal N} x$, then $Q\Downarrow_{\mathcal N} x$.
\end{enumerate}
$P$ is ${\mathcal N}$-barbed bisimilar to $Q$, written
$P \wbbisim_{\mathcal N} Q$, if $P \rel{S}_{\mathcal N} Q$ for some ${\mathcal N}$-barbed bisimulation ${\mathcal S}_{\mathcal N}$.
\end{definition}

$\mathcal{R} \subseteq \pi \times \pi$

$P \mathcal{R} Q => \forall P'. P \red P' \Rightarrow \exists Q'. Q \red Q', P' \mathcal{R} Q'$

$P \vdash x \Rightarrow Q \vdash x$

\begin{mathpar}
  \inferrule*[lab=Out-barb]{x \nameeq y}{{y}!\langle{Q}\rangle \vdash x}
  \and
  \inferrule*[lab=Par-barb]{\mbox{$P\vdash x$ or $Q\vdash x$}}{\binpar{P}{Q} \vdash x}
\end{mathpar}

\subsubsection{Contexts}

One of the principle advantages of computational calculi like the
$\pi$-calculus is a well-defined notion of context,
contextual-equivalence and a correlation between
contextual-equivalence and notions of bisimulation. The notion of
context allows the decomposition of a process into (sub-)process and
its syntactic environment, its context. Thus, a context may be
thought of as a process with a ``hole'' (written $\Box$) in it. The
application of a context $M$ to a process $P$, written $M[P]$, is
tantamount to filling the hole in $M$ with $P$. In this paper we do
not need the full weight of this theory, but do make use of the notion
of context in the proof the main theorem. 

\begin{mathpar}
  \inferrule* [lab=summation] {} {{M_{M},M_{N}} \bc \Box \;|\; x.M_{A} \;|\; M_{M}+M_{N}}
  \and
  \inferrule* [lab=agent] {} {{M_{A}} \bc (\vec{x})M_{P} \;| \; \clift{P_0,\ldots,M_{P},\ldots,P_N}}
  \and \\
  \inferrule* [lab=process] {} {{M_{P}} \bc M_{N} \;| \;P|M_{P} }
\end{mathpar} 

\begin{mathpar}
  \inferrule* [lab=sychronization] {} {M_{N} \bc \Box \;|\; x?M_{F} \;|\; x!M_{C}}
  \and
  \inferrule* [lab=abstraction] {} {{M_{F}} \bc (x)M_{P} }
  \and
  \inferrule* [lab=concretion] {} {{M_{C}} \bc \langle M_{P} \rangle }
  \and \\
  \inferrule* [lab=process] {} {{M_{P}} \bc M_{N} \;| \;P|M_{P} }
\end{mathpar}

\begin{definition}[contextual application] Given a context $M$, and
  process $P$, we define the \emph{contextual application}, $M[P] :=
  M\{P/\Box\}$. That is, the contextual application of M to P is the
  substitution of $P$ for $\Box$ in $M$.
\end{definition}

$\meaningof{-} : L \to \mathcal{P}(\pi)$

\begin{mathpar}
  \inferrule* [lab=collection] {} {\meaningof{true} = \pi, \and \meaningof{~E} = \pi \setminus \meaningof{E}, \and \meaningof{E_{1} \& E_{2}} = \meaningof{E_{1}} \cap \meaningof{E_{2}}}
\end{mathpar}

\begin{mathpar}
  \inferrule* [lab=structure] {} {\meaningof{0} = \{ P \in \pi | P \equiv 0 \}, \and \\ \meaningof{E_1 | E_2} = \{ P \in \pi | P \equiv P_{1} | P_{2}, P_{1} \in \meaningof{E_{1}}, P_{2} \in \meaningof{E_2}\} }
\end{mathpar}

\begin{mathpar}
 \inferrule* [lab=behavior] {} {\meaningof{\langle a?b \rangle E} = \{ P \in \pi | P \equiv Q | u?(y)P', \\ \and \\\\ \and \\ \;\;\; u \in \meaningof{a}, \forall z.P'\{z/y\} \in \meaningof{E\{z/b\}}\}, \and \\ \meaningof{a!E} = \{ P \in \pi | P \equiv Q | x!\langle P' \rangle, x \in \meaningof{a} P' \in \meaningof{E}\} }
\end{mathpar}

\begin{mathpar}
 \inferrule* [lab=nominal] {} {\meaningof{\quotep{E}} = \{ \quotep{P} \in \quotep{\pi} | P \in \meaningof{E} \}, \and \meaningof{\quotep{P}} = \{ \quotep{Q} \in \quotep{\pi} | P \equiv Q \} \and \\ \meaningof{@\quotep{E}} = \{ P \in \pi | P \equiv @x, x \in \meaningof{E} \}}
\end{mathpar}

\begin{eqnarray*}
  \\
  \meaningof{-} : TS \to ST
\end{eqnarray*}

\begin{eqnarray*}
  \\
  L : TS \to ST
\end{eqnarray*}

\begin{eqnarray*}
  \\
  P \models E \iff P \in \meaningof{E}
\end{eqnarray*}

\begin{eqnarray*}
  P \approx_{L} Q \iff \forall E \in L. P \models E \iff Q \models E
\end{eqnarray*}

\begin{eqnarray*}
  P \approx_{K} Q
\end{eqnarray*}

\begin{eqnarray*}
  P \approx Q
\end{eqnarray*}

$\approx_{K} = \approx = \approx_{L}$

\subsubsection{Contextual duality}

Note that contexts extend the quotation operation to a family of
operations from processes to names. Given a context, $M$, we can
define a \emph{nominal context}, $\quotep{M}$ by $\quotep{M}[P] :=
\quotep{M[P]}$. To foreshadow what is to come we observe that these
operations enjoy a duality with processes very much like the duality
between vectors and maps from vectors to scalars.

Further, because the calculus is essentially higher-order, we have a
correspondence between contexts and processes. More specifically,
given a name $x$ and a context $M$ we can construct $M^{*}_{x}$ such
that 

\begin{mathpar}
  M^{*}_{x} | \lift{x}{P} \red M[P]
\end{mathpar}

namely,

\begin{mathpar}
  M^{*}_{x} := x?(u).M[\dropn{u}]
\end{mathpar}

The dependence of $M^{*}_{x}$ on a name makes it an abstraction, 

\begin{mathpar}
  M^{*} := (x)x?(u).M[\dropn{u}]
\end{mathpar}

\subsection{Additional notation}

It will sometimes be convenient to denote the process a name
quotes. We already have the notation $x = \quotep{P}$, but it will be
convenient to introduce an alternate notation, $\procn{x}$, when we
want to emphasize the connection to the use of the name. Note that, by
virtue of name equivalence, $\quotep{\procn{x}} \nameeq x$; so, the
notation is consistent with previous definitions.

Further, because names have structure it is possible to effect
substitutions on the basis of that structure. This means we need to
upgrade our notation for substitutions, which we accomplish by
adapting comprehension notation. Thus,

\begin{mathpar}
  P\{ y / x : x \in S \}
\end{mathpar}

is interpreted to mean the process derived from P by replacing (in a
capture-avoiding manner) each occurrence of $x$ in $S$ by $y$. For example,

\begin{mathpar}
  P\{ \quotep{\procn{x}|\procn{x}} / x : x \in \freenames{P} \}
\end{mathpar}

will replace each (occurrence) of a free name $x$ in $P$ by
$\quotep{\procn{x}|\procn{x}}$.

Also, we will avail ourselves of the notation $x^{L}$ and $x^{R}$ to
denote injections of a name into disjoint copies of the name
space. There are numerous ways to accomplish this. One example can be
found in \cite{MeredithR05}. This notation overloads to vectors of
names: $\vec{x}^{\pi} := (x_{i}^{\pi} \; : \; 0 \leq i < |\vec{x}| )$ where $\pi \in \{L,R\}$.

We also use $P^{\Box} := P|\Box$.

In \cite{MeredithR05} an interpretation of the new operator is
given. It turns out that there are several possible interpretations
all enjoying the requisite algebraic properties of the operator (see
\cite{milner91polyadicpi}). We will therefore make liberal use of
$(\nu\; \vec{x})P$.

% subsection the_syntax_and_semantics_of_the_notation_system (end)   

\input{qm2pi.qmops} 

\input{qm2pi.sterngerlach} 

\input{qm2pi.metric} 

% section concurrent_process_calculi (end)

%\input{qm2pi.proofsketch}

% section proof sketch (end)

%\input{qm2pi.slviaknots} 

% section spatial logic via knots (end)

\input{qm2pi.conclusion}

% section conclusion (end)

%\input{qm2pi.dtcodes} 

% section wiring algorithm (end)

\input{qm2pi.ack} 

% section acknowledgments (end)

\newpage


\bibliographystyle{plain}   
\bibliography{../../biblios/main.bib}

\input{qm2pi.rhodetails}

\end{document}



\end{document}

 

%\documentclass[12pt]{llncs}
%\documentclass{jktr}

\usepackage[pdftex]{hyperref}                   
\usepackage {listings}
\usepackage {mathpartir}
\usepackage{bcprules}
%\usepackage{listings}
                       
\usepackage{graphicx} 
%\usepackage[margins=2.5cm,nohead,nofoot]{geometry}
%\usepackage{geometry}
\usepackage{amsfonts}
\usepackage{amstext}
\usepackage{latexsym}
\usepackage{amssymb}
\usepackage{color}


%\include{myPreamble}
\documentclass[12pt]{llncs}
%\documentclass{jktr}

\usepackage[pdftex]{hyperref}                   
\usepackage {listings}
\usepackage {mathpartir}
\usepackage{bcprules}
%\usepackage{listings}
                       
\usepackage{graphicx} 
%\usepackage[margins=2.5cm,nohead,nofoot]{geometry}
%\usepackage{geometry}
\usepackage{amsfonts}
\usepackage{amstext}
\usepackage{latexsym}
\usepackage{amssymb}
\usepackage{color}


%\include{myPreamble}
\include{qm2pi.local} 

%\ifpdf
%\usepackage[pdftex]{graphicx}
%\else
%\usepackage{graphicx}
%\fi

 % \ifpdf
%  \usepackage{pdfsync}
%  \if


%\title{Brief Article}
%\author{David F. Snyder}
%\author{L.G. Meredith}

%\address{Dept. of Math., Texas State University--San Marcos, San Marcos, TX 78666}
       
\pagestyle{empty}


\begin{document}

\lstset{language=[Objective]Caml,frame=shadowbox}

\input{qm2pi.front}

% section front matter (end)

\input{qm2pi.intro} 
 
% section introduction (end)

% \input{qm2pi.knotations} 

% section notation (end)

\input{qm2pi.process.calculi} 

% section concurrent_process_calculi_and_spatial_logics_ (end)
    
%\input{qm2pi.knots2pi} 

%\input{qm2pi.trefoil} 

%\input{qm2pi.mainthm} 

% subsection basic_interpretation (end)

%\input{qm2pi.rho.presentation} 
\subsection{The syntax and semantics of the notation system}\label{sub:the_syntax_and_semantics_of_the_notation_system} % (fold)

We now summarize a technical presentation of the calculus that
embodies our theory of dynamics. The typical presentation of such a
calculus follows the style of giving generators and relations on
them. The grammar, below, describing term constructors, freely
generates the set of processes, $\Proc$. This set is then quotiented
by a relation known as structural congruence and it is over this set
that the notion of dynamics is expressed. This presentation is
essentially that of \cite{MeredithR05} with the addition of
polyadicity and summation. For readability we have relegated some of
the technical subtleties to an appendix.

\subsubsection{Process grammar}\label{subsub:process_grammar}

\begin{mathpar}
  \inferrule* [lab=synchronization] {} {{M} \bc \pzero \;|\; x?F \;|\; x!C }
  \and
  \inferrule* [lab=abstraction] {} {{F} \bc (x)P}
  \and
  \inferrule* [lab=concretion] {} {{C} \bc \langle Q \rangle}
  \and
  \inferrule* [lab=process] {} {{P,Q} \bc M \;| \;P|Q \;|\; @{x}}
  \and
  \inferrule* [lab=name] {} {{x} \bc \quotep{P}}
\end{mathpar} 

Note that $\vec{x}$ (resp. $\vec{P}$) denotes a vector of names
(resp. processes) of length $|\vec{x}|$ (resp. $|\vec{P}|$). We adopt
the following useful abbreviations.

\begin{mathpar}
   x?(\vec{y}).P := x.(\vec{y})P \and  x\clift{\vec{P}} := x.\clift{\vec{P}}
   \and x!(y) := \lift{x}{\dropn{y}}
   \and \Pi_{i=0}^{n-1}P_i := P_0 | \ldots | P_{n-1}
\end{mathpar}

\subsubsection{Structural congruence}

\paragraph{Free and bound names and alpha-equivalence.} At the
core of structural equivalence is alpha-equivalence which identifies
process that are the same up to a change of variable. Formally, we
recognize the distinction between free and bound names. The free names
of a process, $\freenames{P}$, may be calculated recursively as
follows:

\begin{mathpar}
\freenames{\pzero} := \emptyset
  \and \\
  \freenames{x?(y).P} := \{ x \} \cup (\freenames{P} \setminus \{ y \})
  \and 
  \freenames{x!\langle P \rangle} := \{ x \} \cup \{ P \} 
  \and \\
  \freenames{P|Q} := \freenames{P} \cup \freenames{Q}
  \and \\
  \freenames{@{x}} := \{ x \}
\end{mathpar}

$\pi$
$\quotep{\pi}$

$\freenames{-} : \pi \to \mathcal{P}(\quotep{\pi})$

\begin{eqnarray*}
  \freenames{\pzero} & := & \emptyset \\
  \freenames{x?(y).P} & := & \{ x \} \cup (\freenames{P} \setminus \{ y \}) \\
  \freenames{x!\langle P \rangle} & := & \{ x \} \cup \{ P \} \\
  \freenames{P|Q} & := & \freenames{P} \cup \freenames{Q} \\
  \freenames{\dropn{x}} & := & \{ x \}
\end{eqnarray*}

The bound names of a process, $\boundnames{P}$, are those names occurring in $P$
that are not free. For example, in $x?(y).0$, the name $x$ is free, while $y$ is bound.

\begin{mathpar}
  \inferrule* [lab=monoidal-laws] {} { P|Q \equiv Q|P \and P|0 \equiv P \and P|(Q|R) \equiv (P|Q)|R }
\end{mathpar}

\begin{mathpar}
  \inferrule* [lab=alpha-equivalence] {} { (x)P \equiv (y)P\{y/x\} \and y \not\in \freenames{P} }
\end{mathpar}

\begin{definition}
Then two processes, $P,Q$, are alpha-equivalent if $P = Q\{\vec{y}/\vec{x}\}$ for
some $\vec{x} \in \boundnames{Q},\vec{y} \in \boundnames{P}$, where $Q\{\vec{y}/\vec{x}\}$
denotes the capture-avoiding substitution of $\vec{y}$ for $\vec{x}$ in $Q$.
\end{definition}

\begin{definition}
  The {\em structural congruence} \cite{SangiorgiWalker} , $\equiv$,
  between processes is the least congruence containing
  alpha-equivalence, satisfying the abelian monoid laws
  (associativity, commutativity and $\pzero$ as identity) for parallel
  composition $|$ and for summation $+$.
\end{definition}

\subsection{Name equivalence}

We take name equivalence, written $\nameeq$, to be the smallest
equivalence relation generated by the following rules.

\begin{mathpar}
\inferrule*[lab=Quote-drop]
{ }
{ \quotep{@{x}} \nameeq x }

\inferrule*[lab=Struct-equiv]
{ P \scong Q }
{ \quotep{P} \nameeq \quotep{Q} }
\end{mathpar}

The astute reader will have noticed that the mutual recursion of names
and processes imposes a mutual recursion on alpha-equivalence and
structural equivalence via name-equivalence. Fortunately, all of this
works out pleasantly and we may calculate in the natural way, free of
concern. The reader interested in the details is referred to the
appendix \ref{appendix:rho_details}.

\subsection{Substitution}

We use $\Proc$ for the set of processes, $\QProc$ for the set of
names, and $\id{\{}\vec{y} / \vec{x} \id{\}}$ to denote partial maps,
$s : \QProc \rightarrow \QProc$. A map, $s$ lifts, uniquely, to a map
on process terms, $\widehat{s} : \Proc \rightarrow \Proc$ by the
following equations.

\begin{mathpar}
  (0) \psubstp{Q}{P} := 0 \\
  (R \juxtap S) \psubstp{Q}{P}
  :=    
  (R)\psubstp{Q}{P} \juxtap (S) \psubstp{Q}{P} \\
  (x?(y).R) \psubstp{Q}{P}    
  :=    
  (x)\substp{Q}{P} (z)\concat( (R \psubstn{z}{y}) \psubstp{Q}{P} ) \\
  (\lift{x}{R}) \psubstp{Q}{P}  
  :=
  \lift{(x)\substp{Q}{P}}{ R \psubstp{Q}{P} } \\
%   (\dropn{x})  \psubstp{Q}{P}       
%   := 
%   \left\{ 
%     \begin{array}{ccc} 
%       \dropn{\quotep{Q}} & & x \nameeq \quotep{P} \\
%       \dropn{x} & & otherwise \\
%     \end{array}
%   \right. 
  (\dropn{x})  \psubstp{Q}{P}       
  := 
  \left\{ 
    \begin{array}{ccc} 
      Q & & x \nameeq \quotep{P} \\
      \dropn{x} & & otherwise \\
    \end{array}
  \right.
\end{mathpar}
 

where

\begin{eqnarray}
  (x)\id{\{} \lpquote Q \rpquote / \lpquote P \rpquote \id{\}}            = 
  \left\{ 
    \begin{array}{ccc}
      \lpquote Q \rpquote & & x \nameeq \lpquote P \rpquote \\
      x & & otherwise \\
    \end{array}
  \right. \nonumber
\end{eqnarray}

and $z$ is chosen distinct from $\quotep{P}$, $\quotep{Q}$, the free
names in $Q$, and all the names in $R$. Our $\alpha$-equivalence will
be built in the standard way from this substitution.

\begin{remark}\label{rem:no_self_referential_names}
  One consequence of these definitions is that $\forall P. \quotep{P}
  \not\in \freenames{P}$.
\end{remark}

\subsection{ Dynamic quote: an example }

Anticipating something of what's to come, consider applying the
substitution, $\widehat{\id{\{}u / z \id{\}}}$, to the following pair
of processes, $\lift{w}{y!(z)}$ and $w[ \lpquote y!(z) \rpquote ]$.

\begin{eqnarray}
	\lift{w}{y!(z)}\widehat{\id{\{}u / z \id{\}}}
		& = &
		\lift{w}{y!(u)} \nonumber\\
	w[ \lpquote y!(z) \rpquote ] \widehat{ \id{\{}u / z \id{\}} }
		& = &
		w[ \lpquote y!(z) \rpquote ] \nonumber
\end{eqnarray}

Because the body of the process between quotes is impervious to
substitution, we get radically different answers. In fact, by
examining the first process in an input context,
e.g. $x?(z).\lift{w}{y!(z)}$, we see that the process under the lift
operator may be shaped by prefixed inputs binding a name inside it. In
this sense, the lift operator will be seen as a way to dynamically
construct processes before reifying them as names.

Finally equipped with these standard features we can present the
dynamics of the calculus.

\subsubsection{Operational semantics} 

Finally, we introduce the computational dynamics. What marks these
algebras as distinct from other more traditionally studied algebraic
structures, e.g. vector spaces or polynomial rings, is the manner in
which dynamics is captured. In traditional structures, dynamics is typically
expressed through morphisms between such structures, as in linear maps
between vector spaces or morphisms between rings. In algebras
associated with the semantics of computation, the dynamics is
expressed as part of the algebraic structure itself, through a
reduction reduction relation typically denoted by $\red$. Below, we
give a recursive presentation of this relation for the calculus used
in the encoding.

$\red \subseteq \pi \times \pi$
$\red : \pi \to \mathcal{P}(\pi)$

\begin{mathpar}
  \inferrule* [lab=Comm] { \textsf{match}( x_{src}, x_{trgt} ) } { x_{trgt}?(y)P \; | \; x_{src}!\langle {Q} \rangle \red P\{\quotep{Q}/y}\} }
  \and \\
  \inferrule* [lab=Par] {{P} \red {P}'} {{{P} | {Q}} \red {{P}' | {Q}}}
  \and
  \inferrule* [lab=Equiv]{{{P} \scong {P}'} \andalso {{P}' \red {Q}'} \andalso {{Q}' \scong {Q}}}{{P} \red {Q}}
\end{mathpar}

\begin{eqnarray*}
  match_{\equiv} (\quotep{P},\quotep{Q}) & := & P \equiv Q \\
  match_{\dagger}(\quotep{P},\quotep{Q}) & := & \forall R. P|Q \red^{*} R => R \red^{*} 0 \\
  match_{K}(\quotep{P},\quotep{Q}) & := & K \mbox{ for some context } K
\end{eqnarray*}

$u?(x)P | u!\langle Q \rangle \red P\{\quotep{Q}/x\}$

%We write $\wred$ for $\red^*$, and $P\red$ if $\exists Q $ such that $ P \red Q$.
We write $P\red$ if $\exists Q $ such that $ P \red Q$ and $P\not\red$, otherwise.

\section{Replication}

As mentioned before, it is known that replication (and hence
recursion) can be implemented in a higher-order process algebra
\cite{SangiorgiWalker}. As our first example of calculation with the
machinery thus far presented we give the construction explicitly in
the {\rhoc}.

\begin{eqnarray}
	D_{x} & := & \prefix{x}{y}{(\binpar{\outputp{x}{y}}{@{y}})} \nonumber\\
	\bangp_{x}{P} & := & \binpar{{x}!\langle{\binpar{D_{x}}{P}}\rangle}{D_{x}} \nonumber
\end{eqnarray}

\begin{eqnarray}
	\bangp_{x}{P} & & \nonumber\\
	=
	& {x}!\langle{(\prefix{x}{y}{(\outputp{x}{y} | @{y})) | P}}\rangle 
	      | \prefix{x}{y}{(\outputp{x}{y} | @{y})} & \nonumber\\
	\red
	& (\outputp{x}{y} | @{y})\substn{\quotep{(\prefix{x}{y}{(@{y} | \outputp{x}{y})) | P}}}{y} & \nonumber\\
	=
	& \outputp{x}{\quotep{(\prefix{x}{y}{(\outputp{x}{y} | @{y})) | P}}}
	  | {(\prefix{x}{y}{(\outputp{x}{y} | @{y})) | P}} & \nonumber\\
	\red
	& \ldots & \nonumber\\
	\red^*
	& P | P | \ldots & \nonumber
\end{eqnarray}

Of course, this encoding, as an implementation, runs away, unfolding
$\bangp{P}$ eagerly. A lazier and more implementable replication
operator, restricted to input-guarded processes, may be obtained as follows.

\begin{eqnarray}
\bangp{\prefix{u}{v}{P}} 
	:= 
	\binpar{\lift{x}{\prefix{u}{v}{(\binpar{D(x)}{P})}}}{D(x)} \nonumber
\end{eqnarray}

\begin{remark}
  Note that the lazier definition still does not deal with summation
  or mixed summation (i.e. sums over input and output). The reader is
  invited to construct definitions of replication that deal with these
  features. 

  Further, the definitions are parameterized in a name, $x$. Can you,
  gentle reader, make a definition that eliminates this parameter and
  guarantees no accidental interaction between the replication
  machinery and the process being replicated -- i.e. no accidental
  sharing of names used by the process to get its work done and the
  name(s) used by the replication to effect copying. This latter
  revision of the definition of replication is crucial to obtaining
  the expected identity $!!P \sim !P$.
\end{remark}

\begin{remark}\label{rem:paradoxical_combinator}
  The reader familiar with the lambda calculus will have noticed the
  similarity between $D$ and the paradoxical combinator.

  [Ed. note: the existence of this seems to suggest we have to be more
  restrictive on the set of processes and names we admit if we are to
  support no-cloning.]
\end{remark}

\subsubsection{Bisimulation}

The computational dynamics gives rise to another kind of equivalence,
the equivalence of computational behavior. As previously mentioned
this is typically captured \emph{via} some form of bisimulation.

% The notion we use in this paper is weak barbed bisimulation
% \cite{milner91polyadicpi}.

The notion we use in this paper is derived from weak barbed
bisimulation \cite{milner91polyadicpi}. 

\begin{definition}
An \emph{observation relation}, $\downarrow_{\mathcal N}$, over a set
of names, $\mathcal N$, is the smallest relation satisfying the rules
below.

\infrule[Out-barb]{y \in {\mathcal N}, \; x \nameeq y}
		  {\outputp{x}{v} \downarrow_{\mathcal N} x}
\infrule[Par-barb]{\mbox{$P\downarrow_{\mathcal N} x$ or $Q\downarrow_{\mathcal N} x$}}
		  {\binpar{P}{Q} \downarrow_{\mathcal N} x}

We write $P \Downarrow_{\mathcal N} x$ if there is $Q$ such that 
$P \wred Q$ and $Q \downarrow_{\mathcal N} x$.
\end{definition}

\begin{definition}
%\label{def.bbisim}
An  ${\mathcal N}$-\emph{barbed bisimulation} over a set of names, ${\mathcal N}$, is a symmetric binary relation 
${\mathcal S}_{\mathcal N}$ between agents such that $P\rel{S}_{\mathcal N}Q$ implies:
\begin{enumerate}
\item If $P \red P'$ then $Q \wred Q'$ and $P'\rel{S}_{\mathcal N} Q'$.
\item If $P\downarrow_{\mathcal N} x$, then $Q\Downarrow_{\mathcal N} x$.
\end{enumerate}
$P$ is ${\mathcal N}$-barbed bisimilar to $Q$, written
$P \wbbisim_{\mathcal N} Q$, if $P \rel{S}_{\mathcal N} Q$ for some ${\mathcal N}$-barbed bisimulation ${\mathcal S}_{\mathcal N}$.
\end{definition}

$\mathcal{R} \subseteq \pi \times \pi$

$P \mathcal{R} Q => \forall P'. P \red P' \Rightarrow \exists Q'. Q \red Q', P' \mathcal{R} Q'$

$P \vdash x \Rightarrow Q \vdash x$

\begin{mathpar}
  \inferrule*[lab=Out-barb]{x \nameeq y}{{y}!\langle{Q}\rangle \vdash x}
  \and
  \inferrule*[lab=Par-barb]{\mbox{$P\vdash x$ or $Q\vdash x$}}{\binpar{P}{Q} \vdash x}
\end{mathpar}

\subsubsection{Contexts}

One of the principle advantages of computational calculi like the
$\pi$-calculus is a well-defined notion of context,
contextual-equivalence and a correlation between
contextual-equivalence and notions of bisimulation. The notion of
context allows the decomposition of a process into (sub-)process and
its syntactic environment, its context. Thus, a context may be
thought of as a process with a ``hole'' (written $\Box$) in it. The
application of a context $M$ to a process $P$, written $M[P]$, is
tantamount to filling the hole in $M$ with $P$. In this paper we do
not need the full weight of this theory, but do make use of the notion
of context in the proof the main theorem. 

\begin{mathpar}
  \inferrule* [lab=summation] {} {{M_{M},M_{N}} \bc \Box \;|\; x.M_{A} \;|\; M_{M}+M_{N}}
  \and
  \inferrule* [lab=agent] {} {{M_{A}} \bc (\vec{x})M_{P} \;| \; \clift{P_0,\ldots,M_{P},\ldots,P_N}}
  \and \\
  \inferrule* [lab=process] {} {{M_{P}} \bc M_{N} \;| \;P|M_{P} }
\end{mathpar} 

\begin{mathpar}
  \inferrule* [lab=sychronization] {} {M_{N} \bc \Box \;|\; x?M_{F} \;|\; x!M_{C}}
  \and
  \inferrule* [lab=abstraction] {} {{M_{F}} \bc (x)M_{P} }
  \and
  \inferrule* [lab=concretion] {} {{M_{C}} \bc \langle M_{P} \rangle }
  \and \\
  \inferrule* [lab=process] {} {{M_{P}} \bc M_{N} \;| \;P|M_{P} }
\end{mathpar}

\begin{definition}[contextual application] Given a context $M$, and
  process $P$, we define the \emph{contextual application}, $M[P] :=
  M\{P/\Box\}$. That is, the contextual application of M to P is the
  substitution of $P$ for $\Box$ in $M$.
\end{definition}

$\meaningof{-} : L \to \mathcal{P}(\pi)$

\begin{mathpar}
  \inferrule* [lab=collection] {} {\meaningof{true} = \pi, \and \meaningof{~E} = \pi \setminus \meaningof{E}, \and \meaningof{E_{1} \& E_{2}} = \meaningof{E_{1}} \cap \meaningof{E_{2}}}
\end{mathpar}

\begin{mathpar}
  \inferrule* [lab=structure] {} {\meaningof{0} = \{ P \in \pi | P \equiv 0 \}, \and \\ \meaningof{E_1 | E_2} = \{ P \in \pi | P \equiv P_{1} | P_{2}, P_{1} \in \meaningof{E_{1}}, P_{2} \in \meaningof{E_2}\} }
\end{mathpar}

\begin{mathpar}
 \inferrule* [lab=behavior] {} {\meaningof{\langle a?b \rangle E} = \{ P \in \pi | P \equiv Q | u?(y)P', \\ \and \\\\ \and \\ \;\;\; u \in \meaningof{a}, \forall z.P'\{z/y\} \in \meaningof{E\{z/b\}}\}, \and \\ \meaningof{a!E} = \{ P \in \pi | P \equiv Q | x!\langle P' \rangle, x \in \meaningof{a} P' \in \meaningof{E}\} }
\end{mathpar}

\begin{mathpar}
 \inferrule* [lab=nominal] {} {\meaningof{\quotep{E}} = \{ \quotep{P} \in \quotep{\pi} | P \in \meaningof{E} \}, \and \meaningof{\quotep{P}} = \{ \quotep{Q} \in \quotep{\pi} | P \equiv Q \} \and \\ \meaningof{@\quotep{E}} = \{ P \in \pi | P \equiv @x, x \in \meaningof{E} \}}
\end{mathpar}

\begin{eqnarray*}
  \\
  \meaningof{-} : TS \to ST
\end{eqnarray*}

\begin{eqnarray*}
  \\
  L : TS \to ST
\end{eqnarray*}

\begin{eqnarray*}
  \\
  P \models E \iff P \in \meaningof{E}
\end{eqnarray*}

\begin{eqnarray*}
  P \approx_{L} Q \iff \forall E \in L. P \models E \iff Q \models E
\end{eqnarray*}

\begin{eqnarray*}
  P \approx_{K} Q
\end{eqnarray*}

\begin{eqnarray*}
  P \approx Q
\end{eqnarray*}

$\approx_{K} = \approx = \approx_{L}$

\subsubsection{Contextual duality}

Note that contexts extend the quotation operation to a family of
operations from processes to names. Given a context, $M$, we can
define a \emph{nominal context}, $\quotep{M}$ by $\quotep{M}[P] :=
\quotep{M[P]}$. To foreshadow what is to come we observe that these
operations enjoy a duality with processes very much like the duality
between vectors and maps from vectors to scalars.

Further, because the calculus is essentially higher-order, we have a
correspondence between contexts and processes. More specifically,
given a name $x$ and a context $M$ we can construct $M^{*}_{x}$ such
that 

\begin{mathpar}
  M^{*}_{x} | \lift{x}{P} \red M[P]
\end{mathpar}

namely,

\begin{mathpar}
  M^{*}_{x} := x?(u).M[\dropn{u}]
\end{mathpar}

The dependence of $M^{*}_{x}$ on a name makes it an abstraction, 

\begin{mathpar}
  M^{*} := (x)x?(u).M[\dropn{u}]
\end{mathpar}

\subsection{Additional notation}

It will sometimes be convenient to denote the process a name
quotes. We already have the notation $x = \quotep{P}$, but it will be
convenient to introduce an alternate notation, $\procn{x}$, when we
want to emphasize the connection to the use of the name. Note that, by
virtue of name equivalence, $\quotep{\procn{x}} \nameeq x$; so, the
notation is consistent with previous definitions.

Further, because names have structure it is possible to effect
substitutions on the basis of that structure. This means we need to
upgrade our notation for substitutions, which we accomplish by
adapting comprehension notation. Thus,

\begin{mathpar}
  P\{ y / x : x \in S \}
\end{mathpar}

is interpreted to mean the process derived from P by replacing (in a
capture-avoiding manner) each occurrence of $x$ in $S$ by $y$. For example,

\begin{mathpar}
  P\{ \quotep{\procn{x}|\procn{x}} / x : x \in \freenames{P} \}
\end{mathpar}

will replace each (occurrence) of a free name $x$ in $P$ by
$\quotep{\procn{x}|\procn{x}}$.

Also, we will avail ourselves of the notation $x^{L}$ and $x^{R}$ to
denote injections of a name into disjoint copies of the name
space. There are numerous ways to accomplish this. One example can be
found in \cite{MeredithR05}. This notation overloads to vectors of
names: $\vec{x}^{\pi} := (x_{i}^{\pi} \; : \; 0 \leq i < |\vec{x}| )$ where $\pi \in \{L,R\}$.

We also use $P^{\Box} := P|\Box$.

In \cite{MeredithR05} an interpretation of the new operator is
given. It turns out that there are several possible interpretations
all enjoying the requisite algebraic properties of the operator (see
\cite{milner91polyadicpi}). We will therefore make liberal use of
$(\nu\; \vec{x})P$.

% subsection the_syntax_and_semantics_of_the_notation_system (end)   

\input{qm2pi.qmops} 

\input{qm2pi.sterngerlach} 

\input{qm2pi.metric} 

% section concurrent_process_calculi (end)

%\input{qm2pi.proofsketch}

% section proof sketch (end)

%\input{qm2pi.slviaknots} 

% section spatial logic via knots (end)

\input{qm2pi.conclusion}

% section conclusion (end)

%\input{qm2pi.dtcodes} 

% section wiring algorithm (end)

\input{qm2pi.ack} 

% section acknowledgments (end)

\newpage


\bibliographystyle{plain}   
\bibliography{../../biblios/main.bib}

\input{qm2pi.rhodetails}

\end{document}

 

%\ifpdf
%\usepackage[pdftex]{graphicx}
%\else
%\usepackage{graphicx}
%\fi

 % \ifpdf
%  \usepackage{pdfsync}
%  \if


%\title{Brief Article}
%\author{David F. Snyder}
%\author{L.G. Meredith}

%\address{Dept. of Math., Texas State University--San Marcos, San Marcos, TX 78666}
       
\pagestyle{empty}


\begin{document}

\lstset{language=[Objective]Caml,frame=shadowbox}

\documentclass[12pt]{llncs}
%\documentclass{jktr}

\usepackage[pdftex]{hyperref}                   
\usepackage {listings}
\usepackage {mathpartir}
\usepackage{bcprules}
%\usepackage{listings}
                       
\usepackage{graphicx} 
%\usepackage[margins=2.5cm,nohead,nofoot]{geometry}
%\usepackage{geometry}
\usepackage{amsfonts}
\usepackage{amstext}
\usepackage{latexsym}
\usepackage{amssymb}
\usepackage{color}


%\include{myPreamble}
\include{qm2pi.local} 

%\ifpdf
%\usepackage[pdftex]{graphicx}
%\else
%\usepackage{graphicx}
%\fi

 % \ifpdf
%  \usepackage{pdfsync}
%  \if


%\title{Brief Article}
%\author{David F. Snyder}
%\author{L.G. Meredith}

%\address{Dept. of Math., Texas State University--San Marcos, San Marcos, TX 78666}
       
\pagestyle{empty}


\begin{document}

\lstset{language=[Objective]Caml,frame=shadowbox}

\input{qm2pi.front}

% section front matter (end)

\input{qm2pi.intro} 
 
% section introduction (end)

% \input{qm2pi.knotations} 

% section notation (end)

\input{qm2pi.process.calculi} 

% section concurrent_process_calculi_and_spatial_logics_ (end)
    
%\input{qm2pi.knots2pi} 

%\input{qm2pi.trefoil} 

%\input{qm2pi.mainthm} 

% subsection basic_interpretation (end)

%\input{qm2pi.rho.presentation} 
\subsection{The syntax and semantics of the notation system}\label{sub:the_syntax_and_semantics_of_the_notation_system} % (fold)

We now summarize a technical presentation of the calculus that
embodies our theory of dynamics. The typical presentation of such a
calculus follows the style of giving generators and relations on
them. The grammar, below, describing term constructors, freely
generates the set of processes, $\Proc$. This set is then quotiented
by a relation known as structural congruence and it is over this set
that the notion of dynamics is expressed. This presentation is
essentially that of \cite{MeredithR05} with the addition of
polyadicity and summation. For readability we have relegated some of
the technical subtleties to an appendix.

\subsubsection{Process grammar}\label{subsub:process_grammar}

\begin{mathpar}
  \inferrule* [lab=synchronization] {} {{M} \bc \pzero \;|\; x?F \;|\; x!C }
  \and
  \inferrule* [lab=abstraction] {} {{F} \bc (x)P}
  \and
  \inferrule* [lab=concretion] {} {{C} \bc \langle Q \rangle}
  \and
  \inferrule* [lab=process] {} {{P,Q} \bc M \;| \;P|Q \;|\; @{x}}
  \and
  \inferrule* [lab=name] {} {{x} \bc \quotep{P}}
\end{mathpar} 

Note that $\vec{x}$ (resp. $\vec{P}$) denotes a vector of names
(resp. processes) of length $|\vec{x}|$ (resp. $|\vec{P}|$). We adopt
the following useful abbreviations.

\begin{mathpar}
   x?(\vec{y}).P := x.(\vec{y})P \and  x\clift{\vec{P}} := x.\clift{\vec{P}}
   \and x!(y) := \lift{x}{\dropn{y}}
   \and \Pi_{i=0}^{n-1}P_i := P_0 | \ldots | P_{n-1}
\end{mathpar}

\subsubsection{Structural congruence}

\paragraph{Free and bound names and alpha-equivalence.} At the
core of structural equivalence is alpha-equivalence which identifies
process that are the same up to a change of variable. Formally, we
recognize the distinction between free and bound names. The free names
of a process, $\freenames{P}$, may be calculated recursively as
follows:

\begin{mathpar}
\freenames{\pzero} := \emptyset
  \and \\
  \freenames{x?(y).P} := \{ x \} \cup (\freenames{P} \setminus \{ y \})
  \and 
  \freenames{x!\langle P \rangle} := \{ x \} \cup \{ P \} 
  \and \\
  \freenames{P|Q} := \freenames{P} \cup \freenames{Q}
  \and \\
  \freenames{@{x}} := \{ x \}
\end{mathpar}

$\pi$
$\quotep{\pi}$

$\freenames{-} : \pi \to \mathcal{P}(\quotep{\pi})$

\begin{eqnarray*}
  \freenames{\pzero} & := & \emptyset \\
  \freenames{x?(y).P} & := & \{ x \} \cup (\freenames{P} \setminus \{ y \}) \\
  \freenames{x!\langle P \rangle} & := & \{ x \} \cup \{ P \} \\
  \freenames{P|Q} & := & \freenames{P} \cup \freenames{Q} \\
  \freenames{\dropn{x}} & := & \{ x \}
\end{eqnarray*}

The bound names of a process, $\boundnames{P}$, are those names occurring in $P$
that are not free. For example, in $x?(y).0$, the name $x$ is free, while $y$ is bound.

\begin{mathpar}
  \inferrule* [lab=monoidal-laws] {} { P|Q \equiv Q|P \and P|0 \equiv P \and P|(Q|R) \equiv (P|Q)|R }
\end{mathpar}

\begin{mathpar}
  \inferrule* [lab=alpha-equivalence] {} { (x)P \equiv (y)P\{y/x\} \and y \not\in \freenames{P} }
\end{mathpar}

\begin{definition}
Then two processes, $P,Q$, are alpha-equivalent if $P = Q\{\vec{y}/\vec{x}\}$ for
some $\vec{x} \in \boundnames{Q},\vec{y} \in \boundnames{P}$, where $Q\{\vec{y}/\vec{x}\}$
denotes the capture-avoiding substitution of $\vec{y}$ for $\vec{x}$ in $Q$.
\end{definition}

\begin{definition}
  The {\em structural congruence} \cite{SangiorgiWalker} , $\equiv$,
  between processes is the least congruence containing
  alpha-equivalence, satisfying the abelian monoid laws
  (associativity, commutativity and $\pzero$ as identity) for parallel
  composition $|$ and for summation $+$.
\end{definition}

\subsection{Name equivalence}

We take name equivalence, written $\nameeq$, to be the smallest
equivalence relation generated by the following rules.

\begin{mathpar}
\inferrule*[lab=Quote-drop]
{ }
{ \quotep{@{x}} \nameeq x }

\inferrule*[lab=Struct-equiv]
{ P \scong Q }
{ \quotep{P} \nameeq \quotep{Q} }
\end{mathpar}

The astute reader will have noticed that the mutual recursion of names
and processes imposes a mutual recursion on alpha-equivalence and
structural equivalence via name-equivalence. Fortunately, all of this
works out pleasantly and we may calculate in the natural way, free of
concern. The reader interested in the details is referred to the
appendix \ref{appendix:rho_details}.

\subsection{Substitution}

We use $\Proc$ for the set of processes, $\QProc$ for the set of
names, and $\id{\{}\vec{y} / \vec{x} \id{\}}$ to denote partial maps,
$s : \QProc \rightarrow \QProc$. A map, $s$ lifts, uniquely, to a map
on process terms, $\widehat{s} : \Proc \rightarrow \Proc$ by the
following equations.

\begin{mathpar}
  (0) \psubstp{Q}{P} := 0 \\
  (R \juxtap S) \psubstp{Q}{P}
  :=    
  (R)\psubstp{Q}{P} \juxtap (S) \psubstp{Q}{P} \\
  (x?(y).R) \psubstp{Q}{P}    
  :=    
  (x)\substp{Q}{P} (z)\concat( (R \psubstn{z}{y}) \psubstp{Q}{P} ) \\
  (\lift{x}{R}) \psubstp{Q}{P}  
  :=
  \lift{(x)\substp{Q}{P}}{ R \psubstp{Q}{P} } \\
%   (\dropn{x})  \psubstp{Q}{P}       
%   := 
%   \left\{ 
%     \begin{array}{ccc} 
%       \dropn{\quotep{Q}} & & x \nameeq \quotep{P} \\
%       \dropn{x} & & otherwise \\
%     \end{array}
%   \right. 
  (\dropn{x})  \psubstp{Q}{P}       
  := 
  \left\{ 
    \begin{array}{ccc} 
      Q & & x \nameeq \quotep{P} \\
      \dropn{x} & & otherwise \\
    \end{array}
  \right.
\end{mathpar}
 

where

\begin{eqnarray}
  (x)\id{\{} \lpquote Q \rpquote / \lpquote P \rpquote \id{\}}            = 
  \left\{ 
    \begin{array}{ccc}
      \lpquote Q \rpquote & & x \nameeq \lpquote P \rpquote \\
      x & & otherwise \\
    \end{array}
  \right. \nonumber
\end{eqnarray}

and $z$ is chosen distinct from $\quotep{P}$, $\quotep{Q}$, the free
names in $Q$, and all the names in $R$. Our $\alpha$-equivalence will
be built in the standard way from this substitution.

\begin{remark}\label{rem:no_self_referential_names}
  One consequence of these definitions is that $\forall P. \quotep{P}
  \not\in \freenames{P}$.
\end{remark}

\subsection{ Dynamic quote: an example }

Anticipating something of what's to come, consider applying the
substitution, $\widehat{\id{\{}u / z \id{\}}}$, to the following pair
of processes, $\lift{w}{y!(z)}$ and $w[ \lpquote y!(z) \rpquote ]$.

\begin{eqnarray}
	\lift{w}{y!(z)}\widehat{\id{\{}u / z \id{\}}}
		& = &
		\lift{w}{y!(u)} \nonumber\\
	w[ \lpquote y!(z) \rpquote ] \widehat{ \id{\{}u / z \id{\}} }
		& = &
		w[ \lpquote y!(z) \rpquote ] \nonumber
\end{eqnarray}

Because the body of the process between quotes is impervious to
substitution, we get radically different answers. In fact, by
examining the first process in an input context,
e.g. $x?(z).\lift{w}{y!(z)}$, we see that the process under the lift
operator may be shaped by prefixed inputs binding a name inside it. In
this sense, the lift operator will be seen as a way to dynamically
construct processes before reifying them as names.

Finally equipped with these standard features we can present the
dynamics of the calculus.

\subsubsection{Operational semantics} 

Finally, we introduce the computational dynamics. What marks these
algebras as distinct from other more traditionally studied algebraic
structures, e.g. vector spaces or polynomial rings, is the manner in
which dynamics is captured. In traditional structures, dynamics is typically
expressed through morphisms between such structures, as in linear maps
between vector spaces or morphisms between rings. In algebras
associated with the semantics of computation, the dynamics is
expressed as part of the algebraic structure itself, through a
reduction reduction relation typically denoted by $\red$. Below, we
give a recursive presentation of this relation for the calculus used
in the encoding.

$\red \subseteq \pi \times \pi$
$\red : \pi \to \mathcal{P}(\pi)$

\begin{mathpar}
  \inferrule* [lab=Comm] { \textsf{match}( x_{src}, x_{trgt} ) } { x_{trgt}?(y)P \; | \; x_{src}!\langle {Q} \rangle \red P\{\quotep{Q}/y}\} }
  \and \\
  \inferrule* [lab=Par] {{P} \red {P}'} {{{P} | {Q}} \red {{P}' | {Q}}}
  \and
  \inferrule* [lab=Equiv]{{{P} \scong {P}'} \andalso {{P}' \red {Q}'} \andalso {{Q}' \scong {Q}}}{{P} \red {Q}}
\end{mathpar}

\begin{eqnarray*}
  match_{\equiv} (\quotep{P},\quotep{Q}) & := & P \equiv Q \\
  match_{\dagger}(\quotep{P},\quotep{Q}) & := & \forall R. P|Q \red^{*} R => R \red^{*} 0 \\
  match_{K}(\quotep{P},\quotep{Q}) & := & K \mbox{ for some context } K
\end{eqnarray*}

$u?(x)P | u!\langle Q \rangle \red P\{\quotep{Q}/x\}$

%We write $\wred$ for $\red^*$, and $P\red$ if $\exists Q $ such that $ P \red Q$.
We write $P\red$ if $\exists Q $ such that $ P \red Q$ and $P\not\red$, otherwise.

\section{Replication}

As mentioned before, it is known that replication (and hence
recursion) can be implemented in a higher-order process algebra
\cite{SangiorgiWalker}. As our first example of calculation with the
machinery thus far presented we give the construction explicitly in
the {\rhoc}.

\begin{eqnarray}
	D_{x} & := & \prefix{x}{y}{(\binpar{\outputp{x}{y}}{@{y}})} \nonumber\\
	\bangp_{x}{P} & := & \binpar{{x}!\langle{\binpar{D_{x}}{P}}\rangle}{D_{x}} \nonumber
\end{eqnarray}

\begin{eqnarray}
	\bangp_{x}{P} & & \nonumber\\
	=
	& {x}!\langle{(\prefix{x}{y}{(\outputp{x}{y} | @{y})) | P}}\rangle 
	      | \prefix{x}{y}{(\outputp{x}{y} | @{y})} & \nonumber\\
	\red
	& (\outputp{x}{y} | @{y})\substn{\quotep{(\prefix{x}{y}{(@{y} | \outputp{x}{y})) | P}}}{y} & \nonumber\\
	=
	& \outputp{x}{\quotep{(\prefix{x}{y}{(\outputp{x}{y} | @{y})) | P}}}
	  | {(\prefix{x}{y}{(\outputp{x}{y} | @{y})) | P}} & \nonumber\\
	\red
	& \ldots & \nonumber\\
	\red^*
	& P | P | \ldots & \nonumber
\end{eqnarray}

Of course, this encoding, as an implementation, runs away, unfolding
$\bangp{P}$ eagerly. A lazier and more implementable replication
operator, restricted to input-guarded processes, may be obtained as follows.

\begin{eqnarray}
\bangp{\prefix{u}{v}{P}} 
	:= 
	\binpar{\lift{x}{\prefix{u}{v}{(\binpar{D(x)}{P})}}}{D(x)} \nonumber
\end{eqnarray}

\begin{remark}
  Note that the lazier definition still does not deal with summation
  or mixed summation (i.e. sums over input and output). The reader is
  invited to construct definitions of replication that deal with these
  features. 

  Further, the definitions are parameterized in a name, $x$. Can you,
  gentle reader, make a definition that eliminates this parameter and
  guarantees no accidental interaction between the replication
  machinery and the process being replicated -- i.e. no accidental
  sharing of names used by the process to get its work done and the
  name(s) used by the replication to effect copying. This latter
  revision of the definition of replication is crucial to obtaining
  the expected identity $!!P \sim !P$.
\end{remark}

\begin{remark}\label{rem:paradoxical_combinator}
  The reader familiar with the lambda calculus will have noticed the
  similarity between $D$ and the paradoxical combinator.

  [Ed. note: the existence of this seems to suggest we have to be more
  restrictive on the set of processes and names we admit if we are to
  support no-cloning.]
\end{remark}

\subsubsection{Bisimulation}

The computational dynamics gives rise to another kind of equivalence,
the equivalence of computational behavior. As previously mentioned
this is typically captured \emph{via} some form of bisimulation.

% The notion we use in this paper is weak barbed bisimulation
% \cite{milner91polyadicpi}.

The notion we use in this paper is derived from weak barbed
bisimulation \cite{milner91polyadicpi}. 

\begin{definition}
An \emph{observation relation}, $\downarrow_{\mathcal N}$, over a set
of names, $\mathcal N$, is the smallest relation satisfying the rules
below.

\infrule[Out-barb]{y \in {\mathcal N}, \; x \nameeq y}
		  {\outputp{x}{v} \downarrow_{\mathcal N} x}
\infrule[Par-barb]{\mbox{$P\downarrow_{\mathcal N} x$ or $Q\downarrow_{\mathcal N} x$}}
		  {\binpar{P}{Q} \downarrow_{\mathcal N} x}

We write $P \Downarrow_{\mathcal N} x$ if there is $Q$ such that 
$P \wred Q$ and $Q \downarrow_{\mathcal N} x$.
\end{definition}

\begin{definition}
%\label{def.bbisim}
An  ${\mathcal N}$-\emph{barbed bisimulation} over a set of names, ${\mathcal N}$, is a symmetric binary relation 
${\mathcal S}_{\mathcal N}$ between agents such that $P\rel{S}_{\mathcal N}Q$ implies:
\begin{enumerate}
\item If $P \red P'$ then $Q \wred Q'$ and $P'\rel{S}_{\mathcal N} Q'$.
\item If $P\downarrow_{\mathcal N} x$, then $Q\Downarrow_{\mathcal N} x$.
\end{enumerate}
$P$ is ${\mathcal N}$-barbed bisimilar to $Q$, written
$P \wbbisim_{\mathcal N} Q$, if $P \rel{S}_{\mathcal N} Q$ for some ${\mathcal N}$-barbed bisimulation ${\mathcal S}_{\mathcal N}$.
\end{definition}

$\mathcal{R} \subseteq \pi \times \pi$

$P \mathcal{R} Q => \forall P'. P \red P' \Rightarrow \exists Q'. Q \red Q', P' \mathcal{R} Q'$

$P \vdash x \Rightarrow Q \vdash x$

\begin{mathpar}
  \inferrule*[lab=Out-barb]{x \nameeq y}{{y}!\langle{Q}\rangle \vdash x}
  \and
  \inferrule*[lab=Par-barb]{\mbox{$P\vdash x$ or $Q\vdash x$}}{\binpar{P}{Q} \vdash x}
\end{mathpar}

\subsubsection{Contexts}

One of the principle advantages of computational calculi like the
$\pi$-calculus is a well-defined notion of context,
contextual-equivalence and a correlation between
contextual-equivalence and notions of bisimulation. The notion of
context allows the decomposition of a process into (sub-)process and
its syntactic environment, its context. Thus, a context may be
thought of as a process with a ``hole'' (written $\Box$) in it. The
application of a context $M$ to a process $P$, written $M[P]$, is
tantamount to filling the hole in $M$ with $P$. In this paper we do
not need the full weight of this theory, but do make use of the notion
of context in the proof the main theorem. 

\begin{mathpar}
  \inferrule* [lab=summation] {} {{M_{M},M_{N}} \bc \Box \;|\; x.M_{A} \;|\; M_{M}+M_{N}}
  \and
  \inferrule* [lab=agent] {} {{M_{A}} \bc (\vec{x})M_{P} \;| \; \clift{P_0,\ldots,M_{P},\ldots,P_N}}
  \and \\
  \inferrule* [lab=process] {} {{M_{P}} \bc M_{N} \;| \;P|M_{P} }
\end{mathpar} 

\begin{mathpar}
  \inferrule* [lab=sychronization] {} {M_{N} \bc \Box \;|\; x?M_{F} \;|\; x!M_{C}}
  \and
  \inferrule* [lab=abstraction] {} {{M_{F}} \bc (x)M_{P} }
  \and
  \inferrule* [lab=concretion] {} {{M_{C}} \bc \langle M_{P} \rangle }
  \and \\
  \inferrule* [lab=process] {} {{M_{P}} \bc M_{N} \;| \;P|M_{P} }
\end{mathpar}

\begin{definition}[contextual application] Given a context $M$, and
  process $P$, we define the \emph{contextual application}, $M[P] :=
  M\{P/\Box\}$. That is, the contextual application of M to P is the
  substitution of $P$ for $\Box$ in $M$.
\end{definition}

$\meaningof{-} : L \to \mathcal{P}(\pi)$

\begin{mathpar}
  \inferrule* [lab=collection] {} {\meaningof{true} = \pi, \and \meaningof{~E} = \pi \setminus \meaningof{E}, \and \meaningof{E_{1} \& E_{2}} = \meaningof{E_{1}} \cap \meaningof{E_{2}}}
\end{mathpar}

\begin{mathpar}
  \inferrule* [lab=structure] {} {\meaningof{0} = \{ P \in \pi | P \equiv 0 \}, \and \\ \meaningof{E_1 | E_2} = \{ P \in \pi | P \equiv P_{1} | P_{2}, P_{1} \in \meaningof{E_{1}}, P_{2} \in \meaningof{E_2}\} }
\end{mathpar}

\begin{mathpar}
 \inferrule* [lab=behavior] {} {\meaningof{\langle a?b \rangle E} = \{ P \in \pi | P \equiv Q | u?(y)P', \\ \and \\\\ \and \\ \;\;\; u \in \meaningof{a}, \forall z.P'\{z/y\} \in \meaningof{E\{z/b\}}\}, \and \\ \meaningof{a!E} = \{ P \in \pi | P \equiv Q | x!\langle P' \rangle, x \in \meaningof{a} P' \in \meaningof{E}\} }
\end{mathpar}

\begin{mathpar}
 \inferrule* [lab=nominal] {} {\meaningof{\quotep{E}} = \{ \quotep{P} \in \quotep{\pi} | P \in \meaningof{E} \}, \and \meaningof{\quotep{P}} = \{ \quotep{Q} \in \quotep{\pi} | P \equiv Q \} \and \\ \meaningof{@\quotep{E}} = \{ P \in \pi | P \equiv @x, x \in \meaningof{E} \}}
\end{mathpar}

\begin{eqnarray*}
  \\
  \meaningof{-} : TS \to ST
\end{eqnarray*}

\begin{eqnarray*}
  \\
  L : TS \to ST
\end{eqnarray*}

\begin{eqnarray*}
  \\
  P \models E \iff P \in \meaningof{E}
\end{eqnarray*}

\begin{eqnarray*}
  P \approx_{L} Q \iff \forall E \in L. P \models E \iff Q \models E
\end{eqnarray*}

\begin{eqnarray*}
  P \approx_{K} Q
\end{eqnarray*}

\begin{eqnarray*}
  P \approx Q
\end{eqnarray*}

$\approx_{K} = \approx = \approx_{L}$

\subsubsection{Contextual duality}

Note that contexts extend the quotation operation to a family of
operations from processes to names. Given a context, $M$, we can
define a \emph{nominal context}, $\quotep{M}$ by $\quotep{M}[P] :=
\quotep{M[P]}$. To foreshadow what is to come we observe that these
operations enjoy a duality with processes very much like the duality
between vectors and maps from vectors to scalars.

Further, because the calculus is essentially higher-order, we have a
correspondence between contexts and processes. More specifically,
given a name $x$ and a context $M$ we can construct $M^{*}_{x}$ such
that 

\begin{mathpar}
  M^{*}_{x} | \lift{x}{P} \red M[P]
\end{mathpar}

namely,

\begin{mathpar}
  M^{*}_{x} := x?(u).M[\dropn{u}]
\end{mathpar}

The dependence of $M^{*}_{x}$ on a name makes it an abstraction, 

\begin{mathpar}
  M^{*} := (x)x?(u).M[\dropn{u}]
\end{mathpar}

\subsection{Additional notation}

It will sometimes be convenient to denote the process a name
quotes. We already have the notation $x = \quotep{P}$, but it will be
convenient to introduce an alternate notation, $\procn{x}$, when we
want to emphasize the connection to the use of the name. Note that, by
virtue of name equivalence, $\quotep{\procn{x}} \nameeq x$; so, the
notation is consistent with previous definitions.

Further, because names have structure it is possible to effect
substitutions on the basis of that structure. This means we need to
upgrade our notation for substitutions, which we accomplish by
adapting comprehension notation. Thus,

\begin{mathpar}
  P\{ y / x : x \in S \}
\end{mathpar}

is interpreted to mean the process derived from P by replacing (in a
capture-avoiding manner) each occurrence of $x$ in $S$ by $y$. For example,

\begin{mathpar}
  P\{ \quotep{\procn{x}|\procn{x}} / x : x \in \freenames{P} \}
\end{mathpar}

will replace each (occurrence) of a free name $x$ in $P$ by
$\quotep{\procn{x}|\procn{x}}$.

Also, we will avail ourselves of the notation $x^{L}$ and $x^{R}$ to
denote injections of a name into disjoint copies of the name
space. There are numerous ways to accomplish this. One example can be
found in \cite{MeredithR05}. This notation overloads to vectors of
names: $\vec{x}^{\pi} := (x_{i}^{\pi} \; : \; 0 \leq i < |\vec{x}| )$ where $\pi \in \{L,R\}$.

We also use $P^{\Box} := P|\Box$.

In \cite{MeredithR05} an interpretation of the new operator is
given. It turns out that there are several possible interpretations
all enjoying the requisite algebraic properties of the operator (see
\cite{milner91polyadicpi}). We will therefore make liberal use of
$(\nu\; \vec{x})P$.

% subsection the_syntax_and_semantics_of_the_notation_system (end)   

\input{qm2pi.qmops} 

\input{qm2pi.sterngerlach} 

\input{qm2pi.metric} 

% section concurrent_process_calculi (end)

%\input{qm2pi.proofsketch}

% section proof sketch (end)

%\input{qm2pi.slviaknots} 

% section spatial logic via knots (end)

\input{qm2pi.conclusion}

% section conclusion (end)

%\input{qm2pi.dtcodes} 

% section wiring algorithm (end)

\input{qm2pi.ack} 

% section acknowledgments (end)

\newpage


\bibliographystyle{plain}   
\bibliography{../../biblios/main.bib}

\input{qm2pi.rhodetails}

\end{document}



% section front matter (end)

\section{Introduction}\label{sec:introduction} % (fold)
In this draft of the material i am going to have to dispense with the
usual writing conventions adopted in papers on these topics. i'm going
to have adopt whatever tone i need at the time i'm writing up the
calculations. Sometimes this may be very conversational; others it may
be the barest mathematical grunts; others still it may be that i have
lifted text from one of my other papers because the exposition of some
point was better said there. i hope that my readers are not unduly put
out by this decision. i'm not doing this to flout convention or be
rebellious. i find these calculations very technically challenging. To
keep everything going technically, something has to give; i have to
let go of some cognitive burden. So, the academic writing style --
with all of its trade-offs in terms of facilitating technical
communication -- is what i'm letting go of. Perhaps subsequent drafts
can be tightened and polished, but for now, i'm going to speak as if
we were sitting together in a coffee shop with a laptop, wifi and a
pad of paper and a pencil.

So, here's what i have to say. We -- you and i, comfortably ensconced
in our coffee shop and well-equipped with our tools -- can realize and
carry out the calculations of quantum mechanics over a very different
formal theory of dynamics, a formal theory of dynamics that
corresponds to a theory of concurrent computation with
\emph{reflection}. It has the advantage that the underlying theory is
already `quantized', but supports analogues all of the continuuous
operations. Strikingly, this underlying theory has recently been
connected with a notion of metric that we can show, by calculating
together, coincides with the metric induced by the inner product.

There are a lot of reasons why you might be interested in seeing
calculations of this form. Here's why i'm interested. For the past
several centuries there has been no competitor to the ``Newtonian''
account of dynamics. As a result the predominant share of accounts of
dynamical systems and situations have had to be formulated in terms of
the Newtonian machinery. i view this as an intellectually dangerous
position to occupy. Everything, despite it's intrinsic shape, turns
into a nail to be hit with this hammer. Recently, however, the theory
of computation has matured to the point where we have candidates for
theories of dynamics that offer very different perspective on
reasoning about dynamical systems and situations. Testing these
candidates against very successful accounts of dynamical situations,
like quantum mechanics, is going to give us some sense of how mature
they are and some measure of the quality of these accounts of
dynamics.

\subsection{Summary of contributions and outline of paper}

So, we're going to develop an interpretation of the operations of
quantum mechanics normally interpreted by Hilbert spaces and
operators. We're going to do this over a theory of computation. Note
that this is very different than the usual quantum computation program
which develops notions of computation over quantum mechanics. Rather,
we are developing a story that aligns with Wheeler's slogan: It from
Bit. To do this we will first provide an account of the theory of
computation at play here. Then we will dive into a calculation-driven
interpretation of the operations of quantum mechanics.

The reason we take this approach is that -- until very recently --
there hasn't been an axiomatic account of quantum mechanics. As a
result there has been no sharp delineation of the mathematical theory
supporting interpretation of the physical theory and the physical
theory, itself. So, ambient features of the maths are free to be
exploited (or supressed) without a real accounting of their physical
relevance. There is no sharp statement ``here's the physical theory''
qua \emph{theory} and ``here's the mathematical interpretation''
enabling a judgment of how faithful the interpretation is -- apart
from experimental observation. When there is an axiomatic account we
can judge how well a given mathematical formalism supports an
interpretation of the axioms, independent of
experimentation. Likewise, we can judge how well we have captured our
physical evidence and experience with our axiomatics, independent of
any specific mathematical implementation, with accidental detail that
may or may not have physical significance. 

In lieu of a fully fleshed out and vetted axiomatic account of quantum
mechanics, interpreting the operational notions in service of modeling
physical systems will have to suffice. In other words, we are not in
the business of providing a model of Hilbert spaces and operators. We
are in the business of providing a model of quantum mechanics because
we are motivated by testing our notions of dynamics against physical
theory; and, the predictive calculations of the physical theory must
serve as the best formulation -- shy of a fully fleshed out axiomatic
account -- of the physical theory itself (as they have for scientific
theories since time immemorial). Put another way, despite a
whole-hearted commitment to an It-from-Bit ontology, we are firmly
aligned with the shut-up-and-calculate camp as the best way to obtain
results either from the physical perspective or as a quality assurance
measure of our fledgling theory of dynamics.

In detail, we present a reflective process calculus. Then we develop
intuitive correspondences between the notions available in this
calculus and the usual physical notions supporting quantum mechanical
calculations. Thus, 

\begin{table}[htp]
  \center{
    \fbox{
      \begin{tabular}{c|c}
        quantum mechanics & process calculus \\
        \hline
        scalar & name \\
        state vector & process \\
        dual & contextual duals \\
        matrix & formal sums of process-context-dual pairs \\
        orthogonality & process annihilation \\
        inner product & execution-formula + quoting
      \end{tabular}
    }
  }
  \caption{QM - process calculi correspondences}
\end{table}

Then we tighten up these intuitions to operational definitions. We
employ the Dirac notation as the best proxy we can find for an
abstract syntax of the quantum mechanical notions. The definitions we
develop put us in contact with equational constraints coming from the
theory that we demonstrate the definitions and calculations satisfy.

This puts us in a position to shut up and calculate for the
Stern-Gerlach experimental set up, showing how these predictive
calculations become calculations on processes in our theory of a
reflective process calculus.

Penultimately, we demonstrate that the notion of metric coming from
the inner product coincides with the notion of metric available from
the theory of bisimulation. This demonstration gives us the right to
think of space as arising from behavior. Finally, we consider where we
might go from the new vantage point we have obtained.

% section introduction (end) 
 
% section introduction (end)

% \documentclass[12pt]{llncs}
%\documentclass{jktr}

\usepackage[pdftex]{hyperref}                   
\usepackage {listings}
\usepackage {mathpartir}
\usepackage{bcprules}
%\usepackage{listings}
                       
\usepackage{graphicx} 
%\usepackage[margins=2.5cm,nohead,nofoot]{geometry}
%\usepackage{geometry}
\usepackage{amsfonts}
\usepackage{amstext}
\usepackage{latexsym}
\usepackage{amssymb}
\usepackage{color}


%\include{myPreamble}
\include{qm2pi.local} 

%\ifpdf
%\usepackage[pdftex]{graphicx}
%\else
%\usepackage{graphicx}
%\fi

 % \ifpdf
%  \usepackage{pdfsync}
%  \if


%\title{Brief Article}
%\author{David F. Snyder}
%\author{L.G. Meredith}

%\address{Dept. of Math., Texas State University--San Marcos, San Marcos, TX 78666}
       
\pagestyle{empty}


\begin{document}

\lstset{language=[Objective]Caml,frame=shadowbox}

\input{qm2pi.front}

% section front matter (end)

\input{qm2pi.intro} 
 
% section introduction (end)

% \input{qm2pi.knotations} 

% section notation (end)

\input{qm2pi.process.calculi} 

% section concurrent_process_calculi_and_spatial_logics_ (end)
    
%\input{qm2pi.knots2pi} 

%\input{qm2pi.trefoil} 

%\input{qm2pi.mainthm} 

% subsection basic_interpretation (end)

%\input{qm2pi.rho.presentation} 
\subsection{The syntax and semantics of the notation system}\label{sub:the_syntax_and_semantics_of_the_notation_system} % (fold)

We now summarize a technical presentation of the calculus that
embodies our theory of dynamics. The typical presentation of such a
calculus follows the style of giving generators and relations on
them. The grammar, below, describing term constructors, freely
generates the set of processes, $\Proc$. This set is then quotiented
by a relation known as structural congruence and it is over this set
that the notion of dynamics is expressed. This presentation is
essentially that of \cite{MeredithR05} with the addition of
polyadicity and summation. For readability we have relegated some of
the technical subtleties to an appendix.

\subsubsection{Process grammar}\label{subsub:process_grammar}

\begin{mathpar}
  \inferrule* [lab=synchronization] {} {{M} \bc \pzero \;|\; x?F \;|\; x!C }
  \and
  \inferrule* [lab=abstraction] {} {{F} \bc (x)P}
  \and
  \inferrule* [lab=concretion] {} {{C} \bc \langle Q \rangle}
  \and
  \inferrule* [lab=process] {} {{P,Q} \bc M \;| \;P|Q \;|\; @{x}}
  \and
  \inferrule* [lab=name] {} {{x} \bc \quotep{P}}
\end{mathpar} 

Note that $\vec{x}$ (resp. $\vec{P}$) denotes a vector of names
(resp. processes) of length $|\vec{x}|$ (resp. $|\vec{P}|$). We adopt
the following useful abbreviations.

\begin{mathpar}
   x?(\vec{y}).P := x.(\vec{y})P \and  x\clift{\vec{P}} := x.\clift{\vec{P}}
   \and x!(y) := \lift{x}{\dropn{y}}
   \and \Pi_{i=0}^{n-1}P_i := P_0 | \ldots | P_{n-1}
\end{mathpar}

\subsubsection{Structural congruence}

\paragraph{Free and bound names and alpha-equivalence.} At the
core of structural equivalence is alpha-equivalence which identifies
process that are the same up to a change of variable. Formally, we
recognize the distinction between free and bound names. The free names
of a process, $\freenames{P}$, may be calculated recursively as
follows:

\begin{mathpar}
\freenames{\pzero} := \emptyset
  \and \\
  \freenames{x?(y).P} := \{ x \} \cup (\freenames{P} \setminus \{ y \})
  \and 
  \freenames{x!\langle P \rangle} := \{ x \} \cup \{ P \} 
  \and \\
  \freenames{P|Q} := \freenames{P} \cup \freenames{Q}
  \and \\
  \freenames{@{x}} := \{ x \}
\end{mathpar}

$\pi$
$\quotep{\pi}$

$\freenames{-} : \pi \to \mathcal{P}(\quotep{\pi})$

\begin{eqnarray*}
  \freenames{\pzero} & := & \emptyset \\
  \freenames{x?(y).P} & := & \{ x \} \cup (\freenames{P} \setminus \{ y \}) \\
  \freenames{x!\langle P \rangle} & := & \{ x \} \cup \{ P \} \\
  \freenames{P|Q} & := & \freenames{P} \cup \freenames{Q} \\
  \freenames{\dropn{x}} & := & \{ x \}
\end{eqnarray*}

The bound names of a process, $\boundnames{P}$, are those names occurring in $P$
that are not free. For example, in $x?(y).0$, the name $x$ is free, while $y$ is bound.

\begin{mathpar}
  \inferrule* [lab=monoidal-laws] {} { P|Q \equiv Q|P \and P|0 \equiv P \and P|(Q|R) \equiv (P|Q)|R }
\end{mathpar}

\begin{mathpar}
  \inferrule* [lab=alpha-equivalence] {} { (x)P \equiv (y)P\{y/x\} \and y \not\in \freenames{P} }
\end{mathpar}

\begin{definition}
Then two processes, $P,Q$, are alpha-equivalent if $P = Q\{\vec{y}/\vec{x}\}$ for
some $\vec{x} \in \boundnames{Q},\vec{y} \in \boundnames{P}$, where $Q\{\vec{y}/\vec{x}\}$
denotes the capture-avoiding substitution of $\vec{y}$ for $\vec{x}$ in $Q$.
\end{definition}

\begin{definition}
  The {\em structural congruence} \cite{SangiorgiWalker} , $\equiv$,
  between processes is the least congruence containing
  alpha-equivalence, satisfying the abelian monoid laws
  (associativity, commutativity and $\pzero$ as identity) for parallel
  composition $|$ and for summation $+$.
\end{definition}

\subsection{Name equivalence}

We take name equivalence, written $\nameeq$, to be the smallest
equivalence relation generated by the following rules.

\begin{mathpar}
\inferrule*[lab=Quote-drop]
{ }
{ \quotep{@{x}} \nameeq x }

\inferrule*[lab=Struct-equiv]
{ P \scong Q }
{ \quotep{P} \nameeq \quotep{Q} }
\end{mathpar}

The astute reader will have noticed that the mutual recursion of names
and processes imposes a mutual recursion on alpha-equivalence and
structural equivalence via name-equivalence. Fortunately, all of this
works out pleasantly and we may calculate in the natural way, free of
concern. The reader interested in the details is referred to the
appendix \ref{appendix:rho_details}.

\subsection{Substitution}

We use $\Proc$ for the set of processes, $\QProc$ for the set of
names, and $\id{\{}\vec{y} / \vec{x} \id{\}}$ to denote partial maps,
$s : \QProc \rightarrow \QProc$. A map, $s$ lifts, uniquely, to a map
on process terms, $\widehat{s} : \Proc \rightarrow \Proc$ by the
following equations.

\begin{mathpar}
  (0) \psubstp{Q}{P} := 0 \\
  (R \juxtap S) \psubstp{Q}{P}
  :=    
  (R)\psubstp{Q}{P} \juxtap (S) \psubstp{Q}{P} \\
  (x?(y).R) \psubstp{Q}{P}    
  :=    
  (x)\substp{Q}{P} (z)\concat( (R \psubstn{z}{y}) \psubstp{Q}{P} ) \\
  (\lift{x}{R}) \psubstp{Q}{P}  
  :=
  \lift{(x)\substp{Q}{P}}{ R \psubstp{Q}{P} } \\
%   (\dropn{x})  \psubstp{Q}{P}       
%   := 
%   \left\{ 
%     \begin{array}{ccc} 
%       \dropn{\quotep{Q}} & & x \nameeq \quotep{P} \\
%       \dropn{x} & & otherwise \\
%     \end{array}
%   \right. 
  (\dropn{x})  \psubstp{Q}{P}       
  := 
  \left\{ 
    \begin{array}{ccc} 
      Q & & x \nameeq \quotep{P} \\
      \dropn{x} & & otherwise \\
    \end{array}
  \right.
\end{mathpar}
 

where

\begin{eqnarray}
  (x)\id{\{} \lpquote Q \rpquote / \lpquote P \rpquote \id{\}}            = 
  \left\{ 
    \begin{array}{ccc}
      \lpquote Q \rpquote & & x \nameeq \lpquote P \rpquote \\
      x & & otherwise \\
    \end{array}
  \right. \nonumber
\end{eqnarray}

and $z$ is chosen distinct from $\quotep{P}$, $\quotep{Q}$, the free
names in $Q$, and all the names in $R$. Our $\alpha$-equivalence will
be built in the standard way from this substitution.

\begin{remark}\label{rem:no_self_referential_names}
  One consequence of these definitions is that $\forall P. \quotep{P}
  \not\in \freenames{P}$.
\end{remark}

\subsection{ Dynamic quote: an example }

Anticipating something of what's to come, consider applying the
substitution, $\widehat{\id{\{}u / z \id{\}}}$, to the following pair
of processes, $\lift{w}{y!(z)}$ and $w[ \lpquote y!(z) \rpquote ]$.

\begin{eqnarray}
	\lift{w}{y!(z)}\widehat{\id{\{}u / z \id{\}}}
		& = &
		\lift{w}{y!(u)} \nonumber\\
	w[ \lpquote y!(z) \rpquote ] \widehat{ \id{\{}u / z \id{\}} }
		& = &
		w[ \lpquote y!(z) \rpquote ] \nonumber
\end{eqnarray}

Because the body of the process between quotes is impervious to
substitution, we get radically different answers. In fact, by
examining the first process in an input context,
e.g. $x?(z).\lift{w}{y!(z)}$, we see that the process under the lift
operator may be shaped by prefixed inputs binding a name inside it. In
this sense, the lift operator will be seen as a way to dynamically
construct processes before reifying them as names.

Finally equipped with these standard features we can present the
dynamics of the calculus.

\subsubsection{Operational semantics} 

Finally, we introduce the computational dynamics. What marks these
algebras as distinct from other more traditionally studied algebraic
structures, e.g. vector spaces or polynomial rings, is the manner in
which dynamics is captured. In traditional structures, dynamics is typically
expressed through morphisms between such structures, as in linear maps
between vector spaces or morphisms between rings. In algebras
associated with the semantics of computation, the dynamics is
expressed as part of the algebraic structure itself, through a
reduction reduction relation typically denoted by $\red$. Below, we
give a recursive presentation of this relation for the calculus used
in the encoding.

$\red \subseteq \pi \times \pi$
$\red : \pi \to \mathcal{P}(\pi)$

\begin{mathpar}
  \inferrule* [lab=Comm] { \textsf{match}( x_{src}, x_{trgt} ) } { x_{trgt}?(y)P \; | \; x_{src}!\langle {Q} \rangle \red P\{\quotep{Q}/y}\} }
  \and \\
  \inferrule* [lab=Par] {{P} \red {P}'} {{{P} | {Q}} \red {{P}' | {Q}}}
  \and
  \inferrule* [lab=Equiv]{{{P} \scong {P}'} \andalso {{P}' \red {Q}'} \andalso {{Q}' \scong {Q}}}{{P} \red {Q}}
\end{mathpar}

\begin{eqnarray*}
  match_{\equiv} (\quotep{P},\quotep{Q}) & := & P \equiv Q \\
  match_{\dagger}(\quotep{P},\quotep{Q}) & := & \forall R. P|Q \red^{*} R => R \red^{*} 0 \\
  match_{K}(\quotep{P},\quotep{Q}) & := & K \mbox{ for some context } K
\end{eqnarray*}

$u?(x)P | u!\langle Q \rangle \red P\{\quotep{Q}/x\}$

%We write $\wred$ for $\red^*$, and $P\red$ if $\exists Q $ such that $ P \red Q$.
We write $P\red$ if $\exists Q $ such that $ P \red Q$ and $P\not\red$, otherwise.

\section{Replication}

As mentioned before, it is known that replication (and hence
recursion) can be implemented in a higher-order process algebra
\cite{SangiorgiWalker}. As our first example of calculation with the
machinery thus far presented we give the construction explicitly in
the {\rhoc}.

\begin{eqnarray}
	D_{x} & := & \prefix{x}{y}{(\binpar{\outputp{x}{y}}{@{y}})} \nonumber\\
	\bangp_{x}{P} & := & \binpar{{x}!\langle{\binpar{D_{x}}{P}}\rangle}{D_{x}} \nonumber
\end{eqnarray}

\begin{eqnarray}
	\bangp_{x}{P} & & \nonumber\\
	=
	& {x}!\langle{(\prefix{x}{y}{(\outputp{x}{y} | @{y})) | P}}\rangle 
	      | \prefix{x}{y}{(\outputp{x}{y} | @{y})} & \nonumber\\
	\red
	& (\outputp{x}{y} | @{y})\substn{\quotep{(\prefix{x}{y}{(@{y} | \outputp{x}{y})) | P}}}{y} & \nonumber\\
	=
	& \outputp{x}{\quotep{(\prefix{x}{y}{(\outputp{x}{y} | @{y})) | P}}}
	  | {(\prefix{x}{y}{(\outputp{x}{y} | @{y})) | P}} & \nonumber\\
	\red
	& \ldots & \nonumber\\
	\red^*
	& P | P | \ldots & \nonumber
\end{eqnarray}

Of course, this encoding, as an implementation, runs away, unfolding
$\bangp{P}$ eagerly. A lazier and more implementable replication
operator, restricted to input-guarded processes, may be obtained as follows.

\begin{eqnarray}
\bangp{\prefix{u}{v}{P}} 
	:= 
	\binpar{\lift{x}{\prefix{u}{v}{(\binpar{D(x)}{P})}}}{D(x)} \nonumber
\end{eqnarray}

\begin{remark}
  Note that the lazier definition still does not deal with summation
  or mixed summation (i.e. sums over input and output). The reader is
  invited to construct definitions of replication that deal with these
  features. 

  Further, the definitions are parameterized in a name, $x$. Can you,
  gentle reader, make a definition that eliminates this parameter and
  guarantees no accidental interaction between the replication
  machinery and the process being replicated -- i.e. no accidental
  sharing of names used by the process to get its work done and the
  name(s) used by the replication to effect copying. This latter
  revision of the definition of replication is crucial to obtaining
  the expected identity $!!P \sim !P$.
\end{remark}

\begin{remark}\label{rem:paradoxical_combinator}
  The reader familiar with the lambda calculus will have noticed the
  similarity between $D$ and the paradoxical combinator.

  [Ed. note: the existence of this seems to suggest we have to be more
  restrictive on the set of processes and names we admit if we are to
  support no-cloning.]
\end{remark}

\subsubsection{Bisimulation}

The computational dynamics gives rise to another kind of equivalence,
the equivalence of computational behavior. As previously mentioned
this is typically captured \emph{via} some form of bisimulation.

% The notion we use in this paper is weak barbed bisimulation
% \cite{milner91polyadicpi}.

The notion we use in this paper is derived from weak barbed
bisimulation \cite{milner91polyadicpi}. 

\begin{definition}
An \emph{observation relation}, $\downarrow_{\mathcal N}$, over a set
of names, $\mathcal N$, is the smallest relation satisfying the rules
below.

\infrule[Out-barb]{y \in {\mathcal N}, \; x \nameeq y}
		  {\outputp{x}{v} \downarrow_{\mathcal N} x}
\infrule[Par-barb]{\mbox{$P\downarrow_{\mathcal N} x$ or $Q\downarrow_{\mathcal N} x$}}
		  {\binpar{P}{Q} \downarrow_{\mathcal N} x}

We write $P \Downarrow_{\mathcal N} x$ if there is $Q$ such that 
$P \wred Q$ and $Q \downarrow_{\mathcal N} x$.
\end{definition}

\begin{definition}
%\label{def.bbisim}
An  ${\mathcal N}$-\emph{barbed bisimulation} over a set of names, ${\mathcal N}$, is a symmetric binary relation 
${\mathcal S}_{\mathcal N}$ between agents such that $P\rel{S}_{\mathcal N}Q$ implies:
\begin{enumerate}
\item If $P \red P'$ then $Q \wred Q'$ and $P'\rel{S}_{\mathcal N} Q'$.
\item If $P\downarrow_{\mathcal N} x$, then $Q\Downarrow_{\mathcal N} x$.
\end{enumerate}
$P$ is ${\mathcal N}$-barbed bisimilar to $Q$, written
$P \wbbisim_{\mathcal N} Q$, if $P \rel{S}_{\mathcal N} Q$ for some ${\mathcal N}$-barbed bisimulation ${\mathcal S}_{\mathcal N}$.
\end{definition}

$\mathcal{R} \subseteq \pi \times \pi$

$P \mathcal{R} Q => \forall P'. P \red P' \Rightarrow \exists Q'. Q \red Q', P' \mathcal{R} Q'$

$P \vdash x \Rightarrow Q \vdash x$

\begin{mathpar}
  \inferrule*[lab=Out-barb]{x \nameeq y}{{y}!\langle{Q}\rangle \vdash x}
  \and
  \inferrule*[lab=Par-barb]{\mbox{$P\vdash x$ or $Q\vdash x$}}{\binpar{P}{Q} \vdash x}
\end{mathpar}

\subsubsection{Contexts}

One of the principle advantages of computational calculi like the
$\pi$-calculus is a well-defined notion of context,
contextual-equivalence and a correlation between
contextual-equivalence and notions of bisimulation. The notion of
context allows the decomposition of a process into (sub-)process and
its syntactic environment, its context. Thus, a context may be
thought of as a process with a ``hole'' (written $\Box$) in it. The
application of a context $M$ to a process $P$, written $M[P]$, is
tantamount to filling the hole in $M$ with $P$. In this paper we do
not need the full weight of this theory, but do make use of the notion
of context in the proof the main theorem. 

\begin{mathpar}
  \inferrule* [lab=summation] {} {{M_{M},M_{N}} \bc \Box \;|\; x.M_{A} \;|\; M_{M}+M_{N}}
  \and
  \inferrule* [lab=agent] {} {{M_{A}} \bc (\vec{x})M_{P} \;| \; \clift{P_0,\ldots,M_{P},\ldots,P_N}}
  \and \\
  \inferrule* [lab=process] {} {{M_{P}} \bc M_{N} \;| \;P|M_{P} }
\end{mathpar} 

\begin{mathpar}
  \inferrule* [lab=sychronization] {} {M_{N} \bc \Box \;|\; x?M_{F} \;|\; x!M_{C}}
  \and
  \inferrule* [lab=abstraction] {} {{M_{F}} \bc (x)M_{P} }
  \and
  \inferrule* [lab=concretion] {} {{M_{C}} \bc \langle M_{P} \rangle }
  \and \\
  \inferrule* [lab=process] {} {{M_{P}} \bc M_{N} \;| \;P|M_{P} }
\end{mathpar}

\begin{definition}[contextual application] Given a context $M$, and
  process $P$, we define the \emph{contextual application}, $M[P] :=
  M\{P/\Box\}$. That is, the contextual application of M to P is the
  substitution of $P$ for $\Box$ in $M$.
\end{definition}

$\meaningof{-} : L \to \mathcal{P}(\pi)$

\begin{mathpar}
  \inferrule* [lab=collection] {} {\meaningof{true} = \pi, \and \meaningof{~E} = \pi \setminus \meaningof{E}, \and \meaningof{E_{1} \& E_{2}} = \meaningof{E_{1}} \cap \meaningof{E_{2}}}
\end{mathpar}

\begin{mathpar}
  \inferrule* [lab=structure] {} {\meaningof{0} = \{ P \in \pi | P \equiv 0 \}, \and \\ \meaningof{E_1 | E_2} = \{ P \in \pi | P \equiv P_{1} | P_{2}, P_{1} \in \meaningof{E_{1}}, P_{2} \in \meaningof{E_2}\} }
\end{mathpar}

\begin{mathpar}
 \inferrule* [lab=behavior] {} {\meaningof{\langle a?b \rangle E} = \{ P \in \pi | P \equiv Q | u?(y)P', \\ \and \\\\ \and \\ \;\;\; u \in \meaningof{a}, \forall z.P'\{z/y\} \in \meaningof{E\{z/b\}}\}, \and \\ \meaningof{a!E} = \{ P \in \pi | P \equiv Q | x!\langle P' \rangle, x \in \meaningof{a} P' \in \meaningof{E}\} }
\end{mathpar}

\begin{mathpar}
 \inferrule* [lab=nominal] {} {\meaningof{\quotep{E}} = \{ \quotep{P} \in \quotep{\pi} | P \in \meaningof{E} \}, \and \meaningof{\quotep{P}} = \{ \quotep{Q} \in \quotep{\pi} | P \equiv Q \} \and \\ \meaningof{@\quotep{E}} = \{ P \in \pi | P \equiv @x, x \in \meaningof{E} \}}
\end{mathpar}

\begin{eqnarray*}
  \\
  \meaningof{-} : TS \to ST
\end{eqnarray*}

\begin{eqnarray*}
  \\
  L : TS \to ST
\end{eqnarray*}

\begin{eqnarray*}
  \\
  P \models E \iff P \in \meaningof{E}
\end{eqnarray*}

\begin{eqnarray*}
  P \approx_{L} Q \iff \forall E \in L. P \models E \iff Q \models E
\end{eqnarray*}

\begin{eqnarray*}
  P \approx_{K} Q
\end{eqnarray*}

\begin{eqnarray*}
  P \approx Q
\end{eqnarray*}

$\approx_{K} = \approx = \approx_{L}$

\subsubsection{Contextual duality}

Note that contexts extend the quotation operation to a family of
operations from processes to names. Given a context, $M$, we can
define a \emph{nominal context}, $\quotep{M}$ by $\quotep{M}[P] :=
\quotep{M[P]}$. To foreshadow what is to come we observe that these
operations enjoy a duality with processes very much like the duality
between vectors and maps from vectors to scalars.

Further, because the calculus is essentially higher-order, we have a
correspondence between contexts and processes. More specifically,
given a name $x$ and a context $M$ we can construct $M^{*}_{x}$ such
that 

\begin{mathpar}
  M^{*}_{x} | \lift{x}{P} \red M[P]
\end{mathpar}

namely,

\begin{mathpar}
  M^{*}_{x} := x?(u).M[\dropn{u}]
\end{mathpar}

The dependence of $M^{*}_{x}$ on a name makes it an abstraction, 

\begin{mathpar}
  M^{*} := (x)x?(u).M[\dropn{u}]
\end{mathpar}

\subsection{Additional notation}

It will sometimes be convenient to denote the process a name
quotes. We already have the notation $x = \quotep{P}$, but it will be
convenient to introduce an alternate notation, $\procn{x}$, when we
want to emphasize the connection to the use of the name. Note that, by
virtue of name equivalence, $\quotep{\procn{x}} \nameeq x$; so, the
notation is consistent with previous definitions.

Further, because names have structure it is possible to effect
substitutions on the basis of that structure. This means we need to
upgrade our notation for substitutions, which we accomplish by
adapting comprehension notation. Thus,

\begin{mathpar}
  P\{ y / x : x \in S \}
\end{mathpar}

is interpreted to mean the process derived from P by replacing (in a
capture-avoiding manner) each occurrence of $x$ in $S$ by $y$. For example,

\begin{mathpar}
  P\{ \quotep{\procn{x}|\procn{x}} / x : x \in \freenames{P} \}
\end{mathpar}

will replace each (occurrence) of a free name $x$ in $P$ by
$\quotep{\procn{x}|\procn{x}}$.

Also, we will avail ourselves of the notation $x^{L}$ and $x^{R}$ to
denote injections of a name into disjoint copies of the name
space. There are numerous ways to accomplish this. One example can be
found in \cite{MeredithR05}. This notation overloads to vectors of
names: $\vec{x}^{\pi} := (x_{i}^{\pi} \; : \; 0 \leq i < |\vec{x}| )$ where $\pi \in \{L,R\}$.

We also use $P^{\Box} := P|\Box$.

In \cite{MeredithR05} an interpretation of the new operator is
given. It turns out that there are several possible interpretations
all enjoying the requisite algebraic properties of the operator (see
\cite{milner91polyadicpi}). We will therefore make liberal use of
$(\nu\; \vec{x})P$.

% subsection the_syntax_and_semantics_of_the_notation_system (end)   

\input{qm2pi.qmops} 

\input{qm2pi.sterngerlach} 

\input{qm2pi.metric} 

% section concurrent_process_calculi (end)

%\input{qm2pi.proofsketch}

% section proof sketch (end)

%\input{qm2pi.slviaknots} 

% section spatial logic via knots (end)

\input{qm2pi.conclusion}

% section conclusion (end)

%\input{qm2pi.dtcodes} 

% section wiring algorithm (end)

\input{qm2pi.ack} 

% section acknowledgments (end)

\newpage


\bibliographystyle{plain}   
\bibliography{../../biblios/main.bib}

\input{qm2pi.rhodetails}

\end{document}

 

% section notation (end)

\input{qm2pi.process.calculi} 

% section concurrent_process_calculi_and_spatial_logics_ (end)
    
%\documentclass[12pt]{llncs}
%\documentclass{jktr}

\usepackage[pdftex]{hyperref}                   
\usepackage {listings}
\usepackage {mathpartir}
\usepackage{bcprules}
%\usepackage{listings}
                       
\usepackage{graphicx} 
%\usepackage[margins=2.5cm,nohead,nofoot]{geometry}
%\usepackage{geometry}
\usepackage{amsfonts}
\usepackage{amstext}
\usepackage{latexsym}
\usepackage{amssymb}
\usepackage{color}


%\include{myPreamble}
\include{qm2pi.local} 

%\ifpdf
%\usepackage[pdftex]{graphicx}
%\else
%\usepackage{graphicx}
%\fi

 % \ifpdf
%  \usepackage{pdfsync}
%  \if


%\title{Brief Article}
%\author{David F. Snyder}
%\author{L.G. Meredith}

%\address{Dept. of Math., Texas State University--San Marcos, San Marcos, TX 78666}
       
\pagestyle{empty}


\begin{document}

\lstset{language=[Objective]Caml,frame=shadowbox}

\input{qm2pi.front}

% section front matter (end)

\input{qm2pi.intro} 
 
% section introduction (end)

% \input{qm2pi.knotations} 

% section notation (end)

\input{qm2pi.process.calculi} 

% section concurrent_process_calculi_and_spatial_logics_ (end)
    
%\input{qm2pi.knots2pi} 

%\input{qm2pi.trefoil} 

%\input{qm2pi.mainthm} 

% subsection basic_interpretation (end)

%\input{qm2pi.rho.presentation} 
\subsection{The syntax and semantics of the notation system}\label{sub:the_syntax_and_semantics_of_the_notation_system} % (fold)

We now summarize a technical presentation of the calculus that
embodies our theory of dynamics. The typical presentation of such a
calculus follows the style of giving generators and relations on
them. The grammar, below, describing term constructors, freely
generates the set of processes, $\Proc$. This set is then quotiented
by a relation known as structural congruence and it is over this set
that the notion of dynamics is expressed. This presentation is
essentially that of \cite{MeredithR05} with the addition of
polyadicity and summation. For readability we have relegated some of
the technical subtleties to an appendix.

\subsubsection{Process grammar}\label{subsub:process_grammar}

\begin{mathpar}
  \inferrule* [lab=synchronization] {} {{M} \bc \pzero \;|\; x?F \;|\; x!C }
  \and
  \inferrule* [lab=abstraction] {} {{F} \bc (x)P}
  \and
  \inferrule* [lab=concretion] {} {{C} \bc \langle Q \rangle}
  \and
  \inferrule* [lab=process] {} {{P,Q} \bc M \;| \;P|Q \;|\; @{x}}
  \and
  \inferrule* [lab=name] {} {{x} \bc \quotep{P}}
\end{mathpar} 

Note that $\vec{x}$ (resp. $\vec{P}$) denotes a vector of names
(resp. processes) of length $|\vec{x}|$ (resp. $|\vec{P}|$). We adopt
the following useful abbreviations.

\begin{mathpar}
   x?(\vec{y}).P := x.(\vec{y})P \and  x\clift{\vec{P}} := x.\clift{\vec{P}}
   \and x!(y) := \lift{x}{\dropn{y}}
   \and \Pi_{i=0}^{n-1}P_i := P_0 | \ldots | P_{n-1}
\end{mathpar}

\subsubsection{Structural congruence}

\paragraph{Free and bound names and alpha-equivalence.} At the
core of structural equivalence is alpha-equivalence which identifies
process that are the same up to a change of variable. Formally, we
recognize the distinction between free and bound names. The free names
of a process, $\freenames{P}$, may be calculated recursively as
follows:

\begin{mathpar}
\freenames{\pzero} := \emptyset
  \and \\
  \freenames{x?(y).P} := \{ x \} \cup (\freenames{P} \setminus \{ y \})
  \and 
  \freenames{x!\langle P \rangle} := \{ x \} \cup \{ P \} 
  \and \\
  \freenames{P|Q} := \freenames{P} \cup \freenames{Q}
  \and \\
  \freenames{@{x}} := \{ x \}
\end{mathpar}

$\pi$
$\quotep{\pi}$

$\freenames{-} : \pi \to \mathcal{P}(\quotep{\pi})$

\begin{eqnarray*}
  \freenames{\pzero} & := & \emptyset \\
  \freenames{x?(y).P} & := & \{ x \} \cup (\freenames{P} \setminus \{ y \}) \\
  \freenames{x!\langle P \rangle} & := & \{ x \} \cup \{ P \} \\
  \freenames{P|Q} & := & \freenames{P} \cup \freenames{Q} \\
  \freenames{\dropn{x}} & := & \{ x \}
\end{eqnarray*}

The bound names of a process, $\boundnames{P}$, are those names occurring in $P$
that are not free. For example, in $x?(y).0$, the name $x$ is free, while $y$ is bound.

\begin{mathpar}
  \inferrule* [lab=monoidal-laws] {} { P|Q \equiv Q|P \and P|0 \equiv P \and P|(Q|R) \equiv (P|Q)|R }
\end{mathpar}

\begin{mathpar}
  \inferrule* [lab=alpha-equivalence] {} { (x)P \equiv (y)P\{y/x\} \and y \not\in \freenames{P} }
\end{mathpar}

\begin{definition}
Then two processes, $P,Q$, are alpha-equivalent if $P = Q\{\vec{y}/\vec{x}\}$ for
some $\vec{x} \in \boundnames{Q},\vec{y} \in \boundnames{P}$, where $Q\{\vec{y}/\vec{x}\}$
denotes the capture-avoiding substitution of $\vec{y}$ for $\vec{x}$ in $Q$.
\end{definition}

\begin{definition}
  The {\em structural congruence} \cite{SangiorgiWalker} , $\equiv$,
  between processes is the least congruence containing
  alpha-equivalence, satisfying the abelian monoid laws
  (associativity, commutativity and $\pzero$ as identity) for parallel
  composition $|$ and for summation $+$.
\end{definition}

\subsection{Name equivalence}

We take name equivalence, written $\nameeq$, to be the smallest
equivalence relation generated by the following rules.

\begin{mathpar}
\inferrule*[lab=Quote-drop]
{ }
{ \quotep{@{x}} \nameeq x }

\inferrule*[lab=Struct-equiv]
{ P \scong Q }
{ \quotep{P} \nameeq \quotep{Q} }
\end{mathpar}

The astute reader will have noticed that the mutual recursion of names
and processes imposes a mutual recursion on alpha-equivalence and
structural equivalence via name-equivalence. Fortunately, all of this
works out pleasantly and we may calculate in the natural way, free of
concern. The reader interested in the details is referred to the
appendix \ref{appendix:rho_details}.

\subsection{Substitution}

We use $\Proc$ for the set of processes, $\QProc$ for the set of
names, and $\id{\{}\vec{y} / \vec{x} \id{\}}$ to denote partial maps,
$s : \QProc \rightarrow \QProc$. A map, $s$ lifts, uniquely, to a map
on process terms, $\widehat{s} : \Proc \rightarrow \Proc$ by the
following equations.

\begin{mathpar}
  (0) \psubstp{Q}{P} := 0 \\
  (R \juxtap S) \psubstp{Q}{P}
  :=    
  (R)\psubstp{Q}{P} \juxtap (S) \psubstp{Q}{P} \\
  (x?(y).R) \psubstp{Q}{P}    
  :=    
  (x)\substp{Q}{P} (z)\concat( (R \psubstn{z}{y}) \psubstp{Q}{P} ) \\
  (\lift{x}{R}) \psubstp{Q}{P}  
  :=
  \lift{(x)\substp{Q}{P}}{ R \psubstp{Q}{P} } \\
%   (\dropn{x})  \psubstp{Q}{P}       
%   := 
%   \left\{ 
%     \begin{array}{ccc} 
%       \dropn{\quotep{Q}} & & x \nameeq \quotep{P} \\
%       \dropn{x} & & otherwise \\
%     \end{array}
%   \right. 
  (\dropn{x})  \psubstp{Q}{P}       
  := 
  \left\{ 
    \begin{array}{ccc} 
      Q & & x \nameeq \quotep{P} \\
      \dropn{x} & & otherwise \\
    \end{array}
  \right.
\end{mathpar}
 

where

\begin{eqnarray}
  (x)\id{\{} \lpquote Q \rpquote / \lpquote P \rpquote \id{\}}            = 
  \left\{ 
    \begin{array}{ccc}
      \lpquote Q \rpquote & & x \nameeq \lpquote P \rpquote \\
      x & & otherwise \\
    \end{array}
  \right. \nonumber
\end{eqnarray}

and $z$ is chosen distinct from $\quotep{P}$, $\quotep{Q}$, the free
names in $Q$, and all the names in $R$. Our $\alpha$-equivalence will
be built in the standard way from this substitution.

\begin{remark}\label{rem:no_self_referential_names}
  One consequence of these definitions is that $\forall P. \quotep{P}
  \not\in \freenames{P}$.
\end{remark}

\subsection{ Dynamic quote: an example }

Anticipating something of what's to come, consider applying the
substitution, $\widehat{\id{\{}u / z \id{\}}}$, to the following pair
of processes, $\lift{w}{y!(z)}$ and $w[ \lpquote y!(z) \rpquote ]$.

\begin{eqnarray}
	\lift{w}{y!(z)}\widehat{\id{\{}u / z \id{\}}}
		& = &
		\lift{w}{y!(u)} \nonumber\\
	w[ \lpquote y!(z) \rpquote ] \widehat{ \id{\{}u / z \id{\}} }
		& = &
		w[ \lpquote y!(z) \rpquote ] \nonumber
\end{eqnarray}

Because the body of the process between quotes is impervious to
substitution, we get radically different answers. In fact, by
examining the first process in an input context,
e.g. $x?(z).\lift{w}{y!(z)}$, we see that the process under the lift
operator may be shaped by prefixed inputs binding a name inside it. In
this sense, the lift operator will be seen as a way to dynamically
construct processes before reifying them as names.

Finally equipped with these standard features we can present the
dynamics of the calculus.

\subsubsection{Operational semantics} 

Finally, we introduce the computational dynamics. What marks these
algebras as distinct from other more traditionally studied algebraic
structures, e.g. vector spaces or polynomial rings, is the manner in
which dynamics is captured. In traditional structures, dynamics is typically
expressed through morphisms between such structures, as in linear maps
between vector spaces or morphisms between rings. In algebras
associated with the semantics of computation, the dynamics is
expressed as part of the algebraic structure itself, through a
reduction reduction relation typically denoted by $\red$. Below, we
give a recursive presentation of this relation for the calculus used
in the encoding.

$\red \subseteq \pi \times \pi$
$\red : \pi \to \mathcal{P}(\pi)$

\begin{mathpar}
  \inferrule* [lab=Comm] { \textsf{match}( x_{src}, x_{trgt} ) } { x_{trgt}?(y)P \; | \; x_{src}!\langle {Q} \rangle \red P\{\quotep{Q}/y}\} }
  \and \\
  \inferrule* [lab=Par] {{P} \red {P}'} {{{P} | {Q}} \red {{P}' | {Q}}}
  \and
  \inferrule* [lab=Equiv]{{{P} \scong {P}'} \andalso {{P}' \red {Q}'} \andalso {{Q}' \scong {Q}}}{{P} \red {Q}}
\end{mathpar}

\begin{eqnarray*}
  match_{\equiv} (\quotep{P},\quotep{Q}) & := & P \equiv Q \\
  match_{\dagger}(\quotep{P},\quotep{Q}) & := & \forall R. P|Q \red^{*} R => R \red^{*} 0 \\
  match_{K}(\quotep{P},\quotep{Q}) & := & K \mbox{ for some context } K
\end{eqnarray*}

$u?(x)P | u!\langle Q \rangle \red P\{\quotep{Q}/x\}$

%We write $\wred$ for $\red^*$, and $P\red$ if $\exists Q $ such that $ P \red Q$.
We write $P\red$ if $\exists Q $ such that $ P \red Q$ and $P\not\red$, otherwise.

\section{Replication}

As mentioned before, it is known that replication (and hence
recursion) can be implemented in a higher-order process algebra
\cite{SangiorgiWalker}. As our first example of calculation with the
machinery thus far presented we give the construction explicitly in
the {\rhoc}.

\begin{eqnarray}
	D_{x} & := & \prefix{x}{y}{(\binpar{\outputp{x}{y}}{@{y}})} \nonumber\\
	\bangp_{x}{P} & := & \binpar{{x}!\langle{\binpar{D_{x}}{P}}\rangle}{D_{x}} \nonumber
\end{eqnarray}

\begin{eqnarray}
	\bangp_{x}{P} & & \nonumber\\
	=
	& {x}!\langle{(\prefix{x}{y}{(\outputp{x}{y} | @{y})) | P}}\rangle 
	      | \prefix{x}{y}{(\outputp{x}{y} | @{y})} & \nonumber\\
	\red
	& (\outputp{x}{y} | @{y})\substn{\quotep{(\prefix{x}{y}{(@{y} | \outputp{x}{y})) | P}}}{y} & \nonumber\\
	=
	& \outputp{x}{\quotep{(\prefix{x}{y}{(\outputp{x}{y} | @{y})) | P}}}
	  | {(\prefix{x}{y}{(\outputp{x}{y} | @{y})) | P}} & \nonumber\\
	\red
	& \ldots & \nonumber\\
	\red^*
	& P | P | \ldots & \nonumber
\end{eqnarray}

Of course, this encoding, as an implementation, runs away, unfolding
$\bangp{P}$ eagerly. A lazier and more implementable replication
operator, restricted to input-guarded processes, may be obtained as follows.

\begin{eqnarray}
\bangp{\prefix{u}{v}{P}} 
	:= 
	\binpar{\lift{x}{\prefix{u}{v}{(\binpar{D(x)}{P})}}}{D(x)} \nonumber
\end{eqnarray}

\begin{remark}
  Note that the lazier definition still does not deal with summation
  or mixed summation (i.e. sums over input and output). The reader is
  invited to construct definitions of replication that deal with these
  features. 

  Further, the definitions are parameterized in a name, $x$. Can you,
  gentle reader, make a definition that eliminates this parameter and
  guarantees no accidental interaction between the replication
  machinery and the process being replicated -- i.e. no accidental
  sharing of names used by the process to get its work done and the
  name(s) used by the replication to effect copying. This latter
  revision of the definition of replication is crucial to obtaining
  the expected identity $!!P \sim !P$.
\end{remark}

\begin{remark}\label{rem:paradoxical_combinator}
  The reader familiar with the lambda calculus will have noticed the
  similarity between $D$ and the paradoxical combinator.

  [Ed. note: the existence of this seems to suggest we have to be more
  restrictive on the set of processes and names we admit if we are to
  support no-cloning.]
\end{remark}

\subsubsection{Bisimulation}

The computational dynamics gives rise to another kind of equivalence,
the equivalence of computational behavior. As previously mentioned
this is typically captured \emph{via} some form of bisimulation.

% The notion we use in this paper is weak barbed bisimulation
% \cite{milner91polyadicpi}.

The notion we use in this paper is derived from weak barbed
bisimulation \cite{milner91polyadicpi}. 

\begin{definition}
An \emph{observation relation}, $\downarrow_{\mathcal N}$, over a set
of names, $\mathcal N$, is the smallest relation satisfying the rules
below.

\infrule[Out-barb]{y \in {\mathcal N}, \; x \nameeq y}
		  {\outputp{x}{v} \downarrow_{\mathcal N} x}
\infrule[Par-barb]{\mbox{$P\downarrow_{\mathcal N} x$ or $Q\downarrow_{\mathcal N} x$}}
		  {\binpar{P}{Q} \downarrow_{\mathcal N} x}

We write $P \Downarrow_{\mathcal N} x$ if there is $Q$ such that 
$P \wred Q$ and $Q \downarrow_{\mathcal N} x$.
\end{definition}

\begin{definition}
%\label{def.bbisim}
An  ${\mathcal N}$-\emph{barbed bisimulation} over a set of names, ${\mathcal N}$, is a symmetric binary relation 
${\mathcal S}_{\mathcal N}$ between agents such that $P\rel{S}_{\mathcal N}Q$ implies:
\begin{enumerate}
\item If $P \red P'$ then $Q \wred Q'$ and $P'\rel{S}_{\mathcal N} Q'$.
\item If $P\downarrow_{\mathcal N} x$, then $Q\Downarrow_{\mathcal N} x$.
\end{enumerate}
$P$ is ${\mathcal N}$-barbed bisimilar to $Q$, written
$P \wbbisim_{\mathcal N} Q$, if $P \rel{S}_{\mathcal N} Q$ for some ${\mathcal N}$-barbed bisimulation ${\mathcal S}_{\mathcal N}$.
\end{definition}

$\mathcal{R} \subseteq \pi \times \pi$

$P \mathcal{R} Q => \forall P'. P \red P' \Rightarrow \exists Q'. Q \red Q', P' \mathcal{R} Q'$

$P \vdash x \Rightarrow Q \vdash x$

\begin{mathpar}
  \inferrule*[lab=Out-barb]{x \nameeq y}{{y}!\langle{Q}\rangle \vdash x}
  \and
  \inferrule*[lab=Par-barb]{\mbox{$P\vdash x$ or $Q\vdash x$}}{\binpar{P}{Q} \vdash x}
\end{mathpar}

\subsubsection{Contexts}

One of the principle advantages of computational calculi like the
$\pi$-calculus is a well-defined notion of context,
contextual-equivalence and a correlation between
contextual-equivalence and notions of bisimulation. The notion of
context allows the decomposition of a process into (sub-)process and
its syntactic environment, its context. Thus, a context may be
thought of as a process with a ``hole'' (written $\Box$) in it. The
application of a context $M$ to a process $P$, written $M[P]$, is
tantamount to filling the hole in $M$ with $P$. In this paper we do
not need the full weight of this theory, but do make use of the notion
of context in the proof the main theorem. 

\begin{mathpar}
  \inferrule* [lab=summation] {} {{M_{M},M_{N}} \bc \Box \;|\; x.M_{A} \;|\; M_{M}+M_{N}}
  \and
  \inferrule* [lab=agent] {} {{M_{A}} \bc (\vec{x})M_{P} \;| \; \clift{P_0,\ldots,M_{P},\ldots,P_N}}
  \and \\
  \inferrule* [lab=process] {} {{M_{P}} \bc M_{N} \;| \;P|M_{P} }
\end{mathpar} 

\begin{mathpar}
  \inferrule* [lab=sychronization] {} {M_{N} \bc \Box \;|\; x?M_{F} \;|\; x!M_{C}}
  \and
  \inferrule* [lab=abstraction] {} {{M_{F}} \bc (x)M_{P} }
  \and
  \inferrule* [lab=concretion] {} {{M_{C}} \bc \langle M_{P} \rangle }
  \and \\
  \inferrule* [lab=process] {} {{M_{P}} \bc M_{N} \;| \;P|M_{P} }
\end{mathpar}

\begin{definition}[contextual application] Given a context $M$, and
  process $P$, we define the \emph{contextual application}, $M[P] :=
  M\{P/\Box\}$. That is, the contextual application of M to P is the
  substitution of $P$ for $\Box$ in $M$.
\end{definition}

$\meaningof{-} : L \to \mathcal{P}(\pi)$

\begin{mathpar}
  \inferrule* [lab=collection] {} {\meaningof{true} = \pi, \and \meaningof{~E} = \pi \setminus \meaningof{E}, \and \meaningof{E_{1} \& E_{2}} = \meaningof{E_{1}} \cap \meaningof{E_{2}}}
\end{mathpar}

\begin{mathpar}
  \inferrule* [lab=structure] {} {\meaningof{0} = \{ P \in \pi | P \equiv 0 \}, \and \\ \meaningof{E_1 | E_2} = \{ P \in \pi | P \equiv P_{1} | P_{2}, P_{1} \in \meaningof{E_{1}}, P_{2} \in \meaningof{E_2}\} }
\end{mathpar}

\begin{mathpar}
 \inferrule* [lab=behavior] {} {\meaningof{\langle a?b \rangle E} = \{ P \in \pi | P \equiv Q | u?(y)P', \\ \and \\\\ \and \\ \;\;\; u \in \meaningof{a}, \forall z.P'\{z/y\} \in \meaningof{E\{z/b\}}\}, \and \\ \meaningof{a!E} = \{ P \in \pi | P \equiv Q | x!\langle P' \rangle, x \in \meaningof{a} P' \in \meaningof{E}\} }
\end{mathpar}

\begin{mathpar}
 \inferrule* [lab=nominal] {} {\meaningof{\quotep{E}} = \{ \quotep{P} \in \quotep{\pi} | P \in \meaningof{E} \}, \and \meaningof{\quotep{P}} = \{ \quotep{Q} \in \quotep{\pi} | P \equiv Q \} \and \\ \meaningof{@\quotep{E}} = \{ P \in \pi | P \equiv @x, x \in \meaningof{E} \}}
\end{mathpar}

\begin{eqnarray*}
  \\
  \meaningof{-} : TS \to ST
\end{eqnarray*}

\begin{eqnarray*}
  \\
  L : TS \to ST
\end{eqnarray*}

\begin{eqnarray*}
  \\
  P \models E \iff P \in \meaningof{E}
\end{eqnarray*}

\begin{eqnarray*}
  P \approx_{L} Q \iff \forall E \in L. P \models E \iff Q \models E
\end{eqnarray*}

\begin{eqnarray*}
  P \approx_{K} Q
\end{eqnarray*}

\begin{eqnarray*}
  P \approx Q
\end{eqnarray*}

$\approx_{K} = \approx = \approx_{L}$

\subsubsection{Contextual duality}

Note that contexts extend the quotation operation to a family of
operations from processes to names. Given a context, $M$, we can
define a \emph{nominal context}, $\quotep{M}$ by $\quotep{M}[P] :=
\quotep{M[P]}$. To foreshadow what is to come we observe that these
operations enjoy a duality with processes very much like the duality
between vectors and maps from vectors to scalars.

Further, because the calculus is essentially higher-order, we have a
correspondence between contexts and processes. More specifically,
given a name $x$ and a context $M$ we can construct $M^{*}_{x}$ such
that 

\begin{mathpar}
  M^{*}_{x} | \lift{x}{P} \red M[P]
\end{mathpar}

namely,

\begin{mathpar}
  M^{*}_{x} := x?(u).M[\dropn{u}]
\end{mathpar}

The dependence of $M^{*}_{x}$ on a name makes it an abstraction, 

\begin{mathpar}
  M^{*} := (x)x?(u).M[\dropn{u}]
\end{mathpar}

\subsection{Additional notation}

It will sometimes be convenient to denote the process a name
quotes. We already have the notation $x = \quotep{P}$, but it will be
convenient to introduce an alternate notation, $\procn{x}$, when we
want to emphasize the connection to the use of the name. Note that, by
virtue of name equivalence, $\quotep{\procn{x}} \nameeq x$; so, the
notation is consistent with previous definitions.

Further, because names have structure it is possible to effect
substitutions on the basis of that structure. This means we need to
upgrade our notation for substitutions, which we accomplish by
adapting comprehension notation. Thus,

\begin{mathpar}
  P\{ y / x : x \in S \}
\end{mathpar}

is interpreted to mean the process derived from P by replacing (in a
capture-avoiding manner) each occurrence of $x$ in $S$ by $y$. For example,

\begin{mathpar}
  P\{ \quotep{\procn{x}|\procn{x}} / x : x \in \freenames{P} \}
\end{mathpar}

will replace each (occurrence) of a free name $x$ in $P$ by
$\quotep{\procn{x}|\procn{x}}$.

Also, we will avail ourselves of the notation $x^{L}$ and $x^{R}$ to
denote injections of a name into disjoint copies of the name
space. There are numerous ways to accomplish this. One example can be
found in \cite{MeredithR05}. This notation overloads to vectors of
names: $\vec{x}^{\pi} := (x_{i}^{\pi} \; : \; 0 \leq i < |\vec{x}| )$ where $\pi \in \{L,R\}$.

We also use $P^{\Box} := P|\Box$.

In \cite{MeredithR05} an interpretation of the new operator is
given. It turns out that there are several possible interpretations
all enjoying the requisite algebraic properties of the operator (see
\cite{milner91polyadicpi}). We will therefore make liberal use of
$(\nu\; \vec{x})P$.

% subsection the_syntax_and_semantics_of_the_notation_system (end)   

\input{qm2pi.qmops} 

\input{qm2pi.sterngerlach} 

\input{qm2pi.metric} 

% section concurrent_process_calculi (end)

%\input{qm2pi.proofsketch}

% section proof sketch (end)

%\input{qm2pi.slviaknots} 

% section spatial logic via knots (end)

\input{qm2pi.conclusion}

% section conclusion (end)

%\input{qm2pi.dtcodes} 

% section wiring algorithm (end)

\input{qm2pi.ack} 

% section acknowledgments (end)

\newpage


\bibliographystyle{plain}   
\bibliography{../../biblios/main.bib}

\input{qm2pi.rhodetails}

\end{document}

 

%\documentclass[12pt]{llncs}
%\documentclass{jktr}

\usepackage[pdftex]{hyperref}                   
\usepackage {listings}
\usepackage {mathpartir}
\usepackage{bcprules}
%\usepackage{listings}
                       
\usepackage{graphicx} 
%\usepackage[margins=2.5cm,nohead,nofoot]{geometry}
%\usepackage{geometry}
\usepackage{amsfonts}
\usepackage{amstext}
\usepackage{latexsym}
\usepackage{amssymb}
\usepackage{color}


%\include{myPreamble}
\include{qm2pi.local} 

%\ifpdf
%\usepackage[pdftex]{graphicx}
%\else
%\usepackage{graphicx}
%\fi

 % \ifpdf
%  \usepackage{pdfsync}
%  \if


%\title{Brief Article}
%\author{David F. Snyder}
%\author{L.G. Meredith}

%\address{Dept. of Math., Texas State University--San Marcos, San Marcos, TX 78666}
       
\pagestyle{empty}


\begin{document}

\lstset{language=[Objective]Caml,frame=shadowbox}

\input{qm2pi.front}

% section front matter (end)

\input{qm2pi.intro} 
 
% section introduction (end)

% \input{qm2pi.knotations} 

% section notation (end)

\input{qm2pi.process.calculi} 

% section concurrent_process_calculi_and_spatial_logics_ (end)
    
%\input{qm2pi.knots2pi} 

%\input{qm2pi.trefoil} 

%\input{qm2pi.mainthm} 

% subsection basic_interpretation (end)

%\input{qm2pi.rho.presentation} 
\subsection{The syntax and semantics of the notation system}\label{sub:the_syntax_and_semantics_of_the_notation_system} % (fold)

We now summarize a technical presentation of the calculus that
embodies our theory of dynamics. The typical presentation of such a
calculus follows the style of giving generators and relations on
them. The grammar, below, describing term constructors, freely
generates the set of processes, $\Proc$. This set is then quotiented
by a relation known as structural congruence and it is over this set
that the notion of dynamics is expressed. This presentation is
essentially that of \cite{MeredithR05} with the addition of
polyadicity and summation. For readability we have relegated some of
the technical subtleties to an appendix.

\subsubsection{Process grammar}\label{subsub:process_grammar}

\begin{mathpar}
  \inferrule* [lab=synchronization] {} {{M} \bc \pzero \;|\; x?F \;|\; x!C }
  \and
  \inferrule* [lab=abstraction] {} {{F} \bc (x)P}
  \and
  \inferrule* [lab=concretion] {} {{C} \bc \langle Q \rangle}
  \and
  \inferrule* [lab=process] {} {{P,Q} \bc M \;| \;P|Q \;|\; @{x}}
  \and
  \inferrule* [lab=name] {} {{x} \bc \quotep{P}}
\end{mathpar} 

Note that $\vec{x}$ (resp. $\vec{P}$) denotes a vector of names
(resp. processes) of length $|\vec{x}|$ (resp. $|\vec{P}|$). We adopt
the following useful abbreviations.

\begin{mathpar}
   x?(\vec{y}).P := x.(\vec{y})P \and  x\clift{\vec{P}} := x.\clift{\vec{P}}
   \and x!(y) := \lift{x}{\dropn{y}}
   \and \Pi_{i=0}^{n-1}P_i := P_0 | \ldots | P_{n-1}
\end{mathpar}

\subsubsection{Structural congruence}

\paragraph{Free and bound names and alpha-equivalence.} At the
core of structural equivalence is alpha-equivalence which identifies
process that are the same up to a change of variable. Formally, we
recognize the distinction between free and bound names. The free names
of a process, $\freenames{P}$, may be calculated recursively as
follows:

\begin{mathpar}
\freenames{\pzero} := \emptyset
  \and \\
  \freenames{x?(y).P} := \{ x \} \cup (\freenames{P} \setminus \{ y \})
  \and 
  \freenames{x!\langle P \rangle} := \{ x \} \cup \{ P \} 
  \and \\
  \freenames{P|Q} := \freenames{P} \cup \freenames{Q}
  \and \\
  \freenames{@{x}} := \{ x \}
\end{mathpar}

$\pi$
$\quotep{\pi}$

$\freenames{-} : \pi \to \mathcal{P}(\quotep{\pi})$

\begin{eqnarray*}
  \freenames{\pzero} & := & \emptyset \\
  \freenames{x?(y).P} & := & \{ x \} \cup (\freenames{P} \setminus \{ y \}) \\
  \freenames{x!\langle P \rangle} & := & \{ x \} \cup \{ P \} \\
  \freenames{P|Q} & := & \freenames{P} \cup \freenames{Q} \\
  \freenames{\dropn{x}} & := & \{ x \}
\end{eqnarray*}

The bound names of a process, $\boundnames{P}$, are those names occurring in $P$
that are not free. For example, in $x?(y).0$, the name $x$ is free, while $y$ is bound.

\begin{mathpar}
  \inferrule* [lab=monoidal-laws] {} { P|Q \equiv Q|P \and P|0 \equiv P \and P|(Q|R) \equiv (P|Q)|R }
\end{mathpar}

\begin{mathpar}
  \inferrule* [lab=alpha-equivalence] {} { (x)P \equiv (y)P\{y/x\} \and y \not\in \freenames{P} }
\end{mathpar}

\begin{definition}
Then two processes, $P,Q$, are alpha-equivalent if $P = Q\{\vec{y}/\vec{x}\}$ for
some $\vec{x} \in \boundnames{Q},\vec{y} \in \boundnames{P}$, where $Q\{\vec{y}/\vec{x}\}$
denotes the capture-avoiding substitution of $\vec{y}$ for $\vec{x}$ in $Q$.
\end{definition}

\begin{definition}
  The {\em structural congruence} \cite{SangiorgiWalker} , $\equiv$,
  between processes is the least congruence containing
  alpha-equivalence, satisfying the abelian monoid laws
  (associativity, commutativity and $\pzero$ as identity) for parallel
  composition $|$ and for summation $+$.
\end{definition}

\subsection{Name equivalence}

We take name equivalence, written $\nameeq$, to be the smallest
equivalence relation generated by the following rules.

\begin{mathpar}
\inferrule*[lab=Quote-drop]
{ }
{ \quotep{@{x}} \nameeq x }

\inferrule*[lab=Struct-equiv]
{ P \scong Q }
{ \quotep{P} \nameeq \quotep{Q} }
\end{mathpar}

The astute reader will have noticed that the mutual recursion of names
and processes imposes a mutual recursion on alpha-equivalence and
structural equivalence via name-equivalence. Fortunately, all of this
works out pleasantly and we may calculate in the natural way, free of
concern. The reader interested in the details is referred to the
appendix \ref{appendix:rho_details}.

\subsection{Substitution}

We use $\Proc$ for the set of processes, $\QProc$ for the set of
names, and $\id{\{}\vec{y} / \vec{x} \id{\}}$ to denote partial maps,
$s : \QProc \rightarrow \QProc$. A map, $s$ lifts, uniquely, to a map
on process terms, $\widehat{s} : \Proc \rightarrow \Proc$ by the
following equations.

\begin{mathpar}
  (0) \psubstp{Q}{P} := 0 \\
  (R \juxtap S) \psubstp{Q}{P}
  :=    
  (R)\psubstp{Q}{P} \juxtap (S) \psubstp{Q}{P} \\
  (x?(y).R) \psubstp{Q}{P}    
  :=    
  (x)\substp{Q}{P} (z)\concat( (R \psubstn{z}{y}) \psubstp{Q}{P} ) \\
  (\lift{x}{R}) \psubstp{Q}{P}  
  :=
  \lift{(x)\substp{Q}{P}}{ R \psubstp{Q}{P} } \\
%   (\dropn{x})  \psubstp{Q}{P}       
%   := 
%   \left\{ 
%     \begin{array}{ccc} 
%       \dropn{\quotep{Q}} & & x \nameeq \quotep{P} \\
%       \dropn{x} & & otherwise \\
%     \end{array}
%   \right. 
  (\dropn{x})  \psubstp{Q}{P}       
  := 
  \left\{ 
    \begin{array}{ccc} 
      Q & & x \nameeq \quotep{P} \\
      \dropn{x} & & otherwise \\
    \end{array}
  \right.
\end{mathpar}
 

where

\begin{eqnarray}
  (x)\id{\{} \lpquote Q \rpquote / \lpquote P \rpquote \id{\}}            = 
  \left\{ 
    \begin{array}{ccc}
      \lpquote Q \rpquote & & x \nameeq \lpquote P \rpquote \\
      x & & otherwise \\
    \end{array}
  \right. \nonumber
\end{eqnarray}

and $z$ is chosen distinct from $\quotep{P}$, $\quotep{Q}$, the free
names in $Q$, and all the names in $R$. Our $\alpha$-equivalence will
be built in the standard way from this substitution.

\begin{remark}\label{rem:no_self_referential_names}
  One consequence of these definitions is that $\forall P. \quotep{P}
  \not\in \freenames{P}$.
\end{remark}

\subsection{ Dynamic quote: an example }

Anticipating something of what's to come, consider applying the
substitution, $\widehat{\id{\{}u / z \id{\}}}$, to the following pair
of processes, $\lift{w}{y!(z)}$ and $w[ \lpquote y!(z) \rpquote ]$.

\begin{eqnarray}
	\lift{w}{y!(z)}\widehat{\id{\{}u / z \id{\}}}
		& = &
		\lift{w}{y!(u)} \nonumber\\
	w[ \lpquote y!(z) \rpquote ] \widehat{ \id{\{}u / z \id{\}} }
		& = &
		w[ \lpquote y!(z) \rpquote ] \nonumber
\end{eqnarray}

Because the body of the process between quotes is impervious to
substitution, we get radically different answers. In fact, by
examining the first process in an input context,
e.g. $x?(z).\lift{w}{y!(z)}$, we see that the process under the lift
operator may be shaped by prefixed inputs binding a name inside it. In
this sense, the lift operator will be seen as a way to dynamically
construct processes before reifying them as names.

Finally equipped with these standard features we can present the
dynamics of the calculus.

\subsubsection{Operational semantics} 

Finally, we introduce the computational dynamics. What marks these
algebras as distinct from other more traditionally studied algebraic
structures, e.g. vector spaces or polynomial rings, is the manner in
which dynamics is captured. In traditional structures, dynamics is typically
expressed through morphisms between such structures, as in linear maps
between vector spaces or morphisms between rings. In algebras
associated with the semantics of computation, the dynamics is
expressed as part of the algebraic structure itself, through a
reduction reduction relation typically denoted by $\red$. Below, we
give a recursive presentation of this relation for the calculus used
in the encoding.

$\red \subseteq \pi \times \pi$
$\red : \pi \to \mathcal{P}(\pi)$

\begin{mathpar}
  \inferrule* [lab=Comm] { \textsf{match}( x_{src}, x_{trgt} ) } { x_{trgt}?(y)P \; | \; x_{src}!\langle {Q} \rangle \red P\{\quotep{Q}/y}\} }
  \and \\
  \inferrule* [lab=Par] {{P} \red {P}'} {{{P} | {Q}} \red {{P}' | {Q}}}
  \and
  \inferrule* [lab=Equiv]{{{P} \scong {P}'} \andalso {{P}' \red {Q}'} \andalso {{Q}' \scong {Q}}}{{P} \red {Q}}
\end{mathpar}

\begin{eqnarray*}
  match_{\equiv} (\quotep{P},\quotep{Q}) & := & P \equiv Q \\
  match_{\dagger}(\quotep{P},\quotep{Q}) & := & \forall R. P|Q \red^{*} R => R \red^{*} 0 \\
  match_{K}(\quotep{P},\quotep{Q}) & := & K \mbox{ for some context } K
\end{eqnarray*}

$u?(x)P | u!\langle Q \rangle \red P\{\quotep{Q}/x\}$

%We write $\wred$ for $\red^*$, and $P\red$ if $\exists Q $ such that $ P \red Q$.
We write $P\red$ if $\exists Q $ such that $ P \red Q$ and $P\not\red$, otherwise.

\section{Replication}

As mentioned before, it is known that replication (and hence
recursion) can be implemented in a higher-order process algebra
\cite{SangiorgiWalker}. As our first example of calculation with the
machinery thus far presented we give the construction explicitly in
the {\rhoc}.

\begin{eqnarray}
	D_{x} & := & \prefix{x}{y}{(\binpar{\outputp{x}{y}}{@{y}})} \nonumber\\
	\bangp_{x}{P} & := & \binpar{{x}!\langle{\binpar{D_{x}}{P}}\rangle}{D_{x}} \nonumber
\end{eqnarray}

\begin{eqnarray}
	\bangp_{x}{P} & & \nonumber\\
	=
	& {x}!\langle{(\prefix{x}{y}{(\outputp{x}{y} | @{y})) | P}}\rangle 
	      | \prefix{x}{y}{(\outputp{x}{y} | @{y})} & \nonumber\\
	\red
	& (\outputp{x}{y} | @{y})\substn{\quotep{(\prefix{x}{y}{(@{y} | \outputp{x}{y})) | P}}}{y} & \nonumber\\
	=
	& \outputp{x}{\quotep{(\prefix{x}{y}{(\outputp{x}{y} | @{y})) | P}}}
	  | {(\prefix{x}{y}{(\outputp{x}{y} | @{y})) | P}} & \nonumber\\
	\red
	& \ldots & \nonumber\\
	\red^*
	& P | P | \ldots & \nonumber
\end{eqnarray}

Of course, this encoding, as an implementation, runs away, unfolding
$\bangp{P}$ eagerly. A lazier and more implementable replication
operator, restricted to input-guarded processes, may be obtained as follows.

\begin{eqnarray}
\bangp{\prefix{u}{v}{P}} 
	:= 
	\binpar{\lift{x}{\prefix{u}{v}{(\binpar{D(x)}{P})}}}{D(x)} \nonumber
\end{eqnarray}

\begin{remark}
  Note that the lazier definition still does not deal with summation
  or mixed summation (i.e. sums over input and output). The reader is
  invited to construct definitions of replication that deal with these
  features. 

  Further, the definitions are parameterized in a name, $x$. Can you,
  gentle reader, make a definition that eliminates this parameter and
  guarantees no accidental interaction between the replication
  machinery and the process being replicated -- i.e. no accidental
  sharing of names used by the process to get its work done and the
  name(s) used by the replication to effect copying. This latter
  revision of the definition of replication is crucial to obtaining
  the expected identity $!!P \sim !P$.
\end{remark}

\begin{remark}\label{rem:paradoxical_combinator}
  The reader familiar with the lambda calculus will have noticed the
  similarity between $D$ and the paradoxical combinator.

  [Ed. note: the existence of this seems to suggest we have to be more
  restrictive on the set of processes and names we admit if we are to
  support no-cloning.]
\end{remark}

\subsubsection{Bisimulation}

The computational dynamics gives rise to another kind of equivalence,
the equivalence of computational behavior. As previously mentioned
this is typically captured \emph{via} some form of bisimulation.

% The notion we use in this paper is weak barbed bisimulation
% \cite{milner91polyadicpi}.

The notion we use in this paper is derived from weak barbed
bisimulation \cite{milner91polyadicpi}. 

\begin{definition}
An \emph{observation relation}, $\downarrow_{\mathcal N}$, over a set
of names, $\mathcal N$, is the smallest relation satisfying the rules
below.

\infrule[Out-barb]{y \in {\mathcal N}, \; x \nameeq y}
		  {\outputp{x}{v} \downarrow_{\mathcal N} x}
\infrule[Par-barb]{\mbox{$P\downarrow_{\mathcal N} x$ or $Q\downarrow_{\mathcal N} x$}}
		  {\binpar{P}{Q} \downarrow_{\mathcal N} x}

We write $P \Downarrow_{\mathcal N} x$ if there is $Q$ such that 
$P \wred Q$ and $Q \downarrow_{\mathcal N} x$.
\end{definition}

\begin{definition}
%\label{def.bbisim}
An  ${\mathcal N}$-\emph{barbed bisimulation} over a set of names, ${\mathcal N}$, is a symmetric binary relation 
${\mathcal S}_{\mathcal N}$ between agents such that $P\rel{S}_{\mathcal N}Q$ implies:
\begin{enumerate}
\item If $P \red P'$ then $Q \wred Q'$ and $P'\rel{S}_{\mathcal N} Q'$.
\item If $P\downarrow_{\mathcal N} x$, then $Q\Downarrow_{\mathcal N} x$.
\end{enumerate}
$P$ is ${\mathcal N}$-barbed bisimilar to $Q$, written
$P \wbbisim_{\mathcal N} Q$, if $P \rel{S}_{\mathcal N} Q$ for some ${\mathcal N}$-barbed bisimulation ${\mathcal S}_{\mathcal N}$.
\end{definition}

$\mathcal{R} \subseteq \pi \times \pi$

$P \mathcal{R} Q => \forall P'. P \red P' \Rightarrow \exists Q'. Q \red Q', P' \mathcal{R} Q'$

$P \vdash x \Rightarrow Q \vdash x$

\begin{mathpar}
  \inferrule*[lab=Out-barb]{x \nameeq y}{{y}!\langle{Q}\rangle \vdash x}
  \and
  \inferrule*[lab=Par-barb]{\mbox{$P\vdash x$ or $Q\vdash x$}}{\binpar{P}{Q} \vdash x}
\end{mathpar}

\subsubsection{Contexts}

One of the principle advantages of computational calculi like the
$\pi$-calculus is a well-defined notion of context,
contextual-equivalence and a correlation between
contextual-equivalence and notions of bisimulation. The notion of
context allows the decomposition of a process into (sub-)process and
its syntactic environment, its context. Thus, a context may be
thought of as a process with a ``hole'' (written $\Box$) in it. The
application of a context $M$ to a process $P$, written $M[P]$, is
tantamount to filling the hole in $M$ with $P$. In this paper we do
not need the full weight of this theory, but do make use of the notion
of context in the proof the main theorem. 

\begin{mathpar}
  \inferrule* [lab=summation] {} {{M_{M},M_{N}} \bc \Box \;|\; x.M_{A} \;|\; M_{M}+M_{N}}
  \and
  \inferrule* [lab=agent] {} {{M_{A}} \bc (\vec{x})M_{P} \;| \; \clift{P_0,\ldots,M_{P},\ldots,P_N}}
  \and \\
  \inferrule* [lab=process] {} {{M_{P}} \bc M_{N} \;| \;P|M_{P} }
\end{mathpar} 

\begin{mathpar}
  \inferrule* [lab=sychronization] {} {M_{N} \bc \Box \;|\; x?M_{F} \;|\; x!M_{C}}
  \and
  \inferrule* [lab=abstraction] {} {{M_{F}} \bc (x)M_{P} }
  \and
  \inferrule* [lab=concretion] {} {{M_{C}} \bc \langle M_{P} \rangle }
  \and \\
  \inferrule* [lab=process] {} {{M_{P}} \bc M_{N} \;| \;P|M_{P} }
\end{mathpar}

\begin{definition}[contextual application] Given a context $M$, and
  process $P$, we define the \emph{contextual application}, $M[P] :=
  M\{P/\Box\}$. That is, the contextual application of M to P is the
  substitution of $P$ for $\Box$ in $M$.
\end{definition}

$\meaningof{-} : L \to \mathcal{P}(\pi)$

\begin{mathpar}
  \inferrule* [lab=collection] {} {\meaningof{true} = \pi, \and \meaningof{~E} = \pi \setminus \meaningof{E}, \and \meaningof{E_{1} \& E_{2}} = \meaningof{E_{1}} \cap \meaningof{E_{2}}}
\end{mathpar}

\begin{mathpar}
  \inferrule* [lab=structure] {} {\meaningof{0} = \{ P \in \pi | P \equiv 0 \}, \and \\ \meaningof{E_1 | E_2} = \{ P \in \pi | P \equiv P_{1} | P_{2}, P_{1} \in \meaningof{E_{1}}, P_{2} \in \meaningof{E_2}\} }
\end{mathpar}

\begin{mathpar}
 \inferrule* [lab=behavior] {} {\meaningof{\langle a?b \rangle E} = \{ P \in \pi | P \equiv Q | u?(y)P', \\ \and \\\\ \and \\ \;\;\; u \in \meaningof{a}, \forall z.P'\{z/y\} \in \meaningof{E\{z/b\}}\}, \and \\ \meaningof{a!E} = \{ P \in \pi | P \equiv Q | x!\langle P' \rangle, x \in \meaningof{a} P' \in \meaningof{E}\} }
\end{mathpar}

\begin{mathpar}
 \inferrule* [lab=nominal] {} {\meaningof{\quotep{E}} = \{ \quotep{P} \in \quotep{\pi} | P \in \meaningof{E} \}, \and \meaningof{\quotep{P}} = \{ \quotep{Q} \in \quotep{\pi} | P \equiv Q \} \and \\ \meaningof{@\quotep{E}} = \{ P \in \pi | P \equiv @x, x \in \meaningof{E} \}}
\end{mathpar}

\begin{eqnarray*}
  \\
  \meaningof{-} : TS \to ST
\end{eqnarray*}

\begin{eqnarray*}
  \\
  L : TS \to ST
\end{eqnarray*}

\begin{eqnarray*}
  \\
  P \models E \iff P \in \meaningof{E}
\end{eqnarray*}

\begin{eqnarray*}
  P \approx_{L} Q \iff \forall E \in L. P \models E \iff Q \models E
\end{eqnarray*}

\begin{eqnarray*}
  P \approx_{K} Q
\end{eqnarray*}

\begin{eqnarray*}
  P \approx Q
\end{eqnarray*}

$\approx_{K} = \approx = \approx_{L}$

\subsubsection{Contextual duality}

Note that contexts extend the quotation operation to a family of
operations from processes to names. Given a context, $M$, we can
define a \emph{nominal context}, $\quotep{M}$ by $\quotep{M}[P] :=
\quotep{M[P]}$. To foreshadow what is to come we observe that these
operations enjoy a duality with processes very much like the duality
between vectors and maps from vectors to scalars.

Further, because the calculus is essentially higher-order, we have a
correspondence between contexts and processes. More specifically,
given a name $x$ and a context $M$ we can construct $M^{*}_{x}$ such
that 

\begin{mathpar}
  M^{*}_{x} | \lift{x}{P} \red M[P]
\end{mathpar}

namely,

\begin{mathpar}
  M^{*}_{x} := x?(u).M[\dropn{u}]
\end{mathpar}

The dependence of $M^{*}_{x}$ on a name makes it an abstraction, 

\begin{mathpar}
  M^{*} := (x)x?(u).M[\dropn{u}]
\end{mathpar}

\subsection{Additional notation}

It will sometimes be convenient to denote the process a name
quotes. We already have the notation $x = \quotep{P}$, but it will be
convenient to introduce an alternate notation, $\procn{x}$, when we
want to emphasize the connection to the use of the name. Note that, by
virtue of name equivalence, $\quotep{\procn{x}} \nameeq x$; so, the
notation is consistent with previous definitions.

Further, because names have structure it is possible to effect
substitutions on the basis of that structure. This means we need to
upgrade our notation for substitutions, which we accomplish by
adapting comprehension notation. Thus,

\begin{mathpar}
  P\{ y / x : x \in S \}
\end{mathpar}

is interpreted to mean the process derived from P by replacing (in a
capture-avoiding manner) each occurrence of $x$ in $S$ by $y$. For example,

\begin{mathpar}
  P\{ \quotep{\procn{x}|\procn{x}} / x : x \in \freenames{P} \}
\end{mathpar}

will replace each (occurrence) of a free name $x$ in $P$ by
$\quotep{\procn{x}|\procn{x}}$.

Also, we will avail ourselves of the notation $x^{L}$ and $x^{R}$ to
denote injections of a name into disjoint copies of the name
space. There are numerous ways to accomplish this. One example can be
found in \cite{MeredithR05}. This notation overloads to vectors of
names: $\vec{x}^{\pi} := (x_{i}^{\pi} \; : \; 0 \leq i < |\vec{x}| )$ where $\pi \in \{L,R\}$.

We also use $P^{\Box} := P|\Box$.

In \cite{MeredithR05} an interpretation of the new operator is
given. It turns out that there are several possible interpretations
all enjoying the requisite algebraic properties of the operator (see
\cite{milner91polyadicpi}). We will therefore make liberal use of
$(\nu\; \vec{x})P$.

% subsection the_syntax_and_semantics_of_the_notation_system (end)   

\input{qm2pi.qmops} 

\input{qm2pi.sterngerlach} 

\input{qm2pi.metric} 

% section concurrent_process_calculi (end)

%\input{qm2pi.proofsketch}

% section proof sketch (end)

%\input{qm2pi.slviaknots} 

% section spatial logic via knots (end)

\input{qm2pi.conclusion}

% section conclusion (end)

%\input{qm2pi.dtcodes} 

% section wiring algorithm (end)

\input{qm2pi.ack} 

% section acknowledgments (end)

\newpage


\bibliographystyle{plain}   
\bibliography{../../biblios/main.bib}

\input{qm2pi.rhodetails}

\end{document}

 

%\documentclass[12pt]{llncs}
%\documentclass{jktr}

\usepackage[pdftex]{hyperref}                   
\usepackage {listings}
\usepackage {mathpartir}
\usepackage{bcprules}
%\usepackage{listings}
                       
\usepackage{graphicx} 
%\usepackage[margins=2.5cm,nohead,nofoot]{geometry}
%\usepackage{geometry}
\usepackage{amsfonts}
\usepackage{amstext}
\usepackage{latexsym}
\usepackage{amssymb}
\usepackage{color}


%\include{myPreamble}
\include{qm2pi.local} 

%\ifpdf
%\usepackage[pdftex]{graphicx}
%\else
%\usepackage{graphicx}
%\fi

 % \ifpdf
%  \usepackage{pdfsync}
%  \if


%\title{Brief Article}
%\author{David F. Snyder}
%\author{L.G. Meredith}

%\address{Dept. of Math., Texas State University--San Marcos, San Marcos, TX 78666}
       
\pagestyle{empty}


\begin{document}

\lstset{language=[Objective]Caml,frame=shadowbox}

\input{qm2pi.front}

% section front matter (end)

\input{qm2pi.intro} 
 
% section introduction (end)

% \input{qm2pi.knotations} 

% section notation (end)

\input{qm2pi.process.calculi} 

% section concurrent_process_calculi_and_spatial_logics_ (end)
    
%\input{qm2pi.knots2pi} 

%\input{qm2pi.trefoil} 

%\input{qm2pi.mainthm} 

% subsection basic_interpretation (end)

%\input{qm2pi.rho.presentation} 
\subsection{The syntax and semantics of the notation system}\label{sub:the_syntax_and_semantics_of_the_notation_system} % (fold)

We now summarize a technical presentation of the calculus that
embodies our theory of dynamics. The typical presentation of such a
calculus follows the style of giving generators and relations on
them. The grammar, below, describing term constructors, freely
generates the set of processes, $\Proc$. This set is then quotiented
by a relation known as structural congruence and it is over this set
that the notion of dynamics is expressed. This presentation is
essentially that of \cite{MeredithR05} with the addition of
polyadicity and summation. For readability we have relegated some of
the technical subtleties to an appendix.

\subsubsection{Process grammar}\label{subsub:process_grammar}

\begin{mathpar}
  \inferrule* [lab=synchronization] {} {{M} \bc \pzero \;|\; x?F \;|\; x!C }
  \and
  \inferrule* [lab=abstraction] {} {{F} \bc (x)P}
  \and
  \inferrule* [lab=concretion] {} {{C} \bc \langle Q \rangle}
  \and
  \inferrule* [lab=process] {} {{P,Q} \bc M \;| \;P|Q \;|\; @{x}}
  \and
  \inferrule* [lab=name] {} {{x} \bc \quotep{P}}
\end{mathpar} 

Note that $\vec{x}$ (resp. $\vec{P}$) denotes a vector of names
(resp. processes) of length $|\vec{x}|$ (resp. $|\vec{P}|$). We adopt
the following useful abbreviations.

\begin{mathpar}
   x?(\vec{y}).P := x.(\vec{y})P \and  x\clift{\vec{P}} := x.\clift{\vec{P}}
   \and x!(y) := \lift{x}{\dropn{y}}
   \and \Pi_{i=0}^{n-1}P_i := P_0 | \ldots | P_{n-1}
\end{mathpar}

\subsubsection{Structural congruence}

\paragraph{Free and bound names and alpha-equivalence.} At the
core of structural equivalence is alpha-equivalence which identifies
process that are the same up to a change of variable. Formally, we
recognize the distinction between free and bound names. The free names
of a process, $\freenames{P}$, may be calculated recursively as
follows:

\begin{mathpar}
\freenames{\pzero} := \emptyset
  \and \\
  \freenames{x?(y).P} := \{ x \} \cup (\freenames{P} \setminus \{ y \})
  \and 
  \freenames{x!\langle P \rangle} := \{ x \} \cup \{ P \} 
  \and \\
  \freenames{P|Q} := \freenames{P} \cup \freenames{Q}
  \and \\
  \freenames{@{x}} := \{ x \}
\end{mathpar}

$\pi$
$\quotep{\pi}$

$\freenames{-} : \pi \to \mathcal{P}(\quotep{\pi})$

\begin{eqnarray*}
  \freenames{\pzero} & := & \emptyset \\
  \freenames{x?(y).P} & := & \{ x \} \cup (\freenames{P} \setminus \{ y \}) \\
  \freenames{x!\langle P \rangle} & := & \{ x \} \cup \{ P \} \\
  \freenames{P|Q} & := & \freenames{P} \cup \freenames{Q} \\
  \freenames{\dropn{x}} & := & \{ x \}
\end{eqnarray*}

The bound names of a process, $\boundnames{P}$, are those names occurring in $P$
that are not free. For example, in $x?(y).0$, the name $x$ is free, while $y$ is bound.

\begin{mathpar}
  \inferrule* [lab=monoidal-laws] {} { P|Q \equiv Q|P \and P|0 \equiv P \and P|(Q|R) \equiv (P|Q)|R }
\end{mathpar}

\begin{mathpar}
  \inferrule* [lab=alpha-equivalence] {} { (x)P \equiv (y)P\{y/x\} \and y \not\in \freenames{P} }
\end{mathpar}

\begin{definition}
Then two processes, $P,Q$, are alpha-equivalent if $P = Q\{\vec{y}/\vec{x}\}$ for
some $\vec{x} \in \boundnames{Q},\vec{y} \in \boundnames{P}$, where $Q\{\vec{y}/\vec{x}\}$
denotes the capture-avoiding substitution of $\vec{y}$ for $\vec{x}$ in $Q$.
\end{definition}

\begin{definition}
  The {\em structural congruence} \cite{SangiorgiWalker} , $\equiv$,
  between processes is the least congruence containing
  alpha-equivalence, satisfying the abelian monoid laws
  (associativity, commutativity and $\pzero$ as identity) for parallel
  composition $|$ and for summation $+$.
\end{definition}

\subsection{Name equivalence}

We take name equivalence, written $\nameeq$, to be the smallest
equivalence relation generated by the following rules.

\begin{mathpar}
\inferrule*[lab=Quote-drop]
{ }
{ \quotep{@{x}} \nameeq x }

\inferrule*[lab=Struct-equiv]
{ P \scong Q }
{ \quotep{P} \nameeq \quotep{Q} }
\end{mathpar}

The astute reader will have noticed that the mutual recursion of names
and processes imposes a mutual recursion on alpha-equivalence and
structural equivalence via name-equivalence. Fortunately, all of this
works out pleasantly and we may calculate in the natural way, free of
concern. The reader interested in the details is referred to the
appendix \ref{appendix:rho_details}.

\subsection{Substitution}

We use $\Proc$ for the set of processes, $\QProc$ for the set of
names, and $\id{\{}\vec{y} / \vec{x} \id{\}}$ to denote partial maps,
$s : \QProc \rightarrow \QProc$. A map, $s$ lifts, uniquely, to a map
on process terms, $\widehat{s} : \Proc \rightarrow \Proc$ by the
following equations.

\begin{mathpar}
  (0) \psubstp{Q}{P} := 0 \\
  (R \juxtap S) \psubstp{Q}{P}
  :=    
  (R)\psubstp{Q}{P} \juxtap (S) \psubstp{Q}{P} \\
  (x?(y).R) \psubstp{Q}{P}    
  :=    
  (x)\substp{Q}{P} (z)\concat( (R \psubstn{z}{y}) \psubstp{Q}{P} ) \\
  (\lift{x}{R}) \psubstp{Q}{P}  
  :=
  \lift{(x)\substp{Q}{P}}{ R \psubstp{Q}{P} } \\
%   (\dropn{x})  \psubstp{Q}{P}       
%   := 
%   \left\{ 
%     \begin{array}{ccc} 
%       \dropn{\quotep{Q}} & & x \nameeq \quotep{P} \\
%       \dropn{x} & & otherwise \\
%     \end{array}
%   \right. 
  (\dropn{x})  \psubstp{Q}{P}       
  := 
  \left\{ 
    \begin{array}{ccc} 
      Q & & x \nameeq \quotep{P} \\
      \dropn{x} & & otherwise \\
    \end{array}
  \right.
\end{mathpar}
 

where

\begin{eqnarray}
  (x)\id{\{} \lpquote Q \rpquote / \lpquote P \rpquote \id{\}}            = 
  \left\{ 
    \begin{array}{ccc}
      \lpquote Q \rpquote & & x \nameeq \lpquote P \rpquote \\
      x & & otherwise \\
    \end{array}
  \right. \nonumber
\end{eqnarray}

and $z$ is chosen distinct from $\quotep{P}$, $\quotep{Q}$, the free
names in $Q$, and all the names in $R$. Our $\alpha$-equivalence will
be built in the standard way from this substitution.

\begin{remark}\label{rem:no_self_referential_names}
  One consequence of these definitions is that $\forall P. \quotep{P}
  \not\in \freenames{P}$.
\end{remark}

\subsection{ Dynamic quote: an example }

Anticipating something of what's to come, consider applying the
substitution, $\widehat{\id{\{}u / z \id{\}}}$, to the following pair
of processes, $\lift{w}{y!(z)}$ and $w[ \lpquote y!(z) \rpquote ]$.

\begin{eqnarray}
	\lift{w}{y!(z)}\widehat{\id{\{}u / z \id{\}}}
		& = &
		\lift{w}{y!(u)} \nonumber\\
	w[ \lpquote y!(z) \rpquote ] \widehat{ \id{\{}u / z \id{\}} }
		& = &
		w[ \lpquote y!(z) \rpquote ] \nonumber
\end{eqnarray}

Because the body of the process between quotes is impervious to
substitution, we get radically different answers. In fact, by
examining the first process in an input context,
e.g. $x?(z).\lift{w}{y!(z)}$, we see that the process under the lift
operator may be shaped by prefixed inputs binding a name inside it. In
this sense, the lift operator will be seen as a way to dynamically
construct processes before reifying them as names.

Finally equipped with these standard features we can present the
dynamics of the calculus.

\subsubsection{Operational semantics} 

Finally, we introduce the computational dynamics. What marks these
algebras as distinct from other more traditionally studied algebraic
structures, e.g. vector spaces or polynomial rings, is the manner in
which dynamics is captured. In traditional structures, dynamics is typically
expressed through morphisms between such structures, as in linear maps
between vector spaces or morphisms between rings. In algebras
associated with the semantics of computation, the dynamics is
expressed as part of the algebraic structure itself, through a
reduction reduction relation typically denoted by $\red$. Below, we
give a recursive presentation of this relation for the calculus used
in the encoding.

$\red \subseteq \pi \times \pi$
$\red : \pi \to \mathcal{P}(\pi)$

\begin{mathpar}
  \inferrule* [lab=Comm] { \textsf{match}( x_{src}, x_{trgt} ) } { x_{trgt}?(y)P \; | \; x_{src}!\langle {Q} \rangle \red P\{\quotep{Q}/y}\} }
  \and \\
  \inferrule* [lab=Par] {{P} \red {P}'} {{{P} | {Q}} \red {{P}' | {Q}}}
  \and
  \inferrule* [lab=Equiv]{{{P} \scong {P}'} \andalso {{P}' \red {Q}'} \andalso {{Q}' \scong {Q}}}{{P} \red {Q}}
\end{mathpar}

\begin{eqnarray*}
  match_{\equiv} (\quotep{P},\quotep{Q}) & := & P \equiv Q \\
  match_{\dagger}(\quotep{P},\quotep{Q}) & := & \forall R. P|Q \red^{*} R => R \red^{*} 0 \\
  match_{K}(\quotep{P},\quotep{Q}) & := & K \mbox{ for some context } K
\end{eqnarray*}

$u?(x)P | u!\langle Q \rangle \red P\{\quotep{Q}/x\}$

%We write $\wred$ for $\red^*$, and $P\red$ if $\exists Q $ such that $ P \red Q$.
We write $P\red$ if $\exists Q $ such that $ P \red Q$ and $P\not\red$, otherwise.

\section{Replication}

As mentioned before, it is known that replication (and hence
recursion) can be implemented in a higher-order process algebra
\cite{SangiorgiWalker}. As our first example of calculation with the
machinery thus far presented we give the construction explicitly in
the {\rhoc}.

\begin{eqnarray}
	D_{x} & := & \prefix{x}{y}{(\binpar{\outputp{x}{y}}{@{y}})} \nonumber\\
	\bangp_{x}{P} & := & \binpar{{x}!\langle{\binpar{D_{x}}{P}}\rangle}{D_{x}} \nonumber
\end{eqnarray}

\begin{eqnarray}
	\bangp_{x}{P} & & \nonumber\\
	=
	& {x}!\langle{(\prefix{x}{y}{(\outputp{x}{y} | @{y})) | P}}\rangle 
	      | \prefix{x}{y}{(\outputp{x}{y} | @{y})} & \nonumber\\
	\red
	& (\outputp{x}{y} | @{y})\substn{\quotep{(\prefix{x}{y}{(@{y} | \outputp{x}{y})) | P}}}{y} & \nonumber\\
	=
	& \outputp{x}{\quotep{(\prefix{x}{y}{(\outputp{x}{y} | @{y})) | P}}}
	  | {(\prefix{x}{y}{(\outputp{x}{y} | @{y})) | P}} & \nonumber\\
	\red
	& \ldots & \nonumber\\
	\red^*
	& P | P | \ldots & \nonumber
\end{eqnarray}

Of course, this encoding, as an implementation, runs away, unfolding
$\bangp{P}$ eagerly. A lazier and more implementable replication
operator, restricted to input-guarded processes, may be obtained as follows.

\begin{eqnarray}
\bangp{\prefix{u}{v}{P}} 
	:= 
	\binpar{\lift{x}{\prefix{u}{v}{(\binpar{D(x)}{P})}}}{D(x)} \nonumber
\end{eqnarray}

\begin{remark}
  Note that the lazier definition still does not deal with summation
  or mixed summation (i.e. sums over input and output). The reader is
  invited to construct definitions of replication that deal with these
  features. 

  Further, the definitions are parameterized in a name, $x$. Can you,
  gentle reader, make a definition that eliminates this parameter and
  guarantees no accidental interaction between the replication
  machinery and the process being replicated -- i.e. no accidental
  sharing of names used by the process to get its work done and the
  name(s) used by the replication to effect copying. This latter
  revision of the definition of replication is crucial to obtaining
  the expected identity $!!P \sim !P$.
\end{remark}

\begin{remark}\label{rem:paradoxical_combinator}
  The reader familiar with the lambda calculus will have noticed the
  similarity between $D$ and the paradoxical combinator.

  [Ed. note: the existence of this seems to suggest we have to be more
  restrictive on the set of processes and names we admit if we are to
  support no-cloning.]
\end{remark}

\subsubsection{Bisimulation}

The computational dynamics gives rise to another kind of equivalence,
the equivalence of computational behavior. As previously mentioned
this is typically captured \emph{via} some form of bisimulation.

% The notion we use in this paper is weak barbed bisimulation
% \cite{milner91polyadicpi}.

The notion we use in this paper is derived from weak barbed
bisimulation \cite{milner91polyadicpi}. 

\begin{definition}
An \emph{observation relation}, $\downarrow_{\mathcal N}$, over a set
of names, $\mathcal N$, is the smallest relation satisfying the rules
below.

\infrule[Out-barb]{y \in {\mathcal N}, \; x \nameeq y}
		  {\outputp{x}{v} \downarrow_{\mathcal N} x}
\infrule[Par-barb]{\mbox{$P\downarrow_{\mathcal N} x$ or $Q\downarrow_{\mathcal N} x$}}
		  {\binpar{P}{Q} \downarrow_{\mathcal N} x}

We write $P \Downarrow_{\mathcal N} x$ if there is $Q$ such that 
$P \wred Q$ and $Q \downarrow_{\mathcal N} x$.
\end{definition}

\begin{definition}
%\label{def.bbisim}
An  ${\mathcal N}$-\emph{barbed bisimulation} over a set of names, ${\mathcal N}$, is a symmetric binary relation 
${\mathcal S}_{\mathcal N}$ between agents such that $P\rel{S}_{\mathcal N}Q$ implies:
\begin{enumerate}
\item If $P \red P'$ then $Q \wred Q'$ and $P'\rel{S}_{\mathcal N} Q'$.
\item If $P\downarrow_{\mathcal N} x$, then $Q\Downarrow_{\mathcal N} x$.
\end{enumerate}
$P$ is ${\mathcal N}$-barbed bisimilar to $Q$, written
$P \wbbisim_{\mathcal N} Q$, if $P \rel{S}_{\mathcal N} Q$ for some ${\mathcal N}$-barbed bisimulation ${\mathcal S}_{\mathcal N}$.
\end{definition}

$\mathcal{R} \subseteq \pi \times \pi$

$P \mathcal{R} Q => \forall P'. P \red P' \Rightarrow \exists Q'. Q \red Q', P' \mathcal{R} Q'$

$P \vdash x \Rightarrow Q \vdash x$

\begin{mathpar}
  \inferrule*[lab=Out-barb]{x \nameeq y}{{y}!\langle{Q}\rangle \vdash x}
  \and
  \inferrule*[lab=Par-barb]{\mbox{$P\vdash x$ or $Q\vdash x$}}{\binpar{P}{Q} \vdash x}
\end{mathpar}

\subsubsection{Contexts}

One of the principle advantages of computational calculi like the
$\pi$-calculus is a well-defined notion of context,
contextual-equivalence and a correlation between
contextual-equivalence and notions of bisimulation. The notion of
context allows the decomposition of a process into (sub-)process and
its syntactic environment, its context. Thus, a context may be
thought of as a process with a ``hole'' (written $\Box$) in it. The
application of a context $M$ to a process $P$, written $M[P]$, is
tantamount to filling the hole in $M$ with $P$. In this paper we do
not need the full weight of this theory, but do make use of the notion
of context in the proof the main theorem. 

\begin{mathpar}
  \inferrule* [lab=summation] {} {{M_{M},M_{N}} \bc \Box \;|\; x.M_{A} \;|\; M_{M}+M_{N}}
  \and
  \inferrule* [lab=agent] {} {{M_{A}} \bc (\vec{x})M_{P} \;| \; \clift{P_0,\ldots,M_{P},\ldots,P_N}}
  \and \\
  \inferrule* [lab=process] {} {{M_{P}} \bc M_{N} \;| \;P|M_{P} }
\end{mathpar} 

\begin{mathpar}
  \inferrule* [lab=sychronization] {} {M_{N} \bc \Box \;|\; x?M_{F} \;|\; x!M_{C}}
  \and
  \inferrule* [lab=abstraction] {} {{M_{F}} \bc (x)M_{P} }
  \and
  \inferrule* [lab=concretion] {} {{M_{C}} \bc \langle M_{P} \rangle }
  \and \\
  \inferrule* [lab=process] {} {{M_{P}} \bc M_{N} \;| \;P|M_{P} }
\end{mathpar}

\begin{definition}[contextual application] Given a context $M$, and
  process $P$, we define the \emph{contextual application}, $M[P] :=
  M\{P/\Box\}$. That is, the contextual application of M to P is the
  substitution of $P$ for $\Box$ in $M$.
\end{definition}

$\meaningof{-} : L \to \mathcal{P}(\pi)$

\begin{mathpar}
  \inferrule* [lab=collection] {} {\meaningof{true} = \pi, \and \meaningof{~E} = \pi \setminus \meaningof{E}, \and \meaningof{E_{1} \& E_{2}} = \meaningof{E_{1}} \cap \meaningof{E_{2}}}
\end{mathpar}

\begin{mathpar}
  \inferrule* [lab=structure] {} {\meaningof{0} = \{ P \in \pi | P \equiv 0 \}, \and \\ \meaningof{E_1 | E_2} = \{ P \in \pi | P \equiv P_{1} | P_{2}, P_{1} \in \meaningof{E_{1}}, P_{2} \in \meaningof{E_2}\} }
\end{mathpar}

\begin{mathpar}
 \inferrule* [lab=behavior] {} {\meaningof{\langle a?b \rangle E} = \{ P \in \pi | P \equiv Q | u?(y)P', \\ \and \\\\ \and \\ \;\;\; u \in \meaningof{a}, \forall z.P'\{z/y\} \in \meaningof{E\{z/b\}}\}, \and \\ \meaningof{a!E} = \{ P \in \pi | P \equiv Q | x!\langle P' \rangle, x \in \meaningof{a} P' \in \meaningof{E}\} }
\end{mathpar}

\begin{mathpar}
 \inferrule* [lab=nominal] {} {\meaningof{\quotep{E}} = \{ \quotep{P} \in \quotep{\pi} | P \in \meaningof{E} \}, \and \meaningof{\quotep{P}} = \{ \quotep{Q} \in \quotep{\pi} | P \equiv Q \} \and \\ \meaningof{@\quotep{E}} = \{ P \in \pi | P \equiv @x, x \in \meaningof{E} \}}
\end{mathpar}

\begin{eqnarray*}
  \\
  \meaningof{-} : TS \to ST
\end{eqnarray*}

\begin{eqnarray*}
  \\
  L : TS \to ST
\end{eqnarray*}

\begin{eqnarray*}
  \\
  P \models E \iff P \in \meaningof{E}
\end{eqnarray*}

\begin{eqnarray*}
  P \approx_{L} Q \iff \forall E \in L. P \models E \iff Q \models E
\end{eqnarray*}

\begin{eqnarray*}
  P \approx_{K} Q
\end{eqnarray*}

\begin{eqnarray*}
  P \approx Q
\end{eqnarray*}

$\approx_{K} = \approx = \approx_{L}$

\subsubsection{Contextual duality}

Note that contexts extend the quotation operation to a family of
operations from processes to names. Given a context, $M$, we can
define a \emph{nominal context}, $\quotep{M}$ by $\quotep{M}[P] :=
\quotep{M[P]}$. To foreshadow what is to come we observe that these
operations enjoy a duality with processes very much like the duality
between vectors and maps from vectors to scalars.

Further, because the calculus is essentially higher-order, we have a
correspondence between contexts and processes. More specifically,
given a name $x$ and a context $M$ we can construct $M^{*}_{x}$ such
that 

\begin{mathpar}
  M^{*}_{x} | \lift{x}{P} \red M[P]
\end{mathpar}

namely,

\begin{mathpar}
  M^{*}_{x} := x?(u).M[\dropn{u}]
\end{mathpar}

The dependence of $M^{*}_{x}$ on a name makes it an abstraction, 

\begin{mathpar}
  M^{*} := (x)x?(u).M[\dropn{u}]
\end{mathpar}

\subsection{Additional notation}

It will sometimes be convenient to denote the process a name
quotes. We already have the notation $x = \quotep{P}$, but it will be
convenient to introduce an alternate notation, $\procn{x}$, when we
want to emphasize the connection to the use of the name. Note that, by
virtue of name equivalence, $\quotep{\procn{x}} \nameeq x$; so, the
notation is consistent with previous definitions.

Further, because names have structure it is possible to effect
substitutions on the basis of that structure. This means we need to
upgrade our notation for substitutions, which we accomplish by
adapting comprehension notation. Thus,

\begin{mathpar}
  P\{ y / x : x \in S \}
\end{mathpar}

is interpreted to mean the process derived from P by replacing (in a
capture-avoiding manner) each occurrence of $x$ in $S$ by $y$. For example,

\begin{mathpar}
  P\{ \quotep{\procn{x}|\procn{x}} / x : x \in \freenames{P} \}
\end{mathpar}

will replace each (occurrence) of a free name $x$ in $P$ by
$\quotep{\procn{x}|\procn{x}}$.

Also, we will avail ourselves of the notation $x^{L}$ and $x^{R}$ to
denote injections of a name into disjoint copies of the name
space. There are numerous ways to accomplish this. One example can be
found in \cite{MeredithR05}. This notation overloads to vectors of
names: $\vec{x}^{\pi} := (x_{i}^{\pi} \; : \; 0 \leq i < |\vec{x}| )$ where $\pi \in \{L,R\}$.

We also use $P^{\Box} := P|\Box$.

In \cite{MeredithR05} an interpretation of the new operator is
given. It turns out that there are several possible interpretations
all enjoying the requisite algebraic properties of the operator (see
\cite{milner91polyadicpi}). We will therefore make liberal use of
$(\nu\; \vec{x})P$.

% subsection the_syntax_and_semantics_of_the_notation_system (end)   

\input{qm2pi.qmops} 

\input{qm2pi.sterngerlach} 

\input{qm2pi.metric} 

% section concurrent_process_calculi (end)

%\input{qm2pi.proofsketch}

% section proof sketch (end)

%\input{qm2pi.slviaknots} 

% section spatial logic via knots (end)

\input{qm2pi.conclusion}

% section conclusion (end)

%\input{qm2pi.dtcodes} 

% section wiring algorithm (end)

\input{qm2pi.ack} 

% section acknowledgments (end)

\newpage


\bibliographystyle{plain}   
\bibliography{../../biblios/main.bib}

\input{qm2pi.rhodetails}

\end{document}

 

% subsection basic_interpretation (end)

%\input{qm2pi.rho.presentation} 
\subsection{The syntax and semantics of the notation system}\label{sub:the_syntax_and_semantics_of_the_notation_system} % (fold)

We now summarize a technical presentation of the calculus that
embodies our theory of dynamics. The typical presentation of such a
calculus follows the style of giving generators and relations on
them. The grammar, below, describing term constructors, freely
generates the set of processes, $\Proc$. This set is then quotiented
by a relation known as structural congruence and it is over this set
that the notion of dynamics is expressed. This presentation is
essentially that of \cite{MeredithR05} with the addition of
polyadicity and summation. For readability we have relegated some of
the technical subtleties to an appendix.

\subsubsection{Process grammar}\label{subsub:process_grammar}

\begin{mathpar}
  \inferrule* [lab=synchronization] {} {{M} \bc \pzero \;|\; x?F \;|\; x!C }
  \and
  \inferrule* [lab=abstraction] {} {{F} \bc (x)P}
  \and
  \inferrule* [lab=concretion] {} {{C} \bc \langle Q \rangle}
  \and
  \inferrule* [lab=process] {} {{P,Q} \bc M \;| \;P|Q \;|\; @{x}}
  \and
  \inferrule* [lab=name] {} {{x} \bc \quotep{P}}
\end{mathpar} 

Note that $\vec{x}$ (resp. $\vec{P}$) denotes a vector of names
(resp. processes) of length $|\vec{x}|$ (resp. $|\vec{P}|$). We adopt
the following useful abbreviations.

\begin{mathpar}
   x?(\vec{y}).P := x.(\vec{y})P \and  x\clift{\vec{P}} := x.\clift{\vec{P}}
   \and x!(y) := \lift{x}{\dropn{y}}
   \and \Pi_{i=0}^{n-1}P_i := P_0 | \ldots | P_{n-1}
\end{mathpar}

\subsubsection{Structural congruence}

\paragraph{Free and bound names and alpha-equivalence.} At the
core of structural equivalence is alpha-equivalence which identifies
process that are the same up to a change of variable. Formally, we
recognize the distinction between free and bound names. The free names
of a process, $\freenames{P}$, may be calculated recursively as
follows:

\begin{mathpar}
\freenames{\pzero} := \emptyset
  \and \\
  \freenames{x?(y).P} := \{ x \} \cup (\freenames{P} \setminus \{ y \})
  \and 
  \freenames{x!\langle P \rangle} := \{ x \} \cup \{ P \} 
  \and \\
  \freenames{P|Q} := \freenames{P} \cup \freenames{Q}
  \and \\
  \freenames{@{x}} := \{ x \}
\end{mathpar}

$\pi$
$\quotep{\pi}$

$\freenames{-} : \pi \to \mathcal{P}(\quotep{\pi})$

\begin{eqnarray*}
  \freenames{\pzero} & := & \emptyset \\
  \freenames{x?(y).P} & := & \{ x \} \cup (\freenames{P} \setminus \{ y \}) \\
  \freenames{x!\langle P \rangle} & := & \{ x \} \cup \{ P \} \\
  \freenames{P|Q} & := & \freenames{P} \cup \freenames{Q} \\
  \freenames{\dropn{x}} & := & \{ x \}
\end{eqnarray*}

The bound names of a process, $\boundnames{P}$, are those names occurring in $P$
that are not free. For example, in $x?(y).0$, the name $x$ is free, while $y$ is bound.

\begin{mathpar}
  \inferrule* [lab=monoidal-laws] {} { P|Q \equiv Q|P \and P|0 \equiv P \and P|(Q|R) \equiv (P|Q)|R }
\end{mathpar}

\begin{mathpar}
  \inferrule* [lab=alpha-equivalence] {} { (x)P \equiv (y)P\{y/x\} \and y \not\in \freenames{P} }
\end{mathpar}

\begin{definition}
Then two processes, $P,Q$, are alpha-equivalent if $P = Q\{\vec{y}/\vec{x}\}$ for
some $\vec{x} \in \boundnames{Q},\vec{y} \in \boundnames{P}$, where $Q\{\vec{y}/\vec{x}\}$
denotes the capture-avoiding substitution of $\vec{y}$ for $\vec{x}$ in $Q$.
\end{definition}

\begin{definition}
  The {\em structural congruence} \cite{SangiorgiWalker} , $\equiv$,
  between processes is the least congruence containing
  alpha-equivalence, satisfying the abelian monoid laws
  (associativity, commutativity and $\pzero$ as identity) for parallel
  composition $|$ and for summation $+$.
\end{definition}

\subsection{Name equivalence}

We take name equivalence, written $\nameeq$, to be the smallest
equivalence relation generated by the following rules.

\begin{mathpar}
\inferrule*[lab=Quote-drop]
{ }
{ \quotep{@{x}} \nameeq x }

\inferrule*[lab=Struct-equiv]
{ P \scong Q }
{ \quotep{P} \nameeq \quotep{Q} }
\end{mathpar}

The astute reader will have noticed that the mutual recursion of names
and processes imposes a mutual recursion on alpha-equivalence and
structural equivalence via name-equivalence. Fortunately, all of this
works out pleasantly and we may calculate in the natural way, free of
concern. The reader interested in the details is referred to the
appendix \ref{appendix:rho_details}.

\subsection{Substitution}

We use $\Proc$ for the set of processes, $\QProc$ for the set of
names, and $\id{\{}\vec{y} / \vec{x} \id{\}}$ to denote partial maps,
$s : \QProc \rightarrow \QProc$. A map, $s$ lifts, uniquely, to a map
on process terms, $\widehat{s} : \Proc \rightarrow \Proc$ by the
following equations.

\begin{mathpar}
  (0) \psubstp{Q}{P} := 0 \\
  (R \juxtap S) \psubstp{Q}{P}
  :=    
  (R)\psubstp{Q}{P} \juxtap (S) \psubstp{Q}{P} \\
  (x?(y).R) \psubstp{Q}{P}    
  :=    
  (x)\substp{Q}{P} (z)\concat( (R \psubstn{z}{y}) \psubstp{Q}{P} ) \\
  (\lift{x}{R}) \psubstp{Q}{P}  
  :=
  \lift{(x)\substp{Q}{P}}{ R \psubstp{Q}{P} } \\
%   (\dropn{x})  \psubstp{Q}{P}       
%   := 
%   \left\{ 
%     \begin{array}{ccc} 
%       \dropn{\quotep{Q}} & & x \nameeq \quotep{P} \\
%       \dropn{x} & & otherwise \\
%     \end{array}
%   \right. 
  (\dropn{x})  \psubstp{Q}{P}       
  := 
  \left\{ 
    \begin{array}{ccc} 
      Q & & x \nameeq \quotep{P} \\
      \dropn{x} & & otherwise \\
    \end{array}
  \right.
\end{mathpar}
 

where

\begin{eqnarray}
  (x)\id{\{} \lpquote Q \rpquote / \lpquote P \rpquote \id{\}}            = 
  \left\{ 
    \begin{array}{ccc}
      \lpquote Q \rpquote & & x \nameeq \lpquote P \rpquote \\
      x & & otherwise \\
    \end{array}
  \right. \nonumber
\end{eqnarray}

and $z$ is chosen distinct from $\quotep{P}$, $\quotep{Q}$, the free
names in $Q$, and all the names in $R$. Our $\alpha$-equivalence will
be built in the standard way from this substitution.

\begin{remark}\label{rem:no_self_referential_names}
  One consequence of these definitions is that $\forall P. \quotep{P}
  \not\in \freenames{P}$.
\end{remark}

\subsection{ Dynamic quote: an example }

Anticipating something of what's to come, consider applying the
substitution, $\widehat{\id{\{}u / z \id{\}}}$, to the following pair
of processes, $\lift{w}{y!(z)}$ and $w[ \lpquote y!(z) \rpquote ]$.

\begin{eqnarray}
	\lift{w}{y!(z)}\widehat{\id{\{}u / z \id{\}}}
		& = &
		\lift{w}{y!(u)} \nonumber\\
	w[ \lpquote y!(z) \rpquote ] \widehat{ \id{\{}u / z \id{\}} }
		& = &
		w[ \lpquote y!(z) \rpquote ] \nonumber
\end{eqnarray}

Because the body of the process between quotes is impervious to
substitution, we get radically different answers. In fact, by
examining the first process in an input context,
e.g. $x?(z).\lift{w}{y!(z)}$, we see that the process under the lift
operator may be shaped by prefixed inputs binding a name inside it. In
this sense, the lift operator will be seen as a way to dynamically
construct processes before reifying them as names.

Finally equipped with these standard features we can present the
dynamics of the calculus.

\subsubsection{Operational semantics} 

Finally, we introduce the computational dynamics. What marks these
algebras as distinct from other more traditionally studied algebraic
structures, e.g. vector spaces or polynomial rings, is the manner in
which dynamics is captured. In traditional structures, dynamics is typically
expressed through morphisms between such structures, as in linear maps
between vector spaces or morphisms between rings. In algebras
associated with the semantics of computation, the dynamics is
expressed as part of the algebraic structure itself, through a
reduction reduction relation typically denoted by $\red$. Below, we
give a recursive presentation of this relation for the calculus used
in the encoding.

$\red \subseteq \pi \times \pi$
$\red : \pi \to \mathcal{P}(\pi)$

\begin{mathpar}
  \inferrule* [lab=Comm] { \textsf{match}( x_{src}, x_{trgt} ) } { x_{trgt}?(y)P \; | \; x_{src}!\langle {Q} \rangle \red P\{\quotep{Q}/y}\} }
  \and \\
  \inferrule* [lab=Par] {{P} \red {P}'} {{{P} | {Q}} \red {{P}' | {Q}}}
  \and
  \inferrule* [lab=Equiv]{{{P} \scong {P}'} \andalso {{P}' \red {Q}'} \andalso {{Q}' \scong {Q}}}{{P} \red {Q}}
\end{mathpar}

\begin{eqnarray*}
  match_{\equiv} (\quotep{P},\quotep{Q}) & := & P \equiv Q \\
  match_{\dagger}(\quotep{P},\quotep{Q}) & := & \forall R. P|Q \red^{*} R => R \red^{*} 0 \\
  match_{K}(\quotep{P},\quotep{Q}) & := & K \mbox{ for some context } K
\end{eqnarray*}

$u?(x)P | u!\langle Q \rangle \red P\{\quotep{Q}/x\}$

%We write $\wred$ for $\red^*$, and $P\red$ if $\exists Q $ such that $ P \red Q$.
We write $P\red$ if $\exists Q $ such that $ P \red Q$ and $P\not\red$, otherwise.

\section{Replication}

As mentioned before, it is known that replication (and hence
recursion) can be implemented in a higher-order process algebra
\cite{SangiorgiWalker}. As our first example of calculation with the
machinery thus far presented we give the construction explicitly in
the {\rhoc}.

\begin{eqnarray}
	D_{x} & := & \prefix{x}{y}{(\binpar{\outputp{x}{y}}{@{y}})} \nonumber\\
	\bangp_{x}{P} & := & \binpar{{x}!\langle{\binpar{D_{x}}{P}}\rangle}{D_{x}} \nonumber
\end{eqnarray}

\begin{eqnarray}
	\bangp_{x}{P} & & \nonumber\\
	=
	& {x}!\langle{(\prefix{x}{y}{(\outputp{x}{y} | @{y})) | P}}\rangle 
	      | \prefix{x}{y}{(\outputp{x}{y} | @{y})} & \nonumber\\
	\red
	& (\outputp{x}{y} | @{y})\substn{\quotep{(\prefix{x}{y}{(@{y} | \outputp{x}{y})) | P}}}{y} & \nonumber\\
	=
	& \outputp{x}{\quotep{(\prefix{x}{y}{(\outputp{x}{y} | @{y})) | P}}}
	  | {(\prefix{x}{y}{(\outputp{x}{y} | @{y})) | P}} & \nonumber\\
	\red
	& \ldots & \nonumber\\
	\red^*
	& P | P | \ldots & \nonumber
\end{eqnarray}

Of course, this encoding, as an implementation, runs away, unfolding
$\bangp{P}$ eagerly. A lazier and more implementable replication
operator, restricted to input-guarded processes, may be obtained as follows.

\begin{eqnarray}
\bangp{\prefix{u}{v}{P}} 
	:= 
	\binpar{\lift{x}{\prefix{u}{v}{(\binpar{D(x)}{P})}}}{D(x)} \nonumber
\end{eqnarray}

\begin{remark}
  Note that the lazier definition still does not deal with summation
  or mixed summation (i.e. sums over input and output). The reader is
  invited to construct definitions of replication that deal with these
  features. 

  Further, the definitions are parameterized in a name, $x$. Can you,
  gentle reader, make a definition that eliminates this parameter and
  guarantees no accidental interaction between the replication
  machinery and the process being replicated -- i.e. no accidental
  sharing of names used by the process to get its work done and the
  name(s) used by the replication to effect copying. This latter
  revision of the definition of replication is crucial to obtaining
  the expected identity $!!P \sim !P$.
\end{remark}

\begin{remark}\label{rem:paradoxical_combinator}
  The reader familiar with the lambda calculus will have noticed the
  similarity between $D$ and the paradoxical combinator.

  [Ed. note: the existence of this seems to suggest we have to be more
  restrictive on the set of processes and names we admit if we are to
  support no-cloning.]
\end{remark}

\subsubsection{Bisimulation}

The computational dynamics gives rise to another kind of equivalence,
the equivalence of computational behavior. As previously mentioned
this is typically captured \emph{via} some form of bisimulation.

% The notion we use in this paper is weak barbed bisimulation
% \cite{milner91polyadicpi}.

The notion we use in this paper is derived from weak barbed
bisimulation \cite{milner91polyadicpi}. 

\begin{definition}
An \emph{observation relation}, $\downarrow_{\mathcal N}$, over a set
of names, $\mathcal N$, is the smallest relation satisfying the rules
below.

\infrule[Out-barb]{y \in {\mathcal N}, \; x \nameeq y}
		  {\outputp{x}{v} \downarrow_{\mathcal N} x}
\infrule[Par-barb]{\mbox{$P\downarrow_{\mathcal N} x$ or $Q\downarrow_{\mathcal N} x$}}
		  {\binpar{P}{Q} \downarrow_{\mathcal N} x}

We write $P \Downarrow_{\mathcal N} x$ if there is $Q$ such that 
$P \wred Q$ and $Q \downarrow_{\mathcal N} x$.
\end{definition}

\begin{definition}
%\label{def.bbisim}
An  ${\mathcal N}$-\emph{barbed bisimulation} over a set of names, ${\mathcal N}$, is a symmetric binary relation 
${\mathcal S}_{\mathcal N}$ between agents such that $P\rel{S}_{\mathcal N}Q$ implies:
\begin{enumerate}
\item If $P \red P'$ then $Q \wred Q'$ and $P'\rel{S}_{\mathcal N} Q'$.
\item If $P\downarrow_{\mathcal N} x$, then $Q\Downarrow_{\mathcal N} x$.
\end{enumerate}
$P$ is ${\mathcal N}$-barbed bisimilar to $Q$, written
$P \wbbisim_{\mathcal N} Q$, if $P \rel{S}_{\mathcal N} Q$ for some ${\mathcal N}$-barbed bisimulation ${\mathcal S}_{\mathcal N}$.
\end{definition}

$\mathcal{R} \subseteq \pi \times \pi$

$P \mathcal{R} Q => \forall P'. P \red P' \Rightarrow \exists Q'. Q \red Q', P' \mathcal{R} Q'$

$P \vdash x \Rightarrow Q \vdash x$

\begin{mathpar}
  \inferrule*[lab=Out-barb]{x \nameeq y}{{y}!\langle{Q}\rangle \vdash x}
  \and
  \inferrule*[lab=Par-barb]{\mbox{$P\vdash x$ or $Q\vdash x$}}{\binpar{P}{Q} \vdash x}
\end{mathpar}

\subsubsection{Contexts}

One of the principle advantages of computational calculi like the
$\pi$-calculus is a well-defined notion of context,
contextual-equivalence and a correlation between
contextual-equivalence and notions of bisimulation. The notion of
context allows the decomposition of a process into (sub-)process and
its syntactic environment, its context. Thus, a context may be
thought of as a process with a ``hole'' (written $\Box$) in it. The
application of a context $M$ to a process $P$, written $M[P]$, is
tantamount to filling the hole in $M$ with $P$. In this paper we do
not need the full weight of this theory, but do make use of the notion
of context in the proof the main theorem. 

\begin{mathpar}
  \inferrule* [lab=summation] {} {{M_{M},M_{N}} \bc \Box \;|\; x.M_{A} \;|\; M_{M}+M_{N}}
  \and
  \inferrule* [lab=agent] {} {{M_{A}} \bc (\vec{x})M_{P} \;| \; \clift{P_0,\ldots,M_{P},\ldots,P_N}}
  \and \\
  \inferrule* [lab=process] {} {{M_{P}} \bc M_{N} \;| \;P|M_{P} }
\end{mathpar} 

\begin{mathpar}
  \inferrule* [lab=sychronization] {} {M_{N} \bc \Box \;|\; x?M_{F} \;|\; x!M_{C}}
  \and
  \inferrule* [lab=abstraction] {} {{M_{F}} \bc (x)M_{P} }
  \and
  \inferrule* [lab=concretion] {} {{M_{C}} \bc \langle M_{P} \rangle }
  \and \\
  \inferrule* [lab=process] {} {{M_{P}} \bc M_{N} \;| \;P|M_{P} }
\end{mathpar}

\begin{definition}[contextual application] Given a context $M$, and
  process $P$, we define the \emph{contextual application}, $M[P] :=
  M\{P/\Box\}$. That is, the contextual application of M to P is the
  substitution of $P$ for $\Box$ in $M$.
\end{definition}

$\meaningof{-} : L \to \mathcal{P}(\pi)$

\begin{mathpar}
  \inferrule* [lab=collection] {} {\meaningof{true} = \pi, \and \meaningof{~E} = \pi \setminus \meaningof{E}, \and \meaningof{E_{1} \& E_{2}} = \meaningof{E_{1}} \cap \meaningof{E_{2}}}
\end{mathpar}

\begin{mathpar}
  \inferrule* [lab=structure] {} {\meaningof{0} = \{ P \in \pi | P \equiv 0 \}, \and \\ \meaningof{E_1 | E_2} = \{ P \in \pi | P \equiv P_{1} | P_{2}, P_{1} \in \meaningof{E_{1}}, P_{2} \in \meaningof{E_2}\} }
\end{mathpar}

\begin{mathpar}
 \inferrule* [lab=behavior] {} {\meaningof{\langle a?b \rangle E} = \{ P \in \pi | P \equiv Q | u?(y)P', \\ \and \\\\ \and \\ \;\;\; u \in \meaningof{a}, \forall z.P'\{z/y\} \in \meaningof{E\{z/b\}}\}, \and \\ \meaningof{a!E} = \{ P \in \pi | P \equiv Q | x!\langle P' \rangle, x \in \meaningof{a} P' \in \meaningof{E}\} }
\end{mathpar}

\begin{mathpar}
 \inferrule* [lab=nominal] {} {\meaningof{\quotep{E}} = \{ \quotep{P} \in \quotep{\pi} | P \in \meaningof{E} \}, \and \meaningof{\quotep{P}} = \{ \quotep{Q} \in \quotep{\pi} | P \equiv Q \} \and \\ \meaningof{@\quotep{E}} = \{ P \in \pi | P \equiv @x, x \in \meaningof{E} \}}
\end{mathpar}

\begin{eqnarray*}
  \\
  \meaningof{-} : TS \to ST
\end{eqnarray*}

\begin{eqnarray*}
  \\
  L : TS \to ST
\end{eqnarray*}

\begin{eqnarray*}
  \\
  P \models E \iff P \in \meaningof{E}
\end{eqnarray*}

\begin{eqnarray*}
  P \approx_{L} Q \iff \forall E \in L. P \models E \iff Q \models E
\end{eqnarray*}

\begin{eqnarray*}
  P \approx_{K} Q
\end{eqnarray*}

\begin{eqnarray*}
  P \approx Q
\end{eqnarray*}

$\approx_{K} = \approx = \approx_{L}$

\subsubsection{Contextual duality}

Note that contexts extend the quotation operation to a family of
operations from processes to names. Given a context, $M$, we can
define a \emph{nominal context}, $\quotep{M}$ by $\quotep{M}[P] :=
\quotep{M[P]}$. To foreshadow what is to come we observe that these
operations enjoy a duality with processes very much like the duality
between vectors and maps from vectors to scalars.

Further, because the calculus is essentially higher-order, we have a
correspondence between contexts and processes. More specifically,
given a name $x$ and a context $M$ we can construct $M^{*}_{x}$ such
that 

\begin{mathpar}
  M^{*}_{x} | \lift{x}{P} \red M[P]
\end{mathpar}

namely,

\begin{mathpar}
  M^{*}_{x} := x?(u).M[\dropn{u}]
\end{mathpar}

The dependence of $M^{*}_{x}$ on a name makes it an abstraction, 

\begin{mathpar}
  M^{*} := (x)x?(u).M[\dropn{u}]
\end{mathpar}

\subsection{Additional notation}

It will sometimes be convenient to denote the process a name
quotes. We already have the notation $x = \quotep{P}$, but it will be
convenient to introduce an alternate notation, $\procn{x}$, when we
want to emphasize the connection to the use of the name. Note that, by
virtue of name equivalence, $\quotep{\procn{x}} \nameeq x$; so, the
notation is consistent with previous definitions.

Further, because names have structure it is possible to effect
substitutions on the basis of that structure. This means we need to
upgrade our notation for substitutions, which we accomplish by
adapting comprehension notation. Thus,

\begin{mathpar}
  P\{ y / x : x \in S \}
\end{mathpar}

is interpreted to mean the process derived from P by replacing (in a
capture-avoiding manner) each occurrence of $x$ in $S$ by $y$. For example,

\begin{mathpar}
  P\{ \quotep{\procn{x}|\procn{x}} / x : x \in \freenames{P} \}
\end{mathpar}

will replace each (occurrence) of a free name $x$ in $P$ by
$\quotep{\procn{x}|\procn{x}}$.

Also, we will avail ourselves of the notation $x^{L}$ and $x^{R}$ to
denote injections of a name into disjoint copies of the name
space. There are numerous ways to accomplish this. One example can be
found in \cite{MeredithR05}. This notation overloads to vectors of
names: $\vec{x}^{\pi} := (x_{i}^{\pi} \; : \; 0 \leq i < |\vec{x}| )$ where $\pi \in \{L,R\}$.

We also use $P^{\Box} := P|\Box$.

In \cite{MeredithR05} an interpretation of the new operator is
given. It turns out that there are several possible interpretations
all enjoying the requisite algebraic properties of the operator (see
\cite{milner91polyadicpi}). We will therefore make liberal use of
$(\nu\; \vec{x})P$.

% subsection the_syntax_and_semantics_of_the_notation_system (end)   

\section{Interpretation of QM}
\subsection{Supporting definitions}
\subsubsection{Multiplication}
\begin{mathpar}
  \quotep{Q} \cdot \quotep{R} := \quotep{Q|R}
  \and \\
  \quotep{Q} \cdot P := P\{ \quotep{Q|R} / \quotep{R} : \quotep{R} \in \freenames{P} \}
\end{mathpar}

\paragraph{Discussion}
The first line needs little explanation. The second line says that
each free name of the process is replaced with the multiplication of
that name by the scalar. Multiplication of a scalar (name) by a state
(process) results in a process all the names of which have been `moved
over' by parallel composition with the process the scalar
quotes. There is a subtlety that the bound names have to be
manipulated so that multiplied names aren't accidentally
captured. There are many ways to achieve this.

\begin{remark}\label{rem:multiplication_identities}
  The reader is invited to verify that for all $x,y,z \in \QProc$ and $P \in \Proc$
  \begin{mathpar}
    x \cdot \quotep{0} \equiv x 
    \and
    x \cdot y \equiv y \cdot x
    \and
    x \cdot (y \cdot z) \equiv (x \cdot y) \cdot z
    \and \\
    \quotep{0} \cdot P \equiv P
    \and \\
    x \cdot (y \cdot P) \equiv (x \cdot y) \cdot P
    \and \\
    x \cdot (P|Q) \equiv (x \cdot P) | (x \cdot Q)
    \and \\    
  \end{mathpar}
\end{remark}

\subsubsection{Tensor product}

We define a tensor product on processes by structural induction.

\paragraph{Tensor of sums} First note that all summations, including
$\pzero$ and sequence, can be written $\Sigma_{i} x_{i}.A_{i} +
\Sigma_{j} x_{j}.C_{j}$, where we have grouped input-guarded processes
together and output-guarded processes together.

Thus, we can define the tensor product of two summations, $N_{1}\otimes N_{2}$, where

\begin{mathpar}
  N_{1} := \Sigma_{i} x_{i}.A_{i} + \Sigma_{j} x_{j}.C_{j}
  \and
  N_{2} := \Sigma_{i'} y_{i'}.B_{i'} + \Sigma_{j'} y_{j'}.D_{j'} 
\end{mathpar}

as follows.

\begin{mathpar}
  \Sigma_{i} x_{i}.A_{i} + \Sigma_{j} x_{j}.C_{j} \otimes \Sigma_{i'}
  y_{i'}.B_{i'} + \Sigma_{j'} y_{j'}.D_{j'} 
  \and \\
  := \; \Sigma_{i} \Sigma_{i'} \quotep{\stackrel{\vee}{x_{i}}| \stackrel{\vee}{y_{i'}}}.(A_{i}\otimes B_{i'}) \; | \; \Sigma_{i'} \Sigma_{i} \quotep{\stackrel{\vee}{y_{i'}}|\stackrel{\vee}{x_{i}}}.(B_{i'}\otimes A_{i})
  \and
  \;\; | \;\; \Sigma_{j} \Sigma_{j'} \quotep{\stackrel{\vee}{x_{j}}|\stackrel{\vee}{y_{j'}}}.(A_{j}\otimes B_{j'}) \; | \; \Sigma_{j'} \Sigma_{j} \quotep{\stackrel{\vee}{y_{j'}}|\stackrel{\vee}{x_{j}}}.(B_{j'}\otimes A_{j})
\end{mathpar}

\begin{remark}
  Do we need to $x^{L}$ and $y^{R}$ for this construction as well?
\end{remark}

\paragraph{Tensor of parallel compositions} Next, we distribute tensor
over par.

\begin{mathpar}
  P_{1}|P_{2} \otimes Q_{1}|Q_{2} := (P_{1} \otimes Q_{1}) | (P_{1}
  \otimes Q_{2}) | (P_{2} \otimes Q_{1}) | (P_{2} \otimes Q_{2})
\end{mathpar}

\paragraph{Tensor with dropped names} We treat tensor of a
process with a dropped name as parallel composition.

\begin{mathpar}
  P \otimes \dropn{x} := P | \dropn{x}
\end{mathpar}

\paragraph{Tensor of agents}

Finally, we need to define tensor on agents. Note that the definition
of tensor on normal products only tensors inputs with inputs and
outputs with outputs. Thus, we only have to define the operation on
``homogeneous'' pairings.

\begin{mathpar}
  (\vec{x})P \otimes (\vec{y})Q
  \and \\
  := (x_{0}^{L}|y_{0}^{R},\ldots,x_{0}^{L}|y_{n}^{R},\ldots,x_{m}^{L}|y_{0}^{R},\ldots,x_{m}^{L}|y_{n}^R)(P\{ \vec{x}^{L}/\vec{x}\} \otimes Q \{ \vec{y}^{R}/\vec{y}\})
  \and \\
  \clift{\vec{P}} \otimes \clift{\vec{Q}}
  \and \\
  := \clift{P_{0}\otimes Q_{0},\ldots,P_{0}\otimes Q_{n},\ldots,P_{m}\otimes Q_{0},\ldots,P_{m}\otimes Q_{n}}
\end{mathpar}

\begin{remark}
  Observe that arities of tensored abstractions matches arities of
  tensored concretions if the original arities matched. Note also that
  the length of the arities corresponds to the increase in dimension
  we see in ordinary vector space tensor product.
\end{remark}

\begin{remark}
  Operationally, this definition distributes the tensor down to
  components ``linked'' by summation. Tensor over summation is
  intriguing in that it mixes names. Moreover, as a consequence of the
  way it mixes names we have the identities for all $x \in \QProc$ and
  $P,Q \in \Proc$

  \begin{mathpar}
    (x \cdot P) \otimes Q \equiv x \cdot (P \otimes Q) \equiv P \otimes (x \cdot Q)
    \and
    P \otimes \pzero \equiv P
  \end{mathpar}

  that the reader is invited to verify.
\end{remark}

\subsubsection{Annihilation}
\begin{mathpar}
  P^{\perp} := \{ Q | \forall R. P|Q \red^{*} R \Rightarrow R \red^{*} \pzero \}
  \and \\
  P^{\underline{\perp}} := \Sigma_{Q \in P^{\perp}} \quotep{Q}?(y).(\dropn{y}|Q) | \Sigma_{Q \in P^{\perp}} \quotep{Q}\clift{\Box}
\end{mathpar}

\paragraph{Discussion} The reader will note that $P^{\perp}$ is a
\emph{set} of processes, while $P^{\underline{\perp}}$ is a
\emph{context}. We call the set $P^{\perp}$ the \emph{annihilators} of
$P$. The parallel composition of a process in the annihilators of $P$
with $P$ will result in a process, the state space of which has all
paths eventually leading to $\pzero$. Execution may endure loops; but
under reasonable conditions of fairness (naturally guaranteed under
most notions of bisimulation) such a composite process cannot get
stuck in such a loop and will, eventually pop out and terminate.

The context $P^{\underline{\perp}}$ is ready and willing to ``take the
$P$ out of'' the process to which it is applied. It will effectively
transmit the code of the process to which it is applied to one of the
annihilators and run the process against it.

\subsubsection{Evaluation}
We fix $M$ a domain of fully abstract interpretation with an equality
coincident with bisimulation. We take $\meaningof{\cdot} : \Proc \to
M$ to be the map interpreting processes and $\nmeaningof{\cdot} : \M
\to Proc$ to be the map running the other way. Then we define

\begin{mathpar}
  \int P := \nmeaningof{\meaningof{P}}
\end{mathpar}

\paragraph{Discussion}
There are many fully abstract interpretations of Milner's
$\pi$-calculus. Any of them can be used as a basis for interpreting
the reflective calculus here. Equipped with such a domain it is
largely a matter of grinding through to check that the Yoneda
construction for the normalization-by-evaluation program can be
extended to this setting.

\begin{remark}
  The reader is invited to verify that $\int (P^{\underline{\perp}}[P]) = 0$.
\end{remark}

\subsection{Quantum mechanics}

Table \ref{tbl:core_qm_op_defns} gives the core operational definitions

\begin{table}[htp]\label{tbl:core_qm_op_defns}
  \center{
    \fbox{
      \begin{tabular}{c|c}
        quantum mechanics & process calculus \\
        \hline
        scalar & $x := \quotep{P}$ \\
        state vector & $\state{P} := P$ \\
        dual & $\state{P}^{*} := \event{P^{\underline{\perp}}} := \quotep{P^{\underline{\perp}}}[-]$ \\
        matrix & $ \Sigma_{\alpha} \state{P_{\alpha}}x_{\alpha}\event{Q_{\alpha}}$ \\
        vector addition & $\state{P} + \state{Q} := \state{P | Q}$ \\
        tensor product & $\state{P} \otimes \state{Q} := \state{P \otimes Q}$ \\
        inner product & $\innerprod{P}{Q} := \quotep{\int P^{\underline{\perp}}[Q]}$ \\
      \end{tabular}
    }
  }
  \caption{QM - operational definitions}
\end{table}

where

\begin{mathpar}
  \prmatrix{P}{Q} := \fprmatrix{P}{\quotep{\pzero}}{Q}
  \and
  \fprmatrix{P}{x}{Q} := (\state{P},x,\event{Q})
  \and
  (\fprmatrix{P}{x}{Q})(\state{R}) := x \cdot \innerprod{Q}{R} \cdot \state{P}
  \and
  (\fprmatrix{P}{x}{Q})(\event{R}) := x \cdot \innerprod{R}{P} \cdot \event{Q}
\end{mathpar}

\paragraph{Discussion}
As promised: vectors (aka states) are represented as processes; duals
as contextual duals; inner product definition should be compared with
standard inner product definition for ....

\begin{remark}
  Assuming $\int (P^{\underline{\perp}}[P]) = 0$, the reader is
  invited to verify that $(\fprmatrix{P}{x}{P})(\state{P}) = x \cdot \state{P}$.
\end{remark}

\begin{remark}
  The reader is invited to verify that $\innerprod{P}{Q}$ could
  equally well have been written $\quotep{\int \stackrel{\vee}{x}}$
  where $x = \event{P^{\underline{\perp}}}(Q)$.

  One of the motivations for this remark is that there is another way
  to factor these operations. We could package up evaluation in the dual:

  \begin{mathpar}
    \state{P}^{*} := \event{\int P^{\underline{\perp}}} := \quotep{\int P^{\underline{\perp}}}[-]
  \end{mathpar}

  and then have inner product defined by
  
  \begin{mathpar}
    \innerprod{P}{Q} := \event{P}(Q)
  \end{mathpar}

  Hopefully, experience with the calculations will provide guidance on
  the best factoring.
\end{remark}

\begin{remark}
  Assuming $\int (P^{\underline{\perp}}[P]) = 0$, the reader is
  invited to verify that $\forall P,Q. (\prmatrix{0}{Q})(\state{0}) =
  \state{0}$ and dually $(\prmatrix{P}{0})(\event{0}) = \event{0}$.
\end{remark}

\begin{remark}
  i'm a little worried that i don't (yet) have proper support for
  complex conjugacy. But, the observation above may give us a
  clue. According to Abramsky, it must be the case that the scalars
  are iso to the homset of the identity for the tensor -- which the
  observation above characterizes. 

  For now, we will simply bookmark the notion with $\overline{x}$.
\end{remark}

\subsubsection{Adjointness}

We need to give a definition of $(\cdot)^{\dagger}$ for matrices. The
obvious candidate definition is
\begin{mathpar}
(\Sigma_{\alpha}\fprmatrix{P_{\alpha}}{x_{\alpha}}{Q_{\alpha}})^{\dagger}
= \Sigma_{\alpha}\fprmatrix{(Q_{\alpha}^{\underline{\perp}})^{*}}{\overline{x}_{\alpha}}{P_{\alpha}^{\underline{\perp}}} 
\end{mathpar}

But, $(Q_{\alpha}^{\underline{\perp}})^{*}$ requires a name along
which to communicate the process to achieve the context application.

\subsubsection{Basis for a basis}
If processes label states and ``addition'' of states (a.k.a. vector
addition) is interpreted as parallel composition, what corresponds to
notions of linear independence and basis? Here, we recall that Yoshida
has developed a set of \emph{combinators} for an asynchronous verison
of Milner's $\pi$-calculus. These are a finite set of processes such
any process can be expressed as parallel composition of these
combinators together with liberal uses of the new operator and
replication. We can simply give a translation of these into the
present calculus and have reasonable expectation that the property
carries over. That is, that the resultant set allows to express all
processes via parallel composition. Note, however, that there is no
new operator or replication in this calculus. As a result, we expect
that the corresponding set is actually infinite. That is, we expect
that the space is actually infinite dimensional.

\begin{remark}
  The attentive reader may be a bit concerned. Certainly, the
  collection $S$, $K$ and $I$ is a finite set of
  combinators. Shouldn't we expect to see a finite set of combinators
  for an effectively equivalent system? i am very sympathetic to this
  critique and feel it warrants full attention. On the other hand, i
  also have in mind the following analogy. The natural numbers, as a
  monoid under addition, has exactly $1$ generator, while the natural
  numbers, as a monoid under multiplication, has countably many
  generators (the primes). We observe that the application of the
  lambda calculus is much less resource sensitive than the parallel
  composition of the $\pi$-calculus. Could it be the case that we have
  an analogy of the form
  
  \begin{mathpar}
    m + n : MN :: m*n : M|N
  \end{mathpar}

  giving a similar blow up in the set of ``primes''?  This is such a
  wonderful thought that, even if it's not true, i think it's worth
  writing down.
\end{remark}
 

\documentclass[12pt]{llncs}
%\documentclass{jktr}

\usepackage[pdftex]{hyperref}                   
\usepackage {listings}
\usepackage {mathpartir}
\usepackage{bcprules}
%\usepackage{listings}
                       
\usepackage{graphicx} 
%\usepackage[margins=2.5cm,nohead,nofoot]{geometry}
%\usepackage{geometry}
\usepackage{amsfonts}
\usepackage{amstext}
\usepackage{latexsym}
\usepackage{amssymb}
\usepackage{color}


%\include{myPreamble}
\include{qm2pi.local} 

%\ifpdf
%\usepackage[pdftex]{graphicx}
%\else
%\usepackage{graphicx}
%\fi

 % \ifpdf
%  \usepackage{pdfsync}
%  \if


%\title{Brief Article}
%\author{David F. Snyder}
%\author{L.G. Meredith}

%\address{Dept. of Math., Texas State University--San Marcos, San Marcos, TX 78666}
       
\pagestyle{empty}


\begin{document}

\lstset{language=[Objective]Caml,frame=shadowbox}

\input{qm2pi.front}

% section front matter (end)

\input{qm2pi.intro} 
 
% section introduction (end)

% \input{qm2pi.knotations} 

% section notation (end)

\input{qm2pi.process.calculi} 

% section concurrent_process_calculi_and_spatial_logics_ (end)
    
%\input{qm2pi.knots2pi} 

%\input{qm2pi.trefoil} 

%\input{qm2pi.mainthm} 

% subsection basic_interpretation (end)

%\input{qm2pi.rho.presentation} 
\subsection{The syntax and semantics of the notation system}\label{sub:the_syntax_and_semantics_of_the_notation_system} % (fold)

We now summarize a technical presentation of the calculus that
embodies our theory of dynamics. The typical presentation of such a
calculus follows the style of giving generators and relations on
them. The grammar, below, describing term constructors, freely
generates the set of processes, $\Proc$. This set is then quotiented
by a relation known as structural congruence and it is over this set
that the notion of dynamics is expressed. This presentation is
essentially that of \cite{MeredithR05} with the addition of
polyadicity and summation. For readability we have relegated some of
the technical subtleties to an appendix.

\subsubsection{Process grammar}\label{subsub:process_grammar}

\begin{mathpar}
  \inferrule* [lab=synchronization] {} {{M} \bc \pzero \;|\; x?F \;|\; x!C }
  \and
  \inferrule* [lab=abstraction] {} {{F} \bc (x)P}
  \and
  \inferrule* [lab=concretion] {} {{C} \bc \langle Q \rangle}
  \and
  \inferrule* [lab=process] {} {{P,Q} \bc M \;| \;P|Q \;|\; @{x}}
  \and
  \inferrule* [lab=name] {} {{x} \bc \quotep{P}}
\end{mathpar} 

Note that $\vec{x}$ (resp. $\vec{P}$) denotes a vector of names
(resp. processes) of length $|\vec{x}|$ (resp. $|\vec{P}|$). We adopt
the following useful abbreviations.

\begin{mathpar}
   x?(\vec{y}).P := x.(\vec{y})P \and  x\clift{\vec{P}} := x.\clift{\vec{P}}
   \and x!(y) := \lift{x}{\dropn{y}}
   \and \Pi_{i=0}^{n-1}P_i := P_0 | \ldots | P_{n-1}
\end{mathpar}

\subsubsection{Structural congruence}

\paragraph{Free and bound names and alpha-equivalence.} At the
core of structural equivalence is alpha-equivalence which identifies
process that are the same up to a change of variable. Formally, we
recognize the distinction between free and bound names. The free names
of a process, $\freenames{P}$, may be calculated recursively as
follows:

\begin{mathpar}
\freenames{\pzero} := \emptyset
  \and \\
  \freenames{x?(y).P} := \{ x \} \cup (\freenames{P} \setminus \{ y \})
  \and 
  \freenames{x!\langle P \rangle} := \{ x \} \cup \{ P \} 
  \and \\
  \freenames{P|Q} := \freenames{P} \cup \freenames{Q}
  \and \\
  \freenames{@{x}} := \{ x \}
\end{mathpar}

$\pi$
$\quotep{\pi}$

$\freenames{-} : \pi \to \mathcal{P}(\quotep{\pi})$

\begin{eqnarray*}
  \freenames{\pzero} & := & \emptyset \\
  \freenames{x?(y).P} & := & \{ x \} \cup (\freenames{P} \setminus \{ y \}) \\
  \freenames{x!\langle P \rangle} & := & \{ x \} \cup \{ P \} \\
  \freenames{P|Q} & := & \freenames{P} \cup \freenames{Q} \\
  \freenames{\dropn{x}} & := & \{ x \}
\end{eqnarray*}

The bound names of a process, $\boundnames{P}$, are those names occurring in $P$
that are not free. For example, in $x?(y).0$, the name $x$ is free, while $y$ is bound.

\begin{mathpar}
  \inferrule* [lab=monoidal-laws] {} { P|Q \equiv Q|P \and P|0 \equiv P \and P|(Q|R) \equiv (P|Q)|R }
\end{mathpar}

\begin{mathpar}
  \inferrule* [lab=alpha-equivalence] {} { (x)P \equiv (y)P\{y/x\} \and y \not\in \freenames{P} }
\end{mathpar}

\begin{definition}
Then two processes, $P,Q$, are alpha-equivalent if $P = Q\{\vec{y}/\vec{x}\}$ for
some $\vec{x} \in \boundnames{Q},\vec{y} \in \boundnames{P}$, where $Q\{\vec{y}/\vec{x}\}$
denotes the capture-avoiding substitution of $\vec{y}$ for $\vec{x}$ in $Q$.
\end{definition}

\begin{definition}
  The {\em structural congruence} \cite{SangiorgiWalker} , $\equiv$,
  between processes is the least congruence containing
  alpha-equivalence, satisfying the abelian monoid laws
  (associativity, commutativity and $\pzero$ as identity) for parallel
  composition $|$ and for summation $+$.
\end{definition}

\subsection{Name equivalence}

We take name equivalence, written $\nameeq$, to be the smallest
equivalence relation generated by the following rules.

\begin{mathpar}
\inferrule*[lab=Quote-drop]
{ }
{ \quotep{@{x}} \nameeq x }

\inferrule*[lab=Struct-equiv]
{ P \scong Q }
{ \quotep{P} \nameeq \quotep{Q} }
\end{mathpar}

The astute reader will have noticed that the mutual recursion of names
and processes imposes a mutual recursion on alpha-equivalence and
structural equivalence via name-equivalence. Fortunately, all of this
works out pleasantly and we may calculate in the natural way, free of
concern. The reader interested in the details is referred to the
appendix \ref{appendix:rho_details}.

\subsection{Substitution}

We use $\Proc$ for the set of processes, $\QProc$ for the set of
names, and $\id{\{}\vec{y} / \vec{x} \id{\}}$ to denote partial maps,
$s : \QProc \rightarrow \QProc$. A map, $s$ lifts, uniquely, to a map
on process terms, $\widehat{s} : \Proc \rightarrow \Proc$ by the
following equations.

\begin{mathpar}
  (0) \psubstp{Q}{P} := 0 \\
  (R \juxtap S) \psubstp{Q}{P}
  :=    
  (R)\psubstp{Q}{P} \juxtap (S) \psubstp{Q}{P} \\
  (x?(y).R) \psubstp{Q}{P}    
  :=    
  (x)\substp{Q}{P} (z)\concat( (R \psubstn{z}{y}) \psubstp{Q}{P} ) \\
  (\lift{x}{R}) \psubstp{Q}{P}  
  :=
  \lift{(x)\substp{Q}{P}}{ R \psubstp{Q}{P} } \\
%   (\dropn{x})  \psubstp{Q}{P}       
%   := 
%   \left\{ 
%     \begin{array}{ccc} 
%       \dropn{\quotep{Q}} & & x \nameeq \quotep{P} \\
%       \dropn{x} & & otherwise \\
%     \end{array}
%   \right. 
  (\dropn{x})  \psubstp{Q}{P}       
  := 
  \left\{ 
    \begin{array}{ccc} 
      Q & & x \nameeq \quotep{P} \\
      \dropn{x} & & otherwise \\
    \end{array}
  \right.
\end{mathpar}
 

where

\begin{eqnarray}
  (x)\id{\{} \lpquote Q \rpquote / \lpquote P \rpquote \id{\}}            = 
  \left\{ 
    \begin{array}{ccc}
      \lpquote Q \rpquote & & x \nameeq \lpquote P \rpquote \\
      x & & otherwise \\
    \end{array}
  \right. \nonumber
\end{eqnarray}

and $z$ is chosen distinct from $\quotep{P}$, $\quotep{Q}$, the free
names in $Q$, and all the names in $R$. Our $\alpha$-equivalence will
be built in the standard way from this substitution.

\begin{remark}\label{rem:no_self_referential_names}
  One consequence of these definitions is that $\forall P. \quotep{P}
  \not\in \freenames{P}$.
\end{remark}

\subsection{ Dynamic quote: an example }

Anticipating something of what's to come, consider applying the
substitution, $\widehat{\id{\{}u / z \id{\}}}$, to the following pair
of processes, $\lift{w}{y!(z)}$ and $w[ \lpquote y!(z) \rpquote ]$.

\begin{eqnarray}
	\lift{w}{y!(z)}\widehat{\id{\{}u / z \id{\}}}
		& = &
		\lift{w}{y!(u)} \nonumber\\
	w[ \lpquote y!(z) \rpquote ] \widehat{ \id{\{}u / z \id{\}} }
		& = &
		w[ \lpquote y!(z) \rpquote ] \nonumber
\end{eqnarray}

Because the body of the process between quotes is impervious to
substitution, we get radically different answers. In fact, by
examining the first process in an input context,
e.g. $x?(z).\lift{w}{y!(z)}$, we see that the process under the lift
operator may be shaped by prefixed inputs binding a name inside it. In
this sense, the lift operator will be seen as a way to dynamically
construct processes before reifying them as names.

Finally equipped with these standard features we can present the
dynamics of the calculus.

\subsubsection{Operational semantics} 

Finally, we introduce the computational dynamics. What marks these
algebras as distinct from other more traditionally studied algebraic
structures, e.g. vector spaces or polynomial rings, is the manner in
which dynamics is captured. In traditional structures, dynamics is typically
expressed through morphisms between such structures, as in linear maps
between vector spaces or morphisms between rings. In algebras
associated with the semantics of computation, the dynamics is
expressed as part of the algebraic structure itself, through a
reduction reduction relation typically denoted by $\red$. Below, we
give a recursive presentation of this relation for the calculus used
in the encoding.

$\red \subseteq \pi \times \pi$
$\red : \pi \to \mathcal{P}(\pi)$

\begin{mathpar}
  \inferrule* [lab=Comm] { \textsf{match}( x_{src}, x_{trgt} ) } { x_{trgt}?(y)P \; | \; x_{src}!\langle {Q} \rangle \red P\{\quotep{Q}/y}\} }
  \and \\
  \inferrule* [lab=Par] {{P} \red {P}'} {{{P} | {Q}} \red {{P}' | {Q}}}
  \and
  \inferrule* [lab=Equiv]{{{P} \scong {P}'} \andalso {{P}' \red {Q}'} \andalso {{Q}' \scong {Q}}}{{P} \red {Q}}
\end{mathpar}

\begin{eqnarray*}
  match_{\equiv} (\quotep{P},\quotep{Q}) & := & P \equiv Q \\
  match_{\dagger}(\quotep{P},\quotep{Q}) & := & \forall R. P|Q \red^{*} R => R \red^{*} 0 \\
  match_{K}(\quotep{P},\quotep{Q}) & := & K \mbox{ for some context } K
\end{eqnarray*}

$u?(x)P | u!\langle Q \rangle \red P\{\quotep{Q}/x\}$

%We write $\wred$ for $\red^*$, and $P\red$ if $\exists Q $ such that $ P \red Q$.
We write $P\red$ if $\exists Q $ such that $ P \red Q$ and $P\not\red$, otherwise.

\section{Replication}

As mentioned before, it is known that replication (and hence
recursion) can be implemented in a higher-order process algebra
\cite{SangiorgiWalker}. As our first example of calculation with the
machinery thus far presented we give the construction explicitly in
the {\rhoc}.

\begin{eqnarray}
	D_{x} & := & \prefix{x}{y}{(\binpar{\outputp{x}{y}}{@{y}})} \nonumber\\
	\bangp_{x}{P} & := & \binpar{{x}!\langle{\binpar{D_{x}}{P}}\rangle}{D_{x}} \nonumber
\end{eqnarray}

\begin{eqnarray}
	\bangp_{x}{P} & & \nonumber\\
	=
	& {x}!\langle{(\prefix{x}{y}{(\outputp{x}{y} | @{y})) | P}}\rangle 
	      | \prefix{x}{y}{(\outputp{x}{y} | @{y})} & \nonumber\\
	\red
	& (\outputp{x}{y} | @{y})\substn{\quotep{(\prefix{x}{y}{(@{y} | \outputp{x}{y})) | P}}}{y} & \nonumber\\
	=
	& \outputp{x}{\quotep{(\prefix{x}{y}{(\outputp{x}{y} | @{y})) | P}}}
	  | {(\prefix{x}{y}{(\outputp{x}{y} | @{y})) | P}} & \nonumber\\
	\red
	& \ldots & \nonumber\\
	\red^*
	& P | P | \ldots & \nonumber
\end{eqnarray}

Of course, this encoding, as an implementation, runs away, unfolding
$\bangp{P}$ eagerly. A lazier and more implementable replication
operator, restricted to input-guarded processes, may be obtained as follows.

\begin{eqnarray}
\bangp{\prefix{u}{v}{P}} 
	:= 
	\binpar{\lift{x}{\prefix{u}{v}{(\binpar{D(x)}{P})}}}{D(x)} \nonumber
\end{eqnarray}

\begin{remark}
  Note that the lazier definition still does not deal with summation
  or mixed summation (i.e. sums over input and output). The reader is
  invited to construct definitions of replication that deal with these
  features. 

  Further, the definitions are parameterized in a name, $x$. Can you,
  gentle reader, make a definition that eliminates this parameter and
  guarantees no accidental interaction between the replication
  machinery and the process being replicated -- i.e. no accidental
  sharing of names used by the process to get its work done and the
  name(s) used by the replication to effect copying. This latter
  revision of the definition of replication is crucial to obtaining
  the expected identity $!!P \sim !P$.
\end{remark}

\begin{remark}\label{rem:paradoxical_combinator}
  The reader familiar with the lambda calculus will have noticed the
  similarity between $D$ and the paradoxical combinator.

  [Ed. note: the existence of this seems to suggest we have to be more
  restrictive on the set of processes and names we admit if we are to
  support no-cloning.]
\end{remark}

\subsubsection{Bisimulation}

The computational dynamics gives rise to another kind of equivalence,
the equivalence of computational behavior. As previously mentioned
this is typically captured \emph{via} some form of bisimulation.

% The notion we use in this paper is weak barbed bisimulation
% \cite{milner91polyadicpi}.

The notion we use in this paper is derived from weak barbed
bisimulation \cite{milner91polyadicpi}. 

\begin{definition}
An \emph{observation relation}, $\downarrow_{\mathcal N}$, over a set
of names, $\mathcal N$, is the smallest relation satisfying the rules
below.

\infrule[Out-barb]{y \in {\mathcal N}, \; x \nameeq y}
		  {\outputp{x}{v} \downarrow_{\mathcal N} x}
\infrule[Par-barb]{\mbox{$P\downarrow_{\mathcal N} x$ or $Q\downarrow_{\mathcal N} x$}}
		  {\binpar{P}{Q} \downarrow_{\mathcal N} x}

We write $P \Downarrow_{\mathcal N} x$ if there is $Q$ such that 
$P \wred Q$ and $Q \downarrow_{\mathcal N} x$.
\end{definition}

\begin{definition}
%\label{def.bbisim}
An  ${\mathcal N}$-\emph{barbed bisimulation} over a set of names, ${\mathcal N}$, is a symmetric binary relation 
${\mathcal S}_{\mathcal N}$ between agents such that $P\rel{S}_{\mathcal N}Q$ implies:
\begin{enumerate}
\item If $P \red P'$ then $Q \wred Q'$ and $P'\rel{S}_{\mathcal N} Q'$.
\item If $P\downarrow_{\mathcal N} x$, then $Q\Downarrow_{\mathcal N} x$.
\end{enumerate}
$P$ is ${\mathcal N}$-barbed bisimilar to $Q$, written
$P \wbbisim_{\mathcal N} Q$, if $P \rel{S}_{\mathcal N} Q$ for some ${\mathcal N}$-barbed bisimulation ${\mathcal S}_{\mathcal N}$.
\end{definition}

$\mathcal{R} \subseteq \pi \times \pi$

$P \mathcal{R} Q => \forall P'. P \red P' \Rightarrow \exists Q'. Q \red Q', P' \mathcal{R} Q'$

$P \vdash x \Rightarrow Q \vdash x$

\begin{mathpar}
  \inferrule*[lab=Out-barb]{x \nameeq y}{{y}!\langle{Q}\rangle \vdash x}
  \and
  \inferrule*[lab=Par-barb]{\mbox{$P\vdash x$ or $Q\vdash x$}}{\binpar{P}{Q} \vdash x}
\end{mathpar}

\subsubsection{Contexts}

One of the principle advantages of computational calculi like the
$\pi$-calculus is a well-defined notion of context,
contextual-equivalence and a correlation between
contextual-equivalence and notions of bisimulation. The notion of
context allows the decomposition of a process into (sub-)process and
its syntactic environment, its context. Thus, a context may be
thought of as a process with a ``hole'' (written $\Box$) in it. The
application of a context $M$ to a process $P$, written $M[P]$, is
tantamount to filling the hole in $M$ with $P$. In this paper we do
not need the full weight of this theory, but do make use of the notion
of context in the proof the main theorem. 

\begin{mathpar}
  \inferrule* [lab=summation] {} {{M_{M},M_{N}} \bc \Box \;|\; x.M_{A} \;|\; M_{M}+M_{N}}
  \and
  \inferrule* [lab=agent] {} {{M_{A}} \bc (\vec{x})M_{P} \;| \; \clift{P_0,\ldots,M_{P},\ldots,P_N}}
  \and \\
  \inferrule* [lab=process] {} {{M_{P}} \bc M_{N} \;| \;P|M_{P} }
\end{mathpar} 

\begin{mathpar}
  \inferrule* [lab=sychronization] {} {M_{N} \bc \Box \;|\; x?M_{F} \;|\; x!M_{C}}
  \and
  \inferrule* [lab=abstraction] {} {{M_{F}} \bc (x)M_{P} }
  \and
  \inferrule* [lab=concretion] {} {{M_{C}} \bc \langle M_{P} \rangle }
  \and \\
  \inferrule* [lab=process] {} {{M_{P}} \bc M_{N} \;| \;P|M_{P} }
\end{mathpar}

\begin{definition}[contextual application] Given a context $M$, and
  process $P$, we define the \emph{contextual application}, $M[P] :=
  M\{P/\Box\}$. That is, the contextual application of M to P is the
  substitution of $P$ for $\Box$ in $M$.
\end{definition}

$\meaningof{-} : L \to \mathcal{P}(\pi)$

\begin{mathpar}
  \inferrule* [lab=collection] {} {\meaningof{true} = \pi, \and \meaningof{~E} = \pi \setminus \meaningof{E}, \and \meaningof{E_{1} \& E_{2}} = \meaningof{E_{1}} \cap \meaningof{E_{2}}}
\end{mathpar}

\begin{mathpar}
  \inferrule* [lab=structure] {} {\meaningof{0} = \{ P \in \pi | P \equiv 0 \}, \and \\ \meaningof{E_1 | E_2} = \{ P \in \pi | P \equiv P_{1} | P_{2}, P_{1} \in \meaningof{E_{1}}, P_{2} \in \meaningof{E_2}\} }
\end{mathpar}

\begin{mathpar}
 \inferrule* [lab=behavior] {} {\meaningof{\langle a?b \rangle E} = \{ P \in \pi | P \equiv Q | u?(y)P', \\ \and \\\\ \and \\ \;\;\; u \in \meaningof{a}, \forall z.P'\{z/y\} \in \meaningof{E\{z/b\}}\}, \and \\ \meaningof{a!E} = \{ P \in \pi | P \equiv Q | x!\langle P' \rangle, x \in \meaningof{a} P' \in \meaningof{E}\} }
\end{mathpar}

\begin{mathpar}
 \inferrule* [lab=nominal] {} {\meaningof{\quotep{E}} = \{ \quotep{P} \in \quotep{\pi} | P \in \meaningof{E} \}, \and \meaningof{\quotep{P}} = \{ \quotep{Q} \in \quotep{\pi} | P \equiv Q \} \and \\ \meaningof{@\quotep{E}} = \{ P \in \pi | P \equiv @x, x \in \meaningof{E} \}}
\end{mathpar}

\begin{eqnarray*}
  \\
  \meaningof{-} : TS \to ST
\end{eqnarray*}

\begin{eqnarray*}
  \\
  L : TS \to ST
\end{eqnarray*}

\begin{eqnarray*}
  \\
  P \models E \iff P \in \meaningof{E}
\end{eqnarray*}

\begin{eqnarray*}
  P \approx_{L} Q \iff \forall E \in L. P \models E \iff Q \models E
\end{eqnarray*}

\begin{eqnarray*}
  P \approx_{K} Q
\end{eqnarray*}

\begin{eqnarray*}
  P \approx Q
\end{eqnarray*}

$\approx_{K} = \approx = \approx_{L}$

\subsubsection{Contextual duality}

Note that contexts extend the quotation operation to a family of
operations from processes to names. Given a context, $M$, we can
define a \emph{nominal context}, $\quotep{M}$ by $\quotep{M}[P] :=
\quotep{M[P]}$. To foreshadow what is to come we observe that these
operations enjoy a duality with processes very much like the duality
between vectors and maps from vectors to scalars.

Further, because the calculus is essentially higher-order, we have a
correspondence between contexts and processes. More specifically,
given a name $x$ and a context $M$ we can construct $M^{*}_{x}$ such
that 

\begin{mathpar}
  M^{*}_{x} | \lift{x}{P} \red M[P]
\end{mathpar}

namely,

\begin{mathpar}
  M^{*}_{x} := x?(u).M[\dropn{u}]
\end{mathpar}

The dependence of $M^{*}_{x}$ on a name makes it an abstraction, 

\begin{mathpar}
  M^{*} := (x)x?(u).M[\dropn{u}]
\end{mathpar}

\subsection{Additional notation}

It will sometimes be convenient to denote the process a name
quotes. We already have the notation $x = \quotep{P}$, but it will be
convenient to introduce an alternate notation, $\procn{x}$, when we
want to emphasize the connection to the use of the name. Note that, by
virtue of name equivalence, $\quotep{\procn{x}} \nameeq x$; so, the
notation is consistent with previous definitions.

Further, because names have structure it is possible to effect
substitutions on the basis of that structure. This means we need to
upgrade our notation for substitutions, which we accomplish by
adapting comprehension notation. Thus,

\begin{mathpar}
  P\{ y / x : x \in S \}
\end{mathpar}

is interpreted to mean the process derived from P by replacing (in a
capture-avoiding manner) each occurrence of $x$ in $S$ by $y$. For example,

\begin{mathpar}
  P\{ \quotep{\procn{x}|\procn{x}} / x : x \in \freenames{P} \}
\end{mathpar}

will replace each (occurrence) of a free name $x$ in $P$ by
$\quotep{\procn{x}|\procn{x}}$.

Also, we will avail ourselves of the notation $x^{L}$ and $x^{R}$ to
denote injections of a name into disjoint copies of the name
space. There are numerous ways to accomplish this. One example can be
found in \cite{MeredithR05}. This notation overloads to vectors of
names: $\vec{x}^{\pi} := (x_{i}^{\pi} \; : \; 0 \leq i < |\vec{x}| )$ where $\pi \in \{L,R\}$.

We also use $P^{\Box} := P|\Box$.

In \cite{MeredithR05} an interpretation of the new operator is
given. It turns out that there are several possible interpretations
all enjoying the requisite algebraic properties of the operator (see
\cite{milner91polyadicpi}). We will therefore make liberal use of
$(\nu\; \vec{x})P$.

% subsection the_syntax_and_semantics_of_the_notation_system (end)   

\input{qm2pi.qmops} 

\input{qm2pi.sterngerlach} 

\input{qm2pi.metric} 

% section concurrent_process_calculi (end)

%\input{qm2pi.proofsketch}

% section proof sketch (end)

%\input{qm2pi.slviaknots} 

% section spatial logic via knots (end)

\input{qm2pi.conclusion}

% section conclusion (end)

%\input{qm2pi.dtcodes} 

% section wiring algorithm (end)

\input{qm2pi.ack} 

% section acknowledgments (end)

\newpage


\bibliographystyle{plain}   
\bibliography{../../biblios/main.bib}

\input{qm2pi.rhodetails}

\end{document}

 

\documentclass[12pt]{llncs}
%\documentclass{jktr}

\usepackage[pdftex]{hyperref}                   
\usepackage {listings}
\usepackage {mathpartir}
\usepackage{bcprules}
%\usepackage{listings}
                       
\usepackage{graphicx} 
%\usepackage[margins=2.5cm,nohead,nofoot]{geometry}
%\usepackage{geometry}
\usepackage{amsfonts}
\usepackage{amstext}
\usepackage{latexsym}
\usepackage{amssymb}
\usepackage{color}


%\include{myPreamble}
\include{qm2pi.local} 

%\ifpdf
%\usepackage[pdftex]{graphicx}
%\else
%\usepackage{graphicx}
%\fi

 % \ifpdf
%  \usepackage{pdfsync}
%  \if


%\title{Brief Article}
%\author{David F. Snyder}
%\author{L.G. Meredith}

%\address{Dept. of Math., Texas State University--San Marcos, San Marcos, TX 78666}
       
\pagestyle{empty}


\begin{document}

\lstset{language=[Objective]Caml,frame=shadowbox}

\input{qm2pi.front}

% section front matter (end)

\input{qm2pi.intro} 
 
% section introduction (end)

% \input{qm2pi.knotations} 

% section notation (end)

\input{qm2pi.process.calculi} 

% section concurrent_process_calculi_and_spatial_logics_ (end)
    
%\input{qm2pi.knots2pi} 

%\input{qm2pi.trefoil} 

%\input{qm2pi.mainthm} 

% subsection basic_interpretation (end)

%\input{qm2pi.rho.presentation} 
\subsection{The syntax and semantics of the notation system}\label{sub:the_syntax_and_semantics_of_the_notation_system} % (fold)

We now summarize a technical presentation of the calculus that
embodies our theory of dynamics. The typical presentation of such a
calculus follows the style of giving generators and relations on
them. The grammar, below, describing term constructors, freely
generates the set of processes, $\Proc$. This set is then quotiented
by a relation known as structural congruence and it is over this set
that the notion of dynamics is expressed. This presentation is
essentially that of \cite{MeredithR05} with the addition of
polyadicity and summation. For readability we have relegated some of
the technical subtleties to an appendix.

\subsubsection{Process grammar}\label{subsub:process_grammar}

\begin{mathpar}
  \inferrule* [lab=synchronization] {} {{M} \bc \pzero \;|\; x?F \;|\; x!C }
  \and
  \inferrule* [lab=abstraction] {} {{F} \bc (x)P}
  \and
  \inferrule* [lab=concretion] {} {{C} \bc \langle Q \rangle}
  \and
  \inferrule* [lab=process] {} {{P,Q} \bc M \;| \;P|Q \;|\; @{x}}
  \and
  \inferrule* [lab=name] {} {{x} \bc \quotep{P}}
\end{mathpar} 

Note that $\vec{x}$ (resp. $\vec{P}$) denotes a vector of names
(resp. processes) of length $|\vec{x}|$ (resp. $|\vec{P}|$). We adopt
the following useful abbreviations.

\begin{mathpar}
   x?(\vec{y}).P := x.(\vec{y})P \and  x\clift{\vec{P}} := x.\clift{\vec{P}}
   \and x!(y) := \lift{x}{\dropn{y}}
   \and \Pi_{i=0}^{n-1}P_i := P_0 | \ldots | P_{n-1}
\end{mathpar}

\subsubsection{Structural congruence}

\paragraph{Free and bound names and alpha-equivalence.} At the
core of structural equivalence is alpha-equivalence which identifies
process that are the same up to a change of variable. Formally, we
recognize the distinction between free and bound names. The free names
of a process, $\freenames{P}$, may be calculated recursively as
follows:

\begin{mathpar}
\freenames{\pzero} := \emptyset
  \and \\
  \freenames{x?(y).P} := \{ x \} \cup (\freenames{P} \setminus \{ y \})
  \and 
  \freenames{x!\langle P \rangle} := \{ x \} \cup \{ P \} 
  \and \\
  \freenames{P|Q} := \freenames{P} \cup \freenames{Q}
  \and \\
  \freenames{@{x}} := \{ x \}
\end{mathpar}

$\pi$
$\quotep{\pi}$

$\freenames{-} : \pi \to \mathcal{P}(\quotep{\pi})$

\begin{eqnarray*}
  \freenames{\pzero} & := & \emptyset \\
  \freenames{x?(y).P} & := & \{ x \} \cup (\freenames{P} \setminus \{ y \}) \\
  \freenames{x!\langle P \rangle} & := & \{ x \} \cup \{ P \} \\
  \freenames{P|Q} & := & \freenames{P} \cup \freenames{Q} \\
  \freenames{\dropn{x}} & := & \{ x \}
\end{eqnarray*}

The bound names of a process, $\boundnames{P}$, are those names occurring in $P$
that are not free. For example, in $x?(y).0$, the name $x$ is free, while $y$ is bound.

\begin{mathpar}
  \inferrule* [lab=monoidal-laws] {} { P|Q \equiv Q|P \and P|0 \equiv P \and P|(Q|R) \equiv (P|Q)|R }
\end{mathpar}

\begin{mathpar}
  \inferrule* [lab=alpha-equivalence] {} { (x)P \equiv (y)P\{y/x\} \and y \not\in \freenames{P} }
\end{mathpar}

\begin{definition}
Then two processes, $P,Q$, are alpha-equivalent if $P = Q\{\vec{y}/\vec{x}\}$ for
some $\vec{x} \in \boundnames{Q},\vec{y} \in \boundnames{P}$, where $Q\{\vec{y}/\vec{x}\}$
denotes the capture-avoiding substitution of $\vec{y}$ for $\vec{x}$ in $Q$.
\end{definition}

\begin{definition}
  The {\em structural congruence} \cite{SangiorgiWalker} , $\equiv$,
  between processes is the least congruence containing
  alpha-equivalence, satisfying the abelian monoid laws
  (associativity, commutativity and $\pzero$ as identity) for parallel
  composition $|$ and for summation $+$.
\end{definition}

\subsection{Name equivalence}

We take name equivalence, written $\nameeq$, to be the smallest
equivalence relation generated by the following rules.

\begin{mathpar}
\inferrule*[lab=Quote-drop]
{ }
{ \quotep{@{x}} \nameeq x }

\inferrule*[lab=Struct-equiv]
{ P \scong Q }
{ \quotep{P} \nameeq \quotep{Q} }
\end{mathpar}

The astute reader will have noticed that the mutual recursion of names
and processes imposes a mutual recursion on alpha-equivalence and
structural equivalence via name-equivalence. Fortunately, all of this
works out pleasantly and we may calculate in the natural way, free of
concern. The reader interested in the details is referred to the
appendix \ref{appendix:rho_details}.

\subsection{Substitution}

We use $\Proc$ for the set of processes, $\QProc$ for the set of
names, and $\id{\{}\vec{y} / \vec{x} \id{\}}$ to denote partial maps,
$s : \QProc \rightarrow \QProc$. A map, $s$ lifts, uniquely, to a map
on process terms, $\widehat{s} : \Proc \rightarrow \Proc$ by the
following equations.

\begin{mathpar}
  (0) \psubstp{Q}{P} := 0 \\
  (R \juxtap S) \psubstp{Q}{P}
  :=    
  (R)\psubstp{Q}{P} \juxtap (S) \psubstp{Q}{P} \\
  (x?(y).R) \psubstp{Q}{P}    
  :=    
  (x)\substp{Q}{P} (z)\concat( (R \psubstn{z}{y}) \psubstp{Q}{P} ) \\
  (\lift{x}{R}) \psubstp{Q}{P}  
  :=
  \lift{(x)\substp{Q}{P}}{ R \psubstp{Q}{P} } \\
%   (\dropn{x})  \psubstp{Q}{P}       
%   := 
%   \left\{ 
%     \begin{array}{ccc} 
%       \dropn{\quotep{Q}} & & x \nameeq \quotep{P} \\
%       \dropn{x} & & otherwise \\
%     \end{array}
%   \right. 
  (\dropn{x})  \psubstp{Q}{P}       
  := 
  \left\{ 
    \begin{array}{ccc} 
      Q & & x \nameeq \quotep{P} \\
      \dropn{x} & & otherwise \\
    \end{array}
  \right.
\end{mathpar}
 

where

\begin{eqnarray}
  (x)\id{\{} \lpquote Q \rpquote / \lpquote P \rpquote \id{\}}            = 
  \left\{ 
    \begin{array}{ccc}
      \lpquote Q \rpquote & & x \nameeq \lpquote P \rpquote \\
      x & & otherwise \\
    \end{array}
  \right. \nonumber
\end{eqnarray}

and $z$ is chosen distinct from $\quotep{P}$, $\quotep{Q}$, the free
names in $Q$, and all the names in $R$. Our $\alpha$-equivalence will
be built in the standard way from this substitution.

\begin{remark}\label{rem:no_self_referential_names}
  One consequence of these definitions is that $\forall P. \quotep{P}
  \not\in \freenames{P}$.
\end{remark}

\subsection{ Dynamic quote: an example }

Anticipating something of what's to come, consider applying the
substitution, $\widehat{\id{\{}u / z \id{\}}}$, to the following pair
of processes, $\lift{w}{y!(z)}$ and $w[ \lpquote y!(z) \rpquote ]$.

\begin{eqnarray}
	\lift{w}{y!(z)}\widehat{\id{\{}u / z \id{\}}}
		& = &
		\lift{w}{y!(u)} \nonumber\\
	w[ \lpquote y!(z) \rpquote ] \widehat{ \id{\{}u / z \id{\}} }
		& = &
		w[ \lpquote y!(z) \rpquote ] \nonumber
\end{eqnarray}

Because the body of the process between quotes is impervious to
substitution, we get radically different answers. In fact, by
examining the first process in an input context,
e.g. $x?(z).\lift{w}{y!(z)}$, we see that the process under the lift
operator may be shaped by prefixed inputs binding a name inside it. In
this sense, the lift operator will be seen as a way to dynamically
construct processes before reifying them as names.

Finally equipped with these standard features we can present the
dynamics of the calculus.

\subsubsection{Operational semantics} 

Finally, we introduce the computational dynamics. What marks these
algebras as distinct from other more traditionally studied algebraic
structures, e.g. vector spaces or polynomial rings, is the manner in
which dynamics is captured. In traditional structures, dynamics is typically
expressed through morphisms between such structures, as in linear maps
between vector spaces or morphisms between rings. In algebras
associated with the semantics of computation, the dynamics is
expressed as part of the algebraic structure itself, through a
reduction reduction relation typically denoted by $\red$. Below, we
give a recursive presentation of this relation for the calculus used
in the encoding.

$\red \subseteq \pi \times \pi$
$\red : \pi \to \mathcal{P}(\pi)$

\begin{mathpar}
  \inferrule* [lab=Comm] { \textsf{match}( x_{src}, x_{trgt} ) } { x_{trgt}?(y)P \; | \; x_{src}!\langle {Q} \rangle \red P\{\quotep{Q}/y}\} }
  \and \\
  \inferrule* [lab=Par] {{P} \red {P}'} {{{P} | {Q}} \red {{P}' | {Q}}}
  \and
  \inferrule* [lab=Equiv]{{{P} \scong {P}'} \andalso {{P}' \red {Q}'} \andalso {{Q}' \scong {Q}}}{{P} \red {Q}}
\end{mathpar}

\begin{eqnarray*}
  match_{\equiv} (\quotep{P},\quotep{Q}) & := & P \equiv Q \\
  match_{\dagger}(\quotep{P},\quotep{Q}) & := & \forall R. P|Q \red^{*} R => R \red^{*} 0 \\
  match_{K}(\quotep{P},\quotep{Q}) & := & K \mbox{ for some context } K
\end{eqnarray*}

$u?(x)P | u!\langle Q \rangle \red P\{\quotep{Q}/x\}$

%We write $\wred$ for $\red^*$, and $P\red$ if $\exists Q $ such that $ P \red Q$.
We write $P\red$ if $\exists Q $ such that $ P \red Q$ and $P\not\red$, otherwise.

\section{Replication}

As mentioned before, it is known that replication (and hence
recursion) can be implemented in a higher-order process algebra
\cite{SangiorgiWalker}. As our first example of calculation with the
machinery thus far presented we give the construction explicitly in
the {\rhoc}.

\begin{eqnarray}
	D_{x} & := & \prefix{x}{y}{(\binpar{\outputp{x}{y}}{@{y}})} \nonumber\\
	\bangp_{x}{P} & := & \binpar{{x}!\langle{\binpar{D_{x}}{P}}\rangle}{D_{x}} \nonumber
\end{eqnarray}

\begin{eqnarray}
	\bangp_{x}{P} & & \nonumber\\
	=
	& {x}!\langle{(\prefix{x}{y}{(\outputp{x}{y} | @{y})) | P}}\rangle 
	      | \prefix{x}{y}{(\outputp{x}{y} | @{y})} & \nonumber\\
	\red
	& (\outputp{x}{y} | @{y})\substn{\quotep{(\prefix{x}{y}{(@{y} | \outputp{x}{y})) | P}}}{y} & \nonumber\\
	=
	& \outputp{x}{\quotep{(\prefix{x}{y}{(\outputp{x}{y} | @{y})) | P}}}
	  | {(\prefix{x}{y}{(\outputp{x}{y} | @{y})) | P}} & \nonumber\\
	\red
	& \ldots & \nonumber\\
	\red^*
	& P | P | \ldots & \nonumber
\end{eqnarray}

Of course, this encoding, as an implementation, runs away, unfolding
$\bangp{P}$ eagerly. A lazier and more implementable replication
operator, restricted to input-guarded processes, may be obtained as follows.

\begin{eqnarray}
\bangp{\prefix{u}{v}{P}} 
	:= 
	\binpar{\lift{x}{\prefix{u}{v}{(\binpar{D(x)}{P})}}}{D(x)} \nonumber
\end{eqnarray}

\begin{remark}
  Note that the lazier definition still does not deal with summation
  or mixed summation (i.e. sums over input and output). The reader is
  invited to construct definitions of replication that deal with these
  features. 

  Further, the definitions are parameterized in a name, $x$. Can you,
  gentle reader, make a definition that eliminates this parameter and
  guarantees no accidental interaction between the replication
  machinery and the process being replicated -- i.e. no accidental
  sharing of names used by the process to get its work done and the
  name(s) used by the replication to effect copying. This latter
  revision of the definition of replication is crucial to obtaining
  the expected identity $!!P \sim !P$.
\end{remark}

\begin{remark}\label{rem:paradoxical_combinator}
  The reader familiar with the lambda calculus will have noticed the
  similarity between $D$ and the paradoxical combinator.

  [Ed. note: the existence of this seems to suggest we have to be more
  restrictive on the set of processes and names we admit if we are to
  support no-cloning.]
\end{remark}

\subsubsection{Bisimulation}

The computational dynamics gives rise to another kind of equivalence,
the equivalence of computational behavior. As previously mentioned
this is typically captured \emph{via} some form of bisimulation.

% The notion we use in this paper is weak barbed bisimulation
% \cite{milner91polyadicpi}.

The notion we use in this paper is derived from weak barbed
bisimulation \cite{milner91polyadicpi}. 

\begin{definition}
An \emph{observation relation}, $\downarrow_{\mathcal N}$, over a set
of names, $\mathcal N$, is the smallest relation satisfying the rules
below.

\infrule[Out-barb]{y \in {\mathcal N}, \; x \nameeq y}
		  {\outputp{x}{v} \downarrow_{\mathcal N} x}
\infrule[Par-barb]{\mbox{$P\downarrow_{\mathcal N} x$ or $Q\downarrow_{\mathcal N} x$}}
		  {\binpar{P}{Q} \downarrow_{\mathcal N} x}

We write $P \Downarrow_{\mathcal N} x$ if there is $Q$ such that 
$P \wred Q$ and $Q \downarrow_{\mathcal N} x$.
\end{definition}

\begin{definition}
%\label{def.bbisim}
An  ${\mathcal N}$-\emph{barbed bisimulation} over a set of names, ${\mathcal N}$, is a symmetric binary relation 
${\mathcal S}_{\mathcal N}$ between agents such that $P\rel{S}_{\mathcal N}Q$ implies:
\begin{enumerate}
\item If $P \red P'$ then $Q \wred Q'$ and $P'\rel{S}_{\mathcal N} Q'$.
\item If $P\downarrow_{\mathcal N} x$, then $Q\Downarrow_{\mathcal N} x$.
\end{enumerate}
$P$ is ${\mathcal N}$-barbed bisimilar to $Q$, written
$P \wbbisim_{\mathcal N} Q$, if $P \rel{S}_{\mathcal N} Q$ for some ${\mathcal N}$-barbed bisimulation ${\mathcal S}_{\mathcal N}$.
\end{definition}

$\mathcal{R} \subseteq \pi \times \pi$

$P \mathcal{R} Q => \forall P'. P \red P' \Rightarrow \exists Q'. Q \red Q', P' \mathcal{R} Q'$

$P \vdash x \Rightarrow Q \vdash x$

\begin{mathpar}
  \inferrule*[lab=Out-barb]{x \nameeq y}{{y}!\langle{Q}\rangle \vdash x}
  \and
  \inferrule*[lab=Par-barb]{\mbox{$P\vdash x$ or $Q\vdash x$}}{\binpar{P}{Q} \vdash x}
\end{mathpar}

\subsubsection{Contexts}

One of the principle advantages of computational calculi like the
$\pi$-calculus is a well-defined notion of context,
contextual-equivalence and a correlation between
contextual-equivalence and notions of bisimulation. The notion of
context allows the decomposition of a process into (sub-)process and
its syntactic environment, its context. Thus, a context may be
thought of as a process with a ``hole'' (written $\Box$) in it. The
application of a context $M$ to a process $P$, written $M[P]$, is
tantamount to filling the hole in $M$ with $P$. In this paper we do
not need the full weight of this theory, but do make use of the notion
of context in the proof the main theorem. 

\begin{mathpar}
  \inferrule* [lab=summation] {} {{M_{M},M_{N}} \bc \Box \;|\; x.M_{A} \;|\; M_{M}+M_{N}}
  \and
  \inferrule* [lab=agent] {} {{M_{A}} \bc (\vec{x})M_{P} \;| \; \clift{P_0,\ldots,M_{P},\ldots,P_N}}
  \and \\
  \inferrule* [lab=process] {} {{M_{P}} \bc M_{N} \;| \;P|M_{P} }
\end{mathpar} 

\begin{mathpar}
  \inferrule* [lab=sychronization] {} {M_{N} \bc \Box \;|\; x?M_{F} \;|\; x!M_{C}}
  \and
  \inferrule* [lab=abstraction] {} {{M_{F}} \bc (x)M_{P} }
  \and
  \inferrule* [lab=concretion] {} {{M_{C}} \bc \langle M_{P} \rangle }
  \and \\
  \inferrule* [lab=process] {} {{M_{P}} \bc M_{N} \;| \;P|M_{P} }
\end{mathpar}

\begin{definition}[contextual application] Given a context $M$, and
  process $P$, we define the \emph{contextual application}, $M[P] :=
  M\{P/\Box\}$. That is, the contextual application of M to P is the
  substitution of $P$ for $\Box$ in $M$.
\end{definition}

$\meaningof{-} : L \to \mathcal{P}(\pi)$

\begin{mathpar}
  \inferrule* [lab=collection] {} {\meaningof{true} = \pi, \and \meaningof{~E} = \pi \setminus \meaningof{E}, \and \meaningof{E_{1} \& E_{2}} = \meaningof{E_{1}} \cap \meaningof{E_{2}}}
\end{mathpar}

\begin{mathpar}
  \inferrule* [lab=structure] {} {\meaningof{0} = \{ P \in \pi | P \equiv 0 \}, \and \\ \meaningof{E_1 | E_2} = \{ P \in \pi | P \equiv P_{1} | P_{2}, P_{1} \in \meaningof{E_{1}}, P_{2} \in \meaningof{E_2}\} }
\end{mathpar}

\begin{mathpar}
 \inferrule* [lab=behavior] {} {\meaningof{\langle a?b \rangle E} = \{ P \in \pi | P \equiv Q | u?(y)P', \\ \and \\\\ \and \\ \;\;\; u \in \meaningof{a}, \forall z.P'\{z/y\} \in \meaningof{E\{z/b\}}\}, \and \\ \meaningof{a!E} = \{ P \in \pi | P \equiv Q | x!\langle P' \rangle, x \in \meaningof{a} P' \in \meaningof{E}\} }
\end{mathpar}

\begin{mathpar}
 \inferrule* [lab=nominal] {} {\meaningof{\quotep{E}} = \{ \quotep{P} \in \quotep{\pi} | P \in \meaningof{E} \}, \and \meaningof{\quotep{P}} = \{ \quotep{Q} \in \quotep{\pi} | P \equiv Q \} \and \\ \meaningof{@\quotep{E}} = \{ P \in \pi | P \equiv @x, x \in \meaningof{E} \}}
\end{mathpar}

\begin{eqnarray*}
  \\
  \meaningof{-} : TS \to ST
\end{eqnarray*}

\begin{eqnarray*}
  \\
  L : TS \to ST
\end{eqnarray*}

\begin{eqnarray*}
  \\
  P \models E \iff P \in \meaningof{E}
\end{eqnarray*}

\begin{eqnarray*}
  P \approx_{L} Q \iff \forall E \in L. P \models E \iff Q \models E
\end{eqnarray*}

\begin{eqnarray*}
  P \approx_{K} Q
\end{eqnarray*}

\begin{eqnarray*}
  P \approx Q
\end{eqnarray*}

$\approx_{K} = \approx = \approx_{L}$

\subsubsection{Contextual duality}

Note that contexts extend the quotation operation to a family of
operations from processes to names. Given a context, $M$, we can
define a \emph{nominal context}, $\quotep{M}$ by $\quotep{M}[P] :=
\quotep{M[P]}$. To foreshadow what is to come we observe that these
operations enjoy a duality with processes very much like the duality
between vectors and maps from vectors to scalars.

Further, because the calculus is essentially higher-order, we have a
correspondence between contexts and processes. More specifically,
given a name $x$ and a context $M$ we can construct $M^{*}_{x}$ such
that 

\begin{mathpar}
  M^{*}_{x} | \lift{x}{P} \red M[P]
\end{mathpar}

namely,

\begin{mathpar}
  M^{*}_{x} := x?(u).M[\dropn{u}]
\end{mathpar}

The dependence of $M^{*}_{x}$ on a name makes it an abstraction, 

\begin{mathpar}
  M^{*} := (x)x?(u).M[\dropn{u}]
\end{mathpar}

\subsection{Additional notation}

It will sometimes be convenient to denote the process a name
quotes. We already have the notation $x = \quotep{P}$, but it will be
convenient to introduce an alternate notation, $\procn{x}$, when we
want to emphasize the connection to the use of the name. Note that, by
virtue of name equivalence, $\quotep{\procn{x}} \nameeq x$; so, the
notation is consistent with previous definitions.

Further, because names have structure it is possible to effect
substitutions on the basis of that structure. This means we need to
upgrade our notation for substitutions, which we accomplish by
adapting comprehension notation. Thus,

\begin{mathpar}
  P\{ y / x : x \in S \}
\end{mathpar}

is interpreted to mean the process derived from P by replacing (in a
capture-avoiding manner) each occurrence of $x$ in $S$ by $y$. For example,

\begin{mathpar}
  P\{ \quotep{\procn{x}|\procn{x}} / x : x \in \freenames{P} \}
\end{mathpar}

will replace each (occurrence) of a free name $x$ in $P$ by
$\quotep{\procn{x}|\procn{x}}$.

Also, we will avail ourselves of the notation $x^{L}$ and $x^{R}$ to
denote injections of a name into disjoint copies of the name
space. There are numerous ways to accomplish this. One example can be
found in \cite{MeredithR05}. This notation overloads to vectors of
names: $\vec{x}^{\pi} := (x_{i}^{\pi} \; : \; 0 \leq i < |\vec{x}| )$ where $\pi \in \{L,R\}$.

We also use $P^{\Box} := P|\Box$.

In \cite{MeredithR05} an interpretation of the new operator is
given. It turns out that there are several possible interpretations
all enjoying the requisite algebraic properties of the operator (see
\cite{milner91polyadicpi}). We will therefore make liberal use of
$(\nu\; \vec{x})P$.

% subsection the_syntax_and_semantics_of_the_notation_system (end)   

\input{qm2pi.qmops} 

\input{qm2pi.sterngerlach} 

\input{qm2pi.metric} 

% section concurrent_process_calculi (end)

%\input{qm2pi.proofsketch}

% section proof sketch (end)

%\input{qm2pi.slviaknots} 

% section spatial logic via knots (end)

\input{qm2pi.conclusion}

% section conclusion (end)

%\input{qm2pi.dtcodes} 

% section wiring algorithm (end)

\input{qm2pi.ack} 

% section acknowledgments (end)

\newpage


\bibliographystyle{plain}   
\bibliography{../../biblios/main.bib}

\input{qm2pi.rhodetails}

\end{document}

 

% section concurrent_process_calculi (end)

%\documentclass[12pt]{llncs}
%\documentclass{jktr}

\usepackage[pdftex]{hyperref}                   
\usepackage {listings}
\usepackage {mathpartir}
\usepackage{bcprules}
%\usepackage{listings}
                       
\usepackage{graphicx} 
%\usepackage[margins=2.5cm,nohead,nofoot]{geometry}
%\usepackage{geometry}
\usepackage{amsfonts}
\usepackage{amstext}
\usepackage{latexsym}
\usepackage{amssymb}
\usepackage{color}


%\include{myPreamble}
\include{qm2pi.local} 

%\ifpdf
%\usepackage[pdftex]{graphicx}
%\else
%\usepackage{graphicx}
%\fi

 % \ifpdf
%  \usepackage{pdfsync}
%  \if


%\title{Brief Article}
%\author{David F. Snyder}
%\author{L.G. Meredith}

%\address{Dept. of Math., Texas State University--San Marcos, San Marcos, TX 78666}
       
\pagestyle{empty}


\begin{document}

\lstset{language=[Objective]Caml,frame=shadowbox}

\input{qm2pi.front}

% section front matter (end)

\input{qm2pi.intro} 
 
% section introduction (end)

% \input{qm2pi.knotations} 

% section notation (end)

\input{qm2pi.process.calculi} 

% section concurrent_process_calculi_and_spatial_logics_ (end)
    
%\input{qm2pi.knots2pi} 

%\input{qm2pi.trefoil} 

%\input{qm2pi.mainthm} 

% subsection basic_interpretation (end)

%\input{qm2pi.rho.presentation} 
\subsection{The syntax and semantics of the notation system}\label{sub:the_syntax_and_semantics_of_the_notation_system} % (fold)

We now summarize a technical presentation of the calculus that
embodies our theory of dynamics. The typical presentation of such a
calculus follows the style of giving generators and relations on
them. The grammar, below, describing term constructors, freely
generates the set of processes, $\Proc$. This set is then quotiented
by a relation known as structural congruence and it is over this set
that the notion of dynamics is expressed. This presentation is
essentially that of \cite{MeredithR05} with the addition of
polyadicity and summation. For readability we have relegated some of
the technical subtleties to an appendix.

\subsubsection{Process grammar}\label{subsub:process_grammar}

\begin{mathpar}
  \inferrule* [lab=synchronization] {} {{M} \bc \pzero \;|\; x?F \;|\; x!C }
  \and
  \inferrule* [lab=abstraction] {} {{F} \bc (x)P}
  \and
  \inferrule* [lab=concretion] {} {{C} \bc \langle Q \rangle}
  \and
  \inferrule* [lab=process] {} {{P,Q} \bc M \;| \;P|Q \;|\; @{x}}
  \and
  \inferrule* [lab=name] {} {{x} \bc \quotep{P}}
\end{mathpar} 

Note that $\vec{x}$ (resp. $\vec{P}$) denotes a vector of names
(resp. processes) of length $|\vec{x}|$ (resp. $|\vec{P}|$). We adopt
the following useful abbreviations.

\begin{mathpar}
   x?(\vec{y}).P := x.(\vec{y})P \and  x\clift{\vec{P}} := x.\clift{\vec{P}}
   \and x!(y) := \lift{x}{\dropn{y}}
   \and \Pi_{i=0}^{n-1}P_i := P_0 | \ldots | P_{n-1}
\end{mathpar}

\subsubsection{Structural congruence}

\paragraph{Free and bound names and alpha-equivalence.} At the
core of structural equivalence is alpha-equivalence which identifies
process that are the same up to a change of variable. Formally, we
recognize the distinction between free and bound names. The free names
of a process, $\freenames{P}$, may be calculated recursively as
follows:

\begin{mathpar}
\freenames{\pzero} := \emptyset
  \and \\
  \freenames{x?(y).P} := \{ x \} \cup (\freenames{P} \setminus \{ y \})
  \and 
  \freenames{x!\langle P \rangle} := \{ x \} \cup \{ P \} 
  \and \\
  \freenames{P|Q} := \freenames{P} \cup \freenames{Q}
  \and \\
  \freenames{@{x}} := \{ x \}
\end{mathpar}

$\pi$
$\quotep{\pi}$

$\freenames{-} : \pi \to \mathcal{P}(\quotep{\pi})$

\begin{eqnarray*}
  \freenames{\pzero} & := & \emptyset \\
  \freenames{x?(y).P} & := & \{ x \} \cup (\freenames{P} \setminus \{ y \}) \\
  \freenames{x!\langle P \rangle} & := & \{ x \} \cup \{ P \} \\
  \freenames{P|Q} & := & \freenames{P} \cup \freenames{Q} \\
  \freenames{\dropn{x}} & := & \{ x \}
\end{eqnarray*}

The bound names of a process, $\boundnames{P}$, are those names occurring in $P$
that are not free. For example, in $x?(y).0$, the name $x$ is free, while $y$ is bound.

\begin{mathpar}
  \inferrule* [lab=monoidal-laws] {} { P|Q \equiv Q|P \and P|0 \equiv P \and P|(Q|R) \equiv (P|Q)|R }
\end{mathpar}

\begin{mathpar}
  \inferrule* [lab=alpha-equivalence] {} { (x)P \equiv (y)P\{y/x\} \and y \not\in \freenames{P} }
\end{mathpar}

\begin{definition}
Then two processes, $P,Q$, are alpha-equivalent if $P = Q\{\vec{y}/\vec{x}\}$ for
some $\vec{x} \in \boundnames{Q},\vec{y} \in \boundnames{P}$, where $Q\{\vec{y}/\vec{x}\}$
denotes the capture-avoiding substitution of $\vec{y}$ for $\vec{x}$ in $Q$.
\end{definition}

\begin{definition}
  The {\em structural congruence} \cite{SangiorgiWalker} , $\equiv$,
  between processes is the least congruence containing
  alpha-equivalence, satisfying the abelian monoid laws
  (associativity, commutativity and $\pzero$ as identity) for parallel
  composition $|$ and for summation $+$.
\end{definition}

\subsection{Name equivalence}

We take name equivalence, written $\nameeq$, to be the smallest
equivalence relation generated by the following rules.

\begin{mathpar}
\inferrule*[lab=Quote-drop]
{ }
{ \quotep{@{x}} \nameeq x }

\inferrule*[lab=Struct-equiv]
{ P \scong Q }
{ \quotep{P} \nameeq \quotep{Q} }
\end{mathpar}

The astute reader will have noticed that the mutual recursion of names
and processes imposes a mutual recursion on alpha-equivalence and
structural equivalence via name-equivalence. Fortunately, all of this
works out pleasantly and we may calculate in the natural way, free of
concern. The reader interested in the details is referred to the
appendix \ref{appendix:rho_details}.

\subsection{Substitution}

We use $\Proc$ for the set of processes, $\QProc$ for the set of
names, and $\id{\{}\vec{y} / \vec{x} \id{\}}$ to denote partial maps,
$s : \QProc \rightarrow \QProc$. A map, $s$ lifts, uniquely, to a map
on process terms, $\widehat{s} : \Proc \rightarrow \Proc$ by the
following equations.

\begin{mathpar}
  (0) \psubstp{Q}{P} := 0 \\
  (R \juxtap S) \psubstp{Q}{P}
  :=    
  (R)\psubstp{Q}{P} \juxtap (S) \psubstp{Q}{P} \\
  (x?(y).R) \psubstp{Q}{P}    
  :=    
  (x)\substp{Q}{P} (z)\concat( (R \psubstn{z}{y}) \psubstp{Q}{P} ) \\
  (\lift{x}{R}) \psubstp{Q}{P}  
  :=
  \lift{(x)\substp{Q}{P}}{ R \psubstp{Q}{P} } \\
%   (\dropn{x})  \psubstp{Q}{P}       
%   := 
%   \left\{ 
%     \begin{array}{ccc} 
%       \dropn{\quotep{Q}} & & x \nameeq \quotep{P} \\
%       \dropn{x} & & otherwise \\
%     \end{array}
%   \right. 
  (\dropn{x})  \psubstp{Q}{P}       
  := 
  \left\{ 
    \begin{array}{ccc} 
      Q & & x \nameeq \quotep{P} \\
      \dropn{x} & & otherwise \\
    \end{array}
  \right.
\end{mathpar}
 

where

\begin{eqnarray}
  (x)\id{\{} \lpquote Q \rpquote / \lpquote P \rpquote \id{\}}            = 
  \left\{ 
    \begin{array}{ccc}
      \lpquote Q \rpquote & & x \nameeq \lpquote P \rpquote \\
      x & & otherwise \\
    \end{array}
  \right. \nonumber
\end{eqnarray}

and $z$ is chosen distinct from $\quotep{P}$, $\quotep{Q}$, the free
names in $Q$, and all the names in $R$. Our $\alpha$-equivalence will
be built in the standard way from this substitution.

\begin{remark}\label{rem:no_self_referential_names}
  One consequence of these definitions is that $\forall P. \quotep{P}
  \not\in \freenames{P}$.
\end{remark}

\subsection{ Dynamic quote: an example }

Anticipating something of what's to come, consider applying the
substitution, $\widehat{\id{\{}u / z \id{\}}}$, to the following pair
of processes, $\lift{w}{y!(z)}$ and $w[ \lpquote y!(z) \rpquote ]$.

\begin{eqnarray}
	\lift{w}{y!(z)}\widehat{\id{\{}u / z \id{\}}}
		& = &
		\lift{w}{y!(u)} \nonumber\\
	w[ \lpquote y!(z) \rpquote ] \widehat{ \id{\{}u / z \id{\}} }
		& = &
		w[ \lpquote y!(z) \rpquote ] \nonumber
\end{eqnarray}

Because the body of the process between quotes is impervious to
substitution, we get radically different answers. In fact, by
examining the first process in an input context,
e.g. $x?(z).\lift{w}{y!(z)}$, we see that the process under the lift
operator may be shaped by prefixed inputs binding a name inside it. In
this sense, the lift operator will be seen as a way to dynamically
construct processes before reifying them as names.

Finally equipped with these standard features we can present the
dynamics of the calculus.

\subsubsection{Operational semantics} 

Finally, we introduce the computational dynamics. What marks these
algebras as distinct from other more traditionally studied algebraic
structures, e.g. vector spaces or polynomial rings, is the manner in
which dynamics is captured. In traditional structures, dynamics is typically
expressed through morphisms between such structures, as in linear maps
between vector spaces or morphisms between rings. In algebras
associated with the semantics of computation, the dynamics is
expressed as part of the algebraic structure itself, through a
reduction reduction relation typically denoted by $\red$. Below, we
give a recursive presentation of this relation for the calculus used
in the encoding.

$\red \subseteq \pi \times \pi$
$\red : \pi \to \mathcal{P}(\pi)$

\begin{mathpar}
  \inferrule* [lab=Comm] { \textsf{match}( x_{src}, x_{trgt} ) } { x_{trgt}?(y)P \; | \; x_{src}!\langle {Q} \rangle \red P\{\quotep{Q}/y}\} }
  \and \\
  \inferrule* [lab=Par] {{P} \red {P}'} {{{P} | {Q}} \red {{P}' | {Q}}}
  \and
  \inferrule* [lab=Equiv]{{{P} \scong {P}'} \andalso {{P}' \red {Q}'} \andalso {{Q}' \scong {Q}}}{{P} \red {Q}}
\end{mathpar}

\begin{eqnarray*}
  match_{\equiv} (\quotep{P},\quotep{Q}) & := & P \equiv Q \\
  match_{\dagger}(\quotep{P},\quotep{Q}) & := & \forall R. P|Q \red^{*} R => R \red^{*} 0 \\
  match_{K}(\quotep{P},\quotep{Q}) & := & K \mbox{ for some context } K
\end{eqnarray*}

$u?(x)P | u!\langle Q \rangle \red P\{\quotep{Q}/x\}$

%We write $\wred$ for $\red^*$, and $P\red$ if $\exists Q $ such that $ P \red Q$.
We write $P\red$ if $\exists Q $ such that $ P \red Q$ and $P\not\red$, otherwise.

\section{Replication}

As mentioned before, it is known that replication (and hence
recursion) can be implemented in a higher-order process algebra
\cite{SangiorgiWalker}. As our first example of calculation with the
machinery thus far presented we give the construction explicitly in
the {\rhoc}.

\begin{eqnarray}
	D_{x} & := & \prefix{x}{y}{(\binpar{\outputp{x}{y}}{@{y}})} \nonumber\\
	\bangp_{x}{P} & := & \binpar{{x}!\langle{\binpar{D_{x}}{P}}\rangle}{D_{x}} \nonumber
\end{eqnarray}

\begin{eqnarray}
	\bangp_{x}{P} & & \nonumber\\
	=
	& {x}!\langle{(\prefix{x}{y}{(\outputp{x}{y} | @{y})) | P}}\rangle 
	      | \prefix{x}{y}{(\outputp{x}{y} | @{y})} & \nonumber\\
	\red
	& (\outputp{x}{y} | @{y})\substn{\quotep{(\prefix{x}{y}{(@{y} | \outputp{x}{y})) | P}}}{y} & \nonumber\\
	=
	& \outputp{x}{\quotep{(\prefix{x}{y}{(\outputp{x}{y} | @{y})) | P}}}
	  | {(\prefix{x}{y}{(\outputp{x}{y} | @{y})) | P}} & \nonumber\\
	\red
	& \ldots & \nonumber\\
	\red^*
	& P | P | \ldots & \nonumber
\end{eqnarray}

Of course, this encoding, as an implementation, runs away, unfolding
$\bangp{P}$ eagerly. A lazier and more implementable replication
operator, restricted to input-guarded processes, may be obtained as follows.

\begin{eqnarray}
\bangp{\prefix{u}{v}{P}} 
	:= 
	\binpar{\lift{x}{\prefix{u}{v}{(\binpar{D(x)}{P})}}}{D(x)} \nonumber
\end{eqnarray}

\begin{remark}
  Note that the lazier definition still does not deal with summation
  or mixed summation (i.e. sums over input and output). The reader is
  invited to construct definitions of replication that deal with these
  features. 

  Further, the definitions are parameterized in a name, $x$. Can you,
  gentle reader, make a definition that eliminates this parameter and
  guarantees no accidental interaction between the replication
  machinery and the process being replicated -- i.e. no accidental
  sharing of names used by the process to get its work done and the
  name(s) used by the replication to effect copying. This latter
  revision of the definition of replication is crucial to obtaining
  the expected identity $!!P \sim !P$.
\end{remark}

\begin{remark}\label{rem:paradoxical_combinator}
  The reader familiar with the lambda calculus will have noticed the
  similarity between $D$ and the paradoxical combinator.

  [Ed. note: the existence of this seems to suggest we have to be more
  restrictive on the set of processes and names we admit if we are to
  support no-cloning.]
\end{remark}

\subsubsection{Bisimulation}

The computational dynamics gives rise to another kind of equivalence,
the equivalence of computational behavior. As previously mentioned
this is typically captured \emph{via} some form of bisimulation.

% The notion we use in this paper is weak barbed bisimulation
% \cite{milner91polyadicpi}.

The notion we use in this paper is derived from weak barbed
bisimulation \cite{milner91polyadicpi}. 

\begin{definition}
An \emph{observation relation}, $\downarrow_{\mathcal N}$, over a set
of names, $\mathcal N$, is the smallest relation satisfying the rules
below.

\infrule[Out-barb]{y \in {\mathcal N}, \; x \nameeq y}
		  {\outputp{x}{v} \downarrow_{\mathcal N} x}
\infrule[Par-barb]{\mbox{$P\downarrow_{\mathcal N} x$ or $Q\downarrow_{\mathcal N} x$}}
		  {\binpar{P}{Q} \downarrow_{\mathcal N} x}

We write $P \Downarrow_{\mathcal N} x$ if there is $Q$ such that 
$P \wred Q$ and $Q \downarrow_{\mathcal N} x$.
\end{definition}

\begin{definition}
%\label{def.bbisim}
An  ${\mathcal N}$-\emph{barbed bisimulation} over a set of names, ${\mathcal N}$, is a symmetric binary relation 
${\mathcal S}_{\mathcal N}$ between agents such that $P\rel{S}_{\mathcal N}Q$ implies:
\begin{enumerate}
\item If $P \red P'$ then $Q \wred Q'$ and $P'\rel{S}_{\mathcal N} Q'$.
\item If $P\downarrow_{\mathcal N} x$, then $Q\Downarrow_{\mathcal N} x$.
\end{enumerate}
$P$ is ${\mathcal N}$-barbed bisimilar to $Q$, written
$P \wbbisim_{\mathcal N} Q$, if $P \rel{S}_{\mathcal N} Q$ for some ${\mathcal N}$-barbed bisimulation ${\mathcal S}_{\mathcal N}$.
\end{definition}

$\mathcal{R} \subseteq \pi \times \pi$

$P \mathcal{R} Q => \forall P'. P \red P' \Rightarrow \exists Q'. Q \red Q', P' \mathcal{R} Q'$

$P \vdash x \Rightarrow Q \vdash x$

\begin{mathpar}
  \inferrule*[lab=Out-barb]{x \nameeq y}{{y}!\langle{Q}\rangle \vdash x}
  \and
  \inferrule*[lab=Par-barb]{\mbox{$P\vdash x$ or $Q\vdash x$}}{\binpar{P}{Q} \vdash x}
\end{mathpar}

\subsubsection{Contexts}

One of the principle advantages of computational calculi like the
$\pi$-calculus is a well-defined notion of context,
contextual-equivalence and a correlation between
contextual-equivalence and notions of bisimulation. The notion of
context allows the decomposition of a process into (sub-)process and
its syntactic environment, its context. Thus, a context may be
thought of as a process with a ``hole'' (written $\Box$) in it. The
application of a context $M$ to a process $P$, written $M[P]$, is
tantamount to filling the hole in $M$ with $P$. In this paper we do
not need the full weight of this theory, but do make use of the notion
of context in the proof the main theorem. 

\begin{mathpar}
  \inferrule* [lab=summation] {} {{M_{M},M_{N}} \bc \Box \;|\; x.M_{A} \;|\; M_{M}+M_{N}}
  \and
  \inferrule* [lab=agent] {} {{M_{A}} \bc (\vec{x})M_{P} \;| \; \clift{P_0,\ldots,M_{P},\ldots,P_N}}
  \and \\
  \inferrule* [lab=process] {} {{M_{P}} \bc M_{N} \;| \;P|M_{P} }
\end{mathpar} 

\begin{mathpar}
  \inferrule* [lab=sychronization] {} {M_{N} \bc \Box \;|\; x?M_{F} \;|\; x!M_{C}}
  \and
  \inferrule* [lab=abstraction] {} {{M_{F}} \bc (x)M_{P} }
  \and
  \inferrule* [lab=concretion] {} {{M_{C}} \bc \langle M_{P} \rangle }
  \and \\
  \inferrule* [lab=process] {} {{M_{P}} \bc M_{N} \;| \;P|M_{P} }
\end{mathpar}

\begin{definition}[contextual application] Given a context $M$, and
  process $P$, we define the \emph{contextual application}, $M[P] :=
  M\{P/\Box\}$. That is, the contextual application of M to P is the
  substitution of $P$ for $\Box$ in $M$.
\end{definition}

$\meaningof{-} : L \to \mathcal{P}(\pi)$

\begin{mathpar}
  \inferrule* [lab=collection] {} {\meaningof{true} = \pi, \and \meaningof{~E} = \pi \setminus \meaningof{E}, \and \meaningof{E_{1} \& E_{2}} = \meaningof{E_{1}} \cap \meaningof{E_{2}}}
\end{mathpar}

\begin{mathpar}
  \inferrule* [lab=structure] {} {\meaningof{0} = \{ P \in \pi | P \equiv 0 \}, \and \\ \meaningof{E_1 | E_2} = \{ P \in \pi | P \equiv P_{1} | P_{2}, P_{1} \in \meaningof{E_{1}}, P_{2} \in \meaningof{E_2}\} }
\end{mathpar}

\begin{mathpar}
 \inferrule* [lab=behavior] {} {\meaningof{\langle a?b \rangle E} = \{ P \in \pi | P \equiv Q | u?(y)P', \\ \and \\\\ \and \\ \;\;\; u \in \meaningof{a}, \forall z.P'\{z/y\} \in \meaningof{E\{z/b\}}\}, \and \\ \meaningof{a!E} = \{ P \in \pi | P \equiv Q | x!\langle P' \rangle, x \in \meaningof{a} P' \in \meaningof{E}\} }
\end{mathpar}

\begin{mathpar}
 \inferrule* [lab=nominal] {} {\meaningof{\quotep{E}} = \{ \quotep{P} \in \quotep{\pi} | P \in \meaningof{E} \}, \and \meaningof{\quotep{P}} = \{ \quotep{Q} \in \quotep{\pi} | P \equiv Q \} \and \\ \meaningof{@\quotep{E}} = \{ P \in \pi | P \equiv @x, x \in \meaningof{E} \}}
\end{mathpar}

\begin{eqnarray*}
  \\
  \meaningof{-} : TS \to ST
\end{eqnarray*}

\begin{eqnarray*}
  \\
  L : TS \to ST
\end{eqnarray*}

\begin{eqnarray*}
  \\
  P \models E \iff P \in \meaningof{E}
\end{eqnarray*}

\begin{eqnarray*}
  P \approx_{L} Q \iff \forall E \in L. P \models E \iff Q \models E
\end{eqnarray*}

\begin{eqnarray*}
  P \approx_{K} Q
\end{eqnarray*}

\begin{eqnarray*}
  P \approx Q
\end{eqnarray*}

$\approx_{K} = \approx = \approx_{L}$

\subsubsection{Contextual duality}

Note that contexts extend the quotation operation to a family of
operations from processes to names. Given a context, $M$, we can
define a \emph{nominal context}, $\quotep{M}$ by $\quotep{M}[P] :=
\quotep{M[P]}$. To foreshadow what is to come we observe that these
operations enjoy a duality with processes very much like the duality
between vectors and maps from vectors to scalars.

Further, because the calculus is essentially higher-order, we have a
correspondence between contexts and processes. More specifically,
given a name $x$ and a context $M$ we can construct $M^{*}_{x}$ such
that 

\begin{mathpar}
  M^{*}_{x} | \lift{x}{P} \red M[P]
\end{mathpar}

namely,

\begin{mathpar}
  M^{*}_{x} := x?(u).M[\dropn{u}]
\end{mathpar}

The dependence of $M^{*}_{x}$ on a name makes it an abstraction, 

\begin{mathpar}
  M^{*} := (x)x?(u).M[\dropn{u}]
\end{mathpar}

\subsection{Additional notation}

It will sometimes be convenient to denote the process a name
quotes. We already have the notation $x = \quotep{P}$, but it will be
convenient to introduce an alternate notation, $\procn{x}$, when we
want to emphasize the connection to the use of the name. Note that, by
virtue of name equivalence, $\quotep{\procn{x}} \nameeq x$; so, the
notation is consistent with previous definitions.

Further, because names have structure it is possible to effect
substitutions on the basis of that structure. This means we need to
upgrade our notation for substitutions, which we accomplish by
adapting comprehension notation. Thus,

\begin{mathpar}
  P\{ y / x : x \in S \}
\end{mathpar}

is interpreted to mean the process derived from P by replacing (in a
capture-avoiding manner) each occurrence of $x$ in $S$ by $y$. For example,

\begin{mathpar}
  P\{ \quotep{\procn{x}|\procn{x}} / x : x \in \freenames{P} \}
\end{mathpar}

will replace each (occurrence) of a free name $x$ in $P$ by
$\quotep{\procn{x}|\procn{x}}$.

Also, we will avail ourselves of the notation $x^{L}$ and $x^{R}$ to
denote injections of a name into disjoint copies of the name
space. There are numerous ways to accomplish this. One example can be
found in \cite{MeredithR05}. This notation overloads to vectors of
names: $\vec{x}^{\pi} := (x_{i}^{\pi} \; : \; 0 \leq i < |\vec{x}| )$ where $\pi \in \{L,R\}$.

We also use $P^{\Box} := P|\Box$.

In \cite{MeredithR05} an interpretation of the new operator is
given. It turns out that there are several possible interpretations
all enjoying the requisite algebraic properties of the operator (see
\cite{milner91polyadicpi}). We will therefore make liberal use of
$(\nu\; \vec{x})P$.

% subsection the_syntax_and_semantics_of_the_notation_system (end)   

\input{qm2pi.qmops} 

\input{qm2pi.sterngerlach} 

\input{qm2pi.metric} 

% section concurrent_process_calculi (end)

%\input{qm2pi.proofsketch}

% section proof sketch (end)

%\input{qm2pi.slviaknots} 

% section spatial logic via knots (end)

\input{qm2pi.conclusion}

% section conclusion (end)

%\input{qm2pi.dtcodes} 

% section wiring algorithm (end)

\input{qm2pi.ack} 

% section acknowledgments (end)

\newpage


\bibliographystyle{plain}   
\bibliography{../../biblios/main.bib}

\input{qm2pi.rhodetails}

\end{document}



% section proof sketch (end)

%\section{Unlikely characters: spatial logic for
  knots}\label{sub:characteristic_formulae} % (fold)

Associated to the mobile process calculi are a family of logics known
as the Hennessy-Milner logics. These logics typically enjoy a
semantics interpreting formulae as sets of processes that when
factored through the encoding outlined above allows an identification
of classes of knots with logical formulae. In the context of this
encoding the sub-family known as the spatial logics \cite{CairesC03}
\cite{CairesC04} \cite{Caires04} are of particular interest providing
several important features for expressing and reasoning about
properties (i.e. classes) of knots. We hint here at how this may be done.

%\begin{description}
%\item [structural connectives] 
\subsubsection{Structural connectives} The spatial logics enjoy
structural connectives corresponding, at the logical level, to the
parallel composition ($P | Q$) and new name ($(\nu \; x)P$)
connectives for processes. As illustrated in the examples below, these
connectives are extremely expressive given the shape of our encoding.
%\item [decideable satisfaction]

\subsubsection{Decideable satisfaction}
In \cite{Caires04} the satisfaction relation is shown to be decideable
for a rich class of processes. It further turns out that the image of
the our encoding is a proper subset of that class. This result
provides the basis for an algorithm by which to search for knots
enjoying a given property.
%\item [characteristic formulae]

\subsubsection{Characteristic formulae}
In the same paper \cite{Caires04} , Caires presents a means of calculating
characteristic formulae, selecting equivalence classes of processes
up to a pre--specified depth limit on the support set of names. Composed with our
encoding, this characteristic formula can be used to select
characteristic formulae for knots.
%\end{description}

\subsubsection{Spatial logic formulae}

The grammar below (segmented for comprehension) summarizes the syntax
of spatial logic formulae. We employ illustrative examples in the
sequel to provide an intuitive understanding of their meaning
referring the reader to \cite{Caires04} for a more detailed explication
of the semantics.

\begin{mathpar}
  \inferrule* [lab=boolean] {} {{A,B} \bc T \;|\; \neg A \;|\; A \wedge B \;|\; \eta = \eta'}
  \and
  \inferrule* [lab=spatial] {} {|\; \pzero \;|\; A | B \;|\; x \text{\textregistered} A \;|\; \forall x . A \;|\;  H x . A}
  \and
  \inferrule* [lab=behavioral] {} {|\; \alpha . A}
  \and 
  \inferrule* [lab=recursion] {} {|\; X(\vec{u}) \;|\; \mu X(\vec{u}) . A}
  \and
  \inferrule* [lab=action] {} {\alpha \bc \langle x?(\vec{y}) \rangle \;|\; \langle x!(\vec{y}) \rangle \;|\; \langle \tau \rangle}
  \and 
  \inferrule* [lab=name] {} {\eta \bc x \;|\; \tau}
\end{mathpar} 

% subsection characteristic_formulae (end)   	 

\subsection{Example formulae}\label{sub:example_formulae_} % (fold)

\subsubsection{Crossing as formula.}
% 
% \begin{align*}
%   \frac{d}{dx} \sin x &= \cos x 
%   & \frac{d}{dx} e^x &= e^x \\
%   \frac{d}{dx} \cos x &= - \sin x 
%   & \frac{d}{dx} \log x &= \frac{1}{x} \\
% \end{align*} 

\begin{align*}
 \mu C(x_{0},x_{1},y_{0},y_{1},u).&(\langle x_{0}?(z) \rangle(\langle u! \rangle\langle y_{1}!z \rangle C(x_{0},x_{1},y_{0},y_{1},u)) & \\
  & \wedge \langle y_{1}?(z) \rangle (\langle u! \rangle \langle x_{0}!z \rangle C(x_{0},x_{1},y_{0},y_{1},u)) & \\
  & \wedge \langle x_{1}?(z) \rangle (\langle u? \rangle \langle y_{0}!z \rangle C(x_{0},x_{1},y_{0},y_{1},u)) & \\
  & \wedge \langle y_{0}?(z) \rangle (\langle u? \rangle \langle x_{1}!z \rangle C(x_{0},x_{1},y_{0},y_{1},u))) &
\end{align*}

The lexicographical similarity between the shape of this formulae and
the shape of definition of the process representing a crossing reveals
the intuitive meaning of this formulae. It describes the capabilities
of a process that has the right to represent a crossing. For example
it picks out processes that may perform an input on the port $x_0$ in
its initial menu of capabilities. What differentiates the formula
from the process, however, is that the crossing process is the
smallest candidate to satisfy the formula. Infinitely many other
processes -- with internal behavior hidden behind this interface, so
to speak -- also satisfy this formula. Even this simple formula,
then, can be seen to open a new view onto knots, providing a
computational interpretation of \emph{virtual} knots.

Note that this formula is derived by hand. A similar formula can be
derived by employing Caires' calculation of characteristic formula
\cite{Caires04} to the process representing a crossing. In light of
this discussion, we let
$\meaningof{C}_{\phi}(x0,x1,y0,y1,u)$ denote a formula specifying the
dynamics we wish to capture of a crossing. To guarantee we preserve
the shape of the interface and minimal semantics we demand that
$\meaningof{C}_{\phi}(x0,x1,y0,y1,u) \Rightarrow
\textbf{C}(x0,x1,y0,y1,u)$ where $\textbf{C}(x0,x1,y0,y1,u)$ denotes
the formula above.
                            
\subsubsection{Crossing number constraints.}
The moral content of the context lemma (Lemma \ref{context}) is that the notion of
``locality'' in the Reidemeister moves is effectively captured by the
parallel composition operator of the process calculus. This intuition
extends through the logic. Given a formula,
$\meaningof{C}_{\phi}(x0,x1,y0,y1,u)$, we can use the structural
connectives to specify constraints on crossing numbers, such as at
least $n$ crossings, or exactly $n$ crossings.
\begin{mathpar}
  \inferrule* [lab=at-least-n] {} { K^{\geq n}_{\phi}(\vec{xs},\vec{ys}) := \Pi_{i=0}^{n-1} Hu . \meaningof{C}_{\phi}(xs_i,ys_i,u) | T }
  \and 
  \inferrule* [lab=exactly-n] {} { K^{= n}_{\phi}(\vec{xs},\vec{ys}) := \Pi_{i=0}^{n-1} Hu . \meaningof{C}_{\phi}(xs_i,ys_i,u) | \neg (\forall x_0,y_0,x_1,y_1,u . \meaningof{C}_{\phi}(x_0,y_0,x_1,y_1,u) | T) }
\end{mathpar}

To round out this section, recall that the encoding of an $n$-crossing
knot decomposes into a parallel composition of $n$ \emph{copies} of a
crossing process together with a wiring harness. To specify different
knot classes with the same crossing number amounts to specifying
logical constraints on the wiring harness. In the interest of space,
we defer examples to a forthcoming paper. Suffice it to say that both
the conditions ``alternating knot'' and ``contains the tangle
corresponding to 5/3'' are expressible. For example, it is possible to
calculate the characteristic formula of a process corresponding to the
tangle 5/3 and conjoin it into the classifying formula via the
composition connective of the logic.

Finally, we wish to observe that it is entirely within reason to
contemplate a more domain-specific version of spatial logic tailored
to the shape of processes in the image of the encoding. Such a
domain-specific logic would have a better claim to the title formal
language of knot properties.

% subsection example_formulae_ (end)

% section knots_as_processes (end) 

% section spatial logic via knots (end)

\section{Conclusions and future work}

\paragraph{Testing physical space}
You, gentle reader, may wonder why of all the theorems to be proved
given this set up we pick the one above. In some sense it's hardly
central to quantum mechanics. We see it as central in the sense that
it firmly establishes a notion of physical space arising from a notion
of the equivalence of behavior. Relating bisimulation to a metric is a
big step forward, but one is faced with interpreting the relationship
of that metric space to something more physical. Quantum mechanical
notions of ``physical'' space are still far from intuitive, but by
relating this idea of distance as testing to calculations that predict
physical circumstances we are making a not insignificant step forward
toward an understanding of the physical space we inhabit as
essentially dynamic.

\paragraph{Effectivity and simulation}
One of the observations we have yet to make is that the entire program
spelled out here is effective. We have built various interpreters for
the reflective calculus at work in this interpretation. In principle,
then, we can simulate quantum mechanics on a computer. The place where
the simulation may lose fidelity is the infinitely branching summation
for the annihilator.

In this connection i also want to point out that the evaluation style
calculation of the inner product puts the non-determinism of the
summation right at the heart of measurement. This suggests that
Milner's original reduction-based formulation of the dynamics of his
calculi in terms of sums was not just notationally suggestive of a
notion of measure-and-continue but captured some significant part of
the physics.

\paragraph{Quantum continuations}
In light of this last observation i want to point out that the
predominant account of quantum mechanics is missing a key aspect of a
truly compositional story of the physical situation. In a real lab,
when a measurement is made the observation can be made to feed into
another device that then makes another measurement conditioned on the
results of the first. This means that after the superposition was
collapsed the entire experimental set up remained in
superposition. While QM offers a means of writing this down it doesn't
quite line up well with the well-trodden formulation of computation
and continuation that we see so succinctly expressed in Milner's
calculi. This suggests that there might be advantages to this account
of dynamics waiting to be explored.

\paragraph{Quantum logic}
In this connection, we also note that by virtue of having the
Hennessy-Milner construction, we can pull the construction through the
interpretation of QM. This gives us a natural candidate for a quantum
logic that enjoys an extremely tight connection with it's domain of
interpretation, making the construction much less ad hoc (rather it is
the image of functor!).

\paragraph{Quantum probabiity}
i have questions about the basis of the interpretation of inner
product as probability amplitude. In particular, using which
axiomatization of probability theory does the notion of probability
amplitude earn the right to be so dubbed? In other words, where is the
proof that the operation for calculating a probability amplitude (and
then squaring) satisfies the axioms of what it means to calculate a
probability? Even if such a proof exists (i have yet to find it in the
literature), i wonder if it might not be possible to turn things on
their heads. Can we view the calculation of the probability amplitude
as an axiomatization of probability? If so, then the definition we
give for calculating probability amplitude may provide the basis for
an \emph{effective} theory of probability.

\paragraph{Quantum vs ``biological'' information}
Finally, i want to conclude with a more philosophical observation. At
a recent workshop in which QM was a predominant topic i noticed
something about quantum information. The speaker was giving a riveting
discussion of axiomatic QM and showing how properties of ``no
cloning'' and ``no deleting'' emerged as consequences of the
axiomatization. Theorems of this form are necessary to give us a sense
of confidence that our axioms characterize the physical theory. What
struck me, though, was that if quantum information is neither erasable
nor replicable it is markedly different from \emph{life}. Two of the
things we know about life is that

\begin{itemize}
  \item it ends;
  \item to gain some measure of persistence, to transcend it's
    finitude it is imminently copyable.
\end{itemize}

Both of these qualities are summarized succinctly in the aphorism: all
flesh is grass. For me these two kinds of ``information'' -- call them
quantum and biological -- are end points on a spectrum of strategies
for persistence. At one end, we have those curious entities that enjoy
uniqueness and permanence; at the other, we have those who in the face
of a certain end and an uncertain present make a go of passing
something on. To me one of the more remarkable aspects of the latter
strategy is that in the presence of noise (and certain features of
copying) we get a kind of dynamism, a chance for improvement against a
given persistent condition.

% subsection other_calculi_other_bisimulations_and_geometry_as_behavior (end)




% section conclusion (end)

%\documentclass[12pt]{llncs}
%\documentclass{jktr}

\usepackage[pdftex]{hyperref}                   
\usepackage {listings}
\usepackage {mathpartir}
\usepackage{bcprules}
%\usepackage{listings}
                       
\usepackage{graphicx} 
%\usepackage[margins=2.5cm,nohead,nofoot]{geometry}
%\usepackage{geometry}
\usepackage{amsfonts}
\usepackage{amstext}
\usepackage{latexsym}
\usepackage{amssymb}
\usepackage{color}


%\include{myPreamble}
\include{qm2pi.local} 

%\ifpdf
%\usepackage[pdftex]{graphicx}
%\else
%\usepackage{graphicx}
%\fi

 % \ifpdf
%  \usepackage{pdfsync}
%  \if


%\title{Brief Article}
%\author{David F. Snyder}
%\author{L.G. Meredith}

%\address{Dept. of Math., Texas State University--San Marcos, San Marcos, TX 78666}
       
\pagestyle{empty}


\begin{document}

\lstset{language=[Objective]Caml,frame=shadowbox}

\input{qm2pi.front}

% section front matter (end)

\input{qm2pi.intro} 
 
% section introduction (end)

% \input{qm2pi.knotations} 

% section notation (end)

\input{qm2pi.process.calculi} 

% section concurrent_process_calculi_and_spatial_logics_ (end)
    
%\input{qm2pi.knots2pi} 

%\input{qm2pi.trefoil} 

%\input{qm2pi.mainthm} 

% subsection basic_interpretation (end)

%\input{qm2pi.rho.presentation} 
\subsection{The syntax and semantics of the notation system}\label{sub:the_syntax_and_semantics_of_the_notation_system} % (fold)

We now summarize a technical presentation of the calculus that
embodies our theory of dynamics. The typical presentation of such a
calculus follows the style of giving generators and relations on
them. The grammar, below, describing term constructors, freely
generates the set of processes, $\Proc$. This set is then quotiented
by a relation known as structural congruence and it is over this set
that the notion of dynamics is expressed. This presentation is
essentially that of \cite{MeredithR05} with the addition of
polyadicity and summation. For readability we have relegated some of
the technical subtleties to an appendix.

\subsubsection{Process grammar}\label{subsub:process_grammar}

\begin{mathpar}
  \inferrule* [lab=synchronization] {} {{M} \bc \pzero \;|\; x?F \;|\; x!C }
  \and
  \inferrule* [lab=abstraction] {} {{F} \bc (x)P}
  \and
  \inferrule* [lab=concretion] {} {{C} \bc \langle Q \rangle}
  \and
  \inferrule* [lab=process] {} {{P,Q} \bc M \;| \;P|Q \;|\; @{x}}
  \and
  \inferrule* [lab=name] {} {{x} \bc \quotep{P}}
\end{mathpar} 

Note that $\vec{x}$ (resp. $\vec{P}$) denotes a vector of names
(resp. processes) of length $|\vec{x}|$ (resp. $|\vec{P}|$). We adopt
the following useful abbreviations.

\begin{mathpar}
   x?(\vec{y}).P := x.(\vec{y})P \and  x\clift{\vec{P}} := x.\clift{\vec{P}}
   \and x!(y) := \lift{x}{\dropn{y}}
   \and \Pi_{i=0}^{n-1}P_i := P_0 | \ldots | P_{n-1}
\end{mathpar}

\subsubsection{Structural congruence}

\paragraph{Free and bound names and alpha-equivalence.} At the
core of structural equivalence is alpha-equivalence which identifies
process that are the same up to a change of variable. Formally, we
recognize the distinction between free and bound names. The free names
of a process, $\freenames{P}$, may be calculated recursively as
follows:

\begin{mathpar}
\freenames{\pzero} := \emptyset
  \and \\
  \freenames{x?(y).P} := \{ x \} \cup (\freenames{P} \setminus \{ y \})
  \and 
  \freenames{x!\langle P \rangle} := \{ x \} \cup \{ P \} 
  \and \\
  \freenames{P|Q} := \freenames{P} \cup \freenames{Q}
  \and \\
  \freenames{@{x}} := \{ x \}
\end{mathpar}

$\pi$
$\quotep{\pi}$

$\freenames{-} : \pi \to \mathcal{P}(\quotep{\pi})$

\begin{eqnarray*}
  \freenames{\pzero} & := & \emptyset \\
  \freenames{x?(y).P} & := & \{ x \} \cup (\freenames{P} \setminus \{ y \}) \\
  \freenames{x!\langle P \rangle} & := & \{ x \} \cup \{ P \} \\
  \freenames{P|Q} & := & \freenames{P} \cup \freenames{Q} \\
  \freenames{\dropn{x}} & := & \{ x \}
\end{eqnarray*}

The bound names of a process, $\boundnames{P}$, are those names occurring in $P$
that are not free. For example, in $x?(y).0$, the name $x$ is free, while $y$ is bound.

\begin{mathpar}
  \inferrule* [lab=monoidal-laws] {} { P|Q \equiv Q|P \and P|0 \equiv P \and P|(Q|R) \equiv (P|Q)|R }
\end{mathpar}

\begin{mathpar}
  \inferrule* [lab=alpha-equivalence] {} { (x)P \equiv (y)P\{y/x\} \and y \not\in \freenames{P} }
\end{mathpar}

\begin{definition}
Then two processes, $P,Q$, are alpha-equivalent if $P = Q\{\vec{y}/\vec{x}\}$ for
some $\vec{x} \in \boundnames{Q},\vec{y} \in \boundnames{P}$, where $Q\{\vec{y}/\vec{x}\}$
denotes the capture-avoiding substitution of $\vec{y}$ for $\vec{x}$ in $Q$.
\end{definition}

\begin{definition}
  The {\em structural congruence} \cite{SangiorgiWalker} , $\equiv$,
  between processes is the least congruence containing
  alpha-equivalence, satisfying the abelian monoid laws
  (associativity, commutativity and $\pzero$ as identity) for parallel
  composition $|$ and for summation $+$.
\end{definition}

\subsection{Name equivalence}

We take name equivalence, written $\nameeq$, to be the smallest
equivalence relation generated by the following rules.

\begin{mathpar}
\inferrule*[lab=Quote-drop]
{ }
{ \quotep{@{x}} \nameeq x }

\inferrule*[lab=Struct-equiv]
{ P \scong Q }
{ \quotep{P} \nameeq \quotep{Q} }
\end{mathpar}

The astute reader will have noticed that the mutual recursion of names
and processes imposes a mutual recursion on alpha-equivalence and
structural equivalence via name-equivalence. Fortunately, all of this
works out pleasantly and we may calculate in the natural way, free of
concern. The reader interested in the details is referred to the
appendix \ref{appendix:rho_details}.

\subsection{Substitution}

We use $\Proc$ for the set of processes, $\QProc$ for the set of
names, and $\id{\{}\vec{y} / \vec{x} \id{\}}$ to denote partial maps,
$s : \QProc \rightarrow \QProc$. A map, $s$ lifts, uniquely, to a map
on process terms, $\widehat{s} : \Proc \rightarrow \Proc$ by the
following equations.

\begin{mathpar}
  (0) \psubstp{Q}{P} := 0 \\
  (R \juxtap S) \psubstp{Q}{P}
  :=    
  (R)\psubstp{Q}{P} \juxtap (S) \psubstp{Q}{P} \\
  (x?(y).R) \psubstp{Q}{P}    
  :=    
  (x)\substp{Q}{P} (z)\concat( (R \psubstn{z}{y}) \psubstp{Q}{P} ) \\
  (\lift{x}{R}) \psubstp{Q}{P}  
  :=
  \lift{(x)\substp{Q}{P}}{ R \psubstp{Q}{P} } \\
%   (\dropn{x})  \psubstp{Q}{P}       
%   := 
%   \left\{ 
%     \begin{array}{ccc} 
%       \dropn{\quotep{Q}} & & x \nameeq \quotep{P} \\
%       \dropn{x} & & otherwise \\
%     \end{array}
%   \right. 
  (\dropn{x})  \psubstp{Q}{P}       
  := 
  \left\{ 
    \begin{array}{ccc} 
      Q & & x \nameeq \quotep{P} \\
      \dropn{x} & & otherwise \\
    \end{array}
  \right.
\end{mathpar}
 

where

\begin{eqnarray}
  (x)\id{\{} \lpquote Q \rpquote / \lpquote P \rpquote \id{\}}            = 
  \left\{ 
    \begin{array}{ccc}
      \lpquote Q \rpquote & & x \nameeq \lpquote P \rpquote \\
      x & & otherwise \\
    \end{array}
  \right. \nonumber
\end{eqnarray}

and $z$ is chosen distinct from $\quotep{P}$, $\quotep{Q}$, the free
names in $Q$, and all the names in $R$. Our $\alpha$-equivalence will
be built in the standard way from this substitution.

\begin{remark}\label{rem:no_self_referential_names}
  One consequence of these definitions is that $\forall P. \quotep{P}
  \not\in \freenames{P}$.
\end{remark}

\subsection{ Dynamic quote: an example }

Anticipating something of what's to come, consider applying the
substitution, $\widehat{\id{\{}u / z \id{\}}}$, to the following pair
of processes, $\lift{w}{y!(z)}$ and $w[ \lpquote y!(z) \rpquote ]$.

\begin{eqnarray}
	\lift{w}{y!(z)}\widehat{\id{\{}u / z \id{\}}}
		& = &
		\lift{w}{y!(u)} \nonumber\\
	w[ \lpquote y!(z) \rpquote ] \widehat{ \id{\{}u / z \id{\}} }
		& = &
		w[ \lpquote y!(z) \rpquote ] \nonumber
\end{eqnarray}

Because the body of the process between quotes is impervious to
substitution, we get radically different answers. In fact, by
examining the first process in an input context,
e.g. $x?(z).\lift{w}{y!(z)}$, we see that the process under the lift
operator may be shaped by prefixed inputs binding a name inside it. In
this sense, the lift operator will be seen as a way to dynamically
construct processes before reifying them as names.

Finally equipped with these standard features we can present the
dynamics of the calculus.

\subsubsection{Operational semantics} 

Finally, we introduce the computational dynamics. What marks these
algebras as distinct from other more traditionally studied algebraic
structures, e.g. vector spaces or polynomial rings, is the manner in
which dynamics is captured. In traditional structures, dynamics is typically
expressed through morphisms between such structures, as in linear maps
between vector spaces or morphisms between rings. In algebras
associated with the semantics of computation, the dynamics is
expressed as part of the algebraic structure itself, through a
reduction reduction relation typically denoted by $\red$. Below, we
give a recursive presentation of this relation for the calculus used
in the encoding.

$\red \subseteq \pi \times \pi$
$\red : \pi \to \mathcal{P}(\pi)$

\begin{mathpar}
  \inferrule* [lab=Comm] { \textsf{match}( x_{src}, x_{trgt} ) } { x_{trgt}?(y)P \; | \; x_{src}!\langle {Q} \rangle \red P\{\quotep{Q}/y}\} }
  \and \\
  \inferrule* [lab=Par] {{P} \red {P}'} {{{P} | {Q}} \red {{P}' | {Q}}}
  \and
  \inferrule* [lab=Equiv]{{{P} \scong {P}'} \andalso {{P}' \red {Q}'} \andalso {{Q}' \scong {Q}}}{{P} \red {Q}}
\end{mathpar}

\begin{eqnarray*}
  match_{\equiv} (\quotep{P},\quotep{Q}) & := & P \equiv Q \\
  match_{\dagger}(\quotep{P},\quotep{Q}) & := & \forall R. P|Q \red^{*} R => R \red^{*} 0 \\
  match_{K}(\quotep{P},\quotep{Q}) & := & K \mbox{ for some context } K
\end{eqnarray*}

$u?(x)P | u!\langle Q \rangle \red P\{\quotep{Q}/x\}$

%We write $\wred$ for $\red^*$, and $P\red$ if $\exists Q $ such that $ P \red Q$.
We write $P\red$ if $\exists Q $ such that $ P \red Q$ and $P\not\red$, otherwise.

\section{Replication}

As mentioned before, it is known that replication (and hence
recursion) can be implemented in a higher-order process algebra
\cite{SangiorgiWalker}. As our first example of calculation with the
machinery thus far presented we give the construction explicitly in
the {\rhoc}.

\begin{eqnarray}
	D_{x} & := & \prefix{x}{y}{(\binpar{\outputp{x}{y}}{@{y}})} \nonumber\\
	\bangp_{x}{P} & := & \binpar{{x}!\langle{\binpar{D_{x}}{P}}\rangle}{D_{x}} \nonumber
\end{eqnarray}

\begin{eqnarray}
	\bangp_{x}{P} & & \nonumber\\
	=
	& {x}!\langle{(\prefix{x}{y}{(\outputp{x}{y} | @{y})) | P}}\rangle 
	      | \prefix{x}{y}{(\outputp{x}{y} | @{y})} & \nonumber\\
	\red
	& (\outputp{x}{y} | @{y})\substn{\quotep{(\prefix{x}{y}{(@{y} | \outputp{x}{y})) | P}}}{y} & \nonumber\\
	=
	& \outputp{x}{\quotep{(\prefix{x}{y}{(\outputp{x}{y} | @{y})) | P}}}
	  | {(\prefix{x}{y}{(\outputp{x}{y} | @{y})) | P}} & \nonumber\\
	\red
	& \ldots & \nonumber\\
	\red^*
	& P | P | \ldots & \nonumber
\end{eqnarray}

Of course, this encoding, as an implementation, runs away, unfolding
$\bangp{P}$ eagerly. A lazier and more implementable replication
operator, restricted to input-guarded processes, may be obtained as follows.

\begin{eqnarray}
\bangp{\prefix{u}{v}{P}} 
	:= 
	\binpar{\lift{x}{\prefix{u}{v}{(\binpar{D(x)}{P})}}}{D(x)} \nonumber
\end{eqnarray}

\begin{remark}
  Note that the lazier definition still does not deal with summation
  or mixed summation (i.e. sums over input and output). The reader is
  invited to construct definitions of replication that deal with these
  features. 

  Further, the definitions are parameterized in a name, $x$. Can you,
  gentle reader, make a definition that eliminates this parameter and
  guarantees no accidental interaction between the replication
  machinery and the process being replicated -- i.e. no accidental
  sharing of names used by the process to get its work done and the
  name(s) used by the replication to effect copying. This latter
  revision of the definition of replication is crucial to obtaining
  the expected identity $!!P \sim !P$.
\end{remark}

\begin{remark}\label{rem:paradoxical_combinator}
  The reader familiar with the lambda calculus will have noticed the
  similarity between $D$ and the paradoxical combinator.

  [Ed. note: the existence of this seems to suggest we have to be more
  restrictive on the set of processes and names we admit if we are to
  support no-cloning.]
\end{remark}

\subsubsection{Bisimulation}

The computational dynamics gives rise to another kind of equivalence,
the equivalence of computational behavior. As previously mentioned
this is typically captured \emph{via} some form of bisimulation.

% The notion we use in this paper is weak barbed bisimulation
% \cite{milner91polyadicpi}.

The notion we use in this paper is derived from weak barbed
bisimulation \cite{milner91polyadicpi}. 

\begin{definition}
An \emph{observation relation}, $\downarrow_{\mathcal N}$, over a set
of names, $\mathcal N$, is the smallest relation satisfying the rules
below.

\infrule[Out-barb]{y \in {\mathcal N}, \; x \nameeq y}
		  {\outputp{x}{v} \downarrow_{\mathcal N} x}
\infrule[Par-barb]{\mbox{$P\downarrow_{\mathcal N} x$ or $Q\downarrow_{\mathcal N} x$}}
		  {\binpar{P}{Q} \downarrow_{\mathcal N} x}

We write $P \Downarrow_{\mathcal N} x$ if there is $Q$ such that 
$P \wred Q$ and $Q \downarrow_{\mathcal N} x$.
\end{definition}

\begin{definition}
%\label{def.bbisim}
An  ${\mathcal N}$-\emph{barbed bisimulation} over a set of names, ${\mathcal N}$, is a symmetric binary relation 
${\mathcal S}_{\mathcal N}$ between agents such that $P\rel{S}_{\mathcal N}Q$ implies:
\begin{enumerate}
\item If $P \red P'$ then $Q \wred Q'$ and $P'\rel{S}_{\mathcal N} Q'$.
\item If $P\downarrow_{\mathcal N} x$, then $Q\Downarrow_{\mathcal N} x$.
\end{enumerate}
$P$ is ${\mathcal N}$-barbed bisimilar to $Q$, written
$P \wbbisim_{\mathcal N} Q$, if $P \rel{S}_{\mathcal N} Q$ for some ${\mathcal N}$-barbed bisimulation ${\mathcal S}_{\mathcal N}$.
\end{definition}

$\mathcal{R} \subseteq \pi \times \pi$

$P \mathcal{R} Q => \forall P'. P \red P' \Rightarrow \exists Q'. Q \red Q', P' \mathcal{R} Q'$

$P \vdash x \Rightarrow Q \vdash x$

\begin{mathpar}
  \inferrule*[lab=Out-barb]{x \nameeq y}{{y}!\langle{Q}\rangle \vdash x}
  \and
  \inferrule*[lab=Par-barb]{\mbox{$P\vdash x$ or $Q\vdash x$}}{\binpar{P}{Q} \vdash x}
\end{mathpar}

\subsubsection{Contexts}

One of the principle advantages of computational calculi like the
$\pi$-calculus is a well-defined notion of context,
contextual-equivalence and a correlation between
contextual-equivalence and notions of bisimulation. The notion of
context allows the decomposition of a process into (sub-)process and
its syntactic environment, its context. Thus, a context may be
thought of as a process with a ``hole'' (written $\Box$) in it. The
application of a context $M$ to a process $P$, written $M[P]$, is
tantamount to filling the hole in $M$ with $P$. In this paper we do
not need the full weight of this theory, but do make use of the notion
of context in the proof the main theorem. 

\begin{mathpar}
  \inferrule* [lab=summation] {} {{M_{M},M_{N}} \bc \Box \;|\; x.M_{A} \;|\; M_{M}+M_{N}}
  \and
  \inferrule* [lab=agent] {} {{M_{A}} \bc (\vec{x})M_{P} \;| \; \clift{P_0,\ldots,M_{P},\ldots,P_N}}
  \and \\
  \inferrule* [lab=process] {} {{M_{P}} \bc M_{N} \;| \;P|M_{P} }
\end{mathpar} 

\begin{mathpar}
  \inferrule* [lab=sychronization] {} {M_{N} \bc \Box \;|\; x?M_{F} \;|\; x!M_{C}}
  \and
  \inferrule* [lab=abstraction] {} {{M_{F}} \bc (x)M_{P} }
  \and
  \inferrule* [lab=concretion] {} {{M_{C}} \bc \langle M_{P} \rangle }
  \and \\
  \inferrule* [lab=process] {} {{M_{P}} \bc M_{N} \;| \;P|M_{P} }
\end{mathpar}

\begin{definition}[contextual application] Given a context $M$, and
  process $P$, we define the \emph{contextual application}, $M[P] :=
  M\{P/\Box\}$. That is, the contextual application of M to P is the
  substitution of $P$ for $\Box$ in $M$.
\end{definition}

$\meaningof{-} : L \to \mathcal{P}(\pi)$

\begin{mathpar}
  \inferrule* [lab=collection] {} {\meaningof{true} = \pi, \and \meaningof{~E} = \pi \setminus \meaningof{E}, \and \meaningof{E_{1} \& E_{2}} = \meaningof{E_{1}} \cap \meaningof{E_{2}}}
\end{mathpar}

\begin{mathpar}
  \inferrule* [lab=structure] {} {\meaningof{0} = \{ P \in \pi | P \equiv 0 \}, \and \\ \meaningof{E_1 | E_2} = \{ P \in \pi | P \equiv P_{1} | P_{2}, P_{1} \in \meaningof{E_{1}}, P_{2} \in \meaningof{E_2}\} }
\end{mathpar}

\begin{mathpar}
 \inferrule* [lab=behavior] {} {\meaningof{\langle a?b \rangle E} = \{ P \in \pi | P \equiv Q | u?(y)P', \\ \and \\\\ \and \\ \;\;\; u \in \meaningof{a}, \forall z.P'\{z/y\} \in \meaningof{E\{z/b\}}\}, \and \\ \meaningof{a!E} = \{ P \in \pi | P \equiv Q | x!\langle P' \rangle, x \in \meaningof{a} P' \in \meaningof{E}\} }
\end{mathpar}

\begin{mathpar}
 \inferrule* [lab=nominal] {} {\meaningof{\quotep{E}} = \{ \quotep{P} \in \quotep{\pi} | P \in \meaningof{E} \}, \and \meaningof{\quotep{P}} = \{ \quotep{Q} \in \quotep{\pi} | P \equiv Q \} \and \\ \meaningof{@\quotep{E}} = \{ P \in \pi | P \equiv @x, x \in \meaningof{E} \}}
\end{mathpar}

\begin{eqnarray*}
  \\
  \meaningof{-} : TS \to ST
\end{eqnarray*}

\begin{eqnarray*}
  \\
  L : TS \to ST
\end{eqnarray*}

\begin{eqnarray*}
  \\
  P \models E \iff P \in \meaningof{E}
\end{eqnarray*}

\begin{eqnarray*}
  P \approx_{L} Q \iff \forall E \in L. P \models E \iff Q \models E
\end{eqnarray*}

\begin{eqnarray*}
  P \approx_{K} Q
\end{eqnarray*}

\begin{eqnarray*}
  P \approx Q
\end{eqnarray*}

$\approx_{K} = \approx = \approx_{L}$

\subsubsection{Contextual duality}

Note that contexts extend the quotation operation to a family of
operations from processes to names. Given a context, $M$, we can
define a \emph{nominal context}, $\quotep{M}$ by $\quotep{M}[P] :=
\quotep{M[P]}$. To foreshadow what is to come we observe that these
operations enjoy a duality with processes very much like the duality
between vectors and maps from vectors to scalars.

Further, because the calculus is essentially higher-order, we have a
correspondence between contexts and processes. More specifically,
given a name $x$ and a context $M$ we can construct $M^{*}_{x}$ such
that 

\begin{mathpar}
  M^{*}_{x} | \lift{x}{P} \red M[P]
\end{mathpar}

namely,

\begin{mathpar}
  M^{*}_{x} := x?(u).M[\dropn{u}]
\end{mathpar}

The dependence of $M^{*}_{x}$ on a name makes it an abstraction, 

\begin{mathpar}
  M^{*} := (x)x?(u).M[\dropn{u}]
\end{mathpar}

\subsection{Additional notation}

It will sometimes be convenient to denote the process a name
quotes. We already have the notation $x = \quotep{P}$, but it will be
convenient to introduce an alternate notation, $\procn{x}$, when we
want to emphasize the connection to the use of the name. Note that, by
virtue of name equivalence, $\quotep{\procn{x}} \nameeq x$; so, the
notation is consistent with previous definitions.

Further, because names have structure it is possible to effect
substitutions on the basis of that structure. This means we need to
upgrade our notation for substitutions, which we accomplish by
adapting comprehension notation. Thus,

\begin{mathpar}
  P\{ y / x : x \in S \}
\end{mathpar}

is interpreted to mean the process derived from P by replacing (in a
capture-avoiding manner) each occurrence of $x$ in $S$ by $y$. For example,

\begin{mathpar}
  P\{ \quotep{\procn{x}|\procn{x}} / x : x \in \freenames{P} \}
\end{mathpar}

will replace each (occurrence) of a free name $x$ in $P$ by
$\quotep{\procn{x}|\procn{x}}$.

Also, we will avail ourselves of the notation $x^{L}$ and $x^{R}$ to
denote injections of a name into disjoint copies of the name
space. There are numerous ways to accomplish this. One example can be
found in \cite{MeredithR05}. This notation overloads to vectors of
names: $\vec{x}^{\pi} := (x_{i}^{\pi} \; : \; 0 \leq i < |\vec{x}| )$ where $\pi \in \{L,R\}$.

We also use $P^{\Box} := P|\Box$.

In \cite{MeredithR05} an interpretation of the new operator is
given. It turns out that there are several possible interpretations
all enjoying the requisite algebraic properties of the operator (see
\cite{milner91polyadicpi}). We will therefore make liberal use of
$(\nu\; \vec{x})P$.

% subsection the_syntax_and_semantics_of_the_notation_system (end)   

\input{qm2pi.qmops} 

\input{qm2pi.sterngerlach} 

\input{qm2pi.metric} 

% section concurrent_process_calculi (end)

%\input{qm2pi.proofsketch}

% section proof sketch (end)

%\input{qm2pi.slviaknots} 

% section spatial logic via knots (end)

\input{qm2pi.conclusion}

% section conclusion (end)

%\input{qm2pi.dtcodes} 

% section wiring algorithm (end)

\input{qm2pi.ack} 

% section acknowledgments (end)

\newpage


\bibliographystyle{plain}   
\bibliography{../../biblios/main.bib}

\input{qm2pi.rhodetails}

\end{document}

 

% section wiring algorithm (end)

\documentclass[12pt]{llncs}
%\documentclass{jktr}

\usepackage[pdftex]{hyperref}                   
\usepackage {listings}
\usepackage {mathpartir}
\usepackage{bcprules}
%\usepackage{listings}
                       
\usepackage{graphicx} 
%\usepackage[margins=2.5cm,nohead,nofoot]{geometry}
%\usepackage{geometry}
\usepackage{amsfonts}
\usepackage{amstext}
\usepackage{latexsym}
\usepackage{amssymb}
\usepackage{color}


%\include{myPreamble}
\include{qm2pi.local} 

%\ifpdf
%\usepackage[pdftex]{graphicx}
%\else
%\usepackage{graphicx}
%\fi

 % \ifpdf
%  \usepackage{pdfsync}
%  \if


%\title{Brief Article}
%\author{David F. Snyder}
%\author{L.G. Meredith}

%\address{Dept. of Math., Texas State University--San Marcos, San Marcos, TX 78666}
       
\pagestyle{empty}


\begin{document}

\lstset{language=[Objective]Caml,frame=shadowbox}

\input{qm2pi.front}

% section front matter (end)

\input{qm2pi.intro} 
 
% section introduction (end)

% \input{qm2pi.knotations} 

% section notation (end)

\input{qm2pi.process.calculi} 

% section concurrent_process_calculi_and_spatial_logics_ (end)
    
%\input{qm2pi.knots2pi} 

%\input{qm2pi.trefoil} 

%\input{qm2pi.mainthm} 

% subsection basic_interpretation (end)

%\input{qm2pi.rho.presentation} 
\subsection{The syntax and semantics of the notation system}\label{sub:the_syntax_and_semantics_of_the_notation_system} % (fold)

We now summarize a technical presentation of the calculus that
embodies our theory of dynamics. The typical presentation of such a
calculus follows the style of giving generators and relations on
them. The grammar, below, describing term constructors, freely
generates the set of processes, $\Proc$. This set is then quotiented
by a relation known as structural congruence and it is over this set
that the notion of dynamics is expressed. This presentation is
essentially that of \cite{MeredithR05} with the addition of
polyadicity and summation. For readability we have relegated some of
the technical subtleties to an appendix.

\subsubsection{Process grammar}\label{subsub:process_grammar}

\begin{mathpar}
  \inferrule* [lab=synchronization] {} {{M} \bc \pzero \;|\; x?F \;|\; x!C }
  \and
  \inferrule* [lab=abstraction] {} {{F} \bc (x)P}
  \and
  \inferrule* [lab=concretion] {} {{C} \bc \langle Q \rangle}
  \and
  \inferrule* [lab=process] {} {{P,Q} \bc M \;| \;P|Q \;|\; @{x}}
  \and
  \inferrule* [lab=name] {} {{x} \bc \quotep{P}}
\end{mathpar} 

Note that $\vec{x}$ (resp. $\vec{P}$) denotes a vector of names
(resp. processes) of length $|\vec{x}|$ (resp. $|\vec{P}|$). We adopt
the following useful abbreviations.

\begin{mathpar}
   x?(\vec{y}).P := x.(\vec{y})P \and  x\clift{\vec{P}} := x.\clift{\vec{P}}
   \and x!(y) := \lift{x}{\dropn{y}}
   \and \Pi_{i=0}^{n-1}P_i := P_0 | \ldots | P_{n-1}
\end{mathpar}

\subsubsection{Structural congruence}

\paragraph{Free and bound names and alpha-equivalence.} At the
core of structural equivalence is alpha-equivalence which identifies
process that are the same up to a change of variable. Formally, we
recognize the distinction between free and bound names. The free names
of a process, $\freenames{P}$, may be calculated recursively as
follows:

\begin{mathpar}
\freenames{\pzero} := \emptyset
  \and \\
  \freenames{x?(y).P} := \{ x \} \cup (\freenames{P} \setminus \{ y \})
  \and 
  \freenames{x!\langle P \rangle} := \{ x \} \cup \{ P \} 
  \and \\
  \freenames{P|Q} := \freenames{P} \cup \freenames{Q}
  \and \\
  \freenames{@{x}} := \{ x \}
\end{mathpar}

$\pi$
$\quotep{\pi}$

$\freenames{-} : \pi \to \mathcal{P}(\quotep{\pi})$

\begin{eqnarray*}
  \freenames{\pzero} & := & \emptyset \\
  \freenames{x?(y).P} & := & \{ x \} \cup (\freenames{P} \setminus \{ y \}) \\
  \freenames{x!\langle P \rangle} & := & \{ x \} \cup \{ P \} \\
  \freenames{P|Q} & := & \freenames{P} \cup \freenames{Q} \\
  \freenames{\dropn{x}} & := & \{ x \}
\end{eqnarray*}

The bound names of a process, $\boundnames{P}$, are those names occurring in $P$
that are not free. For example, in $x?(y).0$, the name $x$ is free, while $y$ is bound.

\begin{mathpar}
  \inferrule* [lab=monoidal-laws] {} { P|Q \equiv Q|P \and P|0 \equiv P \and P|(Q|R) \equiv (P|Q)|R }
\end{mathpar}

\begin{mathpar}
  \inferrule* [lab=alpha-equivalence] {} { (x)P \equiv (y)P\{y/x\} \and y \not\in \freenames{P} }
\end{mathpar}

\begin{definition}
Then two processes, $P,Q$, are alpha-equivalent if $P = Q\{\vec{y}/\vec{x}\}$ for
some $\vec{x} \in \boundnames{Q},\vec{y} \in \boundnames{P}$, where $Q\{\vec{y}/\vec{x}\}$
denotes the capture-avoiding substitution of $\vec{y}$ for $\vec{x}$ in $Q$.
\end{definition}

\begin{definition}
  The {\em structural congruence} \cite{SangiorgiWalker} , $\equiv$,
  between processes is the least congruence containing
  alpha-equivalence, satisfying the abelian monoid laws
  (associativity, commutativity and $\pzero$ as identity) for parallel
  composition $|$ and for summation $+$.
\end{definition}

\subsection{Name equivalence}

We take name equivalence, written $\nameeq$, to be the smallest
equivalence relation generated by the following rules.

\begin{mathpar}
\inferrule*[lab=Quote-drop]
{ }
{ \quotep{@{x}} \nameeq x }

\inferrule*[lab=Struct-equiv]
{ P \scong Q }
{ \quotep{P} \nameeq \quotep{Q} }
\end{mathpar}

The astute reader will have noticed that the mutual recursion of names
and processes imposes a mutual recursion on alpha-equivalence and
structural equivalence via name-equivalence. Fortunately, all of this
works out pleasantly and we may calculate in the natural way, free of
concern. The reader interested in the details is referred to the
appendix \ref{appendix:rho_details}.

\subsection{Substitution}

We use $\Proc$ for the set of processes, $\QProc$ for the set of
names, and $\id{\{}\vec{y} / \vec{x} \id{\}}$ to denote partial maps,
$s : \QProc \rightarrow \QProc$. A map, $s$ lifts, uniquely, to a map
on process terms, $\widehat{s} : \Proc \rightarrow \Proc$ by the
following equations.

\begin{mathpar}
  (0) \psubstp{Q}{P} := 0 \\
  (R \juxtap S) \psubstp{Q}{P}
  :=    
  (R)\psubstp{Q}{P} \juxtap (S) \psubstp{Q}{P} \\
  (x?(y).R) \psubstp{Q}{P}    
  :=    
  (x)\substp{Q}{P} (z)\concat( (R \psubstn{z}{y}) \psubstp{Q}{P} ) \\
  (\lift{x}{R}) \psubstp{Q}{P}  
  :=
  \lift{(x)\substp{Q}{P}}{ R \psubstp{Q}{P} } \\
%   (\dropn{x})  \psubstp{Q}{P}       
%   := 
%   \left\{ 
%     \begin{array}{ccc} 
%       \dropn{\quotep{Q}} & & x \nameeq \quotep{P} \\
%       \dropn{x} & & otherwise \\
%     \end{array}
%   \right. 
  (\dropn{x})  \psubstp{Q}{P}       
  := 
  \left\{ 
    \begin{array}{ccc} 
      Q & & x \nameeq \quotep{P} \\
      \dropn{x} & & otherwise \\
    \end{array}
  \right.
\end{mathpar}
 

where

\begin{eqnarray}
  (x)\id{\{} \lpquote Q \rpquote / \lpquote P \rpquote \id{\}}            = 
  \left\{ 
    \begin{array}{ccc}
      \lpquote Q \rpquote & & x \nameeq \lpquote P \rpquote \\
      x & & otherwise \\
    \end{array}
  \right. \nonumber
\end{eqnarray}

and $z$ is chosen distinct from $\quotep{P}$, $\quotep{Q}$, the free
names in $Q$, and all the names in $R$. Our $\alpha$-equivalence will
be built in the standard way from this substitution.

\begin{remark}\label{rem:no_self_referential_names}
  One consequence of these definitions is that $\forall P. \quotep{P}
  \not\in \freenames{P}$.
\end{remark}

\subsection{ Dynamic quote: an example }

Anticipating something of what's to come, consider applying the
substitution, $\widehat{\id{\{}u / z \id{\}}}$, to the following pair
of processes, $\lift{w}{y!(z)}$ and $w[ \lpquote y!(z) \rpquote ]$.

\begin{eqnarray}
	\lift{w}{y!(z)}\widehat{\id{\{}u / z \id{\}}}
		& = &
		\lift{w}{y!(u)} \nonumber\\
	w[ \lpquote y!(z) \rpquote ] \widehat{ \id{\{}u / z \id{\}} }
		& = &
		w[ \lpquote y!(z) \rpquote ] \nonumber
\end{eqnarray}

Because the body of the process between quotes is impervious to
substitution, we get radically different answers. In fact, by
examining the first process in an input context,
e.g. $x?(z).\lift{w}{y!(z)}$, we see that the process under the lift
operator may be shaped by prefixed inputs binding a name inside it. In
this sense, the lift operator will be seen as a way to dynamically
construct processes before reifying them as names.

Finally equipped with these standard features we can present the
dynamics of the calculus.

\subsubsection{Operational semantics} 

Finally, we introduce the computational dynamics. What marks these
algebras as distinct from other more traditionally studied algebraic
structures, e.g. vector spaces or polynomial rings, is the manner in
which dynamics is captured. In traditional structures, dynamics is typically
expressed through morphisms between such structures, as in linear maps
between vector spaces or morphisms between rings. In algebras
associated with the semantics of computation, the dynamics is
expressed as part of the algebraic structure itself, through a
reduction reduction relation typically denoted by $\red$. Below, we
give a recursive presentation of this relation for the calculus used
in the encoding.

$\red \subseteq \pi \times \pi$
$\red : \pi \to \mathcal{P}(\pi)$

\begin{mathpar}
  \inferrule* [lab=Comm] { \textsf{match}( x_{src}, x_{trgt} ) } { x_{trgt}?(y)P \; | \; x_{src}!\langle {Q} \rangle \red P\{\quotep{Q}/y}\} }
  \and \\
  \inferrule* [lab=Par] {{P} \red {P}'} {{{P} | {Q}} \red {{P}' | {Q}}}
  \and
  \inferrule* [lab=Equiv]{{{P} \scong {P}'} \andalso {{P}' \red {Q}'} \andalso {{Q}' \scong {Q}}}{{P} \red {Q}}
\end{mathpar}

\begin{eqnarray*}
  match_{\equiv} (\quotep{P},\quotep{Q}) & := & P \equiv Q \\
  match_{\dagger}(\quotep{P},\quotep{Q}) & := & \forall R. P|Q \red^{*} R => R \red^{*} 0 \\
  match_{K}(\quotep{P},\quotep{Q}) & := & K \mbox{ for some context } K
\end{eqnarray*}

$u?(x)P | u!\langle Q \rangle \red P\{\quotep{Q}/x\}$

%We write $\wred$ for $\red^*$, and $P\red$ if $\exists Q $ such that $ P \red Q$.
We write $P\red$ if $\exists Q $ such that $ P \red Q$ and $P\not\red$, otherwise.

\section{Replication}

As mentioned before, it is known that replication (and hence
recursion) can be implemented in a higher-order process algebra
\cite{SangiorgiWalker}. As our first example of calculation with the
machinery thus far presented we give the construction explicitly in
the {\rhoc}.

\begin{eqnarray}
	D_{x} & := & \prefix{x}{y}{(\binpar{\outputp{x}{y}}{@{y}})} \nonumber\\
	\bangp_{x}{P} & := & \binpar{{x}!\langle{\binpar{D_{x}}{P}}\rangle}{D_{x}} \nonumber
\end{eqnarray}

\begin{eqnarray}
	\bangp_{x}{P} & & \nonumber\\
	=
	& {x}!\langle{(\prefix{x}{y}{(\outputp{x}{y} | @{y})) | P}}\rangle 
	      | \prefix{x}{y}{(\outputp{x}{y} | @{y})} & \nonumber\\
	\red
	& (\outputp{x}{y} | @{y})\substn{\quotep{(\prefix{x}{y}{(@{y} | \outputp{x}{y})) | P}}}{y} & \nonumber\\
	=
	& \outputp{x}{\quotep{(\prefix{x}{y}{(\outputp{x}{y} | @{y})) | P}}}
	  | {(\prefix{x}{y}{(\outputp{x}{y} | @{y})) | P}} & \nonumber\\
	\red
	& \ldots & \nonumber\\
	\red^*
	& P | P | \ldots & \nonumber
\end{eqnarray}

Of course, this encoding, as an implementation, runs away, unfolding
$\bangp{P}$ eagerly. A lazier and more implementable replication
operator, restricted to input-guarded processes, may be obtained as follows.

\begin{eqnarray}
\bangp{\prefix{u}{v}{P}} 
	:= 
	\binpar{\lift{x}{\prefix{u}{v}{(\binpar{D(x)}{P})}}}{D(x)} \nonumber
\end{eqnarray}

\begin{remark}
  Note that the lazier definition still does not deal with summation
  or mixed summation (i.e. sums over input and output). The reader is
  invited to construct definitions of replication that deal with these
  features. 

  Further, the definitions are parameterized in a name, $x$. Can you,
  gentle reader, make a definition that eliminates this parameter and
  guarantees no accidental interaction between the replication
  machinery and the process being replicated -- i.e. no accidental
  sharing of names used by the process to get its work done and the
  name(s) used by the replication to effect copying. This latter
  revision of the definition of replication is crucial to obtaining
  the expected identity $!!P \sim !P$.
\end{remark}

\begin{remark}\label{rem:paradoxical_combinator}
  The reader familiar with the lambda calculus will have noticed the
  similarity between $D$ and the paradoxical combinator.

  [Ed. note: the existence of this seems to suggest we have to be more
  restrictive on the set of processes and names we admit if we are to
  support no-cloning.]
\end{remark}

\subsubsection{Bisimulation}

The computational dynamics gives rise to another kind of equivalence,
the equivalence of computational behavior. As previously mentioned
this is typically captured \emph{via} some form of bisimulation.

% The notion we use in this paper is weak barbed bisimulation
% \cite{milner91polyadicpi}.

The notion we use in this paper is derived from weak barbed
bisimulation \cite{milner91polyadicpi}. 

\begin{definition}
An \emph{observation relation}, $\downarrow_{\mathcal N}$, over a set
of names, $\mathcal N$, is the smallest relation satisfying the rules
below.

\infrule[Out-barb]{y \in {\mathcal N}, \; x \nameeq y}
		  {\outputp{x}{v} \downarrow_{\mathcal N} x}
\infrule[Par-barb]{\mbox{$P\downarrow_{\mathcal N} x$ or $Q\downarrow_{\mathcal N} x$}}
		  {\binpar{P}{Q} \downarrow_{\mathcal N} x}

We write $P \Downarrow_{\mathcal N} x$ if there is $Q$ such that 
$P \wred Q$ and $Q \downarrow_{\mathcal N} x$.
\end{definition}

\begin{definition}
%\label{def.bbisim}
An  ${\mathcal N}$-\emph{barbed bisimulation} over a set of names, ${\mathcal N}$, is a symmetric binary relation 
${\mathcal S}_{\mathcal N}$ between agents such that $P\rel{S}_{\mathcal N}Q$ implies:
\begin{enumerate}
\item If $P \red P'$ then $Q \wred Q'$ and $P'\rel{S}_{\mathcal N} Q'$.
\item If $P\downarrow_{\mathcal N} x$, then $Q\Downarrow_{\mathcal N} x$.
\end{enumerate}
$P$ is ${\mathcal N}$-barbed bisimilar to $Q$, written
$P \wbbisim_{\mathcal N} Q$, if $P \rel{S}_{\mathcal N} Q$ for some ${\mathcal N}$-barbed bisimulation ${\mathcal S}_{\mathcal N}$.
\end{definition}

$\mathcal{R} \subseteq \pi \times \pi$

$P \mathcal{R} Q => \forall P'. P \red P' \Rightarrow \exists Q'. Q \red Q', P' \mathcal{R} Q'$

$P \vdash x \Rightarrow Q \vdash x$

\begin{mathpar}
  \inferrule*[lab=Out-barb]{x \nameeq y}{{y}!\langle{Q}\rangle \vdash x}
  \and
  \inferrule*[lab=Par-barb]{\mbox{$P\vdash x$ or $Q\vdash x$}}{\binpar{P}{Q} \vdash x}
\end{mathpar}

\subsubsection{Contexts}

One of the principle advantages of computational calculi like the
$\pi$-calculus is a well-defined notion of context,
contextual-equivalence and a correlation between
contextual-equivalence and notions of bisimulation. The notion of
context allows the decomposition of a process into (sub-)process and
its syntactic environment, its context. Thus, a context may be
thought of as a process with a ``hole'' (written $\Box$) in it. The
application of a context $M$ to a process $P$, written $M[P]$, is
tantamount to filling the hole in $M$ with $P$. In this paper we do
not need the full weight of this theory, but do make use of the notion
of context in the proof the main theorem. 

\begin{mathpar}
  \inferrule* [lab=summation] {} {{M_{M},M_{N}} \bc \Box \;|\; x.M_{A} \;|\; M_{M}+M_{N}}
  \and
  \inferrule* [lab=agent] {} {{M_{A}} \bc (\vec{x})M_{P} \;| \; \clift{P_0,\ldots,M_{P},\ldots,P_N}}
  \and \\
  \inferrule* [lab=process] {} {{M_{P}} \bc M_{N} \;| \;P|M_{P} }
\end{mathpar} 

\begin{mathpar}
  \inferrule* [lab=sychronization] {} {M_{N} \bc \Box \;|\; x?M_{F} \;|\; x!M_{C}}
  \and
  \inferrule* [lab=abstraction] {} {{M_{F}} \bc (x)M_{P} }
  \and
  \inferrule* [lab=concretion] {} {{M_{C}} \bc \langle M_{P} \rangle }
  \and \\
  \inferrule* [lab=process] {} {{M_{P}} \bc M_{N} \;| \;P|M_{P} }
\end{mathpar}

\begin{definition}[contextual application] Given a context $M$, and
  process $P$, we define the \emph{contextual application}, $M[P] :=
  M\{P/\Box\}$. That is, the contextual application of M to P is the
  substitution of $P$ for $\Box$ in $M$.
\end{definition}

$\meaningof{-} : L \to \mathcal{P}(\pi)$

\begin{mathpar}
  \inferrule* [lab=collection] {} {\meaningof{true} = \pi, \and \meaningof{~E} = \pi \setminus \meaningof{E}, \and \meaningof{E_{1} \& E_{2}} = \meaningof{E_{1}} \cap \meaningof{E_{2}}}
\end{mathpar}

\begin{mathpar}
  \inferrule* [lab=structure] {} {\meaningof{0} = \{ P \in \pi | P \equiv 0 \}, \and \\ \meaningof{E_1 | E_2} = \{ P \in \pi | P \equiv P_{1} | P_{2}, P_{1} \in \meaningof{E_{1}}, P_{2} \in \meaningof{E_2}\} }
\end{mathpar}

\begin{mathpar}
 \inferrule* [lab=behavior] {} {\meaningof{\langle a?b \rangle E} = \{ P \in \pi | P \equiv Q | u?(y)P', \\ \and \\\\ \and \\ \;\;\; u \in \meaningof{a}, \forall z.P'\{z/y\} \in \meaningof{E\{z/b\}}\}, \and \\ \meaningof{a!E} = \{ P \in \pi | P \equiv Q | x!\langle P' \rangle, x \in \meaningof{a} P' \in \meaningof{E}\} }
\end{mathpar}

\begin{mathpar}
 \inferrule* [lab=nominal] {} {\meaningof{\quotep{E}} = \{ \quotep{P} \in \quotep{\pi} | P \in \meaningof{E} \}, \and \meaningof{\quotep{P}} = \{ \quotep{Q} \in \quotep{\pi} | P \equiv Q \} \and \\ \meaningof{@\quotep{E}} = \{ P \in \pi | P \equiv @x, x \in \meaningof{E} \}}
\end{mathpar}

\begin{eqnarray*}
  \\
  \meaningof{-} : TS \to ST
\end{eqnarray*}

\begin{eqnarray*}
  \\
  L : TS \to ST
\end{eqnarray*}

\begin{eqnarray*}
  \\
  P \models E \iff P \in \meaningof{E}
\end{eqnarray*}

\begin{eqnarray*}
  P \approx_{L} Q \iff \forall E \in L. P \models E \iff Q \models E
\end{eqnarray*}

\begin{eqnarray*}
  P \approx_{K} Q
\end{eqnarray*}

\begin{eqnarray*}
  P \approx Q
\end{eqnarray*}

$\approx_{K} = \approx = \approx_{L}$

\subsubsection{Contextual duality}

Note that contexts extend the quotation operation to a family of
operations from processes to names. Given a context, $M$, we can
define a \emph{nominal context}, $\quotep{M}$ by $\quotep{M}[P] :=
\quotep{M[P]}$. To foreshadow what is to come we observe that these
operations enjoy a duality with processes very much like the duality
between vectors and maps from vectors to scalars.

Further, because the calculus is essentially higher-order, we have a
correspondence between contexts and processes. More specifically,
given a name $x$ and a context $M$ we can construct $M^{*}_{x}$ such
that 

\begin{mathpar}
  M^{*}_{x} | \lift{x}{P} \red M[P]
\end{mathpar}

namely,

\begin{mathpar}
  M^{*}_{x} := x?(u).M[\dropn{u}]
\end{mathpar}

The dependence of $M^{*}_{x}$ on a name makes it an abstraction, 

\begin{mathpar}
  M^{*} := (x)x?(u).M[\dropn{u}]
\end{mathpar}

\subsection{Additional notation}

It will sometimes be convenient to denote the process a name
quotes. We already have the notation $x = \quotep{P}$, but it will be
convenient to introduce an alternate notation, $\procn{x}$, when we
want to emphasize the connection to the use of the name. Note that, by
virtue of name equivalence, $\quotep{\procn{x}} \nameeq x$; so, the
notation is consistent with previous definitions.

Further, because names have structure it is possible to effect
substitutions on the basis of that structure. This means we need to
upgrade our notation for substitutions, which we accomplish by
adapting comprehension notation. Thus,

\begin{mathpar}
  P\{ y / x : x \in S \}
\end{mathpar}

is interpreted to mean the process derived from P by replacing (in a
capture-avoiding manner) each occurrence of $x$ in $S$ by $y$. For example,

\begin{mathpar}
  P\{ \quotep{\procn{x}|\procn{x}} / x : x \in \freenames{P} \}
\end{mathpar}

will replace each (occurrence) of a free name $x$ in $P$ by
$\quotep{\procn{x}|\procn{x}}$.

Also, we will avail ourselves of the notation $x^{L}$ and $x^{R}$ to
denote injections of a name into disjoint copies of the name
space. There are numerous ways to accomplish this. One example can be
found in \cite{MeredithR05}. This notation overloads to vectors of
names: $\vec{x}^{\pi} := (x_{i}^{\pi} \; : \; 0 \leq i < |\vec{x}| )$ where $\pi \in \{L,R\}$.

We also use $P^{\Box} := P|\Box$.

In \cite{MeredithR05} an interpretation of the new operator is
given. It turns out that there are several possible interpretations
all enjoying the requisite algebraic properties of the operator (see
\cite{milner91polyadicpi}). We will therefore make liberal use of
$(\nu\; \vec{x})P$.

% subsection the_syntax_and_semantics_of_the_notation_system (end)   

\input{qm2pi.qmops} 

\input{qm2pi.sterngerlach} 

\input{qm2pi.metric} 

% section concurrent_process_calculi (end)

%\input{qm2pi.proofsketch}

% section proof sketch (end)

%\input{qm2pi.slviaknots} 

% section spatial logic via knots (end)

\input{qm2pi.conclusion}

% section conclusion (end)

%\input{qm2pi.dtcodes} 

% section wiring algorithm (end)

\input{qm2pi.ack} 

% section acknowledgments (end)

\newpage


\bibliographystyle{plain}   
\bibliography{../../biblios/main.bib}

\input{qm2pi.rhodetails}

\end{document}

 

% section acknowledgments (end)

\newpage


\bibliographystyle{plain}   
\bibliography{../../biblios/main.bib}

\documentclass[12pt]{llncs}
%\documentclass{jktr}

\usepackage[pdftex]{hyperref}                   
\usepackage {listings}
\usepackage {mathpartir}
\usepackage{bcprules}
%\usepackage{listings}
                       
\usepackage{graphicx} 
%\usepackage[margins=2.5cm,nohead,nofoot]{geometry}
%\usepackage{geometry}
\usepackage{amsfonts}
\usepackage{amstext}
\usepackage{latexsym}
\usepackage{amssymb}
\usepackage{color}


%\include{myPreamble}
\include{qm2pi.local} 

%\ifpdf
%\usepackage[pdftex]{graphicx}
%\else
%\usepackage{graphicx}
%\fi

 % \ifpdf
%  \usepackage{pdfsync}
%  \if


%\title{Brief Article}
%\author{David F. Snyder}
%\author{L.G. Meredith}

%\address{Dept. of Math., Texas State University--San Marcos, San Marcos, TX 78666}
       
\pagestyle{empty}


\begin{document}

\lstset{language=[Objective]Caml,frame=shadowbox}

\input{qm2pi.front}

% section front matter (end)

\input{qm2pi.intro} 
 
% section introduction (end)

% \input{qm2pi.knotations} 

% section notation (end)

\input{qm2pi.process.calculi} 

% section concurrent_process_calculi_and_spatial_logics_ (end)
    
%\input{qm2pi.knots2pi} 

%\input{qm2pi.trefoil} 

%\input{qm2pi.mainthm} 

% subsection basic_interpretation (end)

%\input{qm2pi.rho.presentation} 
\subsection{The syntax and semantics of the notation system}\label{sub:the_syntax_and_semantics_of_the_notation_system} % (fold)

We now summarize a technical presentation of the calculus that
embodies our theory of dynamics. The typical presentation of such a
calculus follows the style of giving generators and relations on
them. The grammar, below, describing term constructors, freely
generates the set of processes, $\Proc$. This set is then quotiented
by a relation known as structural congruence and it is over this set
that the notion of dynamics is expressed. This presentation is
essentially that of \cite{MeredithR05} with the addition of
polyadicity and summation. For readability we have relegated some of
the technical subtleties to an appendix.

\subsubsection{Process grammar}\label{subsub:process_grammar}

\begin{mathpar}
  \inferrule* [lab=synchronization] {} {{M} \bc \pzero \;|\; x?F \;|\; x!C }
  \and
  \inferrule* [lab=abstraction] {} {{F} \bc (x)P}
  \and
  \inferrule* [lab=concretion] {} {{C} \bc \langle Q \rangle}
  \and
  \inferrule* [lab=process] {} {{P,Q} \bc M \;| \;P|Q \;|\; @{x}}
  \and
  \inferrule* [lab=name] {} {{x} \bc \quotep{P}}
\end{mathpar} 

Note that $\vec{x}$ (resp. $\vec{P}$) denotes a vector of names
(resp. processes) of length $|\vec{x}|$ (resp. $|\vec{P}|$). We adopt
the following useful abbreviations.

\begin{mathpar}
   x?(\vec{y}).P := x.(\vec{y})P \and  x\clift{\vec{P}} := x.\clift{\vec{P}}
   \and x!(y) := \lift{x}{\dropn{y}}
   \and \Pi_{i=0}^{n-1}P_i := P_0 | \ldots | P_{n-1}
\end{mathpar}

\subsubsection{Structural congruence}

\paragraph{Free and bound names and alpha-equivalence.} At the
core of structural equivalence is alpha-equivalence which identifies
process that are the same up to a change of variable. Formally, we
recognize the distinction between free and bound names. The free names
of a process, $\freenames{P}$, may be calculated recursively as
follows:

\begin{mathpar}
\freenames{\pzero} := \emptyset
  \and \\
  \freenames{x?(y).P} := \{ x \} \cup (\freenames{P} \setminus \{ y \})
  \and 
  \freenames{x!\langle P \rangle} := \{ x \} \cup \{ P \} 
  \and \\
  \freenames{P|Q} := \freenames{P} \cup \freenames{Q}
  \and \\
  \freenames{@{x}} := \{ x \}
\end{mathpar}

$\pi$
$\quotep{\pi}$

$\freenames{-} : \pi \to \mathcal{P}(\quotep{\pi})$

\begin{eqnarray*}
  \freenames{\pzero} & := & \emptyset \\
  \freenames{x?(y).P} & := & \{ x \} \cup (\freenames{P} \setminus \{ y \}) \\
  \freenames{x!\langle P \rangle} & := & \{ x \} \cup \{ P \} \\
  \freenames{P|Q} & := & \freenames{P} \cup \freenames{Q} \\
  \freenames{\dropn{x}} & := & \{ x \}
\end{eqnarray*}

The bound names of a process, $\boundnames{P}$, are those names occurring in $P$
that are not free. For example, in $x?(y).0$, the name $x$ is free, while $y$ is bound.

\begin{mathpar}
  \inferrule* [lab=monoidal-laws] {} { P|Q \equiv Q|P \and P|0 \equiv P \and P|(Q|R) \equiv (P|Q)|R }
\end{mathpar}

\begin{mathpar}
  \inferrule* [lab=alpha-equivalence] {} { (x)P \equiv (y)P\{y/x\} \and y \not\in \freenames{P} }
\end{mathpar}

\begin{definition}
Then two processes, $P,Q$, are alpha-equivalent if $P = Q\{\vec{y}/\vec{x}\}$ for
some $\vec{x} \in \boundnames{Q},\vec{y} \in \boundnames{P}$, where $Q\{\vec{y}/\vec{x}\}$
denotes the capture-avoiding substitution of $\vec{y}$ for $\vec{x}$ in $Q$.
\end{definition}

\begin{definition}
  The {\em structural congruence} \cite{SangiorgiWalker} , $\equiv$,
  between processes is the least congruence containing
  alpha-equivalence, satisfying the abelian monoid laws
  (associativity, commutativity and $\pzero$ as identity) for parallel
  composition $|$ and for summation $+$.
\end{definition}

\subsection{Name equivalence}

We take name equivalence, written $\nameeq$, to be the smallest
equivalence relation generated by the following rules.

\begin{mathpar}
\inferrule*[lab=Quote-drop]
{ }
{ \quotep{@{x}} \nameeq x }

\inferrule*[lab=Struct-equiv]
{ P \scong Q }
{ \quotep{P} \nameeq \quotep{Q} }
\end{mathpar}

The astute reader will have noticed that the mutual recursion of names
and processes imposes a mutual recursion on alpha-equivalence and
structural equivalence via name-equivalence. Fortunately, all of this
works out pleasantly and we may calculate in the natural way, free of
concern. The reader interested in the details is referred to the
appendix \ref{appendix:rho_details}.

\subsection{Substitution}

We use $\Proc$ for the set of processes, $\QProc$ for the set of
names, and $\id{\{}\vec{y} / \vec{x} \id{\}}$ to denote partial maps,
$s : \QProc \rightarrow \QProc$. A map, $s$ lifts, uniquely, to a map
on process terms, $\widehat{s} : \Proc \rightarrow \Proc$ by the
following equations.

\begin{mathpar}
  (0) \psubstp{Q}{P} := 0 \\
  (R \juxtap S) \psubstp{Q}{P}
  :=    
  (R)\psubstp{Q}{P} \juxtap (S) \psubstp{Q}{P} \\
  (x?(y).R) \psubstp{Q}{P}    
  :=    
  (x)\substp{Q}{P} (z)\concat( (R \psubstn{z}{y}) \psubstp{Q}{P} ) \\
  (\lift{x}{R}) \psubstp{Q}{P}  
  :=
  \lift{(x)\substp{Q}{P}}{ R \psubstp{Q}{P} } \\
%   (\dropn{x})  \psubstp{Q}{P}       
%   := 
%   \left\{ 
%     \begin{array}{ccc} 
%       \dropn{\quotep{Q}} & & x \nameeq \quotep{P} \\
%       \dropn{x} & & otherwise \\
%     \end{array}
%   \right. 
  (\dropn{x})  \psubstp{Q}{P}       
  := 
  \left\{ 
    \begin{array}{ccc} 
      Q & & x \nameeq \quotep{P} \\
      \dropn{x} & & otherwise \\
    \end{array}
  \right.
\end{mathpar}
 

where

\begin{eqnarray}
  (x)\id{\{} \lpquote Q \rpquote / \lpquote P \rpquote \id{\}}            = 
  \left\{ 
    \begin{array}{ccc}
      \lpquote Q \rpquote & & x \nameeq \lpquote P \rpquote \\
      x & & otherwise \\
    \end{array}
  \right. \nonumber
\end{eqnarray}

and $z$ is chosen distinct from $\quotep{P}$, $\quotep{Q}$, the free
names in $Q$, and all the names in $R$. Our $\alpha$-equivalence will
be built in the standard way from this substitution.

\begin{remark}\label{rem:no_self_referential_names}
  One consequence of these definitions is that $\forall P. \quotep{P}
  \not\in \freenames{P}$.
\end{remark}

\subsection{ Dynamic quote: an example }

Anticipating something of what's to come, consider applying the
substitution, $\widehat{\id{\{}u / z \id{\}}}$, to the following pair
of processes, $\lift{w}{y!(z)}$ and $w[ \lpquote y!(z) \rpquote ]$.

\begin{eqnarray}
	\lift{w}{y!(z)}\widehat{\id{\{}u / z \id{\}}}
		& = &
		\lift{w}{y!(u)} \nonumber\\
	w[ \lpquote y!(z) \rpquote ] \widehat{ \id{\{}u / z \id{\}} }
		& = &
		w[ \lpquote y!(z) \rpquote ] \nonumber
\end{eqnarray}

Because the body of the process between quotes is impervious to
substitution, we get radically different answers. In fact, by
examining the first process in an input context,
e.g. $x?(z).\lift{w}{y!(z)}$, we see that the process under the lift
operator may be shaped by prefixed inputs binding a name inside it. In
this sense, the lift operator will be seen as a way to dynamically
construct processes before reifying them as names.

Finally equipped with these standard features we can present the
dynamics of the calculus.

\subsubsection{Operational semantics} 

Finally, we introduce the computational dynamics. What marks these
algebras as distinct from other more traditionally studied algebraic
structures, e.g. vector spaces or polynomial rings, is the manner in
which dynamics is captured. In traditional structures, dynamics is typically
expressed through morphisms between such structures, as in linear maps
between vector spaces or morphisms between rings. In algebras
associated with the semantics of computation, the dynamics is
expressed as part of the algebraic structure itself, through a
reduction reduction relation typically denoted by $\red$. Below, we
give a recursive presentation of this relation for the calculus used
in the encoding.

$\red \subseteq \pi \times \pi$
$\red : \pi \to \mathcal{P}(\pi)$

\begin{mathpar}
  \inferrule* [lab=Comm] { \textsf{match}( x_{src}, x_{trgt} ) } { x_{trgt}?(y)P \; | \; x_{src}!\langle {Q} \rangle \red P\{\quotep{Q}/y}\} }
  \and \\
  \inferrule* [lab=Par] {{P} \red {P}'} {{{P} | {Q}} \red {{P}' | {Q}}}
  \and
  \inferrule* [lab=Equiv]{{{P} \scong {P}'} \andalso {{P}' \red {Q}'} \andalso {{Q}' \scong {Q}}}{{P} \red {Q}}
\end{mathpar}

\begin{eqnarray*}
  match_{\equiv} (\quotep{P},\quotep{Q}) & := & P \equiv Q \\
  match_{\dagger}(\quotep{P},\quotep{Q}) & := & \forall R. P|Q \red^{*} R => R \red^{*} 0 \\
  match_{K}(\quotep{P},\quotep{Q}) & := & K \mbox{ for some context } K
\end{eqnarray*}

$u?(x)P | u!\langle Q \rangle \red P\{\quotep{Q}/x\}$

%We write $\wred$ for $\red^*$, and $P\red$ if $\exists Q $ such that $ P \red Q$.
We write $P\red$ if $\exists Q $ such that $ P \red Q$ and $P\not\red$, otherwise.

\section{Replication}

As mentioned before, it is known that replication (and hence
recursion) can be implemented in a higher-order process algebra
\cite{SangiorgiWalker}. As our first example of calculation with the
machinery thus far presented we give the construction explicitly in
the {\rhoc}.

\begin{eqnarray}
	D_{x} & := & \prefix{x}{y}{(\binpar{\outputp{x}{y}}{@{y}})} \nonumber\\
	\bangp_{x}{P} & := & \binpar{{x}!\langle{\binpar{D_{x}}{P}}\rangle}{D_{x}} \nonumber
\end{eqnarray}

\begin{eqnarray}
	\bangp_{x}{P} & & \nonumber\\
	=
	& {x}!\langle{(\prefix{x}{y}{(\outputp{x}{y} | @{y})) | P}}\rangle 
	      | \prefix{x}{y}{(\outputp{x}{y} | @{y})} & \nonumber\\
	\red
	& (\outputp{x}{y} | @{y})\substn{\quotep{(\prefix{x}{y}{(@{y} | \outputp{x}{y})) | P}}}{y} & \nonumber\\
	=
	& \outputp{x}{\quotep{(\prefix{x}{y}{(\outputp{x}{y} | @{y})) | P}}}
	  | {(\prefix{x}{y}{(\outputp{x}{y} | @{y})) | P}} & \nonumber\\
	\red
	& \ldots & \nonumber\\
	\red^*
	& P | P | \ldots & \nonumber
\end{eqnarray}

Of course, this encoding, as an implementation, runs away, unfolding
$\bangp{P}$ eagerly. A lazier and more implementable replication
operator, restricted to input-guarded processes, may be obtained as follows.

\begin{eqnarray}
\bangp{\prefix{u}{v}{P}} 
	:= 
	\binpar{\lift{x}{\prefix{u}{v}{(\binpar{D(x)}{P})}}}{D(x)} \nonumber
\end{eqnarray}

\begin{remark}
  Note that the lazier definition still does not deal with summation
  or mixed summation (i.e. sums over input and output). The reader is
  invited to construct definitions of replication that deal with these
  features. 

  Further, the definitions are parameterized in a name, $x$. Can you,
  gentle reader, make a definition that eliminates this parameter and
  guarantees no accidental interaction between the replication
  machinery and the process being replicated -- i.e. no accidental
  sharing of names used by the process to get its work done and the
  name(s) used by the replication to effect copying. This latter
  revision of the definition of replication is crucial to obtaining
  the expected identity $!!P \sim !P$.
\end{remark}

\begin{remark}\label{rem:paradoxical_combinator}
  The reader familiar with the lambda calculus will have noticed the
  similarity between $D$ and the paradoxical combinator.

  [Ed. note: the existence of this seems to suggest we have to be more
  restrictive on the set of processes and names we admit if we are to
  support no-cloning.]
\end{remark}

\subsubsection{Bisimulation}

The computational dynamics gives rise to another kind of equivalence,
the equivalence of computational behavior. As previously mentioned
this is typically captured \emph{via} some form of bisimulation.

% The notion we use in this paper is weak barbed bisimulation
% \cite{milner91polyadicpi}.

The notion we use in this paper is derived from weak barbed
bisimulation \cite{milner91polyadicpi}. 

\begin{definition}
An \emph{observation relation}, $\downarrow_{\mathcal N}$, over a set
of names, $\mathcal N$, is the smallest relation satisfying the rules
below.

\infrule[Out-barb]{y \in {\mathcal N}, \; x \nameeq y}
		  {\outputp{x}{v} \downarrow_{\mathcal N} x}
\infrule[Par-barb]{\mbox{$P\downarrow_{\mathcal N} x$ or $Q\downarrow_{\mathcal N} x$}}
		  {\binpar{P}{Q} \downarrow_{\mathcal N} x}

We write $P \Downarrow_{\mathcal N} x$ if there is $Q$ such that 
$P \wred Q$ and $Q \downarrow_{\mathcal N} x$.
\end{definition}

\begin{definition}
%\label{def.bbisim}
An  ${\mathcal N}$-\emph{barbed bisimulation} over a set of names, ${\mathcal N}$, is a symmetric binary relation 
${\mathcal S}_{\mathcal N}$ between agents such that $P\rel{S}_{\mathcal N}Q$ implies:
\begin{enumerate}
\item If $P \red P'$ then $Q \wred Q'$ and $P'\rel{S}_{\mathcal N} Q'$.
\item If $P\downarrow_{\mathcal N} x$, then $Q\Downarrow_{\mathcal N} x$.
\end{enumerate}
$P$ is ${\mathcal N}$-barbed bisimilar to $Q$, written
$P \wbbisim_{\mathcal N} Q$, if $P \rel{S}_{\mathcal N} Q$ for some ${\mathcal N}$-barbed bisimulation ${\mathcal S}_{\mathcal N}$.
\end{definition}

$\mathcal{R} \subseteq \pi \times \pi$

$P \mathcal{R} Q => \forall P'. P \red P' \Rightarrow \exists Q'. Q \red Q', P' \mathcal{R} Q'$

$P \vdash x \Rightarrow Q \vdash x$

\begin{mathpar}
  \inferrule*[lab=Out-barb]{x \nameeq y}{{y}!\langle{Q}\rangle \vdash x}
  \and
  \inferrule*[lab=Par-barb]{\mbox{$P\vdash x$ or $Q\vdash x$}}{\binpar{P}{Q} \vdash x}
\end{mathpar}

\subsubsection{Contexts}

One of the principle advantages of computational calculi like the
$\pi$-calculus is a well-defined notion of context,
contextual-equivalence and a correlation between
contextual-equivalence and notions of bisimulation. The notion of
context allows the decomposition of a process into (sub-)process and
its syntactic environment, its context. Thus, a context may be
thought of as a process with a ``hole'' (written $\Box$) in it. The
application of a context $M$ to a process $P$, written $M[P]$, is
tantamount to filling the hole in $M$ with $P$. In this paper we do
not need the full weight of this theory, but do make use of the notion
of context in the proof the main theorem. 

\begin{mathpar}
  \inferrule* [lab=summation] {} {{M_{M},M_{N}} \bc \Box \;|\; x.M_{A} \;|\; M_{M}+M_{N}}
  \and
  \inferrule* [lab=agent] {} {{M_{A}} \bc (\vec{x})M_{P} \;| \; \clift{P_0,\ldots,M_{P},\ldots,P_N}}
  \and \\
  \inferrule* [lab=process] {} {{M_{P}} \bc M_{N} \;| \;P|M_{P} }
\end{mathpar} 

\begin{mathpar}
  \inferrule* [lab=sychronization] {} {M_{N} \bc \Box \;|\; x?M_{F} \;|\; x!M_{C}}
  \and
  \inferrule* [lab=abstraction] {} {{M_{F}} \bc (x)M_{P} }
  \and
  \inferrule* [lab=concretion] {} {{M_{C}} \bc \langle M_{P} \rangle }
  \and \\
  \inferrule* [lab=process] {} {{M_{P}} \bc M_{N} \;| \;P|M_{P} }
\end{mathpar}

\begin{definition}[contextual application] Given a context $M$, and
  process $P$, we define the \emph{contextual application}, $M[P] :=
  M\{P/\Box\}$. That is, the contextual application of M to P is the
  substitution of $P$ for $\Box$ in $M$.
\end{definition}

$\meaningof{-} : L \to \mathcal{P}(\pi)$

\begin{mathpar}
  \inferrule* [lab=collection] {} {\meaningof{true} = \pi, \and \meaningof{~E} = \pi \setminus \meaningof{E}, \and \meaningof{E_{1} \& E_{2}} = \meaningof{E_{1}} \cap \meaningof{E_{2}}}
\end{mathpar}

\begin{mathpar}
  \inferrule* [lab=structure] {} {\meaningof{0} = \{ P \in \pi | P \equiv 0 \}, \and \\ \meaningof{E_1 | E_2} = \{ P \in \pi | P \equiv P_{1} | P_{2}, P_{1} \in \meaningof{E_{1}}, P_{2} \in \meaningof{E_2}\} }
\end{mathpar}

\begin{mathpar}
 \inferrule* [lab=behavior] {} {\meaningof{\langle a?b \rangle E} = \{ P \in \pi | P \equiv Q | u?(y)P', \\ \and \\\\ \and \\ \;\;\; u \in \meaningof{a}, \forall z.P'\{z/y\} \in \meaningof{E\{z/b\}}\}, \and \\ \meaningof{a!E} = \{ P \in \pi | P \equiv Q | x!\langle P' \rangle, x \in \meaningof{a} P' \in \meaningof{E}\} }
\end{mathpar}

\begin{mathpar}
 \inferrule* [lab=nominal] {} {\meaningof{\quotep{E}} = \{ \quotep{P} \in \quotep{\pi} | P \in \meaningof{E} \}, \and \meaningof{\quotep{P}} = \{ \quotep{Q} \in \quotep{\pi} | P \equiv Q \} \and \\ \meaningof{@\quotep{E}} = \{ P \in \pi | P \equiv @x, x \in \meaningof{E} \}}
\end{mathpar}

\begin{eqnarray*}
  \\
  \meaningof{-} : TS \to ST
\end{eqnarray*}

\begin{eqnarray*}
  \\
  L : TS \to ST
\end{eqnarray*}

\begin{eqnarray*}
  \\
  P \models E \iff P \in \meaningof{E}
\end{eqnarray*}

\begin{eqnarray*}
  P \approx_{L} Q \iff \forall E \in L. P \models E \iff Q \models E
\end{eqnarray*}

\begin{eqnarray*}
  P \approx_{K} Q
\end{eqnarray*}

\begin{eqnarray*}
  P \approx Q
\end{eqnarray*}

$\approx_{K} = \approx = \approx_{L}$

\subsubsection{Contextual duality}

Note that contexts extend the quotation operation to a family of
operations from processes to names. Given a context, $M$, we can
define a \emph{nominal context}, $\quotep{M}$ by $\quotep{M}[P] :=
\quotep{M[P]}$. To foreshadow what is to come we observe that these
operations enjoy a duality with processes very much like the duality
between vectors and maps from vectors to scalars.

Further, because the calculus is essentially higher-order, we have a
correspondence between contexts and processes. More specifically,
given a name $x$ and a context $M$ we can construct $M^{*}_{x}$ such
that 

\begin{mathpar}
  M^{*}_{x} | \lift{x}{P} \red M[P]
\end{mathpar}

namely,

\begin{mathpar}
  M^{*}_{x} := x?(u).M[\dropn{u}]
\end{mathpar}

The dependence of $M^{*}_{x}$ on a name makes it an abstraction, 

\begin{mathpar}
  M^{*} := (x)x?(u).M[\dropn{u}]
\end{mathpar}

\subsection{Additional notation}

It will sometimes be convenient to denote the process a name
quotes. We already have the notation $x = \quotep{P}$, but it will be
convenient to introduce an alternate notation, $\procn{x}$, when we
want to emphasize the connection to the use of the name. Note that, by
virtue of name equivalence, $\quotep{\procn{x}} \nameeq x$; so, the
notation is consistent with previous definitions.

Further, because names have structure it is possible to effect
substitutions on the basis of that structure. This means we need to
upgrade our notation for substitutions, which we accomplish by
adapting comprehension notation. Thus,

\begin{mathpar}
  P\{ y / x : x \in S \}
\end{mathpar}

is interpreted to mean the process derived from P by replacing (in a
capture-avoiding manner) each occurrence of $x$ in $S$ by $y$. For example,

\begin{mathpar}
  P\{ \quotep{\procn{x}|\procn{x}} / x : x \in \freenames{P} \}
\end{mathpar}

will replace each (occurrence) of a free name $x$ in $P$ by
$\quotep{\procn{x}|\procn{x}}$.

Also, we will avail ourselves of the notation $x^{L}$ and $x^{R}$ to
denote injections of a name into disjoint copies of the name
space. There are numerous ways to accomplish this. One example can be
found in \cite{MeredithR05}. This notation overloads to vectors of
names: $\vec{x}^{\pi} := (x_{i}^{\pi} \; : \; 0 \leq i < |\vec{x}| )$ where $\pi \in \{L,R\}$.

We also use $P^{\Box} := P|\Box$.

In \cite{MeredithR05} an interpretation of the new operator is
given. It turns out that there are several possible interpretations
all enjoying the requisite algebraic properties of the operator (see
\cite{milner91polyadicpi}). We will therefore make liberal use of
$(\nu\; \vec{x})P$.

% subsection the_syntax_and_semantics_of_the_notation_system (end)   

\input{qm2pi.qmops} 

\input{qm2pi.sterngerlach} 

\input{qm2pi.metric} 

% section concurrent_process_calculi (end)

%\input{qm2pi.proofsketch}

% section proof sketch (end)

%\input{qm2pi.slviaknots} 

% section spatial logic via knots (end)

\input{qm2pi.conclusion}

% section conclusion (end)

%\input{qm2pi.dtcodes} 

% section wiring algorithm (end)

\input{qm2pi.ack} 

% section acknowledgments (end)

\newpage


\bibliographystyle{plain}   
\bibliography{../../biblios/main.bib}

\input{qm2pi.rhodetails}

\end{document}



\end{document}

 

% subsection basic_interpretation (end)

%\input{qm2pi.rho.presentation} 
\subsection{The syntax and semantics of the notation system}\label{sub:the_syntax_and_semantics_of_the_notation_system} % (fold)

We now summarize a technical presentation of the calculus that
embodies our theory of dynamics. The typical presentation of such a
calculus follows the style of giving generators and relations on
them. The grammar, below, describing term constructors, freely
generates the set of processes, $\Proc$. This set is then quotiented
by a relation known as structural congruence and it is over this set
that the notion of dynamics is expressed. This presentation is
essentially that of \cite{MeredithR05} with the addition of
polyadicity and summation. For readability we have relegated some of
the technical subtleties to an appendix.

\subsubsection{Process grammar}\label{subsub:process_grammar}

\begin{mathpar}
  \inferrule* [lab=synchronization] {} {{M} \bc \pzero \;|\; x?F \;|\; x!C }
  \and
  \inferrule* [lab=abstraction] {} {{F} \bc (x)P}
  \and
  \inferrule* [lab=concretion] {} {{C} \bc \langle Q \rangle}
  \and
  \inferrule* [lab=process] {} {{P,Q} \bc M \;| \;P|Q \;|\; @{x}}
  \and
  \inferrule* [lab=name] {} {{x} \bc \quotep{P}}
\end{mathpar} 

Note that $\vec{x}$ (resp. $\vec{P}$) denotes a vector of names
(resp. processes) of length $|\vec{x}|$ (resp. $|\vec{P}|$). We adopt
the following useful abbreviations.

\begin{mathpar}
   x?(\vec{y}).P := x.(\vec{y})P \and  x\clift{\vec{P}} := x.\clift{\vec{P}}
   \and x!(y) := \lift{x}{\dropn{y}}
   \and \Pi_{i=0}^{n-1}P_i := P_0 | \ldots | P_{n-1}
\end{mathpar}

\subsubsection{Structural congruence}

\paragraph{Free and bound names and alpha-equivalence.} At the
core of structural equivalence is alpha-equivalence which identifies
process that are the same up to a change of variable. Formally, we
recognize the distinction between free and bound names. The free names
of a process, $\freenames{P}$, may be calculated recursively as
follows:

\begin{mathpar}
\freenames{\pzero} := \emptyset
  \and \\
  \freenames{x?(y).P} := \{ x \} \cup (\freenames{P} \setminus \{ y \})
  \and 
  \freenames{x!\langle P \rangle} := \{ x \} \cup \{ P \} 
  \and \\
  \freenames{P|Q} := \freenames{P} \cup \freenames{Q}
  \and \\
  \freenames{@{x}} := \{ x \}
\end{mathpar}

$\pi$
$\quotep{\pi}$

$\freenames{-} : \pi \to \mathcal{P}(\quotep{\pi})$

\begin{eqnarray*}
  \freenames{\pzero} & := & \emptyset \\
  \freenames{x?(y).P} & := & \{ x \} \cup (\freenames{P} \setminus \{ y \}) \\
  \freenames{x!\langle P \rangle} & := & \{ x \} \cup \{ P \} \\
  \freenames{P|Q} & := & \freenames{P} \cup \freenames{Q} \\
  \freenames{\dropn{x}} & := & \{ x \}
\end{eqnarray*}

The bound names of a process, $\boundnames{P}$, are those names occurring in $P$
that are not free. For example, in $x?(y).0$, the name $x$ is free, while $y$ is bound.

\begin{mathpar}
  \inferrule* [lab=monoidal-laws] {} { P|Q \equiv Q|P \and P|0 \equiv P \and P|(Q|R) \equiv (P|Q)|R }
\end{mathpar}

\begin{mathpar}
  \inferrule* [lab=alpha-equivalence] {} { (x)P \equiv (y)P\{y/x\} \and y \not\in \freenames{P} }
\end{mathpar}

\begin{definition}
Then two processes, $P,Q$, are alpha-equivalent if $P = Q\{\vec{y}/\vec{x}\}$ for
some $\vec{x} \in \boundnames{Q},\vec{y} \in \boundnames{P}$, where $Q\{\vec{y}/\vec{x}\}$
denotes the capture-avoiding substitution of $\vec{y}$ for $\vec{x}$ in $Q$.
\end{definition}

\begin{definition}
  The {\em structural congruence} \cite{SangiorgiWalker} , $\equiv$,
  between processes is the least congruence containing
  alpha-equivalence, satisfying the abelian monoid laws
  (associativity, commutativity and $\pzero$ as identity) for parallel
  composition $|$ and for summation $+$.
\end{definition}

\subsection{Name equivalence}

We take name equivalence, written $\nameeq$, to be the smallest
equivalence relation generated by the following rules.

\begin{mathpar}
\inferrule*[lab=Quote-drop]
{ }
{ \quotep{@{x}} \nameeq x }

\inferrule*[lab=Struct-equiv]
{ P \scong Q }
{ \quotep{P} \nameeq \quotep{Q} }
\end{mathpar}

The astute reader will have noticed that the mutual recursion of names
and processes imposes a mutual recursion on alpha-equivalence and
structural equivalence via name-equivalence. Fortunately, all of this
works out pleasantly and we may calculate in the natural way, free of
concern. The reader interested in the details is referred to the
appendix \ref{appendix:rho_details}.

\subsection{Substitution}

We use $\Proc$ for the set of processes, $\QProc$ for the set of
names, and $\id{\{}\vec{y} / \vec{x} \id{\}}$ to denote partial maps,
$s : \QProc \rightarrow \QProc$. A map, $s$ lifts, uniquely, to a map
on process terms, $\widehat{s} : \Proc \rightarrow \Proc$ by the
following equations.

\begin{mathpar}
  (0) \psubstp{Q}{P} := 0 \\
  (R \juxtap S) \psubstp{Q}{P}
  :=    
  (R)\psubstp{Q}{P} \juxtap (S) \psubstp{Q}{P} \\
  (x?(y).R) \psubstp{Q}{P}    
  :=    
  (x)\substp{Q}{P} (z)\concat( (R \psubstn{z}{y}) \psubstp{Q}{P} ) \\
  (\lift{x}{R}) \psubstp{Q}{P}  
  :=
  \lift{(x)\substp{Q}{P}}{ R \psubstp{Q}{P} } \\
%   (\dropn{x})  \psubstp{Q}{P}       
%   := 
%   \left\{ 
%     \begin{array}{ccc} 
%       \dropn{\quotep{Q}} & & x \nameeq \quotep{P} \\
%       \dropn{x} & & otherwise \\
%     \end{array}
%   \right. 
  (\dropn{x})  \psubstp{Q}{P}       
  := 
  \left\{ 
    \begin{array}{ccc} 
      Q & & x \nameeq \quotep{P} \\
      \dropn{x} & & otherwise \\
    \end{array}
  \right.
\end{mathpar}
 

where

\begin{eqnarray}
  (x)\id{\{} \lpquote Q \rpquote / \lpquote P \rpquote \id{\}}            = 
  \left\{ 
    \begin{array}{ccc}
      \lpquote Q \rpquote & & x \nameeq \lpquote P \rpquote \\
      x & & otherwise \\
    \end{array}
  \right. \nonumber
\end{eqnarray}

and $z$ is chosen distinct from $\quotep{P}$, $\quotep{Q}$, the free
names in $Q$, and all the names in $R$. Our $\alpha$-equivalence will
be built in the standard way from this substitution.

\begin{remark}\label{rem:no_self_referential_names}
  One consequence of these definitions is that $\forall P. \quotep{P}
  \not\in \freenames{P}$.
\end{remark}

\subsection{ Dynamic quote: an example }

Anticipating something of what's to come, consider applying the
substitution, $\widehat{\id{\{}u / z \id{\}}}$, to the following pair
of processes, $\lift{w}{y!(z)}$ and $w[ \lpquote y!(z) \rpquote ]$.

\begin{eqnarray}
	\lift{w}{y!(z)}\widehat{\id{\{}u / z \id{\}}}
		& = &
		\lift{w}{y!(u)} \nonumber\\
	w[ \lpquote y!(z) \rpquote ] \widehat{ \id{\{}u / z \id{\}} }
		& = &
		w[ \lpquote y!(z) \rpquote ] \nonumber
\end{eqnarray}

Because the body of the process between quotes is impervious to
substitution, we get radically different answers. In fact, by
examining the first process in an input context,
e.g. $x?(z).\lift{w}{y!(z)}$, we see that the process under the lift
operator may be shaped by prefixed inputs binding a name inside it. In
this sense, the lift operator will be seen as a way to dynamically
construct processes before reifying them as names.

Finally equipped with these standard features we can present the
dynamics of the calculus.

\subsubsection{Operational semantics} 

Finally, we introduce the computational dynamics. What marks these
algebras as distinct from other more traditionally studied algebraic
structures, e.g. vector spaces or polynomial rings, is the manner in
which dynamics is captured. In traditional structures, dynamics is typically
expressed through morphisms between such structures, as in linear maps
between vector spaces or morphisms between rings. In algebras
associated with the semantics of computation, the dynamics is
expressed as part of the algebraic structure itself, through a
reduction reduction relation typically denoted by $\red$. Below, we
give a recursive presentation of this relation for the calculus used
in the encoding.

$\red \subseteq \pi \times \pi$
$\red : \pi \to \mathcal{P}(\pi)$

\begin{mathpar}
  \inferrule* [lab=Comm] { \textsf{match}( x_{src}, x_{trgt} ) } { x_{trgt}?(y)P \; | \; x_{src}!\langle {Q} \rangle \red P\{\quotep{Q}/y}\} }
  \and \\
  \inferrule* [lab=Par] {{P} \red {P}'} {{{P} | {Q}} \red {{P}' | {Q}}}
  \and
  \inferrule* [lab=Equiv]{{{P} \scong {P}'} \andalso {{P}' \red {Q}'} \andalso {{Q}' \scong {Q}}}{{P} \red {Q}}
\end{mathpar}

\begin{eqnarray*}
  match_{\equiv} (\quotep{P},\quotep{Q}) & := & P \equiv Q \\
  match_{\dagger}(\quotep{P},\quotep{Q}) & := & \forall R. P|Q \red^{*} R => R \red^{*} 0 \\
  match_{K}(\quotep{P},\quotep{Q}) & := & K \mbox{ for some context } K
\end{eqnarray*}

$u?(x)P | u!\langle Q \rangle \red P\{\quotep{Q}/x\}$

%We write $\wred$ for $\red^*$, and $P\red$ if $\exists Q $ such that $ P \red Q$.
We write $P\red$ if $\exists Q $ such that $ P \red Q$ and $P\not\red$, otherwise.

\section{Replication}

As mentioned before, it is known that replication (and hence
recursion) can be implemented in a higher-order process algebra
\cite{SangiorgiWalker}. As our first example of calculation with the
machinery thus far presented we give the construction explicitly in
the {\rhoc}.

\begin{eqnarray}
	D_{x} & := & \prefix{x}{y}{(\binpar{\outputp{x}{y}}{@{y}})} \nonumber\\
	\bangp_{x}{P} & := & \binpar{{x}!\langle{\binpar{D_{x}}{P}}\rangle}{D_{x}} \nonumber
\end{eqnarray}

\begin{eqnarray}
	\bangp_{x}{P} & & \nonumber\\
	=
	& {x}!\langle{(\prefix{x}{y}{(\outputp{x}{y} | @{y})) | P}}\rangle 
	      | \prefix{x}{y}{(\outputp{x}{y} | @{y})} & \nonumber\\
	\red
	& (\outputp{x}{y} | @{y})\substn{\quotep{(\prefix{x}{y}{(@{y} | \outputp{x}{y})) | P}}}{y} & \nonumber\\
	=
	& \outputp{x}{\quotep{(\prefix{x}{y}{(\outputp{x}{y} | @{y})) | P}}}
	  | {(\prefix{x}{y}{(\outputp{x}{y} | @{y})) | P}} & \nonumber\\
	\red
	& \ldots & \nonumber\\
	\red^*
	& P | P | \ldots & \nonumber
\end{eqnarray}

Of course, this encoding, as an implementation, runs away, unfolding
$\bangp{P}$ eagerly. A lazier and more implementable replication
operator, restricted to input-guarded processes, may be obtained as follows.

\begin{eqnarray}
\bangp{\prefix{u}{v}{P}} 
	:= 
	\binpar{\lift{x}{\prefix{u}{v}{(\binpar{D(x)}{P})}}}{D(x)} \nonumber
\end{eqnarray}

\begin{remark}
  Note that the lazier definition still does not deal with summation
  or mixed summation (i.e. sums over input and output). The reader is
  invited to construct definitions of replication that deal with these
  features. 

  Further, the definitions are parameterized in a name, $x$. Can you,
  gentle reader, make a definition that eliminates this parameter and
  guarantees no accidental interaction between the replication
  machinery and the process being replicated -- i.e. no accidental
  sharing of names used by the process to get its work done and the
  name(s) used by the replication to effect copying. This latter
  revision of the definition of replication is crucial to obtaining
  the expected identity $!!P \sim !P$.
\end{remark}

\begin{remark}\label{rem:paradoxical_combinator}
  The reader familiar with the lambda calculus will have noticed the
  similarity between $D$ and the paradoxical combinator.

  [Ed. note: the existence of this seems to suggest we have to be more
  restrictive on the set of processes and names we admit if we are to
  support no-cloning.]
\end{remark}

\subsubsection{Bisimulation}

The computational dynamics gives rise to another kind of equivalence,
the equivalence of computational behavior. As previously mentioned
this is typically captured \emph{via} some form of bisimulation.

% The notion we use in this paper is weak barbed bisimulation
% \cite{milner91polyadicpi}.

The notion we use in this paper is derived from weak barbed
bisimulation \cite{milner91polyadicpi}. 

\begin{definition}
An \emph{observation relation}, $\downarrow_{\mathcal N}$, over a set
of names, $\mathcal N$, is the smallest relation satisfying the rules
below.

\infrule[Out-barb]{y \in {\mathcal N}, \; x \nameeq y}
		  {\outputp{x}{v} \downarrow_{\mathcal N} x}
\infrule[Par-barb]{\mbox{$P\downarrow_{\mathcal N} x$ or $Q\downarrow_{\mathcal N} x$}}
		  {\binpar{P}{Q} \downarrow_{\mathcal N} x}

We write $P \Downarrow_{\mathcal N} x$ if there is $Q$ such that 
$P \wred Q$ and $Q \downarrow_{\mathcal N} x$.
\end{definition}

\begin{definition}
%\label{def.bbisim}
An  ${\mathcal N}$-\emph{barbed bisimulation} over a set of names, ${\mathcal N}$, is a symmetric binary relation 
${\mathcal S}_{\mathcal N}$ between agents such that $P\rel{S}_{\mathcal N}Q$ implies:
\begin{enumerate}
\item If $P \red P'$ then $Q \wred Q'$ and $P'\rel{S}_{\mathcal N} Q'$.
\item If $P\downarrow_{\mathcal N} x$, then $Q\Downarrow_{\mathcal N} x$.
\end{enumerate}
$P$ is ${\mathcal N}$-barbed bisimilar to $Q$, written
$P \wbbisim_{\mathcal N} Q$, if $P \rel{S}_{\mathcal N} Q$ for some ${\mathcal N}$-barbed bisimulation ${\mathcal S}_{\mathcal N}$.
\end{definition}

$\mathcal{R} \subseteq \pi \times \pi$

$P \mathcal{R} Q => \forall P'. P \red P' \Rightarrow \exists Q'. Q \red Q', P' \mathcal{R} Q'$

$P \vdash x \Rightarrow Q \vdash x$

\begin{mathpar}
  \inferrule*[lab=Out-barb]{x \nameeq y}{{y}!\langle{Q}\rangle \vdash x}
  \and
  \inferrule*[lab=Par-barb]{\mbox{$P\vdash x$ or $Q\vdash x$}}{\binpar{P}{Q} \vdash x}
\end{mathpar}

\subsubsection{Contexts}

One of the principle advantages of computational calculi like the
$\pi$-calculus is a well-defined notion of context,
contextual-equivalence and a correlation between
contextual-equivalence and notions of bisimulation. The notion of
context allows the decomposition of a process into (sub-)process and
its syntactic environment, its context. Thus, a context may be
thought of as a process with a ``hole'' (written $\Box$) in it. The
application of a context $M$ to a process $P$, written $M[P]$, is
tantamount to filling the hole in $M$ with $P$. In this paper we do
not need the full weight of this theory, but do make use of the notion
of context in the proof the main theorem. 

\begin{mathpar}
  \inferrule* [lab=summation] {} {{M_{M},M_{N}} \bc \Box \;|\; x.M_{A} \;|\; M_{M}+M_{N}}
  \and
  \inferrule* [lab=agent] {} {{M_{A}} \bc (\vec{x})M_{P} \;| \; \clift{P_0,\ldots,M_{P},\ldots,P_N}}
  \and \\
  \inferrule* [lab=process] {} {{M_{P}} \bc M_{N} \;| \;P|M_{P} }
\end{mathpar} 

\begin{mathpar}
  \inferrule* [lab=sychronization] {} {M_{N} \bc \Box \;|\; x?M_{F} \;|\; x!M_{C}}
  \and
  \inferrule* [lab=abstraction] {} {{M_{F}} \bc (x)M_{P} }
  \and
  \inferrule* [lab=concretion] {} {{M_{C}} \bc \langle M_{P} \rangle }
  \and \\
  \inferrule* [lab=process] {} {{M_{P}} \bc M_{N} \;| \;P|M_{P} }
\end{mathpar}

\begin{definition}[contextual application] Given a context $M$, and
  process $P$, we define the \emph{contextual application}, $M[P] :=
  M\{P/\Box\}$. That is, the contextual application of M to P is the
  substitution of $P$ for $\Box$ in $M$.
\end{definition}

$\meaningof{-} : L \to \mathcal{P}(\pi)$

\begin{mathpar}
  \inferrule* [lab=collection] {} {\meaningof{true} = \pi, \and \meaningof{~E} = \pi \setminus \meaningof{E}, \and \meaningof{E_{1} \& E_{2}} = \meaningof{E_{1}} \cap \meaningof{E_{2}}}
\end{mathpar}

\begin{mathpar}
  \inferrule* [lab=structure] {} {\meaningof{0} = \{ P \in \pi | P \equiv 0 \}, \and \\ \meaningof{E_1 | E_2} = \{ P \in \pi | P \equiv P_{1} | P_{2}, P_{1} \in \meaningof{E_{1}}, P_{2} \in \meaningof{E_2}\} }
\end{mathpar}

\begin{mathpar}
 \inferrule* [lab=behavior] {} {\meaningof{\langle a?b \rangle E} = \{ P \in \pi | P \equiv Q | u?(y)P', \\ \and \\\\ \and \\ \;\;\; u \in \meaningof{a}, \forall z.P'\{z/y\} \in \meaningof{E\{z/b\}}\}, \and \\ \meaningof{a!E} = \{ P \in \pi | P \equiv Q | x!\langle P' \rangle, x \in \meaningof{a} P' \in \meaningof{E}\} }
\end{mathpar}

\begin{mathpar}
 \inferrule* [lab=nominal] {} {\meaningof{\quotep{E}} = \{ \quotep{P} \in \quotep{\pi} | P \in \meaningof{E} \}, \and \meaningof{\quotep{P}} = \{ \quotep{Q} \in \quotep{\pi} | P \equiv Q \} \and \\ \meaningof{@\quotep{E}} = \{ P \in \pi | P \equiv @x, x \in \meaningof{E} \}}
\end{mathpar}

\begin{eqnarray*}
  \\
  \meaningof{-} : TS \to ST
\end{eqnarray*}

\begin{eqnarray*}
  \\
  L : TS \to ST
\end{eqnarray*}

\begin{eqnarray*}
  \\
  P \models E \iff P \in \meaningof{E}
\end{eqnarray*}

\begin{eqnarray*}
  P \approx_{L} Q \iff \forall E \in L. P \models E \iff Q \models E
\end{eqnarray*}

\begin{eqnarray*}
  P \approx_{K} Q
\end{eqnarray*}

\begin{eqnarray*}
  P \approx Q
\end{eqnarray*}

$\approx_{K} = \approx = \approx_{L}$

\subsubsection{Contextual duality}

Note that contexts extend the quotation operation to a family of
operations from processes to names. Given a context, $M$, we can
define a \emph{nominal context}, $\quotep{M}$ by $\quotep{M}[P] :=
\quotep{M[P]}$. To foreshadow what is to come we observe that these
operations enjoy a duality with processes very much like the duality
between vectors and maps from vectors to scalars.

Further, because the calculus is essentially higher-order, we have a
correspondence between contexts and processes. More specifically,
given a name $x$ and a context $M$ we can construct $M^{*}_{x}$ such
that 

\begin{mathpar}
  M^{*}_{x} | \lift{x}{P} \red M[P]
\end{mathpar}

namely,

\begin{mathpar}
  M^{*}_{x} := x?(u).M[\dropn{u}]
\end{mathpar}

The dependence of $M^{*}_{x}$ on a name makes it an abstraction, 

\begin{mathpar}
  M^{*} := (x)x?(u).M[\dropn{u}]
\end{mathpar}

\subsection{Additional notation}

It will sometimes be convenient to denote the process a name
quotes. We already have the notation $x = \quotep{P}$, but it will be
convenient to introduce an alternate notation, $\procn{x}$, when we
want to emphasize the connection to the use of the name. Note that, by
virtue of name equivalence, $\quotep{\procn{x}} \nameeq x$; so, the
notation is consistent with previous definitions.

Further, because names have structure it is possible to effect
substitutions on the basis of that structure. This means we need to
upgrade our notation for substitutions, which we accomplish by
adapting comprehension notation. Thus,

\begin{mathpar}
  P\{ y / x : x \in S \}
\end{mathpar}

is interpreted to mean the process derived from P by replacing (in a
capture-avoiding manner) each occurrence of $x$ in $S$ by $y$. For example,

\begin{mathpar}
  P\{ \quotep{\procn{x}|\procn{x}} / x : x \in \freenames{P} \}
\end{mathpar}

will replace each (occurrence) of a free name $x$ in $P$ by
$\quotep{\procn{x}|\procn{x}}$.

Also, we will avail ourselves of the notation $x^{L}$ and $x^{R}$ to
denote injections of a name into disjoint copies of the name
space. There are numerous ways to accomplish this. One example can be
found in \cite{MeredithR05}. This notation overloads to vectors of
names: $\vec{x}^{\pi} := (x_{i}^{\pi} \; : \; 0 \leq i < |\vec{x}| )$ where $\pi \in \{L,R\}$.

We also use $P^{\Box} := P|\Box$.

In \cite{MeredithR05} an interpretation of the new operator is
given. It turns out that there are several possible interpretations
all enjoying the requisite algebraic properties of the operator (see
\cite{milner91polyadicpi}). We will therefore make liberal use of
$(\nu\; \vec{x})P$.

% subsection the_syntax_and_semantics_of_the_notation_system (end)   

\section{Interpretation of QM}
\subsection{Supporting definitions}
\subsubsection{Multiplication}
\begin{mathpar}
  \quotep{Q} \cdot \quotep{R} := \quotep{Q|R}
  \and \\
  \quotep{Q} \cdot P := P\{ \quotep{Q|R} / \quotep{R} : \quotep{R} \in \freenames{P} \}
\end{mathpar}

\paragraph{Discussion}
The first line needs little explanation. The second line says that
each free name of the process is replaced with the multiplication of
that name by the scalar. Multiplication of a scalar (name) by a state
(process) results in a process all the names of which have been `moved
over' by parallel composition with the process the scalar
quotes. There is a subtlety that the bound names have to be
manipulated so that multiplied names aren't accidentally
captured. There are many ways to achieve this.

\begin{remark}\label{rem:multiplication_identities}
  The reader is invited to verify that for all $x,y,z \in \QProc$ and $P \in \Proc$
  \begin{mathpar}
    x \cdot \quotep{0} \equiv x 
    \and
    x \cdot y \equiv y \cdot x
    \and
    x \cdot (y \cdot z) \equiv (x \cdot y) \cdot z
    \and \\
    \quotep{0} \cdot P \equiv P
    \and \\
    x \cdot (y \cdot P) \equiv (x \cdot y) \cdot P
    \and \\
    x \cdot (P|Q) \equiv (x \cdot P) | (x \cdot Q)
    \and \\    
  \end{mathpar}
\end{remark}

\subsubsection{Tensor product}

We define a tensor product on processes by structural induction.

\paragraph{Tensor of sums} First note that all summations, including
$\pzero$ and sequence, can be written $\Sigma_{i} x_{i}.A_{i} +
\Sigma_{j} x_{j}.C_{j}$, where we have grouped input-guarded processes
together and output-guarded processes together.

Thus, we can define the tensor product of two summations, $N_{1}\otimes N_{2}$, where

\begin{mathpar}
  N_{1} := \Sigma_{i} x_{i}.A_{i} + \Sigma_{j} x_{j}.C_{j}
  \and
  N_{2} := \Sigma_{i'} y_{i'}.B_{i'} + \Sigma_{j'} y_{j'}.D_{j'} 
\end{mathpar}

as follows.

\begin{mathpar}
  \Sigma_{i} x_{i}.A_{i} + \Sigma_{j} x_{j}.C_{j} \otimes \Sigma_{i'}
  y_{i'}.B_{i'} + \Sigma_{j'} y_{j'}.D_{j'} 
  \and \\
  := \; \Sigma_{i} \Sigma_{i'} \quotep{\stackrel{\vee}{x_{i}}| \stackrel{\vee}{y_{i'}}}.(A_{i}\otimes B_{i'}) \; | \; \Sigma_{i'} \Sigma_{i} \quotep{\stackrel{\vee}{y_{i'}}|\stackrel{\vee}{x_{i}}}.(B_{i'}\otimes A_{i})
  \and
  \;\; | \;\; \Sigma_{j} \Sigma_{j'} \quotep{\stackrel{\vee}{x_{j}}|\stackrel{\vee}{y_{j'}}}.(A_{j}\otimes B_{j'}) \; | \; \Sigma_{j'} \Sigma_{j} \quotep{\stackrel{\vee}{y_{j'}}|\stackrel{\vee}{x_{j}}}.(B_{j'}\otimes A_{j})
\end{mathpar}

\begin{remark}
  Do we need to $x^{L}$ and $y^{R}$ for this construction as well?
\end{remark}

\paragraph{Tensor of parallel compositions} Next, we distribute tensor
over par.

\begin{mathpar}
  P_{1}|P_{2} \otimes Q_{1}|Q_{2} := (P_{1} \otimes Q_{1}) | (P_{1}
  \otimes Q_{2}) | (P_{2} \otimes Q_{1}) | (P_{2} \otimes Q_{2})
\end{mathpar}

\paragraph{Tensor with dropped names} We treat tensor of a
process with a dropped name as parallel composition.

\begin{mathpar}
  P \otimes \dropn{x} := P | \dropn{x}
\end{mathpar}

\paragraph{Tensor of agents}

Finally, we need to define tensor on agents. Note that the definition
of tensor on normal products only tensors inputs with inputs and
outputs with outputs. Thus, we only have to define the operation on
``homogeneous'' pairings.

\begin{mathpar}
  (\vec{x})P \otimes (\vec{y})Q
  \and \\
  := (x_{0}^{L}|y_{0}^{R},\ldots,x_{0}^{L}|y_{n}^{R},\ldots,x_{m}^{L}|y_{0}^{R},\ldots,x_{m}^{L}|y_{n}^R)(P\{ \vec{x}^{L}/\vec{x}\} \otimes Q \{ \vec{y}^{R}/\vec{y}\})
  \and \\
  \clift{\vec{P}} \otimes \clift{\vec{Q}}
  \and \\
  := \clift{P_{0}\otimes Q_{0},\ldots,P_{0}\otimes Q_{n},\ldots,P_{m}\otimes Q_{0},\ldots,P_{m}\otimes Q_{n}}
\end{mathpar}

\begin{remark}
  Observe that arities of tensored abstractions matches arities of
  tensored concretions if the original arities matched. Note also that
  the length of the arities corresponds to the increase in dimension
  we see in ordinary vector space tensor product.
\end{remark}

\begin{remark}
  Operationally, this definition distributes the tensor down to
  components ``linked'' by summation. Tensor over summation is
  intriguing in that it mixes names. Moreover, as a consequence of the
  way it mixes names we have the identities for all $x \in \QProc$ and
  $P,Q \in \Proc$

  \begin{mathpar}
    (x \cdot P) \otimes Q \equiv x \cdot (P \otimes Q) \equiv P \otimes (x \cdot Q)
    \and
    P \otimes \pzero \equiv P
  \end{mathpar}

  that the reader is invited to verify.
\end{remark}

\subsubsection{Annihilation}
\begin{mathpar}
  P^{\perp} := \{ Q | \forall R. P|Q \red^{*} R \Rightarrow R \red^{*} \pzero \}
  \and \\
  P^{\underline{\perp}} := \Sigma_{Q \in P^{\perp}} \quotep{Q}?(y).(\dropn{y}|Q) | \Sigma_{Q \in P^{\perp}} \quotep{Q}\clift{\Box}
\end{mathpar}

\paragraph{Discussion} The reader will note that $P^{\perp}$ is a
\emph{set} of processes, while $P^{\underline{\perp}}$ is a
\emph{context}. We call the set $P^{\perp}$ the \emph{annihilators} of
$P$. The parallel composition of a process in the annihilators of $P$
with $P$ will result in a process, the state space of which has all
paths eventually leading to $\pzero$. Execution may endure loops; but
under reasonable conditions of fairness (naturally guaranteed under
most notions of bisimulation) such a composite process cannot get
stuck in such a loop and will, eventually pop out and terminate.

The context $P^{\underline{\perp}}$ is ready and willing to ``take the
$P$ out of'' the process to which it is applied. It will effectively
transmit the code of the process to which it is applied to one of the
annihilators and run the process against it.

\subsubsection{Evaluation}
We fix $M$ a domain of fully abstract interpretation with an equality
coincident with bisimulation. We take $\meaningof{\cdot} : \Proc \to
M$ to be the map interpreting processes and $\nmeaningof{\cdot} : \M
\to Proc$ to be the map running the other way. Then we define

\begin{mathpar}
  \int P := \nmeaningof{\meaningof{P}}
\end{mathpar}

\paragraph{Discussion}
There are many fully abstract interpretations of Milner's
$\pi$-calculus. Any of them can be used as a basis for interpreting
the reflective calculus here. Equipped with such a domain it is
largely a matter of grinding through to check that the Yoneda
construction for the normalization-by-evaluation program can be
extended to this setting.

\begin{remark}
  The reader is invited to verify that $\int (P^{\underline{\perp}}[P]) = 0$.
\end{remark}

\subsection{Quantum mechanics}

Table \ref{tbl:core_qm_op_defns} gives the core operational definitions

\begin{table}[htp]\label{tbl:core_qm_op_defns}
  \center{
    \fbox{
      \begin{tabular}{c|c}
        quantum mechanics & process calculus \\
        \hline
        scalar & $x := \quotep{P}$ \\
        state vector & $\state{P} := P$ \\
        dual & $\state{P}^{*} := \event{P^{\underline{\perp}}} := \quotep{P^{\underline{\perp}}}[-]$ \\
        matrix & $ \Sigma_{\alpha} \state{P_{\alpha}}x_{\alpha}\event{Q_{\alpha}}$ \\
        vector addition & $\state{P} + \state{Q} := \state{P | Q}$ \\
        tensor product & $\state{P} \otimes \state{Q} := \state{P \otimes Q}$ \\
        inner product & $\innerprod{P}{Q} := \quotep{\int P^{\underline{\perp}}[Q]}$ \\
      \end{tabular}
    }
  }
  \caption{QM - operational definitions}
\end{table}

where

\begin{mathpar}
  \prmatrix{P}{Q} := \fprmatrix{P}{\quotep{\pzero}}{Q}
  \and
  \fprmatrix{P}{x}{Q} := (\state{P},x,\event{Q})
  \and
  (\fprmatrix{P}{x}{Q})(\state{R}) := x \cdot \innerprod{Q}{R} \cdot \state{P}
  \and
  (\fprmatrix{P}{x}{Q})(\event{R}) := x \cdot \innerprod{R}{P} \cdot \event{Q}
\end{mathpar}

\paragraph{Discussion}
As promised: vectors (aka states) are represented as processes; duals
as contextual duals; inner product definition should be compared with
standard inner product definition for ....

\begin{remark}
  Assuming $\int (P^{\underline{\perp}}[P]) = 0$, the reader is
  invited to verify that $(\fprmatrix{P}{x}{P})(\state{P}) = x \cdot \state{P}$.
\end{remark}

\begin{remark}
  The reader is invited to verify that $\innerprod{P}{Q}$ could
  equally well have been written $\quotep{\int \stackrel{\vee}{x}}$
  where $x = \event{P^{\underline{\perp}}}(Q)$.

  One of the motivations for this remark is that there is another way
  to factor these operations. We could package up evaluation in the dual:

  \begin{mathpar}
    \state{P}^{*} := \event{\int P^{\underline{\perp}}} := \quotep{\int P^{\underline{\perp}}}[-]
  \end{mathpar}

  and then have inner product defined by
  
  \begin{mathpar}
    \innerprod{P}{Q} := \event{P}(Q)
  \end{mathpar}

  Hopefully, experience with the calculations will provide guidance on
  the best factoring.
\end{remark}

\begin{remark}
  Assuming $\int (P^{\underline{\perp}}[P]) = 0$, the reader is
  invited to verify that $\forall P,Q. (\prmatrix{0}{Q})(\state{0}) =
  \state{0}$ and dually $(\prmatrix{P}{0})(\event{0}) = \event{0}$.
\end{remark}

\begin{remark}
  i'm a little worried that i don't (yet) have proper support for
  complex conjugacy. But, the observation above may give us a
  clue. According to Abramsky, it must be the case that the scalars
  are iso to the homset of the identity for the tensor -- which the
  observation above characterizes. 

  For now, we will simply bookmark the notion with $\overline{x}$.
\end{remark}

\subsubsection{Adjointness}

We need to give a definition of $(\cdot)^{\dagger}$ for matrices. The
obvious candidate definition is
\begin{mathpar}
(\Sigma_{\alpha}\fprmatrix{P_{\alpha}}{x_{\alpha}}{Q_{\alpha}})^{\dagger}
= \Sigma_{\alpha}\fprmatrix{(Q_{\alpha}^{\underline{\perp}})^{*}}{\overline{x}_{\alpha}}{P_{\alpha}^{\underline{\perp}}} 
\end{mathpar}

But, $(Q_{\alpha}^{\underline{\perp}})^{*}$ requires a name along
which to communicate the process to achieve the context application.

\subsubsection{Basis for a basis}
If processes label states and ``addition'' of states (a.k.a. vector
addition) is interpreted as parallel composition, what corresponds to
notions of linear independence and basis? Here, we recall that Yoshida
has developed a set of \emph{combinators} for an asynchronous verison
of Milner's $\pi$-calculus. These are a finite set of processes such
any process can be expressed as parallel composition of these
combinators together with liberal uses of the new operator and
replication. We can simply give a translation of these into the
present calculus and have reasonable expectation that the property
carries over. That is, that the resultant set allows to express all
processes via parallel composition. Note, however, that there is no
new operator or replication in this calculus. As a result, we expect
that the corresponding set is actually infinite. That is, we expect
that the space is actually infinite dimensional.

\begin{remark}
  The attentive reader may be a bit concerned. Certainly, the
  collection $S$, $K$ and $I$ is a finite set of
  combinators. Shouldn't we expect to see a finite set of combinators
  for an effectively equivalent system? i am very sympathetic to this
  critique and feel it warrants full attention. On the other hand, i
  also have in mind the following analogy. The natural numbers, as a
  monoid under addition, has exactly $1$ generator, while the natural
  numbers, as a monoid under multiplication, has countably many
  generators (the primes). We observe that the application of the
  lambda calculus is much less resource sensitive than the parallel
  composition of the $\pi$-calculus. Could it be the case that we have
  an analogy of the form
  
  \begin{mathpar}
    m + n : MN :: m*n : M|N
  \end{mathpar}

  giving a similar blow up in the set of ``primes''?  This is such a
  wonderful thought that, even if it's not true, i think it's worth
  writing down.
\end{remark}
 

\documentclass[12pt]{llncs}
%\documentclass{jktr}

\usepackage[pdftex]{hyperref}                   
\usepackage {listings}
\usepackage {mathpartir}
\usepackage{bcprules}
%\usepackage{listings}
                       
\usepackage{graphicx} 
%\usepackage[margins=2.5cm,nohead,nofoot]{geometry}
%\usepackage{geometry}
\usepackage{amsfonts}
\usepackage{amstext}
\usepackage{latexsym}
\usepackage{amssymb}
\usepackage{color}


%\include{myPreamble}
\documentclass[12pt]{llncs}
%\documentclass{jktr}

\usepackage[pdftex]{hyperref}                   
\usepackage {listings}
\usepackage {mathpartir}
\usepackage{bcprules}
%\usepackage{listings}
                       
\usepackage{graphicx} 
%\usepackage[margins=2.5cm,nohead,nofoot]{geometry}
%\usepackage{geometry}
\usepackage{amsfonts}
\usepackage{amstext}
\usepackage{latexsym}
\usepackage{amssymb}
\usepackage{color}


%\include{myPreamble}
\include{qm2pi.local} 

%\ifpdf
%\usepackage[pdftex]{graphicx}
%\else
%\usepackage{graphicx}
%\fi

 % \ifpdf
%  \usepackage{pdfsync}
%  \if


%\title{Brief Article}
%\author{David F. Snyder}
%\author{L.G. Meredith}

%\address{Dept. of Math., Texas State University--San Marcos, San Marcos, TX 78666}
       
\pagestyle{empty}


\begin{document}

\lstset{language=[Objective]Caml,frame=shadowbox}

\input{qm2pi.front}

% section front matter (end)

\input{qm2pi.intro} 
 
% section introduction (end)

% \input{qm2pi.knotations} 

% section notation (end)

\input{qm2pi.process.calculi} 

% section concurrent_process_calculi_and_spatial_logics_ (end)
    
%\input{qm2pi.knots2pi} 

%\input{qm2pi.trefoil} 

%\input{qm2pi.mainthm} 

% subsection basic_interpretation (end)

%\input{qm2pi.rho.presentation} 
\subsection{The syntax and semantics of the notation system}\label{sub:the_syntax_and_semantics_of_the_notation_system} % (fold)

We now summarize a technical presentation of the calculus that
embodies our theory of dynamics. The typical presentation of such a
calculus follows the style of giving generators and relations on
them. The grammar, below, describing term constructors, freely
generates the set of processes, $\Proc$. This set is then quotiented
by a relation known as structural congruence and it is over this set
that the notion of dynamics is expressed. This presentation is
essentially that of \cite{MeredithR05} with the addition of
polyadicity and summation. For readability we have relegated some of
the technical subtleties to an appendix.

\subsubsection{Process grammar}\label{subsub:process_grammar}

\begin{mathpar}
  \inferrule* [lab=synchronization] {} {{M} \bc \pzero \;|\; x?F \;|\; x!C }
  \and
  \inferrule* [lab=abstraction] {} {{F} \bc (x)P}
  \and
  \inferrule* [lab=concretion] {} {{C} \bc \langle Q \rangle}
  \and
  \inferrule* [lab=process] {} {{P,Q} \bc M \;| \;P|Q \;|\; @{x}}
  \and
  \inferrule* [lab=name] {} {{x} \bc \quotep{P}}
\end{mathpar} 

Note that $\vec{x}$ (resp. $\vec{P}$) denotes a vector of names
(resp. processes) of length $|\vec{x}|$ (resp. $|\vec{P}|$). We adopt
the following useful abbreviations.

\begin{mathpar}
   x?(\vec{y}).P := x.(\vec{y})P \and  x\clift{\vec{P}} := x.\clift{\vec{P}}
   \and x!(y) := \lift{x}{\dropn{y}}
   \and \Pi_{i=0}^{n-1}P_i := P_0 | \ldots | P_{n-1}
\end{mathpar}

\subsubsection{Structural congruence}

\paragraph{Free and bound names and alpha-equivalence.} At the
core of structural equivalence is alpha-equivalence which identifies
process that are the same up to a change of variable. Formally, we
recognize the distinction between free and bound names. The free names
of a process, $\freenames{P}$, may be calculated recursively as
follows:

\begin{mathpar}
\freenames{\pzero} := \emptyset
  \and \\
  \freenames{x?(y).P} := \{ x \} \cup (\freenames{P} \setminus \{ y \})
  \and 
  \freenames{x!\langle P \rangle} := \{ x \} \cup \{ P \} 
  \and \\
  \freenames{P|Q} := \freenames{P} \cup \freenames{Q}
  \and \\
  \freenames{@{x}} := \{ x \}
\end{mathpar}

$\pi$
$\quotep{\pi}$

$\freenames{-} : \pi \to \mathcal{P}(\quotep{\pi})$

\begin{eqnarray*}
  \freenames{\pzero} & := & \emptyset \\
  \freenames{x?(y).P} & := & \{ x \} \cup (\freenames{P} \setminus \{ y \}) \\
  \freenames{x!\langle P \rangle} & := & \{ x \} \cup \{ P \} \\
  \freenames{P|Q} & := & \freenames{P} \cup \freenames{Q} \\
  \freenames{\dropn{x}} & := & \{ x \}
\end{eqnarray*}

The bound names of a process, $\boundnames{P}$, are those names occurring in $P$
that are not free. For example, in $x?(y).0$, the name $x$ is free, while $y$ is bound.

\begin{mathpar}
  \inferrule* [lab=monoidal-laws] {} { P|Q \equiv Q|P \and P|0 \equiv P \and P|(Q|R) \equiv (P|Q)|R }
\end{mathpar}

\begin{mathpar}
  \inferrule* [lab=alpha-equivalence] {} { (x)P \equiv (y)P\{y/x\} \and y \not\in \freenames{P} }
\end{mathpar}

\begin{definition}
Then two processes, $P,Q$, are alpha-equivalent if $P = Q\{\vec{y}/\vec{x}\}$ for
some $\vec{x} \in \boundnames{Q},\vec{y} \in \boundnames{P}$, where $Q\{\vec{y}/\vec{x}\}$
denotes the capture-avoiding substitution of $\vec{y}$ for $\vec{x}$ in $Q$.
\end{definition}

\begin{definition}
  The {\em structural congruence} \cite{SangiorgiWalker} , $\equiv$,
  between processes is the least congruence containing
  alpha-equivalence, satisfying the abelian monoid laws
  (associativity, commutativity and $\pzero$ as identity) for parallel
  composition $|$ and for summation $+$.
\end{definition}

\subsection{Name equivalence}

We take name equivalence, written $\nameeq$, to be the smallest
equivalence relation generated by the following rules.

\begin{mathpar}
\inferrule*[lab=Quote-drop]
{ }
{ \quotep{@{x}} \nameeq x }

\inferrule*[lab=Struct-equiv]
{ P \scong Q }
{ \quotep{P} \nameeq \quotep{Q} }
\end{mathpar}

The astute reader will have noticed that the mutual recursion of names
and processes imposes a mutual recursion on alpha-equivalence and
structural equivalence via name-equivalence. Fortunately, all of this
works out pleasantly and we may calculate in the natural way, free of
concern. The reader interested in the details is referred to the
appendix \ref{appendix:rho_details}.

\subsection{Substitution}

We use $\Proc$ for the set of processes, $\QProc$ for the set of
names, and $\id{\{}\vec{y} / \vec{x} \id{\}}$ to denote partial maps,
$s : \QProc \rightarrow \QProc$. A map, $s$ lifts, uniquely, to a map
on process terms, $\widehat{s} : \Proc \rightarrow \Proc$ by the
following equations.

\begin{mathpar}
  (0) \psubstp{Q}{P} := 0 \\
  (R \juxtap S) \psubstp{Q}{P}
  :=    
  (R)\psubstp{Q}{P} \juxtap (S) \psubstp{Q}{P} \\
  (x?(y).R) \psubstp{Q}{P}    
  :=    
  (x)\substp{Q}{P} (z)\concat( (R \psubstn{z}{y}) \psubstp{Q}{P} ) \\
  (\lift{x}{R}) \psubstp{Q}{P}  
  :=
  \lift{(x)\substp{Q}{P}}{ R \psubstp{Q}{P} } \\
%   (\dropn{x})  \psubstp{Q}{P}       
%   := 
%   \left\{ 
%     \begin{array}{ccc} 
%       \dropn{\quotep{Q}} & & x \nameeq \quotep{P} \\
%       \dropn{x} & & otherwise \\
%     \end{array}
%   \right. 
  (\dropn{x})  \psubstp{Q}{P}       
  := 
  \left\{ 
    \begin{array}{ccc} 
      Q & & x \nameeq \quotep{P} \\
      \dropn{x} & & otherwise \\
    \end{array}
  \right.
\end{mathpar}
 

where

\begin{eqnarray}
  (x)\id{\{} \lpquote Q \rpquote / \lpquote P \rpquote \id{\}}            = 
  \left\{ 
    \begin{array}{ccc}
      \lpquote Q \rpquote & & x \nameeq \lpquote P \rpquote \\
      x & & otherwise \\
    \end{array}
  \right. \nonumber
\end{eqnarray}

and $z$ is chosen distinct from $\quotep{P}$, $\quotep{Q}$, the free
names in $Q$, and all the names in $R$. Our $\alpha$-equivalence will
be built in the standard way from this substitution.

\begin{remark}\label{rem:no_self_referential_names}
  One consequence of these definitions is that $\forall P. \quotep{P}
  \not\in \freenames{P}$.
\end{remark}

\subsection{ Dynamic quote: an example }

Anticipating something of what's to come, consider applying the
substitution, $\widehat{\id{\{}u / z \id{\}}}$, to the following pair
of processes, $\lift{w}{y!(z)}$ and $w[ \lpquote y!(z) \rpquote ]$.

\begin{eqnarray}
	\lift{w}{y!(z)}\widehat{\id{\{}u / z \id{\}}}
		& = &
		\lift{w}{y!(u)} \nonumber\\
	w[ \lpquote y!(z) \rpquote ] \widehat{ \id{\{}u / z \id{\}} }
		& = &
		w[ \lpquote y!(z) \rpquote ] \nonumber
\end{eqnarray}

Because the body of the process between quotes is impervious to
substitution, we get radically different answers. In fact, by
examining the first process in an input context,
e.g. $x?(z).\lift{w}{y!(z)}$, we see that the process under the lift
operator may be shaped by prefixed inputs binding a name inside it. In
this sense, the lift operator will be seen as a way to dynamically
construct processes before reifying them as names.

Finally equipped with these standard features we can present the
dynamics of the calculus.

\subsubsection{Operational semantics} 

Finally, we introduce the computational dynamics. What marks these
algebras as distinct from other more traditionally studied algebraic
structures, e.g. vector spaces or polynomial rings, is the manner in
which dynamics is captured. In traditional structures, dynamics is typically
expressed through morphisms between such structures, as in linear maps
between vector spaces or morphisms between rings. In algebras
associated with the semantics of computation, the dynamics is
expressed as part of the algebraic structure itself, through a
reduction reduction relation typically denoted by $\red$. Below, we
give a recursive presentation of this relation for the calculus used
in the encoding.

$\red \subseteq \pi \times \pi$
$\red : \pi \to \mathcal{P}(\pi)$

\begin{mathpar}
  \inferrule* [lab=Comm] { \textsf{match}( x_{src}, x_{trgt} ) } { x_{trgt}?(y)P \; | \; x_{src}!\langle {Q} \rangle \red P\{\quotep{Q}/y}\} }
  \and \\
  \inferrule* [lab=Par] {{P} \red {P}'} {{{P} | {Q}} \red {{P}' | {Q}}}
  \and
  \inferrule* [lab=Equiv]{{{P} \scong {P}'} \andalso {{P}' \red {Q}'} \andalso {{Q}' \scong {Q}}}{{P} \red {Q}}
\end{mathpar}

\begin{eqnarray*}
  match_{\equiv} (\quotep{P},\quotep{Q}) & := & P \equiv Q \\
  match_{\dagger}(\quotep{P},\quotep{Q}) & := & \forall R. P|Q \red^{*} R => R \red^{*} 0 \\
  match_{K}(\quotep{P},\quotep{Q}) & := & K \mbox{ for some context } K
\end{eqnarray*}

$u?(x)P | u!\langle Q \rangle \red P\{\quotep{Q}/x\}$

%We write $\wred$ for $\red^*$, and $P\red$ if $\exists Q $ such that $ P \red Q$.
We write $P\red$ if $\exists Q $ such that $ P \red Q$ and $P\not\red$, otherwise.

\section{Replication}

As mentioned before, it is known that replication (and hence
recursion) can be implemented in a higher-order process algebra
\cite{SangiorgiWalker}. As our first example of calculation with the
machinery thus far presented we give the construction explicitly in
the {\rhoc}.

\begin{eqnarray}
	D_{x} & := & \prefix{x}{y}{(\binpar{\outputp{x}{y}}{@{y}})} \nonumber\\
	\bangp_{x}{P} & := & \binpar{{x}!\langle{\binpar{D_{x}}{P}}\rangle}{D_{x}} \nonumber
\end{eqnarray}

\begin{eqnarray}
	\bangp_{x}{P} & & \nonumber\\
	=
	& {x}!\langle{(\prefix{x}{y}{(\outputp{x}{y} | @{y})) | P}}\rangle 
	      | \prefix{x}{y}{(\outputp{x}{y} | @{y})} & \nonumber\\
	\red
	& (\outputp{x}{y} | @{y})\substn{\quotep{(\prefix{x}{y}{(@{y} | \outputp{x}{y})) | P}}}{y} & \nonumber\\
	=
	& \outputp{x}{\quotep{(\prefix{x}{y}{(\outputp{x}{y} | @{y})) | P}}}
	  | {(\prefix{x}{y}{(\outputp{x}{y} | @{y})) | P}} & \nonumber\\
	\red
	& \ldots & \nonumber\\
	\red^*
	& P | P | \ldots & \nonumber
\end{eqnarray}

Of course, this encoding, as an implementation, runs away, unfolding
$\bangp{P}$ eagerly. A lazier and more implementable replication
operator, restricted to input-guarded processes, may be obtained as follows.

\begin{eqnarray}
\bangp{\prefix{u}{v}{P}} 
	:= 
	\binpar{\lift{x}{\prefix{u}{v}{(\binpar{D(x)}{P})}}}{D(x)} \nonumber
\end{eqnarray}

\begin{remark}
  Note that the lazier definition still does not deal with summation
  or mixed summation (i.e. sums over input and output). The reader is
  invited to construct definitions of replication that deal with these
  features. 

  Further, the definitions are parameterized in a name, $x$. Can you,
  gentle reader, make a definition that eliminates this parameter and
  guarantees no accidental interaction between the replication
  machinery and the process being replicated -- i.e. no accidental
  sharing of names used by the process to get its work done and the
  name(s) used by the replication to effect copying. This latter
  revision of the definition of replication is crucial to obtaining
  the expected identity $!!P \sim !P$.
\end{remark}

\begin{remark}\label{rem:paradoxical_combinator}
  The reader familiar with the lambda calculus will have noticed the
  similarity between $D$ and the paradoxical combinator.

  [Ed. note: the existence of this seems to suggest we have to be more
  restrictive on the set of processes and names we admit if we are to
  support no-cloning.]
\end{remark}

\subsubsection{Bisimulation}

The computational dynamics gives rise to another kind of equivalence,
the equivalence of computational behavior. As previously mentioned
this is typically captured \emph{via} some form of bisimulation.

% The notion we use in this paper is weak barbed bisimulation
% \cite{milner91polyadicpi}.

The notion we use in this paper is derived from weak barbed
bisimulation \cite{milner91polyadicpi}. 

\begin{definition}
An \emph{observation relation}, $\downarrow_{\mathcal N}$, over a set
of names, $\mathcal N$, is the smallest relation satisfying the rules
below.

\infrule[Out-barb]{y \in {\mathcal N}, \; x \nameeq y}
		  {\outputp{x}{v} \downarrow_{\mathcal N} x}
\infrule[Par-barb]{\mbox{$P\downarrow_{\mathcal N} x$ or $Q\downarrow_{\mathcal N} x$}}
		  {\binpar{P}{Q} \downarrow_{\mathcal N} x}

We write $P \Downarrow_{\mathcal N} x$ if there is $Q$ such that 
$P \wred Q$ and $Q \downarrow_{\mathcal N} x$.
\end{definition}

\begin{definition}
%\label{def.bbisim}
An  ${\mathcal N}$-\emph{barbed bisimulation} over a set of names, ${\mathcal N}$, is a symmetric binary relation 
${\mathcal S}_{\mathcal N}$ between agents such that $P\rel{S}_{\mathcal N}Q$ implies:
\begin{enumerate}
\item If $P \red P'$ then $Q \wred Q'$ and $P'\rel{S}_{\mathcal N} Q'$.
\item If $P\downarrow_{\mathcal N} x$, then $Q\Downarrow_{\mathcal N} x$.
\end{enumerate}
$P$ is ${\mathcal N}$-barbed bisimilar to $Q$, written
$P \wbbisim_{\mathcal N} Q$, if $P \rel{S}_{\mathcal N} Q$ for some ${\mathcal N}$-barbed bisimulation ${\mathcal S}_{\mathcal N}$.
\end{definition}

$\mathcal{R} \subseteq \pi \times \pi$

$P \mathcal{R} Q => \forall P'. P \red P' \Rightarrow \exists Q'. Q \red Q', P' \mathcal{R} Q'$

$P \vdash x \Rightarrow Q \vdash x$

\begin{mathpar}
  \inferrule*[lab=Out-barb]{x \nameeq y}{{y}!\langle{Q}\rangle \vdash x}
  \and
  \inferrule*[lab=Par-barb]{\mbox{$P\vdash x$ or $Q\vdash x$}}{\binpar{P}{Q} \vdash x}
\end{mathpar}

\subsubsection{Contexts}

One of the principle advantages of computational calculi like the
$\pi$-calculus is a well-defined notion of context,
contextual-equivalence and a correlation between
contextual-equivalence and notions of bisimulation. The notion of
context allows the decomposition of a process into (sub-)process and
its syntactic environment, its context. Thus, a context may be
thought of as a process with a ``hole'' (written $\Box$) in it. The
application of a context $M$ to a process $P$, written $M[P]$, is
tantamount to filling the hole in $M$ with $P$. In this paper we do
not need the full weight of this theory, but do make use of the notion
of context in the proof the main theorem. 

\begin{mathpar}
  \inferrule* [lab=summation] {} {{M_{M},M_{N}} \bc \Box \;|\; x.M_{A} \;|\; M_{M}+M_{N}}
  \and
  \inferrule* [lab=agent] {} {{M_{A}} \bc (\vec{x})M_{P} \;| \; \clift{P_0,\ldots,M_{P},\ldots,P_N}}
  \and \\
  \inferrule* [lab=process] {} {{M_{P}} \bc M_{N} \;| \;P|M_{P} }
\end{mathpar} 

\begin{mathpar}
  \inferrule* [lab=sychronization] {} {M_{N} \bc \Box \;|\; x?M_{F} \;|\; x!M_{C}}
  \and
  \inferrule* [lab=abstraction] {} {{M_{F}} \bc (x)M_{P} }
  \and
  \inferrule* [lab=concretion] {} {{M_{C}} \bc \langle M_{P} \rangle }
  \and \\
  \inferrule* [lab=process] {} {{M_{P}} \bc M_{N} \;| \;P|M_{P} }
\end{mathpar}

\begin{definition}[contextual application] Given a context $M$, and
  process $P$, we define the \emph{contextual application}, $M[P] :=
  M\{P/\Box\}$. That is, the contextual application of M to P is the
  substitution of $P$ for $\Box$ in $M$.
\end{definition}

$\meaningof{-} : L \to \mathcal{P}(\pi)$

\begin{mathpar}
  \inferrule* [lab=collection] {} {\meaningof{true} = \pi, \and \meaningof{~E} = \pi \setminus \meaningof{E}, \and \meaningof{E_{1} \& E_{2}} = \meaningof{E_{1}} \cap \meaningof{E_{2}}}
\end{mathpar}

\begin{mathpar}
  \inferrule* [lab=structure] {} {\meaningof{0} = \{ P \in \pi | P \equiv 0 \}, \and \\ \meaningof{E_1 | E_2} = \{ P \in \pi | P \equiv P_{1} | P_{2}, P_{1} \in \meaningof{E_{1}}, P_{2} \in \meaningof{E_2}\} }
\end{mathpar}

\begin{mathpar}
 \inferrule* [lab=behavior] {} {\meaningof{\langle a?b \rangle E} = \{ P \in \pi | P \equiv Q | u?(y)P', \\ \and \\\\ \and \\ \;\;\; u \in \meaningof{a}, \forall z.P'\{z/y\} \in \meaningof{E\{z/b\}}\}, \and \\ \meaningof{a!E} = \{ P \in \pi | P \equiv Q | x!\langle P' \rangle, x \in \meaningof{a} P' \in \meaningof{E}\} }
\end{mathpar}

\begin{mathpar}
 \inferrule* [lab=nominal] {} {\meaningof{\quotep{E}} = \{ \quotep{P} \in \quotep{\pi} | P \in \meaningof{E} \}, \and \meaningof{\quotep{P}} = \{ \quotep{Q} \in \quotep{\pi} | P \equiv Q \} \and \\ \meaningof{@\quotep{E}} = \{ P \in \pi | P \equiv @x, x \in \meaningof{E} \}}
\end{mathpar}

\begin{eqnarray*}
  \\
  \meaningof{-} : TS \to ST
\end{eqnarray*}

\begin{eqnarray*}
  \\
  L : TS \to ST
\end{eqnarray*}

\begin{eqnarray*}
  \\
  P \models E \iff P \in \meaningof{E}
\end{eqnarray*}

\begin{eqnarray*}
  P \approx_{L} Q \iff \forall E \in L. P \models E \iff Q \models E
\end{eqnarray*}

\begin{eqnarray*}
  P \approx_{K} Q
\end{eqnarray*}

\begin{eqnarray*}
  P \approx Q
\end{eqnarray*}

$\approx_{K} = \approx = \approx_{L}$

\subsubsection{Contextual duality}

Note that contexts extend the quotation operation to a family of
operations from processes to names. Given a context, $M$, we can
define a \emph{nominal context}, $\quotep{M}$ by $\quotep{M}[P] :=
\quotep{M[P]}$. To foreshadow what is to come we observe that these
operations enjoy a duality with processes very much like the duality
between vectors and maps from vectors to scalars.

Further, because the calculus is essentially higher-order, we have a
correspondence between contexts and processes. More specifically,
given a name $x$ and a context $M$ we can construct $M^{*}_{x}$ such
that 

\begin{mathpar}
  M^{*}_{x} | \lift{x}{P} \red M[P]
\end{mathpar}

namely,

\begin{mathpar}
  M^{*}_{x} := x?(u).M[\dropn{u}]
\end{mathpar}

The dependence of $M^{*}_{x}$ on a name makes it an abstraction, 

\begin{mathpar}
  M^{*} := (x)x?(u).M[\dropn{u}]
\end{mathpar}

\subsection{Additional notation}

It will sometimes be convenient to denote the process a name
quotes. We already have the notation $x = \quotep{P}$, but it will be
convenient to introduce an alternate notation, $\procn{x}$, when we
want to emphasize the connection to the use of the name. Note that, by
virtue of name equivalence, $\quotep{\procn{x}} \nameeq x$; so, the
notation is consistent with previous definitions.

Further, because names have structure it is possible to effect
substitutions on the basis of that structure. This means we need to
upgrade our notation for substitutions, which we accomplish by
adapting comprehension notation. Thus,

\begin{mathpar}
  P\{ y / x : x \in S \}
\end{mathpar}

is interpreted to mean the process derived from P by replacing (in a
capture-avoiding manner) each occurrence of $x$ in $S$ by $y$. For example,

\begin{mathpar}
  P\{ \quotep{\procn{x}|\procn{x}} / x : x \in \freenames{P} \}
\end{mathpar}

will replace each (occurrence) of a free name $x$ in $P$ by
$\quotep{\procn{x}|\procn{x}}$.

Also, we will avail ourselves of the notation $x^{L}$ and $x^{R}$ to
denote injections of a name into disjoint copies of the name
space. There are numerous ways to accomplish this. One example can be
found in \cite{MeredithR05}. This notation overloads to vectors of
names: $\vec{x}^{\pi} := (x_{i}^{\pi} \; : \; 0 \leq i < |\vec{x}| )$ where $\pi \in \{L,R\}$.

We also use $P^{\Box} := P|\Box$.

In \cite{MeredithR05} an interpretation of the new operator is
given. It turns out that there are several possible interpretations
all enjoying the requisite algebraic properties of the operator (see
\cite{milner91polyadicpi}). We will therefore make liberal use of
$(\nu\; \vec{x})P$.

% subsection the_syntax_and_semantics_of_the_notation_system (end)   

\input{qm2pi.qmops} 

\input{qm2pi.sterngerlach} 

\input{qm2pi.metric} 

% section concurrent_process_calculi (end)

%\input{qm2pi.proofsketch}

% section proof sketch (end)

%\input{qm2pi.slviaknots} 

% section spatial logic via knots (end)

\input{qm2pi.conclusion}

% section conclusion (end)

%\input{qm2pi.dtcodes} 

% section wiring algorithm (end)

\input{qm2pi.ack} 

% section acknowledgments (end)

\newpage


\bibliographystyle{plain}   
\bibliography{../../biblios/main.bib}

\input{qm2pi.rhodetails}

\end{document}

 

%\ifpdf
%\usepackage[pdftex]{graphicx}
%\else
%\usepackage{graphicx}
%\fi

 % \ifpdf
%  \usepackage{pdfsync}
%  \if


%\title{Brief Article}
%\author{David F. Snyder}
%\author{L.G. Meredith}

%\address{Dept. of Math., Texas State University--San Marcos, San Marcos, TX 78666}
       
\pagestyle{empty}


\begin{document}

\lstset{language=[Objective]Caml,frame=shadowbox}

\documentclass[12pt]{llncs}
%\documentclass{jktr}

\usepackage[pdftex]{hyperref}                   
\usepackage {listings}
\usepackage {mathpartir}
\usepackage{bcprules}
%\usepackage{listings}
                       
\usepackage{graphicx} 
%\usepackage[margins=2.5cm,nohead,nofoot]{geometry}
%\usepackage{geometry}
\usepackage{amsfonts}
\usepackage{amstext}
\usepackage{latexsym}
\usepackage{amssymb}
\usepackage{color}


%\include{myPreamble}
\include{qm2pi.local} 

%\ifpdf
%\usepackage[pdftex]{graphicx}
%\else
%\usepackage{graphicx}
%\fi

 % \ifpdf
%  \usepackage{pdfsync}
%  \if


%\title{Brief Article}
%\author{David F. Snyder}
%\author{L.G. Meredith}

%\address{Dept. of Math., Texas State University--San Marcos, San Marcos, TX 78666}
       
\pagestyle{empty}


\begin{document}

\lstset{language=[Objective]Caml,frame=shadowbox}

\input{qm2pi.front}

% section front matter (end)

\input{qm2pi.intro} 
 
% section introduction (end)

% \input{qm2pi.knotations} 

% section notation (end)

\input{qm2pi.process.calculi} 

% section concurrent_process_calculi_and_spatial_logics_ (end)
    
%\input{qm2pi.knots2pi} 

%\input{qm2pi.trefoil} 

%\input{qm2pi.mainthm} 

% subsection basic_interpretation (end)

%\input{qm2pi.rho.presentation} 
\subsection{The syntax and semantics of the notation system}\label{sub:the_syntax_and_semantics_of_the_notation_system} % (fold)

We now summarize a technical presentation of the calculus that
embodies our theory of dynamics. The typical presentation of such a
calculus follows the style of giving generators and relations on
them. The grammar, below, describing term constructors, freely
generates the set of processes, $\Proc$. This set is then quotiented
by a relation known as structural congruence and it is over this set
that the notion of dynamics is expressed. This presentation is
essentially that of \cite{MeredithR05} with the addition of
polyadicity and summation. For readability we have relegated some of
the technical subtleties to an appendix.

\subsubsection{Process grammar}\label{subsub:process_grammar}

\begin{mathpar}
  \inferrule* [lab=synchronization] {} {{M} \bc \pzero \;|\; x?F \;|\; x!C }
  \and
  \inferrule* [lab=abstraction] {} {{F} \bc (x)P}
  \and
  \inferrule* [lab=concretion] {} {{C} \bc \langle Q \rangle}
  \and
  \inferrule* [lab=process] {} {{P,Q} \bc M \;| \;P|Q \;|\; @{x}}
  \and
  \inferrule* [lab=name] {} {{x} \bc \quotep{P}}
\end{mathpar} 

Note that $\vec{x}$ (resp. $\vec{P}$) denotes a vector of names
(resp. processes) of length $|\vec{x}|$ (resp. $|\vec{P}|$). We adopt
the following useful abbreviations.

\begin{mathpar}
   x?(\vec{y}).P := x.(\vec{y})P \and  x\clift{\vec{P}} := x.\clift{\vec{P}}
   \and x!(y) := \lift{x}{\dropn{y}}
   \and \Pi_{i=0}^{n-1}P_i := P_0 | \ldots | P_{n-1}
\end{mathpar}

\subsubsection{Structural congruence}

\paragraph{Free and bound names and alpha-equivalence.} At the
core of structural equivalence is alpha-equivalence which identifies
process that are the same up to a change of variable. Formally, we
recognize the distinction between free and bound names. The free names
of a process, $\freenames{P}$, may be calculated recursively as
follows:

\begin{mathpar}
\freenames{\pzero} := \emptyset
  \and \\
  \freenames{x?(y).P} := \{ x \} \cup (\freenames{P} \setminus \{ y \})
  \and 
  \freenames{x!\langle P \rangle} := \{ x \} \cup \{ P \} 
  \and \\
  \freenames{P|Q} := \freenames{P} \cup \freenames{Q}
  \and \\
  \freenames{@{x}} := \{ x \}
\end{mathpar}

$\pi$
$\quotep{\pi}$

$\freenames{-} : \pi \to \mathcal{P}(\quotep{\pi})$

\begin{eqnarray*}
  \freenames{\pzero} & := & \emptyset \\
  \freenames{x?(y).P} & := & \{ x \} \cup (\freenames{P} \setminus \{ y \}) \\
  \freenames{x!\langle P \rangle} & := & \{ x \} \cup \{ P \} \\
  \freenames{P|Q} & := & \freenames{P} \cup \freenames{Q} \\
  \freenames{\dropn{x}} & := & \{ x \}
\end{eqnarray*}

The bound names of a process, $\boundnames{P}$, are those names occurring in $P$
that are not free. For example, in $x?(y).0$, the name $x$ is free, while $y$ is bound.

\begin{mathpar}
  \inferrule* [lab=monoidal-laws] {} { P|Q \equiv Q|P \and P|0 \equiv P \and P|(Q|R) \equiv (P|Q)|R }
\end{mathpar}

\begin{mathpar}
  \inferrule* [lab=alpha-equivalence] {} { (x)P \equiv (y)P\{y/x\} \and y \not\in \freenames{P} }
\end{mathpar}

\begin{definition}
Then two processes, $P,Q$, are alpha-equivalent if $P = Q\{\vec{y}/\vec{x}\}$ for
some $\vec{x} \in \boundnames{Q},\vec{y} \in \boundnames{P}$, where $Q\{\vec{y}/\vec{x}\}$
denotes the capture-avoiding substitution of $\vec{y}$ for $\vec{x}$ in $Q$.
\end{definition}

\begin{definition}
  The {\em structural congruence} \cite{SangiorgiWalker} , $\equiv$,
  between processes is the least congruence containing
  alpha-equivalence, satisfying the abelian monoid laws
  (associativity, commutativity and $\pzero$ as identity) for parallel
  composition $|$ and for summation $+$.
\end{definition}

\subsection{Name equivalence}

We take name equivalence, written $\nameeq$, to be the smallest
equivalence relation generated by the following rules.

\begin{mathpar}
\inferrule*[lab=Quote-drop]
{ }
{ \quotep{@{x}} \nameeq x }

\inferrule*[lab=Struct-equiv]
{ P \scong Q }
{ \quotep{P} \nameeq \quotep{Q} }
\end{mathpar}

The astute reader will have noticed that the mutual recursion of names
and processes imposes a mutual recursion on alpha-equivalence and
structural equivalence via name-equivalence. Fortunately, all of this
works out pleasantly and we may calculate in the natural way, free of
concern. The reader interested in the details is referred to the
appendix \ref{appendix:rho_details}.

\subsection{Substitution}

We use $\Proc$ for the set of processes, $\QProc$ for the set of
names, and $\id{\{}\vec{y} / \vec{x} \id{\}}$ to denote partial maps,
$s : \QProc \rightarrow \QProc$. A map, $s$ lifts, uniquely, to a map
on process terms, $\widehat{s} : \Proc \rightarrow \Proc$ by the
following equations.

\begin{mathpar}
  (0) \psubstp{Q}{P} := 0 \\
  (R \juxtap S) \psubstp{Q}{P}
  :=    
  (R)\psubstp{Q}{P} \juxtap (S) \psubstp{Q}{P} \\
  (x?(y).R) \psubstp{Q}{P}    
  :=    
  (x)\substp{Q}{P} (z)\concat( (R \psubstn{z}{y}) \psubstp{Q}{P} ) \\
  (\lift{x}{R}) \psubstp{Q}{P}  
  :=
  \lift{(x)\substp{Q}{P}}{ R \psubstp{Q}{P} } \\
%   (\dropn{x})  \psubstp{Q}{P}       
%   := 
%   \left\{ 
%     \begin{array}{ccc} 
%       \dropn{\quotep{Q}} & & x \nameeq \quotep{P} \\
%       \dropn{x} & & otherwise \\
%     \end{array}
%   \right. 
  (\dropn{x})  \psubstp{Q}{P}       
  := 
  \left\{ 
    \begin{array}{ccc} 
      Q & & x \nameeq \quotep{P} \\
      \dropn{x} & & otherwise \\
    \end{array}
  \right.
\end{mathpar}
 

where

\begin{eqnarray}
  (x)\id{\{} \lpquote Q \rpquote / \lpquote P \rpquote \id{\}}            = 
  \left\{ 
    \begin{array}{ccc}
      \lpquote Q \rpquote & & x \nameeq \lpquote P \rpquote \\
      x & & otherwise \\
    \end{array}
  \right. \nonumber
\end{eqnarray}

and $z$ is chosen distinct from $\quotep{P}$, $\quotep{Q}$, the free
names in $Q$, and all the names in $R$. Our $\alpha$-equivalence will
be built in the standard way from this substitution.

\begin{remark}\label{rem:no_self_referential_names}
  One consequence of these definitions is that $\forall P. \quotep{P}
  \not\in \freenames{P}$.
\end{remark}

\subsection{ Dynamic quote: an example }

Anticipating something of what's to come, consider applying the
substitution, $\widehat{\id{\{}u / z \id{\}}}$, to the following pair
of processes, $\lift{w}{y!(z)}$ and $w[ \lpquote y!(z) \rpquote ]$.

\begin{eqnarray}
	\lift{w}{y!(z)}\widehat{\id{\{}u / z \id{\}}}
		& = &
		\lift{w}{y!(u)} \nonumber\\
	w[ \lpquote y!(z) \rpquote ] \widehat{ \id{\{}u / z \id{\}} }
		& = &
		w[ \lpquote y!(z) \rpquote ] \nonumber
\end{eqnarray}

Because the body of the process between quotes is impervious to
substitution, we get radically different answers. In fact, by
examining the first process in an input context,
e.g. $x?(z).\lift{w}{y!(z)}$, we see that the process under the lift
operator may be shaped by prefixed inputs binding a name inside it. In
this sense, the lift operator will be seen as a way to dynamically
construct processes before reifying them as names.

Finally equipped with these standard features we can present the
dynamics of the calculus.

\subsubsection{Operational semantics} 

Finally, we introduce the computational dynamics. What marks these
algebras as distinct from other more traditionally studied algebraic
structures, e.g. vector spaces or polynomial rings, is the manner in
which dynamics is captured. In traditional structures, dynamics is typically
expressed through morphisms between such structures, as in linear maps
between vector spaces or morphisms between rings. In algebras
associated with the semantics of computation, the dynamics is
expressed as part of the algebraic structure itself, through a
reduction reduction relation typically denoted by $\red$. Below, we
give a recursive presentation of this relation for the calculus used
in the encoding.

$\red \subseteq \pi \times \pi$
$\red : \pi \to \mathcal{P}(\pi)$

\begin{mathpar}
  \inferrule* [lab=Comm] { \textsf{match}( x_{src}, x_{trgt} ) } { x_{trgt}?(y)P \; | \; x_{src}!\langle {Q} \rangle \red P\{\quotep{Q}/y}\} }
  \and \\
  \inferrule* [lab=Par] {{P} \red {P}'} {{{P} | {Q}} \red {{P}' | {Q}}}
  \and
  \inferrule* [lab=Equiv]{{{P} \scong {P}'} \andalso {{P}' \red {Q}'} \andalso {{Q}' \scong {Q}}}{{P} \red {Q}}
\end{mathpar}

\begin{eqnarray*}
  match_{\equiv} (\quotep{P},\quotep{Q}) & := & P \equiv Q \\
  match_{\dagger}(\quotep{P},\quotep{Q}) & := & \forall R. P|Q \red^{*} R => R \red^{*} 0 \\
  match_{K}(\quotep{P},\quotep{Q}) & := & K \mbox{ for some context } K
\end{eqnarray*}

$u?(x)P | u!\langle Q \rangle \red P\{\quotep{Q}/x\}$

%We write $\wred$ for $\red^*$, and $P\red$ if $\exists Q $ such that $ P \red Q$.
We write $P\red$ if $\exists Q $ such that $ P \red Q$ and $P\not\red$, otherwise.

\section{Replication}

As mentioned before, it is known that replication (and hence
recursion) can be implemented in a higher-order process algebra
\cite{SangiorgiWalker}. As our first example of calculation with the
machinery thus far presented we give the construction explicitly in
the {\rhoc}.

\begin{eqnarray}
	D_{x} & := & \prefix{x}{y}{(\binpar{\outputp{x}{y}}{@{y}})} \nonumber\\
	\bangp_{x}{P} & := & \binpar{{x}!\langle{\binpar{D_{x}}{P}}\rangle}{D_{x}} \nonumber
\end{eqnarray}

\begin{eqnarray}
	\bangp_{x}{P} & & \nonumber\\
	=
	& {x}!\langle{(\prefix{x}{y}{(\outputp{x}{y} | @{y})) | P}}\rangle 
	      | \prefix{x}{y}{(\outputp{x}{y} | @{y})} & \nonumber\\
	\red
	& (\outputp{x}{y} | @{y})\substn{\quotep{(\prefix{x}{y}{(@{y} | \outputp{x}{y})) | P}}}{y} & \nonumber\\
	=
	& \outputp{x}{\quotep{(\prefix{x}{y}{(\outputp{x}{y} | @{y})) | P}}}
	  | {(\prefix{x}{y}{(\outputp{x}{y} | @{y})) | P}} & \nonumber\\
	\red
	& \ldots & \nonumber\\
	\red^*
	& P | P | \ldots & \nonumber
\end{eqnarray}

Of course, this encoding, as an implementation, runs away, unfolding
$\bangp{P}$ eagerly. A lazier and more implementable replication
operator, restricted to input-guarded processes, may be obtained as follows.

\begin{eqnarray}
\bangp{\prefix{u}{v}{P}} 
	:= 
	\binpar{\lift{x}{\prefix{u}{v}{(\binpar{D(x)}{P})}}}{D(x)} \nonumber
\end{eqnarray}

\begin{remark}
  Note that the lazier definition still does not deal with summation
  or mixed summation (i.e. sums over input and output). The reader is
  invited to construct definitions of replication that deal with these
  features. 

  Further, the definitions are parameterized in a name, $x$. Can you,
  gentle reader, make a definition that eliminates this parameter and
  guarantees no accidental interaction between the replication
  machinery and the process being replicated -- i.e. no accidental
  sharing of names used by the process to get its work done and the
  name(s) used by the replication to effect copying. This latter
  revision of the definition of replication is crucial to obtaining
  the expected identity $!!P \sim !P$.
\end{remark}

\begin{remark}\label{rem:paradoxical_combinator}
  The reader familiar with the lambda calculus will have noticed the
  similarity between $D$ and the paradoxical combinator.

  [Ed. note: the existence of this seems to suggest we have to be more
  restrictive on the set of processes and names we admit if we are to
  support no-cloning.]
\end{remark}

\subsubsection{Bisimulation}

The computational dynamics gives rise to another kind of equivalence,
the equivalence of computational behavior. As previously mentioned
this is typically captured \emph{via} some form of bisimulation.

% The notion we use in this paper is weak barbed bisimulation
% \cite{milner91polyadicpi}.

The notion we use in this paper is derived from weak barbed
bisimulation \cite{milner91polyadicpi}. 

\begin{definition}
An \emph{observation relation}, $\downarrow_{\mathcal N}$, over a set
of names, $\mathcal N$, is the smallest relation satisfying the rules
below.

\infrule[Out-barb]{y \in {\mathcal N}, \; x \nameeq y}
		  {\outputp{x}{v} \downarrow_{\mathcal N} x}
\infrule[Par-barb]{\mbox{$P\downarrow_{\mathcal N} x$ or $Q\downarrow_{\mathcal N} x$}}
		  {\binpar{P}{Q} \downarrow_{\mathcal N} x}

We write $P \Downarrow_{\mathcal N} x$ if there is $Q$ such that 
$P \wred Q$ and $Q \downarrow_{\mathcal N} x$.
\end{definition}

\begin{definition}
%\label{def.bbisim}
An  ${\mathcal N}$-\emph{barbed bisimulation} over a set of names, ${\mathcal N}$, is a symmetric binary relation 
${\mathcal S}_{\mathcal N}$ between agents such that $P\rel{S}_{\mathcal N}Q$ implies:
\begin{enumerate}
\item If $P \red P'$ then $Q \wred Q'$ and $P'\rel{S}_{\mathcal N} Q'$.
\item If $P\downarrow_{\mathcal N} x$, then $Q\Downarrow_{\mathcal N} x$.
\end{enumerate}
$P$ is ${\mathcal N}$-barbed bisimilar to $Q$, written
$P \wbbisim_{\mathcal N} Q$, if $P \rel{S}_{\mathcal N} Q$ for some ${\mathcal N}$-barbed bisimulation ${\mathcal S}_{\mathcal N}$.
\end{definition}

$\mathcal{R} \subseteq \pi \times \pi$

$P \mathcal{R} Q => \forall P'. P \red P' \Rightarrow \exists Q'. Q \red Q', P' \mathcal{R} Q'$

$P \vdash x \Rightarrow Q \vdash x$

\begin{mathpar}
  \inferrule*[lab=Out-barb]{x \nameeq y}{{y}!\langle{Q}\rangle \vdash x}
  \and
  \inferrule*[lab=Par-barb]{\mbox{$P\vdash x$ or $Q\vdash x$}}{\binpar{P}{Q} \vdash x}
\end{mathpar}

\subsubsection{Contexts}

One of the principle advantages of computational calculi like the
$\pi$-calculus is a well-defined notion of context,
contextual-equivalence and a correlation between
contextual-equivalence and notions of bisimulation. The notion of
context allows the decomposition of a process into (sub-)process and
its syntactic environment, its context. Thus, a context may be
thought of as a process with a ``hole'' (written $\Box$) in it. The
application of a context $M$ to a process $P$, written $M[P]$, is
tantamount to filling the hole in $M$ with $P$. In this paper we do
not need the full weight of this theory, but do make use of the notion
of context in the proof the main theorem. 

\begin{mathpar}
  \inferrule* [lab=summation] {} {{M_{M},M_{N}} \bc \Box \;|\; x.M_{A} \;|\; M_{M}+M_{N}}
  \and
  \inferrule* [lab=agent] {} {{M_{A}} \bc (\vec{x})M_{P} \;| \; \clift{P_0,\ldots,M_{P},\ldots,P_N}}
  \and \\
  \inferrule* [lab=process] {} {{M_{P}} \bc M_{N} \;| \;P|M_{P} }
\end{mathpar} 

\begin{mathpar}
  \inferrule* [lab=sychronization] {} {M_{N} \bc \Box \;|\; x?M_{F} \;|\; x!M_{C}}
  \and
  \inferrule* [lab=abstraction] {} {{M_{F}} \bc (x)M_{P} }
  \and
  \inferrule* [lab=concretion] {} {{M_{C}} \bc \langle M_{P} \rangle }
  \and \\
  \inferrule* [lab=process] {} {{M_{P}} \bc M_{N} \;| \;P|M_{P} }
\end{mathpar}

\begin{definition}[contextual application] Given a context $M$, and
  process $P$, we define the \emph{contextual application}, $M[P] :=
  M\{P/\Box\}$. That is, the contextual application of M to P is the
  substitution of $P$ for $\Box$ in $M$.
\end{definition}

$\meaningof{-} : L \to \mathcal{P}(\pi)$

\begin{mathpar}
  \inferrule* [lab=collection] {} {\meaningof{true} = \pi, \and \meaningof{~E} = \pi \setminus \meaningof{E}, \and \meaningof{E_{1} \& E_{2}} = \meaningof{E_{1}} \cap \meaningof{E_{2}}}
\end{mathpar}

\begin{mathpar}
  \inferrule* [lab=structure] {} {\meaningof{0} = \{ P \in \pi | P \equiv 0 \}, \and \\ \meaningof{E_1 | E_2} = \{ P \in \pi | P \equiv P_{1} | P_{2}, P_{1} \in \meaningof{E_{1}}, P_{2} \in \meaningof{E_2}\} }
\end{mathpar}

\begin{mathpar}
 \inferrule* [lab=behavior] {} {\meaningof{\langle a?b \rangle E} = \{ P \in \pi | P \equiv Q | u?(y)P', \\ \and \\\\ \and \\ \;\;\; u \in \meaningof{a}, \forall z.P'\{z/y\} \in \meaningof{E\{z/b\}}\}, \and \\ \meaningof{a!E} = \{ P \in \pi | P \equiv Q | x!\langle P' \rangle, x \in \meaningof{a} P' \in \meaningof{E}\} }
\end{mathpar}

\begin{mathpar}
 \inferrule* [lab=nominal] {} {\meaningof{\quotep{E}} = \{ \quotep{P} \in \quotep{\pi} | P \in \meaningof{E} \}, \and \meaningof{\quotep{P}} = \{ \quotep{Q} \in \quotep{\pi} | P \equiv Q \} \and \\ \meaningof{@\quotep{E}} = \{ P \in \pi | P \equiv @x, x \in \meaningof{E} \}}
\end{mathpar}

\begin{eqnarray*}
  \\
  \meaningof{-} : TS \to ST
\end{eqnarray*}

\begin{eqnarray*}
  \\
  L : TS \to ST
\end{eqnarray*}

\begin{eqnarray*}
  \\
  P \models E \iff P \in \meaningof{E}
\end{eqnarray*}

\begin{eqnarray*}
  P \approx_{L} Q \iff \forall E \in L. P \models E \iff Q \models E
\end{eqnarray*}

\begin{eqnarray*}
  P \approx_{K} Q
\end{eqnarray*}

\begin{eqnarray*}
  P \approx Q
\end{eqnarray*}

$\approx_{K} = \approx = \approx_{L}$

\subsubsection{Contextual duality}

Note that contexts extend the quotation operation to a family of
operations from processes to names. Given a context, $M$, we can
define a \emph{nominal context}, $\quotep{M}$ by $\quotep{M}[P] :=
\quotep{M[P]}$. To foreshadow what is to come we observe that these
operations enjoy a duality with processes very much like the duality
between vectors and maps from vectors to scalars.

Further, because the calculus is essentially higher-order, we have a
correspondence between contexts and processes. More specifically,
given a name $x$ and a context $M$ we can construct $M^{*}_{x}$ such
that 

\begin{mathpar}
  M^{*}_{x} | \lift{x}{P} \red M[P]
\end{mathpar}

namely,

\begin{mathpar}
  M^{*}_{x} := x?(u).M[\dropn{u}]
\end{mathpar}

The dependence of $M^{*}_{x}$ on a name makes it an abstraction, 

\begin{mathpar}
  M^{*} := (x)x?(u).M[\dropn{u}]
\end{mathpar}

\subsection{Additional notation}

It will sometimes be convenient to denote the process a name
quotes. We already have the notation $x = \quotep{P}$, but it will be
convenient to introduce an alternate notation, $\procn{x}$, when we
want to emphasize the connection to the use of the name. Note that, by
virtue of name equivalence, $\quotep{\procn{x}} \nameeq x$; so, the
notation is consistent with previous definitions.

Further, because names have structure it is possible to effect
substitutions on the basis of that structure. This means we need to
upgrade our notation for substitutions, which we accomplish by
adapting comprehension notation. Thus,

\begin{mathpar}
  P\{ y / x : x \in S \}
\end{mathpar}

is interpreted to mean the process derived from P by replacing (in a
capture-avoiding manner) each occurrence of $x$ in $S$ by $y$. For example,

\begin{mathpar}
  P\{ \quotep{\procn{x}|\procn{x}} / x : x \in \freenames{P} \}
\end{mathpar}

will replace each (occurrence) of a free name $x$ in $P$ by
$\quotep{\procn{x}|\procn{x}}$.

Also, we will avail ourselves of the notation $x^{L}$ and $x^{R}$ to
denote injections of a name into disjoint copies of the name
space. There are numerous ways to accomplish this. One example can be
found in \cite{MeredithR05}. This notation overloads to vectors of
names: $\vec{x}^{\pi} := (x_{i}^{\pi} \; : \; 0 \leq i < |\vec{x}| )$ where $\pi \in \{L,R\}$.

We also use $P^{\Box} := P|\Box$.

In \cite{MeredithR05} an interpretation of the new operator is
given. It turns out that there are several possible interpretations
all enjoying the requisite algebraic properties of the operator (see
\cite{milner91polyadicpi}). We will therefore make liberal use of
$(\nu\; \vec{x})P$.

% subsection the_syntax_and_semantics_of_the_notation_system (end)   

\input{qm2pi.qmops} 

\input{qm2pi.sterngerlach} 

\input{qm2pi.metric} 

% section concurrent_process_calculi (end)

%\input{qm2pi.proofsketch}

% section proof sketch (end)

%\input{qm2pi.slviaknots} 

% section spatial logic via knots (end)

\input{qm2pi.conclusion}

% section conclusion (end)

%\input{qm2pi.dtcodes} 

% section wiring algorithm (end)

\input{qm2pi.ack} 

% section acknowledgments (end)

\newpage


\bibliographystyle{plain}   
\bibliography{../../biblios/main.bib}

\input{qm2pi.rhodetails}

\end{document}



% section front matter (end)

\section{Introduction}\label{sec:introduction} % (fold)
In this draft of the material i am going to have to dispense with the
usual writing conventions adopted in papers on these topics. i'm going
to have adopt whatever tone i need at the time i'm writing up the
calculations. Sometimes this may be very conversational; others it may
be the barest mathematical grunts; others still it may be that i have
lifted text from one of my other papers because the exposition of some
point was better said there. i hope that my readers are not unduly put
out by this decision. i'm not doing this to flout convention or be
rebellious. i find these calculations very technically challenging. To
keep everything going technically, something has to give; i have to
let go of some cognitive burden. So, the academic writing style --
with all of its trade-offs in terms of facilitating technical
communication -- is what i'm letting go of. Perhaps subsequent drafts
can be tightened and polished, but for now, i'm going to speak as if
we were sitting together in a coffee shop with a laptop, wifi and a
pad of paper and a pencil.

So, here's what i have to say. We -- you and i, comfortably ensconced
in our coffee shop and well-equipped with our tools -- can realize and
carry out the calculations of quantum mechanics over a very different
formal theory of dynamics, a formal theory of dynamics that
corresponds to a theory of concurrent computation with
\emph{reflection}. It has the advantage that the underlying theory is
already `quantized', but supports analogues all of the continuuous
operations. Strikingly, this underlying theory has recently been
connected with a notion of metric that we can show, by calculating
together, coincides with the metric induced by the inner product.

There are a lot of reasons why you might be interested in seeing
calculations of this form. Here's why i'm interested. For the past
several centuries there has been no competitor to the ``Newtonian''
account of dynamics. As a result the predominant share of accounts of
dynamical systems and situations have had to be formulated in terms of
the Newtonian machinery. i view this as an intellectually dangerous
position to occupy. Everything, despite it's intrinsic shape, turns
into a nail to be hit with this hammer. Recently, however, the theory
of computation has matured to the point where we have candidates for
theories of dynamics that offer very different perspective on
reasoning about dynamical systems and situations. Testing these
candidates against very successful accounts of dynamical situations,
like quantum mechanics, is going to give us some sense of how mature
they are and some measure of the quality of these accounts of
dynamics.

\subsection{Summary of contributions and outline of paper}

So, we're going to develop an interpretation of the operations of
quantum mechanics normally interpreted by Hilbert spaces and
operators. We're going to do this over a theory of computation. Note
that this is very different than the usual quantum computation program
which develops notions of computation over quantum mechanics. Rather,
we are developing a story that aligns with Wheeler's slogan: It from
Bit. To do this we will first provide an account of the theory of
computation at play here. Then we will dive into a calculation-driven
interpretation of the operations of quantum mechanics.

The reason we take this approach is that -- until very recently --
there hasn't been an axiomatic account of quantum mechanics. As a
result there has been no sharp delineation of the mathematical theory
supporting interpretation of the physical theory and the physical
theory, itself. So, ambient features of the maths are free to be
exploited (or supressed) without a real accounting of their physical
relevance. There is no sharp statement ``here's the physical theory''
qua \emph{theory} and ``here's the mathematical interpretation''
enabling a judgment of how faithful the interpretation is -- apart
from experimental observation. When there is an axiomatic account we
can judge how well a given mathematical formalism supports an
interpretation of the axioms, independent of
experimentation. Likewise, we can judge how well we have captured our
physical evidence and experience with our axiomatics, independent of
any specific mathematical implementation, with accidental detail that
may or may not have physical significance. 

In lieu of a fully fleshed out and vetted axiomatic account of quantum
mechanics, interpreting the operational notions in service of modeling
physical systems will have to suffice. In other words, we are not in
the business of providing a model of Hilbert spaces and operators. We
are in the business of providing a model of quantum mechanics because
we are motivated by testing our notions of dynamics against physical
theory; and, the predictive calculations of the physical theory must
serve as the best formulation -- shy of a fully fleshed out axiomatic
account -- of the physical theory itself (as they have for scientific
theories since time immemorial). Put another way, despite a
whole-hearted commitment to an It-from-Bit ontology, we are firmly
aligned with the shut-up-and-calculate camp as the best way to obtain
results either from the physical perspective or as a quality assurance
measure of our fledgling theory of dynamics.

In detail, we present a reflective process calculus. Then we develop
intuitive correspondences between the notions available in this
calculus and the usual physical notions supporting quantum mechanical
calculations. Thus, 

\begin{table}[htp]
  \center{
    \fbox{
      \begin{tabular}{c|c}
        quantum mechanics & process calculus \\
        \hline
        scalar & name \\
        state vector & process \\
        dual & contextual duals \\
        matrix & formal sums of process-context-dual pairs \\
        orthogonality & process annihilation \\
        inner product & execution-formula + quoting
      \end{tabular}
    }
  }
  \caption{QM - process calculi correspondences}
\end{table}

Then we tighten up these intuitions to operational definitions. We
employ the Dirac notation as the best proxy we can find for an
abstract syntax of the quantum mechanical notions. The definitions we
develop put us in contact with equational constraints coming from the
theory that we demonstrate the definitions and calculations satisfy.

This puts us in a position to shut up and calculate for the
Stern-Gerlach experimental set up, showing how these predictive
calculations become calculations on processes in our theory of a
reflective process calculus.

Penultimately, we demonstrate that the notion of metric coming from
the inner product coincides with the notion of metric available from
the theory of bisimulation. This demonstration gives us the right to
think of space as arising from behavior. Finally, we consider where we
might go from the new vantage point we have obtained.

% section introduction (end) 
 
% section introduction (end)

% \documentclass[12pt]{llncs}
%\documentclass{jktr}

\usepackage[pdftex]{hyperref}                   
\usepackage {listings}
\usepackage {mathpartir}
\usepackage{bcprules}
%\usepackage{listings}
                       
\usepackage{graphicx} 
%\usepackage[margins=2.5cm,nohead,nofoot]{geometry}
%\usepackage{geometry}
\usepackage{amsfonts}
\usepackage{amstext}
\usepackage{latexsym}
\usepackage{amssymb}
\usepackage{color}


%\include{myPreamble}
\include{qm2pi.local} 

%\ifpdf
%\usepackage[pdftex]{graphicx}
%\else
%\usepackage{graphicx}
%\fi

 % \ifpdf
%  \usepackage{pdfsync}
%  \if


%\title{Brief Article}
%\author{David F. Snyder}
%\author{L.G. Meredith}

%\address{Dept. of Math., Texas State University--San Marcos, San Marcos, TX 78666}
       
\pagestyle{empty}


\begin{document}

\lstset{language=[Objective]Caml,frame=shadowbox}

\input{qm2pi.front}

% section front matter (end)

\input{qm2pi.intro} 
 
% section introduction (end)

% \input{qm2pi.knotations} 

% section notation (end)

\input{qm2pi.process.calculi} 

% section concurrent_process_calculi_and_spatial_logics_ (end)
    
%\input{qm2pi.knots2pi} 

%\input{qm2pi.trefoil} 

%\input{qm2pi.mainthm} 

% subsection basic_interpretation (end)

%\input{qm2pi.rho.presentation} 
\subsection{The syntax and semantics of the notation system}\label{sub:the_syntax_and_semantics_of_the_notation_system} % (fold)

We now summarize a technical presentation of the calculus that
embodies our theory of dynamics. The typical presentation of such a
calculus follows the style of giving generators and relations on
them. The grammar, below, describing term constructors, freely
generates the set of processes, $\Proc$. This set is then quotiented
by a relation known as structural congruence and it is over this set
that the notion of dynamics is expressed. This presentation is
essentially that of \cite{MeredithR05} with the addition of
polyadicity and summation. For readability we have relegated some of
the technical subtleties to an appendix.

\subsubsection{Process grammar}\label{subsub:process_grammar}

\begin{mathpar}
  \inferrule* [lab=synchronization] {} {{M} \bc \pzero \;|\; x?F \;|\; x!C }
  \and
  \inferrule* [lab=abstraction] {} {{F} \bc (x)P}
  \and
  \inferrule* [lab=concretion] {} {{C} \bc \langle Q \rangle}
  \and
  \inferrule* [lab=process] {} {{P,Q} \bc M \;| \;P|Q \;|\; @{x}}
  \and
  \inferrule* [lab=name] {} {{x} \bc \quotep{P}}
\end{mathpar} 

Note that $\vec{x}$ (resp. $\vec{P}$) denotes a vector of names
(resp. processes) of length $|\vec{x}|$ (resp. $|\vec{P}|$). We adopt
the following useful abbreviations.

\begin{mathpar}
   x?(\vec{y}).P := x.(\vec{y})P \and  x\clift{\vec{P}} := x.\clift{\vec{P}}
   \and x!(y) := \lift{x}{\dropn{y}}
   \and \Pi_{i=0}^{n-1}P_i := P_0 | \ldots | P_{n-1}
\end{mathpar}

\subsubsection{Structural congruence}

\paragraph{Free and bound names and alpha-equivalence.} At the
core of structural equivalence is alpha-equivalence which identifies
process that are the same up to a change of variable. Formally, we
recognize the distinction between free and bound names. The free names
of a process, $\freenames{P}$, may be calculated recursively as
follows:

\begin{mathpar}
\freenames{\pzero} := \emptyset
  \and \\
  \freenames{x?(y).P} := \{ x \} \cup (\freenames{P} \setminus \{ y \})
  \and 
  \freenames{x!\langle P \rangle} := \{ x \} \cup \{ P \} 
  \and \\
  \freenames{P|Q} := \freenames{P} \cup \freenames{Q}
  \and \\
  \freenames{@{x}} := \{ x \}
\end{mathpar}

$\pi$
$\quotep{\pi}$

$\freenames{-} : \pi \to \mathcal{P}(\quotep{\pi})$

\begin{eqnarray*}
  \freenames{\pzero} & := & \emptyset \\
  \freenames{x?(y).P} & := & \{ x \} \cup (\freenames{P} \setminus \{ y \}) \\
  \freenames{x!\langle P \rangle} & := & \{ x \} \cup \{ P \} \\
  \freenames{P|Q} & := & \freenames{P} \cup \freenames{Q} \\
  \freenames{\dropn{x}} & := & \{ x \}
\end{eqnarray*}

The bound names of a process, $\boundnames{P}$, are those names occurring in $P$
that are not free. For example, in $x?(y).0$, the name $x$ is free, while $y$ is bound.

\begin{mathpar}
  \inferrule* [lab=monoidal-laws] {} { P|Q \equiv Q|P \and P|0 \equiv P \and P|(Q|R) \equiv (P|Q)|R }
\end{mathpar}

\begin{mathpar}
  \inferrule* [lab=alpha-equivalence] {} { (x)P \equiv (y)P\{y/x\} \and y \not\in \freenames{P} }
\end{mathpar}

\begin{definition}
Then two processes, $P,Q$, are alpha-equivalent if $P = Q\{\vec{y}/\vec{x}\}$ for
some $\vec{x} \in \boundnames{Q},\vec{y} \in \boundnames{P}$, where $Q\{\vec{y}/\vec{x}\}$
denotes the capture-avoiding substitution of $\vec{y}$ for $\vec{x}$ in $Q$.
\end{definition}

\begin{definition}
  The {\em structural congruence} \cite{SangiorgiWalker} , $\equiv$,
  between processes is the least congruence containing
  alpha-equivalence, satisfying the abelian monoid laws
  (associativity, commutativity and $\pzero$ as identity) for parallel
  composition $|$ and for summation $+$.
\end{definition}

\subsection{Name equivalence}

We take name equivalence, written $\nameeq$, to be the smallest
equivalence relation generated by the following rules.

\begin{mathpar}
\inferrule*[lab=Quote-drop]
{ }
{ \quotep{@{x}} \nameeq x }

\inferrule*[lab=Struct-equiv]
{ P \scong Q }
{ \quotep{P} \nameeq \quotep{Q} }
\end{mathpar}

The astute reader will have noticed that the mutual recursion of names
and processes imposes a mutual recursion on alpha-equivalence and
structural equivalence via name-equivalence. Fortunately, all of this
works out pleasantly and we may calculate in the natural way, free of
concern. The reader interested in the details is referred to the
appendix \ref{appendix:rho_details}.

\subsection{Substitution}

We use $\Proc$ for the set of processes, $\QProc$ for the set of
names, and $\id{\{}\vec{y} / \vec{x} \id{\}}$ to denote partial maps,
$s : \QProc \rightarrow \QProc$. A map, $s$ lifts, uniquely, to a map
on process terms, $\widehat{s} : \Proc \rightarrow \Proc$ by the
following equations.

\begin{mathpar}
  (0) \psubstp{Q}{P} := 0 \\
  (R \juxtap S) \psubstp{Q}{P}
  :=    
  (R)\psubstp{Q}{P} \juxtap (S) \psubstp{Q}{P} \\
  (x?(y).R) \psubstp{Q}{P}    
  :=    
  (x)\substp{Q}{P} (z)\concat( (R \psubstn{z}{y}) \psubstp{Q}{P} ) \\
  (\lift{x}{R}) \psubstp{Q}{P}  
  :=
  \lift{(x)\substp{Q}{P}}{ R \psubstp{Q}{P} } \\
%   (\dropn{x})  \psubstp{Q}{P}       
%   := 
%   \left\{ 
%     \begin{array}{ccc} 
%       \dropn{\quotep{Q}} & & x \nameeq \quotep{P} \\
%       \dropn{x} & & otherwise \\
%     \end{array}
%   \right. 
  (\dropn{x})  \psubstp{Q}{P}       
  := 
  \left\{ 
    \begin{array}{ccc} 
      Q & & x \nameeq \quotep{P} \\
      \dropn{x} & & otherwise \\
    \end{array}
  \right.
\end{mathpar}
 

where

\begin{eqnarray}
  (x)\id{\{} \lpquote Q \rpquote / \lpquote P \rpquote \id{\}}            = 
  \left\{ 
    \begin{array}{ccc}
      \lpquote Q \rpquote & & x \nameeq \lpquote P \rpquote \\
      x & & otherwise \\
    \end{array}
  \right. \nonumber
\end{eqnarray}

and $z$ is chosen distinct from $\quotep{P}$, $\quotep{Q}$, the free
names in $Q$, and all the names in $R$. Our $\alpha$-equivalence will
be built in the standard way from this substitution.

\begin{remark}\label{rem:no_self_referential_names}
  One consequence of these definitions is that $\forall P. \quotep{P}
  \not\in \freenames{P}$.
\end{remark}

\subsection{ Dynamic quote: an example }

Anticipating something of what's to come, consider applying the
substitution, $\widehat{\id{\{}u / z \id{\}}}$, to the following pair
of processes, $\lift{w}{y!(z)}$ and $w[ \lpquote y!(z) \rpquote ]$.

\begin{eqnarray}
	\lift{w}{y!(z)}\widehat{\id{\{}u / z \id{\}}}
		& = &
		\lift{w}{y!(u)} \nonumber\\
	w[ \lpquote y!(z) \rpquote ] \widehat{ \id{\{}u / z \id{\}} }
		& = &
		w[ \lpquote y!(z) \rpquote ] \nonumber
\end{eqnarray}

Because the body of the process between quotes is impervious to
substitution, we get radically different answers. In fact, by
examining the first process in an input context,
e.g. $x?(z).\lift{w}{y!(z)}$, we see that the process under the lift
operator may be shaped by prefixed inputs binding a name inside it. In
this sense, the lift operator will be seen as a way to dynamically
construct processes before reifying them as names.

Finally equipped with these standard features we can present the
dynamics of the calculus.

\subsubsection{Operational semantics} 

Finally, we introduce the computational dynamics. What marks these
algebras as distinct from other more traditionally studied algebraic
structures, e.g. vector spaces or polynomial rings, is the manner in
which dynamics is captured. In traditional structures, dynamics is typically
expressed through morphisms between such structures, as in linear maps
between vector spaces or morphisms between rings. In algebras
associated with the semantics of computation, the dynamics is
expressed as part of the algebraic structure itself, through a
reduction reduction relation typically denoted by $\red$. Below, we
give a recursive presentation of this relation for the calculus used
in the encoding.

$\red \subseteq \pi \times \pi$
$\red : \pi \to \mathcal{P}(\pi)$

\begin{mathpar}
  \inferrule* [lab=Comm] { \textsf{match}( x_{src}, x_{trgt} ) } { x_{trgt}?(y)P \; | \; x_{src}!\langle {Q} \rangle \red P\{\quotep{Q}/y}\} }
  \and \\
  \inferrule* [lab=Par] {{P} \red {P}'} {{{P} | {Q}} \red {{P}' | {Q}}}
  \and
  \inferrule* [lab=Equiv]{{{P} \scong {P}'} \andalso {{P}' \red {Q}'} \andalso {{Q}' \scong {Q}}}{{P} \red {Q}}
\end{mathpar}

\begin{eqnarray*}
  match_{\equiv} (\quotep{P},\quotep{Q}) & := & P \equiv Q \\
  match_{\dagger}(\quotep{P},\quotep{Q}) & := & \forall R. P|Q \red^{*} R => R \red^{*} 0 \\
  match_{K}(\quotep{P},\quotep{Q}) & := & K \mbox{ for some context } K
\end{eqnarray*}

$u?(x)P | u!\langle Q \rangle \red P\{\quotep{Q}/x\}$

%We write $\wred$ for $\red^*$, and $P\red$ if $\exists Q $ such that $ P \red Q$.
We write $P\red$ if $\exists Q $ such that $ P \red Q$ and $P\not\red$, otherwise.

\section{Replication}

As mentioned before, it is known that replication (and hence
recursion) can be implemented in a higher-order process algebra
\cite{SangiorgiWalker}. As our first example of calculation with the
machinery thus far presented we give the construction explicitly in
the {\rhoc}.

\begin{eqnarray}
	D_{x} & := & \prefix{x}{y}{(\binpar{\outputp{x}{y}}{@{y}})} \nonumber\\
	\bangp_{x}{P} & := & \binpar{{x}!\langle{\binpar{D_{x}}{P}}\rangle}{D_{x}} \nonumber
\end{eqnarray}

\begin{eqnarray}
	\bangp_{x}{P} & & \nonumber\\
	=
	& {x}!\langle{(\prefix{x}{y}{(\outputp{x}{y} | @{y})) | P}}\rangle 
	      | \prefix{x}{y}{(\outputp{x}{y} | @{y})} & \nonumber\\
	\red
	& (\outputp{x}{y} | @{y})\substn{\quotep{(\prefix{x}{y}{(@{y} | \outputp{x}{y})) | P}}}{y} & \nonumber\\
	=
	& \outputp{x}{\quotep{(\prefix{x}{y}{(\outputp{x}{y} | @{y})) | P}}}
	  | {(\prefix{x}{y}{(\outputp{x}{y} | @{y})) | P}} & \nonumber\\
	\red
	& \ldots & \nonumber\\
	\red^*
	& P | P | \ldots & \nonumber
\end{eqnarray}

Of course, this encoding, as an implementation, runs away, unfolding
$\bangp{P}$ eagerly. A lazier and more implementable replication
operator, restricted to input-guarded processes, may be obtained as follows.

\begin{eqnarray}
\bangp{\prefix{u}{v}{P}} 
	:= 
	\binpar{\lift{x}{\prefix{u}{v}{(\binpar{D(x)}{P})}}}{D(x)} \nonumber
\end{eqnarray}

\begin{remark}
  Note that the lazier definition still does not deal with summation
  or mixed summation (i.e. sums over input and output). The reader is
  invited to construct definitions of replication that deal with these
  features. 

  Further, the definitions are parameterized in a name, $x$. Can you,
  gentle reader, make a definition that eliminates this parameter and
  guarantees no accidental interaction between the replication
  machinery and the process being replicated -- i.e. no accidental
  sharing of names used by the process to get its work done and the
  name(s) used by the replication to effect copying. This latter
  revision of the definition of replication is crucial to obtaining
  the expected identity $!!P \sim !P$.
\end{remark}

\begin{remark}\label{rem:paradoxical_combinator}
  The reader familiar with the lambda calculus will have noticed the
  similarity between $D$ and the paradoxical combinator.

  [Ed. note: the existence of this seems to suggest we have to be more
  restrictive on the set of processes and names we admit if we are to
  support no-cloning.]
\end{remark}

\subsubsection{Bisimulation}

The computational dynamics gives rise to another kind of equivalence,
the equivalence of computational behavior. As previously mentioned
this is typically captured \emph{via} some form of bisimulation.

% The notion we use in this paper is weak barbed bisimulation
% \cite{milner91polyadicpi}.

The notion we use in this paper is derived from weak barbed
bisimulation \cite{milner91polyadicpi}. 

\begin{definition}
An \emph{observation relation}, $\downarrow_{\mathcal N}$, over a set
of names, $\mathcal N$, is the smallest relation satisfying the rules
below.

\infrule[Out-barb]{y \in {\mathcal N}, \; x \nameeq y}
		  {\outputp{x}{v} \downarrow_{\mathcal N} x}
\infrule[Par-barb]{\mbox{$P\downarrow_{\mathcal N} x$ or $Q\downarrow_{\mathcal N} x$}}
		  {\binpar{P}{Q} \downarrow_{\mathcal N} x}

We write $P \Downarrow_{\mathcal N} x$ if there is $Q$ such that 
$P \wred Q$ and $Q \downarrow_{\mathcal N} x$.
\end{definition}

\begin{definition}
%\label{def.bbisim}
An  ${\mathcal N}$-\emph{barbed bisimulation} over a set of names, ${\mathcal N}$, is a symmetric binary relation 
${\mathcal S}_{\mathcal N}$ between agents such that $P\rel{S}_{\mathcal N}Q$ implies:
\begin{enumerate}
\item If $P \red P'$ then $Q \wred Q'$ and $P'\rel{S}_{\mathcal N} Q'$.
\item If $P\downarrow_{\mathcal N} x$, then $Q\Downarrow_{\mathcal N} x$.
\end{enumerate}
$P$ is ${\mathcal N}$-barbed bisimilar to $Q$, written
$P \wbbisim_{\mathcal N} Q$, if $P \rel{S}_{\mathcal N} Q$ for some ${\mathcal N}$-barbed bisimulation ${\mathcal S}_{\mathcal N}$.
\end{definition}

$\mathcal{R} \subseteq \pi \times \pi$

$P \mathcal{R} Q => \forall P'. P \red P' \Rightarrow \exists Q'. Q \red Q', P' \mathcal{R} Q'$

$P \vdash x \Rightarrow Q \vdash x$

\begin{mathpar}
  \inferrule*[lab=Out-barb]{x \nameeq y}{{y}!\langle{Q}\rangle \vdash x}
  \and
  \inferrule*[lab=Par-barb]{\mbox{$P\vdash x$ or $Q\vdash x$}}{\binpar{P}{Q} \vdash x}
\end{mathpar}

\subsubsection{Contexts}

One of the principle advantages of computational calculi like the
$\pi$-calculus is a well-defined notion of context,
contextual-equivalence and a correlation between
contextual-equivalence and notions of bisimulation. The notion of
context allows the decomposition of a process into (sub-)process and
its syntactic environment, its context. Thus, a context may be
thought of as a process with a ``hole'' (written $\Box$) in it. The
application of a context $M$ to a process $P$, written $M[P]$, is
tantamount to filling the hole in $M$ with $P$. In this paper we do
not need the full weight of this theory, but do make use of the notion
of context in the proof the main theorem. 

\begin{mathpar}
  \inferrule* [lab=summation] {} {{M_{M},M_{N}} \bc \Box \;|\; x.M_{A} \;|\; M_{M}+M_{N}}
  \and
  \inferrule* [lab=agent] {} {{M_{A}} \bc (\vec{x})M_{P} \;| \; \clift{P_0,\ldots,M_{P},\ldots,P_N}}
  \and \\
  \inferrule* [lab=process] {} {{M_{P}} \bc M_{N} \;| \;P|M_{P} }
\end{mathpar} 

\begin{mathpar}
  \inferrule* [lab=sychronization] {} {M_{N} \bc \Box \;|\; x?M_{F} \;|\; x!M_{C}}
  \and
  \inferrule* [lab=abstraction] {} {{M_{F}} \bc (x)M_{P} }
  \and
  \inferrule* [lab=concretion] {} {{M_{C}} \bc \langle M_{P} \rangle }
  \and \\
  \inferrule* [lab=process] {} {{M_{P}} \bc M_{N} \;| \;P|M_{P} }
\end{mathpar}

\begin{definition}[contextual application] Given a context $M$, and
  process $P$, we define the \emph{contextual application}, $M[P] :=
  M\{P/\Box\}$. That is, the contextual application of M to P is the
  substitution of $P$ for $\Box$ in $M$.
\end{definition}

$\meaningof{-} : L \to \mathcal{P}(\pi)$

\begin{mathpar}
  \inferrule* [lab=collection] {} {\meaningof{true} = \pi, \and \meaningof{~E} = \pi \setminus \meaningof{E}, \and \meaningof{E_{1} \& E_{2}} = \meaningof{E_{1}} \cap \meaningof{E_{2}}}
\end{mathpar}

\begin{mathpar}
  \inferrule* [lab=structure] {} {\meaningof{0} = \{ P \in \pi | P \equiv 0 \}, \and \\ \meaningof{E_1 | E_2} = \{ P \in \pi | P \equiv P_{1} | P_{2}, P_{1} \in \meaningof{E_{1}}, P_{2} \in \meaningof{E_2}\} }
\end{mathpar}

\begin{mathpar}
 \inferrule* [lab=behavior] {} {\meaningof{\langle a?b \rangle E} = \{ P \in \pi | P \equiv Q | u?(y)P', \\ \and \\\\ \and \\ \;\;\; u \in \meaningof{a}, \forall z.P'\{z/y\} \in \meaningof{E\{z/b\}}\}, \and \\ \meaningof{a!E} = \{ P \in \pi | P \equiv Q | x!\langle P' \rangle, x \in \meaningof{a} P' \in \meaningof{E}\} }
\end{mathpar}

\begin{mathpar}
 \inferrule* [lab=nominal] {} {\meaningof{\quotep{E}} = \{ \quotep{P} \in \quotep{\pi} | P \in \meaningof{E} \}, \and \meaningof{\quotep{P}} = \{ \quotep{Q} \in \quotep{\pi} | P \equiv Q \} \and \\ \meaningof{@\quotep{E}} = \{ P \in \pi | P \equiv @x, x \in \meaningof{E} \}}
\end{mathpar}

\begin{eqnarray*}
  \\
  \meaningof{-} : TS \to ST
\end{eqnarray*}

\begin{eqnarray*}
  \\
  L : TS \to ST
\end{eqnarray*}

\begin{eqnarray*}
  \\
  P \models E \iff P \in \meaningof{E}
\end{eqnarray*}

\begin{eqnarray*}
  P \approx_{L} Q \iff \forall E \in L. P \models E \iff Q \models E
\end{eqnarray*}

\begin{eqnarray*}
  P \approx_{K} Q
\end{eqnarray*}

\begin{eqnarray*}
  P \approx Q
\end{eqnarray*}

$\approx_{K} = \approx = \approx_{L}$

\subsubsection{Contextual duality}

Note that contexts extend the quotation operation to a family of
operations from processes to names. Given a context, $M$, we can
define a \emph{nominal context}, $\quotep{M}$ by $\quotep{M}[P] :=
\quotep{M[P]}$. To foreshadow what is to come we observe that these
operations enjoy a duality with processes very much like the duality
between vectors and maps from vectors to scalars.

Further, because the calculus is essentially higher-order, we have a
correspondence between contexts and processes. More specifically,
given a name $x$ and a context $M$ we can construct $M^{*}_{x}$ such
that 

\begin{mathpar}
  M^{*}_{x} | \lift{x}{P} \red M[P]
\end{mathpar}

namely,

\begin{mathpar}
  M^{*}_{x} := x?(u).M[\dropn{u}]
\end{mathpar}

The dependence of $M^{*}_{x}$ on a name makes it an abstraction, 

\begin{mathpar}
  M^{*} := (x)x?(u).M[\dropn{u}]
\end{mathpar}

\subsection{Additional notation}

It will sometimes be convenient to denote the process a name
quotes. We already have the notation $x = \quotep{P}$, but it will be
convenient to introduce an alternate notation, $\procn{x}$, when we
want to emphasize the connection to the use of the name. Note that, by
virtue of name equivalence, $\quotep{\procn{x}} \nameeq x$; so, the
notation is consistent with previous definitions.

Further, because names have structure it is possible to effect
substitutions on the basis of that structure. This means we need to
upgrade our notation for substitutions, which we accomplish by
adapting comprehension notation. Thus,

\begin{mathpar}
  P\{ y / x : x \in S \}
\end{mathpar}

is interpreted to mean the process derived from P by replacing (in a
capture-avoiding manner) each occurrence of $x$ in $S$ by $y$. For example,

\begin{mathpar}
  P\{ \quotep{\procn{x}|\procn{x}} / x : x \in \freenames{P} \}
\end{mathpar}

will replace each (occurrence) of a free name $x$ in $P$ by
$\quotep{\procn{x}|\procn{x}}$.

Also, we will avail ourselves of the notation $x^{L}$ and $x^{R}$ to
denote injections of a name into disjoint copies of the name
space. There are numerous ways to accomplish this. One example can be
found in \cite{MeredithR05}. This notation overloads to vectors of
names: $\vec{x}^{\pi} := (x_{i}^{\pi} \; : \; 0 \leq i < |\vec{x}| )$ where $\pi \in \{L,R\}$.

We also use $P^{\Box} := P|\Box$.

In \cite{MeredithR05} an interpretation of the new operator is
given. It turns out that there are several possible interpretations
all enjoying the requisite algebraic properties of the operator (see
\cite{milner91polyadicpi}). We will therefore make liberal use of
$(\nu\; \vec{x})P$.

% subsection the_syntax_and_semantics_of_the_notation_system (end)   

\input{qm2pi.qmops} 

\input{qm2pi.sterngerlach} 

\input{qm2pi.metric} 

% section concurrent_process_calculi (end)

%\input{qm2pi.proofsketch}

% section proof sketch (end)

%\input{qm2pi.slviaknots} 

% section spatial logic via knots (end)

\input{qm2pi.conclusion}

% section conclusion (end)

%\input{qm2pi.dtcodes} 

% section wiring algorithm (end)

\input{qm2pi.ack} 

% section acknowledgments (end)

\newpage


\bibliographystyle{plain}   
\bibliography{../../biblios/main.bib}

\input{qm2pi.rhodetails}

\end{document}

 

% section notation (end)

\input{qm2pi.process.calculi} 

% section concurrent_process_calculi_and_spatial_logics_ (end)
    
%\documentclass[12pt]{llncs}
%\documentclass{jktr}

\usepackage[pdftex]{hyperref}                   
\usepackage {listings}
\usepackage {mathpartir}
\usepackage{bcprules}
%\usepackage{listings}
                       
\usepackage{graphicx} 
%\usepackage[margins=2.5cm,nohead,nofoot]{geometry}
%\usepackage{geometry}
\usepackage{amsfonts}
\usepackage{amstext}
\usepackage{latexsym}
\usepackage{amssymb}
\usepackage{color}


%\include{myPreamble}
\include{qm2pi.local} 

%\ifpdf
%\usepackage[pdftex]{graphicx}
%\else
%\usepackage{graphicx}
%\fi

 % \ifpdf
%  \usepackage{pdfsync}
%  \if


%\title{Brief Article}
%\author{David F. Snyder}
%\author{L.G. Meredith}

%\address{Dept. of Math., Texas State University--San Marcos, San Marcos, TX 78666}
       
\pagestyle{empty}


\begin{document}

\lstset{language=[Objective]Caml,frame=shadowbox}

\input{qm2pi.front}

% section front matter (end)

\input{qm2pi.intro} 
 
% section introduction (end)

% \input{qm2pi.knotations} 

% section notation (end)

\input{qm2pi.process.calculi} 

% section concurrent_process_calculi_and_spatial_logics_ (end)
    
%\input{qm2pi.knots2pi} 

%\input{qm2pi.trefoil} 

%\input{qm2pi.mainthm} 

% subsection basic_interpretation (end)

%\input{qm2pi.rho.presentation} 
\subsection{The syntax and semantics of the notation system}\label{sub:the_syntax_and_semantics_of_the_notation_system} % (fold)

We now summarize a technical presentation of the calculus that
embodies our theory of dynamics. The typical presentation of such a
calculus follows the style of giving generators and relations on
them. The grammar, below, describing term constructors, freely
generates the set of processes, $\Proc$. This set is then quotiented
by a relation known as structural congruence and it is over this set
that the notion of dynamics is expressed. This presentation is
essentially that of \cite{MeredithR05} with the addition of
polyadicity and summation. For readability we have relegated some of
the technical subtleties to an appendix.

\subsubsection{Process grammar}\label{subsub:process_grammar}

\begin{mathpar}
  \inferrule* [lab=synchronization] {} {{M} \bc \pzero \;|\; x?F \;|\; x!C }
  \and
  \inferrule* [lab=abstraction] {} {{F} \bc (x)P}
  \and
  \inferrule* [lab=concretion] {} {{C} \bc \langle Q \rangle}
  \and
  \inferrule* [lab=process] {} {{P,Q} \bc M \;| \;P|Q \;|\; @{x}}
  \and
  \inferrule* [lab=name] {} {{x} \bc \quotep{P}}
\end{mathpar} 

Note that $\vec{x}$ (resp. $\vec{P}$) denotes a vector of names
(resp. processes) of length $|\vec{x}|$ (resp. $|\vec{P}|$). We adopt
the following useful abbreviations.

\begin{mathpar}
   x?(\vec{y}).P := x.(\vec{y})P \and  x\clift{\vec{P}} := x.\clift{\vec{P}}
   \and x!(y) := \lift{x}{\dropn{y}}
   \and \Pi_{i=0}^{n-1}P_i := P_0 | \ldots | P_{n-1}
\end{mathpar}

\subsubsection{Structural congruence}

\paragraph{Free and bound names and alpha-equivalence.} At the
core of structural equivalence is alpha-equivalence which identifies
process that are the same up to a change of variable. Formally, we
recognize the distinction between free and bound names. The free names
of a process, $\freenames{P}$, may be calculated recursively as
follows:

\begin{mathpar}
\freenames{\pzero} := \emptyset
  \and \\
  \freenames{x?(y).P} := \{ x \} \cup (\freenames{P} \setminus \{ y \})
  \and 
  \freenames{x!\langle P \rangle} := \{ x \} \cup \{ P \} 
  \and \\
  \freenames{P|Q} := \freenames{P} \cup \freenames{Q}
  \and \\
  \freenames{@{x}} := \{ x \}
\end{mathpar}

$\pi$
$\quotep{\pi}$

$\freenames{-} : \pi \to \mathcal{P}(\quotep{\pi})$

\begin{eqnarray*}
  \freenames{\pzero} & := & \emptyset \\
  \freenames{x?(y).P} & := & \{ x \} \cup (\freenames{P} \setminus \{ y \}) \\
  \freenames{x!\langle P \rangle} & := & \{ x \} \cup \{ P \} \\
  \freenames{P|Q} & := & \freenames{P} \cup \freenames{Q} \\
  \freenames{\dropn{x}} & := & \{ x \}
\end{eqnarray*}

The bound names of a process, $\boundnames{P}$, are those names occurring in $P$
that are not free. For example, in $x?(y).0$, the name $x$ is free, while $y$ is bound.

\begin{mathpar}
  \inferrule* [lab=monoidal-laws] {} { P|Q \equiv Q|P \and P|0 \equiv P \and P|(Q|R) \equiv (P|Q)|R }
\end{mathpar}

\begin{mathpar}
  \inferrule* [lab=alpha-equivalence] {} { (x)P \equiv (y)P\{y/x\} \and y \not\in \freenames{P} }
\end{mathpar}

\begin{definition}
Then two processes, $P,Q$, are alpha-equivalent if $P = Q\{\vec{y}/\vec{x}\}$ for
some $\vec{x} \in \boundnames{Q},\vec{y} \in \boundnames{P}$, where $Q\{\vec{y}/\vec{x}\}$
denotes the capture-avoiding substitution of $\vec{y}$ for $\vec{x}$ in $Q$.
\end{definition}

\begin{definition}
  The {\em structural congruence} \cite{SangiorgiWalker} , $\equiv$,
  between processes is the least congruence containing
  alpha-equivalence, satisfying the abelian monoid laws
  (associativity, commutativity and $\pzero$ as identity) for parallel
  composition $|$ and for summation $+$.
\end{definition}

\subsection{Name equivalence}

We take name equivalence, written $\nameeq$, to be the smallest
equivalence relation generated by the following rules.

\begin{mathpar}
\inferrule*[lab=Quote-drop]
{ }
{ \quotep{@{x}} \nameeq x }

\inferrule*[lab=Struct-equiv]
{ P \scong Q }
{ \quotep{P} \nameeq \quotep{Q} }
\end{mathpar}

The astute reader will have noticed that the mutual recursion of names
and processes imposes a mutual recursion on alpha-equivalence and
structural equivalence via name-equivalence. Fortunately, all of this
works out pleasantly and we may calculate in the natural way, free of
concern. The reader interested in the details is referred to the
appendix \ref{appendix:rho_details}.

\subsection{Substitution}

We use $\Proc$ for the set of processes, $\QProc$ for the set of
names, and $\id{\{}\vec{y} / \vec{x} \id{\}}$ to denote partial maps,
$s : \QProc \rightarrow \QProc$. A map, $s$ lifts, uniquely, to a map
on process terms, $\widehat{s} : \Proc \rightarrow \Proc$ by the
following equations.

\begin{mathpar}
  (0) \psubstp{Q}{P} := 0 \\
  (R \juxtap S) \psubstp{Q}{P}
  :=    
  (R)\psubstp{Q}{P} \juxtap (S) \psubstp{Q}{P} \\
  (x?(y).R) \psubstp{Q}{P}    
  :=    
  (x)\substp{Q}{P} (z)\concat( (R \psubstn{z}{y}) \psubstp{Q}{P} ) \\
  (\lift{x}{R}) \psubstp{Q}{P}  
  :=
  \lift{(x)\substp{Q}{P}}{ R \psubstp{Q}{P} } \\
%   (\dropn{x})  \psubstp{Q}{P}       
%   := 
%   \left\{ 
%     \begin{array}{ccc} 
%       \dropn{\quotep{Q}} & & x \nameeq \quotep{P} \\
%       \dropn{x} & & otherwise \\
%     \end{array}
%   \right. 
  (\dropn{x})  \psubstp{Q}{P}       
  := 
  \left\{ 
    \begin{array}{ccc} 
      Q & & x \nameeq \quotep{P} \\
      \dropn{x} & & otherwise \\
    \end{array}
  \right.
\end{mathpar}
 

where

\begin{eqnarray}
  (x)\id{\{} \lpquote Q \rpquote / \lpquote P \rpquote \id{\}}            = 
  \left\{ 
    \begin{array}{ccc}
      \lpquote Q \rpquote & & x \nameeq \lpquote P \rpquote \\
      x & & otherwise \\
    \end{array}
  \right. \nonumber
\end{eqnarray}

and $z$ is chosen distinct from $\quotep{P}$, $\quotep{Q}$, the free
names in $Q$, and all the names in $R$. Our $\alpha$-equivalence will
be built in the standard way from this substitution.

\begin{remark}\label{rem:no_self_referential_names}
  One consequence of these definitions is that $\forall P. \quotep{P}
  \not\in \freenames{P}$.
\end{remark}

\subsection{ Dynamic quote: an example }

Anticipating something of what's to come, consider applying the
substitution, $\widehat{\id{\{}u / z \id{\}}}$, to the following pair
of processes, $\lift{w}{y!(z)}$ and $w[ \lpquote y!(z) \rpquote ]$.

\begin{eqnarray}
	\lift{w}{y!(z)}\widehat{\id{\{}u / z \id{\}}}
		& = &
		\lift{w}{y!(u)} \nonumber\\
	w[ \lpquote y!(z) \rpquote ] \widehat{ \id{\{}u / z \id{\}} }
		& = &
		w[ \lpquote y!(z) \rpquote ] \nonumber
\end{eqnarray}

Because the body of the process between quotes is impervious to
substitution, we get radically different answers. In fact, by
examining the first process in an input context,
e.g. $x?(z).\lift{w}{y!(z)}$, we see that the process under the lift
operator may be shaped by prefixed inputs binding a name inside it. In
this sense, the lift operator will be seen as a way to dynamically
construct processes before reifying them as names.

Finally equipped with these standard features we can present the
dynamics of the calculus.

\subsubsection{Operational semantics} 

Finally, we introduce the computational dynamics. What marks these
algebras as distinct from other more traditionally studied algebraic
structures, e.g. vector spaces or polynomial rings, is the manner in
which dynamics is captured. In traditional structures, dynamics is typically
expressed through morphisms between such structures, as in linear maps
between vector spaces or morphisms between rings. In algebras
associated with the semantics of computation, the dynamics is
expressed as part of the algebraic structure itself, through a
reduction reduction relation typically denoted by $\red$. Below, we
give a recursive presentation of this relation for the calculus used
in the encoding.

$\red \subseteq \pi \times \pi$
$\red : \pi \to \mathcal{P}(\pi)$

\begin{mathpar}
  \inferrule* [lab=Comm] { \textsf{match}( x_{src}, x_{trgt} ) } { x_{trgt}?(y)P \; | \; x_{src}!\langle {Q} \rangle \red P\{\quotep{Q}/y}\} }
  \and \\
  \inferrule* [lab=Par] {{P} \red {P}'} {{{P} | {Q}} \red {{P}' | {Q}}}
  \and
  \inferrule* [lab=Equiv]{{{P} \scong {P}'} \andalso {{P}' \red {Q}'} \andalso {{Q}' \scong {Q}}}{{P} \red {Q}}
\end{mathpar}

\begin{eqnarray*}
  match_{\equiv} (\quotep{P},\quotep{Q}) & := & P \equiv Q \\
  match_{\dagger}(\quotep{P},\quotep{Q}) & := & \forall R. P|Q \red^{*} R => R \red^{*} 0 \\
  match_{K}(\quotep{P},\quotep{Q}) & := & K \mbox{ for some context } K
\end{eqnarray*}

$u?(x)P | u!\langle Q \rangle \red P\{\quotep{Q}/x\}$

%We write $\wred$ for $\red^*$, and $P\red$ if $\exists Q $ such that $ P \red Q$.
We write $P\red$ if $\exists Q $ such that $ P \red Q$ and $P\not\red$, otherwise.

\section{Replication}

As mentioned before, it is known that replication (and hence
recursion) can be implemented in a higher-order process algebra
\cite{SangiorgiWalker}. As our first example of calculation with the
machinery thus far presented we give the construction explicitly in
the {\rhoc}.

\begin{eqnarray}
	D_{x} & := & \prefix{x}{y}{(\binpar{\outputp{x}{y}}{@{y}})} \nonumber\\
	\bangp_{x}{P} & := & \binpar{{x}!\langle{\binpar{D_{x}}{P}}\rangle}{D_{x}} \nonumber
\end{eqnarray}

\begin{eqnarray}
	\bangp_{x}{P} & & \nonumber\\
	=
	& {x}!\langle{(\prefix{x}{y}{(\outputp{x}{y} | @{y})) | P}}\rangle 
	      | \prefix{x}{y}{(\outputp{x}{y} | @{y})} & \nonumber\\
	\red
	& (\outputp{x}{y} | @{y})\substn{\quotep{(\prefix{x}{y}{(@{y} | \outputp{x}{y})) | P}}}{y} & \nonumber\\
	=
	& \outputp{x}{\quotep{(\prefix{x}{y}{(\outputp{x}{y} | @{y})) | P}}}
	  | {(\prefix{x}{y}{(\outputp{x}{y} | @{y})) | P}} & \nonumber\\
	\red
	& \ldots & \nonumber\\
	\red^*
	& P | P | \ldots & \nonumber
\end{eqnarray}

Of course, this encoding, as an implementation, runs away, unfolding
$\bangp{P}$ eagerly. A lazier and more implementable replication
operator, restricted to input-guarded processes, may be obtained as follows.

\begin{eqnarray}
\bangp{\prefix{u}{v}{P}} 
	:= 
	\binpar{\lift{x}{\prefix{u}{v}{(\binpar{D(x)}{P})}}}{D(x)} \nonumber
\end{eqnarray}

\begin{remark}
  Note that the lazier definition still does not deal with summation
  or mixed summation (i.e. sums over input and output). The reader is
  invited to construct definitions of replication that deal with these
  features. 

  Further, the definitions are parameterized in a name, $x$. Can you,
  gentle reader, make a definition that eliminates this parameter and
  guarantees no accidental interaction between the replication
  machinery and the process being replicated -- i.e. no accidental
  sharing of names used by the process to get its work done and the
  name(s) used by the replication to effect copying. This latter
  revision of the definition of replication is crucial to obtaining
  the expected identity $!!P \sim !P$.
\end{remark}

\begin{remark}\label{rem:paradoxical_combinator}
  The reader familiar with the lambda calculus will have noticed the
  similarity between $D$ and the paradoxical combinator.

  [Ed. note: the existence of this seems to suggest we have to be more
  restrictive on the set of processes and names we admit if we are to
  support no-cloning.]
\end{remark}

\subsubsection{Bisimulation}

The computational dynamics gives rise to another kind of equivalence,
the equivalence of computational behavior. As previously mentioned
this is typically captured \emph{via} some form of bisimulation.

% The notion we use in this paper is weak barbed bisimulation
% \cite{milner91polyadicpi}.

The notion we use in this paper is derived from weak barbed
bisimulation \cite{milner91polyadicpi}. 

\begin{definition}
An \emph{observation relation}, $\downarrow_{\mathcal N}$, over a set
of names, $\mathcal N$, is the smallest relation satisfying the rules
below.

\infrule[Out-barb]{y \in {\mathcal N}, \; x \nameeq y}
		  {\outputp{x}{v} \downarrow_{\mathcal N} x}
\infrule[Par-barb]{\mbox{$P\downarrow_{\mathcal N} x$ or $Q\downarrow_{\mathcal N} x$}}
		  {\binpar{P}{Q} \downarrow_{\mathcal N} x}

We write $P \Downarrow_{\mathcal N} x$ if there is $Q$ such that 
$P \wred Q$ and $Q \downarrow_{\mathcal N} x$.
\end{definition}

\begin{definition}
%\label{def.bbisim}
An  ${\mathcal N}$-\emph{barbed bisimulation} over a set of names, ${\mathcal N}$, is a symmetric binary relation 
${\mathcal S}_{\mathcal N}$ between agents such that $P\rel{S}_{\mathcal N}Q$ implies:
\begin{enumerate}
\item If $P \red P'$ then $Q \wred Q'$ and $P'\rel{S}_{\mathcal N} Q'$.
\item If $P\downarrow_{\mathcal N} x$, then $Q\Downarrow_{\mathcal N} x$.
\end{enumerate}
$P$ is ${\mathcal N}$-barbed bisimilar to $Q$, written
$P \wbbisim_{\mathcal N} Q$, if $P \rel{S}_{\mathcal N} Q$ for some ${\mathcal N}$-barbed bisimulation ${\mathcal S}_{\mathcal N}$.
\end{definition}

$\mathcal{R} \subseteq \pi \times \pi$

$P \mathcal{R} Q => \forall P'. P \red P' \Rightarrow \exists Q'. Q \red Q', P' \mathcal{R} Q'$

$P \vdash x \Rightarrow Q \vdash x$

\begin{mathpar}
  \inferrule*[lab=Out-barb]{x \nameeq y}{{y}!\langle{Q}\rangle \vdash x}
  \and
  \inferrule*[lab=Par-barb]{\mbox{$P\vdash x$ or $Q\vdash x$}}{\binpar{P}{Q} \vdash x}
\end{mathpar}

\subsubsection{Contexts}

One of the principle advantages of computational calculi like the
$\pi$-calculus is a well-defined notion of context,
contextual-equivalence and a correlation between
contextual-equivalence and notions of bisimulation. The notion of
context allows the decomposition of a process into (sub-)process and
its syntactic environment, its context. Thus, a context may be
thought of as a process with a ``hole'' (written $\Box$) in it. The
application of a context $M$ to a process $P$, written $M[P]$, is
tantamount to filling the hole in $M$ with $P$. In this paper we do
not need the full weight of this theory, but do make use of the notion
of context in the proof the main theorem. 

\begin{mathpar}
  \inferrule* [lab=summation] {} {{M_{M},M_{N}} \bc \Box \;|\; x.M_{A} \;|\; M_{M}+M_{N}}
  \and
  \inferrule* [lab=agent] {} {{M_{A}} \bc (\vec{x})M_{P} \;| \; \clift{P_0,\ldots,M_{P},\ldots,P_N}}
  \and \\
  \inferrule* [lab=process] {} {{M_{P}} \bc M_{N} \;| \;P|M_{P} }
\end{mathpar} 

\begin{mathpar}
  \inferrule* [lab=sychronization] {} {M_{N} \bc \Box \;|\; x?M_{F} \;|\; x!M_{C}}
  \and
  \inferrule* [lab=abstraction] {} {{M_{F}} \bc (x)M_{P} }
  \and
  \inferrule* [lab=concretion] {} {{M_{C}} \bc \langle M_{P} \rangle }
  \and \\
  \inferrule* [lab=process] {} {{M_{P}} \bc M_{N} \;| \;P|M_{P} }
\end{mathpar}

\begin{definition}[contextual application] Given a context $M$, and
  process $P$, we define the \emph{contextual application}, $M[P] :=
  M\{P/\Box\}$. That is, the contextual application of M to P is the
  substitution of $P$ for $\Box$ in $M$.
\end{definition}

$\meaningof{-} : L \to \mathcal{P}(\pi)$

\begin{mathpar}
  \inferrule* [lab=collection] {} {\meaningof{true} = \pi, \and \meaningof{~E} = \pi \setminus \meaningof{E}, \and \meaningof{E_{1} \& E_{2}} = \meaningof{E_{1}} \cap \meaningof{E_{2}}}
\end{mathpar}

\begin{mathpar}
  \inferrule* [lab=structure] {} {\meaningof{0} = \{ P \in \pi | P \equiv 0 \}, \and \\ \meaningof{E_1 | E_2} = \{ P \in \pi | P \equiv P_{1} | P_{2}, P_{1} \in \meaningof{E_{1}}, P_{2} \in \meaningof{E_2}\} }
\end{mathpar}

\begin{mathpar}
 \inferrule* [lab=behavior] {} {\meaningof{\langle a?b \rangle E} = \{ P \in \pi | P \equiv Q | u?(y)P', \\ \and \\\\ \and \\ \;\;\; u \in \meaningof{a}, \forall z.P'\{z/y\} \in \meaningof{E\{z/b\}}\}, \and \\ \meaningof{a!E} = \{ P \in \pi | P \equiv Q | x!\langle P' \rangle, x \in \meaningof{a} P' \in \meaningof{E}\} }
\end{mathpar}

\begin{mathpar}
 \inferrule* [lab=nominal] {} {\meaningof{\quotep{E}} = \{ \quotep{P} \in \quotep{\pi} | P \in \meaningof{E} \}, \and \meaningof{\quotep{P}} = \{ \quotep{Q} \in \quotep{\pi} | P \equiv Q \} \and \\ \meaningof{@\quotep{E}} = \{ P \in \pi | P \equiv @x, x \in \meaningof{E} \}}
\end{mathpar}

\begin{eqnarray*}
  \\
  \meaningof{-} : TS \to ST
\end{eqnarray*}

\begin{eqnarray*}
  \\
  L : TS \to ST
\end{eqnarray*}

\begin{eqnarray*}
  \\
  P \models E \iff P \in \meaningof{E}
\end{eqnarray*}

\begin{eqnarray*}
  P \approx_{L} Q \iff \forall E \in L. P \models E \iff Q \models E
\end{eqnarray*}

\begin{eqnarray*}
  P \approx_{K} Q
\end{eqnarray*}

\begin{eqnarray*}
  P \approx Q
\end{eqnarray*}

$\approx_{K} = \approx = \approx_{L}$

\subsubsection{Contextual duality}

Note that contexts extend the quotation operation to a family of
operations from processes to names. Given a context, $M$, we can
define a \emph{nominal context}, $\quotep{M}$ by $\quotep{M}[P] :=
\quotep{M[P]}$. To foreshadow what is to come we observe that these
operations enjoy a duality with processes very much like the duality
between vectors and maps from vectors to scalars.

Further, because the calculus is essentially higher-order, we have a
correspondence between contexts and processes. More specifically,
given a name $x$ and a context $M$ we can construct $M^{*}_{x}$ such
that 

\begin{mathpar}
  M^{*}_{x} | \lift{x}{P} \red M[P]
\end{mathpar}

namely,

\begin{mathpar}
  M^{*}_{x} := x?(u).M[\dropn{u}]
\end{mathpar}

The dependence of $M^{*}_{x}$ on a name makes it an abstraction, 

\begin{mathpar}
  M^{*} := (x)x?(u).M[\dropn{u}]
\end{mathpar}

\subsection{Additional notation}

It will sometimes be convenient to denote the process a name
quotes. We already have the notation $x = \quotep{P}$, but it will be
convenient to introduce an alternate notation, $\procn{x}$, when we
want to emphasize the connection to the use of the name. Note that, by
virtue of name equivalence, $\quotep{\procn{x}} \nameeq x$; so, the
notation is consistent with previous definitions.

Further, because names have structure it is possible to effect
substitutions on the basis of that structure. This means we need to
upgrade our notation for substitutions, which we accomplish by
adapting comprehension notation. Thus,

\begin{mathpar}
  P\{ y / x : x \in S \}
\end{mathpar}

is interpreted to mean the process derived from P by replacing (in a
capture-avoiding manner) each occurrence of $x$ in $S$ by $y$. For example,

\begin{mathpar}
  P\{ \quotep{\procn{x}|\procn{x}} / x : x \in \freenames{P} \}
\end{mathpar}

will replace each (occurrence) of a free name $x$ in $P$ by
$\quotep{\procn{x}|\procn{x}}$.

Also, we will avail ourselves of the notation $x^{L}$ and $x^{R}$ to
denote injections of a name into disjoint copies of the name
space. There are numerous ways to accomplish this. One example can be
found in \cite{MeredithR05}. This notation overloads to vectors of
names: $\vec{x}^{\pi} := (x_{i}^{\pi} \; : \; 0 \leq i < |\vec{x}| )$ where $\pi \in \{L,R\}$.

We also use $P^{\Box} := P|\Box$.

In \cite{MeredithR05} an interpretation of the new operator is
given. It turns out that there are several possible interpretations
all enjoying the requisite algebraic properties of the operator (see
\cite{milner91polyadicpi}). We will therefore make liberal use of
$(\nu\; \vec{x})P$.

% subsection the_syntax_and_semantics_of_the_notation_system (end)   

\input{qm2pi.qmops} 

\input{qm2pi.sterngerlach} 

\input{qm2pi.metric} 

% section concurrent_process_calculi (end)

%\input{qm2pi.proofsketch}

% section proof sketch (end)

%\input{qm2pi.slviaknots} 

% section spatial logic via knots (end)

\input{qm2pi.conclusion}

% section conclusion (end)

%\input{qm2pi.dtcodes} 

% section wiring algorithm (end)

\input{qm2pi.ack} 

% section acknowledgments (end)

\newpage


\bibliographystyle{plain}   
\bibliography{../../biblios/main.bib}

\input{qm2pi.rhodetails}

\end{document}

 

%\documentclass[12pt]{llncs}
%\documentclass{jktr}

\usepackage[pdftex]{hyperref}                   
\usepackage {listings}
\usepackage {mathpartir}
\usepackage{bcprules}
%\usepackage{listings}
                       
\usepackage{graphicx} 
%\usepackage[margins=2.5cm,nohead,nofoot]{geometry}
%\usepackage{geometry}
\usepackage{amsfonts}
\usepackage{amstext}
\usepackage{latexsym}
\usepackage{amssymb}
\usepackage{color}


%\include{myPreamble}
\include{qm2pi.local} 

%\ifpdf
%\usepackage[pdftex]{graphicx}
%\else
%\usepackage{graphicx}
%\fi

 % \ifpdf
%  \usepackage{pdfsync}
%  \if


%\title{Brief Article}
%\author{David F. Snyder}
%\author{L.G. Meredith}

%\address{Dept. of Math., Texas State University--San Marcos, San Marcos, TX 78666}
       
\pagestyle{empty}


\begin{document}

\lstset{language=[Objective]Caml,frame=shadowbox}

\input{qm2pi.front}

% section front matter (end)

\input{qm2pi.intro} 
 
% section introduction (end)

% \input{qm2pi.knotations} 

% section notation (end)

\input{qm2pi.process.calculi} 

% section concurrent_process_calculi_and_spatial_logics_ (end)
    
%\input{qm2pi.knots2pi} 

%\input{qm2pi.trefoil} 

%\input{qm2pi.mainthm} 

% subsection basic_interpretation (end)

%\input{qm2pi.rho.presentation} 
\subsection{The syntax and semantics of the notation system}\label{sub:the_syntax_and_semantics_of_the_notation_system} % (fold)

We now summarize a technical presentation of the calculus that
embodies our theory of dynamics. The typical presentation of such a
calculus follows the style of giving generators and relations on
them. The grammar, below, describing term constructors, freely
generates the set of processes, $\Proc$. This set is then quotiented
by a relation known as structural congruence and it is over this set
that the notion of dynamics is expressed. This presentation is
essentially that of \cite{MeredithR05} with the addition of
polyadicity and summation. For readability we have relegated some of
the technical subtleties to an appendix.

\subsubsection{Process grammar}\label{subsub:process_grammar}

\begin{mathpar}
  \inferrule* [lab=synchronization] {} {{M} \bc \pzero \;|\; x?F \;|\; x!C }
  \and
  \inferrule* [lab=abstraction] {} {{F} \bc (x)P}
  \and
  \inferrule* [lab=concretion] {} {{C} \bc \langle Q \rangle}
  \and
  \inferrule* [lab=process] {} {{P,Q} \bc M \;| \;P|Q \;|\; @{x}}
  \and
  \inferrule* [lab=name] {} {{x} \bc \quotep{P}}
\end{mathpar} 

Note that $\vec{x}$ (resp. $\vec{P}$) denotes a vector of names
(resp. processes) of length $|\vec{x}|$ (resp. $|\vec{P}|$). We adopt
the following useful abbreviations.

\begin{mathpar}
   x?(\vec{y}).P := x.(\vec{y})P \and  x\clift{\vec{P}} := x.\clift{\vec{P}}
   \and x!(y) := \lift{x}{\dropn{y}}
   \and \Pi_{i=0}^{n-1}P_i := P_0 | \ldots | P_{n-1}
\end{mathpar}

\subsubsection{Structural congruence}

\paragraph{Free and bound names and alpha-equivalence.} At the
core of structural equivalence is alpha-equivalence which identifies
process that are the same up to a change of variable. Formally, we
recognize the distinction between free and bound names. The free names
of a process, $\freenames{P}$, may be calculated recursively as
follows:

\begin{mathpar}
\freenames{\pzero} := \emptyset
  \and \\
  \freenames{x?(y).P} := \{ x \} \cup (\freenames{P} \setminus \{ y \})
  \and 
  \freenames{x!\langle P \rangle} := \{ x \} \cup \{ P \} 
  \and \\
  \freenames{P|Q} := \freenames{P} \cup \freenames{Q}
  \and \\
  \freenames{@{x}} := \{ x \}
\end{mathpar}

$\pi$
$\quotep{\pi}$

$\freenames{-} : \pi \to \mathcal{P}(\quotep{\pi})$

\begin{eqnarray*}
  \freenames{\pzero} & := & \emptyset \\
  \freenames{x?(y).P} & := & \{ x \} \cup (\freenames{P} \setminus \{ y \}) \\
  \freenames{x!\langle P \rangle} & := & \{ x \} \cup \{ P \} \\
  \freenames{P|Q} & := & \freenames{P} \cup \freenames{Q} \\
  \freenames{\dropn{x}} & := & \{ x \}
\end{eqnarray*}

The bound names of a process, $\boundnames{P}$, are those names occurring in $P$
that are not free. For example, in $x?(y).0$, the name $x$ is free, while $y$ is bound.

\begin{mathpar}
  \inferrule* [lab=monoidal-laws] {} { P|Q \equiv Q|P \and P|0 \equiv P \and P|(Q|R) \equiv (P|Q)|R }
\end{mathpar}

\begin{mathpar}
  \inferrule* [lab=alpha-equivalence] {} { (x)P \equiv (y)P\{y/x\} \and y \not\in \freenames{P} }
\end{mathpar}

\begin{definition}
Then two processes, $P,Q$, are alpha-equivalent if $P = Q\{\vec{y}/\vec{x}\}$ for
some $\vec{x} \in \boundnames{Q},\vec{y} \in \boundnames{P}$, where $Q\{\vec{y}/\vec{x}\}$
denotes the capture-avoiding substitution of $\vec{y}$ for $\vec{x}$ in $Q$.
\end{definition}

\begin{definition}
  The {\em structural congruence} \cite{SangiorgiWalker} , $\equiv$,
  between processes is the least congruence containing
  alpha-equivalence, satisfying the abelian monoid laws
  (associativity, commutativity and $\pzero$ as identity) for parallel
  composition $|$ and for summation $+$.
\end{definition}

\subsection{Name equivalence}

We take name equivalence, written $\nameeq$, to be the smallest
equivalence relation generated by the following rules.

\begin{mathpar}
\inferrule*[lab=Quote-drop]
{ }
{ \quotep{@{x}} \nameeq x }

\inferrule*[lab=Struct-equiv]
{ P \scong Q }
{ \quotep{P} \nameeq \quotep{Q} }
\end{mathpar}

The astute reader will have noticed that the mutual recursion of names
and processes imposes a mutual recursion on alpha-equivalence and
structural equivalence via name-equivalence. Fortunately, all of this
works out pleasantly and we may calculate in the natural way, free of
concern. The reader interested in the details is referred to the
appendix \ref{appendix:rho_details}.

\subsection{Substitution}

We use $\Proc$ for the set of processes, $\QProc$ for the set of
names, and $\id{\{}\vec{y} / \vec{x} \id{\}}$ to denote partial maps,
$s : \QProc \rightarrow \QProc$. A map, $s$ lifts, uniquely, to a map
on process terms, $\widehat{s} : \Proc \rightarrow \Proc$ by the
following equations.

\begin{mathpar}
  (0) \psubstp{Q}{P} := 0 \\
  (R \juxtap S) \psubstp{Q}{P}
  :=    
  (R)\psubstp{Q}{P} \juxtap (S) \psubstp{Q}{P} \\
  (x?(y).R) \psubstp{Q}{P}    
  :=    
  (x)\substp{Q}{P} (z)\concat( (R \psubstn{z}{y}) \psubstp{Q}{P} ) \\
  (\lift{x}{R}) \psubstp{Q}{P}  
  :=
  \lift{(x)\substp{Q}{P}}{ R \psubstp{Q}{P} } \\
%   (\dropn{x})  \psubstp{Q}{P}       
%   := 
%   \left\{ 
%     \begin{array}{ccc} 
%       \dropn{\quotep{Q}} & & x \nameeq \quotep{P} \\
%       \dropn{x} & & otherwise \\
%     \end{array}
%   \right. 
  (\dropn{x})  \psubstp{Q}{P}       
  := 
  \left\{ 
    \begin{array}{ccc} 
      Q & & x \nameeq \quotep{P} \\
      \dropn{x} & & otherwise \\
    \end{array}
  \right.
\end{mathpar}
 

where

\begin{eqnarray}
  (x)\id{\{} \lpquote Q \rpquote / \lpquote P \rpquote \id{\}}            = 
  \left\{ 
    \begin{array}{ccc}
      \lpquote Q \rpquote & & x \nameeq \lpquote P \rpquote \\
      x & & otherwise \\
    \end{array}
  \right. \nonumber
\end{eqnarray}

and $z$ is chosen distinct from $\quotep{P}$, $\quotep{Q}$, the free
names in $Q$, and all the names in $R$. Our $\alpha$-equivalence will
be built in the standard way from this substitution.

\begin{remark}\label{rem:no_self_referential_names}
  One consequence of these definitions is that $\forall P. \quotep{P}
  \not\in \freenames{P}$.
\end{remark}

\subsection{ Dynamic quote: an example }

Anticipating something of what's to come, consider applying the
substitution, $\widehat{\id{\{}u / z \id{\}}}$, to the following pair
of processes, $\lift{w}{y!(z)}$ and $w[ \lpquote y!(z) \rpquote ]$.

\begin{eqnarray}
	\lift{w}{y!(z)}\widehat{\id{\{}u / z \id{\}}}
		& = &
		\lift{w}{y!(u)} \nonumber\\
	w[ \lpquote y!(z) \rpquote ] \widehat{ \id{\{}u / z \id{\}} }
		& = &
		w[ \lpquote y!(z) \rpquote ] \nonumber
\end{eqnarray}

Because the body of the process between quotes is impervious to
substitution, we get radically different answers. In fact, by
examining the first process in an input context,
e.g. $x?(z).\lift{w}{y!(z)}$, we see that the process under the lift
operator may be shaped by prefixed inputs binding a name inside it. In
this sense, the lift operator will be seen as a way to dynamically
construct processes before reifying them as names.

Finally equipped with these standard features we can present the
dynamics of the calculus.

\subsubsection{Operational semantics} 

Finally, we introduce the computational dynamics. What marks these
algebras as distinct from other more traditionally studied algebraic
structures, e.g. vector spaces or polynomial rings, is the manner in
which dynamics is captured. In traditional structures, dynamics is typically
expressed through morphisms between such structures, as in linear maps
between vector spaces or morphisms between rings. In algebras
associated with the semantics of computation, the dynamics is
expressed as part of the algebraic structure itself, through a
reduction reduction relation typically denoted by $\red$. Below, we
give a recursive presentation of this relation for the calculus used
in the encoding.

$\red \subseteq \pi \times \pi$
$\red : \pi \to \mathcal{P}(\pi)$

\begin{mathpar}
  \inferrule* [lab=Comm] { \textsf{match}( x_{src}, x_{trgt} ) } { x_{trgt}?(y)P \; | \; x_{src}!\langle {Q} \rangle \red P\{\quotep{Q}/y}\} }
  \and \\
  \inferrule* [lab=Par] {{P} \red {P}'} {{{P} | {Q}} \red {{P}' | {Q}}}
  \and
  \inferrule* [lab=Equiv]{{{P} \scong {P}'} \andalso {{P}' \red {Q}'} \andalso {{Q}' \scong {Q}}}{{P} \red {Q}}
\end{mathpar}

\begin{eqnarray*}
  match_{\equiv} (\quotep{P},\quotep{Q}) & := & P \equiv Q \\
  match_{\dagger}(\quotep{P},\quotep{Q}) & := & \forall R. P|Q \red^{*} R => R \red^{*} 0 \\
  match_{K}(\quotep{P},\quotep{Q}) & := & K \mbox{ for some context } K
\end{eqnarray*}

$u?(x)P | u!\langle Q \rangle \red P\{\quotep{Q}/x\}$

%We write $\wred$ for $\red^*$, and $P\red$ if $\exists Q $ such that $ P \red Q$.
We write $P\red$ if $\exists Q $ such that $ P \red Q$ and $P\not\red$, otherwise.

\section{Replication}

As mentioned before, it is known that replication (and hence
recursion) can be implemented in a higher-order process algebra
\cite{SangiorgiWalker}. As our first example of calculation with the
machinery thus far presented we give the construction explicitly in
the {\rhoc}.

\begin{eqnarray}
	D_{x} & := & \prefix{x}{y}{(\binpar{\outputp{x}{y}}{@{y}})} \nonumber\\
	\bangp_{x}{P} & := & \binpar{{x}!\langle{\binpar{D_{x}}{P}}\rangle}{D_{x}} \nonumber
\end{eqnarray}

\begin{eqnarray}
	\bangp_{x}{P} & & \nonumber\\
	=
	& {x}!\langle{(\prefix{x}{y}{(\outputp{x}{y} | @{y})) | P}}\rangle 
	      | \prefix{x}{y}{(\outputp{x}{y} | @{y})} & \nonumber\\
	\red
	& (\outputp{x}{y} | @{y})\substn{\quotep{(\prefix{x}{y}{(@{y} | \outputp{x}{y})) | P}}}{y} & \nonumber\\
	=
	& \outputp{x}{\quotep{(\prefix{x}{y}{(\outputp{x}{y} | @{y})) | P}}}
	  | {(\prefix{x}{y}{(\outputp{x}{y} | @{y})) | P}} & \nonumber\\
	\red
	& \ldots & \nonumber\\
	\red^*
	& P | P | \ldots & \nonumber
\end{eqnarray}

Of course, this encoding, as an implementation, runs away, unfolding
$\bangp{P}$ eagerly. A lazier and more implementable replication
operator, restricted to input-guarded processes, may be obtained as follows.

\begin{eqnarray}
\bangp{\prefix{u}{v}{P}} 
	:= 
	\binpar{\lift{x}{\prefix{u}{v}{(\binpar{D(x)}{P})}}}{D(x)} \nonumber
\end{eqnarray}

\begin{remark}
  Note that the lazier definition still does not deal with summation
  or mixed summation (i.e. sums over input and output). The reader is
  invited to construct definitions of replication that deal with these
  features. 

  Further, the definitions are parameterized in a name, $x$. Can you,
  gentle reader, make a definition that eliminates this parameter and
  guarantees no accidental interaction between the replication
  machinery and the process being replicated -- i.e. no accidental
  sharing of names used by the process to get its work done and the
  name(s) used by the replication to effect copying. This latter
  revision of the definition of replication is crucial to obtaining
  the expected identity $!!P \sim !P$.
\end{remark}

\begin{remark}\label{rem:paradoxical_combinator}
  The reader familiar with the lambda calculus will have noticed the
  similarity between $D$ and the paradoxical combinator.

  [Ed. note: the existence of this seems to suggest we have to be more
  restrictive on the set of processes and names we admit if we are to
  support no-cloning.]
\end{remark}

\subsubsection{Bisimulation}

The computational dynamics gives rise to another kind of equivalence,
the equivalence of computational behavior. As previously mentioned
this is typically captured \emph{via} some form of bisimulation.

% The notion we use in this paper is weak barbed bisimulation
% \cite{milner91polyadicpi}.

The notion we use in this paper is derived from weak barbed
bisimulation \cite{milner91polyadicpi}. 

\begin{definition}
An \emph{observation relation}, $\downarrow_{\mathcal N}$, over a set
of names, $\mathcal N$, is the smallest relation satisfying the rules
below.

\infrule[Out-barb]{y \in {\mathcal N}, \; x \nameeq y}
		  {\outputp{x}{v} \downarrow_{\mathcal N} x}
\infrule[Par-barb]{\mbox{$P\downarrow_{\mathcal N} x$ or $Q\downarrow_{\mathcal N} x$}}
		  {\binpar{P}{Q} \downarrow_{\mathcal N} x}

We write $P \Downarrow_{\mathcal N} x$ if there is $Q$ such that 
$P \wred Q$ and $Q \downarrow_{\mathcal N} x$.
\end{definition}

\begin{definition}
%\label{def.bbisim}
An  ${\mathcal N}$-\emph{barbed bisimulation} over a set of names, ${\mathcal N}$, is a symmetric binary relation 
${\mathcal S}_{\mathcal N}$ between agents such that $P\rel{S}_{\mathcal N}Q$ implies:
\begin{enumerate}
\item If $P \red P'$ then $Q \wred Q'$ and $P'\rel{S}_{\mathcal N} Q'$.
\item If $P\downarrow_{\mathcal N} x$, then $Q\Downarrow_{\mathcal N} x$.
\end{enumerate}
$P$ is ${\mathcal N}$-barbed bisimilar to $Q$, written
$P \wbbisim_{\mathcal N} Q$, if $P \rel{S}_{\mathcal N} Q$ for some ${\mathcal N}$-barbed bisimulation ${\mathcal S}_{\mathcal N}$.
\end{definition}

$\mathcal{R} \subseteq \pi \times \pi$

$P \mathcal{R} Q => \forall P'. P \red P' \Rightarrow \exists Q'. Q \red Q', P' \mathcal{R} Q'$

$P \vdash x \Rightarrow Q \vdash x$

\begin{mathpar}
  \inferrule*[lab=Out-barb]{x \nameeq y}{{y}!\langle{Q}\rangle \vdash x}
  \and
  \inferrule*[lab=Par-barb]{\mbox{$P\vdash x$ or $Q\vdash x$}}{\binpar{P}{Q} \vdash x}
\end{mathpar}

\subsubsection{Contexts}

One of the principle advantages of computational calculi like the
$\pi$-calculus is a well-defined notion of context,
contextual-equivalence and a correlation between
contextual-equivalence and notions of bisimulation. The notion of
context allows the decomposition of a process into (sub-)process and
its syntactic environment, its context. Thus, a context may be
thought of as a process with a ``hole'' (written $\Box$) in it. The
application of a context $M$ to a process $P$, written $M[P]$, is
tantamount to filling the hole in $M$ with $P$. In this paper we do
not need the full weight of this theory, but do make use of the notion
of context in the proof the main theorem. 

\begin{mathpar}
  \inferrule* [lab=summation] {} {{M_{M},M_{N}} \bc \Box \;|\; x.M_{A} \;|\; M_{M}+M_{N}}
  \and
  \inferrule* [lab=agent] {} {{M_{A}} \bc (\vec{x})M_{P} \;| \; \clift{P_0,\ldots,M_{P},\ldots,P_N}}
  \and \\
  \inferrule* [lab=process] {} {{M_{P}} \bc M_{N} \;| \;P|M_{P} }
\end{mathpar} 

\begin{mathpar}
  \inferrule* [lab=sychronization] {} {M_{N} \bc \Box \;|\; x?M_{F} \;|\; x!M_{C}}
  \and
  \inferrule* [lab=abstraction] {} {{M_{F}} \bc (x)M_{P} }
  \and
  \inferrule* [lab=concretion] {} {{M_{C}} \bc \langle M_{P} \rangle }
  \and \\
  \inferrule* [lab=process] {} {{M_{P}} \bc M_{N} \;| \;P|M_{P} }
\end{mathpar}

\begin{definition}[contextual application] Given a context $M$, and
  process $P$, we define the \emph{contextual application}, $M[P] :=
  M\{P/\Box\}$. That is, the contextual application of M to P is the
  substitution of $P$ for $\Box$ in $M$.
\end{definition}

$\meaningof{-} : L \to \mathcal{P}(\pi)$

\begin{mathpar}
  \inferrule* [lab=collection] {} {\meaningof{true} = \pi, \and \meaningof{~E} = \pi \setminus \meaningof{E}, \and \meaningof{E_{1} \& E_{2}} = \meaningof{E_{1}} \cap \meaningof{E_{2}}}
\end{mathpar}

\begin{mathpar}
  \inferrule* [lab=structure] {} {\meaningof{0} = \{ P \in \pi | P \equiv 0 \}, \and \\ \meaningof{E_1 | E_2} = \{ P \in \pi | P \equiv P_{1} | P_{2}, P_{1} \in \meaningof{E_{1}}, P_{2} \in \meaningof{E_2}\} }
\end{mathpar}

\begin{mathpar}
 \inferrule* [lab=behavior] {} {\meaningof{\langle a?b \rangle E} = \{ P \in \pi | P \equiv Q | u?(y)P', \\ \and \\\\ \and \\ \;\;\; u \in \meaningof{a}, \forall z.P'\{z/y\} \in \meaningof{E\{z/b\}}\}, \and \\ \meaningof{a!E} = \{ P \in \pi | P \equiv Q | x!\langle P' \rangle, x \in \meaningof{a} P' \in \meaningof{E}\} }
\end{mathpar}

\begin{mathpar}
 \inferrule* [lab=nominal] {} {\meaningof{\quotep{E}} = \{ \quotep{P} \in \quotep{\pi} | P \in \meaningof{E} \}, \and \meaningof{\quotep{P}} = \{ \quotep{Q} \in \quotep{\pi} | P \equiv Q \} \and \\ \meaningof{@\quotep{E}} = \{ P \in \pi | P \equiv @x, x \in \meaningof{E} \}}
\end{mathpar}

\begin{eqnarray*}
  \\
  \meaningof{-} : TS \to ST
\end{eqnarray*}

\begin{eqnarray*}
  \\
  L : TS \to ST
\end{eqnarray*}

\begin{eqnarray*}
  \\
  P \models E \iff P \in \meaningof{E}
\end{eqnarray*}

\begin{eqnarray*}
  P \approx_{L} Q \iff \forall E \in L. P \models E \iff Q \models E
\end{eqnarray*}

\begin{eqnarray*}
  P \approx_{K} Q
\end{eqnarray*}

\begin{eqnarray*}
  P \approx Q
\end{eqnarray*}

$\approx_{K} = \approx = \approx_{L}$

\subsubsection{Contextual duality}

Note that contexts extend the quotation operation to a family of
operations from processes to names. Given a context, $M$, we can
define a \emph{nominal context}, $\quotep{M}$ by $\quotep{M}[P] :=
\quotep{M[P]}$. To foreshadow what is to come we observe that these
operations enjoy a duality with processes very much like the duality
between vectors and maps from vectors to scalars.

Further, because the calculus is essentially higher-order, we have a
correspondence between contexts and processes. More specifically,
given a name $x$ and a context $M$ we can construct $M^{*}_{x}$ such
that 

\begin{mathpar}
  M^{*}_{x} | \lift{x}{P} \red M[P]
\end{mathpar}

namely,

\begin{mathpar}
  M^{*}_{x} := x?(u).M[\dropn{u}]
\end{mathpar}

The dependence of $M^{*}_{x}$ on a name makes it an abstraction, 

\begin{mathpar}
  M^{*} := (x)x?(u).M[\dropn{u}]
\end{mathpar}

\subsection{Additional notation}

It will sometimes be convenient to denote the process a name
quotes. We already have the notation $x = \quotep{P}$, but it will be
convenient to introduce an alternate notation, $\procn{x}$, when we
want to emphasize the connection to the use of the name. Note that, by
virtue of name equivalence, $\quotep{\procn{x}} \nameeq x$; so, the
notation is consistent with previous definitions.

Further, because names have structure it is possible to effect
substitutions on the basis of that structure. This means we need to
upgrade our notation for substitutions, which we accomplish by
adapting comprehension notation. Thus,

\begin{mathpar}
  P\{ y / x : x \in S \}
\end{mathpar}

is interpreted to mean the process derived from P by replacing (in a
capture-avoiding manner) each occurrence of $x$ in $S$ by $y$. For example,

\begin{mathpar}
  P\{ \quotep{\procn{x}|\procn{x}} / x : x \in \freenames{P} \}
\end{mathpar}

will replace each (occurrence) of a free name $x$ in $P$ by
$\quotep{\procn{x}|\procn{x}}$.

Also, we will avail ourselves of the notation $x^{L}$ and $x^{R}$ to
denote injections of a name into disjoint copies of the name
space. There are numerous ways to accomplish this. One example can be
found in \cite{MeredithR05}. This notation overloads to vectors of
names: $\vec{x}^{\pi} := (x_{i}^{\pi} \; : \; 0 \leq i < |\vec{x}| )$ where $\pi \in \{L,R\}$.

We also use $P^{\Box} := P|\Box$.

In \cite{MeredithR05} an interpretation of the new operator is
given. It turns out that there are several possible interpretations
all enjoying the requisite algebraic properties of the operator (see
\cite{milner91polyadicpi}). We will therefore make liberal use of
$(\nu\; \vec{x})P$.

% subsection the_syntax_and_semantics_of_the_notation_system (end)   

\input{qm2pi.qmops} 

\input{qm2pi.sterngerlach} 

\input{qm2pi.metric} 

% section concurrent_process_calculi (end)

%\input{qm2pi.proofsketch}

% section proof sketch (end)

%\input{qm2pi.slviaknots} 

% section spatial logic via knots (end)

\input{qm2pi.conclusion}

% section conclusion (end)

%\input{qm2pi.dtcodes} 

% section wiring algorithm (end)

\input{qm2pi.ack} 

% section acknowledgments (end)

\newpage


\bibliographystyle{plain}   
\bibliography{../../biblios/main.bib}

\input{qm2pi.rhodetails}

\end{document}

 

%\documentclass[12pt]{llncs}
%\documentclass{jktr}

\usepackage[pdftex]{hyperref}                   
\usepackage {listings}
\usepackage {mathpartir}
\usepackage{bcprules}
%\usepackage{listings}
                       
\usepackage{graphicx} 
%\usepackage[margins=2.5cm,nohead,nofoot]{geometry}
%\usepackage{geometry}
\usepackage{amsfonts}
\usepackage{amstext}
\usepackage{latexsym}
\usepackage{amssymb}
\usepackage{color}


%\include{myPreamble}
\include{qm2pi.local} 

%\ifpdf
%\usepackage[pdftex]{graphicx}
%\else
%\usepackage{graphicx}
%\fi

 % \ifpdf
%  \usepackage{pdfsync}
%  \if


%\title{Brief Article}
%\author{David F. Snyder}
%\author{L.G. Meredith}

%\address{Dept. of Math., Texas State University--San Marcos, San Marcos, TX 78666}
       
\pagestyle{empty}


\begin{document}

\lstset{language=[Objective]Caml,frame=shadowbox}

\input{qm2pi.front}

% section front matter (end)

\input{qm2pi.intro} 
 
% section introduction (end)

% \input{qm2pi.knotations} 

% section notation (end)

\input{qm2pi.process.calculi} 

% section concurrent_process_calculi_and_spatial_logics_ (end)
    
%\input{qm2pi.knots2pi} 

%\input{qm2pi.trefoil} 

%\input{qm2pi.mainthm} 

% subsection basic_interpretation (end)

%\input{qm2pi.rho.presentation} 
\subsection{The syntax and semantics of the notation system}\label{sub:the_syntax_and_semantics_of_the_notation_system} % (fold)

We now summarize a technical presentation of the calculus that
embodies our theory of dynamics. The typical presentation of such a
calculus follows the style of giving generators and relations on
them. The grammar, below, describing term constructors, freely
generates the set of processes, $\Proc$. This set is then quotiented
by a relation known as structural congruence and it is over this set
that the notion of dynamics is expressed. This presentation is
essentially that of \cite{MeredithR05} with the addition of
polyadicity and summation. For readability we have relegated some of
the technical subtleties to an appendix.

\subsubsection{Process grammar}\label{subsub:process_grammar}

\begin{mathpar}
  \inferrule* [lab=synchronization] {} {{M} \bc \pzero \;|\; x?F \;|\; x!C }
  \and
  \inferrule* [lab=abstraction] {} {{F} \bc (x)P}
  \and
  \inferrule* [lab=concretion] {} {{C} \bc \langle Q \rangle}
  \and
  \inferrule* [lab=process] {} {{P,Q} \bc M \;| \;P|Q \;|\; @{x}}
  \and
  \inferrule* [lab=name] {} {{x} \bc \quotep{P}}
\end{mathpar} 

Note that $\vec{x}$ (resp. $\vec{P}$) denotes a vector of names
(resp. processes) of length $|\vec{x}|$ (resp. $|\vec{P}|$). We adopt
the following useful abbreviations.

\begin{mathpar}
   x?(\vec{y}).P := x.(\vec{y})P \and  x\clift{\vec{P}} := x.\clift{\vec{P}}
   \and x!(y) := \lift{x}{\dropn{y}}
   \and \Pi_{i=0}^{n-1}P_i := P_0 | \ldots | P_{n-1}
\end{mathpar}

\subsubsection{Structural congruence}

\paragraph{Free and bound names and alpha-equivalence.} At the
core of structural equivalence is alpha-equivalence which identifies
process that are the same up to a change of variable. Formally, we
recognize the distinction between free and bound names. The free names
of a process, $\freenames{P}$, may be calculated recursively as
follows:

\begin{mathpar}
\freenames{\pzero} := \emptyset
  \and \\
  \freenames{x?(y).P} := \{ x \} \cup (\freenames{P} \setminus \{ y \})
  \and 
  \freenames{x!\langle P \rangle} := \{ x \} \cup \{ P \} 
  \and \\
  \freenames{P|Q} := \freenames{P} \cup \freenames{Q}
  \and \\
  \freenames{@{x}} := \{ x \}
\end{mathpar}

$\pi$
$\quotep{\pi}$

$\freenames{-} : \pi \to \mathcal{P}(\quotep{\pi})$

\begin{eqnarray*}
  \freenames{\pzero} & := & \emptyset \\
  \freenames{x?(y).P} & := & \{ x \} \cup (\freenames{P} \setminus \{ y \}) \\
  \freenames{x!\langle P \rangle} & := & \{ x \} \cup \{ P \} \\
  \freenames{P|Q} & := & \freenames{P} \cup \freenames{Q} \\
  \freenames{\dropn{x}} & := & \{ x \}
\end{eqnarray*}

The bound names of a process, $\boundnames{P}$, are those names occurring in $P$
that are not free. For example, in $x?(y).0$, the name $x$ is free, while $y$ is bound.

\begin{mathpar}
  \inferrule* [lab=monoidal-laws] {} { P|Q \equiv Q|P \and P|0 \equiv P \and P|(Q|R) \equiv (P|Q)|R }
\end{mathpar}

\begin{mathpar}
  \inferrule* [lab=alpha-equivalence] {} { (x)P \equiv (y)P\{y/x\} \and y \not\in \freenames{P} }
\end{mathpar}

\begin{definition}
Then two processes, $P,Q$, are alpha-equivalent if $P = Q\{\vec{y}/\vec{x}\}$ for
some $\vec{x} \in \boundnames{Q},\vec{y} \in \boundnames{P}$, where $Q\{\vec{y}/\vec{x}\}$
denotes the capture-avoiding substitution of $\vec{y}$ for $\vec{x}$ in $Q$.
\end{definition}

\begin{definition}
  The {\em structural congruence} \cite{SangiorgiWalker} , $\equiv$,
  between processes is the least congruence containing
  alpha-equivalence, satisfying the abelian monoid laws
  (associativity, commutativity and $\pzero$ as identity) for parallel
  composition $|$ and for summation $+$.
\end{definition}

\subsection{Name equivalence}

We take name equivalence, written $\nameeq$, to be the smallest
equivalence relation generated by the following rules.

\begin{mathpar}
\inferrule*[lab=Quote-drop]
{ }
{ \quotep{@{x}} \nameeq x }

\inferrule*[lab=Struct-equiv]
{ P \scong Q }
{ \quotep{P} \nameeq \quotep{Q} }
\end{mathpar}

The astute reader will have noticed that the mutual recursion of names
and processes imposes a mutual recursion on alpha-equivalence and
structural equivalence via name-equivalence. Fortunately, all of this
works out pleasantly and we may calculate in the natural way, free of
concern. The reader interested in the details is referred to the
appendix \ref{appendix:rho_details}.

\subsection{Substitution}

We use $\Proc$ for the set of processes, $\QProc$ for the set of
names, and $\id{\{}\vec{y} / \vec{x} \id{\}}$ to denote partial maps,
$s : \QProc \rightarrow \QProc$. A map, $s$ lifts, uniquely, to a map
on process terms, $\widehat{s} : \Proc \rightarrow \Proc$ by the
following equations.

\begin{mathpar}
  (0) \psubstp{Q}{P} := 0 \\
  (R \juxtap S) \psubstp{Q}{P}
  :=    
  (R)\psubstp{Q}{P} \juxtap (S) \psubstp{Q}{P} \\
  (x?(y).R) \psubstp{Q}{P}    
  :=    
  (x)\substp{Q}{P} (z)\concat( (R \psubstn{z}{y}) \psubstp{Q}{P} ) \\
  (\lift{x}{R}) \psubstp{Q}{P}  
  :=
  \lift{(x)\substp{Q}{P}}{ R \psubstp{Q}{P} } \\
%   (\dropn{x})  \psubstp{Q}{P}       
%   := 
%   \left\{ 
%     \begin{array}{ccc} 
%       \dropn{\quotep{Q}} & & x \nameeq \quotep{P} \\
%       \dropn{x} & & otherwise \\
%     \end{array}
%   \right. 
  (\dropn{x})  \psubstp{Q}{P}       
  := 
  \left\{ 
    \begin{array}{ccc} 
      Q & & x \nameeq \quotep{P} \\
      \dropn{x} & & otherwise \\
    \end{array}
  \right.
\end{mathpar}
 

where

\begin{eqnarray}
  (x)\id{\{} \lpquote Q \rpquote / \lpquote P \rpquote \id{\}}            = 
  \left\{ 
    \begin{array}{ccc}
      \lpquote Q \rpquote & & x \nameeq \lpquote P \rpquote \\
      x & & otherwise \\
    \end{array}
  \right. \nonumber
\end{eqnarray}

and $z$ is chosen distinct from $\quotep{P}$, $\quotep{Q}$, the free
names in $Q$, and all the names in $R$. Our $\alpha$-equivalence will
be built in the standard way from this substitution.

\begin{remark}\label{rem:no_self_referential_names}
  One consequence of these definitions is that $\forall P. \quotep{P}
  \not\in \freenames{P}$.
\end{remark}

\subsection{ Dynamic quote: an example }

Anticipating something of what's to come, consider applying the
substitution, $\widehat{\id{\{}u / z \id{\}}}$, to the following pair
of processes, $\lift{w}{y!(z)}$ and $w[ \lpquote y!(z) \rpquote ]$.

\begin{eqnarray}
	\lift{w}{y!(z)}\widehat{\id{\{}u / z \id{\}}}
		& = &
		\lift{w}{y!(u)} \nonumber\\
	w[ \lpquote y!(z) \rpquote ] \widehat{ \id{\{}u / z \id{\}} }
		& = &
		w[ \lpquote y!(z) \rpquote ] \nonumber
\end{eqnarray}

Because the body of the process between quotes is impervious to
substitution, we get radically different answers. In fact, by
examining the first process in an input context,
e.g. $x?(z).\lift{w}{y!(z)}$, we see that the process under the lift
operator may be shaped by prefixed inputs binding a name inside it. In
this sense, the lift operator will be seen as a way to dynamically
construct processes before reifying them as names.

Finally equipped with these standard features we can present the
dynamics of the calculus.

\subsubsection{Operational semantics} 

Finally, we introduce the computational dynamics. What marks these
algebras as distinct from other more traditionally studied algebraic
structures, e.g. vector spaces or polynomial rings, is the manner in
which dynamics is captured. In traditional structures, dynamics is typically
expressed through morphisms between such structures, as in linear maps
between vector spaces or morphisms between rings. In algebras
associated with the semantics of computation, the dynamics is
expressed as part of the algebraic structure itself, through a
reduction reduction relation typically denoted by $\red$. Below, we
give a recursive presentation of this relation for the calculus used
in the encoding.

$\red \subseteq \pi \times \pi$
$\red : \pi \to \mathcal{P}(\pi)$

\begin{mathpar}
  \inferrule* [lab=Comm] { \textsf{match}( x_{src}, x_{trgt} ) } { x_{trgt}?(y)P \; | \; x_{src}!\langle {Q} \rangle \red P\{\quotep{Q}/y}\} }
  \and \\
  \inferrule* [lab=Par] {{P} \red {P}'} {{{P} | {Q}} \red {{P}' | {Q}}}
  \and
  \inferrule* [lab=Equiv]{{{P} \scong {P}'} \andalso {{P}' \red {Q}'} \andalso {{Q}' \scong {Q}}}{{P} \red {Q}}
\end{mathpar}

\begin{eqnarray*}
  match_{\equiv} (\quotep{P},\quotep{Q}) & := & P \equiv Q \\
  match_{\dagger}(\quotep{P},\quotep{Q}) & := & \forall R. P|Q \red^{*} R => R \red^{*} 0 \\
  match_{K}(\quotep{P},\quotep{Q}) & := & K \mbox{ for some context } K
\end{eqnarray*}

$u?(x)P | u!\langle Q \rangle \red P\{\quotep{Q}/x\}$

%We write $\wred$ for $\red^*$, and $P\red$ if $\exists Q $ such that $ P \red Q$.
We write $P\red$ if $\exists Q $ such that $ P \red Q$ and $P\not\red$, otherwise.

\section{Replication}

As mentioned before, it is known that replication (and hence
recursion) can be implemented in a higher-order process algebra
\cite{SangiorgiWalker}. As our first example of calculation with the
machinery thus far presented we give the construction explicitly in
the {\rhoc}.

\begin{eqnarray}
	D_{x} & := & \prefix{x}{y}{(\binpar{\outputp{x}{y}}{@{y}})} \nonumber\\
	\bangp_{x}{P} & := & \binpar{{x}!\langle{\binpar{D_{x}}{P}}\rangle}{D_{x}} \nonumber
\end{eqnarray}

\begin{eqnarray}
	\bangp_{x}{P} & & \nonumber\\
	=
	& {x}!\langle{(\prefix{x}{y}{(\outputp{x}{y} | @{y})) | P}}\rangle 
	      | \prefix{x}{y}{(\outputp{x}{y} | @{y})} & \nonumber\\
	\red
	& (\outputp{x}{y} | @{y})\substn{\quotep{(\prefix{x}{y}{(@{y} | \outputp{x}{y})) | P}}}{y} & \nonumber\\
	=
	& \outputp{x}{\quotep{(\prefix{x}{y}{(\outputp{x}{y} | @{y})) | P}}}
	  | {(\prefix{x}{y}{(\outputp{x}{y} | @{y})) | P}} & \nonumber\\
	\red
	& \ldots & \nonumber\\
	\red^*
	& P | P | \ldots & \nonumber
\end{eqnarray}

Of course, this encoding, as an implementation, runs away, unfolding
$\bangp{P}$ eagerly. A lazier and more implementable replication
operator, restricted to input-guarded processes, may be obtained as follows.

\begin{eqnarray}
\bangp{\prefix{u}{v}{P}} 
	:= 
	\binpar{\lift{x}{\prefix{u}{v}{(\binpar{D(x)}{P})}}}{D(x)} \nonumber
\end{eqnarray}

\begin{remark}
  Note that the lazier definition still does not deal with summation
  or mixed summation (i.e. sums over input and output). The reader is
  invited to construct definitions of replication that deal with these
  features. 

  Further, the definitions are parameterized in a name, $x$. Can you,
  gentle reader, make a definition that eliminates this parameter and
  guarantees no accidental interaction between the replication
  machinery and the process being replicated -- i.e. no accidental
  sharing of names used by the process to get its work done and the
  name(s) used by the replication to effect copying. This latter
  revision of the definition of replication is crucial to obtaining
  the expected identity $!!P \sim !P$.
\end{remark}

\begin{remark}\label{rem:paradoxical_combinator}
  The reader familiar with the lambda calculus will have noticed the
  similarity between $D$ and the paradoxical combinator.

  [Ed. note: the existence of this seems to suggest we have to be more
  restrictive on the set of processes and names we admit if we are to
  support no-cloning.]
\end{remark}

\subsubsection{Bisimulation}

The computational dynamics gives rise to another kind of equivalence,
the equivalence of computational behavior. As previously mentioned
this is typically captured \emph{via} some form of bisimulation.

% The notion we use in this paper is weak barbed bisimulation
% \cite{milner91polyadicpi}.

The notion we use in this paper is derived from weak barbed
bisimulation \cite{milner91polyadicpi}. 

\begin{definition}
An \emph{observation relation}, $\downarrow_{\mathcal N}$, over a set
of names, $\mathcal N$, is the smallest relation satisfying the rules
below.

\infrule[Out-barb]{y \in {\mathcal N}, \; x \nameeq y}
		  {\outputp{x}{v} \downarrow_{\mathcal N} x}
\infrule[Par-barb]{\mbox{$P\downarrow_{\mathcal N} x$ or $Q\downarrow_{\mathcal N} x$}}
		  {\binpar{P}{Q} \downarrow_{\mathcal N} x}

We write $P \Downarrow_{\mathcal N} x$ if there is $Q$ such that 
$P \wred Q$ and $Q \downarrow_{\mathcal N} x$.
\end{definition}

\begin{definition}
%\label{def.bbisim}
An  ${\mathcal N}$-\emph{barbed bisimulation} over a set of names, ${\mathcal N}$, is a symmetric binary relation 
${\mathcal S}_{\mathcal N}$ between agents such that $P\rel{S}_{\mathcal N}Q$ implies:
\begin{enumerate}
\item If $P \red P'$ then $Q \wred Q'$ and $P'\rel{S}_{\mathcal N} Q'$.
\item If $P\downarrow_{\mathcal N} x$, then $Q\Downarrow_{\mathcal N} x$.
\end{enumerate}
$P$ is ${\mathcal N}$-barbed bisimilar to $Q$, written
$P \wbbisim_{\mathcal N} Q$, if $P \rel{S}_{\mathcal N} Q$ for some ${\mathcal N}$-barbed bisimulation ${\mathcal S}_{\mathcal N}$.
\end{definition}

$\mathcal{R} \subseteq \pi \times \pi$

$P \mathcal{R} Q => \forall P'. P \red P' \Rightarrow \exists Q'. Q \red Q', P' \mathcal{R} Q'$

$P \vdash x \Rightarrow Q \vdash x$

\begin{mathpar}
  \inferrule*[lab=Out-barb]{x \nameeq y}{{y}!\langle{Q}\rangle \vdash x}
  \and
  \inferrule*[lab=Par-barb]{\mbox{$P\vdash x$ or $Q\vdash x$}}{\binpar{P}{Q} \vdash x}
\end{mathpar}

\subsubsection{Contexts}

One of the principle advantages of computational calculi like the
$\pi$-calculus is a well-defined notion of context,
contextual-equivalence and a correlation between
contextual-equivalence and notions of bisimulation. The notion of
context allows the decomposition of a process into (sub-)process and
its syntactic environment, its context. Thus, a context may be
thought of as a process with a ``hole'' (written $\Box$) in it. The
application of a context $M$ to a process $P$, written $M[P]$, is
tantamount to filling the hole in $M$ with $P$. In this paper we do
not need the full weight of this theory, but do make use of the notion
of context in the proof the main theorem. 

\begin{mathpar}
  \inferrule* [lab=summation] {} {{M_{M},M_{N}} \bc \Box \;|\; x.M_{A} \;|\; M_{M}+M_{N}}
  \and
  \inferrule* [lab=agent] {} {{M_{A}} \bc (\vec{x})M_{P} \;| \; \clift{P_0,\ldots,M_{P},\ldots,P_N}}
  \and \\
  \inferrule* [lab=process] {} {{M_{P}} \bc M_{N} \;| \;P|M_{P} }
\end{mathpar} 

\begin{mathpar}
  \inferrule* [lab=sychronization] {} {M_{N} \bc \Box \;|\; x?M_{F} \;|\; x!M_{C}}
  \and
  \inferrule* [lab=abstraction] {} {{M_{F}} \bc (x)M_{P} }
  \and
  \inferrule* [lab=concretion] {} {{M_{C}} \bc \langle M_{P} \rangle }
  \and \\
  \inferrule* [lab=process] {} {{M_{P}} \bc M_{N} \;| \;P|M_{P} }
\end{mathpar}

\begin{definition}[contextual application] Given a context $M$, and
  process $P$, we define the \emph{contextual application}, $M[P] :=
  M\{P/\Box\}$. That is, the contextual application of M to P is the
  substitution of $P$ for $\Box$ in $M$.
\end{definition}

$\meaningof{-} : L \to \mathcal{P}(\pi)$

\begin{mathpar}
  \inferrule* [lab=collection] {} {\meaningof{true} = \pi, \and \meaningof{~E} = \pi \setminus \meaningof{E}, \and \meaningof{E_{1} \& E_{2}} = \meaningof{E_{1}} \cap \meaningof{E_{2}}}
\end{mathpar}

\begin{mathpar}
  \inferrule* [lab=structure] {} {\meaningof{0} = \{ P \in \pi | P \equiv 0 \}, \and \\ \meaningof{E_1 | E_2} = \{ P \in \pi | P \equiv P_{1} | P_{2}, P_{1} \in \meaningof{E_{1}}, P_{2} \in \meaningof{E_2}\} }
\end{mathpar}

\begin{mathpar}
 \inferrule* [lab=behavior] {} {\meaningof{\langle a?b \rangle E} = \{ P \in \pi | P \equiv Q | u?(y)P', \\ \and \\\\ \and \\ \;\;\; u \in \meaningof{a}, \forall z.P'\{z/y\} \in \meaningof{E\{z/b\}}\}, \and \\ \meaningof{a!E} = \{ P \in \pi | P \equiv Q | x!\langle P' \rangle, x \in \meaningof{a} P' \in \meaningof{E}\} }
\end{mathpar}

\begin{mathpar}
 \inferrule* [lab=nominal] {} {\meaningof{\quotep{E}} = \{ \quotep{P} \in \quotep{\pi} | P \in \meaningof{E} \}, \and \meaningof{\quotep{P}} = \{ \quotep{Q} \in \quotep{\pi} | P \equiv Q \} \and \\ \meaningof{@\quotep{E}} = \{ P \in \pi | P \equiv @x, x \in \meaningof{E} \}}
\end{mathpar}

\begin{eqnarray*}
  \\
  \meaningof{-} : TS \to ST
\end{eqnarray*}

\begin{eqnarray*}
  \\
  L : TS \to ST
\end{eqnarray*}

\begin{eqnarray*}
  \\
  P \models E \iff P \in \meaningof{E}
\end{eqnarray*}

\begin{eqnarray*}
  P \approx_{L} Q \iff \forall E \in L. P \models E \iff Q \models E
\end{eqnarray*}

\begin{eqnarray*}
  P \approx_{K} Q
\end{eqnarray*}

\begin{eqnarray*}
  P \approx Q
\end{eqnarray*}

$\approx_{K} = \approx = \approx_{L}$

\subsubsection{Contextual duality}

Note that contexts extend the quotation operation to a family of
operations from processes to names. Given a context, $M$, we can
define a \emph{nominal context}, $\quotep{M}$ by $\quotep{M}[P] :=
\quotep{M[P]}$. To foreshadow what is to come we observe that these
operations enjoy a duality with processes very much like the duality
between vectors and maps from vectors to scalars.

Further, because the calculus is essentially higher-order, we have a
correspondence between contexts and processes. More specifically,
given a name $x$ and a context $M$ we can construct $M^{*}_{x}$ such
that 

\begin{mathpar}
  M^{*}_{x} | \lift{x}{P} \red M[P]
\end{mathpar}

namely,

\begin{mathpar}
  M^{*}_{x} := x?(u).M[\dropn{u}]
\end{mathpar}

The dependence of $M^{*}_{x}$ on a name makes it an abstraction, 

\begin{mathpar}
  M^{*} := (x)x?(u).M[\dropn{u}]
\end{mathpar}

\subsection{Additional notation}

It will sometimes be convenient to denote the process a name
quotes. We already have the notation $x = \quotep{P}$, but it will be
convenient to introduce an alternate notation, $\procn{x}$, when we
want to emphasize the connection to the use of the name. Note that, by
virtue of name equivalence, $\quotep{\procn{x}} \nameeq x$; so, the
notation is consistent with previous definitions.

Further, because names have structure it is possible to effect
substitutions on the basis of that structure. This means we need to
upgrade our notation for substitutions, which we accomplish by
adapting comprehension notation. Thus,

\begin{mathpar}
  P\{ y / x : x \in S \}
\end{mathpar}

is interpreted to mean the process derived from P by replacing (in a
capture-avoiding manner) each occurrence of $x$ in $S$ by $y$. For example,

\begin{mathpar}
  P\{ \quotep{\procn{x}|\procn{x}} / x : x \in \freenames{P} \}
\end{mathpar}

will replace each (occurrence) of a free name $x$ in $P$ by
$\quotep{\procn{x}|\procn{x}}$.

Also, we will avail ourselves of the notation $x^{L}$ and $x^{R}$ to
denote injections of a name into disjoint copies of the name
space. There are numerous ways to accomplish this. One example can be
found in \cite{MeredithR05}. This notation overloads to vectors of
names: $\vec{x}^{\pi} := (x_{i}^{\pi} \; : \; 0 \leq i < |\vec{x}| )$ where $\pi \in \{L,R\}$.

We also use $P^{\Box} := P|\Box$.

In \cite{MeredithR05} an interpretation of the new operator is
given. It turns out that there are several possible interpretations
all enjoying the requisite algebraic properties of the operator (see
\cite{milner91polyadicpi}). We will therefore make liberal use of
$(\nu\; \vec{x})P$.

% subsection the_syntax_and_semantics_of_the_notation_system (end)   

\input{qm2pi.qmops} 

\input{qm2pi.sterngerlach} 

\input{qm2pi.metric} 

% section concurrent_process_calculi (end)

%\input{qm2pi.proofsketch}

% section proof sketch (end)

%\input{qm2pi.slviaknots} 

% section spatial logic via knots (end)

\input{qm2pi.conclusion}

% section conclusion (end)

%\input{qm2pi.dtcodes} 

% section wiring algorithm (end)

\input{qm2pi.ack} 

% section acknowledgments (end)

\newpage


\bibliographystyle{plain}   
\bibliography{../../biblios/main.bib}

\input{qm2pi.rhodetails}

\end{document}

 

% subsection basic_interpretation (end)

%\input{qm2pi.rho.presentation} 
\subsection{The syntax and semantics of the notation system}\label{sub:the_syntax_and_semantics_of_the_notation_system} % (fold)

We now summarize a technical presentation of the calculus that
embodies our theory of dynamics. The typical presentation of such a
calculus follows the style of giving generators and relations on
them. The grammar, below, describing term constructors, freely
generates the set of processes, $\Proc$. This set is then quotiented
by a relation known as structural congruence and it is over this set
that the notion of dynamics is expressed. This presentation is
essentially that of \cite{MeredithR05} with the addition of
polyadicity and summation. For readability we have relegated some of
the technical subtleties to an appendix.

\subsubsection{Process grammar}\label{subsub:process_grammar}

\begin{mathpar}
  \inferrule* [lab=synchronization] {} {{M} \bc \pzero \;|\; x?F \;|\; x!C }
  \and
  \inferrule* [lab=abstraction] {} {{F} \bc (x)P}
  \and
  \inferrule* [lab=concretion] {} {{C} \bc \langle Q \rangle}
  \and
  \inferrule* [lab=process] {} {{P,Q} \bc M \;| \;P|Q \;|\; @{x}}
  \and
  \inferrule* [lab=name] {} {{x} \bc \quotep{P}}
\end{mathpar} 

Note that $\vec{x}$ (resp. $\vec{P}$) denotes a vector of names
(resp. processes) of length $|\vec{x}|$ (resp. $|\vec{P}|$). We adopt
the following useful abbreviations.

\begin{mathpar}
   x?(\vec{y}).P := x.(\vec{y})P \and  x\clift{\vec{P}} := x.\clift{\vec{P}}
   \and x!(y) := \lift{x}{\dropn{y}}
   \and \Pi_{i=0}^{n-1}P_i := P_0 | \ldots | P_{n-1}
\end{mathpar}

\subsubsection{Structural congruence}

\paragraph{Free and bound names and alpha-equivalence.} At the
core of structural equivalence is alpha-equivalence which identifies
process that are the same up to a change of variable. Formally, we
recognize the distinction between free and bound names. The free names
of a process, $\freenames{P}$, may be calculated recursively as
follows:

\begin{mathpar}
\freenames{\pzero} := \emptyset
  \and \\
  \freenames{x?(y).P} := \{ x \} \cup (\freenames{P} \setminus \{ y \})
  \and 
  \freenames{x!\langle P \rangle} := \{ x \} \cup \{ P \} 
  \and \\
  \freenames{P|Q} := \freenames{P} \cup \freenames{Q}
  \and \\
  \freenames{@{x}} := \{ x \}
\end{mathpar}

$\pi$
$\quotep{\pi}$

$\freenames{-} : \pi \to \mathcal{P}(\quotep{\pi})$

\begin{eqnarray*}
  \freenames{\pzero} & := & \emptyset \\
  \freenames{x?(y).P} & := & \{ x \} \cup (\freenames{P} \setminus \{ y \}) \\
  \freenames{x!\langle P \rangle} & := & \{ x \} \cup \{ P \} \\
  \freenames{P|Q} & := & \freenames{P} \cup \freenames{Q} \\
  \freenames{\dropn{x}} & := & \{ x \}
\end{eqnarray*}

The bound names of a process, $\boundnames{P}$, are those names occurring in $P$
that are not free. For example, in $x?(y).0$, the name $x$ is free, while $y$ is bound.

\begin{mathpar}
  \inferrule* [lab=monoidal-laws] {} { P|Q \equiv Q|P \and P|0 \equiv P \and P|(Q|R) \equiv (P|Q)|R }
\end{mathpar}

\begin{mathpar}
  \inferrule* [lab=alpha-equivalence] {} { (x)P \equiv (y)P\{y/x\} \and y \not\in \freenames{P} }
\end{mathpar}

\begin{definition}
Then two processes, $P,Q$, are alpha-equivalent if $P = Q\{\vec{y}/\vec{x}\}$ for
some $\vec{x} \in \boundnames{Q},\vec{y} \in \boundnames{P}$, where $Q\{\vec{y}/\vec{x}\}$
denotes the capture-avoiding substitution of $\vec{y}$ for $\vec{x}$ in $Q$.
\end{definition}

\begin{definition}
  The {\em structural congruence} \cite{SangiorgiWalker} , $\equiv$,
  between processes is the least congruence containing
  alpha-equivalence, satisfying the abelian monoid laws
  (associativity, commutativity and $\pzero$ as identity) for parallel
  composition $|$ and for summation $+$.
\end{definition}

\subsection{Name equivalence}

We take name equivalence, written $\nameeq$, to be the smallest
equivalence relation generated by the following rules.

\begin{mathpar}
\inferrule*[lab=Quote-drop]
{ }
{ \quotep{@{x}} \nameeq x }

\inferrule*[lab=Struct-equiv]
{ P \scong Q }
{ \quotep{P} \nameeq \quotep{Q} }
\end{mathpar}

The astute reader will have noticed that the mutual recursion of names
and processes imposes a mutual recursion on alpha-equivalence and
structural equivalence via name-equivalence. Fortunately, all of this
works out pleasantly and we may calculate in the natural way, free of
concern. The reader interested in the details is referred to the
appendix \ref{appendix:rho_details}.

\subsection{Substitution}

We use $\Proc$ for the set of processes, $\QProc$ for the set of
names, and $\id{\{}\vec{y} / \vec{x} \id{\}}$ to denote partial maps,
$s : \QProc \rightarrow \QProc$. A map, $s$ lifts, uniquely, to a map
on process terms, $\widehat{s} : \Proc \rightarrow \Proc$ by the
following equations.

\begin{mathpar}
  (0) \psubstp{Q}{P} := 0 \\
  (R \juxtap S) \psubstp{Q}{P}
  :=    
  (R)\psubstp{Q}{P} \juxtap (S) \psubstp{Q}{P} \\
  (x?(y).R) \psubstp{Q}{P}    
  :=    
  (x)\substp{Q}{P} (z)\concat( (R \psubstn{z}{y}) \psubstp{Q}{P} ) \\
  (\lift{x}{R}) \psubstp{Q}{P}  
  :=
  \lift{(x)\substp{Q}{P}}{ R \psubstp{Q}{P} } \\
%   (\dropn{x})  \psubstp{Q}{P}       
%   := 
%   \left\{ 
%     \begin{array}{ccc} 
%       \dropn{\quotep{Q}} & & x \nameeq \quotep{P} \\
%       \dropn{x} & & otherwise \\
%     \end{array}
%   \right. 
  (\dropn{x})  \psubstp{Q}{P}       
  := 
  \left\{ 
    \begin{array}{ccc} 
      Q & & x \nameeq \quotep{P} \\
      \dropn{x} & & otherwise \\
    \end{array}
  \right.
\end{mathpar}
 

where

\begin{eqnarray}
  (x)\id{\{} \lpquote Q \rpquote / \lpquote P \rpquote \id{\}}            = 
  \left\{ 
    \begin{array}{ccc}
      \lpquote Q \rpquote & & x \nameeq \lpquote P \rpquote \\
      x & & otherwise \\
    \end{array}
  \right. \nonumber
\end{eqnarray}

and $z$ is chosen distinct from $\quotep{P}$, $\quotep{Q}$, the free
names in $Q$, and all the names in $R$. Our $\alpha$-equivalence will
be built in the standard way from this substitution.

\begin{remark}\label{rem:no_self_referential_names}
  One consequence of these definitions is that $\forall P. \quotep{P}
  \not\in \freenames{P}$.
\end{remark}

\subsection{ Dynamic quote: an example }

Anticipating something of what's to come, consider applying the
substitution, $\widehat{\id{\{}u / z \id{\}}}$, to the following pair
of processes, $\lift{w}{y!(z)}$ and $w[ \lpquote y!(z) \rpquote ]$.

\begin{eqnarray}
	\lift{w}{y!(z)}\widehat{\id{\{}u / z \id{\}}}
		& = &
		\lift{w}{y!(u)} \nonumber\\
	w[ \lpquote y!(z) \rpquote ] \widehat{ \id{\{}u / z \id{\}} }
		& = &
		w[ \lpquote y!(z) \rpquote ] \nonumber
\end{eqnarray}

Because the body of the process between quotes is impervious to
substitution, we get radically different answers. In fact, by
examining the first process in an input context,
e.g. $x?(z).\lift{w}{y!(z)}$, we see that the process under the lift
operator may be shaped by prefixed inputs binding a name inside it. In
this sense, the lift operator will be seen as a way to dynamically
construct processes before reifying them as names.

Finally equipped with these standard features we can present the
dynamics of the calculus.

\subsubsection{Operational semantics} 

Finally, we introduce the computational dynamics. What marks these
algebras as distinct from other more traditionally studied algebraic
structures, e.g. vector spaces or polynomial rings, is the manner in
which dynamics is captured. In traditional structures, dynamics is typically
expressed through morphisms between such structures, as in linear maps
between vector spaces or morphisms between rings. In algebras
associated with the semantics of computation, the dynamics is
expressed as part of the algebraic structure itself, through a
reduction reduction relation typically denoted by $\red$. Below, we
give a recursive presentation of this relation for the calculus used
in the encoding.

$\red \subseteq \pi \times \pi$
$\red : \pi \to \mathcal{P}(\pi)$

\begin{mathpar}
  \inferrule* [lab=Comm] { \textsf{match}( x_{src}, x_{trgt} ) } { x_{trgt}?(y)P \; | \; x_{src}!\langle {Q} \rangle \red P\{\quotep{Q}/y}\} }
  \and \\
  \inferrule* [lab=Par] {{P} \red {P}'} {{{P} | {Q}} \red {{P}' | {Q}}}
  \and
  \inferrule* [lab=Equiv]{{{P} \scong {P}'} \andalso {{P}' \red {Q}'} \andalso {{Q}' \scong {Q}}}{{P} \red {Q}}
\end{mathpar}

\begin{eqnarray*}
  match_{\equiv} (\quotep{P},\quotep{Q}) & := & P \equiv Q \\
  match_{\dagger}(\quotep{P},\quotep{Q}) & := & \forall R. P|Q \red^{*} R => R \red^{*} 0 \\
  match_{K}(\quotep{P},\quotep{Q}) & := & K \mbox{ for some context } K
\end{eqnarray*}

$u?(x)P | u!\langle Q \rangle \red P\{\quotep{Q}/x\}$

%We write $\wred$ for $\red^*$, and $P\red$ if $\exists Q $ such that $ P \red Q$.
We write $P\red$ if $\exists Q $ such that $ P \red Q$ and $P\not\red$, otherwise.

\section{Replication}

As mentioned before, it is known that replication (and hence
recursion) can be implemented in a higher-order process algebra
\cite{SangiorgiWalker}. As our first example of calculation with the
machinery thus far presented we give the construction explicitly in
the {\rhoc}.

\begin{eqnarray}
	D_{x} & := & \prefix{x}{y}{(\binpar{\outputp{x}{y}}{@{y}})} \nonumber\\
	\bangp_{x}{P} & := & \binpar{{x}!\langle{\binpar{D_{x}}{P}}\rangle}{D_{x}} \nonumber
\end{eqnarray}

\begin{eqnarray}
	\bangp_{x}{P} & & \nonumber\\
	=
	& {x}!\langle{(\prefix{x}{y}{(\outputp{x}{y} | @{y})) | P}}\rangle 
	      | \prefix{x}{y}{(\outputp{x}{y} | @{y})} & \nonumber\\
	\red
	& (\outputp{x}{y} | @{y})\substn{\quotep{(\prefix{x}{y}{(@{y} | \outputp{x}{y})) | P}}}{y} & \nonumber\\
	=
	& \outputp{x}{\quotep{(\prefix{x}{y}{(\outputp{x}{y} | @{y})) | P}}}
	  | {(\prefix{x}{y}{(\outputp{x}{y} | @{y})) | P}} & \nonumber\\
	\red
	& \ldots & \nonumber\\
	\red^*
	& P | P | \ldots & \nonumber
\end{eqnarray}

Of course, this encoding, as an implementation, runs away, unfolding
$\bangp{P}$ eagerly. A lazier and more implementable replication
operator, restricted to input-guarded processes, may be obtained as follows.

\begin{eqnarray}
\bangp{\prefix{u}{v}{P}} 
	:= 
	\binpar{\lift{x}{\prefix{u}{v}{(\binpar{D(x)}{P})}}}{D(x)} \nonumber
\end{eqnarray}

\begin{remark}
  Note that the lazier definition still does not deal with summation
  or mixed summation (i.e. sums over input and output). The reader is
  invited to construct definitions of replication that deal with these
  features. 

  Further, the definitions are parameterized in a name, $x$. Can you,
  gentle reader, make a definition that eliminates this parameter and
  guarantees no accidental interaction between the replication
  machinery and the process being replicated -- i.e. no accidental
  sharing of names used by the process to get its work done and the
  name(s) used by the replication to effect copying. This latter
  revision of the definition of replication is crucial to obtaining
  the expected identity $!!P \sim !P$.
\end{remark}

\begin{remark}\label{rem:paradoxical_combinator}
  The reader familiar with the lambda calculus will have noticed the
  similarity between $D$ and the paradoxical combinator.

  [Ed. note: the existence of this seems to suggest we have to be more
  restrictive on the set of processes and names we admit if we are to
  support no-cloning.]
\end{remark}

\subsubsection{Bisimulation}

The computational dynamics gives rise to another kind of equivalence,
the equivalence of computational behavior. As previously mentioned
this is typically captured \emph{via} some form of bisimulation.

% The notion we use in this paper is weak barbed bisimulation
% \cite{milner91polyadicpi}.

The notion we use in this paper is derived from weak barbed
bisimulation \cite{milner91polyadicpi}. 

\begin{definition}
An \emph{observation relation}, $\downarrow_{\mathcal N}$, over a set
of names, $\mathcal N$, is the smallest relation satisfying the rules
below.

\infrule[Out-barb]{y \in {\mathcal N}, \; x \nameeq y}
		  {\outputp{x}{v} \downarrow_{\mathcal N} x}
\infrule[Par-barb]{\mbox{$P\downarrow_{\mathcal N} x$ or $Q\downarrow_{\mathcal N} x$}}
		  {\binpar{P}{Q} \downarrow_{\mathcal N} x}

We write $P \Downarrow_{\mathcal N} x$ if there is $Q$ such that 
$P \wred Q$ and $Q \downarrow_{\mathcal N} x$.
\end{definition}

\begin{definition}
%\label{def.bbisim}
An  ${\mathcal N}$-\emph{barbed bisimulation} over a set of names, ${\mathcal N}$, is a symmetric binary relation 
${\mathcal S}_{\mathcal N}$ between agents such that $P\rel{S}_{\mathcal N}Q$ implies:
\begin{enumerate}
\item If $P \red P'$ then $Q \wred Q'$ and $P'\rel{S}_{\mathcal N} Q'$.
\item If $P\downarrow_{\mathcal N} x$, then $Q\Downarrow_{\mathcal N} x$.
\end{enumerate}
$P$ is ${\mathcal N}$-barbed bisimilar to $Q$, written
$P \wbbisim_{\mathcal N} Q$, if $P \rel{S}_{\mathcal N} Q$ for some ${\mathcal N}$-barbed bisimulation ${\mathcal S}_{\mathcal N}$.
\end{definition}

$\mathcal{R} \subseteq \pi \times \pi$

$P \mathcal{R} Q => \forall P'. P \red P' \Rightarrow \exists Q'. Q \red Q', P' \mathcal{R} Q'$

$P \vdash x \Rightarrow Q \vdash x$

\begin{mathpar}
  \inferrule*[lab=Out-barb]{x \nameeq y}{{y}!\langle{Q}\rangle \vdash x}
  \and
  \inferrule*[lab=Par-barb]{\mbox{$P\vdash x$ or $Q\vdash x$}}{\binpar{P}{Q} \vdash x}
\end{mathpar}

\subsubsection{Contexts}

One of the principle advantages of computational calculi like the
$\pi$-calculus is a well-defined notion of context,
contextual-equivalence and a correlation between
contextual-equivalence and notions of bisimulation. The notion of
context allows the decomposition of a process into (sub-)process and
its syntactic environment, its context. Thus, a context may be
thought of as a process with a ``hole'' (written $\Box$) in it. The
application of a context $M$ to a process $P$, written $M[P]$, is
tantamount to filling the hole in $M$ with $P$. In this paper we do
not need the full weight of this theory, but do make use of the notion
of context in the proof the main theorem. 

\begin{mathpar}
  \inferrule* [lab=summation] {} {{M_{M},M_{N}} \bc \Box \;|\; x.M_{A} \;|\; M_{M}+M_{N}}
  \and
  \inferrule* [lab=agent] {} {{M_{A}} \bc (\vec{x})M_{P} \;| \; \clift{P_0,\ldots,M_{P},\ldots,P_N}}
  \and \\
  \inferrule* [lab=process] {} {{M_{P}} \bc M_{N} \;| \;P|M_{P} }
\end{mathpar} 

\begin{mathpar}
  \inferrule* [lab=sychronization] {} {M_{N} \bc \Box \;|\; x?M_{F} \;|\; x!M_{C}}
  \and
  \inferrule* [lab=abstraction] {} {{M_{F}} \bc (x)M_{P} }
  \and
  \inferrule* [lab=concretion] {} {{M_{C}} \bc \langle M_{P} \rangle }
  \and \\
  \inferrule* [lab=process] {} {{M_{P}} \bc M_{N} \;| \;P|M_{P} }
\end{mathpar}

\begin{definition}[contextual application] Given a context $M$, and
  process $P$, we define the \emph{contextual application}, $M[P] :=
  M\{P/\Box\}$. That is, the contextual application of M to P is the
  substitution of $P$ for $\Box$ in $M$.
\end{definition}

$\meaningof{-} : L \to \mathcal{P}(\pi)$

\begin{mathpar}
  \inferrule* [lab=collection] {} {\meaningof{true} = \pi, \and \meaningof{~E} = \pi \setminus \meaningof{E}, \and \meaningof{E_{1} \& E_{2}} = \meaningof{E_{1}} \cap \meaningof{E_{2}}}
\end{mathpar}

\begin{mathpar}
  \inferrule* [lab=structure] {} {\meaningof{0} = \{ P \in \pi | P \equiv 0 \}, \and \\ \meaningof{E_1 | E_2} = \{ P \in \pi | P \equiv P_{1} | P_{2}, P_{1} \in \meaningof{E_{1}}, P_{2} \in \meaningof{E_2}\} }
\end{mathpar}

\begin{mathpar}
 \inferrule* [lab=behavior] {} {\meaningof{\langle a?b \rangle E} = \{ P \in \pi | P \equiv Q | u?(y)P', \\ \and \\\\ \and \\ \;\;\; u \in \meaningof{a}, \forall z.P'\{z/y\} \in \meaningof{E\{z/b\}}\}, \and \\ \meaningof{a!E} = \{ P \in \pi | P \equiv Q | x!\langle P' \rangle, x \in \meaningof{a} P' \in \meaningof{E}\} }
\end{mathpar}

\begin{mathpar}
 \inferrule* [lab=nominal] {} {\meaningof{\quotep{E}} = \{ \quotep{P} \in \quotep{\pi} | P \in \meaningof{E} \}, \and \meaningof{\quotep{P}} = \{ \quotep{Q} \in \quotep{\pi} | P \equiv Q \} \and \\ \meaningof{@\quotep{E}} = \{ P \in \pi | P \equiv @x, x \in \meaningof{E} \}}
\end{mathpar}

\begin{eqnarray*}
  \\
  \meaningof{-} : TS \to ST
\end{eqnarray*}

\begin{eqnarray*}
  \\
  L : TS \to ST
\end{eqnarray*}

\begin{eqnarray*}
  \\
  P \models E \iff P \in \meaningof{E}
\end{eqnarray*}

\begin{eqnarray*}
  P \approx_{L} Q \iff \forall E \in L. P \models E \iff Q \models E
\end{eqnarray*}

\begin{eqnarray*}
  P \approx_{K} Q
\end{eqnarray*}

\begin{eqnarray*}
  P \approx Q
\end{eqnarray*}

$\approx_{K} = \approx = \approx_{L}$

\subsubsection{Contextual duality}

Note that contexts extend the quotation operation to a family of
operations from processes to names. Given a context, $M$, we can
define a \emph{nominal context}, $\quotep{M}$ by $\quotep{M}[P] :=
\quotep{M[P]}$. To foreshadow what is to come we observe that these
operations enjoy a duality with processes very much like the duality
between vectors and maps from vectors to scalars.

Further, because the calculus is essentially higher-order, we have a
correspondence between contexts and processes. More specifically,
given a name $x$ and a context $M$ we can construct $M^{*}_{x}$ such
that 

\begin{mathpar}
  M^{*}_{x} | \lift{x}{P} \red M[P]
\end{mathpar}

namely,

\begin{mathpar}
  M^{*}_{x} := x?(u).M[\dropn{u}]
\end{mathpar}

The dependence of $M^{*}_{x}$ on a name makes it an abstraction, 

\begin{mathpar}
  M^{*} := (x)x?(u).M[\dropn{u}]
\end{mathpar}

\subsection{Additional notation}

It will sometimes be convenient to denote the process a name
quotes. We already have the notation $x = \quotep{P}$, but it will be
convenient to introduce an alternate notation, $\procn{x}$, when we
want to emphasize the connection to the use of the name. Note that, by
virtue of name equivalence, $\quotep{\procn{x}} \nameeq x$; so, the
notation is consistent with previous definitions.

Further, because names have structure it is possible to effect
substitutions on the basis of that structure. This means we need to
upgrade our notation for substitutions, which we accomplish by
adapting comprehension notation. Thus,

\begin{mathpar}
  P\{ y / x : x \in S \}
\end{mathpar}

is interpreted to mean the process derived from P by replacing (in a
capture-avoiding manner) each occurrence of $x$ in $S$ by $y$. For example,

\begin{mathpar}
  P\{ \quotep{\procn{x}|\procn{x}} / x : x \in \freenames{P} \}
\end{mathpar}

will replace each (occurrence) of a free name $x$ in $P$ by
$\quotep{\procn{x}|\procn{x}}$.

Also, we will avail ourselves of the notation $x^{L}$ and $x^{R}$ to
denote injections of a name into disjoint copies of the name
space. There are numerous ways to accomplish this. One example can be
found in \cite{MeredithR05}. This notation overloads to vectors of
names: $\vec{x}^{\pi} := (x_{i}^{\pi} \; : \; 0 \leq i < |\vec{x}| )$ where $\pi \in \{L,R\}$.

We also use $P^{\Box} := P|\Box$.

In \cite{MeredithR05} an interpretation of the new operator is
given. It turns out that there are several possible interpretations
all enjoying the requisite algebraic properties of the operator (see
\cite{milner91polyadicpi}). We will therefore make liberal use of
$(\nu\; \vec{x})P$.

% subsection the_syntax_and_semantics_of_the_notation_system (end)   

\section{Interpretation of QM}
\subsection{Supporting definitions}
\subsubsection{Multiplication}
\begin{mathpar}
  \quotep{Q} \cdot \quotep{R} := \quotep{Q|R}
  \and \\
  \quotep{Q} \cdot P := P\{ \quotep{Q|R} / \quotep{R} : \quotep{R} \in \freenames{P} \}
\end{mathpar}

\paragraph{Discussion}
The first line needs little explanation. The second line says that
each free name of the process is replaced with the multiplication of
that name by the scalar. Multiplication of a scalar (name) by a state
(process) results in a process all the names of which have been `moved
over' by parallel composition with the process the scalar
quotes. There is a subtlety that the bound names have to be
manipulated so that multiplied names aren't accidentally
captured. There are many ways to achieve this.

\begin{remark}\label{rem:multiplication_identities}
  The reader is invited to verify that for all $x,y,z \in \QProc$ and $P \in \Proc$
  \begin{mathpar}
    x \cdot \quotep{0} \equiv x 
    \and
    x \cdot y \equiv y \cdot x
    \and
    x \cdot (y \cdot z) \equiv (x \cdot y) \cdot z
    \and \\
    \quotep{0} \cdot P \equiv P
    \and \\
    x \cdot (y \cdot P) \equiv (x \cdot y) \cdot P
    \and \\
    x \cdot (P|Q) \equiv (x \cdot P) | (x \cdot Q)
    \and \\    
  \end{mathpar}
\end{remark}

\subsubsection{Tensor product}

We define a tensor product on processes by structural induction.

\paragraph{Tensor of sums} First note that all summations, including
$\pzero$ and sequence, can be written $\Sigma_{i} x_{i}.A_{i} +
\Sigma_{j} x_{j}.C_{j}$, where we have grouped input-guarded processes
together and output-guarded processes together.

Thus, we can define the tensor product of two summations, $N_{1}\otimes N_{2}$, where

\begin{mathpar}
  N_{1} := \Sigma_{i} x_{i}.A_{i} + \Sigma_{j} x_{j}.C_{j}
  \and
  N_{2} := \Sigma_{i'} y_{i'}.B_{i'} + \Sigma_{j'} y_{j'}.D_{j'} 
\end{mathpar}

as follows.

\begin{mathpar}
  \Sigma_{i} x_{i}.A_{i} + \Sigma_{j} x_{j}.C_{j} \otimes \Sigma_{i'}
  y_{i'}.B_{i'} + \Sigma_{j'} y_{j'}.D_{j'} 
  \and \\
  := \; \Sigma_{i} \Sigma_{i'} \quotep{\stackrel{\vee}{x_{i}}| \stackrel{\vee}{y_{i'}}}.(A_{i}\otimes B_{i'}) \; | \; \Sigma_{i'} \Sigma_{i} \quotep{\stackrel{\vee}{y_{i'}}|\stackrel{\vee}{x_{i}}}.(B_{i'}\otimes A_{i})
  \and
  \;\; | \;\; \Sigma_{j} \Sigma_{j'} \quotep{\stackrel{\vee}{x_{j}}|\stackrel{\vee}{y_{j'}}}.(A_{j}\otimes B_{j'}) \; | \; \Sigma_{j'} \Sigma_{j} \quotep{\stackrel{\vee}{y_{j'}}|\stackrel{\vee}{x_{j}}}.(B_{j'}\otimes A_{j})
\end{mathpar}

\begin{remark}
  Do we need to $x^{L}$ and $y^{R}$ for this construction as well?
\end{remark}

\paragraph{Tensor of parallel compositions} Next, we distribute tensor
over par.

\begin{mathpar}
  P_{1}|P_{2} \otimes Q_{1}|Q_{2} := (P_{1} \otimes Q_{1}) | (P_{1}
  \otimes Q_{2}) | (P_{2} \otimes Q_{1}) | (P_{2} \otimes Q_{2})
\end{mathpar}

\paragraph{Tensor with dropped names} We treat tensor of a
process with a dropped name as parallel composition.

\begin{mathpar}
  P \otimes \dropn{x} := P | \dropn{x}
\end{mathpar}

\paragraph{Tensor of agents}

Finally, we need to define tensor on agents. Note that the definition
of tensor on normal products only tensors inputs with inputs and
outputs with outputs. Thus, we only have to define the operation on
``homogeneous'' pairings.

\begin{mathpar}
  (\vec{x})P \otimes (\vec{y})Q
  \and \\
  := (x_{0}^{L}|y_{0}^{R},\ldots,x_{0}^{L}|y_{n}^{R},\ldots,x_{m}^{L}|y_{0}^{R},\ldots,x_{m}^{L}|y_{n}^R)(P\{ \vec{x}^{L}/\vec{x}\} \otimes Q \{ \vec{y}^{R}/\vec{y}\})
  \and \\
  \clift{\vec{P}} \otimes \clift{\vec{Q}}
  \and \\
  := \clift{P_{0}\otimes Q_{0},\ldots,P_{0}\otimes Q_{n},\ldots,P_{m}\otimes Q_{0},\ldots,P_{m}\otimes Q_{n}}
\end{mathpar}

\begin{remark}
  Observe that arities of tensored abstractions matches arities of
  tensored concretions if the original arities matched. Note also that
  the length of the arities corresponds to the increase in dimension
  we see in ordinary vector space tensor product.
\end{remark}

\begin{remark}
  Operationally, this definition distributes the tensor down to
  components ``linked'' by summation. Tensor over summation is
  intriguing in that it mixes names. Moreover, as a consequence of the
  way it mixes names we have the identities for all $x \in \QProc$ and
  $P,Q \in \Proc$

  \begin{mathpar}
    (x \cdot P) \otimes Q \equiv x \cdot (P \otimes Q) \equiv P \otimes (x \cdot Q)
    \and
    P \otimes \pzero \equiv P
  \end{mathpar}

  that the reader is invited to verify.
\end{remark}

\subsubsection{Annihilation}
\begin{mathpar}
  P^{\perp} := \{ Q | \forall R. P|Q \red^{*} R \Rightarrow R \red^{*} \pzero \}
  \and \\
  P^{\underline{\perp}} := \Sigma_{Q \in P^{\perp}} \quotep{Q}?(y).(\dropn{y}|Q) | \Sigma_{Q \in P^{\perp}} \quotep{Q}\clift{\Box}
\end{mathpar}

\paragraph{Discussion} The reader will note that $P^{\perp}$ is a
\emph{set} of processes, while $P^{\underline{\perp}}$ is a
\emph{context}. We call the set $P^{\perp}$ the \emph{annihilators} of
$P$. The parallel composition of a process in the annihilators of $P$
with $P$ will result in a process, the state space of which has all
paths eventually leading to $\pzero$. Execution may endure loops; but
under reasonable conditions of fairness (naturally guaranteed under
most notions of bisimulation) such a composite process cannot get
stuck in such a loop and will, eventually pop out and terminate.

The context $P^{\underline{\perp}}$ is ready and willing to ``take the
$P$ out of'' the process to which it is applied. It will effectively
transmit the code of the process to which it is applied to one of the
annihilators and run the process against it.

\subsubsection{Evaluation}
We fix $M$ a domain of fully abstract interpretation with an equality
coincident with bisimulation. We take $\meaningof{\cdot} : \Proc \to
M$ to be the map interpreting processes and $\nmeaningof{\cdot} : \M
\to Proc$ to be the map running the other way. Then we define

\begin{mathpar}
  \int P := \nmeaningof{\meaningof{P}}
\end{mathpar}

\paragraph{Discussion}
There are many fully abstract interpretations of Milner's
$\pi$-calculus. Any of them can be used as a basis for interpreting
the reflective calculus here. Equipped with such a domain it is
largely a matter of grinding through to check that the Yoneda
construction for the normalization-by-evaluation program can be
extended to this setting.

\begin{remark}
  The reader is invited to verify that $\int (P^{\underline{\perp}}[P]) = 0$.
\end{remark}

\subsection{Quantum mechanics}

Table \ref{tbl:core_qm_op_defns} gives the core operational definitions

\begin{table}[htp]\label{tbl:core_qm_op_defns}
  \center{
    \fbox{
      \begin{tabular}{c|c}
        quantum mechanics & process calculus \\
        \hline
        scalar & $x := \quotep{P}$ \\
        state vector & $\state{P} := P$ \\
        dual & $\state{P}^{*} := \event{P^{\underline{\perp}}} := \quotep{P^{\underline{\perp}}}[-]$ \\
        matrix & $ \Sigma_{\alpha} \state{P_{\alpha}}x_{\alpha}\event{Q_{\alpha}}$ \\
        vector addition & $\state{P} + \state{Q} := \state{P | Q}$ \\
        tensor product & $\state{P} \otimes \state{Q} := \state{P \otimes Q}$ \\
        inner product & $\innerprod{P}{Q} := \quotep{\int P^{\underline{\perp}}[Q]}$ \\
      \end{tabular}
    }
  }
  \caption{QM - operational definitions}
\end{table}

where

\begin{mathpar}
  \prmatrix{P}{Q} := \fprmatrix{P}{\quotep{\pzero}}{Q}
  \and
  \fprmatrix{P}{x}{Q} := (\state{P},x,\event{Q})
  \and
  (\fprmatrix{P}{x}{Q})(\state{R}) := x \cdot \innerprod{Q}{R} \cdot \state{P}
  \and
  (\fprmatrix{P}{x}{Q})(\event{R}) := x \cdot \innerprod{R}{P} \cdot \event{Q}
\end{mathpar}

\paragraph{Discussion}
As promised: vectors (aka states) are represented as processes; duals
as contextual duals; inner product definition should be compared with
standard inner product definition for ....

\begin{remark}
  Assuming $\int (P^{\underline{\perp}}[P]) = 0$, the reader is
  invited to verify that $(\fprmatrix{P}{x}{P})(\state{P}) = x \cdot \state{P}$.
\end{remark}

\begin{remark}
  The reader is invited to verify that $\innerprod{P}{Q}$ could
  equally well have been written $\quotep{\int \stackrel{\vee}{x}}$
  where $x = \event{P^{\underline{\perp}}}(Q)$.

  One of the motivations for this remark is that there is another way
  to factor these operations. We could package up evaluation in the dual:

  \begin{mathpar}
    \state{P}^{*} := \event{\int P^{\underline{\perp}}} := \quotep{\int P^{\underline{\perp}}}[-]
  \end{mathpar}

  and then have inner product defined by
  
  \begin{mathpar}
    \innerprod{P}{Q} := \event{P}(Q)
  \end{mathpar}

  Hopefully, experience with the calculations will provide guidance on
  the best factoring.
\end{remark}

\begin{remark}
  Assuming $\int (P^{\underline{\perp}}[P]) = 0$, the reader is
  invited to verify that $\forall P,Q. (\prmatrix{0}{Q})(\state{0}) =
  \state{0}$ and dually $(\prmatrix{P}{0})(\event{0}) = \event{0}$.
\end{remark}

\begin{remark}
  i'm a little worried that i don't (yet) have proper support for
  complex conjugacy. But, the observation above may give us a
  clue. According to Abramsky, it must be the case that the scalars
  are iso to the homset of the identity for the tensor -- which the
  observation above characterizes. 

  For now, we will simply bookmark the notion with $\overline{x}$.
\end{remark}

\subsubsection{Adjointness}

We need to give a definition of $(\cdot)^{\dagger}$ for matrices. The
obvious candidate definition is
\begin{mathpar}
(\Sigma_{\alpha}\fprmatrix{P_{\alpha}}{x_{\alpha}}{Q_{\alpha}})^{\dagger}
= \Sigma_{\alpha}\fprmatrix{(Q_{\alpha}^{\underline{\perp}})^{*}}{\overline{x}_{\alpha}}{P_{\alpha}^{\underline{\perp}}} 
\end{mathpar}

But, $(Q_{\alpha}^{\underline{\perp}})^{*}$ requires a name along
which to communicate the process to achieve the context application.

\subsubsection{Basis for a basis}
If processes label states and ``addition'' of states (a.k.a. vector
addition) is interpreted as parallel composition, what corresponds to
notions of linear independence and basis? Here, we recall that Yoshida
has developed a set of \emph{combinators} for an asynchronous verison
of Milner's $\pi$-calculus. These are a finite set of processes such
any process can be expressed as parallel composition of these
combinators together with liberal uses of the new operator and
replication. We can simply give a translation of these into the
present calculus and have reasonable expectation that the property
carries over. That is, that the resultant set allows to express all
processes via parallel composition. Note, however, that there is no
new operator or replication in this calculus. As a result, we expect
that the corresponding set is actually infinite. That is, we expect
that the space is actually infinite dimensional.

\begin{remark}
  The attentive reader may be a bit concerned. Certainly, the
  collection $S$, $K$ and $I$ is a finite set of
  combinators. Shouldn't we expect to see a finite set of combinators
  for an effectively equivalent system? i am very sympathetic to this
  critique and feel it warrants full attention. On the other hand, i
  also have in mind the following analogy. The natural numbers, as a
  monoid under addition, has exactly $1$ generator, while the natural
  numbers, as a monoid under multiplication, has countably many
  generators (the primes). We observe that the application of the
  lambda calculus is much less resource sensitive than the parallel
  composition of the $\pi$-calculus. Could it be the case that we have
  an analogy of the form
  
  \begin{mathpar}
    m + n : MN :: m*n : M|N
  \end{mathpar}

  giving a similar blow up in the set of ``primes''?  This is such a
  wonderful thought that, even if it's not true, i think it's worth
  writing down.
\end{remark}
 

\documentclass[12pt]{llncs}
%\documentclass{jktr}

\usepackage[pdftex]{hyperref}                   
\usepackage {listings}
\usepackage {mathpartir}
\usepackage{bcprules}
%\usepackage{listings}
                       
\usepackage{graphicx} 
%\usepackage[margins=2.5cm,nohead,nofoot]{geometry}
%\usepackage{geometry}
\usepackage{amsfonts}
\usepackage{amstext}
\usepackage{latexsym}
\usepackage{amssymb}
\usepackage{color}


%\include{myPreamble}
\include{qm2pi.local} 

%\ifpdf
%\usepackage[pdftex]{graphicx}
%\else
%\usepackage{graphicx}
%\fi

 % \ifpdf
%  \usepackage{pdfsync}
%  \if


%\title{Brief Article}
%\author{David F. Snyder}
%\author{L.G. Meredith}

%\address{Dept. of Math., Texas State University--San Marcos, San Marcos, TX 78666}
       
\pagestyle{empty}


\begin{document}

\lstset{language=[Objective]Caml,frame=shadowbox}

\input{qm2pi.front}

% section front matter (end)

\input{qm2pi.intro} 
 
% section introduction (end)

% \input{qm2pi.knotations} 

% section notation (end)

\input{qm2pi.process.calculi} 

% section concurrent_process_calculi_and_spatial_logics_ (end)
    
%\input{qm2pi.knots2pi} 

%\input{qm2pi.trefoil} 

%\input{qm2pi.mainthm} 

% subsection basic_interpretation (end)

%\input{qm2pi.rho.presentation} 
\subsection{The syntax and semantics of the notation system}\label{sub:the_syntax_and_semantics_of_the_notation_system} % (fold)

We now summarize a technical presentation of the calculus that
embodies our theory of dynamics. The typical presentation of such a
calculus follows the style of giving generators and relations on
them. The grammar, below, describing term constructors, freely
generates the set of processes, $\Proc$. This set is then quotiented
by a relation known as structural congruence and it is over this set
that the notion of dynamics is expressed. This presentation is
essentially that of \cite{MeredithR05} with the addition of
polyadicity and summation. For readability we have relegated some of
the technical subtleties to an appendix.

\subsubsection{Process grammar}\label{subsub:process_grammar}

\begin{mathpar}
  \inferrule* [lab=synchronization] {} {{M} \bc \pzero \;|\; x?F \;|\; x!C }
  \and
  \inferrule* [lab=abstraction] {} {{F} \bc (x)P}
  \and
  \inferrule* [lab=concretion] {} {{C} \bc \langle Q \rangle}
  \and
  \inferrule* [lab=process] {} {{P,Q} \bc M \;| \;P|Q \;|\; @{x}}
  \and
  \inferrule* [lab=name] {} {{x} \bc \quotep{P}}
\end{mathpar} 

Note that $\vec{x}$ (resp. $\vec{P}$) denotes a vector of names
(resp. processes) of length $|\vec{x}|$ (resp. $|\vec{P}|$). We adopt
the following useful abbreviations.

\begin{mathpar}
   x?(\vec{y}).P := x.(\vec{y})P \and  x\clift{\vec{P}} := x.\clift{\vec{P}}
   \and x!(y) := \lift{x}{\dropn{y}}
   \and \Pi_{i=0}^{n-1}P_i := P_0 | \ldots | P_{n-1}
\end{mathpar}

\subsubsection{Structural congruence}

\paragraph{Free and bound names and alpha-equivalence.} At the
core of structural equivalence is alpha-equivalence which identifies
process that are the same up to a change of variable. Formally, we
recognize the distinction between free and bound names. The free names
of a process, $\freenames{P}$, may be calculated recursively as
follows:

\begin{mathpar}
\freenames{\pzero} := \emptyset
  \and \\
  \freenames{x?(y).P} := \{ x \} \cup (\freenames{P} \setminus \{ y \})
  \and 
  \freenames{x!\langle P \rangle} := \{ x \} \cup \{ P \} 
  \and \\
  \freenames{P|Q} := \freenames{P} \cup \freenames{Q}
  \and \\
  \freenames{@{x}} := \{ x \}
\end{mathpar}

$\pi$
$\quotep{\pi}$

$\freenames{-} : \pi \to \mathcal{P}(\quotep{\pi})$

\begin{eqnarray*}
  \freenames{\pzero} & := & \emptyset \\
  \freenames{x?(y).P} & := & \{ x \} \cup (\freenames{P} \setminus \{ y \}) \\
  \freenames{x!\langle P \rangle} & := & \{ x \} \cup \{ P \} \\
  \freenames{P|Q} & := & \freenames{P} \cup \freenames{Q} \\
  \freenames{\dropn{x}} & := & \{ x \}
\end{eqnarray*}

The bound names of a process, $\boundnames{P}$, are those names occurring in $P$
that are not free. For example, in $x?(y).0$, the name $x$ is free, while $y$ is bound.

\begin{mathpar}
  \inferrule* [lab=monoidal-laws] {} { P|Q \equiv Q|P \and P|0 \equiv P \and P|(Q|R) \equiv (P|Q)|R }
\end{mathpar}

\begin{mathpar}
  \inferrule* [lab=alpha-equivalence] {} { (x)P \equiv (y)P\{y/x\} \and y \not\in \freenames{P} }
\end{mathpar}

\begin{definition}
Then two processes, $P,Q$, are alpha-equivalent if $P = Q\{\vec{y}/\vec{x}\}$ for
some $\vec{x} \in \boundnames{Q},\vec{y} \in \boundnames{P}$, where $Q\{\vec{y}/\vec{x}\}$
denotes the capture-avoiding substitution of $\vec{y}$ for $\vec{x}$ in $Q$.
\end{definition}

\begin{definition}
  The {\em structural congruence} \cite{SangiorgiWalker} , $\equiv$,
  between processes is the least congruence containing
  alpha-equivalence, satisfying the abelian monoid laws
  (associativity, commutativity and $\pzero$ as identity) for parallel
  composition $|$ and for summation $+$.
\end{definition}

\subsection{Name equivalence}

We take name equivalence, written $\nameeq$, to be the smallest
equivalence relation generated by the following rules.

\begin{mathpar}
\inferrule*[lab=Quote-drop]
{ }
{ \quotep{@{x}} \nameeq x }

\inferrule*[lab=Struct-equiv]
{ P \scong Q }
{ \quotep{P} \nameeq \quotep{Q} }
\end{mathpar}

The astute reader will have noticed that the mutual recursion of names
and processes imposes a mutual recursion on alpha-equivalence and
structural equivalence via name-equivalence. Fortunately, all of this
works out pleasantly and we may calculate in the natural way, free of
concern. The reader interested in the details is referred to the
appendix \ref{appendix:rho_details}.

\subsection{Substitution}

We use $\Proc$ for the set of processes, $\QProc$ for the set of
names, and $\id{\{}\vec{y} / \vec{x} \id{\}}$ to denote partial maps,
$s : \QProc \rightarrow \QProc$. A map, $s$ lifts, uniquely, to a map
on process terms, $\widehat{s} : \Proc \rightarrow \Proc$ by the
following equations.

\begin{mathpar}
  (0) \psubstp{Q}{P} := 0 \\
  (R \juxtap S) \psubstp{Q}{P}
  :=    
  (R)\psubstp{Q}{P} \juxtap (S) \psubstp{Q}{P} \\
  (x?(y).R) \psubstp{Q}{P}    
  :=    
  (x)\substp{Q}{P} (z)\concat( (R \psubstn{z}{y}) \psubstp{Q}{P} ) \\
  (\lift{x}{R}) \psubstp{Q}{P}  
  :=
  \lift{(x)\substp{Q}{P}}{ R \psubstp{Q}{P} } \\
%   (\dropn{x})  \psubstp{Q}{P}       
%   := 
%   \left\{ 
%     \begin{array}{ccc} 
%       \dropn{\quotep{Q}} & & x \nameeq \quotep{P} \\
%       \dropn{x} & & otherwise \\
%     \end{array}
%   \right. 
  (\dropn{x})  \psubstp{Q}{P}       
  := 
  \left\{ 
    \begin{array}{ccc} 
      Q & & x \nameeq \quotep{P} \\
      \dropn{x} & & otherwise \\
    \end{array}
  \right.
\end{mathpar}
 

where

\begin{eqnarray}
  (x)\id{\{} \lpquote Q \rpquote / \lpquote P \rpquote \id{\}}            = 
  \left\{ 
    \begin{array}{ccc}
      \lpquote Q \rpquote & & x \nameeq \lpquote P \rpquote \\
      x & & otherwise \\
    \end{array}
  \right. \nonumber
\end{eqnarray}

and $z$ is chosen distinct from $\quotep{P}$, $\quotep{Q}$, the free
names in $Q$, and all the names in $R$. Our $\alpha$-equivalence will
be built in the standard way from this substitution.

\begin{remark}\label{rem:no_self_referential_names}
  One consequence of these definitions is that $\forall P. \quotep{P}
  \not\in \freenames{P}$.
\end{remark}

\subsection{ Dynamic quote: an example }

Anticipating something of what's to come, consider applying the
substitution, $\widehat{\id{\{}u / z \id{\}}}$, to the following pair
of processes, $\lift{w}{y!(z)}$ and $w[ \lpquote y!(z) \rpquote ]$.

\begin{eqnarray}
	\lift{w}{y!(z)}\widehat{\id{\{}u / z \id{\}}}
		& = &
		\lift{w}{y!(u)} \nonumber\\
	w[ \lpquote y!(z) \rpquote ] \widehat{ \id{\{}u / z \id{\}} }
		& = &
		w[ \lpquote y!(z) \rpquote ] \nonumber
\end{eqnarray}

Because the body of the process between quotes is impervious to
substitution, we get radically different answers. In fact, by
examining the first process in an input context,
e.g. $x?(z).\lift{w}{y!(z)}$, we see that the process under the lift
operator may be shaped by prefixed inputs binding a name inside it. In
this sense, the lift operator will be seen as a way to dynamically
construct processes before reifying them as names.

Finally equipped with these standard features we can present the
dynamics of the calculus.

\subsubsection{Operational semantics} 

Finally, we introduce the computational dynamics. What marks these
algebras as distinct from other more traditionally studied algebraic
structures, e.g. vector spaces or polynomial rings, is the manner in
which dynamics is captured. In traditional structures, dynamics is typically
expressed through morphisms between such structures, as in linear maps
between vector spaces or morphisms between rings. In algebras
associated with the semantics of computation, the dynamics is
expressed as part of the algebraic structure itself, through a
reduction reduction relation typically denoted by $\red$. Below, we
give a recursive presentation of this relation for the calculus used
in the encoding.

$\red \subseteq \pi \times \pi$
$\red : \pi \to \mathcal{P}(\pi)$

\begin{mathpar}
  \inferrule* [lab=Comm] { \textsf{match}( x_{src}, x_{trgt} ) } { x_{trgt}?(y)P \; | \; x_{src}!\langle {Q} \rangle \red P\{\quotep{Q}/y}\} }
  \and \\
  \inferrule* [lab=Par] {{P} \red {P}'} {{{P} | {Q}} \red {{P}' | {Q}}}
  \and
  \inferrule* [lab=Equiv]{{{P} \scong {P}'} \andalso {{P}' \red {Q}'} \andalso {{Q}' \scong {Q}}}{{P} \red {Q}}
\end{mathpar}

\begin{eqnarray*}
  match_{\equiv} (\quotep{P},\quotep{Q}) & := & P \equiv Q \\
  match_{\dagger}(\quotep{P},\quotep{Q}) & := & \forall R. P|Q \red^{*} R => R \red^{*} 0 \\
  match_{K}(\quotep{P},\quotep{Q}) & := & K \mbox{ for some context } K
\end{eqnarray*}

$u?(x)P | u!\langle Q \rangle \red P\{\quotep{Q}/x\}$

%We write $\wred$ for $\red^*$, and $P\red$ if $\exists Q $ such that $ P \red Q$.
We write $P\red$ if $\exists Q $ such that $ P \red Q$ and $P\not\red$, otherwise.

\section{Replication}

As mentioned before, it is known that replication (and hence
recursion) can be implemented in a higher-order process algebra
\cite{SangiorgiWalker}. As our first example of calculation with the
machinery thus far presented we give the construction explicitly in
the {\rhoc}.

\begin{eqnarray}
	D_{x} & := & \prefix{x}{y}{(\binpar{\outputp{x}{y}}{@{y}})} \nonumber\\
	\bangp_{x}{P} & := & \binpar{{x}!\langle{\binpar{D_{x}}{P}}\rangle}{D_{x}} \nonumber
\end{eqnarray}

\begin{eqnarray}
	\bangp_{x}{P} & & \nonumber\\
	=
	& {x}!\langle{(\prefix{x}{y}{(\outputp{x}{y} | @{y})) | P}}\rangle 
	      | \prefix{x}{y}{(\outputp{x}{y} | @{y})} & \nonumber\\
	\red
	& (\outputp{x}{y} | @{y})\substn{\quotep{(\prefix{x}{y}{(@{y} | \outputp{x}{y})) | P}}}{y} & \nonumber\\
	=
	& \outputp{x}{\quotep{(\prefix{x}{y}{(\outputp{x}{y} | @{y})) | P}}}
	  | {(\prefix{x}{y}{(\outputp{x}{y} | @{y})) | P}} & \nonumber\\
	\red
	& \ldots & \nonumber\\
	\red^*
	& P | P | \ldots & \nonumber
\end{eqnarray}

Of course, this encoding, as an implementation, runs away, unfolding
$\bangp{P}$ eagerly. A lazier and more implementable replication
operator, restricted to input-guarded processes, may be obtained as follows.

\begin{eqnarray}
\bangp{\prefix{u}{v}{P}} 
	:= 
	\binpar{\lift{x}{\prefix{u}{v}{(\binpar{D(x)}{P})}}}{D(x)} \nonumber
\end{eqnarray}

\begin{remark}
  Note that the lazier definition still does not deal with summation
  or mixed summation (i.e. sums over input and output). The reader is
  invited to construct definitions of replication that deal with these
  features. 

  Further, the definitions are parameterized in a name, $x$. Can you,
  gentle reader, make a definition that eliminates this parameter and
  guarantees no accidental interaction between the replication
  machinery and the process being replicated -- i.e. no accidental
  sharing of names used by the process to get its work done and the
  name(s) used by the replication to effect copying. This latter
  revision of the definition of replication is crucial to obtaining
  the expected identity $!!P \sim !P$.
\end{remark}

\begin{remark}\label{rem:paradoxical_combinator}
  The reader familiar with the lambda calculus will have noticed the
  similarity between $D$ and the paradoxical combinator.

  [Ed. note: the existence of this seems to suggest we have to be more
  restrictive on the set of processes and names we admit if we are to
  support no-cloning.]
\end{remark}

\subsubsection{Bisimulation}

The computational dynamics gives rise to another kind of equivalence,
the equivalence of computational behavior. As previously mentioned
this is typically captured \emph{via} some form of bisimulation.

% The notion we use in this paper is weak barbed bisimulation
% \cite{milner91polyadicpi}.

The notion we use in this paper is derived from weak barbed
bisimulation \cite{milner91polyadicpi}. 

\begin{definition}
An \emph{observation relation}, $\downarrow_{\mathcal N}$, over a set
of names, $\mathcal N$, is the smallest relation satisfying the rules
below.

\infrule[Out-barb]{y \in {\mathcal N}, \; x \nameeq y}
		  {\outputp{x}{v} \downarrow_{\mathcal N} x}
\infrule[Par-barb]{\mbox{$P\downarrow_{\mathcal N} x$ or $Q\downarrow_{\mathcal N} x$}}
		  {\binpar{P}{Q} \downarrow_{\mathcal N} x}

We write $P \Downarrow_{\mathcal N} x$ if there is $Q$ such that 
$P \wred Q$ and $Q \downarrow_{\mathcal N} x$.
\end{definition}

\begin{definition}
%\label{def.bbisim}
An  ${\mathcal N}$-\emph{barbed bisimulation} over a set of names, ${\mathcal N}$, is a symmetric binary relation 
${\mathcal S}_{\mathcal N}$ between agents such that $P\rel{S}_{\mathcal N}Q$ implies:
\begin{enumerate}
\item If $P \red P'$ then $Q \wred Q'$ and $P'\rel{S}_{\mathcal N} Q'$.
\item If $P\downarrow_{\mathcal N} x$, then $Q\Downarrow_{\mathcal N} x$.
\end{enumerate}
$P$ is ${\mathcal N}$-barbed bisimilar to $Q$, written
$P \wbbisim_{\mathcal N} Q$, if $P \rel{S}_{\mathcal N} Q$ for some ${\mathcal N}$-barbed bisimulation ${\mathcal S}_{\mathcal N}$.
\end{definition}

$\mathcal{R} \subseteq \pi \times \pi$

$P \mathcal{R} Q => \forall P'. P \red P' \Rightarrow \exists Q'. Q \red Q', P' \mathcal{R} Q'$

$P \vdash x \Rightarrow Q \vdash x$

\begin{mathpar}
  \inferrule*[lab=Out-barb]{x \nameeq y}{{y}!\langle{Q}\rangle \vdash x}
  \and
  \inferrule*[lab=Par-barb]{\mbox{$P\vdash x$ or $Q\vdash x$}}{\binpar{P}{Q} \vdash x}
\end{mathpar}

\subsubsection{Contexts}

One of the principle advantages of computational calculi like the
$\pi$-calculus is a well-defined notion of context,
contextual-equivalence and a correlation between
contextual-equivalence and notions of bisimulation. The notion of
context allows the decomposition of a process into (sub-)process and
its syntactic environment, its context. Thus, a context may be
thought of as a process with a ``hole'' (written $\Box$) in it. The
application of a context $M$ to a process $P$, written $M[P]$, is
tantamount to filling the hole in $M$ with $P$. In this paper we do
not need the full weight of this theory, but do make use of the notion
of context in the proof the main theorem. 

\begin{mathpar}
  \inferrule* [lab=summation] {} {{M_{M},M_{N}} \bc \Box \;|\; x.M_{A} \;|\; M_{M}+M_{N}}
  \and
  \inferrule* [lab=agent] {} {{M_{A}} \bc (\vec{x})M_{P} \;| \; \clift{P_0,\ldots,M_{P},\ldots,P_N}}
  \and \\
  \inferrule* [lab=process] {} {{M_{P}} \bc M_{N} \;| \;P|M_{P} }
\end{mathpar} 

\begin{mathpar}
  \inferrule* [lab=sychronization] {} {M_{N} \bc \Box \;|\; x?M_{F} \;|\; x!M_{C}}
  \and
  \inferrule* [lab=abstraction] {} {{M_{F}} \bc (x)M_{P} }
  \and
  \inferrule* [lab=concretion] {} {{M_{C}} \bc \langle M_{P} \rangle }
  \and \\
  \inferrule* [lab=process] {} {{M_{P}} \bc M_{N} \;| \;P|M_{P} }
\end{mathpar}

\begin{definition}[contextual application] Given a context $M$, and
  process $P$, we define the \emph{contextual application}, $M[P] :=
  M\{P/\Box\}$. That is, the contextual application of M to P is the
  substitution of $P$ for $\Box$ in $M$.
\end{definition}

$\meaningof{-} : L \to \mathcal{P}(\pi)$

\begin{mathpar}
  \inferrule* [lab=collection] {} {\meaningof{true} = \pi, \and \meaningof{~E} = \pi \setminus \meaningof{E}, \and \meaningof{E_{1} \& E_{2}} = \meaningof{E_{1}} \cap \meaningof{E_{2}}}
\end{mathpar}

\begin{mathpar}
  \inferrule* [lab=structure] {} {\meaningof{0} = \{ P \in \pi | P \equiv 0 \}, \and \\ \meaningof{E_1 | E_2} = \{ P \in \pi | P \equiv P_{1} | P_{2}, P_{1} \in \meaningof{E_{1}}, P_{2} \in \meaningof{E_2}\} }
\end{mathpar}

\begin{mathpar}
 \inferrule* [lab=behavior] {} {\meaningof{\langle a?b \rangle E} = \{ P \in \pi | P \equiv Q | u?(y)P', \\ \and \\\\ \and \\ \;\;\; u \in \meaningof{a}, \forall z.P'\{z/y\} \in \meaningof{E\{z/b\}}\}, \and \\ \meaningof{a!E} = \{ P \in \pi | P \equiv Q | x!\langle P' \rangle, x \in \meaningof{a} P' \in \meaningof{E}\} }
\end{mathpar}

\begin{mathpar}
 \inferrule* [lab=nominal] {} {\meaningof{\quotep{E}} = \{ \quotep{P} \in \quotep{\pi} | P \in \meaningof{E} \}, \and \meaningof{\quotep{P}} = \{ \quotep{Q} \in \quotep{\pi} | P \equiv Q \} \and \\ \meaningof{@\quotep{E}} = \{ P \in \pi | P \equiv @x, x \in \meaningof{E} \}}
\end{mathpar}

\begin{eqnarray*}
  \\
  \meaningof{-} : TS \to ST
\end{eqnarray*}

\begin{eqnarray*}
  \\
  L : TS \to ST
\end{eqnarray*}

\begin{eqnarray*}
  \\
  P \models E \iff P \in \meaningof{E}
\end{eqnarray*}

\begin{eqnarray*}
  P \approx_{L} Q \iff \forall E \in L. P \models E \iff Q \models E
\end{eqnarray*}

\begin{eqnarray*}
  P \approx_{K} Q
\end{eqnarray*}

\begin{eqnarray*}
  P \approx Q
\end{eqnarray*}

$\approx_{K} = \approx = \approx_{L}$

\subsubsection{Contextual duality}

Note that contexts extend the quotation operation to a family of
operations from processes to names. Given a context, $M$, we can
define a \emph{nominal context}, $\quotep{M}$ by $\quotep{M}[P] :=
\quotep{M[P]}$. To foreshadow what is to come we observe that these
operations enjoy a duality with processes very much like the duality
between vectors and maps from vectors to scalars.

Further, because the calculus is essentially higher-order, we have a
correspondence between contexts and processes. More specifically,
given a name $x$ and a context $M$ we can construct $M^{*}_{x}$ such
that 

\begin{mathpar}
  M^{*}_{x} | \lift{x}{P} \red M[P]
\end{mathpar}

namely,

\begin{mathpar}
  M^{*}_{x} := x?(u).M[\dropn{u}]
\end{mathpar}

The dependence of $M^{*}_{x}$ on a name makes it an abstraction, 

\begin{mathpar}
  M^{*} := (x)x?(u).M[\dropn{u}]
\end{mathpar}

\subsection{Additional notation}

It will sometimes be convenient to denote the process a name
quotes. We already have the notation $x = \quotep{P}$, but it will be
convenient to introduce an alternate notation, $\procn{x}$, when we
want to emphasize the connection to the use of the name. Note that, by
virtue of name equivalence, $\quotep{\procn{x}} \nameeq x$; so, the
notation is consistent with previous definitions.

Further, because names have structure it is possible to effect
substitutions on the basis of that structure. This means we need to
upgrade our notation for substitutions, which we accomplish by
adapting comprehension notation. Thus,

\begin{mathpar}
  P\{ y / x : x \in S \}
\end{mathpar}

is interpreted to mean the process derived from P by replacing (in a
capture-avoiding manner) each occurrence of $x$ in $S$ by $y$. For example,

\begin{mathpar}
  P\{ \quotep{\procn{x}|\procn{x}} / x : x \in \freenames{P} \}
\end{mathpar}

will replace each (occurrence) of a free name $x$ in $P$ by
$\quotep{\procn{x}|\procn{x}}$.

Also, we will avail ourselves of the notation $x^{L}$ and $x^{R}$ to
denote injections of a name into disjoint copies of the name
space. There are numerous ways to accomplish this. One example can be
found in \cite{MeredithR05}. This notation overloads to vectors of
names: $\vec{x}^{\pi} := (x_{i}^{\pi} \; : \; 0 \leq i < |\vec{x}| )$ where $\pi \in \{L,R\}$.

We also use $P^{\Box} := P|\Box$.

In \cite{MeredithR05} an interpretation of the new operator is
given. It turns out that there are several possible interpretations
all enjoying the requisite algebraic properties of the operator (see
\cite{milner91polyadicpi}). We will therefore make liberal use of
$(\nu\; \vec{x})P$.

% subsection the_syntax_and_semantics_of_the_notation_system (end)   

\input{qm2pi.qmops} 

\input{qm2pi.sterngerlach} 

\input{qm2pi.metric} 

% section concurrent_process_calculi (end)

%\input{qm2pi.proofsketch}

% section proof sketch (end)

%\input{qm2pi.slviaknots} 

% section spatial logic via knots (end)

\input{qm2pi.conclusion}

% section conclusion (end)

%\input{qm2pi.dtcodes} 

% section wiring algorithm (end)

\input{qm2pi.ack} 

% section acknowledgments (end)

\newpage


\bibliographystyle{plain}   
\bibliography{../../biblios/main.bib}

\input{qm2pi.rhodetails}

\end{document}

 

\documentclass[12pt]{llncs}
%\documentclass{jktr}

\usepackage[pdftex]{hyperref}                   
\usepackage {listings}
\usepackage {mathpartir}
\usepackage{bcprules}
%\usepackage{listings}
                       
\usepackage{graphicx} 
%\usepackage[margins=2.5cm,nohead,nofoot]{geometry}
%\usepackage{geometry}
\usepackage{amsfonts}
\usepackage{amstext}
\usepackage{latexsym}
\usepackage{amssymb}
\usepackage{color}


%\include{myPreamble}
\include{qm2pi.local} 

%\ifpdf
%\usepackage[pdftex]{graphicx}
%\else
%\usepackage{graphicx}
%\fi

 % \ifpdf
%  \usepackage{pdfsync}
%  \if


%\title{Brief Article}
%\author{David F. Snyder}
%\author{L.G. Meredith}

%\address{Dept. of Math., Texas State University--San Marcos, San Marcos, TX 78666}
       
\pagestyle{empty}


\begin{document}

\lstset{language=[Objective]Caml,frame=shadowbox}

\input{qm2pi.front}

% section front matter (end)

\input{qm2pi.intro} 
 
% section introduction (end)

% \input{qm2pi.knotations} 

% section notation (end)

\input{qm2pi.process.calculi} 

% section concurrent_process_calculi_and_spatial_logics_ (end)
    
%\input{qm2pi.knots2pi} 

%\input{qm2pi.trefoil} 

%\input{qm2pi.mainthm} 

% subsection basic_interpretation (end)

%\input{qm2pi.rho.presentation} 
\subsection{The syntax and semantics of the notation system}\label{sub:the_syntax_and_semantics_of_the_notation_system} % (fold)

We now summarize a technical presentation of the calculus that
embodies our theory of dynamics. The typical presentation of such a
calculus follows the style of giving generators and relations on
them. The grammar, below, describing term constructors, freely
generates the set of processes, $\Proc$. This set is then quotiented
by a relation known as structural congruence and it is over this set
that the notion of dynamics is expressed. This presentation is
essentially that of \cite{MeredithR05} with the addition of
polyadicity and summation. For readability we have relegated some of
the technical subtleties to an appendix.

\subsubsection{Process grammar}\label{subsub:process_grammar}

\begin{mathpar}
  \inferrule* [lab=synchronization] {} {{M} \bc \pzero \;|\; x?F \;|\; x!C }
  \and
  \inferrule* [lab=abstraction] {} {{F} \bc (x)P}
  \and
  \inferrule* [lab=concretion] {} {{C} \bc \langle Q \rangle}
  \and
  \inferrule* [lab=process] {} {{P,Q} \bc M \;| \;P|Q \;|\; @{x}}
  \and
  \inferrule* [lab=name] {} {{x} \bc \quotep{P}}
\end{mathpar} 

Note that $\vec{x}$ (resp. $\vec{P}$) denotes a vector of names
(resp. processes) of length $|\vec{x}|$ (resp. $|\vec{P}|$). We adopt
the following useful abbreviations.

\begin{mathpar}
   x?(\vec{y}).P := x.(\vec{y})P \and  x\clift{\vec{P}} := x.\clift{\vec{P}}
   \and x!(y) := \lift{x}{\dropn{y}}
   \and \Pi_{i=0}^{n-1}P_i := P_0 | \ldots | P_{n-1}
\end{mathpar}

\subsubsection{Structural congruence}

\paragraph{Free and bound names and alpha-equivalence.} At the
core of structural equivalence is alpha-equivalence which identifies
process that are the same up to a change of variable. Formally, we
recognize the distinction between free and bound names. The free names
of a process, $\freenames{P}$, may be calculated recursively as
follows:

\begin{mathpar}
\freenames{\pzero} := \emptyset
  \and \\
  \freenames{x?(y).P} := \{ x \} \cup (\freenames{P} \setminus \{ y \})
  \and 
  \freenames{x!\langle P \rangle} := \{ x \} \cup \{ P \} 
  \and \\
  \freenames{P|Q} := \freenames{P} \cup \freenames{Q}
  \and \\
  \freenames{@{x}} := \{ x \}
\end{mathpar}

$\pi$
$\quotep{\pi}$

$\freenames{-} : \pi \to \mathcal{P}(\quotep{\pi})$

\begin{eqnarray*}
  \freenames{\pzero} & := & \emptyset \\
  \freenames{x?(y).P} & := & \{ x \} \cup (\freenames{P} \setminus \{ y \}) \\
  \freenames{x!\langle P \rangle} & := & \{ x \} \cup \{ P \} \\
  \freenames{P|Q} & := & \freenames{P} \cup \freenames{Q} \\
  \freenames{\dropn{x}} & := & \{ x \}
\end{eqnarray*}

The bound names of a process, $\boundnames{P}$, are those names occurring in $P$
that are not free. For example, in $x?(y).0$, the name $x$ is free, while $y$ is bound.

\begin{mathpar}
  \inferrule* [lab=monoidal-laws] {} { P|Q \equiv Q|P \and P|0 \equiv P \and P|(Q|R) \equiv (P|Q)|R }
\end{mathpar}

\begin{mathpar}
  \inferrule* [lab=alpha-equivalence] {} { (x)P \equiv (y)P\{y/x\} \and y \not\in \freenames{P} }
\end{mathpar}

\begin{definition}
Then two processes, $P,Q$, are alpha-equivalent if $P = Q\{\vec{y}/\vec{x}\}$ for
some $\vec{x} \in \boundnames{Q},\vec{y} \in \boundnames{P}$, where $Q\{\vec{y}/\vec{x}\}$
denotes the capture-avoiding substitution of $\vec{y}$ for $\vec{x}$ in $Q$.
\end{definition}

\begin{definition}
  The {\em structural congruence} \cite{SangiorgiWalker} , $\equiv$,
  between processes is the least congruence containing
  alpha-equivalence, satisfying the abelian monoid laws
  (associativity, commutativity and $\pzero$ as identity) for parallel
  composition $|$ and for summation $+$.
\end{definition}

\subsection{Name equivalence}

We take name equivalence, written $\nameeq$, to be the smallest
equivalence relation generated by the following rules.

\begin{mathpar}
\inferrule*[lab=Quote-drop]
{ }
{ \quotep{@{x}} \nameeq x }

\inferrule*[lab=Struct-equiv]
{ P \scong Q }
{ \quotep{P} \nameeq \quotep{Q} }
\end{mathpar}

The astute reader will have noticed that the mutual recursion of names
and processes imposes a mutual recursion on alpha-equivalence and
structural equivalence via name-equivalence. Fortunately, all of this
works out pleasantly and we may calculate in the natural way, free of
concern. The reader interested in the details is referred to the
appendix \ref{appendix:rho_details}.

\subsection{Substitution}

We use $\Proc$ for the set of processes, $\QProc$ for the set of
names, and $\id{\{}\vec{y} / \vec{x} \id{\}}$ to denote partial maps,
$s : \QProc \rightarrow \QProc$. A map, $s$ lifts, uniquely, to a map
on process terms, $\widehat{s} : \Proc \rightarrow \Proc$ by the
following equations.

\begin{mathpar}
  (0) \psubstp{Q}{P} := 0 \\
  (R \juxtap S) \psubstp{Q}{P}
  :=    
  (R)\psubstp{Q}{P} \juxtap (S) \psubstp{Q}{P} \\
  (x?(y).R) \psubstp{Q}{P}    
  :=    
  (x)\substp{Q}{P} (z)\concat( (R \psubstn{z}{y}) \psubstp{Q}{P} ) \\
  (\lift{x}{R}) \psubstp{Q}{P}  
  :=
  \lift{(x)\substp{Q}{P}}{ R \psubstp{Q}{P} } \\
%   (\dropn{x})  \psubstp{Q}{P}       
%   := 
%   \left\{ 
%     \begin{array}{ccc} 
%       \dropn{\quotep{Q}} & & x \nameeq \quotep{P} \\
%       \dropn{x} & & otherwise \\
%     \end{array}
%   \right. 
  (\dropn{x})  \psubstp{Q}{P}       
  := 
  \left\{ 
    \begin{array}{ccc} 
      Q & & x \nameeq \quotep{P} \\
      \dropn{x} & & otherwise \\
    \end{array}
  \right.
\end{mathpar}
 

where

\begin{eqnarray}
  (x)\id{\{} \lpquote Q \rpquote / \lpquote P \rpquote \id{\}}            = 
  \left\{ 
    \begin{array}{ccc}
      \lpquote Q \rpquote & & x \nameeq \lpquote P \rpquote \\
      x & & otherwise \\
    \end{array}
  \right. \nonumber
\end{eqnarray}

and $z$ is chosen distinct from $\quotep{P}$, $\quotep{Q}$, the free
names in $Q$, and all the names in $R$. Our $\alpha$-equivalence will
be built in the standard way from this substitution.

\begin{remark}\label{rem:no_self_referential_names}
  One consequence of these definitions is that $\forall P. \quotep{P}
  \not\in \freenames{P}$.
\end{remark}

\subsection{ Dynamic quote: an example }

Anticipating something of what's to come, consider applying the
substitution, $\widehat{\id{\{}u / z \id{\}}}$, to the following pair
of processes, $\lift{w}{y!(z)}$ and $w[ \lpquote y!(z) \rpquote ]$.

\begin{eqnarray}
	\lift{w}{y!(z)}\widehat{\id{\{}u / z \id{\}}}
		& = &
		\lift{w}{y!(u)} \nonumber\\
	w[ \lpquote y!(z) \rpquote ] \widehat{ \id{\{}u / z \id{\}} }
		& = &
		w[ \lpquote y!(z) \rpquote ] \nonumber
\end{eqnarray}

Because the body of the process between quotes is impervious to
substitution, we get radically different answers. In fact, by
examining the first process in an input context,
e.g. $x?(z).\lift{w}{y!(z)}$, we see that the process under the lift
operator may be shaped by prefixed inputs binding a name inside it. In
this sense, the lift operator will be seen as a way to dynamically
construct processes before reifying them as names.

Finally equipped with these standard features we can present the
dynamics of the calculus.

\subsubsection{Operational semantics} 

Finally, we introduce the computational dynamics. What marks these
algebras as distinct from other more traditionally studied algebraic
structures, e.g. vector spaces or polynomial rings, is the manner in
which dynamics is captured. In traditional structures, dynamics is typically
expressed through morphisms between such structures, as in linear maps
between vector spaces or morphisms between rings. In algebras
associated with the semantics of computation, the dynamics is
expressed as part of the algebraic structure itself, through a
reduction reduction relation typically denoted by $\red$. Below, we
give a recursive presentation of this relation for the calculus used
in the encoding.

$\red \subseteq \pi \times \pi$
$\red : \pi \to \mathcal{P}(\pi)$

\begin{mathpar}
  \inferrule* [lab=Comm] { \textsf{match}( x_{src}, x_{trgt} ) } { x_{trgt}?(y)P \; | \; x_{src}!\langle {Q} \rangle \red P\{\quotep{Q}/y}\} }
  \and \\
  \inferrule* [lab=Par] {{P} \red {P}'} {{{P} | {Q}} \red {{P}' | {Q}}}
  \and
  \inferrule* [lab=Equiv]{{{P} \scong {P}'} \andalso {{P}' \red {Q}'} \andalso {{Q}' \scong {Q}}}{{P} \red {Q}}
\end{mathpar}

\begin{eqnarray*}
  match_{\equiv} (\quotep{P},\quotep{Q}) & := & P \equiv Q \\
  match_{\dagger}(\quotep{P},\quotep{Q}) & := & \forall R. P|Q \red^{*} R => R \red^{*} 0 \\
  match_{K}(\quotep{P},\quotep{Q}) & := & K \mbox{ for some context } K
\end{eqnarray*}

$u?(x)P | u!\langle Q \rangle \red P\{\quotep{Q}/x\}$

%We write $\wred$ for $\red^*$, and $P\red$ if $\exists Q $ such that $ P \red Q$.
We write $P\red$ if $\exists Q $ such that $ P \red Q$ and $P\not\red$, otherwise.

\section{Replication}

As mentioned before, it is known that replication (and hence
recursion) can be implemented in a higher-order process algebra
\cite{SangiorgiWalker}. As our first example of calculation with the
machinery thus far presented we give the construction explicitly in
the {\rhoc}.

\begin{eqnarray}
	D_{x} & := & \prefix{x}{y}{(\binpar{\outputp{x}{y}}{@{y}})} \nonumber\\
	\bangp_{x}{P} & := & \binpar{{x}!\langle{\binpar{D_{x}}{P}}\rangle}{D_{x}} \nonumber
\end{eqnarray}

\begin{eqnarray}
	\bangp_{x}{P} & & \nonumber\\
	=
	& {x}!\langle{(\prefix{x}{y}{(\outputp{x}{y} | @{y})) | P}}\rangle 
	      | \prefix{x}{y}{(\outputp{x}{y} | @{y})} & \nonumber\\
	\red
	& (\outputp{x}{y} | @{y})\substn{\quotep{(\prefix{x}{y}{(@{y} | \outputp{x}{y})) | P}}}{y} & \nonumber\\
	=
	& \outputp{x}{\quotep{(\prefix{x}{y}{(\outputp{x}{y} | @{y})) | P}}}
	  | {(\prefix{x}{y}{(\outputp{x}{y} | @{y})) | P}} & \nonumber\\
	\red
	& \ldots & \nonumber\\
	\red^*
	& P | P | \ldots & \nonumber
\end{eqnarray}

Of course, this encoding, as an implementation, runs away, unfolding
$\bangp{P}$ eagerly. A lazier and more implementable replication
operator, restricted to input-guarded processes, may be obtained as follows.

\begin{eqnarray}
\bangp{\prefix{u}{v}{P}} 
	:= 
	\binpar{\lift{x}{\prefix{u}{v}{(\binpar{D(x)}{P})}}}{D(x)} \nonumber
\end{eqnarray}

\begin{remark}
  Note that the lazier definition still does not deal with summation
  or mixed summation (i.e. sums over input and output). The reader is
  invited to construct definitions of replication that deal with these
  features. 

  Further, the definitions are parameterized in a name, $x$. Can you,
  gentle reader, make a definition that eliminates this parameter and
  guarantees no accidental interaction between the replication
  machinery and the process being replicated -- i.e. no accidental
  sharing of names used by the process to get its work done and the
  name(s) used by the replication to effect copying. This latter
  revision of the definition of replication is crucial to obtaining
  the expected identity $!!P \sim !P$.
\end{remark}

\begin{remark}\label{rem:paradoxical_combinator}
  The reader familiar with the lambda calculus will have noticed the
  similarity between $D$ and the paradoxical combinator.

  [Ed. note: the existence of this seems to suggest we have to be more
  restrictive on the set of processes and names we admit if we are to
  support no-cloning.]
\end{remark}

\subsubsection{Bisimulation}

The computational dynamics gives rise to another kind of equivalence,
the equivalence of computational behavior. As previously mentioned
this is typically captured \emph{via} some form of bisimulation.

% The notion we use in this paper is weak barbed bisimulation
% \cite{milner91polyadicpi}.

The notion we use in this paper is derived from weak barbed
bisimulation \cite{milner91polyadicpi}. 

\begin{definition}
An \emph{observation relation}, $\downarrow_{\mathcal N}$, over a set
of names, $\mathcal N$, is the smallest relation satisfying the rules
below.

\infrule[Out-barb]{y \in {\mathcal N}, \; x \nameeq y}
		  {\outputp{x}{v} \downarrow_{\mathcal N} x}
\infrule[Par-barb]{\mbox{$P\downarrow_{\mathcal N} x$ or $Q\downarrow_{\mathcal N} x$}}
		  {\binpar{P}{Q} \downarrow_{\mathcal N} x}

We write $P \Downarrow_{\mathcal N} x$ if there is $Q$ such that 
$P \wred Q$ and $Q \downarrow_{\mathcal N} x$.
\end{definition}

\begin{definition}
%\label{def.bbisim}
An  ${\mathcal N}$-\emph{barbed bisimulation} over a set of names, ${\mathcal N}$, is a symmetric binary relation 
${\mathcal S}_{\mathcal N}$ between agents such that $P\rel{S}_{\mathcal N}Q$ implies:
\begin{enumerate}
\item If $P \red P'$ then $Q \wred Q'$ and $P'\rel{S}_{\mathcal N} Q'$.
\item If $P\downarrow_{\mathcal N} x$, then $Q\Downarrow_{\mathcal N} x$.
\end{enumerate}
$P$ is ${\mathcal N}$-barbed bisimilar to $Q$, written
$P \wbbisim_{\mathcal N} Q$, if $P \rel{S}_{\mathcal N} Q$ for some ${\mathcal N}$-barbed bisimulation ${\mathcal S}_{\mathcal N}$.
\end{definition}

$\mathcal{R} \subseteq \pi \times \pi$

$P \mathcal{R} Q => \forall P'. P \red P' \Rightarrow \exists Q'. Q \red Q', P' \mathcal{R} Q'$

$P \vdash x \Rightarrow Q \vdash x$

\begin{mathpar}
  \inferrule*[lab=Out-barb]{x \nameeq y}{{y}!\langle{Q}\rangle \vdash x}
  \and
  \inferrule*[lab=Par-barb]{\mbox{$P\vdash x$ or $Q\vdash x$}}{\binpar{P}{Q} \vdash x}
\end{mathpar}

\subsubsection{Contexts}

One of the principle advantages of computational calculi like the
$\pi$-calculus is a well-defined notion of context,
contextual-equivalence and a correlation between
contextual-equivalence and notions of bisimulation. The notion of
context allows the decomposition of a process into (sub-)process and
its syntactic environment, its context. Thus, a context may be
thought of as a process with a ``hole'' (written $\Box$) in it. The
application of a context $M$ to a process $P$, written $M[P]$, is
tantamount to filling the hole in $M$ with $P$. In this paper we do
not need the full weight of this theory, but do make use of the notion
of context in the proof the main theorem. 

\begin{mathpar}
  \inferrule* [lab=summation] {} {{M_{M},M_{N}} \bc \Box \;|\; x.M_{A} \;|\; M_{M}+M_{N}}
  \and
  \inferrule* [lab=agent] {} {{M_{A}} \bc (\vec{x})M_{P} \;| \; \clift{P_0,\ldots,M_{P},\ldots,P_N}}
  \and \\
  \inferrule* [lab=process] {} {{M_{P}} \bc M_{N} \;| \;P|M_{P} }
\end{mathpar} 

\begin{mathpar}
  \inferrule* [lab=sychronization] {} {M_{N} \bc \Box \;|\; x?M_{F} \;|\; x!M_{C}}
  \and
  \inferrule* [lab=abstraction] {} {{M_{F}} \bc (x)M_{P} }
  \and
  \inferrule* [lab=concretion] {} {{M_{C}} \bc \langle M_{P} \rangle }
  \and \\
  \inferrule* [lab=process] {} {{M_{P}} \bc M_{N} \;| \;P|M_{P} }
\end{mathpar}

\begin{definition}[contextual application] Given a context $M$, and
  process $P$, we define the \emph{contextual application}, $M[P] :=
  M\{P/\Box\}$. That is, the contextual application of M to P is the
  substitution of $P$ for $\Box$ in $M$.
\end{definition}

$\meaningof{-} : L \to \mathcal{P}(\pi)$

\begin{mathpar}
  \inferrule* [lab=collection] {} {\meaningof{true} = \pi, \and \meaningof{~E} = \pi \setminus \meaningof{E}, \and \meaningof{E_{1} \& E_{2}} = \meaningof{E_{1}} \cap \meaningof{E_{2}}}
\end{mathpar}

\begin{mathpar}
  \inferrule* [lab=structure] {} {\meaningof{0} = \{ P \in \pi | P \equiv 0 \}, \and \\ \meaningof{E_1 | E_2} = \{ P \in \pi | P \equiv P_{1} | P_{2}, P_{1} \in \meaningof{E_{1}}, P_{2} \in \meaningof{E_2}\} }
\end{mathpar}

\begin{mathpar}
 \inferrule* [lab=behavior] {} {\meaningof{\langle a?b \rangle E} = \{ P \in \pi | P \equiv Q | u?(y)P', \\ \and \\\\ \and \\ \;\;\; u \in \meaningof{a}, \forall z.P'\{z/y\} \in \meaningof{E\{z/b\}}\}, \and \\ \meaningof{a!E} = \{ P \in \pi | P \equiv Q | x!\langle P' \rangle, x \in \meaningof{a} P' \in \meaningof{E}\} }
\end{mathpar}

\begin{mathpar}
 \inferrule* [lab=nominal] {} {\meaningof{\quotep{E}} = \{ \quotep{P} \in \quotep{\pi} | P \in \meaningof{E} \}, \and \meaningof{\quotep{P}} = \{ \quotep{Q} \in \quotep{\pi} | P \equiv Q \} \and \\ \meaningof{@\quotep{E}} = \{ P \in \pi | P \equiv @x, x \in \meaningof{E} \}}
\end{mathpar}

\begin{eqnarray*}
  \\
  \meaningof{-} : TS \to ST
\end{eqnarray*}

\begin{eqnarray*}
  \\
  L : TS \to ST
\end{eqnarray*}

\begin{eqnarray*}
  \\
  P \models E \iff P \in \meaningof{E}
\end{eqnarray*}

\begin{eqnarray*}
  P \approx_{L} Q \iff \forall E \in L. P \models E \iff Q \models E
\end{eqnarray*}

\begin{eqnarray*}
  P \approx_{K} Q
\end{eqnarray*}

\begin{eqnarray*}
  P \approx Q
\end{eqnarray*}

$\approx_{K} = \approx = \approx_{L}$

\subsubsection{Contextual duality}

Note that contexts extend the quotation operation to a family of
operations from processes to names. Given a context, $M$, we can
define a \emph{nominal context}, $\quotep{M}$ by $\quotep{M}[P] :=
\quotep{M[P]}$. To foreshadow what is to come we observe that these
operations enjoy a duality with processes very much like the duality
between vectors and maps from vectors to scalars.

Further, because the calculus is essentially higher-order, we have a
correspondence between contexts and processes. More specifically,
given a name $x$ and a context $M$ we can construct $M^{*}_{x}$ such
that 

\begin{mathpar}
  M^{*}_{x} | \lift{x}{P} \red M[P]
\end{mathpar}

namely,

\begin{mathpar}
  M^{*}_{x} := x?(u).M[\dropn{u}]
\end{mathpar}

The dependence of $M^{*}_{x}$ on a name makes it an abstraction, 

\begin{mathpar}
  M^{*} := (x)x?(u).M[\dropn{u}]
\end{mathpar}

\subsection{Additional notation}

It will sometimes be convenient to denote the process a name
quotes. We already have the notation $x = \quotep{P}$, but it will be
convenient to introduce an alternate notation, $\procn{x}$, when we
want to emphasize the connection to the use of the name. Note that, by
virtue of name equivalence, $\quotep{\procn{x}} \nameeq x$; so, the
notation is consistent with previous definitions.

Further, because names have structure it is possible to effect
substitutions on the basis of that structure. This means we need to
upgrade our notation for substitutions, which we accomplish by
adapting comprehension notation. Thus,

\begin{mathpar}
  P\{ y / x : x \in S \}
\end{mathpar}

is interpreted to mean the process derived from P by replacing (in a
capture-avoiding manner) each occurrence of $x$ in $S$ by $y$. For example,

\begin{mathpar}
  P\{ \quotep{\procn{x}|\procn{x}} / x : x \in \freenames{P} \}
\end{mathpar}

will replace each (occurrence) of a free name $x$ in $P$ by
$\quotep{\procn{x}|\procn{x}}$.

Also, we will avail ourselves of the notation $x^{L}$ and $x^{R}$ to
denote injections of a name into disjoint copies of the name
space. There are numerous ways to accomplish this. One example can be
found in \cite{MeredithR05}. This notation overloads to vectors of
names: $\vec{x}^{\pi} := (x_{i}^{\pi} \; : \; 0 \leq i < |\vec{x}| )$ where $\pi \in \{L,R\}$.

We also use $P^{\Box} := P|\Box$.

In \cite{MeredithR05} an interpretation of the new operator is
given. It turns out that there are several possible interpretations
all enjoying the requisite algebraic properties of the operator (see
\cite{milner91polyadicpi}). We will therefore make liberal use of
$(\nu\; \vec{x})P$.

% subsection the_syntax_and_semantics_of_the_notation_system (end)   

\input{qm2pi.qmops} 

\input{qm2pi.sterngerlach} 

\input{qm2pi.metric} 

% section concurrent_process_calculi (end)

%\input{qm2pi.proofsketch}

% section proof sketch (end)

%\input{qm2pi.slviaknots} 

% section spatial logic via knots (end)

\input{qm2pi.conclusion}

% section conclusion (end)

%\input{qm2pi.dtcodes} 

% section wiring algorithm (end)

\input{qm2pi.ack} 

% section acknowledgments (end)

\newpage


\bibliographystyle{plain}   
\bibliography{../../biblios/main.bib}

\input{qm2pi.rhodetails}

\end{document}

 

% section concurrent_process_calculi (end)

%\documentclass[12pt]{llncs}
%\documentclass{jktr}

\usepackage[pdftex]{hyperref}                   
\usepackage {listings}
\usepackage {mathpartir}
\usepackage{bcprules}
%\usepackage{listings}
                       
\usepackage{graphicx} 
%\usepackage[margins=2.5cm,nohead,nofoot]{geometry}
%\usepackage{geometry}
\usepackage{amsfonts}
\usepackage{amstext}
\usepackage{latexsym}
\usepackage{amssymb}
\usepackage{color}


%\include{myPreamble}
\include{qm2pi.local} 

%\ifpdf
%\usepackage[pdftex]{graphicx}
%\else
%\usepackage{graphicx}
%\fi

 % \ifpdf
%  \usepackage{pdfsync}
%  \if


%\title{Brief Article}
%\author{David F. Snyder}
%\author{L.G. Meredith}

%\address{Dept. of Math., Texas State University--San Marcos, San Marcos, TX 78666}
       
\pagestyle{empty}


\begin{document}

\lstset{language=[Objective]Caml,frame=shadowbox}

\input{qm2pi.front}

% section front matter (end)

\input{qm2pi.intro} 
 
% section introduction (end)

% \input{qm2pi.knotations} 

% section notation (end)

\input{qm2pi.process.calculi} 

% section concurrent_process_calculi_and_spatial_logics_ (end)
    
%\input{qm2pi.knots2pi} 

%\input{qm2pi.trefoil} 

%\input{qm2pi.mainthm} 

% subsection basic_interpretation (end)

%\input{qm2pi.rho.presentation} 
\subsection{The syntax and semantics of the notation system}\label{sub:the_syntax_and_semantics_of_the_notation_system} % (fold)

We now summarize a technical presentation of the calculus that
embodies our theory of dynamics. The typical presentation of such a
calculus follows the style of giving generators and relations on
them. The grammar, below, describing term constructors, freely
generates the set of processes, $\Proc$. This set is then quotiented
by a relation known as structural congruence and it is over this set
that the notion of dynamics is expressed. This presentation is
essentially that of \cite{MeredithR05} with the addition of
polyadicity and summation. For readability we have relegated some of
the technical subtleties to an appendix.

\subsubsection{Process grammar}\label{subsub:process_grammar}

\begin{mathpar}
  \inferrule* [lab=synchronization] {} {{M} \bc \pzero \;|\; x?F \;|\; x!C }
  \and
  \inferrule* [lab=abstraction] {} {{F} \bc (x)P}
  \and
  \inferrule* [lab=concretion] {} {{C} \bc \langle Q \rangle}
  \and
  \inferrule* [lab=process] {} {{P,Q} \bc M \;| \;P|Q \;|\; @{x}}
  \and
  \inferrule* [lab=name] {} {{x} \bc \quotep{P}}
\end{mathpar} 

Note that $\vec{x}$ (resp. $\vec{P}$) denotes a vector of names
(resp. processes) of length $|\vec{x}|$ (resp. $|\vec{P}|$). We adopt
the following useful abbreviations.

\begin{mathpar}
   x?(\vec{y}).P := x.(\vec{y})P \and  x\clift{\vec{P}} := x.\clift{\vec{P}}
   \and x!(y) := \lift{x}{\dropn{y}}
   \and \Pi_{i=0}^{n-1}P_i := P_0 | \ldots | P_{n-1}
\end{mathpar}

\subsubsection{Structural congruence}

\paragraph{Free and bound names and alpha-equivalence.} At the
core of structural equivalence is alpha-equivalence which identifies
process that are the same up to a change of variable. Formally, we
recognize the distinction between free and bound names. The free names
of a process, $\freenames{P}$, may be calculated recursively as
follows:

\begin{mathpar}
\freenames{\pzero} := \emptyset
  \and \\
  \freenames{x?(y).P} := \{ x \} \cup (\freenames{P} \setminus \{ y \})
  \and 
  \freenames{x!\langle P \rangle} := \{ x \} \cup \{ P \} 
  \and \\
  \freenames{P|Q} := \freenames{P} \cup \freenames{Q}
  \and \\
  \freenames{@{x}} := \{ x \}
\end{mathpar}

$\pi$
$\quotep{\pi}$

$\freenames{-} : \pi \to \mathcal{P}(\quotep{\pi})$

\begin{eqnarray*}
  \freenames{\pzero} & := & \emptyset \\
  \freenames{x?(y).P} & := & \{ x \} \cup (\freenames{P} \setminus \{ y \}) \\
  \freenames{x!\langle P \rangle} & := & \{ x \} \cup \{ P \} \\
  \freenames{P|Q} & := & \freenames{P} \cup \freenames{Q} \\
  \freenames{\dropn{x}} & := & \{ x \}
\end{eqnarray*}

The bound names of a process, $\boundnames{P}$, are those names occurring in $P$
that are not free. For example, in $x?(y).0$, the name $x$ is free, while $y$ is bound.

\begin{mathpar}
  \inferrule* [lab=monoidal-laws] {} { P|Q \equiv Q|P \and P|0 \equiv P \and P|(Q|R) \equiv (P|Q)|R }
\end{mathpar}

\begin{mathpar}
  \inferrule* [lab=alpha-equivalence] {} { (x)P \equiv (y)P\{y/x\} \and y \not\in \freenames{P} }
\end{mathpar}

\begin{definition}
Then two processes, $P,Q$, are alpha-equivalent if $P = Q\{\vec{y}/\vec{x}\}$ for
some $\vec{x} \in \boundnames{Q},\vec{y} \in \boundnames{P}$, where $Q\{\vec{y}/\vec{x}\}$
denotes the capture-avoiding substitution of $\vec{y}$ for $\vec{x}$ in $Q$.
\end{definition}

\begin{definition}
  The {\em structural congruence} \cite{SangiorgiWalker} , $\equiv$,
  between processes is the least congruence containing
  alpha-equivalence, satisfying the abelian monoid laws
  (associativity, commutativity and $\pzero$ as identity) for parallel
  composition $|$ and for summation $+$.
\end{definition}

\subsection{Name equivalence}

We take name equivalence, written $\nameeq$, to be the smallest
equivalence relation generated by the following rules.

\begin{mathpar}
\inferrule*[lab=Quote-drop]
{ }
{ \quotep{@{x}} \nameeq x }

\inferrule*[lab=Struct-equiv]
{ P \scong Q }
{ \quotep{P} \nameeq \quotep{Q} }
\end{mathpar}

The astute reader will have noticed that the mutual recursion of names
and processes imposes a mutual recursion on alpha-equivalence and
structural equivalence via name-equivalence. Fortunately, all of this
works out pleasantly and we may calculate in the natural way, free of
concern. The reader interested in the details is referred to the
appendix \ref{appendix:rho_details}.

\subsection{Substitution}

We use $\Proc$ for the set of processes, $\QProc$ for the set of
names, and $\id{\{}\vec{y} / \vec{x} \id{\}}$ to denote partial maps,
$s : \QProc \rightarrow \QProc$. A map, $s$ lifts, uniquely, to a map
on process terms, $\widehat{s} : \Proc \rightarrow \Proc$ by the
following equations.

\begin{mathpar}
  (0) \psubstp{Q}{P} := 0 \\
  (R \juxtap S) \psubstp{Q}{P}
  :=    
  (R)\psubstp{Q}{P} \juxtap (S) \psubstp{Q}{P} \\
  (x?(y).R) \psubstp{Q}{P}    
  :=    
  (x)\substp{Q}{P} (z)\concat( (R \psubstn{z}{y}) \psubstp{Q}{P} ) \\
  (\lift{x}{R}) \psubstp{Q}{P}  
  :=
  \lift{(x)\substp{Q}{P}}{ R \psubstp{Q}{P} } \\
%   (\dropn{x})  \psubstp{Q}{P}       
%   := 
%   \left\{ 
%     \begin{array}{ccc} 
%       \dropn{\quotep{Q}} & & x \nameeq \quotep{P} \\
%       \dropn{x} & & otherwise \\
%     \end{array}
%   \right. 
  (\dropn{x})  \psubstp{Q}{P}       
  := 
  \left\{ 
    \begin{array}{ccc} 
      Q & & x \nameeq \quotep{P} \\
      \dropn{x} & & otherwise \\
    \end{array}
  \right.
\end{mathpar}
 

where

\begin{eqnarray}
  (x)\id{\{} \lpquote Q \rpquote / \lpquote P \rpquote \id{\}}            = 
  \left\{ 
    \begin{array}{ccc}
      \lpquote Q \rpquote & & x \nameeq \lpquote P \rpquote \\
      x & & otherwise \\
    \end{array}
  \right. \nonumber
\end{eqnarray}

and $z$ is chosen distinct from $\quotep{P}$, $\quotep{Q}$, the free
names in $Q$, and all the names in $R$. Our $\alpha$-equivalence will
be built in the standard way from this substitution.

\begin{remark}\label{rem:no_self_referential_names}
  One consequence of these definitions is that $\forall P. \quotep{P}
  \not\in \freenames{P}$.
\end{remark}

\subsection{ Dynamic quote: an example }

Anticipating something of what's to come, consider applying the
substitution, $\widehat{\id{\{}u / z \id{\}}}$, to the following pair
of processes, $\lift{w}{y!(z)}$ and $w[ \lpquote y!(z) \rpquote ]$.

\begin{eqnarray}
	\lift{w}{y!(z)}\widehat{\id{\{}u / z \id{\}}}
		& = &
		\lift{w}{y!(u)} \nonumber\\
	w[ \lpquote y!(z) \rpquote ] \widehat{ \id{\{}u / z \id{\}} }
		& = &
		w[ \lpquote y!(z) \rpquote ] \nonumber
\end{eqnarray}

Because the body of the process between quotes is impervious to
substitution, we get radically different answers. In fact, by
examining the first process in an input context,
e.g. $x?(z).\lift{w}{y!(z)}$, we see that the process under the lift
operator may be shaped by prefixed inputs binding a name inside it. In
this sense, the lift operator will be seen as a way to dynamically
construct processes before reifying them as names.

Finally equipped with these standard features we can present the
dynamics of the calculus.

\subsubsection{Operational semantics} 

Finally, we introduce the computational dynamics. What marks these
algebras as distinct from other more traditionally studied algebraic
structures, e.g. vector spaces or polynomial rings, is the manner in
which dynamics is captured. In traditional structures, dynamics is typically
expressed through morphisms between such structures, as in linear maps
between vector spaces or morphisms between rings. In algebras
associated with the semantics of computation, the dynamics is
expressed as part of the algebraic structure itself, through a
reduction reduction relation typically denoted by $\red$. Below, we
give a recursive presentation of this relation for the calculus used
in the encoding.

$\red \subseteq \pi \times \pi$
$\red : \pi \to \mathcal{P}(\pi)$

\begin{mathpar}
  \inferrule* [lab=Comm] { \textsf{match}( x_{src}, x_{trgt} ) } { x_{trgt}?(y)P \; | \; x_{src}!\langle {Q} \rangle \red P\{\quotep{Q}/y}\} }
  \and \\
  \inferrule* [lab=Par] {{P} \red {P}'} {{{P} | {Q}} \red {{P}' | {Q}}}
  \and
  \inferrule* [lab=Equiv]{{{P} \scong {P}'} \andalso {{P}' \red {Q}'} \andalso {{Q}' \scong {Q}}}{{P} \red {Q}}
\end{mathpar}

\begin{eqnarray*}
  match_{\equiv} (\quotep{P},\quotep{Q}) & := & P \equiv Q \\
  match_{\dagger}(\quotep{P},\quotep{Q}) & := & \forall R. P|Q \red^{*} R => R \red^{*} 0 \\
  match_{K}(\quotep{P},\quotep{Q}) & := & K \mbox{ for some context } K
\end{eqnarray*}

$u?(x)P | u!\langle Q \rangle \red P\{\quotep{Q}/x\}$

%We write $\wred$ for $\red^*$, and $P\red$ if $\exists Q $ such that $ P \red Q$.
We write $P\red$ if $\exists Q $ such that $ P \red Q$ and $P\not\red$, otherwise.

\section{Replication}

As mentioned before, it is known that replication (and hence
recursion) can be implemented in a higher-order process algebra
\cite{SangiorgiWalker}. As our first example of calculation with the
machinery thus far presented we give the construction explicitly in
the {\rhoc}.

\begin{eqnarray}
	D_{x} & := & \prefix{x}{y}{(\binpar{\outputp{x}{y}}{@{y}})} \nonumber\\
	\bangp_{x}{P} & := & \binpar{{x}!\langle{\binpar{D_{x}}{P}}\rangle}{D_{x}} \nonumber
\end{eqnarray}

\begin{eqnarray}
	\bangp_{x}{P} & & \nonumber\\
	=
	& {x}!\langle{(\prefix{x}{y}{(\outputp{x}{y} | @{y})) | P}}\rangle 
	      | \prefix{x}{y}{(\outputp{x}{y} | @{y})} & \nonumber\\
	\red
	& (\outputp{x}{y} | @{y})\substn{\quotep{(\prefix{x}{y}{(@{y} | \outputp{x}{y})) | P}}}{y} & \nonumber\\
	=
	& \outputp{x}{\quotep{(\prefix{x}{y}{(\outputp{x}{y} | @{y})) | P}}}
	  | {(\prefix{x}{y}{(\outputp{x}{y} | @{y})) | P}} & \nonumber\\
	\red
	& \ldots & \nonumber\\
	\red^*
	& P | P | \ldots & \nonumber
\end{eqnarray}

Of course, this encoding, as an implementation, runs away, unfolding
$\bangp{P}$ eagerly. A lazier and more implementable replication
operator, restricted to input-guarded processes, may be obtained as follows.

\begin{eqnarray}
\bangp{\prefix{u}{v}{P}} 
	:= 
	\binpar{\lift{x}{\prefix{u}{v}{(\binpar{D(x)}{P})}}}{D(x)} \nonumber
\end{eqnarray}

\begin{remark}
  Note that the lazier definition still does not deal with summation
  or mixed summation (i.e. sums over input and output). The reader is
  invited to construct definitions of replication that deal with these
  features. 

  Further, the definitions are parameterized in a name, $x$. Can you,
  gentle reader, make a definition that eliminates this parameter and
  guarantees no accidental interaction between the replication
  machinery and the process being replicated -- i.e. no accidental
  sharing of names used by the process to get its work done and the
  name(s) used by the replication to effect copying. This latter
  revision of the definition of replication is crucial to obtaining
  the expected identity $!!P \sim !P$.
\end{remark}

\begin{remark}\label{rem:paradoxical_combinator}
  The reader familiar with the lambda calculus will have noticed the
  similarity between $D$ and the paradoxical combinator.

  [Ed. note: the existence of this seems to suggest we have to be more
  restrictive on the set of processes and names we admit if we are to
  support no-cloning.]
\end{remark}

\subsubsection{Bisimulation}

The computational dynamics gives rise to another kind of equivalence,
the equivalence of computational behavior. As previously mentioned
this is typically captured \emph{via} some form of bisimulation.

% The notion we use in this paper is weak barbed bisimulation
% \cite{milner91polyadicpi}.

The notion we use in this paper is derived from weak barbed
bisimulation \cite{milner91polyadicpi}. 

\begin{definition}
An \emph{observation relation}, $\downarrow_{\mathcal N}$, over a set
of names, $\mathcal N$, is the smallest relation satisfying the rules
below.

\infrule[Out-barb]{y \in {\mathcal N}, \; x \nameeq y}
		  {\outputp{x}{v} \downarrow_{\mathcal N} x}
\infrule[Par-barb]{\mbox{$P\downarrow_{\mathcal N} x$ or $Q\downarrow_{\mathcal N} x$}}
		  {\binpar{P}{Q} \downarrow_{\mathcal N} x}

We write $P \Downarrow_{\mathcal N} x$ if there is $Q$ such that 
$P \wred Q$ and $Q \downarrow_{\mathcal N} x$.
\end{definition}

\begin{definition}
%\label{def.bbisim}
An  ${\mathcal N}$-\emph{barbed bisimulation} over a set of names, ${\mathcal N}$, is a symmetric binary relation 
${\mathcal S}_{\mathcal N}$ between agents such that $P\rel{S}_{\mathcal N}Q$ implies:
\begin{enumerate}
\item If $P \red P'$ then $Q \wred Q'$ and $P'\rel{S}_{\mathcal N} Q'$.
\item If $P\downarrow_{\mathcal N} x$, then $Q\Downarrow_{\mathcal N} x$.
\end{enumerate}
$P$ is ${\mathcal N}$-barbed bisimilar to $Q$, written
$P \wbbisim_{\mathcal N} Q$, if $P \rel{S}_{\mathcal N} Q$ for some ${\mathcal N}$-barbed bisimulation ${\mathcal S}_{\mathcal N}$.
\end{definition}

$\mathcal{R} \subseteq \pi \times \pi$

$P \mathcal{R} Q => \forall P'. P \red P' \Rightarrow \exists Q'. Q \red Q', P' \mathcal{R} Q'$

$P \vdash x \Rightarrow Q \vdash x$

\begin{mathpar}
  \inferrule*[lab=Out-barb]{x \nameeq y}{{y}!\langle{Q}\rangle \vdash x}
  \and
  \inferrule*[lab=Par-barb]{\mbox{$P\vdash x$ or $Q\vdash x$}}{\binpar{P}{Q} \vdash x}
\end{mathpar}

\subsubsection{Contexts}

One of the principle advantages of computational calculi like the
$\pi$-calculus is a well-defined notion of context,
contextual-equivalence and a correlation between
contextual-equivalence and notions of bisimulation. The notion of
context allows the decomposition of a process into (sub-)process and
its syntactic environment, its context. Thus, a context may be
thought of as a process with a ``hole'' (written $\Box$) in it. The
application of a context $M$ to a process $P$, written $M[P]$, is
tantamount to filling the hole in $M$ with $P$. In this paper we do
not need the full weight of this theory, but do make use of the notion
of context in the proof the main theorem. 

\begin{mathpar}
  \inferrule* [lab=summation] {} {{M_{M},M_{N}} \bc \Box \;|\; x.M_{A} \;|\; M_{M}+M_{N}}
  \and
  \inferrule* [lab=agent] {} {{M_{A}} \bc (\vec{x})M_{P} \;| \; \clift{P_0,\ldots,M_{P},\ldots,P_N}}
  \and \\
  \inferrule* [lab=process] {} {{M_{P}} \bc M_{N} \;| \;P|M_{P} }
\end{mathpar} 

\begin{mathpar}
  \inferrule* [lab=sychronization] {} {M_{N} \bc \Box \;|\; x?M_{F} \;|\; x!M_{C}}
  \and
  \inferrule* [lab=abstraction] {} {{M_{F}} \bc (x)M_{P} }
  \and
  \inferrule* [lab=concretion] {} {{M_{C}} \bc \langle M_{P} \rangle }
  \and \\
  \inferrule* [lab=process] {} {{M_{P}} \bc M_{N} \;| \;P|M_{P} }
\end{mathpar}

\begin{definition}[contextual application] Given a context $M$, and
  process $P$, we define the \emph{contextual application}, $M[P] :=
  M\{P/\Box\}$. That is, the contextual application of M to P is the
  substitution of $P$ for $\Box$ in $M$.
\end{definition}

$\meaningof{-} : L \to \mathcal{P}(\pi)$

\begin{mathpar}
  \inferrule* [lab=collection] {} {\meaningof{true} = \pi, \and \meaningof{~E} = \pi \setminus \meaningof{E}, \and \meaningof{E_{1} \& E_{2}} = \meaningof{E_{1}} \cap \meaningof{E_{2}}}
\end{mathpar}

\begin{mathpar}
  \inferrule* [lab=structure] {} {\meaningof{0} = \{ P \in \pi | P \equiv 0 \}, \and \\ \meaningof{E_1 | E_2} = \{ P \in \pi | P \equiv P_{1} | P_{2}, P_{1} \in \meaningof{E_{1}}, P_{2} \in \meaningof{E_2}\} }
\end{mathpar}

\begin{mathpar}
 \inferrule* [lab=behavior] {} {\meaningof{\langle a?b \rangle E} = \{ P \in \pi | P \equiv Q | u?(y)P', \\ \and \\\\ \and \\ \;\;\; u \in \meaningof{a}, \forall z.P'\{z/y\} \in \meaningof{E\{z/b\}}\}, \and \\ \meaningof{a!E} = \{ P \in \pi | P \equiv Q | x!\langle P' \rangle, x \in \meaningof{a} P' \in \meaningof{E}\} }
\end{mathpar}

\begin{mathpar}
 \inferrule* [lab=nominal] {} {\meaningof{\quotep{E}} = \{ \quotep{P} \in \quotep{\pi} | P \in \meaningof{E} \}, \and \meaningof{\quotep{P}} = \{ \quotep{Q} \in \quotep{\pi} | P \equiv Q \} \and \\ \meaningof{@\quotep{E}} = \{ P \in \pi | P \equiv @x, x \in \meaningof{E} \}}
\end{mathpar}

\begin{eqnarray*}
  \\
  \meaningof{-} : TS \to ST
\end{eqnarray*}

\begin{eqnarray*}
  \\
  L : TS \to ST
\end{eqnarray*}

\begin{eqnarray*}
  \\
  P \models E \iff P \in \meaningof{E}
\end{eqnarray*}

\begin{eqnarray*}
  P \approx_{L} Q \iff \forall E \in L. P \models E \iff Q \models E
\end{eqnarray*}

\begin{eqnarray*}
  P \approx_{K} Q
\end{eqnarray*}

\begin{eqnarray*}
  P \approx Q
\end{eqnarray*}

$\approx_{K} = \approx = \approx_{L}$

\subsubsection{Contextual duality}

Note that contexts extend the quotation operation to a family of
operations from processes to names. Given a context, $M$, we can
define a \emph{nominal context}, $\quotep{M}$ by $\quotep{M}[P] :=
\quotep{M[P]}$. To foreshadow what is to come we observe that these
operations enjoy a duality with processes very much like the duality
between vectors and maps from vectors to scalars.

Further, because the calculus is essentially higher-order, we have a
correspondence between contexts and processes. More specifically,
given a name $x$ and a context $M$ we can construct $M^{*}_{x}$ such
that 

\begin{mathpar}
  M^{*}_{x} | \lift{x}{P} \red M[P]
\end{mathpar}

namely,

\begin{mathpar}
  M^{*}_{x} := x?(u).M[\dropn{u}]
\end{mathpar}

The dependence of $M^{*}_{x}$ on a name makes it an abstraction, 

\begin{mathpar}
  M^{*} := (x)x?(u).M[\dropn{u}]
\end{mathpar}

\subsection{Additional notation}

It will sometimes be convenient to denote the process a name
quotes. We already have the notation $x = \quotep{P}$, but it will be
convenient to introduce an alternate notation, $\procn{x}$, when we
want to emphasize the connection to the use of the name. Note that, by
virtue of name equivalence, $\quotep{\procn{x}} \nameeq x$; so, the
notation is consistent with previous definitions.

Further, because names have structure it is possible to effect
substitutions on the basis of that structure. This means we need to
upgrade our notation for substitutions, which we accomplish by
adapting comprehension notation. Thus,

\begin{mathpar}
  P\{ y / x : x \in S \}
\end{mathpar}

is interpreted to mean the process derived from P by replacing (in a
capture-avoiding manner) each occurrence of $x$ in $S$ by $y$. For example,

\begin{mathpar}
  P\{ \quotep{\procn{x}|\procn{x}} / x : x \in \freenames{P} \}
\end{mathpar}

will replace each (occurrence) of a free name $x$ in $P$ by
$\quotep{\procn{x}|\procn{x}}$.

Also, we will avail ourselves of the notation $x^{L}$ and $x^{R}$ to
denote injections of a name into disjoint copies of the name
space. There are numerous ways to accomplish this. One example can be
found in \cite{MeredithR05}. This notation overloads to vectors of
names: $\vec{x}^{\pi} := (x_{i}^{\pi} \; : \; 0 \leq i < |\vec{x}| )$ where $\pi \in \{L,R\}$.

We also use $P^{\Box} := P|\Box$.

In \cite{MeredithR05} an interpretation of the new operator is
given. It turns out that there are several possible interpretations
all enjoying the requisite algebraic properties of the operator (see
\cite{milner91polyadicpi}). We will therefore make liberal use of
$(\nu\; \vec{x})P$.

% subsection the_syntax_and_semantics_of_the_notation_system (end)   

\input{qm2pi.qmops} 

\input{qm2pi.sterngerlach} 

\input{qm2pi.metric} 

% section concurrent_process_calculi (end)

%\input{qm2pi.proofsketch}

% section proof sketch (end)

%\input{qm2pi.slviaknots} 

% section spatial logic via knots (end)

\input{qm2pi.conclusion}

% section conclusion (end)

%\input{qm2pi.dtcodes} 

% section wiring algorithm (end)

\input{qm2pi.ack} 

% section acknowledgments (end)

\newpage


\bibliographystyle{plain}   
\bibliography{../../biblios/main.bib}

\input{qm2pi.rhodetails}

\end{document}



% section proof sketch (end)

%\section{Unlikely characters: spatial logic for
  knots}\label{sub:characteristic_formulae} % (fold)

Associated to the mobile process calculi are a family of logics known
as the Hennessy-Milner logics. These logics typically enjoy a
semantics interpreting formulae as sets of processes that when
factored through the encoding outlined above allows an identification
of classes of knots with logical formulae. In the context of this
encoding the sub-family known as the spatial logics \cite{CairesC03}
\cite{CairesC04} \cite{Caires04} are of particular interest providing
several important features for expressing and reasoning about
properties (i.e. classes) of knots. We hint here at how this may be done.

%\begin{description}
%\item [structural connectives] 
\subsubsection{Structural connectives} The spatial logics enjoy
structural connectives corresponding, at the logical level, to the
parallel composition ($P | Q$) and new name ($(\nu \; x)P$)
connectives for processes. As illustrated in the examples below, these
connectives are extremely expressive given the shape of our encoding.
%\item [decideable satisfaction]

\subsubsection{Decideable satisfaction}
In \cite{Caires04} the satisfaction relation is shown to be decideable
for a rich class of processes. It further turns out that the image of
the our encoding is a proper subset of that class. This result
provides the basis for an algorithm by which to search for knots
enjoying a given property.
%\item [characteristic formulae]

\subsubsection{Characteristic formulae}
In the same paper \cite{Caires04} , Caires presents a means of calculating
characteristic formulae, selecting equivalence classes of processes
up to a pre--specified depth limit on the support set of names. Composed with our
encoding, this characteristic formula can be used to select
characteristic formulae for knots.
%\end{description}

\subsubsection{Spatial logic formulae}

The grammar below (segmented for comprehension) summarizes the syntax
of spatial logic formulae. We employ illustrative examples in the
sequel to provide an intuitive understanding of their meaning
referring the reader to \cite{Caires04} for a more detailed explication
of the semantics.

\begin{mathpar}
  \inferrule* [lab=boolean] {} {{A,B} \bc T \;|\; \neg A \;|\; A \wedge B \;|\; \eta = \eta'}
  \and
  \inferrule* [lab=spatial] {} {|\; \pzero \;|\; A | B \;|\; x \text{\textregistered} A \;|\; \forall x . A \;|\;  H x . A}
  \and
  \inferrule* [lab=behavioral] {} {|\; \alpha . A}
  \and 
  \inferrule* [lab=recursion] {} {|\; X(\vec{u}) \;|\; \mu X(\vec{u}) . A}
  \and
  \inferrule* [lab=action] {} {\alpha \bc \langle x?(\vec{y}) \rangle \;|\; \langle x!(\vec{y}) \rangle \;|\; \langle \tau \rangle}
  \and 
  \inferrule* [lab=name] {} {\eta \bc x \;|\; \tau}
\end{mathpar} 

% subsection characteristic_formulae (end)   	 

\subsection{Example formulae}\label{sub:example_formulae_} % (fold)

\subsubsection{Crossing as formula.}
% 
% \begin{align*}
%   \frac{d}{dx} \sin x &= \cos x 
%   & \frac{d}{dx} e^x &= e^x \\
%   \frac{d}{dx} \cos x &= - \sin x 
%   & \frac{d}{dx} \log x &= \frac{1}{x} \\
% \end{align*} 

\begin{align*}
 \mu C(x_{0},x_{1},y_{0},y_{1},u).&(\langle x_{0}?(z) \rangle(\langle u! \rangle\langle y_{1}!z \rangle C(x_{0},x_{1},y_{0},y_{1},u)) & \\
  & \wedge \langle y_{1}?(z) \rangle (\langle u! \rangle \langle x_{0}!z \rangle C(x_{0},x_{1},y_{0},y_{1},u)) & \\
  & \wedge \langle x_{1}?(z) \rangle (\langle u? \rangle \langle y_{0}!z \rangle C(x_{0},x_{1},y_{0},y_{1},u)) & \\
  & \wedge \langle y_{0}?(z) \rangle (\langle u? \rangle \langle x_{1}!z \rangle C(x_{0},x_{1},y_{0},y_{1},u))) &
\end{align*}

The lexicographical similarity between the shape of this formulae and
the shape of definition of the process representing a crossing reveals
the intuitive meaning of this formulae. It describes the capabilities
of a process that has the right to represent a crossing. For example
it picks out processes that may perform an input on the port $x_0$ in
its initial menu of capabilities. What differentiates the formula
from the process, however, is that the crossing process is the
smallest candidate to satisfy the formula. Infinitely many other
processes -- with internal behavior hidden behind this interface, so
to speak -- also satisfy this formula. Even this simple formula,
then, can be seen to open a new view onto knots, providing a
computational interpretation of \emph{virtual} knots.

Note that this formula is derived by hand. A similar formula can be
derived by employing Caires' calculation of characteristic formula
\cite{Caires04} to the process representing a crossing. In light of
this discussion, we let
$\meaningof{C}_{\phi}(x0,x1,y0,y1,u)$ denote a formula specifying the
dynamics we wish to capture of a crossing. To guarantee we preserve
the shape of the interface and minimal semantics we demand that
$\meaningof{C}_{\phi}(x0,x1,y0,y1,u) \Rightarrow
\textbf{C}(x0,x1,y0,y1,u)$ where $\textbf{C}(x0,x1,y0,y1,u)$ denotes
the formula above.
                            
\subsubsection{Crossing number constraints.}
The moral content of the context lemma (Lemma \ref{context}) is that the notion of
``locality'' in the Reidemeister moves is effectively captured by the
parallel composition operator of the process calculus. This intuition
extends through the logic. Given a formula,
$\meaningof{C}_{\phi}(x0,x1,y0,y1,u)$, we can use the structural
connectives to specify constraints on crossing numbers, such as at
least $n$ crossings, or exactly $n$ crossings.
\begin{mathpar}
  \inferrule* [lab=at-least-n] {} { K^{\geq n}_{\phi}(\vec{xs},\vec{ys}) := \Pi_{i=0}^{n-1} Hu . \meaningof{C}_{\phi}(xs_i,ys_i,u) | T }
  \and 
  \inferrule* [lab=exactly-n] {} { K^{= n}_{\phi}(\vec{xs},\vec{ys}) := \Pi_{i=0}^{n-1} Hu . \meaningof{C}_{\phi}(xs_i,ys_i,u) | \neg (\forall x_0,y_0,x_1,y_1,u . \meaningof{C}_{\phi}(x_0,y_0,x_1,y_1,u) | T) }
\end{mathpar}

To round out this section, recall that the encoding of an $n$-crossing
knot decomposes into a parallel composition of $n$ \emph{copies} of a
crossing process together with a wiring harness. To specify different
knot classes with the same crossing number amounts to specifying
logical constraints on the wiring harness. In the interest of space,
we defer examples to a forthcoming paper. Suffice it to say that both
the conditions ``alternating knot'' and ``contains the tangle
corresponding to 5/3'' are expressible. For example, it is possible to
calculate the characteristic formula of a process corresponding to the
tangle 5/3 and conjoin it into the classifying formula via the
composition connective of the logic.

Finally, we wish to observe that it is entirely within reason to
contemplate a more domain-specific version of spatial logic tailored
to the shape of processes in the image of the encoding. Such a
domain-specific logic would have a better claim to the title formal
language of knot properties.

% subsection example_formulae_ (end)

% section knots_as_processes (end) 

% section spatial logic via knots (end)

\section{Conclusions and future work}

\paragraph{Testing physical space}
You, gentle reader, may wonder why of all the theorems to be proved
given this set up we pick the one above. In some sense it's hardly
central to quantum mechanics. We see it as central in the sense that
it firmly establishes a notion of physical space arising from a notion
of the equivalence of behavior. Relating bisimulation to a metric is a
big step forward, but one is faced with interpreting the relationship
of that metric space to something more physical. Quantum mechanical
notions of ``physical'' space are still far from intuitive, but by
relating this idea of distance as testing to calculations that predict
physical circumstances we are making a not insignificant step forward
toward an understanding of the physical space we inhabit as
essentially dynamic.

\paragraph{Effectivity and simulation}
One of the observations we have yet to make is that the entire program
spelled out here is effective. We have built various interpreters for
the reflective calculus at work in this interpretation. In principle,
then, we can simulate quantum mechanics on a computer. The place where
the simulation may lose fidelity is the infinitely branching summation
for the annihilator.

In this connection i also want to point out that the evaluation style
calculation of the inner product puts the non-determinism of the
summation right at the heart of measurement. This suggests that
Milner's original reduction-based formulation of the dynamics of his
calculi in terms of sums was not just notationally suggestive of a
notion of measure-and-continue but captured some significant part of
the physics.

\paragraph{Quantum continuations}
In light of this last observation i want to point out that the
predominant account of quantum mechanics is missing a key aspect of a
truly compositional story of the physical situation. In a real lab,
when a measurement is made the observation can be made to feed into
another device that then makes another measurement conditioned on the
results of the first. This means that after the superposition was
collapsed the entire experimental set up remained in
superposition. While QM offers a means of writing this down it doesn't
quite line up well with the well-trodden formulation of computation
and continuation that we see so succinctly expressed in Milner's
calculi. This suggests that there might be advantages to this account
of dynamics waiting to be explored.

\paragraph{Quantum logic}
In this connection, we also note that by virtue of having the
Hennessy-Milner construction, we can pull the construction through the
interpretation of QM. This gives us a natural candidate for a quantum
logic that enjoys an extremely tight connection with it's domain of
interpretation, making the construction much less ad hoc (rather it is
the image of functor!).

\paragraph{Quantum probabiity}
i have questions about the basis of the interpretation of inner
product as probability amplitude. In particular, using which
axiomatization of probability theory does the notion of probability
amplitude earn the right to be so dubbed? In other words, where is the
proof that the operation for calculating a probability amplitude (and
then squaring) satisfies the axioms of what it means to calculate a
probability? Even if such a proof exists (i have yet to find it in the
literature), i wonder if it might not be possible to turn things on
their heads. Can we view the calculation of the probability amplitude
as an axiomatization of probability? If so, then the definition we
give for calculating probability amplitude may provide the basis for
an \emph{effective} theory of probability.

\paragraph{Quantum vs ``biological'' information}
Finally, i want to conclude with a more philosophical observation. At
a recent workshop in which QM was a predominant topic i noticed
something about quantum information. The speaker was giving a riveting
discussion of axiomatic QM and showing how properties of ``no
cloning'' and ``no deleting'' emerged as consequences of the
axiomatization. Theorems of this form are necessary to give us a sense
of confidence that our axioms characterize the physical theory. What
struck me, though, was that if quantum information is neither erasable
nor replicable it is markedly different from \emph{life}. Two of the
things we know about life is that

\begin{itemize}
  \item it ends;
  \item to gain some measure of persistence, to transcend it's
    finitude it is imminently copyable.
\end{itemize}

Both of these qualities are summarized succinctly in the aphorism: all
flesh is grass. For me these two kinds of ``information'' -- call them
quantum and biological -- are end points on a spectrum of strategies
for persistence. At one end, we have those curious entities that enjoy
uniqueness and permanence; at the other, we have those who in the face
of a certain end and an uncertain present make a go of passing
something on. To me one of the more remarkable aspects of the latter
strategy is that in the presence of noise (and certain features of
copying) we get a kind of dynamism, a chance for improvement against a
given persistent condition.

% subsection other_calculi_other_bisimulations_and_geometry_as_behavior (end)




% section conclusion (end)

%\documentclass[12pt]{llncs}
%\documentclass{jktr}

\usepackage[pdftex]{hyperref}                   
\usepackage {listings}
\usepackage {mathpartir}
\usepackage{bcprules}
%\usepackage{listings}
                       
\usepackage{graphicx} 
%\usepackage[margins=2.5cm,nohead,nofoot]{geometry}
%\usepackage{geometry}
\usepackage{amsfonts}
\usepackage{amstext}
\usepackage{latexsym}
\usepackage{amssymb}
\usepackage{color}


%\include{myPreamble}
\include{qm2pi.local} 

%\ifpdf
%\usepackage[pdftex]{graphicx}
%\else
%\usepackage{graphicx}
%\fi

 % \ifpdf
%  \usepackage{pdfsync}
%  \if


%\title{Brief Article}
%\author{David F. Snyder}
%\author{L.G. Meredith}

%\address{Dept. of Math., Texas State University--San Marcos, San Marcos, TX 78666}
       
\pagestyle{empty}


\begin{document}

\lstset{language=[Objective]Caml,frame=shadowbox}

\input{qm2pi.front}

% section front matter (end)

\input{qm2pi.intro} 
 
% section introduction (end)

% \input{qm2pi.knotations} 

% section notation (end)

\input{qm2pi.process.calculi} 

% section concurrent_process_calculi_and_spatial_logics_ (end)
    
%\input{qm2pi.knots2pi} 

%\input{qm2pi.trefoil} 

%\input{qm2pi.mainthm} 

% subsection basic_interpretation (end)

%\input{qm2pi.rho.presentation} 
\subsection{The syntax and semantics of the notation system}\label{sub:the_syntax_and_semantics_of_the_notation_system} % (fold)

We now summarize a technical presentation of the calculus that
embodies our theory of dynamics. The typical presentation of such a
calculus follows the style of giving generators and relations on
them. The grammar, below, describing term constructors, freely
generates the set of processes, $\Proc$. This set is then quotiented
by a relation known as structural congruence and it is over this set
that the notion of dynamics is expressed. This presentation is
essentially that of \cite{MeredithR05} with the addition of
polyadicity and summation. For readability we have relegated some of
the technical subtleties to an appendix.

\subsubsection{Process grammar}\label{subsub:process_grammar}

\begin{mathpar}
  \inferrule* [lab=synchronization] {} {{M} \bc \pzero \;|\; x?F \;|\; x!C }
  \and
  \inferrule* [lab=abstraction] {} {{F} \bc (x)P}
  \and
  \inferrule* [lab=concretion] {} {{C} \bc \langle Q \rangle}
  \and
  \inferrule* [lab=process] {} {{P,Q} \bc M \;| \;P|Q \;|\; @{x}}
  \and
  \inferrule* [lab=name] {} {{x} \bc \quotep{P}}
\end{mathpar} 

Note that $\vec{x}$ (resp. $\vec{P}$) denotes a vector of names
(resp. processes) of length $|\vec{x}|$ (resp. $|\vec{P}|$). We adopt
the following useful abbreviations.

\begin{mathpar}
   x?(\vec{y}).P := x.(\vec{y})P \and  x\clift{\vec{P}} := x.\clift{\vec{P}}
   \and x!(y) := \lift{x}{\dropn{y}}
   \and \Pi_{i=0}^{n-1}P_i := P_0 | \ldots | P_{n-1}
\end{mathpar}

\subsubsection{Structural congruence}

\paragraph{Free and bound names and alpha-equivalence.} At the
core of structural equivalence is alpha-equivalence which identifies
process that are the same up to a change of variable. Formally, we
recognize the distinction between free and bound names. The free names
of a process, $\freenames{P}$, may be calculated recursively as
follows:

\begin{mathpar}
\freenames{\pzero} := \emptyset
  \and \\
  \freenames{x?(y).P} := \{ x \} \cup (\freenames{P} \setminus \{ y \})
  \and 
  \freenames{x!\langle P \rangle} := \{ x \} \cup \{ P \} 
  \and \\
  \freenames{P|Q} := \freenames{P} \cup \freenames{Q}
  \and \\
  \freenames{@{x}} := \{ x \}
\end{mathpar}

$\pi$
$\quotep{\pi}$

$\freenames{-} : \pi \to \mathcal{P}(\quotep{\pi})$

\begin{eqnarray*}
  \freenames{\pzero} & := & \emptyset \\
  \freenames{x?(y).P} & := & \{ x \} \cup (\freenames{P} \setminus \{ y \}) \\
  \freenames{x!\langle P \rangle} & := & \{ x \} \cup \{ P \} \\
  \freenames{P|Q} & := & \freenames{P} \cup \freenames{Q} \\
  \freenames{\dropn{x}} & := & \{ x \}
\end{eqnarray*}

The bound names of a process, $\boundnames{P}$, are those names occurring in $P$
that are not free. For example, in $x?(y).0$, the name $x$ is free, while $y$ is bound.

\begin{mathpar}
  \inferrule* [lab=monoidal-laws] {} { P|Q \equiv Q|P \and P|0 \equiv P \and P|(Q|R) \equiv (P|Q)|R }
\end{mathpar}

\begin{mathpar}
  \inferrule* [lab=alpha-equivalence] {} { (x)P \equiv (y)P\{y/x\} \and y \not\in \freenames{P} }
\end{mathpar}

\begin{definition}
Then two processes, $P,Q$, are alpha-equivalent if $P = Q\{\vec{y}/\vec{x}\}$ for
some $\vec{x} \in \boundnames{Q},\vec{y} \in \boundnames{P}$, where $Q\{\vec{y}/\vec{x}\}$
denotes the capture-avoiding substitution of $\vec{y}$ for $\vec{x}$ in $Q$.
\end{definition}

\begin{definition}
  The {\em structural congruence} \cite{SangiorgiWalker} , $\equiv$,
  between processes is the least congruence containing
  alpha-equivalence, satisfying the abelian monoid laws
  (associativity, commutativity and $\pzero$ as identity) for parallel
  composition $|$ and for summation $+$.
\end{definition}

\subsection{Name equivalence}

We take name equivalence, written $\nameeq$, to be the smallest
equivalence relation generated by the following rules.

\begin{mathpar}
\inferrule*[lab=Quote-drop]
{ }
{ \quotep{@{x}} \nameeq x }

\inferrule*[lab=Struct-equiv]
{ P \scong Q }
{ \quotep{P} \nameeq \quotep{Q} }
\end{mathpar}

The astute reader will have noticed that the mutual recursion of names
and processes imposes a mutual recursion on alpha-equivalence and
structural equivalence via name-equivalence. Fortunately, all of this
works out pleasantly and we may calculate in the natural way, free of
concern. The reader interested in the details is referred to the
appendix \ref{appendix:rho_details}.

\subsection{Substitution}

We use $\Proc$ for the set of processes, $\QProc$ for the set of
names, and $\id{\{}\vec{y} / \vec{x} \id{\}}$ to denote partial maps,
$s : \QProc \rightarrow \QProc$. A map, $s$ lifts, uniquely, to a map
on process terms, $\widehat{s} : \Proc \rightarrow \Proc$ by the
following equations.

\begin{mathpar}
  (0) \psubstp{Q}{P} := 0 \\
  (R \juxtap S) \psubstp{Q}{P}
  :=    
  (R)\psubstp{Q}{P} \juxtap (S) \psubstp{Q}{P} \\
  (x?(y).R) \psubstp{Q}{P}    
  :=    
  (x)\substp{Q}{P} (z)\concat( (R \psubstn{z}{y}) \psubstp{Q}{P} ) \\
  (\lift{x}{R}) \psubstp{Q}{P}  
  :=
  \lift{(x)\substp{Q}{P}}{ R \psubstp{Q}{P} } \\
%   (\dropn{x})  \psubstp{Q}{P}       
%   := 
%   \left\{ 
%     \begin{array}{ccc} 
%       \dropn{\quotep{Q}} & & x \nameeq \quotep{P} \\
%       \dropn{x} & & otherwise \\
%     \end{array}
%   \right. 
  (\dropn{x})  \psubstp{Q}{P}       
  := 
  \left\{ 
    \begin{array}{ccc} 
      Q & & x \nameeq \quotep{P} \\
      \dropn{x} & & otherwise \\
    \end{array}
  \right.
\end{mathpar}
 

where

\begin{eqnarray}
  (x)\id{\{} \lpquote Q \rpquote / \lpquote P \rpquote \id{\}}            = 
  \left\{ 
    \begin{array}{ccc}
      \lpquote Q \rpquote & & x \nameeq \lpquote P \rpquote \\
      x & & otherwise \\
    \end{array}
  \right. \nonumber
\end{eqnarray}

and $z$ is chosen distinct from $\quotep{P}$, $\quotep{Q}$, the free
names in $Q$, and all the names in $R$. Our $\alpha$-equivalence will
be built in the standard way from this substitution.

\begin{remark}\label{rem:no_self_referential_names}
  One consequence of these definitions is that $\forall P. \quotep{P}
  \not\in \freenames{P}$.
\end{remark}

\subsection{ Dynamic quote: an example }

Anticipating something of what's to come, consider applying the
substitution, $\widehat{\id{\{}u / z \id{\}}}$, to the following pair
of processes, $\lift{w}{y!(z)}$ and $w[ \lpquote y!(z) \rpquote ]$.

\begin{eqnarray}
	\lift{w}{y!(z)}\widehat{\id{\{}u / z \id{\}}}
		& = &
		\lift{w}{y!(u)} \nonumber\\
	w[ \lpquote y!(z) \rpquote ] \widehat{ \id{\{}u / z \id{\}} }
		& = &
		w[ \lpquote y!(z) \rpquote ] \nonumber
\end{eqnarray}

Because the body of the process between quotes is impervious to
substitution, we get radically different answers. In fact, by
examining the first process in an input context,
e.g. $x?(z).\lift{w}{y!(z)}$, we see that the process under the lift
operator may be shaped by prefixed inputs binding a name inside it. In
this sense, the lift operator will be seen as a way to dynamically
construct processes before reifying them as names.

Finally equipped with these standard features we can present the
dynamics of the calculus.

\subsubsection{Operational semantics} 

Finally, we introduce the computational dynamics. What marks these
algebras as distinct from other more traditionally studied algebraic
structures, e.g. vector spaces or polynomial rings, is the manner in
which dynamics is captured. In traditional structures, dynamics is typically
expressed through morphisms between such structures, as in linear maps
between vector spaces or morphisms between rings. In algebras
associated with the semantics of computation, the dynamics is
expressed as part of the algebraic structure itself, through a
reduction reduction relation typically denoted by $\red$. Below, we
give a recursive presentation of this relation for the calculus used
in the encoding.

$\red \subseteq \pi \times \pi$
$\red : \pi \to \mathcal{P}(\pi)$

\begin{mathpar}
  \inferrule* [lab=Comm] { \textsf{match}( x_{src}, x_{trgt} ) } { x_{trgt}?(y)P \; | \; x_{src}!\langle {Q} \rangle \red P\{\quotep{Q}/y}\} }
  \and \\
  \inferrule* [lab=Par] {{P} \red {P}'} {{{P} | {Q}} \red {{P}' | {Q}}}
  \and
  \inferrule* [lab=Equiv]{{{P} \scong {P}'} \andalso {{P}' \red {Q}'} \andalso {{Q}' \scong {Q}}}{{P} \red {Q}}
\end{mathpar}

\begin{eqnarray*}
  match_{\equiv} (\quotep{P},\quotep{Q}) & := & P \equiv Q \\
  match_{\dagger}(\quotep{P},\quotep{Q}) & := & \forall R. P|Q \red^{*} R => R \red^{*} 0 \\
  match_{K}(\quotep{P},\quotep{Q}) & := & K \mbox{ for some context } K
\end{eqnarray*}

$u?(x)P | u!\langle Q \rangle \red P\{\quotep{Q}/x\}$

%We write $\wred$ for $\red^*$, and $P\red$ if $\exists Q $ such that $ P \red Q$.
We write $P\red$ if $\exists Q $ such that $ P \red Q$ and $P\not\red$, otherwise.

\section{Replication}

As mentioned before, it is known that replication (and hence
recursion) can be implemented in a higher-order process algebra
\cite{SangiorgiWalker}. As our first example of calculation with the
machinery thus far presented we give the construction explicitly in
the {\rhoc}.

\begin{eqnarray}
	D_{x} & := & \prefix{x}{y}{(\binpar{\outputp{x}{y}}{@{y}})} \nonumber\\
	\bangp_{x}{P} & := & \binpar{{x}!\langle{\binpar{D_{x}}{P}}\rangle}{D_{x}} \nonumber
\end{eqnarray}

\begin{eqnarray}
	\bangp_{x}{P} & & \nonumber\\
	=
	& {x}!\langle{(\prefix{x}{y}{(\outputp{x}{y} | @{y})) | P}}\rangle 
	      | \prefix{x}{y}{(\outputp{x}{y} | @{y})} & \nonumber\\
	\red
	& (\outputp{x}{y} | @{y})\substn{\quotep{(\prefix{x}{y}{(@{y} | \outputp{x}{y})) | P}}}{y} & \nonumber\\
	=
	& \outputp{x}{\quotep{(\prefix{x}{y}{(\outputp{x}{y} | @{y})) | P}}}
	  | {(\prefix{x}{y}{(\outputp{x}{y} | @{y})) | P}} & \nonumber\\
	\red
	& \ldots & \nonumber\\
	\red^*
	& P | P | \ldots & \nonumber
\end{eqnarray}

Of course, this encoding, as an implementation, runs away, unfolding
$\bangp{P}$ eagerly. A lazier and more implementable replication
operator, restricted to input-guarded processes, may be obtained as follows.

\begin{eqnarray}
\bangp{\prefix{u}{v}{P}} 
	:= 
	\binpar{\lift{x}{\prefix{u}{v}{(\binpar{D(x)}{P})}}}{D(x)} \nonumber
\end{eqnarray}

\begin{remark}
  Note that the lazier definition still does not deal with summation
  or mixed summation (i.e. sums over input and output). The reader is
  invited to construct definitions of replication that deal with these
  features. 

  Further, the definitions are parameterized in a name, $x$. Can you,
  gentle reader, make a definition that eliminates this parameter and
  guarantees no accidental interaction between the replication
  machinery and the process being replicated -- i.e. no accidental
  sharing of names used by the process to get its work done and the
  name(s) used by the replication to effect copying. This latter
  revision of the definition of replication is crucial to obtaining
  the expected identity $!!P \sim !P$.
\end{remark}

\begin{remark}\label{rem:paradoxical_combinator}
  The reader familiar with the lambda calculus will have noticed the
  similarity between $D$ and the paradoxical combinator.

  [Ed. note: the existence of this seems to suggest we have to be more
  restrictive on the set of processes and names we admit if we are to
  support no-cloning.]
\end{remark}

\subsubsection{Bisimulation}

The computational dynamics gives rise to another kind of equivalence,
the equivalence of computational behavior. As previously mentioned
this is typically captured \emph{via} some form of bisimulation.

% The notion we use in this paper is weak barbed bisimulation
% \cite{milner91polyadicpi}.

The notion we use in this paper is derived from weak barbed
bisimulation \cite{milner91polyadicpi}. 

\begin{definition}
An \emph{observation relation}, $\downarrow_{\mathcal N}$, over a set
of names, $\mathcal N$, is the smallest relation satisfying the rules
below.

\infrule[Out-barb]{y \in {\mathcal N}, \; x \nameeq y}
		  {\outputp{x}{v} \downarrow_{\mathcal N} x}
\infrule[Par-barb]{\mbox{$P\downarrow_{\mathcal N} x$ or $Q\downarrow_{\mathcal N} x$}}
		  {\binpar{P}{Q} \downarrow_{\mathcal N} x}

We write $P \Downarrow_{\mathcal N} x$ if there is $Q$ such that 
$P \wred Q$ and $Q \downarrow_{\mathcal N} x$.
\end{definition}

\begin{definition}
%\label{def.bbisim}
An  ${\mathcal N}$-\emph{barbed bisimulation} over a set of names, ${\mathcal N}$, is a symmetric binary relation 
${\mathcal S}_{\mathcal N}$ between agents such that $P\rel{S}_{\mathcal N}Q$ implies:
\begin{enumerate}
\item If $P \red P'$ then $Q \wred Q'$ and $P'\rel{S}_{\mathcal N} Q'$.
\item If $P\downarrow_{\mathcal N} x$, then $Q\Downarrow_{\mathcal N} x$.
\end{enumerate}
$P$ is ${\mathcal N}$-barbed bisimilar to $Q$, written
$P \wbbisim_{\mathcal N} Q$, if $P \rel{S}_{\mathcal N} Q$ for some ${\mathcal N}$-barbed bisimulation ${\mathcal S}_{\mathcal N}$.
\end{definition}

$\mathcal{R} \subseteq \pi \times \pi$

$P \mathcal{R} Q => \forall P'. P \red P' \Rightarrow \exists Q'. Q \red Q', P' \mathcal{R} Q'$

$P \vdash x \Rightarrow Q \vdash x$

\begin{mathpar}
  \inferrule*[lab=Out-barb]{x \nameeq y}{{y}!\langle{Q}\rangle \vdash x}
  \and
  \inferrule*[lab=Par-barb]{\mbox{$P\vdash x$ or $Q\vdash x$}}{\binpar{P}{Q} \vdash x}
\end{mathpar}

\subsubsection{Contexts}

One of the principle advantages of computational calculi like the
$\pi$-calculus is a well-defined notion of context,
contextual-equivalence and a correlation between
contextual-equivalence and notions of bisimulation. The notion of
context allows the decomposition of a process into (sub-)process and
its syntactic environment, its context. Thus, a context may be
thought of as a process with a ``hole'' (written $\Box$) in it. The
application of a context $M$ to a process $P$, written $M[P]$, is
tantamount to filling the hole in $M$ with $P$. In this paper we do
not need the full weight of this theory, but do make use of the notion
of context in the proof the main theorem. 

\begin{mathpar}
  \inferrule* [lab=summation] {} {{M_{M},M_{N}} \bc \Box \;|\; x.M_{A} \;|\; M_{M}+M_{N}}
  \and
  \inferrule* [lab=agent] {} {{M_{A}} \bc (\vec{x})M_{P} \;| \; \clift{P_0,\ldots,M_{P},\ldots,P_N}}
  \and \\
  \inferrule* [lab=process] {} {{M_{P}} \bc M_{N} \;| \;P|M_{P} }
\end{mathpar} 

\begin{mathpar}
  \inferrule* [lab=sychronization] {} {M_{N} \bc \Box \;|\; x?M_{F} \;|\; x!M_{C}}
  \and
  \inferrule* [lab=abstraction] {} {{M_{F}} \bc (x)M_{P} }
  \and
  \inferrule* [lab=concretion] {} {{M_{C}} \bc \langle M_{P} \rangle }
  \and \\
  \inferrule* [lab=process] {} {{M_{P}} \bc M_{N} \;| \;P|M_{P} }
\end{mathpar}

\begin{definition}[contextual application] Given a context $M$, and
  process $P$, we define the \emph{contextual application}, $M[P] :=
  M\{P/\Box\}$. That is, the contextual application of M to P is the
  substitution of $P$ for $\Box$ in $M$.
\end{definition}

$\meaningof{-} : L \to \mathcal{P}(\pi)$

\begin{mathpar}
  \inferrule* [lab=collection] {} {\meaningof{true} = \pi, \and \meaningof{~E} = \pi \setminus \meaningof{E}, \and \meaningof{E_{1} \& E_{2}} = \meaningof{E_{1}} \cap \meaningof{E_{2}}}
\end{mathpar}

\begin{mathpar}
  \inferrule* [lab=structure] {} {\meaningof{0} = \{ P \in \pi | P \equiv 0 \}, \and \\ \meaningof{E_1 | E_2} = \{ P \in \pi | P \equiv P_{1} | P_{2}, P_{1} \in \meaningof{E_{1}}, P_{2} \in \meaningof{E_2}\} }
\end{mathpar}

\begin{mathpar}
 \inferrule* [lab=behavior] {} {\meaningof{\langle a?b \rangle E} = \{ P \in \pi | P \equiv Q | u?(y)P', \\ \and \\\\ \and \\ \;\;\; u \in \meaningof{a}, \forall z.P'\{z/y\} \in \meaningof{E\{z/b\}}\}, \and \\ \meaningof{a!E} = \{ P \in \pi | P \equiv Q | x!\langle P' \rangle, x \in \meaningof{a} P' \in \meaningof{E}\} }
\end{mathpar}

\begin{mathpar}
 \inferrule* [lab=nominal] {} {\meaningof{\quotep{E}} = \{ \quotep{P} \in \quotep{\pi} | P \in \meaningof{E} \}, \and \meaningof{\quotep{P}} = \{ \quotep{Q} \in \quotep{\pi} | P \equiv Q \} \and \\ \meaningof{@\quotep{E}} = \{ P \in \pi | P \equiv @x, x \in \meaningof{E} \}}
\end{mathpar}

\begin{eqnarray*}
  \\
  \meaningof{-} : TS \to ST
\end{eqnarray*}

\begin{eqnarray*}
  \\
  L : TS \to ST
\end{eqnarray*}

\begin{eqnarray*}
  \\
  P \models E \iff P \in \meaningof{E}
\end{eqnarray*}

\begin{eqnarray*}
  P \approx_{L} Q \iff \forall E \in L. P \models E \iff Q \models E
\end{eqnarray*}

\begin{eqnarray*}
  P \approx_{K} Q
\end{eqnarray*}

\begin{eqnarray*}
  P \approx Q
\end{eqnarray*}

$\approx_{K} = \approx = \approx_{L}$

\subsubsection{Contextual duality}

Note that contexts extend the quotation operation to a family of
operations from processes to names. Given a context, $M$, we can
define a \emph{nominal context}, $\quotep{M}$ by $\quotep{M}[P] :=
\quotep{M[P]}$. To foreshadow what is to come we observe that these
operations enjoy a duality with processes very much like the duality
between vectors and maps from vectors to scalars.

Further, because the calculus is essentially higher-order, we have a
correspondence between contexts and processes. More specifically,
given a name $x$ and a context $M$ we can construct $M^{*}_{x}$ such
that 

\begin{mathpar}
  M^{*}_{x} | \lift{x}{P} \red M[P]
\end{mathpar}

namely,

\begin{mathpar}
  M^{*}_{x} := x?(u).M[\dropn{u}]
\end{mathpar}

The dependence of $M^{*}_{x}$ on a name makes it an abstraction, 

\begin{mathpar}
  M^{*} := (x)x?(u).M[\dropn{u}]
\end{mathpar}

\subsection{Additional notation}

It will sometimes be convenient to denote the process a name
quotes. We already have the notation $x = \quotep{P}$, but it will be
convenient to introduce an alternate notation, $\procn{x}$, when we
want to emphasize the connection to the use of the name. Note that, by
virtue of name equivalence, $\quotep{\procn{x}} \nameeq x$; so, the
notation is consistent with previous definitions.

Further, because names have structure it is possible to effect
substitutions on the basis of that structure. This means we need to
upgrade our notation for substitutions, which we accomplish by
adapting comprehension notation. Thus,

\begin{mathpar}
  P\{ y / x : x \in S \}
\end{mathpar}

is interpreted to mean the process derived from P by replacing (in a
capture-avoiding manner) each occurrence of $x$ in $S$ by $y$. For example,

\begin{mathpar}
  P\{ \quotep{\procn{x}|\procn{x}} / x : x \in \freenames{P} \}
\end{mathpar}

will replace each (occurrence) of a free name $x$ in $P$ by
$\quotep{\procn{x}|\procn{x}}$.

Also, we will avail ourselves of the notation $x^{L}$ and $x^{R}$ to
denote injections of a name into disjoint copies of the name
space. There are numerous ways to accomplish this. One example can be
found in \cite{MeredithR05}. This notation overloads to vectors of
names: $\vec{x}^{\pi} := (x_{i}^{\pi} \; : \; 0 \leq i < |\vec{x}| )$ where $\pi \in \{L,R\}$.

We also use $P^{\Box} := P|\Box$.

In \cite{MeredithR05} an interpretation of the new operator is
given. It turns out that there are several possible interpretations
all enjoying the requisite algebraic properties of the operator (see
\cite{milner91polyadicpi}). We will therefore make liberal use of
$(\nu\; \vec{x})P$.

% subsection the_syntax_and_semantics_of_the_notation_system (end)   

\input{qm2pi.qmops} 

\input{qm2pi.sterngerlach} 

\input{qm2pi.metric} 

% section concurrent_process_calculi (end)

%\input{qm2pi.proofsketch}

% section proof sketch (end)

%\input{qm2pi.slviaknots} 

% section spatial logic via knots (end)

\input{qm2pi.conclusion}

% section conclusion (end)

%\input{qm2pi.dtcodes} 

% section wiring algorithm (end)

\input{qm2pi.ack} 

% section acknowledgments (end)

\newpage


\bibliographystyle{plain}   
\bibliography{../../biblios/main.bib}

\input{qm2pi.rhodetails}

\end{document}

 

% section wiring algorithm (end)

\documentclass[12pt]{llncs}
%\documentclass{jktr}

\usepackage[pdftex]{hyperref}                   
\usepackage {listings}
\usepackage {mathpartir}
\usepackage{bcprules}
%\usepackage{listings}
                       
\usepackage{graphicx} 
%\usepackage[margins=2.5cm,nohead,nofoot]{geometry}
%\usepackage{geometry}
\usepackage{amsfonts}
\usepackage{amstext}
\usepackage{latexsym}
\usepackage{amssymb}
\usepackage{color}


%\include{myPreamble}
\include{qm2pi.local} 

%\ifpdf
%\usepackage[pdftex]{graphicx}
%\else
%\usepackage{graphicx}
%\fi

 % \ifpdf
%  \usepackage{pdfsync}
%  \if


%\title{Brief Article}
%\author{David F. Snyder}
%\author{L.G. Meredith}

%\address{Dept. of Math., Texas State University--San Marcos, San Marcos, TX 78666}
       
\pagestyle{empty}


\begin{document}

\lstset{language=[Objective]Caml,frame=shadowbox}

\input{qm2pi.front}

% section front matter (end)

\input{qm2pi.intro} 
 
% section introduction (end)

% \input{qm2pi.knotations} 

% section notation (end)

\input{qm2pi.process.calculi} 

% section concurrent_process_calculi_and_spatial_logics_ (end)
    
%\input{qm2pi.knots2pi} 

%\input{qm2pi.trefoil} 

%\input{qm2pi.mainthm} 

% subsection basic_interpretation (end)

%\input{qm2pi.rho.presentation} 
\subsection{The syntax and semantics of the notation system}\label{sub:the_syntax_and_semantics_of_the_notation_system} % (fold)

We now summarize a technical presentation of the calculus that
embodies our theory of dynamics. The typical presentation of such a
calculus follows the style of giving generators and relations on
them. The grammar, below, describing term constructors, freely
generates the set of processes, $\Proc$. This set is then quotiented
by a relation known as structural congruence and it is over this set
that the notion of dynamics is expressed. This presentation is
essentially that of \cite{MeredithR05} with the addition of
polyadicity and summation. For readability we have relegated some of
the technical subtleties to an appendix.

\subsubsection{Process grammar}\label{subsub:process_grammar}

\begin{mathpar}
  \inferrule* [lab=synchronization] {} {{M} \bc \pzero \;|\; x?F \;|\; x!C }
  \and
  \inferrule* [lab=abstraction] {} {{F} \bc (x)P}
  \and
  \inferrule* [lab=concretion] {} {{C} \bc \langle Q \rangle}
  \and
  \inferrule* [lab=process] {} {{P,Q} \bc M \;| \;P|Q \;|\; @{x}}
  \and
  \inferrule* [lab=name] {} {{x} \bc \quotep{P}}
\end{mathpar} 

Note that $\vec{x}$ (resp. $\vec{P}$) denotes a vector of names
(resp. processes) of length $|\vec{x}|$ (resp. $|\vec{P}|$). We adopt
the following useful abbreviations.

\begin{mathpar}
   x?(\vec{y}).P := x.(\vec{y})P \and  x\clift{\vec{P}} := x.\clift{\vec{P}}
   \and x!(y) := \lift{x}{\dropn{y}}
   \and \Pi_{i=0}^{n-1}P_i := P_0 | \ldots | P_{n-1}
\end{mathpar}

\subsubsection{Structural congruence}

\paragraph{Free and bound names and alpha-equivalence.} At the
core of structural equivalence is alpha-equivalence which identifies
process that are the same up to a change of variable. Formally, we
recognize the distinction between free and bound names. The free names
of a process, $\freenames{P}$, may be calculated recursively as
follows:

\begin{mathpar}
\freenames{\pzero} := \emptyset
  \and \\
  \freenames{x?(y).P} := \{ x \} \cup (\freenames{P} \setminus \{ y \})
  \and 
  \freenames{x!\langle P \rangle} := \{ x \} \cup \{ P \} 
  \and \\
  \freenames{P|Q} := \freenames{P} \cup \freenames{Q}
  \and \\
  \freenames{@{x}} := \{ x \}
\end{mathpar}

$\pi$
$\quotep{\pi}$

$\freenames{-} : \pi \to \mathcal{P}(\quotep{\pi})$

\begin{eqnarray*}
  \freenames{\pzero} & := & \emptyset \\
  \freenames{x?(y).P} & := & \{ x \} \cup (\freenames{P} \setminus \{ y \}) \\
  \freenames{x!\langle P \rangle} & := & \{ x \} \cup \{ P \} \\
  \freenames{P|Q} & := & \freenames{P} \cup \freenames{Q} \\
  \freenames{\dropn{x}} & := & \{ x \}
\end{eqnarray*}

The bound names of a process, $\boundnames{P}$, are those names occurring in $P$
that are not free. For example, in $x?(y).0$, the name $x$ is free, while $y$ is bound.

\begin{mathpar}
  \inferrule* [lab=monoidal-laws] {} { P|Q \equiv Q|P \and P|0 \equiv P \and P|(Q|R) \equiv (P|Q)|R }
\end{mathpar}

\begin{mathpar}
  \inferrule* [lab=alpha-equivalence] {} { (x)P \equiv (y)P\{y/x\} \and y \not\in \freenames{P} }
\end{mathpar}

\begin{definition}
Then two processes, $P,Q$, are alpha-equivalent if $P = Q\{\vec{y}/\vec{x}\}$ for
some $\vec{x} \in \boundnames{Q},\vec{y} \in \boundnames{P}$, where $Q\{\vec{y}/\vec{x}\}$
denotes the capture-avoiding substitution of $\vec{y}$ for $\vec{x}$ in $Q$.
\end{definition}

\begin{definition}
  The {\em structural congruence} \cite{SangiorgiWalker} , $\equiv$,
  between processes is the least congruence containing
  alpha-equivalence, satisfying the abelian monoid laws
  (associativity, commutativity and $\pzero$ as identity) for parallel
  composition $|$ and for summation $+$.
\end{definition}

\subsection{Name equivalence}

We take name equivalence, written $\nameeq$, to be the smallest
equivalence relation generated by the following rules.

\begin{mathpar}
\inferrule*[lab=Quote-drop]
{ }
{ \quotep{@{x}} \nameeq x }

\inferrule*[lab=Struct-equiv]
{ P \scong Q }
{ \quotep{P} \nameeq \quotep{Q} }
\end{mathpar}

The astute reader will have noticed that the mutual recursion of names
and processes imposes a mutual recursion on alpha-equivalence and
structural equivalence via name-equivalence. Fortunately, all of this
works out pleasantly and we may calculate in the natural way, free of
concern. The reader interested in the details is referred to the
appendix \ref{appendix:rho_details}.

\subsection{Substitution}

We use $\Proc$ for the set of processes, $\QProc$ for the set of
names, and $\id{\{}\vec{y} / \vec{x} \id{\}}$ to denote partial maps,
$s : \QProc \rightarrow \QProc$. A map, $s$ lifts, uniquely, to a map
on process terms, $\widehat{s} : \Proc \rightarrow \Proc$ by the
following equations.

\begin{mathpar}
  (0) \psubstp{Q}{P} := 0 \\
  (R \juxtap S) \psubstp{Q}{P}
  :=    
  (R)\psubstp{Q}{P} \juxtap (S) \psubstp{Q}{P} \\
  (x?(y).R) \psubstp{Q}{P}    
  :=    
  (x)\substp{Q}{P} (z)\concat( (R \psubstn{z}{y}) \psubstp{Q}{P} ) \\
  (\lift{x}{R}) \psubstp{Q}{P}  
  :=
  \lift{(x)\substp{Q}{P}}{ R \psubstp{Q}{P} } \\
%   (\dropn{x})  \psubstp{Q}{P}       
%   := 
%   \left\{ 
%     \begin{array}{ccc} 
%       \dropn{\quotep{Q}} & & x \nameeq \quotep{P} \\
%       \dropn{x} & & otherwise \\
%     \end{array}
%   \right. 
  (\dropn{x})  \psubstp{Q}{P}       
  := 
  \left\{ 
    \begin{array}{ccc} 
      Q & & x \nameeq \quotep{P} \\
      \dropn{x} & & otherwise \\
    \end{array}
  \right.
\end{mathpar}
 

where

\begin{eqnarray}
  (x)\id{\{} \lpquote Q \rpquote / \lpquote P \rpquote \id{\}}            = 
  \left\{ 
    \begin{array}{ccc}
      \lpquote Q \rpquote & & x \nameeq \lpquote P \rpquote \\
      x & & otherwise \\
    \end{array}
  \right. \nonumber
\end{eqnarray}

and $z$ is chosen distinct from $\quotep{P}$, $\quotep{Q}$, the free
names in $Q$, and all the names in $R$. Our $\alpha$-equivalence will
be built in the standard way from this substitution.

\begin{remark}\label{rem:no_self_referential_names}
  One consequence of these definitions is that $\forall P. \quotep{P}
  \not\in \freenames{P}$.
\end{remark}

\subsection{ Dynamic quote: an example }

Anticipating something of what's to come, consider applying the
substitution, $\widehat{\id{\{}u / z \id{\}}}$, to the following pair
of processes, $\lift{w}{y!(z)}$ and $w[ \lpquote y!(z) \rpquote ]$.

\begin{eqnarray}
	\lift{w}{y!(z)}\widehat{\id{\{}u / z \id{\}}}
		& = &
		\lift{w}{y!(u)} \nonumber\\
	w[ \lpquote y!(z) \rpquote ] \widehat{ \id{\{}u / z \id{\}} }
		& = &
		w[ \lpquote y!(z) \rpquote ] \nonumber
\end{eqnarray}

Because the body of the process between quotes is impervious to
substitution, we get radically different answers. In fact, by
examining the first process in an input context,
e.g. $x?(z).\lift{w}{y!(z)}$, we see that the process under the lift
operator may be shaped by prefixed inputs binding a name inside it. In
this sense, the lift operator will be seen as a way to dynamically
construct processes before reifying them as names.

Finally equipped with these standard features we can present the
dynamics of the calculus.

\subsubsection{Operational semantics} 

Finally, we introduce the computational dynamics. What marks these
algebras as distinct from other more traditionally studied algebraic
structures, e.g. vector spaces or polynomial rings, is the manner in
which dynamics is captured. In traditional structures, dynamics is typically
expressed through morphisms between such structures, as in linear maps
between vector spaces or morphisms between rings. In algebras
associated with the semantics of computation, the dynamics is
expressed as part of the algebraic structure itself, through a
reduction reduction relation typically denoted by $\red$. Below, we
give a recursive presentation of this relation for the calculus used
in the encoding.

$\red \subseteq \pi \times \pi$
$\red : \pi \to \mathcal{P}(\pi)$

\begin{mathpar}
  \inferrule* [lab=Comm] { \textsf{match}( x_{src}, x_{trgt} ) } { x_{trgt}?(y)P \; | \; x_{src}!\langle {Q} \rangle \red P\{\quotep{Q}/y}\} }
  \and \\
  \inferrule* [lab=Par] {{P} \red {P}'} {{{P} | {Q}} \red {{P}' | {Q}}}
  \and
  \inferrule* [lab=Equiv]{{{P} \scong {P}'} \andalso {{P}' \red {Q}'} \andalso {{Q}' \scong {Q}}}{{P} \red {Q}}
\end{mathpar}

\begin{eqnarray*}
  match_{\equiv} (\quotep{P},\quotep{Q}) & := & P \equiv Q \\
  match_{\dagger}(\quotep{P},\quotep{Q}) & := & \forall R. P|Q \red^{*} R => R \red^{*} 0 \\
  match_{K}(\quotep{P},\quotep{Q}) & := & K \mbox{ for some context } K
\end{eqnarray*}

$u?(x)P | u!\langle Q \rangle \red P\{\quotep{Q}/x\}$

%We write $\wred$ for $\red^*$, and $P\red$ if $\exists Q $ such that $ P \red Q$.
We write $P\red$ if $\exists Q $ such that $ P \red Q$ and $P\not\red$, otherwise.

\section{Replication}

As mentioned before, it is known that replication (and hence
recursion) can be implemented in a higher-order process algebra
\cite{SangiorgiWalker}. As our first example of calculation with the
machinery thus far presented we give the construction explicitly in
the {\rhoc}.

\begin{eqnarray}
	D_{x} & := & \prefix{x}{y}{(\binpar{\outputp{x}{y}}{@{y}})} \nonumber\\
	\bangp_{x}{P} & := & \binpar{{x}!\langle{\binpar{D_{x}}{P}}\rangle}{D_{x}} \nonumber
\end{eqnarray}

\begin{eqnarray}
	\bangp_{x}{P} & & \nonumber\\
	=
	& {x}!\langle{(\prefix{x}{y}{(\outputp{x}{y} | @{y})) | P}}\rangle 
	      | \prefix{x}{y}{(\outputp{x}{y} | @{y})} & \nonumber\\
	\red
	& (\outputp{x}{y} | @{y})\substn{\quotep{(\prefix{x}{y}{(@{y} | \outputp{x}{y})) | P}}}{y} & \nonumber\\
	=
	& \outputp{x}{\quotep{(\prefix{x}{y}{(\outputp{x}{y} | @{y})) | P}}}
	  | {(\prefix{x}{y}{(\outputp{x}{y} | @{y})) | P}} & \nonumber\\
	\red
	& \ldots & \nonumber\\
	\red^*
	& P | P | \ldots & \nonumber
\end{eqnarray}

Of course, this encoding, as an implementation, runs away, unfolding
$\bangp{P}$ eagerly. A lazier and more implementable replication
operator, restricted to input-guarded processes, may be obtained as follows.

\begin{eqnarray}
\bangp{\prefix{u}{v}{P}} 
	:= 
	\binpar{\lift{x}{\prefix{u}{v}{(\binpar{D(x)}{P})}}}{D(x)} \nonumber
\end{eqnarray}

\begin{remark}
  Note that the lazier definition still does not deal with summation
  or mixed summation (i.e. sums over input and output). The reader is
  invited to construct definitions of replication that deal with these
  features. 

  Further, the definitions are parameterized in a name, $x$. Can you,
  gentle reader, make a definition that eliminates this parameter and
  guarantees no accidental interaction between the replication
  machinery and the process being replicated -- i.e. no accidental
  sharing of names used by the process to get its work done and the
  name(s) used by the replication to effect copying. This latter
  revision of the definition of replication is crucial to obtaining
  the expected identity $!!P \sim !P$.
\end{remark}

\begin{remark}\label{rem:paradoxical_combinator}
  The reader familiar with the lambda calculus will have noticed the
  similarity between $D$ and the paradoxical combinator.

  [Ed. note: the existence of this seems to suggest we have to be more
  restrictive on the set of processes and names we admit if we are to
  support no-cloning.]
\end{remark}

\subsubsection{Bisimulation}

The computational dynamics gives rise to another kind of equivalence,
the equivalence of computational behavior. As previously mentioned
this is typically captured \emph{via} some form of bisimulation.

% The notion we use in this paper is weak barbed bisimulation
% \cite{milner91polyadicpi}.

The notion we use in this paper is derived from weak barbed
bisimulation \cite{milner91polyadicpi}. 

\begin{definition}
An \emph{observation relation}, $\downarrow_{\mathcal N}$, over a set
of names, $\mathcal N$, is the smallest relation satisfying the rules
below.

\infrule[Out-barb]{y \in {\mathcal N}, \; x \nameeq y}
		  {\outputp{x}{v} \downarrow_{\mathcal N} x}
\infrule[Par-barb]{\mbox{$P\downarrow_{\mathcal N} x$ or $Q\downarrow_{\mathcal N} x$}}
		  {\binpar{P}{Q} \downarrow_{\mathcal N} x}

We write $P \Downarrow_{\mathcal N} x$ if there is $Q$ such that 
$P \wred Q$ and $Q \downarrow_{\mathcal N} x$.
\end{definition}

\begin{definition}
%\label{def.bbisim}
An  ${\mathcal N}$-\emph{barbed bisimulation} over a set of names, ${\mathcal N}$, is a symmetric binary relation 
${\mathcal S}_{\mathcal N}$ between agents such that $P\rel{S}_{\mathcal N}Q$ implies:
\begin{enumerate}
\item If $P \red P'$ then $Q \wred Q'$ and $P'\rel{S}_{\mathcal N} Q'$.
\item If $P\downarrow_{\mathcal N} x$, then $Q\Downarrow_{\mathcal N} x$.
\end{enumerate}
$P$ is ${\mathcal N}$-barbed bisimilar to $Q$, written
$P \wbbisim_{\mathcal N} Q$, if $P \rel{S}_{\mathcal N} Q$ for some ${\mathcal N}$-barbed bisimulation ${\mathcal S}_{\mathcal N}$.
\end{definition}

$\mathcal{R} \subseteq \pi \times \pi$

$P \mathcal{R} Q => \forall P'. P \red P' \Rightarrow \exists Q'. Q \red Q', P' \mathcal{R} Q'$

$P \vdash x \Rightarrow Q \vdash x$

\begin{mathpar}
  \inferrule*[lab=Out-barb]{x \nameeq y}{{y}!\langle{Q}\rangle \vdash x}
  \and
  \inferrule*[lab=Par-barb]{\mbox{$P\vdash x$ or $Q\vdash x$}}{\binpar{P}{Q} \vdash x}
\end{mathpar}

\subsubsection{Contexts}

One of the principle advantages of computational calculi like the
$\pi$-calculus is a well-defined notion of context,
contextual-equivalence and a correlation between
contextual-equivalence and notions of bisimulation. The notion of
context allows the decomposition of a process into (sub-)process and
its syntactic environment, its context. Thus, a context may be
thought of as a process with a ``hole'' (written $\Box$) in it. The
application of a context $M$ to a process $P$, written $M[P]$, is
tantamount to filling the hole in $M$ with $P$. In this paper we do
not need the full weight of this theory, but do make use of the notion
of context in the proof the main theorem. 

\begin{mathpar}
  \inferrule* [lab=summation] {} {{M_{M},M_{N}} \bc \Box \;|\; x.M_{A} \;|\; M_{M}+M_{N}}
  \and
  \inferrule* [lab=agent] {} {{M_{A}} \bc (\vec{x})M_{P} \;| \; \clift{P_0,\ldots,M_{P},\ldots,P_N}}
  \and \\
  \inferrule* [lab=process] {} {{M_{P}} \bc M_{N} \;| \;P|M_{P} }
\end{mathpar} 

\begin{mathpar}
  \inferrule* [lab=sychronization] {} {M_{N} \bc \Box \;|\; x?M_{F} \;|\; x!M_{C}}
  \and
  \inferrule* [lab=abstraction] {} {{M_{F}} \bc (x)M_{P} }
  \and
  \inferrule* [lab=concretion] {} {{M_{C}} \bc \langle M_{P} \rangle }
  \and \\
  \inferrule* [lab=process] {} {{M_{P}} \bc M_{N} \;| \;P|M_{P} }
\end{mathpar}

\begin{definition}[contextual application] Given a context $M$, and
  process $P$, we define the \emph{contextual application}, $M[P] :=
  M\{P/\Box\}$. That is, the contextual application of M to P is the
  substitution of $P$ for $\Box$ in $M$.
\end{definition}

$\meaningof{-} : L \to \mathcal{P}(\pi)$

\begin{mathpar}
  \inferrule* [lab=collection] {} {\meaningof{true} = \pi, \and \meaningof{~E} = \pi \setminus \meaningof{E}, \and \meaningof{E_{1} \& E_{2}} = \meaningof{E_{1}} \cap \meaningof{E_{2}}}
\end{mathpar}

\begin{mathpar}
  \inferrule* [lab=structure] {} {\meaningof{0} = \{ P \in \pi | P \equiv 0 \}, \and \\ \meaningof{E_1 | E_2} = \{ P \in \pi | P \equiv P_{1} | P_{2}, P_{1} \in \meaningof{E_{1}}, P_{2} \in \meaningof{E_2}\} }
\end{mathpar}

\begin{mathpar}
 \inferrule* [lab=behavior] {} {\meaningof{\langle a?b \rangle E} = \{ P \in \pi | P \equiv Q | u?(y)P', \\ \and \\\\ \and \\ \;\;\; u \in \meaningof{a}, \forall z.P'\{z/y\} \in \meaningof{E\{z/b\}}\}, \and \\ \meaningof{a!E} = \{ P \in \pi | P \equiv Q | x!\langle P' \rangle, x \in \meaningof{a} P' \in \meaningof{E}\} }
\end{mathpar}

\begin{mathpar}
 \inferrule* [lab=nominal] {} {\meaningof{\quotep{E}} = \{ \quotep{P} \in \quotep{\pi} | P \in \meaningof{E} \}, \and \meaningof{\quotep{P}} = \{ \quotep{Q} \in \quotep{\pi} | P \equiv Q \} \and \\ \meaningof{@\quotep{E}} = \{ P \in \pi | P \equiv @x, x \in \meaningof{E} \}}
\end{mathpar}

\begin{eqnarray*}
  \\
  \meaningof{-} : TS \to ST
\end{eqnarray*}

\begin{eqnarray*}
  \\
  L : TS \to ST
\end{eqnarray*}

\begin{eqnarray*}
  \\
  P \models E \iff P \in \meaningof{E}
\end{eqnarray*}

\begin{eqnarray*}
  P \approx_{L} Q \iff \forall E \in L. P \models E \iff Q \models E
\end{eqnarray*}

\begin{eqnarray*}
  P \approx_{K} Q
\end{eqnarray*}

\begin{eqnarray*}
  P \approx Q
\end{eqnarray*}

$\approx_{K} = \approx = \approx_{L}$

\subsubsection{Contextual duality}

Note that contexts extend the quotation operation to a family of
operations from processes to names. Given a context, $M$, we can
define a \emph{nominal context}, $\quotep{M}$ by $\quotep{M}[P] :=
\quotep{M[P]}$. To foreshadow what is to come we observe that these
operations enjoy a duality with processes very much like the duality
between vectors and maps from vectors to scalars.

Further, because the calculus is essentially higher-order, we have a
correspondence between contexts and processes. More specifically,
given a name $x$ and a context $M$ we can construct $M^{*}_{x}$ such
that 

\begin{mathpar}
  M^{*}_{x} | \lift{x}{P} \red M[P]
\end{mathpar}

namely,

\begin{mathpar}
  M^{*}_{x} := x?(u).M[\dropn{u}]
\end{mathpar}

The dependence of $M^{*}_{x}$ on a name makes it an abstraction, 

\begin{mathpar}
  M^{*} := (x)x?(u).M[\dropn{u}]
\end{mathpar}

\subsection{Additional notation}

It will sometimes be convenient to denote the process a name
quotes. We already have the notation $x = \quotep{P}$, but it will be
convenient to introduce an alternate notation, $\procn{x}$, when we
want to emphasize the connection to the use of the name. Note that, by
virtue of name equivalence, $\quotep{\procn{x}} \nameeq x$; so, the
notation is consistent with previous definitions.

Further, because names have structure it is possible to effect
substitutions on the basis of that structure. This means we need to
upgrade our notation for substitutions, which we accomplish by
adapting comprehension notation. Thus,

\begin{mathpar}
  P\{ y / x : x \in S \}
\end{mathpar}

is interpreted to mean the process derived from P by replacing (in a
capture-avoiding manner) each occurrence of $x$ in $S$ by $y$. For example,

\begin{mathpar}
  P\{ \quotep{\procn{x}|\procn{x}} / x : x \in \freenames{P} \}
\end{mathpar}

will replace each (occurrence) of a free name $x$ in $P$ by
$\quotep{\procn{x}|\procn{x}}$.

Also, we will avail ourselves of the notation $x^{L}$ and $x^{R}$ to
denote injections of a name into disjoint copies of the name
space. There are numerous ways to accomplish this. One example can be
found in \cite{MeredithR05}. This notation overloads to vectors of
names: $\vec{x}^{\pi} := (x_{i}^{\pi} \; : \; 0 \leq i < |\vec{x}| )$ where $\pi \in \{L,R\}$.

We also use $P^{\Box} := P|\Box$.

In \cite{MeredithR05} an interpretation of the new operator is
given. It turns out that there are several possible interpretations
all enjoying the requisite algebraic properties of the operator (see
\cite{milner91polyadicpi}). We will therefore make liberal use of
$(\nu\; \vec{x})P$.

% subsection the_syntax_and_semantics_of_the_notation_system (end)   

\input{qm2pi.qmops} 

\input{qm2pi.sterngerlach} 

\input{qm2pi.metric} 

% section concurrent_process_calculi (end)

%\input{qm2pi.proofsketch}

% section proof sketch (end)

%\input{qm2pi.slviaknots} 

% section spatial logic via knots (end)

\input{qm2pi.conclusion}

% section conclusion (end)

%\input{qm2pi.dtcodes} 

% section wiring algorithm (end)

\input{qm2pi.ack} 

% section acknowledgments (end)

\newpage


\bibliographystyle{plain}   
\bibliography{../../biblios/main.bib}

\input{qm2pi.rhodetails}

\end{document}

 

% section acknowledgments (end)

\newpage


\bibliographystyle{plain}   
\bibliography{../../biblios/main.bib}

\documentclass[12pt]{llncs}
%\documentclass{jktr}

\usepackage[pdftex]{hyperref}                   
\usepackage {listings}
\usepackage {mathpartir}
\usepackage{bcprules}
%\usepackage{listings}
                       
\usepackage{graphicx} 
%\usepackage[margins=2.5cm,nohead,nofoot]{geometry}
%\usepackage{geometry}
\usepackage{amsfonts}
\usepackage{amstext}
\usepackage{latexsym}
\usepackage{amssymb}
\usepackage{color}


%\include{myPreamble}
\include{qm2pi.local} 

%\ifpdf
%\usepackage[pdftex]{graphicx}
%\else
%\usepackage{graphicx}
%\fi

 % \ifpdf
%  \usepackage{pdfsync}
%  \if


%\title{Brief Article}
%\author{David F. Snyder}
%\author{L.G. Meredith}

%\address{Dept. of Math., Texas State University--San Marcos, San Marcos, TX 78666}
       
\pagestyle{empty}


\begin{document}

\lstset{language=[Objective]Caml,frame=shadowbox}

\input{qm2pi.front}

% section front matter (end)

\input{qm2pi.intro} 
 
% section introduction (end)

% \input{qm2pi.knotations} 

% section notation (end)

\input{qm2pi.process.calculi} 

% section concurrent_process_calculi_and_spatial_logics_ (end)
    
%\input{qm2pi.knots2pi} 

%\input{qm2pi.trefoil} 

%\input{qm2pi.mainthm} 

% subsection basic_interpretation (end)

%\input{qm2pi.rho.presentation} 
\subsection{The syntax and semantics of the notation system}\label{sub:the_syntax_and_semantics_of_the_notation_system} % (fold)

We now summarize a technical presentation of the calculus that
embodies our theory of dynamics. The typical presentation of such a
calculus follows the style of giving generators and relations on
them. The grammar, below, describing term constructors, freely
generates the set of processes, $\Proc$. This set is then quotiented
by a relation known as structural congruence and it is over this set
that the notion of dynamics is expressed. This presentation is
essentially that of \cite{MeredithR05} with the addition of
polyadicity and summation. For readability we have relegated some of
the technical subtleties to an appendix.

\subsubsection{Process grammar}\label{subsub:process_grammar}

\begin{mathpar}
  \inferrule* [lab=synchronization] {} {{M} \bc \pzero \;|\; x?F \;|\; x!C }
  \and
  \inferrule* [lab=abstraction] {} {{F} \bc (x)P}
  \and
  \inferrule* [lab=concretion] {} {{C} \bc \langle Q \rangle}
  \and
  \inferrule* [lab=process] {} {{P,Q} \bc M \;| \;P|Q \;|\; @{x}}
  \and
  \inferrule* [lab=name] {} {{x} \bc \quotep{P}}
\end{mathpar} 

Note that $\vec{x}$ (resp. $\vec{P}$) denotes a vector of names
(resp. processes) of length $|\vec{x}|$ (resp. $|\vec{P}|$). We adopt
the following useful abbreviations.

\begin{mathpar}
   x?(\vec{y}).P := x.(\vec{y})P \and  x\clift{\vec{P}} := x.\clift{\vec{P}}
   \and x!(y) := \lift{x}{\dropn{y}}
   \and \Pi_{i=0}^{n-1}P_i := P_0 | \ldots | P_{n-1}
\end{mathpar}

\subsubsection{Structural congruence}

\paragraph{Free and bound names and alpha-equivalence.} At the
core of structural equivalence is alpha-equivalence which identifies
process that are the same up to a change of variable. Formally, we
recognize the distinction between free and bound names. The free names
of a process, $\freenames{P}$, may be calculated recursively as
follows:

\begin{mathpar}
\freenames{\pzero} := \emptyset
  \and \\
  \freenames{x?(y).P} := \{ x \} \cup (\freenames{P} \setminus \{ y \})
  \and 
  \freenames{x!\langle P \rangle} := \{ x \} \cup \{ P \} 
  \and \\
  \freenames{P|Q} := \freenames{P} \cup \freenames{Q}
  \and \\
  \freenames{@{x}} := \{ x \}
\end{mathpar}

$\pi$
$\quotep{\pi}$

$\freenames{-} : \pi \to \mathcal{P}(\quotep{\pi})$

\begin{eqnarray*}
  \freenames{\pzero} & := & \emptyset \\
  \freenames{x?(y).P} & := & \{ x \} \cup (\freenames{P} \setminus \{ y \}) \\
  \freenames{x!\langle P \rangle} & := & \{ x \} \cup \{ P \} \\
  \freenames{P|Q} & := & \freenames{P} \cup \freenames{Q} \\
  \freenames{\dropn{x}} & := & \{ x \}
\end{eqnarray*}

The bound names of a process, $\boundnames{P}$, are those names occurring in $P$
that are not free. For example, in $x?(y).0$, the name $x$ is free, while $y$ is bound.

\begin{mathpar}
  \inferrule* [lab=monoidal-laws] {} { P|Q \equiv Q|P \and P|0 \equiv P \and P|(Q|R) \equiv (P|Q)|R }
\end{mathpar}

\begin{mathpar}
  \inferrule* [lab=alpha-equivalence] {} { (x)P \equiv (y)P\{y/x\} \and y \not\in \freenames{P} }
\end{mathpar}

\begin{definition}
Then two processes, $P,Q$, are alpha-equivalent if $P = Q\{\vec{y}/\vec{x}\}$ for
some $\vec{x} \in \boundnames{Q},\vec{y} \in \boundnames{P}$, where $Q\{\vec{y}/\vec{x}\}$
denotes the capture-avoiding substitution of $\vec{y}$ for $\vec{x}$ in $Q$.
\end{definition}

\begin{definition}
  The {\em structural congruence} \cite{SangiorgiWalker} , $\equiv$,
  between processes is the least congruence containing
  alpha-equivalence, satisfying the abelian monoid laws
  (associativity, commutativity and $\pzero$ as identity) for parallel
  composition $|$ and for summation $+$.
\end{definition}

\subsection{Name equivalence}

We take name equivalence, written $\nameeq$, to be the smallest
equivalence relation generated by the following rules.

\begin{mathpar}
\inferrule*[lab=Quote-drop]
{ }
{ \quotep{@{x}} \nameeq x }

\inferrule*[lab=Struct-equiv]
{ P \scong Q }
{ \quotep{P} \nameeq \quotep{Q} }
\end{mathpar}

The astute reader will have noticed that the mutual recursion of names
and processes imposes a mutual recursion on alpha-equivalence and
structural equivalence via name-equivalence. Fortunately, all of this
works out pleasantly and we may calculate in the natural way, free of
concern. The reader interested in the details is referred to the
appendix \ref{appendix:rho_details}.

\subsection{Substitution}

We use $\Proc$ for the set of processes, $\QProc$ for the set of
names, and $\id{\{}\vec{y} / \vec{x} \id{\}}$ to denote partial maps,
$s : \QProc \rightarrow \QProc$. A map, $s$ lifts, uniquely, to a map
on process terms, $\widehat{s} : \Proc \rightarrow \Proc$ by the
following equations.

\begin{mathpar}
  (0) \psubstp{Q}{P} := 0 \\
  (R \juxtap S) \psubstp{Q}{P}
  :=    
  (R)\psubstp{Q}{P} \juxtap (S) \psubstp{Q}{P} \\
  (x?(y).R) \psubstp{Q}{P}    
  :=    
  (x)\substp{Q}{P} (z)\concat( (R \psubstn{z}{y}) \psubstp{Q}{P} ) \\
  (\lift{x}{R}) \psubstp{Q}{P}  
  :=
  \lift{(x)\substp{Q}{P}}{ R \psubstp{Q}{P} } \\
%   (\dropn{x})  \psubstp{Q}{P}       
%   := 
%   \left\{ 
%     \begin{array}{ccc} 
%       \dropn{\quotep{Q}} & & x \nameeq \quotep{P} \\
%       \dropn{x} & & otherwise \\
%     \end{array}
%   \right. 
  (\dropn{x})  \psubstp{Q}{P}       
  := 
  \left\{ 
    \begin{array}{ccc} 
      Q & & x \nameeq \quotep{P} \\
      \dropn{x} & & otherwise \\
    \end{array}
  \right.
\end{mathpar}
 

where

\begin{eqnarray}
  (x)\id{\{} \lpquote Q \rpquote / \lpquote P \rpquote \id{\}}            = 
  \left\{ 
    \begin{array}{ccc}
      \lpquote Q \rpquote & & x \nameeq \lpquote P \rpquote \\
      x & & otherwise \\
    \end{array}
  \right. \nonumber
\end{eqnarray}

and $z$ is chosen distinct from $\quotep{P}$, $\quotep{Q}$, the free
names in $Q$, and all the names in $R$. Our $\alpha$-equivalence will
be built in the standard way from this substitution.

\begin{remark}\label{rem:no_self_referential_names}
  One consequence of these definitions is that $\forall P. \quotep{P}
  \not\in \freenames{P}$.
\end{remark}

\subsection{ Dynamic quote: an example }

Anticipating something of what's to come, consider applying the
substitution, $\widehat{\id{\{}u / z \id{\}}}$, to the following pair
of processes, $\lift{w}{y!(z)}$ and $w[ \lpquote y!(z) \rpquote ]$.

\begin{eqnarray}
	\lift{w}{y!(z)}\widehat{\id{\{}u / z \id{\}}}
		& = &
		\lift{w}{y!(u)} \nonumber\\
	w[ \lpquote y!(z) \rpquote ] \widehat{ \id{\{}u / z \id{\}} }
		& = &
		w[ \lpquote y!(z) \rpquote ] \nonumber
\end{eqnarray}

Because the body of the process between quotes is impervious to
substitution, we get radically different answers. In fact, by
examining the first process in an input context,
e.g. $x?(z).\lift{w}{y!(z)}$, we see that the process under the lift
operator may be shaped by prefixed inputs binding a name inside it. In
this sense, the lift operator will be seen as a way to dynamically
construct processes before reifying them as names.

Finally equipped with these standard features we can present the
dynamics of the calculus.

\subsubsection{Operational semantics} 

Finally, we introduce the computational dynamics. What marks these
algebras as distinct from other more traditionally studied algebraic
structures, e.g. vector spaces or polynomial rings, is the manner in
which dynamics is captured. In traditional structures, dynamics is typically
expressed through morphisms between such structures, as in linear maps
between vector spaces or morphisms between rings. In algebras
associated with the semantics of computation, the dynamics is
expressed as part of the algebraic structure itself, through a
reduction reduction relation typically denoted by $\red$. Below, we
give a recursive presentation of this relation for the calculus used
in the encoding.

$\red \subseteq \pi \times \pi$
$\red : \pi \to \mathcal{P}(\pi)$

\begin{mathpar}
  \inferrule* [lab=Comm] { \textsf{match}( x_{src}, x_{trgt} ) } { x_{trgt}?(y)P \; | \; x_{src}!\langle {Q} \rangle \red P\{\quotep{Q}/y}\} }
  \and \\
  \inferrule* [lab=Par] {{P} \red {P}'} {{{P} | {Q}} \red {{P}' | {Q}}}
  \and
  \inferrule* [lab=Equiv]{{{P} \scong {P}'} \andalso {{P}' \red {Q}'} \andalso {{Q}' \scong {Q}}}{{P} \red {Q}}
\end{mathpar}

\begin{eqnarray*}
  match_{\equiv} (\quotep{P},\quotep{Q}) & := & P \equiv Q \\
  match_{\dagger}(\quotep{P},\quotep{Q}) & := & \forall R. P|Q \red^{*} R => R \red^{*} 0 \\
  match_{K}(\quotep{P},\quotep{Q}) & := & K \mbox{ for some context } K
\end{eqnarray*}

$u?(x)P | u!\langle Q \rangle \red P\{\quotep{Q}/x\}$

%We write $\wred$ for $\red^*$, and $P\red$ if $\exists Q $ such that $ P \red Q$.
We write $P\red$ if $\exists Q $ such that $ P \red Q$ and $P\not\red$, otherwise.

\section{Replication}

As mentioned before, it is known that replication (and hence
recursion) can be implemented in a higher-order process algebra
\cite{SangiorgiWalker}. As our first example of calculation with the
machinery thus far presented we give the construction explicitly in
the {\rhoc}.

\begin{eqnarray}
	D_{x} & := & \prefix{x}{y}{(\binpar{\outputp{x}{y}}{@{y}})} \nonumber\\
	\bangp_{x}{P} & := & \binpar{{x}!\langle{\binpar{D_{x}}{P}}\rangle}{D_{x}} \nonumber
\end{eqnarray}

\begin{eqnarray}
	\bangp_{x}{P} & & \nonumber\\
	=
	& {x}!\langle{(\prefix{x}{y}{(\outputp{x}{y} | @{y})) | P}}\rangle 
	      | \prefix{x}{y}{(\outputp{x}{y} | @{y})} & \nonumber\\
	\red
	& (\outputp{x}{y} | @{y})\substn{\quotep{(\prefix{x}{y}{(@{y} | \outputp{x}{y})) | P}}}{y} & \nonumber\\
	=
	& \outputp{x}{\quotep{(\prefix{x}{y}{(\outputp{x}{y} | @{y})) | P}}}
	  | {(\prefix{x}{y}{(\outputp{x}{y} | @{y})) | P}} & \nonumber\\
	\red
	& \ldots & \nonumber\\
	\red^*
	& P | P | \ldots & \nonumber
\end{eqnarray}

Of course, this encoding, as an implementation, runs away, unfolding
$\bangp{P}$ eagerly. A lazier and more implementable replication
operator, restricted to input-guarded processes, may be obtained as follows.

\begin{eqnarray}
\bangp{\prefix{u}{v}{P}} 
	:= 
	\binpar{\lift{x}{\prefix{u}{v}{(\binpar{D(x)}{P})}}}{D(x)} \nonumber
\end{eqnarray}

\begin{remark}
  Note that the lazier definition still does not deal with summation
  or mixed summation (i.e. sums over input and output). The reader is
  invited to construct definitions of replication that deal with these
  features. 

  Further, the definitions are parameterized in a name, $x$. Can you,
  gentle reader, make a definition that eliminates this parameter and
  guarantees no accidental interaction between the replication
  machinery and the process being replicated -- i.e. no accidental
  sharing of names used by the process to get its work done and the
  name(s) used by the replication to effect copying. This latter
  revision of the definition of replication is crucial to obtaining
  the expected identity $!!P \sim !P$.
\end{remark}

\begin{remark}\label{rem:paradoxical_combinator}
  The reader familiar with the lambda calculus will have noticed the
  similarity between $D$ and the paradoxical combinator.

  [Ed. note: the existence of this seems to suggest we have to be more
  restrictive on the set of processes and names we admit if we are to
  support no-cloning.]
\end{remark}

\subsubsection{Bisimulation}

The computational dynamics gives rise to another kind of equivalence,
the equivalence of computational behavior. As previously mentioned
this is typically captured \emph{via} some form of bisimulation.

% The notion we use in this paper is weak barbed bisimulation
% \cite{milner91polyadicpi}.

The notion we use in this paper is derived from weak barbed
bisimulation \cite{milner91polyadicpi}. 

\begin{definition}
An \emph{observation relation}, $\downarrow_{\mathcal N}$, over a set
of names, $\mathcal N$, is the smallest relation satisfying the rules
below.

\infrule[Out-barb]{y \in {\mathcal N}, \; x \nameeq y}
		  {\outputp{x}{v} \downarrow_{\mathcal N} x}
\infrule[Par-barb]{\mbox{$P\downarrow_{\mathcal N} x$ or $Q\downarrow_{\mathcal N} x$}}
		  {\binpar{P}{Q} \downarrow_{\mathcal N} x}

We write $P \Downarrow_{\mathcal N} x$ if there is $Q$ such that 
$P \wred Q$ and $Q \downarrow_{\mathcal N} x$.
\end{definition}

\begin{definition}
%\label{def.bbisim}
An  ${\mathcal N}$-\emph{barbed bisimulation} over a set of names, ${\mathcal N}$, is a symmetric binary relation 
${\mathcal S}_{\mathcal N}$ between agents such that $P\rel{S}_{\mathcal N}Q$ implies:
\begin{enumerate}
\item If $P \red P'$ then $Q \wred Q'$ and $P'\rel{S}_{\mathcal N} Q'$.
\item If $P\downarrow_{\mathcal N} x$, then $Q\Downarrow_{\mathcal N} x$.
\end{enumerate}
$P$ is ${\mathcal N}$-barbed bisimilar to $Q$, written
$P \wbbisim_{\mathcal N} Q$, if $P \rel{S}_{\mathcal N} Q$ for some ${\mathcal N}$-barbed bisimulation ${\mathcal S}_{\mathcal N}$.
\end{definition}

$\mathcal{R} \subseteq \pi \times \pi$

$P \mathcal{R} Q => \forall P'. P \red P' \Rightarrow \exists Q'. Q \red Q', P' \mathcal{R} Q'$

$P \vdash x \Rightarrow Q \vdash x$

\begin{mathpar}
  \inferrule*[lab=Out-barb]{x \nameeq y}{{y}!\langle{Q}\rangle \vdash x}
  \and
  \inferrule*[lab=Par-barb]{\mbox{$P\vdash x$ or $Q\vdash x$}}{\binpar{P}{Q} \vdash x}
\end{mathpar}

\subsubsection{Contexts}

One of the principle advantages of computational calculi like the
$\pi$-calculus is a well-defined notion of context,
contextual-equivalence and a correlation between
contextual-equivalence and notions of bisimulation. The notion of
context allows the decomposition of a process into (sub-)process and
its syntactic environment, its context. Thus, a context may be
thought of as a process with a ``hole'' (written $\Box$) in it. The
application of a context $M$ to a process $P$, written $M[P]$, is
tantamount to filling the hole in $M$ with $P$. In this paper we do
not need the full weight of this theory, but do make use of the notion
of context in the proof the main theorem. 

\begin{mathpar}
  \inferrule* [lab=summation] {} {{M_{M},M_{N}} \bc \Box \;|\; x.M_{A} \;|\; M_{M}+M_{N}}
  \and
  \inferrule* [lab=agent] {} {{M_{A}} \bc (\vec{x})M_{P} \;| \; \clift{P_0,\ldots,M_{P},\ldots,P_N}}
  \and \\
  \inferrule* [lab=process] {} {{M_{P}} \bc M_{N} \;| \;P|M_{P} }
\end{mathpar} 

\begin{mathpar}
  \inferrule* [lab=sychronization] {} {M_{N} \bc \Box \;|\; x?M_{F} \;|\; x!M_{C}}
  \and
  \inferrule* [lab=abstraction] {} {{M_{F}} \bc (x)M_{P} }
  \and
  \inferrule* [lab=concretion] {} {{M_{C}} \bc \langle M_{P} \rangle }
  \and \\
  \inferrule* [lab=process] {} {{M_{P}} \bc M_{N} \;| \;P|M_{P} }
\end{mathpar}

\begin{definition}[contextual application] Given a context $M$, and
  process $P$, we define the \emph{contextual application}, $M[P] :=
  M\{P/\Box\}$. That is, the contextual application of M to P is the
  substitution of $P$ for $\Box$ in $M$.
\end{definition}

$\meaningof{-} : L \to \mathcal{P}(\pi)$

\begin{mathpar}
  \inferrule* [lab=collection] {} {\meaningof{true} = \pi, \and \meaningof{~E} = \pi \setminus \meaningof{E}, \and \meaningof{E_{1} \& E_{2}} = \meaningof{E_{1}} \cap \meaningof{E_{2}}}
\end{mathpar}

\begin{mathpar}
  \inferrule* [lab=structure] {} {\meaningof{0} = \{ P \in \pi | P \equiv 0 \}, \and \\ \meaningof{E_1 | E_2} = \{ P \in \pi | P \equiv P_{1} | P_{2}, P_{1} \in \meaningof{E_{1}}, P_{2} \in \meaningof{E_2}\} }
\end{mathpar}

\begin{mathpar}
 \inferrule* [lab=behavior] {} {\meaningof{\langle a?b \rangle E} = \{ P \in \pi | P \equiv Q | u?(y)P', \\ \and \\\\ \and \\ \;\;\; u \in \meaningof{a}, \forall z.P'\{z/y\} \in \meaningof{E\{z/b\}}\}, \and \\ \meaningof{a!E} = \{ P \in \pi | P \equiv Q | x!\langle P' \rangle, x \in \meaningof{a} P' \in \meaningof{E}\} }
\end{mathpar}

\begin{mathpar}
 \inferrule* [lab=nominal] {} {\meaningof{\quotep{E}} = \{ \quotep{P} \in \quotep{\pi} | P \in \meaningof{E} \}, \and \meaningof{\quotep{P}} = \{ \quotep{Q} \in \quotep{\pi} | P \equiv Q \} \and \\ \meaningof{@\quotep{E}} = \{ P \in \pi | P \equiv @x, x \in \meaningof{E} \}}
\end{mathpar}

\begin{eqnarray*}
  \\
  \meaningof{-} : TS \to ST
\end{eqnarray*}

\begin{eqnarray*}
  \\
  L : TS \to ST
\end{eqnarray*}

\begin{eqnarray*}
  \\
  P \models E \iff P \in \meaningof{E}
\end{eqnarray*}

\begin{eqnarray*}
  P \approx_{L} Q \iff \forall E \in L. P \models E \iff Q \models E
\end{eqnarray*}

\begin{eqnarray*}
  P \approx_{K} Q
\end{eqnarray*}

\begin{eqnarray*}
  P \approx Q
\end{eqnarray*}

$\approx_{K} = \approx = \approx_{L}$

\subsubsection{Contextual duality}

Note that contexts extend the quotation operation to a family of
operations from processes to names. Given a context, $M$, we can
define a \emph{nominal context}, $\quotep{M}$ by $\quotep{M}[P] :=
\quotep{M[P]}$. To foreshadow what is to come we observe that these
operations enjoy a duality with processes very much like the duality
between vectors and maps from vectors to scalars.

Further, because the calculus is essentially higher-order, we have a
correspondence between contexts and processes. More specifically,
given a name $x$ and a context $M$ we can construct $M^{*}_{x}$ such
that 

\begin{mathpar}
  M^{*}_{x} | \lift{x}{P} \red M[P]
\end{mathpar}

namely,

\begin{mathpar}
  M^{*}_{x} := x?(u).M[\dropn{u}]
\end{mathpar}

The dependence of $M^{*}_{x}$ on a name makes it an abstraction, 

\begin{mathpar}
  M^{*} := (x)x?(u).M[\dropn{u}]
\end{mathpar}

\subsection{Additional notation}

It will sometimes be convenient to denote the process a name
quotes. We already have the notation $x = \quotep{P}$, but it will be
convenient to introduce an alternate notation, $\procn{x}$, when we
want to emphasize the connection to the use of the name. Note that, by
virtue of name equivalence, $\quotep{\procn{x}} \nameeq x$; so, the
notation is consistent with previous definitions.

Further, because names have structure it is possible to effect
substitutions on the basis of that structure. This means we need to
upgrade our notation for substitutions, which we accomplish by
adapting comprehension notation. Thus,

\begin{mathpar}
  P\{ y / x : x \in S \}
\end{mathpar}

is interpreted to mean the process derived from P by replacing (in a
capture-avoiding manner) each occurrence of $x$ in $S$ by $y$. For example,

\begin{mathpar}
  P\{ \quotep{\procn{x}|\procn{x}} / x : x \in \freenames{P} \}
\end{mathpar}

will replace each (occurrence) of a free name $x$ in $P$ by
$\quotep{\procn{x}|\procn{x}}$.

Also, we will avail ourselves of the notation $x^{L}$ and $x^{R}$ to
denote injections of a name into disjoint copies of the name
space. There are numerous ways to accomplish this. One example can be
found in \cite{MeredithR05}. This notation overloads to vectors of
names: $\vec{x}^{\pi} := (x_{i}^{\pi} \; : \; 0 \leq i < |\vec{x}| )$ where $\pi \in \{L,R\}$.

We also use $P^{\Box} := P|\Box$.

In \cite{MeredithR05} an interpretation of the new operator is
given. It turns out that there are several possible interpretations
all enjoying the requisite algebraic properties of the operator (see
\cite{milner91polyadicpi}). We will therefore make liberal use of
$(\nu\; \vec{x})P$.

% subsection the_syntax_and_semantics_of_the_notation_system (end)   

\input{qm2pi.qmops} 

\input{qm2pi.sterngerlach} 

\input{qm2pi.metric} 

% section concurrent_process_calculi (end)

%\input{qm2pi.proofsketch}

% section proof sketch (end)

%\input{qm2pi.slviaknots} 

% section spatial logic via knots (end)

\input{qm2pi.conclusion}

% section conclusion (end)

%\input{qm2pi.dtcodes} 

% section wiring algorithm (end)

\input{qm2pi.ack} 

% section acknowledgments (end)

\newpage


\bibliographystyle{plain}   
\bibliography{../../biblios/main.bib}

\input{qm2pi.rhodetails}

\end{document}



\end{document}

 

\documentclass[12pt]{llncs}
%\documentclass{jktr}

\usepackage[pdftex]{hyperref}                   
\usepackage {listings}
\usepackage {mathpartir}
\usepackage{bcprules}
%\usepackage{listings}
                       
\usepackage{graphicx} 
%\usepackage[margins=2.5cm,nohead,nofoot]{geometry}
%\usepackage{geometry}
\usepackage{amsfonts}
\usepackage{amstext}
\usepackage{latexsym}
\usepackage{amssymb}
\usepackage{color}


%\include{myPreamble}
\documentclass[12pt]{llncs}
%\documentclass{jktr}

\usepackage[pdftex]{hyperref}                   
\usepackage {listings}
\usepackage {mathpartir}
\usepackage{bcprules}
%\usepackage{listings}
                       
\usepackage{graphicx} 
%\usepackage[margins=2.5cm,nohead,nofoot]{geometry}
%\usepackage{geometry}
\usepackage{amsfonts}
\usepackage{amstext}
\usepackage{latexsym}
\usepackage{amssymb}
\usepackage{color}


%\include{myPreamble}
\include{qm2pi.local} 

%\ifpdf
%\usepackage[pdftex]{graphicx}
%\else
%\usepackage{graphicx}
%\fi

 % \ifpdf
%  \usepackage{pdfsync}
%  \if


%\title{Brief Article}
%\author{David F. Snyder}
%\author{L.G. Meredith}

%\address{Dept. of Math., Texas State University--San Marcos, San Marcos, TX 78666}
       
\pagestyle{empty}


\begin{document}

\lstset{language=[Objective]Caml,frame=shadowbox}

\input{qm2pi.front}

% section front matter (end)

\input{qm2pi.intro} 
 
% section introduction (end)

% \input{qm2pi.knotations} 

% section notation (end)

\input{qm2pi.process.calculi} 

% section concurrent_process_calculi_and_spatial_logics_ (end)
    
%\input{qm2pi.knots2pi} 

%\input{qm2pi.trefoil} 

%\input{qm2pi.mainthm} 

% subsection basic_interpretation (end)

%\input{qm2pi.rho.presentation} 
\subsection{The syntax and semantics of the notation system}\label{sub:the_syntax_and_semantics_of_the_notation_system} % (fold)

We now summarize a technical presentation of the calculus that
embodies our theory of dynamics. The typical presentation of such a
calculus follows the style of giving generators and relations on
them. The grammar, below, describing term constructors, freely
generates the set of processes, $\Proc$. This set is then quotiented
by a relation known as structural congruence and it is over this set
that the notion of dynamics is expressed. This presentation is
essentially that of \cite{MeredithR05} with the addition of
polyadicity and summation. For readability we have relegated some of
the technical subtleties to an appendix.

\subsubsection{Process grammar}\label{subsub:process_grammar}

\begin{mathpar}
  \inferrule* [lab=synchronization] {} {{M} \bc \pzero \;|\; x?F \;|\; x!C }
  \and
  \inferrule* [lab=abstraction] {} {{F} \bc (x)P}
  \and
  \inferrule* [lab=concretion] {} {{C} \bc \langle Q \rangle}
  \and
  \inferrule* [lab=process] {} {{P,Q} \bc M \;| \;P|Q \;|\; @{x}}
  \and
  \inferrule* [lab=name] {} {{x} \bc \quotep{P}}
\end{mathpar} 

Note that $\vec{x}$ (resp. $\vec{P}$) denotes a vector of names
(resp. processes) of length $|\vec{x}|$ (resp. $|\vec{P}|$). We adopt
the following useful abbreviations.

\begin{mathpar}
   x?(\vec{y}).P := x.(\vec{y})P \and  x\clift{\vec{P}} := x.\clift{\vec{P}}
   \and x!(y) := \lift{x}{\dropn{y}}
   \and \Pi_{i=0}^{n-1}P_i := P_0 | \ldots | P_{n-1}
\end{mathpar}

\subsubsection{Structural congruence}

\paragraph{Free and bound names and alpha-equivalence.} At the
core of structural equivalence is alpha-equivalence which identifies
process that are the same up to a change of variable. Formally, we
recognize the distinction between free and bound names. The free names
of a process, $\freenames{P}$, may be calculated recursively as
follows:

\begin{mathpar}
\freenames{\pzero} := \emptyset
  \and \\
  \freenames{x?(y).P} := \{ x \} \cup (\freenames{P} \setminus \{ y \})
  \and 
  \freenames{x!\langle P \rangle} := \{ x \} \cup \{ P \} 
  \and \\
  \freenames{P|Q} := \freenames{P} \cup \freenames{Q}
  \and \\
  \freenames{@{x}} := \{ x \}
\end{mathpar}

$\pi$
$\quotep{\pi}$

$\freenames{-} : \pi \to \mathcal{P}(\quotep{\pi})$

\begin{eqnarray*}
  \freenames{\pzero} & := & \emptyset \\
  \freenames{x?(y).P} & := & \{ x \} \cup (\freenames{P} \setminus \{ y \}) \\
  \freenames{x!\langle P \rangle} & := & \{ x \} \cup \{ P \} \\
  \freenames{P|Q} & := & \freenames{P} \cup \freenames{Q} \\
  \freenames{\dropn{x}} & := & \{ x \}
\end{eqnarray*}

The bound names of a process, $\boundnames{P}$, are those names occurring in $P$
that are not free. For example, in $x?(y).0$, the name $x$ is free, while $y$ is bound.

\begin{mathpar}
  \inferrule* [lab=monoidal-laws] {} { P|Q \equiv Q|P \and P|0 \equiv P \and P|(Q|R) \equiv (P|Q)|R }
\end{mathpar}

\begin{mathpar}
  \inferrule* [lab=alpha-equivalence] {} { (x)P \equiv (y)P\{y/x\} \and y \not\in \freenames{P} }
\end{mathpar}

\begin{definition}
Then two processes, $P,Q$, are alpha-equivalent if $P = Q\{\vec{y}/\vec{x}\}$ for
some $\vec{x} \in \boundnames{Q},\vec{y} \in \boundnames{P}$, where $Q\{\vec{y}/\vec{x}\}$
denotes the capture-avoiding substitution of $\vec{y}$ for $\vec{x}$ in $Q$.
\end{definition}

\begin{definition}
  The {\em structural congruence} \cite{SangiorgiWalker} , $\equiv$,
  between processes is the least congruence containing
  alpha-equivalence, satisfying the abelian monoid laws
  (associativity, commutativity and $\pzero$ as identity) for parallel
  composition $|$ and for summation $+$.
\end{definition}

\subsection{Name equivalence}

We take name equivalence, written $\nameeq$, to be the smallest
equivalence relation generated by the following rules.

\begin{mathpar}
\inferrule*[lab=Quote-drop]
{ }
{ \quotep{@{x}} \nameeq x }

\inferrule*[lab=Struct-equiv]
{ P \scong Q }
{ \quotep{P} \nameeq \quotep{Q} }
\end{mathpar}

The astute reader will have noticed that the mutual recursion of names
and processes imposes a mutual recursion on alpha-equivalence and
structural equivalence via name-equivalence. Fortunately, all of this
works out pleasantly and we may calculate in the natural way, free of
concern. The reader interested in the details is referred to the
appendix \ref{appendix:rho_details}.

\subsection{Substitution}

We use $\Proc$ for the set of processes, $\QProc$ for the set of
names, and $\id{\{}\vec{y} / \vec{x} \id{\}}$ to denote partial maps,
$s : \QProc \rightarrow \QProc$. A map, $s$ lifts, uniquely, to a map
on process terms, $\widehat{s} : \Proc \rightarrow \Proc$ by the
following equations.

\begin{mathpar}
  (0) \psubstp{Q}{P} := 0 \\
  (R \juxtap S) \psubstp{Q}{P}
  :=    
  (R)\psubstp{Q}{P} \juxtap (S) \psubstp{Q}{P} \\
  (x?(y).R) \psubstp{Q}{P}    
  :=    
  (x)\substp{Q}{P} (z)\concat( (R \psubstn{z}{y}) \psubstp{Q}{P} ) \\
  (\lift{x}{R}) \psubstp{Q}{P}  
  :=
  \lift{(x)\substp{Q}{P}}{ R \psubstp{Q}{P} } \\
%   (\dropn{x})  \psubstp{Q}{P}       
%   := 
%   \left\{ 
%     \begin{array}{ccc} 
%       \dropn{\quotep{Q}} & & x \nameeq \quotep{P} \\
%       \dropn{x} & & otherwise \\
%     \end{array}
%   \right. 
  (\dropn{x})  \psubstp{Q}{P}       
  := 
  \left\{ 
    \begin{array}{ccc} 
      Q & & x \nameeq \quotep{P} \\
      \dropn{x} & & otherwise \\
    \end{array}
  \right.
\end{mathpar}
 

where

\begin{eqnarray}
  (x)\id{\{} \lpquote Q \rpquote / \lpquote P \rpquote \id{\}}            = 
  \left\{ 
    \begin{array}{ccc}
      \lpquote Q \rpquote & & x \nameeq \lpquote P \rpquote \\
      x & & otherwise \\
    \end{array}
  \right. \nonumber
\end{eqnarray}

and $z$ is chosen distinct from $\quotep{P}$, $\quotep{Q}$, the free
names in $Q$, and all the names in $R$. Our $\alpha$-equivalence will
be built in the standard way from this substitution.

\begin{remark}\label{rem:no_self_referential_names}
  One consequence of these definitions is that $\forall P. \quotep{P}
  \not\in \freenames{P}$.
\end{remark}

\subsection{ Dynamic quote: an example }

Anticipating something of what's to come, consider applying the
substitution, $\widehat{\id{\{}u / z \id{\}}}$, to the following pair
of processes, $\lift{w}{y!(z)}$ and $w[ \lpquote y!(z) \rpquote ]$.

\begin{eqnarray}
	\lift{w}{y!(z)}\widehat{\id{\{}u / z \id{\}}}
		& = &
		\lift{w}{y!(u)} \nonumber\\
	w[ \lpquote y!(z) \rpquote ] \widehat{ \id{\{}u / z \id{\}} }
		& = &
		w[ \lpquote y!(z) \rpquote ] \nonumber
\end{eqnarray}

Because the body of the process between quotes is impervious to
substitution, we get radically different answers. In fact, by
examining the first process in an input context,
e.g. $x?(z).\lift{w}{y!(z)}$, we see that the process under the lift
operator may be shaped by prefixed inputs binding a name inside it. In
this sense, the lift operator will be seen as a way to dynamically
construct processes before reifying them as names.

Finally equipped with these standard features we can present the
dynamics of the calculus.

\subsubsection{Operational semantics} 

Finally, we introduce the computational dynamics. What marks these
algebras as distinct from other more traditionally studied algebraic
structures, e.g. vector spaces or polynomial rings, is the manner in
which dynamics is captured. In traditional structures, dynamics is typically
expressed through morphisms between such structures, as in linear maps
between vector spaces or morphisms between rings. In algebras
associated with the semantics of computation, the dynamics is
expressed as part of the algebraic structure itself, through a
reduction reduction relation typically denoted by $\red$. Below, we
give a recursive presentation of this relation for the calculus used
in the encoding.

$\red \subseteq \pi \times \pi$
$\red : \pi \to \mathcal{P}(\pi)$

\begin{mathpar}
  \inferrule* [lab=Comm] { \textsf{match}( x_{src}, x_{trgt} ) } { x_{trgt}?(y)P \; | \; x_{src}!\langle {Q} \rangle \red P\{\quotep{Q}/y}\} }
  \and \\
  \inferrule* [lab=Par] {{P} \red {P}'} {{{P} | {Q}} \red {{P}' | {Q}}}
  \and
  \inferrule* [lab=Equiv]{{{P} \scong {P}'} \andalso {{P}' \red {Q}'} \andalso {{Q}' \scong {Q}}}{{P} \red {Q}}
\end{mathpar}

\begin{eqnarray*}
  match_{\equiv} (\quotep{P},\quotep{Q}) & := & P \equiv Q \\
  match_{\dagger}(\quotep{P},\quotep{Q}) & := & \forall R. P|Q \red^{*} R => R \red^{*} 0 \\
  match_{K}(\quotep{P},\quotep{Q}) & := & K \mbox{ for some context } K
\end{eqnarray*}

$u?(x)P | u!\langle Q \rangle \red P\{\quotep{Q}/x\}$

%We write $\wred$ for $\red^*$, and $P\red$ if $\exists Q $ such that $ P \red Q$.
We write $P\red$ if $\exists Q $ such that $ P \red Q$ and $P\not\red$, otherwise.

\section{Replication}

As mentioned before, it is known that replication (and hence
recursion) can be implemented in a higher-order process algebra
\cite{SangiorgiWalker}. As our first example of calculation with the
machinery thus far presented we give the construction explicitly in
the {\rhoc}.

\begin{eqnarray}
	D_{x} & := & \prefix{x}{y}{(\binpar{\outputp{x}{y}}{@{y}})} \nonumber\\
	\bangp_{x}{P} & := & \binpar{{x}!\langle{\binpar{D_{x}}{P}}\rangle}{D_{x}} \nonumber
\end{eqnarray}

\begin{eqnarray}
	\bangp_{x}{P} & & \nonumber\\
	=
	& {x}!\langle{(\prefix{x}{y}{(\outputp{x}{y} | @{y})) | P}}\rangle 
	      | \prefix{x}{y}{(\outputp{x}{y} | @{y})} & \nonumber\\
	\red
	& (\outputp{x}{y} | @{y})\substn{\quotep{(\prefix{x}{y}{(@{y} | \outputp{x}{y})) | P}}}{y} & \nonumber\\
	=
	& \outputp{x}{\quotep{(\prefix{x}{y}{(\outputp{x}{y} | @{y})) | P}}}
	  | {(\prefix{x}{y}{(\outputp{x}{y} | @{y})) | P}} & \nonumber\\
	\red
	& \ldots & \nonumber\\
	\red^*
	& P | P | \ldots & \nonumber
\end{eqnarray}

Of course, this encoding, as an implementation, runs away, unfolding
$\bangp{P}$ eagerly. A lazier and more implementable replication
operator, restricted to input-guarded processes, may be obtained as follows.

\begin{eqnarray}
\bangp{\prefix{u}{v}{P}} 
	:= 
	\binpar{\lift{x}{\prefix{u}{v}{(\binpar{D(x)}{P})}}}{D(x)} \nonumber
\end{eqnarray}

\begin{remark}
  Note that the lazier definition still does not deal with summation
  or mixed summation (i.e. sums over input and output). The reader is
  invited to construct definitions of replication that deal with these
  features. 

  Further, the definitions are parameterized in a name, $x$. Can you,
  gentle reader, make a definition that eliminates this parameter and
  guarantees no accidental interaction between the replication
  machinery and the process being replicated -- i.e. no accidental
  sharing of names used by the process to get its work done and the
  name(s) used by the replication to effect copying. This latter
  revision of the definition of replication is crucial to obtaining
  the expected identity $!!P \sim !P$.
\end{remark}

\begin{remark}\label{rem:paradoxical_combinator}
  The reader familiar with the lambda calculus will have noticed the
  similarity between $D$ and the paradoxical combinator.

  [Ed. note: the existence of this seems to suggest we have to be more
  restrictive on the set of processes and names we admit if we are to
  support no-cloning.]
\end{remark}

\subsubsection{Bisimulation}

The computational dynamics gives rise to another kind of equivalence,
the equivalence of computational behavior. As previously mentioned
this is typically captured \emph{via} some form of bisimulation.

% The notion we use in this paper is weak barbed bisimulation
% \cite{milner91polyadicpi}.

The notion we use in this paper is derived from weak barbed
bisimulation \cite{milner91polyadicpi}. 

\begin{definition}
An \emph{observation relation}, $\downarrow_{\mathcal N}$, over a set
of names, $\mathcal N$, is the smallest relation satisfying the rules
below.

\infrule[Out-barb]{y \in {\mathcal N}, \; x \nameeq y}
		  {\outputp{x}{v} \downarrow_{\mathcal N} x}
\infrule[Par-barb]{\mbox{$P\downarrow_{\mathcal N} x$ or $Q\downarrow_{\mathcal N} x$}}
		  {\binpar{P}{Q} \downarrow_{\mathcal N} x}

We write $P \Downarrow_{\mathcal N} x$ if there is $Q$ such that 
$P \wred Q$ and $Q \downarrow_{\mathcal N} x$.
\end{definition}

\begin{definition}
%\label{def.bbisim}
An  ${\mathcal N}$-\emph{barbed bisimulation} over a set of names, ${\mathcal N}$, is a symmetric binary relation 
${\mathcal S}_{\mathcal N}$ between agents such that $P\rel{S}_{\mathcal N}Q$ implies:
\begin{enumerate}
\item If $P \red P'$ then $Q \wred Q'$ and $P'\rel{S}_{\mathcal N} Q'$.
\item If $P\downarrow_{\mathcal N} x$, then $Q\Downarrow_{\mathcal N} x$.
\end{enumerate}
$P$ is ${\mathcal N}$-barbed bisimilar to $Q$, written
$P \wbbisim_{\mathcal N} Q$, if $P \rel{S}_{\mathcal N} Q$ for some ${\mathcal N}$-barbed bisimulation ${\mathcal S}_{\mathcal N}$.
\end{definition}

$\mathcal{R} \subseteq \pi \times \pi$

$P \mathcal{R} Q => \forall P'. P \red P' \Rightarrow \exists Q'. Q \red Q', P' \mathcal{R} Q'$

$P \vdash x \Rightarrow Q \vdash x$

\begin{mathpar}
  \inferrule*[lab=Out-barb]{x \nameeq y}{{y}!\langle{Q}\rangle \vdash x}
  \and
  \inferrule*[lab=Par-barb]{\mbox{$P\vdash x$ or $Q\vdash x$}}{\binpar{P}{Q} \vdash x}
\end{mathpar}

\subsubsection{Contexts}

One of the principle advantages of computational calculi like the
$\pi$-calculus is a well-defined notion of context,
contextual-equivalence and a correlation between
contextual-equivalence and notions of bisimulation. The notion of
context allows the decomposition of a process into (sub-)process and
its syntactic environment, its context. Thus, a context may be
thought of as a process with a ``hole'' (written $\Box$) in it. The
application of a context $M$ to a process $P$, written $M[P]$, is
tantamount to filling the hole in $M$ with $P$. In this paper we do
not need the full weight of this theory, but do make use of the notion
of context in the proof the main theorem. 

\begin{mathpar}
  \inferrule* [lab=summation] {} {{M_{M},M_{N}} \bc \Box \;|\; x.M_{A} \;|\; M_{M}+M_{N}}
  \and
  \inferrule* [lab=agent] {} {{M_{A}} \bc (\vec{x})M_{P} \;| \; \clift{P_0,\ldots,M_{P},\ldots,P_N}}
  \and \\
  \inferrule* [lab=process] {} {{M_{P}} \bc M_{N} \;| \;P|M_{P} }
\end{mathpar} 

\begin{mathpar}
  \inferrule* [lab=sychronization] {} {M_{N} \bc \Box \;|\; x?M_{F} \;|\; x!M_{C}}
  \and
  \inferrule* [lab=abstraction] {} {{M_{F}} \bc (x)M_{P} }
  \and
  \inferrule* [lab=concretion] {} {{M_{C}} \bc \langle M_{P} \rangle }
  \and \\
  \inferrule* [lab=process] {} {{M_{P}} \bc M_{N} \;| \;P|M_{P} }
\end{mathpar}

\begin{definition}[contextual application] Given a context $M$, and
  process $P$, we define the \emph{contextual application}, $M[P] :=
  M\{P/\Box\}$. That is, the contextual application of M to P is the
  substitution of $P$ for $\Box$ in $M$.
\end{definition}

$\meaningof{-} : L \to \mathcal{P}(\pi)$

\begin{mathpar}
  \inferrule* [lab=collection] {} {\meaningof{true} = \pi, \and \meaningof{~E} = \pi \setminus \meaningof{E}, \and \meaningof{E_{1} \& E_{2}} = \meaningof{E_{1}} \cap \meaningof{E_{2}}}
\end{mathpar}

\begin{mathpar}
  \inferrule* [lab=structure] {} {\meaningof{0} = \{ P \in \pi | P \equiv 0 \}, \and \\ \meaningof{E_1 | E_2} = \{ P \in \pi | P \equiv P_{1} | P_{2}, P_{1} \in \meaningof{E_{1}}, P_{2} \in \meaningof{E_2}\} }
\end{mathpar}

\begin{mathpar}
 \inferrule* [lab=behavior] {} {\meaningof{\langle a?b \rangle E} = \{ P \in \pi | P \equiv Q | u?(y)P', \\ \and \\\\ \and \\ \;\;\; u \in \meaningof{a}, \forall z.P'\{z/y\} \in \meaningof{E\{z/b\}}\}, \and \\ \meaningof{a!E} = \{ P \in \pi | P \equiv Q | x!\langle P' \rangle, x \in \meaningof{a} P' \in \meaningof{E}\} }
\end{mathpar}

\begin{mathpar}
 \inferrule* [lab=nominal] {} {\meaningof{\quotep{E}} = \{ \quotep{P} \in \quotep{\pi} | P \in \meaningof{E} \}, \and \meaningof{\quotep{P}} = \{ \quotep{Q} \in \quotep{\pi} | P \equiv Q \} \and \\ \meaningof{@\quotep{E}} = \{ P \in \pi | P \equiv @x, x \in \meaningof{E} \}}
\end{mathpar}

\begin{eqnarray*}
  \\
  \meaningof{-} : TS \to ST
\end{eqnarray*}

\begin{eqnarray*}
  \\
  L : TS \to ST
\end{eqnarray*}

\begin{eqnarray*}
  \\
  P \models E \iff P \in \meaningof{E}
\end{eqnarray*}

\begin{eqnarray*}
  P \approx_{L} Q \iff \forall E \in L. P \models E \iff Q \models E
\end{eqnarray*}

\begin{eqnarray*}
  P \approx_{K} Q
\end{eqnarray*}

\begin{eqnarray*}
  P \approx Q
\end{eqnarray*}

$\approx_{K} = \approx = \approx_{L}$

\subsubsection{Contextual duality}

Note that contexts extend the quotation operation to a family of
operations from processes to names. Given a context, $M$, we can
define a \emph{nominal context}, $\quotep{M}$ by $\quotep{M}[P] :=
\quotep{M[P]}$. To foreshadow what is to come we observe that these
operations enjoy a duality with processes very much like the duality
between vectors and maps from vectors to scalars.

Further, because the calculus is essentially higher-order, we have a
correspondence between contexts and processes. More specifically,
given a name $x$ and a context $M$ we can construct $M^{*}_{x}$ such
that 

\begin{mathpar}
  M^{*}_{x} | \lift{x}{P} \red M[P]
\end{mathpar}

namely,

\begin{mathpar}
  M^{*}_{x} := x?(u).M[\dropn{u}]
\end{mathpar}

The dependence of $M^{*}_{x}$ on a name makes it an abstraction, 

\begin{mathpar}
  M^{*} := (x)x?(u).M[\dropn{u}]
\end{mathpar}

\subsection{Additional notation}

It will sometimes be convenient to denote the process a name
quotes. We already have the notation $x = \quotep{P}$, but it will be
convenient to introduce an alternate notation, $\procn{x}$, when we
want to emphasize the connection to the use of the name. Note that, by
virtue of name equivalence, $\quotep{\procn{x}} \nameeq x$; so, the
notation is consistent with previous definitions.

Further, because names have structure it is possible to effect
substitutions on the basis of that structure. This means we need to
upgrade our notation for substitutions, which we accomplish by
adapting comprehension notation. Thus,

\begin{mathpar}
  P\{ y / x : x \in S \}
\end{mathpar}

is interpreted to mean the process derived from P by replacing (in a
capture-avoiding manner) each occurrence of $x$ in $S$ by $y$. For example,

\begin{mathpar}
  P\{ \quotep{\procn{x}|\procn{x}} / x : x \in \freenames{P} \}
\end{mathpar}

will replace each (occurrence) of a free name $x$ in $P$ by
$\quotep{\procn{x}|\procn{x}}$.

Also, we will avail ourselves of the notation $x^{L}$ and $x^{R}$ to
denote injections of a name into disjoint copies of the name
space. There are numerous ways to accomplish this. One example can be
found in \cite{MeredithR05}. This notation overloads to vectors of
names: $\vec{x}^{\pi} := (x_{i}^{\pi} \; : \; 0 \leq i < |\vec{x}| )$ where $\pi \in \{L,R\}$.

We also use $P^{\Box} := P|\Box$.

In \cite{MeredithR05} an interpretation of the new operator is
given. It turns out that there are several possible interpretations
all enjoying the requisite algebraic properties of the operator (see
\cite{milner91polyadicpi}). We will therefore make liberal use of
$(\nu\; \vec{x})P$.

% subsection the_syntax_and_semantics_of_the_notation_system (end)   

\input{qm2pi.qmops} 

\input{qm2pi.sterngerlach} 

\input{qm2pi.metric} 

% section concurrent_process_calculi (end)

%\input{qm2pi.proofsketch}

% section proof sketch (end)

%\input{qm2pi.slviaknots} 

% section spatial logic via knots (end)

\input{qm2pi.conclusion}

% section conclusion (end)

%\input{qm2pi.dtcodes} 

% section wiring algorithm (end)

\input{qm2pi.ack} 

% section acknowledgments (end)

\newpage


\bibliographystyle{plain}   
\bibliography{../../biblios/main.bib}

\input{qm2pi.rhodetails}

\end{document}

 

%\ifpdf
%\usepackage[pdftex]{graphicx}
%\else
%\usepackage{graphicx}
%\fi

 % \ifpdf
%  \usepackage{pdfsync}
%  \if


%\title{Brief Article}
%\author{David F. Snyder}
%\author{L.G. Meredith}

%\address{Dept. of Math., Texas State University--San Marcos, San Marcos, TX 78666}
       
\pagestyle{empty}


\begin{document}

\lstset{language=[Objective]Caml,frame=shadowbox}

\documentclass[12pt]{llncs}
%\documentclass{jktr}

\usepackage[pdftex]{hyperref}                   
\usepackage {listings}
\usepackage {mathpartir}
\usepackage{bcprules}
%\usepackage{listings}
                       
\usepackage{graphicx} 
%\usepackage[margins=2.5cm,nohead,nofoot]{geometry}
%\usepackage{geometry}
\usepackage{amsfonts}
\usepackage{amstext}
\usepackage{latexsym}
\usepackage{amssymb}
\usepackage{color}


%\include{myPreamble}
\include{qm2pi.local} 

%\ifpdf
%\usepackage[pdftex]{graphicx}
%\else
%\usepackage{graphicx}
%\fi

 % \ifpdf
%  \usepackage{pdfsync}
%  \if


%\title{Brief Article}
%\author{David F. Snyder}
%\author{L.G. Meredith}

%\address{Dept. of Math., Texas State University--San Marcos, San Marcos, TX 78666}
       
\pagestyle{empty}


\begin{document}

\lstset{language=[Objective]Caml,frame=shadowbox}

\input{qm2pi.front}

% section front matter (end)

\input{qm2pi.intro} 
 
% section introduction (end)

% \input{qm2pi.knotations} 

% section notation (end)

\input{qm2pi.process.calculi} 

% section concurrent_process_calculi_and_spatial_logics_ (end)
    
%\input{qm2pi.knots2pi} 

%\input{qm2pi.trefoil} 

%\input{qm2pi.mainthm} 

% subsection basic_interpretation (end)

%\input{qm2pi.rho.presentation} 
\subsection{The syntax and semantics of the notation system}\label{sub:the_syntax_and_semantics_of_the_notation_system} % (fold)

We now summarize a technical presentation of the calculus that
embodies our theory of dynamics. The typical presentation of such a
calculus follows the style of giving generators and relations on
them. The grammar, below, describing term constructors, freely
generates the set of processes, $\Proc$. This set is then quotiented
by a relation known as structural congruence and it is over this set
that the notion of dynamics is expressed. This presentation is
essentially that of \cite{MeredithR05} with the addition of
polyadicity and summation. For readability we have relegated some of
the technical subtleties to an appendix.

\subsubsection{Process grammar}\label{subsub:process_grammar}

\begin{mathpar}
  \inferrule* [lab=synchronization] {} {{M} \bc \pzero \;|\; x?F \;|\; x!C }
  \and
  \inferrule* [lab=abstraction] {} {{F} \bc (x)P}
  \and
  \inferrule* [lab=concretion] {} {{C} \bc \langle Q \rangle}
  \and
  \inferrule* [lab=process] {} {{P,Q} \bc M \;| \;P|Q \;|\; @{x}}
  \and
  \inferrule* [lab=name] {} {{x} \bc \quotep{P}}
\end{mathpar} 

Note that $\vec{x}$ (resp. $\vec{P}$) denotes a vector of names
(resp. processes) of length $|\vec{x}|$ (resp. $|\vec{P}|$). We adopt
the following useful abbreviations.

\begin{mathpar}
   x?(\vec{y}).P := x.(\vec{y})P \and  x\clift{\vec{P}} := x.\clift{\vec{P}}
   \and x!(y) := \lift{x}{\dropn{y}}
   \and \Pi_{i=0}^{n-1}P_i := P_0 | \ldots | P_{n-1}
\end{mathpar}

\subsubsection{Structural congruence}

\paragraph{Free and bound names and alpha-equivalence.} At the
core of structural equivalence is alpha-equivalence which identifies
process that are the same up to a change of variable. Formally, we
recognize the distinction between free and bound names. The free names
of a process, $\freenames{P}$, may be calculated recursively as
follows:

\begin{mathpar}
\freenames{\pzero} := \emptyset
  \and \\
  \freenames{x?(y).P} := \{ x \} \cup (\freenames{P} \setminus \{ y \})
  \and 
  \freenames{x!\langle P \rangle} := \{ x \} \cup \{ P \} 
  \and \\
  \freenames{P|Q} := \freenames{P} \cup \freenames{Q}
  \and \\
  \freenames{@{x}} := \{ x \}
\end{mathpar}

$\pi$
$\quotep{\pi}$

$\freenames{-} : \pi \to \mathcal{P}(\quotep{\pi})$

\begin{eqnarray*}
  \freenames{\pzero} & := & \emptyset \\
  \freenames{x?(y).P} & := & \{ x \} \cup (\freenames{P} \setminus \{ y \}) \\
  \freenames{x!\langle P \rangle} & := & \{ x \} \cup \{ P \} \\
  \freenames{P|Q} & := & \freenames{P} \cup \freenames{Q} \\
  \freenames{\dropn{x}} & := & \{ x \}
\end{eqnarray*}

The bound names of a process, $\boundnames{P}$, are those names occurring in $P$
that are not free. For example, in $x?(y).0$, the name $x$ is free, while $y$ is bound.

\begin{mathpar}
  \inferrule* [lab=monoidal-laws] {} { P|Q \equiv Q|P \and P|0 \equiv P \and P|(Q|R) \equiv (P|Q)|R }
\end{mathpar}

\begin{mathpar}
  \inferrule* [lab=alpha-equivalence] {} { (x)P \equiv (y)P\{y/x\} \and y \not\in \freenames{P} }
\end{mathpar}

\begin{definition}
Then two processes, $P,Q$, are alpha-equivalent if $P = Q\{\vec{y}/\vec{x}\}$ for
some $\vec{x} \in \boundnames{Q},\vec{y} \in \boundnames{P}$, where $Q\{\vec{y}/\vec{x}\}$
denotes the capture-avoiding substitution of $\vec{y}$ for $\vec{x}$ in $Q$.
\end{definition}

\begin{definition}
  The {\em structural congruence} \cite{SangiorgiWalker} , $\equiv$,
  between processes is the least congruence containing
  alpha-equivalence, satisfying the abelian monoid laws
  (associativity, commutativity and $\pzero$ as identity) for parallel
  composition $|$ and for summation $+$.
\end{definition}

\subsection{Name equivalence}

We take name equivalence, written $\nameeq$, to be the smallest
equivalence relation generated by the following rules.

\begin{mathpar}
\inferrule*[lab=Quote-drop]
{ }
{ \quotep{@{x}} \nameeq x }

\inferrule*[lab=Struct-equiv]
{ P \scong Q }
{ \quotep{P} \nameeq \quotep{Q} }
\end{mathpar}

The astute reader will have noticed that the mutual recursion of names
and processes imposes a mutual recursion on alpha-equivalence and
structural equivalence via name-equivalence. Fortunately, all of this
works out pleasantly and we may calculate in the natural way, free of
concern. The reader interested in the details is referred to the
appendix \ref{appendix:rho_details}.

\subsection{Substitution}

We use $\Proc$ for the set of processes, $\QProc$ for the set of
names, and $\id{\{}\vec{y} / \vec{x} \id{\}}$ to denote partial maps,
$s : \QProc \rightarrow \QProc$. A map, $s$ lifts, uniquely, to a map
on process terms, $\widehat{s} : \Proc \rightarrow \Proc$ by the
following equations.

\begin{mathpar}
  (0) \psubstp{Q}{P} := 0 \\
  (R \juxtap S) \psubstp{Q}{P}
  :=    
  (R)\psubstp{Q}{P} \juxtap (S) \psubstp{Q}{P} \\
  (x?(y).R) \psubstp{Q}{P}    
  :=    
  (x)\substp{Q}{P} (z)\concat( (R \psubstn{z}{y}) \psubstp{Q}{P} ) \\
  (\lift{x}{R}) \psubstp{Q}{P}  
  :=
  \lift{(x)\substp{Q}{P}}{ R \psubstp{Q}{P} } \\
%   (\dropn{x})  \psubstp{Q}{P}       
%   := 
%   \left\{ 
%     \begin{array}{ccc} 
%       \dropn{\quotep{Q}} & & x \nameeq \quotep{P} \\
%       \dropn{x} & & otherwise \\
%     \end{array}
%   \right. 
  (\dropn{x})  \psubstp{Q}{P}       
  := 
  \left\{ 
    \begin{array}{ccc} 
      Q & & x \nameeq \quotep{P} \\
      \dropn{x} & & otherwise \\
    \end{array}
  \right.
\end{mathpar}
 

where

\begin{eqnarray}
  (x)\id{\{} \lpquote Q \rpquote / \lpquote P \rpquote \id{\}}            = 
  \left\{ 
    \begin{array}{ccc}
      \lpquote Q \rpquote & & x \nameeq \lpquote P \rpquote \\
      x & & otherwise \\
    \end{array}
  \right. \nonumber
\end{eqnarray}

and $z$ is chosen distinct from $\quotep{P}$, $\quotep{Q}$, the free
names in $Q$, and all the names in $R$. Our $\alpha$-equivalence will
be built in the standard way from this substitution.

\begin{remark}\label{rem:no_self_referential_names}
  One consequence of these definitions is that $\forall P. \quotep{P}
  \not\in \freenames{P}$.
\end{remark}

\subsection{ Dynamic quote: an example }

Anticipating something of what's to come, consider applying the
substitution, $\widehat{\id{\{}u / z \id{\}}}$, to the following pair
of processes, $\lift{w}{y!(z)}$ and $w[ \lpquote y!(z) \rpquote ]$.

\begin{eqnarray}
	\lift{w}{y!(z)}\widehat{\id{\{}u / z \id{\}}}
		& = &
		\lift{w}{y!(u)} \nonumber\\
	w[ \lpquote y!(z) \rpquote ] \widehat{ \id{\{}u / z \id{\}} }
		& = &
		w[ \lpquote y!(z) \rpquote ] \nonumber
\end{eqnarray}

Because the body of the process between quotes is impervious to
substitution, we get radically different answers. In fact, by
examining the first process in an input context,
e.g. $x?(z).\lift{w}{y!(z)}$, we see that the process under the lift
operator may be shaped by prefixed inputs binding a name inside it. In
this sense, the lift operator will be seen as a way to dynamically
construct processes before reifying them as names.

Finally equipped with these standard features we can present the
dynamics of the calculus.

\subsubsection{Operational semantics} 

Finally, we introduce the computational dynamics. What marks these
algebras as distinct from other more traditionally studied algebraic
structures, e.g. vector spaces or polynomial rings, is the manner in
which dynamics is captured. In traditional structures, dynamics is typically
expressed through morphisms between such structures, as in linear maps
between vector spaces or morphisms between rings. In algebras
associated with the semantics of computation, the dynamics is
expressed as part of the algebraic structure itself, through a
reduction reduction relation typically denoted by $\red$. Below, we
give a recursive presentation of this relation for the calculus used
in the encoding.

$\red \subseteq \pi \times \pi$
$\red : \pi \to \mathcal{P}(\pi)$

\begin{mathpar}
  \inferrule* [lab=Comm] { \textsf{match}( x_{src}, x_{trgt} ) } { x_{trgt}?(y)P \; | \; x_{src}!\langle {Q} \rangle \red P\{\quotep{Q}/y}\} }
  \and \\
  \inferrule* [lab=Par] {{P} \red {P}'} {{{P} | {Q}} \red {{P}' | {Q}}}
  \and
  \inferrule* [lab=Equiv]{{{P} \scong {P}'} \andalso {{P}' \red {Q}'} \andalso {{Q}' \scong {Q}}}{{P} \red {Q}}
\end{mathpar}

\begin{eqnarray*}
  match_{\equiv} (\quotep{P},\quotep{Q}) & := & P \equiv Q \\
  match_{\dagger}(\quotep{P},\quotep{Q}) & := & \forall R. P|Q \red^{*} R => R \red^{*} 0 \\
  match_{K}(\quotep{P},\quotep{Q}) & := & K \mbox{ for some context } K
\end{eqnarray*}

$u?(x)P | u!\langle Q \rangle \red P\{\quotep{Q}/x\}$

%We write $\wred$ for $\red^*$, and $P\red$ if $\exists Q $ such that $ P \red Q$.
We write $P\red$ if $\exists Q $ such that $ P \red Q$ and $P\not\red$, otherwise.

\section{Replication}

As mentioned before, it is known that replication (and hence
recursion) can be implemented in a higher-order process algebra
\cite{SangiorgiWalker}. As our first example of calculation with the
machinery thus far presented we give the construction explicitly in
the {\rhoc}.

\begin{eqnarray}
	D_{x} & := & \prefix{x}{y}{(\binpar{\outputp{x}{y}}{@{y}})} \nonumber\\
	\bangp_{x}{P} & := & \binpar{{x}!\langle{\binpar{D_{x}}{P}}\rangle}{D_{x}} \nonumber
\end{eqnarray}

\begin{eqnarray}
	\bangp_{x}{P} & & \nonumber\\
	=
	& {x}!\langle{(\prefix{x}{y}{(\outputp{x}{y} | @{y})) | P}}\rangle 
	      | \prefix{x}{y}{(\outputp{x}{y} | @{y})} & \nonumber\\
	\red
	& (\outputp{x}{y} | @{y})\substn{\quotep{(\prefix{x}{y}{(@{y} | \outputp{x}{y})) | P}}}{y} & \nonumber\\
	=
	& \outputp{x}{\quotep{(\prefix{x}{y}{(\outputp{x}{y} | @{y})) | P}}}
	  | {(\prefix{x}{y}{(\outputp{x}{y} | @{y})) | P}} & \nonumber\\
	\red
	& \ldots & \nonumber\\
	\red^*
	& P | P | \ldots & \nonumber
\end{eqnarray}

Of course, this encoding, as an implementation, runs away, unfolding
$\bangp{P}$ eagerly. A lazier and more implementable replication
operator, restricted to input-guarded processes, may be obtained as follows.

\begin{eqnarray}
\bangp{\prefix{u}{v}{P}} 
	:= 
	\binpar{\lift{x}{\prefix{u}{v}{(\binpar{D(x)}{P})}}}{D(x)} \nonumber
\end{eqnarray}

\begin{remark}
  Note that the lazier definition still does not deal with summation
  or mixed summation (i.e. sums over input and output). The reader is
  invited to construct definitions of replication that deal with these
  features. 

  Further, the definitions are parameterized in a name, $x$. Can you,
  gentle reader, make a definition that eliminates this parameter and
  guarantees no accidental interaction between the replication
  machinery and the process being replicated -- i.e. no accidental
  sharing of names used by the process to get its work done and the
  name(s) used by the replication to effect copying. This latter
  revision of the definition of replication is crucial to obtaining
  the expected identity $!!P \sim !P$.
\end{remark}

\begin{remark}\label{rem:paradoxical_combinator}
  The reader familiar with the lambda calculus will have noticed the
  similarity between $D$ and the paradoxical combinator.

  [Ed. note: the existence of this seems to suggest we have to be more
  restrictive on the set of processes and names we admit if we are to
  support no-cloning.]
\end{remark}

\subsubsection{Bisimulation}

The computational dynamics gives rise to another kind of equivalence,
the equivalence of computational behavior. As previously mentioned
this is typically captured \emph{via} some form of bisimulation.

% The notion we use in this paper is weak barbed bisimulation
% \cite{milner91polyadicpi}.

The notion we use in this paper is derived from weak barbed
bisimulation \cite{milner91polyadicpi}. 

\begin{definition}
An \emph{observation relation}, $\downarrow_{\mathcal N}$, over a set
of names, $\mathcal N$, is the smallest relation satisfying the rules
below.

\infrule[Out-barb]{y \in {\mathcal N}, \; x \nameeq y}
		  {\outputp{x}{v} \downarrow_{\mathcal N} x}
\infrule[Par-barb]{\mbox{$P\downarrow_{\mathcal N} x$ or $Q\downarrow_{\mathcal N} x$}}
		  {\binpar{P}{Q} \downarrow_{\mathcal N} x}

We write $P \Downarrow_{\mathcal N} x$ if there is $Q$ such that 
$P \wred Q$ and $Q \downarrow_{\mathcal N} x$.
\end{definition}

\begin{definition}
%\label{def.bbisim}
An  ${\mathcal N}$-\emph{barbed bisimulation} over a set of names, ${\mathcal N}$, is a symmetric binary relation 
${\mathcal S}_{\mathcal N}$ between agents such that $P\rel{S}_{\mathcal N}Q$ implies:
\begin{enumerate}
\item If $P \red P'$ then $Q \wred Q'$ and $P'\rel{S}_{\mathcal N} Q'$.
\item If $P\downarrow_{\mathcal N} x$, then $Q\Downarrow_{\mathcal N} x$.
\end{enumerate}
$P$ is ${\mathcal N}$-barbed bisimilar to $Q$, written
$P \wbbisim_{\mathcal N} Q$, if $P \rel{S}_{\mathcal N} Q$ for some ${\mathcal N}$-barbed bisimulation ${\mathcal S}_{\mathcal N}$.
\end{definition}

$\mathcal{R} \subseteq \pi \times \pi$

$P \mathcal{R} Q => \forall P'. P \red P' \Rightarrow \exists Q'. Q \red Q', P' \mathcal{R} Q'$

$P \vdash x \Rightarrow Q \vdash x$

\begin{mathpar}
  \inferrule*[lab=Out-barb]{x \nameeq y}{{y}!\langle{Q}\rangle \vdash x}
  \and
  \inferrule*[lab=Par-barb]{\mbox{$P\vdash x$ or $Q\vdash x$}}{\binpar{P}{Q} \vdash x}
\end{mathpar}

\subsubsection{Contexts}

One of the principle advantages of computational calculi like the
$\pi$-calculus is a well-defined notion of context,
contextual-equivalence and a correlation between
contextual-equivalence and notions of bisimulation. The notion of
context allows the decomposition of a process into (sub-)process and
its syntactic environment, its context. Thus, a context may be
thought of as a process with a ``hole'' (written $\Box$) in it. The
application of a context $M$ to a process $P$, written $M[P]$, is
tantamount to filling the hole in $M$ with $P$. In this paper we do
not need the full weight of this theory, but do make use of the notion
of context in the proof the main theorem. 

\begin{mathpar}
  \inferrule* [lab=summation] {} {{M_{M},M_{N}} \bc \Box \;|\; x.M_{A} \;|\; M_{M}+M_{N}}
  \and
  \inferrule* [lab=agent] {} {{M_{A}} \bc (\vec{x})M_{P} \;| \; \clift{P_0,\ldots,M_{P},\ldots,P_N}}
  \and \\
  \inferrule* [lab=process] {} {{M_{P}} \bc M_{N} \;| \;P|M_{P} }
\end{mathpar} 

\begin{mathpar}
  \inferrule* [lab=sychronization] {} {M_{N} \bc \Box \;|\; x?M_{F} \;|\; x!M_{C}}
  \and
  \inferrule* [lab=abstraction] {} {{M_{F}} \bc (x)M_{P} }
  \and
  \inferrule* [lab=concretion] {} {{M_{C}} \bc \langle M_{P} \rangle }
  \and \\
  \inferrule* [lab=process] {} {{M_{P}} \bc M_{N} \;| \;P|M_{P} }
\end{mathpar}

\begin{definition}[contextual application] Given a context $M$, and
  process $P$, we define the \emph{contextual application}, $M[P] :=
  M\{P/\Box\}$. That is, the contextual application of M to P is the
  substitution of $P$ for $\Box$ in $M$.
\end{definition}

$\meaningof{-} : L \to \mathcal{P}(\pi)$

\begin{mathpar}
  \inferrule* [lab=collection] {} {\meaningof{true} = \pi, \and \meaningof{~E} = \pi \setminus \meaningof{E}, \and \meaningof{E_{1} \& E_{2}} = \meaningof{E_{1}} \cap \meaningof{E_{2}}}
\end{mathpar}

\begin{mathpar}
  \inferrule* [lab=structure] {} {\meaningof{0} = \{ P \in \pi | P \equiv 0 \}, \and \\ \meaningof{E_1 | E_2} = \{ P \in \pi | P \equiv P_{1} | P_{2}, P_{1} \in \meaningof{E_{1}}, P_{2} \in \meaningof{E_2}\} }
\end{mathpar}

\begin{mathpar}
 \inferrule* [lab=behavior] {} {\meaningof{\langle a?b \rangle E} = \{ P \in \pi | P \equiv Q | u?(y)P', \\ \and \\\\ \and \\ \;\;\; u \in \meaningof{a}, \forall z.P'\{z/y\} \in \meaningof{E\{z/b\}}\}, \and \\ \meaningof{a!E} = \{ P \in \pi | P \equiv Q | x!\langle P' \rangle, x \in \meaningof{a} P' \in \meaningof{E}\} }
\end{mathpar}

\begin{mathpar}
 \inferrule* [lab=nominal] {} {\meaningof{\quotep{E}} = \{ \quotep{P} \in \quotep{\pi} | P \in \meaningof{E} \}, \and \meaningof{\quotep{P}} = \{ \quotep{Q} \in \quotep{\pi} | P \equiv Q \} \and \\ \meaningof{@\quotep{E}} = \{ P \in \pi | P \equiv @x, x \in \meaningof{E} \}}
\end{mathpar}

\begin{eqnarray*}
  \\
  \meaningof{-} : TS \to ST
\end{eqnarray*}

\begin{eqnarray*}
  \\
  L : TS \to ST
\end{eqnarray*}

\begin{eqnarray*}
  \\
  P \models E \iff P \in \meaningof{E}
\end{eqnarray*}

\begin{eqnarray*}
  P \approx_{L} Q \iff \forall E \in L. P \models E \iff Q \models E
\end{eqnarray*}

\begin{eqnarray*}
  P \approx_{K} Q
\end{eqnarray*}

\begin{eqnarray*}
  P \approx Q
\end{eqnarray*}

$\approx_{K} = \approx = \approx_{L}$

\subsubsection{Contextual duality}

Note that contexts extend the quotation operation to a family of
operations from processes to names. Given a context, $M$, we can
define a \emph{nominal context}, $\quotep{M}$ by $\quotep{M}[P] :=
\quotep{M[P]}$. To foreshadow what is to come we observe that these
operations enjoy a duality with processes very much like the duality
between vectors and maps from vectors to scalars.

Further, because the calculus is essentially higher-order, we have a
correspondence between contexts and processes. More specifically,
given a name $x$ and a context $M$ we can construct $M^{*}_{x}$ such
that 

\begin{mathpar}
  M^{*}_{x} | \lift{x}{P} \red M[P]
\end{mathpar}

namely,

\begin{mathpar}
  M^{*}_{x} := x?(u).M[\dropn{u}]
\end{mathpar}

The dependence of $M^{*}_{x}$ on a name makes it an abstraction, 

\begin{mathpar}
  M^{*} := (x)x?(u).M[\dropn{u}]
\end{mathpar}

\subsection{Additional notation}

It will sometimes be convenient to denote the process a name
quotes. We already have the notation $x = \quotep{P}$, but it will be
convenient to introduce an alternate notation, $\procn{x}$, when we
want to emphasize the connection to the use of the name. Note that, by
virtue of name equivalence, $\quotep{\procn{x}} \nameeq x$; so, the
notation is consistent with previous definitions.

Further, because names have structure it is possible to effect
substitutions on the basis of that structure. This means we need to
upgrade our notation for substitutions, which we accomplish by
adapting comprehension notation. Thus,

\begin{mathpar}
  P\{ y / x : x \in S \}
\end{mathpar}

is interpreted to mean the process derived from P by replacing (in a
capture-avoiding manner) each occurrence of $x$ in $S$ by $y$. For example,

\begin{mathpar}
  P\{ \quotep{\procn{x}|\procn{x}} / x : x \in \freenames{P} \}
\end{mathpar}

will replace each (occurrence) of a free name $x$ in $P$ by
$\quotep{\procn{x}|\procn{x}}$.

Also, we will avail ourselves of the notation $x^{L}$ and $x^{R}$ to
denote injections of a name into disjoint copies of the name
space. There are numerous ways to accomplish this. One example can be
found in \cite{MeredithR05}. This notation overloads to vectors of
names: $\vec{x}^{\pi} := (x_{i}^{\pi} \; : \; 0 \leq i < |\vec{x}| )$ where $\pi \in \{L,R\}$.

We also use $P^{\Box} := P|\Box$.

In \cite{MeredithR05} an interpretation of the new operator is
given. It turns out that there are several possible interpretations
all enjoying the requisite algebraic properties of the operator (see
\cite{milner91polyadicpi}). We will therefore make liberal use of
$(\nu\; \vec{x})P$.

% subsection the_syntax_and_semantics_of_the_notation_system (end)   

\input{qm2pi.qmops} 

\input{qm2pi.sterngerlach} 

\input{qm2pi.metric} 

% section concurrent_process_calculi (end)

%\input{qm2pi.proofsketch}

% section proof sketch (end)

%\input{qm2pi.slviaknots} 

% section spatial logic via knots (end)

\input{qm2pi.conclusion}

% section conclusion (end)

%\input{qm2pi.dtcodes} 

% section wiring algorithm (end)

\input{qm2pi.ack} 

% section acknowledgments (end)

\newpage


\bibliographystyle{plain}   
\bibliography{../../biblios/main.bib}

\input{qm2pi.rhodetails}

\end{document}



% section front matter (end)

\section{Introduction}\label{sec:introduction} % (fold)
In this draft of the material i am going to have to dispense with the
usual writing conventions adopted in papers on these topics. i'm going
to have adopt whatever tone i need at the time i'm writing up the
calculations. Sometimes this may be very conversational; others it may
be the barest mathematical grunts; others still it may be that i have
lifted text from one of my other papers because the exposition of some
point was better said there. i hope that my readers are not unduly put
out by this decision. i'm not doing this to flout convention or be
rebellious. i find these calculations very technically challenging. To
keep everything going technically, something has to give; i have to
let go of some cognitive burden. So, the academic writing style --
with all of its trade-offs in terms of facilitating technical
communication -- is what i'm letting go of. Perhaps subsequent drafts
can be tightened and polished, but for now, i'm going to speak as if
we were sitting together in a coffee shop with a laptop, wifi and a
pad of paper and a pencil.

So, here's what i have to say. We -- you and i, comfortably ensconced
in our coffee shop and well-equipped with our tools -- can realize and
carry out the calculations of quantum mechanics over a very different
formal theory of dynamics, a formal theory of dynamics that
corresponds to a theory of concurrent computation with
\emph{reflection}. It has the advantage that the underlying theory is
already `quantized', but supports analogues all of the continuuous
operations. Strikingly, this underlying theory has recently been
connected with a notion of metric that we can show, by calculating
together, coincides with the metric induced by the inner product.

There are a lot of reasons why you might be interested in seeing
calculations of this form. Here's why i'm interested. For the past
several centuries there has been no competitor to the ``Newtonian''
account of dynamics. As a result the predominant share of accounts of
dynamical systems and situations have had to be formulated in terms of
the Newtonian machinery. i view this as an intellectually dangerous
position to occupy. Everything, despite it's intrinsic shape, turns
into a nail to be hit with this hammer. Recently, however, the theory
of computation has matured to the point where we have candidates for
theories of dynamics that offer very different perspective on
reasoning about dynamical systems and situations. Testing these
candidates against very successful accounts of dynamical situations,
like quantum mechanics, is going to give us some sense of how mature
they are and some measure of the quality of these accounts of
dynamics.

\subsection{Summary of contributions and outline of paper}

So, we're going to develop an interpretation of the operations of
quantum mechanics normally interpreted by Hilbert spaces and
operators. We're going to do this over a theory of computation. Note
that this is very different than the usual quantum computation program
which develops notions of computation over quantum mechanics. Rather,
we are developing a story that aligns with Wheeler's slogan: It from
Bit. To do this we will first provide an account of the theory of
computation at play here. Then we will dive into a calculation-driven
interpretation of the operations of quantum mechanics.

The reason we take this approach is that -- until very recently --
there hasn't been an axiomatic account of quantum mechanics. As a
result there has been no sharp delineation of the mathematical theory
supporting interpretation of the physical theory and the physical
theory, itself. So, ambient features of the maths are free to be
exploited (or supressed) without a real accounting of their physical
relevance. There is no sharp statement ``here's the physical theory''
qua \emph{theory} and ``here's the mathematical interpretation''
enabling a judgment of how faithful the interpretation is -- apart
from experimental observation. When there is an axiomatic account we
can judge how well a given mathematical formalism supports an
interpretation of the axioms, independent of
experimentation. Likewise, we can judge how well we have captured our
physical evidence and experience with our axiomatics, independent of
any specific mathematical implementation, with accidental detail that
may or may not have physical significance. 

In lieu of a fully fleshed out and vetted axiomatic account of quantum
mechanics, interpreting the operational notions in service of modeling
physical systems will have to suffice. In other words, we are not in
the business of providing a model of Hilbert spaces and operators. We
are in the business of providing a model of quantum mechanics because
we are motivated by testing our notions of dynamics against physical
theory; and, the predictive calculations of the physical theory must
serve as the best formulation -- shy of a fully fleshed out axiomatic
account -- of the physical theory itself (as they have for scientific
theories since time immemorial). Put another way, despite a
whole-hearted commitment to an It-from-Bit ontology, we are firmly
aligned with the shut-up-and-calculate camp as the best way to obtain
results either from the physical perspective or as a quality assurance
measure of our fledgling theory of dynamics.

In detail, we present a reflective process calculus. Then we develop
intuitive correspondences between the notions available in this
calculus and the usual physical notions supporting quantum mechanical
calculations. Thus, 

\begin{table}[htp]
  \center{
    \fbox{
      \begin{tabular}{c|c}
        quantum mechanics & process calculus \\
        \hline
        scalar & name \\
        state vector & process \\
        dual & contextual duals \\
        matrix & formal sums of process-context-dual pairs \\
        orthogonality & process annihilation \\
        inner product & execution-formula + quoting
      \end{tabular}
    }
  }
  \caption{QM - process calculi correspondences}
\end{table}

Then we tighten up these intuitions to operational definitions. We
employ the Dirac notation as the best proxy we can find for an
abstract syntax of the quantum mechanical notions. The definitions we
develop put us in contact with equational constraints coming from the
theory that we demonstrate the definitions and calculations satisfy.

This puts us in a position to shut up and calculate for the
Stern-Gerlach experimental set up, showing how these predictive
calculations become calculations on processes in our theory of a
reflective process calculus.

Penultimately, we demonstrate that the notion of metric coming from
the inner product coincides with the notion of metric available from
the theory of bisimulation. This demonstration gives us the right to
think of space as arising from behavior. Finally, we consider where we
might go from the new vantage point we have obtained.

% section introduction (end) 
 
% section introduction (end)

% \documentclass[12pt]{llncs}
%\documentclass{jktr}

\usepackage[pdftex]{hyperref}                   
\usepackage {listings}
\usepackage {mathpartir}
\usepackage{bcprules}
%\usepackage{listings}
                       
\usepackage{graphicx} 
%\usepackage[margins=2.5cm,nohead,nofoot]{geometry}
%\usepackage{geometry}
\usepackage{amsfonts}
\usepackage{amstext}
\usepackage{latexsym}
\usepackage{amssymb}
\usepackage{color}


%\include{myPreamble}
\include{qm2pi.local} 

%\ifpdf
%\usepackage[pdftex]{graphicx}
%\else
%\usepackage{graphicx}
%\fi

 % \ifpdf
%  \usepackage{pdfsync}
%  \if


%\title{Brief Article}
%\author{David F. Snyder}
%\author{L.G. Meredith}

%\address{Dept. of Math., Texas State University--San Marcos, San Marcos, TX 78666}
       
\pagestyle{empty}


\begin{document}

\lstset{language=[Objective]Caml,frame=shadowbox}

\input{qm2pi.front}

% section front matter (end)

\input{qm2pi.intro} 
 
% section introduction (end)

% \input{qm2pi.knotations} 

% section notation (end)

\input{qm2pi.process.calculi} 

% section concurrent_process_calculi_and_spatial_logics_ (end)
    
%\input{qm2pi.knots2pi} 

%\input{qm2pi.trefoil} 

%\input{qm2pi.mainthm} 

% subsection basic_interpretation (end)

%\input{qm2pi.rho.presentation} 
\subsection{The syntax and semantics of the notation system}\label{sub:the_syntax_and_semantics_of_the_notation_system} % (fold)

We now summarize a technical presentation of the calculus that
embodies our theory of dynamics. The typical presentation of such a
calculus follows the style of giving generators and relations on
them. The grammar, below, describing term constructors, freely
generates the set of processes, $\Proc$. This set is then quotiented
by a relation known as structural congruence and it is over this set
that the notion of dynamics is expressed. This presentation is
essentially that of \cite{MeredithR05} with the addition of
polyadicity and summation. For readability we have relegated some of
the technical subtleties to an appendix.

\subsubsection{Process grammar}\label{subsub:process_grammar}

\begin{mathpar}
  \inferrule* [lab=synchronization] {} {{M} \bc \pzero \;|\; x?F \;|\; x!C }
  \and
  \inferrule* [lab=abstraction] {} {{F} \bc (x)P}
  \and
  \inferrule* [lab=concretion] {} {{C} \bc \langle Q \rangle}
  \and
  \inferrule* [lab=process] {} {{P,Q} \bc M \;| \;P|Q \;|\; @{x}}
  \and
  \inferrule* [lab=name] {} {{x} \bc \quotep{P}}
\end{mathpar} 

Note that $\vec{x}$ (resp. $\vec{P}$) denotes a vector of names
(resp. processes) of length $|\vec{x}|$ (resp. $|\vec{P}|$). We adopt
the following useful abbreviations.

\begin{mathpar}
   x?(\vec{y}).P := x.(\vec{y})P \and  x\clift{\vec{P}} := x.\clift{\vec{P}}
   \and x!(y) := \lift{x}{\dropn{y}}
   \and \Pi_{i=0}^{n-1}P_i := P_0 | \ldots | P_{n-1}
\end{mathpar}

\subsubsection{Structural congruence}

\paragraph{Free and bound names and alpha-equivalence.} At the
core of structural equivalence is alpha-equivalence which identifies
process that are the same up to a change of variable. Formally, we
recognize the distinction between free and bound names. The free names
of a process, $\freenames{P}$, may be calculated recursively as
follows:

\begin{mathpar}
\freenames{\pzero} := \emptyset
  \and \\
  \freenames{x?(y).P} := \{ x \} \cup (\freenames{P} \setminus \{ y \})
  \and 
  \freenames{x!\langle P \rangle} := \{ x \} \cup \{ P \} 
  \and \\
  \freenames{P|Q} := \freenames{P} \cup \freenames{Q}
  \and \\
  \freenames{@{x}} := \{ x \}
\end{mathpar}

$\pi$
$\quotep{\pi}$

$\freenames{-} : \pi \to \mathcal{P}(\quotep{\pi})$

\begin{eqnarray*}
  \freenames{\pzero} & := & \emptyset \\
  \freenames{x?(y).P} & := & \{ x \} \cup (\freenames{P} \setminus \{ y \}) \\
  \freenames{x!\langle P \rangle} & := & \{ x \} \cup \{ P \} \\
  \freenames{P|Q} & := & \freenames{P} \cup \freenames{Q} \\
  \freenames{\dropn{x}} & := & \{ x \}
\end{eqnarray*}

The bound names of a process, $\boundnames{P}$, are those names occurring in $P$
that are not free. For example, in $x?(y).0$, the name $x$ is free, while $y$ is bound.

\begin{mathpar}
  \inferrule* [lab=monoidal-laws] {} { P|Q \equiv Q|P \and P|0 \equiv P \and P|(Q|R) \equiv (P|Q)|R }
\end{mathpar}

\begin{mathpar}
  \inferrule* [lab=alpha-equivalence] {} { (x)P \equiv (y)P\{y/x\} \and y \not\in \freenames{P} }
\end{mathpar}

\begin{definition}
Then two processes, $P,Q$, are alpha-equivalent if $P = Q\{\vec{y}/\vec{x}\}$ for
some $\vec{x} \in \boundnames{Q},\vec{y} \in \boundnames{P}$, where $Q\{\vec{y}/\vec{x}\}$
denotes the capture-avoiding substitution of $\vec{y}$ for $\vec{x}$ in $Q$.
\end{definition}

\begin{definition}
  The {\em structural congruence} \cite{SangiorgiWalker} , $\equiv$,
  between processes is the least congruence containing
  alpha-equivalence, satisfying the abelian monoid laws
  (associativity, commutativity and $\pzero$ as identity) for parallel
  composition $|$ and for summation $+$.
\end{definition}

\subsection{Name equivalence}

We take name equivalence, written $\nameeq$, to be the smallest
equivalence relation generated by the following rules.

\begin{mathpar}
\inferrule*[lab=Quote-drop]
{ }
{ \quotep{@{x}} \nameeq x }

\inferrule*[lab=Struct-equiv]
{ P \scong Q }
{ \quotep{P} \nameeq \quotep{Q} }
\end{mathpar}

The astute reader will have noticed that the mutual recursion of names
and processes imposes a mutual recursion on alpha-equivalence and
structural equivalence via name-equivalence. Fortunately, all of this
works out pleasantly and we may calculate in the natural way, free of
concern. The reader interested in the details is referred to the
appendix \ref{appendix:rho_details}.

\subsection{Substitution}

We use $\Proc$ for the set of processes, $\QProc$ for the set of
names, and $\id{\{}\vec{y} / \vec{x} \id{\}}$ to denote partial maps,
$s : \QProc \rightarrow \QProc$. A map, $s$ lifts, uniquely, to a map
on process terms, $\widehat{s} : \Proc \rightarrow \Proc$ by the
following equations.

\begin{mathpar}
  (0) \psubstp{Q}{P} := 0 \\
  (R \juxtap S) \psubstp{Q}{P}
  :=    
  (R)\psubstp{Q}{P} \juxtap (S) \psubstp{Q}{P} \\
  (x?(y).R) \psubstp{Q}{P}    
  :=    
  (x)\substp{Q}{P} (z)\concat( (R \psubstn{z}{y}) \psubstp{Q}{P} ) \\
  (\lift{x}{R}) \psubstp{Q}{P}  
  :=
  \lift{(x)\substp{Q}{P}}{ R \psubstp{Q}{P} } \\
%   (\dropn{x})  \psubstp{Q}{P}       
%   := 
%   \left\{ 
%     \begin{array}{ccc} 
%       \dropn{\quotep{Q}} & & x \nameeq \quotep{P} \\
%       \dropn{x} & & otherwise \\
%     \end{array}
%   \right. 
  (\dropn{x})  \psubstp{Q}{P}       
  := 
  \left\{ 
    \begin{array}{ccc} 
      Q & & x \nameeq \quotep{P} \\
      \dropn{x} & & otherwise \\
    \end{array}
  \right.
\end{mathpar}
 

where

\begin{eqnarray}
  (x)\id{\{} \lpquote Q \rpquote / \lpquote P \rpquote \id{\}}            = 
  \left\{ 
    \begin{array}{ccc}
      \lpquote Q \rpquote & & x \nameeq \lpquote P \rpquote \\
      x & & otherwise \\
    \end{array}
  \right. \nonumber
\end{eqnarray}

and $z$ is chosen distinct from $\quotep{P}$, $\quotep{Q}$, the free
names in $Q$, and all the names in $R$. Our $\alpha$-equivalence will
be built in the standard way from this substitution.

\begin{remark}\label{rem:no_self_referential_names}
  One consequence of these definitions is that $\forall P. \quotep{P}
  \not\in \freenames{P}$.
\end{remark}

\subsection{ Dynamic quote: an example }

Anticipating something of what's to come, consider applying the
substitution, $\widehat{\id{\{}u / z \id{\}}}$, to the following pair
of processes, $\lift{w}{y!(z)}$ and $w[ \lpquote y!(z) \rpquote ]$.

\begin{eqnarray}
	\lift{w}{y!(z)}\widehat{\id{\{}u / z \id{\}}}
		& = &
		\lift{w}{y!(u)} \nonumber\\
	w[ \lpquote y!(z) \rpquote ] \widehat{ \id{\{}u / z \id{\}} }
		& = &
		w[ \lpquote y!(z) \rpquote ] \nonumber
\end{eqnarray}

Because the body of the process between quotes is impervious to
substitution, we get radically different answers. In fact, by
examining the first process in an input context,
e.g. $x?(z).\lift{w}{y!(z)}$, we see that the process under the lift
operator may be shaped by prefixed inputs binding a name inside it. In
this sense, the lift operator will be seen as a way to dynamically
construct processes before reifying them as names.

Finally equipped with these standard features we can present the
dynamics of the calculus.

\subsubsection{Operational semantics} 

Finally, we introduce the computational dynamics. What marks these
algebras as distinct from other more traditionally studied algebraic
structures, e.g. vector spaces or polynomial rings, is the manner in
which dynamics is captured. In traditional structures, dynamics is typically
expressed through morphisms between such structures, as in linear maps
between vector spaces or morphisms between rings. In algebras
associated with the semantics of computation, the dynamics is
expressed as part of the algebraic structure itself, through a
reduction reduction relation typically denoted by $\red$. Below, we
give a recursive presentation of this relation for the calculus used
in the encoding.

$\red \subseteq \pi \times \pi$
$\red : \pi \to \mathcal{P}(\pi)$

\begin{mathpar}
  \inferrule* [lab=Comm] { \textsf{match}( x_{src}, x_{trgt} ) } { x_{trgt}?(y)P \; | \; x_{src}!\langle {Q} \rangle \red P\{\quotep{Q}/y}\} }
  \and \\
  \inferrule* [lab=Par] {{P} \red {P}'} {{{P} | {Q}} \red {{P}' | {Q}}}
  \and
  \inferrule* [lab=Equiv]{{{P} \scong {P}'} \andalso {{P}' \red {Q}'} \andalso {{Q}' \scong {Q}}}{{P} \red {Q}}
\end{mathpar}

\begin{eqnarray*}
  match_{\equiv} (\quotep{P},\quotep{Q}) & := & P \equiv Q \\
  match_{\dagger}(\quotep{P},\quotep{Q}) & := & \forall R. P|Q \red^{*} R => R \red^{*} 0 \\
  match_{K}(\quotep{P},\quotep{Q}) & := & K \mbox{ for some context } K
\end{eqnarray*}

$u?(x)P | u!\langle Q \rangle \red P\{\quotep{Q}/x\}$

%We write $\wred$ for $\red^*$, and $P\red$ if $\exists Q $ such that $ P \red Q$.
We write $P\red$ if $\exists Q $ such that $ P \red Q$ and $P\not\red$, otherwise.

\section{Replication}

As mentioned before, it is known that replication (and hence
recursion) can be implemented in a higher-order process algebra
\cite{SangiorgiWalker}. As our first example of calculation with the
machinery thus far presented we give the construction explicitly in
the {\rhoc}.

\begin{eqnarray}
	D_{x} & := & \prefix{x}{y}{(\binpar{\outputp{x}{y}}{@{y}})} \nonumber\\
	\bangp_{x}{P} & := & \binpar{{x}!\langle{\binpar{D_{x}}{P}}\rangle}{D_{x}} \nonumber
\end{eqnarray}

\begin{eqnarray}
	\bangp_{x}{P} & & \nonumber\\
	=
	& {x}!\langle{(\prefix{x}{y}{(\outputp{x}{y} | @{y})) | P}}\rangle 
	      | \prefix{x}{y}{(\outputp{x}{y} | @{y})} & \nonumber\\
	\red
	& (\outputp{x}{y} | @{y})\substn{\quotep{(\prefix{x}{y}{(@{y} | \outputp{x}{y})) | P}}}{y} & \nonumber\\
	=
	& \outputp{x}{\quotep{(\prefix{x}{y}{(\outputp{x}{y} | @{y})) | P}}}
	  | {(\prefix{x}{y}{(\outputp{x}{y} | @{y})) | P}} & \nonumber\\
	\red
	& \ldots & \nonumber\\
	\red^*
	& P | P | \ldots & \nonumber
\end{eqnarray}

Of course, this encoding, as an implementation, runs away, unfolding
$\bangp{P}$ eagerly. A lazier and more implementable replication
operator, restricted to input-guarded processes, may be obtained as follows.

\begin{eqnarray}
\bangp{\prefix{u}{v}{P}} 
	:= 
	\binpar{\lift{x}{\prefix{u}{v}{(\binpar{D(x)}{P})}}}{D(x)} \nonumber
\end{eqnarray}

\begin{remark}
  Note that the lazier definition still does not deal with summation
  or mixed summation (i.e. sums over input and output). The reader is
  invited to construct definitions of replication that deal with these
  features. 

  Further, the definitions are parameterized in a name, $x$. Can you,
  gentle reader, make a definition that eliminates this parameter and
  guarantees no accidental interaction between the replication
  machinery and the process being replicated -- i.e. no accidental
  sharing of names used by the process to get its work done and the
  name(s) used by the replication to effect copying. This latter
  revision of the definition of replication is crucial to obtaining
  the expected identity $!!P \sim !P$.
\end{remark}

\begin{remark}\label{rem:paradoxical_combinator}
  The reader familiar with the lambda calculus will have noticed the
  similarity between $D$ and the paradoxical combinator.

  [Ed. note: the existence of this seems to suggest we have to be more
  restrictive on the set of processes and names we admit if we are to
  support no-cloning.]
\end{remark}

\subsubsection{Bisimulation}

The computational dynamics gives rise to another kind of equivalence,
the equivalence of computational behavior. As previously mentioned
this is typically captured \emph{via} some form of bisimulation.

% The notion we use in this paper is weak barbed bisimulation
% \cite{milner91polyadicpi}.

The notion we use in this paper is derived from weak barbed
bisimulation \cite{milner91polyadicpi}. 

\begin{definition}
An \emph{observation relation}, $\downarrow_{\mathcal N}$, over a set
of names, $\mathcal N$, is the smallest relation satisfying the rules
below.

\infrule[Out-barb]{y \in {\mathcal N}, \; x \nameeq y}
		  {\outputp{x}{v} \downarrow_{\mathcal N} x}
\infrule[Par-barb]{\mbox{$P\downarrow_{\mathcal N} x$ or $Q\downarrow_{\mathcal N} x$}}
		  {\binpar{P}{Q} \downarrow_{\mathcal N} x}

We write $P \Downarrow_{\mathcal N} x$ if there is $Q$ such that 
$P \wred Q$ and $Q \downarrow_{\mathcal N} x$.
\end{definition}

\begin{definition}
%\label{def.bbisim}
An  ${\mathcal N}$-\emph{barbed bisimulation} over a set of names, ${\mathcal N}$, is a symmetric binary relation 
${\mathcal S}_{\mathcal N}$ between agents such that $P\rel{S}_{\mathcal N}Q$ implies:
\begin{enumerate}
\item If $P \red P'$ then $Q \wred Q'$ and $P'\rel{S}_{\mathcal N} Q'$.
\item If $P\downarrow_{\mathcal N} x$, then $Q\Downarrow_{\mathcal N} x$.
\end{enumerate}
$P$ is ${\mathcal N}$-barbed bisimilar to $Q$, written
$P \wbbisim_{\mathcal N} Q$, if $P \rel{S}_{\mathcal N} Q$ for some ${\mathcal N}$-barbed bisimulation ${\mathcal S}_{\mathcal N}$.
\end{definition}

$\mathcal{R} \subseteq \pi \times \pi$

$P \mathcal{R} Q => \forall P'. P \red P' \Rightarrow \exists Q'. Q \red Q', P' \mathcal{R} Q'$

$P \vdash x \Rightarrow Q \vdash x$

\begin{mathpar}
  \inferrule*[lab=Out-barb]{x \nameeq y}{{y}!\langle{Q}\rangle \vdash x}
  \and
  \inferrule*[lab=Par-barb]{\mbox{$P\vdash x$ or $Q\vdash x$}}{\binpar{P}{Q} \vdash x}
\end{mathpar}

\subsubsection{Contexts}

One of the principle advantages of computational calculi like the
$\pi$-calculus is a well-defined notion of context,
contextual-equivalence and a correlation between
contextual-equivalence and notions of bisimulation. The notion of
context allows the decomposition of a process into (sub-)process and
its syntactic environment, its context. Thus, a context may be
thought of as a process with a ``hole'' (written $\Box$) in it. The
application of a context $M$ to a process $P$, written $M[P]$, is
tantamount to filling the hole in $M$ with $P$. In this paper we do
not need the full weight of this theory, but do make use of the notion
of context in the proof the main theorem. 

\begin{mathpar}
  \inferrule* [lab=summation] {} {{M_{M},M_{N}} \bc \Box \;|\; x.M_{A} \;|\; M_{M}+M_{N}}
  \and
  \inferrule* [lab=agent] {} {{M_{A}} \bc (\vec{x})M_{P} \;| \; \clift{P_0,\ldots,M_{P},\ldots,P_N}}
  \and \\
  \inferrule* [lab=process] {} {{M_{P}} \bc M_{N} \;| \;P|M_{P} }
\end{mathpar} 

\begin{mathpar}
  \inferrule* [lab=sychronization] {} {M_{N} \bc \Box \;|\; x?M_{F} \;|\; x!M_{C}}
  \and
  \inferrule* [lab=abstraction] {} {{M_{F}} \bc (x)M_{P} }
  \and
  \inferrule* [lab=concretion] {} {{M_{C}} \bc \langle M_{P} \rangle }
  \and \\
  \inferrule* [lab=process] {} {{M_{P}} \bc M_{N} \;| \;P|M_{P} }
\end{mathpar}

\begin{definition}[contextual application] Given a context $M$, and
  process $P$, we define the \emph{contextual application}, $M[P] :=
  M\{P/\Box\}$. That is, the contextual application of M to P is the
  substitution of $P$ for $\Box$ in $M$.
\end{definition}

$\meaningof{-} : L \to \mathcal{P}(\pi)$

\begin{mathpar}
  \inferrule* [lab=collection] {} {\meaningof{true} = \pi, \and \meaningof{~E} = \pi \setminus \meaningof{E}, \and \meaningof{E_{1} \& E_{2}} = \meaningof{E_{1}} \cap \meaningof{E_{2}}}
\end{mathpar}

\begin{mathpar}
  \inferrule* [lab=structure] {} {\meaningof{0} = \{ P \in \pi | P \equiv 0 \}, \and \\ \meaningof{E_1 | E_2} = \{ P \in \pi | P \equiv P_{1} | P_{2}, P_{1} \in \meaningof{E_{1}}, P_{2} \in \meaningof{E_2}\} }
\end{mathpar}

\begin{mathpar}
 \inferrule* [lab=behavior] {} {\meaningof{\langle a?b \rangle E} = \{ P \in \pi | P \equiv Q | u?(y)P', \\ \and \\\\ \and \\ \;\;\; u \in \meaningof{a}, \forall z.P'\{z/y\} \in \meaningof{E\{z/b\}}\}, \and \\ \meaningof{a!E} = \{ P \in \pi | P \equiv Q | x!\langle P' \rangle, x \in \meaningof{a} P' \in \meaningof{E}\} }
\end{mathpar}

\begin{mathpar}
 \inferrule* [lab=nominal] {} {\meaningof{\quotep{E}} = \{ \quotep{P} \in \quotep{\pi} | P \in \meaningof{E} \}, \and \meaningof{\quotep{P}} = \{ \quotep{Q} \in \quotep{\pi} | P \equiv Q \} \and \\ \meaningof{@\quotep{E}} = \{ P \in \pi | P \equiv @x, x \in \meaningof{E} \}}
\end{mathpar}

\begin{eqnarray*}
  \\
  \meaningof{-} : TS \to ST
\end{eqnarray*}

\begin{eqnarray*}
  \\
  L : TS \to ST
\end{eqnarray*}

\begin{eqnarray*}
  \\
  P \models E \iff P \in \meaningof{E}
\end{eqnarray*}

\begin{eqnarray*}
  P \approx_{L} Q \iff \forall E \in L. P \models E \iff Q \models E
\end{eqnarray*}

\begin{eqnarray*}
  P \approx_{K} Q
\end{eqnarray*}

\begin{eqnarray*}
  P \approx Q
\end{eqnarray*}

$\approx_{K} = \approx = \approx_{L}$

\subsubsection{Contextual duality}

Note that contexts extend the quotation operation to a family of
operations from processes to names. Given a context, $M$, we can
define a \emph{nominal context}, $\quotep{M}$ by $\quotep{M}[P] :=
\quotep{M[P]}$. To foreshadow what is to come we observe that these
operations enjoy a duality with processes very much like the duality
between vectors and maps from vectors to scalars.

Further, because the calculus is essentially higher-order, we have a
correspondence between contexts and processes. More specifically,
given a name $x$ and a context $M$ we can construct $M^{*}_{x}$ such
that 

\begin{mathpar}
  M^{*}_{x} | \lift{x}{P} \red M[P]
\end{mathpar}

namely,

\begin{mathpar}
  M^{*}_{x} := x?(u).M[\dropn{u}]
\end{mathpar}

The dependence of $M^{*}_{x}$ on a name makes it an abstraction, 

\begin{mathpar}
  M^{*} := (x)x?(u).M[\dropn{u}]
\end{mathpar}

\subsection{Additional notation}

It will sometimes be convenient to denote the process a name
quotes. We already have the notation $x = \quotep{P}$, but it will be
convenient to introduce an alternate notation, $\procn{x}$, when we
want to emphasize the connection to the use of the name. Note that, by
virtue of name equivalence, $\quotep{\procn{x}} \nameeq x$; so, the
notation is consistent with previous definitions.

Further, because names have structure it is possible to effect
substitutions on the basis of that structure. This means we need to
upgrade our notation for substitutions, which we accomplish by
adapting comprehension notation. Thus,

\begin{mathpar}
  P\{ y / x : x \in S \}
\end{mathpar}

is interpreted to mean the process derived from P by replacing (in a
capture-avoiding manner) each occurrence of $x$ in $S$ by $y$. For example,

\begin{mathpar}
  P\{ \quotep{\procn{x}|\procn{x}} / x : x \in \freenames{P} \}
\end{mathpar}

will replace each (occurrence) of a free name $x$ in $P$ by
$\quotep{\procn{x}|\procn{x}}$.

Also, we will avail ourselves of the notation $x^{L}$ and $x^{R}$ to
denote injections of a name into disjoint copies of the name
space. There are numerous ways to accomplish this. One example can be
found in \cite{MeredithR05}. This notation overloads to vectors of
names: $\vec{x}^{\pi} := (x_{i}^{\pi} \; : \; 0 \leq i < |\vec{x}| )$ where $\pi \in \{L,R\}$.

We also use $P^{\Box} := P|\Box$.

In \cite{MeredithR05} an interpretation of the new operator is
given. It turns out that there are several possible interpretations
all enjoying the requisite algebraic properties of the operator (see
\cite{milner91polyadicpi}). We will therefore make liberal use of
$(\nu\; \vec{x})P$.

% subsection the_syntax_and_semantics_of_the_notation_system (end)   

\input{qm2pi.qmops} 

\input{qm2pi.sterngerlach} 

\input{qm2pi.metric} 

% section concurrent_process_calculi (end)

%\input{qm2pi.proofsketch}

% section proof sketch (end)

%\input{qm2pi.slviaknots} 

% section spatial logic via knots (end)

\input{qm2pi.conclusion}

% section conclusion (end)

%\input{qm2pi.dtcodes} 

% section wiring algorithm (end)

\input{qm2pi.ack} 

% section acknowledgments (end)

\newpage


\bibliographystyle{plain}   
\bibliography{../../biblios/main.bib}

\input{qm2pi.rhodetails}

\end{document}

 

% section notation (end)

\input{qm2pi.process.calculi} 

% section concurrent_process_calculi_and_spatial_logics_ (end)
    
%\documentclass[12pt]{llncs}
%\documentclass{jktr}

\usepackage[pdftex]{hyperref}                   
\usepackage {listings}
\usepackage {mathpartir}
\usepackage{bcprules}
%\usepackage{listings}
                       
\usepackage{graphicx} 
%\usepackage[margins=2.5cm,nohead,nofoot]{geometry}
%\usepackage{geometry}
\usepackage{amsfonts}
\usepackage{amstext}
\usepackage{latexsym}
\usepackage{amssymb}
\usepackage{color}


%\include{myPreamble}
\include{qm2pi.local} 

%\ifpdf
%\usepackage[pdftex]{graphicx}
%\else
%\usepackage{graphicx}
%\fi

 % \ifpdf
%  \usepackage{pdfsync}
%  \if


%\title{Brief Article}
%\author{David F. Snyder}
%\author{L.G. Meredith}

%\address{Dept. of Math., Texas State University--San Marcos, San Marcos, TX 78666}
       
\pagestyle{empty}


\begin{document}

\lstset{language=[Objective]Caml,frame=shadowbox}

\input{qm2pi.front}

% section front matter (end)

\input{qm2pi.intro} 
 
% section introduction (end)

% \input{qm2pi.knotations} 

% section notation (end)

\input{qm2pi.process.calculi} 

% section concurrent_process_calculi_and_spatial_logics_ (end)
    
%\input{qm2pi.knots2pi} 

%\input{qm2pi.trefoil} 

%\input{qm2pi.mainthm} 

% subsection basic_interpretation (end)

%\input{qm2pi.rho.presentation} 
\subsection{The syntax and semantics of the notation system}\label{sub:the_syntax_and_semantics_of_the_notation_system} % (fold)

We now summarize a technical presentation of the calculus that
embodies our theory of dynamics. The typical presentation of such a
calculus follows the style of giving generators and relations on
them. The grammar, below, describing term constructors, freely
generates the set of processes, $\Proc$. This set is then quotiented
by a relation known as structural congruence and it is over this set
that the notion of dynamics is expressed. This presentation is
essentially that of \cite{MeredithR05} with the addition of
polyadicity and summation. For readability we have relegated some of
the technical subtleties to an appendix.

\subsubsection{Process grammar}\label{subsub:process_grammar}

\begin{mathpar}
  \inferrule* [lab=synchronization] {} {{M} \bc \pzero \;|\; x?F \;|\; x!C }
  \and
  \inferrule* [lab=abstraction] {} {{F} \bc (x)P}
  \and
  \inferrule* [lab=concretion] {} {{C} \bc \langle Q \rangle}
  \and
  \inferrule* [lab=process] {} {{P,Q} \bc M \;| \;P|Q \;|\; @{x}}
  \and
  \inferrule* [lab=name] {} {{x} \bc \quotep{P}}
\end{mathpar} 

Note that $\vec{x}$ (resp. $\vec{P}$) denotes a vector of names
(resp. processes) of length $|\vec{x}|$ (resp. $|\vec{P}|$). We adopt
the following useful abbreviations.

\begin{mathpar}
   x?(\vec{y}).P := x.(\vec{y})P \and  x\clift{\vec{P}} := x.\clift{\vec{P}}
   \and x!(y) := \lift{x}{\dropn{y}}
   \and \Pi_{i=0}^{n-1}P_i := P_0 | \ldots | P_{n-1}
\end{mathpar}

\subsubsection{Structural congruence}

\paragraph{Free and bound names and alpha-equivalence.} At the
core of structural equivalence is alpha-equivalence which identifies
process that are the same up to a change of variable. Formally, we
recognize the distinction between free and bound names. The free names
of a process, $\freenames{P}$, may be calculated recursively as
follows:

\begin{mathpar}
\freenames{\pzero} := \emptyset
  \and \\
  \freenames{x?(y).P} := \{ x \} \cup (\freenames{P} \setminus \{ y \})
  \and 
  \freenames{x!\langle P \rangle} := \{ x \} \cup \{ P \} 
  \and \\
  \freenames{P|Q} := \freenames{P} \cup \freenames{Q}
  \and \\
  \freenames{@{x}} := \{ x \}
\end{mathpar}

$\pi$
$\quotep{\pi}$

$\freenames{-} : \pi \to \mathcal{P}(\quotep{\pi})$

\begin{eqnarray*}
  \freenames{\pzero} & := & \emptyset \\
  \freenames{x?(y).P} & := & \{ x \} \cup (\freenames{P} \setminus \{ y \}) \\
  \freenames{x!\langle P \rangle} & := & \{ x \} \cup \{ P \} \\
  \freenames{P|Q} & := & \freenames{P} \cup \freenames{Q} \\
  \freenames{\dropn{x}} & := & \{ x \}
\end{eqnarray*}

The bound names of a process, $\boundnames{P}$, are those names occurring in $P$
that are not free. For example, in $x?(y).0$, the name $x$ is free, while $y$ is bound.

\begin{mathpar}
  \inferrule* [lab=monoidal-laws] {} { P|Q \equiv Q|P \and P|0 \equiv P \and P|(Q|R) \equiv (P|Q)|R }
\end{mathpar}

\begin{mathpar}
  \inferrule* [lab=alpha-equivalence] {} { (x)P \equiv (y)P\{y/x\} \and y \not\in \freenames{P} }
\end{mathpar}

\begin{definition}
Then two processes, $P,Q$, are alpha-equivalent if $P = Q\{\vec{y}/\vec{x}\}$ for
some $\vec{x} \in \boundnames{Q},\vec{y} \in \boundnames{P}$, where $Q\{\vec{y}/\vec{x}\}$
denotes the capture-avoiding substitution of $\vec{y}$ for $\vec{x}$ in $Q$.
\end{definition}

\begin{definition}
  The {\em structural congruence} \cite{SangiorgiWalker} , $\equiv$,
  between processes is the least congruence containing
  alpha-equivalence, satisfying the abelian monoid laws
  (associativity, commutativity and $\pzero$ as identity) for parallel
  composition $|$ and for summation $+$.
\end{definition}

\subsection{Name equivalence}

We take name equivalence, written $\nameeq$, to be the smallest
equivalence relation generated by the following rules.

\begin{mathpar}
\inferrule*[lab=Quote-drop]
{ }
{ \quotep{@{x}} \nameeq x }

\inferrule*[lab=Struct-equiv]
{ P \scong Q }
{ \quotep{P} \nameeq \quotep{Q} }
\end{mathpar}

The astute reader will have noticed that the mutual recursion of names
and processes imposes a mutual recursion on alpha-equivalence and
structural equivalence via name-equivalence. Fortunately, all of this
works out pleasantly and we may calculate in the natural way, free of
concern. The reader interested in the details is referred to the
appendix \ref{appendix:rho_details}.

\subsection{Substitution}

We use $\Proc$ for the set of processes, $\QProc$ for the set of
names, and $\id{\{}\vec{y} / \vec{x} \id{\}}$ to denote partial maps,
$s : \QProc \rightarrow \QProc$. A map, $s$ lifts, uniquely, to a map
on process terms, $\widehat{s} : \Proc \rightarrow \Proc$ by the
following equations.

\begin{mathpar}
  (0) \psubstp{Q}{P} := 0 \\
  (R \juxtap S) \psubstp{Q}{P}
  :=    
  (R)\psubstp{Q}{P} \juxtap (S) \psubstp{Q}{P} \\
  (x?(y).R) \psubstp{Q}{P}    
  :=    
  (x)\substp{Q}{P} (z)\concat( (R \psubstn{z}{y}) \psubstp{Q}{P} ) \\
  (\lift{x}{R}) \psubstp{Q}{P}  
  :=
  \lift{(x)\substp{Q}{P}}{ R \psubstp{Q}{P} } \\
%   (\dropn{x})  \psubstp{Q}{P}       
%   := 
%   \left\{ 
%     \begin{array}{ccc} 
%       \dropn{\quotep{Q}} & & x \nameeq \quotep{P} \\
%       \dropn{x} & & otherwise \\
%     \end{array}
%   \right. 
  (\dropn{x})  \psubstp{Q}{P}       
  := 
  \left\{ 
    \begin{array}{ccc} 
      Q & & x \nameeq \quotep{P} \\
      \dropn{x} & & otherwise \\
    \end{array}
  \right.
\end{mathpar}
 

where

\begin{eqnarray}
  (x)\id{\{} \lpquote Q \rpquote / \lpquote P \rpquote \id{\}}            = 
  \left\{ 
    \begin{array}{ccc}
      \lpquote Q \rpquote & & x \nameeq \lpquote P \rpquote \\
      x & & otherwise \\
    \end{array}
  \right. \nonumber
\end{eqnarray}

and $z$ is chosen distinct from $\quotep{P}$, $\quotep{Q}$, the free
names in $Q$, and all the names in $R$. Our $\alpha$-equivalence will
be built in the standard way from this substitution.

\begin{remark}\label{rem:no_self_referential_names}
  One consequence of these definitions is that $\forall P. \quotep{P}
  \not\in \freenames{P}$.
\end{remark}

\subsection{ Dynamic quote: an example }

Anticipating something of what's to come, consider applying the
substitution, $\widehat{\id{\{}u / z \id{\}}}$, to the following pair
of processes, $\lift{w}{y!(z)}$ and $w[ \lpquote y!(z) \rpquote ]$.

\begin{eqnarray}
	\lift{w}{y!(z)}\widehat{\id{\{}u / z \id{\}}}
		& = &
		\lift{w}{y!(u)} \nonumber\\
	w[ \lpquote y!(z) \rpquote ] \widehat{ \id{\{}u / z \id{\}} }
		& = &
		w[ \lpquote y!(z) \rpquote ] \nonumber
\end{eqnarray}

Because the body of the process between quotes is impervious to
substitution, we get radically different answers. In fact, by
examining the first process in an input context,
e.g. $x?(z).\lift{w}{y!(z)}$, we see that the process under the lift
operator may be shaped by prefixed inputs binding a name inside it. In
this sense, the lift operator will be seen as a way to dynamically
construct processes before reifying them as names.

Finally equipped with these standard features we can present the
dynamics of the calculus.

\subsubsection{Operational semantics} 

Finally, we introduce the computational dynamics. What marks these
algebras as distinct from other more traditionally studied algebraic
structures, e.g. vector spaces or polynomial rings, is the manner in
which dynamics is captured. In traditional structures, dynamics is typically
expressed through morphisms between such structures, as in linear maps
between vector spaces or morphisms between rings. In algebras
associated with the semantics of computation, the dynamics is
expressed as part of the algebraic structure itself, through a
reduction reduction relation typically denoted by $\red$. Below, we
give a recursive presentation of this relation for the calculus used
in the encoding.

$\red \subseteq \pi \times \pi$
$\red : \pi \to \mathcal{P}(\pi)$

\begin{mathpar}
  \inferrule* [lab=Comm] { \textsf{match}( x_{src}, x_{trgt} ) } { x_{trgt}?(y)P \; | \; x_{src}!\langle {Q} \rangle \red P\{\quotep{Q}/y}\} }
  \and \\
  \inferrule* [lab=Par] {{P} \red {P}'} {{{P} | {Q}} \red {{P}' | {Q}}}
  \and
  \inferrule* [lab=Equiv]{{{P} \scong {P}'} \andalso {{P}' \red {Q}'} \andalso {{Q}' \scong {Q}}}{{P} \red {Q}}
\end{mathpar}

\begin{eqnarray*}
  match_{\equiv} (\quotep{P},\quotep{Q}) & := & P \equiv Q \\
  match_{\dagger}(\quotep{P},\quotep{Q}) & := & \forall R. P|Q \red^{*} R => R \red^{*} 0 \\
  match_{K}(\quotep{P},\quotep{Q}) & := & K \mbox{ for some context } K
\end{eqnarray*}

$u?(x)P | u!\langle Q \rangle \red P\{\quotep{Q}/x\}$

%We write $\wred$ for $\red^*$, and $P\red$ if $\exists Q $ such that $ P \red Q$.
We write $P\red$ if $\exists Q $ such that $ P \red Q$ and $P\not\red$, otherwise.

\section{Replication}

As mentioned before, it is known that replication (and hence
recursion) can be implemented in a higher-order process algebra
\cite{SangiorgiWalker}. As our first example of calculation with the
machinery thus far presented we give the construction explicitly in
the {\rhoc}.

\begin{eqnarray}
	D_{x} & := & \prefix{x}{y}{(\binpar{\outputp{x}{y}}{@{y}})} \nonumber\\
	\bangp_{x}{P} & := & \binpar{{x}!\langle{\binpar{D_{x}}{P}}\rangle}{D_{x}} \nonumber
\end{eqnarray}

\begin{eqnarray}
	\bangp_{x}{P} & & \nonumber\\
	=
	& {x}!\langle{(\prefix{x}{y}{(\outputp{x}{y} | @{y})) | P}}\rangle 
	      | \prefix{x}{y}{(\outputp{x}{y} | @{y})} & \nonumber\\
	\red
	& (\outputp{x}{y} | @{y})\substn{\quotep{(\prefix{x}{y}{(@{y} | \outputp{x}{y})) | P}}}{y} & \nonumber\\
	=
	& \outputp{x}{\quotep{(\prefix{x}{y}{(\outputp{x}{y} | @{y})) | P}}}
	  | {(\prefix{x}{y}{(\outputp{x}{y} | @{y})) | P}} & \nonumber\\
	\red
	& \ldots & \nonumber\\
	\red^*
	& P | P | \ldots & \nonumber
\end{eqnarray}

Of course, this encoding, as an implementation, runs away, unfolding
$\bangp{P}$ eagerly. A lazier and more implementable replication
operator, restricted to input-guarded processes, may be obtained as follows.

\begin{eqnarray}
\bangp{\prefix{u}{v}{P}} 
	:= 
	\binpar{\lift{x}{\prefix{u}{v}{(\binpar{D(x)}{P})}}}{D(x)} \nonumber
\end{eqnarray}

\begin{remark}
  Note that the lazier definition still does not deal with summation
  or mixed summation (i.e. sums over input and output). The reader is
  invited to construct definitions of replication that deal with these
  features. 

  Further, the definitions are parameterized in a name, $x$. Can you,
  gentle reader, make a definition that eliminates this parameter and
  guarantees no accidental interaction between the replication
  machinery and the process being replicated -- i.e. no accidental
  sharing of names used by the process to get its work done and the
  name(s) used by the replication to effect copying. This latter
  revision of the definition of replication is crucial to obtaining
  the expected identity $!!P \sim !P$.
\end{remark}

\begin{remark}\label{rem:paradoxical_combinator}
  The reader familiar with the lambda calculus will have noticed the
  similarity between $D$ and the paradoxical combinator.

  [Ed. note: the existence of this seems to suggest we have to be more
  restrictive on the set of processes and names we admit if we are to
  support no-cloning.]
\end{remark}

\subsubsection{Bisimulation}

The computational dynamics gives rise to another kind of equivalence,
the equivalence of computational behavior. As previously mentioned
this is typically captured \emph{via} some form of bisimulation.

% The notion we use in this paper is weak barbed bisimulation
% \cite{milner91polyadicpi}.

The notion we use in this paper is derived from weak barbed
bisimulation \cite{milner91polyadicpi}. 

\begin{definition}
An \emph{observation relation}, $\downarrow_{\mathcal N}$, over a set
of names, $\mathcal N$, is the smallest relation satisfying the rules
below.

\infrule[Out-barb]{y \in {\mathcal N}, \; x \nameeq y}
		  {\outputp{x}{v} \downarrow_{\mathcal N} x}
\infrule[Par-barb]{\mbox{$P\downarrow_{\mathcal N} x$ or $Q\downarrow_{\mathcal N} x$}}
		  {\binpar{P}{Q} \downarrow_{\mathcal N} x}

We write $P \Downarrow_{\mathcal N} x$ if there is $Q$ such that 
$P \wred Q$ and $Q \downarrow_{\mathcal N} x$.
\end{definition}

\begin{definition}
%\label{def.bbisim}
An  ${\mathcal N}$-\emph{barbed bisimulation} over a set of names, ${\mathcal N}$, is a symmetric binary relation 
${\mathcal S}_{\mathcal N}$ between agents such that $P\rel{S}_{\mathcal N}Q$ implies:
\begin{enumerate}
\item If $P \red P'$ then $Q \wred Q'$ and $P'\rel{S}_{\mathcal N} Q'$.
\item If $P\downarrow_{\mathcal N} x$, then $Q\Downarrow_{\mathcal N} x$.
\end{enumerate}
$P$ is ${\mathcal N}$-barbed bisimilar to $Q$, written
$P \wbbisim_{\mathcal N} Q$, if $P \rel{S}_{\mathcal N} Q$ for some ${\mathcal N}$-barbed bisimulation ${\mathcal S}_{\mathcal N}$.
\end{definition}

$\mathcal{R} \subseteq \pi \times \pi$

$P \mathcal{R} Q => \forall P'. P \red P' \Rightarrow \exists Q'. Q \red Q', P' \mathcal{R} Q'$

$P \vdash x \Rightarrow Q \vdash x$

\begin{mathpar}
  \inferrule*[lab=Out-barb]{x \nameeq y}{{y}!\langle{Q}\rangle \vdash x}
  \and
  \inferrule*[lab=Par-barb]{\mbox{$P\vdash x$ or $Q\vdash x$}}{\binpar{P}{Q} \vdash x}
\end{mathpar}

\subsubsection{Contexts}

One of the principle advantages of computational calculi like the
$\pi$-calculus is a well-defined notion of context,
contextual-equivalence and a correlation between
contextual-equivalence and notions of bisimulation. The notion of
context allows the decomposition of a process into (sub-)process and
its syntactic environment, its context. Thus, a context may be
thought of as a process with a ``hole'' (written $\Box$) in it. The
application of a context $M$ to a process $P$, written $M[P]$, is
tantamount to filling the hole in $M$ with $P$. In this paper we do
not need the full weight of this theory, but do make use of the notion
of context in the proof the main theorem. 

\begin{mathpar}
  \inferrule* [lab=summation] {} {{M_{M},M_{N}} \bc \Box \;|\; x.M_{A} \;|\; M_{M}+M_{N}}
  \and
  \inferrule* [lab=agent] {} {{M_{A}} \bc (\vec{x})M_{P} \;| \; \clift{P_0,\ldots,M_{P},\ldots,P_N}}
  \and \\
  \inferrule* [lab=process] {} {{M_{P}} \bc M_{N} \;| \;P|M_{P} }
\end{mathpar} 

\begin{mathpar}
  \inferrule* [lab=sychronization] {} {M_{N} \bc \Box \;|\; x?M_{F} \;|\; x!M_{C}}
  \and
  \inferrule* [lab=abstraction] {} {{M_{F}} \bc (x)M_{P} }
  \and
  \inferrule* [lab=concretion] {} {{M_{C}} \bc \langle M_{P} \rangle }
  \and \\
  \inferrule* [lab=process] {} {{M_{P}} \bc M_{N} \;| \;P|M_{P} }
\end{mathpar}

\begin{definition}[contextual application] Given a context $M$, and
  process $P$, we define the \emph{contextual application}, $M[P] :=
  M\{P/\Box\}$. That is, the contextual application of M to P is the
  substitution of $P$ for $\Box$ in $M$.
\end{definition}

$\meaningof{-} : L \to \mathcal{P}(\pi)$

\begin{mathpar}
  \inferrule* [lab=collection] {} {\meaningof{true} = \pi, \and \meaningof{~E} = \pi \setminus \meaningof{E}, \and \meaningof{E_{1} \& E_{2}} = \meaningof{E_{1}} \cap \meaningof{E_{2}}}
\end{mathpar}

\begin{mathpar}
  \inferrule* [lab=structure] {} {\meaningof{0} = \{ P \in \pi | P \equiv 0 \}, \and \\ \meaningof{E_1 | E_2} = \{ P \in \pi | P \equiv P_{1} | P_{2}, P_{1} \in \meaningof{E_{1}}, P_{2} \in \meaningof{E_2}\} }
\end{mathpar}

\begin{mathpar}
 \inferrule* [lab=behavior] {} {\meaningof{\langle a?b \rangle E} = \{ P \in \pi | P \equiv Q | u?(y)P', \\ \and \\\\ \and \\ \;\;\; u \in \meaningof{a}, \forall z.P'\{z/y\} \in \meaningof{E\{z/b\}}\}, \and \\ \meaningof{a!E} = \{ P \in \pi | P \equiv Q | x!\langle P' \rangle, x \in \meaningof{a} P' \in \meaningof{E}\} }
\end{mathpar}

\begin{mathpar}
 \inferrule* [lab=nominal] {} {\meaningof{\quotep{E}} = \{ \quotep{P} \in \quotep{\pi} | P \in \meaningof{E} \}, \and \meaningof{\quotep{P}} = \{ \quotep{Q} \in \quotep{\pi} | P \equiv Q \} \and \\ \meaningof{@\quotep{E}} = \{ P \in \pi | P \equiv @x, x \in \meaningof{E} \}}
\end{mathpar}

\begin{eqnarray*}
  \\
  \meaningof{-} : TS \to ST
\end{eqnarray*}

\begin{eqnarray*}
  \\
  L : TS \to ST
\end{eqnarray*}

\begin{eqnarray*}
  \\
  P \models E \iff P \in \meaningof{E}
\end{eqnarray*}

\begin{eqnarray*}
  P \approx_{L} Q \iff \forall E \in L. P \models E \iff Q \models E
\end{eqnarray*}

\begin{eqnarray*}
  P \approx_{K} Q
\end{eqnarray*}

\begin{eqnarray*}
  P \approx Q
\end{eqnarray*}

$\approx_{K} = \approx = \approx_{L}$

\subsubsection{Contextual duality}

Note that contexts extend the quotation operation to a family of
operations from processes to names. Given a context, $M$, we can
define a \emph{nominal context}, $\quotep{M}$ by $\quotep{M}[P] :=
\quotep{M[P]}$. To foreshadow what is to come we observe that these
operations enjoy a duality with processes very much like the duality
between vectors and maps from vectors to scalars.

Further, because the calculus is essentially higher-order, we have a
correspondence between contexts and processes. More specifically,
given a name $x$ and a context $M$ we can construct $M^{*}_{x}$ such
that 

\begin{mathpar}
  M^{*}_{x} | \lift{x}{P} \red M[P]
\end{mathpar}

namely,

\begin{mathpar}
  M^{*}_{x} := x?(u).M[\dropn{u}]
\end{mathpar}

The dependence of $M^{*}_{x}$ on a name makes it an abstraction, 

\begin{mathpar}
  M^{*} := (x)x?(u).M[\dropn{u}]
\end{mathpar}

\subsection{Additional notation}

It will sometimes be convenient to denote the process a name
quotes. We already have the notation $x = \quotep{P}$, but it will be
convenient to introduce an alternate notation, $\procn{x}$, when we
want to emphasize the connection to the use of the name. Note that, by
virtue of name equivalence, $\quotep{\procn{x}} \nameeq x$; so, the
notation is consistent with previous definitions.

Further, because names have structure it is possible to effect
substitutions on the basis of that structure. This means we need to
upgrade our notation for substitutions, which we accomplish by
adapting comprehension notation. Thus,

\begin{mathpar}
  P\{ y / x : x \in S \}
\end{mathpar}

is interpreted to mean the process derived from P by replacing (in a
capture-avoiding manner) each occurrence of $x$ in $S$ by $y$. For example,

\begin{mathpar}
  P\{ \quotep{\procn{x}|\procn{x}} / x : x \in \freenames{P} \}
\end{mathpar}

will replace each (occurrence) of a free name $x$ in $P$ by
$\quotep{\procn{x}|\procn{x}}$.

Also, we will avail ourselves of the notation $x^{L}$ and $x^{R}$ to
denote injections of a name into disjoint copies of the name
space. There are numerous ways to accomplish this. One example can be
found in \cite{MeredithR05}. This notation overloads to vectors of
names: $\vec{x}^{\pi} := (x_{i}^{\pi} \; : \; 0 \leq i < |\vec{x}| )$ where $\pi \in \{L,R\}$.

We also use $P^{\Box} := P|\Box$.

In \cite{MeredithR05} an interpretation of the new operator is
given. It turns out that there are several possible interpretations
all enjoying the requisite algebraic properties of the operator (see
\cite{milner91polyadicpi}). We will therefore make liberal use of
$(\nu\; \vec{x})P$.

% subsection the_syntax_and_semantics_of_the_notation_system (end)   

\input{qm2pi.qmops} 

\input{qm2pi.sterngerlach} 

\input{qm2pi.metric} 

% section concurrent_process_calculi (end)

%\input{qm2pi.proofsketch}

% section proof sketch (end)

%\input{qm2pi.slviaknots} 

% section spatial logic via knots (end)

\input{qm2pi.conclusion}

% section conclusion (end)

%\input{qm2pi.dtcodes} 

% section wiring algorithm (end)

\input{qm2pi.ack} 

% section acknowledgments (end)

\newpage


\bibliographystyle{plain}   
\bibliography{../../biblios/main.bib}

\input{qm2pi.rhodetails}

\end{document}

 

%\documentclass[12pt]{llncs}
%\documentclass{jktr}

\usepackage[pdftex]{hyperref}                   
\usepackage {listings}
\usepackage {mathpartir}
\usepackage{bcprules}
%\usepackage{listings}
                       
\usepackage{graphicx} 
%\usepackage[margins=2.5cm,nohead,nofoot]{geometry}
%\usepackage{geometry}
\usepackage{amsfonts}
\usepackage{amstext}
\usepackage{latexsym}
\usepackage{amssymb}
\usepackage{color}


%\include{myPreamble}
\include{qm2pi.local} 

%\ifpdf
%\usepackage[pdftex]{graphicx}
%\else
%\usepackage{graphicx}
%\fi

 % \ifpdf
%  \usepackage{pdfsync}
%  \if


%\title{Brief Article}
%\author{David F. Snyder}
%\author{L.G. Meredith}

%\address{Dept. of Math., Texas State University--San Marcos, San Marcos, TX 78666}
       
\pagestyle{empty}


\begin{document}

\lstset{language=[Objective]Caml,frame=shadowbox}

\input{qm2pi.front}

% section front matter (end)

\input{qm2pi.intro} 
 
% section introduction (end)

% \input{qm2pi.knotations} 

% section notation (end)

\input{qm2pi.process.calculi} 

% section concurrent_process_calculi_and_spatial_logics_ (end)
    
%\input{qm2pi.knots2pi} 

%\input{qm2pi.trefoil} 

%\input{qm2pi.mainthm} 

% subsection basic_interpretation (end)

%\input{qm2pi.rho.presentation} 
\subsection{The syntax and semantics of the notation system}\label{sub:the_syntax_and_semantics_of_the_notation_system} % (fold)

We now summarize a technical presentation of the calculus that
embodies our theory of dynamics. The typical presentation of such a
calculus follows the style of giving generators and relations on
them. The grammar, below, describing term constructors, freely
generates the set of processes, $\Proc$. This set is then quotiented
by a relation known as structural congruence and it is over this set
that the notion of dynamics is expressed. This presentation is
essentially that of \cite{MeredithR05} with the addition of
polyadicity and summation. For readability we have relegated some of
the technical subtleties to an appendix.

\subsubsection{Process grammar}\label{subsub:process_grammar}

\begin{mathpar}
  \inferrule* [lab=synchronization] {} {{M} \bc \pzero \;|\; x?F \;|\; x!C }
  \and
  \inferrule* [lab=abstraction] {} {{F} \bc (x)P}
  \and
  \inferrule* [lab=concretion] {} {{C} \bc \langle Q \rangle}
  \and
  \inferrule* [lab=process] {} {{P,Q} \bc M \;| \;P|Q \;|\; @{x}}
  \and
  \inferrule* [lab=name] {} {{x} \bc \quotep{P}}
\end{mathpar} 

Note that $\vec{x}$ (resp. $\vec{P}$) denotes a vector of names
(resp. processes) of length $|\vec{x}|$ (resp. $|\vec{P}|$). We adopt
the following useful abbreviations.

\begin{mathpar}
   x?(\vec{y}).P := x.(\vec{y})P \and  x\clift{\vec{P}} := x.\clift{\vec{P}}
   \and x!(y) := \lift{x}{\dropn{y}}
   \and \Pi_{i=0}^{n-1}P_i := P_0 | \ldots | P_{n-1}
\end{mathpar}

\subsubsection{Structural congruence}

\paragraph{Free and bound names and alpha-equivalence.} At the
core of structural equivalence is alpha-equivalence which identifies
process that are the same up to a change of variable. Formally, we
recognize the distinction between free and bound names. The free names
of a process, $\freenames{P}$, may be calculated recursively as
follows:

\begin{mathpar}
\freenames{\pzero} := \emptyset
  \and \\
  \freenames{x?(y).P} := \{ x \} \cup (\freenames{P} \setminus \{ y \})
  \and 
  \freenames{x!\langle P \rangle} := \{ x \} \cup \{ P \} 
  \and \\
  \freenames{P|Q} := \freenames{P} \cup \freenames{Q}
  \and \\
  \freenames{@{x}} := \{ x \}
\end{mathpar}

$\pi$
$\quotep{\pi}$

$\freenames{-} : \pi \to \mathcal{P}(\quotep{\pi})$

\begin{eqnarray*}
  \freenames{\pzero} & := & \emptyset \\
  \freenames{x?(y).P} & := & \{ x \} \cup (\freenames{P} \setminus \{ y \}) \\
  \freenames{x!\langle P \rangle} & := & \{ x \} \cup \{ P \} \\
  \freenames{P|Q} & := & \freenames{P} \cup \freenames{Q} \\
  \freenames{\dropn{x}} & := & \{ x \}
\end{eqnarray*}

The bound names of a process, $\boundnames{P}$, are those names occurring in $P$
that are not free. For example, in $x?(y).0$, the name $x$ is free, while $y$ is bound.

\begin{mathpar}
  \inferrule* [lab=monoidal-laws] {} { P|Q \equiv Q|P \and P|0 \equiv P \and P|(Q|R) \equiv (P|Q)|R }
\end{mathpar}

\begin{mathpar}
  \inferrule* [lab=alpha-equivalence] {} { (x)P \equiv (y)P\{y/x\} \and y \not\in \freenames{P} }
\end{mathpar}

\begin{definition}
Then two processes, $P,Q$, are alpha-equivalent if $P = Q\{\vec{y}/\vec{x}\}$ for
some $\vec{x} \in \boundnames{Q},\vec{y} \in \boundnames{P}$, where $Q\{\vec{y}/\vec{x}\}$
denotes the capture-avoiding substitution of $\vec{y}$ for $\vec{x}$ in $Q$.
\end{definition}

\begin{definition}
  The {\em structural congruence} \cite{SangiorgiWalker} , $\equiv$,
  between processes is the least congruence containing
  alpha-equivalence, satisfying the abelian monoid laws
  (associativity, commutativity and $\pzero$ as identity) for parallel
  composition $|$ and for summation $+$.
\end{definition}

\subsection{Name equivalence}

We take name equivalence, written $\nameeq$, to be the smallest
equivalence relation generated by the following rules.

\begin{mathpar}
\inferrule*[lab=Quote-drop]
{ }
{ \quotep{@{x}} \nameeq x }

\inferrule*[lab=Struct-equiv]
{ P \scong Q }
{ \quotep{P} \nameeq \quotep{Q} }
\end{mathpar}

The astute reader will have noticed that the mutual recursion of names
and processes imposes a mutual recursion on alpha-equivalence and
structural equivalence via name-equivalence. Fortunately, all of this
works out pleasantly and we may calculate in the natural way, free of
concern. The reader interested in the details is referred to the
appendix \ref{appendix:rho_details}.

\subsection{Substitution}

We use $\Proc$ for the set of processes, $\QProc$ for the set of
names, and $\id{\{}\vec{y} / \vec{x} \id{\}}$ to denote partial maps,
$s : \QProc \rightarrow \QProc$. A map, $s$ lifts, uniquely, to a map
on process terms, $\widehat{s} : \Proc \rightarrow \Proc$ by the
following equations.

\begin{mathpar}
  (0) \psubstp{Q}{P} := 0 \\
  (R \juxtap S) \psubstp{Q}{P}
  :=    
  (R)\psubstp{Q}{P} \juxtap (S) \psubstp{Q}{P} \\
  (x?(y).R) \psubstp{Q}{P}    
  :=    
  (x)\substp{Q}{P} (z)\concat( (R \psubstn{z}{y}) \psubstp{Q}{P} ) \\
  (\lift{x}{R}) \psubstp{Q}{P}  
  :=
  \lift{(x)\substp{Q}{P}}{ R \psubstp{Q}{P} } \\
%   (\dropn{x})  \psubstp{Q}{P}       
%   := 
%   \left\{ 
%     \begin{array}{ccc} 
%       \dropn{\quotep{Q}} & & x \nameeq \quotep{P} \\
%       \dropn{x} & & otherwise \\
%     \end{array}
%   \right. 
  (\dropn{x})  \psubstp{Q}{P}       
  := 
  \left\{ 
    \begin{array}{ccc} 
      Q & & x \nameeq \quotep{P} \\
      \dropn{x} & & otherwise \\
    \end{array}
  \right.
\end{mathpar}
 

where

\begin{eqnarray}
  (x)\id{\{} \lpquote Q \rpquote / \lpquote P \rpquote \id{\}}            = 
  \left\{ 
    \begin{array}{ccc}
      \lpquote Q \rpquote & & x \nameeq \lpquote P \rpquote \\
      x & & otherwise \\
    \end{array}
  \right. \nonumber
\end{eqnarray}

and $z$ is chosen distinct from $\quotep{P}$, $\quotep{Q}$, the free
names in $Q$, and all the names in $R$. Our $\alpha$-equivalence will
be built in the standard way from this substitution.

\begin{remark}\label{rem:no_self_referential_names}
  One consequence of these definitions is that $\forall P. \quotep{P}
  \not\in \freenames{P}$.
\end{remark}

\subsection{ Dynamic quote: an example }

Anticipating something of what's to come, consider applying the
substitution, $\widehat{\id{\{}u / z \id{\}}}$, to the following pair
of processes, $\lift{w}{y!(z)}$ and $w[ \lpquote y!(z) \rpquote ]$.

\begin{eqnarray}
	\lift{w}{y!(z)}\widehat{\id{\{}u / z \id{\}}}
		& = &
		\lift{w}{y!(u)} \nonumber\\
	w[ \lpquote y!(z) \rpquote ] \widehat{ \id{\{}u / z \id{\}} }
		& = &
		w[ \lpquote y!(z) \rpquote ] \nonumber
\end{eqnarray}

Because the body of the process between quotes is impervious to
substitution, we get radically different answers. In fact, by
examining the first process in an input context,
e.g. $x?(z).\lift{w}{y!(z)}$, we see that the process under the lift
operator may be shaped by prefixed inputs binding a name inside it. In
this sense, the lift operator will be seen as a way to dynamically
construct processes before reifying them as names.

Finally equipped with these standard features we can present the
dynamics of the calculus.

\subsubsection{Operational semantics} 

Finally, we introduce the computational dynamics. What marks these
algebras as distinct from other more traditionally studied algebraic
structures, e.g. vector spaces or polynomial rings, is the manner in
which dynamics is captured. In traditional structures, dynamics is typically
expressed through morphisms between such structures, as in linear maps
between vector spaces or morphisms between rings. In algebras
associated with the semantics of computation, the dynamics is
expressed as part of the algebraic structure itself, through a
reduction reduction relation typically denoted by $\red$. Below, we
give a recursive presentation of this relation for the calculus used
in the encoding.

$\red \subseteq \pi \times \pi$
$\red : \pi \to \mathcal{P}(\pi)$

\begin{mathpar}
  \inferrule* [lab=Comm] { \textsf{match}( x_{src}, x_{trgt} ) } { x_{trgt}?(y)P \; | \; x_{src}!\langle {Q} \rangle \red P\{\quotep{Q}/y}\} }
  \and \\
  \inferrule* [lab=Par] {{P} \red {P}'} {{{P} | {Q}} \red {{P}' | {Q}}}
  \and
  \inferrule* [lab=Equiv]{{{P} \scong {P}'} \andalso {{P}' \red {Q}'} \andalso {{Q}' \scong {Q}}}{{P} \red {Q}}
\end{mathpar}

\begin{eqnarray*}
  match_{\equiv} (\quotep{P},\quotep{Q}) & := & P \equiv Q \\
  match_{\dagger}(\quotep{P},\quotep{Q}) & := & \forall R. P|Q \red^{*} R => R \red^{*} 0 \\
  match_{K}(\quotep{P},\quotep{Q}) & := & K \mbox{ for some context } K
\end{eqnarray*}

$u?(x)P | u!\langle Q \rangle \red P\{\quotep{Q}/x\}$

%We write $\wred$ for $\red^*$, and $P\red$ if $\exists Q $ such that $ P \red Q$.
We write $P\red$ if $\exists Q $ such that $ P \red Q$ and $P\not\red$, otherwise.

\section{Replication}

As mentioned before, it is known that replication (and hence
recursion) can be implemented in a higher-order process algebra
\cite{SangiorgiWalker}. As our first example of calculation with the
machinery thus far presented we give the construction explicitly in
the {\rhoc}.

\begin{eqnarray}
	D_{x} & := & \prefix{x}{y}{(\binpar{\outputp{x}{y}}{@{y}})} \nonumber\\
	\bangp_{x}{P} & := & \binpar{{x}!\langle{\binpar{D_{x}}{P}}\rangle}{D_{x}} \nonumber
\end{eqnarray}

\begin{eqnarray}
	\bangp_{x}{P} & & \nonumber\\
	=
	& {x}!\langle{(\prefix{x}{y}{(\outputp{x}{y} | @{y})) | P}}\rangle 
	      | \prefix{x}{y}{(\outputp{x}{y} | @{y})} & \nonumber\\
	\red
	& (\outputp{x}{y} | @{y})\substn{\quotep{(\prefix{x}{y}{(@{y} | \outputp{x}{y})) | P}}}{y} & \nonumber\\
	=
	& \outputp{x}{\quotep{(\prefix{x}{y}{(\outputp{x}{y} | @{y})) | P}}}
	  | {(\prefix{x}{y}{(\outputp{x}{y} | @{y})) | P}} & \nonumber\\
	\red
	& \ldots & \nonumber\\
	\red^*
	& P | P | \ldots & \nonumber
\end{eqnarray}

Of course, this encoding, as an implementation, runs away, unfolding
$\bangp{P}$ eagerly. A lazier and more implementable replication
operator, restricted to input-guarded processes, may be obtained as follows.

\begin{eqnarray}
\bangp{\prefix{u}{v}{P}} 
	:= 
	\binpar{\lift{x}{\prefix{u}{v}{(\binpar{D(x)}{P})}}}{D(x)} \nonumber
\end{eqnarray}

\begin{remark}
  Note that the lazier definition still does not deal with summation
  or mixed summation (i.e. sums over input and output). The reader is
  invited to construct definitions of replication that deal with these
  features. 

  Further, the definitions are parameterized in a name, $x$. Can you,
  gentle reader, make a definition that eliminates this parameter and
  guarantees no accidental interaction between the replication
  machinery and the process being replicated -- i.e. no accidental
  sharing of names used by the process to get its work done and the
  name(s) used by the replication to effect copying. This latter
  revision of the definition of replication is crucial to obtaining
  the expected identity $!!P \sim !P$.
\end{remark}

\begin{remark}\label{rem:paradoxical_combinator}
  The reader familiar with the lambda calculus will have noticed the
  similarity between $D$ and the paradoxical combinator.

  [Ed. note: the existence of this seems to suggest we have to be more
  restrictive on the set of processes and names we admit if we are to
  support no-cloning.]
\end{remark}

\subsubsection{Bisimulation}

The computational dynamics gives rise to another kind of equivalence,
the equivalence of computational behavior. As previously mentioned
this is typically captured \emph{via} some form of bisimulation.

% The notion we use in this paper is weak barbed bisimulation
% \cite{milner91polyadicpi}.

The notion we use in this paper is derived from weak barbed
bisimulation \cite{milner91polyadicpi}. 

\begin{definition}
An \emph{observation relation}, $\downarrow_{\mathcal N}$, over a set
of names, $\mathcal N$, is the smallest relation satisfying the rules
below.

\infrule[Out-barb]{y \in {\mathcal N}, \; x \nameeq y}
		  {\outputp{x}{v} \downarrow_{\mathcal N} x}
\infrule[Par-barb]{\mbox{$P\downarrow_{\mathcal N} x$ or $Q\downarrow_{\mathcal N} x$}}
		  {\binpar{P}{Q} \downarrow_{\mathcal N} x}

We write $P \Downarrow_{\mathcal N} x$ if there is $Q$ such that 
$P \wred Q$ and $Q \downarrow_{\mathcal N} x$.
\end{definition}

\begin{definition}
%\label{def.bbisim}
An  ${\mathcal N}$-\emph{barbed bisimulation} over a set of names, ${\mathcal N}$, is a symmetric binary relation 
${\mathcal S}_{\mathcal N}$ between agents such that $P\rel{S}_{\mathcal N}Q$ implies:
\begin{enumerate}
\item If $P \red P'$ then $Q \wred Q'$ and $P'\rel{S}_{\mathcal N} Q'$.
\item If $P\downarrow_{\mathcal N} x$, then $Q\Downarrow_{\mathcal N} x$.
\end{enumerate}
$P$ is ${\mathcal N}$-barbed bisimilar to $Q$, written
$P \wbbisim_{\mathcal N} Q$, if $P \rel{S}_{\mathcal N} Q$ for some ${\mathcal N}$-barbed bisimulation ${\mathcal S}_{\mathcal N}$.
\end{definition}

$\mathcal{R} \subseteq \pi \times \pi$

$P \mathcal{R} Q => \forall P'. P \red P' \Rightarrow \exists Q'. Q \red Q', P' \mathcal{R} Q'$

$P \vdash x \Rightarrow Q \vdash x$

\begin{mathpar}
  \inferrule*[lab=Out-barb]{x \nameeq y}{{y}!\langle{Q}\rangle \vdash x}
  \and
  \inferrule*[lab=Par-barb]{\mbox{$P\vdash x$ or $Q\vdash x$}}{\binpar{P}{Q} \vdash x}
\end{mathpar}

\subsubsection{Contexts}

One of the principle advantages of computational calculi like the
$\pi$-calculus is a well-defined notion of context,
contextual-equivalence and a correlation between
contextual-equivalence and notions of bisimulation. The notion of
context allows the decomposition of a process into (sub-)process and
its syntactic environment, its context. Thus, a context may be
thought of as a process with a ``hole'' (written $\Box$) in it. The
application of a context $M$ to a process $P$, written $M[P]$, is
tantamount to filling the hole in $M$ with $P$. In this paper we do
not need the full weight of this theory, but do make use of the notion
of context in the proof the main theorem. 

\begin{mathpar}
  \inferrule* [lab=summation] {} {{M_{M},M_{N}} \bc \Box \;|\; x.M_{A} \;|\; M_{M}+M_{N}}
  \and
  \inferrule* [lab=agent] {} {{M_{A}} \bc (\vec{x})M_{P} \;| \; \clift{P_0,\ldots,M_{P},\ldots,P_N}}
  \and \\
  \inferrule* [lab=process] {} {{M_{P}} \bc M_{N} \;| \;P|M_{P} }
\end{mathpar} 

\begin{mathpar}
  \inferrule* [lab=sychronization] {} {M_{N} \bc \Box \;|\; x?M_{F} \;|\; x!M_{C}}
  \and
  \inferrule* [lab=abstraction] {} {{M_{F}} \bc (x)M_{P} }
  \and
  \inferrule* [lab=concretion] {} {{M_{C}} \bc \langle M_{P} \rangle }
  \and \\
  \inferrule* [lab=process] {} {{M_{P}} \bc M_{N} \;| \;P|M_{P} }
\end{mathpar}

\begin{definition}[contextual application] Given a context $M$, and
  process $P$, we define the \emph{contextual application}, $M[P] :=
  M\{P/\Box\}$. That is, the contextual application of M to P is the
  substitution of $P$ for $\Box$ in $M$.
\end{definition}

$\meaningof{-} : L \to \mathcal{P}(\pi)$

\begin{mathpar}
  \inferrule* [lab=collection] {} {\meaningof{true} = \pi, \and \meaningof{~E} = \pi \setminus \meaningof{E}, \and \meaningof{E_{1} \& E_{2}} = \meaningof{E_{1}} \cap \meaningof{E_{2}}}
\end{mathpar}

\begin{mathpar}
  \inferrule* [lab=structure] {} {\meaningof{0} = \{ P \in \pi | P \equiv 0 \}, \and \\ \meaningof{E_1 | E_2} = \{ P \in \pi | P \equiv P_{1} | P_{2}, P_{1} \in \meaningof{E_{1}}, P_{2} \in \meaningof{E_2}\} }
\end{mathpar}

\begin{mathpar}
 \inferrule* [lab=behavior] {} {\meaningof{\langle a?b \rangle E} = \{ P \in \pi | P \equiv Q | u?(y)P', \\ \and \\\\ \and \\ \;\;\; u \in \meaningof{a}, \forall z.P'\{z/y\} \in \meaningof{E\{z/b\}}\}, \and \\ \meaningof{a!E} = \{ P \in \pi | P \equiv Q | x!\langle P' \rangle, x \in \meaningof{a} P' \in \meaningof{E}\} }
\end{mathpar}

\begin{mathpar}
 \inferrule* [lab=nominal] {} {\meaningof{\quotep{E}} = \{ \quotep{P} \in \quotep{\pi} | P \in \meaningof{E} \}, \and \meaningof{\quotep{P}} = \{ \quotep{Q} \in \quotep{\pi} | P \equiv Q \} \and \\ \meaningof{@\quotep{E}} = \{ P \in \pi | P \equiv @x, x \in \meaningof{E} \}}
\end{mathpar}

\begin{eqnarray*}
  \\
  \meaningof{-} : TS \to ST
\end{eqnarray*}

\begin{eqnarray*}
  \\
  L : TS \to ST
\end{eqnarray*}

\begin{eqnarray*}
  \\
  P \models E \iff P \in \meaningof{E}
\end{eqnarray*}

\begin{eqnarray*}
  P \approx_{L} Q \iff \forall E \in L. P \models E \iff Q \models E
\end{eqnarray*}

\begin{eqnarray*}
  P \approx_{K} Q
\end{eqnarray*}

\begin{eqnarray*}
  P \approx Q
\end{eqnarray*}

$\approx_{K} = \approx = \approx_{L}$

\subsubsection{Contextual duality}

Note that contexts extend the quotation operation to a family of
operations from processes to names. Given a context, $M$, we can
define a \emph{nominal context}, $\quotep{M}$ by $\quotep{M}[P] :=
\quotep{M[P]}$. To foreshadow what is to come we observe that these
operations enjoy a duality with processes very much like the duality
between vectors and maps from vectors to scalars.

Further, because the calculus is essentially higher-order, we have a
correspondence between contexts and processes. More specifically,
given a name $x$ and a context $M$ we can construct $M^{*}_{x}$ such
that 

\begin{mathpar}
  M^{*}_{x} | \lift{x}{P} \red M[P]
\end{mathpar}

namely,

\begin{mathpar}
  M^{*}_{x} := x?(u).M[\dropn{u}]
\end{mathpar}

The dependence of $M^{*}_{x}$ on a name makes it an abstraction, 

\begin{mathpar}
  M^{*} := (x)x?(u).M[\dropn{u}]
\end{mathpar}

\subsection{Additional notation}

It will sometimes be convenient to denote the process a name
quotes. We already have the notation $x = \quotep{P}$, but it will be
convenient to introduce an alternate notation, $\procn{x}$, when we
want to emphasize the connection to the use of the name. Note that, by
virtue of name equivalence, $\quotep{\procn{x}} \nameeq x$; so, the
notation is consistent with previous definitions.

Further, because names have structure it is possible to effect
substitutions on the basis of that structure. This means we need to
upgrade our notation for substitutions, which we accomplish by
adapting comprehension notation. Thus,

\begin{mathpar}
  P\{ y / x : x \in S \}
\end{mathpar}

is interpreted to mean the process derived from P by replacing (in a
capture-avoiding manner) each occurrence of $x$ in $S$ by $y$. For example,

\begin{mathpar}
  P\{ \quotep{\procn{x}|\procn{x}} / x : x \in \freenames{P} \}
\end{mathpar}

will replace each (occurrence) of a free name $x$ in $P$ by
$\quotep{\procn{x}|\procn{x}}$.

Also, we will avail ourselves of the notation $x^{L}$ and $x^{R}$ to
denote injections of a name into disjoint copies of the name
space. There are numerous ways to accomplish this. One example can be
found in \cite{MeredithR05}. This notation overloads to vectors of
names: $\vec{x}^{\pi} := (x_{i}^{\pi} \; : \; 0 \leq i < |\vec{x}| )$ where $\pi \in \{L,R\}$.

We also use $P^{\Box} := P|\Box$.

In \cite{MeredithR05} an interpretation of the new operator is
given. It turns out that there are several possible interpretations
all enjoying the requisite algebraic properties of the operator (see
\cite{milner91polyadicpi}). We will therefore make liberal use of
$(\nu\; \vec{x})P$.

% subsection the_syntax_and_semantics_of_the_notation_system (end)   

\input{qm2pi.qmops} 

\input{qm2pi.sterngerlach} 

\input{qm2pi.metric} 

% section concurrent_process_calculi (end)

%\input{qm2pi.proofsketch}

% section proof sketch (end)

%\input{qm2pi.slviaknots} 

% section spatial logic via knots (end)

\input{qm2pi.conclusion}

% section conclusion (end)

%\input{qm2pi.dtcodes} 

% section wiring algorithm (end)

\input{qm2pi.ack} 

% section acknowledgments (end)

\newpage


\bibliographystyle{plain}   
\bibliography{../../biblios/main.bib}

\input{qm2pi.rhodetails}

\end{document}

 

%\documentclass[12pt]{llncs}
%\documentclass{jktr}

\usepackage[pdftex]{hyperref}                   
\usepackage {listings}
\usepackage {mathpartir}
\usepackage{bcprules}
%\usepackage{listings}
                       
\usepackage{graphicx} 
%\usepackage[margins=2.5cm,nohead,nofoot]{geometry}
%\usepackage{geometry}
\usepackage{amsfonts}
\usepackage{amstext}
\usepackage{latexsym}
\usepackage{amssymb}
\usepackage{color}


%\include{myPreamble}
\include{qm2pi.local} 

%\ifpdf
%\usepackage[pdftex]{graphicx}
%\else
%\usepackage{graphicx}
%\fi

 % \ifpdf
%  \usepackage{pdfsync}
%  \if


%\title{Brief Article}
%\author{David F. Snyder}
%\author{L.G. Meredith}

%\address{Dept. of Math., Texas State University--San Marcos, San Marcos, TX 78666}
       
\pagestyle{empty}


\begin{document}

\lstset{language=[Objective]Caml,frame=shadowbox}

\input{qm2pi.front}

% section front matter (end)

\input{qm2pi.intro} 
 
% section introduction (end)

% \input{qm2pi.knotations} 

% section notation (end)

\input{qm2pi.process.calculi} 

% section concurrent_process_calculi_and_spatial_logics_ (end)
    
%\input{qm2pi.knots2pi} 

%\input{qm2pi.trefoil} 

%\input{qm2pi.mainthm} 

% subsection basic_interpretation (end)

%\input{qm2pi.rho.presentation} 
\subsection{The syntax and semantics of the notation system}\label{sub:the_syntax_and_semantics_of_the_notation_system} % (fold)

We now summarize a technical presentation of the calculus that
embodies our theory of dynamics. The typical presentation of such a
calculus follows the style of giving generators and relations on
them. The grammar, below, describing term constructors, freely
generates the set of processes, $\Proc$. This set is then quotiented
by a relation known as structural congruence and it is over this set
that the notion of dynamics is expressed. This presentation is
essentially that of \cite{MeredithR05} with the addition of
polyadicity and summation. For readability we have relegated some of
the technical subtleties to an appendix.

\subsubsection{Process grammar}\label{subsub:process_grammar}

\begin{mathpar}
  \inferrule* [lab=synchronization] {} {{M} \bc \pzero \;|\; x?F \;|\; x!C }
  \and
  \inferrule* [lab=abstraction] {} {{F} \bc (x)P}
  \and
  \inferrule* [lab=concretion] {} {{C} \bc \langle Q \rangle}
  \and
  \inferrule* [lab=process] {} {{P,Q} \bc M \;| \;P|Q \;|\; @{x}}
  \and
  \inferrule* [lab=name] {} {{x} \bc \quotep{P}}
\end{mathpar} 

Note that $\vec{x}$ (resp. $\vec{P}$) denotes a vector of names
(resp. processes) of length $|\vec{x}|$ (resp. $|\vec{P}|$). We adopt
the following useful abbreviations.

\begin{mathpar}
   x?(\vec{y}).P := x.(\vec{y})P \and  x\clift{\vec{P}} := x.\clift{\vec{P}}
   \and x!(y) := \lift{x}{\dropn{y}}
   \and \Pi_{i=0}^{n-1}P_i := P_0 | \ldots | P_{n-1}
\end{mathpar}

\subsubsection{Structural congruence}

\paragraph{Free and bound names and alpha-equivalence.} At the
core of structural equivalence is alpha-equivalence which identifies
process that are the same up to a change of variable. Formally, we
recognize the distinction between free and bound names. The free names
of a process, $\freenames{P}$, may be calculated recursively as
follows:

\begin{mathpar}
\freenames{\pzero} := \emptyset
  \and \\
  \freenames{x?(y).P} := \{ x \} \cup (\freenames{P} \setminus \{ y \})
  \and 
  \freenames{x!\langle P \rangle} := \{ x \} \cup \{ P \} 
  \and \\
  \freenames{P|Q} := \freenames{P} \cup \freenames{Q}
  \and \\
  \freenames{@{x}} := \{ x \}
\end{mathpar}

$\pi$
$\quotep{\pi}$

$\freenames{-} : \pi \to \mathcal{P}(\quotep{\pi})$

\begin{eqnarray*}
  \freenames{\pzero} & := & \emptyset \\
  \freenames{x?(y).P} & := & \{ x \} \cup (\freenames{P} \setminus \{ y \}) \\
  \freenames{x!\langle P \rangle} & := & \{ x \} \cup \{ P \} \\
  \freenames{P|Q} & := & \freenames{P} \cup \freenames{Q} \\
  \freenames{\dropn{x}} & := & \{ x \}
\end{eqnarray*}

The bound names of a process, $\boundnames{P}$, are those names occurring in $P$
that are not free. For example, in $x?(y).0$, the name $x$ is free, while $y$ is bound.

\begin{mathpar}
  \inferrule* [lab=monoidal-laws] {} { P|Q \equiv Q|P \and P|0 \equiv P \and P|(Q|R) \equiv (P|Q)|R }
\end{mathpar}

\begin{mathpar}
  \inferrule* [lab=alpha-equivalence] {} { (x)P \equiv (y)P\{y/x\} \and y \not\in \freenames{P} }
\end{mathpar}

\begin{definition}
Then two processes, $P,Q$, are alpha-equivalent if $P = Q\{\vec{y}/\vec{x}\}$ for
some $\vec{x} \in \boundnames{Q},\vec{y} \in \boundnames{P}$, where $Q\{\vec{y}/\vec{x}\}$
denotes the capture-avoiding substitution of $\vec{y}$ for $\vec{x}$ in $Q$.
\end{definition}

\begin{definition}
  The {\em structural congruence} \cite{SangiorgiWalker} , $\equiv$,
  between processes is the least congruence containing
  alpha-equivalence, satisfying the abelian monoid laws
  (associativity, commutativity and $\pzero$ as identity) for parallel
  composition $|$ and for summation $+$.
\end{definition}

\subsection{Name equivalence}

We take name equivalence, written $\nameeq$, to be the smallest
equivalence relation generated by the following rules.

\begin{mathpar}
\inferrule*[lab=Quote-drop]
{ }
{ \quotep{@{x}} \nameeq x }

\inferrule*[lab=Struct-equiv]
{ P \scong Q }
{ \quotep{P} \nameeq \quotep{Q} }
\end{mathpar}

The astute reader will have noticed that the mutual recursion of names
and processes imposes a mutual recursion on alpha-equivalence and
structural equivalence via name-equivalence. Fortunately, all of this
works out pleasantly and we may calculate in the natural way, free of
concern. The reader interested in the details is referred to the
appendix \ref{appendix:rho_details}.

\subsection{Substitution}

We use $\Proc$ for the set of processes, $\QProc$ for the set of
names, and $\id{\{}\vec{y} / \vec{x} \id{\}}$ to denote partial maps,
$s : \QProc \rightarrow \QProc$. A map, $s$ lifts, uniquely, to a map
on process terms, $\widehat{s} : \Proc \rightarrow \Proc$ by the
following equations.

\begin{mathpar}
  (0) \psubstp{Q}{P} := 0 \\
  (R \juxtap S) \psubstp{Q}{P}
  :=    
  (R)\psubstp{Q}{P} \juxtap (S) \psubstp{Q}{P} \\
  (x?(y).R) \psubstp{Q}{P}    
  :=    
  (x)\substp{Q}{P} (z)\concat( (R \psubstn{z}{y}) \psubstp{Q}{P} ) \\
  (\lift{x}{R}) \psubstp{Q}{P}  
  :=
  \lift{(x)\substp{Q}{P}}{ R \psubstp{Q}{P} } \\
%   (\dropn{x})  \psubstp{Q}{P}       
%   := 
%   \left\{ 
%     \begin{array}{ccc} 
%       \dropn{\quotep{Q}} & & x \nameeq \quotep{P} \\
%       \dropn{x} & & otherwise \\
%     \end{array}
%   \right. 
  (\dropn{x})  \psubstp{Q}{P}       
  := 
  \left\{ 
    \begin{array}{ccc} 
      Q & & x \nameeq \quotep{P} \\
      \dropn{x} & & otherwise \\
    \end{array}
  \right.
\end{mathpar}
 

where

\begin{eqnarray}
  (x)\id{\{} \lpquote Q \rpquote / \lpquote P \rpquote \id{\}}            = 
  \left\{ 
    \begin{array}{ccc}
      \lpquote Q \rpquote & & x \nameeq \lpquote P \rpquote \\
      x & & otherwise \\
    \end{array}
  \right. \nonumber
\end{eqnarray}

and $z$ is chosen distinct from $\quotep{P}$, $\quotep{Q}$, the free
names in $Q$, and all the names in $R$. Our $\alpha$-equivalence will
be built in the standard way from this substitution.

\begin{remark}\label{rem:no_self_referential_names}
  One consequence of these definitions is that $\forall P. \quotep{P}
  \not\in \freenames{P}$.
\end{remark}

\subsection{ Dynamic quote: an example }

Anticipating something of what's to come, consider applying the
substitution, $\widehat{\id{\{}u / z \id{\}}}$, to the following pair
of processes, $\lift{w}{y!(z)}$ and $w[ \lpquote y!(z) \rpquote ]$.

\begin{eqnarray}
	\lift{w}{y!(z)}\widehat{\id{\{}u / z \id{\}}}
		& = &
		\lift{w}{y!(u)} \nonumber\\
	w[ \lpquote y!(z) \rpquote ] \widehat{ \id{\{}u / z \id{\}} }
		& = &
		w[ \lpquote y!(z) \rpquote ] \nonumber
\end{eqnarray}

Because the body of the process between quotes is impervious to
substitution, we get radically different answers. In fact, by
examining the first process in an input context,
e.g. $x?(z).\lift{w}{y!(z)}$, we see that the process under the lift
operator may be shaped by prefixed inputs binding a name inside it. In
this sense, the lift operator will be seen as a way to dynamically
construct processes before reifying them as names.

Finally equipped with these standard features we can present the
dynamics of the calculus.

\subsubsection{Operational semantics} 

Finally, we introduce the computational dynamics. What marks these
algebras as distinct from other more traditionally studied algebraic
structures, e.g. vector spaces or polynomial rings, is the manner in
which dynamics is captured. In traditional structures, dynamics is typically
expressed through morphisms between such structures, as in linear maps
between vector spaces or morphisms between rings. In algebras
associated with the semantics of computation, the dynamics is
expressed as part of the algebraic structure itself, through a
reduction reduction relation typically denoted by $\red$. Below, we
give a recursive presentation of this relation for the calculus used
in the encoding.

$\red \subseteq \pi \times \pi$
$\red : \pi \to \mathcal{P}(\pi)$

\begin{mathpar}
  \inferrule* [lab=Comm] { \textsf{match}( x_{src}, x_{trgt} ) } { x_{trgt}?(y)P \; | \; x_{src}!\langle {Q} \rangle \red P\{\quotep{Q}/y}\} }
  \and \\
  \inferrule* [lab=Par] {{P} \red {P}'} {{{P} | {Q}} \red {{P}' | {Q}}}
  \and
  \inferrule* [lab=Equiv]{{{P} \scong {P}'} \andalso {{P}' \red {Q}'} \andalso {{Q}' \scong {Q}}}{{P} \red {Q}}
\end{mathpar}

\begin{eqnarray*}
  match_{\equiv} (\quotep{P},\quotep{Q}) & := & P \equiv Q \\
  match_{\dagger}(\quotep{P},\quotep{Q}) & := & \forall R. P|Q \red^{*} R => R \red^{*} 0 \\
  match_{K}(\quotep{P},\quotep{Q}) & := & K \mbox{ for some context } K
\end{eqnarray*}

$u?(x)P | u!\langle Q \rangle \red P\{\quotep{Q}/x\}$

%We write $\wred$ for $\red^*$, and $P\red$ if $\exists Q $ such that $ P \red Q$.
We write $P\red$ if $\exists Q $ such that $ P \red Q$ and $P\not\red$, otherwise.

\section{Replication}

As mentioned before, it is known that replication (and hence
recursion) can be implemented in a higher-order process algebra
\cite{SangiorgiWalker}. As our first example of calculation with the
machinery thus far presented we give the construction explicitly in
the {\rhoc}.

\begin{eqnarray}
	D_{x} & := & \prefix{x}{y}{(\binpar{\outputp{x}{y}}{@{y}})} \nonumber\\
	\bangp_{x}{P} & := & \binpar{{x}!\langle{\binpar{D_{x}}{P}}\rangle}{D_{x}} \nonumber
\end{eqnarray}

\begin{eqnarray}
	\bangp_{x}{P} & & \nonumber\\
	=
	& {x}!\langle{(\prefix{x}{y}{(\outputp{x}{y} | @{y})) | P}}\rangle 
	      | \prefix{x}{y}{(\outputp{x}{y} | @{y})} & \nonumber\\
	\red
	& (\outputp{x}{y} | @{y})\substn{\quotep{(\prefix{x}{y}{(@{y} | \outputp{x}{y})) | P}}}{y} & \nonumber\\
	=
	& \outputp{x}{\quotep{(\prefix{x}{y}{(\outputp{x}{y} | @{y})) | P}}}
	  | {(\prefix{x}{y}{(\outputp{x}{y} | @{y})) | P}} & \nonumber\\
	\red
	& \ldots & \nonumber\\
	\red^*
	& P | P | \ldots & \nonumber
\end{eqnarray}

Of course, this encoding, as an implementation, runs away, unfolding
$\bangp{P}$ eagerly. A lazier and more implementable replication
operator, restricted to input-guarded processes, may be obtained as follows.

\begin{eqnarray}
\bangp{\prefix{u}{v}{P}} 
	:= 
	\binpar{\lift{x}{\prefix{u}{v}{(\binpar{D(x)}{P})}}}{D(x)} \nonumber
\end{eqnarray}

\begin{remark}
  Note that the lazier definition still does not deal with summation
  or mixed summation (i.e. sums over input and output). The reader is
  invited to construct definitions of replication that deal with these
  features. 

  Further, the definitions are parameterized in a name, $x$. Can you,
  gentle reader, make a definition that eliminates this parameter and
  guarantees no accidental interaction between the replication
  machinery and the process being replicated -- i.e. no accidental
  sharing of names used by the process to get its work done and the
  name(s) used by the replication to effect copying. This latter
  revision of the definition of replication is crucial to obtaining
  the expected identity $!!P \sim !P$.
\end{remark}

\begin{remark}\label{rem:paradoxical_combinator}
  The reader familiar with the lambda calculus will have noticed the
  similarity between $D$ and the paradoxical combinator.

  [Ed. note: the existence of this seems to suggest we have to be more
  restrictive on the set of processes and names we admit if we are to
  support no-cloning.]
\end{remark}

\subsubsection{Bisimulation}

The computational dynamics gives rise to another kind of equivalence,
the equivalence of computational behavior. As previously mentioned
this is typically captured \emph{via} some form of bisimulation.

% The notion we use in this paper is weak barbed bisimulation
% \cite{milner91polyadicpi}.

The notion we use in this paper is derived from weak barbed
bisimulation \cite{milner91polyadicpi}. 

\begin{definition}
An \emph{observation relation}, $\downarrow_{\mathcal N}$, over a set
of names, $\mathcal N$, is the smallest relation satisfying the rules
below.

\infrule[Out-barb]{y \in {\mathcal N}, \; x \nameeq y}
		  {\outputp{x}{v} \downarrow_{\mathcal N} x}
\infrule[Par-barb]{\mbox{$P\downarrow_{\mathcal N} x$ or $Q\downarrow_{\mathcal N} x$}}
		  {\binpar{P}{Q} \downarrow_{\mathcal N} x}

We write $P \Downarrow_{\mathcal N} x$ if there is $Q$ such that 
$P \wred Q$ and $Q \downarrow_{\mathcal N} x$.
\end{definition}

\begin{definition}
%\label{def.bbisim}
An  ${\mathcal N}$-\emph{barbed bisimulation} over a set of names, ${\mathcal N}$, is a symmetric binary relation 
${\mathcal S}_{\mathcal N}$ between agents such that $P\rel{S}_{\mathcal N}Q$ implies:
\begin{enumerate}
\item If $P \red P'$ then $Q \wred Q'$ and $P'\rel{S}_{\mathcal N} Q'$.
\item If $P\downarrow_{\mathcal N} x$, then $Q\Downarrow_{\mathcal N} x$.
\end{enumerate}
$P$ is ${\mathcal N}$-barbed bisimilar to $Q$, written
$P \wbbisim_{\mathcal N} Q$, if $P \rel{S}_{\mathcal N} Q$ for some ${\mathcal N}$-barbed bisimulation ${\mathcal S}_{\mathcal N}$.
\end{definition}

$\mathcal{R} \subseteq \pi \times \pi$

$P \mathcal{R} Q => \forall P'. P \red P' \Rightarrow \exists Q'. Q \red Q', P' \mathcal{R} Q'$

$P \vdash x \Rightarrow Q \vdash x$

\begin{mathpar}
  \inferrule*[lab=Out-barb]{x \nameeq y}{{y}!\langle{Q}\rangle \vdash x}
  \and
  \inferrule*[lab=Par-barb]{\mbox{$P\vdash x$ or $Q\vdash x$}}{\binpar{P}{Q} \vdash x}
\end{mathpar}

\subsubsection{Contexts}

One of the principle advantages of computational calculi like the
$\pi$-calculus is a well-defined notion of context,
contextual-equivalence and a correlation between
contextual-equivalence and notions of bisimulation. The notion of
context allows the decomposition of a process into (sub-)process and
its syntactic environment, its context. Thus, a context may be
thought of as a process with a ``hole'' (written $\Box$) in it. The
application of a context $M$ to a process $P$, written $M[P]$, is
tantamount to filling the hole in $M$ with $P$. In this paper we do
not need the full weight of this theory, but do make use of the notion
of context in the proof the main theorem. 

\begin{mathpar}
  \inferrule* [lab=summation] {} {{M_{M},M_{N}} \bc \Box \;|\; x.M_{A} \;|\; M_{M}+M_{N}}
  \and
  \inferrule* [lab=agent] {} {{M_{A}} \bc (\vec{x})M_{P} \;| \; \clift{P_0,\ldots,M_{P},\ldots,P_N}}
  \and \\
  \inferrule* [lab=process] {} {{M_{P}} \bc M_{N} \;| \;P|M_{P} }
\end{mathpar} 

\begin{mathpar}
  \inferrule* [lab=sychronization] {} {M_{N} \bc \Box \;|\; x?M_{F} \;|\; x!M_{C}}
  \and
  \inferrule* [lab=abstraction] {} {{M_{F}} \bc (x)M_{P} }
  \and
  \inferrule* [lab=concretion] {} {{M_{C}} \bc \langle M_{P} \rangle }
  \and \\
  \inferrule* [lab=process] {} {{M_{P}} \bc M_{N} \;| \;P|M_{P} }
\end{mathpar}

\begin{definition}[contextual application] Given a context $M$, and
  process $P$, we define the \emph{contextual application}, $M[P] :=
  M\{P/\Box\}$. That is, the contextual application of M to P is the
  substitution of $P$ for $\Box$ in $M$.
\end{definition}

$\meaningof{-} : L \to \mathcal{P}(\pi)$

\begin{mathpar}
  \inferrule* [lab=collection] {} {\meaningof{true} = \pi, \and \meaningof{~E} = \pi \setminus \meaningof{E}, \and \meaningof{E_{1} \& E_{2}} = \meaningof{E_{1}} \cap \meaningof{E_{2}}}
\end{mathpar}

\begin{mathpar}
  \inferrule* [lab=structure] {} {\meaningof{0} = \{ P \in \pi | P \equiv 0 \}, \and \\ \meaningof{E_1 | E_2} = \{ P \in \pi | P \equiv P_{1} | P_{2}, P_{1} \in \meaningof{E_{1}}, P_{2} \in \meaningof{E_2}\} }
\end{mathpar}

\begin{mathpar}
 \inferrule* [lab=behavior] {} {\meaningof{\langle a?b \rangle E} = \{ P \in \pi | P \equiv Q | u?(y)P', \\ \and \\\\ \and \\ \;\;\; u \in \meaningof{a}, \forall z.P'\{z/y\} \in \meaningof{E\{z/b\}}\}, \and \\ \meaningof{a!E} = \{ P \in \pi | P \equiv Q | x!\langle P' \rangle, x \in \meaningof{a} P' \in \meaningof{E}\} }
\end{mathpar}

\begin{mathpar}
 \inferrule* [lab=nominal] {} {\meaningof{\quotep{E}} = \{ \quotep{P} \in \quotep{\pi} | P \in \meaningof{E} \}, \and \meaningof{\quotep{P}} = \{ \quotep{Q} \in \quotep{\pi} | P \equiv Q \} \and \\ \meaningof{@\quotep{E}} = \{ P \in \pi | P \equiv @x, x \in \meaningof{E} \}}
\end{mathpar}

\begin{eqnarray*}
  \\
  \meaningof{-} : TS \to ST
\end{eqnarray*}

\begin{eqnarray*}
  \\
  L : TS \to ST
\end{eqnarray*}

\begin{eqnarray*}
  \\
  P \models E \iff P \in \meaningof{E}
\end{eqnarray*}

\begin{eqnarray*}
  P \approx_{L} Q \iff \forall E \in L. P \models E \iff Q \models E
\end{eqnarray*}

\begin{eqnarray*}
  P \approx_{K} Q
\end{eqnarray*}

\begin{eqnarray*}
  P \approx Q
\end{eqnarray*}

$\approx_{K} = \approx = \approx_{L}$

\subsubsection{Contextual duality}

Note that contexts extend the quotation operation to a family of
operations from processes to names. Given a context, $M$, we can
define a \emph{nominal context}, $\quotep{M}$ by $\quotep{M}[P] :=
\quotep{M[P]}$. To foreshadow what is to come we observe that these
operations enjoy a duality with processes very much like the duality
between vectors and maps from vectors to scalars.

Further, because the calculus is essentially higher-order, we have a
correspondence between contexts and processes. More specifically,
given a name $x$ and a context $M$ we can construct $M^{*}_{x}$ such
that 

\begin{mathpar}
  M^{*}_{x} | \lift{x}{P} \red M[P]
\end{mathpar}

namely,

\begin{mathpar}
  M^{*}_{x} := x?(u).M[\dropn{u}]
\end{mathpar}

The dependence of $M^{*}_{x}$ on a name makes it an abstraction, 

\begin{mathpar}
  M^{*} := (x)x?(u).M[\dropn{u}]
\end{mathpar}

\subsection{Additional notation}

It will sometimes be convenient to denote the process a name
quotes. We already have the notation $x = \quotep{P}$, but it will be
convenient to introduce an alternate notation, $\procn{x}$, when we
want to emphasize the connection to the use of the name. Note that, by
virtue of name equivalence, $\quotep{\procn{x}} \nameeq x$; so, the
notation is consistent with previous definitions.

Further, because names have structure it is possible to effect
substitutions on the basis of that structure. This means we need to
upgrade our notation for substitutions, which we accomplish by
adapting comprehension notation. Thus,

\begin{mathpar}
  P\{ y / x : x \in S \}
\end{mathpar}

is interpreted to mean the process derived from P by replacing (in a
capture-avoiding manner) each occurrence of $x$ in $S$ by $y$. For example,

\begin{mathpar}
  P\{ \quotep{\procn{x}|\procn{x}} / x : x \in \freenames{P} \}
\end{mathpar}

will replace each (occurrence) of a free name $x$ in $P$ by
$\quotep{\procn{x}|\procn{x}}$.

Also, we will avail ourselves of the notation $x^{L}$ and $x^{R}$ to
denote injections of a name into disjoint copies of the name
space. There are numerous ways to accomplish this. One example can be
found in \cite{MeredithR05}. This notation overloads to vectors of
names: $\vec{x}^{\pi} := (x_{i}^{\pi} \; : \; 0 \leq i < |\vec{x}| )$ where $\pi \in \{L,R\}$.

We also use $P^{\Box} := P|\Box$.

In \cite{MeredithR05} an interpretation of the new operator is
given. It turns out that there are several possible interpretations
all enjoying the requisite algebraic properties of the operator (see
\cite{milner91polyadicpi}). We will therefore make liberal use of
$(\nu\; \vec{x})P$.

% subsection the_syntax_and_semantics_of_the_notation_system (end)   

\input{qm2pi.qmops} 

\input{qm2pi.sterngerlach} 

\input{qm2pi.metric} 

% section concurrent_process_calculi (end)

%\input{qm2pi.proofsketch}

% section proof sketch (end)

%\input{qm2pi.slviaknots} 

% section spatial logic via knots (end)

\input{qm2pi.conclusion}

% section conclusion (end)

%\input{qm2pi.dtcodes} 

% section wiring algorithm (end)

\input{qm2pi.ack} 

% section acknowledgments (end)

\newpage


\bibliographystyle{plain}   
\bibliography{../../biblios/main.bib}

\input{qm2pi.rhodetails}

\end{document}

 

% subsection basic_interpretation (end)

%\input{qm2pi.rho.presentation} 
\subsection{The syntax and semantics of the notation system}\label{sub:the_syntax_and_semantics_of_the_notation_system} % (fold)

We now summarize a technical presentation of the calculus that
embodies our theory of dynamics. The typical presentation of such a
calculus follows the style of giving generators and relations on
them. The grammar, below, describing term constructors, freely
generates the set of processes, $\Proc$. This set is then quotiented
by a relation known as structural congruence and it is over this set
that the notion of dynamics is expressed. This presentation is
essentially that of \cite{MeredithR05} with the addition of
polyadicity and summation. For readability we have relegated some of
the technical subtleties to an appendix.

\subsubsection{Process grammar}\label{subsub:process_grammar}

\begin{mathpar}
  \inferrule* [lab=synchronization] {} {{M} \bc \pzero \;|\; x?F \;|\; x!C }
  \and
  \inferrule* [lab=abstraction] {} {{F} \bc (x)P}
  \and
  \inferrule* [lab=concretion] {} {{C} \bc \langle Q \rangle}
  \and
  \inferrule* [lab=process] {} {{P,Q} \bc M \;| \;P|Q \;|\; @{x}}
  \and
  \inferrule* [lab=name] {} {{x} \bc \quotep{P}}
\end{mathpar} 

Note that $\vec{x}$ (resp. $\vec{P}$) denotes a vector of names
(resp. processes) of length $|\vec{x}|$ (resp. $|\vec{P}|$). We adopt
the following useful abbreviations.

\begin{mathpar}
   x?(\vec{y}).P := x.(\vec{y})P \and  x\clift{\vec{P}} := x.\clift{\vec{P}}
   \and x!(y) := \lift{x}{\dropn{y}}
   \and \Pi_{i=0}^{n-1}P_i := P_0 | \ldots | P_{n-1}
\end{mathpar}

\subsubsection{Structural congruence}

\paragraph{Free and bound names and alpha-equivalence.} At the
core of structural equivalence is alpha-equivalence which identifies
process that are the same up to a change of variable. Formally, we
recognize the distinction between free and bound names. The free names
of a process, $\freenames{P}$, may be calculated recursively as
follows:

\begin{mathpar}
\freenames{\pzero} := \emptyset
  \and \\
  \freenames{x?(y).P} := \{ x \} \cup (\freenames{P} \setminus \{ y \})
  \and 
  \freenames{x!\langle P \rangle} := \{ x \} \cup \{ P \} 
  \and \\
  \freenames{P|Q} := \freenames{P} \cup \freenames{Q}
  \and \\
  \freenames{@{x}} := \{ x \}
\end{mathpar}

$\pi$
$\quotep{\pi}$

$\freenames{-} : \pi \to \mathcal{P}(\quotep{\pi})$

\begin{eqnarray*}
  \freenames{\pzero} & := & \emptyset \\
  \freenames{x?(y).P} & := & \{ x \} \cup (\freenames{P} \setminus \{ y \}) \\
  \freenames{x!\langle P \rangle} & := & \{ x \} \cup \{ P \} \\
  \freenames{P|Q} & := & \freenames{P} \cup \freenames{Q} \\
  \freenames{\dropn{x}} & := & \{ x \}
\end{eqnarray*}

The bound names of a process, $\boundnames{P}$, are those names occurring in $P$
that are not free. For example, in $x?(y).0$, the name $x$ is free, while $y$ is bound.

\begin{mathpar}
  \inferrule* [lab=monoidal-laws] {} { P|Q \equiv Q|P \and P|0 \equiv P \and P|(Q|R) \equiv (P|Q)|R }
\end{mathpar}

\begin{mathpar}
  \inferrule* [lab=alpha-equivalence] {} { (x)P \equiv (y)P\{y/x\} \and y \not\in \freenames{P} }
\end{mathpar}

\begin{definition}
Then two processes, $P,Q$, are alpha-equivalent if $P = Q\{\vec{y}/\vec{x}\}$ for
some $\vec{x} \in \boundnames{Q},\vec{y} \in \boundnames{P}$, where $Q\{\vec{y}/\vec{x}\}$
denotes the capture-avoiding substitution of $\vec{y}$ for $\vec{x}$ in $Q$.
\end{definition}

\begin{definition}
  The {\em structural congruence} \cite{SangiorgiWalker} , $\equiv$,
  between processes is the least congruence containing
  alpha-equivalence, satisfying the abelian monoid laws
  (associativity, commutativity and $\pzero$ as identity) for parallel
  composition $|$ and for summation $+$.
\end{definition}

\subsection{Name equivalence}

We take name equivalence, written $\nameeq$, to be the smallest
equivalence relation generated by the following rules.

\begin{mathpar}
\inferrule*[lab=Quote-drop]
{ }
{ \quotep{@{x}} \nameeq x }

\inferrule*[lab=Struct-equiv]
{ P \scong Q }
{ \quotep{P} \nameeq \quotep{Q} }
\end{mathpar}

The astute reader will have noticed that the mutual recursion of names
and processes imposes a mutual recursion on alpha-equivalence and
structural equivalence via name-equivalence. Fortunately, all of this
works out pleasantly and we may calculate in the natural way, free of
concern. The reader interested in the details is referred to the
appendix \ref{appendix:rho_details}.

\subsection{Substitution}

We use $\Proc$ for the set of processes, $\QProc$ for the set of
names, and $\id{\{}\vec{y} / \vec{x} \id{\}}$ to denote partial maps,
$s : \QProc \rightarrow \QProc$. A map, $s$ lifts, uniquely, to a map
on process terms, $\widehat{s} : \Proc \rightarrow \Proc$ by the
following equations.

\begin{mathpar}
  (0) \psubstp{Q}{P} := 0 \\
  (R \juxtap S) \psubstp{Q}{P}
  :=    
  (R)\psubstp{Q}{P} \juxtap (S) \psubstp{Q}{P} \\
  (x?(y).R) \psubstp{Q}{P}    
  :=    
  (x)\substp{Q}{P} (z)\concat( (R \psubstn{z}{y}) \psubstp{Q}{P} ) \\
  (\lift{x}{R}) \psubstp{Q}{P}  
  :=
  \lift{(x)\substp{Q}{P}}{ R \psubstp{Q}{P} } \\
%   (\dropn{x})  \psubstp{Q}{P}       
%   := 
%   \left\{ 
%     \begin{array}{ccc} 
%       \dropn{\quotep{Q}} & & x \nameeq \quotep{P} \\
%       \dropn{x} & & otherwise \\
%     \end{array}
%   \right. 
  (\dropn{x})  \psubstp{Q}{P}       
  := 
  \left\{ 
    \begin{array}{ccc} 
      Q & & x \nameeq \quotep{P} \\
      \dropn{x} & & otherwise \\
    \end{array}
  \right.
\end{mathpar}
 

where

\begin{eqnarray}
  (x)\id{\{} \lpquote Q \rpquote / \lpquote P \rpquote \id{\}}            = 
  \left\{ 
    \begin{array}{ccc}
      \lpquote Q \rpquote & & x \nameeq \lpquote P \rpquote \\
      x & & otherwise \\
    \end{array}
  \right. \nonumber
\end{eqnarray}

and $z$ is chosen distinct from $\quotep{P}$, $\quotep{Q}$, the free
names in $Q$, and all the names in $R$. Our $\alpha$-equivalence will
be built in the standard way from this substitution.

\begin{remark}\label{rem:no_self_referential_names}
  One consequence of these definitions is that $\forall P. \quotep{P}
  \not\in \freenames{P}$.
\end{remark}

\subsection{ Dynamic quote: an example }

Anticipating something of what's to come, consider applying the
substitution, $\widehat{\id{\{}u / z \id{\}}}$, to the following pair
of processes, $\lift{w}{y!(z)}$ and $w[ \lpquote y!(z) \rpquote ]$.

\begin{eqnarray}
	\lift{w}{y!(z)}\widehat{\id{\{}u / z \id{\}}}
		& = &
		\lift{w}{y!(u)} \nonumber\\
	w[ \lpquote y!(z) \rpquote ] \widehat{ \id{\{}u / z \id{\}} }
		& = &
		w[ \lpquote y!(z) \rpquote ] \nonumber
\end{eqnarray}

Because the body of the process between quotes is impervious to
substitution, we get radically different answers. In fact, by
examining the first process in an input context,
e.g. $x?(z).\lift{w}{y!(z)}$, we see that the process under the lift
operator may be shaped by prefixed inputs binding a name inside it. In
this sense, the lift operator will be seen as a way to dynamically
construct processes before reifying them as names.

Finally equipped with these standard features we can present the
dynamics of the calculus.

\subsubsection{Operational semantics} 

Finally, we introduce the computational dynamics. What marks these
algebras as distinct from other more traditionally studied algebraic
structures, e.g. vector spaces or polynomial rings, is the manner in
which dynamics is captured. In traditional structures, dynamics is typically
expressed through morphisms between such structures, as in linear maps
between vector spaces or morphisms between rings. In algebras
associated with the semantics of computation, the dynamics is
expressed as part of the algebraic structure itself, through a
reduction reduction relation typically denoted by $\red$. Below, we
give a recursive presentation of this relation for the calculus used
in the encoding.

$\red \subseteq \pi \times \pi$
$\red : \pi \to \mathcal{P}(\pi)$

\begin{mathpar}
  \inferrule* [lab=Comm] { \textsf{match}( x_{src}, x_{trgt} ) } { x_{trgt}?(y)P \; | \; x_{src}!\langle {Q} \rangle \red P\{\quotep{Q}/y}\} }
  \and \\
  \inferrule* [lab=Par] {{P} \red {P}'} {{{P} | {Q}} \red {{P}' | {Q}}}
  \and
  \inferrule* [lab=Equiv]{{{P} \scong {P}'} \andalso {{P}' \red {Q}'} \andalso {{Q}' \scong {Q}}}{{P} \red {Q}}
\end{mathpar}

\begin{eqnarray*}
  match_{\equiv} (\quotep{P},\quotep{Q}) & := & P \equiv Q \\
  match_{\dagger}(\quotep{P},\quotep{Q}) & := & \forall R. P|Q \red^{*} R => R \red^{*} 0 \\
  match_{K}(\quotep{P},\quotep{Q}) & := & K \mbox{ for some context } K
\end{eqnarray*}

$u?(x)P | u!\langle Q \rangle \red P\{\quotep{Q}/x\}$

%We write $\wred$ for $\red^*$, and $P\red$ if $\exists Q $ such that $ P \red Q$.
We write $P\red$ if $\exists Q $ such that $ P \red Q$ and $P\not\red$, otherwise.

\section{Replication}

As mentioned before, it is known that replication (and hence
recursion) can be implemented in a higher-order process algebra
\cite{SangiorgiWalker}. As our first example of calculation with the
machinery thus far presented we give the construction explicitly in
the {\rhoc}.

\begin{eqnarray}
	D_{x} & := & \prefix{x}{y}{(\binpar{\outputp{x}{y}}{@{y}})} \nonumber\\
	\bangp_{x}{P} & := & \binpar{{x}!\langle{\binpar{D_{x}}{P}}\rangle}{D_{x}} \nonumber
\end{eqnarray}

\begin{eqnarray}
	\bangp_{x}{P} & & \nonumber\\
	=
	& {x}!\langle{(\prefix{x}{y}{(\outputp{x}{y} | @{y})) | P}}\rangle 
	      | \prefix{x}{y}{(\outputp{x}{y} | @{y})} & \nonumber\\
	\red
	& (\outputp{x}{y} | @{y})\substn{\quotep{(\prefix{x}{y}{(@{y} | \outputp{x}{y})) | P}}}{y} & \nonumber\\
	=
	& \outputp{x}{\quotep{(\prefix{x}{y}{(\outputp{x}{y} | @{y})) | P}}}
	  | {(\prefix{x}{y}{(\outputp{x}{y} | @{y})) | P}} & \nonumber\\
	\red
	& \ldots & \nonumber\\
	\red^*
	& P | P | \ldots & \nonumber
\end{eqnarray}

Of course, this encoding, as an implementation, runs away, unfolding
$\bangp{P}$ eagerly. A lazier and more implementable replication
operator, restricted to input-guarded processes, may be obtained as follows.

\begin{eqnarray}
\bangp{\prefix{u}{v}{P}} 
	:= 
	\binpar{\lift{x}{\prefix{u}{v}{(\binpar{D(x)}{P})}}}{D(x)} \nonumber
\end{eqnarray}

\begin{remark}
  Note that the lazier definition still does not deal with summation
  or mixed summation (i.e. sums over input and output). The reader is
  invited to construct definitions of replication that deal with these
  features. 

  Further, the definitions are parameterized in a name, $x$. Can you,
  gentle reader, make a definition that eliminates this parameter and
  guarantees no accidental interaction between the replication
  machinery and the process being replicated -- i.e. no accidental
  sharing of names used by the process to get its work done and the
  name(s) used by the replication to effect copying. This latter
  revision of the definition of replication is crucial to obtaining
  the expected identity $!!P \sim !P$.
\end{remark}

\begin{remark}\label{rem:paradoxical_combinator}
  The reader familiar with the lambda calculus will have noticed the
  similarity between $D$ and the paradoxical combinator.

  [Ed. note: the existence of this seems to suggest we have to be more
  restrictive on the set of processes and names we admit if we are to
  support no-cloning.]
\end{remark}

\subsubsection{Bisimulation}

The computational dynamics gives rise to another kind of equivalence,
the equivalence of computational behavior. As previously mentioned
this is typically captured \emph{via} some form of bisimulation.

% The notion we use in this paper is weak barbed bisimulation
% \cite{milner91polyadicpi}.

The notion we use in this paper is derived from weak barbed
bisimulation \cite{milner91polyadicpi}. 

\begin{definition}
An \emph{observation relation}, $\downarrow_{\mathcal N}$, over a set
of names, $\mathcal N$, is the smallest relation satisfying the rules
below.

\infrule[Out-barb]{y \in {\mathcal N}, \; x \nameeq y}
		  {\outputp{x}{v} \downarrow_{\mathcal N} x}
\infrule[Par-barb]{\mbox{$P\downarrow_{\mathcal N} x$ or $Q\downarrow_{\mathcal N} x$}}
		  {\binpar{P}{Q} \downarrow_{\mathcal N} x}

We write $P \Downarrow_{\mathcal N} x$ if there is $Q$ such that 
$P \wred Q$ and $Q \downarrow_{\mathcal N} x$.
\end{definition}

\begin{definition}
%\label{def.bbisim}
An  ${\mathcal N}$-\emph{barbed bisimulation} over a set of names, ${\mathcal N}$, is a symmetric binary relation 
${\mathcal S}_{\mathcal N}$ between agents such that $P\rel{S}_{\mathcal N}Q$ implies:
\begin{enumerate}
\item If $P \red P'$ then $Q \wred Q'$ and $P'\rel{S}_{\mathcal N} Q'$.
\item If $P\downarrow_{\mathcal N} x$, then $Q\Downarrow_{\mathcal N} x$.
\end{enumerate}
$P$ is ${\mathcal N}$-barbed bisimilar to $Q$, written
$P \wbbisim_{\mathcal N} Q$, if $P \rel{S}_{\mathcal N} Q$ for some ${\mathcal N}$-barbed bisimulation ${\mathcal S}_{\mathcal N}$.
\end{definition}

$\mathcal{R} \subseteq \pi \times \pi$

$P \mathcal{R} Q => \forall P'. P \red P' \Rightarrow \exists Q'. Q \red Q', P' \mathcal{R} Q'$

$P \vdash x \Rightarrow Q \vdash x$

\begin{mathpar}
  \inferrule*[lab=Out-barb]{x \nameeq y}{{y}!\langle{Q}\rangle \vdash x}
  \and
  \inferrule*[lab=Par-barb]{\mbox{$P\vdash x$ or $Q\vdash x$}}{\binpar{P}{Q} \vdash x}
\end{mathpar}

\subsubsection{Contexts}

One of the principle advantages of computational calculi like the
$\pi$-calculus is a well-defined notion of context,
contextual-equivalence and a correlation between
contextual-equivalence and notions of bisimulation. The notion of
context allows the decomposition of a process into (sub-)process and
its syntactic environment, its context. Thus, a context may be
thought of as a process with a ``hole'' (written $\Box$) in it. The
application of a context $M$ to a process $P$, written $M[P]$, is
tantamount to filling the hole in $M$ with $P$. In this paper we do
not need the full weight of this theory, but do make use of the notion
of context in the proof the main theorem. 

\begin{mathpar}
  \inferrule* [lab=summation] {} {{M_{M},M_{N}} \bc \Box \;|\; x.M_{A} \;|\; M_{M}+M_{N}}
  \and
  \inferrule* [lab=agent] {} {{M_{A}} \bc (\vec{x})M_{P} \;| \; \clift{P_0,\ldots,M_{P},\ldots,P_N}}
  \and \\
  \inferrule* [lab=process] {} {{M_{P}} \bc M_{N} \;| \;P|M_{P} }
\end{mathpar} 

\begin{mathpar}
  \inferrule* [lab=sychronization] {} {M_{N} \bc \Box \;|\; x?M_{F} \;|\; x!M_{C}}
  \and
  \inferrule* [lab=abstraction] {} {{M_{F}} \bc (x)M_{P} }
  \and
  \inferrule* [lab=concretion] {} {{M_{C}} \bc \langle M_{P} \rangle }
  \and \\
  \inferrule* [lab=process] {} {{M_{P}} \bc M_{N} \;| \;P|M_{P} }
\end{mathpar}

\begin{definition}[contextual application] Given a context $M$, and
  process $P$, we define the \emph{contextual application}, $M[P] :=
  M\{P/\Box\}$. That is, the contextual application of M to P is the
  substitution of $P$ for $\Box$ in $M$.
\end{definition}

$\meaningof{-} : L \to \mathcal{P}(\pi)$

\begin{mathpar}
  \inferrule* [lab=collection] {} {\meaningof{true} = \pi, \and \meaningof{~E} = \pi \setminus \meaningof{E}, \and \meaningof{E_{1} \& E_{2}} = \meaningof{E_{1}} \cap \meaningof{E_{2}}}
\end{mathpar}

\begin{mathpar}
  \inferrule* [lab=structure] {} {\meaningof{0} = \{ P \in \pi | P \equiv 0 \}, \and \\ \meaningof{E_1 | E_2} = \{ P \in \pi | P \equiv P_{1} | P_{2}, P_{1} \in \meaningof{E_{1}}, P_{2} \in \meaningof{E_2}\} }
\end{mathpar}

\begin{mathpar}
 \inferrule* [lab=behavior] {} {\meaningof{\langle a?b \rangle E} = \{ P \in \pi | P \equiv Q | u?(y)P', \\ \and \\\\ \and \\ \;\;\; u \in \meaningof{a}, \forall z.P'\{z/y\} \in \meaningof{E\{z/b\}}\}, \and \\ \meaningof{a!E} = \{ P \in \pi | P \equiv Q | x!\langle P' \rangle, x \in \meaningof{a} P' \in \meaningof{E}\} }
\end{mathpar}

\begin{mathpar}
 \inferrule* [lab=nominal] {} {\meaningof{\quotep{E}} = \{ \quotep{P} \in \quotep{\pi} | P \in \meaningof{E} \}, \and \meaningof{\quotep{P}} = \{ \quotep{Q} \in \quotep{\pi} | P \equiv Q \} \and \\ \meaningof{@\quotep{E}} = \{ P \in \pi | P \equiv @x, x \in \meaningof{E} \}}
\end{mathpar}

\begin{eqnarray*}
  \\
  \meaningof{-} : TS \to ST
\end{eqnarray*}

\begin{eqnarray*}
  \\
  L : TS \to ST
\end{eqnarray*}

\begin{eqnarray*}
  \\
  P \models E \iff P \in \meaningof{E}
\end{eqnarray*}

\begin{eqnarray*}
  P \approx_{L} Q \iff \forall E \in L. P \models E \iff Q \models E
\end{eqnarray*}

\begin{eqnarray*}
  P \approx_{K} Q
\end{eqnarray*}

\begin{eqnarray*}
  P \approx Q
\end{eqnarray*}

$\approx_{K} = \approx = \approx_{L}$

\subsubsection{Contextual duality}

Note that contexts extend the quotation operation to a family of
operations from processes to names. Given a context, $M$, we can
define a \emph{nominal context}, $\quotep{M}$ by $\quotep{M}[P] :=
\quotep{M[P]}$. To foreshadow what is to come we observe that these
operations enjoy a duality with processes very much like the duality
between vectors and maps from vectors to scalars.

Further, because the calculus is essentially higher-order, we have a
correspondence between contexts and processes. More specifically,
given a name $x$ and a context $M$ we can construct $M^{*}_{x}$ such
that 

\begin{mathpar}
  M^{*}_{x} | \lift{x}{P} \red M[P]
\end{mathpar}

namely,

\begin{mathpar}
  M^{*}_{x} := x?(u).M[\dropn{u}]
\end{mathpar}

The dependence of $M^{*}_{x}$ on a name makes it an abstraction, 

\begin{mathpar}
  M^{*} := (x)x?(u).M[\dropn{u}]
\end{mathpar}

\subsection{Additional notation}

It will sometimes be convenient to denote the process a name
quotes. We already have the notation $x = \quotep{P}$, but it will be
convenient to introduce an alternate notation, $\procn{x}$, when we
want to emphasize the connection to the use of the name. Note that, by
virtue of name equivalence, $\quotep{\procn{x}} \nameeq x$; so, the
notation is consistent with previous definitions.

Further, because names have structure it is possible to effect
substitutions on the basis of that structure. This means we need to
upgrade our notation for substitutions, which we accomplish by
adapting comprehension notation. Thus,

\begin{mathpar}
  P\{ y / x : x \in S \}
\end{mathpar}

is interpreted to mean the process derived from P by replacing (in a
capture-avoiding manner) each occurrence of $x$ in $S$ by $y$. For example,

\begin{mathpar}
  P\{ \quotep{\procn{x}|\procn{x}} / x : x \in \freenames{P} \}
\end{mathpar}

will replace each (occurrence) of a free name $x$ in $P$ by
$\quotep{\procn{x}|\procn{x}}$.

Also, we will avail ourselves of the notation $x^{L}$ and $x^{R}$ to
denote injections of a name into disjoint copies of the name
space. There are numerous ways to accomplish this. One example can be
found in \cite{MeredithR05}. This notation overloads to vectors of
names: $\vec{x}^{\pi} := (x_{i}^{\pi} \; : \; 0 \leq i < |\vec{x}| )$ where $\pi \in \{L,R\}$.

We also use $P^{\Box} := P|\Box$.

In \cite{MeredithR05} an interpretation of the new operator is
given. It turns out that there are several possible interpretations
all enjoying the requisite algebraic properties of the operator (see
\cite{milner91polyadicpi}). We will therefore make liberal use of
$(\nu\; \vec{x})P$.

% subsection the_syntax_and_semantics_of_the_notation_system (end)   

\section{Interpretation of QM}
\subsection{Supporting definitions}
\subsubsection{Multiplication}
\begin{mathpar}
  \quotep{Q} \cdot \quotep{R} := \quotep{Q|R}
  \and \\
  \quotep{Q} \cdot P := P\{ \quotep{Q|R} / \quotep{R} : \quotep{R} \in \freenames{P} \}
\end{mathpar}

\paragraph{Discussion}
The first line needs little explanation. The second line says that
each free name of the process is replaced with the multiplication of
that name by the scalar. Multiplication of a scalar (name) by a state
(process) results in a process all the names of which have been `moved
over' by parallel composition with the process the scalar
quotes. There is a subtlety that the bound names have to be
manipulated so that multiplied names aren't accidentally
captured. There are many ways to achieve this.

\begin{remark}\label{rem:multiplication_identities}
  The reader is invited to verify that for all $x,y,z \in \QProc$ and $P \in \Proc$
  \begin{mathpar}
    x \cdot \quotep{0} \equiv x 
    \and
    x \cdot y \equiv y \cdot x
    \and
    x \cdot (y \cdot z) \equiv (x \cdot y) \cdot z
    \and \\
    \quotep{0} \cdot P \equiv P
    \and \\
    x \cdot (y \cdot P) \equiv (x \cdot y) \cdot P
    \and \\
    x \cdot (P|Q) \equiv (x \cdot P) | (x \cdot Q)
    \and \\    
  \end{mathpar}
\end{remark}

\subsubsection{Tensor product}

We define a tensor product on processes by structural induction.

\paragraph{Tensor of sums} First note that all summations, including
$\pzero$ and sequence, can be written $\Sigma_{i} x_{i}.A_{i} +
\Sigma_{j} x_{j}.C_{j}$, where we have grouped input-guarded processes
together and output-guarded processes together.

Thus, we can define the tensor product of two summations, $N_{1}\otimes N_{2}$, where

\begin{mathpar}
  N_{1} := \Sigma_{i} x_{i}.A_{i} + \Sigma_{j} x_{j}.C_{j}
  \and
  N_{2} := \Sigma_{i'} y_{i'}.B_{i'} + \Sigma_{j'} y_{j'}.D_{j'} 
\end{mathpar}

as follows.

\begin{mathpar}
  \Sigma_{i} x_{i}.A_{i} + \Sigma_{j} x_{j}.C_{j} \otimes \Sigma_{i'}
  y_{i'}.B_{i'} + \Sigma_{j'} y_{j'}.D_{j'} 
  \and \\
  := \; \Sigma_{i} \Sigma_{i'} \quotep{\stackrel{\vee}{x_{i}}| \stackrel{\vee}{y_{i'}}}.(A_{i}\otimes B_{i'}) \; | \; \Sigma_{i'} \Sigma_{i} \quotep{\stackrel{\vee}{y_{i'}}|\stackrel{\vee}{x_{i}}}.(B_{i'}\otimes A_{i})
  \and
  \;\; | \;\; \Sigma_{j} \Sigma_{j'} \quotep{\stackrel{\vee}{x_{j}}|\stackrel{\vee}{y_{j'}}}.(A_{j}\otimes B_{j'}) \; | \; \Sigma_{j'} \Sigma_{j} \quotep{\stackrel{\vee}{y_{j'}}|\stackrel{\vee}{x_{j}}}.(B_{j'}\otimes A_{j})
\end{mathpar}

\begin{remark}
  Do we need to $x^{L}$ and $y^{R}$ for this construction as well?
\end{remark}

\paragraph{Tensor of parallel compositions} Next, we distribute tensor
over par.

\begin{mathpar}
  P_{1}|P_{2} \otimes Q_{1}|Q_{2} := (P_{1} \otimes Q_{1}) | (P_{1}
  \otimes Q_{2}) | (P_{2} \otimes Q_{1}) | (P_{2} \otimes Q_{2})
\end{mathpar}

\paragraph{Tensor with dropped names} We treat tensor of a
process with a dropped name as parallel composition.

\begin{mathpar}
  P \otimes \dropn{x} := P | \dropn{x}
\end{mathpar}

\paragraph{Tensor of agents}

Finally, we need to define tensor on agents. Note that the definition
of tensor on normal products only tensors inputs with inputs and
outputs with outputs. Thus, we only have to define the operation on
``homogeneous'' pairings.

\begin{mathpar}
  (\vec{x})P \otimes (\vec{y})Q
  \and \\
  := (x_{0}^{L}|y_{0}^{R},\ldots,x_{0}^{L}|y_{n}^{R},\ldots,x_{m}^{L}|y_{0}^{R},\ldots,x_{m}^{L}|y_{n}^R)(P\{ \vec{x}^{L}/\vec{x}\} \otimes Q \{ \vec{y}^{R}/\vec{y}\})
  \and \\
  \clift{\vec{P}} \otimes \clift{\vec{Q}}
  \and \\
  := \clift{P_{0}\otimes Q_{0},\ldots,P_{0}\otimes Q_{n},\ldots,P_{m}\otimes Q_{0},\ldots,P_{m}\otimes Q_{n}}
\end{mathpar}

\begin{remark}
  Observe that arities of tensored abstractions matches arities of
  tensored concretions if the original arities matched. Note also that
  the length of the arities corresponds to the increase in dimension
  we see in ordinary vector space tensor product.
\end{remark}

\begin{remark}
  Operationally, this definition distributes the tensor down to
  components ``linked'' by summation. Tensor over summation is
  intriguing in that it mixes names. Moreover, as a consequence of the
  way it mixes names we have the identities for all $x \in \QProc$ and
  $P,Q \in \Proc$

  \begin{mathpar}
    (x \cdot P) \otimes Q \equiv x \cdot (P \otimes Q) \equiv P \otimes (x \cdot Q)
    \and
    P \otimes \pzero \equiv P
  \end{mathpar}

  that the reader is invited to verify.
\end{remark}

\subsubsection{Annihilation}
\begin{mathpar}
  P^{\perp} := \{ Q | \forall R. P|Q \red^{*} R \Rightarrow R \red^{*} \pzero \}
  \and \\
  P^{\underline{\perp}} := \Sigma_{Q \in P^{\perp}} \quotep{Q}?(y).(\dropn{y}|Q) | \Sigma_{Q \in P^{\perp}} \quotep{Q}\clift{\Box}
\end{mathpar}

\paragraph{Discussion} The reader will note that $P^{\perp}$ is a
\emph{set} of processes, while $P^{\underline{\perp}}$ is a
\emph{context}. We call the set $P^{\perp}$ the \emph{annihilators} of
$P$. The parallel composition of a process in the annihilators of $P$
with $P$ will result in a process, the state space of which has all
paths eventually leading to $\pzero$. Execution may endure loops; but
under reasonable conditions of fairness (naturally guaranteed under
most notions of bisimulation) such a composite process cannot get
stuck in such a loop and will, eventually pop out and terminate.

The context $P^{\underline{\perp}}$ is ready and willing to ``take the
$P$ out of'' the process to which it is applied. It will effectively
transmit the code of the process to which it is applied to one of the
annihilators and run the process against it.

\subsubsection{Evaluation}
We fix $M$ a domain of fully abstract interpretation with an equality
coincident with bisimulation. We take $\meaningof{\cdot} : \Proc \to
M$ to be the map interpreting processes and $\nmeaningof{\cdot} : \M
\to Proc$ to be the map running the other way. Then we define

\begin{mathpar}
  \int P := \nmeaningof{\meaningof{P}}
\end{mathpar}

\paragraph{Discussion}
There are many fully abstract interpretations of Milner's
$\pi$-calculus. Any of them can be used as a basis for interpreting
the reflective calculus here. Equipped with such a domain it is
largely a matter of grinding through to check that the Yoneda
construction for the normalization-by-evaluation program can be
extended to this setting.

\begin{remark}
  The reader is invited to verify that $\int (P^{\underline{\perp}}[P]) = 0$.
\end{remark}

\subsection{Quantum mechanics}

Table \ref{tbl:core_qm_op_defns} gives the core operational definitions

\begin{table}[htp]\label{tbl:core_qm_op_defns}
  \center{
    \fbox{
      \begin{tabular}{c|c}
        quantum mechanics & process calculus \\
        \hline
        scalar & $x := \quotep{P}$ \\
        state vector & $\state{P} := P$ \\
        dual & $\state{P}^{*} := \event{P^{\underline{\perp}}} := \quotep{P^{\underline{\perp}}}[-]$ \\
        matrix & $ \Sigma_{\alpha} \state{P_{\alpha}}x_{\alpha}\event{Q_{\alpha}}$ \\
        vector addition & $\state{P} + \state{Q} := \state{P | Q}$ \\
        tensor product & $\state{P} \otimes \state{Q} := \state{P \otimes Q}$ \\
        inner product & $\innerprod{P}{Q} := \quotep{\int P^{\underline{\perp}}[Q]}$ \\
      \end{tabular}
    }
  }
  \caption{QM - operational definitions}
\end{table}

where

\begin{mathpar}
  \prmatrix{P}{Q} := \fprmatrix{P}{\quotep{\pzero}}{Q}
  \and
  \fprmatrix{P}{x}{Q} := (\state{P},x,\event{Q})
  \and
  (\fprmatrix{P}{x}{Q})(\state{R}) := x \cdot \innerprod{Q}{R} \cdot \state{P}
  \and
  (\fprmatrix{P}{x}{Q})(\event{R}) := x \cdot \innerprod{R}{P} \cdot \event{Q}
\end{mathpar}

\paragraph{Discussion}
As promised: vectors (aka states) are represented as processes; duals
as contextual duals; inner product definition should be compared with
standard inner product definition for ....

\begin{remark}
  Assuming $\int (P^{\underline{\perp}}[P]) = 0$, the reader is
  invited to verify that $(\fprmatrix{P}{x}{P})(\state{P}) = x \cdot \state{P}$.
\end{remark}

\begin{remark}
  The reader is invited to verify that $\innerprod{P}{Q}$ could
  equally well have been written $\quotep{\int \stackrel{\vee}{x}}$
  where $x = \event{P^{\underline{\perp}}}(Q)$.

  One of the motivations for this remark is that there is another way
  to factor these operations. We could package up evaluation in the dual:

  \begin{mathpar}
    \state{P}^{*} := \event{\int P^{\underline{\perp}}} := \quotep{\int P^{\underline{\perp}}}[-]
  \end{mathpar}

  and then have inner product defined by
  
  \begin{mathpar}
    \innerprod{P}{Q} := \event{P}(Q)
  \end{mathpar}

  Hopefully, experience with the calculations will provide guidance on
  the best factoring.
\end{remark}

\begin{remark}
  Assuming $\int (P^{\underline{\perp}}[P]) = 0$, the reader is
  invited to verify that $\forall P,Q. (\prmatrix{0}{Q})(\state{0}) =
  \state{0}$ and dually $(\prmatrix{P}{0})(\event{0}) = \event{0}$.
\end{remark}

\begin{remark}
  i'm a little worried that i don't (yet) have proper support for
  complex conjugacy. But, the observation above may give us a
  clue. According to Abramsky, it must be the case that the scalars
  are iso to the homset of the identity for the tensor -- which the
  observation above characterizes. 

  For now, we will simply bookmark the notion with $\overline{x}$.
\end{remark}

\subsubsection{Adjointness}

We need to give a definition of $(\cdot)^{\dagger}$ for matrices. The
obvious candidate definition is
\begin{mathpar}
(\Sigma_{\alpha}\fprmatrix{P_{\alpha}}{x_{\alpha}}{Q_{\alpha}})^{\dagger}
= \Sigma_{\alpha}\fprmatrix{(Q_{\alpha}^{\underline{\perp}})^{*}}{\overline{x}_{\alpha}}{P_{\alpha}^{\underline{\perp}}} 
\end{mathpar}

But, $(Q_{\alpha}^{\underline{\perp}})^{*}$ requires a name along
which to communicate the process to achieve the context application.

\subsubsection{Basis for a basis}
If processes label states and ``addition'' of states (a.k.a. vector
addition) is interpreted as parallel composition, what corresponds to
notions of linear independence and basis? Here, we recall that Yoshida
has developed a set of \emph{combinators} for an asynchronous verison
of Milner's $\pi$-calculus. These are a finite set of processes such
any process can be expressed as parallel composition of these
combinators together with liberal uses of the new operator and
replication. We can simply give a translation of these into the
present calculus and have reasonable expectation that the property
carries over. That is, that the resultant set allows to express all
processes via parallel composition. Note, however, that there is no
new operator or replication in this calculus. As a result, we expect
that the corresponding set is actually infinite. That is, we expect
that the space is actually infinite dimensional.

\begin{remark}
  The attentive reader may be a bit concerned. Certainly, the
  collection $S$, $K$ and $I$ is a finite set of
  combinators. Shouldn't we expect to see a finite set of combinators
  for an effectively equivalent system? i am very sympathetic to this
  critique and feel it warrants full attention. On the other hand, i
  also have in mind the following analogy. The natural numbers, as a
  monoid under addition, has exactly $1$ generator, while the natural
  numbers, as a monoid under multiplication, has countably many
  generators (the primes). We observe that the application of the
  lambda calculus is much less resource sensitive than the parallel
  composition of the $\pi$-calculus. Could it be the case that we have
  an analogy of the form
  
  \begin{mathpar}
    m + n : MN :: m*n : M|N
  \end{mathpar}

  giving a similar blow up in the set of ``primes''?  This is such a
  wonderful thought that, even if it's not true, i think it's worth
  writing down.
\end{remark}
 

\documentclass[12pt]{llncs}
%\documentclass{jktr}

\usepackage[pdftex]{hyperref}                   
\usepackage {listings}
\usepackage {mathpartir}
\usepackage{bcprules}
%\usepackage{listings}
                       
\usepackage{graphicx} 
%\usepackage[margins=2.5cm,nohead,nofoot]{geometry}
%\usepackage{geometry}
\usepackage{amsfonts}
\usepackage{amstext}
\usepackage{latexsym}
\usepackage{amssymb}
\usepackage{color}


%\include{myPreamble}
\include{qm2pi.local} 

%\ifpdf
%\usepackage[pdftex]{graphicx}
%\else
%\usepackage{graphicx}
%\fi

 % \ifpdf
%  \usepackage{pdfsync}
%  \if


%\title{Brief Article}
%\author{David F. Snyder}
%\author{L.G. Meredith}

%\address{Dept. of Math., Texas State University--San Marcos, San Marcos, TX 78666}
       
\pagestyle{empty}


\begin{document}

\lstset{language=[Objective]Caml,frame=shadowbox}

\input{qm2pi.front}

% section front matter (end)

\input{qm2pi.intro} 
 
% section introduction (end)

% \input{qm2pi.knotations} 

% section notation (end)

\input{qm2pi.process.calculi} 

% section concurrent_process_calculi_and_spatial_logics_ (end)
    
%\input{qm2pi.knots2pi} 

%\input{qm2pi.trefoil} 

%\input{qm2pi.mainthm} 

% subsection basic_interpretation (end)

%\input{qm2pi.rho.presentation} 
\subsection{The syntax and semantics of the notation system}\label{sub:the_syntax_and_semantics_of_the_notation_system} % (fold)

We now summarize a technical presentation of the calculus that
embodies our theory of dynamics. The typical presentation of such a
calculus follows the style of giving generators and relations on
them. The grammar, below, describing term constructors, freely
generates the set of processes, $\Proc$. This set is then quotiented
by a relation known as structural congruence and it is over this set
that the notion of dynamics is expressed. This presentation is
essentially that of \cite{MeredithR05} with the addition of
polyadicity and summation. For readability we have relegated some of
the technical subtleties to an appendix.

\subsubsection{Process grammar}\label{subsub:process_grammar}

\begin{mathpar}
  \inferrule* [lab=synchronization] {} {{M} \bc \pzero \;|\; x?F \;|\; x!C }
  \and
  \inferrule* [lab=abstraction] {} {{F} \bc (x)P}
  \and
  \inferrule* [lab=concretion] {} {{C} \bc \langle Q \rangle}
  \and
  \inferrule* [lab=process] {} {{P,Q} \bc M \;| \;P|Q \;|\; @{x}}
  \and
  \inferrule* [lab=name] {} {{x} \bc \quotep{P}}
\end{mathpar} 

Note that $\vec{x}$ (resp. $\vec{P}$) denotes a vector of names
(resp. processes) of length $|\vec{x}|$ (resp. $|\vec{P}|$). We adopt
the following useful abbreviations.

\begin{mathpar}
   x?(\vec{y}).P := x.(\vec{y})P \and  x\clift{\vec{P}} := x.\clift{\vec{P}}
   \and x!(y) := \lift{x}{\dropn{y}}
   \and \Pi_{i=0}^{n-1}P_i := P_0 | \ldots | P_{n-1}
\end{mathpar}

\subsubsection{Structural congruence}

\paragraph{Free and bound names and alpha-equivalence.} At the
core of structural equivalence is alpha-equivalence which identifies
process that are the same up to a change of variable. Formally, we
recognize the distinction between free and bound names. The free names
of a process, $\freenames{P}$, may be calculated recursively as
follows:

\begin{mathpar}
\freenames{\pzero} := \emptyset
  \and \\
  \freenames{x?(y).P} := \{ x \} \cup (\freenames{P} \setminus \{ y \})
  \and 
  \freenames{x!\langle P \rangle} := \{ x \} \cup \{ P \} 
  \and \\
  \freenames{P|Q} := \freenames{P} \cup \freenames{Q}
  \and \\
  \freenames{@{x}} := \{ x \}
\end{mathpar}

$\pi$
$\quotep{\pi}$

$\freenames{-} : \pi \to \mathcal{P}(\quotep{\pi})$

\begin{eqnarray*}
  \freenames{\pzero} & := & \emptyset \\
  \freenames{x?(y).P} & := & \{ x \} \cup (\freenames{P} \setminus \{ y \}) \\
  \freenames{x!\langle P \rangle} & := & \{ x \} \cup \{ P \} \\
  \freenames{P|Q} & := & \freenames{P} \cup \freenames{Q} \\
  \freenames{\dropn{x}} & := & \{ x \}
\end{eqnarray*}

The bound names of a process, $\boundnames{P}$, are those names occurring in $P$
that are not free. For example, in $x?(y).0$, the name $x$ is free, while $y$ is bound.

\begin{mathpar}
  \inferrule* [lab=monoidal-laws] {} { P|Q \equiv Q|P \and P|0 \equiv P \and P|(Q|R) \equiv (P|Q)|R }
\end{mathpar}

\begin{mathpar}
  \inferrule* [lab=alpha-equivalence] {} { (x)P \equiv (y)P\{y/x\} \and y \not\in \freenames{P} }
\end{mathpar}

\begin{definition}
Then two processes, $P,Q$, are alpha-equivalent if $P = Q\{\vec{y}/\vec{x}\}$ for
some $\vec{x} \in \boundnames{Q},\vec{y} \in \boundnames{P}$, where $Q\{\vec{y}/\vec{x}\}$
denotes the capture-avoiding substitution of $\vec{y}$ for $\vec{x}$ in $Q$.
\end{definition}

\begin{definition}
  The {\em structural congruence} \cite{SangiorgiWalker} , $\equiv$,
  between processes is the least congruence containing
  alpha-equivalence, satisfying the abelian monoid laws
  (associativity, commutativity and $\pzero$ as identity) for parallel
  composition $|$ and for summation $+$.
\end{definition}

\subsection{Name equivalence}

We take name equivalence, written $\nameeq$, to be the smallest
equivalence relation generated by the following rules.

\begin{mathpar}
\inferrule*[lab=Quote-drop]
{ }
{ \quotep{@{x}} \nameeq x }

\inferrule*[lab=Struct-equiv]
{ P \scong Q }
{ \quotep{P} \nameeq \quotep{Q} }
\end{mathpar}

The astute reader will have noticed that the mutual recursion of names
and processes imposes a mutual recursion on alpha-equivalence and
structural equivalence via name-equivalence. Fortunately, all of this
works out pleasantly and we may calculate in the natural way, free of
concern. The reader interested in the details is referred to the
appendix \ref{appendix:rho_details}.

\subsection{Substitution}

We use $\Proc$ for the set of processes, $\QProc$ for the set of
names, and $\id{\{}\vec{y} / \vec{x} \id{\}}$ to denote partial maps,
$s : \QProc \rightarrow \QProc$. A map, $s$ lifts, uniquely, to a map
on process terms, $\widehat{s} : \Proc \rightarrow \Proc$ by the
following equations.

\begin{mathpar}
  (0) \psubstp{Q}{P} := 0 \\
  (R \juxtap S) \psubstp{Q}{P}
  :=    
  (R)\psubstp{Q}{P} \juxtap (S) \psubstp{Q}{P} \\
  (x?(y).R) \psubstp{Q}{P}    
  :=    
  (x)\substp{Q}{P} (z)\concat( (R \psubstn{z}{y}) \psubstp{Q}{P} ) \\
  (\lift{x}{R}) \psubstp{Q}{P}  
  :=
  \lift{(x)\substp{Q}{P}}{ R \psubstp{Q}{P} } \\
%   (\dropn{x})  \psubstp{Q}{P}       
%   := 
%   \left\{ 
%     \begin{array}{ccc} 
%       \dropn{\quotep{Q}} & & x \nameeq \quotep{P} \\
%       \dropn{x} & & otherwise \\
%     \end{array}
%   \right. 
  (\dropn{x})  \psubstp{Q}{P}       
  := 
  \left\{ 
    \begin{array}{ccc} 
      Q & & x \nameeq \quotep{P} \\
      \dropn{x} & & otherwise \\
    \end{array}
  \right.
\end{mathpar}
 

where

\begin{eqnarray}
  (x)\id{\{} \lpquote Q \rpquote / \lpquote P \rpquote \id{\}}            = 
  \left\{ 
    \begin{array}{ccc}
      \lpquote Q \rpquote & & x \nameeq \lpquote P \rpquote \\
      x & & otherwise \\
    \end{array}
  \right. \nonumber
\end{eqnarray}

and $z$ is chosen distinct from $\quotep{P}$, $\quotep{Q}$, the free
names in $Q$, and all the names in $R$. Our $\alpha$-equivalence will
be built in the standard way from this substitution.

\begin{remark}\label{rem:no_self_referential_names}
  One consequence of these definitions is that $\forall P. \quotep{P}
  \not\in \freenames{P}$.
\end{remark}

\subsection{ Dynamic quote: an example }

Anticipating something of what's to come, consider applying the
substitution, $\widehat{\id{\{}u / z \id{\}}}$, to the following pair
of processes, $\lift{w}{y!(z)}$ and $w[ \lpquote y!(z) \rpquote ]$.

\begin{eqnarray}
	\lift{w}{y!(z)}\widehat{\id{\{}u / z \id{\}}}
		& = &
		\lift{w}{y!(u)} \nonumber\\
	w[ \lpquote y!(z) \rpquote ] \widehat{ \id{\{}u / z \id{\}} }
		& = &
		w[ \lpquote y!(z) \rpquote ] \nonumber
\end{eqnarray}

Because the body of the process between quotes is impervious to
substitution, we get radically different answers. In fact, by
examining the first process in an input context,
e.g. $x?(z).\lift{w}{y!(z)}$, we see that the process under the lift
operator may be shaped by prefixed inputs binding a name inside it. In
this sense, the lift operator will be seen as a way to dynamically
construct processes before reifying them as names.

Finally equipped with these standard features we can present the
dynamics of the calculus.

\subsubsection{Operational semantics} 

Finally, we introduce the computational dynamics. What marks these
algebras as distinct from other more traditionally studied algebraic
structures, e.g. vector spaces or polynomial rings, is the manner in
which dynamics is captured. In traditional structures, dynamics is typically
expressed through morphisms between such structures, as in linear maps
between vector spaces or morphisms between rings. In algebras
associated with the semantics of computation, the dynamics is
expressed as part of the algebraic structure itself, through a
reduction reduction relation typically denoted by $\red$. Below, we
give a recursive presentation of this relation for the calculus used
in the encoding.

$\red \subseteq \pi \times \pi$
$\red : \pi \to \mathcal{P}(\pi)$

\begin{mathpar}
  \inferrule* [lab=Comm] { \textsf{match}( x_{src}, x_{trgt} ) } { x_{trgt}?(y)P \; | \; x_{src}!\langle {Q} \rangle \red P\{\quotep{Q}/y}\} }
  \and \\
  \inferrule* [lab=Par] {{P} \red {P}'} {{{P} | {Q}} \red {{P}' | {Q}}}
  \and
  \inferrule* [lab=Equiv]{{{P} \scong {P}'} \andalso {{P}' \red {Q}'} \andalso {{Q}' \scong {Q}}}{{P} \red {Q}}
\end{mathpar}

\begin{eqnarray*}
  match_{\equiv} (\quotep{P},\quotep{Q}) & := & P \equiv Q \\
  match_{\dagger}(\quotep{P},\quotep{Q}) & := & \forall R. P|Q \red^{*} R => R \red^{*} 0 \\
  match_{K}(\quotep{P},\quotep{Q}) & := & K \mbox{ for some context } K
\end{eqnarray*}

$u?(x)P | u!\langle Q \rangle \red P\{\quotep{Q}/x\}$

%We write $\wred$ for $\red^*$, and $P\red$ if $\exists Q $ such that $ P \red Q$.
We write $P\red$ if $\exists Q $ such that $ P \red Q$ and $P\not\red$, otherwise.

\section{Replication}

As mentioned before, it is known that replication (and hence
recursion) can be implemented in a higher-order process algebra
\cite{SangiorgiWalker}. As our first example of calculation with the
machinery thus far presented we give the construction explicitly in
the {\rhoc}.

\begin{eqnarray}
	D_{x} & := & \prefix{x}{y}{(\binpar{\outputp{x}{y}}{@{y}})} \nonumber\\
	\bangp_{x}{P} & := & \binpar{{x}!\langle{\binpar{D_{x}}{P}}\rangle}{D_{x}} \nonumber
\end{eqnarray}

\begin{eqnarray}
	\bangp_{x}{P} & & \nonumber\\
	=
	& {x}!\langle{(\prefix{x}{y}{(\outputp{x}{y} | @{y})) | P}}\rangle 
	      | \prefix{x}{y}{(\outputp{x}{y} | @{y})} & \nonumber\\
	\red
	& (\outputp{x}{y} | @{y})\substn{\quotep{(\prefix{x}{y}{(@{y} | \outputp{x}{y})) | P}}}{y} & \nonumber\\
	=
	& \outputp{x}{\quotep{(\prefix{x}{y}{(\outputp{x}{y} | @{y})) | P}}}
	  | {(\prefix{x}{y}{(\outputp{x}{y} | @{y})) | P}} & \nonumber\\
	\red
	& \ldots & \nonumber\\
	\red^*
	& P | P | \ldots & \nonumber
\end{eqnarray}

Of course, this encoding, as an implementation, runs away, unfolding
$\bangp{P}$ eagerly. A lazier and more implementable replication
operator, restricted to input-guarded processes, may be obtained as follows.

\begin{eqnarray}
\bangp{\prefix{u}{v}{P}} 
	:= 
	\binpar{\lift{x}{\prefix{u}{v}{(\binpar{D(x)}{P})}}}{D(x)} \nonumber
\end{eqnarray}

\begin{remark}
  Note that the lazier definition still does not deal with summation
  or mixed summation (i.e. sums over input and output). The reader is
  invited to construct definitions of replication that deal with these
  features. 

  Further, the definitions are parameterized in a name, $x$. Can you,
  gentle reader, make a definition that eliminates this parameter and
  guarantees no accidental interaction between the replication
  machinery and the process being replicated -- i.e. no accidental
  sharing of names used by the process to get its work done and the
  name(s) used by the replication to effect copying. This latter
  revision of the definition of replication is crucial to obtaining
  the expected identity $!!P \sim !P$.
\end{remark}

\begin{remark}\label{rem:paradoxical_combinator}
  The reader familiar with the lambda calculus will have noticed the
  similarity between $D$ and the paradoxical combinator.

  [Ed. note: the existence of this seems to suggest we have to be more
  restrictive on the set of processes and names we admit if we are to
  support no-cloning.]
\end{remark}

\subsubsection{Bisimulation}

The computational dynamics gives rise to another kind of equivalence,
the equivalence of computational behavior. As previously mentioned
this is typically captured \emph{via} some form of bisimulation.

% The notion we use in this paper is weak barbed bisimulation
% \cite{milner91polyadicpi}.

The notion we use in this paper is derived from weak barbed
bisimulation \cite{milner91polyadicpi}. 

\begin{definition}
An \emph{observation relation}, $\downarrow_{\mathcal N}$, over a set
of names, $\mathcal N$, is the smallest relation satisfying the rules
below.

\infrule[Out-barb]{y \in {\mathcal N}, \; x \nameeq y}
		  {\outputp{x}{v} \downarrow_{\mathcal N} x}
\infrule[Par-barb]{\mbox{$P\downarrow_{\mathcal N} x$ or $Q\downarrow_{\mathcal N} x$}}
		  {\binpar{P}{Q} \downarrow_{\mathcal N} x}

We write $P \Downarrow_{\mathcal N} x$ if there is $Q$ such that 
$P \wred Q$ and $Q \downarrow_{\mathcal N} x$.
\end{definition}

\begin{definition}
%\label{def.bbisim}
An  ${\mathcal N}$-\emph{barbed bisimulation} over a set of names, ${\mathcal N}$, is a symmetric binary relation 
${\mathcal S}_{\mathcal N}$ between agents such that $P\rel{S}_{\mathcal N}Q$ implies:
\begin{enumerate}
\item If $P \red P'$ then $Q \wred Q'$ and $P'\rel{S}_{\mathcal N} Q'$.
\item If $P\downarrow_{\mathcal N} x$, then $Q\Downarrow_{\mathcal N} x$.
\end{enumerate}
$P$ is ${\mathcal N}$-barbed bisimilar to $Q$, written
$P \wbbisim_{\mathcal N} Q$, if $P \rel{S}_{\mathcal N} Q$ for some ${\mathcal N}$-barbed bisimulation ${\mathcal S}_{\mathcal N}$.
\end{definition}

$\mathcal{R} \subseteq \pi \times \pi$

$P \mathcal{R} Q => \forall P'. P \red P' \Rightarrow \exists Q'. Q \red Q', P' \mathcal{R} Q'$

$P \vdash x \Rightarrow Q \vdash x$

\begin{mathpar}
  \inferrule*[lab=Out-barb]{x \nameeq y}{{y}!\langle{Q}\rangle \vdash x}
  \and
  \inferrule*[lab=Par-barb]{\mbox{$P\vdash x$ or $Q\vdash x$}}{\binpar{P}{Q} \vdash x}
\end{mathpar}

\subsubsection{Contexts}

One of the principle advantages of computational calculi like the
$\pi$-calculus is a well-defined notion of context,
contextual-equivalence and a correlation between
contextual-equivalence and notions of bisimulation. The notion of
context allows the decomposition of a process into (sub-)process and
its syntactic environment, its context. Thus, a context may be
thought of as a process with a ``hole'' (written $\Box$) in it. The
application of a context $M$ to a process $P$, written $M[P]$, is
tantamount to filling the hole in $M$ with $P$. In this paper we do
not need the full weight of this theory, but do make use of the notion
of context in the proof the main theorem. 

\begin{mathpar}
  \inferrule* [lab=summation] {} {{M_{M},M_{N}} \bc \Box \;|\; x.M_{A} \;|\; M_{M}+M_{N}}
  \and
  \inferrule* [lab=agent] {} {{M_{A}} \bc (\vec{x})M_{P} \;| \; \clift{P_0,\ldots,M_{P},\ldots,P_N}}
  \and \\
  \inferrule* [lab=process] {} {{M_{P}} \bc M_{N} \;| \;P|M_{P} }
\end{mathpar} 

\begin{mathpar}
  \inferrule* [lab=sychronization] {} {M_{N} \bc \Box \;|\; x?M_{F} \;|\; x!M_{C}}
  \and
  \inferrule* [lab=abstraction] {} {{M_{F}} \bc (x)M_{P} }
  \and
  \inferrule* [lab=concretion] {} {{M_{C}} \bc \langle M_{P} \rangle }
  \and \\
  \inferrule* [lab=process] {} {{M_{P}} \bc M_{N} \;| \;P|M_{P} }
\end{mathpar}

\begin{definition}[contextual application] Given a context $M$, and
  process $P$, we define the \emph{contextual application}, $M[P] :=
  M\{P/\Box\}$. That is, the contextual application of M to P is the
  substitution of $P$ for $\Box$ in $M$.
\end{definition}

$\meaningof{-} : L \to \mathcal{P}(\pi)$

\begin{mathpar}
  \inferrule* [lab=collection] {} {\meaningof{true} = \pi, \and \meaningof{~E} = \pi \setminus \meaningof{E}, \and \meaningof{E_{1} \& E_{2}} = \meaningof{E_{1}} \cap \meaningof{E_{2}}}
\end{mathpar}

\begin{mathpar}
  \inferrule* [lab=structure] {} {\meaningof{0} = \{ P \in \pi | P \equiv 0 \}, \and \\ \meaningof{E_1 | E_2} = \{ P \in \pi | P \equiv P_{1} | P_{2}, P_{1} \in \meaningof{E_{1}}, P_{2} \in \meaningof{E_2}\} }
\end{mathpar}

\begin{mathpar}
 \inferrule* [lab=behavior] {} {\meaningof{\langle a?b \rangle E} = \{ P \in \pi | P \equiv Q | u?(y)P', \\ \and \\\\ \and \\ \;\;\; u \in \meaningof{a}, \forall z.P'\{z/y\} \in \meaningof{E\{z/b\}}\}, \and \\ \meaningof{a!E} = \{ P \in \pi | P \equiv Q | x!\langle P' \rangle, x \in \meaningof{a} P' \in \meaningof{E}\} }
\end{mathpar}

\begin{mathpar}
 \inferrule* [lab=nominal] {} {\meaningof{\quotep{E}} = \{ \quotep{P} \in \quotep{\pi} | P \in \meaningof{E} \}, \and \meaningof{\quotep{P}} = \{ \quotep{Q} \in \quotep{\pi} | P \equiv Q \} \and \\ \meaningof{@\quotep{E}} = \{ P \in \pi | P \equiv @x, x \in \meaningof{E} \}}
\end{mathpar}

\begin{eqnarray*}
  \\
  \meaningof{-} : TS \to ST
\end{eqnarray*}

\begin{eqnarray*}
  \\
  L : TS \to ST
\end{eqnarray*}

\begin{eqnarray*}
  \\
  P \models E \iff P \in \meaningof{E}
\end{eqnarray*}

\begin{eqnarray*}
  P \approx_{L} Q \iff \forall E \in L. P \models E \iff Q \models E
\end{eqnarray*}

\begin{eqnarray*}
  P \approx_{K} Q
\end{eqnarray*}

\begin{eqnarray*}
  P \approx Q
\end{eqnarray*}

$\approx_{K} = \approx = \approx_{L}$

\subsubsection{Contextual duality}

Note that contexts extend the quotation operation to a family of
operations from processes to names. Given a context, $M$, we can
define a \emph{nominal context}, $\quotep{M}$ by $\quotep{M}[P] :=
\quotep{M[P]}$. To foreshadow what is to come we observe that these
operations enjoy a duality with processes very much like the duality
between vectors and maps from vectors to scalars.

Further, because the calculus is essentially higher-order, we have a
correspondence between contexts and processes. More specifically,
given a name $x$ and a context $M$ we can construct $M^{*}_{x}$ such
that 

\begin{mathpar}
  M^{*}_{x} | \lift{x}{P} \red M[P]
\end{mathpar}

namely,

\begin{mathpar}
  M^{*}_{x} := x?(u).M[\dropn{u}]
\end{mathpar}

The dependence of $M^{*}_{x}$ on a name makes it an abstraction, 

\begin{mathpar}
  M^{*} := (x)x?(u).M[\dropn{u}]
\end{mathpar}

\subsection{Additional notation}

It will sometimes be convenient to denote the process a name
quotes. We already have the notation $x = \quotep{P}$, but it will be
convenient to introduce an alternate notation, $\procn{x}$, when we
want to emphasize the connection to the use of the name. Note that, by
virtue of name equivalence, $\quotep{\procn{x}} \nameeq x$; so, the
notation is consistent with previous definitions.

Further, because names have structure it is possible to effect
substitutions on the basis of that structure. This means we need to
upgrade our notation for substitutions, which we accomplish by
adapting comprehension notation. Thus,

\begin{mathpar}
  P\{ y / x : x \in S \}
\end{mathpar}

is interpreted to mean the process derived from P by replacing (in a
capture-avoiding manner) each occurrence of $x$ in $S$ by $y$. For example,

\begin{mathpar}
  P\{ \quotep{\procn{x}|\procn{x}} / x : x \in \freenames{P} \}
\end{mathpar}

will replace each (occurrence) of a free name $x$ in $P$ by
$\quotep{\procn{x}|\procn{x}}$.

Also, we will avail ourselves of the notation $x^{L}$ and $x^{R}$ to
denote injections of a name into disjoint copies of the name
space. There are numerous ways to accomplish this. One example can be
found in \cite{MeredithR05}. This notation overloads to vectors of
names: $\vec{x}^{\pi} := (x_{i}^{\pi} \; : \; 0 \leq i < |\vec{x}| )$ where $\pi \in \{L,R\}$.

We also use $P^{\Box} := P|\Box$.

In \cite{MeredithR05} an interpretation of the new operator is
given. It turns out that there are several possible interpretations
all enjoying the requisite algebraic properties of the operator (see
\cite{milner91polyadicpi}). We will therefore make liberal use of
$(\nu\; \vec{x})P$.

% subsection the_syntax_and_semantics_of_the_notation_system (end)   

\input{qm2pi.qmops} 

\input{qm2pi.sterngerlach} 

\input{qm2pi.metric} 

% section concurrent_process_calculi (end)

%\input{qm2pi.proofsketch}

% section proof sketch (end)

%\input{qm2pi.slviaknots} 

% section spatial logic via knots (end)

\input{qm2pi.conclusion}

% section conclusion (end)

%\input{qm2pi.dtcodes} 

% section wiring algorithm (end)

\input{qm2pi.ack} 

% section acknowledgments (end)

\newpage


\bibliographystyle{plain}   
\bibliography{../../biblios/main.bib}

\input{qm2pi.rhodetails}

\end{document}

 

\documentclass[12pt]{llncs}
%\documentclass{jktr}

\usepackage[pdftex]{hyperref}                   
\usepackage {listings}
\usepackage {mathpartir}
\usepackage{bcprules}
%\usepackage{listings}
                       
\usepackage{graphicx} 
%\usepackage[margins=2.5cm,nohead,nofoot]{geometry}
%\usepackage{geometry}
\usepackage{amsfonts}
\usepackage{amstext}
\usepackage{latexsym}
\usepackage{amssymb}
\usepackage{color}


%\include{myPreamble}
\include{qm2pi.local} 

%\ifpdf
%\usepackage[pdftex]{graphicx}
%\else
%\usepackage{graphicx}
%\fi

 % \ifpdf
%  \usepackage{pdfsync}
%  \if


%\title{Brief Article}
%\author{David F. Snyder}
%\author{L.G. Meredith}

%\address{Dept. of Math., Texas State University--San Marcos, San Marcos, TX 78666}
       
\pagestyle{empty}


\begin{document}

\lstset{language=[Objective]Caml,frame=shadowbox}

\input{qm2pi.front}

% section front matter (end)

\input{qm2pi.intro} 
 
% section introduction (end)

% \input{qm2pi.knotations} 

% section notation (end)

\input{qm2pi.process.calculi} 

% section concurrent_process_calculi_and_spatial_logics_ (end)
    
%\input{qm2pi.knots2pi} 

%\input{qm2pi.trefoil} 

%\input{qm2pi.mainthm} 

% subsection basic_interpretation (end)

%\input{qm2pi.rho.presentation} 
\subsection{The syntax and semantics of the notation system}\label{sub:the_syntax_and_semantics_of_the_notation_system} % (fold)

We now summarize a technical presentation of the calculus that
embodies our theory of dynamics. The typical presentation of such a
calculus follows the style of giving generators and relations on
them. The grammar, below, describing term constructors, freely
generates the set of processes, $\Proc$. This set is then quotiented
by a relation known as structural congruence and it is over this set
that the notion of dynamics is expressed. This presentation is
essentially that of \cite{MeredithR05} with the addition of
polyadicity and summation. For readability we have relegated some of
the technical subtleties to an appendix.

\subsubsection{Process grammar}\label{subsub:process_grammar}

\begin{mathpar}
  \inferrule* [lab=synchronization] {} {{M} \bc \pzero \;|\; x?F \;|\; x!C }
  \and
  \inferrule* [lab=abstraction] {} {{F} \bc (x)P}
  \and
  \inferrule* [lab=concretion] {} {{C} \bc \langle Q \rangle}
  \and
  \inferrule* [lab=process] {} {{P,Q} \bc M \;| \;P|Q \;|\; @{x}}
  \and
  \inferrule* [lab=name] {} {{x} \bc \quotep{P}}
\end{mathpar} 

Note that $\vec{x}$ (resp. $\vec{P}$) denotes a vector of names
(resp. processes) of length $|\vec{x}|$ (resp. $|\vec{P}|$). We adopt
the following useful abbreviations.

\begin{mathpar}
   x?(\vec{y}).P := x.(\vec{y})P \and  x\clift{\vec{P}} := x.\clift{\vec{P}}
   \and x!(y) := \lift{x}{\dropn{y}}
   \and \Pi_{i=0}^{n-1}P_i := P_0 | \ldots | P_{n-1}
\end{mathpar}

\subsubsection{Structural congruence}

\paragraph{Free and bound names and alpha-equivalence.} At the
core of structural equivalence is alpha-equivalence which identifies
process that are the same up to a change of variable. Formally, we
recognize the distinction between free and bound names. The free names
of a process, $\freenames{P}$, may be calculated recursively as
follows:

\begin{mathpar}
\freenames{\pzero} := \emptyset
  \and \\
  \freenames{x?(y).P} := \{ x \} \cup (\freenames{P} \setminus \{ y \})
  \and 
  \freenames{x!\langle P \rangle} := \{ x \} \cup \{ P \} 
  \and \\
  \freenames{P|Q} := \freenames{P} \cup \freenames{Q}
  \and \\
  \freenames{@{x}} := \{ x \}
\end{mathpar}

$\pi$
$\quotep{\pi}$

$\freenames{-} : \pi \to \mathcal{P}(\quotep{\pi})$

\begin{eqnarray*}
  \freenames{\pzero} & := & \emptyset \\
  \freenames{x?(y).P} & := & \{ x \} \cup (\freenames{P} \setminus \{ y \}) \\
  \freenames{x!\langle P \rangle} & := & \{ x \} \cup \{ P \} \\
  \freenames{P|Q} & := & \freenames{P} \cup \freenames{Q} \\
  \freenames{\dropn{x}} & := & \{ x \}
\end{eqnarray*}

The bound names of a process, $\boundnames{P}$, are those names occurring in $P$
that are not free. For example, in $x?(y).0$, the name $x$ is free, while $y$ is bound.

\begin{mathpar}
  \inferrule* [lab=monoidal-laws] {} { P|Q \equiv Q|P \and P|0 \equiv P \and P|(Q|R) \equiv (P|Q)|R }
\end{mathpar}

\begin{mathpar}
  \inferrule* [lab=alpha-equivalence] {} { (x)P \equiv (y)P\{y/x\} \and y \not\in \freenames{P} }
\end{mathpar}

\begin{definition}
Then two processes, $P,Q$, are alpha-equivalent if $P = Q\{\vec{y}/\vec{x}\}$ for
some $\vec{x} \in \boundnames{Q},\vec{y} \in \boundnames{P}$, where $Q\{\vec{y}/\vec{x}\}$
denotes the capture-avoiding substitution of $\vec{y}$ for $\vec{x}$ in $Q$.
\end{definition}

\begin{definition}
  The {\em structural congruence} \cite{SangiorgiWalker} , $\equiv$,
  between processes is the least congruence containing
  alpha-equivalence, satisfying the abelian monoid laws
  (associativity, commutativity and $\pzero$ as identity) for parallel
  composition $|$ and for summation $+$.
\end{definition}

\subsection{Name equivalence}

We take name equivalence, written $\nameeq$, to be the smallest
equivalence relation generated by the following rules.

\begin{mathpar}
\inferrule*[lab=Quote-drop]
{ }
{ \quotep{@{x}} \nameeq x }

\inferrule*[lab=Struct-equiv]
{ P \scong Q }
{ \quotep{P} \nameeq \quotep{Q} }
\end{mathpar}

The astute reader will have noticed that the mutual recursion of names
and processes imposes a mutual recursion on alpha-equivalence and
structural equivalence via name-equivalence. Fortunately, all of this
works out pleasantly and we may calculate in the natural way, free of
concern. The reader interested in the details is referred to the
appendix \ref{appendix:rho_details}.

\subsection{Substitution}

We use $\Proc$ for the set of processes, $\QProc$ for the set of
names, and $\id{\{}\vec{y} / \vec{x} \id{\}}$ to denote partial maps,
$s : \QProc \rightarrow \QProc$. A map, $s$ lifts, uniquely, to a map
on process terms, $\widehat{s} : \Proc \rightarrow \Proc$ by the
following equations.

\begin{mathpar}
  (0) \psubstp{Q}{P} := 0 \\
  (R \juxtap S) \psubstp{Q}{P}
  :=    
  (R)\psubstp{Q}{P} \juxtap (S) \psubstp{Q}{P} \\
  (x?(y).R) \psubstp{Q}{P}    
  :=    
  (x)\substp{Q}{P} (z)\concat( (R \psubstn{z}{y}) \psubstp{Q}{P} ) \\
  (\lift{x}{R}) \psubstp{Q}{P}  
  :=
  \lift{(x)\substp{Q}{P}}{ R \psubstp{Q}{P} } \\
%   (\dropn{x})  \psubstp{Q}{P}       
%   := 
%   \left\{ 
%     \begin{array}{ccc} 
%       \dropn{\quotep{Q}} & & x \nameeq \quotep{P} \\
%       \dropn{x} & & otherwise \\
%     \end{array}
%   \right. 
  (\dropn{x})  \psubstp{Q}{P}       
  := 
  \left\{ 
    \begin{array}{ccc} 
      Q & & x \nameeq \quotep{P} \\
      \dropn{x} & & otherwise \\
    \end{array}
  \right.
\end{mathpar}
 

where

\begin{eqnarray}
  (x)\id{\{} \lpquote Q \rpquote / \lpquote P \rpquote \id{\}}            = 
  \left\{ 
    \begin{array}{ccc}
      \lpquote Q \rpquote & & x \nameeq \lpquote P \rpquote \\
      x & & otherwise \\
    \end{array}
  \right. \nonumber
\end{eqnarray}

and $z$ is chosen distinct from $\quotep{P}$, $\quotep{Q}$, the free
names in $Q$, and all the names in $R$. Our $\alpha$-equivalence will
be built in the standard way from this substitution.

\begin{remark}\label{rem:no_self_referential_names}
  One consequence of these definitions is that $\forall P. \quotep{P}
  \not\in \freenames{P}$.
\end{remark}

\subsection{ Dynamic quote: an example }

Anticipating something of what's to come, consider applying the
substitution, $\widehat{\id{\{}u / z \id{\}}}$, to the following pair
of processes, $\lift{w}{y!(z)}$ and $w[ \lpquote y!(z) \rpquote ]$.

\begin{eqnarray}
	\lift{w}{y!(z)}\widehat{\id{\{}u / z \id{\}}}
		& = &
		\lift{w}{y!(u)} \nonumber\\
	w[ \lpquote y!(z) \rpquote ] \widehat{ \id{\{}u / z \id{\}} }
		& = &
		w[ \lpquote y!(z) \rpquote ] \nonumber
\end{eqnarray}

Because the body of the process between quotes is impervious to
substitution, we get radically different answers. In fact, by
examining the first process in an input context,
e.g. $x?(z).\lift{w}{y!(z)}$, we see that the process under the lift
operator may be shaped by prefixed inputs binding a name inside it. In
this sense, the lift operator will be seen as a way to dynamically
construct processes before reifying them as names.

Finally equipped with these standard features we can present the
dynamics of the calculus.

\subsubsection{Operational semantics} 

Finally, we introduce the computational dynamics. What marks these
algebras as distinct from other more traditionally studied algebraic
structures, e.g. vector spaces or polynomial rings, is the manner in
which dynamics is captured. In traditional structures, dynamics is typically
expressed through morphisms between such structures, as in linear maps
between vector spaces or morphisms between rings. In algebras
associated with the semantics of computation, the dynamics is
expressed as part of the algebraic structure itself, through a
reduction reduction relation typically denoted by $\red$. Below, we
give a recursive presentation of this relation for the calculus used
in the encoding.

$\red \subseteq \pi \times \pi$
$\red : \pi \to \mathcal{P}(\pi)$

\begin{mathpar}
  \inferrule* [lab=Comm] { \textsf{match}( x_{src}, x_{trgt} ) } { x_{trgt}?(y)P \; | \; x_{src}!\langle {Q} \rangle \red P\{\quotep{Q}/y}\} }
  \and \\
  \inferrule* [lab=Par] {{P} \red {P}'} {{{P} | {Q}} \red {{P}' | {Q}}}
  \and
  \inferrule* [lab=Equiv]{{{P} \scong {P}'} \andalso {{P}' \red {Q}'} \andalso {{Q}' \scong {Q}}}{{P} \red {Q}}
\end{mathpar}

\begin{eqnarray*}
  match_{\equiv} (\quotep{P},\quotep{Q}) & := & P \equiv Q \\
  match_{\dagger}(\quotep{P},\quotep{Q}) & := & \forall R. P|Q \red^{*} R => R \red^{*} 0 \\
  match_{K}(\quotep{P},\quotep{Q}) & := & K \mbox{ for some context } K
\end{eqnarray*}

$u?(x)P | u!\langle Q \rangle \red P\{\quotep{Q}/x\}$

%We write $\wred$ for $\red^*$, and $P\red$ if $\exists Q $ such that $ P \red Q$.
We write $P\red$ if $\exists Q $ such that $ P \red Q$ and $P\not\red$, otherwise.

\section{Replication}

As mentioned before, it is known that replication (and hence
recursion) can be implemented in a higher-order process algebra
\cite{SangiorgiWalker}. As our first example of calculation with the
machinery thus far presented we give the construction explicitly in
the {\rhoc}.

\begin{eqnarray}
	D_{x} & := & \prefix{x}{y}{(\binpar{\outputp{x}{y}}{@{y}})} \nonumber\\
	\bangp_{x}{P} & := & \binpar{{x}!\langle{\binpar{D_{x}}{P}}\rangle}{D_{x}} \nonumber
\end{eqnarray}

\begin{eqnarray}
	\bangp_{x}{P} & & \nonumber\\
	=
	& {x}!\langle{(\prefix{x}{y}{(\outputp{x}{y} | @{y})) | P}}\rangle 
	      | \prefix{x}{y}{(\outputp{x}{y} | @{y})} & \nonumber\\
	\red
	& (\outputp{x}{y} | @{y})\substn{\quotep{(\prefix{x}{y}{(@{y} | \outputp{x}{y})) | P}}}{y} & \nonumber\\
	=
	& \outputp{x}{\quotep{(\prefix{x}{y}{(\outputp{x}{y} | @{y})) | P}}}
	  | {(\prefix{x}{y}{(\outputp{x}{y} | @{y})) | P}} & \nonumber\\
	\red
	& \ldots & \nonumber\\
	\red^*
	& P | P | \ldots & \nonumber
\end{eqnarray}

Of course, this encoding, as an implementation, runs away, unfolding
$\bangp{P}$ eagerly. A lazier and more implementable replication
operator, restricted to input-guarded processes, may be obtained as follows.

\begin{eqnarray}
\bangp{\prefix{u}{v}{P}} 
	:= 
	\binpar{\lift{x}{\prefix{u}{v}{(\binpar{D(x)}{P})}}}{D(x)} \nonumber
\end{eqnarray}

\begin{remark}
  Note that the lazier definition still does not deal with summation
  or mixed summation (i.e. sums over input and output). The reader is
  invited to construct definitions of replication that deal with these
  features. 

  Further, the definitions are parameterized in a name, $x$. Can you,
  gentle reader, make a definition that eliminates this parameter and
  guarantees no accidental interaction between the replication
  machinery and the process being replicated -- i.e. no accidental
  sharing of names used by the process to get its work done and the
  name(s) used by the replication to effect copying. This latter
  revision of the definition of replication is crucial to obtaining
  the expected identity $!!P \sim !P$.
\end{remark}

\begin{remark}\label{rem:paradoxical_combinator}
  The reader familiar with the lambda calculus will have noticed the
  similarity between $D$ and the paradoxical combinator.

  [Ed. note: the existence of this seems to suggest we have to be more
  restrictive on the set of processes and names we admit if we are to
  support no-cloning.]
\end{remark}

\subsubsection{Bisimulation}

The computational dynamics gives rise to another kind of equivalence,
the equivalence of computational behavior. As previously mentioned
this is typically captured \emph{via} some form of bisimulation.

% The notion we use in this paper is weak barbed bisimulation
% \cite{milner91polyadicpi}.

The notion we use in this paper is derived from weak barbed
bisimulation \cite{milner91polyadicpi}. 

\begin{definition}
An \emph{observation relation}, $\downarrow_{\mathcal N}$, over a set
of names, $\mathcal N$, is the smallest relation satisfying the rules
below.

\infrule[Out-barb]{y \in {\mathcal N}, \; x \nameeq y}
		  {\outputp{x}{v} \downarrow_{\mathcal N} x}
\infrule[Par-barb]{\mbox{$P\downarrow_{\mathcal N} x$ or $Q\downarrow_{\mathcal N} x$}}
		  {\binpar{P}{Q} \downarrow_{\mathcal N} x}

We write $P \Downarrow_{\mathcal N} x$ if there is $Q$ such that 
$P \wred Q$ and $Q \downarrow_{\mathcal N} x$.
\end{definition}

\begin{definition}
%\label{def.bbisim}
An  ${\mathcal N}$-\emph{barbed bisimulation} over a set of names, ${\mathcal N}$, is a symmetric binary relation 
${\mathcal S}_{\mathcal N}$ between agents such that $P\rel{S}_{\mathcal N}Q$ implies:
\begin{enumerate}
\item If $P \red P'$ then $Q \wred Q'$ and $P'\rel{S}_{\mathcal N} Q'$.
\item If $P\downarrow_{\mathcal N} x$, then $Q\Downarrow_{\mathcal N} x$.
\end{enumerate}
$P$ is ${\mathcal N}$-barbed bisimilar to $Q$, written
$P \wbbisim_{\mathcal N} Q$, if $P \rel{S}_{\mathcal N} Q$ for some ${\mathcal N}$-barbed bisimulation ${\mathcal S}_{\mathcal N}$.
\end{definition}

$\mathcal{R} \subseteq \pi \times \pi$

$P \mathcal{R} Q => \forall P'. P \red P' \Rightarrow \exists Q'. Q \red Q', P' \mathcal{R} Q'$

$P \vdash x \Rightarrow Q \vdash x$

\begin{mathpar}
  \inferrule*[lab=Out-barb]{x \nameeq y}{{y}!\langle{Q}\rangle \vdash x}
  \and
  \inferrule*[lab=Par-barb]{\mbox{$P\vdash x$ or $Q\vdash x$}}{\binpar{P}{Q} \vdash x}
\end{mathpar}

\subsubsection{Contexts}

One of the principle advantages of computational calculi like the
$\pi$-calculus is a well-defined notion of context,
contextual-equivalence and a correlation between
contextual-equivalence and notions of bisimulation. The notion of
context allows the decomposition of a process into (sub-)process and
its syntactic environment, its context. Thus, a context may be
thought of as a process with a ``hole'' (written $\Box$) in it. The
application of a context $M$ to a process $P$, written $M[P]$, is
tantamount to filling the hole in $M$ with $P$. In this paper we do
not need the full weight of this theory, but do make use of the notion
of context in the proof the main theorem. 

\begin{mathpar}
  \inferrule* [lab=summation] {} {{M_{M},M_{N}} \bc \Box \;|\; x.M_{A} \;|\; M_{M}+M_{N}}
  \and
  \inferrule* [lab=agent] {} {{M_{A}} \bc (\vec{x})M_{P} \;| \; \clift{P_0,\ldots,M_{P},\ldots,P_N}}
  \and \\
  \inferrule* [lab=process] {} {{M_{P}} \bc M_{N} \;| \;P|M_{P} }
\end{mathpar} 

\begin{mathpar}
  \inferrule* [lab=sychronization] {} {M_{N} \bc \Box \;|\; x?M_{F} \;|\; x!M_{C}}
  \and
  \inferrule* [lab=abstraction] {} {{M_{F}} \bc (x)M_{P} }
  \and
  \inferrule* [lab=concretion] {} {{M_{C}} \bc \langle M_{P} \rangle }
  \and \\
  \inferrule* [lab=process] {} {{M_{P}} \bc M_{N} \;| \;P|M_{P} }
\end{mathpar}

\begin{definition}[contextual application] Given a context $M$, and
  process $P$, we define the \emph{contextual application}, $M[P] :=
  M\{P/\Box\}$. That is, the contextual application of M to P is the
  substitution of $P$ for $\Box$ in $M$.
\end{definition}

$\meaningof{-} : L \to \mathcal{P}(\pi)$

\begin{mathpar}
  \inferrule* [lab=collection] {} {\meaningof{true} = \pi, \and \meaningof{~E} = \pi \setminus \meaningof{E}, \and \meaningof{E_{1} \& E_{2}} = \meaningof{E_{1}} \cap \meaningof{E_{2}}}
\end{mathpar}

\begin{mathpar}
  \inferrule* [lab=structure] {} {\meaningof{0} = \{ P \in \pi | P \equiv 0 \}, \and \\ \meaningof{E_1 | E_2} = \{ P \in \pi | P \equiv P_{1} | P_{2}, P_{1} \in \meaningof{E_{1}}, P_{2} \in \meaningof{E_2}\} }
\end{mathpar}

\begin{mathpar}
 \inferrule* [lab=behavior] {} {\meaningof{\langle a?b \rangle E} = \{ P \in \pi | P \equiv Q | u?(y)P', \\ \and \\\\ \and \\ \;\;\; u \in \meaningof{a}, \forall z.P'\{z/y\} \in \meaningof{E\{z/b\}}\}, \and \\ \meaningof{a!E} = \{ P \in \pi | P \equiv Q | x!\langle P' \rangle, x \in \meaningof{a} P' \in \meaningof{E}\} }
\end{mathpar}

\begin{mathpar}
 \inferrule* [lab=nominal] {} {\meaningof{\quotep{E}} = \{ \quotep{P} \in \quotep{\pi} | P \in \meaningof{E} \}, \and \meaningof{\quotep{P}} = \{ \quotep{Q} \in \quotep{\pi} | P \equiv Q \} \and \\ \meaningof{@\quotep{E}} = \{ P \in \pi | P \equiv @x, x \in \meaningof{E} \}}
\end{mathpar}

\begin{eqnarray*}
  \\
  \meaningof{-} : TS \to ST
\end{eqnarray*}

\begin{eqnarray*}
  \\
  L : TS \to ST
\end{eqnarray*}

\begin{eqnarray*}
  \\
  P \models E \iff P \in \meaningof{E}
\end{eqnarray*}

\begin{eqnarray*}
  P \approx_{L} Q \iff \forall E \in L. P \models E \iff Q \models E
\end{eqnarray*}

\begin{eqnarray*}
  P \approx_{K} Q
\end{eqnarray*}

\begin{eqnarray*}
  P \approx Q
\end{eqnarray*}

$\approx_{K} = \approx = \approx_{L}$

\subsubsection{Contextual duality}

Note that contexts extend the quotation operation to a family of
operations from processes to names. Given a context, $M$, we can
define a \emph{nominal context}, $\quotep{M}$ by $\quotep{M}[P] :=
\quotep{M[P]}$. To foreshadow what is to come we observe that these
operations enjoy a duality with processes very much like the duality
between vectors and maps from vectors to scalars.

Further, because the calculus is essentially higher-order, we have a
correspondence between contexts and processes. More specifically,
given a name $x$ and a context $M$ we can construct $M^{*}_{x}$ such
that 

\begin{mathpar}
  M^{*}_{x} | \lift{x}{P} \red M[P]
\end{mathpar}

namely,

\begin{mathpar}
  M^{*}_{x} := x?(u).M[\dropn{u}]
\end{mathpar}

The dependence of $M^{*}_{x}$ on a name makes it an abstraction, 

\begin{mathpar}
  M^{*} := (x)x?(u).M[\dropn{u}]
\end{mathpar}

\subsection{Additional notation}

It will sometimes be convenient to denote the process a name
quotes. We already have the notation $x = \quotep{P}$, but it will be
convenient to introduce an alternate notation, $\procn{x}$, when we
want to emphasize the connection to the use of the name. Note that, by
virtue of name equivalence, $\quotep{\procn{x}} \nameeq x$; so, the
notation is consistent with previous definitions.

Further, because names have structure it is possible to effect
substitutions on the basis of that structure. This means we need to
upgrade our notation for substitutions, which we accomplish by
adapting comprehension notation. Thus,

\begin{mathpar}
  P\{ y / x : x \in S \}
\end{mathpar}

is interpreted to mean the process derived from P by replacing (in a
capture-avoiding manner) each occurrence of $x$ in $S$ by $y$. For example,

\begin{mathpar}
  P\{ \quotep{\procn{x}|\procn{x}} / x : x \in \freenames{P} \}
\end{mathpar}

will replace each (occurrence) of a free name $x$ in $P$ by
$\quotep{\procn{x}|\procn{x}}$.

Also, we will avail ourselves of the notation $x^{L}$ and $x^{R}$ to
denote injections of a name into disjoint copies of the name
space. There are numerous ways to accomplish this. One example can be
found in \cite{MeredithR05}. This notation overloads to vectors of
names: $\vec{x}^{\pi} := (x_{i}^{\pi} \; : \; 0 \leq i < |\vec{x}| )$ where $\pi \in \{L,R\}$.

We also use $P^{\Box} := P|\Box$.

In \cite{MeredithR05} an interpretation of the new operator is
given. It turns out that there are several possible interpretations
all enjoying the requisite algebraic properties of the operator (see
\cite{milner91polyadicpi}). We will therefore make liberal use of
$(\nu\; \vec{x})P$.

% subsection the_syntax_and_semantics_of_the_notation_system (end)   

\input{qm2pi.qmops} 

\input{qm2pi.sterngerlach} 

\input{qm2pi.metric} 

% section concurrent_process_calculi (end)

%\input{qm2pi.proofsketch}

% section proof sketch (end)

%\input{qm2pi.slviaknots} 

% section spatial logic via knots (end)

\input{qm2pi.conclusion}

% section conclusion (end)

%\input{qm2pi.dtcodes} 

% section wiring algorithm (end)

\input{qm2pi.ack} 

% section acknowledgments (end)

\newpage


\bibliographystyle{plain}   
\bibliography{../../biblios/main.bib}

\input{qm2pi.rhodetails}

\end{document}

 

% section concurrent_process_calculi (end)

%\documentclass[12pt]{llncs}
%\documentclass{jktr}

\usepackage[pdftex]{hyperref}                   
\usepackage {listings}
\usepackage {mathpartir}
\usepackage{bcprules}
%\usepackage{listings}
                       
\usepackage{graphicx} 
%\usepackage[margins=2.5cm,nohead,nofoot]{geometry}
%\usepackage{geometry}
\usepackage{amsfonts}
\usepackage{amstext}
\usepackage{latexsym}
\usepackage{amssymb}
\usepackage{color}


%\include{myPreamble}
\include{qm2pi.local} 

%\ifpdf
%\usepackage[pdftex]{graphicx}
%\else
%\usepackage{graphicx}
%\fi

 % \ifpdf
%  \usepackage{pdfsync}
%  \if


%\title{Brief Article}
%\author{David F. Snyder}
%\author{L.G. Meredith}

%\address{Dept. of Math., Texas State University--San Marcos, San Marcos, TX 78666}
       
\pagestyle{empty}


\begin{document}

\lstset{language=[Objective]Caml,frame=shadowbox}

\input{qm2pi.front}

% section front matter (end)

\input{qm2pi.intro} 
 
% section introduction (end)

% \input{qm2pi.knotations} 

% section notation (end)

\input{qm2pi.process.calculi} 

% section concurrent_process_calculi_and_spatial_logics_ (end)
    
%\input{qm2pi.knots2pi} 

%\input{qm2pi.trefoil} 

%\input{qm2pi.mainthm} 

% subsection basic_interpretation (end)

%\input{qm2pi.rho.presentation} 
\subsection{The syntax and semantics of the notation system}\label{sub:the_syntax_and_semantics_of_the_notation_system} % (fold)

We now summarize a technical presentation of the calculus that
embodies our theory of dynamics. The typical presentation of such a
calculus follows the style of giving generators and relations on
them. The grammar, below, describing term constructors, freely
generates the set of processes, $\Proc$. This set is then quotiented
by a relation known as structural congruence and it is over this set
that the notion of dynamics is expressed. This presentation is
essentially that of \cite{MeredithR05} with the addition of
polyadicity and summation. For readability we have relegated some of
the technical subtleties to an appendix.

\subsubsection{Process grammar}\label{subsub:process_grammar}

\begin{mathpar}
  \inferrule* [lab=synchronization] {} {{M} \bc \pzero \;|\; x?F \;|\; x!C }
  \and
  \inferrule* [lab=abstraction] {} {{F} \bc (x)P}
  \and
  \inferrule* [lab=concretion] {} {{C} \bc \langle Q \rangle}
  \and
  \inferrule* [lab=process] {} {{P,Q} \bc M \;| \;P|Q \;|\; @{x}}
  \and
  \inferrule* [lab=name] {} {{x} \bc \quotep{P}}
\end{mathpar} 

Note that $\vec{x}$ (resp. $\vec{P}$) denotes a vector of names
(resp. processes) of length $|\vec{x}|$ (resp. $|\vec{P}|$). We adopt
the following useful abbreviations.

\begin{mathpar}
   x?(\vec{y}).P := x.(\vec{y})P \and  x\clift{\vec{P}} := x.\clift{\vec{P}}
   \and x!(y) := \lift{x}{\dropn{y}}
   \and \Pi_{i=0}^{n-1}P_i := P_0 | \ldots | P_{n-1}
\end{mathpar}

\subsubsection{Structural congruence}

\paragraph{Free and bound names and alpha-equivalence.} At the
core of structural equivalence is alpha-equivalence which identifies
process that are the same up to a change of variable. Formally, we
recognize the distinction between free and bound names. The free names
of a process, $\freenames{P}$, may be calculated recursively as
follows:

\begin{mathpar}
\freenames{\pzero} := \emptyset
  \and \\
  \freenames{x?(y).P} := \{ x \} \cup (\freenames{P} \setminus \{ y \})
  \and 
  \freenames{x!\langle P \rangle} := \{ x \} \cup \{ P \} 
  \and \\
  \freenames{P|Q} := \freenames{P} \cup \freenames{Q}
  \and \\
  \freenames{@{x}} := \{ x \}
\end{mathpar}

$\pi$
$\quotep{\pi}$

$\freenames{-} : \pi \to \mathcal{P}(\quotep{\pi})$

\begin{eqnarray*}
  \freenames{\pzero} & := & \emptyset \\
  \freenames{x?(y).P} & := & \{ x \} \cup (\freenames{P} \setminus \{ y \}) \\
  \freenames{x!\langle P \rangle} & := & \{ x \} \cup \{ P \} \\
  \freenames{P|Q} & := & \freenames{P} \cup \freenames{Q} \\
  \freenames{\dropn{x}} & := & \{ x \}
\end{eqnarray*}

The bound names of a process, $\boundnames{P}$, are those names occurring in $P$
that are not free. For example, in $x?(y).0$, the name $x$ is free, while $y$ is bound.

\begin{mathpar}
  \inferrule* [lab=monoidal-laws] {} { P|Q \equiv Q|P \and P|0 \equiv P \and P|(Q|R) \equiv (P|Q)|R }
\end{mathpar}

\begin{mathpar}
  \inferrule* [lab=alpha-equivalence] {} { (x)P \equiv (y)P\{y/x\} \and y \not\in \freenames{P} }
\end{mathpar}

\begin{definition}
Then two processes, $P,Q$, are alpha-equivalent if $P = Q\{\vec{y}/\vec{x}\}$ for
some $\vec{x} \in \boundnames{Q},\vec{y} \in \boundnames{P}$, where $Q\{\vec{y}/\vec{x}\}$
denotes the capture-avoiding substitution of $\vec{y}$ for $\vec{x}$ in $Q$.
\end{definition}

\begin{definition}
  The {\em structural congruence} \cite{SangiorgiWalker} , $\equiv$,
  between processes is the least congruence containing
  alpha-equivalence, satisfying the abelian monoid laws
  (associativity, commutativity and $\pzero$ as identity) for parallel
  composition $|$ and for summation $+$.
\end{definition}

\subsection{Name equivalence}

We take name equivalence, written $\nameeq$, to be the smallest
equivalence relation generated by the following rules.

\begin{mathpar}
\inferrule*[lab=Quote-drop]
{ }
{ \quotep{@{x}} \nameeq x }

\inferrule*[lab=Struct-equiv]
{ P \scong Q }
{ \quotep{P} \nameeq \quotep{Q} }
\end{mathpar}

The astute reader will have noticed that the mutual recursion of names
and processes imposes a mutual recursion on alpha-equivalence and
structural equivalence via name-equivalence. Fortunately, all of this
works out pleasantly and we may calculate in the natural way, free of
concern. The reader interested in the details is referred to the
appendix \ref{appendix:rho_details}.

\subsection{Substitution}

We use $\Proc$ for the set of processes, $\QProc$ for the set of
names, and $\id{\{}\vec{y} / \vec{x} \id{\}}$ to denote partial maps,
$s : \QProc \rightarrow \QProc$. A map, $s$ lifts, uniquely, to a map
on process terms, $\widehat{s} : \Proc \rightarrow \Proc$ by the
following equations.

\begin{mathpar}
  (0) \psubstp{Q}{P} := 0 \\
  (R \juxtap S) \psubstp{Q}{P}
  :=    
  (R)\psubstp{Q}{P} \juxtap (S) \psubstp{Q}{P} \\
  (x?(y).R) \psubstp{Q}{P}    
  :=    
  (x)\substp{Q}{P} (z)\concat( (R \psubstn{z}{y}) \psubstp{Q}{P} ) \\
  (\lift{x}{R}) \psubstp{Q}{P}  
  :=
  \lift{(x)\substp{Q}{P}}{ R \psubstp{Q}{P} } \\
%   (\dropn{x})  \psubstp{Q}{P}       
%   := 
%   \left\{ 
%     \begin{array}{ccc} 
%       \dropn{\quotep{Q}} & & x \nameeq \quotep{P} \\
%       \dropn{x} & & otherwise \\
%     \end{array}
%   \right. 
  (\dropn{x})  \psubstp{Q}{P}       
  := 
  \left\{ 
    \begin{array}{ccc} 
      Q & & x \nameeq \quotep{P} \\
      \dropn{x} & & otherwise \\
    \end{array}
  \right.
\end{mathpar}
 

where

\begin{eqnarray}
  (x)\id{\{} \lpquote Q \rpquote / \lpquote P \rpquote \id{\}}            = 
  \left\{ 
    \begin{array}{ccc}
      \lpquote Q \rpquote & & x \nameeq \lpquote P \rpquote \\
      x & & otherwise \\
    \end{array}
  \right. \nonumber
\end{eqnarray}

and $z$ is chosen distinct from $\quotep{P}$, $\quotep{Q}$, the free
names in $Q$, and all the names in $R$. Our $\alpha$-equivalence will
be built in the standard way from this substitution.

\begin{remark}\label{rem:no_self_referential_names}
  One consequence of these definitions is that $\forall P. \quotep{P}
  \not\in \freenames{P}$.
\end{remark}

\subsection{ Dynamic quote: an example }

Anticipating something of what's to come, consider applying the
substitution, $\widehat{\id{\{}u / z \id{\}}}$, to the following pair
of processes, $\lift{w}{y!(z)}$ and $w[ \lpquote y!(z) \rpquote ]$.

\begin{eqnarray}
	\lift{w}{y!(z)}\widehat{\id{\{}u / z \id{\}}}
		& = &
		\lift{w}{y!(u)} \nonumber\\
	w[ \lpquote y!(z) \rpquote ] \widehat{ \id{\{}u / z \id{\}} }
		& = &
		w[ \lpquote y!(z) \rpquote ] \nonumber
\end{eqnarray}

Because the body of the process between quotes is impervious to
substitution, we get radically different answers. In fact, by
examining the first process in an input context,
e.g. $x?(z).\lift{w}{y!(z)}$, we see that the process under the lift
operator may be shaped by prefixed inputs binding a name inside it. In
this sense, the lift operator will be seen as a way to dynamically
construct processes before reifying them as names.

Finally equipped with these standard features we can present the
dynamics of the calculus.

\subsubsection{Operational semantics} 

Finally, we introduce the computational dynamics. What marks these
algebras as distinct from other more traditionally studied algebraic
structures, e.g. vector spaces or polynomial rings, is the manner in
which dynamics is captured. In traditional structures, dynamics is typically
expressed through morphisms between such structures, as in linear maps
between vector spaces or morphisms between rings. In algebras
associated with the semantics of computation, the dynamics is
expressed as part of the algebraic structure itself, through a
reduction reduction relation typically denoted by $\red$. Below, we
give a recursive presentation of this relation for the calculus used
in the encoding.

$\red \subseteq \pi \times \pi$
$\red : \pi \to \mathcal{P}(\pi)$

\begin{mathpar}
  \inferrule* [lab=Comm] { \textsf{match}( x_{src}, x_{trgt} ) } { x_{trgt}?(y)P \; | \; x_{src}!\langle {Q} \rangle \red P\{\quotep{Q}/y}\} }
  \and \\
  \inferrule* [lab=Par] {{P} \red {P}'} {{{P} | {Q}} \red {{P}' | {Q}}}
  \and
  \inferrule* [lab=Equiv]{{{P} \scong {P}'} \andalso {{P}' \red {Q}'} \andalso {{Q}' \scong {Q}}}{{P} \red {Q}}
\end{mathpar}

\begin{eqnarray*}
  match_{\equiv} (\quotep{P},\quotep{Q}) & := & P \equiv Q \\
  match_{\dagger}(\quotep{P},\quotep{Q}) & := & \forall R. P|Q \red^{*} R => R \red^{*} 0 \\
  match_{K}(\quotep{P},\quotep{Q}) & := & K \mbox{ for some context } K
\end{eqnarray*}

$u?(x)P | u!\langle Q \rangle \red P\{\quotep{Q}/x\}$

%We write $\wred$ for $\red^*$, and $P\red$ if $\exists Q $ such that $ P \red Q$.
We write $P\red$ if $\exists Q $ such that $ P \red Q$ and $P\not\red$, otherwise.

\section{Replication}

As mentioned before, it is known that replication (and hence
recursion) can be implemented in a higher-order process algebra
\cite{SangiorgiWalker}. As our first example of calculation with the
machinery thus far presented we give the construction explicitly in
the {\rhoc}.

\begin{eqnarray}
	D_{x} & := & \prefix{x}{y}{(\binpar{\outputp{x}{y}}{@{y}})} \nonumber\\
	\bangp_{x}{P} & := & \binpar{{x}!\langle{\binpar{D_{x}}{P}}\rangle}{D_{x}} \nonumber
\end{eqnarray}

\begin{eqnarray}
	\bangp_{x}{P} & & \nonumber\\
	=
	& {x}!\langle{(\prefix{x}{y}{(\outputp{x}{y} | @{y})) | P}}\rangle 
	      | \prefix{x}{y}{(\outputp{x}{y} | @{y})} & \nonumber\\
	\red
	& (\outputp{x}{y} | @{y})\substn{\quotep{(\prefix{x}{y}{(@{y} | \outputp{x}{y})) | P}}}{y} & \nonumber\\
	=
	& \outputp{x}{\quotep{(\prefix{x}{y}{(\outputp{x}{y} | @{y})) | P}}}
	  | {(\prefix{x}{y}{(\outputp{x}{y} | @{y})) | P}} & \nonumber\\
	\red
	& \ldots & \nonumber\\
	\red^*
	& P | P | \ldots & \nonumber
\end{eqnarray}

Of course, this encoding, as an implementation, runs away, unfolding
$\bangp{P}$ eagerly. A lazier and more implementable replication
operator, restricted to input-guarded processes, may be obtained as follows.

\begin{eqnarray}
\bangp{\prefix{u}{v}{P}} 
	:= 
	\binpar{\lift{x}{\prefix{u}{v}{(\binpar{D(x)}{P})}}}{D(x)} \nonumber
\end{eqnarray}

\begin{remark}
  Note that the lazier definition still does not deal with summation
  or mixed summation (i.e. sums over input and output). The reader is
  invited to construct definitions of replication that deal with these
  features. 

  Further, the definitions are parameterized in a name, $x$. Can you,
  gentle reader, make a definition that eliminates this parameter and
  guarantees no accidental interaction between the replication
  machinery and the process being replicated -- i.e. no accidental
  sharing of names used by the process to get its work done and the
  name(s) used by the replication to effect copying. This latter
  revision of the definition of replication is crucial to obtaining
  the expected identity $!!P \sim !P$.
\end{remark}

\begin{remark}\label{rem:paradoxical_combinator}
  The reader familiar with the lambda calculus will have noticed the
  similarity between $D$ and the paradoxical combinator.

  [Ed. note: the existence of this seems to suggest we have to be more
  restrictive on the set of processes and names we admit if we are to
  support no-cloning.]
\end{remark}

\subsubsection{Bisimulation}

The computational dynamics gives rise to another kind of equivalence,
the equivalence of computational behavior. As previously mentioned
this is typically captured \emph{via} some form of bisimulation.

% The notion we use in this paper is weak barbed bisimulation
% \cite{milner91polyadicpi}.

The notion we use in this paper is derived from weak barbed
bisimulation \cite{milner91polyadicpi}. 

\begin{definition}
An \emph{observation relation}, $\downarrow_{\mathcal N}$, over a set
of names, $\mathcal N$, is the smallest relation satisfying the rules
below.

\infrule[Out-barb]{y \in {\mathcal N}, \; x \nameeq y}
		  {\outputp{x}{v} \downarrow_{\mathcal N} x}
\infrule[Par-barb]{\mbox{$P\downarrow_{\mathcal N} x$ or $Q\downarrow_{\mathcal N} x$}}
		  {\binpar{P}{Q} \downarrow_{\mathcal N} x}

We write $P \Downarrow_{\mathcal N} x$ if there is $Q$ such that 
$P \wred Q$ and $Q \downarrow_{\mathcal N} x$.
\end{definition}

\begin{definition}
%\label{def.bbisim}
An  ${\mathcal N}$-\emph{barbed bisimulation} over a set of names, ${\mathcal N}$, is a symmetric binary relation 
${\mathcal S}_{\mathcal N}$ between agents such that $P\rel{S}_{\mathcal N}Q$ implies:
\begin{enumerate}
\item If $P \red P'$ then $Q \wred Q'$ and $P'\rel{S}_{\mathcal N} Q'$.
\item If $P\downarrow_{\mathcal N} x$, then $Q\Downarrow_{\mathcal N} x$.
\end{enumerate}
$P$ is ${\mathcal N}$-barbed bisimilar to $Q$, written
$P \wbbisim_{\mathcal N} Q$, if $P \rel{S}_{\mathcal N} Q$ for some ${\mathcal N}$-barbed bisimulation ${\mathcal S}_{\mathcal N}$.
\end{definition}

$\mathcal{R} \subseteq \pi \times \pi$

$P \mathcal{R} Q => \forall P'. P \red P' \Rightarrow \exists Q'. Q \red Q', P' \mathcal{R} Q'$

$P \vdash x \Rightarrow Q \vdash x$

\begin{mathpar}
  \inferrule*[lab=Out-barb]{x \nameeq y}{{y}!\langle{Q}\rangle \vdash x}
  \and
  \inferrule*[lab=Par-barb]{\mbox{$P\vdash x$ or $Q\vdash x$}}{\binpar{P}{Q} \vdash x}
\end{mathpar}

\subsubsection{Contexts}

One of the principle advantages of computational calculi like the
$\pi$-calculus is a well-defined notion of context,
contextual-equivalence and a correlation between
contextual-equivalence and notions of bisimulation. The notion of
context allows the decomposition of a process into (sub-)process and
its syntactic environment, its context. Thus, a context may be
thought of as a process with a ``hole'' (written $\Box$) in it. The
application of a context $M$ to a process $P$, written $M[P]$, is
tantamount to filling the hole in $M$ with $P$. In this paper we do
not need the full weight of this theory, but do make use of the notion
of context in the proof the main theorem. 

\begin{mathpar}
  \inferrule* [lab=summation] {} {{M_{M},M_{N}} \bc \Box \;|\; x.M_{A} \;|\; M_{M}+M_{N}}
  \and
  \inferrule* [lab=agent] {} {{M_{A}} \bc (\vec{x})M_{P} \;| \; \clift{P_0,\ldots,M_{P},\ldots,P_N}}
  \and \\
  \inferrule* [lab=process] {} {{M_{P}} \bc M_{N} \;| \;P|M_{P} }
\end{mathpar} 

\begin{mathpar}
  \inferrule* [lab=sychronization] {} {M_{N} \bc \Box \;|\; x?M_{F} \;|\; x!M_{C}}
  \and
  \inferrule* [lab=abstraction] {} {{M_{F}} \bc (x)M_{P} }
  \and
  \inferrule* [lab=concretion] {} {{M_{C}} \bc \langle M_{P} \rangle }
  \and \\
  \inferrule* [lab=process] {} {{M_{P}} \bc M_{N} \;| \;P|M_{P} }
\end{mathpar}

\begin{definition}[contextual application] Given a context $M$, and
  process $P$, we define the \emph{contextual application}, $M[P] :=
  M\{P/\Box\}$. That is, the contextual application of M to P is the
  substitution of $P$ for $\Box$ in $M$.
\end{definition}

$\meaningof{-} : L \to \mathcal{P}(\pi)$

\begin{mathpar}
  \inferrule* [lab=collection] {} {\meaningof{true} = \pi, \and \meaningof{~E} = \pi \setminus \meaningof{E}, \and \meaningof{E_{1} \& E_{2}} = \meaningof{E_{1}} \cap \meaningof{E_{2}}}
\end{mathpar}

\begin{mathpar}
  \inferrule* [lab=structure] {} {\meaningof{0} = \{ P \in \pi | P \equiv 0 \}, \and \\ \meaningof{E_1 | E_2} = \{ P \in \pi | P \equiv P_{1} | P_{2}, P_{1} \in \meaningof{E_{1}}, P_{2} \in \meaningof{E_2}\} }
\end{mathpar}

\begin{mathpar}
 \inferrule* [lab=behavior] {} {\meaningof{\langle a?b \rangle E} = \{ P \in \pi | P \equiv Q | u?(y)P', \\ \and \\\\ \and \\ \;\;\; u \in \meaningof{a}, \forall z.P'\{z/y\} \in \meaningof{E\{z/b\}}\}, \and \\ \meaningof{a!E} = \{ P \in \pi | P \equiv Q | x!\langle P' \rangle, x \in \meaningof{a} P' \in \meaningof{E}\} }
\end{mathpar}

\begin{mathpar}
 \inferrule* [lab=nominal] {} {\meaningof{\quotep{E}} = \{ \quotep{P} \in \quotep{\pi} | P \in \meaningof{E} \}, \and \meaningof{\quotep{P}} = \{ \quotep{Q} \in \quotep{\pi} | P \equiv Q \} \and \\ \meaningof{@\quotep{E}} = \{ P \in \pi | P \equiv @x, x \in \meaningof{E} \}}
\end{mathpar}

\begin{eqnarray*}
  \\
  \meaningof{-} : TS \to ST
\end{eqnarray*}

\begin{eqnarray*}
  \\
  L : TS \to ST
\end{eqnarray*}

\begin{eqnarray*}
  \\
  P \models E \iff P \in \meaningof{E}
\end{eqnarray*}

\begin{eqnarray*}
  P \approx_{L} Q \iff \forall E \in L. P \models E \iff Q \models E
\end{eqnarray*}

\begin{eqnarray*}
  P \approx_{K} Q
\end{eqnarray*}

\begin{eqnarray*}
  P \approx Q
\end{eqnarray*}

$\approx_{K} = \approx = \approx_{L}$

\subsubsection{Contextual duality}

Note that contexts extend the quotation operation to a family of
operations from processes to names. Given a context, $M$, we can
define a \emph{nominal context}, $\quotep{M}$ by $\quotep{M}[P] :=
\quotep{M[P]}$. To foreshadow what is to come we observe that these
operations enjoy a duality with processes very much like the duality
between vectors and maps from vectors to scalars.

Further, because the calculus is essentially higher-order, we have a
correspondence between contexts and processes. More specifically,
given a name $x$ and a context $M$ we can construct $M^{*}_{x}$ such
that 

\begin{mathpar}
  M^{*}_{x} | \lift{x}{P} \red M[P]
\end{mathpar}

namely,

\begin{mathpar}
  M^{*}_{x} := x?(u).M[\dropn{u}]
\end{mathpar}

The dependence of $M^{*}_{x}$ on a name makes it an abstraction, 

\begin{mathpar}
  M^{*} := (x)x?(u).M[\dropn{u}]
\end{mathpar}

\subsection{Additional notation}

It will sometimes be convenient to denote the process a name
quotes. We already have the notation $x = \quotep{P}$, but it will be
convenient to introduce an alternate notation, $\procn{x}$, when we
want to emphasize the connection to the use of the name. Note that, by
virtue of name equivalence, $\quotep{\procn{x}} \nameeq x$; so, the
notation is consistent with previous definitions.

Further, because names have structure it is possible to effect
substitutions on the basis of that structure. This means we need to
upgrade our notation for substitutions, which we accomplish by
adapting comprehension notation. Thus,

\begin{mathpar}
  P\{ y / x : x \in S \}
\end{mathpar}

is interpreted to mean the process derived from P by replacing (in a
capture-avoiding manner) each occurrence of $x$ in $S$ by $y$. For example,

\begin{mathpar}
  P\{ \quotep{\procn{x}|\procn{x}} / x : x \in \freenames{P} \}
\end{mathpar}

will replace each (occurrence) of a free name $x$ in $P$ by
$\quotep{\procn{x}|\procn{x}}$.

Also, we will avail ourselves of the notation $x^{L}$ and $x^{R}$ to
denote injections of a name into disjoint copies of the name
space. There are numerous ways to accomplish this. One example can be
found in \cite{MeredithR05}. This notation overloads to vectors of
names: $\vec{x}^{\pi} := (x_{i}^{\pi} \; : \; 0 \leq i < |\vec{x}| )$ where $\pi \in \{L,R\}$.

We also use $P^{\Box} := P|\Box$.

In \cite{MeredithR05} an interpretation of the new operator is
given. It turns out that there are several possible interpretations
all enjoying the requisite algebraic properties of the operator (see
\cite{milner91polyadicpi}). We will therefore make liberal use of
$(\nu\; \vec{x})P$.

% subsection the_syntax_and_semantics_of_the_notation_system (end)   

\input{qm2pi.qmops} 

\input{qm2pi.sterngerlach} 

\input{qm2pi.metric} 

% section concurrent_process_calculi (end)

%\input{qm2pi.proofsketch}

% section proof sketch (end)

%\input{qm2pi.slviaknots} 

% section spatial logic via knots (end)

\input{qm2pi.conclusion}

% section conclusion (end)

%\input{qm2pi.dtcodes} 

% section wiring algorithm (end)

\input{qm2pi.ack} 

% section acknowledgments (end)

\newpage


\bibliographystyle{plain}   
\bibliography{../../biblios/main.bib}

\input{qm2pi.rhodetails}

\end{document}



% section proof sketch (end)

%\section{Unlikely characters: spatial logic for
  knots}\label{sub:characteristic_formulae} % (fold)

Associated to the mobile process calculi are a family of logics known
as the Hennessy-Milner logics. These logics typically enjoy a
semantics interpreting formulae as sets of processes that when
factored through the encoding outlined above allows an identification
of classes of knots with logical formulae. In the context of this
encoding the sub-family known as the spatial logics \cite{CairesC03}
\cite{CairesC04} \cite{Caires04} are of particular interest providing
several important features for expressing and reasoning about
properties (i.e. classes) of knots. We hint here at how this may be done.

%\begin{description}
%\item [structural connectives] 
\subsubsection{Structural connectives} The spatial logics enjoy
structural connectives corresponding, at the logical level, to the
parallel composition ($P | Q$) and new name ($(\nu \; x)P$)
connectives for processes. As illustrated in the examples below, these
connectives are extremely expressive given the shape of our encoding.
%\item [decideable satisfaction]

\subsubsection{Decideable satisfaction}
In \cite{Caires04} the satisfaction relation is shown to be decideable
for a rich class of processes. It further turns out that the image of
the our encoding is a proper subset of that class. This result
provides the basis for an algorithm by which to search for knots
enjoying a given property.
%\item [characteristic formulae]

\subsubsection{Characteristic formulae}
In the same paper \cite{Caires04} , Caires presents a means of calculating
characteristic formulae, selecting equivalence classes of processes
up to a pre--specified depth limit on the support set of names. Composed with our
encoding, this characteristic formula can be used to select
characteristic formulae for knots.
%\end{description}

\subsubsection{Spatial logic formulae}

The grammar below (segmented for comprehension) summarizes the syntax
of spatial logic formulae. We employ illustrative examples in the
sequel to provide an intuitive understanding of their meaning
referring the reader to \cite{Caires04} for a more detailed explication
of the semantics.

\begin{mathpar}
  \inferrule* [lab=boolean] {} {{A,B} \bc T \;|\; \neg A \;|\; A \wedge B \;|\; \eta = \eta'}
  \and
  \inferrule* [lab=spatial] {} {|\; \pzero \;|\; A | B \;|\; x \text{\textregistered} A \;|\; \forall x . A \;|\;  H x . A}
  \and
  \inferrule* [lab=behavioral] {} {|\; \alpha . A}
  \and 
  \inferrule* [lab=recursion] {} {|\; X(\vec{u}) \;|\; \mu X(\vec{u}) . A}
  \and
  \inferrule* [lab=action] {} {\alpha \bc \langle x?(\vec{y}) \rangle \;|\; \langle x!(\vec{y}) \rangle \;|\; \langle \tau \rangle}
  \and 
  \inferrule* [lab=name] {} {\eta \bc x \;|\; \tau}
\end{mathpar} 

% subsection characteristic_formulae (end)   	 

\subsection{Example formulae}\label{sub:example_formulae_} % (fold)

\subsubsection{Crossing as formula.}
% 
% \begin{align*}
%   \frac{d}{dx} \sin x &= \cos x 
%   & \frac{d}{dx} e^x &= e^x \\
%   \frac{d}{dx} \cos x &= - \sin x 
%   & \frac{d}{dx} \log x &= \frac{1}{x} \\
% \end{align*} 

\begin{align*}
 \mu C(x_{0},x_{1},y_{0},y_{1},u).&(\langle x_{0}?(z) \rangle(\langle u! \rangle\langle y_{1}!z \rangle C(x_{0},x_{1},y_{0},y_{1},u)) & \\
  & \wedge \langle y_{1}?(z) \rangle (\langle u! \rangle \langle x_{0}!z \rangle C(x_{0},x_{1},y_{0},y_{1},u)) & \\
  & \wedge \langle x_{1}?(z) \rangle (\langle u? \rangle \langle y_{0}!z \rangle C(x_{0},x_{1},y_{0},y_{1},u)) & \\
  & \wedge \langle y_{0}?(z) \rangle (\langle u? \rangle \langle x_{1}!z \rangle C(x_{0},x_{1},y_{0},y_{1},u))) &
\end{align*}

The lexicographical similarity between the shape of this formulae and
the shape of definition of the process representing a crossing reveals
the intuitive meaning of this formulae. It describes the capabilities
of a process that has the right to represent a crossing. For example
it picks out processes that may perform an input on the port $x_0$ in
its initial menu of capabilities. What differentiates the formula
from the process, however, is that the crossing process is the
smallest candidate to satisfy the formula. Infinitely many other
processes -- with internal behavior hidden behind this interface, so
to speak -- also satisfy this formula. Even this simple formula,
then, can be seen to open a new view onto knots, providing a
computational interpretation of \emph{virtual} knots.

Note that this formula is derived by hand. A similar formula can be
derived by employing Caires' calculation of characteristic formula
\cite{Caires04} to the process representing a crossing. In light of
this discussion, we let
$\meaningof{C}_{\phi}(x0,x1,y0,y1,u)$ denote a formula specifying the
dynamics we wish to capture of a crossing. To guarantee we preserve
the shape of the interface and minimal semantics we demand that
$\meaningof{C}_{\phi}(x0,x1,y0,y1,u) \Rightarrow
\textbf{C}(x0,x1,y0,y1,u)$ where $\textbf{C}(x0,x1,y0,y1,u)$ denotes
the formula above.
                            
\subsubsection{Crossing number constraints.}
The moral content of the context lemma (Lemma \ref{context}) is that the notion of
``locality'' in the Reidemeister moves is effectively captured by the
parallel composition operator of the process calculus. This intuition
extends through the logic. Given a formula,
$\meaningof{C}_{\phi}(x0,x1,y0,y1,u)$, we can use the structural
connectives to specify constraints on crossing numbers, such as at
least $n$ crossings, or exactly $n$ crossings.
\begin{mathpar}
  \inferrule* [lab=at-least-n] {} { K^{\geq n}_{\phi}(\vec{xs},\vec{ys}) := \Pi_{i=0}^{n-1} Hu . \meaningof{C}_{\phi}(xs_i,ys_i,u) | T }
  \and 
  \inferrule* [lab=exactly-n] {} { K^{= n}_{\phi}(\vec{xs},\vec{ys}) := \Pi_{i=0}^{n-1} Hu . \meaningof{C}_{\phi}(xs_i,ys_i,u) | \neg (\forall x_0,y_0,x_1,y_1,u . \meaningof{C}_{\phi}(x_0,y_0,x_1,y_1,u) | T) }
\end{mathpar}

To round out this section, recall that the encoding of an $n$-crossing
knot decomposes into a parallel composition of $n$ \emph{copies} of a
crossing process together with a wiring harness. To specify different
knot classes with the same crossing number amounts to specifying
logical constraints on the wiring harness. In the interest of space,
we defer examples to a forthcoming paper. Suffice it to say that both
the conditions ``alternating knot'' and ``contains the tangle
corresponding to 5/3'' are expressible. For example, it is possible to
calculate the characteristic formula of a process corresponding to the
tangle 5/3 and conjoin it into the classifying formula via the
composition connective of the logic.

Finally, we wish to observe that it is entirely within reason to
contemplate a more domain-specific version of spatial logic tailored
to the shape of processes in the image of the encoding. Such a
domain-specific logic would have a better claim to the title formal
language of knot properties.

% subsection example_formulae_ (end)

% section knots_as_processes (end) 

% section spatial logic via knots (end)

\section{Conclusions and future work}

\paragraph{Testing physical space}
You, gentle reader, may wonder why of all the theorems to be proved
given this set up we pick the one above. In some sense it's hardly
central to quantum mechanics. We see it as central in the sense that
it firmly establishes a notion of physical space arising from a notion
of the equivalence of behavior. Relating bisimulation to a metric is a
big step forward, but one is faced with interpreting the relationship
of that metric space to something more physical. Quantum mechanical
notions of ``physical'' space are still far from intuitive, but by
relating this idea of distance as testing to calculations that predict
physical circumstances we are making a not insignificant step forward
toward an understanding of the physical space we inhabit as
essentially dynamic.

\paragraph{Effectivity and simulation}
One of the observations we have yet to make is that the entire program
spelled out here is effective. We have built various interpreters for
the reflective calculus at work in this interpretation. In principle,
then, we can simulate quantum mechanics on a computer. The place where
the simulation may lose fidelity is the infinitely branching summation
for the annihilator.

In this connection i also want to point out that the evaluation style
calculation of the inner product puts the non-determinism of the
summation right at the heart of measurement. This suggests that
Milner's original reduction-based formulation of the dynamics of his
calculi in terms of sums was not just notationally suggestive of a
notion of measure-and-continue but captured some significant part of
the physics.

\paragraph{Quantum continuations}
In light of this last observation i want to point out that the
predominant account of quantum mechanics is missing a key aspect of a
truly compositional story of the physical situation. In a real lab,
when a measurement is made the observation can be made to feed into
another device that then makes another measurement conditioned on the
results of the first. This means that after the superposition was
collapsed the entire experimental set up remained in
superposition. While QM offers a means of writing this down it doesn't
quite line up well with the well-trodden formulation of computation
and continuation that we see so succinctly expressed in Milner's
calculi. This suggests that there might be advantages to this account
of dynamics waiting to be explored.

\paragraph{Quantum logic}
In this connection, we also note that by virtue of having the
Hennessy-Milner construction, we can pull the construction through the
interpretation of QM. This gives us a natural candidate for a quantum
logic that enjoys an extremely tight connection with it's domain of
interpretation, making the construction much less ad hoc (rather it is
the image of functor!).

\paragraph{Quantum probabiity}
i have questions about the basis of the interpretation of inner
product as probability amplitude. In particular, using which
axiomatization of probability theory does the notion of probability
amplitude earn the right to be so dubbed? In other words, where is the
proof that the operation for calculating a probability amplitude (and
then squaring) satisfies the axioms of what it means to calculate a
probability? Even if such a proof exists (i have yet to find it in the
literature), i wonder if it might not be possible to turn things on
their heads. Can we view the calculation of the probability amplitude
as an axiomatization of probability? If so, then the definition we
give for calculating probability amplitude may provide the basis for
an \emph{effective} theory of probability.

\paragraph{Quantum vs ``biological'' information}
Finally, i want to conclude with a more philosophical observation. At
a recent workshop in which QM was a predominant topic i noticed
something about quantum information. The speaker was giving a riveting
discussion of axiomatic QM and showing how properties of ``no
cloning'' and ``no deleting'' emerged as consequences of the
axiomatization. Theorems of this form are necessary to give us a sense
of confidence that our axioms characterize the physical theory. What
struck me, though, was that if quantum information is neither erasable
nor replicable it is markedly different from \emph{life}. Two of the
things we know about life is that

\begin{itemize}
  \item it ends;
  \item to gain some measure of persistence, to transcend it's
    finitude it is imminently copyable.
\end{itemize}

Both of these qualities are summarized succinctly in the aphorism: all
flesh is grass. For me these two kinds of ``information'' -- call them
quantum and biological -- are end points on a spectrum of strategies
for persistence. At one end, we have those curious entities that enjoy
uniqueness and permanence; at the other, we have those who in the face
of a certain end and an uncertain present make a go of passing
something on. To me one of the more remarkable aspects of the latter
strategy is that in the presence of noise (and certain features of
copying) we get a kind of dynamism, a chance for improvement against a
given persistent condition.

% subsection other_calculi_other_bisimulations_and_geometry_as_behavior (end)




% section conclusion (end)

%\documentclass[12pt]{llncs}
%\documentclass{jktr}

\usepackage[pdftex]{hyperref}                   
\usepackage {listings}
\usepackage {mathpartir}
\usepackage{bcprules}
%\usepackage{listings}
                       
\usepackage{graphicx} 
%\usepackage[margins=2.5cm,nohead,nofoot]{geometry}
%\usepackage{geometry}
\usepackage{amsfonts}
\usepackage{amstext}
\usepackage{latexsym}
\usepackage{amssymb}
\usepackage{color}


%\include{myPreamble}
\include{qm2pi.local} 

%\ifpdf
%\usepackage[pdftex]{graphicx}
%\else
%\usepackage{graphicx}
%\fi

 % \ifpdf
%  \usepackage{pdfsync}
%  \if


%\title{Brief Article}
%\author{David F. Snyder}
%\author{L.G. Meredith}

%\address{Dept. of Math., Texas State University--San Marcos, San Marcos, TX 78666}
       
\pagestyle{empty}


\begin{document}

\lstset{language=[Objective]Caml,frame=shadowbox}

\input{qm2pi.front}

% section front matter (end)

\input{qm2pi.intro} 
 
% section introduction (end)

% \input{qm2pi.knotations} 

% section notation (end)

\input{qm2pi.process.calculi} 

% section concurrent_process_calculi_and_spatial_logics_ (end)
    
%\input{qm2pi.knots2pi} 

%\input{qm2pi.trefoil} 

%\input{qm2pi.mainthm} 

% subsection basic_interpretation (end)

%\input{qm2pi.rho.presentation} 
\subsection{The syntax and semantics of the notation system}\label{sub:the_syntax_and_semantics_of_the_notation_system} % (fold)

We now summarize a technical presentation of the calculus that
embodies our theory of dynamics. The typical presentation of such a
calculus follows the style of giving generators and relations on
them. The grammar, below, describing term constructors, freely
generates the set of processes, $\Proc$. This set is then quotiented
by a relation known as structural congruence and it is over this set
that the notion of dynamics is expressed. This presentation is
essentially that of \cite{MeredithR05} with the addition of
polyadicity and summation. For readability we have relegated some of
the technical subtleties to an appendix.

\subsubsection{Process grammar}\label{subsub:process_grammar}

\begin{mathpar}
  \inferrule* [lab=synchronization] {} {{M} \bc \pzero \;|\; x?F \;|\; x!C }
  \and
  \inferrule* [lab=abstraction] {} {{F} \bc (x)P}
  \and
  \inferrule* [lab=concretion] {} {{C} \bc \langle Q \rangle}
  \and
  \inferrule* [lab=process] {} {{P,Q} \bc M \;| \;P|Q \;|\; @{x}}
  \and
  \inferrule* [lab=name] {} {{x} \bc \quotep{P}}
\end{mathpar} 

Note that $\vec{x}$ (resp. $\vec{P}$) denotes a vector of names
(resp. processes) of length $|\vec{x}|$ (resp. $|\vec{P}|$). We adopt
the following useful abbreviations.

\begin{mathpar}
   x?(\vec{y}).P := x.(\vec{y})P \and  x\clift{\vec{P}} := x.\clift{\vec{P}}
   \and x!(y) := \lift{x}{\dropn{y}}
   \and \Pi_{i=0}^{n-1}P_i := P_0 | \ldots | P_{n-1}
\end{mathpar}

\subsubsection{Structural congruence}

\paragraph{Free and bound names and alpha-equivalence.} At the
core of structural equivalence is alpha-equivalence which identifies
process that are the same up to a change of variable. Formally, we
recognize the distinction between free and bound names. The free names
of a process, $\freenames{P}$, may be calculated recursively as
follows:

\begin{mathpar}
\freenames{\pzero} := \emptyset
  \and \\
  \freenames{x?(y).P} := \{ x \} \cup (\freenames{P} \setminus \{ y \})
  \and 
  \freenames{x!\langle P \rangle} := \{ x \} \cup \{ P \} 
  \and \\
  \freenames{P|Q} := \freenames{P} \cup \freenames{Q}
  \and \\
  \freenames{@{x}} := \{ x \}
\end{mathpar}

$\pi$
$\quotep{\pi}$

$\freenames{-} : \pi \to \mathcal{P}(\quotep{\pi})$

\begin{eqnarray*}
  \freenames{\pzero} & := & \emptyset \\
  \freenames{x?(y).P} & := & \{ x \} \cup (\freenames{P} \setminus \{ y \}) \\
  \freenames{x!\langle P \rangle} & := & \{ x \} \cup \{ P \} \\
  \freenames{P|Q} & := & \freenames{P} \cup \freenames{Q} \\
  \freenames{\dropn{x}} & := & \{ x \}
\end{eqnarray*}

The bound names of a process, $\boundnames{P}$, are those names occurring in $P$
that are not free. For example, in $x?(y).0$, the name $x$ is free, while $y$ is bound.

\begin{mathpar}
  \inferrule* [lab=monoidal-laws] {} { P|Q \equiv Q|P \and P|0 \equiv P \and P|(Q|R) \equiv (P|Q)|R }
\end{mathpar}

\begin{mathpar}
  \inferrule* [lab=alpha-equivalence] {} { (x)P \equiv (y)P\{y/x\} \and y \not\in \freenames{P} }
\end{mathpar}

\begin{definition}
Then two processes, $P,Q$, are alpha-equivalent if $P = Q\{\vec{y}/\vec{x}\}$ for
some $\vec{x} \in \boundnames{Q},\vec{y} \in \boundnames{P}$, where $Q\{\vec{y}/\vec{x}\}$
denotes the capture-avoiding substitution of $\vec{y}$ for $\vec{x}$ in $Q$.
\end{definition}

\begin{definition}
  The {\em structural congruence} \cite{SangiorgiWalker} , $\equiv$,
  between processes is the least congruence containing
  alpha-equivalence, satisfying the abelian monoid laws
  (associativity, commutativity and $\pzero$ as identity) for parallel
  composition $|$ and for summation $+$.
\end{definition}

\subsection{Name equivalence}

We take name equivalence, written $\nameeq$, to be the smallest
equivalence relation generated by the following rules.

\begin{mathpar}
\inferrule*[lab=Quote-drop]
{ }
{ \quotep{@{x}} \nameeq x }

\inferrule*[lab=Struct-equiv]
{ P \scong Q }
{ \quotep{P} \nameeq \quotep{Q} }
\end{mathpar}

The astute reader will have noticed that the mutual recursion of names
and processes imposes a mutual recursion on alpha-equivalence and
structural equivalence via name-equivalence. Fortunately, all of this
works out pleasantly and we may calculate in the natural way, free of
concern. The reader interested in the details is referred to the
appendix \ref{appendix:rho_details}.

\subsection{Substitution}

We use $\Proc$ for the set of processes, $\QProc$ for the set of
names, and $\id{\{}\vec{y} / \vec{x} \id{\}}$ to denote partial maps,
$s : \QProc \rightarrow \QProc$. A map, $s$ lifts, uniquely, to a map
on process terms, $\widehat{s} : \Proc \rightarrow \Proc$ by the
following equations.

\begin{mathpar}
  (0) \psubstp{Q}{P} := 0 \\
  (R \juxtap S) \psubstp{Q}{P}
  :=    
  (R)\psubstp{Q}{P} \juxtap (S) \psubstp{Q}{P} \\
  (x?(y).R) \psubstp{Q}{P}    
  :=    
  (x)\substp{Q}{P} (z)\concat( (R \psubstn{z}{y}) \psubstp{Q}{P} ) \\
  (\lift{x}{R}) \psubstp{Q}{P}  
  :=
  \lift{(x)\substp{Q}{P}}{ R \psubstp{Q}{P} } \\
%   (\dropn{x})  \psubstp{Q}{P}       
%   := 
%   \left\{ 
%     \begin{array}{ccc} 
%       \dropn{\quotep{Q}} & & x \nameeq \quotep{P} \\
%       \dropn{x} & & otherwise \\
%     \end{array}
%   \right. 
  (\dropn{x})  \psubstp{Q}{P}       
  := 
  \left\{ 
    \begin{array}{ccc} 
      Q & & x \nameeq \quotep{P} \\
      \dropn{x} & & otherwise \\
    \end{array}
  \right.
\end{mathpar}
 

where

\begin{eqnarray}
  (x)\id{\{} \lpquote Q \rpquote / \lpquote P \rpquote \id{\}}            = 
  \left\{ 
    \begin{array}{ccc}
      \lpquote Q \rpquote & & x \nameeq \lpquote P \rpquote \\
      x & & otherwise \\
    \end{array}
  \right. \nonumber
\end{eqnarray}

and $z$ is chosen distinct from $\quotep{P}$, $\quotep{Q}$, the free
names in $Q$, and all the names in $R$. Our $\alpha$-equivalence will
be built in the standard way from this substitution.

\begin{remark}\label{rem:no_self_referential_names}
  One consequence of these definitions is that $\forall P. \quotep{P}
  \not\in \freenames{P}$.
\end{remark}

\subsection{ Dynamic quote: an example }

Anticipating something of what's to come, consider applying the
substitution, $\widehat{\id{\{}u / z \id{\}}}$, to the following pair
of processes, $\lift{w}{y!(z)}$ and $w[ \lpquote y!(z) \rpquote ]$.

\begin{eqnarray}
	\lift{w}{y!(z)}\widehat{\id{\{}u / z \id{\}}}
		& = &
		\lift{w}{y!(u)} \nonumber\\
	w[ \lpquote y!(z) \rpquote ] \widehat{ \id{\{}u / z \id{\}} }
		& = &
		w[ \lpquote y!(z) \rpquote ] \nonumber
\end{eqnarray}

Because the body of the process between quotes is impervious to
substitution, we get radically different answers. In fact, by
examining the first process in an input context,
e.g. $x?(z).\lift{w}{y!(z)}$, we see that the process under the lift
operator may be shaped by prefixed inputs binding a name inside it. In
this sense, the lift operator will be seen as a way to dynamically
construct processes before reifying them as names.

Finally equipped with these standard features we can present the
dynamics of the calculus.

\subsubsection{Operational semantics} 

Finally, we introduce the computational dynamics. What marks these
algebras as distinct from other more traditionally studied algebraic
structures, e.g. vector spaces or polynomial rings, is the manner in
which dynamics is captured. In traditional structures, dynamics is typically
expressed through morphisms between such structures, as in linear maps
between vector spaces or morphisms between rings. In algebras
associated with the semantics of computation, the dynamics is
expressed as part of the algebraic structure itself, through a
reduction reduction relation typically denoted by $\red$. Below, we
give a recursive presentation of this relation for the calculus used
in the encoding.

$\red \subseteq \pi \times \pi$
$\red : \pi \to \mathcal{P}(\pi)$

\begin{mathpar}
  \inferrule* [lab=Comm] { \textsf{match}( x_{src}, x_{trgt} ) } { x_{trgt}?(y)P \; | \; x_{src}!\langle {Q} \rangle \red P\{\quotep{Q}/y}\} }
  \and \\
  \inferrule* [lab=Par] {{P} \red {P}'} {{{P} | {Q}} \red {{P}' | {Q}}}
  \and
  \inferrule* [lab=Equiv]{{{P} \scong {P}'} \andalso {{P}' \red {Q}'} \andalso {{Q}' \scong {Q}}}{{P} \red {Q}}
\end{mathpar}

\begin{eqnarray*}
  match_{\equiv} (\quotep{P},\quotep{Q}) & := & P \equiv Q \\
  match_{\dagger}(\quotep{P},\quotep{Q}) & := & \forall R. P|Q \red^{*} R => R \red^{*} 0 \\
  match_{K}(\quotep{P},\quotep{Q}) & := & K \mbox{ for some context } K
\end{eqnarray*}

$u?(x)P | u!\langle Q \rangle \red P\{\quotep{Q}/x\}$

%We write $\wred$ for $\red^*$, and $P\red$ if $\exists Q $ such that $ P \red Q$.
We write $P\red$ if $\exists Q $ such that $ P \red Q$ and $P\not\red$, otherwise.

\section{Replication}

As mentioned before, it is known that replication (and hence
recursion) can be implemented in a higher-order process algebra
\cite{SangiorgiWalker}. As our first example of calculation with the
machinery thus far presented we give the construction explicitly in
the {\rhoc}.

\begin{eqnarray}
	D_{x} & := & \prefix{x}{y}{(\binpar{\outputp{x}{y}}{@{y}})} \nonumber\\
	\bangp_{x}{P} & := & \binpar{{x}!\langle{\binpar{D_{x}}{P}}\rangle}{D_{x}} \nonumber
\end{eqnarray}

\begin{eqnarray}
	\bangp_{x}{P} & & \nonumber\\
	=
	& {x}!\langle{(\prefix{x}{y}{(\outputp{x}{y} | @{y})) | P}}\rangle 
	      | \prefix{x}{y}{(\outputp{x}{y} | @{y})} & \nonumber\\
	\red
	& (\outputp{x}{y} | @{y})\substn{\quotep{(\prefix{x}{y}{(@{y} | \outputp{x}{y})) | P}}}{y} & \nonumber\\
	=
	& \outputp{x}{\quotep{(\prefix{x}{y}{(\outputp{x}{y} | @{y})) | P}}}
	  | {(\prefix{x}{y}{(\outputp{x}{y} | @{y})) | P}} & \nonumber\\
	\red
	& \ldots & \nonumber\\
	\red^*
	& P | P | \ldots & \nonumber
\end{eqnarray}

Of course, this encoding, as an implementation, runs away, unfolding
$\bangp{P}$ eagerly. A lazier and more implementable replication
operator, restricted to input-guarded processes, may be obtained as follows.

\begin{eqnarray}
\bangp{\prefix{u}{v}{P}} 
	:= 
	\binpar{\lift{x}{\prefix{u}{v}{(\binpar{D(x)}{P})}}}{D(x)} \nonumber
\end{eqnarray}

\begin{remark}
  Note that the lazier definition still does not deal with summation
  or mixed summation (i.e. sums over input and output). The reader is
  invited to construct definitions of replication that deal with these
  features. 

  Further, the definitions are parameterized in a name, $x$. Can you,
  gentle reader, make a definition that eliminates this parameter and
  guarantees no accidental interaction between the replication
  machinery and the process being replicated -- i.e. no accidental
  sharing of names used by the process to get its work done and the
  name(s) used by the replication to effect copying. This latter
  revision of the definition of replication is crucial to obtaining
  the expected identity $!!P \sim !P$.
\end{remark}

\begin{remark}\label{rem:paradoxical_combinator}
  The reader familiar with the lambda calculus will have noticed the
  similarity between $D$ and the paradoxical combinator.

  [Ed. note: the existence of this seems to suggest we have to be more
  restrictive on the set of processes and names we admit if we are to
  support no-cloning.]
\end{remark}

\subsubsection{Bisimulation}

The computational dynamics gives rise to another kind of equivalence,
the equivalence of computational behavior. As previously mentioned
this is typically captured \emph{via} some form of bisimulation.

% The notion we use in this paper is weak barbed bisimulation
% \cite{milner91polyadicpi}.

The notion we use in this paper is derived from weak barbed
bisimulation \cite{milner91polyadicpi}. 

\begin{definition}
An \emph{observation relation}, $\downarrow_{\mathcal N}$, over a set
of names, $\mathcal N$, is the smallest relation satisfying the rules
below.

\infrule[Out-barb]{y \in {\mathcal N}, \; x \nameeq y}
		  {\outputp{x}{v} \downarrow_{\mathcal N} x}
\infrule[Par-barb]{\mbox{$P\downarrow_{\mathcal N} x$ or $Q\downarrow_{\mathcal N} x$}}
		  {\binpar{P}{Q} \downarrow_{\mathcal N} x}

We write $P \Downarrow_{\mathcal N} x$ if there is $Q$ such that 
$P \wred Q$ and $Q \downarrow_{\mathcal N} x$.
\end{definition}

\begin{definition}
%\label{def.bbisim}
An  ${\mathcal N}$-\emph{barbed bisimulation} over a set of names, ${\mathcal N}$, is a symmetric binary relation 
${\mathcal S}_{\mathcal N}$ between agents such that $P\rel{S}_{\mathcal N}Q$ implies:
\begin{enumerate}
\item If $P \red P'$ then $Q \wred Q'$ and $P'\rel{S}_{\mathcal N} Q'$.
\item If $P\downarrow_{\mathcal N} x$, then $Q\Downarrow_{\mathcal N} x$.
\end{enumerate}
$P$ is ${\mathcal N}$-barbed bisimilar to $Q$, written
$P \wbbisim_{\mathcal N} Q$, if $P \rel{S}_{\mathcal N} Q$ for some ${\mathcal N}$-barbed bisimulation ${\mathcal S}_{\mathcal N}$.
\end{definition}

$\mathcal{R} \subseteq \pi \times \pi$

$P \mathcal{R} Q => \forall P'. P \red P' \Rightarrow \exists Q'. Q \red Q', P' \mathcal{R} Q'$

$P \vdash x \Rightarrow Q \vdash x$

\begin{mathpar}
  \inferrule*[lab=Out-barb]{x \nameeq y}{{y}!\langle{Q}\rangle \vdash x}
  \and
  \inferrule*[lab=Par-barb]{\mbox{$P\vdash x$ or $Q\vdash x$}}{\binpar{P}{Q} \vdash x}
\end{mathpar}

\subsubsection{Contexts}

One of the principle advantages of computational calculi like the
$\pi$-calculus is a well-defined notion of context,
contextual-equivalence and a correlation between
contextual-equivalence and notions of bisimulation. The notion of
context allows the decomposition of a process into (sub-)process and
its syntactic environment, its context. Thus, a context may be
thought of as a process with a ``hole'' (written $\Box$) in it. The
application of a context $M$ to a process $P$, written $M[P]$, is
tantamount to filling the hole in $M$ with $P$. In this paper we do
not need the full weight of this theory, but do make use of the notion
of context in the proof the main theorem. 

\begin{mathpar}
  \inferrule* [lab=summation] {} {{M_{M},M_{N}} \bc \Box \;|\; x.M_{A} \;|\; M_{M}+M_{N}}
  \and
  \inferrule* [lab=agent] {} {{M_{A}} \bc (\vec{x})M_{P} \;| \; \clift{P_0,\ldots,M_{P},\ldots,P_N}}
  \and \\
  \inferrule* [lab=process] {} {{M_{P}} \bc M_{N} \;| \;P|M_{P} }
\end{mathpar} 

\begin{mathpar}
  \inferrule* [lab=sychronization] {} {M_{N} \bc \Box \;|\; x?M_{F} \;|\; x!M_{C}}
  \and
  \inferrule* [lab=abstraction] {} {{M_{F}} \bc (x)M_{P} }
  \and
  \inferrule* [lab=concretion] {} {{M_{C}} \bc \langle M_{P} \rangle }
  \and \\
  \inferrule* [lab=process] {} {{M_{P}} \bc M_{N} \;| \;P|M_{P} }
\end{mathpar}

\begin{definition}[contextual application] Given a context $M$, and
  process $P$, we define the \emph{contextual application}, $M[P] :=
  M\{P/\Box\}$. That is, the contextual application of M to P is the
  substitution of $P$ for $\Box$ in $M$.
\end{definition}

$\meaningof{-} : L \to \mathcal{P}(\pi)$

\begin{mathpar}
  \inferrule* [lab=collection] {} {\meaningof{true} = \pi, \and \meaningof{~E} = \pi \setminus \meaningof{E}, \and \meaningof{E_{1} \& E_{2}} = \meaningof{E_{1}} \cap \meaningof{E_{2}}}
\end{mathpar}

\begin{mathpar}
  \inferrule* [lab=structure] {} {\meaningof{0} = \{ P \in \pi | P \equiv 0 \}, \and \\ \meaningof{E_1 | E_2} = \{ P \in \pi | P \equiv P_{1} | P_{2}, P_{1} \in \meaningof{E_{1}}, P_{2} \in \meaningof{E_2}\} }
\end{mathpar}

\begin{mathpar}
 \inferrule* [lab=behavior] {} {\meaningof{\langle a?b \rangle E} = \{ P \in \pi | P \equiv Q | u?(y)P', \\ \and \\\\ \and \\ \;\;\; u \in \meaningof{a}, \forall z.P'\{z/y\} \in \meaningof{E\{z/b\}}\}, \and \\ \meaningof{a!E} = \{ P \in \pi | P \equiv Q | x!\langle P' \rangle, x \in \meaningof{a} P' \in \meaningof{E}\} }
\end{mathpar}

\begin{mathpar}
 \inferrule* [lab=nominal] {} {\meaningof{\quotep{E}} = \{ \quotep{P} \in \quotep{\pi} | P \in \meaningof{E} \}, \and \meaningof{\quotep{P}} = \{ \quotep{Q} \in \quotep{\pi} | P \equiv Q \} \and \\ \meaningof{@\quotep{E}} = \{ P \in \pi | P \equiv @x, x \in \meaningof{E} \}}
\end{mathpar}

\begin{eqnarray*}
  \\
  \meaningof{-} : TS \to ST
\end{eqnarray*}

\begin{eqnarray*}
  \\
  L : TS \to ST
\end{eqnarray*}

\begin{eqnarray*}
  \\
  P \models E \iff P \in \meaningof{E}
\end{eqnarray*}

\begin{eqnarray*}
  P \approx_{L} Q \iff \forall E \in L. P \models E \iff Q \models E
\end{eqnarray*}

\begin{eqnarray*}
  P \approx_{K} Q
\end{eqnarray*}

\begin{eqnarray*}
  P \approx Q
\end{eqnarray*}

$\approx_{K} = \approx = \approx_{L}$

\subsubsection{Contextual duality}

Note that contexts extend the quotation operation to a family of
operations from processes to names. Given a context, $M$, we can
define a \emph{nominal context}, $\quotep{M}$ by $\quotep{M}[P] :=
\quotep{M[P]}$. To foreshadow what is to come we observe that these
operations enjoy a duality with processes very much like the duality
between vectors and maps from vectors to scalars.

Further, because the calculus is essentially higher-order, we have a
correspondence between contexts and processes. More specifically,
given a name $x$ and a context $M$ we can construct $M^{*}_{x}$ such
that 

\begin{mathpar}
  M^{*}_{x} | \lift{x}{P} \red M[P]
\end{mathpar}

namely,

\begin{mathpar}
  M^{*}_{x} := x?(u).M[\dropn{u}]
\end{mathpar}

The dependence of $M^{*}_{x}$ on a name makes it an abstraction, 

\begin{mathpar}
  M^{*} := (x)x?(u).M[\dropn{u}]
\end{mathpar}

\subsection{Additional notation}

It will sometimes be convenient to denote the process a name
quotes. We already have the notation $x = \quotep{P}$, but it will be
convenient to introduce an alternate notation, $\procn{x}$, when we
want to emphasize the connection to the use of the name. Note that, by
virtue of name equivalence, $\quotep{\procn{x}} \nameeq x$; so, the
notation is consistent with previous definitions.

Further, because names have structure it is possible to effect
substitutions on the basis of that structure. This means we need to
upgrade our notation for substitutions, which we accomplish by
adapting comprehension notation. Thus,

\begin{mathpar}
  P\{ y / x : x \in S \}
\end{mathpar}

is interpreted to mean the process derived from P by replacing (in a
capture-avoiding manner) each occurrence of $x$ in $S$ by $y$. For example,

\begin{mathpar}
  P\{ \quotep{\procn{x}|\procn{x}} / x : x \in \freenames{P} \}
\end{mathpar}

will replace each (occurrence) of a free name $x$ in $P$ by
$\quotep{\procn{x}|\procn{x}}$.

Also, we will avail ourselves of the notation $x^{L}$ and $x^{R}$ to
denote injections of a name into disjoint copies of the name
space. There are numerous ways to accomplish this. One example can be
found in \cite{MeredithR05}. This notation overloads to vectors of
names: $\vec{x}^{\pi} := (x_{i}^{\pi} \; : \; 0 \leq i < |\vec{x}| )$ where $\pi \in \{L,R\}$.

We also use $P^{\Box} := P|\Box$.

In \cite{MeredithR05} an interpretation of the new operator is
given. It turns out that there are several possible interpretations
all enjoying the requisite algebraic properties of the operator (see
\cite{milner91polyadicpi}). We will therefore make liberal use of
$(\nu\; \vec{x})P$.

% subsection the_syntax_and_semantics_of_the_notation_system (end)   

\input{qm2pi.qmops} 

\input{qm2pi.sterngerlach} 

\input{qm2pi.metric} 

% section concurrent_process_calculi (end)

%\input{qm2pi.proofsketch}

% section proof sketch (end)

%\input{qm2pi.slviaknots} 

% section spatial logic via knots (end)

\input{qm2pi.conclusion}

% section conclusion (end)

%\input{qm2pi.dtcodes} 

% section wiring algorithm (end)

\input{qm2pi.ack} 

% section acknowledgments (end)

\newpage


\bibliographystyle{plain}   
\bibliography{../../biblios/main.bib}

\input{qm2pi.rhodetails}

\end{document}

 

% section wiring algorithm (end)

\documentclass[12pt]{llncs}
%\documentclass{jktr}

\usepackage[pdftex]{hyperref}                   
\usepackage {listings}
\usepackage {mathpartir}
\usepackage{bcprules}
%\usepackage{listings}
                       
\usepackage{graphicx} 
%\usepackage[margins=2.5cm,nohead,nofoot]{geometry}
%\usepackage{geometry}
\usepackage{amsfonts}
\usepackage{amstext}
\usepackage{latexsym}
\usepackage{amssymb}
\usepackage{color}


%\include{myPreamble}
\include{qm2pi.local} 

%\ifpdf
%\usepackage[pdftex]{graphicx}
%\else
%\usepackage{graphicx}
%\fi

 % \ifpdf
%  \usepackage{pdfsync}
%  \if


%\title{Brief Article}
%\author{David F. Snyder}
%\author{L.G. Meredith}

%\address{Dept. of Math., Texas State University--San Marcos, San Marcos, TX 78666}
       
\pagestyle{empty}


\begin{document}

\lstset{language=[Objective]Caml,frame=shadowbox}

\input{qm2pi.front}

% section front matter (end)

\input{qm2pi.intro} 
 
% section introduction (end)

% \input{qm2pi.knotations} 

% section notation (end)

\input{qm2pi.process.calculi} 

% section concurrent_process_calculi_and_spatial_logics_ (end)
    
%\input{qm2pi.knots2pi} 

%\input{qm2pi.trefoil} 

%\input{qm2pi.mainthm} 

% subsection basic_interpretation (end)

%\input{qm2pi.rho.presentation} 
\subsection{The syntax and semantics of the notation system}\label{sub:the_syntax_and_semantics_of_the_notation_system} % (fold)

We now summarize a technical presentation of the calculus that
embodies our theory of dynamics. The typical presentation of such a
calculus follows the style of giving generators and relations on
them. The grammar, below, describing term constructors, freely
generates the set of processes, $\Proc$. This set is then quotiented
by a relation known as structural congruence and it is over this set
that the notion of dynamics is expressed. This presentation is
essentially that of \cite{MeredithR05} with the addition of
polyadicity and summation. For readability we have relegated some of
the technical subtleties to an appendix.

\subsubsection{Process grammar}\label{subsub:process_grammar}

\begin{mathpar}
  \inferrule* [lab=synchronization] {} {{M} \bc \pzero \;|\; x?F \;|\; x!C }
  \and
  \inferrule* [lab=abstraction] {} {{F} \bc (x)P}
  \and
  \inferrule* [lab=concretion] {} {{C} \bc \langle Q \rangle}
  \and
  \inferrule* [lab=process] {} {{P,Q} \bc M \;| \;P|Q \;|\; @{x}}
  \and
  \inferrule* [lab=name] {} {{x} \bc \quotep{P}}
\end{mathpar} 

Note that $\vec{x}$ (resp. $\vec{P}$) denotes a vector of names
(resp. processes) of length $|\vec{x}|$ (resp. $|\vec{P}|$). We adopt
the following useful abbreviations.

\begin{mathpar}
   x?(\vec{y}).P := x.(\vec{y})P \and  x\clift{\vec{P}} := x.\clift{\vec{P}}
   \and x!(y) := \lift{x}{\dropn{y}}
   \and \Pi_{i=0}^{n-1}P_i := P_0 | \ldots | P_{n-1}
\end{mathpar}

\subsubsection{Structural congruence}

\paragraph{Free and bound names and alpha-equivalence.} At the
core of structural equivalence is alpha-equivalence which identifies
process that are the same up to a change of variable. Formally, we
recognize the distinction between free and bound names. The free names
of a process, $\freenames{P}$, may be calculated recursively as
follows:

\begin{mathpar}
\freenames{\pzero} := \emptyset
  \and \\
  \freenames{x?(y).P} := \{ x \} \cup (\freenames{P} \setminus \{ y \})
  \and 
  \freenames{x!\langle P \rangle} := \{ x \} \cup \{ P \} 
  \and \\
  \freenames{P|Q} := \freenames{P} \cup \freenames{Q}
  \and \\
  \freenames{@{x}} := \{ x \}
\end{mathpar}

$\pi$
$\quotep{\pi}$

$\freenames{-} : \pi \to \mathcal{P}(\quotep{\pi})$

\begin{eqnarray*}
  \freenames{\pzero} & := & \emptyset \\
  \freenames{x?(y).P} & := & \{ x \} \cup (\freenames{P} \setminus \{ y \}) \\
  \freenames{x!\langle P \rangle} & := & \{ x \} \cup \{ P \} \\
  \freenames{P|Q} & := & \freenames{P} \cup \freenames{Q} \\
  \freenames{\dropn{x}} & := & \{ x \}
\end{eqnarray*}

The bound names of a process, $\boundnames{P}$, are those names occurring in $P$
that are not free. For example, in $x?(y).0$, the name $x$ is free, while $y$ is bound.

\begin{mathpar}
  \inferrule* [lab=monoidal-laws] {} { P|Q \equiv Q|P \and P|0 \equiv P \and P|(Q|R) \equiv (P|Q)|R }
\end{mathpar}

\begin{mathpar}
  \inferrule* [lab=alpha-equivalence] {} { (x)P \equiv (y)P\{y/x\} \and y \not\in \freenames{P} }
\end{mathpar}

\begin{definition}
Then two processes, $P,Q$, are alpha-equivalent if $P = Q\{\vec{y}/\vec{x}\}$ for
some $\vec{x} \in \boundnames{Q},\vec{y} \in \boundnames{P}$, where $Q\{\vec{y}/\vec{x}\}$
denotes the capture-avoiding substitution of $\vec{y}$ for $\vec{x}$ in $Q$.
\end{definition}

\begin{definition}
  The {\em structural congruence} \cite{SangiorgiWalker} , $\equiv$,
  between processes is the least congruence containing
  alpha-equivalence, satisfying the abelian monoid laws
  (associativity, commutativity and $\pzero$ as identity) for parallel
  composition $|$ and for summation $+$.
\end{definition}

\subsection{Name equivalence}

We take name equivalence, written $\nameeq$, to be the smallest
equivalence relation generated by the following rules.

\begin{mathpar}
\inferrule*[lab=Quote-drop]
{ }
{ \quotep{@{x}} \nameeq x }

\inferrule*[lab=Struct-equiv]
{ P \scong Q }
{ \quotep{P} \nameeq \quotep{Q} }
\end{mathpar}

The astute reader will have noticed that the mutual recursion of names
and processes imposes a mutual recursion on alpha-equivalence and
structural equivalence via name-equivalence. Fortunately, all of this
works out pleasantly and we may calculate in the natural way, free of
concern. The reader interested in the details is referred to the
appendix \ref{appendix:rho_details}.

\subsection{Substitution}

We use $\Proc$ for the set of processes, $\QProc$ for the set of
names, and $\id{\{}\vec{y} / \vec{x} \id{\}}$ to denote partial maps,
$s : \QProc \rightarrow \QProc$. A map, $s$ lifts, uniquely, to a map
on process terms, $\widehat{s} : \Proc \rightarrow \Proc$ by the
following equations.

\begin{mathpar}
  (0) \psubstp{Q}{P} := 0 \\
  (R \juxtap S) \psubstp{Q}{P}
  :=    
  (R)\psubstp{Q}{P} \juxtap (S) \psubstp{Q}{P} \\
  (x?(y).R) \psubstp{Q}{P}    
  :=    
  (x)\substp{Q}{P} (z)\concat( (R \psubstn{z}{y}) \psubstp{Q}{P} ) \\
  (\lift{x}{R}) \psubstp{Q}{P}  
  :=
  \lift{(x)\substp{Q}{P}}{ R \psubstp{Q}{P} } \\
%   (\dropn{x})  \psubstp{Q}{P}       
%   := 
%   \left\{ 
%     \begin{array}{ccc} 
%       \dropn{\quotep{Q}} & & x \nameeq \quotep{P} \\
%       \dropn{x} & & otherwise \\
%     \end{array}
%   \right. 
  (\dropn{x})  \psubstp{Q}{P}       
  := 
  \left\{ 
    \begin{array}{ccc} 
      Q & & x \nameeq \quotep{P} \\
      \dropn{x} & & otherwise \\
    \end{array}
  \right.
\end{mathpar}
 

where

\begin{eqnarray}
  (x)\id{\{} \lpquote Q \rpquote / \lpquote P \rpquote \id{\}}            = 
  \left\{ 
    \begin{array}{ccc}
      \lpquote Q \rpquote & & x \nameeq \lpquote P \rpquote \\
      x & & otherwise \\
    \end{array}
  \right. \nonumber
\end{eqnarray}

and $z$ is chosen distinct from $\quotep{P}$, $\quotep{Q}$, the free
names in $Q$, and all the names in $R$. Our $\alpha$-equivalence will
be built in the standard way from this substitution.

\begin{remark}\label{rem:no_self_referential_names}
  One consequence of these definitions is that $\forall P. \quotep{P}
  \not\in \freenames{P}$.
\end{remark}

\subsection{ Dynamic quote: an example }

Anticipating something of what's to come, consider applying the
substitution, $\widehat{\id{\{}u / z \id{\}}}$, to the following pair
of processes, $\lift{w}{y!(z)}$ and $w[ \lpquote y!(z) \rpquote ]$.

\begin{eqnarray}
	\lift{w}{y!(z)}\widehat{\id{\{}u / z \id{\}}}
		& = &
		\lift{w}{y!(u)} \nonumber\\
	w[ \lpquote y!(z) \rpquote ] \widehat{ \id{\{}u / z \id{\}} }
		& = &
		w[ \lpquote y!(z) \rpquote ] \nonumber
\end{eqnarray}

Because the body of the process between quotes is impervious to
substitution, we get radically different answers. In fact, by
examining the first process in an input context,
e.g. $x?(z).\lift{w}{y!(z)}$, we see that the process under the lift
operator may be shaped by prefixed inputs binding a name inside it. In
this sense, the lift operator will be seen as a way to dynamically
construct processes before reifying them as names.

Finally equipped with these standard features we can present the
dynamics of the calculus.

\subsubsection{Operational semantics} 

Finally, we introduce the computational dynamics. What marks these
algebras as distinct from other more traditionally studied algebraic
structures, e.g. vector spaces or polynomial rings, is the manner in
which dynamics is captured. In traditional structures, dynamics is typically
expressed through morphisms between such structures, as in linear maps
between vector spaces or morphisms between rings. In algebras
associated with the semantics of computation, the dynamics is
expressed as part of the algebraic structure itself, through a
reduction reduction relation typically denoted by $\red$. Below, we
give a recursive presentation of this relation for the calculus used
in the encoding.

$\red \subseteq \pi \times \pi$
$\red : \pi \to \mathcal{P}(\pi)$

\begin{mathpar}
  \inferrule* [lab=Comm] { \textsf{match}( x_{src}, x_{trgt} ) } { x_{trgt}?(y)P \; | \; x_{src}!\langle {Q} \rangle \red P\{\quotep{Q}/y}\} }
  \and \\
  \inferrule* [lab=Par] {{P} \red {P}'} {{{P} | {Q}} \red {{P}' | {Q}}}
  \and
  \inferrule* [lab=Equiv]{{{P} \scong {P}'} \andalso {{P}' \red {Q}'} \andalso {{Q}' \scong {Q}}}{{P} \red {Q}}
\end{mathpar}

\begin{eqnarray*}
  match_{\equiv} (\quotep{P},\quotep{Q}) & := & P \equiv Q \\
  match_{\dagger}(\quotep{P},\quotep{Q}) & := & \forall R. P|Q \red^{*} R => R \red^{*} 0 \\
  match_{K}(\quotep{P},\quotep{Q}) & := & K \mbox{ for some context } K
\end{eqnarray*}

$u?(x)P | u!\langle Q \rangle \red P\{\quotep{Q}/x\}$

%We write $\wred$ for $\red^*$, and $P\red$ if $\exists Q $ such that $ P \red Q$.
We write $P\red$ if $\exists Q $ such that $ P \red Q$ and $P\not\red$, otherwise.

\section{Replication}

As mentioned before, it is known that replication (and hence
recursion) can be implemented in a higher-order process algebra
\cite{SangiorgiWalker}. As our first example of calculation with the
machinery thus far presented we give the construction explicitly in
the {\rhoc}.

\begin{eqnarray}
	D_{x} & := & \prefix{x}{y}{(\binpar{\outputp{x}{y}}{@{y}})} \nonumber\\
	\bangp_{x}{P} & := & \binpar{{x}!\langle{\binpar{D_{x}}{P}}\rangle}{D_{x}} \nonumber
\end{eqnarray}

\begin{eqnarray}
	\bangp_{x}{P} & & \nonumber\\
	=
	& {x}!\langle{(\prefix{x}{y}{(\outputp{x}{y} | @{y})) | P}}\rangle 
	      | \prefix{x}{y}{(\outputp{x}{y} | @{y})} & \nonumber\\
	\red
	& (\outputp{x}{y} | @{y})\substn{\quotep{(\prefix{x}{y}{(@{y} | \outputp{x}{y})) | P}}}{y} & \nonumber\\
	=
	& \outputp{x}{\quotep{(\prefix{x}{y}{(\outputp{x}{y} | @{y})) | P}}}
	  | {(\prefix{x}{y}{(\outputp{x}{y} | @{y})) | P}} & \nonumber\\
	\red
	& \ldots & \nonumber\\
	\red^*
	& P | P | \ldots & \nonumber
\end{eqnarray}

Of course, this encoding, as an implementation, runs away, unfolding
$\bangp{P}$ eagerly. A lazier and more implementable replication
operator, restricted to input-guarded processes, may be obtained as follows.

\begin{eqnarray}
\bangp{\prefix{u}{v}{P}} 
	:= 
	\binpar{\lift{x}{\prefix{u}{v}{(\binpar{D(x)}{P})}}}{D(x)} \nonumber
\end{eqnarray}

\begin{remark}
  Note that the lazier definition still does not deal with summation
  or mixed summation (i.e. sums over input and output). The reader is
  invited to construct definitions of replication that deal with these
  features. 

  Further, the definitions are parameterized in a name, $x$. Can you,
  gentle reader, make a definition that eliminates this parameter and
  guarantees no accidental interaction between the replication
  machinery and the process being replicated -- i.e. no accidental
  sharing of names used by the process to get its work done and the
  name(s) used by the replication to effect copying. This latter
  revision of the definition of replication is crucial to obtaining
  the expected identity $!!P \sim !P$.
\end{remark}

\begin{remark}\label{rem:paradoxical_combinator}
  The reader familiar with the lambda calculus will have noticed the
  similarity between $D$ and the paradoxical combinator.

  [Ed. note: the existence of this seems to suggest we have to be more
  restrictive on the set of processes and names we admit if we are to
  support no-cloning.]
\end{remark}

\subsubsection{Bisimulation}

The computational dynamics gives rise to another kind of equivalence,
the equivalence of computational behavior. As previously mentioned
this is typically captured \emph{via} some form of bisimulation.

% The notion we use in this paper is weak barbed bisimulation
% \cite{milner91polyadicpi}.

The notion we use in this paper is derived from weak barbed
bisimulation \cite{milner91polyadicpi}. 

\begin{definition}
An \emph{observation relation}, $\downarrow_{\mathcal N}$, over a set
of names, $\mathcal N$, is the smallest relation satisfying the rules
below.

\infrule[Out-barb]{y \in {\mathcal N}, \; x \nameeq y}
		  {\outputp{x}{v} \downarrow_{\mathcal N} x}
\infrule[Par-barb]{\mbox{$P\downarrow_{\mathcal N} x$ or $Q\downarrow_{\mathcal N} x$}}
		  {\binpar{P}{Q} \downarrow_{\mathcal N} x}

We write $P \Downarrow_{\mathcal N} x$ if there is $Q$ such that 
$P \wred Q$ and $Q \downarrow_{\mathcal N} x$.
\end{definition}

\begin{definition}
%\label{def.bbisim}
An  ${\mathcal N}$-\emph{barbed bisimulation} over a set of names, ${\mathcal N}$, is a symmetric binary relation 
${\mathcal S}_{\mathcal N}$ between agents such that $P\rel{S}_{\mathcal N}Q$ implies:
\begin{enumerate}
\item If $P \red P'$ then $Q \wred Q'$ and $P'\rel{S}_{\mathcal N} Q'$.
\item If $P\downarrow_{\mathcal N} x$, then $Q\Downarrow_{\mathcal N} x$.
\end{enumerate}
$P$ is ${\mathcal N}$-barbed bisimilar to $Q$, written
$P \wbbisim_{\mathcal N} Q$, if $P \rel{S}_{\mathcal N} Q$ for some ${\mathcal N}$-barbed bisimulation ${\mathcal S}_{\mathcal N}$.
\end{definition}

$\mathcal{R} \subseteq \pi \times \pi$

$P \mathcal{R} Q => \forall P'. P \red P' \Rightarrow \exists Q'. Q \red Q', P' \mathcal{R} Q'$

$P \vdash x \Rightarrow Q \vdash x$

\begin{mathpar}
  \inferrule*[lab=Out-barb]{x \nameeq y}{{y}!\langle{Q}\rangle \vdash x}
  \and
  \inferrule*[lab=Par-barb]{\mbox{$P\vdash x$ or $Q\vdash x$}}{\binpar{P}{Q} \vdash x}
\end{mathpar}

\subsubsection{Contexts}

One of the principle advantages of computational calculi like the
$\pi$-calculus is a well-defined notion of context,
contextual-equivalence and a correlation between
contextual-equivalence and notions of bisimulation. The notion of
context allows the decomposition of a process into (sub-)process and
its syntactic environment, its context. Thus, a context may be
thought of as a process with a ``hole'' (written $\Box$) in it. The
application of a context $M$ to a process $P$, written $M[P]$, is
tantamount to filling the hole in $M$ with $P$. In this paper we do
not need the full weight of this theory, but do make use of the notion
of context in the proof the main theorem. 

\begin{mathpar}
  \inferrule* [lab=summation] {} {{M_{M},M_{N}} \bc \Box \;|\; x.M_{A} \;|\; M_{M}+M_{N}}
  \and
  \inferrule* [lab=agent] {} {{M_{A}} \bc (\vec{x})M_{P} \;| \; \clift{P_0,\ldots,M_{P},\ldots,P_N}}
  \and \\
  \inferrule* [lab=process] {} {{M_{P}} \bc M_{N} \;| \;P|M_{P} }
\end{mathpar} 

\begin{mathpar}
  \inferrule* [lab=sychronization] {} {M_{N} \bc \Box \;|\; x?M_{F} \;|\; x!M_{C}}
  \and
  \inferrule* [lab=abstraction] {} {{M_{F}} \bc (x)M_{P} }
  \and
  \inferrule* [lab=concretion] {} {{M_{C}} \bc \langle M_{P} \rangle }
  \and \\
  \inferrule* [lab=process] {} {{M_{P}} \bc M_{N} \;| \;P|M_{P} }
\end{mathpar}

\begin{definition}[contextual application] Given a context $M$, and
  process $P$, we define the \emph{contextual application}, $M[P] :=
  M\{P/\Box\}$. That is, the contextual application of M to P is the
  substitution of $P$ for $\Box$ in $M$.
\end{definition}

$\meaningof{-} : L \to \mathcal{P}(\pi)$

\begin{mathpar}
  \inferrule* [lab=collection] {} {\meaningof{true} = \pi, \and \meaningof{~E} = \pi \setminus \meaningof{E}, \and \meaningof{E_{1} \& E_{2}} = \meaningof{E_{1}} \cap \meaningof{E_{2}}}
\end{mathpar}

\begin{mathpar}
  \inferrule* [lab=structure] {} {\meaningof{0} = \{ P \in \pi | P \equiv 0 \}, \and \\ \meaningof{E_1 | E_2} = \{ P \in \pi | P \equiv P_{1} | P_{2}, P_{1} \in \meaningof{E_{1}}, P_{2} \in \meaningof{E_2}\} }
\end{mathpar}

\begin{mathpar}
 \inferrule* [lab=behavior] {} {\meaningof{\langle a?b \rangle E} = \{ P \in \pi | P \equiv Q | u?(y)P', \\ \and \\\\ \and \\ \;\;\; u \in \meaningof{a}, \forall z.P'\{z/y\} \in \meaningof{E\{z/b\}}\}, \and \\ \meaningof{a!E} = \{ P \in \pi | P \equiv Q | x!\langle P' \rangle, x \in \meaningof{a} P' \in \meaningof{E}\} }
\end{mathpar}

\begin{mathpar}
 \inferrule* [lab=nominal] {} {\meaningof{\quotep{E}} = \{ \quotep{P} \in \quotep{\pi} | P \in \meaningof{E} \}, \and \meaningof{\quotep{P}} = \{ \quotep{Q} \in \quotep{\pi} | P \equiv Q \} \and \\ \meaningof{@\quotep{E}} = \{ P \in \pi | P \equiv @x, x \in \meaningof{E} \}}
\end{mathpar}

\begin{eqnarray*}
  \\
  \meaningof{-} : TS \to ST
\end{eqnarray*}

\begin{eqnarray*}
  \\
  L : TS \to ST
\end{eqnarray*}

\begin{eqnarray*}
  \\
  P \models E \iff P \in \meaningof{E}
\end{eqnarray*}

\begin{eqnarray*}
  P \approx_{L} Q \iff \forall E \in L. P \models E \iff Q \models E
\end{eqnarray*}

\begin{eqnarray*}
  P \approx_{K} Q
\end{eqnarray*}

\begin{eqnarray*}
  P \approx Q
\end{eqnarray*}

$\approx_{K} = \approx = \approx_{L}$

\subsubsection{Contextual duality}

Note that contexts extend the quotation operation to a family of
operations from processes to names. Given a context, $M$, we can
define a \emph{nominal context}, $\quotep{M}$ by $\quotep{M}[P] :=
\quotep{M[P]}$. To foreshadow what is to come we observe that these
operations enjoy a duality with processes very much like the duality
between vectors and maps from vectors to scalars.

Further, because the calculus is essentially higher-order, we have a
correspondence between contexts and processes. More specifically,
given a name $x$ and a context $M$ we can construct $M^{*}_{x}$ such
that 

\begin{mathpar}
  M^{*}_{x} | \lift{x}{P} \red M[P]
\end{mathpar}

namely,

\begin{mathpar}
  M^{*}_{x} := x?(u).M[\dropn{u}]
\end{mathpar}

The dependence of $M^{*}_{x}$ on a name makes it an abstraction, 

\begin{mathpar}
  M^{*} := (x)x?(u).M[\dropn{u}]
\end{mathpar}

\subsection{Additional notation}

It will sometimes be convenient to denote the process a name
quotes. We already have the notation $x = \quotep{P}$, but it will be
convenient to introduce an alternate notation, $\procn{x}$, when we
want to emphasize the connection to the use of the name. Note that, by
virtue of name equivalence, $\quotep{\procn{x}} \nameeq x$; so, the
notation is consistent with previous definitions.

Further, because names have structure it is possible to effect
substitutions on the basis of that structure. This means we need to
upgrade our notation for substitutions, which we accomplish by
adapting comprehension notation. Thus,

\begin{mathpar}
  P\{ y / x : x \in S \}
\end{mathpar}

is interpreted to mean the process derived from P by replacing (in a
capture-avoiding manner) each occurrence of $x$ in $S$ by $y$. For example,

\begin{mathpar}
  P\{ \quotep{\procn{x}|\procn{x}} / x : x \in \freenames{P} \}
\end{mathpar}

will replace each (occurrence) of a free name $x$ in $P$ by
$\quotep{\procn{x}|\procn{x}}$.

Also, we will avail ourselves of the notation $x^{L}$ and $x^{R}$ to
denote injections of a name into disjoint copies of the name
space. There are numerous ways to accomplish this. One example can be
found in \cite{MeredithR05}. This notation overloads to vectors of
names: $\vec{x}^{\pi} := (x_{i}^{\pi} \; : \; 0 \leq i < |\vec{x}| )$ where $\pi \in \{L,R\}$.

We also use $P^{\Box} := P|\Box$.

In \cite{MeredithR05} an interpretation of the new operator is
given. It turns out that there are several possible interpretations
all enjoying the requisite algebraic properties of the operator (see
\cite{milner91polyadicpi}). We will therefore make liberal use of
$(\nu\; \vec{x})P$.

% subsection the_syntax_and_semantics_of_the_notation_system (end)   

\input{qm2pi.qmops} 

\input{qm2pi.sterngerlach} 

\input{qm2pi.metric} 

% section concurrent_process_calculi (end)

%\input{qm2pi.proofsketch}

% section proof sketch (end)

%\input{qm2pi.slviaknots} 

% section spatial logic via knots (end)

\input{qm2pi.conclusion}

% section conclusion (end)

%\input{qm2pi.dtcodes} 

% section wiring algorithm (end)

\input{qm2pi.ack} 

% section acknowledgments (end)

\newpage


\bibliographystyle{plain}   
\bibliography{../../biblios/main.bib}

\input{qm2pi.rhodetails}

\end{document}

 

% section acknowledgments (end)

\newpage


\bibliographystyle{plain}   
\bibliography{../../biblios/main.bib}

\documentclass[12pt]{llncs}
%\documentclass{jktr}

\usepackage[pdftex]{hyperref}                   
\usepackage {listings}
\usepackage {mathpartir}
\usepackage{bcprules}
%\usepackage{listings}
                       
\usepackage{graphicx} 
%\usepackage[margins=2.5cm,nohead,nofoot]{geometry}
%\usepackage{geometry}
\usepackage{amsfonts}
\usepackage{amstext}
\usepackage{latexsym}
\usepackage{amssymb}
\usepackage{color}


%\include{myPreamble}
\include{qm2pi.local} 

%\ifpdf
%\usepackage[pdftex]{graphicx}
%\else
%\usepackage{graphicx}
%\fi

 % \ifpdf
%  \usepackage{pdfsync}
%  \if


%\title{Brief Article}
%\author{David F. Snyder}
%\author{L.G. Meredith}

%\address{Dept. of Math., Texas State University--San Marcos, San Marcos, TX 78666}
       
\pagestyle{empty}


\begin{document}

\lstset{language=[Objective]Caml,frame=shadowbox}

\input{qm2pi.front}

% section front matter (end)

\input{qm2pi.intro} 
 
% section introduction (end)

% \input{qm2pi.knotations} 

% section notation (end)

\input{qm2pi.process.calculi} 

% section concurrent_process_calculi_and_spatial_logics_ (end)
    
%\input{qm2pi.knots2pi} 

%\input{qm2pi.trefoil} 

%\input{qm2pi.mainthm} 

% subsection basic_interpretation (end)

%\input{qm2pi.rho.presentation} 
\subsection{The syntax and semantics of the notation system}\label{sub:the_syntax_and_semantics_of_the_notation_system} % (fold)

We now summarize a technical presentation of the calculus that
embodies our theory of dynamics. The typical presentation of such a
calculus follows the style of giving generators and relations on
them. The grammar, below, describing term constructors, freely
generates the set of processes, $\Proc$. This set is then quotiented
by a relation known as structural congruence and it is over this set
that the notion of dynamics is expressed. This presentation is
essentially that of \cite{MeredithR05} with the addition of
polyadicity and summation. For readability we have relegated some of
the technical subtleties to an appendix.

\subsubsection{Process grammar}\label{subsub:process_grammar}

\begin{mathpar}
  \inferrule* [lab=synchronization] {} {{M} \bc \pzero \;|\; x?F \;|\; x!C }
  \and
  \inferrule* [lab=abstraction] {} {{F} \bc (x)P}
  \and
  \inferrule* [lab=concretion] {} {{C} \bc \langle Q \rangle}
  \and
  \inferrule* [lab=process] {} {{P,Q} \bc M \;| \;P|Q \;|\; @{x}}
  \and
  \inferrule* [lab=name] {} {{x} \bc \quotep{P}}
\end{mathpar} 

Note that $\vec{x}$ (resp. $\vec{P}$) denotes a vector of names
(resp. processes) of length $|\vec{x}|$ (resp. $|\vec{P}|$). We adopt
the following useful abbreviations.

\begin{mathpar}
   x?(\vec{y}).P := x.(\vec{y})P \and  x\clift{\vec{P}} := x.\clift{\vec{P}}
   \and x!(y) := \lift{x}{\dropn{y}}
   \and \Pi_{i=0}^{n-1}P_i := P_0 | \ldots | P_{n-1}
\end{mathpar}

\subsubsection{Structural congruence}

\paragraph{Free and bound names and alpha-equivalence.} At the
core of structural equivalence is alpha-equivalence which identifies
process that are the same up to a change of variable. Formally, we
recognize the distinction between free and bound names. The free names
of a process, $\freenames{P}$, may be calculated recursively as
follows:

\begin{mathpar}
\freenames{\pzero} := \emptyset
  \and \\
  \freenames{x?(y).P} := \{ x \} \cup (\freenames{P} \setminus \{ y \})
  \and 
  \freenames{x!\langle P \rangle} := \{ x \} \cup \{ P \} 
  \and \\
  \freenames{P|Q} := \freenames{P} \cup \freenames{Q}
  \and \\
  \freenames{@{x}} := \{ x \}
\end{mathpar}

$\pi$
$\quotep{\pi}$

$\freenames{-} : \pi \to \mathcal{P}(\quotep{\pi})$

\begin{eqnarray*}
  \freenames{\pzero} & := & \emptyset \\
  \freenames{x?(y).P} & := & \{ x \} \cup (\freenames{P} \setminus \{ y \}) \\
  \freenames{x!\langle P \rangle} & := & \{ x \} \cup \{ P \} \\
  \freenames{P|Q} & := & \freenames{P} \cup \freenames{Q} \\
  \freenames{\dropn{x}} & := & \{ x \}
\end{eqnarray*}

The bound names of a process, $\boundnames{P}$, are those names occurring in $P$
that are not free. For example, in $x?(y).0$, the name $x$ is free, while $y$ is bound.

\begin{mathpar}
  \inferrule* [lab=monoidal-laws] {} { P|Q \equiv Q|P \and P|0 \equiv P \and P|(Q|R) \equiv (P|Q)|R }
\end{mathpar}

\begin{mathpar}
  \inferrule* [lab=alpha-equivalence] {} { (x)P \equiv (y)P\{y/x\} \and y \not\in \freenames{P} }
\end{mathpar}

\begin{definition}
Then two processes, $P,Q$, are alpha-equivalent if $P = Q\{\vec{y}/\vec{x}\}$ for
some $\vec{x} \in \boundnames{Q},\vec{y} \in \boundnames{P}$, where $Q\{\vec{y}/\vec{x}\}$
denotes the capture-avoiding substitution of $\vec{y}$ for $\vec{x}$ in $Q$.
\end{definition}

\begin{definition}
  The {\em structural congruence} \cite{SangiorgiWalker} , $\equiv$,
  between processes is the least congruence containing
  alpha-equivalence, satisfying the abelian monoid laws
  (associativity, commutativity and $\pzero$ as identity) for parallel
  composition $|$ and for summation $+$.
\end{definition}

\subsection{Name equivalence}

We take name equivalence, written $\nameeq$, to be the smallest
equivalence relation generated by the following rules.

\begin{mathpar}
\inferrule*[lab=Quote-drop]
{ }
{ \quotep{@{x}} \nameeq x }

\inferrule*[lab=Struct-equiv]
{ P \scong Q }
{ \quotep{P} \nameeq \quotep{Q} }
\end{mathpar}

The astute reader will have noticed that the mutual recursion of names
and processes imposes a mutual recursion on alpha-equivalence and
structural equivalence via name-equivalence. Fortunately, all of this
works out pleasantly and we may calculate in the natural way, free of
concern. The reader interested in the details is referred to the
appendix \ref{appendix:rho_details}.

\subsection{Substitution}

We use $\Proc$ for the set of processes, $\QProc$ for the set of
names, and $\id{\{}\vec{y} / \vec{x} \id{\}}$ to denote partial maps,
$s : \QProc \rightarrow \QProc$. A map, $s$ lifts, uniquely, to a map
on process terms, $\widehat{s} : \Proc \rightarrow \Proc$ by the
following equations.

\begin{mathpar}
  (0) \psubstp{Q}{P} := 0 \\
  (R \juxtap S) \psubstp{Q}{P}
  :=    
  (R)\psubstp{Q}{P} \juxtap (S) \psubstp{Q}{P} \\
  (x?(y).R) \psubstp{Q}{P}    
  :=    
  (x)\substp{Q}{P} (z)\concat( (R \psubstn{z}{y}) \psubstp{Q}{P} ) \\
  (\lift{x}{R}) \psubstp{Q}{P}  
  :=
  \lift{(x)\substp{Q}{P}}{ R \psubstp{Q}{P} } \\
%   (\dropn{x})  \psubstp{Q}{P}       
%   := 
%   \left\{ 
%     \begin{array}{ccc} 
%       \dropn{\quotep{Q}} & & x \nameeq \quotep{P} \\
%       \dropn{x} & & otherwise \\
%     \end{array}
%   \right. 
  (\dropn{x})  \psubstp{Q}{P}       
  := 
  \left\{ 
    \begin{array}{ccc} 
      Q & & x \nameeq \quotep{P} \\
      \dropn{x} & & otherwise \\
    \end{array}
  \right.
\end{mathpar}
 

where

\begin{eqnarray}
  (x)\id{\{} \lpquote Q \rpquote / \lpquote P \rpquote \id{\}}            = 
  \left\{ 
    \begin{array}{ccc}
      \lpquote Q \rpquote & & x \nameeq \lpquote P \rpquote \\
      x & & otherwise \\
    \end{array}
  \right. \nonumber
\end{eqnarray}

and $z$ is chosen distinct from $\quotep{P}$, $\quotep{Q}$, the free
names in $Q$, and all the names in $R$. Our $\alpha$-equivalence will
be built in the standard way from this substitution.

\begin{remark}\label{rem:no_self_referential_names}
  One consequence of these definitions is that $\forall P. \quotep{P}
  \not\in \freenames{P}$.
\end{remark}

\subsection{ Dynamic quote: an example }

Anticipating something of what's to come, consider applying the
substitution, $\widehat{\id{\{}u / z \id{\}}}$, to the following pair
of processes, $\lift{w}{y!(z)}$ and $w[ \lpquote y!(z) \rpquote ]$.

\begin{eqnarray}
	\lift{w}{y!(z)}\widehat{\id{\{}u / z \id{\}}}
		& = &
		\lift{w}{y!(u)} \nonumber\\
	w[ \lpquote y!(z) \rpquote ] \widehat{ \id{\{}u / z \id{\}} }
		& = &
		w[ \lpquote y!(z) \rpquote ] \nonumber
\end{eqnarray}

Because the body of the process between quotes is impervious to
substitution, we get radically different answers. In fact, by
examining the first process in an input context,
e.g. $x?(z).\lift{w}{y!(z)}$, we see that the process under the lift
operator may be shaped by prefixed inputs binding a name inside it. In
this sense, the lift operator will be seen as a way to dynamically
construct processes before reifying them as names.

Finally equipped with these standard features we can present the
dynamics of the calculus.

\subsubsection{Operational semantics} 

Finally, we introduce the computational dynamics. What marks these
algebras as distinct from other more traditionally studied algebraic
structures, e.g. vector spaces or polynomial rings, is the manner in
which dynamics is captured. In traditional structures, dynamics is typically
expressed through morphisms between such structures, as in linear maps
between vector spaces or morphisms between rings. In algebras
associated with the semantics of computation, the dynamics is
expressed as part of the algebraic structure itself, through a
reduction reduction relation typically denoted by $\red$. Below, we
give a recursive presentation of this relation for the calculus used
in the encoding.

$\red \subseteq \pi \times \pi$
$\red : \pi \to \mathcal{P}(\pi)$

\begin{mathpar}
  \inferrule* [lab=Comm] { \textsf{match}( x_{src}, x_{trgt} ) } { x_{trgt}?(y)P \; | \; x_{src}!\langle {Q} \rangle \red P\{\quotep{Q}/y}\} }
  \and \\
  \inferrule* [lab=Par] {{P} \red {P}'} {{{P} | {Q}} \red {{P}' | {Q}}}
  \and
  \inferrule* [lab=Equiv]{{{P} \scong {P}'} \andalso {{P}' \red {Q}'} \andalso {{Q}' \scong {Q}}}{{P} \red {Q}}
\end{mathpar}

\begin{eqnarray*}
  match_{\equiv} (\quotep{P},\quotep{Q}) & := & P \equiv Q \\
  match_{\dagger}(\quotep{P},\quotep{Q}) & := & \forall R. P|Q \red^{*} R => R \red^{*} 0 \\
  match_{K}(\quotep{P},\quotep{Q}) & := & K \mbox{ for some context } K
\end{eqnarray*}

$u?(x)P | u!\langle Q \rangle \red P\{\quotep{Q}/x\}$

%We write $\wred$ for $\red^*$, and $P\red$ if $\exists Q $ such that $ P \red Q$.
We write $P\red$ if $\exists Q $ such that $ P \red Q$ and $P\not\red$, otherwise.

\section{Replication}

As mentioned before, it is known that replication (and hence
recursion) can be implemented in a higher-order process algebra
\cite{SangiorgiWalker}. As our first example of calculation with the
machinery thus far presented we give the construction explicitly in
the {\rhoc}.

\begin{eqnarray}
	D_{x} & := & \prefix{x}{y}{(\binpar{\outputp{x}{y}}{@{y}})} \nonumber\\
	\bangp_{x}{P} & := & \binpar{{x}!\langle{\binpar{D_{x}}{P}}\rangle}{D_{x}} \nonumber
\end{eqnarray}

\begin{eqnarray}
	\bangp_{x}{P} & & \nonumber\\
	=
	& {x}!\langle{(\prefix{x}{y}{(\outputp{x}{y} | @{y})) | P}}\rangle 
	      | \prefix{x}{y}{(\outputp{x}{y} | @{y})} & \nonumber\\
	\red
	& (\outputp{x}{y} | @{y})\substn{\quotep{(\prefix{x}{y}{(@{y} | \outputp{x}{y})) | P}}}{y} & \nonumber\\
	=
	& \outputp{x}{\quotep{(\prefix{x}{y}{(\outputp{x}{y} | @{y})) | P}}}
	  | {(\prefix{x}{y}{(\outputp{x}{y} | @{y})) | P}} & \nonumber\\
	\red
	& \ldots & \nonumber\\
	\red^*
	& P | P | \ldots & \nonumber
\end{eqnarray}

Of course, this encoding, as an implementation, runs away, unfolding
$\bangp{P}$ eagerly. A lazier and more implementable replication
operator, restricted to input-guarded processes, may be obtained as follows.

\begin{eqnarray}
\bangp{\prefix{u}{v}{P}} 
	:= 
	\binpar{\lift{x}{\prefix{u}{v}{(\binpar{D(x)}{P})}}}{D(x)} \nonumber
\end{eqnarray}

\begin{remark}
  Note that the lazier definition still does not deal with summation
  or mixed summation (i.e. sums over input and output). The reader is
  invited to construct definitions of replication that deal with these
  features. 

  Further, the definitions are parameterized in a name, $x$. Can you,
  gentle reader, make a definition that eliminates this parameter and
  guarantees no accidental interaction between the replication
  machinery and the process being replicated -- i.e. no accidental
  sharing of names used by the process to get its work done and the
  name(s) used by the replication to effect copying. This latter
  revision of the definition of replication is crucial to obtaining
  the expected identity $!!P \sim !P$.
\end{remark}

\begin{remark}\label{rem:paradoxical_combinator}
  The reader familiar with the lambda calculus will have noticed the
  similarity between $D$ and the paradoxical combinator.

  [Ed. note: the existence of this seems to suggest we have to be more
  restrictive on the set of processes and names we admit if we are to
  support no-cloning.]
\end{remark}

\subsubsection{Bisimulation}

The computational dynamics gives rise to another kind of equivalence,
the equivalence of computational behavior. As previously mentioned
this is typically captured \emph{via} some form of bisimulation.

% The notion we use in this paper is weak barbed bisimulation
% \cite{milner91polyadicpi}.

The notion we use in this paper is derived from weak barbed
bisimulation \cite{milner91polyadicpi}. 

\begin{definition}
An \emph{observation relation}, $\downarrow_{\mathcal N}$, over a set
of names, $\mathcal N$, is the smallest relation satisfying the rules
below.

\infrule[Out-barb]{y \in {\mathcal N}, \; x \nameeq y}
		  {\outputp{x}{v} \downarrow_{\mathcal N} x}
\infrule[Par-barb]{\mbox{$P\downarrow_{\mathcal N} x$ or $Q\downarrow_{\mathcal N} x$}}
		  {\binpar{P}{Q} \downarrow_{\mathcal N} x}

We write $P \Downarrow_{\mathcal N} x$ if there is $Q$ such that 
$P \wred Q$ and $Q \downarrow_{\mathcal N} x$.
\end{definition}

\begin{definition}
%\label{def.bbisim}
An  ${\mathcal N}$-\emph{barbed bisimulation} over a set of names, ${\mathcal N}$, is a symmetric binary relation 
${\mathcal S}_{\mathcal N}$ between agents such that $P\rel{S}_{\mathcal N}Q$ implies:
\begin{enumerate}
\item If $P \red P'$ then $Q \wred Q'$ and $P'\rel{S}_{\mathcal N} Q'$.
\item If $P\downarrow_{\mathcal N} x$, then $Q\Downarrow_{\mathcal N} x$.
\end{enumerate}
$P$ is ${\mathcal N}$-barbed bisimilar to $Q$, written
$P \wbbisim_{\mathcal N} Q$, if $P \rel{S}_{\mathcal N} Q$ for some ${\mathcal N}$-barbed bisimulation ${\mathcal S}_{\mathcal N}$.
\end{definition}

$\mathcal{R} \subseteq \pi \times \pi$

$P \mathcal{R} Q => \forall P'. P \red P' \Rightarrow \exists Q'. Q \red Q', P' \mathcal{R} Q'$

$P \vdash x \Rightarrow Q \vdash x$

\begin{mathpar}
  \inferrule*[lab=Out-barb]{x \nameeq y}{{y}!\langle{Q}\rangle \vdash x}
  \and
  \inferrule*[lab=Par-barb]{\mbox{$P\vdash x$ or $Q\vdash x$}}{\binpar{P}{Q} \vdash x}
\end{mathpar}

\subsubsection{Contexts}

One of the principle advantages of computational calculi like the
$\pi$-calculus is a well-defined notion of context,
contextual-equivalence and a correlation between
contextual-equivalence and notions of bisimulation. The notion of
context allows the decomposition of a process into (sub-)process and
its syntactic environment, its context. Thus, a context may be
thought of as a process with a ``hole'' (written $\Box$) in it. The
application of a context $M$ to a process $P$, written $M[P]$, is
tantamount to filling the hole in $M$ with $P$. In this paper we do
not need the full weight of this theory, but do make use of the notion
of context in the proof the main theorem. 

\begin{mathpar}
  \inferrule* [lab=summation] {} {{M_{M},M_{N}} \bc \Box \;|\; x.M_{A} \;|\; M_{M}+M_{N}}
  \and
  \inferrule* [lab=agent] {} {{M_{A}} \bc (\vec{x})M_{P} \;| \; \clift{P_0,\ldots,M_{P},\ldots,P_N}}
  \and \\
  \inferrule* [lab=process] {} {{M_{P}} \bc M_{N} \;| \;P|M_{P} }
\end{mathpar} 

\begin{mathpar}
  \inferrule* [lab=sychronization] {} {M_{N} \bc \Box \;|\; x?M_{F} \;|\; x!M_{C}}
  \and
  \inferrule* [lab=abstraction] {} {{M_{F}} \bc (x)M_{P} }
  \and
  \inferrule* [lab=concretion] {} {{M_{C}} \bc \langle M_{P} \rangle }
  \and \\
  \inferrule* [lab=process] {} {{M_{P}} \bc M_{N} \;| \;P|M_{P} }
\end{mathpar}

\begin{definition}[contextual application] Given a context $M$, and
  process $P$, we define the \emph{contextual application}, $M[P] :=
  M\{P/\Box\}$. That is, the contextual application of M to P is the
  substitution of $P$ for $\Box$ in $M$.
\end{definition}

$\meaningof{-} : L \to \mathcal{P}(\pi)$

\begin{mathpar}
  \inferrule* [lab=collection] {} {\meaningof{true} = \pi, \and \meaningof{~E} = \pi \setminus \meaningof{E}, \and \meaningof{E_{1} \& E_{2}} = \meaningof{E_{1}} \cap \meaningof{E_{2}}}
\end{mathpar}

\begin{mathpar}
  \inferrule* [lab=structure] {} {\meaningof{0} = \{ P \in \pi | P \equiv 0 \}, \and \\ \meaningof{E_1 | E_2} = \{ P \in \pi | P \equiv P_{1} | P_{2}, P_{1} \in \meaningof{E_{1}}, P_{2} \in \meaningof{E_2}\} }
\end{mathpar}

\begin{mathpar}
 \inferrule* [lab=behavior] {} {\meaningof{\langle a?b \rangle E} = \{ P \in \pi | P \equiv Q | u?(y)P', \\ \and \\\\ \and \\ \;\;\; u \in \meaningof{a}, \forall z.P'\{z/y\} \in \meaningof{E\{z/b\}}\}, \and \\ \meaningof{a!E} = \{ P \in \pi | P \equiv Q | x!\langle P' \rangle, x \in \meaningof{a} P' \in \meaningof{E}\} }
\end{mathpar}

\begin{mathpar}
 \inferrule* [lab=nominal] {} {\meaningof{\quotep{E}} = \{ \quotep{P} \in \quotep{\pi} | P \in \meaningof{E} \}, \and \meaningof{\quotep{P}} = \{ \quotep{Q} \in \quotep{\pi} | P \equiv Q \} \and \\ \meaningof{@\quotep{E}} = \{ P \in \pi | P \equiv @x, x \in \meaningof{E} \}}
\end{mathpar}

\begin{eqnarray*}
  \\
  \meaningof{-} : TS \to ST
\end{eqnarray*}

\begin{eqnarray*}
  \\
  L : TS \to ST
\end{eqnarray*}

\begin{eqnarray*}
  \\
  P \models E \iff P \in \meaningof{E}
\end{eqnarray*}

\begin{eqnarray*}
  P \approx_{L} Q \iff \forall E \in L. P \models E \iff Q \models E
\end{eqnarray*}

\begin{eqnarray*}
  P \approx_{K} Q
\end{eqnarray*}

\begin{eqnarray*}
  P \approx Q
\end{eqnarray*}

$\approx_{K} = \approx = \approx_{L}$

\subsubsection{Contextual duality}

Note that contexts extend the quotation operation to a family of
operations from processes to names. Given a context, $M$, we can
define a \emph{nominal context}, $\quotep{M}$ by $\quotep{M}[P] :=
\quotep{M[P]}$. To foreshadow what is to come we observe that these
operations enjoy a duality with processes very much like the duality
between vectors and maps from vectors to scalars.

Further, because the calculus is essentially higher-order, we have a
correspondence between contexts and processes. More specifically,
given a name $x$ and a context $M$ we can construct $M^{*}_{x}$ such
that 

\begin{mathpar}
  M^{*}_{x} | \lift{x}{P} \red M[P]
\end{mathpar}

namely,

\begin{mathpar}
  M^{*}_{x} := x?(u).M[\dropn{u}]
\end{mathpar}

The dependence of $M^{*}_{x}$ on a name makes it an abstraction, 

\begin{mathpar}
  M^{*} := (x)x?(u).M[\dropn{u}]
\end{mathpar}

\subsection{Additional notation}

It will sometimes be convenient to denote the process a name
quotes. We already have the notation $x = \quotep{P}$, but it will be
convenient to introduce an alternate notation, $\procn{x}$, when we
want to emphasize the connection to the use of the name. Note that, by
virtue of name equivalence, $\quotep{\procn{x}} \nameeq x$; so, the
notation is consistent with previous definitions.

Further, because names have structure it is possible to effect
substitutions on the basis of that structure. This means we need to
upgrade our notation for substitutions, which we accomplish by
adapting comprehension notation. Thus,

\begin{mathpar}
  P\{ y / x : x \in S \}
\end{mathpar}

is interpreted to mean the process derived from P by replacing (in a
capture-avoiding manner) each occurrence of $x$ in $S$ by $y$. For example,

\begin{mathpar}
  P\{ \quotep{\procn{x}|\procn{x}} / x : x \in \freenames{P} \}
\end{mathpar}

will replace each (occurrence) of a free name $x$ in $P$ by
$\quotep{\procn{x}|\procn{x}}$.

Also, we will avail ourselves of the notation $x^{L}$ and $x^{R}$ to
denote injections of a name into disjoint copies of the name
space. There are numerous ways to accomplish this. One example can be
found in \cite{MeredithR05}. This notation overloads to vectors of
names: $\vec{x}^{\pi} := (x_{i}^{\pi} \; : \; 0 \leq i < |\vec{x}| )$ where $\pi \in \{L,R\}$.

We also use $P^{\Box} := P|\Box$.

In \cite{MeredithR05} an interpretation of the new operator is
given. It turns out that there are several possible interpretations
all enjoying the requisite algebraic properties of the operator (see
\cite{milner91polyadicpi}). We will therefore make liberal use of
$(\nu\; \vec{x})P$.

% subsection the_syntax_and_semantics_of_the_notation_system (end)   

\input{qm2pi.qmops} 

\input{qm2pi.sterngerlach} 

\input{qm2pi.metric} 

% section concurrent_process_calculi (end)

%\input{qm2pi.proofsketch}

% section proof sketch (end)

%\input{qm2pi.slviaknots} 

% section spatial logic via knots (end)

\input{qm2pi.conclusion}

% section conclusion (end)

%\input{qm2pi.dtcodes} 

% section wiring algorithm (end)

\input{qm2pi.ack} 

% section acknowledgments (end)

\newpage


\bibliographystyle{plain}   
\bibliography{../../biblios/main.bib}

\input{qm2pi.rhodetails}

\end{document}



\end{document}

 

% section concurrent_process_calculi (end)

%\documentclass[12pt]{llncs}
%\documentclass{jktr}

\usepackage[pdftex]{hyperref}                   
\usepackage {listings}
\usepackage {mathpartir}
\usepackage{bcprules}
%\usepackage{listings}
                       
\usepackage{graphicx} 
%\usepackage[margins=2.5cm,nohead,nofoot]{geometry}
%\usepackage{geometry}
\usepackage{amsfonts}
\usepackage{amstext}
\usepackage{latexsym}
\usepackage{amssymb}
\usepackage{color}


%\include{myPreamble}
\documentclass[12pt]{llncs}
%\documentclass{jktr}

\usepackage[pdftex]{hyperref}                   
\usepackage {listings}
\usepackage {mathpartir}
\usepackage{bcprules}
%\usepackage{listings}
                       
\usepackage{graphicx} 
%\usepackage[margins=2.5cm,nohead,nofoot]{geometry}
%\usepackage{geometry}
\usepackage{amsfonts}
\usepackage{amstext}
\usepackage{latexsym}
\usepackage{amssymb}
\usepackage{color}


%\include{myPreamble}
\include{qm2pi.local} 

%\ifpdf
%\usepackage[pdftex]{graphicx}
%\else
%\usepackage{graphicx}
%\fi

 % \ifpdf
%  \usepackage{pdfsync}
%  \if


%\title{Brief Article}
%\author{David F. Snyder}
%\author{L.G. Meredith}

%\address{Dept. of Math., Texas State University--San Marcos, San Marcos, TX 78666}
       
\pagestyle{empty}


\begin{document}

\lstset{language=[Objective]Caml,frame=shadowbox}

\input{qm2pi.front}

% section front matter (end)

\input{qm2pi.intro} 
 
% section introduction (end)

% \input{qm2pi.knotations} 

% section notation (end)

\input{qm2pi.process.calculi} 

% section concurrent_process_calculi_and_spatial_logics_ (end)
    
%\input{qm2pi.knots2pi} 

%\input{qm2pi.trefoil} 

%\input{qm2pi.mainthm} 

% subsection basic_interpretation (end)

%\input{qm2pi.rho.presentation} 
\subsection{The syntax and semantics of the notation system}\label{sub:the_syntax_and_semantics_of_the_notation_system} % (fold)

We now summarize a technical presentation of the calculus that
embodies our theory of dynamics. The typical presentation of such a
calculus follows the style of giving generators and relations on
them. The grammar, below, describing term constructors, freely
generates the set of processes, $\Proc$. This set is then quotiented
by a relation known as structural congruence and it is over this set
that the notion of dynamics is expressed. This presentation is
essentially that of \cite{MeredithR05} with the addition of
polyadicity and summation. For readability we have relegated some of
the technical subtleties to an appendix.

\subsubsection{Process grammar}\label{subsub:process_grammar}

\begin{mathpar}
  \inferrule* [lab=synchronization] {} {{M} \bc \pzero \;|\; x?F \;|\; x!C }
  \and
  \inferrule* [lab=abstraction] {} {{F} \bc (x)P}
  \and
  \inferrule* [lab=concretion] {} {{C} \bc \langle Q \rangle}
  \and
  \inferrule* [lab=process] {} {{P,Q} \bc M \;| \;P|Q \;|\; @{x}}
  \and
  \inferrule* [lab=name] {} {{x} \bc \quotep{P}}
\end{mathpar} 

Note that $\vec{x}$ (resp. $\vec{P}$) denotes a vector of names
(resp. processes) of length $|\vec{x}|$ (resp. $|\vec{P}|$). We adopt
the following useful abbreviations.

\begin{mathpar}
   x?(\vec{y}).P := x.(\vec{y})P \and  x\clift{\vec{P}} := x.\clift{\vec{P}}
   \and x!(y) := \lift{x}{\dropn{y}}
   \and \Pi_{i=0}^{n-1}P_i := P_0 | \ldots | P_{n-1}
\end{mathpar}

\subsubsection{Structural congruence}

\paragraph{Free and bound names and alpha-equivalence.} At the
core of structural equivalence is alpha-equivalence which identifies
process that are the same up to a change of variable. Formally, we
recognize the distinction between free and bound names. The free names
of a process, $\freenames{P}$, may be calculated recursively as
follows:

\begin{mathpar}
\freenames{\pzero} := \emptyset
  \and \\
  \freenames{x?(y).P} := \{ x \} \cup (\freenames{P} \setminus \{ y \})
  \and 
  \freenames{x!\langle P \rangle} := \{ x \} \cup \{ P \} 
  \and \\
  \freenames{P|Q} := \freenames{P} \cup \freenames{Q}
  \and \\
  \freenames{@{x}} := \{ x \}
\end{mathpar}

$\pi$
$\quotep{\pi}$

$\freenames{-} : \pi \to \mathcal{P}(\quotep{\pi})$

\begin{eqnarray*}
  \freenames{\pzero} & := & \emptyset \\
  \freenames{x?(y).P} & := & \{ x \} \cup (\freenames{P} \setminus \{ y \}) \\
  \freenames{x!\langle P \rangle} & := & \{ x \} \cup \{ P \} \\
  \freenames{P|Q} & := & \freenames{P} \cup \freenames{Q} \\
  \freenames{\dropn{x}} & := & \{ x \}
\end{eqnarray*}

The bound names of a process, $\boundnames{P}$, are those names occurring in $P$
that are not free. For example, in $x?(y).0$, the name $x$ is free, while $y$ is bound.

\begin{mathpar}
  \inferrule* [lab=monoidal-laws] {} { P|Q \equiv Q|P \and P|0 \equiv P \and P|(Q|R) \equiv (P|Q)|R }
\end{mathpar}

\begin{mathpar}
  \inferrule* [lab=alpha-equivalence] {} { (x)P \equiv (y)P\{y/x\} \and y \not\in \freenames{P} }
\end{mathpar}

\begin{definition}
Then two processes, $P,Q$, are alpha-equivalent if $P = Q\{\vec{y}/\vec{x}\}$ for
some $\vec{x} \in \boundnames{Q},\vec{y} \in \boundnames{P}$, where $Q\{\vec{y}/\vec{x}\}$
denotes the capture-avoiding substitution of $\vec{y}$ for $\vec{x}$ in $Q$.
\end{definition}

\begin{definition}
  The {\em structural congruence} \cite{SangiorgiWalker} , $\equiv$,
  between processes is the least congruence containing
  alpha-equivalence, satisfying the abelian monoid laws
  (associativity, commutativity and $\pzero$ as identity) for parallel
  composition $|$ and for summation $+$.
\end{definition}

\subsection{Name equivalence}

We take name equivalence, written $\nameeq$, to be the smallest
equivalence relation generated by the following rules.

\begin{mathpar}
\inferrule*[lab=Quote-drop]
{ }
{ \quotep{@{x}} \nameeq x }

\inferrule*[lab=Struct-equiv]
{ P \scong Q }
{ \quotep{P} \nameeq \quotep{Q} }
\end{mathpar}

The astute reader will have noticed that the mutual recursion of names
and processes imposes a mutual recursion on alpha-equivalence and
structural equivalence via name-equivalence. Fortunately, all of this
works out pleasantly and we may calculate in the natural way, free of
concern. The reader interested in the details is referred to the
appendix \ref{appendix:rho_details}.

\subsection{Substitution}

We use $\Proc$ for the set of processes, $\QProc$ for the set of
names, and $\id{\{}\vec{y} / \vec{x} \id{\}}$ to denote partial maps,
$s : \QProc \rightarrow \QProc$. A map, $s$ lifts, uniquely, to a map
on process terms, $\widehat{s} : \Proc \rightarrow \Proc$ by the
following equations.

\begin{mathpar}
  (0) \psubstp{Q}{P} := 0 \\
  (R \juxtap S) \psubstp{Q}{P}
  :=    
  (R)\psubstp{Q}{P} \juxtap (S) \psubstp{Q}{P} \\
  (x?(y).R) \psubstp{Q}{P}    
  :=    
  (x)\substp{Q}{P} (z)\concat( (R \psubstn{z}{y}) \psubstp{Q}{P} ) \\
  (\lift{x}{R}) \psubstp{Q}{P}  
  :=
  \lift{(x)\substp{Q}{P}}{ R \psubstp{Q}{P} } \\
%   (\dropn{x})  \psubstp{Q}{P}       
%   := 
%   \left\{ 
%     \begin{array}{ccc} 
%       \dropn{\quotep{Q}} & & x \nameeq \quotep{P} \\
%       \dropn{x} & & otherwise \\
%     \end{array}
%   \right. 
  (\dropn{x})  \psubstp{Q}{P}       
  := 
  \left\{ 
    \begin{array}{ccc} 
      Q & & x \nameeq \quotep{P} \\
      \dropn{x} & & otherwise \\
    \end{array}
  \right.
\end{mathpar}
 

where

\begin{eqnarray}
  (x)\id{\{} \lpquote Q \rpquote / \lpquote P \rpquote \id{\}}            = 
  \left\{ 
    \begin{array}{ccc}
      \lpquote Q \rpquote & & x \nameeq \lpquote P \rpquote \\
      x & & otherwise \\
    \end{array}
  \right. \nonumber
\end{eqnarray}

and $z$ is chosen distinct from $\quotep{P}$, $\quotep{Q}$, the free
names in $Q$, and all the names in $R$. Our $\alpha$-equivalence will
be built in the standard way from this substitution.

\begin{remark}\label{rem:no_self_referential_names}
  One consequence of these definitions is that $\forall P. \quotep{P}
  \not\in \freenames{P}$.
\end{remark}

\subsection{ Dynamic quote: an example }

Anticipating something of what's to come, consider applying the
substitution, $\widehat{\id{\{}u / z \id{\}}}$, to the following pair
of processes, $\lift{w}{y!(z)}$ and $w[ \lpquote y!(z) \rpquote ]$.

\begin{eqnarray}
	\lift{w}{y!(z)}\widehat{\id{\{}u / z \id{\}}}
		& = &
		\lift{w}{y!(u)} \nonumber\\
	w[ \lpquote y!(z) \rpquote ] \widehat{ \id{\{}u / z \id{\}} }
		& = &
		w[ \lpquote y!(z) \rpquote ] \nonumber
\end{eqnarray}

Because the body of the process between quotes is impervious to
substitution, we get radically different answers. In fact, by
examining the first process in an input context,
e.g. $x?(z).\lift{w}{y!(z)}$, we see that the process under the lift
operator may be shaped by prefixed inputs binding a name inside it. In
this sense, the lift operator will be seen as a way to dynamically
construct processes before reifying them as names.

Finally equipped with these standard features we can present the
dynamics of the calculus.

\subsubsection{Operational semantics} 

Finally, we introduce the computational dynamics. What marks these
algebras as distinct from other more traditionally studied algebraic
structures, e.g. vector spaces or polynomial rings, is the manner in
which dynamics is captured. In traditional structures, dynamics is typically
expressed through morphisms between such structures, as in linear maps
between vector spaces or morphisms between rings. In algebras
associated with the semantics of computation, the dynamics is
expressed as part of the algebraic structure itself, through a
reduction reduction relation typically denoted by $\red$. Below, we
give a recursive presentation of this relation for the calculus used
in the encoding.

$\red \subseteq \pi \times \pi$
$\red : \pi \to \mathcal{P}(\pi)$

\begin{mathpar}
  \inferrule* [lab=Comm] { \textsf{match}( x_{src}, x_{trgt} ) } { x_{trgt}?(y)P \; | \; x_{src}!\langle {Q} \rangle \red P\{\quotep{Q}/y}\} }
  \and \\
  \inferrule* [lab=Par] {{P} \red {P}'} {{{P} | {Q}} \red {{P}' | {Q}}}
  \and
  \inferrule* [lab=Equiv]{{{P} \scong {P}'} \andalso {{P}' \red {Q}'} \andalso {{Q}' \scong {Q}}}{{P} \red {Q}}
\end{mathpar}

\begin{eqnarray*}
  match_{\equiv} (\quotep{P},\quotep{Q}) & := & P \equiv Q \\
  match_{\dagger}(\quotep{P},\quotep{Q}) & := & \forall R. P|Q \red^{*} R => R \red^{*} 0 \\
  match_{K}(\quotep{P},\quotep{Q}) & := & K \mbox{ for some context } K
\end{eqnarray*}

$u?(x)P | u!\langle Q \rangle \red P\{\quotep{Q}/x\}$

%We write $\wred$ for $\red^*$, and $P\red$ if $\exists Q $ such that $ P \red Q$.
We write $P\red$ if $\exists Q $ such that $ P \red Q$ and $P\not\red$, otherwise.

\section{Replication}

As mentioned before, it is known that replication (and hence
recursion) can be implemented in a higher-order process algebra
\cite{SangiorgiWalker}. As our first example of calculation with the
machinery thus far presented we give the construction explicitly in
the {\rhoc}.

\begin{eqnarray}
	D_{x} & := & \prefix{x}{y}{(\binpar{\outputp{x}{y}}{@{y}})} \nonumber\\
	\bangp_{x}{P} & := & \binpar{{x}!\langle{\binpar{D_{x}}{P}}\rangle}{D_{x}} \nonumber
\end{eqnarray}

\begin{eqnarray}
	\bangp_{x}{P} & & \nonumber\\
	=
	& {x}!\langle{(\prefix{x}{y}{(\outputp{x}{y} | @{y})) | P}}\rangle 
	      | \prefix{x}{y}{(\outputp{x}{y} | @{y})} & \nonumber\\
	\red
	& (\outputp{x}{y} | @{y})\substn{\quotep{(\prefix{x}{y}{(@{y} | \outputp{x}{y})) | P}}}{y} & \nonumber\\
	=
	& \outputp{x}{\quotep{(\prefix{x}{y}{(\outputp{x}{y} | @{y})) | P}}}
	  | {(\prefix{x}{y}{(\outputp{x}{y} | @{y})) | P}} & \nonumber\\
	\red
	& \ldots & \nonumber\\
	\red^*
	& P | P | \ldots & \nonumber
\end{eqnarray}

Of course, this encoding, as an implementation, runs away, unfolding
$\bangp{P}$ eagerly. A lazier and more implementable replication
operator, restricted to input-guarded processes, may be obtained as follows.

\begin{eqnarray}
\bangp{\prefix{u}{v}{P}} 
	:= 
	\binpar{\lift{x}{\prefix{u}{v}{(\binpar{D(x)}{P})}}}{D(x)} \nonumber
\end{eqnarray}

\begin{remark}
  Note that the lazier definition still does not deal with summation
  or mixed summation (i.e. sums over input and output). The reader is
  invited to construct definitions of replication that deal with these
  features. 

  Further, the definitions are parameterized in a name, $x$. Can you,
  gentle reader, make a definition that eliminates this parameter and
  guarantees no accidental interaction between the replication
  machinery and the process being replicated -- i.e. no accidental
  sharing of names used by the process to get its work done and the
  name(s) used by the replication to effect copying. This latter
  revision of the definition of replication is crucial to obtaining
  the expected identity $!!P \sim !P$.
\end{remark}

\begin{remark}\label{rem:paradoxical_combinator}
  The reader familiar with the lambda calculus will have noticed the
  similarity between $D$ and the paradoxical combinator.

  [Ed. note: the existence of this seems to suggest we have to be more
  restrictive on the set of processes and names we admit if we are to
  support no-cloning.]
\end{remark}

\subsubsection{Bisimulation}

The computational dynamics gives rise to another kind of equivalence,
the equivalence of computational behavior. As previously mentioned
this is typically captured \emph{via} some form of bisimulation.

% The notion we use in this paper is weak barbed bisimulation
% \cite{milner91polyadicpi}.

The notion we use in this paper is derived from weak barbed
bisimulation \cite{milner91polyadicpi}. 

\begin{definition}
An \emph{observation relation}, $\downarrow_{\mathcal N}$, over a set
of names, $\mathcal N$, is the smallest relation satisfying the rules
below.

\infrule[Out-barb]{y \in {\mathcal N}, \; x \nameeq y}
		  {\outputp{x}{v} \downarrow_{\mathcal N} x}
\infrule[Par-barb]{\mbox{$P\downarrow_{\mathcal N} x$ or $Q\downarrow_{\mathcal N} x$}}
		  {\binpar{P}{Q} \downarrow_{\mathcal N} x}

We write $P \Downarrow_{\mathcal N} x$ if there is $Q$ such that 
$P \wred Q$ and $Q \downarrow_{\mathcal N} x$.
\end{definition}

\begin{definition}
%\label{def.bbisim}
An  ${\mathcal N}$-\emph{barbed bisimulation} over a set of names, ${\mathcal N}$, is a symmetric binary relation 
${\mathcal S}_{\mathcal N}$ between agents such that $P\rel{S}_{\mathcal N}Q$ implies:
\begin{enumerate}
\item If $P \red P'$ then $Q \wred Q'$ and $P'\rel{S}_{\mathcal N} Q'$.
\item If $P\downarrow_{\mathcal N} x$, then $Q\Downarrow_{\mathcal N} x$.
\end{enumerate}
$P$ is ${\mathcal N}$-barbed bisimilar to $Q$, written
$P \wbbisim_{\mathcal N} Q$, if $P \rel{S}_{\mathcal N} Q$ for some ${\mathcal N}$-barbed bisimulation ${\mathcal S}_{\mathcal N}$.
\end{definition}

$\mathcal{R} \subseteq \pi \times \pi$

$P \mathcal{R} Q => \forall P'. P \red P' \Rightarrow \exists Q'. Q \red Q', P' \mathcal{R} Q'$

$P \vdash x \Rightarrow Q \vdash x$

\begin{mathpar}
  \inferrule*[lab=Out-barb]{x \nameeq y}{{y}!\langle{Q}\rangle \vdash x}
  \and
  \inferrule*[lab=Par-barb]{\mbox{$P\vdash x$ or $Q\vdash x$}}{\binpar{P}{Q} \vdash x}
\end{mathpar}

\subsubsection{Contexts}

One of the principle advantages of computational calculi like the
$\pi$-calculus is a well-defined notion of context,
contextual-equivalence and a correlation between
contextual-equivalence and notions of bisimulation. The notion of
context allows the decomposition of a process into (sub-)process and
its syntactic environment, its context. Thus, a context may be
thought of as a process with a ``hole'' (written $\Box$) in it. The
application of a context $M$ to a process $P$, written $M[P]$, is
tantamount to filling the hole in $M$ with $P$. In this paper we do
not need the full weight of this theory, but do make use of the notion
of context in the proof the main theorem. 

\begin{mathpar}
  \inferrule* [lab=summation] {} {{M_{M},M_{N}} \bc \Box \;|\; x.M_{A} \;|\; M_{M}+M_{N}}
  \and
  \inferrule* [lab=agent] {} {{M_{A}} \bc (\vec{x})M_{P} \;| \; \clift{P_0,\ldots,M_{P},\ldots,P_N}}
  \and \\
  \inferrule* [lab=process] {} {{M_{P}} \bc M_{N} \;| \;P|M_{P} }
\end{mathpar} 

\begin{mathpar}
  \inferrule* [lab=sychronization] {} {M_{N} \bc \Box \;|\; x?M_{F} \;|\; x!M_{C}}
  \and
  \inferrule* [lab=abstraction] {} {{M_{F}} \bc (x)M_{P} }
  \and
  \inferrule* [lab=concretion] {} {{M_{C}} \bc \langle M_{P} \rangle }
  \and \\
  \inferrule* [lab=process] {} {{M_{P}} \bc M_{N} \;| \;P|M_{P} }
\end{mathpar}

\begin{definition}[contextual application] Given a context $M$, and
  process $P$, we define the \emph{contextual application}, $M[P] :=
  M\{P/\Box\}$. That is, the contextual application of M to P is the
  substitution of $P$ for $\Box$ in $M$.
\end{definition}

$\meaningof{-} : L \to \mathcal{P}(\pi)$

\begin{mathpar}
  \inferrule* [lab=collection] {} {\meaningof{true} = \pi, \and \meaningof{~E} = \pi \setminus \meaningof{E}, \and \meaningof{E_{1} \& E_{2}} = \meaningof{E_{1}} \cap \meaningof{E_{2}}}
\end{mathpar}

\begin{mathpar}
  \inferrule* [lab=structure] {} {\meaningof{0} = \{ P \in \pi | P \equiv 0 \}, \and \\ \meaningof{E_1 | E_2} = \{ P \in \pi | P \equiv P_{1} | P_{2}, P_{1} \in \meaningof{E_{1}}, P_{2} \in \meaningof{E_2}\} }
\end{mathpar}

\begin{mathpar}
 \inferrule* [lab=behavior] {} {\meaningof{\langle a?b \rangle E} = \{ P \in \pi | P \equiv Q | u?(y)P', \\ \and \\\\ \and \\ \;\;\; u \in \meaningof{a}, \forall z.P'\{z/y\} \in \meaningof{E\{z/b\}}\}, \and \\ \meaningof{a!E} = \{ P \in \pi | P \equiv Q | x!\langle P' \rangle, x \in \meaningof{a} P' \in \meaningof{E}\} }
\end{mathpar}

\begin{mathpar}
 \inferrule* [lab=nominal] {} {\meaningof{\quotep{E}} = \{ \quotep{P} \in \quotep{\pi} | P \in \meaningof{E} \}, \and \meaningof{\quotep{P}} = \{ \quotep{Q} \in \quotep{\pi} | P \equiv Q \} \and \\ \meaningof{@\quotep{E}} = \{ P \in \pi | P \equiv @x, x \in \meaningof{E} \}}
\end{mathpar}

\begin{eqnarray*}
  \\
  \meaningof{-} : TS \to ST
\end{eqnarray*}

\begin{eqnarray*}
  \\
  L : TS \to ST
\end{eqnarray*}

\begin{eqnarray*}
  \\
  P \models E \iff P \in \meaningof{E}
\end{eqnarray*}

\begin{eqnarray*}
  P \approx_{L} Q \iff \forall E \in L. P \models E \iff Q \models E
\end{eqnarray*}

\begin{eqnarray*}
  P \approx_{K} Q
\end{eqnarray*}

\begin{eqnarray*}
  P \approx Q
\end{eqnarray*}

$\approx_{K} = \approx = \approx_{L}$

\subsubsection{Contextual duality}

Note that contexts extend the quotation operation to a family of
operations from processes to names. Given a context, $M$, we can
define a \emph{nominal context}, $\quotep{M}$ by $\quotep{M}[P] :=
\quotep{M[P]}$. To foreshadow what is to come we observe that these
operations enjoy a duality with processes very much like the duality
between vectors and maps from vectors to scalars.

Further, because the calculus is essentially higher-order, we have a
correspondence between contexts and processes. More specifically,
given a name $x$ and a context $M$ we can construct $M^{*}_{x}$ such
that 

\begin{mathpar}
  M^{*}_{x} | \lift{x}{P} \red M[P]
\end{mathpar}

namely,

\begin{mathpar}
  M^{*}_{x} := x?(u).M[\dropn{u}]
\end{mathpar}

The dependence of $M^{*}_{x}$ on a name makes it an abstraction, 

\begin{mathpar}
  M^{*} := (x)x?(u).M[\dropn{u}]
\end{mathpar}

\subsection{Additional notation}

It will sometimes be convenient to denote the process a name
quotes. We already have the notation $x = \quotep{P}$, but it will be
convenient to introduce an alternate notation, $\procn{x}$, when we
want to emphasize the connection to the use of the name. Note that, by
virtue of name equivalence, $\quotep{\procn{x}} \nameeq x$; so, the
notation is consistent with previous definitions.

Further, because names have structure it is possible to effect
substitutions on the basis of that structure. This means we need to
upgrade our notation for substitutions, which we accomplish by
adapting comprehension notation. Thus,

\begin{mathpar}
  P\{ y / x : x \in S \}
\end{mathpar}

is interpreted to mean the process derived from P by replacing (in a
capture-avoiding manner) each occurrence of $x$ in $S$ by $y$. For example,

\begin{mathpar}
  P\{ \quotep{\procn{x}|\procn{x}} / x : x \in \freenames{P} \}
\end{mathpar}

will replace each (occurrence) of a free name $x$ in $P$ by
$\quotep{\procn{x}|\procn{x}}$.

Also, we will avail ourselves of the notation $x^{L}$ and $x^{R}$ to
denote injections of a name into disjoint copies of the name
space. There are numerous ways to accomplish this. One example can be
found in \cite{MeredithR05}. This notation overloads to vectors of
names: $\vec{x}^{\pi} := (x_{i}^{\pi} \; : \; 0 \leq i < |\vec{x}| )$ where $\pi \in \{L,R\}$.

We also use $P^{\Box} := P|\Box$.

In \cite{MeredithR05} an interpretation of the new operator is
given. It turns out that there are several possible interpretations
all enjoying the requisite algebraic properties of the operator (see
\cite{milner91polyadicpi}). We will therefore make liberal use of
$(\nu\; \vec{x})P$.

% subsection the_syntax_and_semantics_of_the_notation_system (end)   

\input{qm2pi.qmops} 

\input{qm2pi.sterngerlach} 

\input{qm2pi.metric} 

% section concurrent_process_calculi (end)

%\input{qm2pi.proofsketch}

% section proof sketch (end)

%\input{qm2pi.slviaknots} 

% section spatial logic via knots (end)

\input{qm2pi.conclusion}

% section conclusion (end)

%\input{qm2pi.dtcodes} 

% section wiring algorithm (end)

\input{qm2pi.ack} 

% section acknowledgments (end)

\newpage


\bibliographystyle{plain}   
\bibliography{../../biblios/main.bib}

\input{qm2pi.rhodetails}

\end{document}

 

%\ifpdf
%\usepackage[pdftex]{graphicx}
%\else
%\usepackage{graphicx}
%\fi

 % \ifpdf
%  \usepackage{pdfsync}
%  \if


%\title{Brief Article}
%\author{David F. Snyder}
%\author{L.G. Meredith}

%\address{Dept. of Math., Texas State University--San Marcos, San Marcos, TX 78666}
       
\pagestyle{empty}


\begin{document}

\lstset{language=[Objective]Caml,frame=shadowbox}

\documentclass[12pt]{llncs}
%\documentclass{jktr}

\usepackage[pdftex]{hyperref}                   
\usepackage {listings}
\usepackage {mathpartir}
\usepackage{bcprules}
%\usepackage{listings}
                       
\usepackage{graphicx} 
%\usepackage[margins=2.5cm,nohead,nofoot]{geometry}
%\usepackage{geometry}
\usepackage{amsfonts}
\usepackage{amstext}
\usepackage{latexsym}
\usepackage{amssymb}
\usepackage{color}


%\include{myPreamble}
\include{qm2pi.local} 

%\ifpdf
%\usepackage[pdftex]{graphicx}
%\else
%\usepackage{graphicx}
%\fi

 % \ifpdf
%  \usepackage{pdfsync}
%  \if


%\title{Brief Article}
%\author{David F. Snyder}
%\author{L.G. Meredith}

%\address{Dept. of Math., Texas State University--San Marcos, San Marcos, TX 78666}
       
\pagestyle{empty}


\begin{document}

\lstset{language=[Objective]Caml,frame=shadowbox}

\input{qm2pi.front}

% section front matter (end)

\input{qm2pi.intro} 
 
% section introduction (end)

% \input{qm2pi.knotations} 

% section notation (end)

\input{qm2pi.process.calculi} 

% section concurrent_process_calculi_and_spatial_logics_ (end)
    
%\input{qm2pi.knots2pi} 

%\input{qm2pi.trefoil} 

%\input{qm2pi.mainthm} 

% subsection basic_interpretation (end)

%\input{qm2pi.rho.presentation} 
\subsection{The syntax and semantics of the notation system}\label{sub:the_syntax_and_semantics_of_the_notation_system} % (fold)

We now summarize a technical presentation of the calculus that
embodies our theory of dynamics. The typical presentation of such a
calculus follows the style of giving generators and relations on
them. The grammar, below, describing term constructors, freely
generates the set of processes, $\Proc$. This set is then quotiented
by a relation known as structural congruence and it is over this set
that the notion of dynamics is expressed. This presentation is
essentially that of \cite{MeredithR05} with the addition of
polyadicity and summation. For readability we have relegated some of
the technical subtleties to an appendix.

\subsubsection{Process grammar}\label{subsub:process_grammar}

\begin{mathpar}
  \inferrule* [lab=synchronization] {} {{M} \bc \pzero \;|\; x?F \;|\; x!C }
  \and
  \inferrule* [lab=abstraction] {} {{F} \bc (x)P}
  \and
  \inferrule* [lab=concretion] {} {{C} \bc \langle Q \rangle}
  \and
  \inferrule* [lab=process] {} {{P,Q} \bc M \;| \;P|Q \;|\; @{x}}
  \and
  \inferrule* [lab=name] {} {{x} \bc \quotep{P}}
\end{mathpar} 

Note that $\vec{x}$ (resp. $\vec{P}$) denotes a vector of names
(resp. processes) of length $|\vec{x}|$ (resp. $|\vec{P}|$). We adopt
the following useful abbreviations.

\begin{mathpar}
   x?(\vec{y}).P := x.(\vec{y})P \and  x\clift{\vec{P}} := x.\clift{\vec{P}}
   \and x!(y) := \lift{x}{\dropn{y}}
   \and \Pi_{i=0}^{n-1}P_i := P_0 | \ldots | P_{n-1}
\end{mathpar}

\subsubsection{Structural congruence}

\paragraph{Free and bound names and alpha-equivalence.} At the
core of structural equivalence is alpha-equivalence which identifies
process that are the same up to a change of variable. Formally, we
recognize the distinction between free and bound names. The free names
of a process, $\freenames{P}$, may be calculated recursively as
follows:

\begin{mathpar}
\freenames{\pzero} := \emptyset
  \and \\
  \freenames{x?(y).P} := \{ x \} \cup (\freenames{P} \setminus \{ y \})
  \and 
  \freenames{x!\langle P \rangle} := \{ x \} \cup \{ P \} 
  \and \\
  \freenames{P|Q} := \freenames{P} \cup \freenames{Q}
  \and \\
  \freenames{@{x}} := \{ x \}
\end{mathpar}

$\pi$
$\quotep{\pi}$

$\freenames{-} : \pi \to \mathcal{P}(\quotep{\pi})$

\begin{eqnarray*}
  \freenames{\pzero} & := & \emptyset \\
  \freenames{x?(y).P} & := & \{ x \} \cup (\freenames{P} \setminus \{ y \}) \\
  \freenames{x!\langle P \rangle} & := & \{ x \} \cup \{ P \} \\
  \freenames{P|Q} & := & \freenames{P} \cup \freenames{Q} \\
  \freenames{\dropn{x}} & := & \{ x \}
\end{eqnarray*}

The bound names of a process, $\boundnames{P}$, are those names occurring in $P$
that are not free. For example, in $x?(y).0$, the name $x$ is free, while $y$ is bound.

\begin{mathpar}
  \inferrule* [lab=monoidal-laws] {} { P|Q \equiv Q|P \and P|0 \equiv P \and P|(Q|R) \equiv (P|Q)|R }
\end{mathpar}

\begin{mathpar}
  \inferrule* [lab=alpha-equivalence] {} { (x)P \equiv (y)P\{y/x\} \and y \not\in \freenames{P} }
\end{mathpar}

\begin{definition}
Then two processes, $P,Q$, are alpha-equivalent if $P = Q\{\vec{y}/\vec{x}\}$ for
some $\vec{x} \in \boundnames{Q},\vec{y} \in \boundnames{P}$, where $Q\{\vec{y}/\vec{x}\}$
denotes the capture-avoiding substitution of $\vec{y}$ for $\vec{x}$ in $Q$.
\end{definition}

\begin{definition}
  The {\em structural congruence} \cite{SangiorgiWalker} , $\equiv$,
  between processes is the least congruence containing
  alpha-equivalence, satisfying the abelian monoid laws
  (associativity, commutativity and $\pzero$ as identity) for parallel
  composition $|$ and for summation $+$.
\end{definition}

\subsection{Name equivalence}

We take name equivalence, written $\nameeq$, to be the smallest
equivalence relation generated by the following rules.

\begin{mathpar}
\inferrule*[lab=Quote-drop]
{ }
{ \quotep{@{x}} \nameeq x }

\inferrule*[lab=Struct-equiv]
{ P \scong Q }
{ \quotep{P} \nameeq \quotep{Q} }
\end{mathpar}

The astute reader will have noticed that the mutual recursion of names
and processes imposes a mutual recursion on alpha-equivalence and
structural equivalence via name-equivalence. Fortunately, all of this
works out pleasantly and we may calculate in the natural way, free of
concern. The reader interested in the details is referred to the
appendix \ref{appendix:rho_details}.

\subsection{Substitution}

We use $\Proc$ for the set of processes, $\QProc$ for the set of
names, and $\id{\{}\vec{y} / \vec{x} \id{\}}$ to denote partial maps,
$s : \QProc \rightarrow \QProc$. A map, $s$ lifts, uniquely, to a map
on process terms, $\widehat{s} : \Proc \rightarrow \Proc$ by the
following equations.

\begin{mathpar}
  (0) \psubstp{Q}{P} := 0 \\
  (R \juxtap S) \psubstp{Q}{P}
  :=    
  (R)\psubstp{Q}{P} \juxtap (S) \psubstp{Q}{P} \\
  (x?(y).R) \psubstp{Q}{P}    
  :=    
  (x)\substp{Q}{P} (z)\concat( (R \psubstn{z}{y}) \psubstp{Q}{P} ) \\
  (\lift{x}{R}) \psubstp{Q}{P}  
  :=
  \lift{(x)\substp{Q}{P}}{ R \psubstp{Q}{P} } \\
%   (\dropn{x})  \psubstp{Q}{P}       
%   := 
%   \left\{ 
%     \begin{array}{ccc} 
%       \dropn{\quotep{Q}} & & x \nameeq \quotep{P} \\
%       \dropn{x} & & otherwise \\
%     \end{array}
%   \right. 
  (\dropn{x})  \psubstp{Q}{P}       
  := 
  \left\{ 
    \begin{array}{ccc} 
      Q & & x \nameeq \quotep{P} \\
      \dropn{x} & & otherwise \\
    \end{array}
  \right.
\end{mathpar}
 

where

\begin{eqnarray}
  (x)\id{\{} \lpquote Q \rpquote / \lpquote P \rpquote \id{\}}            = 
  \left\{ 
    \begin{array}{ccc}
      \lpquote Q \rpquote & & x \nameeq \lpquote P \rpquote \\
      x & & otherwise \\
    \end{array}
  \right. \nonumber
\end{eqnarray}

and $z$ is chosen distinct from $\quotep{P}$, $\quotep{Q}$, the free
names in $Q$, and all the names in $R$. Our $\alpha$-equivalence will
be built in the standard way from this substitution.

\begin{remark}\label{rem:no_self_referential_names}
  One consequence of these definitions is that $\forall P. \quotep{P}
  \not\in \freenames{P}$.
\end{remark}

\subsection{ Dynamic quote: an example }

Anticipating something of what's to come, consider applying the
substitution, $\widehat{\id{\{}u / z \id{\}}}$, to the following pair
of processes, $\lift{w}{y!(z)}$ and $w[ \lpquote y!(z) \rpquote ]$.

\begin{eqnarray}
	\lift{w}{y!(z)}\widehat{\id{\{}u / z \id{\}}}
		& = &
		\lift{w}{y!(u)} \nonumber\\
	w[ \lpquote y!(z) \rpquote ] \widehat{ \id{\{}u / z \id{\}} }
		& = &
		w[ \lpquote y!(z) \rpquote ] \nonumber
\end{eqnarray}

Because the body of the process between quotes is impervious to
substitution, we get radically different answers. In fact, by
examining the first process in an input context,
e.g. $x?(z).\lift{w}{y!(z)}$, we see that the process under the lift
operator may be shaped by prefixed inputs binding a name inside it. In
this sense, the lift operator will be seen as a way to dynamically
construct processes before reifying them as names.

Finally equipped with these standard features we can present the
dynamics of the calculus.

\subsubsection{Operational semantics} 

Finally, we introduce the computational dynamics. What marks these
algebras as distinct from other more traditionally studied algebraic
structures, e.g. vector spaces or polynomial rings, is the manner in
which dynamics is captured. In traditional structures, dynamics is typically
expressed through morphisms between such structures, as in linear maps
between vector spaces or morphisms between rings. In algebras
associated with the semantics of computation, the dynamics is
expressed as part of the algebraic structure itself, through a
reduction reduction relation typically denoted by $\red$. Below, we
give a recursive presentation of this relation for the calculus used
in the encoding.

$\red \subseteq \pi \times \pi$
$\red : \pi \to \mathcal{P}(\pi)$

\begin{mathpar}
  \inferrule* [lab=Comm] { \textsf{match}( x_{src}, x_{trgt} ) } { x_{trgt}?(y)P \; | \; x_{src}!\langle {Q} \rangle \red P\{\quotep{Q}/y}\} }
  \and \\
  \inferrule* [lab=Par] {{P} \red {P}'} {{{P} | {Q}} \red {{P}' | {Q}}}
  \and
  \inferrule* [lab=Equiv]{{{P} \scong {P}'} \andalso {{P}' \red {Q}'} \andalso {{Q}' \scong {Q}}}{{P} \red {Q}}
\end{mathpar}

\begin{eqnarray*}
  match_{\equiv} (\quotep{P},\quotep{Q}) & := & P \equiv Q \\
  match_{\dagger}(\quotep{P},\quotep{Q}) & := & \forall R. P|Q \red^{*} R => R \red^{*} 0 \\
  match_{K}(\quotep{P},\quotep{Q}) & := & K \mbox{ for some context } K
\end{eqnarray*}

$u?(x)P | u!\langle Q \rangle \red P\{\quotep{Q}/x\}$

%We write $\wred$ for $\red^*$, and $P\red$ if $\exists Q $ such that $ P \red Q$.
We write $P\red$ if $\exists Q $ such that $ P \red Q$ and $P\not\red$, otherwise.

\section{Replication}

As mentioned before, it is known that replication (and hence
recursion) can be implemented in a higher-order process algebra
\cite{SangiorgiWalker}. As our first example of calculation with the
machinery thus far presented we give the construction explicitly in
the {\rhoc}.

\begin{eqnarray}
	D_{x} & := & \prefix{x}{y}{(\binpar{\outputp{x}{y}}{@{y}})} \nonumber\\
	\bangp_{x}{P} & := & \binpar{{x}!\langle{\binpar{D_{x}}{P}}\rangle}{D_{x}} \nonumber
\end{eqnarray}

\begin{eqnarray}
	\bangp_{x}{P} & & \nonumber\\
	=
	& {x}!\langle{(\prefix{x}{y}{(\outputp{x}{y} | @{y})) | P}}\rangle 
	      | \prefix{x}{y}{(\outputp{x}{y} | @{y})} & \nonumber\\
	\red
	& (\outputp{x}{y} | @{y})\substn{\quotep{(\prefix{x}{y}{(@{y} | \outputp{x}{y})) | P}}}{y} & \nonumber\\
	=
	& \outputp{x}{\quotep{(\prefix{x}{y}{(\outputp{x}{y} | @{y})) | P}}}
	  | {(\prefix{x}{y}{(\outputp{x}{y} | @{y})) | P}} & \nonumber\\
	\red
	& \ldots & \nonumber\\
	\red^*
	& P | P | \ldots & \nonumber
\end{eqnarray}

Of course, this encoding, as an implementation, runs away, unfolding
$\bangp{P}$ eagerly. A lazier and more implementable replication
operator, restricted to input-guarded processes, may be obtained as follows.

\begin{eqnarray}
\bangp{\prefix{u}{v}{P}} 
	:= 
	\binpar{\lift{x}{\prefix{u}{v}{(\binpar{D(x)}{P})}}}{D(x)} \nonumber
\end{eqnarray}

\begin{remark}
  Note that the lazier definition still does not deal with summation
  or mixed summation (i.e. sums over input and output). The reader is
  invited to construct definitions of replication that deal with these
  features. 

  Further, the definitions are parameterized in a name, $x$. Can you,
  gentle reader, make a definition that eliminates this parameter and
  guarantees no accidental interaction between the replication
  machinery and the process being replicated -- i.e. no accidental
  sharing of names used by the process to get its work done and the
  name(s) used by the replication to effect copying. This latter
  revision of the definition of replication is crucial to obtaining
  the expected identity $!!P \sim !P$.
\end{remark}

\begin{remark}\label{rem:paradoxical_combinator}
  The reader familiar with the lambda calculus will have noticed the
  similarity between $D$ and the paradoxical combinator.

  [Ed. note: the existence of this seems to suggest we have to be more
  restrictive on the set of processes and names we admit if we are to
  support no-cloning.]
\end{remark}

\subsubsection{Bisimulation}

The computational dynamics gives rise to another kind of equivalence,
the equivalence of computational behavior. As previously mentioned
this is typically captured \emph{via} some form of bisimulation.

% The notion we use in this paper is weak barbed bisimulation
% \cite{milner91polyadicpi}.

The notion we use in this paper is derived from weak barbed
bisimulation \cite{milner91polyadicpi}. 

\begin{definition}
An \emph{observation relation}, $\downarrow_{\mathcal N}$, over a set
of names, $\mathcal N$, is the smallest relation satisfying the rules
below.

\infrule[Out-barb]{y \in {\mathcal N}, \; x \nameeq y}
		  {\outputp{x}{v} \downarrow_{\mathcal N} x}
\infrule[Par-barb]{\mbox{$P\downarrow_{\mathcal N} x$ or $Q\downarrow_{\mathcal N} x$}}
		  {\binpar{P}{Q} \downarrow_{\mathcal N} x}

We write $P \Downarrow_{\mathcal N} x$ if there is $Q$ such that 
$P \wred Q$ and $Q \downarrow_{\mathcal N} x$.
\end{definition}

\begin{definition}
%\label{def.bbisim}
An  ${\mathcal N}$-\emph{barbed bisimulation} over a set of names, ${\mathcal N}$, is a symmetric binary relation 
${\mathcal S}_{\mathcal N}$ between agents such that $P\rel{S}_{\mathcal N}Q$ implies:
\begin{enumerate}
\item If $P \red P'$ then $Q \wred Q'$ and $P'\rel{S}_{\mathcal N} Q'$.
\item If $P\downarrow_{\mathcal N} x$, then $Q\Downarrow_{\mathcal N} x$.
\end{enumerate}
$P$ is ${\mathcal N}$-barbed bisimilar to $Q$, written
$P \wbbisim_{\mathcal N} Q$, if $P \rel{S}_{\mathcal N} Q$ for some ${\mathcal N}$-barbed bisimulation ${\mathcal S}_{\mathcal N}$.
\end{definition}

$\mathcal{R} \subseteq \pi \times \pi$

$P \mathcal{R} Q => \forall P'. P \red P' \Rightarrow \exists Q'. Q \red Q', P' \mathcal{R} Q'$

$P \vdash x \Rightarrow Q \vdash x$

\begin{mathpar}
  \inferrule*[lab=Out-barb]{x \nameeq y}{{y}!\langle{Q}\rangle \vdash x}
  \and
  \inferrule*[lab=Par-barb]{\mbox{$P\vdash x$ or $Q\vdash x$}}{\binpar{P}{Q} \vdash x}
\end{mathpar}

\subsubsection{Contexts}

One of the principle advantages of computational calculi like the
$\pi$-calculus is a well-defined notion of context,
contextual-equivalence and a correlation between
contextual-equivalence and notions of bisimulation. The notion of
context allows the decomposition of a process into (sub-)process and
its syntactic environment, its context. Thus, a context may be
thought of as a process with a ``hole'' (written $\Box$) in it. The
application of a context $M$ to a process $P$, written $M[P]$, is
tantamount to filling the hole in $M$ with $P$. In this paper we do
not need the full weight of this theory, but do make use of the notion
of context in the proof the main theorem. 

\begin{mathpar}
  \inferrule* [lab=summation] {} {{M_{M},M_{N}} \bc \Box \;|\; x.M_{A} \;|\; M_{M}+M_{N}}
  \and
  \inferrule* [lab=agent] {} {{M_{A}} \bc (\vec{x})M_{P} \;| \; \clift{P_0,\ldots,M_{P},\ldots,P_N}}
  \and \\
  \inferrule* [lab=process] {} {{M_{P}} \bc M_{N} \;| \;P|M_{P} }
\end{mathpar} 

\begin{mathpar}
  \inferrule* [lab=sychronization] {} {M_{N} \bc \Box \;|\; x?M_{F} \;|\; x!M_{C}}
  \and
  \inferrule* [lab=abstraction] {} {{M_{F}} \bc (x)M_{P} }
  \and
  \inferrule* [lab=concretion] {} {{M_{C}} \bc \langle M_{P} \rangle }
  \and \\
  \inferrule* [lab=process] {} {{M_{P}} \bc M_{N} \;| \;P|M_{P} }
\end{mathpar}

\begin{definition}[contextual application] Given a context $M$, and
  process $P$, we define the \emph{contextual application}, $M[P] :=
  M\{P/\Box\}$. That is, the contextual application of M to P is the
  substitution of $P$ for $\Box$ in $M$.
\end{definition}

$\meaningof{-} : L \to \mathcal{P}(\pi)$

\begin{mathpar}
  \inferrule* [lab=collection] {} {\meaningof{true} = \pi, \and \meaningof{~E} = \pi \setminus \meaningof{E}, \and \meaningof{E_{1} \& E_{2}} = \meaningof{E_{1}} \cap \meaningof{E_{2}}}
\end{mathpar}

\begin{mathpar}
  \inferrule* [lab=structure] {} {\meaningof{0} = \{ P \in \pi | P \equiv 0 \}, \and \\ \meaningof{E_1 | E_2} = \{ P \in \pi | P \equiv P_{1} | P_{2}, P_{1} \in \meaningof{E_{1}}, P_{2} \in \meaningof{E_2}\} }
\end{mathpar}

\begin{mathpar}
 \inferrule* [lab=behavior] {} {\meaningof{\langle a?b \rangle E} = \{ P \in \pi | P \equiv Q | u?(y)P', \\ \and \\\\ \and \\ \;\;\; u \in \meaningof{a}, \forall z.P'\{z/y\} \in \meaningof{E\{z/b\}}\}, \and \\ \meaningof{a!E} = \{ P \in \pi | P \equiv Q | x!\langle P' \rangle, x \in \meaningof{a} P' \in \meaningof{E}\} }
\end{mathpar}

\begin{mathpar}
 \inferrule* [lab=nominal] {} {\meaningof{\quotep{E}} = \{ \quotep{P} \in \quotep{\pi} | P \in \meaningof{E} \}, \and \meaningof{\quotep{P}} = \{ \quotep{Q} \in \quotep{\pi} | P \equiv Q \} \and \\ \meaningof{@\quotep{E}} = \{ P \in \pi | P \equiv @x, x \in \meaningof{E} \}}
\end{mathpar}

\begin{eqnarray*}
  \\
  \meaningof{-} : TS \to ST
\end{eqnarray*}

\begin{eqnarray*}
  \\
  L : TS \to ST
\end{eqnarray*}

\begin{eqnarray*}
  \\
  P \models E \iff P \in \meaningof{E}
\end{eqnarray*}

\begin{eqnarray*}
  P \approx_{L} Q \iff \forall E \in L. P \models E \iff Q \models E
\end{eqnarray*}

\begin{eqnarray*}
  P \approx_{K} Q
\end{eqnarray*}

\begin{eqnarray*}
  P \approx Q
\end{eqnarray*}

$\approx_{K} = \approx = \approx_{L}$

\subsubsection{Contextual duality}

Note that contexts extend the quotation operation to a family of
operations from processes to names. Given a context, $M$, we can
define a \emph{nominal context}, $\quotep{M}$ by $\quotep{M}[P] :=
\quotep{M[P]}$. To foreshadow what is to come we observe that these
operations enjoy a duality with processes very much like the duality
between vectors and maps from vectors to scalars.

Further, because the calculus is essentially higher-order, we have a
correspondence between contexts and processes. More specifically,
given a name $x$ and a context $M$ we can construct $M^{*}_{x}$ such
that 

\begin{mathpar}
  M^{*}_{x} | \lift{x}{P} \red M[P]
\end{mathpar}

namely,

\begin{mathpar}
  M^{*}_{x} := x?(u).M[\dropn{u}]
\end{mathpar}

The dependence of $M^{*}_{x}$ on a name makes it an abstraction, 

\begin{mathpar}
  M^{*} := (x)x?(u).M[\dropn{u}]
\end{mathpar}

\subsection{Additional notation}

It will sometimes be convenient to denote the process a name
quotes. We already have the notation $x = \quotep{P}$, but it will be
convenient to introduce an alternate notation, $\procn{x}$, when we
want to emphasize the connection to the use of the name. Note that, by
virtue of name equivalence, $\quotep{\procn{x}} \nameeq x$; so, the
notation is consistent with previous definitions.

Further, because names have structure it is possible to effect
substitutions on the basis of that structure. This means we need to
upgrade our notation for substitutions, which we accomplish by
adapting comprehension notation. Thus,

\begin{mathpar}
  P\{ y / x : x \in S \}
\end{mathpar}

is interpreted to mean the process derived from P by replacing (in a
capture-avoiding manner) each occurrence of $x$ in $S$ by $y$. For example,

\begin{mathpar}
  P\{ \quotep{\procn{x}|\procn{x}} / x : x \in \freenames{P} \}
\end{mathpar}

will replace each (occurrence) of a free name $x$ in $P$ by
$\quotep{\procn{x}|\procn{x}}$.

Also, we will avail ourselves of the notation $x^{L}$ and $x^{R}$ to
denote injections of a name into disjoint copies of the name
space. There are numerous ways to accomplish this. One example can be
found in \cite{MeredithR05}. This notation overloads to vectors of
names: $\vec{x}^{\pi} := (x_{i}^{\pi} \; : \; 0 \leq i < |\vec{x}| )$ where $\pi \in \{L,R\}$.

We also use $P^{\Box} := P|\Box$.

In \cite{MeredithR05} an interpretation of the new operator is
given. It turns out that there are several possible interpretations
all enjoying the requisite algebraic properties of the operator (see
\cite{milner91polyadicpi}). We will therefore make liberal use of
$(\nu\; \vec{x})P$.

% subsection the_syntax_and_semantics_of_the_notation_system (end)   

\input{qm2pi.qmops} 

\input{qm2pi.sterngerlach} 

\input{qm2pi.metric} 

% section concurrent_process_calculi (end)

%\input{qm2pi.proofsketch}

% section proof sketch (end)

%\input{qm2pi.slviaknots} 

% section spatial logic via knots (end)

\input{qm2pi.conclusion}

% section conclusion (end)

%\input{qm2pi.dtcodes} 

% section wiring algorithm (end)

\input{qm2pi.ack} 

% section acknowledgments (end)

\newpage


\bibliographystyle{plain}   
\bibliography{../../biblios/main.bib}

\input{qm2pi.rhodetails}

\end{document}



% section front matter (end)

\section{Introduction}\label{sec:introduction} % (fold)
In this draft of the material i am going to have to dispense with the
usual writing conventions adopted in papers on these topics. i'm going
to have adopt whatever tone i need at the time i'm writing up the
calculations. Sometimes this may be very conversational; others it may
be the barest mathematical grunts; others still it may be that i have
lifted text from one of my other papers because the exposition of some
point was better said there. i hope that my readers are not unduly put
out by this decision. i'm not doing this to flout convention or be
rebellious. i find these calculations very technically challenging. To
keep everything going technically, something has to give; i have to
let go of some cognitive burden. So, the academic writing style --
with all of its trade-offs in terms of facilitating technical
communication -- is what i'm letting go of. Perhaps subsequent drafts
can be tightened and polished, but for now, i'm going to speak as if
we were sitting together in a coffee shop with a laptop, wifi and a
pad of paper and a pencil.

So, here's what i have to say. We -- you and i, comfortably ensconced
in our coffee shop and well-equipped with our tools -- can realize and
carry out the calculations of quantum mechanics over a very different
formal theory of dynamics, a formal theory of dynamics that
corresponds to a theory of concurrent computation with
\emph{reflection}. It has the advantage that the underlying theory is
already `quantized', but supports analogues all of the continuuous
operations. Strikingly, this underlying theory has recently been
connected with a notion of metric that we can show, by calculating
together, coincides with the metric induced by the inner product.

There are a lot of reasons why you might be interested in seeing
calculations of this form. Here's why i'm interested. For the past
several centuries there has been no competitor to the ``Newtonian''
account of dynamics. As a result the predominant share of accounts of
dynamical systems and situations have had to be formulated in terms of
the Newtonian machinery. i view this as an intellectually dangerous
position to occupy. Everything, despite it's intrinsic shape, turns
into a nail to be hit with this hammer. Recently, however, the theory
of computation has matured to the point where we have candidates for
theories of dynamics that offer very different perspective on
reasoning about dynamical systems and situations. Testing these
candidates against very successful accounts of dynamical situations,
like quantum mechanics, is going to give us some sense of how mature
they are and some measure of the quality of these accounts of
dynamics.

\subsection{Summary of contributions and outline of paper}

So, we're going to develop an interpretation of the operations of
quantum mechanics normally interpreted by Hilbert spaces and
operators. We're going to do this over a theory of computation. Note
that this is very different than the usual quantum computation program
which develops notions of computation over quantum mechanics. Rather,
we are developing a story that aligns with Wheeler's slogan: It from
Bit. To do this we will first provide an account of the theory of
computation at play here. Then we will dive into a calculation-driven
interpretation of the operations of quantum mechanics.

The reason we take this approach is that -- until very recently --
there hasn't been an axiomatic account of quantum mechanics. As a
result there has been no sharp delineation of the mathematical theory
supporting interpretation of the physical theory and the physical
theory, itself. So, ambient features of the maths are free to be
exploited (or supressed) without a real accounting of their physical
relevance. There is no sharp statement ``here's the physical theory''
qua \emph{theory} and ``here's the mathematical interpretation''
enabling a judgment of how faithful the interpretation is -- apart
from experimental observation. When there is an axiomatic account we
can judge how well a given mathematical formalism supports an
interpretation of the axioms, independent of
experimentation. Likewise, we can judge how well we have captured our
physical evidence and experience with our axiomatics, independent of
any specific mathematical implementation, with accidental detail that
may or may not have physical significance. 

In lieu of a fully fleshed out and vetted axiomatic account of quantum
mechanics, interpreting the operational notions in service of modeling
physical systems will have to suffice. In other words, we are not in
the business of providing a model of Hilbert spaces and operators. We
are in the business of providing a model of quantum mechanics because
we are motivated by testing our notions of dynamics against physical
theory; and, the predictive calculations of the physical theory must
serve as the best formulation -- shy of a fully fleshed out axiomatic
account -- of the physical theory itself (as they have for scientific
theories since time immemorial). Put another way, despite a
whole-hearted commitment to an It-from-Bit ontology, we are firmly
aligned with the shut-up-and-calculate camp as the best way to obtain
results either from the physical perspective or as a quality assurance
measure of our fledgling theory of dynamics.

In detail, we present a reflective process calculus. Then we develop
intuitive correspondences between the notions available in this
calculus and the usual physical notions supporting quantum mechanical
calculations. Thus, 

\begin{table}[htp]
  \center{
    \fbox{
      \begin{tabular}{c|c}
        quantum mechanics & process calculus \\
        \hline
        scalar & name \\
        state vector & process \\
        dual & contextual duals \\
        matrix & formal sums of process-context-dual pairs \\
        orthogonality & process annihilation \\
        inner product & execution-formula + quoting
      \end{tabular}
    }
  }
  \caption{QM - process calculi correspondences}
\end{table}

Then we tighten up these intuitions to operational definitions. We
employ the Dirac notation as the best proxy we can find for an
abstract syntax of the quantum mechanical notions. The definitions we
develop put us in contact with equational constraints coming from the
theory that we demonstrate the definitions and calculations satisfy.

This puts us in a position to shut up and calculate for the
Stern-Gerlach experimental set up, showing how these predictive
calculations become calculations on processes in our theory of a
reflective process calculus.

Penultimately, we demonstrate that the notion of metric coming from
the inner product coincides with the notion of metric available from
the theory of bisimulation. This demonstration gives us the right to
think of space as arising from behavior. Finally, we consider where we
might go from the new vantage point we have obtained.

% section introduction (end) 
 
% section introduction (end)

% \documentclass[12pt]{llncs}
%\documentclass{jktr}

\usepackage[pdftex]{hyperref}                   
\usepackage {listings}
\usepackage {mathpartir}
\usepackage{bcprules}
%\usepackage{listings}
                       
\usepackage{graphicx} 
%\usepackage[margins=2.5cm,nohead,nofoot]{geometry}
%\usepackage{geometry}
\usepackage{amsfonts}
\usepackage{amstext}
\usepackage{latexsym}
\usepackage{amssymb}
\usepackage{color}


%\include{myPreamble}
\include{qm2pi.local} 

%\ifpdf
%\usepackage[pdftex]{graphicx}
%\else
%\usepackage{graphicx}
%\fi

 % \ifpdf
%  \usepackage{pdfsync}
%  \if


%\title{Brief Article}
%\author{David F. Snyder}
%\author{L.G. Meredith}

%\address{Dept. of Math., Texas State University--San Marcos, San Marcos, TX 78666}
       
\pagestyle{empty}


\begin{document}

\lstset{language=[Objective]Caml,frame=shadowbox}

\input{qm2pi.front}

% section front matter (end)

\input{qm2pi.intro} 
 
% section introduction (end)

% \input{qm2pi.knotations} 

% section notation (end)

\input{qm2pi.process.calculi} 

% section concurrent_process_calculi_and_spatial_logics_ (end)
    
%\input{qm2pi.knots2pi} 

%\input{qm2pi.trefoil} 

%\input{qm2pi.mainthm} 

% subsection basic_interpretation (end)

%\input{qm2pi.rho.presentation} 
\subsection{The syntax and semantics of the notation system}\label{sub:the_syntax_and_semantics_of_the_notation_system} % (fold)

We now summarize a technical presentation of the calculus that
embodies our theory of dynamics. The typical presentation of such a
calculus follows the style of giving generators and relations on
them. The grammar, below, describing term constructors, freely
generates the set of processes, $\Proc$. This set is then quotiented
by a relation known as structural congruence and it is over this set
that the notion of dynamics is expressed. This presentation is
essentially that of \cite{MeredithR05} with the addition of
polyadicity and summation. For readability we have relegated some of
the technical subtleties to an appendix.

\subsubsection{Process grammar}\label{subsub:process_grammar}

\begin{mathpar}
  \inferrule* [lab=synchronization] {} {{M} \bc \pzero \;|\; x?F \;|\; x!C }
  \and
  \inferrule* [lab=abstraction] {} {{F} \bc (x)P}
  \and
  \inferrule* [lab=concretion] {} {{C} \bc \langle Q \rangle}
  \and
  \inferrule* [lab=process] {} {{P,Q} \bc M \;| \;P|Q \;|\; @{x}}
  \and
  \inferrule* [lab=name] {} {{x} \bc \quotep{P}}
\end{mathpar} 

Note that $\vec{x}$ (resp. $\vec{P}$) denotes a vector of names
(resp. processes) of length $|\vec{x}|$ (resp. $|\vec{P}|$). We adopt
the following useful abbreviations.

\begin{mathpar}
   x?(\vec{y}).P := x.(\vec{y})P \and  x\clift{\vec{P}} := x.\clift{\vec{P}}
   \and x!(y) := \lift{x}{\dropn{y}}
   \and \Pi_{i=0}^{n-1}P_i := P_0 | \ldots | P_{n-1}
\end{mathpar}

\subsubsection{Structural congruence}

\paragraph{Free and bound names and alpha-equivalence.} At the
core of structural equivalence is alpha-equivalence which identifies
process that are the same up to a change of variable. Formally, we
recognize the distinction between free and bound names. The free names
of a process, $\freenames{P}$, may be calculated recursively as
follows:

\begin{mathpar}
\freenames{\pzero} := \emptyset
  \and \\
  \freenames{x?(y).P} := \{ x \} \cup (\freenames{P} \setminus \{ y \})
  \and 
  \freenames{x!\langle P \rangle} := \{ x \} \cup \{ P \} 
  \and \\
  \freenames{P|Q} := \freenames{P} \cup \freenames{Q}
  \and \\
  \freenames{@{x}} := \{ x \}
\end{mathpar}

$\pi$
$\quotep{\pi}$

$\freenames{-} : \pi \to \mathcal{P}(\quotep{\pi})$

\begin{eqnarray*}
  \freenames{\pzero} & := & \emptyset \\
  \freenames{x?(y).P} & := & \{ x \} \cup (\freenames{P} \setminus \{ y \}) \\
  \freenames{x!\langle P \rangle} & := & \{ x \} \cup \{ P \} \\
  \freenames{P|Q} & := & \freenames{P} \cup \freenames{Q} \\
  \freenames{\dropn{x}} & := & \{ x \}
\end{eqnarray*}

The bound names of a process, $\boundnames{P}$, are those names occurring in $P$
that are not free. For example, in $x?(y).0$, the name $x$ is free, while $y$ is bound.

\begin{mathpar}
  \inferrule* [lab=monoidal-laws] {} { P|Q \equiv Q|P \and P|0 \equiv P \and P|(Q|R) \equiv (P|Q)|R }
\end{mathpar}

\begin{mathpar}
  \inferrule* [lab=alpha-equivalence] {} { (x)P \equiv (y)P\{y/x\} \and y \not\in \freenames{P} }
\end{mathpar}

\begin{definition}
Then two processes, $P,Q$, are alpha-equivalent if $P = Q\{\vec{y}/\vec{x}\}$ for
some $\vec{x} \in \boundnames{Q},\vec{y} \in \boundnames{P}$, where $Q\{\vec{y}/\vec{x}\}$
denotes the capture-avoiding substitution of $\vec{y}$ for $\vec{x}$ in $Q$.
\end{definition}

\begin{definition}
  The {\em structural congruence} \cite{SangiorgiWalker} , $\equiv$,
  between processes is the least congruence containing
  alpha-equivalence, satisfying the abelian monoid laws
  (associativity, commutativity and $\pzero$ as identity) for parallel
  composition $|$ and for summation $+$.
\end{definition}

\subsection{Name equivalence}

We take name equivalence, written $\nameeq$, to be the smallest
equivalence relation generated by the following rules.

\begin{mathpar}
\inferrule*[lab=Quote-drop]
{ }
{ \quotep{@{x}} \nameeq x }

\inferrule*[lab=Struct-equiv]
{ P \scong Q }
{ \quotep{P} \nameeq \quotep{Q} }
\end{mathpar}

The astute reader will have noticed that the mutual recursion of names
and processes imposes a mutual recursion on alpha-equivalence and
structural equivalence via name-equivalence. Fortunately, all of this
works out pleasantly and we may calculate in the natural way, free of
concern. The reader interested in the details is referred to the
appendix \ref{appendix:rho_details}.

\subsection{Substitution}

We use $\Proc$ for the set of processes, $\QProc$ for the set of
names, and $\id{\{}\vec{y} / \vec{x} \id{\}}$ to denote partial maps,
$s : \QProc \rightarrow \QProc$. A map, $s$ lifts, uniquely, to a map
on process terms, $\widehat{s} : \Proc \rightarrow \Proc$ by the
following equations.

\begin{mathpar}
  (0) \psubstp{Q}{P} := 0 \\
  (R \juxtap S) \psubstp{Q}{P}
  :=    
  (R)\psubstp{Q}{P} \juxtap (S) \psubstp{Q}{P} \\
  (x?(y).R) \psubstp{Q}{P}    
  :=    
  (x)\substp{Q}{P} (z)\concat( (R \psubstn{z}{y}) \psubstp{Q}{P} ) \\
  (\lift{x}{R}) \psubstp{Q}{P}  
  :=
  \lift{(x)\substp{Q}{P}}{ R \psubstp{Q}{P} } \\
%   (\dropn{x})  \psubstp{Q}{P}       
%   := 
%   \left\{ 
%     \begin{array}{ccc} 
%       \dropn{\quotep{Q}} & & x \nameeq \quotep{P} \\
%       \dropn{x} & & otherwise \\
%     \end{array}
%   \right. 
  (\dropn{x})  \psubstp{Q}{P}       
  := 
  \left\{ 
    \begin{array}{ccc} 
      Q & & x \nameeq \quotep{P} \\
      \dropn{x} & & otherwise \\
    \end{array}
  \right.
\end{mathpar}
 

where

\begin{eqnarray}
  (x)\id{\{} \lpquote Q \rpquote / \lpquote P \rpquote \id{\}}            = 
  \left\{ 
    \begin{array}{ccc}
      \lpquote Q \rpquote & & x \nameeq \lpquote P \rpquote \\
      x & & otherwise \\
    \end{array}
  \right. \nonumber
\end{eqnarray}

and $z$ is chosen distinct from $\quotep{P}$, $\quotep{Q}$, the free
names in $Q$, and all the names in $R$. Our $\alpha$-equivalence will
be built in the standard way from this substitution.

\begin{remark}\label{rem:no_self_referential_names}
  One consequence of these definitions is that $\forall P. \quotep{P}
  \not\in \freenames{P}$.
\end{remark}

\subsection{ Dynamic quote: an example }

Anticipating something of what's to come, consider applying the
substitution, $\widehat{\id{\{}u / z \id{\}}}$, to the following pair
of processes, $\lift{w}{y!(z)}$ and $w[ \lpquote y!(z) \rpquote ]$.

\begin{eqnarray}
	\lift{w}{y!(z)}\widehat{\id{\{}u / z \id{\}}}
		& = &
		\lift{w}{y!(u)} \nonumber\\
	w[ \lpquote y!(z) \rpquote ] \widehat{ \id{\{}u / z \id{\}} }
		& = &
		w[ \lpquote y!(z) \rpquote ] \nonumber
\end{eqnarray}

Because the body of the process between quotes is impervious to
substitution, we get radically different answers. In fact, by
examining the first process in an input context,
e.g. $x?(z).\lift{w}{y!(z)}$, we see that the process under the lift
operator may be shaped by prefixed inputs binding a name inside it. In
this sense, the lift operator will be seen as a way to dynamically
construct processes before reifying them as names.

Finally equipped with these standard features we can present the
dynamics of the calculus.

\subsubsection{Operational semantics} 

Finally, we introduce the computational dynamics. What marks these
algebras as distinct from other more traditionally studied algebraic
structures, e.g. vector spaces or polynomial rings, is the manner in
which dynamics is captured. In traditional structures, dynamics is typically
expressed through morphisms between such structures, as in linear maps
between vector spaces or morphisms between rings. In algebras
associated with the semantics of computation, the dynamics is
expressed as part of the algebraic structure itself, through a
reduction reduction relation typically denoted by $\red$. Below, we
give a recursive presentation of this relation for the calculus used
in the encoding.

$\red \subseteq \pi \times \pi$
$\red : \pi \to \mathcal{P}(\pi)$

\begin{mathpar}
  \inferrule* [lab=Comm] { \textsf{match}( x_{src}, x_{trgt} ) } { x_{trgt}?(y)P \; | \; x_{src}!\langle {Q} \rangle \red P\{\quotep{Q}/y}\} }
  \and \\
  \inferrule* [lab=Par] {{P} \red {P}'} {{{P} | {Q}} \red {{P}' | {Q}}}
  \and
  \inferrule* [lab=Equiv]{{{P} \scong {P}'} \andalso {{P}' \red {Q}'} \andalso {{Q}' \scong {Q}}}{{P} \red {Q}}
\end{mathpar}

\begin{eqnarray*}
  match_{\equiv} (\quotep{P},\quotep{Q}) & := & P \equiv Q \\
  match_{\dagger}(\quotep{P},\quotep{Q}) & := & \forall R. P|Q \red^{*} R => R \red^{*} 0 \\
  match_{K}(\quotep{P},\quotep{Q}) & := & K \mbox{ for some context } K
\end{eqnarray*}

$u?(x)P | u!\langle Q \rangle \red P\{\quotep{Q}/x\}$

%We write $\wred$ for $\red^*$, and $P\red$ if $\exists Q $ such that $ P \red Q$.
We write $P\red$ if $\exists Q $ such that $ P \red Q$ and $P\not\red$, otherwise.

\section{Replication}

As mentioned before, it is known that replication (and hence
recursion) can be implemented in a higher-order process algebra
\cite{SangiorgiWalker}. As our first example of calculation with the
machinery thus far presented we give the construction explicitly in
the {\rhoc}.

\begin{eqnarray}
	D_{x} & := & \prefix{x}{y}{(\binpar{\outputp{x}{y}}{@{y}})} \nonumber\\
	\bangp_{x}{P} & := & \binpar{{x}!\langle{\binpar{D_{x}}{P}}\rangle}{D_{x}} \nonumber
\end{eqnarray}

\begin{eqnarray}
	\bangp_{x}{P} & & \nonumber\\
	=
	& {x}!\langle{(\prefix{x}{y}{(\outputp{x}{y} | @{y})) | P}}\rangle 
	      | \prefix{x}{y}{(\outputp{x}{y} | @{y})} & \nonumber\\
	\red
	& (\outputp{x}{y} | @{y})\substn{\quotep{(\prefix{x}{y}{(@{y} | \outputp{x}{y})) | P}}}{y} & \nonumber\\
	=
	& \outputp{x}{\quotep{(\prefix{x}{y}{(\outputp{x}{y} | @{y})) | P}}}
	  | {(\prefix{x}{y}{(\outputp{x}{y} | @{y})) | P}} & \nonumber\\
	\red
	& \ldots & \nonumber\\
	\red^*
	& P | P | \ldots & \nonumber
\end{eqnarray}

Of course, this encoding, as an implementation, runs away, unfolding
$\bangp{P}$ eagerly. A lazier and more implementable replication
operator, restricted to input-guarded processes, may be obtained as follows.

\begin{eqnarray}
\bangp{\prefix{u}{v}{P}} 
	:= 
	\binpar{\lift{x}{\prefix{u}{v}{(\binpar{D(x)}{P})}}}{D(x)} \nonumber
\end{eqnarray}

\begin{remark}
  Note that the lazier definition still does not deal with summation
  or mixed summation (i.e. sums over input and output). The reader is
  invited to construct definitions of replication that deal with these
  features. 

  Further, the definitions are parameterized in a name, $x$. Can you,
  gentle reader, make a definition that eliminates this parameter and
  guarantees no accidental interaction between the replication
  machinery and the process being replicated -- i.e. no accidental
  sharing of names used by the process to get its work done and the
  name(s) used by the replication to effect copying. This latter
  revision of the definition of replication is crucial to obtaining
  the expected identity $!!P \sim !P$.
\end{remark}

\begin{remark}\label{rem:paradoxical_combinator}
  The reader familiar with the lambda calculus will have noticed the
  similarity between $D$ and the paradoxical combinator.

  [Ed. note: the existence of this seems to suggest we have to be more
  restrictive on the set of processes and names we admit if we are to
  support no-cloning.]
\end{remark}

\subsubsection{Bisimulation}

The computational dynamics gives rise to another kind of equivalence,
the equivalence of computational behavior. As previously mentioned
this is typically captured \emph{via} some form of bisimulation.

% The notion we use in this paper is weak barbed bisimulation
% \cite{milner91polyadicpi}.

The notion we use in this paper is derived from weak barbed
bisimulation \cite{milner91polyadicpi}. 

\begin{definition}
An \emph{observation relation}, $\downarrow_{\mathcal N}$, over a set
of names, $\mathcal N$, is the smallest relation satisfying the rules
below.

\infrule[Out-barb]{y \in {\mathcal N}, \; x \nameeq y}
		  {\outputp{x}{v} \downarrow_{\mathcal N} x}
\infrule[Par-barb]{\mbox{$P\downarrow_{\mathcal N} x$ or $Q\downarrow_{\mathcal N} x$}}
		  {\binpar{P}{Q} \downarrow_{\mathcal N} x}

We write $P \Downarrow_{\mathcal N} x$ if there is $Q$ such that 
$P \wred Q$ and $Q \downarrow_{\mathcal N} x$.
\end{definition}

\begin{definition}
%\label{def.bbisim}
An  ${\mathcal N}$-\emph{barbed bisimulation} over a set of names, ${\mathcal N}$, is a symmetric binary relation 
${\mathcal S}_{\mathcal N}$ between agents such that $P\rel{S}_{\mathcal N}Q$ implies:
\begin{enumerate}
\item If $P \red P'$ then $Q \wred Q'$ and $P'\rel{S}_{\mathcal N} Q'$.
\item If $P\downarrow_{\mathcal N} x$, then $Q\Downarrow_{\mathcal N} x$.
\end{enumerate}
$P$ is ${\mathcal N}$-barbed bisimilar to $Q$, written
$P \wbbisim_{\mathcal N} Q$, if $P \rel{S}_{\mathcal N} Q$ for some ${\mathcal N}$-barbed bisimulation ${\mathcal S}_{\mathcal N}$.
\end{definition}

$\mathcal{R} \subseteq \pi \times \pi$

$P \mathcal{R} Q => \forall P'. P \red P' \Rightarrow \exists Q'. Q \red Q', P' \mathcal{R} Q'$

$P \vdash x \Rightarrow Q \vdash x$

\begin{mathpar}
  \inferrule*[lab=Out-barb]{x \nameeq y}{{y}!\langle{Q}\rangle \vdash x}
  \and
  \inferrule*[lab=Par-barb]{\mbox{$P\vdash x$ or $Q\vdash x$}}{\binpar{P}{Q} \vdash x}
\end{mathpar}

\subsubsection{Contexts}

One of the principle advantages of computational calculi like the
$\pi$-calculus is a well-defined notion of context,
contextual-equivalence and a correlation between
contextual-equivalence and notions of bisimulation. The notion of
context allows the decomposition of a process into (sub-)process and
its syntactic environment, its context. Thus, a context may be
thought of as a process with a ``hole'' (written $\Box$) in it. The
application of a context $M$ to a process $P$, written $M[P]$, is
tantamount to filling the hole in $M$ with $P$. In this paper we do
not need the full weight of this theory, but do make use of the notion
of context in the proof the main theorem. 

\begin{mathpar}
  \inferrule* [lab=summation] {} {{M_{M},M_{N}} \bc \Box \;|\; x.M_{A} \;|\; M_{M}+M_{N}}
  \and
  \inferrule* [lab=agent] {} {{M_{A}} \bc (\vec{x})M_{P} \;| \; \clift{P_0,\ldots,M_{P},\ldots,P_N}}
  \and \\
  \inferrule* [lab=process] {} {{M_{P}} \bc M_{N} \;| \;P|M_{P} }
\end{mathpar} 

\begin{mathpar}
  \inferrule* [lab=sychronization] {} {M_{N} \bc \Box \;|\; x?M_{F} \;|\; x!M_{C}}
  \and
  \inferrule* [lab=abstraction] {} {{M_{F}} \bc (x)M_{P} }
  \and
  \inferrule* [lab=concretion] {} {{M_{C}} \bc \langle M_{P} \rangle }
  \and \\
  \inferrule* [lab=process] {} {{M_{P}} \bc M_{N} \;| \;P|M_{P} }
\end{mathpar}

\begin{definition}[contextual application] Given a context $M$, and
  process $P$, we define the \emph{contextual application}, $M[P] :=
  M\{P/\Box\}$. That is, the contextual application of M to P is the
  substitution of $P$ for $\Box$ in $M$.
\end{definition}

$\meaningof{-} : L \to \mathcal{P}(\pi)$

\begin{mathpar}
  \inferrule* [lab=collection] {} {\meaningof{true} = \pi, \and \meaningof{~E} = \pi \setminus \meaningof{E}, \and \meaningof{E_{1} \& E_{2}} = \meaningof{E_{1}} \cap \meaningof{E_{2}}}
\end{mathpar}

\begin{mathpar}
  \inferrule* [lab=structure] {} {\meaningof{0} = \{ P \in \pi | P \equiv 0 \}, \and \\ \meaningof{E_1 | E_2} = \{ P \in \pi | P \equiv P_{1} | P_{2}, P_{1} \in \meaningof{E_{1}}, P_{2} \in \meaningof{E_2}\} }
\end{mathpar}

\begin{mathpar}
 \inferrule* [lab=behavior] {} {\meaningof{\langle a?b \rangle E} = \{ P \in \pi | P \equiv Q | u?(y)P', \\ \and \\\\ \and \\ \;\;\; u \in \meaningof{a}, \forall z.P'\{z/y\} \in \meaningof{E\{z/b\}}\}, \and \\ \meaningof{a!E} = \{ P \in \pi | P \equiv Q | x!\langle P' \rangle, x \in \meaningof{a} P' \in \meaningof{E}\} }
\end{mathpar}

\begin{mathpar}
 \inferrule* [lab=nominal] {} {\meaningof{\quotep{E}} = \{ \quotep{P} \in \quotep{\pi} | P \in \meaningof{E} \}, \and \meaningof{\quotep{P}} = \{ \quotep{Q} \in \quotep{\pi} | P \equiv Q \} \and \\ \meaningof{@\quotep{E}} = \{ P \in \pi | P \equiv @x, x \in \meaningof{E} \}}
\end{mathpar}

\begin{eqnarray*}
  \\
  \meaningof{-} : TS \to ST
\end{eqnarray*}

\begin{eqnarray*}
  \\
  L : TS \to ST
\end{eqnarray*}

\begin{eqnarray*}
  \\
  P \models E \iff P \in \meaningof{E}
\end{eqnarray*}

\begin{eqnarray*}
  P \approx_{L} Q \iff \forall E \in L. P \models E \iff Q \models E
\end{eqnarray*}

\begin{eqnarray*}
  P \approx_{K} Q
\end{eqnarray*}

\begin{eqnarray*}
  P \approx Q
\end{eqnarray*}

$\approx_{K} = \approx = \approx_{L}$

\subsubsection{Contextual duality}

Note that contexts extend the quotation operation to a family of
operations from processes to names. Given a context, $M$, we can
define a \emph{nominal context}, $\quotep{M}$ by $\quotep{M}[P] :=
\quotep{M[P]}$. To foreshadow what is to come we observe that these
operations enjoy a duality with processes very much like the duality
between vectors and maps from vectors to scalars.

Further, because the calculus is essentially higher-order, we have a
correspondence between contexts and processes. More specifically,
given a name $x$ and a context $M$ we can construct $M^{*}_{x}$ such
that 

\begin{mathpar}
  M^{*}_{x} | \lift{x}{P} \red M[P]
\end{mathpar}

namely,

\begin{mathpar}
  M^{*}_{x} := x?(u).M[\dropn{u}]
\end{mathpar}

The dependence of $M^{*}_{x}$ on a name makes it an abstraction, 

\begin{mathpar}
  M^{*} := (x)x?(u).M[\dropn{u}]
\end{mathpar}

\subsection{Additional notation}

It will sometimes be convenient to denote the process a name
quotes. We already have the notation $x = \quotep{P}$, but it will be
convenient to introduce an alternate notation, $\procn{x}$, when we
want to emphasize the connection to the use of the name. Note that, by
virtue of name equivalence, $\quotep{\procn{x}} \nameeq x$; so, the
notation is consistent with previous definitions.

Further, because names have structure it is possible to effect
substitutions on the basis of that structure. This means we need to
upgrade our notation for substitutions, which we accomplish by
adapting comprehension notation. Thus,

\begin{mathpar}
  P\{ y / x : x \in S \}
\end{mathpar}

is interpreted to mean the process derived from P by replacing (in a
capture-avoiding manner) each occurrence of $x$ in $S$ by $y$. For example,

\begin{mathpar}
  P\{ \quotep{\procn{x}|\procn{x}} / x : x \in \freenames{P} \}
\end{mathpar}

will replace each (occurrence) of a free name $x$ in $P$ by
$\quotep{\procn{x}|\procn{x}}$.

Also, we will avail ourselves of the notation $x^{L}$ and $x^{R}$ to
denote injections of a name into disjoint copies of the name
space. There are numerous ways to accomplish this. One example can be
found in \cite{MeredithR05}. This notation overloads to vectors of
names: $\vec{x}^{\pi} := (x_{i}^{\pi} \; : \; 0 \leq i < |\vec{x}| )$ where $\pi \in \{L,R\}$.

We also use $P^{\Box} := P|\Box$.

In \cite{MeredithR05} an interpretation of the new operator is
given. It turns out that there are several possible interpretations
all enjoying the requisite algebraic properties of the operator (see
\cite{milner91polyadicpi}). We will therefore make liberal use of
$(\nu\; \vec{x})P$.

% subsection the_syntax_and_semantics_of_the_notation_system (end)   

\input{qm2pi.qmops} 

\input{qm2pi.sterngerlach} 

\input{qm2pi.metric} 

% section concurrent_process_calculi (end)

%\input{qm2pi.proofsketch}

% section proof sketch (end)

%\input{qm2pi.slviaknots} 

% section spatial logic via knots (end)

\input{qm2pi.conclusion}

% section conclusion (end)

%\input{qm2pi.dtcodes} 

% section wiring algorithm (end)

\input{qm2pi.ack} 

% section acknowledgments (end)

\newpage


\bibliographystyle{plain}   
\bibliography{../../biblios/main.bib}

\input{qm2pi.rhodetails}

\end{document}

 

% section notation (end)

\input{qm2pi.process.calculi} 

% section concurrent_process_calculi_and_spatial_logics_ (end)
    
%\documentclass[12pt]{llncs}
%\documentclass{jktr}

\usepackage[pdftex]{hyperref}                   
\usepackage {listings}
\usepackage {mathpartir}
\usepackage{bcprules}
%\usepackage{listings}
                       
\usepackage{graphicx} 
%\usepackage[margins=2.5cm,nohead,nofoot]{geometry}
%\usepackage{geometry}
\usepackage{amsfonts}
\usepackage{amstext}
\usepackage{latexsym}
\usepackage{amssymb}
\usepackage{color}


%\include{myPreamble}
\include{qm2pi.local} 

%\ifpdf
%\usepackage[pdftex]{graphicx}
%\else
%\usepackage{graphicx}
%\fi

 % \ifpdf
%  \usepackage{pdfsync}
%  \if


%\title{Brief Article}
%\author{David F. Snyder}
%\author{L.G. Meredith}

%\address{Dept. of Math., Texas State University--San Marcos, San Marcos, TX 78666}
       
\pagestyle{empty}


\begin{document}

\lstset{language=[Objective]Caml,frame=shadowbox}

\input{qm2pi.front}

% section front matter (end)

\input{qm2pi.intro} 
 
% section introduction (end)

% \input{qm2pi.knotations} 

% section notation (end)

\input{qm2pi.process.calculi} 

% section concurrent_process_calculi_and_spatial_logics_ (end)
    
%\input{qm2pi.knots2pi} 

%\input{qm2pi.trefoil} 

%\input{qm2pi.mainthm} 

% subsection basic_interpretation (end)

%\input{qm2pi.rho.presentation} 
\subsection{The syntax and semantics of the notation system}\label{sub:the_syntax_and_semantics_of_the_notation_system} % (fold)

We now summarize a technical presentation of the calculus that
embodies our theory of dynamics. The typical presentation of such a
calculus follows the style of giving generators and relations on
them. The grammar, below, describing term constructors, freely
generates the set of processes, $\Proc$. This set is then quotiented
by a relation known as structural congruence and it is over this set
that the notion of dynamics is expressed. This presentation is
essentially that of \cite{MeredithR05} with the addition of
polyadicity and summation. For readability we have relegated some of
the technical subtleties to an appendix.

\subsubsection{Process grammar}\label{subsub:process_grammar}

\begin{mathpar}
  \inferrule* [lab=synchronization] {} {{M} \bc \pzero \;|\; x?F \;|\; x!C }
  \and
  \inferrule* [lab=abstraction] {} {{F} \bc (x)P}
  \and
  \inferrule* [lab=concretion] {} {{C} \bc \langle Q \rangle}
  \and
  \inferrule* [lab=process] {} {{P,Q} \bc M \;| \;P|Q \;|\; @{x}}
  \and
  \inferrule* [lab=name] {} {{x} \bc \quotep{P}}
\end{mathpar} 

Note that $\vec{x}$ (resp. $\vec{P}$) denotes a vector of names
(resp. processes) of length $|\vec{x}|$ (resp. $|\vec{P}|$). We adopt
the following useful abbreviations.

\begin{mathpar}
   x?(\vec{y}).P := x.(\vec{y})P \and  x\clift{\vec{P}} := x.\clift{\vec{P}}
   \and x!(y) := \lift{x}{\dropn{y}}
   \and \Pi_{i=0}^{n-1}P_i := P_0 | \ldots | P_{n-1}
\end{mathpar}

\subsubsection{Structural congruence}

\paragraph{Free and bound names and alpha-equivalence.} At the
core of structural equivalence is alpha-equivalence which identifies
process that are the same up to a change of variable. Formally, we
recognize the distinction between free and bound names. The free names
of a process, $\freenames{P}$, may be calculated recursively as
follows:

\begin{mathpar}
\freenames{\pzero} := \emptyset
  \and \\
  \freenames{x?(y).P} := \{ x \} \cup (\freenames{P} \setminus \{ y \})
  \and 
  \freenames{x!\langle P \rangle} := \{ x \} \cup \{ P \} 
  \and \\
  \freenames{P|Q} := \freenames{P} \cup \freenames{Q}
  \and \\
  \freenames{@{x}} := \{ x \}
\end{mathpar}

$\pi$
$\quotep{\pi}$

$\freenames{-} : \pi \to \mathcal{P}(\quotep{\pi})$

\begin{eqnarray*}
  \freenames{\pzero} & := & \emptyset \\
  \freenames{x?(y).P} & := & \{ x \} \cup (\freenames{P} \setminus \{ y \}) \\
  \freenames{x!\langle P \rangle} & := & \{ x \} \cup \{ P \} \\
  \freenames{P|Q} & := & \freenames{P} \cup \freenames{Q} \\
  \freenames{\dropn{x}} & := & \{ x \}
\end{eqnarray*}

The bound names of a process, $\boundnames{P}$, are those names occurring in $P$
that are not free. For example, in $x?(y).0$, the name $x$ is free, while $y$ is bound.

\begin{mathpar}
  \inferrule* [lab=monoidal-laws] {} { P|Q \equiv Q|P \and P|0 \equiv P \and P|(Q|R) \equiv (P|Q)|R }
\end{mathpar}

\begin{mathpar}
  \inferrule* [lab=alpha-equivalence] {} { (x)P \equiv (y)P\{y/x\} \and y \not\in \freenames{P} }
\end{mathpar}

\begin{definition}
Then two processes, $P,Q$, are alpha-equivalent if $P = Q\{\vec{y}/\vec{x}\}$ for
some $\vec{x} \in \boundnames{Q},\vec{y} \in \boundnames{P}$, where $Q\{\vec{y}/\vec{x}\}$
denotes the capture-avoiding substitution of $\vec{y}$ for $\vec{x}$ in $Q$.
\end{definition}

\begin{definition}
  The {\em structural congruence} \cite{SangiorgiWalker} , $\equiv$,
  between processes is the least congruence containing
  alpha-equivalence, satisfying the abelian monoid laws
  (associativity, commutativity and $\pzero$ as identity) for parallel
  composition $|$ and for summation $+$.
\end{definition}

\subsection{Name equivalence}

We take name equivalence, written $\nameeq$, to be the smallest
equivalence relation generated by the following rules.

\begin{mathpar}
\inferrule*[lab=Quote-drop]
{ }
{ \quotep{@{x}} \nameeq x }

\inferrule*[lab=Struct-equiv]
{ P \scong Q }
{ \quotep{P} \nameeq \quotep{Q} }
\end{mathpar}

The astute reader will have noticed that the mutual recursion of names
and processes imposes a mutual recursion on alpha-equivalence and
structural equivalence via name-equivalence. Fortunately, all of this
works out pleasantly and we may calculate in the natural way, free of
concern. The reader interested in the details is referred to the
appendix \ref{appendix:rho_details}.

\subsection{Substitution}

We use $\Proc$ for the set of processes, $\QProc$ for the set of
names, and $\id{\{}\vec{y} / \vec{x} \id{\}}$ to denote partial maps,
$s : \QProc \rightarrow \QProc$. A map, $s$ lifts, uniquely, to a map
on process terms, $\widehat{s} : \Proc \rightarrow \Proc$ by the
following equations.

\begin{mathpar}
  (0) \psubstp{Q}{P} := 0 \\
  (R \juxtap S) \psubstp{Q}{P}
  :=    
  (R)\psubstp{Q}{P} \juxtap (S) \psubstp{Q}{P} \\
  (x?(y).R) \psubstp{Q}{P}    
  :=    
  (x)\substp{Q}{P} (z)\concat( (R \psubstn{z}{y}) \psubstp{Q}{P} ) \\
  (\lift{x}{R}) \psubstp{Q}{P}  
  :=
  \lift{(x)\substp{Q}{P}}{ R \psubstp{Q}{P} } \\
%   (\dropn{x})  \psubstp{Q}{P}       
%   := 
%   \left\{ 
%     \begin{array}{ccc} 
%       \dropn{\quotep{Q}} & & x \nameeq \quotep{P} \\
%       \dropn{x} & & otherwise \\
%     \end{array}
%   \right. 
  (\dropn{x})  \psubstp{Q}{P}       
  := 
  \left\{ 
    \begin{array}{ccc} 
      Q & & x \nameeq \quotep{P} \\
      \dropn{x} & & otherwise \\
    \end{array}
  \right.
\end{mathpar}
 

where

\begin{eqnarray}
  (x)\id{\{} \lpquote Q \rpquote / \lpquote P \rpquote \id{\}}            = 
  \left\{ 
    \begin{array}{ccc}
      \lpquote Q \rpquote & & x \nameeq \lpquote P \rpquote \\
      x & & otherwise \\
    \end{array}
  \right. \nonumber
\end{eqnarray}

and $z$ is chosen distinct from $\quotep{P}$, $\quotep{Q}$, the free
names in $Q$, and all the names in $R$. Our $\alpha$-equivalence will
be built in the standard way from this substitution.

\begin{remark}\label{rem:no_self_referential_names}
  One consequence of these definitions is that $\forall P. \quotep{P}
  \not\in \freenames{P}$.
\end{remark}

\subsection{ Dynamic quote: an example }

Anticipating something of what's to come, consider applying the
substitution, $\widehat{\id{\{}u / z \id{\}}}$, to the following pair
of processes, $\lift{w}{y!(z)}$ and $w[ \lpquote y!(z) \rpquote ]$.

\begin{eqnarray}
	\lift{w}{y!(z)}\widehat{\id{\{}u / z \id{\}}}
		& = &
		\lift{w}{y!(u)} \nonumber\\
	w[ \lpquote y!(z) \rpquote ] \widehat{ \id{\{}u / z \id{\}} }
		& = &
		w[ \lpquote y!(z) \rpquote ] \nonumber
\end{eqnarray}

Because the body of the process between quotes is impervious to
substitution, we get radically different answers. In fact, by
examining the first process in an input context,
e.g. $x?(z).\lift{w}{y!(z)}$, we see that the process under the lift
operator may be shaped by prefixed inputs binding a name inside it. In
this sense, the lift operator will be seen as a way to dynamically
construct processes before reifying them as names.

Finally equipped with these standard features we can present the
dynamics of the calculus.

\subsubsection{Operational semantics} 

Finally, we introduce the computational dynamics. What marks these
algebras as distinct from other more traditionally studied algebraic
structures, e.g. vector spaces or polynomial rings, is the manner in
which dynamics is captured. In traditional structures, dynamics is typically
expressed through morphisms between such structures, as in linear maps
between vector spaces or morphisms between rings. In algebras
associated with the semantics of computation, the dynamics is
expressed as part of the algebraic structure itself, through a
reduction reduction relation typically denoted by $\red$. Below, we
give a recursive presentation of this relation for the calculus used
in the encoding.

$\red \subseteq \pi \times \pi$
$\red : \pi \to \mathcal{P}(\pi)$

\begin{mathpar}
  \inferrule* [lab=Comm] { \textsf{match}( x_{src}, x_{trgt} ) } { x_{trgt}?(y)P \; | \; x_{src}!\langle {Q} \rangle \red P\{\quotep{Q}/y}\} }
  \and \\
  \inferrule* [lab=Par] {{P} \red {P}'} {{{P} | {Q}} \red {{P}' | {Q}}}
  \and
  \inferrule* [lab=Equiv]{{{P} \scong {P}'} \andalso {{P}' \red {Q}'} \andalso {{Q}' \scong {Q}}}{{P} \red {Q}}
\end{mathpar}

\begin{eqnarray*}
  match_{\equiv} (\quotep{P},\quotep{Q}) & := & P \equiv Q \\
  match_{\dagger}(\quotep{P},\quotep{Q}) & := & \forall R. P|Q \red^{*} R => R \red^{*} 0 \\
  match_{K}(\quotep{P},\quotep{Q}) & := & K \mbox{ for some context } K
\end{eqnarray*}

$u?(x)P | u!\langle Q \rangle \red P\{\quotep{Q}/x\}$

%We write $\wred$ for $\red^*$, and $P\red$ if $\exists Q $ such that $ P \red Q$.
We write $P\red$ if $\exists Q $ such that $ P \red Q$ and $P\not\red$, otherwise.

\section{Replication}

As mentioned before, it is known that replication (and hence
recursion) can be implemented in a higher-order process algebra
\cite{SangiorgiWalker}. As our first example of calculation with the
machinery thus far presented we give the construction explicitly in
the {\rhoc}.

\begin{eqnarray}
	D_{x} & := & \prefix{x}{y}{(\binpar{\outputp{x}{y}}{@{y}})} \nonumber\\
	\bangp_{x}{P} & := & \binpar{{x}!\langle{\binpar{D_{x}}{P}}\rangle}{D_{x}} \nonumber
\end{eqnarray}

\begin{eqnarray}
	\bangp_{x}{P} & & \nonumber\\
	=
	& {x}!\langle{(\prefix{x}{y}{(\outputp{x}{y} | @{y})) | P}}\rangle 
	      | \prefix{x}{y}{(\outputp{x}{y} | @{y})} & \nonumber\\
	\red
	& (\outputp{x}{y} | @{y})\substn{\quotep{(\prefix{x}{y}{(@{y} | \outputp{x}{y})) | P}}}{y} & \nonumber\\
	=
	& \outputp{x}{\quotep{(\prefix{x}{y}{(\outputp{x}{y} | @{y})) | P}}}
	  | {(\prefix{x}{y}{(\outputp{x}{y} | @{y})) | P}} & \nonumber\\
	\red
	& \ldots & \nonumber\\
	\red^*
	& P | P | \ldots & \nonumber
\end{eqnarray}

Of course, this encoding, as an implementation, runs away, unfolding
$\bangp{P}$ eagerly. A lazier and more implementable replication
operator, restricted to input-guarded processes, may be obtained as follows.

\begin{eqnarray}
\bangp{\prefix{u}{v}{P}} 
	:= 
	\binpar{\lift{x}{\prefix{u}{v}{(\binpar{D(x)}{P})}}}{D(x)} \nonumber
\end{eqnarray}

\begin{remark}
  Note that the lazier definition still does not deal with summation
  or mixed summation (i.e. sums over input and output). The reader is
  invited to construct definitions of replication that deal with these
  features. 

  Further, the definitions are parameterized in a name, $x$. Can you,
  gentle reader, make a definition that eliminates this parameter and
  guarantees no accidental interaction between the replication
  machinery and the process being replicated -- i.e. no accidental
  sharing of names used by the process to get its work done and the
  name(s) used by the replication to effect copying. This latter
  revision of the definition of replication is crucial to obtaining
  the expected identity $!!P \sim !P$.
\end{remark}

\begin{remark}\label{rem:paradoxical_combinator}
  The reader familiar with the lambda calculus will have noticed the
  similarity between $D$ and the paradoxical combinator.

  [Ed. note: the existence of this seems to suggest we have to be more
  restrictive on the set of processes and names we admit if we are to
  support no-cloning.]
\end{remark}

\subsubsection{Bisimulation}

The computational dynamics gives rise to another kind of equivalence,
the equivalence of computational behavior. As previously mentioned
this is typically captured \emph{via} some form of bisimulation.

% The notion we use in this paper is weak barbed bisimulation
% \cite{milner91polyadicpi}.

The notion we use in this paper is derived from weak barbed
bisimulation \cite{milner91polyadicpi}. 

\begin{definition}
An \emph{observation relation}, $\downarrow_{\mathcal N}$, over a set
of names, $\mathcal N$, is the smallest relation satisfying the rules
below.

\infrule[Out-barb]{y \in {\mathcal N}, \; x \nameeq y}
		  {\outputp{x}{v} \downarrow_{\mathcal N} x}
\infrule[Par-barb]{\mbox{$P\downarrow_{\mathcal N} x$ or $Q\downarrow_{\mathcal N} x$}}
		  {\binpar{P}{Q} \downarrow_{\mathcal N} x}

We write $P \Downarrow_{\mathcal N} x$ if there is $Q$ such that 
$P \wred Q$ and $Q \downarrow_{\mathcal N} x$.
\end{definition}

\begin{definition}
%\label{def.bbisim}
An  ${\mathcal N}$-\emph{barbed bisimulation} over a set of names, ${\mathcal N}$, is a symmetric binary relation 
${\mathcal S}_{\mathcal N}$ between agents such that $P\rel{S}_{\mathcal N}Q$ implies:
\begin{enumerate}
\item If $P \red P'$ then $Q \wred Q'$ and $P'\rel{S}_{\mathcal N} Q'$.
\item If $P\downarrow_{\mathcal N} x$, then $Q\Downarrow_{\mathcal N} x$.
\end{enumerate}
$P$ is ${\mathcal N}$-barbed bisimilar to $Q$, written
$P \wbbisim_{\mathcal N} Q$, if $P \rel{S}_{\mathcal N} Q$ for some ${\mathcal N}$-barbed bisimulation ${\mathcal S}_{\mathcal N}$.
\end{definition}

$\mathcal{R} \subseteq \pi \times \pi$

$P \mathcal{R} Q => \forall P'. P \red P' \Rightarrow \exists Q'. Q \red Q', P' \mathcal{R} Q'$

$P \vdash x \Rightarrow Q \vdash x$

\begin{mathpar}
  \inferrule*[lab=Out-barb]{x \nameeq y}{{y}!\langle{Q}\rangle \vdash x}
  \and
  \inferrule*[lab=Par-barb]{\mbox{$P\vdash x$ or $Q\vdash x$}}{\binpar{P}{Q} \vdash x}
\end{mathpar}

\subsubsection{Contexts}

One of the principle advantages of computational calculi like the
$\pi$-calculus is a well-defined notion of context,
contextual-equivalence and a correlation between
contextual-equivalence and notions of bisimulation. The notion of
context allows the decomposition of a process into (sub-)process and
its syntactic environment, its context. Thus, a context may be
thought of as a process with a ``hole'' (written $\Box$) in it. The
application of a context $M$ to a process $P$, written $M[P]$, is
tantamount to filling the hole in $M$ with $P$. In this paper we do
not need the full weight of this theory, but do make use of the notion
of context in the proof the main theorem. 

\begin{mathpar}
  \inferrule* [lab=summation] {} {{M_{M},M_{N}} \bc \Box \;|\; x.M_{A} \;|\; M_{M}+M_{N}}
  \and
  \inferrule* [lab=agent] {} {{M_{A}} \bc (\vec{x})M_{P} \;| \; \clift{P_0,\ldots,M_{P},\ldots,P_N}}
  \and \\
  \inferrule* [lab=process] {} {{M_{P}} \bc M_{N} \;| \;P|M_{P} }
\end{mathpar} 

\begin{mathpar}
  \inferrule* [lab=sychronization] {} {M_{N} \bc \Box \;|\; x?M_{F} \;|\; x!M_{C}}
  \and
  \inferrule* [lab=abstraction] {} {{M_{F}} \bc (x)M_{P} }
  \and
  \inferrule* [lab=concretion] {} {{M_{C}} \bc \langle M_{P} \rangle }
  \and \\
  \inferrule* [lab=process] {} {{M_{P}} \bc M_{N} \;| \;P|M_{P} }
\end{mathpar}

\begin{definition}[contextual application] Given a context $M$, and
  process $P$, we define the \emph{contextual application}, $M[P] :=
  M\{P/\Box\}$. That is, the contextual application of M to P is the
  substitution of $P$ for $\Box$ in $M$.
\end{definition}

$\meaningof{-} : L \to \mathcal{P}(\pi)$

\begin{mathpar}
  \inferrule* [lab=collection] {} {\meaningof{true} = \pi, \and \meaningof{~E} = \pi \setminus \meaningof{E}, \and \meaningof{E_{1} \& E_{2}} = \meaningof{E_{1}} \cap \meaningof{E_{2}}}
\end{mathpar}

\begin{mathpar}
  \inferrule* [lab=structure] {} {\meaningof{0} = \{ P \in \pi | P \equiv 0 \}, \and \\ \meaningof{E_1 | E_2} = \{ P \in \pi | P \equiv P_{1} | P_{2}, P_{1} \in \meaningof{E_{1}}, P_{2} \in \meaningof{E_2}\} }
\end{mathpar}

\begin{mathpar}
 \inferrule* [lab=behavior] {} {\meaningof{\langle a?b \rangle E} = \{ P \in \pi | P \equiv Q | u?(y)P', \\ \and \\\\ \and \\ \;\;\; u \in \meaningof{a}, \forall z.P'\{z/y\} \in \meaningof{E\{z/b\}}\}, \and \\ \meaningof{a!E} = \{ P \in \pi | P \equiv Q | x!\langle P' \rangle, x \in \meaningof{a} P' \in \meaningof{E}\} }
\end{mathpar}

\begin{mathpar}
 \inferrule* [lab=nominal] {} {\meaningof{\quotep{E}} = \{ \quotep{P} \in \quotep{\pi} | P \in \meaningof{E} \}, \and \meaningof{\quotep{P}} = \{ \quotep{Q} \in \quotep{\pi} | P \equiv Q \} \and \\ \meaningof{@\quotep{E}} = \{ P \in \pi | P \equiv @x, x \in \meaningof{E} \}}
\end{mathpar}

\begin{eqnarray*}
  \\
  \meaningof{-} : TS \to ST
\end{eqnarray*}

\begin{eqnarray*}
  \\
  L : TS \to ST
\end{eqnarray*}

\begin{eqnarray*}
  \\
  P \models E \iff P \in \meaningof{E}
\end{eqnarray*}

\begin{eqnarray*}
  P \approx_{L} Q \iff \forall E \in L. P \models E \iff Q \models E
\end{eqnarray*}

\begin{eqnarray*}
  P \approx_{K} Q
\end{eqnarray*}

\begin{eqnarray*}
  P \approx Q
\end{eqnarray*}

$\approx_{K} = \approx = \approx_{L}$

\subsubsection{Contextual duality}

Note that contexts extend the quotation operation to a family of
operations from processes to names. Given a context, $M$, we can
define a \emph{nominal context}, $\quotep{M}$ by $\quotep{M}[P] :=
\quotep{M[P]}$. To foreshadow what is to come we observe that these
operations enjoy a duality with processes very much like the duality
between vectors and maps from vectors to scalars.

Further, because the calculus is essentially higher-order, we have a
correspondence between contexts and processes. More specifically,
given a name $x$ and a context $M$ we can construct $M^{*}_{x}$ such
that 

\begin{mathpar}
  M^{*}_{x} | \lift{x}{P} \red M[P]
\end{mathpar}

namely,

\begin{mathpar}
  M^{*}_{x} := x?(u).M[\dropn{u}]
\end{mathpar}

The dependence of $M^{*}_{x}$ on a name makes it an abstraction, 

\begin{mathpar}
  M^{*} := (x)x?(u).M[\dropn{u}]
\end{mathpar}

\subsection{Additional notation}

It will sometimes be convenient to denote the process a name
quotes. We already have the notation $x = \quotep{P}$, but it will be
convenient to introduce an alternate notation, $\procn{x}$, when we
want to emphasize the connection to the use of the name. Note that, by
virtue of name equivalence, $\quotep{\procn{x}} \nameeq x$; so, the
notation is consistent with previous definitions.

Further, because names have structure it is possible to effect
substitutions on the basis of that structure. This means we need to
upgrade our notation for substitutions, which we accomplish by
adapting comprehension notation. Thus,

\begin{mathpar}
  P\{ y / x : x \in S \}
\end{mathpar}

is interpreted to mean the process derived from P by replacing (in a
capture-avoiding manner) each occurrence of $x$ in $S$ by $y$. For example,

\begin{mathpar}
  P\{ \quotep{\procn{x}|\procn{x}} / x : x \in \freenames{P} \}
\end{mathpar}

will replace each (occurrence) of a free name $x$ in $P$ by
$\quotep{\procn{x}|\procn{x}}$.

Also, we will avail ourselves of the notation $x^{L}$ and $x^{R}$ to
denote injections of a name into disjoint copies of the name
space. There are numerous ways to accomplish this. One example can be
found in \cite{MeredithR05}. This notation overloads to vectors of
names: $\vec{x}^{\pi} := (x_{i}^{\pi} \; : \; 0 \leq i < |\vec{x}| )$ where $\pi \in \{L,R\}$.

We also use $P^{\Box} := P|\Box$.

In \cite{MeredithR05} an interpretation of the new operator is
given. It turns out that there are several possible interpretations
all enjoying the requisite algebraic properties of the operator (see
\cite{milner91polyadicpi}). We will therefore make liberal use of
$(\nu\; \vec{x})P$.

% subsection the_syntax_and_semantics_of_the_notation_system (end)   

\input{qm2pi.qmops} 

\input{qm2pi.sterngerlach} 

\input{qm2pi.metric} 

% section concurrent_process_calculi (end)

%\input{qm2pi.proofsketch}

% section proof sketch (end)

%\input{qm2pi.slviaknots} 

% section spatial logic via knots (end)

\input{qm2pi.conclusion}

% section conclusion (end)

%\input{qm2pi.dtcodes} 

% section wiring algorithm (end)

\input{qm2pi.ack} 

% section acknowledgments (end)

\newpage


\bibliographystyle{plain}   
\bibliography{../../biblios/main.bib}

\input{qm2pi.rhodetails}

\end{document}

 

%\documentclass[12pt]{llncs}
%\documentclass{jktr}

\usepackage[pdftex]{hyperref}                   
\usepackage {listings}
\usepackage {mathpartir}
\usepackage{bcprules}
%\usepackage{listings}
                       
\usepackage{graphicx} 
%\usepackage[margins=2.5cm,nohead,nofoot]{geometry}
%\usepackage{geometry}
\usepackage{amsfonts}
\usepackage{amstext}
\usepackage{latexsym}
\usepackage{amssymb}
\usepackage{color}


%\include{myPreamble}
\include{qm2pi.local} 

%\ifpdf
%\usepackage[pdftex]{graphicx}
%\else
%\usepackage{graphicx}
%\fi

 % \ifpdf
%  \usepackage{pdfsync}
%  \if


%\title{Brief Article}
%\author{David F. Snyder}
%\author{L.G. Meredith}

%\address{Dept. of Math., Texas State University--San Marcos, San Marcos, TX 78666}
       
\pagestyle{empty}


\begin{document}

\lstset{language=[Objective]Caml,frame=shadowbox}

\input{qm2pi.front}

% section front matter (end)

\input{qm2pi.intro} 
 
% section introduction (end)

% \input{qm2pi.knotations} 

% section notation (end)

\input{qm2pi.process.calculi} 

% section concurrent_process_calculi_and_spatial_logics_ (end)
    
%\input{qm2pi.knots2pi} 

%\input{qm2pi.trefoil} 

%\input{qm2pi.mainthm} 

% subsection basic_interpretation (end)

%\input{qm2pi.rho.presentation} 
\subsection{The syntax and semantics of the notation system}\label{sub:the_syntax_and_semantics_of_the_notation_system} % (fold)

We now summarize a technical presentation of the calculus that
embodies our theory of dynamics. The typical presentation of such a
calculus follows the style of giving generators and relations on
them. The grammar, below, describing term constructors, freely
generates the set of processes, $\Proc$. This set is then quotiented
by a relation known as structural congruence and it is over this set
that the notion of dynamics is expressed. This presentation is
essentially that of \cite{MeredithR05} with the addition of
polyadicity and summation. For readability we have relegated some of
the technical subtleties to an appendix.

\subsubsection{Process grammar}\label{subsub:process_grammar}

\begin{mathpar}
  \inferrule* [lab=synchronization] {} {{M} \bc \pzero \;|\; x?F \;|\; x!C }
  \and
  \inferrule* [lab=abstraction] {} {{F} \bc (x)P}
  \and
  \inferrule* [lab=concretion] {} {{C} \bc \langle Q \rangle}
  \and
  \inferrule* [lab=process] {} {{P,Q} \bc M \;| \;P|Q \;|\; @{x}}
  \and
  \inferrule* [lab=name] {} {{x} \bc \quotep{P}}
\end{mathpar} 

Note that $\vec{x}$ (resp. $\vec{P}$) denotes a vector of names
(resp. processes) of length $|\vec{x}|$ (resp. $|\vec{P}|$). We adopt
the following useful abbreviations.

\begin{mathpar}
   x?(\vec{y}).P := x.(\vec{y})P \and  x\clift{\vec{P}} := x.\clift{\vec{P}}
   \and x!(y) := \lift{x}{\dropn{y}}
   \and \Pi_{i=0}^{n-1}P_i := P_0 | \ldots | P_{n-1}
\end{mathpar}

\subsubsection{Structural congruence}

\paragraph{Free and bound names and alpha-equivalence.} At the
core of structural equivalence is alpha-equivalence which identifies
process that are the same up to a change of variable. Formally, we
recognize the distinction between free and bound names. The free names
of a process, $\freenames{P}$, may be calculated recursively as
follows:

\begin{mathpar}
\freenames{\pzero} := \emptyset
  \and \\
  \freenames{x?(y).P} := \{ x \} \cup (\freenames{P} \setminus \{ y \})
  \and 
  \freenames{x!\langle P \rangle} := \{ x \} \cup \{ P \} 
  \and \\
  \freenames{P|Q} := \freenames{P} \cup \freenames{Q}
  \and \\
  \freenames{@{x}} := \{ x \}
\end{mathpar}

$\pi$
$\quotep{\pi}$

$\freenames{-} : \pi \to \mathcal{P}(\quotep{\pi})$

\begin{eqnarray*}
  \freenames{\pzero} & := & \emptyset \\
  \freenames{x?(y).P} & := & \{ x \} \cup (\freenames{P} \setminus \{ y \}) \\
  \freenames{x!\langle P \rangle} & := & \{ x \} \cup \{ P \} \\
  \freenames{P|Q} & := & \freenames{P} \cup \freenames{Q} \\
  \freenames{\dropn{x}} & := & \{ x \}
\end{eqnarray*}

The bound names of a process, $\boundnames{P}$, are those names occurring in $P$
that are not free. For example, in $x?(y).0$, the name $x$ is free, while $y$ is bound.

\begin{mathpar}
  \inferrule* [lab=monoidal-laws] {} { P|Q \equiv Q|P \and P|0 \equiv P \and P|(Q|R) \equiv (P|Q)|R }
\end{mathpar}

\begin{mathpar}
  \inferrule* [lab=alpha-equivalence] {} { (x)P \equiv (y)P\{y/x\} \and y \not\in \freenames{P} }
\end{mathpar}

\begin{definition}
Then two processes, $P,Q$, are alpha-equivalent if $P = Q\{\vec{y}/\vec{x}\}$ for
some $\vec{x} \in \boundnames{Q},\vec{y} \in \boundnames{P}$, where $Q\{\vec{y}/\vec{x}\}$
denotes the capture-avoiding substitution of $\vec{y}$ for $\vec{x}$ in $Q$.
\end{definition}

\begin{definition}
  The {\em structural congruence} \cite{SangiorgiWalker} , $\equiv$,
  between processes is the least congruence containing
  alpha-equivalence, satisfying the abelian monoid laws
  (associativity, commutativity and $\pzero$ as identity) for parallel
  composition $|$ and for summation $+$.
\end{definition}

\subsection{Name equivalence}

We take name equivalence, written $\nameeq$, to be the smallest
equivalence relation generated by the following rules.

\begin{mathpar}
\inferrule*[lab=Quote-drop]
{ }
{ \quotep{@{x}} \nameeq x }

\inferrule*[lab=Struct-equiv]
{ P \scong Q }
{ \quotep{P} \nameeq \quotep{Q} }
\end{mathpar}

The astute reader will have noticed that the mutual recursion of names
and processes imposes a mutual recursion on alpha-equivalence and
structural equivalence via name-equivalence. Fortunately, all of this
works out pleasantly and we may calculate in the natural way, free of
concern. The reader interested in the details is referred to the
appendix \ref{appendix:rho_details}.

\subsection{Substitution}

We use $\Proc$ for the set of processes, $\QProc$ for the set of
names, and $\id{\{}\vec{y} / \vec{x} \id{\}}$ to denote partial maps,
$s : \QProc \rightarrow \QProc$. A map, $s$ lifts, uniquely, to a map
on process terms, $\widehat{s} : \Proc \rightarrow \Proc$ by the
following equations.

\begin{mathpar}
  (0) \psubstp{Q}{P} := 0 \\
  (R \juxtap S) \psubstp{Q}{P}
  :=    
  (R)\psubstp{Q}{P} \juxtap (S) \psubstp{Q}{P} \\
  (x?(y).R) \psubstp{Q}{P}    
  :=    
  (x)\substp{Q}{P} (z)\concat( (R \psubstn{z}{y}) \psubstp{Q}{P} ) \\
  (\lift{x}{R}) \psubstp{Q}{P}  
  :=
  \lift{(x)\substp{Q}{P}}{ R \psubstp{Q}{P} } \\
%   (\dropn{x})  \psubstp{Q}{P}       
%   := 
%   \left\{ 
%     \begin{array}{ccc} 
%       \dropn{\quotep{Q}} & & x \nameeq \quotep{P} \\
%       \dropn{x} & & otherwise \\
%     \end{array}
%   \right. 
  (\dropn{x})  \psubstp{Q}{P}       
  := 
  \left\{ 
    \begin{array}{ccc} 
      Q & & x \nameeq \quotep{P} \\
      \dropn{x} & & otherwise \\
    \end{array}
  \right.
\end{mathpar}
 

where

\begin{eqnarray}
  (x)\id{\{} \lpquote Q \rpquote / \lpquote P \rpquote \id{\}}            = 
  \left\{ 
    \begin{array}{ccc}
      \lpquote Q \rpquote & & x \nameeq \lpquote P \rpquote \\
      x & & otherwise \\
    \end{array}
  \right. \nonumber
\end{eqnarray}

and $z$ is chosen distinct from $\quotep{P}$, $\quotep{Q}$, the free
names in $Q$, and all the names in $R$. Our $\alpha$-equivalence will
be built in the standard way from this substitution.

\begin{remark}\label{rem:no_self_referential_names}
  One consequence of these definitions is that $\forall P. \quotep{P}
  \not\in \freenames{P}$.
\end{remark}

\subsection{ Dynamic quote: an example }

Anticipating something of what's to come, consider applying the
substitution, $\widehat{\id{\{}u / z \id{\}}}$, to the following pair
of processes, $\lift{w}{y!(z)}$ and $w[ \lpquote y!(z) \rpquote ]$.

\begin{eqnarray}
	\lift{w}{y!(z)}\widehat{\id{\{}u / z \id{\}}}
		& = &
		\lift{w}{y!(u)} \nonumber\\
	w[ \lpquote y!(z) \rpquote ] \widehat{ \id{\{}u / z \id{\}} }
		& = &
		w[ \lpquote y!(z) \rpquote ] \nonumber
\end{eqnarray}

Because the body of the process between quotes is impervious to
substitution, we get radically different answers. In fact, by
examining the first process in an input context,
e.g. $x?(z).\lift{w}{y!(z)}$, we see that the process under the lift
operator may be shaped by prefixed inputs binding a name inside it. In
this sense, the lift operator will be seen as a way to dynamically
construct processes before reifying them as names.

Finally equipped with these standard features we can present the
dynamics of the calculus.

\subsubsection{Operational semantics} 

Finally, we introduce the computational dynamics. What marks these
algebras as distinct from other more traditionally studied algebraic
structures, e.g. vector spaces or polynomial rings, is the manner in
which dynamics is captured. In traditional structures, dynamics is typically
expressed through morphisms between such structures, as in linear maps
between vector spaces or morphisms between rings. In algebras
associated with the semantics of computation, the dynamics is
expressed as part of the algebraic structure itself, through a
reduction reduction relation typically denoted by $\red$. Below, we
give a recursive presentation of this relation for the calculus used
in the encoding.

$\red \subseteq \pi \times \pi$
$\red : \pi \to \mathcal{P}(\pi)$

\begin{mathpar}
  \inferrule* [lab=Comm] { \textsf{match}( x_{src}, x_{trgt} ) } { x_{trgt}?(y)P \; | \; x_{src}!\langle {Q} \rangle \red P\{\quotep{Q}/y}\} }
  \and \\
  \inferrule* [lab=Par] {{P} \red {P}'} {{{P} | {Q}} \red {{P}' | {Q}}}
  \and
  \inferrule* [lab=Equiv]{{{P} \scong {P}'} \andalso {{P}' \red {Q}'} \andalso {{Q}' \scong {Q}}}{{P} \red {Q}}
\end{mathpar}

\begin{eqnarray*}
  match_{\equiv} (\quotep{P},\quotep{Q}) & := & P \equiv Q \\
  match_{\dagger}(\quotep{P},\quotep{Q}) & := & \forall R. P|Q \red^{*} R => R \red^{*} 0 \\
  match_{K}(\quotep{P},\quotep{Q}) & := & K \mbox{ for some context } K
\end{eqnarray*}

$u?(x)P | u!\langle Q \rangle \red P\{\quotep{Q}/x\}$

%We write $\wred$ for $\red^*$, and $P\red$ if $\exists Q $ such that $ P \red Q$.
We write $P\red$ if $\exists Q $ such that $ P \red Q$ and $P\not\red$, otherwise.

\section{Replication}

As mentioned before, it is known that replication (and hence
recursion) can be implemented in a higher-order process algebra
\cite{SangiorgiWalker}. As our first example of calculation with the
machinery thus far presented we give the construction explicitly in
the {\rhoc}.

\begin{eqnarray}
	D_{x} & := & \prefix{x}{y}{(\binpar{\outputp{x}{y}}{@{y}})} \nonumber\\
	\bangp_{x}{P} & := & \binpar{{x}!\langle{\binpar{D_{x}}{P}}\rangle}{D_{x}} \nonumber
\end{eqnarray}

\begin{eqnarray}
	\bangp_{x}{P} & & \nonumber\\
	=
	& {x}!\langle{(\prefix{x}{y}{(\outputp{x}{y} | @{y})) | P}}\rangle 
	      | \prefix{x}{y}{(\outputp{x}{y} | @{y})} & \nonumber\\
	\red
	& (\outputp{x}{y} | @{y})\substn{\quotep{(\prefix{x}{y}{(@{y} | \outputp{x}{y})) | P}}}{y} & \nonumber\\
	=
	& \outputp{x}{\quotep{(\prefix{x}{y}{(\outputp{x}{y} | @{y})) | P}}}
	  | {(\prefix{x}{y}{(\outputp{x}{y} | @{y})) | P}} & \nonumber\\
	\red
	& \ldots & \nonumber\\
	\red^*
	& P | P | \ldots & \nonumber
\end{eqnarray}

Of course, this encoding, as an implementation, runs away, unfolding
$\bangp{P}$ eagerly. A lazier and more implementable replication
operator, restricted to input-guarded processes, may be obtained as follows.

\begin{eqnarray}
\bangp{\prefix{u}{v}{P}} 
	:= 
	\binpar{\lift{x}{\prefix{u}{v}{(\binpar{D(x)}{P})}}}{D(x)} \nonumber
\end{eqnarray}

\begin{remark}
  Note that the lazier definition still does not deal with summation
  or mixed summation (i.e. sums over input and output). The reader is
  invited to construct definitions of replication that deal with these
  features. 

  Further, the definitions are parameterized in a name, $x$. Can you,
  gentle reader, make a definition that eliminates this parameter and
  guarantees no accidental interaction between the replication
  machinery and the process being replicated -- i.e. no accidental
  sharing of names used by the process to get its work done and the
  name(s) used by the replication to effect copying. This latter
  revision of the definition of replication is crucial to obtaining
  the expected identity $!!P \sim !P$.
\end{remark}

\begin{remark}\label{rem:paradoxical_combinator}
  The reader familiar with the lambda calculus will have noticed the
  similarity between $D$ and the paradoxical combinator.

  [Ed. note: the existence of this seems to suggest we have to be more
  restrictive on the set of processes and names we admit if we are to
  support no-cloning.]
\end{remark}

\subsubsection{Bisimulation}

The computational dynamics gives rise to another kind of equivalence,
the equivalence of computational behavior. As previously mentioned
this is typically captured \emph{via} some form of bisimulation.

% The notion we use in this paper is weak barbed bisimulation
% \cite{milner91polyadicpi}.

The notion we use in this paper is derived from weak barbed
bisimulation \cite{milner91polyadicpi}. 

\begin{definition}
An \emph{observation relation}, $\downarrow_{\mathcal N}$, over a set
of names, $\mathcal N$, is the smallest relation satisfying the rules
below.

\infrule[Out-barb]{y \in {\mathcal N}, \; x \nameeq y}
		  {\outputp{x}{v} \downarrow_{\mathcal N} x}
\infrule[Par-barb]{\mbox{$P\downarrow_{\mathcal N} x$ or $Q\downarrow_{\mathcal N} x$}}
		  {\binpar{P}{Q} \downarrow_{\mathcal N} x}

We write $P \Downarrow_{\mathcal N} x$ if there is $Q$ such that 
$P \wred Q$ and $Q \downarrow_{\mathcal N} x$.
\end{definition}

\begin{definition}
%\label{def.bbisim}
An  ${\mathcal N}$-\emph{barbed bisimulation} over a set of names, ${\mathcal N}$, is a symmetric binary relation 
${\mathcal S}_{\mathcal N}$ between agents such that $P\rel{S}_{\mathcal N}Q$ implies:
\begin{enumerate}
\item If $P \red P'$ then $Q \wred Q'$ and $P'\rel{S}_{\mathcal N} Q'$.
\item If $P\downarrow_{\mathcal N} x$, then $Q\Downarrow_{\mathcal N} x$.
\end{enumerate}
$P$ is ${\mathcal N}$-barbed bisimilar to $Q$, written
$P \wbbisim_{\mathcal N} Q$, if $P \rel{S}_{\mathcal N} Q$ for some ${\mathcal N}$-barbed bisimulation ${\mathcal S}_{\mathcal N}$.
\end{definition}

$\mathcal{R} \subseteq \pi \times \pi$

$P \mathcal{R} Q => \forall P'. P \red P' \Rightarrow \exists Q'. Q \red Q', P' \mathcal{R} Q'$

$P \vdash x \Rightarrow Q \vdash x$

\begin{mathpar}
  \inferrule*[lab=Out-barb]{x \nameeq y}{{y}!\langle{Q}\rangle \vdash x}
  \and
  \inferrule*[lab=Par-barb]{\mbox{$P\vdash x$ or $Q\vdash x$}}{\binpar{P}{Q} \vdash x}
\end{mathpar}

\subsubsection{Contexts}

One of the principle advantages of computational calculi like the
$\pi$-calculus is a well-defined notion of context,
contextual-equivalence and a correlation between
contextual-equivalence and notions of bisimulation. The notion of
context allows the decomposition of a process into (sub-)process and
its syntactic environment, its context. Thus, a context may be
thought of as a process with a ``hole'' (written $\Box$) in it. The
application of a context $M$ to a process $P$, written $M[P]$, is
tantamount to filling the hole in $M$ with $P$. In this paper we do
not need the full weight of this theory, but do make use of the notion
of context in the proof the main theorem. 

\begin{mathpar}
  \inferrule* [lab=summation] {} {{M_{M},M_{N}} \bc \Box \;|\; x.M_{A} \;|\; M_{M}+M_{N}}
  \and
  \inferrule* [lab=agent] {} {{M_{A}} \bc (\vec{x})M_{P} \;| \; \clift{P_0,\ldots,M_{P},\ldots,P_N}}
  \and \\
  \inferrule* [lab=process] {} {{M_{P}} \bc M_{N} \;| \;P|M_{P} }
\end{mathpar} 

\begin{mathpar}
  \inferrule* [lab=sychronization] {} {M_{N} \bc \Box \;|\; x?M_{F} \;|\; x!M_{C}}
  \and
  \inferrule* [lab=abstraction] {} {{M_{F}} \bc (x)M_{P} }
  \and
  \inferrule* [lab=concretion] {} {{M_{C}} \bc \langle M_{P} \rangle }
  \and \\
  \inferrule* [lab=process] {} {{M_{P}} \bc M_{N} \;| \;P|M_{P} }
\end{mathpar}

\begin{definition}[contextual application] Given a context $M$, and
  process $P$, we define the \emph{contextual application}, $M[P] :=
  M\{P/\Box\}$. That is, the contextual application of M to P is the
  substitution of $P$ for $\Box$ in $M$.
\end{definition}

$\meaningof{-} : L \to \mathcal{P}(\pi)$

\begin{mathpar}
  \inferrule* [lab=collection] {} {\meaningof{true} = \pi, \and \meaningof{~E} = \pi \setminus \meaningof{E}, \and \meaningof{E_{1} \& E_{2}} = \meaningof{E_{1}} \cap \meaningof{E_{2}}}
\end{mathpar}

\begin{mathpar}
  \inferrule* [lab=structure] {} {\meaningof{0} = \{ P \in \pi | P \equiv 0 \}, \and \\ \meaningof{E_1 | E_2} = \{ P \in \pi | P \equiv P_{1} | P_{2}, P_{1} \in \meaningof{E_{1}}, P_{2} \in \meaningof{E_2}\} }
\end{mathpar}

\begin{mathpar}
 \inferrule* [lab=behavior] {} {\meaningof{\langle a?b \rangle E} = \{ P \in \pi | P \equiv Q | u?(y)P', \\ \and \\\\ \and \\ \;\;\; u \in \meaningof{a}, \forall z.P'\{z/y\} \in \meaningof{E\{z/b\}}\}, \and \\ \meaningof{a!E} = \{ P \in \pi | P \equiv Q | x!\langle P' \rangle, x \in \meaningof{a} P' \in \meaningof{E}\} }
\end{mathpar}

\begin{mathpar}
 \inferrule* [lab=nominal] {} {\meaningof{\quotep{E}} = \{ \quotep{P} \in \quotep{\pi} | P \in \meaningof{E} \}, \and \meaningof{\quotep{P}} = \{ \quotep{Q} \in \quotep{\pi} | P \equiv Q \} \and \\ \meaningof{@\quotep{E}} = \{ P \in \pi | P \equiv @x, x \in \meaningof{E} \}}
\end{mathpar}

\begin{eqnarray*}
  \\
  \meaningof{-} : TS \to ST
\end{eqnarray*}

\begin{eqnarray*}
  \\
  L : TS \to ST
\end{eqnarray*}

\begin{eqnarray*}
  \\
  P \models E \iff P \in \meaningof{E}
\end{eqnarray*}

\begin{eqnarray*}
  P \approx_{L} Q \iff \forall E \in L. P \models E \iff Q \models E
\end{eqnarray*}

\begin{eqnarray*}
  P \approx_{K} Q
\end{eqnarray*}

\begin{eqnarray*}
  P \approx Q
\end{eqnarray*}

$\approx_{K} = \approx = \approx_{L}$

\subsubsection{Contextual duality}

Note that contexts extend the quotation operation to a family of
operations from processes to names. Given a context, $M$, we can
define a \emph{nominal context}, $\quotep{M}$ by $\quotep{M}[P] :=
\quotep{M[P]}$. To foreshadow what is to come we observe that these
operations enjoy a duality with processes very much like the duality
between vectors and maps from vectors to scalars.

Further, because the calculus is essentially higher-order, we have a
correspondence between contexts and processes. More specifically,
given a name $x$ and a context $M$ we can construct $M^{*}_{x}$ such
that 

\begin{mathpar}
  M^{*}_{x} | \lift{x}{P} \red M[P]
\end{mathpar}

namely,

\begin{mathpar}
  M^{*}_{x} := x?(u).M[\dropn{u}]
\end{mathpar}

The dependence of $M^{*}_{x}$ on a name makes it an abstraction, 

\begin{mathpar}
  M^{*} := (x)x?(u).M[\dropn{u}]
\end{mathpar}

\subsection{Additional notation}

It will sometimes be convenient to denote the process a name
quotes. We already have the notation $x = \quotep{P}$, but it will be
convenient to introduce an alternate notation, $\procn{x}$, when we
want to emphasize the connection to the use of the name. Note that, by
virtue of name equivalence, $\quotep{\procn{x}} \nameeq x$; so, the
notation is consistent with previous definitions.

Further, because names have structure it is possible to effect
substitutions on the basis of that structure. This means we need to
upgrade our notation for substitutions, which we accomplish by
adapting comprehension notation. Thus,

\begin{mathpar}
  P\{ y / x : x \in S \}
\end{mathpar}

is interpreted to mean the process derived from P by replacing (in a
capture-avoiding manner) each occurrence of $x$ in $S$ by $y$. For example,

\begin{mathpar}
  P\{ \quotep{\procn{x}|\procn{x}} / x : x \in \freenames{P} \}
\end{mathpar}

will replace each (occurrence) of a free name $x$ in $P$ by
$\quotep{\procn{x}|\procn{x}}$.

Also, we will avail ourselves of the notation $x^{L}$ and $x^{R}$ to
denote injections of a name into disjoint copies of the name
space. There are numerous ways to accomplish this. One example can be
found in \cite{MeredithR05}. This notation overloads to vectors of
names: $\vec{x}^{\pi} := (x_{i}^{\pi} \; : \; 0 \leq i < |\vec{x}| )$ where $\pi \in \{L,R\}$.

We also use $P^{\Box} := P|\Box$.

In \cite{MeredithR05} an interpretation of the new operator is
given. It turns out that there are several possible interpretations
all enjoying the requisite algebraic properties of the operator (see
\cite{milner91polyadicpi}). We will therefore make liberal use of
$(\nu\; \vec{x})P$.

% subsection the_syntax_and_semantics_of_the_notation_system (end)   

\input{qm2pi.qmops} 

\input{qm2pi.sterngerlach} 

\input{qm2pi.metric} 

% section concurrent_process_calculi (end)

%\input{qm2pi.proofsketch}

% section proof sketch (end)

%\input{qm2pi.slviaknots} 

% section spatial logic via knots (end)

\input{qm2pi.conclusion}

% section conclusion (end)

%\input{qm2pi.dtcodes} 

% section wiring algorithm (end)

\input{qm2pi.ack} 

% section acknowledgments (end)

\newpage


\bibliographystyle{plain}   
\bibliography{../../biblios/main.bib}

\input{qm2pi.rhodetails}

\end{document}

 

%\documentclass[12pt]{llncs}
%\documentclass{jktr}

\usepackage[pdftex]{hyperref}                   
\usepackage {listings}
\usepackage {mathpartir}
\usepackage{bcprules}
%\usepackage{listings}
                       
\usepackage{graphicx} 
%\usepackage[margins=2.5cm,nohead,nofoot]{geometry}
%\usepackage{geometry}
\usepackage{amsfonts}
\usepackage{amstext}
\usepackage{latexsym}
\usepackage{amssymb}
\usepackage{color}


%\include{myPreamble}
\include{qm2pi.local} 

%\ifpdf
%\usepackage[pdftex]{graphicx}
%\else
%\usepackage{graphicx}
%\fi

 % \ifpdf
%  \usepackage{pdfsync}
%  \if


%\title{Brief Article}
%\author{David F. Snyder}
%\author{L.G. Meredith}

%\address{Dept. of Math., Texas State University--San Marcos, San Marcos, TX 78666}
       
\pagestyle{empty}


\begin{document}

\lstset{language=[Objective]Caml,frame=shadowbox}

\input{qm2pi.front}

% section front matter (end)

\input{qm2pi.intro} 
 
% section introduction (end)

% \input{qm2pi.knotations} 

% section notation (end)

\input{qm2pi.process.calculi} 

% section concurrent_process_calculi_and_spatial_logics_ (end)
    
%\input{qm2pi.knots2pi} 

%\input{qm2pi.trefoil} 

%\input{qm2pi.mainthm} 

% subsection basic_interpretation (end)

%\input{qm2pi.rho.presentation} 
\subsection{The syntax and semantics of the notation system}\label{sub:the_syntax_and_semantics_of_the_notation_system} % (fold)

We now summarize a technical presentation of the calculus that
embodies our theory of dynamics. The typical presentation of such a
calculus follows the style of giving generators and relations on
them. The grammar, below, describing term constructors, freely
generates the set of processes, $\Proc$. This set is then quotiented
by a relation known as structural congruence and it is over this set
that the notion of dynamics is expressed. This presentation is
essentially that of \cite{MeredithR05} with the addition of
polyadicity and summation. For readability we have relegated some of
the technical subtleties to an appendix.

\subsubsection{Process grammar}\label{subsub:process_grammar}

\begin{mathpar}
  \inferrule* [lab=synchronization] {} {{M} \bc \pzero \;|\; x?F \;|\; x!C }
  \and
  \inferrule* [lab=abstraction] {} {{F} \bc (x)P}
  \and
  \inferrule* [lab=concretion] {} {{C} \bc \langle Q \rangle}
  \and
  \inferrule* [lab=process] {} {{P,Q} \bc M \;| \;P|Q \;|\; @{x}}
  \and
  \inferrule* [lab=name] {} {{x} \bc \quotep{P}}
\end{mathpar} 

Note that $\vec{x}$ (resp. $\vec{P}$) denotes a vector of names
(resp. processes) of length $|\vec{x}|$ (resp. $|\vec{P}|$). We adopt
the following useful abbreviations.

\begin{mathpar}
   x?(\vec{y}).P := x.(\vec{y})P \and  x\clift{\vec{P}} := x.\clift{\vec{P}}
   \and x!(y) := \lift{x}{\dropn{y}}
   \and \Pi_{i=0}^{n-1}P_i := P_0 | \ldots | P_{n-1}
\end{mathpar}

\subsubsection{Structural congruence}

\paragraph{Free and bound names and alpha-equivalence.} At the
core of structural equivalence is alpha-equivalence which identifies
process that are the same up to a change of variable. Formally, we
recognize the distinction between free and bound names. The free names
of a process, $\freenames{P}$, may be calculated recursively as
follows:

\begin{mathpar}
\freenames{\pzero} := \emptyset
  \and \\
  \freenames{x?(y).P} := \{ x \} \cup (\freenames{P} \setminus \{ y \})
  \and 
  \freenames{x!\langle P \rangle} := \{ x \} \cup \{ P \} 
  \and \\
  \freenames{P|Q} := \freenames{P} \cup \freenames{Q}
  \and \\
  \freenames{@{x}} := \{ x \}
\end{mathpar}

$\pi$
$\quotep{\pi}$

$\freenames{-} : \pi \to \mathcal{P}(\quotep{\pi})$

\begin{eqnarray*}
  \freenames{\pzero} & := & \emptyset \\
  \freenames{x?(y).P} & := & \{ x \} \cup (\freenames{P} \setminus \{ y \}) \\
  \freenames{x!\langle P \rangle} & := & \{ x \} \cup \{ P \} \\
  \freenames{P|Q} & := & \freenames{P} \cup \freenames{Q} \\
  \freenames{\dropn{x}} & := & \{ x \}
\end{eqnarray*}

The bound names of a process, $\boundnames{P}$, are those names occurring in $P$
that are not free. For example, in $x?(y).0$, the name $x$ is free, while $y$ is bound.

\begin{mathpar}
  \inferrule* [lab=monoidal-laws] {} { P|Q \equiv Q|P \and P|0 \equiv P \and P|(Q|R) \equiv (P|Q)|R }
\end{mathpar}

\begin{mathpar}
  \inferrule* [lab=alpha-equivalence] {} { (x)P \equiv (y)P\{y/x\} \and y \not\in \freenames{P} }
\end{mathpar}

\begin{definition}
Then two processes, $P,Q$, are alpha-equivalent if $P = Q\{\vec{y}/\vec{x}\}$ for
some $\vec{x} \in \boundnames{Q},\vec{y} \in \boundnames{P}$, where $Q\{\vec{y}/\vec{x}\}$
denotes the capture-avoiding substitution of $\vec{y}$ for $\vec{x}$ in $Q$.
\end{definition}

\begin{definition}
  The {\em structural congruence} \cite{SangiorgiWalker} , $\equiv$,
  between processes is the least congruence containing
  alpha-equivalence, satisfying the abelian monoid laws
  (associativity, commutativity and $\pzero$ as identity) for parallel
  composition $|$ and for summation $+$.
\end{definition}

\subsection{Name equivalence}

We take name equivalence, written $\nameeq$, to be the smallest
equivalence relation generated by the following rules.

\begin{mathpar}
\inferrule*[lab=Quote-drop]
{ }
{ \quotep{@{x}} \nameeq x }

\inferrule*[lab=Struct-equiv]
{ P \scong Q }
{ \quotep{P} \nameeq \quotep{Q} }
\end{mathpar}

The astute reader will have noticed that the mutual recursion of names
and processes imposes a mutual recursion on alpha-equivalence and
structural equivalence via name-equivalence. Fortunately, all of this
works out pleasantly and we may calculate in the natural way, free of
concern. The reader interested in the details is referred to the
appendix \ref{appendix:rho_details}.

\subsection{Substitution}

We use $\Proc$ for the set of processes, $\QProc$ for the set of
names, and $\id{\{}\vec{y} / \vec{x} \id{\}}$ to denote partial maps,
$s : \QProc \rightarrow \QProc$. A map, $s$ lifts, uniquely, to a map
on process terms, $\widehat{s} : \Proc \rightarrow \Proc$ by the
following equations.

\begin{mathpar}
  (0) \psubstp{Q}{P} := 0 \\
  (R \juxtap S) \psubstp{Q}{P}
  :=    
  (R)\psubstp{Q}{P} \juxtap (S) \psubstp{Q}{P} \\
  (x?(y).R) \psubstp{Q}{P}    
  :=    
  (x)\substp{Q}{P} (z)\concat( (R \psubstn{z}{y}) \psubstp{Q}{P} ) \\
  (\lift{x}{R}) \psubstp{Q}{P}  
  :=
  \lift{(x)\substp{Q}{P}}{ R \psubstp{Q}{P} } \\
%   (\dropn{x})  \psubstp{Q}{P}       
%   := 
%   \left\{ 
%     \begin{array}{ccc} 
%       \dropn{\quotep{Q}} & & x \nameeq \quotep{P} \\
%       \dropn{x} & & otherwise \\
%     \end{array}
%   \right. 
  (\dropn{x})  \psubstp{Q}{P}       
  := 
  \left\{ 
    \begin{array}{ccc} 
      Q & & x \nameeq \quotep{P} \\
      \dropn{x} & & otherwise \\
    \end{array}
  \right.
\end{mathpar}
 

where

\begin{eqnarray}
  (x)\id{\{} \lpquote Q \rpquote / \lpquote P \rpquote \id{\}}            = 
  \left\{ 
    \begin{array}{ccc}
      \lpquote Q \rpquote & & x \nameeq \lpquote P \rpquote \\
      x & & otherwise \\
    \end{array}
  \right. \nonumber
\end{eqnarray}

and $z$ is chosen distinct from $\quotep{P}$, $\quotep{Q}$, the free
names in $Q$, and all the names in $R$. Our $\alpha$-equivalence will
be built in the standard way from this substitution.

\begin{remark}\label{rem:no_self_referential_names}
  One consequence of these definitions is that $\forall P. \quotep{P}
  \not\in \freenames{P}$.
\end{remark}

\subsection{ Dynamic quote: an example }

Anticipating something of what's to come, consider applying the
substitution, $\widehat{\id{\{}u / z \id{\}}}$, to the following pair
of processes, $\lift{w}{y!(z)}$ and $w[ \lpquote y!(z) \rpquote ]$.

\begin{eqnarray}
	\lift{w}{y!(z)}\widehat{\id{\{}u / z \id{\}}}
		& = &
		\lift{w}{y!(u)} \nonumber\\
	w[ \lpquote y!(z) \rpquote ] \widehat{ \id{\{}u / z \id{\}} }
		& = &
		w[ \lpquote y!(z) \rpquote ] \nonumber
\end{eqnarray}

Because the body of the process between quotes is impervious to
substitution, we get radically different answers. In fact, by
examining the first process in an input context,
e.g. $x?(z).\lift{w}{y!(z)}$, we see that the process under the lift
operator may be shaped by prefixed inputs binding a name inside it. In
this sense, the lift operator will be seen as a way to dynamically
construct processes before reifying them as names.

Finally equipped with these standard features we can present the
dynamics of the calculus.

\subsubsection{Operational semantics} 

Finally, we introduce the computational dynamics. What marks these
algebras as distinct from other more traditionally studied algebraic
structures, e.g. vector spaces or polynomial rings, is the manner in
which dynamics is captured. In traditional structures, dynamics is typically
expressed through morphisms between such structures, as in linear maps
between vector spaces or morphisms between rings. In algebras
associated with the semantics of computation, the dynamics is
expressed as part of the algebraic structure itself, through a
reduction reduction relation typically denoted by $\red$. Below, we
give a recursive presentation of this relation for the calculus used
in the encoding.

$\red \subseteq \pi \times \pi$
$\red : \pi \to \mathcal{P}(\pi)$

\begin{mathpar}
  \inferrule* [lab=Comm] { \textsf{match}( x_{src}, x_{trgt} ) } { x_{trgt}?(y)P \; | \; x_{src}!\langle {Q} \rangle \red P\{\quotep{Q}/y}\} }
  \and \\
  \inferrule* [lab=Par] {{P} \red {P}'} {{{P} | {Q}} \red {{P}' | {Q}}}
  \and
  \inferrule* [lab=Equiv]{{{P} \scong {P}'} \andalso {{P}' \red {Q}'} \andalso {{Q}' \scong {Q}}}{{P} \red {Q}}
\end{mathpar}

\begin{eqnarray*}
  match_{\equiv} (\quotep{P},\quotep{Q}) & := & P \equiv Q \\
  match_{\dagger}(\quotep{P},\quotep{Q}) & := & \forall R. P|Q \red^{*} R => R \red^{*} 0 \\
  match_{K}(\quotep{P},\quotep{Q}) & := & K \mbox{ for some context } K
\end{eqnarray*}

$u?(x)P | u!\langle Q \rangle \red P\{\quotep{Q}/x\}$

%We write $\wred$ for $\red^*$, and $P\red$ if $\exists Q $ such that $ P \red Q$.
We write $P\red$ if $\exists Q $ such that $ P \red Q$ and $P\not\red$, otherwise.

\section{Replication}

As mentioned before, it is known that replication (and hence
recursion) can be implemented in a higher-order process algebra
\cite{SangiorgiWalker}. As our first example of calculation with the
machinery thus far presented we give the construction explicitly in
the {\rhoc}.

\begin{eqnarray}
	D_{x} & := & \prefix{x}{y}{(\binpar{\outputp{x}{y}}{@{y}})} \nonumber\\
	\bangp_{x}{P} & := & \binpar{{x}!\langle{\binpar{D_{x}}{P}}\rangle}{D_{x}} \nonumber
\end{eqnarray}

\begin{eqnarray}
	\bangp_{x}{P} & & \nonumber\\
	=
	& {x}!\langle{(\prefix{x}{y}{(\outputp{x}{y} | @{y})) | P}}\rangle 
	      | \prefix{x}{y}{(\outputp{x}{y} | @{y})} & \nonumber\\
	\red
	& (\outputp{x}{y} | @{y})\substn{\quotep{(\prefix{x}{y}{(@{y} | \outputp{x}{y})) | P}}}{y} & \nonumber\\
	=
	& \outputp{x}{\quotep{(\prefix{x}{y}{(\outputp{x}{y} | @{y})) | P}}}
	  | {(\prefix{x}{y}{(\outputp{x}{y} | @{y})) | P}} & \nonumber\\
	\red
	& \ldots & \nonumber\\
	\red^*
	& P | P | \ldots & \nonumber
\end{eqnarray}

Of course, this encoding, as an implementation, runs away, unfolding
$\bangp{P}$ eagerly. A lazier and more implementable replication
operator, restricted to input-guarded processes, may be obtained as follows.

\begin{eqnarray}
\bangp{\prefix{u}{v}{P}} 
	:= 
	\binpar{\lift{x}{\prefix{u}{v}{(\binpar{D(x)}{P})}}}{D(x)} \nonumber
\end{eqnarray}

\begin{remark}
  Note that the lazier definition still does not deal with summation
  or mixed summation (i.e. sums over input and output). The reader is
  invited to construct definitions of replication that deal with these
  features. 

  Further, the definitions are parameterized in a name, $x$. Can you,
  gentle reader, make a definition that eliminates this parameter and
  guarantees no accidental interaction between the replication
  machinery and the process being replicated -- i.e. no accidental
  sharing of names used by the process to get its work done and the
  name(s) used by the replication to effect copying. This latter
  revision of the definition of replication is crucial to obtaining
  the expected identity $!!P \sim !P$.
\end{remark}

\begin{remark}\label{rem:paradoxical_combinator}
  The reader familiar with the lambda calculus will have noticed the
  similarity between $D$ and the paradoxical combinator.

  [Ed. note: the existence of this seems to suggest we have to be more
  restrictive on the set of processes and names we admit if we are to
  support no-cloning.]
\end{remark}

\subsubsection{Bisimulation}

The computational dynamics gives rise to another kind of equivalence,
the equivalence of computational behavior. As previously mentioned
this is typically captured \emph{via} some form of bisimulation.

% The notion we use in this paper is weak barbed bisimulation
% \cite{milner91polyadicpi}.

The notion we use in this paper is derived from weak barbed
bisimulation \cite{milner91polyadicpi}. 

\begin{definition}
An \emph{observation relation}, $\downarrow_{\mathcal N}$, over a set
of names, $\mathcal N$, is the smallest relation satisfying the rules
below.

\infrule[Out-barb]{y \in {\mathcal N}, \; x \nameeq y}
		  {\outputp{x}{v} \downarrow_{\mathcal N} x}
\infrule[Par-barb]{\mbox{$P\downarrow_{\mathcal N} x$ or $Q\downarrow_{\mathcal N} x$}}
		  {\binpar{P}{Q} \downarrow_{\mathcal N} x}

We write $P \Downarrow_{\mathcal N} x$ if there is $Q$ such that 
$P \wred Q$ and $Q \downarrow_{\mathcal N} x$.
\end{definition}

\begin{definition}
%\label{def.bbisim}
An  ${\mathcal N}$-\emph{barbed bisimulation} over a set of names, ${\mathcal N}$, is a symmetric binary relation 
${\mathcal S}_{\mathcal N}$ between agents such that $P\rel{S}_{\mathcal N}Q$ implies:
\begin{enumerate}
\item If $P \red P'$ then $Q \wred Q'$ and $P'\rel{S}_{\mathcal N} Q'$.
\item If $P\downarrow_{\mathcal N} x$, then $Q\Downarrow_{\mathcal N} x$.
\end{enumerate}
$P$ is ${\mathcal N}$-barbed bisimilar to $Q$, written
$P \wbbisim_{\mathcal N} Q$, if $P \rel{S}_{\mathcal N} Q$ for some ${\mathcal N}$-barbed bisimulation ${\mathcal S}_{\mathcal N}$.
\end{definition}

$\mathcal{R} \subseteq \pi \times \pi$

$P \mathcal{R} Q => \forall P'. P \red P' \Rightarrow \exists Q'. Q \red Q', P' \mathcal{R} Q'$

$P \vdash x \Rightarrow Q \vdash x$

\begin{mathpar}
  \inferrule*[lab=Out-barb]{x \nameeq y}{{y}!\langle{Q}\rangle \vdash x}
  \and
  \inferrule*[lab=Par-barb]{\mbox{$P\vdash x$ or $Q\vdash x$}}{\binpar{P}{Q} \vdash x}
\end{mathpar}

\subsubsection{Contexts}

One of the principle advantages of computational calculi like the
$\pi$-calculus is a well-defined notion of context,
contextual-equivalence and a correlation between
contextual-equivalence and notions of bisimulation. The notion of
context allows the decomposition of a process into (sub-)process and
its syntactic environment, its context. Thus, a context may be
thought of as a process with a ``hole'' (written $\Box$) in it. The
application of a context $M$ to a process $P$, written $M[P]$, is
tantamount to filling the hole in $M$ with $P$. In this paper we do
not need the full weight of this theory, but do make use of the notion
of context in the proof the main theorem. 

\begin{mathpar}
  \inferrule* [lab=summation] {} {{M_{M},M_{N}} \bc \Box \;|\; x.M_{A} \;|\; M_{M}+M_{N}}
  \and
  \inferrule* [lab=agent] {} {{M_{A}} \bc (\vec{x})M_{P} \;| \; \clift{P_0,\ldots,M_{P},\ldots,P_N}}
  \and \\
  \inferrule* [lab=process] {} {{M_{P}} \bc M_{N} \;| \;P|M_{P} }
\end{mathpar} 

\begin{mathpar}
  \inferrule* [lab=sychronization] {} {M_{N} \bc \Box \;|\; x?M_{F} \;|\; x!M_{C}}
  \and
  \inferrule* [lab=abstraction] {} {{M_{F}} \bc (x)M_{P} }
  \and
  \inferrule* [lab=concretion] {} {{M_{C}} \bc \langle M_{P} \rangle }
  \and \\
  \inferrule* [lab=process] {} {{M_{P}} \bc M_{N} \;| \;P|M_{P} }
\end{mathpar}

\begin{definition}[contextual application] Given a context $M$, and
  process $P$, we define the \emph{contextual application}, $M[P] :=
  M\{P/\Box\}$. That is, the contextual application of M to P is the
  substitution of $P$ for $\Box$ in $M$.
\end{definition}

$\meaningof{-} : L \to \mathcal{P}(\pi)$

\begin{mathpar}
  \inferrule* [lab=collection] {} {\meaningof{true} = \pi, \and \meaningof{~E} = \pi \setminus \meaningof{E}, \and \meaningof{E_{1} \& E_{2}} = \meaningof{E_{1}} \cap \meaningof{E_{2}}}
\end{mathpar}

\begin{mathpar}
  \inferrule* [lab=structure] {} {\meaningof{0} = \{ P \in \pi | P \equiv 0 \}, \and \\ \meaningof{E_1 | E_2} = \{ P \in \pi | P \equiv P_{1} | P_{2}, P_{1} \in \meaningof{E_{1}}, P_{2} \in \meaningof{E_2}\} }
\end{mathpar}

\begin{mathpar}
 \inferrule* [lab=behavior] {} {\meaningof{\langle a?b \rangle E} = \{ P \in \pi | P \equiv Q | u?(y)P', \\ \and \\\\ \and \\ \;\;\; u \in \meaningof{a}, \forall z.P'\{z/y\} \in \meaningof{E\{z/b\}}\}, \and \\ \meaningof{a!E} = \{ P \in \pi | P \equiv Q | x!\langle P' \rangle, x \in \meaningof{a} P' \in \meaningof{E}\} }
\end{mathpar}

\begin{mathpar}
 \inferrule* [lab=nominal] {} {\meaningof{\quotep{E}} = \{ \quotep{P} \in \quotep{\pi} | P \in \meaningof{E} \}, \and \meaningof{\quotep{P}} = \{ \quotep{Q} \in \quotep{\pi} | P \equiv Q \} \and \\ \meaningof{@\quotep{E}} = \{ P \in \pi | P \equiv @x, x \in \meaningof{E} \}}
\end{mathpar}

\begin{eqnarray*}
  \\
  \meaningof{-} : TS \to ST
\end{eqnarray*}

\begin{eqnarray*}
  \\
  L : TS \to ST
\end{eqnarray*}

\begin{eqnarray*}
  \\
  P \models E \iff P \in \meaningof{E}
\end{eqnarray*}

\begin{eqnarray*}
  P \approx_{L} Q \iff \forall E \in L. P \models E \iff Q \models E
\end{eqnarray*}

\begin{eqnarray*}
  P \approx_{K} Q
\end{eqnarray*}

\begin{eqnarray*}
  P \approx Q
\end{eqnarray*}

$\approx_{K} = \approx = \approx_{L}$

\subsubsection{Contextual duality}

Note that contexts extend the quotation operation to a family of
operations from processes to names. Given a context, $M$, we can
define a \emph{nominal context}, $\quotep{M}$ by $\quotep{M}[P] :=
\quotep{M[P]}$. To foreshadow what is to come we observe that these
operations enjoy a duality with processes very much like the duality
between vectors and maps from vectors to scalars.

Further, because the calculus is essentially higher-order, we have a
correspondence between contexts and processes. More specifically,
given a name $x$ and a context $M$ we can construct $M^{*}_{x}$ such
that 

\begin{mathpar}
  M^{*}_{x} | \lift{x}{P} \red M[P]
\end{mathpar}

namely,

\begin{mathpar}
  M^{*}_{x} := x?(u).M[\dropn{u}]
\end{mathpar}

The dependence of $M^{*}_{x}$ on a name makes it an abstraction, 

\begin{mathpar}
  M^{*} := (x)x?(u).M[\dropn{u}]
\end{mathpar}

\subsection{Additional notation}

It will sometimes be convenient to denote the process a name
quotes. We already have the notation $x = \quotep{P}$, but it will be
convenient to introduce an alternate notation, $\procn{x}$, when we
want to emphasize the connection to the use of the name. Note that, by
virtue of name equivalence, $\quotep{\procn{x}} \nameeq x$; so, the
notation is consistent with previous definitions.

Further, because names have structure it is possible to effect
substitutions on the basis of that structure. This means we need to
upgrade our notation for substitutions, which we accomplish by
adapting comprehension notation. Thus,

\begin{mathpar}
  P\{ y / x : x \in S \}
\end{mathpar}

is interpreted to mean the process derived from P by replacing (in a
capture-avoiding manner) each occurrence of $x$ in $S$ by $y$. For example,

\begin{mathpar}
  P\{ \quotep{\procn{x}|\procn{x}} / x : x \in \freenames{P} \}
\end{mathpar}

will replace each (occurrence) of a free name $x$ in $P$ by
$\quotep{\procn{x}|\procn{x}}$.

Also, we will avail ourselves of the notation $x^{L}$ and $x^{R}$ to
denote injections of a name into disjoint copies of the name
space. There are numerous ways to accomplish this. One example can be
found in \cite{MeredithR05}. This notation overloads to vectors of
names: $\vec{x}^{\pi} := (x_{i}^{\pi} \; : \; 0 \leq i < |\vec{x}| )$ where $\pi \in \{L,R\}$.

We also use $P^{\Box} := P|\Box$.

In \cite{MeredithR05} an interpretation of the new operator is
given. It turns out that there are several possible interpretations
all enjoying the requisite algebraic properties of the operator (see
\cite{milner91polyadicpi}). We will therefore make liberal use of
$(\nu\; \vec{x})P$.

% subsection the_syntax_and_semantics_of_the_notation_system (end)   

\input{qm2pi.qmops} 

\input{qm2pi.sterngerlach} 

\input{qm2pi.metric} 

% section concurrent_process_calculi (end)

%\input{qm2pi.proofsketch}

% section proof sketch (end)

%\input{qm2pi.slviaknots} 

% section spatial logic via knots (end)

\input{qm2pi.conclusion}

% section conclusion (end)

%\input{qm2pi.dtcodes} 

% section wiring algorithm (end)

\input{qm2pi.ack} 

% section acknowledgments (end)

\newpage


\bibliographystyle{plain}   
\bibliography{../../biblios/main.bib}

\input{qm2pi.rhodetails}

\end{document}

 

% subsection basic_interpretation (end)

%\input{qm2pi.rho.presentation} 
\subsection{The syntax and semantics of the notation system}\label{sub:the_syntax_and_semantics_of_the_notation_system} % (fold)

We now summarize a technical presentation of the calculus that
embodies our theory of dynamics. The typical presentation of such a
calculus follows the style of giving generators and relations on
them. The grammar, below, describing term constructors, freely
generates the set of processes, $\Proc$. This set is then quotiented
by a relation known as structural congruence and it is over this set
that the notion of dynamics is expressed. This presentation is
essentially that of \cite{MeredithR05} with the addition of
polyadicity and summation. For readability we have relegated some of
the technical subtleties to an appendix.

\subsubsection{Process grammar}\label{subsub:process_grammar}

\begin{mathpar}
  \inferrule* [lab=synchronization] {} {{M} \bc \pzero \;|\; x?F \;|\; x!C }
  \and
  \inferrule* [lab=abstraction] {} {{F} \bc (x)P}
  \and
  \inferrule* [lab=concretion] {} {{C} \bc \langle Q \rangle}
  \and
  \inferrule* [lab=process] {} {{P,Q} \bc M \;| \;P|Q \;|\; @{x}}
  \and
  \inferrule* [lab=name] {} {{x} \bc \quotep{P}}
\end{mathpar} 

Note that $\vec{x}$ (resp. $\vec{P}$) denotes a vector of names
(resp. processes) of length $|\vec{x}|$ (resp. $|\vec{P}|$). We adopt
the following useful abbreviations.

\begin{mathpar}
   x?(\vec{y}).P := x.(\vec{y})P \and  x\clift{\vec{P}} := x.\clift{\vec{P}}
   \and x!(y) := \lift{x}{\dropn{y}}
   \and \Pi_{i=0}^{n-1}P_i := P_0 | \ldots | P_{n-1}
\end{mathpar}

\subsubsection{Structural congruence}

\paragraph{Free and bound names and alpha-equivalence.} At the
core of structural equivalence is alpha-equivalence which identifies
process that are the same up to a change of variable. Formally, we
recognize the distinction between free and bound names. The free names
of a process, $\freenames{P}$, may be calculated recursively as
follows:

\begin{mathpar}
\freenames{\pzero} := \emptyset
  \and \\
  \freenames{x?(y).P} := \{ x \} \cup (\freenames{P} \setminus \{ y \})
  \and 
  \freenames{x!\langle P \rangle} := \{ x \} \cup \{ P \} 
  \and \\
  \freenames{P|Q} := \freenames{P} \cup \freenames{Q}
  \and \\
  \freenames{@{x}} := \{ x \}
\end{mathpar}

$\pi$
$\quotep{\pi}$

$\freenames{-} : \pi \to \mathcal{P}(\quotep{\pi})$

\begin{eqnarray*}
  \freenames{\pzero} & := & \emptyset \\
  \freenames{x?(y).P} & := & \{ x \} \cup (\freenames{P} \setminus \{ y \}) \\
  \freenames{x!\langle P \rangle} & := & \{ x \} \cup \{ P \} \\
  \freenames{P|Q} & := & \freenames{P} \cup \freenames{Q} \\
  \freenames{\dropn{x}} & := & \{ x \}
\end{eqnarray*}

The bound names of a process, $\boundnames{P}$, are those names occurring in $P$
that are not free. For example, in $x?(y).0$, the name $x$ is free, while $y$ is bound.

\begin{mathpar}
  \inferrule* [lab=monoidal-laws] {} { P|Q \equiv Q|P \and P|0 \equiv P \and P|(Q|R) \equiv (P|Q)|R }
\end{mathpar}

\begin{mathpar}
  \inferrule* [lab=alpha-equivalence] {} { (x)P \equiv (y)P\{y/x\} \and y \not\in \freenames{P} }
\end{mathpar}

\begin{definition}
Then two processes, $P,Q$, are alpha-equivalent if $P = Q\{\vec{y}/\vec{x}\}$ for
some $\vec{x} \in \boundnames{Q},\vec{y} \in \boundnames{P}$, where $Q\{\vec{y}/\vec{x}\}$
denotes the capture-avoiding substitution of $\vec{y}$ for $\vec{x}$ in $Q$.
\end{definition}

\begin{definition}
  The {\em structural congruence} \cite{SangiorgiWalker} , $\equiv$,
  between processes is the least congruence containing
  alpha-equivalence, satisfying the abelian monoid laws
  (associativity, commutativity and $\pzero$ as identity) for parallel
  composition $|$ and for summation $+$.
\end{definition}

\subsection{Name equivalence}

We take name equivalence, written $\nameeq$, to be the smallest
equivalence relation generated by the following rules.

\begin{mathpar}
\inferrule*[lab=Quote-drop]
{ }
{ \quotep{@{x}} \nameeq x }

\inferrule*[lab=Struct-equiv]
{ P \scong Q }
{ \quotep{P} \nameeq \quotep{Q} }
\end{mathpar}

The astute reader will have noticed that the mutual recursion of names
and processes imposes a mutual recursion on alpha-equivalence and
structural equivalence via name-equivalence. Fortunately, all of this
works out pleasantly and we may calculate in the natural way, free of
concern. The reader interested in the details is referred to the
appendix \ref{appendix:rho_details}.

\subsection{Substitution}

We use $\Proc$ for the set of processes, $\QProc$ for the set of
names, and $\id{\{}\vec{y} / \vec{x} \id{\}}$ to denote partial maps,
$s : \QProc \rightarrow \QProc$. A map, $s$ lifts, uniquely, to a map
on process terms, $\widehat{s} : \Proc \rightarrow \Proc$ by the
following equations.

\begin{mathpar}
  (0) \psubstp{Q}{P} := 0 \\
  (R \juxtap S) \psubstp{Q}{P}
  :=    
  (R)\psubstp{Q}{P} \juxtap (S) \psubstp{Q}{P} \\
  (x?(y).R) \psubstp{Q}{P}    
  :=    
  (x)\substp{Q}{P} (z)\concat( (R \psubstn{z}{y}) \psubstp{Q}{P} ) \\
  (\lift{x}{R}) \psubstp{Q}{P}  
  :=
  \lift{(x)\substp{Q}{P}}{ R \psubstp{Q}{P} } \\
%   (\dropn{x})  \psubstp{Q}{P}       
%   := 
%   \left\{ 
%     \begin{array}{ccc} 
%       \dropn{\quotep{Q}} & & x \nameeq \quotep{P} \\
%       \dropn{x} & & otherwise \\
%     \end{array}
%   \right. 
  (\dropn{x})  \psubstp{Q}{P}       
  := 
  \left\{ 
    \begin{array}{ccc} 
      Q & & x \nameeq \quotep{P} \\
      \dropn{x} & & otherwise \\
    \end{array}
  \right.
\end{mathpar}
 

where

\begin{eqnarray}
  (x)\id{\{} \lpquote Q \rpquote / \lpquote P \rpquote \id{\}}            = 
  \left\{ 
    \begin{array}{ccc}
      \lpquote Q \rpquote & & x \nameeq \lpquote P \rpquote \\
      x & & otherwise \\
    \end{array}
  \right. \nonumber
\end{eqnarray}

and $z$ is chosen distinct from $\quotep{P}$, $\quotep{Q}$, the free
names in $Q$, and all the names in $R$. Our $\alpha$-equivalence will
be built in the standard way from this substitution.

\begin{remark}\label{rem:no_self_referential_names}
  One consequence of these definitions is that $\forall P. \quotep{P}
  \not\in \freenames{P}$.
\end{remark}

\subsection{ Dynamic quote: an example }

Anticipating something of what's to come, consider applying the
substitution, $\widehat{\id{\{}u / z \id{\}}}$, to the following pair
of processes, $\lift{w}{y!(z)}$ and $w[ \lpquote y!(z) \rpquote ]$.

\begin{eqnarray}
	\lift{w}{y!(z)}\widehat{\id{\{}u / z \id{\}}}
		& = &
		\lift{w}{y!(u)} \nonumber\\
	w[ \lpquote y!(z) \rpquote ] \widehat{ \id{\{}u / z \id{\}} }
		& = &
		w[ \lpquote y!(z) \rpquote ] \nonumber
\end{eqnarray}

Because the body of the process between quotes is impervious to
substitution, we get radically different answers. In fact, by
examining the first process in an input context,
e.g. $x?(z).\lift{w}{y!(z)}$, we see that the process under the lift
operator may be shaped by prefixed inputs binding a name inside it. In
this sense, the lift operator will be seen as a way to dynamically
construct processes before reifying them as names.

Finally equipped with these standard features we can present the
dynamics of the calculus.

\subsubsection{Operational semantics} 

Finally, we introduce the computational dynamics. What marks these
algebras as distinct from other more traditionally studied algebraic
structures, e.g. vector spaces or polynomial rings, is the manner in
which dynamics is captured. In traditional structures, dynamics is typically
expressed through morphisms between such structures, as in linear maps
between vector spaces or morphisms between rings. In algebras
associated with the semantics of computation, the dynamics is
expressed as part of the algebraic structure itself, through a
reduction reduction relation typically denoted by $\red$. Below, we
give a recursive presentation of this relation for the calculus used
in the encoding.

$\red \subseteq \pi \times \pi$
$\red : \pi \to \mathcal{P}(\pi)$

\begin{mathpar}
  \inferrule* [lab=Comm] { \textsf{match}( x_{src}, x_{trgt} ) } { x_{trgt}?(y)P \; | \; x_{src}!\langle {Q} \rangle \red P\{\quotep{Q}/y}\} }
  \and \\
  \inferrule* [lab=Par] {{P} \red {P}'} {{{P} | {Q}} \red {{P}' | {Q}}}
  \and
  \inferrule* [lab=Equiv]{{{P} \scong {P}'} \andalso {{P}' \red {Q}'} \andalso {{Q}' \scong {Q}}}{{P} \red {Q}}
\end{mathpar}

\begin{eqnarray*}
  match_{\equiv} (\quotep{P},\quotep{Q}) & := & P \equiv Q \\
  match_{\dagger}(\quotep{P},\quotep{Q}) & := & \forall R. P|Q \red^{*} R => R \red^{*} 0 \\
  match_{K}(\quotep{P},\quotep{Q}) & := & K \mbox{ for some context } K
\end{eqnarray*}

$u?(x)P | u!\langle Q \rangle \red P\{\quotep{Q}/x\}$

%We write $\wred$ for $\red^*$, and $P\red$ if $\exists Q $ such that $ P \red Q$.
We write $P\red$ if $\exists Q $ such that $ P \red Q$ and $P\not\red$, otherwise.

\section{Replication}

As mentioned before, it is known that replication (and hence
recursion) can be implemented in a higher-order process algebra
\cite{SangiorgiWalker}. As our first example of calculation with the
machinery thus far presented we give the construction explicitly in
the {\rhoc}.

\begin{eqnarray}
	D_{x} & := & \prefix{x}{y}{(\binpar{\outputp{x}{y}}{@{y}})} \nonumber\\
	\bangp_{x}{P} & := & \binpar{{x}!\langle{\binpar{D_{x}}{P}}\rangle}{D_{x}} \nonumber
\end{eqnarray}

\begin{eqnarray}
	\bangp_{x}{P} & & \nonumber\\
	=
	& {x}!\langle{(\prefix{x}{y}{(\outputp{x}{y} | @{y})) | P}}\rangle 
	      | \prefix{x}{y}{(\outputp{x}{y} | @{y})} & \nonumber\\
	\red
	& (\outputp{x}{y} | @{y})\substn{\quotep{(\prefix{x}{y}{(@{y} | \outputp{x}{y})) | P}}}{y} & \nonumber\\
	=
	& \outputp{x}{\quotep{(\prefix{x}{y}{(\outputp{x}{y} | @{y})) | P}}}
	  | {(\prefix{x}{y}{(\outputp{x}{y} | @{y})) | P}} & \nonumber\\
	\red
	& \ldots & \nonumber\\
	\red^*
	& P | P | \ldots & \nonumber
\end{eqnarray}

Of course, this encoding, as an implementation, runs away, unfolding
$\bangp{P}$ eagerly. A lazier and more implementable replication
operator, restricted to input-guarded processes, may be obtained as follows.

\begin{eqnarray}
\bangp{\prefix{u}{v}{P}} 
	:= 
	\binpar{\lift{x}{\prefix{u}{v}{(\binpar{D(x)}{P})}}}{D(x)} \nonumber
\end{eqnarray}

\begin{remark}
  Note that the lazier definition still does not deal with summation
  or mixed summation (i.e. sums over input and output). The reader is
  invited to construct definitions of replication that deal with these
  features. 

  Further, the definitions are parameterized in a name, $x$. Can you,
  gentle reader, make a definition that eliminates this parameter and
  guarantees no accidental interaction between the replication
  machinery and the process being replicated -- i.e. no accidental
  sharing of names used by the process to get its work done and the
  name(s) used by the replication to effect copying. This latter
  revision of the definition of replication is crucial to obtaining
  the expected identity $!!P \sim !P$.
\end{remark}

\begin{remark}\label{rem:paradoxical_combinator}
  The reader familiar with the lambda calculus will have noticed the
  similarity between $D$ and the paradoxical combinator.

  [Ed. note: the existence of this seems to suggest we have to be more
  restrictive on the set of processes and names we admit if we are to
  support no-cloning.]
\end{remark}

\subsubsection{Bisimulation}

The computational dynamics gives rise to another kind of equivalence,
the equivalence of computational behavior. As previously mentioned
this is typically captured \emph{via} some form of bisimulation.

% The notion we use in this paper is weak barbed bisimulation
% \cite{milner91polyadicpi}.

The notion we use in this paper is derived from weak barbed
bisimulation \cite{milner91polyadicpi}. 

\begin{definition}
An \emph{observation relation}, $\downarrow_{\mathcal N}$, over a set
of names, $\mathcal N$, is the smallest relation satisfying the rules
below.

\infrule[Out-barb]{y \in {\mathcal N}, \; x \nameeq y}
		  {\outputp{x}{v} \downarrow_{\mathcal N} x}
\infrule[Par-barb]{\mbox{$P\downarrow_{\mathcal N} x$ or $Q\downarrow_{\mathcal N} x$}}
		  {\binpar{P}{Q} \downarrow_{\mathcal N} x}

We write $P \Downarrow_{\mathcal N} x$ if there is $Q$ such that 
$P \wred Q$ and $Q \downarrow_{\mathcal N} x$.
\end{definition}

\begin{definition}
%\label{def.bbisim}
An  ${\mathcal N}$-\emph{barbed bisimulation} over a set of names, ${\mathcal N}$, is a symmetric binary relation 
${\mathcal S}_{\mathcal N}$ between agents such that $P\rel{S}_{\mathcal N}Q$ implies:
\begin{enumerate}
\item If $P \red P'$ then $Q \wred Q'$ and $P'\rel{S}_{\mathcal N} Q'$.
\item If $P\downarrow_{\mathcal N} x$, then $Q\Downarrow_{\mathcal N} x$.
\end{enumerate}
$P$ is ${\mathcal N}$-barbed bisimilar to $Q$, written
$P \wbbisim_{\mathcal N} Q$, if $P \rel{S}_{\mathcal N} Q$ for some ${\mathcal N}$-barbed bisimulation ${\mathcal S}_{\mathcal N}$.
\end{definition}

$\mathcal{R} \subseteq \pi \times \pi$

$P \mathcal{R} Q => \forall P'. P \red P' \Rightarrow \exists Q'. Q \red Q', P' \mathcal{R} Q'$

$P \vdash x \Rightarrow Q \vdash x$

\begin{mathpar}
  \inferrule*[lab=Out-barb]{x \nameeq y}{{y}!\langle{Q}\rangle \vdash x}
  \and
  \inferrule*[lab=Par-barb]{\mbox{$P\vdash x$ or $Q\vdash x$}}{\binpar{P}{Q} \vdash x}
\end{mathpar}

\subsubsection{Contexts}

One of the principle advantages of computational calculi like the
$\pi$-calculus is a well-defined notion of context,
contextual-equivalence and a correlation between
contextual-equivalence and notions of bisimulation. The notion of
context allows the decomposition of a process into (sub-)process and
its syntactic environment, its context. Thus, a context may be
thought of as a process with a ``hole'' (written $\Box$) in it. The
application of a context $M$ to a process $P$, written $M[P]$, is
tantamount to filling the hole in $M$ with $P$. In this paper we do
not need the full weight of this theory, but do make use of the notion
of context in the proof the main theorem. 

\begin{mathpar}
  \inferrule* [lab=summation] {} {{M_{M},M_{N}} \bc \Box \;|\; x.M_{A} \;|\; M_{M}+M_{N}}
  \and
  \inferrule* [lab=agent] {} {{M_{A}} \bc (\vec{x})M_{P} \;| \; \clift{P_0,\ldots,M_{P},\ldots,P_N}}
  \and \\
  \inferrule* [lab=process] {} {{M_{P}} \bc M_{N} \;| \;P|M_{P} }
\end{mathpar} 

\begin{mathpar}
  \inferrule* [lab=sychronization] {} {M_{N} \bc \Box \;|\; x?M_{F} \;|\; x!M_{C}}
  \and
  \inferrule* [lab=abstraction] {} {{M_{F}} \bc (x)M_{P} }
  \and
  \inferrule* [lab=concretion] {} {{M_{C}} \bc \langle M_{P} \rangle }
  \and \\
  \inferrule* [lab=process] {} {{M_{P}} \bc M_{N} \;| \;P|M_{P} }
\end{mathpar}

\begin{definition}[contextual application] Given a context $M$, and
  process $P$, we define the \emph{contextual application}, $M[P] :=
  M\{P/\Box\}$. That is, the contextual application of M to P is the
  substitution of $P$ for $\Box$ in $M$.
\end{definition}

$\meaningof{-} : L \to \mathcal{P}(\pi)$

\begin{mathpar}
  \inferrule* [lab=collection] {} {\meaningof{true} = \pi, \and \meaningof{~E} = \pi \setminus \meaningof{E}, \and \meaningof{E_{1} \& E_{2}} = \meaningof{E_{1}} \cap \meaningof{E_{2}}}
\end{mathpar}

\begin{mathpar}
  \inferrule* [lab=structure] {} {\meaningof{0} = \{ P \in \pi | P \equiv 0 \}, \and \\ \meaningof{E_1 | E_2} = \{ P \in \pi | P \equiv P_{1} | P_{2}, P_{1} \in \meaningof{E_{1}}, P_{2} \in \meaningof{E_2}\} }
\end{mathpar}

\begin{mathpar}
 \inferrule* [lab=behavior] {} {\meaningof{\langle a?b \rangle E} = \{ P \in \pi | P \equiv Q | u?(y)P', \\ \and \\\\ \and \\ \;\;\; u \in \meaningof{a}, \forall z.P'\{z/y\} \in \meaningof{E\{z/b\}}\}, \and \\ \meaningof{a!E} = \{ P \in \pi | P \equiv Q | x!\langle P' \rangle, x \in \meaningof{a} P' \in \meaningof{E}\} }
\end{mathpar}

\begin{mathpar}
 \inferrule* [lab=nominal] {} {\meaningof{\quotep{E}} = \{ \quotep{P} \in \quotep{\pi} | P \in \meaningof{E} \}, \and \meaningof{\quotep{P}} = \{ \quotep{Q} \in \quotep{\pi} | P \equiv Q \} \and \\ \meaningof{@\quotep{E}} = \{ P \in \pi | P \equiv @x, x \in \meaningof{E} \}}
\end{mathpar}

\begin{eqnarray*}
  \\
  \meaningof{-} : TS \to ST
\end{eqnarray*}

\begin{eqnarray*}
  \\
  L : TS \to ST
\end{eqnarray*}

\begin{eqnarray*}
  \\
  P \models E \iff P \in \meaningof{E}
\end{eqnarray*}

\begin{eqnarray*}
  P \approx_{L} Q \iff \forall E \in L. P \models E \iff Q \models E
\end{eqnarray*}

\begin{eqnarray*}
  P \approx_{K} Q
\end{eqnarray*}

\begin{eqnarray*}
  P \approx Q
\end{eqnarray*}

$\approx_{K} = \approx = \approx_{L}$

\subsubsection{Contextual duality}

Note that contexts extend the quotation operation to a family of
operations from processes to names. Given a context, $M$, we can
define a \emph{nominal context}, $\quotep{M}$ by $\quotep{M}[P] :=
\quotep{M[P]}$. To foreshadow what is to come we observe that these
operations enjoy a duality with processes very much like the duality
between vectors and maps from vectors to scalars.

Further, because the calculus is essentially higher-order, we have a
correspondence between contexts and processes. More specifically,
given a name $x$ and a context $M$ we can construct $M^{*}_{x}$ such
that 

\begin{mathpar}
  M^{*}_{x} | \lift{x}{P} \red M[P]
\end{mathpar}

namely,

\begin{mathpar}
  M^{*}_{x} := x?(u).M[\dropn{u}]
\end{mathpar}

The dependence of $M^{*}_{x}$ on a name makes it an abstraction, 

\begin{mathpar}
  M^{*} := (x)x?(u).M[\dropn{u}]
\end{mathpar}

\subsection{Additional notation}

It will sometimes be convenient to denote the process a name
quotes. We already have the notation $x = \quotep{P}$, but it will be
convenient to introduce an alternate notation, $\procn{x}$, when we
want to emphasize the connection to the use of the name. Note that, by
virtue of name equivalence, $\quotep{\procn{x}} \nameeq x$; so, the
notation is consistent with previous definitions.

Further, because names have structure it is possible to effect
substitutions on the basis of that structure. This means we need to
upgrade our notation for substitutions, which we accomplish by
adapting comprehension notation. Thus,

\begin{mathpar}
  P\{ y / x : x \in S \}
\end{mathpar}

is interpreted to mean the process derived from P by replacing (in a
capture-avoiding manner) each occurrence of $x$ in $S$ by $y$. For example,

\begin{mathpar}
  P\{ \quotep{\procn{x}|\procn{x}} / x : x \in \freenames{P} \}
\end{mathpar}

will replace each (occurrence) of a free name $x$ in $P$ by
$\quotep{\procn{x}|\procn{x}}$.

Also, we will avail ourselves of the notation $x^{L}$ and $x^{R}$ to
denote injections of a name into disjoint copies of the name
space. There are numerous ways to accomplish this. One example can be
found in \cite{MeredithR05}. This notation overloads to vectors of
names: $\vec{x}^{\pi} := (x_{i}^{\pi} \; : \; 0 \leq i < |\vec{x}| )$ where $\pi \in \{L,R\}$.

We also use $P^{\Box} := P|\Box$.

In \cite{MeredithR05} an interpretation of the new operator is
given. It turns out that there are several possible interpretations
all enjoying the requisite algebraic properties of the operator (see
\cite{milner91polyadicpi}). We will therefore make liberal use of
$(\nu\; \vec{x})P$.

% subsection the_syntax_and_semantics_of_the_notation_system (end)   

\section{Interpretation of QM}
\subsection{Supporting definitions}
\subsubsection{Multiplication}
\begin{mathpar}
  \quotep{Q} \cdot \quotep{R} := \quotep{Q|R}
  \and \\
  \quotep{Q} \cdot P := P\{ \quotep{Q|R} / \quotep{R} : \quotep{R} \in \freenames{P} \}
\end{mathpar}

\paragraph{Discussion}
The first line needs little explanation. The second line says that
each free name of the process is replaced with the multiplication of
that name by the scalar. Multiplication of a scalar (name) by a state
(process) results in a process all the names of which have been `moved
over' by parallel composition with the process the scalar
quotes. There is a subtlety that the bound names have to be
manipulated so that multiplied names aren't accidentally
captured. There are many ways to achieve this.

\begin{remark}\label{rem:multiplication_identities}
  The reader is invited to verify that for all $x,y,z \in \QProc$ and $P \in \Proc$
  \begin{mathpar}
    x \cdot \quotep{0} \equiv x 
    \and
    x \cdot y \equiv y \cdot x
    \and
    x \cdot (y \cdot z) \equiv (x \cdot y) \cdot z
    \and \\
    \quotep{0} \cdot P \equiv P
    \and \\
    x \cdot (y \cdot P) \equiv (x \cdot y) \cdot P
    \and \\
    x \cdot (P|Q) \equiv (x \cdot P) | (x \cdot Q)
    \and \\    
  \end{mathpar}
\end{remark}

\subsubsection{Tensor product}

We define a tensor product on processes by structural induction.

\paragraph{Tensor of sums} First note that all summations, including
$\pzero$ and sequence, can be written $\Sigma_{i} x_{i}.A_{i} +
\Sigma_{j} x_{j}.C_{j}$, where we have grouped input-guarded processes
together and output-guarded processes together.

Thus, we can define the tensor product of two summations, $N_{1}\otimes N_{2}$, where

\begin{mathpar}
  N_{1} := \Sigma_{i} x_{i}.A_{i} + \Sigma_{j} x_{j}.C_{j}
  \and
  N_{2} := \Sigma_{i'} y_{i'}.B_{i'} + \Sigma_{j'} y_{j'}.D_{j'} 
\end{mathpar}

as follows.

\begin{mathpar}
  \Sigma_{i} x_{i}.A_{i} + \Sigma_{j} x_{j}.C_{j} \otimes \Sigma_{i'}
  y_{i'}.B_{i'} + \Sigma_{j'} y_{j'}.D_{j'} 
  \and \\
  := \; \Sigma_{i} \Sigma_{i'} \quotep{\stackrel{\vee}{x_{i}}| \stackrel{\vee}{y_{i'}}}.(A_{i}\otimes B_{i'}) \; | \; \Sigma_{i'} \Sigma_{i} \quotep{\stackrel{\vee}{y_{i'}}|\stackrel{\vee}{x_{i}}}.(B_{i'}\otimes A_{i})
  \and
  \;\; | \;\; \Sigma_{j} \Sigma_{j'} \quotep{\stackrel{\vee}{x_{j}}|\stackrel{\vee}{y_{j'}}}.(A_{j}\otimes B_{j'}) \; | \; \Sigma_{j'} \Sigma_{j} \quotep{\stackrel{\vee}{y_{j'}}|\stackrel{\vee}{x_{j}}}.(B_{j'}\otimes A_{j})
\end{mathpar}

\begin{remark}
  Do we need to $x^{L}$ and $y^{R}$ for this construction as well?
\end{remark}

\paragraph{Tensor of parallel compositions} Next, we distribute tensor
over par.

\begin{mathpar}
  P_{1}|P_{2} \otimes Q_{1}|Q_{2} := (P_{1} \otimes Q_{1}) | (P_{1}
  \otimes Q_{2}) | (P_{2} \otimes Q_{1}) | (P_{2} \otimes Q_{2})
\end{mathpar}

\paragraph{Tensor with dropped names} We treat tensor of a
process with a dropped name as parallel composition.

\begin{mathpar}
  P \otimes \dropn{x} := P | \dropn{x}
\end{mathpar}

\paragraph{Tensor of agents}

Finally, we need to define tensor on agents. Note that the definition
of tensor on normal products only tensors inputs with inputs and
outputs with outputs. Thus, we only have to define the operation on
``homogeneous'' pairings.

\begin{mathpar}
  (\vec{x})P \otimes (\vec{y})Q
  \and \\
  := (x_{0}^{L}|y_{0}^{R},\ldots,x_{0}^{L}|y_{n}^{R},\ldots,x_{m}^{L}|y_{0}^{R},\ldots,x_{m}^{L}|y_{n}^R)(P\{ \vec{x}^{L}/\vec{x}\} \otimes Q \{ \vec{y}^{R}/\vec{y}\})
  \and \\
  \clift{\vec{P}} \otimes \clift{\vec{Q}}
  \and \\
  := \clift{P_{0}\otimes Q_{0},\ldots,P_{0}\otimes Q_{n},\ldots,P_{m}\otimes Q_{0},\ldots,P_{m}\otimes Q_{n}}
\end{mathpar}

\begin{remark}
  Observe that arities of tensored abstractions matches arities of
  tensored concretions if the original arities matched. Note also that
  the length of the arities corresponds to the increase in dimension
  we see in ordinary vector space tensor product.
\end{remark}

\begin{remark}
  Operationally, this definition distributes the tensor down to
  components ``linked'' by summation. Tensor over summation is
  intriguing in that it mixes names. Moreover, as a consequence of the
  way it mixes names we have the identities for all $x \in \QProc$ and
  $P,Q \in \Proc$

  \begin{mathpar}
    (x \cdot P) \otimes Q \equiv x \cdot (P \otimes Q) \equiv P \otimes (x \cdot Q)
    \and
    P \otimes \pzero \equiv P
  \end{mathpar}

  that the reader is invited to verify.
\end{remark}

\subsubsection{Annihilation}
\begin{mathpar}
  P^{\perp} := \{ Q | \forall R. P|Q \red^{*} R \Rightarrow R \red^{*} \pzero \}
  \and \\
  P^{\underline{\perp}} := \Sigma_{Q \in P^{\perp}} \quotep{Q}?(y).(\dropn{y}|Q) | \Sigma_{Q \in P^{\perp}} \quotep{Q}\clift{\Box}
\end{mathpar}

\paragraph{Discussion} The reader will note that $P^{\perp}$ is a
\emph{set} of processes, while $P^{\underline{\perp}}$ is a
\emph{context}. We call the set $P^{\perp}$ the \emph{annihilators} of
$P$. The parallel composition of a process in the annihilators of $P$
with $P$ will result in a process, the state space of which has all
paths eventually leading to $\pzero$. Execution may endure loops; but
under reasonable conditions of fairness (naturally guaranteed under
most notions of bisimulation) such a composite process cannot get
stuck in such a loop and will, eventually pop out and terminate.

The context $P^{\underline{\perp}}$ is ready and willing to ``take the
$P$ out of'' the process to which it is applied. It will effectively
transmit the code of the process to which it is applied to one of the
annihilators and run the process against it.

\subsubsection{Evaluation}
We fix $M$ a domain of fully abstract interpretation with an equality
coincident with bisimulation. We take $\meaningof{\cdot} : \Proc \to
M$ to be the map interpreting processes and $\nmeaningof{\cdot} : \M
\to Proc$ to be the map running the other way. Then we define

\begin{mathpar}
  \int P := \nmeaningof{\meaningof{P}}
\end{mathpar}

\paragraph{Discussion}
There are many fully abstract interpretations of Milner's
$\pi$-calculus. Any of them can be used as a basis for interpreting
the reflective calculus here. Equipped with such a domain it is
largely a matter of grinding through to check that the Yoneda
construction for the normalization-by-evaluation program can be
extended to this setting.

\begin{remark}
  The reader is invited to verify that $\int (P^{\underline{\perp}}[P]) = 0$.
\end{remark}

\subsection{Quantum mechanics}

Table \ref{tbl:core_qm_op_defns} gives the core operational definitions

\begin{table}[htp]\label{tbl:core_qm_op_defns}
  \center{
    \fbox{
      \begin{tabular}{c|c}
        quantum mechanics & process calculus \\
        \hline
        scalar & $x := \quotep{P}$ \\
        state vector & $\state{P} := P$ \\
        dual & $\state{P}^{*} := \event{P^{\underline{\perp}}} := \quotep{P^{\underline{\perp}}}[-]$ \\
        matrix & $ \Sigma_{\alpha} \state{P_{\alpha}}x_{\alpha}\event{Q_{\alpha}}$ \\
        vector addition & $\state{P} + \state{Q} := \state{P | Q}$ \\
        tensor product & $\state{P} \otimes \state{Q} := \state{P \otimes Q}$ \\
        inner product & $\innerprod{P}{Q} := \quotep{\int P^{\underline{\perp}}[Q]}$ \\
      \end{tabular}
    }
  }
  \caption{QM - operational definitions}
\end{table}

where

\begin{mathpar}
  \prmatrix{P}{Q} := \fprmatrix{P}{\quotep{\pzero}}{Q}
  \and
  \fprmatrix{P}{x}{Q} := (\state{P},x,\event{Q})
  \and
  (\fprmatrix{P}{x}{Q})(\state{R}) := x \cdot \innerprod{Q}{R} \cdot \state{P}
  \and
  (\fprmatrix{P}{x}{Q})(\event{R}) := x \cdot \innerprod{R}{P} \cdot \event{Q}
\end{mathpar}

\paragraph{Discussion}
As promised: vectors (aka states) are represented as processes; duals
as contextual duals; inner product definition should be compared with
standard inner product definition for ....

\begin{remark}
  Assuming $\int (P^{\underline{\perp}}[P]) = 0$, the reader is
  invited to verify that $(\fprmatrix{P}{x}{P})(\state{P}) = x \cdot \state{P}$.
\end{remark}

\begin{remark}
  The reader is invited to verify that $\innerprod{P}{Q}$ could
  equally well have been written $\quotep{\int \stackrel{\vee}{x}}$
  where $x = \event{P^{\underline{\perp}}}(Q)$.

  One of the motivations for this remark is that there is another way
  to factor these operations. We could package up evaluation in the dual:

  \begin{mathpar}
    \state{P}^{*} := \event{\int P^{\underline{\perp}}} := \quotep{\int P^{\underline{\perp}}}[-]
  \end{mathpar}

  and then have inner product defined by
  
  \begin{mathpar}
    \innerprod{P}{Q} := \event{P}(Q)
  \end{mathpar}

  Hopefully, experience with the calculations will provide guidance on
  the best factoring.
\end{remark}

\begin{remark}
  Assuming $\int (P^{\underline{\perp}}[P]) = 0$, the reader is
  invited to verify that $\forall P,Q. (\prmatrix{0}{Q})(\state{0}) =
  \state{0}$ and dually $(\prmatrix{P}{0})(\event{0}) = \event{0}$.
\end{remark}

\begin{remark}
  i'm a little worried that i don't (yet) have proper support for
  complex conjugacy. But, the observation above may give us a
  clue. According to Abramsky, it must be the case that the scalars
  are iso to the homset of the identity for the tensor -- which the
  observation above characterizes. 

  For now, we will simply bookmark the notion with $\overline{x}$.
\end{remark}

\subsubsection{Adjointness}

We need to give a definition of $(\cdot)^{\dagger}$ for matrices. The
obvious candidate definition is
\begin{mathpar}
(\Sigma_{\alpha}\fprmatrix{P_{\alpha}}{x_{\alpha}}{Q_{\alpha}})^{\dagger}
= \Sigma_{\alpha}\fprmatrix{(Q_{\alpha}^{\underline{\perp}})^{*}}{\overline{x}_{\alpha}}{P_{\alpha}^{\underline{\perp}}} 
\end{mathpar}

But, $(Q_{\alpha}^{\underline{\perp}})^{*}$ requires a name along
which to communicate the process to achieve the context application.

\subsubsection{Basis for a basis}
If processes label states and ``addition'' of states (a.k.a. vector
addition) is interpreted as parallel composition, what corresponds to
notions of linear independence and basis? Here, we recall that Yoshida
has developed a set of \emph{combinators} for an asynchronous verison
of Milner's $\pi$-calculus. These are a finite set of processes such
any process can be expressed as parallel composition of these
combinators together with liberal uses of the new operator and
replication. We can simply give a translation of these into the
present calculus and have reasonable expectation that the property
carries over. That is, that the resultant set allows to express all
processes via parallel composition. Note, however, that there is no
new operator or replication in this calculus. As a result, we expect
that the corresponding set is actually infinite. That is, we expect
that the space is actually infinite dimensional.

\begin{remark}
  The attentive reader may be a bit concerned. Certainly, the
  collection $S$, $K$ and $I$ is a finite set of
  combinators. Shouldn't we expect to see a finite set of combinators
  for an effectively equivalent system? i am very sympathetic to this
  critique and feel it warrants full attention. On the other hand, i
  also have in mind the following analogy. The natural numbers, as a
  monoid under addition, has exactly $1$ generator, while the natural
  numbers, as a monoid under multiplication, has countably many
  generators (the primes). We observe that the application of the
  lambda calculus is much less resource sensitive than the parallel
  composition of the $\pi$-calculus. Could it be the case that we have
  an analogy of the form
  
  \begin{mathpar}
    m + n : MN :: m*n : M|N
  \end{mathpar}

  giving a similar blow up in the set of ``primes''?  This is such a
  wonderful thought that, even if it's not true, i think it's worth
  writing down.
\end{remark}
 

\documentclass[12pt]{llncs}
%\documentclass{jktr}

\usepackage[pdftex]{hyperref}                   
\usepackage {listings}
\usepackage {mathpartir}
\usepackage{bcprules}
%\usepackage{listings}
                       
\usepackage{graphicx} 
%\usepackage[margins=2.5cm,nohead,nofoot]{geometry}
%\usepackage{geometry}
\usepackage{amsfonts}
\usepackage{amstext}
\usepackage{latexsym}
\usepackage{amssymb}
\usepackage{color}


%\include{myPreamble}
\include{qm2pi.local} 

%\ifpdf
%\usepackage[pdftex]{graphicx}
%\else
%\usepackage{graphicx}
%\fi

 % \ifpdf
%  \usepackage{pdfsync}
%  \if


%\title{Brief Article}
%\author{David F. Snyder}
%\author{L.G. Meredith}

%\address{Dept. of Math., Texas State University--San Marcos, San Marcos, TX 78666}
       
\pagestyle{empty}


\begin{document}

\lstset{language=[Objective]Caml,frame=shadowbox}

\input{qm2pi.front}

% section front matter (end)

\input{qm2pi.intro} 
 
% section introduction (end)

% \input{qm2pi.knotations} 

% section notation (end)

\input{qm2pi.process.calculi} 

% section concurrent_process_calculi_and_spatial_logics_ (end)
    
%\input{qm2pi.knots2pi} 

%\input{qm2pi.trefoil} 

%\input{qm2pi.mainthm} 

% subsection basic_interpretation (end)

%\input{qm2pi.rho.presentation} 
\subsection{The syntax and semantics of the notation system}\label{sub:the_syntax_and_semantics_of_the_notation_system} % (fold)

We now summarize a technical presentation of the calculus that
embodies our theory of dynamics. The typical presentation of such a
calculus follows the style of giving generators and relations on
them. The grammar, below, describing term constructors, freely
generates the set of processes, $\Proc$. This set is then quotiented
by a relation known as structural congruence and it is over this set
that the notion of dynamics is expressed. This presentation is
essentially that of \cite{MeredithR05} with the addition of
polyadicity and summation. For readability we have relegated some of
the technical subtleties to an appendix.

\subsubsection{Process grammar}\label{subsub:process_grammar}

\begin{mathpar}
  \inferrule* [lab=synchronization] {} {{M} \bc \pzero \;|\; x?F \;|\; x!C }
  \and
  \inferrule* [lab=abstraction] {} {{F} \bc (x)P}
  \and
  \inferrule* [lab=concretion] {} {{C} \bc \langle Q \rangle}
  \and
  \inferrule* [lab=process] {} {{P,Q} \bc M \;| \;P|Q \;|\; @{x}}
  \and
  \inferrule* [lab=name] {} {{x} \bc \quotep{P}}
\end{mathpar} 

Note that $\vec{x}$ (resp. $\vec{P}$) denotes a vector of names
(resp. processes) of length $|\vec{x}|$ (resp. $|\vec{P}|$). We adopt
the following useful abbreviations.

\begin{mathpar}
   x?(\vec{y}).P := x.(\vec{y})P \and  x\clift{\vec{P}} := x.\clift{\vec{P}}
   \and x!(y) := \lift{x}{\dropn{y}}
   \and \Pi_{i=0}^{n-1}P_i := P_0 | \ldots | P_{n-1}
\end{mathpar}

\subsubsection{Structural congruence}

\paragraph{Free and bound names and alpha-equivalence.} At the
core of structural equivalence is alpha-equivalence which identifies
process that are the same up to a change of variable. Formally, we
recognize the distinction between free and bound names. The free names
of a process, $\freenames{P}$, may be calculated recursively as
follows:

\begin{mathpar}
\freenames{\pzero} := \emptyset
  \and \\
  \freenames{x?(y).P} := \{ x \} \cup (\freenames{P} \setminus \{ y \})
  \and 
  \freenames{x!\langle P \rangle} := \{ x \} \cup \{ P \} 
  \and \\
  \freenames{P|Q} := \freenames{P} \cup \freenames{Q}
  \and \\
  \freenames{@{x}} := \{ x \}
\end{mathpar}

$\pi$
$\quotep{\pi}$

$\freenames{-} : \pi \to \mathcal{P}(\quotep{\pi})$

\begin{eqnarray*}
  \freenames{\pzero} & := & \emptyset \\
  \freenames{x?(y).P} & := & \{ x \} \cup (\freenames{P} \setminus \{ y \}) \\
  \freenames{x!\langle P \rangle} & := & \{ x \} \cup \{ P \} \\
  \freenames{P|Q} & := & \freenames{P} \cup \freenames{Q} \\
  \freenames{\dropn{x}} & := & \{ x \}
\end{eqnarray*}

The bound names of a process, $\boundnames{P}$, are those names occurring in $P$
that are not free. For example, in $x?(y).0$, the name $x$ is free, while $y$ is bound.

\begin{mathpar}
  \inferrule* [lab=monoidal-laws] {} { P|Q \equiv Q|P \and P|0 \equiv P \and P|(Q|R) \equiv (P|Q)|R }
\end{mathpar}

\begin{mathpar}
  \inferrule* [lab=alpha-equivalence] {} { (x)P \equiv (y)P\{y/x\} \and y \not\in \freenames{P} }
\end{mathpar}

\begin{definition}
Then two processes, $P,Q$, are alpha-equivalent if $P = Q\{\vec{y}/\vec{x}\}$ for
some $\vec{x} \in \boundnames{Q},\vec{y} \in \boundnames{P}$, where $Q\{\vec{y}/\vec{x}\}$
denotes the capture-avoiding substitution of $\vec{y}$ for $\vec{x}$ in $Q$.
\end{definition}

\begin{definition}
  The {\em structural congruence} \cite{SangiorgiWalker} , $\equiv$,
  between processes is the least congruence containing
  alpha-equivalence, satisfying the abelian monoid laws
  (associativity, commutativity and $\pzero$ as identity) for parallel
  composition $|$ and for summation $+$.
\end{definition}

\subsection{Name equivalence}

We take name equivalence, written $\nameeq$, to be the smallest
equivalence relation generated by the following rules.

\begin{mathpar}
\inferrule*[lab=Quote-drop]
{ }
{ \quotep{@{x}} \nameeq x }

\inferrule*[lab=Struct-equiv]
{ P \scong Q }
{ \quotep{P} \nameeq \quotep{Q} }
\end{mathpar}

The astute reader will have noticed that the mutual recursion of names
and processes imposes a mutual recursion on alpha-equivalence and
structural equivalence via name-equivalence. Fortunately, all of this
works out pleasantly and we may calculate in the natural way, free of
concern. The reader interested in the details is referred to the
appendix \ref{appendix:rho_details}.

\subsection{Substitution}

We use $\Proc$ for the set of processes, $\QProc$ for the set of
names, and $\id{\{}\vec{y} / \vec{x} \id{\}}$ to denote partial maps,
$s : \QProc \rightarrow \QProc$. A map, $s$ lifts, uniquely, to a map
on process terms, $\widehat{s} : \Proc \rightarrow \Proc$ by the
following equations.

\begin{mathpar}
  (0) \psubstp{Q}{P} := 0 \\
  (R \juxtap S) \psubstp{Q}{P}
  :=    
  (R)\psubstp{Q}{P} \juxtap (S) \psubstp{Q}{P} \\
  (x?(y).R) \psubstp{Q}{P}    
  :=    
  (x)\substp{Q}{P} (z)\concat( (R \psubstn{z}{y}) \psubstp{Q}{P} ) \\
  (\lift{x}{R}) \psubstp{Q}{P}  
  :=
  \lift{(x)\substp{Q}{P}}{ R \psubstp{Q}{P} } \\
%   (\dropn{x})  \psubstp{Q}{P}       
%   := 
%   \left\{ 
%     \begin{array}{ccc} 
%       \dropn{\quotep{Q}} & & x \nameeq \quotep{P} \\
%       \dropn{x} & & otherwise \\
%     \end{array}
%   \right. 
  (\dropn{x})  \psubstp{Q}{P}       
  := 
  \left\{ 
    \begin{array}{ccc} 
      Q & & x \nameeq \quotep{P} \\
      \dropn{x} & & otherwise \\
    \end{array}
  \right.
\end{mathpar}
 

where

\begin{eqnarray}
  (x)\id{\{} \lpquote Q \rpquote / \lpquote P \rpquote \id{\}}            = 
  \left\{ 
    \begin{array}{ccc}
      \lpquote Q \rpquote & & x \nameeq \lpquote P \rpquote \\
      x & & otherwise \\
    \end{array}
  \right. \nonumber
\end{eqnarray}

and $z$ is chosen distinct from $\quotep{P}$, $\quotep{Q}$, the free
names in $Q$, and all the names in $R$. Our $\alpha$-equivalence will
be built in the standard way from this substitution.

\begin{remark}\label{rem:no_self_referential_names}
  One consequence of these definitions is that $\forall P. \quotep{P}
  \not\in \freenames{P}$.
\end{remark}

\subsection{ Dynamic quote: an example }

Anticipating something of what's to come, consider applying the
substitution, $\widehat{\id{\{}u / z \id{\}}}$, to the following pair
of processes, $\lift{w}{y!(z)}$ and $w[ \lpquote y!(z) \rpquote ]$.

\begin{eqnarray}
	\lift{w}{y!(z)}\widehat{\id{\{}u / z \id{\}}}
		& = &
		\lift{w}{y!(u)} \nonumber\\
	w[ \lpquote y!(z) \rpquote ] \widehat{ \id{\{}u / z \id{\}} }
		& = &
		w[ \lpquote y!(z) \rpquote ] \nonumber
\end{eqnarray}

Because the body of the process between quotes is impervious to
substitution, we get radically different answers. In fact, by
examining the first process in an input context,
e.g. $x?(z).\lift{w}{y!(z)}$, we see that the process under the lift
operator may be shaped by prefixed inputs binding a name inside it. In
this sense, the lift operator will be seen as a way to dynamically
construct processes before reifying them as names.

Finally equipped with these standard features we can present the
dynamics of the calculus.

\subsubsection{Operational semantics} 

Finally, we introduce the computational dynamics. What marks these
algebras as distinct from other more traditionally studied algebraic
structures, e.g. vector spaces or polynomial rings, is the manner in
which dynamics is captured. In traditional structures, dynamics is typically
expressed through morphisms between such structures, as in linear maps
between vector spaces or morphisms between rings. In algebras
associated with the semantics of computation, the dynamics is
expressed as part of the algebraic structure itself, through a
reduction reduction relation typically denoted by $\red$. Below, we
give a recursive presentation of this relation for the calculus used
in the encoding.

$\red \subseteq \pi \times \pi$
$\red : \pi \to \mathcal{P}(\pi)$

\begin{mathpar}
  \inferrule* [lab=Comm] { \textsf{match}( x_{src}, x_{trgt} ) } { x_{trgt}?(y)P \; | \; x_{src}!\langle {Q} \rangle \red P\{\quotep{Q}/y}\} }
  \and \\
  \inferrule* [lab=Par] {{P} \red {P}'} {{{P} | {Q}} \red {{P}' | {Q}}}
  \and
  \inferrule* [lab=Equiv]{{{P} \scong {P}'} \andalso {{P}' \red {Q}'} \andalso {{Q}' \scong {Q}}}{{P} \red {Q}}
\end{mathpar}

\begin{eqnarray*}
  match_{\equiv} (\quotep{P},\quotep{Q}) & := & P \equiv Q \\
  match_{\dagger}(\quotep{P},\quotep{Q}) & := & \forall R. P|Q \red^{*} R => R \red^{*} 0 \\
  match_{K}(\quotep{P},\quotep{Q}) & := & K \mbox{ for some context } K
\end{eqnarray*}

$u?(x)P | u!\langle Q \rangle \red P\{\quotep{Q}/x\}$

%We write $\wred$ for $\red^*$, and $P\red$ if $\exists Q $ such that $ P \red Q$.
We write $P\red$ if $\exists Q $ such that $ P \red Q$ and $P\not\red$, otherwise.

\section{Replication}

As mentioned before, it is known that replication (and hence
recursion) can be implemented in a higher-order process algebra
\cite{SangiorgiWalker}. As our first example of calculation with the
machinery thus far presented we give the construction explicitly in
the {\rhoc}.

\begin{eqnarray}
	D_{x} & := & \prefix{x}{y}{(\binpar{\outputp{x}{y}}{@{y}})} \nonumber\\
	\bangp_{x}{P} & := & \binpar{{x}!\langle{\binpar{D_{x}}{P}}\rangle}{D_{x}} \nonumber
\end{eqnarray}

\begin{eqnarray}
	\bangp_{x}{P} & & \nonumber\\
	=
	& {x}!\langle{(\prefix{x}{y}{(\outputp{x}{y} | @{y})) | P}}\rangle 
	      | \prefix{x}{y}{(\outputp{x}{y} | @{y})} & \nonumber\\
	\red
	& (\outputp{x}{y} | @{y})\substn{\quotep{(\prefix{x}{y}{(@{y} | \outputp{x}{y})) | P}}}{y} & \nonumber\\
	=
	& \outputp{x}{\quotep{(\prefix{x}{y}{(\outputp{x}{y} | @{y})) | P}}}
	  | {(\prefix{x}{y}{(\outputp{x}{y} | @{y})) | P}} & \nonumber\\
	\red
	& \ldots & \nonumber\\
	\red^*
	& P | P | \ldots & \nonumber
\end{eqnarray}

Of course, this encoding, as an implementation, runs away, unfolding
$\bangp{P}$ eagerly. A lazier and more implementable replication
operator, restricted to input-guarded processes, may be obtained as follows.

\begin{eqnarray}
\bangp{\prefix{u}{v}{P}} 
	:= 
	\binpar{\lift{x}{\prefix{u}{v}{(\binpar{D(x)}{P})}}}{D(x)} \nonumber
\end{eqnarray}

\begin{remark}
  Note that the lazier definition still does not deal with summation
  or mixed summation (i.e. sums over input and output). The reader is
  invited to construct definitions of replication that deal with these
  features. 

  Further, the definitions are parameterized in a name, $x$. Can you,
  gentle reader, make a definition that eliminates this parameter and
  guarantees no accidental interaction between the replication
  machinery and the process being replicated -- i.e. no accidental
  sharing of names used by the process to get its work done and the
  name(s) used by the replication to effect copying. This latter
  revision of the definition of replication is crucial to obtaining
  the expected identity $!!P \sim !P$.
\end{remark}

\begin{remark}\label{rem:paradoxical_combinator}
  The reader familiar with the lambda calculus will have noticed the
  similarity between $D$ and the paradoxical combinator.

  [Ed. note: the existence of this seems to suggest we have to be more
  restrictive on the set of processes and names we admit if we are to
  support no-cloning.]
\end{remark}

\subsubsection{Bisimulation}

The computational dynamics gives rise to another kind of equivalence,
the equivalence of computational behavior. As previously mentioned
this is typically captured \emph{via} some form of bisimulation.

% The notion we use in this paper is weak barbed bisimulation
% \cite{milner91polyadicpi}.

The notion we use in this paper is derived from weak barbed
bisimulation \cite{milner91polyadicpi}. 

\begin{definition}
An \emph{observation relation}, $\downarrow_{\mathcal N}$, over a set
of names, $\mathcal N$, is the smallest relation satisfying the rules
below.

\infrule[Out-barb]{y \in {\mathcal N}, \; x \nameeq y}
		  {\outputp{x}{v} \downarrow_{\mathcal N} x}
\infrule[Par-barb]{\mbox{$P\downarrow_{\mathcal N} x$ or $Q\downarrow_{\mathcal N} x$}}
		  {\binpar{P}{Q} \downarrow_{\mathcal N} x}

We write $P \Downarrow_{\mathcal N} x$ if there is $Q$ such that 
$P \wred Q$ and $Q \downarrow_{\mathcal N} x$.
\end{definition}

\begin{definition}
%\label{def.bbisim}
An  ${\mathcal N}$-\emph{barbed bisimulation} over a set of names, ${\mathcal N}$, is a symmetric binary relation 
${\mathcal S}_{\mathcal N}$ between agents such that $P\rel{S}_{\mathcal N}Q$ implies:
\begin{enumerate}
\item If $P \red P'$ then $Q \wred Q'$ and $P'\rel{S}_{\mathcal N} Q'$.
\item If $P\downarrow_{\mathcal N} x$, then $Q\Downarrow_{\mathcal N} x$.
\end{enumerate}
$P$ is ${\mathcal N}$-barbed bisimilar to $Q$, written
$P \wbbisim_{\mathcal N} Q$, if $P \rel{S}_{\mathcal N} Q$ for some ${\mathcal N}$-barbed bisimulation ${\mathcal S}_{\mathcal N}$.
\end{definition}

$\mathcal{R} \subseteq \pi \times \pi$

$P \mathcal{R} Q => \forall P'. P \red P' \Rightarrow \exists Q'. Q \red Q', P' \mathcal{R} Q'$

$P \vdash x \Rightarrow Q \vdash x$

\begin{mathpar}
  \inferrule*[lab=Out-barb]{x \nameeq y}{{y}!\langle{Q}\rangle \vdash x}
  \and
  \inferrule*[lab=Par-barb]{\mbox{$P\vdash x$ or $Q\vdash x$}}{\binpar{P}{Q} \vdash x}
\end{mathpar}

\subsubsection{Contexts}

One of the principle advantages of computational calculi like the
$\pi$-calculus is a well-defined notion of context,
contextual-equivalence and a correlation between
contextual-equivalence and notions of bisimulation. The notion of
context allows the decomposition of a process into (sub-)process and
its syntactic environment, its context. Thus, a context may be
thought of as a process with a ``hole'' (written $\Box$) in it. The
application of a context $M$ to a process $P$, written $M[P]$, is
tantamount to filling the hole in $M$ with $P$. In this paper we do
not need the full weight of this theory, but do make use of the notion
of context in the proof the main theorem. 

\begin{mathpar}
  \inferrule* [lab=summation] {} {{M_{M},M_{N}} \bc \Box \;|\; x.M_{A} \;|\; M_{M}+M_{N}}
  \and
  \inferrule* [lab=agent] {} {{M_{A}} \bc (\vec{x})M_{P} \;| \; \clift{P_0,\ldots,M_{P},\ldots,P_N}}
  \and \\
  \inferrule* [lab=process] {} {{M_{P}} \bc M_{N} \;| \;P|M_{P} }
\end{mathpar} 

\begin{mathpar}
  \inferrule* [lab=sychronization] {} {M_{N} \bc \Box \;|\; x?M_{F} \;|\; x!M_{C}}
  \and
  \inferrule* [lab=abstraction] {} {{M_{F}} \bc (x)M_{P} }
  \and
  \inferrule* [lab=concretion] {} {{M_{C}} \bc \langle M_{P} \rangle }
  \and \\
  \inferrule* [lab=process] {} {{M_{P}} \bc M_{N} \;| \;P|M_{P} }
\end{mathpar}

\begin{definition}[contextual application] Given a context $M$, and
  process $P$, we define the \emph{contextual application}, $M[P] :=
  M\{P/\Box\}$. That is, the contextual application of M to P is the
  substitution of $P$ for $\Box$ in $M$.
\end{definition}

$\meaningof{-} : L \to \mathcal{P}(\pi)$

\begin{mathpar}
  \inferrule* [lab=collection] {} {\meaningof{true} = \pi, \and \meaningof{~E} = \pi \setminus \meaningof{E}, \and \meaningof{E_{1} \& E_{2}} = \meaningof{E_{1}} \cap \meaningof{E_{2}}}
\end{mathpar}

\begin{mathpar}
  \inferrule* [lab=structure] {} {\meaningof{0} = \{ P \in \pi | P \equiv 0 \}, \and \\ \meaningof{E_1 | E_2} = \{ P \in \pi | P \equiv P_{1} | P_{2}, P_{1} \in \meaningof{E_{1}}, P_{2} \in \meaningof{E_2}\} }
\end{mathpar}

\begin{mathpar}
 \inferrule* [lab=behavior] {} {\meaningof{\langle a?b \rangle E} = \{ P \in \pi | P \equiv Q | u?(y)P', \\ \and \\\\ \and \\ \;\;\; u \in \meaningof{a}, \forall z.P'\{z/y\} \in \meaningof{E\{z/b\}}\}, \and \\ \meaningof{a!E} = \{ P \in \pi | P \equiv Q | x!\langle P' \rangle, x \in \meaningof{a} P' \in \meaningof{E}\} }
\end{mathpar}

\begin{mathpar}
 \inferrule* [lab=nominal] {} {\meaningof{\quotep{E}} = \{ \quotep{P} \in \quotep{\pi} | P \in \meaningof{E} \}, \and \meaningof{\quotep{P}} = \{ \quotep{Q} \in \quotep{\pi} | P \equiv Q \} \and \\ \meaningof{@\quotep{E}} = \{ P \in \pi | P \equiv @x, x \in \meaningof{E} \}}
\end{mathpar}

\begin{eqnarray*}
  \\
  \meaningof{-} : TS \to ST
\end{eqnarray*}

\begin{eqnarray*}
  \\
  L : TS \to ST
\end{eqnarray*}

\begin{eqnarray*}
  \\
  P \models E \iff P \in \meaningof{E}
\end{eqnarray*}

\begin{eqnarray*}
  P \approx_{L} Q \iff \forall E \in L. P \models E \iff Q \models E
\end{eqnarray*}

\begin{eqnarray*}
  P \approx_{K} Q
\end{eqnarray*}

\begin{eqnarray*}
  P \approx Q
\end{eqnarray*}

$\approx_{K} = \approx = \approx_{L}$

\subsubsection{Contextual duality}

Note that contexts extend the quotation operation to a family of
operations from processes to names. Given a context, $M$, we can
define a \emph{nominal context}, $\quotep{M}$ by $\quotep{M}[P] :=
\quotep{M[P]}$. To foreshadow what is to come we observe that these
operations enjoy a duality with processes very much like the duality
between vectors and maps from vectors to scalars.

Further, because the calculus is essentially higher-order, we have a
correspondence between contexts and processes. More specifically,
given a name $x$ and a context $M$ we can construct $M^{*}_{x}$ such
that 

\begin{mathpar}
  M^{*}_{x} | \lift{x}{P} \red M[P]
\end{mathpar}

namely,

\begin{mathpar}
  M^{*}_{x} := x?(u).M[\dropn{u}]
\end{mathpar}

The dependence of $M^{*}_{x}$ on a name makes it an abstraction, 

\begin{mathpar}
  M^{*} := (x)x?(u).M[\dropn{u}]
\end{mathpar}

\subsection{Additional notation}

It will sometimes be convenient to denote the process a name
quotes. We already have the notation $x = \quotep{P}$, but it will be
convenient to introduce an alternate notation, $\procn{x}$, when we
want to emphasize the connection to the use of the name. Note that, by
virtue of name equivalence, $\quotep{\procn{x}} \nameeq x$; so, the
notation is consistent with previous definitions.

Further, because names have structure it is possible to effect
substitutions on the basis of that structure. This means we need to
upgrade our notation for substitutions, which we accomplish by
adapting comprehension notation. Thus,

\begin{mathpar}
  P\{ y / x : x \in S \}
\end{mathpar}

is interpreted to mean the process derived from P by replacing (in a
capture-avoiding manner) each occurrence of $x$ in $S$ by $y$. For example,

\begin{mathpar}
  P\{ \quotep{\procn{x}|\procn{x}} / x : x \in \freenames{P} \}
\end{mathpar}

will replace each (occurrence) of a free name $x$ in $P$ by
$\quotep{\procn{x}|\procn{x}}$.

Also, we will avail ourselves of the notation $x^{L}$ and $x^{R}$ to
denote injections of a name into disjoint copies of the name
space. There are numerous ways to accomplish this. One example can be
found in \cite{MeredithR05}. This notation overloads to vectors of
names: $\vec{x}^{\pi} := (x_{i}^{\pi} \; : \; 0 \leq i < |\vec{x}| )$ where $\pi \in \{L,R\}$.

We also use $P^{\Box} := P|\Box$.

In \cite{MeredithR05} an interpretation of the new operator is
given. It turns out that there are several possible interpretations
all enjoying the requisite algebraic properties of the operator (see
\cite{milner91polyadicpi}). We will therefore make liberal use of
$(\nu\; \vec{x})P$.

% subsection the_syntax_and_semantics_of_the_notation_system (end)   

\input{qm2pi.qmops} 

\input{qm2pi.sterngerlach} 

\input{qm2pi.metric} 

% section concurrent_process_calculi (end)

%\input{qm2pi.proofsketch}

% section proof sketch (end)

%\input{qm2pi.slviaknots} 

% section spatial logic via knots (end)

\input{qm2pi.conclusion}

% section conclusion (end)

%\input{qm2pi.dtcodes} 

% section wiring algorithm (end)

\input{qm2pi.ack} 

% section acknowledgments (end)

\newpage


\bibliographystyle{plain}   
\bibliography{../../biblios/main.bib}

\input{qm2pi.rhodetails}

\end{document}

 

\documentclass[12pt]{llncs}
%\documentclass{jktr}

\usepackage[pdftex]{hyperref}                   
\usepackage {listings}
\usepackage {mathpartir}
\usepackage{bcprules}
%\usepackage{listings}
                       
\usepackage{graphicx} 
%\usepackage[margins=2.5cm,nohead,nofoot]{geometry}
%\usepackage{geometry}
\usepackage{amsfonts}
\usepackage{amstext}
\usepackage{latexsym}
\usepackage{amssymb}
\usepackage{color}


%\include{myPreamble}
\include{qm2pi.local} 

%\ifpdf
%\usepackage[pdftex]{graphicx}
%\else
%\usepackage{graphicx}
%\fi

 % \ifpdf
%  \usepackage{pdfsync}
%  \if


%\title{Brief Article}
%\author{David F. Snyder}
%\author{L.G. Meredith}

%\address{Dept. of Math., Texas State University--San Marcos, San Marcos, TX 78666}
       
\pagestyle{empty}


\begin{document}

\lstset{language=[Objective]Caml,frame=shadowbox}

\input{qm2pi.front}

% section front matter (end)

\input{qm2pi.intro} 
 
% section introduction (end)

% \input{qm2pi.knotations} 

% section notation (end)

\input{qm2pi.process.calculi} 

% section concurrent_process_calculi_and_spatial_logics_ (end)
    
%\input{qm2pi.knots2pi} 

%\input{qm2pi.trefoil} 

%\input{qm2pi.mainthm} 

% subsection basic_interpretation (end)

%\input{qm2pi.rho.presentation} 
\subsection{The syntax and semantics of the notation system}\label{sub:the_syntax_and_semantics_of_the_notation_system} % (fold)

We now summarize a technical presentation of the calculus that
embodies our theory of dynamics. The typical presentation of such a
calculus follows the style of giving generators and relations on
them. The grammar, below, describing term constructors, freely
generates the set of processes, $\Proc$. This set is then quotiented
by a relation known as structural congruence and it is over this set
that the notion of dynamics is expressed. This presentation is
essentially that of \cite{MeredithR05} with the addition of
polyadicity and summation. For readability we have relegated some of
the technical subtleties to an appendix.

\subsubsection{Process grammar}\label{subsub:process_grammar}

\begin{mathpar}
  \inferrule* [lab=synchronization] {} {{M} \bc \pzero \;|\; x?F \;|\; x!C }
  \and
  \inferrule* [lab=abstraction] {} {{F} \bc (x)P}
  \and
  \inferrule* [lab=concretion] {} {{C} \bc \langle Q \rangle}
  \and
  \inferrule* [lab=process] {} {{P,Q} \bc M \;| \;P|Q \;|\; @{x}}
  \and
  \inferrule* [lab=name] {} {{x} \bc \quotep{P}}
\end{mathpar} 

Note that $\vec{x}$ (resp. $\vec{P}$) denotes a vector of names
(resp. processes) of length $|\vec{x}|$ (resp. $|\vec{P}|$). We adopt
the following useful abbreviations.

\begin{mathpar}
   x?(\vec{y}).P := x.(\vec{y})P \and  x\clift{\vec{P}} := x.\clift{\vec{P}}
   \and x!(y) := \lift{x}{\dropn{y}}
   \and \Pi_{i=0}^{n-1}P_i := P_0 | \ldots | P_{n-1}
\end{mathpar}

\subsubsection{Structural congruence}

\paragraph{Free and bound names and alpha-equivalence.} At the
core of structural equivalence is alpha-equivalence which identifies
process that are the same up to a change of variable. Formally, we
recognize the distinction between free and bound names. The free names
of a process, $\freenames{P}$, may be calculated recursively as
follows:

\begin{mathpar}
\freenames{\pzero} := \emptyset
  \and \\
  \freenames{x?(y).P} := \{ x \} \cup (\freenames{P} \setminus \{ y \})
  \and 
  \freenames{x!\langle P \rangle} := \{ x \} \cup \{ P \} 
  \and \\
  \freenames{P|Q} := \freenames{P} \cup \freenames{Q}
  \and \\
  \freenames{@{x}} := \{ x \}
\end{mathpar}

$\pi$
$\quotep{\pi}$

$\freenames{-} : \pi \to \mathcal{P}(\quotep{\pi})$

\begin{eqnarray*}
  \freenames{\pzero} & := & \emptyset \\
  \freenames{x?(y).P} & := & \{ x \} \cup (\freenames{P} \setminus \{ y \}) \\
  \freenames{x!\langle P \rangle} & := & \{ x \} \cup \{ P \} \\
  \freenames{P|Q} & := & \freenames{P} \cup \freenames{Q} \\
  \freenames{\dropn{x}} & := & \{ x \}
\end{eqnarray*}

The bound names of a process, $\boundnames{P}$, are those names occurring in $P$
that are not free. For example, in $x?(y).0$, the name $x$ is free, while $y$ is bound.

\begin{mathpar}
  \inferrule* [lab=monoidal-laws] {} { P|Q \equiv Q|P \and P|0 \equiv P \and P|(Q|R) \equiv (P|Q)|R }
\end{mathpar}

\begin{mathpar}
  \inferrule* [lab=alpha-equivalence] {} { (x)P \equiv (y)P\{y/x\} \and y \not\in \freenames{P} }
\end{mathpar}

\begin{definition}
Then two processes, $P,Q$, are alpha-equivalent if $P = Q\{\vec{y}/\vec{x}\}$ for
some $\vec{x} \in \boundnames{Q},\vec{y} \in \boundnames{P}$, where $Q\{\vec{y}/\vec{x}\}$
denotes the capture-avoiding substitution of $\vec{y}$ for $\vec{x}$ in $Q$.
\end{definition}

\begin{definition}
  The {\em structural congruence} \cite{SangiorgiWalker} , $\equiv$,
  between processes is the least congruence containing
  alpha-equivalence, satisfying the abelian monoid laws
  (associativity, commutativity and $\pzero$ as identity) for parallel
  composition $|$ and for summation $+$.
\end{definition}

\subsection{Name equivalence}

We take name equivalence, written $\nameeq$, to be the smallest
equivalence relation generated by the following rules.

\begin{mathpar}
\inferrule*[lab=Quote-drop]
{ }
{ \quotep{@{x}} \nameeq x }

\inferrule*[lab=Struct-equiv]
{ P \scong Q }
{ \quotep{P} \nameeq \quotep{Q} }
\end{mathpar}

The astute reader will have noticed that the mutual recursion of names
and processes imposes a mutual recursion on alpha-equivalence and
structural equivalence via name-equivalence. Fortunately, all of this
works out pleasantly and we may calculate in the natural way, free of
concern. The reader interested in the details is referred to the
appendix \ref{appendix:rho_details}.

\subsection{Substitution}

We use $\Proc$ for the set of processes, $\QProc$ for the set of
names, and $\id{\{}\vec{y} / \vec{x} \id{\}}$ to denote partial maps,
$s : \QProc \rightarrow \QProc$. A map, $s$ lifts, uniquely, to a map
on process terms, $\widehat{s} : \Proc \rightarrow \Proc$ by the
following equations.

\begin{mathpar}
  (0) \psubstp{Q}{P} := 0 \\
  (R \juxtap S) \psubstp{Q}{P}
  :=    
  (R)\psubstp{Q}{P} \juxtap (S) \psubstp{Q}{P} \\
  (x?(y).R) \psubstp{Q}{P}    
  :=    
  (x)\substp{Q}{P} (z)\concat( (R \psubstn{z}{y}) \psubstp{Q}{P} ) \\
  (\lift{x}{R}) \psubstp{Q}{P}  
  :=
  \lift{(x)\substp{Q}{P}}{ R \psubstp{Q}{P} } \\
%   (\dropn{x})  \psubstp{Q}{P}       
%   := 
%   \left\{ 
%     \begin{array}{ccc} 
%       \dropn{\quotep{Q}} & & x \nameeq \quotep{P} \\
%       \dropn{x} & & otherwise \\
%     \end{array}
%   \right. 
  (\dropn{x})  \psubstp{Q}{P}       
  := 
  \left\{ 
    \begin{array}{ccc} 
      Q & & x \nameeq \quotep{P} \\
      \dropn{x} & & otherwise \\
    \end{array}
  \right.
\end{mathpar}
 

where

\begin{eqnarray}
  (x)\id{\{} \lpquote Q \rpquote / \lpquote P \rpquote \id{\}}            = 
  \left\{ 
    \begin{array}{ccc}
      \lpquote Q \rpquote & & x \nameeq \lpquote P \rpquote \\
      x & & otherwise \\
    \end{array}
  \right. \nonumber
\end{eqnarray}

and $z$ is chosen distinct from $\quotep{P}$, $\quotep{Q}$, the free
names in $Q$, and all the names in $R$. Our $\alpha$-equivalence will
be built in the standard way from this substitution.

\begin{remark}\label{rem:no_self_referential_names}
  One consequence of these definitions is that $\forall P. \quotep{P}
  \not\in \freenames{P}$.
\end{remark}

\subsection{ Dynamic quote: an example }

Anticipating something of what's to come, consider applying the
substitution, $\widehat{\id{\{}u / z \id{\}}}$, to the following pair
of processes, $\lift{w}{y!(z)}$ and $w[ \lpquote y!(z) \rpquote ]$.

\begin{eqnarray}
	\lift{w}{y!(z)}\widehat{\id{\{}u / z \id{\}}}
		& = &
		\lift{w}{y!(u)} \nonumber\\
	w[ \lpquote y!(z) \rpquote ] \widehat{ \id{\{}u / z \id{\}} }
		& = &
		w[ \lpquote y!(z) \rpquote ] \nonumber
\end{eqnarray}

Because the body of the process between quotes is impervious to
substitution, we get radically different answers. In fact, by
examining the first process in an input context,
e.g. $x?(z).\lift{w}{y!(z)}$, we see that the process under the lift
operator may be shaped by prefixed inputs binding a name inside it. In
this sense, the lift operator will be seen as a way to dynamically
construct processes before reifying them as names.

Finally equipped with these standard features we can present the
dynamics of the calculus.

\subsubsection{Operational semantics} 

Finally, we introduce the computational dynamics. What marks these
algebras as distinct from other more traditionally studied algebraic
structures, e.g. vector spaces or polynomial rings, is the manner in
which dynamics is captured. In traditional structures, dynamics is typically
expressed through morphisms between such structures, as in linear maps
between vector spaces or morphisms between rings. In algebras
associated with the semantics of computation, the dynamics is
expressed as part of the algebraic structure itself, through a
reduction reduction relation typically denoted by $\red$. Below, we
give a recursive presentation of this relation for the calculus used
in the encoding.

$\red \subseteq \pi \times \pi$
$\red : \pi \to \mathcal{P}(\pi)$

\begin{mathpar}
  \inferrule* [lab=Comm] { \textsf{match}( x_{src}, x_{trgt} ) } { x_{trgt}?(y)P \; | \; x_{src}!\langle {Q} \rangle \red P\{\quotep{Q}/y}\} }
  \and \\
  \inferrule* [lab=Par] {{P} \red {P}'} {{{P} | {Q}} \red {{P}' | {Q}}}
  \and
  \inferrule* [lab=Equiv]{{{P} \scong {P}'} \andalso {{P}' \red {Q}'} \andalso {{Q}' \scong {Q}}}{{P} \red {Q}}
\end{mathpar}

\begin{eqnarray*}
  match_{\equiv} (\quotep{P},\quotep{Q}) & := & P \equiv Q \\
  match_{\dagger}(\quotep{P},\quotep{Q}) & := & \forall R. P|Q \red^{*} R => R \red^{*} 0 \\
  match_{K}(\quotep{P},\quotep{Q}) & := & K \mbox{ for some context } K
\end{eqnarray*}

$u?(x)P | u!\langle Q \rangle \red P\{\quotep{Q}/x\}$

%We write $\wred$ for $\red^*$, and $P\red$ if $\exists Q $ such that $ P \red Q$.
We write $P\red$ if $\exists Q $ such that $ P \red Q$ and $P\not\red$, otherwise.

\section{Replication}

As mentioned before, it is known that replication (and hence
recursion) can be implemented in a higher-order process algebra
\cite{SangiorgiWalker}. As our first example of calculation with the
machinery thus far presented we give the construction explicitly in
the {\rhoc}.

\begin{eqnarray}
	D_{x} & := & \prefix{x}{y}{(\binpar{\outputp{x}{y}}{@{y}})} \nonumber\\
	\bangp_{x}{P} & := & \binpar{{x}!\langle{\binpar{D_{x}}{P}}\rangle}{D_{x}} \nonumber
\end{eqnarray}

\begin{eqnarray}
	\bangp_{x}{P} & & \nonumber\\
	=
	& {x}!\langle{(\prefix{x}{y}{(\outputp{x}{y} | @{y})) | P}}\rangle 
	      | \prefix{x}{y}{(\outputp{x}{y} | @{y})} & \nonumber\\
	\red
	& (\outputp{x}{y} | @{y})\substn{\quotep{(\prefix{x}{y}{(@{y} | \outputp{x}{y})) | P}}}{y} & \nonumber\\
	=
	& \outputp{x}{\quotep{(\prefix{x}{y}{(\outputp{x}{y} | @{y})) | P}}}
	  | {(\prefix{x}{y}{(\outputp{x}{y} | @{y})) | P}} & \nonumber\\
	\red
	& \ldots & \nonumber\\
	\red^*
	& P | P | \ldots & \nonumber
\end{eqnarray}

Of course, this encoding, as an implementation, runs away, unfolding
$\bangp{P}$ eagerly. A lazier and more implementable replication
operator, restricted to input-guarded processes, may be obtained as follows.

\begin{eqnarray}
\bangp{\prefix{u}{v}{P}} 
	:= 
	\binpar{\lift{x}{\prefix{u}{v}{(\binpar{D(x)}{P})}}}{D(x)} \nonumber
\end{eqnarray}

\begin{remark}
  Note that the lazier definition still does not deal with summation
  or mixed summation (i.e. sums over input and output). The reader is
  invited to construct definitions of replication that deal with these
  features. 

  Further, the definitions are parameterized in a name, $x$. Can you,
  gentle reader, make a definition that eliminates this parameter and
  guarantees no accidental interaction between the replication
  machinery and the process being replicated -- i.e. no accidental
  sharing of names used by the process to get its work done and the
  name(s) used by the replication to effect copying. This latter
  revision of the definition of replication is crucial to obtaining
  the expected identity $!!P \sim !P$.
\end{remark}

\begin{remark}\label{rem:paradoxical_combinator}
  The reader familiar with the lambda calculus will have noticed the
  similarity between $D$ and the paradoxical combinator.

  [Ed. note: the existence of this seems to suggest we have to be more
  restrictive on the set of processes and names we admit if we are to
  support no-cloning.]
\end{remark}

\subsubsection{Bisimulation}

The computational dynamics gives rise to another kind of equivalence,
the equivalence of computational behavior. As previously mentioned
this is typically captured \emph{via} some form of bisimulation.

% The notion we use in this paper is weak barbed bisimulation
% \cite{milner91polyadicpi}.

The notion we use in this paper is derived from weak barbed
bisimulation \cite{milner91polyadicpi}. 

\begin{definition}
An \emph{observation relation}, $\downarrow_{\mathcal N}$, over a set
of names, $\mathcal N$, is the smallest relation satisfying the rules
below.

\infrule[Out-barb]{y \in {\mathcal N}, \; x \nameeq y}
		  {\outputp{x}{v} \downarrow_{\mathcal N} x}
\infrule[Par-barb]{\mbox{$P\downarrow_{\mathcal N} x$ or $Q\downarrow_{\mathcal N} x$}}
		  {\binpar{P}{Q} \downarrow_{\mathcal N} x}

We write $P \Downarrow_{\mathcal N} x$ if there is $Q$ such that 
$P \wred Q$ and $Q \downarrow_{\mathcal N} x$.
\end{definition}

\begin{definition}
%\label{def.bbisim}
An  ${\mathcal N}$-\emph{barbed bisimulation} over a set of names, ${\mathcal N}$, is a symmetric binary relation 
${\mathcal S}_{\mathcal N}$ between agents such that $P\rel{S}_{\mathcal N}Q$ implies:
\begin{enumerate}
\item If $P \red P'$ then $Q \wred Q'$ and $P'\rel{S}_{\mathcal N} Q'$.
\item If $P\downarrow_{\mathcal N} x$, then $Q\Downarrow_{\mathcal N} x$.
\end{enumerate}
$P$ is ${\mathcal N}$-barbed bisimilar to $Q$, written
$P \wbbisim_{\mathcal N} Q$, if $P \rel{S}_{\mathcal N} Q$ for some ${\mathcal N}$-barbed bisimulation ${\mathcal S}_{\mathcal N}$.
\end{definition}

$\mathcal{R} \subseteq \pi \times \pi$

$P \mathcal{R} Q => \forall P'. P \red P' \Rightarrow \exists Q'. Q \red Q', P' \mathcal{R} Q'$

$P \vdash x \Rightarrow Q \vdash x$

\begin{mathpar}
  \inferrule*[lab=Out-barb]{x \nameeq y}{{y}!\langle{Q}\rangle \vdash x}
  \and
  \inferrule*[lab=Par-barb]{\mbox{$P\vdash x$ or $Q\vdash x$}}{\binpar{P}{Q} \vdash x}
\end{mathpar}

\subsubsection{Contexts}

One of the principle advantages of computational calculi like the
$\pi$-calculus is a well-defined notion of context,
contextual-equivalence and a correlation between
contextual-equivalence and notions of bisimulation. The notion of
context allows the decomposition of a process into (sub-)process and
its syntactic environment, its context. Thus, a context may be
thought of as a process with a ``hole'' (written $\Box$) in it. The
application of a context $M$ to a process $P$, written $M[P]$, is
tantamount to filling the hole in $M$ with $P$. In this paper we do
not need the full weight of this theory, but do make use of the notion
of context in the proof the main theorem. 

\begin{mathpar}
  \inferrule* [lab=summation] {} {{M_{M},M_{N}} \bc \Box \;|\; x.M_{A} \;|\; M_{M}+M_{N}}
  \and
  \inferrule* [lab=agent] {} {{M_{A}} \bc (\vec{x})M_{P} \;| \; \clift{P_0,\ldots,M_{P},\ldots,P_N}}
  \and \\
  \inferrule* [lab=process] {} {{M_{P}} \bc M_{N} \;| \;P|M_{P} }
\end{mathpar} 

\begin{mathpar}
  \inferrule* [lab=sychronization] {} {M_{N} \bc \Box \;|\; x?M_{F} \;|\; x!M_{C}}
  \and
  \inferrule* [lab=abstraction] {} {{M_{F}} \bc (x)M_{P} }
  \and
  \inferrule* [lab=concretion] {} {{M_{C}} \bc \langle M_{P} \rangle }
  \and \\
  \inferrule* [lab=process] {} {{M_{P}} \bc M_{N} \;| \;P|M_{P} }
\end{mathpar}

\begin{definition}[contextual application] Given a context $M$, and
  process $P$, we define the \emph{contextual application}, $M[P] :=
  M\{P/\Box\}$. That is, the contextual application of M to P is the
  substitution of $P$ for $\Box$ in $M$.
\end{definition}

$\meaningof{-} : L \to \mathcal{P}(\pi)$

\begin{mathpar}
  \inferrule* [lab=collection] {} {\meaningof{true} = \pi, \and \meaningof{~E} = \pi \setminus \meaningof{E}, \and \meaningof{E_{1} \& E_{2}} = \meaningof{E_{1}} \cap \meaningof{E_{2}}}
\end{mathpar}

\begin{mathpar}
  \inferrule* [lab=structure] {} {\meaningof{0} = \{ P \in \pi | P \equiv 0 \}, \and \\ \meaningof{E_1 | E_2} = \{ P \in \pi | P \equiv P_{1} | P_{2}, P_{1} \in \meaningof{E_{1}}, P_{2} \in \meaningof{E_2}\} }
\end{mathpar}

\begin{mathpar}
 \inferrule* [lab=behavior] {} {\meaningof{\langle a?b \rangle E} = \{ P \in \pi | P \equiv Q | u?(y)P', \\ \and \\\\ \and \\ \;\;\; u \in \meaningof{a}, \forall z.P'\{z/y\} \in \meaningof{E\{z/b\}}\}, \and \\ \meaningof{a!E} = \{ P \in \pi | P \equiv Q | x!\langle P' \rangle, x \in \meaningof{a} P' \in \meaningof{E}\} }
\end{mathpar}

\begin{mathpar}
 \inferrule* [lab=nominal] {} {\meaningof{\quotep{E}} = \{ \quotep{P} \in \quotep{\pi} | P \in \meaningof{E} \}, \and \meaningof{\quotep{P}} = \{ \quotep{Q} \in \quotep{\pi} | P \equiv Q \} \and \\ \meaningof{@\quotep{E}} = \{ P \in \pi | P \equiv @x, x \in \meaningof{E} \}}
\end{mathpar}

\begin{eqnarray*}
  \\
  \meaningof{-} : TS \to ST
\end{eqnarray*}

\begin{eqnarray*}
  \\
  L : TS \to ST
\end{eqnarray*}

\begin{eqnarray*}
  \\
  P \models E \iff P \in \meaningof{E}
\end{eqnarray*}

\begin{eqnarray*}
  P \approx_{L} Q \iff \forall E \in L. P \models E \iff Q \models E
\end{eqnarray*}

\begin{eqnarray*}
  P \approx_{K} Q
\end{eqnarray*}

\begin{eqnarray*}
  P \approx Q
\end{eqnarray*}

$\approx_{K} = \approx = \approx_{L}$

\subsubsection{Contextual duality}

Note that contexts extend the quotation operation to a family of
operations from processes to names. Given a context, $M$, we can
define a \emph{nominal context}, $\quotep{M}$ by $\quotep{M}[P] :=
\quotep{M[P]}$. To foreshadow what is to come we observe that these
operations enjoy a duality with processes very much like the duality
between vectors and maps from vectors to scalars.

Further, because the calculus is essentially higher-order, we have a
correspondence between contexts and processes. More specifically,
given a name $x$ and a context $M$ we can construct $M^{*}_{x}$ such
that 

\begin{mathpar}
  M^{*}_{x} | \lift{x}{P} \red M[P]
\end{mathpar}

namely,

\begin{mathpar}
  M^{*}_{x} := x?(u).M[\dropn{u}]
\end{mathpar}

The dependence of $M^{*}_{x}$ on a name makes it an abstraction, 

\begin{mathpar}
  M^{*} := (x)x?(u).M[\dropn{u}]
\end{mathpar}

\subsection{Additional notation}

It will sometimes be convenient to denote the process a name
quotes. We already have the notation $x = \quotep{P}$, but it will be
convenient to introduce an alternate notation, $\procn{x}$, when we
want to emphasize the connection to the use of the name. Note that, by
virtue of name equivalence, $\quotep{\procn{x}} \nameeq x$; so, the
notation is consistent with previous definitions.

Further, because names have structure it is possible to effect
substitutions on the basis of that structure. This means we need to
upgrade our notation for substitutions, which we accomplish by
adapting comprehension notation. Thus,

\begin{mathpar}
  P\{ y / x : x \in S \}
\end{mathpar}

is interpreted to mean the process derived from P by replacing (in a
capture-avoiding manner) each occurrence of $x$ in $S$ by $y$. For example,

\begin{mathpar}
  P\{ \quotep{\procn{x}|\procn{x}} / x : x \in \freenames{P} \}
\end{mathpar}

will replace each (occurrence) of a free name $x$ in $P$ by
$\quotep{\procn{x}|\procn{x}}$.

Also, we will avail ourselves of the notation $x^{L}$ and $x^{R}$ to
denote injections of a name into disjoint copies of the name
space. There are numerous ways to accomplish this. One example can be
found in \cite{MeredithR05}. This notation overloads to vectors of
names: $\vec{x}^{\pi} := (x_{i}^{\pi} \; : \; 0 \leq i < |\vec{x}| )$ where $\pi \in \{L,R\}$.

We also use $P^{\Box} := P|\Box$.

In \cite{MeredithR05} an interpretation of the new operator is
given. It turns out that there are several possible interpretations
all enjoying the requisite algebraic properties of the operator (see
\cite{milner91polyadicpi}). We will therefore make liberal use of
$(\nu\; \vec{x})P$.

% subsection the_syntax_and_semantics_of_the_notation_system (end)   

\input{qm2pi.qmops} 

\input{qm2pi.sterngerlach} 

\input{qm2pi.metric} 

% section concurrent_process_calculi (end)

%\input{qm2pi.proofsketch}

% section proof sketch (end)

%\input{qm2pi.slviaknots} 

% section spatial logic via knots (end)

\input{qm2pi.conclusion}

% section conclusion (end)

%\input{qm2pi.dtcodes} 

% section wiring algorithm (end)

\input{qm2pi.ack} 

% section acknowledgments (end)

\newpage


\bibliographystyle{plain}   
\bibliography{../../biblios/main.bib}

\input{qm2pi.rhodetails}

\end{document}

 

% section concurrent_process_calculi (end)

%\documentclass[12pt]{llncs}
%\documentclass{jktr}

\usepackage[pdftex]{hyperref}                   
\usepackage {listings}
\usepackage {mathpartir}
\usepackage{bcprules}
%\usepackage{listings}
                       
\usepackage{graphicx} 
%\usepackage[margins=2.5cm,nohead,nofoot]{geometry}
%\usepackage{geometry}
\usepackage{amsfonts}
\usepackage{amstext}
\usepackage{latexsym}
\usepackage{amssymb}
\usepackage{color}


%\include{myPreamble}
\include{qm2pi.local} 

%\ifpdf
%\usepackage[pdftex]{graphicx}
%\else
%\usepackage{graphicx}
%\fi

 % \ifpdf
%  \usepackage{pdfsync}
%  \if


%\title{Brief Article}
%\author{David F. Snyder}
%\author{L.G. Meredith}

%\address{Dept. of Math., Texas State University--San Marcos, San Marcos, TX 78666}
       
\pagestyle{empty}


\begin{document}

\lstset{language=[Objective]Caml,frame=shadowbox}

\input{qm2pi.front}

% section front matter (end)

\input{qm2pi.intro} 
 
% section introduction (end)

% \input{qm2pi.knotations} 

% section notation (end)

\input{qm2pi.process.calculi} 

% section concurrent_process_calculi_and_spatial_logics_ (end)
    
%\input{qm2pi.knots2pi} 

%\input{qm2pi.trefoil} 

%\input{qm2pi.mainthm} 

% subsection basic_interpretation (end)

%\input{qm2pi.rho.presentation} 
\subsection{The syntax and semantics of the notation system}\label{sub:the_syntax_and_semantics_of_the_notation_system} % (fold)

We now summarize a technical presentation of the calculus that
embodies our theory of dynamics. The typical presentation of such a
calculus follows the style of giving generators and relations on
them. The grammar, below, describing term constructors, freely
generates the set of processes, $\Proc$. This set is then quotiented
by a relation known as structural congruence and it is over this set
that the notion of dynamics is expressed. This presentation is
essentially that of \cite{MeredithR05} with the addition of
polyadicity and summation. For readability we have relegated some of
the technical subtleties to an appendix.

\subsubsection{Process grammar}\label{subsub:process_grammar}

\begin{mathpar}
  \inferrule* [lab=synchronization] {} {{M} \bc \pzero \;|\; x?F \;|\; x!C }
  \and
  \inferrule* [lab=abstraction] {} {{F} \bc (x)P}
  \and
  \inferrule* [lab=concretion] {} {{C} \bc \langle Q \rangle}
  \and
  \inferrule* [lab=process] {} {{P,Q} \bc M \;| \;P|Q \;|\; @{x}}
  \and
  \inferrule* [lab=name] {} {{x} \bc \quotep{P}}
\end{mathpar} 

Note that $\vec{x}$ (resp. $\vec{P}$) denotes a vector of names
(resp. processes) of length $|\vec{x}|$ (resp. $|\vec{P}|$). We adopt
the following useful abbreviations.

\begin{mathpar}
   x?(\vec{y}).P := x.(\vec{y})P \and  x\clift{\vec{P}} := x.\clift{\vec{P}}
   \and x!(y) := \lift{x}{\dropn{y}}
   \and \Pi_{i=0}^{n-1}P_i := P_0 | \ldots | P_{n-1}
\end{mathpar}

\subsubsection{Structural congruence}

\paragraph{Free and bound names and alpha-equivalence.} At the
core of structural equivalence is alpha-equivalence which identifies
process that are the same up to a change of variable. Formally, we
recognize the distinction between free and bound names. The free names
of a process, $\freenames{P}$, may be calculated recursively as
follows:

\begin{mathpar}
\freenames{\pzero} := \emptyset
  \and \\
  \freenames{x?(y).P} := \{ x \} \cup (\freenames{P} \setminus \{ y \})
  \and 
  \freenames{x!\langle P \rangle} := \{ x \} \cup \{ P \} 
  \and \\
  \freenames{P|Q} := \freenames{P} \cup \freenames{Q}
  \and \\
  \freenames{@{x}} := \{ x \}
\end{mathpar}

$\pi$
$\quotep{\pi}$

$\freenames{-} : \pi \to \mathcal{P}(\quotep{\pi})$

\begin{eqnarray*}
  \freenames{\pzero} & := & \emptyset \\
  \freenames{x?(y).P} & := & \{ x \} \cup (\freenames{P} \setminus \{ y \}) \\
  \freenames{x!\langle P \rangle} & := & \{ x \} \cup \{ P \} \\
  \freenames{P|Q} & := & \freenames{P} \cup \freenames{Q} \\
  \freenames{\dropn{x}} & := & \{ x \}
\end{eqnarray*}

The bound names of a process, $\boundnames{P}$, are those names occurring in $P$
that are not free. For example, in $x?(y).0$, the name $x$ is free, while $y$ is bound.

\begin{mathpar}
  \inferrule* [lab=monoidal-laws] {} { P|Q \equiv Q|P \and P|0 \equiv P \and P|(Q|R) \equiv (P|Q)|R }
\end{mathpar}

\begin{mathpar}
  \inferrule* [lab=alpha-equivalence] {} { (x)P \equiv (y)P\{y/x\} \and y \not\in \freenames{P} }
\end{mathpar}

\begin{definition}
Then two processes, $P,Q$, are alpha-equivalent if $P = Q\{\vec{y}/\vec{x}\}$ for
some $\vec{x} \in \boundnames{Q},\vec{y} \in \boundnames{P}$, where $Q\{\vec{y}/\vec{x}\}$
denotes the capture-avoiding substitution of $\vec{y}$ for $\vec{x}$ in $Q$.
\end{definition}

\begin{definition}
  The {\em structural congruence} \cite{SangiorgiWalker} , $\equiv$,
  between processes is the least congruence containing
  alpha-equivalence, satisfying the abelian monoid laws
  (associativity, commutativity and $\pzero$ as identity) for parallel
  composition $|$ and for summation $+$.
\end{definition}

\subsection{Name equivalence}

We take name equivalence, written $\nameeq$, to be the smallest
equivalence relation generated by the following rules.

\begin{mathpar}
\inferrule*[lab=Quote-drop]
{ }
{ \quotep{@{x}} \nameeq x }

\inferrule*[lab=Struct-equiv]
{ P \scong Q }
{ \quotep{P} \nameeq \quotep{Q} }
\end{mathpar}

The astute reader will have noticed that the mutual recursion of names
and processes imposes a mutual recursion on alpha-equivalence and
structural equivalence via name-equivalence. Fortunately, all of this
works out pleasantly and we may calculate in the natural way, free of
concern. The reader interested in the details is referred to the
appendix \ref{appendix:rho_details}.

\subsection{Substitution}

We use $\Proc$ for the set of processes, $\QProc$ for the set of
names, and $\id{\{}\vec{y} / \vec{x} \id{\}}$ to denote partial maps,
$s : \QProc \rightarrow \QProc$. A map, $s$ lifts, uniquely, to a map
on process terms, $\widehat{s} : \Proc \rightarrow \Proc$ by the
following equations.

\begin{mathpar}
  (0) \psubstp{Q}{P} := 0 \\
  (R \juxtap S) \psubstp{Q}{P}
  :=    
  (R)\psubstp{Q}{P} \juxtap (S) \psubstp{Q}{P} \\
  (x?(y).R) \psubstp{Q}{P}    
  :=    
  (x)\substp{Q}{P} (z)\concat( (R \psubstn{z}{y}) \psubstp{Q}{P} ) \\
  (\lift{x}{R}) \psubstp{Q}{P}  
  :=
  \lift{(x)\substp{Q}{P}}{ R \psubstp{Q}{P} } \\
%   (\dropn{x})  \psubstp{Q}{P}       
%   := 
%   \left\{ 
%     \begin{array}{ccc} 
%       \dropn{\quotep{Q}} & & x \nameeq \quotep{P} \\
%       \dropn{x} & & otherwise \\
%     \end{array}
%   \right. 
  (\dropn{x})  \psubstp{Q}{P}       
  := 
  \left\{ 
    \begin{array}{ccc} 
      Q & & x \nameeq \quotep{P} \\
      \dropn{x} & & otherwise \\
    \end{array}
  \right.
\end{mathpar}
 

where

\begin{eqnarray}
  (x)\id{\{} \lpquote Q \rpquote / \lpquote P \rpquote \id{\}}            = 
  \left\{ 
    \begin{array}{ccc}
      \lpquote Q \rpquote & & x \nameeq \lpquote P \rpquote \\
      x & & otherwise \\
    \end{array}
  \right. \nonumber
\end{eqnarray}

and $z$ is chosen distinct from $\quotep{P}$, $\quotep{Q}$, the free
names in $Q$, and all the names in $R$. Our $\alpha$-equivalence will
be built in the standard way from this substitution.

\begin{remark}\label{rem:no_self_referential_names}
  One consequence of these definitions is that $\forall P. \quotep{P}
  \not\in \freenames{P}$.
\end{remark}

\subsection{ Dynamic quote: an example }

Anticipating something of what's to come, consider applying the
substitution, $\widehat{\id{\{}u / z \id{\}}}$, to the following pair
of processes, $\lift{w}{y!(z)}$ and $w[ \lpquote y!(z) \rpquote ]$.

\begin{eqnarray}
	\lift{w}{y!(z)}\widehat{\id{\{}u / z \id{\}}}
		& = &
		\lift{w}{y!(u)} \nonumber\\
	w[ \lpquote y!(z) \rpquote ] \widehat{ \id{\{}u / z \id{\}} }
		& = &
		w[ \lpquote y!(z) \rpquote ] \nonumber
\end{eqnarray}

Because the body of the process between quotes is impervious to
substitution, we get radically different answers. In fact, by
examining the first process in an input context,
e.g. $x?(z).\lift{w}{y!(z)}$, we see that the process under the lift
operator may be shaped by prefixed inputs binding a name inside it. In
this sense, the lift operator will be seen as a way to dynamically
construct processes before reifying them as names.

Finally equipped with these standard features we can present the
dynamics of the calculus.

\subsubsection{Operational semantics} 

Finally, we introduce the computational dynamics. What marks these
algebras as distinct from other more traditionally studied algebraic
structures, e.g. vector spaces or polynomial rings, is the manner in
which dynamics is captured. In traditional structures, dynamics is typically
expressed through morphisms between such structures, as in linear maps
between vector spaces or morphisms between rings. In algebras
associated with the semantics of computation, the dynamics is
expressed as part of the algebraic structure itself, through a
reduction reduction relation typically denoted by $\red$. Below, we
give a recursive presentation of this relation for the calculus used
in the encoding.

$\red \subseteq \pi \times \pi$
$\red : \pi \to \mathcal{P}(\pi)$

\begin{mathpar}
  \inferrule* [lab=Comm] { \textsf{match}( x_{src}, x_{trgt} ) } { x_{trgt}?(y)P \; | \; x_{src}!\langle {Q} \rangle \red P\{\quotep{Q}/y}\} }
  \and \\
  \inferrule* [lab=Par] {{P} \red {P}'} {{{P} | {Q}} \red {{P}' | {Q}}}
  \and
  \inferrule* [lab=Equiv]{{{P} \scong {P}'} \andalso {{P}' \red {Q}'} \andalso {{Q}' \scong {Q}}}{{P} \red {Q}}
\end{mathpar}

\begin{eqnarray*}
  match_{\equiv} (\quotep{P},\quotep{Q}) & := & P \equiv Q \\
  match_{\dagger}(\quotep{P},\quotep{Q}) & := & \forall R. P|Q \red^{*} R => R \red^{*} 0 \\
  match_{K}(\quotep{P},\quotep{Q}) & := & K \mbox{ for some context } K
\end{eqnarray*}

$u?(x)P | u!\langle Q \rangle \red P\{\quotep{Q}/x\}$

%We write $\wred$ for $\red^*$, and $P\red$ if $\exists Q $ such that $ P \red Q$.
We write $P\red$ if $\exists Q $ such that $ P \red Q$ and $P\not\red$, otherwise.

\section{Replication}

As mentioned before, it is known that replication (and hence
recursion) can be implemented in a higher-order process algebra
\cite{SangiorgiWalker}. As our first example of calculation with the
machinery thus far presented we give the construction explicitly in
the {\rhoc}.

\begin{eqnarray}
	D_{x} & := & \prefix{x}{y}{(\binpar{\outputp{x}{y}}{@{y}})} \nonumber\\
	\bangp_{x}{P} & := & \binpar{{x}!\langle{\binpar{D_{x}}{P}}\rangle}{D_{x}} \nonumber
\end{eqnarray}

\begin{eqnarray}
	\bangp_{x}{P} & & \nonumber\\
	=
	& {x}!\langle{(\prefix{x}{y}{(\outputp{x}{y} | @{y})) | P}}\rangle 
	      | \prefix{x}{y}{(\outputp{x}{y} | @{y})} & \nonumber\\
	\red
	& (\outputp{x}{y} | @{y})\substn{\quotep{(\prefix{x}{y}{(@{y} | \outputp{x}{y})) | P}}}{y} & \nonumber\\
	=
	& \outputp{x}{\quotep{(\prefix{x}{y}{(\outputp{x}{y} | @{y})) | P}}}
	  | {(\prefix{x}{y}{(\outputp{x}{y} | @{y})) | P}} & \nonumber\\
	\red
	& \ldots & \nonumber\\
	\red^*
	& P | P | \ldots & \nonumber
\end{eqnarray}

Of course, this encoding, as an implementation, runs away, unfolding
$\bangp{P}$ eagerly. A lazier and more implementable replication
operator, restricted to input-guarded processes, may be obtained as follows.

\begin{eqnarray}
\bangp{\prefix{u}{v}{P}} 
	:= 
	\binpar{\lift{x}{\prefix{u}{v}{(\binpar{D(x)}{P})}}}{D(x)} \nonumber
\end{eqnarray}

\begin{remark}
  Note that the lazier definition still does not deal with summation
  or mixed summation (i.e. sums over input and output). The reader is
  invited to construct definitions of replication that deal with these
  features. 

  Further, the definitions are parameterized in a name, $x$. Can you,
  gentle reader, make a definition that eliminates this parameter and
  guarantees no accidental interaction between the replication
  machinery and the process being replicated -- i.e. no accidental
  sharing of names used by the process to get its work done and the
  name(s) used by the replication to effect copying. This latter
  revision of the definition of replication is crucial to obtaining
  the expected identity $!!P \sim !P$.
\end{remark}

\begin{remark}\label{rem:paradoxical_combinator}
  The reader familiar with the lambda calculus will have noticed the
  similarity between $D$ and the paradoxical combinator.

  [Ed. note: the existence of this seems to suggest we have to be more
  restrictive on the set of processes and names we admit if we are to
  support no-cloning.]
\end{remark}

\subsubsection{Bisimulation}

The computational dynamics gives rise to another kind of equivalence,
the equivalence of computational behavior. As previously mentioned
this is typically captured \emph{via} some form of bisimulation.

% The notion we use in this paper is weak barbed bisimulation
% \cite{milner91polyadicpi}.

The notion we use in this paper is derived from weak barbed
bisimulation \cite{milner91polyadicpi}. 

\begin{definition}
An \emph{observation relation}, $\downarrow_{\mathcal N}$, over a set
of names, $\mathcal N$, is the smallest relation satisfying the rules
below.

\infrule[Out-barb]{y \in {\mathcal N}, \; x \nameeq y}
		  {\outputp{x}{v} \downarrow_{\mathcal N} x}
\infrule[Par-barb]{\mbox{$P\downarrow_{\mathcal N} x$ or $Q\downarrow_{\mathcal N} x$}}
		  {\binpar{P}{Q} \downarrow_{\mathcal N} x}

We write $P \Downarrow_{\mathcal N} x$ if there is $Q$ such that 
$P \wred Q$ and $Q \downarrow_{\mathcal N} x$.
\end{definition}

\begin{definition}
%\label{def.bbisim}
An  ${\mathcal N}$-\emph{barbed bisimulation} over a set of names, ${\mathcal N}$, is a symmetric binary relation 
${\mathcal S}_{\mathcal N}$ between agents such that $P\rel{S}_{\mathcal N}Q$ implies:
\begin{enumerate}
\item If $P \red P'$ then $Q \wred Q'$ and $P'\rel{S}_{\mathcal N} Q'$.
\item If $P\downarrow_{\mathcal N} x$, then $Q\Downarrow_{\mathcal N} x$.
\end{enumerate}
$P$ is ${\mathcal N}$-barbed bisimilar to $Q$, written
$P \wbbisim_{\mathcal N} Q$, if $P \rel{S}_{\mathcal N} Q$ for some ${\mathcal N}$-barbed bisimulation ${\mathcal S}_{\mathcal N}$.
\end{definition}

$\mathcal{R} \subseteq \pi \times \pi$

$P \mathcal{R} Q => \forall P'. P \red P' \Rightarrow \exists Q'. Q \red Q', P' \mathcal{R} Q'$

$P \vdash x \Rightarrow Q \vdash x$

\begin{mathpar}
  \inferrule*[lab=Out-barb]{x \nameeq y}{{y}!\langle{Q}\rangle \vdash x}
  \and
  \inferrule*[lab=Par-barb]{\mbox{$P\vdash x$ or $Q\vdash x$}}{\binpar{P}{Q} \vdash x}
\end{mathpar}

\subsubsection{Contexts}

One of the principle advantages of computational calculi like the
$\pi$-calculus is a well-defined notion of context,
contextual-equivalence and a correlation between
contextual-equivalence and notions of bisimulation. The notion of
context allows the decomposition of a process into (sub-)process and
its syntactic environment, its context. Thus, a context may be
thought of as a process with a ``hole'' (written $\Box$) in it. The
application of a context $M$ to a process $P$, written $M[P]$, is
tantamount to filling the hole in $M$ with $P$. In this paper we do
not need the full weight of this theory, but do make use of the notion
of context in the proof the main theorem. 

\begin{mathpar}
  \inferrule* [lab=summation] {} {{M_{M},M_{N}} \bc \Box \;|\; x.M_{A} \;|\; M_{M}+M_{N}}
  \and
  \inferrule* [lab=agent] {} {{M_{A}} \bc (\vec{x})M_{P} \;| \; \clift{P_0,\ldots,M_{P},\ldots,P_N}}
  \and \\
  \inferrule* [lab=process] {} {{M_{P}} \bc M_{N} \;| \;P|M_{P} }
\end{mathpar} 

\begin{mathpar}
  \inferrule* [lab=sychronization] {} {M_{N} \bc \Box \;|\; x?M_{F} \;|\; x!M_{C}}
  \and
  \inferrule* [lab=abstraction] {} {{M_{F}} \bc (x)M_{P} }
  \and
  \inferrule* [lab=concretion] {} {{M_{C}} \bc \langle M_{P} \rangle }
  \and \\
  \inferrule* [lab=process] {} {{M_{P}} \bc M_{N} \;| \;P|M_{P} }
\end{mathpar}

\begin{definition}[contextual application] Given a context $M$, and
  process $P$, we define the \emph{contextual application}, $M[P] :=
  M\{P/\Box\}$. That is, the contextual application of M to P is the
  substitution of $P$ for $\Box$ in $M$.
\end{definition}

$\meaningof{-} : L \to \mathcal{P}(\pi)$

\begin{mathpar}
  \inferrule* [lab=collection] {} {\meaningof{true} = \pi, \and \meaningof{~E} = \pi \setminus \meaningof{E}, \and \meaningof{E_{1} \& E_{2}} = \meaningof{E_{1}} \cap \meaningof{E_{2}}}
\end{mathpar}

\begin{mathpar}
  \inferrule* [lab=structure] {} {\meaningof{0} = \{ P \in \pi | P \equiv 0 \}, \and \\ \meaningof{E_1 | E_2} = \{ P \in \pi | P \equiv P_{1} | P_{2}, P_{1} \in \meaningof{E_{1}}, P_{2} \in \meaningof{E_2}\} }
\end{mathpar}

\begin{mathpar}
 \inferrule* [lab=behavior] {} {\meaningof{\langle a?b \rangle E} = \{ P \in \pi | P \equiv Q | u?(y)P', \\ \and \\\\ \and \\ \;\;\; u \in \meaningof{a}, \forall z.P'\{z/y\} \in \meaningof{E\{z/b\}}\}, \and \\ \meaningof{a!E} = \{ P \in \pi | P \equiv Q | x!\langle P' \rangle, x \in \meaningof{a} P' \in \meaningof{E}\} }
\end{mathpar}

\begin{mathpar}
 \inferrule* [lab=nominal] {} {\meaningof{\quotep{E}} = \{ \quotep{P} \in \quotep{\pi} | P \in \meaningof{E} \}, \and \meaningof{\quotep{P}} = \{ \quotep{Q} \in \quotep{\pi} | P \equiv Q \} \and \\ \meaningof{@\quotep{E}} = \{ P \in \pi | P \equiv @x, x \in \meaningof{E} \}}
\end{mathpar}

\begin{eqnarray*}
  \\
  \meaningof{-} : TS \to ST
\end{eqnarray*}

\begin{eqnarray*}
  \\
  L : TS \to ST
\end{eqnarray*}

\begin{eqnarray*}
  \\
  P \models E \iff P \in \meaningof{E}
\end{eqnarray*}

\begin{eqnarray*}
  P \approx_{L} Q \iff \forall E \in L. P \models E \iff Q \models E
\end{eqnarray*}

\begin{eqnarray*}
  P \approx_{K} Q
\end{eqnarray*}

\begin{eqnarray*}
  P \approx Q
\end{eqnarray*}

$\approx_{K} = \approx = \approx_{L}$

\subsubsection{Contextual duality}

Note that contexts extend the quotation operation to a family of
operations from processes to names. Given a context, $M$, we can
define a \emph{nominal context}, $\quotep{M}$ by $\quotep{M}[P] :=
\quotep{M[P]}$. To foreshadow what is to come we observe that these
operations enjoy a duality with processes very much like the duality
between vectors and maps from vectors to scalars.

Further, because the calculus is essentially higher-order, we have a
correspondence between contexts and processes. More specifically,
given a name $x$ and a context $M$ we can construct $M^{*}_{x}$ such
that 

\begin{mathpar}
  M^{*}_{x} | \lift{x}{P} \red M[P]
\end{mathpar}

namely,

\begin{mathpar}
  M^{*}_{x} := x?(u).M[\dropn{u}]
\end{mathpar}

The dependence of $M^{*}_{x}$ on a name makes it an abstraction, 

\begin{mathpar}
  M^{*} := (x)x?(u).M[\dropn{u}]
\end{mathpar}

\subsection{Additional notation}

It will sometimes be convenient to denote the process a name
quotes. We already have the notation $x = \quotep{P}$, but it will be
convenient to introduce an alternate notation, $\procn{x}$, when we
want to emphasize the connection to the use of the name. Note that, by
virtue of name equivalence, $\quotep{\procn{x}} \nameeq x$; so, the
notation is consistent with previous definitions.

Further, because names have structure it is possible to effect
substitutions on the basis of that structure. This means we need to
upgrade our notation for substitutions, which we accomplish by
adapting comprehension notation. Thus,

\begin{mathpar}
  P\{ y / x : x \in S \}
\end{mathpar}

is interpreted to mean the process derived from P by replacing (in a
capture-avoiding manner) each occurrence of $x$ in $S$ by $y$. For example,

\begin{mathpar}
  P\{ \quotep{\procn{x}|\procn{x}} / x : x \in \freenames{P} \}
\end{mathpar}

will replace each (occurrence) of a free name $x$ in $P$ by
$\quotep{\procn{x}|\procn{x}}$.

Also, we will avail ourselves of the notation $x^{L}$ and $x^{R}$ to
denote injections of a name into disjoint copies of the name
space. There are numerous ways to accomplish this. One example can be
found in \cite{MeredithR05}. This notation overloads to vectors of
names: $\vec{x}^{\pi} := (x_{i}^{\pi} \; : \; 0 \leq i < |\vec{x}| )$ where $\pi \in \{L,R\}$.

We also use $P^{\Box} := P|\Box$.

In \cite{MeredithR05} an interpretation of the new operator is
given. It turns out that there are several possible interpretations
all enjoying the requisite algebraic properties of the operator (see
\cite{milner91polyadicpi}). We will therefore make liberal use of
$(\nu\; \vec{x})P$.

% subsection the_syntax_and_semantics_of_the_notation_system (end)   

\input{qm2pi.qmops} 

\input{qm2pi.sterngerlach} 

\input{qm2pi.metric} 

% section concurrent_process_calculi (end)

%\input{qm2pi.proofsketch}

% section proof sketch (end)

%\input{qm2pi.slviaknots} 

% section spatial logic via knots (end)

\input{qm2pi.conclusion}

% section conclusion (end)

%\input{qm2pi.dtcodes} 

% section wiring algorithm (end)

\input{qm2pi.ack} 

% section acknowledgments (end)

\newpage


\bibliographystyle{plain}   
\bibliography{../../biblios/main.bib}

\input{qm2pi.rhodetails}

\end{document}



% section proof sketch (end)

%\section{Unlikely characters: spatial logic for
  knots}\label{sub:characteristic_formulae} % (fold)

Associated to the mobile process calculi are a family of logics known
as the Hennessy-Milner logics. These logics typically enjoy a
semantics interpreting formulae as sets of processes that when
factored through the encoding outlined above allows an identification
of classes of knots with logical formulae. In the context of this
encoding the sub-family known as the spatial logics \cite{CairesC03}
\cite{CairesC04} \cite{Caires04} are of particular interest providing
several important features for expressing and reasoning about
properties (i.e. classes) of knots. We hint here at how this may be done.

%\begin{description}
%\item [structural connectives] 
\subsubsection{Structural connectives} The spatial logics enjoy
structural connectives corresponding, at the logical level, to the
parallel composition ($P | Q$) and new name ($(\nu \; x)P$)
connectives for processes. As illustrated in the examples below, these
connectives are extremely expressive given the shape of our encoding.
%\item [decideable satisfaction]

\subsubsection{Decideable satisfaction}
In \cite{Caires04} the satisfaction relation is shown to be decideable
for a rich class of processes. It further turns out that the image of
the our encoding is a proper subset of that class. This result
provides the basis for an algorithm by which to search for knots
enjoying a given property.
%\item [characteristic formulae]

\subsubsection{Characteristic formulae}
In the same paper \cite{Caires04} , Caires presents a means of calculating
characteristic formulae, selecting equivalence classes of processes
up to a pre--specified depth limit on the support set of names. Composed with our
encoding, this characteristic formula can be used to select
characteristic formulae for knots.
%\end{description}

\subsubsection{Spatial logic formulae}

The grammar below (segmented for comprehension) summarizes the syntax
of spatial logic formulae. We employ illustrative examples in the
sequel to provide an intuitive understanding of their meaning
referring the reader to \cite{Caires04} for a more detailed explication
of the semantics.

\begin{mathpar}
  \inferrule* [lab=boolean] {} {{A,B} \bc T \;|\; \neg A \;|\; A \wedge B \;|\; \eta = \eta'}
  \and
  \inferrule* [lab=spatial] {} {|\; \pzero \;|\; A | B \;|\; x \text{\textregistered} A \;|\; \forall x . A \;|\;  H x . A}
  \and
  \inferrule* [lab=behavioral] {} {|\; \alpha . A}
  \and 
  \inferrule* [lab=recursion] {} {|\; X(\vec{u}) \;|\; \mu X(\vec{u}) . A}
  \and
  \inferrule* [lab=action] {} {\alpha \bc \langle x?(\vec{y}) \rangle \;|\; \langle x!(\vec{y}) \rangle \;|\; \langle \tau \rangle}
  \and 
  \inferrule* [lab=name] {} {\eta \bc x \;|\; \tau}
\end{mathpar} 

% subsection characteristic_formulae (end)   	 

\subsection{Example formulae}\label{sub:example_formulae_} % (fold)

\subsubsection{Crossing as formula.}
% 
% \begin{align*}
%   \frac{d}{dx} \sin x &= \cos x 
%   & \frac{d}{dx} e^x &= e^x \\
%   \frac{d}{dx} \cos x &= - \sin x 
%   & \frac{d}{dx} \log x &= \frac{1}{x} \\
% \end{align*} 

\begin{align*}
 \mu C(x_{0},x_{1},y_{0},y_{1},u).&(\langle x_{0}?(z) \rangle(\langle u! \rangle\langle y_{1}!z \rangle C(x_{0},x_{1},y_{0},y_{1},u)) & \\
  & \wedge \langle y_{1}?(z) \rangle (\langle u! \rangle \langle x_{0}!z \rangle C(x_{0},x_{1},y_{0},y_{1},u)) & \\
  & \wedge \langle x_{1}?(z) \rangle (\langle u? \rangle \langle y_{0}!z \rangle C(x_{0},x_{1},y_{0},y_{1},u)) & \\
  & \wedge \langle y_{0}?(z) \rangle (\langle u? \rangle \langle x_{1}!z \rangle C(x_{0},x_{1},y_{0},y_{1},u))) &
\end{align*}

The lexicographical similarity between the shape of this formulae and
the shape of definition of the process representing a crossing reveals
the intuitive meaning of this formulae. It describes the capabilities
of a process that has the right to represent a crossing. For example
it picks out processes that may perform an input on the port $x_0$ in
its initial menu of capabilities. What differentiates the formula
from the process, however, is that the crossing process is the
smallest candidate to satisfy the formula. Infinitely many other
processes -- with internal behavior hidden behind this interface, so
to speak -- also satisfy this formula. Even this simple formula,
then, can be seen to open a new view onto knots, providing a
computational interpretation of \emph{virtual} knots.

Note that this formula is derived by hand. A similar formula can be
derived by employing Caires' calculation of characteristic formula
\cite{Caires04} to the process representing a crossing. In light of
this discussion, we let
$\meaningof{C}_{\phi}(x0,x1,y0,y1,u)$ denote a formula specifying the
dynamics we wish to capture of a crossing. To guarantee we preserve
the shape of the interface and minimal semantics we demand that
$\meaningof{C}_{\phi}(x0,x1,y0,y1,u) \Rightarrow
\textbf{C}(x0,x1,y0,y1,u)$ where $\textbf{C}(x0,x1,y0,y1,u)$ denotes
the formula above.
                            
\subsubsection{Crossing number constraints.}
The moral content of the context lemma (Lemma \ref{context}) is that the notion of
``locality'' in the Reidemeister moves is effectively captured by the
parallel composition operator of the process calculus. This intuition
extends through the logic. Given a formula,
$\meaningof{C}_{\phi}(x0,x1,y0,y1,u)$, we can use the structural
connectives to specify constraints on crossing numbers, such as at
least $n$ crossings, or exactly $n$ crossings.
\begin{mathpar}
  \inferrule* [lab=at-least-n] {} { K^{\geq n}_{\phi}(\vec{xs},\vec{ys}) := \Pi_{i=0}^{n-1} Hu . \meaningof{C}_{\phi}(xs_i,ys_i,u) | T }
  \and 
  \inferrule* [lab=exactly-n] {} { K^{= n}_{\phi}(\vec{xs},\vec{ys}) := \Pi_{i=0}^{n-1} Hu . \meaningof{C}_{\phi}(xs_i,ys_i,u) | \neg (\forall x_0,y_0,x_1,y_1,u . \meaningof{C}_{\phi}(x_0,y_0,x_1,y_1,u) | T) }
\end{mathpar}

To round out this section, recall that the encoding of an $n$-crossing
knot decomposes into a parallel composition of $n$ \emph{copies} of a
crossing process together with a wiring harness. To specify different
knot classes with the same crossing number amounts to specifying
logical constraints on the wiring harness. In the interest of space,
we defer examples to a forthcoming paper. Suffice it to say that both
the conditions ``alternating knot'' and ``contains the tangle
corresponding to 5/3'' are expressible. For example, it is possible to
calculate the characteristic formula of a process corresponding to the
tangle 5/3 and conjoin it into the classifying formula via the
composition connective of the logic.

Finally, we wish to observe that it is entirely within reason to
contemplate a more domain-specific version of spatial logic tailored
to the shape of processes in the image of the encoding. Such a
domain-specific logic would have a better claim to the title formal
language of knot properties.

% subsection example_formulae_ (end)

% section knots_as_processes (end) 

% section spatial logic via knots (end)

\section{Conclusions and future work}

\paragraph{Testing physical space}
You, gentle reader, may wonder why of all the theorems to be proved
given this set up we pick the one above. In some sense it's hardly
central to quantum mechanics. We see it as central in the sense that
it firmly establishes a notion of physical space arising from a notion
of the equivalence of behavior. Relating bisimulation to a metric is a
big step forward, but one is faced with interpreting the relationship
of that metric space to something more physical. Quantum mechanical
notions of ``physical'' space are still far from intuitive, but by
relating this idea of distance as testing to calculations that predict
physical circumstances we are making a not insignificant step forward
toward an understanding of the physical space we inhabit as
essentially dynamic.

\paragraph{Effectivity and simulation}
One of the observations we have yet to make is that the entire program
spelled out here is effective. We have built various interpreters for
the reflective calculus at work in this interpretation. In principle,
then, we can simulate quantum mechanics on a computer. The place where
the simulation may lose fidelity is the infinitely branching summation
for the annihilator.

In this connection i also want to point out that the evaluation style
calculation of the inner product puts the non-determinism of the
summation right at the heart of measurement. This suggests that
Milner's original reduction-based formulation of the dynamics of his
calculi in terms of sums was not just notationally suggestive of a
notion of measure-and-continue but captured some significant part of
the physics.

\paragraph{Quantum continuations}
In light of this last observation i want to point out that the
predominant account of quantum mechanics is missing a key aspect of a
truly compositional story of the physical situation. In a real lab,
when a measurement is made the observation can be made to feed into
another device that then makes another measurement conditioned on the
results of the first. This means that after the superposition was
collapsed the entire experimental set up remained in
superposition. While QM offers a means of writing this down it doesn't
quite line up well with the well-trodden formulation of computation
and continuation that we see so succinctly expressed in Milner's
calculi. This suggests that there might be advantages to this account
of dynamics waiting to be explored.

\paragraph{Quantum logic}
In this connection, we also note that by virtue of having the
Hennessy-Milner construction, we can pull the construction through the
interpretation of QM. This gives us a natural candidate for a quantum
logic that enjoys an extremely tight connection with it's domain of
interpretation, making the construction much less ad hoc (rather it is
the image of functor!).

\paragraph{Quantum probabiity}
i have questions about the basis of the interpretation of inner
product as probability amplitude. In particular, using which
axiomatization of probability theory does the notion of probability
amplitude earn the right to be so dubbed? In other words, where is the
proof that the operation for calculating a probability amplitude (and
then squaring) satisfies the axioms of what it means to calculate a
probability? Even if such a proof exists (i have yet to find it in the
literature), i wonder if it might not be possible to turn things on
their heads. Can we view the calculation of the probability amplitude
as an axiomatization of probability? If so, then the definition we
give for calculating probability amplitude may provide the basis for
an \emph{effective} theory of probability.

\paragraph{Quantum vs ``biological'' information}
Finally, i want to conclude with a more philosophical observation. At
a recent workshop in which QM was a predominant topic i noticed
something about quantum information. The speaker was giving a riveting
discussion of axiomatic QM and showing how properties of ``no
cloning'' and ``no deleting'' emerged as consequences of the
axiomatization. Theorems of this form are necessary to give us a sense
of confidence that our axioms characterize the physical theory. What
struck me, though, was that if quantum information is neither erasable
nor replicable it is markedly different from \emph{life}. Two of the
things we know about life is that

\begin{itemize}
  \item it ends;
  \item to gain some measure of persistence, to transcend it's
    finitude it is imminently copyable.
\end{itemize}

Both of these qualities are summarized succinctly in the aphorism: all
flesh is grass. For me these two kinds of ``information'' -- call them
quantum and biological -- are end points on a spectrum of strategies
for persistence. At one end, we have those curious entities that enjoy
uniqueness and permanence; at the other, we have those who in the face
of a certain end and an uncertain present make a go of passing
something on. To me one of the more remarkable aspects of the latter
strategy is that in the presence of noise (and certain features of
copying) we get a kind of dynamism, a chance for improvement against a
given persistent condition.

% subsection other_calculi_other_bisimulations_and_geometry_as_behavior (end)




% section conclusion (end)

%\documentclass[12pt]{llncs}
%\documentclass{jktr}

\usepackage[pdftex]{hyperref}                   
\usepackage {listings}
\usepackage {mathpartir}
\usepackage{bcprules}
%\usepackage{listings}
                       
\usepackage{graphicx} 
%\usepackage[margins=2.5cm,nohead,nofoot]{geometry}
%\usepackage{geometry}
\usepackage{amsfonts}
\usepackage{amstext}
\usepackage{latexsym}
\usepackage{amssymb}
\usepackage{color}


%\include{myPreamble}
\include{qm2pi.local} 

%\ifpdf
%\usepackage[pdftex]{graphicx}
%\else
%\usepackage{graphicx}
%\fi

 % \ifpdf
%  \usepackage{pdfsync}
%  \if


%\title{Brief Article}
%\author{David F. Snyder}
%\author{L.G. Meredith}

%\address{Dept. of Math., Texas State University--San Marcos, San Marcos, TX 78666}
       
\pagestyle{empty}


\begin{document}

\lstset{language=[Objective]Caml,frame=shadowbox}

\input{qm2pi.front}

% section front matter (end)

\input{qm2pi.intro} 
 
% section introduction (end)

% \input{qm2pi.knotations} 

% section notation (end)

\input{qm2pi.process.calculi} 

% section concurrent_process_calculi_and_spatial_logics_ (end)
    
%\input{qm2pi.knots2pi} 

%\input{qm2pi.trefoil} 

%\input{qm2pi.mainthm} 

% subsection basic_interpretation (end)

%\input{qm2pi.rho.presentation} 
\subsection{The syntax and semantics of the notation system}\label{sub:the_syntax_and_semantics_of_the_notation_system} % (fold)

We now summarize a technical presentation of the calculus that
embodies our theory of dynamics. The typical presentation of such a
calculus follows the style of giving generators and relations on
them. The grammar, below, describing term constructors, freely
generates the set of processes, $\Proc$. This set is then quotiented
by a relation known as structural congruence and it is over this set
that the notion of dynamics is expressed. This presentation is
essentially that of \cite{MeredithR05} with the addition of
polyadicity and summation. For readability we have relegated some of
the technical subtleties to an appendix.

\subsubsection{Process grammar}\label{subsub:process_grammar}

\begin{mathpar}
  \inferrule* [lab=synchronization] {} {{M} \bc \pzero \;|\; x?F \;|\; x!C }
  \and
  \inferrule* [lab=abstraction] {} {{F} \bc (x)P}
  \and
  \inferrule* [lab=concretion] {} {{C} \bc \langle Q \rangle}
  \and
  \inferrule* [lab=process] {} {{P,Q} \bc M \;| \;P|Q \;|\; @{x}}
  \and
  \inferrule* [lab=name] {} {{x} \bc \quotep{P}}
\end{mathpar} 

Note that $\vec{x}$ (resp. $\vec{P}$) denotes a vector of names
(resp. processes) of length $|\vec{x}|$ (resp. $|\vec{P}|$). We adopt
the following useful abbreviations.

\begin{mathpar}
   x?(\vec{y}).P := x.(\vec{y})P \and  x\clift{\vec{P}} := x.\clift{\vec{P}}
   \and x!(y) := \lift{x}{\dropn{y}}
   \and \Pi_{i=0}^{n-1}P_i := P_0 | \ldots | P_{n-1}
\end{mathpar}

\subsubsection{Structural congruence}

\paragraph{Free and bound names and alpha-equivalence.} At the
core of structural equivalence is alpha-equivalence which identifies
process that are the same up to a change of variable. Formally, we
recognize the distinction between free and bound names. The free names
of a process, $\freenames{P}$, may be calculated recursively as
follows:

\begin{mathpar}
\freenames{\pzero} := \emptyset
  \and \\
  \freenames{x?(y).P} := \{ x \} \cup (\freenames{P} \setminus \{ y \})
  \and 
  \freenames{x!\langle P \rangle} := \{ x \} \cup \{ P \} 
  \and \\
  \freenames{P|Q} := \freenames{P} \cup \freenames{Q}
  \and \\
  \freenames{@{x}} := \{ x \}
\end{mathpar}

$\pi$
$\quotep{\pi}$

$\freenames{-} : \pi \to \mathcal{P}(\quotep{\pi})$

\begin{eqnarray*}
  \freenames{\pzero} & := & \emptyset \\
  \freenames{x?(y).P} & := & \{ x \} \cup (\freenames{P} \setminus \{ y \}) \\
  \freenames{x!\langle P \rangle} & := & \{ x \} \cup \{ P \} \\
  \freenames{P|Q} & := & \freenames{P} \cup \freenames{Q} \\
  \freenames{\dropn{x}} & := & \{ x \}
\end{eqnarray*}

The bound names of a process, $\boundnames{P}$, are those names occurring in $P$
that are not free. For example, in $x?(y).0$, the name $x$ is free, while $y$ is bound.

\begin{mathpar}
  \inferrule* [lab=monoidal-laws] {} { P|Q \equiv Q|P \and P|0 \equiv P \and P|(Q|R) \equiv (P|Q)|R }
\end{mathpar}

\begin{mathpar}
  \inferrule* [lab=alpha-equivalence] {} { (x)P \equiv (y)P\{y/x\} \and y \not\in \freenames{P} }
\end{mathpar}

\begin{definition}
Then two processes, $P,Q$, are alpha-equivalent if $P = Q\{\vec{y}/\vec{x}\}$ for
some $\vec{x} \in \boundnames{Q},\vec{y} \in \boundnames{P}$, where $Q\{\vec{y}/\vec{x}\}$
denotes the capture-avoiding substitution of $\vec{y}$ for $\vec{x}$ in $Q$.
\end{definition}

\begin{definition}
  The {\em structural congruence} \cite{SangiorgiWalker} , $\equiv$,
  between processes is the least congruence containing
  alpha-equivalence, satisfying the abelian monoid laws
  (associativity, commutativity and $\pzero$ as identity) for parallel
  composition $|$ and for summation $+$.
\end{definition}

\subsection{Name equivalence}

We take name equivalence, written $\nameeq$, to be the smallest
equivalence relation generated by the following rules.

\begin{mathpar}
\inferrule*[lab=Quote-drop]
{ }
{ \quotep{@{x}} \nameeq x }

\inferrule*[lab=Struct-equiv]
{ P \scong Q }
{ \quotep{P} \nameeq \quotep{Q} }
\end{mathpar}

The astute reader will have noticed that the mutual recursion of names
and processes imposes a mutual recursion on alpha-equivalence and
structural equivalence via name-equivalence. Fortunately, all of this
works out pleasantly and we may calculate in the natural way, free of
concern. The reader interested in the details is referred to the
appendix \ref{appendix:rho_details}.

\subsection{Substitution}

We use $\Proc$ for the set of processes, $\QProc$ for the set of
names, and $\id{\{}\vec{y} / \vec{x} \id{\}}$ to denote partial maps,
$s : \QProc \rightarrow \QProc$. A map, $s$ lifts, uniquely, to a map
on process terms, $\widehat{s} : \Proc \rightarrow \Proc$ by the
following equations.

\begin{mathpar}
  (0) \psubstp{Q}{P} := 0 \\
  (R \juxtap S) \psubstp{Q}{P}
  :=    
  (R)\psubstp{Q}{P} \juxtap (S) \psubstp{Q}{P} \\
  (x?(y).R) \psubstp{Q}{P}    
  :=    
  (x)\substp{Q}{P} (z)\concat( (R \psubstn{z}{y}) \psubstp{Q}{P} ) \\
  (\lift{x}{R}) \psubstp{Q}{P}  
  :=
  \lift{(x)\substp{Q}{P}}{ R \psubstp{Q}{P} } \\
%   (\dropn{x})  \psubstp{Q}{P}       
%   := 
%   \left\{ 
%     \begin{array}{ccc} 
%       \dropn{\quotep{Q}} & & x \nameeq \quotep{P} \\
%       \dropn{x} & & otherwise \\
%     \end{array}
%   \right. 
  (\dropn{x})  \psubstp{Q}{P}       
  := 
  \left\{ 
    \begin{array}{ccc} 
      Q & & x \nameeq \quotep{P} \\
      \dropn{x} & & otherwise \\
    \end{array}
  \right.
\end{mathpar}
 

where

\begin{eqnarray}
  (x)\id{\{} \lpquote Q \rpquote / \lpquote P \rpquote \id{\}}            = 
  \left\{ 
    \begin{array}{ccc}
      \lpquote Q \rpquote & & x \nameeq \lpquote P \rpquote \\
      x & & otherwise \\
    \end{array}
  \right. \nonumber
\end{eqnarray}

and $z$ is chosen distinct from $\quotep{P}$, $\quotep{Q}$, the free
names in $Q$, and all the names in $R$. Our $\alpha$-equivalence will
be built in the standard way from this substitution.

\begin{remark}\label{rem:no_self_referential_names}
  One consequence of these definitions is that $\forall P. \quotep{P}
  \not\in \freenames{P}$.
\end{remark}

\subsection{ Dynamic quote: an example }

Anticipating something of what's to come, consider applying the
substitution, $\widehat{\id{\{}u / z \id{\}}}$, to the following pair
of processes, $\lift{w}{y!(z)}$ and $w[ \lpquote y!(z) \rpquote ]$.

\begin{eqnarray}
	\lift{w}{y!(z)}\widehat{\id{\{}u / z \id{\}}}
		& = &
		\lift{w}{y!(u)} \nonumber\\
	w[ \lpquote y!(z) \rpquote ] \widehat{ \id{\{}u / z \id{\}} }
		& = &
		w[ \lpquote y!(z) \rpquote ] \nonumber
\end{eqnarray}

Because the body of the process between quotes is impervious to
substitution, we get radically different answers. In fact, by
examining the first process in an input context,
e.g. $x?(z).\lift{w}{y!(z)}$, we see that the process under the lift
operator may be shaped by prefixed inputs binding a name inside it. In
this sense, the lift operator will be seen as a way to dynamically
construct processes before reifying them as names.

Finally equipped with these standard features we can present the
dynamics of the calculus.

\subsubsection{Operational semantics} 

Finally, we introduce the computational dynamics. What marks these
algebras as distinct from other more traditionally studied algebraic
structures, e.g. vector spaces or polynomial rings, is the manner in
which dynamics is captured. In traditional structures, dynamics is typically
expressed through morphisms between such structures, as in linear maps
between vector spaces or morphisms between rings. In algebras
associated with the semantics of computation, the dynamics is
expressed as part of the algebraic structure itself, through a
reduction reduction relation typically denoted by $\red$. Below, we
give a recursive presentation of this relation for the calculus used
in the encoding.

$\red \subseteq \pi \times \pi$
$\red : \pi \to \mathcal{P}(\pi)$

\begin{mathpar}
  \inferrule* [lab=Comm] { \textsf{match}( x_{src}, x_{trgt} ) } { x_{trgt}?(y)P \; | \; x_{src}!\langle {Q} \rangle \red P\{\quotep{Q}/y}\} }
  \and \\
  \inferrule* [lab=Par] {{P} \red {P}'} {{{P} | {Q}} \red {{P}' | {Q}}}
  \and
  \inferrule* [lab=Equiv]{{{P} \scong {P}'} \andalso {{P}' \red {Q}'} \andalso {{Q}' \scong {Q}}}{{P} \red {Q}}
\end{mathpar}

\begin{eqnarray*}
  match_{\equiv} (\quotep{P},\quotep{Q}) & := & P \equiv Q \\
  match_{\dagger}(\quotep{P},\quotep{Q}) & := & \forall R. P|Q \red^{*} R => R \red^{*} 0 \\
  match_{K}(\quotep{P},\quotep{Q}) & := & K \mbox{ for some context } K
\end{eqnarray*}

$u?(x)P | u!\langle Q \rangle \red P\{\quotep{Q}/x\}$

%We write $\wred$ for $\red^*$, and $P\red$ if $\exists Q $ such that $ P \red Q$.
We write $P\red$ if $\exists Q $ such that $ P \red Q$ and $P\not\red$, otherwise.

\section{Replication}

As mentioned before, it is known that replication (and hence
recursion) can be implemented in a higher-order process algebra
\cite{SangiorgiWalker}. As our first example of calculation with the
machinery thus far presented we give the construction explicitly in
the {\rhoc}.

\begin{eqnarray}
	D_{x} & := & \prefix{x}{y}{(\binpar{\outputp{x}{y}}{@{y}})} \nonumber\\
	\bangp_{x}{P} & := & \binpar{{x}!\langle{\binpar{D_{x}}{P}}\rangle}{D_{x}} \nonumber
\end{eqnarray}

\begin{eqnarray}
	\bangp_{x}{P} & & \nonumber\\
	=
	& {x}!\langle{(\prefix{x}{y}{(\outputp{x}{y} | @{y})) | P}}\rangle 
	      | \prefix{x}{y}{(\outputp{x}{y} | @{y})} & \nonumber\\
	\red
	& (\outputp{x}{y} | @{y})\substn{\quotep{(\prefix{x}{y}{(@{y} | \outputp{x}{y})) | P}}}{y} & \nonumber\\
	=
	& \outputp{x}{\quotep{(\prefix{x}{y}{(\outputp{x}{y} | @{y})) | P}}}
	  | {(\prefix{x}{y}{(\outputp{x}{y} | @{y})) | P}} & \nonumber\\
	\red
	& \ldots & \nonumber\\
	\red^*
	& P | P | \ldots & \nonumber
\end{eqnarray}

Of course, this encoding, as an implementation, runs away, unfolding
$\bangp{P}$ eagerly. A lazier and more implementable replication
operator, restricted to input-guarded processes, may be obtained as follows.

\begin{eqnarray}
\bangp{\prefix{u}{v}{P}} 
	:= 
	\binpar{\lift{x}{\prefix{u}{v}{(\binpar{D(x)}{P})}}}{D(x)} \nonumber
\end{eqnarray}

\begin{remark}
  Note that the lazier definition still does not deal with summation
  or mixed summation (i.e. sums over input and output). The reader is
  invited to construct definitions of replication that deal with these
  features. 

  Further, the definitions are parameterized in a name, $x$. Can you,
  gentle reader, make a definition that eliminates this parameter and
  guarantees no accidental interaction between the replication
  machinery and the process being replicated -- i.e. no accidental
  sharing of names used by the process to get its work done and the
  name(s) used by the replication to effect copying. This latter
  revision of the definition of replication is crucial to obtaining
  the expected identity $!!P \sim !P$.
\end{remark}

\begin{remark}\label{rem:paradoxical_combinator}
  The reader familiar with the lambda calculus will have noticed the
  similarity between $D$ and the paradoxical combinator.

  [Ed. note: the existence of this seems to suggest we have to be more
  restrictive on the set of processes and names we admit if we are to
  support no-cloning.]
\end{remark}

\subsubsection{Bisimulation}

The computational dynamics gives rise to another kind of equivalence,
the equivalence of computational behavior. As previously mentioned
this is typically captured \emph{via} some form of bisimulation.

% The notion we use in this paper is weak barbed bisimulation
% \cite{milner91polyadicpi}.

The notion we use in this paper is derived from weak barbed
bisimulation \cite{milner91polyadicpi}. 

\begin{definition}
An \emph{observation relation}, $\downarrow_{\mathcal N}$, over a set
of names, $\mathcal N$, is the smallest relation satisfying the rules
below.

\infrule[Out-barb]{y \in {\mathcal N}, \; x \nameeq y}
		  {\outputp{x}{v} \downarrow_{\mathcal N} x}
\infrule[Par-barb]{\mbox{$P\downarrow_{\mathcal N} x$ or $Q\downarrow_{\mathcal N} x$}}
		  {\binpar{P}{Q} \downarrow_{\mathcal N} x}

We write $P \Downarrow_{\mathcal N} x$ if there is $Q$ such that 
$P \wred Q$ and $Q \downarrow_{\mathcal N} x$.
\end{definition}

\begin{definition}
%\label{def.bbisim}
An  ${\mathcal N}$-\emph{barbed bisimulation} over a set of names, ${\mathcal N}$, is a symmetric binary relation 
${\mathcal S}_{\mathcal N}$ between agents such that $P\rel{S}_{\mathcal N}Q$ implies:
\begin{enumerate}
\item If $P \red P'$ then $Q \wred Q'$ and $P'\rel{S}_{\mathcal N} Q'$.
\item If $P\downarrow_{\mathcal N} x$, then $Q\Downarrow_{\mathcal N} x$.
\end{enumerate}
$P$ is ${\mathcal N}$-barbed bisimilar to $Q$, written
$P \wbbisim_{\mathcal N} Q$, if $P \rel{S}_{\mathcal N} Q$ for some ${\mathcal N}$-barbed bisimulation ${\mathcal S}_{\mathcal N}$.
\end{definition}

$\mathcal{R} \subseteq \pi \times \pi$

$P \mathcal{R} Q => \forall P'. P \red P' \Rightarrow \exists Q'. Q \red Q', P' \mathcal{R} Q'$

$P \vdash x \Rightarrow Q \vdash x$

\begin{mathpar}
  \inferrule*[lab=Out-barb]{x \nameeq y}{{y}!\langle{Q}\rangle \vdash x}
  \and
  \inferrule*[lab=Par-barb]{\mbox{$P\vdash x$ or $Q\vdash x$}}{\binpar{P}{Q} \vdash x}
\end{mathpar}

\subsubsection{Contexts}

One of the principle advantages of computational calculi like the
$\pi$-calculus is a well-defined notion of context,
contextual-equivalence and a correlation between
contextual-equivalence and notions of bisimulation. The notion of
context allows the decomposition of a process into (sub-)process and
its syntactic environment, its context. Thus, a context may be
thought of as a process with a ``hole'' (written $\Box$) in it. The
application of a context $M$ to a process $P$, written $M[P]$, is
tantamount to filling the hole in $M$ with $P$. In this paper we do
not need the full weight of this theory, but do make use of the notion
of context in the proof the main theorem. 

\begin{mathpar}
  \inferrule* [lab=summation] {} {{M_{M},M_{N}} \bc \Box \;|\; x.M_{A} \;|\; M_{M}+M_{N}}
  \and
  \inferrule* [lab=agent] {} {{M_{A}} \bc (\vec{x})M_{P} \;| \; \clift{P_0,\ldots,M_{P},\ldots,P_N}}
  \and \\
  \inferrule* [lab=process] {} {{M_{P}} \bc M_{N} \;| \;P|M_{P} }
\end{mathpar} 

\begin{mathpar}
  \inferrule* [lab=sychronization] {} {M_{N} \bc \Box \;|\; x?M_{F} \;|\; x!M_{C}}
  \and
  \inferrule* [lab=abstraction] {} {{M_{F}} \bc (x)M_{P} }
  \and
  \inferrule* [lab=concretion] {} {{M_{C}} \bc \langle M_{P} \rangle }
  \and \\
  \inferrule* [lab=process] {} {{M_{P}} \bc M_{N} \;| \;P|M_{P} }
\end{mathpar}

\begin{definition}[contextual application] Given a context $M$, and
  process $P$, we define the \emph{contextual application}, $M[P] :=
  M\{P/\Box\}$. That is, the contextual application of M to P is the
  substitution of $P$ for $\Box$ in $M$.
\end{definition}

$\meaningof{-} : L \to \mathcal{P}(\pi)$

\begin{mathpar}
  \inferrule* [lab=collection] {} {\meaningof{true} = \pi, \and \meaningof{~E} = \pi \setminus \meaningof{E}, \and \meaningof{E_{1} \& E_{2}} = \meaningof{E_{1}} \cap \meaningof{E_{2}}}
\end{mathpar}

\begin{mathpar}
  \inferrule* [lab=structure] {} {\meaningof{0} = \{ P \in \pi | P \equiv 0 \}, \and \\ \meaningof{E_1 | E_2} = \{ P \in \pi | P \equiv P_{1} | P_{2}, P_{1} \in \meaningof{E_{1}}, P_{2} \in \meaningof{E_2}\} }
\end{mathpar}

\begin{mathpar}
 \inferrule* [lab=behavior] {} {\meaningof{\langle a?b \rangle E} = \{ P \in \pi | P \equiv Q | u?(y)P', \\ \and \\\\ \and \\ \;\;\; u \in \meaningof{a}, \forall z.P'\{z/y\} \in \meaningof{E\{z/b\}}\}, \and \\ \meaningof{a!E} = \{ P \in \pi | P \equiv Q | x!\langle P' \rangle, x \in \meaningof{a} P' \in \meaningof{E}\} }
\end{mathpar}

\begin{mathpar}
 \inferrule* [lab=nominal] {} {\meaningof{\quotep{E}} = \{ \quotep{P} \in \quotep{\pi} | P \in \meaningof{E} \}, \and \meaningof{\quotep{P}} = \{ \quotep{Q} \in \quotep{\pi} | P \equiv Q \} \and \\ \meaningof{@\quotep{E}} = \{ P \in \pi | P \equiv @x, x \in \meaningof{E} \}}
\end{mathpar}

\begin{eqnarray*}
  \\
  \meaningof{-} : TS \to ST
\end{eqnarray*}

\begin{eqnarray*}
  \\
  L : TS \to ST
\end{eqnarray*}

\begin{eqnarray*}
  \\
  P \models E \iff P \in \meaningof{E}
\end{eqnarray*}

\begin{eqnarray*}
  P \approx_{L} Q \iff \forall E \in L. P \models E \iff Q \models E
\end{eqnarray*}

\begin{eqnarray*}
  P \approx_{K} Q
\end{eqnarray*}

\begin{eqnarray*}
  P \approx Q
\end{eqnarray*}

$\approx_{K} = \approx = \approx_{L}$

\subsubsection{Contextual duality}

Note that contexts extend the quotation operation to a family of
operations from processes to names. Given a context, $M$, we can
define a \emph{nominal context}, $\quotep{M}$ by $\quotep{M}[P] :=
\quotep{M[P]}$. To foreshadow what is to come we observe that these
operations enjoy a duality with processes very much like the duality
between vectors and maps from vectors to scalars.

Further, because the calculus is essentially higher-order, we have a
correspondence between contexts and processes. More specifically,
given a name $x$ and a context $M$ we can construct $M^{*}_{x}$ such
that 

\begin{mathpar}
  M^{*}_{x} | \lift{x}{P} \red M[P]
\end{mathpar}

namely,

\begin{mathpar}
  M^{*}_{x} := x?(u).M[\dropn{u}]
\end{mathpar}

The dependence of $M^{*}_{x}$ on a name makes it an abstraction, 

\begin{mathpar}
  M^{*} := (x)x?(u).M[\dropn{u}]
\end{mathpar}

\subsection{Additional notation}

It will sometimes be convenient to denote the process a name
quotes. We already have the notation $x = \quotep{P}$, but it will be
convenient to introduce an alternate notation, $\procn{x}$, when we
want to emphasize the connection to the use of the name. Note that, by
virtue of name equivalence, $\quotep{\procn{x}} \nameeq x$; so, the
notation is consistent with previous definitions.

Further, because names have structure it is possible to effect
substitutions on the basis of that structure. This means we need to
upgrade our notation for substitutions, which we accomplish by
adapting comprehension notation. Thus,

\begin{mathpar}
  P\{ y / x : x \in S \}
\end{mathpar}

is interpreted to mean the process derived from P by replacing (in a
capture-avoiding manner) each occurrence of $x$ in $S$ by $y$. For example,

\begin{mathpar}
  P\{ \quotep{\procn{x}|\procn{x}} / x : x \in \freenames{P} \}
\end{mathpar}

will replace each (occurrence) of a free name $x$ in $P$ by
$\quotep{\procn{x}|\procn{x}}$.

Also, we will avail ourselves of the notation $x^{L}$ and $x^{R}$ to
denote injections of a name into disjoint copies of the name
space. There are numerous ways to accomplish this. One example can be
found in \cite{MeredithR05}. This notation overloads to vectors of
names: $\vec{x}^{\pi} := (x_{i}^{\pi} \; : \; 0 \leq i < |\vec{x}| )$ where $\pi \in \{L,R\}$.

We also use $P^{\Box} := P|\Box$.

In \cite{MeredithR05} an interpretation of the new operator is
given. It turns out that there are several possible interpretations
all enjoying the requisite algebraic properties of the operator (see
\cite{milner91polyadicpi}). We will therefore make liberal use of
$(\nu\; \vec{x})P$.

% subsection the_syntax_and_semantics_of_the_notation_system (end)   

\input{qm2pi.qmops} 

\input{qm2pi.sterngerlach} 

\input{qm2pi.metric} 

% section concurrent_process_calculi (end)

%\input{qm2pi.proofsketch}

% section proof sketch (end)

%\input{qm2pi.slviaknots} 

% section spatial logic via knots (end)

\input{qm2pi.conclusion}

% section conclusion (end)

%\input{qm2pi.dtcodes} 

% section wiring algorithm (end)

\input{qm2pi.ack} 

% section acknowledgments (end)

\newpage


\bibliographystyle{plain}   
\bibliography{../../biblios/main.bib}

\input{qm2pi.rhodetails}

\end{document}

 

% section wiring algorithm (end)

\documentclass[12pt]{llncs}
%\documentclass{jktr}

\usepackage[pdftex]{hyperref}                   
\usepackage {listings}
\usepackage {mathpartir}
\usepackage{bcprules}
%\usepackage{listings}
                       
\usepackage{graphicx} 
%\usepackage[margins=2.5cm,nohead,nofoot]{geometry}
%\usepackage{geometry}
\usepackage{amsfonts}
\usepackage{amstext}
\usepackage{latexsym}
\usepackage{amssymb}
\usepackage{color}


%\include{myPreamble}
\include{qm2pi.local} 

%\ifpdf
%\usepackage[pdftex]{graphicx}
%\else
%\usepackage{graphicx}
%\fi

 % \ifpdf
%  \usepackage{pdfsync}
%  \if


%\title{Brief Article}
%\author{David F. Snyder}
%\author{L.G. Meredith}

%\address{Dept. of Math., Texas State University--San Marcos, San Marcos, TX 78666}
       
\pagestyle{empty}


\begin{document}

\lstset{language=[Objective]Caml,frame=shadowbox}

\input{qm2pi.front}

% section front matter (end)

\input{qm2pi.intro} 
 
% section introduction (end)

% \input{qm2pi.knotations} 

% section notation (end)

\input{qm2pi.process.calculi} 

% section concurrent_process_calculi_and_spatial_logics_ (end)
    
%\input{qm2pi.knots2pi} 

%\input{qm2pi.trefoil} 

%\input{qm2pi.mainthm} 

% subsection basic_interpretation (end)

%\input{qm2pi.rho.presentation} 
\subsection{The syntax and semantics of the notation system}\label{sub:the_syntax_and_semantics_of_the_notation_system} % (fold)

We now summarize a technical presentation of the calculus that
embodies our theory of dynamics. The typical presentation of such a
calculus follows the style of giving generators and relations on
them. The grammar, below, describing term constructors, freely
generates the set of processes, $\Proc$. This set is then quotiented
by a relation known as structural congruence and it is over this set
that the notion of dynamics is expressed. This presentation is
essentially that of \cite{MeredithR05} with the addition of
polyadicity and summation. For readability we have relegated some of
the technical subtleties to an appendix.

\subsubsection{Process grammar}\label{subsub:process_grammar}

\begin{mathpar}
  \inferrule* [lab=synchronization] {} {{M} \bc \pzero \;|\; x?F \;|\; x!C }
  \and
  \inferrule* [lab=abstraction] {} {{F} \bc (x)P}
  \and
  \inferrule* [lab=concretion] {} {{C} \bc \langle Q \rangle}
  \and
  \inferrule* [lab=process] {} {{P,Q} \bc M \;| \;P|Q \;|\; @{x}}
  \and
  \inferrule* [lab=name] {} {{x} \bc \quotep{P}}
\end{mathpar} 

Note that $\vec{x}$ (resp. $\vec{P}$) denotes a vector of names
(resp. processes) of length $|\vec{x}|$ (resp. $|\vec{P}|$). We adopt
the following useful abbreviations.

\begin{mathpar}
   x?(\vec{y}).P := x.(\vec{y})P \and  x\clift{\vec{P}} := x.\clift{\vec{P}}
   \and x!(y) := \lift{x}{\dropn{y}}
   \and \Pi_{i=0}^{n-1}P_i := P_0 | \ldots | P_{n-1}
\end{mathpar}

\subsubsection{Structural congruence}

\paragraph{Free and bound names and alpha-equivalence.} At the
core of structural equivalence is alpha-equivalence which identifies
process that are the same up to a change of variable. Formally, we
recognize the distinction between free and bound names. The free names
of a process, $\freenames{P}$, may be calculated recursively as
follows:

\begin{mathpar}
\freenames{\pzero} := \emptyset
  \and \\
  \freenames{x?(y).P} := \{ x \} \cup (\freenames{P} \setminus \{ y \})
  \and 
  \freenames{x!\langle P \rangle} := \{ x \} \cup \{ P \} 
  \and \\
  \freenames{P|Q} := \freenames{P} \cup \freenames{Q}
  \and \\
  \freenames{@{x}} := \{ x \}
\end{mathpar}

$\pi$
$\quotep{\pi}$

$\freenames{-} : \pi \to \mathcal{P}(\quotep{\pi})$

\begin{eqnarray*}
  \freenames{\pzero} & := & \emptyset \\
  \freenames{x?(y).P} & := & \{ x \} \cup (\freenames{P} \setminus \{ y \}) \\
  \freenames{x!\langle P \rangle} & := & \{ x \} \cup \{ P \} \\
  \freenames{P|Q} & := & \freenames{P} \cup \freenames{Q} \\
  \freenames{\dropn{x}} & := & \{ x \}
\end{eqnarray*}

The bound names of a process, $\boundnames{P}$, are those names occurring in $P$
that are not free. For example, in $x?(y).0$, the name $x$ is free, while $y$ is bound.

\begin{mathpar}
  \inferrule* [lab=monoidal-laws] {} { P|Q \equiv Q|P \and P|0 \equiv P \and P|(Q|R) \equiv (P|Q)|R }
\end{mathpar}

\begin{mathpar}
  \inferrule* [lab=alpha-equivalence] {} { (x)P \equiv (y)P\{y/x\} \and y \not\in \freenames{P} }
\end{mathpar}

\begin{definition}
Then two processes, $P,Q$, are alpha-equivalent if $P = Q\{\vec{y}/\vec{x}\}$ for
some $\vec{x} \in \boundnames{Q},\vec{y} \in \boundnames{P}$, where $Q\{\vec{y}/\vec{x}\}$
denotes the capture-avoiding substitution of $\vec{y}$ for $\vec{x}$ in $Q$.
\end{definition}

\begin{definition}
  The {\em structural congruence} \cite{SangiorgiWalker} , $\equiv$,
  between processes is the least congruence containing
  alpha-equivalence, satisfying the abelian monoid laws
  (associativity, commutativity and $\pzero$ as identity) for parallel
  composition $|$ and for summation $+$.
\end{definition}

\subsection{Name equivalence}

We take name equivalence, written $\nameeq$, to be the smallest
equivalence relation generated by the following rules.

\begin{mathpar}
\inferrule*[lab=Quote-drop]
{ }
{ \quotep{@{x}} \nameeq x }

\inferrule*[lab=Struct-equiv]
{ P \scong Q }
{ \quotep{P} \nameeq \quotep{Q} }
\end{mathpar}

The astute reader will have noticed that the mutual recursion of names
and processes imposes a mutual recursion on alpha-equivalence and
structural equivalence via name-equivalence. Fortunately, all of this
works out pleasantly and we may calculate in the natural way, free of
concern. The reader interested in the details is referred to the
appendix \ref{appendix:rho_details}.

\subsection{Substitution}

We use $\Proc$ for the set of processes, $\QProc$ for the set of
names, and $\id{\{}\vec{y} / \vec{x} \id{\}}$ to denote partial maps,
$s : \QProc \rightarrow \QProc$. A map, $s$ lifts, uniquely, to a map
on process terms, $\widehat{s} : \Proc \rightarrow \Proc$ by the
following equations.

\begin{mathpar}
  (0) \psubstp{Q}{P} := 0 \\
  (R \juxtap S) \psubstp{Q}{P}
  :=    
  (R)\psubstp{Q}{P} \juxtap (S) \psubstp{Q}{P} \\
  (x?(y).R) \psubstp{Q}{P}    
  :=    
  (x)\substp{Q}{P} (z)\concat( (R \psubstn{z}{y}) \psubstp{Q}{P} ) \\
  (\lift{x}{R}) \psubstp{Q}{P}  
  :=
  \lift{(x)\substp{Q}{P}}{ R \psubstp{Q}{P} } \\
%   (\dropn{x})  \psubstp{Q}{P}       
%   := 
%   \left\{ 
%     \begin{array}{ccc} 
%       \dropn{\quotep{Q}} & & x \nameeq \quotep{P} \\
%       \dropn{x} & & otherwise \\
%     \end{array}
%   \right. 
  (\dropn{x})  \psubstp{Q}{P}       
  := 
  \left\{ 
    \begin{array}{ccc} 
      Q & & x \nameeq \quotep{P} \\
      \dropn{x} & & otherwise \\
    \end{array}
  \right.
\end{mathpar}
 

where

\begin{eqnarray}
  (x)\id{\{} \lpquote Q \rpquote / \lpquote P \rpquote \id{\}}            = 
  \left\{ 
    \begin{array}{ccc}
      \lpquote Q \rpquote & & x \nameeq \lpquote P \rpquote \\
      x & & otherwise \\
    \end{array}
  \right. \nonumber
\end{eqnarray}

and $z$ is chosen distinct from $\quotep{P}$, $\quotep{Q}$, the free
names in $Q$, and all the names in $R$. Our $\alpha$-equivalence will
be built in the standard way from this substitution.

\begin{remark}\label{rem:no_self_referential_names}
  One consequence of these definitions is that $\forall P. \quotep{P}
  \not\in \freenames{P}$.
\end{remark}

\subsection{ Dynamic quote: an example }

Anticipating something of what's to come, consider applying the
substitution, $\widehat{\id{\{}u / z \id{\}}}$, to the following pair
of processes, $\lift{w}{y!(z)}$ and $w[ \lpquote y!(z) \rpquote ]$.

\begin{eqnarray}
	\lift{w}{y!(z)}\widehat{\id{\{}u / z \id{\}}}
		& = &
		\lift{w}{y!(u)} \nonumber\\
	w[ \lpquote y!(z) \rpquote ] \widehat{ \id{\{}u / z \id{\}} }
		& = &
		w[ \lpquote y!(z) \rpquote ] \nonumber
\end{eqnarray}

Because the body of the process between quotes is impervious to
substitution, we get radically different answers. In fact, by
examining the first process in an input context,
e.g. $x?(z).\lift{w}{y!(z)}$, we see that the process under the lift
operator may be shaped by prefixed inputs binding a name inside it. In
this sense, the lift operator will be seen as a way to dynamically
construct processes before reifying them as names.

Finally equipped with these standard features we can present the
dynamics of the calculus.

\subsubsection{Operational semantics} 

Finally, we introduce the computational dynamics. What marks these
algebras as distinct from other more traditionally studied algebraic
structures, e.g. vector spaces or polynomial rings, is the manner in
which dynamics is captured. In traditional structures, dynamics is typically
expressed through morphisms between such structures, as in linear maps
between vector spaces or morphisms between rings. In algebras
associated with the semantics of computation, the dynamics is
expressed as part of the algebraic structure itself, through a
reduction reduction relation typically denoted by $\red$. Below, we
give a recursive presentation of this relation for the calculus used
in the encoding.

$\red \subseteq \pi \times \pi$
$\red : \pi \to \mathcal{P}(\pi)$

\begin{mathpar}
  \inferrule* [lab=Comm] { \textsf{match}( x_{src}, x_{trgt} ) } { x_{trgt}?(y)P \; | \; x_{src}!\langle {Q} \rangle \red P\{\quotep{Q}/y}\} }
  \and \\
  \inferrule* [lab=Par] {{P} \red {P}'} {{{P} | {Q}} \red {{P}' | {Q}}}
  \and
  \inferrule* [lab=Equiv]{{{P} \scong {P}'} \andalso {{P}' \red {Q}'} \andalso {{Q}' \scong {Q}}}{{P} \red {Q}}
\end{mathpar}

\begin{eqnarray*}
  match_{\equiv} (\quotep{P},\quotep{Q}) & := & P \equiv Q \\
  match_{\dagger}(\quotep{P},\quotep{Q}) & := & \forall R. P|Q \red^{*} R => R \red^{*} 0 \\
  match_{K}(\quotep{P},\quotep{Q}) & := & K \mbox{ for some context } K
\end{eqnarray*}

$u?(x)P | u!\langle Q \rangle \red P\{\quotep{Q}/x\}$

%We write $\wred$ for $\red^*$, and $P\red$ if $\exists Q $ such that $ P \red Q$.
We write $P\red$ if $\exists Q $ such that $ P \red Q$ and $P\not\red$, otherwise.

\section{Replication}

As mentioned before, it is known that replication (and hence
recursion) can be implemented in a higher-order process algebra
\cite{SangiorgiWalker}. As our first example of calculation with the
machinery thus far presented we give the construction explicitly in
the {\rhoc}.

\begin{eqnarray}
	D_{x} & := & \prefix{x}{y}{(\binpar{\outputp{x}{y}}{@{y}})} \nonumber\\
	\bangp_{x}{P} & := & \binpar{{x}!\langle{\binpar{D_{x}}{P}}\rangle}{D_{x}} \nonumber
\end{eqnarray}

\begin{eqnarray}
	\bangp_{x}{P} & & \nonumber\\
	=
	& {x}!\langle{(\prefix{x}{y}{(\outputp{x}{y} | @{y})) | P}}\rangle 
	      | \prefix{x}{y}{(\outputp{x}{y} | @{y})} & \nonumber\\
	\red
	& (\outputp{x}{y} | @{y})\substn{\quotep{(\prefix{x}{y}{(@{y} | \outputp{x}{y})) | P}}}{y} & \nonumber\\
	=
	& \outputp{x}{\quotep{(\prefix{x}{y}{(\outputp{x}{y} | @{y})) | P}}}
	  | {(\prefix{x}{y}{(\outputp{x}{y} | @{y})) | P}} & \nonumber\\
	\red
	& \ldots & \nonumber\\
	\red^*
	& P | P | \ldots & \nonumber
\end{eqnarray}

Of course, this encoding, as an implementation, runs away, unfolding
$\bangp{P}$ eagerly. A lazier and more implementable replication
operator, restricted to input-guarded processes, may be obtained as follows.

\begin{eqnarray}
\bangp{\prefix{u}{v}{P}} 
	:= 
	\binpar{\lift{x}{\prefix{u}{v}{(\binpar{D(x)}{P})}}}{D(x)} \nonumber
\end{eqnarray}

\begin{remark}
  Note that the lazier definition still does not deal with summation
  or mixed summation (i.e. sums over input and output). The reader is
  invited to construct definitions of replication that deal with these
  features. 

  Further, the definitions are parameterized in a name, $x$. Can you,
  gentle reader, make a definition that eliminates this parameter and
  guarantees no accidental interaction between the replication
  machinery and the process being replicated -- i.e. no accidental
  sharing of names used by the process to get its work done and the
  name(s) used by the replication to effect copying. This latter
  revision of the definition of replication is crucial to obtaining
  the expected identity $!!P \sim !P$.
\end{remark}

\begin{remark}\label{rem:paradoxical_combinator}
  The reader familiar with the lambda calculus will have noticed the
  similarity between $D$ and the paradoxical combinator.

  [Ed. note: the existence of this seems to suggest we have to be more
  restrictive on the set of processes and names we admit if we are to
  support no-cloning.]
\end{remark}

\subsubsection{Bisimulation}

The computational dynamics gives rise to another kind of equivalence,
the equivalence of computational behavior. As previously mentioned
this is typically captured \emph{via} some form of bisimulation.

% The notion we use in this paper is weak barbed bisimulation
% \cite{milner91polyadicpi}.

The notion we use in this paper is derived from weak barbed
bisimulation \cite{milner91polyadicpi}. 

\begin{definition}
An \emph{observation relation}, $\downarrow_{\mathcal N}$, over a set
of names, $\mathcal N$, is the smallest relation satisfying the rules
below.

\infrule[Out-barb]{y \in {\mathcal N}, \; x \nameeq y}
		  {\outputp{x}{v} \downarrow_{\mathcal N} x}
\infrule[Par-barb]{\mbox{$P\downarrow_{\mathcal N} x$ or $Q\downarrow_{\mathcal N} x$}}
		  {\binpar{P}{Q} \downarrow_{\mathcal N} x}

We write $P \Downarrow_{\mathcal N} x$ if there is $Q$ such that 
$P \wred Q$ and $Q \downarrow_{\mathcal N} x$.
\end{definition}

\begin{definition}
%\label{def.bbisim}
An  ${\mathcal N}$-\emph{barbed bisimulation} over a set of names, ${\mathcal N}$, is a symmetric binary relation 
${\mathcal S}_{\mathcal N}$ between agents such that $P\rel{S}_{\mathcal N}Q$ implies:
\begin{enumerate}
\item If $P \red P'$ then $Q \wred Q'$ and $P'\rel{S}_{\mathcal N} Q'$.
\item If $P\downarrow_{\mathcal N} x$, then $Q\Downarrow_{\mathcal N} x$.
\end{enumerate}
$P$ is ${\mathcal N}$-barbed bisimilar to $Q$, written
$P \wbbisim_{\mathcal N} Q$, if $P \rel{S}_{\mathcal N} Q$ for some ${\mathcal N}$-barbed bisimulation ${\mathcal S}_{\mathcal N}$.
\end{definition}

$\mathcal{R} \subseteq \pi \times \pi$

$P \mathcal{R} Q => \forall P'. P \red P' \Rightarrow \exists Q'. Q \red Q', P' \mathcal{R} Q'$

$P \vdash x \Rightarrow Q \vdash x$

\begin{mathpar}
  \inferrule*[lab=Out-barb]{x \nameeq y}{{y}!\langle{Q}\rangle \vdash x}
  \and
  \inferrule*[lab=Par-barb]{\mbox{$P\vdash x$ or $Q\vdash x$}}{\binpar{P}{Q} \vdash x}
\end{mathpar}

\subsubsection{Contexts}

One of the principle advantages of computational calculi like the
$\pi$-calculus is a well-defined notion of context,
contextual-equivalence and a correlation between
contextual-equivalence and notions of bisimulation. The notion of
context allows the decomposition of a process into (sub-)process and
its syntactic environment, its context. Thus, a context may be
thought of as a process with a ``hole'' (written $\Box$) in it. The
application of a context $M$ to a process $P$, written $M[P]$, is
tantamount to filling the hole in $M$ with $P$. In this paper we do
not need the full weight of this theory, but do make use of the notion
of context in the proof the main theorem. 

\begin{mathpar}
  \inferrule* [lab=summation] {} {{M_{M},M_{N}} \bc \Box \;|\; x.M_{A} \;|\; M_{M}+M_{N}}
  \and
  \inferrule* [lab=agent] {} {{M_{A}} \bc (\vec{x})M_{P} \;| \; \clift{P_0,\ldots,M_{P},\ldots,P_N}}
  \and \\
  \inferrule* [lab=process] {} {{M_{P}} \bc M_{N} \;| \;P|M_{P} }
\end{mathpar} 

\begin{mathpar}
  \inferrule* [lab=sychronization] {} {M_{N} \bc \Box \;|\; x?M_{F} \;|\; x!M_{C}}
  \and
  \inferrule* [lab=abstraction] {} {{M_{F}} \bc (x)M_{P} }
  \and
  \inferrule* [lab=concretion] {} {{M_{C}} \bc \langle M_{P} \rangle }
  \and \\
  \inferrule* [lab=process] {} {{M_{P}} \bc M_{N} \;| \;P|M_{P} }
\end{mathpar}

\begin{definition}[contextual application] Given a context $M$, and
  process $P$, we define the \emph{contextual application}, $M[P] :=
  M\{P/\Box\}$. That is, the contextual application of M to P is the
  substitution of $P$ for $\Box$ in $M$.
\end{definition}

$\meaningof{-} : L \to \mathcal{P}(\pi)$

\begin{mathpar}
  \inferrule* [lab=collection] {} {\meaningof{true} = \pi, \and \meaningof{~E} = \pi \setminus \meaningof{E}, \and \meaningof{E_{1} \& E_{2}} = \meaningof{E_{1}} \cap \meaningof{E_{2}}}
\end{mathpar}

\begin{mathpar}
  \inferrule* [lab=structure] {} {\meaningof{0} = \{ P \in \pi | P \equiv 0 \}, \and \\ \meaningof{E_1 | E_2} = \{ P \in \pi | P \equiv P_{1} | P_{2}, P_{1} \in \meaningof{E_{1}}, P_{2} \in \meaningof{E_2}\} }
\end{mathpar}

\begin{mathpar}
 \inferrule* [lab=behavior] {} {\meaningof{\langle a?b \rangle E} = \{ P \in \pi | P \equiv Q | u?(y)P', \\ \and \\\\ \and \\ \;\;\; u \in \meaningof{a}, \forall z.P'\{z/y\} \in \meaningof{E\{z/b\}}\}, \and \\ \meaningof{a!E} = \{ P \in \pi | P \equiv Q | x!\langle P' \rangle, x \in \meaningof{a} P' \in \meaningof{E}\} }
\end{mathpar}

\begin{mathpar}
 \inferrule* [lab=nominal] {} {\meaningof{\quotep{E}} = \{ \quotep{P} \in \quotep{\pi} | P \in \meaningof{E} \}, \and \meaningof{\quotep{P}} = \{ \quotep{Q} \in \quotep{\pi} | P \equiv Q \} \and \\ \meaningof{@\quotep{E}} = \{ P \in \pi | P \equiv @x, x \in \meaningof{E} \}}
\end{mathpar}

\begin{eqnarray*}
  \\
  \meaningof{-} : TS \to ST
\end{eqnarray*}

\begin{eqnarray*}
  \\
  L : TS \to ST
\end{eqnarray*}

\begin{eqnarray*}
  \\
  P \models E \iff P \in \meaningof{E}
\end{eqnarray*}

\begin{eqnarray*}
  P \approx_{L} Q \iff \forall E \in L. P \models E \iff Q \models E
\end{eqnarray*}

\begin{eqnarray*}
  P \approx_{K} Q
\end{eqnarray*}

\begin{eqnarray*}
  P \approx Q
\end{eqnarray*}

$\approx_{K} = \approx = \approx_{L}$

\subsubsection{Contextual duality}

Note that contexts extend the quotation operation to a family of
operations from processes to names. Given a context, $M$, we can
define a \emph{nominal context}, $\quotep{M}$ by $\quotep{M}[P] :=
\quotep{M[P]}$. To foreshadow what is to come we observe that these
operations enjoy a duality with processes very much like the duality
between vectors and maps from vectors to scalars.

Further, because the calculus is essentially higher-order, we have a
correspondence between contexts and processes. More specifically,
given a name $x$ and a context $M$ we can construct $M^{*}_{x}$ such
that 

\begin{mathpar}
  M^{*}_{x} | \lift{x}{P} \red M[P]
\end{mathpar}

namely,

\begin{mathpar}
  M^{*}_{x} := x?(u).M[\dropn{u}]
\end{mathpar}

The dependence of $M^{*}_{x}$ on a name makes it an abstraction, 

\begin{mathpar}
  M^{*} := (x)x?(u).M[\dropn{u}]
\end{mathpar}

\subsection{Additional notation}

It will sometimes be convenient to denote the process a name
quotes. We already have the notation $x = \quotep{P}$, but it will be
convenient to introduce an alternate notation, $\procn{x}$, when we
want to emphasize the connection to the use of the name. Note that, by
virtue of name equivalence, $\quotep{\procn{x}} \nameeq x$; so, the
notation is consistent with previous definitions.

Further, because names have structure it is possible to effect
substitutions on the basis of that structure. This means we need to
upgrade our notation for substitutions, which we accomplish by
adapting comprehension notation. Thus,

\begin{mathpar}
  P\{ y / x : x \in S \}
\end{mathpar}

is interpreted to mean the process derived from P by replacing (in a
capture-avoiding manner) each occurrence of $x$ in $S$ by $y$. For example,

\begin{mathpar}
  P\{ \quotep{\procn{x}|\procn{x}} / x : x \in \freenames{P} \}
\end{mathpar}

will replace each (occurrence) of a free name $x$ in $P$ by
$\quotep{\procn{x}|\procn{x}}$.

Also, we will avail ourselves of the notation $x^{L}$ and $x^{R}$ to
denote injections of a name into disjoint copies of the name
space. There are numerous ways to accomplish this. One example can be
found in \cite{MeredithR05}. This notation overloads to vectors of
names: $\vec{x}^{\pi} := (x_{i}^{\pi} \; : \; 0 \leq i < |\vec{x}| )$ where $\pi \in \{L,R\}$.

We also use $P^{\Box} := P|\Box$.

In \cite{MeredithR05} an interpretation of the new operator is
given. It turns out that there are several possible interpretations
all enjoying the requisite algebraic properties of the operator (see
\cite{milner91polyadicpi}). We will therefore make liberal use of
$(\nu\; \vec{x})P$.

% subsection the_syntax_and_semantics_of_the_notation_system (end)   

\input{qm2pi.qmops} 

\input{qm2pi.sterngerlach} 

\input{qm2pi.metric} 

% section concurrent_process_calculi (end)

%\input{qm2pi.proofsketch}

% section proof sketch (end)

%\input{qm2pi.slviaknots} 

% section spatial logic via knots (end)

\input{qm2pi.conclusion}

% section conclusion (end)

%\input{qm2pi.dtcodes} 

% section wiring algorithm (end)

\input{qm2pi.ack} 

% section acknowledgments (end)

\newpage


\bibliographystyle{plain}   
\bibliography{../../biblios/main.bib}

\input{qm2pi.rhodetails}

\end{document}

 

% section acknowledgments (end)

\newpage


\bibliographystyle{plain}   
\bibliography{../../biblios/main.bib}

\documentclass[12pt]{llncs}
%\documentclass{jktr}

\usepackage[pdftex]{hyperref}                   
\usepackage {listings}
\usepackage {mathpartir}
\usepackage{bcprules}
%\usepackage{listings}
                       
\usepackage{graphicx} 
%\usepackage[margins=2.5cm,nohead,nofoot]{geometry}
%\usepackage{geometry}
\usepackage{amsfonts}
\usepackage{amstext}
\usepackage{latexsym}
\usepackage{amssymb}
\usepackage{color}


%\include{myPreamble}
\include{qm2pi.local} 

%\ifpdf
%\usepackage[pdftex]{graphicx}
%\else
%\usepackage{graphicx}
%\fi

 % \ifpdf
%  \usepackage{pdfsync}
%  \if


%\title{Brief Article}
%\author{David F. Snyder}
%\author{L.G. Meredith}

%\address{Dept. of Math., Texas State University--San Marcos, San Marcos, TX 78666}
       
\pagestyle{empty}


\begin{document}

\lstset{language=[Objective]Caml,frame=shadowbox}

\input{qm2pi.front}

% section front matter (end)

\input{qm2pi.intro} 
 
% section introduction (end)

% \input{qm2pi.knotations} 

% section notation (end)

\input{qm2pi.process.calculi} 

% section concurrent_process_calculi_and_spatial_logics_ (end)
    
%\input{qm2pi.knots2pi} 

%\input{qm2pi.trefoil} 

%\input{qm2pi.mainthm} 

% subsection basic_interpretation (end)

%\input{qm2pi.rho.presentation} 
\subsection{The syntax and semantics of the notation system}\label{sub:the_syntax_and_semantics_of_the_notation_system} % (fold)

We now summarize a technical presentation of the calculus that
embodies our theory of dynamics. The typical presentation of such a
calculus follows the style of giving generators and relations on
them. The grammar, below, describing term constructors, freely
generates the set of processes, $\Proc$. This set is then quotiented
by a relation known as structural congruence and it is over this set
that the notion of dynamics is expressed. This presentation is
essentially that of \cite{MeredithR05} with the addition of
polyadicity and summation. For readability we have relegated some of
the technical subtleties to an appendix.

\subsubsection{Process grammar}\label{subsub:process_grammar}

\begin{mathpar}
  \inferrule* [lab=synchronization] {} {{M} \bc \pzero \;|\; x?F \;|\; x!C }
  \and
  \inferrule* [lab=abstraction] {} {{F} \bc (x)P}
  \and
  \inferrule* [lab=concretion] {} {{C} \bc \langle Q \rangle}
  \and
  \inferrule* [lab=process] {} {{P,Q} \bc M \;| \;P|Q \;|\; @{x}}
  \and
  \inferrule* [lab=name] {} {{x} \bc \quotep{P}}
\end{mathpar} 

Note that $\vec{x}$ (resp. $\vec{P}$) denotes a vector of names
(resp. processes) of length $|\vec{x}|$ (resp. $|\vec{P}|$). We adopt
the following useful abbreviations.

\begin{mathpar}
   x?(\vec{y}).P := x.(\vec{y})P \and  x\clift{\vec{P}} := x.\clift{\vec{P}}
   \and x!(y) := \lift{x}{\dropn{y}}
   \and \Pi_{i=0}^{n-1}P_i := P_0 | \ldots | P_{n-1}
\end{mathpar}

\subsubsection{Structural congruence}

\paragraph{Free and bound names and alpha-equivalence.} At the
core of structural equivalence is alpha-equivalence which identifies
process that are the same up to a change of variable. Formally, we
recognize the distinction between free and bound names. The free names
of a process, $\freenames{P}$, may be calculated recursively as
follows:

\begin{mathpar}
\freenames{\pzero} := \emptyset
  \and \\
  \freenames{x?(y).P} := \{ x \} \cup (\freenames{P} \setminus \{ y \})
  \and 
  \freenames{x!\langle P \rangle} := \{ x \} \cup \{ P \} 
  \and \\
  \freenames{P|Q} := \freenames{P} \cup \freenames{Q}
  \and \\
  \freenames{@{x}} := \{ x \}
\end{mathpar}

$\pi$
$\quotep{\pi}$

$\freenames{-} : \pi \to \mathcal{P}(\quotep{\pi})$

\begin{eqnarray*}
  \freenames{\pzero} & := & \emptyset \\
  \freenames{x?(y).P} & := & \{ x \} \cup (\freenames{P} \setminus \{ y \}) \\
  \freenames{x!\langle P \rangle} & := & \{ x \} \cup \{ P \} \\
  \freenames{P|Q} & := & \freenames{P} \cup \freenames{Q} \\
  \freenames{\dropn{x}} & := & \{ x \}
\end{eqnarray*}

The bound names of a process, $\boundnames{P}$, are those names occurring in $P$
that are not free. For example, in $x?(y).0$, the name $x$ is free, while $y$ is bound.

\begin{mathpar}
  \inferrule* [lab=monoidal-laws] {} { P|Q \equiv Q|P \and P|0 \equiv P \and P|(Q|R) \equiv (P|Q)|R }
\end{mathpar}

\begin{mathpar}
  \inferrule* [lab=alpha-equivalence] {} { (x)P \equiv (y)P\{y/x\} \and y \not\in \freenames{P} }
\end{mathpar}

\begin{definition}
Then two processes, $P,Q$, are alpha-equivalent if $P = Q\{\vec{y}/\vec{x}\}$ for
some $\vec{x} \in \boundnames{Q},\vec{y} \in \boundnames{P}$, where $Q\{\vec{y}/\vec{x}\}$
denotes the capture-avoiding substitution of $\vec{y}$ for $\vec{x}$ in $Q$.
\end{definition}

\begin{definition}
  The {\em structural congruence} \cite{SangiorgiWalker} , $\equiv$,
  between processes is the least congruence containing
  alpha-equivalence, satisfying the abelian monoid laws
  (associativity, commutativity and $\pzero$ as identity) for parallel
  composition $|$ and for summation $+$.
\end{definition}

\subsection{Name equivalence}

We take name equivalence, written $\nameeq$, to be the smallest
equivalence relation generated by the following rules.

\begin{mathpar}
\inferrule*[lab=Quote-drop]
{ }
{ \quotep{@{x}} \nameeq x }

\inferrule*[lab=Struct-equiv]
{ P \scong Q }
{ \quotep{P} \nameeq \quotep{Q} }
\end{mathpar}

The astute reader will have noticed that the mutual recursion of names
and processes imposes a mutual recursion on alpha-equivalence and
structural equivalence via name-equivalence. Fortunately, all of this
works out pleasantly and we may calculate in the natural way, free of
concern. The reader interested in the details is referred to the
appendix \ref{appendix:rho_details}.

\subsection{Substitution}

We use $\Proc$ for the set of processes, $\QProc$ for the set of
names, and $\id{\{}\vec{y} / \vec{x} \id{\}}$ to denote partial maps,
$s : \QProc \rightarrow \QProc$. A map, $s$ lifts, uniquely, to a map
on process terms, $\widehat{s} : \Proc \rightarrow \Proc$ by the
following equations.

\begin{mathpar}
  (0) \psubstp{Q}{P} := 0 \\
  (R \juxtap S) \psubstp{Q}{P}
  :=    
  (R)\psubstp{Q}{P} \juxtap (S) \psubstp{Q}{P} \\
  (x?(y).R) \psubstp{Q}{P}    
  :=    
  (x)\substp{Q}{P} (z)\concat( (R \psubstn{z}{y}) \psubstp{Q}{P} ) \\
  (\lift{x}{R}) \psubstp{Q}{P}  
  :=
  \lift{(x)\substp{Q}{P}}{ R \psubstp{Q}{P} } \\
%   (\dropn{x})  \psubstp{Q}{P}       
%   := 
%   \left\{ 
%     \begin{array}{ccc} 
%       \dropn{\quotep{Q}} & & x \nameeq \quotep{P} \\
%       \dropn{x} & & otherwise \\
%     \end{array}
%   \right. 
  (\dropn{x})  \psubstp{Q}{P}       
  := 
  \left\{ 
    \begin{array}{ccc} 
      Q & & x \nameeq \quotep{P} \\
      \dropn{x} & & otherwise \\
    \end{array}
  \right.
\end{mathpar}
 

where

\begin{eqnarray}
  (x)\id{\{} \lpquote Q \rpquote / \lpquote P \rpquote \id{\}}            = 
  \left\{ 
    \begin{array}{ccc}
      \lpquote Q \rpquote & & x \nameeq \lpquote P \rpquote \\
      x & & otherwise \\
    \end{array}
  \right. \nonumber
\end{eqnarray}

and $z$ is chosen distinct from $\quotep{P}$, $\quotep{Q}$, the free
names in $Q$, and all the names in $R$. Our $\alpha$-equivalence will
be built in the standard way from this substitution.

\begin{remark}\label{rem:no_self_referential_names}
  One consequence of these definitions is that $\forall P. \quotep{P}
  \not\in \freenames{P}$.
\end{remark}

\subsection{ Dynamic quote: an example }

Anticipating something of what's to come, consider applying the
substitution, $\widehat{\id{\{}u / z \id{\}}}$, to the following pair
of processes, $\lift{w}{y!(z)}$ and $w[ \lpquote y!(z) \rpquote ]$.

\begin{eqnarray}
	\lift{w}{y!(z)}\widehat{\id{\{}u / z \id{\}}}
		& = &
		\lift{w}{y!(u)} \nonumber\\
	w[ \lpquote y!(z) \rpquote ] \widehat{ \id{\{}u / z \id{\}} }
		& = &
		w[ \lpquote y!(z) \rpquote ] \nonumber
\end{eqnarray}

Because the body of the process between quotes is impervious to
substitution, we get radically different answers. In fact, by
examining the first process in an input context,
e.g. $x?(z).\lift{w}{y!(z)}$, we see that the process under the lift
operator may be shaped by prefixed inputs binding a name inside it. In
this sense, the lift operator will be seen as a way to dynamically
construct processes before reifying them as names.

Finally equipped with these standard features we can present the
dynamics of the calculus.

\subsubsection{Operational semantics} 

Finally, we introduce the computational dynamics. What marks these
algebras as distinct from other more traditionally studied algebraic
structures, e.g. vector spaces or polynomial rings, is the manner in
which dynamics is captured. In traditional structures, dynamics is typically
expressed through morphisms between such structures, as in linear maps
between vector spaces or morphisms between rings. In algebras
associated with the semantics of computation, the dynamics is
expressed as part of the algebraic structure itself, through a
reduction reduction relation typically denoted by $\red$. Below, we
give a recursive presentation of this relation for the calculus used
in the encoding.

$\red \subseteq \pi \times \pi$
$\red : \pi \to \mathcal{P}(\pi)$

\begin{mathpar}
  \inferrule* [lab=Comm] { \textsf{match}( x_{src}, x_{trgt} ) } { x_{trgt}?(y)P \; | \; x_{src}!\langle {Q} \rangle \red P\{\quotep{Q}/y}\} }
  \and \\
  \inferrule* [lab=Par] {{P} \red {P}'} {{{P} | {Q}} \red {{P}' | {Q}}}
  \and
  \inferrule* [lab=Equiv]{{{P} \scong {P}'} \andalso {{P}' \red {Q}'} \andalso {{Q}' \scong {Q}}}{{P} \red {Q}}
\end{mathpar}

\begin{eqnarray*}
  match_{\equiv} (\quotep{P},\quotep{Q}) & := & P \equiv Q \\
  match_{\dagger}(\quotep{P},\quotep{Q}) & := & \forall R. P|Q \red^{*} R => R \red^{*} 0 \\
  match_{K}(\quotep{P},\quotep{Q}) & := & K \mbox{ for some context } K
\end{eqnarray*}

$u?(x)P | u!\langle Q \rangle \red P\{\quotep{Q}/x\}$

%We write $\wred$ for $\red^*$, and $P\red$ if $\exists Q $ such that $ P \red Q$.
We write $P\red$ if $\exists Q $ such that $ P \red Q$ and $P\not\red$, otherwise.

\section{Replication}

As mentioned before, it is known that replication (and hence
recursion) can be implemented in a higher-order process algebra
\cite{SangiorgiWalker}. As our first example of calculation with the
machinery thus far presented we give the construction explicitly in
the {\rhoc}.

\begin{eqnarray}
	D_{x} & := & \prefix{x}{y}{(\binpar{\outputp{x}{y}}{@{y}})} \nonumber\\
	\bangp_{x}{P} & := & \binpar{{x}!\langle{\binpar{D_{x}}{P}}\rangle}{D_{x}} \nonumber
\end{eqnarray}

\begin{eqnarray}
	\bangp_{x}{P} & & \nonumber\\
	=
	& {x}!\langle{(\prefix{x}{y}{(\outputp{x}{y} | @{y})) | P}}\rangle 
	      | \prefix{x}{y}{(\outputp{x}{y} | @{y})} & \nonumber\\
	\red
	& (\outputp{x}{y} | @{y})\substn{\quotep{(\prefix{x}{y}{(@{y} | \outputp{x}{y})) | P}}}{y} & \nonumber\\
	=
	& \outputp{x}{\quotep{(\prefix{x}{y}{(\outputp{x}{y} | @{y})) | P}}}
	  | {(\prefix{x}{y}{(\outputp{x}{y} | @{y})) | P}} & \nonumber\\
	\red
	& \ldots & \nonumber\\
	\red^*
	& P | P | \ldots & \nonumber
\end{eqnarray}

Of course, this encoding, as an implementation, runs away, unfolding
$\bangp{P}$ eagerly. A lazier and more implementable replication
operator, restricted to input-guarded processes, may be obtained as follows.

\begin{eqnarray}
\bangp{\prefix{u}{v}{P}} 
	:= 
	\binpar{\lift{x}{\prefix{u}{v}{(\binpar{D(x)}{P})}}}{D(x)} \nonumber
\end{eqnarray}

\begin{remark}
  Note that the lazier definition still does not deal with summation
  or mixed summation (i.e. sums over input and output). The reader is
  invited to construct definitions of replication that deal with these
  features. 

  Further, the definitions are parameterized in a name, $x$. Can you,
  gentle reader, make a definition that eliminates this parameter and
  guarantees no accidental interaction between the replication
  machinery and the process being replicated -- i.e. no accidental
  sharing of names used by the process to get its work done and the
  name(s) used by the replication to effect copying. This latter
  revision of the definition of replication is crucial to obtaining
  the expected identity $!!P \sim !P$.
\end{remark}

\begin{remark}\label{rem:paradoxical_combinator}
  The reader familiar with the lambda calculus will have noticed the
  similarity between $D$ and the paradoxical combinator.

  [Ed. note: the existence of this seems to suggest we have to be more
  restrictive on the set of processes and names we admit if we are to
  support no-cloning.]
\end{remark}

\subsubsection{Bisimulation}

The computational dynamics gives rise to another kind of equivalence,
the equivalence of computational behavior. As previously mentioned
this is typically captured \emph{via} some form of bisimulation.

% The notion we use in this paper is weak barbed bisimulation
% \cite{milner91polyadicpi}.

The notion we use in this paper is derived from weak barbed
bisimulation \cite{milner91polyadicpi}. 

\begin{definition}
An \emph{observation relation}, $\downarrow_{\mathcal N}$, over a set
of names, $\mathcal N$, is the smallest relation satisfying the rules
below.

\infrule[Out-barb]{y \in {\mathcal N}, \; x \nameeq y}
		  {\outputp{x}{v} \downarrow_{\mathcal N} x}
\infrule[Par-barb]{\mbox{$P\downarrow_{\mathcal N} x$ or $Q\downarrow_{\mathcal N} x$}}
		  {\binpar{P}{Q} \downarrow_{\mathcal N} x}

We write $P \Downarrow_{\mathcal N} x$ if there is $Q$ such that 
$P \wred Q$ and $Q \downarrow_{\mathcal N} x$.
\end{definition}

\begin{definition}
%\label{def.bbisim}
An  ${\mathcal N}$-\emph{barbed bisimulation} over a set of names, ${\mathcal N}$, is a symmetric binary relation 
${\mathcal S}_{\mathcal N}$ between agents such that $P\rel{S}_{\mathcal N}Q$ implies:
\begin{enumerate}
\item If $P \red P'$ then $Q \wred Q'$ and $P'\rel{S}_{\mathcal N} Q'$.
\item If $P\downarrow_{\mathcal N} x$, then $Q\Downarrow_{\mathcal N} x$.
\end{enumerate}
$P$ is ${\mathcal N}$-barbed bisimilar to $Q$, written
$P \wbbisim_{\mathcal N} Q$, if $P \rel{S}_{\mathcal N} Q$ for some ${\mathcal N}$-barbed bisimulation ${\mathcal S}_{\mathcal N}$.
\end{definition}

$\mathcal{R} \subseteq \pi \times \pi$

$P \mathcal{R} Q => \forall P'. P \red P' \Rightarrow \exists Q'. Q \red Q', P' \mathcal{R} Q'$

$P \vdash x \Rightarrow Q \vdash x$

\begin{mathpar}
  \inferrule*[lab=Out-barb]{x \nameeq y}{{y}!\langle{Q}\rangle \vdash x}
  \and
  \inferrule*[lab=Par-barb]{\mbox{$P\vdash x$ or $Q\vdash x$}}{\binpar{P}{Q} \vdash x}
\end{mathpar}

\subsubsection{Contexts}

One of the principle advantages of computational calculi like the
$\pi$-calculus is a well-defined notion of context,
contextual-equivalence and a correlation between
contextual-equivalence and notions of bisimulation. The notion of
context allows the decomposition of a process into (sub-)process and
its syntactic environment, its context. Thus, a context may be
thought of as a process with a ``hole'' (written $\Box$) in it. The
application of a context $M$ to a process $P$, written $M[P]$, is
tantamount to filling the hole in $M$ with $P$. In this paper we do
not need the full weight of this theory, but do make use of the notion
of context in the proof the main theorem. 

\begin{mathpar}
  \inferrule* [lab=summation] {} {{M_{M},M_{N}} \bc \Box \;|\; x.M_{A} \;|\; M_{M}+M_{N}}
  \and
  \inferrule* [lab=agent] {} {{M_{A}} \bc (\vec{x})M_{P} \;| \; \clift{P_0,\ldots,M_{P},\ldots,P_N}}
  \and \\
  \inferrule* [lab=process] {} {{M_{P}} \bc M_{N} \;| \;P|M_{P} }
\end{mathpar} 

\begin{mathpar}
  \inferrule* [lab=sychronization] {} {M_{N} \bc \Box \;|\; x?M_{F} \;|\; x!M_{C}}
  \and
  \inferrule* [lab=abstraction] {} {{M_{F}} \bc (x)M_{P} }
  \and
  \inferrule* [lab=concretion] {} {{M_{C}} \bc \langle M_{P} \rangle }
  \and \\
  \inferrule* [lab=process] {} {{M_{P}} \bc M_{N} \;| \;P|M_{P} }
\end{mathpar}

\begin{definition}[contextual application] Given a context $M$, and
  process $P$, we define the \emph{contextual application}, $M[P] :=
  M\{P/\Box\}$. That is, the contextual application of M to P is the
  substitution of $P$ for $\Box$ in $M$.
\end{definition}

$\meaningof{-} : L \to \mathcal{P}(\pi)$

\begin{mathpar}
  \inferrule* [lab=collection] {} {\meaningof{true} = \pi, \and \meaningof{~E} = \pi \setminus \meaningof{E}, \and \meaningof{E_{1} \& E_{2}} = \meaningof{E_{1}} \cap \meaningof{E_{2}}}
\end{mathpar}

\begin{mathpar}
  \inferrule* [lab=structure] {} {\meaningof{0} = \{ P \in \pi | P \equiv 0 \}, \and \\ \meaningof{E_1 | E_2} = \{ P \in \pi | P \equiv P_{1} | P_{2}, P_{1} \in \meaningof{E_{1}}, P_{2} \in \meaningof{E_2}\} }
\end{mathpar}

\begin{mathpar}
 \inferrule* [lab=behavior] {} {\meaningof{\langle a?b \rangle E} = \{ P \in \pi | P \equiv Q | u?(y)P', \\ \and \\\\ \and \\ \;\;\; u \in \meaningof{a}, \forall z.P'\{z/y\} \in \meaningof{E\{z/b\}}\}, \and \\ \meaningof{a!E} = \{ P \in \pi | P \equiv Q | x!\langle P' \rangle, x \in \meaningof{a} P' \in \meaningof{E}\} }
\end{mathpar}

\begin{mathpar}
 \inferrule* [lab=nominal] {} {\meaningof{\quotep{E}} = \{ \quotep{P} \in \quotep{\pi} | P \in \meaningof{E} \}, \and \meaningof{\quotep{P}} = \{ \quotep{Q} \in \quotep{\pi} | P \equiv Q \} \and \\ \meaningof{@\quotep{E}} = \{ P \in \pi | P \equiv @x, x \in \meaningof{E} \}}
\end{mathpar}

\begin{eqnarray*}
  \\
  \meaningof{-} : TS \to ST
\end{eqnarray*}

\begin{eqnarray*}
  \\
  L : TS \to ST
\end{eqnarray*}

\begin{eqnarray*}
  \\
  P \models E \iff P \in \meaningof{E}
\end{eqnarray*}

\begin{eqnarray*}
  P \approx_{L} Q \iff \forall E \in L. P \models E \iff Q \models E
\end{eqnarray*}

\begin{eqnarray*}
  P \approx_{K} Q
\end{eqnarray*}

\begin{eqnarray*}
  P \approx Q
\end{eqnarray*}

$\approx_{K} = \approx = \approx_{L}$

\subsubsection{Contextual duality}

Note that contexts extend the quotation operation to a family of
operations from processes to names. Given a context, $M$, we can
define a \emph{nominal context}, $\quotep{M}$ by $\quotep{M}[P] :=
\quotep{M[P]}$. To foreshadow what is to come we observe that these
operations enjoy a duality with processes very much like the duality
between vectors and maps from vectors to scalars.

Further, because the calculus is essentially higher-order, we have a
correspondence between contexts and processes. More specifically,
given a name $x$ and a context $M$ we can construct $M^{*}_{x}$ such
that 

\begin{mathpar}
  M^{*}_{x} | \lift{x}{P} \red M[P]
\end{mathpar}

namely,

\begin{mathpar}
  M^{*}_{x} := x?(u).M[\dropn{u}]
\end{mathpar}

The dependence of $M^{*}_{x}$ on a name makes it an abstraction, 

\begin{mathpar}
  M^{*} := (x)x?(u).M[\dropn{u}]
\end{mathpar}

\subsection{Additional notation}

It will sometimes be convenient to denote the process a name
quotes. We already have the notation $x = \quotep{P}$, but it will be
convenient to introduce an alternate notation, $\procn{x}$, when we
want to emphasize the connection to the use of the name. Note that, by
virtue of name equivalence, $\quotep{\procn{x}} \nameeq x$; so, the
notation is consistent with previous definitions.

Further, because names have structure it is possible to effect
substitutions on the basis of that structure. This means we need to
upgrade our notation for substitutions, which we accomplish by
adapting comprehension notation. Thus,

\begin{mathpar}
  P\{ y / x : x \in S \}
\end{mathpar}

is interpreted to mean the process derived from P by replacing (in a
capture-avoiding manner) each occurrence of $x$ in $S$ by $y$. For example,

\begin{mathpar}
  P\{ \quotep{\procn{x}|\procn{x}} / x : x \in \freenames{P} \}
\end{mathpar}

will replace each (occurrence) of a free name $x$ in $P$ by
$\quotep{\procn{x}|\procn{x}}$.

Also, we will avail ourselves of the notation $x^{L}$ and $x^{R}$ to
denote injections of a name into disjoint copies of the name
space. There are numerous ways to accomplish this. One example can be
found in \cite{MeredithR05}. This notation overloads to vectors of
names: $\vec{x}^{\pi} := (x_{i}^{\pi} \; : \; 0 \leq i < |\vec{x}| )$ where $\pi \in \{L,R\}$.

We also use $P^{\Box} := P|\Box$.

In \cite{MeredithR05} an interpretation of the new operator is
given. It turns out that there are several possible interpretations
all enjoying the requisite algebraic properties of the operator (see
\cite{milner91polyadicpi}). We will therefore make liberal use of
$(\nu\; \vec{x})P$.

% subsection the_syntax_and_semantics_of_the_notation_system (end)   

\input{qm2pi.qmops} 

\input{qm2pi.sterngerlach} 

\input{qm2pi.metric} 

% section concurrent_process_calculi (end)

%\input{qm2pi.proofsketch}

% section proof sketch (end)

%\input{qm2pi.slviaknots} 

% section spatial logic via knots (end)

\input{qm2pi.conclusion}

% section conclusion (end)

%\input{qm2pi.dtcodes} 

% section wiring algorithm (end)

\input{qm2pi.ack} 

% section acknowledgments (end)

\newpage


\bibliographystyle{plain}   
\bibliography{../../biblios/main.bib}

\input{qm2pi.rhodetails}

\end{document}



\end{document}



% section proof sketch (end)

%\section{Unlikely characters: spatial logic for
  knots}\label{sub:characteristic_formulae} % (fold)

Associated to the mobile process calculi are a family of logics known
as the Hennessy-Milner logics. These logics typically enjoy a
semantics interpreting formulae as sets of processes that when
factored through the encoding outlined above allows an identification
of classes of knots with logical formulae. In the context of this
encoding the sub-family known as the spatial logics \cite{CairesC03}
\cite{CairesC04} \cite{Caires04} are of particular interest providing
several important features for expressing and reasoning about
properties (i.e. classes) of knots. We hint here at how this may be done.

%\begin{description}
%\item [structural connectives] 
\subsubsection{Structural connectives} The spatial logics enjoy
structural connectives corresponding, at the logical level, to the
parallel composition ($P | Q$) and new name ($(\nu \; x)P$)
connectives for processes. As illustrated in the examples below, these
connectives are extremely expressive given the shape of our encoding.
%\item [decideable satisfaction]

\subsubsection{Decideable satisfaction}
In \cite{Caires04} the satisfaction relation is shown to be decideable
for a rich class of processes. It further turns out that the image of
the our encoding is a proper subset of that class. This result
provides the basis for an algorithm by which to search for knots
enjoying a given property.
%\item [characteristic formulae]

\subsubsection{Characteristic formulae}
In the same paper \cite{Caires04} , Caires presents a means of calculating
characteristic formulae, selecting equivalence classes of processes
up to a pre--specified depth limit on the support set of names. Composed with our
encoding, this characteristic formula can be used to select
characteristic formulae for knots.
%\end{description}

\subsubsection{Spatial logic formulae}

The grammar below (segmented for comprehension) summarizes the syntax
of spatial logic formulae. We employ illustrative examples in the
sequel to provide an intuitive understanding of their meaning
referring the reader to \cite{Caires04} for a more detailed explication
of the semantics.

\begin{mathpar}
  \inferrule* [lab=boolean] {} {{A,B} \bc T \;|\; \neg A \;|\; A \wedge B \;|\; \eta = \eta'}
  \and
  \inferrule* [lab=spatial] {} {|\; \pzero \;|\; A | B \;|\; x \text{\textregistered} A \;|\; \forall x . A \;|\;  H x . A}
  \and
  \inferrule* [lab=behavioral] {} {|\; \alpha . A}
  \and 
  \inferrule* [lab=recursion] {} {|\; X(\vec{u}) \;|\; \mu X(\vec{u}) . A}
  \and
  \inferrule* [lab=action] {} {\alpha \bc \langle x?(\vec{y}) \rangle \;|\; \langle x!(\vec{y}) \rangle \;|\; \langle \tau \rangle}
  \and 
  \inferrule* [lab=name] {} {\eta \bc x \;|\; \tau}
\end{mathpar} 

% subsection characteristic_formulae (end)   	 

\subsection{Example formulae}\label{sub:example_formulae_} % (fold)

\subsubsection{Crossing as formula.}
% 
% \begin{align*}
%   \frac{d}{dx} \sin x &= \cos x 
%   & \frac{d}{dx} e^x &= e^x \\
%   \frac{d}{dx} \cos x &= - \sin x 
%   & \frac{d}{dx} \log x &= \frac{1}{x} \\
% \end{align*} 

\begin{align*}
 \mu C(x_{0},x_{1},y_{0},y_{1},u).&(\langle x_{0}?(z) \rangle(\langle u! \rangle\langle y_{1}!z \rangle C(x_{0},x_{1},y_{0},y_{1},u)) & \\
  & \wedge \langle y_{1}?(z) \rangle (\langle u! \rangle \langle x_{0}!z \rangle C(x_{0},x_{1},y_{0},y_{1},u)) & \\
  & \wedge \langle x_{1}?(z) \rangle (\langle u? \rangle \langle y_{0}!z \rangle C(x_{0},x_{1},y_{0},y_{1},u)) & \\
  & \wedge \langle y_{0}?(z) \rangle (\langle u? \rangle \langle x_{1}!z \rangle C(x_{0},x_{1},y_{0},y_{1},u))) &
\end{align*}

The lexicographical similarity between the shape of this formulae and
the shape of definition of the process representing a crossing reveals
the intuitive meaning of this formulae. It describes the capabilities
of a process that has the right to represent a crossing. For example
it picks out processes that may perform an input on the port $x_0$ in
its initial menu of capabilities. What differentiates the formula
from the process, however, is that the crossing process is the
smallest candidate to satisfy the formula. Infinitely many other
processes -- with internal behavior hidden behind this interface, so
to speak -- also satisfy this formula. Even this simple formula,
then, can be seen to open a new view onto knots, providing a
computational interpretation of \emph{virtual} knots.

Note that this formula is derived by hand. A similar formula can be
derived by employing Caires' calculation of characteristic formula
\cite{Caires04} to the process representing a crossing. In light of
this discussion, we let
$\meaningof{C}_{\phi}(x0,x1,y0,y1,u)$ denote a formula specifying the
dynamics we wish to capture of a crossing. To guarantee we preserve
the shape of the interface and minimal semantics we demand that
$\meaningof{C}_{\phi}(x0,x1,y0,y1,u) \Rightarrow
\textbf{C}(x0,x1,y0,y1,u)$ where $\textbf{C}(x0,x1,y0,y1,u)$ denotes
the formula above.
                            
\subsubsection{Crossing number constraints.}
The moral content of the context lemma (Lemma \ref{context}) is that the notion of
``locality'' in the Reidemeister moves is effectively captured by the
parallel composition operator of the process calculus. This intuition
extends through the logic. Given a formula,
$\meaningof{C}_{\phi}(x0,x1,y0,y1,u)$, we can use the structural
connectives to specify constraints on crossing numbers, such as at
least $n$ crossings, or exactly $n$ crossings.
\begin{mathpar}
  \inferrule* [lab=at-least-n] {} { K^{\geq n}_{\phi}(\vec{xs},\vec{ys}) := \Pi_{i=0}^{n-1} Hu . \meaningof{C}_{\phi}(xs_i,ys_i,u) | T }
  \and 
  \inferrule* [lab=exactly-n] {} { K^{= n}_{\phi}(\vec{xs},\vec{ys}) := \Pi_{i=0}^{n-1} Hu . \meaningof{C}_{\phi}(xs_i,ys_i,u) | \neg (\forall x_0,y_0,x_1,y_1,u . \meaningof{C}_{\phi}(x_0,y_0,x_1,y_1,u) | T) }
\end{mathpar}

To round out this section, recall that the encoding of an $n$-crossing
knot decomposes into a parallel composition of $n$ \emph{copies} of a
crossing process together with a wiring harness. To specify different
knot classes with the same crossing number amounts to specifying
logical constraints on the wiring harness. In the interest of space,
we defer examples to a forthcoming paper. Suffice it to say that both
the conditions ``alternating knot'' and ``contains the tangle
corresponding to 5/3'' are expressible. For example, it is possible to
calculate the characteristic formula of a process corresponding to the
tangle 5/3 and conjoin it into the classifying formula via the
composition connective of the logic.

Finally, we wish to observe that it is entirely within reason to
contemplate a more domain-specific version of spatial logic tailored
to the shape of processes in the image of the encoding. Such a
domain-specific logic would have a better claim to the title formal
language of knot properties.

% subsection example_formulae_ (end)

% section knots_as_processes (end) 

% section spatial logic via knots (end)

\section{Conclusions and future work}

\paragraph{Testing physical space}
You, gentle reader, may wonder why of all the theorems to be proved
given this set up we pick the one above. In some sense it's hardly
central to quantum mechanics. We see it as central in the sense that
it firmly establishes a notion of physical space arising from a notion
of the equivalence of behavior. Relating bisimulation to a metric is a
big step forward, but one is faced with interpreting the relationship
of that metric space to something more physical. Quantum mechanical
notions of ``physical'' space are still far from intuitive, but by
relating this idea of distance as testing to calculations that predict
physical circumstances we are making a not insignificant step forward
toward an understanding of the physical space we inhabit as
essentially dynamic.

\paragraph{Effectivity and simulation}
One of the observations we have yet to make is that the entire program
spelled out here is effective. We have built various interpreters for
the reflective calculus at work in this interpretation. In principle,
then, we can simulate quantum mechanics on a computer. The place where
the simulation may lose fidelity is the infinitely branching summation
for the annihilator.

In this connection i also want to point out that the evaluation style
calculation of the inner product puts the non-determinism of the
summation right at the heart of measurement. This suggests that
Milner's original reduction-based formulation of the dynamics of his
calculi in terms of sums was not just notationally suggestive of a
notion of measure-and-continue but captured some significant part of
the physics.

\paragraph{Quantum continuations}
In light of this last observation i want to point out that the
predominant account of quantum mechanics is missing a key aspect of a
truly compositional story of the physical situation. In a real lab,
when a measurement is made the observation can be made to feed into
another device that then makes another measurement conditioned on the
results of the first. This means that after the superposition was
collapsed the entire experimental set up remained in
superposition. While QM offers a means of writing this down it doesn't
quite line up well with the well-trodden formulation of computation
and continuation that we see so succinctly expressed in Milner's
calculi. This suggests that there might be advantages to this account
of dynamics waiting to be explored.

\paragraph{Quantum logic}
In this connection, we also note that by virtue of having the
Hennessy-Milner construction, we can pull the construction through the
interpretation of QM. This gives us a natural candidate for a quantum
logic that enjoys an extremely tight connection with it's domain of
interpretation, making the construction much less ad hoc (rather it is
the image of functor!).

\paragraph{Quantum probabiity}
i have questions about the basis of the interpretation of inner
product as probability amplitude. In particular, using which
axiomatization of probability theory does the notion of probability
amplitude earn the right to be so dubbed? In other words, where is the
proof that the operation for calculating a probability amplitude (and
then squaring) satisfies the axioms of what it means to calculate a
probability? Even if such a proof exists (i have yet to find it in the
literature), i wonder if it might not be possible to turn things on
their heads. Can we view the calculation of the probability amplitude
as an axiomatization of probability? If so, then the definition we
give for calculating probability amplitude may provide the basis for
an \emph{effective} theory of probability.

\paragraph{Quantum vs ``biological'' information}
Finally, i want to conclude with a more philosophical observation. At
a recent workshop in which QM was a predominant topic i noticed
something about quantum information. The speaker was giving a riveting
discussion of axiomatic QM and showing how properties of ``no
cloning'' and ``no deleting'' emerged as consequences of the
axiomatization. Theorems of this form are necessary to give us a sense
of confidence that our axioms characterize the physical theory. What
struck me, though, was that if quantum information is neither erasable
nor replicable it is markedly different from \emph{life}. Two of the
things we know about life is that

\begin{itemize}
  \item it ends;
  \item to gain some measure of persistence, to transcend it's
    finitude it is imminently copyable.
\end{itemize}

Both of these qualities are summarized succinctly in the aphorism: all
flesh is grass. For me these two kinds of ``information'' -- call them
quantum and biological -- are end points on a spectrum of strategies
for persistence. At one end, we have those curious entities that enjoy
uniqueness and permanence; at the other, we have those who in the face
of a certain end and an uncertain present make a go of passing
something on. To me one of the more remarkable aspects of the latter
strategy is that in the presence of noise (and certain features of
copying) we get a kind of dynamism, a chance for improvement against a
given persistent condition.

% subsection other_calculi_other_bisimulations_and_geometry_as_behavior (end)




% section conclusion (end)

%\documentclass[12pt]{llncs}
%\documentclass{jktr}

\usepackage[pdftex]{hyperref}                   
\usepackage {listings}
\usepackage {mathpartir}
\usepackage{bcprules}
%\usepackage{listings}
                       
\usepackage{graphicx} 
%\usepackage[margins=2.5cm,nohead,nofoot]{geometry}
%\usepackage{geometry}
\usepackage{amsfonts}
\usepackage{amstext}
\usepackage{latexsym}
\usepackage{amssymb}
\usepackage{color}


%\include{myPreamble}
\documentclass[12pt]{llncs}
%\documentclass{jktr}

\usepackage[pdftex]{hyperref}                   
\usepackage {listings}
\usepackage {mathpartir}
\usepackage{bcprules}
%\usepackage{listings}
                       
\usepackage{graphicx} 
%\usepackage[margins=2.5cm,nohead,nofoot]{geometry}
%\usepackage{geometry}
\usepackage{amsfonts}
\usepackage{amstext}
\usepackage{latexsym}
\usepackage{amssymb}
\usepackage{color}


%\include{myPreamble}
\include{qm2pi.local} 

%\ifpdf
%\usepackage[pdftex]{graphicx}
%\else
%\usepackage{graphicx}
%\fi

 % \ifpdf
%  \usepackage{pdfsync}
%  \if


%\title{Brief Article}
%\author{David F. Snyder}
%\author{L.G. Meredith}

%\address{Dept. of Math., Texas State University--San Marcos, San Marcos, TX 78666}
       
\pagestyle{empty}


\begin{document}

\lstset{language=[Objective]Caml,frame=shadowbox}

\input{qm2pi.front}

% section front matter (end)

\input{qm2pi.intro} 
 
% section introduction (end)

% \input{qm2pi.knotations} 

% section notation (end)

\input{qm2pi.process.calculi} 

% section concurrent_process_calculi_and_spatial_logics_ (end)
    
%\input{qm2pi.knots2pi} 

%\input{qm2pi.trefoil} 

%\input{qm2pi.mainthm} 

% subsection basic_interpretation (end)

%\input{qm2pi.rho.presentation} 
\subsection{The syntax and semantics of the notation system}\label{sub:the_syntax_and_semantics_of_the_notation_system} % (fold)

We now summarize a technical presentation of the calculus that
embodies our theory of dynamics. The typical presentation of such a
calculus follows the style of giving generators and relations on
them. The grammar, below, describing term constructors, freely
generates the set of processes, $\Proc$. This set is then quotiented
by a relation known as structural congruence and it is over this set
that the notion of dynamics is expressed. This presentation is
essentially that of \cite{MeredithR05} with the addition of
polyadicity and summation. For readability we have relegated some of
the technical subtleties to an appendix.

\subsubsection{Process grammar}\label{subsub:process_grammar}

\begin{mathpar}
  \inferrule* [lab=synchronization] {} {{M} \bc \pzero \;|\; x?F \;|\; x!C }
  \and
  \inferrule* [lab=abstraction] {} {{F} \bc (x)P}
  \and
  \inferrule* [lab=concretion] {} {{C} \bc \langle Q \rangle}
  \and
  \inferrule* [lab=process] {} {{P,Q} \bc M \;| \;P|Q \;|\; @{x}}
  \and
  \inferrule* [lab=name] {} {{x} \bc \quotep{P}}
\end{mathpar} 

Note that $\vec{x}$ (resp. $\vec{P}$) denotes a vector of names
(resp. processes) of length $|\vec{x}|$ (resp. $|\vec{P}|$). We adopt
the following useful abbreviations.

\begin{mathpar}
   x?(\vec{y}).P := x.(\vec{y})P \and  x\clift{\vec{P}} := x.\clift{\vec{P}}
   \and x!(y) := \lift{x}{\dropn{y}}
   \and \Pi_{i=0}^{n-1}P_i := P_0 | \ldots | P_{n-1}
\end{mathpar}

\subsubsection{Structural congruence}

\paragraph{Free and bound names and alpha-equivalence.} At the
core of structural equivalence is alpha-equivalence which identifies
process that are the same up to a change of variable. Formally, we
recognize the distinction between free and bound names. The free names
of a process, $\freenames{P}$, may be calculated recursively as
follows:

\begin{mathpar}
\freenames{\pzero} := \emptyset
  \and \\
  \freenames{x?(y).P} := \{ x \} \cup (\freenames{P} \setminus \{ y \})
  \and 
  \freenames{x!\langle P \rangle} := \{ x \} \cup \{ P \} 
  \and \\
  \freenames{P|Q} := \freenames{P} \cup \freenames{Q}
  \and \\
  \freenames{@{x}} := \{ x \}
\end{mathpar}

$\pi$
$\quotep{\pi}$

$\freenames{-} : \pi \to \mathcal{P}(\quotep{\pi})$

\begin{eqnarray*}
  \freenames{\pzero} & := & \emptyset \\
  \freenames{x?(y).P} & := & \{ x \} \cup (\freenames{P} \setminus \{ y \}) \\
  \freenames{x!\langle P \rangle} & := & \{ x \} \cup \{ P \} \\
  \freenames{P|Q} & := & \freenames{P} \cup \freenames{Q} \\
  \freenames{\dropn{x}} & := & \{ x \}
\end{eqnarray*}

The bound names of a process, $\boundnames{P}$, are those names occurring in $P$
that are not free. For example, in $x?(y).0$, the name $x$ is free, while $y$ is bound.

\begin{mathpar}
  \inferrule* [lab=monoidal-laws] {} { P|Q \equiv Q|P \and P|0 \equiv P \and P|(Q|R) \equiv (P|Q)|R }
\end{mathpar}

\begin{mathpar}
  \inferrule* [lab=alpha-equivalence] {} { (x)P \equiv (y)P\{y/x\} \and y \not\in \freenames{P} }
\end{mathpar}

\begin{definition}
Then two processes, $P,Q$, are alpha-equivalent if $P = Q\{\vec{y}/\vec{x}\}$ for
some $\vec{x} \in \boundnames{Q},\vec{y} \in \boundnames{P}$, where $Q\{\vec{y}/\vec{x}\}$
denotes the capture-avoiding substitution of $\vec{y}$ for $\vec{x}$ in $Q$.
\end{definition}

\begin{definition}
  The {\em structural congruence} \cite{SangiorgiWalker} , $\equiv$,
  between processes is the least congruence containing
  alpha-equivalence, satisfying the abelian monoid laws
  (associativity, commutativity and $\pzero$ as identity) for parallel
  composition $|$ and for summation $+$.
\end{definition}

\subsection{Name equivalence}

We take name equivalence, written $\nameeq$, to be the smallest
equivalence relation generated by the following rules.

\begin{mathpar}
\inferrule*[lab=Quote-drop]
{ }
{ \quotep{@{x}} \nameeq x }

\inferrule*[lab=Struct-equiv]
{ P \scong Q }
{ \quotep{P} \nameeq \quotep{Q} }
\end{mathpar}

The astute reader will have noticed that the mutual recursion of names
and processes imposes a mutual recursion on alpha-equivalence and
structural equivalence via name-equivalence. Fortunately, all of this
works out pleasantly and we may calculate in the natural way, free of
concern. The reader interested in the details is referred to the
appendix \ref{appendix:rho_details}.

\subsection{Substitution}

We use $\Proc$ for the set of processes, $\QProc$ for the set of
names, and $\id{\{}\vec{y} / \vec{x} \id{\}}$ to denote partial maps,
$s : \QProc \rightarrow \QProc$. A map, $s$ lifts, uniquely, to a map
on process terms, $\widehat{s} : \Proc \rightarrow \Proc$ by the
following equations.

\begin{mathpar}
  (0) \psubstp{Q}{P} := 0 \\
  (R \juxtap S) \psubstp{Q}{P}
  :=    
  (R)\psubstp{Q}{P} \juxtap (S) \psubstp{Q}{P} \\
  (x?(y).R) \psubstp{Q}{P}    
  :=    
  (x)\substp{Q}{P} (z)\concat( (R \psubstn{z}{y}) \psubstp{Q}{P} ) \\
  (\lift{x}{R}) \psubstp{Q}{P}  
  :=
  \lift{(x)\substp{Q}{P}}{ R \psubstp{Q}{P} } \\
%   (\dropn{x})  \psubstp{Q}{P}       
%   := 
%   \left\{ 
%     \begin{array}{ccc} 
%       \dropn{\quotep{Q}} & & x \nameeq \quotep{P} \\
%       \dropn{x} & & otherwise \\
%     \end{array}
%   \right. 
  (\dropn{x})  \psubstp{Q}{P}       
  := 
  \left\{ 
    \begin{array}{ccc} 
      Q & & x \nameeq \quotep{P} \\
      \dropn{x} & & otherwise \\
    \end{array}
  \right.
\end{mathpar}
 

where

\begin{eqnarray}
  (x)\id{\{} \lpquote Q \rpquote / \lpquote P \rpquote \id{\}}            = 
  \left\{ 
    \begin{array}{ccc}
      \lpquote Q \rpquote & & x \nameeq \lpquote P \rpquote \\
      x & & otherwise \\
    \end{array}
  \right. \nonumber
\end{eqnarray}

and $z$ is chosen distinct from $\quotep{P}$, $\quotep{Q}$, the free
names in $Q$, and all the names in $R$. Our $\alpha$-equivalence will
be built in the standard way from this substitution.

\begin{remark}\label{rem:no_self_referential_names}
  One consequence of these definitions is that $\forall P. \quotep{P}
  \not\in \freenames{P}$.
\end{remark}

\subsection{ Dynamic quote: an example }

Anticipating something of what's to come, consider applying the
substitution, $\widehat{\id{\{}u / z \id{\}}}$, to the following pair
of processes, $\lift{w}{y!(z)}$ and $w[ \lpquote y!(z) \rpquote ]$.

\begin{eqnarray}
	\lift{w}{y!(z)}\widehat{\id{\{}u / z \id{\}}}
		& = &
		\lift{w}{y!(u)} \nonumber\\
	w[ \lpquote y!(z) \rpquote ] \widehat{ \id{\{}u / z \id{\}} }
		& = &
		w[ \lpquote y!(z) \rpquote ] \nonumber
\end{eqnarray}

Because the body of the process between quotes is impervious to
substitution, we get radically different answers. In fact, by
examining the first process in an input context,
e.g. $x?(z).\lift{w}{y!(z)}$, we see that the process under the lift
operator may be shaped by prefixed inputs binding a name inside it. In
this sense, the lift operator will be seen as a way to dynamically
construct processes before reifying them as names.

Finally equipped with these standard features we can present the
dynamics of the calculus.

\subsubsection{Operational semantics} 

Finally, we introduce the computational dynamics. What marks these
algebras as distinct from other more traditionally studied algebraic
structures, e.g. vector spaces or polynomial rings, is the manner in
which dynamics is captured. In traditional structures, dynamics is typically
expressed through morphisms between such structures, as in linear maps
between vector spaces or morphisms between rings. In algebras
associated with the semantics of computation, the dynamics is
expressed as part of the algebraic structure itself, through a
reduction reduction relation typically denoted by $\red$. Below, we
give a recursive presentation of this relation for the calculus used
in the encoding.

$\red \subseteq \pi \times \pi$
$\red : \pi \to \mathcal{P}(\pi)$

\begin{mathpar}
  \inferrule* [lab=Comm] { \textsf{match}( x_{src}, x_{trgt} ) } { x_{trgt}?(y)P \; | \; x_{src}!\langle {Q} \rangle \red P\{\quotep{Q}/y}\} }
  \and \\
  \inferrule* [lab=Par] {{P} \red {P}'} {{{P} | {Q}} \red {{P}' | {Q}}}
  \and
  \inferrule* [lab=Equiv]{{{P} \scong {P}'} \andalso {{P}' \red {Q}'} \andalso {{Q}' \scong {Q}}}{{P} \red {Q}}
\end{mathpar}

\begin{eqnarray*}
  match_{\equiv} (\quotep{P},\quotep{Q}) & := & P \equiv Q \\
  match_{\dagger}(\quotep{P},\quotep{Q}) & := & \forall R. P|Q \red^{*} R => R \red^{*} 0 \\
  match_{K}(\quotep{P},\quotep{Q}) & := & K \mbox{ for some context } K
\end{eqnarray*}

$u?(x)P | u!\langle Q \rangle \red P\{\quotep{Q}/x\}$

%We write $\wred$ for $\red^*$, and $P\red$ if $\exists Q $ such that $ P \red Q$.
We write $P\red$ if $\exists Q $ such that $ P \red Q$ and $P\not\red$, otherwise.

\section{Replication}

As mentioned before, it is known that replication (and hence
recursion) can be implemented in a higher-order process algebra
\cite{SangiorgiWalker}. As our first example of calculation with the
machinery thus far presented we give the construction explicitly in
the {\rhoc}.

\begin{eqnarray}
	D_{x} & := & \prefix{x}{y}{(\binpar{\outputp{x}{y}}{@{y}})} \nonumber\\
	\bangp_{x}{P} & := & \binpar{{x}!\langle{\binpar{D_{x}}{P}}\rangle}{D_{x}} \nonumber
\end{eqnarray}

\begin{eqnarray}
	\bangp_{x}{P} & & \nonumber\\
	=
	& {x}!\langle{(\prefix{x}{y}{(\outputp{x}{y} | @{y})) | P}}\rangle 
	      | \prefix{x}{y}{(\outputp{x}{y} | @{y})} & \nonumber\\
	\red
	& (\outputp{x}{y} | @{y})\substn{\quotep{(\prefix{x}{y}{(@{y} | \outputp{x}{y})) | P}}}{y} & \nonumber\\
	=
	& \outputp{x}{\quotep{(\prefix{x}{y}{(\outputp{x}{y} | @{y})) | P}}}
	  | {(\prefix{x}{y}{(\outputp{x}{y} | @{y})) | P}} & \nonumber\\
	\red
	& \ldots & \nonumber\\
	\red^*
	& P | P | \ldots & \nonumber
\end{eqnarray}

Of course, this encoding, as an implementation, runs away, unfolding
$\bangp{P}$ eagerly. A lazier and more implementable replication
operator, restricted to input-guarded processes, may be obtained as follows.

\begin{eqnarray}
\bangp{\prefix{u}{v}{P}} 
	:= 
	\binpar{\lift{x}{\prefix{u}{v}{(\binpar{D(x)}{P})}}}{D(x)} \nonumber
\end{eqnarray}

\begin{remark}
  Note that the lazier definition still does not deal with summation
  or mixed summation (i.e. sums over input and output). The reader is
  invited to construct definitions of replication that deal with these
  features. 

  Further, the definitions are parameterized in a name, $x$. Can you,
  gentle reader, make a definition that eliminates this parameter and
  guarantees no accidental interaction between the replication
  machinery and the process being replicated -- i.e. no accidental
  sharing of names used by the process to get its work done and the
  name(s) used by the replication to effect copying. This latter
  revision of the definition of replication is crucial to obtaining
  the expected identity $!!P \sim !P$.
\end{remark}

\begin{remark}\label{rem:paradoxical_combinator}
  The reader familiar with the lambda calculus will have noticed the
  similarity between $D$ and the paradoxical combinator.

  [Ed. note: the existence of this seems to suggest we have to be more
  restrictive on the set of processes and names we admit if we are to
  support no-cloning.]
\end{remark}

\subsubsection{Bisimulation}

The computational dynamics gives rise to another kind of equivalence,
the equivalence of computational behavior. As previously mentioned
this is typically captured \emph{via} some form of bisimulation.

% The notion we use in this paper is weak barbed bisimulation
% \cite{milner91polyadicpi}.

The notion we use in this paper is derived from weak barbed
bisimulation \cite{milner91polyadicpi}. 

\begin{definition}
An \emph{observation relation}, $\downarrow_{\mathcal N}$, over a set
of names, $\mathcal N$, is the smallest relation satisfying the rules
below.

\infrule[Out-barb]{y \in {\mathcal N}, \; x \nameeq y}
		  {\outputp{x}{v} \downarrow_{\mathcal N} x}
\infrule[Par-barb]{\mbox{$P\downarrow_{\mathcal N} x$ or $Q\downarrow_{\mathcal N} x$}}
		  {\binpar{P}{Q} \downarrow_{\mathcal N} x}

We write $P \Downarrow_{\mathcal N} x$ if there is $Q$ such that 
$P \wred Q$ and $Q \downarrow_{\mathcal N} x$.
\end{definition}

\begin{definition}
%\label{def.bbisim}
An  ${\mathcal N}$-\emph{barbed bisimulation} over a set of names, ${\mathcal N}$, is a symmetric binary relation 
${\mathcal S}_{\mathcal N}$ between agents such that $P\rel{S}_{\mathcal N}Q$ implies:
\begin{enumerate}
\item If $P \red P'$ then $Q \wred Q'$ and $P'\rel{S}_{\mathcal N} Q'$.
\item If $P\downarrow_{\mathcal N} x$, then $Q\Downarrow_{\mathcal N} x$.
\end{enumerate}
$P$ is ${\mathcal N}$-barbed bisimilar to $Q$, written
$P \wbbisim_{\mathcal N} Q$, if $P \rel{S}_{\mathcal N} Q$ for some ${\mathcal N}$-barbed bisimulation ${\mathcal S}_{\mathcal N}$.
\end{definition}

$\mathcal{R} \subseteq \pi \times \pi$

$P \mathcal{R} Q => \forall P'. P \red P' \Rightarrow \exists Q'. Q \red Q', P' \mathcal{R} Q'$

$P \vdash x \Rightarrow Q \vdash x$

\begin{mathpar}
  \inferrule*[lab=Out-barb]{x \nameeq y}{{y}!\langle{Q}\rangle \vdash x}
  \and
  \inferrule*[lab=Par-barb]{\mbox{$P\vdash x$ or $Q\vdash x$}}{\binpar{P}{Q} \vdash x}
\end{mathpar}

\subsubsection{Contexts}

One of the principle advantages of computational calculi like the
$\pi$-calculus is a well-defined notion of context,
contextual-equivalence and a correlation between
contextual-equivalence and notions of bisimulation. The notion of
context allows the decomposition of a process into (sub-)process and
its syntactic environment, its context. Thus, a context may be
thought of as a process with a ``hole'' (written $\Box$) in it. The
application of a context $M$ to a process $P$, written $M[P]$, is
tantamount to filling the hole in $M$ with $P$. In this paper we do
not need the full weight of this theory, but do make use of the notion
of context in the proof the main theorem. 

\begin{mathpar}
  \inferrule* [lab=summation] {} {{M_{M},M_{N}} \bc \Box \;|\; x.M_{A} \;|\; M_{M}+M_{N}}
  \and
  \inferrule* [lab=agent] {} {{M_{A}} \bc (\vec{x})M_{P} \;| \; \clift{P_0,\ldots,M_{P},\ldots,P_N}}
  \and \\
  \inferrule* [lab=process] {} {{M_{P}} \bc M_{N} \;| \;P|M_{P} }
\end{mathpar} 

\begin{mathpar}
  \inferrule* [lab=sychronization] {} {M_{N} \bc \Box \;|\; x?M_{F} \;|\; x!M_{C}}
  \and
  \inferrule* [lab=abstraction] {} {{M_{F}} \bc (x)M_{P} }
  \and
  \inferrule* [lab=concretion] {} {{M_{C}} \bc \langle M_{P} \rangle }
  \and \\
  \inferrule* [lab=process] {} {{M_{P}} \bc M_{N} \;| \;P|M_{P} }
\end{mathpar}

\begin{definition}[contextual application] Given a context $M$, and
  process $P$, we define the \emph{contextual application}, $M[P] :=
  M\{P/\Box\}$. That is, the contextual application of M to P is the
  substitution of $P$ for $\Box$ in $M$.
\end{definition}

$\meaningof{-} : L \to \mathcal{P}(\pi)$

\begin{mathpar}
  \inferrule* [lab=collection] {} {\meaningof{true} = \pi, \and \meaningof{~E} = \pi \setminus \meaningof{E}, \and \meaningof{E_{1} \& E_{2}} = \meaningof{E_{1}} \cap \meaningof{E_{2}}}
\end{mathpar}

\begin{mathpar}
  \inferrule* [lab=structure] {} {\meaningof{0} = \{ P \in \pi | P \equiv 0 \}, \and \\ \meaningof{E_1 | E_2} = \{ P \in \pi | P \equiv P_{1} | P_{2}, P_{1} \in \meaningof{E_{1}}, P_{2} \in \meaningof{E_2}\} }
\end{mathpar}

\begin{mathpar}
 \inferrule* [lab=behavior] {} {\meaningof{\langle a?b \rangle E} = \{ P \in \pi | P \equiv Q | u?(y)P', \\ \and \\\\ \and \\ \;\;\; u \in \meaningof{a}, \forall z.P'\{z/y\} \in \meaningof{E\{z/b\}}\}, \and \\ \meaningof{a!E} = \{ P \in \pi | P \equiv Q | x!\langle P' \rangle, x \in \meaningof{a} P' \in \meaningof{E}\} }
\end{mathpar}

\begin{mathpar}
 \inferrule* [lab=nominal] {} {\meaningof{\quotep{E}} = \{ \quotep{P} \in \quotep{\pi} | P \in \meaningof{E} \}, \and \meaningof{\quotep{P}} = \{ \quotep{Q} \in \quotep{\pi} | P \equiv Q \} \and \\ \meaningof{@\quotep{E}} = \{ P \in \pi | P \equiv @x, x \in \meaningof{E} \}}
\end{mathpar}

\begin{eqnarray*}
  \\
  \meaningof{-} : TS \to ST
\end{eqnarray*}

\begin{eqnarray*}
  \\
  L : TS \to ST
\end{eqnarray*}

\begin{eqnarray*}
  \\
  P \models E \iff P \in \meaningof{E}
\end{eqnarray*}

\begin{eqnarray*}
  P \approx_{L} Q \iff \forall E \in L. P \models E \iff Q \models E
\end{eqnarray*}

\begin{eqnarray*}
  P \approx_{K} Q
\end{eqnarray*}

\begin{eqnarray*}
  P \approx Q
\end{eqnarray*}

$\approx_{K} = \approx = \approx_{L}$

\subsubsection{Contextual duality}

Note that contexts extend the quotation operation to a family of
operations from processes to names. Given a context, $M$, we can
define a \emph{nominal context}, $\quotep{M}$ by $\quotep{M}[P] :=
\quotep{M[P]}$. To foreshadow what is to come we observe that these
operations enjoy a duality with processes very much like the duality
between vectors and maps from vectors to scalars.

Further, because the calculus is essentially higher-order, we have a
correspondence between contexts and processes. More specifically,
given a name $x$ and a context $M$ we can construct $M^{*}_{x}$ such
that 

\begin{mathpar}
  M^{*}_{x} | \lift{x}{P} \red M[P]
\end{mathpar}

namely,

\begin{mathpar}
  M^{*}_{x} := x?(u).M[\dropn{u}]
\end{mathpar}

The dependence of $M^{*}_{x}$ on a name makes it an abstraction, 

\begin{mathpar}
  M^{*} := (x)x?(u).M[\dropn{u}]
\end{mathpar}

\subsection{Additional notation}

It will sometimes be convenient to denote the process a name
quotes. We already have the notation $x = \quotep{P}$, but it will be
convenient to introduce an alternate notation, $\procn{x}$, when we
want to emphasize the connection to the use of the name. Note that, by
virtue of name equivalence, $\quotep{\procn{x}} \nameeq x$; so, the
notation is consistent with previous definitions.

Further, because names have structure it is possible to effect
substitutions on the basis of that structure. This means we need to
upgrade our notation for substitutions, which we accomplish by
adapting comprehension notation. Thus,

\begin{mathpar}
  P\{ y / x : x \in S \}
\end{mathpar}

is interpreted to mean the process derived from P by replacing (in a
capture-avoiding manner) each occurrence of $x$ in $S$ by $y$. For example,

\begin{mathpar}
  P\{ \quotep{\procn{x}|\procn{x}} / x : x \in \freenames{P} \}
\end{mathpar}

will replace each (occurrence) of a free name $x$ in $P$ by
$\quotep{\procn{x}|\procn{x}}$.

Also, we will avail ourselves of the notation $x^{L}$ and $x^{R}$ to
denote injections of a name into disjoint copies of the name
space. There are numerous ways to accomplish this. One example can be
found in \cite{MeredithR05}. This notation overloads to vectors of
names: $\vec{x}^{\pi} := (x_{i}^{\pi} \; : \; 0 \leq i < |\vec{x}| )$ where $\pi \in \{L,R\}$.

We also use $P^{\Box} := P|\Box$.

In \cite{MeredithR05} an interpretation of the new operator is
given. It turns out that there are several possible interpretations
all enjoying the requisite algebraic properties of the operator (see
\cite{milner91polyadicpi}). We will therefore make liberal use of
$(\nu\; \vec{x})P$.

% subsection the_syntax_and_semantics_of_the_notation_system (end)   

\input{qm2pi.qmops} 

\input{qm2pi.sterngerlach} 

\input{qm2pi.metric} 

% section concurrent_process_calculi (end)

%\input{qm2pi.proofsketch}

% section proof sketch (end)

%\input{qm2pi.slviaknots} 

% section spatial logic via knots (end)

\input{qm2pi.conclusion}

% section conclusion (end)

%\input{qm2pi.dtcodes} 

% section wiring algorithm (end)

\input{qm2pi.ack} 

% section acknowledgments (end)

\newpage


\bibliographystyle{plain}   
\bibliography{../../biblios/main.bib}

\input{qm2pi.rhodetails}

\end{document}

 

%\ifpdf
%\usepackage[pdftex]{graphicx}
%\else
%\usepackage{graphicx}
%\fi

 % \ifpdf
%  \usepackage{pdfsync}
%  \if


%\title{Brief Article}
%\author{David F. Snyder}
%\author{L.G. Meredith}

%\address{Dept. of Math., Texas State University--San Marcos, San Marcos, TX 78666}
       
\pagestyle{empty}


\begin{document}

\lstset{language=[Objective]Caml,frame=shadowbox}

\documentclass[12pt]{llncs}
%\documentclass{jktr}

\usepackage[pdftex]{hyperref}                   
\usepackage {listings}
\usepackage {mathpartir}
\usepackage{bcprules}
%\usepackage{listings}
                       
\usepackage{graphicx} 
%\usepackage[margins=2.5cm,nohead,nofoot]{geometry}
%\usepackage{geometry}
\usepackage{amsfonts}
\usepackage{amstext}
\usepackage{latexsym}
\usepackage{amssymb}
\usepackage{color}


%\include{myPreamble}
\include{qm2pi.local} 

%\ifpdf
%\usepackage[pdftex]{graphicx}
%\else
%\usepackage{graphicx}
%\fi

 % \ifpdf
%  \usepackage{pdfsync}
%  \if


%\title{Brief Article}
%\author{David F. Snyder}
%\author{L.G. Meredith}

%\address{Dept. of Math., Texas State University--San Marcos, San Marcos, TX 78666}
       
\pagestyle{empty}


\begin{document}

\lstset{language=[Objective]Caml,frame=shadowbox}

\input{qm2pi.front}

% section front matter (end)

\input{qm2pi.intro} 
 
% section introduction (end)

% \input{qm2pi.knotations} 

% section notation (end)

\input{qm2pi.process.calculi} 

% section concurrent_process_calculi_and_spatial_logics_ (end)
    
%\input{qm2pi.knots2pi} 

%\input{qm2pi.trefoil} 

%\input{qm2pi.mainthm} 

% subsection basic_interpretation (end)

%\input{qm2pi.rho.presentation} 
\subsection{The syntax and semantics of the notation system}\label{sub:the_syntax_and_semantics_of_the_notation_system} % (fold)

We now summarize a technical presentation of the calculus that
embodies our theory of dynamics. The typical presentation of such a
calculus follows the style of giving generators and relations on
them. The grammar, below, describing term constructors, freely
generates the set of processes, $\Proc$. This set is then quotiented
by a relation known as structural congruence and it is over this set
that the notion of dynamics is expressed. This presentation is
essentially that of \cite{MeredithR05} with the addition of
polyadicity and summation. For readability we have relegated some of
the technical subtleties to an appendix.

\subsubsection{Process grammar}\label{subsub:process_grammar}

\begin{mathpar}
  \inferrule* [lab=synchronization] {} {{M} \bc \pzero \;|\; x?F \;|\; x!C }
  \and
  \inferrule* [lab=abstraction] {} {{F} \bc (x)P}
  \and
  \inferrule* [lab=concretion] {} {{C} \bc \langle Q \rangle}
  \and
  \inferrule* [lab=process] {} {{P,Q} \bc M \;| \;P|Q \;|\; @{x}}
  \and
  \inferrule* [lab=name] {} {{x} \bc \quotep{P}}
\end{mathpar} 

Note that $\vec{x}$ (resp. $\vec{P}$) denotes a vector of names
(resp. processes) of length $|\vec{x}|$ (resp. $|\vec{P}|$). We adopt
the following useful abbreviations.

\begin{mathpar}
   x?(\vec{y}).P := x.(\vec{y})P \and  x\clift{\vec{P}} := x.\clift{\vec{P}}
   \and x!(y) := \lift{x}{\dropn{y}}
   \and \Pi_{i=0}^{n-1}P_i := P_0 | \ldots | P_{n-1}
\end{mathpar}

\subsubsection{Structural congruence}

\paragraph{Free and bound names and alpha-equivalence.} At the
core of structural equivalence is alpha-equivalence which identifies
process that are the same up to a change of variable. Formally, we
recognize the distinction between free and bound names. The free names
of a process, $\freenames{P}$, may be calculated recursively as
follows:

\begin{mathpar}
\freenames{\pzero} := \emptyset
  \and \\
  \freenames{x?(y).P} := \{ x \} \cup (\freenames{P} \setminus \{ y \})
  \and 
  \freenames{x!\langle P \rangle} := \{ x \} \cup \{ P \} 
  \and \\
  \freenames{P|Q} := \freenames{P} \cup \freenames{Q}
  \and \\
  \freenames{@{x}} := \{ x \}
\end{mathpar}

$\pi$
$\quotep{\pi}$

$\freenames{-} : \pi \to \mathcal{P}(\quotep{\pi})$

\begin{eqnarray*}
  \freenames{\pzero} & := & \emptyset \\
  \freenames{x?(y).P} & := & \{ x \} \cup (\freenames{P} \setminus \{ y \}) \\
  \freenames{x!\langle P \rangle} & := & \{ x \} \cup \{ P \} \\
  \freenames{P|Q} & := & \freenames{P} \cup \freenames{Q} \\
  \freenames{\dropn{x}} & := & \{ x \}
\end{eqnarray*}

The bound names of a process, $\boundnames{P}$, are those names occurring in $P$
that are not free. For example, in $x?(y).0$, the name $x$ is free, while $y$ is bound.

\begin{mathpar}
  \inferrule* [lab=monoidal-laws] {} { P|Q \equiv Q|P \and P|0 \equiv P \and P|(Q|R) \equiv (P|Q)|R }
\end{mathpar}

\begin{mathpar}
  \inferrule* [lab=alpha-equivalence] {} { (x)P \equiv (y)P\{y/x\} \and y \not\in \freenames{P} }
\end{mathpar}

\begin{definition}
Then two processes, $P,Q$, are alpha-equivalent if $P = Q\{\vec{y}/\vec{x}\}$ for
some $\vec{x} \in \boundnames{Q},\vec{y} \in \boundnames{P}$, where $Q\{\vec{y}/\vec{x}\}$
denotes the capture-avoiding substitution of $\vec{y}$ for $\vec{x}$ in $Q$.
\end{definition}

\begin{definition}
  The {\em structural congruence} \cite{SangiorgiWalker} , $\equiv$,
  between processes is the least congruence containing
  alpha-equivalence, satisfying the abelian monoid laws
  (associativity, commutativity and $\pzero$ as identity) for parallel
  composition $|$ and for summation $+$.
\end{definition}

\subsection{Name equivalence}

We take name equivalence, written $\nameeq$, to be the smallest
equivalence relation generated by the following rules.

\begin{mathpar}
\inferrule*[lab=Quote-drop]
{ }
{ \quotep{@{x}} \nameeq x }

\inferrule*[lab=Struct-equiv]
{ P \scong Q }
{ \quotep{P} \nameeq \quotep{Q} }
\end{mathpar}

The astute reader will have noticed that the mutual recursion of names
and processes imposes a mutual recursion on alpha-equivalence and
structural equivalence via name-equivalence. Fortunately, all of this
works out pleasantly and we may calculate in the natural way, free of
concern. The reader interested in the details is referred to the
appendix \ref{appendix:rho_details}.

\subsection{Substitution}

We use $\Proc$ for the set of processes, $\QProc$ for the set of
names, and $\id{\{}\vec{y} / \vec{x} \id{\}}$ to denote partial maps,
$s : \QProc \rightarrow \QProc$. A map, $s$ lifts, uniquely, to a map
on process terms, $\widehat{s} : \Proc \rightarrow \Proc$ by the
following equations.

\begin{mathpar}
  (0) \psubstp{Q}{P} := 0 \\
  (R \juxtap S) \psubstp{Q}{P}
  :=    
  (R)\psubstp{Q}{P} \juxtap (S) \psubstp{Q}{P} \\
  (x?(y).R) \psubstp{Q}{P}    
  :=    
  (x)\substp{Q}{P} (z)\concat( (R \psubstn{z}{y}) \psubstp{Q}{P} ) \\
  (\lift{x}{R}) \psubstp{Q}{P}  
  :=
  \lift{(x)\substp{Q}{P}}{ R \psubstp{Q}{P} } \\
%   (\dropn{x})  \psubstp{Q}{P}       
%   := 
%   \left\{ 
%     \begin{array}{ccc} 
%       \dropn{\quotep{Q}} & & x \nameeq \quotep{P} \\
%       \dropn{x} & & otherwise \\
%     \end{array}
%   \right. 
  (\dropn{x})  \psubstp{Q}{P}       
  := 
  \left\{ 
    \begin{array}{ccc} 
      Q & & x \nameeq \quotep{P} \\
      \dropn{x} & & otherwise \\
    \end{array}
  \right.
\end{mathpar}
 

where

\begin{eqnarray}
  (x)\id{\{} \lpquote Q \rpquote / \lpquote P \rpquote \id{\}}            = 
  \left\{ 
    \begin{array}{ccc}
      \lpquote Q \rpquote & & x \nameeq \lpquote P \rpquote \\
      x & & otherwise \\
    \end{array}
  \right. \nonumber
\end{eqnarray}

and $z$ is chosen distinct from $\quotep{P}$, $\quotep{Q}$, the free
names in $Q$, and all the names in $R$. Our $\alpha$-equivalence will
be built in the standard way from this substitution.

\begin{remark}\label{rem:no_self_referential_names}
  One consequence of these definitions is that $\forall P. \quotep{P}
  \not\in \freenames{P}$.
\end{remark}

\subsection{ Dynamic quote: an example }

Anticipating something of what's to come, consider applying the
substitution, $\widehat{\id{\{}u / z \id{\}}}$, to the following pair
of processes, $\lift{w}{y!(z)}$ and $w[ \lpquote y!(z) \rpquote ]$.

\begin{eqnarray}
	\lift{w}{y!(z)}\widehat{\id{\{}u / z \id{\}}}
		& = &
		\lift{w}{y!(u)} \nonumber\\
	w[ \lpquote y!(z) \rpquote ] \widehat{ \id{\{}u / z \id{\}} }
		& = &
		w[ \lpquote y!(z) \rpquote ] \nonumber
\end{eqnarray}

Because the body of the process between quotes is impervious to
substitution, we get radically different answers. In fact, by
examining the first process in an input context,
e.g. $x?(z).\lift{w}{y!(z)}$, we see that the process under the lift
operator may be shaped by prefixed inputs binding a name inside it. In
this sense, the lift operator will be seen as a way to dynamically
construct processes before reifying them as names.

Finally equipped with these standard features we can present the
dynamics of the calculus.

\subsubsection{Operational semantics} 

Finally, we introduce the computational dynamics. What marks these
algebras as distinct from other more traditionally studied algebraic
structures, e.g. vector spaces or polynomial rings, is the manner in
which dynamics is captured. In traditional structures, dynamics is typically
expressed through morphisms between such structures, as in linear maps
between vector spaces or morphisms between rings. In algebras
associated with the semantics of computation, the dynamics is
expressed as part of the algebraic structure itself, through a
reduction reduction relation typically denoted by $\red$. Below, we
give a recursive presentation of this relation for the calculus used
in the encoding.

$\red \subseteq \pi \times \pi$
$\red : \pi \to \mathcal{P}(\pi)$

\begin{mathpar}
  \inferrule* [lab=Comm] { \textsf{match}( x_{src}, x_{trgt} ) } { x_{trgt}?(y)P \; | \; x_{src}!\langle {Q} \rangle \red P\{\quotep{Q}/y}\} }
  \and \\
  \inferrule* [lab=Par] {{P} \red {P}'} {{{P} | {Q}} \red {{P}' | {Q}}}
  \and
  \inferrule* [lab=Equiv]{{{P} \scong {P}'} \andalso {{P}' \red {Q}'} \andalso {{Q}' \scong {Q}}}{{P} \red {Q}}
\end{mathpar}

\begin{eqnarray*}
  match_{\equiv} (\quotep{P},\quotep{Q}) & := & P \equiv Q \\
  match_{\dagger}(\quotep{P},\quotep{Q}) & := & \forall R. P|Q \red^{*} R => R \red^{*} 0 \\
  match_{K}(\quotep{P},\quotep{Q}) & := & K \mbox{ for some context } K
\end{eqnarray*}

$u?(x)P | u!\langle Q \rangle \red P\{\quotep{Q}/x\}$

%We write $\wred$ for $\red^*$, and $P\red$ if $\exists Q $ such that $ P \red Q$.
We write $P\red$ if $\exists Q $ such that $ P \red Q$ and $P\not\red$, otherwise.

\section{Replication}

As mentioned before, it is known that replication (and hence
recursion) can be implemented in a higher-order process algebra
\cite{SangiorgiWalker}. As our first example of calculation with the
machinery thus far presented we give the construction explicitly in
the {\rhoc}.

\begin{eqnarray}
	D_{x} & := & \prefix{x}{y}{(\binpar{\outputp{x}{y}}{@{y}})} \nonumber\\
	\bangp_{x}{P} & := & \binpar{{x}!\langle{\binpar{D_{x}}{P}}\rangle}{D_{x}} \nonumber
\end{eqnarray}

\begin{eqnarray}
	\bangp_{x}{P} & & \nonumber\\
	=
	& {x}!\langle{(\prefix{x}{y}{(\outputp{x}{y} | @{y})) | P}}\rangle 
	      | \prefix{x}{y}{(\outputp{x}{y} | @{y})} & \nonumber\\
	\red
	& (\outputp{x}{y} | @{y})\substn{\quotep{(\prefix{x}{y}{(@{y} | \outputp{x}{y})) | P}}}{y} & \nonumber\\
	=
	& \outputp{x}{\quotep{(\prefix{x}{y}{(\outputp{x}{y} | @{y})) | P}}}
	  | {(\prefix{x}{y}{(\outputp{x}{y} | @{y})) | P}} & \nonumber\\
	\red
	& \ldots & \nonumber\\
	\red^*
	& P | P | \ldots & \nonumber
\end{eqnarray}

Of course, this encoding, as an implementation, runs away, unfolding
$\bangp{P}$ eagerly. A lazier and more implementable replication
operator, restricted to input-guarded processes, may be obtained as follows.

\begin{eqnarray}
\bangp{\prefix{u}{v}{P}} 
	:= 
	\binpar{\lift{x}{\prefix{u}{v}{(\binpar{D(x)}{P})}}}{D(x)} \nonumber
\end{eqnarray}

\begin{remark}
  Note that the lazier definition still does not deal with summation
  or mixed summation (i.e. sums over input and output). The reader is
  invited to construct definitions of replication that deal with these
  features. 

  Further, the definitions are parameterized in a name, $x$. Can you,
  gentle reader, make a definition that eliminates this parameter and
  guarantees no accidental interaction between the replication
  machinery and the process being replicated -- i.e. no accidental
  sharing of names used by the process to get its work done and the
  name(s) used by the replication to effect copying. This latter
  revision of the definition of replication is crucial to obtaining
  the expected identity $!!P \sim !P$.
\end{remark}

\begin{remark}\label{rem:paradoxical_combinator}
  The reader familiar with the lambda calculus will have noticed the
  similarity between $D$ and the paradoxical combinator.

  [Ed. note: the existence of this seems to suggest we have to be more
  restrictive on the set of processes and names we admit if we are to
  support no-cloning.]
\end{remark}

\subsubsection{Bisimulation}

The computational dynamics gives rise to another kind of equivalence,
the equivalence of computational behavior. As previously mentioned
this is typically captured \emph{via} some form of bisimulation.

% The notion we use in this paper is weak barbed bisimulation
% \cite{milner91polyadicpi}.

The notion we use in this paper is derived from weak barbed
bisimulation \cite{milner91polyadicpi}. 

\begin{definition}
An \emph{observation relation}, $\downarrow_{\mathcal N}$, over a set
of names, $\mathcal N$, is the smallest relation satisfying the rules
below.

\infrule[Out-barb]{y \in {\mathcal N}, \; x \nameeq y}
		  {\outputp{x}{v} \downarrow_{\mathcal N} x}
\infrule[Par-barb]{\mbox{$P\downarrow_{\mathcal N} x$ or $Q\downarrow_{\mathcal N} x$}}
		  {\binpar{P}{Q} \downarrow_{\mathcal N} x}

We write $P \Downarrow_{\mathcal N} x$ if there is $Q$ such that 
$P \wred Q$ and $Q \downarrow_{\mathcal N} x$.
\end{definition}

\begin{definition}
%\label{def.bbisim}
An  ${\mathcal N}$-\emph{barbed bisimulation} over a set of names, ${\mathcal N}$, is a symmetric binary relation 
${\mathcal S}_{\mathcal N}$ between agents such that $P\rel{S}_{\mathcal N}Q$ implies:
\begin{enumerate}
\item If $P \red P'$ then $Q \wred Q'$ and $P'\rel{S}_{\mathcal N} Q'$.
\item If $P\downarrow_{\mathcal N} x$, then $Q\Downarrow_{\mathcal N} x$.
\end{enumerate}
$P$ is ${\mathcal N}$-barbed bisimilar to $Q$, written
$P \wbbisim_{\mathcal N} Q$, if $P \rel{S}_{\mathcal N} Q$ for some ${\mathcal N}$-barbed bisimulation ${\mathcal S}_{\mathcal N}$.
\end{definition}

$\mathcal{R} \subseteq \pi \times \pi$

$P \mathcal{R} Q => \forall P'. P \red P' \Rightarrow \exists Q'. Q \red Q', P' \mathcal{R} Q'$

$P \vdash x \Rightarrow Q \vdash x$

\begin{mathpar}
  \inferrule*[lab=Out-barb]{x \nameeq y}{{y}!\langle{Q}\rangle \vdash x}
  \and
  \inferrule*[lab=Par-barb]{\mbox{$P\vdash x$ or $Q\vdash x$}}{\binpar{P}{Q} \vdash x}
\end{mathpar}

\subsubsection{Contexts}

One of the principle advantages of computational calculi like the
$\pi$-calculus is a well-defined notion of context,
contextual-equivalence and a correlation between
contextual-equivalence and notions of bisimulation. The notion of
context allows the decomposition of a process into (sub-)process and
its syntactic environment, its context. Thus, a context may be
thought of as a process with a ``hole'' (written $\Box$) in it. The
application of a context $M$ to a process $P$, written $M[P]$, is
tantamount to filling the hole in $M$ with $P$. In this paper we do
not need the full weight of this theory, but do make use of the notion
of context in the proof the main theorem. 

\begin{mathpar}
  \inferrule* [lab=summation] {} {{M_{M},M_{N}} \bc \Box \;|\; x.M_{A} \;|\; M_{M}+M_{N}}
  \and
  \inferrule* [lab=agent] {} {{M_{A}} \bc (\vec{x})M_{P} \;| \; \clift{P_0,\ldots,M_{P},\ldots,P_N}}
  \and \\
  \inferrule* [lab=process] {} {{M_{P}} \bc M_{N} \;| \;P|M_{P} }
\end{mathpar} 

\begin{mathpar}
  \inferrule* [lab=sychronization] {} {M_{N} \bc \Box \;|\; x?M_{F} \;|\; x!M_{C}}
  \and
  \inferrule* [lab=abstraction] {} {{M_{F}} \bc (x)M_{P} }
  \and
  \inferrule* [lab=concretion] {} {{M_{C}} \bc \langle M_{P} \rangle }
  \and \\
  \inferrule* [lab=process] {} {{M_{P}} \bc M_{N} \;| \;P|M_{P} }
\end{mathpar}

\begin{definition}[contextual application] Given a context $M$, and
  process $P$, we define the \emph{contextual application}, $M[P] :=
  M\{P/\Box\}$. That is, the contextual application of M to P is the
  substitution of $P$ for $\Box$ in $M$.
\end{definition}

$\meaningof{-} : L \to \mathcal{P}(\pi)$

\begin{mathpar}
  \inferrule* [lab=collection] {} {\meaningof{true} = \pi, \and \meaningof{~E} = \pi \setminus \meaningof{E}, \and \meaningof{E_{1} \& E_{2}} = \meaningof{E_{1}} \cap \meaningof{E_{2}}}
\end{mathpar}

\begin{mathpar}
  \inferrule* [lab=structure] {} {\meaningof{0} = \{ P \in \pi | P \equiv 0 \}, \and \\ \meaningof{E_1 | E_2} = \{ P \in \pi | P \equiv P_{1} | P_{2}, P_{1} \in \meaningof{E_{1}}, P_{2} \in \meaningof{E_2}\} }
\end{mathpar}

\begin{mathpar}
 \inferrule* [lab=behavior] {} {\meaningof{\langle a?b \rangle E} = \{ P \in \pi | P \equiv Q | u?(y)P', \\ \and \\\\ \and \\ \;\;\; u \in \meaningof{a}, \forall z.P'\{z/y\} \in \meaningof{E\{z/b\}}\}, \and \\ \meaningof{a!E} = \{ P \in \pi | P \equiv Q | x!\langle P' \rangle, x \in \meaningof{a} P' \in \meaningof{E}\} }
\end{mathpar}

\begin{mathpar}
 \inferrule* [lab=nominal] {} {\meaningof{\quotep{E}} = \{ \quotep{P} \in \quotep{\pi} | P \in \meaningof{E} \}, \and \meaningof{\quotep{P}} = \{ \quotep{Q} \in \quotep{\pi} | P \equiv Q \} \and \\ \meaningof{@\quotep{E}} = \{ P \in \pi | P \equiv @x, x \in \meaningof{E} \}}
\end{mathpar}

\begin{eqnarray*}
  \\
  \meaningof{-} : TS \to ST
\end{eqnarray*}

\begin{eqnarray*}
  \\
  L : TS \to ST
\end{eqnarray*}

\begin{eqnarray*}
  \\
  P \models E \iff P \in \meaningof{E}
\end{eqnarray*}

\begin{eqnarray*}
  P \approx_{L} Q \iff \forall E \in L. P \models E \iff Q \models E
\end{eqnarray*}

\begin{eqnarray*}
  P \approx_{K} Q
\end{eqnarray*}

\begin{eqnarray*}
  P \approx Q
\end{eqnarray*}

$\approx_{K} = \approx = \approx_{L}$

\subsubsection{Contextual duality}

Note that contexts extend the quotation operation to a family of
operations from processes to names. Given a context, $M$, we can
define a \emph{nominal context}, $\quotep{M}$ by $\quotep{M}[P] :=
\quotep{M[P]}$. To foreshadow what is to come we observe that these
operations enjoy a duality with processes very much like the duality
between vectors and maps from vectors to scalars.

Further, because the calculus is essentially higher-order, we have a
correspondence between contexts and processes. More specifically,
given a name $x$ and a context $M$ we can construct $M^{*}_{x}$ such
that 

\begin{mathpar}
  M^{*}_{x} | \lift{x}{P} \red M[P]
\end{mathpar}

namely,

\begin{mathpar}
  M^{*}_{x} := x?(u).M[\dropn{u}]
\end{mathpar}

The dependence of $M^{*}_{x}$ on a name makes it an abstraction, 

\begin{mathpar}
  M^{*} := (x)x?(u).M[\dropn{u}]
\end{mathpar}

\subsection{Additional notation}

It will sometimes be convenient to denote the process a name
quotes. We already have the notation $x = \quotep{P}$, but it will be
convenient to introduce an alternate notation, $\procn{x}$, when we
want to emphasize the connection to the use of the name. Note that, by
virtue of name equivalence, $\quotep{\procn{x}} \nameeq x$; so, the
notation is consistent with previous definitions.

Further, because names have structure it is possible to effect
substitutions on the basis of that structure. This means we need to
upgrade our notation for substitutions, which we accomplish by
adapting comprehension notation. Thus,

\begin{mathpar}
  P\{ y / x : x \in S \}
\end{mathpar}

is interpreted to mean the process derived from P by replacing (in a
capture-avoiding manner) each occurrence of $x$ in $S$ by $y$. For example,

\begin{mathpar}
  P\{ \quotep{\procn{x}|\procn{x}} / x : x \in \freenames{P} \}
\end{mathpar}

will replace each (occurrence) of a free name $x$ in $P$ by
$\quotep{\procn{x}|\procn{x}}$.

Also, we will avail ourselves of the notation $x^{L}$ and $x^{R}$ to
denote injections of a name into disjoint copies of the name
space. There are numerous ways to accomplish this. One example can be
found in \cite{MeredithR05}. This notation overloads to vectors of
names: $\vec{x}^{\pi} := (x_{i}^{\pi} \; : \; 0 \leq i < |\vec{x}| )$ where $\pi \in \{L,R\}$.

We also use $P^{\Box} := P|\Box$.

In \cite{MeredithR05} an interpretation of the new operator is
given. It turns out that there are several possible interpretations
all enjoying the requisite algebraic properties of the operator (see
\cite{milner91polyadicpi}). We will therefore make liberal use of
$(\nu\; \vec{x})P$.

% subsection the_syntax_and_semantics_of_the_notation_system (end)   

\input{qm2pi.qmops} 

\input{qm2pi.sterngerlach} 

\input{qm2pi.metric} 

% section concurrent_process_calculi (end)

%\input{qm2pi.proofsketch}

% section proof sketch (end)

%\input{qm2pi.slviaknots} 

% section spatial logic via knots (end)

\input{qm2pi.conclusion}

% section conclusion (end)

%\input{qm2pi.dtcodes} 

% section wiring algorithm (end)

\input{qm2pi.ack} 

% section acknowledgments (end)

\newpage


\bibliographystyle{plain}   
\bibliography{../../biblios/main.bib}

\input{qm2pi.rhodetails}

\end{document}



% section front matter (end)

\section{Introduction}\label{sec:introduction} % (fold)
In this draft of the material i am going to have to dispense with the
usual writing conventions adopted in papers on these topics. i'm going
to have adopt whatever tone i need at the time i'm writing up the
calculations. Sometimes this may be very conversational; others it may
be the barest mathematical grunts; others still it may be that i have
lifted text from one of my other papers because the exposition of some
point was better said there. i hope that my readers are not unduly put
out by this decision. i'm not doing this to flout convention or be
rebellious. i find these calculations very technically challenging. To
keep everything going technically, something has to give; i have to
let go of some cognitive burden. So, the academic writing style --
with all of its trade-offs in terms of facilitating technical
communication -- is what i'm letting go of. Perhaps subsequent drafts
can be tightened and polished, but for now, i'm going to speak as if
we were sitting together in a coffee shop with a laptop, wifi and a
pad of paper and a pencil.

So, here's what i have to say. We -- you and i, comfortably ensconced
in our coffee shop and well-equipped with our tools -- can realize and
carry out the calculations of quantum mechanics over a very different
formal theory of dynamics, a formal theory of dynamics that
corresponds to a theory of concurrent computation with
\emph{reflection}. It has the advantage that the underlying theory is
already `quantized', but supports analogues all of the continuuous
operations. Strikingly, this underlying theory has recently been
connected with a notion of metric that we can show, by calculating
together, coincides with the metric induced by the inner product.

There are a lot of reasons why you might be interested in seeing
calculations of this form. Here's why i'm interested. For the past
several centuries there has been no competitor to the ``Newtonian''
account of dynamics. As a result the predominant share of accounts of
dynamical systems and situations have had to be formulated in terms of
the Newtonian machinery. i view this as an intellectually dangerous
position to occupy. Everything, despite it's intrinsic shape, turns
into a nail to be hit with this hammer. Recently, however, the theory
of computation has matured to the point where we have candidates for
theories of dynamics that offer very different perspective on
reasoning about dynamical systems and situations. Testing these
candidates against very successful accounts of dynamical situations,
like quantum mechanics, is going to give us some sense of how mature
they are and some measure of the quality of these accounts of
dynamics.

\subsection{Summary of contributions and outline of paper}

So, we're going to develop an interpretation of the operations of
quantum mechanics normally interpreted by Hilbert spaces and
operators. We're going to do this over a theory of computation. Note
that this is very different than the usual quantum computation program
which develops notions of computation over quantum mechanics. Rather,
we are developing a story that aligns with Wheeler's slogan: It from
Bit. To do this we will first provide an account of the theory of
computation at play here. Then we will dive into a calculation-driven
interpretation of the operations of quantum mechanics.

The reason we take this approach is that -- until very recently --
there hasn't been an axiomatic account of quantum mechanics. As a
result there has been no sharp delineation of the mathematical theory
supporting interpretation of the physical theory and the physical
theory, itself. So, ambient features of the maths are free to be
exploited (or supressed) without a real accounting of their physical
relevance. There is no sharp statement ``here's the physical theory''
qua \emph{theory} and ``here's the mathematical interpretation''
enabling a judgment of how faithful the interpretation is -- apart
from experimental observation. When there is an axiomatic account we
can judge how well a given mathematical formalism supports an
interpretation of the axioms, independent of
experimentation. Likewise, we can judge how well we have captured our
physical evidence and experience with our axiomatics, independent of
any specific mathematical implementation, with accidental detail that
may or may not have physical significance. 

In lieu of a fully fleshed out and vetted axiomatic account of quantum
mechanics, interpreting the operational notions in service of modeling
physical systems will have to suffice. In other words, we are not in
the business of providing a model of Hilbert spaces and operators. We
are in the business of providing a model of quantum mechanics because
we are motivated by testing our notions of dynamics against physical
theory; and, the predictive calculations of the physical theory must
serve as the best formulation -- shy of a fully fleshed out axiomatic
account -- of the physical theory itself (as they have for scientific
theories since time immemorial). Put another way, despite a
whole-hearted commitment to an It-from-Bit ontology, we are firmly
aligned with the shut-up-and-calculate camp as the best way to obtain
results either from the physical perspective or as a quality assurance
measure of our fledgling theory of dynamics.

In detail, we present a reflective process calculus. Then we develop
intuitive correspondences between the notions available in this
calculus and the usual physical notions supporting quantum mechanical
calculations. Thus, 

\begin{table}[htp]
  \center{
    \fbox{
      \begin{tabular}{c|c}
        quantum mechanics & process calculus \\
        \hline
        scalar & name \\
        state vector & process \\
        dual & contextual duals \\
        matrix & formal sums of process-context-dual pairs \\
        orthogonality & process annihilation \\
        inner product & execution-formula + quoting
      \end{tabular}
    }
  }
  \caption{QM - process calculi correspondences}
\end{table}

Then we tighten up these intuitions to operational definitions. We
employ the Dirac notation as the best proxy we can find for an
abstract syntax of the quantum mechanical notions. The definitions we
develop put us in contact with equational constraints coming from the
theory that we demonstrate the definitions and calculations satisfy.

This puts us in a position to shut up and calculate for the
Stern-Gerlach experimental set up, showing how these predictive
calculations become calculations on processes in our theory of a
reflective process calculus.

Penultimately, we demonstrate that the notion of metric coming from
the inner product coincides with the notion of metric available from
the theory of bisimulation. This demonstration gives us the right to
think of space as arising from behavior. Finally, we consider where we
might go from the new vantage point we have obtained.

% section introduction (end) 
 
% section introduction (end)

% \documentclass[12pt]{llncs}
%\documentclass{jktr}

\usepackage[pdftex]{hyperref}                   
\usepackage {listings}
\usepackage {mathpartir}
\usepackage{bcprules}
%\usepackage{listings}
                       
\usepackage{graphicx} 
%\usepackage[margins=2.5cm,nohead,nofoot]{geometry}
%\usepackage{geometry}
\usepackage{amsfonts}
\usepackage{amstext}
\usepackage{latexsym}
\usepackage{amssymb}
\usepackage{color}


%\include{myPreamble}
\include{qm2pi.local} 

%\ifpdf
%\usepackage[pdftex]{graphicx}
%\else
%\usepackage{graphicx}
%\fi

 % \ifpdf
%  \usepackage{pdfsync}
%  \if


%\title{Brief Article}
%\author{David F. Snyder}
%\author{L.G. Meredith}

%\address{Dept. of Math., Texas State University--San Marcos, San Marcos, TX 78666}
       
\pagestyle{empty}


\begin{document}

\lstset{language=[Objective]Caml,frame=shadowbox}

\input{qm2pi.front}

% section front matter (end)

\input{qm2pi.intro} 
 
% section introduction (end)

% \input{qm2pi.knotations} 

% section notation (end)

\input{qm2pi.process.calculi} 

% section concurrent_process_calculi_and_spatial_logics_ (end)
    
%\input{qm2pi.knots2pi} 

%\input{qm2pi.trefoil} 

%\input{qm2pi.mainthm} 

% subsection basic_interpretation (end)

%\input{qm2pi.rho.presentation} 
\subsection{The syntax and semantics of the notation system}\label{sub:the_syntax_and_semantics_of_the_notation_system} % (fold)

We now summarize a technical presentation of the calculus that
embodies our theory of dynamics. The typical presentation of such a
calculus follows the style of giving generators and relations on
them. The grammar, below, describing term constructors, freely
generates the set of processes, $\Proc$. This set is then quotiented
by a relation known as structural congruence and it is over this set
that the notion of dynamics is expressed. This presentation is
essentially that of \cite{MeredithR05} with the addition of
polyadicity and summation. For readability we have relegated some of
the technical subtleties to an appendix.

\subsubsection{Process grammar}\label{subsub:process_grammar}

\begin{mathpar}
  \inferrule* [lab=synchronization] {} {{M} \bc \pzero \;|\; x?F \;|\; x!C }
  \and
  \inferrule* [lab=abstraction] {} {{F} \bc (x)P}
  \and
  \inferrule* [lab=concretion] {} {{C} \bc \langle Q \rangle}
  \and
  \inferrule* [lab=process] {} {{P,Q} \bc M \;| \;P|Q \;|\; @{x}}
  \and
  \inferrule* [lab=name] {} {{x} \bc \quotep{P}}
\end{mathpar} 

Note that $\vec{x}$ (resp. $\vec{P}$) denotes a vector of names
(resp. processes) of length $|\vec{x}|$ (resp. $|\vec{P}|$). We adopt
the following useful abbreviations.

\begin{mathpar}
   x?(\vec{y}).P := x.(\vec{y})P \and  x\clift{\vec{P}} := x.\clift{\vec{P}}
   \and x!(y) := \lift{x}{\dropn{y}}
   \and \Pi_{i=0}^{n-1}P_i := P_0 | \ldots | P_{n-1}
\end{mathpar}

\subsubsection{Structural congruence}

\paragraph{Free and bound names and alpha-equivalence.} At the
core of structural equivalence is alpha-equivalence which identifies
process that are the same up to a change of variable. Formally, we
recognize the distinction between free and bound names. The free names
of a process, $\freenames{P}$, may be calculated recursively as
follows:

\begin{mathpar}
\freenames{\pzero} := \emptyset
  \and \\
  \freenames{x?(y).P} := \{ x \} \cup (\freenames{P} \setminus \{ y \})
  \and 
  \freenames{x!\langle P \rangle} := \{ x \} \cup \{ P \} 
  \and \\
  \freenames{P|Q} := \freenames{P} \cup \freenames{Q}
  \and \\
  \freenames{@{x}} := \{ x \}
\end{mathpar}

$\pi$
$\quotep{\pi}$

$\freenames{-} : \pi \to \mathcal{P}(\quotep{\pi})$

\begin{eqnarray*}
  \freenames{\pzero} & := & \emptyset \\
  \freenames{x?(y).P} & := & \{ x \} \cup (\freenames{P} \setminus \{ y \}) \\
  \freenames{x!\langle P \rangle} & := & \{ x \} \cup \{ P \} \\
  \freenames{P|Q} & := & \freenames{P} \cup \freenames{Q} \\
  \freenames{\dropn{x}} & := & \{ x \}
\end{eqnarray*}

The bound names of a process, $\boundnames{P}$, are those names occurring in $P$
that are not free. For example, in $x?(y).0$, the name $x$ is free, while $y$ is bound.

\begin{mathpar}
  \inferrule* [lab=monoidal-laws] {} { P|Q \equiv Q|P \and P|0 \equiv P \and P|(Q|R) \equiv (P|Q)|R }
\end{mathpar}

\begin{mathpar}
  \inferrule* [lab=alpha-equivalence] {} { (x)P \equiv (y)P\{y/x\} \and y \not\in \freenames{P} }
\end{mathpar}

\begin{definition}
Then two processes, $P,Q$, are alpha-equivalent if $P = Q\{\vec{y}/\vec{x}\}$ for
some $\vec{x} \in \boundnames{Q},\vec{y} \in \boundnames{P}$, where $Q\{\vec{y}/\vec{x}\}$
denotes the capture-avoiding substitution of $\vec{y}$ for $\vec{x}$ in $Q$.
\end{definition}

\begin{definition}
  The {\em structural congruence} \cite{SangiorgiWalker} , $\equiv$,
  between processes is the least congruence containing
  alpha-equivalence, satisfying the abelian monoid laws
  (associativity, commutativity and $\pzero$ as identity) for parallel
  composition $|$ and for summation $+$.
\end{definition}

\subsection{Name equivalence}

We take name equivalence, written $\nameeq$, to be the smallest
equivalence relation generated by the following rules.

\begin{mathpar}
\inferrule*[lab=Quote-drop]
{ }
{ \quotep{@{x}} \nameeq x }

\inferrule*[lab=Struct-equiv]
{ P \scong Q }
{ \quotep{P} \nameeq \quotep{Q} }
\end{mathpar}

The astute reader will have noticed that the mutual recursion of names
and processes imposes a mutual recursion on alpha-equivalence and
structural equivalence via name-equivalence. Fortunately, all of this
works out pleasantly and we may calculate in the natural way, free of
concern. The reader interested in the details is referred to the
appendix \ref{appendix:rho_details}.

\subsection{Substitution}

We use $\Proc$ for the set of processes, $\QProc$ for the set of
names, and $\id{\{}\vec{y} / \vec{x} \id{\}}$ to denote partial maps,
$s : \QProc \rightarrow \QProc$. A map, $s$ lifts, uniquely, to a map
on process terms, $\widehat{s} : \Proc \rightarrow \Proc$ by the
following equations.

\begin{mathpar}
  (0) \psubstp{Q}{P} := 0 \\
  (R \juxtap S) \psubstp{Q}{P}
  :=    
  (R)\psubstp{Q}{P} \juxtap (S) \psubstp{Q}{P} \\
  (x?(y).R) \psubstp{Q}{P}    
  :=    
  (x)\substp{Q}{P} (z)\concat( (R \psubstn{z}{y}) \psubstp{Q}{P} ) \\
  (\lift{x}{R}) \psubstp{Q}{P}  
  :=
  \lift{(x)\substp{Q}{P}}{ R \psubstp{Q}{P} } \\
%   (\dropn{x})  \psubstp{Q}{P}       
%   := 
%   \left\{ 
%     \begin{array}{ccc} 
%       \dropn{\quotep{Q}} & & x \nameeq \quotep{P} \\
%       \dropn{x} & & otherwise \\
%     \end{array}
%   \right. 
  (\dropn{x})  \psubstp{Q}{P}       
  := 
  \left\{ 
    \begin{array}{ccc} 
      Q & & x \nameeq \quotep{P} \\
      \dropn{x} & & otherwise \\
    \end{array}
  \right.
\end{mathpar}
 

where

\begin{eqnarray}
  (x)\id{\{} \lpquote Q \rpquote / \lpquote P \rpquote \id{\}}            = 
  \left\{ 
    \begin{array}{ccc}
      \lpquote Q \rpquote & & x \nameeq \lpquote P \rpquote \\
      x & & otherwise \\
    \end{array}
  \right. \nonumber
\end{eqnarray}

and $z$ is chosen distinct from $\quotep{P}$, $\quotep{Q}$, the free
names in $Q$, and all the names in $R$. Our $\alpha$-equivalence will
be built in the standard way from this substitution.

\begin{remark}\label{rem:no_self_referential_names}
  One consequence of these definitions is that $\forall P. \quotep{P}
  \not\in \freenames{P}$.
\end{remark}

\subsection{ Dynamic quote: an example }

Anticipating something of what's to come, consider applying the
substitution, $\widehat{\id{\{}u / z \id{\}}}$, to the following pair
of processes, $\lift{w}{y!(z)}$ and $w[ \lpquote y!(z) \rpquote ]$.

\begin{eqnarray}
	\lift{w}{y!(z)}\widehat{\id{\{}u / z \id{\}}}
		& = &
		\lift{w}{y!(u)} \nonumber\\
	w[ \lpquote y!(z) \rpquote ] \widehat{ \id{\{}u / z \id{\}} }
		& = &
		w[ \lpquote y!(z) \rpquote ] \nonumber
\end{eqnarray}

Because the body of the process between quotes is impervious to
substitution, we get radically different answers. In fact, by
examining the first process in an input context,
e.g. $x?(z).\lift{w}{y!(z)}$, we see that the process under the lift
operator may be shaped by prefixed inputs binding a name inside it. In
this sense, the lift operator will be seen as a way to dynamically
construct processes before reifying them as names.

Finally equipped with these standard features we can present the
dynamics of the calculus.

\subsubsection{Operational semantics} 

Finally, we introduce the computational dynamics. What marks these
algebras as distinct from other more traditionally studied algebraic
structures, e.g. vector spaces or polynomial rings, is the manner in
which dynamics is captured. In traditional structures, dynamics is typically
expressed through morphisms between such structures, as in linear maps
between vector spaces or morphisms between rings. In algebras
associated with the semantics of computation, the dynamics is
expressed as part of the algebraic structure itself, through a
reduction reduction relation typically denoted by $\red$. Below, we
give a recursive presentation of this relation for the calculus used
in the encoding.

$\red \subseteq \pi \times \pi$
$\red : \pi \to \mathcal{P}(\pi)$

\begin{mathpar}
  \inferrule* [lab=Comm] { \textsf{match}( x_{src}, x_{trgt} ) } { x_{trgt}?(y)P \; | \; x_{src}!\langle {Q} \rangle \red P\{\quotep{Q}/y}\} }
  \and \\
  \inferrule* [lab=Par] {{P} \red {P}'} {{{P} | {Q}} \red {{P}' | {Q}}}
  \and
  \inferrule* [lab=Equiv]{{{P} \scong {P}'} \andalso {{P}' \red {Q}'} \andalso {{Q}' \scong {Q}}}{{P} \red {Q}}
\end{mathpar}

\begin{eqnarray*}
  match_{\equiv} (\quotep{P},\quotep{Q}) & := & P \equiv Q \\
  match_{\dagger}(\quotep{P},\quotep{Q}) & := & \forall R. P|Q \red^{*} R => R \red^{*} 0 \\
  match_{K}(\quotep{P},\quotep{Q}) & := & K \mbox{ for some context } K
\end{eqnarray*}

$u?(x)P | u!\langle Q \rangle \red P\{\quotep{Q}/x\}$

%We write $\wred$ for $\red^*$, and $P\red$ if $\exists Q $ such that $ P \red Q$.
We write $P\red$ if $\exists Q $ such that $ P \red Q$ and $P\not\red$, otherwise.

\section{Replication}

As mentioned before, it is known that replication (and hence
recursion) can be implemented in a higher-order process algebra
\cite{SangiorgiWalker}. As our first example of calculation with the
machinery thus far presented we give the construction explicitly in
the {\rhoc}.

\begin{eqnarray}
	D_{x} & := & \prefix{x}{y}{(\binpar{\outputp{x}{y}}{@{y}})} \nonumber\\
	\bangp_{x}{P} & := & \binpar{{x}!\langle{\binpar{D_{x}}{P}}\rangle}{D_{x}} \nonumber
\end{eqnarray}

\begin{eqnarray}
	\bangp_{x}{P} & & \nonumber\\
	=
	& {x}!\langle{(\prefix{x}{y}{(\outputp{x}{y} | @{y})) | P}}\rangle 
	      | \prefix{x}{y}{(\outputp{x}{y} | @{y})} & \nonumber\\
	\red
	& (\outputp{x}{y} | @{y})\substn{\quotep{(\prefix{x}{y}{(@{y} | \outputp{x}{y})) | P}}}{y} & \nonumber\\
	=
	& \outputp{x}{\quotep{(\prefix{x}{y}{(\outputp{x}{y} | @{y})) | P}}}
	  | {(\prefix{x}{y}{(\outputp{x}{y} | @{y})) | P}} & \nonumber\\
	\red
	& \ldots & \nonumber\\
	\red^*
	& P | P | \ldots & \nonumber
\end{eqnarray}

Of course, this encoding, as an implementation, runs away, unfolding
$\bangp{P}$ eagerly. A lazier and more implementable replication
operator, restricted to input-guarded processes, may be obtained as follows.

\begin{eqnarray}
\bangp{\prefix{u}{v}{P}} 
	:= 
	\binpar{\lift{x}{\prefix{u}{v}{(\binpar{D(x)}{P})}}}{D(x)} \nonumber
\end{eqnarray}

\begin{remark}
  Note that the lazier definition still does not deal with summation
  or mixed summation (i.e. sums over input and output). The reader is
  invited to construct definitions of replication that deal with these
  features. 

  Further, the definitions are parameterized in a name, $x$. Can you,
  gentle reader, make a definition that eliminates this parameter and
  guarantees no accidental interaction between the replication
  machinery and the process being replicated -- i.e. no accidental
  sharing of names used by the process to get its work done and the
  name(s) used by the replication to effect copying. This latter
  revision of the definition of replication is crucial to obtaining
  the expected identity $!!P \sim !P$.
\end{remark}

\begin{remark}\label{rem:paradoxical_combinator}
  The reader familiar with the lambda calculus will have noticed the
  similarity between $D$ and the paradoxical combinator.

  [Ed. note: the existence of this seems to suggest we have to be more
  restrictive on the set of processes and names we admit if we are to
  support no-cloning.]
\end{remark}

\subsubsection{Bisimulation}

The computational dynamics gives rise to another kind of equivalence,
the equivalence of computational behavior. As previously mentioned
this is typically captured \emph{via} some form of bisimulation.

% The notion we use in this paper is weak barbed bisimulation
% \cite{milner91polyadicpi}.

The notion we use in this paper is derived from weak barbed
bisimulation \cite{milner91polyadicpi}. 

\begin{definition}
An \emph{observation relation}, $\downarrow_{\mathcal N}$, over a set
of names, $\mathcal N$, is the smallest relation satisfying the rules
below.

\infrule[Out-barb]{y \in {\mathcal N}, \; x \nameeq y}
		  {\outputp{x}{v} \downarrow_{\mathcal N} x}
\infrule[Par-barb]{\mbox{$P\downarrow_{\mathcal N} x$ or $Q\downarrow_{\mathcal N} x$}}
		  {\binpar{P}{Q} \downarrow_{\mathcal N} x}

We write $P \Downarrow_{\mathcal N} x$ if there is $Q$ such that 
$P \wred Q$ and $Q \downarrow_{\mathcal N} x$.
\end{definition}

\begin{definition}
%\label{def.bbisim}
An  ${\mathcal N}$-\emph{barbed bisimulation} over a set of names, ${\mathcal N}$, is a symmetric binary relation 
${\mathcal S}_{\mathcal N}$ between agents such that $P\rel{S}_{\mathcal N}Q$ implies:
\begin{enumerate}
\item If $P \red P'$ then $Q \wred Q'$ and $P'\rel{S}_{\mathcal N} Q'$.
\item If $P\downarrow_{\mathcal N} x$, then $Q\Downarrow_{\mathcal N} x$.
\end{enumerate}
$P$ is ${\mathcal N}$-barbed bisimilar to $Q$, written
$P \wbbisim_{\mathcal N} Q$, if $P \rel{S}_{\mathcal N} Q$ for some ${\mathcal N}$-barbed bisimulation ${\mathcal S}_{\mathcal N}$.
\end{definition}

$\mathcal{R} \subseteq \pi \times \pi$

$P \mathcal{R} Q => \forall P'. P \red P' \Rightarrow \exists Q'. Q \red Q', P' \mathcal{R} Q'$

$P \vdash x \Rightarrow Q \vdash x$

\begin{mathpar}
  \inferrule*[lab=Out-barb]{x \nameeq y}{{y}!\langle{Q}\rangle \vdash x}
  \and
  \inferrule*[lab=Par-barb]{\mbox{$P\vdash x$ or $Q\vdash x$}}{\binpar{P}{Q} \vdash x}
\end{mathpar}

\subsubsection{Contexts}

One of the principle advantages of computational calculi like the
$\pi$-calculus is a well-defined notion of context,
contextual-equivalence and a correlation between
contextual-equivalence and notions of bisimulation. The notion of
context allows the decomposition of a process into (sub-)process and
its syntactic environment, its context. Thus, a context may be
thought of as a process with a ``hole'' (written $\Box$) in it. The
application of a context $M$ to a process $P$, written $M[P]$, is
tantamount to filling the hole in $M$ with $P$. In this paper we do
not need the full weight of this theory, but do make use of the notion
of context in the proof the main theorem. 

\begin{mathpar}
  \inferrule* [lab=summation] {} {{M_{M},M_{N}} \bc \Box \;|\; x.M_{A} \;|\; M_{M}+M_{N}}
  \and
  \inferrule* [lab=agent] {} {{M_{A}} \bc (\vec{x})M_{P} \;| \; \clift{P_0,\ldots,M_{P},\ldots,P_N}}
  \and \\
  \inferrule* [lab=process] {} {{M_{P}} \bc M_{N} \;| \;P|M_{P} }
\end{mathpar} 

\begin{mathpar}
  \inferrule* [lab=sychronization] {} {M_{N} \bc \Box \;|\; x?M_{F} \;|\; x!M_{C}}
  \and
  \inferrule* [lab=abstraction] {} {{M_{F}} \bc (x)M_{P} }
  \and
  \inferrule* [lab=concretion] {} {{M_{C}} \bc \langle M_{P} \rangle }
  \and \\
  \inferrule* [lab=process] {} {{M_{P}} \bc M_{N} \;| \;P|M_{P} }
\end{mathpar}

\begin{definition}[contextual application] Given a context $M$, and
  process $P$, we define the \emph{contextual application}, $M[P] :=
  M\{P/\Box\}$. That is, the contextual application of M to P is the
  substitution of $P$ for $\Box$ in $M$.
\end{definition}

$\meaningof{-} : L \to \mathcal{P}(\pi)$

\begin{mathpar}
  \inferrule* [lab=collection] {} {\meaningof{true} = \pi, \and \meaningof{~E} = \pi \setminus \meaningof{E}, \and \meaningof{E_{1} \& E_{2}} = \meaningof{E_{1}} \cap \meaningof{E_{2}}}
\end{mathpar}

\begin{mathpar}
  \inferrule* [lab=structure] {} {\meaningof{0} = \{ P \in \pi | P \equiv 0 \}, \and \\ \meaningof{E_1 | E_2} = \{ P \in \pi | P \equiv P_{1} | P_{2}, P_{1} \in \meaningof{E_{1}}, P_{2} \in \meaningof{E_2}\} }
\end{mathpar}

\begin{mathpar}
 \inferrule* [lab=behavior] {} {\meaningof{\langle a?b \rangle E} = \{ P \in \pi | P \equiv Q | u?(y)P', \\ \and \\\\ \and \\ \;\;\; u \in \meaningof{a}, \forall z.P'\{z/y\} \in \meaningof{E\{z/b\}}\}, \and \\ \meaningof{a!E} = \{ P \in \pi | P \equiv Q | x!\langle P' \rangle, x \in \meaningof{a} P' \in \meaningof{E}\} }
\end{mathpar}

\begin{mathpar}
 \inferrule* [lab=nominal] {} {\meaningof{\quotep{E}} = \{ \quotep{P} \in \quotep{\pi} | P \in \meaningof{E} \}, \and \meaningof{\quotep{P}} = \{ \quotep{Q} \in \quotep{\pi} | P \equiv Q \} \and \\ \meaningof{@\quotep{E}} = \{ P \in \pi | P \equiv @x, x \in \meaningof{E} \}}
\end{mathpar}

\begin{eqnarray*}
  \\
  \meaningof{-} : TS \to ST
\end{eqnarray*}

\begin{eqnarray*}
  \\
  L : TS \to ST
\end{eqnarray*}

\begin{eqnarray*}
  \\
  P \models E \iff P \in \meaningof{E}
\end{eqnarray*}

\begin{eqnarray*}
  P \approx_{L} Q \iff \forall E \in L. P \models E \iff Q \models E
\end{eqnarray*}

\begin{eqnarray*}
  P \approx_{K} Q
\end{eqnarray*}

\begin{eqnarray*}
  P \approx Q
\end{eqnarray*}

$\approx_{K} = \approx = \approx_{L}$

\subsubsection{Contextual duality}

Note that contexts extend the quotation operation to a family of
operations from processes to names. Given a context, $M$, we can
define a \emph{nominal context}, $\quotep{M}$ by $\quotep{M}[P] :=
\quotep{M[P]}$. To foreshadow what is to come we observe that these
operations enjoy a duality with processes very much like the duality
between vectors and maps from vectors to scalars.

Further, because the calculus is essentially higher-order, we have a
correspondence between contexts and processes. More specifically,
given a name $x$ and a context $M$ we can construct $M^{*}_{x}$ such
that 

\begin{mathpar}
  M^{*}_{x} | \lift{x}{P} \red M[P]
\end{mathpar}

namely,

\begin{mathpar}
  M^{*}_{x} := x?(u).M[\dropn{u}]
\end{mathpar}

The dependence of $M^{*}_{x}$ on a name makes it an abstraction, 

\begin{mathpar}
  M^{*} := (x)x?(u).M[\dropn{u}]
\end{mathpar}

\subsection{Additional notation}

It will sometimes be convenient to denote the process a name
quotes. We already have the notation $x = \quotep{P}$, but it will be
convenient to introduce an alternate notation, $\procn{x}$, when we
want to emphasize the connection to the use of the name. Note that, by
virtue of name equivalence, $\quotep{\procn{x}} \nameeq x$; so, the
notation is consistent with previous definitions.

Further, because names have structure it is possible to effect
substitutions on the basis of that structure. This means we need to
upgrade our notation for substitutions, which we accomplish by
adapting comprehension notation. Thus,

\begin{mathpar}
  P\{ y / x : x \in S \}
\end{mathpar}

is interpreted to mean the process derived from P by replacing (in a
capture-avoiding manner) each occurrence of $x$ in $S$ by $y$. For example,

\begin{mathpar}
  P\{ \quotep{\procn{x}|\procn{x}} / x : x \in \freenames{P} \}
\end{mathpar}

will replace each (occurrence) of a free name $x$ in $P$ by
$\quotep{\procn{x}|\procn{x}}$.

Also, we will avail ourselves of the notation $x^{L}$ and $x^{R}$ to
denote injections of a name into disjoint copies of the name
space. There are numerous ways to accomplish this. One example can be
found in \cite{MeredithR05}. This notation overloads to vectors of
names: $\vec{x}^{\pi} := (x_{i}^{\pi} \; : \; 0 \leq i < |\vec{x}| )$ where $\pi \in \{L,R\}$.

We also use $P^{\Box} := P|\Box$.

In \cite{MeredithR05} an interpretation of the new operator is
given. It turns out that there are several possible interpretations
all enjoying the requisite algebraic properties of the operator (see
\cite{milner91polyadicpi}). We will therefore make liberal use of
$(\nu\; \vec{x})P$.

% subsection the_syntax_and_semantics_of_the_notation_system (end)   

\input{qm2pi.qmops} 

\input{qm2pi.sterngerlach} 

\input{qm2pi.metric} 

% section concurrent_process_calculi (end)

%\input{qm2pi.proofsketch}

% section proof sketch (end)

%\input{qm2pi.slviaknots} 

% section spatial logic via knots (end)

\input{qm2pi.conclusion}

% section conclusion (end)

%\input{qm2pi.dtcodes} 

% section wiring algorithm (end)

\input{qm2pi.ack} 

% section acknowledgments (end)

\newpage


\bibliographystyle{plain}   
\bibliography{../../biblios/main.bib}

\input{qm2pi.rhodetails}

\end{document}

 

% section notation (end)

\input{qm2pi.process.calculi} 

% section concurrent_process_calculi_and_spatial_logics_ (end)
    
%\documentclass[12pt]{llncs}
%\documentclass{jktr}

\usepackage[pdftex]{hyperref}                   
\usepackage {listings}
\usepackage {mathpartir}
\usepackage{bcprules}
%\usepackage{listings}
                       
\usepackage{graphicx} 
%\usepackage[margins=2.5cm,nohead,nofoot]{geometry}
%\usepackage{geometry}
\usepackage{amsfonts}
\usepackage{amstext}
\usepackage{latexsym}
\usepackage{amssymb}
\usepackage{color}


%\include{myPreamble}
\include{qm2pi.local} 

%\ifpdf
%\usepackage[pdftex]{graphicx}
%\else
%\usepackage{graphicx}
%\fi

 % \ifpdf
%  \usepackage{pdfsync}
%  \if


%\title{Brief Article}
%\author{David F. Snyder}
%\author{L.G. Meredith}

%\address{Dept. of Math., Texas State University--San Marcos, San Marcos, TX 78666}
       
\pagestyle{empty}


\begin{document}

\lstset{language=[Objective]Caml,frame=shadowbox}

\input{qm2pi.front}

% section front matter (end)

\input{qm2pi.intro} 
 
% section introduction (end)

% \input{qm2pi.knotations} 

% section notation (end)

\input{qm2pi.process.calculi} 

% section concurrent_process_calculi_and_spatial_logics_ (end)
    
%\input{qm2pi.knots2pi} 

%\input{qm2pi.trefoil} 

%\input{qm2pi.mainthm} 

% subsection basic_interpretation (end)

%\input{qm2pi.rho.presentation} 
\subsection{The syntax and semantics of the notation system}\label{sub:the_syntax_and_semantics_of_the_notation_system} % (fold)

We now summarize a technical presentation of the calculus that
embodies our theory of dynamics. The typical presentation of such a
calculus follows the style of giving generators and relations on
them. The grammar, below, describing term constructors, freely
generates the set of processes, $\Proc$. This set is then quotiented
by a relation known as structural congruence and it is over this set
that the notion of dynamics is expressed. This presentation is
essentially that of \cite{MeredithR05} with the addition of
polyadicity and summation. For readability we have relegated some of
the technical subtleties to an appendix.

\subsubsection{Process grammar}\label{subsub:process_grammar}

\begin{mathpar}
  \inferrule* [lab=synchronization] {} {{M} \bc \pzero \;|\; x?F \;|\; x!C }
  \and
  \inferrule* [lab=abstraction] {} {{F} \bc (x)P}
  \and
  \inferrule* [lab=concretion] {} {{C} \bc \langle Q \rangle}
  \and
  \inferrule* [lab=process] {} {{P,Q} \bc M \;| \;P|Q \;|\; @{x}}
  \and
  \inferrule* [lab=name] {} {{x} \bc \quotep{P}}
\end{mathpar} 

Note that $\vec{x}$ (resp. $\vec{P}$) denotes a vector of names
(resp. processes) of length $|\vec{x}|$ (resp. $|\vec{P}|$). We adopt
the following useful abbreviations.

\begin{mathpar}
   x?(\vec{y}).P := x.(\vec{y})P \and  x\clift{\vec{P}} := x.\clift{\vec{P}}
   \and x!(y) := \lift{x}{\dropn{y}}
   \and \Pi_{i=0}^{n-1}P_i := P_0 | \ldots | P_{n-1}
\end{mathpar}

\subsubsection{Structural congruence}

\paragraph{Free and bound names and alpha-equivalence.} At the
core of structural equivalence is alpha-equivalence which identifies
process that are the same up to a change of variable. Formally, we
recognize the distinction between free and bound names. The free names
of a process, $\freenames{P}$, may be calculated recursively as
follows:

\begin{mathpar}
\freenames{\pzero} := \emptyset
  \and \\
  \freenames{x?(y).P} := \{ x \} \cup (\freenames{P} \setminus \{ y \})
  \and 
  \freenames{x!\langle P \rangle} := \{ x \} \cup \{ P \} 
  \and \\
  \freenames{P|Q} := \freenames{P} \cup \freenames{Q}
  \and \\
  \freenames{@{x}} := \{ x \}
\end{mathpar}

$\pi$
$\quotep{\pi}$

$\freenames{-} : \pi \to \mathcal{P}(\quotep{\pi})$

\begin{eqnarray*}
  \freenames{\pzero} & := & \emptyset \\
  \freenames{x?(y).P} & := & \{ x \} \cup (\freenames{P} \setminus \{ y \}) \\
  \freenames{x!\langle P \rangle} & := & \{ x \} \cup \{ P \} \\
  \freenames{P|Q} & := & \freenames{P} \cup \freenames{Q} \\
  \freenames{\dropn{x}} & := & \{ x \}
\end{eqnarray*}

The bound names of a process, $\boundnames{P}$, are those names occurring in $P$
that are not free. For example, in $x?(y).0$, the name $x$ is free, while $y$ is bound.

\begin{mathpar}
  \inferrule* [lab=monoidal-laws] {} { P|Q \equiv Q|P \and P|0 \equiv P \and P|(Q|R) \equiv (P|Q)|R }
\end{mathpar}

\begin{mathpar}
  \inferrule* [lab=alpha-equivalence] {} { (x)P \equiv (y)P\{y/x\} \and y \not\in \freenames{P} }
\end{mathpar}

\begin{definition}
Then two processes, $P,Q$, are alpha-equivalent if $P = Q\{\vec{y}/\vec{x}\}$ for
some $\vec{x} \in \boundnames{Q},\vec{y} \in \boundnames{P}$, where $Q\{\vec{y}/\vec{x}\}$
denotes the capture-avoiding substitution of $\vec{y}$ for $\vec{x}$ in $Q$.
\end{definition}

\begin{definition}
  The {\em structural congruence} \cite{SangiorgiWalker} , $\equiv$,
  between processes is the least congruence containing
  alpha-equivalence, satisfying the abelian monoid laws
  (associativity, commutativity and $\pzero$ as identity) for parallel
  composition $|$ and for summation $+$.
\end{definition}

\subsection{Name equivalence}

We take name equivalence, written $\nameeq$, to be the smallest
equivalence relation generated by the following rules.

\begin{mathpar}
\inferrule*[lab=Quote-drop]
{ }
{ \quotep{@{x}} \nameeq x }

\inferrule*[lab=Struct-equiv]
{ P \scong Q }
{ \quotep{P} \nameeq \quotep{Q} }
\end{mathpar}

The astute reader will have noticed that the mutual recursion of names
and processes imposes a mutual recursion on alpha-equivalence and
structural equivalence via name-equivalence. Fortunately, all of this
works out pleasantly and we may calculate in the natural way, free of
concern. The reader interested in the details is referred to the
appendix \ref{appendix:rho_details}.

\subsection{Substitution}

We use $\Proc$ for the set of processes, $\QProc$ for the set of
names, and $\id{\{}\vec{y} / \vec{x} \id{\}}$ to denote partial maps,
$s : \QProc \rightarrow \QProc$. A map, $s$ lifts, uniquely, to a map
on process terms, $\widehat{s} : \Proc \rightarrow \Proc$ by the
following equations.

\begin{mathpar}
  (0) \psubstp{Q}{P} := 0 \\
  (R \juxtap S) \psubstp{Q}{P}
  :=    
  (R)\psubstp{Q}{P} \juxtap (S) \psubstp{Q}{P} \\
  (x?(y).R) \psubstp{Q}{P}    
  :=    
  (x)\substp{Q}{P} (z)\concat( (R \psubstn{z}{y}) \psubstp{Q}{P} ) \\
  (\lift{x}{R}) \psubstp{Q}{P}  
  :=
  \lift{(x)\substp{Q}{P}}{ R \psubstp{Q}{P} } \\
%   (\dropn{x})  \psubstp{Q}{P}       
%   := 
%   \left\{ 
%     \begin{array}{ccc} 
%       \dropn{\quotep{Q}} & & x \nameeq \quotep{P} \\
%       \dropn{x} & & otherwise \\
%     \end{array}
%   \right. 
  (\dropn{x})  \psubstp{Q}{P}       
  := 
  \left\{ 
    \begin{array}{ccc} 
      Q & & x \nameeq \quotep{P} \\
      \dropn{x} & & otherwise \\
    \end{array}
  \right.
\end{mathpar}
 

where

\begin{eqnarray}
  (x)\id{\{} \lpquote Q \rpquote / \lpquote P \rpquote \id{\}}            = 
  \left\{ 
    \begin{array}{ccc}
      \lpquote Q \rpquote & & x \nameeq \lpquote P \rpquote \\
      x & & otherwise \\
    \end{array}
  \right. \nonumber
\end{eqnarray}

and $z$ is chosen distinct from $\quotep{P}$, $\quotep{Q}$, the free
names in $Q$, and all the names in $R$. Our $\alpha$-equivalence will
be built in the standard way from this substitution.

\begin{remark}\label{rem:no_self_referential_names}
  One consequence of these definitions is that $\forall P. \quotep{P}
  \not\in \freenames{P}$.
\end{remark}

\subsection{ Dynamic quote: an example }

Anticipating something of what's to come, consider applying the
substitution, $\widehat{\id{\{}u / z \id{\}}}$, to the following pair
of processes, $\lift{w}{y!(z)}$ and $w[ \lpquote y!(z) \rpquote ]$.

\begin{eqnarray}
	\lift{w}{y!(z)}\widehat{\id{\{}u / z \id{\}}}
		& = &
		\lift{w}{y!(u)} \nonumber\\
	w[ \lpquote y!(z) \rpquote ] \widehat{ \id{\{}u / z \id{\}} }
		& = &
		w[ \lpquote y!(z) \rpquote ] \nonumber
\end{eqnarray}

Because the body of the process between quotes is impervious to
substitution, we get radically different answers. In fact, by
examining the first process in an input context,
e.g. $x?(z).\lift{w}{y!(z)}$, we see that the process under the lift
operator may be shaped by prefixed inputs binding a name inside it. In
this sense, the lift operator will be seen as a way to dynamically
construct processes before reifying them as names.

Finally equipped with these standard features we can present the
dynamics of the calculus.

\subsubsection{Operational semantics} 

Finally, we introduce the computational dynamics. What marks these
algebras as distinct from other more traditionally studied algebraic
structures, e.g. vector spaces or polynomial rings, is the manner in
which dynamics is captured. In traditional structures, dynamics is typically
expressed through morphisms between such structures, as in linear maps
between vector spaces or morphisms between rings. In algebras
associated with the semantics of computation, the dynamics is
expressed as part of the algebraic structure itself, through a
reduction reduction relation typically denoted by $\red$. Below, we
give a recursive presentation of this relation for the calculus used
in the encoding.

$\red \subseteq \pi \times \pi$
$\red : \pi \to \mathcal{P}(\pi)$

\begin{mathpar}
  \inferrule* [lab=Comm] { \textsf{match}( x_{src}, x_{trgt} ) } { x_{trgt}?(y)P \; | \; x_{src}!\langle {Q} \rangle \red P\{\quotep{Q}/y}\} }
  \and \\
  \inferrule* [lab=Par] {{P} \red {P}'} {{{P} | {Q}} \red {{P}' | {Q}}}
  \and
  \inferrule* [lab=Equiv]{{{P} \scong {P}'} \andalso {{P}' \red {Q}'} \andalso {{Q}' \scong {Q}}}{{P} \red {Q}}
\end{mathpar}

\begin{eqnarray*}
  match_{\equiv} (\quotep{P},\quotep{Q}) & := & P \equiv Q \\
  match_{\dagger}(\quotep{P},\quotep{Q}) & := & \forall R. P|Q \red^{*} R => R \red^{*} 0 \\
  match_{K}(\quotep{P},\quotep{Q}) & := & K \mbox{ for some context } K
\end{eqnarray*}

$u?(x)P | u!\langle Q \rangle \red P\{\quotep{Q}/x\}$

%We write $\wred$ for $\red^*$, and $P\red$ if $\exists Q $ such that $ P \red Q$.
We write $P\red$ if $\exists Q $ such that $ P \red Q$ and $P\not\red$, otherwise.

\section{Replication}

As mentioned before, it is known that replication (and hence
recursion) can be implemented in a higher-order process algebra
\cite{SangiorgiWalker}. As our first example of calculation with the
machinery thus far presented we give the construction explicitly in
the {\rhoc}.

\begin{eqnarray}
	D_{x} & := & \prefix{x}{y}{(\binpar{\outputp{x}{y}}{@{y}})} \nonumber\\
	\bangp_{x}{P} & := & \binpar{{x}!\langle{\binpar{D_{x}}{P}}\rangle}{D_{x}} \nonumber
\end{eqnarray}

\begin{eqnarray}
	\bangp_{x}{P} & & \nonumber\\
	=
	& {x}!\langle{(\prefix{x}{y}{(\outputp{x}{y} | @{y})) | P}}\rangle 
	      | \prefix{x}{y}{(\outputp{x}{y} | @{y})} & \nonumber\\
	\red
	& (\outputp{x}{y} | @{y})\substn{\quotep{(\prefix{x}{y}{(@{y} | \outputp{x}{y})) | P}}}{y} & \nonumber\\
	=
	& \outputp{x}{\quotep{(\prefix{x}{y}{(\outputp{x}{y} | @{y})) | P}}}
	  | {(\prefix{x}{y}{(\outputp{x}{y} | @{y})) | P}} & \nonumber\\
	\red
	& \ldots & \nonumber\\
	\red^*
	& P | P | \ldots & \nonumber
\end{eqnarray}

Of course, this encoding, as an implementation, runs away, unfolding
$\bangp{P}$ eagerly. A lazier and more implementable replication
operator, restricted to input-guarded processes, may be obtained as follows.

\begin{eqnarray}
\bangp{\prefix{u}{v}{P}} 
	:= 
	\binpar{\lift{x}{\prefix{u}{v}{(\binpar{D(x)}{P})}}}{D(x)} \nonumber
\end{eqnarray}

\begin{remark}
  Note that the lazier definition still does not deal with summation
  or mixed summation (i.e. sums over input and output). The reader is
  invited to construct definitions of replication that deal with these
  features. 

  Further, the definitions are parameterized in a name, $x$. Can you,
  gentle reader, make a definition that eliminates this parameter and
  guarantees no accidental interaction between the replication
  machinery and the process being replicated -- i.e. no accidental
  sharing of names used by the process to get its work done and the
  name(s) used by the replication to effect copying. This latter
  revision of the definition of replication is crucial to obtaining
  the expected identity $!!P \sim !P$.
\end{remark}

\begin{remark}\label{rem:paradoxical_combinator}
  The reader familiar with the lambda calculus will have noticed the
  similarity between $D$ and the paradoxical combinator.

  [Ed. note: the existence of this seems to suggest we have to be more
  restrictive on the set of processes and names we admit if we are to
  support no-cloning.]
\end{remark}

\subsubsection{Bisimulation}

The computational dynamics gives rise to another kind of equivalence,
the equivalence of computational behavior. As previously mentioned
this is typically captured \emph{via} some form of bisimulation.

% The notion we use in this paper is weak barbed bisimulation
% \cite{milner91polyadicpi}.

The notion we use in this paper is derived from weak barbed
bisimulation \cite{milner91polyadicpi}. 

\begin{definition}
An \emph{observation relation}, $\downarrow_{\mathcal N}$, over a set
of names, $\mathcal N$, is the smallest relation satisfying the rules
below.

\infrule[Out-barb]{y \in {\mathcal N}, \; x \nameeq y}
		  {\outputp{x}{v} \downarrow_{\mathcal N} x}
\infrule[Par-barb]{\mbox{$P\downarrow_{\mathcal N} x$ or $Q\downarrow_{\mathcal N} x$}}
		  {\binpar{P}{Q} \downarrow_{\mathcal N} x}

We write $P \Downarrow_{\mathcal N} x$ if there is $Q$ such that 
$P \wred Q$ and $Q \downarrow_{\mathcal N} x$.
\end{definition}

\begin{definition}
%\label{def.bbisim}
An  ${\mathcal N}$-\emph{barbed bisimulation} over a set of names, ${\mathcal N}$, is a symmetric binary relation 
${\mathcal S}_{\mathcal N}$ between agents such that $P\rel{S}_{\mathcal N}Q$ implies:
\begin{enumerate}
\item If $P \red P'$ then $Q \wred Q'$ and $P'\rel{S}_{\mathcal N} Q'$.
\item If $P\downarrow_{\mathcal N} x$, then $Q\Downarrow_{\mathcal N} x$.
\end{enumerate}
$P$ is ${\mathcal N}$-barbed bisimilar to $Q$, written
$P \wbbisim_{\mathcal N} Q$, if $P \rel{S}_{\mathcal N} Q$ for some ${\mathcal N}$-barbed bisimulation ${\mathcal S}_{\mathcal N}$.
\end{definition}

$\mathcal{R} \subseteq \pi \times \pi$

$P \mathcal{R} Q => \forall P'. P \red P' \Rightarrow \exists Q'. Q \red Q', P' \mathcal{R} Q'$

$P \vdash x \Rightarrow Q \vdash x$

\begin{mathpar}
  \inferrule*[lab=Out-barb]{x \nameeq y}{{y}!\langle{Q}\rangle \vdash x}
  \and
  \inferrule*[lab=Par-barb]{\mbox{$P\vdash x$ or $Q\vdash x$}}{\binpar{P}{Q} \vdash x}
\end{mathpar}

\subsubsection{Contexts}

One of the principle advantages of computational calculi like the
$\pi$-calculus is a well-defined notion of context,
contextual-equivalence and a correlation between
contextual-equivalence and notions of bisimulation. The notion of
context allows the decomposition of a process into (sub-)process and
its syntactic environment, its context. Thus, a context may be
thought of as a process with a ``hole'' (written $\Box$) in it. The
application of a context $M$ to a process $P$, written $M[P]$, is
tantamount to filling the hole in $M$ with $P$. In this paper we do
not need the full weight of this theory, but do make use of the notion
of context in the proof the main theorem. 

\begin{mathpar}
  \inferrule* [lab=summation] {} {{M_{M},M_{N}} \bc \Box \;|\; x.M_{A} \;|\; M_{M}+M_{N}}
  \and
  \inferrule* [lab=agent] {} {{M_{A}} \bc (\vec{x})M_{P} \;| \; \clift{P_0,\ldots,M_{P},\ldots,P_N}}
  \and \\
  \inferrule* [lab=process] {} {{M_{P}} \bc M_{N} \;| \;P|M_{P} }
\end{mathpar} 

\begin{mathpar}
  \inferrule* [lab=sychronization] {} {M_{N} \bc \Box \;|\; x?M_{F} \;|\; x!M_{C}}
  \and
  \inferrule* [lab=abstraction] {} {{M_{F}} \bc (x)M_{P} }
  \and
  \inferrule* [lab=concretion] {} {{M_{C}} \bc \langle M_{P} \rangle }
  \and \\
  \inferrule* [lab=process] {} {{M_{P}} \bc M_{N} \;| \;P|M_{P} }
\end{mathpar}

\begin{definition}[contextual application] Given a context $M$, and
  process $P$, we define the \emph{contextual application}, $M[P] :=
  M\{P/\Box\}$. That is, the contextual application of M to P is the
  substitution of $P$ for $\Box$ in $M$.
\end{definition}

$\meaningof{-} : L \to \mathcal{P}(\pi)$

\begin{mathpar}
  \inferrule* [lab=collection] {} {\meaningof{true} = \pi, \and \meaningof{~E} = \pi \setminus \meaningof{E}, \and \meaningof{E_{1} \& E_{2}} = \meaningof{E_{1}} \cap \meaningof{E_{2}}}
\end{mathpar}

\begin{mathpar}
  \inferrule* [lab=structure] {} {\meaningof{0} = \{ P \in \pi | P \equiv 0 \}, \and \\ \meaningof{E_1 | E_2} = \{ P \in \pi | P \equiv P_{1} | P_{2}, P_{1} \in \meaningof{E_{1}}, P_{2} \in \meaningof{E_2}\} }
\end{mathpar}

\begin{mathpar}
 \inferrule* [lab=behavior] {} {\meaningof{\langle a?b \rangle E} = \{ P \in \pi | P \equiv Q | u?(y)P', \\ \and \\\\ \and \\ \;\;\; u \in \meaningof{a}, \forall z.P'\{z/y\} \in \meaningof{E\{z/b\}}\}, \and \\ \meaningof{a!E} = \{ P \in \pi | P \equiv Q | x!\langle P' \rangle, x \in \meaningof{a} P' \in \meaningof{E}\} }
\end{mathpar}

\begin{mathpar}
 \inferrule* [lab=nominal] {} {\meaningof{\quotep{E}} = \{ \quotep{P} \in \quotep{\pi} | P \in \meaningof{E} \}, \and \meaningof{\quotep{P}} = \{ \quotep{Q} \in \quotep{\pi} | P \equiv Q \} \and \\ \meaningof{@\quotep{E}} = \{ P \in \pi | P \equiv @x, x \in \meaningof{E} \}}
\end{mathpar}

\begin{eqnarray*}
  \\
  \meaningof{-} : TS \to ST
\end{eqnarray*}

\begin{eqnarray*}
  \\
  L : TS \to ST
\end{eqnarray*}

\begin{eqnarray*}
  \\
  P \models E \iff P \in \meaningof{E}
\end{eqnarray*}

\begin{eqnarray*}
  P \approx_{L} Q \iff \forall E \in L. P \models E \iff Q \models E
\end{eqnarray*}

\begin{eqnarray*}
  P \approx_{K} Q
\end{eqnarray*}

\begin{eqnarray*}
  P \approx Q
\end{eqnarray*}

$\approx_{K} = \approx = \approx_{L}$

\subsubsection{Contextual duality}

Note that contexts extend the quotation operation to a family of
operations from processes to names. Given a context, $M$, we can
define a \emph{nominal context}, $\quotep{M}$ by $\quotep{M}[P] :=
\quotep{M[P]}$. To foreshadow what is to come we observe that these
operations enjoy a duality with processes very much like the duality
between vectors and maps from vectors to scalars.

Further, because the calculus is essentially higher-order, we have a
correspondence between contexts and processes. More specifically,
given a name $x$ and a context $M$ we can construct $M^{*}_{x}$ such
that 

\begin{mathpar}
  M^{*}_{x} | \lift{x}{P} \red M[P]
\end{mathpar}

namely,

\begin{mathpar}
  M^{*}_{x} := x?(u).M[\dropn{u}]
\end{mathpar}

The dependence of $M^{*}_{x}$ on a name makes it an abstraction, 

\begin{mathpar}
  M^{*} := (x)x?(u).M[\dropn{u}]
\end{mathpar}

\subsection{Additional notation}

It will sometimes be convenient to denote the process a name
quotes. We already have the notation $x = \quotep{P}$, but it will be
convenient to introduce an alternate notation, $\procn{x}$, when we
want to emphasize the connection to the use of the name. Note that, by
virtue of name equivalence, $\quotep{\procn{x}} \nameeq x$; so, the
notation is consistent with previous definitions.

Further, because names have structure it is possible to effect
substitutions on the basis of that structure. This means we need to
upgrade our notation for substitutions, which we accomplish by
adapting comprehension notation. Thus,

\begin{mathpar}
  P\{ y / x : x \in S \}
\end{mathpar}

is interpreted to mean the process derived from P by replacing (in a
capture-avoiding manner) each occurrence of $x$ in $S$ by $y$. For example,

\begin{mathpar}
  P\{ \quotep{\procn{x}|\procn{x}} / x : x \in \freenames{P} \}
\end{mathpar}

will replace each (occurrence) of a free name $x$ in $P$ by
$\quotep{\procn{x}|\procn{x}}$.

Also, we will avail ourselves of the notation $x^{L}$ and $x^{R}$ to
denote injections of a name into disjoint copies of the name
space. There are numerous ways to accomplish this. One example can be
found in \cite{MeredithR05}. This notation overloads to vectors of
names: $\vec{x}^{\pi} := (x_{i}^{\pi} \; : \; 0 \leq i < |\vec{x}| )$ where $\pi \in \{L,R\}$.

We also use $P^{\Box} := P|\Box$.

In \cite{MeredithR05} an interpretation of the new operator is
given. It turns out that there are several possible interpretations
all enjoying the requisite algebraic properties of the operator (see
\cite{milner91polyadicpi}). We will therefore make liberal use of
$(\nu\; \vec{x})P$.

% subsection the_syntax_and_semantics_of_the_notation_system (end)   

\input{qm2pi.qmops} 

\input{qm2pi.sterngerlach} 

\input{qm2pi.metric} 

% section concurrent_process_calculi (end)

%\input{qm2pi.proofsketch}

% section proof sketch (end)

%\input{qm2pi.slviaknots} 

% section spatial logic via knots (end)

\input{qm2pi.conclusion}

% section conclusion (end)

%\input{qm2pi.dtcodes} 

% section wiring algorithm (end)

\input{qm2pi.ack} 

% section acknowledgments (end)

\newpage


\bibliographystyle{plain}   
\bibliography{../../biblios/main.bib}

\input{qm2pi.rhodetails}

\end{document}

 

%\documentclass[12pt]{llncs}
%\documentclass{jktr}

\usepackage[pdftex]{hyperref}                   
\usepackage {listings}
\usepackage {mathpartir}
\usepackage{bcprules}
%\usepackage{listings}
                       
\usepackage{graphicx} 
%\usepackage[margins=2.5cm,nohead,nofoot]{geometry}
%\usepackage{geometry}
\usepackage{amsfonts}
\usepackage{amstext}
\usepackage{latexsym}
\usepackage{amssymb}
\usepackage{color}


%\include{myPreamble}
\include{qm2pi.local} 

%\ifpdf
%\usepackage[pdftex]{graphicx}
%\else
%\usepackage{graphicx}
%\fi

 % \ifpdf
%  \usepackage{pdfsync}
%  \if


%\title{Brief Article}
%\author{David F. Snyder}
%\author{L.G. Meredith}

%\address{Dept. of Math., Texas State University--San Marcos, San Marcos, TX 78666}
       
\pagestyle{empty}


\begin{document}

\lstset{language=[Objective]Caml,frame=shadowbox}

\input{qm2pi.front}

% section front matter (end)

\input{qm2pi.intro} 
 
% section introduction (end)

% \input{qm2pi.knotations} 

% section notation (end)

\input{qm2pi.process.calculi} 

% section concurrent_process_calculi_and_spatial_logics_ (end)
    
%\input{qm2pi.knots2pi} 

%\input{qm2pi.trefoil} 

%\input{qm2pi.mainthm} 

% subsection basic_interpretation (end)

%\input{qm2pi.rho.presentation} 
\subsection{The syntax and semantics of the notation system}\label{sub:the_syntax_and_semantics_of_the_notation_system} % (fold)

We now summarize a technical presentation of the calculus that
embodies our theory of dynamics. The typical presentation of such a
calculus follows the style of giving generators and relations on
them. The grammar, below, describing term constructors, freely
generates the set of processes, $\Proc$. This set is then quotiented
by a relation known as structural congruence and it is over this set
that the notion of dynamics is expressed. This presentation is
essentially that of \cite{MeredithR05} with the addition of
polyadicity and summation. For readability we have relegated some of
the technical subtleties to an appendix.

\subsubsection{Process grammar}\label{subsub:process_grammar}

\begin{mathpar}
  \inferrule* [lab=synchronization] {} {{M} \bc \pzero \;|\; x?F \;|\; x!C }
  \and
  \inferrule* [lab=abstraction] {} {{F} \bc (x)P}
  \and
  \inferrule* [lab=concretion] {} {{C} \bc \langle Q \rangle}
  \and
  \inferrule* [lab=process] {} {{P,Q} \bc M \;| \;P|Q \;|\; @{x}}
  \and
  \inferrule* [lab=name] {} {{x} \bc \quotep{P}}
\end{mathpar} 

Note that $\vec{x}$ (resp. $\vec{P}$) denotes a vector of names
(resp. processes) of length $|\vec{x}|$ (resp. $|\vec{P}|$). We adopt
the following useful abbreviations.

\begin{mathpar}
   x?(\vec{y}).P := x.(\vec{y})P \and  x\clift{\vec{P}} := x.\clift{\vec{P}}
   \and x!(y) := \lift{x}{\dropn{y}}
   \and \Pi_{i=0}^{n-1}P_i := P_0 | \ldots | P_{n-1}
\end{mathpar}

\subsubsection{Structural congruence}

\paragraph{Free and bound names and alpha-equivalence.} At the
core of structural equivalence is alpha-equivalence which identifies
process that are the same up to a change of variable. Formally, we
recognize the distinction between free and bound names. The free names
of a process, $\freenames{P}$, may be calculated recursively as
follows:

\begin{mathpar}
\freenames{\pzero} := \emptyset
  \and \\
  \freenames{x?(y).P} := \{ x \} \cup (\freenames{P} \setminus \{ y \})
  \and 
  \freenames{x!\langle P \rangle} := \{ x \} \cup \{ P \} 
  \and \\
  \freenames{P|Q} := \freenames{P} \cup \freenames{Q}
  \and \\
  \freenames{@{x}} := \{ x \}
\end{mathpar}

$\pi$
$\quotep{\pi}$

$\freenames{-} : \pi \to \mathcal{P}(\quotep{\pi})$

\begin{eqnarray*}
  \freenames{\pzero} & := & \emptyset \\
  \freenames{x?(y).P} & := & \{ x \} \cup (\freenames{P} \setminus \{ y \}) \\
  \freenames{x!\langle P \rangle} & := & \{ x \} \cup \{ P \} \\
  \freenames{P|Q} & := & \freenames{P} \cup \freenames{Q} \\
  \freenames{\dropn{x}} & := & \{ x \}
\end{eqnarray*}

The bound names of a process, $\boundnames{P}$, are those names occurring in $P$
that are not free. For example, in $x?(y).0$, the name $x$ is free, while $y$ is bound.

\begin{mathpar}
  \inferrule* [lab=monoidal-laws] {} { P|Q \equiv Q|P \and P|0 \equiv P \and P|(Q|R) \equiv (P|Q)|R }
\end{mathpar}

\begin{mathpar}
  \inferrule* [lab=alpha-equivalence] {} { (x)P \equiv (y)P\{y/x\} \and y \not\in \freenames{P} }
\end{mathpar}

\begin{definition}
Then two processes, $P,Q$, are alpha-equivalent if $P = Q\{\vec{y}/\vec{x}\}$ for
some $\vec{x} \in \boundnames{Q},\vec{y} \in \boundnames{P}$, where $Q\{\vec{y}/\vec{x}\}$
denotes the capture-avoiding substitution of $\vec{y}$ for $\vec{x}$ in $Q$.
\end{definition}

\begin{definition}
  The {\em structural congruence} \cite{SangiorgiWalker} , $\equiv$,
  between processes is the least congruence containing
  alpha-equivalence, satisfying the abelian monoid laws
  (associativity, commutativity and $\pzero$ as identity) for parallel
  composition $|$ and for summation $+$.
\end{definition}

\subsection{Name equivalence}

We take name equivalence, written $\nameeq$, to be the smallest
equivalence relation generated by the following rules.

\begin{mathpar}
\inferrule*[lab=Quote-drop]
{ }
{ \quotep{@{x}} \nameeq x }

\inferrule*[lab=Struct-equiv]
{ P \scong Q }
{ \quotep{P} \nameeq \quotep{Q} }
\end{mathpar}

The astute reader will have noticed that the mutual recursion of names
and processes imposes a mutual recursion on alpha-equivalence and
structural equivalence via name-equivalence. Fortunately, all of this
works out pleasantly and we may calculate in the natural way, free of
concern. The reader interested in the details is referred to the
appendix \ref{appendix:rho_details}.

\subsection{Substitution}

We use $\Proc$ for the set of processes, $\QProc$ for the set of
names, and $\id{\{}\vec{y} / \vec{x} \id{\}}$ to denote partial maps,
$s : \QProc \rightarrow \QProc$. A map, $s$ lifts, uniquely, to a map
on process terms, $\widehat{s} : \Proc \rightarrow \Proc$ by the
following equations.

\begin{mathpar}
  (0) \psubstp{Q}{P} := 0 \\
  (R \juxtap S) \psubstp{Q}{P}
  :=    
  (R)\psubstp{Q}{P} \juxtap (S) \psubstp{Q}{P} \\
  (x?(y).R) \psubstp{Q}{P}    
  :=    
  (x)\substp{Q}{P} (z)\concat( (R \psubstn{z}{y}) \psubstp{Q}{P} ) \\
  (\lift{x}{R}) \psubstp{Q}{P}  
  :=
  \lift{(x)\substp{Q}{P}}{ R \psubstp{Q}{P} } \\
%   (\dropn{x})  \psubstp{Q}{P}       
%   := 
%   \left\{ 
%     \begin{array}{ccc} 
%       \dropn{\quotep{Q}} & & x \nameeq \quotep{P} \\
%       \dropn{x} & & otherwise \\
%     \end{array}
%   \right. 
  (\dropn{x})  \psubstp{Q}{P}       
  := 
  \left\{ 
    \begin{array}{ccc} 
      Q & & x \nameeq \quotep{P} \\
      \dropn{x} & & otherwise \\
    \end{array}
  \right.
\end{mathpar}
 

where

\begin{eqnarray}
  (x)\id{\{} \lpquote Q \rpquote / \lpquote P \rpquote \id{\}}            = 
  \left\{ 
    \begin{array}{ccc}
      \lpquote Q \rpquote & & x \nameeq \lpquote P \rpquote \\
      x & & otherwise \\
    \end{array}
  \right. \nonumber
\end{eqnarray}

and $z$ is chosen distinct from $\quotep{P}$, $\quotep{Q}$, the free
names in $Q$, and all the names in $R$. Our $\alpha$-equivalence will
be built in the standard way from this substitution.

\begin{remark}\label{rem:no_self_referential_names}
  One consequence of these definitions is that $\forall P. \quotep{P}
  \not\in \freenames{P}$.
\end{remark}

\subsection{ Dynamic quote: an example }

Anticipating something of what's to come, consider applying the
substitution, $\widehat{\id{\{}u / z \id{\}}}$, to the following pair
of processes, $\lift{w}{y!(z)}$ and $w[ \lpquote y!(z) \rpquote ]$.

\begin{eqnarray}
	\lift{w}{y!(z)}\widehat{\id{\{}u / z \id{\}}}
		& = &
		\lift{w}{y!(u)} \nonumber\\
	w[ \lpquote y!(z) \rpquote ] \widehat{ \id{\{}u / z \id{\}} }
		& = &
		w[ \lpquote y!(z) \rpquote ] \nonumber
\end{eqnarray}

Because the body of the process between quotes is impervious to
substitution, we get radically different answers. In fact, by
examining the first process in an input context,
e.g. $x?(z).\lift{w}{y!(z)}$, we see that the process under the lift
operator may be shaped by prefixed inputs binding a name inside it. In
this sense, the lift operator will be seen as a way to dynamically
construct processes before reifying them as names.

Finally equipped with these standard features we can present the
dynamics of the calculus.

\subsubsection{Operational semantics} 

Finally, we introduce the computational dynamics. What marks these
algebras as distinct from other more traditionally studied algebraic
structures, e.g. vector spaces or polynomial rings, is the manner in
which dynamics is captured. In traditional structures, dynamics is typically
expressed through morphisms between such structures, as in linear maps
between vector spaces or morphisms between rings. In algebras
associated with the semantics of computation, the dynamics is
expressed as part of the algebraic structure itself, through a
reduction reduction relation typically denoted by $\red$. Below, we
give a recursive presentation of this relation for the calculus used
in the encoding.

$\red \subseteq \pi \times \pi$
$\red : \pi \to \mathcal{P}(\pi)$

\begin{mathpar}
  \inferrule* [lab=Comm] { \textsf{match}( x_{src}, x_{trgt} ) } { x_{trgt}?(y)P \; | \; x_{src}!\langle {Q} \rangle \red P\{\quotep{Q}/y}\} }
  \and \\
  \inferrule* [lab=Par] {{P} \red {P}'} {{{P} | {Q}} \red {{P}' | {Q}}}
  \and
  \inferrule* [lab=Equiv]{{{P} \scong {P}'} \andalso {{P}' \red {Q}'} \andalso {{Q}' \scong {Q}}}{{P} \red {Q}}
\end{mathpar}

\begin{eqnarray*}
  match_{\equiv} (\quotep{P},\quotep{Q}) & := & P \equiv Q \\
  match_{\dagger}(\quotep{P},\quotep{Q}) & := & \forall R. P|Q \red^{*} R => R \red^{*} 0 \\
  match_{K}(\quotep{P},\quotep{Q}) & := & K \mbox{ for some context } K
\end{eqnarray*}

$u?(x)P | u!\langle Q \rangle \red P\{\quotep{Q}/x\}$

%We write $\wred$ for $\red^*$, and $P\red$ if $\exists Q $ such that $ P \red Q$.
We write $P\red$ if $\exists Q $ such that $ P \red Q$ and $P\not\red$, otherwise.

\section{Replication}

As mentioned before, it is known that replication (and hence
recursion) can be implemented in a higher-order process algebra
\cite{SangiorgiWalker}. As our first example of calculation with the
machinery thus far presented we give the construction explicitly in
the {\rhoc}.

\begin{eqnarray}
	D_{x} & := & \prefix{x}{y}{(\binpar{\outputp{x}{y}}{@{y}})} \nonumber\\
	\bangp_{x}{P} & := & \binpar{{x}!\langle{\binpar{D_{x}}{P}}\rangle}{D_{x}} \nonumber
\end{eqnarray}

\begin{eqnarray}
	\bangp_{x}{P} & & \nonumber\\
	=
	& {x}!\langle{(\prefix{x}{y}{(\outputp{x}{y} | @{y})) | P}}\rangle 
	      | \prefix{x}{y}{(\outputp{x}{y} | @{y})} & \nonumber\\
	\red
	& (\outputp{x}{y} | @{y})\substn{\quotep{(\prefix{x}{y}{(@{y} | \outputp{x}{y})) | P}}}{y} & \nonumber\\
	=
	& \outputp{x}{\quotep{(\prefix{x}{y}{(\outputp{x}{y} | @{y})) | P}}}
	  | {(\prefix{x}{y}{(\outputp{x}{y} | @{y})) | P}} & \nonumber\\
	\red
	& \ldots & \nonumber\\
	\red^*
	& P | P | \ldots & \nonumber
\end{eqnarray}

Of course, this encoding, as an implementation, runs away, unfolding
$\bangp{P}$ eagerly. A lazier and more implementable replication
operator, restricted to input-guarded processes, may be obtained as follows.

\begin{eqnarray}
\bangp{\prefix{u}{v}{P}} 
	:= 
	\binpar{\lift{x}{\prefix{u}{v}{(\binpar{D(x)}{P})}}}{D(x)} \nonumber
\end{eqnarray}

\begin{remark}
  Note that the lazier definition still does not deal with summation
  or mixed summation (i.e. sums over input and output). The reader is
  invited to construct definitions of replication that deal with these
  features. 

  Further, the definitions are parameterized in a name, $x$. Can you,
  gentle reader, make a definition that eliminates this parameter and
  guarantees no accidental interaction between the replication
  machinery and the process being replicated -- i.e. no accidental
  sharing of names used by the process to get its work done and the
  name(s) used by the replication to effect copying. This latter
  revision of the definition of replication is crucial to obtaining
  the expected identity $!!P \sim !P$.
\end{remark}

\begin{remark}\label{rem:paradoxical_combinator}
  The reader familiar with the lambda calculus will have noticed the
  similarity between $D$ and the paradoxical combinator.

  [Ed. note: the existence of this seems to suggest we have to be more
  restrictive on the set of processes and names we admit if we are to
  support no-cloning.]
\end{remark}

\subsubsection{Bisimulation}

The computational dynamics gives rise to another kind of equivalence,
the equivalence of computational behavior. As previously mentioned
this is typically captured \emph{via} some form of bisimulation.

% The notion we use in this paper is weak barbed bisimulation
% \cite{milner91polyadicpi}.

The notion we use in this paper is derived from weak barbed
bisimulation \cite{milner91polyadicpi}. 

\begin{definition}
An \emph{observation relation}, $\downarrow_{\mathcal N}$, over a set
of names, $\mathcal N$, is the smallest relation satisfying the rules
below.

\infrule[Out-barb]{y \in {\mathcal N}, \; x \nameeq y}
		  {\outputp{x}{v} \downarrow_{\mathcal N} x}
\infrule[Par-barb]{\mbox{$P\downarrow_{\mathcal N} x$ or $Q\downarrow_{\mathcal N} x$}}
		  {\binpar{P}{Q} \downarrow_{\mathcal N} x}

We write $P \Downarrow_{\mathcal N} x$ if there is $Q$ such that 
$P \wred Q$ and $Q \downarrow_{\mathcal N} x$.
\end{definition}

\begin{definition}
%\label{def.bbisim}
An  ${\mathcal N}$-\emph{barbed bisimulation} over a set of names, ${\mathcal N}$, is a symmetric binary relation 
${\mathcal S}_{\mathcal N}$ between agents such that $P\rel{S}_{\mathcal N}Q$ implies:
\begin{enumerate}
\item If $P \red P'$ then $Q \wred Q'$ and $P'\rel{S}_{\mathcal N} Q'$.
\item If $P\downarrow_{\mathcal N} x$, then $Q\Downarrow_{\mathcal N} x$.
\end{enumerate}
$P$ is ${\mathcal N}$-barbed bisimilar to $Q$, written
$P \wbbisim_{\mathcal N} Q$, if $P \rel{S}_{\mathcal N} Q$ for some ${\mathcal N}$-barbed bisimulation ${\mathcal S}_{\mathcal N}$.
\end{definition}

$\mathcal{R} \subseteq \pi \times \pi$

$P \mathcal{R} Q => \forall P'. P \red P' \Rightarrow \exists Q'. Q \red Q', P' \mathcal{R} Q'$

$P \vdash x \Rightarrow Q \vdash x$

\begin{mathpar}
  \inferrule*[lab=Out-barb]{x \nameeq y}{{y}!\langle{Q}\rangle \vdash x}
  \and
  \inferrule*[lab=Par-barb]{\mbox{$P\vdash x$ or $Q\vdash x$}}{\binpar{P}{Q} \vdash x}
\end{mathpar}

\subsubsection{Contexts}

One of the principle advantages of computational calculi like the
$\pi$-calculus is a well-defined notion of context,
contextual-equivalence and a correlation between
contextual-equivalence and notions of bisimulation. The notion of
context allows the decomposition of a process into (sub-)process and
its syntactic environment, its context. Thus, a context may be
thought of as a process with a ``hole'' (written $\Box$) in it. The
application of a context $M$ to a process $P$, written $M[P]$, is
tantamount to filling the hole in $M$ with $P$. In this paper we do
not need the full weight of this theory, but do make use of the notion
of context in the proof the main theorem. 

\begin{mathpar}
  \inferrule* [lab=summation] {} {{M_{M},M_{N}} \bc \Box \;|\; x.M_{A} \;|\; M_{M}+M_{N}}
  \and
  \inferrule* [lab=agent] {} {{M_{A}} \bc (\vec{x})M_{P} \;| \; \clift{P_0,\ldots,M_{P},\ldots,P_N}}
  \and \\
  \inferrule* [lab=process] {} {{M_{P}} \bc M_{N} \;| \;P|M_{P} }
\end{mathpar} 

\begin{mathpar}
  \inferrule* [lab=sychronization] {} {M_{N} \bc \Box \;|\; x?M_{F} \;|\; x!M_{C}}
  \and
  \inferrule* [lab=abstraction] {} {{M_{F}} \bc (x)M_{P} }
  \and
  \inferrule* [lab=concretion] {} {{M_{C}} \bc \langle M_{P} \rangle }
  \and \\
  \inferrule* [lab=process] {} {{M_{P}} \bc M_{N} \;| \;P|M_{P} }
\end{mathpar}

\begin{definition}[contextual application] Given a context $M$, and
  process $P$, we define the \emph{contextual application}, $M[P] :=
  M\{P/\Box\}$. That is, the contextual application of M to P is the
  substitution of $P$ for $\Box$ in $M$.
\end{definition}

$\meaningof{-} : L \to \mathcal{P}(\pi)$

\begin{mathpar}
  \inferrule* [lab=collection] {} {\meaningof{true} = \pi, \and \meaningof{~E} = \pi \setminus \meaningof{E}, \and \meaningof{E_{1} \& E_{2}} = \meaningof{E_{1}} \cap \meaningof{E_{2}}}
\end{mathpar}

\begin{mathpar}
  \inferrule* [lab=structure] {} {\meaningof{0} = \{ P \in \pi | P \equiv 0 \}, \and \\ \meaningof{E_1 | E_2} = \{ P \in \pi | P \equiv P_{1} | P_{2}, P_{1} \in \meaningof{E_{1}}, P_{2} \in \meaningof{E_2}\} }
\end{mathpar}

\begin{mathpar}
 \inferrule* [lab=behavior] {} {\meaningof{\langle a?b \rangle E} = \{ P \in \pi | P \equiv Q | u?(y)P', \\ \and \\\\ \and \\ \;\;\; u \in \meaningof{a}, \forall z.P'\{z/y\} \in \meaningof{E\{z/b\}}\}, \and \\ \meaningof{a!E} = \{ P \in \pi | P \equiv Q | x!\langle P' \rangle, x \in \meaningof{a} P' \in \meaningof{E}\} }
\end{mathpar}

\begin{mathpar}
 \inferrule* [lab=nominal] {} {\meaningof{\quotep{E}} = \{ \quotep{P} \in \quotep{\pi} | P \in \meaningof{E} \}, \and \meaningof{\quotep{P}} = \{ \quotep{Q} \in \quotep{\pi} | P \equiv Q \} \and \\ \meaningof{@\quotep{E}} = \{ P \in \pi | P \equiv @x, x \in \meaningof{E} \}}
\end{mathpar}

\begin{eqnarray*}
  \\
  \meaningof{-} : TS \to ST
\end{eqnarray*}

\begin{eqnarray*}
  \\
  L : TS \to ST
\end{eqnarray*}

\begin{eqnarray*}
  \\
  P \models E \iff P \in \meaningof{E}
\end{eqnarray*}

\begin{eqnarray*}
  P \approx_{L} Q \iff \forall E \in L. P \models E \iff Q \models E
\end{eqnarray*}

\begin{eqnarray*}
  P \approx_{K} Q
\end{eqnarray*}

\begin{eqnarray*}
  P \approx Q
\end{eqnarray*}

$\approx_{K} = \approx = \approx_{L}$

\subsubsection{Contextual duality}

Note that contexts extend the quotation operation to a family of
operations from processes to names. Given a context, $M$, we can
define a \emph{nominal context}, $\quotep{M}$ by $\quotep{M}[P] :=
\quotep{M[P]}$. To foreshadow what is to come we observe that these
operations enjoy a duality with processes very much like the duality
between vectors and maps from vectors to scalars.

Further, because the calculus is essentially higher-order, we have a
correspondence between contexts and processes. More specifically,
given a name $x$ and a context $M$ we can construct $M^{*}_{x}$ such
that 

\begin{mathpar}
  M^{*}_{x} | \lift{x}{P} \red M[P]
\end{mathpar}

namely,

\begin{mathpar}
  M^{*}_{x} := x?(u).M[\dropn{u}]
\end{mathpar}

The dependence of $M^{*}_{x}$ on a name makes it an abstraction, 

\begin{mathpar}
  M^{*} := (x)x?(u).M[\dropn{u}]
\end{mathpar}

\subsection{Additional notation}

It will sometimes be convenient to denote the process a name
quotes. We already have the notation $x = \quotep{P}$, but it will be
convenient to introduce an alternate notation, $\procn{x}$, when we
want to emphasize the connection to the use of the name. Note that, by
virtue of name equivalence, $\quotep{\procn{x}} \nameeq x$; so, the
notation is consistent with previous definitions.

Further, because names have structure it is possible to effect
substitutions on the basis of that structure. This means we need to
upgrade our notation for substitutions, which we accomplish by
adapting comprehension notation. Thus,

\begin{mathpar}
  P\{ y / x : x \in S \}
\end{mathpar}

is interpreted to mean the process derived from P by replacing (in a
capture-avoiding manner) each occurrence of $x$ in $S$ by $y$. For example,

\begin{mathpar}
  P\{ \quotep{\procn{x}|\procn{x}} / x : x \in \freenames{P} \}
\end{mathpar}

will replace each (occurrence) of a free name $x$ in $P$ by
$\quotep{\procn{x}|\procn{x}}$.

Also, we will avail ourselves of the notation $x^{L}$ and $x^{R}$ to
denote injections of a name into disjoint copies of the name
space. There are numerous ways to accomplish this. One example can be
found in \cite{MeredithR05}. This notation overloads to vectors of
names: $\vec{x}^{\pi} := (x_{i}^{\pi} \; : \; 0 \leq i < |\vec{x}| )$ where $\pi \in \{L,R\}$.

We also use $P^{\Box} := P|\Box$.

In \cite{MeredithR05} an interpretation of the new operator is
given. It turns out that there are several possible interpretations
all enjoying the requisite algebraic properties of the operator (see
\cite{milner91polyadicpi}). We will therefore make liberal use of
$(\nu\; \vec{x})P$.

% subsection the_syntax_and_semantics_of_the_notation_system (end)   

\input{qm2pi.qmops} 

\input{qm2pi.sterngerlach} 

\input{qm2pi.metric} 

% section concurrent_process_calculi (end)

%\input{qm2pi.proofsketch}

% section proof sketch (end)

%\input{qm2pi.slviaknots} 

% section spatial logic via knots (end)

\input{qm2pi.conclusion}

% section conclusion (end)

%\input{qm2pi.dtcodes} 

% section wiring algorithm (end)

\input{qm2pi.ack} 

% section acknowledgments (end)

\newpage


\bibliographystyle{plain}   
\bibliography{../../biblios/main.bib}

\input{qm2pi.rhodetails}

\end{document}

 

%\documentclass[12pt]{llncs}
%\documentclass{jktr}

\usepackage[pdftex]{hyperref}                   
\usepackage {listings}
\usepackage {mathpartir}
\usepackage{bcprules}
%\usepackage{listings}
                       
\usepackage{graphicx} 
%\usepackage[margins=2.5cm,nohead,nofoot]{geometry}
%\usepackage{geometry}
\usepackage{amsfonts}
\usepackage{amstext}
\usepackage{latexsym}
\usepackage{amssymb}
\usepackage{color}


%\include{myPreamble}
\include{qm2pi.local} 

%\ifpdf
%\usepackage[pdftex]{graphicx}
%\else
%\usepackage{graphicx}
%\fi

 % \ifpdf
%  \usepackage{pdfsync}
%  \if


%\title{Brief Article}
%\author{David F. Snyder}
%\author{L.G. Meredith}

%\address{Dept. of Math., Texas State University--San Marcos, San Marcos, TX 78666}
       
\pagestyle{empty}


\begin{document}

\lstset{language=[Objective]Caml,frame=shadowbox}

\input{qm2pi.front}

% section front matter (end)

\input{qm2pi.intro} 
 
% section introduction (end)

% \input{qm2pi.knotations} 

% section notation (end)

\input{qm2pi.process.calculi} 

% section concurrent_process_calculi_and_spatial_logics_ (end)
    
%\input{qm2pi.knots2pi} 

%\input{qm2pi.trefoil} 

%\input{qm2pi.mainthm} 

% subsection basic_interpretation (end)

%\input{qm2pi.rho.presentation} 
\subsection{The syntax and semantics of the notation system}\label{sub:the_syntax_and_semantics_of_the_notation_system} % (fold)

We now summarize a technical presentation of the calculus that
embodies our theory of dynamics. The typical presentation of such a
calculus follows the style of giving generators and relations on
them. The grammar, below, describing term constructors, freely
generates the set of processes, $\Proc$. This set is then quotiented
by a relation known as structural congruence and it is over this set
that the notion of dynamics is expressed. This presentation is
essentially that of \cite{MeredithR05} with the addition of
polyadicity and summation. For readability we have relegated some of
the technical subtleties to an appendix.

\subsubsection{Process grammar}\label{subsub:process_grammar}

\begin{mathpar}
  \inferrule* [lab=synchronization] {} {{M} \bc \pzero \;|\; x?F \;|\; x!C }
  \and
  \inferrule* [lab=abstraction] {} {{F} \bc (x)P}
  \and
  \inferrule* [lab=concretion] {} {{C} \bc \langle Q \rangle}
  \and
  \inferrule* [lab=process] {} {{P,Q} \bc M \;| \;P|Q \;|\; @{x}}
  \and
  \inferrule* [lab=name] {} {{x} \bc \quotep{P}}
\end{mathpar} 

Note that $\vec{x}$ (resp. $\vec{P}$) denotes a vector of names
(resp. processes) of length $|\vec{x}|$ (resp. $|\vec{P}|$). We adopt
the following useful abbreviations.

\begin{mathpar}
   x?(\vec{y}).P := x.(\vec{y})P \and  x\clift{\vec{P}} := x.\clift{\vec{P}}
   \and x!(y) := \lift{x}{\dropn{y}}
   \and \Pi_{i=0}^{n-1}P_i := P_0 | \ldots | P_{n-1}
\end{mathpar}

\subsubsection{Structural congruence}

\paragraph{Free and bound names and alpha-equivalence.} At the
core of structural equivalence is alpha-equivalence which identifies
process that are the same up to a change of variable. Formally, we
recognize the distinction between free and bound names. The free names
of a process, $\freenames{P}$, may be calculated recursively as
follows:

\begin{mathpar}
\freenames{\pzero} := \emptyset
  \and \\
  \freenames{x?(y).P} := \{ x \} \cup (\freenames{P} \setminus \{ y \})
  \and 
  \freenames{x!\langle P \rangle} := \{ x \} \cup \{ P \} 
  \and \\
  \freenames{P|Q} := \freenames{P} \cup \freenames{Q}
  \and \\
  \freenames{@{x}} := \{ x \}
\end{mathpar}

$\pi$
$\quotep{\pi}$

$\freenames{-} : \pi \to \mathcal{P}(\quotep{\pi})$

\begin{eqnarray*}
  \freenames{\pzero} & := & \emptyset \\
  \freenames{x?(y).P} & := & \{ x \} \cup (\freenames{P} \setminus \{ y \}) \\
  \freenames{x!\langle P \rangle} & := & \{ x \} \cup \{ P \} \\
  \freenames{P|Q} & := & \freenames{P} \cup \freenames{Q} \\
  \freenames{\dropn{x}} & := & \{ x \}
\end{eqnarray*}

The bound names of a process, $\boundnames{P}$, are those names occurring in $P$
that are not free. For example, in $x?(y).0$, the name $x$ is free, while $y$ is bound.

\begin{mathpar}
  \inferrule* [lab=monoidal-laws] {} { P|Q \equiv Q|P \and P|0 \equiv P \and P|(Q|R) \equiv (P|Q)|R }
\end{mathpar}

\begin{mathpar}
  \inferrule* [lab=alpha-equivalence] {} { (x)P \equiv (y)P\{y/x\} \and y \not\in \freenames{P} }
\end{mathpar}

\begin{definition}
Then two processes, $P,Q$, are alpha-equivalent if $P = Q\{\vec{y}/\vec{x}\}$ for
some $\vec{x} \in \boundnames{Q},\vec{y} \in \boundnames{P}$, where $Q\{\vec{y}/\vec{x}\}$
denotes the capture-avoiding substitution of $\vec{y}$ for $\vec{x}$ in $Q$.
\end{definition}

\begin{definition}
  The {\em structural congruence} \cite{SangiorgiWalker} , $\equiv$,
  between processes is the least congruence containing
  alpha-equivalence, satisfying the abelian monoid laws
  (associativity, commutativity and $\pzero$ as identity) for parallel
  composition $|$ and for summation $+$.
\end{definition}

\subsection{Name equivalence}

We take name equivalence, written $\nameeq$, to be the smallest
equivalence relation generated by the following rules.

\begin{mathpar}
\inferrule*[lab=Quote-drop]
{ }
{ \quotep{@{x}} \nameeq x }

\inferrule*[lab=Struct-equiv]
{ P \scong Q }
{ \quotep{P} \nameeq \quotep{Q} }
\end{mathpar}

The astute reader will have noticed that the mutual recursion of names
and processes imposes a mutual recursion on alpha-equivalence and
structural equivalence via name-equivalence. Fortunately, all of this
works out pleasantly and we may calculate in the natural way, free of
concern. The reader interested in the details is referred to the
appendix \ref{appendix:rho_details}.

\subsection{Substitution}

We use $\Proc$ for the set of processes, $\QProc$ for the set of
names, and $\id{\{}\vec{y} / \vec{x} \id{\}}$ to denote partial maps,
$s : \QProc \rightarrow \QProc$. A map, $s$ lifts, uniquely, to a map
on process terms, $\widehat{s} : \Proc \rightarrow \Proc$ by the
following equations.

\begin{mathpar}
  (0) \psubstp{Q}{P} := 0 \\
  (R \juxtap S) \psubstp{Q}{P}
  :=    
  (R)\psubstp{Q}{P} \juxtap (S) \psubstp{Q}{P} \\
  (x?(y).R) \psubstp{Q}{P}    
  :=    
  (x)\substp{Q}{P} (z)\concat( (R \psubstn{z}{y}) \psubstp{Q}{P} ) \\
  (\lift{x}{R}) \psubstp{Q}{P}  
  :=
  \lift{(x)\substp{Q}{P}}{ R \psubstp{Q}{P} } \\
%   (\dropn{x})  \psubstp{Q}{P}       
%   := 
%   \left\{ 
%     \begin{array}{ccc} 
%       \dropn{\quotep{Q}} & & x \nameeq \quotep{P} \\
%       \dropn{x} & & otherwise \\
%     \end{array}
%   \right. 
  (\dropn{x})  \psubstp{Q}{P}       
  := 
  \left\{ 
    \begin{array}{ccc} 
      Q & & x \nameeq \quotep{P} \\
      \dropn{x} & & otherwise \\
    \end{array}
  \right.
\end{mathpar}
 

where

\begin{eqnarray}
  (x)\id{\{} \lpquote Q \rpquote / \lpquote P \rpquote \id{\}}            = 
  \left\{ 
    \begin{array}{ccc}
      \lpquote Q \rpquote & & x \nameeq \lpquote P \rpquote \\
      x & & otherwise \\
    \end{array}
  \right. \nonumber
\end{eqnarray}

and $z$ is chosen distinct from $\quotep{P}$, $\quotep{Q}$, the free
names in $Q$, and all the names in $R$. Our $\alpha$-equivalence will
be built in the standard way from this substitution.

\begin{remark}\label{rem:no_self_referential_names}
  One consequence of these definitions is that $\forall P. \quotep{P}
  \not\in \freenames{P}$.
\end{remark}

\subsection{ Dynamic quote: an example }

Anticipating something of what's to come, consider applying the
substitution, $\widehat{\id{\{}u / z \id{\}}}$, to the following pair
of processes, $\lift{w}{y!(z)}$ and $w[ \lpquote y!(z) \rpquote ]$.

\begin{eqnarray}
	\lift{w}{y!(z)}\widehat{\id{\{}u / z \id{\}}}
		& = &
		\lift{w}{y!(u)} \nonumber\\
	w[ \lpquote y!(z) \rpquote ] \widehat{ \id{\{}u / z \id{\}} }
		& = &
		w[ \lpquote y!(z) \rpquote ] \nonumber
\end{eqnarray}

Because the body of the process between quotes is impervious to
substitution, we get radically different answers. In fact, by
examining the first process in an input context,
e.g. $x?(z).\lift{w}{y!(z)}$, we see that the process under the lift
operator may be shaped by prefixed inputs binding a name inside it. In
this sense, the lift operator will be seen as a way to dynamically
construct processes before reifying them as names.

Finally equipped with these standard features we can present the
dynamics of the calculus.

\subsubsection{Operational semantics} 

Finally, we introduce the computational dynamics. What marks these
algebras as distinct from other more traditionally studied algebraic
structures, e.g. vector spaces or polynomial rings, is the manner in
which dynamics is captured. In traditional structures, dynamics is typically
expressed through morphisms between such structures, as in linear maps
between vector spaces or morphisms between rings. In algebras
associated with the semantics of computation, the dynamics is
expressed as part of the algebraic structure itself, through a
reduction reduction relation typically denoted by $\red$. Below, we
give a recursive presentation of this relation for the calculus used
in the encoding.

$\red \subseteq \pi \times \pi$
$\red : \pi \to \mathcal{P}(\pi)$

\begin{mathpar}
  \inferrule* [lab=Comm] { \textsf{match}( x_{src}, x_{trgt} ) } { x_{trgt}?(y)P \; | \; x_{src}!\langle {Q} \rangle \red P\{\quotep{Q}/y}\} }
  \and \\
  \inferrule* [lab=Par] {{P} \red {P}'} {{{P} | {Q}} \red {{P}' | {Q}}}
  \and
  \inferrule* [lab=Equiv]{{{P} \scong {P}'} \andalso {{P}' \red {Q}'} \andalso {{Q}' \scong {Q}}}{{P} \red {Q}}
\end{mathpar}

\begin{eqnarray*}
  match_{\equiv} (\quotep{P},\quotep{Q}) & := & P \equiv Q \\
  match_{\dagger}(\quotep{P},\quotep{Q}) & := & \forall R. P|Q \red^{*} R => R \red^{*} 0 \\
  match_{K}(\quotep{P},\quotep{Q}) & := & K \mbox{ for some context } K
\end{eqnarray*}

$u?(x)P | u!\langle Q \rangle \red P\{\quotep{Q}/x\}$

%We write $\wred$ for $\red^*$, and $P\red$ if $\exists Q $ such that $ P \red Q$.
We write $P\red$ if $\exists Q $ such that $ P \red Q$ and $P\not\red$, otherwise.

\section{Replication}

As mentioned before, it is known that replication (and hence
recursion) can be implemented in a higher-order process algebra
\cite{SangiorgiWalker}. As our first example of calculation with the
machinery thus far presented we give the construction explicitly in
the {\rhoc}.

\begin{eqnarray}
	D_{x} & := & \prefix{x}{y}{(\binpar{\outputp{x}{y}}{@{y}})} \nonumber\\
	\bangp_{x}{P} & := & \binpar{{x}!\langle{\binpar{D_{x}}{P}}\rangle}{D_{x}} \nonumber
\end{eqnarray}

\begin{eqnarray}
	\bangp_{x}{P} & & \nonumber\\
	=
	& {x}!\langle{(\prefix{x}{y}{(\outputp{x}{y} | @{y})) | P}}\rangle 
	      | \prefix{x}{y}{(\outputp{x}{y} | @{y})} & \nonumber\\
	\red
	& (\outputp{x}{y} | @{y})\substn{\quotep{(\prefix{x}{y}{(@{y} | \outputp{x}{y})) | P}}}{y} & \nonumber\\
	=
	& \outputp{x}{\quotep{(\prefix{x}{y}{(\outputp{x}{y} | @{y})) | P}}}
	  | {(\prefix{x}{y}{(\outputp{x}{y} | @{y})) | P}} & \nonumber\\
	\red
	& \ldots & \nonumber\\
	\red^*
	& P | P | \ldots & \nonumber
\end{eqnarray}

Of course, this encoding, as an implementation, runs away, unfolding
$\bangp{P}$ eagerly. A lazier and more implementable replication
operator, restricted to input-guarded processes, may be obtained as follows.

\begin{eqnarray}
\bangp{\prefix{u}{v}{P}} 
	:= 
	\binpar{\lift{x}{\prefix{u}{v}{(\binpar{D(x)}{P})}}}{D(x)} \nonumber
\end{eqnarray}

\begin{remark}
  Note that the lazier definition still does not deal with summation
  or mixed summation (i.e. sums over input and output). The reader is
  invited to construct definitions of replication that deal with these
  features. 

  Further, the definitions are parameterized in a name, $x$. Can you,
  gentle reader, make a definition that eliminates this parameter and
  guarantees no accidental interaction between the replication
  machinery and the process being replicated -- i.e. no accidental
  sharing of names used by the process to get its work done and the
  name(s) used by the replication to effect copying. This latter
  revision of the definition of replication is crucial to obtaining
  the expected identity $!!P \sim !P$.
\end{remark}

\begin{remark}\label{rem:paradoxical_combinator}
  The reader familiar with the lambda calculus will have noticed the
  similarity between $D$ and the paradoxical combinator.

  [Ed. note: the existence of this seems to suggest we have to be more
  restrictive on the set of processes and names we admit if we are to
  support no-cloning.]
\end{remark}

\subsubsection{Bisimulation}

The computational dynamics gives rise to another kind of equivalence,
the equivalence of computational behavior. As previously mentioned
this is typically captured \emph{via} some form of bisimulation.

% The notion we use in this paper is weak barbed bisimulation
% \cite{milner91polyadicpi}.

The notion we use in this paper is derived from weak barbed
bisimulation \cite{milner91polyadicpi}. 

\begin{definition}
An \emph{observation relation}, $\downarrow_{\mathcal N}$, over a set
of names, $\mathcal N$, is the smallest relation satisfying the rules
below.

\infrule[Out-barb]{y \in {\mathcal N}, \; x \nameeq y}
		  {\outputp{x}{v} \downarrow_{\mathcal N} x}
\infrule[Par-barb]{\mbox{$P\downarrow_{\mathcal N} x$ or $Q\downarrow_{\mathcal N} x$}}
		  {\binpar{P}{Q} \downarrow_{\mathcal N} x}

We write $P \Downarrow_{\mathcal N} x$ if there is $Q$ such that 
$P \wred Q$ and $Q \downarrow_{\mathcal N} x$.
\end{definition}

\begin{definition}
%\label{def.bbisim}
An  ${\mathcal N}$-\emph{barbed bisimulation} over a set of names, ${\mathcal N}$, is a symmetric binary relation 
${\mathcal S}_{\mathcal N}$ between agents such that $P\rel{S}_{\mathcal N}Q$ implies:
\begin{enumerate}
\item If $P \red P'$ then $Q \wred Q'$ and $P'\rel{S}_{\mathcal N} Q'$.
\item If $P\downarrow_{\mathcal N} x$, then $Q\Downarrow_{\mathcal N} x$.
\end{enumerate}
$P$ is ${\mathcal N}$-barbed bisimilar to $Q$, written
$P \wbbisim_{\mathcal N} Q$, if $P \rel{S}_{\mathcal N} Q$ for some ${\mathcal N}$-barbed bisimulation ${\mathcal S}_{\mathcal N}$.
\end{definition}

$\mathcal{R} \subseteq \pi \times \pi$

$P \mathcal{R} Q => \forall P'. P \red P' \Rightarrow \exists Q'. Q \red Q', P' \mathcal{R} Q'$

$P \vdash x \Rightarrow Q \vdash x$

\begin{mathpar}
  \inferrule*[lab=Out-barb]{x \nameeq y}{{y}!\langle{Q}\rangle \vdash x}
  \and
  \inferrule*[lab=Par-barb]{\mbox{$P\vdash x$ or $Q\vdash x$}}{\binpar{P}{Q} \vdash x}
\end{mathpar}

\subsubsection{Contexts}

One of the principle advantages of computational calculi like the
$\pi$-calculus is a well-defined notion of context,
contextual-equivalence and a correlation between
contextual-equivalence and notions of bisimulation. The notion of
context allows the decomposition of a process into (sub-)process and
its syntactic environment, its context. Thus, a context may be
thought of as a process with a ``hole'' (written $\Box$) in it. The
application of a context $M$ to a process $P$, written $M[P]$, is
tantamount to filling the hole in $M$ with $P$. In this paper we do
not need the full weight of this theory, but do make use of the notion
of context in the proof the main theorem. 

\begin{mathpar}
  \inferrule* [lab=summation] {} {{M_{M},M_{N}} \bc \Box \;|\; x.M_{A} \;|\; M_{M}+M_{N}}
  \and
  \inferrule* [lab=agent] {} {{M_{A}} \bc (\vec{x})M_{P} \;| \; \clift{P_0,\ldots,M_{P},\ldots,P_N}}
  \and \\
  \inferrule* [lab=process] {} {{M_{P}} \bc M_{N} \;| \;P|M_{P} }
\end{mathpar} 

\begin{mathpar}
  \inferrule* [lab=sychronization] {} {M_{N} \bc \Box \;|\; x?M_{F} \;|\; x!M_{C}}
  \and
  \inferrule* [lab=abstraction] {} {{M_{F}} \bc (x)M_{P} }
  \and
  \inferrule* [lab=concretion] {} {{M_{C}} \bc \langle M_{P} \rangle }
  \and \\
  \inferrule* [lab=process] {} {{M_{P}} \bc M_{N} \;| \;P|M_{P} }
\end{mathpar}

\begin{definition}[contextual application] Given a context $M$, and
  process $P$, we define the \emph{contextual application}, $M[P] :=
  M\{P/\Box\}$. That is, the contextual application of M to P is the
  substitution of $P$ for $\Box$ in $M$.
\end{definition}

$\meaningof{-} : L \to \mathcal{P}(\pi)$

\begin{mathpar}
  \inferrule* [lab=collection] {} {\meaningof{true} = \pi, \and \meaningof{~E} = \pi \setminus \meaningof{E}, \and \meaningof{E_{1} \& E_{2}} = \meaningof{E_{1}} \cap \meaningof{E_{2}}}
\end{mathpar}

\begin{mathpar}
  \inferrule* [lab=structure] {} {\meaningof{0} = \{ P \in \pi | P \equiv 0 \}, \and \\ \meaningof{E_1 | E_2} = \{ P \in \pi | P \equiv P_{1} | P_{2}, P_{1} \in \meaningof{E_{1}}, P_{2} \in \meaningof{E_2}\} }
\end{mathpar}

\begin{mathpar}
 \inferrule* [lab=behavior] {} {\meaningof{\langle a?b \rangle E} = \{ P \in \pi | P \equiv Q | u?(y)P', \\ \and \\\\ \and \\ \;\;\; u \in \meaningof{a}, \forall z.P'\{z/y\} \in \meaningof{E\{z/b\}}\}, \and \\ \meaningof{a!E} = \{ P \in \pi | P \equiv Q | x!\langle P' \rangle, x \in \meaningof{a} P' \in \meaningof{E}\} }
\end{mathpar}

\begin{mathpar}
 \inferrule* [lab=nominal] {} {\meaningof{\quotep{E}} = \{ \quotep{P} \in \quotep{\pi} | P \in \meaningof{E} \}, \and \meaningof{\quotep{P}} = \{ \quotep{Q} \in \quotep{\pi} | P \equiv Q \} \and \\ \meaningof{@\quotep{E}} = \{ P \in \pi | P \equiv @x, x \in \meaningof{E} \}}
\end{mathpar}

\begin{eqnarray*}
  \\
  \meaningof{-} : TS \to ST
\end{eqnarray*}

\begin{eqnarray*}
  \\
  L : TS \to ST
\end{eqnarray*}

\begin{eqnarray*}
  \\
  P \models E \iff P \in \meaningof{E}
\end{eqnarray*}

\begin{eqnarray*}
  P \approx_{L} Q \iff \forall E \in L. P \models E \iff Q \models E
\end{eqnarray*}

\begin{eqnarray*}
  P \approx_{K} Q
\end{eqnarray*}

\begin{eqnarray*}
  P \approx Q
\end{eqnarray*}

$\approx_{K} = \approx = \approx_{L}$

\subsubsection{Contextual duality}

Note that contexts extend the quotation operation to a family of
operations from processes to names. Given a context, $M$, we can
define a \emph{nominal context}, $\quotep{M}$ by $\quotep{M}[P] :=
\quotep{M[P]}$. To foreshadow what is to come we observe that these
operations enjoy a duality with processes very much like the duality
between vectors and maps from vectors to scalars.

Further, because the calculus is essentially higher-order, we have a
correspondence between contexts and processes. More specifically,
given a name $x$ and a context $M$ we can construct $M^{*}_{x}$ such
that 

\begin{mathpar}
  M^{*}_{x} | \lift{x}{P} \red M[P]
\end{mathpar}

namely,

\begin{mathpar}
  M^{*}_{x} := x?(u).M[\dropn{u}]
\end{mathpar}

The dependence of $M^{*}_{x}$ on a name makes it an abstraction, 

\begin{mathpar}
  M^{*} := (x)x?(u).M[\dropn{u}]
\end{mathpar}

\subsection{Additional notation}

It will sometimes be convenient to denote the process a name
quotes. We already have the notation $x = \quotep{P}$, but it will be
convenient to introduce an alternate notation, $\procn{x}$, when we
want to emphasize the connection to the use of the name. Note that, by
virtue of name equivalence, $\quotep{\procn{x}} \nameeq x$; so, the
notation is consistent with previous definitions.

Further, because names have structure it is possible to effect
substitutions on the basis of that structure. This means we need to
upgrade our notation for substitutions, which we accomplish by
adapting comprehension notation. Thus,

\begin{mathpar}
  P\{ y / x : x \in S \}
\end{mathpar}

is interpreted to mean the process derived from P by replacing (in a
capture-avoiding manner) each occurrence of $x$ in $S$ by $y$. For example,

\begin{mathpar}
  P\{ \quotep{\procn{x}|\procn{x}} / x : x \in \freenames{P} \}
\end{mathpar}

will replace each (occurrence) of a free name $x$ in $P$ by
$\quotep{\procn{x}|\procn{x}}$.

Also, we will avail ourselves of the notation $x^{L}$ and $x^{R}$ to
denote injections of a name into disjoint copies of the name
space. There are numerous ways to accomplish this. One example can be
found in \cite{MeredithR05}. This notation overloads to vectors of
names: $\vec{x}^{\pi} := (x_{i}^{\pi} \; : \; 0 \leq i < |\vec{x}| )$ where $\pi \in \{L,R\}$.

We also use $P^{\Box} := P|\Box$.

In \cite{MeredithR05} an interpretation of the new operator is
given. It turns out that there are several possible interpretations
all enjoying the requisite algebraic properties of the operator (see
\cite{milner91polyadicpi}). We will therefore make liberal use of
$(\nu\; \vec{x})P$.

% subsection the_syntax_and_semantics_of_the_notation_system (end)   

\input{qm2pi.qmops} 

\input{qm2pi.sterngerlach} 

\input{qm2pi.metric} 

% section concurrent_process_calculi (end)

%\input{qm2pi.proofsketch}

% section proof sketch (end)

%\input{qm2pi.slviaknots} 

% section spatial logic via knots (end)

\input{qm2pi.conclusion}

% section conclusion (end)

%\input{qm2pi.dtcodes} 

% section wiring algorithm (end)

\input{qm2pi.ack} 

% section acknowledgments (end)

\newpage


\bibliographystyle{plain}   
\bibliography{../../biblios/main.bib}

\input{qm2pi.rhodetails}

\end{document}

 

% subsection basic_interpretation (end)

%\input{qm2pi.rho.presentation} 
\subsection{The syntax and semantics of the notation system}\label{sub:the_syntax_and_semantics_of_the_notation_system} % (fold)

We now summarize a technical presentation of the calculus that
embodies our theory of dynamics. The typical presentation of such a
calculus follows the style of giving generators and relations on
them. The grammar, below, describing term constructors, freely
generates the set of processes, $\Proc$. This set is then quotiented
by a relation known as structural congruence and it is over this set
that the notion of dynamics is expressed. This presentation is
essentially that of \cite{MeredithR05} with the addition of
polyadicity and summation. For readability we have relegated some of
the technical subtleties to an appendix.

\subsubsection{Process grammar}\label{subsub:process_grammar}

\begin{mathpar}
  \inferrule* [lab=synchronization] {} {{M} \bc \pzero \;|\; x?F \;|\; x!C }
  \and
  \inferrule* [lab=abstraction] {} {{F} \bc (x)P}
  \and
  \inferrule* [lab=concretion] {} {{C} \bc \langle Q \rangle}
  \and
  \inferrule* [lab=process] {} {{P,Q} \bc M \;| \;P|Q \;|\; @{x}}
  \and
  \inferrule* [lab=name] {} {{x} \bc \quotep{P}}
\end{mathpar} 

Note that $\vec{x}$ (resp. $\vec{P}$) denotes a vector of names
(resp. processes) of length $|\vec{x}|$ (resp. $|\vec{P}|$). We adopt
the following useful abbreviations.

\begin{mathpar}
   x?(\vec{y}).P := x.(\vec{y})P \and  x\clift{\vec{P}} := x.\clift{\vec{P}}
   \and x!(y) := \lift{x}{\dropn{y}}
   \and \Pi_{i=0}^{n-1}P_i := P_0 | \ldots | P_{n-1}
\end{mathpar}

\subsubsection{Structural congruence}

\paragraph{Free and bound names and alpha-equivalence.} At the
core of structural equivalence is alpha-equivalence which identifies
process that are the same up to a change of variable. Formally, we
recognize the distinction between free and bound names. The free names
of a process, $\freenames{P}$, may be calculated recursively as
follows:

\begin{mathpar}
\freenames{\pzero} := \emptyset
  \and \\
  \freenames{x?(y).P} := \{ x \} \cup (\freenames{P} \setminus \{ y \})
  \and 
  \freenames{x!\langle P \rangle} := \{ x \} \cup \{ P \} 
  \and \\
  \freenames{P|Q} := \freenames{P} \cup \freenames{Q}
  \and \\
  \freenames{@{x}} := \{ x \}
\end{mathpar}

$\pi$
$\quotep{\pi}$

$\freenames{-} : \pi \to \mathcal{P}(\quotep{\pi})$

\begin{eqnarray*}
  \freenames{\pzero} & := & \emptyset \\
  \freenames{x?(y).P} & := & \{ x \} \cup (\freenames{P} \setminus \{ y \}) \\
  \freenames{x!\langle P \rangle} & := & \{ x \} \cup \{ P \} \\
  \freenames{P|Q} & := & \freenames{P} \cup \freenames{Q} \\
  \freenames{\dropn{x}} & := & \{ x \}
\end{eqnarray*}

The bound names of a process, $\boundnames{P}$, are those names occurring in $P$
that are not free. For example, in $x?(y).0$, the name $x$ is free, while $y$ is bound.

\begin{mathpar}
  \inferrule* [lab=monoidal-laws] {} { P|Q \equiv Q|P \and P|0 \equiv P \and P|(Q|R) \equiv (P|Q)|R }
\end{mathpar}

\begin{mathpar}
  \inferrule* [lab=alpha-equivalence] {} { (x)P \equiv (y)P\{y/x\} \and y \not\in \freenames{P} }
\end{mathpar}

\begin{definition}
Then two processes, $P,Q$, are alpha-equivalent if $P = Q\{\vec{y}/\vec{x}\}$ for
some $\vec{x} \in \boundnames{Q},\vec{y} \in \boundnames{P}$, where $Q\{\vec{y}/\vec{x}\}$
denotes the capture-avoiding substitution of $\vec{y}$ for $\vec{x}$ in $Q$.
\end{definition}

\begin{definition}
  The {\em structural congruence} \cite{SangiorgiWalker} , $\equiv$,
  between processes is the least congruence containing
  alpha-equivalence, satisfying the abelian monoid laws
  (associativity, commutativity and $\pzero$ as identity) for parallel
  composition $|$ and for summation $+$.
\end{definition}

\subsection{Name equivalence}

We take name equivalence, written $\nameeq$, to be the smallest
equivalence relation generated by the following rules.

\begin{mathpar}
\inferrule*[lab=Quote-drop]
{ }
{ \quotep{@{x}} \nameeq x }

\inferrule*[lab=Struct-equiv]
{ P \scong Q }
{ \quotep{P} \nameeq \quotep{Q} }
\end{mathpar}

The astute reader will have noticed that the mutual recursion of names
and processes imposes a mutual recursion on alpha-equivalence and
structural equivalence via name-equivalence. Fortunately, all of this
works out pleasantly and we may calculate in the natural way, free of
concern. The reader interested in the details is referred to the
appendix \ref{appendix:rho_details}.

\subsection{Substitution}

We use $\Proc$ for the set of processes, $\QProc$ for the set of
names, and $\id{\{}\vec{y} / \vec{x} \id{\}}$ to denote partial maps,
$s : \QProc \rightarrow \QProc$. A map, $s$ lifts, uniquely, to a map
on process terms, $\widehat{s} : \Proc \rightarrow \Proc$ by the
following equations.

\begin{mathpar}
  (0) \psubstp{Q}{P} := 0 \\
  (R \juxtap S) \psubstp{Q}{P}
  :=    
  (R)\psubstp{Q}{P} \juxtap (S) \psubstp{Q}{P} \\
  (x?(y).R) \psubstp{Q}{P}    
  :=    
  (x)\substp{Q}{P} (z)\concat( (R \psubstn{z}{y}) \psubstp{Q}{P} ) \\
  (\lift{x}{R}) \psubstp{Q}{P}  
  :=
  \lift{(x)\substp{Q}{P}}{ R \psubstp{Q}{P} } \\
%   (\dropn{x})  \psubstp{Q}{P}       
%   := 
%   \left\{ 
%     \begin{array}{ccc} 
%       \dropn{\quotep{Q}} & & x \nameeq \quotep{P} \\
%       \dropn{x} & & otherwise \\
%     \end{array}
%   \right. 
  (\dropn{x})  \psubstp{Q}{P}       
  := 
  \left\{ 
    \begin{array}{ccc} 
      Q & & x \nameeq \quotep{P} \\
      \dropn{x} & & otherwise \\
    \end{array}
  \right.
\end{mathpar}
 

where

\begin{eqnarray}
  (x)\id{\{} \lpquote Q \rpquote / \lpquote P \rpquote \id{\}}            = 
  \left\{ 
    \begin{array}{ccc}
      \lpquote Q \rpquote & & x \nameeq \lpquote P \rpquote \\
      x & & otherwise \\
    \end{array}
  \right. \nonumber
\end{eqnarray}

and $z$ is chosen distinct from $\quotep{P}$, $\quotep{Q}$, the free
names in $Q$, and all the names in $R$. Our $\alpha$-equivalence will
be built in the standard way from this substitution.

\begin{remark}\label{rem:no_self_referential_names}
  One consequence of these definitions is that $\forall P. \quotep{P}
  \not\in \freenames{P}$.
\end{remark}

\subsection{ Dynamic quote: an example }

Anticipating something of what's to come, consider applying the
substitution, $\widehat{\id{\{}u / z \id{\}}}$, to the following pair
of processes, $\lift{w}{y!(z)}$ and $w[ \lpquote y!(z) \rpquote ]$.

\begin{eqnarray}
	\lift{w}{y!(z)}\widehat{\id{\{}u / z \id{\}}}
		& = &
		\lift{w}{y!(u)} \nonumber\\
	w[ \lpquote y!(z) \rpquote ] \widehat{ \id{\{}u / z \id{\}} }
		& = &
		w[ \lpquote y!(z) \rpquote ] \nonumber
\end{eqnarray}

Because the body of the process between quotes is impervious to
substitution, we get radically different answers. In fact, by
examining the first process in an input context,
e.g. $x?(z).\lift{w}{y!(z)}$, we see that the process under the lift
operator may be shaped by prefixed inputs binding a name inside it. In
this sense, the lift operator will be seen as a way to dynamically
construct processes before reifying them as names.

Finally equipped with these standard features we can present the
dynamics of the calculus.

\subsubsection{Operational semantics} 

Finally, we introduce the computational dynamics. What marks these
algebras as distinct from other more traditionally studied algebraic
structures, e.g. vector spaces or polynomial rings, is the manner in
which dynamics is captured. In traditional structures, dynamics is typically
expressed through morphisms between such structures, as in linear maps
between vector spaces or morphisms between rings. In algebras
associated with the semantics of computation, the dynamics is
expressed as part of the algebraic structure itself, through a
reduction reduction relation typically denoted by $\red$. Below, we
give a recursive presentation of this relation for the calculus used
in the encoding.

$\red \subseteq \pi \times \pi$
$\red : \pi \to \mathcal{P}(\pi)$

\begin{mathpar}
  \inferrule* [lab=Comm] { \textsf{match}( x_{src}, x_{trgt} ) } { x_{trgt}?(y)P \; | \; x_{src}!\langle {Q} \rangle \red P\{\quotep{Q}/y}\} }
  \and \\
  \inferrule* [lab=Par] {{P} \red {P}'} {{{P} | {Q}} \red {{P}' | {Q}}}
  \and
  \inferrule* [lab=Equiv]{{{P} \scong {P}'} \andalso {{P}' \red {Q}'} \andalso {{Q}' \scong {Q}}}{{P} \red {Q}}
\end{mathpar}

\begin{eqnarray*}
  match_{\equiv} (\quotep{P},\quotep{Q}) & := & P \equiv Q \\
  match_{\dagger}(\quotep{P},\quotep{Q}) & := & \forall R. P|Q \red^{*} R => R \red^{*} 0 \\
  match_{K}(\quotep{P},\quotep{Q}) & := & K \mbox{ for some context } K
\end{eqnarray*}

$u?(x)P | u!\langle Q \rangle \red P\{\quotep{Q}/x\}$

%We write $\wred$ for $\red^*$, and $P\red$ if $\exists Q $ such that $ P \red Q$.
We write $P\red$ if $\exists Q $ such that $ P \red Q$ and $P\not\red$, otherwise.

\section{Replication}

As mentioned before, it is known that replication (and hence
recursion) can be implemented in a higher-order process algebra
\cite{SangiorgiWalker}. As our first example of calculation with the
machinery thus far presented we give the construction explicitly in
the {\rhoc}.

\begin{eqnarray}
	D_{x} & := & \prefix{x}{y}{(\binpar{\outputp{x}{y}}{@{y}})} \nonumber\\
	\bangp_{x}{P} & := & \binpar{{x}!\langle{\binpar{D_{x}}{P}}\rangle}{D_{x}} \nonumber
\end{eqnarray}

\begin{eqnarray}
	\bangp_{x}{P} & & \nonumber\\
	=
	& {x}!\langle{(\prefix{x}{y}{(\outputp{x}{y} | @{y})) | P}}\rangle 
	      | \prefix{x}{y}{(\outputp{x}{y} | @{y})} & \nonumber\\
	\red
	& (\outputp{x}{y} | @{y})\substn{\quotep{(\prefix{x}{y}{(@{y} | \outputp{x}{y})) | P}}}{y} & \nonumber\\
	=
	& \outputp{x}{\quotep{(\prefix{x}{y}{(\outputp{x}{y} | @{y})) | P}}}
	  | {(\prefix{x}{y}{(\outputp{x}{y} | @{y})) | P}} & \nonumber\\
	\red
	& \ldots & \nonumber\\
	\red^*
	& P | P | \ldots & \nonumber
\end{eqnarray}

Of course, this encoding, as an implementation, runs away, unfolding
$\bangp{P}$ eagerly. A lazier and more implementable replication
operator, restricted to input-guarded processes, may be obtained as follows.

\begin{eqnarray}
\bangp{\prefix{u}{v}{P}} 
	:= 
	\binpar{\lift{x}{\prefix{u}{v}{(\binpar{D(x)}{P})}}}{D(x)} \nonumber
\end{eqnarray}

\begin{remark}
  Note that the lazier definition still does not deal with summation
  or mixed summation (i.e. sums over input and output). The reader is
  invited to construct definitions of replication that deal with these
  features. 

  Further, the definitions are parameterized in a name, $x$. Can you,
  gentle reader, make a definition that eliminates this parameter and
  guarantees no accidental interaction between the replication
  machinery and the process being replicated -- i.e. no accidental
  sharing of names used by the process to get its work done and the
  name(s) used by the replication to effect copying. This latter
  revision of the definition of replication is crucial to obtaining
  the expected identity $!!P \sim !P$.
\end{remark}

\begin{remark}\label{rem:paradoxical_combinator}
  The reader familiar with the lambda calculus will have noticed the
  similarity between $D$ and the paradoxical combinator.

  [Ed. note: the existence of this seems to suggest we have to be more
  restrictive on the set of processes and names we admit if we are to
  support no-cloning.]
\end{remark}

\subsubsection{Bisimulation}

The computational dynamics gives rise to another kind of equivalence,
the equivalence of computational behavior. As previously mentioned
this is typically captured \emph{via} some form of bisimulation.

% The notion we use in this paper is weak barbed bisimulation
% \cite{milner91polyadicpi}.

The notion we use in this paper is derived from weak barbed
bisimulation \cite{milner91polyadicpi}. 

\begin{definition}
An \emph{observation relation}, $\downarrow_{\mathcal N}$, over a set
of names, $\mathcal N$, is the smallest relation satisfying the rules
below.

\infrule[Out-barb]{y \in {\mathcal N}, \; x \nameeq y}
		  {\outputp{x}{v} \downarrow_{\mathcal N} x}
\infrule[Par-barb]{\mbox{$P\downarrow_{\mathcal N} x$ or $Q\downarrow_{\mathcal N} x$}}
		  {\binpar{P}{Q} \downarrow_{\mathcal N} x}

We write $P \Downarrow_{\mathcal N} x$ if there is $Q$ such that 
$P \wred Q$ and $Q \downarrow_{\mathcal N} x$.
\end{definition}

\begin{definition}
%\label{def.bbisim}
An  ${\mathcal N}$-\emph{barbed bisimulation} over a set of names, ${\mathcal N}$, is a symmetric binary relation 
${\mathcal S}_{\mathcal N}$ between agents such that $P\rel{S}_{\mathcal N}Q$ implies:
\begin{enumerate}
\item If $P \red P'$ then $Q \wred Q'$ and $P'\rel{S}_{\mathcal N} Q'$.
\item If $P\downarrow_{\mathcal N} x$, then $Q\Downarrow_{\mathcal N} x$.
\end{enumerate}
$P$ is ${\mathcal N}$-barbed bisimilar to $Q$, written
$P \wbbisim_{\mathcal N} Q$, if $P \rel{S}_{\mathcal N} Q$ for some ${\mathcal N}$-barbed bisimulation ${\mathcal S}_{\mathcal N}$.
\end{definition}

$\mathcal{R} \subseteq \pi \times \pi$

$P \mathcal{R} Q => \forall P'. P \red P' \Rightarrow \exists Q'. Q \red Q', P' \mathcal{R} Q'$

$P \vdash x \Rightarrow Q \vdash x$

\begin{mathpar}
  \inferrule*[lab=Out-barb]{x \nameeq y}{{y}!\langle{Q}\rangle \vdash x}
  \and
  \inferrule*[lab=Par-barb]{\mbox{$P\vdash x$ or $Q\vdash x$}}{\binpar{P}{Q} \vdash x}
\end{mathpar}

\subsubsection{Contexts}

One of the principle advantages of computational calculi like the
$\pi$-calculus is a well-defined notion of context,
contextual-equivalence and a correlation between
contextual-equivalence and notions of bisimulation. The notion of
context allows the decomposition of a process into (sub-)process and
its syntactic environment, its context. Thus, a context may be
thought of as a process with a ``hole'' (written $\Box$) in it. The
application of a context $M$ to a process $P$, written $M[P]$, is
tantamount to filling the hole in $M$ with $P$. In this paper we do
not need the full weight of this theory, but do make use of the notion
of context in the proof the main theorem. 

\begin{mathpar}
  \inferrule* [lab=summation] {} {{M_{M},M_{N}} \bc \Box \;|\; x.M_{A} \;|\; M_{M}+M_{N}}
  \and
  \inferrule* [lab=agent] {} {{M_{A}} \bc (\vec{x})M_{P} \;| \; \clift{P_0,\ldots,M_{P},\ldots,P_N}}
  \and \\
  \inferrule* [lab=process] {} {{M_{P}} \bc M_{N} \;| \;P|M_{P} }
\end{mathpar} 

\begin{mathpar}
  \inferrule* [lab=sychronization] {} {M_{N} \bc \Box \;|\; x?M_{F} \;|\; x!M_{C}}
  \and
  \inferrule* [lab=abstraction] {} {{M_{F}} \bc (x)M_{P} }
  \and
  \inferrule* [lab=concretion] {} {{M_{C}} \bc \langle M_{P} \rangle }
  \and \\
  \inferrule* [lab=process] {} {{M_{P}} \bc M_{N} \;| \;P|M_{P} }
\end{mathpar}

\begin{definition}[contextual application] Given a context $M$, and
  process $P$, we define the \emph{contextual application}, $M[P] :=
  M\{P/\Box\}$. That is, the contextual application of M to P is the
  substitution of $P$ for $\Box$ in $M$.
\end{definition}

$\meaningof{-} : L \to \mathcal{P}(\pi)$

\begin{mathpar}
  \inferrule* [lab=collection] {} {\meaningof{true} = \pi, \and \meaningof{~E} = \pi \setminus \meaningof{E}, \and \meaningof{E_{1} \& E_{2}} = \meaningof{E_{1}} \cap \meaningof{E_{2}}}
\end{mathpar}

\begin{mathpar}
  \inferrule* [lab=structure] {} {\meaningof{0} = \{ P \in \pi | P \equiv 0 \}, \and \\ \meaningof{E_1 | E_2} = \{ P \in \pi | P \equiv P_{1} | P_{2}, P_{1} \in \meaningof{E_{1}}, P_{2} \in \meaningof{E_2}\} }
\end{mathpar}

\begin{mathpar}
 \inferrule* [lab=behavior] {} {\meaningof{\langle a?b \rangle E} = \{ P \in \pi | P \equiv Q | u?(y)P', \\ \and \\\\ \and \\ \;\;\; u \in \meaningof{a}, \forall z.P'\{z/y\} \in \meaningof{E\{z/b\}}\}, \and \\ \meaningof{a!E} = \{ P \in \pi | P \equiv Q | x!\langle P' \rangle, x \in \meaningof{a} P' \in \meaningof{E}\} }
\end{mathpar}

\begin{mathpar}
 \inferrule* [lab=nominal] {} {\meaningof{\quotep{E}} = \{ \quotep{P} \in \quotep{\pi} | P \in \meaningof{E} \}, \and \meaningof{\quotep{P}} = \{ \quotep{Q} \in \quotep{\pi} | P \equiv Q \} \and \\ \meaningof{@\quotep{E}} = \{ P \in \pi | P \equiv @x, x \in \meaningof{E} \}}
\end{mathpar}

\begin{eqnarray*}
  \\
  \meaningof{-} : TS \to ST
\end{eqnarray*}

\begin{eqnarray*}
  \\
  L : TS \to ST
\end{eqnarray*}

\begin{eqnarray*}
  \\
  P \models E \iff P \in \meaningof{E}
\end{eqnarray*}

\begin{eqnarray*}
  P \approx_{L} Q \iff \forall E \in L. P \models E \iff Q \models E
\end{eqnarray*}

\begin{eqnarray*}
  P \approx_{K} Q
\end{eqnarray*}

\begin{eqnarray*}
  P \approx Q
\end{eqnarray*}

$\approx_{K} = \approx = \approx_{L}$

\subsubsection{Contextual duality}

Note that contexts extend the quotation operation to a family of
operations from processes to names. Given a context, $M$, we can
define a \emph{nominal context}, $\quotep{M}$ by $\quotep{M}[P] :=
\quotep{M[P]}$. To foreshadow what is to come we observe that these
operations enjoy a duality with processes very much like the duality
between vectors and maps from vectors to scalars.

Further, because the calculus is essentially higher-order, we have a
correspondence between contexts and processes. More specifically,
given a name $x$ and a context $M$ we can construct $M^{*}_{x}$ such
that 

\begin{mathpar}
  M^{*}_{x} | \lift{x}{P} \red M[P]
\end{mathpar}

namely,

\begin{mathpar}
  M^{*}_{x} := x?(u).M[\dropn{u}]
\end{mathpar}

The dependence of $M^{*}_{x}$ on a name makes it an abstraction, 

\begin{mathpar}
  M^{*} := (x)x?(u).M[\dropn{u}]
\end{mathpar}

\subsection{Additional notation}

It will sometimes be convenient to denote the process a name
quotes. We already have the notation $x = \quotep{P}$, but it will be
convenient to introduce an alternate notation, $\procn{x}$, when we
want to emphasize the connection to the use of the name. Note that, by
virtue of name equivalence, $\quotep{\procn{x}} \nameeq x$; so, the
notation is consistent with previous definitions.

Further, because names have structure it is possible to effect
substitutions on the basis of that structure. This means we need to
upgrade our notation for substitutions, which we accomplish by
adapting comprehension notation. Thus,

\begin{mathpar}
  P\{ y / x : x \in S \}
\end{mathpar}

is interpreted to mean the process derived from P by replacing (in a
capture-avoiding manner) each occurrence of $x$ in $S$ by $y$. For example,

\begin{mathpar}
  P\{ \quotep{\procn{x}|\procn{x}} / x : x \in \freenames{P} \}
\end{mathpar}

will replace each (occurrence) of a free name $x$ in $P$ by
$\quotep{\procn{x}|\procn{x}}$.

Also, we will avail ourselves of the notation $x^{L}$ and $x^{R}$ to
denote injections of a name into disjoint copies of the name
space. There are numerous ways to accomplish this. One example can be
found in \cite{MeredithR05}. This notation overloads to vectors of
names: $\vec{x}^{\pi} := (x_{i}^{\pi} \; : \; 0 \leq i < |\vec{x}| )$ where $\pi \in \{L,R\}$.

We also use $P^{\Box} := P|\Box$.

In \cite{MeredithR05} an interpretation of the new operator is
given. It turns out that there are several possible interpretations
all enjoying the requisite algebraic properties of the operator (see
\cite{milner91polyadicpi}). We will therefore make liberal use of
$(\nu\; \vec{x})P$.

% subsection the_syntax_and_semantics_of_the_notation_system (end)   

\section{Interpretation of QM}
\subsection{Supporting definitions}
\subsubsection{Multiplication}
\begin{mathpar}
  \quotep{Q} \cdot \quotep{R} := \quotep{Q|R}
  \and \\
  \quotep{Q} \cdot P := P\{ \quotep{Q|R} / \quotep{R} : \quotep{R} \in \freenames{P} \}
\end{mathpar}

\paragraph{Discussion}
The first line needs little explanation. The second line says that
each free name of the process is replaced with the multiplication of
that name by the scalar. Multiplication of a scalar (name) by a state
(process) results in a process all the names of which have been `moved
over' by parallel composition with the process the scalar
quotes. There is a subtlety that the bound names have to be
manipulated so that multiplied names aren't accidentally
captured. There are many ways to achieve this.

\begin{remark}\label{rem:multiplication_identities}
  The reader is invited to verify that for all $x,y,z \in \QProc$ and $P \in \Proc$
  \begin{mathpar}
    x \cdot \quotep{0} \equiv x 
    \and
    x \cdot y \equiv y \cdot x
    \and
    x \cdot (y \cdot z) \equiv (x \cdot y) \cdot z
    \and \\
    \quotep{0} \cdot P \equiv P
    \and \\
    x \cdot (y \cdot P) \equiv (x \cdot y) \cdot P
    \and \\
    x \cdot (P|Q) \equiv (x \cdot P) | (x \cdot Q)
    \and \\    
  \end{mathpar}
\end{remark}

\subsubsection{Tensor product}

We define a tensor product on processes by structural induction.

\paragraph{Tensor of sums} First note that all summations, including
$\pzero$ and sequence, can be written $\Sigma_{i} x_{i}.A_{i} +
\Sigma_{j} x_{j}.C_{j}$, where we have grouped input-guarded processes
together and output-guarded processes together.

Thus, we can define the tensor product of two summations, $N_{1}\otimes N_{2}$, where

\begin{mathpar}
  N_{1} := \Sigma_{i} x_{i}.A_{i} + \Sigma_{j} x_{j}.C_{j}
  \and
  N_{2} := \Sigma_{i'} y_{i'}.B_{i'} + \Sigma_{j'} y_{j'}.D_{j'} 
\end{mathpar}

as follows.

\begin{mathpar}
  \Sigma_{i} x_{i}.A_{i} + \Sigma_{j} x_{j}.C_{j} \otimes \Sigma_{i'}
  y_{i'}.B_{i'} + \Sigma_{j'} y_{j'}.D_{j'} 
  \and \\
  := \; \Sigma_{i} \Sigma_{i'} \quotep{\stackrel{\vee}{x_{i}}| \stackrel{\vee}{y_{i'}}}.(A_{i}\otimes B_{i'}) \; | \; \Sigma_{i'} \Sigma_{i} \quotep{\stackrel{\vee}{y_{i'}}|\stackrel{\vee}{x_{i}}}.(B_{i'}\otimes A_{i})
  \and
  \;\; | \;\; \Sigma_{j} \Sigma_{j'} \quotep{\stackrel{\vee}{x_{j}}|\stackrel{\vee}{y_{j'}}}.(A_{j}\otimes B_{j'}) \; | \; \Sigma_{j'} \Sigma_{j} \quotep{\stackrel{\vee}{y_{j'}}|\stackrel{\vee}{x_{j}}}.(B_{j'}\otimes A_{j})
\end{mathpar}

\begin{remark}
  Do we need to $x^{L}$ and $y^{R}$ for this construction as well?
\end{remark}

\paragraph{Tensor of parallel compositions} Next, we distribute tensor
over par.

\begin{mathpar}
  P_{1}|P_{2} \otimes Q_{1}|Q_{2} := (P_{1} \otimes Q_{1}) | (P_{1}
  \otimes Q_{2}) | (P_{2} \otimes Q_{1}) | (P_{2} \otimes Q_{2})
\end{mathpar}

\paragraph{Tensor with dropped names} We treat tensor of a
process with a dropped name as parallel composition.

\begin{mathpar}
  P \otimes \dropn{x} := P | \dropn{x}
\end{mathpar}

\paragraph{Tensor of agents}

Finally, we need to define tensor on agents. Note that the definition
of tensor on normal products only tensors inputs with inputs and
outputs with outputs. Thus, we only have to define the operation on
``homogeneous'' pairings.

\begin{mathpar}
  (\vec{x})P \otimes (\vec{y})Q
  \and \\
  := (x_{0}^{L}|y_{0}^{R},\ldots,x_{0}^{L}|y_{n}^{R},\ldots,x_{m}^{L}|y_{0}^{R},\ldots,x_{m}^{L}|y_{n}^R)(P\{ \vec{x}^{L}/\vec{x}\} \otimes Q \{ \vec{y}^{R}/\vec{y}\})
  \and \\
  \clift{\vec{P}} \otimes \clift{\vec{Q}}
  \and \\
  := \clift{P_{0}\otimes Q_{0},\ldots,P_{0}\otimes Q_{n},\ldots,P_{m}\otimes Q_{0},\ldots,P_{m}\otimes Q_{n}}
\end{mathpar}

\begin{remark}
  Observe that arities of tensored abstractions matches arities of
  tensored concretions if the original arities matched. Note also that
  the length of the arities corresponds to the increase in dimension
  we see in ordinary vector space tensor product.
\end{remark}

\begin{remark}
  Operationally, this definition distributes the tensor down to
  components ``linked'' by summation. Tensor over summation is
  intriguing in that it mixes names. Moreover, as a consequence of the
  way it mixes names we have the identities for all $x \in \QProc$ and
  $P,Q \in \Proc$

  \begin{mathpar}
    (x \cdot P) \otimes Q \equiv x \cdot (P \otimes Q) \equiv P \otimes (x \cdot Q)
    \and
    P \otimes \pzero \equiv P
  \end{mathpar}

  that the reader is invited to verify.
\end{remark}

\subsubsection{Annihilation}
\begin{mathpar}
  P^{\perp} := \{ Q | \forall R. P|Q \red^{*} R \Rightarrow R \red^{*} \pzero \}
  \and \\
  P^{\underline{\perp}} := \Sigma_{Q \in P^{\perp}} \quotep{Q}?(y).(\dropn{y}|Q) | \Sigma_{Q \in P^{\perp}} \quotep{Q}\clift{\Box}
\end{mathpar}

\paragraph{Discussion} The reader will note that $P^{\perp}$ is a
\emph{set} of processes, while $P^{\underline{\perp}}$ is a
\emph{context}. We call the set $P^{\perp}$ the \emph{annihilators} of
$P$. The parallel composition of a process in the annihilators of $P$
with $P$ will result in a process, the state space of which has all
paths eventually leading to $\pzero$. Execution may endure loops; but
under reasonable conditions of fairness (naturally guaranteed under
most notions of bisimulation) such a composite process cannot get
stuck in such a loop and will, eventually pop out and terminate.

The context $P^{\underline{\perp}}$ is ready and willing to ``take the
$P$ out of'' the process to which it is applied. It will effectively
transmit the code of the process to which it is applied to one of the
annihilators and run the process against it.

\subsubsection{Evaluation}
We fix $M$ a domain of fully abstract interpretation with an equality
coincident with bisimulation. We take $\meaningof{\cdot} : \Proc \to
M$ to be the map interpreting processes and $\nmeaningof{\cdot} : \M
\to Proc$ to be the map running the other way. Then we define

\begin{mathpar}
  \int P := \nmeaningof{\meaningof{P}}
\end{mathpar}

\paragraph{Discussion}
There are many fully abstract interpretations of Milner's
$\pi$-calculus. Any of them can be used as a basis for interpreting
the reflective calculus here. Equipped with such a domain it is
largely a matter of grinding through to check that the Yoneda
construction for the normalization-by-evaluation program can be
extended to this setting.

\begin{remark}
  The reader is invited to verify that $\int (P^{\underline{\perp}}[P]) = 0$.
\end{remark}

\subsection{Quantum mechanics}

Table \ref{tbl:core_qm_op_defns} gives the core operational definitions

\begin{table}[htp]\label{tbl:core_qm_op_defns}
  \center{
    \fbox{
      \begin{tabular}{c|c}
        quantum mechanics & process calculus \\
        \hline
        scalar & $x := \quotep{P}$ \\
        state vector & $\state{P} := P$ \\
        dual & $\state{P}^{*} := \event{P^{\underline{\perp}}} := \quotep{P^{\underline{\perp}}}[-]$ \\
        matrix & $ \Sigma_{\alpha} \state{P_{\alpha}}x_{\alpha}\event{Q_{\alpha}}$ \\
        vector addition & $\state{P} + \state{Q} := \state{P | Q}$ \\
        tensor product & $\state{P} \otimes \state{Q} := \state{P \otimes Q}$ \\
        inner product & $\innerprod{P}{Q} := \quotep{\int P^{\underline{\perp}}[Q]}$ \\
      \end{tabular}
    }
  }
  \caption{QM - operational definitions}
\end{table}

where

\begin{mathpar}
  \prmatrix{P}{Q} := \fprmatrix{P}{\quotep{\pzero}}{Q}
  \and
  \fprmatrix{P}{x}{Q} := (\state{P},x,\event{Q})
  \and
  (\fprmatrix{P}{x}{Q})(\state{R}) := x \cdot \innerprod{Q}{R} \cdot \state{P}
  \and
  (\fprmatrix{P}{x}{Q})(\event{R}) := x \cdot \innerprod{R}{P} \cdot \event{Q}
\end{mathpar}

\paragraph{Discussion}
As promised: vectors (aka states) are represented as processes; duals
as contextual duals; inner product definition should be compared with
standard inner product definition for ....

\begin{remark}
  Assuming $\int (P^{\underline{\perp}}[P]) = 0$, the reader is
  invited to verify that $(\fprmatrix{P}{x}{P})(\state{P}) = x \cdot \state{P}$.
\end{remark}

\begin{remark}
  The reader is invited to verify that $\innerprod{P}{Q}$ could
  equally well have been written $\quotep{\int \stackrel{\vee}{x}}$
  where $x = \event{P^{\underline{\perp}}}(Q)$.

  One of the motivations for this remark is that there is another way
  to factor these operations. We could package up evaluation in the dual:

  \begin{mathpar}
    \state{P}^{*} := \event{\int P^{\underline{\perp}}} := \quotep{\int P^{\underline{\perp}}}[-]
  \end{mathpar}

  and then have inner product defined by
  
  \begin{mathpar}
    \innerprod{P}{Q} := \event{P}(Q)
  \end{mathpar}

  Hopefully, experience with the calculations will provide guidance on
  the best factoring.
\end{remark}

\begin{remark}
  Assuming $\int (P^{\underline{\perp}}[P]) = 0$, the reader is
  invited to verify that $\forall P,Q. (\prmatrix{0}{Q})(\state{0}) =
  \state{0}$ and dually $(\prmatrix{P}{0})(\event{0}) = \event{0}$.
\end{remark}

\begin{remark}
  i'm a little worried that i don't (yet) have proper support for
  complex conjugacy. But, the observation above may give us a
  clue. According to Abramsky, it must be the case that the scalars
  are iso to the homset of the identity for the tensor -- which the
  observation above characterizes. 

  For now, we will simply bookmark the notion with $\overline{x}$.
\end{remark}

\subsubsection{Adjointness}

We need to give a definition of $(\cdot)^{\dagger}$ for matrices. The
obvious candidate definition is
\begin{mathpar}
(\Sigma_{\alpha}\fprmatrix{P_{\alpha}}{x_{\alpha}}{Q_{\alpha}})^{\dagger}
= \Sigma_{\alpha}\fprmatrix{(Q_{\alpha}^{\underline{\perp}})^{*}}{\overline{x}_{\alpha}}{P_{\alpha}^{\underline{\perp}}} 
\end{mathpar}

But, $(Q_{\alpha}^{\underline{\perp}})^{*}$ requires a name along
which to communicate the process to achieve the context application.

\subsubsection{Basis for a basis}
If processes label states and ``addition'' of states (a.k.a. vector
addition) is interpreted as parallel composition, what corresponds to
notions of linear independence and basis? Here, we recall that Yoshida
has developed a set of \emph{combinators} for an asynchronous verison
of Milner's $\pi$-calculus. These are a finite set of processes such
any process can be expressed as parallel composition of these
combinators together with liberal uses of the new operator and
replication. We can simply give a translation of these into the
present calculus and have reasonable expectation that the property
carries over. That is, that the resultant set allows to express all
processes via parallel composition. Note, however, that there is no
new operator or replication in this calculus. As a result, we expect
that the corresponding set is actually infinite. That is, we expect
that the space is actually infinite dimensional.

\begin{remark}
  The attentive reader may be a bit concerned. Certainly, the
  collection $S$, $K$ and $I$ is a finite set of
  combinators. Shouldn't we expect to see a finite set of combinators
  for an effectively equivalent system? i am very sympathetic to this
  critique and feel it warrants full attention. On the other hand, i
  also have in mind the following analogy. The natural numbers, as a
  monoid under addition, has exactly $1$ generator, while the natural
  numbers, as a monoid under multiplication, has countably many
  generators (the primes). We observe that the application of the
  lambda calculus is much less resource sensitive than the parallel
  composition of the $\pi$-calculus. Could it be the case that we have
  an analogy of the form
  
  \begin{mathpar}
    m + n : MN :: m*n : M|N
  \end{mathpar}

  giving a similar blow up in the set of ``primes''?  This is such a
  wonderful thought that, even if it's not true, i think it's worth
  writing down.
\end{remark}
 

\documentclass[12pt]{llncs}
%\documentclass{jktr}

\usepackage[pdftex]{hyperref}                   
\usepackage {listings}
\usepackage {mathpartir}
\usepackage{bcprules}
%\usepackage{listings}
                       
\usepackage{graphicx} 
%\usepackage[margins=2.5cm,nohead,nofoot]{geometry}
%\usepackage{geometry}
\usepackage{amsfonts}
\usepackage{amstext}
\usepackage{latexsym}
\usepackage{amssymb}
\usepackage{color}


%\include{myPreamble}
\include{qm2pi.local} 

%\ifpdf
%\usepackage[pdftex]{graphicx}
%\else
%\usepackage{graphicx}
%\fi

 % \ifpdf
%  \usepackage{pdfsync}
%  \if


%\title{Brief Article}
%\author{David F. Snyder}
%\author{L.G. Meredith}

%\address{Dept. of Math., Texas State University--San Marcos, San Marcos, TX 78666}
       
\pagestyle{empty}


\begin{document}

\lstset{language=[Objective]Caml,frame=shadowbox}

\input{qm2pi.front}

% section front matter (end)

\input{qm2pi.intro} 
 
% section introduction (end)

% \input{qm2pi.knotations} 

% section notation (end)

\input{qm2pi.process.calculi} 

% section concurrent_process_calculi_and_spatial_logics_ (end)
    
%\input{qm2pi.knots2pi} 

%\input{qm2pi.trefoil} 

%\input{qm2pi.mainthm} 

% subsection basic_interpretation (end)

%\input{qm2pi.rho.presentation} 
\subsection{The syntax and semantics of the notation system}\label{sub:the_syntax_and_semantics_of_the_notation_system} % (fold)

We now summarize a technical presentation of the calculus that
embodies our theory of dynamics. The typical presentation of such a
calculus follows the style of giving generators and relations on
them. The grammar, below, describing term constructors, freely
generates the set of processes, $\Proc$. This set is then quotiented
by a relation known as structural congruence and it is over this set
that the notion of dynamics is expressed. This presentation is
essentially that of \cite{MeredithR05} with the addition of
polyadicity and summation. For readability we have relegated some of
the technical subtleties to an appendix.

\subsubsection{Process grammar}\label{subsub:process_grammar}

\begin{mathpar}
  \inferrule* [lab=synchronization] {} {{M} \bc \pzero \;|\; x?F \;|\; x!C }
  \and
  \inferrule* [lab=abstraction] {} {{F} \bc (x)P}
  \and
  \inferrule* [lab=concretion] {} {{C} \bc \langle Q \rangle}
  \and
  \inferrule* [lab=process] {} {{P,Q} \bc M \;| \;P|Q \;|\; @{x}}
  \and
  \inferrule* [lab=name] {} {{x} \bc \quotep{P}}
\end{mathpar} 

Note that $\vec{x}$ (resp. $\vec{P}$) denotes a vector of names
(resp. processes) of length $|\vec{x}|$ (resp. $|\vec{P}|$). We adopt
the following useful abbreviations.

\begin{mathpar}
   x?(\vec{y}).P := x.(\vec{y})P \and  x\clift{\vec{P}} := x.\clift{\vec{P}}
   \and x!(y) := \lift{x}{\dropn{y}}
   \and \Pi_{i=0}^{n-1}P_i := P_0 | \ldots | P_{n-1}
\end{mathpar}

\subsubsection{Structural congruence}

\paragraph{Free and bound names and alpha-equivalence.} At the
core of structural equivalence is alpha-equivalence which identifies
process that are the same up to a change of variable. Formally, we
recognize the distinction between free and bound names. The free names
of a process, $\freenames{P}$, may be calculated recursively as
follows:

\begin{mathpar}
\freenames{\pzero} := \emptyset
  \and \\
  \freenames{x?(y).P} := \{ x \} \cup (\freenames{P} \setminus \{ y \})
  \and 
  \freenames{x!\langle P \rangle} := \{ x \} \cup \{ P \} 
  \and \\
  \freenames{P|Q} := \freenames{P} \cup \freenames{Q}
  \and \\
  \freenames{@{x}} := \{ x \}
\end{mathpar}

$\pi$
$\quotep{\pi}$

$\freenames{-} : \pi \to \mathcal{P}(\quotep{\pi})$

\begin{eqnarray*}
  \freenames{\pzero} & := & \emptyset \\
  \freenames{x?(y).P} & := & \{ x \} \cup (\freenames{P} \setminus \{ y \}) \\
  \freenames{x!\langle P \rangle} & := & \{ x \} \cup \{ P \} \\
  \freenames{P|Q} & := & \freenames{P} \cup \freenames{Q} \\
  \freenames{\dropn{x}} & := & \{ x \}
\end{eqnarray*}

The bound names of a process, $\boundnames{P}$, are those names occurring in $P$
that are not free. For example, in $x?(y).0$, the name $x$ is free, while $y$ is bound.

\begin{mathpar}
  \inferrule* [lab=monoidal-laws] {} { P|Q \equiv Q|P \and P|0 \equiv P \and P|(Q|R) \equiv (P|Q)|R }
\end{mathpar}

\begin{mathpar}
  \inferrule* [lab=alpha-equivalence] {} { (x)P \equiv (y)P\{y/x\} \and y \not\in \freenames{P} }
\end{mathpar}

\begin{definition}
Then two processes, $P,Q$, are alpha-equivalent if $P = Q\{\vec{y}/\vec{x}\}$ for
some $\vec{x} \in \boundnames{Q},\vec{y} \in \boundnames{P}$, where $Q\{\vec{y}/\vec{x}\}$
denotes the capture-avoiding substitution of $\vec{y}$ for $\vec{x}$ in $Q$.
\end{definition}

\begin{definition}
  The {\em structural congruence} \cite{SangiorgiWalker} , $\equiv$,
  between processes is the least congruence containing
  alpha-equivalence, satisfying the abelian monoid laws
  (associativity, commutativity and $\pzero$ as identity) for parallel
  composition $|$ and for summation $+$.
\end{definition}

\subsection{Name equivalence}

We take name equivalence, written $\nameeq$, to be the smallest
equivalence relation generated by the following rules.

\begin{mathpar}
\inferrule*[lab=Quote-drop]
{ }
{ \quotep{@{x}} \nameeq x }

\inferrule*[lab=Struct-equiv]
{ P \scong Q }
{ \quotep{P} \nameeq \quotep{Q} }
\end{mathpar}

The astute reader will have noticed that the mutual recursion of names
and processes imposes a mutual recursion on alpha-equivalence and
structural equivalence via name-equivalence. Fortunately, all of this
works out pleasantly and we may calculate in the natural way, free of
concern. The reader interested in the details is referred to the
appendix \ref{appendix:rho_details}.

\subsection{Substitution}

We use $\Proc$ for the set of processes, $\QProc$ for the set of
names, and $\id{\{}\vec{y} / \vec{x} \id{\}}$ to denote partial maps,
$s : \QProc \rightarrow \QProc$. A map, $s$ lifts, uniquely, to a map
on process terms, $\widehat{s} : \Proc \rightarrow \Proc$ by the
following equations.

\begin{mathpar}
  (0) \psubstp{Q}{P} := 0 \\
  (R \juxtap S) \psubstp{Q}{P}
  :=    
  (R)\psubstp{Q}{P} \juxtap (S) \psubstp{Q}{P} \\
  (x?(y).R) \psubstp{Q}{P}    
  :=    
  (x)\substp{Q}{P} (z)\concat( (R \psubstn{z}{y}) \psubstp{Q}{P} ) \\
  (\lift{x}{R}) \psubstp{Q}{P}  
  :=
  \lift{(x)\substp{Q}{P}}{ R \psubstp{Q}{P} } \\
%   (\dropn{x})  \psubstp{Q}{P}       
%   := 
%   \left\{ 
%     \begin{array}{ccc} 
%       \dropn{\quotep{Q}} & & x \nameeq \quotep{P} \\
%       \dropn{x} & & otherwise \\
%     \end{array}
%   \right. 
  (\dropn{x})  \psubstp{Q}{P}       
  := 
  \left\{ 
    \begin{array}{ccc} 
      Q & & x \nameeq \quotep{P} \\
      \dropn{x} & & otherwise \\
    \end{array}
  \right.
\end{mathpar}
 

where

\begin{eqnarray}
  (x)\id{\{} \lpquote Q \rpquote / \lpquote P \rpquote \id{\}}            = 
  \left\{ 
    \begin{array}{ccc}
      \lpquote Q \rpquote & & x \nameeq \lpquote P \rpquote \\
      x & & otherwise \\
    \end{array}
  \right. \nonumber
\end{eqnarray}

and $z$ is chosen distinct from $\quotep{P}$, $\quotep{Q}$, the free
names in $Q$, and all the names in $R$. Our $\alpha$-equivalence will
be built in the standard way from this substitution.

\begin{remark}\label{rem:no_self_referential_names}
  One consequence of these definitions is that $\forall P. \quotep{P}
  \not\in \freenames{P}$.
\end{remark}

\subsection{ Dynamic quote: an example }

Anticipating something of what's to come, consider applying the
substitution, $\widehat{\id{\{}u / z \id{\}}}$, to the following pair
of processes, $\lift{w}{y!(z)}$ and $w[ \lpquote y!(z) \rpquote ]$.

\begin{eqnarray}
	\lift{w}{y!(z)}\widehat{\id{\{}u / z \id{\}}}
		& = &
		\lift{w}{y!(u)} \nonumber\\
	w[ \lpquote y!(z) \rpquote ] \widehat{ \id{\{}u / z \id{\}} }
		& = &
		w[ \lpquote y!(z) \rpquote ] \nonumber
\end{eqnarray}

Because the body of the process between quotes is impervious to
substitution, we get radically different answers. In fact, by
examining the first process in an input context,
e.g. $x?(z).\lift{w}{y!(z)}$, we see that the process under the lift
operator may be shaped by prefixed inputs binding a name inside it. In
this sense, the lift operator will be seen as a way to dynamically
construct processes before reifying them as names.

Finally equipped with these standard features we can present the
dynamics of the calculus.

\subsubsection{Operational semantics} 

Finally, we introduce the computational dynamics. What marks these
algebras as distinct from other more traditionally studied algebraic
structures, e.g. vector spaces or polynomial rings, is the manner in
which dynamics is captured. In traditional structures, dynamics is typically
expressed through morphisms between such structures, as in linear maps
between vector spaces or morphisms between rings. In algebras
associated with the semantics of computation, the dynamics is
expressed as part of the algebraic structure itself, through a
reduction reduction relation typically denoted by $\red$. Below, we
give a recursive presentation of this relation for the calculus used
in the encoding.

$\red \subseteq \pi \times \pi$
$\red : \pi \to \mathcal{P}(\pi)$

\begin{mathpar}
  \inferrule* [lab=Comm] { \textsf{match}( x_{src}, x_{trgt} ) } { x_{trgt}?(y)P \; | \; x_{src}!\langle {Q} \rangle \red P\{\quotep{Q}/y}\} }
  \and \\
  \inferrule* [lab=Par] {{P} \red {P}'} {{{P} | {Q}} \red {{P}' | {Q}}}
  \and
  \inferrule* [lab=Equiv]{{{P} \scong {P}'} \andalso {{P}' \red {Q}'} \andalso {{Q}' \scong {Q}}}{{P} \red {Q}}
\end{mathpar}

\begin{eqnarray*}
  match_{\equiv} (\quotep{P},\quotep{Q}) & := & P \equiv Q \\
  match_{\dagger}(\quotep{P},\quotep{Q}) & := & \forall R. P|Q \red^{*} R => R \red^{*} 0 \\
  match_{K}(\quotep{P},\quotep{Q}) & := & K \mbox{ for some context } K
\end{eqnarray*}

$u?(x)P | u!\langle Q \rangle \red P\{\quotep{Q}/x\}$

%We write $\wred$ for $\red^*$, and $P\red$ if $\exists Q $ such that $ P \red Q$.
We write $P\red$ if $\exists Q $ such that $ P \red Q$ and $P\not\red$, otherwise.

\section{Replication}

As mentioned before, it is known that replication (and hence
recursion) can be implemented in a higher-order process algebra
\cite{SangiorgiWalker}. As our first example of calculation with the
machinery thus far presented we give the construction explicitly in
the {\rhoc}.

\begin{eqnarray}
	D_{x} & := & \prefix{x}{y}{(\binpar{\outputp{x}{y}}{@{y}})} \nonumber\\
	\bangp_{x}{P} & := & \binpar{{x}!\langle{\binpar{D_{x}}{P}}\rangle}{D_{x}} \nonumber
\end{eqnarray}

\begin{eqnarray}
	\bangp_{x}{P} & & \nonumber\\
	=
	& {x}!\langle{(\prefix{x}{y}{(\outputp{x}{y} | @{y})) | P}}\rangle 
	      | \prefix{x}{y}{(\outputp{x}{y} | @{y})} & \nonumber\\
	\red
	& (\outputp{x}{y} | @{y})\substn{\quotep{(\prefix{x}{y}{(@{y} | \outputp{x}{y})) | P}}}{y} & \nonumber\\
	=
	& \outputp{x}{\quotep{(\prefix{x}{y}{(\outputp{x}{y} | @{y})) | P}}}
	  | {(\prefix{x}{y}{(\outputp{x}{y} | @{y})) | P}} & \nonumber\\
	\red
	& \ldots & \nonumber\\
	\red^*
	& P | P | \ldots & \nonumber
\end{eqnarray}

Of course, this encoding, as an implementation, runs away, unfolding
$\bangp{P}$ eagerly. A lazier and more implementable replication
operator, restricted to input-guarded processes, may be obtained as follows.

\begin{eqnarray}
\bangp{\prefix{u}{v}{P}} 
	:= 
	\binpar{\lift{x}{\prefix{u}{v}{(\binpar{D(x)}{P})}}}{D(x)} \nonumber
\end{eqnarray}

\begin{remark}
  Note that the lazier definition still does not deal with summation
  or mixed summation (i.e. sums over input and output). The reader is
  invited to construct definitions of replication that deal with these
  features. 

  Further, the definitions are parameterized in a name, $x$. Can you,
  gentle reader, make a definition that eliminates this parameter and
  guarantees no accidental interaction between the replication
  machinery and the process being replicated -- i.e. no accidental
  sharing of names used by the process to get its work done and the
  name(s) used by the replication to effect copying. This latter
  revision of the definition of replication is crucial to obtaining
  the expected identity $!!P \sim !P$.
\end{remark}

\begin{remark}\label{rem:paradoxical_combinator}
  The reader familiar with the lambda calculus will have noticed the
  similarity between $D$ and the paradoxical combinator.

  [Ed. note: the existence of this seems to suggest we have to be more
  restrictive on the set of processes and names we admit if we are to
  support no-cloning.]
\end{remark}

\subsubsection{Bisimulation}

The computational dynamics gives rise to another kind of equivalence,
the equivalence of computational behavior. As previously mentioned
this is typically captured \emph{via} some form of bisimulation.

% The notion we use in this paper is weak barbed bisimulation
% \cite{milner91polyadicpi}.

The notion we use in this paper is derived from weak barbed
bisimulation \cite{milner91polyadicpi}. 

\begin{definition}
An \emph{observation relation}, $\downarrow_{\mathcal N}$, over a set
of names, $\mathcal N$, is the smallest relation satisfying the rules
below.

\infrule[Out-barb]{y \in {\mathcal N}, \; x \nameeq y}
		  {\outputp{x}{v} \downarrow_{\mathcal N} x}
\infrule[Par-barb]{\mbox{$P\downarrow_{\mathcal N} x$ or $Q\downarrow_{\mathcal N} x$}}
		  {\binpar{P}{Q} \downarrow_{\mathcal N} x}

We write $P \Downarrow_{\mathcal N} x$ if there is $Q$ such that 
$P \wred Q$ and $Q \downarrow_{\mathcal N} x$.
\end{definition}

\begin{definition}
%\label{def.bbisim}
An  ${\mathcal N}$-\emph{barbed bisimulation} over a set of names, ${\mathcal N}$, is a symmetric binary relation 
${\mathcal S}_{\mathcal N}$ between agents such that $P\rel{S}_{\mathcal N}Q$ implies:
\begin{enumerate}
\item If $P \red P'$ then $Q \wred Q'$ and $P'\rel{S}_{\mathcal N} Q'$.
\item If $P\downarrow_{\mathcal N} x$, then $Q\Downarrow_{\mathcal N} x$.
\end{enumerate}
$P$ is ${\mathcal N}$-barbed bisimilar to $Q$, written
$P \wbbisim_{\mathcal N} Q$, if $P \rel{S}_{\mathcal N} Q$ for some ${\mathcal N}$-barbed bisimulation ${\mathcal S}_{\mathcal N}$.
\end{definition}

$\mathcal{R} \subseteq \pi \times \pi$

$P \mathcal{R} Q => \forall P'. P \red P' \Rightarrow \exists Q'. Q \red Q', P' \mathcal{R} Q'$

$P \vdash x \Rightarrow Q \vdash x$

\begin{mathpar}
  \inferrule*[lab=Out-barb]{x \nameeq y}{{y}!\langle{Q}\rangle \vdash x}
  \and
  \inferrule*[lab=Par-barb]{\mbox{$P\vdash x$ or $Q\vdash x$}}{\binpar{P}{Q} \vdash x}
\end{mathpar}

\subsubsection{Contexts}

One of the principle advantages of computational calculi like the
$\pi$-calculus is a well-defined notion of context,
contextual-equivalence and a correlation between
contextual-equivalence and notions of bisimulation. The notion of
context allows the decomposition of a process into (sub-)process and
its syntactic environment, its context. Thus, a context may be
thought of as a process with a ``hole'' (written $\Box$) in it. The
application of a context $M$ to a process $P$, written $M[P]$, is
tantamount to filling the hole in $M$ with $P$. In this paper we do
not need the full weight of this theory, but do make use of the notion
of context in the proof the main theorem. 

\begin{mathpar}
  \inferrule* [lab=summation] {} {{M_{M},M_{N}} \bc \Box \;|\; x.M_{A} \;|\; M_{M}+M_{N}}
  \and
  \inferrule* [lab=agent] {} {{M_{A}} \bc (\vec{x})M_{P} \;| \; \clift{P_0,\ldots,M_{P},\ldots,P_N}}
  \and \\
  \inferrule* [lab=process] {} {{M_{P}} \bc M_{N} \;| \;P|M_{P} }
\end{mathpar} 

\begin{mathpar}
  \inferrule* [lab=sychronization] {} {M_{N} \bc \Box \;|\; x?M_{F} \;|\; x!M_{C}}
  \and
  \inferrule* [lab=abstraction] {} {{M_{F}} \bc (x)M_{P} }
  \and
  \inferrule* [lab=concretion] {} {{M_{C}} \bc \langle M_{P} \rangle }
  \and \\
  \inferrule* [lab=process] {} {{M_{P}} \bc M_{N} \;| \;P|M_{P} }
\end{mathpar}

\begin{definition}[contextual application] Given a context $M$, and
  process $P$, we define the \emph{contextual application}, $M[P] :=
  M\{P/\Box\}$. That is, the contextual application of M to P is the
  substitution of $P$ for $\Box$ in $M$.
\end{definition}

$\meaningof{-} : L \to \mathcal{P}(\pi)$

\begin{mathpar}
  \inferrule* [lab=collection] {} {\meaningof{true} = \pi, \and \meaningof{~E} = \pi \setminus \meaningof{E}, \and \meaningof{E_{1} \& E_{2}} = \meaningof{E_{1}} \cap \meaningof{E_{2}}}
\end{mathpar}

\begin{mathpar}
  \inferrule* [lab=structure] {} {\meaningof{0} = \{ P \in \pi | P \equiv 0 \}, \and \\ \meaningof{E_1 | E_2} = \{ P \in \pi | P \equiv P_{1} | P_{2}, P_{1} \in \meaningof{E_{1}}, P_{2} \in \meaningof{E_2}\} }
\end{mathpar}

\begin{mathpar}
 \inferrule* [lab=behavior] {} {\meaningof{\langle a?b \rangle E} = \{ P \in \pi | P \equiv Q | u?(y)P', \\ \and \\\\ \and \\ \;\;\; u \in \meaningof{a}, \forall z.P'\{z/y\} \in \meaningof{E\{z/b\}}\}, \and \\ \meaningof{a!E} = \{ P \in \pi | P \equiv Q | x!\langle P' \rangle, x \in \meaningof{a} P' \in \meaningof{E}\} }
\end{mathpar}

\begin{mathpar}
 \inferrule* [lab=nominal] {} {\meaningof{\quotep{E}} = \{ \quotep{P} \in \quotep{\pi} | P \in \meaningof{E} \}, \and \meaningof{\quotep{P}} = \{ \quotep{Q} \in \quotep{\pi} | P \equiv Q \} \and \\ \meaningof{@\quotep{E}} = \{ P \in \pi | P \equiv @x, x \in \meaningof{E} \}}
\end{mathpar}

\begin{eqnarray*}
  \\
  \meaningof{-} : TS \to ST
\end{eqnarray*}

\begin{eqnarray*}
  \\
  L : TS \to ST
\end{eqnarray*}

\begin{eqnarray*}
  \\
  P \models E \iff P \in \meaningof{E}
\end{eqnarray*}

\begin{eqnarray*}
  P \approx_{L} Q \iff \forall E \in L. P \models E \iff Q \models E
\end{eqnarray*}

\begin{eqnarray*}
  P \approx_{K} Q
\end{eqnarray*}

\begin{eqnarray*}
  P \approx Q
\end{eqnarray*}

$\approx_{K} = \approx = \approx_{L}$

\subsubsection{Contextual duality}

Note that contexts extend the quotation operation to a family of
operations from processes to names. Given a context, $M$, we can
define a \emph{nominal context}, $\quotep{M}$ by $\quotep{M}[P] :=
\quotep{M[P]}$. To foreshadow what is to come we observe that these
operations enjoy a duality with processes very much like the duality
between vectors and maps from vectors to scalars.

Further, because the calculus is essentially higher-order, we have a
correspondence between contexts and processes. More specifically,
given a name $x$ and a context $M$ we can construct $M^{*}_{x}$ such
that 

\begin{mathpar}
  M^{*}_{x} | \lift{x}{P} \red M[P]
\end{mathpar}

namely,

\begin{mathpar}
  M^{*}_{x} := x?(u).M[\dropn{u}]
\end{mathpar}

The dependence of $M^{*}_{x}$ on a name makes it an abstraction, 

\begin{mathpar}
  M^{*} := (x)x?(u).M[\dropn{u}]
\end{mathpar}

\subsection{Additional notation}

It will sometimes be convenient to denote the process a name
quotes. We already have the notation $x = \quotep{P}$, but it will be
convenient to introduce an alternate notation, $\procn{x}$, when we
want to emphasize the connection to the use of the name. Note that, by
virtue of name equivalence, $\quotep{\procn{x}} \nameeq x$; so, the
notation is consistent with previous definitions.

Further, because names have structure it is possible to effect
substitutions on the basis of that structure. This means we need to
upgrade our notation for substitutions, which we accomplish by
adapting comprehension notation. Thus,

\begin{mathpar}
  P\{ y / x : x \in S \}
\end{mathpar}

is interpreted to mean the process derived from P by replacing (in a
capture-avoiding manner) each occurrence of $x$ in $S$ by $y$. For example,

\begin{mathpar}
  P\{ \quotep{\procn{x}|\procn{x}} / x : x \in \freenames{P} \}
\end{mathpar}

will replace each (occurrence) of a free name $x$ in $P$ by
$\quotep{\procn{x}|\procn{x}}$.

Also, we will avail ourselves of the notation $x^{L}$ and $x^{R}$ to
denote injections of a name into disjoint copies of the name
space. There are numerous ways to accomplish this. One example can be
found in \cite{MeredithR05}. This notation overloads to vectors of
names: $\vec{x}^{\pi} := (x_{i}^{\pi} \; : \; 0 \leq i < |\vec{x}| )$ where $\pi \in \{L,R\}$.

We also use $P^{\Box} := P|\Box$.

In \cite{MeredithR05} an interpretation of the new operator is
given. It turns out that there are several possible interpretations
all enjoying the requisite algebraic properties of the operator (see
\cite{milner91polyadicpi}). We will therefore make liberal use of
$(\nu\; \vec{x})P$.

% subsection the_syntax_and_semantics_of_the_notation_system (end)   

\input{qm2pi.qmops} 

\input{qm2pi.sterngerlach} 

\input{qm2pi.metric} 

% section concurrent_process_calculi (end)

%\input{qm2pi.proofsketch}

% section proof sketch (end)

%\input{qm2pi.slviaknots} 

% section spatial logic via knots (end)

\input{qm2pi.conclusion}

% section conclusion (end)

%\input{qm2pi.dtcodes} 

% section wiring algorithm (end)

\input{qm2pi.ack} 

% section acknowledgments (end)

\newpage


\bibliographystyle{plain}   
\bibliography{../../biblios/main.bib}

\input{qm2pi.rhodetails}

\end{document}

 

\documentclass[12pt]{llncs}
%\documentclass{jktr}

\usepackage[pdftex]{hyperref}                   
\usepackage {listings}
\usepackage {mathpartir}
\usepackage{bcprules}
%\usepackage{listings}
                       
\usepackage{graphicx} 
%\usepackage[margins=2.5cm,nohead,nofoot]{geometry}
%\usepackage{geometry}
\usepackage{amsfonts}
\usepackage{amstext}
\usepackage{latexsym}
\usepackage{amssymb}
\usepackage{color}


%\include{myPreamble}
\include{qm2pi.local} 

%\ifpdf
%\usepackage[pdftex]{graphicx}
%\else
%\usepackage{graphicx}
%\fi

 % \ifpdf
%  \usepackage{pdfsync}
%  \if


%\title{Brief Article}
%\author{David F. Snyder}
%\author{L.G. Meredith}

%\address{Dept. of Math., Texas State University--San Marcos, San Marcos, TX 78666}
       
\pagestyle{empty}


\begin{document}

\lstset{language=[Objective]Caml,frame=shadowbox}

\input{qm2pi.front}

% section front matter (end)

\input{qm2pi.intro} 
 
% section introduction (end)

% \input{qm2pi.knotations} 

% section notation (end)

\input{qm2pi.process.calculi} 

% section concurrent_process_calculi_and_spatial_logics_ (end)
    
%\input{qm2pi.knots2pi} 

%\input{qm2pi.trefoil} 

%\input{qm2pi.mainthm} 

% subsection basic_interpretation (end)

%\input{qm2pi.rho.presentation} 
\subsection{The syntax and semantics of the notation system}\label{sub:the_syntax_and_semantics_of_the_notation_system} % (fold)

We now summarize a technical presentation of the calculus that
embodies our theory of dynamics. The typical presentation of such a
calculus follows the style of giving generators and relations on
them. The grammar, below, describing term constructors, freely
generates the set of processes, $\Proc$. This set is then quotiented
by a relation known as structural congruence and it is over this set
that the notion of dynamics is expressed. This presentation is
essentially that of \cite{MeredithR05} with the addition of
polyadicity and summation. For readability we have relegated some of
the technical subtleties to an appendix.

\subsubsection{Process grammar}\label{subsub:process_grammar}

\begin{mathpar}
  \inferrule* [lab=synchronization] {} {{M} \bc \pzero \;|\; x?F \;|\; x!C }
  \and
  \inferrule* [lab=abstraction] {} {{F} \bc (x)P}
  \and
  \inferrule* [lab=concretion] {} {{C} \bc \langle Q \rangle}
  \and
  \inferrule* [lab=process] {} {{P,Q} \bc M \;| \;P|Q \;|\; @{x}}
  \and
  \inferrule* [lab=name] {} {{x} \bc \quotep{P}}
\end{mathpar} 

Note that $\vec{x}$ (resp. $\vec{P}$) denotes a vector of names
(resp. processes) of length $|\vec{x}|$ (resp. $|\vec{P}|$). We adopt
the following useful abbreviations.

\begin{mathpar}
   x?(\vec{y}).P := x.(\vec{y})P \and  x\clift{\vec{P}} := x.\clift{\vec{P}}
   \and x!(y) := \lift{x}{\dropn{y}}
   \and \Pi_{i=0}^{n-1}P_i := P_0 | \ldots | P_{n-1}
\end{mathpar}

\subsubsection{Structural congruence}

\paragraph{Free and bound names and alpha-equivalence.} At the
core of structural equivalence is alpha-equivalence which identifies
process that are the same up to a change of variable. Formally, we
recognize the distinction between free and bound names. The free names
of a process, $\freenames{P}$, may be calculated recursively as
follows:

\begin{mathpar}
\freenames{\pzero} := \emptyset
  \and \\
  \freenames{x?(y).P} := \{ x \} \cup (\freenames{P} \setminus \{ y \})
  \and 
  \freenames{x!\langle P \rangle} := \{ x \} \cup \{ P \} 
  \and \\
  \freenames{P|Q} := \freenames{P} \cup \freenames{Q}
  \and \\
  \freenames{@{x}} := \{ x \}
\end{mathpar}

$\pi$
$\quotep{\pi}$

$\freenames{-} : \pi \to \mathcal{P}(\quotep{\pi})$

\begin{eqnarray*}
  \freenames{\pzero} & := & \emptyset \\
  \freenames{x?(y).P} & := & \{ x \} \cup (\freenames{P} \setminus \{ y \}) \\
  \freenames{x!\langle P \rangle} & := & \{ x \} \cup \{ P \} \\
  \freenames{P|Q} & := & \freenames{P} \cup \freenames{Q} \\
  \freenames{\dropn{x}} & := & \{ x \}
\end{eqnarray*}

The bound names of a process, $\boundnames{P}$, are those names occurring in $P$
that are not free. For example, in $x?(y).0$, the name $x$ is free, while $y$ is bound.

\begin{mathpar}
  \inferrule* [lab=monoidal-laws] {} { P|Q \equiv Q|P \and P|0 \equiv P \and P|(Q|R) \equiv (P|Q)|R }
\end{mathpar}

\begin{mathpar}
  \inferrule* [lab=alpha-equivalence] {} { (x)P \equiv (y)P\{y/x\} \and y \not\in \freenames{P} }
\end{mathpar}

\begin{definition}
Then two processes, $P,Q$, are alpha-equivalent if $P = Q\{\vec{y}/\vec{x}\}$ for
some $\vec{x} \in \boundnames{Q},\vec{y} \in \boundnames{P}$, where $Q\{\vec{y}/\vec{x}\}$
denotes the capture-avoiding substitution of $\vec{y}$ for $\vec{x}$ in $Q$.
\end{definition}

\begin{definition}
  The {\em structural congruence} \cite{SangiorgiWalker} , $\equiv$,
  between processes is the least congruence containing
  alpha-equivalence, satisfying the abelian monoid laws
  (associativity, commutativity and $\pzero$ as identity) for parallel
  composition $|$ and for summation $+$.
\end{definition}

\subsection{Name equivalence}

We take name equivalence, written $\nameeq$, to be the smallest
equivalence relation generated by the following rules.

\begin{mathpar}
\inferrule*[lab=Quote-drop]
{ }
{ \quotep{@{x}} \nameeq x }

\inferrule*[lab=Struct-equiv]
{ P \scong Q }
{ \quotep{P} \nameeq \quotep{Q} }
\end{mathpar}

The astute reader will have noticed that the mutual recursion of names
and processes imposes a mutual recursion on alpha-equivalence and
structural equivalence via name-equivalence. Fortunately, all of this
works out pleasantly and we may calculate in the natural way, free of
concern. The reader interested in the details is referred to the
appendix \ref{appendix:rho_details}.

\subsection{Substitution}

We use $\Proc$ for the set of processes, $\QProc$ for the set of
names, and $\id{\{}\vec{y} / \vec{x} \id{\}}$ to denote partial maps,
$s : \QProc \rightarrow \QProc$. A map, $s$ lifts, uniquely, to a map
on process terms, $\widehat{s} : \Proc \rightarrow \Proc$ by the
following equations.

\begin{mathpar}
  (0) \psubstp{Q}{P} := 0 \\
  (R \juxtap S) \psubstp{Q}{P}
  :=    
  (R)\psubstp{Q}{P} \juxtap (S) \psubstp{Q}{P} \\
  (x?(y).R) \psubstp{Q}{P}    
  :=    
  (x)\substp{Q}{P} (z)\concat( (R \psubstn{z}{y}) \psubstp{Q}{P} ) \\
  (\lift{x}{R}) \psubstp{Q}{P}  
  :=
  \lift{(x)\substp{Q}{P}}{ R \psubstp{Q}{P} } \\
%   (\dropn{x})  \psubstp{Q}{P}       
%   := 
%   \left\{ 
%     \begin{array}{ccc} 
%       \dropn{\quotep{Q}} & & x \nameeq \quotep{P} \\
%       \dropn{x} & & otherwise \\
%     \end{array}
%   \right. 
  (\dropn{x})  \psubstp{Q}{P}       
  := 
  \left\{ 
    \begin{array}{ccc} 
      Q & & x \nameeq \quotep{P} \\
      \dropn{x} & & otherwise \\
    \end{array}
  \right.
\end{mathpar}
 

where

\begin{eqnarray}
  (x)\id{\{} \lpquote Q \rpquote / \lpquote P \rpquote \id{\}}            = 
  \left\{ 
    \begin{array}{ccc}
      \lpquote Q \rpquote & & x \nameeq \lpquote P \rpquote \\
      x & & otherwise \\
    \end{array}
  \right. \nonumber
\end{eqnarray}

and $z$ is chosen distinct from $\quotep{P}$, $\quotep{Q}$, the free
names in $Q$, and all the names in $R$. Our $\alpha$-equivalence will
be built in the standard way from this substitution.

\begin{remark}\label{rem:no_self_referential_names}
  One consequence of these definitions is that $\forall P. \quotep{P}
  \not\in \freenames{P}$.
\end{remark}

\subsection{ Dynamic quote: an example }

Anticipating something of what's to come, consider applying the
substitution, $\widehat{\id{\{}u / z \id{\}}}$, to the following pair
of processes, $\lift{w}{y!(z)}$ and $w[ \lpquote y!(z) \rpquote ]$.

\begin{eqnarray}
	\lift{w}{y!(z)}\widehat{\id{\{}u / z \id{\}}}
		& = &
		\lift{w}{y!(u)} \nonumber\\
	w[ \lpquote y!(z) \rpquote ] \widehat{ \id{\{}u / z \id{\}} }
		& = &
		w[ \lpquote y!(z) \rpquote ] \nonumber
\end{eqnarray}

Because the body of the process between quotes is impervious to
substitution, we get radically different answers. In fact, by
examining the first process in an input context,
e.g. $x?(z).\lift{w}{y!(z)}$, we see that the process under the lift
operator may be shaped by prefixed inputs binding a name inside it. In
this sense, the lift operator will be seen as a way to dynamically
construct processes before reifying them as names.

Finally equipped with these standard features we can present the
dynamics of the calculus.

\subsubsection{Operational semantics} 

Finally, we introduce the computational dynamics. What marks these
algebras as distinct from other more traditionally studied algebraic
structures, e.g. vector spaces or polynomial rings, is the manner in
which dynamics is captured. In traditional structures, dynamics is typically
expressed through morphisms between such structures, as in linear maps
between vector spaces or morphisms between rings. In algebras
associated with the semantics of computation, the dynamics is
expressed as part of the algebraic structure itself, through a
reduction reduction relation typically denoted by $\red$. Below, we
give a recursive presentation of this relation for the calculus used
in the encoding.

$\red \subseteq \pi \times \pi$
$\red : \pi \to \mathcal{P}(\pi)$

\begin{mathpar}
  \inferrule* [lab=Comm] { \textsf{match}( x_{src}, x_{trgt} ) } { x_{trgt}?(y)P \; | \; x_{src}!\langle {Q} \rangle \red P\{\quotep{Q}/y}\} }
  \and \\
  \inferrule* [lab=Par] {{P} \red {P}'} {{{P} | {Q}} \red {{P}' | {Q}}}
  \and
  \inferrule* [lab=Equiv]{{{P} \scong {P}'} \andalso {{P}' \red {Q}'} \andalso {{Q}' \scong {Q}}}{{P} \red {Q}}
\end{mathpar}

\begin{eqnarray*}
  match_{\equiv} (\quotep{P},\quotep{Q}) & := & P \equiv Q \\
  match_{\dagger}(\quotep{P},\quotep{Q}) & := & \forall R. P|Q \red^{*} R => R \red^{*} 0 \\
  match_{K}(\quotep{P},\quotep{Q}) & := & K \mbox{ for some context } K
\end{eqnarray*}

$u?(x)P | u!\langle Q \rangle \red P\{\quotep{Q}/x\}$

%We write $\wred$ for $\red^*$, and $P\red$ if $\exists Q $ such that $ P \red Q$.
We write $P\red$ if $\exists Q $ such that $ P \red Q$ and $P\not\red$, otherwise.

\section{Replication}

As mentioned before, it is known that replication (and hence
recursion) can be implemented in a higher-order process algebra
\cite{SangiorgiWalker}. As our first example of calculation with the
machinery thus far presented we give the construction explicitly in
the {\rhoc}.

\begin{eqnarray}
	D_{x} & := & \prefix{x}{y}{(\binpar{\outputp{x}{y}}{@{y}})} \nonumber\\
	\bangp_{x}{P} & := & \binpar{{x}!\langle{\binpar{D_{x}}{P}}\rangle}{D_{x}} \nonumber
\end{eqnarray}

\begin{eqnarray}
	\bangp_{x}{P} & & \nonumber\\
	=
	& {x}!\langle{(\prefix{x}{y}{(\outputp{x}{y} | @{y})) | P}}\rangle 
	      | \prefix{x}{y}{(\outputp{x}{y} | @{y})} & \nonumber\\
	\red
	& (\outputp{x}{y} | @{y})\substn{\quotep{(\prefix{x}{y}{(@{y} | \outputp{x}{y})) | P}}}{y} & \nonumber\\
	=
	& \outputp{x}{\quotep{(\prefix{x}{y}{(\outputp{x}{y} | @{y})) | P}}}
	  | {(\prefix{x}{y}{(\outputp{x}{y} | @{y})) | P}} & \nonumber\\
	\red
	& \ldots & \nonumber\\
	\red^*
	& P | P | \ldots & \nonumber
\end{eqnarray}

Of course, this encoding, as an implementation, runs away, unfolding
$\bangp{P}$ eagerly. A lazier and more implementable replication
operator, restricted to input-guarded processes, may be obtained as follows.

\begin{eqnarray}
\bangp{\prefix{u}{v}{P}} 
	:= 
	\binpar{\lift{x}{\prefix{u}{v}{(\binpar{D(x)}{P})}}}{D(x)} \nonumber
\end{eqnarray}

\begin{remark}
  Note that the lazier definition still does not deal with summation
  or mixed summation (i.e. sums over input and output). The reader is
  invited to construct definitions of replication that deal with these
  features. 

  Further, the definitions are parameterized in a name, $x$. Can you,
  gentle reader, make a definition that eliminates this parameter and
  guarantees no accidental interaction between the replication
  machinery and the process being replicated -- i.e. no accidental
  sharing of names used by the process to get its work done and the
  name(s) used by the replication to effect copying. This latter
  revision of the definition of replication is crucial to obtaining
  the expected identity $!!P \sim !P$.
\end{remark}

\begin{remark}\label{rem:paradoxical_combinator}
  The reader familiar with the lambda calculus will have noticed the
  similarity between $D$ and the paradoxical combinator.

  [Ed. note: the existence of this seems to suggest we have to be more
  restrictive on the set of processes and names we admit if we are to
  support no-cloning.]
\end{remark}

\subsubsection{Bisimulation}

The computational dynamics gives rise to another kind of equivalence,
the equivalence of computational behavior. As previously mentioned
this is typically captured \emph{via} some form of bisimulation.

% The notion we use in this paper is weak barbed bisimulation
% \cite{milner91polyadicpi}.

The notion we use in this paper is derived from weak barbed
bisimulation \cite{milner91polyadicpi}. 

\begin{definition}
An \emph{observation relation}, $\downarrow_{\mathcal N}$, over a set
of names, $\mathcal N$, is the smallest relation satisfying the rules
below.

\infrule[Out-barb]{y \in {\mathcal N}, \; x \nameeq y}
		  {\outputp{x}{v} \downarrow_{\mathcal N} x}
\infrule[Par-barb]{\mbox{$P\downarrow_{\mathcal N} x$ or $Q\downarrow_{\mathcal N} x$}}
		  {\binpar{P}{Q} \downarrow_{\mathcal N} x}

We write $P \Downarrow_{\mathcal N} x$ if there is $Q$ such that 
$P \wred Q$ and $Q \downarrow_{\mathcal N} x$.
\end{definition}

\begin{definition}
%\label{def.bbisim}
An  ${\mathcal N}$-\emph{barbed bisimulation} over a set of names, ${\mathcal N}$, is a symmetric binary relation 
${\mathcal S}_{\mathcal N}$ between agents such that $P\rel{S}_{\mathcal N}Q$ implies:
\begin{enumerate}
\item If $P \red P'$ then $Q \wred Q'$ and $P'\rel{S}_{\mathcal N} Q'$.
\item If $P\downarrow_{\mathcal N} x$, then $Q\Downarrow_{\mathcal N} x$.
\end{enumerate}
$P$ is ${\mathcal N}$-barbed bisimilar to $Q$, written
$P \wbbisim_{\mathcal N} Q$, if $P \rel{S}_{\mathcal N} Q$ for some ${\mathcal N}$-barbed bisimulation ${\mathcal S}_{\mathcal N}$.
\end{definition}

$\mathcal{R} \subseteq \pi \times \pi$

$P \mathcal{R} Q => \forall P'. P \red P' \Rightarrow \exists Q'. Q \red Q', P' \mathcal{R} Q'$

$P \vdash x \Rightarrow Q \vdash x$

\begin{mathpar}
  \inferrule*[lab=Out-barb]{x \nameeq y}{{y}!\langle{Q}\rangle \vdash x}
  \and
  \inferrule*[lab=Par-barb]{\mbox{$P\vdash x$ or $Q\vdash x$}}{\binpar{P}{Q} \vdash x}
\end{mathpar}

\subsubsection{Contexts}

One of the principle advantages of computational calculi like the
$\pi$-calculus is a well-defined notion of context,
contextual-equivalence and a correlation between
contextual-equivalence and notions of bisimulation. The notion of
context allows the decomposition of a process into (sub-)process and
its syntactic environment, its context. Thus, a context may be
thought of as a process with a ``hole'' (written $\Box$) in it. The
application of a context $M$ to a process $P$, written $M[P]$, is
tantamount to filling the hole in $M$ with $P$. In this paper we do
not need the full weight of this theory, but do make use of the notion
of context in the proof the main theorem. 

\begin{mathpar}
  \inferrule* [lab=summation] {} {{M_{M},M_{N}} \bc \Box \;|\; x.M_{A} \;|\; M_{M}+M_{N}}
  \and
  \inferrule* [lab=agent] {} {{M_{A}} \bc (\vec{x})M_{P} \;| \; \clift{P_0,\ldots,M_{P},\ldots,P_N}}
  \and \\
  \inferrule* [lab=process] {} {{M_{P}} \bc M_{N} \;| \;P|M_{P} }
\end{mathpar} 

\begin{mathpar}
  \inferrule* [lab=sychronization] {} {M_{N} \bc \Box \;|\; x?M_{F} \;|\; x!M_{C}}
  \and
  \inferrule* [lab=abstraction] {} {{M_{F}} \bc (x)M_{P} }
  \and
  \inferrule* [lab=concretion] {} {{M_{C}} \bc \langle M_{P} \rangle }
  \and \\
  \inferrule* [lab=process] {} {{M_{P}} \bc M_{N} \;| \;P|M_{P} }
\end{mathpar}

\begin{definition}[contextual application] Given a context $M$, and
  process $P$, we define the \emph{contextual application}, $M[P] :=
  M\{P/\Box\}$. That is, the contextual application of M to P is the
  substitution of $P$ for $\Box$ in $M$.
\end{definition}

$\meaningof{-} : L \to \mathcal{P}(\pi)$

\begin{mathpar}
  \inferrule* [lab=collection] {} {\meaningof{true} = \pi, \and \meaningof{~E} = \pi \setminus \meaningof{E}, \and \meaningof{E_{1} \& E_{2}} = \meaningof{E_{1}} \cap \meaningof{E_{2}}}
\end{mathpar}

\begin{mathpar}
  \inferrule* [lab=structure] {} {\meaningof{0} = \{ P \in \pi | P \equiv 0 \}, \and \\ \meaningof{E_1 | E_2} = \{ P \in \pi | P \equiv P_{1} | P_{2}, P_{1} \in \meaningof{E_{1}}, P_{2} \in \meaningof{E_2}\} }
\end{mathpar}

\begin{mathpar}
 \inferrule* [lab=behavior] {} {\meaningof{\langle a?b \rangle E} = \{ P \in \pi | P \equiv Q | u?(y)P', \\ \and \\\\ \and \\ \;\;\; u \in \meaningof{a}, \forall z.P'\{z/y\} \in \meaningof{E\{z/b\}}\}, \and \\ \meaningof{a!E} = \{ P \in \pi | P \equiv Q | x!\langle P' \rangle, x \in \meaningof{a} P' \in \meaningof{E}\} }
\end{mathpar}

\begin{mathpar}
 \inferrule* [lab=nominal] {} {\meaningof{\quotep{E}} = \{ \quotep{P} \in \quotep{\pi} | P \in \meaningof{E} \}, \and \meaningof{\quotep{P}} = \{ \quotep{Q} \in \quotep{\pi} | P \equiv Q \} \and \\ \meaningof{@\quotep{E}} = \{ P \in \pi | P \equiv @x, x \in \meaningof{E} \}}
\end{mathpar}

\begin{eqnarray*}
  \\
  \meaningof{-} : TS \to ST
\end{eqnarray*}

\begin{eqnarray*}
  \\
  L : TS \to ST
\end{eqnarray*}

\begin{eqnarray*}
  \\
  P \models E \iff P \in \meaningof{E}
\end{eqnarray*}

\begin{eqnarray*}
  P \approx_{L} Q \iff \forall E \in L. P \models E \iff Q \models E
\end{eqnarray*}

\begin{eqnarray*}
  P \approx_{K} Q
\end{eqnarray*}

\begin{eqnarray*}
  P \approx Q
\end{eqnarray*}

$\approx_{K} = \approx = \approx_{L}$

\subsubsection{Contextual duality}

Note that contexts extend the quotation operation to a family of
operations from processes to names. Given a context, $M$, we can
define a \emph{nominal context}, $\quotep{M}$ by $\quotep{M}[P] :=
\quotep{M[P]}$. To foreshadow what is to come we observe that these
operations enjoy a duality with processes very much like the duality
between vectors and maps from vectors to scalars.

Further, because the calculus is essentially higher-order, we have a
correspondence between contexts and processes. More specifically,
given a name $x$ and a context $M$ we can construct $M^{*}_{x}$ such
that 

\begin{mathpar}
  M^{*}_{x} | \lift{x}{P} \red M[P]
\end{mathpar}

namely,

\begin{mathpar}
  M^{*}_{x} := x?(u).M[\dropn{u}]
\end{mathpar}

The dependence of $M^{*}_{x}$ on a name makes it an abstraction, 

\begin{mathpar}
  M^{*} := (x)x?(u).M[\dropn{u}]
\end{mathpar}

\subsection{Additional notation}

It will sometimes be convenient to denote the process a name
quotes. We already have the notation $x = \quotep{P}$, but it will be
convenient to introduce an alternate notation, $\procn{x}$, when we
want to emphasize the connection to the use of the name. Note that, by
virtue of name equivalence, $\quotep{\procn{x}} \nameeq x$; so, the
notation is consistent with previous definitions.

Further, because names have structure it is possible to effect
substitutions on the basis of that structure. This means we need to
upgrade our notation for substitutions, which we accomplish by
adapting comprehension notation. Thus,

\begin{mathpar}
  P\{ y / x : x \in S \}
\end{mathpar}

is interpreted to mean the process derived from P by replacing (in a
capture-avoiding manner) each occurrence of $x$ in $S$ by $y$. For example,

\begin{mathpar}
  P\{ \quotep{\procn{x}|\procn{x}} / x : x \in \freenames{P} \}
\end{mathpar}

will replace each (occurrence) of a free name $x$ in $P$ by
$\quotep{\procn{x}|\procn{x}}$.

Also, we will avail ourselves of the notation $x^{L}$ and $x^{R}$ to
denote injections of a name into disjoint copies of the name
space. There are numerous ways to accomplish this. One example can be
found in \cite{MeredithR05}. This notation overloads to vectors of
names: $\vec{x}^{\pi} := (x_{i}^{\pi} \; : \; 0 \leq i < |\vec{x}| )$ where $\pi \in \{L,R\}$.

We also use $P^{\Box} := P|\Box$.

In \cite{MeredithR05} an interpretation of the new operator is
given. It turns out that there are several possible interpretations
all enjoying the requisite algebraic properties of the operator (see
\cite{milner91polyadicpi}). We will therefore make liberal use of
$(\nu\; \vec{x})P$.

% subsection the_syntax_and_semantics_of_the_notation_system (end)   

\input{qm2pi.qmops} 

\input{qm2pi.sterngerlach} 

\input{qm2pi.metric} 

% section concurrent_process_calculi (end)

%\input{qm2pi.proofsketch}

% section proof sketch (end)

%\input{qm2pi.slviaknots} 

% section spatial logic via knots (end)

\input{qm2pi.conclusion}

% section conclusion (end)

%\input{qm2pi.dtcodes} 

% section wiring algorithm (end)

\input{qm2pi.ack} 

% section acknowledgments (end)

\newpage


\bibliographystyle{plain}   
\bibliography{../../biblios/main.bib}

\input{qm2pi.rhodetails}

\end{document}

 

% section concurrent_process_calculi (end)

%\documentclass[12pt]{llncs}
%\documentclass{jktr}

\usepackage[pdftex]{hyperref}                   
\usepackage {listings}
\usepackage {mathpartir}
\usepackage{bcprules}
%\usepackage{listings}
                       
\usepackage{graphicx} 
%\usepackage[margins=2.5cm,nohead,nofoot]{geometry}
%\usepackage{geometry}
\usepackage{amsfonts}
\usepackage{amstext}
\usepackage{latexsym}
\usepackage{amssymb}
\usepackage{color}


%\include{myPreamble}
\include{qm2pi.local} 

%\ifpdf
%\usepackage[pdftex]{graphicx}
%\else
%\usepackage{graphicx}
%\fi

 % \ifpdf
%  \usepackage{pdfsync}
%  \if


%\title{Brief Article}
%\author{David F. Snyder}
%\author{L.G. Meredith}

%\address{Dept. of Math., Texas State University--San Marcos, San Marcos, TX 78666}
       
\pagestyle{empty}


\begin{document}

\lstset{language=[Objective]Caml,frame=shadowbox}

\input{qm2pi.front}

% section front matter (end)

\input{qm2pi.intro} 
 
% section introduction (end)

% \input{qm2pi.knotations} 

% section notation (end)

\input{qm2pi.process.calculi} 

% section concurrent_process_calculi_and_spatial_logics_ (end)
    
%\input{qm2pi.knots2pi} 

%\input{qm2pi.trefoil} 

%\input{qm2pi.mainthm} 

% subsection basic_interpretation (end)

%\input{qm2pi.rho.presentation} 
\subsection{The syntax and semantics of the notation system}\label{sub:the_syntax_and_semantics_of_the_notation_system} % (fold)

We now summarize a technical presentation of the calculus that
embodies our theory of dynamics. The typical presentation of such a
calculus follows the style of giving generators and relations on
them. The grammar, below, describing term constructors, freely
generates the set of processes, $\Proc$. This set is then quotiented
by a relation known as structural congruence and it is over this set
that the notion of dynamics is expressed. This presentation is
essentially that of \cite{MeredithR05} with the addition of
polyadicity and summation. For readability we have relegated some of
the technical subtleties to an appendix.

\subsubsection{Process grammar}\label{subsub:process_grammar}

\begin{mathpar}
  \inferrule* [lab=synchronization] {} {{M} \bc \pzero \;|\; x?F \;|\; x!C }
  \and
  \inferrule* [lab=abstraction] {} {{F} \bc (x)P}
  \and
  \inferrule* [lab=concretion] {} {{C} \bc \langle Q \rangle}
  \and
  \inferrule* [lab=process] {} {{P,Q} \bc M \;| \;P|Q \;|\; @{x}}
  \and
  \inferrule* [lab=name] {} {{x} \bc \quotep{P}}
\end{mathpar} 

Note that $\vec{x}$ (resp. $\vec{P}$) denotes a vector of names
(resp. processes) of length $|\vec{x}|$ (resp. $|\vec{P}|$). We adopt
the following useful abbreviations.

\begin{mathpar}
   x?(\vec{y}).P := x.(\vec{y})P \and  x\clift{\vec{P}} := x.\clift{\vec{P}}
   \and x!(y) := \lift{x}{\dropn{y}}
   \and \Pi_{i=0}^{n-1}P_i := P_0 | \ldots | P_{n-1}
\end{mathpar}

\subsubsection{Structural congruence}

\paragraph{Free and bound names and alpha-equivalence.} At the
core of structural equivalence is alpha-equivalence which identifies
process that are the same up to a change of variable. Formally, we
recognize the distinction between free and bound names. The free names
of a process, $\freenames{P}$, may be calculated recursively as
follows:

\begin{mathpar}
\freenames{\pzero} := \emptyset
  \and \\
  \freenames{x?(y).P} := \{ x \} \cup (\freenames{P} \setminus \{ y \})
  \and 
  \freenames{x!\langle P \rangle} := \{ x \} \cup \{ P \} 
  \and \\
  \freenames{P|Q} := \freenames{P} \cup \freenames{Q}
  \and \\
  \freenames{@{x}} := \{ x \}
\end{mathpar}

$\pi$
$\quotep{\pi}$

$\freenames{-} : \pi \to \mathcal{P}(\quotep{\pi})$

\begin{eqnarray*}
  \freenames{\pzero} & := & \emptyset \\
  \freenames{x?(y).P} & := & \{ x \} \cup (\freenames{P} \setminus \{ y \}) \\
  \freenames{x!\langle P \rangle} & := & \{ x \} \cup \{ P \} \\
  \freenames{P|Q} & := & \freenames{P} \cup \freenames{Q} \\
  \freenames{\dropn{x}} & := & \{ x \}
\end{eqnarray*}

The bound names of a process, $\boundnames{P}$, are those names occurring in $P$
that are not free. For example, in $x?(y).0$, the name $x$ is free, while $y$ is bound.

\begin{mathpar}
  \inferrule* [lab=monoidal-laws] {} { P|Q \equiv Q|P \and P|0 \equiv P \and P|(Q|R) \equiv (P|Q)|R }
\end{mathpar}

\begin{mathpar}
  \inferrule* [lab=alpha-equivalence] {} { (x)P \equiv (y)P\{y/x\} \and y \not\in \freenames{P} }
\end{mathpar}

\begin{definition}
Then two processes, $P,Q$, are alpha-equivalent if $P = Q\{\vec{y}/\vec{x}\}$ for
some $\vec{x} \in \boundnames{Q},\vec{y} \in \boundnames{P}$, where $Q\{\vec{y}/\vec{x}\}$
denotes the capture-avoiding substitution of $\vec{y}$ for $\vec{x}$ in $Q$.
\end{definition}

\begin{definition}
  The {\em structural congruence} \cite{SangiorgiWalker} , $\equiv$,
  between processes is the least congruence containing
  alpha-equivalence, satisfying the abelian monoid laws
  (associativity, commutativity and $\pzero$ as identity) for parallel
  composition $|$ and for summation $+$.
\end{definition}

\subsection{Name equivalence}

We take name equivalence, written $\nameeq$, to be the smallest
equivalence relation generated by the following rules.

\begin{mathpar}
\inferrule*[lab=Quote-drop]
{ }
{ \quotep{@{x}} \nameeq x }

\inferrule*[lab=Struct-equiv]
{ P \scong Q }
{ \quotep{P} \nameeq \quotep{Q} }
\end{mathpar}

The astute reader will have noticed that the mutual recursion of names
and processes imposes a mutual recursion on alpha-equivalence and
structural equivalence via name-equivalence. Fortunately, all of this
works out pleasantly and we may calculate in the natural way, free of
concern. The reader interested in the details is referred to the
appendix \ref{appendix:rho_details}.

\subsection{Substitution}

We use $\Proc$ for the set of processes, $\QProc$ for the set of
names, and $\id{\{}\vec{y} / \vec{x} \id{\}}$ to denote partial maps,
$s : \QProc \rightarrow \QProc$. A map, $s$ lifts, uniquely, to a map
on process terms, $\widehat{s} : \Proc \rightarrow \Proc$ by the
following equations.

\begin{mathpar}
  (0) \psubstp{Q}{P} := 0 \\
  (R \juxtap S) \psubstp{Q}{P}
  :=    
  (R)\psubstp{Q}{P} \juxtap (S) \psubstp{Q}{P} \\
  (x?(y).R) \psubstp{Q}{P}    
  :=    
  (x)\substp{Q}{P} (z)\concat( (R \psubstn{z}{y}) \psubstp{Q}{P} ) \\
  (\lift{x}{R}) \psubstp{Q}{P}  
  :=
  \lift{(x)\substp{Q}{P}}{ R \psubstp{Q}{P} } \\
%   (\dropn{x})  \psubstp{Q}{P}       
%   := 
%   \left\{ 
%     \begin{array}{ccc} 
%       \dropn{\quotep{Q}} & & x \nameeq \quotep{P} \\
%       \dropn{x} & & otherwise \\
%     \end{array}
%   \right. 
  (\dropn{x})  \psubstp{Q}{P}       
  := 
  \left\{ 
    \begin{array}{ccc} 
      Q & & x \nameeq \quotep{P} \\
      \dropn{x} & & otherwise \\
    \end{array}
  \right.
\end{mathpar}
 

where

\begin{eqnarray}
  (x)\id{\{} \lpquote Q \rpquote / \lpquote P \rpquote \id{\}}            = 
  \left\{ 
    \begin{array}{ccc}
      \lpquote Q \rpquote & & x \nameeq \lpquote P \rpquote \\
      x & & otherwise \\
    \end{array}
  \right. \nonumber
\end{eqnarray}

and $z$ is chosen distinct from $\quotep{P}$, $\quotep{Q}$, the free
names in $Q$, and all the names in $R$. Our $\alpha$-equivalence will
be built in the standard way from this substitution.

\begin{remark}\label{rem:no_self_referential_names}
  One consequence of these definitions is that $\forall P. \quotep{P}
  \not\in \freenames{P}$.
\end{remark}

\subsection{ Dynamic quote: an example }

Anticipating something of what's to come, consider applying the
substitution, $\widehat{\id{\{}u / z \id{\}}}$, to the following pair
of processes, $\lift{w}{y!(z)}$ and $w[ \lpquote y!(z) \rpquote ]$.

\begin{eqnarray}
	\lift{w}{y!(z)}\widehat{\id{\{}u / z \id{\}}}
		& = &
		\lift{w}{y!(u)} \nonumber\\
	w[ \lpquote y!(z) \rpquote ] \widehat{ \id{\{}u / z \id{\}} }
		& = &
		w[ \lpquote y!(z) \rpquote ] \nonumber
\end{eqnarray}

Because the body of the process between quotes is impervious to
substitution, we get radically different answers. In fact, by
examining the first process in an input context,
e.g. $x?(z).\lift{w}{y!(z)}$, we see that the process under the lift
operator may be shaped by prefixed inputs binding a name inside it. In
this sense, the lift operator will be seen as a way to dynamically
construct processes before reifying them as names.

Finally equipped with these standard features we can present the
dynamics of the calculus.

\subsubsection{Operational semantics} 

Finally, we introduce the computational dynamics. What marks these
algebras as distinct from other more traditionally studied algebraic
structures, e.g. vector spaces or polynomial rings, is the manner in
which dynamics is captured. In traditional structures, dynamics is typically
expressed through morphisms between such structures, as in linear maps
between vector spaces or morphisms between rings. In algebras
associated with the semantics of computation, the dynamics is
expressed as part of the algebraic structure itself, through a
reduction reduction relation typically denoted by $\red$. Below, we
give a recursive presentation of this relation for the calculus used
in the encoding.

$\red \subseteq \pi \times \pi$
$\red : \pi \to \mathcal{P}(\pi)$

\begin{mathpar}
  \inferrule* [lab=Comm] { \textsf{match}( x_{src}, x_{trgt} ) } { x_{trgt}?(y)P \; | \; x_{src}!\langle {Q} \rangle \red P\{\quotep{Q}/y}\} }
  \and \\
  \inferrule* [lab=Par] {{P} \red {P}'} {{{P} | {Q}} \red {{P}' | {Q}}}
  \and
  \inferrule* [lab=Equiv]{{{P} \scong {P}'} \andalso {{P}' \red {Q}'} \andalso {{Q}' \scong {Q}}}{{P} \red {Q}}
\end{mathpar}

\begin{eqnarray*}
  match_{\equiv} (\quotep{P},\quotep{Q}) & := & P \equiv Q \\
  match_{\dagger}(\quotep{P},\quotep{Q}) & := & \forall R. P|Q \red^{*} R => R \red^{*} 0 \\
  match_{K}(\quotep{P},\quotep{Q}) & := & K \mbox{ for some context } K
\end{eqnarray*}

$u?(x)P | u!\langle Q \rangle \red P\{\quotep{Q}/x\}$

%We write $\wred$ for $\red^*$, and $P\red$ if $\exists Q $ such that $ P \red Q$.
We write $P\red$ if $\exists Q $ such that $ P \red Q$ and $P\not\red$, otherwise.

\section{Replication}

As mentioned before, it is known that replication (and hence
recursion) can be implemented in a higher-order process algebra
\cite{SangiorgiWalker}. As our first example of calculation with the
machinery thus far presented we give the construction explicitly in
the {\rhoc}.

\begin{eqnarray}
	D_{x} & := & \prefix{x}{y}{(\binpar{\outputp{x}{y}}{@{y}})} \nonumber\\
	\bangp_{x}{P} & := & \binpar{{x}!\langle{\binpar{D_{x}}{P}}\rangle}{D_{x}} \nonumber
\end{eqnarray}

\begin{eqnarray}
	\bangp_{x}{P} & & \nonumber\\
	=
	& {x}!\langle{(\prefix{x}{y}{(\outputp{x}{y} | @{y})) | P}}\rangle 
	      | \prefix{x}{y}{(\outputp{x}{y} | @{y})} & \nonumber\\
	\red
	& (\outputp{x}{y} | @{y})\substn{\quotep{(\prefix{x}{y}{(@{y} | \outputp{x}{y})) | P}}}{y} & \nonumber\\
	=
	& \outputp{x}{\quotep{(\prefix{x}{y}{(\outputp{x}{y} | @{y})) | P}}}
	  | {(\prefix{x}{y}{(\outputp{x}{y} | @{y})) | P}} & \nonumber\\
	\red
	& \ldots & \nonumber\\
	\red^*
	& P | P | \ldots & \nonumber
\end{eqnarray}

Of course, this encoding, as an implementation, runs away, unfolding
$\bangp{P}$ eagerly. A lazier and more implementable replication
operator, restricted to input-guarded processes, may be obtained as follows.

\begin{eqnarray}
\bangp{\prefix{u}{v}{P}} 
	:= 
	\binpar{\lift{x}{\prefix{u}{v}{(\binpar{D(x)}{P})}}}{D(x)} \nonumber
\end{eqnarray}

\begin{remark}
  Note that the lazier definition still does not deal with summation
  or mixed summation (i.e. sums over input and output). The reader is
  invited to construct definitions of replication that deal with these
  features. 

  Further, the definitions are parameterized in a name, $x$. Can you,
  gentle reader, make a definition that eliminates this parameter and
  guarantees no accidental interaction between the replication
  machinery and the process being replicated -- i.e. no accidental
  sharing of names used by the process to get its work done and the
  name(s) used by the replication to effect copying. This latter
  revision of the definition of replication is crucial to obtaining
  the expected identity $!!P \sim !P$.
\end{remark}

\begin{remark}\label{rem:paradoxical_combinator}
  The reader familiar with the lambda calculus will have noticed the
  similarity between $D$ and the paradoxical combinator.

  [Ed. note: the existence of this seems to suggest we have to be more
  restrictive on the set of processes and names we admit if we are to
  support no-cloning.]
\end{remark}

\subsubsection{Bisimulation}

The computational dynamics gives rise to another kind of equivalence,
the equivalence of computational behavior. As previously mentioned
this is typically captured \emph{via} some form of bisimulation.

% The notion we use in this paper is weak barbed bisimulation
% \cite{milner91polyadicpi}.

The notion we use in this paper is derived from weak barbed
bisimulation \cite{milner91polyadicpi}. 

\begin{definition}
An \emph{observation relation}, $\downarrow_{\mathcal N}$, over a set
of names, $\mathcal N$, is the smallest relation satisfying the rules
below.

\infrule[Out-barb]{y \in {\mathcal N}, \; x \nameeq y}
		  {\outputp{x}{v} \downarrow_{\mathcal N} x}
\infrule[Par-barb]{\mbox{$P\downarrow_{\mathcal N} x$ or $Q\downarrow_{\mathcal N} x$}}
		  {\binpar{P}{Q} \downarrow_{\mathcal N} x}

We write $P \Downarrow_{\mathcal N} x$ if there is $Q$ such that 
$P \wred Q$ and $Q \downarrow_{\mathcal N} x$.
\end{definition}

\begin{definition}
%\label{def.bbisim}
An  ${\mathcal N}$-\emph{barbed bisimulation} over a set of names, ${\mathcal N}$, is a symmetric binary relation 
${\mathcal S}_{\mathcal N}$ between agents such that $P\rel{S}_{\mathcal N}Q$ implies:
\begin{enumerate}
\item If $P \red P'$ then $Q \wred Q'$ and $P'\rel{S}_{\mathcal N} Q'$.
\item If $P\downarrow_{\mathcal N} x$, then $Q\Downarrow_{\mathcal N} x$.
\end{enumerate}
$P$ is ${\mathcal N}$-barbed bisimilar to $Q$, written
$P \wbbisim_{\mathcal N} Q$, if $P \rel{S}_{\mathcal N} Q$ for some ${\mathcal N}$-barbed bisimulation ${\mathcal S}_{\mathcal N}$.
\end{definition}

$\mathcal{R} \subseteq \pi \times \pi$

$P \mathcal{R} Q => \forall P'. P \red P' \Rightarrow \exists Q'. Q \red Q', P' \mathcal{R} Q'$

$P \vdash x \Rightarrow Q \vdash x$

\begin{mathpar}
  \inferrule*[lab=Out-barb]{x \nameeq y}{{y}!\langle{Q}\rangle \vdash x}
  \and
  \inferrule*[lab=Par-barb]{\mbox{$P\vdash x$ or $Q\vdash x$}}{\binpar{P}{Q} \vdash x}
\end{mathpar}

\subsubsection{Contexts}

One of the principle advantages of computational calculi like the
$\pi$-calculus is a well-defined notion of context,
contextual-equivalence and a correlation between
contextual-equivalence and notions of bisimulation. The notion of
context allows the decomposition of a process into (sub-)process and
its syntactic environment, its context. Thus, a context may be
thought of as a process with a ``hole'' (written $\Box$) in it. The
application of a context $M$ to a process $P$, written $M[P]$, is
tantamount to filling the hole in $M$ with $P$. In this paper we do
not need the full weight of this theory, but do make use of the notion
of context in the proof the main theorem. 

\begin{mathpar}
  \inferrule* [lab=summation] {} {{M_{M},M_{N}} \bc \Box \;|\; x.M_{A} \;|\; M_{M}+M_{N}}
  \and
  \inferrule* [lab=agent] {} {{M_{A}} \bc (\vec{x})M_{P} \;| \; \clift{P_0,\ldots,M_{P},\ldots,P_N}}
  \and \\
  \inferrule* [lab=process] {} {{M_{P}} \bc M_{N} \;| \;P|M_{P} }
\end{mathpar} 

\begin{mathpar}
  \inferrule* [lab=sychronization] {} {M_{N} \bc \Box \;|\; x?M_{F} \;|\; x!M_{C}}
  \and
  \inferrule* [lab=abstraction] {} {{M_{F}} \bc (x)M_{P} }
  \and
  \inferrule* [lab=concretion] {} {{M_{C}} \bc \langle M_{P} \rangle }
  \and \\
  \inferrule* [lab=process] {} {{M_{P}} \bc M_{N} \;| \;P|M_{P} }
\end{mathpar}

\begin{definition}[contextual application] Given a context $M$, and
  process $P$, we define the \emph{contextual application}, $M[P] :=
  M\{P/\Box\}$. That is, the contextual application of M to P is the
  substitution of $P$ for $\Box$ in $M$.
\end{definition}

$\meaningof{-} : L \to \mathcal{P}(\pi)$

\begin{mathpar}
  \inferrule* [lab=collection] {} {\meaningof{true} = \pi, \and \meaningof{~E} = \pi \setminus \meaningof{E}, \and \meaningof{E_{1} \& E_{2}} = \meaningof{E_{1}} \cap \meaningof{E_{2}}}
\end{mathpar}

\begin{mathpar}
  \inferrule* [lab=structure] {} {\meaningof{0} = \{ P \in \pi | P \equiv 0 \}, \and \\ \meaningof{E_1 | E_2} = \{ P \in \pi | P \equiv P_{1} | P_{2}, P_{1} \in \meaningof{E_{1}}, P_{2} \in \meaningof{E_2}\} }
\end{mathpar}

\begin{mathpar}
 \inferrule* [lab=behavior] {} {\meaningof{\langle a?b \rangle E} = \{ P \in \pi | P \equiv Q | u?(y)P', \\ \and \\\\ \and \\ \;\;\; u \in \meaningof{a}, \forall z.P'\{z/y\} \in \meaningof{E\{z/b\}}\}, \and \\ \meaningof{a!E} = \{ P \in \pi | P \equiv Q | x!\langle P' \rangle, x \in \meaningof{a} P' \in \meaningof{E}\} }
\end{mathpar}

\begin{mathpar}
 \inferrule* [lab=nominal] {} {\meaningof{\quotep{E}} = \{ \quotep{P} \in \quotep{\pi} | P \in \meaningof{E} \}, \and \meaningof{\quotep{P}} = \{ \quotep{Q} \in \quotep{\pi} | P \equiv Q \} \and \\ \meaningof{@\quotep{E}} = \{ P \in \pi | P \equiv @x, x \in \meaningof{E} \}}
\end{mathpar}

\begin{eqnarray*}
  \\
  \meaningof{-} : TS \to ST
\end{eqnarray*}

\begin{eqnarray*}
  \\
  L : TS \to ST
\end{eqnarray*}

\begin{eqnarray*}
  \\
  P \models E \iff P \in \meaningof{E}
\end{eqnarray*}

\begin{eqnarray*}
  P \approx_{L} Q \iff \forall E \in L. P \models E \iff Q \models E
\end{eqnarray*}

\begin{eqnarray*}
  P \approx_{K} Q
\end{eqnarray*}

\begin{eqnarray*}
  P \approx Q
\end{eqnarray*}

$\approx_{K} = \approx = \approx_{L}$

\subsubsection{Contextual duality}

Note that contexts extend the quotation operation to a family of
operations from processes to names. Given a context, $M$, we can
define a \emph{nominal context}, $\quotep{M}$ by $\quotep{M}[P] :=
\quotep{M[P]}$. To foreshadow what is to come we observe that these
operations enjoy a duality with processes very much like the duality
between vectors and maps from vectors to scalars.

Further, because the calculus is essentially higher-order, we have a
correspondence between contexts and processes. More specifically,
given a name $x$ and a context $M$ we can construct $M^{*}_{x}$ such
that 

\begin{mathpar}
  M^{*}_{x} | \lift{x}{P} \red M[P]
\end{mathpar}

namely,

\begin{mathpar}
  M^{*}_{x} := x?(u).M[\dropn{u}]
\end{mathpar}

The dependence of $M^{*}_{x}$ on a name makes it an abstraction, 

\begin{mathpar}
  M^{*} := (x)x?(u).M[\dropn{u}]
\end{mathpar}

\subsection{Additional notation}

It will sometimes be convenient to denote the process a name
quotes. We already have the notation $x = \quotep{P}$, but it will be
convenient to introduce an alternate notation, $\procn{x}$, when we
want to emphasize the connection to the use of the name. Note that, by
virtue of name equivalence, $\quotep{\procn{x}} \nameeq x$; so, the
notation is consistent with previous definitions.

Further, because names have structure it is possible to effect
substitutions on the basis of that structure. This means we need to
upgrade our notation for substitutions, which we accomplish by
adapting comprehension notation. Thus,

\begin{mathpar}
  P\{ y / x : x \in S \}
\end{mathpar}

is interpreted to mean the process derived from P by replacing (in a
capture-avoiding manner) each occurrence of $x$ in $S$ by $y$. For example,

\begin{mathpar}
  P\{ \quotep{\procn{x}|\procn{x}} / x : x \in \freenames{P} \}
\end{mathpar}

will replace each (occurrence) of a free name $x$ in $P$ by
$\quotep{\procn{x}|\procn{x}}$.

Also, we will avail ourselves of the notation $x^{L}$ and $x^{R}$ to
denote injections of a name into disjoint copies of the name
space. There are numerous ways to accomplish this. One example can be
found in \cite{MeredithR05}. This notation overloads to vectors of
names: $\vec{x}^{\pi} := (x_{i}^{\pi} \; : \; 0 \leq i < |\vec{x}| )$ where $\pi \in \{L,R\}$.

We also use $P^{\Box} := P|\Box$.

In \cite{MeredithR05} an interpretation of the new operator is
given. It turns out that there are several possible interpretations
all enjoying the requisite algebraic properties of the operator (see
\cite{milner91polyadicpi}). We will therefore make liberal use of
$(\nu\; \vec{x})P$.

% subsection the_syntax_and_semantics_of_the_notation_system (end)   

\input{qm2pi.qmops} 

\input{qm2pi.sterngerlach} 

\input{qm2pi.metric} 

% section concurrent_process_calculi (end)

%\input{qm2pi.proofsketch}

% section proof sketch (end)

%\input{qm2pi.slviaknots} 

% section spatial logic via knots (end)

\input{qm2pi.conclusion}

% section conclusion (end)

%\input{qm2pi.dtcodes} 

% section wiring algorithm (end)

\input{qm2pi.ack} 

% section acknowledgments (end)

\newpage


\bibliographystyle{plain}   
\bibliography{../../biblios/main.bib}

\input{qm2pi.rhodetails}

\end{document}



% section proof sketch (end)

%\section{Unlikely characters: spatial logic for
  knots}\label{sub:characteristic_formulae} % (fold)

Associated to the mobile process calculi are a family of logics known
as the Hennessy-Milner logics. These logics typically enjoy a
semantics interpreting formulae as sets of processes that when
factored through the encoding outlined above allows an identification
of classes of knots with logical formulae. In the context of this
encoding the sub-family known as the spatial logics \cite{CairesC03}
\cite{CairesC04} \cite{Caires04} are of particular interest providing
several important features for expressing and reasoning about
properties (i.e. classes) of knots. We hint here at how this may be done.

%\begin{description}
%\item [structural connectives] 
\subsubsection{Structural connectives} The spatial logics enjoy
structural connectives corresponding, at the logical level, to the
parallel composition ($P | Q$) and new name ($(\nu \; x)P$)
connectives for processes. As illustrated in the examples below, these
connectives are extremely expressive given the shape of our encoding.
%\item [decideable satisfaction]

\subsubsection{Decideable satisfaction}
In \cite{Caires04} the satisfaction relation is shown to be decideable
for a rich class of processes. It further turns out that the image of
the our encoding is a proper subset of that class. This result
provides the basis for an algorithm by which to search for knots
enjoying a given property.
%\item [characteristic formulae]

\subsubsection{Characteristic formulae}
In the same paper \cite{Caires04} , Caires presents a means of calculating
characteristic formulae, selecting equivalence classes of processes
up to a pre--specified depth limit on the support set of names. Composed with our
encoding, this characteristic formula can be used to select
characteristic formulae for knots.
%\end{description}

\subsubsection{Spatial logic formulae}

The grammar below (segmented for comprehension) summarizes the syntax
of spatial logic formulae. We employ illustrative examples in the
sequel to provide an intuitive understanding of their meaning
referring the reader to \cite{Caires04} for a more detailed explication
of the semantics.

\begin{mathpar}
  \inferrule* [lab=boolean] {} {{A,B} \bc T \;|\; \neg A \;|\; A \wedge B \;|\; \eta = \eta'}
  \and
  \inferrule* [lab=spatial] {} {|\; \pzero \;|\; A | B \;|\; x \text{\textregistered} A \;|\; \forall x . A \;|\;  H x . A}
  \and
  \inferrule* [lab=behavioral] {} {|\; \alpha . A}
  \and 
  \inferrule* [lab=recursion] {} {|\; X(\vec{u}) \;|\; \mu X(\vec{u}) . A}
  \and
  \inferrule* [lab=action] {} {\alpha \bc \langle x?(\vec{y}) \rangle \;|\; \langle x!(\vec{y}) \rangle \;|\; \langle \tau \rangle}
  \and 
  \inferrule* [lab=name] {} {\eta \bc x \;|\; \tau}
\end{mathpar} 

% subsection characteristic_formulae (end)   	 

\subsection{Example formulae}\label{sub:example_formulae_} % (fold)

\subsubsection{Crossing as formula.}
% 
% \begin{align*}
%   \frac{d}{dx} \sin x &= \cos x 
%   & \frac{d}{dx} e^x &= e^x \\
%   \frac{d}{dx} \cos x &= - \sin x 
%   & \frac{d}{dx} \log x &= \frac{1}{x} \\
% \end{align*} 

\begin{align*}
 \mu C(x_{0},x_{1},y_{0},y_{1},u).&(\langle x_{0}?(z) \rangle(\langle u! \rangle\langle y_{1}!z \rangle C(x_{0},x_{1},y_{0},y_{1},u)) & \\
  & \wedge \langle y_{1}?(z) \rangle (\langle u! \rangle \langle x_{0}!z \rangle C(x_{0},x_{1},y_{0},y_{1},u)) & \\
  & \wedge \langle x_{1}?(z) \rangle (\langle u? \rangle \langle y_{0}!z \rangle C(x_{0},x_{1},y_{0},y_{1},u)) & \\
  & \wedge \langle y_{0}?(z) \rangle (\langle u? \rangle \langle x_{1}!z \rangle C(x_{0},x_{1},y_{0},y_{1},u))) &
\end{align*}

The lexicographical similarity between the shape of this formulae and
the shape of definition of the process representing a crossing reveals
the intuitive meaning of this formulae. It describes the capabilities
of a process that has the right to represent a crossing. For example
it picks out processes that may perform an input on the port $x_0$ in
its initial menu of capabilities. What differentiates the formula
from the process, however, is that the crossing process is the
smallest candidate to satisfy the formula. Infinitely many other
processes -- with internal behavior hidden behind this interface, so
to speak -- also satisfy this formula. Even this simple formula,
then, can be seen to open a new view onto knots, providing a
computational interpretation of \emph{virtual} knots.

Note that this formula is derived by hand. A similar formula can be
derived by employing Caires' calculation of characteristic formula
\cite{Caires04} to the process representing a crossing. In light of
this discussion, we let
$\meaningof{C}_{\phi}(x0,x1,y0,y1,u)$ denote a formula specifying the
dynamics we wish to capture of a crossing. To guarantee we preserve
the shape of the interface and minimal semantics we demand that
$\meaningof{C}_{\phi}(x0,x1,y0,y1,u) \Rightarrow
\textbf{C}(x0,x1,y0,y1,u)$ where $\textbf{C}(x0,x1,y0,y1,u)$ denotes
the formula above.
                            
\subsubsection{Crossing number constraints.}
The moral content of the context lemma (Lemma \ref{context}) is that the notion of
``locality'' in the Reidemeister moves is effectively captured by the
parallel composition operator of the process calculus. This intuition
extends through the logic. Given a formula,
$\meaningof{C}_{\phi}(x0,x1,y0,y1,u)$, we can use the structural
connectives to specify constraints on crossing numbers, such as at
least $n$ crossings, or exactly $n$ crossings.
\begin{mathpar}
  \inferrule* [lab=at-least-n] {} { K^{\geq n}_{\phi}(\vec{xs},\vec{ys}) := \Pi_{i=0}^{n-1} Hu . \meaningof{C}_{\phi}(xs_i,ys_i,u) | T }
  \and 
  \inferrule* [lab=exactly-n] {} { K^{= n}_{\phi}(\vec{xs},\vec{ys}) := \Pi_{i=0}^{n-1} Hu . \meaningof{C}_{\phi}(xs_i,ys_i,u) | \neg (\forall x_0,y_0,x_1,y_1,u . \meaningof{C}_{\phi}(x_0,y_0,x_1,y_1,u) | T) }
\end{mathpar}

To round out this section, recall that the encoding of an $n$-crossing
knot decomposes into a parallel composition of $n$ \emph{copies} of a
crossing process together with a wiring harness. To specify different
knot classes with the same crossing number amounts to specifying
logical constraints on the wiring harness. In the interest of space,
we defer examples to a forthcoming paper. Suffice it to say that both
the conditions ``alternating knot'' and ``contains the tangle
corresponding to 5/3'' are expressible. For example, it is possible to
calculate the characteristic formula of a process corresponding to the
tangle 5/3 and conjoin it into the classifying formula via the
composition connective of the logic.

Finally, we wish to observe that it is entirely within reason to
contemplate a more domain-specific version of spatial logic tailored
to the shape of processes in the image of the encoding. Such a
domain-specific logic would have a better claim to the title formal
language of knot properties.

% subsection example_formulae_ (end)

% section knots_as_processes (end) 

% section spatial logic via knots (end)

\section{Conclusions and future work}

\paragraph{Testing physical space}
You, gentle reader, may wonder why of all the theorems to be proved
given this set up we pick the one above. In some sense it's hardly
central to quantum mechanics. We see it as central in the sense that
it firmly establishes a notion of physical space arising from a notion
of the equivalence of behavior. Relating bisimulation to a metric is a
big step forward, but one is faced with interpreting the relationship
of that metric space to something more physical. Quantum mechanical
notions of ``physical'' space are still far from intuitive, but by
relating this idea of distance as testing to calculations that predict
physical circumstances we are making a not insignificant step forward
toward an understanding of the physical space we inhabit as
essentially dynamic.

\paragraph{Effectivity and simulation}
One of the observations we have yet to make is that the entire program
spelled out here is effective. We have built various interpreters for
the reflective calculus at work in this interpretation. In principle,
then, we can simulate quantum mechanics on a computer. The place where
the simulation may lose fidelity is the infinitely branching summation
for the annihilator.

In this connection i also want to point out that the evaluation style
calculation of the inner product puts the non-determinism of the
summation right at the heart of measurement. This suggests that
Milner's original reduction-based formulation of the dynamics of his
calculi in terms of sums was not just notationally suggestive of a
notion of measure-and-continue but captured some significant part of
the physics.

\paragraph{Quantum continuations}
In light of this last observation i want to point out that the
predominant account of quantum mechanics is missing a key aspect of a
truly compositional story of the physical situation. In a real lab,
when a measurement is made the observation can be made to feed into
another device that then makes another measurement conditioned on the
results of the first. This means that after the superposition was
collapsed the entire experimental set up remained in
superposition. While QM offers a means of writing this down it doesn't
quite line up well with the well-trodden formulation of computation
and continuation that we see so succinctly expressed in Milner's
calculi. This suggests that there might be advantages to this account
of dynamics waiting to be explored.

\paragraph{Quantum logic}
In this connection, we also note that by virtue of having the
Hennessy-Milner construction, we can pull the construction through the
interpretation of QM. This gives us a natural candidate for a quantum
logic that enjoys an extremely tight connection with it's domain of
interpretation, making the construction much less ad hoc (rather it is
the image of functor!).

\paragraph{Quantum probabiity}
i have questions about the basis of the interpretation of inner
product as probability amplitude. In particular, using which
axiomatization of probability theory does the notion of probability
amplitude earn the right to be so dubbed? In other words, where is the
proof that the operation for calculating a probability amplitude (and
then squaring) satisfies the axioms of what it means to calculate a
probability? Even if such a proof exists (i have yet to find it in the
literature), i wonder if it might not be possible to turn things on
their heads. Can we view the calculation of the probability amplitude
as an axiomatization of probability? If so, then the definition we
give for calculating probability amplitude may provide the basis for
an \emph{effective} theory of probability.

\paragraph{Quantum vs ``biological'' information}
Finally, i want to conclude with a more philosophical observation. At
a recent workshop in which QM was a predominant topic i noticed
something about quantum information. The speaker was giving a riveting
discussion of axiomatic QM and showing how properties of ``no
cloning'' and ``no deleting'' emerged as consequences of the
axiomatization. Theorems of this form are necessary to give us a sense
of confidence that our axioms characterize the physical theory. What
struck me, though, was that if quantum information is neither erasable
nor replicable it is markedly different from \emph{life}. Two of the
things we know about life is that

\begin{itemize}
  \item it ends;
  \item to gain some measure of persistence, to transcend it's
    finitude it is imminently copyable.
\end{itemize}

Both of these qualities are summarized succinctly in the aphorism: all
flesh is grass. For me these two kinds of ``information'' -- call them
quantum and biological -- are end points on a spectrum of strategies
for persistence. At one end, we have those curious entities that enjoy
uniqueness and permanence; at the other, we have those who in the face
of a certain end and an uncertain present make a go of passing
something on. To me one of the more remarkable aspects of the latter
strategy is that in the presence of noise (and certain features of
copying) we get a kind of dynamism, a chance for improvement against a
given persistent condition.

% subsection other_calculi_other_bisimulations_and_geometry_as_behavior (end)




% section conclusion (end)

%\documentclass[12pt]{llncs}
%\documentclass{jktr}

\usepackage[pdftex]{hyperref}                   
\usepackage {listings}
\usepackage {mathpartir}
\usepackage{bcprules}
%\usepackage{listings}
                       
\usepackage{graphicx} 
%\usepackage[margins=2.5cm,nohead,nofoot]{geometry}
%\usepackage{geometry}
\usepackage{amsfonts}
\usepackage{amstext}
\usepackage{latexsym}
\usepackage{amssymb}
\usepackage{color}


%\include{myPreamble}
\include{qm2pi.local} 

%\ifpdf
%\usepackage[pdftex]{graphicx}
%\else
%\usepackage{graphicx}
%\fi

 % \ifpdf
%  \usepackage{pdfsync}
%  \if


%\title{Brief Article}
%\author{David F. Snyder}
%\author{L.G. Meredith}

%\address{Dept. of Math., Texas State University--San Marcos, San Marcos, TX 78666}
       
\pagestyle{empty}


\begin{document}

\lstset{language=[Objective]Caml,frame=shadowbox}

\input{qm2pi.front}

% section front matter (end)

\input{qm2pi.intro} 
 
% section introduction (end)

% \input{qm2pi.knotations} 

% section notation (end)

\input{qm2pi.process.calculi} 

% section concurrent_process_calculi_and_spatial_logics_ (end)
    
%\input{qm2pi.knots2pi} 

%\input{qm2pi.trefoil} 

%\input{qm2pi.mainthm} 

% subsection basic_interpretation (end)

%\input{qm2pi.rho.presentation} 
\subsection{The syntax and semantics of the notation system}\label{sub:the_syntax_and_semantics_of_the_notation_system} % (fold)

We now summarize a technical presentation of the calculus that
embodies our theory of dynamics. The typical presentation of such a
calculus follows the style of giving generators and relations on
them. The grammar, below, describing term constructors, freely
generates the set of processes, $\Proc$. This set is then quotiented
by a relation known as structural congruence and it is over this set
that the notion of dynamics is expressed. This presentation is
essentially that of \cite{MeredithR05} with the addition of
polyadicity and summation. For readability we have relegated some of
the technical subtleties to an appendix.

\subsubsection{Process grammar}\label{subsub:process_grammar}

\begin{mathpar}
  \inferrule* [lab=synchronization] {} {{M} \bc \pzero \;|\; x?F \;|\; x!C }
  \and
  \inferrule* [lab=abstraction] {} {{F} \bc (x)P}
  \and
  \inferrule* [lab=concretion] {} {{C} \bc \langle Q \rangle}
  \and
  \inferrule* [lab=process] {} {{P,Q} \bc M \;| \;P|Q \;|\; @{x}}
  \and
  \inferrule* [lab=name] {} {{x} \bc \quotep{P}}
\end{mathpar} 

Note that $\vec{x}$ (resp. $\vec{P}$) denotes a vector of names
(resp. processes) of length $|\vec{x}|$ (resp. $|\vec{P}|$). We adopt
the following useful abbreviations.

\begin{mathpar}
   x?(\vec{y}).P := x.(\vec{y})P \and  x\clift{\vec{P}} := x.\clift{\vec{P}}
   \and x!(y) := \lift{x}{\dropn{y}}
   \and \Pi_{i=0}^{n-1}P_i := P_0 | \ldots | P_{n-1}
\end{mathpar}

\subsubsection{Structural congruence}

\paragraph{Free and bound names and alpha-equivalence.} At the
core of structural equivalence is alpha-equivalence which identifies
process that are the same up to a change of variable. Formally, we
recognize the distinction between free and bound names. The free names
of a process, $\freenames{P}$, may be calculated recursively as
follows:

\begin{mathpar}
\freenames{\pzero} := \emptyset
  \and \\
  \freenames{x?(y).P} := \{ x \} \cup (\freenames{P} \setminus \{ y \})
  \and 
  \freenames{x!\langle P \rangle} := \{ x \} \cup \{ P \} 
  \and \\
  \freenames{P|Q} := \freenames{P} \cup \freenames{Q}
  \and \\
  \freenames{@{x}} := \{ x \}
\end{mathpar}

$\pi$
$\quotep{\pi}$

$\freenames{-} : \pi \to \mathcal{P}(\quotep{\pi})$

\begin{eqnarray*}
  \freenames{\pzero} & := & \emptyset \\
  \freenames{x?(y).P} & := & \{ x \} \cup (\freenames{P} \setminus \{ y \}) \\
  \freenames{x!\langle P \rangle} & := & \{ x \} \cup \{ P \} \\
  \freenames{P|Q} & := & \freenames{P} \cup \freenames{Q} \\
  \freenames{\dropn{x}} & := & \{ x \}
\end{eqnarray*}

The bound names of a process, $\boundnames{P}$, are those names occurring in $P$
that are not free. For example, in $x?(y).0$, the name $x$ is free, while $y$ is bound.

\begin{mathpar}
  \inferrule* [lab=monoidal-laws] {} { P|Q \equiv Q|P \and P|0 \equiv P \and P|(Q|R) \equiv (P|Q)|R }
\end{mathpar}

\begin{mathpar}
  \inferrule* [lab=alpha-equivalence] {} { (x)P \equiv (y)P\{y/x\} \and y \not\in \freenames{P} }
\end{mathpar}

\begin{definition}
Then two processes, $P,Q$, are alpha-equivalent if $P = Q\{\vec{y}/\vec{x}\}$ for
some $\vec{x} \in \boundnames{Q},\vec{y} \in \boundnames{P}$, where $Q\{\vec{y}/\vec{x}\}$
denotes the capture-avoiding substitution of $\vec{y}$ for $\vec{x}$ in $Q$.
\end{definition}

\begin{definition}
  The {\em structural congruence} \cite{SangiorgiWalker} , $\equiv$,
  between processes is the least congruence containing
  alpha-equivalence, satisfying the abelian monoid laws
  (associativity, commutativity and $\pzero$ as identity) for parallel
  composition $|$ and for summation $+$.
\end{definition}

\subsection{Name equivalence}

We take name equivalence, written $\nameeq$, to be the smallest
equivalence relation generated by the following rules.

\begin{mathpar}
\inferrule*[lab=Quote-drop]
{ }
{ \quotep{@{x}} \nameeq x }

\inferrule*[lab=Struct-equiv]
{ P \scong Q }
{ \quotep{P} \nameeq \quotep{Q} }
\end{mathpar}

The astute reader will have noticed that the mutual recursion of names
and processes imposes a mutual recursion on alpha-equivalence and
structural equivalence via name-equivalence. Fortunately, all of this
works out pleasantly and we may calculate in the natural way, free of
concern. The reader interested in the details is referred to the
appendix \ref{appendix:rho_details}.

\subsection{Substitution}

We use $\Proc$ for the set of processes, $\QProc$ for the set of
names, and $\id{\{}\vec{y} / \vec{x} \id{\}}$ to denote partial maps,
$s : \QProc \rightarrow \QProc$. A map, $s$ lifts, uniquely, to a map
on process terms, $\widehat{s} : \Proc \rightarrow \Proc$ by the
following equations.

\begin{mathpar}
  (0) \psubstp{Q}{P} := 0 \\
  (R \juxtap S) \psubstp{Q}{P}
  :=    
  (R)\psubstp{Q}{P} \juxtap (S) \psubstp{Q}{P} \\
  (x?(y).R) \psubstp{Q}{P}    
  :=    
  (x)\substp{Q}{P} (z)\concat( (R \psubstn{z}{y}) \psubstp{Q}{P} ) \\
  (\lift{x}{R}) \psubstp{Q}{P}  
  :=
  \lift{(x)\substp{Q}{P}}{ R \psubstp{Q}{P} } \\
%   (\dropn{x})  \psubstp{Q}{P}       
%   := 
%   \left\{ 
%     \begin{array}{ccc} 
%       \dropn{\quotep{Q}} & & x \nameeq \quotep{P} \\
%       \dropn{x} & & otherwise \\
%     \end{array}
%   \right. 
  (\dropn{x})  \psubstp{Q}{P}       
  := 
  \left\{ 
    \begin{array}{ccc} 
      Q & & x \nameeq \quotep{P} \\
      \dropn{x} & & otherwise \\
    \end{array}
  \right.
\end{mathpar}
 

where

\begin{eqnarray}
  (x)\id{\{} \lpquote Q \rpquote / \lpquote P \rpquote \id{\}}            = 
  \left\{ 
    \begin{array}{ccc}
      \lpquote Q \rpquote & & x \nameeq \lpquote P \rpquote \\
      x & & otherwise \\
    \end{array}
  \right. \nonumber
\end{eqnarray}

and $z$ is chosen distinct from $\quotep{P}$, $\quotep{Q}$, the free
names in $Q$, and all the names in $R$. Our $\alpha$-equivalence will
be built in the standard way from this substitution.

\begin{remark}\label{rem:no_self_referential_names}
  One consequence of these definitions is that $\forall P. \quotep{P}
  \not\in \freenames{P}$.
\end{remark}

\subsection{ Dynamic quote: an example }

Anticipating something of what's to come, consider applying the
substitution, $\widehat{\id{\{}u / z \id{\}}}$, to the following pair
of processes, $\lift{w}{y!(z)}$ and $w[ \lpquote y!(z) \rpquote ]$.

\begin{eqnarray}
	\lift{w}{y!(z)}\widehat{\id{\{}u / z \id{\}}}
		& = &
		\lift{w}{y!(u)} \nonumber\\
	w[ \lpquote y!(z) \rpquote ] \widehat{ \id{\{}u / z \id{\}} }
		& = &
		w[ \lpquote y!(z) \rpquote ] \nonumber
\end{eqnarray}

Because the body of the process between quotes is impervious to
substitution, we get radically different answers. In fact, by
examining the first process in an input context,
e.g. $x?(z).\lift{w}{y!(z)}$, we see that the process under the lift
operator may be shaped by prefixed inputs binding a name inside it. In
this sense, the lift operator will be seen as a way to dynamically
construct processes before reifying them as names.

Finally equipped with these standard features we can present the
dynamics of the calculus.

\subsubsection{Operational semantics} 

Finally, we introduce the computational dynamics. What marks these
algebras as distinct from other more traditionally studied algebraic
structures, e.g. vector spaces or polynomial rings, is the manner in
which dynamics is captured. In traditional structures, dynamics is typically
expressed through morphisms between such structures, as in linear maps
between vector spaces or morphisms between rings. In algebras
associated with the semantics of computation, the dynamics is
expressed as part of the algebraic structure itself, through a
reduction reduction relation typically denoted by $\red$. Below, we
give a recursive presentation of this relation for the calculus used
in the encoding.

$\red \subseteq \pi \times \pi$
$\red : \pi \to \mathcal{P}(\pi)$

\begin{mathpar}
  \inferrule* [lab=Comm] { \textsf{match}( x_{src}, x_{trgt} ) } { x_{trgt}?(y)P \; | \; x_{src}!\langle {Q} \rangle \red P\{\quotep{Q}/y}\} }
  \and \\
  \inferrule* [lab=Par] {{P} \red {P}'} {{{P} | {Q}} \red {{P}' | {Q}}}
  \and
  \inferrule* [lab=Equiv]{{{P} \scong {P}'} \andalso {{P}' \red {Q}'} \andalso {{Q}' \scong {Q}}}{{P} \red {Q}}
\end{mathpar}

\begin{eqnarray*}
  match_{\equiv} (\quotep{P},\quotep{Q}) & := & P \equiv Q \\
  match_{\dagger}(\quotep{P},\quotep{Q}) & := & \forall R. P|Q \red^{*} R => R \red^{*} 0 \\
  match_{K}(\quotep{P},\quotep{Q}) & := & K \mbox{ for some context } K
\end{eqnarray*}

$u?(x)P | u!\langle Q \rangle \red P\{\quotep{Q}/x\}$

%We write $\wred$ for $\red^*$, and $P\red$ if $\exists Q $ such that $ P \red Q$.
We write $P\red$ if $\exists Q $ such that $ P \red Q$ and $P\not\red$, otherwise.

\section{Replication}

As mentioned before, it is known that replication (and hence
recursion) can be implemented in a higher-order process algebra
\cite{SangiorgiWalker}. As our first example of calculation with the
machinery thus far presented we give the construction explicitly in
the {\rhoc}.

\begin{eqnarray}
	D_{x} & := & \prefix{x}{y}{(\binpar{\outputp{x}{y}}{@{y}})} \nonumber\\
	\bangp_{x}{P} & := & \binpar{{x}!\langle{\binpar{D_{x}}{P}}\rangle}{D_{x}} \nonumber
\end{eqnarray}

\begin{eqnarray}
	\bangp_{x}{P} & & \nonumber\\
	=
	& {x}!\langle{(\prefix{x}{y}{(\outputp{x}{y} | @{y})) | P}}\rangle 
	      | \prefix{x}{y}{(\outputp{x}{y} | @{y})} & \nonumber\\
	\red
	& (\outputp{x}{y} | @{y})\substn{\quotep{(\prefix{x}{y}{(@{y} | \outputp{x}{y})) | P}}}{y} & \nonumber\\
	=
	& \outputp{x}{\quotep{(\prefix{x}{y}{(\outputp{x}{y} | @{y})) | P}}}
	  | {(\prefix{x}{y}{(\outputp{x}{y} | @{y})) | P}} & \nonumber\\
	\red
	& \ldots & \nonumber\\
	\red^*
	& P | P | \ldots & \nonumber
\end{eqnarray}

Of course, this encoding, as an implementation, runs away, unfolding
$\bangp{P}$ eagerly. A lazier and more implementable replication
operator, restricted to input-guarded processes, may be obtained as follows.

\begin{eqnarray}
\bangp{\prefix{u}{v}{P}} 
	:= 
	\binpar{\lift{x}{\prefix{u}{v}{(\binpar{D(x)}{P})}}}{D(x)} \nonumber
\end{eqnarray}

\begin{remark}
  Note that the lazier definition still does not deal with summation
  or mixed summation (i.e. sums over input and output). The reader is
  invited to construct definitions of replication that deal with these
  features. 

  Further, the definitions are parameterized in a name, $x$. Can you,
  gentle reader, make a definition that eliminates this parameter and
  guarantees no accidental interaction between the replication
  machinery and the process being replicated -- i.e. no accidental
  sharing of names used by the process to get its work done and the
  name(s) used by the replication to effect copying. This latter
  revision of the definition of replication is crucial to obtaining
  the expected identity $!!P \sim !P$.
\end{remark}

\begin{remark}\label{rem:paradoxical_combinator}
  The reader familiar with the lambda calculus will have noticed the
  similarity between $D$ and the paradoxical combinator.

  [Ed. note: the existence of this seems to suggest we have to be more
  restrictive on the set of processes and names we admit if we are to
  support no-cloning.]
\end{remark}

\subsubsection{Bisimulation}

The computational dynamics gives rise to another kind of equivalence,
the equivalence of computational behavior. As previously mentioned
this is typically captured \emph{via} some form of bisimulation.

% The notion we use in this paper is weak barbed bisimulation
% \cite{milner91polyadicpi}.

The notion we use in this paper is derived from weak barbed
bisimulation \cite{milner91polyadicpi}. 

\begin{definition}
An \emph{observation relation}, $\downarrow_{\mathcal N}$, over a set
of names, $\mathcal N$, is the smallest relation satisfying the rules
below.

\infrule[Out-barb]{y \in {\mathcal N}, \; x \nameeq y}
		  {\outputp{x}{v} \downarrow_{\mathcal N} x}
\infrule[Par-barb]{\mbox{$P\downarrow_{\mathcal N} x$ or $Q\downarrow_{\mathcal N} x$}}
		  {\binpar{P}{Q} \downarrow_{\mathcal N} x}

We write $P \Downarrow_{\mathcal N} x$ if there is $Q$ such that 
$P \wred Q$ and $Q \downarrow_{\mathcal N} x$.
\end{definition}

\begin{definition}
%\label{def.bbisim}
An  ${\mathcal N}$-\emph{barbed bisimulation} over a set of names, ${\mathcal N}$, is a symmetric binary relation 
${\mathcal S}_{\mathcal N}$ between agents such that $P\rel{S}_{\mathcal N}Q$ implies:
\begin{enumerate}
\item If $P \red P'$ then $Q \wred Q'$ and $P'\rel{S}_{\mathcal N} Q'$.
\item If $P\downarrow_{\mathcal N} x$, then $Q\Downarrow_{\mathcal N} x$.
\end{enumerate}
$P$ is ${\mathcal N}$-barbed bisimilar to $Q$, written
$P \wbbisim_{\mathcal N} Q$, if $P \rel{S}_{\mathcal N} Q$ for some ${\mathcal N}$-barbed bisimulation ${\mathcal S}_{\mathcal N}$.
\end{definition}

$\mathcal{R} \subseteq \pi \times \pi$

$P \mathcal{R} Q => \forall P'. P \red P' \Rightarrow \exists Q'. Q \red Q', P' \mathcal{R} Q'$

$P \vdash x \Rightarrow Q \vdash x$

\begin{mathpar}
  \inferrule*[lab=Out-barb]{x \nameeq y}{{y}!\langle{Q}\rangle \vdash x}
  \and
  \inferrule*[lab=Par-barb]{\mbox{$P\vdash x$ or $Q\vdash x$}}{\binpar{P}{Q} \vdash x}
\end{mathpar}

\subsubsection{Contexts}

One of the principle advantages of computational calculi like the
$\pi$-calculus is a well-defined notion of context,
contextual-equivalence and a correlation between
contextual-equivalence and notions of bisimulation. The notion of
context allows the decomposition of a process into (sub-)process and
its syntactic environment, its context. Thus, a context may be
thought of as a process with a ``hole'' (written $\Box$) in it. The
application of a context $M$ to a process $P$, written $M[P]$, is
tantamount to filling the hole in $M$ with $P$. In this paper we do
not need the full weight of this theory, but do make use of the notion
of context in the proof the main theorem. 

\begin{mathpar}
  \inferrule* [lab=summation] {} {{M_{M},M_{N}} \bc \Box \;|\; x.M_{A} \;|\; M_{M}+M_{N}}
  \and
  \inferrule* [lab=agent] {} {{M_{A}} \bc (\vec{x})M_{P} \;| \; \clift{P_0,\ldots,M_{P},\ldots,P_N}}
  \and \\
  \inferrule* [lab=process] {} {{M_{P}} \bc M_{N} \;| \;P|M_{P} }
\end{mathpar} 

\begin{mathpar}
  \inferrule* [lab=sychronization] {} {M_{N} \bc \Box \;|\; x?M_{F} \;|\; x!M_{C}}
  \and
  \inferrule* [lab=abstraction] {} {{M_{F}} \bc (x)M_{P} }
  \and
  \inferrule* [lab=concretion] {} {{M_{C}} \bc \langle M_{P} \rangle }
  \and \\
  \inferrule* [lab=process] {} {{M_{P}} \bc M_{N} \;| \;P|M_{P} }
\end{mathpar}

\begin{definition}[contextual application] Given a context $M$, and
  process $P$, we define the \emph{contextual application}, $M[P] :=
  M\{P/\Box\}$. That is, the contextual application of M to P is the
  substitution of $P$ for $\Box$ in $M$.
\end{definition}

$\meaningof{-} : L \to \mathcal{P}(\pi)$

\begin{mathpar}
  \inferrule* [lab=collection] {} {\meaningof{true} = \pi, \and \meaningof{~E} = \pi \setminus \meaningof{E}, \and \meaningof{E_{1} \& E_{2}} = \meaningof{E_{1}} \cap \meaningof{E_{2}}}
\end{mathpar}

\begin{mathpar}
  \inferrule* [lab=structure] {} {\meaningof{0} = \{ P \in \pi | P \equiv 0 \}, \and \\ \meaningof{E_1 | E_2} = \{ P \in \pi | P \equiv P_{1} | P_{2}, P_{1} \in \meaningof{E_{1}}, P_{2} \in \meaningof{E_2}\} }
\end{mathpar}

\begin{mathpar}
 \inferrule* [lab=behavior] {} {\meaningof{\langle a?b \rangle E} = \{ P \in \pi | P \equiv Q | u?(y)P', \\ \and \\\\ \and \\ \;\;\; u \in \meaningof{a}, \forall z.P'\{z/y\} \in \meaningof{E\{z/b\}}\}, \and \\ \meaningof{a!E} = \{ P \in \pi | P \equiv Q | x!\langle P' \rangle, x \in \meaningof{a} P' \in \meaningof{E}\} }
\end{mathpar}

\begin{mathpar}
 \inferrule* [lab=nominal] {} {\meaningof{\quotep{E}} = \{ \quotep{P} \in \quotep{\pi} | P \in \meaningof{E} \}, \and \meaningof{\quotep{P}} = \{ \quotep{Q} \in \quotep{\pi} | P \equiv Q \} \and \\ \meaningof{@\quotep{E}} = \{ P \in \pi | P \equiv @x, x \in \meaningof{E} \}}
\end{mathpar}

\begin{eqnarray*}
  \\
  \meaningof{-} : TS \to ST
\end{eqnarray*}

\begin{eqnarray*}
  \\
  L : TS \to ST
\end{eqnarray*}

\begin{eqnarray*}
  \\
  P \models E \iff P \in \meaningof{E}
\end{eqnarray*}

\begin{eqnarray*}
  P \approx_{L} Q \iff \forall E \in L. P \models E \iff Q \models E
\end{eqnarray*}

\begin{eqnarray*}
  P \approx_{K} Q
\end{eqnarray*}

\begin{eqnarray*}
  P \approx Q
\end{eqnarray*}

$\approx_{K} = \approx = \approx_{L}$

\subsubsection{Contextual duality}

Note that contexts extend the quotation operation to a family of
operations from processes to names. Given a context, $M$, we can
define a \emph{nominal context}, $\quotep{M}$ by $\quotep{M}[P] :=
\quotep{M[P]}$. To foreshadow what is to come we observe that these
operations enjoy a duality with processes very much like the duality
between vectors and maps from vectors to scalars.

Further, because the calculus is essentially higher-order, we have a
correspondence between contexts and processes. More specifically,
given a name $x$ and a context $M$ we can construct $M^{*}_{x}$ such
that 

\begin{mathpar}
  M^{*}_{x} | \lift{x}{P} \red M[P]
\end{mathpar}

namely,

\begin{mathpar}
  M^{*}_{x} := x?(u).M[\dropn{u}]
\end{mathpar}

The dependence of $M^{*}_{x}$ on a name makes it an abstraction, 

\begin{mathpar}
  M^{*} := (x)x?(u).M[\dropn{u}]
\end{mathpar}

\subsection{Additional notation}

It will sometimes be convenient to denote the process a name
quotes. We already have the notation $x = \quotep{P}$, but it will be
convenient to introduce an alternate notation, $\procn{x}$, when we
want to emphasize the connection to the use of the name. Note that, by
virtue of name equivalence, $\quotep{\procn{x}} \nameeq x$; so, the
notation is consistent with previous definitions.

Further, because names have structure it is possible to effect
substitutions on the basis of that structure. This means we need to
upgrade our notation for substitutions, which we accomplish by
adapting comprehension notation. Thus,

\begin{mathpar}
  P\{ y / x : x \in S \}
\end{mathpar}

is interpreted to mean the process derived from P by replacing (in a
capture-avoiding manner) each occurrence of $x$ in $S$ by $y$. For example,

\begin{mathpar}
  P\{ \quotep{\procn{x}|\procn{x}} / x : x \in \freenames{P} \}
\end{mathpar}

will replace each (occurrence) of a free name $x$ in $P$ by
$\quotep{\procn{x}|\procn{x}}$.

Also, we will avail ourselves of the notation $x^{L}$ and $x^{R}$ to
denote injections of a name into disjoint copies of the name
space. There are numerous ways to accomplish this. One example can be
found in \cite{MeredithR05}. This notation overloads to vectors of
names: $\vec{x}^{\pi} := (x_{i}^{\pi} \; : \; 0 \leq i < |\vec{x}| )$ where $\pi \in \{L,R\}$.

We also use $P^{\Box} := P|\Box$.

In \cite{MeredithR05} an interpretation of the new operator is
given. It turns out that there are several possible interpretations
all enjoying the requisite algebraic properties of the operator (see
\cite{milner91polyadicpi}). We will therefore make liberal use of
$(\nu\; \vec{x})P$.

% subsection the_syntax_and_semantics_of_the_notation_system (end)   

\input{qm2pi.qmops} 

\input{qm2pi.sterngerlach} 

\input{qm2pi.metric} 

% section concurrent_process_calculi (end)

%\input{qm2pi.proofsketch}

% section proof sketch (end)

%\input{qm2pi.slviaknots} 

% section spatial logic via knots (end)

\input{qm2pi.conclusion}

% section conclusion (end)

%\input{qm2pi.dtcodes} 

% section wiring algorithm (end)

\input{qm2pi.ack} 

% section acknowledgments (end)

\newpage


\bibliographystyle{plain}   
\bibliography{../../biblios/main.bib}

\input{qm2pi.rhodetails}

\end{document}

 

% section wiring algorithm (end)

\documentclass[12pt]{llncs}
%\documentclass{jktr}

\usepackage[pdftex]{hyperref}                   
\usepackage {listings}
\usepackage {mathpartir}
\usepackage{bcprules}
%\usepackage{listings}
                       
\usepackage{graphicx} 
%\usepackage[margins=2.5cm,nohead,nofoot]{geometry}
%\usepackage{geometry}
\usepackage{amsfonts}
\usepackage{amstext}
\usepackage{latexsym}
\usepackage{amssymb}
\usepackage{color}


%\include{myPreamble}
\include{qm2pi.local} 

%\ifpdf
%\usepackage[pdftex]{graphicx}
%\else
%\usepackage{graphicx}
%\fi

 % \ifpdf
%  \usepackage{pdfsync}
%  \if


%\title{Brief Article}
%\author{David F. Snyder}
%\author{L.G. Meredith}

%\address{Dept. of Math., Texas State University--San Marcos, San Marcos, TX 78666}
       
\pagestyle{empty}


\begin{document}

\lstset{language=[Objective]Caml,frame=shadowbox}

\input{qm2pi.front}

% section front matter (end)

\input{qm2pi.intro} 
 
% section introduction (end)

% \input{qm2pi.knotations} 

% section notation (end)

\input{qm2pi.process.calculi} 

% section concurrent_process_calculi_and_spatial_logics_ (end)
    
%\input{qm2pi.knots2pi} 

%\input{qm2pi.trefoil} 

%\input{qm2pi.mainthm} 

% subsection basic_interpretation (end)

%\input{qm2pi.rho.presentation} 
\subsection{The syntax and semantics of the notation system}\label{sub:the_syntax_and_semantics_of_the_notation_system} % (fold)

We now summarize a technical presentation of the calculus that
embodies our theory of dynamics. The typical presentation of such a
calculus follows the style of giving generators and relations on
them. The grammar, below, describing term constructors, freely
generates the set of processes, $\Proc$. This set is then quotiented
by a relation known as structural congruence and it is over this set
that the notion of dynamics is expressed. This presentation is
essentially that of \cite{MeredithR05} with the addition of
polyadicity and summation. For readability we have relegated some of
the technical subtleties to an appendix.

\subsubsection{Process grammar}\label{subsub:process_grammar}

\begin{mathpar}
  \inferrule* [lab=synchronization] {} {{M} \bc \pzero \;|\; x?F \;|\; x!C }
  \and
  \inferrule* [lab=abstraction] {} {{F} \bc (x)P}
  \and
  \inferrule* [lab=concretion] {} {{C} \bc \langle Q \rangle}
  \and
  \inferrule* [lab=process] {} {{P,Q} \bc M \;| \;P|Q \;|\; @{x}}
  \and
  \inferrule* [lab=name] {} {{x} \bc \quotep{P}}
\end{mathpar} 

Note that $\vec{x}$ (resp. $\vec{P}$) denotes a vector of names
(resp. processes) of length $|\vec{x}|$ (resp. $|\vec{P}|$). We adopt
the following useful abbreviations.

\begin{mathpar}
   x?(\vec{y}).P := x.(\vec{y})P \and  x\clift{\vec{P}} := x.\clift{\vec{P}}
   \and x!(y) := \lift{x}{\dropn{y}}
   \and \Pi_{i=0}^{n-1}P_i := P_0 | \ldots | P_{n-1}
\end{mathpar}

\subsubsection{Structural congruence}

\paragraph{Free and bound names and alpha-equivalence.} At the
core of structural equivalence is alpha-equivalence which identifies
process that are the same up to a change of variable. Formally, we
recognize the distinction between free and bound names. The free names
of a process, $\freenames{P}$, may be calculated recursively as
follows:

\begin{mathpar}
\freenames{\pzero} := \emptyset
  \and \\
  \freenames{x?(y).P} := \{ x \} \cup (\freenames{P} \setminus \{ y \})
  \and 
  \freenames{x!\langle P \rangle} := \{ x \} \cup \{ P \} 
  \and \\
  \freenames{P|Q} := \freenames{P} \cup \freenames{Q}
  \and \\
  \freenames{@{x}} := \{ x \}
\end{mathpar}

$\pi$
$\quotep{\pi}$

$\freenames{-} : \pi \to \mathcal{P}(\quotep{\pi})$

\begin{eqnarray*}
  \freenames{\pzero} & := & \emptyset \\
  \freenames{x?(y).P} & := & \{ x \} \cup (\freenames{P} \setminus \{ y \}) \\
  \freenames{x!\langle P \rangle} & := & \{ x \} \cup \{ P \} \\
  \freenames{P|Q} & := & \freenames{P} \cup \freenames{Q} \\
  \freenames{\dropn{x}} & := & \{ x \}
\end{eqnarray*}

The bound names of a process, $\boundnames{P}$, are those names occurring in $P$
that are not free. For example, in $x?(y).0$, the name $x$ is free, while $y$ is bound.

\begin{mathpar}
  \inferrule* [lab=monoidal-laws] {} { P|Q \equiv Q|P \and P|0 \equiv P \and P|(Q|R) \equiv (P|Q)|R }
\end{mathpar}

\begin{mathpar}
  \inferrule* [lab=alpha-equivalence] {} { (x)P \equiv (y)P\{y/x\} \and y \not\in \freenames{P} }
\end{mathpar}

\begin{definition}
Then two processes, $P,Q$, are alpha-equivalent if $P = Q\{\vec{y}/\vec{x}\}$ for
some $\vec{x} \in \boundnames{Q},\vec{y} \in \boundnames{P}$, where $Q\{\vec{y}/\vec{x}\}$
denotes the capture-avoiding substitution of $\vec{y}$ for $\vec{x}$ in $Q$.
\end{definition}

\begin{definition}
  The {\em structural congruence} \cite{SangiorgiWalker} , $\equiv$,
  between processes is the least congruence containing
  alpha-equivalence, satisfying the abelian monoid laws
  (associativity, commutativity and $\pzero$ as identity) for parallel
  composition $|$ and for summation $+$.
\end{definition}

\subsection{Name equivalence}

We take name equivalence, written $\nameeq$, to be the smallest
equivalence relation generated by the following rules.

\begin{mathpar}
\inferrule*[lab=Quote-drop]
{ }
{ \quotep{@{x}} \nameeq x }

\inferrule*[lab=Struct-equiv]
{ P \scong Q }
{ \quotep{P} \nameeq \quotep{Q} }
\end{mathpar}

The astute reader will have noticed that the mutual recursion of names
and processes imposes a mutual recursion on alpha-equivalence and
structural equivalence via name-equivalence. Fortunately, all of this
works out pleasantly and we may calculate in the natural way, free of
concern. The reader interested in the details is referred to the
appendix \ref{appendix:rho_details}.

\subsection{Substitution}

We use $\Proc$ for the set of processes, $\QProc$ for the set of
names, and $\id{\{}\vec{y} / \vec{x} \id{\}}$ to denote partial maps,
$s : \QProc \rightarrow \QProc$. A map, $s$ lifts, uniquely, to a map
on process terms, $\widehat{s} : \Proc \rightarrow \Proc$ by the
following equations.

\begin{mathpar}
  (0) \psubstp{Q}{P} := 0 \\
  (R \juxtap S) \psubstp{Q}{P}
  :=    
  (R)\psubstp{Q}{P} \juxtap (S) \psubstp{Q}{P} \\
  (x?(y).R) \psubstp{Q}{P}    
  :=    
  (x)\substp{Q}{P} (z)\concat( (R \psubstn{z}{y}) \psubstp{Q}{P} ) \\
  (\lift{x}{R}) \psubstp{Q}{P}  
  :=
  \lift{(x)\substp{Q}{P}}{ R \psubstp{Q}{P} } \\
%   (\dropn{x})  \psubstp{Q}{P}       
%   := 
%   \left\{ 
%     \begin{array}{ccc} 
%       \dropn{\quotep{Q}} & & x \nameeq \quotep{P} \\
%       \dropn{x} & & otherwise \\
%     \end{array}
%   \right. 
  (\dropn{x})  \psubstp{Q}{P}       
  := 
  \left\{ 
    \begin{array}{ccc} 
      Q & & x \nameeq \quotep{P} \\
      \dropn{x} & & otherwise \\
    \end{array}
  \right.
\end{mathpar}
 

where

\begin{eqnarray}
  (x)\id{\{} \lpquote Q \rpquote / \lpquote P \rpquote \id{\}}            = 
  \left\{ 
    \begin{array}{ccc}
      \lpquote Q \rpquote & & x \nameeq \lpquote P \rpquote \\
      x & & otherwise \\
    \end{array}
  \right. \nonumber
\end{eqnarray}

and $z$ is chosen distinct from $\quotep{P}$, $\quotep{Q}$, the free
names in $Q$, and all the names in $R$. Our $\alpha$-equivalence will
be built in the standard way from this substitution.

\begin{remark}\label{rem:no_self_referential_names}
  One consequence of these definitions is that $\forall P. \quotep{P}
  \not\in \freenames{P}$.
\end{remark}

\subsection{ Dynamic quote: an example }

Anticipating something of what's to come, consider applying the
substitution, $\widehat{\id{\{}u / z \id{\}}}$, to the following pair
of processes, $\lift{w}{y!(z)}$ and $w[ \lpquote y!(z) \rpquote ]$.

\begin{eqnarray}
	\lift{w}{y!(z)}\widehat{\id{\{}u / z \id{\}}}
		& = &
		\lift{w}{y!(u)} \nonumber\\
	w[ \lpquote y!(z) \rpquote ] \widehat{ \id{\{}u / z \id{\}} }
		& = &
		w[ \lpquote y!(z) \rpquote ] \nonumber
\end{eqnarray}

Because the body of the process between quotes is impervious to
substitution, we get radically different answers. In fact, by
examining the first process in an input context,
e.g. $x?(z).\lift{w}{y!(z)}$, we see that the process under the lift
operator may be shaped by prefixed inputs binding a name inside it. In
this sense, the lift operator will be seen as a way to dynamically
construct processes before reifying them as names.

Finally equipped with these standard features we can present the
dynamics of the calculus.

\subsubsection{Operational semantics} 

Finally, we introduce the computational dynamics. What marks these
algebras as distinct from other more traditionally studied algebraic
structures, e.g. vector spaces or polynomial rings, is the manner in
which dynamics is captured. In traditional structures, dynamics is typically
expressed through morphisms between such structures, as in linear maps
between vector spaces or morphisms between rings. In algebras
associated with the semantics of computation, the dynamics is
expressed as part of the algebraic structure itself, through a
reduction reduction relation typically denoted by $\red$. Below, we
give a recursive presentation of this relation for the calculus used
in the encoding.

$\red \subseteq \pi \times \pi$
$\red : \pi \to \mathcal{P}(\pi)$

\begin{mathpar}
  \inferrule* [lab=Comm] { \textsf{match}( x_{src}, x_{trgt} ) } { x_{trgt}?(y)P \; | \; x_{src}!\langle {Q} \rangle \red P\{\quotep{Q}/y}\} }
  \and \\
  \inferrule* [lab=Par] {{P} \red {P}'} {{{P} | {Q}} \red {{P}' | {Q}}}
  \and
  \inferrule* [lab=Equiv]{{{P} \scong {P}'} \andalso {{P}' \red {Q}'} \andalso {{Q}' \scong {Q}}}{{P} \red {Q}}
\end{mathpar}

\begin{eqnarray*}
  match_{\equiv} (\quotep{P},\quotep{Q}) & := & P \equiv Q \\
  match_{\dagger}(\quotep{P},\quotep{Q}) & := & \forall R. P|Q \red^{*} R => R \red^{*} 0 \\
  match_{K}(\quotep{P},\quotep{Q}) & := & K \mbox{ for some context } K
\end{eqnarray*}

$u?(x)P | u!\langle Q \rangle \red P\{\quotep{Q}/x\}$

%We write $\wred$ for $\red^*$, and $P\red$ if $\exists Q $ such that $ P \red Q$.
We write $P\red$ if $\exists Q $ such that $ P \red Q$ and $P\not\red$, otherwise.

\section{Replication}

As mentioned before, it is known that replication (and hence
recursion) can be implemented in a higher-order process algebra
\cite{SangiorgiWalker}. As our first example of calculation with the
machinery thus far presented we give the construction explicitly in
the {\rhoc}.

\begin{eqnarray}
	D_{x} & := & \prefix{x}{y}{(\binpar{\outputp{x}{y}}{@{y}})} \nonumber\\
	\bangp_{x}{P} & := & \binpar{{x}!\langle{\binpar{D_{x}}{P}}\rangle}{D_{x}} \nonumber
\end{eqnarray}

\begin{eqnarray}
	\bangp_{x}{P} & & \nonumber\\
	=
	& {x}!\langle{(\prefix{x}{y}{(\outputp{x}{y} | @{y})) | P}}\rangle 
	      | \prefix{x}{y}{(\outputp{x}{y} | @{y})} & \nonumber\\
	\red
	& (\outputp{x}{y} | @{y})\substn{\quotep{(\prefix{x}{y}{(@{y} | \outputp{x}{y})) | P}}}{y} & \nonumber\\
	=
	& \outputp{x}{\quotep{(\prefix{x}{y}{(\outputp{x}{y} | @{y})) | P}}}
	  | {(\prefix{x}{y}{(\outputp{x}{y} | @{y})) | P}} & \nonumber\\
	\red
	& \ldots & \nonumber\\
	\red^*
	& P | P | \ldots & \nonumber
\end{eqnarray}

Of course, this encoding, as an implementation, runs away, unfolding
$\bangp{P}$ eagerly. A lazier and more implementable replication
operator, restricted to input-guarded processes, may be obtained as follows.

\begin{eqnarray}
\bangp{\prefix{u}{v}{P}} 
	:= 
	\binpar{\lift{x}{\prefix{u}{v}{(\binpar{D(x)}{P})}}}{D(x)} \nonumber
\end{eqnarray}

\begin{remark}
  Note that the lazier definition still does not deal with summation
  or mixed summation (i.e. sums over input and output). The reader is
  invited to construct definitions of replication that deal with these
  features. 

  Further, the definitions are parameterized in a name, $x$. Can you,
  gentle reader, make a definition that eliminates this parameter and
  guarantees no accidental interaction between the replication
  machinery and the process being replicated -- i.e. no accidental
  sharing of names used by the process to get its work done and the
  name(s) used by the replication to effect copying. This latter
  revision of the definition of replication is crucial to obtaining
  the expected identity $!!P \sim !P$.
\end{remark}

\begin{remark}\label{rem:paradoxical_combinator}
  The reader familiar with the lambda calculus will have noticed the
  similarity between $D$ and the paradoxical combinator.

  [Ed. note: the existence of this seems to suggest we have to be more
  restrictive on the set of processes and names we admit if we are to
  support no-cloning.]
\end{remark}

\subsubsection{Bisimulation}

The computational dynamics gives rise to another kind of equivalence,
the equivalence of computational behavior. As previously mentioned
this is typically captured \emph{via} some form of bisimulation.

% The notion we use in this paper is weak barbed bisimulation
% \cite{milner91polyadicpi}.

The notion we use in this paper is derived from weak barbed
bisimulation \cite{milner91polyadicpi}. 

\begin{definition}
An \emph{observation relation}, $\downarrow_{\mathcal N}$, over a set
of names, $\mathcal N$, is the smallest relation satisfying the rules
below.

\infrule[Out-barb]{y \in {\mathcal N}, \; x \nameeq y}
		  {\outputp{x}{v} \downarrow_{\mathcal N} x}
\infrule[Par-barb]{\mbox{$P\downarrow_{\mathcal N} x$ or $Q\downarrow_{\mathcal N} x$}}
		  {\binpar{P}{Q} \downarrow_{\mathcal N} x}

We write $P \Downarrow_{\mathcal N} x$ if there is $Q$ such that 
$P \wred Q$ and $Q \downarrow_{\mathcal N} x$.
\end{definition}

\begin{definition}
%\label{def.bbisim}
An  ${\mathcal N}$-\emph{barbed bisimulation} over a set of names, ${\mathcal N}$, is a symmetric binary relation 
${\mathcal S}_{\mathcal N}$ between agents such that $P\rel{S}_{\mathcal N}Q$ implies:
\begin{enumerate}
\item If $P \red P'$ then $Q \wred Q'$ and $P'\rel{S}_{\mathcal N} Q'$.
\item If $P\downarrow_{\mathcal N} x$, then $Q\Downarrow_{\mathcal N} x$.
\end{enumerate}
$P$ is ${\mathcal N}$-barbed bisimilar to $Q$, written
$P \wbbisim_{\mathcal N} Q$, if $P \rel{S}_{\mathcal N} Q$ for some ${\mathcal N}$-barbed bisimulation ${\mathcal S}_{\mathcal N}$.
\end{definition}

$\mathcal{R} \subseteq \pi \times \pi$

$P \mathcal{R} Q => \forall P'. P \red P' \Rightarrow \exists Q'. Q \red Q', P' \mathcal{R} Q'$

$P \vdash x \Rightarrow Q \vdash x$

\begin{mathpar}
  \inferrule*[lab=Out-barb]{x \nameeq y}{{y}!\langle{Q}\rangle \vdash x}
  \and
  \inferrule*[lab=Par-barb]{\mbox{$P\vdash x$ or $Q\vdash x$}}{\binpar{P}{Q} \vdash x}
\end{mathpar}

\subsubsection{Contexts}

One of the principle advantages of computational calculi like the
$\pi$-calculus is a well-defined notion of context,
contextual-equivalence and a correlation between
contextual-equivalence and notions of bisimulation. The notion of
context allows the decomposition of a process into (sub-)process and
its syntactic environment, its context. Thus, a context may be
thought of as a process with a ``hole'' (written $\Box$) in it. The
application of a context $M$ to a process $P$, written $M[P]$, is
tantamount to filling the hole in $M$ with $P$. In this paper we do
not need the full weight of this theory, but do make use of the notion
of context in the proof the main theorem. 

\begin{mathpar}
  \inferrule* [lab=summation] {} {{M_{M},M_{N}} \bc \Box \;|\; x.M_{A} \;|\; M_{M}+M_{N}}
  \and
  \inferrule* [lab=agent] {} {{M_{A}} \bc (\vec{x})M_{P} \;| \; \clift{P_0,\ldots,M_{P},\ldots,P_N}}
  \and \\
  \inferrule* [lab=process] {} {{M_{P}} \bc M_{N} \;| \;P|M_{P} }
\end{mathpar} 

\begin{mathpar}
  \inferrule* [lab=sychronization] {} {M_{N} \bc \Box \;|\; x?M_{F} \;|\; x!M_{C}}
  \and
  \inferrule* [lab=abstraction] {} {{M_{F}} \bc (x)M_{P} }
  \and
  \inferrule* [lab=concretion] {} {{M_{C}} \bc \langle M_{P} \rangle }
  \and \\
  \inferrule* [lab=process] {} {{M_{P}} \bc M_{N} \;| \;P|M_{P} }
\end{mathpar}

\begin{definition}[contextual application] Given a context $M$, and
  process $P$, we define the \emph{contextual application}, $M[P] :=
  M\{P/\Box\}$. That is, the contextual application of M to P is the
  substitution of $P$ for $\Box$ in $M$.
\end{definition}

$\meaningof{-} : L \to \mathcal{P}(\pi)$

\begin{mathpar}
  \inferrule* [lab=collection] {} {\meaningof{true} = \pi, \and \meaningof{~E} = \pi \setminus \meaningof{E}, \and \meaningof{E_{1} \& E_{2}} = \meaningof{E_{1}} \cap \meaningof{E_{2}}}
\end{mathpar}

\begin{mathpar}
  \inferrule* [lab=structure] {} {\meaningof{0} = \{ P \in \pi | P \equiv 0 \}, \and \\ \meaningof{E_1 | E_2} = \{ P \in \pi | P \equiv P_{1} | P_{2}, P_{1} \in \meaningof{E_{1}}, P_{2} \in \meaningof{E_2}\} }
\end{mathpar}

\begin{mathpar}
 \inferrule* [lab=behavior] {} {\meaningof{\langle a?b \rangle E} = \{ P \in \pi | P \equiv Q | u?(y)P', \\ \and \\\\ \and \\ \;\;\; u \in \meaningof{a}, \forall z.P'\{z/y\} \in \meaningof{E\{z/b\}}\}, \and \\ \meaningof{a!E} = \{ P \in \pi | P \equiv Q | x!\langle P' \rangle, x \in \meaningof{a} P' \in \meaningof{E}\} }
\end{mathpar}

\begin{mathpar}
 \inferrule* [lab=nominal] {} {\meaningof{\quotep{E}} = \{ \quotep{P} \in \quotep{\pi} | P \in \meaningof{E} \}, \and \meaningof{\quotep{P}} = \{ \quotep{Q} \in \quotep{\pi} | P \equiv Q \} \and \\ \meaningof{@\quotep{E}} = \{ P \in \pi | P \equiv @x, x \in \meaningof{E} \}}
\end{mathpar}

\begin{eqnarray*}
  \\
  \meaningof{-} : TS \to ST
\end{eqnarray*}

\begin{eqnarray*}
  \\
  L : TS \to ST
\end{eqnarray*}

\begin{eqnarray*}
  \\
  P \models E \iff P \in \meaningof{E}
\end{eqnarray*}

\begin{eqnarray*}
  P \approx_{L} Q \iff \forall E \in L. P \models E \iff Q \models E
\end{eqnarray*}

\begin{eqnarray*}
  P \approx_{K} Q
\end{eqnarray*}

\begin{eqnarray*}
  P \approx Q
\end{eqnarray*}

$\approx_{K} = \approx = \approx_{L}$

\subsubsection{Contextual duality}

Note that contexts extend the quotation operation to a family of
operations from processes to names. Given a context, $M$, we can
define a \emph{nominal context}, $\quotep{M}$ by $\quotep{M}[P] :=
\quotep{M[P]}$. To foreshadow what is to come we observe that these
operations enjoy a duality with processes very much like the duality
between vectors and maps from vectors to scalars.

Further, because the calculus is essentially higher-order, we have a
correspondence between contexts and processes. More specifically,
given a name $x$ and a context $M$ we can construct $M^{*}_{x}$ such
that 

\begin{mathpar}
  M^{*}_{x} | \lift{x}{P} \red M[P]
\end{mathpar}

namely,

\begin{mathpar}
  M^{*}_{x} := x?(u).M[\dropn{u}]
\end{mathpar}

The dependence of $M^{*}_{x}$ on a name makes it an abstraction, 

\begin{mathpar}
  M^{*} := (x)x?(u).M[\dropn{u}]
\end{mathpar}

\subsection{Additional notation}

It will sometimes be convenient to denote the process a name
quotes. We already have the notation $x = \quotep{P}$, but it will be
convenient to introduce an alternate notation, $\procn{x}$, when we
want to emphasize the connection to the use of the name. Note that, by
virtue of name equivalence, $\quotep{\procn{x}} \nameeq x$; so, the
notation is consistent with previous definitions.

Further, because names have structure it is possible to effect
substitutions on the basis of that structure. This means we need to
upgrade our notation for substitutions, which we accomplish by
adapting comprehension notation. Thus,

\begin{mathpar}
  P\{ y / x : x \in S \}
\end{mathpar}

is interpreted to mean the process derived from P by replacing (in a
capture-avoiding manner) each occurrence of $x$ in $S$ by $y$. For example,

\begin{mathpar}
  P\{ \quotep{\procn{x}|\procn{x}} / x : x \in \freenames{P} \}
\end{mathpar}

will replace each (occurrence) of a free name $x$ in $P$ by
$\quotep{\procn{x}|\procn{x}}$.

Also, we will avail ourselves of the notation $x^{L}$ and $x^{R}$ to
denote injections of a name into disjoint copies of the name
space. There are numerous ways to accomplish this. One example can be
found in \cite{MeredithR05}. This notation overloads to vectors of
names: $\vec{x}^{\pi} := (x_{i}^{\pi} \; : \; 0 \leq i < |\vec{x}| )$ where $\pi \in \{L,R\}$.

We also use $P^{\Box} := P|\Box$.

In \cite{MeredithR05} an interpretation of the new operator is
given. It turns out that there are several possible interpretations
all enjoying the requisite algebraic properties of the operator (see
\cite{milner91polyadicpi}). We will therefore make liberal use of
$(\nu\; \vec{x})P$.

% subsection the_syntax_and_semantics_of_the_notation_system (end)   

\input{qm2pi.qmops} 

\input{qm2pi.sterngerlach} 

\input{qm2pi.metric} 

% section concurrent_process_calculi (end)

%\input{qm2pi.proofsketch}

% section proof sketch (end)

%\input{qm2pi.slviaknots} 

% section spatial logic via knots (end)

\input{qm2pi.conclusion}

% section conclusion (end)

%\input{qm2pi.dtcodes} 

% section wiring algorithm (end)

\input{qm2pi.ack} 

% section acknowledgments (end)

\newpage


\bibliographystyle{plain}   
\bibliography{../../biblios/main.bib}

\input{qm2pi.rhodetails}

\end{document}

 

% section acknowledgments (end)

\newpage


\bibliographystyle{plain}   
\bibliography{../../biblios/main.bib}

\documentclass[12pt]{llncs}
%\documentclass{jktr}

\usepackage[pdftex]{hyperref}                   
\usepackage {listings}
\usepackage {mathpartir}
\usepackage{bcprules}
%\usepackage{listings}
                       
\usepackage{graphicx} 
%\usepackage[margins=2.5cm,nohead,nofoot]{geometry}
%\usepackage{geometry}
\usepackage{amsfonts}
\usepackage{amstext}
\usepackage{latexsym}
\usepackage{amssymb}
\usepackage{color}


%\include{myPreamble}
\include{qm2pi.local} 

%\ifpdf
%\usepackage[pdftex]{graphicx}
%\else
%\usepackage{graphicx}
%\fi

 % \ifpdf
%  \usepackage{pdfsync}
%  \if


%\title{Brief Article}
%\author{David F. Snyder}
%\author{L.G. Meredith}

%\address{Dept. of Math., Texas State University--San Marcos, San Marcos, TX 78666}
       
\pagestyle{empty}


\begin{document}

\lstset{language=[Objective]Caml,frame=shadowbox}

\input{qm2pi.front}

% section front matter (end)

\input{qm2pi.intro} 
 
% section introduction (end)

% \input{qm2pi.knotations} 

% section notation (end)

\input{qm2pi.process.calculi} 

% section concurrent_process_calculi_and_spatial_logics_ (end)
    
%\input{qm2pi.knots2pi} 

%\input{qm2pi.trefoil} 

%\input{qm2pi.mainthm} 

% subsection basic_interpretation (end)

%\input{qm2pi.rho.presentation} 
\subsection{The syntax and semantics of the notation system}\label{sub:the_syntax_and_semantics_of_the_notation_system} % (fold)

We now summarize a technical presentation of the calculus that
embodies our theory of dynamics. The typical presentation of such a
calculus follows the style of giving generators and relations on
them. The grammar, below, describing term constructors, freely
generates the set of processes, $\Proc$. This set is then quotiented
by a relation known as structural congruence and it is over this set
that the notion of dynamics is expressed. This presentation is
essentially that of \cite{MeredithR05} with the addition of
polyadicity and summation. For readability we have relegated some of
the technical subtleties to an appendix.

\subsubsection{Process grammar}\label{subsub:process_grammar}

\begin{mathpar}
  \inferrule* [lab=synchronization] {} {{M} \bc \pzero \;|\; x?F \;|\; x!C }
  \and
  \inferrule* [lab=abstraction] {} {{F} \bc (x)P}
  \and
  \inferrule* [lab=concretion] {} {{C} \bc \langle Q \rangle}
  \and
  \inferrule* [lab=process] {} {{P,Q} \bc M \;| \;P|Q \;|\; @{x}}
  \and
  \inferrule* [lab=name] {} {{x} \bc \quotep{P}}
\end{mathpar} 

Note that $\vec{x}$ (resp. $\vec{P}$) denotes a vector of names
(resp. processes) of length $|\vec{x}|$ (resp. $|\vec{P}|$). We adopt
the following useful abbreviations.

\begin{mathpar}
   x?(\vec{y}).P := x.(\vec{y})P \and  x\clift{\vec{P}} := x.\clift{\vec{P}}
   \and x!(y) := \lift{x}{\dropn{y}}
   \and \Pi_{i=0}^{n-1}P_i := P_0 | \ldots | P_{n-1}
\end{mathpar}

\subsubsection{Structural congruence}

\paragraph{Free and bound names and alpha-equivalence.} At the
core of structural equivalence is alpha-equivalence which identifies
process that are the same up to a change of variable. Formally, we
recognize the distinction between free and bound names. The free names
of a process, $\freenames{P}$, may be calculated recursively as
follows:

\begin{mathpar}
\freenames{\pzero} := \emptyset
  \and \\
  \freenames{x?(y).P} := \{ x \} \cup (\freenames{P} \setminus \{ y \})
  \and 
  \freenames{x!\langle P \rangle} := \{ x \} \cup \{ P \} 
  \and \\
  \freenames{P|Q} := \freenames{P} \cup \freenames{Q}
  \and \\
  \freenames{@{x}} := \{ x \}
\end{mathpar}

$\pi$
$\quotep{\pi}$

$\freenames{-} : \pi \to \mathcal{P}(\quotep{\pi})$

\begin{eqnarray*}
  \freenames{\pzero} & := & \emptyset \\
  \freenames{x?(y).P} & := & \{ x \} \cup (\freenames{P} \setminus \{ y \}) \\
  \freenames{x!\langle P \rangle} & := & \{ x \} \cup \{ P \} \\
  \freenames{P|Q} & := & \freenames{P} \cup \freenames{Q} \\
  \freenames{\dropn{x}} & := & \{ x \}
\end{eqnarray*}

The bound names of a process, $\boundnames{P}$, are those names occurring in $P$
that are not free. For example, in $x?(y).0$, the name $x$ is free, while $y$ is bound.

\begin{mathpar}
  \inferrule* [lab=monoidal-laws] {} { P|Q \equiv Q|P \and P|0 \equiv P \and P|(Q|R) \equiv (P|Q)|R }
\end{mathpar}

\begin{mathpar}
  \inferrule* [lab=alpha-equivalence] {} { (x)P \equiv (y)P\{y/x\} \and y \not\in \freenames{P} }
\end{mathpar}

\begin{definition}
Then two processes, $P,Q$, are alpha-equivalent if $P = Q\{\vec{y}/\vec{x}\}$ for
some $\vec{x} \in \boundnames{Q},\vec{y} \in \boundnames{P}$, where $Q\{\vec{y}/\vec{x}\}$
denotes the capture-avoiding substitution of $\vec{y}$ for $\vec{x}$ in $Q$.
\end{definition}

\begin{definition}
  The {\em structural congruence} \cite{SangiorgiWalker} , $\equiv$,
  between processes is the least congruence containing
  alpha-equivalence, satisfying the abelian monoid laws
  (associativity, commutativity and $\pzero$ as identity) for parallel
  composition $|$ and for summation $+$.
\end{definition}

\subsection{Name equivalence}

We take name equivalence, written $\nameeq$, to be the smallest
equivalence relation generated by the following rules.

\begin{mathpar}
\inferrule*[lab=Quote-drop]
{ }
{ \quotep{@{x}} \nameeq x }

\inferrule*[lab=Struct-equiv]
{ P \scong Q }
{ \quotep{P} \nameeq \quotep{Q} }
\end{mathpar}

The astute reader will have noticed that the mutual recursion of names
and processes imposes a mutual recursion on alpha-equivalence and
structural equivalence via name-equivalence. Fortunately, all of this
works out pleasantly and we may calculate in the natural way, free of
concern. The reader interested in the details is referred to the
appendix \ref{appendix:rho_details}.

\subsection{Substitution}

We use $\Proc$ for the set of processes, $\QProc$ for the set of
names, and $\id{\{}\vec{y} / \vec{x} \id{\}}$ to denote partial maps,
$s : \QProc \rightarrow \QProc$. A map, $s$ lifts, uniquely, to a map
on process terms, $\widehat{s} : \Proc \rightarrow \Proc$ by the
following equations.

\begin{mathpar}
  (0) \psubstp{Q}{P} := 0 \\
  (R \juxtap S) \psubstp{Q}{P}
  :=    
  (R)\psubstp{Q}{P} \juxtap (S) \psubstp{Q}{P} \\
  (x?(y).R) \psubstp{Q}{P}    
  :=    
  (x)\substp{Q}{P} (z)\concat( (R \psubstn{z}{y}) \psubstp{Q}{P} ) \\
  (\lift{x}{R}) \psubstp{Q}{P}  
  :=
  \lift{(x)\substp{Q}{P}}{ R \psubstp{Q}{P} } \\
%   (\dropn{x})  \psubstp{Q}{P}       
%   := 
%   \left\{ 
%     \begin{array}{ccc} 
%       \dropn{\quotep{Q}} & & x \nameeq \quotep{P} \\
%       \dropn{x} & & otherwise \\
%     \end{array}
%   \right. 
  (\dropn{x})  \psubstp{Q}{P}       
  := 
  \left\{ 
    \begin{array}{ccc} 
      Q & & x \nameeq \quotep{P} \\
      \dropn{x} & & otherwise \\
    \end{array}
  \right.
\end{mathpar}
 

where

\begin{eqnarray}
  (x)\id{\{} \lpquote Q \rpquote / \lpquote P \rpquote \id{\}}            = 
  \left\{ 
    \begin{array}{ccc}
      \lpquote Q \rpquote & & x \nameeq \lpquote P \rpquote \\
      x & & otherwise \\
    \end{array}
  \right. \nonumber
\end{eqnarray}

and $z$ is chosen distinct from $\quotep{P}$, $\quotep{Q}$, the free
names in $Q$, and all the names in $R$. Our $\alpha$-equivalence will
be built in the standard way from this substitution.

\begin{remark}\label{rem:no_self_referential_names}
  One consequence of these definitions is that $\forall P. \quotep{P}
  \not\in \freenames{P}$.
\end{remark}

\subsection{ Dynamic quote: an example }

Anticipating something of what's to come, consider applying the
substitution, $\widehat{\id{\{}u / z \id{\}}}$, to the following pair
of processes, $\lift{w}{y!(z)}$ and $w[ \lpquote y!(z) \rpquote ]$.

\begin{eqnarray}
	\lift{w}{y!(z)}\widehat{\id{\{}u / z \id{\}}}
		& = &
		\lift{w}{y!(u)} \nonumber\\
	w[ \lpquote y!(z) \rpquote ] \widehat{ \id{\{}u / z \id{\}} }
		& = &
		w[ \lpquote y!(z) \rpquote ] \nonumber
\end{eqnarray}

Because the body of the process between quotes is impervious to
substitution, we get radically different answers. In fact, by
examining the first process in an input context,
e.g. $x?(z).\lift{w}{y!(z)}$, we see that the process under the lift
operator may be shaped by prefixed inputs binding a name inside it. In
this sense, the lift operator will be seen as a way to dynamically
construct processes before reifying them as names.

Finally equipped with these standard features we can present the
dynamics of the calculus.

\subsubsection{Operational semantics} 

Finally, we introduce the computational dynamics. What marks these
algebras as distinct from other more traditionally studied algebraic
structures, e.g. vector spaces or polynomial rings, is the manner in
which dynamics is captured. In traditional structures, dynamics is typically
expressed through morphisms between such structures, as in linear maps
between vector spaces or morphisms between rings. In algebras
associated with the semantics of computation, the dynamics is
expressed as part of the algebraic structure itself, through a
reduction reduction relation typically denoted by $\red$. Below, we
give a recursive presentation of this relation for the calculus used
in the encoding.

$\red \subseteq \pi \times \pi$
$\red : \pi \to \mathcal{P}(\pi)$

\begin{mathpar}
  \inferrule* [lab=Comm] { \textsf{match}( x_{src}, x_{trgt} ) } { x_{trgt}?(y)P \; | \; x_{src}!\langle {Q} \rangle \red P\{\quotep{Q}/y}\} }
  \and \\
  \inferrule* [lab=Par] {{P} \red {P}'} {{{P} | {Q}} \red {{P}' | {Q}}}
  \and
  \inferrule* [lab=Equiv]{{{P} \scong {P}'} \andalso {{P}' \red {Q}'} \andalso {{Q}' \scong {Q}}}{{P} \red {Q}}
\end{mathpar}

\begin{eqnarray*}
  match_{\equiv} (\quotep{P},\quotep{Q}) & := & P \equiv Q \\
  match_{\dagger}(\quotep{P},\quotep{Q}) & := & \forall R. P|Q \red^{*} R => R \red^{*} 0 \\
  match_{K}(\quotep{P},\quotep{Q}) & := & K \mbox{ for some context } K
\end{eqnarray*}

$u?(x)P | u!\langle Q \rangle \red P\{\quotep{Q}/x\}$

%We write $\wred$ for $\red^*$, and $P\red$ if $\exists Q $ such that $ P \red Q$.
We write $P\red$ if $\exists Q $ such that $ P \red Q$ and $P\not\red$, otherwise.

\section{Replication}

As mentioned before, it is known that replication (and hence
recursion) can be implemented in a higher-order process algebra
\cite{SangiorgiWalker}. As our first example of calculation with the
machinery thus far presented we give the construction explicitly in
the {\rhoc}.

\begin{eqnarray}
	D_{x} & := & \prefix{x}{y}{(\binpar{\outputp{x}{y}}{@{y}})} \nonumber\\
	\bangp_{x}{P} & := & \binpar{{x}!\langle{\binpar{D_{x}}{P}}\rangle}{D_{x}} \nonumber
\end{eqnarray}

\begin{eqnarray}
	\bangp_{x}{P} & & \nonumber\\
	=
	& {x}!\langle{(\prefix{x}{y}{(\outputp{x}{y} | @{y})) | P}}\rangle 
	      | \prefix{x}{y}{(\outputp{x}{y} | @{y})} & \nonumber\\
	\red
	& (\outputp{x}{y} | @{y})\substn{\quotep{(\prefix{x}{y}{(@{y} | \outputp{x}{y})) | P}}}{y} & \nonumber\\
	=
	& \outputp{x}{\quotep{(\prefix{x}{y}{(\outputp{x}{y} | @{y})) | P}}}
	  | {(\prefix{x}{y}{(\outputp{x}{y} | @{y})) | P}} & \nonumber\\
	\red
	& \ldots & \nonumber\\
	\red^*
	& P | P | \ldots & \nonumber
\end{eqnarray}

Of course, this encoding, as an implementation, runs away, unfolding
$\bangp{P}$ eagerly. A lazier and more implementable replication
operator, restricted to input-guarded processes, may be obtained as follows.

\begin{eqnarray}
\bangp{\prefix{u}{v}{P}} 
	:= 
	\binpar{\lift{x}{\prefix{u}{v}{(\binpar{D(x)}{P})}}}{D(x)} \nonumber
\end{eqnarray}

\begin{remark}
  Note that the lazier definition still does not deal with summation
  or mixed summation (i.e. sums over input and output). The reader is
  invited to construct definitions of replication that deal with these
  features. 

  Further, the definitions are parameterized in a name, $x$. Can you,
  gentle reader, make a definition that eliminates this parameter and
  guarantees no accidental interaction between the replication
  machinery and the process being replicated -- i.e. no accidental
  sharing of names used by the process to get its work done and the
  name(s) used by the replication to effect copying. This latter
  revision of the definition of replication is crucial to obtaining
  the expected identity $!!P \sim !P$.
\end{remark}

\begin{remark}\label{rem:paradoxical_combinator}
  The reader familiar with the lambda calculus will have noticed the
  similarity between $D$ and the paradoxical combinator.

  [Ed. note: the existence of this seems to suggest we have to be more
  restrictive on the set of processes and names we admit if we are to
  support no-cloning.]
\end{remark}

\subsubsection{Bisimulation}

The computational dynamics gives rise to another kind of equivalence,
the equivalence of computational behavior. As previously mentioned
this is typically captured \emph{via} some form of bisimulation.

% The notion we use in this paper is weak barbed bisimulation
% \cite{milner91polyadicpi}.

The notion we use in this paper is derived from weak barbed
bisimulation \cite{milner91polyadicpi}. 

\begin{definition}
An \emph{observation relation}, $\downarrow_{\mathcal N}$, over a set
of names, $\mathcal N$, is the smallest relation satisfying the rules
below.

\infrule[Out-barb]{y \in {\mathcal N}, \; x \nameeq y}
		  {\outputp{x}{v} \downarrow_{\mathcal N} x}
\infrule[Par-barb]{\mbox{$P\downarrow_{\mathcal N} x$ or $Q\downarrow_{\mathcal N} x$}}
		  {\binpar{P}{Q} \downarrow_{\mathcal N} x}

We write $P \Downarrow_{\mathcal N} x$ if there is $Q$ such that 
$P \wred Q$ and $Q \downarrow_{\mathcal N} x$.
\end{definition}

\begin{definition}
%\label{def.bbisim}
An  ${\mathcal N}$-\emph{barbed bisimulation} over a set of names, ${\mathcal N}$, is a symmetric binary relation 
${\mathcal S}_{\mathcal N}$ between agents such that $P\rel{S}_{\mathcal N}Q$ implies:
\begin{enumerate}
\item If $P \red P'$ then $Q \wred Q'$ and $P'\rel{S}_{\mathcal N} Q'$.
\item If $P\downarrow_{\mathcal N} x$, then $Q\Downarrow_{\mathcal N} x$.
\end{enumerate}
$P$ is ${\mathcal N}$-barbed bisimilar to $Q$, written
$P \wbbisim_{\mathcal N} Q$, if $P \rel{S}_{\mathcal N} Q$ for some ${\mathcal N}$-barbed bisimulation ${\mathcal S}_{\mathcal N}$.
\end{definition}

$\mathcal{R} \subseteq \pi \times \pi$

$P \mathcal{R} Q => \forall P'. P \red P' \Rightarrow \exists Q'. Q \red Q', P' \mathcal{R} Q'$

$P \vdash x \Rightarrow Q \vdash x$

\begin{mathpar}
  \inferrule*[lab=Out-barb]{x \nameeq y}{{y}!\langle{Q}\rangle \vdash x}
  \and
  \inferrule*[lab=Par-barb]{\mbox{$P\vdash x$ or $Q\vdash x$}}{\binpar{P}{Q} \vdash x}
\end{mathpar}

\subsubsection{Contexts}

One of the principle advantages of computational calculi like the
$\pi$-calculus is a well-defined notion of context,
contextual-equivalence and a correlation between
contextual-equivalence and notions of bisimulation. The notion of
context allows the decomposition of a process into (sub-)process and
its syntactic environment, its context. Thus, a context may be
thought of as a process with a ``hole'' (written $\Box$) in it. The
application of a context $M$ to a process $P$, written $M[P]$, is
tantamount to filling the hole in $M$ with $P$. In this paper we do
not need the full weight of this theory, but do make use of the notion
of context in the proof the main theorem. 

\begin{mathpar}
  \inferrule* [lab=summation] {} {{M_{M},M_{N}} \bc \Box \;|\; x.M_{A} \;|\; M_{M}+M_{N}}
  \and
  \inferrule* [lab=agent] {} {{M_{A}} \bc (\vec{x})M_{P} \;| \; \clift{P_0,\ldots,M_{P},\ldots,P_N}}
  \and \\
  \inferrule* [lab=process] {} {{M_{P}} \bc M_{N} \;| \;P|M_{P} }
\end{mathpar} 

\begin{mathpar}
  \inferrule* [lab=sychronization] {} {M_{N} \bc \Box \;|\; x?M_{F} \;|\; x!M_{C}}
  \and
  \inferrule* [lab=abstraction] {} {{M_{F}} \bc (x)M_{P} }
  \and
  \inferrule* [lab=concretion] {} {{M_{C}} \bc \langle M_{P} \rangle }
  \and \\
  \inferrule* [lab=process] {} {{M_{P}} \bc M_{N} \;| \;P|M_{P} }
\end{mathpar}

\begin{definition}[contextual application] Given a context $M$, and
  process $P$, we define the \emph{contextual application}, $M[P] :=
  M\{P/\Box\}$. That is, the contextual application of M to P is the
  substitution of $P$ for $\Box$ in $M$.
\end{definition}

$\meaningof{-} : L \to \mathcal{P}(\pi)$

\begin{mathpar}
  \inferrule* [lab=collection] {} {\meaningof{true} = \pi, \and \meaningof{~E} = \pi \setminus \meaningof{E}, \and \meaningof{E_{1} \& E_{2}} = \meaningof{E_{1}} \cap \meaningof{E_{2}}}
\end{mathpar}

\begin{mathpar}
  \inferrule* [lab=structure] {} {\meaningof{0} = \{ P \in \pi | P \equiv 0 \}, \and \\ \meaningof{E_1 | E_2} = \{ P \in \pi | P \equiv P_{1} | P_{2}, P_{1} \in \meaningof{E_{1}}, P_{2} \in \meaningof{E_2}\} }
\end{mathpar}

\begin{mathpar}
 \inferrule* [lab=behavior] {} {\meaningof{\langle a?b \rangle E} = \{ P \in \pi | P \equiv Q | u?(y)P', \\ \and \\\\ \and \\ \;\;\; u \in \meaningof{a}, \forall z.P'\{z/y\} \in \meaningof{E\{z/b\}}\}, \and \\ \meaningof{a!E} = \{ P \in \pi | P \equiv Q | x!\langle P' \rangle, x \in \meaningof{a} P' \in \meaningof{E}\} }
\end{mathpar}

\begin{mathpar}
 \inferrule* [lab=nominal] {} {\meaningof{\quotep{E}} = \{ \quotep{P} \in \quotep{\pi} | P \in \meaningof{E} \}, \and \meaningof{\quotep{P}} = \{ \quotep{Q} \in \quotep{\pi} | P \equiv Q \} \and \\ \meaningof{@\quotep{E}} = \{ P \in \pi | P \equiv @x, x \in \meaningof{E} \}}
\end{mathpar}

\begin{eqnarray*}
  \\
  \meaningof{-} : TS \to ST
\end{eqnarray*}

\begin{eqnarray*}
  \\
  L : TS \to ST
\end{eqnarray*}

\begin{eqnarray*}
  \\
  P \models E \iff P \in \meaningof{E}
\end{eqnarray*}

\begin{eqnarray*}
  P \approx_{L} Q \iff \forall E \in L. P \models E \iff Q \models E
\end{eqnarray*}

\begin{eqnarray*}
  P \approx_{K} Q
\end{eqnarray*}

\begin{eqnarray*}
  P \approx Q
\end{eqnarray*}

$\approx_{K} = \approx = \approx_{L}$

\subsubsection{Contextual duality}

Note that contexts extend the quotation operation to a family of
operations from processes to names. Given a context, $M$, we can
define a \emph{nominal context}, $\quotep{M}$ by $\quotep{M}[P] :=
\quotep{M[P]}$. To foreshadow what is to come we observe that these
operations enjoy a duality with processes very much like the duality
between vectors and maps from vectors to scalars.

Further, because the calculus is essentially higher-order, we have a
correspondence between contexts and processes. More specifically,
given a name $x$ and a context $M$ we can construct $M^{*}_{x}$ such
that 

\begin{mathpar}
  M^{*}_{x} | \lift{x}{P} \red M[P]
\end{mathpar}

namely,

\begin{mathpar}
  M^{*}_{x} := x?(u).M[\dropn{u}]
\end{mathpar}

The dependence of $M^{*}_{x}$ on a name makes it an abstraction, 

\begin{mathpar}
  M^{*} := (x)x?(u).M[\dropn{u}]
\end{mathpar}

\subsection{Additional notation}

It will sometimes be convenient to denote the process a name
quotes. We already have the notation $x = \quotep{P}$, but it will be
convenient to introduce an alternate notation, $\procn{x}$, when we
want to emphasize the connection to the use of the name. Note that, by
virtue of name equivalence, $\quotep{\procn{x}} \nameeq x$; so, the
notation is consistent with previous definitions.

Further, because names have structure it is possible to effect
substitutions on the basis of that structure. This means we need to
upgrade our notation for substitutions, which we accomplish by
adapting comprehension notation. Thus,

\begin{mathpar}
  P\{ y / x : x \in S \}
\end{mathpar}

is interpreted to mean the process derived from P by replacing (in a
capture-avoiding manner) each occurrence of $x$ in $S$ by $y$. For example,

\begin{mathpar}
  P\{ \quotep{\procn{x}|\procn{x}} / x : x \in \freenames{P} \}
\end{mathpar}

will replace each (occurrence) of a free name $x$ in $P$ by
$\quotep{\procn{x}|\procn{x}}$.

Also, we will avail ourselves of the notation $x^{L}$ and $x^{R}$ to
denote injections of a name into disjoint copies of the name
space. There are numerous ways to accomplish this. One example can be
found in \cite{MeredithR05}. This notation overloads to vectors of
names: $\vec{x}^{\pi} := (x_{i}^{\pi} \; : \; 0 \leq i < |\vec{x}| )$ where $\pi \in \{L,R\}$.

We also use $P^{\Box} := P|\Box$.

In \cite{MeredithR05} an interpretation of the new operator is
given. It turns out that there are several possible interpretations
all enjoying the requisite algebraic properties of the operator (see
\cite{milner91polyadicpi}). We will therefore make liberal use of
$(\nu\; \vec{x})P$.

% subsection the_syntax_and_semantics_of_the_notation_system (end)   

\input{qm2pi.qmops} 

\input{qm2pi.sterngerlach} 

\input{qm2pi.metric} 

% section concurrent_process_calculi (end)

%\input{qm2pi.proofsketch}

% section proof sketch (end)

%\input{qm2pi.slviaknots} 

% section spatial logic via knots (end)

\input{qm2pi.conclusion}

% section conclusion (end)

%\input{qm2pi.dtcodes} 

% section wiring algorithm (end)

\input{qm2pi.ack} 

% section acknowledgments (end)

\newpage


\bibliographystyle{plain}   
\bibliography{../../biblios/main.bib}

\input{qm2pi.rhodetails}

\end{document}



\end{document}

 

% section wiring algorithm (end)

\documentclass[12pt]{llncs}
%\documentclass{jktr}

\usepackage[pdftex]{hyperref}                   
\usepackage {listings}
\usepackage {mathpartir}
\usepackage{bcprules}
%\usepackage{listings}
                       
\usepackage{graphicx} 
%\usepackage[margins=2.5cm,nohead,nofoot]{geometry}
%\usepackage{geometry}
\usepackage{amsfonts}
\usepackage{amstext}
\usepackage{latexsym}
\usepackage{amssymb}
\usepackage{color}


%\include{myPreamble}
\documentclass[12pt]{llncs}
%\documentclass{jktr}

\usepackage[pdftex]{hyperref}                   
\usepackage {listings}
\usepackage {mathpartir}
\usepackage{bcprules}
%\usepackage{listings}
                       
\usepackage{graphicx} 
%\usepackage[margins=2.5cm,nohead,nofoot]{geometry}
%\usepackage{geometry}
\usepackage{amsfonts}
\usepackage{amstext}
\usepackage{latexsym}
\usepackage{amssymb}
\usepackage{color}


%\include{myPreamble}
\include{qm2pi.local} 

%\ifpdf
%\usepackage[pdftex]{graphicx}
%\else
%\usepackage{graphicx}
%\fi

 % \ifpdf
%  \usepackage{pdfsync}
%  \if


%\title{Brief Article}
%\author{David F. Snyder}
%\author{L.G. Meredith}

%\address{Dept. of Math., Texas State University--San Marcos, San Marcos, TX 78666}
       
\pagestyle{empty}


\begin{document}

\lstset{language=[Objective]Caml,frame=shadowbox}

\input{qm2pi.front}

% section front matter (end)

\input{qm2pi.intro} 
 
% section introduction (end)

% \input{qm2pi.knotations} 

% section notation (end)

\input{qm2pi.process.calculi} 

% section concurrent_process_calculi_and_spatial_logics_ (end)
    
%\input{qm2pi.knots2pi} 

%\input{qm2pi.trefoil} 

%\input{qm2pi.mainthm} 

% subsection basic_interpretation (end)

%\input{qm2pi.rho.presentation} 
\subsection{The syntax and semantics of the notation system}\label{sub:the_syntax_and_semantics_of_the_notation_system} % (fold)

We now summarize a technical presentation of the calculus that
embodies our theory of dynamics. The typical presentation of such a
calculus follows the style of giving generators and relations on
them. The grammar, below, describing term constructors, freely
generates the set of processes, $\Proc$. This set is then quotiented
by a relation known as structural congruence and it is over this set
that the notion of dynamics is expressed. This presentation is
essentially that of \cite{MeredithR05} with the addition of
polyadicity and summation. For readability we have relegated some of
the technical subtleties to an appendix.

\subsubsection{Process grammar}\label{subsub:process_grammar}

\begin{mathpar}
  \inferrule* [lab=synchronization] {} {{M} \bc \pzero \;|\; x?F \;|\; x!C }
  \and
  \inferrule* [lab=abstraction] {} {{F} \bc (x)P}
  \and
  \inferrule* [lab=concretion] {} {{C} \bc \langle Q \rangle}
  \and
  \inferrule* [lab=process] {} {{P,Q} \bc M \;| \;P|Q \;|\; @{x}}
  \and
  \inferrule* [lab=name] {} {{x} \bc \quotep{P}}
\end{mathpar} 

Note that $\vec{x}$ (resp. $\vec{P}$) denotes a vector of names
(resp. processes) of length $|\vec{x}|$ (resp. $|\vec{P}|$). We adopt
the following useful abbreviations.

\begin{mathpar}
   x?(\vec{y}).P := x.(\vec{y})P \and  x\clift{\vec{P}} := x.\clift{\vec{P}}
   \and x!(y) := \lift{x}{\dropn{y}}
   \and \Pi_{i=0}^{n-1}P_i := P_0 | \ldots | P_{n-1}
\end{mathpar}

\subsubsection{Structural congruence}

\paragraph{Free and bound names and alpha-equivalence.} At the
core of structural equivalence is alpha-equivalence which identifies
process that are the same up to a change of variable. Formally, we
recognize the distinction between free and bound names. The free names
of a process, $\freenames{P}$, may be calculated recursively as
follows:

\begin{mathpar}
\freenames{\pzero} := \emptyset
  \and \\
  \freenames{x?(y).P} := \{ x \} \cup (\freenames{P} \setminus \{ y \})
  \and 
  \freenames{x!\langle P \rangle} := \{ x \} \cup \{ P \} 
  \and \\
  \freenames{P|Q} := \freenames{P} \cup \freenames{Q}
  \and \\
  \freenames{@{x}} := \{ x \}
\end{mathpar}

$\pi$
$\quotep{\pi}$

$\freenames{-} : \pi \to \mathcal{P}(\quotep{\pi})$

\begin{eqnarray*}
  \freenames{\pzero} & := & \emptyset \\
  \freenames{x?(y).P} & := & \{ x \} \cup (\freenames{P} \setminus \{ y \}) \\
  \freenames{x!\langle P \rangle} & := & \{ x \} \cup \{ P \} \\
  \freenames{P|Q} & := & \freenames{P} \cup \freenames{Q} \\
  \freenames{\dropn{x}} & := & \{ x \}
\end{eqnarray*}

The bound names of a process, $\boundnames{P}$, are those names occurring in $P$
that are not free. For example, in $x?(y).0$, the name $x$ is free, while $y$ is bound.

\begin{mathpar}
  \inferrule* [lab=monoidal-laws] {} { P|Q \equiv Q|P \and P|0 \equiv P \and P|(Q|R) \equiv (P|Q)|R }
\end{mathpar}

\begin{mathpar}
  \inferrule* [lab=alpha-equivalence] {} { (x)P \equiv (y)P\{y/x\} \and y \not\in \freenames{P} }
\end{mathpar}

\begin{definition}
Then two processes, $P,Q$, are alpha-equivalent if $P = Q\{\vec{y}/\vec{x}\}$ for
some $\vec{x} \in \boundnames{Q},\vec{y} \in \boundnames{P}$, where $Q\{\vec{y}/\vec{x}\}$
denotes the capture-avoiding substitution of $\vec{y}$ for $\vec{x}$ in $Q$.
\end{definition}

\begin{definition}
  The {\em structural congruence} \cite{SangiorgiWalker} , $\equiv$,
  between processes is the least congruence containing
  alpha-equivalence, satisfying the abelian monoid laws
  (associativity, commutativity and $\pzero$ as identity) for parallel
  composition $|$ and for summation $+$.
\end{definition}

\subsection{Name equivalence}

We take name equivalence, written $\nameeq$, to be the smallest
equivalence relation generated by the following rules.

\begin{mathpar}
\inferrule*[lab=Quote-drop]
{ }
{ \quotep{@{x}} \nameeq x }

\inferrule*[lab=Struct-equiv]
{ P \scong Q }
{ \quotep{P} \nameeq \quotep{Q} }
\end{mathpar}

The astute reader will have noticed that the mutual recursion of names
and processes imposes a mutual recursion on alpha-equivalence and
structural equivalence via name-equivalence. Fortunately, all of this
works out pleasantly and we may calculate in the natural way, free of
concern. The reader interested in the details is referred to the
appendix \ref{appendix:rho_details}.

\subsection{Substitution}

We use $\Proc$ for the set of processes, $\QProc$ for the set of
names, and $\id{\{}\vec{y} / \vec{x} \id{\}}$ to denote partial maps,
$s : \QProc \rightarrow \QProc$. A map, $s$ lifts, uniquely, to a map
on process terms, $\widehat{s} : \Proc \rightarrow \Proc$ by the
following equations.

\begin{mathpar}
  (0) \psubstp{Q}{P} := 0 \\
  (R \juxtap S) \psubstp{Q}{P}
  :=    
  (R)\psubstp{Q}{P} \juxtap (S) \psubstp{Q}{P} \\
  (x?(y).R) \psubstp{Q}{P}    
  :=    
  (x)\substp{Q}{P} (z)\concat( (R \psubstn{z}{y}) \psubstp{Q}{P} ) \\
  (\lift{x}{R}) \psubstp{Q}{P}  
  :=
  \lift{(x)\substp{Q}{P}}{ R \psubstp{Q}{P} } \\
%   (\dropn{x})  \psubstp{Q}{P}       
%   := 
%   \left\{ 
%     \begin{array}{ccc} 
%       \dropn{\quotep{Q}} & & x \nameeq \quotep{P} \\
%       \dropn{x} & & otherwise \\
%     \end{array}
%   \right. 
  (\dropn{x})  \psubstp{Q}{P}       
  := 
  \left\{ 
    \begin{array}{ccc} 
      Q & & x \nameeq \quotep{P} \\
      \dropn{x} & & otherwise \\
    \end{array}
  \right.
\end{mathpar}
 

where

\begin{eqnarray}
  (x)\id{\{} \lpquote Q \rpquote / \lpquote P \rpquote \id{\}}            = 
  \left\{ 
    \begin{array}{ccc}
      \lpquote Q \rpquote & & x \nameeq \lpquote P \rpquote \\
      x & & otherwise \\
    \end{array}
  \right. \nonumber
\end{eqnarray}

and $z$ is chosen distinct from $\quotep{P}$, $\quotep{Q}$, the free
names in $Q$, and all the names in $R$. Our $\alpha$-equivalence will
be built in the standard way from this substitution.

\begin{remark}\label{rem:no_self_referential_names}
  One consequence of these definitions is that $\forall P. \quotep{P}
  \not\in \freenames{P}$.
\end{remark}

\subsection{ Dynamic quote: an example }

Anticipating something of what's to come, consider applying the
substitution, $\widehat{\id{\{}u / z \id{\}}}$, to the following pair
of processes, $\lift{w}{y!(z)}$ and $w[ \lpquote y!(z) \rpquote ]$.

\begin{eqnarray}
	\lift{w}{y!(z)}\widehat{\id{\{}u / z \id{\}}}
		& = &
		\lift{w}{y!(u)} \nonumber\\
	w[ \lpquote y!(z) \rpquote ] \widehat{ \id{\{}u / z \id{\}} }
		& = &
		w[ \lpquote y!(z) \rpquote ] \nonumber
\end{eqnarray}

Because the body of the process between quotes is impervious to
substitution, we get radically different answers. In fact, by
examining the first process in an input context,
e.g. $x?(z).\lift{w}{y!(z)}$, we see that the process under the lift
operator may be shaped by prefixed inputs binding a name inside it. In
this sense, the lift operator will be seen as a way to dynamically
construct processes before reifying them as names.

Finally equipped with these standard features we can present the
dynamics of the calculus.

\subsubsection{Operational semantics} 

Finally, we introduce the computational dynamics. What marks these
algebras as distinct from other more traditionally studied algebraic
structures, e.g. vector spaces or polynomial rings, is the manner in
which dynamics is captured. In traditional structures, dynamics is typically
expressed through morphisms between such structures, as in linear maps
between vector spaces or morphisms between rings. In algebras
associated with the semantics of computation, the dynamics is
expressed as part of the algebraic structure itself, through a
reduction reduction relation typically denoted by $\red$. Below, we
give a recursive presentation of this relation for the calculus used
in the encoding.

$\red \subseteq \pi \times \pi$
$\red : \pi \to \mathcal{P}(\pi)$

\begin{mathpar}
  \inferrule* [lab=Comm] { \textsf{match}( x_{src}, x_{trgt} ) } { x_{trgt}?(y)P \; | \; x_{src}!\langle {Q} \rangle \red P\{\quotep{Q}/y}\} }
  \and \\
  \inferrule* [lab=Par] {{P} \red {P}'} {{{P} | {Q}} \red {{P}' | {Q}}}
  \and
  \inferrule* [lab=Equiv]{{{P} \scong {P}'} \andalso {{P}' \red {Q}'} \andalso {{Q}' \scong {Q}}}{{P} \red {Q}}
\end{mathpar}

\begin{eqnarray*}
  match_{\equiv} (\quotep{P},\quotep{Q}) & := & P \equiv Q \\
  match_{\dagger}(\quotep{P},\quotep{Q}) & := & \forall R. P|Q \red^{*} R => R \red^{*} 0 \\
  match_{K}(\quotep{P},\quotep{Q}) & := & K \mbox{ for some context } K
\end{eqnarray*}

$u?(x)P | u!\langle Q \rangle \red P\{\quotep{Q}/x\}$

%We write $\wred$ for $\red^*$, and $P\red$ if $\exists Q $ such that $ P \red Q$.
We write $P\red$ if $\exists Q $ such that $ P \red Q$ and $P\not\red$, otherwise.

\section{Replication}

As mentioned before, it is known that replication (and hence
recursion) can be implemented in a higher-order process algebra
\cite{SangiorgiWalker}. As our first example of calculation with the
machinery thus far presented we give the construction explicitly in
the {\rhoc}.

\begin{eqnarray}
	D_{x} & := & \prefix{x}{y}{(\binpar{\outputp{x}{y}}{@{y}})} \nonumber\\
	\bangp_{x}{P} & := & \binpar{{x}!\langle{\binpar{D_{x}}{P}}\rangle}{D_{x}} \nonumber
\end{eqnarray}

\begin{eqnarray}
	\bangp_{x}{P} & & \nonumber\\
	=
	& {x}!\langle{(\prefix{x}{y}{(\outputp{x}{y} | @{y})) | P}}\rangle 
	      | \prefix{x}{y}{(\outputp{x}{y} | @{y})} & \nonumber\\
	\red
	& (\outputp{x}{y} | @{y})\substn{\quotep{(\prefix{x}{y}{(@{y} | \outputp{x}{y})) | P}}}{y} & \nonumber\\
	=
	& \outputp{x}{\quotep{(\prefix{x}{y}{(\outputp{x}{y} | @{y})) | P}}}
	  | {(\prefix{x}{y}{(\outputp{x}{y} | @{y})) | P}} & \nonumber\\
	\red
	& \ldots & \nonumber\\
	\red^*
	& P | P | \ldots & \nonumber
\end{eqnarray}

Of course, this encoding, as an implementation, runs away, unfolding
$\bangp{P}$ eagerly. A lazier and more implementable replication
operator, restricted to input-guarded processes, may be obtained as follows.

\begin{eqnarray}
\bangp{\prefix{u}{v}{P}} 
	:= 
	\binpar{\lift{x}{\prefix{u}{v}{(\binpar{D(x)}{P})}}}{D(x)} \nonumber
\end{eqnarray}

\begin{remark}
  Note that the lazier definition still does not deal with summation
  or mixed summation (i.e. sums over input and output). The reader is
  invited to construct definitions of replication that deal with these
  features. 

  Further, the definitions are parameterized in a name, $x$. Can you,
  gentle reader, make a definition that eliminates this parameter and
  guarantees no accidental interaction between the replication
  machinery and the process being replicated -- i.e. no accidental
  sharing of names used by the process to get its work done and the
  name(s) used by the replication to effect copying. This latter
  revision of the definition of replication is crucial to obtaining
  the expected identity $!!P \sim !P$.
\end{remark}

\begin{remark}\label{rem:paradoxical_combinator}
  The reader familiar with the lambda calculus will have noticed the
  similarity between $D$ and the paradoxical combinator.

  [Ed. note: the existence of this seems to suggest we have to be more
  restrictive on the set of processes and names we admit if we are to
  support no-cloning.]
\end{remark}

\subsubsection{Bisimulation}

The computational dynamics gives rise to another kind of equivalence,
the equivalence of computational behavior. As previously mentioned
this is typically captured \emph{via} some form of bisimulation.

% The notion we use in this paper is weak barbed bisimulation
% \cite{milner91polyadicpi}.

The notion we use in this paper is derived from weak barbed
bisimulation \cite{milner91polyadicpi}. 

\begin{definition}
An \emph{observation relation}, $\downarrow_{\mathcal N}$, over a set
of names, $\mathcal N$, is the smallest relation satisfying the rules
below.

\infrule[Out-barb]{y \in {\mathcal N}, \; x \nameeq y}
		  {\outputp{x}{v} \downarrow_{\mathcal N} x}
\infrule[Par-barb]{\mbox{$P\downarrow_{\mathcal N} x$ or $Q\downarrow_{\mathcal N} x$}}
		  {\binpar{P}{Q} \downarrow_{\mathcal N} x}

We write $P \Downarrow_{\mathcal N} x$ if there is $Q$ such that 
$P \wred Q$ and $Q \downarrow_{\mathcal N} x$.
\end{definition}

\begin{definition}
%\label{def.bbisim}
An  ${\mathcal N}$-\emph{barbed bisimulation} over a set of names, ${\mathcal N}$, is a symmetric binary relation 
${\mathcal S}_{\mathcal N}$ between agents such that $P\rel{S}_{\mathcal N}Q$ implies:
\begin{enumerate}
\item If $P \red P'$ then $Q \wred Q'$ and $P'\rel{S}_{\mathcal N} Q'$.
\item If $P\downarrow_{\mathcal N} x$, then $Q\Downarrow_{\mathcal N} x$.
\end{enumerate}
$P$ is ${\mathcal N}$-barbed bisimilar to $Q$, written
$P \wbbisim_{\mathcal N} Q$, if $P \rel{S}_{\mathcal N} Q$ for some ${\mathcal N}$-barbed bisimulation ${\mathcal S}_{\mathcal N}$.
\end{definition}

$\mathcal{R} \subseteq \pi \times \pi$

$P \mathcal{R} Q => \forall P'. P \red P' \Rightarrow \exists Q'. Q \red Q', P' \mathcal{R} Q'$

$P \vdash x \Rightarrow Q \vdash x$

\begin{mathpar}
  \inferrule*[lab=Out-barb]{x \nameeq y}{{y}!\langle{Q}\rangle \vdash x}
  \and
  \inferrule*[lab=Par-barb]{\mbox{$P\vdash x$ or $Q\vdash x$}}{\binpar{P}{Q} \vdash x}
\end{mathpar}

\subsubsection{Contexts}

One of the principle advantages of computational calculi like the
$\pi$-calculus is a well-defined notion of context,
contextual-equivalence and a correlation between
contextual-equivalence and notions of bisimulation. The notion of
context allows the decomposition of a process into (sub-)process and
its syntactic environment, its context. Thus, a context may be
thought of as a process with a ``hole'' (written $\Box$) in it. The
application of a context $M$ to a process $P$, written $M[P]$, is
tantamount to filling the hole in $M$ with $P$. In this paper we do
not need the full weight of this theory, but do make use of the notion
of context in the proof the main theorem. 

\begin{mathpar}
  \inferrule* [lab=summation] {} {{M_{M},M_{N}} \bc \Box \;|\; x.M_{A} \;|\; M_{M}+M_{N}}
  \and
  \inferrule* [lab=agent] {} {{M_{A}} \bc (\vec{x})M_{P} \;| \; \clift{P_0,\ldots,M_{P},\ldots,P_N}}
  \and \\
  \inferrule* [lab=process] {} {{M_{P}} \bc M_{N} \;| \;P|M_{P} }
\end{mathpar} 

\begin{mathpar}
  \inferrule* [lab=sychronization] {} {M_{N} \bc \Box \;|\; x?M_{F} \;|\; x!M_{C}}
  \and
  \inferrule* [lab=abstraction] {} {{M_{F}} \bc (x)M_{P} }
  \and
  \inferrule* [lab=concretion] {} {{M_{C}} \bc \langle M_{P} \rangle }
  \and \\
  \inferrule* [lab=process] {} {{M_{P}} \bc M_{N} \;| \;P|M_{P} }
\end{mathpar}

\begin{definition}[contextual application] Given a context $M$, and
  process $P$, we define the \emph{contextual application}, $M[P] :=
  M\{P/\Box\}$. That is, the contextual application of M to P is the
  substitution of $P$ for $\Box$ in $M$.
\end{definition}

$\meaningof{-} : L \to \mathcal{P}(\pi)$

\begin{mathpar}
  \inferrule* [lab=collection] {} {\meaningof{true} = \pi, \and \meaningof{~E} = \pi \setminus \meaningof{E}, \and \meaningof{E_{1} \& E_{2}} = \meaningof{E_{1}} \cap \meaningof{E_{2}}}
\end{mathpar}

\begin{mathpar}
  \inferrule* [lab=structure] {} {\meaningof{0} = \{ P \in \pi | P \equiv 0 \}, \and \\ \meaningof{E_1 | E_2} = \{ P \in \pi | P \equiv P_{1} | P_{2}, P_{1} \in \meaningof{E_{1}}, P_{2} \in \meaningof{E_2}\} }
\end{mathpar}

\begin{mathpar}
 \inferrule* [lab=behavior] {} {\meaningof{\langle a?b \rangle E} = \{ P \in \pi | P \equiv Q | u?(y)P', \\ \and \\\\ \and \\ \;\;\; u \in \meaningof{a}, \forall z.P'\{z/y\} \in \meaningof{E\{z/b\}}\}, \and \\ \meaningof{a!E} = \{ P \in \pi | P \equiv Q | x!\langle P' \rangle, x \in \meaningof{a} P' \in \meaningof{E}\} }
\end{mathpar}

\begin{mathpar}
 \inferrule* [lab=nominal] {} {\meaningof{\quotep{E}} = \{ \quotep{P} \in \quotep{\pi} | P \in \meaningof{E} \}, \and \meaningof{\quotep{P}} = \{ \quotep{Q} \in \quotep{\pi} | P \equiv Q \} \and \\ \meaningof{@\quotep{E}} = \{ P \in \pi | P \equiv @x, x \in \meaningof{E} \}}
\end{mathpar}

\begin{eqnarray*}
  \\
  \meaningof{-} : TS \to ST
\end{eqnarray*}

\begin{eqnarray*}
  \\
  L : TS \to ST
\end{eqnarray*}

\begin{eqnarray*}
  \\
  P \models E \iff P \in \meaningof{E}
\end{eqnarray*}

\begin{eqnarray*}
  P \approx_{L} Q \iff \forall E \in L. P \models E \iff Q \models E
\end{eqnarray*}

\begin{eqnarray*}
  P \approx_{K} Q
\end{eqnarray*}

\begin{eqnarray*}
  P \approx Q
\end{eqnarray*}

$\approx_{K} = \approx = \approx_{L}$

\subsubsection{Contextual duality}

Note that contexts extend the quotation operation to a family of
operations from processes to names. Given a context, $M$, we can
define a \emph{nominal context}, $\quotep{M}$ by $\quotep{M}[P] :=
\quotep{M[P]}$. To foreshadow what is to come we observe that these
operations enjoy a duality with processes very much like the duality
between vectors and maps from vectors to scalars.

Further, because the calculus is essentially higher-order, we have a
correspondence between contexts and processes. More specifically,
given a name $x$ and a context $M$ we can construct $M^{*}_{x}$ such
that 

\begin{mathpar}
  M^{*}_{x} | \lift{x}{P} \red M[P]
\end{mathpar}

namely,

\begin{mathpar}
  M^{*}_{x} := x?(u).M[\dropn{u}]
\end{mathpar}

The dependence of $M^{*}_{x}$ on a name makes it an abstraction, 

\begin{mathpar}
  M^{*} := (x)x?(u).M[\dropn{u}]
\end{mathpar}

\subsection{Additional notation}

It will sometimes be convenient to denote the process a name
quotes. We already have the notation $x = \quotep{P}$, but it will be
convenient to introduce an alternate notation, $\procn{x}$, when we
want to emphasize the connection to the use of the name. Note that, by
virtue of name equivalence, $\quotep{\procn{x}} \nameeq x$; so, the
notation is consistent with previous definitions.

Further, because names have structure it is possible to effect
substitutions on the basis of that structure. This means we need to
upgrade our notation for substitutions, which we accomplish by
adapting comprehension notation. Thus,

\begin{mathpar}
  P\{ y / x : x \in S \}
\end{mathpar}

is interpreted to mean the process derived from P by replacing (in a
capture-avoiding manner) each occurrence of $x$ in $S$ by $y$. For example,

\begin{mathpar}
  P\{ \quotep{\procn{x}|\procn{x}} / x : x \in \freenames{P} \}
\end{mathpar}

will replace each (occurrence) of a free name $x$ in $P$ by
$\quotep{\procn{x}|\procn{x}}$.

Also, we will avail ourselves of the notation $x^{L}$ and $x^{R}$ to
denote injections of a name into disjoint copies of the name
space. There are numerous ways to accomplish this. One example can be
found in \cite{MeredithR05}. This notation overloads to vectors of
names: $\vec{x}^{\pi} := (x_{i}^{\pi} \; : \; 0 \leq i < |\vec{x}| )$ where $\pi \in \{L,R\}$.

We also use $P^{\Box} := P|\Box$.

In \cite{MeredithR05} an interpretation of the new operator is
given. It turns out that there are several possible interpretations
all enjoying the requisite algebraic properties of the operator (see
\cite{milner91polyadicpi}). We will therefore make liberal use of
$(\nu\; \vec{x})P$.

% subsection the_syntax_and_semantics_of_the_notation_system (end)   

\input{qm2pi.qmops} 

\input{qm2pi.sterngerlach} 

\input{qm2pi.metric} 

% section concurrent_process_calculi (end)

%\input{qm2pi.proofsketch}

% section proof sketch (end)

%\input{qm2pi.slviaknots} 

% section spatial logic via knots (end)

\input{qm2pi.conclusion}

% section conclusion (end)

%\input{qm2pi.dtcodes} 

% section wiring algorithm (end)

\input{qm2pi.ack} 

% section acknowledgments (end)

\newpage


\bibliographystyle{plain}   
\bibliography{../../biblios/main.bib}

\input{qm2pi.rhodetails}

\end{document}

 

%\ifpdf
%\usepackage[pdftex]{graphicx}
%\else
%\usepackage{graphicx}
%\fi

 % \ifpdf
%  \usepackage{pdfsync}
%  \if


%\title{Brief Article}
%\author{David F. Snyder}
%\author{L.G. Meredith}

%\address{Dept. of Math., Texas State University--San Marcos, San Marcos, TX 78666}
       
\pagestyle{empty}


\begin{document}

\lstset{language=[Objective]Caml,frame=shadowbox}

\documentclass[12pt]{llncs}
%\documentclass{jktr}

\usepackage[pdftex]{hyperref}                   
\usepackage {listings}
\usepackage {mathpartir}
\usepackage{bcprules}
%\usepackage{listings}
                       
\usepackage{graphicx} 
%\usepackage[margins=2.5cm,nohead,nofoot]{geometry}
%\usepackage{geometry}
\usepackage{amsfonts}
\usepackage{amstext}
\usepackage{latexsym}
\usepackage{amssymb}
\usepackage{color}


%\include{myPreamble}
\include{qm2pi.local} 

%\ifpdf
%\usepackage[pdftex]{graphicx}
%\else
%\usepackage{graphicx}
%\fi

 % \ifpdf
%  \usepackage{pdfsync}
%  \if


%\title{Brief Article}
%\author{David F. Snyder}
%\author{L.G. Meredith}

%\address{Dept. of Math., Texas State University--San Marcos, San Marcos, TX 78666}
       
\pagestyle{empty}


\begin{document}

\lstset{language=[Objective]Caml,frame=shadowbox}

\input{qm2pi.front}

% section front matter (end)

\input{qm2pi.intro} 
 
% section introduction (end)

% \input{qm2pi.knotations} 

% section notation (end)

\input{qm2pi.process.calculi} 

% section concurrent_process_calculi_and_spatial_logics_ (end)
    
%\input{qm2pi.knots2pi} 

%\input{qm2pi.trefoil} 

%\input{qm2pi.mainthm} 

% subsection basic_interpretation (end)

%\input{qm2pi.rho.presentation} 
\subsection{The syntax and semantics of the notation system}\label{sub:the_syntax_and_semantics_of_the_notation_system} % (fold)

We now summarize a technical presentation of the calculus that
embodies our theory of dynamics. The typical presentation of such a
calculus follows the style of giving generators and relations on
them. The grammar, below, describing term constructors, freely
generates the set of processes, $\Proc$. This set is then quotiented
by a relation known as structural congruence and it is over this set
that the notion of dynamics is expressed. This presentation is
essentially that of \cite{MeredithR05} with the addition of
polyadicity and summation. For readability we have relegated some of
the technical subtleties to an appendix.

\subsubsection{Process grammar}\label{subsub:process_grammar}

\begin{mathpar}
  \inferrule* [lab=synchronization] {} {{M} \bc \pzero \;|\; x?F \;|\; x!C }
  \and
  \inferrule* [lab=abstraction] {} {{F} \bc (x)P}
  \and
  \inferrule* [lab=concretion] {} {{C} \bc \langle Q \rangle}
  \and
  \inferrule* [lab=process] {} {{P,Q} \bc M \;| \;P|Q \;|\; @{x}}
  \and
  \inferrule* [lab=name] {} {{x} \bc \quotep{P}}
\end{mathpar} 

Note that $\vec{x}$ (resp. $\vec{P}$) denotes a vector of names
(resp. processes) of length $|\vec{x}|$ (resp. $|\vec{P}|$). We adopt
the following useful abbreviations.

\begin{mathpar}
   x?(\vec{y}).P := x.(\vec{y})P \and  x\clift{\vec{P}} := x.\clift{\vec{P}}
   \and x!(y) := \lift{x}{\dropn{y}}
   \and \Pi_{i=0}^{n-1}P_i := P_0 | \ldots | P_{n-1}
\end{mathpar}

\subsubsection{Structural congruence}

\paragraph{Free and bound names and alpha-equivalence.} At the
core of structural equivalence is alpha-equivalence which identifies
process that are the same up to a change of variable. Formally, we
recognize the distinction between free and bound names. The free names
of a process, $\freenames{P}$, may be calculated recursively as
follows:

\begin{mathpar}
\freenames{\pzero} := \emptyset
  \and \\
  \freenames{x?(y).P} := \{ x \} \cup (\freenames{P} \setminus \{ y \})
  \and 
  \freenames{x!\langle P \rangle} := \{ x \} \cup \{ P \} 
  \and \\
  \freenames{P|Q} := \freenames{P} \cup \freenames{Q}
  \and \\
  \freenames{@{x}} := \{ x \}
\end{mathpar}

$\pi$
$\quotep{\pi}$

$\freenames{-} : \pi \to \mathcal{P}(\quotep{\pi})$

\begin{eqnarray*}
  \freenames{\pzero} & := & \emptyset \\
  \freenames{x?(y).P} & := & \{ x \} \cup (\freenames{P} \setminus \{ y \}) \\
  \freenames{x!\langle P \rangle} & := & \{ x \} \cup \{ P \} \\
  \freenames{P|Q} & := & \freenames{P} \cup \freenames{Q} \\
  \freenames{\dropn{x}} & := & \{ x \}
\end{eqnarray*}

The bound names of a process, $\boundnames{P}$, are those names occurring in $P$
that are not free. For example, in $x?(y).0$, the name $x$ is free, while $y$ is bound.

\begin{mathpar}
  \inferrule* [lab=monoidal-laws] {} { P|Q \equiv Q|P \and P|0 \equiv P \and P|(Q|R) \equiv (P|Q)|R }
\end{mathpar}

\begin{mathpar}
  \inferrule* [lab=alpha-equivalence] {} { (x)P \equiv (y)P\{y/x\} \and y \not\in \freenames{P} }
\end{mathpar}

\begin{definition}
Then two processes, $P,Q$, are alpha-equivalent if $P = Q\{\vec{y}/\vec{x}\}$ for
some $\vec{x} \in \boundnames{Q},\vec{y} \in \boundnames{P}$, where $Q\{\vec{y}/\vec{x}\}$
denotes the capture-avoiding substitution of $\vec{y}$ for $\vec{x}$ in $Q$.
\end{definition}

\begin{definition}
  The {\em structural congruence} \cite{SangiorgiWalker} , $\equiv$,
  between processes is the least congruence containing
  alpha-equivalence, satisfying the abelian monoid laws
  (associativity, commutativity and $\pzero$ as identity) for parallel
  composition $|$ and for summation $+$.
\end{definition}

\subsection{Name equivalence}

We take name equivalence, written $\nameeq$, to be the smallest
equivalence relation generated by the following rules.

\begin{mathpar}
\inferrule*[lab=Quote-drop]
{ }
{ \quotep{@{x}} \nameeq x }

\inferrule*[lab=Struct-equiv]
{ P \scong Q }
{ \quotep{P} \nameeq \quotep{Q} }
\end{mathpar}

The astute reader will have noticed that the mutual recursion of names
and processes imposes a mutual recursion on alpha-equivalence and
structural equivalence via name-equivalence. Fortunately, all of this
works out pleasantly and we may calculate in the natural way, free of
concern. The reader interested in the details is referred to the
appendix \ref{appendix:rho_details}.

\subsection{Substitution}

We use $\Proc$ for the set of processes, $\QProc$ for the set of
names, and $\id{\{}\vec{y} / \vec{x} \id{\}}$ to denote partial maps,
$s : \QProc \rightarrow \QProc$. A map, $s$ lifts, uniquely, to a map
on process terms, $\widehat{s} : \Proc \rightarrow \Proc$ by the
following equations.

\begin{mathpar}
  (0) \psubstp{Q}{P} := 0 \\
  (R \juxtap S) \psubstp{Q}{P}
  :=    
  (R)\psubstp{Q}{P} \juxtap (S) \psubstp{Q}{P} \\
  (x?(y).R) \psubstp{Q}{P}    
  :=    
  (x)\substp{Q}{P} (z)\concat( (R \psubstn{z}{y}) \psubstp{Q}{P} ) \\
  (\lift{x}{R}) \psubstp{Q}{P}  
  :=
  \lift{(x)\substp{Q}{P}}{ R \psubstp{Q}{P} } \\
%   (\dropn{x})  \psubstp{Q}{P}       
%   := 
%   \left\{ 
%     \begin{array}{ccc} 
%       \dropn{\quotep{Q}} & & x \nameeq \quotep{P} \\
%       \dropn{x} & & otherwise \\
%     \end{array}
%   \right. 
  (\dropn{x})  \psubstp{Q}{P}       
  := 
  \left\{ 
    \begin{array}{ccc} 
      Q & & x \nameeq \quotep{P} \\
      \dropn{x} & & otherwise \\
    \end{array}
  \right.
\end{mathpar}
 

where

\begin{eqnarray}
  (x)\id{\{} \lpquote Q \rpquote / \lpquote P \rpquote \id{\}}            = 
  \left\{ 
    \begin{array}{ccc}
      \lpquote Q \rpquote & & x \nameeq \lpquote P \rpquote \\
      x & & otherwise \\
    \end{array}
  \right. \nonumber
\end{eqnarray}

and $z$ is chosen distinct from $\quotep{P}$, $\quotep{Q}$, the free
names in $Q$, and all the names in $R$. Our $\alpha$-equivalence will
be built in the standard way from this substitution.

\begin{remark}\label{rem:no_self_referential_names}
  One consequence of these definitions is that $\forall P. \quotep{P}
  \not\in \freenames{P}$.
\end{remark}

\subsection{ Dynamic quote: an example }

Anticipating something of what's to come, consider applying the
substitution, $\widehat{\id{\{}u / z \id{\}}}$, to the following pair
of processes, $\lift{w}{y!(z)}$ and $w[ \lpquote y!(z) \rpquote ]$.

\begin{eqnarray}
	\lift{w}{y!(z)}\widehat{\id{\{}u / z \id{\}}}
		& = &
		\lift{w}{y!(u)} \nonumber\\
	w[ \lpquote y!(z) \rpquote ] \widehat{ \id{\{}u / z \id{\}} }
		& = &
		w[ \lpquote y!(z) \rpquote ] \nonumber
\end{eqnarray}

Because the body of the process between quotes is impervious to
substitution, we get radically different answers. In fact, by
examining the first process in an input context,
e.g. $x?(z).\lift{w}{y!(z)}$, we see that the process under the lift
operator may be shaped by prefixed inputs binding a name inside it. In
this sense, the lift operator will be seen as a way to dynamically
construct processes before reifying them as names.

Finally equipped with these standard features we can present the
dynamics of the calculus.

\subsubsection{Operational semantics} 

Finally, we introduce the computational dynamics. What marks these
algebras as distinct from other more traditionally studied algebraic
structures, e.g. vector spaces or polynomial rings, is the manner in
which dynamics is captured. In traditional structures, dynamics is typically
expressed through morphisms between such structures, as in linear maps
between vector spaces or morphisms between rings. In algebras
associated with the semantics of computation, the dynamics is
expressed as part of the algebraic structure itself, through a
reduction reduction relation typically denoted by $\red$. Below, we
give a recursive presentation of this relation for the calculus used
in the encoding.

$\red \subseteq \pi \times \pi$
$\red : \pi \to \mathcal{P}(\pi)$

\begin{mathpar}
  \inferrule* [lab=Comm] { \textsf{match}( x_{src}, x_{trgt} ) } { x_{trgt}?(y)P \; | \; x_{src}!\langle {Q} \rangle \red P\{\quotep{Q}/y}\} }
  \and \\
  \inferrule* [lab=Par] {{P} \red {P}'} {{{P} | {Q}} \red {{P}' | {Q}}}
  \and
  \inferrule* [lab=Equiv]{{{P} \scong {P}'} \andalso {{P}' \red {Q}'} \andalso {{Q}' \scong {Q}}}{{P} \red {Q}}
\end{mathpar}

\begin{eqnarray*}
  match_{\equiv} (\quotep{P},\quotep{Q}) & := & P \equiv Q \\
  match_{\dagger}(\quotep{P},\quotep{Q}) & := & \forall R. P|Q \red^{*} R => R \red^{*} 0 \\
  match_{K}(\quotep{P},\quotep{Q}) & := & K \mbox{ for some context } K
\end{eqnarray*}

$u?(x)P | u!\langle Q \rangle \red P\{\quotep{Q}/x\}$

%We write $\wred$ for $\red^*$, and $P\red$ if $\exists Q $ such that $ P \red Q$.
We write $P\red$ if $\exists Q $ such that $ P \red Q$ and $P\not\red$, otherwise.

\section{Replication}

As mentioned before, it is known that replication (and hence
recursion) can be implemented in a higher-order process algebra
\cite{SangiorgiWalker}. As our first example of calculation with the
machinery thus far presented we give the construction explicitly in
the {\rhoc}.

\begin{eqnarray}
	D_{x} & := & \prefix{x}{y}{(\binpar{\outputp{x}{y}}{@{y}})} \nonumber\\
	\bangp_{x}{P} & := & \binpar{{x}!\langle{\binpar{D_{x}}{P}}\rangle}{D_{x}} \nonumber
\end{eqnarray}

\begin{eqnarray}
	\bangp_{x}{P} & & \nonumber\\
	=
	& {x}!\langle{(\prefix{x}{y}{(\outputp{x}{y} | @{y})) | P}}\rangle 
	      | \prefix{x}{y}{(\outputp{x}{y} | @{y})} & \nonumber\\
	\red
	& (\outputp{x}{y} | @{y})\substn{\quotep{(\prefix{x}{y}{(@{y} | \outputp{x}{y})) | P}}}{y} & \nonumber\\
	=
	& \outputp{x}{\quotep{(\prefix{x}{y}{(\outputp{x}{y} | @{y})) | P}}}
	  | {(\prefix{x}{y}{(\outputp{x}{y} | @{y})) | P}} & \nonumber\\
	\red
	& \ldots & \nonumber\\
	\red^*
	& P | P | \ldots & \nonumber
\end{eqnarray}

Of course, this encoding, as an implementation, runs away, unfolding
$\bangp{P}$ eagerly. A lazier and more implementable replication
operator, restricted to input-guarded processes, may be obtained as follows.

\begin{eqnarray}
\bangp{\prefix{u}{v}{P}} 
	:= 
	\binpar{\lift{x}{\prefix{u}{v}{(\binpar{D(x)}{P})}}}{D(x)} \nonumber
\end{eqnarray}

\begin{remark}
  Note that the lazier definition still does not deal with summation
  or mixed summation (i.e. sums over input and output). The reader is
  invited to construct definitions of replication that deal with these
  features. 

  Further, the definitions are parameterized in a name, $x$. Can you,
  gentle reader, make a definition that eliminates this parameter and
  guarantees no accidental interaction between the replication
  machinery and the process being replicated -- i.e. no accidental
  sharing of names used by the process to get its work done and the
  name(s) used by the replication to effect copying. This latter
  revision of the definition of replication is crucial to obtaining
  the expected identity $!!P \sim !P$.
\end{remark}

\begin{remark}\label{rem:paradoxical_combinator}
  The reader familiar with the lambda calculus will have noticed the
  similarity between $D$ and the paradoxical combinator.

  [Ed. note: the existence of this seems to suggest we have to be more
  restrictive on the set of processes and names we admit if we are to
  support no-cloning.]
\end{remark}

\subsubsection{Bisimulation}

The computational dynamics gives rise to another kind of equivalence,
the equivalence of computational behavior. As previously mentioned
this is typically captured \emph{via} some form of bisimulation.

% The notion we use in this paper is weak barbed bisimulation
% \cite{milner91polyadicpi}.

The notion we use in this paper is derived from weak barbed
bisimulation \cite{milner91polyadicpi}. 

\begin{definition}
An \emph{observation relation}, $\downarrow_{\mathcal N}$, over a set
of names, $\mathcal N$, is the smallest relation satisfying the rules
below.

\infrule[Out-barb]{y \in {\mathcal N}, \; x \nameeq y}
		  {\outputp{x}{v} \downarrow_{\mathcal N} x}
\infrule[Par-barb]{\mbox{$P\downarrow_{\mathcal N} x$ or $Q\downarrow_{\mathcal N} x$}}
		  {\binpar{P}{Q} \downarrow_{\mathcal N} x}

We write $P \Downarrow_{\mathcal N} x$ if there is $Q$ such that 
$P \wred Q$ and $Q \downarrow_{\mathcal N} x$.
\end{definition}

\begin{definition}
%\label{def.bbisim}
An  ${\mathcal N}$-\emph{barbed bisimulation} over a set of names, ${\mathcal N}$, is a symmetric binary relation 
${\mathcal S}_{\mathcal N}$ between agents such that $P\rel{S}_{\mathcal N}Q$ implies:
\begin{enumerate}
\item If $P \red P'$ then $Q \wred Q'$ and $P'\rel{S}_{\mathcal N} Q'$.
\item If $P\downarrow_{\mathcal N} x$, then $Q\Downarrow_{\mathcal N} x$.
\end{enumerate}
$P$ is ${\mathcal N}$-barbed bisimilar to $Q$, written
$P \wbbisim_{\mathcal N} Q$, if $P \rel{S}_{\mathcal N} Q$ for some ${\mathcal N}$-barbed bisimulation ${\mathcal S}_{\mathcal N}$.
\end{definition}

$\mathcal{R} \subseteq \pi \times \pi$

$P \mathcal{R} Q => \forall P'. P \red P' \Rightarrow \exists Q'. Q \red Q', P' \mathcal{R} Q'$

$P \vdash x \Rightarrow Q \vdash x$

\begin{mathpar}
  \inferrule*[lab=Out-barb]{x \nameeq y}{{y}!\langle{Q}\rangle \vdash x}
  \and
  \inferrule*[lab=Par-barb]{\mbox{$P\vdash x$ or $Q\vdash x$}}{\binpar{P}{Q} \vdash x}
\end{mathpar}

\subsubsection{Contexts}

One of the principle advantages of computational calculi like the
$\pi$-calculus is a well-defined notion of context,
contextual-equivalence and a correlation between
contextual-equivalence and notions of bisimulation. The notion of
context allows the decomposition of a process into (sub-)process and
its syntactic environment, its context. Thus, a context may be
thought of as a process with a ``hole'' (written $\Box$) in it. The
application of a context $M$ to a process $P$, written $M[P]$, is
tantamount to filling the hole in $M$ with $P$. In this paper we do
not need the full weight of this theory, but do make use of the notion
of context in the proof the main theorem. 

\begin{mathpar}
  \inferrule* [lab=summation] {} {{M_{M},M_{N}} \bc \Box \;|\; x.M_{A} \;|\; M_{M}+M_{N}}
  \and
  \inferrule* [lab=agent] {} {{M_{A}} \bc (\vec{x})M_{P} \;| \; \clift{P_0,\ldots,M_{P},\ldots,P_N}}
  \and \\
  \inferrule* [lab=process] {} {{M_{P}} \bc M_{N} \;| \;P|M_{P} }
\end{mathpar} 

\begin{mathpar}
  \inferrule* [lab=sychronization] {} {M_{N} \bc \Box \;|\; x?M_{F} \;|\; x!M_{C}}
  \and
  \inferrule* [lab=abstraction] {} {{M_{F}} \bc (x)M_{P} }
  \and
  \inferrule* [lab=concretion] {} {{M_{C}} \bc \langle M_{P} \rangle }
  \and \\
  \inferrule* [lab=process] {} {{M_{P}} \bc M_{N} \;| \;P|M_{P} }
\end{mathpar}

\begin{definition}[contextual application] Given a context $M$, and
  process $P$, we define the \emph{contextual application}, $M[P] :=
  M\{P/\Box\}$. That is, the contextual application of M to P is the
  substitution of $P$ for $\Box$ in $M$.
\end{definition}

$\meaningof{-} : L \to \mathcal{P}(\pi)$

\begin{mathpar}
  \inferrule* [lab=collection] {} {\meaningof{true} = \pi, \and \meaningof{~E} = \pi \setminus \meaningof{E}, \and \meaningof{E_{1} \& E_{2}} = \meaningof{E_{1}} \cap \meaningof{E_{2}}}
\end{mathpar}

\begin{mathpar}
  \inferrule* [lab=structure] {} {\meaningof{0} = \{ P \in \pi | P \equiv 0 \}, \and \\ \meaningof{E_1 | E_2} = \{ P \in \pi | P \equiv P_{1} | P_{2}, P_{1} \in \meaningof{E_{1}}, P_{2} \in \meaningof{E_2}\} }
\end{mathpar}

\begin{mathpar}
 \inferrule* [lab=behavior] {} {\meaningof{\langle a?b \rangle E} = \{ P \in \pi | P \equiv Q | u?(y)P', \\ \and \\\\ \and \\ \;\;\; u \in \meaningof{a}, \forall z.P'\{z/y\} \in \meaningof{E\{z/b\}}\}, \and \\ \meaningof{a!E} = \{ P \in \pi | P \equiv Q | x!\langle P' \rangle, x \in \meaningof{a} P' \in \meaningof{E}\} }
\end{mathpar}

\begin{mathpar}
 \inferrule* [lab=nominal] {} {\meaningof{\quotep{E}} = \{ \quotep{P} \in \quotep{\pi} | P \in \meaningof{E} \}, \and \meaningof{\quotep{P}} = \{ \quotep{Q} \in \quotep{\pi} | P \equiv Q \} \and \\ \meaningof{@\quotep{E}} = \{ P \in \pi | P \equiv @x, x \in \meaningof{E} \}}
\end{mathpar}

\begin{eqnarray*}
  \\
  \meaningof{-} : TS \to ST
\end{eqnarray*}

\begin{eqnarray*}
  \\
  L : TS \to ST
\end{eqnarray*}

\begin{eqnarray*}
  \\
  P \models E \iff P \in \meaningof{E}
\end{eqnarray*}

\begin{eqnarray*}
  P \approx_{L} Q \iff \forall E \in L. P \models E \iff Q \models E
\end{eqnarray*}

\begin{eqnarray*}
  P \approx_{K} Q
\end{eqnarray*}

\begin{eqnarray*}
  P \approx Q
\end{eqnarray*}

$\approx_{K} = \approx = \approx_{L}$

\subsubsection{Contextual duality}

Note that contexts extend the quotation operation to a family of
operations from processes to names. Given a context, $M$, we can
define a \emph{nominal context}, $\quotep{M}$ by $\quotep{M}[P] :=
\quotep{M[P]}$. To foreshadow what is to come we observe that these
operations enjoy a duality with processes very much like the duality
between vectors and maps from vectors to scalars.

Further, because the calculus is essentially higher-order, we have a
correspondence between contexts and processes. More specifically,
given a name $x$ and a context $M$ we can construct $M^{*}_{x}$ such
that 

\begin{mathpar}
  M^{*}_{x} | \lift{x}{P} \red M[P]
\end{mathpar}

namely,

\begin{mathpar}
  M^{*}_{x} := x?(u).M[\dropn{u}]
\end{mathpar}

The dependence of $M^{*}_{x}$ on a name makes it an abstraction, 

\begin{mathpar}
  M^{*} := (x)x?(u).M[\dropn{u}]
\end{mathpar}

\subsection{Additional notation}

It will sometimes be convenient to denote the process a name
quotes. We already have the notation $x = \quotep{P}$, but it will be
convenient to introduce an alternate notation, $\procn{x}$, when we
want to emphasize the connection to the use of the name. Note that, by
virtue of name equivalence, $\quotep{\procn{x}} \nameeq x$; so, the
notation is consistent with previous definitions.

Further, because names have structure it is possible to effect
substitutions on the basis of that structure. This means we need to
upgrade our notation for substitutions, which we accomplish by
adapting comprehension notation. Thus,

\begin{mathpar}
  P\{ y / x : x \in S \}
\end{mathpar}

is interpreted to mean the process derived from P by replacing (in a
capture-avoiding manner) each occurrence of $x$ in $S$ by $y$. For example,

\begin{mathpar}
  P\{ \quotep{\procn{x}|\procn{x}} / x : x \in \freenames{P} \}
\end{mathpar}

will replace each (occurrence) of a free name $x$ in $P$ by
$\quotep{\procn{x}|\procn{x}}$.

Also, we will avail ourselves of the notation $x^{L}$ and $x^{R}$ to
denote injections of a name into disjoint copies of the name
space. There are numerous ways to accomplish this. One example can be
found in \cite{MeredithR05}. This notation overloads to vectors of
names: $\vec{x}^{\pi} := (x_{i}^{\pi} \; : \; 0 \leq i < |\vec{x}| )$ where $\pi \in \{L,R\}$.

We also use $P^{\Box} := P|\Box$.

In \cite{MeredithR05} an interpretation of the new operator is
given. It turns out that there are several possible interpretations
all enjoying the requisite algebraic properties of the operator (see
\cite{milner91polyadicpi}). We will therefore make liberal use of
$(\nu\; \vec{x})P$.

% subsection the_syntax_and_semantics_of_the_notation_system (end)   

\input{qm2pi.qmops} 

\input{qm2pi.sterngerlach} 

\input{qm2pi.metric} 

% section concurrent_process_calculi (end)

%\input{qm2pi.proofsketch}

% section proof sketch (end)

%\input{qm2pi.slviaknots} 

% section spatial logic via knots (end)

\input{qm2pi.conclusion}

% section conclusion (end)

%\input{qm2pi.dtcodes} 

% section wiring algorithm (end)

\input{qm2pi.ack} 

% section acknowledgments (end)

\newpage


\bibliographystyle{plain}   
\bibliography{../../biblios/main.bib}

\input{qm2pi.rhodetails}

\end{document}



% section front matter (end)

\section{Introduction}\label{sec:introduction} % (fold)
In this draft of the material i am going to have to dispense with the
usual writing conventions adopted in papers on these topics. i'm going
to have adopt whatever tone i need at the time i'm writing up the
calculations. Sometimes this may be very conversational; others it may
be the barest mathematical grunts; others still it may be that i have
lifted text from one of my other papers because the exposition of some
point was better said there. i hope that my readers are not unduly put
out by this decision. i'm not doing this to flout convention or be
rebellious. i find these calculations very technically challenging. To
keep everything going technically, something has to give; i have to
let go of some cognitive burden. So, the academic writing style --
with all of its trade-offs in terms of facilitating technical
communication -- is what i'm letting go of. Perhaps subsequent drafts
can be tightened and polished, but for now, i'm going to speak as if
we were sitting together in a coffee shop with a laptop, wifi and a
pad of paper and a pencil.

So, here's what i have to say. We -- you and i, comfortably ensconced
in our coffee shop and well-equipped with our tools -- can realize and
carry out the calculations of quantum mechanics over a very different
formal theory of dynamics, a formal theory of dynamics that
corresponds to a theory of concurrent computation with
\emph{reflection}. It has the advantage that the underlying theory is
already `quantized', but supports analogues all of the continuuous
operations. Strikingly, this underlying theory has recently been
connected with a notion of metric that we can show, by calculating
together, coincides with the metric induced by the inner product.

There are a lot of reasons why you might be interested in seeing
calculations of this form. Here's why i'm interested. For the past
several centuries there has been no competitor to the ``Newtonian''
account of dynamics. As a result the predominant share of accounts of
dynamical systems and situations have had to be formulated in terms of
the Newtonian machinery. i view this as an intellectually dangerous
position to occupy. Everything, despite it's intrinsic shape, turns
into a nail to be hit with this hammer. Recently, however, the theory
of computation has matured to the point where we have candidates for
theories of dynamics that offer very different perspective on
reasoning about dynamical systems and situations. Testing these
candidates against very successful accounts of dynamical situations,
like quantum mechanics, is going to give us some sense of how mature
they are and some measure of the quality of these accounts of
dynamics.

\subsection{Summary of contributions and outline of paper}

So, we're going to develop an interpretation of the operations of
quantum mechanics normally interpreted by Hilbert spaces and
operators. We're going to do this over a theory of computation. Note
that this is very different than the usual quantum computation program
which develops notions of computation over quantum mechanics. Rather,
we are developing a story that aligns with Wheeler's slogan: It from
Bit. To do this we will first provide an account of the theory of
computation at play here. Then we will dive into a calculation-driven
interpretation of the operations of quantum mechanics.

The reason we take this approach is that -- until very recently --
there hasn't been an axiomatic account of quantum mechanics. As a
result there has been no sharp delineation of the mathematical theory
supporting interpretation of the physical theory and the physical
theory, itself. So, ambient features of the maths are free to be
exploited (or supressed) without a real accounting of their physical
relevance. There is no sharp statement ``here's the physical theory''
qua \emph{theory} and ``here's the mathematical interpretation''
enabling a judgment of how faithful the interpretation is -- apart
from experimental observation. When there is an axiomatic account we
can judge how well a given mathematical formalism supports an
interpretation of the axioms, independent of
experimentation. Likewise, we can judge how well we have captured our
physical evidence and experience with our axiomatics, independent of
any specific mathematical implementation, with accidental detail that
may or may not have physical significance. 

In lieu of a fully fleshed out and vetted axiomatic account of quantum
mechanics, interpreting the operational notions in service of modeling
physical systems will have to suffice. In other words, we are not in
the business of providing a model of Hilbert spaces and operators. We
are in the business of providing a model of quantum mechanics because
we are motivated by testing our notions of dynamics against physical
theory; and, the predictive calculations of the physical theory must
serve as the best formulation -- shy of a fully fleshed out axiomatic
account -- of the physical theory itself (as they have for scientific
theories since time immemorial). Put another way, despite a
whole-hearted commitment to an It-from-Bit ontology, we are firmly
aligned with the shut-up-and-calculate camp as the best way to obtain
results either from the physical perspective or as a quality assurance
measure of our fledgling theory of dynamics.

In detail, we present a reflective process calculus. Then we develop
intuitive correspondences between the notions available in this
calculus and the usual physical notions supporting quantum mechanical
calculations. Thus, 

\begin{table}[htp]
  \center{
    \fbox{
      \begin{tabular}{c|c}
        quantum mechanics & process calculus \\
        \hline
        scalar & name \\
        state vector & process \\
        dual & contextual duals \\
        matrix & formal sums of process-context-dual pairs \\
        orthogonality & process annihilation \\
        inner product & execution-formula + quoting
      \end{tabular}
    }
  }
  \caption{QM - process calculi correspondences}
\end{table}

Then we tighten up these intuitions to operational definitions. We
employ the Dirac notation as the best proxy we can find for an
abstract syntax of the quantum mechanical notions. The definitions we
develop put us in contact with equational constraints coming from the
theory that we demonstrate the definitions and calculations satisfy.

This puts us in a position to shut up and calculate for the
Stern-Gerlach experimental set up, showing how these predictive
calculations become calculations on processes in our theory of a
reflective process calculus.

Penultimately, we demonstrate that the notion of metric coming from
the inner product coincides with the notion of metric available from
the theory of bisimulation. This demonstration gives us the right to
think of space as arising from behavior. Finally, we consider where we
might go from the new vantage point we have obtained.

% section introduction (end) 
 
% section introduction (end)

% \documentclass[12pt]{llncs}
%\documentclass{jktr}

\usepackage[pdftex]{hyperref}                   
\usepackage {listings}
\usepackage {mathpartir}
\usepackage{bcprules}
%\usepackage{listings}
                       
\usepackage{graphicx} 
%\usepackage[margins=2.5cm,nohead,nofoot]{geometry}
%\usepackage{geometry}
\usepackage{amsfonts}
\usepackage{amstext}
\usepackage{latexsym}
\usepackage{amssymb}
\usepackage{color}


%\include{myPreamble}
\include{qm2pi.local} 

%\ifpdf
%\usepackage[pdftex]{graphicx}
%\else
%\usepackage{graphicx}
%\fi

 % \ifpdf
%  \usepackage{pdfsync}
%  \if


%\title{Brief Article}
%\author{David F. Snyder}
%\author{L.G. Meredith}

%\address{Dept. of Math., Texas State University--San Marcos, San Marcos, TX 78666}
       
\pagestyle{empty}


\begin{document}

\lstset{language=[Objective]Caml,frame=shadowbox}

\input{qm2pi.front}

% section front matter (end)

\input{qm2pi.intro} 
 
% section introduction (end)

% \input{qm2pi.knotations} 

% section notation (end)

\input{qm2pi.process.calculi} 

% section concurrent_process_calculi_and_spatial_logics_ (end)
    
%\input{qm2pi.knots2pi} 

%\input{qm2pi.trefoil} 

%\input{qm2pi.mainthm} 

% subsection basic_interpretation (end)

%\input{qm2pi.rho.presentation} 
\subsection{The syntax and semantics of the notation system}\label{sub:the_syntax_and_semantics_of_the_notation_system} % (fold)

We now summarize a technical presentation of the calculus that
embodies our theory of dynamics. The typical presentation of such a
calculus follows the style of giving generators and relations on
them. The grammar, below, describing term constructors, freely
generates the set of processes, $\Proc$. This set is then quotiented
by a relation known as structural congruence and it is over this set
that the notion of dynamics is expressed. This presentation is
essentially that of \cite{MeredithR05} with the addition of
polyadicity and summation. For readability we have relegated some of
the technical subtleties to an appendix.

\subsubsection{Process grammar}\label{subsub:process_grammar}

\begin{mathpar}
  \inferrule* [lab=synchronization] {} {{M} \bc \pzero \;|\; x?F \;|\; x!C }
  \and
  \inferrule* [lab=abstraction] {} {{F} \bc (x)P}
  \and
  \inferrule* [lab=concretion] {} {{C} \bc \langle Q \rangle}
  \and
  \inferrule* [lab=process] {} {{P,Q} \bc M \;| \;P|Q \;|\; @{x}}
  \and
  \inferrule* [lab=name] {} {{x} \bc \quotep{P}}
\end{mathpar} 

Note that $\vec{x}$ (resp. $\vec{P}$) denotes a vector of names
(resp. processes) of length $|\vec{x}|$ (resp. $|\vec{P}|$). We adopt
the following useful abbreviations.

\begin{mathpar}
   x?(\vec{y}).P := x.(\vec{y})P \and  x\clift{\vec{P}} := x.\clift{\vec{P}}
   \and x!(y) := \lift{x}{\dropn{y}}
   \and \Pi_{i=0}^{n-1}P_i := P_0 | \ldots | P_{n-1}
\end{mathpar}

\subsubsection{Structural congruence}

\paragraph{Free and bound names and alpha-equivalence.} At the
core of structural equivalence is alpha-equivalence which identifies
process that are the same up to a change of variable. Formally, we
recognize the distinction between free and bound names. The free names
of a process, $\freenames{P}$, may be calculated recursively as
follows:

\begin{mathpar}
\freenames{\pzero} := \emptyset
  \and \\
  \freenames{x?(y).P} := \{ x \} \cup (\freenames{P} \setminus \{ y \})
  \and 
  \freenames{x!\langle P \rangle} := \{ x \} \cup \{ P \} 
  \and \\
  \freenames{P|Q} := \freenames{P} \cup \freenames{Q}
  \and \\
  \freenames{@{x}} := \{ x \}
\end{mathpar}

$\pi$
$\quotep{\pi}$

$\freenames{-} : \pi \to \mathcal{P}(\quotep{\pi})$

\begin{eqnarray*}
  \freenames{\pzero} & := & \emptyset \\
  \freenames{x?(y).P} & := & \{ x \} \cup (\freenames{P} \setminus \{ y \}) \\
  \freenames{x!\langle P \rangle} & := & \{ x \} \cup \{ P \} \\
  \freenames{P|Q} & := & \freenames{P} \cup \freenames{Q} \\
  \freenames{\dropn{x}} & := & \{ x \}
\end{eqnarray*}

The bound names of a process, $\boundnames{P}$, are those names occurring in $P$
that are not free. For example, in $x?(y).0$, the name $x$ is free, while $y$ is bound.

\begin{mathpar}
  \inferrule* [lab=monoidal-laws] {} { P|Q \equiv Q|P \and P|0 \equiv P \and P|(Q|R) \equiv (P|Q)|R }
\end{mathpar}

\begin{mathpar}
  \inferrule* [lab=alpha-equivalence] {} { (x)P \equiv (y)P\{y/x\} \and y \not\in \freenames{P} }
\end{mathpar}

\begin{definition}
Then two processes, $P,Q$, are alpha-equivalent if $P = Q\{\vec{y}/\vec{x}\}$ for
some $\vec{x} \in \boundnames{Q},\vec{y} \in \boundnames{P}$, where $Q\{\vec{y}/\vec{x}\}$
denotes the capture-avoiding substitution of $\vec{y}$ for $\vec{x}$ in $Q$.
\end{definition}

\begin{definition}
  The {\em structural congruence} \cite{SangiorgiWalker} , $\equiv$,
  between processes is the least congruence containing
  alpha-equivalence, satisfying the abelian monoid laws
  (associativity, commutativity and $\pzero$ as identity) for parallel
  composition $|$ and for summation $+$.
\end{definition}

\subsection{Name equivalence}

We take name equivalence, written $\nameeq$, to be the smallest
equivalence relation generated by the following rules.

\begin{mathpar}
\inferrule*[lab=Quote-drop]
{ }
{ \quotep{@{x}} \nameeq x }

\inferrule*[lab=Struct-equiv]
{ P \scong Q }
{ \quotep{P} \nameeq \quotep{Q} }
\end{mathpar}

The astute reader will have noticed that the mutual recursion of names
and processes imposes a mutual recursion on alpha-equivalence and
structural equivalence via name-equivalence. Fortunately, all of this
works out pleasantly and we may calculate in the natural way, free of
concern. The reader interested in the details is referred to the
appendix \ref{appendix:rho_details}.

\subsection{Substitution}

We use $\Proc$ for the set of processes, $\QProc$ for the set of
names, and $\id{\{}\vec{y} / \vec{x} \id{\}}$ to denote partial maps,
$s : \QProc \rightarrow \QProc$. A map, $s$ lifts, uniquely, to a map
on process terms, $\widehat{s} : \Proc \rightarrow \Proc$ by the
following equations.

\begin{mathpar}
  (0) \psubstp{Q}{P} := 0 \\
  (R \juxtap S) \psubstp{Q}{P}
  :=    
  (R)\psubstp{Q}{P} \juxtap (S) \psubstp{Q}{P} \\
  (x?(y).R) \psubstp{Q}{P}    
  :=    
  (x)\substp{Q}{P} (z)\concat( (R \psubstn{z}{y}) \psubstp{Q}{P} ) \\
  (\lift{x}{R}) \psubstp{Q}{P}  
  :=
  \lift{(x)\substp{Q}{P}}{ R \psubstp{Q}{P} } \\
%   (\dropn{x})  \psubstp{Q}{P}       
%   := 
%   \left\{ 
%     \begin{array}{ccc} 
%       \dropn{\quotep{Q}} & & x \nameeq \quotep{P} \\
%       \dropn{x} & & otherwise \\
%     \end{array}
%   \right. 
  (\dropn{x})  \psubstp{Q}{P}       
  := 
  \left\{ 
    \begin{array}{ccc} 
      Q & & x \nameeq \quotep{P} \\
      \dropn{x} & & otherwise \\
    \end{array}
  \right.
\end{mathpar}
 

where

\begin{eqnarray}
  (x)\id{\{} \lpquote Q \rpquote / \lpquote P \rpquote \id{\}}            = 
  \left\{ 
    \begin{array}{ccc}
      \lpquote Q \rpquote & & x \nameeq \lpquote P \rpquote \\
      x & & otherwise \\
    \end{array}
  \right. \nonumber
\end{eqnarray}

and $z$ is chosen distinct from $\quotep{P}$, $\quotep{Q}$, the free
names in $Q$, and all the names in $R$. Our $\alpha$-equivalence will
be built in the standard way from this substitution.

\begin{remark}\label{rem:no_self_referential_names}
  One consequence of these definitions is that $\forall P. \quotep{P}
  \not\in \freenames{P}$.
\end{remark}

\subsection{ Dynamic quote: an example }

Anticipating something of what's to come, consider applying the
substitution, $\widehat{\id{\{}u / z \id{\}}}$, to the following pair
of processes, $\lift{w}{y!(z)}$ and $w[ \lpquote y!(z) \rpquote ]$.

\begin{eqnarray}
	\lift{w}{y!(z)}\widehat{\id{\{}u / z \id{\}}}
		& = &
		\lift{w}{y!(u)} \nonumber\\
	w[ \lpquote y!(z) \rpquote ] \widehat{ \id{\{}u / z \id{\}} }
		& = &
		w[ \lpquote y!(z) \rpquote ] \nonumber
\end{eqnarray}

Because the body of the process between quotes is impervious to
substitution, we get radically different answers. In fact, by
examining the first process in an input context,
e.g. $x?(z).\lift{w}{y!(z)}$, we see that the process under the lift
operator may be shaped by prefixed inputs binding a name inside it. In
this sense, the lift operator will be seen as a way to dynamically
construct processes before reifying them as names.

Finally equipped with these standard features we can present the
dynamics of the calculus.

\subsubsection{Operational semantics} 

Finally, we introduce the computational dynamics. What marks these
algebras as distinct from other more traditionally studied algebraic
structures, e.g. vector spaces or polynomial rings, is the manner in
which dynamics is captured. In traditional structures, dynamics is typically
expressed through morphisms between such structures, as in linear maps
between vector spaces or morphisms between rings. In algebras
associated with the semantics of computation, the dynamics is
expressed as part of the algebraic structure itself, through a
reduction reduction relation typically denoted by $\red$. Below, we
give a recursive presentation of this relation for the calculus used
in the encoding.

$\red \subseteq \pi \times \pi$
$\red : \pi \to \mathcal{P}(\pi)$

\begin{mathpar}
  \inferrule* [lab=Comm] { \textsf{match}( x_{src}, x_{trgt} ) } { x_{trgt}?(y)P \; | \; x_{src}!\langle {Q} \rangle \red P\{\quotep{Q}/y}\} }
  \and \\
  \inferrule* [lab=Par] {{P} \red {P}'} {{{P} | {Q}} \red {{P}' | {Q}}}
  \and
  \inferrule* [lab=Equiv]{{{P} \scong {P}'} \andalso {{P}' \red {Q}'} \andalso {{Q}' \scong {Q}}}{{P} \red {Q}}
\end{mathpar}

\begin{eqnarray*}
  match_{\equiv} (\quotep{P},\quotep{Q}) & := & P \equiv Q \\
  match_{\dagger}(\quotep{P},\quotep{Q}) & := & \forall R. P|Q \red^{*} R => R \red^{*} 0 \\
  match_{K}(\quotep{P},\quotep{Q}) & := & K \mbox{ for some context } K
\end{eqnarray*}

$u?(x)P | u!\langle Q \rangle \red P\{\quotep{Q}/x\}$

%We write $\wred$ for $\red^*$, and $P\red$ if $\exists Q $ such that $ P \red Q$.
We write $P\red$ if $\exists Q $ such that $ P \red Q$ and $P\not\red$, otherwise.

\section{Replication}

As mentioned before, it is known that replication (and hence
recursion) can be implemented in a higher-order process algebra
\cite{SangiorgiWalker}. As our first example of calculation with the
machinery thus far presented we give the construction explicitly in
the {\rhoc}.

\begin{eqnarray}
	D_{x} & := & \prefix{x}{y}{(\binpar{\outputp{x}{y}}{@{y}})} \nonumber\\
	\bangp_{x}{P} & := & \binpar{{x}!\langle{\binpar{D_{x}}{P}}\rangle}{D_{x}} \nonumber
\end{eqnarray}

\begin{eqnarray}
	\bangp_{x}{P} & & \nonumber\\
	=
	& {x}!\langle{(\prefix{x}{y}{(\outputp{x}{y} | @{y})) | P}}\rangle 
	      | \prefix{x}{y}{(\outputp{x}{y} | @{y})} & \nonumber\\
	\red
	& (\outputp{x}{y} | @{y})\substn{\quotep{(\prefix{x}{y}{(@{y} | \outputp{x}{y})) | P}}}{y} & \nonumber\\
	=
	& \outputp{x}{\quotep{(\prefix{x}{y}{(\outputp{x}{y} | @{y})) | P}}}
	  | {(\prefix{x}{y}{(\outputp{x}{y} | @{y})) | P}} & \nonumber\\
	\red
	& \ldots & \nonumber\\
	\red^*
	& P | P | \ldots & \nonumber
\end{eqnarray}

Of course, this encoding, as an implementation, runs away, unfolding
$\bangp{P}$ eagerly. A lazier and more implementable replication
operator, restricted to input-guarded processes, may be obtained as follows.

\begin{eqnarray}
\bangp{\prefix{u}{v}{P}} 
	:= 
	\binpar{\lift{x}{\prefix{u}{v}{(\binpar{D(x)}{P})}}}{D(x)} \nonumber
\end{eqnarray}

\begin{remark}
  Note that the lazier definition still does not deal with summation
  or mixed summation (i.e. sums over input and output). The reader is
  invited to construct definitions of replication that deal with these
  features. 

  Further, the definitions are parameterized in a name, $x$. Can you,
  gentle reader, make a definition that eliminates this parameter and
  guarantees no accidental interaction between the replication
  machinery and the process being replicated -- i.e. no accidental
  sharing of names used by the process to get its work done and the
  name(s) used by the replication to effect copying. This latter
  revision of the definition of replication is crucial to obtaining
  the expected identity $!!P \sim !P$.
\end{remark}

\begin{remark}\label{rem:paradoxical_combinator}
  The reader familiar with the lambda calculus will have noticed the
  similarity between $D$ and the paradoxical combinator.

  [Ed. note: the existence of this seems to suggest we have to be more
  restrictive on the set of processes and names we admit if we are to
  support no-cloning.]
\end{remark}

\subsubsection{Bisimulation}

The computational dynamics gives rise to another kind of equivalence,
the equivalence of computational behavior. As previously mentioned
this is typically captured \emph{via} some form of bisimulation.

% The notion we use in this paper is weak barbed bisimulation
% \cite{milner91polyadicpi}.

The notion we use in this paper is derived from weak barbed
bisimulation \cite{milner91polyadicpi}. 

\begin{definition}
An \emph{observation relation}, $\downarrow_{\mathcal N}$, over a set
of names, $\mathcal N$, is the smallest relation satisfying the rules
below.

\infrule[Out-barb]{y \in {\mathcal N}, \; x \nameeq y}
		  {\outputp{x}{v} \downarrow_{\mathcal N} x}
\infrule[Par-barb]{\mbox{$P\downarrow_{\mathcal N} x$ or $Q\downarrow_{\mathcal N} x$}}
		  {\binpar{P}{Q} \downarrow_{\mathcal N} x}

We write $P \Downarrow_{\mathcal N} x$ if there is $Q$ such that 
$P \wred Q$ and $Q \downarrow_{\mathcal N} x$.
\end{definition}

\begin{definition}
%\label{def.bbisim}
An  ${\mathcal N}$-\emph{barbed bisimulation} over a set of names, ${\mathcal N}$, is a symmetric binary relation 
${\mathcal S}_{\mathcal N}$ between agents such that $P\rel{S}_{\mathcal N}Q$ implies:
\begin{enumerate}
\item If $P \red P'$ then $Q \wred Q'$ and $P'\rel{S}_{\mathcal N} Q'$.
\item If $P\downarrow_{\mathcal N} x$, then $Q\Downarrow_{\mathcal N} x$.
\end{enumerate}
$P$ is ${\mathcal N}$-barbed bisimilar to $Q$, written
$P \wbbisim_{\mathcal N} Q$, if $P \rel{S}_{\mathcal N} Q$ for some ${\mathcal N}$-barbed bisimulation ${\mathcal S}_{\mathcal N}$.
\end{definition}

$\mathcal{R} \subseteq \pi \times \pi$

$P \mathcal{R} Q => \forall P'. P \red P' \Rightarrow \exists Q'. Q \red Q', P' \mathcal{R} Q'$

$P \vdash x \Rightarrow Q \vdash x$

\begin{mathpar}
  \inferrule*[lab=Out-barb]{x \nameeq y}{{y}!\langle{Q}\rangle \vdash x}
  \and
  \inferrule*[lab=Par-barb]{\mbox{$P\vdash x$ or $Q\vdash x$}}{\binpar{P}{Q} \vdash x}
\end{mathpar}

\subsubsection{Contexts}

One of the principle advantages of computational calculi like the
$\pi$-calculus is a well-defined notion of context,
contextual-equivalence and a correlation between
contextual-equivalence and notions of bisimulation. The notion of
context allows the decomposition of a process into (sub-)process and
its syntactic environment, its context. Thus, a context may be
thought of as a process with a ``hole'' (written $\Box$) in it. The
application of a context $M$ to a process $P$, written $M[P]$, is
tantamount to filling the hole in $M$ with $P$. In this paper we do
not need the full weight of this theory, but do make use of the notion
of context in the proof the main theorem. 

\begin{mathpar}
  \inferrule* [lab=summation] {} {{M_{M},M_{N}} \bc \Box \;|\; x.M_{A} \;|\; M_{M}+M_{N}}
  \and
  \inferrule* [lab=agent] {} {{M_{A}} \bc (\vec{x})M_{P} \;| \; \clift{P_0,\ldots,M_{P},\ldots,P_N}}
  \and \\
  \inferrule* [lab=process] {} {{M_{P}} \bc M_{N} \;| \;P|M_{P} }
\end{mathpar} 

\begin{mathpar}
  \inferrule* [lab=sychronization] {} {M_{N} \bc \Box \;|\; x?M_{F} \;|\; x!M_{C}}
  \and
  \inferrule* [lab=abstraction] {} {{M_{F}} \bc (x)M_{P} }
  \and
  \inferrule* [lab=concretion] {} {{M_{C}} \bc \langle M_{P} \rangle }
  \and \\
  \inferrule* [lab=process] {} {{M_{P}} \bc M_{N} \;| \;P|M_{P} }
\end{mathpar}

\begin{definition}[contextual application] Given a context $M$, and
  process $P$, we define the \emph{contextual application}, $M[P] :=
  M\{P/\Box\}$. That is, the contextual application of M to P is the
  substitution of $P$ for $\Box$ in $M$.
\end{definition}

$\meaningof{-} : L \to \mathcal{P}(\pi)$

\begin{mathpar}
  \inferrule* [lab=collection] {} {\meaningof{true} = \pi, \and \meaningof{~E} = \pi \setminus \meaningof{E}, \and \meaningof{E_{1} \& E_{2}} = \meaningof{E_{1}} \cap \meaningof{E_{2}}}
\end{mathpar}

\begin{mathpar}
  \inferrule* [lab=structure] {} {\meaningof{0} = \{ P \in \pi | P \equiv 0 \}, \and \\ \meaningof{E_1 | E_2} = \{ P \in \pi | P \equiv P_{1} | P_{2}, P_{1} \in \meaningof{E_{1}}, P_{2} \in \meaningof{E_2}\} }
\end{mathpar}

\begin{mathpar}
 \inferrule* [lab=behavior] {} {\meaningof{\langle a?b \rangle E} = \{ P \in \pi | P \equiv Q | u?(y)P', \\ \and \\\\ \and \\ \;\;\; u \in \meaningof{a}, \forall z.P'\{z/y\} \in \meaningof{E\{z/b\}}\}, \and \\ \meaningof{a!E} = \{ P \in \pi | P \equiv Q | x!\langle P' \rangle, x \in \meaningof{a} P' \in \meaningof{E}\} }
\end{mathpar}

\begin{mathpar}
 \inferrule* [lab=nominal] {} {\meaningof{\quotep{E}} = \{ \quotep{P} \in \quotep{\pi} | P \in \meaningof{E} \}, \and \meaningof{\quotep{P}} = \{ \quotep{Q} \in \quotep{\pi} | P \equiv Q \} \and \\ \meaningof{@\quotep{E}} = \{ P \in \pi | P \equiv @x, x \in \meaningof{E} \}}
\end{mathpar}

\begin{eqnarray*}
  \\
  \meaningof{-} : TS \to ST
\end{eqnarray*}

\begin{eqnarray*}
  \\
  L : TS \to ST
\end{eqnarray*}

\begin{eqnarray*}
  \\
  P \models E \iff P \in \meaningof{E}
\end{eqnarray*}

\begin{eqnarray*}
  P \approx_{L} Q \iff \forall E \in L. P \models E \iff Q \models E
\end{eqnarray*}

\begin{eqnarray*}
  P \approx_{K} Q
\end{eqnarray*}

\begin{eqnarray*}
  P \approx Q
\end{eqnarray*}

$\approx_{K} = \approx = \approx_{L}$

\subsubsection{Contextual duality}

Note that contexts extend the quotation operation to a family of
operations from processes to names. Given a context, $M$, we can
define a \emph{nominal context}, $\quotep{M}$ by $\quotep{M}[P] :=
\quotep{M[P]}$. To foreshadow what is to come we observe that these
operations enjoy a duality with processes very much like the duality
between vectors and maps from vectors to scalars.

Further, because the calculus is essentially higher-order, we have a
correspondence between contexts and processes. More specifically,
given a name $x$ and a context $M$ we can construct $M^{*}_{x}$ such
that 

\begin{mathpar}
  M^{*}_{x} | \lift{x}{P} \red M[P]
\end{mathpar}

namely,

\begin{mathpar}
  M^{*}_{x} := x?(u).M[\dropn{u}]
\end{mathpar}

The dependence of $M^{*}_{x}$ on a name makes it an abstraction, 

\begin{mathpar}
  M^{*} := (x)x?(u).M[\dropn{u}]
\end{mathpar}

\subsection{Additional notation}

It will sometimes be convenient to denote the process a name
quotes. We already have the notation $x = \quotep{P}$, but it will be
convenient to introduce an alternate notation, $\procn{x}$, when we
want to emphasize the connection to the use of the name. Note that, by
virtue of name equivalence, $\quotep{\procn{x}} \nameeq x$; so, the
notation is consistent with previous definitions.

Further, because names have structure it is possible to effect
substitutions on the basis of that structure. This means we need to
upgrade our notation for substitutions, which we accomplish by
adapting comprehension notation. Thus,

\begin{mathpar}
  P\{ y / x : x \in S \}
\end{mathpar}

is interpreted to mean the process derived from P by replacing (in a
capture-avoiding manner) each occurrence of $x$ in $S$ by $y$. For example,

\begin{mathpar}
  P\{ \quotep{\procn{x}|\procn{x}} / x : x \in \freenames{P} \}
\end{mathpar}

will replace each (occurrence) of a free name $x$ in $P$ by
$\quotep{\procn{x}|\procn{x}}$.

Also, we will avail ourselves of the notation $x^{L}$ and $x^{R}$ to
denote injections of a name into disjoint copies of the name
space. There are numerous ways to accomplish this. One example can be
found in \cite{MeredithR05}. This notation overloads to vectors of
names: $\vec{x}^{\pi} := (x_{i}^{\pi} \; : \; 0 \leq i < |\vec{x}| )$ where $\pi \in \{L,R\}$.

We also use $P^{\Box} := P|\Box$.

In \cite{MeredithR05} an interpretation of the new operator is
given. It turns out that there are several possible interpretations
all enjoying the requisite algebraic properties of the operator (see
\cite{milner91polyadicpi}). We will therefore make liberal use of
$(\nu\; \vec{x})P$.

% subsection the_syntax_and_semantics_of_the_notation_system (end)   

\input{qm2pi.qmops} 

\input{qm2pi.sterngerlach} 

\input{qm2pi.metric} 

% section concurrent_process_calculi (end)

%\input{qm2pi.proofsketch}

% section proof sketch (end)

%\input{qm2pi.slviaknots} 

% section spatial logic via knots (end)

\input{qm2pi.conclusion}

% section conclusion (end)

%\input{qm2pi.dtcodes} 

% section wiring algorithm (end)

\input{qm2pi.ack} 

% section acknowledgments (end)

\newpage


\bibliographystyle{plain}   
\bibliography{../../biblios/main.bib}

\input{qm2pi.rhodetails}

\end{document}

 

% section notation (end)

\input{qm2pi.process.calculi} 

% section concurrent_process_calculi_and_spatial_logics_ (end)
    
%\documentclass[12pt]{llncs}
%\documentclass{jktr}

\usepackage[pdftex]{hyperref}                   
\usepackage {listings}
\usepackage {mathpartir}
\usepackage{bcprules}
%\usepackage{listings}
                       
\usepackage{graphicx} 
%\usepackage[margins=2.5cm,nohead,nofoot]{geometry}
%\usepackage{geometry}
\usepackage{amsfonts}
\usepackage{amstext}
\usepackage{latexsym}
\usepackage{amssymb}
\usepackage{color}


%\include{myPreamble}
\include{qm2pi.local} 

%\ifpdf
%\usepackage[pdftex]{graphicx}
%\else
%\usepackage{graphicx}
%\fi

 % \ifpdf
%  \usepackage{pdfsync}
%  \if


%\title{Brief Article}
%\author{David F. Snyder}
%\author{L.G. Meredith}

%\address{Dept. of Math., Texas State University--San Marcos, San Marcos, TX 78666}
       
\pagestyle{empty}


\begin{document}

\lstset{language=[Objective]Caml,frame=shadowbox}

\input{qm2pi.front}

% section front matter (end)

\input{qm2pi.intro} 
 
% section introduction (end)

% \input{qm2pi.knotations} 

% section notation (end)

\input{qm2pi.process.calculi} 

% section concurrent_process_calculi_and_spatial_logics_ (end)
    
%\input{qm2pi.knots2pi} 

%\input{qm2pi.trefoil} 

%\input{qm2pi.mainthm} 

% subsection basic_interpretation (end)

%\input{qm2pi.rho.presentation} 
\subsection{The syntax and semantics of the notation system}\label{sub:the_syntax_and_semantics_of_the_notation_system} % (fold)

We now summarize a technical presentation of the calculus that
embodies our theory of dynamics. The typical presentation of such a
calculus follows the style of giving generators and relations on
them. The grammar, below, describing term constructors, freely
generates the set of processes, $\Proc$. This set is then quotiented
by a relation known as structural congruence and it is over this set
that the notion of dynamics is expressed. This presentation is
essentially that of \cite{MeredithR05} with the addition of
polyadicity and summation. For readability we have relegated some of
the technical subtleties to an appendix.

\subsubsection{Process grammar}\label{subsub:process_grammar}

\begin{mathpar}
  \inferrule* [lab=synchronization] {} {{M} \bc \pzero \;|\; x?F \;|\; x!C }
  \and
  \inferrule* [lab=abstraction] {} {{F} \bc (x)P}
  \and
  \inferrule* [lab=concretion] {} {{C} \bc \langle Q \rangle}
  \and
  \inferrule* [lab=process] {} {{P,Q} \bc M \;| \;P|Q \;|\; @{x}}
  \and
  \inferrule* [lab=name] {} {{x} \bc \quotep{P}}
\end{mathpar} 

Note that $\vec{x}$ (resp. $\vec{P}$) denotes a vector of names
(resp. processes) of length $|\vec{x}|$ (resp. $|\vec{P}|$). We adopt
the following useful abbreviations.

\begin{mathpar}
   x?(\vec{y}).P := x.(\vec{y})P \and  x\clift{\vec{P}} := x.\clift{\vec{P}}
   \and x!(y) := \lift{x}{\dropn{y}}
   \and \Pi_{i=0}^{n-1}P_i := P_0 | \ldots | P_{n-1}
\end{mathpar}

\subsubsection{Structural congruence}

\paragraph{Free and bound names and alpha-equivalence.} At the
core of structural equivalence is alpha-equivalence which identifies
process that are the same up to a change of variable. Formally, we
recognize the distinction between free and bound names. The free names
of a process, $\freenames{P}$, may be calculated recursively as
follows:

\begin{mathpar}
\freenames{\pzero} := \emptyset
  \and \\
  \freenames{x?(y).P} := \{ x \} \cup (\freenames{P} \setminus \{ y \})
  \and 
  \freenames{x!\langle P \rangle} := \{ x \} \cup \{ P \} 
  \and \\
  \freenames{P|Q} := \freenames{P} \cup \freenames{Q}
  \and \\
  \freenames{@{x}} := \{ x \}
\end{mathpar}

$\pi$
$\quotep{\pi}$

$\freenames{-} : \pi \to \mathcal{P}(\quotep{\pi})$

\begin{eqnarray*}
  \freenames{\pzero} & := & \emptyset \\
  \freenames{x?(y).P} & := & \{ x \} \cup (\freenames{P} \setminus \{ y \}) \\
  \freenames{x!\langle P \rangle} & := & \{ x \} \cup \{ P \} \\
  \freenames{P|Q} & := & \freenames{P} \cup \freenames{Q} \\
  \freenames{\dropn{x}} & := & \{ x \}
\end{eqnarray*}

The bound names of a process, $\boundnames{P}$, are those names occurring in $P$
that are not free. For example, in $x?(y).0$, the name $x$ is free, while $y$ is bound.

\begin{mathpar}
  \inferrule* [lab=monoidal-laws] {} { P|Q \equiv Q|P \and P|0 \equiv P \and P|(Q|R) \equiv (P|Q)|R }
\end{mathpar}

\begin{mathpar}
  \inferrule* [lab=alpha-equivalence] {} { (x)P \equiv (y)P\{y/x\} \and y \not\in \freenames{P} }
\end{mathpar}

\begin{definition}
Then two processes, $P,Q$, are alpha-equivalent if $P = Q\{\vec{y}/\vec{x}\}$ for
some $\vec{x} \in \boundnames{Q},\vec{y} \in \boundnames{P}$, where $Q\{\vec{y}/\vec{x}\}$
denotes the capture-avoiding substitution of $\vec{y}$ for $\vec{x}$ in $Q$.
\end{definition}

\begin{definition}
  The {\em structural congruence} \cite{SangiorgiWalker} , $\equiv$,
  between processes is the least congruence containing
  alpha-equivalence, satisfying the abelian monoid laws
  (associativity, commutativity and $\pzero$ as identity) for parallel
  composition $|$ and for summation $+$.
\end{definition}

\subsection{Name equivalence}

We take name equivalence, written $\nameeq$, to be the smallest
equivalence relation generated by the following rules.

\begin{mathpar}
\inferrule*[lab=Quote-drop]
{ }
{ \quotep{@{x}} \nameeq x }

\inferrule*[lab=Struct-equiv]
{ P \scong Q }
{ \quotep{P} \nameeq \quotep{Q} }
\end{mathpar}

The astute reader will have noticed that the mutual recursion of names
and processes imposes a mutual recursion on alpha-equivalence and
structural equivalence via name-equivalence. Fortunately, all of this
works out pleasantly and we may calculate in the natural way, free of
concern. The reader interested in the details is referred to the
appendix \ref{appendix:rho_details}.

\subsection{Substitution}

We use $\Proc$ for the set of processes, $\QProc$ for the set of
names, and $\id{\{}\vec{y} / \vec{x} \id{\}}$ to denote partial maps,
$s : \QProc \rightarrow \QProc$. A map, $s$ lifts, uniquely, to a map
on process terms, $\widehat{s} : \Proc \rightarrow \Proc$ by the
following equations.

\begin{mathpar}
  (0) \psubstp{Q}{P} := 0 \\
  (R \juxtap S) \psubstp{Q}{P}
  :=    
  (R)\psubstp{Q}{P} \juxtap (S) \psubstp{Q}{P} \\
  (x?(y).R) \psubstp{Q}{P}    
  :=    
  (x)\substp{Q}{P} (z)\concat( (R \psubstn{z}{y}) \psubstp{Q}{P} ) \\
  (\lift{x}{R}) \psubstp{Q}{P}  
  :=
  \lift{(x)\substp{Q}{P}}{ R \psubstp{Q}{P} } \\
%   (\dropn{x})  \psubstp{Q}{P}       
%   := 
%   \left\{ 
%     \begin{array}{ccc} 
%       \dropn{\quotep{Q}} & & x \nameeq \quotep{P} \\
%       \dropn{x} & & otherwise \\
%     \end{array}
%   \right. 
  (\dropn{x})  \psubstp{Q}{P}       
  := 
  \left\{ 
    \begin{array}{ccc} 
      Q & & x \nameeq \quotep{P} \\
      \dropn{x} & & otherwise \\
    \end{array}
  \right.
\end{mathpar}
 

where

\begin{eqnarray}
  (x)\id{\{} \lpquote Q \rpquote / \lpquote P \rpquote \id{\}}            = 
  \left\{ 
    \begin{array}{ccc}
      \lpquote Q \rpquote & & x \nameeq \lpquote P \rpquote \\
      x & & otherwise \\
    \end{array}
  \right. \nonumber
\end{eqnarray}

and $z$ is chosen distinct from $\quotep{P}$, $\quotep{Q}$, the free
names in $Q$, and all the names in $R$. Our $\alpha$-equivalence will
be built in the standard way from this substitution.

\begin{remark}\label{rem:no_self_referential_names}
  One consequence of these definitions is that $\forall P. \quotep{P}
  \not\in \freenames{P}$.
\end{remark}

\subsection{ Dynamic quote: an example }

Anticipating something of what's to come, consider applying the
substitution, $\widehat{\id{\{}u / z \id{\}}}$, to the following pair
of processes, $\lift{w}{y!(z)}$ and $w[ \lpquote y!(z) \rpquote ]$.

\begin{eqnarray}
	\lift{w}{y!(z)}\widehat{\id{\{}u / z \id{\}}}
		& = &
		\lift{w}{y!(u)} \nonumber\\
	w[ \lpquote y!(z) \rpquote ] \widehat{ \id{\{}u / z \id{\}} }
		& = &
		w[ \lpquote y!(z) \rpquote ] \nonumber
\end{eqnarray}

Because the body of the process between quotes is impervious to
substitution, we get radically different answers. In fact, by
examining the first process in an input context,
e.g. $x?(z).\lift{w}{y!(z)}$, we see that the process under the lift
operator may be shaped by prefixed inputs binding a name inside it. In
this sense, the lift operator will be seen as a way to dynamically
construct processes before reifying them as names.

Finally equipped with these standard features we can present the
dynamics of the calculus.

\subsubsection{Operational semantics} 

Finally, we introduce the computational dynamics. What marks these
algebras as distinct from other more traditionally studied algebraic
structures, e.g. vector spaces or polynomial rings, is the manner in
which dynamics is captured. In traditional structures, dynamics is typically
expressed through morphisms between such structures, as in linear maps
between vector spaces or morphisms between rings. In algebras
associated with the semantics of computation, the dynamics is
expressed as part of the algebraic structure itself, through a
reduction reduction relation typically denoted by $\red$. Below, we
give a recursive presentation of this relation for the calculus used
in the encoding.

$\red \subseteq \pi \times \pi$
$\red : \pi \to \mathcal{P}(\pi)$

\begin{mathpar}
  \inferrule* [lab=Comm] { \textsf{match}( x_{src}, x_{trgt} ) } { x_{trgt}?(y)P \; | \; x_{src}!\langle {Q} \rangle \red P\{\quotep{Q}/y}\} }
  \and \\
  \inferrule* [lab=Par] {{P} \red {P}'} {{{P} | {Q}} \red {{P}' | {Q}}}
  \and
  \inferrule* [lab=Equiv]{{{P} \scong {P}'} \andalso {{P}' \red {Q}'} \andalso {{Q}' \scong {Q}}}{{P} \red {Q}}
\end{mathpar}

\begin{eqnarray*}
  match_{\equiv} (\quotep{P},\quotep{Q}) & := & P \equiv Q \\
  match_{\dagger}(\quotep{P},\quotep{Q}) & := & \forall R. P|Q \red^{*} R => R \red^{*} 0 \\
  match_{K}(\quotep{P},\quotep{Q}) & := & K \mbox{ for some context } K
\end{eqnarray*}

$u?(x)P | u!\langle Q \rangle \red P\{\quotep{Q}/x\}$

%We write $\wred$ for $\red^*$, and $P\red$ if $\exists Q $ such that $ P \red Q$.
We write $P\red$ if $\exists Q $ such that $ P \red Q$ and $P\not\red$, otherwise.

\section{Replication}

As mentioned before, it is known that replication (and hence
recursion) can be implemented in a higher-order process algebra
\cite{SangiorgiWalker}. As our first example of calculation with the
machinery thus far presented we give the construction explicitly in
the {\rhoc}.

\begin{eqnarray}
	D_{x} & := & \prefix{x}{y}{(\binpar{\outputp{x}{y}}{@{y}})} \nonumber\\
	\bangp_{x}{P} & := & \binpar{{x}!\langle{\binpar{D_{x}}{P}}\rangle}{D_{x}} \nonumber
\end{eqnarray}

\begin{eqnarray}
	\bangp_{x}{P} & & \nonumber\\
	=
	& {x}!\langle{(\prefix{x}{y}{(\outputp{x}{y} | @{y})) | P}}\rangle 
	      | \prefix{x}{y}{(\outputp{x}{y} | @{y})} & \nonumber\\
	\red
	& (\outputp{x}{y} | @{y})\substn{\quotep{(\prefix{x}{y}{(@{y} | \outputp{x}{y})) | P}}}{y} & \nonumber\\
	=
	& \outputp{x}{\quotep{(\prefix{x}{y}{(\outputp{x}{y} | @{y})) | P}}}
	  | {(\prefix{x}{y}{(\outputp{x}{y} | @{y})) | P}} & \nonumber\\
	\red
	& \ldots & \nonumber\\
	\red^*
	& P | P | \ldots & \nonumber
\end{eqnarray}

Of course, this encoding, as an implementation, runs away, unfolding
$\bangp{P}$ eagerly. A lazier and more implementable replication
operator, restricted to input-guarded processes, may be obtained as follows.

\begin{eqnarray}
\bangp{\prefix{u}{v}{P}} 
	:= 
	\binpar{\lift{x}{\prefix{u}{v}{(\binpar{D(x)}{P})}}}{D(x)} \nonumber
\end{eqnarray}

\begin{remark}
  Note that the lazier definition still does not deal with summation
  or mixed summation (i.e. sums over input and output). The reader is
  invited to construct definitions of replication that deal with these
  features. 

  Further, the definitions are parameterized in a name, $x$. Can you,
  gentle reader, make a definition that eliminates this parameter and
  guarantees no accidental interaction between the replication
  machinery and the process being replicated -- i.e. no accidental
  sharing of names used by the process to get its work done and the
  name(s) used by the replication to effect copying. This latter
  revision of the definition of replication is crucial to obtaining
  the expected identity $!!P \sim !P$.
\end{remark}

\begin{remark}\label{rem:paradoxical_combinator}
  The reader familiar with the lambda calculus will have noticed the
  similarity between $D$ and the paradoxical combinator.

  [Ed. note: the existence of this seems to suggest we have to be more
  restrictive on the set of processes and names we admit if we are to
  support no-cloning.]
\end{remark}

\subsubsection{Bisimulation}

The computational dynamics gives rise to another kind of equivalence,
the equivalence of computational behavior. As previously mentioned
this is typically captured \emph{via} some form of bisimulation.

% The notion we use in this paper is weak barbed bisimulation
% \cite{milner91polyadicpi}.

The notion we use in this paper is derived from weak barbed
bisimulation \cite{milner91polyadicpi}. 

\begin{definition}
An \emph{observation relation}, $\downarrow_{\mathcal N}$, over a set
of names, $\mathcal N$, is the smallest relation satisfying the rules
below.

\infrule[Out-barb]{y \in {\mathcal N}, \; x \nameeq y}
		  {\outputp{x}{v} \downarrow_{\mathcal N} x}
\infrule[Par-barb]{\mbox{$P\downarrow_{\mathcal N} x$ or $Q\downarrow_{\mathcal N} x$}}
		  {\binpar{P}{Q} \downarrow_{\mathcal N} x}

We write $P \Downarrow_{\mathcal N} x$ if there is $Q$ such that 
$P \wred Q$ and $Q \downarrow_{\mathcal N} x$.
\end{definition}

\begin{definition}
%\label{def.bbisim}
An  ${\mathcal N}$-\emph{barbed bisimulation} over a set of names, ${\mathcal N}$, is a symmetric binary relation 
${\mathcal S}_{\mathcal N}$ between agents such that $P\rel{S}_{\mathcal N}Q$ implies:
\begin{enumerate}
\item If $P \red P'$ then $Q \wred Q'$ and $P'\rel{S}_{\mathcal N} Q'$.
\item If $P\downarrow_{\mathcal N} x$, then $Q\Downarrow_{\mathcal N} x$.
\end{enumerate}
$P$ is ${\mathcal N}$-barbed bisimilar to $Q$, written
$P \wbbisim_{\mathcal N} Q$, if $P \rel{S}_{\mathcal N} Q$ for some ${\mathcal N}$-barbed bisimulation ${\mathcal S}_{\mathcal N}$.
\end{definition}

$\mathcal{R} \subseteq \pi \times \pi$

$P \mathcal{R} Q => \forall P'. P \red P' \Rightarrow \exists Q'. Q \red Q', P' \mathcal{R} Q'$

$P \vdash x \Rightarrow Q \vdash x$

\begin{mathpar}
  \inferrule*[lab=Out-barb]{x \nameeq y}{{y}!\langle{Q}\rangle \vdash x}
  \and
  \inferrule*[lab=Par-barb]{\mbox{$P\vdash x$ or $Q\vdash x$}}{\binpar{P}{Q} \vdash x}
\end{mathpar}

\subsubsection{Contexts}

One of the principle advantages of computational calculi like the
$\pi$-calculus is a well-defined notion of context,
contextual-equivalence and a correlation between
contextual-equivalence and notions of bisimulation. The notion of
context allows the decomposition of a process into (sub-)process and
its syntactic environment, its context. Thus, a context may be
thought of as a process with a ``hole'' (written $\Box$) in it. The
application of a context $M$ to a process $P$, written $M[P]$, is
tantamount to filling the hole in $M$ with $P$. In this paper we do
not need the full weight of this theory, but do make use of the notion
of context in the proof the main theorem. 

\begin{mathpar}
  \inferrule* [lab=summation] {} {{M_{M},M_{N}} \bc \Box \;|\; x.M_{A} \;|\; M_{M}+M_{N}}
  \and
  \inferrule* [lab=agent] {} {{M_{A}} \bc (\vec{x})M_{P} \;| \; \clift{P_0,\ldots,M_{P},\ldots,P_N}}
  \and \\
  \inferrule* [lab=process] {} {{M_{P}} \bc M_{N} \;| \;P|M_{P} }
\end{mathpar} 

\begin{mathpar}
  \inferrule* [lab=sychronization] {} {M_{N} \bc \Box \;|\; x?M_{F} \;|\; x!M_{C}}
  \and
  \inferrule* [lab=abstraction] {} {{M_{F}} \bc (x)M_{P} }
  \and
  \inferrule* [lab=concretion] {} {{M_{C}} \bc \langle M_{P} \rangle }
  \and \\
  \inferrule* [lab=process] {} {{M_{P}} \bc M_{N} \;| \;P|M_{P} }
\end{mathpar}

\begin{definition}[contextual application] Given a context $M$, and
  process $P$, we define the \emph{contextual application}, $M[P] :=
  M\{P/\Box\}$. That is, the contextual application of M to P is the
  substitution of $P$ for $\Box$ in $M$.
\end{definition}

$\meaningof{-} : L \to \mathcal{P}(\pi)$

\begin{mathpar}
  \inferrule* [lab=collection] {} {\meaningof{true} = \pi, \and \meaningof{~E} = \pi \setminus \meaningof{E}, \and \meaningof{E_{1} \& E_{2}} = \meaningof{E_{1}} \cap \meaningof{E_{2}}}
\end{mathpar}

\begin{mathpar}
  \inferrule* [lab=structure] {} {\meaningof{0} = \{ P \in \pi | P \equiv 0 \}, \and \\ \meaningof{E_1 | E_2} = \{ P \in \pi | P \equiv P_{1} | P_{2}, P_{1} \in \meaningof{E_{1}}, P_{2} \in \meaningof{E_2}\} }
\end{mathpar}

\begin{mathpar}
 \inferrule* [lab=behavior] {} {\meaningof{\langle a?b \rangle E} = \{ P \in \pi | P \equiv Q | u?(y)P', \\ \and \\\\ \and \\ \;\;\; u \in \meaningof{a}, \forall z.P'\{z/y\} \in \meaningof{E\{z/b\}}\}, \and \\ \meaningof{a!E} = \{ P \in \pi | P \equiv Q | x!\langle P' \rangle, x \in \meaningof{a} P' \in \meaningof{E}\} }
\end{mathpar}

\begin{mathpar}
 \inferrule* [lab=nominal] {} {\meaningof{\quotep{E}} = \{ \quotep{P} \in \quotep{\pi} | P \in \meaningof{E} \}, \and \meaningof{\quotep{P}} = \{ \quotep{Q} \in \quotep{\pi} | P \equiv Q \} \and \\ \meaningof{@\quotep{E}} = \{ P \in \pi | P \equiv @x, x \in \meaningof{E} \}}
\end{mathpar}

\begin{eqnarray*}
  \\
  \meaningof{-} : TS \to ST
\end{eqnarray*}

\begin{eqnarray*}
  \\
  L : TS \to ST
\end{eqnarray*}

\begin{eqnarray*}
  \\
  P \models E \iff P \in \meaningof{E}
\end{eqnarray*}

\begin{eqnarray*}
  P \approx_{L} Q \iff \forall E \in L. P \models E \iff Q \models E
\end{eqnarray*}

\begin{eqnarray*}
  P \approx_{K} Q
\end{eqnarray*}

\begin{eqnarray*}
  P \approx Q
\end{eqnarray*}

$\approx_{K} = \approx = \approx_{L}$

\subsubsection{Contextual duality}

Note that contexts extend the quotation operation to a family of
operations from processes to names. Given a context, $M$, we can
define a \emph{nominal context}, $\quotep{M}$ by $\quotep{M}[P] :=
\quotep{M[P]}$. To foreshadow what is to come we observe that these
operations enjoy a duality with processes very much like the duality
between vectors and maps from vectors to scalars.

Further, because the calculus is essentially higher-order, we have a
correspondence between contexts and processes. More specifically,
given a name $x$ and a context $M$ we can construct $M^{*}_{x}$ such
that 

\begin{mathpar}
  M^{*}_{x} | \lift{x}{P} \red M[P]
\end{mathpar}

namely,

\begin{mathpar}
  M^{*}_{x} := x?(u).M[\dropn{u}]
\end{mathpar}

The dependence of $M^{*}_{x}$ on a name makes it an abstraction, 

\begin{mathpar}
  M^{*} := (x)x?(u).M[\dropn{u}]
\end{mathpar}

\subsection{Additional notation}

It will sometimes be convenient to denote the process a name
quotes. We already have the notation $x = \quotep{P}$, but it will be
convenient to introduce an alternate notation, $\procn{x}$, when we
want to emphasize the connection to the use of the name. Note that, by
virtue of name equivalence, $\quotep{\procn{x}} \nameeq x$; so, the
notation is consistent with previous definitions.

Further, because names have structure it is possible to effect
substitutions on the basis of that structure. This means we need to
upgrade our notation for substitutions, which we accomplish by
adapting comprehension notation. Thus,

\begin{mathpar}
  P\{ y / x : x \in S \}
\end{mathpar}

is interpreted to mean the process derived from P by replacing (in a
capture-avoiding manner) each occurrence of $x$ in $S$ by $y$. For example,

\begin{mathpar}
  P\{ \quotep{\procn{x}|\procn{x}} / x : x \in \freenames{P} \}
\end{mathpar}

will replace each (occurrence) of a free name $x$ in $P$ by
$\quotep{\procn{x}|\procn{x}}$.

Also, we will avail ourselves of the notation $x^{L}$ and $x^{R}$ to
denote injections of a name into disjoint copies of the name
space. There are numerous ways to accomplish this. One example can be
found in \cite{MeredithR05}. This notation overloads to vectors of
names: $\vec{x}^{\pi} := (x_{i}^{\pi} \; : \; 0 \leq i < |\vec{x}| )$ where $\pi \in \{L,R\}$.

We also use $P^{\Box} := P|\Box$.

In \cite{MeredithR05} an interpretation of the new operator is
given. It turns out that there are several possible interpretations
all enjoying the requisite algebraic properties of the operator (see
\cite{milner91polyadicpi}). We will therefore make liberal use of
$(\nu\; \vec{x})P$.

% subsection the_syntax_and_semantics_of_the_notation_system (end)   

\input{qm2pi.qmops} 

\input{qm2pi.sterngerlach} 

\input{qm2pi.metric} 

% section concurrent_process_calculi (end)

%\input{qm2pi.proofsketch}

% section proof sketch (end)

%\input{qm2pi.slviaknots} 

% section spatial logic via knots (end)

\input{qm2pi.conclusion}

% section conclusion (end)

%\input{qm2pi.dtcodes} 

% section wiring algorithm (end)

\input{qm2pi.ack} 

% section acknowledgments (end)

\newpage


\bibliographystyle{plain}   
\bibliography{../../biblios/main.bib}

\input{qm2pi.rhodetails}

\end{document}

 

%\documentclass[12pt]{llncs}
%\documentclass{jktr}

\usepackage[pdftex]{hyperref}                   
\usepackage {listings}
\usepackage {mathpartir}
\usepackage{bcprules}
%\usepackage{listings}
                       
\usepackage{graphicx} 
%\usepackage[margins=2.5cm,nohead,nofoot]{geometry}
%\usepackage{geometry}
\usepackage{amsfonts}
\usepackage{amstext}
\usepackage{latexsym}
\usepackage{amssymb}
\usepackage{color}


%\include{myPreamble}
\include{qm2pi.local} 

%\ifpdf
%\usepackage[pdftex]{graphicx}
%\else
%\usepackage{graphicx}
%\fi

 % \ifpdf
%  \usepackage{pdfsync}
%  \if


%\title{Brief Article}
%\author{David F. Snyder}
%\author{L.G. Meredith}

%\address{Dept. of Math., Texas State University--San Marcos, San Marcos, TX 78666}
       
\pagestyle{empty}


\begin{document}

\lstset{language=[Objective]Caml,frame=shadowbox}

\input{qm2pi.front}

% section front matter (end)

\input{qm2pi.intro} 
 
% section introduction (end)

% \input{qm2pi.knotations} 

% section notation (end)

\input{qm2pi.process.calculi} 

% section concurrent_process_calculi_and_spatial_logics_ (end)
    
%\input{qm2pi.knots2pi} 

%\input{qm2pi.trefoil} 

%\input{qm2pi.mainthm} 

% subsection basic_interpretation (end)

%\input{qm2pi.rho.presentation} 
\subsection{The syntax and semantics of the notation system}\label{sub:the_syntax_and_semantics_of_the_notation_system} % (fold)

We now summarize a technical presentation of the calculus that
embodies our theory of dynamics. The typical presentation of such a
calculus follows the style of giving generators and relations on
them. The grammar, below, describing term constructors, freely
generates the set of processes, $\Proc$. This set is then quotiented
by a relation known as structural congruence and it is over this set
that the notion of dynamics is expressed. This presentation is
essentially that of \cite{MeredithR05} with the addition of
polyadicity and summation. For readability we have relegated some of
the technical subtleties to an appendix.

\subsubsection{Process grammar}\label{subsub:process_grammar}

\begin{mathpar}
  \inferrule* [lab=synchronization] {} {{M} \bc \pzero \;|\; x?F \;|\; x!C }
  \and
  \inferrule* [lab=abstraction] {} {{F} \bc (x)P}
  \and
  \inferrule* [lab=concretion] {} {{C} \bc \langle Q \rangle}
  \and
  \inferrule* [lab=process] {} {{P,Q} \bc M \;| \;P|Q \;|\; @{x}}
  \and
  \inferrule* [lab=name] {} {{x} \bc \quotep{P}}
\end{mathpar} 

Note that $\vec{x}$ (resp. $\vec{P}$) denotes a vector of names
(resp. processes) of length $|\vec{x}|$ (resp. $|\vec{P}|$). We adopt
the following useful abbreviations.

\begin{mathpar}
   x?(\vec{y}).P := x.(\vec{y})P \and  x\clift{\vec{P}} := x.\clift{\vec{P}}
   \and x!(y) := \lift{x}{\dropn{y}}
   \and \Pi_{i=0}^{n-1}P_i := P_0 | \ldots | P_{n-1}
\end{mathpar}

\subsubsection{Structural congruence}

\paragraph{Free and bound names and alpha-equivalence.} At the
core of structural equivalence is alpha-equivalence which identifies
process that are the same up to a change of variable. Formally, we
recognize the distinction between free and bound names. The free names
of a process, $\freenames{P}$, may be calculated recursively as
follows:

\begin{mathpar}
\freenames{\pzero} := \emptyset
  \and \\
  \freenames{x?(y).P} := \{ x \} \cup (\freenames{P} \setminus \{ y \})
  \and 
  \freenames{x!\langle P \rangle} := \{ x \} \cup \{ P \} 
  \and \\
  \freenames{P|Q} := \freenames{P} \cup \freenames{Q}
  \and \\
  \freenames{@{x}} := \{ x \}
\end{mathpar}

$\pi$
$\quotep{\pi}$

$\freenames{-} : \pi \to \mathcal{P}(\quotep{\pi})$

\begin{eqnarray*}
  \freenames{\pzero} & := & \emptyset \\
  \freenames{x?(y).P} & := & \{ x \} \cup (\freenames{P} \setminus \{ y \}) \\
  \freenames{x!\langle P \rangle} & := & \{ x \} \cup \{ P \} \\
  \freenames{P|Q} & := & \freenames{P} \cup \freenames{Q} \\
  \freenames{\dropn{x}} & := & \{ x \}
\end{eqnarray*}

The bound names of a process, $\boundnames{P}$, are those names occurring in $P$
that are not free. For example, in $x?(y).0$, the name $x$ is free, while $y$ is bound.

\begin{mathpar}
  \inferrule* [lab=monoidal-laws] {} { P|Q \equiv Q|P \and P|0 \equiv P \and P|(Q|R) \equiv (P|Q)|R }
\end{mathpar}

\begin{mathpar}
  \inferrule* [lab=alpha-equivalence] {} { (x)P \equiv (y)P\{y/x\} \and y \not\in \freenames{P} }
\end{mathpar}

\begin{definition}
Then two processes, $P,Q$, are alpha-equivalent if $P = Q\{\vec{y}/\vec{x}\}$ for
some $\vec{x} \in \boundnames{Q},\vec{y} \in \boundnames{P}$, where $Q\{\vec{y}/\vec{x}\}$
denotes the capture-avoiding substitution of $\vec{y}$ for $\vec{x}$ in $Q$.
\end{definition}

\begin{definition}
  The {\em structural congruence} \cite{SangiorgiWalker} , $\equiv$,
  between processes is the least congruence containing
  alpha-equivalence, satisfying the abelian monoid laws
  (associativity, commutativity and $\pzero$ as identity) for parallel
  composition $|$ and for summation $+$.
\end{definition}

\subsection{Name equivalence}

We take name equivalence, written $\nameeq$, to be the smallest
equivalence relation generated by the following rules.

\begin{mathpar}
\inferrule*[lab=Quote-drop]
{ }
{ \quotep{@{x}} \nameeq x }

\inferrule*[lab=Struct-equiv]
{ P \scong Q }
{ \quotep{P} \nameeq \quotep{Q} }
\end{mathpar}

The astute reader will have noticed that the mutual recursion of names
and processes imposes a mutual recursion on alpha-equivalence and
structural equivalence via name-equivalence. Fortunately, all of this
works out pleasantly and we may calculate in the natural way, free of
concern. The reader interested in the details is referred to the
appendix \ref{appendix:rho_details}.

\subsection{Substitution}

We use $\Proc$ for the set of processes, $\QProc$ for the set of
names, and $\id{\{}\vec{y} / \vec{x} \id{\}}$ to denote partial maps,
$s : \QProc \rightarrow \QProc$. A map, $s$ lifts, uniquely, to a map
on process terms, $\widehat{s} : \Proc \rightarrow \Proc$ by the
following equations.

\begin{mathpar}
  (0) \psubstp{Q}{P} := 0 \\
  (R \juxtap S) \psubstp{Q}{P}
  :=    
  (R)\psubstp{Q}{P} \juxtap (S) \psubstp{Q}{P} \\
  (x?(y).R) \psubstp{Q}{P}    
  :=    
  (x)\substp{Q}{P} (z)\concat( (R \psubstn{z}{y}) \psubstp{Q}{P} ) \\
  (\lift{x}{R}) \psubstp{Q}{P}  
  :=
  \lift{(x)\substp{Q}{P}}{ R \psubstp{Q}{P} } \\
%   (\dropn{x})  \psubstp{Q}{P}       
%   := 
%   \left\{ 
%     \begin{array}{ccc} 
%       \dropn{\quotep{Q}} & & x \nameeq \quotep{P} \\
%       \dropn{x} & & otherwise \\
%     \end{array}
%   \right. 
  (\dropn{x})  \psubstp{Q}{P}       
  := 
  \left\{ 
    \begin{array}{ccc} 
      Q & & x \nameeq \quotep{P} \\
      \dropn{x} & & otherwise \\
    \end{array}
  \right.
\end{mathpar}
 

where

\begin{eqnarray}
  (x)\id{\{} \lpquote Q \rpquote / \lpquote P \rpquote \id{\}}            = 
  \left\{ 
    \begin{array}{ccc}
      \lpquote Q \rpquote & & x \nameeq \lpquote P \rpquote \\
      x & & otherwise \\
    \end{array}
  \right. \nonumber
\end{eqnarray}

and $z$ is chosen distinct from $\quotep{P}$, $\quotep{Q}$, the free
names in $Q$, and all the names in $R$. Our $\alpha$-equivalence will
be built in the standard way from this substitution.

\begin{remark}\label{rem:no_self_referential_names}
  One consequence of these definitions is that $\forall P. \quotep{P}
  \not\in \freenames{P}$.
\end{remark}

\subsection{ Dynamic quote: an example }

Anticipating something of what's to come, consider applying the
substitution, $\widehat{\id{\{}u / z \id{\}}}$, to the following pair
of processes, $\lift{w}{y!(z)}$ and $w[ \lpquote y!(z) \rpquote ]$.

\begin{eqnarray}
	\lift{w}{y!(z)}\widehat{\id{\{}u / z \id{\}}}
		& = &
		\lift{w}{y!(u)} \nonumber\\
	w[ \lpquote y!(z) \rpquote ] \widehat{ \id{\{}u / z \id{\}} }
		& = &
		w[ \lpquote y!(z) \rpquote ] \nonumber
\end{eqnarray}

Because the body of the process between quotes is impervious to
substitution, we get radically different answers. In fact, by
examining the first process in an input context,
e.g. $x?(z).\lift{w}{y!(z)}$, we see that the process under the lift
operator may be shaped by prefixed inputs binding a name inside it. In
this sense, the lift operator will be seen as a way to dynamically
construct processes before reifying them as names.

Finally equipped with these standard features we can present the
dynamics of the calculus.

\subsubsection{Operational semantics} 

Finally, we introduce the computational dynamics. What marks these
algebras as distinct from other more traditionally studied algebraic
structures, e.g. vector spaces or polynomial rings, is the manner in
which dynamics is captured. In traditional structures, dynamics is typically
expressed through morphisms between such structures, as in linear maps
between vector spaces or morphisms between rings. In algebras
associated with the semantics of computation, the dynamics is
expressed as part of the algebraic structure itself, through a
reduction reduction relation typically denoted by $\red$. Below, we
give a recursive presentation of this relation for the calculus used
in the encoding.

$\red \subseteq \pi \times \pi$
$\red : \pi \to \mathcal{P}(\pi)$

\begin{mathpar}
  \inferrule* [lab=Comm] { \textsf{match}( x_{src}, x_{trgt} ) } { x_{trgt}?(y)P \; | \; x_{src}!\langle {Q} \rangle \red P\{\quotep{Q}/y}\} }
  \and \\
  \inferrule* [lab=Par] {{P} \red {P}'} {{{P} | {Q}} \red {{P}' | {Q}}}
  \and
  \inferrule* [lab=Equiv]{{{P} \scong {P}'} \andalso {{P}' \red {Q}'} \andalso {{Q}' \scong {Q}}}{{P} \red {Q}}
\end{mathpar}

\begin{eqnarray*}
  match_{\equiv} (\quotep{P},\quotep{Q}) & := & P \equiv Q \\
  match_{\dagger}(\quotep{P},\quotep{Q}) & := & \forall R. P|Q \red^{*} R => R \red^{*} 0 \\
  match_{K}(\quotep{P},\quotep{Q}) & := & K \mbox{ for some context } K
\end{eqnarray*}

$u?(x)P | u!\langle Q \rangle \red P\{\quotep{Q}/x\}$

%We write $\wred$ for $\red^*$, and $P\red$ if $\exists Q $ such that $ P \red Q$.
We write $P\red$ if $\exists Q $ such that $ P \red Q$ and $P\not\red$, otherwise.

\section{Replication}

As mentioned before, it is known that replication (and hence
recursion) can be implemented in a higher-order process algebra
\cite{SangiorgiWalker}. As our first example of calculation with the
machinery thus far presented we give the construction explicitly in
the {\rhoc}.

\begin{eqnarray}
	D_{x} & := & \prefix{x}{y}{(\binpar{\outputp{x}{y}}{@{y}})} \nonumber\\
	\bangp_{x}{P} & := & \binpar{{x}!\langle{\binpar{D_{x}}{P}}\rangle}{D_{x}} \nonumber
\end{eqnarray}

\begin{eqnarray}
	\bangp_{x}{P} & & \nonumber\\
	=
	& {x}!\langle{(\prefix{x}{y}{(\outputp{x}{y} | @{y})) | P}}\rangle 
	      | \prefix{x}{y}{(\outputp{x}{y} | @{y})} & \nonumber\\
	\red
	& (\outputp{x}{y} | @{y})\substn{\quotep{(\prefix{x}{y}{(@{y} | \outputp{x}{y})) | P}}}{y} & \nonumber\\
	=
	& \outputp{x}{\quotep{(\prefix{x}{y}{(\outputp{x}{y} | @{y})) | P}}}
	  | {(\prefix{x}{y}{(\outputp{x}{y} | @{y})) | P}} & \nonumber\\
	\red
	& \ldots & \nonumber\\
	\red^*
	& P | P | \ldots & \nonumber
\end{eqnarray}

Of course, this encoding, as an implementation, runs away, unfolding
$\bangp{P}$ eagerly. A lazier and more implementable replication
operator, restricted to input-guarded processes, may be obtained as follows.

\begin{eqnarray}
\bangp{\prefix{u}{v}{P}} 
	:= 
	\binpar{\lift{x}{\prefix{u}{v}{(\binpar{D(x)}{P})}}}{D(x)} \nonumber
\end{eqnarray}

\begin{remark}
  Note that the lazier definition still does not deal with summation
  or mixed summation (i.e. sums over input and output). The reader is
  invited to construct definitions of replication that deal with these
  features. 

  Further, the definitions are parameterized in a name, $x$. Can you,
  gentle reader, make a definition that eliminates this parameter and
  guarantees no accidental interaction between the replication
  machinery and the process being replicated -- i.e. no accidental
  sharing of names used by the process to get its work done and the
  name(s) used by the replication to effect copying. This latter
  revision of the definition of replication is crucial to obtaining
  the expected identity $!!P \sim !P$.
\end{remark}

\begin{remark}\label{rem:paradoxical_combinator}
  The reader familiar with the lambda calculus will have noticed the
  similarity between $D$ and the paradoxical combinator.

  [Ed. note: the existence of this seems to suggest we have to be more
  restrictive on the set of processes and names we admit if we are to
  support no-cloning.]
\end{remark}

\subsubsection{Bisimulation}

The computational dynamics gives rise to another kind of equivalence,
the equivalence of computational behavior. As previously mentioned
this is typically captured \emph{via} some form of bisimulation.

% The notion we use in this paper is weak barbed bisimulation
% \cite{milner91polyadicpi}.

The notion we use in this paper is derived from weak barbed
bisimulation \cite{milner91polyadicpi}. 

\begin{definition}
An \emph{observation relation}, $\downarrow_{\mathcal N}$, over a set
of names, $\mathcal N$, is the smallest relation satisfying the rules
below.

\infrule[Out-barb]{y \in {\mathcal N}, \; x \nameeq y}
		  {\outputp{x}{v} \downarrow_{\mathcal N} x}
\infrule[Par-barb]{\mbox{$P\downarrow_{\mathcal N} x$ or $Q\downarrow_{\mathcal N} x$}}
		  {\binpar{P}{Q} \downarrow_{\mathcal N} x}

We write $P \Downarrow_{\mathcal N} x$ if there is $Q$ such that 
$P \wred Q$ and $Q \downarrow_{\mathcal N} x$.
\end{definition}

\begin{definition}
%\label{def.bbisim}
An  ${\mathcal N}$-\emph{barbed bisimulation} over a set of names, ${\mathcal N}$, is a symmetric binary relation 
${\mathcal S}_{\mathcal N}$ between agents such that $P\rel{S}_{\mathcal N}Q$ implies:
\begin{enumerate}
\item If $P \red P'$ then $Q \wred Q'$ and $P'\rel{S}_{\mathcal N} Q'$.
\item If $P\downarrow_{\mathcal N} x$, then $Q\Downarrow_{\mathcal N} x$.
\end{enumerate}
$P$ is ${\mathcal N}$-barbed bisimilar to $Q$, written
$P \wbbisim_{\mathcal N} Q$, if $P \rel{S}_{\mathcal N} Q$ for some ${\mathcal N}$-barbed bisimulation ${\mathcal S}_{\mathcal N}$.
\end{definition}

$\mathcal{R} \subseteq \pi \times \pi$

$P \mathcal{R} Q => \forall P'. P \red P' \Rightarrow \exists Q'. Q \red Q', P' \mathcal{R} Q'$

$P \vdash x \Rightarrow Q \vdash x$

\begin{mathpar}
  \inferrule*[lab=Out-barb]{x \nameeq y}{{y}!\langle{Q}\rangle \vdash x}
  \and
  \inferrule*[lab=Par-barb]{\mbox{$P\vdash x$ or $Q\vdash x$}}{\binpar{P}{Q} \vdash x}
\end{mathpar}

\subsubsection{Contexts}

One of the principle advantages of computational calculi like the
$\pi$-calculus is a well-defined notion of context,
contextual-equivalence and a correlation between
contextual-equivalence and notions of bisimulation. The notion of
context allows the decomposition of a process into (sub-)process and
its syntactic environment, its context. Thus, a context may be
thought of as a process with a ``hole'' (written $\Box$) in it. The
application of a context $M$ to a process $P$, written $M[P]$, is
tantamount to filling the hole in $M$ with $P$. In this paper we do
not need the full weight of this theory, but do make use of the notion
of context in the proof the main theorem. 

\begin{mathpar}
  \inferrule* [lab=summation] {} {{M_{M},M_{N}} \bc \Box \;|\; x.M_{A} \;|\; M_{M}+M_{N}}
  \and
  \inferrule* [lab=agent] {} {{M_{A}} \bc (\vec{x})M_{P} \;| \; \clift{P_0,\ldots,M_{P},\ldots,P_N}}
  \and \\
  \inferrule* [lab=process] {} {{M_{P}} \bc M_{N} \;| \;P|M_{P} }
\end{mathpar} 

\begin{mathpar}
  \inferrule* [lab=sychronization] {} {M_{N} \bc \Box \;|\; x?M_{F} \;|\; x!M_{C}}
  \and
  \inferrule* [lab=abstraction] {} {{M_{F}} \bc (x)M_{P} }
  \and
  \inferrule* [lab=concretion] {} {{M_{C}} \bc \langle M_{P} \rangle }
  \and \\
  \inferrule* [lab=process] {} {{M_{P}} \bc M_{N} \;| \;P|M_{P} }
\end{mathpar}

\begin{definition}[contextual application] Given a context $M$, and
  process $P$, we define the \emph{contextual application}, $M[P] :=
  M\{P/\Box\}$. That is, the contextual application of M to P is the
  substitution of $P$ for $\Box$ in $M$.
\end{definition}

$\meaningof{-} : L \to \mathcal{P}(\pi)$

\begin{mathpar}
  \inferrule* [lab=collection] {} {\meaningof{true} = \pi, \and \meaningof{~E} = \pi \setminus \meaningof{E}, \and \meaningof{E_{1} \& E_{2}} = \meaningof{E_{1}} \cap \meaningof{E_{2}}}
\end{mathpar}

\begin{mathpar}
  \inferrule* [lab=structure] {} {\meaningof{0} = \{ P \in \pi | P \equiv 0 \}, \and \\ \meaningof{E_1 | E_2} = \{ P \in \pi | P \equiv P_{1} | P_{2}, P_{1} \in \meaningof{E_{1}}, P_{2} \in \meaningof{E_2}\} }
\end{mathpar}

\begin{mathpar}
 \inferrule* [lab=behavior] {} {\meaningof{\langle a?b \rangle E} = \{ P \in \pi | P \equiv Q | u?(y)P', \\ \and \\\\ \and \\ \;\;\; u \in \meaningof{a}, \forall z.P'\{z/y\} \in \meaningof{E\{z/b\}}\}, \and \\ \meaningof{a!E} = \{ P \in \pi | P \equiv Q | x!\langle P' \rangle, x \in \meaningof{a} P' \in \meaningof{E}\} }
\end{mathpar}

\begin{mathpar}
 \inferrule* [lab=nominal] {} {\meaningof{\quotep{E}} = \{ \quotep{P} \in \quotep{\pi} | P \in \meaningof{E} \}, \and \meaningof{\quotep{P}} = \{ \quotep{Q} \in \quotep{\pi} | P \equiv Q \} \and \\ \meaningof{@\quotep{E}} = \{ P \in \pi | P \equiv @x, x \in \meaningof{E} \}}
\end{mathpar}

\begin{eqnarray*}
  \\
  \meaningof{-} : TS \to ST
\end{eqnarray*}

\begin{eqnarray*}
  \\
  L : TS \to ST
\end{eqnarray*}

\begin{eqnarray*}
  \\
  P \models E \iff P \in \meaningof{E}
\end{eqnarray*}

\begin{eqnarray*}
  P \approx_{L} Q \iff \forall E \in L. P \models E \iff Q \models E
\end{eqnarray*}

\begin{eqnarray*}
  P \approx_{K} Q
\end{eqnarray*}

\begin{eqnarray*}
  P \approx Q
\end{eqnarray*}

$\approx_{K} = \approx = \approx_{L}$

\subsubsection{Contextual duality}

Note that contexts extend the quotation operation to a family of
operations from processes to names. Given a context, $M$, we can
define a \emph{nominal context}, $\quotep{M}$ by $\quotep{M}[P] :=
\quotep{M[P]}$. To foreshadow what is to come we observe that these
operations enjoy a duality with processes very much like the duality
between vectors and maps from vectors to scalars.

Further, because the calculus is essentially higher-order, we have a
correspondence between contexts and processes. More specifically,
given a name $x$ and a context $M$ we can construct $M^{*}_{x}$ such
that 

\begin{mathpar}
  M^{*}_{x} | \lift{x}{P} \red M[P]
\end{mathpar}

namely,

\begin{mathpar}
  M^{*}_{x} := x?(u).M[\dropn{u}]
\end{mathpar}

The dependence of $M^{*}_{x}$ on a name makes it an abstraction, 

\begin{mathpar}
  M^{*} := (x)x?(u).M[\dropn{u}]
\end{mathpar}

\subsection{Additional notation}

It will sometimes be convenient to denote the process a name
quotes. We already have the notation $x = \quotep{P}$, but it will be
convenient to introduce an alternate notation, $\procn{x}$, when we
want to emphasize the connection to the use of the name. Note that, by
virtue of name equivalence, $\quotep{\procn{x}} \nameeq x$; so, the
notation is consistent with previous definitions.

Further, because names have structure it is possible to effect
substitutions on the basis of that structure. This means we need to
upgrade our notation for substitutions, which we accomplish by
adapting comprehension notation. Thus,

\begin{mathpar}
  P\{ y / x : x \in S \}
\end{mathpar}

is interpreted to mean the process derived from P by replacing (in a
capture-avoiding manner) each occurrence of $x$ in $S$ by $y$. For example,

\begin{mathpar}
  P\{ \quotep{\procn{x}|\procn{x}} / x : x \in \freenames{P} \}
\end{mathpar}

will replace each (occurrence) of a free name $x$ in $P$ by
$\quotep{\procn{x}|\procn{x}}$.

Also, we will avail ourselves of the notation $x^{L}$ and $x^{R}$ to
denote injections of a name into disjoint copies of the name
space. There are numerous ways to accomplish this. One example can be
found in \cite{MeredithR05}. This notation overloads to vectors of
names: $\vec{x}^{\pi} := (x_{i}^{\pi} \; : \; 0 \leq i < |\vec{x}| )$ where $\pi \in \{L,R\}$.

We also use $P^{\Box} := P|\Box$.

In \cite{MeredithR05} an interpretation of the new operator is
given. It turns out that there are several possible interpretations
all enjoying the requisite algebraic properties of the operator (see
\cite{milner91polyadicpi}). We will therefore make liberal use of
$(\nu\; \vec{x})P$.

% subsection the_syntax_and_semantics_of_the_notation_system (end)   

\input{qm2pi.qmops} 

\input{qm2pi.sterngerlach} 

\input{qm2pi.metric} 

% section concurrent_process_calculi (end)

%\input{qm2pi.proofsketch}

% section proof sketch (end)

%\input{qm2pi.slviaknots} 

% section spatial logic via knots (end)

\input{qm2pi.conclusion}

% section conclusion (end)

%\input{qm2pi.dtcodes} 

% section wiring algorithm (end)

\input{qm2pi.ack} 

% section acknowledgments (end)

\newpage


\bibliographystyle{plain}   
\bibliography{../../biblios/main.bib}

\input{qm2pi.rhodetails}

\end{document}

 

%\documentclass[12pt]{llncs}
%\documentclass{jktr}

\usepackage[pdftex]{hyperref}                   
\usepackage {listings}
\usepackage {mathpartir}
\usepackage{bcprules}
%\usepackage{listings}
                       
\usepackage{graphicx} 
%\usepackage[margins=2.5cm,nohead,nofoot]{geometry}
%\usepackage{geometry}
\usepackage{amsfonts}
\usepackage{amstext}
\usepackage{latexsym}
\usepackage{amssymb}
\usepackage{color}


%\include{myPreamble}
\include{qm2pi.local} 

%\ifpdf
%\usepackage[pdftex]{graphicx}
%\else
%\usepackage{graphicx}
%\fi

 % \ifpdf
%  \usepackage{pdfsync}
%  \if


%\title{Brief Article}
%\author{David F. Snyder}
%\author{L.G. Meredith}

%\address{Dept. of Math., Texas State University--San Marcos, San Marcos, TX 78666}
       
\pagestyle{empty}


\begin{document}

\lstset{language=[Objective]Caml,frame=shadowbox}

\input{qm2pi.front}

% section front matter (end)

\input{qm2pi.intro} 
 
% section introduction (end)

% \input{qm2pi.knotations} 

% section notation (end)

\input{qm2pi.process.calculi} 

% section concurrent_process_calculi_and_spatial_logics_ (end)
    
%\input{qm2pi.knots2pi} 

%\input{qm2pi.trefoil} 

%\input{qm2pi.mainthm} 

% subsection basic_interpretation (end)

%\input{qm2pi.rho.presentation} 
\subsection{The syntax and semantics of the notation system}\label{sub:the_syntax_and_semantics_of_the_notation_system} % (fold)

We now summarize a technical presentation of the calculus that
embodies our theory of dynamics. The typical presentation of such a
calculus follows the style of giving generators and relations on
them. The grammar, below, describing term constructors, freely
generates the set of processes, $\Proc$. This set is then quotiented
by a relation known as structural congruence and it is over this set
that the notion of dynamics is expressed. This presentation is
essentially that of \cite{MeredithR05} with the addition of
polyadicity and summation. For readability we have relegated some of
the technical subtleties to an appendix.

\subsubsection{Process grammar}\label{subsub:process_grammar}

\begin{mathpar}
  \inferrule* [lab=synchronization] {} {{M} \bc \pzero \;|\; x?F \;|\; x!C }
  \and
  \inferrule* [lab=abstraction] {} {{F} \bc (x)P}
  \and
  \inferrule* [lab=concretion] {} {{C} \bc \langle Q \rangle}
  \and
  \inferrule* [lab=process] {} {{P,Q} \bc M \;| \;P|Q \;|\; @{x}}
  \and
  \inferrule* [lab=name] {} {{x} \bc \quotep{P}}
\end{mathpar} 

Note that $\vec{x}$ (resp. $\vec{P}$) denotes a vector of names
(resp. processes) of length $|\vec{x}|$ (resp. $|\vec{P}|$). We adopt
the following useful abbreviations.

\begin{mathpar}
   x?(\vec{y}).P := x.(\vec{y})P \and  x\clift{\vec{P}} := x.\clift{\vec{P}}
   \and x!(y) := \lift{x}{\dropn{y}}
   \and \Pi_{i=0}^{n-1}P_i := P_0 | \ldots | P_{n-1}
\end{mathpar}

\subsubsection{Structural congruence}

\paragraph{Free and bound names and alpha-equivalence.} At the
core of structural equivalence is alpha-equivalence which identifies
process that are the same up to a change of variable. Formally, we
recognize the distinction between free and bound names. The free names
of a process, $\freenames{P}$, may be calculated recursively as
follows:

\begin{mathpar}
\freenames{\pzero} := \emptyset
  \and \\
  \freenames{x?(y).P} := \{ x \} \cup (\freenames{P} \setminus \{ y \})
  \and 
  \freenames{x!\langle P \rangle} := \{ x \} \cup \{ P \} 
  \and \\
  \freenames{P|Q} := \freenames{P} \cup \freenames{Q}
  \and \\
  \freenames{@{x}} := \{ x \}
\end{mathpar}

$\pi$
$\quotep{\pi}$

$\freenames{-} : \pi \to \mathcal{P}(\quotep{\pi})$

\begin{eqnarray*}
  \freenames{\pzero} & := & \emptyset \\
  \freenames{x?(y).P} & := & \{ x \} \cup (\freenames{P} \setminus \{ y \}) \\
  \freenames{x!\langle P \rangle} & := & \{ x \} \cup \{ P \} \\
  \freenames{P|Q} & := & \freenames{P} \cup \freenames{Q} \\
  \freenames{\dropn{x}} & := & \{ x \}
\end{eqnarray*}

The bound names of a process, $\boundnames{P}$, are those names occurring in $P$
that are not free. For example, in $x?(y).0$, the name $x$ is free, while $y$ is bound.

\begin{mathpar}
  \inferrule* [lab=monoidal-laws] {} { P|Q \equiv Q|P \and P|0 \equiv P \and P|(Q|R) \equiv (P|Q)|R }
\end{mathpar}

\begin{mathpar}
  \inferrule* [lab=alpha-equivalence] {} { (x)P \equiv (y)P\{y/x\} \and y \not\in \freenames{P} }
\end{mathpar}

\begin{definition}
Then two processes, $P,Q$, are alpha-equivalent if $P = Q\{\vec{y}/\vec{x}\}$ for
some $\vec{x} \in \boundnames{Q},\vec{y} \in \boundnames{P}$, where $Q\{\vec{y}/\vec{x}\}$
denotes the capture-avoiding substitution of $\vec{y}$ for $\vec{x}$ in $Q$.
\end{definition}

\begin{definition}
  The {\em structural congruence} \cite{SangiorgiWalker} , $\equiv$,
  between processes is the least congruence containing
  alpha-equivalence, satisfying the abelian monoid laws
  (associativity, commutativity and $\pzero$ as identity) for parallel
  composition $|$ and for summation $+$.
\end{definition}

\subsection{Name equivalence}

We take name equivalence, written $\nameeq$, to be the smallest
equivalence relation generated by the following rules.

\begin{mathpar}
\inferrule*[lab=Quote-drop]
{ }
{ \quotep{@{x}} \nameeq x }

\inferrule*[lab=Struct-equiv]
{ P \scong Q }
{ \quotep{P} \nameeq \quotep{Q} }
\end{mathpar}

The astute reader will have noticed that the mutual recursion of names
and processes imposes a mutual recursion on alpha-equivalence and
structural equivalence via name-equivalence. Fortunately, all of this
works out pleasantly and we may calculate in the natural way, free of
concern. The reader interested in the details is referred to the
appendix \ref{appendix:rho_details}.

\subsection{Substitution}

We use $\Proc$ for the set of processes, $\QProc$ for the set of
names, and $\id{\{}\vec{y} / \vec{x} \id{\}}$ to denote partial maps,
$s : \QProc \rightarrow \QProc$. A map, $s$ lifts, uniquely, to a map
on process terms, $\widehat{s} : \Proc \rightarrow \Proc$ by the
following equations.

\begin{mathpar}
  (0) \psubstp{Q}{P} := 0 \\
  (R \juxtap S) \psubstp{Q}{P}
  :=    
  (R)\psubstp{Q}{P} \juxtap (S) \psubstp{Q}{P} \\
  (x?(y).R) \psubstp{Q}{P}    
  :=    
  (x)\substp{Q}{P} (z)\concat( (R \psubstn{z}{y}) \psubstp{Q}{P} ) \\
  (\lift{x}{R}) \psubstp{Q}{P}  
  :=
  \lift{(x)\substp{Q}{P}}{ R \psubstp{Q}{P} } \\
%   (\dropn{x})  \psubstp{Q}{P}       
%   := 
%   \left\{ 
%     \begin{array}{ccc} 
%       \dropn{\quotep{Q}} & & x \nameeq \quotep{P} \\
%       \dropn{x} & & otherwise \\
%     \end{array}
%   \right. 
  (\dropn{x})  \psubstp{Q}{P}       
  := 
  \left\{ 
    \begin{array}{ccc} 
      Q & & x \nameeq \quotep{P} \\
      \dropn{x} & & otherwise \\
    \end{array}
  \right.
\end{mathpar}
 

where

\begin{eqnarray}
  (x)\id{\{} \lpquote Q \rpquote / \lpquote P \rpquote \id{\}}            = 
  \left\{ 
    \begin{array}{ccc}
      \lpquote Q \rpquote & & x \nameeq \lpquote P \rpquote \\
      x & & otherwise \\
    \end{array}
  \right. \nonumber
\end{eqnarray}

and $z$ is chosen distinct from $\quotep{P}$, $\quotep{Q}$, the free
names in $Q$, and all the names in $R$. Our $\alpha$-equivalence will
be built in the standard way from this substitution.

\begin{remark}\label{rem:no_self_referential_names}
  One consequence of these definitions is that $\forall P. \quotep{P}
  \not\in \freenames{P}$.
\end{remark}

\subsection{ Dynamic quote: an example }

Anticipating something of what's to come, consider applying the
substitution, $\widehat{\id{\{}u / z \id{\}}}$, to the following pair
of processes, $\lift{w}{y!(z)}$ and $w[ \lpquote y!(z) \rpquote ]$.

\begin{eqnarray}
	\lift{w}{y!(z)}\widehat{\id{\{}u / z \id{\}}}
		& = &
		\lift{w}{y!(u)} \nonumber\\
	w[ \lpquote y!(z) \rpquote ] \widehat{ \id{\{}u / z \id{\}} }
		& = &
		w[ \lpquote y!(z) \rpquote ] \nonumber
\end{eqnarray}

Because the body of the process between quotes is impervious to
substitution, we get radically different answers. In fact, by
examining the first process in an input context,
e.g. $x?(z).\lift{w}{y!(z)}$, we see that the process under the lift
operator may be shaped by prefixed inputs binding a name inside it. In
this sense, the lift operator will be seen as a way to dynamically
construct processes before reifying them as names.

Finally equipped with these standard features we can present the
dynamics of the calculus.

\subsubsection{Operational semantics} 

Finally, we introduce the computational dynamics. What marks these
algebras as distinct from other more traditionally studied algebraic
structures, e.g. vector spaces or polynomial rings, is the manner in
which dynamics is captured. In traditional structures, dynamics is typically
expressed through morphisms between such structures, as in linear maps
between vector spaces or morphisms between rings. In algebras
associated with the semantics of computation, the dynamics is
expressed as part of the algebraic structure itself, through a
reduction reduction relation typically denoted by $\red$. Below, we
give a recursive presentation of this relation for the calculus used
in the encoding.

$\red \subseteq \pi \times \pi$
$\red : \pi \to \mathcal{P}(\pi)$

\begin{mathpar}
  \inferrule* [lab=Comm] { \textsf{match}( x_{src}, x_{trgt} ) } { x_{trgt}?(y)P \; | \; x_{src}!\langle {Q} \rangle \red P\{\quotep{Q}/y}\} }
  \and \\
  \inferrule* [lab=Par] {{P} \red {P}'} {{{P} | {Q}} \red {{P}' | {Q}}}
  \and
  \inferrule* [lab=Equiv]{{{P} \scong {P}'} \andalso {{P}' \red {Q}'} \andalso {{Q}' \scong {Q}}}{{P} \red {Q}}
\end{mathpar}

\begin{eqnarray*}
  match_{\equiv} (\quotep{P},\quotep{Q}) & := & P \equiv Q \\
  match_{\dagger}(\quotep{P},\quotep{Q}) & := & \forall R. P|Q \red^{*} R => R \red^{*} 0 \\
  match_{K}(\quotep{P},\quotep{Q}) & := & K \mbox{ for some context } K
\end{eqnarray*}

$u?(x)P | u!\langle Q \rangle \red P\{\quotep{Q}/x\}$

%We write $\wred$ for $\red^*$, and $P\red$ if $\exists Q $ such that $ P \red Q$.
We write $P\red$ if $\exists Q $ such that $ P \red Q$ and $P\not\red$, otherwise.

\section{Replication}

As mentioned before, it is known that replication (and hence
recursion) can be implemented in a higher-order process algebra
\cite{SangiorgiWalker}. As our first example of calculation with the
machinery thus far presented we give the construction explicitly in
the {\rhoc}.

\begin{eqnarray}
	D_{x} & := & \prefix{x}{y}{(\binpar{\outputp{x}{y}}{@{y}})} \nonumber\\
	\bangp_{x}{P} & := & \binpar{{x}!\langle{\binpar{D_{x}}{P}}\rangle}{D_{x}} \nonumber
\end{eqnarray}

\begin{eqnarray}
	\bangp_{x}{P} & & \nonumber\\
	=
	& {x}!\langle{(\prefix{x}{y}{(\outputp{x}{y} | @{y})) | P}}\rangle 
	      | \prefix{x}{y}{(\outputp{x}{y} | @{y})} & \nonumber\\
	\red
	& (\outputp{x}{y} | @{y})\substn{\quotep{(\prefix{x}{y}{(@{y} | \outputp{x}{y})) | P}}}{y} & \nonumber\\
	=
	& \outputp{x}{\quotep{(\prefix{x}{y}{(\outputp{x}{y} | @{y})) | P}}}
	  | {(\prefix{x}{y}{(\outputp{x}{y} | @{y})) | P}} & \nonumber\\
	\red
	& \ldots & \nonumber\\
	\red^*
	& P | P | \ldots & \nonumber
\end{eqnarray}

Of course, this encoding, as an implementation, runs away, unfolding
$\bangp{P}$ eagerly. A lazier and more implementable replication
operator, restricted to input-guarded processes, may be obtained as follows.

\begin{eqnarray}
\bangp{\prefix{u}{v}{P}} 
	:= 
	\binpar{\lift{x}{\prefix{u}{v}{(\binpar{D(x)}{P})}}}{D(x)} \nonumber
\end{eqnarray}

\begin{remark}
  Note that the lazier definition still does not deal with summation
  or mixed summation (i.e. sums over input and output). The reader is
  invited to construct definitions of replication that deal with these
  features. 

  Further, the definitions are parameterized in a name, $x$. Can you,
  gentle reader, make a definition that eliminates this parameter and
  guarantees no accidental interaction between the replication
  machinery and the process being replicated -- i.e. no accidental
  sharing of names used by the process to get its work done and the
  name(s) used by the replication to effect copying. This latter
  revision of the definition of replication is crucial to obtaining
  the expected identity $!!P \sim !P$.
\end{remark}

\begin{remark}\label{rem:paradoxical_combinator}
  The reader familiar with the lambda calculus will have noticed the
  similarity between $D$ and the paradoxical combinator.

  [Ed. note: the existence of this seems to suggest we have to be more
  restrictive on the set of processes and names we admit if we are to
  support no-cloning.]
\end{remark}

\subsubsection{Bisimulation}

The computational dynamics gives rise to another kind of equivalence,
the equivalence of computational behavior. As previously mentioned
this is typically captured \emph{via} some form of bisimulation.

% The notion we use in this paper is weak barbed bisimulation
% \cite{milner91polyadicpi}.

The notion we use in this paper is derived from weak barbed
bisimulation \cite{milner91polyadicpi}. 

\begin{definition}
An \emph{observation relation}, $\downarrow_{\mathcal N}$, over a set
of names, $\mathcal N$, is the smallest relation satisfying the rules
below.

\infrule[Out-barb]{y \in {\mathcal N}, \; x \nameeq y}
		  {\outputp{x}{v} \downarrow_{\mathcal N} x}
\infrule[Par-barb]{\mbox{$P\downarrow_{\mathcal N} x$ or $Q\downarrow_{\mathcal N} x$}}
		  {\binpar{P}{Q} \downarrow_{\mathcal N} x}

We write $P \Downarrow_{\mathcal N} x$ if there is $Q$ such that 
$P \wred Q$ and $Q \downarrow_{\mathcal N} x$.
\end{definition}

\begin{definition}
%\label{def.bbisim}
An  ${\mathcal N}$-\emph{barbed bisimulation} over a set of names, ${\mathcal N}$, is a symmetric binary relation 
${\mathcal S}_{\mathcal N}$ between agents such that $P\rel{S}_{\mathcal N}Q$ implies:
\begin{enumerate}
\item If $P \red P'$ then $Q \wred Q'$ and $P'\rel{S}_{\mathcal N} Q'$.
\item If $P\downarrow_{\mathcal N} x$, then $Q\Downarrow_{\mathcal N} x$.
\end{enumerate}
$P$ is ${\mathcal N}$-barbed bisimilar to $Q$, written
$P \wbbisim_{\mathcal N} Q$, if $P \rel{S}_{\mathcal N} Q$ for some ${\mathcal N}$-barbed bisimulation ${\mathcal S}_{\mathcal N}$.
\end{definition}

$\mathcal{R} \subseteq \pi \times \pi$

$P \mathcal{R} Q => \forall P'. P \red P' \Rightarrow \exists Q'. Q \red Q', P' \mathcal{R} Q'$

$P \vdash x \Rightarrow Q \vdash x$

\begin{mathpar}
  \inferrule*[lab=Out-barb]{x \nameeq y}{{y}!\langle{Q}\rangle \vdash x}
  \and
  \inferrule*[lab=Par-barb]{\mbox{$P\vdash x$ or $Q\vdash x$}}{\binpar{P}{Q} \vdash x}
\end{mathpar}

\subsubsection{Contexts}

One of the principle advantages of computational calculi like the
$\pi$-calculus is a well-defined notion of context,
contextual-equivalence and a correlation between
contextual-equivalence and notions of bisimulation. The notion of
context allows the decomposition of a process into (sub-)process and
its syntactic environment, its context. Thus, a context may be
thought of as a process with a ``hole'' (written $\Box$) in it. The
application of a context $M$ to a process $P$, written $M[P]$, is
tantamount to filling the hole in $M$ with $P$. In this paper we do
not need the full weight of this theory, but do make use of the notion
of context in the proof the main theorem. 

\begin{mathpar}
  \inferrule* [lab=summation] {} {{M_{M},M_{N}} \bc \Box \;|\; x.M_{A} \;|\; M_{M}+M_{N}}
  \and
  \inferrule* [lab=agent] {} {{M_{A}} \bc (\vec{x})M_{P} \;| \; \clift{P_0,\ldots,M_{P},\ldots,P_N}}
  \and \\
  \inferrule* [lab=process] {} {{M_{P}} \bc M_{N} \;| \;P|M_{P} }
\end{mathpar} 

\begin{mathpar}
  \inferrule* [lab=sychronization] {} {M_{N} \bc \Box \;|\; x?M_{F} \;|\; x!M_{C}}
  \and
  \inferrule* [lab=abstraction] {} {{M_{F}} \bc (x)M_{P} }
  \and
  \inferrule* [lab=concretion] {} {{M_{C}} \bc \langle M_{P} \rangle }
  \and \\
  \inferrule* [lab=process] {} {{M_{P}} \bc M_{N} \;| \;P|M_{P} }
\end{mathpar}

\begin{definition}[contextual application] Given a context $M$, and
  process $P$, we define the \emph{contextual application}, $M[P] :=
  M\{P/\Box\}$. That is, the contextual application of M to P is the
  substitution of $P$ for $\Box$ in $M$.
\end{definition}

$\meaningof{-} : L \to \mathcal{P}(\pi)$

\begin{mathpar}
  \inferrule* [lab=collection] {} {\meaningof{true} = \pi, \and \meaningof{~E} = \pi \setminus \meaningof{E}, \and \meaningof{E_{1} \& E_{2}} = \meaningof{E_{1}} \cap \meaningof{E_{2}}}
\end{mathpar}

\begin{mathpar}
  \inferrule* [lab=structure] {} {\meaningof{0} = \{ P \in \pi | P \equiv 0 \}, \and \\ \meaningof{E_1 | E_2} = \{ P \in \pi | P \equiv P_{1} | P_{2}, P_{1} \in \meaningof{E_{1}}, P_{2} \in \meaningof{E_2}\} }
\end{mathpar}

\begin{mathpar}
 \inferrule* [lab=behavior] {} {\meaningof{\langle a?b \rangle E} = \{ P \in \pi | P \equiv Q | u?(y)P', \\ \and \\\\ \and \\ \;\;\; u \in \meaningof{a}, \forall z.P'\{z/y\} \in \meaningof{E\{z/b\}}\}, \and \\ \meaningof{a!E} = \{ P \in \pi | P \equiv Q | x!\langle P' \rangle, x \in \meaningof{a} P' \in \meaningof{E}\} }
\end{mathpar}

\begin{mathpar}
 \inferrule* [lab=nominal] {} {\meaningof{\quotep{E}} = \{ \quotep{P} \in \quotep{\pi} | P \in \meaningof{E} \}, \and \meaningof{\quotep{P}} = \{ \quotep{Q} \in \quotep{\pi} | P \equiv Q \} \and \\ \meaningof{@\quotep{E}} = \{ P \in \pi | P \equiv @x, x \in \meaningof{E} \}}
\end{mathpar}

\begin{eqnarray*}
  \\
  \meaningof{-} : TS \to ST
\end{eqnarray*}

\begin{eqnarray*}
  \\
  L : TS \to ST
\end{eqnarray*}

\begin{eqnarray*}
  \\
  P \models E \iff P \in \meaningof{E}
\end{eqnarray*}

\begin{eqnarray*}
  P \approx_{L} Q \iff \forall E \in L. P \models E \iff Q \models E
\end{eqnarray*}

\begin{eqnarray*}
  P \approx_{K} Q
\end{eqnarray*}

\begin{eqnarray*}
  P \approx Q
\end{eqnarray*}

$\approx_{K} = \approx = \approx_{L}$

\subsubsection{Contextual duality}

Note that contexts extend the quotation operation to a family of
operations from processes to names. Given a context, $M$, we can
define a \emph{nominal context}, $\quotep{M}$ by $\quotep{M}[P] :=
\quotep{M[P]}$. To foreshadow what is to come we observe that these
operations enjoy a duality with processes very much like the duality
between vectors and maps from vectors to scalars.

Further, because the calculus is essentially higher-order, we have a
correspondence between contexts and processes. More specifically,
given a name $x$ and a context $M$ we can construct $M^{*}_{x}$ such
that 

\begin{mathpar}
  M^{*}_{x} | \lift{x}{P} \red M[P]
\end{mathpar}

namely,

\begin{mathpar}
  M^{*}_{x} := x?(u).M[\dropn{u}]
\end{mathpar}

The dependence of $M^{*}_{x}$ on a name makes it an abstraction, 

\begin{mathpar}
  M^{*} := (x)x?(u).M[\dropn{u}]
\end{mathpar}

\subsection{Additional notation}

It will sometimes be convenient to denote the process a name
quotes. We already have the notation $x = \quotep{P}$, but it will be
convenient to introduce an alternate notation, $\procn{x}$, when we
want to emphasize the connection to the use of the name. Note that, by
virtue of name equivalence, $\quotep{\procn{x}} \nameeq x$; so, the
notation is consistent with previous definitions.

Further, because names have structure it is possible to effect
substitutions on the basis of that structure. This means we need to
upgrade our notation for substitutions, which we accomplish by
adapting comprehension notation. Thus,

\begin{mathpar}
  P\{ y / x : x \in S \}
\end{mathpar}

is interpreted to mean the process derived from P by replacing (in a
capture-avoiding manner) each occurrence of $x$ in $S$ by $y$. For example,

\begin{mathpar}
  P\{ \quotep{\procn{x}|\procn{x}} / x : x \in \freenames{P} \}
\end{mathpar}

will replace each (occurrence) of a free name $x$ in $P$ by
$\quotep{\procn{x}|\procn{x}}$.

Also, we will avail ourselves of the notation $x^{L}$ and $x^{R}$ to
denote injections of a name into disjoint copies of the name
space. There are numerous ways to accomplish this. One example can be
found in \cite{MeredithR05}. This notation overloads to vectors of
names: $\vec{x}^{\pi} := (x_{i}^{\pi} \; : \; 0 \leq i < |\vec{x}| )$ where $\pi \in \{L,R\}$.

We also use $P^{\Box} := P|\Box$.

In \cite{MeredithR05} an interpretation of the new operator is
given. It turns out that there are several possible interpretations
all enjoying the requisite algebraic properties of the operator (see
\cite{milner91polyadicpi}). We will therefore make liberal use of
$(\nu\; \vec{x})P$.

% subsection the_syntax_and_semantics_of_the_notation_system (end)   

\input{qm2pi.qmops} 

\input{qm2pi.sterngerlach} 

\input{qm2pi.metric} 

% section concurrent_process_calculi (end)

%\input{qm2pi.proofsketch}

% section proof sketch (end)

%\input{qm2pi.slviaknots} 

% section spatial logic via knots (end)

\input{qm2pi.conclusion}

% section conclusion (end)

%\input{qm2pi.dtcodes} 

% section wiring algorithm (end)

\input{qm2pi.ack} 

% section acknowledgments (end)

\newpage


\bibliographystyle{plain}   
\bibliography{../../biblios/main.bib}

\input{qm2pi.rhodetails}

\end{document}

 

% subsection basic_interpretation (end)

%\input{qm2pi.rho.presentation} 
\subsection{The syntax and semantics of the notation system}\label{sub:the_syntax_and_semantics_of_the_notation_system} % (fold)

We now summarize a technical presentation of the calculus that
embodies our theory of dynamics. The typical presentation of such a
calculus follows the style of giving generators and relations on
them. The grammar, below, describing term constructors, freely
generates the set of processes, $\Proc$. This set is then quotiented
by a relation known as structural congruence and it is over this set
that the notion of dynamics is expressed. This presentation is
essentially that of \cite{MeredithR05} with the addition of
polyadicity and summation. For readability we have relegated some of
the technical subtleties to an appendix.

\subsubsection{Process grammar}\label{subsub:process_grammar}

\begin{mathpar}
  \inferrule* [lab=synchronization] {} {{M} \bc \pzero \;|\; x?F \;|\; x!C }
  \and
  \inferrule* [lab=abstraction] {} {{F} \bc (x)P}
  \and
  \inferrule* [lab=concretion] {} {{C} \bc \langle Q \rangle}
  \and
  \inferrule* [lab=process] {} {{P,Q} \bc M \;| \;P|Q \;|\; @{x}}
  \and
  \inferrule* [lab=name] {} {{x} \bc \quotep{P}}
\end{mathpar} 

Note that $\vec{x}$ (resp. $\vec{P}$) denotes a vector of names
(resp. processes) of length $|\vec{x}|$ (resp. $|\vec{P}|$). We adopt
the following useful abbreviations.

\begin{mathpar}
   x?(\vec{y}).P := x.(\vec{y})P \and  x\clift{\vec{P}} := x.\clift{\vec{P}}
   \and x!(y) := \lift{x}{\dropn{y}}
   \and \Pi_{i=0}^{n-1}P_i := P_0 | \ldots | P_{n-1}
\end{mathpar}

\subsubsection{Structural congruence}

\paragraph{Free and bound names and alpha-equivalence.} At the
core of structural equivalence is alpha-equivalence which identifies
process that are the same up to a change of variable. Formally, we
recognize the distinction between free and bound names. The free names
of a process, $\freenames{P}$, may be calculated recursively as
follows:

\begin{mathpar}
\freenames{\pzero} := \emptyset
  \and \\
  \freenames{x?(y).P} := \{ x \} \cup (\freenames{P} \setminus \{ y \})
  \and 
  \freenames{x!\langle P \rangle} := \{ x \} \cup \{ P \} 
  \and \\
  \freenames{P|Q} := \freenames{P} \cup \freenames{Q}
  \and \\
  \freenames{@{x}} := \{ x \}
\end{mathpar}

$\pi$
$\quotep{\pi}$

$\freenames{-} : \pi \to \mathcal{P}(\quotep{\pi})$

\begin{eqnarray*}
  \freenames{\pzero} & := & \emptyset \\
  \freenames{x?(y).P} & := & \{ x \} \cup (\freenames{P} \setminus \{ y \}) \\
  \freenames{x!\langle P \rangle} & := & \{ x \} \cup \{ P \} \\
  \freenames{P|Q} & := & \freenames{P} \cup \freenames{Q} \\
  \freenames{\dropn{x}} & := & \{ x \}
\end{eqnarray*}

The bound names of a process, $\boundnames{P}$, are those names occurring in $P$
that are not free. For example, in $x?(y).0$, the name $x$ is free, while $y$ is bound.

\begin{mathpar}
  \inferrule* [lab=monoidal-laws] {} { P|Q \equiv Q|P \and P|0 \equiv P \and P|(Q|R) \equiv (P|Q)|R }
\end{mathpar}

\begin{mathpar}
  \inferrule* [lab=alpha-equivalence] {} { (x)P \equiv (y)P\{y/x\} \and y \not\in \freenames{P} }
\end{mathpar}

\begin{definition}
Then two processes, $P,Q$, are alpha-equivalent if $P = Q\{\vec{y}/\vec{x}\}$ for
some $\vec{x} \in \boundnames{Q},\vec{y} \in \boundnames{P}$, where $Q\{\vec{y}/\vec{x}\}$
denotes the capture-avoiding substitution of $\vec{y}$ for $\vec{x}$ in $Q$.
\end{definition}

\begin{definition}
  The {\em structural congruence} \cite{SangiorgiWalker} , $\equiv$,
  between processes is the least congruence containing
  alpha-equivalence, satisfying the abelian monoid laws
  (associativity, commutativity and $\pzero$ as identity) for parallel
  composition $|$ and for summation $+$.
\end{definition}

\subsection{Name equivalence}

We take name equivalence, written $\nameeq$, to be the smallest
equivalence relation generated by the following rules.

\begin{mathpar}
\inferrule*[lab=Quote-drop]
{ }
{ \quotep{@{x}} \nameeq x }

\inferrule*[lab=Struct-equiv]
{ P \scong Q }
{ \quotep{P} \nameeq \quotep{Q} }
\end{mathpar}

The astute reader will have noticed that the mutual recursion of names
and processes imposes a mutual recursion on alpha-equivalence and
structural equivalence via name-equivalence. Fortunately, all of this
works out pleasantly and we may calculate in the natural way, free of
concern. The reader interested in the details is referred to the
appendix \ref{appendix:rho_details}.

\subsection{Substitution}

We use $\Proc$ for the set of processes, $\QProc$ for the set of
names, and $\id{\{}\vec{y} / \vec{x} \id{\}}$ to denote partial maps,
$s : \QProc \rightarrow \QProc$. A map, $s$ lifts, uniquely, to a map
on process terms, $\widehat{s} : \Proc \rightarrow \Proc$ by the
following equations.

\begin{mathpar}
  (0) \psubstp{Q}{P} := 0 \\
  (R \juxtap S) \psubstp{Q}{P}
  :=    
  (R)\psubstp{Q}{P} \juxtap (S) \psubstp{Q}{P} \\
  (x?(y).R) \psubstp{Q}{P}    
  :=    
  (x)\substp{Q}{P} (z)\concat( (R \psubstn{z}{y}) \psubstp{Q}{P} ) \\
  (\lift{x}{R}) \psubstp{Q}{P}  
  :=
  \lift{(x)\substp{Q}{P}}{ R \psubstp{Q}{P} } \\
%   (\dropn{x})  \psubstp{Q}{P}       
%   := 
%   \left\{ 
%     \begin{array}{ccc} 
%       \dropn{\quotep{Q}} & & x \nameeq \quotep{P} \\
%       \dropn{x} & & otherwise \\
%     \end{array}
%   \right. 
  (\dropn{x})  \psubstp{Q}{P}       
  := 
  \left\{ 
    \begin{array}{ccc} 
      Q & & x \nameeq \quotep{P} \\
      \dropn{x} & & otherwise \\
    \end{array}
  \right.
\end{mathpar}
 

where

\begin{eqnarray}
  (x)\id{\{} \lpquote Q \rpquote / \lpquote P \rpquote \id{\}}            = 
  \left\{ 
    \begin{array}{ccc}
      \lpquote Q \rpquote & & x \nameeq \lpquote P \rpquote \\
      x & & otherwise \\
    \end{array}
  \right. \nonumber
\end{eqnarray}

and $z$ is chosen distinct from $\quotep{P}$, $\quotep{Q}$, the free
names in $Q$, and all the names in $R$. Our $\alpha$-equivalence will
be built in the standard way from this substitution.

\begin{remark}\label{rem:no_self_referential_names}
  One consequence of these definitions is that $\forall P. \quotep{P}
  \not\in \freenames{P}$.
\end{remark}

\subsection{ Dynamic quote: an example }

Anticipating something of what's to come, consider applying the
substitution, $\widehat{\id{\{}u / z \id{\}}}$, to the following pair
of processes, $\lift{w}{y!(z)}$ and $w[ \lpquote y!(z) \rpquote ]$.

\begin{eqnarray}
	\lift{w}{y!(z)}\widehat{\id{\{}u / z \id{\}}}
		& = &
		\lift{w}{y!(u)} \nonumber\\
	w[ \lpquote y!(z) \rpquote ] \widehat{ \id{\{}u / z \id{\}} }
		& = &
		w[ \lpquote y!(z) \rpquote ] \nonumber
\end{eqnarray}

Because the body of the process between quotes is impervious to
substitution, we get radically different answers. In fact, by
examining the first process in an input context,
e.g. $x?(z).\lift{w}{y!(z)}$, we see that the process under the lift
operator may be shaped by prefixed inputs binding a name inside it. In
this sense, the lift operator will be seen as a way to dynamically
construct processes before reifying them as names.

Finally equipped with these standard features we can present the
dynamics of the calculus.

\subsubsection{Operational semantics} 

Finally, we introduce the computational dynamics. What marks these
algebras as distinct from other more traditionally studied algebraic
structures, e.g. vector spaces or polynomial rings, is the manner in
which dynamics is captured. In traditional structures, dynamics is typically
expressed through morphisms between such structures, as in linear maps
between vector spaces or morphisms between rings. In algebras
associated with the semantics of computation, the dynamics is
expressed as part of the algebraic structure itself, through a
reduction reduction relation typically denoted by $\red$. Below, we
give a recursive presentation of this relation for the calculus used
in the encoding.

$\red \subseteq \pi \times \pi$
$\red : \pi \to \mathcal{P}(\pi)$

\begin{mathpar}
  \inferrule* [lab=Comm] { \textsf{match}( x_{src}, x_{trgt} ) } { x_{trgt}?(y)P \; | \; x_{src}!\langle {Q} \rangle \red P\{\quotep{Q}/y}\} }
  \and \\
  \inferrule* [lab=Par] {{P} \red {P}'} {{{P} | {Q}} \red {{P}' | {Q}}}
  \and
  \inferrule* [lab=Equiv]{{{P} \scong {P}'} \andalso {{P}' \red {Q}'} \andalso {{Q}' \scong {Q}}}{{P} \red {Q}}
\end{mathpar}

\begin{eqnarray*}
  match_{\equiv} (\quotep{P},\quotep{Q}) & := & P \equiv Q \\
  match_{\dagger}(\quotep{P},\quotep{Q}) & := & \forall R. P|Q \red^{*} R => R \red^{*} 0 \\
  match_{K}(\quotep{P},\quotep{Q}) & := & K \mbox{ for some context } K
\end{eqnarray*}

$u?(x)P | u!\langle Q \rangle \red P\{\quotep{Q}/x\}$

%We write $\wred$ for $\red^*$, and $P\red$ if $\exists Q $ such that $ P \red Q$.
We write $P\red$ if $\exists Q $ such that $ P \red Q$ and $P\not\red$, otherwise.

\section{Replication}

As mentioned before, it is known that replication (and hence
recursion) can be implemented in a higher-order process algebra
\cite{SangiorgiWalker}. As our first example of calculation with the
machinery thus far presented we give the construction explicitly in
the {\rhoc}.

\begin{eqnarray}
	D_{x} & := & \prefix{x}{y}{(\binpar{\outputp{x}{y}}{@{y}})} \nonumber\\
	\bangp_{x}{P} & := & \binpar{{x}!\langle{\binpar{D_{x}}{P}}\rangle}{D_{x}} \nonumber
\end{eqnarray}

\begin{eqnarray}
	\bangp_{x}{P} & & \nonumber\\
	=
	& {x}!\langle{(\prefix{x}{y}{(\outputp{x}{y} | @{y})) | P}}\rangle 
	      | \prefix{x}{y}{(\outputp{x}{y} | @{y})} & \nonumber\\
	\red
	& (\outputp{x}{y} | @{y})\substn{\quotep{(\prefix{x}{y}{(@{y} | \outputp{x}{y})) | P}}}{y} & \nonumber\\
	=
	& \outputp{x}{\quotep{(\prefix{x}{y}{(\outputp{x}{y} | @{y})) | P}}}
	  | {(\prefix{x}{y}{(\outputp{x}{y} | @{y})) | P}} & \nonumber\\
	\red
	& \ldots & \nonumber\\
	\red^*
	& P | P | \ldots & \nonumber
\end{eqnarray}

Of course, this encoding, as an implementation, runs away, unfolding
$\bangp{P}$ eagerly. A lazier and more implementable replication
operator, restricted to input-guarded processes, may be obtained as follows.

\begin{eqnarray}
\bangp{\prefix{u}{v}{P}} 
	:= 
	\binpar{\lift{x}{\prefix{u}{v}{(\binpar{D(x)}{P})}}}{D(x)} \nonumber
\end{eqnarray}

\begin{remark}
  Note that the lazier definition still does not deal with summation
  or mixed summation (i.e. sums over input and output). The reader is
  invited to construct definitions of replication that deal with these
  features. 

  Further, the definitions are parameterized in a name, $x$. Can you,
  gentle reader, make a definition that eliminates this parameter and
  guarantees no accidental interaction between the replication
  machinery and the process being replicated -- i.e. no accidental
  sharing of names used by the process to get its work done and the
  name(s) used by the replication to effect copying. This latter
  revision of the definition of replication is crucial to obtaining
  the expected identity $!!P \sim !P$.
\end{remark}

\begin{remark}\label{rem:paradoxical_combinator}
  The reader familiar with the lambda calculus will have noticed the
  similarity between $D$ and the paradoxical combinator.

  [Ed. note: the existence of this seems to suggest we have to be more
  restrictive on the set of processes and names we admit if we are to
  support no-cloning.]
\end{remark}

\subsubsection{Bisimulation}

The computational dynamics gives rise to another kind of equivalence,
the equivalence of computational behavior. As previously mentioned
this is typically captured \emph{via} some form of bisimulation.

% The notion we use in this paper is weak barbed bisimulation
% \cite{milner91polyadicpi}.

The notion we use in this paper is derived from weak barbed
bisimulation \cite{milner91polyadicpi}. 

\begin{definition}
An \emph{observation relation}, $\downarrow_{\mathcal N}$, over a set
of names, $\mathcal N$, is the smallest relation satisfying the rules
below.

\infrule[Out-barb]{y \in {\mathcal N}, \; x \nameeq y}
		  {\outputp{x}{v} \downarrow_{\mathcal N} x}
\infrule[Par-barb]{\mbox{$P\downarrow_{\mathcal N} x$ or $Q\downarrow_{\mathcal N} x$}}
		  {\binpar{P}{Q} \downarrow_{\mathcal N} x}

We write $P \Downarrow_{\mathcal N} x$ if there is $Q$ such that 
$P \wred Q$ and $Q \downarrow_{\mathcal N} x$.
\end{definition}

\begin{definition}
%\label{def.bbisim}
An  ${\mathcal N}$-\emph{barbed bisimulation} over a set of names, ${\mathcal N}$, is a symmetric binary relation 
${\mathcal S}_{\mathcal N}$ between agents such that $P\rel{S}_{\mathcal N}Q$ implies:
\begin{enumerate}
\item If $P \red P'$ then $Q \wred Q'$ and $P'\rel{S}_{\mathcal N} Q'$.
\item If $P\downarrow_{\mathcal N} x$, then $Q\Downarrow_{\mathcal N} x$.
\end{enumerate}
$P$ is ${\mathcal N}$-barbed bisimilar to $Q$, written
$P \wbbisim_{\mathcal N} Q$, if $P \rel{S}_{\mathcal N} Q$ for some ${\mathcal N}$-barbed bisimulation ${\mathcal S}_{\mathcal N}$.
\end{definition}

$\mathcal{R} \subseteq \pi \times \pi$

$P \mathcal{R} Q => \forall P'. P \red P' \Rightarrow \exists Q'. Q \red Q', P' \mathcal{R} Q'$

$P \vdash x \Rightarrow Q \vdash x$

\begin{mathpar}
  \inferrule*[lab=Out-barb]{x \nameeq y}{{y}!\langle{Q}\rangle \vdash x}
  \and
  \inferrule*[lab=Par-barb]{\mbox{$P\vdash x$ or $Q\vdash x$}}{\binpar{P}{Q} \vdash x}
\end{mathpar}

\subsubsection{Contexts}

One of the principle advantages of computational calculi like the
$\pi$-calculus is a well-defined notion of context,
contextual-equivalence and a correlation between
contextual-equivalence and notions of bisimulation. The notion of
context allows the decomposition of a process into (sub-)process and
its syntactic environment, its context. Thus, a context may be
thought of as a process with a ``hole'' (written $\Box$) in it. The
application of a context $M$ to a process $P$, written $M[P]$, is
tantamount to filling the hole in $M$ with $P$. In this paper we do
not need the full weight of this theory, but do make use of the notion
of context in the proof the main theorem. 

\begin{mathpar}
  \inferrule* [lab=summation] {} {{M_{M},M_{N}} \bc \Box \;|\; x.M_{A} \;|\; M_{M}+M_{N}}
  \and
  \inferrule* [lab=agent] {} {{M_{A}} \bc (\vec{x})M_{P} \;| \; \clift{P_0,\ldots,M_{P},\ldots,P_N}}
  \and \\
  \inferrule* [lab=process] {} {{M_{P}} \bc M_{N} \;| \;P|M_{P} }
\end{mathpar} 

\begin{mathpar}
  \inferrule* [lab=sychronization] {} {M_{N} \bc \Box \;|\; x?M_{F} \;|\; x!M_{C}}
  \and
  \inferrule* [lab=abstraction] {} {{M_{F}} \bc (x)M_{P} }
  \and
  \inferrule* [lab=concretion] {} {{M_{C}} \bc \langle M_{P} \rangle }
  \and \\
  \inferrule* [lab=process] {} {{M_{P}} \bc M_{N} \;| \;P|M_{P} }
\end{mathpar}

\begin{definition}[contextual application] Given a context $M$, and
  process $P$, we define the \emph{contextual application}, $M[P] :=
  M\{P/\Box\}$. That is, the contextual application of M to P is the
  substitution of $P$ for $\Box$ in $M$.
\end{definition}

$\meaningof{-} : L \to \mathcal{P}(\pi)$

\begin{mathpar}
  \inferrule* [lab=collection] {} {\meaningof{true} = \pi, \and \meaningof{~E} = \pi \setminus \meaningof{E}, \and \meaningof{E_{1} \& E_{2}} = \meaningof{E_{1}} \cap \meaningof{E_{2}}}
\end{mathpar}

\begin{mathpar}
  \inferrule* [lab=structure] {} {\meaningof{0} = \{ P \in \pi | P \equiv 0 \}, \and \\ \meaningof{E_1 | E_2} = \{ P \in \pi | P \equiv P_{1} | P_{2}, P_{1} \in \meaningof{E_{1}}, P_{2} \in \meaningof{E_2}\} }
\end{mathpar}

\begin{mathpar}
 \inferrule* [lab=behavior] {} {\meaningof{\langle a?b \rangle E} = \{ P \in \pi | P \equiv Q | u?(y)P', \\ \and \\\\ \and \\ \;\;\; u \in \meaningof{a}, \forall z.P'\{z/y\} \in \meaningof{E\{z/b\}}\}, \and \\ \meaningof{a!E} = \{ P \in \pi | P \equiv Q | x!\langle P' \rangle, x \in \meaningof{a} P' \in \meaningof{E}\} }
\end{mathpar}

\begin{mathpar}
 \inferrule* [lab=nominal] {} {\meaningof{\quotep{E}} = \{ \quotep{P} \in \quotep{\pi} | P \in \meaningof{E} \}, \and \meaningof{\quotep{P}} = \{ \quotep{Q} \in \quotep{\pi} | P \equiv Q \} \and \\ \meaningof{@\quotep{E}} = \{ P \in \pi | P \equiv @x, x \in \meaningof{E} \}}
\end{mathpar}

\begin{eqnarray*}
  \\
  \meaningof{-} : TS \to ST
\end{eqnarray*}

\begin{eqnarray*}
  \\
  L : TS \to ST
\end{eqnarray*}

\begin{eqnarray*}
  \\
  P \models E \iff P \in \meaningof{E}
\end{eqnarray*}

\begin{eqnarray*}
  P \approx_{L} Q \iff \forall E \in L. P \models E \iff Q \models E
\end{eqnarray*}

\begin{eqnarray*}
  P \approx_{K} Q
\end{eqnarray*}

\begin{eqnarray*}
  P \approx Q
\end{eqnarray*}

$\approx_{K} = \approx = \approx_{L}$

\subsubsection{Contextual duality}

Note that contexts extend the quotation operation to a family of
operations from processes to names. Given a context, $M$, we can
define a \emph{nominal context}, $\quotep{M}$ by $\quotep{M}[P] :=
\quotep{M[P]}$. To foreshadow what is to come we observe that these
operations enjoy a duality with processes very much like the duality
between vectors and maps from vectors to scalars.

Further, because the calculus is essentially higher-order, we have a
correspondence between contexts and processes. More specifically,
given a name $x$ and a context $M$ we can construct $M^{*}_{x}$ such
that 

\begin{mathpar}
  M^{*}_{x} | \lift{x}{P} \red M[P]
\end{mathpar}

namely,

\begin{mathpar}
  M^{*}_{x} := x?(u).M[\dropn{u}]
\end{mathpar}

The dependence of $M^{*}_{x}$ on a name makes it an abstraction, 

\begin{mathpar}
  M^{*} := (x)x?(u).M[\dropn{u}]
\end{mathpar}

\subsection{Additional notation}

It will sometimes be convenient to denote the process a name
quotes. We already have the notation $x = \quotep{P}$, but it will be
convenient to introduce an alternate notation, $\procn{x}$, when we
want to emphasize the connection to the use of the name. Note that, by
virtue of name equivalence, $\quotep{\procn{x}} \nameeq x$; so, the
notation is consistent with previous definitions.

Further, because names have structure it is possible to effect
substitutions on the basis of that structure. This means we need to
upgrade our notation for substitutions, which we accomplish by
adapting comprehension notation. Thus,

\begin{mathpar}
  P\{ y / x : x \in S \}
\end{mathpar}

is interpreted to mean the process derived from P by replacing (in a
capture-avoiding manner) each occurrence of $x$ in $S$ by $y$. For example,

\begin{mathpar}
  P\{ \quotep{\procn{x}|\procn{x}} / x : x \in \freenames{P} \}
\end{mathpar}

will replace each (occurrence) of a free name $x$ in $P$ by
$\quotep{\procn{x}|\procn{x}}$.

Also, we will avail ourselves of the notation $x^{L}$ and $x^{R}$ to
denote injections of a name into disjoint copies of the name
space. There are numerous ways to accomplish this. One example can be
found in \cite{MeredithR05}. This notation overloads to vectors of
names: $\vec{x}^{\pi} := (x_{i}^{\pi} \; : \; 0 \leq i < |\vec{x}| )$ where $\pi \in \{L,R\}$.

We also use $P^{\Box} := P|\Box$.

In \cite{MeredithR05} an interpretation of the new operator is
given. It turns out that there are several possible interpretations
all enjoying the requisite algebraic properties of the operator (see
\cite{milner91polyadicpi}). We will therefore make liberal use of
$(\nu\; \vec{x})P$.

% subsection the_syntax_and_semantics_of_the_notation_system (end)   

\section{Interpretation of QM}
\subsection{Supporting definitions}
\subsubsection{Multiplication}
\begin{mathpar}
  \quotep{Q} \cdot \quotep{R} := \quotep{Q|R}
  \and \\
  \quotep{Q} \cdot P := P\{ \quotep{Q|R} / \quotep{R} : \quotep{R} \in \freenames{P} \}
\end{mathpar}

\paragraph{Discussion}
The first line needs little explanation. The second line says that
each free name of the process is replaced with the multiplication of
that name by the scalar. Multiplication of a scalar (name) by a state
(process) results in a process all the names of which have been `moved
over' by parallel composition with the process the scalar
quotes. There is a subtlety that the bound names have to be
manipulated so that multiplied names aren't accidentally
captured. There are many ways to achieve this.

\begin{remark}\label{rem:multiplication_identities}
  The reader is invited to verify that for all $x,y,z \in \QProc$ and $P \in \Proc$
  \begin{mathpar}
    x \cdot \quotep{0} \equiv x 
    \and
    x \cdot y \equiv y \cdot x
    \and
    x \cdot (y \cdot z) \equiv (x \cdot y) \cdot z
    \and \\
    \quotep{0} \cdot P \equiv P
    \and \\
    x \cdot (y \cdot P) \equiv (x \cdot y) \cdot P
    \and \\
    x \cdot (P|Q) \equiv (x \cdot P) | (x \cdot Q)
    \and \\    
  \end{mathpar}
\end{remark}

\subsubsection{Tensor product}

We define a tensor product on processes by structural induction.

\paragraph{Tensor of sums} First note that all summations, including
$\pzero$ and sequence, can be written $\Sigma_{i} x_{i}.A_{i} +
\Sigma_{j} x_{j}.C_{j}$, where we have grouped input-guarded processes
together and output-guarded processes together.

Thus, we can define the tensor product of two summations, $N_{1}\otimes N_{2}$, where

\begin{mathpar}
  N_{1} := \Sigma_{i} x_{i}.A_{i} + \Sigma_{j} x_{j}.C_{j}
  \and
  N_{2} := \Sigma_{i'} y_{i'}.B_{i'} + \Sigma_{j'} y_{j'}.D_{j'} 
\end{mathpar}

as follows.

\begin{mathpar}
  \Sigma_{i} x_{i}.A_{i} + \Sigma_{j} x_{j}.C_{j} \otimes \Sigma_{i'}
  y_{i'}.B_{i'} + \Sigma_{j'} y_{j'}.D_{j'} 
  \and \\
  := \; \Sigma_{i} \Sigma_{i'} \quotep{\stackrel{\vee}{x_{i}}| \stackrel{\vee}{y_{i'}}}.(A_{i}\otimes B_{i'}) \; | \; \Sigma_{i'} \Sigma_{i} \quotep{\stackrel{\vee}{y_{i'}}|\stackrel{\vee}{x_{i}}}.(B_{i'}\otimes A_{i})
  \and
  \;\; | \;\; \Sigma_{j} \Sigma_{j'} \quotep{\stackrel{\vee}{x_{j}}|\stackrel{\vee}{y_{j'}}}.(A_{j}\otimes B_{j'}) \; | \; \Sigma_{j'} \Sigma_{j} \quotep{\stackrel{\vee}{y_{j'}}|\stackrel{\vee}{x_{j}}}.(B_{j'}\otimes A_{j})
\end{mathpar}

\begin{remark}
  Do we need to $x^{L}$ and $y^{R}$ for this construction as well?
\end{remark}

\paragraph{Tensor of parallel compositions} Next, we distribute tensor
over par.

\begin{mathpar}
  P_{1}|P_{2} \otimes Q_{1}|Q_{2} := (P_{1} \otimes Q_{1}) | (P_{1}
  \otimes Q_{2}) | (P_{2} \otimes Q_{1}) | (P_{2} \otimes Q_{2})
\end{mathpar}

\paragraph{Tensor with dropped names} We treat tensor of a
process with a dropped name as parallel composition.

\begin{mathpar}
  P \otimes \dropn{x} := P | \dropn{x}
\end{mathpar}

\paragraph{Tensor of agents}

Finally, we need to define tensor on agents. Note that the definition
of tensor on normal products only tensors inputs with inputs and
outputs with outputs. Thus, we only have to define the operation on
``homogeneous'' pairings.

\begin{mathpar}
  (\vec{x})P \otimes (\vec{y})Q
  \and \\
  := (x_{0}^{L}|y_{0}^{R},\ldots,x_{0}^{L}|y_{n}^{R},\ldots,x_{m}^{L}|y_{0}^{R},\ldots,x_{m}^{L}|y_{n}^R)(P\{ \vec{x}^{L}/\vec{x}\} \otimes Q \{ \vec{y}^{R}/\vec{y}\})
  \and \\
  \clift{\vec{P}} \otimes \clift{\vec{Q}}
  \and \\
  := \clift{P_{0}\otimes Q_{0},\ldots,P_{0}\otimes Q_{n},\ldots,P_{m}\otimes Q_{0},\ldots,P_{m}\otimes Q_{n}}
\end{mathpar}

\begin{remark}
  Observe that arities of tensored abstractions matches arities of
  tensored concretions if the original arities matched. Note also that
  the length of the arities corresponds to the increase in dimension
  we see in ordinary vector space tensor product.
\end{remark}

\begin{remark}
  Operationally, this definition distributes the tensor down to
  components ``linked'' by summation. Tensor over summation is
  intriguing in that it mixes names. Moreover, as a consequence of the
  way it mixes names we have the identities for all $x \in \QProc$ and
  $P,Q \in \Proc$

  \begin{mathpar}
    (x \cdot P) \otimes Q \equiv x \cdot (P \otimes Q) \equiv P \otimes (x \cdot Q)
    \and
    P \otimes \pzero \equiv P
  \end{mathpar}

  that the reader is invited to verify.
\end{remark}

\subsubsection{Annihilation}
\begin{mathpar}
  P^{\perp} := \{ Q | \forall R. P|Q \red^{*} R \Rightarrow R \red^{*} \pzero \}
  \and \\
  P^{\underline{\perp}} := \Sigma_{Q \in P^{\perp}} \quotep{Q}?(y).(\dropn{y}|Q) | \Sigma_{Q \in P^{\perp}} \quotep{Q}\clift{\Box}
\end{mathpar}

\paragraph{Discussion} The reader will note that $P^{\perp}$ is a
\emph{set} of processes, while $P^{\underline{\perp}}$ is a
\emph{context}. We call the set $P^{\perp}$ the \emph{annihilators} of
$P$. The parallel composition of a process in the annihilators of $P$
with $P$ will result in a process, the state space of which has all
paths eventually leading to $\pzero$. Execution may endure loops; but
under reasonable conditions of fairness (naturally guaranteed under
most notions of bisimulation) such a composite process cannot get
stuck in such a loop and will, eventually pop out and terminate.

The context $P^{\underline{\perp}}$ is ready and willing to ``take the
$P$ out of'' the process to which it is applied. It will effectively
transmit the code of the process to which it is applied to one of the
annihilators and run the process against it.

\subsubsection{Evaluation}
We fix $M$ a domain of fully abstract interpretation with an equality
coincident with bisimulation. We take $\meaningof{\cdot} : \Proc \to
M$ to be the map interpreting processes and $\nmeaningof{\cdot} : \M
\to Proc$ to be the map running the other way. Then we define

\begin{mathpar}
  \int P := \nmeaningof{\meaningof{P}}
\end{mathpar}

\paragraph{Discussion}
There are many fully abstract interpretations of Milner's
$\pi$-calculus. Any of them can be used as a basis for interpreting
the reflective calculus here. Equipped with such a domain it is
largely a matter of grinding through to check that the Yoneda
construction for the normalization-by-evaluation program can be
extended to this setting.

\begin{remark}
  The reader is invited to verify that $\int (P^{\underline{\perp}}[P]) = 0$.
\end{remark}

\subsection{Quantum mechanics}

Table \ref{tbl:core_qm_op_defns} gives the core operational definitions

\begin{table}[htp]\label{tbl:core_qm_op_defns}
  \center{
    \fbox{
      \begin{tabular}{c|c}
        quantum mechanics & process calculus \\
        \hline
        scalar & $x := \quotep{P}$ \\
        state vector & $\state{P} := P$ \\
        dual & $\state{P}^{*} := \event{P^{\underline{\perp}}} := \quotep{P^{\underline{\perp}}}[-]$ \\
        matrix & $ \Sigma_{\alpha} \state{P_{\alpha}}x_{\alpha}\event{Q_{\alpha}}$ \\
        vector addition & $\state{P} + \state{Q} := \state{P | Q}$ \\
        tensor product & $\state{P} \otimes \state{Q} := \state{P \otimes Q}$ \\
        inner product & $\innerprod{P}{Q} := \quotep{\int P^{\underline{\perp}}[Q]}$ \\
      \end{tabular}
    }
  }
  \caption{QM - operational definitions}
\end{table}

where

\begin{mathpar}
  \prmatrix{P}{Q} := \fprmatrix{P}{\quotep{\pzero}}{Q}
  \and
  \fprmatrix{P}{x}{Q} := (\state{P},x,\event{Q})
  \and
  (\fprmatrix{P}{x}{Q})(\state{R}) := x \cdot \innerprod{Q}{R} \cdot \state{P}
  \and
  (\fprmatrix{P}{x}{Q})(\event{R}) := x \cdot \innerprod{R}{P} \cdot \event{Q}
\end{mathpar}

\paragraph{Discussion}
As promised: vectors (aka states) are represented as processes; duals
as contextual duals; inner product definition should be compared with
standard inner product definition for ....

\begin{remark}
  Assuming $\int (P^{\underline{\perp}}[P]) = 0$, the reader is
  invited to verify that $(\fprmatrix{P}{x}{P})(\state{P}) = x \cdot \state{P}$.
\end{remark}

\begin{remark}
  The reader is invited to verify that $\innerprod{P}{Q}$ could
  equally well have been written $\quotep{\int \stackrel{\vee}{x}}$
  where $x = \event{P^{\underline{\perp}}}(Q)$.

  One of the motivations for this remark is that there is another way
  to factor these operations. We could package up evaluation in the dual:

  \begin{mathpar}
    \state{P}^{*} := \event{\int P^{\underline{\perp}}} := \quotep{\int P^{\underline{\perp}}}[-]
  \end{mathpar}

  and then have inner product defined by
  
  \begin{mathpar}
    \innerprod{P}{Q} := \event{P}(Q)
  \end{mathpar}

  Hopefully, experience with the calculations will provide guidance on
  the best factoring.
\end{remark}

\begin{remark}
  Assuming $\int (P^{\underline{\perp}}[P]) = 0$, the reader is
  invited to verify that $\forall P,Q. (\prmatrix{0}{Q})(\state{0}) =
  \state{0}$ and dually $(\prmatrix{P}{0})(\event{0}) = \event{0}$.
\end{remark}

\begin{remark}
  i'm a little worried that i don't (yet) have proper support for
  complex conjugacy. But, the observation above may give us a
  clue. According to Abramsky, it must be the case that the scalars
  are iso to the homset of the identity for the tensor -- which the
  observation above characterizes. 

  For now, we will simply bookmark the notion with $\overline{x}$.
\end{remark}

\subsubsection{Adjointness}

We need to give a definition of $(\cdot)^{\dagger}$ for matrices. The
obvious candidate definition is
\begin{mathpar}
(\Sigma_{\alpha}\fprmatrix{P_{\alpha}}{x_{\alpha}}{Q_{\alpha}})^{\dagger}
= \Sigma_{\alpha}\fprmatrix{(Q_{\alpha}^{\underline{\perp}})^{*}}{\overline{x}_{\alpha}}{P_{\alpha}^{\underline{\perp}}} 
\end{mathpar}

But, $(Q_{\alpha}^{\underline{\perp}})^{*}$ requires a name along
which to communicate the process to achieve the context application.

\subsubsection{Basis for a basis}
If processes label states and ``addition'' of states (a.k.a. vector
addition) is interpreted as parallel composition, what corresponds to
notions of linear independence and basis? Here, we recall that Yoshida
has developed a set of \emph{combinators} for an asynchronous verison
of Milner's $\pi$-calculus. These are a finite set of processes such
any process can be expressed as parallel composition of these
combinators together with liberal uses of the new operator and
replication. We can simply give a translation of these into the
present calculus and have reasonable expectation that the property
carries over. That is, that the resultant set allows to express all
processes via parallel composition. Note, however, that there is no
new operator or replication in this calculus. As a result, we expect
that the corresponding set is actually infinite. That is, we expect
that the space is actually infinite dimensional.

\begin{remark}
  The attentive reader may be a bit concerned. Certainly, the
  collection $S$, $K$ and $I$ is a finite set of
  combinators. Shouldn't we expect to see a finite set of combinators
  for an effectively equivalent system? i am very sympathetic to this
  critique and feel it warrants full attention. On the other hand, i
  also have in mind the following analogy. The natural numbers, as a
  monoid under addition, has exactly $1$ generator, while the natural
  numbers, as a monoid under multiplication, has countably many
  generators (the primes). We observe that the application of the
  lambda calculus is much less resource sensitive than the parallel
  composition of the $\pi$-calculus. Could it be the case that we have
  an analogy of the form
  
  \begin{mathpar}
    m + n : MN :: m*n : M|N
  \end{mathpar}

  giving a similar blow up in the set of ``primes''?  This is such a
  wonderful thought that, even if it's not true, i think it's worth
  writing down.
\end{remark}
 

\documentclass[12pt]{llncs}
%\documentclass{jktr}

\usepackage[pdftex]{hyperref}                   
\usepackage {listings}
\usepackage {mathpartir}
\usepackage{bcprules}
%\usepackage{listings}
                       
\usepackage{graphicx} 
%\usepackage[margins=2.5cm,nohead,nofoot]{geometry}
%\usepackage{geometry}
\usepackage{amsfonts}
\usepackage{amstext}
\usepackage{latexsym}
\usepackage{amssymb}
\usepackage{color}


%\include{myPreamble}
\include{qm2pi.local} 

%\ifpdf
%\usepackage[pdftex]{graphicx}
%\else
%\usepackage{graphicx}
%\fi

 % \ifpdf
%  \usepackage{pdfsync}
%  \if


%\title{Brief Article}
%\author{David F. Snyder}
%\author{L.G. Meredith}

%\address{Dept. of Math., Texas State University--San Marcos, San Marcos, TX 78666}
       
\pagestyle{empty}


\begin{document}

\lstset{language=[Objective]Caml,frame=shadowbox}

\input{qm2pi.front}

% section front matter (end)

\input{qm2pi.intro} 
 
% section introduction (end)

% \input{qm2pi.knotations} 

% section notation (end)

\input{qm2pi.process.calculi} 

% section concurrent_process_calculi_and_spatial_logics_ (end)
    
%\input{qm2pi.knots2pi} 

%\input{qm2pi.trefoil} 

%\input{qm2pi.mainthm} 

% subsection basic_interpretation (end)

%\input{qm2pi.rho.presentation} 
\subsection{The syntax and semantics of the notation system}\label{sub:the_syntax_and_semantics_of_the_notation_system} % (fold)

We now summarize a technical presentation of the calculus that
embodies our theory of dynamics. The typical presentation of such a
calculus follows the style of giving generators and relations on
them. The grammar, below, describing term constructors, freely
generates the set of processes, $\Proc$. This set is then quotiented
by a relation known as structural congruence and it is over this set
that the notion of dynamics is expressed. This presentation is
essentially that of \cite{MeredithR05} with the addition of
polyadicity and summation. For readability we have relegated some of
the technical subtleties to an appendix.

\subsubsection{Process grammar}\label{subsub:process_grammar}

\begin{mathpar}
  \inferrule* [lab=synchronization] {} {{M} \bc \pzero \;|\; x?F \;|\; x!C }
  \and
  \inferrule* [lab=abstraction] {} {{F} \bc (x)P}
  \and
  \inferrule* [lab=concretion] {} {{C} \bc \langle Q \rangle}
  \and
  \inferrule* [lab=process] {} {{P,Q} \bc M \;| \;P|Q \;|\; @{x}}
  \and
  \inferrule* [lab=name] {} {{x} \bc \quotep{P}}
\end{mathpar} 

Note that $\vec{x}$ (resp. $\vec{P}$) denotes a vector of names
(resp. processes) of length $|\vec{x}|$ (resp. $|\vec{P}|$). We adopt
the following useful abbreviations.

\begin{mathpar}
   x?(\vec{y}).P := x.(\vec{y})P \and  x\clift{\vec{P}} := x.\clift{\vec{P}}
   \and x!(y) := \lift{x}{\dropn{y}}
   \and \Pi_{i=0}^{n-1}P_i := P_0 | \ldots | P_{n-1}
\end{mathpar}

\subsubsection{Structural congruence}

\paragraph{Free and bound names and alpha-equivalence.} At the
core of structural equivalence is alpha-equivalence which identifies
process that are the same up to a change of variable. Formally, we
recognize the distinction between free and bound names. The free names
of a process, $\freenames{P}$, may be calculated recursively as
follows:

\begin{mathpar}
\freenames{\pzero} := \emptyset
  \and \\
  \freenames{x?(y).P} := \{ x \} \cup (\freenames{P} \setminus \{ y \})
  \and 
  \freenames{x!\langle P \rangle} := \{ x \} \cup \{ P \} 
  \and \\
  \freenames{P|Q} := \freenames{P} \cup \freenames{Q}
  \and \\
  \freenames{@{x}} := \{ x \}
\end{mathpar}

$\pi$
$\quotep{\pi}$

$\freenames{-} : \pi \to \mathcal{P}(\quotep{\pi})$

\begin{eqnarray*}
  \freenames{\pzero} & := & \emptyset \\
  \freenames{x?(y).P} & := & \{ x \} \cup (\freenames{P} \setminus \{ y \}) \\
  \freenames{x!\langle P \rangle} & := & \{ x \} \cup \{ P \} \\
  \freenames{P|Q} & := & \freenames{P} \cup \freenames{Q} \\
  \freenames{\dropn{x}} & := & \{ x \}
\end{eqnarray*}

The bound names of a process, $\boundnames{P}$, are those names occurring in $P$
that are not free. For example, in $x?(y).0$, the name $x$ is free, while $y$ is bound.

\begin{mathpar}
  \inferrule* [lab=monoidal-laws] {} { P|Q \equiv Q|P \and P|0 \equiv P \and P|(Q|R) \equiv (P|Q)|R }
\end{mathpar}

\begin{mathpar}
  \inferrule* [lab=alpha-equivalence] {} { (x)P \equiv (y)P\{y/x\} \and y \not\in \freenames{P} }
\end{mathpar}

\begin{definition}
Then two processes, $P,Q$, are alpha-equivalent if $P = Q\{\vec{y}/\vec{x}\}$ for
some $\vec{x} \in \boundnames{Q},\vec{y} \in \boundnames{P}$, where $Q\{\vec{y}/\vec{x}\}$
denotes the capture-avoiding substitution of $\vec{y}$ for $\vec{x}$ in $Q$.
\end{definition}

\begin{definition}
  The {\em structural congruence} \cite{SangiorgiWalker} , $\equiv$,
  between processes is the least congruence containing
  alpha-equivalence, satisfying the abelian monoid laws
  (associativity, commutativity and $\pzero$ as identity) for parallel
  composition $|$ and for summation $+$.
\end{definition}

\subsection{Name equivalence}

We take name equivalence, written $\nameeq$, to be the smallest
equivalence relation generated by the following rules.

\begin{mathpar}
\inferrule*[lab=Quote-drop]
{ }
{ \quotep{@{x}} \nameeq x }

\inferrule*[lab=Struct-equiv]
{ P \scong Q }
{ \quotep{P} \nameeq \quotep{Q} }
\end{mathpar}

The astute reader will have noticed that the mutual recursion of names
and processes imposes a mutual recursion on alpha-equivalence and
structural equivalence via name-equivalence. Fortunately, all of this
works out pleasantly and we may calculate in the natural way, free of
concern. The reader interested in the details is referred to the
appendix \ref{appendix:rho_details}.

\subsection{Substitution}

We use $\Proc$ for the set of processes, $\QProc$ for the set of
names, and $\id{\{}\vec{y} / \vec{x} \id{\}}$ to denote partial maps,
$s : \QProc \rightarrow \QProc$. A map, $s$ lifts, uniquely, to a map
on process terms, $\widehat{s} : \Proc \rightarrow \Proc$ by the
following equations.

\begin{mathpar}
  (0) \psubstp{Q}{P} := 0 \\
  (R \juxtap S) \psubstp{Q}{P}
  :=    
  (R)\psubstp{Q}{P} \juxtap (S) \psubstp{Q}{P} \\
  (x?(y).R) \psubstp{Q}{P}    
  :=    
  (x)\substp{Q}{P} (z)\concat( (R \psubstn{z}{y}) \psubstp{Q}{P} ) \\
  (\lift{x}{R}) \psubstp{Q}{P}  
  :=
  \lift{(x)\substp{Q}{P}}{ R \psubstp{Q}{P} } \\
%   (\dropn{x})  \psubstp{Q}{P}       
%   := 
%   \left\{ 
%     \begin{array}{ccc} 
%       \dropn{\quotep{Q}} & & x \nameeq \quotep{P} \\
%       \dropn{x} & & otherwise \\
%     \end{array}
%   \right. 
  (\dropn{x})  \psubstp{Q}{P}       
  := 
  \left\{ 
    \begin{array}{ccc} 
      Q & & x \nameeq \quotep{P} \\
      \dropn{x} & & otherwise \\
    \end{array}
  \right.
\end{mathpar}
 

where

\begin{eqnarray}
  (x)\id{\{} \lpquote Q \rpquote / \lpquote P \rpquote \id{\}}            = 
  \left\{ 
    \begin{array}{ccc}
      \lpquote Q \rpquote & & x \nameeq \lpquote P \rpquote \\
      x & & otherwise \\
    \end{array}
  \right. \nonumber
\end{eqnarray}

and $z$ is chosen distinct from $\quotep{P}$, $\quotep{Q}$, the free
names in $Q$, and all the names in $R$. Our $\alpha$-equivalence will
be built in the standard way from this substitution.

\begin{remark}\label{rem:no_self_referential_names}
  One consequence of these definitions is that $\forall P. \quotep{P}
  \not\in \freenames{P}$.
\end{remark}

\subsection{ Dynamic quote: an example }

Anticipating something of what's to come, consider applying the
substitution, $\widehat{\id{\{}u / z \id{\}}}$, to the following pair
of processes, $\lift{w}{y!(z)}$ and $w[ \lpquote y!(z) \rpquote ]$.

\begin{eqnarray}
	\lift{w}{y!(z)}\widehat{\id{\{}u / z \id{\}}}
		& = &
		\lift{w}{y!(u)} \nonumber\\
	w[ \lpquote y!(z) \rpquote ] \widehat{ \id{\{}u / z \id{\}} }
		& = &
		w[ \lpquote y!(z) \rpquote ] \nonumber
\end{eqnarray}

Because the body of the process between quotes is impervious to
substitution, we get radically different answers. In fact, by
examining the first process in an input context,
e.g. $x?(z).\lift{w}{y!(z)}$, we see that the process under the lift
operator may be shaped by prefixed inputs binding a name inside it. In
this sense, the lift operator will be seen as a way to dynamically
construct processes before reifying them as names.

Finally equipped with these standard features we can present the
dynamics of the calculus.

\subsubsection{Operational semantics} 

Finally, we introduce the computational dynamics. What marks these
algebras as distinct from other more traditionally studied algebraic
structures, e.g. vector spaces or polynomial rings, is the manner in
which dynamics is captured. In traditional structures, dynamics is typically
expressed through morphisms between such structures, as in linear maps
between vector spaces or morphisms between rings. In algebras
associated with the semantics of computation, the dynamics is
expressed as part of the algebraic structure itself, through a
reduction reduction relation typically denoted by $\red$. Below, we
give a recursive presentation of this relation for the calculus used
in the encoding.

$\red \subseteq \pi \times \pi$
$\red : \pi \to \mathcal{P}(\pi)$

\begin{mathpar}
  \inferrule* [lab=Comm] { \textsf{match}( x_{src}, x_{trgt} ) } { x_{trgt}?(y)P \; | \; x_{src}!\langle {Q} \rangle \red P\{\quotep{Q}/y}\} }
  \and \\
  \inferrule* [lab=Par] {{P} \red {P}'} {{{P} | {Q}} \red {{P}' | {Q}}}
  \and
  \inferrule* [lab=Equiv]{{{P} \scong {P}'} \andalso {{P}' \red {Q}'} \andalso {{Q}' \scong {Q}}}{{P} \red {Q}}
\end{mathpar}

\begin{eqnarray*}
  match_{\equiv} (\quotep{P},\quotep{Q}) & := & P \equiv Q \\
  match_{\dagger}(\quotep{P},\quotep{Q}) & := & \forall R. P|Q \red^{*} R => R \red^{*} 0 \\
  match_{K}(\quotep{P},\quotep{Q}) & := & K \mbox{ for some context } K
\end{eqnarray*}

$u?(x)P | u!\langle Q \rangle \red P\{\quotep{Q}/x\}$

%We write $\wred$ for $\red^*$, and $P\red$ if $\exists Q $ such that $ P \red Q$.
We write $P\red$ if $\exists Q $ such that $ P \red Q$ and $P\not\red$, otherwise.

\section{Replication}

As mentioned before, it is known that replication (and hence
recursion) can be implemented in a higher-order process algebra
\cite{SangiorgiWalker}. As our first example of calculation with the
machinery thus far presented we give the construction explicitly in
the {\rhoc}.

\begin{eqnarray}
	D_{x} & := & \prefix{x}{y}{(\binpar{\outputp{x}{y}}{@{y}})} \nonumber\\
	\bangp_{x}{P} & := & \binpar{{x}!\langle{\binpar{D_{x}}{P}}\rangle}{D_{x}} \nonumber
\end{eqnarray}

\begin{eqnarray}
	\bangp_{x}{P} & & \nonumber\\
	=
	& {x}!\langle{(\prefix{x}{y}{(\outputp{x}{y} | @{y})) | P}}\rangle 
	      | \prefix{x}{y}{(\outputp{x}{y} | @{y})} & \nonumber\\
	\red
	& (\outputp{x}{y} | @{y})\substn{\quotep{(\prefix{x}{y}{(@{y} | \outputp{x}{y})) | P}}}{y} & \nonumber\\
	=
	& \outputp{x}{\quotep{(\prefix{x}{y}{(\outputp{x}{y} | @{y})) | P}}}
	  | {(\prefix{x}{y}{(\outputp{x}{y} | @{y})) | P}} & \nonumber\\
	\red
	& \ldots & \nonumber\\
	\red^*
	& P | P | \ldots & \nonumber
\end{eqnarray}

Of course, this encoding, as an implementation, runs away, unfolding
$\bangp{P}$ eagerly. A lazier and more implementable replication
operator, restricted to input-guarded processes, may be obtained as follows.

\begin{eqnarray}
\bangp{\prefix{u}{v}{P}} 
	:= 
	\binpar{\lift{x}{\prefix{u}{v}{(\binpar{D(x)}{P})}}}{D(x)} \nonumber
\end{eqnarray}

\begin{remark}
  Note that the lazier definition still does not deal with summation
  or mixed summation (i.e. sums over input and output). The reader is
  invited to construct definitions of replication that deal with these
  features. 

  Further, the definitions are parameterized in a name, $x$. Can you,
  gentle reader, make a definition that eliminates this parameter and
  guarantees no accidental interaction between the replication
  machinery and the process being replicated -- i.e. no accidental
  sharing of names used by the process to get its work done and the
  name(s) used by the replication to effect copying. This latter
  revision of the definition of replication is crucial to obtaining
  the expected identity $!!P \sim !P$.
\end{remark}

\begin{remark}\label{rem:paradoxical_combinator}
  The reader familiar with the lambda calculus will have noticed the
  similarity between $D$ and the paradoxical combinator.

  [Ed. note: the existence of this seems to suggest we have to be more
  restrictive on the set of processes and names we admit if we are to
  support no-cloning.]
\end{remark}

\subsubsection{Bisimulation}

The computational dynamics gives rise to another kind of equivalence,
the equivalence of computational behavior. As previously mentioned
this is typically captured \emph{via} some form of bisimulation.

% The notion we use in this paper is weak barbed bisimulation
% \cite{milner91polyadicpi}.

The notion we use in this paper is derived from weak barbed
bisimulation \cite{milner91polyadicpi}. 

\begin{definition}
An \emph{observation relation}, $\downarrow_{\mathcal N}$, over a set
of names, $\mathcal N$, is the smallest relation satisfying the rules
below.

\infrule[Out-barb]{y \in {\mathcal N}, \; x \nameeq y}
		  {\outputp{x}{v} \downarrow_{\mathcal N} x}
\infrule[Par-barb]{\mbox{$P\downarrow_{\mathcal N} x$ or $Q\downarrow_{\mathcal N} x$}}
		  {\binpar{P}{Q} \downarrow_{\mathcal N} x}

We write $P \Downarrow_{\mathcal N} x$ if there is $Q$ such that 
$P \wred Q$ and $Q \downarrow_{\mathcal N} x$.
\end{definition}

\begin{definition}
%\label{def.bbisim}
An  ${\mathcal N}$-\emph{barbed bisimulation} over a set of names, ${\mathcal N}$, is a symmetric binary relation 
${\mathcal S}_{\mathcal N}$ between agents such that $P\rel{S}_{\mathcal N}Q$ implies:
\begin{enumerate}
\item If $P \red P'$ then $Q \wred Q'$ and $P'\rel{S}_{\mathcal N} Q'$.
\item If $P\downarrow_{\mathcal N} x$, then $Q\Downarrow_{\mathcal N} x$.
\end{enumerate}
$P$ is ${\mathcal N}$-barbed bisimilar to $Q$, written
$P \wbbisim_{\mathcal N} Q$, if $P \rel{S}_{\mathcal N} Q$ for some ${\mathcal N}$-barbed bisimulation ${\mathcal S}_{\mathcal N}$.
\end{definition}

$\mathcal{R} \subseteq \pi \times \pi$

$P \mathcal{R} Q => \forall P'. P \red P' \Rightarrow \exists Q'. Q \red Q', P' \mathcal{R} Q'$

$P \vdash x \Rightarrow Q \vdash x$

\begin{mathpar}
  \inferrule*[lab=Out-barb]{x \nameeq y}{{y}!\langle{Q}\rangle \vdash x}
  \and
  \inferrule*[lab=Par-barb]{\mbox{$P\vdash x$ or $Q\vdash x$}}{\binpar{P}{Q} \vdash x}
\end{mathpar}

\subsubsection{Contexts}

One of the principle advantages of computational calculi like the
$\pi$-calculus is a well-defined notion of context,
contextual-equivalence and a correlation between
contextual-equivalence and notions of bisimulation. The notion of
context allows the decomposition of a process into (sub-)process and
its syntactic environment, its context. Thus, a context may be
thought of as a process with a ``hole'' (written $\Box$) in it. The
application of a context $M$ to a process $P$, written $M[P]$, is
tantamount to filling the hole in $M$ with $P$. In this paper we do
not need the full weight of this theory, but do make use of the notion
of context in the proof the main theorem. 

\begin{mathpar}
  \inferrule* [lab=summation] {} {{M_{M},M_{N}} \bc \Box \;|\; x.M_{A} \;|\; M_{M}+M_{N}}
  \and
  \inferrule* [lab=agent] {} {{M_{A}} \bc (\vec{x})M_{P} \;| \; \clift{P_0,\ldots,M_{P},\ldots,P_N}}
  \and \\
  \inferrule* [lab=process] {} {{M_{P}} \bc M_{N} \;| \;P|M_{P} }
\end{mathpar} 

\begin{mathpar}
  \inferrule* [lab=sychronization] {} {M_{N} \bc \Box \;|\; x?M_{F} \;|\; x!M_{C}}
  \and
  \inferrule* [lab=abstraction] {} {{M_{F}} \bc (x)M_{P} }
  \and
  \inferrule* [lab=concretion] {} {{M_{C}} \bc \langle M_{P} \rangle }
  \and \\
  \inferrule* [lab=process] {} {{M_{P}} \bc M_{N} \;| \;P|M_{P} }
\end{mathpar}

\begin{definition}[contextual application] Given a context $M$, and
  process $P$, we define the \emph{contextual application}, $M[P] :=
  M\{P/\Box\}$. That is, the contextual application of M to P is the
  substitution of $P$ for $\Box$ in $M$.
\end{definition}

$\meaningof{-} : L \to \mathcal{P}(\pi)$

\begin{mathpar}
  \inferrule* [lab=collection] {} {\meaningof{true} = \pi, \and \meaningof{~E} = \pi \setminus \meaningof{E}, \and \meaningof{E_{1} \& E_{2}} = \meaningof{E_{1}} \cap \meaningof{E_{2}}}
\end{mathpar}

\begin{mathpar}
  \inferrule* [lab=structure] {} {\meaningof{0} = \{ P \in \pi | P \equiv 0 \}, \and \\ \meaningof{E_1 | E_2} = \{ P \in \pi | P \equiv P_{1} | P_{2}, P_{1} \in \meaningof{E_{1}}, P_{2} \in \meaningof{E_2}\} }
\end{mathpar}

\begin{mathpar}
 \inferrule* [lab=behavior] {} {\meaningof{\langle a?b \rangle E} = \{ P \in \pi | P \equiv Q | u?(y)P', \\ \and \\\\ \and \\ \;\;\; u \in \meaningof{a}, \forall z.P'\{z/y\} \in \meaningof{E\{z/b\}}\}, \and \\ \meaningof{a!E} = \{ P \in \pi | P \equiv Q | x!\langle P' \rangle, x \in \meaningof{a} P' \in \meaningof{E}\} }
\end{mathpar}

\begin{mathpar}
 \inferrule* [lab=nominal] {} {\meaningof{\quotep{E}} = \{ \quotep{P} \in \quotep{\pi} | P \in \meaningof{E} \}, \and \meaningof{\quotep{P}} = \{ \quotep{Q} \in \quotep{\pi} | P \equiv Q \} \and \\ \meaningof{@\quotep{E}} = \{ P \in \pi | P \equiv @x, x \in \meaningof{E} \}}
\end{mathpar}

\begin{eqnarray*}
  \\
  \meaningof{-} : TS \to ST
\end{eqnarray*}

\begin{eqnarray*}
  \\
  L : TS \to ST
\end{eqnarray*}

\begin{eqnarray*}
  \\
  P \models E \iff P \in \meaningof{E}
\end{eqnarray*}

\begin{eqnarray*}
  P \approx_{L} Q \iff \forall E \in L. P \models E \iff Q \models E
\end{eqnarray*}

\begin{eqnarray*}
  P \approx_{K} Q
\end{eqnarray*}

\begin{eqnarray*}
  P \approx Q
\end{eqnarray*}

$\approx_{K} = \approx = \approx_{L}$

\subsubsection{Contextual duality}

Note that contexts extend the quotation operation to a family of
operations from processes to names. Given a context, $M$, we can
define a \emph{nominal context}, $\quotep{M}$ by $\quotep{M}[P] :=
\quotep{M[P]}$. To foreshadow what is to come we observe that these
operations enjoy a duality with processes very much like the duality
between vectors and maps from vectors to scalars.

Further, because the calculus is essentially higher-order, we have a
correspondence between contexts and processes. More specifically,
given a name $x$ and a context $M$ we can construct $M^{*}_{x}$ such
that 

\begin{mathpar}
  M^{*}_{x} | \lift{x}{P} \red M[P]
\end{mathpar}

namely,

\begin{mathpar}
  M^{*}_{x} := x?(u).M[\dropn{u}]
\end{mathpar}

The dependence of $M^{*}_{x}$ on a name makes it an abstraction, 

\begin{mathpar}
  M^{*} := (x)x?(u).M[\dropn{u}]
\end{mathpar}

\subsection{Additional notation}

It will sometimes be convenient to denote the process a name
quotes. We already have the notation $x = \quotep{P}$, but it will be
convenient to introduce an alternate notation, $\procn{x}$, when we
want to emphasize the connection to the use of the name. Note that, by
virtue of name equivalence, $\quotep{\procn{x}} \nameeq x$; so, the
notation is consistent with previous definitions.

Further, because names have structure it is possible to effect
substitutions on the basis of that structure. This means we need to
upgrade our notation for substitutions, which we accomplish by
adapting comprehension notation. Thus,

\begin{mathpar}
  P\{ y / x : x \in S \}
\end{mathpar}

is interpreted to mean the process derived from P by replacing (in a
capture-avoiding manner) each occurrence of $x$ in $S$ by $y$. For example,

\begin{mathpar}
  P\{ \quotep{\procn{x}|\procn{x}} / x : x \in \freenames{P} \}
\end{mathpar}

will replace each (occurrence) of a free name $x$ in $P$ by
$\quotep{\procn{x}|\procn{x}}$.

Also, we will avail ourselves of the notation $x^{L}$ and $x^{R}$ to
denote injections of a name into disjoint copies of the name
space. There are numerous ways to accomplish this. One example can be
found in \cite{MeredithR05}. This notation overloads to vectors of
names: $\vec{x}^{\pi} := (x_{i}^{\pi} \; : \; 0 \leq i < |\vec{x}| )$ where $\pi \in \{L,R\}$.

We also use $P^{\Box} := P|\Box$.

In \cite{MeredithR05} an interpretation of the new operator is
given. It turns out that there are several possible interpretations
all enjoying the requisite algebraic properties of the operator (see
\cite{milner91polyadicpi}). We will therefore make liberal use of
$(\nu\; \vec{x})P$.

% subsection the_syntax_and_semantics_of_the_notation_system (end)   

\input{qm2pi.qmops} 

\input{qm2pi.sterngerlach} 

\input{qm2pi.metric} 

% section concurrent_process_calculi (end)

%\input{qm2pi.proofsketch}

% section proof sketch (end)

%\input{qm2pi.slviaknots} 

% section spatial logic via knots (end)

\input{qm2pi.conclusion}

% section conclusion (end)

%\input{qm2pi.dtcodes} 

% section wiring algorithm (end)

\input{qm2pi.ack} 

% section acknowledgments (end)

\newpage


\bibliographystyle{plain}   
\bibliography{../../biblios/main.bib}

\input{qm2pi.rhodetails}

\end{document}

 

\documentclass[12pt]{llncs}
%\documentclass{jktr}

\usepackage[pdftex]{hyperref}                   
\usepackage {listings}
\usepackage {mathpartir}
\usepackage{bcprules}
%\usepackage{listings}
                       
\usepackage{graphicx} 
%\usepackage[margins=2.5cm,nohead,nofoot]{geometry}
%\usepackage{geometry}
\usepackage{amsfonts}
\usepackage{amstext}
\usepackage{latexsym}
\usepackage{amssymb}
\usepackage{color}


%\include{myPreamble}
\include{qm2pi.local} 

%\ifpdf
%\usepackage[pdftex]{graphicx}
%\else
%\usepackage{graphicx}
%\fi

 % \ifpdf
%  \usepackage{pdfsync}
%  \if


%\title{Brief Article}
%\author{David F. Snyder}
%\author{L.G. Meredith}

%\address{Dept. of Math., Texas State University--San Marcos, San Marcos, TX 78666}
       
\pagestyle{empty}


\begin{document}

\lstset{language=[Objective]Caml,frame=shadowbox}

\input{qm2pi.front}

% section front matter (end)

\input{qm2pi.intro} 
 
% section introduction (end)

% \input{qm2pi.knotations} 

% section notation (end)

\input{qm2pi.process.calculi} 

% section concurrent_process_calculi_and_spatial_logics_ (end)
    
%\input{qm2pi.knots2pi} 

%\input{qm2pi.trefoil} 

%\input{qm2pi.mainthm} 

% subsection basic_interpretation (end)

%\input{qm2pi.rho.presentation} 
\subsection{The syntax and semantics of the notation system}\label{sub:the_syntax_and_semantics_of_the_notation_system} % (fold)

We now summarize a technical presentation of the calculus that
embodies our theory of dynamics. The typical presentation of such a
calculus follows the style of giving generators and relations on
them. The grammar, below, describing term constructors, freely
generates the set of processes, $\Proc$. This set is then quotiented
by a relation known as structural congruence and it is over this set
that the notion of dynamics is expressed. This presentation is
essentially that of \cite{MeredithR05} with the addition of
polyadicity and summation. For readability we have relegated some of
the technical subtleties to an appendix.

\subsubsection{Process grammar}\label{subsub:process_grammar}

\begin{mathpar}
  \inferrule* [lab=synchronization] {} {{M} \bc \pzero \;|\; x?F \;|\; x!C }
  \and
  \inferrule* [lab=abstraction] {} {{F} \bc (x)P}
  \and
  \inferrule* [lab=concretion] {} {{C} \bc \langle Q \rangle}
  \and
  \inferrule* [lab=process] {} {{P,Q} \bc M \;| \;P|Q \;|\; @{x}}
  \and
  \inferrule* [lab=name] {} {{x} \bc \quotep{P}}
\end{mathpar} 

Note that $\vec{x}$ (resp. $\vec{P}$) denotes a vector of names
(resp. processes) of length $|\vec{x}|$ (resp. $|\vec{P}|$). We adopt
the following useful abbreviations.

\begin{mathpar}
   x?(\vec{y}).P := x.(\vec{y})P \and  x\clift{\vec{P}} := x.\clift{\vec{P}}
   \and x!(y) := \lift{x}{\dropn{y}}
   \and \Pi_{i=0}^{n-1}P_i := P_0 | \ldots | P_{n-1}
\end{mathpar}

\subsubsection{Structural congruence}

\paragraph{Free and bound names and alpha-equivalence.} At the
core of structural equivalence is alpha-equivalence which identifies
process that are the same up to a change of variable. Formally, we
recognize the distinction between free and bound names. The free names
of a process, $\freenames{P}$, may be calculated recursively as
follows:

\begin{mathpar}
\freenames{\pzero} := \emptyset
  \and \\
  \freenames{x?(y).P} := \{ x \} \cup (\freenames{P} \setminus \{ y \})
  \and 
  \freenames{x!\langle P \rangle} := \{ x \} \cup \{ P \} 
  \and \\
  \freenames{P|Q} := \freenames{P} \cup \freenames{Q}
  \and \\
  \freenames{@{x}} := \{ x \}
\end{mathpar}

$\pi$
$\quotep{\pi}$

$\freenames{-} : \pi \to \mathcal{P}(\quotep{\pi})$

\begin{eqnarray*}
  \freenames{\pzero} & := & \emptyset \\
  \freenames{x?(y).P} & := & \{ x \} \cup (\freenames{P} \setminus \{ y \}) \\
  \freenames{x!\langle P \rangle} & := & \{ x \} \cup \{ P \} \\
  \freenames{P|Q} & := & \freenames{P} \cup \freenames{Q} \\
  \freenames{\dropn{x}} & := & \{ x \}
\end{eqnarray*}

The bound names of a process, $\boundnames{P}$, are those names occurring in $P$
that are not free. For example, in $x?(y).0$, the name $x$ is free, while $y$ is bound.

\begin{mathpar}
  \inferrule* [lab=monoidal-laws] {} { P|Q \equiv Q|P \and P|0 \equiv P \and P|(Q|R) \equiv (P|Q)|R }
\end{mathpar}

\begin{mathpar}
  \inferrule* [lab=alpha-equivalence] {} { (x)P \equiv (y)P\{y/x\} \and y \not\in \freenames{P} }
\end{mathpar}

\begin{definition}
Then two processes, $P,Q$, are alpha-equivalent if $P = Q\{\vec{y}/\vec{x}\}$ for
some $\vec{x} \in \boundnames{Q},\vec{y} \in \boundnames{P}$, where $Q\{\vec{y}/\vec{x}\}$
denotes the capture-avoiding substitution of $\vec{y}$ for $\vec{x}$ in $Q$.
\end{definition}

\begin{definition}
  The {\em structural congruence} \cite{SangiorgiWalker} , $\equiv$,
  between processes is the least congruence containing
  alpha-equivalence, satisfying the abelian monoid laws
  (associativity, commutativity and $\pzero$ as identity) for parallel
  composition $|$ and for summation $+$.
\end{definition}

\subsection{Name equivalence}

We take name equivalence, written $\nameeq$, to be the smallest
equivalence relation generated by the following rules.

\begin{mathpar}
\inferrule*[lab=Quote-drop]
{ }
{ \quotep{@{x}} \nameeq x }

\inferrule*[lab=Struct-equiv]
{ P \scong Q }
{ \quotep{P} \nameeq \quotep{Q} }
\end{mathpar}

The astute reader will have noticed that the mutual recursion of names
and processes imposes a mutual recursion on alpha-equivalence and
structural equivalence via name-equivalence. Fortunately, all of this
works out pleasantly and we may calculate in the natural way, free of
concern. The reader interested in the details is referred to the
appendix \ref{appendix:rho_details}.

\subsection{Substitution}

We use $\Proc$ for the set of processes, $\QProc$ for the set of
names, and $\id{\{}\vec{y} / \vec{x} \id{\}}$ to denote partial maps,
$s : \QProc \rightarrow \QProc$. A map, $s$ lifts, uniquely, to a map
on process terms, $\widehat{s} : \Proc \rightarrow \Proc$ by the
following equations.

\begin{mathpar}
  (0) \psubstp{Q}{P} := 0 \\
  (R \juxtap S) \psubstp{Q}{P}
  :=    
  (R)\psubstp{Q}{P} \juxtap (S) \psubstp{Q}{P} \\
  (x?(y).R) \psubstp{Q}{P}    
  :=    
  (x)\substp{Q}{P} (z)\concat( (R \psubstn{z}{y}) \psubstp{Q}{P} ) \\
  (\lift{x}{R}) \psubstp{Q}{P}  
  :=
  \lift{(x)\substp{Q}{P}}{ R \psubstp{Q}{P} } \\
%   (\dropn{x})  \psubstp{Q}{P}       
%   := 
%   \left\{ 
%     \begin{array}{ccc} 
%       \dropn{\quotep{Q}} & & x \nameeq \quotep{P} \\
%       \dropn{x} & & otherwise \\
%     \end{array}
%   \right. 
  (\dropn{x})  \psubstp{Q}{P}       
  := 
  \left\{ 
    \begin{array}{ccc} 
      Q & & x \nameeq \quotep{P} \\
      \dropn{x} & & otherwise \\
    \end{array}
  \right.
\end{mathpar}
 

where

\begin{eqnarray}
  (x)\id{\{} \lpquote Q \rpquote / \lpquote P \rpquote \id{\}}            = 
  \left\{ 
    \begin{array}{ccc}
      \lpquote Q \rpquote & & x \nameeq \lpquote P \rpquote \\
      x & & otherwise \\
    \end{array}
  \right. \nonumber
\end{eqnarray}

and $z$ is chosen distinct from $\quotep{P}$, $\quotep{Q}$, the free
names in $Q$, and all the names in $R$. Our $\alpha$-equivalence will
be built in the standard way from this substitution.

\begin{remark}\label{rem:no_self_referential_names}
  One consequence of these definitions is that $\forall P. \quotep{P}
  \not\in \freenames{P}$.
\end{remark}

\subsection{ Dynamic quote: an example }

Anticipating something of what's to come, consider applying the
substitution, $\widehat{\id{\{}u / z \id{\}}}$, to the following pair
of processes, $\lift{w}{y!(z)}$ and $w[ \lpquote y!(z) \rpquote ]$.

\begin{eqnarray}
	\lift{w}{y!(z)}\widehat{\id{\{}u / z \id{\}}}
		& = &
		\lift{w}{y!(u)} \nonumber\\
	w[ \lpquote y!(z) \rpquote ] \widehat{ \id{\{}u / z \id{\}} }
		& = &
		w[ \lpquote y!(z) \rpquote ] \nonumber
\end{eqnarray}

Because the body of the process between quotes is impervious to
substitution, we get radically different answers. In fact, by
examining the first process in an input context,
e.g. $x?(z).\lift{w}{y!(z)}$, we see that the process under the lift
operator may be shaped by prefixed inputs binding a name inside it. In
this sense, the lift operator will be seen as a way to dynamically
construct processes before reifying them as names.

Finally equipped with these standard features we can present the
dynamics of the calculus.

\subsubsection{Operational semantics} 

Finally, we introduce the computational dynamics. What marks these
algebras as distinct from other more traditionally studied algebraic
structures, e.g. vector spaces or polynomial rings, is the manner in
which dynamics is captured. In traditional structures, dynamics is typically
expressed through morphisms between such structures, as in linear maps
between vector spaces or morphisms between rings. In algebras
associated with the semantics of computation, the dynamics is
expressed as part of the algebraic structure itself, through a
reduction reduction relation typically denoted by $\red$. Below, we
give a recursive presentation of this relation for the calculus used
in the encoding.

$\red \subseteq \pi \times \pi$
$\red : \pi \to \mathcal{P}(\pi)$

\begin{mathpar}
  \inferrule* [lab=Comm] { \textsf{match}( x_{src}, x_{trgt} ) } { x_{trgt}?(y)P \; | \; x_{src}!\langle {Q} \rangle \red P\{\quotep{Q}/y}\} }
  \and \\
  \inferrule* [lab=Par] {{P} \red {P}'} {{{P} | {Q}} \red {{P}' | {Q}}}
  \and
  \inferrule* [lab=Equiv]{{{P} \scong {P}'} \andalso {{P}' \red {Q}'} \andalso {{Q}' \scong {Q}}}{{P} \red {Q}}
\end{mathpar}

\begin{eqnarray*}
  match_{\equiv} (\quotep{P},\quotep{Q}) & := & P \equiv Q \\
  match_{\dagger}(\quotep{P},\quotep{Q}) & := & \forall R. P|Q \red^{*} R => R \red^{*} 0 \\
  match_{K}(\quotep{P},\quotep{Q}) & := & K \mbox{ for some context } K
\end{eqnarray*}

$u?(x)P | u!\langle Q \rangle \red P\{\quotep{Q}/x\}$

%We write $\wred$ for $\red^*$, and $P\red$ if $\exists Q $ such that $ P \red Q$.
We write $P\red$ if $\exists Q $ such that $ P \red Q$ and $P\not\red$, otherwise.

\section{Replication}

As mentioned before, it is known that replication (and hence
recursion) can be implemented in a higher-order process algebra
\cite{SangiorgiWalker}. As our first example of calculation with the
machinery thus far presented we give the construction explicitly in
the {\rhoc}.

\begin{eqnarray}
	D_{x} & := & \prefix{x}{y}{(\binpar{\outputp{x}{y}}{@{y}})} \nonumber\\
	\bangp_{x}{P} & := & \binpar{{x}!\langle{\binpar{D_{x}}{P}}\rangle}{D_{x}} \nonumber
\end{eqnarray}

\begin{eqnarray}
	\bangp_{x}{P} & & \nonumber\\
	=
	& {x}!\langle{(\prefix{x}{y}{(\outputp{x}{y} | @{y})) | P}}\rangle 
	      | \prefix{x}{y}{(\outputp{x}{y} | @{y})} & \nonumber\\
	\red
	& (\outputp{x}{y} | @{y})\substn{\quotep{(\prefix{x}{y}{(@{y} | \outputp{x}{y})) | P}}}{y} & \nonumber\\
	=
	& \outputp{x}{\quotep{(\prefix{x}{y}{(\outputp{x}{y} | @{y})) | P}}}
	  | {(\prefix{x}{y}{(\outputp{x}{y} | @{y})) | P}} & \nonumber\\
	\red
	& \ldots & \nonumber\\
	\red^*
	& P | P | \ldots & \nonumber
\end{eqnarray}

Of course, this encoding, as an implementation, runs away, unfolding
$\bangp{P}$ eagerly. A lazier and more implementable replication
operator, restricted to input-guarded processes, may be obtained as follows.

\begin{eqnarray}
\bangp{\prefix{u}{v}{P}} 
	:= 
	\binpar{\lift{x}{\prefix{u}{v}{(\binpar{D(x)}{P})}}}{D(x)} \nonumber
\end{eqnarray}

\begin{remark}
  Note that the lazier definition still does not deal with summation
  or mixed summation (i.e. sums over input and output). The reader is
  invited to construct definitions of replication that deal with these
  features. 

  Further, the definitions are parameterized in a name, $x$. Can you,
  gentle reader, make a definition that eliminates this parameter and
  guarantees no accidental interaction between the replication
  machinery and the process being replicated -- i.e. no accidental
  sharing of names used by the process to get its work done and the
  name(s) used by the replication to effect copying. This latter
  revision of the definition of replication is crucial to obtaining
  the expected identity $!!P \sim !P$.
\end{remark}

\begin{remark}\label{rem:paradoxical_combinator}
  The reader familiar with the lambda calculus will have noticed the
  similarity between $D$ and the paradoxical combinator.

  [Ed. note: the existence of this seems to suggest we have to be more
  restrictive on the set of processes and names we admit if we are to
  support no-cloning.]
\end{remark}

\subsubsection{Bisimulation}

The computational dynamics gives rise to another kind of equivalence,
the equivalence of computational behavior. As previously mentioned
this is typically captured \emph{via} some form of bisimulation.

% The notion we use in this paper is weak barbed bisimulation
% \cite{milner91polyadicpi}.

The notion we use in this paper is derived from weak barbed
bisimulation \cite{milner91polyadicpi}. 

\begin{definition}
An \emph{observation relation}, $\downarrow_{\mathcal N}$, over a set
of names, $\mathcal N$, is the smallest relation satisfying the rules
below.

\infrule[Out-barb]{y \in {\mathcal N}, \; x \nameeq y}
		  {\outputp{x}{v} \downarrow_{\mathcal N} x}
\infrule[Par-barb]{\mbox{$P\downarrow_{\mathcal N} x$ or $Q\downarrow_{\mathcal N} x$}}
		  {\binpar{P}{Q} \downarrow_{\mathcal N} x}

We write $P \Downarrow_{\mathcal N} x$ if there is $Q$ such that 
$P \wred Q$ and $Q \downarrow_{\mathcal N} x$.
\end{definition}

\begin{definition}
%\label{def.bbisim}
An  ${\mathcal N}$-\emph{barbed bisimulation} over a set of names, ${\mathcal N}$, is a symmetric binary relation 
${\mathcal S}_{\mathcal N}$ between agents such that $P\rel{S}_{\mathcal N}Q$ implies:
\begin{enumerate}
\item If $P \red P'$ then $Q \wred Q'$ and $P'\rel{S}_{\mathcal N} Q'$.
\item If $P\downarrow_{\mathcal N} x$, then $Q\Downarrow_{\mathcal N} x$.
\end{enumerate}
$P$ is ${\mathcal N}$-barbed bisimilar to $Q$, written
$P \wbbisim_{\mathcal N} Q$, if $P \rel{S}_{\mathcal N} Q$ for some ${\mathcal N}$-barbed bisimulation ${\mathcal S}_{\mathcal N}$.
\end{definition}

$\mathcal{R} \subseteq \pi \times \pi$

$P \mathcal{R} Q => \forall P'. P \red P' \Rightarrow \exists Q'. Q \red Q', P' \mathcal{R} Q'$

$P \vdash x \Rightarrow Q \vdash x$

\begin{mathpar}
  \inferrule*[lab=Out-barb]{x \nameeq y}{{y}!\langle{Q}\rangle \vdash x}
  \and
  \inferrule*[lab=Par-barb]{\mbox{$P\vdash x$ or $Q\vdash x$}}{\binpar{P}{Q} \vdash x}
\end{mathpar}

\subsubsection{Contexts}

One of the principle advantages of computational calculi like the
$\pi$-calculus is a well-defined notion of context,
contextual-equivalence and a correlation between
contextual-equivalence and notions of bisimulation. The notion of
context allows the decomposition of a process into (sub-)process and
its syntactic environment, its context. Thus, a context may be
thought of as a process with a ``hole'' (written $\Box$) in it. The
application of a context $M$ to a process $P$, written $M[P]$, is
tantamount to filling the hole in $M$ with $P$. In this paper we do
not need the full weight of this theory, but do make use of the notion
of context in the proof the main theorem. 

\begin{mathpar}
  \inferrule* [lab=summation] {} {{M_{M},M_{N}} \bc \Box \;|\; x.M_{A} \;|\; M_{M}+M_{N}}
  \and
  \inferrule* [lab=agent] {} {{M_{A}} \bc (\vec{x})M_{P} \;| \; \clift{P_0,\ldots,M_{P},\ldots,P_N}}
  \and \\
  \inferrule* [lab=process] {} {{M_{P}} \bc M_{N} \;| \;P|M_{P} }
\end{mathpar} 

\begin{mathpar}
  \inferrule* [lab=sychronization] {} {M_{N} \bc \Box \;|\; x?M_{F} \;|\; x!M_{C}}
  \and
  \inferrule* [lab=abstraction] {} {{M_{F}} \bc (x)M_{P} }
  \and
  \inferrule* [lab=concretion] {} {{M_{C}} \bc \langle M_{P} \rangle }
  \and \\
  \inferrule* [lab=process] {} {{M_{P}} \bc M_{N} \;| \;P|M_{P} }
\end{mathpar}

\begin{definition}[contextual application] Given a context $M$, and
  process $P$, we define the \emph{contextual application}, $M[P] :=
  M\{P/\Box\}$. That is, the contextual application of M to P is the
  substitution of $P$ for $\Box$ in $M$.
\end{definition}

$\meaningof{-} : L \to \mathcal{P}(\pi)$

\begin{mathpar}
  \inferrule* [lab=collection] {} {\meaningof{true} = \pi, \and \meaningof{~E} = \pi \setminus \meaningof{E}, \and \meaningof{E_{1} \& E_{2}} = \meaningof{E_{1}} \cap \meaningof{E_{2}}}
\end{mathpar}

\begin{mathpar}
  \inferrule* [lab=structure] {} {\meaningof{0} = \{ P \in \pi | P \equiv 0 \}, \and \\ \meaningof{E_1 | E_2} = \{ P \in \pi | P \equiv P_{1} | P_{2}, P_{1} \in \meaningof{E_{1}}, P_{2} \in \meaningof{E_2}\} }
\end{mathpar}

\begin{mathpar}
 \inferrule* [lab=behavior] {} {\meaningof{\langle a?b \rangle E} = \{ P \in \pi | P \equiv Q | u?(y)P', \\ \and \\\\ \and \\ \;\;\; u \in \meaningof{a}, \forall z.P'\{z/y\} \in \meaningof{E\{z/b\}}\}, \and \\ \meaningof{a!E} = \{ P \in \pi | P \equiv Q | x!\langle P' \rangle, x \in \meaningof{a} P' \in \meaningof{E}\} }
\end{mathpar}

\begin{mathpar}
 \inferrule* [lab=nominal] {} {\meaningof{\quotep{E}} = \{ \quotep{P} \in \quotep{\pi} | P \in \meaningof{E} \}, \and \meaningof{\quotep{P}} = \{ \quotep{Q} \in \quotep{\pi} | P \equiv Q \} \and \\ \meaningof{@\quotep{E}} = \{ P \in \pi | P \equiv @x, x \in \meaningof{E} \}}
\end{mathpar}

\begin{eqnarray*}
  \\
  \meaningof{-} : TS \to ST
\end{eqnarray*}

\begin{eqnarray*}
  \\
  L : TS \to ST
\end{eqnarray*}

\begin{eqnarray*}
  \\
  P \models E \iff P \in \meaningof{E}
\end{eqnarray*}

\begin{eqnarray*}
  P \approx_{L} Q \iff \forall E \in L. P \models E \iff Q \models E
\end{eqnarray*}

\begin{eqnarray*}
  P \approx_{K} Q
\end{eqnarray*}

\begin{eqnarray*}
  P \approx Q
\end{eqnarray*}

$\approx_{K} = \approx = \approx_{L}$

\subsubsection{Contextual duality}

Note that contexts extend the quotation operation to a family of
operations from processes to names. Given a context, $M$, we can
define a \emph{nominal context}, $\quotep{M}$ by $\quotep{M}[P] :=
\quotep{M[P]}$. To foreshadow what is to come we observe that these
operations enjoy a duality with processes very much like the duality
between vectors and maps from vectors to scalars.

Further, because the calculus is essentially higher-order, we have a
correspondence between contexts and processes. More specifically,
given a name $x$ and a context $M$ we can construct $M^{*}_{x}$ such
that 

\begin{mathpar}
  M^{*}_{x} | \lift{x}{P} \red M[P]
\end{mathpar}

namely,

\begin{mathpar}
  M^{*}_{x} := x?(u).M[\dropn{u}]
\end{mathpar}

The dependence of $M^{*}_{x}$ on a name makes it an abstraction, 

\begin{mathpar}
  M^{*} := (x)x?(u).M[\dropn{u}]
\end{mathpar}

\subsection{Additional notation}

It will sometimes be convenient to denote the process a name
quotes. We already have the notation $x = \quotep{P}$, but it will be
convenient to introduce an alternate notation, $\procn{x}$, when we
want to emphasize the connection to the use of the name. Note that, by
virtue of name equivalence, $\quotep{\procn{x}} \nameeq x$; so, the
notation is consistent with previous definitions.

Further, because names have structure it is possible to effect
substitutions on the basis of that structure. This means we need to
upgrade our notation for substitutions, which we accomplish by
adapting comprehension notation. Thus,

\begin{mathpar}
  P\{ y / x : x \in S \}
\end{mathpar}

is interpreted to mean the process derived from P by replacing (in a
capture-avoiding manner) each occurrence of $x$ in $S$ by $y$. For example,

\begin{mathpar}
  P\{ \quotep{\procn{x}|\procn{x}} / x : x \in \freenames{P} \}
\end{mathpar}

will replace each (occurrence) of a free name $x$ in $P$ by
$\quotep{\procn{x}|\procn{x}}$.

Also, we will avail ourselves of the notation $x^{L}$ and $x^{R}$ to
denote injections of a name into disjoint copies of the name
space. There are numerous ways to accomplish this. One example can be
found in \cite{MeredithR05}. This notation overloads to vectors of
names: $\vec{x}^{\pi} := (x_{i}^{\pi} \; : \; 0 \leq i < |\vec{x}| )$ where $\pi \in \{L,R\}$.

We also use $P^{\Box} := P|\Box$.

In \cite{MeredithR05} an interpretation of the new operator is
given. It turns out that there are several possible interpretations
all enjoying the requisite algebraic properties of the operator (see
\cite{milner91polyadicpi}). We will therefore make liberal use of
$(\nu\; \vec{x})P$.

% subsection the_syntax_and_semantics_of_the_notation_system (end)   

\input{qm2pi.qmops} 

\input{qm2pi.sterngerlach} 

\input{qm2pi.metric} 

% section concurrent_process_calculi (end)

%\input{qm2pi.proofsketch}

% section proof sketch (end)

%\input{qm2pi.slviaknots} 

% section spatial logic via knots (end)

\input{qm2pi.conclusion}

% section conclusion (end)

%\input{qm2pi.dtcodes} 

% section wiring algorithm (end)

\input{qm2pi.ack} 

% section acknowledgments (end)

\newpage


\bibliographystyle{plain}   
\bibliography{../../biblios/main.bib}

\input{qm2pi.rhodetails}

\end{document}

 

% section concurrent_process_calculi (end)

%\documentclass[12pt]{llncs}
%\documentclass{jktr}

\usepackage[pdftex]{hyperref}                   
\usepackage {listings}
\usepackage {mathpartir}
\usepackage{bcprules}
%\usepackage{listings}
                       
\usepackage{graphicx} 
%\usepackage[margins=2.5cm,nohead,nofoot]{geometry}
%\usepackage{geometry}
\usepackage{amsfonts}
\usepackage{amstext}
\usepackage{latexsym}
\usepackage{amssymb}
\usepackage{color}


%\include{myPreamble}
\include{qm2pi.local} 

%\ifpdf
%\usepackage[pdftex]{graphicx}
%\else
%\usepackage{graphicx}
%\fi

 % \ifpdf
%  \usepackage{pdfsync}
%  \if


%\title{Brief Article}
%\author{David F. Snyder}
%\author{L.G. Meredith}

%\address{Dept. of Math., Texas State University--San Marcos, San Marcos, TX 78666}
       
\pagestyle{empty}


\begin{document}

\lstset{language=[Objective]Caml,frame=shadowbox}

\input{qm2pi.front}

% section front matter (end)

\input{qm2pi.intro} 
 
% section introduction (end)

% \input{qm2pi.knotations} 

% section notation (end)

\input{qm2pi.process.calculi} 

% section concurrent_process_calculi_and_spatial_logics_ (end)
    
%\input{qm2pi.knots2pi} 

%\input{qm2pi.trefoil} 

%\input{qm2pi.mainthm} 

% subsection basic_interpretation (end)

%\input{qm2pi.rho.presentation} 
\subsection{The syntax and semantics of the notation system}\label{sub:the_syntax_and_semantics_of_the_notation_system} % (fold)

We now summarize a technical presentation of the calculus that
embodies our theory of dynamics. The typical presentation of such a
calculus follows the style of giving generators and relations on
them. The grammar, below, describing term constructors, freely
generates the set of processes, $\Proc$. This set is then quotiented
by a relation known as structural congruence and it is over this set
that the notion of dynamics is expressed. This presentation is
essentially that of \cite{MeredithR05} with the addition of
polyadicity and summation. For readability we have relegated some of
the technical subtleties to an appendix.

\subsubsection{Process grammar}\label{subsub:process_grammar}

\begin{mathpar}
  \inferrule* [lab=synchronization] {} {{M} \bc \pzero \;|\; x?F \;|\; x!C }
  \and
  \inferrule* [lab=abstraction] {} {{F} \bc (x)P}
  \and
  \inferrule* [lab=concretion] {} {{C} \bc \langle Q \rangle}
  \and
  \inferrule* [lab=process] {} {{P,Q} \bc M \;| \;P|Q \;|\; @{x}}
  \and
  \inferrule* [lab=name] {} {{x} \bc \quotep{P}}
\end{mathpar} 

Note that $\vec{x}$ (resp. $\vec{P}$) denotes a vector of names
(resp. processes) of length $|\vec{x}|$ (resp. $|\vec{P}|$). We adopt
the following useful abbreviations.

\begin{mathpar}
   x?(\vec{y}).P := x.(\vec{y})P \and  x\clift{\vec{P}} := x.\clift{\vec{P}}
   \and x!(y) := \lift{x}{\dropn{y}}
   \and \Pi_{i=0}^{n-1}P_i := P_0 | \ldots | P_{n-1}
\end{mathpar}

\subsubsection{Structural congruence}

\paragraph{Free and bound names and alpha-equivalence.} At the
core of structural equivalence is alpha-equivalence which identifies
process that are the same up to a change of variable. Formally, we
recognize the distinction between free and bound names. The free names
of a process, $\freenames{P}$, may be calculated recursively as
follows:

\begin{mathpar}
\freenames{\pzero} := \emptyset
  \and \\
  \freenames{x?(y).P} := \{ x \} \cup (\freenames{P} \setminus \{ y \})
  \and 
  \freenames{x!\langle P \rangle} := \{ x \} \cup \{ P \} 
  \and \\
  \freenames{P|Q} := \freenames{P} \cup \freenames{Q}
  \and \\
  \freenames{@{x}} := \{ x \}
\end{mathpar}

$\pi$
$\quotep{\pi}$

$\freenames{-} : \pi \to \mathcal{P}(\quotep{\pi})$

\begin{eqnarray*}
  \freenames{\pzero} & := & \emptyset \\
  \freenames{x?(y).P} & := & \{ x \} \cup (\freenames{P} \setminus \{ y \}) \\
  \freenames{x!\langle P \rangle} & := & \{ x \} \cup \{ P \} \\
  \freenames{P|Q} & := & \freenames{P} \cup \freenames{Q} \\
  \freenames{\dropn{x}} & := & \{ x \}
\end{eqnarray*}

The bound names of a process, $\boundnames{P}$, are those names occurring in $P$
that are not free. For example, in $x?(y).0$, the name $x$ is free, while $y$ is bound.

\begin{mathpar}
  \inferrule* [lab=monoidal-laws] {} { P|Q \equiv Q|P \and P|0 \equiv P \and P|(Q|R) \equiv (P|Q)|R }
\end{mathpar}

\begin{mathpar}
  \inferrule* [lab=alpha-equivalence] {} { (x)P \equiv (y)P\{y/x\} \and y \not\in \freenames{P} }
\end{mathpar}

\begin{definition}
Then two processes, $P,Q$, are alpha-equivalent if $P = Q\{\vec{y}/\vec{x}\}$ for
some $\vec{x} \in \boundnames{Q},\vec{y} \in \boundnames{P}$, where $Q\{\vec{y}/\vec{x}\}$
denotes the capture-avoiding substitution of $\vec{y}$ for $\vec{x}$ in $Q$.
\end{definition}

\begin{definition}
  The {\em structural congruence} \cite{SangiorgiWalker} , $\equiv$,
  between processes is the least congruence containing
  alpha-equivalence, satisfying the abelian monoid laws
  (associativity, commutativity and $\pzero$ as identity) for parallel
  composition $|$ and for summation $+$.
\end{definition}

\subsection{Name equivalence}

We take name equivalence, written $\nameeq$, to be the smallest
equivalence relation generated by the following rules.

\begin{mathpar}
\inferrule*[lab=Quote-drop]
{ }
{ \quotep{@{x}} \nameeq x }

\inferrule*[lab=Struct-equiv]
{ P \scong Q }
{ \quotep{P} \nameeq \quotep{Q} }
\end{mathpar}

The astute reader will have noticed that the mutual recursion of names
and processes imposes a mutual recursion on alpha-equivalence and
structural equivalence via name-equivalence. Fortunately, all of this
works out pleasantly and we may calculate in the natural way, free of
concern. The reader interested in the details is referred to the
appendix \ref{appendix:rho_details}.

\subsection{Substitution}

We use $\Proc$ for the set of processes, $\QProc$ for the set of
names, and $\id{\{}\vec{y} / \vec{x} \id{\}}$ to denote partial maps,
$s : \QProc \rightarrow \QProc$. A map, $s$ lifts, uniquely, to a map
on process terms, $\widehat{s} : \Proc \rightarrow \Proc$ by the
following equations.

\begin{mathpar}
  (0) \psubstp{Q}{P} := 0 \\
  (R \juxtap S) \psubstp{Q}{P}
  :=    
  (R)\psubstp{Q}{P} \juxtap (S) \psubstp{Q}{P} \\
  (x?(y).R) \psubstp{Q}{P}    
  :=    
  (x)\substp{Q}{P} (z)\concat( (R \psubstn{z}{y}) \psubstp{Q}{P} ) \\
  (\lift{x}{R}) \psubstp{Q}{P}  
  :=
  \lift{(x)\substp{Q}{P}}{ R \psubstp{Q}{P} } \\
%   (\dropn{x})  \psubstp{Q}{P}       
%   := 
%   \left\{ 
%     \begin{array}{ccc} 
%       \dropn{\quotep{Q}} & & x \nameeq \quotep{P} \\
%       \dropn{x} & & otherwise \\
%     \end{array}
%   \right. 
  (\dropn{x})  \psubstp{Q}{P}       
  := 
  \left\{ 
    \begin{array}{ccc} 
      Q & & x \nameeq \quotep{P} \\
      \dropn{x} & & otherwise \\
    \end{array}
  \right.
\end{mathpar}
 

where

\begin{eqnarray}
  (x)\id{\{} \lpquote Q \rpquote / \lpquote P \rpquote \id{\}}            = 
  \left\{ 
    \begin{array}{ccc}
      \lpquote Q \rpquote & & x \nameeq \lpquote P \rpquote \\
      x & & otherwise \\
    \end{array}
  \right. \nonumber
\end{eqnarray}

and $z$ is chosen distinct from $\quotep{P}$, $\quotep{Q}$, the free
names in $Q$, and all the names in $R$. Our $\alpha$-equivalence will
be built in the standard way from this substitution.

\begin{remark}\label{rem:no_self_referential_names}
  One consequence of these definitions is that $\forall P. \quotep{P}
  \not\in \freenames{P}$.
\end{remark}

\subsection{ Dynamic quote: an example }

Anticipating something of what's to come, consider applying the
substitution, $\widehat{\id{\{}u / z \id{\}}}$, to the following pair
of processes, $\lift{w}{y!(z)}$ and $w[ \lpquote y!(z) \rpquote ]$.

\begin{eqnarray}
	\lift{w}{y!(z)}\widehat{\id{\{}u / z \id{\}}}
		& = &
		\lift{w}{y!(u)} \nonumber\\
	w[ \lpquote y!(z) \rpquote ] \widehat{ \id{\{}u / z \id{\}} }
		& = &
		w[ \lpquote y!(z) \rpquote ] \nonumber
\end{eqnarray}

Because the body of the process between quotes is impervious to
substitution, we get radically different answers. In fact, by
examining the first process in an input context,
e.g. $x?(z).\lift{w}{y!(z)}$, we see that the process under the lift
operator may be shaped by prefixed inputs binding a name inside it. In
this sense, the lift operator will be seen as a way to dynamically
construct processes before reifying them as names.

Finally equipped with these standard features we can present the
dynamics of the calculus.

\subsubsection{Operational semantics} 

Finally, we introduce the computational dynamics. What marks these
algebras as distinct from other more traditionally studied algebraic
structures, e.g. vector spaces or polynomial rings, is the manner in
which dynamics is captured. In traditional structures, dynamics is typically
expressed through morphisms between such structures, as in linear maps
between vector spaces or morphisms between rings. In algebras
associated with the semantics of computation, the dynamics is
expressed as part of the algebraic structure itself, through a
reduction reduction relation typically denoted by $\red$. Below, we
give a recursive presentation of this relation for the calculus used
in the encoding.

$\red \subseteq \pi \times \pi$
$\red : \pi \to \mathcal{P}(\pi)$

\begin{mathpar}
  \inferrule* [lab=Comm] { \textsf{match}( x_{src}, x_{trgt} ) } { x_{trgt}?(y)P \; | \; x_{src}!\langle {Q} \rangle \red P\{\quotep{Q}/y}\} }
  \and \\
  \inferrule* [lab=Par] {{P} \red {P}'} {{{P} | {Q}} \red {{P}' | {Q}}}
  \and
  \inferrule* [lab=Equiv]{{{P} \scong {P}'} \andalso {{P}' \red {Q}'} \andalso {{Q}' \scong {Q}}}{{P} \red {Q}}
\end{mathpar}

\begin{eqnarray*}
  match_{\equiv} (\quotep{P},\quotep{Q}) & := & P \equiv Q \\
  match_{\dagger}(\quotep{P},\quotep{Q}) & := & \forall R. P|Q \red^{*} R => R \red^{*} 0 \\
  match_{K}(\quotep{P},\quotep{Q}) & := & K \mbox{ for some context } K
\end{eqnarray*}

$u?(x)P | u!\langle Q \rangle \red P\{\quotep{Q}/x\}$

%We write $\wred$ for $\red^*$, and $P\red$ if $\exists Q $ such that $ P \red Q$.
We write $P\red$ if $\exists Q $ such that $ P \red Q$ and $P\not\red$, otherwise.

\section{Replication}

As mentioned before, it is known that replication (and hence
recursion) can be implemented in a higher-order process algebra
\cite{SangiorgiWalker}. As our first example of calculation with the
machinery thus far presented we give the construction explicitly in
the {\rhoc}.

\begin{eqnarray}
	D_{x} & := & \prefix{x}{y}{(\binpar{\outputp{x}{y}}{@{y}})} \nonumber\\
	\bangp_{x}{P} & := & \binpar{{x}!\langle{\binpar{D_{x}}{P}}\rangle}{D_{x}} \nonumber
\end{eqnarray}

\begin{eqnarray}
	\bangp_{x}{P} & & \nonumber\\
	=
	& {x}!\langle{(\prefix{x}{y}{(\outputp{x}{y} | @{y})) | P}}\rangle 
	      | \prefix{x}{y}{(\outputp{x}{y} | @{y})} & \nonumber\\
	\red
	& (\outputp{x}{y} | @{y})\substn{\quotep{(\prefix{x}{y}{(@{y} | \outputp{x}{y})) | P}}}{y} & \nonumber\\
	=
	& \outputp{x}{\quotep{(\prefix{x}{y}{(\outputp{x}{y} | @{y})) | P}}}
	  | {(\prefix{x}{y}{(\outputp{x}{y} | @{y})) | P}} & \nonumber\\
	\red
	& \ldots & \nonumber\\
	\red^*
	& P | P | \ldots & \nonumber
\end{eqnarray}

Of course, this encoding, as an implementation, runs away, unfolding
$\bangp{P}$ eagerly. A lazier and more implementable replication
operator, restricted to input-guarded processes, may be obtained as follows.

\begin{eqnarray}
\bangp{\prefix{u}{v}{P}} 
	:= 
	\binpar{\lift{x}{\prefix{u}{v}{(\binpar{D(x)}{P})}}}{D(x)} \nonumber
\end{eqnarray}

\begin{remark}
  Note that the lazier definition still does not deal with summation
  or mixed summation (i.e. sums over input and output). The reader is
  invited to construct definitions of replication that deal with these
  features. 

  Further, the definitions are parameterized in a name, $x$. Can you,
  gentle reader, make a definition that eliminates this parameter and
  guarantees no accidental interaction between the replication
  machinery and the process being replicated -- i.e. no accidental
  sharing of names used by the process to get its work done and the
  name(s) used by the replication to effect copying. This latter
  revision of the definition of replication is crucial to obtaining
  the expected identity $!!P \sim !P$.
\end{remark}

\begin{remark}\label{rem:paradoxical_combinator}
  The reader familiar with the lambda calculus will have noticed the
  similarity between $D$ and the paradoxical combinator.

  [Ed. note: the existence of this seems to suggest we have to be more
  restrictive on the set of processes and names we admit if we are to
  support no-cloning.]
\end{remark}

\subsubsection{Bisimulation}

The computational dynamics gives rise to another kind of equivalence,
the equivalence of computational behavior. As previously mentioned
this is typically captured \emph{via} some form of bisimulation.

% The notion we use in this paper is weak barbed bisimulation
% \cite{milner91polyadicpi}.

The notion we use in this paper is derived from weak barbed
bisimulation \cite{milner91polyadicpi}. 

\begin{definition}
An \emph{observation relation}, $\downarrow_{\mathcal N}$, over a set
of names, $\mathcal N$, is the smallest relation satisfying the rules
below.

\infrule[Out-barb]{y \in {\mathcal N}, \; x \nameeq y}
		  {\outputp{x}{v} \downarrow_{\mathcal N} x}
\infrule[Par-barb]{\mbox{$P\downarrow_{\mathcal N} x$ or $Q\downarrow_{\mathcal N} x$}}
		  {\binpar{P}{Q} \downarrow_{\mathcal N} x}

We write $P \Downarrow_{\mathcal N} x$ if there is $Q$ such that 
$P \wred Q$ and $Q \downarrow_{\mathcal N} x$.
\end{definition}

\begin{definition}
%\label{def.bbisim}
An  ${\mathcal N}$-\emph{barbed bisimulation} over a set of names, ${\mathcal N}$, is a symmetric binary relation 
${\mathcal S}_{\mathcal N}$ between agents such that $P\rel{S}_{\mathcal N}Q$ implies:
\begin{enumerate}
\item If $P \red P'$ then $Q \wred Q'$ and $P'\rel{S}_{\mathcal N} Q'$.
\item If $P\downarrow_{\mathcal N} x$, then $Q\Downarrow_{\mathcal N} x$.
\end{enumerate}
$P$ is ${\mathcal N}$-barbed bisimilar to $Q$, written
$P \wbbisim_{\mathcal N} Q$, if $P \rel{S}_{\mathcal N} Q$ for some ${\mathcal N}$-barbed bisimulation ${\mathcal S}_{\mathcal N}$.
\end{definition}

$\mathcal{R} \subseteq \pi \times \pi$

$P \mathcal{R} Q => \forall P'. P \red P' \Rightarrow \exists Q'. Q \red Q', P' \mathcal{R} Q'$

$P \vdash x \Rightarrow Q \vdash x$

\begin{mathpar}
  \inferrule*[lab=Out-barb]{x \nameeq y}{{y}!\langle{Q}\rangle \vdash x}
  \and
  \inferrule*[lab=Par-barb]{\mbox{$P\vdash x$ or $Q\vdash x$}}{\binpar{P}{Q} \vdash x}
\end{mathpar}

\subsubsection{Contexts}

One of the principle advantages of computational calculi like the
$\pi$-calculus is a well-defined notion of context,
contextual-equivalence and a correlation between
contextual-equivalence and notions of bisimulation. The notion of
context allows the decomposition of a process into (sub-)process and
its syntactic environment, its context. Thus, a context may be
thought of as a process with a ``hole'' (written $\Box$) in it. The
application of a context $M$ to a process $P$, written $M[P]$, is
tantamount to filling the hole in $M$ with $P$. In this paper we do
not need the full weight of this theory, but do make use of the notion
of context in the proof the main theorem. 

\begin{mathpar}
  \inferrule* [lab=summation] {} {{M_{M},M_{N}} \bc \Box \;|\; x.M_{A} \;|\; M_{M}+M_{N}}
  \and
  \inferrule* [lab=agent] {} {{M_{A}} \bc (\vec{x})M_{P} \;| \; \clift{P_0,\ldots,M_{P},\ldots,P_N}}
  \and \\
  \inferrule* [lab=process] {} {{M_{P}} \bc M_{N} \;| \;P|M_{P} }
\end{mathpar} 

\begin{mathpar}
  \inferrule* [lab=sychronization] {} {M_{N} \bc \Box \;|\; x?M_{F} \;|\; x!M_{C}}
  \and
  \inferrule* [lab=abstraction] {} {{M_{F}} \bc (x)M_{P} }
  \and
  \inferrule* [lab=concretion] {} {{M_{C}} \bc \langle M_{P} \rangle }
  \and \\
  \inferrule* [lab=process] {} {{M_{P}} \bc M_{N} \;| \;P|M_{P} }
\end{mathpar}

\begin{definition}[contextual application] Given a context $M$, and
  process $P$, we define the \emph{contextual application}, $M[P] :=
  M\{P/\Box\}$. That is, the contextual application of M to P is the
  substitution of $P$ for $\Box$ in $M$.
\end{definition}

$\meaningof{-} : L \to \mathcal{P}(\pi)$

\begin{mathpar}
  \inferrule* [lab=collection] {} {\meaningof{true} = \pi, \and \meaningof{~E} = \pi \setminus \meaningof{E}, \and \meaningof{E_{1} \& E_{2}} = \meaningof{E_{1}} \cap \meaningof{E_{2}}}
\end{mathpar}

\begin{mathpar}
  \inferrule* [lab=structure] {} {\meaningof{0} = \{ P \in \pi | P \equiv 0 \}, \and \\ \meaningof{E_1 | E_2} = \{ P \in \pi | P \equiv P_{1} | P_{2}, P_{1} \in \meaningof{E_{1}}, P_{2} \in \meaningof{E_2}\} }
\end{mathpar}

\begin{mathpar}
 \inferrule* [lab=behavior] {} {\meaningof{\langle a?b \rangle E} = \{ P \in \pi | P \equiv Q | u?(y)P', \\ \and \\\\ \and \\ \;\;\; u \in \meaningof{a}, \forall z.P'\{z/y\} \in \meaningof{E\{z/b\}}\}, \and \\ \meaningof{a!E} = \{ P \in \pi | P \equiv Q | x!\langle P' \rangle, x \in \meaningof{a} P' \in \meaningof{E}\} }
\end{mathpar}

\begin{mathpar}
 \inferrule* [lab=nominal] {} {\meaningof{\quotep{E}} = \{ \quotep{P} \in \quotep{\pi} | P \in \meaningof{E} \}, \and \meaningof{\quotep{P}} = \{ \quotep{Q} \in \quotep{\pi} | P \equiv Q \} \and \\ \meaningof{@\quotep{E}} = \{ P \in \pi | P \equiv @x, x \in \meaningof{E} \}}
\end{mathpar}

\begin{eqnarray*}
  \\
  \meaningof{-} : TS \to ST
\end{eqnarray*}

\begin{eqnarray*}
  \\
  L : TS \to ST
\end{eqnarray*}

\begin{eqnarray*}
  \\
  P \models E \iff P \in \meaningof{E}
\end{eqnarray*}

\begin{eqnarray*}
  P \approx_{L} Q \iff \forall E \in L. P \models E \iff Q \models E
\end{eqnarray*}

\begin{eqnarray*}
  P \approx_{K} Q
\end{eqnarray*}

\begin{eqnarray*}
  P \approx Q
\end{eqnarray*}

$\approx_{K} = \approx = \approx_{L}$

\subsubsection{Contextual duality}

Note that contexts extend the quotation operation to a family of
operations from processes to names. Given a context, $M$, we can
define a \emph{nominal context}, $\quotep{M}$ by $\quotep{M}[P] :=
\quotep{M[P]}$. To foreshadow what is to come we observe that these
operations enjoy a duality with processes very much like the duality
between vectors and maps from vectors to scalars.

Further, because the calculus is essentially higher-order, we have a
correspondence between contexts and processes. More specifically,
given a name $x$ and a context $M$ we can construct $M^{*}_{x}$ such
that 

\begin{mathpar}
  M^{*}_{x} | \lift{x}{P} \red M[P]
\end{mathpar}

namely,

\begin{mathpar}
  M^{*}_{x} := x?(u).M[\dropn{u}]
\end{mathpar}

The dependence of $M^{*}_{x}$ on a name makes it an abstraction, 

\begin{mathpar}
  M^{*} := (x)x?(u).M[\dropn{u}]
\end{mathpar}

\subsection{Additional notation}

It will sometimes be convenient to denote the process a name
quotes. We already have the notation $x = \quotep{P}$, but it will be
convenient to introduce an alternate notation, $\procn{x}$, when we
want to emphasize the connection to the use of the name. Note that, by
virtue of name equivalence, $\quotep{\procn{x}} \nameeq x$; so, the
notation is consistent with previous definitions.

Further, because names have structure it is possible to effect
substitutions on the basis of that structure. This means we need to
upgrade our notation for substitutions, which we accomplish by
adapting comprehension notation. Thus,

\begin{mathpar}
  P\{ y / x : x \in S \}
\end{mathpar}

is interpreted to mean the process derived from P by replacing (in a
capture-avoiding manner) each occurrence of $x$ in $S$ by $y$. For example,

\begin{mathpar}
  P\{ \quotep{\procn{x}|\procn{x}} / x : x \in \freenames{P} \}
\end{mathpar}

will replace each (occurrence) of a free name $x$ in $P$ by
$\quotep{\procn{x}|\procn{x}}$.

Also, we will avail ourselves of the notation $x^{L}$ and $x^{R}$ to
denote injections of a name into disjoint copies of the name
space. There are numerous ways to accomplish this. One example can be
found in \cite{MeredithR05}. This notation overloads to vectors of
names: $\vec{x}^{\pi} := (x_{i}^{\pi} \; : \; 0 \leq i < |\vec{x}| )$ where $\pi \in \{L,R\}$.

We also use $P^{\Box} := P|\Box$.

In \cite{MeredithR05} an interpretation of the new operator is
given. It turns out that there are several possible interpretations
all enjoying the requisite algebraic properties of the operator (see
\cite{milner91polyadicpi}). We will therefore make liberal use of
$(\nu\; \vec{x})P$.

% subsection the_syntax_and_semantics_of_the_notation_system (end)   

\input{qm2pi.qmops} 

\input{qm2pi.sterngerlach} 

\input{qm2pi.metric} 

% section concurrent_process_calculi (end)

%\input{qm2pi.proofsketch}

% section proof sketch (end)

%\input{qm2pi.slviaknots} 

% section spatial logic via knots (end)

\input{qm2pi.conclusion}

% section conclusion (end)

%\input{qm2pi.dtcodes} 

% section wiring algorithm (end)

\input{qm2pi.ack} 

% section acknowledgments (end)

\newpage


\bibliographystyle{plain}   
\bibliography{../../biblios/main.bib}

\input{qm2pi.rhodetails}

\end{document}



% section proof sketch (end)

%\section{Unlikely characters: spatial logic for
  knots}\label{sub:characteristic_formulae} % (fold)

Associated to the mobile process calculi are a family of logics known
as the Hennessy-Milner logics. These logics typically enjoy a
semantics interpreting formulae as sets of processes that when
factored through the encoding outlined above allows an identification
of classes of knots with logical formulae. In the context of this
encoding the sub-family known as the spatial logics \cite{CairesC03}
\cite{CairesC04} \cite{Caires04} are of particular interest providing
several important features for expressing and reasoning about
properties (i.e. classes) of knots. We hint here at how this may be done.

%\begin{description}
%\item [structural connectives] 
\subsubsection{Structural connectives} The spatial logics enjoy
structural connectives corresponding, at the logical level, to the
parallel composition ($P | Q$) and new name ($(\nu \; x)P$)
connectives for processes. As illustrated in the examples below, these
connectives are extremely expressive given the shape of our encoding.
%\item [decideable satisfaction]

\subsubsection{Decideable satisfaction}
In \cite{Caires04} the satisfaction relation is shown to be decideable
for a rich class of processes. It further turns out that the image of
the our encoding is a proper subset of that class. This result
provides the basis for an algorithm by which to search for knots
enjoying a given property.
%\item [characteristic formulae]

\subsubsection{Characteristic formulae}
In the same paper \cite{Caires04} , Caires presents a means of calculating
characteristic formulae, selecting equivalence classes of processes
up to a pre--specified depth limit on the support set of names. Composed with our
encoding, this characteristic formula can be used to select
characteristic formulae for knots.
%\end{description}

\subsubsection{Spatial logic formulae}

The grammar below (segmented for comprehension) summarizes the syntax
of spatial logic formulae. We employ illustrative examples in the
sequel to provide an intuitive understanding of their meaning
referring the reader to \cite{Caires04} for a more detailed explication
of the semantics.

\begin{mathpar}
  \inferrule* [lab=boolean] {} {{A,B} \bc T \;|\; \neg A \;|\; A \wedge B \;|\; \eta = \eta'}
  \and
  \inferrule* [lab=spatial] {} {|\; \pzero \;|\; A | B \;|\; x \text{\textregistered} A \;|\; \forall x . A \;|\;  H x . A}
  \and
  \inferrule* [lab=behavioral] {} {|\; \alpha . A}
  \and 
  \inferrule* [lab=recursion] {} {|\; X(\vec{u}) \;|\; \mu X(\vec{u}) . A}
  \and
  \inferrule* [lab=action] {} {\alpha \bc \langle x?(\vec{y}) \rangle \;|\; \langle x!(\vec{y}) \rangle \;|\; \langle \tau \rangle}
  \and 
  \inferrule* [lab=name] {} {\eta \bc x \;|\; \tau}
\end{mathpar} 

% subsection characteristic_formulae (end)   	 

\subsection{Example formulae}\label{sub:example_formulae_} % (fold)

\subsubsection{Crossing as formula.}
% 
% \begin{align*}
%   \frac{d}{dx} \sin x &= \cos x 
%   & \frac{d}{dx} e^x &= e^x \\
%   \frac{d}{dx} \cos x &= - \sin x 
%   & \frac{d}{dx} \log x &= \frac{1}{x} \\
% \end{align*} 

\begin{align*}
 \mu C(x_{0},x_{1},y_{0},y_{1},u).&(\langle x_{0}?(z) \rangle(\langle u! \rangle\langle y_{1}!z \rangle C(x_{0},x_{1},y_{0},y_{1},u)) & \\
  & \wedge \langle y_{1}?(z) \rangle (\langle u! \rangle \langle x_{0}!z \rangle C(x_{0},x_{1},y_{0},y_{1},u)) & \\
  & \wedge \langle x_{1}?(z) \rangle (\langle u? \rangle \langle y_{0}!z \rangle C(x_{0},x_{1},y_{0},y_{1},u)) & \\
  & \wedge \langle y_{0}?(z) \rangle (\langle u? \rangle \langle x_{1}!z \rangle C(x_{0},x_{1},y_{0},y_{1},u))) &
\end{align*}

The lexicographical similarity between the shape of this formulae and
the shape of definition of the process representing a crossing reveals
the intuitive meaning of this formulae. It describes the capabilities
of a process that has the right to represent a crossing. For example
it picks out processes that may perform an input on the port $x_0$ in
its initial menu of capabilities. What differentiates the formula
from the process, however, is that the crossing process is the
smallest candidate to satisfy the formula. Infinitely many other
processes -- with internal behavior hidden behind this interface, so
to speak -- also satisfy this formula. Even this simple formula,
then, can be seen to open a new view onto knots, providing a
computational interpretation of \emph{virtual} knots.

Note that this formula is derived by hand. A similar formula can be
derived by employing Caires' calculation of characteristic formula
\cite{Caires04} to the process representing a crossing. In light of
this discussion, we let
$\meaningof{C}_{\phi}(x0,x1,y0,y1,u)$ denote a formula specifying the
dynamics we wish to capture of a crossing. To guarantee we preserve
the shape of the interface and minimal semantics we demand that
$\meaningof{C}_{\phi}(x0,x1,y0,y1,u) \Rightarrow
\textbf{C}(x0,x1,y0,y1,u)$ where $\textbf{C}(x0,x1,y0,y1,u)$ denotes
the formula above.
                            
\subsubsection{Crossing number constraints.}
The moral content of the context lemma (Lemma \ref{context}) is that the notion of
``locality'' in the Reidemeister moves is effectively captured by the
parallel composition operator of the process calculus. This intuition
extends through the logic. Given a formula,
$\meaningof{C}_{\phi}(x0,x1,y0,y1,u)$, we can use the structural
connectives to specify constraints on crossing numbers, such as at
least $n$ crossings, or exactly $n$ crossings.
\begin{mathpar}
  \inferrule* [lab=at-least-n] {} { K^{\geq n}_{\phi}(\vec{xs},\vec{ys}) := \Pi_{i=0}^{n-1} Hu . \meaningof{C}_{\phi}(xs_i,ys_i,u) | T }
  \and 
  \inferrule* [lab=exactly-n] {} { K^{= n}_{\phi}(\vec{xs},\vec{ys}) := \Pi_{i=0}^{n-1} Hu . \meaningof{C}_{\phi}(xs_i,ys_i,u) | \neg (\forall x_0,y_0,x_1,y_1,u . \meaningof{C}_{\phi}(x_0,y_0,x_1,y_1,u) | T) }
\end{mathpar}

To round out this section, recall that the encoding of an $n$-crossing
knot decomposes into a parallel composition of $n$ \emph{copies} of a
crossing process together with a wiring harness. To specify different
knot classes with the same crossing number amounts to specifying
logical constraints on the wiring harness. In the interest of space,
we defer examples to a forthcoming paper. Suffice it to say that both
the conditions ``alternating knot'' and ``contains the tangle
corresponding to 5/3'' are expressible. For example, it is possible to
calculate the characteristic formula of a process corresponding to the
tangle 5/3 and conjoin it into the classifying formula via the
composition connective of the logic.

Finally, we wish to observe that it is entirely within reason to
contemplate a more domain-specific version of spatial logic tailored
to the shape of processes in the image of the encoding. Such a
domain-specific logic would have a better claim to the title formal
language of knot properties.

% subsection example_formulae_ (end)

% section knots_as_processes (end) 

% section spatial logic via knots (end)

\section{Conclusions and future work}

\paragraph{Testing physical space}
You, gentle reader, may wonder why of all the theorems to be proved
given this set up we pick the one above. In some sense it's hardly
central to quantum mechanics. We see it as central in the sense that
it firmly establishes a notion of physical space arising from a notion
of the equivalence of behavior. Relating bisimulation to a metric is a
big step forward, but one is faced with interpreting the relationship
of that metric space to something more physical. Quantum mechanical
notions of ``physical'' space are still far from intuitive, but by
relating this idea of distance as testing to calculations that predict
physical circumstances we are making a not insignificant step forward
toward an understanding of the physical space we inhabit as
essentially dynamic.

\paragraph{Effectivity and simulation}
One of the observations we have yet to make is that the entire program
spelled out here is effective. We have built various interpreters for
the reflective calculus at work in this interpretation. In principle,
then, we can simulate quantum mechanics on a computer. The place where
the simulation may lose fidelity is the infinitely branching summation
for the annihilator.

In this connection i also want to point out that the evaluation style
calculation of the inner product puts the non-determinism of the
summation right at the heart of measurement. This suggests that
Milner's original reduction-based formulation of the dynamics of his
calculi in terms of sums was not just notationally suggestive of a
notion of measure-and-continue but captured some significant part of
the physics.

\paragraph{Quantum continuations}
In light of this last observation i want to point out that the
predominant account of quantum mechanics is missing a key aspect of a
truly compositional story of the physical situation. In a real lab,
when a measurement is made the observation can be made to feed into
another device that then makes another measurement conditioned on the
results of the first. This means that after the superposition was
collapsed the entire experimental set up remained in
superposition. While QM offers a means of writing this down it doesn't
quite line up well with the well-trodden formulation of computation
and continuation that we see so succinctly expressed in Milner's
calculi. This suggests that there might be advantages to this account
of dynamics waiting to be explored.

\paragraph{Quantum logic}
In this connection, we also note that by virtue of having the
Hennessy-Milner construction, we can pull the construction through the
interpretation of QM. This gives us a natural candidate for a quantum
logic that enjoys an extremely tight connection with it's domain of
interpretation, making the construction much less ad hoc (rather it is
the image of functor!).

\paragraph{Quantum probabiity}
i have questions about the basis of the interpretation of inner
product as probability amplitude. In particular, using which
axiomatization of probability theory does the notion of probability
amplitude earn the right to be so dubbed? In other words, where is the
proof that the operation for calculating a probability amplitude (and
then squaring) satisfies the axioms of what it means to calculate a
probability? Even if such a proof exists (i have yet to find it in the
literature), i wonder if it might not be possible to turn things on
their heads. Can we view the calculation of the probability amplitude
as an axiomatization of probability? If so, then the definition we
give for calculating probability amplitude may provide the basis for
an \emph{effective} theory of probability.

\paragraph{Quantum vs ``biological'' information}
Finally, i want to conclude with a more philosophical observation. At
a recent workshop in which QM was a predominant topic i noticed
something about quantum information. The speaker was giving a riveting
discussion of axiomatic QM and showing how properties of ``no
cloning'' and ``no deleting'' emerged as consequences of the
axiomatization. Theorems of this form are necessary to give us a sense
of confidence that our axioms characterize the physical theory. What
struck me, though, was that if quantum information is neither erasable
nor replicable it is markedly different from \emph{life}. Two of the
things we know about life is that

\begin{itemize}
  \item it ends;
  \item to gain some measure of persistence, to transcend it's
    finitude it is imminently copyable.
\end{itemize}

Both of these qualities are summarized succinctly in the aphorism: all
flesh is grass. For me these two kinds of ``information'' -- call them
quantum and biological -- are end points on a spectrum of strategies
for persistence. At one end, we have those curious entities that enjoy
uniqueness and permanence; at the other, we have those who in the face
of a certain end and an uncertain present make a go of passing
something on. To me one of the more remarkable aspects of the latter
strategy is that in the presence of noise (and certain features of
copying) we get a kind of dynamism, a chance for improvement against a
given persistent condition.

% subsection other_calculi_other_bisimulations_and_geometry_as_behavior (end)




% section conclusion (end)

%\documentclass[12pt]{llncs}
%\documentclass{jktr}

\usepackage[pdftex]{hyperref}                   
\usepackage {listings}
\usepackage {mathpartir}
\usepackage{bcprules}
%\usepackage{listings}
                       
\usepackage{graphicx} 
%\usepackage[margins=2.5cm,nohead,nofoot]{geometry}
%\usepackage{geometry}
\usepackage{amsfonts}
\usepackage{amstext}
\usepackage{latexsym}
\usepackage{amssymb}
\usepackage{color}


%\include{myPreamble}
\include{qm2pi.local} 

%\ifpdf
%\usepackage[pdftex]{graphicx}
%\else
%\usepackage{graphicx}
%\fi

 % \ifpdf
%  \usepackage{pdfsync}
%  \if


%\title{Brief Article}
%\author{David F. Snyder}
%\author{L.G. Meredith}

%\address{Dept. of Math., Texas State University--San Marcos, San Marcos, TX 78666}
       
\pagestyle{empty}


\begin{document}

\lstset{language=[Objective]Caml,frame=shadowbox}

\input{qm2pi.front}

% section front matter (end)

\input{qm2pi.intro} 
 
% section introduction (end)

% \input{qm2pi.knotations} 

% section notation (end)

\input{qm2pi.process.calculi} 

% section concurrent_process_calculi_and_spatial_logics_ (end)
    
%\input{qm2pi.knots2pi} 

%\input{qm2pi.trefoil} 

%\input{qm2pi.mainthm} 

% subsection basic_interpretation (end)

%\input{qm2pi.rho.presentation} 
\subsection{The syntax and semantics of the notation system}\label{sub:the_syntax_and_semantics_of_the_notation_system} % (fold)

We now summarize a technical presentation of the calculus that
embodies our theory of dynamics. The typical presentation of such a
calculus follows the style of giving generators and relations on
them. The grammar, below, describing term constructors, freely
generates the set of processes, $\Proc$. This set is then quotiented
by a relation known as structural congruence and it is over this set
that the notion of dynamics is expressed. This presentation is
essentially that of \cite{MeredithR05} with the addition of
polyadicity and summation. For readability we have relegated some of
the technical subtleties to an appendix.

\subsubsection{Process grammar}\label{subsub:process_grammar}

\begin{mathpar}
  \inferrule* [lab=synchronization] {} {{M} \bc \pzero \;|\; x?F \;|\; x!C }
  \and
  \inferrule* [lab=abstraction] {} {{F} \bc (x)P}
  \and
  \inferrule* [lab=concretion] {} {{C} \bc \langle Q \rangle}
  \and
  \inferrule* [lab=process] {} {{P,Q} \bc M \;| \;P|Q \;|\; @{x}}
  \and
  \inferrule* [lab=name] {} {{x} \bc \quotep{P}}
\end{mathpar} 

Note that $\vec{x}$ (resp. $\vec{P}$) denotes a vector of names
(resp. processes) of length $|\vec{x}|$ (resp. $|\vec{P}|$). We adopt
the following useful abbreviations.

\begin{mathpar}
   x?(\vec{y}).P := x.(\vec{y})P \and  x\clift{\vec{P}} := x.\clift{\vec{P}}
   \and x!(y) := \lift{x}{\dropn{y}}
   \and \Pi_{i=0}^{n-1}P_i := P_0 | \ldots | P_{n-1}
\end{mathpar}

\subsubsection{Structural congruence}

\paragraph{Free and bound names and alpha-equivalence.} At the
core of structural equivalence is alpha-equivalence which identifies
process that are the same up to a change of variable. Formally, we
recognize the distinction between free and bound names. The free names
of a process, $\freenames{P}$, may be calculated recursively as
follows:

\begin{mathpar}
\freenames{\pzero} := \emptyset
  \and \\
  \freenames{x?(y).P} := \{ x \} \cup (\freenames{P} \setminus \{ y \})
  \and 
  \freenames{x!\langle P \rangle} := \{ x \} \cup \{ P \} 
  \and \\
  \freenames{P|Q} := \freenames{P} \cup \freenames{Q}
  \and \\
  \freenames{@{x}} := \{ x \}
\end{mathpar}

$\pi$
$\quotep{\pi}$

$\freenames{-} : \pi \to \mathcal{P}(\quotep{\pi})$

\begin{eqnarray*}
  \freenames{\pzero} & := & \emptyset \\
  \freenames{x?(y).P} & := & \{ x \} \cup (\freenames{P} \setminus \{ y \}) \\
  \freenames{x!\langle P \rangle} & := & \{ x \} \cup \{ P \} \\
  \freenames{P|Q} & := & \freenames{P} \cup \freenames{Q} \\
  \freenames{\dropn{x}} & := & \{ x \}
\end{eqnarray*}

The bound names of a process, $\boundnames{P}$, are those names occurring in $P$
that are not free. For example, in $x?(y).0$, the name $x$ is free, while $y$ is bound.

\begin{mathpar}
  \inferrule* [lab=monoidal-laws] {} { P|Q \equiv Q|P \and P|0 \equiv P \and P|(Q|R) \equiv (P|Q)|R }
\end{mathpar}

\begin{mathpar}
  \inferrule* [lab=alpha-equivalence] {} { (x)P \equiv (y)P\{y/x\} \and y \not\in \freenames{P} }
\end{mathpar}

\begin{definition}
Then two processes, $P,Q$, are alpha-equivalent if $P = Q\{\vec{y}/\vec{x}\}$ for
some $\vec{x} \in \boundnames{Q},\vec{y} \in \boundnames{P}$, where $Q\{\vec{y}/\vec{x}\}$
denotes the capture-avoiding substitution of $\vec{y}$ for $\vec{x}$ in $Q$.
\end{definition}

\begin{definition}
  The {\em structural congruence} \cite{SangiorgiWalker} , $\equiv$,
  between processes is the least congruence containing
  alpha-equivalence, satisfying the abelian monoid laws
  (associativity, commutativity and $\pzero$ as identity) for parallel
  composition $|$ and for summation $+$.
\end{definition}

\subsection{Name equivalence}

We take name equivalence, written $\nameeq$, to be the smallest
equivalence relation generated by the following rules.

\begin{mathpar}
\inferrule*[lab=Quote-drop]
{ }
{ \quotep{@{x}} \nameeq x }

\inferrule*[lab=Struct-equiv]
{ P \scong Q }
{ \quotep{P} \nameeq \quotep{Q} }
\end{mathpar}

The astute reader will have noticed that the mutual recursion of names
and processes imposes a mutual recursion on alpha-equivalence and
structural equivalence via name-equivalence. Fortunately, all of this
works out pleasantly and we may calculate in the natural way, free of
concern. The reader interested in the details is referred to the
appendix \ref{appendix:rho_details}.

\subsection{Substitution}

We use $\Proc$ for the set of processes, $\QProc$ for the set of
names, and $\id{\{}\vec{y} / \vec{x} \id{\}}$ to denote partial maps,
$s : \QProc \rightarrow \QProc$. A map, $s$ lifts, uniquely, to a map
on process terms, $\widehat{s} : \Proc \rightarrow \Proc$ by the
following equations.

\begin{mathpar}
  (0) \psubstp{Q}{P} := 0 \\
  (R \juxtap S) \psubstp{Q}{P}
  :=    
  (R)\psubstp{Q}{P} \juxtap (S) \psubstp{Q}{P} \\
  (x?(y).R) \psubstp{Q}{P}    
  :=    
  (x)\substp{Q}{P} (z)\concat( (R \psubstn{z}{y}) \psubstp{Q}{P} ) \\
  (\lift{x}{R}) \psubstp{Q}{P}  
  :=
  \lift{(x)\substp{Q}{P}}{ R \psubstp{Q}{P} } \\
%   (\dropn{x})  \psubstp{Q}{P}       
%   := 
%   \left\{ 
%     \begin{array}{ccc} 
%       \dropn{\quotep{Q}} & & x \nameeq \quotep{P} \\
%       \dropn{x} & & otherwise \\
%     \end{array}
%   \right. 
  (\dropn{x})  \psubstp{Q}{P}       
  := 
  \left\{ 
    \begin{array}{ccc} 
      Q & & x \nameeq \quotep{P} \\
      \dropn{x} & & otherwise \\
    \end{array}
  \right.
\end{mathpar}
 

where

\begin{eqnarray}
  (x)\id{\{} \lpquote Q \rpquote / \lpquote P \rpquote \id{\}}            = 
  \left\{ 
    \begin{array}{ccc}
      \lpquote Q \rpquote & & x \nameeq \lpquote P \rpquote \\
      x & & otherwise \\
    \end{array}
  \right. \nonumber
\end{eqnarray}

and $z$ is chosen distinct from $\quotep{P}$, $\quotep{Q}$, the free
names in $Q$, and all the names in $R$. Our $\alpha$-equivalence will
be built in the standard way from this substitution.

\begin{remark}\label{rem:no_self_referential_names}
  One consequence of these definitions is that $\forall P. \quotep{P}
  \not\in \freenames{P}$.
\end{remark}

\subsection{ Dynamic quote: an example }

Anticipating something of what's to come, consider applying the
substitution, $\widehat{\id{\{}u / z \id{\}}}$, to the following pair
of processes, $\lift{w}{y!(z)}$ and $w[ \lpquote y!(z) \rpquote ]$.

\begin{eqnarray}
	\lift{w}{y!(z)}\widehat{\id{\{}u / z \id{\}}}
		& = &
		\lift{w}{y!(u)} \nonumber\\
	w[ \lpquote y!(z) \rpquote ] \widehat{ \id{\{}u / z \id{\}} }
		& = &
		w[ \lpquote y!(z) \rpquote ] \nonumber
\end{eqnarray}

Because the body of the process between quotes is impervious to
substitution, we get radically different answers. In fact, by
examining the first process in an input context,
e.g. $x?(z).\lift{w}{y!(z)}$, we see that the process under the lift
operator may be shaped by prefixed inputs binding a name inside it. In
this sense, the lift operator will be seen as a way to dynamically
construct processes before reifying them as names.

Finally equipped with these standard features we can present the
dynamics of the calculus.

\subsubsection{Operational semantics} 

Finally, we introduce the computational dynamics. What marks these
algebras as distinct from other more traditionally studied algebraic
structures, e.g. vector spaces or polynomial rings, is the manner in
which dynamics is captured. In traditional structures, dynamics is typically
expressed through morphisms between such structures, as in linear maps
between vector spaces or morphisms between rings. In algebras
associated with the semantics of computation, the dynamics is
expressed as part of the algebraic structure itself, through a
reduction reduction relation typically denoted by $\red$. Below, we
give a recursive presentation of this relation for the calculus used
in the encoding.

$\red \subseteq \pi \times \pi$
$\red : \pi \to \mathcal{P}(\pi)$

\begin{mathpar}
  \inferrule* [lab=Comm] { \textsf{match}( x_{src}, x_{trgt} ) } { x_{trgt}?(y)P \; | \; x_{src}!\langle {Q} \rangle \red P\{\quotep{Q}/y}\} }
  \and \\
  \inferrule* [lab=Par] {{P} \red {P}'} {{{P} | {Q}} \red {{P}' | {Q}}}
  \and
  \inferrule* [lab=Equiv]{{{P} \scong {P}'} \andalso {{P}' \red {Q}'} \andalso {{Q}' \scong {Q}}}{{P} \red {Q}}
\end{mathpar}

\begin{eqnarray*}
  match_{\equiv} (\quotep{P},\quotep{Q}) & := & P \equiv Q \\
  match_{\dagger}(\quotep{P},\quotep{Q}) & := & \forall R. P|Q \red^{*} R => R \red^{*} 0 \\
  match_{K}(\quotep{P},\quotep{Q}) & := & K \mbox{ for some context } K
\end{eqnarray*}

$u?(x)P | u!\langle Q \rangle \red P\{\quotep{Q}/x\}$

%We write $\wred$ for $\red^*$, and $P\red$ if $\exists Q $ such that $ P \red Q$.
We write $P\red$ if $\exists Q $ such that $ P \red Q$ and $P\not\red$, otherwise.

\section{Replication}

As mentioned before, it is known that replication (and hence
recursion) can be implemented in a higher-order process algebra
\cite{SangiorgiWalker}. As our first example of calculation with the
machinery thus far presented we give the construction explicitly in
the {\rhoc}.

\begin{eqnarray}
	D_{x} & := & \prefix{x}{y}{(\binpar{\outputp{x}{y}}{@{y}})} \nonumber\\
	\bangp_{x}{P} & := & \binpar{{x}!\langle{\binpar{D_{x}}{P}}\rangle}{D_{x}} \nonumber
\end{eqnarray}

\begin{eqnarray}
	\bangp_{x}{P} & & \nonumber\\
	=
	& {x}!\langle{(\prefix{x}{y}{(\outputp{x}{y} | @{y})) | P}}\rangle 
	      | \prefix{x}{y}{(\outputp{x}{y} | @{y})} & \nonumber\\
	\red
	& (\outputp{x}{y} | @{y})\substn{\quotep{(\prefix{x}{y}{(@{y} | \outputp{x}{y})) | P}}}{y} & \nonumber\\
	=
	& \outputp{x}{\quotep{(\prefix{x}{y}{(\outputp{x}{y} | @{y})) | P}}}
	  | {(\prefix{x}{y}{(\outputp{x}{y} | @{y})) | P}} & \nonumber\\
	\red
	& \ldots & \nonumber\\
	\red^*
	& P | P | \ldots & \nonumber
\end{eqnarray}

Of course, this encoding, as an implementation, runs away, unfolding
$\bangp{P}$ eagerly. A lazier and more implementable replication
operator, restricted to input-guarded processes, may be obtained as follows.

\begin{eqnarray}
\bangp{\prefix{u}{v}{P}} 
	:= 
	\binpar{\lift{x}{\prefix{u}{v}{(\binpar{D(x)}{P})}}}{D(x)} \nonumber
\end{eqnarray}

\begin{remark}
  Note that the lazier definition still does not deal with summation
  or mixed summation (i.e. sums over input and output). The reader is
  invited to construct definitions of replication that deal with these
  features. 

  Further, the definitions are parameterized in a name, $x$. Can you,
  gentle reader, make a definition that eliminates this parameter and
  guarantees no accidental interaction between the replication
  machinery and the process being replicated -- i.e. no accidental
  sharing of names used by the process to get its work done and the
  name(s) used by the replication to effect copying. This latter
  revision of the definition of replication is crucial to obtaining
  the expected identity $!!P \sim !P$.
\end{remark}

\begin{remark}\label{rem:paradoxical_combinator}
  The reader familiar with the lambda calculus will have noticed the
  similarity between $D$ and the paradoxical combinator.

  [Ed. note: the existence of this seems to suggest we have to be more
  restrictive on the set of processes and names we admit if we are to
  support no-cloning.]
\end{remark}

\subsubsection{Bisimulation}

The computational dynamics gives rise to another kind of equivalence,
the equivalence of computational behavior. As previously mentioned
this is typically captured \emph{via} some form of bisimulation.

% The notion we use in this paper is weak barbed bisimulation
% \cite{milner91polyadicpi}.

The notion we use in this paper is derived from weak barbed
bisimulation \cite{milner91polyadicpi}. 

\begin{definition}
An \emph{observation relation}, $\downarrow_{\mathcal N}$, over a set
of names, $\mathcal N$, is the smallest relation satisfying the rules
below.

\infrule[Out-barb]{y \in {\mathcal N}, \; x \nameeq y}
		  {\outputp{x}{v} \downarrow_{\mathcal N} x}
\infrule[Par-barb]{\mbox{$P\downarrow_{\mathcal N} x$ or $Q\downarrow_{\mathcal N} x$}}
		  {\binpar{P}{Q} \downarrow_{\mathcal N} x}

We write $P \Downarrow_{\mathcal N} x$ if there is $Q$ such that 
$P \wred Q$ and $Q \downarrow_{\mathcal N} x$.
\end{definition}

\begin{definition}
%\label{def.bbisim}
An  ${\mathcal N}$-\emph{barbed bisimulation} over a set of names, ${\mathcal N}$, is a symmetric binary relation 
${\mathcal S}_{\mathcal N}$ between agents such that $P\rel{S}_{\mathcal N}Q$ implies:
\begin{enumerate}
\item If $P \red P'$ then $Q \wred Q'$ and $P'\rel{S}_{\mathcal N} Q'$.
\item If $P\downarrow_{\mathcal N} x$, then $Q\Downarrow_{\mathcal N} x$.
\end{enumerate}
$P$ is ${\mathcal N}$-barbed bisimilar to $Q$, written
$P \wbbisim_{\mathcal N} Q$, if $P \rel{S}_{\mathcal N} Q$ for some ${\mathcal N}$-barbed bisimulation ${\mathcal S}_{\mathcal N}$.
\end{definition}

$\mathcal{R} \subseteq \pi \times \pi$

$P \mathcal{R} Q => \forall P'. P \red P' \Rightarrow \exists Q'. Q \red Q', P' \mathcal{R} Q'$

$P \vdash x \Rightarrow Q \vdash x$

\begin{mathpar}
  \inferrule*[lab=Out-barb]{x \nameeq y}{{y}!\langle{Q}\rangle \vdash x}
  \and
  \inferrule*[lab=Par-barb]{\mbox{$P\vdash x$ or $Q\vdash x$}}{\binpar{P}{Q} \vdash x}
\end{mathpar}

\subsubsection{Contexts}

One of the principle advantages of computational calculi like the
$\pi$-calculus is a well-defined notion of context,
contextual-equivalence and a correlation between
contextual-equivalence and notions of bisimulation. The notion of
context allows the decomposition of a process into (sub-)process and
its syntactic environment, its context. Thus, a context may be
thought of as a process with a ``hole'' (written $\Box$) in it. The
application of a context $M$ to a process $P$, written $M[P]$, is
tantamount to filling the hole in $M$ with $P$. In this paper we do
not need the full weight of this theory, but do make use of the notion
of context in the proof the main theorem. 

\begin{mathpar}
  \inferrule* [lab=summation] {} {{M_{M},M_{N}} \bc \Box \;|\; x.M_{A} \;|\; M_{M}+M_{N}}
  \and
  \inferrule* [lab=agent] {} {{M_{A}} \bc (\vec{x})M_{P} \;| \; \clift{P_0,\ldots,M_{P},\ldots,P_N}}
  \and \\
  \inferrule* [lab=process] {} {{M_{P}} \bc M_{N} \;| \;P|M_{P} }
\end{mathpar} 

\begin{mathpar}
  \inferrule* [lab=sychronization] {} {M_{N} \bc \Box \;|\; x?M_{F} \;|\; x!M_{C}}
  \and
  \inferrule* [lab=abstraction] {} {{M_{F}} \bc (x)M_{P} }
  \and
  \inferrule* [lab=concretion] {} {{M_{C}} \bc \langle M_{P} \rangle }
  \and \\
  \inferrule* [lab=process] {} {{M_{P}} \bc M_{N} \;| \;P|M_{P} }
\end{mathpar}

\begin{definition}[contextual application] Given a context $M$, and
  process $P$, we define the \emph{contextual application}, $M[P] :=
  M\{P/\Box\}$. That is, the contextual application of M to P is the
  substitution of $P$ for $\Box$ in $M$.
\end{definition}

$\meaningof{-} : L \to \mathcal{P}(\pi)$

\begin{mathpar}
  \inferrule* [lab=collection] {} {\meaningof{true} = \pi, \and \meaningof{~E} = \pi \setminus \meaningof{E}, \and \meaningof{E_{1} \& E_{2}} = \meaningof{E_{1}} \cap \meaningof{E_{2}}}
\end{mathpar}

\begin{mathpar}
  \inferrule* [lab=structure] {} {\meaningof{0} = \{ P \in \pi | P \equiv 0 \}, \and \\ \meaningof{E_1 | E_2} = \{ P \in \pi | P \equiv P_{1} | P_{2}, P_{1} \in \meaningof{E_{1}}, P_{2} \in \meaningof{E_2}\} }
\end{mathpar}

\begin{mathpar}
 \inferrule* [lab=behavior] {} {\meaningof{\langle a?b \rangle E} = \{ P \in \pi | P \equiv Q | u?(y)P', \\ \and \\\\ \and \\ \;\;\; u \in \meaningof{a}, \forall z.P'\{z/y\} \in \meaningof{E\{z/b\}}\}, \and \\ \meaningof{a!E} = \{ P \in \pi | P \equiv Q | x!\langle P' \rangle, x \in \meaningof{a} P' \in \meaningof{E}\} }
\end{mathpar}

\begin{mathpar}
 \inferrule* [lab=nominal] {} {\meaningof{\quotep{E}} = \{ \quotep{P} \in \quotep{\pi} | P \in \meaningof{E} \}, \and \meaningof{\quotep{P}} = \{ \quotep{Q} \in \quotep{\pi} | P \equiv Q \} \and \\ \meaningof{@\quotep{E}} = \{ P \in \pi | P \equiv @x, x \in \meaningof{E} \}}
\end{mathpar}

\begin{eqnarray*}
  \\
  \meaningof{-} : TS \to ST
\end{eqnarray*}

\begin{eqnarray*}
  \\
  L : TS \to ST
\end{eqnarray*}

\begin{eqnarray*}
  \\
  P \models E \iff P \in \meaningof{E}
\end{eqnarray*}

\begin{eqnarray*}
  P \approx_{L} Q \iff \forall E \in L. P \models E \iff Q \models E
\end{eqnarray*}

\begin{eqnarray*}
  P \approx_{K} Q
\end{eqnarray*}

\begin{eqnarray*}
  P \approx Q
\end{eqnarray*}

$\approx_{K} = \approx = \approx_{L}$

\subsubsection{Contextual duality}

Note that contexts extend the quotation operation to a family of
operations from processes to names. Given a context, $M$, we can
define a \emph{nominal context}, $\quotep{M}$ by $\quotep{M}[P] :=
\quotep{M[P]}$. To foreshadow what is to come we observe that these
operations enjoy a duality with processes very much like the duality
between vectors and maps from vectors to scalars.

Further, because the calculus is essentially higher-order, we have a
correspondence between contexts and processes. More specifically,
given a name $x$ and a context $M$ we can construct $M^{*}_{x}$ such
that 

\begin{mathpar}
  M^{*}_{x} | \lift{x}{P} \red M[P]
\end{mathpar}

namely,

\begin{mathpar}
  M^{*}_{x} := x?(u).M[\dropn{u}]
\end{mathpar}

The dependence of $M^{*}_{x}$ on a name makes it an abstraction, 

\begin{mathpar}
  M^{*} := (x)x?(u).M[\dropn{u}]
\end{mathpar}

\subsection{Additional notation}

It will sometimes be convenient to denote the process a name
quotes. We already have the notation $x = \quotep{P}$, but it will be
convenient to introduce an alternate notation, $\procn{x}$, when we
want to emphasize the connection to the use of the name. Note that, by
virtue of name equivalence, $\quotep{\procn{x}} \nameeq x$; so, the
notation is consistent with previous definitions.

Further, because names have structure it is possible to effect
substitutions on the basis of that structure. This means we need to
upgrade our notation for substitutions, which we accomplish by
adapting comprehension notation. Thus,

\begin{mathpar}
  P\{ y / x : x \in S \}
\end{mathpar}

is interpreted to mean the process derived from P by replacing (in a
capture-avoiding manner) each occurrence of $x$ in $S$ by $y$. For example,

\begin{mathpar}
  P\{ \quotep{\procn{x}|\procn{x}} / x : x \in \freenames{P} \}
\end{mathpar}

will replace each (occurrence) of a free name $x$ in $P$ by
$\quotep{\procn{x}|\procn{x}}$.

Also, we will avail ourselves of the notation $x^{L}$ and $x^{R}$ to
denote injections of a name into disjoint copies of the name
space. There are numerous ways to accomplish this. One example can be
found in \cite{MeredithR05}. This notation overloads to vectors of
names: $\vec{x}^{\pi} := (x_{i}^{\pi} \; : \; 0 \leq i < |\vec{x}| )$ where $\pi \in \{L,R\}$.

We also use $P^{\Box} := P|\Box$.

In \cite{MeredithR05} an interpretation of the new operator is
given. It turns out that there are several possible interpretations
all enjoying the requisite algebraic properties of the operator (see
\cite{milner91polyadicpi}). We will therefore make liberal use of
$(\nu\; \vec{x})P$.

% subsection the_syntax_and_semantics_of_the_notation_system (end)   

\input{qm2pi.qmops} 

\input{qm2pi.sterngerlach} 

\input{qm2pi.metric} 

% section concurrent_process_calculi (end)

%\input{qm2pi.proofsketch}

% section proof sketch (end)

%\input{qm2pi.slviaknots} 

% section spatial logic via knots (end)

\input{qm2pi.conclusion}

% section conclusion (end)

%\input{qm2pi.dtcodes} 

% section wiring algorithm (end)

\input{qm2pi.ack} 

% section acknowledgments (end)

\newpage


\bibliographystyle{plain}   
\bibliography{../../biblios/main.bib}

\input{qm2pi.rhodetails}

\end{document}

 

% section wiring algorithm (end)

\documentclass[12pt]{llncs}
%\documentclass{jktr}

\usepackage[pdftex]{hyperref}                   
\usepackage {listings}
\usepackage {mathpartir}
\usepackage{bcprules}
%\usepackage{listings}
                       
\usepackage{graphicx} 
%\usepackage[margins=2.5cm,nohead,nofoot]{geometry}
%\usepackage{geometry}
\usepackage{amsfonts}
\usepackage{amstext}
\usepackage{latexsym}
\usepackage{amssymb}
\usepackage{color}


%\include{myPreamble}
\include{qm2pi.local} 

%\ifpdf
%\usepackage[pdftex]{graphicx}
%\else
%\usepackage{graphicx}
%\fi

 % \ifpdf
%  \usepackage{pdfsync}
%  \if


%\title{Brief Article}
%\author{David F. Snyder}
%\author{L.G. Meredith}

%\address{Dept. of Math., Texas State University--San Marcos, San Marcos, TX 78666}
       
\pagestyle{empty}


\begin{document}

\lstset{language=[Objective]Caml,frame=shadowbox}

\input{qm2pi.front}

% section front matter (end)

\input{qm2pi.intro} 
 
% section introduction (end)

% \input{qm2pi.knotations} 

% section notation (end)

\input{qm2pi.process.calculi} 

% section concurrent_process_calculi_and_spatial_logics_ (end)
    
%\input{qm2pi.knots2pi} 

%\input{qm2pi.trefoil} 

%\input{qm2pi.mainthm} 

% subsection basic_interpretation (end)

%\input{qm2pi.rho.presentation} 
\subsection{The syntax and semantics of the notation system}\label{sub:the_syntax_and_semantics_of_the_notation_system} % (fold)

We now summarize a technical presentation of the calculus that
embodies our theory of dynamics. The typical presentation of such a
calculus follows the style of giving generators and relations on
them. The grammar, below, describing term constructors, freely
generates the set of processes, $\Proc$. This set is then quotiented
by a relation known as structural congruence and it is over this set
that the notion of dynamics is expressed. This presentation is
essentially that of \cite{MeredithR05} with the addition of
polyadicity and summation. For readability we have relegated some of
the technical subtleties to an appendix.

\subsubsection{Process grammar}\label{subsub:process_grammar}

\begin{mathpar}
  \inferrule* [lab=synchronization] {} {{M} \bc \pzero \;|\; x?F \;|\; x!C }
  \and
  \inferrule* [lab=abstraction] {} {{F} \bc (x)P}
  \and
  \inferrule* [lab=concretion] {} {{C} \bc \langle Q \rangle}
  \and
  \inferrule* [lab=process] {} {{P,Q} \bc M \;| \;P|Q \;|\; @{x}}
  \and
  \inferrule* [lab=name] {} {{x} \bc \quotep{P}}
\end{mathpar} 

Note that $\vec{x}$ (resp. $\vec{P}$) denotes a vector of names
(resp. processes) of length $|\vec{x}|$ (resp. $|\vec{P}|$). We adopt
the following useful abbreviations.

\begin{mathpar}
   x?(\vec{y}).P := x.(\vec{y})P \and  x\clift{\vec{P}} := x.\clift{\vec{P}}
   \and x!(y) := \lift{x}{\dropn{y}}
   \and \Pi_{i=0}^{n-1}P_i := P_0 | \ldots | P_{n-1}
\end{mathpar}

\subsubsection{Structural congruence}

\paragraph{Free and bound names and alpha-equivalence.} At the
core of structural equivalence is alpha-equivalence which identifies
process that are the same up to a change of variable. Formally, we
recognize the distinction between free and bound names. The free names
of a process, $\freenames{P}$, may be calculated recursively as
follows:

\begin{mathpar}
\freenames{\pzero} := \emptyset
  \and \\
  \freenames{x?(y).P} := \{ x \} \cup (\freenames{P} \setminus \{ y \})
  \and 
  \freenames{x!\langle P \rangle} := \{ x \} \cup \{ P \} 
  \and \\
  \freenames{P|Q} := \freenames{P} \cup \freenames{Q}
  \and \\
  \freenames{@{x}} := \{ x \}
\end{mathpar}

$\pi$
$\quotep{\pi}$

$\freenames{-} : \pi \to \mathcal{P}(\quotep{\pi})$

\begin{eqnarray*}
  \freenames{\pzero} & := & \emptyset \\
  \freenames{x?(y).P} & := & \{ x \} \cup (\freenames{P} \setminus \{ y \}) \\
  \freenames{x!\langle P \rangle} & := & \{ x \} \cup \{ P \} \\
  \freenames{P|Q} & := & \freenames{P} \cup \freenames{Q} \\
  \freenames{\dropn{x}} & := & \{ x \}
\end{eqnarray*}

The bound names of a process, $\boundnames{P}$, are those names occurring in $P$
that are not free. For example, in $x?(y).0$, the name $x$ is free, while $y$ is bound.

\begin{mathpar}
  \inferrule* [lab=monoidal-laws] {} { P|Q \equiv Q|P \and P|0 \equiv P \and P|(Q|R) \equiv (P|Q)|R }
\end{mathpar}

\begin{mathpar}
  \inferrule* [lab=alpha-equivalence] {} { (x)P \equiv (y)P\{y/x\} \and y \not\in \freenames{P} }
\end{mathpar}

\begin{definition}
Then two processes, $P,Q$, are alpha-equivalent if $P = Q\{\vec{y}/\vec{x}\}$ for
some $\vec{x} \in \boundnames{Q},\vec{y} \in \boundnames{P}$, where $Q\{\vec{y}/\vec{x}\}$
denotes the capture-avoiding substitution of $\vec{y}$ for $\vec{x}$ in $Q$.
\end{definition}

\begin{definition}
  The {\em structural congruence} \cite{SangiorgiWalker} , $\equiv$,
  between processes is the least congruence containing
  alpha-equivalence, satisfying the abelian monoid laws
  (associativity, commutativity and $\pzero$ as identity) for parallel
  composition $|$ and for summation $+$.
\end{definition}

\subsection{Name equivalence}

We take name equivalence, written $\nameeq$, to be the smallest
equivalence relation generated by the following rules.

\begin{mathpar}
\inferrule*[lab=Quote-drop]
{ }
{ \quotep{@{x}} \nameeq x }

\inferrule*[lab=Struct-equiv]
{ P \scong Q }
{ \quotep{P} \nameeq \quotep{Q} }
\end{mathpar}

The astute reader will have noticed that the mutual recursion of names
and processes imposes a mutual recursion on alpha-equivalence and
structural equivalence via name-equivalence. Fortunately, all of this
works out pleasantly and we may calculate in the natural way, free of
concern. The reader interested in the details is referred to the
appendix \ref{appendix:rho_details}.

\subsection{Substitution}

We use $\Proc$ for the set of processes, $\QProc$ for the set of
names, and $\id{\{}\vec{y} / \vec{x} \id{\}}$ to denote partial maps,
$s : \QProc \rightarrow \QProc$. A map, $s$ lifts, uniquely, to a map
on process terms, $\widehat{s} : \Proc \rightarrow \Proc$ by the
following equations.

\begin{mathpar}
  (0) \psubstp{Q}{P} := 0 \\
  (R \juxtap S) \psubstp{Q}{P}
  :=    
  (R)\psubstp{Q}{P} \juxtap (S) \psubstp{Q}{P} \\
  (x?(y).R) \psubstp{Q}{P}    
  :=    
  (x)\substp{Q}{P} (z)\concat( (R \psubstn{z}{y}) \psubstp{Q}{P} ) \\
  (\lift{x}{R}) \psubstp{Q}{P}  
  :=
  \lift{(x)\substp{Q}{P}}{ R \psubstp{Q}{P} } \\
%   (\dropn{x})  \psubstp{Q}{P}       
%   := 
%   \left\{ 
%     \begin{array}{ccc} 
%       \dropn{\quotep{Q}} & & x \nameeq \quotep{P} \\
%       \dropn{x} & & otherwise \\
%     \end{array}
%   \right. 
  (\dropn{x})  \psubstp{Q}{P}       
  := 
  \left\{ 
    \begin{array}{ccc} 
      Q & & x \nameeq \quotep{P} \\
      \dropn{x} & & otherwise \\
    \end{array}
  \right.
\end{mathpar}
 

where

\begin{eqnarray}
  (x)\id{\{} \lpquote Q \rpquote / \lpquote P \rpquote \id{\}}            = 
  \left\{ 
    \begin{array}{ccc}
      \lpquote Q \rpquote & & x \nameeq \lpquote P \rpquote \\
      x & & otherwise \\
    \end{array}
  \right. \nonumber
\end{eqnarray}

and $z$ is chosen distinct from $\quotep{P}$, $\quotep{Q}$, the free
names in $Q$, and all the names in $R$. Our $\alpha$-equivalence will
be built in the standard way from this substitution.

\begin{remark}\label{rem:no_self_referential_names}
  One consequence of these definitions is that $\forall P. \quotep{P}
  \not\in \freenames{P}$.
\end{remark}

\subsection{ Dynamic quote: an example }

Anticipating something of what's to come, consider applying the
substitution, $\widehat{\id{\{}u / z \id{\}}}$, to the following pair
of processes, $\lift{w}{y!(z)}$ and $w[ \lpquote y!(z) \rpquote ]$.

\begin{eqnarray}
	\lift{w}{y!(z)}\widehat{\id{\{}u / z \id{\}}}
		& = &
		\lift{w}{y!(u)} \nonumber\\
	w[ \lpquote y!(z) \rpquote ] \widehat{ \id{\{}u / z \id{\}} }
		& = &
		w[ \lpquote y!(z) \rpquote ] \nonumber
\end{eqnarray}

Because the body of the process between quotes is impervious to
substitution, we get radically different answers. In fact, by
examining the first process in an input context,
e.g. $x?(z).\lift{w}{y!(z)}$, we see that the process under the lift
operator may be shaped by prefixed inputs binding a name inside it. In
this sense, the lift operator will be seen as a way to dynamically
construct processes before reifying them as names.

Finally equipped with these standard features we can present the
dynamics of the calculus.

\subsubsection{Operational semantics} 

Finally, we introduce the computational dynamics. What marks these
algebras as distinct from other more traditionally studied algebraic
structures, e.g. vector spaces or polynomial rings, is the manner in
which dynamics is captured. In traditional structures, dynamics is typically
expressed through morphisms between such structures, as in linear maps
between vector spaces or morphisms between rings. In algebras
associated with the semantics of computation, the dynamics is
expressed as part of the algebraic structure itself, through a
reduction reduction relation typically denoted by $\red$. Below, we
give a recursive presentation of this relation for the calculus used
in the encoding.

$\red \subseteq \pi \times \pi$
$\red : \pi \to \mathcal{P}(\pi)$

\begin{mathpar}
  \inferrule* [lab=Comm] { \textsf{match}( x_{src}, x_{trgt} ) } { x_{trgt}?(y)P \; | \; x_{src}!\langle {Q} \rangle \red P\{\quotep{Q}/y}\} }
  \and \\
  \inferrule* [lab=Par] {{P} \red {P}'} {{{P} | {Q}} \red {{P}' | {Q}}}
  \and
  \inferrule* [lab=Equiv]{{{P} \scong {P}'} \andalso {{P}' \red {Q}'} \andalso {{Q}' \scong {Q}}}{{P} \red {Q}}
\end{mathpar}

\begin{eqnarray*}
  match_{\equiv} (\quotep{P},\quotep{Q}) & := & P \equiv Q \\
  match_{\dagger}(\quotep{P},\quotep{Q}) & := & \forall R. P|Q \red^{*} R => R \red^{*} 0 \\
  match_{K}(\quotep{P},\quotep{Q}) & := & K \mbox{ for some context } K
\end{eqnarray*}

$u?(x)P | u!\langle Q \rangle \red P\{\quotep{Q}/x\}$

%We write $\wred$ for $\red^*$, and $P\red$ if $\exists Q $ such that $ P \red Q$.
We write $P\red$ if $\exists Q $ such that $ P \red Q$ and $P\not\red$, otherwise.

\section{Replication}

As mentioned before, it is known that replication (and hence
recursion) can be implemented in a higher-order process algebra
\cite{SangiorgiWalker}. As our first example of calculation with the
machinery thus far presented we give the construction explicitly in
the {\rhoc}.

\begin{eqnarray}
	D_{x} & := & \prefix{x}{y}{(\binpar{\outputp{x}{y}}{@{y}})} \nonumber\\
	\bangp_{x}{P} & := & \binpar{{x}!\langle{\binpar{D_{x}}{P}}\rangle}{D_{x}} \nonumber
\end{eqnarray}

\begin{eqnarray}
	\bangp_{x}{P} & & \nonumber\\
	=
	& {x}!\langle{(\prefix{x}{y}{(\outputp{x}{y} | @{y})) | P}}\rangle 
	      | \prefix{x}{y}{(\outputp{x}{y} | @{y})} & \nonumber\\
	\red
	& (\outputp{x}{y} | @{y})\substn{\quotep{(\prefix{x}{y}{(@{y} | \outputp{x}{y})) | P}}}{y} & \nonumber\\
	=
	& \outputp{x}{\quotep{(\prefix{x}{y}{(\outputp{x}{y} | @{y})) | P}}}
	  | {(\prefix{x}{y}{(\outputp{x}{y} | @{y})) | P}} & \nonumber\\
	\red
	& \ldots & \nonumber\\
	\red^*
	& P | P | \ldots & \nonumber
\end{eqnarray}

Of course, this encoding, as an implementation, runs away, unfolding
$\bangp{P}$ eagerly. A lazier and more implementable replication
operator, restricted to input-guarded processes, may be obtained as follows.

\begin{eqnarray}
\bangp{\prefix{u}{v}{P}} 
	:= 
	\binpar{\lift{x}{\prefix{u}{v}{(\binpar{D(x)}{P})}}}{D(x)} \nonumber
\end{eqnarray}

\begin{remark}
  Note that the lazier definition still does not deal with summation
  or mixed summation (i.e. sums over input and output). The reader is
  invited to construct definitions of replication that deal with these
  features. 

  Further, the definitions are parameterized in a name, $x$. Can you,
  gentle reader, make a definition that eliminates this parameter and
  guarantees no accidental interaction between the replication
  machinery and the process being replicated -- i.e. no accidental
  sharing of names used by the process to get its work done and the
  name(s) used by the replication to effect copying. This latter
  revision of the definition of replication is crucial to obtaining
  the expected identity $!!P \sim !P$.
\end{remark}

\begin{remark}\label{rem:paradoxical_combinator}
  The reader familiar with the lambda calculus will have noticed the
  similarity between $D$ and the paradoxical combinator.

  [Ed. note: the existence of this seems to suggest we have to be more
  restrictive on the set of processes and names we admit if we are to
  support no-cloning.]
\end{remark}

\subsubsection{Bisimulation}

The computational dynamics gives rise to another kind of equivalence,
the equivalence of computational behavior. As previously mentioned
this is typically captured \emph{via} some form of bisimulation.

% The notion we use in this paper is weak barbed bisimulation
% \cite{milner91polyadicpi}.

The notion we use in this paper is derived from weak barbed
bisimulation \cite{milner91polyadicpi}. 

\begin{definition}
An \emph{observation relation}, $\downarrow_{\mathcal N}$, over a set
of names, $\mathcal N$, is the smallest relation satisfying the rules
below.

\infrule[Out-barb]{y \in {\mathcal N}, \; x \nameeq y}
		  {\outputp{x}{v} \downarrow_{\mathcal N} x}
\infrule[Par-barb]{\mbox{$P\downarrow_{\mathcal N} x$ or $Q\downarrow_{\mathcal N} x$}}
		  {\binpar{P}{Q} \downarrow_{\mathcal N} x}

We write $P \Downarrow_{\mathcal N} x$ if there is $Q$ such that 
$P \wred Q$ and $Q \downarrow_{\mathcal N} x$.
\end{definition}

\begin{definition}
%\label{def.bbisim}
An  ${\mathcal N}$-\emph{barbed bisimulation} over a set of names, ${\mathcal N}$, is a symmetric binary relation 
${\mathcal S}_{\mathcal N}$ between agents such that $P\rel{S}_{\mathcal N}Q$ implies:
\begin{enumerate}
\item If $P \red P'$ then $Q \wred Q'$ and $P'\rel{S}_{\mathcal N} Q'$.
\item If $P\downarrow_{\mathcal N} x$, then $Q\Downarrow_{\mathcal N} x$.
\end{enumerate}
$P$ is ${\mathcal N}$-barbed bisimilar to $Q$, written
$P \wbbisim_{\mathcal N} Q$, if $P \rel{S}_{\mathcal N} Q$ for some ${\mathcal N}$-barbed bisimulation ${\mathcal S}_{\mathcal N}$.
\end{definition}

$\mathcal{R} \subseteq \pi \times \pi$

$P \mathcal{R} Q => \forall P'. P \red P' \Rightarrow \exists Q'. Q \red Q', P' \mathcal{R} Q'$

$P \vdash x \Rightarrow Q \vdash x$

\begin{mathpar}
  \inferrule*[lab=Out-barb]{x \nameeq y}{{y}!\langle{Q}\rangle \vdash x}
  \and
  \inferrule*[lab=Par-barb]{\mbox{$P\vdash x$ or $Q\vdash x$}}{\binpar{P}{Q} \vdash x}
\end{mathpar}

\subsubsection{Contexts}

One of the principle advantages of computational calculi like the
$\pi$-calculus is a well-defined notion of context,
contextual-equivalence and a correlation between
contextual-equivalence and notions of bisimulation. The notion of
context allows the decomposition of a process into (sub-)process and
its syntactic environment, its context. Thus, a context may be
thought of as a process with a ``hole'' (written $\Box$) in it. The
application of a context $M$ to a process $P$, written $M[P]$, is
tantamount to filling the hole in $M$ with $P$. In this paper we do
not need the full weight of this theory, but do make use of the notion
of context in the proof the main theorem. 

\begin{mathpar}
  \inferrule* [lab=summation] {} {{M_{M},M_{N}} \bc \Box \;|\; x.M_{A} \;|\; M_{M}+M_{N}}
  \and
  \inferrule* [lab=agent] {} {{M_{A}} \bc (\vec{x})M_{P} \;| \; \clift{P_0,\ldots,M_{P},\ldots,P_N}}
  \and \\
  \inferrule* [lab=process] {} {{M_{P}} \bc M_{N} \;| \;P|M_{P} }
\end{mathpar} 

\begin{mathpar}
  \inferrule* [lab=sychronization] {} {M_{N} \bc \Box \;|\; x?M_{F} \;|\; x!M_{C}}
  \and
  \inferrule* [lab=abstraction] {} {{M_{F}} \bc (x)M_{P} }
  \and
  \inferrule* [lab=concretion] {} {{M_{C}} \bc \langle M_{P} \rangle }
  \and \\
  \inferrule* [lab=process] {} {{M_{P}} \bc M_{N} \;| \;P|M_{P} }
\end{mathpar}

\begin{definition}[contextual application] Given a context $M$, and
  process $P$, we define the \emph{contextual application}, $M[P] :=
  M\{P/\Box\}$. That is, the contextual application of M to P is the
  substitution of $P$ for $\Box$ in $M$.
\end{definition}

$\meaningof{-} : L \to \mathcal{P}(\pi)$

\begin{mathpar}
  \inferrule* [lab=collection] {} {\meaningof{true} = \pi, \and \meaningof{~E} = \pi \setminus \meaningof{E}, \and \meaningof{E_{1} \& E_{2}} = \meaningof{E_{1}} \cap \meaningof{E_{2}}}
\end{mathpar}

\begin{mathpar}
  \inferrule* [lab=structure] {} {\meaningof{0} = \{ P \in \pi | P \equiv 0 \}, \and \\ \meaningof{E_1 | E_2} = \{ P \in \pi | P \equiv P_{1} | P_{2}, P_{1} \in \meaningof{E_{1}}, P_{2} \in \meaningof{E_2}\} }
\end{mathpar}

\begin{mathpar}
 \inferrule* [lab=behavior] {} {\meaningof{\langle a?b \rangle E} = \{ P \in \pi | P \equiv Q | u?(y)P', \\ \and \\\\ \and \\ \;\;\; u \in \meaningof{a}, \forall z.P'\{z/y\} \in \meaningof{E\{z/b\}}\}, \and \\ \meaningof{a!E} = \{ P \in \pi | P \equiv Q | x!\langle P' \rangle, x \in \meaningof{a} P' \in \meaningof{E}\} }
\end{mathpar}

\begin{mathpar}
 \inferrule* [lab=nominal] {} {\meaningof{\quotep{E}} = \{ \quotep{P} \in \quotep{\pi} | P \in \meaningof{E} \}, \and \meaningof{\quotep{P}} = \{ \quotep{Q} \in \quotep{\pi} | P \equiv Q \} \and \\ \meaningof{@\quotep{E}} = \{ P \in \pi | P \equiv @x, x \in \meaningof{E} \}}
\end{mathpar}

\begin{eqnarray*}
  \\
  \meaningof{-} : TS \to ST
\end{eqnarray*}

\begin{eqnarray*}
  \\
  L : TS \to ST
\end{eqnarray*}

\begin{eqnarray*}
  \\
  P \models E \iff P \in \meaningof{E}
\end{eqnarray*}

\begin{eqnarray*}
  P \approx_{L} Q \iff \forall E \in L. P \models E \iff Q \models E
\end{eqnarray*}

\begin{eqnarray*}
  P \approx_{K} Q
\end{eqnarray*}

\begin{eqnarray*}
  P \approx Q
\end{eqnarray*}

$\approx_{K} = \approx = \approx_{L}$

\subsubsection{Contextual duality}

Note that contexts extend the quotation operation to a family of
operations from processes to names. Given a context, $M$, we can
define a \emph{nominal context}, $\quotep{M}$ by $\quotep{M}[P] :=
\quotep{M[P]}$. To foreshadow what is to come we observe that these
operations enjoy a duality with processes very much like the duality
between vectors and maps from vectors to scalars.

Further, because the calculus is essentially higher-order, we have a
correspondence between contexts and processes. More specifically,
given a name $x$ and a context $M$ we can construct $M^{*}_{x}$ such
that 

\begin{mathpar}
  M^{*}_{x} | \lift{x}{P} \red M[P]
\end{mathpar}

namely,

\begin{mathpar}
  M^{*}_{x} := x?(u).M[\dropn{u}]
\end{mathpar}

The dependence of $M^{*}_{x}$ on a name makes it an abstraction, 

\begin{mathpar}
  M^{*} := (x)x?(u).M[\dropn{u}]
\end{mathpar}

\subsection{Additional notation}

It will sometimes be convenient to denote the process a name
quotes. We already have the notation $x = \quotep{P}$, but it will be
convenient to introduce an alternate notation, $\procn{x}$, when we
want to emphasize the connection to the use of the name. Note that, by
virtue of name equivalence, $\quotep{\procn{x}} \nameeq x$; so, the
notation is consistent with previous definitions.

Further, because names have structure it is possible to effect
substitutions on the basis of that structure. This means we need to
upgrade our notation for substitutions, which we accomplish by
adapting comprehension notation. Thus,

\begin{mathpar}
  P\{ y / x : x \in S \}
\end{mathpar}

is interpreted to mean the process derived from P by replacing (in a
capture-avoiding manner) each occurrence of $x$ in $S$ by $y$. For example,

\begin{mathpar}
  P\{ \quotep{\procn{x}|\procn{x}} / x : x \in \freenames{P} \}
\end{mathpar}

will replace each (occurrence) of a free name $x$ in $P$ by
$\quotep{\procn{x}|\procn{x}}$.

Also, we will avail ourselves of the notation $x^{L}$ and $x^{R}$ to
denote injections of a name into disjoint copies of the name
space. There are numerous ways to accomplish this. One example can be
found in \cite{MeredithR05}. This notation overloads to vectors of
names: $\vec{x}^{\pi} := (x_{i}^{\pi} \; : \; 0 \leq i < |\vec{x}| )$ where $\pi \in \{L,R\}$.

We also use $P^{\Box} := P|\Box$.

In \cite{MeredithR05} an interpretation of the new operator is
given. It turns out that there are several possible interpretations
all enjoying the requisite algebraic properties of the operator (see
\cite{milner91polyadicpi}). We will therefore make liberal use of
$(\nu\; \vec{x})P$.

% subsection the_syntax_and_semantics_of_the_notation_system (end)   

\input{qm2pi.qmops} 

\input{qm2pi.sterngerlach} 

\input{qm2pi.metric} 

% section concurrent_process_calculi (end)

%\input{qm2pi.proofsketch}

% section proof sketch (end)

%\input{qm2pi.slviaknots} 

% section spatial logic via knots (end)

\input{qm2pi.conclusion}

% section conclusion (end)

%\input{qm2pi.dtcodes} 

% section wiring algorithm (end)

\input{qm2pi.ack} 

% section acknowledgments (end)

\newpage


\bibliographystyle{plain}   
\bibliography{../../biblios/main.bib}

\input{qm2pi.rhodetails}

\end{document}

 

% section acknowledgments (end)

\newpage


\bibliographystyle{plain}   
\bibliography{../../biblios/main.bib}

\documentclass[12pt]{llncs}
%\documentclass{jktr}

\usepackage[pdftex]{hyperref}                   
\usepackage {listings}
\usepackage {mathpartir}
\usepackage{bcprules}
%\usepackage{listings}
                       
\usepackage{graphicx} 
%\usepackage[margins=2.5cm,nohead,nofoot]{geometry}
%\usepackage{geometry}
\usepackage{amsfonts}
\usepackage{amstext}
\usepackage{latexsym}
\usepackage{amssymb}
\usepackage{color}


%\include{myPreamble}
\include{qm2pi.local} 

%\ifpdf
%\usepackage[pdftex]{graphicx}
%\else
%\usepackage{graphicx}
%\fi

 % \ifpdf
%  \usepackage{pdfsync}
%  \if


%\title{Brief Article}
%\author{David F. Snyder}
%\author{L.G. Meredith}

%\address{Dept. of Math., Texas State University--San Marcos, San Marcos, TX 78666}
       
\pagestyle{empty}


\begin{document}

\lstset{language=[Objective]Caml,frame=shadowbox}

\input{qm2pi.front}

% section front matter (end)

\input{qm2pi.intro} 
 
% section introduction (end)

% \input{qm2pi.knotations} 

% section notation (end)

\input{qm2pi.process.calculi} 

% section concurrent_process_calculi_and_spatial_logics_ (end)
    
%\input{qm2pi.knots2pi} 

%\input{qm2pi.trefoil} 

%\input{qm2pi.mainthm} 

% subsection basic_interpretation (end)

%\input{qm2pi.rho.presentation} 
\subsection{The syntax and semantics of the notation system}\label{sub:the_syntax_and_semantics_of_the_notation_system} % (fold)

We now summarize a technical presentation of the calculus that
embodies our theory of dynamics. The typical presentation of such a
calculus follows the style of giving generators and relations on
them. The grammar, below, describing term constructors, freely
generates the set of processes, $\Proc$. This set is then quotiented
by a relation known as structural congruence and it is over this set
that the notion of dynamics is expressed. This presentation is
essentially that of \cite{MeredithR05} with the addition of
polyadicity and summation. For readability we have relegated some of
the technical subtleties to an appendix.

\subsubsection{Process grammar}\label{subsub:process_grammar}

\begin{mathpar}
  \inferrule* [lab=synchronization] {} {{M} \bc \pzero \;|\; x?F \;|\; x!C }
  \and
  \inferrule* [lab=abstraction] {} {{F} \bc (x)P}
  \and
  \inferrule* [lab=concretion] {} {{C} \bc \langle Q \rangle}
  \and
  \inferrule* [lab=process] {} {{P,Q} \bc M \;| \;P|Q \;|\; @{x}}
  \and
  \inferrule* [lab=name] {} {{x} \bc \quotep{P}}
\end{mathpar} 

Note that $\vec{x}$ (resp. $\vec{P}$) denotes a vector of names
(resp. processes) of length $|\vec{x}|$ (resp. $|\vec{P}|$). We adopt
the following useful abbreviations.

\begin{mathpar}
   x?(\vec{y}).P := x.(\vec{y})P \and  x\clift{\vec{P}} := x.\clift{\vec{P}}
   \and x!(y) := \lift{x}{\dropn{y}}
   \and \Pi_{i=0}^{n-1}P_i := P_0 | \ldots | P_{n-1}
\end{mathpar}

\subsubsection{Structural congruence}

\paragraph{Free and bound names and alpha-equivalence.} At the
core of structural equivalence is alpha-equivalence which identifies
process that are the same up to a change of variable. Formally, we
recognize the distinction between free and bound names. The free names
of a process, $\freenames{P}$, may be calculated recursively as
follows:

\begin{mathpar}
\freenames{\pzero} := \emptyset
  \and \\
  \freenames{x?(y).P} := \{ x \} \cup (\freenames{P} \setminus \{ y \})
  \and 
  \freenames{x!\langle P \rangle} := \{ x \} \cup \{ P \} 
  \and \\
  \freenames{P|Q} := \freenames{P} \cup \freenames{Q}
  \and \\
  \freenames{@{x}} := \{ x \}
\end{mathpar}

$\pi$
$\quotep{\pi}$

$\freenames{-} : \pi \to \mathcal{P}(\quotep{\pi})$

\begin{eqnarray*}
  \freenames{\pzero} & := & \emptyset \\
  \freenames{x?(y).P} & := & \{ x \} \cup (\freenames{P} \setminus \{ y \}) \\
  \freenames{x!\langle P \rangle} & := & \{ x \} \cup \{ P \} \\
  \freenames{P|Q} & := & \freenames{P} \cup \freenames{Q} \\
  \freenames{\dropn{x}} & := & \{ x \}
\end{eqnarray*}

The bound names of a process, $\boundnames{P}$, are those names occurring in $P$
that are not free. For example, in $x?(y).0$, the name $x$ is free, while $y$ is bound.

\begin{mathpar}
  \inferrule* [lab=monoidal-laws] {} { P|Q \equiv Q|P \and P|0 \equiv P \and P|(Q|R) \equiv (P|Q)|R }
\end{mathpar}

\begin{mathpar}
  \inferrule* [lab=alpha-equivalence] {} { (x)P \equiv (y)P\{y/x\} \and y \not\in \freenames{P} }
\end{mathpar}

\begin{definition}
Then two processes, $P,Q$, are alpha-equivalent if $P = Q\{\vec{y}/\vec{x}\}$ for
some $\vec{x} \in \boundnames{Q},\vec{y} \in \boundnames{P}$, where $Q\{\vec{y}/\vec{x}\}$
denotes the capture-avoiding substitution of $\vec{y}$ for $\vec{x}$ in $Q$.
\end{definition}

\begin{definition}
  The {\em structural congruence} \cite{SangiorgiWalker} , $\equiv$,
  between processes is the least congruence containing
  alpha-equivalence, satisfying the abelian monoid laws
  (associativity, commutativity and $\pzero$ as identity) for parallel
  composition $|$ and for summation $+$.
\end{definition}

\subsection{Name equivalence}

We take name equivalence, written $\nameeq$, to be the smallest
equivalence relation generated by the following rules.

\begin{mathpar}
\inferrule*[lab=Quote-drop]
{ }
{ \quotep{@{x}} \nameeq x }

\inferrule*[lab=Struct-equiv]
{ P \scong Q }
{ \quotep{P} \nameeq \quotep{Q} }
\end{mathpar}

The astute reader will have noticed that the mutual recursion of names
and processes imposes a mutual recursion on alpha-equivalence and
structural equivalence via name-equivalence. Fortunately, all of this
works out pleasantly and we may calculate in the natural way, free of
concern. The reader interested in the details is referred to the
appendix \ref{appendix:rho_details}.

\subsection{Substitution}

We use $\Proc$ for the set of processes, $\QProc$ for the set of
names, and $\id{\{}\vec{y} / \vec{x} \id{\}}$ to denote partial maps,
$s : \QProc \rightarrow \QProc$. A map, $s$ lifts, uniquely, to a map
on process terms, $\widehat{s} : \Proc \rightarrow \Proc$ by the
following equations.

\begin{mathpar}
  (0) \psubstp{Q}{P} := 0 \\
  (R \juxtap S) \psubstp{Q}{P}
  :=    
  (R)\psubstp{Q}{P} \juxtap (S) \psubstp{Q}{P} \\
  (x?(y).R) \psubstp{Q}{P}    
  :=    
  (x)\substp{Q}{P} (z)\concat( (R \psubstn{z}{y}) \psubstp{Q}{P} ) \\
  (\lift{x}{R}) \psubstp{Q}{P}  
  :=
  \lift{(x)\substp{Q}{P}}{ R \psubstp{Q}{P} } \\
%   (\dropn{x})  \psubstp{Q}{P}       
%   := 
%   \left\{ 
%     \begin{array}{ccc} 
%       \dropn{\quotep{Q}} & & x \nameeq \quotep{P} \\
%       \dropn{x} & & otherwise \\
%     \end{array}
%   \right. 
  (\dropn{x})  \psubstp{Q}{P}       
  := 
  \left\{ 
    \begin{array}{ccc} 
      Q & & x \nameeq \quotep{P} \\
      \dropn{x} & & otherwise \\
    \end{array}
  \right.
\end{mathpar}
 

where

\begin{eqnarray}
  (x)\id{\{} \lpquote Q \rpquote / \lpquote P \rpquote \id{\}}            = 
  \left\{ 
    \begin{array}{ccc}
      \lpquote Q \rpquote & & x \nameeq \lpquote P \rpquote \\
      x & & otherwise \\
    \end{array}
  \right. \nonumber
\end{eqnarray}

and $z$ is chosen distinct from $\quotep{P}$, $\quotep{Q}$, the free
names in $Q$, and all the names in $R$. Our $\alpha$-equivalence will
be built in the standard way from this substitution.

\begin{remark}\label{rem:no_self_referential_names}
  One consequence of these definitions is that $\forall P. \quotep{P}
  \not\in \freenames{P}$.
\end{remark}

\subsection{ Dynamic quote: an example }

Anticipating something of what's to come, consider applying the
substitution, $\widehat{\id{\{}u / z \id{\}}}$, to the following pair
of processes, $\lift{w}{y!(z)}$ and $w[ \lpquote y!(z) \rpquote ]$.

\begin{eqnarray}
	\lift{w}{y!(z)}\widehat{\id{\{}u / z \id{\}}}
		& = &
		\lift{w}{y!(u)} \nonumber\\
	w[ \lpquote y!(z) \rpquote ] \widehat{ \id{\{}u / z \id{\}} }
		& = &
		w[ \lpquote y!(z) \rpquote ] \nonumber
\end{eqnarray}

Because the body of the process between quotes is impervious to
substitution, we get radically different answers. In fact, by
examining the first process in an input context,
e.g. $x?(z).\lift{w}{y!(z)}$, we see that the process under the lift
operator may be shaped by prefixed inputs binding a name inside it. In
this sense, the lift operator will be seen as a way to dynamically
construct processes before reifying them as names.

Finally equipped with these standard features we can present the
dynamics of the calculus.

\subsubsection{Operational semantics} 

Finally, we introduce the computational dynamics. What marks these
algebras as distinct from other more traditionally studied algebraic
structures, e.g. vector spaces or polynomial rings, is the manner in
which dynamics is captured. In traditional structures, dynamics is typically
expressed through morphisms between such structures, as in linear maps
between vector spaces or morphisms between rings. In algebras
associated with the semantics of computation, the dynamics is
expressed as part of the algebraic structure itself, through a
reduction reduction relation typically denoted by $\red$. Below, we
give a recursive presentation of this relation for the calculus used
in the encoding.

$\red \subseteq \pi \times \pi$
$\red : \pi \to \mathcal{P}(\pi)$

\begin{mathpar}
  \inferrule* [lab=Comm] { \textsf{match}( x_{src}, x_{trgt} ) } { x_{trgt}?(y)P \; | \; x_{src}!\langle {Q} \rangle \red P\{\quotep{Q}/y}\} }
  \and \\
  \inferrule* [lab=Par] {{P} \red {P}'} {{{P} | {Q}} \red {{P}' | {Q}}}
  \and
  \inferrule* [lab=Equiv]{{{P} \scong {P}'} \andalso {{P}' \red {Q}'} \andalso {{Q}' \scong {Q}}}{{P} \red {Q}}
\end{mathpar}

\begin{eqnarray*}
  match_{\equiv} (\quotep{P},\quotep{Q}) & := & P \equiv Q \\
  match_{\dagger}(\quotep{P},\quotep{Q}) & := & \forall R. P|Q \red^{*} R => R \red^{*} 0 \\
  match_{K}(\quotep{P},\quotep{Q}) & := & K \mbox{ for some context } K
\end{eqnarray*}

$u?(x)P | u!\langle Q \rangle \red P\{\quotep{Q}/x\}$

%We write $\wred$ for $\red^*$, and $P\red$ if $\exists Q $ such that $ P \red Q$.
We write $P\red$ if $\exists Q $ such that $ P \red Q$ and $P\not\red$, otherwise.

\section{Replication}

As mentioned before, it is known that replication (and hence
recursion) can be implemented in a higher-order process algebra
\cite{SangiorgiWalker}. As our first example of calculation with the
machinery thus far presented we give the construction explicitly in
the {\rhoc}.

\begin{eqnarray}
	D_{x} & := & \prefix{x}{y}{(\binpar{\outputp{x}{y}}{@{y}})} \nonumber\\
	\bangp_{x}{P} & := & \binpar{{x}!\langle{\binpar{D_{x}}{P}}\rangle}{D_{x}} \nonumber
\end{eqnarray}

\begin{eqnarray}
	\bangp_{x}{P} & & \nonumber\\
	=
	& {x}!\langle{(\prefix{x}{y}{(\outputp{x}{y} | @{y})) | P}}\rangle 
	      | \prefix{x}{y}{(\outputp{x}{y} | @{y})} & \nonumber\\
	\red
	& (\outputp{x}{y} | @{y})\substn{\quotep{(\prefix{x}{y}{(@{y} | \outputp{x}{y})) | P}}}{y} & \nonumber\\
	=
	& \outputp{x}{\quotep{(\prefix{x}{y}{(\outputp{x}{y} | @{y})) | P}}}
	  | {(\prefix{x}{y}{(\outputp{x}{y} | @{y})) | P}} & \nonumber\\
	\red
	& \ldots & \nonumber\\
	\red^*
	& P | P | \ldots & \nonumber
\end{eqnarray}

Of course, this encoding, as an implementation, runs away, unfolding
$\bangp{P}$ eagerly. A lazier and more implementable replication
operator, restricted to input-guarded processes, may be obtained as follows.

\begin{eqnarray}
\bangp{\prefix{u}{v}{P}} 
	:= 
	\binpar{\lift{x}{\prefix{u}{v}{(\binpar{D(x)}{P})}}}{D(x)} \nonumber
\end{eqnarray}

\begin{remark}
  Note that the lazier definition still does not deal with summation
  or mixed summation (i.e. sums over input and output). The reader is
  invited to construct definitions of replication that deal with these
  features. 

  Further, the definitions are parameterized in a name, $x$. Can you,
  gentle reader, make a definition that eliminates this parameter and
  guarantees no accidental interaction between the replication
  machinery and the process being replicated -- i.e. no accidental
  sharing of names used by the process to get its work done and the
  name(s) used by the replication to effect copying. This latter
  revision of the definition of replication is crucial to obtaining
  the expected identity $!!P \sim !P$.
\end{remark}

\begin{remark}\label{rem:paradoxical_combinator}
  The reader familiar with the lambda calculus will have noticed the
  similarity between $D$ and the paradoxical combinator.

  [Ed. note: the existence of this seems to suggest we have to be more
  restrictive on the set of processes and names we admit if we are to
  support no-cloning.]
\end{remark}

\subsubsection{Bisimulation}

The computational dynamics gives rise to another kind of equivalence,
the equivalence of computational behavior. As previously mentioned
this is typically captured \emph{via} some form of bisimulation.

% The notion we use in this paper is weak barbed bisimulation
% \cite{milner91polyadicpi}.

The notion we use in this paper is derived from weak barbed
bisimulation \cite{milner91polyadicpi}. 

\begin{definition}
An \emph{observation relation}, $\downarrow_{\mathcal N}$, over a set
of names, $\mathcal N$, is the smallest relation satisfying the rules
below.

\infrule[Out-barb]{y \in {\mathcal N}, \; x \nameeq y}
		  {\outputp{x}{v} \downarrow_{\mathcal N} x}
\infrule[Par-barb]{\mbox{$P\downarrow_{\mathcal N} x$ or $Q\downarrow_{\mathcal N} x$}}
		  {\binpar{P}{Q} \downarrow_{\mathcal N} x}

We write $P \Downarrow_{\mathcal N} x$ if there is $Q$ such that 
$P \wred Q$ and $Q \downarrow_{\mathcal N} x$.
\end{definition}

\begin{definition}
%\label{def.bbisim}
An  ${\mathcal N}$-\emph{barbed bisimulation} over a set of names, ${\mathcal N}$, is a symmetric binary relation 
${\mathcal S}_{\mathcal N}$ between agents such that $P\rel{S}_{\mathcal N}Q$ implies:
\begin{enumerate}
\item If $P \red P'$ then $Q \wred Q'$ and $P'\rel{S}_{\mathcal N} Q'$.
\item If $P\downarrow_{\mathcal N} x$, then $Q\Downarrow_{\mathcal N} x$.
\end{enumerate}
$P$ is ${\mathcal N}$-barbed bisimilar to $Q$, written
$P \wbbisim_{\mathcal N} Q$, if $P \rel{S}_{\mathcal N} Q$ for some ${\mathcal N}$-barbed bisimulation ${\mathcal S}_{\mathcal N}$.
\end{definition}

$\mathcal{R} \subseteq \pi \times \pi$

$P \mathcal{R} Q => \forall P'. P \red P' \Rightarrow \exists Q'. Q \red Q', P' \mathcal{R} Q'$

$P \vdash x \Rightarrow Q \vdash x$

\begin{mathpar}
  \inferrule*[lab=Out-barb]{x \nameeq y}{{y}!\langle{Q}\rangle \vdash x}
  \and
  \inferrule*[lab=Par-barb]{\mbox{$P\vdash x$ or $Q\vdash x$}}{\binpar{P}{Q} \vdash x}
\end{mathpar}

\subsubsection{Contexts}

One of the principle advantages of computational calculi like the
$\pi$-calculus is a well-defined notion of context,
contextual-equivalence and a correlation between
contextual-equivalence and notions of bisimulation. The notion of
context allows the decomposition of a process into (sub-)process and
its syntactic environment, its context. Thus, a context may be
thought of as a process with a ``hole'' (written $\Box$) in it. The
application of a context $M$ to a process $P$, written $M[P]$, is
tantamount to filling the hole in $M$ with $P$. In this paper we do
not need the full weight of this theory, but do make use of the notion
of context in the proof the main theorem. 

\begin{mathpar}
  \inferrule* [lab=summation] {} {{M_{M},M_{N}} \bc \Box \;|\; x.M_{A} \;|\; M_{M}+M_{N}}
  \and
  \inferrule* [lab=agent] {} {{M_{A}} \bc (\vec{x})M_{P} \;| \; \clift{P_0,\ldots,M_{P},\ldots,P_N}}
  \and \\
  \inferrule* [lab=process] {} {{M_{P}} \bc M_{N} \;| \;P|M_{P} }
\end{mathpar} 

\begin{mathpar}
  \inferrule* [lab=sychronization] {} {M_{N} \bc \Box \;|\; x?M_{F} \;|\; x!M_{C}}
  \and
  \inferrule* [lab=abstraction] {} {{M_{F}} \bc (x)M_{P} }
  \and
  \inferrule* [lab=concretion] {} {{M_{C}} \bc \langle M_{P} \rangle }
  \and \\
  \inferrule* [lab=process] {} {{M_{P}} \bc M_{N} \;| \;P|M_{P} }
\end{mathpar}

\begin{definition}[contextual application] Given a context $M$, and
  process $P$, we define the \emph{contextual application}, $M[P] :=
  M\{P/\Box\}$. That is, the contextual application of M to P is the
  substitution of $P$ for $\Box$ in $M$.
\end{definition}

$\meaningof{-} : L \to \mathcal{P}(\pi)$

\begin{mathpar}
  \inferrule* [lab=collection] {} {\meaningof{true} = \pi, \and \meaningof{~E} = \pi \setminus \meaningof{E}, \and \meaningof{E_{1} \& E_{2}} = \meaningof{E_{1}} \cap \meaningof{E_{2}}}
\end{mathpar}

\begin{mathpar}
  \inferrule* [lab=structure] {} {\meaningof{0} = \{ P \in \pi | P \equiv 0 \}, \and \\ \meaningof{E_1 | E_2} = \{ P \in \pi | P \equiv P_{1} | P_{2}, P_{1} \in \meaningof{E_{1}}, P_{2} \in \meaningof{E_2}\} }
\end{mathpar}

\begin{mathpar}
 \inferrule* [lab=behavior] {} {\meaningof{\langle a?b \rangle E} = \{ P \in \pi | P \equiv Q | u?(y)P', \\ \and \\\\ \and \\ \;\;\; u \in \meaningof{a}, \forall z.P'\{z/y\} \in \meaningof{E\{z/b\}}\}, \and \\ \meaningof{a!E} = \{ P \in \pi | P \equiv Q | x!\langle P' \rangle, x \in \meaningof{a} P' \in \meaningof{E}\} }
\end{mathpar}

\begin{mathpar}
 \inferrule* [lab=nominal] {} {\meaningof{\quotep{E}} = \{ \quotep{P} \in \quotep{\pi} | P \in \meaningof{E} \}, \and \meaningof{\quotep{P}} = \{ \quotep{Q} \in \quotep{\pi} | P \equiv Q \} \and \\ \meaningof{@\quotep{E}} = \{ P \in \pi | P \equiv @x, x \in \meaningof{E} \}}
\end{mathpar}

\begin{eqnarray*}
  \\
  \meaningof{-} : TS \to ST
\end{eqnarray*}

\begin{eqnarray*}
  \\
  L : TS \to ST
\end{eqnarray*}

\begin{eqnarray*}
  \\
  P \models E \iff P \in \meaningof{E}
\end{eqnarray*}

\begin{eqnarray*}
  P \approx_{L} Q \iff \forall E \in L. P \models E \iff Q \models E
\end{eqnarray*}

\begin{eqnarray*}
  P \approx_{K} Q
\end{eqnarray*}

\begin{eqnarray*}
  P \approx Q
\end{eqnarray*}

$\approx_{K} = \approx = \approx_{L}$

\subsubsection{Contextual duality}

Note that contexts extend the quotation operation to a family of
operations from processes to names. Given a context, $M$, we can
define a \emph{nominal context}, $\quotep{M}$ by $\quotep{M}[P] :=
\quotep{M[P]}$. To foreshadow what is to come we observe that these
operations enjoy a duality with processes very much like the duality
between vectors and maps from vectors to scalars.

Further, because the calculus is essentially higher-order, we have a
correspondence between contexts and processes. More specifically,
given a name $x$ and a context $M$ we can construct $M^{*}_{x}$ such
that 

\begin{mathpar}
  M^{*}_{x} | \lift{x}{P} \red M[P]
\end{mathpar}

namely,

\begin{mathpar}
  M^{*}_{x} := x?(u).M[\dropn{u}]
\end{mathpar}

The dependence of $M^{*}_{x}$ on a name makes it an abstraction, 

\begin{mathpar}
  M^{*} := (x)x?(u).M[\dropn{u}]
\end{mathpar}

\subsection{Additional notation}

It will sometimes be convenient to denote the process a name
quotes. We already have the notation $x = \quotep{P}$, but it will be
convenient to introduce an alternate notation, $\procn{x}$, when we
want to emphasize the connection to the use of the name. Note that, by
virtue of name equivalence, $\quotep{\procn{x}} \nameeq x$; so, the
notation is consistent with previous definitions.

Further, because names have structure it is possible to effect
substitutions on the basis of that structure. This means we need to
upgrade our notation for substitutions, which we accomplish by
adapting comprehension notation. Thus,

\begin{mathpar}
  P\{ y / x : x \in S \}
\end{mathpar}

is interpreted to mean the process derived from P by replacing (in a
capture-avoiding manner) each occurrence of $x$ in $S$ by $y$. For example,

\begin{mathpar}
  P\{ \quotep{\procn{x}|\procn{x}} / x : x \in \freenames{P} \}
\end{mathpar}

will replace each (occurrence) of a free name $x$ in $P$ by
$\quotep{\procn{x}|\procn{x}}$.

Also, we will avail ourselves of the notation $x^{L}$ and $x^{R}$ to
denote injections of a name into disjoint copies of the name
space. There are numerous ways to accomplish this. One example can be
found in \cite{MeredithR05}. This notation overloads to vectors of
names: $\vec{x}^{\pi} := (x_{i}^{\pi} \; : \; 0 \leq i < |\vec{x}| )$ where $\pi \in \{L,R\}$.

We also use $P^{\Box} := P|\Box$.

In \cite{MeredithR05} an interpretation of the new operator is
given. It turns out that there are several possible interpretations
all enjoying the requisite algebraic properties of the operator (see
\cite{milner91polyadicpi}). We will therefore make liberal use of
$(\nu\; \vec{x})P$.

% subsection the_syntax_and_semantics_of_the_notation_system (end)   

\input{qm2pi.qmops} 

\input{qm2pi.sterngerlach} 

\input{qm2pi.metric} 

% section concurrent_process_calculi (end)

%\input{qm2pi.proofsketch}

% section proof sketch (end)

%\input{qm2pi.slviaknots} 

% section spatial logic via knots (end)

\input{qm2pi.conclusion}

% section conclusion (end)

%\input{qm2pi.dtcodes} 

% section wiring algorithm (end)

\input{qm2pi.ack} 

% section acknowledgments (end)

\newpage


\bibliographystyle{plain}   
\bibliography{../../biblios/main.bib}

\input{qm2pi.rhodetails}

\end{document}



\end{document}

 

% section acknowledgments (end)

\newpage


\bibliographystyle{plain}   
\bibliography{../../biblios/main.bib}

\documentclass[12pt]{llncs}
%\documentclass{jktr}

\usepackage[pdftex]{hyperref}                   
\usepackage {listings}
\usepackage {mathpartir}
\usepackage{bcprules}
%\usepackage{listings}
                       
\usepackage{graphicx} 
%\usepackage[margins=2.5cm,nohead,nofoot]{geometry}
%\usepackage{geometry}
\usepackage{amsfonts}
\usepackage{amstext}
\usepackage{latexsym}
\usepackage{amssymb}
\usepackage{color}


%\include{myPreamble}
\documentclass[12pt]{llncs}
%\documentclass{jktr}

\usepackage[pdftex]{hyperref}                   
\usepackage {listings}
\usepackage {mathpartir}
\usepackage{bcprules}
%\usepackage{listings}
                       
\usepackage{graphicx} 
%\usepackage[margins=2.5cm,nohead,nofoot]{geometry}
%\usepackage{geometry}
\usepackage{amsfonts}
\usepackage{amstext}
\usepackage{latexsym}
\usepackage{amssymb}
\usepackage{color}


%\include{myPreamble}
\include{qm2pi.local} 

%\ifpdf
%\usepackage[pdftex]{graphicx}
%\else
%\usepackage{graphicx}
%\fi

 % \ifpdf
%  \usepackage{pdfsync}
%  \if


%\title{Brief Article}
%\author{David F. Snyder}
%\author{L.G. Meredith}

%\address{Dept. of Math., Texas State University--San Marcos, San Marcos, TX 78666}
       
\pagestyle{empty}


\begin{document}

\lstset{language=[Objective]Caml,frame=shadowbox}

\input{qm2pi.front}

% section front matter (end)

\input{qm2pi.intro} 
 
% section introduction (end)

% \input{qm2pi.knotations} 

% section notation (end)

\input{qm2pi.process.calculi} 

% section concurrent_process_calculi_and_spatial_logics_ (end)
    
%\input{qm2pi.knots2pi} 

%\input{qm2pi.trefoil} 

%\input{qm2pi.mainthm} 

% subsection basic_interpretation (end)

%\input{qm2pi.rho.presentation} 
\subsection{The syntax and semantics of the notation system}\label{sub:the_syntax_and_semantics_of_the_notation_system} % (fold)

We now summarize a technical presentation of the calculus that
embodies our theory of dynamics. The typical presentation of such a
calculus follows the style of giving generators and relations on
them. The grammar, below, describing term constructors, freely
generates the set of processes, $\Proc$. This set is then quotiented
by a relation known as structural congruence and it is over this set
that the notion of dynamics is expressed. This presentation is
essentially that of \cite{MeredithR05} with the addition of
polyadicity and summation. For readability we have relegated some of
the technical subtleties to an appendix.

\subsubsection{Process grammar}\label{subsub:process_grammar}

\begin{mathpar}
  \inferrule* [lab=synchronization] {} {{M} \bc \pzero \;|\; x?F \;|\; x!C }
  \and
  \inferrule* [lab=abstraction] {} {{F} \bc (x)P}
  \and
  \inferrule* [lab=concretion] {} {{C} \bc \langle Q \rangle}
  \and
  \inferrule* [lab=process] {} {{P,Q} \bc M \;| \;P|Q \;|\; @{x}}
  \and
  \inferrule* [lab=name] {} {{x} \bc \quotep{P}}
\end{mathpar} 

Note that $\vec{x}$ (resp. $\vec{P}$) denotes a vector of names
(resp. processes) of length $|\vec{x}|$ (resp. $|\vec{P}|$). We adopt
the following useful abbreviations.

\begin{mathpar}
   x?(\vec{y}).P := x.(\vec{y})P \and  x\clift{\vec{P}} := x.\clift{\vec{P}}
   \and x!(y) := \lift{x}{\dropn{y}}
   \and \Pi_{i=0}^{n-1}P_i := P_0 | \ldots | P_{n-1}
\end{mathpar}

\subsubsection{Structural congruence}

\paragraph{Free and bound names and alpha-equivalence.} At the
core of structural equivalence is alpha-equivalence which identifies
process that are the same up to a change of variable. Formally, we
recognize the distinction between free and bound names. The free names
of a process, $\freenames{P}$, may be calculated recursively as
follows:

\begin{mathpar}
\freenames{\pzero} := \emptyset
  \and \\
  \freenames{x?(y).P} := \{ x \} \cup (\freenames{P} \setminus \{ y \})
  \and 
  \freenames{x!\langle P \rangle} := \{ x \} \cup \{ P \} 
  \and \\
  \freenames{P|Q} := \freenames{P} \cup \freenames{Q}
  \and \\
  \freenames{@{x}} := \{ x \}
\end{mathpar}

$\pi$
$\quotep{\pi}$

$\freenames{-} : \pi \to \mathcal{P}(\quotep{\pi})$

\begin{eqnarray*}
  \freenames{\pzero} & := & \emptyset \\
  \freenames{x?(y).P} & := & \{ x \} \cup (\freenames{P} \setminus \{ y \}) \\
  \freenames{x!\langle P \rangle} & := & \{ x \} \cup \{ P \} \\
  \freenames{P|Q} & := & \freenames{P} \cup \freenames{Q} \\
  \freenames{\dropn{x}} & := & \{ x \}
\end{eqnarray*}

The bound names of a process, $\boundnames{P}$, are those names occurring in $P$
that are not free. For example, in $x?(y).0$, the name $x$ is free, while $y$ is bound.

\begin{mathpar}
  \inferrule* [lab=monoidal-laws] {} { P|Q \equiv Q|P \and P|0 \equiv P \and P|(Q|R) \equiv (P|Q)|R }
\end{mathpar}

\begin{mathpar}
  \inferrule* [lab=alpha-equivalence] {} { (x)P \equiv (y)P\{y/x\} \and y \not\in \freenames{P} }
\end{mathpar}

\begin{definition}
Then two processes, $P,Q$, are alpha-equivalent if $P = Q\{\vec{y}/\vec{x}\}$ for
some $\vec{x} \in \boundnames{Q},\vec{y} \in \boundnames{P}$, where $Q\{\vec{y}/\vec{x}\}$
denotes the capture-avoiding substitution of $\vec{y}$ for $\vec{x}$ in $Q$.
\end{definition}

\begin{definition}
  The {\em structural congruence} \cite{SangiorgiWalker} , $\equiv$,
  between processes is the least congruence containing
  alpha-equivalence, satisfying the abelian monoid laws
  (associativity, commutativity and $\pzero$ as identity) for parallel
  composition $|$ and for summation $+$.
\end{definition}

\subsection{Name equivalence}

We take name equivalence, written $\nameeq$, to be the smallest
equivalence relation generated by the following rules.

\begin{mathpar}
\inferrule*[lab=Quote-drop]
{ }
{ \quotep{@{x}} \nameeq x }

\inferrule*[lab=Struct-equiv]
{ P \scong Q }
{ \quotep{P} \nameeq \quotep{Q} }
\end{mathpar}

The astute reader will have noticed that the mutual recursion of names
and processes imposes a mutual recursion on alpha-equivalence and
structural equivalence via name-equivalence. Fortunately, all of this
works out pleasantly and we may calculate in the natural way, free of
concern. The reader interested in the details is referred to the
appendix \ref{appendix:rho_details}.

\subsection{Substitution}

We use $\Proc$ for the set of processes, $\QProc$ for the set of
names, and $\id{\{}\vec{y} / \vec{x} \id{\}}$ to denote partial maps,
$s : \QProc \rightarrow \QProc$. A map, $s$ lifts, uniquely, to a map
on process terms, $\widehat{s} : \Proc \rightarrow \Proc$ by the
following equations.

\begin{mathpar}
  (0) \psubstp{Q}{P} := 0 \\
  (R \juxtap S) \psubstp{Q}{P}
  :=    
  (R)\psubstp{Q}{P} \juxtap (S) \psubstp{Q}{P} \\
  (x?(y).R) \psubstp{Q}{P}    
  :=    
  (x)\substp{Q}{P} (z)\concat( (R \psubstn{z}{y}) \psubstp{Q}{P} ) \\
  (\lift{x}{R}) \psubstp{Q}{P}  
  :=
  \lift{(x)\substp{Q}{P}}{ R \psubstp{Q}{P} } \\
%   (\dropn{x})  \psubstp{Q}{P}       
%   := 
%   \left\{ 
%     \begin{array}{ccc} 
%       \dropn{\quotep{Q}} & & x \nameeq \quotep{P} \\
%       \dropn{x} & & otherwise \\
%     \end{array}
%   \right. 
  (\dropn{x})  \psubstp{Q}{P}       
  := 
  \left\{ 
    \begin{array}{ccc} 
      Q & & x \nameeq \quotep{P} \\
      \dropn{x} & & otherwise \\
    \end{array}
  \right.
\end{mathpar}
 

where

\begin{eqnarray}
  (x)\id{\{} \lpquote Q \rpquote / \lpquote P \rpquote \id{\}}            = 
  \left\{ 
    \begin{array}{ccc}
      \lpquote Q \rpquote & & x \nameeq \lpquote P \rpquote \\
      x & & otherwise \\
    \end{array}
  \right. \nonumber
\end{eqnarray}

and $z$ is chosen distinct from $\quotep{P}$, $\quotep{Q}$, the free
names in $Q$, and all the names in $R$. Our $\alpha$-equivalence will
be built in the standard way from this substitution.

\begin{remark}\label{rem:no_self_referential_names}
  One consequence of these definitions is that $\forall P. \quotep{P}
  \not\in \freenames{P}$.
\end{remark}

\subsection{ Dynamic quote: an example }

Anticipating something of what's to come, consider applying the
substitution, $\widehat{\id{\{}u / z \id{\}}}$, to the following pair
of processes, $\lift{w}{y!(z)}$ and $w[ \lpquote y!(z) \rpquote ]$.

\begin{eqnarray}
	\lift{w}{y!(z)}\widehat{\id{\{}u / z \id{\}}}
		& = &
		\lift{w}{y!(u)} \nonumber\\
	w[ \lpquote y!(z) \rpquote ] \widehat{ \id{\{}u / z \id{\}} }
		& = &
		w[ \lpquote y!(z) \rpquote ] \nonumber
\end{eqnarray}

Because the body of the process between quotes is impervious to
substitution, we get radically different answers. In fact, by
examining the first process in an input context,
e.g. $x?(z).\lift{w}{y!(z)}$, we see that the process under the lift
operator may be shaped by prefixed inputs binding a name inside it. In
this sense, the lift operator will be seen as a way to dynamically
construct processes before reifying them as names.

Finally equipped with these standard features we can present the
dynamics of the calculus.

\subsubsection{Operational semantics} 

Finally, we introduce the computational dynamics. What marks these
algebras as distinct from other more traditionally studied algebraic
structures, e.g. vector spaces or polynomial rings, is the manner in
which dynamics is captured. In traditional structures, dynamics is typically
expressed through morphisms between such structures, as in linear maps
between vector spaces or morphisms between rings. In algebras
associated with the semantics of computation, the dynamics is
expressed as part of the algebraic structure itself, through a
reduction reduction relation typically denoted by $\red$. Below, we
give a recursive presentation of this relation for the calculus used
in the encoding.

$\red \subseteq \pi \times \pi$
$\red : \pi \to \mathcal{P}(\pi)$

\begin{mathpar}
  \inferrule* [lab=Comm] { \textsf{match}( x_{src}, x_{trgt} ) } { x_{trgt}?(y)P \; | \; x_{src}!\langle {Q} \rangle \red P\{\quotep{Q}/y}\} }
  \and \\
  \inferrule* [lab=Par] {{P} \red {P}'} {{{P} | {Q}} \red {{P}' | {Q}}}
  \and
  \inferrule* [lab=Equiv]{{{P} \scong {P}'} \andalso {{P}' \red {Q}'} \andalso {{Q}' \scong {Q}}}{{P} \red {Q}}
\end{mathpar}

\begin{eqnarray*}
  match_{\equiv} (\quotep{P},\quotep{Q}) & := & P \equiv Q \\
  match_{\dagger}(\quotep{P},\quotep{Q}) & := & \forall R. P|Q \red^{*} R => R \red^{*} 0 \\
  match_{K}(\quotep{P},\quotep{Q}) & := & K \mbox{ for some context } K
\end{eqnarray*}

$u?(x)P | u!\langle Q \rangle \red P\{\quotep{Q}/x\}$

%We write $\wred$ for $\red^*$, and $P\red$ if $\exists Q $ such that $ P \red Q$.
We write $P\red$ if $\exists Q $ such that $ P \red Q$ and $P\not\red$, otherwise.

\section{Replication}

As mentioned before, it is known that replication (and hence
recursion) can be implemented in a higher-order process algebra
\cite{SangiorgiWalker}. As our first example of calculation with the
machinery thus far presented we give the construction explicitly in
the {\rhoc}.

\begin{eqnarray}
	D_{x} & := & \prefix{x}{y}{(\binpar{\outputp{x}{y}}{@{y}})} \nonumber\\
	\bangp_{x}{P} & := & \binpar{{x}!\langle{\binpar{D_{x}}{P}}\rangle}{D_{x}} \nonumber
\end{eqnarray}

\begin{eqnarray}
	\bangp_{x}{P} & & \nonumber\\
	=
	& {x}!\langle{(\prefix{x}{y}{(\outputp{x}{y} | @{y})) | P}}\rangle 
	      | \prefix{x}{y}{(\outputp{x}{y} | @{y})} & \nonumber\\
	\red
	& (\outputp{x}{y} | @{y})\substn{\quotep{(\prefix{x}{y}{(@{y} | \outputp{x}{y})) | P}}}{y} & \nonumber\\
	=
	& \outputp{x}{\quotep{(\prefix{x}{y}{(\outputp{x}{y} | @{y})) | P}}}
	  | {(\prefix{x}{y}{(\outputp{x}{y} | @{y})) | P}} & \nonumber\\
	\red
	& \ldots & \nonumber\\
	\red^*
	& P | P | \ldots & \nonumber
\end{eqnarray}

Of course, this encoding, as an implementation, runs away, unfolding
$\bangp{P}$ eagerly. A lazier and more implementable replication
operator, restricted to input-guarded processes, may be obtained as follows.

\begin{eqnarray}
\bangp{\prefix{u}{v}{P}} 
	:= 
	\binpar{\lift{x}{\prefix{u}{v}{(\binpar{D(x)}{P})}}}{D(x)} \nonumber
\end{eqnarray}

\begin{remark}
  Note that the lazier definition still does not deal with summation
  or mixed summation (i.e. sums over input and output). The reader is
  invited to construct definitions of replication that deal with these
  features. 

  Further, the definitions are parameterized in a name, $x$. Can you,
  gentle reader, make a definition that eliminates this parameter and
  guarantees no accidental interaction between the replication
  machinery and the process being replicated -- i.e. no accidental
  sharing of names used by the process to get its work done and the
  name(s) used by the replication to effect copying. This latter
  revision of the definition of replication is crucial to obtaining
  the expected identity $!!P \sim !P$.
\end{remark}

\begin{remark}\label{rem:paradoxical_combinator}
  The reader familiar with the lambda calculus will have noticed the
  similarity between $D$ and the paradoxical combinator.

  [Ed. note: the existence of this seems to suggest we have to be more
  restrictive on the set of processes and names we admit if we are to
  support no-cloning.]
\end{remark}

\subsubsection{Bisimulation}

The computational dynamics gives rise to another kind of equivalence,
the equivalence of computational behavior. As previously mentioned
this is typically captured \emph{via} some form of bisimulation.

% The notion we use in this paper is weak barbed bisimulation
% \cite{milner91polyadicpi}.

The notion we use in this paper is derived from weak barbed
bisimulation \cite{milner91polyadicpi}. 

\begin{definition}
An \emph{observation relation}, $\downarrow_{\mathcal N}$, over a set
of names, $\mathcal N$, is the smallest relation satisfying the rules
below.

\infrule[Out-barb]{y \in {\mathcal N}, \; x \nameeq y}
		  {\outputp{x}{v} \downarrow_{\mathcal N} x}
\infrule[Par-barb]{\mbox{$P\downarrow_{\mathcal N} x$ or $Q\downarrow_{\mathcal N} x$}}
		  {\binpar{P}{Q} \downarrow_{\mathcal N} x}

We write $P \Downarrow_{\mathcal N} x$ if there is $Q$ such that 
$P \wred Q$ and $Q \downarrow_{\mathcal N} x$.
\end{definition}

\begin{definition}
%\label{def.bbisim}
An  ${\mathcal N}$-\emph{barbed bisimulation} over a set of names, ${\mathcal N}$, is a symmetric binary relation 
${\mathcal S}_{\mathcal N}$ between agents such that $P\rel{S}_{\mathcal N}Q$ implies:
\begin{enumerate}
\item If $P \red P'$ then $Q \wred Q'$ and $P'\rel{S}_{\mathcal N} Q'$.
\item If $P\downarrow_{\mathcal N} x$, then $Q\Downarrow_{\mathcal N} x$.
\end{enumerate}
$P$ is ${\mathcal N}$-barbed bisimilar to $Q$, written
$P \wbbisim_{\mathcal N} Q$, if $P \rel{S}_{\mathcal N} Q$ for some ${\mathcal N}$-barbed bisimulation ${\mathcal S}_{\mathcal N}$.
\end{definition}

$\mathcal{R} \subseteq \pi \times \pi$

$P \mathcal{R} Q => \forall P'. P \red P' \Rightarrow \exists Q'. Q \red Q', P' \mathcal{R} Q'$

$P \vdash x \Rightarrow Q \vdash x$

\begin{mathpar}
  \inferrule*[lab=Out-barb]{x \nameeq y}{{y}!\langle{Q}\rangle \vdash x}
  \and
  \inferrule*[lab=Par-barb]{\mbox{$P\vdash x$ or $Q\vdash x$}}{\binpar{P}{Q} \vdash x}
\end{mathpar}

\subsubsection{Contexts}

One of the principle advantages of computational calculi like the
$\pi$-calculus is a well-defined notion of context,
contextual-equivalence and a correlation between
contextual-equivalence and notions of bisimulation. The notion of
context allows the decomposition of a process into (sub-)process and
its syntactic environment, its context. Thus, a context may be
thought of as a process with a ``hole'' (written $\Box$) in it. The
application of a context $M$ to a process $P$, written $M[P]$, is
tantamount to filling the hole in $M$ with $P$. In this paper we do
not need the full weight of this theory, but do make use of the notion
of context in the proof the main theorem. 

\begin{mathpar}
  \inferrule* [lab=summation] {} {{M_{M},M_{N}} \bc \Box \;|\; x.M_{A} \;|\; M_{M}+M_{N}}
  \and
  \inferrule* [lab=agent] {} {{M_{A}} \bc (\vec{x})M_{P} \;| \; \clift{P_0,\ldots,M_{P},\ldots,P_N}}
  \and \\
  \inferrule* [lab=process] {} {{M_{P}} \bc M_{N} \;| \;P|M_{P} }
\end{mathpar} 

\begin{mathpar}
  \inferrule* [lab=sychronization] {} {M_{N} \bc \Box \;|\; x?M_{F} \;|\; x!M_{C}}
  \and
  \inferrule* [lab=abstraction] {} {{M_{F}} \bc (x)M_{P} }
  \and
  \inferrule* [lab=concretion] {} {{M_{C}} \bc \langle M_{P} \rangle }
  \and \\
  \inferrule* [lab=process] {} {{M_{P}} \bc M_{N} \;| \;P|M_{P} }
\end{mathpar}

\begin{definition}[contextual application] Given a context $M$, and
  process $P$, we define the \emph{contextual application}, $M[P] :=
  M\{P/\Box\}$. That is, the contextual application of M to P is the
  substitution of $P$ for $\Box$ in $M$.
\end{definition}

$\meaningof{-} : L \to \mathcal{P}(\pi)$

\begin{mathpar}
  \inferrule* [lab=collection] {} {\meaningof{true} = \pi, \and \meaningof{~E} = \pi \setminus \meaningof{E}, \and \meaningof{E_{1} \& E_{2}} = \meaningof{E_{1}} \cap \meaningof{E_{2}}}
\end{mathpar}

\begin{mathpar}
  \inferrule* [lab=structure] {} {\meaningof{0} = \{ P \in \pi | P \equiv 0 \}, \and \\ \meaningof{E_1 | E_2} = \{ P \in \pi | P \equiv P_{1} | P_{2}, P_{1} \in \meaningof{E_{1}}, P_{2} \in \meaningof{E_2}\} }
\end{mathpar}

\begin{mathpar}
 \inferrule* [lab=behavior] {} {\meaningof{\langle a?b \rangle E} = \{ P \in \pi | P \equiv Q | u?(y)P', \\ \and \\\\ \and \\ \;\;\; u \in \meaningof{a}, \forall z.P'\{z/y\} \in \meaningof{E\{z/b\}}\}, \and \\ \meaningof{a!E} = \{ P \in \pi | P \equiv Q | x!\langle P' \rangle, x \in \meaningof{a} P' \in \meaningof{E}\} }
\end{mathpar}

\begin{mathpar}
 \inferrule* [lab=nominal] {} {\meaningof{\quotep{E}} = \{ \quotep{P} \in \quotep{\pi} | P \in \meaningof{E} \}, \and \meaningof{\quotep{P}} = \{ \quotep{Q} \in \quotep{\pi} | P \equiv Q \} \and \\ \meaningof{@\quotep{E}} = \{ P \in \pi | P \equiv @x, x \in \meaningof{E} \}}
\end{mathpar}

\begin{eqnarray*}
  \\
  \meaningof{-} : TS \to ST
\end{eqnarray*}

\begin{eqnarray*}
  \\
  L : TS \to ST
\end{eqnarray*}

\begin{eqnarray*}
  \\
  P \models E \iff P \in \meaningof{E}
\end{eqnarray*}

\begin{eqnarray*}
  P \approx_{L} Q \iff \forall E \in L. P \models E \iff Q \models E
\end{eqnarray*}

\begin{eqnarray*}
  P \approx_{K} Q
\end{eqnarray*}

\begin{eqnarray*}
  P \approx Q
\end{eqnarray*}

$\approx_{K} = \approx = \approx_{L}$

\subsubsection{Contextual duality}

Note that contexts extend the quotation operation to a family of
operations from processes to names. Given a context, $M$, we can
define a \emph{nominal context}, $\quotep{M}$ by $\quotep{M}[P] :=
\quotep{M[P]}$. To foreshadow what is to come we observe that these
operations enjoy a duality with processes very much like the duality
between vectors and maps from vectors to scalars.

Further, because the calculus is essentially higher-order, we have a
correspondence between contexts and processes. More specifically,
given a name $x$ and a context $M$ we can construct $M^{*}_{x}$ such
that 

\begin{mathpar}
  M^{*}_{x} | \lift{x}{P} \red M[P]
\end{mathpar}

namely,

\begin{mathpar}
  M^{*}_{x} := x?(u).M[\dropn{u}]
\end{mathpar}

The dependence of $M^{*}_{x}$ on a name makes it an abstraction, 

\begin{mathpar}
  M^{*} := (x)x?(u).M[\dropn{u}]
\end{mathpar}

\subsection{Additional notation}

It will sometimes be convenient to denote the process a name
quotes. We already have the notation $x = \quotep{P}$, but it will be
convenient to introduce an alternate notation, $\procn{x}$, when we
want to emphasize the connection to the use of the name. Note that, by
virtue of name equivalence, $\quotep{\procn{x}} \nameeq x$; so, the
notation is consistent with previous definitions.

Further, because names have structure it is possible to effect
substitutions on the basis of that structure. This means we need to
upgrade our notation for substitutions, which we accomplish by
adapting comprehension notation. Thus,

\begin{mathpar}
  P\{ y / x : x \in S \}
\end{mathpar}

is interpreted to mean the process derived from P by replacing (in a
capture-avoiding manner) each occurrence of $x$ in $S$ by $y$. For example,

\begin{mathpar}
  P\{ \quotep{\procn{x}|\procn{x}} / x : x \in \freenames{P} \}
\end{mathpar}

will replace each (occurrence) of a free name $x$ in $P$ by
$\quotep{\procn{x}|\procn{x}}$.

Also, we will avail ourselves of the notation $x^{L}$ and $x^{R}$ to
denote injections of a name into disjoint copies of the name
space. There are numerous ways to accomplish this. One example can be
found in \cite{MeredithR05}. This notation overloads to vectors of
names: $\vec{x}^{\pi} := (x_{i}^{\pi} \; : \; 0 \leq i < |\vec{x}| )$ where $\pi \in \{L,R\}$.

We also use $P^{\Box} := P|\Box$.

In \cite{MeredithR05} an interpretation of the new operator is
given. It turns out that there are several possible interpretations
all enjoying the requisite algebraic properties of the operator (see
\cite{milner91polyadicpi}). We will therefore make liberal use of
$(\nu\; \vec{x})P$.

% subsection the_syntax_and_semantics_of_the_notation_system (end)   

\input{qm2pi.qmops} 

\input{qm2pi.sterngerlach} 

\input{qm2pi.metric} 

% section concurrent_process_calculi (end)

%\input{qm2pi.proofsketch}

% section proof sketch (end)

%\input{qm2pi.slviaknots} 

% section spatial logic via knots (end)

\input{qm2pi.conclusion}

% section conclusion (end)

%\input{qm2pi.dtcodes} 

% section wiring algorithm (end)

\input{qm2pi.ack} 

% section acknowledgments (end)

\newpage


\bibliographystyle{plain}   
\bibliography{../../biblios/main.bib}

\input{qm2pi.rhodetails}

\end{document}

 

%\ifpdf
%\usepackage[pdftex]{graphicx}
%\else
%\usepackage{graphicx}
%\fi

 % \ifpdf
%  \usepackage{pdfsync}
%  \if


%\title{Brief Article}
%\author{David F. Snyder}
%\author{L.G. Meredith}

%\address{Dept. of Math., Texas State University--San Marcos, San Marcos, TX 78666}
       
\pagestyle{empty}


\begin{document}

\lstset{language=[Objective]Caml,frame=shadowbox}

\documentclass[12pt]{llncs}
%\documentclass{jktr}

\usepackage[pdftex]{hyperref}                   
\usepackage {listings}
\usepackage {mathpartir}
\usepackage{bcprules}
%\usepackage{listings}
                       
\usepackage{graphicx} 
%\usepackage[margins=2.5cm,nohead,nofoot]{geometry}
%\usepackage{geometry}
\usepackage{amsfonts}
\usepackage{amstext}
\usepackage{latexsym}
\usepackage{amssymb}
\usepackage{color}


%\include{myPreamble}
\include{qm2pi.local} 

%\ifpdf
%\usepackage[pdftex]{graphicx}
%\else
%\usepackage{graphicx}
%\fi

 % \ifpdf
%  \usepackage{pdfsync}
%  \if


%\title{Brief Article}
%\author{David F. Snyder}
%\author{L.G. Meredith}

%\address{Dept. of Math., Texas State University--San Marcos, San Marcos, TX 78666}
       
\pagestyle{empty}


\begin{document}

\lstset{language=[Objective]Caml,frame=shadowbox}

\input{qm2pi.front}

% section front matter (end)

\input{qm2pi.intro} 
 
% section introduction (end)

% \input{qm2pi.knotations} 

% section notation (end)

\input{qm2pi.process.calculi} 

% section concurrent_process_calculi_and_spatial_logics_ (end)
    
%\input{qm2pi.knots2pi} 

%\input{qm2pi.trefoil} 

%\input{qm2pi.mainthm} 

% subsection basic_interpretation (end)

%\input{qm2pi.rho.presentation} 
\subsection{The syntax and semantics of the notation system}\label{sub:the_syntax_and_semantics_of_the_notation_system} % (fold)

We now summarize a technical presentation of the calculus that
embodies our theory of dynamics. The typical presentation of such a
calculus follows the style of giving generators and relations on
them. The grammar, below, describing term constructors, freely
generates the set of processes, $\Proc$. This set is then quotiented
by a relation known as structural congruence and it is over this set
that the notion of dynamics is expressed. This presentation is
essentially that of \cite{MeredithR05} with the addition of
polyadicity and summation. For readability we have relegated some of
the technical subtleties to an appendix.

\subsubsection{Process grammar}\label{subsub:process_grammar}

\begin{mathpar}
  \inferrule* [lab=synchronization] {} {{M} \bc \pzero \;|\; x?F \;|\; x!C }
  \and
  \inferrule* [lab=abstraction] {} {{F} \bc (x)P}
  \and
  \inferrule* [lab=concretion] {} {{C} \bc \langle Q \rangle}
  \and
  \inferrule* [lab=process] {} {{P,Q} \bc M \;| \;P|Q \;|\; @{x}}
  \and
  \inferrule* [lab=name] {} {{x} \bc \quotep{P}}
\end{mathpar} 

Note that $\vec{x}$ (resp. $\vec{P}$) denotes a vector of names
(resp. processes) of length $|\vec{x}|$ (resp. $|\vec{P}|$). We adopt
the following useful abbreviations.

\begin{mathpar}
   x?(\vec{y}).P := x.(\vec{y})P \and  x\clift{\vec{P}} := x.\clift{\vec{P}}
   \and x!(y) := \lift{x}{\dropn{y}}
   \and \Pi_{i=0}^{n-1}P_i := P_0 | \ldots | P_{n-1}
\end{mathpar}

\subsubsection{Structural congruence}

\paragraph{Free and bound names and alpha-equivalence.} At the
core of structural equivalence is alpha-equivalence which identifies
process that are the same up to a change of variable. Formally, we
recognize the distinction between free and bound names. The free names
of a process, $\freenames{P}$, may be calculated recursively as
follows:

\begin{mathpar}
\freenames{\pzero} := \emptyset
  \and \\
  \freenames{x?(y).P} := \{ x \} \cup (\freenames{P} \setminus \{ y \})
  \and 
  \freenames{x!\langle P \rangle} := \{ x \} \cup \{ P \} 
  \and \\
  \freenames{P|Q} := \freenames{P} \cup \freenames{Q}
  \and \\
  \freenames{@{x}} := \{ x \}
\end{mathpar}

$\pi$
$\quotep{\pi}$

$\freenames{-} : \pi \to \mathcal{P}(\quotep{\pi})$

\begin{eqnarray*}
  \freenames{\pzero} & := & \emptyset \\
  \freenames{x?(y).P} & := & \{ x \} \cup (\freenames{P} \setminus \{ y \}) \\
  \freenames{x!\langle P \rangle} & := & \{ x \} \cup \{ P \} \\
  \freenames{P|Q} & := & \freenames{P} \cup \freenames{Q} \\
  \freenames{\dropn{x}} & := & \{ x \}
\end{eqnarray*}

The bound names of a process, $\boundnames{P}$, are those names occurring in $P$
that are not free. For example, in $x?(y).0$, the name $x$ is free, while $y$ is bound.

\begin{mathpar}
  \inferrule* [lab=monoidal-laws] {} { P|Q \equiv Q|P \and P|0 \equiv P \and P|(Q|R) \equiv (P|Q)|R }
\end{mathpar}

\begin{mathpar}
  \inferrule* [lab=alpha-equivalence] {} { (x)P \equiv (y)P\{y/x\} \and y \not\in \freenames{P} }
\end{mathpar}

\begin{definition}
Then two processes, $P,Q$, are alpha-equivalent if $P = Q\{\vec{y}/\vec{x}\}$ for
some $\vec{x} \in \boundnames{Q},\vec{y} \in \boundnames{P}$, where $Q\{\vec{y}/\vec{x}\}$
denotes the capture-avoiding substitution of $\vec{y}$ for $\vec{x}$ in $Q$.
\end{definition}

\begin{definition}
  The {\em structural congruence} \cite{SangiorgiWalker} , $\equiv$,
  between processes is the least congruence containing
  alpha-equivalence, satisfying the abelian monoid laws
  (associativity, commutativity and $\pzero$ as identity) for parallel
  composition $|$ and for summation $+$.
\end{definition}

\subsection{Name equivalence}

We take name equivalence, written $\nameeq$, to be the smallest
equivalence relation generated by the following rules.

\begin{mathpar}
\inferrule*[lab=Quote-drop]
{ }
{ \quotep{@{x}} \nameeq x }

\inferrule*[lab=Struct-equiv]
{ P \scong Q }
{ \quotep{P} \nameeq \quotep{Q} }
\end{mathpar}

The astute reader will have noticed that the mutual recursion of names
and processes imposes a mutual recursion on alpha-equivalence and
structural equivalence via name-equivalence. Fortunately, all of this
works out pleasantly and we may calculate in the natural way, free of
concern. The reader interested in the details is referred to the
appendix \ref{appendix:rho_details}.

\subsection{Substitution}

We use $\Proc$ for the set of processes, $\QProc$ for the set of
names, and $\id{\{}\vec{y} / \vec{x} \id{\}}$ to denote partial maps,
$s : \QProc \rightarrow \QProc$. A map, $s$ lifts, uniquely, to a map
on process terms, $\widehat{s} : \Proc \rightarrow \Proc$ by the
following equations.

\begin{mathpar}
  (0) \psubstp{Q}{P} := 0 \\
  (R \juxtap S) \psubstp{Q}{P}
  :=    
  (R)\psubstp{Q}{P} \juxtap (S) \psubstp{Q}{P} \\
  (x?(y).R) \psubstp{Q}{P}    
  :=    
  (x)\substp{Q}{P} (z)\concat( (R \psubstn{z}{y}) \psubstp{Q}{P} ) \\
  (\lift{x}{R}) \psubstp{Q}{P}  
  :=
  \lift{(x)\substp{Q}{P}}{ R \psubstp{Q}{P} } \\
%   (\dropn{x})  \psubstp{Q}{P}       
%   := 
%   \left\{ 
%     \begin{array}{ccc} 
%       \dropn{\quotep{Q}} & & x \nameeq \quotep{P} \\
%       \dropn{x} & & otherwise \\
%     \end{array}
%   \right. 
  (\dropn{x})  \psubstp{Q}{P}       
  := 
  \left\{ 
    \begin{array}{ccc} 
      Q & & x \nameeq \quotep{P} \\
      \dropn{x} & & otherwise \\
    \end{array}
  \right.
\end{mathpar}
 

where

\begin{eqnarray}
  (x)\id{\{} \lpquote Q \rpquote / \lpquote P \rpquote \id{\}}            = 
  \left\{ 
    \begin{array}{ccc}
      \lpquote Q \rpquote & & x \nameeq \lpquote P \rpquote \\
      x & & otherwise \\
    \end{array}
  \right. \nonumber
\end{eqnarray}

and $z$ is chosen distinct from $\quotep{P}$, $\quotep{Q}$, the free
names in $Q$, and all the names in $R$. Our $\alpha$-equivalence will
be built in the standard way from this substitution.

\begin{remark}\label{rem:no_self_referential_names}
  One consequence of these definitions is that $\forall P. \quotep{P}
  \not\in \freenames{P}$.
\end{remark}

\subsection{ Dynamic quote: an example }

Anticipating something of what's to come, consider applying the
substitution, $\widehat{\id{\{}u / z \id{\}}}$, to the following pair
of processes, $\lift{w}{y!(z)}$ and $w[ \lpquote y!(z) \rpquote ]$.

\begin{eqnarray}
	\lift{w}{y!(z)}\widehat{\id{\{}u / z \id{\}}}
		& = &
		\lift{w}{y!(u)} \nonumber\\
	w[ \lpquote y!(z) \rpquote ] \widehat{ \id{\{}u / z \id{\}} }
		& = &
		w[ \lpquote y!(z) \rpquote ] \nonumber
\end{eqnarray}

Because the body of the process between quotes is impervious to
substitution, we get radically different answers. In fact, by
examining the first process in an input context,
e.g. $x?(z).\lift{w}{y!(z)}$, we see that the process under the lift
operator may be shaped by prefixed inputs binding a name inside it. In
this sense, the lift operator will be seen as a way to dynamically
construct processes before reifying them as names.

Finally equipped with these standard features we can present the
dynamics of the calculus.

\subsubsection{Operational semantics} 

Finally, we introduce the computational dynamics. What marks these
algebras as distinct from other more traditionally studied algebraic
structures, e.g. vector spaces or polynomial rings, is the manner in
which dynamics is captured. In traditional structures, dynamics is typically
expressed through morphisms between such structures, as in linear maps
between vector spaces or morphisms between rings. In algebras
associated with the semantics of computation, the dynamics is
expressed as part of the algebraic structure itself, through a
reduction reduction relation typically denoted by $\red$. Below, we
give a recursive presentation of this relation for the calculus used
in the encoding.

$\red \subseteq \pi \times \pi$
$\red : \pi \to \mathcal{P}(\pi)$

\begin{mathpar}
  \inferrule* [lab=Comm] { \textsf{match}( x_{src}, x_{trgt} ) } { x_{trgt}?(y)P \; | \; x_{src}!\langle {Q} \rangle \red P\{\quotep{Q}/y}\} }
  \and \\
  \inferrule* [lab=Par] {{P} \red {P}'} {{{P} | {Q}} \red {{P}' | {Q}}}
  \and
  \inferrule* [lab=Equiv]{{{P} \scong {P}'} \andalso {{P}' \red {Q}'} \andalso {{Q}' \scong {Q}}}{{P} \red {Q}}
\end{mathpar}

\begin{eqnarray*}
  match_{\equiv} (\quotep{P},\quotep{Q}) & := & P \equiv Q \\
  match_{\dagger}(\quotep{P},\quotep{Q}) & := & \forall R. P|Q \red^{*} R => R \red^{*} 0 \\
  match_{K}(\quotep{P},\quotep{Q}) & := & K \mbox{ for some context } K
\end{eqnarray*}

$u?(x)P | u!\langle Q \rangle \red P\{\quotep{Q}/x\}$

%We write $\wred$ for $\red^*$, and $P\red$ if $\exists Q $ such that $ P \red Q$.
We write $P\red$ if $\exists Q $ such that $ P \red Q$ and $P\not\red$, otherwise.

\section{Replication}

As mentioned before, it is known that replication (and hence
recursion) can be implemented in a higher-order process algebra
\cite{SangiorgiWalker}. As our first example of calculation with the
machinery thus far presented we give the construction explicitly in
the {\rhoc}.

\begin{eqnarray}
	D_{x} & := & \prefix{x}{y}{(\binpar{\outputp{x}{y}}{@{y}})} \nonumber\\
	\bangp_{x}{P} & := & \binpar{{x}!\langle{\binpar{D_{x}}{P}}\rangle}{D_{x}} \nonumber
\end{eqnarray}

\begin{eqnarray}
	\bangp_{x}{P} & & \nonumber\\
	=
	& {x}!\langle{(\prefix{x}{y}{(\outputp{x}{y} | @{y})) | P}}\rangle 
	      | \prefix{x}{y}{(\outputp{x}{y} | @{y})} & \nonumber\\
	\red
	& (\outputp{x}{y} | @{y})\substn{\quotep{(\prefix{x}{y}{(@{y} | \outputp{x}{y})) | P}}}{y} & \nonumber\\
	=
	& \outputp{x}{\quotep{(\prefix{x}{y}{(\outputp{x}{y} | @{y})) | P}}}
	  | {(\prefix{x}{y}{(\outputp{x}{y} | @{y})) | P}} & \nonumber\\
	\red
	& \ldots & \nonumber\\
	\red^*
	& P | P | \ldots & \nonumber
\end{eqnarray}

Of course, this encoding, as an implementation, runs away, unfolding
$\bangp{P}$ eagerly. A lazier and more implementable replication
operator, restricted to input-guarded processes, may be obtained as follows.

\begin{eqnarray}
\bangp{\prefix{u}{v}{P}} 
	:= 
	\binpar{\lift{x}{\prefix{u}{v}{(\binpar{D(x)}{P})}}}{D(x)} \nonumber
\end{eqnarray}

\begin{remark}
  Note that the lazier definition still does not deal with summation
  or mixed summation (i.e. sums over input and output). The reader is
  invited to construct definitions of replication that deal with these
  features. 

  Further, the definitions are parameterized in a name, $x$. Can you,
  gentle reader, make a definition that eliminates this parameter and
  guarantees no accidental interaction between the replication
  machinery and the process being replicated -- i.e. no accidental
  sharing of names used by the process to get its work done and the
  name(s) used by the replication to effect copying. This latter
  revision of the definition of replication is crucial to obtaining
  the expected identity $!!P \sim !P$.
\end{remark}

\begin{remark}\label{rem:paradoxical_combinator}
  The reader familiar with the lambda calculus will have noticed the
  similarity between $D$ and the paradoxical combinator.

  [Ed. note: the existence of this seems to suggest we have to be more
  restrictive on the set of processes and names we admit if we are to
  support no-cloning.]
\end{remark}

\subsubsection{Bisimulation}

The computational dynamics gives rise to another kind of equivalence,
the equivalence of computational behavior. As previously mentioned
this is typically captured \emph{via} some form of bisimulation.

% The notion we use in this paper is weak barbed bisimulation
% \cite{milner91polyadicpi}.

The notion we use in this paper is derived from weak barbed
bisimulation \cite{milner91polyadicpi}. 

\begin{definition}
An \emph{observation relation}, $\downarrow_{\mathcal N}$, over a set
of names, $\mathcal N$, is the smallest relation satisfying the rules
below.

\infrule[Out-barb]{y \in {\mathcal N}, \; x \nameeq y}
		  {\outputp{x}{v} \downarrow_{\mathcal N} x}
\infrule[Par-barb]{\mbox{$P\downarrow_{\mathcal N} x$ or $Q\downarrow_{\mathcal N} x$}}
		  {\binpar{P}{Q} \downarrow_{\mathcal N} x}

We write $P \Downarrow_{\mathcal N} x$ if there is $Q$ such that 
$P \wred Q$ and $Q \downarrow_{\mathcal N} x$.
\end{definition}

\begin{definition}
%\label{def.bbisim}
An  ${\mathcal N}$-\emph{barbed bisimulation} over a set of names, ${\mathcal N}$, is a symmetric binary relation 
${\mathcal S}_{\mathcal N}$ between agents such that $P\rel{S}_{\mathcal N}Q$ implies:
\begin{enumerate}
\item If $P \red P'$ then $Q \wred Q'$ and $P'\rel{S}_{\mathcal N} Q'$.
\item If $P\downarrow_{\mathcal N} x$, then $Q\Downarrow_{\mathcal N} x$.
\end{enumerate}
$P$ is ${\mathcal N}$-barbed bisimilar to $Q$, written
$P \wbbisim_{\mathcal N} Q$, if $P \rel{S}_{\mathcal N} Q$ for some ${\mathcal N}$-barbed bisimulation ${\mathcal S}_{\mathcal N}$.
\end{definition}

$\mathcal{R} \subseteq \pi \times \pi$

$P \mathcal{R} Q => \forall P'. P \red P' \Rightarrow \exists Q'. Q \red Q', P' \mathcal{R} Q'$

$P \vdash x \Rightarrow Q \vdash x$

\begin{mathpar}
  \inferrule*[lab=Out-barb]{x \nameeq y}{{y}!\langle{Q}\rangle \vdash x}
  \and
  \inferrule*[lab=Par-barb]{\mbox{$P\vdash x$ or $Q\vdash x$}}{\binpar{P}{Q} \vdash x}
\end{mathpar}

\subsubsection{Contexts}

One of the principle advantages of computational calculi like the
$\pi$-calculus is a well-defined notion of context,
contextual-equivalence and a correlation between
contextual-equivalence and notions of bisimulation. The notion of
context allows the decomposition of a process into (sub-)process and
its syntactic environment, its context. Thus, a context may be
thought of as a process with a ``hole'' (written $\Box$) in it. The
application of a context $M$ to a process $P$, written $M[P]$, is
tantamount to filling the hole in $M$ with $P$. In this paper we do
not need the full weight of this theory, but do make use of the notion
of context in the proof the main theorem. 

\begin{mathpar}
  \inferrule* [lab=summation] {} {{M_{M},M_{N}} \bc \Box \;|\; x.M_{A} \;|\; M_{M}+M_{N}}
  \and
  \inferrule* [lab=agent] {} {{M_{A}} \bc (\vec{x})M_{P} \;| \; \clift{P_0,\ldots,M_{P},\ldots,P_N}}
  \and \\
  \inferrule* [lab=process] {} {{M_{P}} \bc M_{N} \;| \;P|M_{P} }
\end{mathpar} 

\begin{mathpar}
  \inferrule* [lab=sychronization] {} {M_{N} \bc \Box \;|\; x?M_{F} \;|\; x!M_{C}}
  \and
  \inferrule* [lab=abstraction] {} {{M_{F}} \bc (x)M_{P} }
  \and
  \inferrule* [lab=concretion] {} {{M_{C}} \bc \langle M_{P} \rangle }
  \and \\
  \inferrule* [lab=process] {} {{M_{P}} \bc M_{N} \;| \;P|M_{P} }
\end{mathpar}

\begin{definition}[contextual application] Given a context $M$, and
  process $P$, we define the \emph{contextual application}, $M[P] :=
  M\{P/\Box\}$. That is, the contextual application of M to P is the
  substitution of $P$ for $\Box$ in $M$.
\end{definition}

$\meaningof{-} : L \to \mathcal{P}(\pi)$

\begin{mathpar}
  \inferrule* [lab=collection] {} {\meaningof{true} = \pi, \and \meaningof{~E} = \pi \setminus \meaningof{E}, \and \meaningof{E_{1} \& E_{2}} = \meaningof{E_{1}} \cap \meaningof{E_{2}}}
\end{mathpar}

\begin{mathpar}
  \inferrule* [lab=structure] {} {\meaningof{0} = \{ P \in \pi | P \equiv 0 \}, \and \\ \meaningof{E_1 | E_2} = \{ P \in \pi | P \equiv P_{1} | P_{2}, P_{1} \in \meaningof{E_{1}}, P_{2} \in \meaningof{E_2}\} }
\end{mathpar}

\begin{mathpar}
 \inferrule* [lab=behavior] {} {\meaningof{\langle a?b \rangle E} = \{ P \in \pi | P \equiv Q | u?(y)P', \\ \and \\\\ \and \\ \;\;\; u \in \meaningof{a}, \forall z.P'\{z/y\} \in \meaningof{E\{z/b\}}\}, \and \\ \meaningof{a!E} = \{ P \in \pi | P \equiv Q | x!\langle P' \rangle, x \in \meaningof{a} P' \in \meaningof{E}\} }
\end{mathpar}

\begin{mathpar}
 \inferrule* [lab=nominal] {} {\meaningof{\quotep{E}} = \{ \quotep{P} \in \quotep{\pi} | P \in \meaningof{E} \}, \and \meaningof{\quotep{P}} = \{ \quotep{Q} \in \quotep{\pi} | P \equiv Q \} \and \\ \meaningof{@\quotep{E}} = \{ P \in \pi | P \equiv @x, x \in \meaningof{E} \}}
\end{mathpar}

\begin{eqnarray*}
  \\
  \meaningof{-} : TS \to ST
\end{eqnarray*}

\begin{eqnarray*}
  \\
  L : TS \to ST
\end{eqnarray*}

\begin{eqnarray*}
  \\
  P \models E \iff P \in \meaningof{E}
\end{eqnarray*}

\begin{eqnarray*}
  P \approx_{L} Q \iff \forall E \in L. P \models E \iff Q \models E
\end{eqnarray*}

\begin{eqnarray*}
  P \approx_{K} Q
\end{eqnarray*}

\begin{eqnarray*}
  P \approx Q
\end{eqnarray*}

$\approx_{K} = \approx = \approx_{L}$

\subsubsection{Contextual duality}

Note that contexts extend the quotation operation to a family of
operations from processes to names. Given a context, $M$, we can
define a \emph{nominal context}, $\quotep{M}$ by $\quotep{M}[P] :=
\quotep{M[P]}$. To foreshadow what is to come we observe that these
operations enjoy a duality with processes very much like the duality
between vectors and maps from vectors to scalars.

Further, because the calculus is essentially higher-order, we have a
correspondence between contexts and processes. More specifically,
given a name $x$ and a context $M$ we can construct $M^{*}_{x}$ such
that 

\begin{mathpar}
  M^{*}_{x} | \lift{x}{P} \red M[P]
\end{mathpar}

namely,

\begin{mathpar}
  M^{*}_{x} := x?(u).M[\dropn{u}]
\end{mathpar}

The dependence of $M^{*}_{x}$ on a name makes it an abstraction, 

\begin{mathpar}
  M^{*} := (x)x?(u).M[\dropn{u}]
\end{mathpar}

\subsection{Additional notation}

It will sometimes be convenient to denote the process a name
quotes. We already have the notation $x = \quotep{P}$, but it will be
convenient to introduce an alternate notation, $\procn{x}$, when we
want to emphasize the connection to the use of the name. Note that, by
virtue of name equivalence, $\quotep{\procn{x}} \nameeq x$; so, the
notation is consistent with previous definitions.

Further, because names have structure it is possible to effect
substitutions on the basis of that structure. This means we need to
upgrade our notation for substitutions, which we accomplish by
adapting comprehension notation. Thus,

\begin{mathpar}
  P\{ y / x : x \in S \}
\end{mathpar}

is interpreted to mean the process derived from P by replacing (in a
capture-avoiding manner) each occurrence of $x$ in $S$ by $y$. For example,

\begin{mathpar}
  P\{ \quotep{\procn{x}|\procn{x}} / x : x \in \freenames{P} \}
\end{mathpar}

will replace each (occurrence) of a free name $x$ in $P$ by
$\quotep{\procn{x}|\procn{x}}$.

Also, we will avail ourselves of the notation $x^{L}$ and $x^{R}$ to
denote injections of a name into disjoint copies of the name
space. There are numerous ways to accomplish this. One example can be
found in \cite{MeredithR05}. This notation overloads to vectors of
names: $\vec{x}^{\pi} := (x_{i}^{\pi} \; : \; 0 \leq i < |\vec{x}| )$ where $\pi \in \{L,R\}$.

We also use $P^{\Box} := P|\Box$.

In \cite{MeredithR05} an interpretation of the new operator is
given. It turns out that there are several possible interpretations
all enjoying the requisite algebraic properties of the operator (see
\cite{milner91polyadicpi}). We will therefore make liberal use of
$(\nu\; \vec{x})P$.

% subsection the_syntax_and_semantics_of_the_notation_system (end)   

\input{qm2pi.qmops} 

\input{qm2pi.sterngerlach} 

\input{qm2pi.metric} 

% section concurrent_process_calculi (end)

%\input{qm2pi.proofsketch}

% section proof sketch (end)

%\input{qm2pi.slviaknots} 

% section spatial logic via knots (end)

\input{qm2pi.conclusion}

% section conclusion (end)

%\input{qm2pi.dtcodes} 

% section wiring algorithm (end)

\input{qm2pi.ack} 

% section acknowledgments (end)

\newpage


\bibliographystyle{plain}   
\bibliography{../../biblios/main.bib}

\input{qm2pi.rhodetails}

\end{document}



% section front matter (end)

\section{Introduction}\label{sec:introduction} % (fold)
In this draft of the material i am going to have to dispense with the
usual writing conventions adopted in papers on these topics. i'm going
to have adopt whatever tone i need at the time i'm writing up the
calculations. Sometimes this may be very conversational; others it may
be the barest mathematical grunts; others still it may be that i have
lifted text from one of my other papers because the exposition of some
point was better said there. i hope that my readers are not unduly put
out by this decision. i'm not doing this to flout convention or be
rebellious. i find these calculations very technically challenging. To
keep everything going technically, something has to give; i have to
let go of some cognitive burden. So, the academic writing style --
with all of its trade-offs in terms of facilitating technical
communication -- is what i'm letting go of. Perhaps subsequent drafts
can be tightened and polished, but for now, i'm going to speak as if
we were sitting together in a coffee shop with a laptop, wifi and a
pad of paper and a pencil.

So, here's what i have to say. We -- you and i, comfortably ensconced
in our coffee shop and well-equipped with our tools -- can realize and
carry out the calculations of quantum mechanics over a very different
formal theory of dynamics, a formal theory of dynamics that
corresponds to a theory of concurrent computation with
\emph{reflection}. It has the advantage that the underlying theory is
already `quantized', but supports analogues all of the continuuous
operations. Strikingly, this underlying theory has recently been
connected with a notion of metric that we can show, by calculating
together, coincides with the metric induced by the inner product.

There are a lot of reasons why you might be interested in seeing
calculations of this form. Here's why i'm interested. For the past
several centuries there has been no competitor to the ``Newtonian''
account of dynamics. As a result the predominant share of accounts of
dynamical systems and situations have had to be formulated in terms of
the Newtonian machinery. i view this as an intellectually dangerous
position to occupy. Everything, despite it's intrinsic shape, turns
into a nail to be hit with this hammer. Recently, however, the theory
of computation has matured to the point where we have candidates for
theories of dynamics that offer very different perspective on
reasoning about dynamical systems and situations. Testing these
candidates against very successful accounts of dynamical situations,
like quantum mechanics, is going to give us some sense of how mature
they are and some measure of the quality of these accounts of
dynamics.

\subsection{Summary of contributions and outline of paper}

So, we're going to develop an interpretation of the operations of
quantum mechanics normally interpreted by Hilbert spaces and
operators. We're going to do this over a theory of computation. Note
that this is very different than the usual quantum computation program
which develops notions of computation over quantum mechanics. Rather,
we are developing a story that aligns with Wheeler's slogan: It from
Bit. To do this we will first provide an account of the theory of
computation at play here. Then we will dive into a calculation-driven
interpretation of the operations of quantum mechanics.

The reason we take this approach is that -- until very recently --
there hasn't been an axiomatic account of quantum mechanics. As a
result there has been no sharp delineation of the mathematical theory
supporting interpretation of the physical theory and the physical
theory, itself. So, ambient features of the maths are free to be
exploited (or supressed) without a real accounting of their physical
relevance. There is no sharp statement ``here's the physical theory''
qua \emph{theory} and ``here's the mathematical interpretation''
enabling a judgment of how faithful the interpretation is -- apart
from experimental observation. When there is an axiomatic account we
can judge how well a given mathematical formalism supports an
interpretation of the axioms, independent of
experimentation. Likewise, we can judge how well we have captured our
physical evidence and experience with our axiomatics, independent of
any specific mathematical implementation, with accidental detail that
may or may not have physical significance. 

In lieu of a fully fleshed out and vetted axiomatic account of quantum
mechanics, interpreting the operational notions in service of modeling
physical systems will have to suffice. In other words, we are not in
the business of providing a model of Hilbert spaces and operators. We
are in the business of providing a model of quantum mechanics because
we are motivated by testing our notions of dynamics against physical
theory; and, the predictive calculations of the physical theory must
serve as the best formulation -- shy of a fully fleshed out axiomatic
account -- of the physical theory itself (as they have for scientific
theories since time immemorial). Put another way, despite a
whole-hearted commitment to an It-from-Bit ontology, we are firmly
aligned with the shut-up-and-calculate camp as the best way to obtain
results either from the physical perspective or as a quality assurance
measure of our fledgling theory of dynamics.

In detail, we present a reflective process calculus. Then we develop
intuitive correspondences between the notions available in this
calculus and the usual physical notions supporting quantum mechanical
calculations. Thus, 

\begin{table}[htp]
  \center{
    \fbox{
      \begin{tabular}{c|c}
        quantum mechanics & process calculus \\
        \hline
        scalar & name \\
        state vector & process \\
        dual & contextual duals \\
        matrix & formal sums of process-context-dual pairs \\
        orthogonality & process annihilation \\
        inner product & execution-formula + quoting
      \end{tabular}
    }
  }
  \caption{QM - process calculi correspondences}
\end{table}

Then we tighten up these intuitions to operational definitions. We
employ the Dirac notation as the best proxy we can find for an
abstract syntax of the quantum mechanical notions. The definitions we
develop put us in contact with equational constraints coming from the
theory that we demonstrate the definitions and calculations satisfy.

This puts us in a position to shut up and calculate for the
Stern-Gerlach experimental set up, showing how these predictive
calculations become calculations on processes in our theory of a
reflective process calculus.

Penultimately, we demonstrate that the notion of metric coming from
the inner product coincides with the notion of metric available from
the theory of bisimulation. This demonstration gives us the right to
think of space as arising from behavior. Finally, we consider where we
might go from the new vantage point we have obtained.

% section introduction (end) 
 
% section introduction (end)

% \documentclass[12pt]{llncs}
%\documentclass{jktr}

\usepackage[pdftex]{hyperref}                   
\usepackage {listings}
\usepackage {mathpartir}
\usepackage{bcprules}
%\usepackage{listings}
                       
\usepackage{graphicx} 
%\usepackage[margins=2.5cm,nohead,nofoot]{geometry}
%\usepackage{geometry}
\usepackage{amsfonts}
\usepackage{amstext}
\usepackage{latexsym}
\usepackage{amssymb}
\usepackage{color}


%\include{myPreamble}
\include{qm2pi.local} 

%\ifpdf
%\usepackage[pdftex]{graphicx}
%\else
%\usepackage{graphicx}
%\fi

 % \ifpdf
%  \usepackage{pdfsync}
%  \if


%\title{Brief Article}
%\author{David F. Snyder}
%\author{L.G. Meredith}

%\address{Dept. of Math., Texas State University--San Marcos, San Marcos, TX 78666}
       
\pagestyle{empty}


\begin{document}

\lstset{language=[Objective]Caml,frame=shadowbox}

\input{qm2pi.front}

% section front matter (end)

\input{qm2pi.intro} 
 
% section introduction (end)

% \input{qm2pi.knotations} 

% section notation (end)

\input{qm2pi.process.calculi} 

% section concurrent_process_calculi_and_spatial_logics_ (end)
    
%\input{qm2pi.knots2pi} 

%\input{qm2pi.trefoil} 

%\input{qm2pi.mainthm} 

% subsection basic_interpretation (end)

%\input{qm2pi.rho.presentation} 
\subsection{The syntax and semantics of the notation system}\label{sub:the_syntax_and_semantics_of_the_notation_system} % (fold)

We now summarize a technical presentation of the calculus that
embodies our theory of dynamics. The typical presentation of such a
calculus follows the style of giving generators and relations on
them. The grammar, below, describing term constructors, freely
generates the set of processes, $\Proc$. This set is then quotiented
by a relation known as structural congruence and it is over this set
that the notion of dynamics is expressed. This presentation is
essentially that of \cite{MeredithR05} with the addition of
polyadicity and summation. For readability we have relegated some of
the technical subtleties to an appendix.

\subsubsection{Process grammar}\label{subsub:process_grammar}

\begin{mathpar}
  \inferrule* [lab=synchronization] {} {{M} \bc \pzero \;|\; x?F \;|\; x!C }
  \and
  \inferrule* [lab=abstraction] {} {{F} \bc (x)P}
  \and
  \inferrule* [lab=concretion] {} {{C} \bc \langle Q \rangle}
  \and
  \inferrule* [lab=process] {} {{P,Q} \bc M \;| \;P|Q \;|\; @{x}}
  \and
  \inferrule* [lab=name] {} {{x} \bc \quotep{P}}
\end{mathpar} 

Note that $\vec{x}$ (resp. $\vec{P}$) denotes a vector of names
(resp. processes) of length $|\vec{x}|$ (resp. $|\vec{P}|$). We adopt
the following useful abbreviations.

\begin{mathpar}
   x?(\vec{y}).P := x.(\vec{y})P \and  x\clift{\vec{P}} := x.\clift{\vec{P}}
   \and x!(y) := \lift{x}{\dropn{y}}
   \and \Pi_{i=0}^{n-1}P_i := P_0 | \ldots | P_{n-1}
\end{mathpar}

\subsubsection{Structural congruence}

\paragraph{Free and bound names and alpha-equivalence.} At the
core of structural equivalence is alpha-equivalence which identifies
process that are the same up to a change of variable. Formally, we
recognize the distinction between free and bound names. The free names
of a process, $\freenames{P}$, may be calculated recursively as
follows:

\begin{mathpar}
\freenames{\pzero} := \emptyset
  \and \\
  \freenames{x?(y).P} := \{ x \} \cup (\freenames{P} \setminus \{ y \})
  \and 
  \freenames{x!\langle P \rangle} := \{ x \} \cup \{ P \} 
  \and \\
  \freenames{P|Q} := \freenames{P} \cup \freenames{Q}
  \and \\
  \freenames{@{x}} := \{ x \}
\end{mathpar}

$\pi$
$\quotep{\pi}$

$\freenames{-} : \pi \to \mathcal{P}(\quotep{\pi})$

\begin{eqnarray*}
  \freenames{\pzero} & := & \emptyset \\
  \freenames{x?(y).P} & := & \{ x \} \cup (\freenames{P} \setminus \{ y \}) \\
  \freenames{x!\langle P \rangle} & := & \{ x \} \cup \{ P \} \\
  \freenames{P|Q} & := & \freenames{P} \cup \freenames{Q} \\
  \freenames{\dropn{x}} & := & \{ x \}
\end{eqnarray*}

The bound names of a process, $\boundnames{P}$, are those names occurring in $P$
that are not free. For example, in $x?(y).0$, the name $x$ is free, while $y$ is bound.

\begin{mathpar}
  \inferrule* [lab=monoidal-laws] {} { P|Q \equiv Q|P \and P|0 \equiv P \and P|(Q|R) \equiv (P|Q)|R }
\end{mathpar}

\begin{mathpar}
  \inferrule* [lab=alpha-equivalence] {} { (x)P \equiv (y)P\{y/x\} \and y \not\in \freenames{P} }
\end{mathpar}

\begin{definition}
Then two processes, $P,Q$, are alpha-equivalent if $P = Q\{\vec{y}/\vec{x}\}$ for
some $\vec{x} \in \boundnames{Q},\vec{y} \in \boundnames{P}$, where $Q\{\vec{y}/\vec{x}\}$
denotes the capture-avoiding substitution of $\vec{y}$ for $\vec{x}$ in $Q$.
\end{definition}

\begin{definition}
  The {\em structural congruence} \cite{SangiorgiWalker} , $\equiv$,
  between processes is the least congruence containing
  alpha-equivalence, satisfying the abelian monoid laws
  (associativity, commutativity and $\pzero$ as identity) for parallel
  composition $|$ and for summation $+$.
\end{definition}

\subsection{Name equivalence}

We take name equivalence, written $\nameeq$, to be the smallest
equivalence relation generated by the following rules.

\begin{mathpar}
\inferrule*[lab=Quote-drop]
{ }
{ \quotep{@{x}} \nameeq x }

\inferrule*[lab=Struct-equiv]
{ P \scong Q }
{ \quotep{P} \nameeq \quotep{Q} }
\end{mathpar}

The astute reader will have noticed that the mutual recursion of names
and processes imposes a mutual recursion on alpha-equivalence and
structural equivalence via name-equivalence. Fortunately, all of this
works out pleasantly and we may calculate in the natural way, free of
concern. The reader interested in the details is referred to the
appendix \ref{appendix:rho_details}.

\subsection{Substitution}

We use $\Proc$ for the set of processes, $\QProc$ for the set of
names, and $\id{\{}\vec{y} / \vec{x} \id{\}}$ to denote partial maps,
$s : \QProc \rightarrow \QProc$. A map, $s$ lifts, uniquely, to a map
on process terms, $\widehat{s} : \Proc \rightarrow \Proc$ by the
following equations.

\begin{mathpar}
  (0) \psubstp{Q}{P} := 0 \\
  (R \juxtap S) \psubstp{Q}{P}
  :=    
  (R)\psubstp{Q}{P} \juxtap (S) \psubstp{Q}{P} \\
  (x?(y).R) \psubstp{Q}{P}    
  :=    
  (x)\substp{Q}{P} (z)\concat( (R \psubstn{z}{y}) \psubstp{Q}{P} ) \\
  (\lift{x}{R}) \psubstp{Q}{P}  
  :=
  \lift{(x)\substp{Q}{P}}{ R \psubstp{Q}{P} } \\
%   (\dropn{x})  \psubstp{Q}{P}       
%   := 
%   \left\{ 
%     \begin{array}{ccc} 
%       \dropn{\quotep{Q}} & & x \nameeq \quotep{P} \\
%       \dropn{x} & & otherwise \\
%     \end{array}
%   \right. 
  (\dropn{x})  \psubstp{Q}{P}       
  := 
  \left\{ 
    \begin{array}{ccc} 
      Q & & x \nameeq \quotep{P} \\
      \dropn{x} & & otherwise \\
    \end{array}
  \right.
\end{mathpar}
 

where

\begin{eqnarray}
  (x)\id{\{} \lpquote Q \rpquote / \lpquote P \rpquote \id{\}}            = 
  \left\{ 
    \begin{array}{ccc}
      \lpquote Q \rpquote & & x \nameeq \lpquote P \rpquote \\
      x & & otherwise \\
    \end{array}
  \right. \nonumber
\end{eqnarray}

and $z$ is chosen distinct from $\quotep{P}$, $\quotep{Q}$, the free
names in $Q$, and all the names in $R$. Our $\alpha$-equivalence will
be built in the standard way from this substitution.

\begin{remark}\label{rem:no_self_referential_names}
  One consequence of these definitions is that $\forall P. \quotep{P}
  \not\in \freenames{P}$.
\end{remark}

\subsection{ Dynamic quote: an example }

Anticipating something of what's to come, consider applying the
substitution, $\widehat{\id{\{}u / z \id{\}}}$, to the following pair
of processes, $\lift{w}{y!(z)}$ and $w[ \lpquote y!(z) \rpquote ]$.

\begin{eqnarray}
	\lift{w}{y!(z)}\widehat{\id{\{}u / z \id{\}}}
		& = &
		\lift{w}{y!(u)} \nonumber\\
	w[ \lpquote y!(z) \rpquote ] \widehat{ \id{\{}u / z \id{\}} }
		& = &
		w[ \lpquote y!(z) \rpquote ] \nonumber
\end{eqnarray}

Because the body of the process between quotes is impervious to
substitution, we get radically different answers. In fact, by
examining the first process in an input context,
e.g. $x?(z).\lift{w}{y!(z)}$, we see that the process under the lift
operator may be shaped by prefixed inputs binding a name inside it. In
this sense, the lift operator will be seen as a way to dynamically
construct processes before reifying them as names.

Finally equipped with these standard features we can present the
dynamics of the calculus.

\subsubsection{Operational semantics} 

Finally, we introduce the computational dynamics. What marks these
algebras as distinct from other more traditionally studied algebraic
structures, e.g. vector spaces or polynomial rings, is the manner in
which dynamics is captured. In traditional structures, dynamics is typically
expressed through morphisms between such structures, as in linear maps
between vector spaces or morphisms between rings. In algebras
associated with the semantics of computation, the dynamics is
expressed as part of the algebraic structure itself, through a
reduction reduction relation typically denoted by $\red$. Below, we
give a recursive presentation of this relation for the calculus used
in the encoding.

$\red \subseteq \pi \times \pi$
$\red : \pi \to \mathcal{P}(\pi)$

\begin{mathpar}
  \inferrule* [lab=Comm] { \textsf{match}( x_{src}, x_{trgt} ) } { x_{trgt}?(y)P \; | \; x_{src}!\langle {Q} \rangle \red P\{\quotep{Q}/y}\} }
  \and \\
  \inferrule* [lab=Par] {{P} \red {P}'} {{{P} | {Q}} \red {{P}' | {Q}}}
  \and
  \inferrule* [lab=Equiv]{{{P} \scong {P}'} \andalso {{P}' \red {Q}'} \andalso {{Q}' \scong {Q}}}{{P} \red {Q}}
\end{mathpar}

\begin{eqnarray*}
  match_{\equiv} (\quotep{P},\quotep{Q}) & := & P \equiv Q \\
  match_{\dagger}(\quotep{P},\quotep{Q}) & := & \forall R. P|Q \red^{*} R => R \red^{*} 0 \\
  match_{K}(\quotep{P},\quotep{Q}) & := & K \mbox{ for some context } K
\end{eqnarray*}

$u?(x)P | u!\langle Q \rangle \red P\{\quotep{Q}/x\}$

%We write $\wred$ for $\red^*$, and $P\red$ if $\exists Q $ such that $ P \red Q$.
We write $P\red$ if $\exists Q $ such that $ P \red Q$ and $P\not\red$, otherwise.

\section{Replication}

As mentioned before, it is known that replication (and hence
recursion) can be implemented in a higher-order process algebra
\cite{SangiorgiWalker}. As our first example of calculation with the
machinery thus far presented we give the construction explicitly in
the {\rhoc}.

\begin{eqnarray}
	D_{x} & := & \prefix{x}{y}{(\binpar{\outputp{x}{y}}{@{y}})} \nonumber\\
	\bangp_{x}{P} & := & \binpar{{x}!\langle{\binpar{D_{x}}{P}}\rangle}{D_{x}} \nonumber
\end{eqnarray}

\begin{eqnarray}
	\bangp_{x}{P} & & \nonumber\\
	=
	& {x}!\langle{(\prefix{x}{y}{(\outputp{x}{y} | @{y})) | P}}\rangle 
	      | \prefix{x}{y}{(\outputp{x}{y} | @{y})} & \nonumber\\
	\red
	& (\outputp{x}{y} | @{y})\substn{\quotep{(\prefix{x}{y}{(@{y} | \outputp{x}{y})) | P}}}{y} & \nonumber\\
	=
	& \outputp{x}{\quotep{(\prefix{x}{y}{(\outputp{x}{y} | @{y})) | P}}}
	  | {(\prefix{x}{y}{(\outputp{x}{y} | @{y})) | P}} & \nonumber\\
	\red
	& \ldots & \nonumber\\
	\red^*
	& P | P | \ldots & \nonumber
\end{eqnarray}

Of course, this encoding, as an implementation, runs away, unfolding
$\bangp{P}$ eagerly. A lazier and more implementable replication
operator, restricted to input-guarded processes, may be obtained as follows.

\begin{eqnarray}
\bangp{\prefix{u}{v}{P}} 
	:= 
	\binpar{\lift{x}{\prefix{u}{v}{(\binpar{D(x)}{P})}}}{D(x)} \nonumber
\end{eqnarray}

\begin{remark}
  Note that the lazier definition still does not deal with summation
  or mixed summation (i.e. sums over input and output). The reader is
  invited to construct definitions of replication that deal with these
  features. 

  Further, the definitions are parameterized in a name, $x$. Can you,
  gentle reader, make a definition that eliminates this parameter and
  guarantees no accidental interaction between the replication
  machinery and the process being replicated -- i.e. no accidental
  sharing of names used by the process to get its work done and the
  name(s) used by the replication to effect copying. This latter
  revision of the definition of replication is crucial to obtaining
  the expected identity $!!P \sim !P$.
\end{remark}

\begin{remark}\label{rem:paradoxical_combinator}
  The reader familiar with the lambda calculus will have noticed the
  similarity between $D$ and the paradoxical combinator.

  [Ed. note: the existence of this seems to suggest we have to be more
  restrictive on the set of processes and names we admit if we are to
  support no-cloning.]
\end{remark}

\subsubsection{Bisimulation}

The computational dynamics gives rise to another kind of equivalence,
the equivalence of computational behavior. As previously mentioned
this is typically captured \emph{via} some form of bisimulation.

% The notion we use in this paper is weak barbed bisimulation
% \cite{milner91polyadicpi}.

The notion we use in this paper is derived from weak barbed
bisimulation \cite{milner91polyadicpi}. 

\begin{definition}
An \emph{observation relation}, $\downarrow_{\mathcal N}$, over a set
of names, $\mathcal N$, is the smallest relation satisfying the rules
below.

\infrule[Out-barb]{y \in {\mathcal N}, \; x \nameeq y}
		  {\outputp{x}{v} \downarrow_{\mathcal N} x}
\infrule[Par-barb]{\mbox{$P\downarrow_{\mathcal N} x$ or $Q\downarrow_{\mathcal N} x$}}
		  {\binpar{P}{Q} \downarrow_{\mathcal N} x}

We write $P \Downarrow_{\mathcal N} x$ if there is $Q$ such that 
$P \wred Q$ and $Q \downarrow_{\mathcal N} x$.
\end{definition}

\begin{definition}
%\label{def.bbisim}
An  ${\mathcal N}$-\emph{barbed bisimulation} over a set of names, ${\mathcal N}$, is a symmetric binary relation 
${\mathcal S}_{\mathcal N}$ between agents such that $P\rel{S}_{\mathcal N}Q$ implies:
\begin{enumerate}
\item If $P \red P'$ then $Q \wred Q'$ and $P'\rel{S}_{\mathcal N} Q'$.
\item If $P\downarrow_{\mathcal N} x$, then $Q\Downarrow_{\mathcal N} x$.
\end{enumerate}
$P$ is ${\mathcal N}$-barbed bisimilar to $Q$, written
$P \wbbisim_{\mathcal N} Q$, if $P \rel{S}_{\mathcal N} Q$ for some ${\mathcal N}$-barbed bisimulation ${\mathcal S}_{\mathcal N}$.
\end{definition}

$\mathcal{R} \subseteq \pi \times \pi$

$P \mathcal{R} Q => \forall P'. P \red P' \Rightarrow \exists Q'. Q \red Q', P' \mathcal{R} Q'$

$P \vdash x \Rightarrow Q \vdash x$

\begin{mathpar}
  \inferrule*[lab=Out-barb]{x \nameeq y}{{y}!\langle{Q}\rangle \vdash x}
  \and
  \inferrule*[lab=Par-barb]{\mbox{$P\vdash x$ or $Q\vdash x$}}{\binpar{P}{Q} \vdash x}
\end{mathpar}

\subsubsection{Contexts}

One of the principle advantages of computational calculi like the
$\pi$-calculus is a well-defined notion of context,
contextual-equivalence and a correlation between
contextual-equivalence and notions of bisimulation. The notion of
context allows the decomposition of a process into (sub-)process and
its syntactic environment, its context. Thus, a context may be
thought of as a process with a ``hole'' (written $\Box$) in it. The
application of a context $M$ to a process $P$, written $M[P]$, is
tantamount to filling the hole in $M$ with $P$. In this paper we do
not need the full weight of this theory, but do make use of the notion
of context in the proof the main theorem. 

\begin{mathpar}
  \inferrule* [lab=summation] {} {{M_{M},M_{N}} \bc \Box \;|\; x.M_{A} \;|\; M_{M}+M_{N}}
  \and
  \inferrule* [lab=agent] {} {{M_{A}} \bc (\vec{x})M_{P} \;| \; \clift{P_0,\ldots,M_{P},\ldots,P_N}}
  \and \\
  \inferrule* [lab=process] {} {{M_{P}} \bc M_{N} \;| \;P|M_{P} }
\end{mathpar} 

\begin{mathpar}
  \inferrule* [lab=sychronization] {} {M_{N} \bc \Box \;|\; x?M_{F} \;|\; x!M_{C}}
  \and
  \inferrule* [lab=abstraction] {} {{M_{F}} \bc (x)M_{P} }
  \and
  \inferrule* [lab=concretion] {} {{M_{C}} \bc \langle M_{P} \rangle }
  \and \\
  \inferrule* [lab=process] {} {{M_{P}} \bc M_{N} \;| \;P|M_{P} }
\end{mathpar}

\begin{definition}[contextual application] Given a context $M$, and
  process $P$, we define the \emph{contextual application}, $M[P] :=
  M\{P/\Box\}$. That is, the contextual application of M to P is the
  substitution of $P$ for $\Box$ in $M$.
\end{definition}

$\meaningof{-} : L \to \mathcal{P}(\pi)$

\begin{mathpar}
  \inferrule* [lab=collection] {} {\meaningof{true} = \pi, \and \meaningof{~E} = \pi \setminus \meaningof{E}, \and \meaningof{E_{1} \& E_{2}} = \meaningof{E_{1}} \cap \meaningof{E_{2}}}
\end{mathpar}

\begin{mathpar}
  \inferrule* [lab=structure] {} {\meaningof{0} = \{ P \in \pi | P \equiv 0 \}, \and \\ \meaningof{E_1 | E_2} = \{ P \in \pi | P \equiv P_{1} | P_{2}, P_{1} \in \meaningof{E_{1}}, P_{2} \in \meaningof{E_2}\} }
\end{mathpar}

\begin{mathpar}
 \inferrule* [lab=behavior] {} {\meaningof{\langle a?b \rangle E} = \{ P \in \pi | P \equiv Q | u?(y)P', \\ \and \\\\ \and \\ \;\;\; u \in \meaningof{a}, \forall z.P'\{z/y\} \in \meaningof{E\{z/b\}}\}, \and \\ \meaningof{a!E} = \{ P \in \pi | P \equiv Q | x!\langle P' \rangle, x \in \meaningof{a} P' \in \meaningof{E}\} }
\end{mathpar}

\begin{mathpar}
 \inferrule* [lab=nominal] {} {\meaningof{\quotep{E}} = \{ \quotep{P} \in \quotep{\pi} | P \in \meaningof{E} \}, \and \meaningof{\quotep{P}} = \{ \quotep{Q} \in \quotep{\pi} | P \equiv Q \} \and \\ \meaningof{@\quotep{E}} = \{ P \in \pi | P \equiv @x, x \in \meaningof{E} \}}
\end{mathpar}

\begin{eqnarray*}
  \\
  \meaningof{-} : TS \to ST
\end{eqnarray*}

\begin{eqnarray*}
  \\
  L : TS \to ST
\end{eqnarray*}

\begin{eqnarray*}
  \\
  P \models E \iff P \in \meaningof{E}
\end{eqnarray*}

\begin{eqnarray*}
  P \approx_{L} Q \iff \forall E \in L. P \models E \iff Q \models E
\end{eqnarray*}

\begin{eqnarray*}
  P \approx_{K} Q
\end{eqnarray*}

\begin{eqnarray*}
  P \approx Q
\end{eqnarray*}

$\approx_{K} = \approx = \approx_{L}$

\subsubsection{Contextual duality}

Note that contexts extend the quotation operation to a family of
operations from processes to names. Given a context, $M$, we can
define a \emph{nominal context}, $\quotep{M}$ by $\quotep{M}[P] :=
\quotep{M[P]}$. To foreshadow what is to come we observe that these
operations enjoy a duality with processes very much like the duality
between vectors and maps from vectors to scalars.

Further, because the calculus is essentially higher-order, we have a
correspondence between contexts and processes. More specifically,
given a name $x$ and a context $M$ we can construct $M^{*}_{x}$ such
that 

\begin{mathpar}
  M^{*}_{x} | \lift{x}{P} \red M[P]
\end{mathpar}

namely,

\begin{mathpar}
  M^{*}_{x} := x?(u).M[\dropn{u}]
\end{mathpar}

The dependence of $M^{*}_{x}$ on a name makes it an abstraction, 

\begin{mathpar}
  M^{*} := (x)x?(u).M[\dropn{u}]
\end{mathpar}

\subsection{Additional notation}

It will sometimes be convenient to denote the process a name
quotes. We already have the notation $x = \quotep{P}$, but it will be
convenient to introduce an alternate notation, $\procn{x}$, when we
want to emphasize the connection to the use of the name. Note that, by
virtue of name equivalence, $\quotep{\procn{x}} \nameeq x$; so, the
notation is consistent with previous definitions.

Further, because names have structure it is possible to effect
substitutions on the basis of that structure. This means we need to
upgrade our notation for substitutions, which we accomplish by
adapting comprehension notation. Thus,

\begin{mathpar}
  P\{ y / x : x \in S \}
\end{mathpar}

is interpreted to mean the process derived from P by replacing (in a
capture-avoiding manner) each occurrence of $x$ in $S$ by $y$. For example,

\begin{mathpar}
  P\{ \quotep{\procn{x}|\procn{x}} / x : x \in \freenames{P} \}
\end{mathpar}

will replace each (occurrence) of a free name $x$ in $P$ by
$\quotep{\procn{x}|\procn{x}}$.

Also, we will avail ourselves of the notation $x^{L}$ and $x^{R}$ to
denote injections of a name into disjoint copies of the name
space. There are numerous ways to accomplish this. One example can be
found in \cite{MeredithR05}. This notation overloads to vectors of
names: $\vec{x}^{\pi} := (x_{i}^{\pi} \; : \; 0 \leq i < |\vec{x}| )$ where $\pi \in \{L,R\}$.

We also use $P^{\Box} := P|\Box$.

In \cite{MeredithR05} an interpretation of the new operator is
given. It turns out that there are several possible interpretations
all enjoying the requisite algebraic properties of the operator (see
\cite{milner91polyadicpi}). We will therefore make liberal use of
$(\nu\; \vec{x})P$.

% subsection the_syntax_and_semantics_of_the_notation_system (end)   

\input{qm2pi.qmops} 

\input{qm2pi.sterngerlach} 

\input{qm2pi.metric} 

% section concurrent_process_calculi (end)

%\input{qm2pi.proofsketch}

% section proof sketch (end)

%\input{qm2pi.slviaknots} 

% section spatial logic via knots (end)

\input{qm2pi.conclusion}

% section conclusion (end)

%\input{qm2pi.dtcodes} 

% section wiring algorithm (end)

\input{qm2pi.ack} 

% section acknowledgments (end)

\newpage


\bibliographystyle{plain}   
\bibliography{../../biblios/main.bib}

\input{qm2pi.rhodetails}

\end{document}

 

% section notation (end)

\input{qm2pi.process.calculi} 

% section concurrent_process_calculi_and_spatial_logics_ (end)
    
%\documentclass[12pt]{llncs}
%\documentclass{jktr}

\usepackage[pdftex]{hyperref}                   
\usepackage {listings}
\usepackage {mathpartir}
\usepackage{bcprules}
%\usepackage{listings}
                       
\usepackage{graphicx} 
%\usepackage[margins=2.5cm,nohead,nofoot]{geometry}
%\usepackage{geometry}
\usepackage{amsfonts}
\usepackage{amstext}
\usepackage{latexsym}
\usepackage{amssymb}
\usepackage{color}


%\include{myPreamble}
\include{qm2pi.local} 

%\ifpdf
%\usepackage[pdftex]{graphicx}
%\else
%\usepackage{graphicx}
%\fi

 % \ifpdf
%  \usepackage{pdfsync}
%  \if


%\title{Brief Article}
%\author{David F. Snyder}
%\author{L.G. Meredith}

%\address{Dept. of Math., Texas State University--San Marcos, San Marcos, TX 78666}
       
\pagestyle{empty}


\begin{document}

\lstset{language=[Objective]Caml,frame=shadowbox}

\input{qm2pi.front}

% section front matter (end)

\input{qm2pi.intro} 
 
% section introduction (end)

% \input{qm2pi.knotations} 

% section notation (end)

\input{qm2pi.process.calculi} 

% section concurrent_process_calculi_and_spatial_logics_ (end)
    
%\input{qm2pi.knots2pi} 

%\input{qm2pi.trefoil} 

%\input{qm2pi.mainthm} 

% subsection basic_interpretation (end)

%\input{qm2pi.rho.presentation} 
\subsection{The syntax and semantics of the notation system}\label{sub:the_syntax_and_semantics_of_the_notation_system} % (fold)

We now summarize a technical presentation of the calculus that
embodies our theory of dynamics. The typical presentation of such a
calculus follows the style of giving generators and relations on
them. The grammar, below, describing term constructors, freely
generates the set of processes, $\Proc$. This set is then quotiented
by a relation known as structural congruence and it is over this set
that the notion of dynamics is expressed. This presentation is
essentially that of \cite{MeredithR05} with the addition of
polyadicity and summation. For readability we have relegated some of
the technical subtleties to an appendix.

\subsubsection{Process grammar}\label{subsub:process_grammar}

\begin{mathpar}
  \inferrule* [lab=synchronization] {} {{M} \bc \pzero \;|\; x?F \;|\; x!C }
  \and
  \inferrule* [lab=abstraction] {} {{F} \bc (x)P}
  \and
  \inferrule* [lab=concretion] {} {{C} \bc \langle Q \rangle}
  \and
  \inferrule* [lab=process] {} {{P,Q} \bc M \;| \;P|Q \;|\; @{x}}
  \and
  \inferrule* [lab=name] {} {{x} \bc \quotep{P}}
\end{mathpar} 

Note that $\vec{x}$ (resp. $\vec{P}$) denotes a vector of names
(resp. processes) of length $|\vec{x}|$ (resp. $|\vec{P}|$). We adopt
the following useful abbreviations.

\begin{mathpar}
   x?(\vec{y}).P := x.(\vec{y})P \and  x\clift{\vec{P}} := x.\clift{\vec{P}}
   \and x!(y) := \lift{x}{\dropn{y}}
   \and \Pi_{i=0}^{n-1}P_i := P_0 | \ldots | P_{n-1}
\end{mathpar}

\subsubsection{Structural congruence}

\paragraph{Free and bound names and alpha-equivalence.} At the
core of structural equivalence is alpha-equivalence which identifies
process that are the same up to a change of variable. Formally, we
recognize the distinction between free and bound names. The free names
of a process, $\freenames{P}$, may be calculated recursively as
follows:

\begin{mathpar}
\freenames{\pzero} := \emptyset
  \and \\
  \freenames{x?(y).P} := \{ x \} \cup (\freenames{P} \setminus \{ y \})
  \and 
  \freenames{x!\langle P \rangle} := \{ x \} \cup \{ P \} 
  \and \\
  \freenames{P|Q} := \freenames{P} \cup \freenames{Q}
  \and \\
  \freenames{@{x}} := \{ x \}
\end{mathpar}

$\pi$
$\quotep{\pi}$

$\freenames{-} : \pi \to \mathcal{P}(\quotep{\pi})$

\begin{eqnarray*}
  \freenames{\pzero} & := & \emptyset \\
  \freenames{x?(y).P} & := & \{ x \} \cup (\freenames{P} \setminus \{ y \}) \\
  \freenames{x!\langle P \rangle} & := & \{ x \} \cup \{ P \} \\
  \freenames{P|Q} & := & \freenames{P} \cup \freenames{Q} \\
  \freenames{\dropn{x}} & := & \{ x \}
\end{eqnarray*}

The bound names of a process, $\boundnames{P}$, are those names occurring in $P$
that are not free. For example, in $x?(y).0$, the name $x$ is free, while $y$ is bound.

\begin{mathpar}
  \inferrule* [lab=monoidal-laws] {} { P|Q \equiv Q|P \and P|0 \equiv P \and P|(Q|R) \equiv (P|Q)|R }
\end{mathpar}

\begin{mathpar}
  \inferrule* [lab=alpha-equivalence] {} { (x)P \equiv (y)P\{y/x\} \and y \not\in \freenames{P} }
\end{mathpar}

\begin{definition}
Then two processes, $P,Q$, are alpha-equivalent if $P = Q\{\vec{y}/\vec{x}\}$ for
some $\vec{x} \in \boundnames{Q},\vec{y} \in \boundnames{P}$, where $Q\{\vec{y}/\vec{x}\}$
denotes the capture-avoiding substitution of $\vec{y}$ for $\vec{x}$ in $Q$.
\end{definition}

\begin{definition}
  The {\em structural congruence} \cite{SangiorgiWalker} , $\equiv$,
  between processes is the least congruence containing
  alpha-equivalence, satisfying the abelian monoid laws
  (associativity, commutativity and $\pzero$ as identity) for parallel
  composition $|$ and for summation $+$.
\end{definition}

\subsection{Name equivalence}

We take name equivalence, written $\nameeq$, to be the smallest
equivalence relation generated by the following rules.

\begin{mathpar}
\inferrule*[lab=Quote-drop]
{ }
{ \quotep{@{x}} \nameeq x }

\inferrule*[lab=Struct-equiv]
{ P \scong Q }
{ \quotep{P} \nameeq \quotep{Q} }
\end{mathpar}

The astute reader will have noticed that the mutual recursion of names
and processes imposes a mutual recursion on alpha-equivalence and
structural equivalence via name-equivalence. Fortunately, all of this
works out pleasantly and we may calculate in the natural way, free of
concern. The reader interested in the details is referred to the
appendix \ref{appendix:rho_details}.

\subsection{Substitution}

We use $\Proc$ for the set of processes, $\QProc$ for the set of
names, and $\id{\{}\vec{y} / \vec{x} \id{\}}$ to denote partial maps,
$s : \QProc \rightarrow \QProc$. A map, $s$ lifts, uniquely, to a map
on process terms, $\widehat{s} : \Proc \rightarrow \Proc$ by the
following equations.

\begin{mathpar}
  (0) \psubstp{Q}{P} := 0 \\
  (R \juxtap S) \psubstp{Q}{P}
  :=    
  (R)\psubstp{Q}{P} \juxtap (S) \psubstp{Q}{P} \\
  (x?(y).R) \psubstp{Q}{P}    
  :=    
  (x)\substp{Q}{P} (z)\concat( (R \psubstn{z}{y}) \psubstp{Q}{P} ) \\
  (\lift{x}{R}) \psubstp{Q}{P}  
  :=
  \lift{(x)\substp{Q}{P}}{ R \psubstp{Q}{P} } \\
%   (\dropn{x})  \psubstp{Q}{P}       
%   := 
%   \left\{ 
%     \begin{array}{ccc} 
%       \dropn{\quotep{Q}} & & x \nameeq \quotep{P} \\
%       \dropn{x} & & otherwise \\
%     \end{array}
%   \right. 
  (\dropn{x})  \psubstp{Q}{P}       
  := 
  \left\{ 
    \begin{array}{ccc} 
      Q & & x \nameeq \quotep{P} \\
      \dropn{x} & & otherwise \\
    \end{array}
  \right.
\end{mathpar}
 

where

\begin{eqnarray}
  (x)\id{\{} \lpquote Q \rpquote / \lpquote P \rpquote \id{\}}            = 
  \left\{ 
    \begin{array}{ccc}
      \lpquote Q \rpquote & & x \nameeq \lpquote P \rpquote \\
      x & & otherwise \\
    \end{array}
  \right. \nonumber
\end{eqnarray}

and $z$ is chosen distinct from $\quotep{P}$, $\quotep{Q}$, the free
names in $Q$, and all the names in $R$. Our $\alpha$-equivalence will
be built in the standard way from this substitution.

\begin{remark}\label{rem:no_self_referential_names}
  One consequence of these definitions is that $\forall P. \quotep{P}
  \not\in \freenames{P}$.
\end{remark}

\subsection{ Dynamic quote: an example }

Anticipating something of what's to come, consider applying the
substitution, $\widehat{\id{\{}u / z \id{\}}}$, to the following pair
of processes, $\lift{w}{y!(z)}$ and $w[ \lpquote y!(z) \rpquote ]$.

\begin{eqnarray}
	\lift{w}{y!(z)}\widehat{\id{\{}u / z \id{\}}}
		& = &
		\lift{w}{y!(u)} \nonumber\\
	w[ \lpquote y!(z) \rpquote ] \widehat{ \id{\{}u / z \id{\}} }
		& = &
		w[ \lpquote y!(z) \rpquote ] \nonumber
\end{eqnarray}

Because the body of the process between quotes is impervious to
substitution, we get radically different answers. In fact, by
examining the first process in an input context,
e.g. $x?(z).\lift{w}{y!(z)}$, we see that the process under the lift
operator may be shaped by prefixed inputs binding a name inside it. In
this sense, the lift operator will be seen as a way to dynamically
construct processes before reifying them as names.

Finally equipped with these standard features we can present the
dynamics of the calculus.

\subsubsection{Operational semantics} 

Finally, we introduce the computational dynamics. What marks these
algebras as distinct from other more traditionally studied algebraic
structures, e.g. vector spaces or polynomial rings, is the manner in
which dynamics is captured. In traditional structures, dynamics is typically
expressed through morphisms between such structures, as in linear maps
between vector spaces or morphisms between rings. In algebras
associated with the semantics of computation, the dynamics is
expressed as part of the algebraic structure itself, through a
reduction reduction relation typically denoted by $\red$. Below, we
give a recursive presentation of this relation for the calculus used
in the encoding.

$\red \subseteq \pi \times \pi$
$\red : \pi \to \mathcal{P}(\pi)$

\begin{mathpar}
  \inferrule* [lab=Comm] { \textsf{match}( x_{src}, x_{trgt} ) } { x_{trgt}?(y)P \; | \; x_{src}!\langle {Q} \rangle \red P\{\quotep{Q}/y}\} }
  \and \\
  \inferrule* [lab=Par] {{P} \red {P}'} {{{P} | {Q}} \red {{P}' | {Q}}}
  \and
  \inferrule* [lab=Equiv]{{{P} \scong {P}'} \andalso {{P}' \red {Q}'} \andalso {{Q}' \scong {Q}}}{{P} \red {Q}}
\end{mathpar}

\begin{eqnarray*}
  match_{\equiv} (\quotep{P},\quotep{Q}) & := & P \equiv Q \\
  match_{\dagger}(\quotep{P},\quotep{Q}) & := & \forall R. P|Q \red^{*} R => R \red^{*} 0 \\
  match_{K}(\quotep{P},\quotep{Q}) & := & K \mbox{ for some context } K
\end{eqnarray*}

$u?(x)P | u!\langle Q \rangle \red P\{\quotep{Q}/x\}$

%We write $\wred$ for $\red^*$, and $P\red$ if $\exists Q $ such that $ P \red Q$.
We write $P\red$ if $\exists Q $ such that $ P \red Q$ and $P\not\red$, otherwise.

\section{Replication}

As mentioned before, it is known that replication (and hence
recursion) can be implemented in a higher-order process algebra
\cite{SangiorgiWalker}. As our first example of calculation with the
machinery thus far presented we give the construction explicitly in
the {\rhoc}.

\begin{eqnarray}
	D_{x} & := & \prefix{x}{y}{(\binpar{\outputp{x}{y}}{@{y}})} \nonumber\\
	\bangp_{x}{P} & := & \binpar{{x}!\langle{\binpar{D_{x}}{P}}\rangle}{D_{x}} \nonumber
\end{eqnarray}

\begin{eqnarray}
	\bangp_{x}{P} & & \nonumber\\
	=
	& {x}!\langle{(\prefix{x}{y}{(\outputp{x}{y} | @{y})) | P}}\rangle 
	      | \prefix{x}{y}{(\outputp{x}{y} | @{y})} & \nonumber\\
	\red
	& (\outputp{x}{y} | @{y})\substn{\quotep{(\prefix{x}{y}{(@{y} | \outputp{x}{y})) | P}}}{y} & \nonumber\\
	=
	& \outputp{x}{\quotep{(\prefix{x}{y}{(\outputp{x}{y} | @{y})) | P}}}
	  | {(\prefix{x}{y}{(\outputp{x}{y} | @{y})) | P}} & \nonumber\\
	\red
	& \ldots & \nonumber\\
	\red^*
	& P | P | \ldots & \nonumber
\end{eqnarray}

Of course, this encoding, as an implementation, runs away, unfolding
$\bangp{P}$ eagerly. A lazier and more implementable replication
operator, restricted to input-guarded processes, may be obtained as follows.

\begin{eqnarray}
\bangp{\prefix{u}{v}{P}} 
	:= 
	\binpar{\lift{x}{\prefix{u}{v}{(\binpar{D(x)}{P})}}}{D(x)} \nonumber
\end{eqnarray}

\begin{remark}
  Note that the lazier definition still does not deal with summation
  or mixed summation (i.e. sums over input and output). The reader is
  invited to construct definitions of replication that deal with these
  features. 

  Further, the definitions are parameterized in a name, $x$. Can you,
  gentle reader, make a definition that eliminates this parameter and
  guarantees no accidental interaction between the replication
  machinery and the process being replicated -- i.e. no accidental
  sharing of names used by the process to get its work done and the
  name(s) used by the replication to effect copying. This latter
  revision of the definition of replication is crucial to obtaining
  the expected identity $!!P \sim !P$.
\end{remark}

\begin{remark}\label{rem:paradoxical_combinator}
  The reader familiar with the lambda calculus will have noticed the
  similarity between $D$ and the paradoxical combinator.

  [Ed. note: the existence of this seems to suggest we have to be more
  restrictive on the set of processes and names we admit if we are to
  support no-cloning.]
\end{remark}

\subsubsection{Bisimulation}

The computational dynamics gives rise to another kind of equivalence,
the equivalence of computational behavior. As previously mentioned
this is typically captured \emph{via} some form of bisimulation.

% The notion we use in this paper is weak barbed bisimulation
% \cite{milner91polyadicpi}.

The notion we use in this paper is derived from weak barbed
bisimulation \cite{milner91polyadicpi}. 

\begin{definition}
An \emph{observation relation}, $\downarrow_{\mathcal N}$, over a set
of names, $\mathcal N$, is the smallest relation satisfying the rules
below.

\infrule[Out-barb]{y \in {\mathcal N}, \; x \nameeq y}
		  {\outputp{x}{v} \downarrow_{\mathcal N} x}
\infrule[Par-barb]{\mbox{$P\downarrow_{\mathcal N} x$ or $Q\downarrow_{\mathcal N} x$}}
		  {\binpar{P}{Q} \downarrow_{\mathcal N} x}

We write $P \Downarrow_{\mathcal N} x$ if there is $Q$ such that 
$P \wred Q$ and $Q \downarrow_{\mathcal N} x$.
\end{definition}

\begin{definition}
%\label{def.bbisim}
An  ${\mathcal N}$-\emph{barbed bisimulation} over a set of names, ${\mathcal N}$, is a symmetric binary relation 
${\mathcal S}_{\mathcal N}$ between agents such that $P\rel{S}_{\mathcal N}Q$ implies:
\begin{enumerate}
\item If $P \red P'$ then $Q \wred Q'$ and $P'\rel{S}_{\mathcal N} Q'$.
\item If $P\downarrow_{\mathcal N} x$, then $Q\Downarrow_{\mathcal N} x$.
\end{enumerate}
$P$ is ${\mathcal N}$-barbed bisimilar to $Q$, written
$P \wbbisim_{\mathcal N} Q$, if $P \rel{S}_{\mathcal N} Q$ for some ${\mathcal N}$-barbed bisimulation ${\mathcal S}_{\mathcal N}$.
\end{definition}

$\mathcal{R} \subseteq \pi \times \pi$

$P \mathcal{R} Q => \forall P'. P \red P' \Rightarrow \exists Q'. Q \red Q', P' \mathcal{R} Q'$

$P \vdash x \Rightarrow Q \vdash x$

\begin{mathpar}
  \inferrule*[lab=Out-barb]{x \nameeq y}{{y}!\langle{Q}\rangle \vdash x}
  \and
  \inferrule*[lab=Par-barb]{\mbox{$P\vdash x$ or $Q\vdash x$}}{\binpar{P}{Q} \vdash x}
\end{mathpar}

\subsubsection{Contexts}

One of the principle advantages of computational calculi like the
$\pi$-calculus is a well-defined notion of context,
contextual-equivalence and a correlation between
contextual-equivalence and notions of bisimulation. The notion of
context allows the decomposition of a process into (sub-)process and
its syntactic environment, its context. Thus, a context may be
thought of as a process with a ``hole'' (written $\Box$) in it. The
application of a context $M$ to a process $P$, written $M[P]$, is
tantamount to filling the hole in $M$ with $P$. In this paper we do
not need the full weight of this theory, but do make use of the notion
of context in the proof the main theorem. 

\begin{mathpar}
  \inferrule* [lab=summation] {} {{M_{M},M_{N}} \bc \Box \;|\; x.M_{A} \;|\; M_{M}+M_{N}}
  \and
  \inferrule* [lab=agent] {} {{M_{A}} \bc (\vec{x})M_{P} \;| \; \clift{P_0,\ldots,M_{P},\ldots,P_N}}
  \and \\
  \inferrule* [lab=process] {} {{M_{P}} \bc M_{N} \;| \;P|M_{P} }
\end{mathpar} 

\begin{mathpar}
  \inferrule* [lab=sychronization] {} {M_{N} \bc \Box \;|\; x?M_{F} \;|\; x!M_{C}}
  \and
  \inferrule* [lab=abstraction] {} {{M_{F}} \bc (x)M_{P} }
  \and
  \inferrule* [lab=concretion] {} {{M_{C}} \bc \langle M_{P} \rangle }
  \and \\
  \inferrule* [lab=process] {} {{M_{P}} \bc M_{N} \;| \;P|M_{P} }
\end{mathpar}

\begin{definition}[contextual application] Given a context $M$, and
  process $P$, we define the \emph{contextual application}, $M[P] :=
  M\{P/\Box\}$. That is, the contextual application of M to P is the
  substitution of $P$ for $\Box$ in $M$.
\end{definition}

$\meaningof{-} : L \to \mathcal{P}(\pi)$

\begin{mathpar}
  \inferrule* [lab=collection] {} {\meaningof{true} = \pi, \and \meaningof{~E} = \pi \setminus \meaningof{E}, \and \meaningof{E_{1} \& E_{2}} = \meaningof{E_{1}} \cap \meaningof{E_{2}}}
\end{mathpar}

\begin{mathpar}
  \inferrule* [lab=structure] {} {\meaningof{0} = \{ P \in \pi | P \equiv 0 \}, \and \\ \meaningof{E_1 | E_2} = \{ P \in \pi | P \equiv P_{1} | P_{2}, P_{1} \in \meaningof{E_{1}}, P_{2} \in \meaningof{E_2}\} }
\end{mathpar}

\begin{mathpar}
 \inferrule* [lab=behavior] {} {\meaningof{\langle a?b \rangle E} = \{ P \in \pi | P \equiv Q | u?(y)P', \\ \and \\\\ \and \\ \;\;\; u \in \meaningof{a}, \forall z.P'\{z/y\} \in \meaningof{E\{z/b\}}\}, \and \\ \meaningof{a!E} = \{ P \in \pi | P \equiv Q | x!\langle P' \rangle, x \in \meaningof{a} P' \in \meaningof{E}\} }
\end{mathpar}

\begin{mathpar}
 \inferrule* [lab=nominal] {} {\meaningof{\quotep{E}} = \{ \quotep{P} \in \quotep{\pi} | P \in \meaningof{E} \}, \and \meaningof{\quotep{P}} = \{ \quotep{Q} \in \quotep{\pi} | P \equiv Q \} \and \\ \meaningof{@\quotep{E}} = \{ P \in \pi | P \equiv @x, x \in \meaningof{E} \}}
\end{mathpar}

\begin{eqnarray*}
  \\
  \meaningof{-} : TS \to ST
\end{eqnarray*}

\begin{eqnarray*}
  \\
  L : TS \to ST
\end{eqnarray*}

\begin{eqnarray*}
  \\
  P \models E \iff P \in \meaningof{E}
\end{eqnarray*}

\begin{eqnarray*}
  P \approx_{L} Q \iff \forall E \in L. P \models E \iff Q \models E
\end{eqnarray*}

\begin{eqnarray*}
  P \approx_{K} Q
\end{eqnarray*}

\begin{eqnarray*}
  P \approx Q
\end{eqnarray*}

$\approx_{K} = \approx = \approx_{L}$

\subsubsection{Contextual duality}

Note that contexts extend the quotation operation to a family of
operations from processes to names. Given a context, $M$, we can
define a \emph{nominal context}, $\quotep{M}$ by $\quotep{M}[P] :=
\quotep{M[P]}$. To foreshadow what is to come we observe that these
operations enjoy a duality with processes very much like the duality
between vectors and maps from vectors to scalars.

Further, because the calculus is essentially higher-order, we have a
correspondence between contexts and processes. More specifically,
given a name $x$ and a context $M$ we can construct $M^{*}_{x}$ such
that 

\begin{mathpar}
  M^{*}_{x} | \lift{x}{P} \red M[P]
\end{mathpar}

namely,

\begin{mathpar}
  M^{*}_{x} := x?(u).M[\dropn{u}]
\end{mathpar}

The dependence of $M^{*}_{x}$ on a name makes it an abstraction, 

\begin{mathpar}
  M^{*} := (x)x?(u).M[\dropn{u}]
\end{mathpar}

\subsection{Additional notation}

It will sometimes be convenient to denote the process a name
quotes. We already have the notation $x = \quotep{P}$, but it will be
convenient to introduce an alternate notation, $\procn{x}$, when we
want to emphasize the connection to the use of the name. Note that, by
virtue of name equivalence, $\quotep{\procn{x}} \nameeq x$; so, the
notation is consistent with previous definitions.

Further, because names have structure it is possible to effect
substitutions on the basis of that structure. This means we need to
upgrade our notation for substitutions, which we accomplish by
adapting comprehension notation. Thus,

\begin{mathpar}
  P\{ y / x : x \in S \}
\end{mathpar}

is interpreted to mean the process derived from P by replacing (in a
capture-avoiding manner) each occurrence of $x$ in $S$ by $y$. For example,

\begin{mathpar}
  P\{ \quotep{\procn{x}|\procn{x}} / x : x \in \freenames{P} \}
\end{mathpar}

will replace each (occurrence) of a free name $x$ in $P$ by
$\quotep{\procn{x}|\procn{x}}$.

Also, we will avail ourselves of the notation $x^{L}$ and $x^{R}$ to
denote injections of a name into disjoint copies of the name
space. There are numerous ways to accomplish this. One example can be
found in \cite{MeredithR05}. This notation overloads to vectors of
names: $\vec{x}^{\pi} := (x_{i}^{\pi} \; : \; 0 \leq i < |\vec{x}| )$ where $\pi \in \{L,R\}$.

We also use $P^{\Box} := P|\Box$.

In \cite{MeredithR05} an interpretation of the new operator is
given. It turns out that there are several possible interpretations
all enjoying the requisite algebraic properties of the operator (see
\cite{milner91polyadicpi}). We will therefore make liberal use of
$(\nu\; \vec{x})P$.

% subsection the_syntax_and_semantics_of_the_notation_system (end)   

\input{qm2pi.qmops} 

\input{qm2pi.sterngerlach} 

\input{qm2pi.metric} 

% section concurrent_process_calculi (end)

%\input{qm2pi.proofsketch}

% section proof sketch (end)

%\input{qm2pi.slviaknots} 

% section spatial logic via knots (end)

\input{qm2pi.conclusion}

% section conclusion (end)

%\input{qm2pi.dtcodes} 

% section wiring algorithm (end)

\input{qm2pi.ack} 

% section acknowledgments (end)

\newpage


\bibliographystyle{plain}   
\bibliography{../../biblios/main.bib}

\input{qm2pi.rhodetails}

\end{document}

 

%\documentclass[12pt]{llncs}
%\documentclass{jktr}

\usepackage[pdftex]{hyperref}                   
\usepackage {listings}
\usepackage {mathpartir}
\usepackage{bcprules}
%\usepackage{listings}
                       
\usepackage{graphicx} 
%\usepackage[margins=2.5cm,nohead,nofoot]{geometry}
%\usepackage{geometry}
\usepackage{amsfonts}
\usepackage{amstext}
\usepackage{latexsym}
\usepackage{amssymb}
\usepackage{color}


%\include{myPreamble}
\include{qm2pi.local} 

%\ifpdf
%\usepackage[pdftex]{graphicx}
%\else
%\usepackage{graphicx}
%\fi

 % \ifpdf
%  \usepackage{pdfsync}
%  \if


%\title{Brief Article}
%\author{David F. Snyder}
%\author{L.G. Meredith}

%\address{Dept. of Math., Texas State University--San Marcos, San Marcos, TX 78666}
       
\pagestyle{empty}


\begin{document}

\lstset{language=[Objective]Caml,frame=shadowbox}

\input{qm2pi.front}

% section front matter (end)

\input{qm2pi.intro} 
 
% section introduction (end)

% \input{qm2pi.knotations} 

% section notation (end)

\input{qm2pi.process.calculi} 

% section concurrent_process_calculi_and_spatial_logics_ (end)
    
%\input{qm2pi.knots2pi} 

%\input{qm2pi.trefoil} 

%\input{qm2pi.mainthm} 

% subsection basic_interpretation (end)

%\input{qm2pi.rho.presentation} 
\subsection{The syntax and semantics of the notation system}\label{sub:the_syntax_and_semantics_of_the_notation_system} % (fold)

We now summarize a technical presentation of the calculus that
embodies our theory of dynamics. The typical presentation of such a
calculus follows the style of giving generators and relations on
them. The grammar, below, describing term constructors, freely
generates the set of processes, $\Proc$. This set is then quotiented
by a relation known as structural congruence and it is over this set
that the notion of dynamics is expressed. This presentation is
essentially that of \cite{MeredithR05} with the addition of
polyadicity and summation. For readability we have relegated some of
the technical subtleties to an appendix.

\subsubsection{Process grammar}\label{subsub:process_grammar}

\begin{mathpar}
  \inferrule* [lab=synchronization] {} {{M} \bc \pzero \;|\; x?F \;|\; x!C }
  \and
  \inferrule* [lab=abstraction] {} {{F} \bc (x)P}
  \and
  \inferrule* [lab=concretion] {} {{C} \bc \langle Q \rangle}
  \and
  \inferrule* [lab=process] {} {{P,Q} \bc M \;| \;P|Q \;|\; @{x}}
  \and
  \inferrule* [lab=name] {} {{x} \bc \quotep{P}}
\end{mathpar} 

Note that $\vec{x}$ (resp. $\vec{P}$) denotes a vector of names
(resp. processes) of length $|\vec{x}|$ (resp. $|\vec{P}|$). We adopt
the following useful abbreviations.

\begin{mathpar}
   x?(\vec{y}).P := x.(\vec{y})P \and  x\clift{\vec{P}} := x.\clift{\vec{P}}
   \and x!(y) := \lift{x}{\dropn{y}}
   \and \Pi_{i=0}^{n-1}P_i := P_0 | \ldots | P_{n-1}
\end{mathpar}

\subsubsection{Structural congruence}

\paragraph{Free and bound names and alpha-equivalence.} At the
core of structural equivalence is alpha-equivalence which identifies
process that are the same up to a change of variable. Formally, we
recognize the distinction between free and bound names. The free names
of a process, $\freenames{P}$, may be calculated recursively as
follows:

\begin{mathpar}
\freenames{\pzero} := \emptyset
  \and \\
  \freenames{x?(y).P} := \{ x \} \cup (\freenames{P} \setminus \{ y \})
  \and 
  \freenames{x!\langle P \rangle} := \{ x \} \cup \{ P \} 
  \and \\
  \freenames{P|Q} := \freenames{P} \cup \freenames{Q}
  \and \\
  \freenames{@{x}} := \{ x \}
\end{mathpar}

$\pi$
$\quotep{\pi}$

$\freenames{-} : \pi \to \mathcal{P}(\quotep{\pi})$

\begin{eqnarray*}
  \freenames{\pzero} & := & \emptyset \\
  \freenames{x?(y).P} & := & \{ x \} \cup (\freenames{P} \setminus \{ y \}) \\
  \freenames{x!\langle P \rangle} & := & \{ x \} \cup \{ P \} \\
  \freenames{P|Q} & := & \freenames{P} \cup \freenames{Q} \\
  \freenames{\dropn{x}} & := & \{ x \}
\end{eqnarray*}

The bound names of a process, $\boundnames{P}$, are those names occurring in $P$
that are not free. For example, in $x?(y).0$, the name $x$ is free, while $y$ is bound.

\begin{mathpar}
  \inferrule* [lab=monoidal-laws] {} { P|Q \equiv Q|P \and P|0 \equiv P \and P|(Q|R) \equiv (P|Q)|R }
\end{mathpar}

\begin{mathpar}
  \inferrule* [lab=alpha-equivalence] {} { (x)P \equiv (y)P\{y/x\} \and y \not\in \freenames{P} }
\end{mathpar}

\begin{definition}
Then two processes, $P,Q$, are alpha-equivalent if $P = Q\{\vec{y}/\vec{x}\}$ for
some $\vec{x} \in \boundnames{Q},\vec{y} \in \boundnames{P}$, where $Q\{\vec{y}/\vec{x}\}$
denotes the capture-avoiding substitution of $\vec{y}$ for $\vec{x}$ in $Q$.
\end{definition}

\begin{definition}
  The {\em structural congruence} \cite{SangiorgiWalker} , $\equiv$,
  between processes is the least congruence containing
  alpha-equivalence, satisfying the abelian monoid laws
  (associativity, commutativity and $\pzero$ as identity) for parallel
  composition $|$ and for summation $+$.
\end{definition}

\subsection{Name equivalence}

We take name equivalence, written $\nameeq$, to be the smallest
equivalence relation generated by the following rules.

\begin{mathpar}
\inferrule*[lab=Quote-drop]
{ }
{ \quotep{@{x}} \nameeq x }

\inferrule*[lab=Struct-equiv]
{ P \scong Q }
{ \quotep{P} \nameeq \quotep{Q} }
\end{mathpar}

The astute reader will have noticed that the mutual recursion of names
and processes imposes a mutual recursion on alpha-equivalence and
structural equivalence via name-equivalence. Fortunately, all of this
works out pleasantly and we may calculate in the natural way, free of
concern. The reader interested in the details is referred to the
appendix \ref{appendix:rho_details}.

\subsection{Substitution}

We use $\Proc$ for the set of processes, $\QProc$ for the set of
names, and $\id{\{}\vec{y} / \vec{x} \id{\}}$ to denote partial maps,
$s : \QProc \rightarrow \QProc$. A map, $s$ lifts, uniquely, to a map
on process terms, $\widehat{s} : \Proc \rightarrow \Proc$ by the
following equations.

\begin{mathpar}
  (0) \psubstp{Q}{P} := 0 \\
  (R \juxtap S) \psubstp{Q}{P}
  :=    
  (R)\psubstp{Q}{P} \juxtap (S) \psubstp{Q}{P} \\
  (x?(y).R) \psubstp{Q}{P}    
  :=    
  (x)\substp{Q}{P} (z)\concat( (R \psubstn{z}{y}) \psubstp{Q}{P} ) \\
  (\lift{x}{R}) \psubstp{Q}{P}  
  :=
  \lift{(x)\substp{Q}{P}}{ R \psubstp{Q}{P} } \\
%   (\dropn{x})  \psubstp{Q}{P}       
%   := 
%   \left\{ 
%     \begin{array}{ccc} 
%       \dropn{\quotep{Q}} & & x \nameeq \quotep{P} \\
%       \dropn{x} & & otherwise \\
%     \end{array}
%   \right. 
  (\dropn{x})  \psubstp{Q}{P}       
  := 
  \left\{ 
    \begin{array}{ccc} 
      Q & & x \nameeq \quotep{P} \\
      \dropn{x} & & otherwise \\
    \end{array}
  \right.
\end{mathpar}
 

where

\begin{eqnarray}
  (x)\id{\{} \lpquote Q \rpquote / \lpquote P \rpquote \id{\}}            = 
  \left\{ 
    \begin{array}{ccc}
      \lpquote Q \rpquote & & x \nameeq \lpquote P \rpquote \\
      x & & otherwise \\
    \end{array}
  \right. \nonumber
\end{eqnarray}

and $z$ is chosen distinct from $\quotep{P}$, $\quotep{Q}$, the free
names in $Q$, and all the names in $R$. Our $\alpha$-equivalence will
be built in the standard way from this substitution.

\begin{remark}\label{rem:no_self_referential_names}
  One consequence of these definitions is that $\forall P. \quotep{P}
  \not\in \freenames{P}$.
\end{remark}

\subsection{ Dynamic quote: an example }

Anticipating something of what's to come, consider applying the
substitution, $\widehat{\id{\{}u / z \id{\}}}$, to the following pair
of processes, $\lift{w}{y!(z)}$ and $w[ \lpquote y!(z) \rpquote ]$.

\begin{eqnarray}
	\lift{w}{y!(z)}\widehat{\id{\{}u / z \id{\}}}
		& = &
		\lift{w}{y!(u)} \nonumber\\
	w[ \lpquote y!(z) \rpquote ] \widehat{ \id{\{}u / z \id{\}} }
		& = &
		w[ \lpquote y!(z) \rpquote ] \nonumber
\end{eqnarray}

Because the body of the process between quotes is impervious to
substitution, we get radically different answers. In fact, by
examining the first process in an input context,
e.g. $x?(z).\lift{w}{y!(z)}$, we see that the process under the lift
operator may be shaped by prefixed inputs binding a name inside it. In
this sense, the lift operator will be seen as a way to dynamically
construct processes before reifying them as names.

Finally equipped with these standard features we can present the
dynamics of the calculus.

\subsubsection{Operational semantics} 

Finally, we introduce the computational dynamics. What marks these
algebras as distinct from other more traditionally studied algebraic
structures, e.g. vector spaces or polynomial rings, is the manner in
which dynamics is captured. In traditional structures, dynamics is typically
expressed through morphisms between such structures, as in linear maps
between vector spaces or morphisms between rings. In algebras
associated with the semantics of computation, the dynamics is
expressed as part of the algebraic structure itself, through a
reduction reduction relation typically denoted by $\red$. Below, we
give a recursive presentation of this relation for the calculus used
in the encoding.

$\red \subseteq \pi \times \pi$
$\red : \pi \to \mathcal{P}(\pi)$

\begin{mathpar}
  \inferrule* [lab=Comm] { \textsf{match}( x_{src}, x_{trgt} ) } { x_{trgt}?(y)P \; | \; x_{src}!\langle {Q} \rangle \red P\{\quotep{Q}/y}\} }
  \and \\
  \inferrule* [lab=Par] {{P} \red {P}'} {{{P} | {Q}} \red {{P}' | {Q}}}
  \and
  \inferrule* [lab=Equiv]{{{P} \scong {P}'} \andalso {{P}' \red {Q}'} \andalso {{Q}' \scong {Q}}}{{P} \red {Q}}
\end{mathpar}

\begin{eqnarray*}
  match_{\equiv} (\quotep{P},\quotep{Q}) & := & P \equiv Q \\
  match_{\dagger}(\quotep{P},\quotep{Q}) & := & \forall R. P|Q \red^{*} R => R \red^{*} 0 \\
  match_{K}(\quotep{P},\quotep{Q}) & := & K \mbox{ for some context } K
\end{eqnarray*}

$u?(x)P | u!\langle Q \rangle \red P\{\quotep{Q}/x\}$

%We write $\wred$ for $\red^*$, and $P\red$ if $\exists Q $ such that $ P \red Q$.
We write $P\red$ if $\exists Q $ such that $ P \red Q$ and $P\not\red$, otherwise.

\section{Replication}

As mentioned before, it is known that replication (and hence
recursion) can be implemented in a higher-order process algebra
\cite{SangiorgiWalker}. As our first example of calculation with the
machinery thus far presented we give the construction explicitly in
the {\rhoc}.

\begin{eqnarray}
	D_{x} & := & \prefix{x}{y}{(\binpar{\outputp{x}{y}}{@{y}})} \nonumber\\
	\bangp_{x}{P} & := & \binpar{{x}!\langle{\binpar{D_{x}}{P}}\rangle}{D_{x}} \nonumber
\end{eqnarray}

\begin{eqnarray}
	\bangp_{x}{P} & & \nonumber\\
	=
	& {x}!\langle{(\prefix{x}{y}{(\outputp{x}{y} | @{y})) | P}}\rangle 
	      | \prefix{x}{y}{(\outputp{x}{y} | @{y})} & \nonumber\\
	\red
	& (\outputp{x}{y} | @{y})\substn{\quotep{(\prefix{x}{y}{(@{y} | \outputp{x}{y})) | P}}}{y} & \nonumber\\
	=
	& \outputp{x}{\quotep{(\prefix{x}{y}{(\outputp{x}{y} | @{y})) | P}}}
	  | {(\prefix{x}{y}{(\outputp{x}{y} | @{y})) | P}} & \nonumber\\
	\red
	& \ldots & \nonumber\\
	\red^*
	& P | P | \ldots & \nonumber
\end{eqnarray}

Of course, this encoding, as an implementation, runs away, unfolding
$\bangp{P}$ eagerly. A lazier and more implementable replication
operator, restricted to input-guarded processes, may be obtained as follows.

\begin{eqnarray}
\bangp{\prefix{u}{v}{P}} 
	:= 
	\binpar{\lift{x}{\prefix{u}{v}{(\binpar{D(x)}{P})}}}{D(x)} \nonumber
\end{eqnarray}

\begin{remark}
  Note that the lazier definition still does not deal with summation
  or mixed summation (i.e. sums over input and output). The reader is
  invited to construct definitions of replication that deal with these
  features. 

  Further, the definitions are parameterized in a name, $x$. Can you,
  gentle reader, make a definition that eliminates this parameter and
  guarantees no accidental interaction between the replication
  machinery and the process being replicated -- i.e. no accidental
  sharing of names used by the process to get its work done and the
  name(s) used by the replication to effect copying. This latter
  revision of the definition of replication is crucial to obtaining
  the expected identity $!!P \sim !P$.
\end{remark}

\begin{remark}\label{rem:paradoxical_combinator}
  The reader familiar with the lambda calculus will have noticed the
  similarity between $D$ and the paradoxical combinator.

  [Ed. note: the existence of this seems to suggest we have to be more
  restrictive on the set of processes and names we admit if we are to
  support no-cloning.]
\end{remark}

\subsubsection{Bisimulation}

The computational dynamics gives rise to another kind of equivalence,
the equivalence of computational behavior. As previously mentioned
this is typically captured \emph{via} some form of bisimulation.

% The notion we use in this paper is weak barbed bisimulation
% \cite{milner91polyadicpi}.

The notion we use in this paper is derived from weak barbed
bisimulation \cite{milner91polyadicpi}. 

\begin{definition}
An \emph{observation relation}, $\downarrow_{\mathcal N}$, over a set
of names, $\mathcal N$, is the smallest relation satisfying the rules
below.

\infrule[Out-barb]{y \in {\mathcal N}, \; x \nameeq y}
		  {\outputp{x}{v} \downarrow_{\mathcal N} x}
\infrule[Par-barb]{\mbox{$P\downarrow_{\mathcal N} x$ or $Q\downarrow_{\mathcal N} x$}}
		  {\binpar{P}{Q} \downarrow_{\mathcal N} x}

We write $P \Downarrow_{\mathcal N} x$ if there is $Q$ such that 
$P \wred Q$ and $Q \downarrow_{\mathcal N} x$.
\end{definition}

\begin{definition}
%\label{def.bbisim}
An  ${\mathcal N}$-\emph{barbed bisimulation} over a set of names, ${\mathcal N}$, is a symmetric binary relation 
${\mathcal S}_{\mathcal N}$ between agents such that $P\rel{S}_{\mathcal N}Q$ implies:
\begin{enumerate}
\item If $P \red P'$ then $Q \wred Q'$ and $P'\rel{S}_{\mathcal N} Q'$.
\item If $P\downarrow_{\mathcal N} x$, then $Q\Downarrow_{\mathcal N} x$.
\end{enumerate}
$P$ is ${\mathcal N}$-barbed bisimilar to $Q$, written
$P \wbbisim_{\mathcal N} Q$, if $P \rel{S}_{\mathcal N} Q$ for some ${\mathcal N}$-barbed bisimulation ${\mathcal S}_{\mathcal N}$.
\end{definition}

$\mathcal{R} \subseteq \pi \times \pi$

$P \mathcal{R} Q => \forall P'. P \red P' \Rightarrow \exists Q'. Q \red Q', P' \mathcal{R} Q'$

$P \vdash x \Rightarrow Q \vdash x$

\begin{mathpar}
  \inferrule*[lab=Out-barb]{x \nameeq y}{{y}!\langle{Q}\rangle \vdash x}
  \and
  \inferrule*[lab=Par-barb]{\mbox{$P\vdash x$ or $Q\vdash x$}}{\binpar{P}{Q} \vdash x}
\end{mathpar}

\subsubsection{Contexts}

One of the principle advantages of computational calculi like the
$\pi$-calculus is a well-defined notion of context,
contextual-equivalence and a correlation between
contextual-equivalence and notions of bisimulation. The notion of
context allows the decomposition of a process into (sub-)process and
its syntactic environment, its context. Thus, a context may be
thought of as a process with a ``hole'' (written $\Box$) in it. The
application of a context $M$ to a process $P$, written $M[P]$, is
tantamount to filling the hole in $M$ with $P$. In this paper we do
not need the full weight of this theory, but do make use of the notion
of context in the proof the main theorem. 

\begin{mathpar}
  \inferrule* [lab=summation] {} {{M_{M},M_{N}} \bc \Box \;|\; x.M_{A} \;|\; M_{M}+M_{N}}
  \and
  \inferrule* [lab=agent] {} {{M_{A}} \bc (\vec{x})M_{P} \;| \; \clift{P_0,\ldots,M_{P},\ldots,P_N}}
  \and \\
  \inferrule* [lab=process] {} {{M_{P}} \bc M_{N} \;| \;P|M_{P} }
\end{mathpar} 

\begin{mathpar}
  \inferrule* [lab=sychronization] {} {M_{N} \bc \Box \;|\; x?M_{F} \;|\; x!M_{C}}
  \and
  \inferrule* [lab=abstraction] {} {{M_{F}} \bc (x)M_{P} }
  \and
  \inferrule* [lab=concretion] {} {{M_{C}} \bc \langle M_{P} \rangle }
  \and \\
  \inferrule* [lab=process] {} {{M_{P}} \bc M_{N} \;| \;P|M_{P} }
\end{mathpar}

\begin{definition}[contextual application] Given a context $M$, and
  process $P$, we define the \emph{contextual application}, $M[P] :=
  M\{P/\Box\}$. That is, the contextual application of M to P is the
  substitution of $P$ for $\Box$ in $M$.
\end{definition}

$\meaningof{-} : L \to \mathcal{P}(\pi)$

\begin{mathpar}
  \inferrule* [lab=collection] {} {\meaningof{true} = \pi, \and \meaningof{~E} = \pi \setminus \meaningof{E}, \and \meaningof{E_{1} \& E_{2}} = \meaningof{E_{1}} \cap \meaningof{E_{2}}}
\end{mathpar}

\begin{mathpar}
  \inferrule* [lab=structure] {} {\meaningof{0} = \{ P \in \pi | P \equiv 0 \}, \and \\ \meaningof{E_1 | E_2} = \{ P \in \pi | P \equiv P_{1} | P_{2}, P_{1} \in \meaningof{E_{1}}, P_{2} \in \meaningof{E_2}\} }
\end{mathpar}

\begin{mathpar}
 \inferrule* [lab=behavior] {} {\meaningof{\langle a?b \rangle E} = \{ P \in \pi | P \equiv Q | u?(y)P', \\ \and \\\\ \and \\ \;\;\; u \in \meaningof{a}, \forall z.P'\{z/y\} \in \meaningof{E\{z/b\}}\}, \and \\ \meaningof{a!E} = \{ P \in \pi | P \equiv Q | x!\langle P' \rangle, x \in \meaningof{a} P' \in \meaningof{E}\} }
\end{mathpar}

\begin{mathpar}
 \inferrule* [lab=nominal] {} {\meaningof{\quotep{E}} = \{ \quotep{P} \in \quotep{\pi} | P \in \meaningof{E} \}, \and \meaningof{\quotep{P}} = \{ \quotep{Q} \in \quotep{\pi} | P \equiv Q \} \and \\ \meaningof{@\quotep{E}} = \{ P \in \pi | P \equiv @x, x \in \meaningof{E} \}}
\end{mathpar}

\begin{eqnarray*}
  \\
  \meaningof{-} : TS \to ST
\end{eqnarray*}

\begin{eqnarray*}
  \\
  L : TS \to ST
\end{eqnarray*}

\begin{eqnarray*}
  \\
  P \models E \iff P \in \meaningof{E}
\end{eqnarray*}

\begin{eqnarray*}
  P \approx_{L} Q \iff \forall E \in L. P \models E \iff Q \models E
\end{eqnarray*}

\begin{eqnarray*}
  P \approx_{K} Q
\end{eqnarray*}

\begin{eqnarray*}
  P \approx Q
\end{eqnarray*}

$\approx_{K} = \approx = \approx_{L}$

\subsubsection{Contextual duality}

Note that contexts extend the quotation operation to a family of
operations from processes to names. Given a context, $M$, we can
define a \emph{nominal context}, $\quotep{M}$ by $\quotep{M}[P] :=
\quotep{M[P]}$. To foreshadow what is to come we observe that these
operations enjoy a duality with processes very much like the duality
between vectors and maps from vectors to scalars.

Further, because the calculus is essentially higher-order, we have a
correspondence between contexts and processes. More specifically,
given a name $x$ and a context $M$ we can construct $M^{*}_{x}$ such
that 

\begin{mathpar}
  M^{*}_{x} | \lift{x}{P} \red M[P]
\end{mathpar}

namely,

\begin{mathpar}
  M^{*}_{x} := x?(u).M[\dropn{u}]
\end{mathpar}

The dependence of $M^{*}_{x}$ on a name makes it an abstraction, 

\begin{mathpar}
  M^{*} := (x)x?(u).M[\dropn{u}]
\end{mathpar}

\subsection{Additional notation}

It will sometimes be convenient to denote the process a name
quotes. We already have the notation $x = \quotep{P}$, but it will be
convenient to introduce an alternate notation, $\procn{x}$, when we
want to emphasize the connection to the use of the name. Note that, by
virtue of name equivalence, $\quotep{\procn{x}} \nameeq x$; so, the
notation is consistent with previous definitions.

Further, because names have structure it is possible to effect
substitutions on the basis of that structure. This means we need to
upgrade our notation for substitutions, which we accomplish by
adapting comprehension notation. Thus,

\begin{mathpar}
  P\{ y / x : x \in S \}
\end{mathpar}

is interpreted to mean the process derived from P by replacing (in a
capture-avoiding manner) each occurrence of $x$ in $S$ by $y$. For example,

\begin{mathpar}
  P\{ \quotep{\procn{x}|\procn{x}} / x : x \in \freenames{P} \}
\end{mathpar}

will replace each (occurrence) of a free name $x$ in $P$ by
$\quotep{\procn{x}|\procn{x}}$.

Also, we will avail ourselves of the notation $x^{L}$ and $x^{R}$ to
denote injections of a name into disjoint copies of the name
space. There are numerous ways to accomplish this. One example can be
found in \cite{MeredithR05}. This notation overloads to vectors of
names: $\vec{x}^{\pi} := (x_{i}^{\pi} \; : \; 0 \leq i < |\vec{x}| )$ where $\pi \in \{L,R\}$.

We also use $P^{\Box} := P|\Box$.

In \cite{MeredithR05} an interpretation of the new operator is
given. It turns out that there are several possible interpretations
all enjoying the requisite algebraic properties of the operator (see
\cite{milner91polyadicpi}). We will therefore make liberal use of
$(\nu\; \vec{x})P$.

% subsection the_syntax_and_semantics_of_the_notation_system (end)   

\input{qm2pi.qmops} 

\input{qm2pi.sterngerlach} 

\input{qm2pi.metric} 

% section concurrent_process_calculi (end)

%\input{qm2pi.proofsketch}

% section proof sketch (end)

%\input{qm2pi.slviaknots} 

% section spatial logic via knots (end)

\input{qm2pi.conclusion}

% section conclusion (end)

%\input{qm2pi.dtcodes} 

% section wiring algorithm (end)

\input{qm2pi.ack} 

% section acknowledgments (end)

\newpage


\bibliographystyle{plain}   
\bibliography{../../biblios/main.bib}

\input{qm2pi.rhodetails}

\end{document}

 

%\documentclass[12pt]{llncs}
%\documentclass{jktr}

\usepackage[pdftex]{hyperref}                   
\usepackage {listings}
\usepackage {mathpartir}
\usepackage{bcprules}
%\usepackage{listings}
                       
\usepackage{graphicx} 
%\usepackage[margins=2.5cm,nohead,nofoot]{geometry}
%\usepackage{geometry}
\usepackage{amsfonts}
\usepackage{amstext}
\usepackage{latexsym}
\usepackage{amssymb}
\usepackage{color}


%\include{myPreamble}
\include{qm2pi.local} 

%\ifpdf
%\usepackage[pdftex]{graphicx}
%\else
%\usepackage{graphicx}
%\fi

 % \ifpdf
%  \usepackage{pdfsync}
%  \if


%\title{Brief Article}
%\author{David F. Snyder}
%\author{L.G. Meredith}

%\address{Dept. of Math., Texas State University--San Marcos, San Marcos, TX 78666}
       
\pagestyle{empty}


\begin{document}

\lstset{language=[Objective]Caml,frame=shadowbox}

\input{qm2pi.front}

% section front matter (end)

\input{qm2pi.intro} 
 
% section introduction (end)

% \input{qm2pi.knotations} 

% section notation (end)

\input{qm2pi.process.calculi} 

% section concurrent_process_calculi_and_spatial_logics_ (end)
    
%\input{qm2pi.knots2pi} 

%\input{qm2pi.trefoil} 

%\input{qm2pi.mainthm} 

% subsection basic_interpretation (end)

%\input{qm2pi.rho.presentation} 
\subsection{The syntax and semantics of the notation system}\label{sub:the_syntax_and_semantics_of_the_notation_system} % (fold)

We now summarize a technical presentation of the calculus that
embodies our theory of dynamics. The typical presentation of such a
calculus follows the style of giving generators and relations on
them. The grammar, below, describing term constructors, freely
generates the set of processes, $\Proc$. This set is then quotiented
by a relation known as structural congruence and it is over this set
that the notion of dynamics is expressed. This presentation is
essentially that of \cite{MeredithR05} with the addition of
polyadicity and summation. For readability we have relegated some of
the technical subtleties to an appendix.

\subsubsection{Process grammar}\label{subsub:process_grammar}

\begin{mathpar}
  \inferrule* [lab=synchronization] {} {{M} \bc \pzero \;|\; x?F \;|\; x!C }
  \and
  \inferrule* [lab=abstraction] {} {{F} \bc (x)P}
  \and
  \inferrule* [lab=concretion] {} {{C} \bc \langle Q \rangle}
  \and
  \inferrule* [lab=process] {} {{P,Q} \bc M \;| \;P|Q \;|\; @{x}}
  \and
  \inferrule* [lab=name] {} {{x} \bc \quotep{P}}
\end{mathpar} 

Note that $\vec{x}$ (resp. $\vec{P}$) denotes a vector of names
(resp. processes) of length $|\vec{x}|$ (resp. $|\vec{P}|$). We adopt
the following useful abbreviations.

\begin{mathpar}
   x?(\vec{y}).P := x.(\vec{y})P \and  x\clift{\vec{P}} := x.\clift{\vec{P}}
   \and x!(y) := \lift{x}{\dropn{y}}
   \and \Pi_{i=0}^{n-1}P_i := P_0 | \ldots | P_{n-1}
\end{mathpar}

\subsubsection{Structural congruence}

\paragraph{Free and bound names and alpha-equivalence.} At the
core of structural equivalence is alpha-equivalence which identifies
process that are the same up to a change of variable. Formally, we
recognize the distinction between free and bound names. The free names
of a process, $\freenames{P}$, may be calculated recursively as
follows:

\begin{mathpar}
\freenames{\pzero} := \emptyset
  \and \\
  \freenames{x?(y).P} := \{ x \} \cup (\freenames{P} \setminus \{ y \})
  \and 
  \freenames{x!\langle P \rangle} := \{ x \} \cup \{ P \} 
  \and \\
  \freenames{P|Q} := \freenames{P} \cup \freenames{Q}
  \and \\
  \freenames{@{x}} := \{ x \}
\end{mathpar}

$\pi$
$\quotep{\pi}$

$\freenames{-} : \pi \to \mathcal{P}(\quotep{\pi})$

\begin{eqnarray*}
  \freenames{\pzero} & := & \emptyset \\
  \freenames{x?(y).P} & := & \{ x \} \cup (\freenames{P} \setminus \{ y \}) \\
  \freenames{x!\langle P \rangle} & := & \{ x \} \cup \{ P \} \\
  \freenames{P|Q} & := & \freenames{P} \cup \freenames{Q} \\
  \freenames{\dropn{x}} & := & \{ x \}
\end{eqnarray*}

The bound names of a process, $\boundnames{P}$, are those names occurring in $P$
that are not free. For example, in $x?(y).0$, the name $x$ is free, while $y$ is bound.

\begin{mathpar}
  \inferrule* [lab=monoidal-laws] {} { P|Q \equiv Q|P \and P|0 \equiv P \and P|(Q|R) \equiv (P|Q)|R }
\end{mathpar}

\begin{mathpar}
  \inferrule* [lab=alpha-equivalence] {} { (x)P \equiv (y)P\{y/x\} \and y \not\in \freenames{P} }
\end{mathpar}

\begin{definition}
Then two processes, $P,Q$, are alpha-equivalent if $P = Q\{\vec{y}/\vec{x}\}$ for
some $\vec{x} \in \boundnames{Q},\vec{y} \in \boundnames{P}$, where $Q\{\vec{y}/\vec{x}\}$
denotes the capture-avoiding substitution of $\vec{y}$ for $\vec{x}$ in $Q$.
\end{definition}

\begin{definition}
  The {\em structural congruence} \cite{SangiorgiWalker} , $\equiv$,
  between processes is the least congruence containing
  alpha-equivalence, satisfying the abelian monoid laws
  (associativity, commutativity and $\pzero$ as identity) for parallel
  composition $|$ and for summation $+$.
\end{definition}

\subsection{Name equivalence}

We take name equivalence, written $\nameeq$, to be the smallest
equivalence relation generated by the following rules.

\begin{mathpar}
\inferrule*[lab=Quote-drop]
{ }
{ \quotep{@{x}} \nameeq x }

\inferrule*[lab=Struct-equiv]
{ P \scong Q }
{ \quotep{P} \nameeq \quotep{Q} }
\end{mathpar}

The astute reader will have noticed that the mutual recursion of names
and processes imposes a mutual recursion on alpha-equivalence and
structural equivalence via name-equivalence. Fortunately, all of this
works out pleasantly and we may calculate in the natural way, free of
concern. The reader interested in the details is referred to the
appendix \ref{appendix:rho_details}.

\subsection{Substitution}

We use $\Proc$ for the set of processes, $\QProc$ for the set of
names, and $\id{\{}\vec{y} / \vec{x} \id{\}}$ to denote partial maps,
$s : \QProc \rightarrow \QProc$. A map, $s$ lifts, uniquely, to a map
on process terms, $\widehat{s} : \Proc \rightarrow \Proc$ by the
following equations.

\begin{mathpar}
  (0) \psubstp{Q}{P} := 0 \\
  (R \juxtap S) \psubstp{Q}{P}
  :=    
  (R)\psubstp{Q}{P} \juxtap (S) \psubstp{Q}{P} \\
  (x?(y).R) \psubstp{Q}{P}    
  :=    
  (x)\substp{Q}{P} (z)\concat( (R \psubstn{z}{y}) \psubstp{Q}{P} ) \\
  (\lift{x}{R}) \psubstp{Q}{P}  
  :=
  \lift{(x)\substp{Q}{P}}{ R \psubstp{Q}{P} } \\
%   (\dropn{x})  \psubstp{Q}{P}       
%   := 
%   \left\{ 
%     \begin{array}{ccc} 
%       \dropn{\quotep{Q}} & & x \nameeq \quotep{P} \\
%       \dropn{x} & & otherwise \\
%     \end{array}
%   \right. 
  (\dropn{x})  \psubstp{Q}{P}       
  := 
  \left\{ 
    \begin{array}{ccc} 
      Q & & x \nameeq \quotep{P} \\
      \dropn{x} & & otherwise \\
    \end{array}
  \right.
\end{mathpar}
 

where

\begin{eqnarray}
  (x)\id{\{} \lpquote Q \rpquote / \lpquote P \rpquote \id{\}}            = 
  \left\{ 
    \begin{array}{ccc}
      \lpquote Q \rpquote & & x \nameeq \lpquote P \rpquote \\
      x & & otherwise \\
    \end{array}
  \right. \nonumber
\end{eqnarray}

and $z$ is chosen distinct from $\quotep{P}$, $\quotep{Q}$, the free
names in $Q$, and all the names in $R$. Our $\alpha$-equivalence will
be built in the standard way from this substitution.

\begin{remark}\label{rem:no_self_referential_names}
  One consequence of these definitions is that $\forall P. \quotep{P}
  \not\in \freenames{P}$.
\end{remark}

\subsection{ Dynamic quote: an example }

Anticipating something of what's to come, consider applying the
substitution, $\widehat{\id{\{}u / z \id{\}}}$, to the following pair
of processes, $\lift{w}{y!(z)}$ and $w[ \lpquote y!(z) \rpquote ]$.

\begin{eqnarray}
	\lift{w}{y!(z)}\widehat{\id{\{}u / z \id{\}}}
		& = &
		\lift{w}{y!(u)} \nonumber\\
	w[ \lpquote y!(z) \rpquote ] \widehat{ \id{\{}u / z \id{\}} }
		& = &
		w[ \lpquote y!(z) \rpquote ] \nonumber
\end{eqnarray}

Because the body of the process between quotes is impervious to
substitution, we get radically different answers. In fact, by
examining the first process in an input context,
e.g. $x?(z).\lift{w}{y!(z)}$, we see that the process under the lift
operator may be shaped by prefixed inputs binding a name inside it. In
this sense, the lift operator will be seen as a way to dynamically
construct processes before reifying them as names.

Finally equipped with these standard features we can present the
dynamics of the calculus.

\subsubsection{Operational semantics} 

Finally, we introduce the computational dynamics. What marks these
algebras as distinct from other more traditionally studied algebraic
structures, e.g. vector spaces or polynomial rings, is the manner in
which dynamics is captured. In traditional structures, dynamics is typically
expressed through morphisms between such structures, as in linear maps
between vector spaces or morphisms between rings. In algebras
associated with the semantics of computation, the dynamics is
expressed as part of the algebraic structure itself, through a
reduction reduction relation typically denoted by $\red$. Below, we
give a recursive presentation of this relation for the calculus used
in the encoding.

$\red \subseteq \pi \times \pi$
$\red : \pi \to \mathcal{P}(\pi)$

\begin{mathpar}
  \inferrule* [lab=Comm] { \textsf{match}( x_{src}, x_{trgt} ) } { x_{trgt}?(y)P \; | \; x_{src}!\langle {Q} \rangle \red P\{\quotep{Q}/y}\} }
  \and \\
  \inferrule* [lab=Par] {{P} \red {P}'} {{{P} | {Q}} \red {{P}' | {Q}}}
  \and
  \inferrule* [lab=Equiv]{{{P} \scong {P}'} \andalso {{P}' \red {Q}'} \andalso {{Q}' \scong {Q}}}{{P} \red {Q}}
\end{mathpar}

\begin{eqnarray*}
  match_{\equiv} (\quotep{P},\quotep{Q}) & := & P \equiv Q \\
  match_{\dagger}(\quotep{P},\quotep{Q}) & := & \forall R. P|Q \red^{*} R => R \red^{*} 0 \\
  match_{K}(\quotep{P},\quotep{Q}) & := & K \mbox{ for some context } K
\end{eqnarray*}

$u?(x)P | u!\langle Q \rangle \red P\{\quotep{Q}/x\}$

%We write $\wred$ for $\red^*$, and $P\red$ if $\exists Q $ such that $ P \red Q$.
We write $P\red$ if $\exists Q $ such that $ P \red Q$ and $P\not\red$, otherwise.

\section{Replication}

As mentioned before, it is known that replication (and hence
recursion) can be implemented in a higher-order process algebra
\cite{SangiorgiWalker}. As our first example of calculation with the
machinery thus far presented we give the construction explicitly in
the {\rhoc}.

\begin{eqnarray}
	D_{x} & := & \prefix{x}{y}{(\binpar{\outputp{x}{y}}{@{y}})} \nonumber\\
	\bangp_{x}{P} & := & \binpar{{x}!\langle{\binpar{D_{x}}{P}}\rangle}{D_{x}} \nonumber
\end{eqnarray}

\begin{eqnarray}
	\bangp_{x}{P} & & \nonumber\\
	=
	& {x}!\langle{(\prefix{x}{y}{(\outputp{x}{y} | @{y})) | P}}\rangle 
	      | \prefix{x}{y}{(\outputp{x}{y} | @{y})} & \nonumber\\
	\red
	& (\outputp{x}{y} | @{y})\substn{\quotep{(\prefix{x}{y}{(@{y} | \outputp{x}{y})) | P}}}{y} & \nonumber\\
	=
	& \outputp{x}{\quotep{(\prefix{x}{y}{(\outputp{x}{y} | @{y})) | P}}}
	  | {(\prefix{x}{y}{(\outputp{x}{y} | @{y})) | P}} & \nonumber\\
	\red
	& \ldots & \nonumber\\
	\red^*
	& P | P | \ldots & \nonumber
\end{eqnarray}

Of course, this encoding, as an implementation, runs away, unfolding
$\bangp{P}$ eagerly. A lazier and more implementable replication
operator, restricted to input-guarded processes, may be obtained as follows.

\begin{eqnarray}
\bangp{\prefix{u}{v}{P}} 
	:= 
	\binpar{\lift{x}{\prefix{u}{v}{(\binpar{D(x)}{P})}}}{D(x)} \nonumber
\end{eqnarray}

\begin{remark}
  Note that the lazier definition still does not deal with summation
  or mixed summation (i.e. sums over input and output). The reader is
  invited to construct definitions of replication that deal with these
  features. 

  Further, the definitions are parameterized in a name, $x$. Can you,
  gentle reader, make a definition that eliminates this parameter and
  guarantees no accidental interaction between the replication
  machinery and the process being replicated -- i.e. no accidental
  sharing of names used by the process to get its work done and the
  name(s) used by the replication to effect copying. This latter
  revision of the definition of replication is crucial to obtaining
  the expected identity $!!P \sim !P$.
\end{remark}

\begin{remark}\label{rem:paradoxical_combinator}
  The reader familiar with the lambda calculus will have noticed the
  similarity between $D$ and the paradoxical combinator.

  [Ed. note: the existence of this seems to suggest we have to be more
  restrictive on the set of processes and names we admit if we are to
  support no-cloning.]
\end{remark}

\subsubsection{Bisimulation}

The computational dynamics gives rise to another kind of equivalence,
the equivalence of computational behavior. As previously mentioned
this is typically captured \emph{via} some form of bisimulation.

% The notion we use in this paper is weak barbed bisimulation
% \cite{milner91polyadicpi}.

The notion we use in this paper is derived from weak barbed
bisimulation \cite{milner91polyadicpi}. 

\begin{definition}
An \emph{observation relation}, $\downarrow_{\mathcal N}$, over a set
of names, $\mathcal N$, is the smallest relation satisfying the rules
below.

\infrule[Out-barb]{y \in {\mathcal N}, \; x \nameeq y}
		  {\outputp{x}{v} \downarrow_{\mathcal N} x}
\infrule[Par-barb]{\mbox{$P\downarrow_{\mathcal N} x$ or $Q\downarrow_{\mathcal N} x$}}
		  {\binpar{P}{Q} \downarrow_{\mathcal N} x}

We write $P \Downarrow_{\mathcal N} x$ if there is $Q$ such that 
$P \wred Q$ and $Q \downarrow_{\mathcal N} x$.
\end{definition}

\begin{definition}
%\label{def.bbisim}
An  ${\mathcal N}$-\emph{barbed bisimulation} over a set of names, ${\mathcal N}$, is a symmetric binary relation 
${\mathcal S}_{\mathcal N}$ between agents such that $P\rel{S}_{\mathcal N}Q$ implies:
\begin{enumerate}
\item If $P \red P'$ then $Q \wred Q'$ and $P'\rel{S}_{\mathcal N} Q'$.
\item If $P\downarrow_{\mathcal N} x$, then $Q\Downarrow_{\mathcal N} x$.
\end{enumerate}
$P$ is ${\mathcal N}$-barbed bisimilar to $Q$, written
$P \wbbisim_{\mathcal N} Q$, if $P \rel{S}_{\mathcal N} Q$ for some ${\mathcal N}$-barbed bisimulation ${\mathcal S}_{\mathcal N}$.
\end{definition}

$\mathcal{R} \subseteq \pi \times \pi$

$P \mathcal{R} Q => \forall P'. P \red P' \Rightarrow \exists Q'. Q \red Q', P' \mathcal{R} Q'$

$P \vdash x \Rightarrow Q \vdash x$

\begin{mathpar}
  \inferrule*[lab=Out-barb]{x \nameeq y}{{y}!\langle{Q}\rangle \vdash x}
  \and
  \inferrule*[lab=Par-barb]{\mbox{$P\vdash x$ or $Q\vdash x$}}{\binpar{P}{Q} \vdash x}
\end{mathpar}

\subsubsection{Contexts}

One of the principle advantages of computational calculi like the
$\pi$-calculus is a well-defined notion of context,
contextual-equivalence and a correlation between
contextual-equivalence and notions of bisimulation. The notion of
context allows the decomposition of a process into (sub-)process and
its syntactic environment, its context. Thus, a context may be
thought of as a process with a ``hole'' (written $\Box$) in it. The
application of a context $M$ to a process $P$, written $M[P]$, is
tantamount to filling the hole in $M$ with $P$. In this paper we do
not need the full weight of this theory, but do make use of the notion
of context in the proof the main theorem. 

\begin{mathpar}
  \inferrule* [lab=summation] {} {{M_{M},M_{N}} \bc \Box \;|\; x.M_{A} \;|\; M_{M}+M_{N}}
  \and
  \inferrule* [lab=agent] {} {{M_{A}} \bc (\vec{x})M_{P} \;| \; \clift{P_0,\ldots,M_{P},\ldots,P_N}}
  \and \\
  \inferrule* [lab=process] {} {{M_{P}} \bc M_{N} \;| \;P|M_{P} }
\end{mathpar} 

\begin{mathpar}
  \inferrule* [lab=sychronization] {} {M_{N} \bc \Box \;|\; x?M_{F} \;|\; x!M_{C}}
  \and
  \inferrule* [lab=abstraction] {} {{M_{F}} \bc (x)M_{P} }
  \and
  \inferrule* [lab=concretion] {} {{M_{C}} \bc \langle M_{P} \rangle }
  \and \\
  \inferrule* [lab=process] {} {{M_{P}} \bc M_{N} \;| \;P|M_{P} }
\end{mathpar}

\begin{definition}[contextual application] Given a context $M$, and
  process $P$, we define the \emph{contextual application}, $M[P] :=
  M\{P/\Box\}$. That is, the contextual application of M to P is the
  substitution of $P$ for $\Box$ in $M$.
\end{definition}

$\meaningof{-} : L \to \mathcal{P}(\pi)$

\begin{mathpar}
  \inferrule* [lab=collection] {} {\meaningof{true} = \pi, \and \meaningof{~E} = \pi \setminus \meaningof{E}, \and \meaningof{E_{1} \& E_{2}} = \meaningof{E_{1}} \cap \meaningof{E_{2}}}
\end{mathpar}

\begin{mathpar}
  \inferrule* [lab=structure] {} {\meaningof{0} = \{ P \in \pi | P \equiv 0 \}, \and \\ \meaningof{E_1 | E_2} = \{ P \in \pi | P \equiv P_{1} | P_{2}, P_{1} \in \meaningof{E_{1}}, P_{2} \in \meaningof{E_2}\} }
\end{mathpar}

\begin{mathpar}
 \inferrule* [lab=behavior] {} {\meaningof{\langle a?b \rangle E} = \{ P \in \pi | P \equiv Q | u?(y)P', \\ \and \\\\ \and \\ \;\;\; u \in \meaningof{a}, \forall z.P'\{z/y\} \in \meaningof{E\{z/b\}}\}, \and \\ \meaningof{a!E} = \{ P \in \pi | P \equiv Q | x!\langle P' \rangle, x \in \meaningof{a} P' \in \meaningof{E}\} }
\end{mathpar}

\begin{mathpar}
 \inferrule* [lab=nominal] {} {\meaningof{\quotep{E}} = \{ \quotep{P} \in \quotep{\pi} | P \in \meaningof{E} \}, \and \meaningof{\quotep{P}} = \{ \quotep{Q} \in \quotep{\pi} | P \equiv Q \} \and \\ \meaningof{@\quotep{E}} = \{ P \in \pi | P \equiv @x, x \in \meaningof{E} \}}
\end{mathpar}

\begin{eqnarray*}
  \\
  \meaningof{-} : TS \to ST
\end{eqnarray*}

\begin{eqnarray*}
  \\
  L : TS \to ST
\end{eqnarray*}

\begin{eqnarray*}
  \\
  P \models E \iff P \in \meaningof{E}
\end{eqnarray*}

\begin{eqnarray*}
  P \approx_{L} Q \iff \forall E \in L. P \models E \iff Q \models E
\end{eqnarray*}

\begin{eqnarray*}
  P \approx_{K} Q
\end{eqnarray*}

\begin{eqnarray*}
  P \approx Q
\end{eqnarray*}

$\approx_{K} = \approx = \approx_{L}$

\subsubsection{Contextual duality}

Note that contexts extend the quotation operation to a family of
operations from processes to names. Given a context, $M$, we can
define a \emph{nominal context}, $\quotep{M}$ by $\quotep{M}[P] :=
\quotep{M[P]}$. To foreshadow what is to come we observe that these
operations enjoy a duality with processes very much like the duality
between vectors and maps from vectors to scalars.

Further, because the calculus is essentially higher-order, we have a
correspondence between contexts and processes. More specifically,
given a name $x$ and a context $M$ we can construct $M^{*}_{x}$ such
that 

\begin{mathpar}
  M^{*}_{x} | \lift{x}{P} \red M[P]
\end{mathpar}

namely,

\begin{mathpar}
  M^{*}_{x} := x?(u).M[\dropn{u}]
\end{mathpar}

The dependence of $M^{*}_{x}$ on a name makes it an abstraction, 

\begin{mathpar}
  M^{*} := (x)x?(u).M[\dropn{u}]
\end{mathpar}

\subsection{Additional notation}

It will sometimes be convenient to denote the process a name
quotes. We already have the notation $x = \quotep{P}$, but it will be
convenient to introduce an alternate notation, $\procn{x}$, when we
want to emphasize the connection to the use of the name. Note that, by
virtue of name equivalence, $\quotep{\procn{x}} \nameeq x$; so, the
notation is consistent with previous definitions.

Further, because names have structure it is possible to effect
substitutions on the basis of that structure. This means we need to
upgrade our notation for substitutions, which we accomplish by
adapting comprehension notation. Thus,

\begin{mathpar}
  P\{ y / x : x \in S \}
\end{mathpar}

is interpreted to mean the process derived from P by replacing (in a
capture-avoiding manner) each occurrence of $x$ in $S$ by $y$. For example,

\begin{mathpar}
  P\{ \quotep{\procn{x}|\procn{x}} / x : x \in \freenames{P} \}
\end{mathpar}

will replace each (occurrence) of a free name $x$ in $P$ by
$\quotep{\procn{x}|\procn{x}}$.

Also, we will avail ourselves of the notation $x^{L}$ and $x^{R}$ to
denote injections of a name into disjoint copies of the name
space. There are numerous ways to accomplish this. One example can be
found in \cite{MeredithR05}. This notation overloads to vectors of
names: $\vec{x}^{\pi} := (x_{i}^{\pi} \; : \; 0 \leq i < |\vec{x}| )$ where $\pi \in \{L,R\}$.

We also use $P^{\Box} := P|\Box$.

In \cite{MeredithR05} an interpretation of the new operator is
given. It turns out that there are several possible interpretations
all enjoying the requisite algebraic properties of the operator (see
\cite{milner91polyadicpi}). We will therefore make liberal use of
$(\nu\; \vec{x})P$.

% subsection the_syntax_and_semantics_of_the_notation_system (end)   

\input{qm2pi.qmops} 

\input{qm2pi.sterngerlach} 

\input{qm2pi.metric} 

% section concurrent_process_calculi (end)

%\input{qm2pi.proofsketch}

% section proof sketch (end)

%\input{qm2pi.slviaknots} 

% section spatial logic via knots (end)

\input{qm2pi.conclusion}

% section conclusion (end)

%\input{qm2pi.dtcodes} 

% section wiring algorithm (end)

\input{qm2pi.ack} 

% section acknowledgments (end)

\newpage


\bibliographystyle{plain}   
\bibliography{../../biblios/main.bib}

\input{qm2pi.rhodetails}

\end{document}

 

% subsection basic_interpretation (end)

%\input{qm2pi.rho.presentation} 
\subsection{The syntax and semantics of the notation system}\label{sub:the_syntax_and_semantics_of_the_notation_system} % (fold)

We now summarize a technical presentation of the calculus that
embodies our theory of dynamics. The typical presentation of such a
calculus follows the style of giving generators and relations on
them. The grammar, below, describing term constructors, freely
generates the set of processes, $\Proc$. This set is then quotiented
by a relation known as structural congruence and it is over this set
that the notion of dynamics is expressed. This presentation is
essentially that of \cite{MeredithR05} with the addition of
polyadicity and summation. For readability we have relegated some of
the technical subtleties to an appendix.

\subsubsection{Process grammar}\label{subsub:process_grammar}

\begin{mathpar}
  \inferrule* [lab=synchronization] {} {{M} \bc \pzero \;|\; x?F \;|\; x!C }
  \and
  \inferrule* [lab=abstraction] {} {{F} \bc (x)P}
  \and
  \inferrule* [lab=concretion] {} {{C} \bc \langle Q \rangle}
  \and
  \inferrule* [lab=process] {} {{P,Q} \bc M \;| \;P|Q \;|\; @{x}}
  \and
  \inferrule* [lab=name] {} {{x} \bc \quotep{P}}
\end{mathpar} 

Note that $\vec{x}$ (resp. $\vec{P}$) denotes a vector of names
(resp. processes) of length $|\vec{x}|$ (resp. $|\vec{P}|$). We adopt
the following useful abbreviations.

\begin{mathpar}
   x?(\vec{y}).P := x.(\vec{y})P \and  x\clift{\vec{P}} := x.\clift{\vec{P}}
   \and x!(y) := \lift{x}{\dropn{y}}
   \and \Pi_{i=0}^{n-1}P_i := P_0 | \ldots | P_{n-1}
\end{mathpar}

\subsubsection{Structural congruence}

\paragraph{Free and bound names and alpha-equivalence.} At the
core of structural equivalence is alpha-equivalence which identifies
process that are the same up to a change of variable. Formally, we
recognize the distinction between free and bound names. The free names
of a process, $\freenames{P}$, may be calculated recursively as
follows:

\begin{mathpar}
\freenames{\pzero} := \emptyset
  \and \\
  \freenames{x?(y).P} := \{ x \} \cup (\freenames{P} \setminus \{ y \})
  \and 
  \freenames{x!\langle P \rangle} := \{ x \} \cup \{ P \} 
  \and \\
  \freenames{P|Q} := \freenames{P} \cup \freenames{Q}
  \and \\
  \freenames{@{x}} := \{ x \}
\end{mathpar}

$\pi$
$\quotep{\pi}$

$\freenames{-} : \pi \to \mathcal{P}(\quotep{\pi})$

\begin{eqnarray*}
  \freenames{\pzero} & := & \emptyset \\
  \freenames{x?(y).P} & := & \{ x \} \cup (\freenames{P} \setminus \{ y \}) \\
  \freenames{x!\langle P \rangle} & := & \{ x \} \cup \{ P \} \\
  \freenames{P|Q} & := & \freenames{P} \cup \freenames{Q} \\
  \freenames{\dropn{x}} & := & \{ x \}
\end{eqnarray*}

The bound names of a process, $\boundnames{P}$, are those names occurring in $P$
that are not free. For example, in $x?(y).0$, the name $x$ is free, while $y$ is bound.

\begin{mathpar}
  \inferrule* [lab=monoidal-laws] {} { P|Q \equiv Q|P \and P|0 \equiv P \and P|(Q|R) \equiv (P|Q)|R }
\end{mathpar}

\begin{mathpar}
  \inferrule* [lab=alpha-equivalence] {} { (x)P \equiv (y)P\{y/x\} \and y \not\in \freenames{P} }
\end{mathpar}

\begin{definition}
Then two processes, $P,Q$, are alpha-equivalent if $P = Q\{\vec{y}/\vec{x}\}$ for
some $\vec{x} \in \boundnames{Q},\vec{y} \in \boundnames{P}$, where $Q\{\vec{y}/\vec{x}\}$
denotes the capture-avoiding substitution of $\vec{y}$ for $\vec{x}$ in $Q$.
\end{definition}

\begin{definition}
  The {\em structural congruence} \cite{SangiorgiWalker} , $\equiv$,
  between processes is the least congruence containing
  alpha-equivalence, satisfying the abelian monoid laws
  (associativity, commutativity and $\pzero$ as identity) for parallel
  composition $|$ and for summation $+$.
\end{definition}

\subsection{Name equivalence}

We take name equivalence, written $\nameeq$, to be the smallest
equivalence relation generated by the following rules.

\begin{mathpar}
\inferrule*[lab=Quote-drop]
{ }
{ \quotep{@{x}} \nameeq x }

\inferrule*[lab=Struct-equiv]
{ P \scong Q }
{ \quotep{P} \nameeq \quotep{Q} }
\end{mathpar}

The astute reader will have noticed that the mutual recursion of names
and processes imposes a mutual recursion on alpha-equivalence and
structural equivalence via name-equivalence. Fortunately, all of this
works out pleasantly and we may calculate in the natural way, free of
concern. The reader interested in the details is referred to the
appendix \ref{appendix:rho_details}.

\subsection{Substitution}

We use $\Proc$ for the set of processes, $\QProc$ for the set of
names, and $\id{\{}\vec{y} / \vec{x} \id{\}}$ to denote partial maps,
$s : \QProc \rightarrow \QProc$. A map, $s$ lifts, uniquely, to a map
on process terms, $\widehat{s} : \Proc \rightarrow \Proc$ by the
following equations.

\begin{mathpar}
  (0) \psubstp{Q}{P} := 0 \\
  (R \juxtap S) \psubstp{Q}{P}
  :=    
  (R)\psubstp{Q}{P} \juxtap (S) \psubstp{Q}{P} \\
  (x?(y).R) \psubstp{Q}{P}    
  :=    
  (x)\substp{Q}{P} (z)\concat( (R \psubstn{z}{y}) \psubstp{Q}{P} ) \\
  (\lift{x}{R}) \psubstp{Q}{P}  
  :=
  \lift{(x)\substp{Q}{P}}{ R \psubstp{Q}{P} } \\
%   (\dropn{x})  \psubstp{Q}{P}       
%   := 
%   \left\{ 
%     \begin{array}{ccc} 
%       \dropn{\quotep{Q}} & & x \nameeq \quotep{P} \\
%       \dropn{x} & & otherwise \\
%     \end{array}
%   \right. 
  (\dropn{x})  \psubstp{Q}{P}       
  := 
  \left\{ 
    \begin{array}{ccc} 
      Q & & x \nameeq \quotep{P} \\
      \dropn{x} & & otherwise \\
    \end{array}
  \right.
\end{mathpar}
 

where

\begin{eqnarray}
  (x)\id{\{} \lpquote Q \rpquote / \lpquote P \rpquote \id{\}}            = 
  \left\{ 
    \begin{array}{ccc}
      \lpquote Q \rpquote & & x \nameeq \lpquote P \rpquote \\
      x & & otherwise \\
    \end{array}
  \right. \nonumber
\end{eqnarray}

and $z$ is chosen distinct from $\quotep{P}$, $\quotep{Q}$, the free
names in $Q$, and all the names in $R$. Our $\alpha$-equivalence will
be built in the standard way from this substitution.

\begin{remark}\label{rem:no_self_referential_names}
  One consequence of these definitions is that $\forall P. \quotep{P}
  \not\in \freenames{P}$.
\end{remark}

\subsection{ Dynamic quote: an example }

Anticipating something of what's to come, consider applying the
substitution, $\widehat{\id{\{}u / z \id{\}}}$, to the following pair
of processes, $\lift{w}{y!(z)}$ and $w[ \lpquote y!(z) \rpquote ]$.

\begin{eqnarray}
	\lift{w}{y!(z)}\widehat{\id{\{}u / z \id{\}}}
		& = &
		\lift{w}{y!(u)} \nonumber\\
	w[ \lpquote y!(z) \rpquote ] \widehat{ \id{\{}u / z \id{\}} }
		& = &
		w[ \lpquote y!(z) \rpquote ] \nonumber
\end{eqnarray}

Because the body of the process between quotes is impervious to
substitution, we get radically different answers. In fact, by
examining the first process in an input context,
e.g. $x?(z).\lift{w}{y!(z)}$, we see that the process under the lift
operator may be shaped by prefixed inputs binding a name inside it. In
this sense, the lift operator will be seen as a way to dynamically
construct processes before reifying them as names.

Finally equipped with these standard features we can present the
dynamics of the calculus.

\subsubsection{Operational semantics} 

Finally, we introduce the computational dynamics. What marks these
algebras as distinct from other more traditionally studied algebraic
structures, e.g. vector spaces or polynomial rings, is the manner in
which dynamics is captured. In traditional structures, dynamics is typically
expressed through morphisms between such structures, as in linear maps
between vector spaces or morphisms between rings. In algebras
associated with the semantics of computation, the dynamics is
expressed as part of the algebraic structure itself, through a
reduction reduction relation typically denoted by $\red$. Below, we
give a recursive presentation of this relation for the calculus used
in the encoding.

$\red \subseteq \pi \times \pi$
$\red : \pi \to \mathcal{P}(\pi)$

\begin{mathpar}
  \inferrule* [lab=Comm] { \textsf{match}( x_{src}, x_{trgt} ) } { x_{trgt}?(y)P \; | \; x_{src}!\langle {Q} \rangle \red P\{\quotep{Q}/y}\} }
  \and \\
  \inferrule* [lab=Par] {{P} \red {P}'} {{{P} | {Q}} \red {{P}' | {Q}}}
  \and
  \inferrule* [lab=Equiv]{{{P} \scong {P}'} \andalso {{P}' \red {Q}'} \andalso {{Q}' \scong {Q}}}{{P} \red {Q}}
\end{mathpar}

\begin{eqnarray*}
  match_{\equiv} (\quotep{P},\quotep{Q}) & := & P \equiv Q \\
  match_{\dagger}(\quotep{P},\quotep{Q}) & := & \forall R. P|Q \red^{*} R => R \red^{*} 0 \\
  match_{K}(\quotep{P},\quotep{Q}) & := & K \mbox{ for some context } K
\end{eqnarray*}

$u?(x)P | u!\langle Q \rangle \red P\{\quotep{Q}/x\}$

%We write $\wred$ for $\red^*$, and $P\red$ if $\exists Q $ such that $ P \red Q$.
We write $P\red$ if $\exists Q $ such that $ P \red Q$ and $P\not\red$, otherwise.

\section{Replication}

As mentioned before, it is known that replication (and hence
recursion) can be implemented in a higher-order process algebra
\cite{SangiorgiWalker}. As our first example of calculation with the
machinery thus far presented we give the construction explicitly in
the {\rhoc}.

\begin{eqnarray}
	D_{x} & := & \prefix{x}{y}{(\binpar{\outputp{x}{y}}{@{y}})} \nonumber\\
	\bangp_{x}{P} & := & \binpar{{x}!\langle{\binpar{D_{x}}{P}}\rangle}{D_{x}} \nonumber
\end{eqnarray}

\begin{eqnarray}
	\bangp_{x}{P} & & \nonumber\\
	=
	& {x}!\langle{(\prefix{x}{y}{(\outputp{x}{y} | @{y})) | P}}\rangle 
	      | \prefix{x}{y}{(\outputp{x}{y} | @{y})} & \nonumber\\
	\red
	& (\outputp{x}{y} | @{y})\substn{\quotep{(\prefix{x}{y}{(@{y} | \outputp{x}{y})) | P}}}{y} & \nonumber\\
	=
	& \outputp{x}{\quotep{(\prefix{x}{y}{(\outputp{x}{y} | @{y})) | P}}}
	  | {(\prefix{x}{y}{(\outputp{x}{y} | @{y})) | P}} & \nonumber\\
	\red
	& \ldots & \nonumber\\
	\red^*
	& P | P | \ldots & \nonumber
\end{eqnarray}

Of course, this encoding, as an implementation, runs away, unfolding
$\bangp{P}$ eagerly. A lazier and more implementable replication
operator, restricted to input-guarded processes, may be obtained as follows.

\begin{eqnarray}
\bangp{\prefix{u}{v}{P}} 
	:= 
	\binpar{\lift{x}{\prefix{u}{v}{(\binpar{D(x)}{P})}}}{D(x)} \nonumber
\end{eqnarray}

\begin{remark}
  Note that the lazier definition still does not deal with summation
  or mixed summation (i.e. sums over input and output). The reader is
  invited to construct definitions of replication that deal with these
  features. 

  Further, the definitions are parameterized in a name, $x$. Can you,
  gentle reader, make a definition that eliminates this parameter and
  guarantees no accidental interaction between the replication
  machinery and the process being replicated -- i.e. no accidental
  sharing of names used by the process to get its work done and the
  name(s) used by the replication to effect copying. This latter
  revision of the definition of replication is crucial to obtaining
  the expected identity $!!P \sim !P$.
\end{remark}

\begin{remark}\label{rem:paradoxical_combinator}
  The reader familiar with the lambda calculus will have noticed the
  similarity between $D$ and the paradoxical combinator.

  [Ed. note: the existence of this seems to suggest we have to be more
  restrictive on the set of processes and names we admit if we are to
  support no-cloning.]
\end{remark}

\subsubsection{Bisimulation}

The computational dynamics gives rise to another kind of equivalence,
the equivalence of computational behavior. As previously mentioned
this is typically captured \emph{via} some form of bisimulation.

% The notion we use in this paper is weak barbed bisimulation
% \cite{milner91polyadicpi}.

The notion we use in this paper is derived from weak barbed
bisimulation \cite{milner91polyadicpi}. 

\begin{definition}
An \emph{observation relation}, $\downarrow_{\mathcal N}$, over a set
of names, $\mathcal N$, is the smallest relation satisfying the rules
below.

\infrule[Out-barb]{y \in {\mathcal N}, \; x \nameeq y}
		  {\outputp{x}{v} \downarrow_{\mathcal N} x}
\infrule[Par-barb]{\mbox{$P\downarrow_{\mathcal N} x$ or $Q\downarrow_{\mathcal N} x$}}
		  {\binpar{P}{Q} \downarrow_{\mathcal N} x}

We write $P \Downarrow_{\mathcal N} x$ if there is $Q$ such that 
$P \wred Q$ and $Q \downarrow_{\mathcal N} x$.
\end{definition}

\begin{definition}
%\label{def.bbisim}
An  ${\mathcal N}$-\emph{barbed bisimulation} over a set of names, ${\mathcal N}$, is a symmetric binary relation 
${\mathcal S}_{\mathcal N}$ between agents such that $P\rel{S}_{\mathcal N}Q$ implies:
\begin{enumerate}
\item If $P \red P'$ then $Q \wred Q'$ and $P'\rel{S}_{\mathcal N} Q'$.
\item If $P\downarrow_{\mathcal N} x$, then $Q\Downarrow_{\mathcal N} x$.
\end{enumerate}
$P$ is ${\mathcal N}$-barbed bisimilar to $Q$, written
$P \wbbisim_{\mathcal N} Q$, if $P \rel{S}_{\mathcal N} Q$ for some ${\mathcal N}$-barbed bisimulation ${\mathcal S}_{\mathcal N}$.
\end{definition}

$\mathcal{R} \subseteq \pi \times \pi$

$P \mathcal{R} Q => \forall P'. P \red P' \Rightarrow \exists Q'. Q \red Q', P' \mathcal{R} Q'$

$P \vdash x \Rightarrow Q \vdash x$

\begin{mathpar}
  \inferrule*[lab=Out-barb]{x \nameeq y}{{y}!\langle{Q}\rangle \vdash x}
  \and
  \inferrule*[lab=Par-barb]{\mbox{$P\vdash x$ or $Q\vdash x$}}{\binpar{P}{Q} \vdash x}
\end{mathpar}

\subsubsection{Contexts}

One of the principle advantages of computational calculi like the
$\pi$-calculus is a well-defined notion of context,
contextual-equivalence and a correlation between
contextual-equivalence and notions of bisimulation. The notion of
context allows the decomposition of a process into (sub-)process and
its syntactic environment, its context. Thus, a context may be
thought of as a process with a ``hole'' (written $\Box$) in it. The
application of a context $M$ to a process $P$, written $M[P]$, is
tantamount to filling the hole in $M$ with $P$. In this paper we do
not need the full weight of this theory, but do make use of the notion
of context in the proof the main theorem. 

\begin{mathpar}
  \inferrule* [lab=summation] {} {{M_{M},M_{N}} \bc \Box \;|\; x.M_{A} \;|\; M_{M}+M_{N}}
  \and
  \inferrule* [lab=agent] {} {{M_{A}} \bc (\vec{x})M_{P} \;| \; \clift{P_0,\ldots,M_{P},\ldots,P_N}}
  \and \\
  \inferrule* [lab=process] {} {{M_{P}} \bc M_{N} \;| \;P|M_{P} }
\end{mathpar} 

\begin{mathpar}
  \inferrule* [lab=sychronization] {} {M_{N} \bc \Box \;|\; x?M_{F} \;|\; x!M_{C}}
  \and
  \inferrule* [lab=abstraction] {} {{M_{F}} \bc (x)M_{P} }
  \and
  \inferrule* [lab=concretion] {} {{M_{C}} \bc \langle M_{P} \rangle }
  \and \\
  \inferrule* [lab=process] {} {{M_{P}} \bc M_{N} \;| \;P|M_{P} }
\end{mathpar}

\begin{definition}[contextual application] Given a context $M$, and
  process $P$, we define the \emph{contextual application}, $M[P] :=
  M\{P/\Box\}$. That is, the contextual application of M to P is the
  substitution of $P$ for $\Box$ in $M$.
\end{definition}

$\meaningof{-} : L \to \mathcal{P}(\pi)$

\begin{mathpar}
  \inferrule* [lab=collection] {} {\meaningof{true} = \pi, \and \meaningof{~E} = \pi \setminus \meaningof{E}, \and \meaningof{E_{1} \& E_{2}} = \meaningof{E_{1}} \cap \meaningof{E_{2}}}
\end{mathpar}

\begin{mathpar}
  \inferrule* [lab=structure] {} {\meaningof{0} = \{ P \in \pi | P \equiv 0 \}, \and \\ \meaningof{E_1 | E_2} = \{ P \in \pi | P \equiv P_{1} | P_{2}, P_{1} \in \meaningof{E_{1}}, P_{2} \in \meaningof{E_2}\} }
\end{mathpar}

\begin{mathpar}
 \inferrule* [lab=behavior] {} {\meaningof{\langle a?b \rangle E} = \{ P \in \pi | P \equiv Q | u?(y)P', \\ \and \\\\ \and \\ \;\;\; u \in \meaningof{a}, \forall z.P'\{z/y\} \in \meaningof{E\{z/b\}}\}, \and \\ \meaningof{a!E} = \{ P \in \pi | P \equiv Q | x!\langle P' \rangle, x \in \meaningof{a} P' \in \meaningof{E}\} }
\end{mathpar}

\begin{mathpar}
 \inferrule* [lab=nominal] {} {\meaningof{\quotep{E}} = \{ \quotep{P} \in \quotep{\pi} | P \in \meaningof{E} \}, \and \meaningof{\quotep{P}} = \{ \quotep{Q} \in \quotep{\pi} | P \equiv Q \} \and \\ \meaningof{@\quotep{E}} = \{ P \in \pi | P \equiv @x, x \in \meaningof{E} \}}
\end{mathpar}

\begin{eqnarray*}
  \\
  \meaningof{-} : TS \to ST
\end{eqnarray*}

\begin{eqnarray*}
  \\
  L : TS \to ST
\end{eqnarray*}

\begin{eqnarray*}
  \\
  P \models E \iff P \in \meaningof{E}
\end{eqnarray*}

\begin{eqnarray*}
  P \approx_{L} Q \iff \forall E \in L. P \models E \iff Q \models E
\end{eqnarray*}

\begin{eqnarray*}
  P \approx_{K} Q
\end{eqnarray*}

\begin{eqnarray*}
  P \approx Q
\end{eqnarray*}

$\approx_{K} = \approx = \approx_{L}$

\subsubsection{Contextual duality}

Note that contexts extend the quotation operation to a family of
operations from processes to names. Given a context, $M$, we can
define a \emph{nominal context}, $\quotep{M}$ by $\quotep{M}[P] :=
\quotep{M[P]}$. To foreshadow what is to come we observe that these
operations enjoy a duality with processes very much like the duality
between vectors and maps from vectors to scalars.

Further, because the calculus is essentially higher-order, we have a
correspondence between contexts and processes. More specifically,
given a name $x$ and a context $M$ we can construct $M^{*}_{x}$ such
that 

\begin{mathpar}
  M^{*}_{x} | \lift{x}{P} \red M[P]
\end{mathpar}

namely,

\begin{mathpar}
  M^{*}_{x} := x?(u).M[\dropn{u}]
\end{mathpar}

The dependence of $M^{*}_{x}$ on a name makes it an abstraction, 

\begin{mathpar}
  M^{*} := (x)x?(u).M[\dropn{u}]
\end{mathpar}

\subsection{Additional notation}

It will sometimes be convenient to denote the process a name
quotes. We already have the notation $x = \quotep{P}$, but it will be
convenient to introduce an alternate notation, $\procn{x}$, when we
want to emphasize the connection to the use of the name. Note that, by
virtue of name equivalence, $\quotep{\procn{x}} \nameeq x$; so, the
notation is consistent with previous definitions.

Further, because names have structure it is possible to effect
substitutions on the basis of that structure. This means we need to
upgrade our notation for substitutions, which we accomplish by
adapting comprehension notation. Thus,

\begin{mathpar}
  P\{ y / x : x \in S \}
\end{mathpar}

is interpreted to mean the process derived from P by replacing (in a
capture-avoiding manner) each occurrence of $x$ in $S$ by $y$. For example,

\begin{mathpar}
  P\{ \quotep{\procn{x}|\procn{x}} / x : x \in \freenames{P} \}
\end{mathpar}

will replace each (occurrence) of a free name $x$ in $P$ by
$\quotep{\procn{x}|\procn{x}}$.

Also, we will avail ourselves of the notation $x^{L}$ and $x^{R}$ to
denote injections of a name into disjoint copies of the name
space. There are numerous ways to accomplish this. One example can be
found in \cite{MeredithR05}. This notation overloads to vectors of
names: $\vec{x}^{\pi} := (x_{i}^{\pi} \; : \; 0 \leq i < |\vec{x}| )$ where $\pi \in \{L,R\}$.

We also use $P^{\Box} := P|\Box$.

In \cite{MeredithR05} an interpretation of the new operator is
given. It turns out that there are several possible interpretations
all enjoying the requisite algebraic properties of the operator (see
\cite{milner91polyadicpi}). We will therefore make liberal use of
$(\nu\; \vec{x})P$.

% subsection the_syntax_and_semantics_of_the_notation_system (end)   

\section{Interpretation of QM}
\subsection{Supporting definitions}
\subsubsection{Multiplication}
\begin{mathpar}
  \quotep{Q} \cdot \quotep{R} := \quotep{Q|R}
  \and \\
  \quotep{Q} \cdot P := P\{ \quotep{Q|R} / \quotep{R} : \quotep{R} \in \freenames{P} \}
\end{mathpar}

\paragraph{Discussion}
The first line needs little explanation. The second line says that
each free name of the process is replaced with the multiplication of
that name by the scalar. Multiplication of a scalar (name) by a state
(process) results in a process all the names of which have been `moved
over' by parallel composition with the process the scalar
quotes. There is a subtlety that the bound names have to be
manipulated so that multiplied names aren't accidentally
captured. There are many ways to achieve this.

\begin{remark}\label{rem:multiplication_identities}
  The reader is invited to verify that for all $x,y,z \in \QProc$ and $P \in \Proc$
  \begin{mathpar}
    x \cdot \quotep{0} \equiv x 
    \and
    x \cdot y \equiv y \cdot x
    \and
    x \cdot (y \cdot z) \equiv (x \cdot y) \cdot z
    \and \\
    \quotep{0} \cdot P \equiv P
    \and \\
    x \cdot (y \cdot P) \equiv (x \cdot y) \cdot P
    \and \\
    x \cdot (P|Q) \equiv (x \cdot P) | (x \cdot Q)
    \and \\    
  \end{mathpar}
\end{remark}

\subsubsection{Tensor product}

We define a tensor product on processes by structural induction.

\paragraph{Tensor of sums} First note that all summations, including
$\pzero$ and sequence, can be written $\Sigma_{i} x_{i}.A_{i} +
\Sigma_{j} x_{j}.C_{j}$, where we have grouped input-guarded processes
together and output-guarded processes together.

Thus, we can define the tensor product of two summations, $N_{1}\otimes N_{2}$, where

\begin{mathpar}
  N_{1} := \Sigma_{i} x_{i}.A_{i} + \Sigma_{j} x_{j}.C_{j}
  \and
  N_{2} := \Sigma_{i'} y_{i'}.B_{i'} + \Sigma_{j'} y_{j'}.D_{j'} 
\end{mathpar}

as follows.

\begin{mathpar}
  \Sigma_{i} x_{i}.A_{i} + \Sigma_{j} x_{j}.C_{j} \otimes \Sigma_{i'}
  y_{i'}.B_{i'} + \Sigma_{j'} y_{j'}.D_{j'} 
  \and \\
  := \; \Sigma_{i} \Sigma_{i'} \quotep{\stackrel{\vee}{x_{i}}| \stackrel{\vee}{y_{i'}}}.(A_{i}\otimes B_{i'}) \; | \; \Sigma_{i'} \Sigma_{i} \quotep{\stackrel{\vee}{y_{i'}}|\stackrel{\vee}{x_{i}}}.(B_{i'}\otimes A_{i})
  \and
  \;\; | \;\; \Sigma_{j} \Sigma_{j'} \quotep{\stackrel{\vee}{x_{j}}|\stackrel{\vee}{y_{j'}}}.(A_{j}\otimes B_{j'}) \; | \; \Sigma_{j'} \Sigma_{j} \quotep{\stackrel{\vee}{y_{j'}}|\stackrel{\vee}{x_{j}}}.(B_{j'}\otimes A_{j})
\end{mathpar}

\begin{remark}
  Do we need to $x^{L}$ and $y^{R}$ for this construction as well?
\end{remark}

\paragraph{Tensor of parallel compositions} Next, we distribute tensor
over par.

\begin{mathpar}
  P_{1}|P_{2} \otimes Q_{1}|Q_{2} := (P_{1} \otimes Q_{1}) | (P_{1}
  \otimes Q_{2}) | (P_{2} \otimes Q_{1}) | (P_{2} \otimes Q_{2})
\end{mathpar}

\paragraph{Tensor with dropped names} We treat tensor of a
process with a dropped name as parallel composition.

\begin{mathpar}
  P \otimes \dropn{x} := P | \dropn{x}
\end{mathpar}

\paragraph{Tensor of agents}

Finally, we need to define tensor on agents. Note that the definition
of tensor on normal products only tensors inputs with inputs and
outputs with outputs. Thus, we only have to define the operation on
``homogeneous'' pairings.

\begin{mathpar}
  (\vec{x})P \otimes (\vec{y})Q
  \and \\
  := (x_{0}^{L}|y_{0}^{R},\ldots,x_{0}^{L}|y_{n}^{R},\ldots,x_{m}^{L}|y_{0}^{R},\ldots,x_{m}^{L}|y_{n}^R)(P\{ \vec{x}^{L}/\vec{x}\} \otimes Q \{ \vec{y}^{R}/\vec{y}\})
  \and \\
  \clift{\vec{P}} \otimes \clift{\vec{Q}}
  \and \\
  := \clift{P_{0}\otimes Q_{0},\ldots,P_{0}\otimes Q_{n},\ldots,P_{m}\otimes Q_{0},\ldots,P_{m}\otimes Q_{n}}
\end{mathpar}

\begin{remark}
  Observe that arities of tensored abstractions matches arities of
  tensored concretions if the original arities matched. Note also that
  the length of the arities corresponds to the increase in dimension
  we see in ordinary vector space tensor product.
\end{remark}

\begin{remark}
  Operationally, this definition distributes the tensor down to
  components ``linked'' by summation. Tensor over summation is
  intriguing in that it mixes names. Moreover, as a consequence of the
  way it mixes names we have the identities for all $x \in \QProc$ and
  $P,Q \in \Proc$

  \begin{mathpar}
    (x \cdot P) \otimes Q \equiv x \cdot (P \otimes Q) \equiv P \otimes (x \cdot Q)
    \and
    P \otimes \pzero \equiv P
  \end{mathpar}

  that the reader is invited to verify.
\end{remark}

\subsubsection{Annihilation}
\begin{mathpar}
  P^{\perp} := \{ Q | \forall R. P|Q \red^{*} R \Rightarrow R \red^{*} \pzero \}
  \and \\
  P^{\underline{\perp}} := \Sigma_{Q \in P^{\perp}} \quotep{Q}?(y).(\dropn{y}|Q) | \Sigma_{Q \in P^{\perp}} \quotep{Q}\clift{\Box}
\end{mathpar}

\paragraph{Discussion} The reader will note that $P^{\perp}$ is a
\emph{set} of processes, while $P^{\underline{\perp}}$ is a
\emph{context}. We call the set $P^{\perp}$ the \emph{annihilators} of
$P$. The parallel composition of a process in the annihilators of $P$
with $P$ will result in a process, the state space of which has all
paths eventually leading to $\pzero$. Execution may endure loops; but
under reasonable conditions of fairness (naturally guaranteed under
most notions of bisimulation) such a composite process cannot get
stuck in such a loop and will, eventually pop out and terminate.

The context $P^{\underline{\perp}}$ is ready and willing to ``take the
$P$ out of'' the process to which it is applied. It will effectively
transmit the code of the process to which it is applied to one of the
annihilators and run the process against it.

\subsubsection{Evaluation}
We fix $M$ a domain of fully abstract interpretation with an equality
coincident with bisimulation. We take $\meaningof{\cdot} : \Proc \to
M$ to be the map interpreting processes and $\nmeaningof{\cdot} : \M
\to Proc$ to be the map running the other way. Then we define

\begin{mathpar}
  \int P := \nmeaningof{\meaningof{P}}
\end{mathpar}

\paragraph{Discussion}
There are many fully abstract interpretations of Milner's
$\pi$-calculus. Any of them can be used as a basis for interpreting
the reflective calculus here. Equipped with such a domain it is
largely a matter of grinding through to check that the Yoneda
construction for the normalization-by-evaluation program can be
extended to this setting.

\begin{remark}
  The reader is invited to verify that $\int (P^{\underline{\perp}}[P]) = 0$.
\end{remark}

\subsection{Quantum mechanics}

Table \ref{tbl:core_qm_op_defns} gives the core operational definitions

\begin{table}[htp]\label{tbl:core_qm_op_defns}
  \center{
    \fbox{
      \begin{tabular}{c|c}
        quantum mechanics & process calculus \\
        \hline
        scalar & $x := \quotep{P}$ \\
        state vector & $\state{P} := P$ \\
        dual & $\state{P}^{*} := \event{P^{\underline{\perp}}} := \quotep{P^{\underline{\perp}}}[-]$ \\
        matrix & $ \Sigma_{\alpha} \state{P_{\alpha}}x_{\alpha}\event{Q_{\alpha}}$ \\
        vector addition & $\state{P} + \state{Q} := \state{P | Q}$ \\
        tensor product & $\state{P} \otimes \state{Q} := \state{P \otimes Q}$ \\
        inner product & $\innerprod{P}{Q} := \quotep{\int P^{\underline{\perp}}[Q]}$ \\
      \end{tabular}
    }
  }
  \caption{QM - operational definitions}
\end{table}

where

\begin{mathpar}
  \prmatrix{P}{Q} := \fprmatrix{P}{\quotep{\pzero}}{Q}
  \and
  \fprmatrix{P}{x}{Q} := (\state{P},x,\event{Q})
  \and
  (\fprmatrix{P}{x}{Q})(\state{R}) := x \cdot \innerprod{Q}{R} \cdot \state{P}
  \and
  (\fprmatrix{P}{x}{Q})(\event{R}) := x \cdot \innerprod{R}{P} \cdot \event{Q}
\end{mathpar}

\paragraph{Discussion}
As promised: vectors (aka states) are represented as processes; duals
as contextual duals; inner product definition should be compared with
standard inner product definition for ....

\begin{remark}
  Assuming $\int (P^{\underline{\perp}}[P]) = 0$, the reader is
  invited to verify that $(\fprmatrix{P}{x}{P})(\state{P}) = x \cdot \state{P}$.
\end{remark}

\begin{remark}
  The reader is invited to verify that $\innerprod{P}{Q}$ could
  equally well have been written $\quotep{\int \stackrel{\vee}{x}}$
  where $x = \event{P^{\underline{\perp}}}(Q)$.

  One of the motivations for this remark is that there is another way
  to factor these operations. We could package up evaluation in the dual:

  \begin{mathpar}
    \state{P}^{*} := \event{\int P^{\underline{\perp}}} := \quotep{\int P^{\underline{\perp}}}[-]
  \end{mathpar}

  and then have inner product defined by
  
  \begin{mathpar}
    \innerprod{P}{Q} := \event{P}(Q)
  \end{mathpar}

  Hopefully, experience with the calculations will provide guidance on
  the best factoring.
\end{remark}

\begin{remark}
  Assuming $\int (P^{\underline{\perp}}[P]) = 0$, the reader is
  invited to verify that $\forall P,Q. (\prmatrix{0}{Q})(\state{0}) =
  \state{0}$ and dually $(\prmatrix{P}{0})(\event{0}) = \event{0}$.
\end{remark}

\begin{remark}
  i'm a little worried that i don't (yet) have proper support for
  complex conjugacy. But, the observation above may give us a
  clue. According to Abramsky, it must be the case that the scalars
  are iso to the homset of the identity for the tensor -- which the
  observation above characterizes. 

  For now, we will simply bookmark the notion with $\overline{x}$.
\end{remark}

\subsubsection{Adjointness}

We need to give a definition of $(\cdot)^{\dagger}$ for matrices. The
obvious candidate definition is
\begin{mathpar}
(\Sigma_{\alpha}\fprmatrix{P_{\alpha}}{x_{\alpha}}{Q_{\alpha}})^{\dagger}
= \Sigma_{\alpha}\fprmatrix{(Q_{\alpha}^{\underline{\perp}})^{*}}{\overline{x}_{\alpha}}{P_{\alpha}^{\underline{\perp}}} 
\end{mathpar}

But, $(Q_{\alpha}^{\underline{\perp}})^{*}$ requires a name along
which to communicate the process to achieve the context application.

\subsubsection{Basis for a basis}
If processes label states and ``addition'' of states (a.k.a. vector
addition) is interpreted as parallel composition, what corresponds to
notions of linear independence and basis? Here, we recall that Yoshida
has developed a set of \emph{combinators} for an asynchronous verison
of Milner's $\pi$-calculus. These are a finite set of processes such
any process can be expressed as parallel composition of these
combinators together with liberal uses of the new operator and
replication. We can simply give a translation of these into the
present calculus and have reasonable expectation that the property
carries over. That is, that the resultant set allows to express all
processes via parallel composition. Note, however, that there is no
new operator or replication in this calculus. As a result, we expect
that the corresponding set is actually infinite. That is, we expect
that the space is actually infinite dimensional.

\begin{remark}
  The attentive reader may be a bit concerned. Certainly, the
  collection $S$, $K$ and $I$ is a finite set of
  combinators. Shouldn't we expect to see a finite set of combinators
  for an effectively equivalent system? i am very sympathetic to this
  critique and feel it warrants full attention. On the other hand, i
  also have in mind the following analogy. The natural numbers, as a
  monoid under addition, has exactly $1$ generator, while the natural
  numbers, as a monoid under multiplication, has countably many
  generators (the primes). We observe that the application of the
  lambda calculus is much less resource sensitive than the parallel
  composition of the $\pi$-calculus. Could it be the case that we have
  an analogy of the form
  
  \begin{mathpar}
    m + n : MN :: m*n : M|N
  \end{mathpar}

  giving a similar blow up in the set of ``primes''?  This is such a
  wonderful thought that, even if it's not true, i think it's worth
  writing down.
\end{remark}
 

\documentclass[12pt]{llncs}
%\documentclass{jktr}

\usepackage[pdftex]{hyperref}                   
\usepackage {listings}
\usepackage {mathpartir}
\usepackage{bcprules}
%\usepackage{listings}
                       
\usepackage{graphicx} 
%\usepackage[margins=2.5cm,nohead,nofoot]{geometry}
%\usepackage{geometry}
\usepackage{amsfonts}
\usepackage{amstext}
\usepackage{latexsym}
\usepackage{amssymb}
\usepackage{color}


%\include{myPreamble}
\include{qm2pi.local} 

%\ifpdf
%\usepackage[pdftex]{graphicx}
%\else
%\usepackage{graphicx}
%\fi

 % \ifpdf
%  \usepackage{pdfsync}
%  \if


%\title{Brief Article}
%\author{David F. Snyder}
%\author{L.G. Meredith}

%\address{Dept. of Math., Texas State University--San Marcos, San Marcos, TX 78666}
       
\pagestyle{empty}


\begin{document}

\lstset{language=[Objective]Caml,frame=shadowbox}

\input{qm2pi.front}

% section front matter (end)

\input{qm2pi.intro} 
 
% section introduction (end)

% \input{qm2pi.knotations} 

% section notation (end)

\input{qm2pi.process.calculi} 

% section concurrent_process_calculi_and_spatial_logics_ (end)
    
%\input{qm2pi.knots2pi} 

%\input{qm2pi.trefoil} 

%\input{qm2pi.mainthm} 

% subsection basic_interpretation (end)

%\input{qm2pi.rho.presentation} 
\subsection{The syntax and semantics of the notation system}\label{sub:the_syntax_and_semantics_of_the_notation_system} % (fold)

We now summarize a technical presentation of the calculus that
embodies our theory of dynamics. The typical presentation of such a
calculus follows the style of giving generators and relations on
them. The grammar, below, describing term constructors, freely
generates the set of processes, $\Proc$. This set is then quotiented
by a relation known as structural congruence and it is over this set
that the notion of dynamics is expressed. This presentation is
essentially that of \cite{MeredithR05} with the addition of
polyadicity and summation. For readability we have relegated some of
the technical subtleties to an appendix.

\subsubsection{Process grammar}\label{subsub:process_grammar}

\begin{mathpar}
  \inferrule* [lab=synchronization] {} {{M} \bc \pzero \;|\; x?F \;|\; x!C }
  \and
  \inferrule* [lab=abstraction] {} {{F} \bc (x)P}
  \and
  \inferrule* [lab=concretion] {} {{C} \bc \langle Q \rangle}
  \and
  \inferrule* [lab=process] {} {{P,Q} \bc M \;| \;P|Q \;|\; @{x}}
  \and
  \inferrule* [lab=name] {} {{x} \bc \quotep{P}}
\end{mathpar} 

Note that $\vec{x}$ (resp. $\vec{P}$) denotes a vector of names
(resp. processes) of length $|\vec{x}|$ (resp. $|\vec{P}|$). We adopt
the following useful abbreviations.

\begin{mathpar}
   x?(\vec{y}).P := x.(\vec{y})P \and  x\clift{\vec{P}} := x.\clift{\vec{P}}
   \and x!(y) := \lift{x}{\dropn{y}}
   \and \Pi_{i=0}^{n-1}P_i := P_0 | \ldots | P_{n-1}
\end{mathpar}

\subsubsection{Structural congruence}

\paragraph{Free and bound names and alpha-equivalence.} At the
core of structural equivalence is alpha-equivalence which identifies
process that are the same up to a change of variable. Formally, we
recognize the distinction between free and bound names. The free names
of a process, $\freenames{P}$, may be calculated recursively as
follows:

\begin{mathpar}
\freenames{\pzero} := \emptyset
  \and \\
  \freenames{x?(y).P} := \{ x \} \cup (\freenames{P} \setminus \{ y \})
  \and 
  \freenames{x!\langle P \rangle} := \{ x \} \cup \{ P \} 
  \and \\
  \freenames{P|Q} := \freenames{P} \cup \freenames{Q}
  \and \\
  \freenames{@{x}} := \{ x \}
\end{mathpar}

$\pi$
$\quotep{\pi}$

$\freenames{-} : \pi \to \mathcal{P}(\quotep{\pi})$

\begin{eqnarray*}
  \freenames{\pzero} & := & \emptyset \\
  \freenames{x?(y).P} & := & \{ x \} \cup (\freenames{P} \setminus \{ y \}) \\
  \freenames{x!\langle P \rangle} & := & \{ x \} \cup \{ P \} \\
  \freenames{P|Q} & := & \freenames{P} \cup \freenames{Q} \\
  \freenames{\dropn{x}} & := & \{ x \}
\end{eqnarray*}

The bound names of a process, $\boundnames{P}$, are those names occurring in $P$
that are not free. For example, in $x?(y).0$, the name $x$ is free, while $y$ is bound.

\begin{mathpar}
  \inferrule* [lab=monoidal-laws] {} { P|Q \equiv Q|P \and P|0 \equiv P \and P|(Q|R) \equiv (P|Q)|R }
\end{mathpar}

\begin{mathpar}
  \inferrule* [lab=alpha-equivalence] {} { (x)P \equiv (y)P\{y/x\} \and y \not\in \freenames{P} }
\end{mathpar}

\begin{definition}
Then two processes, $P,Q$, are alpha-equivalent if $P = Q\{\vec{y}/\vec{x}\}$ for
some $\vec{x} \in \boundnames{Q},\vec{y} \in \boundnames{P}$, where $Q\{\vec{y}/\vec{x}\}$
denotes the capture-avoiding substitution of $\vec{y}$ for $\vec{x}$ in $Q$.
\end{definition}

\begin{definition}
  The {\em structural congruence} \cite{SangiorgiWalker} , $\equiv$,
  between processes is the least congruence containing
  alpha-equivalence, satisfying the abelian monoid laws
  (associativity, commutativity and $\pzero$ as identity) for parallel
  composition $|$ and for summation $+$.
\end{definition}

\subsection{Name equivalence}

We take name equivalence, written $\nameeq$, to be the smallest
equivalence relation generated by the following rules.

\begin{mathpar}
\inferrule*[lab=Quote-drop]
{ }
{ \quotep{@{x}} \nameeq x }

\inferrule*[lab=Struct-equiv]
{ P \scong Q }
{ \quotep{P} \nameeq \quotep{Q} }
\end{mathpar}

The astute reader will have noticed that the mutual recursion of names
and processes imposes a mutual recursion on alpha-equivalence and
structural equivalence via name-equivalence. Fortunately, all of this
works out pleasantly and we may calculate in the natural way, free of
concern. The reader interested in the details is referred to the
appendix \ref{appendix:rho_details}.

\subsection{Substitution}

We use $\Proc$ for the set of processes, $\QProc$ for the set of
names, and $\id{\{}\vec{y} / \vec{x} \id{\}}$ to denote partial maps,
$s : \QProc \rightarrow \QProc$. A map, $s$ lifts, uniquely, to a map
on process terms, $\widehat{s} : \Proc \rightarrow \Proc$ by the
following equations.

\begin{mathpar}
  (0) \psubstp{Q}{P} := 0 \\
  (R \juxtap S) \psubstp{Q}{P}
  :=    
  (R)\psubstp{Q}{P} \juxtap (S) \psubstp{Q}{P} \\
  (x?(y).R) \psubstp{Q}{P}    
  :=    
  (x)\substp{Q}{P} (z)\concat( (R \psubstn{z}{y}) \psubstp{Q}{P} ) \\
  (\lift{x}{R}) \psubstp{Q}{P}  
  :=
  \lift{(x)\substp{Q}{P}}{ R \psubstp{Q}{P} } \\
%   (\dropn{x})  \psubstp{Q}{P}       
%   := 
%   \left\{ 
%     \begin{array}{ccc} 
%       \dropn{\quotep{Q}} & & x \nameeq \quotep{P} \\
%       \dropn{x} & & otherwise \\
%     \end{array}
%   \right. 
  (\dropn{x})  \psubstp{Q}{P}       
  := 
  \left\{ 
    \begin{array}{ccc} 
      Q & & x \nameeq \quotep{P} \\
      \dropn{x} & & otherwise \\
    \end{array}
  \right.
\end{mathpar}
 

where

\begin{eqnarray}
  (x)\id{\{} \lpquote Q \rpquote / \lpquote P \rpquote \id{\}}            = 
  \left\{ 
    \begin{array}{ccc}
      \lpquote Q \rpquote & & x \nameeq \lpquote P \rpquote \\
      x & & otherwise \\
    \end{array}
  \right. \nonumber
\end{eqnarray}

and $z$ is chosen distinct from $\quotep{P}$, $\quotep{Q}$, the free
names in $Q$, and all the names in $R$. Our $\alpha$-equivalence will
be built in the standard way from this substitution.

\begin{remark}\label{rem:no_self_referential_names}
  One consequence of these definitions is that $\forall P. \quotep{P}
  \not\in \freenames{P}$.
\end{remark}

\subsection{ Dynamic quote: an example }

Anticipating something of what's to come, consider applying the
substitution, $\widehat{\id{\{}u / z \id{\}}}$, to the following pair
of processes, $\lift{w}{y!(z)}$ and $w[ \lpquote y!(z) \rpquote ]$.

\begin{eqnarray}
	\lift{w}{y!(z)}\widehat{\id{\{}u / z \id{\}}}
		& = &
		\lift{w}{y!(u)} \nonumber\\
	w[ \lpquote y!(z) \rpquote ] \widehat{ \id{\{}u / z \id{\}} }
		& = &
		w[ \lpquote y!(z) \rpquote ] \nonumber
\end{eqnarray}

Because the body of the process between quotes is impervious to
substitution, we get radically different answers. In fact, by
examining the first process in an input context,
e.g. $x?(z).\lift{w}{y!(z)}$, we see that the process under the lift
operator may be shaped by prefixed inputs binding a name inside it. In
this sense, the lift operator will be seen as a way to dynamically
construct processes before reifying them as names.

Finally equipped with these standard features we can present the
dynamics of the calculus.

\subsubsection{Operational semantics} 

Finally, we introduce the computational dynamics. What marks these
algebras as distinct from other more traditionally studied algebraic
structures, e.g. vector spaces or polynomial rings, is the manner in
which dynamics is captured. In traditional structures, dynamics is typically
expressed through morphisms between such structures, as in linear maps
between vector spaces or morphisms between rings. In algebras
associated with the semantics of computation, the dynamics is
expressed as part of the algebraic structure itself, through a
reduction reduction relation typically denoted by $\red$. Below, we
give a recursive presentation of this relation for the calculus used
in the encoding.

$\red \subseteq \pi \times \pi$
$\red : \pi \to \mathcal{P}(\pi)$

\begin{mathpar}
  \inferrule* [lab=Comm] { \textsf{match}( x_{src}, x_{trgt} ) } { x_{trgt}?(y)P \; | \; x_{src}!\langle {Q} \rangle \red P\{\quotep{Q}/y}\} }
  \and \\
  \inferrule* [lab=Par] {{P} \red {P}'} {{{P} | {Q}} \red {{P}' | {Q}}}
  \and
  \inferrule* [lab=Equiv]{{{P} \scong {P}'} \andalso {{P}' \red {Q}'} \andalso {{Q}' \scong {Q}}}{{P} \red {Q}}
\end{mathpar}

\begin{eqnarray*}
  match_{\equiv} (\quotep{P},\quotep{Q}) & := & P \equiv Q \\
  match_{\dagger}(\quotep{P},\quotep{Q}) & := & \forall R. P|Q \red^{*} R => R \red^{*} 0 \\
  match_{K}(\quotep{P},\quotep{Q}) & := & K \mbox{ for some context } K
\end{eqnarray*}

$u?(x)P | u!\langle Q \rangle \red P\{\quotep{Q}/x\}$

%We write $\wred$ for $\red^*$, and $P\red$ if $\exists Q $ such that $ P \red Q$.
We write $P\red$ if $\exists Q $ such that $ P \red Q$ and $P\not\red$, otherwise.

\section{Replication}

As mentioned before, it is known that replication (and hence
recursion) can be implemented in a higher-order process algebra
\cite{SangiorgiWalker}. As our first example of calculation with the
machinery thus far presented we give the construction explicitly in
the {\rhoc}.

\begin{eqnarray}
	D_{x} & := & \prefix{x}{y}{(\binpar{\outputp{x}{y}}{@{y}})} \nonumber\\
	\bangp_{x}{P} & := & \binpar{{x}!\langle{\binpar{D_{x}}{P}}\rangle}{D_{x}} \nonumber
\end{eqnarray}

\begin{eqnarray}
	\bangp_{x}{P} & & \nonumber\\
	=
	& {x}!\langle{(\prefix{x}{y}{(\outputp{x}{y} | @{y})) | P}}\rangle 
	      | \prefix{x}{y}{(\outputp{x}{y} | @{y})} & \nonumber\\
	\red
	& (\outputp{x}{y} | @{y})\substn{\quotep{(\prefix{x}{y}{(@{y} | \outputp{x}{y})) | P}}}{y} & \nonumber\\
	=
	& \outputp{x}{\quotep{(\prefix{x}{y}{(\outputp{x}{y} | @{y})) | P}}}
	  | {(\prefix{x}{y}{(\outputp{x}{y} | @{y})) | P}} & \nonumber\\
	\red
	& \ldots & \nonumber\\
	\red^*
	& P | P | \ldots & \nonumber
\end{eqnarray}

Of course, this encoding, as an implementation, runs away, unfolding
$\bangp{P}$ eagerly. A lazier and more implementable replication
operator, restricted to input-guarded processes, may be obtained as follows.

\begin{eqnarray}
\bangp{\prefix{u}{v}{P}} 
	:= 
	\binpar{\lift{x}{\prefix{u}{v}{(\binpar{D(x)}{P})}}}{D(x)} \nonumber
\end{eqnarray}

\begin{remark}
  Note that the lazier definition still does not deal with summation
  or mixed summation (i.e. sums over input and output). The reader is
  invited to construct definitions of replication that deal with these
  features. 

  Further, the definitions are parameterized in a name, $x$. Can you,
  gentle reader, make a definition that eliminates this parameter and
  guarantees no accidental interaction between the replication
  machinery and the process being replicated -- i.e. no accidental
  sharing of names used by the process to get its work done and the
  name(s) used by the replication to effect copying. This latter
  revision of the definition of replication is crucial to obtaining
  the expected identity $!!P \sim !P$.
\end{remark}

\begin{remark}\label{rem:paradoxical_combinator}
  The reader familiar with the lambda calculus will have noticed the
  similarity between $D$ and the paradoxical combinator.

  [Ed. note: the existence of this seems to suggest we have to be more
  restrictive on the set of processes and names we admit if we are to
  support no-cloning.]
\end{remark}

\subsubsection{Bisimulation}

The computational dynamics gives rise to another kind of equivalence,
the equivalence of computational behavior. As previously mentioned
this is typically captured \emph{via} some form of bisimulation.

% The notion we use in this paper is weak barbed bisimulation
% \cite{milner91polyadicpi}.

The notion we use in this paper is derived from weak barbed
bisimulation \cite{milner91polyadicpi}. 

\begin{definition}
An \emph{observation relation}, $\downarrow_{\mathcal N}$, over a set
of names, $\mathcal N$, is the smallest relation satisfying the rules
below.

\infrule[Out-barb]{y \in {\mathcal N}, \; x \nameeq y}
		  {\outputp{x}{v} \downarrow_{\mathcal N} x}
\infrule[Par-barb]{\mbox{$P\downarrow_{\mathcal N} x$ or $Q\downarrow_{\mathcal N} x$}}
		  {\binpar{P}{Q} \downarrow_{\mathcal N} x}

We write $P \Downarrow_{\mathcal N} x$ if there is $Q$ such that 
$P \wred Q$ and $Q \downarrow_{\mathcal N} x$.
\end{definition}

\begin{definition}
%\label{def.bbisim}
An  ${\mathcal N}$-\emph{barbed bisimulation} over a set of names, ${\mathcal N}$, is a symmetric binary relation 
${\mathcal S}_{\mathcal N}$ between agents such that $P\rel{S}_{\mathcal N}Q$ implies:
\begin{enumerate}
\item If $P \red P'$ then $Q \wred Q'$ and $P'\rel{S}_{\mathcal N} Q'$.
\item If $P\downarrow_{\mathcal N} x$, then $Q\Downarrow_{\mathcal N} x$.
\end{enumerate}
$P$ is ${\mathcal N}$-barbed bisimilar to $Q$, written
$P \wbbisim_{\mathcal N} Q$, if $P \rel{S}_{\mathcal N} Q$ for some ${\mathcal N}$-barbed bisimulation ${\mathcal S}_{\mathcal N}$.
\end{definition}

$\mathcal{R} \subseteq \pi \times \pi$

$P \mathcal{R} Q => \forall P'. P \red P' \Rightarrow \exists Q'. Q \red Q', P' \mathcal{R} Q'$

$P \vdash x \Rightarrow Q \vdash x$

\begin{mathpar}
  \inferrule*[lab=Out-barb]{x \nameeq y}{{y}!\langle{Q}\rangle \vdash x}
  \and
  \inferrule*[lab=Par-barb]{\mbox{$P\vdash x$ or $Q\vdash x$}}{\binpar{P}{Q} \vdash x}
\end{mathpar}

\subsubsection{Contexts}

One of the principle advantages of computational calculi like the
$\pi$-calculus is a well-defined notion of context,
contextual-equivalence and a correlation between
contextual-equivalence and notions of bisimulation. The notion of
context allows the decomposition of a process into (sub-)process and
its syntactic environment, its context. Thus, a context may be
thought of as a process with a ``hole'' (written $\Box$) in it. The
application of a context $M$ to a process $P$, written $M[P]$, is
tantamount to filling the hole in $M$ with $P$. In this paper we do
not need the full weight of this theory, but do make use of the notion
of context in the proof the main theorem. 

\begin{mathpar}
  \inferrule* [lab=summation] {} {{M_{M},M_{N}} \bc \Box \;|\; x.M_{A} \;|\; M_{M}+M_{N}}
  \and
  \inferrule* [lab=agent] {} {{M_{A}} \bc (\vec{x})M_{P} \;| \; \clift{P_0,\ldots,M_{P},\ldots,P_N}}
  \and \\
  \inferrule* [lab=process] {} {{M_{P}} \bc M_{N} \;| \;P|M_{P} }
\end{mathpar} 

\begin{mathpar}
  \inferrule* [lab=sychronization] {} {M_{N} \bc \Box \;|\; x?M_{F} \;|\; x!M_{C}}
  \and
  \inferrule* [lab=abstraction] {} {{M_{F}} \bc (x)M_{P} }
  \and
  \inferrule* [lab=concretion] {} {{M_{C}} \bc \langle M_{P} \rangle }
  \and \\
  \inferrule* [lab=process] {} {{M_{P}} \bc M_{N} \;| \;P|M_{P} }
\end{mathpar}

\begin{definition}[contextual application] Given a context $M$, and
  process $P$, we define the \emph{contextual application}, $M[P] :=
  M\{P/\Box\}$. That is, the contextual application of M to P is the
  substitution of $P$ for $\Box$ in $M$.
\end{definition}

$\meaningof{-} : L \to \mathcal{P}(\pi)$

\begin{mathpar}
  \inferrule* [lab=collection] {} {\meaningof{true} = \pi, \and \meaningof{~E} = \pi \setminus \meaningof{E}, \and \meaningof{E_{1} \& E_{2}} = \meaningof{E_{1}} \cap \meaningof{E_{2}}}
\end{mathpar}

\begin{mathpar}
  \inferrule* [lab=structure] {} {\meaningof{0} = \{ P \in \pi | P \equiv 0 \}, \and \\ \meaningof{E_1 | E_2} = \{ P \in \pi | P \equiv P_{1} | P_{2}, P_{1} \in \meaningof{E_{1}}, P_{2} \in \meaningof{E_2}\} }
\end{mathpar}

\begin{mathpar}
 \inferrule* [lab=behavior] {} {\meaningof{\langle a?b \rangle E} = \{ P \in \pi | P \equiv Q | u?(y)P', \\ \and \\\\ \and \\ \;\;\; u \in \meaningof{a}, \forall z.P'\{z/y\} \in \meaningof{E\{z/b\}}\}, \and \\ \meaningof{a!E} = \{ P \in \pi | P \equiv Q | x!\langle P' \rangle, x \in \meaningof{a} P' \in \meaningof{E}\} }
\end{mathpar}

\begin{mathpar}
 \inferrule* [lab=nominal] {} {\meaningof{\quotep{E}} = \{ \quotep{P} \in \quotep{\pi} | P \in \meaningof{E} \}, \and \meaningof{\quotep{P}} = \{ \quotep{Q} \in \quotep{\pi} | P \equiv Q \} \and \\ \meaningof{@\quotep{E}} = \{ P \in \pi | P \equiv @x, x \in \meaningof{E} \}}
\end{mathpar}

\begin{eqnarray*}
  \\
  \meaningof{-} : TS \to ST
\end{eqnarray*}

\begin{eqnarray*}
  \\
  L : TS \to ST
\end{eqnarray*}

\begin{eqnarray*}
  \\
  P \models E \iff P \in \meaningof{E}
\end{eqnarray*}

\begin{eqnarray*}
  P \approx_{L} Q \iff \forall E \in L. P \models E \iff Q \models E
\end{eqnarray*}

\begin{eqnarray*}
  P \approx_{K} Q
\end{eqnarray*}

\begin{eqnarray*}
  P \approx Q
\end{eqnarray*}

$\approx_{K} = \approx = \approx_{L}$

\subsubsection{Contextual duality}

Note that contexts extend the quotation operation to a family of
operations from processes to names. Given a context, $M$, we can
define a \emph{nominal context}, $\quotep{M}$ by $\quotep{M}[P] :=
\quotep{M[P]}$. To foreshadow what is to come we observe that these
operations enjoy a duality with processes very much like the duality
between vectors and maps from vectors to scalars.

Further, because the calculus is essentially higher-order, we have a
correspondence between contexts and processes. More specifically,
given a name $x$ and a context $M$ we can construct $M^{*}_{x}$ such
that 

\begin{mathpar}
  M^{*}_{x} | \lift{x}{P} \red M[P]
\end{mathpar}

namely,

\begin{mathpar}
  M^{*}_{x} := x?(u).M[\dropn{u}]
\end{mathpar}

The dependence of $M^{*}_{x}$ on a name makes it an abstraction, 

\begin{mathpar}
  M^{*} := (x)x?(u).M[\dropn{u}]
\end{mathpar}

\subsection{Additional notation}

It will sometimes be convenient to denote the process a name
quotes. We already have the notation $x = \quotep{P}$, but it will be
convenient to introduce an alternate notation, $\procn{x}$, when we
want to emphasize the connection to the use of the name. Note that, by
virtue of name equivalence, $\quotep{\procn{x}} \nameeq x$; so, the
notation is consistent with previous definitions.

Further, because names have structure it is possible to effect
substitutions on the basis of that structure. This means we need to
upgrade our notation for substitutions, which we accomplish by
adapting comprehension notation. Thus,

\begin{mathpar}
  P\{ y / x : x \in S \}
\end{mathpar}

is interpreted to mean the process derived from P by replacing (in a
capture-avoiding manner) each occurrence of $x$ in $S$ by $y$. For example,

\begin{mathpar}
  P\{ \quotep{\procn{x}|\procn{x}} / x : x \in \freenames{P} \}
\end{mathpar}

will replace each (occurrence) of a free name $x$ in $P$ by
$\quotep{\procn{x}|\procn{x}}$.

Also, we will avail ourselves of the notation $x^{L}$ and $x^{R}$ to
denote injections of a name into disjoint copies of the name
space. There are numerous ways to accomplish this. One example can be
found in \cite{MeredithR05}. This notation overloads to vectors of
names: $\vec{x}^{\pi} := (x_{i}^{\pi} \; : \; 0 \leq i < |\vec{x}| )$ where $\pi \in \{L,R\}$.

We also use $P^{\Box} := P|\Box$.

In \cite{MeredithR05} an interpretation of the new operator is
given. It turns out that there are several possible interpretations
all enjoying the requisite algebraic properties of the operator (see
\cite{milner91polyadicpi}). We will therefore make liberal use of
$(\nu\; \vec{x})P$.

% subsection the_syntax_and_semantics_of_the_notation_system (end)   

\input{qm2pi.qmops} 

\input{qm2pi.sterngerlach} 

\input{qm2pi.metric} 

% section concurrent_process_calculi (end)

%\input{qm2pi.proofsketch}

% section proof sketch (end)

%\input{qm2pi.slviaknots} 

% section spatial logic via knots (end)

\input{qm2pi.conclusion}

% section conclusion (end)

%\input{qm2pi.dtcodes} 

% section wiring algorithm (end)

\input{qm2pi.ack} 

% section acknowledgments (end)

\newpage


\bibliographystyle{plain}   
\bibliography{../../biblios/main.bib}

\input{qm2pi.rhodetails}

\end{document}

 

\documentclass[12pt]{llncs}
%\documentclass{jktr}

\usepackage[pdftex]{hyperref}                   
\usepackage {listings}
\usepackage {mathpartir}
\usepackage{bcprules}
%\usepackage{listings}
                       
\usepackage{graphicx} 
%\usepackage[margins=2.5cm,nohead,nofoot]{geometry}
%\usepackage{geometry}
\usepackage{amsfonts}
\usepackage{amstext}
\usepackage{latexsym}
\usepackage{amssymb}
\usepackage{color}


%\include{myPreamble}
\include{qm2pi.local} 

%\ifpdf
%\usepackage[pdftex]{graphicx}
%\else
%\usepackage{graphicx}
%\fi

 % \ifpdf
%  \usepackage{pdfsync}
%  \if


%\title{Brief Article}
%\author{David F. Snyder}
%\author{L.G. Meredith}

%\address{Dept. of Math., Texas State University--San Marcos, San Marcos, TX 78666}
       
\pagestyle{empty}


\begin{document}

\lstset{language=[Objective]Caml,frame=shadowbox}

\input{qm2pi.front}

% section front matter (end)

\input{qm2pi.intro} 
 
% section introduction (end)

% \input{qm2pi.knotations} 

% section notation (end)

\input{qm2pi.process.calculi} 

% section concurrent_process_calculi_and_spatial_logics_ (end)
    
%\input{qm2pi.knots2pi} 

%\input{qm2pi.trefoil} 

%\input{qm2pi.mainthm} 

% subsection basic_interpretation (end)

%\input{qm2pi.rho.presentation} 
\subsection{The syntax and semantics of the notation system}\label{sub:the_syntax_and_semantics_of_the_notation_system} % (fold)

We now summarize a technical presentation of the calculus that
embodies our theory of dynamics. The typical presentation of such a
calculus follows the style of giving generators and relations on
them. The grammar, below, describing term constructors, freely
generates the set of processes, $\Proc$. This set is then quotiented
by a relation known as structural congruence and it is over this set
that the notion of dynamics is expressed. This presentation is
essentially that of \cite{MeredithR05} with the addition of
polyadicity and summation. For readability we have relegated some of
the technical subtleties to an appendix.

\subsubsection{Process grammar}\label{subsub:process_grammar}

\begin{mathpar}
  \inferrule* [lab=synchronization] {} {{M} \bc \pzero \;|\; x?F \;|\; x!C }
  \and
  \inferrule* [lab=abstraction] {} {{F} \bc (x)P}
  \and
  \inferrule* [lab=concretion] {} {{C} \bc \langle Q \rangle}
  \and
  \inferrule* [lab=process] {} {{P,Q} \bc M \;| \;P|Q \;|\; @{x}}
  \and
  \inferrule* [lab=name] {} {{x} \bc \quotep{P}}
\end{mathpar} 

Note that $\vec{x}$ (resp. $\vec{P}$) denotes a vector of names
(resp. processes) of length $|\vec{x}|$ (resp. $|\vec{P}|$). We adopt
the following useful abbreviations.

\begin{mathpar}
   x?(\vec{y}).P := x.(\vec{y})P \and  x\clift{\vec{P}} := x.\clift{\vec{P}}
   \and x!(y) := \lift{x}{\dropn{y}}
   \and \Pi_{i=0}^{n-1}P_i := P_0 | \ldots | P_{n-1}
\end{mathpar}

\subsubsection{Structural congruence}

\paragraph{Free and bound names and alpha-equivalence.} At the
core of structural equivalence is alpha-equivalence which identifies
process that are the same up to a change of variable. Formally, we
recognize the distinction between free and bound names. The free names
of a process, $\freenames{P}$, may be calculated recursively as
follows:

\begin{mathpar}
\freenames{\pzero} := \emptyset
  \and \\
  \freenames{x?(y).P} := \{ x \} \cup (\freenames{P} \setminus \{ y \})
  \and 
  \freenames{x!\langle P \rangle} := \{ x \} \cup \{ P \} 
  \and \\
  \freenames{P|Q} := \freenames{P} \cup \freenames{Q}
  \and \\
  \freenames{@{x}} := \{ x \}
\end{mathpar}

$\pi$
$\quotep{\pi}$

$\freenames{-} : \pi \to \mathcal{P}(\quotep{\pi})$

\begin{eqnarray*}
  \freenames{\pzero} & := & \emptyset \\
  \freenames{x?(y).P} & := & \{ x \} \cup (\freenames{P} \setminus \{ y \}) \\
  \freenames{x!\langle P \rangle} & := & \{ x \} \cup \{ P \} \\
  \freenames{P|Q} & := & \freenames{P} \cup \freenames{Q} \\
  \freenames{\dropn{x}} & := & \{ x \}
\end{eqnarray*}

The bound names of a process, $\boundnames{P}$, are those names occurring in $P$
that are not free. For example, in $x?(y).0$, the name $x$ is free, while $y$ is bound.

\begin{mathpar}
  \inferrule* [lab=monoidal-laws] {} { P|Q \equiv Q|P \and P|0 \equiv P \and P|(Q|R) \equiv (P|Q)|R }
\end{mathpar}

\begin{mathpar}
  \inferrule* [lab=alpha-equivalence] {} { (x)P \equiv (y)P\{y/x\} \and y \not\in \freenames{P} }
\end{mathpar}

\begin{definition}
Then two processes, $P,Q$, are alpha-equivalent if $P = Q\{\vec{y}/\vec{x}\}$ for
some $\vec{x} \in \boundnames{Q},\vec{y} \in \boundnames{P}$, where $Q\{\vec{y}/\vec{x}\}$
denotes the capture-avoiding substitution of $\vec{y}$ for $\vec{x}$ in $Q$.
\end{definition}

\begin{definition}
  The {\em structural congruence} \cite{SangiorgiWalker} , $\equiv$,
  between processes is the least congruence containing
  alpha-equivalence, satisfying the abelian monoid laws
  (associativity, commutativity and $\pzero$ as identity) for parallel
  composition $|$ and for summation $+$.
\end{definition}

\subsection{Name equivalence}

We take name equivalence, written $\nameeq$, to be the smallest
equivalence relation generated by the following rules.

\begin{mathpar}
\inferrule*[lab=Quote-drop]
{ }
{ \quotep{@{x}} \nameeq x }

\inferrule*[lab=Struct-equiv]
{ P \scong Q }
{ \quotep{P} \nameeq \quotep{Q} }
\end{mathpar}

The astute reader will have noticed that the mutual recursion of names
and processes imposes a mutual recursion on alpha-equivalence and
structural equivalence via name-equivalence. Fortunately, all of this
works out pleasantly and we may calculate in the natural way, free of
concern. The reader interested in the details is referred to the
appendix \ref{appendix:rho_details}.

\subsection{Substitution}

We use $\Proc$ for the set of processes, $\QProc$ for the set of
names, and $\id{\{}\vec{y} / \vec{x} \id{\}}$ to denote partial maps,
$s : \QProc \rightarrow \QProc$. A map, $s$ lifts, uniquely, to a map
on process terms, $\widehat{s} : \Proc \rightarrow \Proc$ by the
following equations.

\begin{mathpar}
  (0) \psubstp{Q}{P} := 0 \\
  (R \juxtap S) \psubstp{Q}{P}
  :=    
  (R)\psubstp{Q}{P} \juxtap (S) \psubstp{Q}{P} \\
  (x?(y).R) \psubstp{Q}{P}    
  :=    
  (x)\substp{Q}{P} (z)\concat( (R \psubstn{z}{y}) \psubstp{Q}{P} ) \\
  (\lift{x}{R}) \psubstp{Q}{P}  
  :=
  \lift{(x)\substp{Q}{P}}{ R \psubstp{Q}{P} } \\
%   (\dropn{x})  \psubstp{Q}{P}       
%   := 
%   \left\{ 
%     \begin{array}{ccc} 
%       \dropn{\quotep{Q}} & & x \nameeq \quotep{P} \\
%       \dropn{x} & & otherwise \\
%     \end{array}
%   \right. 
  (\dropn{x})  \psubstp{Q}{P}       
  := 
  \left\{ 
    \begin{array}{ccc} 
      Q & & x \nameeq \quotep{P} \\
      \dropn{x} & & otherwise \\
    \end{array}
  \right.
\end{mathpar}
 

where

\begin{eqnarray}
  (x)\id{\{} \lpquote Q \rpquote / \lpquote P \rpquote \id{\}}            = 
  \left\{ 
    \begin{array}{ccc}
      \lpquote Q \rpquote & & x \nameeq \lpquote P \rpquote \\
      x & & otherwise \\
    \end{array}
  \right. \nonumber
\end{eqnarray}

and $z$ is chosen distinct from $\quotep{P}$, $\quotep{Q}$, the free
names in $Q$, and all the names in $R$. Our $\alpha$-equivalence will
be built in the standard way from this substitution.

\begin{remark}\label{rem:no_self_referential_names}
  One consequence of these definitions is that $\forall P. \quotep{P}
  \not\in \freenames{P}$.
\end{remark}

\subsection{ Dynamic quote: an example }

Anticipating something of what's to come, consider applying the
substitution, $\widehat{\id{\{}u / z \id{\}}}$, to the following pair
of processes, $\lift{w}{y!(z)}$ and $w[ \lpquote y!(z) \rpquote ]$.

\begin{eqnarray}
	\lift{w}{y!(z)}\widehat{\id{\{}u / z \id{\}}}
		& = &
		\lift{w}{y!(u)} \nonumber\\
	w[ \lpquote y!(z) \rpquote ] \widehat{ \id{\{}u / z \id{\}} }
		& = &
		w[ \lpquote y!(z) \rpquote ] \nonumber
\end{eqnarray}

Because the body of the process between quotes is impervious to
substitution, we get radically different answers. In fact, by
examining the first process in an input context,
e.g. $x?(z).\lift{w}{y!(z)}$, we see that the process under the lift
operator may be shaped by prefixed inputs binding a name inside it. In
this sense, the lift operator will be seen as a way to dynamically
construct processes before reifying them as names.

Finally equipped with these standard features we can present the
dynamics of the calculus.

\subsubsection{Operational semantics} 

Finally, we introduce the computational dynamics. What marks these
algebras as distinct from other more traditionally studied algebraic
structures, e.g. vector spaces or polynomial rings, is the manner in
which dynamics is captured. In traditional structures, dynamics is typically
expressed through morphisms between such structures, as in linear maps
between vector spaces or morphisms between rings. In algebras
associated with the semantics of computation, the dynamics is
expressed as part of the algebraic structure itself, through a
reduction reduction relation typically denoted by $\red$. Below, we
give a recursive presentation of this relation for the calculus used
in the encoding.

$\red \subseteq \pi \times \pi$
$\red : \pi \to \mathcal{P}(\pi)$

\begin{mathpar}
  \inferrule* [lab=Comm] { \textsf{match}( x_{src}, x_{trgt} ) } { x_{trgt}?(y)P \; | \; x_{src}!\langle {Q} \rangle \red P\{\quotep{Q}/y}\} }
  \and \\
  \inferrule* [lab=Par] {{P} \red {P}'} {{{P} | {Q}} \red {{P}' | {Q}}}
  \and
  \inferrule* [lab=Equiv]{{{P} \scong {P}'} \andalso {{P}' \red {Q}'} \andalso {{Q}' \scong {Q}}}{{P} \red {Q}}
\end{mathpar}

\begin{eqnarray*}
  match_{\equiv} (\quotep{P},\quotep{Q}) & := & P \equiv Q \\
  match_{\dagger}(\quotep{P},\quotep{Q}) & := & \forall R. P|Q \red^{*} R => R \red^{*} 0 \\
  match_{K}(\quotep{P},\quotep{Q}) & := & K \mbox{ for some context } K
\end{eqnarray*}

$u?(x)P | u!\langle Q \rangle \red P\{\quotep{Q}/x\}$

%We write $\wred$ for $\red^*$, and $P\red$ if $\exists Q $ such that $ P \red Q$.
We write $P\red$ if $\exists Q $ such that $ P \red Q$ and $P\not\red$, otherwise.

\section{Replication}

As mentioned before, it is known that replication (and hence
recursion) can be implemented in a higher-order process algebra
\cite{SangiorgiWalker}. As our first example of calculation with the
machinery thus far presented we give the construction explicitly in
the {\rhoc}.

\begin{eqnarray}
	D_{x} & := & \prefix{x}{y}{(\binpar{\outputp{x}{y}}{@{y}})} \nonumber\\
	\bangp_{x}{P} & := & \binpar{{x}!\langle{\binpar{D_{x}}{P}}\rangle}{D_{x}} \nonumber
\end{eqnarray}

\begin{eqnarray}
	\bangp_{x}{P} & & \nonumber\\
	=
	& {x}!\langle{(\prefix{x}{y}{(\outputp{x}{y} | @{y})) | P}}\rangle 
	      | \prefix{x}{y}{(\outputp{x}{y} | @{y})} & \nonumber\\
	\red
	& (\outputp{x}{y} | @{y})\substn{\quotep{(\prefix{x}{y}{(@{y} | \outputp{x}{y})) | P}}}{y} & \nonumber\\
	=
	& \outputp{x}{\quotep{(\prefix{x}{y}{(\outputp{x}{y} | @{y})) | P}}}
	  | {(\prefix{x}{y}{(\outputp{x}{y} | @{y})) | P}} & \nonumber\\
	\red
	& \ldots & \nonumber\\
	\red^*
	& P | P | \ldots & \nonumber
\end{eqnarray}

Of course, this encoding, as an implementation, runs away, unfolding
$\bangp{P}$ eagerly. A lazier and more implementable replication
operator, restricted to input-guarded processes, may be obtained as follows.

\begin{eqnarray}
\bangp{\prefix{u}{v}{P}} 
	:= 
	\binpar{\lift{x}{\prefix{u}{v}{(\binpar{D(x)}{P})}}}{D(x)} \nonumber
\end{eqnarray}

\begin{remark}
  Note that the lazier definition still does not deal with summation
  or mixed summation (i.e. sums over input and output). The reader is
  invited to construct definitions of replication that deal with these
  features. 

  Further, the definitions are parameterized in a name, $x$. Can you,
  gentle reader, make a definition that eliminates this parameter and
  guarantees no accidental interaction between the replication
  machinery and the process being replicated -- i.e. no accidental
  sharing of names used by the process to get its work done and the
  name(s) used by the replication to effect copying. This latter
  revision of the definition of replication is crucial to obtaining
  the expected identity $!!P \sim !P$.
\end{remark}

\begin{remark}\label{rem:paradoxical_combinator}
  The reader familiar with the lambda calculus will have noticed the
  similarity between $D$ and the paradoxical combinator.

  [Ed. note: the existence of this seems to suggest we have to be more
  restrictive on the set of processes and names we admit if we are to
  support no-cloning.]
\end{remark}

\subsubsection{Bisimulation}

The computational dynamics gives rise to another kind of equivalence,
the equivalence of computational behavior. As previously mentioned
this is typically captured \emph{via} some form of bisimulation.

% The notion we use in this paper is weak barbed bisimulation
% \cite{milner91polyadicpi}.

The notion we use in this paper is derived from weak barbed
bisimulation \cite{milner91polyadicpi}. 

\begin{definition}
An \emph{observation relation}, $\downarrow_{\mathcal N}$, over a set
of names, $\mathcal N$, is the smallest relation satisfying the rules
below.

\infrule[Out-barb]{y \in {\mathcal N}, \; x \nameeq y}
		  {\outputp{x}{v} \downarrow_{\mathcal N} x}
\infrule[Par-barb]{\mbox{$P\downarrow_{\mathcal N} x$ or $Q\downarrow_{\mathcal N} x$}}
		  {\binpar{P}{Q} \downarrow_{\mathcal N} x}

We write $P \Downarrow_{\mathcal N} x$ if there is $Q$ such that 
$P \wred Q$ and $Q \downarrow_{\mathcal N} x$.
\end{definition}

\begin{definition}
%\label{def.bbisim}
An  ${\mathcal N}$-\emph{barbed bisimulation} over a set of names, ${\mathcal N}$, is a symmetric binary relation 
${\mathcal S}_{\mathcal N}$ between agents such that $P\rel{S}_{\mathcal N}Q$ implies:
\begin{enumerate}
\item If $P \red P'$ then $Q \wred Q'$ and $P'\rel{S}_{\mathcal N} Q'$.
\item If $P\downarrow_{\mathcal N} x$, then $Q\Downarrow_{\mathcal N} x$.
\end{enumerate}
$P$ is ${\mathcal N}$-barbed bisimilar to $Q$, written
$P \wbbisim_{\mathcal N} Q$, if $P \rel{S}_{\mathcal N} Q$ for some ${\mathcal N}$-barbed bisimulation ${\mathcal S}_{\mathcal N}$.
\end{definition}

$\mathcal{R} \subseteq \pi \times \pi$

$P \mathcal{R} Q => \forall P'. P \red P' \Rightarrow \exists Q'. Q \red Q', P' \mathcal{R} Q'$

$P \vdash x \Rightarrow Q \vdash x$

\begin{mathpar}
  \inferrule*[lab=Out-barb]{x \nameeq y}{{y}!\langle{Q}\rangle \vdash x}
  \and
  \inferrule*[lab=Par-barb]{\mbox{$P\vdash x$ or $Q\vdash x$}}{\binpar{P}{Q} \vdash x}
\end{mathpar}

\subsubsection{Contexts}

One of the principle advantages of computational calculi like the
$\pi$-calculus is a well-defined notion of context,
contextual-equivalence and a correlation between
contextual-equivalence and notions of bisimulation. The notion of
context allows the decomposition of a process into (sub-)process and
its syntactic environment, its context. Thus, a context may be
thought of as a process with a ``hole'' (written $\Box$) in it. The
application of a context $M$ to a process $P$, written $M[P]$, is
tantamount to filling the hole in $M$ with $P$. In this paper we do
not need the full weight of this theory, but do make use of the notion
of context in the proof the main theorem. 

\begin{mathpar}
  \inferrule* [lab=summation] {} {{M_{M},M_{N}} \bc \Box \;|\; x.M_{A} \;|\; M_{M}+M_{N}}
  \and
  \inferrule* [lab=agent] {} {{M_{A}} \bc (\vec{x})M_{P} \;| \; \clift{P_0,\ldots,M_{P},\ldots,P_N}}
  \and \\
  \inferrule* [lab=process] {} {{M_{P}} \bc M_{N} \;| \;P|M_{P} }
\end{mathpar} 

\begin{mathpar}
  \inferrule* [lab=sychronization] {} {M_{N} \bc \Box \;|\; x?M_{F} \;|\; x!M_{C}}
  \and
  \inferrule* [lab=abstraction] {} {{M_{F}} \bc (x)M_{P} }
  \and
  \inferrule* [lab=concretion] {} {{M_{C}} \bc \langle M_{P} \rangle }
  \and \\
  \inferrule* [lab=process] {} {{M_{P}} \bc M_{N} \;| \;P|M_{P} }
\end{mathpar}

\begin{definition}[contextual application] Given a context $M$, and
  process $P$, we define the \emph{contextual application}, $M[P] :=
  M\{P/\Box\}$. That is, the contextual application of M to P is the
  substitution of $P$ for $\Box$ in $M$.
\end{definition}

$\meaningof{-} : L \to \mathcal{P}(\pi)$

\begin{mathpar}
  \inferrule* [lab=collection] {} {\meaningof{true} = \pi, \and \meaningof{~E} = \pi \setminus \meaningof{E}, \and \meaningof{E_{1} \& E_{2}} = \meaningof{E_{1}} \cap \meaningof{E_{2}}}
\end{mathpar}

\begin{mathpar}
  \inferrule* [lab=structure] {} {\meaningof{0} = \{ P \in \pi | P \equiv 0 \}, \and \\ \meaningof{E_1 | E_2} = \{ P \in \pi | P \equiv P_{1} | P_{2}, P_{1} \in \meaningof{E_{1}}, P_{2} \in \meaningof{E_2}\} }
\end{mathpar}

\begin{mathpar}
 \inferrule* [lab=behavior] {} {\meaningof{\langle a?b \rangle E} = \{ P \in \pi | P \equiv Q | u?(y)P', \\ \and \\\\ \and \\ \;\;\; u \in \meaningof{a}, \forall z.P'\{z/y\} \in \meaningof{E\{z/b\}}\}, \and \\ \meaningof{a!E} = \{ P \in \pi | P \equiv Q | x!\langle P' \rangle, x \in \meaningof{a} P' \in \meaningof{E}\} }
\end{mathpar}

\begin{mathpar}
 \inferrule* [lab=nominal] {} {\meaningof{\quotep{E}} = \{ \quotep{P} \in \quotep{\pi} | P \in \meaningof{E} \}, \and \meaningof{\quotep{P}} = \{ \quotep{Q} \in \quotep{\pi} | P \equiv Q \} \and \\ \meaningof{@\quotep{E}} = \{ P \in \pi | P \equiv @x, x \in \meaningof{E} \}}
\end{mathpar}

\begin{eqnarray*}
  \\
  \meaningof{-} : TS \to ST
\end{eqnarray*}

\begin{eqnarray*}
  \\
  L : TS \to ST
\end{eqnarray*}

\begin{eqnarray*}
  \\
  P \models E \iff P \in \meaningof{E}
\end{eqnarray*}

\begin{eqnarray*}
  P \approx_{L} Q \iff \forall E \in L. P \models E \iff Q \models E
\end{eqnarray*}

\begin{eqnarray*}
  P \approx_{K} Q
\end{eqnarray*}

\begin{eqnarray*}
  P \approx Q
\end{eqnarray*}

$\approx_{K} = \approx = \approx_{L}$

\subsubsection{Contextual duality}

Note that contexts extend the quotation operation to a family of
operations from processes to names. Given a context, $M$, we can
define a \emph{nominal context}, $\quotep{M}$ by $\quotep{M}[P] :=
\quotep{M[P]}$. To foreshadow what is to come we observe that these
operations enjoy a duality with processes very much like the duality
between vectors and maps from vectors to scalars.

Further, because the calculus is essentially higher-order, we have a
correspondence between contexts and processes. More specifically,
given a name $x$ and a context $M$ we can construct $M^{*}_{x}$ such
that 

\begin{mathpar}
  M^{*}_{x} | \lift{x}{P} \red M[P]
\end{mathpar}

namely,

\begin{mathpar}
  M^{*}_{x} := x?(u).M[\dropn{u}]
\end{mathpar}

The dependence of $M^{*}_{x}$ on a name makes it an abstraction, 

\begin{mathpar}
  M^{*} := (x)x?(u).M[\dropn{u}]
\end{mathpar}

\subsection{Additional notation}

It will sometimes be convenient to denote the process a name
quotes. We already have the notation $x = \quotep{P}$, but it will be
convenient to introduce an alternate notation, $\procn{x}$, when we
want to emphasize the connection to the use of the name. Note that, by
virtue of name equivalence, $\quotep{\procn{x}} \nameeq x$; so, the
notation is consistent with previous definitions.

Further, because names have structure it is possible to effect
substitutions on the basis of that structure. This means we need to
upgrade our notation for substitutions, which we accomplish by
adapting comprehension notation. Thus,

\begin{mathpar}
  P\{ y / x : x \in S \}
\end{mathpar}

is interpreted to mean the process derived from P by replacing (in a
capture-avoiding manner) each occurrence of $x$ in $S$ by $y$. For example,

\begin{mathpar}
  P\{ \quotep{\procn{x}|\procn{x}} / x : x \in \freenames{P} \}
\end{mathpar}

will replace each (occurrence) of a free name $x$ in $P$ by
$\quotep{\procn{x}|\procn{x}}$.

Also, we will avail ourselves of the notation $x^{L}$ and $x^{R}$ to
denote injections of a name into disjoint copies of the name
space. There are numerous ways to accomplish this. One example can be
found in \cite{MeredithR05}. This notation overloads to vectors of
names: $\vec{x}^{\pi} := (x_{i}^{\pi} \; : \; 0 \leq i < |\vec{x}| )$ where $\pi \in \{L,R\}$.

We also use $P^{\Box} := P|\Box$.

In \cite{MeredithR05} an interpretation of the new operator is
given. It turns out that there are several possible interpretations
all enjoying the requisite algebraic properties of the operator (see
\cite{milner91polyadicpi}). We will therefore make liberal use of
$(\nu\; \vec{x})P$.

% subsection the_syntax_and_semantics_of_the_notation_system (end)   

\input{qm2pi.qmops} 

\input{qm2pi.sterngerlach} 

\input{qm2pi.metric} 

% section concurrent_process_calculi (end)

%\input{qm2pi.proofsketch}

% section proof sketch (end)

%\input{qm2pi.slviaknots} 

% section spatial logic via knots (end)

\input{qm2pi.conclusion}

% section conclusion (end)

%\input{qm2pi.dtcodes} 

% section wiring algorithm (end)

\input{qm2pi.ack} 

% section acknowledgments (end)

\newpage


\bibliographystyle{plain}   
\bibliography{../../biblios/main.bib}

\input{qm2pi.rhodetails}

\end{document}

 

% section concurrent_process_calculi (end)

%\documentclass[12pt]{llncs}
%\documentclass{jktr}

\usepackage[pdftex]{hyperref}                   
\usepackage {listings}
\usepackage {mathpartir}
\usepackage{bcprules}
%\usepackage{listings}
                       
\usepackage{graphicx} 
%\usepackage[margins=2.5cm,nohead,nofoot]{geometry}
%\usepackage{geometry}
\usepackage{amsfonts}
\usepackage{amstext}
\usepackage{latexsym}
\usepackage{amssymb}
\usepackage{color}


%\include{myPreamble}
\include{qm2pi.local} 

%\ifpdf
%\usepackage[pdftex]{graphicx}
%\else
%\usepackage{graphicx}
%\fi

 % \ifpdf
%  \usepackage{pdfsync}
%  \if


%\title{Brief Article}
%\author{David F. Snyder}
%\author{L.G. Meredith}

%\address{Dept. of Math., Texas State University--San Marcos, San Marcos, TX 78666}
       
\pagestyle{empty}


\begin{document}

\lstset{language=[Objective]Caml,frame=shadowbox}

\input{qm2pi.front}

% section front matter (end)

\input{qm2pi.intro} 
 
% section introduction (end)

% \input{qm2pi.knotations} 

% section notation (end)

\input{qm2pi.process.calculi} 

% section concurrent_process_calculi_and_spatial_logics_ (end)
    
%\input{qm2pi.knots2pi} 

%\input{qm2pi.trefoil} 

%\input{qm2pi.mainthm} 

% subsection basic_interpretation (end)

%\input{qm2pi.rho.presentation} 
\subsection{The syntax and semantics of the notation system}\label{sub:the_syntax_and_semantics_of_the_notation_system} % (fold)

We now summarize a technical presentation of the calculus that
embodies our theory of dynamics. The typical presentation of such a
calculus follows the style of giving generators and relations on
them. The grammar, below, describing term constructors, freely
generates the set of processes, $\Proc$. This set is then quotiented
by a relation known as structural congruence and it is over this set
that the notion of dynamics is expressed. This presentation is
essentially that of \cite{MeredithR05} with the addition of
polyadicity and summation. For readability we have relegated some of
the technical subtleties to an appendix.

\subsubsection{Process grammar}\label{subsub:process_grammar}

\begin{mathpar}
  \inferrule* [lab=synchronization] {} {{M} \bc \pzero \;|\; x?F \;|\; x!C }
  \and
  \inferrule* [lab=abstraction] {} {{F} \bc (x)P}
  \and
  \inferrule* [lab=concretion] {} {{C} \bc \langle Q \rangle}
  \and
  \inferrule* [lab=process] {} {{P,Q} \bc M \;| \;P|Q \;|\; @{x}}
  \and
  \inferrule* [lab=name] {} {{x} \bc \quotep{P}}
\end{mathpar} 

Note that $\vec{x}$ (resp. $\vec{P}$) denotes a vector of names
(resp. processes) of length $|\vec{x}|$ (resp. $|\vec{P}|$). We adopt
the following useful abbreviations.

\begin{mathpar}
   x?(\vec{y}).P := x.(\vec{y})P \and  x\clift{\vec{P}} := x.\clift{\vec{P}}
   \and x!(y) := \lift{x}{\dropn{y}}
   \and \Pi_{i=0}^{n-1}P_i := P_0 | \ldots | P_{n-1}
\end{mathpar}

\subsubsection{Structural congruence}

\paragraph{Free and bound names and alpha-equivalence.} At the
core of structural equivalence is alpha-equivalence which identifies
process that are the same up to a change of variable. Formally, we
recognize the distinction between free and bound names. The free names
of a process, $\freenames{P}$, may be calculated recursively as
follows:

\begin{mathpar}
\freenames{\pzero} := \emptyset
  \and \\
  \freenames{x?(y).P} := \{ x \} \cup (\freenames{P} \setminus \{ y \})
  \and 
  \freenames{x!\langle P \rangle} := \{ x \} \cup \{ P \} 
  \and \\
  \freenames{P|Q} := \freenames{P} \cup \freenames{Q}
  \and \\
  \freenames{@{x}} := \{ x \}
\end{mathpar}

$\pi$
$\quotep{\pi}$

$\freenames{-} : \pi \to \mathcal{P}(\quotep{\pi})$

\begin{eqnarray*}
  \freenames{\pzero} & := & \emptyset \\
  \freenames{x?(y).P} & := & \{ x \} \cup (\freenames{P} \setminus \{ y \}) \\
  \freenames{x!\langle P \rangle} & := & \{ x \} \cup \{ P \} \\
  \freenames{P|Q} & := & \freenames{P} \cup \freenames{Q} \\
  \freenames{\dropn{x}} & := & \{ x \}
\end{eqnarray*}

The bound names of a process, $\boundnames{P}$, are those names occurring in $P$
that are not free. For example, in $x?(y).0$, the name $x$ is free, while $y$ is bound.

\begin{mathpar}
  \inferrule* [lab=monoidal-laws] {} { P|Q \equiv Q|P \and P|0 \equiv P \and P|(Q|R) \equiv (P|Q)|R }
\end{mathpar}

\begin{mathpar}
  \inferrule* [lab=alpha-equivalence] {} { (x)P \equiv (y)P\{y/x\} \and y \not\in \freenames{P} }
\end{mathpar}

\begin{definition}
Then two processes, $P,Q$, are alpha-equivalent if $P = Q\{\vec{y}/\vec{x}\}$ for
some $\vec{x} \in \boundnames{Q},\vec{y} \in \boundnames{P}$, where $Q\{\vec{y}/\vec{x}\}$
denotes the capture-avoiding substitution of $\vec{y}$ for $\vec{x}$ in $Q$.
\end{definition}

\begin{definition}
  The {\em structural congruence} \cite{SangiorgiWalker} , $\equiv$,
  between processes is the least congruence containing
  alpha-equivalence, satisfying the abelian monoid laws
  (associativity, commutativity and $\pzero$ as identity) for parallel
  composition $|$ and for summation $+$.
\end{definition}

\subsection{Name equivalence}

We take name equivalence, written $\nameeq$, to be the smallest
equivalence relation generated by the following rules.

\begin{mathpar}
\inferrule*[lab=Quote-drop]
{ }
{ \quotep{@{x}} \nameeq x }

\inferrule*[lab=Struct-equiv]
{ P \scong Q }
{ \quotep{P} \nameeq \quotep{Q} }
\end{mathpar}

The astute reader will have noticed that the mutual recursion of names
and processes imposes a mutual recursion on alpha-equivalence and
structural equivalence via name-equivalence. Fortunately, all of this
works out pleasantly and we may calculate in the natural way, free of
concern. The reader interested in the details is referred to the
appendix \ref{appendix:rho_details}.

\subsection{Substitution}

We use $\Proc$ for the set of processes, $\QProc$ for the set of
names, and $\id{\{}\vec{y} / \vec{x} \id{\}}$ to denote partial maps,
$s : \QProc \rightarrow \QProc$. A map, $s$ lifts, uniquely, to a map
on process terms, $\widehat{s} : \Proc \rightarrow \Proc$ by the
following equations.

\begin{mathpar}
  (0) \psubstp{Q}{P} := 0 \\
  (R \juxtap S) \psubstp{Q}{P}
  :=    
  (R)\psubstp{Q}{P} \juxtap (S) \psubstp{Q}{P} \\
  (x?(y).R) \psubstp{Q}{P}    
  :=    
  (x)\substp{Q}{P} (z)\concat( (R \psubstn{z}{y}) \psubstp{Q}{P} ) \\
  (\lift{x}{R}) \psubstp{Q}{P}  
  :=
  \lift{(x)\substp{Q}{P}}{ R \psubstp{Q}{P} } \\
%   (\dropn{x})  \psubstp{Q}{P}       
%   := 
%   \left\{ 
%     \begin{array}{ccc} 
%       \dropn{\quotep{Q}} & & x \nameeq \quotep{P} \\
%       \dropn{x} & & otherwise \\
%     \end{array}
%   \right. 
  (\dropn{x})  \psubstp{Q}{P}       
  := 
  \left\{ 
    \begin{array}{ccc} 
      Q & & x \nameeq \quotep{P} \\
      \dropn{x} & & otherwise \\
    \end{array}
  \right.
\end{mathpar}
 

where

\begin{eqnarray}
  (x)\id{\{} \lpquote Q \rpquote / \lpquote P \rpquote \id{\}}            = 
  \left\{ 
    \begin{array}{ccc}
      \lpquote Q \rpquote & & x \nameeq \lpquote P \rpquote \\
      x & & otherwise \\
    \end{array}
  \right. \nonumber
\end{eqnarray}

and $z$ is chosen distinct from $\quotep{P}$, $\quotep{Q}$, the free
names in $Q$, and all the names in $R$. Our $\alpha$-equivalence will
be built in the standard way from this substitution.

\begin{remark}\label{rem:no_self_referential_names}
  One consequence of these definitions is that $\forall P. \quotep{P}
  \not\in \freenames{P}$.
\end{remark}

\subsection{ Dynamic quote: an example }

Anticipating something of what's to come, consider applying the
substitution, $\widehat{\id{\{}u / z \id{\}}}$, to the following pair
of processes, $\lift{w}{y!(z)}$ and $w[ \lpquote y!(z) \rpquote ]$.

\begin{eqnarray}
	\lift{w}{y!(z)}\widehat{\id{\{}u / z \id{\}}}
		& = &
		\lift{w}{y!(u)} \nonumber\\
	w[ \lpquote y!(z) \rpquote ] \widehat{ \id{\{}u / z \id{\}} }
		& = &
		w[ \lpquote y!(z) \rpquote ] \nonumber
\end{eqnarray}

Because the body of the process between quotes is impervious to
substitution, we get radically different answers. In fact, by
examining the first process in an input context,
e.g. $x?(z).\lift{w}{y!(z)}$, we see that the process under the lift
operator may be shaped by prefixed inputs binding a name inside it. In
this sense, the lift operator will be seen as a way to dynamically
construct processes before reifying them as names.

Finally equipped with these standard features we can present the
dynamics of the calculus.

\subsubsection{Operational semantics} 

Finally, we introduce the computational dynamics. What marks these
algebras as distinct from other more traditionally studied algebraic
structures, e.g. vector spaces or polynomial rings, is the manner in
which dynamics is captured. In traditional structures, dynamics is typically
expressed through morphisms between such structures, as in linear maps
between vector spaces or morphisms between rings. In algebras
associated with the semantics of computation, the dynamics is
expressed as part of the algebraic structure itself, through a
reduction reduction relation typically denoted by $\red$. Below, we
give a recursive presentation of this relation for the calculus used
in the encoding.

$\red \subseteq \pi \times \pi$
$\red : \pi \to \mathcal{P}(\pi)$

\begin{mathpar}
  \inferrule* [lab=Comm] { \textsf{match}( x_{src}, x_{trgt} ) } { x_{trgt}?(y)P \; | \; x_{src}!\langle {Q} \rangle \red P\{\quotep{Q}/y}\} }
  \and \\
  \inferrule* [lab=Par] {{P} \red {P}'} {{{P} | {Q}} \red {{P}' | {Q}}}
  \and
  \inferrule* [lab=Equiv]{{{P} \scong {P}'} \andalso {{P}' \red {Q}'} \andalso {{Q}' \scong {Q}}}{{P} \red {Q}}
\end{mathpar}

\begin{eqnarray*}
  match_{\equiv} (\quotep{P},\quotep{Q}) & := & P \equiv Q \\
  match_{\dagger}(\quotep{P},\quotep{Q}) & := & \forall R. P|Q \red^{*} R => R \red^{*} 0 \\
  match_{K}(\quotep{P},\quotep{Q}) & := & K \mbox{ for some context } K
\end{eqnarray*}

$u?(x)P | u!\langle Q \rangle \red P\{\quotep{Q}/x\}$

%We write $\wred$ for $\red^*$, and $P\red$ if $\exists Q $ such that $ P \red Q$.
We write $P\red$ if $\exists Q $ such that $ P \red Q$ and $P\not\red$, otherwise.

\section{Replication}

As mentioned before, it is known that replication (and hence
recursion) can be implemented in a higher-order process algebra
\cite{SangiorgiWalker}. As our first example of calculation with the
machinery thus far presented we give the construction explicitly in
the {\rhoc}.

\begin{eqnarray}
	D_{x} & := & \prefix{x}{y}{(\binpar{\outputp{x}{y}}{@{y}})} \nonumber\\
	\bangp_{x}{P} & := & \binpar{{x}!\langle{\binpar{D_{x}}{P}}\rangle}{D_{x}} \nonumber
\end{eqnarray}

\begin{eqnarray}
	\bangp_{x}{P} & & \nonumber\\
	=
	& {x}!\langle{(\prefix{x}{y}{(\outputp{x}{y} | @{y})) | P}}\rangle 
	      | \prefix{x}{y}{(\outputp{x}{y} | @{y})} & \nonumber\\
	\red
	& (\outputp{x}{y} | @{y})\substn{\quotep{(\prefix{x}{y}{(@{y} | \outputp{x}{y})) | P}}}{y} & \nonumber\\
	=
	& \outputp{x}{\quotep{(\prefix{x}{y}{(\outputp{x}{y} | @{y})) | P}}}
	  | {(\prefix{x}{y}{(\outputp{x}{y} | @{y})) | P}} & \nonumber\\
	\red
	& \ldots & \nonumber\\
	\red^*
	& P | P | \ldots & \nonumber
\end{eqnarray}

Of course, this encoding, as an implementation, runs away, unfolding
$\bangp{P}$ eagerly. A lazier and more implementable replication
operator, restricted to input-guarded processes, may be obtained as follows.

\begin{eqnarray}
\bangp{\prefix{u}{v}{P}} 
	:= 
	\binpar{\lift{x}{\prefix{u}{v}{(\binpar{D(x)}{P})}}}{D(x)} \nonumber
\end{eqnarray}

\begin{remark}
  Note that the lazier definition still does not deal with summation
  or mixed summation (i.e. sums over input and output). The reader is
  invited to construct definitions of replication that deal with these
  features. 

  Further, the definitions are parameterized in a name, $x$. Can you,
  gentle reader, make a definition that eliminates this parameter and
  guarantees no accidental interaction between the replication
  machinery and the process being replicated -- i.e. no accidental
  sharing of names used by the process to get its work done and the
  name(s) used by the replication to effect copying. This latter
  revision of the definition of replication is crucial to obtaining
  the expected identity $!!P \sim !P$.
\end{remark}

\begin{remark}\label{rem:paradoxical_combinator}
  The reader familiar with the lambda calculus will have noticed the
  similarity between $D$ and the paradoxical combinator.

  [Ed. note: the existence of this seems to suggest we have to be more
  restrictive on the set of processes and names we admit if we are to
  support no-cloning.]
\end{remark}

\subsubsection{Bisimulation}

The computational dynamics gives rise to another kind of equivalence,
the equivalence of computational behavior. As previously mentioned
this is typically captured \emph{via} some form of bisimulation.

% The notion we use in this paper is weak barbed bisimulation
% \cite{milner91polyadicpi}.

The notion we use in this paper is derived from weak barbed
bisimulation \cite{milner91polyadicpi}. 

\begin{definition}
An \emph{observation relation}, $\downarrow_{\mathcal N}$, over a set
of names, $\mathcal N$, is the smallest relation satisfying the rules
below.

\infrule[Out-barb]{y \in {\mathcal N}, \; x \nameeq y}
		  {\outputp{x}{v} \downarrow_{\mathcal N} x}
\infrule[Par-barb]{\mbox{$P\downarrow_{\mathcal N} x$ or $Q\downarrow_{\mathcal N} x$}}
		  {\binpar{P}{Q} \downarrow_{\mathcal N} x}

We write $P \Downarrow_{\mathcal N} x$ if there is $Q$ such that 
$P \wred Q$ and $Q \downarrow_{\mathcal N} x$.
\end{definition}

\begin{definition}
%\label{def.bbisim}
An  ${\mathcal N}$-\emph{barbed bisimulation} over a set of names, ${\mathcal N}$, is a symmetric binary relation 
${\mathcal S}_{\mathcal N}$ between agents such that $P\rel{S}_{\mathcal N}Q$ implies:
\begin{enumerate}
\item If $P \red P'$ then $Q \wred Q'$ and $P'\rel{S}_{\mathcal N} Q'$.
\item If $P\downarrow_{\mathcal N} x$, then $Q\Downarrow_{\mathcal N} x$.
\end{enumerate}
$P$ is ${\mathcal N}$-barbed bisimilar to $Q$, written
$P \wbbisim_{\mathcal N} Q$, if $P \rel{S}_{\mathcal N} Q$ for some ${\mathcal N}$-barbed bisimulation ${\mathcal S}_{\mathcal N}$.
\end{definition}

$\mathcal{R} \subseteq \pi \times \pi$

$P \mathcal{R} Q => \forall P'. P \red P' \Rightarrow \exists Q'. Q \red Q', P' \mathcal{R} Q'$

$P \vdash x \Rightarrow Q \vdash x$

\begin{mathpar}
  \inferrule*[lab=Out-barb]{x \nameeq y}{{y}!\langle{Q}\rangle \vdash x}
  \and
  \inferrule*[lab=Par-barb]{\mbox{$P\vdash x$ or $Q\vdash x$}}{\binpar{P}{Q} \vdash x}
\end{mathpar}

\subsubsection{Contexts}

One of the principle advantages of computational calculi like the
$\pi$-calculus is a well-defined notion of context,
contextual-equivalence and a correlation between
contextual-equivalence and notions of bisimulation. The notion of
context allows the decomposition of a process into (sub-)process and
its syntactic environment, its context. Thus, a context may be
thought of as a process with a ``hole'' (written $\Box$) in it. The
application of a context $M$ to a process $P$, written $M[P]$, is
tantamount to filling the hole in $M$ with $P$. In this paper we do
not need the full weight of this theory, but do make use of the notion
of context in the proof the main theorem. 

\begin{mathpar}
  \inferrule* [lab=summation] {} {{M_{M},M_{N}} \bc \Box \;|\; x.M_{A} \;|\; M_{M}+M_{N}}
  \and
  \inferrule* [lab=agent] {} {{M_{A}} \bc (\vec{x})M_{P} \;| \; \clift{P_0,\ldots,M_{P},\ldots,P_N}}
  \and \\
  \inferrule* [lab=process] {} {{M_{P}} \bc M_{N} \;| \;P|M_{P} }
\end{mathpar} 

\begin{mathpar}
  \inferrule* [lab=sychronization] {} {M_{N} \bc \Box \;|\; x?M_{F} \;|\; x!M_{C}}
  \and
  \inferrule* [lab=abstraction] {} {{M_{F}} \bc (x)M_{P} }
  \and
  \inferrule* [lab=concretion] {} {{M_{C}} \bc \langle M_{P} \rangle }
  \and \\
  \inferrule* [lab=process] {} {{M_{P}} \bc M_{N} \;| \;P|M_{P} }
\end{mathpar}

\begin{definition}[contextual application] Given a context $M$, and
  process $P$, we define the \emph{contextual application}, $M[P] :=
  M\{P/\Box\}$. That is, the contextual application of M to P is the
  substitution of $P$ for $\Box$ in $M$.
\end{definition}

$\meaningof{-} : L \to \mathcal{P}(\pi)$

\begin{mathpar}
  \inferrule* [lab=collection] {} {\meaningof{true} = \pi, \and \meaningof{~E} = \pi \setminus \meaningof{E}, \and \meaningof{E_{1} \& E_{2}} = \meaningof{E_{1}} \cap \meaningof{E_{2}}}
\end{mathpar}

\begin{mathpar}
  \inferrule* [lab=structure] {} {\meaningof{0} = \{ P \in \pi | P \equiv 0 \}, \and \\ \meaningof{E_1 | E_2} = \{ P \in \pi | P \equiv P_{1} | P_{2}, P_{1} \in \meaningof{E_{1}}, P_{2} \in \meaningof{E_2}\} }
\end{mathpar}

\begin{mathpar}
 \inferrule* [lab=behavior] {} {\meaningof{\langle a?b \rangle E} = \{ P \in \pi | P \equiv Q | u?(y)P', \\ \and \\\\ \and \\ \;\;\; u \in \meaningof{a}, \forall z.P'\{z/y\} \in \meaningof{E\{z/b\}}\}, \and \\ \meaningof{a!E} = \{ P \in \pi | P \equiv Q | x!\langle P' \rangle, x \in \meaningof{a} P' \in \meaningof{E}\} }
\end{mathpar}

\begin{mathpar}
 \inferrule* [lab=nominal] {} {\meaningof{\quotep{E}} = \{ \quotep{P} \in \quotep{\pi} | P \in \meaningof{E} \}, \and \meaningof{\quotep{P}} = \{ \quotep{Q} \in \quotep{\pi} | P \equiv Q \} \and \\ \meaningof{@\quotep{E}} = \{ P \in \pi | P \equiv @x, x \in \meaningof{E} \}}
\end{mathpar}

\begin{eqnarray*}
  \\
  \meaningof{-} : TS \to ST
\end{eqnarray*}

\begin{eqnarray*}
  \\
  L : TS \to ST
\end{eqnarray*}

\begin{eqnarray*}
  \\
  P \models E \iff P \in \meaningof{E}
\end{eqnarray*}

\begin{eqnarray*}
  P \approx_{L} Q \iff \forall E \in L. P \models E \iff Q \models E
\end{eqnarray*}

\begin{eqnarray*}
  P \approx_{K} Q
\end{eqnarray*}

\begin{eqnarray*}
  P \approx Q
\end{eqnarray*}

$\approx_{K} = \approx = \approx_{L}$

\subsubsection{Contextual duality}

Note that contexts extend the quotation operation to a family of
operations from processes to names. Given a context, $M$, we can
define a \emph{nominal context}, $\quotep{M}$ by $\quotep{M}[P] :=
\quotep{M[P]}$. To foreshadow what is to come we observe that these
operations enjoy a duality with processes very much like the duality
between vectors and maps from vectors to scalars.

Further, because the calculus is essentially higher-order, we have a
correspondence between contexts and processes. More specifically,
given a name $x$ and a context $M$ we can construct $M^{*}_{x}$ such
that 

\begin{mathpar}
  M^{*}_{x} | \lift{x}{P} \red M[P]
\end{mathpar}

namely,

\begin{mathpar}
  M^{*}_{x} := x?(u).M[\dropn{u}]
\end{mathpar}

The dependence of $M^{*}_{x}$ on a name makes it an abstraction, 

\begin{mathpar}
  M^{*} := (x)x?(u).M[\dropn{u}]
\end{mathpar}

\subsection{Additional notation}

It will sometimes be convenient to denote the process a name
quotes. We already have the notation $x = \quotep{P}$, but it will be
convenient to introduce an alternate notation, $\procn{x}$, when we
want to emphasize the connection to the use of the name. Note that, by
virtue of name equivalence, $\quotep{\procn{x}} \nameeq x$; so, the
notation is consistent with previous definitions.

Further, because names have structure it is possible to effect
substitutions on the basis of that structure. This means we need to
upgrade our notation for substitutions, which we accomplish by
adapting comprehension notation. Thus,

\begin{mathpar}
  P\{ y / x : x \in S \}
\end{mathpar}

is interpreted to mean the process derived from P by replacing (in a
capture-avoiding manner) each occurrence of $x$ in $S$ by $y$. For example,

\begin{mathpar}
  P\{ \quotep{\procn{x}|\procn{x}} / x : x \in \freenames{P} \}
\end{mathpar}

will replace each (occurrence) of a free name $x$ in $P$ by
$\quotep{\procn{x}|\procn{x}}$.

Also, we will avail ourselves of the notation $x^{L}$ and $x^{R}$ to
denote injections of a name into disjoint copies of the name
space. There are numerous ways to accomplish this. One example can be
found in \cite{MeredithR05}. This notation overloads to vectors of
names: $\vec{x}^{\pi} := (x_{i}^{\pi} \; : \; 0 \leq i < |\vec{x}| )$ where $\pi \in \{L,R\}$.

We also use $P^{\Box} := P|\Box$.

In \cite{MeredithR05} an interpretation of the new operator is
given. It turns out that there are several possible interpretations
all enjoying the requisite algebraic properties of the operator (see
\cite{milner91polyadicpi}). We will therefore make liberal use of
$(\nu\; \vec{x})P$.

% subsection the_syntax_and_semantics_of_the_notation_system (end)   

\input{qm2pi.qmops} 

\input{qm2pi.sterngerlach} 

\input{qm2pi.metric} 

% section concurrent_process_calculi (end)

%\input{qm2pi.proofsketch}

% section proof sketch (end)

%\input{qm2pi.slviaknots} 

% section spatial logic via knots (end)

\input{qm2pi.conclusion}

% section conclusion (end)

%\input{qm2pi.dtcodes} 

% section wiring algorithm (end)

\input{qm2pi.ack} 

% section acknowledgments (end)

\newpage


\bibliographystyle{plain}   
\bibliography{../../biblios/main.bib}

\input{qm2pi.rhodetails}

\end{document}



% section proof sketch (end)

%\section{Unlikely characters: spatial logic for
  knots}\label{sub:characteristic_formulae} % (fold)

Associated to the mobile process calculi are a family of logics known
as the Hennessy-Milner logics. These logics typically enjoy a
semantics interpreting formulae as sets of processes that when
factored through the encoding outlined above allows an identification
of classes of knots with logical formulae. In the context of this
encoding the sub-family known as the spatial logics \cite{CairesC03}
\cite{CairesC04} \cite{Caires04} are of particular interest providing
several important features for expressing and reasoning about
properties (i.e. classes) of knots. We hint here at how this may be done.

%\begin{description}
%\item [structural connectives] 
\subsubsection{Structural connectives} The spatial logics enjoy
structural connectives corresponding, at the logical level, to the
parallel composition ($P | Q$) and new name ($(\nu \; x)P$)
connectives for processes. As illustrated in the examples below, these
connectives are extremely expressive given the shape of our encoding.
%\item [decideable satisfaction]

\subsubsection{Decideable satisfaction}
In \cite{Caires04} the satisfaction relation is shown to be decideable
for a rich class of processes. It further turns out that the image of
the our encoding is a proper subset of that class. This result
provides the basis for an algorithm by which to search for knots
enjoying a given property.
%\item [characteristic formulae]

\subsubsection{Characteristic formulae}
In the same paper \cite{Caires04} , Caires presents a means of calculating
characteristic formulae, selecting equivalence classes of processes
up to a pre--specified depth limit on the support set of names. Composed with our
encoding, this characteristic formula can be used to select
characteristic formulae for knots.
%\end{description}

\subsubsection{Spatial logic formulae}

The grammar below (segmented for comprehension) summarizes the syntax
of spatial logic formulae. We employ illustrative examples in the
sequel to provide an intuitive understanding of their meaning
referring the reader to \cite{Caires04} for a more detailed explication
of the semantics.

\begin{mathpar}
  \inferrule* [lab=boolean] {} {{A,B} \bc T \;|\; \neg A \;|\; A \wedge B \;|\; \eta = \eta'}
  \and
  \inferrule* [lab=spatial] {} {|\; \pzero \;|\; A | B \;|\; x \text{\textregistered} A \;|\; \forall x . A \;|\;  H x . A}
  \and
  \inferrule* [lab=behavioral] {} {|\; \alpha . A}
  \and 
  \inferrule* [lab=recursion] {} {|\; X(\vec{u}) \;|\; \mu X(\vec{u}) . A}
  \and
  \inferrule* [lab=action] {} {\alpha \bc \langle x?(\vec{y}) \rangle \;|\; \langle x!(\vec{y}) \rangle \;|\; \langle \tau \rangle}
  \and 
  \inferrule* [lab=name] {} {\eta \bc x \;|\; \tau}
\end{mathpar} 

% subsection characteristic_formulae (end)   	 

\subsection{Example formulae}\label{sub:example_formulae_} % (fold)

\subsubsection{Crossing as formula.}
% 
% \begin{align*}
%   \frac{d}{dx} \sin x &= \cos x 
%   & \frac{d}{dx} e^x &= e^x \\
%   \frac{d}{dx} \cos x &= - \sin x 
%   & \frac{d}{dx} \log x &= \frac{1}{x} \\
% \end{align*} 

\begin{align*}
 \mu C(x_{0},x_{1},y_{0},y_{1},u).&(\langle x_{0}?(z) \rangle(\langle u! \rangle\langle y_{1}!z \rangle C(x_{0},x_{1},y_{0},y_{1},u)) & \\
  & \wedge \langle y_{1}?(z) \rangle (\langle u! \rangle \langle x_{0}!z \rangle C(x_{0},x_{1},y_{0},y_{1},u)) & \\
  & \wedge \langle x_{1}?(z) \rangle (\langle u? \rangle \langle y_{0}!z \rangle C(x_{0},x_{1},y_{0},y_{1},u)) & \\
  & \wedge \langle y_{0}?(z) \rangle (\langle u? \rangle \langle x_{1}!z \rangle C(x_{0},x_{1},y_{0},y_{1},u))) &
\end{align*}

The lexicographical similarity between the shape of this formulae and
the shape of definition of the process representing a crossing reveals
the intuitive meaning of this formulae. It describes the capabilities
of a process that has the right to represent a crossing. For example
it picks out processes that may perform an input on the port $x_0$ in
its initial menu of capabilities. What differentiates the formula
from the process, however, is that the crossing process is the
smallest candidate to satisfy the formula. Infinitely many other
processes -- with internal behavior hidden behind this interface, so
to speak -- also satisfy this formula. Even this simple formula,
then, can be seen to open a new view onto knots, providing a
computational interpretation of \emph{virtual} knots.

Note that this formula is derived by hand. A similar formula can be
derived by employing Caires' calculation of characteristic formula
\cite{Caires04} to the process representing a crossing. In light of
this discussion, we let
$\meaningof{C}_{\phi}(x0,x1,y0,y1,u)$ denote a formula specifying the
dynamics we wish to capture of a crossing. To guarantee we preserve
the shape of the interface and minimal semantics we demand that
$\meaningof{C}_{\phi}(x0,x1,y0,y1,u) \Rightarrow
\textbf{C}(x0,x1,y0,y1,u)$ where $\textbf{C}(x0,x1,y0,y1,u)$ denotes
the formula above.
                            
\subsubsection{Crossing number constraints.}
The moral content of the context lemma (Lemma \ref{context}) is that the notion of
``locality'' in the Reidemeister moves is effectively captured by the
parallel composition operator of the process calculus. This intuition
extends through the logic. Given a formula,
$\meaningof{C}_{\phi}(x0,x1,y0,y1,u)$, we can use the structural
connectives to specify constraints on crossing numbers, such as at
least $n$ crossings, or exactly $n$ crossings.
\begin{mathpar}
  \inferrule* [lab=at-least-n] {} { K^{\geq n}_{\phi}(\vec{xs},\vec{ys}) := \Pi_{i=0}^{n-1} Hu . \meaningof{C}_{\phi}(xs_i,ys_i,u) | T }
  \and 
  \inferrule* [lab=exactly-n] {} { K^{= n}_{\phi}(\vec{xs},\vec{ys}) := \Pi_{i=0}^{n-1} Hu . \meaningof{C}_{\phi}(xs_i,ys_i,u) | \neg (\forall x_0,y_0,x_1,y_1,u . \meaningof{C}_{\phi}(x_0,y_0,x_1,y_1,u) | T) }
\end{mathpar}

To round out this section, recall that the encoding of an $n$-crossing
knot decomposes into a parallel composition of $n$ \emph{copies} of a
crossing process together with a wiring harness. To specify different
knot classes with the same crossing number amounts to specifying
logical constraints on the wiring harness. In the interest of space,
we defer examples to a forthcoming paper. Suffice it to say that both
the conditions ``alternating knot'' and ``contains the tangle
corresponding to 5/3'' are expressible. For example, it is possible to
calculate the characteristic formula of a process corresponding to the
tangle 5/3 and conjoin it into the classifying formula via the
composition connective of the logic.

Finally, we wish to observe that it is entirely within reason to
contemplate a more domain-specific version of spatial logic tailored
to the shape of processes in the image of the encoding. Such a
domain-specific logic would have a better claim to the title formal
language of knot properties.

% subsection example_formulae_ (end)

% section knots_as_processes (end) 

% section spatial logic via knots (end)

\section{Conclusions and future work}

\paragraph{Testing physical space}
You, gentle reader, may wonder why of all the theorems to be proved
given this set up we pick the one above. In some sense it's hardly
central to quantum mechanics. We see it as central in the sense that
it firmly establishes a notion of physical space arising from a notion
of the equivalence of behavior. Relating bisimulation to a metric is a
big step forward, but one is faced with interpreting the relationship
of that metric space to something more physical. Quantum mechanical
notions of ``physical'' space are still far from intuitive, but by
relating this idea of distance as testing to calculations that predict
physical circumstances we are making a not insignificant step forward
toward an understanding of the physical space we inhabit as
essentially dynamic.

\paragraph{Effectivity and simulation}
One of the observations we have yet to make is that the entire program
spelled out here is effective. We have built various interpreters for
the reflective calculus at work in this interpretation. In principle,
then, we can simulate quantum mechanics on a computer. The place where
the simulation may lose fidelity is the infinitely branching summation
for the annihilator.

In this connection i also want to point out that the evaluation style
calculation of the inner product puts the non-determinism of the
summation right at the heart of measurement. This suggests that
Milner's original reduction-based formulation of the dynamics of his
calculi in terms of sums was not just notationally suggestive of a
notion of measure-and-continue but captured some significant part of
the physics.

\paragraph{Quantum continuations}
In light of this last observation i want to point out that the
predominant account of quantum mechanics is missing a key aspect of a
truly compositional story of the physical situation. In a real lab,
when a measurement is made the observation can be made to feed into
another device that then makes another measurement conditioned on the
results of the first. This means that after the superposition was
collapsed the entire experimental set up remained in
superposition. While QM offers a means of writing this down it doesn't
quite line up well with the well-trodden formulation of computation
and continuation that we see so succinctly expressed in Milner's
calculi. This suggests that there might be advantages to this account
of dynamics waiting to be explored.

\paragraph{Quantum logic}
In this connection, we also note that by virtue of having the
Hennessy-Milner construction, we can pull the construction through the
interpretation of QM. This gives us a natural candidate for a quantum
logic that enjoys an extremely tight connection with it's domain of
interpretation, making the construction much less ad hoc (rather it is
the image of functor!).

\paragraph{Quantum probabiity}
i have questions about the basis of the interpretation of inner
product as probability amplitude. In particular, using which
axiomatization of probability theory does the notion of probability
amplitude earn the right to be so dubbed? In other words, where is the
proof that the operation for calculating a probability amplitude (and
then squaring) satisfies the axioms of what it means to calculate a
probability? Even if such a proof exists (i have yet to find it in the
literature), i wonder if it might not be possible to turn things on
their heads. Can we view the calculation of the probability amplitude
as an axiomatization of probability? If so, then the definition we
give for calculating probability amplitude may provide the basis for
an \emph{effective} theory of probability.

\paragraph{Quantum vs ``biological'' information}
Finally, i want to conclude with a more philosophical observation. At
a recent workshop in which QM was a predominant topic i noticed
something about quantum information. The speaker was giving a riveting
discussion of axiomatic QM and showing how properties of ``no
cloning'' and ``no deleting'' emerged as consequences of the
axiomatization. Theorems of this form are necessary to give us a sense
of confidence that our axioms characterize the physical theory. What
struck me, though, was that if quantum information is neither erasable
nor replicable it is markedly different from \emph{life}. Two of the
things we know about life is that

\begin{itemize}
  \item it ends;
  \item to gain some measure of persistence, to transcend it's
    finitude it is imminently copyable.
\end{itemize}

Both of these qualities are summarized succinctly in the aphorism: all
flesh is grass. For me these two kinds of ``information'' -- call them
quantum and biological -- are end points on a spectrum of strategies
for persistence. At one end, we have those curious entities that enjoy
uniqueness and permanence; at the other, we have those who in the face
of a certain end and an uncertain present make a go of passing
something on. To me one of the more remarkable aspects of the latter
strategy is that in the presence of noise (and certain features of
copying) we get a kind of dynamism, a chance for improvement against a
given persistent condition.

% subsection other_calculi_other_bisimulations_and_geometry_as_behavior (end)




% section conclusion (end)

%\documentclass[12pt]{llncs}
%\documentclass{jktr}

\usepackage[pdftex]{hyperref}                   
\usepackage {listings}
\usepackage {mathpartir}
\usepackage{bcprules}
%\usepackage{listings}
                       
\usepackage{graphicx} 
%\usepackage[margins=2.5cm,nohead,nofoot]{geometry}
%\usepackage{geometry}
\usepackage{amsfonts}
\usepackage{amstext}
\usepackage{latexsym}
\usepackage{amssymb}
\usepackage{color}


%\include{myPreamble}
\include{qm2pi.local} 

%\ifpdf
%\usepackage[pdftex]{graphicx}
%\else
%\usepackage{graphicx}
%\fi

 % \ifpdf
%  \usepackage{pdfsync}
%  \if


%\title{Brief Article}
%\author{David F. Snyder}
%\author{L.G. Meredith}

%\address{Dept. of Math., Texas State University--San Marcos, San Marcos, TX 78666}
       
\pagestyle{empty}


\begin{document}

\lstset{language=[Objective]Caml,frame=shadowbox}

\input{qm2pi.front}

% section front matter (end)

\input{qm2pi.intro} 
 
% section introduction (end)

% \input{qm2pi.knotations} 

% section notation (end)

\input{qm2pi.process.calculi} 

% section concurrent_process_calculi_and_spatial_logics_ (end)
    
%\input{qm2pi.knots2pi} 

%\input{qm2pi.trefoil} 

%\input{qm2pi.mainthm} 

% subsection basic_interpretation (end)

%\input{qm2pi.rho.presentation} 
\subsection{The syntax and semantics of the notation system}\label{sub:the_syntax_and_semantics_of_the_notation_system} % (fold)

We now summarize a technical presentation of the calculus that
embodies our theory of dynamics. The typical presentation of such a
calculus follows the style of giving generators and relations on
them. The grammar, below, describing term constructors, freely
generates the set of processes, $\Proc$. This set is then quotiented
by a relation known as structural congruence and it is over this set
that the notion of dynamics is expressed. This presentation is
essentially that of \cite{MeredithR05} with the addition of
polyadicity and summation. For readability we have relegated some of
the technical subtleties to an appendix.

\subsubsection{Process grammar}\label{subsub:process_grammar}

\begin{mathpar}
  \inferrule* [lab=synchronization] {} {{M} \bc \pzero \;|\; x?F \;|\; x!C }
  \and
  \inferrule* [lab=abstraction] {} {{F} \bc (x)P}
  \and
  \inferrule* [lab=concretion] {} {{C} \bc \langle Q \rangle}
  \and
  \inferrule* [lab=process] {} {{P,Q} \bc M \;| \;P|Q \;|\; @{x}}
  \and
  \inferrule* [lab=name] {} {{x} \bc \quotep{P}}
\end{mathpar} 

Note that $\vec{x}$ (resp. $\vec{P}$) denotes a vector of names
(resp. processes) of length $|\vec{x}|$ (resp. $|\vec{P}|$). We adopt
the following useful abbreviations.

\begin{mathpar}
   x?(\vec{y}).P := x.(\vec{y})P \and  x\clift{\vec{P}} := x.\clift{\vec{P}}
   \and x!(y) := \lift{x}{\dropn{y}}
   \and \Pi_{i=0}^{n-1}P_i := P_0 | \ldots | P_{n-1}
\end{mathpar}

\subsubsection{Structural congruence}

\paragraph{Free and bound names and alpha-equivalence.} At the
core of structural equivalence is alpha-equivalence which identifies
process that are the same up to a change of variable. Formally, we
recognize the distinction between free and bound names. The free names
of a process, $\freenames{P}$, may be calculated recursively as
follows:

\begin{mathpar}
\freenames{\pzero} := \emptyset
  \and \\
  \freenames{x?(y).P} := \{ x \} \cup (\freenames{P} \setminus \{ y \})
  \and 
  \freenames{x!\langle P \rangle} := \{ x \} \cup \{ P \} 
  \and \\
  \freenames{P|Q} := \freenames{P} \cup \freenames{Q}
  \and \\
  \freenames{@{x}} := \{ x \}
\end{mathpar}

$\pi$
$\quotep{\pi}$

$\freenames{-} : \pi \to \mathcal{P}(\quotep{\pi})$

\begin{eqnarray*}
  \freenames{\pzero} & := & \emptyset \\
  \freenames{x?(y).P} & := & \{ x \} \cup (\freenames{P} \setminus \{ y \}) \\
  \freenames{x!\langle P \rangle} & := & \{ x \} \cup \{ P \} \\
  \freenames{P|Q} & := & \freenames{P} \cup \freenames{Q} \\
  \freenames{\dropn{x}} & := & \{ x \}
\end{eqnarray*}

The bound names of a process, $\boundnames{P}$, are those names occurring in $P$
that are not free. For example, in $x?(y).0$, the name $x$ is free, while $y$ is bound.

\begin{mathpar}
  \inferrule* [lab=monoidal-laws] {} { P|Q \equiv Q|P \and P|0 \equiv P \and P|(Q|R) \equiv (P|Q)|R }
\end{mathpar}

\begin{mathpar}
  \inferrule* [lab=alpha-equivalence] {} { (x)P \equiv (y)P\{y/x\} \and y \not\in \freenames{P} }
\end{mathpar}

\begin{definition}
Then two processes, $P,Q$, are alpha-equivalent if $P = Q\{\vec{y}/\vec{x}\}$ for
some $\vec{x} \in \boundnames{Q},\vec{y} \in \boundnames{P}$, where $Q\{\vec{y}/\vec{x}\}$
denotes the capture-avoiding substitution of $\vec{y}$ for $\vec{x}$ in $Q$.
\end{definition}

\begin{definition}
  The {\em structural congruence} \cite{SangiorgiWalker} , $\equiv$,
  between processes is the least congruence containing
  alpha-equivalence, satisfying the abelian monoid laws
  (associativity, commutativity and $\pzero$ as identity) for parallel
  composition $|$ and for summation $+$.
\end{definition}

\subsection{Name equivalence}

We take name equivalence, written $\nameeq$, to be the smallest
equivalence relation generated by the following rules.

\begin{mathpar}
\inferrule*[lab=Quote-drop]
{ }
{ \quotep{@{x}} \nameeq x }

\inferrule*[lab=Struct-equiv]
{ P \scong Q }
{ \quotep{P} \nameeq \quotep{Q} }
\end{mathpar}

The astute reader will have noticed that the mutual recursion of names
and processes imposes a mutual recursion on alpha-equivalence and
structural equivalence via name-equivalence. Fortunately, all of this
works out pleasantly and we may calculate in the natural way, free of
concern. The reader interested in the details is referred to the
appendix \ref{appendix:rho_details}.

\subsection{Substitution}

We use $\Proc$ for the set of processes, $\QProc$ for the set of
names, and $\id{\{}\vec{y} / \vec{x} \id{\}}$ to denote partial maps,
$s : \QProc \rightarrow \QProc$. A map, $s$ lifts, uniquely, to a map
on process terms, $\widehat{s} : \Proc \rightarrow \Proc$ by the
following equations.

\begin{mathpar}
  (0) \psubstp{Q}{P} := 0 \\
  (R \juxtap S) \psubstp{Q}{P}
  :=    
  (R)\psubstp{Q}{P} \juxtap (S) \psubstp{Q}{P} \\
  (x?(y).R) \psubstp{Q}{P}    
  :=    
  (x)\substp{Q}{P} (z)\concat( (R \psubstn{z}{y}) \psubstp{Q}{P} ) \\
  (\lift{x}{R}) \psubstp{Q}{P}  
  :=
  \lift{(x)\substp{Q}{P}}{ R \psubstp{Q}{P} } \\
%   (\dropn{x})  \psubstp{Q}{P}       
%   := 
%   \left\{ 
%     \begin{array}{ccc} 
%       \dropn{\quotep{Q}} & & x \nameeq \quotep{P} \\
%       \dropn{x} & & otherwise \\
%     \end{array}
%   \right. 
  (\dropn{x})  \psubstp{Q}{P}       
  := 
  \left\{ 
    \begin{array}{ccc} 
      Q & & x \nameeq \quotep{P} \\
      \dropn{x} & & otherwise \\
    \end{array}
  \right.
\end{mathpar}
 

where

\begin{eqnarray}
  (x)\id{\{} \lpquote Q \rpquote / \lpquote P \rpquote \id{\}}            = 
  \left\{ 
    \begin{array}{ccc}
      \lpquote Q \rpquote & & x \nameeq \lpquote P \rpquote \\
      x & & otherwise \\
    \end{array}
  \right. \nonumber
\end{eqnarray}

and $z$ is chosen distinct from $\quotep{P}$, $\quotep{Q}$, the free
names in $Q$, and all the names in $R$. Our $\alpha$-equivalence will
be built in the standard way from this substitution.

\begin{remark}\label{rem:no_self_referential_names}
  One consequence of these definitions is that $\forall P. \quotep{P}
  \not\in \freenames{P}$.
\end{remark}

\subsection{ Dynamic quote: an example }

Anticipating something of what's to come, consider applying the
substitution, $\widehat{\id{\{}u / z \id{\}}}$, to the following pair
of processes, $\lift{w}{y!(z)}$ and $w[ \lpquote y!(z) \rpquote ]$.

\begin{eqnarray}
	\lift{w}{y!(z)}\widehat{\id{\{}u / z \id{\}}}
		& = &
		\lift{w}{y!(u)} \nonumber\\
	w[ \lpquote y!(z) \rpquote ] \widehat{ \id{\{}u / z \id{\}} }
		& = &
		w[ \lpquote y!(z) \rpquote ] \nonumber
\end{eqnarray}

Because the body of the process between quotes is impervious to
substitution, we get radically different answers. In fact, by
examining the first process in an input context,
e.g. $x?(z).\lift{w}{y!(z)}$, we see that the process under the lift
operator may be shaped by prefixed inputs binding a name inside it. In
this sense, the lift operator will be seen as a way to dynamically
construct processes before reifying them as names.

Finally equipped with these standard features we can present the
dynamics of the calculus.

\subsubsection{Operational semantics} 

Finally, we introduce the computational dynamics. What marks these
algebras as distinct from other more traditionally studied algebraic
structures, e.g. vector spaces or polynomial rings, is the manner in
which dynamics is captured. In traditional structures, dynamics is typically
expressed through morphisms between such structures, as in linear maps
between vector spaces or morphisms between rings. In algebras
associated with the semantics of computation, the dynamics is
expressed as part of the algebraic structure itself, through a
reduction reduction relation typically denoted by $\red$. Below, we
give a recursive presentation of this relation for the calculus used
in the encoding.

$\red \subseteq \pi \times \pi$
$\red : \pi \to \mathcal{P}(\pi)$

\begin{mathpar}
  \inferrule* [lab=Comm] { \textsf{match}( x_{src}, x_{trgt} ) } { x_{trgt}?(y)P \; | \; x_{src}!\langle {Q} \rangle \red P\{\quotep{Q}/y}\} }
  \and \\
  \inferrule* [lab=Par] {{P} \red {P}'} {{{P} | {Q}} \red {{P}' | {Q}}}
  \and
  \inferrule* [lab=Equiv]{{{P} \scong {P}'} \andalso {{P}' \red {Q}'} \andalso {{Q}' \scong {Q}}}{{P} \red {Q}}
\end{mathpar}

\begin{eqnarray*}
  match_{\equiv} (\quotep{P},\quotep{Q}) & := & P \equiv Q \\
  match_{\dagger}(\quotep{P},\quotep{Q}) & := & \forall R. P|Q \red^{*} R => R \red^{*} 0 \\
  match_{K}(\quotep{P},\quotep{Q}) & := & K \mbox{ for some context } K
\end{eqnarray*}

$u?(x)P | u!\langle Q \rangle \red P\{\quotep{Q}/x\}$

%We write $\wred$ for $\red^*$, and $P\red$ if $\exists Q $ such that $ P \red Q$.
We write $P\red$ if $\exists Q $ such that $ P \red Q$ and $P\not\red$, otherwise.

\section{Replication}

As mentioned before, it is known that replication (and hence
recursion) can be implemented in a higher-order process algebra
\cite{SangiorgiWalker}. As our first example of calculation with the
machinery thus far presented we give the construction explicitly in
the {\rhoc}.

\begin{eqnarray}
	D_{x} & := & \prefix{x}{y}{(\binpar{\outputp{x}{y}}{@{y}})} \nonumber\\
	\bangp_{x}{P} & := & \binpar{{x}!\langle{\binpar{D_{x}}{P}}\rangle}{D_{x}} \nonumber
\end{eqnarray}

\begin{eqnarray}
	\bangp_{x}{P} & & \nonumber\\
	=
	& {x}!\langle{(\prefix{x}{y}{(\outputp{x}{y} | @{y})) | P}}\rangle 
	      | \prefix{x}{y}{(\outputp{x}{y} | @{y})} & \nonumber\\
	\red
	& (\outputp{x}{y} | @{y})\substn{\quotep{(\prefix{x}{y}{(@{y} | \outputp{x}{y})) | P}}}{y} & \nonumber\\
	=
	& \outputp{x}{\quotep{(\prefix{x}{y}{(\outputp{x}{y} | @{y})) | P}}}
	  | {(\prefix{x}{y}{(\outputp{x}{y} | @{y})) | P}} & \nonumber\\
	\red
	& \ldots & \nonumber\\
	\red^*
	& P | P | \ldots & \nonumber
\end{eqnarray}

Of course, this encoding, as an implementation, runs away, unfolding
$\bangp{P}$ eagerly. A lazier and more implementable replication
operator, restricted to input-guarded processes, may be obtained as follows.

\begin{eqnarray}
\bangp{\prefix{u}{v}{P}} 
	:= 
	\binpar{\lift{x}{\prefix{u}{v}{(\binpar{D(x)}{P})}}}{D(x)} \nonumber
\end{eqnarray}

\begin{remark}
  Note that the lazier definition still does not deal with summation
  or mixed summation (i.e. sums over input and output). The reader is
  invited to construct definitions of replication that deal with these
  features. 

  Further, the definitions are parameterized in a name, $x$. Can you,
  gentle reader, make a definition that eliminates this parameter and
  guarantees no accidental interaction between the replication
  machinery and the process being replicated -- i.e. no accidental
  sharing of names used by the process to get its work done and the
  name(s) used by the replication to effect copying. This latter
  revision of the definition of replication is crucial to obtaining
  the expected identity $!!P \sim !P$.
\end{remark}

\begin{remark}\label{rem:paradoxical_combinator}
  The reader familiar with the lambda calculus will have noticed the
  similarity between $D$ and the paradoxical combinator.

  [Ed. note: the existence of this seems to suggest we have to be more
  restrictive on the set of processes and names we admit if we are to
  support no-cloning.]
\end{remark}

\subsubsection{Bisimulation}

The computational dynamics gives rise to another kind of equivalence,
the equivalence of computational behavior. As previously mentioned
this is typically captured \emph{via} some form of bisimulation.

% The notion we use in this paper is weak barbed bisimulation
% \cite{milner91polyadicpi}.

The notion we use in this paper is derived from weak barbed
bisimulation \cite{milner91polyadicpi}. 

\begin{definition}
An \emph{observation relation}, $\downarrow_{\mathcal N}$, over a set
of names, $\mathcal N$, is the smallest relation satisfying the rules
below.

\infrule[Out-barb]{y \in {\mathcal N}, \; x \nameeq y}
		  {\outputp{x}{v} \downarrow_{\mathcal N} x}
\infrule[Par-barb]{\mbox{$P\downarrow_{\mathcal N} x$ or $Q\downarrow_{\mathcal N} x$}}
		  {\binpar{P}{Q} \downarrow_{\mathcal N} x}

We write $P \Downarrow_{\mathcal N} x$ if there is $Q$ such that 
$P \wred Q$ and $Q \downarrow_{\mathcal N} x$.
\end{definition}

\begin{definition}
%\label{def.bbisim}
An  ${\mathcal N}$-\emph{barbed bisimulation} over a set of names, ${\mathcal N}$, is a symmetric binary relation 
${\mathcal S}_{\mathcal N}$ between agents such that $P\rel{S}_{\mathcal N}Q$ implies:
\begin{enumerate}
\item If $P \red P'$ then $Q \wred Q'$ and $P'\rel{S}_{\mathcal N} Q'$.
\item If $P\downarrow_{\mathcal N} x$, then $Q\Downarrow_{\mathcal N} x$.
\end{enumerate}
$P$ is ${\mathcal N}$-barbed bisimilar to $Q$, written
$P \wbbisim_{\mathcal N} Q$, if $P \rel{S}_{\mathcal N} Q$ for some ${\mathcal N}$-barbed bisimulation ${\mathcal S}_{\mathcal N}$.
\end{definition}

$\mathcal{R} \subseteq \pi \times \pi$

$P \mathcal{R} Q => \forall P'. P \red P' \Rightarrow \exists Q'. Q \red Q', P' \mathcal{R} Q'$

$P \vdash x \Rightarrow Q \vdash x$

\begin{mathpar}
  \inferrule*[lab=Out-barb]{x \nameeq y}{{y}!\langle{Q}\rangle \vdash x}
  \and
  \inferrule*[lab=Par-barb]{\mbox{$P\vdash x$ or $Q\vdash x$}}{\binpar{P}{Q} \vdash x}
\end{mathpar}

\subsubsection{Contexts}

One of the principle advantages of computational calculi like the
$\pi$-calculus is a well-defined notion of context,
contextual-equivalence and a correlation between
contextual-equivalence and notions of bisimulation. The notion of
context allows the decomposition of a process into (sub-)process and
its syntactic environment, its context. Thus, a context may be
thought of as a process with a ``hole'' (written $\Box$) in it. The
application of a context $M$ to a process $P$, written $M[P]$, is
tantamount to filling the hole in $M$ with $P$. In this paper we do
not need the full weight of this theory, but do make use of the notion
of context in the proof the main theorem. 

\begin{mathpar}
  \inferrule* [lab=summation] {} {{M_{M},M_{N}} \bc \Box \;|\; x.M_{A} \;|\; M_{M}+M_{N}}
  \and
  \inferrule* [lab=agent] {} {{M_{A}} \bc (\vec{x})M_{P} \;| \; \clift{P_0,\ldots,M_{P},\ldots,P_N}}
  \and \\
  \inferrule* [lab=process] {} {{M_{P}} \bc M_{N} \;| \;P|M_{P} }
\end{mathpar} 

\begin{mathpar}
  \inferrule* [lab=sychronization] {} {M_{N} \bc \Box \;|\; x?M_{F} \;|\; x!M_{C}}
  \and
  \inferrule* [lab=abstraction] {} {{M_{F}} \bc (x)M_{P} }
  \and
  \inferrule* [lab=concretion] {} {{M_{C}} \bc \langle M_{P} \rangle }
  \and \\
  \inferrule* [lab=process] {} {{M_{P}} \bc M_{N} \;| \;P|M_{P} }
\end{mathpar}

\begin{definition}[contextual application] Given a context $M$, and
  process $P$, we define the \emph{contextual application}, $M[P] :=
  M\{P/\Box\}$. That is, the contextual application of M to P is the
  substitution of $P$ for $\Box$ in $M$.
\end{definition}

$\meaningof{-} : L \to \mathcal{P}(\pi)$

\begin{mathpar}
  \inferrule* [lab=collection] {} {\meaningof{true} = \pi, \and \meaningof{~E} = \pi \setminus \meaningof{E}, \and \meaningof{E_{1} \& E_{2}} = \meaningof{E_{1}} \cap \meaningof{E_{2}}}
\end{mathpar}

\begin{mathpar}
  \inferrule* [lab=structure] {} {\meaningof{0} = \{ P \in \pi | P \equiv 0 \}, \and \\ \meaningof{E_1 | E_2} = \{ P \in \pi | P \equiv P_{1} | P_{2}, P_{1} \in \meaningof{E_{1}}, P_{2} \in \meaningof{E_2}\} }
\end{mathpar}

\begin{mathpar}
 \inferrule* [lab=behavior] {} {\meaningof{\langle a?b \rangle E} = \{ P \in \pi | P \equiv Q | u?(y)P', \\ \and \\\\ \and \\ \;\;\; u \in \meaningof{a}, \forall z.P'\{z/y\} \in \meaningof{E\{z/b\}}\}, \and \\ \meaningof{a!E} = \{ P \in \pi | P \equiv Q | x!\langle P' \rangle, x \in \meaningof{a} P' \in \meaningof{E}\} }
\end{mathpar}

\begin{mathpar}
 \inferrule* [lab=nominal] {} {\meaningof{\quotep{E}} = \{ \quotep{P} \in \quotep{\pi} | P \in \meaningof{E} \}, \and \meaningof{\quotep{P}} = \{ \quotep{Q} \in \quotep{\pi} | P \equiv Q \} \and \\ \meaningof{@\quotep{E}} = \{ P \in \pi | P \equiv @x, x \in \meaningof{E} \}}
\end{mathpar}

\begin{eqnarray*}
  \\
  \meaningof{-} : TS \to ST
\end{eqnarray*}

\begin{eqnarray*}
  \\
  L : TS \to ST
\end{eqnarray*}

\begin{eqnarray*}
  \\
  P \models E \iff P \in \meaningof{E}
\end{eqnarray*}

\begin{eqnarray*}
  P \approx_{L} Q \iff \forall E \in L. P \models E \iff Q \models E
\end{eqnarray*}

\begin{eqnarray*}
  P \approx_{K} Q
\end{eqnarray*}

\begin{eqnarray*}
  P \approx Q
\end{eqnarray*}

$\approx_{K} = \approx = \approx_{L}$

\subsubsection{Contextual duality}

Note that contexts extend the quotation operation to a family of
operations from processes to names. Given a context, $M$, we can
define a \emph{nominal context}, $\quotep{M}$ by $\quotep{M}[P] :=
\quotep{M[P]}$. To foreshadow what is to come we observe that these
operations enjoy a duality with processes very much like the duality
between vectors and maps from vectors to scalars.

Further, because the calculus is essentially higher-order, we have a
correspondence between contexts and processes. More specifically,
given a name $x$ and a context $M$ we can construct $M^{*}_{x}$ such
that 

\begin{mathpar}
  M^{*}_{x} | \lift{x}{P} \red M[P]
\end{mathpar}

namely,

\begin{mathpar}
  M^{*}_{x} := x?(u).M[\dropn{u}]
\end{mathpar}

The dependence of $M^{*}_{x}$ on a name makes it an abstraction, 

\begin{mathpar}
  M^{*} := (x)x?(u).M[\dropn{u}]
\end{mathpar}

\subsection{Additional notation}

It will sometimes be convenient to denote the process a name
quotes. We already have the notation $x = \quotep{P}$, but it will be
convenient to introduce an alternate notation, $\procn{x}$, when we
want to emphasize the connection to the use of the name. Note that, by
virtue of name equivalence, $\quotep{\procn{x}} \nameeq x$; so, the
notation is consistent with previous definitions.

Further, because names have structure it is possible to effect
substitutions on the basis of that structure. This means we need to
upgrade our notation for substitutions, which we accomplish by
adapting comprehension notation. Thus,

\begin{mathpar}
  P\{ y / x : x \in S \}
\end{mathpar}

is interpreted to mean the process derived from P by replacing (in a
capture-avoiding manner) each occurrence of $x$ in $S$ by $y$. For example,

\begin{mathpar}
  P\{ \quotep{\procn{x}|\procn{x}} / x : x \in \freenames{P} \}
\end{mathpar}

will replace each (occurrence) of a free name $x$ in $P$ by
$\quotep{\procn{x}|\procn{x}}$.

Also, we will avail ourselves of the notation $x^{L}$ and $x^{R}$ to
denote injections of a name into disjoint copies of the name
space. There are numerous ways to accomplish this. One example can be
found in \cite{MeredithR05}. This notation overloads to vectors of
names: $\vec{x}^{\pi} := (x_{i}^{\pi} \; : \; 0 \leq i < |\vec{x}| )$ where $\pi \in \{L,R\}$.

We also use $P^{\Box} := P|\Box$.

In \cite{MeredithR05} an interpretation of the new operator is
given. It turns out that there are several possible interpretations
all enjoying the requisite algebraic properties of the operator (see
\cite{milner91polyadicpi}). We will therefore make liberal use of
$(\nu\; \vec{x})P$.

% subsection the_syntax_and_semantics_of_the_notation_system (end)   

\input{qm2pi.qmops} 

\input{qm2pi.sterngerlach} 

\input{qm2pi.metric} 

% section concurrent_process_calculi (end)

%\input{qm2pi.proofsketch}

% section proof sketch (end)

%\input{qm2pi.slviaknots} 

% section spatial logic via knots (end)

\input{qm2pi.conclusion}

% section conclusion (end)

%\input{qm2pi.dtcodes} 

% section wiring algorithm (end)

\input{qm2pi.ack} 

% section acknowledgments (end)

\newpage


\bibliographystyle{plain}   
\bibliography{../../biblios/main.bib}

\input{qm2pi.rhodetails}

\end{document}

 

% section wiring algorithm (end)

\documentclass[12pt]{llncs}
%\documentclass{jktr}

\usepackage[pdftex]{hyperref}                   
\usepackage {listings}
\usepackage {mathpartir}
\usepackage{bcprules}
%\usepackage{listings}
                       
\usepackage{graphicx} 
%\usepackage[margins=2.5cm,nohead,nofoot]{geometry}
%\usepackage{geometry}
\usepackage{amsfonts}
\usepackage{amstext}
\usepackage{latexsym}
\usepackage{amssymb}
\usepackage{color}


%\include{myPreamble}
\include{qm2pi.local} 

%\ifpdf
%\usepackage[pdftex]{graphicx}
%\else
%\usepackage{graphicx}
%\fi

 % \ifpdf
%  \usepackage{pdfsync}
%  \if


%\title{Brief Article}
%\author{David F. Snyder}
%\author{L.G. Meredith}

%\address{Dept. of Math., Texas State University--San Marcos, San Marcos, TX 78666}
       
\pagestyle{empty}


\begin{document}

\lstset{language=[Objective]Caml,frame=shadowbox}

\input{qm2pi.front}

% section front matter (end)

\input{qm2pi.intro} 
 
% section introduction (end)

% \input{qm2pi.knotations} 

% section notation (end)

\input{qm2pi.process.calculi} 

% section concurrent_process_calculi_and_spatial_logics_ (end)
    
%\input{qm2pi.knots2pi} 

%\input{qm2pi.trefoil} 

%\input{qm2pi.mainthm} 

% subsection basic_interpretation (end)

%\input{qm2pi.rho.presentation} 
\subsection{The syntax and semantics of the notation system}\label{sub:the_syntax_and_semantics_of_the_notation_system} % (fold)

We now summarize a technical presentation of the calculus that
embodies our theory of dynamics. The typical presentation of such a
calculus follows the style of giving generators and relations on
them. The grammar, below, describing term constructors, freely
generates the set of processes, $\Proc$. This set is then quotiented
by a relation known as structural congruence and it is over this set
that the notion of dynamics is expressed. This presentation is
essentially that of \cite{MeredithR05} with the addition of
polyadicity and summation. For readability we have relegated some of
the technical subtleties to an appendix.

\subsubsection{Process grammar}\label{subsub:process_grammar}

\begin{mathpar}
  \inferrule* [lab=synchronization] {} {{M} \bc \pzero \;|\; x?F \;|\; x!C }
  \and
  \inferrule* [lab=abstraction] {} {{F} \bc (x)P}
  \and
  \inferrule* [lab=concretion] {} {{C} \bc \langle Q \rangle}
  \and
  \inferrule* [lab=process] {} {{P,Q} \bc M \;| \;P|Q \;|\; @{x}}
  \and
  \inferrule* [lab=name] {} {{x} \bc \quotep{P}}
\end{mathpar} 

Note that $\vec{x}$ (resp. $\vec{P}$) denotes a vector of names
(resp. processes) of length $|\vec{x}|$ (resp. $|\vec{P}|$). We adopt
the following useful abbreviations.

\begin{mathpar}
   x?(\vec{y}).P := x.(\vec{y})P \and  x\clift{\vec{P}} := x.\clift{\vec{P}}
   \and x!(y) := \lift{x}{\dropn{y}}
   \and \Pi_{i=0}^{n-1}P_i := P_0 | \ldots | P_{n-1}
\end{mathpar}

\subsubsection{Structural congruence}

\paragraph{Free and bound names and alpha-equivalence.} At the
core of structural equivalence is alpha-equivalence which identifies
process that are the same up to a change of variable. Formally, we
recognize the distinction between free and bound names. The free names
of a process, $\freenames{P}$, may be calculated recursively as
follows:

\begin{mathpar}
\freenames{\pzero} := \emptyset
  \and \\
  \freenames{x?(y).P} := \{ x \} \cup (\freenames{P} \setminus \{ y \})
  \and 
  \freenames{x!\langle P \rangle} := \{ x \} \cup \{ P \} 
  \and \\
  \freenames{P|Q} := \freenames{P} \cup \freenames{Q}
  \and \\
  \freenames{@{x}} := \{ x \}
\end{mathpar}

$\pi$
$\quotep{\pi}$

$\freenames{-} : \pi \to \mathcal{P}(\quotep{\pi})$

\begin{eqnarray*}
  \freenames{\pzero} & := & \emptyset \\
  \freenames{x?(y).P} & := & \{ x \} \cup (\freenames{P} \setminus \{ y \}) \\
  \freenames{x!\langle P \rangle} & := & \{ x \} \cup \{ P \} \\
  \freenames{P|Q} & := & \freenames{P} \cup \freenames{Q} \\
  \freenames{\dropn{x}} & := & \{ x \}
\end{eqnarray*}

The bound names of a process, $\boundnames{P}$, are those names occurring in $P$
that are not free. For example, in $x?(y).0$, the name $x$ is free, while $y$ is bound.

\begin{mathpar}
  \inferrule* [lab=monoidal-laws] {} { P|Q \equiv Q|P \and P|0 \equiv P \and P|(Q|R) \equiv (P|Q)|R }
\end{mathpar}

\begin{mathpar}
  \inferrule* [lab=alpha-equivalence] {} { (x)P \equiv (y)P\{y/x\} \and y \not\in \freenames{P} }
\end{mathpar}

\begin{definition}
Then two processes, $P,Q$, are alpha-equivalent if $P = Q\{\vec{y}/\vec{x}\}$ for
some $\vec{x} \in \boundnames{Q},\vec{y} \in \boundnames{P}$, where $Q\{\vec{y}/\vec{x}\}$
denotes the capture-avoiding substitution of $\vec{y}$ for $\vec{x}$ in $Q$.
\end{definition}

\begin{definition}
  The {\em structural congruence} \cite{SangiorgiWalker} , $\equiv$,
  between processes is the least congruence containing
  alpha-equivalence, satisfying the abelian monoid laws
  (associativity, commutativity and $\pzero$ as identity) for parallel
  composition $|$ and for summation $+$.
\end{definition}

\subsection{Name equivalence}

We take name equivalence, written $\nameeq$, to be the smallest
equivalence relation generated by the following rules.

\begin{mathpar}
\inferrule*[lab=Quote-drop]
{ }
{ \quotep{@{x}} \nameeq x }

\inferrule*[lab=Struct-equiv]
{ P \scong Q }
{ \quotep{P} \nameeq \quotep{Q} }
\end{mathpar}

The astute reader will have noticed that the mutual recursion of names
and processes imposes a mutual recursion on alpha-equivalence and
structural equivalence via name-equivalence. Fortunately, all of this
works out pleasantly and we may calculate in the natural way, free of
concern. The reader interested in the details is referred to the
appendix \ref{appendix:rho_details}.

\subsection{Substitution}

We use $\Proc$ for the set of processes, $\QProc$ for the set of
names, and $\id{\{}\vec{y} / \vec{x} \id{\}}$ to denote partial maps,
$s : \QProc \rightarrow \QProc$. A map, $s$ lifts, uniquely, to a map
on process terms, $\widehat{s} : \Proc \rightarrow \Proc$ by the
following equations.

\begin{mathpar}
  (0) \psubstp{Q}{P} := 0 \\
  (R \juxtap S) \psubstp{Q}{P}
  :=    
  (R)\psubstp{Q}{P} \juxtap (S) \psubstp{Q}{P} \\
  (x?(y).R) \psubstp{Q}{P}    
  :=    
  (x)\substp{Q}{P} (z)\concat( (R \psubstn{z}{y}) \psubstp{Q}{P} ) \\
  (\lift{x}{R}) \psubstp{Q}{P}  
  :=
  \lift{(x)\substp{Q}{P}}{ R \psubstp{Q}{P} } \\
%   (\dropn{x})  \psubstp{Q}{P}       
%   := 
%   \left\{ 
%     \begin{array}{ccc} 
%       \dropn{\quotep{Q}} & & x \nameeq \quotep{P} \\
%       \dropn{x} & & otherwise \\
%     \end{array}
%   \right. 
  (\dropn{x})  \psubstp{Q}{P}       
  := 
  \left\{ 
    \begin{array}{ccc} 
      Q & & x \nameeq \quotep{P} \\
      \dropn{x} & & otherwise \\
    \end{array}
  \right.
\end{mathpar}
 

where

\begin{eqnarray}
  (x)\id{\{} \lpquote Q \rpquote / \lpquote P \rpquote \id{\}}            = 
  \left\{ 
    \begin{array}{ccc}
      \lpquote Q \rpquote & & x \nameeq \lpquote P \rpquote \\
      x & & otherwise \\
    \end{array}
  \right. \nonumber
\end{eqnarray}

and $z$ is chosen distinct from $\quotep{P}$, $\quotep{Q}$, the free
names in $Q$, and all the names in $R$. Our $\alpha$-equivalence will
be built in the standard way from this substitution.

\begin{remark}\label{rem:no_self_referential_names}
  One consequence of these definitions is that $\forall P. \quotep{P}
  \not\in \freenames{P}$.
\end{remark}

\subsection{ Dynamic quote: an example }

Anticipating something of what's to come, consider applying the
substitution, $\widehat{\id{\{}u / z \id{\}}}$, to the following pair
of processes, $\lift{w}{y!(z)}$ and $w[ \lpquote y!(z) \rpquote ]$.

\begin{eqnarray}
	\lift{w}{y!(z)}\widehat{\id{\{}u / z \id{\}}}
		& = &
		\lift{w}{y!(u)} \nonumber\\
	w[ \lpquote y!(z) \rpquote ] \widehat{ \id{\{}u / z \id{\}} }
		& = &
		w[ \lpquote y!(z) \rpquote ] \nonumber
\end{eqnarray}

Because the body of the process between quotes is impervious to
substitution, we get radically different answers. In fact, by
examining the first process in an input context,
e.g. $x?(z).\lift{w}{y!(z)}$, we see that the process under the lift
operator may be shaped by prefixed inputs binding a name inside it. In
this sense, the lift operator will be seen as a way to dynamically
construct processes before reifying them as names.

Finally equipped with these standard features we can present the
dynamics of the calculus.

\subsubsection{Operational semantics} 

Finally, we introduce the computational dynamics. What marks these
algebras as distinct from other more traditionally studied algebraic
structures, e.g. vector spaces or polynomial rings, is the manner in
which dynamics is captured. In traditional structures, dynamics is typically
expressed through morphisms between such structures, as in linear maps
between vector spaces or morphisms between rings. In algebras
associated with the semantics of computation, the dynamics is
expressed as part of the algebraic structure itself, through a
reduction reduction relation typically denoted by $\red$. Below, we
give a recursive presentation of this relation for the calculus used
in the encoding.

$\red \subseteq \pi \times \pi$
$\red : \pi \to \mathcal{P}(\pi)$

\begin{mathpar}
  \inferrule* [lab=Comm] { \textsf{match}( x_{src}, x_{trgt} ) } { x_{trgt}?(y)P \; | \; x_{src}!\langle {Q} \rangle \red P\{\quotep{Q}/y}\} }
  \and \\
  \inferrule* [lab=Par] {{P} \red {P}'} {{{P} | {Q}} \red {{P}' | {Q}}}
  \and
  \inferrule* [lab=Equiv]{{{P} \scong {P}'} \andalso {{P}' \red {Q}'} \andalso {{Q}' \scong {Q}}}{{P} \red {Q}}
\end{mathpar}

\begin{eqnarray*}
  match_{\equiv} (\quotep{P},\quotep{Q}) & := & P \equiv Q \\
  match_{\dagger}(\quotep{P},\quotep{Q}) & := & \forall R. P|Q \red^{*} R => R \red^{*} 0 \\
  match_{K}(\quotep{P},\quotep{Q}) & := & K \mbox{ for some context } K
\end{eqnarray*}

$u?(x)P | u!\langle Q \rangle \red P\{\quotep{Q}/x\}$

%We write $\wred$ for $\red^*$, and $P\red$ if $\exists Q $ such that $ P \red Q$.
We write $P\red$ if $\exists Q $ such that $ P \red Q$ and $P\not\red$, otherwise.

\section{Replication}

As mentioned before, it is known that replication (and hence
recursion) can be implemented in a higher-order process algebra
\cite{SangiorgiWalker}. As our first example of calculation with the
machinery thus far presented we give the construction explicitly in
the {\rhoc}.

\begin{eqnarray}
	D_{x} & := & \prefix{x}{y}{(\binpar{\outputp{x}{y}}{@{y}})} \nonumber\\
	\bangp_{x}{P} & := & \binpar{{x}!\langle{\binpar{D_{x}}{P}}\rangle}{D_{x}} \nonumber
\end{eqnarray}

\begin{eqnarray}
	\bangp_{x}{P} & & \nonumber\\
	=
	& {x}!\langle{(\prefix{x}{y}{(\outputp{x}{y} | @{y})) | P}}\rangle 
	      | \prefix{x}{y}{(\outputp{x}{y} | @{y})} & \nonumber\\
	\red
	& (\outputp{x}{y} | @{y})\substn{\quotep{(\prefix{x}{y}{(@{y} | \outputp{x}{y})) | P}}}{y} & \nonumber\\
	=
	& \outputp{x}{\quotep{(\prefix{x}{y}{(\outputp{x}{y} | @{y})) | P}}}
	  | {(\prefix{x}{y}{(\outputp{x}{y} | @{y})) | P}} & \nonumber\\
	\red
	& \ldots & \nonumber\\
	\red^*
	& P | P | \ldots & \nonumber
\end{eqnarray}

Of course, this encoding, as an implementation, runs away, unfolding
$\bangp{P}$ eagerly. A lazier and more implementable replication
operator, restricted to input-guarded processes, may be obtained as follows.

\begin{eqnarray}
\bangp{\prefix{u}{v}{P}} 
	:= 
	\binpar{\lift{x}{\prefix{u}{v}{(\binpar{D(x)}{P})}}}{D(x)} \nonumber
\end{eqnarray}

\begin{remark}
  Note that the lazier definition still does not deal with summation
  or mixed summation (i.e. sums over input and output). The reader is
  invited to construct definitions of replication that deal with these
  features. 

  Further, the definitions are parameterized in a name, $x$. Can you,
  gentle reader, make a definition that eliminates this parameter and
  guarantees no accidental interaction between the replication
  machinery and the process being replicated -- i.e. no accidental
  sharing of names used by the process to get its work done and the
  name(s) used by the replication to effect copying. This latter
  revision of the definition of replication is crucial to obtaining
  the expected identity $!!P \sim !P$.
\end{remark}

\begin{remark}\label{rem:paradoxical_combinator}
  The reader familiar with the lambda calculus will have noticed the
  similarity between $D$ and the paradoxical combinator.

  [Ed. note: the existence of this seems to suggest we have to be more
  restrictive on the set of processes and names we admit if we are to
  support no-cloning.]
\end{remark}

\subsubsection{Bisimulation}

The computational dynamics gives rise to another kind of equivalence,
the equivalence of computational behavior. As previously mentioned
this is typically captured \emph{via} some form of bisimulation.

% The notion we use in this paper is weak barbed bisimulation
% \cite{milner91polyadicpi}.

The notion we use in this paper is derived from weak barbed
bisimulation \cite{milner91polyadicpi}. 

\begin{definition}
An \emph{observation relation}, $\downarrow_{\mathcal N}$, over a set
of names, $\mathcal N$, is the smallest relation satisfying the rules
below.

\infrule[Out-barb]{y \in {\mathcal N}, \; x \nameeq y}
		  {\outputp{x}{v} \downarrow_{\mathcal N} x}
\infrule[Par-barb]{\mbox{$P\downarrow_{\mathcal N} x$ or $Q\downarrow_{\mathcal N} x$}}
		  {\binpar{P}{Q} \downarrow_{\mathcal N} x}

We write $P \Downarrow_{\mathcal N} x$ if there is $Q$ such that 
$P \wred Q$ and $Q \downarrow_{\mathcal N} x$.
\end{definition}

\begin{definition}
%\label{def.bbisim}
An  ${\mathcal N}$-\emph{barbed bisimulation} over a set of names, ${\mathcal N}$, is a symmetric binary relation 
${\mathcal S}_{\mathcal N}$ between agents such that $P\rel{S}_{\mathcal N}Q$ implies:
\begin{enumerate}
\item If $P \red P'$ then $Q \wred Q'$ and $P'\rel{S}_{\mathcal N} Q'$.
\item If $P\downarrow_{\mathcal N} x$, then $Q\Downarrow_{\mathcal N} x$.
\end{enumerate}
$P$ is ${\mathcal N}$-barbed bisimilar to $Q$, written
$P \wbbisim_{\mathcal N} Q$, if $P \rel{S}_{\mathcal N} Q$ for some ${\mathcal N}$-barbed bisimulation ${\mathcal S}_{\mathcal N}$.
\end{definition}

$\mathcal{R} \subseteq \pi \times \pi$

$P \mathcal{R} Q => \forall P'. P \red P' \Rightarrow \exists Q'. Q \red Q', P' \mathcal{R} Q'$

$P \vdash x \Rightarrow Q \vdash x$

\begin{mathpar}
  \inferrule*[lab=Out-barb]{x \nameeq y}{{y}!\langle{Q}\rangle \vdash x}
  \and
  \inferrule*[lab=Par-barb]{\mbox{$P\vdash x$ or $Q\vdash x$}}{\binpar{P}{Q} \vdash x}
\end{mathpar}

\subsubsection{Contexts}

One of the principle advantages of computational calculi like the
$\pi$-calculus is a well-defined notion of context,
contextual-equivalence and a correlation between
contextual-equivalence and notions of bisimulation. The notion of
context allows the decomposition of a process into (sub-)process and
its syntactic environment, its context. Thus, a context may be
thought of as a process with a ``hole'' (written $\Box$) in it. The
application of a context $M$ to a process $P$, written $M[P]$, is
tantamount to filling the hole in $M$ with $P$. In this paper we do
not need the full weight of this theory, but do make use of the notion
of context in the proof the main theorem. 

\begin{mathpar}
  \inferrule* [lab=summation] {} {{M_{M},M_{N}} \bc \Box \;|\; x.M_{A} \;|\; M_{M}+M_{N}}
  \and
  \inferrule* [lab=agent] {} {{M_{A}} \bc (\vec{x})M_{P} \;| \; \clift{P_0,\ldots,M_{P},\ldots,P_N}}
  \and \\
  \inferrule* [lab=process] {} {{M_{P}} \bc M_{N} \;| \;P|M_{P} }
\end{mathpar} 

\begin{mathpar}
  \inferrule* [lab=sychronization] {} {M_{N} \bc \Box \;|\; x?M_{F} \;|\; x!M_{C}}
  \and
  \inferrule* [lab=abstraction] {} {{M_{F}} \bc (x)M_{P} }
  \and
  \inferrule* [lab=concretion] {} {{M_{C}} \bc \langle M_{P} \rangle }
  \and \\
  \inferrule* [lab=process] {} {{M_{P}} \bc M_{N} \;| \;P|M_{P} }
\end{mathpar}

\begin{definition}[contextual application] Given a context $M$, and
  process $P$, we define the \emph{contextual application}, $M[P] :=
  M\{P/\Box\}$. That is, the contextual application of M to P is the
  substitution of $P$ for $\Box$ in $M$.
\end{definition}

$\meaningof{-} : L \to \mathcal{P}(\pi)$

\begin{mathpar}
  \inferrule* [lab=collection] {} {\meaningof{true} = \pi, \and \meaningof{~E} = \pi \setminus \meaningof{E}, \and \meaningof{E_{1} \& E_{2}} = \meaningof{E_{1}} \cap \meaningof{E_{2}}}
\end{mathpar}

\begin{mathpar}
  \inferrule* [lab=structure] {} {\meaningof{0} = \{ P \in \pi | P \equiv 0 \}, \and \\ \meaningof{E_1 | E_2} = \{ P \in \pi | P \equiv P_{1} | P_{2}, P_{1} \in \meaningof{E_{1}}, P_{2} \in \meaningof{E_2}\} }
\end{mathpar}

\begin{mathpar}
 \inferrule* [lab=behavior] {} {\meaningof{\langle a?b \rangle E} = \{ P \in \pi | P \equiv Q | u?(y)P', \\ \and \\\\ \and \\ \;\;\; u \in \meaningof{a}, \forall z.P'\{z/y\} \in \meaningof{E\{z/b\}}\}, \and \\ \meaningof{a!E} = \{ P \in \pi | P \equiv Q | x!\langle P' \rangle, x \in \meaningof{a} P' \in \meaningof{E}\} }
\end{mathpar}

\begin{mathpar}
 \inferrule* [lab=nominal] {} {\meaningof{\quotep{E}} = \{ \quotep{P} \in \quotep{\pi} | P \in \meaningof{E} \}, \and \meaningof{\quotep{P}} = \{ \quotep{Q} \in \quotep{\pi} | P \equiv Q \} \and \\ \meaningof{@\quotep{E}} = \{ P \in \pi | P \equiv @x, x \in \meaningof{E} \}}
\end{mathpar}

\begin{eqnarray*}
  \\
  \meaningof{-} : TS \to ST
\end{eqnarray*}

\begin{eqnarray*}
  \\
  L : TS \to ST
\end{eqnarray*}

\begin{eqnarray*}
  \\
  P \models E \iff P \in \meaningof{E}
\end{eqnarray*}

\begin{eqnarray*}
  P \approx_{L} Q \iff \forall E \in L. P \models E \iff Q \models E
\end{eqnarray*}

\begin{eqnarray*}
  P \approx_{K} Q
\end{eqnarray*}

\begin{eqnarray*}
  P \approx Q
\end{eqnarray*}

$\approx_{K} = \approx = \approx_{L}$

\subsubsection{Contextual duality}

Note that contexts extend the quotation operation to a family of
operations from processes to names. Given a context, $M$, we can
define a \emph{nominal context}, $\quotep{M}$ by $\quotep{M}[P] :=
\quotep{M[P]}$. To foreshadow what is to come we observe that these
operations enjoy a duality with processes very much like the duality
between vectors and maps from vectors to scalars.

Further, because the calculus is essentially higher-order, we have a
correspondence between contexts and processes. More specifically,
given a name $x$ and a context $M$ we can construct $M^{*}_{x}$ such
that 

\begin{mathpar}
  M^{*}_{x} | \lift{x}{P} \red M[P]
\end{mathpar}

namely,

\begin{mathpar}
  M^{*}_{x} := x?(u).M[\dropn{u}]
\end{mathpar}

The dependence of $M^{*}_{x}$ on a name makes it an abstraction, 

\begin{mathpar}
  M^{*} := (x)x?(u).M[\dropn{u}]
\end{mathpar}

\subsection{Additional notation}

It will sometimes be convenient to denote the process a name
quotes. We already have the notation $x = \quotep{P}$, but it will be
convenient to introduce an alternate notation, $\procn{x}$, when we
want to emphasize the connection to the use of the name. Note that, by
virtue of name equivalence, $\quotep{\procn{x}} \nameeq x$; so, the
notation is consistent with previous definitions.

Further, because names have structure it is possible to effect
substitutions on the basis of that structure. This means we need to
upgrade our notation for substitutions, which we accomplish by
adapting comprehension notation. Thus,

\begin{mathpar}
  P\{ y / x : x \in S \}
\end{mathpar}

is interpreted to mean the process derived from P by replacing (in a
capture-avoiding manner) each occurrence of $x$ in $S$ by $y$. For example,

\begin{mathpar}
  P\{ \quotep{\procn{x}|\procn{x}} / x : x \in \freenames{P} \}
\end{mathpar}

will replace each (occurrence) of a free name $x$ in $P$ by
$\quotep{\procn{x}|\procn{x}}$.

Also, we will avail ourselves of the notation $x^{L}$ and $x^{R}$ to
denote injections of a name into disjoint copies of the name
space. There are numerous ways to accomplish this. One example can be
found in \cite{MeredithR05}. This notation overloads to vectors of
names: $\vec{x}^{\pi} := (x_{i}^{\pi} \; : \; 0 \leq i < |\vec{x}| )$ where $\pi \in \{L,R\}$.

We also use $P^{\Box} := P|\Box$.

In \cite{MeredithR05} an interpretation of the new operator is
given. It turns out that there are several possible interpretations
all enjoying the requisite algebraic properties of the operator (see
\cite{milner91polyadicpi}). We will therefore make liberal use of
$(\nu\; \vec{x})P$.

% subsection the_syntax_and_semantics_of_the_notation_system (end)   

\input{qm2pi.qmops} 

\input{qm2pi.sterngerlach} 

\input{qm2pi.metric} 

% section concurrent_process_calculi (end)

%\input{qm2pi.proofsketch}

% section proof sketch (end)

%\input{qm2pi.slviaknots} 

% section spatial logic via knots (end)

\input{qm2pi.conclusion}

% section conclusion (end)

%\input{qm2pi.dtcodes} 

% section wiring algorithm (end)

\input{qm2pi.ack} 

% section acknowledgments (end)

\newpage


\bibliographystyle{plain}   
\bibliography{../../biblios/main.bib}

\input{qm2pi.rhodetails}

\end{document}

 

% section acknowledgments (end)

\newpage


\bibliographystyle{plain}   
\bibliography{../../biblios/main.bib}

\documentclass[12pt]{llncs}
%\documentclass{jktr}

\usepackage[pdftex]{hyperref}                   
\usepackage {listings}
\usepackage {mathpartir}
\usepackage{bcprules}
%\usepackage{listings}
                       
\usepackage{graphicx} 
%\usepackage[margins=2.5cm,nohead,nofoot]{geometry}
%\usepackage{geometry}
\usepackage{amsfonts}
\usepackage{amstext}
\usepackage{latexsym}
\usepackage{amssymb}
\usepackage{color}


%\include{myPreamble}
\include{qm2pi.local} 

%\ifpdf
%\usepackage[pdftex]{graphicx}
%\else
%\usepackage{graphicx}
%\fi

 % \ifpdf
%  \usepackage{pdfsync}
%  \if


%\title{Brief Article}
%\author{David F. Snyder}
%\author{L.G. Meredith}

%\address{Dept. of Math., Texas State University--San Marcos, San Marcos, TX 78666}
       
\pagestyle{empty}


\begin{document}

\lstset{language=[Objective]Caml,frame=shadowbox}

\input{qm2pi.front}

% section front matter (end)

\input{qm2pi.intro} 
 
% section introduction (end)

% \input{qm2pi.knotations} 

% section notation (end)

\input{qm2pi.process.calculi} 

% section concurrent_process_calculi_and_spatial_logics_ (end)
    
%\input{qm2pi.knots2pi} 

%\input{qm2pi.trefoil} 

%\input{qm2pi.mainthm} 

% subsection basic_interpretation (end)

%\input{qm2pi.rho.presentation} 
\subsection{The syntax and semantics of the notation system}\label{sub:the_syntax_and_semantics_of_the_notation_system} % (fold)

We now summarize a technical presentation of the calculus that
embodies our theory of dynamics. The typical presentation of such a
calculus follows the style of giving generators and relations on
them. The grammar, below, describing term constructors, freely
generates the set of processes, $\Proc$. This set is then quotiented
by a relation known as structural congruence and it is over this set
that the notion of dynamics is expressed. This presentation is
essentially that of \cite{MeredithR05} with the addition of
polyadicity and summation. For readability we have relegated some of
the technical subtleties to an appendix.

\subsubsection{Process grammar}\label{subsub:process_grammar}

\begin{mathpar}
  \inferrule* [lab=synchronization] {} {{M} \bc \pzero \;|\; x?F \;|\; x!C }
  \and
  \inferrule* [lab=abstraction] {} {{F} \bc (x)P}
  \and
  \inferrule* [lab=concretion] {} {{C} \bc \langle Q \rangle}
  \and
  \inferrule* [lab=process] {} {{P,Q} \bc M \;| \;P|Q \;|\; @{x}}
  \and
  \inferrule* [lab=name] {} {{x} \bc \quotep{P}}
\end{mathpar} 

Note that $\vec{x}$ (resp. $\vec{P}$) denotes a vector of names
(resp. processes) of length $|\vec{x}|$ (resp. $|\vec{P}|$). We adopt
the following useful abbreviations.

\begin{mathpar}
   x?(\vec{y}).P := x.(\vec{y})P \and  x\clift{\vec{P}} := x.\clift{\vec{P}}
   \and x!(y) := \lift{x}{\dropn{y}}
   \and \Pi_{i=0}^{n-1}P_i := P_0 | \ldots | P_{n-1}
\end{mathpar}

\subsubsection{Structural congruence}

\paragraph{Free and bound names and alpha-equivalence.} At the
core of structural equivalence is alpha-equivalence which identifies
process that are the same up to a change of variable. Formally, we
recognize the distinction between free and bound names. The free names
of a process, $\freenames{P}$, may be calculated recursively as
follows:

\begin{mathpar}
\freenames{\pzero} := \emptyset
  \and \\
  \freenames{x?(y).P} := \{ x \} \cup (\freenames{P} \setminus \{ y \})
  \and 
  \freenames{x!\langle P \rangle} := \{ x \} \cup \{ P \} 
  \and \\
  \freenames{P|Q} := \freenames{P} \cup \freenames{Q}
  \and \\
  \freenames{@{x}} := \{ x \}
\end{mathpar}

$\pi$
$\quotep{\pi}$

$\freenames{-} : \pi \to \mathcal{P}(\quotep{\pi})$

\begin{eqnarray*}
  \freenames{\pzero} & := & \emptyset \\
  \freenames{x?(y).P} & := & \{ x \} \cup (\freenames{P} \setminus \{ y \}) \\
  \freenames{x!\langle P \rangle} & := & \{ x \} \cup \{ P \} \\
  \freenames{P|Q} & := & \freenames{P} \cup \freenames{Q} \\
  \freenames{\dropn{x}} & := & \{ x \}
\end{eqnarray*}

The bound names of a process, $\boundnames{P}$, are those names occurring in $P$
that are not free. For example, in $x?(y).0$, the name $x$ is free, while $y$ is bound.

\begin{mathpar}
  \inferrule* [lab=monoidal-laws] {} { P|Q \equiv Q|P \and P|0 \equiv P \and P|(Q|R) \equiv (P|Q)|R }
\end{mathpar}

\begin{mathpar}
  \inferrule* [lab=alpha-equivalence] {} { (x)P \equiv (y)P\{y/x\} \and y \not\in \freenames{P} }
\end{mathpar}

\begin{definition}
Then two processes, $P,Q$, are alpha-equivalent if $P = Q\{\vec{y}/\vec{x}\}$ for
some $\vec{x} \in \boundnames{Q},\vec{y} \in \boundnames{P}$, where $Q\{\vec{y}/\vec{x}\}$
denotes the capture-avoiding substitution of $\vec{y}$ for $\vec{x}$ in $Q$.
\end{definition}

\begin{definition}
  The {\em structural congruence} \cite{SangiorgiWalker} , $\equiv$,
  between processes is the least congruence containing
  alpha-equivalence, satisfying the abelian monoid laws
  (associativity, commutativity and $\pzero$ as identity) for parallel
  composition $|$ and for summation $+$.
\end{definition}

\subsection{Name equivalence}

We take name equivalence, written $\nameeq$, to be the smallest
equivalence relation generated by the following rules.

\begin{mathpar}
\inferrule*[lab=Quote-drop]
{ }
{ \quotep{@{x}} \nameeq x }

\inferrule*[lab=Struct-equiv]
{ P \scong Q }
{ \quotep{P} \nameeq \quotep{Q} }
\end{mathpar}

The astute reader will have noticed that the mutual recursion of names
and processes imposes a mutual recursion on alpha-equivalence and
structural equivalence via name-equivalence. Fortunately, all of this
works out pleasantly and we may calculate in the natural way, free of
concern. The reader interested in the details is referred to the
appendix \ref{appendix:rho_details}.

\subsection{Substitution}

We use $\Proc$ for the set of processes, $\QProc$ for the set of
names, and $\id{\{}\vec{y} / \vec{x} \id{\}}$ to denote partial maps,
$s : \QProc \rightarrow \QProc$. A map, $s$ lifts, uniquely, to a map
on process terms, $\widehat{s} : \Proc \rightarrow \Proc$ by the
following equations.

\begin{mathpar}
  (0) \psubstp{Q}{P} := 0 \\
  (R \juxtap S) \psubstp{Q}{P}
  :=    
  (R)\psubstp{Q}{P} \juxtap (S) \psubstp{Q}{P} \\
  (x?(y).R) \psubstp{Q}{P}    
  :=    
  (x)\substp{Q}{P} (z)\concat( (R \psubstn{z}{y}) \psubstp{Q}{P} ) \\
  (\lift{x}{R}) \psubstp{Q}{P}  
  :=
  \lift{(x)\substp{Q}{P}}{ R \psubstp{Q}{P} } \\
%   (\dropn{x})  \psubstp{Q}{P}       
%   := 
%   \left\{ 
%     \begin{array}{ccc} 
%       \dropn{\quotep{Q}} & & x \nameeq \quotep{P} \\
%       \dropn{x} & & otherwise \\
%     \end{array}
%   \right. 
  (\dropn{x})  \psubstp{Q}{P}       
  := 
  \left\{ 
    \begin{array}{ccc} 
      Q & & x \nameeq \quotep{P} \\
      \dropn{x} & & otherwise \\
    \end{array}
  \right.
\end{mathpar}
 

where

\begin{eqnarray}
  (x)\id{\{} \lpquote Q \rpquote / \lpquote P \rpquote \id{\}}            = 
  \left\{ 
    \begin{array}{ccc}
      \lpquote Q \rpquote & & x \nameeq \lpquote P \rpquote \\
      x & & otherwise \\
    \end{array}
  \right. \nonumber
\end{eqnarray}

and $z$ is chosen distinct from $\quotep{P}$, $\quotep{Q}$, the free
names in $Q$, and all the names in $R$. Our $\alpha$-equivalence will
be built in the standard way from this substitution.

\begin{remark}\label{rem:no_self_referential_names}
  One consequence of these definitions is that $\forall P. \quotep{P}
  \not\in \freenames{P}$.
\end{remark}

\subsection{ Dynamic quote: an example }

Anticipating something of what's to come, consider applying the
substitution, $\widehat{\id{\{}u / z \id{\}}}$, to the following pair
of processes, $\lift{w}{y!(z)}$ and $w[ \lpquote y!(z) \rpquote ]$.

\begin{eqnarray}
	\lift{w}{y!(z)}\widehat{\id{\{}u / z \id{\}}}
		& = &
		\lift{w}{y!(u)} \nonumber\\
	w[ \lpquote y!(z) \rpquote ] \widehat{ \id{\{}u / z \id{\}} }
		& = &
		w[ \lpquote y!(z) \rpquote ] \nonumber
\end{eqnarray}

Because the body of the process between quotes is impervious to
substitution, we get radically different answers. In fact, by
examining the first process in an input context,
e.g. $x?(z).\lift{w}{y!(z)}$, we see that the process under the lift
operator may be shaped by prefixed inputs binding a name inside it. In
this sense, the lift operator will be seen as a way to dynamically
construct processes before reifying them as names.

Finally equipped with these standard features we can present the
dynamics of the calculus.

\subsubsection{Operational semantics} 

Finally, we introduce the computational dynamics. What marks these
algebras as distinct from other more traditionally studied algebraic
structures, e.g. vector spaces or polynomial rings, is the manner in
which dynamics is captured. In traditional structures, dynamics is typically
expressed through morphisms between such structures, as in linear maps
between vector spaces or morphisms between rings. In algebras
associated with the semantics of computation, the dynamics is
expressed as part of the algebraic structure itself, through a
reduction reduction relation typically denoted by $\red$. Below, we
give a recursive presentation of this relation for the calculus used
in the encoding.

$\red \subseteq \pi \times \pi$
$\red : \pi \to \mathcal{P}(\pi)$

\begin{mathpar}
  \inferrule* [lab=Comm] { \textsf{match}( x_{src}, x_{trgt} ) } { x_{trgt}?(y)P \; | \; x_{src}!\langle {Q} \rangle \red P\{\quotep{Q}/y}\} }
  \and \\
  \inferrule* [lab=Par] {{P} \red {P}'} {{{P} | {Q}} \red {{P}' | {Q}}}
  \and
  \inferrule* [lab=Equiv]{{{P} \scong {P}'} \andalso {{P}' \red {Q}'} \andalso {{Q}' \scong {Q}}}{{P} \red {Q}}
\end{mathpar}

\begin{eqnarray*}
  match_{\equiv} (\quotep{P},\quotep{Q}) & := & P \equiv Q \\
  match_{\dagger}(\quotep{P},\quotep{Q}) & := & \forall R. P|Q \red^{*} R => R \red^{*} 0 \\
  match_{K}(\quotep{P},\quotep{Q}) & := & K \mbox{ for some context } K
\end{eqnarray*}

$u?(x)P | u!\langle Q \rangle \red P\{\quotep{Q}/x\}$

%We write $\wred$ for $\red^*$, and $P\red$ if $\exists Q $ such that $ P \red Q$.
We write $P\red$ if $\exists Q $ such that $ P \red Q$ and $P\not\red$, otherwise.

\section{Replication}

As mentioned before, it is known that replication (and hence
recursion) can be implemented in a higher-order process algebra
\cite{SangiorgiWalker}. As our first example of calculation with the
machinery thus far presented we give the construction explicitly in
the {\rhoc}.

\begin{eqnarray}
	D_{x} & := & \prefix{x}{y}{(\binpar{\outputp{x}{y}}{@{y}})} \nonumber\\
	\bangp_{x}{P} & := & \binpar{{x}!\langle{\binpar{D_{x}}{P}}\rangle}{D_{x}} \nonumber
\end{eqnarray}

\begin{eqnarray}
	\bangp_{x}{P} & & \nonumber\\
	=
	& {x}!\langle{(\prefix{x}{y}{(\outputp{x}{y} | @{y})) | P}}\rangle 
	      | \prefix{x}{y}{(\outputp{x}{y} | @{y})} & \nonumber\\
	\red
	& (\outputp{x}{y} | @{y})\substn{\quotep{(\prefix{x}{y}{(@{y} | \outputp{x}{y})) | P}}}{y} & \nonumber\\
	=
	& \outputp{x}{\quotep{(\prefix{x}{y}{(\outputp{x}{y} | @{y})) | P}}}
	  | {(\prefix{x}{y}{(\outputp{x}{y} | @{y})) | P}} & \nonumber\\
	\red
	& \ldots & \nonumber\\
	\red^*
	& P | P | \ldots & \nonumber
\end{eqnarray}

Of course, this encoding, as an implementation, runs away, unfolding
$\bangp{P}$ eagerly. A lazier and more implementable replication
operator, restricted to input-guarded processes, may be obtained as follows.

\begin{eqnarray}
\bangp{\prefix{u}{v}{P}} 
	:= 
	\binpar{\lift{x}{\prefix{u}{v}{(\binpar{D(x)}{P})}}}{D(x)} \nonumber
\end{eqnarray}

\begin{remark}
  Note that the lazier definition still does not deal with summation
  or mixed summation (i.e. sums over input and output). The reader is
  invited to construct definitions of replication that deal with these
  features. 

  Further, the definitions are parameterized in a name, $x$. Can you,
  gentle reader, make a definition that eliminates this parameter and
  guarantees no accidental interaction between the replication
  machinery and the process being replicated -- i.e. no accidental
  sharing of names used by the process to get its work done and the
  name(s) used by the replication to effect copying. This latter
  revision of the definition of replication is crucial to obtaining
  the expected identity $!!P \sim !P$.
\end{remark}

\begin{remark}\label{rem:paradoxical_combinator}
  The reader familiar with the lambda calculus will have noticed the
  similarity between $D$ and the paradoxical combinator.

  [Ed. note: the existence of this seems to suggest we have to be more
  restrictive on the set of processes and names we admit if we are to
  support no-cloning.]
\end{remark}

\subsubsection{Bisimulation}

The computational dynamics gives rise to another kind of equivalence,
the equivalence of computational behavior. As previously mentioned
this is typically captured \emph{via} some form of bisimulation.

% The notion we use in this paper is weak barbed bisimulation
% \cite{milner91polyadicpi}.

The notion we use in this paper is derived from weak barbed
bisimulation \cite{milner91polyadicpi}. 

\begin{definition}
An \emph{observation relation}, $\downarrow_{\mathcal N}$, over a set
of names, $\mathcal N$, is the smallest relation satisfying the rules
below.

\infrule[Out-barb]{y \in {\mathcal N}, \; x \nameeq y}
		  {\outputp{x}{v} \downarrow_{\mathcal N} x}
\infrule[Par-barb]{\mbox{$P\downarrow_{\mathcal N} x$ or $Q\downarrow_{\mathcal N} x$}}
		  {\binpar{P}{Q} \downarrow_{\mathcal N} x}

We write $P \Downarrow_{\mathcal N} x$ if there is $Q$ such that 
$P \wred Q$ and $Q \downarrow_{\mathcal N} x$.
\end{definition}

\begin{definition}
%\label{def.bbisim}
An  ${\mathcal N}$-\emph{barbed bisimulation} over a set of names, ${\mathcal N}$, is a symmetric binary relation 
${\mathcal S}_{\mathcal N}$ between agents such that $P\rel{S}_{\mathcal N}Q$ implies:
\begin{enumerate}
\item If $P \red P'$ then $Q \wred Q'$ and $P'\rel{S}_{\mathcal N} Q'$.
\item If $P\downarrow_{\mathcal N} x$, then $Q\Downarrow_{\mathcal N} x$.
\end{enumerate}
$P$ is ${\mathcal N}$-barbed bisimilar to $Q$, written
$P \wbbisim_{\mathcal N} Q$, if $P \rel{S}_{\mathcal N} Q$ for some ${\mathcal N}$-barbed bisimulation ${\mathcal S}_{\mathcal N}$.
\end{definition}

$\mathcal{R} \subseteq \pi \times \pi$

$P \mathcal{R} Q => \forall P'. P \red P' \Rightarrow \exists Q'. Q \red Q', P' \mathcal{R} Q'$

$P \vdash x \Rightarrow Q \vdash x$

\begin{mathpar}
  \inferrule*[lab=Out-barb]{x \nameeq y}{{y}!\langle{Q}\rangle \vdash x}
  \and
  \inferrule*[lab=Par-barb]{\mbox{$P\vdash x$ or $Q\vdash x$}}{\binpar{P}{Q} \vdash x}
\end{mathpar}

\subsubsection{Contexts}

One of the principle advantages of computational calculi like the
$\pi$-calculus is a well-defined notion of context,
contextual-equivalence and a correlation between
contextual-equivalence and notions of bisimulation. The notion of
context allows the decomposition of a process into (sub-)process and
its syntactic environment, its context. Thus, a context may be
thought of as a process with a ``hole'' (written $\Box$) in it. The
application of a context $M$ to a process $P$, written $M[P]$, is
tantamount to filling the hole in $M$ with $P$. In this paper we do
not need the full weight of this theory, but do make use of the notion
of context in the proof the main theorem. 

\begin{mathpar}
  \inferrule* [lab=summation] {} {{M_{M},M_{N}} \bc \Box \;|\; x.M_{A} \;|\; M_{M}+M_{N}}
  \and
  \inferrule* [lab=agent] {} {{M_{A}} \bc (\vec{x})M_{P} \;| \; \clift{P_0,\ldots,M_{P},\ldots,P_N}}
  \and \\
  \inferrule* [lab=process] {} {{M_{P}} \bc M_{N} \;| \;P|M_{P} }
\end{mathpar} 

\begin{mathpar}
  \inferrule* [lab=sychronization] {} {M_{N} \bc \Box \;|\; x?M_{F} \;|\; x!M_{C}}
  \and
  \inferrule* [lab=abstraction] {} {{M_{F}} \bc (x)M_{P} }
  \and
  \inferrule* [lab=concretion] {} {{M_{C}} \bc \langle M_{P} \rangle }
  \and \\
  \inferrule* [lab=process] {} {{M_{P}} \bc M_{N} \;| \;P|M_{P} }
\end{mathpar}

\begin{definition}[contextual application] Given a context $M$, and
  process $P$, we define the \emph{contextual application}, $M[P] :=
  M\{P/\Box\}$. That is, the contextual application of M to P is the
  substitution of $P$ for $\Box$ in $M$.
\end{definition}

$\meaningof{-} : L \to \mathcal{P}(\pi)$

\begin{mathpar}
  \inferrule* [lab=collection] {} {\meaningof{true} = \pi, \and \meaningof{~E} = \pi \setminus \meaningof{E}, \and \meaningof{E_{1} \& E_{2}} = \meaningof{E_{1}} \cap \meaningof{E_{2}}}
\end{mathpar}

\begin{mathpar}
  \inferrule* [lab=structure] {} {\meaningof{0} = \{ P \in \pi | P \equiv 0 \}, \and \\ \meaningof{E_1 | E_2} = \{ P \in \pi | P \equiv P_{1} | P_{2}, P_{1} \in \meaningof{E_{1}}, P_{2} \in \meaningof{E_2}\} }
\end{mathpar}

\begin{mathpar}
 \inferrule* [lab=behavior] {} {\meaningof{\langle a?b \rangle E} = \{ P \in \pi | P \equiv Q | u?(y)P', \\ \and \\\\ \and \\ \;\;\; u \in \meaningof{a}, \forall z.P'\{z/y\} \in \meaningof{E\{z/b\}}\}, \and \\ \meaningof{a!E} = \{ P \in \pi | P \equiv Q | x!\langle P' \rangle, x \in \meaningof{a} P' \in \meaningof{E}\} }
\end{mathpar}

\begin{mathpar}
 \inferrule* [lab=nominal] {} {\meaningof{\quotep{E}} = \{ \quotep{P} \in \quotep{\pi} | P \in \meaningof{E} \}, \and \meaningof{\quotep{P}} = \{ \quotep{Q} \in \quotep{\pi} | P \equiv Q \} \and \\ \meaningof{@\quotep{E}} = \{ P \in \pi | P \equiv @x, x \in \meaningof{E} \}}
\end{mathpar}

\begin{eqnarray*}
  \\
  \meaningof{-} : TS \to ST
\end{eqnarray*}

\begin{eqnarray*}
  \\
  L : TS \to ST
\end{eqnarray*}

\begin{eqnarray*}
  \\
  P \models E \iff P \in \meaningof{E}
\end{eqnarray*}

\begin{eqnarray*}
  P \approx_{L} Q \iff \forall E \in L. P \models E \iff Q \models E
\end{eqnarray*}

\begin{eqnarray*}
  P \approx_{K} Q
\end{eqnarray*}

\begin{eqnarray*}
  P \approx Q
\end{eqnarray*}

$\approx_{K} = \approx = \approx_{L}$

\subsubsection{Contextual duality}

Note that contexts extend the quotation operation to a family of
operations from processes to names. Given a context, $M$, we can
define a \emph{nominal context}, $\quotep{M}$ by $\quotep{M}[P] :=
\quotep{M[P]}$. To foreshadow what is to come we observe that these
operations enjoy a duality with processes very much like the duality
between vectors and maps from vectors to scalars.

Further, because the calculus is essentially higher-order, we have a
correspondence between contexts and processes. More specifically,
given a name $x$ and a context $M$ we can construct $M^{*}_{x}$ such
that 

\begin{mathpar}
  M^{*}_{x} | \lift{x}{P} \red M[P]
\end{mathpar}

namely,

\begin{mathpar}
  M^{*}_{x} := x?(u).M[\dropn{u}]
\end{mathpar}

The dependence of $M^{*}_{x}$ on a name makes it an abstraction, 

\begin{mathpar}
  M^{*} := (x)x?(u).M[\dropn{u}]
\end{mathpar}

\subsection{Additional notation}

It will sometimes be convenient to denote the process a name
quotes. We already have the notation $x = \quotep{P}$, but it will be
convenient to introduce an alternate notation, $\procn{x}$, when we
want to emphasize the connection to the use of the name. Note that, by
virtue of name equivalence, $\quotep{\procn{x}} \nameeq x$; so, the
notation is consistent with previous definitions.

Further, because names have structure it is possible to effect
substitutions on the basis of that structure. This means we need to
upgrade our notation for substitutions, which we accomplish by
adapting comprehension notation. Thus,

\begin{mathpar}
  P\{ y / x : x \in S \}
\end{mathpar}

is interpreted to mean the process derived from P by replacing (in a
capture-avoiding manner) each occurrence of $x$ in $S$ by $y$. For example,

\begin{mathpar}
  P\{ \quotep{\procn{x}|\procn{x}} / x : x \in \freenames{P} \}
\end{mathpar}

will replace each (occurrence) of a free name $x$ in $P$ by
$\quotep{\procn{x}|\procn{x}}$.

Also, we will avail ourselves of the notation $x^{L}$ and $x^{R}$ to
denote injections of a name into disjoint copies of the name
space. There are numerous ways to accomplish this. One example can be
found in \cite{MeredithR05}. This notation overloads to vectors of
names: $\vec{x}^{\pi} := (x_{i}^{\pi} \; : \; 0 \leq i < |\vec{x}| )$ where $\pi \in \{L,R\}$.

We also use $P^{\Box} := P|\Box$.

In \cite{MeredithR05} an interpretation of the new operator is
given. It turns out that there are several possible interpretations
all enjoying the requisite algebraic properties of the operator (see
\cite{milner91polyadicpi}). We will therefore make liberal use of
$(\nu\; \vec{x})P$.

% subsection the_syntax_and_semantics_of_the_notation_system (end)   

\input{qm2pi.qmops} 

\input{qm2pi.sterngerlach} 

\input{qm2pi.metric} 

% section concurrent_process_calculi (end)

%\input{qm2pi.proofsketch}

% section proof sketch (end)

%\input{qm2pi.slviaknots} 

% section spatial logic via knots (end)

\input{qm2pi.conclusion}

% section conclusion (end)

%\input{qm2pi.dtcodes} 

% section wiring algorithm (end)

\input{qm2pi.ack} 

% section acknowledgments (end)

\newpage


\bibliographystyle{plain}   
\bibliography{../../biblios/main.bib}

\input{qm2pi.rhodetails}

\end{document}



\end{document}



\end{document}

 

%\documentclass[12pt]{llncs}
%\documentclass{jktr}

\usepackage[pdftex]{hyperref}                   
\usepackage {listings}
\usepackage {mathpartir}
\usepackage{bcprules}
%\usepackage{listings}
                       
\usepackage{graphicx} 
%\usepackage[margins=2.5cm,nohead,nofoot]{geometry}
%\usepackage{geometry}
\usepackage{amsfonts}
\usepackage{amstext}
\usepackage{latexsym}
\usepackage{amssymb}
\usepackage{color}


%\include{myPreamble}
\documentclass[12pt]{llncs}
%\documentclass{jktr}

\usepackage[pdftex]{hyperref}                   
\usepackage {listings}
\usepackage {mathpartir}
\usepackage{bcprules}
%\usepackage{listings}
                       
\usepackage{graphicx} 
%\usepackage[margins=2.5cm,nohead,nofoot]{geometry}
%\usepackage{geometry}
\usepackage{amsfonts}
\usepackage{amstext}
\usepackage{latexsym}
\usepackage{amssymb}
\usepackage{color}


%\include{myPreamble}
\documentclass[12pt]{llncs}
%\documentclass{jktr}

\usepackage[pdftex]{hyperref}                   
\usepackage {listings}
\usepackage {mathpartir}
\usepackage{bcprules}
%\usepackage{listings}
                       
\usepackage{graphicx} 
%\usepackage[margins=2.5cm,nohead,nofoot]{geometry}
%\usepackage{geometry}
\usepackage{amsfonts}
\usepackage{amstext}
\usepackage{latexsym}
\usepackage{amssymb}
\usepackage{color}


%\include{myPreamble}
\include{qm2pi.local} 

%\ifpdf
%\usepackage[pdftex]{graphicx}
%\else
%\usepackage{graphicx}
%\fi

 % \ifpdf
%  \usepackage{pdfsync}
%  \if


%\title{Brief Article}
%\author{David F. Snyder}
%\author{L.G. Meredith}

%\address{Dept. of Math., Texas State University--San Marcos, San Marcos, TX 78666}
       
\pagestyle{empty}


\begin{document}

\lstset{language=[Objective]Caml,frame=shadowbox}

\input{qm2pi.front}

% section front matter (end)

\input{qm2pi.intro} 
 
% section introduction (end)

% \input{qm2pi.knotations} 

% section notation (end)

\input{qm2pi.process.calculi} 

% section concurrent_process_calculi_and_spatial_logics_ (end)
    
%\input{qm2pi.knots2pi} 

%\input{qm2pi.trefoil} 

%\input{qm2pi.mainthm} 

% subsection basic_interpretation (end)

%\input{qm2pi.rho.presentation} 
\subsection{The syntax and semantics of the notation system}\label{sub:the_syntax_and_semantics_of_the_notation_system} % (fold)

We now summarize a technical presentation of the calculus that
embodies our theory of dynamics. The typical presentation of such a
calculus follows the style of giving generators and relations on
them. The grammar, below, describing term constructors, freely
generates the set of processes, $\Proc$. This set is then quotiented
by a relation known as structural congruence and it is over this set
that the notion of dynamics is expressed. This presentation is
essentially that of \cite{MeredithR05} with the addition of
polyadicity and summation. For readability we have relegated some of
the technical subtleties to an appendix.

\subsubsection{Process grammar}\label{subsub:process_grammar}

\begin{mathpar}
  \inferrule* [lab=synchronization] {} {{M} \bc \pzero \;|\; x?F \;|\; x!C }
  \and
  \inferrule* [lab=abstraction] {} {{F} \bc (x)P}
  \and
  \inferrule* [lab=concretion] {} {{C} \bc \langle Q \rangle}
  \and
  \inferrule* [lab=process] {} {{P,Q} \bc M \;| \;P|Q \;|\; @{x}}
  \and
  \inferrule* [lab=name] {} {{x} \bc \quotep{P}}
\end{mathpar} 

Note that $\vec{x}$ (resp. $\vec{P}$) denotes a vector of names
(resp. processes) of length $|\vec{x}|$ (resp. $|\vec{P}|$). We adopt
the following useful abbreviations.

\begin{mathpar}
   x?(\vec{y}).P := x.(\vec{y})P \and  x\clift{\vec{P}} := x.\clift{\vec{P}}
   \and x!(y) := \lift{x}{\dropn{y}}
   \and \Pi_{i=0}^{n-1}P_i := P_0 | \ldots | P_{n-1}
\end{mathpar}

\subsubsection{Structural congruence}

\paragraph{Free and bound names and alpha-equivalence.} At the
core of structural equivalence is alpha-equivalence which identifies
process that are the same up to a change of variable. Formally, we
recognize the distinction between free and bound names. The free names
of a process, $\freenames{P}$, may be calculated recursively as
follows:

\begin{mathpar}
\freenames{\pzero} := \emptyset
  \and \\
  \freenames{x?(y).P} := \{ x \} \cup (\freenames{P} \setminus \{ y \})
  \and 
  \freenames{x!\langle P \rangle} := \{ x \} \cup \{ P \} 
  \and \\
  \freenames{P|Q} := \freenames{P} \cup \freenames{Q}
  \and \\
  \freenames{@{x}} := \{ x \}
\end{mathpar}

$\pi$
$\quotep{\pi}$

$\freenames{-} : \pi \to \mathcal{P}(\quotep{\pi})$

\begin{eqnarray*}
  \freenames{\pzero} & := & \emptyset \\
  \freenames{x?(y).P} & := & \{ x \} \cup (\freenames{P} \setminus \{ y \}) \\
  \freenames{x!\langle P \rangle} & := & \{ x \} \cup \{ P \} \\
  \freenames{P|Q} & := & \freenames{P} \cup \freenames{Q} \\
  \freenames{\dropn{x}} & := & \{ x \}
\end{eqnarray*}

The bound names of a process, $\boundnames{P}$, are those names occurring in $P$
that are not free. For example, in $x?(y).0$, the name $x$ is free, while $y$ is bound.

\begin{mathpar}
  \inferrule* [lab=monoidal-laws] {} { P|Q \equiv Q|P \and P|0 \equiv P \and P|(Q|R) \equiv (P|Q)|R }
\end{mathpar}

\begin{mathpar}
  \inferrule* [lab=alpha-equivalence] {} { (x)P \equiv (y)P\{y/x\} \and y \not\in \freenames{P} }
\end{mathpar}

\begin{definition}
Then two processes, $P,Q$, are alpha-equivalent if $P = Q\{\vec{y}/\vec{x}\}$ for
some $\vec{x} \in \boundnames{Q},\vec{y} \in \boundnames{P}$, where $Q\{\vec{y}/\vec{x}\}$
denotes the capture-avoiding substitution of $\vec{y}$ for $\vec{x}$ in $Q$.
\end{definition}

\begin{definition}
  The {\em structural congruence} \cite{SangiorgiWalker} , $\equiv$,
  between processes is the least congruence containing
  alpha-equivalence, satisfying the abelian monoid laws
  (associativity, commutativity and $\pzero$ as identity) for parallel
  composition $|$ and for summation $+$.
\end{definition}

\subsection{Name equivalence}

We take name equivalence, written $\nameeq$, to be the smallest
equivalence relation generated by the following rules.

\begin{mathpar}
\inferrule*[lab=Quote-drop]
{ }
{ \quotep{@{x}} \nameeq x }

\inferrule*[lab=Struct-equiv]
{ P \scong Q }
{ \quotep{P} \nameeq \quotep{Q} }
\end{mathpar}

The astute reader will have noticed that the mutual recursion of names
and processes imposes a mutual recursion on alpha-equivalence and
structural equivalence via name-equivalence. Fortunately, all of this
works out pleasantly and we may calculate in the natural way, free of
concern. The reader interested in the details is referred to the
appendix \ref{appendix:rho_details}.

\subsection{Substitution}

We use $\Proc$ for the set of processes, $\QProc$ for the set of
names, and $\id{\{}\vec{y} / \vec{x} \id{\}}$ to denote partial maps,
$s : \QProc \rightarrow \QProc$. A map, $s$ lifts, uniquely, to a map
on process terms, $\widehat{s} : \Proc \rightarrow \Proc$ by the
following equations.

\begin{mathpar}
  (0) \psubstp{Q}{P} := 0 \\
  (R \juxtap S) \psubstp{Q}{P}
  :=    
  (R)\psubstp{Q}{P} \juxtap (S) \psubstp{Q}{P} \\
  (x?(y).R) \psubstp{Q}{P}    
  :=    
  (x)\substp{Q}{P} (z)\concat( (R \psubstn{z}{y}) \psubstp{Q}{P} ) \\
  (\lift{x}{R}) \psubstp{Q}{P}  
  :=
  \lift{(x)\substp{Q}{P}}{ R \psubstp{Q}{P} } \\
%   (\dropn{x})  \psubstp{Q}{P}       
%   := 
%   \left\{ 
%     \begin{array}{ccc} 
%       \dropn{\quotep{Q}} & & x \nameeq \quotep{P} \\
%       \dropn{x} & & otherwise \\
%     \end{array}
%   \right. 
  (\dropn{x})  \psubstp{Q}{P}       
  := 
  \left\{ 
    \begin{array}{ccc} 
      Q & & x \nameeq \quotep{P} \\
      \dropn{x} & & otherwise \\
    \end{array}
  \right.
\end{mathpar}
 

where

\begin{eqnarray}
  (x)\id{\{} \lpquote Q \rpquote / \lpquote P \rpquote \id{\}}            = 
  \left\{ 
    \begin{array}{ccc}
      \lpquote Q \rpquote & & x \nameeq \lpquote P \rpquote \\
      x & & otherwise \\
    \end{array}
  \right. \nonumber
\end{eqnarray}

and $z$ is chosen distinct from $\quotep{P}$, $\quotep{Q}$, the free
names in $Q$, and all the names in $R$. Our $\alpha$-equivalence will
be built in the standard way from this substitution.

\begin{remark}\label{rem:no_self_referential_names}
  One consequence of these definitions is that $\forall P. \quotep{P}
  \not\in \freenames{P}$.
\end{remark}

\subsection{ Dynamic quote: an example }

Anticipating something of what's to come, consider applying the
substitution, $\widehat{\id{\{}u / z \id{\}}}$, to the following pair
of processes, $\lift{w}{y!(z)}$ and $w[ \lpquote y!(z) \rpquote ]$.

\begin{eqnarray}
	\lift{w}{y!(z)}\widehat{\id{\{}u / z \id{\}}}
		& = &
		\lift{w}{y!(u)} \nonumber\\
	w[ \lpquote y!(z) \rpquote ] \widehat{ \id{\{}u / z \id{\}} }
		& = &
		w[ \lpquote y!(z) \rpquote ] \nonumber
\end{eqnarray}

Because the body of the process between quotes is impervious to
substitution, we get radically different answers. In fact, by
examining the first process in an input context,
e.g. $x?(z).\lift{w}{y!(z)}$, we see that the process under the lift
operator may be shaped by prefixed inputs binding a name inside it. In
this sense, the lift operator will be seen as a way to dynamically
construct processes before reifying them as names.

Finally equipped with these standard features we can present the
dynamics of the calculus.

\subsubsection{Operational semantics} 

Finally, we introduce the computational dynamics. What marks these
algebras as distinct from other more traditionally studied algebraic
structures, e.g. vector spaces or polynomial rings, is the manner in
which dynamics is captured. In traditional structures, dynamics is typically
expressed through morphisms between such structures, as in linear maps
between vector spaces or morphisms between rings. In algebras
associated with the semantics of computation, the dynamics is
expressed as part of the algebraic structure itself, through a
reduction reduction relation typically denoted by $\red$. Below, we
give a recursive presentation of this relation for the calculus used
in the encoding.

$\red \subseteq \pi \times \pi$
$\red : \pi \to \mathcal{P}(\pi)$

\begin{mathpar}
  \inferrule* [lab=Comm] { \textsf{match}( x_{src}, x_{trgt} ) } { x_{trgt}?(y)P \; | \; x_{src}!\langle {Q} \rangle \red P\{\quotep{Q}/y}\} }
  \and \\
  \inferrule* [lab=Par] {{P} \red {P}'} {{{P} | {Q}} \red {{P}' | {Q}}}
  \and
  \inferrule* [lab=Equiv]{{{P} \scong {P}'} \andalso {{P}' \red {Q}'} \andalso {{Q}' \scong {Q}}}{{P} \red {Q}}
\end{mathpar}

\begin{eqnarray*}
  match_{\equiv} (\quotep{P},\quotep{Q}) & := & P \equiv Q \\
  match_{\dagger}(\quotep{P},\quotep{Q}) & := & \forall R. P|Q \red^{*} R => R \red^{*} 0 \\
  match_{K}(\quotep{P},\quotep{Q}) & := & K \mbox{ for some context } K
\end{eqnarray*}

$u?(x)P | u!\langle Q \rangle \red P\{\quotep{Q}/x\}$

%We write $\wred$ for $\red^*$, and $P\red$ if $\exists Q $ such that $ P \red Q$.
We write $P\red$ if $\exists Q $ such that $ P \red Q$ and $P\not\red$, otherwise.

\section{Replication}

As mentioned before, it is known that replication (and hence
recursion) can be implemented in a higher-order process algebra
\cite{SangiorgiWalker}. As our first example of calculation with the
machinery thus far presented we give the construction explicitly in
the {\rhoc}.

\begin{eqnarray}
	D_{x} & := & \prefix{x}{y}{(\binpar{\outputp{x}{y}}{@{y}})} \nonumber\\
	\bangp_{x}{P} & := & \binpar{{x}!\langle{\binpar{D_{x}}{P}}\rangle}{D_{x}} \nonumber
\end{eqnarray}

\begin{eqnarray}
	\bangp_{x}{P} & & \nonumber\\
	=
	& {x}!\langle{(\prefix{x}{y}{(\outputp{x}{y} | @{y})) | P}}\rangle 
	      | \prefix{x}{y}{(\outputp{x}{y} | @{y})} & \nonumber\\
	\red
	& (\outputp{x}{y} | @{y})\substn{\quotep{(\prefix{x}{y}{(@{y} | \outputp{x}{y})) | P}}}{y} & \nonumber\\
	=
	& \outputp{x}{\quotep{(\prefix{x}{y}{(\outputp{x}{y} | @{y})) | P}}}
	  | {(\prefix{x}{y}{(\outputp{x}{y} | @{y})) | P}} & \nonumber\\
	\red
	& \ldots & \nonumber\\
	\red^*
	& P | P | \ldots & \nonumber
\end{eqnarray}

Of course, this encoding, as an implementation, runs away, unfolding
$\bangp{P}$ eagerly. A lazier and more implementable replication
operator, restricted to input-guarded processes, may be obtained as follows.

\begin{eqnarray}
\bangp{\prefix{u}{v}{P}} 
	:= 
	\binpar{\lift{x}{\prefix{u}{v}{(\binpar{D(x)}{P})}}}{D(x)} \nonumber
\end{eqnarray}

\begin{remark}
  Note that the lazier definition still does not deal with summation
  or mixed summation (i.e. sums over input and output). The reader is
  invited to construct definitions of replication that deal with these
  features. 

  Further, the definitions are parameterized in a name, $x$. Can you,
  gentle reader, make a definition that eliminates this parameter and
  guarantees no accidental interaction between the replication
  machinery and the process being replicated -- i.e. no accidental
  sharing of names used by the process to get its work done and the
  name(s) used by the replication to effect copying. This latter
  revision of the definition of replication is crucial to obtaining
  the expected identity $!!P \sim !P$.
\end{remark}

\begin{remark}\label{rem:paradoxical_combinator}
  The reader familiar with the lambda calculus will have noticed the
  similarity between $D$ and the paradoxical combinator.

  [Ed. note: the existence of this seems to suggest we have to be more
  restrictive on the set of processes and names we admit if we are to
  support no-cloning.]
\end{remark}

\subsubsection{Bisimulation}

The computational dynamics gives rise to another kind of equivalence,
the equivalence of computational behavior. As previously mentioned
this is typically captured \emph{via} some form of bisimulation.

% The notion we use in this paper is weak barbed bisimulation
% \cite{milner91polyadicpi}.

The notion we use in this paper is derived from weak barbed
bisimulation \cite{milner91polyadicpi}. 

\begin{definition}
An \emph{observation relation}, $\downarrow_{\mathcal N}$, over a set
of names, $\mathcal N$, is the smallest relation satisfying the rules
below.

\infrule[Out-barb]{y \in {\mathcal N}, \; x \nameeq y}
		  {\outputp{x}{v} \downarrow_{\mathcal N} x}
\infrule[Par-barb]{\mbox{$P\downarrow_{\mathcal N} x$ or $Q\downarrow_{\mathcal N} x$}}
		  {\binpar{P}{Q} \downarrow_{\mathcal N} x}

We write $P \Downarrow_{\mathcal N} x$ if there is $Q$ such that 
$P \wred Q$ and $Q \downarrow_{\mathcal N} x$.
\end{definition}

\begin{definition}
%\label{def.bbisim}
An  ${\mathcal N}$-\emph{barbed bisimulation} over a set of names, ${\mathcal N}$, is a symmetric binary relation 
${\mathcal S}_{\mathcal N}$ between agents such that $P\rel{S}_{\mathcal N}Q$ implies:
\begin{enumerate}
\item If $P \red P'$ then $Q \wred Q'$ and $P'\rel{S}_{\mathcal N} Q'$.
\item If $P\downarrow_{\mathcal N} x$, then $Q\Downarrow_{\mathcal N} x$.
\end{enumerate}
$P$ is ${\mathcal N}$-barbed bisimilar to $Q$, written
$P \wbbisim_{\mathcal N} Q$, if $P \rel{S}_{\mathcal N} Q$ for some ${\mathcal N}$-barbed bisimulation ${\mathcal S}_{\mathcal N}$.
\end{definition}

$\mathcal{R} \subseteq \pi \times \pi$

$P \mathcal{R} Q => \forall P'. P \red P' \Rightarrow \exists Q'. Q \red Q', P' \mathcal{R} Q'$

$P \vdash x \Rightarrow Q \vdash x$

\begin{mathpar}
  \inferrule*[lab=Out-barb]{x \nameeq y}{{y}!\langle{Q}\rangle \vdash x}
  \and
  \inferrule*[lab=Par-barb]{\mbox{$P\vdash x$ or $Q\vdash x$}}{\binpar{P}{Q} \vdash x}
\end{mathpar}

\subsubsection{Contexts}

One of the principle advantages of computational calculi like the
$\pi$-calculus is a well-defined notion of context,
contextual-equivalence and a correlation between
contextual-equivalence and notions of bisimulation. The notion of
context allows the decomposition of a process into (sub-)process and
its syntactic environment, its context. Thus, a context may be
thought of as a process with a ``hole'' (written $\Box$) in it. The
application of a context $M$ to a process $P$, written $M[P]$, is
tantamount to filling the hole in $M$ with $P$. In this paper we do
not need the full weight of this theory, but do make use of the notion
of context in the proof the main theorem. 

\begin{mathpar}
  \inferrule* [lab=summation] {} {{M_{M},M_{N}} \bc \Box \;|\; x.M_{A} \;|\; M_{M}+M_{N}}
  \and
  \inferrule* [lab=agent] {} {{M_{A}} \bc (\vec{x})M_{P} \;| \; \clift{P_0,\ldots,M_{P},\ldots,P_N}}
  \and \\
  \inferrule* [lab=process] {} {{M_{P}} \bc M_{N} \;| \;P|M_{P} }
\end{mathpar} 

\begin{mathpar}
  \inferrule* [lab=sychronization] {} {M_{N} \bc \Box \;|\; x?M_{F} \;|\; x!M_{C}}
  \and
  \inferrule* [lab=abstraction] {} {{M_{F}} \bc (x)M_{P} }
  \and
  \inferrule* [lab=concretion] {} {{M_{C}} \bc \langle M_{P} \rangle }
  \and \\
  \inferrule* [lab=process] {} {{M_{P}} \bc M_{N} \;| \;P|M_{P} }
\end{mathpar}

\begin{definition}[contextual application] Given a context $M$, and
  process $P$, we define the \emph{contextual application}, $M[P] :=
  M\{P/\Box\}$. That is, the contextual application of M to P is the
  substitution of $P$ for $\Box$ in $M$.
\end{definition}

$\meaningof{-} : L \to \mathcal{P}(\pi)$

\begin{mathpar}
  \inferrule* [lab=collection] {} {\meaningof{true} = \pi, \and \meaningof{~E} = \pi \setminus \meaningof{E}, \and \meaningof{E_{1} \& E_{2}} = \meaningof{E_{1}} \cap \meaningof{E_{2}}}
\end{mathpar}

\begin{mathpar}
  \inferrule* [lab=structure] {} {\meaningof{0} = \{ P \in \pi | P \equiv 0 \}, \and \\ \meaningof{E_1 | E_2} = \{ P \in \pi | P \equiv P_{1} | P_{2}, P_{1} \in \meaningof{E_{1}}, P_{2} \in \meaningof{E_2}\} }
\end{mathpar}

\begin{mathpar}
 \inferrule* [lab=behavior] {} {\meaningof{\langle a?b \rangle E} = \{ P \in \pi | P \equiv Q | u?(y)P', \\ \and \\\\ \and \\ \;\;\; u \in \meaningof{a}, \forall z.P'\{z/y\} \in \meaningof{E\{z/b\}}\}, \and \\ \meaningof{a!E} = \{ P \in \pi | P \equiv Q | x!\langle P' \rangle, x \in \meaningof{a} P' \in \meaningof{E}\} }
\end{mathpar}

\begin{mathpar}
 \inferrule* [lab=nominal] {} {\meaningof{\quotep{E}} = \{ \quotep{P} \in \quotep{\pi} | P \in \meaningof{E} \}, \and \meaningof{\quotep{P}} = \{ \quotep{Q} \in \quotep{\pi} | P \equiv Q \} \and \\ \meaningof{@\quotep{E}} = \{ P \in \pi | P \equiv @x, x \in \meaningof{E} \}}
\end{mathpar}

\begin{eqnarray*}
  \\
  \meaningof{-} : TS \to ST
\end{eqnarray*}

\begin{eqnarray*}
  \\
  L : TS \to ST
\end{eqnarray*}

\begin{eqnarray*}
  \\
  P \models E \iff P \in \meaningof{E}
\end{eqnarray*}

\begin{eqnarray*}
  P \approx_{L} Q \iff \forall E \in L. P \models E \iff Q \models E
\end{eqnarray*}

\begin{eqnarray*}
  P \approx_{K} Q
\end{eqnarray*}

\begin{eqnarray*}
  P \approx Q
\end{eqnarray*}

$\approx_{K} = \approx = \approx_{L}$

\subsubsection{Contextual duality}

Note that contexts extend the quotation operation to a family of
operations from processes to names. Given a context, $M$, we can
define a \emph{nominal context}, $\quotep{M}$ by $\quotep{M}[P] :=
\quotep{M[P]}$. To foreshadow what is to come we observe that these
operations enjoy a duality with processes very much like the duality
between vectors and maps from vectors to scalars.

Further, because the calculus is essentially higher-order, we have a
correspondence between contexts and processes. More specifically,
given a name $x$ and a context $M$ we can construct $M^{*}_{x}$ such
that 

\begin{mathpar}
  M^{*}_{x} | \lift{x}{P} \red M[P]
\end{mathpar}

namely,

\begin{mathpar}
  M^{*}_{x} := x?(u).M[\dropn{u}]
\end{mathpar}

The dependence of $M^{*}_{x}$ on a name makes it an abstraction, 

\begin{mathpar}
  M^{*} := (x)x?(u).M[\dropn{u}]
\end{mathpar}

\subsection{Additional notation}

It will sometimes be convenient to denote the process a name
quotes. We already have the notation $x = \quotep{P}$, but it will be
convenient to introduce an alternate notation, $\procn{x}$, when we
want to emphasize the connection to the use of the name. Note that, by
virtue of name equivalence, $\quotep{\procn{x}} \nameeq x$; so, the
notation is consistent with previous definitions.

Further, because names have structure it is possible to effect
substitutions on the basis of that structure. This means we need to
upgrade our notation for substitutions, which we accomplish by
adapting comprehension notation. Thus,

\begin{mathpar}
  P\{ y / x : x \in S \}
\end{mathpar}

is interpreted to mean the process derived from P by replacing (in a
capture-avoiding manner) each occurrence of $x$ in $S$ by $y$. For example,

\begin{mathpar}
  P\{ \quotep{\procn{x}|\procn{x}} / x : x \in \freenames{P} \}
\end{mathpar}

will replace each (occurrence) of a free name $x$ in $P$ by
$\quotep{\procn{x}|\procn{x}}$.

Also, we will avail ourselves of the notation $x^{L}$ and $x^{R}$ to
denote injections of a name into disjoint copies of the name
space. There are numerous ways to accomplish this. One example can be
found in \cite{MeredithR05}. This notation overloads to vectors of
names: $\vec{x}^{\pi} := (x_{i}^{\pi} \; : \; 0 \leq i < |\vec{x}| )$ where $\pi \in \{L,R\}$.

We also use $P^{\Box} := P|\Box$.

In \cite{MeredithR05} an interpretation of the new operator is
given. It turns out that there are several possible interpretations
all enjoying the requisite algebraic properties of the operator (see
\cite{milner91polyadicpi}). We will therefore make liberal use of
$(\nu\; \vec{x})P$.

% subsection the_syntax_and_semantics_of_the_notation_system (end)   

\input{qm2pi.qmops} 

\input{qm2pi.sterngerlach} 

\input{qm2pi.metric} 

% section concurrent_process_calculi (end)

%\input{qm2pi.proofsketch}

% section proof sketch (end)

%\input{qm2pi.slviaknots} 

% section spatial logic via knots (end)

\input{qm2pi.conclusion}

% section conclusion (end)

%\input{qm2pi.dtcodes} 

% section wiring algorithm (end)

\input{qm2pi.ack} 

% section acknowledgments (end)

\newpage


\bibliographystyle{plain}   
\bibliography{../../biblios/main.bib}

\input{qm2pi.rhodetails}

\end{document}

 

%\ifpdf
%\usepackage[pdftex]{graphicx}
%\else
%\usepackage{graphicx}
%\fi

 % \ifpdf
%  \usepackage{pdfsync}
%  \if


%\title{Brief Article}
%\author{David F. Snyder}
%\author{L.G. Meredith}

%\address{Dept. of Math., Texas State University--San Marcos, San Marcos, TX 78666}
       
\pagestyle{empty}


\begin{document}

\lstset{language=[Objective]Caml,frame=shadowbox}

\documentclass[12pt]{llncs}
%\documentclass{jktr}

\usepackage[pdftex]{hyperref}                   
\usepackage {listings}
\usepackage {mathpartir}
\usepackage{bcprules}
%\usepackage{listings}
                       
\usepackage{graphicx} 
%\usepackage[margins=2.5cm,nohead,nofoot]{geometry}
%\usepackage{geometry}
\usepackage{amsfonts}
\usepackage{amstext}
\usepackage{latexsym}
\usepackage{amssymb}
\usepackage{color}


%\include{myPreamble}
\include{qm2pi.local} 

%\ifpdf
%\usepackage[pdftex]{graphicx}
%\else
%\usepackage{graphicx}
%\fi

 % \ifpdf
%  \usepackage{pdfsync}
%  \if


%\title{Brief Article}
%\author{David F. Snyder}
%\author{L.G. Meredith}

%\address{Dept. of Math., Texas State University--San Marcos, San Marcos, TX 78666}
       
\pagestyle{empty}


\begin{document}

\lstset{language=[Objective]Caml,frame=shadowbox}

\input{qm2pi.front}

% section front matter (end)

\input{qm2pi.intro} 
 
% section introduction (end)

% \input{qm2pi.knotations} 

% section notation (end)

\input{qm2pi.process.calculi} 

% section concurrent_process_calculi_and_spatial_logics_ (end)
    
%\input{qm2pi.knots2pi} 

%\input{qm2pi.trefoil} 

%\input{qm2pi.mainthm} 

% subsection basic_interpretation (end)

%\input{qm2pi.rho.presentation} 
\subsection{The syntax and semantics of the notation system}\label{sub:the_syntax_and_semantics_of_the_notation_system} % (fold)

We now summarize a technical presentation of the calculus that
embodies our theory of dynamics. The typical presentation of such a
calculus follows the style of giving generators and relations on
them. The grammar, below, describing term constructors, freely
generates the set of processes, $\Proc$. This set is then quotiented
by a relation known as structural congruence and it is over this set
that the notion of dynamics is expressed. This presentation is
essentially that of \cite{MeredithR05} with the addition of
polyadicity and summation. For readability we have relegated some of
the technical subtleties to an appendix.

\subsubsection{Process grammar}\label{subsub:process_grammar}

\begin{mathpar}
  \inferrule* [lab=synchronization] {} {{M} \bc \pzero \;|\; x?F \;|\; x!C }
  \and
  \inferrule* [lab=abstraction] {} {{F} \bc (x)P}
  \and
  \inferrule* [lab=concretion] {} {{C} \bc \langle Q \rangle}
  \and
  \inferrule* [lab=process] {} {{P,Q} \bc M \;| \;P|Q \;|\; @{x}}
  \and
  \inferrule* [lab=name] {} {{x} \bc \quotep{P}}
\end{mathpar} 

Note that $\vec{x}$ (resp. $\vec{P}$) denotes a vector of names
(resp. processes) of length $|\vec{x}|$ (resp. $|\vec{P}|$). We adopt
the following useful abbreviations.

\begin{mathpar}
   x?(\vec{y}).P := x.(\vec{y})P \and  x\clift{\vec{P}} := x.\clift{\vec{P}}
   \and x!(y) := \lift{x}{\dropn{y}}
   \and \Pi_{i=0}^{n-1}P_i := P_0 | \ldots | P_{n-1}
\end{mathpar}

\subsubsection{Structural congruence}

\paragraph{Free and bound names and alpha-equivalence.} At the
core of structural equivalence is alpha-equivalence which identifies
process that are the same up to a change of variable. Formally, we
recognize the distinction between free and bound names. The free names
of a process, $\freenames{P}$, may be calculated recursively as
follows:

\begin{mathpar}
\freenames{\pzero} := \emptyset
  \and \\
  \freenames{x?(y).P} := \{ x \} \cup (\freenames{P} \setminus \{ y \})
  \and 
  \freenames{x!\langle P \rangle} := \{ x \} \cup \{ P \} 
  \and \\
  \freenames{P|Q} := \freenames{P} \cup \freenames{Q}
  \and \\
  \freenames{@{x}} := \{ x \}
\end{mathpar}

$\pi$
$\quotep{\pi}$

$\freenames{-} : \pi \to \mathcal{P}(\quotep{\pi})$

\begin{eqnarray*}
  \freenames{\pzero} & := & \emptyset \\
  \freenames{x?(y).P} & := & \{ x \} \cup (\freenames{P} \setminus \{ y \}) \\
  \freenames{x!\langle P \rangle} & := & \{ x \} \cup \{ P \} \\
  \freenames{P|Q} & := & \freenames{P} \cup \freenames{Q} \\
  \freenames{\dropn{x}} & := & \{ x \}
\end{eqnarray*}

The bound names of a process, $\boundnames{P}$, are those names occurring in $P$
that are not free. For example, in $x?(y).0$, the name $x$ is free, while $y$ is bound.

\begin{mathpar}
  \inferrule* [lab=monoidal-laws] {} { P|Q \equiv Q|P \and P|0 \equiv P \and P|(Q|R) \equiv (P|Q)|R }
\end{mathpar}

\begin{mathpar}
  \inferrule* [lab=alpha-equivalence] {} { (x)P \equiv (y)P\{y/x\} \and y \not\in \freenames{P} }
\end{mathpar}

\begin{definition}
Then two processes, $P,Q$, are alpha-equivalent if $P = Q\{\vec{y}/\vec{x}\}$ for
some $\vec{x} \in \boundnames{Q},\vec{y} \in \boundnames{P}$, where $Q\{\vec{y}/\vec{x}\}$
denotes the capture-avoiding substitution of $\vec{y}$ for $\vec{x}$ in $Q$.
\end{definition}

\begin{definition}
  The {\em structural congruence} \cite{SangiorgiWalker} , $\equiv$,
  between processes is the least congruence containing
  alpha-equivalence, satisfying the abelian monoid laws
  (associativity, commutativity and $\pzero$ as identity) for parallel
  composition $|$ and for summation $+$.
\end{definition}

\subsection{Name equivalence}

We take name equivalence, written $\nameeq$, to be the smallest
equivalence relation generated by the following rules.

\begin{mathpar}
\inferrule*[lab=Quote-drop]
{ }
{ \quotep{@{x}} \nameeq x }

\inferrule*[lab=Struct-equiv]
{ P \scong Q }
{ \quotep{P} \nameeq \quotep{Q} }
\end{mathpar}

The astute reader will have noticed that the mutual recursion of names
and processes imposes a mutual recursion on alpha-equivalence and
structural equivalence via name-equivalence. Fortunately, all of this
works out pleasantly and we may calculate in the natural way, free of
concern. The reader interested in the details is referred to the
appendix \ref{appendix:rho_details}.

\subsection{Substitution}

We use $\Proc$ for the set of processes, $\QProc$ for the set of
names, and $\id{\{}\vec{y} / \vec{x} \id{\}}$ to denote partial maps,
$s : \QProc \rightarrow \QProc$. A map, $s$ lifts, uniquely, to a map
on process terms, $\widehat{s} : \Proc \rightarrow \Proc$ by the
following equations.

\begin{mathpar}
  (0) \psubstp{Q}{P} := 0 \\
  (R \juxtap S) \psubstp{Q}{P}
  :=    
  (R)\psubstp{Q}{P} \juxtap (S) \psubstp{Q}{P} \\
  (x?(y).R) \psubstp{Q}{P}    
  :=    
  (x)\substp{Q}{P} (z)\concat( (R \psubstn{z}{y}) \psubstp{Q}{P} ) \\
  (\lift{x}{R}) \psubstp{Q}{P}  
  :=
  \lift{(x)\substp{Q}{P}}{ R \psubstp{Q}{P} } \\
%   (\dropn{x})  \psubstp{Q}{P}       
%   := 
%   \left\{ 
%     \begin{array}{ccc} 
%       \dropn{\quotep{Q}} & & x \nameeq \quotep{P} \\
%       \dropn{x} & & otherwise \\
%     \end{array}
%   \right. 
  (\dropn{x})  \psubstp{Q}{P}       
  := 
  \left\{ 
    \begin{array}{ccc} 
      Q & & x \nameeq \quotep{P} \\
      \dropn{x} & & otherwise \\
    \end{array}
  \right.
\end{mathpar}
 

where

\begin{eqnarray}
  (x)\id{\{} \lpquote Q \rpquote / \lpquote P \rpquote \id{\}}            = 
  \left\{ 
    \begin{array}{ccc}
      \lpquote Q \rpquote & & x \nameeq \lpquote P \rpquote \\
      x & & otherwise \\
    \end{array}
  \right. \nonumber
\end{eqnarray}

and $z$ is chosen distinct from $\quotep{P}$, $\quotep{Q}$, the free
names in $Q$, and all the names in $R$. Our $\alpha$-equivalence will
be built in the standard way from this substitution.

\begin{remark}\label{rem:no_self_referential_names}
  One consequence of these definitions is that $\forall P. \quotep{P}
  \not\in \freenames{P}$.
\end{remark}

\subsection{ Dynamic quote: an example }

Anticipating something of what's to come, consider applying the
substitution, $\widehat{\id{\{}u / z \id{\}}}$, to the following pair
of processes, $\lift{w}{y!(z)}$ and $w[ \lpquote y!(z) \rpquote ]$.

\begin{eqnarray}
	\lift{w}{y!(z)}\widehat{\id{\{}u / z \id{\}}}
		& = &
		\lift{w}{y!(u)} \nonumber\\
	w[ \lpquote y!(z) \rpquote ] \widehat{ \id{\{}u / z \id{\}} }
		& = &
		w[ \lpquote y!(z) \rpquote ] \nonumber
\end{eqnarray}

Because the body of the process between quotes is impervious to
substitution, we get radically different answers. In fact, by
examining the first process in an input context,
e.g. $x?(z).\lift{w}{y!(z)}$, we see that the process under the lift
operator may be shaped by prefixed inputs binding a name inside it. In
this sense, the lift operator will be seen as a way to dynamically
construct processes before reifying them as names.

Finally equipped with these standard features we can present the
dynamics of the calculus.

\subsubsection{Operational semantics} 

Finally, we introduce the computational dynamics. What marks these
algebras as distinct from other more traditionally studied algebraic
structures, e.g. vector spaces or polynomial rings, is the manner in
which dynamics is captured. In traditional structures, dynamics is typically
expressed through morphisms between such structures, as in linear maps
between vector spaces or morphisms between rings. In algebras
associated with the semantics of computation, the dynamics is
expressed as part of the algebraic structure itself, through a
reduction reduction relation typically denoted by $\red$. Below, we
give a recursive presentation of this relation for the calculus used
in the encoding.

$\red \subseteq \pi \times \pi$
$\red : \pi \to \mathcal{P}(\pi)$

\begin{mathpar}
  \inferrule* [lab=Comm] { \textsf{match}( x_{src}, x_{trgt} ) } { x_{trgt}?(y)P \; | \; x_{src}!\langle {Q} \rangle \red P\{\quotep{Q}/y}\} }
  \and \\
  \inferrule* [lab=Par] {{P} \red {P}'} {{{P} | {Q}} \red {{P}' | {Q}}}
  \and
  \inferrule* [lab=Equiv]{{{P} \scong {P}'} \andalso {{P}' \red {Q}'} \andalso {{Q}' \scong {Q}}}{{P} \red {Q}}
\end{mathpar}

\begin{eqnarray*}
  match_{\equiv} (\quotep{P},\quotep{Q}) & := & P \equiv Q \\
  match_{\dagger}(\quotep{P},\quotep{Q}) & := & \forall R. P|Q \red^{*} R => R \red^{*} 0 \\
  match_{K}(\quotep{P},\quotep{Q}) & := & K \mbox{ for some context } K
\end{eqnarray*}

$u?(x)P | u!\langle Q \rangle \red P\{\quotep{Q}/x\}$

%We write $\wred$ for $\red^*$, and $P\red$ if $\exists Q $ such that $ P \red Q$.
We write $P\red$ if $\exists Q $ such that $ P \red Q$ and $P\not\red$, otherwise.

\section{Replication}

As mentioned before, it is known that replication (and hence
recursion) can be implemented in a higher-order process algebra
\cite{SangiorgiWalker}. As our first example of calculation with the
machinery thus far presented we give the construction explicitly in
the {\rhoc}.

\begin{eqnarray}
	D_{x} & := & \prefix{x}{y}{(\binpar{\outputp{x}{y}}{@{y}})} \nonumber\\
	\bangp_{x}{P} & := & \binpar{{x}!\langle{\binpar{D_{x}}{P}}\rangle}{D_{x}} \nonumber
\end{eqnarray}

\begin{eqnarray}
	\bangp_{x}{P} & & \nonumber\\
	=
	& {x}!\langle{(\prefix{x}{y}{(\outputp{x}{y} | @{y})) | P}}\rangle 
	      | \prefix{x}{y}{(\outputp{x}{y} | @{y})} & \nonumber\\
	\red
	& (\outputp{x}{y} | @{y})\substn{\quotep{(\prefix{x}{y}{(@{y} | \outputp{x}{y})) | P}}}{y} & \nonumber\\
	=
	& \outputp{x}{\quotep{(\prefix{x}{y}{(\outputp{x}{y} | @{y})) | P}}}
	  | {(\prefix{x}{y}{(\outputp{x}{y} | @{y})) | P}} & \nonumber\\
	\red
	& \ldots & \nonumber\\
	\red^*
	& P | P | \ldots & \nonumber
\end{eqnarray}

Of course, this encoding, as an implementation, runs away, unfolding
$\bangp{P}$ eagerly. A lazier and more implementable replication
operator, restricted to input-guarded processes, may be obtained as follows.

\begin{eqnarray}
\bangp{\prefix{u}{v}{P}} 
	:= 
	\binpar{\lift{x}{\prefix{u}{v}{(\binpar{D(x)}{P})}}}{D(x)} \nonumber
\end{eqnarray}

\begin{remark}
  Note that the lazier definition still does not deal with summation
  or mixed summation (i.e. sums over input and output). The reader is
  invited to construct definitions of replication that deal with these
  features. 

  Further, the definitions are parameterized in a name, $x$. Can you,
  gentle reader, make a definition that eliminates this parameter and
  guarantees no accidental interaction between the replication
  machinery and the process being replicated -- i.e. no accidental
  sharing of names used by the process to get its work done and the
  name(s) used by the replication to effect copying. This latter
  revision of the definition of replication is crucial to obtaining
  the expected identity $!!P \sim !P$.
\end{remark}

\begin{remark}\label{rem:paradoxical_combinator}
  The reader familiar with the lambda calculus will have noticed the
  similarity between $D$ and the paradoxical combinator.

  [Ed. note: the existence of this seems to suggest we have to be more
  restrictive on the set of processes and names we admit if we are to
  support no-cloning.]
\end{remark}

\subsubsection{Bisimulation}

The computational dynamics gives rise to another kind of equivalence,
the equivalence of computational behavior. As previously mentioned
this is typically captured \emph{via} some form of bisimulation.

% The notion we use in this paper is weak barbed bisimulation
% \cite{milner91polyadicpi}.

The notion we use in this paper is derived from weak barbed
bisimulation \cite{milner91polyadicpi}. 

\begin{definition}
An \emph{observation relation}, $\downarrow_{\mathcal N}$, over a set
of names, $\mathcal N$, is the smallest relation satisfying the rules
below.

\infrule[Out-barb]{y \in {\mathcal N}, \; x \nameeq y}
		  {\outputp{x}{v} \downarrow_{\mathcal N} x}
\infrule[Par-barb]{\mbox{$P\downarrow_{\mathcal N} x$ or $Q\downarrow_{\mathcal N} x$}}
		  {\binpar{P}{Q} \downarrow_{\mathcal N} x}

We write $P \Downarrow_{\mathcal N} x$ if there is $Q$ such that 
$P \wred Q$ and $Q \downarrow_{\mathcal N} x$.
\end{definition}

\begin{definition}
%\label{def.bbisim}
An  ${\mathcal N}$-\emph{barbed bisimulation} over a set of names, ${\mathcal N}$, is a symmetric binary relation 
${\mathcal S}_{\mathcal N}$ between agents such that $P\rel{S}_{\mathcal N}Q$ implies:
\begin{enumerate}
\item If $P \red P'$ then $Q \wred Q'$ and $P'\rel{S}_{\mathcal N} Q'$.
\item If $P\downarrow_{\mathcal N} x$, then $Q\Downarrow_{\mathcal N} x$.
\end{enumerate}
$P$ is ${\mathcal N}$-barbed bisimilar to $Q$, written
$P \wbbisim_{\mathcal N} Q$, if $P \rel{S}_{\mathcal N} Q$ for some ${\mathcal N}$-barbed bisimulation ${\mathcal S}_{\mathcal N}$.
\end{definition}

$\mathcal{R} \subseteq \pi \times \pi$

$P \mathcal{R} Q => \forall P'. P \red P' \Rightarrow \exists Q'. Q \red Q', P' \mathcal{R} Q'$

$P \vdash x \Rightarrow Q \vdash x$

\begin{mathpar}
  \inferrule*[lab=Out-barb]{x \nameeq y}{{y}!\langle{Q}\rangle \vdash x}
  \and
  \inferrule*[lab=Par-barb]{\mbox{$P\vdash x$ or $Q\vdash x$}}{\binpar{P}{Q} \vdash x}
\end{mathpar}

\subsubsection{Contexts}

One of the principle advantages of computational calculi like the
$\pi$-calculus is a well-defined notion of context,
contextual-equivalence and a correlation between
contextual-equivalence and notions of bisimulation. The notion of
context allows the decomposition of a process into (sub-)process and
its syntactic environment, its context. Thus, a context may be
thought of as a process with a ``hole'' (written $\Box$) in it. The
application of a context $M$ to a process $P$, written $M[P]$, is
tantamount to filling the hole in $M$ with $P$. In this paper we do
not need the full weight of this theory, but do make use of the notion
of context in the proof the main theorem. 

\begin{mathpar}
  \inferrule* [lab=summation] {} {{M_{M},M_{N}} \bc \Box \;|\; x.M_{A} \;|\; M_{M}+M_{N}}
  \and
  \inferrule* [lab=agent] {} {{M_{A}} \bc (\vec{x})M_{P} \;| \; \clift{P_0,\ldots,M_{P},\ldots,P_N}}
  \and \\
  \inferrule* [lab=process] {} {{M_{P}} \bc M_{N} \;| \;P|M_{P} }
\end{mathpar} 

\begin{mathpar}
  \inferrule* [lab=sychronization] {} {M_{N} \bc \Box \;|\; x?M_{F} \;|\; x!M_{C}}
  \and
  \inferrule* [lab=abstraction] {} {{M_{F}} \bc (x)M_{P} }
  \and
  \inferrule* [lab=concretion] {} {{M_{C}} \bc \langle M_{P} \rangle }
  \and \\
  \inferrule* [lab=process] {} {{M_{P}} \bc M_{N} \;| \;P|M_{P} }
\end{mathpar}

\begin{definition}[contextual application] Given a context $M$, and
  process $P$, we define the \emph{contextual application}, $M[P] :=
  M\{P/\Box\}$. That is, the contextual application of M to P is the
  substitution of $P$ for $\Box$ in $M$.
\end{definition}

$\meaningof{-} : L \to \mathcal{P}(\pi)$

\begin{mathpar}
  \inferrule* [lab=collection] {} {\meaningof{true} = \pi, \and \meaningof{~E} = \pi \setminus \meaningof{E}, \and \meaningof{E_{1} \& E_{2}} = \meaningof{E_{1}} \cap \meaningof{E_{2}}}
\end{mathpar}

\begin{mathpar}
  \inferrule* [lab=structure] {} {\meaningof{0} = \{ P \in \pi | P \equiv 0 \}, \and \\ \meaningof{E_1 | E_2} = \{ P \in \pi | P \equiv P_{1} | P_{2}, P_{1} \in \meaningof{E_{1}}, P_{2} \in \meaningof{E_2}\} }
\end{mathpar}

\begin{mathpar}
 \inferrule* [lab=behavior] {} {\meaningof{\langle a?b \rangle E} = \{ P \in \pi | P \equiv Q | u?(y)P', \\ \and \\\\ \and \\ \;\;\; u \in \meaningof{a}, \forall z.P'\{z/y\} \in \meaningof{E\{z/b\}}\}, \and \\ \meaningof{a!E} = \{ P \in \pi | P \equiv Q | x!\langle P' \rangle, x \in \meaningof{a} P' \in \meaningof{E}\} }
\end{mathpar}

\begin{mathpar}
 \inferrule* [lab=nominal] {} {\meaningof{\quotep{E}} = \{ \quotep{P} \in \quotep{\pi} | P \in \meaningof{E} \}, \and \meaningof{\quotep{P}} = \{ \quotep{Q} \in \quotep{\pi} | P \equiv Q \} \and \\ \meaningof{@\quotep{E}} = \{ P \in \pi | P \equiv @x, x \in \meaningof{E} \}}
\end{mathpar}

\begin{eqnarray*}
  \\
  \meaningof{-} : TS \to ST
\end{eqnarray*}

\begin{eqnarray*}
  \\
  L : TS \to ST
\end{eqnarray*}

\begin{eqnarray*}
  \\
  P \models E \iff P \in \meaningof{E}
\end{eqnarray*}

\begin{eqnarray*}
  P \approx_{L} Q \iff \forall E \in L. P \models E \iff Q \models E
\end{eqnarray*}

\begin{eqnarray*}
  P \approx_{K} Q
\end{eqnarray*}

\begin{eqnarray*}
  P \approx Q
\end{eqnarray*}

$\approx_{K} = \approx = \approx_{L}$

\subsubsection{Contextual duality}

Note that contexts extend the quotation operation to a family of
operations from processes to names. Given a context, $M$, we can
define a \emph{nominal context}, $\quotep{M}$ by $\quotep{M}[P] :=
\quotep{M[P]}$. To foreshadow what is to come we observe that these
operations enjoy a duality with processes very much like the duality
between vectors and maps from vectors to scalars.

Further, because the calculus is essentially higher-order, we have a
correspondence between contexts and processes. More specifically,
given a name $x$ and a context $M$ we can construct $M^{*}_{x}$ such
that 

\begin{mathpar}
  M^{*}_{x} | \lift{x}{P} \red M[P]
\end{mathpar}

namely,

\begin{mathpar}
  M^{*}_{x} := x?(u).M[\dropn{u}]
\end{mathpar}

The dependence of $M^{*}_{x}$ on a name makes it an abstraction, 

\begin{mathpar}
  M^{*} := (x)x?(u).M[\dropn{u}]
\end{mathpar}

\subsection{Additional notation}

It will sometimes be convenient to denote the process a name
quotes. We already have the notation $x = \quotep{P}$, but it will be
convenient to introduce an alternate notation, $\procn{x}$, when we
want to emphasize the connection to the use of the name. Note that, by
virtue of name equivalence, $\quotep{\procn{x}} \nameeq x$; so, the
notation is consistent with previous definitions.

Further, because names have structure it is possible to effect
substitutions on the basis of that structure. This means we need to
upgrade our notation for substitutions, which we accomplish by
adapting comprehension notation. Thus,

\begin{mathpar}
  P\{ y / x : x \in S \}
\end{mathpar}

is interpreted to mean the process derived from P by replacing (in a
capture-avoiding manner) each occurrence of $x$ in $S$ by $y$. For example,

\begin{mathpar}
  P\{ \quotep{\procn{x}|\procn{x}} / x : x \in \freenames{P} \}
\end{mathpar}

will replace each (occurrence) of a free name $x$ in $P$ by
$\quotep{\procn{x}|\procn{x}}$.

Also, we will avail ourselves of the notation $x^{L}$ and $x^{R}$ to
denote injections of a name into disjoint copies of the name
space. There are numerous ways to accomplish this. One example can be
found in \cite{MeredithR05}. This notation overloads to vectors of
names: $\vec{x}^{\pi} := (x_{i}^{\pi} \; : \; 0 \leq i < |\vec{x}| )$ where $\pi \in \{L,R\}$.

We also use $P^{\Box} := P|\Box$.

In \cite{MeredithR05} an interpretation of the new operator is
given. It turns out that there are several possible interpretations
all enjoying the requisite algebraic properties of the operator (see
\cite{milner91polyadicpi}). We will therefore make liberal use of
$(\nu\; \vec{x})P$.

% subsection the_syntax_and_semantics_of_the_notation_system (end)   

\input{qm2pi.qmops} 

\input{qm2pi.sterngerlach} 

\input{qm2pi.metric} 

% section concurrent_process_calculi (end)

%\input{qm2pi.proofsketch}

% section proof sketch (end)

%\input{qm2pi.slviaknots} 

% section spatial logic via knots (end)

\input{qm2pi.conclusion}

% section conclusion (end)

%\input{qm2pi.dtcodes} 

% section wiring algorithm (end)

\input{qm2pi.ack} 

% section acknowledgments (end)

\newpage


\bibliographystyle{plain}   
\bibliography{../../biblios/main.bib}

\input{qm2pi.rhodetails}

\end{document}



% section front matter (end)

\section{Introduction}\label{sec:introduction} % (fold)
In this draft of the material i am going to have to dispense with the
usual writing conventions adopted in papers on these topics. i'm going
to have adopt whatever tone i need at the time i'm writing up the
calculations. Sometimes this may be very conversational; others it may
be the barest mathematical grunts; others still it may be that i have
lifted text from one of my other papers because the exposition of some
point was better said there. i hope that my readers are not unduly put
out by this decision. i'm not doing this to flout convention or be
rebellious. i find these calculations very technically challenging. To
keep everything going technically, something has to give; i have to
let go of some cognitive burden. So, the academic writing style --
with all of its trade-offs in terms of facilitating technical
communication -- is what i'm letting go of. Perhaps subsequent drafts
can be tightened and polished, but for now, i'm going to speak as if
we were sitting together in a coffee shop with a laptop, wifi and a
pad of paper and a pencil.

So, here's what i have to say. We -- you and i, comfortably ensconced
in our coffee shop and well-equipped with our tools -- can realize and
carry out the calculations of quantum mechanics over a very different
formal theory of dynamics, a formal theory of dynamics that
corresponds to a theory of concurrent computation with
\emph{reflection}. It has the advantage that the underlying theory is
already `quantized', but supports analogues all of the continuuous
operations. Strikingly, this underlying theory has recently been
connected with a notion of metric that we can show, by calculating
together, coincides with the metric induced by the inner product.

There are a lot of reasons why you might be interested in seeing
calculations of this form. Here's why i'm interested. For the past
several centuries there has been no competitor to the ``Newtonian''
account of dynamics. As a result the predominant share of accounts of
dynamical systems and situations have had to be formulated in terms of
the Newtonian machinery. i view this as an intellectually dangerous
position to occupy. Everything, despite it's intrinsic shape, turns
into a nail to be hit with this hammer. Recently, however, the theory
of computation has matured to the point where we have candidates for
theories of dynamics that offer very different perspective on
reasoning about dynamical systems and situations. Testing these
candidates against very successful accounts of dynamical situations,
like quantum mechanics, is going to give us some sense of how mature
they are and some measure of the quality of these accounts of
dynamics.

\subsection{Summary of contributions and outline of paper}

So, we're going to develop an interpretation of the operations of
quantum mechanics normally interpreted by Hilbert spaces and
operators. We're going to do this over a theory of computation. Note
that this is very different than the usual quantum computation program
which develops notions of computation over quantum mechanics. Rather,
we are developing a story that aligns with Wheeler's slogan: It from
Bit. To do this we will first provide an account of the theory of
computation at play here. Then we will dive into a calculation-driven
interpretation of the operations of quantum mechanics.

The reason we take this approach is that -- until very recently --
there hasn't been an axiomatic account of quantum mechanics. As a
result there has been no sharp delineation of the mathematical theory
supporting interpretation of the physical theory and the physical
theory, itself. So, ambient features of the maths are free to be
exploited (or supressed) without a real accounting of their physical
relevance. There is no sharp statement ``here's the physical theory''
qua \emph{theory} and ``here's the mathematical interpretation''
enabling a judgment of how faithful the interpretation is -- apart
from experimental observation. When there is an axiomatic account we
can judge how well a given mathematical formalism supports an
interpretation of the axioms, independent of
experimentation. Likewise, we can judge how well we have captured our
physical evidence and experience with our axiomatics, independent of
any specific mathematical implementation, with accidental detail that
may or may not have physical significance. 

In lieu of a fully fleshed out and vetted axiomatic account of quantum
mechanics, interpreting the operational notions in service of modeling
physical systems will have to suffice. In other words, we are not in
the business of providing a model of Hilbert spaces and operators. We
are in the business of providing a model of quantum mechanics because
we are motivated by testing our notions of dynamics against physical
theory; and, the predictive calculations of the physical theory must
serve as the best formulation -- shy of a fully fleshed out axiomatic
account -- of the physical theory itself (as they have for scientific
theories since time immemorial). Put another way, despite a
whole-hearted commitment to an It-from-Bit ontology, we are firmly
aligned with the shut-up-and-calculate camp as the best way to obtain
results either from the physical perspective or as a quality assurance
measure of our fledgling theory of dynamics.

In detail, we present a reflective process calculus. Then we develop
intuitive correspondences between the notions available in this
calculus and the usual physical notions supporting quantum mechanical
calculations. Thus, 

\begin{table}[htp]
  \center{
    \fbox{
      \begin{tabular}{c|c}
        quantum mechanics & process calculus \\
        \hline
        scalar & name \\
        state vector & process \\
        dual & contextual duals \\
        matrix & formal sums of process-context-dual pairs \\
        orthogonality & process annihilation \\
        inner product & execution-formula + quoting
      \end{tabular}
    }
  }
  \caption{QM - process calculi correspondences}
\end{table}

Then we tighten up these intuitions to operational definitions. We
employ the Dirac notation as the best proxy we can find for an
abstract syntax of the quantum mechanical notions. The definitions we
develop put us in contact with equational constraints coming from the
theory that we demonstrate the definitions and calculations satisfy.

This puts us in a position to shut up and calculate for the
Stern-Gerlach experimental set up, showing how these predictive
calculations become calculations on processes in our theory of a
reflective process calculus.

Penultimately, we demonstrate that the notion of metric coming from
the inner product coincides with the notion of metric available from
the theory of bisimulation. This demonstration gives us the right to
think of space as arising from behavior. Finally, we consider where we
might go from the new vantage point we have obtained.

% section introduction (end) 
 
% section introduction (end)

% \documentclass[12pt]{llncs}
%\documentclass{jktr}

\usepackage[pdftex]{hyperref}                   
\usepackage {listings}
\usepackage {mathpartir}
\usepackage{bcprules}
%\usepackage{listings}
                       
\usepackage{graphicx} 
%\usepackage[margins=2.5cm,nohead,nofoot]{geometry}
%\usepackage{geometry}
\usepackage{amsfonts}
\usepackage{amstext}
\usepackage{latexsym}
\usepackage{amssymb}
\usepackage{color}


%\include{myPreamble}
\include{qm2pi.local} 

%\ifpdf
%\usepackage[pdftex]{graphicx}
%\else
%\usepackage{graphicx}
%\fi

 % \ifpdf
%  \usepackage{pdfsync}
%  \if


%\title{Brief Article}
%\author{David F. Snyder}
%\author{L.G. Meredith}

%\address{Dept. of Math., Texas State University--San Marcos, San Marcos, TX 78666}
       
\pagestyle{empty}


\begin{document}

\lstset{language=[Objective]Caml,frame=shadowbox}

\input{qm2pi.front}

% section front matter (end)

\input{qm2pi.intro} 
 
% section introduction (end)

% \input{qm2pi.knotations} 

% section notation (end)

\input{qm2pi.process.calculi} 

% section concurrent_process_calculi_and_spatial_logics_ (end)
    
%\input{qm2pi.knots2pi} 

%\input{qm2pi.trefoil} 

%\input{qm2pi.mainthm} 

% subsection basic_interpretation (end)

%\input{qm2pi.rho.presentation} 
\subsection{The syntax and semantics of the notation system}\label{sub:the_syntax_and_semantics_of_the_notation_system} % (fold)

We now summarize a technical presentation of the calculus that
embodies our theory of dynamics. The typical presentation of such a
calculus follows the style of giving generators and relations on
them. The grammar, below, describing term constructors, freely
generates the set of processes, $\Proc$. This set is then quotiented
by a relation known as structural congruence and it is over this set
that the notion of dynamics is expressed. This presentation is
essentially that of \cite{MeredithR05} with the addition of
polyadicity and summation. For readability we have relegated some of
the technical subtleties to an appendix.

\subsubsection{Process grammar}\label{subsub:process_grammar}

\begin{mathpar}
  \inferrule* [lab=synchronization] {} {{M} \bc \pzero \;|\; x?F \;|\; x!C }
  \and
  \inferrule* [lab=abstraction] {} {{F} \bc (x)P}
  \and
  \inferrule* [lab=concretion] {} {{C} \bc \langle Q \rangle}
  \and
  \inferrule* [lab=process] {} {{P,Q} \bc M \;| \;P|Q \;|\; @{x}}
  \and
  \inferrule* [lab=name] {} {{x} \bc \quotep{P}}
\end{mathpar} 

Note that $\vec{x}$ (resp. $\vec{P}$) denotes a vector of names
(resp. processes) of length $|\vec{x}|$ (resp. $|\vec{P}|$). We adopt
the following useful abbreviations.

\begin{mathpar}
   x?(\vec{y}).P := x.(\vec{y})P \and  x\clift{\vec{P}} := x.\clift{\vec{P}}
   \and x!(y) := \lift{x}{\dropn{y}}
   \and \Pi_{i=0}^{n-1}P_i := P_0 | \ldots | P_{n-1}
\end{mathpar}

\subsubsection{Structural congruence}

\paragraph{Free and bound names and alpha-equivalence.} At the
core of structural equivalence is alpha-equivalence which identifies
process that are the same up to a change of variable. Formally, we
recognize the distinction between free and bound names. The free names
of a process, $\freenames{P}$, may be calculated recursively as
follows:

\begin{mathpar}
\freenames{\pzero} := \emptyset
  \and \\
  \freenames{x?(y).P} := \{ x \} \cup (\freenames{P} \setminus \{ y \})
  \and 
  \freenames{x!\langle P \rangle} := \{ x \} \cup \{ P \} 
  \and \\
  \freenames{P|Q} := \freenames{P} \cup \freenames{Q}
  \and \\
  \freenames{@{x}} := \{ x \}
\end{mathpar}

$\pi$
$\quotep{\pi}$

$\freenames{-} : \pi \to \mathcal{P}(\quotep{\pi})$

\begin{eqnarray*}
  \freenames{\pzero} & := & \emptyset \\
  \freenames{x?(y).P} & := & \{ x \} \cup (\freenames{P} \setminus \{ y \}) \\
  \freenames{x!\langle P \rangle} & := & \{ x \} \cup \{ P \} \\
  \freenames{P|Q} & := & \freenames{P} \cup \freenames{Q} \\
  \freenames{\dropn{x}} & := & \{ x \}
\end{eqnarray*}

The bound names of a process, $\boundnames{P}$, are those names occurring in $P$
that are not free. For example, in $x?(y).0$, the name $x$ is free, while $y$ is bound.

\begin{mathpar}
  \inferrule* [lab=monoidal-laws] {} { P|Q \equiv Q|P \and P|0 \equiv P \and P|(Q|R) \equiv (P|Q)|R }
\end{mathpar}

\begin{mathpar}
  \inferrule* [lab=alpha-equivalence] {} { (x)P \equiv (y)P\{y/x\} \and y \not\in \freenames{P} }
\end{mathpar}

\begin{definition}
Then two processes, $P,Q$, are alpha-equivalent if $P = Q\{\vec{y}/\vec{x}\}$ for
some $\vec{x} \in \boundnames{Q},\vec{y} \in \boundnames{P}$, where $Q\{\vec{y}/\vec{x}\}$
denotes the capture-avoiding substitution of $\vec{y}$ for $\vec{x}$ in $Q$.
\end{definition}

\begin{definition}
  The {\em structural congruence} \cite{SangiorgiWalker} , $\equiv$,
  between processes is the least congruence containing
  alpha-equivalence, satisfying the abelian monoid laws
  (associativity, commutativity and $\pzero$ as identity) for parallel
  composition $|$ and for summation $+$.
\end{definition}

\subsection{Name equivalence}

We take name equivalence, written $\nameeq$, to be the smallest
equivalence relation generated by the following rules.

\begin{mathpar}
\inferrule*[lab=Quote-drop]
{ }
{ \quotep{@{x}} \nameeq x }

\inferrule*[lab=Struct-equiv]
{ P \scong Q }
{ \quotep{P} \nameeq \quotep{Q} }
\end{mathpar}

The astute reader will have noticed that the mutual recursion of names
and processes imposes a mutual recursion on alpha-equivalence and
structural equivalence via name-equivalence. Fortunately, all of this
works out pleasantly and we may calculate in the natural way, free of
concern. The reader interested in the details is referred to the
appendix \ref{appendix:rho_details}.

\subsection{Substitution}

We use $\Proc$ for the set of processes, $\QProc$ for the set of
names, and $\id{\{}\vec{y} / \vec{x} \id{\}}$ to denote partial maps,
$s : \QProc \rightarrow \QProc$. A map, $s$ lifts, uniquely, to a map
on process terms, $\widehat{s} : \Proc \rightarrow \Proc$ by the
following equations.

\begin{mathpar}
  (0) \psubstp{Q}{P} := 0 \\
  (R \juxtap S) \psubstp{Q}{P}
  :=    
  (R)\psubstp{Q}{P} \juxtap (S) \psubstp{Q}{P} \\
  (x?(y).R) \psubstp{Q}{P}    
  :=    
  (x)\substp{Q}{P} (z)\concat( (R \psubstn{z}{y}) \psubstp{Q}{P} ) \\
  (\lift{x}{R}) \psubstp{Q}{P}  
  :=
  \lift{(x)\substp{Q}{P}}{ R \psubstp{Q}{P} } \\
%   (\dropn{x})  \psubstp{Q}{P}       
%   := 
%   \left\{ 
%     \begin{array}{ccc} 
%       \dropn{\quotep{Q}} & & x \nameeq \quotep{P} \\
%       \dropn{x} & & otherwise \\
%     \end{array}
%   \right. 
  (\dropn{x})  \psubstp{Q}{P}       
  := 
  \left\{ 
    \begin{array}{ccc} 
      Q & & x \nameeq \quotep{P} \\
      \dropn{x} & & otherwise \\
    \end{array}
  \right.
\end{mathpar}
 

where

\begin{eqnarray}
  (x)\id{\{} \lpquote Q \rpquote / \lpquote P \rpquote \id{\}}            = 
  \left\{ 
    \begin{array}{ccc}
      \lpquote Q \rpquote & & x \nameeq \lpquote P \rpquote \\
      x & & otherwise \\
    \end{array}
  \right. \nonumber
\end{eqnarray}

and $z$ is chosen distinct from $\quotep{P}$, $\quotep{Q}$, the free
names in $Q$, and all the names in $R$. Our $\alpha$-equivalence will
be built in the standard way from this substitution.

\begin{remark}\label{rem:no_self_referential_names}
  One consequence of these definitions is that $\forall P. \quotep{P}
  \not\in \freenames{P}$.
\end{remark}

\subsection{ Dynamic quote: an example }

Anticipating something of what's to come, consider applying the
substitution, $\widehat{\id{\{}u / z \id{\}}}$, to the following pair
of processes, $\lift{w}{y!(z)}$ and $w[ \lpquote y!(z) \rpquote ]$.

\begin{eqnarray}
	\lift{w}{y!(z)}\widehat{\id{\{}u / z \id{\}}}
		& = &
		\lift{w}{y!(u)} \nonumber\\
	w[ \lpquote y!(z) \rpquote ] \widehat{ \id{\{}u / z \id{\}} }
		& = &
		w[ \lpquote y!(z) \rpquote ] \nonumber
\end{eqnarray}

Because the body of the process between quotes is impervious to
substitution, we get radically different answers. In fact, by
examining the first process in an input context,
e.g. $x?(z).\lift{w}{y!(z)}$, we see that the process under the lift
operator may be shaped by prefixed inputs binding a name inside it. In
this sense, the lift operator will be seen as a way to dynamically
construct processes before reifying them as names.

Finally equipped with these standard features we can present the
dynamics of the calculus.

\subsubsection{Operational semantics} 

Finally, we introduce the computational dynamics. What marks these
algebras as distinct from other more traditionally studied algebraic
structures, e.g. vector spaces or polynomial rings, is the manner in
which dynamics is captured. In traditional structures, dynamics is typically
expressed through morphisms between such structures, as in linear maps
between vector spaces or morphisms between rings. In algebras
associated with the semantics of computation, the dynamics is
expressed as part of the algebraic structure itself, through a
reduction reduction relation typically denoted by $\red$. Below, we
give a recursive presentation of this relation for the calculus used
in the encoding.

$\red \subseteq \pi \times \pi$
$\red : \pi \to \mathcal{P}(\pi)$

\begin{mathpar}
  \inferrule* [lab=Comm] { \textsf{match}( x_{src}, x_{trgt} ) } { x_{trgt}?(y)P \; | \; x_{src}!\langle {Q} \rangle \red P\{\quotep{Q}/y}\} }
  \and \\
  \inferrule* [lab=Par] {{P} \red {P}'} {{{P} | {Q}} \red {{P}' | {Q}}}
  \and
  \inferrule* [lab=Equiv]{{{P} \scong {P}'} \andalso {{P}' \red {Q}'} \andalso {{Q}' \scong {Q}}}{{P} \red {Q}}
\end{mathpar}

\begin{eqnarray*}
  match_{\equiv} (\quotep{P},\quotep{Q}) & := & P \equiv Q \\
  match_{\dagger}(\quotep{P},\quotep{Q}) & := & \forall R. P|Q \red^{*} R => R \red^{*} 0 \\
  match_{K}(\quotep{P},\quotep{Q}) & := & K \mbox{ for some context } K
\end{eqnarray*}

$u?(x)P | u!\langle Q \rangle \red P\{\quotep{Q}/x\}$

%We write $\wred$ for $\red^*$, and $P\red$ if $\exists Q $ such that $ P \red Q$.
We write $P\red$ if $\exists Q $ such that $ P \red Q$ and $P\not\red$, otherwise.

\section{Replication}

As mentioned before, it is known that replication (and hence
recursion) can be implemented in a higher-order process algebra
\cite{SangiorgiWalker}. As our first example of calculation with the
machinery thus far presented we give the construction explicitly in
the {\rhoc}.

\begin{eqnarray}
	D_{x} & := & \prefix{x}{y}{(\binpar{\outputp{x}{y}}{@{y}})} \nonumber\\
	\bangp_{x}{P} & := & \binpar{{x}!\langle{\binpar{D_{x}}{P}}\rangle}{D_{x}} \nonumber
\end{eqnarray}

\begin{eqnarray}
	\bangp_{x}{P} & & \nonumber\\
	=
	& {x}!\langle{(\prefix{x}{y}{(\outputp{x}{y} | @{y})) | P}}\rangle 
	      | \prefix{x}{y}{(\outputp{x}{y} | @{y})} & \nonumber\\
	\red
	& (\outputp{x}{y} | @{y})\substn{\quotep{(\prefix{x}{y}{(@{y} | \outputp{x}{y})) | P}}}{y} & \nonumber\\
	=
	& \outputp{x}{\quotep{(\prefix{x}{y}{(\outputp{x}{y} | @{y})) | P}}}
	  | {(\prefix{x}{y}{(\outputp{x}{y} | @{y})) | P}} & \nonumber\\
	\red
	& \ldots & \nonumber\\
	\red^*
	& P | P | \ldots & \nonumber
\end{eqnarray}

Of course, this encoding, as an implementation, runs away, unfolding
$\bangp{P}$ eagerly. A lazier and more implementable replication
operator, restricted to input-guarded processes, may be obtained as follows.

\begin{eqnarray}
\bangp{\prefix{u}{v}{P}} 
	:= 
	\binpar{\lift{x}{\prefix{u}{v}{(\binpar{D(x)}{P})}}}{D(x)} \nonumber
\end{eqnarray}

\begin{remark}
  Note that the lazier definition still does not deal with summation
  or mixed summation (i.e. sums over input and output). The reader is
  invited to construct definitions of replication that deal with these
  features. 

  Further, the definitions are parameterized in a name, $x$. Can you,
  gentle reader, make a definition that eliminates this parameter and
  guarantees no accidental interaction between the replication
  machinery and the process being replicated -- i.e. no accidental
  sharing of names used by the process to get its work done and the
  name(s) used by the replication to effect copying. This latter
  revision of the definition of replication is crucial to obtaining
  the expected identity $!!P \sim !P$.
\end{remark}

\begin{remark}\label{rem:paradoxical_combinator}
  The reader familiar with the lambda calculus will have noticed the
  similarity between $D$ and the paradoxical combinator.

  [Ed. note: the existence of this seems to suggest we have to be more
  restrictive on the set of processes and names we admit if we are to
  support no-cloning.]
\end{remark}

\subsubsection{Bisimulation}

The computational dynamics gives rise to another kind of equivalence,
the equivalence of computational behavior. As previously mentioned
this is typically captured \emph{via} some form of bisimulation.

% The notion we use in this paper is weak barbed bisimulation
% \cite{milner91polyadicpi}.

The notion we use in this paper is derived from weak barbed
bisimulation \cite{milner91polyadicpi}. 

\begin{definition}
An \emph{observation relation}, $\downarrow_{\mathcal N}$, over a set
of names, $\mathcal N$, is the smallest relation satisfying the rules
below.

\infrule[Out-barb]{y \in {\mathcal N}, \; x \nameeq y}
		  {\outputp{x}{v} \downarrow_{\mathcal N} x}
\infrule[Par-barb]{\mbox{$P\downarrow_{\mathcal N} x$ or $Q\downarrow_{\mathcal N} x$}}
		  {\binpar{P}{Q} \downarrow_{\mathcal N} x}

We write $P \Downarrow_{\mathcal N} x$ if there is $Q$ such that 
$P \wred Q$ and $Q \downarrow_{\mathcal N} x$.
\end{definition}

\begin{definition}
%\label{def.bbisim}
An  ${\mathcal N}$-\emph{barbed bisimulation} over a set of names, ${\mathcal N}$, is a symmetric binary relation 
${\mathcal S}_{\mathcal N}$ between agents such that $P\rel{S}_{\mathcal N}Q$ implies:
\begin{enumerate}
\item If $P \red P'$ then $Q \wred Q'$ and $P'\rel{S}_{\mathcal N} Q'$.
\item If $P\downarrow_{\mathcal N} x$, then $Q\Downarrow_{\mathcal N} x$.
\end{enumerate}
$P$ is ${\mathcal N}$-barbed bisimilar to $Q$, written
$P \wbbisim_{\mathcal N} Q$, if $P \rel{S}_{\mathcal N} Q$ for some ${\mathcal N}$-barbed bisimulation ${\mathcal S}_{\mathcal N}$.
\end{definition}

$\mathcal{R} \subseteq \pi \times \pi$

$P \mathcal{R} Q => \forall P'. P \red P' \Rightarrow \exists Q'. Q \red Q', P' \mathcal{R} Q'$

$P \vdash x \Rightarrow Q \vdash x$

\begin{mathpar}
  \inferrule*[lab=Out-barb]{x \nameeq y}{{y}!\langle{Q}\rangle \vdash x}
  \and
  \inferrule*[lab=Par-barb]{\mbox{$P\vdash x$ or $Q\vdash x$}}{\binpar{P}{Q} \vdash x}
\end{mathpar}

\subsubsection{Contexts}

One of the principle advantages of computational calculi like the
$\pi$-calculus is a well-defined notion of context,
contextual-equivalence and a correlation between
contextual-equivalence and notions of bisimulation. The notion of
context allows the decomposition of a process into (sub-)process and
its syntactic environment, its context. Thus, a context may be
thought of as a process with a ``hole'' (written $\Box$) in it. The
application of a context $M$ to a process $P$, written $M[P]$, is
tantamount to filling the hole in $M$ with $P$. In this paper we do
not need the full weight of this theory, but do make use of the notion
of context in the proof the main theorem. 

\begin{mathpar}
  \inferrule* [lab=summation] {} {{M_{M},M_{N}} \bc \Box \;|\; x.M_{A} \;|\; M_{M}+M_{N}}
  \and
  \inferrule* [lab=agent] {} {{M_{A}} \bc (\vec{x})M_{P} \;| \; \clift{P_0,\ldots,M_{P},\ldots,P_N}}
  \and \\
  \inferrule* [lab=process] {} {{M_{P}} \bc M_{N} \;| \;P|M_{P} }
\end{mathpar} 

\begin{mathpar}
  \inferrule* [lab=sychronization] {} {M_{N} \bc \Box \;|\; x?M_{F} \;|\; x!M_{C}}
  \and
  \inferrule* [lab=abstraction] {} {{M_{F}} \bc (x)M_{P} }
  \and
  \inferrule* [lab=concretion] {} {{M_{C}} \bc \langle M_{P} \rangle }
  \and \\
  \inferrule* [lab=process] {} {{M_{P}} \bc M_{N} \;| \;P|M_{P} }
\end{mathpar}

\begin{definition}[contextual application] Given a context $M$, and
  process $P$, we define the \emph{contextual application}, $M[P] :=
  M\{P/\Box\}$. That is, the contextual application of M to P is the
  substitution of $P$ for $\Box$ in $M$.
\end{definition}

$\meaningof{-} : L \to \mathcal{P}(\pi)$

\begin{mathpar}
  \inferrule* [lab=collection] {} {\meaningof{true} = \pi, \and \meaningof{~E} = \pi \setminus \meaningof{E}, \and \meaningof{E_{1} \& E_{2}} = \meaningof{E_{1}} \cap \meaningof{E_{2}}}
\end{mathpar}

\begin{mathpar}
  \inferrule* [lab=structure] {} {\meaningof{0} = \{ P \in \pi | P \equiv 0 \}, \and \\ \meaningof{E_1 | E_2} = \{ P \in \pi | P \equiv P_{1} | P_{2}, P_{1} \in \meaningof{E_{1}}, P_{2} \in \meaningof{E_2}\} }
\end{mathpar}

\begin{mathpar}
 \inferrule* [lab=behavior] {} {\meaningof{\langle a?b \rangle E} = \{ P \in \pi | P \equiv Q | u?(y)P', \\ \and \\\\ \and \\ \;\;\; u \in \meaningof{a}, \forall z.P'\{z/y\} \in \meaningof{E\{z/b\}}\}, \and \\ \meaningof{a!E} = \{ P \in \pi | P \equiv Q | x!\langle P' \rangle, x \in \meaningof{a} P' \in \meaningof{E}\} }
\end{mathpar}

\begin{mathpar}
 \inferrule* [lab=nominal] {} {\meaningof{\quotep{E}} = \{ \quotep{P} \in \quotep{\pi} | P \in \meaningof{E} \}, \and \meaningof{\quotep{P}} = \{ \quotep{Q} \in \quotep{\pi} | P \equiv Q \} \and \\ \meaningof{@\quotep{E}} = \{ P \in \pi | P \equiv @x, x \in \meaningof{E} \}}
\end{mathpar}

\begin{eqnarray*}
  \\
  \meaningof{-} : TS \to ST
\end{eqnarray*}

\begin{eqnarray*}
  \\
  L : TS \to ST
\end{eqnarray*}

\begin{eqnarray*}
  \\
  P \models E \iff P \in \meaningof{E}
\end{eqnarray*}

\begin{eqnarray*}
  P \approx_{L} Q \iff \forall E \in L. P \models E \iff Q \models E
\end{eqnarray*}

\begin{eqnarray*}
  P \approx_{K} Q
\end{eqnarray*}

\begin{eqnarray*}
  P \approx Q
\end{eqnarray*}

$\approx_{K} = \approx = \approx_{L}$

\subsubsection{Contextual duality}

Note that contexts extend the quotation operation to a family of
operations from processes to names. Given a context, $M$, we can
define a \emph{nominal context}, $\quotep{M}$ by $\quotep{M}[P] :=
\quotep{M[P]}$. To foreshadow what is to come we observe that these
operations enjoy a duality with processes very much like the duality
between vectors and maps from vectors to scalars.

Further, because the calculus is essentially higher-order, we have a
correspondence between contexts and processes. More specifically,
given a name $x$ and a context $M$ we can construct $M^{*}_{x}$ such
that 

\begin{mathpar}
  M^{*}_{x} | \lift{x}{P} \red M[P]
\end{mathpar}

namely,

\begin{mathpar}
  M^{*}_{x} := x?(u).M[\dropn{u}]
\end{mathpar}

The dependence of $M^{*}_{x}$ on a name makes it an abstraction, 

\begin{mathpar}
  M^{*} := (x)x?(u).M[\dropn{u}]
\end{mathpar}

\subsection{Additional notation}

It will sometimes be convenient to denote the process a name
quotes. We already have the notation $x = \quotep{P}$, but it will be
convenient to introduce an alternate notation, $\procn{x}$, when we
want to emphasize the connection to the use of the name. Note that, by
virtue of name equivalence, $\quotep{\procn{x}} \nameeq x$; so, the
notation is consistent with previous definitions.

Further, because names have structure it is possible to effect
substitutions on the basis of that structure. This means we need to
upgrade our notation for substitutions, which we accomplish by
adapting comprehension notation. Thus,

\begin{mathpar}
  P\{ y / x : x \in S \}
\end{mathpar}

is interpreted to mean the process derived from P by replacing (in a
capture-avoiding manner) each occurrence of $x$ in $S$ by $y$. For example,

\begin{mathpar}
  P\{ \quotep{\procn{x}|\procn{x}} / x : x \in \freenames{P} \}
\end{mathpar}

will replace each (occurrence) of a free name $x$ in $P$ by
$\quotep{\procn{x}|\procn{x}}$.

Also, we will avail ourselves of the notation $x^{L}$ and $x^{R}$ to
denote injections of a name into disjoint copies of the name
space. There are numerous ways to accomplish this. One example can be
found in \cite{MeredithR05}. This notation overloads to vectors of
names: $\vec{x}^{\pi} := (x_{i}^{\pi} \; : \; 0 \leq i < |\vec{x}| )$ where $\pi \in \{L,R\}$.

We also use $P^{\Box} := P|\Box$.

In \cite{MeredithR05} an interpretation of the new operator is
given. It turns out that there are several possible interpretations
all enjoying the requisite algebraic properties of the operator (see
\cite{milner91polyadicpi}). We will therefore make liberal use of
$(\nu\; \vec{x})P$.

% subsection the_syntax_and_semantics_of_the_notation_system (end)   

\input{qm2pi.qmops} 

\input{qm2pi.sterngerlach} 

\input{qm2pi.metric} 

% section concurrent_process_calculi (end)

%\input{qm2pi.proofsketch}

% section proof sketch (end)

%\input{qm2pi.slviaknots} 

% section spatial logic via knots (end)

\input{qm2pi.conclusion}

% section conclusion (end)

%\input{qm2pi.dtcodes} 

% section wiring algorithm (end)

\input{qm2pi.ack} 

% section acknowledgments (end)

\newpage


\bibliographystyle{plain}   
\bibliography{../../biblios/main.bib}

\input{qm2pi.rhodetails}

\end{document}

 

% section notation (end)

\input{qm2pi.process.calculi} 

% section concurrent_process_calculi_and_spatial_logics_ (end)
    
%\documentclass[12pt]{llncs}
%\documentclass{jktr}

\usepackage[pdftex]{hyperref}                   
\usepackage {listings}
\usepackage {mathpartir}
\usepackage{bcprules}
%\usepackage{listings}
                       
\usepackage{graphicx} 
%\usepackage[margins=2.5cm,nohead,nofoot]{geometry}
%\usepackage{geometry}
\usepackage{amsfonts}
\usepackage{amstext}
\usepackage{latexsym}
\usepackage{amssymb}
\usepackage{color}


%\include{myPreamble}
\include{qm2pi.local} 

%\ifpdf
%\usepackage[pdftex]{graphicx}
%\else
%\usepackage{graphicx}
%\fi

 % \ifpdf
%  \usepackage{pdfsync}
%  \if


%\title{Brief Article}
%\author{David F. Snyder}
%\author{L.G. Meredith}

%\address{Dept. of Math., Texas State University--San Marcos, San Marcos, TX 78666}
       
\pagestyle{empty}


\begin{document}

\lstset{language=[Objective]Caml,frame=shadowbox}

\input{qm2pi.front}

% section front matter (end)

\input{qm2pi.intro} 
 
% section introduction (end)

% \input{qm2pi.knotations} 

% section notation (end)

\input{qm2pi.process.calculi} 

% section concurrent_process_calculi_and_spatial_logics_ (end)
    
%\input{qm2pi.knots2pi} 

%\input{qm2pi.trefoil} 

%\input{qm2pi.mainthm} 

% subsection basic_interpretation (end)

%\input{qm2pi.rho.presentation} 
\subsection{The syntax and semantics of the notation system}\label{sub:the_syntax_and_semantics_of_the_notation_system} % (fold)

We now summarize a technical presentation of the calculus that
embodies our theory of dynamics. The typical presentation of such a
calculus follows the style of giving generators and relations on
them. The grammar, below, describing term constructors, freely
generates the set of processes, $\Proc$. This set is then quotiented
by a relation known as structural congruence and it is over this set
that the notion of dynamics is expressed. This presentation is
essentially that of \cite{MeredithR05} with the addition of
polyadicity and summation. For readability we have relegated some of
the technical subtleties to an appendix.

\subsubsection{Process grammar}\label{subsub:process_grammar}

\begin{mathpar}
  \inferrule* [lab=synchronization] {} {{M} \bc \pzero \;|\; x?F \;|\; x!C }
  \and
  \inferrule* [lab=abstraction] {} {{F} \bc (x)P}
  \and
  \inferrule* [lab=concretion] {} {{C} \bc \langle Q \rangle}
  \and
  \inferrule* [lab=process] {} {{P,Q} \bc M \;| \;P|Q \;|\; @{x}}
  \and
  \inferrule* [lab=name] {} {{x} \bc \quotep{P}}
\end{mathpar} 

Note that $\vec{x}$ (resp. $\vec{P}$) denotes a vector of names
(resp. processes) of length $|\vec{x}|$ (resp. $|\vec{P}|$). We adopt
the following useful abbreviations.

\begin{mathpar}
   x?(\vec{y}).P := x.(\vec{y})P \and  x\clift{\vec{P}} := x.\clift{\vec{P}}
   \and x!(y) := \lift{x}{\dropn{y}}
   \and \Pi_{i=0}^{n-1}P_i := P_0 | \ldots | P_{n-1}
\end{mathpar}

\subsubsection{Structural congruence}

\paragraph{Free and bound names and alpha-equivalence.} At the
core of structural equivalence is alpha-equivalence which identifies
process that are the same up to a change of variable. Formally, we
recognize the distinction between free and bound names. The free names
of a process, $\freenames{P}$, may be calculated recursively as
follows:

\begin{mathpar}
\freenames{\pzero} := \emptyset
  \and \\
  \freenames{x?(y).P} := \{ x \} \cup (\freenames{P} \setminus \{ y \})
  \and 
  \freenames{x!\langle P \rangle} := \{ x \} \cup \{ P \} 
  \and \\
  \freenames{P|Q} := \freenames{P} \cup \freenames{Q}
  \and \\
  \freenames{@{x}} := \{ x \}
\end{mathpar}

$\pi$
$\quotep{\pi}$

$\freenames{-} : \pi \to \mathcal{P}(\quotep{\pi})$

\begin{eqnarray*}
  \freenames{\pzero} & := & \emptyset \\
  \freenames{x?(y).P} & := & \{ x \} \cup (\freenames{P} \setminus \{ y \}) \\
  \freenames{x!\langle P \rangle} & := & \{ x \} \cup \{ P \} \\
  \freenames{P|Q} & := & \freenames{P} \cup \freenames{Q} \\
  \freenames{\dropn{x}} & := & \{ x \}
\end{eqnarray*}

The bound names of a process, $\boundnames{P}$, are those names occurring in $P$
that are not free. For example, in $x?(y).0$, the name $x$ is free, while $y$ is bound.

\begin{mathpar}
  \inferrule* [lab=monoidal-laws] {} { P|Q \equiv Q|P \and P|0 \equiv P \and P|(Q|R) \equiv (P|Q)|R }
\end{mathpar}

\begin{mathpar}
  \inferrule* [lab=alpha-equivalence] {} { (x)P \equiv (y)P\{y/x\} \and y \not\in \freenames{P} }
\end{mathpar}

\begin{definition}
Then two processes, $P,Q$, are alpha-equivalent if $P = Q\{\vec{y}/\vec{x}\}$ for
some $\vec{x} \in \boundnames{Q},\vec{y} \in \boundnames{P}$, where $Q\{\vec{y}/\vec{x}\}$
denotes the capture-avoiding substitution of $\vec{y}$ for $\vec{x}$ in $Q$.
\end{definition}

\begin{definition}
  The {\em structural congruence} \cite{SangiorgiWalker} , $\equiv$,
  between processes is the least congruence containing
  alpha-equivalence, satisfying the abelian monoid laws
  (associativity, commutativity and $\pzero$ as identity) for parallel
  composition $|$ and for summation $+$.
\end{definition}

\subsection{Name equivalence}

We take name equivalence, written $\nameeq$, to be the smallest
equivalence relation generated by the following rules.

\begin{mathpar}
\inferrule*[lab=Quote-drop]
{ }
{ \quotep{@{x}} \nameeq x }

\inferrule*[lab=Struct-equiv]
{ P \scong Q }
{ \quotep{P} \nameeq \quotep{Q} }
\end{mathpar}

The astute reader will have noticed that the mutual recursion of names
and processes imposes a mutual recursion on alpha-equivalence and
structural equivalence via name-equivalence. Fortunately, all of this
works out pleasantly and we may calculate in the natural way, free of
concern. The reader interested in the details is referred to the
appendix \ref{appendix:rho_details}.

\subsection{Substitution}

We use $\Proc$ for the set of processes, $\QProc$ for the set of
names, and $\id{\{}\vec{y} / \vec{x} \id{\}}$ to denote partial maps,
$s : \QProc \rightarrow \QProc$. A map, $s$ lifts, uniquely, to a map
on process terms, $\widehat{s} : \Proc \rightarrow \Proc$ by the
following equations.

\begin{mathpar}
  (0) \psubstp{Q}{P} := 0 \\
  (R \juxtap S) \psubstp{Q}{P}
  :=    
  (R)\psubstp{Q}{P} \juxtap (S) \psubstp{Q}{P} \\
  (x?(y).R) \psubstp{Q}{P}    
  :=    
  (x)\substp{Q}{P} (z)\concat( (R \psubstn{z}{y}) \psubstp{Q}{P} ) \\
  (\lift{x}{R}) \psubstp{Q}{P}  
  :=
  \lift{(x)\substp{Q}{P}}{ R \psubstp{Q}{P} } \\
%   (\dropn{x})  \psubstp{Q}{P}       
%   := 
%   \left\{ 
%     \begin{array}{ccc} 
%       \dropn{\quotep{Q}} & & x \nameeq \quotep{P} \\
%       \dropn{x} & & otherwise \\
%     \end{array}
%   \right. 
  (\dropn{x})  \psubstp{Q}{P}       
  := 
  \left\{ 
    \begin{array}{ccc} 
      Q & & x \nameeq \quotep{P} \\
      \dropn{x} & & otherwise \\
    \end{array}
  \right.
\end{mathpar}
 

where

\begin{eqnarray}
  (x)\id{\{} \lpquote Q \rpquote / \lpquote P \rpquote \id{\}}            = 
  \left\{ 
    \begin{array}{ccc}
      \lpquote Q \rpquote & & x \nameeq \lpquote P \rpquote \\
      x & & otherwise \\
    \end{array}
  \right. \nonumber
\end{eqnarray}

and $z$ is chosen distinct from $\quotep{P}$, $\quotep{Q}$, the free
names in $Q$, and all the names in $R$. Our $\alpha$-equivalence will
be built in the standard way from this substitution.

\begin{remark}\label{rem:no_self_referential_names}
  One consequence of these definitions is that $\forall P. \quotep{P}
  \not\in \freenames{P}$.
\end{remark}

\subsection{ Dynamic quote: an example }

Anticipating something of what's to come, consider applying the
substitution, $\widehat{\id{\{}u / z \id{\}}}$, to the following pair
of processes, $\lift{w}{y!(z)}$ and $w[ \lpquote y!(z) \rpquote ]$.

\begin{eqnarray}
	\lift{w}{y!(z)}\widehat{\id{\{}u / z \id{\}}}
		& = &
		\lift{w}{y!(u)} \nonumber\\
	w[ \lpquote y!(z) \rpquote ] \widehat{ \id{\{}u / z \id{\}} }
		& = &
		w[ \lpquote y!(z) \rpquote ] \nonumber
\end{eqnarray}

Because the body of the process between quotes is impervious to
substitution, we get radically different answers. In fact, by
examining the first process in an input context,
e.g. $x?(z).\lift{w}{y!(z)}$, we see that the process under the lift
operator may be shaped by prefixed inputs binding a name inside it. In
this sense, the lift operator will be seen as a way to dynamically
construct processes before reifying them as names.

Finally equipped with these standard features we can present the
dynamics of the calculus.

\subsubsection{Operational semantics} 

Finally, we introduce the computational dynamics. What marks these
algebras as distinct from other more traditionally studied algebraic
structures, e.g. vector spaces or polynomial rings, is the manner in
which dynamics is captured. In traditional structures, dynamics is typically
expressed through morphisms between such structures, as in linear maps
between vector spaces or morphisms between rings. In algebras
associated with the semantics of computation, the dynamics is
expressed as part of the algebraic structure itself, through a
reduction reduction relation typically denoted by $\red$. Below, we
give a recursive presentation of this relation for the calculus used
in the encoding.

$\red \subseteq \pi \times \pi$
$\red : \pi \to \mathcal{P}(\pi)$

\begin{mathpar}
  \inferrule* [lab=Comm] { \textsf{match}( x_{src}, x_{trgt} ) } { x_{trgt}?(y)P \; | \; x_{src}!\langle {Q} \rangle \red P\{\quotep{Q}/y}\} }
  \and \\
  \inferrule* [lab=Par] {{P} \red {P}'} {{{P} | {Q}} \red {{P}' | {Q}}}
  \and
  \inferrule* [lab=Equiv]{{{P} \scong {P}'} \andalso {{P}' \red {Q}'} \andalso {{Q}' \scong {Q}}}{{P} \red {Q}}
\end{mathpar}

\begin{eqnarray*}
  match_{\equiv} (\quotep{P},\quotep{Q}) & := & P \equiv Q \\
  match_{\dagger}(\quotep{P},\quotep{Q}) & := & \forall R. P|Q \red^{*} R => R \red^{*} 0 \\
  match_{K}(\quotep{P},\quotep{Q}) & := & K \mbox{ for some context } K
\end{eqnarray*}

$u?(x)P | u!\langle Q \rangle \red P\{\quotep{Q}/x\}$

%We write $\wred$ for $\red^*$, and $P\red$ if $\exists Q $ such that $ P \red Q$.
We write $P\red$ if $\exists Q $ such that $ P \red Q$ and $P\not\red$, otherwise.

\section{Replication}

As mentioned before, it is known that replication (and hence
recursion) can be implemented in a higher-order process algebra
\cite{SangiorgiWalker}. As our first example of calculation with the
machinery thus far presented we give the construction explicitly in
the {\rhoc}.

\begin{eqnarray}
	D_{x} & := & \prefix{x}{y}{(\binpar{\outputp{x}{y}}{@{y}})} \nonumber\\
	\bangp_{x}{P} & := & \binpar{{x}!\langle{\binpar{D_{x}}{P}}\rangle}{D_{x}} \nonumber
\end{eqnarray}

\begin{eqnarray}
	\bangp_{x}{P} & & \nonumber\\
	=
	& {x}!\langle{(\prefix{x}{y}{(\outputp{x}{y} | @{y})) | P}}\rangle 
	      | \prefix{x}{y}{(\outputp{x}{y} | @{y})} & \nonumber\\
	\red
	& (\outputp{x}{y} | @{y})\substn{\quotep{(\prefix{x}{y}{(@{y} | \outputp{x}{y})) | P}}}{y} & \nonumber\\
	=
	& \outputp{x}{\quotep{(\prefix{x}{y}{(\outputp{x}{y} | @{y})) | P}}}
	  | {(\prefix{x}{y}{(\outputp{x}{y} | @{y})) | P}} & \nonumber\\
	\red
	& \ldots & \nonumber\\
	\red^*
	& P | P | \ldots & \nonumber
\end{eqnarray}

Of course, this encoding, as an implementation, runs away, unfolding
$\bangp{P}$ eagerly. A lazier and more implementable replication
operator, restricted to input-guarded processes, may be obtained as follows.

\begin{eqnarray}
\bangp{\prefix{u}{v}{P}} 
	:= 
	\binpar{\lift{x}{\prefix{u}{v}{(\binpar{D(x)}{P})}}}{D(x)} \nonumber
\end{eqnarray}

\begin{remark}
  Note that the lazier definition still does not deal with summation
  or mixed summation (i.e. sums over input and output). The reader is
  invited to construct definitions of replication that deal with these
  features. 

  Further, the definitions are parameterized in a name, $x$. Can you,
  gentle reader, make a definition that eliminates this parameter and
  guarantees no accidental interaction between the replication
  machinery and the process being replicated -- i.e. no accidental
  sharing of names used by the process to get its work done and the
  name(s) used by the replication to effect copying. This latter
  revision of the definition of replication is crucial to obtaining
  the expected identity $!!P \sim !P$.
\end{remark}

\begin{remark}\label{rem:paradoxical_combinator}
  The reader familiar with the lambda calculus will have noticed the
  similarity between $D$ and the paradoxical combinator.

  [Ed. note: the existence of this seems to suggest we have to be more
  restrictive on the set of processes and names we admit if we are to
  support no-cloning.]
\end{remark}

\subsubsection{Bisimulation}

The computational dynamics gives rise to another kind of equivalence,
the equivalence of computational behavior. As previously mentioned
this is typically captured \emph{via} some form of bisimulation.

% The notion we use in this paper is weak barbed bisimulation
% \cite{milner91polyadicpi}.

The notion we use in this paper is derived from weak barbed
bisimulation \cite{milner91polyadicpi}. 

\begin{definition}
An \emph{observation relation}, $\downarrow_{\mathcal N}$, over a set
of names, $\mathcal N$, is the smallest relation satisfying the rules
below.

\infrule[Out-barb]{y \in {\mathcal N}, \; x \nameeq y}
		  {\outputp{x}{v} \downarrow_{\mathcal N} x}
\infrule[Par-barb]{\mbox{$P\downarrow_{\mathcal N} x$ or $Q\downarrow_{\mathcal N} x$}}
		  {\binpar{P}{Q} \downarrow_{\mathcal N} x}

We write $P \Downarrow_{\mathcal N} x$ if there is $Q$ such that 
$P \wred Q$ and $Q \downarrow_{\mathcal N} x$.
\end{definition}

\begin{definition}
%\label{def.bbisim}
An  ${\mathcal N}$-\emph{barbed bisimulation} over a set of names, ${\mathcal N}$, is a symmetric binary relation 
${\mathcal S}_{\mathcal N}$ between agents such that $P\rel{S}_{\mathcal N}Q$ implies:
\begin{enumerate}
\item If $P \red P'$ then $Q \wred Q'$ and $P'\rel{S}_{\mathcal N} Q'$.
\item If $P\downarrow_{\mathcal N} x$, then $Q\Downarrow_{\mathcal N} x$.
\end{enumerate}
$P$ is ${\mathcal N}$-barbed bisimilar to $Q$, written
$P \wbbisim_{\mathcal N} Q$, if $P \rel{S}_{\mathcal N} Q$ for some ${\mathcal N}$-barbed bisimulation ${\mathcal S}_{\mathcal N}$.
\end{definition}

$\mathcal{R} \subseteq \pi \times \pi$

$P \mathcal{R} Q => \forall P'. P \red P' \Rightarrow \exists Q'. Q \red Q', P' \mathcal{R} Q'$

$P \vdash x \Rightarrow Q \vdash x$

\begin{mathpar}
  \inferrule*[lab=Out-barb]{x \nameeq y}{{y}!\langle{Q}\rangle \vdash x}
  \and
  \inferrule*[lab=Par-barb]{\mbox{$P\vdash x$ or $Q\vdash x$}}{\binpar{P}{Q} \vdash x}
\end{mathpar}

\subsubsection{Contexts}

One of the principle advantages of computational calculi like the
$\pi$-calculus is a well-defined notion of context,
contextual-equivalence and a correlation between
contextual-equivalence and notions of bisimulation. The notion of
context allows the decomposition of a process into (sub-)process and
its syntactic environment, its context. Thus, a context may be
thought of as a process with a ``hole'' (written $\Box$) in it. The
application of a context $M$ to a process $P$, written $M[P]$, is
tantamount to filling the hole in $M$ with $P$. In this paper we do
not need the full weight of this theory, but do make use of the notion
of context in the proof the main theorem. 

\begin{mathpar}
  \inferrule* [lab=summation] {} {{M_{M},M_{N}} \bc \Box \;|\; x.M_{A} \;|\; M_{M}+M_{N}}
  \and
  \inferrule* [lab=agent] {} {{M_{A}} \bc (\vec{x})M_{P} \;| \; \clift{P_0,\ldots,M_{P},\ldots,P_N}}
  \and \\
  \inferrule* [lab=process] {} {{M_{P}} \bc M_{N} \;| \;P|M_{P} }
\end{mathpar} 

\begin{mathpar}
  \inferrule* [lab=sychronization] {} {M_{N} \bc \Box \;|\; x?M_{F} \;|\; x!M_{C}}
  \and
  \inferrule* [lab=abstraction] {} {{M_{F}} \bc (x)M_{P} }
  \and
  \inferrule* [lab=concretion] {} {{M_{C}} \bc \langle M_{P} \rangle }
  \and \\
  \inferrule* [lab=process] {} {{M_{P}} \bc M_{N} \;| \;P|M_{P} }
\end{mathpar}

\begin{definition}[contextual application] Given a context $M$, and
  process $P$, we define the \emph{contextual application}, $M[P] :=
  M\{P/\Box\}$. That is, the contextual application of M to P is the
  substitution of $P$ for $\Box$ in $M$.
\end{definition}

$\meaningof{-} : L \to \mathcal{P}(\pi)$

\begin{mathpar}
  \inferrule* [lab=collection] {} {\meaningof{true} = \pi, \and \meaningof{~E} = \pi \setminus \meaningof{E}, \and \meaningof{E_{1} \& E_{2}} = \meaningof{E_{1}} \cap \meaningof{E_{2}}}
\end{mathpar}

\begin{mathpar}
  \inferrule* [lab=structure] {} {\meaningof{0} = \{ P \in \pi | P \equiv 0 \}, \and \\ \meaningof{E_1 | E_2} = \{ P \in \pi | P \equiv P_{1} | P_{2}, P_{1} \in \meaningof{E_{1}}, P_{2} \in \meaningof{E_2}\} }
\end{mathpar}

\begin{mathpar}
 \inferrule* [lab=behavior] {} {\meaningof{\langle a?b \rangle E} = \{ P \in \pi | P \equiv Q | u?(y)P', \\ \and \\\\ \and \\ \;\;\; u \in \meaningof{a}, \forall z.P'\{z/y\} \in \meaningof{E\{z/b\}}\}, \and \\ \meaningof{a!E} = \{ P \in \pi | P \equiv Q | x!\langle P' \rangle, x \in \meaningof{a} P' \in \meaningof{E}\} }
\end{mathpar}

\begin{mathpar}
 \inferrule* [lab=nominal] {} {\meaningof{\quotep{E}} = \{ \quotep{P} \in \quotep{\pi} | P \in \meaningof{E} \}, \and \meaningof{\quotep{P}} = \{ \quotep{Q} \in \quotep{\pi} | P \equiv Q \} \and \\ \meaningof{@\quotep{E}} = \{ P \in \pi | P \equiv @x, x \in \meaningof{E} \}}
\end{mathpar}

\begin{eqnarray*}
  \\
  \meaningof{-} : TS \to ST
\end{eqnarray*}

\begin{eqnarray*}
  \\
  L : TS \to ST
\end{eqnarray*}

\begin{eqnarray*}
  \\
  P \models E \iff P \in \meaningof{E}
\end{eqnarray*}

\begin{eqnarray*}
  P \approx_{L} Q \iff \forall E \in L. P \models E \iff Q \models E
\end{eqnarray*}

\begin{eqnarray*}
  P \approx_{K} Q
\end{eqnarray*}

\begin{eqnarray*}
  P \approx Q
\end{eqnarray*}

$\approx_{K} = \approx = \approx_{L}$

\subsubsection{Contextual duality}

Note that contexts extend the quotation operation to a family of
operations from processes to names. Given a context, $M$, we can
define a \emph{nominal context}, $\quotep{M}$ by $\quotep{M}[P] :=
\quotep{M[P]}$. To foreshadow what is to come we observe that these
operations enjoy a duality with processes very much like the duality
between vectors and maps from vectors to scalars.

Further, because the calculus is essentially higher-order, we have a
correspondence between contexts and processes. More specifically,
given a name $x$ and a context $M$ we can construct $M^{*}_{x}$ such
that 

\begin{mathpar}
  M^{*}_{x} | \lift{x}{P} \red M[P]
\end{mathpar}

namely,

\begin{mathpar}
  M^{*}_{x} := x?(u).M[\dropn{u}]
\end{mathpar}

The dependence of $M^{*}_{x}$ on a name makes it an abstraction, 

\begin{mathpar}
  M^{*} := (x)x?(u).M[\dropn{u}]
\end{mathpar}

\subsection{Additional notation}

It will sometimes be convenient to denote the process a name
quotes. We already have the notation $x = \quotep{P}$, but it will be
convenient to introduce an alternate notation, $\procn{x}$, when we
want to emphasize the connection to the use of the name. Note that, by
virtue of name equivalence, $\quotep{\procn{x}} \nameeq x$; so, the
notation is consistent with previous definitions.

Further, because names have structure it is possible to effect
substitutions on the basis of that structure. This means we need to
upgrade our notation for substitutions, which we accomplish by
adapting comprehension notation. Thus,

\begin{mathpar}
  P\{ y / x : x \in S \}
\end{mathpar}

is interpreted to mean the process derived from P by replacing (in a
capture-avoiding manner) each occurrence of $x$ in $S$ by $y$. For example,

\begin{mathpar}
  P\{ \quotep{\procn{x}|\procn{x}} / x : x \in \freenames{P} \}
\end{mathpar}

will replace each (occurrence) of a free name $x$ in $P$ by
$\quotep{\procn{x}|\procn{x}}$.

Also, we will avail ourselves of the notation $x^{L}$ and $x^{R}$ to
denote injections of a name into disjoint copies of the name
space. There are numerous ways to accomplish this. One example can be
found in \cite{MeredithR05}. This notation overloads to vectors of
names: $\vec{x}^{\pi} := (x_{i}^{\pi} \; : \; 0 \leq i < |\vec{x}| )$ where $\pi \in \{L,R\}$.

We also use $P^{\Box} := P|\Box$.

In \cite{MeredithR05} an interpretation of the new operator is
given. It turns out that there are several possible interpretations
all enjoying the requisite algebraic properties of the operator (see
\cite{milner91polyadicpi}). We will therefore make liberal use of
$(\nu\; \vec{x})P$.

% subsection the_syntax_and_semantics_of_the_notation_system (end)   

\input{qm2pi.qmops} 

\input{qm2pi.sterngerlach} 

\input{qm2pi.metric} 

% section concurrent_process_calculi (end)

%\input{qm2pi.proofsketch}

% section proof sketch (end)

%\input{qm2pi.slviaknots} 

% section spatial logic via knots (end)

\input{qm2pi.conclusion}

% section conclusion (end)

%\input{qm2pi.dtcodes} 

% section wiring algorithm (end)

\input{qm2pi.ack} 

% section acknowledgments (end)

\newpage


\bibliographystyle{plain}   
\bibliography{../../biblios/main.bib}

\input{qm2pi.rhodetails}

\end{document}

 

%\documentclass[12pt]{llncs}
%\documentclass{jktr}

\usepackage[pdftex]{hyperref}                   
\usepackage {listings}
\usepackage {mathpartir}
\usepackage{bcprules}
%\usepackage{listings}
                       
\usepackage{graphicx} 
%\usepackage[margins=2.5cm,nohead,nofoot]{geometry}
%\usepackage{geometry}
\usepackage{amsfonts}
\usepackage{amstext}
\usepackage{latexsym}
\usepackage{amssymb}
\usepackage{color}


%\include{myPreamble}
\include{qm2pi.local} 

%\ifpdf
%\usepackage[pdftex]{graphicx}
%\else
%\usepackage{graphicx}
%\fi

 % \ifpdf
%  \usepackage{pdfsync}
%  \if


%\title{Brief Article}
%\author{David F. Snyder}
%\author{L.G. Meredith}

%\address{Dept. of Math., Texas State University--San Marcos, San Marcos, TX 78666}
       
\pagestyle{empty}


\begin{document}

\lstset{language=[Objective]Caml,frame=shadowbox}

\input{qm2pi.front}

% section front matter (end)

\input{qm2pi.intro} 
 
% section introduction (end)

% \input{qm2pi.knotations} 

% section notation (end)

\input{qm2pi.process.calculi} 

% section concurrent_process_calculi_and_spatial_logics_ (end)
    
%\input{qm2pi.knots2pi} 

%\input{qm2pi.trefoil} 

%\input{qm2pi.mainthm} 

% subsection basic_interpretation (end)

%\input{qm2pi.rho.presentation} 
\subsection{The syntax and semantics of the notation system}\label{sub:the_syntax_and_semantics_of_the_notation_system} % (fold)

We now summarize a technical presentation of the calculus that
embodies our theory of dynamics. The typical presentation of such a
calculus follows the style of giving generators and relations on
them. The grammar, below, describing term constructors, freely
generates the set of processes, $\Proc$. This set is then quotiented
by a relation known as structural congruence and it is over this set
that the notion of dynamics is expressed. This presentation is
essentially that of \cite{MeredithR05} with the addition of
polyadicity and summation. For readability we have relegated some of
the technical subtleties to an appendix.

\subsubsection{Process grammar}\label{subsub:process_grammar}

\begin{mathpar}
  \inferrule* [lab=synchronization] {} {{M} \bc \pzero \;|\; x?F \;|\; x!C }
  \and
  \inferrule* [lab=abstraction] {} {{F} \bc (x)P}
  \and
  \inferrule* [lab=concretion] {} {{C} \bc \langle Q \rangle}
  \and
  \inferrule* [lab=process] {} {{P,Q} \bc M \;| \;P|Q \;|\; @{x}}
  \and
  \inferrule* [lab=name] {} {{x} \bc \quotep{P}}
\end{mathpar} 

Note that $\vec{x}$ (resp. $\vec{P}$) denotes a vector of names
(resp. processes) of length $|\vec{x}|$ (resp. $|\vec{P}|$). We adopt
the following useful abbreviations.

\begin{mathpar}
   x?(\vec{y}).P := x.(\vec{y})P \and  x\clift{\vec{P}} := x.\clift{\vec{P}}
   \and x!(y) := \lift{x}{\dropn{y}}
   \and \Pi_{i=0}^{n-1}P_i := P_0 | \ldots | P_{n-1}
\end{mathpar}

\subsubsection{Structural congruence}

\paragraph{Free and bound names and alpha-equivalence.} At the
core of structural equivalence is alpha-equivalence which identifies
process that are the same up to a change of variable. Formally, we
recognize the distinction between free and bound names. The free names
of a process, $\freenames{P}$, may be calculated recursively as
follows:

\begin{mathpar}
\freenames{\pzero} := \emptyset
  \and \\
  \freenames{x?(y).P} := \{ x \} \cup (\freenames{P} \setminus \{ y \})
  \and 
  \freenames{x!\langle P \rangle} := \{ x \} \cup \{ P \} 
  \and \\
  \freenames{P|Q} := \freenames{P} \cup \freenames{Q}
  \and \\
  \freenames{@{x}} := \{ x \}
\end{mathpar}

$\pi$
$\quotep{\pi}$

$\freenames{-} : \pi \to \mathcal{P}(\quotep{\pi})$

\begin{eqnarray*}
  \freenames{\pzero} & := & \emptyset \\
  \freenames{x?(y).P} & := & \{ x \} \cup (\freenames{P} \setminus \{ y \}) \\
  \freenames{x!\langle P \rangle} & := & \{ x \} \cup \{ P \} \\
  \freenames{P|Q} & := & \freenames{P} \cup \freenames{Q} \\
  \freenames{\dropn{x}} & := & \{ x \}
\end{eqnarray*}

The bound names of a process, $\boundnames{P}$, are those names occurring in $P$
that are not free. For example, in $x?(y).0$, the name $x$ is free, while $y$ is bound.

\begin{mathpar}
  \inferrule* [lab=monoidal-laws] {} { P|Q \equiv Q|P \and P|0 \equiv P \and P|(Q|R) \equiv (P|Q)|R }
\end{mathpar}

\begin{mathpar}
  \inferrule* [lab=alpha-equivalence] {} { (x)P \equiv (y)P\{y/x\} \and y \not\in \freenames{P} }
\end{mathpar}

\begin{definition}
Then two processes, $P,Q$, are alpha-equivalent if $P = Q\{\vec{y}/\vec{x}\}$ for
some $\vec{x} \in \boundnames{Q},\vec{y} \in \boundnames{P}$, where $Q\{\vec{y}/\vec{x}\}$
denotes the capture-avoiding substitution of $\vec{y}$ for $\vec{x}$ in $Q$.
\end{definition}

\begin{definition}
  The {\em structural congruence} \cite{SangiorgiWalker} , $\equiv$,
  between processes is the least congruence containing
  alpha-equivalence, satisfying the abelian monoid laws
  (associativity, commutativity and $\pzero$ as identity) for parallel
  composition $|$ and for summation $+$.
\end{definition}

\subsection{Name equivalence}

We take name equivalence, written $\nameeq$, to be the smallest
equivalence relation generated by the following rules.

\begin{mathpar}
\inferrule*[lab=Quote-drop]
{ }
{ \quotep{@{x}} \nameeq x }

\inferrule*[lab=Struct-equiv]
{ P \scong Q }
{ \quotep{P} \nameeq \quotep{Q} }
\end{mathpar}

The astute reader will have noticed that the mutual recursion of names
and processes imposes a mutual recursion on alpha-equivalence and
structural equivalence via name-equivalence. Fortunately, all of this
works out pleasantly and we may calculate in the natural way, free of
concern. The reader interested in the details is referred to the
appendix \ref{appendix:rho_details}.

\subsection{Substitution}

We use $\Proc$ for the set of processes, $\QProc$ for the set of
names, and $\id{\{}\vec{y} / \vec{x} \id{\}}$ to denote partial maps,
$s : \QProc \rightarrow \QProc$. A map, $s$ lifts, uniquely, to a map
on process terms, $\widehat{s} : \Proc \rightarrow \Proc$ by the
following equations.

\begin{mathpar}
  (0) \psubstp{Q}{P} := 0 \\
  (R \juxtap S) \psubstp{Q}{P}
  :=    
  (R)\psubstp{Q}{P} \juxtap (S) \psubstp{Q}{P} \\
  (x?(y).R) \psubstp{Q}{P}    
  :=    
  (x)\substp{Q}{P} (z)\concat( (R \psubstn{z}{y}) \psubstp{Q}{P} ) \\
  (\lift{x}{R}) \psubstp{Q}{P}  
  :=
  \lift{(x)\substp{Q}{P}}{ R \psubstp{Q}{P} } \\
%   (\dropn{x})  \psubstp{Q}{P}       
%   := 
%   \left\{ 
%     \begin{array}{ccc} 
%       \dropn{\quotep{Q}} & & x \nameeq \quotep{P} \\
%       \dropn{x} & & otherwise \\
%     \end{array}
%   \right. 
  (\dropn{x})  \psubstp{Q}{P}       
  := 
  \left\{ 
    \begin{array}{ccc} 
      Q & & x \nameeq \quotep{P} \\
      \dropn{x} & & otherwise \\
    \end{array}
  \right.
\end{mathpar}
 

where

\begin{eqnarray}
  (x)\id{\{} \lpquote Q \rpquote / \lpquote P \rpquote \id{\}}            = 
  \left\{ 
    \begin{array}{ccc}
      \lpquote Q \rpquote & & x \nameeq \lpquote P \rpquote \\
      x & & otherwise \\
    \end{array}
  \right. \nonumber
\end{eqnarray}

and $z$ is chosen distinct from $\quotep{P}$, $\quotep{Q}$, the free
names in $Q$, and all the names in $R$. Our $\alpha$-equivalence will
be built in the standard way from this substitution.

\begin{remark}\label{rem:no_self_referential_names}
  One consequence of these definitions is that $\forall P. \quotep{P}
  \not\in \freenames{P}$.
\end{remark}

\subsection{ Dynamic quote: an example }

Anticipating something of what's to come, consider applying the
substitution, $\widehat{\id{\{}u / z \id{\}}}$, to the following pair
of processes, $\lift{w}{y!(z)}$ and $w[ \lpquote y!(z) \rpquote ]$.

\begin{eqnarray}
	\lift{w}{y!(z)}\widehat{\id{\{}u / z \id{\}}}
		& = &
		\lift{w}{y!(u)} \nonumber\\
	w[ \lpquote y!(z) \rpquote ] \widehat{ \id{\{}u / z \id{\}} }
		& = &
		w[ \lpquote y!(z) \rpquote ] \nonumber
\end{eqnarray}

Because the body of the process between quotes is impervious to
substitution, we get radically different answers. In fact, by
examining the first process in an input context,
e.g. $x?(z).\lift{w}{y!(z)}$, we see that the process under the lift
operator may be shaped by prefixed inputs binding a name inside it. In
this sense, the lift operator will be seen as a way to dynamically
construct processes before reifying them as names.

Finally equipped with these standard features we can present the
dynamics of the calculus.

\subsubsection{Operational semantics} 

Finally, we introduce the computational dynamics. What marks these
algebras as distinct from other more traditionally studied algebraic
structures, e.g. vector spaces or polynomial rings, is the manner in
which dynamics is captured. In traditional structures, dynamics is typically
expressed through morphisms between such structures, as in linear maps
between vector spaces or morphisms between rings. In algebras
associated with the semantics of computation, the dynamics is
expressed as part of the algebraic structure itself, through a
reduction reduction relation typically denoted by $\red$. Below, we
give a recursive presentation of this relation for the calculus used
in the encoding.

$\red \subseteq \pi \times \pi$
$\red : \pi \to \mathcal{P}(\pi)$

\begin{mathpar}
  \inferrule* [lab=Comm] { \textsf{match}( x_{src}, x_{trgt} ) } { x_{trgt}?(y)P \; | \; x_{src}!\langle {Q} \rangle \red P\{\quotep{Q}/y}\} }
  \and \\
  \inferrule* [lab=Par] {{P} \red {P}'} {{{P} | {Q}} \red {{P}' | {Q}}}
  \and
  \inferrule* [lab=Equiv]{{{P} \scong {P}'} \andalso {{P}' \red {Q}'} \andalso {{Q}' \scong {Q}}}{{P} \red {Q}}
\end{mathpar}

\begin{eqnarray*}
  match_{\equiv} (\quotep{P},\quotep{Q}) & := & P \equiv Q \\
  match_{\dagger}(\quotep{P},\quotep{Q}) & := & \forall R. P|Q \red^{*} R => R \red^{*} 0 \\
  match_{K}(\quotep{P},\quotep{Q}) & := & K \mbox{ for some context } K
\end{eqnarray*}

$u?(x)P | u!\langle Q \rangle \red P\{\quotep{Q}/x\}$

%We write $\wred$ for $\red^*$, and $P\red$ if $\exists Q $ such that $ P \red Q$.
We write $P\red$ if $\exists Q $ such that $ P \red Q$ and $P\not\red$, otherwise.

\section{Replication}

As mentioned before, it is known that replication (and hence
recursion) can be implemented in a higher-order process algebra
\cite{SangiorgiWalker}. As our first example of calculation with the
machinery thus far presented we give the construction explicitly in
the {\rhoc}.

\begin{eqnarray}
	D_{x} & := & \prefix{x}{y}{(\binpar{\outputp{x}{y}}{@{y}})} \nonumber\\
	\bangp_{x}{P} & := & \binpar{{x}!\langle{\binpar{D_{x}}{P}}\rangle}{D_{x}} \nonumber
\end{eqnarray}

\begin{eqnarray}
	\bangp_{x}{P} & & \nonumber\\
	=
	& {x}!\langle{(\prefix{x}{y}{(\outputp{x}{y} | @{y})) | P}}\rangle 
	      | \prefix{x}{y}{(\outputp{x}{y} | @{y})} & \nonumber\\
	\red
	& (\outputp{x}{y} | @{y})\substn{\quotep{(\prefix{x}{y}{(@{y} | \outputp{x}{y})) | P}}}{y} & \nonumber\\
	=
	& \outputp{x}{\quotep{(\prefix{x}{y}{(\outputp{x}{y} | @{y})) | P}}}
	  | {(\prefix{x}{y}{(\outputp{x}{y} | @{y})) | P}} & \nonumber\\
	\red
	& \ldots & \nonumber\\
	\red^*
	& P | P | \ldots & \nonumber
\end{eqnarray}

Of course, this encoding, as an implementation, runs away, unfolding
$\bangp{P}$ eagerly. A lazier and more implementable replication
operator, restricted to input-guarded processes, may be obtained as follows.

\begin{eqnarray}
\bangp{\prefix{u}{v}{P}} 
	:= 
	\binpar{\lift{x}{\prefix{u}{v}{(\binpar{D(x)}{P})}}}{D(x)} \nonumber
\end{eqnarray}

\begin{remark}
  Note that the lazier definition still does not deal with summation
  or mixed summation (i.e. sums over input and output). The reader is
  invited to construct definitions of replication that deal with these
  features. 

  Further, the definitions are parameterized in a name, $x$. Can you,
  gentle reader, make a definition that eliminates this parameter and
  guarantees no accidental interaction between the replication
  machinery and the process being replicated -- i.e. no accidental
  sharing of names used by the process to get its work done and the
  name(s) used by the replication to effect copying. This latter
  revision of the definition of replication is crucial to obtaining
  the expected identity $!!P \sim !P$.
\end{remark}

\begin{remark}\label{rem:paradoxical_combinator}
  The reader familiar with the lambda calculus will have noticed the
  similarity between $D$ and the paradoxical combinator.

  [Ed. note: the existence of this seems to suggest we have to be more
  restrictive on the set of processes and names we admit if we are to
  support no-cloning.]
\end{remark}

\subsubsection{Bisimulation}

The computational dynamics gives rise to another kind of equivalence,
the equivalence of computational behavior. As previously mentioned
this is typically captured \emph{via} some form of bisimulation.

% The notion we use in this paper is weak barbed bisimulation
% \cite{milner91polyadicpi}.

The notion we use in this paper is derived from weak barbed
bisimulation \cite{milner91polyadicpi}. 

\begin{definition}
An \emph{observation relation}, $\downarrow_{\mathcal N}$, over a set
of names, $\mathcal N$, is the smallest relation satisfying the rules
below.

\infrule[Out-barb]{y \in {\mathcal N}, \; x \nameeq y}
		  {\outputp{x}{v} \downarrow_{\mathcal N} x}
\infrule[Par-barb]{\mbox{$P\downarrow_{\mathcal N} x$ or $Q\downarrow_{\mathcal N} x$}}
		  {\binpar{P}{Q} \downarrow_{\mathcal N} x}

We write $P \Downarrow_{\mathcal N} x$ if there is $Q$ such that 
$P \wred Q$ and $Q \downarrow_{\mathcal N} x$.
\end{definition}

\begin{definition}
%\label{def.bbisim}
An  ${\mathcal N}$-\emph{barbed bisimulation} over a set of names, ${\mathcal N}$, is a symmetric binary relation 
${\mathcal S}_{\mathcal N}$ between agents such that $P\rel{S}_{\mathcal N}Q$ implies:
\begin{enumerate}
\item If $P \red P'$ then $Q \wred Q'$ and $P'\rel{S}_{\mathcal N} Q'$.
\item If $P\downarrow_{\mathcal N} x$, then $Q\Downarrow_{\mathcal N} x$.
\end{enumerate}
$P$ is ${\mathcal N}$-barbed bisimilar to $Q$, written
$P \wbbisim_{\mathcal N} Q$, if $P \rel{S}_{\mathcal N} Q$ for some ${\mathcal N}$-barbed bisimulation ${\mathcal S}_{\mathcal N}$.
\end{definition}

$\mathcal{R} \subseteq \pi \times \pi$

$P \mathcal{R} Q => \forall P'. P \red P' \Rightarrow \exists Q'. Q \red Q', P' \mathcal{R} Q'$

$P \vdash x \Rightarrow Q \vdash x$

\begin{mathpar}
  \inferrule*[lab=Out-barb]{x \nameeq y}{{y}!\langle{Q}\rangle \vdash x}
  \and
  \inferrule*[lab=Par-barb]{\mbox{$P\vdash x$ or $Q\vdash x$}}{\binpar{P}{Q} \vdash x}
\end{mathpar}

\subsubsection{Contexts}

One of the principle advantages of computational calculi like the
$\pi$-calculus is a well-defined notion of context,
contextual-equivalence and a correlation between
contextual-equivalence and notions of bisimulation. The notion of
context allows the decomposition of a process into (sub-)process and
its syntactic environment, its context. Thus, a context may be
thought of as a process with a ``hole'' (written $\Box$) in it. The
application of a context $M$ to a process $P$, written $M[P]$, is
tantamount to filling the hole in $M$ with $P$. In this paper we do
not need the full weight of this theory, but do make use of the notion
of context in the proof the main theorem. 

\begin{mathpar}
  \inferrule* [lab=summation] {} {{M_{M},M_{N}} \bc \Box \;|\; x.M_{A} \;|\; M_{M}+M_{N}}
  \and
  \inferrule* [lab=agent] {} {{M_{A}} \bc (\vec{x})M_{P} \;| \; \clift{P_0,\ldots,M_{P},\ldots,P_N}}
  \and \\
  \inferrule* [lab=process] {} {{M_{P}} \bc M_{N} \;| \;P|M_{P} }
\end{mathpar} 

\begin{mathpar}
  \inferrule* [lab=sychronization] {} {M_{N} \bc \Box \;|\; x?M_{F} \;|\; x!M_{C}}
  \and
  \inferrule* [lab=abstraction] {} {{M_{F}} \bc (x)M_{P} }
  \and
  \inferrule* [lab=concretion] {} {{M_{C}} \bc \langle M_{P} \rangle }
  \and \\
  \inferrule* [lab=process] {} {{M_{P}} \bc M_{N} \;| \;P|M_{P} }
\end{mathpar}

\begin{definition}[contextual application] Given a context $M$, and
  process $P$, we define the \emph{contextual application}, $M[P] :=
  M\{P/\Box\}$. That is, the contextual application of M to P is the
  substitution of $P$ for $\Box$ in $M$.
\end{definition}

$\meaningof{-} : L \to \mathcal{P}(\pi)$

\begin{mathpar}
  \inferrule* [lab=collection] {} {\meaningof{true} = \pi, \and \meaningof{~E} = \pi \setminus \meaningof{E}, \and \meaningof{E_{1} \& E_{2}} = \meaningof{E_{1}} \cap \meaningof{E_{2}}}
\end{mathpar}

\begin{mathpar}
  \inferrule* [lab=structure] {} {\meaningof{0} = \{ P \in \pi | P \equiv 0 \}, \and \\ \meaningof{E_1 | E_2} = \{ P \in \pi | P \equiv P_{1} | P_{2}, P_{1} \in \meaningof{E_{1}}, P_{2} \in \meaningof{E_2}\} }
\end{mathpar}

\begin{mathpar}
 \inferrule* [lab=behavior] {} {\meaningof{\langle a?b \rangle E} = \{ P \in \pi | P \equiv Q | u?(y)P', \\ \and \\\\ \and \\ \;\;\; u \in \meaningof{a}, \forall z.P'\{z/y\} \in \meaningof{E\{z/b\}}\}, \and \\ \meaningof{a!E} = \{ P \in \pi | P \equiv Q | x!\langle P' \rangle, x \in \meaningof{a} P' \in \meaningof{E}\} }
\end{mathpar}

\begin{mathpar}
 \inferrule* [lab=nominal] {} {\meaningof{\quotep{E}} = \{ \quotep{P} \in \quotep{\pi} | P \in \meaningof{E} \}, \and \meaningof{\quotep{P}} = \{ \quotep{Q} \in \quotep{\pi} | P \equiv Q \} \and \\ \meaningof{@\quotep{E}} = \{ P \in \pi | P \equiv @x, x \in \meaningof{E} \}}
\end{mathpar}

\begin{eqnarray*}
  \\
  \meaningof{-} : TS \to ST
\end{eqnarray*}

\begin{eqnarray*}
  \\
  L : TS \to ST
\end{eqnarray*}

\begin{eqnarray*}
  \\
  P \models E \iff P \in \meaningof{E}
\end{eqnarray*}

\begin{eqnarray*}
  P \approx_{L} Q \iff \forall E \in L. P \models E \iff Q \models E
\end{eqnarray*}

\begin{eqnarray*}
  P \approx_{K} Q
\end{eqnarray*}

\begin{eqnarray*}
  P \approx Q
\end{eqnarray*}

$\approx_{K} = \approx = \approx_{L}$

\subsubsection{Contextual duality}

Note that contexts extend the quotation operation to a family of
operations from processes to names. Given a context, $M$, we can
define a \emph{nominal context}, $\quotep{M}$ by $\quotep{M}[P] :=
\quotep{M[P]}$. To foreshadow what is to come we observe that these
operations enjoy a duality with processes very much like the duality
between vectors and maps from vectors to scalars.

Further, because the calculus is essentially higher-order, we have a
correspondence between contexts and processes. More specifically,
given a name $x$ and a context $M$ we can construct $M^{*}_{x}$ such
that 

\begin{mathpar}
  M^{*}_{x} | \lift{x}{P} \red M[P]
\end{mathpar}

namely,

\begin{mathpar}
  M^{*}_{x} := x?(u).M[\dropn{u}]
\end{mathpar}

The dependence of $M^{*}_{x}$ on a name makes it an abstraction, 

\begin{mathpar}
  M^{*} := (x)x?(u).M[\dropn{u}]
\end{mathpar}

\subsection{Additional notation}

It will sometimes be convenient to denote the process a name
quotes. We already have the notation $x = \quotep{P}$, but it will be
convenient to introduce an alternate notation, $\procn{x}$, when we
want to emphasize the connection to the use of the name. Note that, by
virtue of name equivalence, $\quotep{\procn{x}} \nameeq x$; so, the
notation is consistent with previous definitions.

Further, because names have structure it is possible to effect
substitutions on the basis of that structure. This means we need to
upgrade our notation for substitutions, which we accomplish by
adapting comprehension notation. Thus,

\begin{mathpar}
  P\{ y / x : x \in S \}
\end{mathpar}

is interpreted to mean the process derived from P by replacing (in a
capture-avoiding manner) each occurrence of $x$ in $S$ by $y$. For example,

\begin{mathpar}
  P\{ \quotep{\procn{x}|\procn{x}} / x : x \in \freenames{P} \}
\end{mathpar}

will replace each (occurrence) of a free name $x$ in $P$ by
$\quotep{\procn{x}|\procn{x}}$.

Also, we will avail ourselves of the notation $x^{L}$ and $x^{R}$ to
denote injections of a name into disjoint copies of the name
space. There are numerous ways to accomplish this. One example can be
found in \cite{MeredithR05}. This notation overloads to vectors of
names: $\vec{x}^{\pi} := (x_{i}^{\pi} \; : \; 0 \leq i < |\vec{x}| )$ where $\pi \in \{L,R\}$.

We also use $P^{\Box} := P|\Box$.

In \cite{MeredithR05} an interpretation of the new operator is
given. It turns out that there are several possible interpretations
all enjoying the requisite algebraic properties of the operator (see
\cite{milner91polyadicpi}). We will therefore make liberal use of
$(\nu\; \vec{x})P$.

% subsection the_syntax_and_semantics_of_the_notation_system (end)   

\input{qm2pi.qmops} 

\input{qm2pi.sterngerlach} 

\input{qm2pi.metric} 

% section concurrent_process_calculi (end)

%\input{qm2pi.proofsketch}

% section proof sketch (end)

%\input{qm2pi.slviaknots} 

% section spatial logic via knots (end)

\input{qm2pi.conclusion}

% section conclusion (end)

%\input{qm2pi.dtcodes} 

% section wiring algorithm (end)

\input{qm2pi.ack} 

% section acknowledgments (end)

\newpage


\bibliographystyle{plain}   
\bibliography{../../biblios/main.bib}

\input{qm2pi.rhodetails}

\end{document}

 

%\documentclass[12pt]{llncs}
%\documentclass{jktr}

\usepackage[pdftex]{hyperref}                   
\usepackage {listings}
\usepackage {mathpartir}
\usepackage{bcprules}
%\usepackage{listings}
                       
\usepackage{graphicx} 
%\usepackage[margins=2.5cm,nohead,nofoot]{geometry}
%\usepackage{geometry}
\usepackage{amsfonts}
\usepackage{amstext}
\usepackage{latexsym}
\usepackage{amssymb}
\usepackage{color}


%\include{myPreamble}
\include{qm2pi.local} 

%\ifpdf
%\usepackage[pdftex]{graphicx}
%\else
%\usepackage{graphicx}
%\fi

 % \ifpdf
%  \usepackage{pdfsync}
%  \if


%\title{Brief Article}
%\author{David F. Snyder}
%\author{L.G. Meredith}

%\address{Dept. of Math., Texas State University--San Marcos, San Marcos, TX 78666}
       
\pagestyle{empty}


\begin{document}

\lstset{language=[Objective]Caml,frame=shadowbox}

\input{qm2pi.front}

% section front matter (end)

\input{qm2pi.intro} 
 
% section introduction (end)

% \input{qm2pi.knotations} 

% section notation (end)

\input{qm2pi.process.calculi} 

% section concurrent_process_calculi_and_spatial_logics_ (end)
    
%\input{qm2pi.knots2pi} 

%\input{qm2pi.trefoil} 

%\input{qm2pi.mainthm} 

% subsection basic_interpretation (end)

%\input{qm2pi.rho.presentation} 
\subsection{The syntax and semantics of the notation system}\label{sub:the_syntax_and_semantics_of_the_notation_system} % (fold)

We now summarize a technical presentation of the calculus that
embodies our theory of dynamics. The typical presentation of such a
calculus follows the style of giving generators and relations on
them. The grammar, below, describing term constructors, freely
generates the set of processes, $\Proc$. This set is then quotiented
by a relation known as structural congruence and it is over this set
that the notion of dynamics is expressed. This presentation is
essentially that of \cite{MeredithR05} with the addition of
polyadicity and summation. For readability we have relegated some of
the technical subtleties to an appendix.

\subsubsection{Process grammar}\label{subsub:process_grammar}

\begin{mathpar}
  \inferrule* [lab=synchronization] {} {{M} \bc \pzero \;|\; x?F \;|\; x!C }
  \and
  \inferrule* [lab=abstraction] {} {{F} \bc (x)P}
  \and
  \inferrule* [lab=concretion] {} {{C} \bc \langle Q \rangle}
  \and
  \inferrule* [lab=process] {} {{P,Q} \bc M \;| \;P|Q \;|\; @{x}}
  \and
  \inferrule* [lab=name] {} {{x} \bc \quotep{P}}
\end{mathpar} 

Note that $\vec{x}$ (resp. $\vec{P}$) denotes a vector of names
(resp. processes) of length $|\vec{x}|$ (resp. $|\vec{P}|$). We adopt
the following useful abbreviations.

\begin{mathpar}
   x?(\vec{y}).P := x.(\vec{y})P \and  x\clift{\vec{P}} := x.\clift{\vec{P}}
   \and x!(y) := \lift{x}{\dropn{y}}
   \and \Pi_{i=0}^{n-1}P_i := P_0 | \ldots | P_{n-1}
\end{mathpar}

\subsubsection{Structural congruence}

\paragraph{Free and bound names and alpha-equivalence.} At the
core of structural equivalence is alpha-equivalence which identifies
process that are the same up to a change of variable. Formally, we
recognize the distinction between free and bound names. The free names
of a process, $\freenames{P}$, may be calculated recursively as
follows:

\begin{mathpar}
\freenames{\pzero} := \emptyset
  \and \\
  \freenames{x?(y).P} := \{ x \} \cup (\freenames{P} \setminus \{ y \})
  \and 
  \freenames{x!\langle P \rangle} := \{ x \} \cup \{ P \} 
  \and \\
  \freenames{P|Q} := \freenames{P} \cup \freenames{Q}
  \and \\
  \freenames{@{x}} := \{ x \}
\end{mathpar}

$\pi$
$\quotep{\pi}$

$\freenames{-} : \pi \to \mathcal{P}(\quotep{\pi})$

\begin{eqnarray*}
  \freenames{\pzero} & := & \emptyset \\
  \freenames{x?(y).P} & := & \{ x \} \cup (\freenames{P} \setminus \{ y \}) \\
  \freenames{x!\langle P \rangle} & := & \{ x \} \cup \{ P \} \\
  \freenames{P|Q} & := & \freenames{P} \cup \freenames{Q} \\
  \freenames{\dropn{x}} & := & \{ x \}
\end{eqnarray*}

The bound names of a process, $\boundnames{P}$, are those names occurring in $P$
that are not free. For example, in $x?(y).0$, the name $x$ is free, while $y$ is bound.

\begin{mathpar}
  \inferrule* [lab=monoidal-laws] {} { P|Q \equiv Q|P \and P|0 \equiv P \and P|(Q|R) \equiv (P|Q)|R }
\end{mathpar}

\begin{mathpar}
  \inferrule* [lab=alpha-equivalence] {} { (x)P \equiv (y)P\{y/x\} \and y \not\in \freenames{P} }
\end{mathpar}

\begin{definition}
Then two processes, $P,Q$, are alpha-equivalent if $P = Q\{\vec{y}/\vec{x}\}$ for
some $\vec{x} \in \boundnames{Q},\vec{y} \in \boundnames{P}$, where $Q\{\vec{y}/\vec{x}\}$
denotes the capture-avoiding substitution of $\vec{y}$ for $\vec{x}$ in $Q$.
\end{definition}

\begin{definition}
  The {\em structural congruence} \cite{SangiorgiWalker} , $\equiv$,
  between processes is the least congruence containing
  alpha-equivalence, satisfying the abelian monoid laws
  (associativity, commutativity and $\pzero$ as identity) for parallel
  composition $|$ and for summation $+$.
\end{definition}

\subsection{Name equivalence}

We take name equivalence, written $\nameeq$, to be the smallest
equivalence relation generated by the following rules.

\begin{mathpar}
\inferrule*[lab=Quote-drop]
{ }
{ \quotep{@{x}} \nameeq x }

\inferrule*[lab=Struct-equiv]
{ P \scong Q }
{ \quotep{P} \nameeq \quotep{Q} }
\end{mathpar}

The astute reader will have noticed that the mutual recursion of names
and processes imposes a mutual recursion on alpha-equivalence and
structural equivalence via name-equivalence. Fortunately, all of this
works out pleasantly and we may calculate in the natural way, free of
concern. The reader interested in the details is referred to the
appendix \ref{appendix:rho_details}.

\subsection{Substitution}

We use $\Proc$ for the set of processes, $\QProc$ for the set of
names, and $\id{\{}\vec{y} / \vec{x} \id{\}}$ to denote partial maps,
$s : \QProc \rightarrow \QProc$. A map, $s$ lifts, uniquely, to a map
on process terms, $\widehat{s} : \Proc \rightarrow \Proc$ by the
following equations.

\begin{mathpar}
  (0) \psubstp{Q}{P} := 0 \\
  (R \juxtap S) \psubstp{Q}{P}
  :=    
  (R)\psubstp{Q}{P} \juxtap (S) \psubstp{Q}{P} \\
  (x?(y).R) \psubstp{Q}{P}    
  :=    
  (x)\substp{Q}{P} (z)\concat( (R \psubstn{z}{y}) \psubstp{Q}{P} ) \\
  (\lift{x}{R}) \psubstp{Q}{P}  
  :=
  \lift{(x)\substp{Q}{P}}{ R \psubstp{Q}{P} } \\
%   (\dropn{x})  \psubstp{Q}{P}       
%   := 
%   \left\{ 
%     \begin{array}{ccc} 
%       \dropn{\quotep{Q}} & & x \nameeq \quotep{P} \\
%       \dropn{x} & & otherwise \\
%     \end{array}
%   \right. 
  (\dropn{x})  \psubstp{Q}{P}       
  := 
  \left\{ 
    \begin{array}{ccc} 
      Q & & x \nameeq \quotep{P} \\
      \dropn{x} & & otherwise \\
    \end{array}
  \right.
\end{mathpar}
 

where

\begin{eqnarray}
  (x)\id{\{} \lpquote Q \rpquote / \lpquote P \rpquote \id{\}}            = 
  \left\{ 
    \begin{array}{ccc}
      \lpquote Q \rpquote & & x \nameeq \lpquote P \rpquote \\
      x & & otherwise \\
    \end{array}
  \right. \nonumber
\end{eqnarray}

and $z$ is chosen distinct from $\quotep{P}$, $\quotep{Q}$, the free
names in $Q$, and all the names in $R$. Our $\alpha$-equivalence will
be built in the standard way from this substitution.

\begin{remark}\label{rem:no_self_referential_names}
  One consequence of these definitions is that $\forall P. \quotep{P}
  \not\in \freenames{P}$.
\end{remark}

\subsection{ Dynamic quote: an example }

Anticipating something of what's to come, consider applying the
substitution, $\widehat{\id{\{}u / z \id{\}}}$, to the following pair
of processes, $\lift{w}{y!(z)}$ and $w[ \lpquote y!(z) \rpquote ]$.

\begin{eqnarray}
	\lift{w}{y!(z)}\widehat{\id{\{}u / z \id{\}}}
		& = &
		\lift{w}{y!(u)} \nonumber\\
	w[ \lpquote y!(z) \rpquote ] \widehat{ \id{\{}u / z \id{\}} }
		& = &
		w[ \lpquote y!(z) \rpquote ] \nonumber
\end{eqnarray}

Because the body of the process between quotes is impervious to
substitution, we get radically different answers. In fact, by
examining the first process in an input context,
e.g. $x?(z).\lift{w}{y!(z)}$, we see that the process under the lift
operator may be shaped by prefixed inputs binding a name inside it. In
this sense, the lift operator will be seen as a way to dynamically
construct processes before reifying them as names.

Finally equipped with these standard features we can present the
dynamics of the calculus.

\subsubsection{Operational semantics} 

Finally, we introduce the computational dynamics. What marks these
algebras as distinct from other more traditionally studied algebraic
structures, e.g. vector spaces or polynomial rings, is the manner in
which dynamics is captured. In traditional structures, dynamics is typically
expressed through morphisms between such structures, as in linear maps
between vector spaces or morphisms between rings. In algebras
associated with the semantics of computation, the dynamics is
expressed as part of the algebraic structure itself, through a
reduction reduction relation typically denoted by $\red$. Below, we
give a recursive presentation of this relation for the calculus used
in the encoding.

$\red \subseteq \pi \times \pi$
$\red : \pi \to \mathcal{P}(\pi)$

\begin{mathpar}
  \inferrule* [lab=Comm] { \textsf{match}( x_{src}, x_{trgt} ) } { x_{trgt}?(y)P \; | \; x_{src}!\langle {Q} \rangle \red P\{\quotep{Q}/y}\} }
  \and \\
  \inferrule* [lab=Par] {{P} \red {P}'} {{{P} | {Q}} \red {{P}' | {Q}}}
  \and
  \inferrule* [lab=Equiv]{{{P} \scong {P}'} \andalso {{P}' \red {Q}'} \andalso {{Q}' \scong {Q}}}{{P} \red {Q}}
\end{mathpar}

\begin{eqnarray*}
  match_{\equiv} (\quotep{P},\quotep{Q}) & := & P \equiv Q \\
  match_{\dagger}(\quotep{P},\quotep{Q}) & := & \forall R. P|Q \red^{*} R => R \red^{*} 0 \\
  match_{K}(\quotep{P},\quotep{Q}) & := & K \mbox{ for some context } K
\end{eqnarray*}

$u?(x)P | u!\langle Q \rangle \red P\{\quotep{Q}/x\}$

%We write $\wred$ for $\red^*$, and $P\red$ if $\exists Q $ such that $ P \red Q$.
We write $P\red$ if $\exists Q $ such that $ P \red Q$ and $P\not\red$, otherwise.

\section{Replication}

As mentioned before, it is known that replication (and hence
recursion) can be implemented in a higher-order process algebra
\cite{SangiorgiWalker}. As our first example of calculation with the
machinery thus far presented we give the construction explicitly in
the {\rhoc}.

\begin{eqnarray}
	D_{x} & := & \prefix{x}{y}{(\binpar{\outputp{x}{y}}{@{y}})} \nonumber\\
	\bangp_{x}{P} & := & \binpar{{x}!\langle{\binpar{D_{x}}{P}}\rangle}{D_{x}} \nonumber
\end{eqnarray}

\begin{eqnarray}
	\bangp_{x}{P} & & \nonumber\\
	=
	& {x}!\langle{(\prefix{x}{y}{(\outputp{x}{y} | @{y})) | P}}\rangle 
	      | \prefix{x}{y}{(\outputp{x}{y} | @{y})} & \nonumber\\
	\red
	& (\outputp{x}{y} | @{y})\substn{\quotep{(\prefix{x}{y}{(@{y} | \outputp{x}{y})) | P}}}{y} & \nonumber\\
	=
	& \outputp{x}{\quotep{(\prefix{x}{y}{(\outputp{x}{y} | @{y})) | P}}}
	  | {(\prefix{x}{y}{(\outputp{x}{y} | @{y})) | P}} & \nonumber\\
	\red
	& \ldots & \nonumber\\
	\red^*
	& P | P | \ldots & \nonumber
\end{eqnarray}

Of course, this encoding, as an implementation, runs away, unfolding
$\bangp{P}$ eagerly. A lazier and more implementable replication
operator, restricted to input-guarded processes, may be obtained as follows.

\begin{eqnarray}
\bangp{\prefix{u}{v}{P}} 
	:= 
	\binpar{\lift{x}{\prefix{u}{v}{(\binpar{D(x)}{P})}}}{D(x)} \nonumber
\end{eqnarray}

\begin{remark}
  Note that the lazier definition still does not deal with summation
  or mixed summation (i.e. sums over input and output). The reader is
  invited to construct definitions of replication that deal with these
  features. 

  Further, the definitions are parameterized in a name, $x$. Can you,
  gentle reader, make a definition that eliminates this parameter and
  guarantees no accidental interaction between the replication
  machinery and the process being replicated -- i.e. no accidental
  sharing of names used by the process to get its work done and the
  name(s) used by the replication to effect copying. This latter
  revision of the definition of replication is crucial to obtaining
  the expected identity $!!P \sim !P$.
\end{remark}

\begin{remark}\label{rem:paradoxical_combinator}
  The reader familiar with the lambda calculus will have noticed the
  similarity between $D$ and the paradoxical combinator.

  [Ed. note: the existence of this seems to suggest we have to be more
  restrictive on the set of processes and names we admit if we are to
  support no-cloning.]
\end{remark}

\subsubsection{Bisimulation}

The computational dynamics gives rise to another kind of equivalence,
the equivalence of computational behavior. As previously mentioned
this is typically captured \emph{via} some form of bisimulation.

% The notion we use in this paper is weak barbed bisimulation
% \cite{milner91polyadicpi}.

The notion we use in this paper is derived from weak barbed
bisimulation \cite{milner91polyadicpi}. 

\begin{definition}
An \emph{observation relation}, $\downarrow_{\mathcal N}$, over a set
of names, $\mathcal N$, is the smallest relation satisfying the rules
below.

\infrule[Out-barb]{y \in {\mathcal N}, \; x \nameeq y}
		  {\outputp{x}{v} \downarrow_{\mathcal N} x}
\infrule[Par-barb]{\mbox{$P\downarrow_{\mathcal N} x$ or $Q\downarrow_{\mathcal N} x$}}
		  {\binpar{P}{Q} \downarrow_{\mathcal N} x}

We write $P \Downarrow_{\mathcal N} x$ if there is $Q$ such that 
$P \wred Q$ and $Q \downarrow_{\mathcal N} x$.
\end{definition}

\begin{definition}
%\label{def.bbisim}
An  ${\mathcal N}$-\emph{barbed bisimulation} over a set of names, ${\mathcal N}$, is a symmetric binary relation 
${\mathcal S}_{\mathcal N}$ between agents such that $P\rel{S}_{\mathcal N}Q$ implies:
\begin{enumerate}
\item If $P \red P'$ then $Q \wred Q'$ and $P'\rel{S}_{\mathcal N} Q'$.
\item If $P\downarrow_{\mathcal N} x$, then $Q\Downarrow_{\mathcal N} x$.
\end{enumerate}
$P$ is ${\mathcal N}$-barbed bisimilar to $Q$, written
$P \wbbisim_{\mathcal N} Q$, if $P \rel{S}_{\mathcal N} Q$ for some ${\mathcal N}$-barbed bisimulation ${\mathcal S}_{\mathcal N}$.
\end{definition}

$\mathcal{R} \subseteq \pi \times \pi$

$P \mathcal{R} Q => \forall P'. P \red P' \Rightarrow \exists Q'. Q \red Q', P' \mathcal{R} Q'$

$P \vdash x \Rightarrow Q \vdash x$

\begin{mathpar}
  \inferrule*[lab=Out-barb]{x \nameeq y}{{y}!\langle{Q}\rangle \vdash x}
  \and
  \inferrule*[lab=Par-barb]{\mbox{$P\vdash x$ or $Q\vdash x$}}{\binpar{P}{Q} \vdash x}
\end{mathpar}

\subsubsection{Contexts}

One of the principle advantages of computational calculi like the
$\pi$-calculus is a well-defined notion of context,
contextual-equivalence and a correlation between
contextual-equivalence and notions of bisimulation. The notion of
context allows the decomposition of a process into (sub-)process and
its syntactic environment, its context. Thus, a context may be
thought of as a process with a ``hole'' (written $\Box$) in it. The
application of a context $M$ to a process $P$, written $M[P]$, is
tantamount to filling the hole in $M$ with $P$. In this paper we do
not need the full weight of this theory, but do make use of the notion
of context in the proof the main theorem. 

\begin{mathpar}
  \inferrule* [lab=summation] {} {{M_{M},M_{N}} \bc \Box \;|\; x.M_{A} \;|\; M_{M}+M_{N}}
  \and
  \inferrule* [lab=agent] {} {{M_{A}} \bc (\vec{x})M_{P} \;| \; \clift{P_0,\ldots,M_{P},\ldots,P_N}}
  \and \\
  \inferrule* [lab=process] {} {{M_{P}} \bc M_{N} \;| \;P|M_{P} }
\end{mathpar} 

\begin{mathpar}
  \inferrule* [lab=sychronization] {} {M_{N} \bc \Box \;|\; x?M_{F} \;|\; x!M_{C}}
  \and
  \inferrule* [lab=abstraction] {} {{M_{F}} \bc (x)M_{P} }
  \and
  \inferrule* [lab=concretion] {} {{M_{C}} \bc \langle M_{P} \rangle }
  \and \\
  \inferrule* [lab=process] {} {{M_{P}} \bc M_{N} \;| \;P|M_{P} }
\end{mathpar}

\begin{definition}[contextual application] Given a context $M$, and
  process $P$, we define the \emph{contextual application}, $M[P] :=
  M\{P/\Box\}$. That is, the contextual application of M to P is the
  substitution of $P$ for $\Box$ in $M$.
\end{definition}

$\meaningof{-} : L \to \mathcal{P}(\pi)$

\begin{mathpar}
  \inferrule* [lab=collection] {} {\meaningof{true} = \pi, \and \meaningof{~E} = \pi \setminus \meaningof{E}, \and \meaningof{E_{1} \& E_{2}} = \meaningof{E_{1}} \cap \meaningof{E_{2}}}
\end{mathpar}

\begin{mathpar}
  \inferrule* [lab=structure] {} {\meaningof{0} = \{ P \in \pi | P \equiv 0 \}, \and \\ \meaningof{E_1 | E_2} = \{ P \in \pi | P \equiv P_{1} | P_{2}, P_{1} \in \meaningof{E_{1}}, P_{2} \in \meaningof{E_2}\} }
\end{mathpar}

\begin{mathpar}
 \inferrule* [lab=behavior] {} {\meaningof{\langle a?b \rangle E} = \{ P \in \pi | P \equiv Q | u?(y)P', \\ \and \\\\ \and \\ \;\;\; u \in \meaningof{a}, \forall z.P'\{z/y\} \in \meaningof{E\{z/b\}}\}, \and \\ \meaningof{a!E} = \{ P \in \pi | P \equiv Q | x!\langle P' \rangle, x \in \meaningof{a} P' \in \meaningof{E}\} }
\end{mathpar}

\begin{mathpar}
 \inferrule* [lab=nominal] {} {\meaningof{\quotep{E}} = \{ \quotep{P} \in \quotep{\pi} | P \in \meaningof{E} \}, \and \meaningof{\quotep{P}} = \{ \quotep{Q} \in \quotep{\pi} | P \equiv Q \} \and \\ \meaningof{@\quotep{E}} = \{ P \in \pi | P \equiv @x, x \in \meaningof{E} \}}
\end{mathpar}

\begin{eqnarray*}
  \\
  \meaningof{-} : TS \to ST
\end{eqnarray*}

\begin{eqnarray*}
  \\
  L : TS \to ST
\end{eqnarray*}

\begin{eqnarray*}
  \\
  P \models E \iff P \in \meaningof{E}
\end{eqnarray*}

\begin{eqnarray*}
  P \approx_{L} Q \iff \forall E \in L. P \models E \iff Q \models E
\end{eqnarray*}

\begin{eqnarray*}
  P \approx_{K} Q
\end{eqnarray*}

\begin{eqnarray*}
  P \approx Q
\end{eqnarray*}

$\approx_{K} = \approx = \approx_{L}$

\subsubsection{Contextual duality}

Note that contexts extend the quotation operation to a family of
operations from processes to names. Given a context, $M$, we can
define a \emph{nominal context}, $\quotep{M}$ by $\quotep{M}[P] :=
\quotep{M[P]}$. To foreshadow what is to come we observe that these
operations enjoy a duality with processes very much like the duality
between vectors and maps from vectors to scalars.

Further, because the calculus is essentially higher-order, we have a
correspondence between contexts and processes. More specifically,
given a name $x$ and a context $M$ we can construct $M^{*}_{x}$ such
that 

\begin{mathpar}
  M^{*}_{x} | \lift{x}{P} \red M[P]
\end{mathpar}

namely,

\begin{mathpar}
  M^{*}_{x} := x?(u).M[\dropn{u}]
\end{mathpar}

The dependence of $M^{*}_{x}$ on a name makes it an abstraction, 

\begin{mathpar}
  M^{*} := (x)x?(u).M[\dropn{u}]
\end{mathpar}

\subsection{Additional notation}

It will sometimes be convenient to denote the process a name
quotes. We already have the notation $x = \quotep{P}$, but it will be
convenient to introduce an alternate notation, $\procn{x}$, when we
want to emphasize the connection to the use of the name. Note that, by
virtue of name equivalence, $\quotep{\procn{x}} \nameeq x$; so, the
notation is consistent with previous definitions.

Further, because names have structure it is possible to effect
substitutions on the basis of that structure. This means we need to
upgrade our notation for substitutions, which we accomplish by
adapting comprehension notation. Thus,

\begin{mathpar}
  P\{ y / x : x \in S \}
\end{mathpar}

is interpreted to mean the process derived from P by replacing (in a
capture-avoiding manner) each occurrence of $x$ in $S$ by $y$. For example,

\begin{mathpar}
  P\{ \quotep{\procn{x}|\procn{x}} / x : x \in \freenames{P} \}
\end{mathpar}

will replace each (occurrence) of a free name $x$ in $P$ by
$\quotep{\procn{x}|\procn{x}}$.

Also, we will avail ourselves of the notation $x^{L}$ and $x^{R}$ to
denote injections of a name into disjoint copies of the name
space. There are numerous ways to accomplish this. One example can be
found in \cite{MeredithR05}. This notation overloads to vectors of
names: $\vec{x}^{\pi} := (x_{i}^{\pi} \; : \; 0 \leq i < |\vec{x}| )$ where $\pi \in \{L,R\}$.

We also use $P^{\Box} := P|\Box$.

In \cite{MeredithR05} an interpretation of the new operator is
given. It turns out that there are several possible interpretations
all enjoying the requisite algebraic properties of the operator (see
\cite{milner91polyadicpi}). We will therefore make liberal use of
$(\nu\; \vec{x})P$.

% subsection the_syntax_and_semantics_of_the_notation_system (end)   

\input{qm2pi.qmops} 

\input{qm2pi.sterngerlach} 

\input{qm2pi.metric} 

% section concurrent_process_calculi (end)

%\input{qm2pi.proofsketch}

% section proof sketch (end)

%\input{qm2pi.slviaknots} 

% section spatial logic via knots (end)

\input{qm2pi.conclusion}

% section conclusion (end)

%\input{qm2pi.dtcodes} 

% section wiring algorithm (end)

\input{qm2pi.ack} 

% section acknowledgments (end)

\newpage


\bibliographystyle{plain}   
\bibliography{../../biblios/main.bib}

\input{qm2pi.rhodetails}

\end{document}

 

% subsection basic_interpretation (end)

%\input{qm2pi.rho.presentation} 
\subsection{The syntax and semantics of the notation system}\label{sub:the_syntax_and_semantics_of_the_notation_system} % (fold)

We now summarize a technical presentation of the calculus that
embodies our theory of dynamics. The typical presentation of such a
calculus follows the style of giving generators and relations on
them. The grammar, below, describing term constructors, freely
generates the set of processes, $\Proc$. This set is then quotiented
by a relation known as structural congruence and it is over this set
that the notion of dynamics is expressed. This presentation is
essentially that of \cite{MeredithR05} with the addition of
polyadicity and summation. For readability we have relegated some of
the technical subtleties to an appendix.

\subsubsection{Process grammar}\label{subsub:process_grammar}

\begin{mathpar}
  \inferrule* [lab=synchronization] {} {{M} \bc \pzero \;|\; x?F \;|\; x!C }
  \and
  \inferrule* [lab=abstraction] {} {{F} \bc (x)P}
  \and
  \inferrule* [lab=concretion] {} {{C} \bc \langle Q \rangle}
  \and
  \inferrule* [lab=process] {} {{P,Q} \bc M \;| \;P|Q \;|\; @{x}}
  \and
  \inferrule* [lab=name] {} {{x} \bc \quotep{P}}
\end{mathpar} 

Note that $\vec{x}$ (resp. $\vec{P}$) denotes a vector of names
(resp. processes) of length $|\vec{x}|$ (resp. $|\vec{P}|$). We adopt
the following useful abbreviations.

\begin{mathpar}
   x?(\vec{y}).P := x.(\vec{y})P \and  x\clift{\vec{P}} := x.\clift{\vec{P}}
   \and x!(y) := \lift{x}{\dropn{y}}
   \and \Pi_{i=0}^{n-1}P_i := P_0 | \ldots | P_{n-1}
\end{mathpar}

\subsubsection{Structural congruence}

\paragraph{Free and bound names and alpha-equivalence.} At the
core of structural equivalence is alpha-equivalence which identifies
process that are the same up to a change of variable. Formally, we
recognize the distinction between free and bound names. The free names
of a process, $\freenames{P}$, may be calculated recursively as
follows:

\begin{mathpar}
\freenames{\pzero} := \emptyset
  \and \\
  \freenames{x?(y).P} := \{ x \} \cup (\freenames{P} \setminus \{ y \})
  \and 
  \freenames{x!\langle P \rangle} := \{ x \} \cup \{ P \} 
  \and \\
  \freenames{P|Q} := \freenames{P} \cup \freenames{Q}
  \and \\
  \freenames{@{x}} := \{ x \}
\end{mathpar}

$\pi$
$\quotep{\pi}$

$\freenames{-} : \pi \to \mathcal{P}(\quotep{\pi})$

\begin{eqnarray*}
  \freenames{\pzero} & := & \emptyset \\
  \freenames{x?(y).P} & := & \{ x \} \cup (\freenames{P} \setminus \{ y \}) \\
  \freenames{x!\langle P \rangle} & := & \{ x \} \cup \{ P \} \\
  \freenames{P|Q} & := & \freenames{P} \cup \freenames{Q} \\
  \freenames{\dropn{x}} & := & \{ x \}
\end{eqnarray*}

The bound names of a process, $\boundnames{P}$, are those names occurring in $P$
that are not free. For example, in $x?(y).0$, the name $x$ is free, while $y$ is bound.

\begin{mathpar}
  \inferrule* [lab=monoidal-laws] {} { P|Q \equiv Q|P \and P|0 \equiv P \and P|(Q|R) \equiv (P|Q)|R }
\end{mathpar}

\begin{mathpar}
  \inferrule* [lab=alpha-equivalence] {} { (x)P \equiv (y)P\{y/x\} \and y \not\in \freenames{P} }
\end{mathpar}

\begin{definition}
Then two processes, $P,Q$, are alpha-equivalent if $P = Q\{\vec{y}/\vec{x}\}$ for
some $\vec{x} \in \boundnames{Q},\vec{y} \in \boundnames{P}$, where $Q\{\vec{y}/\vec{x}\}$
denotes the capture-avoiding substitution of $\vec{y}$ for $\vec{x}$ in $Q$.
\end{definition}

\begin{definition}
  The {\em structural congruence} \cite{SangiorgiWalker} , $\equiv$,
  between processes is the least congruence containing
  alpha-equivalence, satisfying the abelian monoid laws
  (associativity, commutativity and $\pzero$ as identity) for parallel
  composition $|$ and for summation $+$.
\end{definition}

\subsection{Name equivalence}

We take name equivalence, written $\nameeq$, to be the smallest
equivalence relation generated by the following rules.

\begin{mathpar}
\inferrule*[lab=Quote-drop]
{ }
{ \quotep{@{x}} \nameeq x }

\inferrule*[lab=Struct-equiv]
{ P \scong Q }
{ \quotep{P} \nameeq \quotep{Q} }
\end{mathpar}

The astute reader will have noticed that the mutual recursion of names
and processes imposes a mutual recursion on alpha-equivalence and
structural equivalence via name-equivalence. Fortunately, all of this
works out pleasantly and we may calculate in the natural way, free of
concern. The reader interested in the details is referred to the
appendix \ref{appendix:rho_details}.

\subsection{Substitution}

We use $\Proc$ for the set of processes, $\QProc$ for the set of
names, and $\id{\{}\vec{y} / \vec{x} \id{\}}$ to denote partial maps,
$s : \QProc \rightarrow \QProc$. A map, $s$ lifts, uniquely, to a map
on process terms, $\widehat{s} : \Proc \rightarrow \Proc$ by the
following equations.

\begin{mathpar}
  (0) \psubstp{Q}{P} := 0 \\
  (R \juxtap S) \psubstp{Q}{P}
  :=    
  (R)\psubstp{Q}{P} \juxtap (S) \psubstp{Q}{P} \\
  (x?(y).R) \psubstp{Q}{P}    
  :=    
  (x)\substp{Q}{P} (z)\concat( (R \psubstn{z}{y}) \psubstp{Q}{P} ) \\
  (\lift{x}{R}) \psubstp{Q}{P}  
  :=
  \lift{(x)\substp{Q}{P}}{ R \psubstp{Q}{P} } \\
%   (\dropn{x})  \psubstp{Q}{P}       
%   := 
%   \left\{ 
%     \begin{array}{ccc} 
%       \dropn{\quotep{Q}} & & x \nameeq \quotep{P} \\
%       \dropn{x} & & otherwise \\
%     \end{array}
%   \right. 
  (\dropn{x})  \psubstp{Q}{P}       
  := 
  \left\{ 
    \begin{array}{ccc} 
      Q & & x \nameeq \quotep{P} \\
      \dropn{x} & & otherwise \\
    \end{array}
  \right.
\end{mathpar}
 

where

\begin{eqnarray}
  (x)\id{\{} \lpquote Q \rpquote / \lpquote P \rpquote \id{\}}            = 
  \left\{ 
    \begin{array}{ccc}
      \lpquote Q \rpquote & & x \nameeq \lpquote P \rpquote \\
      x & & otherwise \\
    \end{array}
  \right. \nonumber
\end{eqnarray}

and $z$ is chosen distinct from $\quotep{P}$, $\quotep{Q}$, the free
names in $Q$, and all the names in $R$. Our $\alpha$-equivalence will
be built in the standard way from this substitution.

\begin{remark}\label{rem:no_self_referential_names}
  One consequence of these definitions is that $\forall P. \quotep{P}
  \not\in \freenames{P}$.
\end{remark}

\subsection{ Dynamic quote: an example }

Anticipating something of what's to come, consider applying the
substitution, $\widehat{\id{\{}u / z \id{\}}}$, to the following pair
of processes, $\lift{w}{y!(z)}$ and $w[ \lpquote y!(z) \rpquote ]$.

\begin{eqnarray}
	\lift{w}{y!(z)}\widehat{\id{\{}u / z \id{\}}}
		& = &
		\lift{w}{y!(u)} \nonumber\\
	w[ \lpquote y!(z) \rpquote ] \widehat{ \id{\{}u / z \id{\}} }
		& = &
		w[ \lpquote y!(z) \rpquote ] \nonumber
\end{eqnarray}

Because the body of the process between quotes is impervious to
substitution, we get radically different answers. In fact, by
examining the first process in an input context,
e.g. $x?(z).\lift{w}{y!(z)}$, we see that the process under the lift
operator may be shaped by prefixed inputs binding a name inside it. In
this sense, the lift operator will be seen as a way to dynamically
construct processes before reifying them as names.

Finally equipped with these standard features we can present the
dynamics of the calculus.

\subsubsection{Operational semantics} 

Finally, we introduce the computational dynamics. What marks these
algebras as distinct from other more traditionally studied algebraic
structures, e.g. vector spaces or polynomial rings, is the manner in
which dynamics is captured. In traditional structures, dynamics is typically
expressed through morphisms between such structures, as in linear maps
between vector spaces or morphisms between rings. In algebras
associated with the semantics of computation, the dynamics is
expressed as part of the algebraic structure itself, through a
reduction reduction relation typically denoted by $\red$. Below, we
give a recursive presentation of this relation for the calculus used
in the encoding.

$\red \subseteq \pi \times \pi$
$\red : \pi \to \mathcal{P}(\pi)$

\begin{mathpar}
  \inferrule* [lab=Comm] { \textsf{match}( x_{src}, x_{trgt} ) } { x_{trgt}?(y)P \; | \; x_{src}!\langle {Q} \rangle \red P\{\quotep{Q}/y}\} }
  \and \\
  \inferrule* [lab=Par] {{P} \red {P}'} {{{P} | {Q}} \red {{P}' | {Q}}}
  \and
  \inferrule* [lab=Equiv]{{{P} \scong {P}'} \andalso {{P}' \red {Q}'} \andalso {{Q}' \scong {Q}}}{{P} \red {Q}}
\end{mathpar}

\begin{eqnarray*}
  match_{\equiv} (\quotep{P},\quotep{Q}) & := & P \equiv Q \\
  match_{\dagger}(\quotep{P},\quotep{Q}) & := & \forall R. P|Q \red^{*} R => R \red^{*} 0 \\
  match_{K}(\quotep{P},\quotep{Q}) & := & K \mbox{ for some context } K
\end{eqnarray*}

$u?(x)P | u!\langle Q \rangle \red P\{\quotep{Q}/x\}$

%We write $\wred$ for $\red^*$, and $P\red$ if $\exists Q $ such that $ P \red Q$.
We write $P\red$ if $\exists Q $ such that $ P \red Q$ and $P\not\red$, otherwise.

\section{Replication}

As mentioned before, it is known that replication (and hence
recursion) can be implemented in a higher-order process algebra
\cite{SangiorgiWalker}. As our first example of calculation with the
machinery thus far presented we give the construction explicitly in
the {\rhoc}.

\begin{eqnarray}
	D_{x} & := & \prefix{x}{y}{(\binpar{\outputp{x}{y}}{@{y}})} \nonumber\\
	\bangp_{x}{P} & := & \binpar{{x}!\langle{\binpar{D_{x}}{P}}\rangle}{D_{x}} \nonumber
\end{eqnarray}

\begin{eqnarray}
	\bangp_{x}{P} & & \nonumber\\
	=
	& {x}!\langle{(\prefix{x}{y}{(\outputp{x}{y} | @{y})) | P}}\rangle 
	      | \prefix{x}{y}{(\outputp{x}{y} | @{y})} & \nonumber\\
	\red
	& (\outputp{x}{y} | @{y})\substn{\quotep{(\prefix{x}{y}{(@{y} | \outputp{x}{y})) | P}}}{y} & \nonumber\\
	=
	& \outputp{x}{\quotep{(\prefix{x}{y}{(\outputp{x}{y} | @{y})) | P}}}
	  | {(\prefix{x}{y}{(\outputp{x}{y} | @{y})) | P}} & \nonumber\\
	\red
	& \ldots & \nonumber\\
	\red^*
	& P | P | \ldots & \nonumber
\end{eqnarray}

Of course, this encoding, as an implementation, runs away, unfolding
$\bangp{P}$ eagerly. A lazier and more implementable replication
operator, restricted to input-guarded processes, may be obtained as follows.

\begin{eqnarray}
\bangp{\prefix{u}{v}{P}} 
	:= 
	\binpar{\lift{x}{\prefix{u}{v}{(\binpar{D(x)}{P})}}}{D(x)} \nonumber
\end{eqnarray}

\begin{remark}
  Note that the lazier definition still does not deal with summation
  or mixed summation (i.e. sums over input and output). The reader is
  invited to construct definitions of replication that deal with these
  features. 

  Further, the definitions are parameterized in a name, $x$. Can you,
  gentle reader, make a definition that eliminates this parameter and
  guarantees no accidental interaction between the replication
  machinery and the process being replicated -- i.e. no accidental
  sharing of names used by the process to get its work done and the
  name(s) used by the replication to effect copying. This latter
  revision of the definition of replication is crucial to obtaining
  the expected identity $!!P \sim !P$.
\end{remark}

\begin{remark}\label{rem:paradoxical_combinator}
  The reader familiar with the lambda calculus will have noticed the
  similarity between $D$ and the paradoxical combinator.

  [Ed. note: the existence of this seems to suggest we have to be more
  restrictive on the set of processes and names we admit if we are to
  support no-cloning.]
\end{remark}

\subsubsection{Bisimulation}

The computational dynamics gives rise to another kind of equivalence,
the equivalence of computational behavior. As previously mentioned
this is typically captured \emph{via} some form of bisimulation.

% The notion we use in this paper is weak barbed bisimulation
% \cite{milner91polyadicpi}.

The notion we use in this paper is derived from weak barbed
bisimulation \cite{milner91polyadicpi}. 

\begin{definition}
An \emph{observation relation}, $\downarrow_{\mathcal N}$, over a set
of names, $\mathcal N$, is the smallest relation satisfying the rules
below.

\infrule[Out-barb]{y \in {\mathcal N}, \; x \nameeq y}
		  {\outputp{x}{v} \downarrow_{\mathcal N} x}
\infrule[Par-barb]{\mbox{$P\downarrow_{\mathcal N} x$ or $Q\downarrow_{\mathcal N} x$}}
		  {\binpar{P}{Q} \downarrow_{\mathcal N} x}

We write $P \Downarrow_{\mathcal N} x$ if there is $Q$ such that 
$P \wred Q$ and $Q \downarrow_{\mathcal N} x$.
\end{definition}

\begin{definition}
%\label{def.bbisim}
An  ${\mathcal N}$-\emph{barbed bisimulation} over a set of names, ${\mathcal N}$, is a symmetric binary relation 
${\mathcal S}_{\mathcal N}$ between agents such that $P\rel{S}_{\mathcal N}Q$ implies:
\begin{enumerate}
\item If $P \red P'$ then $Q \wred Q'$ and $P'\rel{S}_{\mathcal N} Q'$.
\item If $P\downarrow_{\mathcal N} x$, then $Q\Downarrow_{\mathcal N} x$.
\end{enumerate}
$P$ is ${\mathcal N}$-barbed bisimilar to $Q$, written
$P \wbbisim_{\mathcal N} Q$, if $P \rel{S}_{\mathcal N} Q$ for some ${\mathcal N}$-barbed bisimulation ${\mathcal S}_{\mathcal N}$.
\end{definition}

$\mathcal{R} \subseteq \pi \times \pi$

$P \mathcal{R} Q => \forall P'. P \red P' \Rightarrow \exists Q'. Q \red Q', P' \mathcal{R} Q'$

$P \vdash x \Rightarrow Q \vdash x$

\begin{mathpar}
  \inferrule*[lab=Out-barb]{x \nameeq y}{{y}!\langle{Q}\rangle \vdash x}
  \and
  \inferrule*[lab=Par-barb]{\mbox{$P\vdash x$ or $Q\vdash x$}}{\binpar{P}{Q} \vdash x}
\end{mathpar}

\subsubsection{Contexts}

One of the principle advantages of computational calculi like the
$\pi$-calculus is a well-defined notion of context,
contextual-equivalence and a correlation between
contextual-equivalence and notions of bisimulation. The notion of
context allows the decomposition of a process into (sub-)process and
its syntactic environment, its context. Thus, a context may be
thought of as a process with a ``hole'' (written $\Box$) in it. The
application of a context $M$ to a process $P$, written $M[P]$, is
tantamount to filling the hole in $M$ with $P$. In this paper we do
not need the full weight of this theory, but do make use of the notion
of context in the proof the main theorem. 

\begin{mathpar}
  \inferrule* [lab=summation] {} {{M_{M},M_{N}} \bc \Box \;|\; x.M_{A} \;|\; M_{M}+M_{N}}
  \and
  \inferrule* [lab=agent] {} {{M_{A}} \bc (\vec{x})M_{P} \;| \; \clift{P_0,\ldots,M_{P},\ldots,P_N}}
  \and \\
  \inferrule* [lab=process] {} {{M_{P}} \bc M_{N} \;| \;P|M_{P} }
\end{mathpar} 

\begin{mathpar}
  \inferrule* [lab=sychronization] {} {M_{N} \bc \Box \;|\; x?M_{F} \;|\; x!M_{C}}
  \and
  \inferrule* [lab=abstraction] {} {{M_{F}} \bc (x)M_{P} }
  \and
  \inferrule* [lab=concretion] {} {{M_{C}} \bc \langle M_{P} \rangle }
  \and \\
  \inferrule* [lab=process] {} {{M_{P}} \bc M_{N} \;| \;P|M_{P} }
\end{mathpar}

\begin{definition}[contextual application] Given a context $M$, and
  process $P$, we define the \emph{contextual application}, $M[P] :=
  M\{P/\Box\}$. That is, the contextual application of M to P is the
  substitution of $P$ for $\Box$ in $M$.
\end{definition}

$\meaningof{-} : L \to \mathcal{P}(\pi)$

\begin{mathpar}
  \inferrule* [lab=collection] {} {\meaningof{true} = \pi, \and \meaningof{~E} = \pi \setminus \meaningof{E}, \and \meaningof{E_{1} \& E_{2}} = \meaningof{E_{1}} \cap \meaningof{E_{2}}}
\end{mathpar}

\begin{mathpar}
  \inferrule* [lab=structure] {} {\meaningof{0} = \{ P \in \pi | P \equiv 0 \}, \and \\ \meaningof{E_1 | E_2} = \{ P \in \pi | P \equiv P_{1} | P_{2}, P_{1} \in \meaningof{E_{1}}, P_{2} \in \meaningof{E_2}\} }
\end{mathpar}

\begin{mathpar}
 \inferrule* [lab=behavior] {} {\meaningof{\langle a?b \rangle E} = \{ P \in \pi | P \equiv Q | u?(y)P', \\ \and \\\\ \and \\ \;\;\; u \in \meaningof{a}, \forall z.P'\{z/y\} \in \meaningof{E\{z/b\}}\}, \and \\ \meaningof{a!E} = \{ P \in \pi | P \equiv Q | x!\langle P' \rangle, x \in \meaningof{a} P' \in \meaningof{E}\} }
\end{mathpar}

\begin{mathpar}
 \inferrule* [lab=nominal] {} {\meaningof{\quotep{E}} = \{ \quotep{P} \in \quotep{\pi} | P \in \meaningof{E} \}, \and \meaningof{\quotep{P}} = \{ \quotep{Q} \in \quotep{\pi} | P \equiv Q \} \and \\ \meaningof{@\quotep{E}} = \{ P \in \pi | P \equiv @x, x \in \meaningof{E} \}}
\end{mathpar}

\begin{eqnarray*}
  \\
  \meaningof{-} : TS \to ST
\end{eqnarray*}

\begin{eqnarray*}
  \\
  L : TS \to ST
\end{eqnarray*}

\begin{eqnarray*}
  \\
  P \models E \iff P \in \meaningof{E}
\end{eqnarray*}

\begin{eqnarray*}
  P \approx_{L} Q \iff \forall E \in L. P \models E \iff Q \models E
\end{eqnarray*}

\begin{eqnarray*}
  P \approx_{K} Q
\end{eqnarray*}

\begin{eqnarray*}
  P \approx Q
\end{eqnarray*}

$\approx_{K} = \approx = \approx_{L}$

\subsubsection{Contextual duality}

Note that contexts extend the quotation operation to a family of
operations from processes to names. Given a context, $M$, we can
define a \emph{nominal context}, $\quotep{M}$ by $\quotep{M}[P] :=
\quotep{M[P]}$. To foreshadow what is to come we observe that these
operations enjoy a duality with processes very much like the duality
between vectors and maps from vectors to scalars.

Further, because the calculus is essentially higher-order, we have a
correspondence between contexts and processes. More specifically,
given a name $x$ and a context $M$ we can construct $M^{*}_{x}$ such
that 

\begin{mathpar}
  M^{*}_{x} | \lift{x}{P} \red M[P]
\end{mathpar}

namely,

\begin{mathpar}
  M^{*}_{x} := x?(u).M[\dropn{u}]
\end{mathpar}

The dependence of $M^{*}_{x}$ on a name makes it an abstraction, 

\begin{mathpar}
  M^{*} := (x)x?(u).M[\dropn{u}]
\end{mathpar}

\subsection{Additional notation}

It will sometimes be convenient to denote the process a name
quotes. We already have the notation $x = \quotep{P}$, but it will be
convenient to introduce an alternate notation, $\procn{x}$, when we
want to emphasize the connection to the use of the name. Note that, by
virtue of name equivalence, $\quotep{\procn{x}} \nameeq x$; so, the
notation is consistent with previous definitions.

Further, because names have structure it is possible to effect
substitutions on the basis of that structure. This means we need to
upgrade our notation for substitutions, which we accomplish by
adapting comprehension notation. Thus,

\begin{mathpar}
  P\{ y / x : x \in S \}
\end{mathpar}

is interpreted to mean the process derived from P by replacing (in a
capture-avoiding manner) each occurrence of $x$ in $S$ by $y$. For example,

\begin{mathpar}
  P\{ \quotep{\procn{x}|\procn{x}} / x : x \in \freenames{P} \}
\end{mathpar}

will replace each (occurrence) of a free name $x$ in $P$ by
$\quotep{\procn{x}|\procn{x}}$.

Also, we will avail ourselves of the notation $x^{L}$ and $x^{R}$ to
denote injections of a name into disjoint copies of the name
space. There are numerous ways to accomplish this. One example can be
found in \cite{MeredithR05}. This notation overloads to vectors of
names: $\vec{x}^{\pi} := (x_{i}^{\pi} \; : \; 0 \leq i < |\vec{x}| )$ where $\pi \in \{L,R\}$.

We also use $P^{\Box} := P|\Box$.

In \cite{MeredithR05} an interpretation of the new operator is
given. It turns out that there are several possible interpretations
all enjoying the requisite algebraic properties of the operator (see
\cite{milner91polyadicpi}). We will therefore make liberal use of
$(\nu\; \vec{x})P$.

% subsection the_syntax_and_semantics_of_the_notation_system (end)   

\section{Interpretation of QM}
\subsection{Supporting definitions}
\subsubsection{Multiplication}
\begin{mathpar}
  \quotep{Q} \cdot \quotep{R} := \quotep{Q|R}
  \and \\
  \quotep{Q} \cdot P := P\{ \quotep{Q|R} / \quotep{R} : \quotep{R} \in \freenames{P} \}
\end{mathpar}

\paragraph{Discussion}
The first line needs little explanation. The second line says that
each free name of the process is replaced with the multiplication of
that name by the scalar. Multiplication of a scalar (name) by a state
(process) results in a process all the names of which have been `moved
over' by parallel composition with the process the scalar
quotes. There is a subtlety that the bound names have to be
manipulated so that multiplied names aren't accidentally
captured. There are many ways to achieve this.

\begin{remark}\label{rem:multiplication_identities}
  The reader is invited to verify that for all $x,y,z \in \QProc$ and $P \in \Proc$
  \begin{mathpar}
    x \cdot \quotep{0} \equiv x 
    \and
    x \cdot y \equiv y \cdot x
    \and
    x \cdot (y \cdot z) \equiv (x \cdot y) \cdot z
    \and \\
    \quotep{0} \cdot P \equiv P
    \and \\
    x \cdot (y \cdot P) \equiv (x \cdot y) \cdot P
    \and \\
    x \cdot (P|Q) \equiv (x \cdot P) | (x \cdot Q)
    \and \\    
  \end{mathpar}
\end{remark}

\subsubsection{Tensor product}

We define a tensor product on processes by structural induction.

\paragraph{Tensor of sums} First note that all summations, including
$\pzero$ and sequence, can be written $\Sigma_{i} x_{i}.A_{i} +
\Sigma_{j} x_{j}.C_{j}$, where we have grouped input-guarded processes
together and output-guarded processes together.

Thus, we can define the tensor product of two summations, $N_{1}\otimes N_{2}$, where

\begin{mathpar}
  N_{1} := \Sigma_{i} x_{i}.A_{i} + \Sigma_{j} x_{j}.C_{j}
  \and
  N_{2} := \Sigma_{i'} y_{i'}.B_{i'} + \Sigma_{j'} y_{j'}.D_{j'} 
\end{mathpar}

as follows.

\begin{mathpar}
  \Sigma_{i} x_{i}.A_{i} + \Sigma_{j} x_{j}.C_{j} \otimes \Sigma_{i'}
  y_{i'}.B_{i'} + \Sigma_{j'} y_{j'}.D_{j'} 
  \and \\
  := \; \Sigma_{i} \Sigma_{i'} \quotep{\stackrel{\vee}{x_{i}}| \stackrel{\vee}{y_{i'}}}.(A_{i}\otimes B_{i'}) \; | \; \Sigma_{i'} \Sigma_{i} \quotep{\stackrel{\vee}{y_{i'}}|\stackrel{\vee}{x_{i}}}.(B_{i'}\otimes A_{i})
  \and
  \;\; | \;\; \Sigma_{j} \Sigma_{j'} \quotep{\stackrel{\vee}{x_{j}}|\stackrel{\vee}{y_{j'}}}.(A_{j}\otimes B_{j'}) \; | \; \Sigma_{j'} \Sigma_{j} \quotep{\stackrel{\vee}{y_{j'}}|\stackrel{\vee}{x_{j}}}.(B_{j'}\otimes A_{j})
\end{mathpar}

\begin{remark}
  Do we need to $x^{L}$ and $y^{R}$ for this construction as well?
\end{remark}

\paragraph{Tensor of parallel compositions} Next, we distribute tensor
over par.

\begin{mathpar}
  P_{1}|P_{2} \otimes Q_{1}|Q_{2} := (P_{1} \otimes Q_{1}) | (P_{1}
  \otimes Q_{2}) | (P_{2} \otimes Q_{1}) | (P_{2} \otimes Q_{2})
\end{mathpar}

\paragraph{Tensor with dropped names} We treat tensor of a
process with a dropped name as parallel composition.

\begin{mathpar}
  P \otimes \dropn{x} := P | \dropn{x}
\end{mathpar}

\paragraph{Tensor of agents}

Finally, we need to define tensor on agents. Note that the definition
of tensor on normal products only tensors inputs with inputs and
outputs with outputs. Thus, we only have to define the operation on
``homogeneous'' pairings.

\begin{mathpar}
  (\vec{x})P \otimes (\vec{y})Q
  \and \\
  := (x_{0}^{L}|y_{0}^{R},\ldots,x_{0}^{L}|y_{n}^{R},\ldots,x_{m}^{L}|y_{0}^{R},\ldots,x_{m}^{L}|y_{n}^R)(P\{ \vec{x}^{L}/\vec{x}\} \otimes Q \{ \vec{y}^{R}/\vec{y}\})
  \and \\
  \clift{\vec{P}} \otimes \clift{\vec{Q}}
  \and \\
  := \clift{P_{0}\otimes Q_{0},\ldots,P_{0}\otimes Q_{n},\ldots,P_{m}\otimes Q_{0},\ldots,P_{m}\otimes Q_{n}}
\end{mathpar}

\begin{remark}
  Observe that arities of tensored abstractions matches arities of
  tensored concretions if the original arities matched. Note also that
  the length of the arities corresponds to the increase in dimension
  we see in ordinary vector space tensor product.
\end{remark}

\begin{remark}
  Operationally, this definition distributes the tensor down to
  components ``linked'' by summation. Tensor over summation is
  intriguing in that it mixes names. Moreover, as a consequence of the
  way it mixes names we have the identities for all $x \in \QProc$ and
  $P,Q \in \Proc$

  \begin{mathpar}
    (x \cdot P) \otimes Q \equiv x \cdot (P \otimes Q) \equiv P \otimes (x \cdot Q)
    \and
    P \otimes \pzero \equiv P
  \end{mathpar}

  that the reader is invited to verify.
\end{remark}

\subsubsection{Annihilation}
\begin{mathpar}
  P^{\perp} := \{ Q | \forall R. P|Q \red^{*} R \Rightarrow R \red^{*} \pzero \}
  \and \\
  P^{\underline{\perp}} := \Sigma_{Q \in P^{\perp}} \quotep{Q}?(y).(\dropn{y}|Q) | \Sigma_{Q \in P^{\perp}} \quotep{Q}\clift{\Box}
\end{mathpar}

\paragraph{Discussion} The reader will note that $P^{\perp}$ is a
\emph{set} of processes, while $P^{\underline{\perp}}$ is a
\emph{context}. We call the set $P^{\perp}$ the \emph{annihilators} of
$P$. The parallel composition of a process in the annihilators of $P$
with $P$ will result in a process, the state space of which has all
paths eventually leading to $\pzero$. Execution may endure loops; but
under reasonable conditions of fairness (naturally guaranteed under
most notions of bisimulation) such a composite process cannot get
stuck in such a loop and will, eventually pop out and terminate.

The context $P^{\underline{\perp}}$ is ready and willing to ``take the
$P$ out of'' the process to which it is applied. It will effectively
transmit the code of the process to which it is applied to one of the
annihilators and run the process against it.

\subsubsection{Evaluation}
We fix $M$ a domain of fully abstract interpretation with an equality
coincident with bisimulation. We take $\meaningof{\cdot} : \Proc \to
M$ to be the map interpreting processes and $\nmeaningof{\cdot} : \M
\to Proc$ to be the map running the other way. Then we define

\begin{mathpar}
  \int P := \nmeaningof{\meaningof{P}}
\end{mathpar}

\paragraph{Discussion}
There are many fully abstract interpretations of Milner's
$\pi$-calculus. Any of them can be used as a basis for interpreting
the reflective calculus here. Equipped with such a domain it is
largely a matter of grinding through to check that the Yoneda
construction for the normalization-by-evaluation program can be
extended to this setting.

\begin{remark}
  The reader is invited to verify that $\int (P^{\underline{\perp}}[P]) = 0$.
\end{remark}

\subsection{Quantum mechanics}

Table \ref{tbl:core_qm_op_defns} gives the core operational definitions

\begin{table}[htp]\label{tbl:core_qm_op_defns}
  \center{
    \fbox{
      \begin{tabular}{c|c}
        quantum mechanics & process calculus \\
        \hline
        scalar & $x := \quotep{P}$ \\
        state vector & $\state{P} := P$ \\
        dual & $\state{P}^{*} := \event{P^{\underline{\perp}}} := \quotep{P^{\underline{\perp}}}[-]$ \\
        matrix & $ \Sigma_{\alpha} \state{P_{\alpha}}x_{\alpha}\event{Q_{\alpha}}$ \\
        vector addition & $\state{P} + \state{Q} := \state{P | Q}$ \\
        tensor product & $\state{P} \otimes \state{Q} := \state{P \otimes Q}$ \\
        inner product & $\innerprod{P}{Q} := \quotep{\int P^{\underline{\perp}}[Q]}$ \\
      \end{tabular}
    }
  }
  \caption{QM - operational definitions}
\end{table}

where

\begin{mathpar}
  \prmatrix{P}{Q} := \fprmatrix{P}{\quotep{\pzero}}{Q}
  \and
  \fprmatrix{P}{x}{Q} := (\state{P},x,\event{Q})
  \and
  (\fprmatrix{P}{x}{Q})(\state{R}) := x \cdot \innerprod{Q}{R} \cdot \state{P}
  \and
  (\fprmatrix{P}{x}{Q})(\event{R}) := x \cdot \innerprod{R}{P} \cdot \event{Q}
\end{mathpar}

\paragraph{Discussion}
As promised: vectors (aka states) are represented as processes; duals
as contextual duals; inner product definition should be compared with
standard inner product definition for ....

\begin{remark}
  Assuming $\int (P^{\underline{\perp}}[P]) = 0$, the reader is
  invited to verify that $(\fprmatrix{P}{x}{P})(\state{P}) = x \cdot \state{P}$.
\end{remark}

\begin{remark}
  The reader is invited to verify that $\innerprod{P}{Q}$ could
  equally well have been written $\quotep{\int \stackrel{\vee}{x}}$
  where $x = \event{P^{\underline{\perp}}}(Q)$.

  One of the motivations for this remark is that there is another way
  to factor these operations. We could package up evaluation in the dual:

  \begin{mathpar}
    \state{P}^{*} := \event{\int P^{\underline{\perp}}} := \quotep{\int P^{\underline{\perp}}}[-]
  \end{mathpar}

  and then have inner product defined by
  
  \begin{mathpar}
    \innerprod{P}{Q} := \event{P}(Q)
  \end{mathpar}

  Hopefully, experience with the calculations will provide guidance on
  the best factoring.
\end{remark}

\begin{remark}
  Assuming $\int (P^{\underline{\perp}}[P]) = 0$, the reader is
  invited to verify that $\forall P,Q. (\prmatrix{0}{Q})(\state{0}) =
  \state{0}$ and dually $(\prmatrix{P}{0})(\event{0}) = \event{0}$.
\end{remark}

\begin{remark}
  i'm a little worried that i don't (yet) have proper support for
  complex conjugacy. But, the observation above may give us a
  clue. According to Abramsky, it must be the case that the scalars
  are iso to the homset of the identity for the tensor -- which the
  observation above characterizes. 

  For now, we will simply bookmark the notion with $\overline{x}$.
\end{remark}

\subsubsection{Adjointness}

We need to give a definition of $(\cdot)^{\dagger}$ for matrices. The
obvious candidate definition is
\begin{mathpar}
(\Sigma_{\alpha}\fprmatrix{P_{\alpha}}{x_{\alpha}}{Q_{\alpha}})^{\dagger}
= \Sigma_{\alpha}\fprmatrix{(Q_{\alpha}^{\underline{\perp}})^{*}}{\overline{x}_{\alpha}}{P_{\alpha}^{\underline{\perp}}} 
\end{mathpar}

But, $(Q_{\alpha}^{\underline{\perp}})^{*}$ requires a name along
which to communicate the process to achieve the context application.

\subsubsection{Basis for a basis}
If processes label states and ``addition'' of states (a.k.a. vector
addition) is interpreted as parallel composition, what corresponds to
notions of linear independence and basis? Here, we recall that Yoshida
has developed a set of \emph{combinators} for an asynchronous verison
of Milner's $\pi$-calculus. These are a finite set of processes such
any process can be expressed as parallel composition of these
combinators together with liberal uses of the new operator and
replication. We can simply give a translation of these into the
present calculus and have reasonable expectation that the property
carries over. That is, that the resultant set allows to express all
processes via parallel composition. Note, however, that there is no
new operator or replication in this calculus. As a result, we expect
that the corresponding set is actually infinite. That is, we expect
that the space is actually infinite dimensional.

\begin{remark}
  The attentive reader may be a bit concerned. Certainly, the
  collection $S$, $K$ and $I$ is a finite set of
  combinators. Shouldn't we expect to see a finite set of combinators
  for an effectively equivalent system? i am very sympathetic to this
  critique and feel it warrants full attention. On the other hand, i
  also have in mind the following analogy. The natural numbers, as a
  monoid under addition, has exactly $1$ generator, while the natural
  numbers, as a monoid under multiplication, has countably many
  generators (the primes). We observe that the application of the
  lambda calculus is much less resource sensitive than the parallel
  composition of the $\pi$-calculus. Could it be the case that we have
  an analogy of the form
  
  \begin{mathpar}
    m + n : MN :: m*n : M|N
  \end{mathpar}

  giving a similar blow up in the set of ``primes''?  This is such a
  wonderful thought that, even if it's not true, i think it's worth
  writing down.
\end{remark}
 

\documentclass[12pt]{llncs}
%\documentclass{jktr}

\usepackage[pdftex]{hyperref}                   
\usepackage {listings}
\usepackage {mathpartir}
\usepackage{bcprules}
%\usepackage{listings}
                       
\usepackage{graphicx} 
%\usepackage[margins=2.5cm,nohead,nofoot]{geometry}
%\usepackage{geometry}
\usepackage{amsfonts}
\usepackage{amstext}
\usepackage{latexsym}
\usepackage{amssymb}
\usepackage{color}


%\include{myPreamble}
\include{qm2pi.local} 

%\ifpdf
%\usepackage[pdftex]{graphicx}
%\else
%\usepackage{graphicx}
%\fi

 % \ifpdf
%  \usepackage{pdfsync}
%  \if


%\title{Brief Article}
%\author{David F. Snyder}
%\author{L.G. Meredith}

%\address{Dept. of Math., Texas State University--San Marcos, San Marcos, TX 78666}
       
\pagestyle{empty}


\begin{document}

\lstset{language=[Objective]Caml,frame=shadowbox}

\input{qm2pi.front}

% section front matter (end)

\input{qm2pi.intro} 
 
% section introduction (end)

% \input{qm2pi.knotations} 

% section notation (end)

\input{qm2pi.process.calculi} 

% section concurrent_process_calculi_and_spatial_logics_ (end)
    
%\input{qm2pi.knots2pi} 

%\input{qm2pi.trefoil} 

%\input{qm2pi.mainthm} 

% subsection basic_interpretation (end)

%\input{qm2pi.rho.presentation} 
\subsection{The syntax and semantics of the notation system}\label{sub:the_syntax_and_semantics_of_the_notation_system} % (fold)

We now summarize a technical presentation of the calculus that
embodies our theory of dynamics. The typical presentation of such a
calculus follows the style of giving generators and relations on
them. The grammar, below, describing term constructors, freely
generates the set of processes, $\Proc$. This set is then quotiented
by a relation known as structural congruence and it is over this set
that the notion of dynamics is expressed. This presentation is
essentially that of \cite{MeredithR05} with the addition of
polyadicity and summation. For readability we have relegated some of
the technical subtleties to an appendix.

\subsubsection{Process grammar}\label{subsub:process_grammar}

\begin{mathpar}
  \inferrule* [lab=synchronization] {} {{M} \bc \pzero \;|\; x?F \;|\; x!C }
  \and
  \inferrule* [lab=abstraction] {} {{F} \bc (x)P}
  \and
  \inferrule* [lab=concretion] {} {{C} \bc \langle Q \rangle}
  \and
  \inferrule* [lab=process] {} {{P,Q} \bc M \;| \;P|Q \;|\; @{x}}
  \and
  \inferrule* [lab=name] {} {{x} \bc \quotep{P}}
\end{mathpar} 

Note that $\vec{x}$ (resp. $\vec{P}$) denotes a vector of names
(resp. processes) of length $|\vec{x}|$ (resp. $|\vec{P}|$). We adopt
the following useful abbreviations.

\begin{mathpar}
   x?(\vec{y}).P := x.(\vec{y})P \and  x\clift{\vec{P}} := x.\clift{\vec{P}}
   \and x!(y) := \lift{x}{\dropn{y}}
   \and \Pi_{i=0}^{n-1}P_i := P_0 | \ldots | P_{n-1}
\end{mathpar}

\subsubsection{Structural congruence}

\paragraph{Free and bound names and alpha-equivalence.} At the
core of structural equivalence is alpha-equivalence which identifies
process that are the same up to a change of variable. Formally, we
recognize the distinction between free and bound names. The free names
of a process, $\freenames{P}$, may be calculated recursively as
follows:

\begin{mathpar}
\freenames{\pzero} := \emptyset
  \and \\
  \freenames{x?(y).P} := \{ x \} \cup (\freenames{P} \setminus \{ y \})
  \and 
  \freenames{x!\langle P \rangle} := \{ x \} \cup \{ P \} 
  \and \\
  \freenames{P|Q} := \freenames{P} \cup \freenames{Q}
  \and \\
  \freenames{@{x}} := \{ x \}
\end{mathpar}

$\pi$
$\quotep{\pi}$

$\freenames{-} : \pi \to \mathcal{P}(\quotep{\pi})$

\begin{eqnarray*}
  \freenames{\pzero} & := & \emptyset \\
  \freenames{x?(y).P} & := & \{ x \} \cup (\freenames{P} \setminus \{ y \}) \\
  \freenames{x!\langle P \rangle} & := & \{ x \} \cup \{ P \} \\
  \freenames{P|Q} & := & \freenames{P} \cup \freenames{Q} \\
  \freenames{\dropn{x}} & := & \{ x \}
\end{eqnarray*}

The bound names of a process, $\boundnames{P}$, are those names occurring in $P$
that are not free. For example, in $x?(y).0$, the name $x$ is free, while $y$ is bound.

\begin{mathpar}
  \inferrule* [lab=monoidal-laws] {} { P|Q \equiv Q|P \and P|0 \equiv P \and P|(Q|R) \equiv (P|Q)|R }
\end{mathpar}

\begin{mathpar}
  \inferrule* [lab=alpha-equivalence] {} { (x)P \equiv (y)P\{y/x\} \and y \not\in \freenames{P} }
\end{mathpar}

\begin{definition}
Then two processes, $P,Q$, are alpha-equivalent if $P = Q\{\vec{y}/\vec{x}\}$ for
some $\vec{x} \in \boundnames{Q},\vec{y} \in \boundnames{P}$, where $Q\{\vec{y}/\vec{x}\}$
denotes the capture-avoiding substitution of $\vec{y}$ for $\vec{x}$ in $Q$.
\end{definition}

\begin{definition}
  The {\em structural congruence} \cite{SangiorgiWalker} , $\equiv$,
  between processes is the least congruence containing
  alpha-equivalence, satisfying the abelian monoid laws
  (associativity, commutativity and $\pzero$ as identity) for parallel
  composition $|$ and for summation $+$.
\end{definition}

\subsection{Name equivalence}

We take name equivalence, written $\nameeq$, to be the smallest
equivalence relation generated by the following rules.

\begin{mathpar}
\inferrule*[lab=Quote-drop]
{ }
{ \quotep{@{x}} \nameeq x }

\inferrule*[lab=Struct-equiv]
{ P \scong Q }
{ \quotep{P} \nameeq \quotep{Q} }
\end{mathpar}

The astute reader will have noticed that the mutual recursion of names
and processes imposes a mutual recursion on alpha-equivalence and
structural equivalence via name-equivalence. Fortunately, all of this
works out pleasantly and we may calculate in the natural way, free of
concern. The reader interested in the details is referred to the
appendix \ref{appendix:rho_details}.

\subsection{Substitution}

We use $\Proc$ for the set of processes, $\QProc$ for the set of
names, and $\id{\{}\vec{y} / \vec{x} \id{\}}$ to denote partial maps,
$s : \QProc \rightarrow \QProc$. A map, $s$ lifts, uniquely, to a map
on process terms, $\widehat{s} : \Proc \rightarrow \Proc$ by the
following equations.

\begin{mathpar}
  (0) \psubstp{Q}{P} := 0 \\
  (R \juxtap S) \psubstp{Q}{P}
  :=    
  (R)\psubstp{Q}{P} \juxtap (S) \psubstp{Q}{P} \\
  (x?(y).R) \psubstp{Q}{P}    
  :=    
  (x)\substp{Q}{P} (z)\concat( (R \psubstn{z}{y}) \psubstp{Q}{P} ) \\
  (\lift{x}{R}) \psubstp{Q}{P}  
  :=
  \lift{(x)\substp{Q}{P}}{ R \psubstp{Q}{P} } \\
%   (\dropn{x})  \psubstp{Q}{P}       
%   := 
%   \left\{ 
%     \begin{array}{ccc} 
%       \dropn{\quotep{Q}} & & x \nameeq \quotep{P} \\
%       \dropn{x} & & otherwise \\
%     \end{array}
%   \right. 
  (\dropn{x})  \psubstp{Q}{P}       
  := 
  \left\{ 
    \begin{array}{ccc} 
      Q & & x \nameeq \quotep{P} \\
      \dropn{x} & & otherwise \\
    \end{array}
  \right.
\end{mathpar}
 

where

\begin{eqnarray}
  (x)\id{\{} \lpquote Q \rpquote / \lpquote P \rpquote \id{\}}            = 
  \left\{ 
    \begin{array}{ccc}
      \lpquote Q \rpquote & & x \nameeq \lpquote P \rpquote \\
      x & & otherwise \\
    \end{array}
  \right. \nonumber
\end{eqnarray}

and $z$ is chosen distinct from $\quotep{P}$, $\quotep{Q}$, the free
names in $Q$, and all the names in $R$. Our $\alpha$-equivalence will
be built in the standard way from this substitution.

\begin{remark}\label{rem:no_self_referential_names}
  One consequence of these definitions is that $\forall P. \quotep{P}
  \not\in \freenames{P}$.
\end{remark}

\subsection{ Dynamic quote: an example }

Anticipating something of what's to come, consider applying the
substitution, $\widehat{\id{\{}u / z \id{\}}}$, to the following pair
of processes, $\lift{w}{y!(z)}$ and $w[ \lpquote y!(z) \rpquote ]$.

\begin{eqnarray}
	\lift{w}{y!(z)}\widehat{\id{\{}u / z \id{\}}}
		& = &
		\lift{w}{y!(u)} \nonumber\\
	w[ \lpquote y!(z) \rpquote ] \widehat{ \id{\{}u / z \id{\}} }
		& = &
		w[ \lpquote y!(z) \rpquote ] \nonumber
\end{eqnarray}

Because the body of the process between quotes is impervious to
substitution, we get radically different answers. In fact, by
examining the first process in an input context,
e.g. $x?(z).\lift{w}{y!(z)}$, we see that the process under the lift
operator may be shaped by prefixed inputs binding a name inside it. In
this sense, the lift operator will be seen as a way to dynamically
construct processes before reifying them as names.

Finally equipped with these standard features we can present the
dynamics of the calculus.

\subsubsection{Operational semantics} 

Finally, we introduce the computational dynamics. What marks these
algebras as distinct from other more traditionally studied algebraic
structures, e.g. vector spaces or polynomial rings, is the manner in
which dynamics is captured. In traditional structures, dynamics is typically
expressed through morphisms between such structures, as in linear maps
between vector spaces or morphisms between rings. In algebras
associated with the semantics of computation, the dynamics is
expressed as part of the algebraic structure itself, through a
reduction reduction relation typically denoted by $\red$. Below, we
give a recursive presentation of this relation for the calculus used
in the encoding.

$\red \subseteq \pi \times \pi$
$\red : \pi \to \mathcal{P}(\pi)$

\begin{mathpar}
  \inferrule* [lab=Comm] { \textsf{match}( x_{src}, x_{trgt} ) } { x_{trgt}?(y)P \; | \; x_{src}!\langle {Q} \rangle \red P\{\quotep{Q}/y}\} }
  \and \\
  \inferrule* [lab=Par] {{P} \red {P}'} {{{P} | {Q}} \red {{P}' | {Q}}}
  \and
  \inferrule* [lab=Equiv]{{{P} \scong {P}'} \andalso {{P}' \red {Q}'} \andalso {{Q}' \scong {Q}}}{{P} \red {Q}}
\end{mathpar}

\begin{eqnarray*}
  match_{\equiv} (\quotep{P},\quotep{Q}) & := & P \equiv Q \\
  match_{\dagger}(\quotep{P},\quotep{Q}) & := & \forall R. P|Q \red^{*} R => R \red^{*} 0 \\
  match_{K}(\quotep{P},\quotep{Q}) & := & K \mbox{ for some context } K
\end{eqnarray*}

$u?(x)P | u!\langle Q \rangle \red P\{\quotep{Q}/x\}$

%We write $\wred$ for $\red^*$, and $P\red$ if $\exists Q $ such that $ P \red Q$.
We write $P\red$ if $\exists Q $ such that $ P \red Q$ and $P\not\red$, otherwise.

\section{Replication}

As mentioned before, it is known that replication (and hence
recursion) can be implemented in a higher-order process algebra
\cite{SangiorgiWalker}. As our first example of calculation with the
machinery thus far presented we give the construction explicitly in
the {\rhoc}.

\begin{eqnarray}
	D_{x} & := & \prefix{x}{y}{(\binpar{\outputp{x}{y}}{@{y}})} \nonumber\\
	\bangp_{x}{P} & := & \binpar{{x}!\langle{\binpar{D_{x}}{P}}\rangle}{D_{x}} \nonumber
\end{eqnarray}

\begin{eqnarray}
	\bangp_{x}{P} & & \nonumber\\
	=
	& {x}!\langle{(\prefix{x}{y}{(\outputp{x}{y} | @{y})) | P}}\rangle 
	      | \prefix{x}{y}{(\outputp{x}{y} | @{y})} & \nonumber\\
	\red
	& (\outputp{x}{y} | @{y})\substn{\quotep{(\prefix{x}{y}{(@{y} | \outputp{x}{y})) | P}}}{y} & \nonumber\\
	=
	& \outputp{x}{\quotep{(\prefix{x}{y}{(\outputp{x}{y} | @{y})) | P}}}
	  | {(\prefix{x}{y}{(\outputp{x}{y} | @{y})) | P}} & \nonumber\\
	\red
	& \ldots & \nonumber\\
	\red^*
	& P | P | \ldots & \nonumber
\end{eqnarray}

Of course, this encoding, as an implementation, runs away, unfolding
$\bangp{P}$ eagerly. A lazier and more implementable replication
operator, restricted to input-guarded processes, may be obtained as follows.

\begin{eqnarray}
\bangp{\prefix{u}{v}{P}} 
	:= 
	\binpar{\lift{x}{\prefix{u}{v}{(\binpar{D(x)}{P})}}}{D(x)} \nonumber
\end{eqnarray}

\begin{remark}
  Note that the lazier definition still does not deal with summation
  or mixed summation (i.e. sums over input and output). The reader is
  invited to construct definitions of replication that deal with these
  features. 

  Further, the definitions are parameterized in a name, $x$. Can you,
  gentle reader, make a definition that eliminates this parameter and
  guarantees no accidental interaction between the replication
  machinery and the process being replicated -- i.e. no accidental
  sharing of names used by the process to get its work done and the
  name(s) used by the replication to effect copying. This latter
  revision of the definition of replication is crucial to obtaining
  the expected identity $!!P \sim !P$.
\end{remark}

\begin{remark}\label{rem:paradoxical_combinator}
  The reader familiar with the lambda calculus will have noticed the
  similarity between $D$ and the paradoxical combinator.

  [Ed. note: the existence of this seems to suggest we have to be more
  restrictive on the set of processes and names we admit if we are to
  support no-cloning.]
\end{remark}

\subsubsection{Bisimulation}

The computational dynamics gives rise to another kind of equivalence,
the equivalence of computational behavior. As previously mentioned
this is typically captured \emph{via} some form of bisimulation.

% The notion we use in this paper is weak barbed bisimulation
% \cite{milner91polyadicpi}.

The notion we use in this paper is derived from weak barbed
bisimulation \cite{milner91polyadicpi}. 

\begin{definition}
An \emph{observation relation}, $\downarrow_{\mathcal N}$, over a set
of names, $\mathcal N$, is the smallest relation satisfying the rules
below.

\infrule[Out-barb]{y \in {\mathcal N}, \; x \nameeq y}
		  {\outputp{x}{v} \downarrow_{\mathcal N} x}
\infrule[Par-barb]{\mbox{$P\downarrow_{\mathcal N} x$ or $Q\downarrow_{\mathcal N} x$}}
		  {\binpar{P}{Q} \downarrow_{\mathcal N} x}

We write $P \Downarrow_{\mathcal N} x$ if there is $Q$ such that 
$P \wred Q$ and $Q \downarrow_{\mathcal N} x$.
\end{definition}

\begin{definition}
%\label{def.bbisim}
An  ${\mathcal N}$-\emph{barbed bisimulation} over a set of names, ${\mathcal N}$, is a symmetric binary relation 
${\mathcal S}_{\mathcal N}$ between agents such that $P\rel{S}_{\mathcal N}Q$ implies:
\begin{enumerate}
\item If $P \red P'$ then $Q \wred Q'$ and $P'\rel{S}_{\mathcal N} Q'$.
\item If $P\downarrow_{\mathcal N} x$, then $Q\Downarrow_{\mathcal N} x$.
\end{enumerate}
$P$ is ${\mathcal N}$-barbed bisimilar to $Q$, written
$P \wbbisim_{\mathcal N} Q$, if $P \rel{S}_{\mathcal N} Q$ for some ${\mathcal N}$-barbed bisimulation ${\mathcal S}_{\mathcal N}$.
\end{definition}

$\mathcal{R} \subseteq \pi \times \pi$

$P \mathcal{R} Q => \forall P'. P \red P' \Rightarrow \exists Q'. Q \red Q', P' \mathcal{R} Q'$

$P \vdash x \Rightarrow Q \vdash x$

\begin{mathpar}
  \inferrule*[lab=Out-barb]{x \nameeq y}{{y}!\langle{Q}\rangle \vdash x}
  \and
  \inferrule*[lab=Par-barb]{\mbox{$P\vdash x$ or $Q\vdash x$}}{\binpar{P}{Q} \vdash x}
\end{mathpar}

\subsubsection{Contexts}

One of the principle advantages of computational calculi like the
$\pi$-calculus is a well-defined notion of context,
contextual-equivalence and a correlation between
contextual-equivalence and notions of bisimulation. The notion of
context allows the decomposition of a process into (sub-)process and
its syntactic environment, its context. Thus, a context may be
thought of as a process with a ``hole'' (written $\Box$) in it. The
application of a context $M$ to a process $P$, written $M[P]$, is
tantamount to filling the hole in $M$ with $P$. In this paper we do
not need the full weight of this theory, but do make use of the notion
of context in the proof the main theorem. 

\begin{mathpar}
  \inferrule* [lab=summation] {} {{M_{M},M_{N}} \bc \Box \;|\; x.M_{A} \;|\; M_{M}+M_{N}}
  \and
  \inferrule* [lab=agent] {} {{M_{A}} \bc (\vec{x})M_{P} \;| \; \clift{P_0,\ldots,M_{P},\ldots,P_N}}
  \and \\
  \inferrule* [lab=process] {} {{M_{P}} \bc M_{N} \;| \;P|M_{P} }
\end{mathpar} 

\begin{mathpar}
  \inferrule* [lab=sychronization] {} {M_{N} \bc \Box \;|\; x?M_{F} \;|\; x!M_{C}}
  \and
  \inferrule* [lab=abstraction] {} {{M_{F}} \bc (x)M_{P} }
  \and
  \inferrule* [lab=concretion] {} {{M_{C}} \bc \langle M_{P} \rangle }
  \and \\
  \inferrule* [lab=process] {} {{M_{P}} \bc M_{N} \;| \;P|M_{P} }
\end{mathpar}

\begin{definition}[contextual application] Given a context $M$, and
  process $P$, we define the \emph{contextual application}, $M[P] :=
  M\{P/\Box\}$. That is, the contextual application of M to P is the
  substitution of $P$ for $\Box$ in $M$.
\end{definition}

$\meaningof{-} : L \to \mathcal{P}(\pi)$

\begin{mathpar}
  \inferrule* [lab=collection] {} {\meaningof{true} = \pi, \and \meaningof{~E} = \pi \setminus \meaningof{E}, \and \meaningof{E_{1} \& E_{2}} = \meaningof{E_{1}} \cap \meaningof{E_{2}}}
\end{mathpar}

\begin{mathpar}
  \inferrule* [lab=structure] {} {\meaningof{0} = \{ P \in \pi | P \equiv 0 \}, \and \\ \meaningof{E_1 | E_2} = \{ P \in \pi | P \equiv P_{1} | P_{2}, P_{1} \in \meaningof{E_{1}}, P_{2} \in \meaningof{E_2}\} }
\end{mathpar}

\begin{mathpar}
 \inferrule* [lab=behavior] {} {\meaningof{\langle a?b \rangle E} = \{ P \in \pi | P \equiv Q | u?(y)P', \\ \and \\\\ \and \\ \;\;\; u \in \meaningof{a}, \forall z.P'\{z/y\} \in \meaningof{E\{z/b\}}\}, \and \\ \meaningof{a!E} = \{ P \in \pi | P \equiv Q | x!\langle P' \rangle, x \in \meaningof{a} P' \in \meaningof{E}\} }
\end{mathpar}

\begin{mathpar}
 \inferrule* [lab=nominal] {} {\meaningof{\quotep{E}} = \{ \quotep{P} \in \quotep{\pi} | P \in \meaningof{E} \}, \and \meaningof{\quotep{P}} = \{ \quotep{Q} \in \quotep{\pi} | P \equiv Q \} \and \\ \meaningof{@\quotep{E}} = \{ P \in \pi | P \equiv @x, x \in \meaningof{E} \}}
\end{mathpar}

\begin{eqnarray*}
  \\
  \meaningof{-} : TS \to ST
\end{eqnarray*}

\begin{eqnarray*}
  \\
  L : TS \to ST
\end{eqnarray*}

\begin{eqnarray*}
  \\
  P \models E \iff P \in \meaningof{E}
\end{eqnarray*}

\begin{eqnarray*}
  P \approx_{L} Q \iff \forall E \in L. P \models E \iff Q \models E
\end{eqnarray*}

\begin{eqnarray*}
  P \approx_{K} Q
\end{eqnarray*}

\begin{eqnarray*}
  P \approx Q
\end{eqnarray*}

$\approx_{K} = \approx = \approx_{L}$

\subsubsection{Contextual duality}

Note that contexts extend the quotation operation to a family of
operations from processes to names. Given a context, $M$, we can
define a \emph{nominal context}, $\quotep{M}$ by $\quotep{M}[P] :=
\quotep{M[P]}$. To foreshadow what is to come we observe that these
operations enjoy a duality with processes very much like the duality
between vectors and maps from vectors to scalars.

Further, because the calculus is essentially higher-order, we have a
correspondence between contexts and processes. More specifically,
given a name $x$ and a context $M$ we can construct $M^{*}_{x}$ such
that 

\begin{mathpar}
  M^{*}_{x} | \lift{x}{P} \red M[P]
\end{mathpar}

namely,

\begin{mathpar}
  M^{*}_{x} := x?(u).M[\dropn{u}]
\end{mathpar}

The dependence of $M^{*}_{x}$ on a name makes it an abstraction, 

\begin{mathpar}
  M^{*} := (x)x?(u).M[\dropn{u}]
\end{mathpar}

\subsection{Additional notation}

It will sometimes be convenient to denote the process a name
quotes. We already have the notation $x = \quotep{P}$, but it will be
convenient to introduce an alternate notation, $\procn{x}$, when we
want to emphasize the connection to the use of the name. Note that, by
virtue of name equivalence, $\quotep{\procn{x}} \nameeq x$; so, the
notation is consistent with previous definitions.

Further, because names have structure it is possible to effect
substitutions on the basis of that structure. This means we need to
upgrade our notation for substitutions, which we accomplish by
adapting comprehension notation. Thus,

\begin{mathpar}
  P\{ y / x : x \in S \}
\end{mathpar}

is interpreted to mean the process derived from P by replacing (in a
capture-avoiding manner) each occurrence of $x$ in $S$ by $y$. For example,

\begin{mathpar}
  P\{ \quotep{\procn{x}|\procn{x}} / x : x \in \freenames{P} \}
\end{mathpar}

will replace each (occurrence) of a free name $x$ in $P$ by
$\quotep{\procn{x}|\procn{x}}$.

Also, we will avail ourselves of the notation $x^{L}$ and $x^{R}$ to
denote injections of a name into disjoint copies of the name
space. There are numerous ways to accomplish this. One example can be
found in \cite{MeredithR05}. This notation overloads to vectors of
names: $\vec{x}^{\pi} := (x_{i}^{\pi} \; : \; 0 \leq i < |\vec{x}| )$ where $\pi \in \{L,R\}$.

We also use $P^{\Box} := P|\Box$.

In \cite{MeredithR05} an interpretation of the new operator is
given. It turns out that there are several possible interpretations
all enjoying the requisite algebraic properties of the operator (see
\cite{milner91polyadicpi}). We will therefore make liberal use of
$(\nu\; \vec{x})P$.

% subsection the_syntax_and_semantics_of_the_notation_system (end)   

\input{qm2pi.qmops} 

\input{qm2pi.sterngerlach} 

\input{qm2pi.metric} 

% section concurrent_process_calculi (end)

%\input{qm2pi.proofsketch}

% section proof sketch (end)

%\input{qm2pi.slviaknots} 

% section spatial logic via knots (end)

\input{qm2pi.conclusion}

% section conclusion (end)

%\input{qm2pi.dtcodes} 

% section wiring algorithm (end)

\input{qm2pi.ack} 

% section acknowledgments (end)

\newpage


\bibliographystyle{plain}   
\bibliography{../../biblios/main.bib}

\input{qm2pi.rhodetails}

\end{document}

 

\documentclass[12pt]{llncs}
%\documentclass{jktr}

\usepackage[pdftex]{hyperref}                   
\usepackage {listings}
\usepackage {mathpartir}
\usepackage{bcprules}
%\usepackage{listings}
                       
\usepackage{graphicx} 
%\usepackage[margins=2.5cm,nohead,nofoot]{geometry}
%\usepackage{geometry}
\usepackage{amsfonts}
\usepackage{amstext}
\usepackage{latexsym}
\usepackage{amssymb}
\usepackage{color}


%\include{myPreamble}
\include{qm2pi.local} 

%\ifpdf
%\usepackage[pdftex]{graphicx}
%\else
%\usepackage{graphicx}
%\fi

 % \ifpdf
%  \usepackage{pdfsync}
%  \if


%\title{Brief Article}
%\author{David F. Snyder}
%\author{L.G. Meredith}

%\address{Dept. of Math., Texas State University--San Marcos, San Marcos, TX 78666}
       
\pagestyle{empty}


\begin{document}

\lstset{language=[Objective]Caml,frame=shadowbox}

\input{qm2pi.front}

% section front matter (end)

\input{qm2pi.intro} 
 
% section introduction (end)

% \input{qm2pi.knotations} 

% section notation (end)

\input{qm2pi.process.calculi} 

% section concurrent_process_calculi_and_spatial_logics_ (end)
    
%\input{qm2pi.knots2pi} 

%\input{qm2pi.trefoil} 

%\input{qm2pi.mainthm} 

% subsection basic_interpretation (end)

%\input{qm2pi.rho.presentation} 
\subsection{The syntax and semantics of the notation system}\label{sub:the_syntax_and_semantics_of_the_notation_system} % (fold)

We now summarize a technical presentation of the calculus that
embodies our theory of dynamics. The typical presentation of such a
calculus follows the style of giving generators and relations on
them. The grammar, below, describing term constructors, freely
generates the set of processes, $\Proc$. This set is then quotiented
by a relation known as structural congruence and it is over this set
that the notion of dynamics is expressed. This presentation is
essentially that of \cite{MeredithR05} with the addition of
polyadicity and summation. For readability we have relegated some of
the technical subtleties to an appendix.

\subsubsection{Process grammar}\label{subsub:process_grammar}

\begin{mathpar}
  \inferrule* [lab=synchronization] {} {{M} \bc \pzero \;|\; x?F \;|\; x!C }
  \and
  \inferrule* [lab=abstraction] {} {{F} \bc (x)P}
  \and
  \inferrule* [lab=concretion] {} {{C} \bc \langle Q \rangle}
  \and
  \inferrule* [lab=process] {} {{P,Q} \bc M \;| \;P|Q \;|\; @{x}}
  \and
  \inferrule* [lab=name] {} {{x} \bc \quotep{P}}
\end{mathpar} 

Note that $\vec{x}$ (resp. $\vec{P}$) denotes a vector of names
(resp. processes) of length $|\vec{x}|$ (resp. $|\vec{P}|$). We adopt
the following useful abbreviations.

\begin{mathpar}
   x?(\vec{y}).P := x.(\vec{y})P \and  x\clift{\vec{P}} := x.\clift{\vec{P}}
   \and x!(y) := \lift{x}{\dropn{y}}
   \and \Pi_{i=0}^{n-1}P_i := P_0 | \ldots | P_{n-1}
\end{mathpar}

\subsubsection{Structural congruence}

\paragraph{Free and bound names and alpha-equivalence.} At the
core of structural equivalence is alpha-equivalence which identifies
process that are the same up to a change of variable. Formally, we
recognize the distinction between free and bound names. The free names
of a process, $\freenames{P}$, may be calculated recursively as
follows:

\begin{mathpar}
\freenames{\pzero} := \emptyset
  \and \\
  \freenames{x?(y).P} := \{ x \} \cup (\freenames{P} \setminus \{ y \})
  \and 
  \freenames{x!\langle P \rangle} := \{ x \} \cup \{ P \} 
  \and \\
  \freenames{P|Q} := \freenames{P} \cup \freenames{Q}
  \and \\
  \freenames{@{x}} := \{ x \}
\end{mathpar}

$\pi$
$\quotep{\pi}$

$\freenames{-} : \pi \to \mathcal{P}(\quotep{\pi})$

\begin{eqnarray*}
  \freenames{\pzero} & := & \emptyset \\
  \freenames{x?(y).P} & := & \{ x \} \cup (\freenames{P} \setminus \{ y \}) \\
  \freenames{x!\langle P \rangle} & := & \{ x \} \cup \{ P \} \\
  \freenames{P|Q} & := & \freenames{P} \cup \freenames{Q} \\
  \freenames{\dropn{x}} & := & \{ x \}
\end{eqnarray*}

The bound names of a process, $\boundnames{P}$, are those names occurring in $P$
that are not free. For example, in $x?(y).0$, the name $x$ is free, while $y$ is bound.

\begin{mathpar}
  \inferrule* [lab=monoidal-laws] {} { P|Q \equiv Q|P \and P|0 \equiv P \and P|(Q|R) \equiv (P|Q)|R }
\end{mathpar}

\begin{mathpar}
  \inferrule* [lab=alpha-equivalence] {} { (x)P \equiv (y)P\{y/x\} \and y \not\in \freenames{P} }
\end{mathpar}

\begin{definition}
Then two processes, $P,Q$, are alpha-equivalent if $P = Q\{\vec{y}/\vec{x}\}$ for
some $\vec{x} \in \boundnames{Q},\vec{y} \in \boundnames{P}$, where $Q\{\vec{y}/\vec{x}\}$
denotes the capture-avoiding substitution of $\vec{y}$ for $\vec{x}$ in $Q$.
\end{definition}

\begin{definition}
  The {\em structural congruence} \cite{SangiorgiWalker} , $\equiv$,
  between processes is the least congruence containing
  alpha-equivalence, satisfying the abelian monoid laws
  (associativity, commutativity and $\pzero$ as identity) for parallel
  composition $|$ and for summation $+$.
\end{definition}

\subsection{Name equivalence}

We take name equivalence, written $\nameeq$, to be the smallest
equivalence relation generated by the following rules.

\begin{mathpar}
\inferrule*[lab=Quote-drop]
{ }
{ \quotep{@{x}} \nameeq x }

\inferrule*[lab=Struct-equiv]
{ P \scong Q }
{ \quotep{P} \nameeq \quotep{Q} }
\end{mathpar}

The astute reader will have noticed that the mutual recursion of names
and processes imposes a mutual recursion on alpha-equivalence and
structural equivalence via name-equivalence. Fortunately, all of this
works out pleasantly and we may calculate in the natural way, free of
concern. The reader interested in the details is referred to the
appendix \ref{appendix:rho_details}.

\subsection{Substitution}

We use $\Proc$ for the set of processes, $\QProc$ for the set of
names, and $\id{\{}\vec{y} / \vec{x} \id{\}}$ to denote partial maps,
$s : \QProc \rightarrow \QProc$. A map, $s$ lifts, uniquely, to a map
on process terms, $\widehat{s} : \Proc \rightarrow \Proc$ by the
following equations.

\begin{mathpar}
  (0) \psubstp{Q}{P} := 0 \\
  (R \juxtap S) \psubstp{Q}{P}
  :=    
  (R)\psubstp{Q}{P} \juxtap (S) \psubstp{Q}{P} \\
  (x?(y).R) \psubstp{Q}{P}    
  :=    
  (x)\substp{Q}{P} (z)\concat( (R \psubstn{z}{y}) \psubstp{Q}{P} ) \\
  (\lift{x}{R}) \psubstp{Q}{P}  
  :=
  \lift{(x)\substp{Q}{P}}{ R \psubstp{Q}{P} } \\
%   (\dropn{x})  \psubstp{Q}{P}       
%   := 
%   \left\{ 
%     \begin{array}{ccc} 
%       \dropn{\quotep{Q}} & & x \nameeq \quotep{P} \\
%       \dropn{x} & & otherwise \\
%     \end{array}
%   \right. 
  (\dropn{x})  \psubstp{Q}{P}       
  := 
  \left\{ 
    \begin{array}{ccc} 
      Q & & x \nameeq \quotep{P} \\
      \dropn{x} & & otherwise \\
    \end{array}
  \right.
\end{mathpar}
 

where

\begin{eqnarray}
  (x)\id{\{} \lpquote Q \rpquote / \lpquote P \rpquote \id{\}}            = 
  \left\{ 
    \begin{array}{ccc}
      \lpquote Q \rpquote & & x \nameeq \lpquote P \rpquote \\
      x & & otherwise \\
    \end{array}
  \right. \nonumber
\end{eqnarray}

and $z$ is chosen distinct from $\quotep{P}$, $\quotep{Q}$, the free
names in $Q$, and all the names in $R$. Our $\alpha$-equivalence will
be built in the standard way from this substitution.

\begin{remark}\label{rem:no_self_referential_names}
  One consequence of these definitions is that $\forall P. \quotep{P}
  \not\in \freenames{P}$.
\end{remark}

\subsection{ Dynamic quote: an example }

Anticipating something of what's to come, consider applying the
substitution, $\widehat{\id{\{}u / z \id{\}}}$, to the following pair
of processes, $\lift{w}{y!(z)}$ and $w[ \lpquote y!(z) \rpquote ]$.

\begin{eqnarray}
	\lift{w}{y!(z)}\widehat{\id{\{}u / z \id{\}}}
		& = &
		\lift{w}{y!(u)} \nonumber\\
	w[ \lpquote y!(z) \rpquote ] \widehat{ \id{\{}u / z \id{\}} }
		& = &
		w[ \lpquote y!(z) \rpquote ] \nonumber
\end{eqnarray}

Because the body of the process between quotes is impervious to
substitution, we get radically different answers. In fact, by
examining the first process in an input context,
e.g. $x?(z).\lift{w}{y!(z)}$, we see that the process under the lift
operator may be shaped by prefixed inputs binding a name inside it. In
this sense, the lift operator will be seen as a way to dynamically
construct processes before reifying them as names.

Finally equipped with these standard features we can present the
dynamics of the calculus.

\subsubsection{Operational semantics} 

Finally, we introduce the computational dynamics. What marks these
algebras as distinct from other more traditionally studied algebraic
structures, e.g. vector spaces or polynomial rings, is the manner in
which dynamics is captured. In traditional structures, dynamics is typically
expressed through morphisms between such structures, as in linear maps
between vector spaces or morphisms between rings. In algebras
associated with the semantics of computation, the dynamics is
expressed as part of the algebraic structure itself, through a
reduction reduction relation typically denoted by $\red$. Below, we
give a recursive presentation of this relation for the calculus used
in the encoding.

$\red \subseteq \pi \times \pi$
$\red : \pi \to \mathcal{P}(\pi)$

\begin{mathpar}
  \inferrule* [lab=Comm] { \textsf{match}( x_{src}, x_{trgt} ) } { x_{trgt}?(y)P \; | \; x_{src}!\langle {Q} \rangle \red P\{\quotep{Q}/y}\} }
  \and \\
  \inferrule* [lab=Par] {{P} \red {P}'} {{{P} | {Q}} \red {{P}' | {Q}}}
  \and
  \inferrule* [lab=Equiv]{{{P} \scong {P}'} \andalso {{P}' \red {Q}'} \andalso {{Q}' \scong {Q}}}{{P} \red {Q}}
\end{mathpar}

\begin{eqnarray*}
  match_{\equiv} (\quotep{P},\quotep{Q}) & := & P \equiv Q \\
  match_{\dagger}(\quotep{P},\quotep{Q}) & := & \forall R. P|Q \red^{*} R => R \red^{*} 0 \\
  match_{K}(\quotep{P},\quotep{Q}) & := & K \mbox{ for some context } K
\end{eqnarray*}

$u?(x)P | u!\langle Q \rangle \red P\{\quotep{Q}/x\}$

%We write $\wred$ for $\red^*$, and $P\red$ if $\exists Q $ such that $ P \red Q$.
We write $P\red$ if $\exists Q $ such that $ P \red Q$ and $P\not\red$, otherwise.

\section{Replication}

As mentioned before, it is known that replication (and hence
recursion) can be implemented in a higher-order process algebra
\cite{SangiorgiWalker}. As our first example of calculation with the
machinery thus far presented we give the construction explicitly in
the {\rhoc}.

\begin{eqnarray}
	D_{x} & := & \prefix{x}{y}{(\binpar{\outputp{x}{y}}{@{y}})} \nonumber\\
	\bangp_{x}{P} & := & \binpar{{x}!\langle{\binpar{D_{x}}{P}}\rangle}{D_{x}} \nonumber
\end{eqnarray}

\begin{eqnarray}
	\bangp_{x}{P} & & \nonumber\\
	=
	& {x}!\langle{(\prefix{x}{y}{(\outputp{x}{y} | @{y})) | P}}\rangle 
	      | \prefix{x}{y}{(\outputp{x}{y} | @{y})} & \nonumber\\
	\red
	& (\outputp{x}{y} | @{y})\substn{\quotep{(\prefix{x}{y}{(@{y} | \outputp{x}{y})) | P}}}{y} & \nonumber\\
	=
	& \outputp{x}{\quotep{(\prefix{x}{y}{(\outputp{x}{y} | @{y})) | P}}}
	  | {(\prefix{x}{y}{(\outputp{x}{y} | @{y})) | P}} & \nonumber\\
	\red
	& \ldots & \nonumber\\
	\red^*
	& P | P | \ldots & \nonumber
\end{eqnarray}

Of course, this encoding, as an implementation, runs away, unfolding
$\bangp{P}$ eagerly. A lazier and more implementable replication
operator, restricted to input-guarded processes, may be obtained as follows.

\begin{eqnarray}
\bangp{\prefix{u}{v}{P}} 
	:= 
	\binpar{\lift{x}{\prefix{u}{v}{(\binpar{D(x)}{P})}}}{D(x)} \nonumber
\end{eqnarray}

\begin{remark}
  Note that the lazier definition still does not deal with summation
  or mixed summation (i.e. sums over input and output). The reader is
  invited to construct definitions of replication that deal with these
  features. 

  Further, the definitions are parameterized in a name, $x$. Can you,
  gentle reader, make a definition that eliminates this parameter and
  guarantees no accidental interaction between the replication
  machinery and the process being replicated -- i.e. no accidental
  sharing of names used by the process to get its work done and the
  name(s) used by the replication to effect copying. This latter
  revision of the definition of replication is crucial to obtaining
  the expected identity $!!P \sim !P$.
\end{remark}

\begin{remark}\label{rem:paradoxical_combinator}
  The reader familiar with the lambda calculus will have noticed the
  similarity between $D$ and the paradoxical combinator.

  [Ed. note: the existence of this seems to suggest we have to be more
  restrictive on the set of processes and names we admit if we are to
  support no-cloning.]
\end{remark}

\subsubsection{Bisimulation}

The computational dynamics gives rise to another kind of equivalence,
the equivalence of computational behavior. As previously mentioned
this is typically captured \emph{via} some form of bisimulation.

% The notion we use in this paper is weak barbed bisimulation
% \cite{milner91polyadicpi}.

The notion we use in this paper is derived from weak barbed
bisimulation \cite{milner91polyadicpi}. 

\begin{definition}
An \emph{observation relation}, $\downarrow_{\mathcal N}$, over a set
of names, $\mathcal N$, is the smallest relation satisfying the rules
below.

\infrule[Out-barb]{y \in {\mathcal N}, \; x \nameeq y}
		  {\outputp{x}{v} \downarrow_{\mathcal N} x}
\infrule[Par-barb]{\mbox{$P\downarrow_{\mathcal N} x$ or $Q\downarrow_{\mathcal N} x$}}
		  {\binpar{P}{Q} \downarrow_{\mathcal N} x}

We write $P \Downarrow_{\mathcal N} x$ if there is $Q$ such that 
$P \wred Q$ and $Q \downarrow_{\mathcal N} x$.
\end{definition}

\begin{definition}
%\label{def.bbisim}
An  ${\mathcal N}$-\emph{barbed bisimulation} over a set of names, ${\mathcal N}$, is a symmetric binary relation 
${\mathcal S}_{\mathcal N}$ between agents such that $P\rel{S}_{\mathcal N}Q$ implies:
\begin{enumerate}
\item If $P \red P'$ then $Q \wred Q'$ and $P'\rel{S}_{\mathcal N} Q'$.
\item If $P\downarrow_{\mathcal N} x$, then $Q\Downarrow_{\mathcal N} x$.
\end{enumerate}
$P$ is ${\mathcal N}$-barbed bisimilar to $Q$, written
$P \wbbisim_{\mathcal N} Q$, if $P \rel{S}_{\mathcal N} Q$ for some ${\mathcal N}$-barbed bisimulation ${\mathcal S}_{\mathcal N}$.
\end{definition}

$\mathcal{R} \subseteq \pi \times \pi$

$P \mathcal{R} Q => \forall P'. P \red P' \Rightarrow \exists Q'. Q \red Q', P' \mathcal{R} Q'$

$P \vdash x \Rightarrow Q \vdash x$

\begin{mathpar}
  \inferrule*[lab=Out-barb]{x \nameeq y}{{y}!\langle{Q}\rangle \vdash x}
  \and
  \inferrule*[lab=Par-barb]{\mbox{$P\vdash x$ or $Q\vdash x$}}{\binpar{P}{Q} \vdash x}
\end{mathpar}

\subsubsection{Contexts}

One of the principle advantages of computational calculi like the
$\pi$-calculus is a well-defined notion of context,
contextual-equivalence and a correlation between
contextual-equivalence and notions of bisimulation. The notion of
context allows the decomposition of a process into (sub-)process and
its syntactic environment, its context. Thus, a context may be
thought of as a process with a ``hole'' (written $\Box$) in it. The
application of a context $M$ to a process $P$, written $M[P]$, is
tantamount to filling the hole in $M$ with $P$. In this paper we do
not need the full weight of this theory, but do make use of the notion
of context in the proof the main theorem. 

\begin{mathpar}
  \inferrule* [lab=summation] {} {{M_{M},M_{N}} \bc \Box \;|\; x.M_{A} \;|\; M_{M}+M_{N}}
  \and
  \inferrule* [lab=agent] {} {{M_{A}} \bc (\vec{x})M_{P} \;| \; \clift{P_0,\ldots,M_{P},\ldots,P_N}}
  \and \\
  \inferrule* [lab=process] {} {{M_{P}} \bc M_{N} \;| \;P|M_{P} }
\end{mathpar} 

\begin{mathpar}
  \inferrule* [lab=sychronization] {} {M_{N} \bc \Box \;|\; x?M_{F} \;|\; x!M_{C}}
  \and
  \inferrule* [lab=abstraction] {} {{M_{F}} \bc (x)M_{P} }
  \and
  \inferrule* [lab=concretion] {} {{M_{C}} \bc \langle M_{P} \rangle }
  \and \\
  \inferrule* [lab=process] {} {{M_{P}} \bc M_{N} \;| \;P|M_{P} }
\end{mathpar}

\begin{definition}[contextual application] Given a context $M$, and
  process $P$, we define the \emph{contextual application}, $M[P] :=
  M\{P/\Box\}$. That is, the contextual application of M to P is the
  substitution of $P$ for $\Box$ in $M$.
\end{definition}

$\meaningof{-} : L \to \mathcal{P}(\pi)$

\begin{mathpar}
  \inferrule* [lab=collection] {} {\meaningof{true} = \pi, \and \meaningof{~E} = \pi \setminus \meaningof{E}, \and \meaningof{E_{1} \& E_{2}} = \meaningof{E_{1}} \cap \meaningof{E_{2}}}
\end{mathpar}

\begin{mathpar}
  \inferrule* [lab=structure] {} {\meaningof{0} = \{ P \in \pi | P \equiv 0 \}, \and \\ \meaningof{E_1 | E_2} = \{ P \in \pi | P \equiv P_{1} | P_{2}, P_{1} \in \meaningof{E_{1}}, P_{2} \in \meaningof{E_2}\} }
\end{mathpar}

\begin{mathpar}
 \inferrule* [lab=behavior] {} {\meaningof{\langle a?b \rangle E} = \{ P \in \pi | P \equiv Q | u?(y)P', \\ \and \\\\ \and \\ \;\;\; u \in \meaningof{a}, \forall z.P'\{z/y\} \in \meaningof{E\{z/b\}}\}, \and \\ \meaningof{a!E} = \{ P \in \pi | P \equiv Q | x!\langle P' \rangle, x \in \meaningof{a} P' \in \meaningof{E}\} }
\end{mathpar}

\begin{mathpar}
 \inferrule* [lab=nominal] {} {\meaningof{\quotep{E}} = \{ \quotep{P} \in \quotep{\pi} | P \in \meaningof{E} \}, \and \meaningof{\quotep{P}} = \{ \quotep{Q} \in \quotep{\pi} | P \equiv Q \} \and \\ \meaningof{@\quotep{E}} = \{ P \in \pi | P \equiv @x, x \in \meaningof{E} \}}
\end{mathpar}

\begin{eqnarray*}
  \\
  \meaningof{-} : TS \to ST
\end{eqnarray*}

\begin{eqnarray*}
  \\
  L : TS \to ST
\end{eqnarray*}

\begin{eqnarray*}
  \\
  P \models E \iff P \in \meaningof{E}
\end{eqnarray*}

\begin{eqnarray*}
  P \approx_{L} Q \iff \forall E \in L. P \models E \iff Q \models E
\end{eqnarray*}

\begin{eqnarray*}
  P \approx_{K} Q
\end{eqnarray*}

\begin{eqnarray*}
  P \approx Q
\end{eqnarray*}

$\approx_{K} = \approx = \approx_{L}$

\subsubsection{Contextual duality}

Note that contexts extend the quotation operation to a family of
operations from processes to names. Given a context, $M$, we can
define a \emph{nominal context}, $\quotep{M}$ by $\quotep{M}[P] :=
\quotep{M[P]}$. To foreshadow what is to come we observe that these
operations enjoy a duality with processes very much like the duality
between vectors and maps from vectors to scalars.

Further, because the calculus is essentially higher-order, we have a
correspondence between contexts and processes. More specifically,
given a name $x$ and a context $M$ we can construct $M^{*}_{x}$ such
that 

\begin{mathpar}
  M^{*}_{x} | \lift{x}{P} \red M[P]
\end{mathpar}

namely,

\begin{mathpar}
  M^{*}_{x} := x?(u).M[\dropn{u}]
\end{mathpar}

The dependence of $M^{*}_{x}$ on a name makes it an abstraction, 

\begin{mathpar}
  M^{*} := (x)x?(u).M[\dropn{u}]
\end{mathpar}

\subsection{Additional notation}

It will sometimes be convenient to denote the process a name
quotes. We already have the notation $x = \quotep{P}$, but it will be
convenient to introduce an alternate notation, $\procn{x}$, when we
want to emphasize the connection to the use of the name. Note that, by
virtue of name equivalence, $\quotep{\procn{x}} \nameeq x$; so, the
notation is consistent with previous definitions.

Further, because names have structure it is possible to effect
substitutions on the basis of that structure. This means we need to
upgrade our notation for substitutions, which we accomplish by
adapting comprehension notation. Thus,

\begin{mathpar}
  P\{ y / x : x \in S \}
\end{mathpar}

is interpreted to mean the process derived from P by replacing (in a
capture-avoiding manner) each occurrence of $x$ in $S$ by $y$. For example,

\begin{mathpar}
  P\{ \quotep{\procn{x}|\procn{x}} / x : x \in \freenames{P} \}
\end{mathpar}

will replace each (occurrence) of a free name $x$ in $P$ by
$\quotep{\procn{x}|\procn{x}}$.

Also, we will avail ourselves of the notation $x^{L}$ and $x^{R}$ to
denote injections of a name into disjoint copies of the name
space. There are numerous ways to accomplish this. One example can be
found in \cite{MeredithR05}. This notation overloads to vectors of
names: $\vec{x}^{\pi} := (x_{i}^{\pi} \; : \; 0 \leq i < |\vec{x}| )$ where $\pi \in \{L,R\}$.

We also use $P^{\Box} := P|\Box$.

In \cite{MeredithR05} an interpretation of the new operator is
given. It turns out that there are several possible interpretations
all enjoying the requisite algebraic properties of the operator (see
\cite{milner91polyadicpi}). We will therefore make liberal use of
$(\nu\; \vec{x})P$.

% subsection the_syntax_and_semantics_of_the_notation_system (end)   

\input{qm2pi.qmops} 

\input{qm2pi.sterngerlach} 

\input{qm2pi.metric} 

% section concurrent_process_calculi (end)

%\input{qm2pi.proofsketch}

% section proof sketch (end)

%\input{qm2pi.slviaknots} 

% section spatial logic via knots (end)

\input{qm2pi.conclusion}

% section conclusion (end)

%\input{qm2pi.dtcodes} 

% section wiring algorithm (end)

\input{qm2pi.ack} 

% section acknowledgments (end)

\newpage


\bibliographystyle{plain}   
\bibliography{../../biblios/main.bib}

\input{qm2pi.rhodetails}

\end{document}

 

% section concurrent_process_calculi (end)

%\documentclass[12pt]{llncs}
%\documentclass{jktr}

\usepackage[pdftex]{hyperref}                   
\usepackage {listings}
\usepackage {mathpartir}
\usepackage{bcprules}
%\usepackage{listings}
                       
\usepackage{graphicx} 
%\usepackage[margins=2.5cm,nohead,nofoot]{geometry}
%\usepackage{geometry}
\usepackage{amsfonts}
\usepackage{amstext}
\usepackage{latexsym}
\usepackage{amssymb}
\usepackage{color}


%\include{myPreamble}
\include{qm2pi.local} 

%\ifpdf
%\usepackage[pdftex]{graphicx}
%\else
%\usepackage{graphicx}
%\fi

 % \ifpdf
%  \usepackage{pdfsync}
%  \if


%\title{Brief Article}
%\author{David F. Snyder}
%\author{L.G. Meredith}

%\address{Dept. of Math., Texas State University--San Marcos, San Marcos, TX 78666}
       
\pagestyle{empty}


\begin{document}

\lstset{language=[Objective]Caml,frame=shadowbox}

\input{qm2pi.front}

% section front matter (end)

\input{qm2pi.intro} 
 
% section introduction (end)

% \input{qm2pi.knotations} 

% section notation (end)

\input{qm2pi.process.calculi} 

% section concurrent_process_calculi_and_spatial_logics_ (end)
    
%\input{qm2pi.knots2pi} 

%\input{qm2pi.trefoil} 

%\input{qm2pi.mainthm} 

% subsection basic_interpretation (end)

%\input{qm2pi.rho.presentation} 
\subsection{The syntax and semantics of the notation system}\label{sub:the_syntax_and_semantics_of_the_notation_system} % (fold)

We now summarize a technical presentation of the calculus that
embodies our theory of dynamics. The typical presentation of such a
calculus follows the style of giving generators and relations on
them. The grammar, below, describing term constructors, freely
generates the set of processes, $\Proc$. This set is then quotiented
by a relation known as structural congruence and it is over this set
that the notion of dynamics is expressed. This presentation is
essentially that of \cite{MeredithR05} with the addition of
polyadicity and summation. For readability we have relegated some of
the technical subtleties to an appendix.

\subsubsection{Process grammar}\label{subsub:process_grammar}

\begin{mathpar}
  \inferrule* [lab=synchronization] {} {{M} \bc \pzero \;|\; x?F \;|\; x!C }
  \and
  \inferrule* [lab=abstraction] {} {{F} \bc (x)P}
  \and
  \inferrule* [lab=concretion] {} {{C} \bc \langle Q \rangle}
  \and
  \inferrule* [lab=process] {} {{P,Q} \bc M \;| \;P|Q \;|\; @{x}}
  \and
  \inferrule* [lab=name] {} {{x} \bc \quotep{P}}
\end{mathpar} 

Note that $\vec{x}$ (resp. $\vec{P}$) denotes a vector of names
(resp. processes) of length $|\vec{x}|$ (resp. $|\vec{P}|$). We adopt
the following useful abbreviations.

\begin{mathpar}
   x?(\vec{y}).P := x.(\vec{y})P \and  x\clift{\vec{P}} := x.\clift{\vec{P}}
   \and x!(y) := \lift{x}{\dropn{y}}
   \and \Pi_{i=0}^{n-1}P_i := P_0 | \ldots | P_{n-1}
\end{mathpar}

\subsubsection{Structural congruence}

\paragraph{Free and bound names and alpha-equivalence.} At the
core of structural equivalence is alpha-equivalence which identifies
process that are the same up to a change of variable. Formally, we
recognize the distinction between free and bound names. The free names
of a process, $\freenames{P}$, may be calculated recursively as
follows:

\begin{mathpar}
\freenames{\pzero} := \emptyset
  \and \\
  \freenames{x?(y).P} := \{ x \} \cup (\freenames{P} \setminus \{ y \})
  \and 
  \freenames{x!\langle P \rangle} := \{ x \} \cup \{ P \} 
  \and \\
  \freenames{P|Q} := \freenames{P} \cup \freenames{Q}
  \and \\
  \freenames{@{x}} := \{ x \}
\end{mathpar}

$\pi$
$\quotep{\pi}$

$\freenames{-} : \pi \to \mathcal{P}(\quotep{\pi})$

\begin{eqnarray*}
  \freenames{\pzero} & := & \emptyset \\
  \freenames{x?(y).P} & := & \{ x \} \cup (\freenames{P} \setminus \{ y \}) \\
  \freenames{x!\langle P \rangle} & := & \{ x \} \cup \{ P \} \\
  \freenames{P|Q} & := & \freenames{P} \cup \freenames{Q} \\
  \freenames{\dropn{x}} & := & \{ x \}
\end{eqnarray*}

The bound names of a process, $\boundnames{P}$, are those names occurring in $P$
that are not free. For example, in $x?(y).0$, the name $x$ is free, while $y$ is bound.

\begin{mathpar}
  \inferrule* [lab=monoidal-laws] {} { P|Q \equiv Q|P \and P|0 \equiv P \and P|(Q|R) \equiv (P|Q)|R }
\end{mathpar}

\begin{mathpar}
  \inferrule* [lab=alpha-equivalence] {} { (x)P \equiv (y)P\{y/x\} \and y \not\in \freenames{P} }
\end{mathpar}

\begin{definition}
Then two processes, $P,Q$, are alpha-equivalent if $P = Q\{\vec{y}/\vec{x}\}$ for
some $\vec{x} \in \boundnames{Q},\vec{y} \in \boundnames{P}$, where $Q\{\vec{y}/\vec{x}\}$
denotes the capture-avoiding substitution of $\vec{y}$ for $\vec{x}$ in $Q$.
\end{definition}

\begin{definition}
  The {\em structural congruence} \cite{SangiorgiWalker} , $\equiv$,
  between processes is the least congruence containing
  alpha-equivalence, satisfying the abelian monoid laws
  (associativity, commutativity and $\pzero$ as identity) for parallel
  composition $|$ and for summation $+$.
\end{definition}

\subsection{Name equivalence}

We take name equivalence, written $\nameeq$, to be the smallest
equivalence relation generated by the following rules.

\begin{mathpar}
\inferrule*[lab=Quote-drop]
{ }
{ \quotep{@{x}} \nameeq x }

\inferrule*[lab=Struct-equiv]
{ P \scong Q }
{ \quotep{P} \nameeq \quotep{Q} }
\end{mathpar}

The astute reader will have noticed that the mutual recursion of names
and processes imposes a mutual recursion on alpha-equivalence and
structural equivalence via name-equivalence. Fortunately, all of this
works out pleasantly and we may calculate in the natural way, free of
concern. The reader interested in the details is referred to the
appendix \ref{appendix:rho_details}.

\subsection{Substitution}

We use $\Proc$ for the set of processes, $\QProc$ for the set of
names, and $\id{\{}\vec{y} / \vec{x} \id{\}}$ to denote partial maps,
$s : \QProc \rightarrow \QProc$. A map, $s$ lifts, uniquely, to a map
on process terms, $\widehat{s} : \Proc \rightarrow \Proc$ by the
following equations.

\begin{mathpar}
  (0) \psubstp{Q}{P} := 0 \\
  (R \juxtap S) \psubstp{Q}{P}
  :=    
  (R)\psubstp{Q}{P} \juxtap (S) \psubstp{Q}{P} \\
  (x?(y).R) \psubstp{Q}{P}    
  :=    
  (x)\substp{Q}{P} (z)\concat( (R \psubstn{z}{y}) \psubstp{Q}{P} ) \\
  (\lift{x}{R}) \psubstp{Q}{P}  
  :=
  \lift{(x)\substp{Q}{P}}{ R \psubstp{Q}{P} } \\
%   (\dropn{x})  \psubstp{Q}{P}       
%   := 
%   \left\{ 
%     \begin{array}{ccc} 
%       \dropn{\quotep{Q}} & & x \nameeq \quotep{P} \\
%       \dropn{x} & & otherwise \\
%     \end{array}
%   \right. 
  (\dropn{x})  \psubstp{Q}{P}       
  := 
  \left\{ 
    \begin{array}{ccc} 
      Q & & x \nameeq \quotep{P} \\
      \dropn{x} & & otherwise \\
    \end{array}
  \right.
\end{mathpar}
 

where

\begin{eqnarray}
  (x)\id{\{} \lpquote Q \rpquote / \lpquote P \rpquote \id{\}}            = 
  \left\{ 
    \begin{array}{ccc}
      \lpquote Q \rpquote & & x \nameeq \lpquote P \rpquote \\
      x & & otherwise \\
    \end{array}
  \right. \nonumber
\end{eqnarray}

and $z$ is chosen distinct from $\quotep{P}$, $\quotep{Q}$, the free
names in $Q$, and all the names in $R$. Our $\alpha$-equivalence will
be built in the standard way from this substitution.

\begin{remark}\label{rem:no_self_referential_names}
  One consequence of these definitions is that $\forall P. \quotep{P}
  \not\in \freenames{P}$.
\end{remark}

\subsection{ Dynamic quote: an example }

Anticipating something of what's to come, consider applying the
substitution, $\widehat{\id{\{}u / z \id{\}}}$, to the following pair
of processes, $\lift{w}{y!(z)}$ and $w[ \lpquote y!(z) \rpquote ]$.

\begin{eqnarray}
	\lift{w}{y!(z)}\widehat{\id{\{}u / z \id{\}}}
		& = &
		\lift{w}{y!(u)} \nonumber\\
	w[ \lpquote y!(z) \rpquote ] \widehat{ \id{\{}u / z \id{\}} }
		& = &
		w[ \lpquote y!(z) \rpquote ] \nonumber
\end{eqnarray}

Because the body of the process between quotes is impervious to
substitution, we get radically different answers. In fact, by
examining the first process in an input context,
e.g. $x?(z).\lift{w}{y!(z)}$, we see that the process under the lift
operator may be shaped by prefixed inputs binding a name inside it. In
this sense, the lift operator will be seen as a way to dynamically
construct processes before reifying them as names.

Finally equipped with these standard features we can present the
dynamics of the calculus.

\subsubsection{Operational semantics} 

Finally, we introduce the computational dynamics. What marks these
algebras as distinct from other more traditionally studied algebraic
structures, e.g. vector spaces or polynomial rings, is the manner in
which dynamics is captured. In traditional structures, dynamics is typically
expressed through morphisms between such structures, as in linear maps
between vector spaces or morphisms between rings. In algebras
associated with the semantics of computation, the dynamics is
expressed as part of the algebraic structure itself, through a
reduction reduction relation typically denoted by $\red$. Below, we
give a recursive presentation of this relation for the calculus used
in the encoding.

$\red \subseteq \pi \times \pi$
$\red : \pi \to \mathcal{P}(\pi)$

\begin{mathpar}
  \inferrule* [lab=Comm] { \textsf{match}( x_{src}, x_{trgt} ) } { x_{trgt}?(y)P \; | \; x_{src}!\langle {Q} \rangle \red P\{\quotep{Q}/y}\} }
  \and \\
  \inferrule* [lab=Par] {{P} \red {P}'} {{{P} | {Q}} \red {{P}' | {Q}}}
  \and
  \inferrule* [lab=Equiv]{{{P} \scong {P}'} \andalso {{P}' \red {Q}'} \andalso {{Q}' \scong {Q}}}{{P} \red {Q}}
\end{mathpar}

\begin{eqnarray*}
  match_{\equiv} (\quotep{P},\quotep{Q}) & := & P \equiv Q \\
  match_{\dagger}(\quotep{P},\quotep{Q}) & := & \forall R. P|Q \red^{*} R => R \red^{*} 0 \\
  match_{K}(\quotep{P},\quotep{Q}) & := & K \mbox{ for some context } K
\end{eqnarray*}

$u?(x)P | u!\langle Q \rangle \red P\{\quotep{Q}/x\}$

%We write $\wred$ for $\red^*$, and $P\red$ if $\exists Q $ such that $ P \red Q$.
We write $P\red$ if $\exists Q $ such that $ P \red Q$ and $P\not\red$, otherwise.

\section{Replication}

As mentioned before, it is known that replication (and hence
recursion) can be implemented in a higher-order process algebra
\cite{SangiorgiWalker}. As our first example of calculation with the
machinery thus far presented we give the construction explicitly in
the {\rhoc}.

\begin{eqnarray}
	D_{x} & := & \prefix{x}{y}{(\binpar{\outputp{x}{y}}{@{y}})} \nonumber\\
	\bangp_{x}{P} & := & \binpar{{x}!\langle{\binpar{D_{x}}{P}}\rangle}{D_{x}} \nonumber
\end{eqnarray}

\begin{eqnarray}
	\bangp_{x}{P} & & \nonumber\\
	=
	& {x}!\langle{(\prefix{x}{y}{(\outputp{x}{y} | @{y})) | P}}\rangle 
	      | \prefix{x}{y}{(\outputp{x}{y} | @{y})} & \nonumber\\
	\red
	& (\outputp{x}{y} | @{y})\substn{\quotep{(\prefix{x}{y}{(@{y} | \outputp{x}{y})) | P}}}{y} & \nonumber\\
	=
	& \outputp{x}{\quotep{(\prefix{x}{y}{(\outputp{x}{y} | @{y})) | P}}}
	  | {(\prefix{x}{y}{(\outputp{x}{y} | @{y})) | P}} & \nonumber\\
	\red
	& \ldots & \nonumber\\
	\red^*
	& P | P | \ldots & \nonumber
\end{eqnarray}

Of course, this encoding, as an implementation, runs away, unfolding
$\bangp{P}$ eagerly. A lazier and more implementable replication
operator, restricted to input-guarded processes, may be obtained as follows.

\begin{eqnarray}
\bangp{\prefix{u}{v}{P}} 
	:= 
	\binpar{\lift{x}{\prefix{u}{v}{(\binpar{D(x)}{P})}}}{D(x)} \nonumber
\end{eqnarray}

\begin{remark}
  Note that the lazier definition still does not deal with summation
  or mixed summation (i.e. sums over input and output). The reader is
  invited to construct definitions of replication that deal with these
  features. 

  Further, the definitions are parameterized in a name, $x$. Can you,
  gentle reader, make a definition that eliminates this parameter and
  guarantees no accidental interaction between the replication
  machinery and the process being replicated -- i.e. no accidental
  sharing of names used by the process to get its work done and the
  name(s) used by the replication to effect copying. This latter
  revision of the definition of replication is crucial to obtaining
  the expected identity $!!P \sim !P$.
\end{remark}

\begin{remark}\label{rem:paradoxical_combinator}
  The reader familiar with the lambda calculus will have noticed the
  similarity between $D$ and the paradoxical combinator.

  [Ed. note: the existence of this seems to suggest we have to be more
  restrictive on the set of processes and names we admit if we are to
  support no-cloning.]
\end{remark}

\subsubsection{Bisimulation}

The computational dynamics gives rise to another kind of equivalence,
the equivalence of computational behavior. As previously mentioned
this is typically captured \emph{via} some form of bisimulation.

% The notion we use in this paper is weak barbed bisimulation
% \cite{milner91polyadicpi}.

The notion we use in this paper is derived from weak barbed
bisimulation \cite{milner91polyadicpi}. 

\begin{definition}
An \emph{observation relation}, $\downarrow_{\mathcal N}$, over a set
of names, $\mathcal N$, is the smallest relation satisfying the rules
below.

\infrule[Out-barb]{y \in {\mathcal N}, \; x \nameeq y}
		  {\outputp{x}{v} \downarrow_{\mathcal N} x}
\infrule[Par-barb]{\mbox{$P\downarrow_{\mathcal N} x$ or $Q\downarrow_{\mathcal N} x$}}
		  {\binpar{P}{Q} \downarrow_{\mathcal N} x}

We write $P \Downarrow_{\mathcal N} x$ if there is $Q$ such that 
$P \wred Q$ and $Q \downarrow_{\mathcal N} x$.
\end{definition}

\begin{definition}
%\label{def.bbisim}
An  ${\mathcal N}$-\emph{barbed bisimulation} over a set of names, ${\mathcal N}$, is a symmetric binary relation 
${\mathcal S}_{\mathcal N}$ between agents such that $P\rel{S}_{\mathcal N}Q$ implies:
\begin{enumerate}
\item If $P \red P'$ then $Q \wred Q'$ and $P'\rel{S}_{\mathcal N} Q'$.
\item If $P\downarrow_{\mathcal N} x$, then $Q\Downarrow_{\mathcal N} x$.
\end{enumerate}
$P$ is ${\mathcal N}$-barbed bisimilar to $Q$, written
$P \wbbisim_{\mathcal N} Q$, if $P \rel{S}_{\mathcal N} Q$ for some ${\mathcal N}$-barbed bisimulation ${\mathcal S}_{\mathcal N}$.
\end{definition}

$\mathcal{R} \subseteq \pi \times \pi$

$P \mathcal{R} Q => \forall P'. P \red P' \Rightarrow \exists Q'. Q \red Q', P' \mathcal{R} Q'$

$P \vdash x \Rightarrow Q \vdash x$

\begin{mathpar}
  \inferrule*[lab=Out-barb]{x \nameeq y}{{y}!\langle{Q}\rangle \vdash x}
  \and
  \inferrule*[lab=Par-barb]{\mbox{$P\vdash x$ or $Q\vdash x$}}{\binpar{P}{Q} \vdash x}
\end{mathpar}

\subsubsection{Contexts}

One of the principle advantages of computational calculi like the
$\pi$-calculus is a well-defined notion of context,
contextual-equivalence and a correlation between
contextual-equivalence and notions of bisimulation. The notion of
context allows the decomposition of a process into (sub-)process and
its syntactic environment, its context. Thus, a context may be
thought of as a process with a ``hole'' (written $\Box$) in it. The
application of a context $M$ to a process $P$, written $M[P]$, is
tantamount to filling the hole in $M$ with $P$. In this paper we do
not need the full weight of this theory, but do make use of the notion
of context in the proof the main theorem. 

\begin{mathpar}
  \inferrule* [lab=summation] {} {{M_{M},M_{N}} \bc \Box \;|\; x.M_{A} \;|\; M_{M}+M_{N}}
  \and
  \inferrule* [lab=agent] {} {{M_{A}} \bc (\vec{x})M_{P} \;| \; \clift{P_0,\ldots,M_{P},\ldots,P_N}}
  \and \\
  \inferrule* [lab=process] {} {{M_{P}} \bc M_{N} \;| \;P|M_{P} }
\end{mathpar} 

\begin{mathpar}
  \inferrule* [lab=sychronization] {} {M_{N} \bc \Box \;|\; x?M_{F} \;|\; x!M_{C}}
  \and
  \inferrule* [lab=abstraction] {} {{M_{F}} \bc (x)M_{P} }
  \and
  \inferrule* [lab=concretion] {} {{M_{C}} \bc \langle M_{P} \rangle }
  \and \\
  \inferrule* [lab=process] {} {{M_{P}} \bc M_{N} \;| \;P|M_{P} }
\end{mathpar}

\begin{definition}[contextual application] Given a context $M$, and
  process $P$, we define the \emph{contextual application}, $M[P] :=
  M\{P/\Box\}$. That is, the contextual application of M to P is the
  substitution of $P$ for $\Box$ in $M$.
\end{definition}

$\meaningof{-} : L \to \mathcal{P}(\pi)$

\begin{mathpar}
  \inferrule* [lab=collection] {} {\meaningof{true} = \pi, \and \meaningof{~E} = \pi \setminus \meaningof{E}, \and \meaningof{E_{1} \& E_{2}} = \meaningof{E_{1}} \cap \meaningof{E_{2}}}
\end{mathpar}

\begin{mathpar}
  \inferrule* [lab=structure] {} {\meaningof{0} = \{ P \in \pi | P \equiv 0 \}, \and \\ \meaningof{E_1 | E_2} = \{ P \in \pi | P \equiv P_{1} | P_{2}, P_{1} \in \meaningof{E_{1}}, P_{2} \in \meaningof{E_2}\} }
\end{mathpar}

\begin{mathpar}
 \inferrule* [lab=behavior] {} {\meaningof{\langle a?b \rangle E} = \{ P \in \pi | P \equiv Q | u?(y)P', \\ \and \\\\ \and \\ \;\;\; u \in \meaningof{a}, \forall z.P'\{z/y\} \in \meaningof{E\{z/b\}}\}, \and \\ \meaningof{a!E} = \{ P \in \pi | P \equiv Q | x!\langle P' \rangle, x \in \meaningof{a} P' \in \meaningof{E}\} }
\end{mathpar}

\begin{mathpar}
 \inferrule* [lab=nominal] {} {\meaningof{\quotep{E}} = \{ \quotep{P} \in \quotep{\pi} | P \in \meaningof{E} \}, \and \meaningof{\quotep{P}} = \{ \quotep{Q} \in \quotep{\pi} | P \equiv Q \} \and \\ \meaningof{@\quotep{E}} = \{ P \in \pi | P \equiv @x, x \in \meaningof{E} \}}
\end{mathpar}

\begin{eqnarray*}
  \\
  \meaningof{-} : TS \to ST
\end{eqnarray*}

\begin{eqnarray*}
  \\
  L : TS \to ST
\end{eqnarray*}

\begin{eqnarray*}
  \\
  P \models E \iff P \in \meaningof{E}
\end{eqnarray*}

\begin{eqnarray*}
  P \approx_{L} Q \iff \forall E \in L. P \models E \iff Q \models E
\end{eqnarray*}

\begin{eqnarray*}
  P \approx_{K} Q
\end{eqnarray*}

\begin{eqnarray*}
  P \approx Q
\end{eqnarray*}

$\approx_{K} = \approx = \approx_{L}$

\subsubsection{Contextual duality}

Note that contexts extend the quotation operation to a family of
operations from processes to names. Given a context, $M$, we can
define a \emph{nominal context}, $\quotep{M}$ by $\quotep{M}[P] :=
\quotep{M[P]}$. To foreshadow what is to come we observe that these
operations enjoy a duality with processes very much like the duality
between vectors and maps from vectors to scalars.

Further, because the calculus is essentially higher-order, we have a
correspondence between contexts and processes. More specifically,
given a name $x$ and a context $M$ we can construct $M^{*}_{x}$ such
that 

\begin{mathpar}
  M^{*}_{x} | \lift{x}{P} \red M[P]
\end{mathpar}

namely,

\begin{mathpar}
  M^{*}_{x} := x?(u).M[\dropn{u}]
\end{mathpar}

The dependence of $M^{*}_{x}$ on a name makes it an abstraction, 

\begin{mathpar}
  M^{*} := (x)x?(u).M[\dropn{u}]
\end{mathpar}

\subsection{Additional notation}

It will sometimes be convenient to denote the process a name
quotes. We already have the notation $x = \quotep{P}$, but it will be
convenient to introduce an alternate notation, $\procn{x}$, when we
want to emphasize the connection to the use of the name. Note that, by
virtue of name equivalence, $\quotep{\procn{x}} \nameeq x$; so, the
notation is consistent with previous definitions.

Further, because names have structure it is possible to effect
substitutions on the basis of that structure. This means we need to
upgrade our notation for substitutions, which we accomplish by
adapting comprehension notation. Thus,

\begin{mathpar}
  P\{ y / x : x \in S \}
\end{mathpar}

is interpreted to mean the process derived from P by replacing (in a
capture-avoiding manner) each occurrence of $x$ in $S$ by $y$. For example,

\begin{mathpar}
  P\{ \quotep{\procn{x}|\procn{x}} / x : x \in \freenames{P} \}
\end{mathpar}

will replace each (occurrence) of a free name $x$ in $P$ by
$\quotep{\procn{x}|\procn{x}}$.

Also, we will avail ourselves of the notation $x^{L}$ and $x^{R}$ to
denote injections of a name into disjoint copies of the name
space. There are numerous ways to accomplish this. One example can be
found in \cite{MeredithR05}. This notation overloads to vectors of
names: $\vec{x}^{\pi} := (x_{i}^{\pi} \; : \; 0 \leq i < |\vec{x}| )$ where $\pi \in \{L,R\}$.

We also use $P^{\Box} := P|\Box$.

In \cite{MeredithR05} an interpretation of the new operator is
given. It turns out that there are several possible interpretations
all enjoying the requisite algebraic properties of the operator (see
\cite{milner91polyadicpi}). We will therefore make liberal use of
$(\nu\; \vec{x})P$.

% subsection the_syntax_and_semantics_of_the_notation_system (end)   

\input{qm2pi.qmops} 

\input{qm2pi.sterngerlach} 

\input{qm2pi.metric} 

% section concurrent_process_calculi (end)

%\input{qm2pi.proofsketch}

% section proof sketch (end)

%\input{qm2pi.slviaknots} 

% section spatial logic via knots (end)

\input{qm2pi.conclusion}

% section conclusion (end)

%\input{qm2pi.dtcodes} 

% section wiring algorithm (end)

\input{qm2pi.ack} 

% section acknowledgments (end)

\newpage


\bibliographystyle{plain}   
\bibliography{../../biblios/main.bib}

\input{qm2pi.rhodetails}

\end{document}



% section proof sketch (end)

%\section{Unlikely characters: spatial logic for
  knots}\label{sub:characteristic_formulae} % (fold)

Associated to the mobile process calculi are a family of logics known
as the Hennessy-Milner logics. These logics typically enjoy a
semantics interpreting formulae as sets of processes that when
factored through the encoding outlined above allows an identification
of classes of knots with logical formulae. In the context of this
encoding the sub-family known as the spatial logics \cite{CairesC03}
\cite{CairesC04} \cite{Caires04} are of particular interest providing
several important features for expressing and reasoning about
properties (i.e. classes) of knots. We hint here at how this may be done.

%\begin{description}
%\item [structural connectives] 
\subsubsection{Structural connectives} The spatial logics enjoy
structural connectives corresponding, at the logical level, to the
parallel composition ($P | Q$) and new name ($(\nu \; x)P$)
connectives for processes. As illustrated in the examples below, these
connectives are extremely expressive given the shape of our encoding.
%\item [decideable satisfaction]

\subsubsection{Decideable satisfaction}
In \cite{Caires04} the satisfaction relation is shown to be decideable
for a rich class of processes. It further turns out that the image of
the our encoding is a proper subset of that class. This result
provides the basis for an algorithm by which to search for knots
enjoying a given property.
%\item [characteristic formulae]

\subsubsection{Characteristic formulae}
In the same paper \cite{Caires04} , Caires presents a means of calculating
characteristic formulae, selecting equivalence classes of processes
up to a pre--specified depth limit on the support set of names. Composed with our
encoding, this characteristic formula can be used to select
characteristic formulae for knots.
%\end{description}

\subsubsection{Spatial logic formulae}

The grammar below (segmented for comprehension) summarizes the syntax
of spatial logic formulae. We employ illustrative examples in the
sequel to provide an intuitive understanding of their meaning
referring the reader to \cite{Caires04} for a more detailed explication
of the semantics.

\begin{mathpar}
  \inferrule* [lab=boolean] {} {{A,B} \bc T \;|\; \neg A \;|\; A \wedge B \;|\; \eta = \eta'}
  \and
  \inferrule* [lab=spatial] {} {|\; \pzero \;|\; A | B \;|\; x \text{\textregistered} A \;|\; \forall x . A \;|\;  H x . A}
  \and
  \inferrule* [lab=behavioral] {} {|\; \alpha . A}
  \and 
  \inferrule* [lab=recursion] {} {|\; X(\vec{u}) \;|\; \mu X(\vec{u}) . A}
  \and
  \inferrule* [lab=action] {} {\alpha \bc \langle x?(\vec{y}) \rangle \;|\; \langle x!(\vec{y}) \rangle \;|\; \langle \tau \rangle}
  \and 
  \inferrule* [lab=name] {} {\eta \bc x \;|\; \tau}
\end{mathpar} 

% subsection characteristic_formulae (end)   	 

\subsection{Example formulae}\label{sub:example_formulae_} % (fold)

\subsubsection{Crossing as formula.}
% 
% \begin{align*}
%   \frac{d}{dx} \sin x &= \cos x 
%   & \frac{d}{dx} e^x &= e^x \\
%   \frac{d}{dx} \cos x &= - \sin x 
%   & \frac{d}{dx} \log x &= \frac{1}{x} \\
% \end{align*} 

\begin{align*}
 \mu C(x_{0},x_{1},y_{0},y_{1},u).&(\langle x_{0}?(z) \rangle(\langle u! \rangle\langle y_{1}!z \rangle C(x_{0},x_{1},y_{0},y_{1},u)) & \\
  & \wedge \langle y_{1}?(z) \rangle (\langle u! \rangle \langle x_{0}!z \rangle C(x_{0},x_{1},y_{0},y_{1},u)) & \\
  & \wedge \langle x_{1}?(z) \rangle (\langle u? \rangle \langle y_{0}!z \rangle C(x_{0},x_{1},y_{0},y_{1},u)) & \\
  & \wedge \langle y_{0}?(z) \rangle (\langle u? \rangle \langle x_{1}!z \rangle C(x_{0},x_{1},y_{0},y_{1},u))) &
\end{align*}

The lexicographical similarity between the shape of this formulae and
the shape of definition of the process representing a crossing reveals
the intuitive meaning of this formulae. It describes the capabilities
of a process that has the right to represent a crossing. For example
it picks out processes that may perform an input on the port $x_0$ in
its initial menu of capabilities. What differentiates the formula
from the process, however, is that the crossing process is the
smallest candidate to satisfy the formula. Infinitely many other
processes -- with internal behavior hidden behind this interface, so
to speak -- also satisfy this formula. Even this simple formula,
then, can be seen to open a new view onto knots, providing a
computational interpretation of \emph{virtual} knots.

Note that this formula is derived by hand. A similar formula can be
derived by employing Caires' calculation of characteristic formula
\cite{Caires04} to the process representing a crossing. In light of
this discussion, we let
$\meaningof{C}_{\phi}(x0,x1,y0,y1,u)$ denote a formula specifying the
dynamics we wish to capture of a crossing. To guarantee we preserve
the shape of the interface and minimal semantics we demand that
$\meaningof{C}_{\phi}(x0,x1,y0,y1,u) \Rightarrow
\textbf{C}(x0,x1,y0,y1,u)$ where $\textbf{C}(x0,x1,y0,y1,u)$ denotes
the formula above.
                            
\subsubsection{Crossing number constraints.}
The moral content of the context lemma (Lemma \ref{context}) is that the notion of
``locality'' in the Reidemeister moves is effectively captured by the
parallel composition operator of the process calculus. This intuition
extends through the logic. Given a formula,
$\meaningof{C}_{\phi}(x0,x1,y0,y1,u)$, we can use the structural
connectives to specify constraints on crossing numbers, such as at
least $n$ crossings, or exactly $n$ crossings.
\begin{mathpar}
  \inferrule* [lab=at-least-n] {} { K^{\geq n}_{\phi}(\vec{xs},\vec{ys}) := \Pi_{i=0}^{n-1} Hu . \meaningof{C}_{\phi}(xs_i,ys_i,u) | T }
  \and 
  \inferrule* [lab=exactly-n] {} { K^{= n}_{\phi}(\vec{xs},\vec{ys}) := \Pi_{i=0}^{n-1} Hu . \meaningof{C}_{\phi}(xs_i,ys_i,u) | \neg (\forall x_0,y_0,x_1,y_1,u . \meaningof{C}_{\phi}(x_0,y_0,x_1,y_1,u) | T) }
\end{mathpar}

To round out this section, recall that the encoding of an $n$-crossing
knot decomposes into a parallel composition of $n$ \emph{copies} of a
crossing process together with a wiring harness. To specify different
knot classes with the same crossing number amounts to specifying
logical constraints on the wiring harness. In the interest of space,
we defer examples to a forthcoming paper. Suffice it to say that both
the conditions ``alternating knot'' and ``contains the tangle
corresponding to 5/3'' are expressible. For example, it is possible to
calculate the characteristic formula of a process corresponding to the
tangle 5/3 and conjoin it into the classifying formula via the
composition connective of the logic.

Finally, we wish to observe that it is entirely within reason to
contemplate a more domain-specific version of spatial logic tailored
to the shape of processes in the image of the encoding. Such a
domain-specific logic would have a better claim to the title formal
language of knot properties.

% subsection example_formulae_ (end)

% section knots_as_processes (end) 

% section spatial logic via knots (end)

\section{Conclusions and future work}

\paragraph{Testing physical space}
You, gentle reader, may wonder why of all the theorems to be proved
given this set up we pick the one above. In some sense it's hardly
central to quantum mechanics. We see it as central in the sense that
it firmly establishes a notion of physical space arising from a notion
of the equivalence of behavior. Relating bisimulation to a metric is a
big step forward, but one is faced with interpreting the relationship
of that metric space to something more physical. Quantum mechanical
notions of ``physical'' space are still far from intuitive, but by
relating this idea of distance as testing to calculations that predict
physical circumstances we are making a not insignificant step forward
toward an understanding of the physical space we inhabit as
essentially dynamic.

\paragraph{Effectivity and simulation}
One of the observations we have yet to make is that the entire program
spelled out here is effective. We have built various interpreters for
the reflective calculus at work in this interpretation. In principle,
then, we can simulate quantum mechanics on a computer. The place where
the simulation may lose fidelity is the infinitely branching summation
for the annihilator.

In this connection i also want to point out that the evaluation style
calculation of the inner product puts the non-determinism of the
summation right at the heart of measurement. This suggests that
Milner's original reduction-based formulation of the dynamics of his
calculi in terms of sums was not just notationally suggestive of a
notion of measure-and-continue but captured some significant part of
the physics.

\paragraph{Quantum continuations}
In light of this last observation i want to point out that the
predominant account of quantum mechanics is missing a key aspect of a
truly compositional story of the physical situation. In a real lab,
when a measurement is made the observation can be made to feed into
another device that then makes another measurement conditioned on the
results of the first. This means that after the superposition was
collapsed the entire experimental set up remained in
superposition. While QM offers a means of writing this down it doesn't
quite line up well with the well-trodden formulation of computation
and continuation that we see so succinctly expressed in Milner's
calculi. This suggests that there might be advantages to this account
of dynamics waiting to be explored.

\paragraph{Quantum logic}
In this connection, we also note that by virtue of having the
Hennessy-Milner construction, we can pull the construction through the
interpretation of QM. This gives us a natural candidate for a quantum
logic that enjoys an extremely tight connection with it's domain of
interpretation, making the construction much less ad hoc (rather it is
the image of functor!).

\paragraph{Quantum probabiity}
i have questions about the basis of the interpretation of inner
product as probability amplitude. In particular, using which
axiomatization of probability theory does the notion of probability
amplitude earn the right to be so dubbed? In other words, where is the
proof that the operation for calculating a probability amplitude (and
then squaring) satisfies the axioms of what it means to calculate a
probability? Even if such a proof exists (i have yet to find it in the
literature), i wonder if it might not be possible to turn things on
their heads. Can we view the calculation of the probability amplitude
as an axiomatization of probability? If so, then the definition we
give for calculating probability amplitude may provide the basis for
an \emph{effective} theory of probability.

\paragraph{Quantum vs ``biological'' information}
Finally, i want to conclude with a more philosophical observation. At
a recent workshop in which QM was a predominant topic i noticed
something about quantum information. The speaker was giving a riveting
discussion of axiomatic QM and showing how properties of ``no
cloning'' and ``no deleting'' emerged as consequences of the
axiomatization. Theorems of this form are necessary to give us a sense
of confidence that our axioms characterize the physical theory. What
struck me, though, was that if quantum information is neither erasable
nor replicable it is markedly different from \emph{life}. Two of the
things we know about life is that

\begin{itemize}
  \item it ends;
  \item to gain some measure of persistence, to transcend it's
    finitude it is imminently copyable.
\end{itemize}

Both of these qualities are summarized succinctly in the aphorism: all
flesh is grass. For me these two kinds of ``information'' -- call them
quantum and biological -- are end points on a spectrum of strategies
for persistence. At one end, we have those curious entities that enjoy
uniqueness and permanence; at the other, we have those who in the face
of a certain end and an uncertain present make a go of passing
something on. To me one of the more remarkable aspects of the latter
strategy is that in the presence of noise (and certain features of
copying) we get a kind of dynamism, a chance for improvement against a
given persistent condition.

% subsection other_calculi_other_bisimulations_and_geometry_as_behavior (end)




% section conclusion (end)

%\documentclass[12pt]{llncs}
%\documentclass{jktr}

\usepackage[pdftex]{hyperref}                   
\usepackage {listings}
\usepackage {mathpartir}
\usepackage{bcprules}
%\usepackage{listings}
                       
\usepackage{graphicx} 
%\usepackage[margins=2.5cm,nohead,nofoot]{geometry}
%\usepackage{geometry}
\usepackage{amsfonts}
\usepackage{amstext}
\usepackage{latexsym}
\usepackage{amssymb}
\usepackage{color}


%\include{myPreamble}
\include{qm2pi.local} 

%\ifpdf
%\usepackage[pdftex]{graphicx}
%\else
%\usepackage{graphicx}
%\fi

 % \ifpdf
%  \usepackage{pdfsync}
%  \if


%\title{Brief Article}
%\author{David F. Snyder}
%\author{L.G. Meredith}

%\address{Dept. of Math., Texas State University--San Marcos, San Marcos, TX 78666}
       
\pagestyle{empty}


\begin{document}

\lstset{language=[Objective]Caml,frame=shadowbox}

\input{qm2pi.front}

% section front matter (end)

\input{qm2pi.intro} 
 
% section introduction (end)

% \input{qm2pi.knotations} 

% section notation (end)

\input{qm2pi.process.calculi} 

% section concurrent_process_calculi_and_spatial_logics_ (end)
    
%\input{qm2pi.knots2pi} 

%\input{qm2pi.trefoil} 

%\input{qm2pi.mainthm} 

% subsection basic_interpretation (end)

%\input{qm2pi.rho.presentation} 
\subsection{The syntax and semantics of the notation system}\label{sub:the_syntax_and_semantics_of_the_notation_system} % (fold)

We now summarize a technical presentation of the calculus that
embodies our theory of dynamics. The typical presentation of such a
calculus follows the style of giving generators and relations on
them. The grammar, below, describing term constructors, freely
generates the set of processes, $\Proc$. This set is then quotiented
by a relation known as structural congruence and it is over this set
that the notion of dynamics is expressed. This presentation is
essentially that of \cite{MeredithR05} with the addition of
polyadicity and summation. For readability we have relegated some of
the technical subtleties to an appendix.

\subsubsection{Process grammar}\label{subsub:process_grammar}

\begin{mathpar}
  \inferrule* [lab=synchronization] {} {{M} \bc \pzero \;|\; x?F \;|\; x!C }
  \and
  \inferrule* [lab=abstraction] {} {{F} \bc (x)P}
  \and
  \inferrule* [lab=concretion] {} {{C} \bc \langle Q \rangle}
  \and
  \inferrule* [lab=process] {} {{P,Q} \bc M \;| \;P|Q \;|\; @{x}}
  \and
  \inferrule* [lab=name] {} {{x} \bc \quotep{P}}
\end{mathpar} 

Note that $\vec{x}$ (resp. $\vec{P}$) denotes a vector of names
(resp. processes) of length $|\vec{x}|$ (resp. $|\vec{P}|$). We adopt
the following useful abbreviations.

\begin{mathpar}
   x?(\vec{y}).P := x.(\vec{y})P \and  x\clift{\vec{P}} := x.\clift{\vec{P}}
   \and x!(y) := \lift{x}{\dropn{y}}
   \and \Pi_{i=0}^{n-1}P_i := P_0 | \ldots | P_{n-1}
\end{mathpar}

\subsubsection{Structural congruence}

\paragraph{Free and bound names and alpha-equivalence.} At the
core of structural equivalence is alpha-equivalence which identifies
process that are the same up to a change of variable. Formally, we
recognize the distinction between free and bound names. The free names
of a process, $\freenames{P}$, may be calculated recursively as
follows:

\begin{mathpar}
\freenames{\pzero} := \emptyset
  \and \\
  \freenames{x?(y).P} := \{ x \} \cup (\freenames{P} \setminus \{ y \})
  \and 
  \freenames{x!\langle P \rangle} := \{ x \} \cup \{ P \} 
  \and \\
  \freenames{P|Q} := \freenames{P} \cup \freenames{Q}
  \and \\
  \freenames{@{x}} := \{ x \}
\end{mathpar}

$\pi$
$\quotep{\pi}$

$\freenames{-} : \pi \to \mathcal{P}(\quotep{\pi})$

\begin{eqnarray*}
  \freenames{\pzero} & := & \emptyset \\
  \freenames{x?(y).P} & := & \{ x \} \cup (\freenames{P} \setminus \{ y \}) \\
  \freenames{x!\langle P \rangle} & := & \{ x \} \cup \{ P \} \\
  \freenames{P|Q} & := & \freenames{P} \cup \freenames{Q} \\
  \freenames{\dropn{x}} & := & \{ x \}
\end{eqnarray*}

The bound names of a process, $\boundnames{P}$, are those names occurring in $P$
that are not free. For example, in $x?(y).0$, the name $x$ is free, while $y$ is bound.

\begin{mathpar}
  \inferrule* [lab=monoidal-laws] {} { P|Q \equiv Q|P \and P|0 \equiv P \and P|(Q|R) \equiv (P|Q)|R }
\end{mathpar}

\begin{mathpar}
  \inferrule* [lab=alpha-equivalence] {} { (x)P \equiv (y)P\{y/x\} \and y \not\in \freenames{P} }
\end{mathpar}

\begin{definition}
Then two processes, $P,Q$, are alpha-equivalent if $P = Q\{\vec{y}/\vec{x}\}$ for
some $\vec{x} \in \boundnames{Q},\vec{y} \in \boundnames{P}$, where $Q\{\vec{y}/\vec{x}\}$
denotes the capture-avoiding substitution of $\vec{y}$ for $\vec{x}$ in $Q$.
\end{definition}

\begin{definition}
  The {\em structural congruence} \cite{SangiorgiWalker} , $\equiv$,
  between processes is the least congruence containing
  alpha-equivalence, satisfying the abelian monoid laws
  (associativity, commutativity and $\pzero$ as identity) for parallel
  composition $|$ and for summation $+$.
\end{definition}

\subsection{Name equivalence}

We take name equivalence, written $\nameeq$, to be the smallest
equivalence relation generated by the following rules.

\begin{mathpar}
\inferrule*[lab=Quote-drop]
{ }
{ \quotep{@{x}} \nameeq x }

\inferrule*[lab=Struct-equiv]
{ P \scong Q }
{ \quotep{P} \nameeq \quotep{Q} }
\end{mathpar}

The astute reader will have noticed that the mutual recursion of names
and processes imposes a mutual recursion on alpha-equivalence and
structural equivalence via name-equivalence. Fortunately, all of this
works out pleasantly and we may calculate in the natural way, free of
concern. The reader interested in the details is referred to the
appendix \ref{appendix:rho_details}.

\subsection{Substitution}

We use $\Proc$ for the set of processes, $\QProc$ for the set of
names, and $\id{\{}\vec{y} / \vec{x} \id{\}}$ to denote partial maps,
$s : \QProc \rightarrow \QProc$. A map, $s$ lifts, uniquely, to a map
on process terms, $\widehat{s} : \Proc \rightarrow \Proc$ by the
following equations.

\begin{mathpar}
  (0) \psubstp{Q}{P} := 0 \\
  (R \juxtap S) \psubstp{Q}{P}
  :=    
  (R)\psubstp{Q}{P} \juxtap (S) \psubstp{Q}{P} \\
  (x?(y).R) \psubstp{Q}{P}    
  :=    
  (x)\substp{Q}{P} (z)\concat( (R \psubstn{z}{y}) \psubstp{Q}{P} ) \\
  (\lift{x}{R}) \psubstp{Q}{P}  
  :=
  \lift{(x)\substp{Q}{P}}{ R \psubstp{Q}{P} } \\
%   (\dropn{x})  \psubstp{Q}{P}       
%   := 
%   \left\{ 
%     \begin{array}{ccc} 
%       \dropn{\quotep{Q}} & & x \nameeq \quotep{P} \\
%       \dropn{x} & & otherwise \\
%     \end{array}
%   \right. 
  (\dropn{x})  \psubstp{Q}{P}       
  := 
  \left\{ 
    \begin{array}{ccc} 
      Q & & x \nameeq \quotep{P} \\
      \dropn{x} & & otherwise \\
    \end{array}
  \right.
\end{mathpar}
 

where

\begin{eqnarray}
  (x)\id{\{} \lpquote Q \rpquote / \lpquote P \rpquote \id{\}}            = 
  \left\{ 
    \begin{array}{ccc}
      \lpquote Q \rpquote & & x \nameeq \lpquote P \rpquote \\
      x & & otherwise \\
    \end{array}
  \right. \nonumber
\end{eqnarray}

and $z$ is chosen distinct from $\quotep{P}$, $\quotep{Q}$, the free
names in $Q$, and all the names in $R$. Our $\alpha$-equivalence will
be built in the standard way from this substitution.

\begin{remark}\label{rem:no_self_referential_names}
  One consequence of these definitions is that $\forall P. \quotep{P}
  \not\in \freenames{P}$.
\end{remark}

\subsection{ Dynamic quote: an example }

Anticipating something of what's to come, consider applying the
substitution, $\widehat{\id{\{}u / z \id{\}}}$, to the following pair
of processes, $\lift{w}{y!(z)}$ and $w[ \lpquote y!(z) \rpquote ]$.

\begin{eqnarray}
	\lift{w}{y!(z)}\widehat{\id{\{}u / z \id{\}}}
		& = &
		\lift{w}{y!(u)} \nonumber\\
	w[ \lpquote y!(z) \rpquote ] \widehat{ \id{\{}u / z \id{\}} }
		& = &
		w[ \lpquote y!(z) \rpquote ] \nonumber
\end{eqnarray}

Because the body of the process between quotes is impervious to
substitution, we get radically different answers. In fact, by
examining the first process in an input context,
e.g. $x?(z).\lift{w}{y!(z)}$, we see that the process under the lift
operator may be shaped by prefixed inputs binding a name inside it. In
this sense, the lift operator will be seen as a way to dynamically
construct processes before reifying them as names.

Finally equipped with these standard features we can present the
dynamics of the calculus.

\subsubsection{Operational semantics} 

Finally, we introduce the computational dynamics. What marks these
algebras as distinct from other more traditionally studied algebraic
structures, e.g. vector spaces or polynomial rings, is the manner in
which dynamics is captured. In traditional structures, dynamics is typically
expressed through morphisms between such structures, as in linear maps
between vector spaces or morphisms between rings. In algebras
associated with the semantics of computation, the dynamics is
expressed as part of the algebraic structure itself, through a
reduction reduction relation typically denoted by $\red$. Below, we
give a recursive presentation of this relation for the calculus used
in the encoding.

$\red \subseteq \pi \times \pi$
$\red : \pi \to \mathcal{P}(\pi)$

\begin{mathpar}
  \inferrule* [lab=Comm] { \textsf{match}( x_{src}, x_{trgt} ) } { x_{trgt}?(y)P \; | \; x_{src}!\langle {Q} \rangle \red P\{\quotep{Q}/y}\} }
  \and \\
  \inferrule* [lab=Par] {{P} \red {P}'} {{{P} | {Q}} \red {{P}' | {Q}}}
  \and
  \inferrule* [lab=Equiv]{{{P} \scong {P}'} \andalso {{P}' \red {Q}'} \andalso {{Q}' \scong {Q}}}{{P} \red {Q}}
\end{mathpar}

\begin{eqnarray*}
  match_{\equiv} (\quotep{P},\quotep{Q}) & := & P \equiv Q \\
  match_{\dagger}(\quotep{P},\quotep{Q}) & := & \forall R. P|Q \red^{*} R => R \red^{*} 0 \\
  match_{K}(\quotep{P},\quotep{Q}) & := & K \mbox{ for some context } K
\end{eqnarray*}

$u?(x)P | u!\langle Q \rangle \red P\{\quotep{Q}/x\}$

%We write $\wred$ for $\red^*$, and $P\red$ if $\exists Q $ such that $ P \red Q$.
We write $P\red$ if $\exists Q $ such that $ P \red Q$ and $P\not\red$, otherwise.

\section{Replication}

As mentioned before, it is known that replication (and hence
recursion) can be implemented in a higher-order process algebra
\cite{SangiorgiWalker}. As our first example of calculation with the
machinery thus far presented we give the construction explicitly in
the {\rhoc}.

\begin{eqnarray}
	D_{x} & := & \prefix{x}{y}{(\binpar{\outputp{x}{y}}{@{y}})} \nonumber\\
	\bangp_{x}{P} & := & \binpar{{x}!\langle{\binpar{D_{x}}{P}}\rangle}{D_{x}} \nonumber
\end{eqnarray}

\begin{eqnarray}
	\bangp_{x}{P} & & \nonumber\\
	=
	& {x}!\langle{(\prefix{x}{y}{(\outputp{x}{y} | @{y})) | P}}\rangle 
	      | \prefix{x}{y}{(\outputp{x}{y} | @{y})} & \nonumber\\
	\red
	& (\outputp{x}{y} | @{y})\substn{\quotep{(\prefix{x}{y}{(@{y} | \outputp{x}{y})) | P}}}{y} & \nonumber\\
	=
	& \outputp{x}{\quotep{(\prefix{x}{y}{(\outputp{x}{y} | @{y})) | P}}}
	  | {(\prefix{x}{y}{(\outputp{x}{y} | @{y})) | P}} & \nonumber\\
	\red
	& \ldots & \nonumber\\
	\red^*
	& P | P | \ldots & \nonumber
\end{eqnarray}

Of course, this encoding, as an implementation, runs away, unfolding
$\bangp{P}$ eagerly. A lazier and more implementable replication
operator, restricted to input-guarded processes, may be obtained as follows.

\begin{eqnarray}
\bangp{\prefix{u}{v}{P}} 
	:= 
	\binpar{\lift{x}{\prefix{u}{v}{(\binpar{D(x)}{P})}}}{D(x)} \nonumber
\end{eqnarray}

\begin{remark}
  Note that the lazier definition still does not deal with summation
  or mixed summation (i.e. sums over input and output). The reader is
  invited to construct definitions of replication that deal with these
  features. 

  Further, the definitions are parameterized in a name, $x$. Can you,
  gentle reader, make a definition that eliminates this parameter and
  guarantees no accidental interaction between the replication
  machinery and the process being replicated -- i.e. no accidental
  sharing of names used by the process to get its work done and the
  name(s) used by the replication to effect copying. This latter
  revision of the definition of replication is crucial to obtaining
  the expected identity $!!P \sim !P$.
\end{remark}

\begin{remark}\label{rem:paradoxical_combinator}
  The reader familiar with the lambda calculus will have noticed the
  similarity between $D$ and the paradoxical combinator.

  [Ed. note: the existence of this seems to suggest we have to be more
  restrictive on the set of processes and names we admit if we are to
  support no-cloning.]
\end{remark}

\subsubsection{Bisimulation}

The computational dynamics gives rise to another kind of equivalence,
the equivalence of computational behavior. As previously mentioned
this is typically captured \emph{via} some form of bisimulation.

% The notion we use in this paper is weak barbed bisimulation
% \cite{milner91polyadicpi}.

The notion we use in this paper is derived from weak barbed
bisimulation \cite{milner91polyadicpi}. 

\begin{definition}
An \emph{observation relation}, $\downarrow_{\mathcal N}$, over a set
of names, $\mathcal N$, is the smallest relation satisfying the rules
below.

\infrule[Out-barb]{y \in {\mathcal N}, \; x \nameeq y}
		  {\outputp{x}{v} \downarrow_{\mathcal N} x}
\infrule[Par-barb]{\mbox{$P\downarrow_{\mathcal N} x$ or $Q\downarrow_{\mathcal N} x$}}
		  {\binpar{P}{Q} \downarrow_{\mathcal N} x}

We write $P \Downarrow_{\mathcal N} x$ if there is $Q$ such that 
$P \wred Q$ and $Q \downarrow_{\mathcal N} x$.
\end{definition}

\begin{definition}
%\label{def.bbisim}
An  ${\mathcal N}$-\emph{barbed bisimulation} over a set of names, ${\mathcal N}$, is a symmetric binary relation 
${\mathcal S}_{\mathcal N}$ between agents such that $P\rel{S}_{\mathcal N}Q$ implies:
\begin{enumerate}
\item If $P \red P'$ then $Q \wred Q'$ and $P'\rel{S}_{\mathcal N} Q'$.
\item If $P\downarrow_{\mathcal N} x$, then $Q\Downarrow_{\mathcal N} x$.
\end{enumerate}
$P$ is ${\mathcal N}$-barbed bisimilar to $Q$, written
$P \wbbisim_{\mathcal N} Q$, if $P \rel{S}_{\mathcal N} Q$ for some ${\mathcal N}$-barbed bisimulation ${\mathcal S}_{\mathcal N}$.
\end{definition}

$\mathcal{R} \subseteq \pi \times \pi$

$P \mathcal{R} Q => \forall P'. P \red P' \Rightarrow \exists Q'. Q \red Q', P' \mathcal{R} Q'$

$P \vdash x \Rightarrow Q \vdash x$

\begin{mathpar}
  \inferrule*[lab=Out-barb]{x \nameeq y}{{y}!\langle{Q}\rangle \vdash x}
  \and
  \inferrule*[lab=Par-barb]{\mbox{$P\vdash x$ or $Q\vdash x$}}{\binpar{P}{Q} \vdash x}
\end{mathpar}

\subsubsection{Contexts}

One of the principle advantages of computational calculi like the
$\pi$-calculus is a well-defined notion of context,
contextual-equivalence and a correlation between
contextual-equivalence and notions of bisimulation. The notion of
context allows the decomposition of a process into (sub-)process and
its syntactic environment, its context. Thus, a context may be
thought of as a process with a ``hole'' (written $\Box$) in it. The
application of a context $M$ to a process $P$, written $M[P]$, is
tantamount to filling the hole in $M$ with $P$. In this paper we do
not need the full weight of this theory, but do make use of the notion
of context in the proof the main theorem. 

\begin{mathpar}
  \inferrule* [lab=summation] {} {{M_{M},M_{N}} \bc \Box \;|\; x.M_{A} \;|\; M_{M}+M_{N}}
  \and
  \inferrule* [lab=agent] {} {{M_{A}} \bc (\vec{x})M_{P} \;| \; \clift{P_0,\ldots,M_{P},\ldots,P_N}}
  \and \\
  \inferrule* [lab=process] {} {{M_{P}} \bc M_{N} \;| \;P|M_{P} }
\end{mathpar} 

\begin{mathpar}
  \inferrule* [lab=sychronization] {} {M_{N} \bc \Box \;|\; x?M_{F} \;|\; x!M_{C}}
  \and
  \inferrule* [lab=abstraction] {} {{M_{F}} \bc (x)M_{P} }
  \and
  \inferrule* [lab=concretion] {} {{M_{C}} \bc \langle M_{P} \rangle }
  \and \\
  \inferrule* [lab=process] {} {{M_{P}} \bc M_{N} \;| \;P|M_{P} }
\end{mathpar}

\begin{definition}[contextual application] Given a context $M$, and
  process $P$, we define the \emph{contextual application}, $M[P] :=
  M\{P/\Box\}$. That is, the contextual application of M to P is the
  substitution of $P$ for $\Box$ in $M$.
\end{definition}

$\meaningof{-} : L \to \mathcal{P}(\pi)$

\begin{mathpar}
  \inferrule* [lab=collection] {} {\meaningof{true} = \pi, \and \meaningof{~E} = \pi \setminus \meaningof{E}, \and \meaningof{E_{1} \& E_{2}} = \meaningof{E_{1}} \cap \meaningof{E_{2}}}
\end{mathpar}

\begin{mathpar}
  \inferrule* [lab=structure] {} {\meaningof{0} = \{ P \in \pi | P \equiv 0 \}, \and \\ \meaningof{E_1 | E_2} = \{ P \in \pi | P \equiv P_{1} | P_{2}, P_{1} \in \meaningof{E_{1}}, P_{2} \in \meaningof{E_2}\} }
\end{mathpar}

\begin{mathpar}
 \inferrule* [lab=behavior] {} {\meaningof{\langle a?b \rangle E} = \{ P \in \pi | P \equiv Q | u?(y)P', \\ \and \\\\ \and \\ \;\;\; u \in \meaningof{a}, \forall z.P'\{z/y\} \in \meaningof{E\{z/b\}}\}, \and \\ \meaningof{a!E} = \{ P \in \pi | P \equiv Q | x!\langle P' \rangle, x \in \meaningof{a} P' \in \meaningof{E}\} }
\end{mathpar}

\begin{mathpar}
 \inferrule* [lab=nominal] {} {\meaningof{\quotep{E}} = \{ \quotep{P} \in \quotep{\pi} | P \in \meaningof{E} \}, \and \meaningof{\quotep{P}} = \{ \quotep{Q} \in \quotep{\pi} | P \equiv Q \} \and \\ \meaningof{@\quotep{E}} = \{ P \in \pi | P \equiv @x, x \in \meaningof{E} \}}
\end{mathpar}

\begin{eqnarray*}
  \\
  \meaningof{-} : TS \to ST
\end{eqnarray*}

\begin{eqnarray*}
  \\
  L : TS \to ST
\end{eqnarray*}

\begin{eqnarray*}
  \\
  P \models E \iff P \in \meaningof{E}
\end{eqnarray*}

\begin{eqnarray*}
  P \approx_{L} Q \iff \forall E \in L. P \models E \iff Q \models E
\end{eqnarray*}

\begin{eqnarray*}
  P \approx_{K} Q
\end{eqnarray*}

\begin{eqnarray*}
  P \approx Q
\end{eqnarray*}

$\approx_{K} = \approx = \approx_{L}$

\subsubsection{Contextual duality}

Note that contexts extend the quotation operation to a family of
operations from processes to names. Given a context, $M$, we can
define a \emph{nominal context}, $\quotep{M}$ by $\quotep{M}[P] :=
\quotep{M[P]}$. To foreshadow what is to come we observe that these
operations enjoy a duality with processes very much like the duality
between vectors and maps from vectors to scalars.

Further, because the calculus is essentially higher-order, we have a
correspondence between contexts and processes. More specifically,
given a name $x$ and a context $M$ we can construct $M^{*}_{x}$ such
that 

\begin{mathpar}
  M^{*}_{x} | \lift{x}{P} \red M[P]
\end{mathpar}

namely,

\begin{mathpar}
  M^{*}_{x} := x?(u).M[\dropn{u}]
\end{mathpar}

The dependence of $M^{*}_{x}$ on a name makes it an abstraction, 

\begin{mathpar}
  M^{*} := (x)x?(u).M[\dropn{u}]
\end{mathpar}

\subsection{Additional notation}

It will sometimes be convenient to denote the process a name
quotes. We already have the notation $x = \quotep{P}$, but it will be
convenient to introduce an alternate notation, $\procn{x}$, when we
want to emphasize the connection to the use of the name. Note that, by
virtue of name equivalence, $\quotep{\procn{x}} \nameeq x$; so, the
notation is consistent with previous definitions.

Further, because names have structure it is possible to effect
substitutions on the basis of that structure. This means we need to
upgrade our notation for substitutions, which we accomplish by
adapting comprehension notation. Thus,

\begin{mathpar}
  P\{ y / x : x \in S \}
\end{mathpar}

is interpreted to mean the process derived from P by replacing (in a
capture-avoiding manner) each occurrence of $x$ in $S$ by $y$. For example,

\begin{mathpar}
  P\{ \quotep{\procn{x}|\procn{x}} / x : x \in \freenames{P} \}
\end{mathpar}

will replace each (occurrence) of a free name $x$ in $P$ by
$\quotep{\procn{x}|\procn{x}}$.

Also, we will avail ourselves of the notation $x^{L}$ and $x^{R}$ to
denote injections of a name into disjoint copies of the name
space. There are numerous ways to accomplish this. One example can be
found in \cite{MeredithR05}. This notation overloads to vectors of
names: $\vec{x}^{\pi} := (x_{i}^{\pi} \; : \; 0 \leq i < |\vec{x}| )$ where $\pi \in \{L,R\}$.

We also use $P^{\Box} := P|\Box$.

In \cite{MeredithR05} an interpretation of the new operator is
given. It turns out that there are several possible interpretations
all enjoying the requisite algebraic properties of the operator (see
\cite{milner91polyadicpi}). We will therefore make liberal use of
$(\nu\; \vec{x})P$.

% subsection the_syntax_and_semantics_of_the_notation_system (end)   

\input{qm2pi.qmops} 

\input{qm2pi.sterngerlach} 

\input{qm2pi.metric} 

% section concurrent_process_calculi (end)

%\input{qm2pi.proofsketch}

% section proof sketch (end)

%\input{qm2pi.slviaknots} 

% section spatial logic via knots (end)

\input{qm2pi.conclusion}

% section conclusion (end)

%\input{qm2pi.dtcodes} 

% section wiring algorithm (end)

\input{qm2pi.ack} 

% section acknowledgments (end)

\newpage


\bibliographystyle{plain}   
\bibliography{../../biblios/main.bib}

\input{qm2pi.rhodetails}

\end{document}

 

% section wiring algorithm (end)

\documentclass[12pt]{llncs}
%\documentclass{jktr}

\usepackage[pdftex]{hyperref}                   
\usepackage {listings}
\usepackage {mathpartir}
\usepackage{bcprules}
%\usepackage{listings}
                       
\usepackage{graphicx} 
%\usepackage[margins=2.5cm,nohead,nofoot]{geometry}
%\usepackage{geometry}
\usepackage{amsfonts}
\usepackage{amstext}
\usepackage{latexsym}
\usepackage{amssymb}
\usepackage{color}


%\include{myPreamble}
\include{qm2pi.local} 

%\ifpdf
%\usepackage[pdftex]{graphicx}
%\else
%\usepackage{graphicx}
%\fi

 % \ifpdf
%  \usepackage{pdfsync}
%  \if


%\title{Brief Article}
%\author{David F. Snyder}
%\author{L.G. Meredith}

%\address{Dept. of Math., Texas State University--San Marcos, San Marcos, TX 78666}
       
\pagestyle{empty}


\begin{document}

\lstset{language=[Objective]Caml,frame=shadowbox}

\input{qm2pi.front}

% section front matter (end)

\input{qm2pi.intro} 
 
% section introduction (end)

% \input{qm2pi.knotations} 

% section notation (end)

\input{qm2pi.process.calculi} 

% section concurrent_process_calculi_and_spatial_logics_ (end)
    
%\input{qm2pi.knots2pi} 

%\input{qm2pi.trefoil} 

%\input{qm2pi.mainthm} 

% subsection basic_interpretation (end)

%\input{qm2pi.rho.presentation} 
\subsection{The syntax and semantics of the notation system}\label{sub:the_syntax_and_semantics_of_the_notation_system} % (fold)

We now summarize a technical presentation of the calculus that
embodies our theory of dynamics. The typical presentation of such a
calculus follows the style of giving generators and relations on
them. The grammar, below, describing term constructors, freely
generates the set of processes, $\Proc$. This set is then quotiented
by a relation known as structural congruence and it is over this set
that the notion of dynamics is expressed. This presentation is
essentially that of \cite{MeredithR05} with the addition of
polyadicity and summation. For readability we have relegated some of
the technical subtleties to an appendix.

\subsubsection{Process grammar}\label{subsub:process_grammar}

\begin{mathpar}
  \inferrule* [lab=synchronization] {} {{M} \bc \pzero \;|\; x?F \;|\; x!C }
  \and
  \inferrule* [lab=abstraction] {} {{F} \bc (x)P}
  \and
  \inferrule* [lab=concretion] {} {{C} \bc \langle Q \rangle}
  \and
  \inferrule* [lab=process] {} {{P,Q} \bc M \;| \;P|Q \;|\; @{x}}
  \and
  \inferrule* [lab=name] {} {{x} \bc \quotep{P}}
\end{mathpar} 

Note that $\vec{x}$ (resp. $\vec{P}$) denotes a vector of names
(resp. processes) of length $|\vec{x}|$ (resp. $|\vec{P}|$). We adopt
the following useful abbreviations.

\begin{mathpar}
   x?(\vec{y}).P := x.(\vec{y})P \and  x\clift{\vec{P}} := x.\clift{\vec{P}}
   \and x!(y) := \lift{x}{\dropn{y}}
   \and \Pi_{i=0}^{n-1}P_i := P_0 | \ldots | P_{n-1}
\end{mathpar}

\subsubsection{Structural congruence}

\paragraph{Free and bound names and alpha-equivalence.} At the
core of structural equivalence is alpha-equivalence which identifies
process that are the same up to a change of variable. Formally, we
recognize the distinction between free and bound names. The free names
of a process, $\freenames{P}$, may be calculated recursively as
follows:

\begin{mathpar}
\freenames{\pzero} := \emptyset
  \and \\
  \freenames{x?(y).P} := \{ x \} \cup (\freenames{P} \setminus \{ y \})
  \and 
  \freenames{x!\langle P \rangle} := \{ x \} \cup \{ P \} 
  \and \\
  \freenames{P|Q} := \freenames{P} \cup \freenames{Q}
  \and \\
  \freenames{@{x}} := \{ x \}
\end{mathpar}

$\pi$
$\quotep{\pi}$

$\freenames{-} : \pi \to \mathcal{P}(\quotep{\pi})$

\begin{eqnarray*}
  \freenames{\pzero} & := & \emptyset \\
  \freenames{x?(y).P} & := & \{ x \} \cup (\freenames{P} \setminus \{ y \}) \\
  \freenames{x!\langle P \rangle} & := & \{ x \} \cup \{ P \} \\
  \freenames{P|Q} & := & \freenames{P} \cup \freenames{Q} \\
  \freenames{\dropn{x}} & := & \{ x \}
\end{eqnarray*}

The bound names of a process, $\boundnames{P}$, are those names occurring in $P$
that are not free. For example, in $x?(y).0$, the name $x$ is free, while $y$ is bound.

\begin{mathpar}
  \inferrule* [lab=monoidal-laws] {} { P|Q \equiv Q|P \and P|0 \equiv P \and P|(Q|R) \equiv (P|Q)|R }
\end{mathpar}

\begin{mathpar}
  \inferrule* [lab=alpha-equivalence] {} { (x)P \equiv (y)P\{y/x\} \and y \not\in \freenames{P} }
\end{mathpar}

\begin{definition}
Then two processes, $P,Q$, are alpha-equivalent if $P = Q\{\vec{y}/\vec{x}\}$ for
some $\vec{x} \in \boundnames{Q},\vec{y} \in \boundnames{P}$, where $Q\{\vec{y}/\vec{x}\}$
denotes the capture-avoiding substitution of $\vec{y}$ for $\vec{x}$ in $Q$.
\end{definition}

\begin{definition}
  The {\em structural congruence} \cite{SangiorgiWalker} , $\equiv$,
  between processes is the least congruence containing
  alpha-equivalence, satisfying the abelian monoid laws
  (associativity, commutativity and $\pzero$ as identity) for parallel
  composition $|$ and for summation $+$.
\end{definition}

\subsection{Name equivalence}

We take name equivalence, written $\nameeq$, to be the smallest
equivalence relation generated by the following rules.

\begin{mathpar}
\inferrule*[lab=Quote-drop]
{ }
{ \quotep{@{x}} \nameeq x }

\inferrule*[lab=Struct-equiv]
{ P \scong Q }
{ \quotep{P} \nameeq \quotep{Q} }
\end{mathpar}

The astute reader will have noticed that the mutual recursion of names
and processes imposes a mutual recursion on alpha-equivalence and
structural equivalence via name-equivalence. Fortunately, all of this
works out pleasantly and we may calculate in the natural way, free of
concern. The reader interested in the details is referred to the
appendix \ref{appendix:rho_details}.

\subsection{Substitution}

We use $\Proc$ for the set of processes, $\QProc$ for the set of
names, and $\id{\{}\vec{y} / \vec{x} \id{\}}$ to denote partial maps,
$s : \QProc \rightarrow \QProc$. A map, $s$ lifts, uniquely, to a map
on process terms, $\widehat{s} : \Proc \rightarrow \Proc$ by the
following equations.

\begin{mathpar}
  (0) \psubstp{Q}{P} := 0 \\
  (R \juxtap S) \psubstp{Q}{P}
  :=    
  (R)\psubstp{Q}{P} \juxtap (S) \psubstp{Q}{P} \\
  (x?(y).R) \psubstp{Q}{P}    
  :=    
  (x)\substp{Q}{P} (z)\concat( (R \psubstn{z}{y}) \psubstp{Q}{P} ) \\
  (\lift{x}{R}) \psubstp{Q}{P}  
  :=
  \lift{(x)\substp{Q}{P}}{ R \psubstp{Q}{P} } \\
%   (\dropn{x})  \psubstp{Q}{P}       
%   := 
%   \left\{ 
%     \begin{array}{ccc} 
%       \dropn{\quotep{Q}} & & x \nameeq \quotep{P} \\
%       \dropn{x} & & otherwise \\
%     \end{array}
%   \right. 
  (\dropn{x})  \psubstp{Q}{P}       
  := 
  \left\{ 
    \begin{array}{ccc} 
      Q & & x \nameeq \quotep{P} \\
      \dropn{x} & & otherwise \\
    \end{array}
  \right.
\end{mathpar}
 

where

\begin{eqnarray}
  (x)\id{\{} \lpquote Q \rpquote / \lpquote P \rpquote \id{\}}            = 
  \left\{ 
    \begin{array}{ccc}
      \lpquote Q \rpquote & & x \nameeq \lpquote P \rpquote \\
      x & & otherwise \\
    \end{array}
  \right. \nonumber
\end{eqnarray}

and $z$ is chosen distinct from $\quotep{P}$, $\quotep{Q}$, the free
names in $Q$, and all the names in $R$. Our $\alpha$-equivalence will
be built in the standard way from this substitution.

\begin{remark}\label{rem:no_self_referential_names}
  One consequence of these definitions is that $\forall P. \quotep{P}
  \not\in \freenames{P}$.
\end{remark}

\subsection{ Dynamic quote: an example }

Anticipating something of what's to come, consider applying the
substitution, $\widehat{\id{\{}u / z \id{\}}}$, to the following pair
of processes, $\lift{w}{y!(z)}$ and $w[ \lpquote y!(z) \rpquote ]$.

\begin{eqnarray}
	\lift{w}{y!(z)}\widehat{\id{\{}u / z \id{\}}}
		& = &
		\lift{w}{y!(u)} \nonumber\\
	w[ \lpquote y!(z) \rpquote ] \widehat{ \id{\{}u / z \id{\}} }
		& = &
		w[ \lpquote y!(z) \rpquote ] \nonumber
\end{eqnarray}

Because the body of the process between quotes is impervious to
substitution, we get radically different answers. In fact, by
examining the first process in an input context,
e.g. $x?(z).\lift{w}{y!(z)}$, we see that the process under the lift
operator may be shaped by prefixed inputs binding a name inside it. In
this sense, the lift operator will be seen as a way to dynamically
construct processes before reifying them as names.

Finally equipped with these standard features we can present the
dynamics of the calculus.

\subsubsection{Operational semantics} 

Finally, we introduce the computational dynamics. What marks these
algebras as distinct from other more traditionally studied algebraic
structures, e.g. vector spaces or polynomial rings, is the manner in
which dynamics is captured. In traditional structures, dynamics is typically
expressed through morphisms between such structures, as in linear maps
between vector spaces or morphisms between rings. In algebras
associated with the semantics of computation, the dynamics is
expressed as part of the algebraic structure itself, through a
reduction reduction relation typically denoted by $\red$. Below, we
give a recursive presentation of this relation for the calculus used
in the encoding.

$\red \subseteq \pi \times \pi$
$\red : \pi \to \mathcal{P}(\pi)$

\begin{mathpar}
  \inferrule* [lab=Comm] { \textsf{match}( x_{src}, x_{trgt} ) } { x_{trgt}?(y)P \; | \; x_{src}!\langle {Q} \rangle \red P\{\quotep{Q}/y}\} }
  \and \\
  \inferrule* [lab=Par] {{P} \red {P}'} {{{P} | {Q}} \red {{P}' | {Q}}}
  \and
  \inferrule* [lab=Equiv]{{{P} \scong {P}'} \andalso {{P}' \red {Q}'} \andalso {{Q}' \scong {Q}}}{{P} \red {Q}}
\end{mathpar}

\begin{eqnarray*}
  match_{\equiv} (\quotep{P},\quotep{Q}) & := & P \equiv Q \\
  match_{\dagger}(\quotep{P},\quotep{Q}) & := & \forall R. P|Q \red^{*} R => R \red^{*} 0 \\
  match_{K}(\quotep{P},\quotep{Q}) & := & K \mbox{ for some context } K
\end{eqnarray*}

$u?(x)P | u!\langle Q \rangle \red P\{\quotep{Q}/x\}$

%We write $\wred$ for $\red^*$, and $P\red$ if $\exists Q $ such that $ P \red Q$.
We write $P\red$ if $\exists Q $ such that $ P \red Q$ and $P\not\red$, otherwise.

\section{Replication}

As mentioned before, it is known that replication (and hence
recursion) can be implemented in a higher-order process algebra
\cite{SangiorgiWalker}. As our first example of calculation with the
machinery thus far presented we give the construction explicitly in
the {\rhoc}.

\begin{eqnarray}
	D_{x} & := & \prefix{x}{y}{(\binpar{\outputp{x}{y}}{@{y}})} \nonumber\\
	\bangp_{x}{P} & := & \binpar{{x}!\langle{\binpar{D_{x}}{P}}\rangle}{D_{x}} \nonumber
\end{eqnarray}

\begin{eqnarray}
	\bangp_{x}{P} & & \nonumber\\
	=
	& {x}!\langle{(\prefix{x}{y}{(\outputp{x}{y} | @{y})) | P}}\rangle 
	      | \prefix{x}{y}{(\outputp{x}{y} | @{y})} & \nonumber\\
	\red
	& (\outputp{x}{y} | @{y})\substn{\quotep{(\prefix{x}{y}{(@{y} | \outputp{x}{y})) | P}}}{y} & \nonumber\\
	=
	& \outputp{x}{\quotep{(\prefix{x}{y}{(\outputp{x}{y} | @{y})) | P}}}
	  | {(\prefix{x}{y}{(\outputp{x}{y} | @{y})) | P}} & \nonumber\\
	\red
	& \ldots & \nonumber\\
	\red^*
	& P | P | \ldots & \nonumber
\end{eqnarray}

Of course, this encoding, as an implementation, runs away, unfolding
$\bangp{P}$ eagerly. A lazier and more implementable replication
operator, restricted to input-guarded processes, may be obtained as follows.

\begin{eqnarray}
\bangp{\prefix{u}{v}{P}} 
	:= 
	\binpar{\lift{x}{\prefix{u}{v}{(\binpar{D(x)}{P})}}}{D(x)} \nonumber
\end{eqnarray}

\begin{remark}
  Note that the lazier definition still does not deal with summation
  or mixed summation (i.e. sums over input and output). The reader is
  invited to construct definitions of replication that deal with these
  features. 

  Further, the definitions are parameterized in a name, $x$. Can you,
  gentle reader, make a definition that eliminates this parameter and
  guarantees no accidental interaction between the replication
  machinery and the process being replicated -- i.e. no accidental
  sharing of names used by the process to get its work done and the
  name(s) used by the replication to effect copying. This latter
  revision of the definition of replication is crucial to obtaining
  the expected identity $!!P \sim !P$.
\end{remark}

\begin{remark}\label{rem:paradoxical_combinator}
  The reader familiar with the lambda calculus will have noticed the
  similarity between $D$ and the paradoxical combinator.

  [Ed. note: the existence of this seems to suggest we have to be more
  restrictive on the set of processes and names we admit if we are to
  support no-cloning.]
\end{remark}

\subsubsection{Bisimulation}

The computational dynamics gives rise to another kind of equivalence,
the equivalence of computational behavior. As previously mentioned
this is typically captured \emph{via} some form of bisimulation.

% The notion we use in this paper is weak barbed bisimulation
% \cite{milner91polyadicpi}.

The notion we use in this paper is derived from weak barbed
bisimulation \cite{milner91polyadicpi}. 

\begin{definition}
An \emph{observation relation}, $\downarrow_{\mathcal N}$, over a set
of names, $\mathcal N$, is the smallest relation satisfying the rules
below.

\infrule[Out-barb]{y \in {\mathcal N}, \; x \nameeq y}
		  {\outputp{x}{v} \downarrow_{\mathcal N} x}
\infrule[Par-barb]{\mbox{$P\downarrow_{\mathcal N} x$ or $Q\downarrow_{\mathcal N} x$}}
		  {\binpar{P}{Q} \downarrow_{\mathcal N} x}

We write $P \Downarrow_{\mathcal N} x$ if there is $Q$ such that 
$P \wred Q$ and $Q \downarrow_{\mathcal N} x$.
\end{definition}

\begin{definition}
%\label{def.bbisim}
An  ${\mathcal N}$-\emph{barbed bisimulation} over a set of names, ${\mathcal N}$, is a symmetric binary relation 
${\mathcal S}_{\mathcal N}$ between agents such that $P\rel{S}_{\mathcal N}Q$ implies:
\begin{enumerate}
\item If $P \red P'$ then $Q \wred Q'$ and $P'\rel{S}_{\mathcal N} Q'$.
\item If $P\downarrow_{\mathcal N} x$, then $Q\Downarrow_{\mathcal N} x$.
\end{enumerate}
$P$ is ${\mathcal N}$-barbed bisimilar to $Q$, written
$P \wbbisim_{\mathcal N} Q$, if $P \rel{S}_{\mathcal N} Q$ for some ${\mathcal N}$-barbed bisimulation ${\mathcal S}_{\mathcal N}$.
\end{definition}

$\mathcal{R} \subseteq \pi \times \pi$

$P \mathcal{R} Q => \forall P'. P \red P' \Rightarrow \exists Q'. Q \red Q', P' \mathcal{R} Q'$

$P \vdash x \Rightarrow Q \vdash x$

\begin{mathpar}
  \inferrule*[lab=Out-barb]{x \nameeq y}{{y}!\langle{Q}\rangle \vdash x}
  \and
  \inferrule*[lab=Par-barb]{\mbox{$P\vdash x$ or $Q\vdash x$}}{\binpar{P}{Q} \vdash x}
\end{mathpar}

\subsubsection{Contexts}

One of the principle advantages of computational calculi like the
$\pi$-calculus is a well-defined notion of context,
contextual-equivalence and a correlation between
contextual-equivalence and notions of bisimulation. The notion of
context allows the decomposition of a process into (sub-)process and
its syntactic environment, its context. Thus, a context may be
thought of as a process with a ``hole'' (written $\Box$) in it. The
application of a context $M$ to a process $P$, written $M[P]$, is
tantamount to filling the hole in $M$ with $P$. In this paper we do
not need the full weight of this theory, but do make use of the notion
of context in the proof the main theorem. 

\begin{mathpar}
  \inferrule* [lab=summation] {} {{M_{M},M_{N}} \bc \Box \;|\; x.M_{A} \;|\; M_{M}+M_{N}}
  \and
  \inferrule* [lab=agent] {} {{M_{A}} \bc (\vec{x})M_{P} \;| \; \clift{P_0,\ldots,M_{P},\ldots,P_N}}
  \and \\
  \inferrule* [lab=process] {} {{M_{P}} \bc M_{N} \;| \;P|M_{P} }
\end{mathpar} 

\begin{mathpar}
  \inferrule* [lab=sychronization] {} {M_{N} \bc \Box \;|\; x?M_{F} \;|\; x!M_{C}}
  \and
  \inferrule* [lab=abstraction] {} {{M_{F}} \bc (x)M_{P} }
  \and
  \inferrule* [lab=concretion] {} {{M_{C}} \bc \langle M_{P} \rangle }
  \and \\
  \inferrule* [lab=process] {} {{M_{P}} \bc M_{N} \;| \;P|M_{P} }
\end{mathpar}

\begin{definition}[contextual application] Given a context $M$, and
  process $P$, we define the \emph{contextual application}, $M[P] :=
  M\{P/\Box\}$. That is, the contextual application of M to P is the
  substitution of $P$ for $\Box$ in $M$.
\end{definition}

$\meaningof{-} : L \to \mathcal{P}(\pi)$

\begin{mathpar}
  \inferrule* [lab=collection] {} {\meaningof{true} = \pi, \and \meaningof{~E} = \pi \setminus \meaningof{E}, \and \meaningof{E_{1} \& E_{2}} = \meaningof{E_{1}} \cap \meaningof{E_{2}}}
\end{mathpar}

\begin{mathpar}
  \inferrule* [lab=structure] {} {\meaningof{0} = \{ P \in \pi | P \equiv 0 \}, \and \\ \meaningof{E_1 | E_2} = \{ P \in \pi | P \equiv P_{1} | P_{2}, P_{1} \in \meaningof{E_{1}}, P_{2} \in \meaningof{E_2}\} }
\end{mathpar}

\begin{mathpar}
 \inferrule* [lab=behavior] {} {\meaningof{\langle a?b \rangle E} = \{ P \in \pi | P \equiv Q | u?(y)P', \\ \and \\\\ \and \\ \;\;\; u \in \meaningof{a}, \forall z.P'\{z/y\} \in \meaningof{E\{z/b\}}\}, \and \\ \meaningof{a!E} = \{ P \in \pi | P \equiv Q | x!\langle P' \rangle, x \in \meaningof{a} P' \in \meaningof{E}\} }
\end{mathpar}

\begin{mathpar}
 \inferrule* [lab=nominal] {} {\meaningof{\quotep{E}} = \{ \quotep{P} \in \quotep{\pi} | P \in \meaningof{E} \}, \and \meaningof{\quotep{P}} = \{ \quotep{Q} \in \quotep{\pi} | P \equiv Q \} \and \\ \meaningof{@\quotep{E}} = \{ P \in \pi | P \equiv @x, x \in \meaningof{E} \}}
\end{mathpar}

\begin{eqnarray*}
  \\
  \meaningof{-} : TS \to ST
\end{eqnarray*}

\begin{eqnarray*}
  \\
  L : TS \to ST
\end{eqnarray*}

\begin{eqnarray*}
  \\
  P \models E \iff P \in \meaningof{E}
\end{eqnarray*}

\begin{eqnarray*}
  P \approx_{L} Q \iff \forall E \in L. P \models E \iff Q \models E
\end{eqnarray*}

\begin{eqnarray*}
  P \approx_{K} Q
\end{eqnarray*}

\begin{eqnarray*}
  P \approx Q
\end{eqnarray*}

$\approx_{K} = \approx = \approx_{L}$

\subsubsection{Contextual duality}

Note that contexts extend the quotation operation to a family of
operations from processes to names. Given a context, $M$, we can
define a \emph{nominal context}, $\quotep{M}$ by $\quotep{M}[P] :=
\quotep{M[P]}$. To foreshadow what is to come we observe that these
operations enjoy a duality with processes very much like the duality
between vectors and maps from vectors to scalars.

Further, because the calculus is essentially higher-order, we have a
correspondence between contexts and processes. More specifically,
given a name $x$ and a context $M$ we can construct $M^{*}_{x}$ such
that 

\begin{mathpar}
  M^{*}_{x} | \lift{x}{P} \red M[P]
\end{mathpar}

namely,

\begin{mathpar}
  M^{*}_{x} := x?(u).M[\dropn{u}]
\end{mathpar}

The dependence of $M^{*}_{x}$ on a name makes it an abstraction, 

\begin{mathpar}
  M^{*} := (x)x?(u).M[\dropn{u}]
\end{mathpar}

\subsection{Additional notation}

It will sometimes be convenient to denote the process a name
quotes. We already have the notation $x = \quotep{P}$, but it will be
convenient to introduce an alternate notation, $\procn{x}$, when we
want to emphasize the connection to the use of the name. Note that, by
virtue of name equivalence, $\quotep{\procn{x}} \nameeq x$; so, the
notation is consistent with previous definitions.

Further, because names have structure it is possible to effect
substitutions on the basis of that structure. This means we need to
upgrade our notation for substitutions, which we accomplish by
adapting comprehension notation. Thus,

\begin{mathpar}
  P\{ y / x : x \in S \}
\end{mathpar}

is interpreted to mean the process derived from P by replacing (in a
capture-avoiding manner) each occurrence of $x$ in $S$ by $y$. For example,

\begin{mathpar}
  P\{ \quotep{\procn{x}|\procn{x}} / x : x \in \freenames{P} \}
\end{mathpar}

will replace each (occurrence) of a free name $x$ in $P$ by
$\quotep{\procn{x}|\procn{x}}$.

Also, we will avail ourselves of the notation $x^{L}$ and $x^{R}$ to
denote injections of a name into disjoint copies of the name
space. There are numerous ways to accomplish this. One example can be
found in \cite{MeredithR05}. This notation overloads to vectors of
names: $\vec{x}^{\pi} := (x_{i}^{\pi} \; : \; 0 \leq i < |\vec{x}| )$ where $\pi \in \{L,R\}$.

We also use $P^{\Box} := P|\Box$.

In \cite{MeredithR05} an interpretation of the new operator is
given. It turns out that there are several possible interpretations
all enjoying the requisite algebraic properties of the operator (see
\cite{milner91polyadicpi}). We will therefore make liberal use of
$(\nu\; \vec{x})P$.

% subsection the_syntax_and_semantics_of_the_notation_system (end)   

\input{qm2pi.qmops} 

\input{qm2pi.sterngerlach} 

\input{qm2pi.metric} 

% section concurrent_process_calculi (end)

%\input{qm2pi.proofsketch}

% section proof sketch (end)

%\input{qm2pi.slviaknots} 

% section spatial logic via knots (end)

\input{qm2pi.conclusion}

% section conclusion (end)

%\input{qm2pi.dtcodes} 

% section wiring algorithm (end)

\input{qm2pi.ack} 

% section acknowledgments (end)

\newpage


\bibliographystyle{plain}   
\bibliography{../../biblios/main.bib}

\input{qm2pi.rhodetails}

\end{document}

 

% section acknowledgments (end)

\newpage


\bibliographystyle{plain}   
\bibliography{../../biblios/main.bib}

\documentclass[12pt]{llncs}
%\documentclass{jktr}

\usepackage[pdftex]{hyperref}                   
\usepackage {listings}
\usepackage {mathpartir}
\usepackage{bcprules}
%\usepackage{listings}
                       
\usepackage{graphicx} 
%\usepackage[margins=2.5cm,nohead,nofoot]{geometry}
%\usepackage{geometry}
\usepackage{amsfonts}
\usepackage{amstext}
\usepackage{latexsym}
\usepackage{amssymb}
\usepackage{color}


%\include{myPreamble}
\include{qm2pi.local} 

%\ifpdf
%\usepackage[pdftex]{graphicx}
%\else
%\usepackage{graphicx}
%\fi

 % \ifpdf
%  \usepackage{pdfsync}
%  \if


%\title{Brief Article}
%\author{David F. Snyder}
%\author{L.G. Meredith}

%\address{Dept. of Math., Texas State University--San Marcos, San Marcos, TX 78666}
       
\pagestyle{empty}


\begin{document}

\lstset{language=[Objective]Caml,frame=shadowbox}

\input{qm2pi.front}

% section front matter (end)

\input{qm2pi.intro} 
 
% section introduction (end)

% \input{qm2pi.knotations} 

% section notation (end)

\input{qm2pi.process.calculi} 

% section concurrent_process_calculi_and_spatial_logics_ (end)
    
%\input{qm2pi.knots2pi} 

%\input{qm2pi.trefoil} 

%\input{qm2pi.mainthm} 

% subsection basic_interpretation (end)

%\input{qm2pi.rho.presentation} 
\subsection{The syntax and semantics of the notation system}\label{sub:the_syntax_and_semantics_of_the_notation_system} % (fold)

We now summarize a technical presentation of the calculus that
embodies our theory of dynamics. The typical presentation of such a
calculus follows the style of giving generators and relations on
them. The grammar, below, describing term constructors, freely
generates the set of processes, $\Proc$. This set is then quotiented
by a relation known as structural congruence and it is over this set
that the notion of dynamics is expressed. This presentation is
essentially that of \cite{MeredithR05} with the addition of
polyadicity and summation. For readability we have relegated some of
the technical subtleties to an appendix.

\subsubsection{Process grammar}\label{subsub:process_grammar}

\begin{mathpar}
  \inferrule* [lab=synchronization] {} {{M} \bc \pzero \;|\; x?F \;|\; x!C }
  \and
  \inferrule* [lab=abstraction] {} {{F} \bc (x)P}
  \and
  \inferrule* [lab=concretion] {} {{C} \bc \langle Q \rangle}
  \and
  \inferrule* [lab=process] {} {{P,Q} \bc M \;| \;P|Q \;|\; @{x}}
  \and
  \inferrule* [lab=name] {} {{x} \bc \quotep{P}}
\end{mathpar} 

Note that $\vec{x}$ (resp. $\vec{P}$) denotes a vector of names
(resp. processes) of length $|\vec{x}|$ (resp. $|\vec{P}|$). We adopt
the following useful abbreviations.

\begin{mathpar}
   x?(\vec{y}).P := x.(\vec{y})P \and  x\clift{\vec{P}} := x.\clift{\vec{P}}
   \and x!(y) := \lift{x}{\dropn{y}}
   \and \Pi_{i=0}^{n-1}P_i := P_0 | \ldots | P_{n-1}
\end{mathpar}

\subsubsection{Structural congruence}

\paragraph{Free and bound names and alpha-equivalence.} At the
core of structural equivalence is alpha-equivalence which identifies
process that are the same up to a change of variable. Formally, we
recognize the distinction between free and bound names. The free names
of a process, $\freenames{P}$, may be calculated recursively as
follows:

\begin{mathpar}
\freenames{\pzero} := \emptyset
  \and \\
  \freenames{x?(y).P} := \{ x \} \cup (\freenames{P} \setminus \{ y \})
  \and 
  \freenames{x!\langle P \rangle} := \{ x \} \cup \{ P \} 
  \and \\
  \freenames{P|Q} := \freenames{P} \cup \freenames{Q}
  \and \\
  \freenames{@{x}} := \{ x \}
\end{mathpar}

$\pi$
$\quotep{\pi}$

$\freenames{-} : \pi \to \mathcal{P}(\quotep{\pi})$

\begin{eqnarray*}
  \freenames{\pzero} & := & \emptyset \\
  \freenames{x?(y).P} & := & \{ x \} \cup (\freenames{P} \setminus \{ y \}) \\
  \freenames{x!\langle P \rangle} & := & \{ x \} \cup \{ P \} \\
  \freenames{P|Q} & := & \freenames{P} \cup \freenames{Q} \\
  \freenames{\dropn{x}} & := & \{ x \}
\end{eqnarray*}

The bound names of a process, $\boundnames{P}$, are those names occurring in $P$
that are not free. For example, in $x?(y).0$, the name $x$ is free, while $y$ is bound.

\begin{mathpar}
  \inferrule* [lab=monoidal-laws] {} { P|Q \equiv Q|P \and P|0 \equiv P \and P|(Q|R) \equiv (P|Q)|R }
\end{mathpar}

\begin{mathpar}
  \inferrule* [lab=alpha-equivalence] {} { (x)P \equiv (y)P\{y/x\} \and y \not\in \freenames{P} }
\end{mathpar}

\begin{definition}
Then two processes, $P,Q$, are alpha-equivalent if $P = Q\{\vec{y}/\vec{x}\}$ for
some $\vec{x} \in \boundnames{Q},\vec{y} \in \boundnames{P}$, where $Q\{\vec{y}/\vec{x}\}$
denotes the capture-avoiding substitution of $\vec{y}$ for $\vec{x}$ in $Q$.
\end{definition}

\begin{definition}
  The {\em structural congruence} \cite{SangiorgiWalker} , $\equiv$,
  between processes is the least congruence containing
  alpha-equivalence, satisfying the abelian monoid laws
  (associativity, commutativity and $\pzero$ as identity) for parallel
  composition $|$ and for summation $+$.
\end{definition}

\subsection{Name equivalence}

We take name equivalence, written $\nameeq$, to be the smallest
equivalence relation generated by the following rules.

\begin{mathpar}
\inferrule*[lab=Quote-drop]
{ }
{ \quotep{@{x}} \nameeq x }

\inferrule*[lab=Struct-equiv]
{ P \scong Q }
{ \quotep{P} \nameeq \quotep{Q} }
\end{mathpar}

The astute reader will have noticed that the mutual recursion of names
and processes imposes a mutual recursion on alpha-equivalence and
structural equivalence via name-equivalence. Fortunately, all of this
works out pleasantly and we may calculate in the natural way, free of
concern. The reader interested in the details is referred to the
appendix \ref{appendix:rho_details}.

\subsection{Substitution}

We use $\Proc$ for the set of processes, $\QProc$ for the set of
names, and $\id{\{}\vec{y} / \vec{x} \id{\}}$ to denote partial maps,
$s : \QProc \rightarrow \QProc$. A map, $s$ lifts, uniquely, to a map
on process terms, $\widehat{s} : \Proc \rightarrow \Proc$ by the
following equations.

\begin{mathpar}
  (0) \psubstp{Q}{P} := 0 \\
  (R \juxtap S) \psubstp{Q}{P}
  :=    
  (R)\psubstp{Q}{P} \juxtap (S) \psubstp{Q}{P} \\
  (x?(y).R) \psubstp{Q}{P}    
  :=    
  (x)\substp{Q}{P} (z)\concat( (R \psubstn{z}{y}) \psubstp{Q}{P} ) \\
  (\lift{x}{R}) \psubstp{Q}{P}  
  :=
  \lift{(x)\substp{Q}{P}}{ R \psubstp{Q}{P} } \\
%   (\dropn{x})  \psubstp{Q}{P}       
%   := 
%   \left\{ 
%     \begin{array}{ccc} 
%       \dropn{\quotep{Q}} & & x \nameeq \quotep{P} \\
%       \dropn{x} & & otherwise \\
%     \end{array}
%   \right. 
  (\dropn{x})  \psubstp{Q}{P}       
  := 
  \left\{ 
    \begin{array}{ccc} 
      Q & & x \nameeq \quotep{P} \\
      \dropn{x} & & otherwise \\
    \end{array}
  \right.
\end{mathpar}
 

where

\begin{eqnarray}
  (x)\id{\{} \lpquote Q \rpquote / \lpquote P \rpquote \id{\}}            = 
  \left\{ 
    \begin{array}{ccc}
      \lpquote Q \rpquote & & x \nameeq \lpquote P \rpquote \\
      x & & otherwise \\
    \end{array}
  \right. \nonumber
\end{eqnarray}

and $z$ is chosen distinct from $\quotep{P}$, $\quotep{Q}$, the free
names in $Q$, and all the names in $R$. Our $\alpha$-equivalence will
be built in the standard way from this substitution.

\begin{remark}\label{rem:no_self_referential_names}
  One consequence of these definitions is that $\forall P. \quotep{P}
  \not\in \freenames{P}$.
\end{remark}

\subsection{ Dynamic quote: an example }

Anticipating something of what's to come, consider applying the
substitution, $\widehat{\id{\{}u / z \id{\}}}$, to the following pair
of processes, $\lift{w}{y!(z)}$ and $w[ \lpquote y!(z) \rpquote ]$.

\begin{eqnarray}
	\lift{w}{y!(z)}\widehat{\id{\{}u / z \id{\}}}
		& = &
		\lift{w}{y!(u)} \nonumber\\
	w[ \lpquote y!(z) \rpquote ] \widehat{ \id{\{}u / z \id{\}} }
		& = &
		w[ \lpquote y!(z) \rpquote ] \nonumber
\end{eqnarray}

Because the body of the process between quotes is impervious to
substitution, we get radically different answers. In fact, by
examining the first process in an input context,
e.g. $x?(z).\lift{w}{y!(z)}$, we see that the process under the lift
operator may be shaped by prefixed inputs binding a name inside it. In
this sense, the lift operator will be seen as a way to dynamically
construct processes before reifying them as names.

Finally equipped with these standard features we can present the
dynamics of the calculus.

\subsubsection{Operational semantics} 

Finally, we introduce the computational dynamics. What marks these
algebras as distinct from other more traditionally studied algebraic
structures, e.g. vector spaces or polynomial rings, is the manner in
which dynamics is captured. In traditional structures, dynamics is typically
expressed through morphisms between such structures, as in linear maps
between vector spaces or morphisms between rings. In algebras
associated with the semantics of computation, the dynamics is
expressed as part of the algebraic structure itself, through a
reduction reduction relation typically denoted by $\red$. Below, we
give a recursive presentation of this relation for the calculus used
in the encoding.

$\red \subseteq \pi \times \pi$
$\red : \pi \to \mathcal{P}(\pi)$

\begin{mathpar}
  \inferrule* [lab=Comm] { \textsf{match}( x_{src}, x_{trgt} ) } { x_{trgt}?(y)P \; | \; x_{src}!\langle {Q} \rangle \red P\{\quotep{Q}/y}\} }
  \and \\
  \inferrule* [lab=Par] {{P} \red {P}'} {{{P} | {Q}} \red {{P}' | {Q}}}
  \and
  \inferrule* [lab=Equiv]{{{P} \scong {P}'} \andalso {{P}' \red {Q}'} \andalso {{Q}' \scong {Q}}}{{P} \red {Q}}
\end{mathpar}

\begin{eqnarray*}
  match_{\equiv} (\quotep{P},\quotep{Q}) & := & P \equiv Q \\
  match_{\dagger}(\quotep{P},\quotep{Q}) & := & \forall R. P|Q \red^{*} R => R \red^{*} 0 \\
  match_{K}(\quotep{P},\quotep{Q}) & := & K \mbox{ for some context } K
\end{eqnarray*}

$u?(x)P | u!\langle Q \rangle \red P\{\quotep{Q}/x\}$

%We write $\wred$ for $\red^*$, and $P\red$ if $\exists Q $ such that $ P \red Q$.
We write $P\red$ if $\exists Q $ such that $ P \red Q$ and $P\not\red$, otherwise.

\section{Replication}

As mentioned before, it is known that replication (and hence
recursion) can be implemented in a higher-order process algebra
\cite{SangiorgiWalker}. As our first example of calculation with the
machinery thus far presented we give the construction explicitly in
the {\rhoc}.

\begin{eqnarray}
	D_{x} & := & \prefix{x}{y}{(\binpar{\outputp{x}{y}}{@{y}})} \nonumber\\
	\bangp_{x}{P} & := & \binpar{{x}!\langle{\binpar{D_{x}}{P}}\rangle}{D_{x}} \nonumber
\end{eqnarray}

\begin{eqnarray}
	\bangp_{x}{P} & & \nonumber\\
	=
	& {x}!\langle{(\prefix{x}{y}{(\outputp{x}{y} | @{y})) | P}}\rangle 
	      | \prefix{x}{y}{(\outputp{x}{y} | @{y})} & \nonumber\\
	\red
	& (\outputp{x}{y} | @{y})\substn{\quotep{(\prefix{x}{y}{(@{y} | \outputp{x}{y})) | P}}}{y} & \nonumber\\
	=
	& \outputp{x}{\quotep{(\prefix{x}{y}{(\outputp{x}{y} | @{y})) | P}}}
	  | {(\prefix{x}{y}{(\outputp{x}{y} | @{y})) | P}} & \nonumber\\
	\red
	& \ldots & \nonumber\\
	\red^*
	& P | P | \ldots & \nonumber
\end{eqnarray}

Of course, this encoding, as an implementation, runs away, unfolding
$\bangp{P}$ eagerly. A lazier and more implementable replication
operator, restricted to input-guarded processes, may be obtained as follows.

\begin{eqnarray}
\bangp{\prefix{u}{v}{P}} 
	:= 
	\binpar{\lift{x}{\prefix{u}{v}{(\binpar{D(x)}{P})}}}{D(x)} \nonumber
\end{eqnarray}

\begin{remark}
  Note that the lazier definition still does not deal with summation
  or mixed summation (i.e. sums over input and output). The reader is
  invited to construct definitions of replication that deal with these
  features. 

  Further, the definitions are parameterized in a name, $x$. Can you,
  gentle reader, make a definition that eliminates this parameter and
  guarantees no accidental interaction between the replication
  machinery and the process being replicated -- i.e. no accidental
  sharing of names used by the process to get its work done and the
  name(s) used by the replication to effect copying. This latter
  revision of the definition of replication is crucial to obtaining
  the expected identity $!!P \sim !P$.
\end{remark}

\begin{remark}\label{rem:paradoxical_combinator}
  The reader familiar with the lambda calculus will have noticed the
  similarity between $D$ and the paradoxical combinator.

  [Ed. note: the existence of this seems to suggest we have to be more
  restrictive on the set of processes and names we admit if we are to
  support no-cloning.]
\end{remark}

\subsubsection{Bisimulation}

The computational dynamics gives rise to another kind of equivalence,
the equivalence of computational behavior. As previously mentioned
this is typically captured \emph{via} some form of bisimulation.

% The notion we use in this paper is weak barbed bisimulation
% \cite{milner91polyadicpi}.

The notion we use in this paper is derived from weak barbed
bisimulation \cite{milner91polyadicpi}. 

\begin{definition}
An \emph{observation relation}, $\downarrow_{\mathcal N}$, over a set
of names, $\mathcal N$, is the smallest relation satisfying the rules
below.

\infrule[Out-barb]{y \in {\mathcal N}, \; x \nameeq y}
		  {\outputp{x}{v} \downarrow_{\mathcal N} x}
\infrule[Par-barb]{\mbox{$P\downarrow_{\mathcal N} x$ or $Q\downarrow_{\mathcal N} x$}}
		  {\binpar{P}{Q} \downarrow_{\mathcal N} x}

We write $P \Downarrow_{\mathcal N} x$ if there is $Q$ such that 
$P \wred Q$ and $Q \downarrow_{\mathcal N} x$.
\end{definition}

\begin{definition}
%\label{def.bbisim}
An  ${\mathcal N}$-\emph{barbed bisimulation} over a set of names, ${\mathcal N}$, is a symmetric binary relation 
${\mathcal S}_{\mathcal N}$ between agents such that $P\rel{S}_{\mathcal N}Q$ implies:
\begin{enumerate}
\item If $P \red P'$ then $Q \wred Q'$ and $P'\rel{S}_{\mathcal N} Q'$.
\item If $P\downarrow_{\mathcal N} x$, then $Q\Downarrow_{\mathcal N} x$.
\end{enumerate}
$P$ is ${\mathcal N}$-barbed bisimilar to $Q$, written
$P \wbbisim_{\mathcal N} Q$, if $P \rel{S}_{\mathcal N} Q$ for some ${\mathcal N}$-barbed bisimulation ${\mathcal S}_{\mathcal N}$.
\end{definition}

$\mathcal{R} \subseteq \pi \times \pi$

$P \mathcal{R} Q => \forall P'. P \red P' \Rightarrow \exists Q'. Q \red Q', P' \mathcal{R} Q'$

$P \vdash x \Rightarrow Q \vdash x$

\begin{mathpar}
  \inferrule*[lab=Out-barb]{x \nameeq y}{{y}!\langle{Q}\rangle \vdash x}
  \and
  \inferrule*[lab=Par-barb]{\mbox{$P\vdash x$ or $Q\vdash x$}}{\binpar{P}{Q} \vdash x}
\end{mathpar}

\subsubsection{Contexts}

One of the principle advantages of computational calculi like the
$\pi$-calculus is a well-defined notion of context,
contextual-equivalence and a correlation between
contextual-equivalence and notions of bisimulation. The notion of
context allows the decomposition of a process into (sub-)process and
its syntactic environment, its context. Thus, a context may be
thought of as a process with a ``hole'' (written $\Box$) in it. The
application of a context $M$ to a process $P$, written $M[P]$, is
tantamount to filling the hole in $M$ with $P$. In this paper we do
not need the full weight of this theory, but do make use of the notion
of context in the proof the main theorem. 

\begin{mathpar}
  \inferrule* [lab=summation] {} {{M_{M},M_{N}} \bc \Box \;|\; x.M_{A} \;|\; M_{M}+M_{N}}
  \and
  \inferrule* [lab=agent] {} {{M_{A}} \bc (\vec{x})M_{P} \;| \; \clift{P_0,\ldots,M_{P},\ldots,P_N}}
  \and \\
  \inferrule* [lab=process] {} {{M_{P}} \bc M_{N} \;| \;P|M_{P} }
\end{mathpar} 

\begin{mathpar}
  \inferrule* [lab=sychronization] {} {M_{N} \bc \Box \;|\; x?M_{F} \;|\; x!M_{C}}
  \and
  \inferrule* [lab=abstraction] {} {{M_{F}} \bc (x)M_{P} }
  \and
  \inferrule* [lab=concretion] {} {{M_{C}} \bc \langle M_{P} \rangle }
  \and \\
  \inferrule* [lab=process] {} {{M_{P}} \bc M_{N} \;| \;P|M_{P} }
\end{mathpar}

\begin{definition}[contextual application] Given a context $M$, and
  process $P$, we define the \emph{contextual application}, $M[P] :=
  M\{P/\Box\}$. That is, the contextual application of M to P is the
  substitution of $P$ for $\Box$ in $M$.
\end{definition}

$\meaningof{-} : L \to \mathcal{P}(\pi)$

\begin{mathpar}
  \inferrule* [lab=collection] {} {\meaningof{true} = \pi, \and \meaningof{~E} = \pi \setminus \meaningof{E}, \and \meaningof{E_{1} \& E_{2}} = \meaningof{E_{1}} \cap \meaningof{E_{2}}}
\end{mathpar}

\begin{mathpar}
  \inferrule* [lab=structure] {} {\meaningof{0} = \{ P \in \pi | P \equiv 0 \}, \and \\ \meaningof{E_1 | E_2} = \{ P \in \pi | P \equiv P_{1} | P_{2}, P_{1} \in \meaningof{E_{1}}, P_{2} \in \meaningof{E_2}\} }
\end{mathpar}

\begin{mathpar}
 \inferrule* [lab=behavior] {} {\meaningof{\langle a?b \rangle E} = \{ P \in \pi | P \equiv Q | u?(y)P', \\ \and \\\\ \and \\ \;\;\; u \in \meaningof{a}, \forall z.P'\{z/y\} \in \meaningof{E\{z/b\}}\}, \and \\ \meaningof{a!E} = \{ P \in \pi | P \equiv Q | x!\langle P' \rangle, x \in \meaningof{a} P' \in \meaningof{E}\} }
\end{mathpar}

\begin{mathpar}
 \inferrule* [lab=nominal] {} {\meaningof{\quotep{E}} = \{ \quotep{P} \in \quotep{\pi} | P \in \meaningof{E} \}, \and \meaningof{\quotep{P}} = \{ \quotep{Q} \in \quotep{\pi} | P \equiv Q \} \and \\ \meaningof{@\quotep{E}} = \{ P \in \pi | P \equiv @x, x \in \meaningof{E} \}}
\end{mathpar}

\begin{eqnarray*}
  \\
  \meaningof{-} : TS \to ST
\end{eqnarray*}

\begin{eqnarray*}
  \\
  L : TS \to ST
\end{eqnarray*}

\begin{eqnarray*}
  \\
  P \models E \iff P \in \meaningof{E}
\end{eqnarray*}

\begin{eqnarray*}
  P \approx_{L} Q \iff \forall E \in L. P \models E \iff Q \models E
\end{eqnarray*}

\begin{eqnarray*}
  P \approx_{K} Q
\end{eqnarray*}

\begin{eqnarray*}
  P \approx Q
\end{eqnarray*}

$\approx_{K} = \approx = \approx_{L}$

\subsubsection{Contextual duality}

Note that contexts extend the quotation operation to a family of
operations from processes to names. Given a context, $M$, we can
define a \emph{nominal context}, $\quotep{M}$ by $\quotep{M}[P] :=
\quotep{M[P]}$. To foreshadow what is to come we observe that these
operations enjoy a duality with processes very much like the duality
between vectors and maps from vectors to scalars.

Further, because the calculus is essentially higher-order, we have a
correspondence between contexts and processes. More specifically,
given a name $x$ and a context $M$ we can construct $M^{*}_{x}$ such
that 

\begin{mathpar}
  M^{*}_{x} | \lift{x}{P} \red M[P]
\end{mathpar}

namely,

\begin{mathpar}
  M^{*}_{x} := x?(u).M[\dropn{u}]
\end{mathpar}

The dependence of $M^{*}_{x}$ on a name makes it an abstraction, 

\begin{mathpar}
  M^{*} := (x)x?(u).M[\dropn{u}]
\end{mathpar}

\subsection{Additional notation}

It will sometimes be convenient to denote the process a name
quotes. We already have the notation $x = \quotep{P}$, but it will be
convenient to introduce an alternate notation, $\procn{x}$, when we
want to emphasize the connection to the use of the name. Note that, by
virtue of name equivalence, $\quotep{\procn{x}} \nameeq x$; so, the
notation is consistent with previous definitions.

Further, because names have structure it is possible to effect
substitutions on the basis of that structure. This means we need to
upgrade our notation for substitutions, which we accomplish by
adapting comprehension notation. Thus,

\begin{mathpar}
  P\{ y / x : x \in S \}
\end{mathpar}

is interpreted to mean the process derived from P by replacing (in a
capture-avoiding manner) each occurrence of $x$ in $S$ by $y$. For example,

\begin{mathpar}
  P\{ \quotep{\procn{x}|\procn{x}} / x : x \in \freenames{P} \}
\end{mathpar}

will replace each (occurrence) of a free name $x$ in $P$ by
$\quotep{\procn{x}|\procn{x}}$.

Also, we will avail ourselves of the notation $x^{L}$ and $x^{R}$ to
denote injections of a name into disjoint copies of the name
space. There are numerous ways to accomplish this. One example can be
found in \cite{MeredithR05}. This notation overloads to vectors of
names: $\vec{x}^{\pi} := (x_{i}^{\pi} \; : \; 0 \leq i < |\vec{x}| )$ where $\pi \in \{L,R\}$.

We also use $P^{\Box} := P|\Box$.

In \cite{MeredithR05} an interpretation of the new operator is
given. It turns out that there are several possible interpretations
all enjoying the requisite algebraic properties of the operator (see
\cite{milner91polyadicpi}). We will therefore make liberal use of
$(\nu\; \vec{x})P$.

% subsection the_syntax_and_semantics_of_the_notation_system (end)   

\input{qm2pi.qmops} 

\input{qm2pi.sterngerlach} 

\input{qm2pi.metric} 

% section concurrent_process_calculi (end)

%\input{qm2pi.proofsketch}

% section proof sketch (end)

%\input{qm2pi.slviaknots} 

% section spatial logic via knots (end)

\input{qm2pi.conclusion}

% section conclusion (end)

%\input{qm2pi.dtcodes} 

% section wiring algorithm (end)

\input{qm2pi.ack} 

% section acknowledgments (end)

\newpage


\bibliographystyle{plain}   
\bibliography{../../biblios/main.bib}

\input{qm2pi.rhodetails}

\end{document}



\end{document}

 

%\ifpdf
%\usepackage[pdftex]{graphicx}
%\else
%\usepackage{graphicx}
%\fi

 % \ifpdf
%  \usepackage{pdfsync}
%  \if


%\title{Brief Article}
%\author{David F. Snyder}
%\author{L.G. Meredith}

%\address{Dept. of Math., Texas State University--San Marcos, San Marcos, TX 78666}
       
\pagestyle{empty}


\begin{document}

\lstset{language=[Objective]Caml,frame=shadowbox}

\documentclass[12pt]{llncs}
%\documentclass{jktr}

\usepackage[pdftex]{hyperref}                   
\usepackage {listings}
\usepackage {mathpartir}
\usepackage{bcprules}
%\usepackage{listings}
                       
\usepackage{graphicx} 
%\usepackage[margins=2.5cm,nohead,nofoot]{geometry}
%\usepackage{geometry}
\usepackage{amsfonts}
\usepackage{amstext}
\usepackage{latexsym}
\usepackage{amssymb}
\usepackage{color}


%\include{myPreamble}
\documentclass[12pt]{llncs}
%\documentclass{jktr}

\usepackage[pdftex]{hyperref}                   
\usepackage {listings}
\usepackage {mathpartir}
\usepackage{bcprules}
%\usepackage{listings}
                       
\usepackage{graphicx} 
%\usepackage[margins=2.5cm,nohead,nofoot]{geometry}
%\usepackage{geometry}
\usepackage{amsfonts}
\usepackage{amstext}
\usepackage{latexsym}
\usepackage{amssymb}
\usepackage{color}


%\include{myPreamble}
\include{qm2pi.local} 

%\ifpdf
%\usepackage[pdftex]{graphicx}
%\else
%\usepackage{graphicx}
%\fi

 % \ifpdf
%  \usepackage{pdfsync}
%  \if


%\title{Brief Article}
%\author{David F. Snyder}
%\author{L.G. Meredith}

%\address{Dept. of Math., Texas State University--San Marcos, San Marcos, TX 78666}
       
\pagestyle{empty}


\begin{document}

\lstset{language=[Objective]Caml,frame=shadowbox}

\input{qm2pi.front}

% section front matter (end)

\input{qm2pi.intro} 
 
% section introduction (end)

% \input{qm2pi.knotations} 

% section notation (end)

\input{qm2pi.process.calculi} 

% section concurrent_process_calculi_and_spatial_logics_ (end)
    
%\input{qm2pi.knots2pi} 

%\input{qm2pi.trefoil} 

%\input{qm2pi.mainthm} 

% subsection basic_interpretation (end)

%\input{qm2pi.rho.presentation} 
\subsection{The syntax and semantics of the notation system}\label{sub:the_syntax_and_semantics_of_the_notation_system} % (fold)

We now summarize a technical presentation of the calculus that
embodies our theory of dynamics. The typical presentation of such a
calculus follows the style of giving generators and relations on
them. The grammar, below, describing term constructors, freely
generates the set of processes, $\Proc$. This set is then quotiented
by a relation known as structural congruence and it is over this set
that the notion of dynamics is expressed. This presentation is
essentially that of \cite{MeredithR05} with the addition of
polyadicity and summation. For readability we have relegated some of
the technical subtleties to an appendix.

\subsubsection{Process grammar}\label{subsub:process_grammar}

\begin{mathpar}
  \inferrule* [lab=synchronization] {} {{M} \bc \pzero \;|\; x?F \;|\; x!C }
  \and
  \inferrule* [lab=abstraction] {} {{F} \bc (x)P}
  \and
  \inferrule* [lab=concretion] {} {{C} \bc \langle Q \rangle}
  \and
  \inferrule* [lab=process] {} {{P,Q} \bc M \;| \;P|Q \;|\; @{x}}
  \and
  \inferrule* [lab=name] {} {{x} \bc \quotep{P}}
\end{mathpar} 

Note that $\vec{x}$ (resp. $\vec{P}$) denotes a vector of names
(resp. processes) of length $|\vec{x}|$ (resp. $|\vec{P}|$). We adopt
the following useful abbreviations.

\begin{mathpar}
   x?(\vec{y}).P := x.(\vec{y})P \and  x\clift{\vec{P}} := x.\clift{\vec{P}}
   \and x!(y) := \lift{x}{\dropn{y}}
   \and \Pi_{i=0}^{n-1}P_i := P_0 | \ldots | P_{n-1}
\end{mathpar}

\subsubsection{Structural congruence}

\paragraph{Free and bound names and alpha-equivalence.} At the
core of structural equivalence is alpha-equivalence which identifies
process that are the same up to a change of variable. Formally, we
recognize the distinction between free and bound names. The free names
of a process, $\freenames{P}$, may be calculated recursively as
follows:

\begin{mathpar}
\freenames{\pzero} := \emptyset
  \and \\
  \freenames{x?(y).P} := \{ x \} \cup (\freenames{P} \setminus \{ y \})
  \and 
  \freenames{x!\langle P \rangle} := \{ x \} \cup \{ P \} 
  \and \\
  \freenames{P|Q} := \freenames{P} \cup \freenames{Q}
  \and \\
  \freenames{@{x}} := \{ x \}
\end{mathpar}

$\pi$
$\quotep{\pi}$

$\freenames{-} : \pi \to \mathcal{P}(\quotep{\pi})$

\begin{eqnarray*}
  \freenames{\pzero} & := & \emptyset \\
  \freenames{x?(y).P} & := & \{ x \} \cup (\freenames{P} \setminus \{ y \}) \\
  \freenames{x!\langle P \rangle} & := & \{ x \} \cup \{ P \} \\
  \freenames{P|Q} & := & \freenames{P} \cup \freenames{Q} \\
  \freenames{\dropn{x}} & := & \{ x \}
\end{eqnarray*}

The bound names of a process, $\boundnames{P}$, are those names occurring in $P$
that are not free. For example, in $x?(y).0$, the name $x$ is free, while $y$ is bound.

\begin{mathpar}
  \inferrule* [lab=monoidal-laws] {} { P|Q \equiv Q|P \and P|0 \equiv P \and P|(Q|R) \equiv (P|Q)|R }
\end{mathpar}

\begin{mathpar}
  \inferrule* [lab=alpha-equivalence] {} { (x)P \equiv (y)P\{y/x\} \and y \not\in \freenames{P} }
\end{mathpar}

\begin{definition}
Then two processes, $P,Q$, are alpha-equivalent if $P = Q\{\vec{y}/\vec{x}\}$ for
some $\vec{x} \in \boundnames{Q},\vec{y} \in \boundnames{P}$, where $Q\{\vec{y}/\vec{x}\}$
denotes the capture-avoiding substitution of $\vec{y}$ for $\vec{x}$ in $Q$.
\end{definition}

\begin{definition}
  The {\em structural congruence} \cite{SangiorgiWalker} , $\equiv$,
  between processes is the least congruence containing
  alpha-equivalence, satisfying the abelian monoid laws
  (associativity, commutativity and $\pzero$ as identity) for parallel
  composition $|$ and for summation $+$.
\end{definition}

\subsection{Name equivalence}

We take name equivalence, written $\nameeq$, to be the smallest
equivalence relation generated by the following rules.

\begin{mathpar}
\inferrule*[lab=Quote-drop]
{ }
{ \quotep{@{x}} \nameeq x }

\inferrule*[lab=Struct-equiv]
{ P \scong Q }
{ \quotep{P} \nameeq \quotep{Q} }
\end{mathpar}

The astute reader will have noticed that the mutual recursion of names
and processes imposes a mutual recursion on alpha-equivalence and
structural equivalence via name-equivalence. Fortunately, all of this
works out pleasantly and we may calculate in the natural way, free of
concern. The reader interested in the details is referred to the
appendix \ref{appendix:rho_details}.

\subsection{Substitution}

We use $\Proc$ for the set of processes, $\QProc$ for the set of
names, and $\id{\{}\vec{y} / \vec{x} \id{\}}$ to denote partial maps,
$s : \QProc \rightarrow \QProc$. A map, $s$ lifts, uniquely, to a map
on process terms, $\widehat{s} : \Proc \rightarrow \Proc$ by the
following equations.

\begin{mathpar}
  (0) \psubstp{Q}{P} := 0 \\
  (R \juxtap S) \psubstp{Q}{P}
  :=    
  (R)\psubstp{Q}{P} \juxtap (S) \psubstp{Q}{P} \\
  (x?(y).R) \psubstp{Q}{P}    
  :=    
  (x)\substp{Q}{P} (z)\concat( (R \psubstn{z}{y}) \psubstp{Q}{P} ) \\
  (\lift{x}{R}) \psubstp{Q}{P}  
  :=
  \lift{(x)\substp{Q}{P}}{ R \psubstp{Q}{P} } \\
%   (\dropn{x})  \psubstp{Q}{P}       
%   := 
%   \left\{ 
%     \begin{array}{ccc} 
%       \dropn{\quotep{Q}} & & x \nameeq \quotep{P} \\
%       \dropn{x} & & otherwise \\
%     \end{array}
%   \right. 
  (\dropn{x})  \psubstp{Q}{P}       
  := 
  \left\{ 
    \begin{array}{ccc} 
      Q & & x \nameeq \quotep{P} \\
      \dropn{x} & & otherwise \\
    \end{array}
  \right.
\end{mathpar}
 

where

\begin{eqnarray}
  (x)\id{\{} \lpquote Q \rpquote / \lpquote P \rpquote \id{\}}            = 
  \left\{ 
    \begin{array}{ccc}
      \lpquote Q \rpquote & & x \nameeq \lpquote P \rpquote \\
      x & & otherwise \\
    \end{array}
  \right. \nonumber
\end{eqnarray}

and $z$ is chosen distinct from $\quotep{P}$, $\quotep{Q}$, the free
names in $Q$, and all the names in $R$. Our $\alpha$-equivalence will
be built in the standard way from this substitution.

\begin{remark}\label{rem:no_self_referential_names}
  One consequence of these definitions is that $\forall P. \quotep{P}
  \not\in \freenames{P}$.
\end{remark}

\subsection{ Dynamic quote: an example }

Anticipating something of what's to come, consider applying the
substitution, $\widehat{\id{\{}u / z \id{\}}}$, to the following pair
of processes, $\lift{w}{y!(z)}$ and $w[ \lpquote y!(z) \rpquote ]$.

\begin{eqnarray}
	\lift{w}{y!(z)}\widehat{\id{\{}u / z \id{\}}}
		& = &
		\lift{w}{y!(u)} \nonumber\\
	w[ \lpquote y!(z) \rpquote ] \widehat{ \id{\{}u / z \id{\}} }
		& = &
		w[ \lpquote y!(z) \rpquote ] \nonumber
\end{eqnarray}

Because the body of the process between quotes is impervious to
substitution, we get radically different answers. In fact, by
examining the first process in an input context,
e.g. $x?(z).\lift{w}{y!(z)}$, we see that the process under the lift
operator may be shaped by prefixed inputs binding a name inside it. In
this sense, the lift operator will be seen as a way to dynamically
construct processes before reifying them as names.

Finally equipped with these standard features we can present the
dynamics of the calculus.

\subsubsection{Operational semantics} 

Finally, we introduce the computational dynamics. What marks these
algebras as distinct from other more traditionally studied algebraic
structures, e.g. vector spaces or polynomial rings, is the manner in
which dynamics is captured. In traditional structures, dynamics is typically
expressed through morphisms between such structures, as in linear maps
between vector spaces or morphisms between rings. In algebras
associated with the semantics of computation, the dynamics is
expressed as part of the algebraic structure itself, through a
reduction reduction relation typically denoted by $\red$. Below, we
give a recursive presentation of this relation for the calculus used
in the encoding.

$\red \subseteq \pi \times \pi$
$\red : \pi \to \mathcal{P}(\pi)$

\begin{mathpar}
  \inferrule* [lab=Comm] { \textsf{match}( x_{src}, x_{trgt} ) } { x_{trgt}?(y)P \; | \; x_{src}!\langle {Q} \rangle \red P\{\quotep{Q}/y}\} }
  \and \\
  \inferrule* [lab=Par] {{P} \red {P}'} {{{P} | {Q}} \red {{P}' | {Q}}}
  \and
  \inferrule* [lab=Equiv]{{{P} \scong {P}'} \andalso {{P}' \red {Q}'} \andalso {{Q}' \scong {Q}}}{{P} \red {Q}}
\end{mathpar}

\begin{eqnarray*}
  match_{\equiv} (\quotep{P},\quotep{Q}) & := & P \equiv Q \\
  match_{\dagger}(\quotep{P},\quotep{Q}) & := & \forall R. P|Q \red^{*} R => R \red^{*} 0 \\
  match_{K}(\quotep{P},\quotep{Q}) & := & K \mbox{ for some context } K
\end{eqnarray*}

$u?(x)P | u!\langle Q \rangle \red P\{\quotep{Q}/x\}$

%We write $\wred$ for $\red^*$, and $P\red$ if $\exists Q $ such that $ P \red Q$.
We write $P\red$ if $\exists Q $ such that $ P \red Q$ and $P\not\red$, otherwise.

\section{Replication}

As mentioned before, it is known that replication (and hence
recursion) can be implemented in a higher-order process algebra
\cite{SangiorgiWalker}. As our first example of calculation with the
machinery thus far presented we give the construction explicitly in
the {\rhoc}.

\begin{eqnarray}
	D_{x} & := & \prefix{x}{y}{(\binpar{\outputp{x}{y}}{@{y}})} \nonumber\\
	\bangp_{x}{P} & := & \binpar{{x}!\langle{\binpar{D_{x}}{P}}\rangle}{D_{x}} \nonumber
\end{eqnarray}

\begin{eqnarray}
	\bangp_{x}{P} & & \nonumber\\
	=
	& {x}!\langle{(\prefix{x}{y}{(\outputp{x}{y} | @{y})) | P}}\rangle 
	      | \prefix{x}{y}{(\outputp{x}{y} | @{y})} & \nonumber\\
	\red
	& (\outputp{x}{y} | @{y})\substn{\quotep{(\prefix{x}{y}{(@{y} | \outputp{x}{y})) | P}}}{y} & \nonumber\\
	=
	& \outputp{x}{\quotep{(\prefix{x}{y}{(\outputp{x}{y} | @{y})) | P}}}
	  | {(\prefix{x}{y}{(\outputp{x}{y} | @{y})) | P}} & \nonumber\\
	\red
	& \ldots & \nonumber\\
	\red^*
	& P | P | \ldots & \nonumber
\end{eqnarray}

Of course, this encoding, as an implementation, runs away, unfolding
$\bangp{P}$ eagerly. A lazier and more implementable replication
operator, restricted to input-guarded processes, may be obtained as follows.

\begin{eqnarray}
\bangp{\prefix{u}{v}{P}} 
	:= 
	\binpar{\lift{x}{\prefix{u}{v}{(\binpar{D(x)}{P})}}}{D(x)} \nonumber
\end{eqnarray}

\begin{remark}
  Note that the lazier definition still does not deal with summation
  or mixed summation (i.e. sums over input and output). The reader is
  invited to construct definitions of replication that deal with these
  features. 

  Further, the definitions are parameterized in a name, $x$. Can you,
  gentle reader, make a definition that eliminates this parameter and
  guarantees no accidental interaction between the replication
  machinery and the process being replicated -- i.e. no accidental
  sharing of names used by the process to get its work done and the
  name(s) used by the replication to effect copying. This latter
  revision of the definition of replication is crucial to obtaining
  the expected identity $!!P \sim !P$.
\end{remark}

\begin{remark}\label{rem:paradoxical_combinator}
  The reader familiar with the lambda calculus will have noticed the
  similarity between $D$ and the paradoxical combinator.

  [Ed. note: the existence of this seems to suggest we have to be more
  restrictive on the set of processes and names we admit if we are to
  support no-cloning.]
\end{remark}

\subsubsection{Bisimulation}

The computational dynamics gives rise to another kind of equivalence,
the equivalence of computational behavior. As previously mentioned
this is typically captured \emph{via} some form of bisimulation.

% The notion we use in this paper is weak barbed bisimulation
% \cite{milner91polyadicpi}.

The notion we use in this paper is derived from weak barbed
bisimulation \cite{milner91polyadicpi}. 

\begin{definition}
An \emph{observation relation}, $\downarrow_{\mathcal N}$, over a set
of names, $\mathcal N$, is the smallest relation satisfying the rules
below.

\infrule[Out-barb]{y \in {\mathcal N}, \; x \nameeq y}
		  {\outputp{x}{v} \downarrow_{\mathcal N} x}
\infrule[Par-barb]{\mbox{$P\downarrow_{\mathcal N} x$ or $Q\downarrow_{\mathcal N} x$}}
		  {\binpar{P}{Q} \downarrow_{\mathcal N} x}

We write $P \Downarrow_{\mathcal N} x$ if there is $Q$ such that 
$P \wred Q$ and $Q \downarrow_{\mathcal N} x$.
\end{definition}

\begin{definition}
%\label{def.bbisim}
An  ${\mathcal N}$-\emph{barbed bisimulation} over a set of names, ${\mathcal N}$, is a symmetric binary relation 
${\mathcal S}_{\mathcal N}$ between agents such that $P\rel{S}_{\mathcal N}Q$ implies:
\begin{enumerate}
\item If $P \red P'$ then $Q \wred Q'$ and $P'\rel{S}_{\mathcal N} Q'$.
\item If $P\downarrow_{\mathcal N} x$, then $Q\Downarrow_{\mathcal N} x$.
\end{enumerate}
$P$ is ${\mathcal N}$-barbed bisimilar to $Q$, written
$P \wbbisim_{\mathcal N} Q$, if $P \rel{S}_{\mathcal N} Q$ for some ${\mathcal N}$-barbed bisimulation ${\mathcal S}_{\mathcal N}$.
\end{definition}

$\mathcal{R} \subseteq \pi \times \pi$

$P \mathcal{R} Q => \forall P'. P \red P' \Rightarrow \exists Q'. Q \red Q', P' \mathcal{R} Q'$

$P \vdash x \Rightarrow Q \vdash x$

\begin{mathpar}
  \inferrule*[lab=Out-barb]{x \nameeq y}{{y}!\langle{Q}\rangle \vdash x}
  \and
  \inferrule*[lab=Par-barb]{\mbox{$P\vdash x$ or $Q\vdash x$}}{\binpar{P}{Q} \vdash x}
\end{mathpar}

\subsubsection{Contexts}

One of the principle advantages of computational calculi like the
$\pi$-calculus is a well-defined notion of context,
contextual-equivalence and a correlation between
contextual-equivalence and notions of bisimulation. The notion of
context allows the decomposition of a process into (sub-)process and
its syntactic environment, its context. Thus, a context may be
thought of as a process with a ``hole'' (written $\Box$) in it. The
application of a context $M$ to a process $P$, written $M[P]$, is
tantamount to filling the hole in $M$ with $P$. In this paper we do
not need the full weight of this theory, but do make use of the notion
of context in the proof the main theorem. 

\begin{mathpar}
  \inferrule* [lab=summation] {} {{M_{M},M_{N}} \bc \Box \;|\; x.M_{A} \;|\; M_{M}+M_{N}}
  \and
  \inferrule* [lab=agent] {} {{M_{A}} \bc (\vec{x})M_{P} \;| \; \clift{P_0,\ldots,M_{P},\ldots,P_N}}
  \and \\
  \inferrule* [lab=process] {} {{M_{P}} \bc M_{N} \;| \;P|M_{P} }
\end{mathpar} 

\begin{mathpar}
  \inferrule* [lab=sychronization] {} {M_{N} \bc \Box \;|\; x?M_{F} \;|\; x!M_{C}}
  \and
  \inferrule* [lab=abstraction] {} {{M_{F}} \bc (x)M_{P} }
  \and
  \inferrule* [lab=concretion] {} {{M_{C}} \bc \langle M_{P} \rangle }
  \and \\
  \inferrule* [lab=process] {} {{M_{P}} \bc M_{N} \;| \;P|M_{P} }
\end{mathpar}

\begin{definition}[contextual application] Given a context $M$, and
  process $P$, we define the \emph{contextual application}, $M[P] :=
  M\{P/\Box\}$. That is, the contextual application of M to P is the
  substitution of $P$ for $\Box$ in $M$.
\end{definition}

$\meaningof{-} : L \to \mathcal{P}(\pi)$

\begin{mathpar}
  \inferrule* [lab=collection] {} {\meaningof{true} = \pi, \and \meaningof{~E} = \pi \setminus \meaningof{E}, \and \meaningof{E_{1} \& E_{2}} = \meaningof{E_{1}} \cap \meaningof{E_{2}}}
\end{mathpar}

\begin{mathpar}
  \inferrule* [lab=structure] {} {\meaningof{0} = \{ P \in \pi | P \equiv 0 \}, \and \\ \meaningof{E_1 | E_2} = \{ P \in \pi | P \equiv P_{1} | P_{2}, P_{1} \in \meaningof{E_{1}}, P_{2} \in \meaningof{E_2}\} }
\end{mathpar}

\begin{mathpar}
 \inferrule* [lab=behavior] {} {\meaningof{\langle a?b \rangle E} = \{ P \in \pi | P \equiv Q | u?(y)P', \\ \and \\\\ \and \\ \;\;\; u \in \meaningof{a}, \forall z.P'\{z/y\} \in \meaningof{E\{z/b\}}\}, \and \\ \meaningof{a!E} = \{ P \in \pi | P \equiv Q | x!\langle P' \rangle, x \in \meaningof{a} P' \in \meaningof{E}\} }
\end{mathpar}

\begin{mathpar}
 \inferrule* [lab=nominal] {} {\meaningof{\quotep{E}} = \{ \quotep{P} \in \quotep{\pi} | P \in \meaningof{E} \}, \and \meaningof{\quotep{P}} = \{ \quotep{Q} \in \quotep{\pi} | P \equiv Q \} \and \\ \meaningof{@\quotep{E}} = \{ P \in \pi | P \equiv @x, x \in \meaningof{E} \}}
\end{mathpar}

\begin{eqnarray*}
  \\
  \meaningof{-} : TS \to ST
\end{eqnarray*}

\begin{eqnarray*}
  \\
  L : TS \to ST
\end{eqnarray*}

\begin{eqnarray*}
  \\
  P \models E \iff P \in \meaningof{E}
\end{eqnarray*}

\begin{eqnarray*}
  P \approx_{L} Q \iff \forall E \in L. P \models E \iff Q \models E
\end{eqnarray*}

\begin{eqnarray*}
  P \approx_{K} Q
\end{eqnarray*}

\begin{eqnarray*}
  P \approx Q
\end{eqnarray*}

$\approx_{K} = \approx = \approx_{L}$

\subsubsection{Contextual duality}

Note that contexts extend the quotation operation to a family of
operations from processes to names. Given a context, $M$, we can
define a \emph{nominal context}, $\quotep{M}$ by $\quotep{M}[P] :=
\quotep{M[P]}$. To foreshadow what is to come we observe that these
operations enjoy a duality with processes very much like the duality
between vectors and maps from vectors to scalars.

Further, because the calculus is essentially higher-order, we have a
correspondence between contexts and processes. More specifically,
given a name $x$ and a context $M$ we can construct $M^{*}_{x}$ such
that 

\begin{mathpar}
  M^{*}_{x} | \lift{x}{P} \red M[P]
\end{mathpar}

namely,

\begin{mathpar}
  M^{*}_{x} := x?(u).M[\dropn{u}]
\end{mathpar}

The dependence of $M^{*}_{x}$ on a name makes it an abstraction, 

\begin{mathpar}
  M^{*} := (x)x?(u).M[\dropn{u}]
\end{mathpar}

\subsection{Additional notation}

It will sometimes be convenient to denote the process a name
quotes. We already have the notation $x = \quotep{P}$, but it will be
convenient to introduce an alternate notation, $\procn{x}$, when we
want to emphasize the connection to the use of the name. Note that, by
virtue of name equivalence, $\quotep{\procn{x}} \nameeq x$; so, the
notation is consistent with previous definitions.

Further, because names have structure it is possible to effect
substitutions on the basis of that structure. This means we need to
upgrade our notation for substitutions, which we accomplish by
adapting comprehension notation. Thus,

\begin{mathpar}
  P\{ y / x : x \in S \}
\end{mathpar}

is interpreted to mean the process derived from P by replacing (in a
capture-avoiding manner) each occurrence of $x$ in $S$ by $y$. For example,

\begin{mathpar}
  P\{ \quotep{\procn{x}|\procn{x}} / x : x \in \freenames{P} \}
\end{mathpar}

will replace each (occurrence) of a free name $x$ in $P$ by
$\quotep{\procn{x}|\procn{x}}$.

Also, we will avail ourselves of the notation $x^{L}$ and $x^{R}$ to
denote injections of a name into disjoint copies of the name
space. There are numerous ways to accomplish this. One example can be
found in \cite{MeredithR05}. This notation overloads to vectors of
names: $\vec{x}^{\pi} := (x_{i}^{\pi} \; : \; 0 \leq i < |\vec{x}| )$ where $\pi \in \{L,R\}$.

We also use $P^{\Box} := P|\Box$.

In \cite{MeredithR05} an interpretation of the new operator is
given. It turns out that there are several possible interpretations
all enjoying the requisite algebraic properties of the operator (see
\cite{milner91polyadicpi}). We will therefore make liberal use of
$(\nu\; \vec{x})P$.

% subsection the_syntax_and_semantics_of_the_notation_system (end)   

\input{qm2pi.qmops} 

\input{qm2pi.sterngerlach} 

\input{qm2pi.metric} 

% section concurrent_process_calculi (end)

%\input{qm2pi.proofsketch}

% section proof sketch (end)

%\input{qm2pi.slviaknots} 

% section spatial logic via knots (end)

\input{qm2pi.conclusion}

% section conclusion (end)

%\input{qm2pi.dtcodes} 

% section wiring algorithm (end)

\input{qm2pi.ack} 

% section acknowledgments (end)

\newpage


\bibliographystyle{plain}   
\bibliography{../../biblios/main.bib}

\input{qm2pi.rhodetails}

\end{document}

 

%\ifpdf
%\usepackage[pdftex]{graphicx}
%\else
%\usepackage{graphicx}
%\fi

 % \ifpdf
%  \usepackage{pdfsync}
%  \if


%\title{Brief Article}
%\author{David F. Snyder}
%\author{L.G. Meredith}

%\address{Dept. of Math., Texas State University--San Marcos, San Marcos, TX 78666}
       
\pagestyle{empty}


\begin{document}

\lstset{language=[Objective]Caml,frame=shadowbox}

\documentclass[12pt]{llncs}
%\documentclass{jktr}

\usepackage[pdftex]{hyperref}                   
\usepackage {listings}
\usepackage {mathpartir}
\usepackage{bcprules}
%\usepackage{listings}
                       
\usepackage{graphicx} 
%\usepackage[margins=2.5cm,nohead,nofoot]{geometry}
%\usepackage{geometry}
\usepackage{amsfonts}
\usepackage{amstext}
\usepackage{latexsym}
\usepackage{amssymb}
\usepackage{color}


%\include{myPreamble}
\include{qm2pi.local} 

%\ifpdf
%\usepackage[pdftex]{graphicx}
%\else
%\usepackage{graphicx}
%\fi

 % \ifpdf
%  \usepackage{pdfsync}
%  \if


%\title{Brief Article}
%\author{David F. Snyder}
%\author{L.G. Meredith}

%\address{Dept. of Math., Texas State University--San Marcos, San Marcos, TX 78666}
       
\pagestyle{empty}


\begin{document}

\lstset{language=[Objective]Caml,frame=shadowbox}

\input{qm2pi.front}

% section front matter (end)

\input{qm2pi.intro} 
 
% section introduction (end)

% \input{qm2pi.knotations} 

% section notation (end)

\input{qm2pi.process.calculi} 

% section concurrent_process_calculi_and_spatial_logics_ (end)
    
%\input{qm2pi.knots2pi} 

%\input{qm2pi.trefoil} 

%\input{qm2pi.mainthm} 

% subsection basic_interpretation (end)

%\input{qm2pi.rho.presentation} 
\subsection{The syntax and semantics of the notation system}\label{sub:the_syntax_and_semantics_of_the_notation_system} % (fold)

We now summarize a technical presentation of the calculus that
embodies our theory of dynamics. The typical presentation of such a
calculus follows the style of giving generators and relations on
them. The grammar, below, describing term constructors, freely
generates the set of processes, $\Proc$. This set is then quotiented
by a relation known as structural congruence and it is over this set
that the notion of dynamics is expressed. This presentation is
essentially that of \cite{MeredithR05} with the addition of
polyadicity and summation. For readability we have relegated some of
the technical subtleties to an appendix.

\subsubsection{Process grammar}\label{subsub:process_grammar}

\begin{mathpar}
  \inferrule* [lab=synchronization] {} {{M} \bc \pzero \;|\; x?F \;|\; x!C }
  \and
  \inferrule* [lab=abstraction] {} {{F} \bc (x)P}
  \and
  \inferrule* [lab=concretion] {} {{C} \bc \langle Q \rangle}
  \and
  \inferrule* [lab=process] {} {{P,Q} \bc M \;| \;P|Q \;|\; @{x}}
  \and
  \inferrule* [lab=name] {} {{x} \bc \quotep{P}}
\end{mathpar} 

Note that $\vec{x}$ (resp. $\vec{P}$) denotes a vector of names
(resp. processes) of length $|\vec{x}|$ (resp. $|\vec{P}|$). We adopt
the following useful abbreviations.

\begin{mathpar}
   x?(\vec{y}).P := x.(\vec{y})P \and  x\clift{\vec{P}} := x.\clift{\vec{P}}
   \and x!(y) := \lift{x}{\dropn{y}}
   \and \Pi_{i=0}^{n-1}P_i := P_0 | \ldots | P_{n-1}
\end{mathpar}

\subsubsection{Structural congruence}

\paragraph{Free and bound names and alpha-equivalence.} At the
core of structural equivalence is alpha-equivalence which identifies
process that are the same up to a change of variable. Formally, we
recognize the distinction between free and bound names. The free names
of a process, $\freenames{P}$, may be calculated recursively as
follows:

\begin{mathpar}
\freenames{\pzero} := \emptyset
  \and \\
  \freenames{x?(y).P} := \{ x \} \cup (\freenames{P} \setminus \{ y \})
  \and 
  \freenames{x!\langle P \rangle} := \{ x \} \cup \{ P \} 
  \and \\
  \freenames{P|Q} := \freenames{P} \cup \freenames{Q}
  \and \\
  \freenames{@{x}} := \{ x \}
\end{mathpar}

$\pi$
$\quotep{\pi}$

$\freenames{-} : \pi \to \mathcal{P}(\quotep{\pi})$

\begin{eqnarray*}
  \freenames{\pzero} & := & \emptyset \\
  \freenames{x?(y).P} & := & \{ x \} \cup (\freenames{P} \setminus \{ y \}) \\
  \freenames{x!\langle P \rangle} & := & \{ x \} \cup \{ P \} \\
  \freenames{P|Q} & := & \freenames{P} \cup \freenames{Q} \\
  \freenames{\dropn{x}} & := & \{ x \}
\end{eqnarray*}

The bound names of a process, $\boundnames{P}$, are those names occurring in $P$
that are not free. For example, in $x?(y).0$, the name $x$ is free, while $y$ is bound.

\begin{mathpar}
  \inferrule* [lab=monoidal-laws] {} { P|Q \equiv Q|P \and P|0 \equiv P \and P|(Q|R) \equiv (P|Q)|R }
\end{mathpar}

\begin{mathpar}
  \inferrule* [lab=alpha-equivalence] {} { (x)P \equiv (y)P\{y/x\} \and y \not\in \freenames{P} }
\end{mathpar}

\begin{definition}
Then two processes, $P,Q$, are alpha-equivalent if $P = Q\{\vec{y}/\vec{x}\}$ for
some $\vec{x} \in \boundnames{Q},\vec{y} \in \boundnames{P}$, where $Q\{\vec{y}/\vec{x}\}$
denotes the capture-avoiding substitution of $\vec{y}$ for $\vec{x}$ in $Q$.
\end{definition}

\begin{definition}
  The {\em structural congruence} \cite{SangiorgiWalker} , $\equiv$,
  between processes is the least congruence containing
  alpha-equivalence, satisfying the abelian monoid laws
  (associativity, commutativity and $\pzero$ as identity) for parallel
  composition $|$ and for summation $+$.
\end{definition}

\subsection{Name equivalence}

We take name equivalence, written $\nameeq$, to be the smallest
equivalence relation generated by the following rules.

\begin{mathpar}
\inferrule*[lab=Quote-drop]
{ }
{ \quotep{@{x}} \nameeq x }

\inferrule*[lab=Struct-equiv]
{ P \scong Q }
{ \quotep{P} \nameeq \quotep{Q} }
\end{mathpar}

The astute reader will have noticed that the mutual recursion of names
and processes imposes a mutual recursion on alpha-equivalence and
structural equivalence via name-equivalence. Fortunately, all of this
works out pleasantly and we may calculate in the natural way, free of
concern. The reader interested in the details is referred to the
appendix \ref{appendix:rho_details}.

\subsection{Substitution}

We use $\Proc$ for the set of processes, $\QProc$ for the set of
names, and $\id{\{}\vec{y} / \vec{x} \id{\}}$ to denote partial maps,
$s : \QProc \rightarrow \QProc$. A map, $s$ lifts, uniquely, to a map
on process terms, $\widehat{s} : \Proc \rightarrow \Proc$ by the
following equations.

\begin{mathpar}
  (0) \psubstp{Q}{P} := 0 \\
  (R \juxtap S) \psubstp{Q}{P}
  :=    
  (R)\psubstp{Q}{P} \juxtap (S) \psubstp{Q}{P} \\
  (x?(y).R) \psubstp{Q}{P}    
  :=    
  (x)\substp{Q}{P} (z)\concat( (R \psubstn{z}{y}) \psubstp{Q}{P} ) \\
  (\lift{x}{R}) \psubstp{Q}{P}  
  :=
  \lift{(x)\substp{Q}{P}}{ R \psubstp{Q}{P} } \\
%   (\dropn{x})  \psubstp{Q}{P}       
%   := 
%   \left\{ 
%     \begin{array}{ccc} 
%       \dropn{\quotep{Q}} & & x \nameeq \quotep{P} \\
%       \dropn{x} & & otherwise \\
%     \end{array}
%   \right. 
  (\dropn{x})  \psubstp{Q}{P}       
  := 
  \left\{ 
    \begin{array}{ccc} 
      Q & & x \nameeq \quotep{P} \\
      \dropn{x} & & otherwise \\
    \end{array}
  \right.
\end{mathpar}
 

where

\begin{eqnarray}
  (x)\id{\{} \lpquote Q \rpquote / \lpquote P \rpquote \id{\}}            = 
  \left\{ 
    \begin{array}{ccc}
      \lpquote Q \rpquote & & x \nameeq \lpquote P \rpquote \\
      x & & otherwise \\
    \end{array}
  \right. \nonumber
\end{eqnarray}

and $z$ is chosen distinct from $\quotep{P}$, $\quotep{Q}$, the free
names in $Q$, and all the names in $R$. Our $\alpha$-equivalence will
be built in the standard way from this substitution.

\begin{remark}\label{rem:no_self_referential_names}
  One consequence of these definitions is that $\forall P. \quotep{P}
  \not\in \freenames{P}$.
\end{remark}

\subsection{ Dynamic quote: an example }

Anticipating something of what's to come, consider applying the
substitution, $\widehat{\id{\{}u / z \id{\}}}$, to the following pair
of processes, $\lift{w}{y!(z)}$ and $w[ \lpquote y!(z) \rpquote ]$.

\begin{eqnarray}
	\lift{w}{y!(z)}\widehat{\id{\{}u / z \id{\}}}
		& = &
		\lift{w}{y!(u)} \nonumber\\
	w[ \lpquote y!(z) \rpquote ] \widehat{ \id{\{}u / z \id{\}} }
		& = &
		w[ \lpquote y!(z) \rpquote ] \nonumber
\end{eqnarray}

Because the body of the process between quotes is impervious to
substitution, we get radically different answers. In fact, by
examining the first process in an input context,
e.g. $x?(z).\lift{w}{y!(z)}$, we see that the process under the lift
operator may be shaped by prefixed inputs binding a name inside it. In
this sense, the lift operator will be seen as a way to dynamically
construct processes before reifying them as names.

Finally equipped with these standard features we can present the
dynamics of the calculus.

\subsubsection{Operational semantics} 

Finally, we introduce the computational dynamics. What marks these
algebras as distinct from other more traditionally studied algebraic
structures, e.g. vector spaces or polynomial rings, is the manner in
which dynamics is captured. In traditional structures, dynamics is typically
expressed through morphisms between such structures, as in linear maps
between vector spaces or morphisms between rings. In algebras
associated with the semantics of computation, the dynamics is
expressed as part of the algebraic structure itself, through a
reduction reduction relation typically denoted by $\red$. Below, we
give a recursive presentation of this relation for the calculus used
in the encoding.

$\red \subseteq \pi \times \pi$
$\red : \pi \to \mathcal{P}(\pi)$

\begin{mathpar}
  \inferrule* [lab=Comm] { \textsf{match}( x_{src}, x_{trgt} ) } { x_{trgt}?(y)P \; | \; x_{src}!\langle {Q} \rangle \red P\{\quotep{Q}/y}\} }
  \and \\
  \inferrule* [lab=Par] {{P} \red {P}'} {{{P} | {Q}} \red {{P}' | {Q}}}
  \and
  \inferrule* [lab=Equiv]{{{P} \scong {P}'} \andalso {{P}' \red {Q}'} \andalso {{Q}' \scong {Q}}}{{P} \red {Q}}
\end{mathpar}

\begin{eqnarray*}
  match_{\equiv} (\quotep{P},\quotep{Q}) & := & P \equiv Q \\
  match_{\dagger}(\quotep{P},\quotep{Q}) & := & \forall R. P|Q \red^{*} R => R \red^{*} 0 \\
  match_{K}(\quotep{P},\quotep{Q}) & := & K \mbox{ for some context } K
\end{eqnarray*}

$u?(x)P | u!\langle Q \rangle \red P\{\quotep{Q}/x\}$

%We write $\wred$ for $\red^*$, and $P\red$ if $\exists Q $ such that $ P \red Q$.
We write $P\red$ if $\exists Q $ such that $ P \red Q$ and $P\not\red$, otherwise.

\section{Replication}

As mentioned before, it is known that replication (and hence
recursion) can be implemented in a higher-order process algebra
\cite{SangiorgiWalker}. As our first example of calculation with the
machinery thus far presented we give the construction explicitly in
the {\rhoc}.

\begin{eqnarray}
	D_{x} & := & \prefix{x}{y}{(\binpar{\outputp{x}{y}}{@{y}})} \nonumber\\
	\bangp_{x}{P} & := & \binpar{{x}!\langle{\binpar{D_{x}}{P}}\rangle}{D_{x}} \nonumber
\end{eqnarray}

\begin{eqnarray}
	\bangp_{x}{P} & & \nonumber\\
	=
	& {x}!\langle{(\prefix{x}{y}{(\outputp{x}{y} | @{y})) | P}}\rangle 
	      | \prefix{x}{y}{(\outputp{x}{y} | @{y})} & \nonumber\\
	\red
	& (\outputp{x}{y} | @{y})\substn{\quotep{(\prefix{x}{y}{(@{y} | \outputp{x}{y})) | P}}}{y} & \nonumber\\
	=
	& \outputp{x}{\quotep{(\prefix{x}{y}{(\outputp{x}{y} | @{y})) | P}}}
	  | {(\prefix{x}{y}{(\outputp{x}{y} | @{y})) | P}} & \nonumber\\
	\red
	& \ldots & \nonumber\\
	\red^*
	& P | P | \ldots & \nonumber
\end{eqnarray}

Of course, this encoding, as an implementation, runs away, unfolding
$\bangp{P}$ eagerly. A lazier and more implementable replication
operator, restricted to input-guarded processes, may be obtained as follows.

\begin{eqnarray}
\bangp{\prefix{u}{v}{P}} 
	:= 
	\binpar{\lift{x}{\prefix{u}{v}{(\binpar{D(x)}{P})}}}{D(x)} \nonumber
\end{eqnarray}

\begin{remark}
  Note that the lazier definition still does not deal with summation
  or mixed summation (i.e. sums over input and output). The reader is
  invited to construct definitions of replication that deal with these
  features. 

  Further, the definitions are parameterized in a name, $x$. Can you,
  gentle reader, make a definition that eliminates this parameter and
  guarantees no accidental interaction between the replication
  machinery and the process being replicated -- i.e. no accidental
  sharing of names used by the process to get its work done and the
  name(s) used by the replication to effect copying. This latter
  revision of the definition of replication is crucial to obtaining
  the expected identity $!!P \sim !P$.
\end{remark}

\begin{remark}\label{rem:paradoxical_combinator}
  The reader familiar with the lambda calculus will have noticed the
  similarity between $D$ and the paradoxical combinator.

  [Ed. note: the existence of this seems to suggest we have to be more
  restrictive on the set of processes and names we admit if we are to
  support no-cloning.]
\end{remark}

\subsubsection{Bisimulation}

The computational dynamics gives rise to another kind of equivalence,
the equivalence of computational behavior. As previously mentioned
this is typically captured \emph{via} some form of bisimulation.

% The notion we use in this paper is weak barbed bisimulation
% \cite{milner91polyadicpi}.

The notion we use in this paper is derived from weak barbed
bisimulation \cite{milner91polyadicpi}. 

\begin{definition}
An \emph{observation relation}, $\downarrow_{\mathcal N}$, over a set
of names, $\mathcal N$, is the smallest relation satisfying the rules
below.

\infrule[Out-barb]{y \in {\mathcal N}, \; x \nameeq y}
		  {\outputp{x}{v} \downarrow_{\mathcal N} x}
\infrule[Par-barb]{\mbox{$P\downarrow_{\mathcal N} x$ or $Q\downarrow_{\mathcal N} x$}}
		  {\binpar{P}{Q} \downarrow_{\mathcal N} x}

We write $P \Downarrow_{\mathcal N} x$ if there is $Q$ such that 
$P \wred Q$ and $Q \downarrow_{\mathcal N} x$.
\end{definition}

\begin{definition}
%\label{def.bbisim}
An  ${\mathcal N}$-\emph{barbed bisimulation} over a set of names, ${\mathcal N}$, is a symmetric binary relation 
${\mathcal S}_{\mathcal N}$ between agents such that $P\rel{S}_{\mathcal N}Q$ implies:
\begin{enumerate}
\item If $P \red P'$ then $Q \wred Q'$ and $P'\rel{S}_{\mathcal N} Q'$.
\item If $P\downarrow_{\mathcal N} x$, then $Q\Downarrow_{\mathcal N} x$.
\end{enumerate}
$P$ is ${\mathcal N}$-barbed bisimilar to $Q$, written
$P \wbbisim_{\mathcal N} Q$, if $P \rel{S}_{\mathcal N} Q$ for some ${\mathcal N}$-barbed bisimulation ${\mathcal S}_{\mathcal N}$.
\end{definition}

$\mathcal{R} \subseteq \pi \times \pi$

$P \mathcal{R} Q => \forall P'. P \red P' \Rightarrow \exists Q'. Q \red Q', P' \mathcal{R} Q'$

$P \vdash x \Rightarrow Q \vdash x$

\begin{mathpar}
  \inferrule*[lab=Out-barb]{x \nameeq y}{{y}!\langle{Q}\rangle \vdash x}
  \and
  \inferrule*[lab=Par-barb]{\mbox{$P\vdash x$ or $Q\vdash x$}}{\binpar{P}{Q} \vdash x}
\end{mathpar}

\subsubsection{Contexts}

One of the principle advantages of computational calculi like the
$\pi$-calculus is a well-defined notion of context,
contextual-equivalence and a correlation between
contextual-equivalence and notions of bisimulation. The notion of
context allows the decomposition of a process into (sub-)process and
its syntactic environment, its context. Thus, a context may be
thought of as a process with a ``hole'' (written $\Box$) in it. The
application of a context $M$ to a process $P$, written $M[P]$, is
tantamount to filling the hole in $M$ with $P$. In this paper we do
not need the full weight of this theory, but do make use of the notion
of context in the proof the main theorem. 

\begin{mathpar}
  \inferrule* [lab=summation] {} {{M_{M},M_{N}} \bc \Box \;|\; x.M_{A} \;|\; M_{M}+M_{N}}
  \and
  \inferrule* [lab=agent] {} {{M_{A}} \bc (\vec{x})M_{P} \;| \; \clift{P_0,\ldots,M_{P},\ldots,P_N}}
  \and \\
  \inferrule* [lab=process] {} {{M_{P}} \bc M_{N} \;| \;P|M_{P} }
\end{mathpar} 

\begin{mathpar}
  \inferrule* [lab=sychronization] {} {M_{N} \bc \Box \;|\; x?M_{F} \;|\; x!M_{C}}
  \and
  \inferrule* [lab=abstraction] {} {{M_{F}} \bc (x)M_{P} }
  \and
  \inferrule* [lab=concretion] {} {{M_{C}} \bc \langle M_{P} \rangle }
  \and \\
  \inferrule* [lab=process] {} {{M_{P}} \bc M_{N} \;| \;P|M_{P} }
\end{mathpar}

\begin{definition}[contextual application] Given a context $M$, and
  process $P$, we define the \emph{contextual application}, $M[P] :=
  M\{P/\Box\}$. That is, the contextual application of M to P is the
  substitution of $P$ for $\Box$ in $M$.
\end{definition}

$\meaningof{-} : L \to \mathcal{P}(\pi)$

\begin{mathpar}
  \inferrule* [lab=collection] {} {\meaningof{true} = \pi, \and \meaningof{~E} = \pi \setminus \meaningof{E}, \and \meaningof{E_{1} \& E_{2}} = \meaningof{E_{1}} \cap \meaningof{E_{2}}}
\end{mathpar}

\begin{mathpar}
  \inferrule* [lab=structure] {} {\meaningof{0} = \{ P \in \pi | P \equiv 0 \}, \and \\ \meaningof{E_1 | E_2} = \{ P \in \pi | P \equiv P_{1} | P_{2}, P_{1} \in \meaningof{E_{1}}, P_{2} \in \meaningof{E_2}\} }
\end{mathpar}

\begin{mathpar}
 \inferrule* [lab=behavior] {} {\meaningof{\langle a?b \rangle E} = \{ P \in \pi | P \equiv Q | u?(y)P', \\ \and \\\\ \and \\ \;\;\; u \in \meaningof{a}, \forall z.P'\{z/y\} \in \meaningof{E\{z/b\}}\}, \and \\ \meaningof{a!E} = \{ P \in \pi | P \equiv Q | x!\langle P' \rangle, x \in \meaningof{a} P' \in \meaningof{E}\} }
\end{mathpar}

\begin{mathpar}
 \inferrule* [lab=nominal] {} {\meaningof{\quotep{E}} = \{ \quotep{P} \in \quotep{\pi} | P \in \meaningof{E} \}, \and \meaningof{\quotep{P}} = \{ \quotep{Q} \in \quotep{\pi} | P \equiv Q \} \and \\ \meaningof{@\quotep{E}} = \{ P \in \pi | P \equiv @x, x \in \meaningof{E} \}}
\end{mathpar}

\begin{eqnarray*}
  \\
  \meaningof{-} : TS \to ST
\end{eqnarray*}

\begin{eqnarray*}
  \\
  L : TS \to ST
\end{eqnarray*}

\begin{eqnarray*}
  \\
  P \models E \iff P \in \meaningof{E}
\end{eqnarray*}

\begin{eqnarray*}
  P \approx_{L} Q \iff \forall E \in L. P \models E \iff Q \models E
\end{eqnarray*}

\begin{eqnarray*}
  P \approx_{K} Q
\end{eqnarray*}

\begin{eqnarray*}
  P \approx Q
\end{eqnarray*}

$\approx_{K} = \approx = \approx_{L}$

\subsubsection{Contextual duality}

Note that contexts extend the quotation operation to a family of
operations from processes to names. Given a context, $M$, we can
define a \emph{nominal context}, $\quotep{M}$ by $\quotep{M}[P] :=
\quotep{M[P]}$. To foreshadow what is to come we observe that these
operations enjoy a duality with processes very much like the duality
between vectors and maps from vectors to scalars.

Further, because the calculus is essentially higher-order, we have a
correspondence between contexts and processes. More specifically,
given a name $x$ and a context $M$ we can construct $M^{*}_{x}$ such
that 

\begin{mathpar}
  M^{*}_{x} | \lift{x}{P} \red M[P]
\end{mathpar}

namely,

\begin{mathpar}
  M^{*}_{x} := x?(u).M[\dropn{u}]
\end{mathpar}

The dependence of $M^{*}_{x}$ on a name makes it an abstraction, 

\begin{mathpar}
  M^{*} := (x)x?(u).M[\dropn{u}]
\end{mathpar}

\subsection{Additional notation}

It will sometimes be convenient to denote the process a name
quotes. We already have the notation $x = \quotep{P}$, but it will be
convenient to introduce an alternate notation, $\procn{x}$, when we
want to emphasize the connection to the use of the name. Note that, by
virtue of name equivalence, $\quotep{\procn{x}} \nameeq x$; so, the
notation is consistent with previous definitions.

Further, because names have structure it is possible to effect
substitutions on the basis of that structure. This means we need to
upgrade our notation for substitutions, which we accomplish by
adapting comprehension notation. Thus,

\begin{mathpar}
  P\{ y / x : x \in S \}
\end{mathpar}

is interpreted to mean the process derived from P by replacing (in a
capture-avoiding manner) each occurrence of $x$ in $S$ by $y$. For example,

\begin{mathpar}
  P\{ \quotep{\procn{x}|\procn{x}} / x : x \in \freenames{P} \}
\end{mathpar}

will replace each (occurrence) of a free name $x$ in $P$ by
$\quotep{\procn{x}|\procn{x}}$.

Also, we will avail ourselves of the notation $x^{L}$ and $x^{R}$ to
denote injections of a name into disjoint copies of the name
space. There are numerous ways to accomplish this. One example can be
found in \cite{MeredithR05}. This notation overloads to vectors of
names: $\vec{x}^{\pi} := (x_{i}^{\pi} \; : \; 0 \leq i < |\vec{x}| )$ where $\pi \in \{L,R\}$.

We also use $P^{\Box} := P|\Box$.

In \cite{MeredithR05} an interpretation of the new operator is
given. It turns out that there are several possible interpretations
all enjoying the requisite algebraic properties of the operator (see
\cite{milner91polyadicpi}). We will therefore make liberal use of
$(\nu\; \vec{x})P$.

% subsection the_syntax_and_semantics_of_the_notation_system (end)   

\input{qm2pi.qmops} 

\input{qm2pi.sterngerlach} 

\input{qm2pi.metric} 

% section concurrent_process_calculi (end)

%\input{qm2pi.proofsketch}

% section proof sketch (end)

%\input{qm2pi.slviaknots} 

% section spatial logic via knots (end)

\input{qm2pi.conclusion}

% section conclusion (end)

%\input{qm2pi.dtcodes} 

% section wiring algorithm (end)

\input{qm2pi.ack} 

% section acknowledgments (end)

\newpage


\bibliographystyle{plain}   
\bibliography{../../biblios/main.bib}

\input{qm2pi.rhodetails}

\end{document}



% section front matter (end)

\section{Introduction}\label{sec:introduction} % (fold)
In this draft of the material i am going to have to dispense with the
usual writing conventions adopted in papers on these topics. i'm going
to have adopt whatever tone i need at the time i'm writing up the
calculations. Sometimes this may be very conversational; others it may
be the barest mathematical grunts; others still it may be that i have
lifted text from one of my other papers because the exposition of some
point was better said there. i hope that my readers are not unduly put
out by this decision. i'm not doing this to flout convention or be
rebellious. i find these calculations very technically challenging. To
keep everything going technically, something has to give; i have to
let go of some cognitive burden. So, the academic writing style --
with all of its trade-offs in terms of facilitating technical
communication -- is what i'm letting go of. Perhaps subsequent drafts
can be tightened and polished, but for now, i'm going to speak as if
we were sitting together in a coffee shop with a laptop, wifi and a
pad of paper and a pencil.

So, here's what i have to say. We -- you and i, comfortably ensconced
in our coffee shop and well-equipped with our tools -- can realize and
carry out the calculations of quantum mechanics over a very different
formal theory of dynamics, a formal theory of dynamics that
corresponds to a theory of concurrent computation with
\emph{reflection}. It has the advantage that the underlying theory is
already `quantized', but supports analogues all of the continuuous
operations. Strikingly, this underlying theory has recently been
connected with a notion of metric that we can show, by calculating
together, coincides with the metric induced by the inner product.

There are a lot of reasons why you might be interested in seeing
calculations of this form. Here's why i'm interested. For the past
several centuries there has been no competitor to the ``Newtonian''
account of dynamics. As a result the predominant share of accounts of
dynamical systems and situations have had to be formulated in terms of
the Newtonian machinery. i view this as an intellectually dangerous
position to occupy. Everything, despite it's intrinsic shape, turns
into a nail to be hit with this hammer. Recently, however, the theory
of computation has matured to the point where we have candidates for
theories of dynamics that offer very different perspective on
reasoning about dynamical systems and situations. Testing these
candidates against very successful accounts of dynamical situations,
like quantum mechanics, is going to give us some sense of how mature
they are and some measure of the quality of these accounts of
dynamics.

\subsection{Summary of contributions and outline of paper}

So, we're going to develop an interpretation of the operations of
quantum mechanics normally interpreted by Hilbert spaces and
operators. We're going to do this over a theory of computation. Note
that this is very different than the usual quantum computation program
which develops notions of computation over quantum mechanics. Rather,
we are developing a story that aligns with Wheeler's slogan: It from
Bit. To do this we will first provide an account of the theory of
computation at play here. Then we will dive into a calculation-driven
interpretation of the operations of quantum mechanics.

The reason we take this approach is that -- until very recently --
there hasn't been an axiomatic account of quantum mechanics. As a
result there has been no sharp delineation of the mathematical theory
supporting interpretation of the physical theory and the physical
theory, itself. So, ambient features of the maths are free to be
exploited (or supressed) without a real accounting of their physical
relevance. There is no sharp statement ``here's the physical theory''
qua \emph{theory} and ``here's the mathematical interpretation''
enabling a judgment of how faithful the interpretation is -- apart
from experimental observation. When there is an axiomatic account we
can judge how well a given mathematical formalism supports an
interpretation of the axioms, independent of
experimentation. Likewise, we can judge how well we have captured our
physical evidence and experience with our axiomatics, independent of
any specific mathematical implementation, with accidental detail that
may or may not have physical significance. 

In lieu of a fully fleshed out and vetted axiomatic account of quantum
mechanics, interpreting the operational notions in service of modeling
physical systems will have to suffice. In other words, we are not in
the business of providing a model of Hilbert spaces and operators. We
are in the business of providing a model of quantum mechanics because
we are motivated by testing our notions of dynamics against physical
theory; and, the predictive calculations of the physical theory must
serve as the best formulation -- shy of a fully fleshed out axiomatic
account -- of the physical theory itself (as they have for scientific
theories since time immemorial). Put another way, despite a
whole-hearted commitment to an It-from-Bit ontology, we are firmly
aligned with the shut-up-and-calculate camp as the best way to obtain
results either from the physical perspective or as a quality assurance
measure of our fledgling theory of dynamics.

In detail, we present a reflective process calculus. Then we develop
intuitive correspondences between the notions available in this
calculus and the usual physical notions supporting quantum mechanical
calculations. Thus, 

\begin{table}[htp]
  \center{
    \fbox{
      \begin{tabular}{c|c}
        quantum mechanics & process calculus \\
        \hline
        scalar & name \\
        state vector & process \\
        dual & contextual duals \\
        matrix & formal sums of process-context-dual pairs \\
        orthogonality & process annihilation \\
        inner product & execution-formula + quoting
      \end{tabular}
    }
  }
  \caption{QM - process calculi correspondences}
\end{table}

Then we tighten up these intuitions to operational definitions. We
employ the Dirac notation as the best proxy we can find for an
abstract syntax of the quantum mechanical notions. The definitions we
develop put us in contact with equational constraints coming from the
theory that we demonstrate the definitions and calculations satisfy.

This puts us in a position to shut up and calculate for the
Stern-Gerlach experimental set up, showing how these predictive
calculations become calculations on processes in our theory of a
reflective process calculus.

Penultimately, we demonstrate that the notion of metric coming from
the inner product coincides with the notion of metric available from
the theory of bisimulation. This demonstration gives us the right to
think of space as arising from behavior. Finally, we consider where we
might go from the new vantage point we have obtained.

% section introduction (end) 
 
% section introduction (end)

% \documentclass[12pt]{llncs}
%\documentclass{jktr}

\usepackage[pdftex]{hyperref}                   
\usepackage {listings}
\usepackage {mathpartir}
\usepackage{bcprules}
%\usepackage{listings}
                       
\usepackage{graphicx} 
%\usepackage[margins=2.5cm,nohead,nofoot]{geometry}
%\usepackage{geometry}
\usepackage{amsfonts}
\usepackage{amstext}
\usepackage{latexsym}
\usepackage{amssymb}
\usepackage{color}


%\include{myPreamble}
\include{qm2pi.local} 

%\ifpdf
%\usepackage[pdftex]{graphicx}
%\else
%\usepackage{graphicx}
%\fi

 % \ifpdf
%  \usepackage{pdfsync}
%  \if


%\title{Brief Article}
%\author{David F. Snyder}
%\author{L.G. Meredith}

%\address{Dept. of Math., Texas State University--San Marcos, San Marcos, TX 78666}
       
\pagestyle{empty}


\begin{document}

\lstset{language=[Objective]Caml,frame=shadowbox}

\input{qm2pi.front}

% section front matter (end)

\input{qm2pi.intro} 
 
% section introduction (end)

% \input{qm2pi.knotations} 

% section notation (end)

\input{qm2pi.process.calculi} 

% section concurrent_process_calculi_and_spatial_logics_ (end)
    
%\input{qm2pi.knots2pi} 

%\input{qm2pi.trefoil} 

%\input{qm2pi.mainthm} 

% subsection basic_interpretation (end)

%\input{qm2pi.rho.presentation} 
\subsection{The syntax and semantics of the notation system}\label{sub:the_syntax_and_semantics_of_the_notation_system} % (fold)

We now summarize a technical presentation of the calculus that
embodies our theory of dynamics. The typical presentation of such a
calculus follows the style of giving generators and relations on
them. The grammar, below, describing term constructors, freely
generates the set of processes, $\Proc$. This set is then quotiented
by a relation known as structural congruence and it is over this set
that the notion of dynamics is expressed. This presentation is
essentially that of \cite{MeredithR05} with the addition of
polyadicity and summation. For readability we have relegated some of
the technical subtleties to an appendix.

\subsubsection{Process grammar}\label{subsub:process_grammar}

\begin{mathpar}
  \inferrule* [lab=synchronization] {} {{M} \bc \pzero \;|\; x?F \;|\; x!C }
  \and
  \inferrule* [lab=abstraction] {} {{F} \bc (x)P}
  \and
  \inferrule* [lab=concretion] {} {{C} \bc \langle Q \rangle}
  \and
  \inferrule* [lab=process] {} {{P,Q} \bc M \;| \;P|Q \;|\; @{x}}
  \and
  \inferrule* [lab=name] {} {{x} \bc \quotep{P}}
\end{mathpar} 

Note that $\vec{x}$ (resp. $\vec{P}$) denotes a vector of names
(resp. processes) of length $|\vec{x}|$ (resp. $|\vec{P}|$). We adopt
the following useful abbreviations.

\begin{mathpar}
   x?(\vec{y}).P := x.(\vec{y})P \and  x\clift{\vec{P}} := x.\clift{\vec{P}}
   \and x!(y) := \lift{x}{\dropn{y}}
   \and \Pi_{i=0}^{n-1}P_i := P_0 | \ldots | P_{n-1}
\end{mathpar}

\subsubsection{Structural congruence}

\paragraph{Free and bound names and alpha-equivalence.} At the
core of structural equivalence is alpha-equivalence which identifies
process that are the same up to a change of variable. Formally, we
recognize the distinction between free and bound names. The free names
of a process, $\freenames{P}$, may be calculated recursively as
follows:

\begin{mathpar}
\freenames{\pzero} := \emptyset
  \and \\
  \freenames{x?(y).P} := \{ x \} \cup (\freenames{P} \setminus \{ y \})
  \and 
  \freenames{x!\langle P \rangle} := \{ x \} \cup \{ P \} 
  \and \\
  \freenames{P|Q} := \freenames{P} \cup \freenames{Q}
  \and \\
  \freenames{@{x}} := \{ x \}
\end{mathpar}

$\pi$
$\quotep{\pi}$

$\freenames{-} : \pi \to \mathcal{P}(\quotep{\pi})$

\begin{eqnarray*}
  \freenames{\pzero} & := & \emptyset \\
  \freenames{x?(y).P} & := & \{ x \} \cup (\freenames{P} \setminus \{ y \}) \\
  \freenames{x!\langle P \rangle} & := & \{ x \} \cup \{ P \} \\
  \freenames{P|Q} & := & \freenames{P} \cup \freenames{Q} \\
  \freenames{\dropn{x}} & := & \{ x \}
\end{eqnarray*}

The bound names of a process, $\boundnames{P}$, are those names occurring in $P$
that are not free. For example, in $x?(y).0$, the name $x$ is free, while $y$ is bound.

\begin{mathpar}
  \inferrule* [lab=monoidal-laws] {} { P|Q \equiv Q|P \and P|0 \equiv P \and P|(Q|R) \equiv (P|Q)|R }
\end{mathpar}

\begin{mathpar}
  \inferrule* [lab=alpha-equivalence] {} { (x)P \equiv (y)P\{y/x\} \and y \not\in \freenames{P} }
\end{mathpar}

\begin{definition}
Then two processes, $P,Q$, are alpha-equivalent if $P = Q\{\vec{y}/\vec{x}\}$ for
some $\vec{x} \in \boundnames{Q},\vec{y} \in \boundnames{P}$, where $Q\{\vec{y}/\vec{x}\}$
denotes the capture-avoiding substitution of $\vec{y}$ for $\vec{x}$ in $Q$.
\end{definition}

\begin{definition}
  The {\em structural congruence} \cite{SangiorgiWalker} , $\equiv$,
  between processes is the least congruence containing
  alpha-equivalence, satisfying the abelian monoid laws
  (associativity, commutativity and $\pzero$ as identity) for parallel
  composition $|$ and for summation $+$.
\end{definition}

\subsection{Name equivalence}

We take name equivalence, written $\nameeq$, to be the smallest
equivalence relation generated by the following rules.

\begin{mathpar}
\inferrule*[lab=Quote-drop]
{ }
{ \quotep{@{x}} \nameeq x }

\inferrule*[lab=Struct-equiv]
{ P \scong Q }
{ \quotep{P} \nameeq \quotep{Q} }
\end{mathpar}

The astute reader will have noticed that the mutual recursion of names
and processes imposes a mutual recursion on alpha-equivalence and
structural equivalence via name-equivalence. Fortunately, all of this
works out pleasantly and we may calculate in the natural way, free of
concern. The reader interested in the details is referred to the
appendix \ref{appendix:rho_details}.

\subsection{Substitution}

We use $\Proc$ for the set of processes, $\QProc$ for the set of
names, and $\id{\{}\vec{y} / \vec{x} \id{\}}$ to denote partial maps,
$s : \QProc \rightarrow \QProc$. A map, $s$ lifts, uniquely, to a map
on process terms, $\widehat{s} : \Proc \rightarrow \Proc$ by the
following equations.

\begin{mathpar}
  (0) \psubstp{Q}{P} := 0 \\
  (R \juxtap S) \psubstp{Q}{P}
  :=    
  (R)\psubstp{Q}{P} \juxtap (S) \psubstp{Q}{P} \\
  (x?(y).R) \psubstp{Q}{P}    
  :=    
  (x)\substp{Q}{P} (z)\concat( (R \psubstn{z}{y}) \psubstp{Q}{P} ) \\
  (\lift{x}{R}) \psubstp{Q}{P}  
  :=
  \lift{(x)\substp{Q}{P}}{ R \psubstp{Q}{P} } \\
%   (\dropn{x})  \psubstp{Q}{P}       
%   := 
%   \left\{ 
%     \begin{array}{ccc} 
%       \dropn{\quotep{Q}} & & x \nameeq \quotep{P} \\
%       \dropn{x} & & otherwise \\
%     \end{array}
%   \right. 
  (\dropn{x})  \psubstp{Q}{P}       
  := 
  \left\{ 
    \begin{array}{ccc} 
      Q & & x \nameeq \quotep{P} \\
      \dropn{x} & & otherwise \\
    \end{array}
  \right.
\end{mathpar}
 

where

\begin{eqnarray}
  (x)\id{\{} \lpquote Q \rpquote / \lpquote P \rpquote \id{\}}            = 
  \left\{ 
    \begin{array}{ccc}
      \lpquote Q \rpquote & & x \nameeq \lpquote P \rpquote \\
      x & & otherwise \\
    \end{array}
  \right. \nonumber
\end{eqnarray}

and $z$ is chosen distinct from $\quotep{P}$, $\quotep{Q}$, the free
names in $Q$, and all the names in $R$. Our $\alpha$-equivalence will
be built in the standard way from this substitution.

\begin{remark}\label{rem:no_self_referential_names}
  One consequence of these definitions is that $\forall P. \quotep{P}
  \not\in \freenames{P}$.
\end{remark}

\subsection{ Dynamic quote: an example }

Anticipating something of what's to come, consider applying the
substitution, $\widehat{\id{\{}u / z \id{\}}}$, to the following pair
of processes, $\lift{w}{y!(z)}$ and $w[ \lpquote y!(z) \rpquote ]$.

\begin{eqnarray}
	\lift{w}{y!(z)}\widehat{\id{\{}u / z \id{\}}}
		& = &
		\lift{w}{y!(u)} \nonumber\\
	w[ \lpquote y!(z) \rpquote ] \widehat{ \id{\{}u / z \id{\}} }
		& = &
		w[ \lpquote y!(z) \rpquote ] \nonumber
\end{eqnarray}

Because the body of the process between quotes is impervious to
substitution, we get radically different answers. In fact, by
examining the first process in an input context,
e.g. $x?(z).\lift{w}{y!(z)}$, we see that the process under the lift
operator may be shaped by prefixed inputs binding a name inside it. In
this sense, the lift operator will be seen as a way to dynamically
construct processes before reifying them as names.

Finally equipped with these standard features we can present the
dynamics of the calculus.

\subsubsection{Operational semantics} 

Finally, we introduce the computational dynamics. What marks these
algebras as distinct from other more traditionally studied algebraic
structures, e.g. vector spaces or polynomial rings, is the manner in
which dynamics is captured. In traditional structures, dynamics is typically
expressed through morphisms between such structures, as in linear maps
between vector spaces or morphisms between rings. In algebras
associated with the semantics of computation, the dynamics is
expressed as part of the algebraic structure itself, through a
reduction reduction relation typically denoted by $\red$. Below, we
give a recursive presentation of this relation for the calculus used
in the encoding.

$\red \subseteq \pi \times \pi$
$\red : \pi \to \mathcal{P}(\pi)$

\begin{mathpar}
  \inferrule* [lab=Comm] { \textsf{match}( x_{src}, x_{trgt} ) } { x_{trgt}?(y)P \; | \; x_{src}!\langle {Q} \rangle \red P\{\quotep{Q}/y}\} }
  \and \\
  \inferrule* [lab=Par] {{P} \red {P}'} {{{P} | {Q}} \red {{P}' | {Q}}}
  \and
  \inferrule* [lab=Equiv]{{{P} \scong {P}'} \andalso {{P}' \red {Q}'} \andalso {{Q}' \scong {Q}}}{{P} \red {Q}}
\end{mathpar}

\begin{eqnarray*}
  match_{\equiv} (\quotep{P},\quotep{Q}) & := & P \equiv Q \\
  match_{\dagger}(\quotep{P},\quotep{Q}) & := & \forall R. P|Q \red^{*} R => R \red^{*} 0 \\
  match_{K}(\quotep{P},\quotep{Q}) & := & K \mbox{ for some context } K
\end{eqnarray*}

$u?(x)P | u!\langle Q \rangle \red P\{\quotep{Q}/x\}$

%We write $\wred$ for $\red^*$, and $P\red$ if $\exists Q $ such that $ P \red Q$.
We write $P\red$ if $\exists Q $ such that $ P \red Q$ and $P\not\red$, otherwise.

\section{Replication}

As mentioned before, it is known that replication (and hence
recursion) can be implemented in a higher-order process algebra
\cite{SangiorgiWalker}. As our first example of calculation with the
machinery thus far presented we give the construction explicitly in
the {\rhoc}.

\begin{eqnarray}
	D_{x} & := & \prefix{x}{y}{(\binpar{\outputp{x}{y}}{@{y}})} \nonumber\\
	\bangp_{x}{P} & := & \binpar{{x}!\langle{\binpar{D_{x}}{P}}\rangle}{D_{x}} \nonumber
\end{eqnarray}

\begin{eqnarray}
	\bangp_{x}{P} & & \nonumber\\
	=
	& {x}!\langle{(\prefix{x}{y}{(\outputp{x}{y} | @{y})) | P}}\rangle 
	      | \prefix{x}{y}{(\outputp{x}{y} | @{y})} & \nonumber\\
	\red
	& (\outputp{x}{y} | @{y})\substn{\quotep{(\prefix{x}{y}{(@{y} | \outputp{x}{y})) | P}}}{y} & \nonumber\\
	=
	& \outputp{x}{\quotep{(\prefix{x}{y}{(\outputp{x}{y} | @{y})) | P}}}
	  | {(\prefix{x}{y}{(\outputp{x}{y} | @{y})) | P}} & \nonumber\\
	\red
	& \ldots & \nonumber\\
	\red^*
	& P | P | \ldots & \nonumber
\end{eqnarray}

Of course, this encoding, as an implementation, runs away, unfolding
$\bangp{P}$ eagerly. A lazier and more implementable replication
operator, restricted to input-guarded processes, may be obtained as follows.

\begin{eqnarray}
\bangp{\prefix{u}{v}{P}} 
	:= 
	\binpar{\lift{x}{\prefix{u}{v}{(\binpar{D(x)}{P})}}}{D(x)} \nonumber
\end{eqnarray}

\begin{remark}
  Note that the lazier definition still does not deal with summation
  or mixed summation (i.e. sums over input and output). The reader is
  invited to construct definitions of replication that deal with these
  features. 

  Further, the definitions are parameterized in a name, $x$. Can you,
  gentle reader, make a definition that eliminates this parameter and
  guarantees no accidental interaction between the replication
  machinery and the process being replicated -- i.e. no accidental
  sharing of names used by the process to get its work done and the
  name(s) used by the replication to effect copying. This latter
  revision of the definition of replication is crucial to obtaining
  the expected identity $!!P \sim !P$.
\end{remark}

\begin{remark}\label{rem:paradoxical_combinator}
  The reader familiar with the lambda calculus will have noticed the
  similarity between $D$ and the paradoxical combinator.

  [Ed. note: the existence of this seems to suggest we have to be more
  restrictive on the set of processes and names we admit if we are to
  support no-cloning.]
\end{remark}

\subsubsection{Bisimulation}

The computational dynamics gives rise to another kind of equivalence,
the equivalence of computational behavior. As previously mentioned
this is typically captured \emph{via} some form of bisimulation.

% The notion we use in this paper is weak barbed bisimulation
% \cite{milner91polyadicpi}.

The notion we use in this paper is derived from weak barbed
bisimulation \cite{milner91polyadicpi}. 

\begin{definition}
An \emph{observation relation}, $\downarrow_{\mathcal N}$, over a set
of names, $\mathcal N$, is the smallest relation satisfying the rules
below.

\infrule[Out-barb]{y \in {\mathcal N}, \; x \nameeq y}
		  {\outputp{x}{v} \downarrow_{\mathcal N} x}
\infrule[Par-barb]{\mbox{$P\downarrow_{\mathcal N} x$ or $Q\downarrow_{\mathcal N} x$}}
		  {\binpar{P}{Q} \downarrow_{\mathcal N} x}

We write $P \Downarrow_{\mathcal N} x$ if there is $Q$ such that 
$P \wred Q$ and $Q \downarrow_{\mathcal N} x$.
\end{definition}

\begin{definition}
%\label{def.bbisim}
An  ${\mathcal N}$-\emph{barbed bisimulation} over a set of names, ${\mathcal N}$, is a symmetric binary relation 
${\mathcal S}_{\mathcal N}$ between agents such that $P\rel{S}_{\mathcal N}Q$ implies:
\begin{enumerate}
\item If $P \red P'$ then $Q \wred Q'$ and $P'\rel{S}_{\mathcal N} Q'$.
\item If $P\downarrow_{\mathcal N} x$, then $Q\Downarrow_{\mathcal N} x$.
\end{enumerate}
$P$ is ${\mathcal N}$-barbed bisimilar to $Q$, written
$P \wbbisim_{\mathcal N} Q$, if $P \rel{S}_{\mathcal N} Q$ for some ${\mathcal N}$-barbed bisimulation ${\mathcal S}_{\mathcal N}$.
\end{definition}

$\mathcal{R} \subseteq \pi \times \pi$

$P \mathcal{R} Q => \forall P'. P \red P' \Rightarrow \exists Q'. Q \red Q', P' \mathcal{R} Q'$

$P \vdash x \Rightarrow Q \vdash x$

\begin{mathpar}
  \inferrule*[lab=Out-barb]{x \nameeq y}{{y}!\langle{Q}\rangle \vdash x}
  \and
  \inferrule*[lab=Par-barb]{\mbox{$P\vdash x$ or $Q\vdash x$}}{\binpar{P}{Q} \vdash x}
\end{mathpar}

\subsubsection{Contexts}

One of the principle advantages of computational calculi like the
$\pi$-calculus is a well-defined notion of context,
contextual-equivalence and a correlation between
contextual-equivalence and notions of bisimulation. The notion of
context allows the decomposition of a process into (sub-)process and
its syntactic environment, its context. Thus, a context may be
thought of as a process with a ``hole'' (written $\Box$) in it. The
application of a context $M$ to a process $P$, written $M[P]$, is
tantamount to filling the hole in $M$ with $P$. In this paper we do
not need the full weight of this theory, but do make use of the notion
of context in the proof the main theorem. 

\begin{mathpar}
  \inferrule* [lab=summation] {} {{M_{M},M_{N}} \bc \Box \;|\; x.M_{A} \;|\; M_{M}+M_{N}}
  \and
  \inferrule* [lab=agent] {} {{M_{A}} \bc (\vec{x})M_{P} \;| \; \clift{P_0,\ldots,M_{P},\ldots,P_N}}
  \and \\
  \inferrule* [lab=process] {} {{M_{P}} \bc M_{N} \;| \;P|M_{P} }
\end{mathpar} 

\begin{mathpar}
  \inferrule* [lab=sychronization] {} {M_{N} \bc \Box \;|\; x?M_{F} \;|\; x!M_{C}}
  \and
  \inferrule* [lab=abstraction] {} {{M_{F}} \bc (x)M_{P} }
  \and
  \inferrule* [lab=concretion] {} {{M_{C}} \bc \langle M_{P} \rangle }
  \and \\
  \inferrule* [lab=process] {} {{M_{P}} \bc M_{N} \;| \;P|M_{P} }
\end{mathpar}

\begin{definition}[contextual application] Given a context $M$, and
  process $P$, we define the \emph{contextual application}, $M[P] :=
  M\{P/\Box\}$. That is, the contextual application of M to P is the
  substitution of $P$ for $\Box$ in $M$.
\end{definition}

$\meaningof{-} : L \to \mathcal{P}(\pi)$

\begin{mathpar}
  \inferrule* [lab=collection] {} {\meaningof{true} = \pi, \and \meaningof{~E} = \pi \setminus \meaningof{E}, \and \meaningof{E_{1} \& E_{2}} = \meaningof{E_{1}} \cap \meaningof{E_{2}}}
\end{mathpar}

\begin{mathpar}
  \inferrule* [lab=structure] {} {\meaningof{0} = \{ P \in \pi | P \equiv 0 \}, \and \\ \meaningof{E_1 | E_2} = \{ P \in \pi | P \equiv P_{1} | P_{2}, P_{1} \in \meaningof{E_{1}}, P_{2} \in \meaningof{E_2}\} }
\end{mathpar}

\begin{mathpar}
 \inferrule* [lab=behavior] {} {\meaningof{\langle a?b \rangle E} = \{ P \in \pi | P \equiv Q | u?(y)P', \\ \and \\\\ \and \\ \;\;\; u \in \meaningof{a}, \forall z.P'\{z/y\} \in \meaningof{E\{z/b\}}\}, \and \\ \meaningof{a!E} = \{ P \in \pi | P \equiv Q | x!\langle P' \rangle, x \in \meaningof{a} P' \in \meaningof{E}\} }
\end{mathpar}

\begin{mathpar}
 \inferrule* [lab=nominal] {} {\meaningof{\quotep{E}} = \{ \quotep{P} \in \quotep{\pi} | P \in \meaningof{E} \}, \and \meaningof{\quotep{P}} = \{ \quotep{Q} \in \quotep{\pi} | P \equiv Q \} \and \\ \meaningof{@\quotep{E}} = \{ P \in \pi | P \equiv @x, x \in \meaningof{E} \}}
\end{mathpar}

\begin{eqnarray*}
  \\
  \meaningof{-} : TS \to ST
\end{eqnarray*}

\begin{eqnarray*}
  \\
  L : TS \to ST
\end{eqnarray*}

\begin{eqnarray*}
  \\
  P \models E \iff P \in \meaningof{E}
\end{eqnarray*}

\begin{eqnarray*}
  P \approx_{L} Q \iff \forall E \in L. P \models E \iff Q \models E
\end{eqnarray*}

\begin{eqnarray*}
  P \approx_{K} Q
\end{eqnarray*}

\begin{eqnarray*}
  P \approx Q
\end{eqnarray*}

$\approx_{K} = \approx = \approx_{L}$

\subsubsection{Contextual duality}

Note that contexts extend the quotation operation to a family of
operations from processes to names. Given a context, $M$, we can
define a \emph{nominal context}, $\quotep{M}$ by $\quotep{M}[P] :=
\quotep{M[P]}$. To foreshadow what is to come we observe that these
operations enjoy a duality with processes very much like the duality
between vectors and maps from vectors to scalars.

Further, because the calculus is essentially higher-order, we have a
correspondence between contexts and processes. More specifically,
given a name $x$ and a context $M$ we can construct $M^{*}_{x}$ such
that 

\begin{mathpar}
  M^{*}_{x} | \lift{x}{P} \red M[P]
\end{mathpar}

namely,

\begin{mathpar}
  M^{*}_{x} := x?(u).M[\dropn{u}]
\end{mathpar}

The dependence of $M^{*}_{x}$ on a name makes it an abstraction, 

\begin{mathpar}
  M^{*} := (x)x?(u).M[\dropn{u}]
\end{mathpar}

\subsection{Additional notation}

It will sometimes be convenient to denote the process a name
quotes. We already have the notation $x = \quotep{P}$, but it will be
convenient to introduce an alternate notation, $\procn{x}$, when we
want to emphasize the connection to the use of the name. Note that, by
virtue of name equivalence, $\quotep{\procn{x}} \nameeq x$; so, the
notation is consistent with previous definitions.

Further, because names have structure it is possible to effect
substitutions on the basis of that structure. This means we need to
upgrade our notation for substitutions, which we accomplish by
adapting comprehension notation. Thus,

\begin{mathpar}
  P\{ y / x : x \in S \}
\end{mathpar}

is interpreted to mean the process derived from P by replacing (in a
capture-avoiding manner) each occurrence of $x$ in $S$ by $y$. For example,

\begin{mathpar}
  P\{ \quotep{\procn{x}|\procn{x}} / x : x \in \freenames{P} \}
\end{mathpar}

will replace each (occurrence) of a free name $x$ in $P$ by
$\quotep{\procn{x}|\procn{x}}$.

Also, we will avail ourselves of the notation $x^{L}$ and $x^{R}$ to
denote injections of a name into disjoint copies of the name
space. There are numerous ways to accomplish this. One example can be
found in \cite{MeredithR05}. This notation overloads to vectors of
names: $\vec{x}^{\pi} := (x_{i}^{\pi} \; : \; 0 \leq i < |\vec{x}| )$ where $\pi \in \{L,R\}$.

We also use $P^{\Box} := P|\Box$.

In \cite{MeredithR05} an interpretation of the new operator is
given. It turns out that there are several possible interpretations
all enjoying the requisite algebraic properties of the operator (see
\cite{milner91polyadicpi}). We will therefore make liberal use of
$(\nu\; \vec{x})P$.

% subsection the_syntax_and_semantics_of_the_notation_system (end)   

\input{qm2pi.qmops} 

\input{qm2pi.sterngerlach} 

\input{qm2pi.metric} 

% section concurrent_process_calculi (end)

%\input{qm2pi.proofsketch}

% section proof sketch (end)

%\input{qm2pi.slviaknots} 

% section spatial logic via knots (end)

\input{qm2pi.conclusion}

% section conclusion (end)

%\input{qm2pi.dtcodes} 

% section wiring algorithm (end)

\input{qm2pi.ack} 

% section acknowledgments (end)

\newpage


\bibliographystyle{plain}   
\bibliography{../../biblios/main.bib}

\input{qm2pi.rhodetails}

\end{document}

 

% section notation (end)

\input{qm2pi.process.calculi} 

% section concurrent_process_calculi_and_spatial_logics_ (end)
    
%\documentclass[12pt]{llncs}
%\documentclass{jktr}

\usepackage[pdftex]{hyperref}                   
\usepackage {listings}
\usepackage {mathpartir}
\usepackage{bcprules}
%\usepackage{listings}
                       
\usepackage{graphicx} 
%\usepackage[margins=2.5cm,nohead,nofoot]{geometry}
%\usepackage{geometry}
\usepackage{amsfonts}
\usepackage{amstext}
\usepackage{latexsym}
\usepackage{amssymb}
\usepackage{color}


%\include{myPreamble}
\include{qm2pi.local} 

%\ifpdf
%\usepackage[pdftex]{graphicx}
%\else
%\usepackage{graphicx}
%\fi

 % \ifpdf
%  \usepackage{pdfsync}
%  \if


%\title{Brief Article}
%\author{David F. Snyder}
%\author{L.G. Meredith}

%\address{Dept. of Math., Texas State University--San Marcos, San Marcos, TX 78666}
       
\pagestyle{empty}


\begin{document}

\lstset{language=[Objective]Caml,frame=shadowbox}

\input{qm2pi.front}

% section front matter (end)

\input{qm2pi.intro} 
 
% section introduction (end)

% \input{qm2pi.knotations} 

% section notation (end)

\input{qm2pi.process.calculi} 

% section concurrent_process_calculi_and_spatial_logics_ (end)
    
%\input{qm2pi.knots2pi} 

%\input{qm2pi.trefoil} 

%\input{qm2pi.mainthm} 

% subsection basic_interpretation (end)

%\input{qm2pi.rho.presentation} 
\subsection{The syntax and semantics of the notation system}\label{sub:the_syntax_and_semantics_of_the_notation_system} % (fold)

We now summarize a technical presentation of the calculus that
embodies our theory of dynamics. The typical presentation of such a
calculus follows the style of giving generators and relations on
them. The grammar, below, describing term constructors, freely
generates the set of processes, $\Proc$. This set is then quotiented
by a relation known as structural congruence and it is over this set
that the notion of dynamics is expressed. This presentation is
essentially that of \cite{MeredithR05} with the addition of
polyadicity and summation. For readability we have relegated some of
the technical subtleties to an appendix.

\subsubsection{Process grammar}\label{subsub:process_grammar}

\begin{mathpar}
  \inferrule* [lab=synchronization] {} {{M} \bc \pzero \;|\; x?F \;|\; x!C }
  \and
  \inferrule* [lab=abstraction] {} {{F} \bc (x)P}
  \and
  \inferrule* [lab=concretion] {} {{C} \bc \langle Q \rangle}
  \and
  \inferrule* [lab=process] {} {{P,Q} \bc M \;| \;P|Q \;|\; @{x}}
  \and
  \inferrule* [lab=name] {} {{x} \bc \quotep{P}}
\end{mathpar} 

Note that $\vec{x}$ (resp. $\vec{P}$) denotes a vector of names
(resp. processes) of length $|\vec{x}|$ (resp. $|\vec{P}|$). We adopt
the following useful abbreviations.

\begin{mathpar}
   x?(\vec{y}).P := x.(\vec{y})P \and  x\clift{\vec{P}} := x.\clift{\vec{P}}
   \and x!(y) := \lift{x}{\dropn{y}}
   \and \Pi_{i=0}^{n-1}P_i := P_0 | \ldots | P_{n-1}
\end{mathpar}

\subsubsection{Structural congruence}

\paragraph{Free and bound names and alpha-equivalence.} At the
core of structural equivalence is alpha-equivalence which identifies
process that are the same up to a change of variable. Formally, we
recognize the distinction between free and bound names. The free names
of a process, $\freenames{P}$, may be calculated recursively as
follows:

\begin{mathpar}
\freenames{\pzero} := \emptyset
  \and \\
  \freenames{x?(y).P} := \{ x \} \cup (\freenames{P} \setminus \{ y \})
  \and 
  \freenames{x!\langle P \rangle} := \{ x \} \cup \{ P \} 
  \and \\
  \freenames{P|Q} := \freenames{P} \cup \freenames{Q}
  \and \\
  \freenames{@{x}} := \{ x \}
\end{mathpar}

$\pi$
$\quotep{\pi}$

$\freenames{-} : \pi \to \mathcal{P}(\quotep{\pi})$

\begin{eqnarray*}
  \freenames{\pzero} & := & \emptyset \\
  \freenames{x?(y).P} & := & \{ x \} \cup (\freenames{P} \setminus \{ y \}) \\
  \freenames{x!\langle P \rangle} & := & \{ x \} \cup \{ P \} \\
  \freenames{P|Q} & := & \freenames{P} \cup \freenames{Q} \\
  \freenames{\dropn{x}} & := & \{ x \}
\end{eqnarray*}

The bound names of a process, $\boundnames{P}$, are those names occurring in $P$
that are not free. For example, in $x?(y).0$, the name $x$ is free, while $y$ is bound.

\begin{mathpar}
  \inferrule* [lab=monoidal-laws] {} { P|Q \equiv Q|P \and P|0 \equiv P \and P|(Q|R) \equiv (P|Q)|R }
\end{mathpar}

\begin{mathpar}
  \inferrule* [lab=alpha-equivalence] {} { (x)P \equiv (y)P\{y/x\} \and y \not\in \freenames{P} }
\end{mathpar}

\begin{definition}
Then two processes, $P,Q$, are alpha-equivalent if $P = Q\{\vec{y}/\vec{x}\}$ for
some $\vec{x} \in \boundnames{Q},\vec{y} \in \boundnames{P}$, where $Q\{\vec{y}/\vec{x}\}$
denotes the capture-avoiding substitution of $\vec{y}$ for $\vec{x}$ in $Q$.
\end{definition}

\begin{definition}
  The {\em structural congruence} \cite{SangiorgiWalker} , $\equiv$,
  between processes is the least congruence containing
  alpha-equivalence, satisfying the abelian monoid laws
  (associativity, commutativity and $\pzero$ as identity) for parallel
  composition $|$ and for summation $+$.
\end{definition}

\subsection{Name equivalence}

We take name equivalence, written $\nameeq$, to be the smallest
equivalence relation generated by the following rules.

\begin{mathpar}
\inferrule*[lab=Quote-drop]
{ }
{ \quotep{@{x}} \nameeq x }

\inferrule*[lab=Struct-equiv]
{ P \scong Q }
{ \quotep{P} \nameeq \quotep{Q} }
\end{mathpar}

The astute reader will have noticed that the mutual recursion of names
and processes imposes a mutual recursion on alpha-equivalence and
structural equivalence via name-equivalence. Fortunately, all of this
works out pleasantly and we may calculate in the natural way, free of
concern. The reader interested in the details is referred to the
appendix \ref{appendix:rho_details}.

\subsection{Substitution}

We use $\Proc$ for the set of processes, $\QProc$ for the set of
names, and $\id{\{}\vec{y} / \vec{x} \id{\}}$ to denote partial maps,
$s : \QProc \rightarrow \QProc$. A map, $s$ lifts, uniquely, to a map
on process terms, $\widehat{s} : \Proc \rightarrow \Proc$ by the
following equations.

\begin{mathpar}
  (0) \psubstp{Q}{P} := 0 \\
  (R \juxtap S) \psubstp{Q}{P}
  :=    
  (R)\psubstp{Q}{P} \juxtap (S) \psubstp{Q}{P} \\
  (x?(y).R) \psubstp{Q}{P}    
  :=    
  (x)\substp{Q}{P} (z)\concat( (R \psubstn{z}{y}) \psubstp{Q}{P} ) \\
  (\lift{x}{R}) \psubstp{Q}{P}  
  :=
  \lift{(x)\substp{Q}{P}}{ R \psubstp{Q}{P} } \\
%   (\dropn{x})  \psubstp{Q}{P}       
%   := 
%   \left\{ 
%     \begin{array}{ccc} 
%       \dropn{\quotep{Q}} & & x \nameeq \quotep{P} \\
%       \dropn{x} & & otherwise \\
%     \end{array}
%   \right. 
  (\dropn{x})  \psubstp{Q}{P}       
  := 
  \left\{ 
    \begin{array}{ccc} 
      Q & & x \nameeq \quotep{P} \\
      \dropn{x} & & otherwise \\
    \end{array}
  \right.
\end{mathpar}
 

where

\begin{eqnarray}
  (x)\id{\{} \lpquote Q \rpquote / \lpquote P \rpquote \id{\}}            = 
  \left\{ 
    \begin{array}{ccc}
      \lpquote Q \rpquote & & x \nameeq \lpquote P \rpquote \\
      x & & otherwise \\
    \end{array}
  \right. \nonumber
\end{eqnarray}

and $z$ is chosen distinct from $\quotep{P}$, $\quotep{Q}$, the free
names in $Q$, and all the names in $R$. Our $\alpha$-equivalence will
be built in the standard way from this substitution.

\begin{remark}\label{rem:no_self_referential_names}
  One consequence of these definitions is that $\forall P. \quotep{P}
  \not\in \freenames{P}$.
\end{remark}

\subsection{ Dynamic quote: an example }

Anticipating something of what's to come, consider applying the
substitution, $\widehat{\id{\{}u / z \id{\}}}$, to the following pair
of processes, $\lift{w}{y!(z)}$ and $w[ \lpquote y!(z) \rpquote ]$.

\begin{eqnarray}
	\lift{w}{y!(z)}\widehat{\id{\{}u / z \id{\}}}
		& = &
		\lift{w}{y!(u)} \nonumber\\
	w[ \lpquote y!(z) \rpquote ] \widehat{ \id{\{}u / z \id{\}} }
		& = &
		w[ \lpquote y!(z) \rpquote ] \nonumber
\end{eqnarray}

Because the body of the process between quotes is impervious to
substitution, we get radically different answers. In fact, by
examining the first process in an input context,
e.g. $x?(z).\lift{w}{y!(z)}$, we see that the process under the lift
operator may be shaped by prefixed inputs binding a name inside it. In
this sense, the lift operator will be seen as a way to dynamically
construct processes before reifying them as names.

Finally equipped with these standard features we can present the
dynamics of the calculus.

\subsubsection{Operational semantics} 

Finally, we introduce the computational dynamics. What marks these
algebras as distinct from other more traditionally studied algebraic
structures, e.g. vector spaces or polynomial rings, is the manner in
which dynamics is captured. In traditional structures, dynamics is typically
expressed through morphisms between such structures, as in linear maps
between vector spaces or morphisms between rings. In algebras
associated with the semantics of computation, the dynamics is
expressed as part of the algebraic structure itself, through a
reduction reduction relation typically denoted by $\red$. Below, we
give a recursive presentation of this relation for the calculus used
in the encoding.

$\red \subseteq \pi \times \pi$
$\red : \pi \to \mathcal{P}(\pi)$

\begin{mathpar}
  \inferrule* [lab=Comm] { \textsf{match}( x_{src}, x_{trgt} ) } { x_{trgt}?(y)P \; | \; x_{src}!\langle {Q} \rangle \red P\{\quotep{Q}/y}\} }
  \and \\
  \inferrule* [lab=Par] {{P} \red {P}'} {{{P} | {Q}} \red {{P}' | {Q}}}
  \and
  \inferrule* [lab=Equiv]{{{P} \scong {P}'} \andalso {{P}' \red {Q}'} \andalso {{Q}' \scong {Q}}}{{P} \red {Q}}
\end{mathpar}

\begin{eqnarray*}
  match_{\equiv} (\quotep{P},\quotep{Q}) & := & P \equiv Q \\
  match_{\dagger}(\quotep{P},\quotep{Q}) & := & \forall R. P|Q \red^{*} R => R \red^{*} 0 \\
  match_{K}(\quotep{P},\quotep{Q}) & := & K \mbox{ for some context } K
\end{eqnarray*}

$u?(x)P | u!\langle Q \rangle \red P\{\quotep{Q}/x\}$

%We write $\wred$ for $\red^*$, and $P\red$ if $\exists Q $ such that $ P \red Q$.
We write $P\red$ if $\exists Q $ such that $ P \red Q$ and $P\not\red$, otherwise.

\section{Replication}

As mentioned before, it is known that replication (and hence
recursion) can be implemented in a higher-order process algebra
\cite{SangiorgiWalker}. As our first example of calculation with the
machinery thus far presented we give the construction explicitly in
the {\rhoc}.

\begin{eqnarray}
	D_{x} & := & \prefix{x}{y}{(\binpar{\outputp{x}{y}}{@{y}})} \nonumber\\
	\bangp_{x}{P} & := & \binpar{{x}!\langle{\binpar{D_{x}}{P}}\rangle}{D_{x}} \nonumber
\end{eqnarray}

\begin{eqnarray}
	\bangp_{x}{P} & & \nonumber\\
	=
	& {x}!\langle{(\prefix{x}{y}{(\outputp{x}{y} | @{y})) | P}}\rangle 
	      | \prefix{x}{y}{(\outputp{x}{y} | @{y})} & \nonumber\\
	\red
	& (\outputp{x}{y} | @{y})\substn{\quotep{(\prefix{x}{y}{(@{y} | \outputp{x}{y})) | P}}}{y} & \nonumber\\
	=
	& \outputp{x}{\quotep{(\prefix{x}{y}{(\outputp{x}{y} | @{y})) | P}}}
	  | {(\prefix{x}{y}{(\outputp{x}{y} | @{y})) | P}} & \nonumber\\
	\red
	& \ldots & \nonumber\\
	\red^*
	& P | P | \ldots & \nonumber
\end{eqnarray}

Of course, this encoding, as an implementation, runs away, unfolding
$\bangp{P}$ eagerly. A lazier and more implementable replication
operator, restricted to input-guarded processes, may be obtained as follows.

\begin{eqnarray}
\bangp{\prefix{u}{v}{P}} 
	:= 
	\binpar{\lift{x}{\prefix{u}{v}{(\binpar{D(x)}{P})}}}{D(x)} \nonumber
\end{eqnarray}

\begin{remark}
  Note that the lazier definition still does not deal with summation
  or mixed summation (i.e. sums over input and output). The reader is
  invited to construct definitions of replication that deal with these
  features. 

  Further, the definitions are parameterized in a name, $x$. Can you,
  gentle reader, make a definition that eliminates this parameter and
  guarantees no accidental interaction between the replication
  machinery and the process being replicated -- i.e. no accidental
  sharing of names used by the process to get its work done and the
  name(s) used by the replication to effect copying. This latter
  revision of the definition of replication is crucial to obtaining
  the expected identity $!!P \sim !P$.
\end{remark}

\begin{remark}\label{rem:paradoxical_combinator}
  The reader familiar with the lambda calculus will have noticed the
  similarity between $D$ and the paradoxical combinator.

  [Ed. note: the existence of this seems to suggest we have to be more
  restrictive on the set of processes and names we admit if we are to
  support no-cloning.]
\end{remark}

\subsubsection{Bisimulation}

The computational dynamics gives rise to another kind of equivalence,
the equivalence of computational behavior. As previously mentioned
this is typically captured \emph{via} some form of bisimulation.

% The notion we use in this paper is weak barbed bisimulation
% \cite{milner91polyadicpi}.

The notion we use in this paper is derived from weak barbed
bisimulation \cite{milner91polyadicpi}. 

\begin{definition}
An \emph{observation relation}, $\downarrow_{\mathcal N}$, over a set
of names, $\mathcal N$, is the smallest relation satisfying the rules
below.

\infrule[Out-barb]{y \in {\mathcal N}, \; x \nameeq y}
		  {\outputp{x}{v} \downarrow_{\mathcal N} x}
\infrule[Par-barb]{\mbox{$P\downarrow_{\mathcal N} x$ or $Q\downarrow_{\mathcal N} x$}}
		  {\binpar{P}{Q} \downarrow_{\mathcal N} x}

We write $P \Downarrow_{\mathcal N} x$ if there is $Q$ such that 
$P \wred Q$ and $Q \downarrow_{\mathcal N} x$.
\end{definition}

\begin{definition}
%\label{def.bbisim}
An  ${\mathcal N}$-\emph{barbed bisimulation} over a set of names, ${\mathcal N}$, is a symmetric binary relation 
${\mathcal S}_{\mathcal N}$ between agents such that $P\rel{S}_{\mathcal N}Q$ implies:
\begin{enumerate}
\item If $P \red P'$ then $Q \wred Q'$ and $P'\rel{S}_{\mathcal N} Q'$.
\item If $P\downarrow_{\mathcal N} x$, then $Q\Downarrow_{\mathcal N} x$.
\end{enumerate}
$P$ is ${\mathcal N}$-barbed bisimilar to $Q$, written
$P \wbbisim_{\mathcal N} Q$, if $P \rel{S}_{\mathcal N} Q$ for some ${\mathcal N}$-barbed bisimulation ${\mathcal S}_{\mathcal N}$.
\end{definition}

$\mathcal{R} \subseteq \pi \times \pi$

$P \mathcal{R} Q => \forall P'. P \red P' \Rightarrow \exists Q'. Q \red Q', P' \mathcal{R} Q'$

$P \vdash x \Rightarrow Q \vdash x$

\begin{mathpar}
  \inferrule*[lab=Out-barb]{x \nameeq y}{{y}!\langle{Q}\rangle \vdash x}
  \and
  \inferrule*[lab=Par-barb]{\mbox{$P\vdash x$ or $Q\vdash x$}}{\binpar{P}{Q} \vdash x}
\end{mathpar}

\subsubsection{Contexts}

One of the principle advantages of computational calculi like the
$\pi$-calculus is a well-defined notion of context,
contextual-equivalence and a correlation between
contextual-equivalence and notions of bisimulation. The notion of
context allows the decomposition of a process into (sub-)process and
its syntactic environment, its context. Thus, a context may be
thought of as a process with a ``hole'' (written $\Box$) in it. The
application of a context $M$ to a process $P$, written $M[P]$, is
tantamount to filling the hole in $M$ with $P$. In this paper we do
not need the full weight of this theory, but do make use of the notion
of context in the proof the main theorem. 

\begin{mathpar}
  \inferrule* [lab=summation] {} {{M_{M},M_{N}} \bc \Box \;|\; x.M_{A} \;|\; M_{M}+M_{N}}
  \and
  \inferrule* [lab=agent] {} {{M_{A}} \bc (\vec{x})M_{P} \;| \; \clift{P_0,\ldots,M_{P},\ldots,P_N}}
  \and \\
  \inferrule* [lab=process] {} {{M_{P}} \bc M_{N} \;| \;P|M_{P} }
\end{mathpar} 

\begin{mathpar}
  \inferrule* [lab=sychronization] {} {M_{N} \bc \Box \;|\; x?M_{F} \;|\; x!M_{C}}
  \and
  \inferrule* [lab=abstraction] {} {{M_{F}} \bc (x)M_{P} }
  \and
  \inferrule* [lab=concretion] {} {{M_{C}} \bc \langle M_{P} \rangle }
  \and \\
  \inferrule* [lab=process] {} {{M_{P}} \bc M_{N} \;| \;P|M_{P} }
\end{mathpar}

\begin{definition}[contextual application] Given a context $M$, and
  process $P$, we define the \emph{contextual application}, $M[P] :=
  M\{P/\Box\}$. That is, the contextual application of M to P is the
  substitution of $P$ for $\Box$ in $M$.
\end{definition}

$\meaningof{-} : L \to \mathcal{P}(\pi)$

\begin{mathpar}
  \inferrule* [lab=collection] {} {\meaningof{true} = \pi, \and \meaningof{~E} = \pi \setminus \meaningof{E}, \and \meaningof{E_{1} \& E_{2}} = \meaningof{E_{1}} \cap \meaningof{E_{2}}}
\end{mathpar}

\begin{mathpar}
  \inferrule* [lab=structure] {} {\meaningof{0} = \{ P \in \pi | P \equiv 0 \}, \and \\ \meaningof{E_1 | E_2} = \{ P \in \pi | P \equiv P_{1} | P_{2}, P_{1} \in \meaningof{E_{1}}, P_{2} \in \meaningof{E_2}\} }
\end{mathpar}

\begin{mathpar}
 \inferrule* [lab=behavior] {} {\meaningof{\langle a?b \rangle E} = \{ P \in \pi | P \equiv Q | u?(y)P', \\ \and \\\\ \and \\ \;\;\; u \in \meaningof{a}, \forall z.P'\{z/y\} \in \meaningof{E\{z/b\}}\}, \and \\ \meaningof{a!E} = \{ P \in \pi | P \equiv Q | x!\langle P' \rangle, x \in \meaningof{a} P' \in \meaningof{E}\} }
\end{mathpar}

\begin{mathpar}
 \inferrule* [lab=nominal] {} {\meaningof{\quotep{E}} = \{ \quotep{P} \in \quotep{\pi} | P \in \meaningof{E} \}, \and \meaningof{\quotep{P}} = \{ \quotep{Q} \in \quotep{\pi} | P \equiv Q \} \and \\ \meaningof{@\quotep{E}} = \{ P \in \pi | P \equiv @x, x \in \meaningof{E} \}}
\end{mathpar}

\begin{eqnarray*}
  \\
  \meaningof{-} : TS \to ST
\end{eqnarray*}

\begin{eqnarray*}
  \\
  L : TS \to ST
\end{eqnarray*}

\begin{eqnarray*}
  \\
  P \models E \iff P \in \meaningof{E}
\end{eqnarray*}

\begin{eqnarray*}
  P \approx_{L} Q \iff \forall E \in L. P \models E \iff Q \models E
\end{eqnarray*}

\begin{eqnarray*}
  P \approx_{K} Q
\end{eqnarray*}

\begin{eqnarray*}
  P \approx Q
\end{eqnarray*}

$\approx_{K} = \approx = \approx_{L}$

\subsubsection{Contextual duality}

Note that contexts extend the quotation operation to a family of
operations from processes to names. Given a context, $M$, we can
define a \emph{nominal context}, $\quotep{M}$ by $\quotep{M}[P] :=
\quotep{M[P]}$. To foreshadow what is to come we observe that these
operations enjoy a duality with processes very much like the duality
between vectors and maps from vectors to scalars.

Further, because the calculus is essentially higher-order, we have a
correspondence between contexts and processes. More specifically,
given a name $x$ and a context $M$ we can construct $M^{*}_{x}$ such
that 

\begin{mathpar}
  M^{*}_{x} | \lift{x}{P} \red M[P]
\end{mathpar}

namely,

\begin{mathpar}
  M^{*}_{x} := x?(u).M[\dropn{u}]
\end{mathpar}

The dependence of $M^{*}_{x}$ on a name makes it an abstraction, 

\begin{mathpar}
  M^{*} := (x)x?(u).M[\dropn{u}]
\end{mathpar}

\subsection{Additional notation}

It will sometimes be convenient to denote the process a name
quotes. We already have the notation $x = \quotep{P}$, but it will be
convenient to introduce an alternate notation, $\procn{x}$, when we
want to emphasize the connection to the use of the name. Note that, by
virtue of name equivalence, $\quotep{\procn{x}} \nameeq x$; so, the
notation is consistent with previous definitions.

Further, because names have structure it is possible to effect
substitutions on the basis of that structure. This means we need to
upgrade our notation for substitutions, which we accomplish by
adapting comprehension notation. Thus,

\begin{mathpar}
  P\{ y / x : x \in S \}
\end{mathpar}

is interpreted to mean the process derived from P by replacing (in a
capture-avoiding manner) each occurrence of $x$ in $S$ by $y$. For example,

\begin{mathpar}
  P\{ \quotep{\procn{x}|\procn{x}} / x : x \in \freenames{P} \}
\end{mathpar}

will replace each (occurrence) of a free name $x$ in $P$ by
$\quotep{\procn{x}|\procn{x}}$.

Also, we will avail ourselves of the notation $x^{L}$ and $x^{R}$ to
denote injections of a name into disjoint copies of the name
space. There are numerous ways to accomplish this. One example can be
found in \cite{MeredithR05}. This notation overloads to vectors of
names: $\vec{x}^{\pi} := (x_{i}^{\pi} \; : \; 0 \leq i < |\vec{x}| )$ where $\pi \in \{L,R\}$.

We also use $P^{\Box} := P|\Box$.

In \cite{MeredithR05} an interpretation of the new operator is
given. It turns out that there are several possible interpretations
all enjoying the requisite algebraic properties of the operator (see
\cite{milner91polyadicpi}). We will therefore make liberal use of
$(\nu\; \vec{x})P$.

% subsection the_syntax_and_semantics_of_the_notation_system (end)   

\input{qm2pi.qmops} 

\input{qm2pi.sterngerlach} 

\input{qm2pi.metric} 

% section concurrent_process_calculi (end)

%\input{qm2pi.proofsketch}

% section proof sketch (end)

%\input{qm2pi.slviaknots} 

% section spatial logic via knots (end)

\input{qm2pi.conclusion}

% section conclusion (end)

%\input{qm2pi.dtcodes} 

% section wiring algorithm (end)

\input{qm2pi.ack} 

% section acknowledgments (end)

\newpage


\bibliographystyle{plain}   
\bibliography{../../biblios/main.bib}

\input{qm2pi.rhodetails}

\end{document}

 

%\documentclass[12pt]{llncs}
%\documentclass{jktr}

\usepackage[pdftex]{hyperref}                   
\usepackage {listings}
\usepackage {mathpartir}
\usepackage{bcprules}
%\usepackage{listings}
                       
\usepackage{graphicx} 
%\usepackage[margins=2.5cm,nohead,nofoot]{geometry}
%\usepackage{geometry}
\usepackage{amsfonts}
\usepackage{amstext}
\usepackage{latexsym}
\usepackage{amssymb}
\usepackage{color}


%\include{myPreamble}
\include{qm2pi.local} 

%\ifpdf
%\usepackage[pdftex]{graphicx}
%\else
%\usepackage{graphicx}
%\fi

 % \ifpdf
%  \usepackage{pdfsync}
%  \if


%\title{Brief Article}
%\author{David F. Snyder}
%\author{L.G. Meredith}

%\address{Dept. of Math., Texas State University--San Marcos, San Marcos, TX 78666}
       
\pagestyle{empty}


\begin{document}

\lstset{language=[Objective]Caml,frame=shadowbox}

\input{qm2pi.front}

% section front matter (end)

\input{qm2pi.intro} 
 
% section introduction (end)

% \input{qm2pi.knotations} 

% section notation (end)

\input{qm2pi.process.calculi} 

% section concurrent_process_calculi_and_spatial_logics_ (end)
    
%\input{qm2pi.knots2pi} 

%\input{qm2pi.trefoil} 

%\input{qm2pi.mainthm} 

% subsection basic_interpretation (end)

%\input{qm2pi.rho.presentation} 
\subsection{The syntax and semantics of the notation system}\label{sub:the_syntax_and_semantics_of_the_notation_system} % (fold)

We now summarize a technical presentation of the calculus that
embodies our theory of dynamics. The typical presentation of such a
calculus follows the style of giving generators and relations on
them. The grammar, below, describing term constructors, freely
generates the set of processes, $\Proc$. This set is then quotiented
by a relation known as structural congruence and it is over this set
that the notion of dynamics is expressed. This presentation is
essentially that of \cite{MeredithR05} with the addition of
polyadicity and summation. For readability we have relegated some of
the technical subtleties to an appendix.

\subsubsection{Process grammar}\label{subsub:process_grammar}

\begin{mathpar}
  \inferrule* [lab=synchronization] {} {{M} \bc \pzero \;|\; x?F \;|\; x!C }
  \and
  \inferrule* [lab=abstraction] {} {{F} \bc (x)P}
  \and
  \inferrule* [lab=concretion] {} {{C} \bc \langle Q \rangle}
  \and
  \inferrule* [lab=process] {} {{P,Q} \bc M \;| \;P|Q \;|\; @{x}}
  \and
  \inferrule* [lab=name] {} {{x} \bc \quotep{P}}
\end{mathpar} 

Note that $\vec{x}$ (resp. $\vec{P}$) denotes a vector of names
(resp. processes) of length $|\vec{x}|$ (resp. $|\vec{P}|$). We adopt
the following useful abbreviations.

\begin{mathpar}
   x?(\vec{y}).P := x.(\vec{y})P \and  x\clift{\vec{P}} := x.\clift{\vec{P}}
   \and x!(y) := \lift{x}{\dropn{y}}
   \and \Pi_{i=0}^{n-1}P_i := P_0 | \ldots | P_{n-1}
\end{mathpar}

\subsubsection{Structural congruence}

\paragraph{Free and bound names and alpha-equivalence.} At the
core of structural equivalence is alpha-equivalence which identifies
process that are the same up to a change of variable. Formally, we
recognize the distinction between free and bound names. The free names
of a process, $\freenames{P}$, may be calculated recursively as
follows:

\begin{mathpar}
\freenames{\pzero} := \emptyset
  \and \\
  \freenames{x?(y).P} := \{ x \} \cup (\freenames{P} \setminus \{ y \})
  \and 
  \freenames{x!\langle P \rangle} := \{ x \} \cup \{ P \} 
  \and \\
  \freenames{P|Q} := \freenames{P} \cup \freenames{Q}
  \and \\
  \freenames{@{x}} := \{ x \}
\end{mathpar}

$\pi$
$\quotep{\pi}$

$\freenames{-} : \pi \to \mathcal{P}(\quotep{\pi})$

\begin{eqnarray*}
  \freenames{\pzero} & := & \emptyset \\
  \freenames{x?(y).P} & := & \{ x \} \cup (\freenames{P} \setminus \{ y \}) \\
  \freenames{x!\langle P \rangle} & := & \{ x \} \cup \{ P \} \\
  \freenames{P|Q} & := & \freenames{P} \cup \freenames{Q} \\
  \freenames{\dropn{x}} & := & \{ x \}
\end{eqnarray*}

The bound names of a process, $\boundnames{P}$, are those names occurring in $P$
that are not free. For example, in $x?(y).0$, the name $x$ is free, while $y$ is bound.

\begin{mathpar}
  \inferrule* [lab=monoidal-laws] {} { P|Q \equiv Q|P \and P|0 \equiv P \and P|(Q|R) \equiv (P|Q)|R }
\end{mathpar}

\begin{mathpar}
  \inferrule* [lab=alpha-equivalence] {} { (x)P \equiv (y)P\{y/x\} \and y \not\in \freenames{P} }
\end{mathpar}

\begin{definition}
Then two processes, $P,Q$, are alpha-equivalent if $P = Q\{\vec{y}/\vec{x}\}$ for
some $\vec{x} \in \boundnames{Q},\vec{y} \in \boundnames{P}$, where $Q\{\vec{y}/\vec{x}\}$
denotes the capture-avoiding substitution of $\vec{y}$ for $\vec{x}$ in $Q$.
\end{definition}

\begin{definition}
  The {\em structural congruence} \cite{SangiorgiWalker} , $\equiv$,
  between processes is the least congruence containing
  alpha-equivalence, satisfying the abelian monoid laws
  (associativity, commutativity and $\pzero$ as identity) for parallel
  composition $|$ and for summation $+$.
\end{definition}

\subsection{Name equivalence}

We take name equivalence, written $\nameeq$, to be the smallest
equivalence relation generated by the following rules.

\begin{mathpar}
\inferrule*[lab=Quote-drop]
{ }
{ \quotep{@{x}} \nameeq x }

\inferrule*[lab=Struct-equiv]
{ P \scong Q }
{ \quotep{P} \nameeq \quotep{Q} }
\end{mathpar}

The astute reader will have noticed that the mutual recursion of names
and processes imposes a mutual recursion on alpha-equivalence and
structural equivalence via name-equivalence. Fortunately, all of this
works out pleasantly and we may calculate in the natural way, free of
concern. The reader interested in the details is referred to the
appendix \ref{appendix:rho_details}.

\subsection{Substitution}

We use $\Proc$ for the set of processes, $\QProc$ for the set of
names, and $\id{\{}\vec{y} / \vec{x} \id{\}}$ to denote partial maps,
$s : \QProc \rightarrow \QProc$. A map, $s$ lifts, uniquely, to a map
on process terms, $\widehat{s} : \Proc \rightarrow \Proc$ by the
following equations.

\begin{mathpar}
  (0) \psubstp{Q}{P} := 0 \\
  (R \juxtap S) \psubstp{Q}{P}
  :=    
  (R)\psubstp{Q}{P} \juxtap (S) \psubstp{Q}{P} \\
  (x?(y).R) \psubstp{Q}{P}    
  :=    
  (x)\substp{Q}{P} (z)\concat( (R \psubstn{z}{y}) \psubstp{Q}{P} ) \\
  (\lift{x}{R}) \psubstp{Q}{P}  
  :=
  \lift{(x)\substp{Q}{P}}{ R \psubstp{Q}{P} } \\
%   (\dropn{x})  \psubstp{Q}{P}       
%   := 
%   \left\{ 
%     \begin{array}{ccc} 
%       \dropn{\quotep{Q}} & & x \nameeq \quotep{P} \\
%       \dropn{x} & & otherwise \\
%     \end{array}
%   \right. 
  (\dropn{x})  \psubstp{Q}{P}       
  := 
  \left\{ 
    \begin{array}{ccc} 
      Q & & x \nameeq \quotep{P} \\
      \dropn{x} & & otherwise \\
    \end{array}
  \right.
\end{mathpar}
 

where

\begin{eqnarray}
  (x)\id{\{} \lpquote Q \rpquote / \lpquote P \rpquote \id{\}}            = 
  \left\{ 
    \begin{array}{ccc}
      \lpquote Q \rpquote & & x \nameeq \lpquote P \rpquote \\
      x & & otherwise \\
    \end{array}
  \right. \nonumber
\end{eqnarray}

and $z$ is chosen distinct from $\quotep{P}$, $\quotep{Q}$, the free
names in $Q$, and all the names in $R$. Our $\alpha$-equivalence will
be built in the standard way from this substitution.

\begin{remark}\label{rem:no_self_referential_names}
  One consequence of these definitions is that $\forall P. \quotep{P}
  \not\in \freenames{P}$.
\end{remark}

\subsection{ Dynamic quote: an example }

Anticipating something of what's to come, consider applying the
substitution, $\widehat{\id{\{}u / z \id{\}}}$, to the following pair
of processes, $\lift{w}{y!(z)}$ and $w[ \lpquote y!(z) \rpquote ]$.

\begin{eqnarray}
	\lift{w}{y!(z)}\widehat{\id{\{}u / z \id{\}}}
		& = &
		\lift{w}{y!(u)} \nonumber\\
	w[ \lpquote y!(z) \rpquote ] \widehat{ \id{\{}u / z \id{\}} }
		& = &
		w[ \lpquote y!(z) \rpquote ] \nonumber
\end{eqnarray}

Because the body of the process between quotes is impervious to
substitution, we get radically different answers. In fact, by
examining the first process in an input context,
e.g. $x?(z).\lift{w}{y!(z)}$, we see that the process under the lift
operator may be shaped by prefixed inputs binding a name inside it. In
this sense, the lift operator will be seen as a way to dynamically
construct processes before reifying them as names.

Finally equipped with these standard features we can present the
dynamics of the calculus.

\subsubsection{Operational semantics} 

Finally, we introduce the computational dynamics. What marks these
algebras as distinct from other more traditionally studied algebraic
structures, e.g. vector spaces or polynomial rings, is the manner in
which dynamics is captured. In traditional structures, dynamics is typically
expressed through morphisms between such structures, as in linear maps
between vector spaces or morphisms between rings. In algebras
associated with the semantics of computation, the dynamics is
expressed as part of the algebraic structure itself, through a
reduction reduction relation typically denoted by $\red$. Below, we
give a recursive presentation of this relation for the calculus used
in the encoding.

$\red \subseteq \pi \times \pi$
$\red : \pi \to \mathcal{P}(\pi)$

\begin{mathpar}
  \inferrule* [lab=Comm] { \textsf{match}( x_{src}, x_{trgt} ) } { x_{trgt}?(y)P \; | \; x_{src}!\langle {Q} \rangle \red P\{\quotep{Q}/y}\} }
  \and \\
  \inferrule* [lab=Par] {{P} \red {P}'} {{{P} | {Q}} \red {{P}' | {Q}}}
  \and
  \inferrule* [lab=Equiv]{{{P} \scong {P}'} \andalso {{P}' \red {Q}'} \andalso {{Q}' \scong {Q}}}{{P} \red {Q}}
\end{mathpar}

\begin{eqnarray*}
  match_{\equiv} (\quotep{P},\quotep{Q}) & := & P \equiv Q \\
  match_{\dagger}(\quotep{P},\quotep{Q}) & := & \forall R. P|Q \red^{*} R => R \red^{*} 0 \\
  match_{K}(\quotep{P},\quotep{Q}) & := & K \mbox{ for some context } K
\end{eqnarray*}

$u?(x)P | u!\langle Q \rangle \red P\{\quotep{Q}/x\}$

%We write $\wred$ for $\red^*$, and $P\red$ if $\exists Q $ such that $ P \red Q$.
We write $P\red$ if $\exists Q $ such that $ P \red Q$ and $P\not\red$, otherwise.

\section{Replication}

As mentioned before, it is known that replication (and hence
recursion) can be implemented in a higher-order process algebra
\cite{SangiorgiWalker}. As our first example of calculation with the
machinery thus far presented we give the construction explicitly in
the {\rhoc}.

\begin{eqnarray}
	D_{x} & := & \prefix{x}{y}{(\binpar{\outputp{x}{y}}{@{y}})} \nonumber\\
	\bangp_{x}{P} & := & \binpar{{x}!\langle{\binpar{D_{x}}{P}}\rangle}{D_{x}} \nonumber
\end{eqnarray}

\begin{eqnarray}
	\bangp_{x}{P} & & \nonumber\\
	=
	& {x}!\langle{(\prefix{x}{y}{(\outputp{x}{y} | @{y})) | P}}\rangle 
	      | \prefix{x}{y}{(\outputp{x}{y} | @{y})} & \nonumber\\
	\red
	& (\outputp{x}{y} | @{y})\substn{\quotep{(\prefix{x}{y}{(@{y} | \outputp{x}{y})) | P}}}{y} & \nonumber\\
	=
	& \outputp{x}{\quotep{(\prefix{x}{y}{(\outputp{x}{y} | @{y})) | P}}}
	  | {(\prefix{x}{y}{(\outputp{x}{y} | @{y})) | P}} & \nonumber\\
	\red
	& \ldots & \nonumber\\
	\red^*
	& P | P | \ldots & \nonumber
\end{eqnarray}

Of course, this encoding, as an implementation, runs away, unfolding
$\bangp{P}$ eagerly. A lazier and more implementable replication
operator, restricted to input-guarded processes, may be obtained as follows.

\begin{eqnarray}
\bangp{\prefix{u}{v}{P}} 
	:= 
	\binpar{\lift{x}{\prefix{u}{v}{(\binpar{D(x)}{P})}}}{D(x)} \nonumber
\end{eqnarray}

\begin{remark}
  Note that the lazier definition still does not deal with summation
  or mixed summation (i.e. sums over input and output). The reader is
  invited to construct definitions of replication that deal with these
  features. 

  Further, the definitions are parameterized in a name, $x$. Can you,
  gentle reader, make a definition that eliminates this parameter and
  guarantees no accidental interaction between the replication
  machinery and the process being replicated -- i.e. no accidental
  sharing of names used by the process to get its work done and the
  name(s) used by the replication to effect copying. This latter
  revision of the definition of replication is crucial to obtaining
  the expected identity $!!P \sim !P$.
\end{remark}

\begin{remark}\label{rem:paradoxical_combinator}
  The reader familiar with the lambda calculus will have noticed the
  similarity between $D$ and the paradoxical combinator.

  [Ed. note: the existence of this seems to suggest we have to be more
  restrictive on the set of processes and names we admit if we are to
  support no-cloning.]
\end{remark}

\subsubsection{Bisimulation}

The computational dynamics gives rise to another kind of equivalence,
the equivalence of computational behavior. As previously mentioned
this is typically captured \emph{via} some form of bisimulation.

% The notion we use in this paper is weak barbed bisimulation
% \cite{milner91polyadicpi}.

The notion we use in this paper is derived from weak barbed
bisimulation \cite{milner91polyadicpi}. 

\begin{definition}
An \emph{observation relation}, $\downarrow_{\mathcal N}$, over a set
of names, $\mathcal N$, is the smallest relation satisfying the rules
below.

\infrule[Out-barb]{y \in {\mathcal N}, \; x \nameeq y}
		  {\outputp{x}{v} \downarrow_{\mathcal N} x}
\infrule[Par-barb]{\mbox{$P\downarrow_{\mathcal N} x$ or $Q\downarrow_{\mathcal N} x$}}
		  {\binpar{P}{Q} \downarrow_{\mathcal N} x}

We write $P \Downarrow_{\mathcal N} x$ if there is $Q$ such that 
$P \wred Q$ and $Q \downarrow_{\mathcal N} x$.
\end{definition}

\begin{definition}
%\label{def.bbisim}
An  ${\mathcal N}$-\emph{barbed bisimulation} over a set of names, ${\mathcal N}$, is a symmetric binary relation 
${\mathcal S}_{\mathcal N}$ between agents such that $P\rel{S}_{\mathcal N}Q$ implies:
\begin{enumerate}
\item If $P \red P'$ then $Q \wred Q'$ and $P'\rel{S}_{\mathcal N} Q'$.
\item If $P\downarrow_{\mathcal N} x$, then $Q\Downarrow_{\mathcal N} x$.
\end{enumerate}
$P$ is ${\mathcal N}$-barbed bisimilar to $Q$, written
$P \wbbisim_{\mathcal N} Q$, if $P \rel{S}_{\mathcal N} Q$ for some ${\mathcal N}$-barbed bisimulation ${\mathcal S}_{\mathcal N}$.
\end{definition}

$\mathcal{R} \subseteq \pi \times \pi$

$P \mathcal{R} Q => \forall P'. P \red P' \Rightarrow \exists Q'. Q \red Q', P' \mathcal{R} Q'$

$P \vdash x \Rightarrow Q \vdash x$

\begin{mathpar}
  \inferrule*[lab=Out-barb]{x \nameeq y}{{y}!\langle{Q}\rangle \vdash x}
  \and
  \inferrule*[lab=Par-barb]{\mbox{$P\vdash x$ or $Q\vdash x$}}{\binpar{P}{Q} \vdash x}
\end{mathpar}

\subsubsection{Contexts}

One of the principle advantages of computational calculi like the
$\pi$-calculus is a well-defined notion of context,
contextual-equivalence and a correlation between
contextual-equivalence and notions of bisimulation. The notion of
context allows the decomposition of a process into (sub-)process and
its syntactic environment, its context. Thus, a context may be
thought of as a process with a ``hole'' (written $\Box$) in it. The
application of a context $M$ to a process $P$, written $M[P]$, is
tantamount to filling the hole in $M$ with $P$. In this paper we do
not need the full weight of this theory, but do make use of the notion
of context in the proof the main theorem. 

\begin{mathpar}
  \inferrule* [lab=summation] {} {{M_{M},M_{N}} \bc \Box \;|\; x.M_{A} \;|\; M_{M}+M_{N}}
  \and
  \inferrule* [lab=agent] {} {{M_{A}} \bc (\vec{x})M_{P} \;| \; \clift{P_0,\ldots,M_{P},\ldots,P_N}}
  \and \\
  \inferrule* [lab=process] {} {{M_{P}} \bc M_{N} \;| \;P|M_{P} }
\end{mathpar} 

\begin{mathpar}
  \inferrule* [lab=sychronization] {} {M_{N} \bc \Box \;|\; x?M_{F} \;|\; x!M_{C}}
  \and
  \inferrule* [lab=abstraction] {} {{M_{F}} \bc (x)M_{P} }
  \and
  \inferrule* [lab=concretion] {} {{M_{C}} \bc \langle M_{P} \rangle }
  \and \\
  \inferrule* [lab=process] {} {{M_{P}} \bc M_{N} \;| \;P|M_{P} }
\end{mathpar}

\begin{definition}[contextual application] Given a context $M$, and
  process $P$, we define the \emph{contextual application}, $M[P] :=
  M\{P/\Box\}$. That is, the contextual application of M to P is the
  substitution of $P$ for $\Box$ in $M$.
\end{definition}

$\meaningof{-} : L \to \mathcal{P}(\pi)$

\begin{mathpar}
  \inferrule* [lab=collection] {} {\meaningof{true} = \pi, \and \meaningof{~E} = \pi \setminus \meaningof{E}, \and \meaningof{E_{1} \& E_{2}} = \meaningof{E_{1}} \cap \meaningof{E_{2}}}
\end{mathpar}

\begin{mathpar}
  \inferrule* [lab=structure] {} {\meaningof{0} = \{ P \in \pi | P \equiv 0 \}, \and \\ \meaningof{E_1 | E_2} = \{ P \in \pi | P \equiv P_{1} | P_{2}, P_{1} \in \meaningof{E_{1}}, P_{2} \in \meaningof{E_2}\} }
\end{mathpar}

\begin{mathpar}
 \inferrule* [lab=behavior] {} {\meaningof{\langle a?b \rangle E} = \{ P \in \pi | P \equiv Q | u?(y)P', \\ \and \\\\ \and \\ \;\;\; u \in \meaningof{a}, \forall z.P'\{z/y\} \in \meaningof{E\{z/b\}}\}, \and \\ \meaningof{a!E} = \{ P \in \pi | P \equiv Q | x!\langle P' \rangle, x \in \meaningof{a} P' \in \meaningof{E}\} }
\end{mathpar}

\begin{mathpar}
 \inferrule* [lab=nominal] {} {\meaningof{\quotep{E}} = \{ \quotep{P} \in \quotep{\pi} | P \in \meaningof{E} \}, \and \meaningof{\quotep{P}} = \{ \quotep{Q} \in \quotep{\pi} | P \equiv Q \} \and \\ \meaningof{@\quotep{E}} = \{ P \in \pi | P \equiv @x, x \in \meaningof{E} \}}
\end{mathpar}

\begin{eqnarray*}
  \\
  \meaningof{-} : TS \to ST
\end{eqnarray*}

\begin{eqnarray*}
  \\
  L : TS \to ST
\end{eqnarray*}

\begin{eqnarray*}
  \\
  P \models E \iff P \in \meaningof{E}
\end{eqnarray*}

\begin{eqnarray*}
  P \approx_{L} Q \iff \forall E \in L. P \models E \iff Q \models E
\end{eqnarray*}

\begin{eqnarray*}
  P \approx_{K} Q
\end{eqnarray*}

\begin{eqnarray*}
  P \approx Q
\end{eqnarray*}

$\approx_{K} = \approx = \approx_{L}$

\subsubsection{Contextual duality}

Note that contexts extend the quotation operation to a family of
operations from processes to names. Given a context, $M$, we can
define a \emph{nominal context}, $\quotep{M}$ by $\quotep{M}[P] :=
\quotep{M[P]}$. To foreshadow what is to come we observe that these
operations enjoy a duality with processes very much like the duality
between vectors and maps from vectors to scalars.

Further, because the calculus is essentially higher-order, we have a
correspondence between contexts and processes. More specifically,
given a name $x$ and a context $M$ we can construct $M^{*}_{x}$ such
that 

\begin{mathpar}
  M^{*}_{x} | \lift{x}{P} \red M[P]
\end{mathpar}

namely,

\begin{mathpar}
  M^{*}_{x} := x?(u).M[\dropn{u}]
\end{mathpar}

The dependence of $M^{*}_{x}$ on a name makes it an abstraction, 

\begin{mathpar}
  M^{*} := (x)x?(u).M[\dropn{u}]
\end{mathpar}

\subsection{Additional notation}

It will sometimes be convenient to denote the process a name
quotes. We already have the notation $x = \quotep{P}$, but it will be
convenient to introduce an alternate notation, $\procn{x}$, when we
want to emphasize the connection to the use of the name. Note that, by
virtue of name equivalence, $\quotep{\procn{x}} \nameeq x$; so, the
notation is consistent with previous definitions.

Further, because names have structure it is possible to effect
substitutions on the basis of that structure. This means we need to
upgrade our notation for substitutions, which we accomplish by
adapting comprehension notation. Thus,

\begin{mathpar}
  P\{ y / x : x \in S \}
\end{mathpar}

is interpreted to mean the process derived from P by replacing (in a
capture-avoiding manner) each occurrence of $x$ in $S$ by $y$. For example,

\begin{mathpar}
  P\{ \quotep{\procn{x}|\procn{x}} / x : x \in \freenames{P} \}
\end{mathpar}

will replace each (occurrence) of a free name $x$ in $P$ by
$\quotep{\procn{x}|\procn{x}}$.

Also, we will avail ourselves of the notation $x^{L}$ and $x^{R}$ to
denote injections of a name into disjoint copies of the name
space. There are numerous ways to accomplish this. One example can be
found in \cite{MeredithR05}. This notation overloads to vectors of
names: $\vec{x}^{\pi} := (x_{i}^{\pi} \; : \; 0 \leq i < |\vec{x}| )$ where $\pi \in \{L,R\}$.

We also use $P^{\Box} := P|\Box$.

In \cite{MeredithR05} an interpretation of the new operator is
given. It turns out that there are several possible interpretations
all enjoying the requisite algebraic properties of the operator (see
\cite{milner91polyadicpi}). We will therefore make liberal use of
$(\nu\; \vec{x})P$.

% subsection the_syntax_and_semantics_of_the_notation_system (end)   

\input{qm2pi.qmops} 

\input{qm2pi.sterngerlach} 

\input{qm2pi.metric} 

% section concurrent_process_calculi (end)

%\input{qm2pi.proofsketch}

% section proof sketch (end)

%\input{qm2pi.slviaknots} 

% section spatial logic via knots (end)

\input{qm2pi.conclusion}

% section conclusion (end)

%\input{qm2pi.dtcodes} 

% section wiring algorithm (end)

\input{qm2pi.ack} 

% section acknowledgments (end)

\newpage


\bibliographystyle{plain}   
\bibliography{../../biblios/main.bib}

\input{qm2pi.rhodetails}

\end{document}

 

%\documentclass[12pt]{llncs}
%\documentclass{jktr}

\usepackage[pdftex]{hyperref}                   
\usepackage {listings}
\usepackage {mathpartir}
\usepackage{bcprules}
%\usepackage{listings}
                       
\usepackage{graphicx} 
%\usepackage[margins=2.5cm,nohead,nofoot]{geometry}
%\usepackage{geometry}
\usepackage{amsfonts}
\usepackage{amstext}
\usepackage{latexsym}
\usepackage{amssymb}
\usepackage{color}


%\include{myPreamble}
\include{qm2pi.local} 

%\ifpdf
%\usepackage[pdftex]{graphicx}
%\else
%\usepackage{graphicx}
%\fi

 % \ifpdf
%  \usepackage{pdfsync}
%  \if


%\title{Brief Article}
%\author{David F. Snyder}
%\author{L.G. Meredith}

%\address{Dept. of Math., Texas State University--San Marcos, San Marcos, TX 78666}
       
\pagestyle{empty}


\begin{document}

\lstset{language=[Objective]Caml,frame=shadowbox}

\input{qm2pi.front}

% section front matter (end)

\input{qm2pi.intro} 
 
% section introduction (end)

% \input{qm2pi.knotations} 

% section notation (end)

\input{qm2pi.process.calculi} 

% section concurrent_process_calculi_and_spatial_logics_ (end)
    
%\input{qm2pi.knots2pi} 

%\input{qm2pi.trefoil} 

%\input{qm2pi.mainthm} 

% subsection basic_interpretation (end)

%\input{qm2pi.rho.presentation} 
\subsection{The syntax and semantics of the notation system}\label{sub:the_syntax_and_semantics_of_the_notation_system} % (fold)

We now summarize a technical presentation of the calculus that
embodies our theory of dynamics. The typical presentation of such a
calculus follows the style of giving generators and relations on
them. The grammar, below, describing term constructors, freely
generates the set of processes, $\Proc$. This set is then quotiented
by a relation known as structural congruence and it is over this set
that the notion of dynamics is expressed. This presentation is
essentially that of \cite{MeredithR05} with the addition of
polyadicity and summation. For readability we have relegated some of
the technical subtleties to an appendix.

\subsubsection{Process grammar}\label{subsub:process_grammar}

\begin{mathpar}
  \inferrule* [lab=synchronization] {} {{M} \bc \pzero \;|\; x?F \;|\; x!C }
  \and
  \inferrule* [lab=abstraction] {} {{F} \bc (x)P}
  \and
  \inferrule* [lab=concretion] {} {{C} \bc \langle Q \rangle}
  \and
  \inferrule* [lab=process] {} {{P,Q} \bc M \;| \;P|Q \;|\; @{x}}
  \and
  \inferrule* [lab=name] {} {{x} \bc \quotep{P}}
\end{mathpar} 

Note that $\vec{x}$ (resp. $\vec{P}$) denotes a vector of names
(resp. processes) of length $|\vec{x}|$ (resp. $|\vec{P}|$). We adopt
the following useful abbreviations.

\begin{mathpar}
   x?(\vec{y}).P := x.(\vec{y})P \and  x\clift{\vec{P}} := x.\clift{\vec{P}}
   \and x!(y) := \lift{x}{\dropn{y}}
   \and \Pi_{i=0}^{n-1}P_i := P_0 | \ldots | P_{n-1}
\end{mathpar}

\subsubsection{Structural congruence}

\paragraph{Free and bound names and alpha-equivalence.} At the
core of structural equivalence is alpha-equivalence which identifies
process that are the same up to a change of variable. Formally, we
recognize the distinction between free and bound names. The free names
of a process, $\freenames{P}$, may be calculated recursively as
follows:

\begin{mathpar}
\freenames{\pzero} := \emptyset
  \and \\
  \freenames{x?(y).P} := \{ x \} \cup (\freenames{P} \setminus \{ y \})
  \and 
  \freenames{x!\langle P \rangle} := \{ x \} \cup \{ P \} 
  \and \\
  \freenames{P|Q} := \freenames{P} \cup \freenames{Q}
  \and \\
  \freenames{@{x}} := \{ x \}
\end{mathpar}

$\pi$
$\quotep{\pi}$

$\freenames{-} : \pi \to \mathcal{P}(\quotep{\pi})$

\begin{eqnarray*}
  \freenames{\pzero} & := & \emptyset \\
  \freenames{x?(y).P} & := & \{ x \} \cup (\freenames{P} \setminus \{ y \}) \\
  \freenames{x!\langle P \rangle} & := & \{ x \} \cup \{ P \} \\
  \freenames{P|Q} & := & \freenames{P} \cup \freenames{Q} \\
  \freenames{\dropn{x}} & := & \{ x \}
\end{eqnarray*}

The bound names of a process, $\boundnames{P}$, are those names occurring in $P$
that are not free. For example, in $x?(y).0$, the name $x$ is free, while $y$ is bound.

\begin{mathpar}
  \inferrule* [lab=monoidal-laws] {} { P|Q \equiv Q|P \and P|0 \equiv P \and P|(Q|R) \equiv (P|Q)|R }
\end{mathpar}

\begin{mathpar}
  \inferrule* [lab=alpha-equivalence] {} { (x)P \equiv (y)P\{y/x\} \and y \not\in \freenames{P} }
\end{mathpar}

\begin{definition}
Then two processes, $P,Q$, are alpha-equivalent if $P = Q\{\vec{y}/\vec{x}\}$ for
some $\vec{x} \in \boundnames{Q},\vec{y} \in \boundnames{P}$, where $Q\{\vec{y}/\vec{x}\}$
denotes the capture-avoiding substitution of $\vec{y}$ for $\vec{x}$ in $Q$.
\end{definition}

\begin{definition}
  The {\em structural congruence} \cite{SangiorgiWalker} , $\equiv$,
  between processes is the least congruence containing
  alpha-equivalence, satisfying the abelian monoid laws
  (associativity, commutativity and $\pzero$ as identity) for parallel
  composition $|$ and for summation $+$.
\end{definition}

\subsection{Name equivalence}

We take name equivalence, written $\nameeq$, to be the smallest
equivalence relation generated by the following rules.

\begin{mathpar}
\inferrule*[lab=Quote-drop]
{ }
{ \quotep{@{x}} \nameeq x }

\inferrule*[lab=Struct-equiv]
{ P \scong Q }
{ \quotep{P} \nameeq \quotep{Q} }
\end{mathpar}

The astute reader will have noticed that the mutual recursion of names
and processes imposes a mutual recursion on alpha-equivalence and
structural equivalence via name-equivalence. Fortunately, all of this
works out pleasantly and we may calculate in the natural way, free of
concern. The reader interested in the details is referred to the
appendix \ref{appendix:rho_details}.

\subsection{Substitution}

We use $\Proc$ for the set of processes, $\QProc$ for the set of
names, and $\id{\{}\vec{y} / \vec{x} \id{\}}$ to denote partial maps,
$s : \QProc \rightarrow \QProc$. A map, $s$ lifts, uniquely, to a map
on process terms, $\widehat{s} : \Proc \rightarrow \Proc$ by the
following equations.

\begin{mathpar}
  (0) \psubstp{Q}{P} := 0 \\
  (R \juxtap S) \psubstp{Q}{P}
  :=    
  (R)\psubstp{Q}{P} \juxtap (S) \psubstp{Q}{P} \\
  (x?(y).R) \psubstp{Q}{P}    
  :=    
  (x)\substp{Q}{P} (z)\concat( (R \psubstn{z}{y}) \psubstp{Q}{P} ) \\
  (\lift{x}{R}) \psubstp{Q}{P}  
  :=
  \lift{(x)\substp{Q}{P}}{ R \psubstp{Q}{P} } \\
%   (\dropn{x})  \psubstp{Q}{P}       
%   := 
%   \left\{ 
%     \begin{array}{ccc} 
%       \dropn{\quotep{Q}} & & x \nameeq \quotep{P} \\
%       \dropn{x} & & otherwise \\
%     \end{array}
%   \right. 
  (\dropn{x})  \psubstp{Q}{P}       
  := 
  \left\{ 
    \begin{array}{ccc} 
      Q & & x \nameeq \quotep{P} \\
      \dropn{x} & & otherwise \\
    \end{array}
  \right.
\end{mathpar}
 

where

\begin{eqnarray}
  (x)\id{\{} \lpquote Q \rpquote / \lpquote P \rpquote \id{\}}            = 
  \left\{ 
    \begin{array}{ccc}
      \lpquote Q \rpquote & & x \nameeq \lpquote P \rpquote \\
      x & & otherwise \\
    \end{array}
  \right. \nonumber
\end{eqnarray}

and $z$ is chosen distinct from $\quotep{P}$, $\quotep{Q}$, the free
names in $Q$, and all the names in $R$. Our $\alpha$-equivalence will
be built in the standard way from this substitution.

\begin{remark}\label{rem:no_self_referential_names}
  One consequence of these definitions is that $\forall P. \quotep{P}
  \not\in \freenames{P}$.
\end{remark}

\subsection{ Dynamic quote: an example }

Anticipating something of what's to come, consider applying the
substitution, $\widehat{\id{\{}u / z \id{\}}}$, to the following pair
of processes, $\lift{w}{y!(z)}$ and $w[ \lpquote y!(z) \rpquote ]$.

\begin{eqnarray}
	\lift{w}{y!(z)}\widehat{\id{\{}u / z \id{\}}}
		& = &
		\lift{w}{y!(u)} \nonumber\\
	w[ \lpquote y!(z) \rpquote ] \widehat{ \id{\{}u / z \id{\}} }
		& = &
		w[ \lpquote y!(z) \rpquote ] \nonumber
\end{eqnarray}

Because the body of the process between quotes is impervious to
substitution, we get radically different answers. In fact, by
examining the first process in an input context,
e.g. $x?(z).\lift{w}{y!(z)}$, we see that the process under the lift
operator may be shaped by prefixed inputs binding a name inside it. In
this sense, the lift operator will be seen as a way to dynamically
construct processes before reifying them as names.

Finally equipped with these standard features we can present the
dynamics of the calculus.

\subsubsection{Operational semantics} 

Finally, we introduce the computational dynamics. What marks these
algebras as distinct from other more traditionally studied algebraic
structures, e.g. vector spaces or polynomial rings, is the manner in
which dynamics is captured. In traditional structures, dynamics is typically
expressed through morphisms between such structures, as in linear maps
between vector spaces or morphisms between rings. In algebras
associated with the semantics of computation, the dynamics is
expressed as part of the algebraic structure itself, through a
reduction reduction relation typically denoted by $\red$. Below, we
give a recursive presentation of this relation for the calculus used
in the encoding.

$\red \subseteq \pi \times \pi$
$\red : \pi \to \mathcal{P}(\pi)$

\begin{mathpar}
  \inferrule* [lab=Comm] { \textsf{match}( x_{src}, x_{trgt} ) } { x_{trgt}?(y)P \; | \; x_{src}!\langle {Q} \rangle \red P\{\quotep{Q}/y}\} }
  \and \\
  \inferrule* [lab=Par] {{P} \red {P}'} {{{P} | {Q}} \red {{P}' | {Q}}}
  \and
  \inferrule* [lab=Equiv]{{{P} \scong {P}'} \andalso {{P}' \red {Q}'} \andalso {{Q}' \scong {Q}}}{{P} \red {Q}}
\end{mathpar}

\begin{eqnarray*}
  match_{\equiv} (\quotep{P},\quotep{Q}) & := & P \equiv Q \\
  match_{\dagger}(\quotep{P},\quotep{Q}) & := & \forall R. P|Q \red^{*} R => R \red^{*} 0 \\
  match_{K}(\quotep{P},\quotep{Q}) & := & K \mbox{ for some context } K
\end{eqnarray*}

$u?(x)P | u!\langle Q \rangle \red P\{\quotep{Q}/x\}$

%We write $\wred$ for $\red^*$, and $P\red$ if $\exists Q $ such that $ P \red Q$.
We write $P\red$ if $\exists Q $ such that $ P \red Q$ and $P\not\red$, otherwise.

\section{Replication}

As mentioned before, it is known that replication (and hence
recursion) can be implemented in a higher-order process algebra
\cite{SangiorgiWalker}. As our first example of calculation with the
machinery thus far presented we give the construction explicitly in
the {\rhoc}.

\begin{eqnarray}
	D_{x} & := & \prefix{x}{y}{(\binpar{\outputp{x}{y}}{@{y}})} \nonumber\\
	\bangp_{x}{P} & := & \binpar{{x}!\langle{\binpar{D_{x}}{P}}\rangle}{D_{x}} \nonumber
\end{eqnarray}

\begin{eqnarray}
	\bangp_{x}{P} & & \nonumber\\
	=
	& {x}!\langle{(\prefix{x}{y}{(\outputp{x}{y} | @{y})) | P}}\rangle 
	      | \prefix{x}{y}{(\outputp{x}{y} | @{y})} & \nonumber\\
	\red
	& (\outputp{x}{y} | @{y})\substn{\quotep{(\prefix{x}{y}{(@{y} | \outputp{x}{y})) | P}}}{y} & \nonumber\\
	=
	& \outputp{x}{\quotep{(\prefix{x}{y}{(\outputp{x}{y} | @{y})) | P}}}
	  | {(\prefix{x}{y}{(\outputp{x}{y} | @{y})) | P}} & \nonumber\\
	\red
	& \ldots & \nonumber\\
	\red^*
	& P | P | \ldots & \nonumber
\end{eqnarray}

Of course, this encoding, as an implementation, runs away, unfolding
$\bangp{P}$ eagerly. A lazier and more implementable replication
operator, restricted to input-guarded processes, may be obtained as follows.

\begin{eqnarray}
\bangp{\prefix{u}{v}{P}} 
	:= 
	\binpar{\lift{x}{\prefix{u}{v}{(\binpar{D(x)}{P})}}}{D(x)} \nonumber
\end{eqnarray}

\begin{remark}
  Note that the lazier definition still does not deal with summation
  or mixed summation (i.e. sums over input and output). The reader is
  invited to construct definitions of replication that deal with these
  features. 

  Further, the definitions are parameterized in a name, $x$. Can you,
  gentle reader, make a definition that eliminates this parameter and
  guarantees no accidental interaction between the replication
  machinery and the process being replicated -- i.e. no accidental
  sharing of names used by the process to get its work done and the
  name(s) used by the replication to effect copying. This latter
  revision of the definition of replication is crucial to obtaining
  the expected identity $!!P \sim !P$.
\end{remark}

\begin{remark}\label{rem:paradoxical_combinator}
  The reader familiar with the lambda calculus will have noticed the
  similarity between $D$ and the paradoxical combinator.

  [Ed. note: the existence of this seems to suggest we have to be more
  restrictive on the set of processes and names we admit if we are to
  support no-cloning.]
\end{remark}

\subsubsection{Bisimulation}

The computational dynamics gives rise to another kind of equivalence,
the equivalence of computational behavior. As previously mentioned
this is typically captured \emph{via} some form of bisimulation.

% The notion we use in this paper is weak barbed bisimulation
% \cite{milner91polyadicpi}.

The notion we use in this paper is derived from weak barbed
bisimulation \cite{milner91polyadicpi}. 

\begin{definition}
An \emph{observation relation}, $\downarrow_{\mathcal N}$, over a set
of names, $\mathcal N$, is the smallest relation satisfying the rules
below.

\infrule[Out-barb]{y \in {\mathcal N}, \; x \nameeq y}
		  {\outputp{x}{v} \downarrow_{\mathcal N} x}
\infrule[Par-barb]{\mbox{$P\downarrow_{\mathcal N} x$ or $Q\downarrow_{\mathcal N} x$}}
		  {\binpar{P}{Q} \downarrow_{\mathcal N} x}

We write $P \Downarrow_{\mathcal N} x$ if there is $Q$ such that 
$P \wred Q$ and $Q \downarrow_{\mathcal N} x$.
\end{definition}

\begin{definition}
%\label{def.bbisim}
An  ${\mathcal N}$-\emph{barbed bisimulation} over a set of names, ${\mathcal N}$, is a symmetric binary relation 
${\mathcal S}_{\mathcal N}$ between agents such that $P\rel{S}_{\mathcal N}Q$ implies:
\begin{enumerate}
\item If $P \red P'$ then $Q \wred Q'$ and $P'\rel{S}_{\mathcal N} Q'$.
\item If $P\downarrow_{\mathcal N} x$, then $Q\Downarrow_{\mathcal N} x$.
\end{enumerate}
$P$ is ${\mathcal N}$-barbed bisimilar to $Q$, written
$P \wbbisim_{\mathcal N} Q$, if $P \rel{S}_{\mathcal N} Q$ for some ${\mathcal N}$-barbed bisimulation ${\mathcal S}_{\mathcal N}$.
\end{definition}

$\mathcal{R} \subseteq \pi \times \pi$

$P \mathcal{R} Q => \forall P'. P \red P' \Rightarrow \exists Q'. Q \red Q', P' \mathcal{R} Q'$

$P \vdash x \Rightarrow Q \vdash x$

\begin{mathpar}
  \inferrule*[lab=Out-barb]{x \nameeq y}{{y}!\langle{Q}\rangle \vdash x}
  \and
  \inferrule*[lab=Par-barb]{\mbox{$P\vdash x$ or $Q\vdash x$}}{\binpar{P}{Q} \vdash x}
\end{mathpar}

\subsubsection{Contexts}

One of the principle advantages of computational calculi like the
$\pi$-calculus is a well-defined notion of context,
contextual-equivalence and a correlation between
contextual-equivalence and notions of bisimulation. The notion of
context allows the decomposition of a process into (sub-)process and
its syntactic environment, its context. Thus, a context may be
thought of as a process with a ``hole'' (written $\Box$) in it. The
application of a context $M$ to a process $P$, written $M[P]$, is
tantamount to filling the hole in $M$ with $P$. In this paper we do
not need the full weight of this theory, but do make use of the notion
of context in the proof the main theorem. 

\begin{mathpar}
  \inferrule* [lab=summation] {} {{M_{M},M_{N}} \bc \Box \;|\; x.M_{A} \;|\; M_{M}+M_{N}}
  \and
  \inferrule* [lab=agent] {} {{M_{A}} \bc (\vec{x})M_{P} \;| \; \clift{P_0,\ldots,M_{P},\ldots,P_N}}
  \and \\
  \inferrule* [lab=process] {} {{M_{P}} \bc M_{N} \;| \;P|M_{P} }
\end{mathpar} 

\begin{mathpar}
  \inferrule* [lab=sychronization] {} {M_{N} \bc \Box \;|\; x?M_{F} \;|\; x!M_{C}}
  \and
  \inferrule* [lab=abstraction] {} {{M_{F}} \bc (x)M_{P} }
  \and
  \inferrule* [lab=concretion] {} {{M_{C}} \bc \langle M_{P} \rangle }
  \and \\
  \inferrule* [lab=process] {} {{M_{P}} \bc M_{N} \;| \;P|M_{P} }
\end{mathpar}

\begin{definition}[contextual application] Given a context $M$, and
  process $P$, we define the \emph{contextual application}, $M[P] :=
  M\{P/\Box\}$. That is, the contextual application of M to P is the
  substitution of $P$ for $\Box$ in $M$.
\end{definition}

$\meaningof{-} : L \to \mathcal{P}(\pi)$

\begin{mathpar}
  \inferrule* [lab=collection] {} {\meaningof{true} = \pi, \and \meaningof{~E} = \pi \setminus \meaningof{E}, \and \meaningof{E_{1} \& E_{2}} = \meaningof{E_{1}} \cap \meaningof{E_{2}}}
\end{mathpar}

\begin{mathpar}
  \inferrule* [lab=structure] {} {\meaningof{0} = \{ P \in \pi | P \equiv 0 \}, \and \\ \meaningof{E_1 | E_2} = \{ P \in \pi | P \equiv P_{1} | P_{2}, P_{1} \in \meaningof{E_{1}}, P_{2} \in \meaningof{E_2}\} }
\end{mathpar}

\begin{mathpar}
 \inferrule* [lab=behavior] {} {\meaningof{\langle a?b \rangle E} = \{ P \in \pi | P \equiv Q | u?(y)P', \\ \and \\\\ \and \\ \;\;\; u \in \meaningof{a}, \forall z.P'\{z/y\} \in \meaningof{E\{z/b\}}\}, \and \\ \meaningof{a!E} = \{ P \in \pi | P \equiv Q | x!\langle P' \rangle, x \in \meaningof{a} P' \in \meaningof{E}\} }
\end{mathpar}

\begin{mathpar}
 \inferrule* [lab=nominal] {} {\meaningof{\quotep{E}} = \{ \quotep{P} \in \quotep{\pi} | P \in \meaningof{E} \}, \and \meaningof{\quotep{P}} = \{ \quotep{Q} \in \quotep{\pi} | P \equiv Q \} \and \\ \meaningof{@\quotep{E}} = \{ P \in \pi | P \equiv @x, x \in \meaningof{E} \}}
\end{mathpar}

\begin{eqnarray*}
  \\
  \meaningof{-} : TS \to ST
\end{eqnarray*}

\begin{eqnarray*}
  \\
  L : TS \to ST
\end{eqnarray*}

\begin{eqnarray*}
  \\
  P \models E \iff P \in \meaningof{E}
\end{eqnarray*}

\begin{eqnarray*}
  P \approx_{L} Q \iff \forall E \in L. P \models E \iff Q \models E
\end{eqnarray*}

\begin{eqnarray*}
  P \approx_{K} Q
\end{eqnarray*}

\begin{eqnarray*}
  P \approx Q
\end{eqnarray*}

$\approx_{K} = \approx = \approx_{L}$

\subsubsection{Contextual duality}

Note that contexts extend the quotation operation to a family of
operations from processes to names. Given a context, $M$, we can
define a \emph{nominal context}, $\quotep{M}$ by $\quotep{M}[P] :=
\quotep{M[P]}$. To foreshadow what is to come we observe that these
operations enjoy a duality with processes very much like the duality
between vectors and maps from vectors to scalars.

Further, because the calculus is essentially higher-order, we have a
correspondence between contexts and processes. More specifically,
given a name $x$ and a context $M$ we can construct $M^{*}_{x}$ such
that 

\begin{mathpar}
  M^{*}_{x} | \lift{x}{P} \red M[P]
\end{mathpar}

namely,

\begin{mathpar}
  M^{*}_{x} := x?(u).M[\dropn{u}]
\end{mathpar}

The dependence of $M^{*}_{x}$ on a name makes it an abstraction, 

\begin{mathpar}
  M^{*} := (x)x?(u).M[\dropn{u}]
\end{mathpar}

\subsection{Additional notation}

It will sometimes be convenient to denote the process a name
quotes. We already have the notation $x = \quotep{P}$, but it will be
convenient to introduce an alternate notation, $\procn{x}$, when we
want to emphasize the connection to the use of the name. Note that, by
virtue of name equivalence, $\quotep{\procn{x}} \nameeq x$; so, the
notation is consistent with previous definitions.

Further, because names have structure it is possible to effect
substitutions on the basis of that structure. This means we need to
upgrade our notation for substitutions, which we accomplish by
adapting comprehension notation. Thus,

\begin{mathpar}
  P\{ y / x : x \in S \}
\end{mathpar}

is interpreted to mean the process derived from P by replacing (in a
capture-avoiding manner) each occurrence of $x$ in $S$ by $y$. For example,

\begin{mathpar}
  P\{ \quotep{\procn{x}|\procn{x}} / x : x \in \freenames{P} \}
\end{mathpar}

will replace each (occurrence) of a free name $x$ in $P$ by
$\quotep{\procn{x}|\procn{x}}$.

Also, we will avail ourselves of the notation $x^{L}$ and $x^{R}$ to
denote injections of a name into disjoint copies of the name
space. There are numerous ways to accomplish this. One example can be
found in \cite{MeredithR05}. This notation overloads to vectors of
names: $\vec{x}^{\pi} := (x_{i}^{\pi} \; : \; 0 \leq i < |\vec{x}| )$ where $\pi \in \{L,R\}$.

We also use $P^{\Box} := P|\Box$.

In \cite{MeredithR05} an interpretation of the new operator is
given. It turns out that there are several possible interpretations
all enjoying the requisite algebraic properties of the operator (see
\cite{milner91polyadicpi}). We will therefore make liberal use of
$(\nu\; \vec{x})P$.

% subsection the_syntax_and_semantics_of_the_notation_system (end)   

\input{qm2pi.qmops} 

\input{qm2pi.sterngerlach} 

\input{qm2pi.metric} 

% section concurrent_process_calculi (end)

%\input{qm2pi.proofsketch}

% section proof sketch (end)

%\input{qm2pi.slviaknots} 

% section spatial logic via knots (end)

\input{qm2pi.conclusion}

% section conclusion (end)

%\input{qm2pi.dtcodes} 

% section wiring algorithm (end)

\input{qm2pi.ack} 

% section acknowledgments (end)

\newpage


\bibliographystyle{plain}   
\bibliography{../../biblios/main.bib}

\input{qm2pi.rhodetails}

\end{document}

 

% subsection basic_interpretation (end)

%\input{qm2pi.rho.presentation} 
\subsection{The syntax and semantics of the notation system}\label{sub:the_syntax_and_semantics_of_the_notation_system} % (fold)

We now summarize a technical presentation of the calculus that
embodies our theory of dynamics. The typical presentation of such a
calculus follows the style of giving generators and relations on
them. The grammar, below, describing term constructors, freely
generates the set of processes, $\Proc$. This set is then quotiented
by a relation known as structural congruence and it is over this set
that the notion of dynamics is expressed. This presentation is
essentially that of \cite{MeredithR05} with the addition of
polyadicity and summation. For readability we have relegated some of
the technical subtleties to an appendix.

\subsubsection{Process grammar}\label{subsub:process_grammar}

\begin{mathpar}
  \inferrule* [lab=synchronization] {} {{M} \bc \pzero \;|\; x?F \;|\; x!C }
  \and
  \inferrule* [lab=abstraction] {} {{F} \bc (x)P}
  \and
  \inferrule* [lab=concretion] {} {{C} \bc \langle Q \rangle}
  \and
  \inferrule* [lab=process] {} {{P,Q} \bc M \;| \;P|Q \;|\; @{x}}
  \and
  \inferrule* [lab=name] {} {{x} \bc \quotep{P}}
\end{mathpar} 

Note that $\vec{x}$ (resp. $\vec{P}$) denotes a vector of names
(resp. processes) of length $|\vec{x}|$ (resp. $|\vec{P}|$). We adopt
the following useful abbreviations.

\begin{mathpar}
   x?(\vec{y}).P := x.(\vec{y})P \and  x\clift{\vec{P}} := x.\clift{\vec{P}}
   \and x!(y) := \lift{x}{\dropn{y}}
   \and \Pi_{i=0}^{n-1}P_i := P_0 | \ldots | P_{n-1}
\end{mathpar}

\subsubsection{Structural congruence}

\paragraph{Free and bound names and alpha-equivalence.} At the
core of structural equivalence is alpha-equivalence which identifies
process that are the same up to a change of variable. Formally, we
recognize the distinction between free and bound names. The free names
of a process, $\freenames{P}$, may be calculated recursively as
follows:

\begin{mathpar}
\freenames{\pzero} := \emptyset
  \and \\
  \freenames{x?(y).P} := \{ x \} \cup (\freenames{P} \setminus \{ y \})
  \and 
  \freenames{x!\langle P \rangle} := \{ x \} \cup \{ P \} 
  \and \\
  \freenames{P|Q} := \freenames{P} \cup \freenames{Q}
  \and \\
  \freenames{@{x}} := \{ x \}
\end{mathpar}

$\pi$
$\quotep{\pi}$

$\freenames{-} : \pi \to \mathcal{P}(\quotep{\pi})$

\begin{eqnarray*}
  \freenames{\pzero} & := & \emptyset \\
  \freenames{x?(y).P} & := & \{ x \} \cup (\freenames{P} \setminus \{ y \}) \\
  \freenames{x!\langle P \rangle} & := & \{ x \} \cup \{ P \} \\
  \freenames{P|Q} & := & \freenames{P} \cup \freenames{Q} \\
  \freenames{\dropn{x}} & := & \{ x \}
\end{eqnarray*}

The bound names of a process, $\boundnames{P}$, are those names occurring in $P$
that are not free. For example, in $x?(y).0$, the name $x$ is free, while $y$ is bound.

\begin{mathpar}
  \inferrule* [lab=monoidal-laws] {} { P|Q \equiv Q|P \and P|0 \equiv P \and P|(Q|R) \equiv (P|Q)|R }
\end{mathpar}

\begin{mathpar}
  \inferrule* [lab=alpha-equivalence] {} { (x)P \equiv (y)P\{y/x\} \and y \not\in \freenames{P} }
\end{mathpar}

\begin{definition}
Then two processes, $P,Q$, are alpha-equivalent if $P = Q\{\vec{y}/\vec{x}\}$ for
some $\vec{x} \in \boundnames{Q},\vec{y} \in \boundnames{P}$, where $Q\{\vec{y}/\vec{x}\}$
denotes the capture-avoiding substitution of $\vec{y}$ for $\vec{x}$ in $Q$.
\end{definition}

\begin{definition}
  The {\em structural congruence} \cite{SangiorgiWalker} , $\equiv$,
  between processes is the least congruence containing
  alpha-equivalence, satisfying the abelian monoid laws
  (associativity, commutativity and $\pzero$ as identity) for parallel
  composition $|$ and for summation $+$.
\end{definition}

\subsection{Name equivalence}

We take name equivalence, written $\nameeq$, to be the smallest
equivalence relation generated by the following rules.

\begin{mathpar}
\inferrule*[lab=Quote-drop]
{ }
{ \quotep{@{x}} \nameeq x }

\inferrule*[lab=Struct-equiv]
{ P \scong Q }
{ \quotep{P} \nameeq \quotep{Q} }
\end{mathpar}

The astute reader will have noticed that the mutual recursion of names
and processes imposes a mutual recursion on alpha-equivalence and
structural equivalence via name-equivalence. Fortunately, all of this
works out pleasantly and we may calculate in the natural way, free of
concern. The reader interested in the details is referred to the
appendix \ref{appendix:rho_details}.

\subsection{Substitution}

We use $\Proc$ for the set of processes, $\QProc$ for the set of
names, and $\id{\{}\vec{y} / \vec{x} \id{\}}$ to denote partial maps,
$s : \QProc \rightarrow \QProc$. A map, $s$ lifts, uniquely, to a map
on process terms, $\widehat{s} : \Proc \rightarrow \Proc$ by the
following equations.

\begin{mathpar}
  (0) \psubstp{Q}{P} := 0 \\
  (R \juxtap S) \psubstp{Q}{P}
  :=    
  (R)\psubstp{Q}{P} \juxtap (S) \psubstp{Q}{P} \\
  (x?(y).R) \psubstp{Q}{P}    
  :=    
  (x)\substp{Q}{P} (z)\concat( (R \psubstn{z}{y}) \psubstp{Q}{P} ) \\
  (\lift{x}{R}) \psubstp{Q}{P}  
  :=
  \lift{(x)\substp{Q}{P}}{ R \psubstp{Q}{P} } \\
%   (\dropn{x})  \psubstp{Q}{P}       
%   := 
%   \left\{ 
%     \begin{array}{ccc} 
%       \dropn{\quotep{Q}} & & x \nameeq \quotep{P} \\
%       \dropn{x} & & otherwise \\
%     \end{array}
%   \right. 
  (\dropn{x})  \psubstp{Q}{P}       
  := 
  \left\{ 
    \begin{array}{ccc} 
      Q & & x \nameeq \quotep{P} \\
      \dropn{x} & & otherwise \\
    \end{array}
  \right.
\end{mathpar}
 

where

\begin{eqnarray}
  (x)\id{\{} \lpquote Q \rpquote / \lpquote P \rpquote \id{\}}            = 
  \left\{ 
    \begin{array}{ccc}
      \lpquote Q \rpquote & & x \nameeq \lpquote P \rpquote \\
      x & & otherwise \\
    \end{array}
  \right. \nonumber
\end{eqnarray}

and $z$ is chosen distinct from $\quotep{P}$, $\quotep{Q}$, the free
names in $Q$, and all the names in $R$. Our $\alpha$-equivalence will
be built in the standard way from this substitution.

\begin{remark}\label{rem:no_self_referential_names}
  One consequence of these definitions is that $\forall P. \quotep{P}
  \not\in \freenames{P}$.
\end{remark}

\subsection{ Dynamic quote: an example }

Anticipating something of what's to come, consider applying the
substitution, $\widehat{\id{\{}u / z \id{\}}}$, to the following pair
of processes, $\lift{w}{y!(z)}$ and $w[ \lpquote y!(z) \rpquote ]$.

\begin{eqnarray}
	\lift{w}{y!(z)}\widehat{\id{\{}u / z \id{\}}}
		& = &
		\lift{w}{y!(u)} \nonumber\\
	w[ \lpquote y!(z) \rpquote ] \widehat{ \id{\{}u / z \id{\}} }
		& = &
		w[ \lpquote y!(z) \rpquote ] \nonumber
\end{eqnarray}

Because the body of the process between quotes is impervious to
substitution, we get radically different answers. In fact, by
examining the first process in an input context,
e.g. $x?(z).\lift{w}{y!(z)}$, we see that the process under the lift
operator may be shaped by prefixed inputs binding a name inside it. In
this sense, the lift operator will be seen as a way to dynamically
construct processes before reifying them as names.

Finally equipped with these standard features we can present the
dynamics of the calculus.

\subsubsection{Operational semantics} 

Finally, we introduce the computational dynamics. What marks these
algebras as distinct from other more traditionally studied algebraic
structures, e.g. vector spaces or polynomial rings, is the manner in
which dynamics is captured. In traditional structures, dynamics is typically
expressed through morphisms between such structures, as in linear maps
between vector spaces or morphisms between rings. In algebras
associated with the semantics of computation, the dynamics is
expressed as part of the algebraic structure itself, through a
reduction reduction relation typically denoted by $\red$. Below, we
give a recursive presentation of this relation for the calculus used
in the encoding.

$\red \subseteq \pi \times \pi$
$\red : \pi \to \mathcal{P}(\pi)$

\begin{mathpar}
  \inferrule* [lab=Comm] { \textsf{match}( x_{src}, x_{trgt} ) } { x_{trgt}?(y)P \; | \; x_{src}!\langle {Q} \rangle \red P\{\quotep{Q}/y}\} }
  \and \\
  \inferrule* [lab=Par] {{P} \red {P}'} {{{P} | {Q}} \red {{P}' | {Q}}}
  \and
  \inferrule* [lab=Equiv]{{{P} \scong {P}'} \andalso {{P}' \red {Q}'} \andalso {{Q}' \scong {Q}}}{{P} \red {Q}}
\end{mathpar}

\begin{eqnarray*}
  match_{\equiv} (\quotep{P},\quotep{Q}) & := & P \equiv Q \\
  match_{\dagger}(\quotep{P},\quotep{Q}) & := & \forall R. P|Q \red^{*} R => R \red^{*} 0 \\
  match_{K}(\quotep{P},\quotep{Q}) & := & K \mbox{ for some context } K
\end{eqnarray*}

$u?(x)P | u!\langle Q \rangle \red P\{\quotep{Q}/x\}$

%We write $\wred$ for $\red^*$, and $P\red$ if $\exists Q $ such that $ P \red Q$.
We write $P\red$ if $\exists Q $ such that $ P \red Q$ and $P\not\red$, otherwise.

\section{Replication}

As mentioned before, it is known that replication (and hence
recursion) can be implemented in a higher-order process algebra
\cite{SangiorgiWalker}. As our first example of calculation with the
machinery thus far presented we give the construction explicitly in
the {\rhoc}.

\begin{eqnarray}
	D_{x} & := & \prefix{x}{y}{(\binpar{\outputp{x}{y}}{@{y}})} \nonumber\\
	\bangp_{x}{P} & := & \binpar{{x}!\langle{\binpar{D_{x}}{P}}\rangle}{D_{x}} \nonumber
\end{eqnarray}

\begin{eqnarray}
	\bangp_{x}{P} & & \nonumber\\
	=
	& {x}!\langle{(\prefix{x}{y}{(\outputp{x}{y} | @{y})) | P}}\rangle 
	      | \prefix{x}{y}{(\outputp{x}{y} | @{y})} & \nonumber\\
	\red
	& (\outputp{x}{y} | @{y})\substn{\quotep{(\prefix{x}{y}{(@{y} | \outputp{x}{y})) | P}}}{y} & \nonumber\\
	=
	& \outputp{x}{\quotep{(\prefix{x}{y}{(\outputp{x}{y} | @{y})) | P}}}
	  | {(\prefix{x}{y}{(\outputp{x}{y} | @{y})) | P}} & \nonumber\\
	\red
	& \ldots & \nonumber\\
	\red^*
	& P | P | \ldots & \nonumber
\end{eqnarray}

Of course, this encoding, as an implementation, runs away, unfolding
$\bangp{P}$ eagerly. A lazier and more implementable replication
operator, restricted to input-guarded processes, may be obtained as follows.

\begin{eqnarray}
\bangp{\prefix{u}{v}{P}} 
	:= 
	\binpar{\lift{x}{\prefix{u}{v}{(\binpar{D(x)}{P})}}}{D(x)} \nonumber
\end{eqnarray}

\begin{remark}
  Note that the lazier definition still does not deal with summation
  or mixed summation (i.e. sums over input and output). The reader is
  invited to construct definitions of replication that deal with these
  features. 

  Further, the definitions are parameterized in a name, $x$. Can you,
  gentle reader, make a definition that eliminates this parameter and
  guarantees no accidental interaction between the replication
  machinery and the process being replicated -- i.e. no accidental
  sharing of names used by the process to get its work done and the
  name(s) used by the replication to effect copying. This latter
  revision of the definition of replication is crucial to obtaining
  the expected identity $!!P \sim !P$.
\end{remark}

\begin{remark}\label{rem:paradoxical_combinator}
  The reader familiar with the lambda calculus will have noticed the
  similarity between $D$ and the paradoxical combinator.

  [Ed. note: the existence of this seems to suggest we have to be more
  restrictive on the set of processes and names we admit if we are to
  support no-cloning.]
\end{remark}

\subsubsection{Bisimulation}

The computational dynamics gives rise to another kind of equivalence,
the equivalence of computational behavior. As previously mentioned
this is typically captured \emph{via} some form of bisimulation.

% The notion we use in this paper is weak barbed bisimulation
% \cite{milner91polyadicpi}.

The notion we use in this paper is derived from weak barbed
bisimulation \cite{milner91polyadicpi}. 

\begin{definition}
An \emph{observation relation}, $\downarrow_{\mathcal N}$, over a set
of names, $\mathcal N$, is the smallest relation satisfying the rules
below.

\infrule[Out-barb]{y \in {\mathcal N}, \; x \nameeq y}
		  {\outputp{x}{v} \downarrow_{\mathcal N} x}
\infrule[Par-barb]{\mbox{$P\downarrow_{\mathcal N} x$ or $Q\downarrow_{\mathcal N} x$}}
		  {\binpar{P}{Q} \downarrow_{\mathcal N} x}

We write $P \Downarrow_{\mathcal N} x$ if there is $Q$ such that 
$P \wred Q$ and $Q \downarrow_{\mathcal N} x$.
\end{definition}

\begin{definition}
%\label{def.bbisim}
An  ${\mathcal N}$-\emph{barbed bisimulation} over a set of names, ${\mathcal N}$, is a symmetric binary relation 
${\mathcal S}_{\mathcal N}$ between agents such that $P\rel{S}_{\mathcal N}Q$ implies:
\begin{enumerate}
\item If $P \red P'$ then $Q \wred Q'$ and $P'\rel{S}_{\mathcal N} Q'$.
\item If $P\downarrow_{\mathcal N} x$, then $Q\Downarrow_{\mathcal N} x$.
\end{enumerate}
$P$ is ${\mathcal N}$-barbed bisimilar to $Q$, written
$P \wbbisim_{\mathcal N} Q$, if $P \rel{S}_{\mathcal N} Q$ for some ${\mathcal N}$-barbed bisimulation ${\mathcal S}_{\mathcal N}$.
\end{definition}

$\mathcal{R} \subseteq \pi \times \pi$

$P \mathcal{R} Q => \forall P'. P \red P' \Rightarrow \exists Q'. Q \red Q', P' \mathcal{R} Q'$

$P \vdash x \Rightarrow Q \vdash x$

\begin{mathpar}
  \inferrule*[lab=Out-barb]{x \nameeq y}{{y}!\langle{Q}\rangle \vdash x}
  \and
  \inferrule*[lab=Par-barb]{\mbox{$P\vdash x$ or $Q\vdash x$}}{\binpar{P}{Q} \vdash x}
\end{mathpar}

\subsubsection{Contexts}

One of the principle advantages of computational calculi like the
$\pi$-calculus is a well-defined notion of context,
contextual-equivalence and a correlation between
contextual-equivalence and notions of bisimulation. The notion of
context allows the decomposition of a process into (sub-)process and
its syntactic environment, its context. Thus, a context may be
thought of as a process with a ``hole'' (written $\Box$) in it. The
application of a context $M$ to a process $P$, written $M[P]$, is
tantamount to filling the hole in $M$ with $P$. In this paper we do
not need the full weight of this theory, but do make use of the notion
of context in the proof the main theorem. 

\begin{mathpar}
  \inferrule* [lab=summation] {} {{M_{M},M_{N}} \bc \Box \;|\; x.M_{A} \;|\; M_{M}+M_{N}}
  \and
  \inferrule* [lab=agent] {} {{M_{A}} \bc (\vec{x})M_{P} \;| \; \clift{P_0,\ldots,M_{P},\ldots,P_N}}
  \and \\
  \inferrule* [lab=process] {} {{M_{P}} \bc M_{N} \;| \;P|M_{P} }
\end{mathpar} 

\begin{mathpar}
  \inferrule* [lab=sychronization] {} {M_{N} \bc \Box \;|\; x?M_{F} \;|\; x!M_{C}}
  \and
  \inferrule* [lab=abstraction] {} {{M_{F}} \bc (x)M_{P} }
  \and
  \inferrule* [lab=concretion] {} {{M_{C}} \bc \langle M_{P} \rangle }
  \and \\
  \inferrule* [lab=process] {} {{M_{P}} \bc M_{N} \;| \;P|M_{P} }
\end{mathpar}

\begin{definition}[contextual application] Given a context $M$, and
  process $P$, we define the \emph{contextual application}, $M[P] :=
  M\{P/\Box\}$. That is, the contextual application of M to P is the
  substitution of $P$ for $\Box$ in $M$.
\end{definition}

$\meaningof{-} : L \to \mathcal{P}(\pi)$

\begin{mathpar}
  \inferrule* [lab=collection] {} {\meaningof{true} = \pi, \and \meaningof{~E} = \pi \setminus \meaningof{E}, \and \meaningof{E_{1} \& E_{2}} = \meaningof{E_{1}} \cap \meaningof{E_{2}}}
\end{mathpar}

\begin{mathpar}
  \inferrule* [lab=structure] {} {\meaningof{0} = \{ P \in \pi | P \equiv 0 \}, \and \\ \meaningof{E_1 | E_2} = \{ P \in \pi | P \equiv P_{1} | P_{2}, P_{1} \in \meaningof{E_{1}}, P_{2} \in \meaningof{E_2}\} }
\end{mathpar}

\begin{mathpar}
 \inferrule* [lab=behavior] {} {\meaningof{\langle a?b \rangle E} = \{ P \in \pi | P \equiv Q | u?(y)P', \\ \and \\\\ \and \\ \;\;\; u \in \meaningof{a}, \forall z.P'\{z/y\} \in \meaningof{E\{z/b\}}\}, \and \\ \meaningof{a!E} = \{ P \in \pi | P \equiv Q | x!\langle P' \rangle, x \in \meaningof{a} P' \in \meaningof{E}\} }
\end{mathpar}

\begin{mathpar}
 \inferrule* [lab=nominal] {} {\meaningof{\quotep{E}} = \{ \quotep{P} \in \quotep{\pi} | P \in \meaningof{E} \}, \and \meaningof{\quotep{P}} = \{ \quotep{Q} \in \quotep{\pi} | P \equiv Q \} \and \\ \meaningof{@\quotep{E}} = \{ P \in \pi | P \equiv @x, x \in \meaningof{E} \}}
\end{mathpar}

\begin{eqnarray*}
  \\
  \meaningof{-} : TS \to ST
\end{eqnarray*}

\begin{eqnarray*}
  \\
  L : TS \to ST
\end{eqnarray*}

\begin{eqnarray*}
  \\
  P \models E \iff P \in \meaningof{E}
\end{eqnarray*}

\begin{eqnarray*}
  P \approx_{L} Q \iff \forall E \in L. P \models E \iff Q \models E
\end{eqnarray*}

\begin{eqnarray*}
  P \approx_{K} Q
\end{eqnarray*}

\begin{eqnarray*}
  P \approx Q
\end{eqnarray*}

$\approx_{K} = \approx = \approx_{L}$

\subsubsection{Contextual duality}

Note that contexts extend the quotation operation to a family of
operations from processes to names. Given a context, $M$, we can
define a \emph{nominal context}, $\quotep{M}$ by $\quotep{M}[P] :=
\quotep{M[P]}$. To foreshadow what is to come we observe that these
operations enjoy a duality with processes very much like the duality
between vectors and maps from vectors to scalars.

Further, because the calculus is essentially higher-order, we have a
correspondence between contexts and processes. More specifically,
given a name $x$ and a context $M$ we can construct $M^{*}_{x}$ such
that 

\begin{mathpar}
  M^{*}_{x} | \lift{x}{P} \red M[P]
\end{mathpar}

namely,

\begin{mathpar}
  M^{*}_{x} := x?(u).M[\dropn{u}]
\end{mathpar}

The dependence of $M^{*}_{x}$ on a name makes it an abstraction, 

\begin{mathpar}
  M^{*} := (x)x?(u).M[\dropn{u}]
\end{mathpar}

\subsection{Additional notation}

It will sometimes be convenient to denote the process a name
quotes. We already have the notation $x = \quotep{P}$, but it will be
convenient to introduce an alternate notation, $\procn{x}$, when we
want to emphasize the connection to the use of the name. Note that, by
virtue of name equivalence, $\quotep{\procn{x}} \nameeq x$; so, the
notation is consistent with previous definitions.

Further, because names have structure it is possible to effect
substitutions on the basis of that structure. This means we need to
upgrade our notation for substitutions, which we accomplish by
adapting comprehension notation. Thus,

\begin{mathpar}
  P\{ y / x : x \in S \}
\end{mathpar}

is interpreted to mean the process derived from P by replacing (in a
capture-avoiding manner) each occurrence of $x$ in $S$ by $y$. For example,

\begin{mathpar}
  P\{ \quotep{\procn{x}|\procn{x}} / x : x \in \freenames{P} \}
\end{mathpar}

will replace each (occurrence) of a free name $x$ in $P$ by
$\quotep{\procn{x}|\procn{x}}$.

Also, we will avail ourselves of the notation $x^{L}$ and $x^{R}$ to
denote injections of a name into disjoint copies of the name
space. There are numerous ways to accomplish this. One example can be
found in \cite{MeredithR05}. This notation overloads to vectors of
names: $\vec{x}^{\pi} := (x_{i}^{\pi} \; : \; 0 \leq i < |\vec{x}| )$ where $\pi \in \{L,R\}$.

We also use $P^{\Box} := P|\Box$.

In \cite{MeredithR05} an interpretation of the new operator is
given. It turns out that there are several possible interpretations
all enjoying the requisite algebraic properties of the operator (see
\cite{milner91polyadicpi}). We will therefore make liberal use of
$(\nu\; \vec{x})P$.

% subsection the_syntax_and_semantics_of_the_notation_system (end)   

\section{Interpretation of QM}
\subsection{Supporting definitions}
\subsubsection{Multiplication}
\begin{mathpar}
  \quotep{Q} \cdot \quotep{R} := \quotep{Q|R}
  \and \\
  \quotep{Q} \cdot P := P\{ \quotep{Q|R} / \quotep{R} : \quotep{R} \in \freenames{P} \}
\end{mathpar}

\paragraph{Discussion}
The first line needs little explanation. The second line says that
each free name of the process is replaced with the multiplication of
that name by the scalar. Multiplication of a scalar (name) by a state
(process) results in a process all the names of which have been `moved
over' by parallel composition with the process the scalar
quotes. There is a subtlety that the bound names have to be
manipulated so that multiplied names aren't accidentally
captured. There are many ways to achieve this.

\begin{remark}\label{rem:multiplication_identities}
  The reader is invited to verify that for all $x,y,z \in \QProc$ and $P \in \Proc$
  \begin{mathpar}
    x \cdot \quotep{0} \equiv x 
    \and
    x \cdot y \equiv y \cdot x
    \and
    x \cdot (y \cdot z) \equiv (x \cdot y) \cdot z
    \and \\
    \quotep{0} \cdot P \equiv P
    \and \\
    x \cdot (y \cdot P) \equiv (x \cdot y) \cdot P
    \and \\
    x \cdot (P|Q) \equiv (x \cdot P) | (x \cdot Q)
    \and \\    
  \end{mathpar}
\end{remark}

\subsubsection{Tensor product}

We define a tensor product on processes by structural induction.

\paragraph{Tensor of sums} First note that all summations, including
$\pzero$ and sequence, can be written $\Sigma_{i} x_{i}.A_{i} +
\Sigma_{j} x_{j}.C_{j}$, where we have grouped input-guarded processes
together and output-guarded processes together.

Thus, we can define the tensor product of two summations, $N_{1}\otimes N_{2}$, where

\begin{mathpar}
  N_{1} := \Sigma_{i} x_{i}.A_{i} + \Sigma_{j} x_{j}.C_{j}
  \and
  N_{2} := \Sigma_{i'} y_{i'}.B_{i'} + \Sigma_{j'} y_{j'}.D_{j'} 
\end{mathpar}

as follows.

\begin{mathpar}
  \Sigma_{i} x_{i}.A_{i} + \Sigma_{j} x_{j}.C_{j} \otimes \Sigma_{i'}
  y_{i'}.B_{i'} + \Sigma_{j'} y_{j'}.D_{j'} 
  \and \\
  := \; \Sigma_{i} \Sigma_{i'} \quotep{\stackrel{\vee}{x_{i}}| \stackrel{\vee}{y_{i'}}}.(A_{i}\otimes B_{i'}) \; | \; \Sigma_{i'} \Sigma_{i} \quotep{\stackrel{\vee}{y_{i'}}|\stackrel{\vee}{x_{i}}}.(B_{i'}\otimes A_{i})
  \and
  \;\; | \;\; \Sigma_{j} \Sigma_{j'} \quotep{\stackrel{\vee}{x_{j}}|\stackrel{\vee}{y_{j'}}}.(A_{j}\otimes B_{j'}) \; | \; \Sigma_{j'} \Sigma_{j} \quotep{\stackrel{\vee}{y_{j'}}|\stackrel{\vee}{x_{j}}}.(B_{j'}\otimes A_{j})
\end{mathpar}

\begin{remark}
  Do we need to $x^{L}$ and $y^{R}$ for this construction as well?
\end{remark}

\paragraph{Tensor of parallel compositions} Next, we distribute tensor
over par.

\begin{mathpar}
  P_{1}|P_{2} \otimes Q_{1}|Q_{2} := (P_{1} \otimes Q_{1}) | (P_{1}
  \otimes Q_{2}) | (P_{2} \otimes Q_{1}) | (P_{2} \otimes Q_{2})
\end{mathpar}

\paragraph{Tensor with dropped names} We treat tensor of a
process with a dropped name as parallel composition.

\begin{mathpar}
  P \otimes \dropn{x} := P | \dropn{x}
\end{mathpar}

\paragraph{Tensor of agents}

Finally, we need to define tensor on agents. Note that the definition
of tensor on normal products only tensors inputs with inputs and
outputs with outputs. Thus, we only have to define the operation on
``homogeneous'' pairings.

\begin{mathpar}
  (\vec{x})P \otimes (\vec{y})Q
  \and \\
  := (x_{0}^{L}|y_{0}^{R},\ldots,x_{0}^{L}|y_{n}^{R},\ldots,x_{m}^{L}|y_{0}^{R},\ldots,x_{m}^{L}|y_{n}^R)(P\{ \vec{x}^{L}/\vec{x}\} \otimes Q \{ \vec{y}^{R}/\vec{y}\})
  \and \\
  \clift{\vec{P}} \otimes \clift{\vec{Q}}
  \and \\
  := \clift{P_{0}\otimes Q_{0},\ldots,P_{0}\otimes Q_{n},\ldots,P_{m}\otimes Q_{0},\ldots,P_{m}\otimes Q_{n}}
\end{mathpar}

\begin{remark}
  Observe that arities of tensored abstractions matches arities of
  tensored concretions if the original arities matched. Note also that
  the length of the arities corresponds to the increase in dimension
  we see in ordinary vector space tensor product.
\end{remark}

\begin{remark}
  Operationally, this definition distributes the tensor down to
  components ``linked'' by summation. Tensor over summation is
  intriguing in that it mixes names. Moreover, as a consequence of the
  way it mixes names we have the identities for all $x \in \QProc$ and
  $P,Q \in \Proc$

  \begin{mathpar}
    (x \cdot P) \otimes Q \equiv x \cdot (P \otimes Q) \equiv P \otimes (x \cdot Q)
    \and
    P \otimes \pzero \equiv P
  \end{mathpar}

  that the reader is invited to verify.
\end{remark}

\subsubsection{Annihilation}
\begin{mathpar}
  P^{\perp} := \{ Q | \forall R. P|Q \red^{*} R \Rightarrow R \red^{*} \pzero \}
  \and \\
  P^{\underline{\perp}} := \Sigma_{Q \in P^{\perp}} \quotep{Q}?(y).(\dropn{y}|Q) | \Sigma_{Q \in P^{\perp}} \quotep{Q}\clift{\Box}
\end{mathpar}

\paragraph{Discussion} The reader will note that $P^{\perp}$ is a
\emph{set} of processes, while $P^{\underline{\perp}}$ is a
\emph{context}. We call the set $P^{\perp}$ the \emph{annihilators} of
$P$. The parallel composition of a process in the annihilators of $P$
with $P$ will result in a process, the state space of which has all
paths eventually leading to $\pzero$. Execution may endure loops; but
under reasonable conditions of fairness (naturally guaranteed under
most notions of bisimulation) such a composite process cannot get
stuck in such a loop and will, eventually pop out and terminate.

The context $P^{\underline{\perp}}$ is ready and willing to ``take the
$P$ out of'' the process to which it is applied. It will effectively
transmit the code of the process to which it is applied to one of the
annihilators and run the process against it.

\subsubsection{Evaluation}
We fix $M$ a domain of fully abstract interpretation with an equality
coincident with bisimulation. We take $\meaningof{\cdot} : \Proc \to
M$ to be the map interpreting processes and $\nmeaningof{\cdot} : \M
\to Proc$ to be the map running the other way. Then we define

\begin{mathpar}
  \int P := \nmeaningof{\meaningof{P}}
\end{mathpar}

\paragraph{Discussion}
There are many fully abstract interpretations of Milner's
$\pi$-calculus. Any of them can be used as a basis for interpreting
the reflective calculus here. Equipped with such a domain it is
largely a matter of grinding through to check that the Yoneda
construction for the normalization-by-evaluation program can be
extended to this setting.

\begin{remark}
  The reader is invited to verify that $\int (P^{\underline{\perp}}[P]) = 0$.
\end{remark}

\subsection{Quantum mechanics}

Table \ref{tbl:core_qm_op_defns} gives the core operational definitions

\begin{table}[htp]\label{tbl:core_qm_op_defns}
  \center{
    \fbox{
      \begin{tabular}{c|c}
        quantum mechanics & process calculus \\
        \hline
        scalar & $x := \quotep{P}$ \\
        state vector & $\state{P} := P$ \\
        dual & $\state{P}^{*} := \event{P^{\underline{\perp}}} := \quotep{P^{\underline{\perp}}}[-]$ \\
        matrix & $ \Sigma_{\alpha} \state{P_{\alpha}}x_{\alpha}\event{Q_{\alpha}}$ \\
        vector addition & $\state{P} + \state{Q} := \state{P | Q}$ \\
        tensor product & $\state{P} \otimes \state{Q} := \state{P \otimes Q}$ \\
        inner product & $\innerprod{P}{Q} := \quotep{\int P^{\underline{\perp}}[Q]}$ \\
      \end{tabular}
    }
  }
  \caption{QM - operational definitions}
\end{table}

where

\begin{mathpar}
  \prmatrix{P}{Q} := \fprmatrix{P}{\quotep{\pzero}}{Q}
  \and
  \fprmatrix{P}{x}{Q} := (\state{P},x,\event{Q})
  \and
  (\fprmatrix{P}{x}{Q})(\state{R}) := x \cdot \innerprod{Q}{R} \cdot \state{P}
  \and
  (\fprmatrix{P}{x}{Q})(\event{R}) := x \cdot \innerprod{R}{P} \cdot \event{Q}
\end{mathpar}

\paragraph{Discussion}
As promised: vectors (aka states) are represented as processes; duals
as contextual duals; inner product definition should be compared with
standard inner product definition for ....

\begin{remark}
  Assuming $\int (P^{\underline{\perp}}[P]) = 0$, the reader is
  invited to verify that $(\fprmatrix{P}{x}{P})(\state{P}) = x \cdot \state{P}$.
\end{remark}

\begin{remark}
  The reader is invited to verify that $\innerprod{P}{Q}$ could
  equally well have been written $\quotep{\int \stackrel{\vee}{x}}$
  where $x = \event{P^{\underline{\perp}}}(Q)$.

  One of the motivations for this remark is that there is another way
  to factor these operations. We could package up evaluation in the dual:

  \begin{mathpar}
    \state{P}^{*} := \event{\int P^{\underline{\perp}}} := \quotep{\int P^{\underline{\perp}}}[-]
  \end{mathpar}

  and then have inner product defined by
  
  \begin{mathpar}
    \innerprod{P}{Q} := \event{P}(Q)
  \end{mathpar}

  Hopefully, experience with the calculations will provide guidance on
  the best factoring.
\end{remark}

\begin{remark}
  Assuming $\int (P^{\underline{\perp}}[P]) = 0$, the reader is
  invited to verify that $\forall P,Q. (\prmatrix{0}{Q})(\state{0}) =
  \state{0}$ and dually $(\prmatrix{P}{0})(\event{0}) = \event{0}$.
\end{remark}

\begin{remark}
  i'm a little worried that i don't (yet) have proper support for
  complex conjugacy. But, the observation above may give us a
  clue. According to Abramsky, it must be the case that the scalars
  are iso to the homset of the identity for the tensor -- which the
  observation above characterizes. 

  For now, we will simply bookmark the notion with $\overline{x}$.
\end{remark}

\subsubsection{Adjointness}

We need to give a definition of $(\cdot)^{\dagger}$ for matrices. The
obvious candidate definition is
\begin{mathpar}
(\Sigma_{\alpha}\fprmatrix{P_{\alpha}}{x_{\alpha}}{Q_{\alpha}})^{\dagger}
= \Sigma_{\alpha}\fprmatrix{(Q_{\alpha}^{\underline{\perp}})^{*}}{\overline{x}_{\alpha}}{P_{\alpha}^{\underline{\perp}}} 
\end{mathpar}

But, $(Q_{\alpha}^{\underline{\perp}})^{*}$ requires a name along
which to communicate the process to achieve the context application.

\subsubsection{Basis for a basis}
If processes label states and ``addition'' of states (a.k.a. vector
addition) is interpreted as parallel composition, what corresponds to
notions of linear independence and basis? Here, we recall that Yoshida
has developed a set of \emph{combinators} for an asynchronous verison
of Milner's $\pi$-calculus. These are a finite set of processes such
any process can be expressed as parallel composition of these
combinators together with liberal uses of the new operator and
replication. We can simply give a translation of these into the
present calculus and have reasonable expectation that the property
carries over. That is, that the resultant set allows to express all
processes via parallel composition. Note, however, that there is no
new operator or replication in this calculus. As a result, we expect
that the corresponding set is actually infinite. That is, we expect
that the space is actually infinite dimensional.

\begin{remark}
  The attentive reader may be a bit concerned. Certainly, the
  collection $S$, $K$ and $I$ is a finite set of
  combinators. Shouldn't we expect to see a finite set of combinators
  for an effectively equivalent system? i am very sympathetic to this
  critique and feel it warrants full attention. On the other hand, i
  also have in mind the following analogy. The natural numbers, as a
  monoid under addition, has exactly $1$ generator, while the natural
  numbers, as a monoid under multiplication, has countably many
  generators (the primes). We observe that the application of the
  lambda calculus is much less resource sensitive than the parallel
  composition of the $\pi$-calculus. Could it be the case that we have
  an analogy of the form
  
  \begin{mathpar}
    m + n : MN :: m*n : M|N
  \end{mathpar}

  giving a similar blow up in the set of ``primes''?  This is such a
  wonderful thought that, even if it's not true, i think it's worth
  writing down.
\end{remark}
 

\documentclass[12pt]{llncs}
%\documentclass{jktr}

\usepackage[pdftex]{hyperref}                   
\usepackage {listings}
\usepackage {mathpartir}
\usepackage{bcprules}
%\usepackage{listings}
                       
\usepackage{graphicx} 
%\usepackage[margins=2.5cm,nohead,nofoot]{geometry}
%\usepackage{geometry}
\usepackage{amsfonts}
\usepackage{amstext}
\usepackage{latexsym}
\usepackage{amssymb}
\usepackage{color}


%\include{myPreamble}
\include{qm2pi.local} 

%\ifpdf
%\usepackage[pdftex]{graphicx}
%\else
%\usepackage{graphicx}
%\fi

 % \ifpdf
%  \usepackage{pdfsync}
%  \if


%\title{Brief Article}
%\author{David F. Snyder}
%\author{L.G. Meredith}

%\address{Dept. of Math., Texas State University--San Marcos, San Marcos, TX 78666}
       
\pagestyle{empty}


\begin{document}

\lstset{language=[Objective]Caml,frame=shadowbox}

\input{qm2pi.front}

% section front matter (end)

\input{qm2pi.intro} 
 
% section introduction (end)

% \input{qm2pi.knotations} 

% section notation (end)

\input{qm2pi.process.calculi} 

% section concurrent_process_calculi_and_spatial_logics_ (end)
    
%\input{qm2pi.knots2pi} 

%\input{qm2pi.trefoil} 

%\input{qm2pi.mainthm} 

% subsection basic_interpretation (end)

%\input{qm2pi.rho.presentation} 
\subsection{The syntax and semantics of the notation system}\label{sub:the_syntax_and_semantics_of_the_notation_system} % (fold)

We now summarize a technical presentation of the calculus that
embodies our theory of dynamics. The typical presentation of such a
calculus follows the style of giving generators and relations on
them. The grammar, below, describing term constructors, freely
generates the set of processes, $\Proc$. This set is then quotiented
by a relation known as structural congruence and it is over this set
that the notion of dynamics is expressed. This presentation is
essentially that of \cite{MeredithR05} with the addition of
polyadicity and summation. For readability we have relegated some of
the technical subtleties to an appendix.

\subsubsection{Process grammar}\label{subsub:process_grammar}

\begin{mathpar}
  \inferrule* [lab=synchronization] {} {{M} \bc \pzero \;|\; x?F \;|\; x!C }
  \and
  \inferrule* [lab=abstraction] {} {{F} \bc (x)P}
  \and
  \inferrule* [lab=concretion] {} {{C} \bc \langle Q \rangle}
  \and
  \inferrule* [lab=process] {} {{P,Q} \bc M \;| \;P|Q \;|\; @{x}}
  \and
  \inferrule* [lab=name] {} {{x} \bc \quotep{P}}
\end{mathpar} 

Note that $\vec{x}$ (resp. $\vec{P}$) denotes a vector of names
(resp. processes) of length $|\vec{x}|$ (resp. $|\vec{P}|$). We adopt
the following useful abbreviations.

\begin{mathpar}
   x?(\vec{y}).P := x.(\vec{y})P \and  x\clift{\vec{P}} := x.\clift{\vec{P}}
   \and x!(y) := \lift{x}{\dropn{y}}
   \and \Pi_{i=0}^{n-1}P_i := P_0 | \ldots | P_{n-1}
\end{mathpar}

\subsubsection{Structural congruence}

\paragraph{Free and bound names and alpha-equivalence.} At the
core of structural equivalence is alpha-equivalence which identifies
process that are the same up to a change of variable. Formally, we
recognize the distinction between free and bound names. The free names
of a process, $\freenames{P}$, may be calculated recursively as
follows:

\begin{mathpar}
\freenames{\pzero} := \emptyset
  \and \\
  \freenames{x?(y).P} := \{ x \} \cup (\freenames{P} \setminus \{ y \})
  \and 
  \freenames{x!\langle P \rangle} := \{ x \} \cup \{ P \} 
  \and \\
  \freenames{P|Q} := \freenames{P} \cup \freenames{Q}
  \and \\
  \freenames{@{x}} := \{ x \}
\end{mathpar}

$\pi$
$\quotep{\pi}$

$\freenames{-} : \pi \to \mathcal{P}(\quotep{\pi})$

\begin{eqnarray*}
  \freenames{\pzero} & := & \emptyset \\
  \freenames{x?(y).P} & := & \{ x \} \cup (\freenames{P} \setminus \{ y \}) \\
  \freenames{x!\langle P \rangle} & := & \{ x \} \cup \{ P \} \\
  \freenames{P|Q} & := & \freenames{P} \cup \freenames{Q} \\
  \freenames{\dropn{x}} & := & \{ x \}
\end{eqnarray*}

The bound names of a process, $\boundnames{P}$, are those names occurring in $P$
that are not free. For example, in $x?(y).0$, the name $x$ is free, while $y$ is bound.

\begin{mathpar}
  \inferrule* [lab=monoidal-laws] {} { P|Q \equiv Q|P \and P|0 \equiv P \and P|(Q|R) \equiv (P|Q)|R }
\end{mathpar}

\begin{mathpar}
  \inferrule* [lab=alpha-equivalence] {} { (x)P \equiv (y)P\{y/x\} \and y \not\in \freenames{P} }
\end{mathpar}

\begin{definition}
Then two processes, $P,Q$, are alpha-equivalent if $P = Q\{\vec{y}/\vec{x}\}$ for
some $\vec{x} \in \boundnames{Q},\vec{y} \in \boundnames{P}$, where $Q\{\vec{y}/\vec{x}\}$
denotes the capture-avoiding substitution of $\vec{y}$ for $\vec{x}$ in $Q$.
\end{definition}

\begin{definition}
  The {\em structural congruence} \cite{SangiorgiWalker} , $\equiv$,
  between processes is the least congruence containing
  alpha-equivalence, satisfying the abelian monoid laws
  (associativity, commutativity and $\pzero$ as identity) for parallel
  composition $|$ and for summation $+$.
\end{definition}

\subsection{Name equivalence}

We take name equivalence, written $\nameeq$, to be the smallest
equivalence relation generated by the following rules.

\begin{mathpar}
\inferrule*[lab=Quote-drop]
{ }
{ \quotep{@{x}} \nameeq x }

\inferrule*[lab=Struct-equiv]
{ P \scong Q }
{ \quotep{P} \nameeq \quotep{Q} }
\end{mathpar}

The astute reader will have noticed that the mutual recursion of names
and processes imposes a mutual recursion on alpha-equivalence and
structural equivalence via name-equivalence. Fortunately, all of this
works out pleasantly and we may calculate in the natural way, free of
concern. The reader interested in the details is referred to the
appendix \ref{appendix:rho_details}.

\subsection{Substitution}

We use $\Proc$ for the set of processes, $\QProc$ for the set of
names, and $\id{\{}\vec{y} / \vec{x} \id{\}}$ to denote partial maps,
$s : \QProc \rightarrow \QProc$. A map, $s$ lifts, uniquely, to a map
on process terms, $\widehat{s} : \Proc \rightarrow \Proc$ by the
following equations.

\begin{mathpar}
  (0) \psubstp{Q}{P} := 0 \\
  (R \juxtap S) \psubstp{Q}{P}
  :=    
  (R)\psubstp{Q}{P} \juxtap (S) \psubstp{Q}{P} \\
  (x?(y).R) \psubstp{Q}{P}    
  :=    
  (x)\substp{Q}{P} (z)\concat( (R \psubstn{z}{y}) \psubstp{Q}{P} ) \\
  (\lift{x}{R}) \psubstp{Q}{P}  
  :=
  \lift{(x)\substp{Q}{P}}{ R \psubstp{Q}{P} } \\
%   (\dropn{x})  \psubstp{Q}{P}       
%   := 
%   \left\{ 
%     \begin{array}{ccc} 
%       \dropn{\quotep{Q}} & & x \nameeq \quotep{P} \\
%       \dropn{x} & & otherwise \\
%     \end{array}
%   \right. 
  (\dropn{x})  \psubstp{Q}{P}       
  := 
  \left\{ 
    \begin{array}{ccc} 
      Q & & x \nameeq \quotep{P} \\
      \dropn{x} & & otherwise \\
    \end{array}
  \right.
\end{mathpar}
 

where

\begin{eqnarray}
  (x)\id{\{} \lpquote Q \rpquote / \lpquote P \rpquote \id{\}}            = 
  \left\{ 
    \begin{array}{ccc}
      \lpquote Q \rpquote & & x \nameeq \lpquote P \rpquote \\
      x & & otherwise \\
    \end{array}
  \right. \nonumber
\end{eqnarray}

and $z$ is chosen distinct from $\quotep{P}$, $\quotep{Q}$, the free
names in $Q$, and all the names in $R$. Our $\alpha$-equivalence will
be built in the standard way from this substitution.

\begin{remark}\label{rem:no_self_referential_names}
  One consequence of these definitions is that $\forall P. \quotep{P}
  \not\in \freenames{P}$.
\end{remark}

\subsection{ Dynamic quote: an example }

Anticipating something of what's to come, consider applying the
substitution, $\widehat{\id{\{}u / z \id{\}}}$, to the following pair
of processes, $\lift{w}{y!(z)}$ and $w[ \lpquote y!(z) \rpquote ]$.

\begin{eqnarray}
	\lift{w}{y!(z)}\widehat{\id{\{}u / z \id{\}}}
		& = &
		\lift{w}{y!(u)} \nonumber\\
	w[ \lpquote y!(z) \rpquote ] \widehat{ \id{\{}u / z \id{\}} }
		& = &
		w[ \lpquote y!(z) \rpquote ] \nonumber
\end{eqnarray}

Because the body of the process between quotes is impervious to
substitution, we get radically different answers. In fact, by
examining the first process in an input context,
e.g. $x?(z).\lift{w}{y!(z)}$, we see that the process under the lift
operator may be shaped by prefixed inputs binding a name inside it. In
this sense, the lift operator will be seen as a way to dynamically
construct processes before reifying them as names.

Finally equipped with these standard features we can present the
dynamics of the calculus.

\subsubsection{Operational semantics} 

Finally, we introduce the computational dynamics. What marks these
algebras as distinct from other more traditionally studied algebraic
structures, e.g. vector spaces or polynomial rings, is the manner in
which dynamics is captured. In traditional structures, dynamics is typically
expressed through morphisms between such structures, as in linear maps
between vector spaces or morphisms between rings. In algebras
associated with the semantics of computation, the dynamics is
expressed as part of the algebraic structure itself, through a
reduction reduction relation typically denoted by $\red$. Below, we
give a recursive presentation of this relation for the calculus used
in the encoding.

$\red \subseteq \pi \times \pi$
$\red : \pi \to \mathcal{P}(\pi)$

\begin{mathpar}
  \inferrule* [lab=Comm] { \textsf{match}( x_{src}, x_{trgt} ) } { x_{trgt}?(y)P \; | \; x_{src}!\langle {Q} \rangle \red P\{\quotep{Q}/y}\} }
  \and \\
  \inferrule* [lab=Par] {{P} \red {P}'} {{{P} | {Q}} \red {{P}' | {Q}}}
  \and
  \inferrule* [lab=Equiv]{{{P} \scong {P}'} \andalso {{P}' \red {Q}'} \andalso {{Q}' \scong {Q}}}{{P} \red {Q}}
\end{mathpar}

\begin{eqnarray*}
  match_{\equiv} (\quotep{P},\quotep{Q}) & := & P \equiv Q \\
  match_{\dagger}(\quotep{P},\quotep{Q}) & := & \forall R. P|Q \red^{*} R => R \red^{*} 0 \\
  match_{K}(\quotep{P},\quotep{Q}) & := & K \mbox{ for some context } K
\end{eqnarray*}

$u?(x)P | u!\langle Q \rangle \red P\{\quotep{Q}/x\}$

%We write $\wred$ for $\red^*$, and $P\red$ if $\exists Q $ such that $ P \red Q$.
We write $P\red$ if $\exists Q $ such that $ P \red Q$ and $P\not\red$, otherwise.

\section{Replication}

As mentioned before, it is known that replication (and hence
recursion) can be implemented in a higher-order process algebra
\cite{SangiorgiWalker}. As our first example of calculation with the
machinery thus far presented we give the construction explicitly in
the {\rhoc}.

\begin{eqnarray}
	D_{x} & := & \prefix{x}{y}{(\binpar{\outputp{x}{y}}{@{y}})} \nonumber\\
	\bangp_{x}{P} & := & \binpar{{x}!\langle{\binpar{D_{x}}{P}}\rangle}{D_{x}} \nonumber
\end{eqnarray}

\begin{eqnarray}
	\bangp_{x}{P} & & \nonumber\\
	=
	& {x}!\langle{(\prefix{x}{y}{(\outputp{x}{y} | @{y})) | P}}\rangle 
	      | \prefix{x}{y}{(\outputp{x}{y} | @{y})} & \nonumber\\
	\red
	& (\outputp{x}{y} | @{y})\substn{\quotep{(\prefix{x}{y}{(@{y} | \outputp{x}{y})) | P}}}{y} & \nonumber\\
	=
	& \outputp{x}{\quotep{(\prefix{x}{y}{(\outputp{x}{y} | @{y})) | P}}}
	  | {(\prefix{x}{y}{(\outputp{x}{y} | @{y})) | P}} & \nonumber\\
	\red
	& \ldots & \nonumber\\
	\red^*
	& P | P | \ldots & \nonumber
\end{eqnarray}

Of course, this encoding, as an implementation, runs away, unfolding
$\bangp{P}$ eagerly. A lazier and more implementable replication
operator, restricted to input-guarded processes, may be obtained as follows.

\begin{eqnarray}
\bangp{\prefix{u}{v}{P}} 
	:= 
	\binpar{\lift{x}{\prefix{u}{v}{(\binpar{D(x)}{P})}}}{D(x)} \nonumber
\end{eqnarray}

\begin{remark}
  Note that the lazier definition still does not deal with summation
  or mixed summation (i.e. sums over input and output). The reader is
  invited to construct definitions of replication that deal with these
  features. 

  Further, the definitions are parameterized in a name, $x$. Can you,
  gentle reader, make a definition that eliminates this parameter and
  guarantees no accidental interaction between the replication
  machinery and the process being replicated -- i.e. no accidental
  sharing of names used by the process to get its work done and the
  name(s) used by the replication to effect copying. This latter
  revision of the definition of replication is crucial to obtaining
  the expected identity $!!P \sim !P$.
\end{remark}

\begin{remark}\label{rem:paradoxical_combinator}
  The reader familiar with the lambda calculus will have noticed the
  similarity between $D$ and the paradoxical combinator.

  [Ed. note: the existence of this seems to suggest we have to be more
  restrictive on the set of processes and names we admit if we are to
  support no-cloning.]
\end{remark}

\subsubsection{Bisimulation}

The computational dynamics gives rise to another kind of equivalence,
the equivalence of computational behavior. As previously mentioned
this is typically captured \emph{via} some form of bisimulation.

% The notion we use in this paper is weak barbed bisimulation
% \cite{milner91polyadicpi}.

The notion we use in this paper is derived from weak barbed
bisimulation \cite{milner91polyadicpi}. 

\begin{definition}
An \emph{observation relation}, $\downarrow_{\mathcal N}$, over a set
of names, $\mathcal N$, is the smallest relation satisfying the rules
below.

\infrule[Out-barb]{y \in {\mathcal N}, \; x \nameeq y}
		  {\outputp{x}{v} \downarrow_{\mathcal N} x}
\infrule[Par-barb]{\mbox{$P\downarrow_{\mathcal N} x$ or $Q\downarrow_{\mathcal N} x$}}
		  {\binpar{P}{Q} \downarrow_{\mathcal N} x}

We write $P \Downarrow_{\mathcal N} x$ if there is $Q$ such that 
$P \wred Q$ and $Q \downarrow_{\mathcal N} x$.
\end{definition}

\begin{definition}
%\label{def.bbisim}
An  ${\mathcal N}$-\emph{barbed bisimulation} over a set of names, ${\mathcal N}$, is a symmetric binary relation 
${\mathcal S}_{\mathcal N}$ between agents such that $P\rel{S}_{\mathcal N}Q$ implies:
\begin{enumerate}
\item If $P \red P'$ then $Q \wred Q'$ and $P'\rel{S}_{\mathcal N} Q'$.
\item If $P\downarrow_{\mathcal N} x$, then $Q\Downarrow_{\mathcal N} x$.
\end{enumerate}
$P$ is ${\mathcal N}$-barbed bisimilar to $Q$, written
$P \wbbisim_{\mathcal N} Q$, if $P \rel{S}_{\mathcal N} Q$ for some ${\mathcal N}$-barbed bisimulation ${\mathcal S}_{\mathcal N}$.
\end{definition}

$\mathcal{R} \subseteq \pi \times \pi$

$P \mathcal{R} Q => \forall P'. P \red P' \Rightarrow \exists Q'. Q \red Q', P' \mathcal{R} Q'$

$P \vdash x \Rightarrow Q \vdash x$

\begin{mathpar}
  \inferrule*[lab=Out-barb]{x \nameeq y}{{y}!\langle{Q}\rangle \vdash x}
  \and
  \inferrule*[lab=Par-barb]{\mbox{$P\vdash x$ or $Q\vdash x$}}{\binpar{P}{Q} \vdash x}
\end{mathpar}

\subsubsection{Contexts}

One of the principle advantages of computational calculi like the
$\pi$-calculus is a well-defined notion of context,
contextual-equivalence and a correlation between
contextual-equivalence and notions of bisimulation. The notion of
context allows the decomposition of a process into (sub-)process and
its syntactic environment, its context. Thus, a context may be
thought of as a process with a ``hole'' (written $\Box$) in it. The
application of a context $M$ to a process $P$, written $M[P]$, is
tantamount to filling the hole in $M$ with $P$. In this paper we do
not need the full weight of this theory, but do make use of the notion
of context in the proof the main theorem. 

\begin{mathpar}
  \inferrule* [lab=summation] {} {{M_{M},M_{N}} \bc \Box \;|\; x.M_{A} \;|\; M_{M}+M_{N}}
  \and
  \inferrule* [lab=agent] {} {{M_{A}} \bc (\vec{x})M_{P} \;| \; \clift{P_0,\ldots,M_{P},\ldots,P_N}}
  \and \\
  \inferrule* [lab=process] {} {{M_{P}} \bc M_{N} \;| \;P|M_{P} }
\end{mathpar} 

\begin{mathpar}
  \inferrule* [lab=sychronization] {} {M_{N} \bc \Box \;|\; x?M_{F} \;|\; x!M_{C}}
  \and
  \inferrule* [lab=abstraction] {} {{M_{F}} \bc (x)M_{P} }
  \and
  \inferrule* [lab=concretion] {} {{M_{C}} \bc \langle M_{P} \rangle }
  \and \\
  \inferrule* [lab=process] {} {{M_{P}} \bc M_{N} \;| \;P|M_{P} }
\end{mathpar}

\begin{definition}[contextual application] Given a context $M$, and
  process $P$, we define the \emph{contextual application}, $M[P] :=
  M\{P/\Box\}$. That is, the contextual application of M to P is the
  substitution of $P$ for $\Box$ in $M$.
\end{definition}

$\meaningof{-} : L \to \mathcal{P}(\pi)$

\begin{mathpar}
  \inferrule* [lab=collection] {} {\meaningof{true} = \pi, \and \meaningof{~E} = \pi \setminus \meaningof{E}, \and \meaningof{E_{1} \& E_{2}} = \meaningof{E_{1}} \cap \meaningof{E_{2}}}
\end{mathpar}

\begin{mathpar}
  \inferrule* [lab=structure] {} {\meaningof{0} = \{ P \in \pi | P \equiv 0 \}, \and \\ \meaningof{E_1 | E_2} = \{ P \in \pi | P \equiv P_{1} | P_{2}, P_{1} \in \meaningof{E_{1}}, P_{2} \in \meaningof{E_2}\} }
\end{mathpar}

\begin{mathpar}
 \inferrule* [lab=behavior] {} {\meaningof{\langle a?b \rangle E} = \{ P \in \pi | P \equiv Q | u?(y)P', \\ \and \\\\ \and \\ \;\;\; u \in \meaningof{a}, \forall z.P'\{z/y\} \in \meaningof{E\{z/b\}}\}, \and \\ \meaningof{a!E} = \{ P \in \pi | P \equiv Q | x!\langle P' \rangle, x \in \meaningof{a} P' \in \meaningof{E}\} }
\end{mathpar}

\begin{mathpar}
 \inferrule* [lab=nominal] {} {\meaningof{\quotep{E}} = \{ \quotep{P} \in \quotep{\pi} | P \in \meaningof{E} \}, \and \meaningof{\quotep{P}} = \{ \quotep{Q} \in \quotep{\pi} | P \equiv Q \} \and \\ \meaningof{@\quotep{E}} = \{ P \in \pi | P \equiv @x, x \in \meaningof{E} \}}
\end{mathpar}

\begin{eqnarray*}
  \\
  \meaningof{-} : TS \to ST
\end{eqnarray*}

\begin{eqnarray*}
  \\
  L : TS \to ST
\end{eqnarray*}

\begin{eqnarray*}
  \\
  P \models E \iff P \in \meaningof{E}
\end{eqnarray*}

\begin{eqnarray*}
  P \approx_{L} Q \iff \forall E \in L. P \models E \iff Q \models E
\end{eqnarray*}

\begin{eqnarray*}
  P \approx_{K} Q
\end{eqnarray*}

\begin{eqnarray*}
  P \approx Q
\end{eqnarray*}

$\approx_{K} = \approx = \approx_{L}$

\subsubsection{Contextual duality}

Note that contexts extend the quotation operation to a family of
operations from processes to names. Given a context, $M$, we can
define a \emph{nominal context}, $\quotep{M}$ by $\quotep{M}[P] :=
\quotep{M[P]}$. To foreshadow what is to come we observe that these
operations enjoy a duality with processes very much like the duality
between vectors and maps from vectors to scalars.

Further, because the calculus is essentially higher-order, we have a
correspondence between contexts and processes. More specifically,
given a name $x$ and a context $M$ we can construct $M^{*}_{x}$ such
that 

\begin{mathpar}
  M^{*}_{x} | \lift{x}{P} \red M[P]
\end{mathpar}

namely,

\begin{mathpar}
  M^{*}_{x} := x?(u).M[\dropn{u}]
\end{mathpar}

The dependence of $M^{*}_{x}$ on a name makes it an abstraction, 

\begin{mathpar}
  M^{*} := (x)x?(u).M[\dropn{u}]
\end{mathpar}

\subsection{Additional notation}

It will sometimes be convenient to denote the process a name
quotes. We already have the notation $x = \quotep{P}$, but it will be
convenient to introduce an alternate notation, $\procn{x}$, when we
want to emphasize the connection to the use of the name. Note that, by
virtue of name equivalence, $\quotep{\procn{x}} \nameeq x$; so, the
notation is consistent with previous definitions.

Further, because names have structure it is possible to effect
substitutions on the basis of that structure. This means we need to
upgrade our notation for substitutions, which we accomplish by
adapting comprehension notation. Thus,

\begin{mathpar}
  P\{ y / x : x \in S \}
\end{mathpar}

is interpreted to mean the process derived from P by replacing (in a
capture-avoiding manner) each occurrence of $x$ in $S$ by $y$. For example,

\begin{mathpar}
  P\{ \quotep{\procn{x}|\procn{x}} / x : x \in \freenames{P} \}
\end{mathpar}

will replace each (occurrence) of a free name $x$ in $P$ by
$\quotep{\procn{x}|\procn{x}}$.

Also, we will avail ourselves of the notation $x^{L}$ and $x^{R}$ to
denote injections of a name into disjoint copies of the name
space. There are numerous ways to accomplish this. One example can be
found in \cite{MeredithR05}. This notation overloads to vectors of
names: $\vec{x}^{\pi} := (x_{i}^{\pi} \; : \; 0 \leq i < |\vec{x}| )$ where $\pi \in \{L,R\}$.

We also use $P^{\Box} := P|\Box$.

In \cite{MeredithR05} an interpretation of the new operator is
given. It turns out that there are several possible interpretations
all enjoying the requisite algebraic properties of the operator (see
\cite{milner91polyadicpi}). We will therefore make liberal use of
$(\nu\; \vec{x})P$.

% subsection the_syntax_and_semantics_of_the_notation_system (end)   

\input{qm2pi.qmops} 

\input{qm2pi.sterngerlach} 

\input{qm2pi.metric} 

% section concurrent_process_calculi (end)

%\input{qm2pi.proofsketch}

% section proof sketch (end)

%\input{qm2pi.slviaknots} 

% section spatial logic via knots (end)

\input{qm2pi.conclusion}

% section conclusion (end)

%\input{qm2pi.dtcodes} 

% section wiring algorithm (end)

\input{qm2pi.ack} 

% section acknowledgments (end)

\newpage


\bibliographystyle{plain}   
\bibliography{../../biblios/main.bib}

\input{qm2pi.rhodetails}

\end{document}

 

\documentclass[12pt]{llncs}
%\documentclass{jktr}

\usepackage[pdftex]{hyperref}                   
\usepackage {listings}
\usepackage {mathpartir}
\usepackage{bcprules}
%\usepackage{listings}
                       
\usepackage{graphicx} 
%\usepackage[margins=2.5cm,nohead,nofoot]{geometry}
%\usepackage{geometry}
\usepackage{amsfonts}
\usepackage{amstext}
\usepackage{latexsym}
\usepackage{amssymb}
\usepackage{color}


%\include{myPreamble}
\include{qm2pi.local} 

%\ifpdf
%\usepackage[pdftex]{graphicx}
%\else
%\usepackage{graphicx}
%\fi

 % \ifpdf
%  \usepackage{pdfsync}
%  \if


%\title{Brief Article}
%\author{David F. Snyder}
%\author{L.G. Meredith}

%\address{Dept. of Math., Texas State University--San Marcos, San Marcos, TX 78666}
       
\pagestyle{empty}


\begin{document}

\lstset{language=[Objective]Caml,frame=shadowbox}

\input{qm2pi.front}

% section front matter (end)

\input{qm2pi.intro} 
 
% section introduction (end)

% \input{qm2pi.knotations} 

% section notation (end)

\input{qm2pi.process.calculi} 

% section concurrent_process_calculi_and_spatial_logics_ (end)
    
%\input{qm2pi.knots2pi} 

%\input{qm2pi.trefoil} 

%\input{qm2pi.mainthm} 

% subsection basic_interpretation (end)

%\input{qm2pi.rho.presentation} 
\subsection{The syntax and semantics of the notation system}\label{sub:the_syntax_and_semantics_of_the_notation_system} % (fold)

We now summarize a technical presentation of the calculus that
embodies our theory of dynamics. The typical presentation of such a
calculus follows the style of giving generators and relations on
them. The grammar, below, describing term constructors, freely
generates the set of processes, $\Proc$. This set is then quotiented
by a relation known as structural congruence and it is over this set
that the notion of dynamics is expressed. This presentation is
essentially that of \cite{MeredithR05} with the addition of
polyadicity and summation. For readability we have relegated some of
the technical subtleties to an appendix.

\subsubsection{Process grammar}\label{subsub:process_grammar}

\begin{mathpar}
  \inferrule* [lab=synchronization] {} {{M} \bc \pzero \;|\; x?F \;|\; x!C }
  \and
  \inferrule* [lab=abstraction] {} {{F} \bc (x)P}
  \and
  \inferrule* [lab=concretion] {} {{C} \bc \langle Q \rangle}
  \and
  \inferrule* [lab=process] {} {{P,Q} \bc M \;| \;P|Q \;|\; @{x}}
  \and
  \inferrule* [lab=name] {} {{x} \bc \quotep{P}}
\end{mathpar} 

Note that $\vec{x}$ (resp. $\vec{P}$) denotes a vector of names
(resp. processes) of length $|\vec{x}|$ (resp. $|\vec{P}|$). We adopt
the following useful abbreviations.

\begin{mathpar}
   x?(\vec{y}).P := x.(\vec{y})P \and  x\clift{\vec{P}} := x.\clift{\vec{P}}
   \and x!(y) := \lift{x}{\dropn{y}}
   \and \Pi_{i=0}^{n-1}P_i := P_0 | \ldots | P_{n-1}
\end{mathpar}

\subsubsection{Structural congruence}

\paragraph{Free and bound names and alpha-equivalence.} At the
core of structural equivalence is alpha-equivalence which identifies
process that are the same up to a change of variable. Formally, we
recognize the distinction between free and bound names. The free names
of a process, $\freenames{P}$, may be calculated recursively as
follows:

\begin{mathpar}
\freenames{\pzero} := \emptyset
  \and \\
  \freenames{x?(y).P} := \{ x \} \cup (\freenames{P} \setminus \{ y \})
  \and 
  \freenames{x!\langle P \rangle} := \{ x \} \cup \{ P \} 
  \and \\
  \freenames{P|Q} := \freenames{P} \cup \freenames{Q}
  \and \\
  \freenames{@{x}} := \{ x \}
\end{mathpar}

$\pi$
$\quotep{\pi}$

$\freenames{-} : \pi \to \mathcal{P}(\quotep{\pi})$

\begin{eqnarray*}
  \freenames{\pzero} & := & \emptyset \\
  \freenames{x?(y).P} & := & \{ x \} \cup (\freenames{P} \setminus \{ y \}) \\
  \freenames{x!\langle P \rangle} & := & \{ x \} \cup \{ P \} \\
  \freenames{P|Q} & := & \freenames{P} \cup \freenames{Q} \\
  \freenames{\dropn{x}} & := & \{ x \}
\end{eqnarray*}

The bound names of a process, $\boundnames{P}$, are those names occurring in $P$
that are not free. For example, in $x?(y).0$, the name $x$ is free, while $y$ is bound.

\begin{mathpar}
  \inferrule* [lab=monoidal-laws] {} { P|Q \equiv Q|P \and P|0 \equiv P \and P|(Q|R) \equiv (P|Q)|R }
\end{mathpar}

\begin{mathpar}
  \inferrule* [lab=alpha-equivalence] {} { (x)P \equiv (y)P\{y/x\} \and y \not\in \freenames{P} }
\end{mathpar}

\begin{definition}
Then two processes, $P,Q$, are alpha-equivalent if $P = Q\{\vec{y}/\vec{x}\}$ for
some $\vec{x} \in \boundnames{Q},\vec{y} \in \boundnames{P}$, where $Q\{\vec{y}/\vec{x}\}$
denotes the capture-avoiding substitution of $\vec{y}$ for $\vec{x}$ in $Q$.
\end{definition}

\begin{definition}
  The {\em structural congruence} \cite{SangiorgiWalker} , $\equiv$,
  between processes is the least congruence containing
  alpha-equivalence, satisfying the abelian monoid laws
  (associativity, commutativity and $\pzero$ as identity) for parallel
  composition $|$ and for summation $+$.
\end{definition}

\subsection{Name equivalence}

We take name equivalence, written $\nameeq$, to be the smallest
equivalence relation generated by the following rules.

\begin{mathpar}
\inferrule*[lab=Quote-drop]
{ }
{ \quotep{@{x}} \nameeq x }

\inferrule*[lab=Struct-equiv]
{ P \scong Q }
{ \quotep{P} \nameeq \quotep{Q} }
\end{mathpar}

The astute reader will have noticed that the mutual recursion of names
and processes imposes a mutual recursion on alpha-equivalence and
structural equivalence via name-equivalence. Fortunately, all of this
works out pleasantly and we may calculate in the natural way, free of
concern. The reader interested in the details is referred to the
appendix \ref{appendix:rho_details}.

\subsection{Substitution}

We use $\Proc$ for the set of processes, $\QProc$ for the set of
names, and $\id{\{}\vec{y} / \vec{x} \id{\}}$ to denote partial maps,
$s : \QProc \rightarrow \QProc$. A map, $s$ lifts, uniquely, to a map
on process terms, $\widehat{s} : \Proc \rightarrow \Proc$ by the
following equations.

\begin{mathpar}
  (0) \psubstp{Q}{P} := 0 \\
  (R \juxtap S) \psubstp{Q}{P}
  :=    
  (R)\psubstp{Q}{P} \juxtap (S) \psubstp{Q}{P} \\
  (x?(y).R) \psubstp{Q}{P}    
  :=    
  (x)\substp{Q}{P} (z)\concat( (R \psubstn{z}{y}) \psubstp{Q}{P} ) \\
  (\lift{x}{R}) \psubstp{Q}{P}  
  :=
  \lift{(x)\substp{Q}{P}}{ R \psubstp{Q}{P} } \\
%   (\dropn{x})  \psubstp{Q}{P}       
%   := 
%   \left\{ 
%     \begin{array}{ccc} 
%       \dropn{\quotep{Q}} & & x \nameeq \quotep{P} \\
%       \dropn{x} & & otherwise \\
%     \end{array}
%   \right. 
  (\dropn{x})  \psubstp{Q}{P}       
  := 
  \left\{ 
    \begin{array}{ccc} 
      Q & & x \nameeq \quotep{P} \\
      \dropn{x} & & otherwise \\
    \end{array}
  \right.
\end{mathpar}
 

where

\begin{eqnarray}
  (x)\id{\{} \lpquote Q \rpquote / \lpquote P \rpquote \id{\}}            = 
  \left\{ 
    \begin{array}{ccc}
      \lpquote Q \rpquote & & x \nameeq \lpquote P \rpquote \\
      x & & otherwise \\
    \end{array}
  \right. \nonumber
\end{eqnarray}

and $z$ is chosen distinct from $\quotep{P}$, $\quotep{Q}$, the free
names in $Q$, and all the names in $R$. Our $\alpha$-equivalence will
be built in the standard way from this substitution.

\begin{remark}\label{rem:no_self_referential_names}
  One consequence of these definitions is that $\forall P. \quotep{P}
  \not\in \freenames{P}$.
\end{remark}

\subsection{ Dynamic quote: an example }

Anticipating something of what's to come, consider applying the
substitution, $\widehat{\id{\{}u / z \id{\}}}$, to the following pair
of processes, $\lift{w}{y!(z)}$ and $w[ \lpquote y!(z) \rpquote ]$.

\begin{eqnarray}
	\lift{w}{y!(z)}\widehat{\id{\{}u / z \id{\}}}
		& = &
		\lift{w}{y!(u)} \nonumber\\
	w[ \lpquote y!(z) \rpquote ] \widehat{ \id{\{}u / z \id{\}} }
		& = &
		w[ \lpquote y!(z) \rpquote ] \nonumber
\end{eqnarray}

Because the body of the process between quotes is impervious to
substitution, we get radically different answers. In fact, by
examining the first process in an input context,
e.g. $x?(z).\lift{w}{y!(z)}$, we see that the process under the lift
operator may be shaped by prefixed inputs binding a name inside it. In
this sense, the lift operator will be seen as a way to dynamically
construct processes before reifying them as names.

Finally equipped with these standard features we can present the
dynamics of the calculus.

\subsubsection{Operational semantics} 

Finally, we introduce the computational dynamics. What marks these
algebras as distinct from other more traditionally studied algebraic
structures, e.g. vector spaces or polynomial rings, is the manner in
which dynamics is captured. In traditional structures, dynamics is typically
expressed through morphisms between such structures, as in linear maps
between vector spaces or morphisms between rings. In algebras
associated with the semantics of computation, the dynamics is
expressed as part of the algebraic structure itself, through a
reduction reduction relation typically denoted by $\red$. Below, we
give a recursive presentation of this relation for the calculus used
in the encoding.

$\red \subseteq \pi \times \pi$
$\red : \pi \to \mathcal{P}(\pi)$

\begin{mathpar}
  \inferrule* [lab=Comm] { \textsf{match}( x_{src}, x_{trgt} ) } { x_{trgt}?(y)P \; | \; x_{src}!\langle {Q} \rangle \red P\{\quotep{Q}/y}\} }
  \and \\
  \inferrule* [lab=Par] {{P} \red {P}'} {{{P} | {Q}} \red {{P}' | {Q}}}
  \and
  \inferrule* [lab=Equiv]{{{P} \scong {P}'} \andalso {{P}' \red {Q}'} \andalso {{Q}' \scong {Q}}}{{P} \red {Q}}
\end{mathpar}

\begin{eqnarray*}
  match_{\equiv} (\quotep{P},\quotep{Q}) & := & P \equiv Q \\
  match_{\dagger}(\quotep{P},\quotep{Q}) & := & \forall R. P|Q \red^{*} R => R \red^{*} 0 \\
  match_{K}(\quotep{P},\quotep{Q}) & := & K \mbox{ for some context } K
\end{eqnarray*}

$u?(x)P | u!\langle Q \rangle \red P\{\quotep{Q}/x\}$

%We write $\wred$ for $\red^*$, and $P\red$ if $\exists Q $ such that $ P \red Q$.
We write $P\red$ if $\exists Q $ such that $ P \red Q$ and $P\not\red$, otherwise.

\section{Replication}

As mentioned before, it is known that replication (and hence
recursion) can be implemented in a higher-order process algebra
\cite{SangiorgiWalker}. As our first example of calculation with the
machinery thus far presented we give the construction explicitly in
the {\rhoc}.

\begin{eqnarray}
	D_{x} & := & \prefix{x}{y}{(\binpar{\outputp{x}{y}}{@{y}})} \nonumber\\
	\bangp_{x}{P} & := & \binpar{{x}!\langle{\binpar{D_{x}}{P}}\rangle}{D_{x}} \nonumber
\end{eqnarray}

\begin{eqnarray}
	\bangp_{x}{P} & & \nonumber\\
	=
	& {x}!\langle{(\prefix{x}{y}{(\outputp{x}{y} | @{y})) | P}}\rangle 
	      | \prefix{x}{y}{(\outputp{x}{y} | @{y})} & \nonumber\\
	\red
	& (\outputp{x}{y} | @{y})\substn{\quotep{(\prefix{x}{y}{(@{y} | \outputp{x}{y})) | P}}}{y} & \nonumber\\
	=
	& \outputp{x}{\quotep{(\prefix{x}{y}{(\outputp{x}{y} | @{y})) | P}}}
	  | {(\prefix{x}{y}{(\outputp{x}{y} | @{y})) | P}} & \nonumber\\
	\red
	& \ldots & \nonumber\\
	\red^*
	& P | P | \ldots & \nonumber
\end{eqnarray}

Of course, this encoding, as an implementation, runs away, unfolding
$\bangp{P}$ eagerly. A lazier and more implementable replication
operator, restricted to input-guarded processes, may be obtained as follows.

\begin{eqnarray}
\bangp{\prefix{u}{v}{P}} 
	:= 
	\binpar{\lift{x}{\prefix{u}{v}{(\binpar{D(x)}{P})}}}{D(x)} \nonumber
\end{eqnarray}

\begin{remark}
  Note that the lazier definition still does not deal with summation
  or mixed summation (i.e. sums over input and output). The reader is
  invited to construct definitions of replication that deal with these
  features. 

  Further, the definitions are parameterized in a name, $x$. Can you,
  gentle reader, make a definition that eliminates this parameter and
  guarantees no accidental interaction between the replication
  machinery and the process being replicated -- i.e. no accidental
  sharing of names used by the process to get its work done and the
  name(s) used by the replication to effect copying. This latter
  revision of the definition of replication is crucial to obtaining
  the expected identity $!!P \sim !P$.
\end{remark}

\begin{remark}\label{rem:paradoxical_combinator}
  The reader familiar with the lambda calculus will have noticed the
  similarity between $D$ and the paradoxical combinator.

  [Ed. note: the existence of this seems to suggest we have to be more
  restrictive on the set of processes and names we admit if we are to
  support no-cloning.]
\end{remark}

\subsubsection{Bisimulation}

The computational dynamics gives rise to another kind of equivalence,
the equivalence of computational behavior. As previously mentioned
this is typically captured \emph{via} some form of bisimulation.

% The notion we use in this paper is weak barbed bisimulation
% \cite{milner91polyadicpi}.

The notion we use in this paper is derived from weak barbed
bisimulation \cite{milner91polyadicpi}. 

\begin{definition}
An \emph{observation relation}, $\downarrow_{\mathcal N}$, over a set
of names, $\mathcal N$, is the smallest relation satisfying the rules
below.

\infrule[Out-barb]{y \in {\mathcal N}, \; x \nameeq y}
		  {\outputp{x}{v} \downarrow_{\mathcal N} x}
\infrule[Par-barb]{\mbox{$P\downarrow_{\mathcal N} x$ or $Q\downarrow_{\mathcal N} x$}}
		  {\binpar{P}{Q} \downarrow_{\mathcal N} x}

We write $P \Downarrow_{\mathcal N} x$ if there is $Q$ such that 
$P \wred Q$ and $Q \downarrow_{\mathcal N} x$.
\end{definition}

\begin{definition}
%\label{def.bbisim}
An  ${\mathcal N}$-\emph{barbed bisimulation} over a set of names, ${\mathcal N}$, is a symmetric binary relation 
${\mathcal S}_{\mathcal N}$ between agents such that $P\rel{S}_{\mathcal N}Q$ implies:
\begin{enumerate}
\item If $P \red P'$ then $Q \wred Q'$ and $P'\rel{S}_{\mathcal N} Q'$.
\item If $P\downarrow_{\mathcal N} x$, then $Q\Downarrow_{\mathcal N} x$.
\end{enumerate}
$P$ is ${\mathcal N}$-barbed bisimilar to $Q$, written
$P \wbbisim_{\mathcal N} Q$, if $P \rel{S}_{\mathcal N} Q$ for some ${\mathcal N}$-barbed bisimulation ${\mathcal S}_{\mathcal N}$.
\end{definition}

$\mathcal{R} \subseteq \pi \times \pi$

$P \mathcal{R} Q => \forall P'. P \red P' \Rightarrow \exists Q'. Q \red Q', P' \mathcal{R} Q'$

$P \vdash x \Rightarrow Q \vdash x$

\begin{mathpar}
  \inferrule*[lab=Out-barb]{x \nameeq y}{{y}!\langle{Q}\rangle \vdash x}
  \and
  \inferrule*[lab=Par-barb]{\mbox{$P\vdash x$ or $Q\vdash x$}}{\binpar{P}{Q} \vdash x}
\end{mathpar}

\subsubsection{Contexts}

One of the principle advantages of computational calculi like the
$\pi$-calculus is a well-defined notion of context,
contextual-equivalence and a correlation between
contextual-equivalence and notions of bisimulation. The notion of
context allows the decomposition of a process into (sub-)process and
its syntactic environment, its context. Thus, a context may be
thought of as a process with a ``hole'' (written $\Box$) in it. The
application of a context $M$ to a process $P$, written $M[P]$, is
tantamount to filling the hole in $M$ with $P$. In this paper we do
not need the full weight of this theory, but do make use of the notion
of context in the proof the main theorem. 

\begin{mathpar}
  \inferrule* [lab=summation] {} {{M_{M},M_{N}} \bc \Box \;|\; x.M_{A} \;|\; M_{M}+M_{N}}
  \and
  \inferrule* [lab=agent] {} {{M_{A}} \bc (\vec{x})M_{P} \;| \; \clift{P_0,\ldots,M_{P},\ldots,P_N}}
  \and \\
  \inferrule* [lab=process] {} {{M_{P}} \bc M_{N} \;| \;P|M_{P} }
\end{mathpar} 

\begin{mathpar}
  \inferrule* [lab=sychronization] {} {M_{N} \bc \Box \;|\; x?M_{F} \;|\; x!M_{C}}
  \and
  \inferrule* [lab=abstraction] {} {{M_{F}} \bc (x)M_{P} }
  \and
  \inferrule* [lab=concretion] {} {{M_{C}} \bc \langle M_{P} \rangle }
  \and \\
  \inferrule* [lab=process] {} {{M_{P}} \bc M_{N} \;| \;P|M_{P} }
\end{mathpar}

\begin{definition}[contextual application] Given a context $M$, and
  process $P$, we define the \emph{contextual application}, $M[P] :=
  M\{P/\Box\}$. That is, the contextual application of M to P is the
  substitution of $P$ for $\Box$ in $M$.
\end{definition}

$\meaningof{-} : L \to \mathcal{P}(\pi)$

\begin{mathpar}
  \inferrule* [lab=collection] {} {\meaningof{true} = \pi, \and \meaningof{~E} = \pi \setminus \meaningof{E}, \and \meaningof{E_{1} \& E_{2}} = \meaningof{E_{1}} \cap \meaningof{E_{2}}}
\end{mathpar}

\begin{mathpar}
  \inferrule* [lab=structure] {} {\meaningof{0} = \{ P \in \pi | P \equiv 0 \}, \and \\ \meaningof{E_1 | E_2} = \{ P \in \pi | P \equiv P_{1} | P_{2}, P_{1} \in \meaningof{E_{1}}, P_{2} \in \meaningof{E_2}\} }
\end{mathpar}

\begin{mathpar}
 \inferrule* [lab=behavior] {} {\meaningof{\langle a?b \rangle E} = \{ P \in \pi | P \equiv Q | u?(y)P', \\ \and \\\\ \and \\ \;\;\; u \in \meaningof{a}, \forall z.P'\{z/y\} \in \meaningof{E\{z/b\}}\}, \and \\ \meaningof{a!E} = \{ P \in \pi | P \equiv Q | x!\langle P' \rangle, x \in \meaningof{a} P' \in \meaningof{E}\} }
\end{mathpar}

\begin{mathpar}
 \inferrule* [lab=nominal] {} {\meaningof{\quotep{E}} = \{ \quotep{P} \in \quotep{\pi} | P \in \meaningof{E} \}, \and \meaningof{\quotep{P}} = \{ \quotep{Q} \in \quotep{\pi} | P \equiv Q \} \and \\ \meaningof{@\quotep{E}} = \{ P \in \pi | P \equiv @x, x \in \meaningof{E} \}}
\end{mathpar}

\begin{eqnarray*}
  \\
  \meaningof{-} : TS \to ST
\end{eqnarray*}

\begin{eqnarray*}
  \\
  L : TS \to ST
\end{eqnarray*}

\begin{eqnarray*}
  \\
  P \models E \iff P \in \meaningof{E}
\end{eqnarray*}

\begin{eqnarray*}
  P \approx_{L} Q \iff \forall E \in L. P \models E \iff Q \models E
\end{eqnarray*}

\begin{eqnarray*}
  P \approx_{K} Q
\end{eqnarray*}

\begin{eqnarray*}
  P \approx Q
\end{eqnarray*}

$\approx_{K} = \approx = \approx_{L}$

\subsubsection{Contextual duality}

Note that contexts extend the quotation operation to a family of
operations from processes to names. Given a context, $M$, we can
define a \emph{nominal context}, $\quotep{M}$ by $\quotep{M}[P] :=
\quotep{M[P]}$. To foreshadow what is to come we observe that these
operations enjoy a duality with processes very much like the duality
between vectors and maps from vectors to scalars.

Further, because the calculus is essentially higher-order, we have a
correspondence between contexts and processes. More specifically,
given a name $x$ and a context $M$ we can construct $M^{*}_{x}$ such
that 

\begin{mathpar}
  M^{*}_{x} | \lift{x}{P} \red M[P]
\end{mathpar}

namely,

\begin{mathpar}
  M^{*}_{x} := x?(u).M[\dropn{u}]
\end{mathpar}

The dependence of $M^{*}_{x}$ on a name makes it an abstraction, 

\begin{mathpar}
  M^{*} := (x)x?(u).M[\dropn{u}]
\end{mathpar}

\subsection{Additional notation}

It will sometimes be convenient to denote the process a name
quotes. We already have the notation $x = \quotep{P}$, but it will be
convenient to introduce an alternate notation, $\procn{x}$, when we
want to emphasize the connection to the use of the name. Note that, by
virtue of name equivalence, $\quotep{\procn{x}} \nameeq x$; so, the
notation is consistent with previous definitions.

Further, because names have structure it is possible to effect
substitutions on the basis of that structure. This means we need to
upgrade our notation for substitutions, which we accomplish by
adapting comprehension notation. Thus,

\begin{mathpar}
  P\{ y / x : x \in S \}
\end{mathpar}

is interpreted to mean the process derived from P by replacing (in a
capture-avoiding manner) each occurrence of $x$ in $S$ by $y$. For example,

\begin{mathpar}
  P\{ \quotep{\procn{x}|\procn{x}} / x : x \in \freenames{P} \}
\end{mathpar}

will replace each (occurrence) of a free name $x$ in $P$ by
$\quotep{\procn{x}|\procn{x}}$.

Also, we will avail ourselves of the notation $x^{L}$ and $x^{R}$ to
denote injections of a name into disjoint copies of the name
space. There are numerous ways to accomplish this. One example can be
found in \cite{MeredithR05}. This notation overloads to vectors of
names: $\vec{x}^{\pi} := (x_{i}^{\pi} \; : \; 0 \leq i < |\vec{x}| )$ where $\pi \in \{L,R\}$.

We also use $P^{\Box} := P|\Box$.

In \cite{MeredithR05} an interpretation of the new operator is
given. It turns out that there are several possible interpretations
all enjoying the requisite algebraic properties of the operator (see
\cite{milner91polyadicpi}). We will therefore make liberal use of
$(\nu\; \vec{x})P$.

% subsection the_syntax_and_semantics_of_the_notation_system (end)   

\input{qm2pi.qmops} 

\input{qm2pi.sterngerlach} 

\input{qm2pi.metric} 

% section concurrent_process_calculi (end)

%\input{qm2pi.proofsketch}

% section proof sketch (end)

%\input{qm2pi.slviaknots} 

% section spatial logic via knots (end)

\input{qm2pi.conclusion}

% section conclusion (end)

%\input{qm2pi.dtcodes} 

% section wiring algorithm (end)

\input{qm2pi.ack} 

% section acknowledgments (end)

\newpage


\bibliographystyle{plain}   
\bibliography{../../biblios/main.bib}

\input{qm2pi.rhodetails}

\end{document}

 

% section concurrent_process_calculi (end)

%\documentclass[12pt]{llncs}
%\documentclass{jktr}

\usepackage[pdftex]{hyperref}                   
\usepackage {listings}
\usepackage {mathpartir}
\usepackage{bcprules}
%\usepackage{listings}
                       
\usepackage{graphicx} 
%\usepackage[margins=2.5cm,nohead,nofoot]{geometry}
%\usepackage{geometry}
\usepackage{amsfonts}
\usepackage{amstext}
\usepackage{latexsym}
\usepackage{amssymb}
\usepackage{color}


%\include{myPreamble}
\include{qm2pi.local} 

%\ifpdf
%\usepackage[pdftex]{graphicx}
%\else
%\usepackage{graphicx}
%\fi

 % \ifpdf
%  \usepackage{pdfsync}
%  \if


%\title{Brief Article}
%\author{David F. Snyder}
%\author{L.G. Meredith}

%\address{Dept. of Math., Texas State University--San Marcos, San Marcos, TX 78666}
       
\pagestyle{empty}


\begin{document}

\lstset{language=[Objective]Caml,frame=shadowbox}

\input{qm2pi.front}

% section front matter (end)

\input{qm2pi.intro} 
 
% section introduction (end)

% \input{qm2pi.knotations} 

% section notation (end)

\input{qm2pi.process.calculi} 

% section concurrent_process_calculi_and_spatial_logics_ (end)
    
%\input{qm2pi.knots2pi} 

%\input{qm2pi.trefoil} 

%\input{qm2pi.mainthm} 

% subsection basic_interpretation (end)

%\input{qm2pi.rho.presentation} 
\subsection{The syntax and semantics of the notation system}\label{sub:the_syntax_and_semantics_of_the_notation_system} % (fold)

We now summarize a technical presentation of the calculus that
embodies our theory of dynamics. The typical presentation of such a
calculus follows the style of giving generators and relations on
them. The grammar, below, describing term constructors, freely
generates the set of processes, $\Proc$. This set is then quotiented
by a relation known as structural congruence and it is over this set
that the notion of dynamics is expressed. This presentation is
essentially that of \cite{MeredithR05} with the addition of
polyadicity and summation. For readability we have relegated some of
the technical subtleties to an appendix.

\subsubsection{Process grammar}\label{subsub:process_grammar}

\begin{mathpar}
  \inferrule* [lab=synchronization] {} {{M} \bc \pzero \;|\; x?F \;|\; x!C }
  \and
  \inferrule* [lab=abstraction] {} {{F} \bc (x)P}
  \and
  \inferrule* [lab=concretion] {} {{C} \bc \langle Q \rangle}
  \and
  \inferrule* [lab=process] {} {{P,Q} \bc M \;| \;P|Q \;|\; @{x}}
  \and
  \inferrule* [lab=name] {} {{x} \bc \quotep{P}}
\end{mathpar} 

Note that $\vec{x}$ (resp. $\vec{P}$) denotes a vector of names
(resp. processes) of length $|\vec{x}|$ (resp. $|\vec{P}|$). We adopt
the following useful abbreviations.

\begin{mathpar}
   x?(\vec{y}).P := x.(\vec{y})P \and  x\clift{\vec{P}} := x.\clift{\vec{P}}
   \and x!(y) := \lift{x}{\dropn{y}}
   \and \Pi_{i=0}^{n-1}P_i := P_0 | \ldots | P_{n-1}
\end{mathpar}

\subsubsection{Structural congruence}

\paragraph{Free and bound names and alpha-equivalence.} At the
core of structural equivalence is alpha-equivalence which identifies
process that are the same up to a change of variable. Formally, we
recognize the distinction between free and bound names. The free names
of a process, $\freenames{P}$, may be calculated recursively as
follows:

\begin{mathpar}
\freenames{\pzero} := \emptyset
  \and \\
  \freenames{x?(y).P} := \{ x \} \cup (\freenames{P} \setminus \{ y \})
  \and 
  \freenames{x!\langle P \rangle} := \{ x \} \cup \{ P \} 
  \and \\
  \freenames{P|Q} := \freenames{P} \cup \freenames{Q}
  \and \\
  \freenames{@{x}} := \{ x \}
\end{mathpar}

$\pi$
$\quotep{\pi}$

$\freenames{-} : \pi \to \mathcal{P}(\quotep{\pi})$

\begin{eqnarray*}
  \freenames{\pzero} & := & \emptyset \\
  \freenames{x?(y).P} & := & \{ x \} \cup (\freenames{P} \setminus \{ y \}) \\
  \freenames{x!\langle P \rangle} & := & \{ x \} \cup \{ P \} \\
  \freenames{P|Q} & := & \freenames{P} \cup \freenames{Q} \\
  \freenames{\dropn{x}} & := & \{ x \}
\end{eqnarray*}

The bound names of a process, $\boundnames{P}$, are those names occurring in $P$
that are not free. For example, in $x?(y).0$, the name $x$ is free, while $y$ is bound.

\begin{mathpar}
  \inferrule* [lab=monoidal-laws] {} { P|Q \equiv Q|P \and P|0 \equiv P \and P|(Q|R) \equiv (P|Q)|R }
\end{mathpar}

\begin{mathpar}
  \inferrule* [lab=alpha-equivalence] {} { (x)P \equiv (y)P\{y/x\} \and y \not\in \freenames{P} }
\end{mathpar}

\begin{definition}
Then two processes, $P,Q$, are alpha-equivalent if $P = Q\{\vec{y}/\vec{x}\}$ for
some $\vec{x} \in \boundnames{Q},\vec{y} \in \boundnames{P}$, where $Q\{\vec{y}/\vec{x}\}$
denotes the capture-avoiding substitution of $\vec{y}$ for $\vec{x}$ in $Q$.
\end{definition}

\begin{definition}
  The {\em structural congruence} \cite{SangiorgiWalker} , $\equiv$,
  between processes is the least congruence containing
  alpha-equivalence, satisfying the abelian monoid laws
  (associativity, commutativity and $\pzero$ as identity) for parallel
  composition $|$ and for summation $+$.
\end{definition}

\subsection{Name equivalence}

We take name equivalence, written $\nameeq$, to be the smallest
equivalence relation generated by the following rules.

\begin{mathpar}
\inferrule*[lab=Quote-drop]
{ }
{ \quotep{@{x}} \nameeq x }

\inferrule*[lab=Struct-equiv]
{ P \scong Q }
{ \quotep{P} \nameeq \quotep{Q} }
\end{mathpar}

The astute reader will have noticed that the mutual recursion of names
and processes imposes a mutual recursion on alpha-equivalence and
structural equivalence via name-equivalence. Fortunately, all of this
works out pleasantly and we may calculate in the natural way, free of
concern. The reader interested in the details is referred to the
appendix \ref{appendix:rho_details}.

\subsection{Substitution}

We use $\Proc$ for the set of processes, $\QProc$ for the set of
names, and $\id{\{}\vec{y} / \vec{x} \id{\}}$ to denote partial maps,
$s : \QProc \rightarrow \QProc$. A map, $s$ lifts, uniquely, to a map
on process terms, $\widehat{s} : \Proc \rightarrow \Proc$ by the
following equations.

\begin{mathpar}
  (0) \psubstp{Q}{P} := 0 \\
  (R \juxtap S) \psubstp{Q}{P}
  :=    
  (R)\psubstp{Q}{P} \juxtap (S) \psubstp{Q}{P} \\
  (x?(y).R) \psubstp{Q}{P}    
  :=    
  (x)\substp{Q}{P} (z)\concat( (R \psubstn{z}{y}) \psubstp{Q}{P} ) \\
  (\lift{x}{R}) \psubstp{Q}{P}  
  :=
  \lift{(x)\substp{Q}{P}}{ R \psubstp{Q}{P} } \\
%   (\dropn{x})  \psubstp{Q}{P}       
%   := 
%   \left\{ 
%     \begin{array}{ccc} 
%       \dropn{\quotep{Q}} & & x \nameeq \quotep{P} \\
%       \dropn{x} & & otherwise \\
%     \end{array}
%   \right. 
  (\dropn{x})  \psubstp{Q}{P}       
  := 
  \left\{ 
    \begin{array}{ccc} 
      Q & & x \nameeq \quotep{P} \\
      \dropn{x} & & otherwise \\
    \end{array}
  \right.
\end{mathpar}
 

where

\begin{eqnarray}
  (x)\id{\{} \lpquote Q \rpquote / \lpquote P \rpquote \id{\}}            = 
  \left\{ 
    \begin{array}{ccc}
      \lpquote Q \rpquote & & x \nameeq \lpquote P \rpquote \\
      x & & otherwise \\
    \end{array}
  \right. \nonumber
\end{eqnarray}

and $z$ is chosen distinct from $\quotep{P}$, $\quotep{Q}$, the free
names in $Q$, and all the names in $R$. Our $\alpha$-equivalence will
be built in the standard way from this substitution.

\begin{remark}\label{rem:no_self_referential_names}
  One consequence of these definitions is that $\forall P. \quotep{P}
  \not\in \freenames{P}$.
\end{remark}

\subsection{ Dynamic quote: an example }

Anticipating something of what's to come, consider applying the
substitution, $\widehat{\id{\{}u / z \id{\}}}$, to the following pair
of processes, $\lift{w}{y!(z)}$ and $w[ \lpquote y!(z) \rpquote ]$.

\begin{eqnarray}
	\lift{w}{y!(z)}\widehat{\id{\{}u / z \id{\}}}
		& = &
		\lift{w}{y!(u)} \nonumber\\
	w[ \lpquote y!(z) \rpquote ] \widehat{ \id{\{}u / z \id{\}} }
		& = &
		w[ \lpquote y!(z) \rpquote ] \nonumber
\end{eqnarray}

Because the body of the process between quotes is impervious to
substitution, we get radically different answers. In fact, by
examining the first process in an input context,
e.g. $x?(z).\lift{w}{y!(z)}$, we see that the process under the lift
operator may be shaped by prefixed inputs binding a name inside it. In
this sense, the lift operator will be seen as a way to dynamically
construct processes before reifying them as names.

Finally equipped with these standard features we can present the
dynamics of the calculus.

\subsubsection{Operational semantics} 

Finally, we introduce the computational dynamics. What marks these
algebras as distinct from other more traditionally studied algebraic
structures, e.g. vector spaces or polynomial rings, is the manner in
which dynamics is captured. In traditional structures, dynamics is typically
expressed through morphisms between such structures, as in linear maps
between vector spaces or morphisms between rings. In algebras
associated with the semantics of computation, the dynamics is
expressed as part of the algebraic structure itself, through a
reduction reduction relation typically denoted by $\red$. Below, we
give a recursive presentation of this relation for the calculus used
in the encoding.

$\red \subseteq \pi \times \pi$
$\red : \pi \to \mathcal{P}(\pi)$

\begin{mathpar}
  \inferrule* [lab=Comm] { \textsf{match}( x_{src}, x_{trgt} ) } { x_{trgt}?(y)P \; | \; x_{src}!\langle {Q} \rangle \red P\{\quotep{Q}/y}\} }
  \and \\
  \inferrule* [lab=Par] {{P} \red {P}'} {{{P} | {Q}} \red {{P}' | {Q}}}
  \and
  \inferrule* [lab=Equiv]{{{P} \scong {P}'} \andalso {{P}' \red {Q}'} \andalso {{Q}' \scong {Q}}}{{P} \red {Q}}
\end{mathpar}

\begin{eqnarray*}
  match_{\equiv} (\quotep{P},\quotep{Q}) & := & P \equiv Q \\
  match_{\dagger}(\quotep{P},\quotep{Q}) & := & \forall R. P|Q \red^{*} R => R \red^{*} 0 \\
  match_{K}(\quotep{P},\quotep{Q}) & := & K \mbox{ for some context } K
\end{eqnarray*}

$u?(x)P | u!\langle Q \rangle \red P\{\quotep{Q}/x\}$

%We write $\wred$ for $\red^*$, and $P\red$ if $\exists Q $ such that $ P \red Q$.
We write $P\red$ if $\exists Q $ such that $ P \red Q$ and $P\not\red$, otherwise.

\section{Replication}

As mentioned before, it is known that replication (and hence
recursion) can be implemented in a higher-order process algebra
\cite{SangiorgiWalker}. As our first example of calculation with the
machinery thus far presented we give the construction explicitly in
the {\rhoc}.

\begin{eqnarray}
	D_{x} & := & \prefix{x}{y}{(\binpar{\outputp{x}{y}}{@{y}})} \nonumber\\
	\bangp_{x}{P} & := & \binpar{{x}!\langle{\binpar{D_{x}}{P}}\rangle}{D_{x}} \nonumber
\end{eqnarray}

\begin{eqnarray}
	\bangp_{x}{P} & & \nonumber\\
	=
	& {x}!\langle{(\prefix{x}{y}{(\outputp{x}{y} | @{y})) | P}}\rangle 
	      | \prefix{x}{y}{(\outputp{x}{y} | @{y})} & \nonumber\\
	\red
	& (\outputp{x}{y} | @{y})\substn{\quotep{(\prefix{x}{y}{(@{y} | \outputp{x}{y})) | P}}}{y} & \nonumber\\
	=
	& \outputp{x}{\quotep{(\prefix{x}{y}{(\outputp{x}{y} | @{y})) | P}}}
	  | {(\prefix{x}{y}{(\outputp{x}{y} | @{y})) | P}} & \nonumber\\
	\red
	& \ldots & \nonumber\\
	\red^*
	& P | P | \ldots & \nonumber
\end{eqnarray}

Of course, this encoding, as an implementation, runs away, unfolding
$\bangp{P}$ eagerly. A lazier and more implementable replication
operator, restricted to input-guarded processes, may be obtained as follows.

\begin{eqnarray}
\bangp{\prefix{u}{v}{P}} 
	:= 
	\binpar{\lift{x}{\prefix{u}{v}{(\binpar{D(x)}{P})}}}{D(x)} \nonumber
\end{eqnarray}

\begin{remark}
  Note that the lazier definition still does not deal with summation
  or mixed summation (i.e. sums over input and output). The reader is
  invited to construct definitions of replication that deal with these
  features. 

  Further, the definitions are parameterized in a name, $x$. Can you,
  gentle reader, make a definition that eliminates this parameter and
  guarantees no accidental interaction between the replication
  machinery and the process being replicated -- i.e. no accidental
  sharing of names used by the process to get its work done and the
  name(s) used by the replication to effect copying. This latter
  revision of the definition of replication is crucial to obtaining
  the expected identity $!!P \sim !P$.
\end{remark}

\begin{remark}\label{rem:paradoxical_combinator}
  The reader familiar with the lambda calculus will have noticed the
  similarity between $D$ and the paradoxical combinator.

  [Ed. note: the existence of this seems to suggest we have to be more
  restrictive on the set of processes and names we admit if we are to
  support no-cloning.]
\end{remark}

\subsubsection{Bisimulation}

The computational dynamics gives rise to another kind of equivalence,
the equivalence of computational behavior. As previously mentioned
this is typically captured \emph{via} some form of bisimulation.

% The notion we use in this paper is weak barbed bisimulation
% \cite{milner91polyadicpi}.

The notion we use in this paper is derived from weak barbed
bisimulation \cite{milner91polyadicpi}. 

\begin{definition}
An \emph{observation relation}, $\downarrow_{\mathcal N}$, over a set
of names, $\mathcal N$, is the smallest relation satisfying the rules
below.

\infrule[Out-barb]{y \in {\mathcal N}, \; x \nameeq y}
		  {\outputp{x}{v} \downarrow_{\mathcal N} x}
\infrule[Par-barb]{\mbox{$P\downarrow_{\mathcal N} x$ or $Q\downarrow_{\mathcal N} x$}}
		  {\binpar{P}{Q} \downarrow_{\mathcal N} x}

We write $P \Downarrow_{\mathcal N} x$ if there is $Q$ such that 
$P \wred Q$ and $Q \downarrow_{\mathcal N} x$.
\end{definition}

\begin{definition}
%\label{def.bbisim}
An  ${\mathcal N}$-\emph{barbed bisimulation} over a set of names, ${\mathcal N}$, is a symmetric binary relation 
${\mathcal S}_{\mathcal N}$ between agents such that $P\rel{S}_{\mathcal N}Q$ implies:
\begin{enumerate}
\item If $P \red P'$ then $Q \wred Q'$ and $P'\rel{S}_{\mathcal N} Q'$.
\item If $P\downarrow_{\mathcal N} x$, then $Q\Downarrow_{\mathcal N} x$.
\end{enumerate}
$P$ is ${\mathcal N}$-barbed bisimilar to $Q$, written
$P \wbbisim_{\mathcal N} Q$, if $P \rel{S}_{\mathcal N} Q$ for some ${\mathcal N}$-barbed bisimulation ${\mathcal S}_{\mathcal N}$.
\end{definition}

$\mathcal{R} \subseteq \pi \times \pi$

$P \mathcal{R} Q => \forall P'. P \red P' \Rightarrow \exists Q'. Q \red Q', P' \mathcal{R} Q'$

$P \vdash x \Rightarrow Q \vdash x$

\begin{mathpar}
  \inferrule*[lab=Out-barb]{x \nameeq y}{{y}!\langle{Q}\rangle \vdash x}
  \and
  \inferrule*[lab=Par-barb]{\mbox{$P\vdash x$ or $Q\vdash x$}}{\binpar{P}{Q} \vdash x}
\end{mathpar}

\subsubsection{Contexts}

One of the principle advantages of computational calculi like the
$\pi$-calculus is a well-defined notion of context,
contextual-equivalence and a correlation between
contextual-equivalence and notions of bisimulation. The notion of
context allows the decomposition of a process into (sub-)process and
its syntactic environment, its context. Thus, a context may be
thought of as a process with a ``hole'' (written $\Box$) in it. The
application of a context $M$ to a process $P$, written $M[P]$, is
tantamount to filling the hole in $M$ with $P$. In this paper we do
not need the full weight of this theory, but do make use of the notion
of context in the proof the main theorem. 

\begin{mathpar}
  \inferrule* [lab=summation] {} {{M_{M},M_{N}} \bc \Box \;|\; x.M_{A} \;|\; M_{M}+M_{N}}
  \and
  \inferrule* [lab=agent] {} {{M_{A}} \bc (\vec{x})M_{P} \;| \; \clift{P_0,\ldots,M_{P},\ldots,P_N}}
  \and \\
  \inferrule* [lab=process] {} {{M_{P}} \bc M_{N} \;| \;P|M_{P} }
\end{mathpar} 

\begin{mathpar}
  \inferrule* [lab=sychronization] {} {M_{N} \bc \Box \;|\; x?M_{F} \;|\; x!M_{C}}
  \and
  \inferrule* [lab=abstraction] {} {{M_{F}} \bc (x)M_{P} }
  \and
  \inferrule* [lab=concretion] {} {{M_{C}} \bc \langle M_{P} \rangle }
  \and \\
  \inferrule* [lab=process] {} {{M_{P}} \bc M_{N} \;| \;P|M_{P} }
\end{mathpar}

\begin{definition}[contextual application] Given a context $M$, and
  process $P$, we define the \emph{contextual application}, $M[P] :=
  M\{P/\Box\}$. That is, the contextual application of M to P is the
  substitution of $P$ for $\Box$ in $M$.
\end{definition}

$\meaningof{-} : L \to \mathcal{P}(\pi)$

\begin{mathpar}
  \inferrule* [lab=collection] {} {\meaningof{true} = \pi, \and \meaningof{~E} = \pi \setminus \meaningof{E}, \and \meaningof{E_{1} \& E_{2}} = \meaningof{E_{1}} \cap \meaningof{E_{2}}}
\end{mathpar}

\begin{mathpar}
  \inferrule* [lab=structure] {} {\meaningof{0} = \{ P \in \pi | P \equiv 0 \}, \and \\ \meaningof{E_1 | E_2} = \{ P \in \pi | P \equiv P_{1} | P_{2}, P_{1} \in \meaningof{E_{1}}, P_{2} \in \meaningof{E_2}\} }
\end{mathpar}

\begin{mathpar}
 \inferrule* [lab=behavior] {} {\meaningof{\langle a?b \rangle E} = \{ P \in \pi | P \equiv Q | u?(y)P', \\ \and \\\\ \and \\ \;\;\; u \in \meaningof{a}, \forall z.P'\{z/y\} \in \meaningof{E\{z/b\}}\}, \and \\ \meaningof{a!E} = \{ P \in \pi | P \equiv Q | x!\langle P' \rangle, x \in \meaningof{a} P' \in \meaningof{E}\} }
\end{mathpar}

\begin{mathpar}
 \inferrule* [lab=nominal] {} {\meaningof{\quotep{E}} = \{ \quotep{P} \in \quotep{\pi} | P \in \meaningof{E} \}, \and \meaningof{\quotep{P}} = \{ \quotep{Q} \in \quotep{\pi} | P \equiv Q \} \and \\ \meaningof{@\quotep{E}} = \{ P \in \pi | P \equiv @x, x \in \meaningof{E} \}}
\end{mathpar}

\begin{eqnarray*}
  \\
  \meaningof{-} : TS \to ST
\end{eqnarray*}

\begin{eqnarray*}
  \\
  L : TS \to ST
\end{eqnarray*}

\begin{eqnarray*}
  \\
  P \models E \iff P \in \meaningof{E}
\end{eqnarray*}

\begin{eqnarray*}
  P \approx_{L} Q \iff \forall E \in L. P \models E \iff Q \models E
\end{eqnarray*}

\begin{eqnarray*}
  P \approx_{K} Q
\end{eqnarray*}

\begin{eqnarray*}
  P \approx Q
\end{eqnarray*}

$\approx_{K} = \approx = \approx_{L}$

\subsubsection{Contextual duality}

Note that contexts extend the quotation operation to a family of
operations from processes to names. Given a context, $M$, we can
define a \emph{nominal context}, $\quotep{M}$ by $\quotep{M}[P] :=
\quotep{M[P]}$. To foreshadow what is to come we observe that these
operations enjoy a duality with processes very much like the duality
between vectors and maps from vectors to scalars.

Further, because the calculus is essentially higher-order, we have a
correspondence between contexts and processes. More specifically,
given a name $x$ and a context $M$ we can construct $M^{*}_{x}$ such
that 

\begin{mathpar}
  M^{*}_{x} | \lift{x}{P} \red M[P]
\end{mathpar}

namely,

\begin{mathpar}
  M^{*}_{x} := x?(u).M[\dropn{u}]
\end{mathpar}

The dependence of $M^{*}_{x}$ on a name makes it an abstraction, 

\begin{mathpar}
  M^{*} := (x)x?(u).M[\dropn{u}]
\end{mathpar}

\subsection{Additional notation}

It will sometimes be convenient to denote the process a name
quotes. We already have the notation $x = \quotep{P}$, but it will be
convenient to introduce an alternate notation, $\procn{x}$, when we
want to emphasize the connection to the use of the name. Note that, by
virtue of name equivalence, $\quotep{\procn{x}} \nameeq x$; so, the
notation is consistent with previous definitions.

Further, because names have structure it is possible to effect
substitutions on the basis of that structure. This means we need to
upgrade our notation for substitutions, which we accomplish by
adapting comprehension notation. Thus,

\begin{mathpar}
  P\{ y / x : x \in S \}
\end{mathpar}

is interpreted to mean the process derived from P by replacing (in a
capture-avoiding manner) each occurrence of $x$ in $S$ by $y$. For example,

\begin{mathpar}
  P\{ \quotep{\procn{x}|\procn{x}} / x : x \in \freenames{P} \}
\end{mathpar}

will replace each (occurrence) of a free name $x$ in $P$ by
$\quotep{\procn{x}|\procn{x}}$.

Also, we will avail ourselves of the notation $x^{L}$ and $x^{R}$ to
denote injections of a name into disjoint copies of the name
space. There are numerous ways to accomplish this. One example can be
found in \cite{MeredithR05}. This notation overloads to vectors of
names: $\vec{x}^{\pi} := (x_{i}^{\pi} \; : \; 0 \leq i < |\vec{x}| )$ where $\pi \in \{L,R\}$.

We also use $P^{\Box} := P|\Box$.

In \cite{MeredithR05} an interpretation of the new operator is
given. It turns out that there are several possible interpretations
all enjoying the requisite algebraic properties of the operator (see
\cite{milner91polyadicpi}). We will therefore make liberal use of
$(\nu\; \vec{x})P$.

% subsection the_syntax_and_semantics_of_the_notation_system (end)   

\input{qm2pi.qmops} 

\input{qm2pi.sterngerlach} 

\input{qm2pi.metric} 

% section concurrent_process_calculi (end)

%\input{qm2pi.proofsketch}

% section proof sketch (end)

%\input{qm2pi.slviaknots} 

% section spatial logic via knots (end)

\input{qm2pi.conclusion}

% section conclusion (end)

%\input{qm2pi.dtcodes} 

% section wiring algorithm (end)

\input{qm2pi.ack} 

% section acknowledgments (end)

\newpage


\bibliographystyle{plain}   
\bibliography{../../biblios/main.bib}

\input{qm2pi.rhodetails}

\end{document}



% section proof sketch (end)

%\section{Unlikely characters: spatial logic for
  knots}\label{sub:characteristic_formulae} % (fold)

Associated to the mobile process calculi are a family of logics known
as the Hennessy-Milner logics. These logics typically enjoy a
semantics interpreting formulae as sets of processes that when
factored through the encoding outlined above allows an identification
of classes of knots with logical formulae. In the context of this
encoding the sub-family known as the spatial logics \cite{CairesC03}
\cite{CairesC04} \cite{Caires04} are of particular interest providing
several important features for expressing and reasoning about
properties (i.e. classes) of knots. We hint here at how this may be done.

%\begin{description}
%\item [structural connectives] 
\subsubsection{Structural connectives} The spatial logics enjoy
structural connectives corresponding, at the logical level, to the
parallel composition ($P | Q$) and new name ($(\nu \; x)P$)
connectives for processes. As illustrated in the examples below, these
connectives are extremely expressive given the shape of our encoding.
%\item [decideable satisfaction]

\subsubsection{Decideable satisfaction}
In \cite{Caires04} the satisfaction relation is shown to be decideable
for a rich class of processes. It further turns out that the image of
the our encoding is a proper subset of that class. This result
provides the basis for an algorithm by which to search for knots
enjoying a given property.
%\item [characteristic formulae]

\subsubsection{Characteristic formulae}
In the same paper \cite{Caires04} , Caires presents a means of calculating
characteristic formulae, selecting equivalence classes of processes
up to a pre--specified depth limit on the support set of names. Composed with our
encoding, this characteristic formula can be used to select
characteristic formulae for knots.
%\end{description}

\subsubsection{Spatial logic formulae}

The grammar below (segmented for comprehension) summarizes the syntax
of spatial logic formulae. We employ illustrative examples in the
sequel to provide an intuitive understanding of their meaning
referring the reader to \cite{Caires04} for a more detailed explication
of the semantics.

\begin{mathpar}
  \inferrule* [lab=boolean] {} {{A,B} \bc T \;|\; \neg A \;|\; A \wedge B \;|\; \eta = \eta'}
  \and
  \inferrule* [lab=spatial] {} {|\; \pzero \;|\; A | B \;|\; x \text{\textregistered} A \;|\; \forall x . A \;|\;  H x . A}
  \and
  \inferrule* [lab=behavioral] {} {|\; \alpha . A}
  \and 
  \inferrule* [lab=recursion] {} {|\; X(\vec{u}) \;|\; \mu X(\vec{u}) . A}
  \and
  \inferrule* [lab=action] {} {\alpha \bc \langle x?(\vec{y}) \rangle \;|\; \langle x!(\vec{y}) \rangle \;|\; \langle \tau \rangle}
  \and 
  \inferrule* [lab=name] {} {\eta \bc x \;|\; \tau}
\end{mathpar} 

% subsection characteristic_formulae (end)   	 

\subsection{Example formulae}\label{sub:example_formulae_} % (fold)

\subsubsection{Crossing as formula.}
% 
% \begin{align*}
%   \frac{d}{dx} \sin x &= \cos x 
%   & \frac{d}{dx} e^x &= e^x \\
%   \frac{d}{dx} \cos x &= - \sin x 
%   & \frac{d}{dx} \log x &= \frac{1}{x} \\
% \end{align*} 

\begin{align*}
 \mu C(x_{0},x_{1},y_{0},y_{1},u).&(\langle x_{0}?(z) \rangle(\langle u! \rangle\langle y_{1}!z \rangle C(x_{0},x_{1},y_{0},y_{1},u)) & \\
  & \wedge \langle y_{1}?(z) \rangle (\langle u! \rangle \langle x_{0}!z \rangle C(x_{0},x_{1},y_{0},y_{1},u)) & \\
  & \wedge \langle x_{1}?(z) \rangle (\langle u? \rangle \langle y_{0}!z \rangle C(x_{0},x_{1},y_{0},y_{1},u)) & \\
  & \wedge \langle y_{0}?(z) \rangle (\langle u? \rangle \langle x_{1}!z \rangle C(x_{0},x_{1},y_{0},y_{1},u))) &
\end{align*}

The lexicographical similarity between the shape of this formulae and
the shape of definition of the process representing a crossing reveals
the intuitive meaning of this formulae. It describes the capabilities
of a process that has the right to represent a crossing. For example
it picks out processes that may perform an input on the port $x_0$ in
its initial menu of capabilities. What differentiates the formula
from the process, however, is that the crossing process is the
smallest candidate to satisfy the formula. Infinitely many other
processes -- with internal behavior hidden behind this interface, so
to speak -- also satisfy this formula. Even this simple formula,
then, can be seen to open a new view onto knots, providing a
computational interpretation of \emph{virtual} knots.

Note that this formula is derived by hand. A similar formula can be
derived by employing Caires' calculation of characteristic formula
\cite{Caires04} to the process representing a crossing. In light of
this discussion, we let
$\meaningof{C}_{\phi}(x0,x1,y0,y1,u)$ denote a formula specifying the
dynamics we wish to capture of a crossing. To guarantee we preserve
the shape of the interface and minimal semantics we demand that
$\meaningof{C}_{\phi}(x0,x1,y0,y1,u) \Rightarrow
\textbf{C}(x0,x1,y0,y1,u)$ where $\textbf{C}(x0,x1,y0,y1,u)$ denotes
the formula above.
                            
\subsubsection{Crossing number constraints.}
The moral content of the context lemma (Lemma \ref{context}) is that the notion of
``locality'' in the Reidemeister moves is effectively captured by the
parallel composition operator of the process calculus. This intuition
extends through the logic. Given a formula,
$\meaningof{C}_{\phi}(x0,x1,y0,y1,u)$, we can use the structural
connectives to specify constraints on crossing numbers, such as at
least $n$ crossings, or exactly $n$ crossings.
\begin{mathpar}
  \inferrule* [lab=at-least-n] {} { K^{\geq n}_{\phi}(\vec{xs},\vec{ys}) := \Pi_{i=0}^{n-1} Hu . \meaningof{C}_{\phi}(xs_i,ys_i,u) | T }
  \and 
  \inferrule* [lab=exactly-n] {} { K^{= n}_{\phi}(\vec{xs},\vec{ys}) := \Pi_{i=0}^{n-1} Hu . \meaningof{C}_{\phi}(xs_i,ys_i,u) | \neg (\forall x_0,y_0,x_1,y_1,u . \meaningof{C}_{\phi}(x_0,y_0,x_1,y_1,u) | T) }
\end{mathpar}

To round out this section, recall that the encoding of an $n$-crossing
knot decomposes into a parallel composition of $n$ \emph{copies} of a
crossing process together with a wiring harness. To specify different
knot classes with the same crossing number amounts to specifying
logical constraints on the wiring harness. In the interest of space,
we defer examples to a forthcoming paper. Suffice it to say that both
the conditions ``alternating knot'' and ``contains the tangle
corresponding to 5/3'' are expressible. For example, it is possible to
calculate the characteristic formula of a process corresponding to the
tangle 5/3 and conjoin it into the classifying formula via the
composition connective of the logic.

Finally, we wish to observe that it is entirely within reason to
contemplate a more domain-specific version of spatial logic tailored
to the shape of processes in the image of the encoding. Such a
domain-specific logic would have a better claim to the title formal
language of knot properties.

% subsection example_formulae_ (end)

% section knots_as_processes (end) 

% section spatial logic via knots (end)

\section{Conclusions and future work}

\paragraph{Testing physical space}
You, gentle reader, may wonder why of all the theorems to be proved
given this set up we pick the one above. In some sense it's hardly
central to quantum mechanics. We see it as central in the sense that
it firmly establishes a notion of physical space arising from a notion
of the equivalence of behavior. Relating bisimulation to a metric is a
big step forward, but one is faced with interpreting the relationship
of that metric space to something more physical. Quantum mechanical
notions of ``physical'' space are still far from intuitive, but by
relating this idea of distance as testing to calculations that predict
physical circumstances we are making a not insignificant step forward
toward an understanding of the physical space we inhabit as
essentially dynamic.

\paragraph{Effectivity and simulation}
One of the observations we have yet to make is that the entire program
spelled out here is effective. We have built various interpreters for
the reflective calculus at work in this interpretation. In principle,
then, we can simulate quantum mechanics on a computer. The place where
the simulation may lose fidelity is the infinitely branching summation
for the annihilator.

In this connection i also want to point out that the evaluation style
calculation of the inner product puts the non-determinism of the
summation right at the heart of measurement. This suggests that
Milner's original reduction-based formulation of the dynamics of his
calculi in terms of sums was not just notationally suggestive of a
notion of measure-and-continue but captured some significant part of
the physics.

\paragraph{Quantum continuations}
In light of this last observation i want to point out that the
predominant account of quantum mechanics is missing a key aspect of a
truly compositional story of the physical situation. In a real lab,
when a measurement is made the observation can be made to feed into
another device that then makes another measurement conditioned on the
results of the first. This means that after the superposition was
collapsed the entire experimental set up remained in
superposition. While QM offers a means of writing this down it doesn't
quite line up well with the well-trodden formulation of computation
and continuation that we see so succinctly expressed in Milner's
calculi. This suggests that there might be advantages to this account
of dynamics waiting to be explored.

\paragraph{Quantum logic}
In this connection, we also note that by virtue of having the
Hennessy-Milner construction, we can pull the construction through the
interpretation of QM. This gives us a natural candidate for a quantum
logic that enjoys an extremely tight connection with it's domain of
interpretation, making the construction much less ad hoc (rather it is
the image of functor!).

\paragraph{Quantum probabiity}
i have questions about the basis of the interpretation of inner
product as probability amplitude. In particular, using which
axiomatization of probability theory does the notion of probability
amplitude earn the right to be so dubbed? In other words, where is the
proof that the operation for calculating a probability amplitude (and
then squaring) satisfies the axioms of what it means to calculate a
probability? Even if such a proof exists (i have yet to find it in the
literature), i wonder if it might not be possible to turn things on
their heads. Can we view the calculation of the probability amplitude
as an axiomatization of probability? If so, then the definition we
give for calculating probability amplitude may provide the basis for
an \emph{effective} theory of probability.

\paragraph{Quantum vs ``biological'' information}
Finally, i want to conclude with a more philosophical observation. At
a recent workshop in which QM was a predominant topic i noticed
something about quantum information. The speaker was giving a riveting
discussion of axiomatic QM and showing how properties of ``no
cloning'' and ``no deleting'' emerged as consequences of the
axiomatization. Theorems of this form are necessary to give us a sense
of confidence that our axioms characterize the physical theory. What
struck me, though, was that if quantum information is neither erasable
nor replicable it is markedly different from \emph{life}. Two of the
things we know about life is that

\begin{itemize}
  \item it ends;
  \item to gain some measure of persistence, to transcend it's
    finitude it is imminently copyable.
\end{itemize}

Both of these qualities are summarized succinctly in the aphorism: all
flesh is grass. For me these two kinds of ``information'' -- call them
quantum and biological -- are end points on a spectrum of strategies
for persistence. At one end, we have those curious entities that enjoy
uniqueness and permanence; at the other, we have those who in the face
of a certain end and an uncertain present make a go of passing
something on. To me one of the more remarkable aspects of the latter
strategy is that in the presence of noise (and certain features of
copying) we get a kind of dynamism, a chance for improvement against a
given persistent condition.

% subsection other_calculi_other_bisimulations_and_geometry_as_behavior (end)




% section conclusion (end)

%\documentclass[12pt]{llncs}
%\documentclass{jktr}

\usepackage[pdftex]{hyperref}                   
\usepackage {listings}
\usepackage {mathpartir}
\usepackage{bcprules}
%\usepackage{listings}
                       
\usepackage{graphicx} 
%\usepackage[margins=2.5cm,nohead,nofoot]{geometry}
%\usepackage{geometry}
\usepackage{amsfonts}
\usepackage{amstext}
\usepackage{latexsym}
\usepackage{amssymb}
\usepackage{color}


%\include{myPreamble}
\include{qm2pi.local} 

%\ifpdf
%\usepackage[pdftex]{graphicx}
%\else
%\usepackage{graphicx}
%\fi

 % \ifpdf
%  \usepackage{pdfsync}
%  \if


%\title{Brief Article}
%\author{David F. Snyder}
%\author{L.G. Meredith}

%\address{Dept. of Math., Texas State University--San Marcos, San Marcos, TX 78666}
       
\pagestyle{empty}


\begin{document}

\lstset{language=[Objective]Caml,frame=shadowbox}

\input{qm2pi.front}

% section front matter (end)

\input{qm2pi.intro} 
 
% section introduction (end)

% \input{qm2pi.knotations} 

% section notation (end)

\input{qm2pi.process.calculi} 

% section concurrent_process_calculi_and_spatial_logics_ (end)
    
%\input{qm2pi.knots2pi} 

%\input{qm2pi.trefoil} 

%\input{qm2pi.mainthm} 

% subsection basic_interpretation (end)

%\input{qm2pi.rho.presentation} 
\subsection{The syntax and semantics of the notation system}\label{sub:the_syntax_and_semantics_of_the_notation_system} % (fold)

We now summarize a technical presentation of the calculus that
embodies our theory of dynamics. The typical presentation of such a
calculus follows the style of giving generators and relations on
them. The grammar, below, describing term constructors, freely
generates the set of processes, $\Proc$. This set is then quotiented
by a relation known as structural congruence and it is over this set
that the notion of dynamics is expressed. This presentation is
essentially that of \cite{MeredithR05} with the addition of
polyadicity and summation. For readability we have relegated some of
the technical subtleties to an appendix.

\subsubsection{Process grammar}\label{subsub:process_grammar}

\begin{mathpar}
  \inferrule* [lab=synchronization] {} {{M} \bc \pzero \;|\; x?F \;|\; x!C }
  \and
  \inferrule* [lab=abstraction] {} {{F} \bc (x)P}
  \and
  \inferrule* [lab=concretion] {} {{C} \bc \langle Q \rangle}
  \and
  \inferrule* [lab=process] {} {{P,Q} \bc M \;| \;P|Q \;|\; @{x}}
  \and
  \inferrule* [lab=name] {} {{x} \bc \quotep{P}}
\end{mathpar} 

Note that $\vec{x}$ (resp. $\vec{P}$) denotes a vector of names
(resp. processes) of length $|\vec{x}|$ (resp. $|\vec{P}|$). We adopt
the following useful abbreviations.

\begin{mathpar}
   x?(\vec{y}).P := x.(\vec{y})P \and  x\clift{\vec{P}} := x.\clift{\vec{P}}
   \and x!(y) := \lift{x}{\dropn{y}}
   \and \Pi_{i=0}^{n-1}P_i := P_0 | \ldots | P_{n-1}
\end{mathpar}

\subsubsection{Structural congruence}

\paragraph{Free and bound names and alpha-equivalence.} At the
core of structural equivalence is alpha-equivalence which identifies
process that are the same up to a change of variable. Formally, we
recognize the distinction between free and bound names. The free names
of a process, $\freenames{P}$, may be calculated recursively as
follows:

\begin{mathpar}
\freenames{\pzero} := \emptyset
  \and \\
  \freenames{x?(y).P} := \{ x \} \cup (\freenames{P} \setminus \{ y \})
  \and 
  \freenames{x!\langle P \rangle} := \{ x \} \cup \{ P \} 
  \and \\
  \freenames{P|Q} := \freenames{P} \cup \freenames{Q}
  \and \\
  \freenames{@{x}} := \{ x \}
\end{mathpar}

$\pi$
$\quotep{\pi}$

$\freenames{-} : \pi \to \mathcal{P}(\quotep{\pi})$

\begin{eqnarray*}
  \freenames{\pzero} & := & \emptyset \\
  \freenames{x?(y).P} & := & \{ x \} \cup (\freenames{P} \setminus \{ y \}) \\
  \freenames{x!\langle P \rangle} & := & \{ x \} \cup \{ P \} \\
  \freenames{P|Q} & := & \freenames{P} \cup \freenames{Q} \\
  \freenames{\dropn{x}} & := & \{ x \}
\end{eqnarray*}

The bound names of a process, $\boundnames{P}$, are those names occurring in $P$
that are not free. For example, in $x?(y).0$, the name $x$ is free, while $y$ is bound.

\begin{mathpar}
  \inferrule* [lab=monoidal-laws] {} { P|Q \equiv Q|P \and P|0 \equiv P \and P|(Q|R) \equiv (P|Q)|R }
\end{mathpar}

\begin{mathpar}
  \inferrule* [lab=alpha-equivalence] {} { (x)P \equiv (y)P\{y/x\} \and y \not\in \freenames{P} }
\end{mathpar}

\begin{definition}
Then two processes, $P,Q$, are alpha-equivalent if $P = Q\{\vec{y}/\vec{x}\}$ for
some $\vec{x} \in \boundnames{Q},\vec{y} \in \boundnames{P}$, where $Q\{\vec{y}/\vec{x}\}$
denotes the capture-avoiding substitution of $\vec{y}$ for $\vec{x}$ in $Q$.
\end{definition}

\begin{definition}
  The {\em structural congruence} \cite{SangiorgiWalker} , $\equiv$,
  between processes is the least congruence containing
  alpha-equivalence, satisfying the abelian monoid laws
  (associativity, commutativity and $\pzero$ as identity) for parallel
  composition $|$ and for summation $+$.
\end{definition}

\subsection{Name equivalence}

We take name equivalence, written $\nameeq$, to be the smallest
equivalence relation generated by the following rules.

\begin{mathpar}
\inferrule*[lab=Quote-drop]
{ }
{ \quotep{@{x}} \nameeq x }

\inferrule*[lab=Struct-equiv]
{ P \scong Q }
{ \quotep{P} \nameeq \quotep{Q} }
\end{mathpar}

The astute reader will have noticed that the mutual recursion of names
and processes imposes a mutual recursion on alpha-equivalence and
structural equivalence via name-equivalence. Fortunately, all of this
works out pleasantly and we may calculate in the natural way, free of
concern. The reader interested in the details is referred to the
appendix \ref{appendix:rho_details}.

\subsection{Substitution}

We use $\Proc$ for the set of processes, $\QProc$ for the set of
names, and $\id{\{}\vec{y} / \vec{x} \id{\}}$ to denote partial maps,
$s : \QProc \rightarrow \QProc$. A map, $s$ lifts, uniquely, to a map
on process terms, $\widehat{s} : \Proc \rightarrow \Proc$ by the
following equations.

\begin{mathpar}
  (0) \psubstp{Q}{P} := 0 \\
  (R \juxtap S) \psubstp{Q}{P}
  :=    
  (R)\psubstp{Q}{P} \juxtap (S) \psubstp{Q}{P} \\
  (x?(y).R) \psubstp{Q}{P}    
  :=    
  (x)\substp{Q}{P} (z)\concat( (R \psubstn{z}{y}) \psubstp{Q}{P} ) \\
  (\lift{x}{R}) \psubstp{Q}{P}  
  :=
  \lift{(x)\substp{Q}{P}}{ R \psubstp{Q}{P} } \\
%   (\dropn{x})  \psubstp{Q}{P}       
%   := 
%   \left\{ 
%     \begin{array}{ccc} 
%       \dropn{\quotep{Q}} & & x \nameeq \quotep{P} \\
%       \dropn{x} & & otherwise \\
%     \end{array}
%   \right. 
  (\dropn{x})  \psubstp{Q}{P}       
  := 
  \left\{ 
    \begin{array}{ccc} 
      Q & & x \nameeq \quotep{P} \\
      \dropn{x} & & otherwise \\
    \end{array}
  \right.
\end{mathpar}
 

where

\begin{eqnarray}
  (x)\id{\{} \lpquote Q \rpquote / \lpquote P \rpquote \id{\}}            = 
  \left\{ 
    \begin{array}{ccc}
      \lpquote Q \rpquote & & x \nameeq \lpquote P \rpquote \\
      x & & otherwise \\
    \end{array}
  \right. \nonumber
\end{eqnarray}

and $z$ is chosen distinct from $\quotep{P}$, $\quotep{Q}$, the free
names in $Q$, and all the names in $R$. Our $\alpha$-equivalence will
be built in the standard way from this substitution.

\begin{remark}\label{rem:no_self_referential_names}
  One consequence of these definitions is that $\forall P. \quotep{P}
  \not\in \freenames{P}$.
\end{remark}

\subsection{ Dynamic quote: an example }

Anticipating something of what's to come, consider applying the
substitution, $\widehat{\id{\{}u / z \id{\}}}$, to the following pair
of processes, $\lift{w}{y!(z)}$ and $w[ \lpquote y!(z) \rpquote ]$.

\begin{eqnarray}
	\lift{w}{y!(z)}\widehat{\id{\{}u / z \id{\}}}
		& = &
		\lift{w}{y!(u)} \nonumber\\
	w[ \lpquote y!(z) \rpquote ] \widehat{ \id{\{}u / z \id{\}} }
		& = &
		w[ \lpquote y!(z) \rpquote ] \nonumber
\end{eqnarray}

Because the body of the process between quotes is impervious to
substitution, we get radically different answers. In fact, by
examining the first process in an input context,
e.g. $x?(z).\lift{w}{y!(z)}$, we see that the process under the lift
operator may be shaped by prefixed inputs binding a name inside it. In
this sense, the lift operator will be seen as a way to dynamically
construct processes before reifying them as names.

Finally equipped with these standard features we can present the
dynamics of the calculus.

\subsubsection{Operational semantics} 

Finally, we introduce the computational dynamics. What marks these
algebras as distinct from other more traditionally studied algebraic
structures, e.g. vector spaces or polynomial rings, is the manner in
which dynamics is captured. In traditional structures, dynamics is typically
expressed through morphisms between such structures, as in linear maps
between vector spaces or morphisms between rings. In algebras
associated with the semantics of computation, the dynamics is
expressed as part of the algebraic structure itself, through a
reduction reduction relation typically denoted by $\red$. Below, we
give a recursive presentation of this relation for the calculus used
in the encoding.

$\red \subseteq \pi \times \pi$
$\red : \pi \to \mathcal{P}(\pi)$

\begin{mathpar}
  \inferrule* [lab=Comm] { \textsf{match}( x_{src}, x_{trgt} ) } { x_{trgt}?(y)P \; | \; x_{src}!\langle {Q} \rangle \red P\{\quotep{Q}/y}\} }
  \and \\
  \inferrule* [lab=Par] {{P} \red {P}'} {{{P} | {Q}} \red {{P}' | {Q}}}
  \and
  \inferrule* [lab=Equiv]{{{P} \scong {P}'} \andalso {{P}' \red {Q}'} \andalso {{Q}' \scong {Q}}}{{P} \red {Q}}
\end{mathpar}

\begin{eqnarray*}
  match_{\equiv} (\quotep{P},\quotep{Q}) & := & P \equiv Q \\
  match_{\dagger}(\quotep{P},\quotep{Q}) & := & \forall R. P|Q \red^{*} R => R \red^{*} 0 \\
  match_{K}(\quotep{P},\quotep{Q}) & := & K \mbox{ for some context } K
\end{eqnarray*}

$u?(x)P | u!\langle Q \rangle \red P\{\quotep{Q}/x\}$

%We write $\wred$ for $\red^*$, and $P\red$ if $\exists Q $ such that $ P \red Q$.
We write $P\red$ if $\exists Q $ such that $ P \red Q$ and $P\not\red$, otherwise.

\section{Replication}

As mentioned before, it is known that replication (and hence
recursion) can be implemented in a higher-order process algebra
\cite{SangiorgiWalker}. As our first example of calculation with the
machinery thus far presented we give the construction explicitly in
the {\rhoc}.

\begin{eqnarray}
	D_{x} & := & \prefix{x}{y}{(\binpar{\outputp{x}{y}}{@{y}})} \nonumber\\
	\bangp_{x}{P} & := & \binpar{{x}!\langle{\binpar{D_{x}}{P}}\rangle}{D_{x}} \nonumber
\end{eqnarray}

\begin{eqnarray}
	\bangp_{x}{P} & & \nonumber\\
	=
	& {x}!\langle{(\prefix{x}{y}{(\outputp{x}{y} | @{y})) | P}}\rangle 
	      | \prefix{x}{y}{(\outputp{x}{y} | @{y})} & \nonumber\\
	\red
	& (\outputp{x}{y} | @{y})\substn{\quotep{(\prefix{x}{y}{(@{y} | \outputp{x}{y})) | P}}}{y} & \nonumber\\
	=
	& \outputp{x}{\quotep{(\prefix{x}{y}{(\outputp{x}{y} | @{y})) | P}}}
	  | {(\prefix{x}{y}{(\outputp{x}{y} | @{y})) | P}} & \nonumber\\
	\red
	& \ldots & \nonumber\\
	\red^*
	& P | P | \ldots & \nonumber
\end{eqnarray}

Of course, this encoding, as an implementation, runs away, unfolding
$\bangp{P}$ eagerly. A lazier and more implementable replication
operator, restricted to input-guarded processes, may be obtained as follows.

\begin{eqnarray}
\bangp{\prefix{u}{v}{P}} 
	:= 
	\binpar{\lift{x}{\prefix{u}{v}{(\binpar{D(x)}{P})}}}{D(x)} \nonumber
\end{eqnarray}

\begin{remark}
  Note that the lazier definition still does not deal with summation
  or mixed summation (i.e. sums over input and output). The reader is
  invited to construct definitions of replication that deal with these
  features. 

  Further, the definitions are parameterized in a name, $x$. Can you,
  gentle reader, make a definition that eliminates this parameter and
  guarantees no accidental interaction between the replication
  machinery and the process being replicated -- i.e. no accidental
  sharing of names used by the process to get its work done and the
  name(s) used by the replication to effect copying. This latter
  revision of the definition of replication is crucial to obtaining
  the expected identity $!!P \sim !P$.
\end{remark}

\begin{remark}\label{rem:paradoxical_combinator}
  The reader familiar with the lambda calculus will have noticed the
  similarity between $D$ and the paradoxical combinator.

  [Ed. note: the existence of this seems to suggest we have to be more
  restrictive on the set of processes and names we admit if we are to
  support no-cloning.]
\end{remark}

\subsubsection{Bisimulation}

The computational dynamics gives rise to another kind of equivalence,
the equivalence of computational behavior. As previously mentioned
this is typically captured \emph{via} some form of bisimulation.

% The notion we use in this paper is weak barbed bisimulation
% \cite{milner91polyadicpi}.

The notion we use in this paper is derived from weak barbed
bisimulation \cite{milner91polyadicpi}. 

\begin{definition}
An \emph{observation relation}, $\downarrow_{\mathcal N}$, over a set
of names, $\mathcal N$, is the smallest relation satisfying the rules
below.

\infrule[Out-barb]{y \in {\mathcal N}, \; x \nameeq y}
		  {\outputp{x}{v} \downarrow_{\mathcal N} x}
\infrule[Par-barb]{\mbox{$P\downarrow_{\mathcal N} x$ or $Q\downarrow_{\mathcal N} x$}}
		  {\binpar{P}{Q} \downarrow_{\mathcal N} x}

We write $P \Downarrow_{\mathcal N} x$ if there is $Q$ such that 
$P \wred Q$ and $Q \downarrow_{\mathcal N} x$.
\end{definition}

\begin{definition}
%\label{def.bbisim}
An  ${\mathcal N}$-\emph{barbed bisimulation} over a set of names, ${\mathcal N}$, is a symmetric binary relation 
${\mathcal S}_{\mathcal N}$ between agents such that $P\rel{S}_{\mathcal N}Q$ implies:
\begin{enumerate}
\item If $P \red P'$ then $Q \wred Q'$ and $P'\rel{S}_{\mathcal N} Q'$.
\item If $P\downarrow_{\mathcal N} x$, then $Q\Downarrow_{\mathcal N} x$.
\end{enumerate}
$P$ is ${\mathcal N}$-barbed bisimilar to $Q$, written
$P \wbbisim_{\mathcal N} Q$, if $P \rel{S}_{\mathcal N} Q$ for some ${\mathcal N}$-barbed bisimulation ${\mathcal S}_{\mathcal N}$.
\end{definition}

$\mathcal{R} \subseteq \pi \times \pi$

$P \mathcal{R} Q => \forall P'. P \red P' \Rightarrow \exists Q'. Q \red Q', P' \mathcal{R} Q'$

$P \vdash x \Rightarrow Q \vdash x$

\begin{mathpar}
  \inferrule*[lab=Out-barb]{x \nameeq y}{{y}!\langle{Q}\rangle \vdash x}
  \and
  \inferrule*[lab=Par-barb]{\mbox{$P\vdash x$ or $Q\vdash x$}}{\binpar{P}{Q} \vdash x}
\end{mathpar}

\subsubsection{Contexts}

One of the principle advantages of computational calculi like the
$\pi$-calculus is a well-defined notion of context,
contextual-equivalence and a correlation between
contextual-equivalence and notions of bisimulation. The notion of
context allows the decomposition of a process into (sub-)process and
its syntactic environment, its context. Thus, a context may be
thought of as a process with a ``hole'' (written $\Box$) in it. The
application of a context $M$ to a process $P$, written $M[P]$, is
tantamount to filling the hole in $M$ with $P$. In this paper we do
not need the full weight of this theory, but do make use of the notion
of context in the proof the main theorem. 

\begin{mathpar}
  \inferrule* [lab=summation] {} {{M_{M},M_{N}} \bc \Box \;|\; x.M_{A} \;|\; M_{M}+M_{N}}
  \and
  \inferrule* [lab=agent] {} {{M_{A}} \bc (\vec{x})M_{P} \;| \; \clift{P_0,\ldots,M_{P},\ldots,P_N}}
  \and \\
  \inferrule* [lab=process] {} {{M_{P}} \bc M_{N} \;| \;P|M_{P} }
\end{mathpar} 

\begin{mathpar}
  \inferrule* [lab=sychronization] {} {M_{N} \bc \Box \;|\; x?M_{F} \;|\; x!M_{C}}
  \and
  \inferrule* [lab=abstraction] {} {{M_{F}} \bc (x)M_{P} }
  \and
  \inferrule* [lab=concretion] {} {{M_{C}} \bc \langle M_{P} \rangle }
  \and \\
  \inferrule* [lab=process] {} {{M_{P}} \bc M_{N} \;| \;P|M_{P} }
\end{mathpar}

\begin{definition}[contextual application] Given a context $M$, and
  process $P$, we define the \emph{contextual application}, $M[P] :=
  M\{P/\Box\}$. That is, the contextual application of M to P is the
  substitution of $P$ for $\Box$ in $M$.
\end{definition}

$\meaningof{-} : L \to \mathcal{P}(\pi)$

\begin{mathpar}
  \inferrule* [lab=collection] {} {\meaningof{true} = \pi, \and \meaningof{~E} = \pi \setminus \meaningof{E}, \and \meaningof{E_{1} \& E_{2}} = \meaningof{E_{1}} \cap \meaningof{E_{2}}}
\end{mathpar}

\begin{mathpar}
  \inferrule* [lab=structure] {} {\meaningof{0} = \{ P \in \pi | P \equiv 0 \}, \and \\ \meaningof{E_1 | E_2} = \{ P \in \pi | P \equiv P_{1} | P_{2}, P_{1} \in \meaningof{E_{1}}, P_{2} \in \meaningof{E_2}\} }
\end{mathpar}

\begin{mathpar}
 \inferrule* [lab=behavior] {} {\meaningof{\langle a?b \rangle E} = \{ P \in \pi | P \equiv Q | u?(y)P', \\ \and \\\\ \and \\ \;\;\; u \in \meaningof{a}, \forall z.P'\{z/y\} \in \meaningof{E\{z/b\}}\}, \and \\ \meaningof{a!E} = \{ P \in \pi | P \equiv Q | x!\langle P' \rangle, x \in \meaningof{a} P' \in \meaningof{E}\} }
\end{mathpar}

\begin{mathpar}
 \inferrule* [lab=nominal] {} {\meaningof{\quotep{E}} = \{ \quotep{P} \in \quotep{\pi} | P \in \meaningof{E} \}, \and \meaningof{\quotep{P}} = \{ \quotep{Q} \in \quotep{\pi} | P \equiv Q \} \and \\ \meaningof{@\quotep{E}} = \{ P \in \pi | P \equiv @x, x \in \meaningof{E} \}}
\end{mathpar}

\begin{eqnarray*}
  \\
  \meaningof{-} : TS \to ST
\end{eqnarray*}

\begin{eqnarray*}
  \\
  L : TS \to ST
\end{eqnarray*}

\begin{eqnarray*}
  \\
  P \models E \iff P \in \meaningof{E}
\end{eqnarray*}

\begin{eqnarray*}
  P \approx_{L} Q \iff \forall E \in L. P \models E \iff Q \models E
\end{eqnarray*}

\begin{eqnarray*}
  P \approx_{K} Q
\end{eqnarray*}

\begin{eqnarray*}
  P \approx Q
\end{eqnarray*}

$\approx_{K} = \approx = \approx_{L}$

\subsubsection{Contextual duality}

Note that contexts extend the quotation operation to a family of
operations from processes to names. Given a context, $M$, we can
define a \emph{nominal context}, $\quotep{M}$ by $\quotep{M}[P] :=
\quotep{M[P]}$. To foreshadow what is to come we observe that these
operations enjoy a duality with processes very much like the duality
between vectors and maps from vectors to scalars.

Further, because the calculus is essentially higher-order, we have a
correspondence between contexts and processes. More specifically,
given a name $x$ and a context $M$ we can construct $M^{*}_{x}$ such
that 

\begin{mathpar}
  M^{*}_{x} | \lift{x}{P} \red M[P]
\end{mathpar}

namely,

\begin{mathpar}
  M^{*}_{x} := x?(u).M[\dropn{u}]
\end{mathpar}

The dependence of $M^{*}_{x}$ on a name makes it an abstraction, 

\begin{mathpar}
  M^{*} := (x)x?(u).M[\dropn{u}]
\end{mathpar}

\subsection{Additional notation}

It will sometimes be convenient to denote the process a name
quotes. We already have the notation $x = \quotep{P}$, but it will be
convenient to introduce an alternate notation, $\procn{x}$, when we
want to emphasize the connection to the use of the name. Note that, by
virtue of name equivalence, $\quotep{\procn{x}} \nameeq x$; so, the
notation is consistent with previous definitions.

Further, because names have structure it is possible to effect
substitutions on the basis of that structure. This means we need to
upgrade our notation for substitutions, which we accomplish by
adapting comprehension notation. Thus,

\begin{mathpar}
  P\{ y / x : x \in S \}
\end{mathpar}

is interpreted to mean the process derived from P by replacing (in a
capture-avoiding manner) each occurrence of $x$ in $S$ by $y$. For example,

\begin{mathpar}
  P\{ \quotep{\procn{x}|\procn{x}} / x : x \in \freenames{P} \}
\end{mathpar}

will replace each (occurrence) of a free name $x$ in $P$ by
$\quotep{\procn{x}|\procn{x}}$.

Also, we will avail ourselves of the notation $x^{L}$ and $x^{R}$ to
denote injections of a name into disjoint copies of the name
space. There are numerous ways to accomplish this. One example can be
found in \cite{MeredithR05}. This notation overloads to vectors of
names: $\vec{x}^{\pi} := (x_{i}^{\pi} \; : \; 0 \leq i < |\vec{x}| )$ where $\pi \in \{L,R\}$.

We also use $P^{\Box} := P|\Box$.

In \cite{MeredithR05} an interpretation of the new operator is
given. It turns out that there are several possible interpretations
all enjoying the requisite algebraic properties of the operator (see
\cite{milner91polyadicpi}). We will therefore make liberal use of
$(\nu\; \vec{x})P$.

% subsection the_syntax_and_semantics_of_the_notation_system (end)   

\input{qm2pi.qmops} 

\input{qm2pi.sterngerlach} 

\input{qm2pi.metric} 

% section concurrent_process_calculi (end)

%\input{qm2pi.proofsketch}

% section proof sketch (end)

%\input{qm2pi.slviaknots} 

% section spatial logic via knots (end)

\input{qm2pi.conclusion}

% section conclusion (end)

%\input{qm2pi.dtcodes} 

% section wiring algorithm (end)

\input{qm2pi.ack} 

% section acknowledgments (end)

\newpage


\bibliographystyle{plain}   
\bibliography{../../biblios/main.bib}

\input{qm2pi.rhodetails}

\end{document}

 

% section wiring algorithm (end)

\documentclass[12pt]{llncs}
%\documentclass{jktr}

\usepackage[pdftex]{hyperref}                   
\usepackage {listings}
\usepackage {mathpartir}
\usepackage{bcprules}
%\usepackage{listings}
                       
\usepackage{graphicx} 
%\usepackage[margins=2.5cm,nohead,nofoot]{geometry}
%\usepackage{geometry}
\usepackage{amsfonts}
\usepackage{amstext}
\usepackage{latexsym}
\usepackage{amssymb}
\usepackage{color}


%\include{myPreamble}
\include{qm2pi.local} 

%\ifpdf
%\usepackage[pdftex]{graphicx}
%\else
%\usepackage{graphicx}
%\fi

 % \ifpdf
%  \usepackage{pdfsync}
%  \if


%\title{Brief Article}
%\author{David F. Snyder}
%\author{L.G. Meredith}

%\address{Dept. of Math., Texas State University--San Marcos, San Marcos, TX 78666}
       
\pagestyle{empty}


\begin{document}

\lstset{language=[Objective]Caml,frame=shadowbox}

\input{qm2pi.front}

% section front matter (end)

\input{qm2pi.intro} 
 
% section introduction (end)

% \input{qm2pi.knotations} 

% section notation (end)

\input{qm2pi.process.calculi} 

% section concurrent_process_calculi_and_spatial_logics_ (end)
    
%\input{qm2pi.knots2pi} 

%\input{qm2pi.trefoil} 

%\input{qm2pi.mainthm} 

% subsection basic_interpretation (end)

%\input{qm2pi.rho.presentation} 
\subsection{The syntax and semantics of the notation system}\label{sub:the_syntax_and_semantics_of_the_notation_system} % (fold)

We now summarize a technical presentation of the calculus that
embodies our theory of dynamics. The typical presentation of such a
calculus follows the style of giving generators and relations on
them. The grammar, below, describing term constructors, freely
generates the set of processes, $\Proc$. This set is then quotiented
by a relation known as structural congruence and it is over this set
that the notion of dynamics is expressed. This presentation is
essentially that of \cite{MeredithR05} with the addition of
polyadicity and summation. For readability we have relegated some of
the technical subtleties to an appendix.

\subsubsection{Process grammar}\label{subsub:process_grammar}

\begin{mathpar}
  \inferrule* [lab=synchronization] {} {{M} \bc \pzero \;|\; x?F \;|\; x!C }
  \and
  \inferrule* [lab=abstraction] {} {{F} \bc (x)P}
  \and
  \inferrule* [lab=concretion] {} {{C} \bc \langle Q \rangle}
  \and
  \inferrule* [lab=process] {} {{P,Q} \bc M \;| \;P|Q \;|\; @{x}}
  \and
  \inferrule* [lab=name] {} {{x} \bc \quotep{P}}
\end{mathpar} 

Note that $\vec{x}$ (resp. $\vec{P}$) denotes a vector of names
(resp. processes) of length $|\vec{x}|$ (resp. $|\vec{P}|$). We adopt
the following useful abbreviations.

\begin{mathpar}
   x?(\vec{y}).P := x.(\vec{y})P \and  x\clift{\vec{P}} := x.\clift{\vec{P}}
   \and x!(y) := \lift{x}{\dropn{y}}
   \and \Pi_{i=0}^{n-1}P_i := P_0 | \ldots | P_{n-1}
\end{mathpar}

\subsubsection{Structural congruence}

\paragraph{Free and bound names and alpha-equivalence.} At the
core of structural equivalence is alpha-equivalence which identifies
process that are the same up to a change of variable. Formally, we
recognize the distinction between free and bound names. The free names
of a process, $\freenames{P}$, may be calculated recursively as
follows:

\begin{mathpar}
\freenames{\pzero} := \emptyset
  \and \\
  \freenames{x?(y).P} := \{ x \} \cup (\freenames{P} \setminus \{ y \})
  \and 
  \freenames{x!\langle P \rangle} := \{ x \} \cup \{ P \} 
  \and \\
  \freenames{P|Q} := \freenames{P} \cup \freenames{Q}
  \and \\
  \freenames{@{x}} := \{ x \}
\end{mathpar}

$\pi$
$\quotep{\pi}$

$\freenames{-} : \pi \to \mathcal{P}(\quotep{\pi})$

\begin{eqnarray*}
  \freenames{\pzero} & := & \emptyset \\
  \freenames{x?(y).P} & := & \{ x \} \cup (\freenames{P} \setminus \{ y \}) \\
  \freenames{x!\langle P \rangle} & := & \{ x \} \cup \{ P \} \\
  \freenames{P|Q} & := & \freenames{P} \cup \freenames{Q} \\
  \freenames{\dropn{x}} & := & \{ x \}
\end{eqnarray*}

The bound names of a process, $\boundnames{P}$, are those names occurring in $P$
that are not free. For example, in $x?(y).0$, the name $x$ is free, while $y$ is bound.

\begin{mathpar}
  \inferrule* [lab=monoidal-laws] {} { P|Q \equiv Q|P \and P|0 \equiv P \and P|(Q|R) \equiv (P|Q)|R }
\end{mathpar}

\begin{mathpar}
  \inferrule* [lab=alpha-equivalence] {} { (x)P \equiv (y)P\{y/x\} \and y \not\in \freenames{P} }
\end{mathpar}

\begin{definition}
Then two processes, $P,Q$, are alpha-equivalent if $P = Q\{\vec{y}/\vec{x}\}$ for
some $\vec{x} \in \boundnames{Q},\vec{y} \in \boundnames{P}$, where $Q\{\vec{y}/\vec{x}\}$
denotes the capture-avoiding substitution of $\vec{y}$ for $\vec{x}$ in $Q$.
\end{definition}

\begin{definition}
  The {\em structural congruence} \cite{SangiorgiWalker} , $\equiv$,
  between processes is the least congruence containing
  alpha-equivalence, satisfying the abelian monoid laws
  (associativity, commutativity and $\pzero$ as identity) for parallel
  composition $|$ and for summation $+$.
\end{definition}

\subsection{Name equivalence}

We take name equivalence, written $\nameeq$, to be the smallest
equivalence relation generated by the following rules.

\begin{mathpar}
\inferrule*[lab=Quote-drop]
{ }
{ \quotep{@{x}} \nameeq x }

\inferrule*[lab=Struct-equiv]
{ P \scong Q }
{ \quotep{P} \nameeq \quotep{Q} }
\end{mathpar}

The astute reader will have noticed that the mutual recursion of names
and processes imposes a mutual recursion on alpha-equivalence and
structural equivalence via name-equivalence. Fortunately, all of this
works out pleasantly and we may calculate in the natural way, free of
concern. The reader interested in the details is referred to the
appendix \ref{appendix:rho_details}.

\subsection{Substitution}

We use $\Proc$ for the set of processes, $\QProc$ for the set of
names, and $\id{\{}\vec{y} / \vec{x} \id{\}}$ to denote partial maps,
$s : \QProc \rightarrow \QProc$. A map, $s$ lifts, uniquely, to a map
on process terms, $\widehat{s} : \Proc \rightarrow \Proc$ by the
following equations.

\begin{mathpar}
  (0) \psubstp{Q}{P} := 0 \\
  (R \juxtap S) \psubstp{Q}{P}
  :=    
  (R)\psubstp{Q}{P} \juxtap (S) \psubstp{Q}{P} \\
  (x?(y).R) \psubstp{Q}{P}    
  :=    
  (x)\substp{Q}{P} (z)\concat( (R \psubstn{z}{y}) \psubstp{Q}{P} ) \\
  (\lift{x}{R}) \psubstp{Q}{P}  
  :=
  \lift{(x)\substp{Q}{P}}{ R \psubstp{Q}{P} } \\
%   (\dropn{x})  \psubstp{Q}{P}       
%   := 
%   \left\{ 
%     \begin{array}{ccc} 
%       \dropn{\quotep{Q}} & & x \nameeq \quotep{P} \\
%       \dropn{x} & & otherwise \\
%     \end{array}
%   \right. 
  (\dropn{x})  \psubstp{Q}{P}       
  := 
  \left\{ 
    \begin{array}{ccc} 
      Q & & x \nameeq \quotep{P} \\
      \dropn{x} & & otherwise \\
    \end{array}
  \right.
\end{mathpar}
 

where

\begin{eqnarray}
  (x)\id{\{} \lpquote Q \rpquote / \lpquote P \rpquote \id{\}}            = 
  \left\{ 
    \begin{array}{ccc}
      \lpquote Q \rpquote & & x \nameeq \lpquote P \rpquote \\
      x & & otherwise \\
    \end{array}
  \right. \nonumber
\end{eqnarray}

and $z$ is chosen distinct from $\quotep{P}$, $\quotep{Q}$, the free
names in $Q$, and all the names in $R$. Our $\alpha$-equivalence will
be built in the standard way from this substitution.

\begin{remark}\label{rem:no_self_referential_names}
  One consequence of these definitions is that $\forall P. \quotep{P}
  \not\in \freenames{P}$.
\end{remark}

\subsection{ Dynamic quote: an example }

Anticipating something of what's to come, consider applying the
substitution, $\widehat{\id{\{}u / z \id{\}}}$, to the following pair
of processes, $\lift{w}{y!(z)}$ and $w[ \lpquote y!(z) \rpquote ]$.

\begin{eqnarray}
	\lift{w}{y!(z)}\widehat{\id{\{}u / z \id{\}}}
		& = &
		\lift{w}{y!(u)} \nonumber\\
	w[ \lpquote y!(z) \rpquote ] \widehat{ \id{\{}u / z \id{\}} }
		& = &
		w[ \lpquote y!(z) \rpquote ] \nonumber
\end{eqnarray}

Because the body of the process between quotes is impervious to
substitution, we get radically different answers. In fact, by
examining the first process in an input context,
e.g. $x?(z).\lift{w}{y!(z)}$, we see that the process under the lift
operator may be shaped by prefixed inputs binding a name inside it. In
this sense, the lift operator will be seen as a way to dynamically
construct processes before reifying them as names.

Finally equipped with these standard features we can present the
dynamics of the calculus.

\subsubsection{Operational semantics} 

Finally, we introduce the computational dynamics. What marks these
algebras as distinct from other more traditionally studied algebraic
structures, e.g. vector spaces or polynomial rings, is the manner in
which dynamics is captured. In traditional structures, dynamics is typically
expressed through morphisms between such structures, as in linear maps
between vector spaces or morphisms between rings. In algebras
associated with the semantics of computation, the dynamics is
expressed as part of the algebraic structure itself, through a
reduction reduction relation typically denoted by $\red$. Below, we
give a recursive presentation of this relation for the calculus used
in the encoding.

$\red \subseteq \pi \times \pi$
$\red : \pi \to \mathcal{P}(\pi)$

\begin{mathpar}
  \inferrule* [lab=Comm] { \textsf{match}( x_{src}, x_{trgt} ) } { x_{trgt}?(y)P \; | \; x_{src}!\langle {Q} \rangle \red P\{\quotep{Q}/y}\} }
  \and \\
  \inferrule* [lab=Par] {{P} \red {P}'} {{{P} | {Q}} \red {{P}' | {Q}}}
  \and
  \inferrule* [lab=Equiv]{{{P} \scong {P}'} \andalso {{P}' \red {Q}'} \andalso {{Q}' \scong {Q}}}{{P} \red {Q}}
\end{mathpar}

\begin{eqnarray*}
  match_{\equiv} (\quotep{P},\quotep{Q}) & := & P \equiv Q \\
  match_{\dagger}(\quotep{P},\quotep{Q}) & := & \forall R. P|Q \red^{*} R => R \red^{*} 0 \\
  match_{K}(\quotep{P},\quotep{Q}) & := & K \mbox{ for some context } K
\end{eqnarray*}

$u?(x)P | u!\langle Q \rangle \red P\{\quotep{Q}/x\}$

%We write $\wred$ for $\red^*$, and $P\red$ if $\exists Q $ such that $ P \red Q$.
We write $P\red$ if $\exists Q $ such that $ P \red Q$ and $P\not\red$, otherwise.

\section{Replication}

As mentioned before, it is known that replication (and hence
recursion) can be implemented in a higher-order process algebra
\cite{SangiorgiWalker}. As our first example of calculation with the
machinery thus far presented we give the construction explicitly in
the {\rhoc}.

\begin{eqnarray}
	D_{x} & := & \prefix{x}{y}{(\binpar{\outputp{x}{y}}{@{y}})} \nonumber\\
	\bangp_{x}{P} & := & \binpar{{x}!\langle{\binpar{D_{x}}{P}}\rangle}{D_{x}} \nonumber
\end{eqnarray}

\begin{eqnarray}
	\bangp_{x}{P} & & \nonumber\\
	=
	& {x}!\langle{(\prefix{x}{y}{(\outputp{x}{y} | @{y})) | P}}\rangle 
	      | \prefix{x}{y}{(\outputp{x}{y} | @{y})} & \nonumber\\
	\red
	& (\outputp{x}{y} | @{y})\substn{\quotep{(\prefix{x}{y}{(@{y} | \outputp{x}{y})) | P}}}{y} & \nonumber\\
	=
	& \outputp{x}{\quotep{(\prefix{x}{y}{(\outputp{x}{y} | @{y})) | P}}}
	  | {(\prefix{x}{y}{(\outputp{x}{y} | @{y})) | P}} & \nonumber\\
	\red
	& \ldots & \nonumber\\
	\red^*
	& P | P | \ldots & \nonumber
\end{eqnarray}

Of course, this encoding, as an implementation, runs away, unfolding
$\bangp{P}$ eagerly. A lazier and more implementable replication
operator, restricted to input-guarded processes, may be obtained as follows.

\begin{eqnarray}
\bangp{\prefix{u}{v}{P}} 
	:= 
	\binpar{\lift{x}{\prefix{u}{v}{(\binpar{D(x)}{P})}}}{D(x)} \nonumber
\end{eqnarray}

\begin{remark}
  Note that the lazier definition still does not deal with summation
  or mixed summation (i.e. sums over input and output). The reader is
  invited to construct definitions of replication that deal with these
  features. 

  Further, the definitions are parameterized in a name, $x$. Can you,
  gentle reader, make a definition that eliminates this parameter and
  guarantees no accidental interaction between the replication
  machinery and the process being replicated -- i.e. no accidental
  sharing of names used by the process to get its work done and the
  name(s) used by the replication to effect copying. This latter
  revision of the definition of replication is crucial to obtaining
  the expected identity $!!P \sim !P$.
\end{remark}

\begin{remark}\label{rem:paradoxical_combinator}
  The reader familiar with the lambda calculus will have noticed the
  similarity between $D$ and the paradoxical combinator.

  [Ed. note: the existence of this seems to suggest we have to be more
  restrictive on the set of processes and names we admit if we are to
  support no-cloning.]
\end{remark}

\subsubsection{Bisimulation}

The computational dynamics gives rise to another kind of equivalence,
the equivalence of computational behavior. As previously mentioned
this is typically captured \emph{via} some form of bisimulation.

% The notion we use in this paper is weak barbed bisimulation
% \cite{milner91polyadicpi}.

The notion we use in this paper is derived from weak barbed
bisimulation \cite{milner91polyadicpi}. 

\begin{definition}
An \emph{observation relation}, $\downarrow_{\mathcal N}$, over a set
of names, $\mathcal N$, is the smallest relation satisfying the rules
below.

\infrule[Out-barb]{y \in {\mathcal N}, \; x \nameeq y}
		  {\outputp{x}{v} \downarrow_{\mathcal N} x}
\infrule[Par-barb]{\mbox{$P\downarrow_{\mathcal N} x$ or $Q\downarrow_{\mathcal N} x$}}
		  {\binpar{P}{Q} \downarrow_{\mathcal N} x}

We write $P \Downarrow_{\mathcal N} x$ if there is $Q$ such that 
$P \wred Q$ and $Q \downarrow_{\mathcal N} x$.
\end{definition}

\begin{definition}
%\label{def.bbisim}
An  ${\mathcal N}$-\emph{barbed bisimulation} over a set of names, ${\mathcal N}$, is a symmetric binary relation 
${\mathcal S}_{\mathcal N}$ between agents such that $P\rel{S}_{\mathcal N}Q$ implies:
\begin{enumerate}
\item If $P \red P'$ then $Q \wred Q'$ and $P'\rel{S}_{\mathcal N} Q'$.
\item If $P\downarrow_{\mathcal N} x$, then $Q\Downarrow_{\mathcal N} x$.
\end{enumerate}
$P$ is ${\mathcal N}$-barbed bisimilar to $Q$, written
$P \wbbisim_{\mathcal N} Q$, if $P \rel{S}_{\mathcal N} Q$ for some ${\mathcal N}$-barbed bisimulation ${\mathcal S}_{\mathcal N}$.
\end{definition}

$\mathcal{R} \subseteq \pi \times \pi$

$P \mathcal{R} Q => \forall P'. P \red P' \Rightarrow \exists Q'. Q \red Q', P' \mathcal{R} Q'$

$P \vdash x \Rightarrow Q \vdash x$

\begin{mathpar}
  \inferrule*[lab=Out-barb]{x \nameeq y}{{y}!\langle{Q}\rangle \vdash x}
  \and
  \inferrule*[lab=Par-barb]{\mbox{$P\vdash x$ or $Q\vdash x$}}{\binpar{P}{Q} \vdash x}
\end{mathpar}

\subsubsection{Contexts}

One of the principle advantages of computational calculi like the
$\pi$-calculus is a well-defined notion of context,
contextual-equivalence and a correlation between
contextual-equivalence and notions of bisimulation. The notion of
context allows the decomposition of a process into (sub-)process and
its syntactic environment, its context. Thus, a context may be
thought of as a process with a ``hole'' (written $\Box$) in it. The
application of a context $M$ to a process $P$, written $M[P]$, is
tantamount to filling the hole in $M$ with $P$. In this paper we do
not need the full weight of this theory, but do make use of the notion
of context in the proof the main theorem. 

\begin{mathpar}
  \inferrule* [lab=summation] {} {{M_{M},M_{N}} \bc \Box \;|\; x.M_{A} \;|\; M_{M}+M_{N}}
  \and
  \inferrule* [lab=agent] {} {{M_{A}} \bc (\vec{x})M_{P} \;| \; \clift{P_0,\ldots,M_{P},\ldots,P_N}}
  \and \\
  \inferrule* [lab=process] {} {{M_{P}} \bc M_{N} \;| \;P|M_{P} }
\end{mathpar} 

\begin{mathpar}
  \inferrule* [lab=sychronization] {} {M_{N} \bc \Box \;|\; x?M_{F} \;|\; x!M_{C}}
  \and
  \inferrule* [lab=abstraction] {} {{M_{F}} \bc (x)M_{P} }
  \and
  \inferrule* [lab=concretion] {} {{M_{C}} \bc \langle M_{P} \rangle }
  \and \\
  \inferrule* [lab=process] {} {{M_{P}} \bc M_{N} \;| \;P|M_{P} }
\end{mathpar}

\begin{definition}[contextual application] Given a context $M$, and
  process $P$, we define the \emph{contextual application}, $M[P] :=
  M\{P/\Box\}$. That is, the contextual application of M to P is the
  substitution of $P$ for $\Box$ in $M$.
\end{definition}

$\meaningof{-} : L \to \mathcal{P}(\pi)$

\begin{mathpar}
  \inferrule* [lab=collection] {} {\meaningof{true} = \pi, \and \meaningof{~E} = \pi \setminus \meaningof{E}, \and \meaningof{E_{1} \& E_{2}} = \meaningof{E_{1}} \cap \meaningof{E_{2}}}
\end{mathpar}

\begin{mathpar}
  \inferrule* [lab=structure] {} {\meaningof{0} = \{ P \in \pi | P \equiv 0 \}, \and \\ \meaningof{E_1 | E_2} = \{ P \in \pi | P \equiv P_{1} | P_{2}, P_{1} \in \meaningof{E_{1}}, P_{2} \in \meaningof{E_2}\} }
\end{mathpar}

\begin{mathpar}
 \inferrule* [lab=behavior] {} {\meaningof{\langle a?b \rangle E} = \{ P \in \pi | P \equiv Q | u?(y)P', \\ \and \\\\ \and \\ \;\;\; u \in \meaningof{a}, \forall z.P'\{z/y\} \in \meaningof{E\{z/b\}}\}, \and \\ \meaningof{a!E} = \{ P \in \pi | P \equiv Q | x!\langle P' \rangle, x \in \meaningof{a} P' \in \meaningof{E}\} }
\end{mathpar}

\begin{mathpar}
 \inferrule* [lab=nominal] {} {\meaningof{\quotep{E}} = \{ \quotep{P} \in \quotep{\pi} | P \in \meaningof{E} \}, \and \meaningof{\quotep{P}} = \{ \quotep{Q} \in \quotep{\pi} | P \equiv Q \} \and \\ \meaningof{@\quotep{E}} = \{ P \in \pi | P \equiv @x, x \in \meaningof{E} \}}
\end{mathpar}

\begin{eqnarray*}
  \\
  \meaningof{-} : TS \to ST
\end{eqnarray*}

\begin{eqnarray*}
  \\
  L : TS \to ST
\end{eqnarray*}

\begin{eqnarray*}
  \\
  P \models E \iff P \in \meaningof{E}
\end{eqnarray*}

\begin{eqnarray*}
  P \approx_{L} Q \iff \forall E \in L. P \models E \iff Q \models E
\end{eqnarray*}

\begin{eqnarray*}
  P \approx_{K} Q
\end{eqnarray*}

\begin{eqnarray*}
  P \approx Q
\end{eqnarray*}

$\approx_{K} = \approx = \approx_{L}$

\subsubsection{Contextual duality}

Note that contexts extend the quotation operation to a family of
operations from processes to names. Given a context, $M$, we can
define a \emph{nominal context}, $\quotep{M}$ by $\quotep{M}[P] :=
\quotep{M[P]}$. To foreshadow what is to come we observe that these
operations enjoy a duality with processes very much like the duality
between vectors and maps from vectors to scalars.

Further, because the calculus is essentially higher-order, we have a
correspondence between contexts and processes. More specifically,
given a name $x$ and a context $M$ we can construct $M^{*}_{x}$ such
that 

\begin{mathpar}
  M^{*}_{x} | \lift{x}{P} \red M[P]
\end{mathpar}

namely,

\begin{mathpar}
  M^{*}_{x} := x?(u).M[\dropn{u}]
\end{mathpar}

The dependence of $M^{*}_{x}$ on a name makes it an abstraction, 

\begin{mathpar}
  M^{*} := (x)x?(u).M[\dropn{u}]
\end{mathpar}

\subsection{Additional notation}

It will sometimes be convenient to denote the process a name
quotes. We already have the notation $x = \quotep{P}$, but it will be
convenient to introduce an alternate notation, $\procn{x}$, when we
want to emphasize the connection to the use of the name. Note that, by
virtue of name equivalence, $\quotep{\procn{x}} \nameeq x$; so, the
notation is consistent with previous definitions.

Further, because names have structure it is possible to effect
substitutions on the basis of that structure. This means we need to
upgrade our notation for substitutions, which we accomplish by
adapting comprehension notation. Thus,

\begin{mathpar}
  P\{ y / x : x \in S \}
\end{mathpar}

is interpreted to mean the process derived from P by replacing (in a
capture-avoiding manner) each occurrence of $x$ in $S$ by $y$. For example,

\begin{mathpar}
  P\{ \quotep{\procn{x}|\procn{x}} / x : x \in \freenames{P} \}
\end{mathpar}

will replace each (occurrence) of a free name $x$ in $P$ by
$\quotep{\procn{x}|\procn{x}}$.

Also, we will avail ourselves of the notation $x^{L}$ and $x^{R}$ to
denote injections of a name into disjoint copies of the name
space. There are numerous ways to accomplish this. One example can be
found in \cite{MeredithR05}. This notation overloads to vectors of
names: $\vec{x}^{\pi} := (x_{i}^{\pi} \; : \; 0 \leq i < |\vec{x}| )$ where $\pi \in \{L,R\}$.

We also use $P^{\Box} := P|\Box$.

In \cite{MeredithR05} an interpretation of the new operator is
given. It turns out that there are several possible interpretations
all enjoying the requisite algebraic properties of the operator (see
\cite{milner91polyadicpi}). We will therefore make liberal use of
$(\nu\; \vec{x})P$.

% subsection the_syntax_and_semantics_of_the_notation_system (end)   

\input{qm2pi.qmops} 

\input{qm2pi.sterngerlach} 

\input{qm2pi.metric} 

% section concurrent_process_calculi (end)

%\input{qm2pi.proofsketch}

% section proof sketch (end)

%\input{qm2pi.slviaknots} 

% section spatial logic via knots (end)

\input{qm2pi.conclusion}

% section conclusion (end)

%\input{qm2pi.dtcodes} 

% section wiring algorithm (end)

\input{qm2pi.ack} 

% section acknowledgments (end)

\newpage


\bibliographystyle{plain}   
\bibliography{../../biblios/main.bib}

\input{qm2pi.rhodetails}

\end{document}

 

% section acknowledgments (end)

\newpage


\bibliographystyle{plain}   
\bibliography{../../biblios/main.bib}

\documentclass[12pt]{llncs}
%\documentclass{jktr}

\usepackage[pdftex]{hyperref}                   
\usepackage {listings}
\usepackage {mathpartir}
\usepackage{bcprules}
%\usepackage{listings}
                       
\usepackage{graphicx} 
%\usepackage[margins=2.5cm,nohead,nofoot]{geometry}
%\usepackage{geometry}
\usepackage{amsfonts}
\usepackage{amstext}
\usepackage{latexsym}
\usepackage{amssymb}
\usepackage{color}


%\include{myPreamble}
\include{qm2pi.local} 

%\ifpdf
%\usepackage[pdftex]{graphicx}
%\else
%\usepackage{graphicx}
%\fi

 % \ifpdf
%  \usepackage{pdfsync}
%  \if


%\title{Brief Article}
%\author{David F. Snyder}
%\author{L.G. Meredith}

%\address{Dept. of Math., Texas State University--San Marcos, San Marcos, TX 78666}
       
\pagestyle{empty}


\begin{document}

\lstset{language=[Objective]Caml,frame=shadowbox}

\input{qm2pi.front}

% section front matter (end)

\input{qm2pi.intro} 
 
% section introduction (end)

% \input{qm2pi.knotations} 

% section notation (end)

\input{qm2pi.process.calculi} 

% section concurrent_process_calculi_and_spatial_logics_ (end)
    
%\input{qm2pi.knots2pi} 

%\input{qm2pi.trefoil} 

%\input{qm2pi.mainthm} 

% subsection basic_interpretation (end)

%\input{qm2pi.rho.presentation} 
\subsection{The syntax and semantics of the notation system}\label{sub:the_syntax_and_semantics_of_the_notation_system} % (fold)

We now summarize a technical presentation of the calculus that
embodies our theory of dynamics. The typical presentation of such a
calculus follows the style of giving generators and relations on
them. The grammar, below, describing term constructors, freely
generates the set of processes, $\Proc$. This set is then quotiented
by a relation known as structural congruence and it is over this set
that the notion of dynamics is expressed. This presentation is
essentially that of \cite{MeredithR05} with the addition of
polyadicity and summation. For readability we have relegated some of
the technical subtleties to an appendix.

\subsubsection{Process grammar}\label{subsub:process_grammar}

\begin{mathpar}
  \inferrule* [lab=synchronization] {} {{M} \bc \pzero \;|\; x?F \;|\; x!C }
  \and
  \inferrule* [lab=abstraction] {} {{F} \bc (x)P}
  \and
  \inferrule* [lab=concretion] {} {{C} \bc \langle Q \rangle}
  \and
  \inferrule* [lab=process] {} {{P,Q} \bc M \;| \;P|Q \;|\; @{x}}
  \and
  \inferrule* [lab=name] {} {{x} \bc \quotep{P}}
\end{mathpar} 

Note that $\vec{x}$ (resp. $\vec{P}$) denotes a vector of names
(resp. processes) of length $|\vec{x}|$ (resp. $|\vec{P}|$). We adopt
the following useful abbreviations.

\begin{mathpar}
   x?(\vec{y}).P := x.(\vec{y})P \and  x\clift{\vec{P}} := x.\clift{\vec{P}}
   \and x!(y) := \lift{x}{\dropn{y}}
   \and \Pi_{i=0}^{n-1}P_i := P_0 | \ldots | P_{n-1}
\end{mathpar}

\subsubsection{Structural congruence}

\paragraph{Free and bound names and alpha-equivalence.} At the
core of structural equivalence is alpha-equivalence which identifies
process that are the same up to a change of variable. Formally, we
recognize the distinction between free and bound names. The free names
of a process, $\freenames{P}$, may be calculated recursively as
follows:

\begin{mathpar}
\freenames{\pzero} := \emptyset
  \and \\
  \freenames{x?(y).P} := \{ x \} \cup (\freenames{P} \setminus \{ y \})
  \and 
  \freenames{x!\langle P \rangle} := \{ x \} \cup \{ P \} 
  \and \\
  \freenames{P|Q} := \freenames{P} \cup \freenames{Q}
  \and \\
  \freenames{@{x}} := \{ x \}
\end{mathpar}

$\pi$
$\quotep{\pi}$

$\freenames{-} : \pi \to \mathcal{P}(\quotep{\pi})$

\begin{eqnarray*}
  \freenames{\pzero} & := & \emptyset \\
  \freenames{x?(y).P} & := & \{ x \} \cup (\freenames{P} \setminus \{ y \}) \\
  \freenames{x!\langle P \rangle} & := & \{ x \} \cup \{ P \} \\
  \freenames{P|Q} & := & \freenames{P} \cup \freenames{Q} \\
  \freenames{\dropn{x}} & := & \{ x \}
\end{eqnarray*}

The bound names of a process, $\boundnames{P}$, are those names occurring in $P$
that are not free. For example, in $x?(y).0$, the name $x$ is free, while $y$ is bound.

\begin{mathpar}
  \inferrule* [lab=monoidal-laws] {} { P|Q \equiv Q|P \and P|0 \equiv P \and P|(Q|R) \equiv (P|Q)|R }
\end{mathpar}

\begin{mathpar}
  \inferrule* [lab=alpha-equivalence] {} { (x)P \equiv (y)P\{y/x\} \and y \not\in \freenames{P} }
\end{mathpar}

\begin{definition}
Then two processes, $P,Q$, are alpha-equivalent if $P = Q\{\vec{y}/\vec{x}\}$ for
some $\vec{x} \in \boundnames{Q},\vec{y} \in \boundnames{P}$, where $Q\{\vec{y}/\vec{x}\}$
denotes the capture-avoiding substitution of $\vec{y}$ for $\vec{x}$ in $Q$.
\end{definition}

\begin{definition}
  The {\em structural congruence} \cite{SangiorgiWalker} , $\equiv$,
  between processes is the least congruence containing
  alpha-equivalence, satisfying the abelian monoid laws
  (associativity, commutativity and $\pzero$ as identity) for parallel
  composition $|$ and for summation $+$.
\end{definition}

\subsection{Name equivalence}

We take name equivalence, written $\nameeq$, to be the smallest
equivalence relation generated by the following rules.

\begin{mathpar}
\inferrule*[lab=Quote-drop]
{ }
{ \quotep{@{x}} \nameeq x }

\inferrule*[lab=Struct-equiv]
{ P \scong Q }
{ \quotep{P} \nameeq \quotep{Q} }
\end{mathpar}

The astute reader will have noticed that the mutual recursion of names
and processes imposes a mutual recursion on alpha-equivalence and
structural equivalence via name-equivalence. Fortunately, all of this
works out pleasantly and we may calculate in the natural way, free of
concern. The reader interested in the details is referred to the
appendix \ref{appendix:rho_details}.

\subsection{Substitution}

We use $\Proc$ for the set of processes, $\QProc$ for the set of
names, and $\id{\{}\vec{y} / \vec{x} \id{\}}$ to denote partial maps,
$s : \QProc \rightarrow \QProc$. A map, $s$ lifts, uniquely, to a map
on process terms, $\widehat{s} : \Proc \rightarrow \Proc$ by the
following equations.

\begin{mathpar}
  (0) \psubstp{Q}{P} := 0 \\
  (R \juxtap S) \psubstp{Q}{P}
  :=    
  (R)\psubstp{Q}{P} \juxtap (S) \psubstp{Q}{P} \\
  (x?(y).R) \psubstp{Q}{P}    
  :=    
  (x)\substp{Q}{P} (z)\concat( (R \psubstn{z}{y}) \psubstp{Q}{P} ) \\
  (\lift{x}{R}) \psubstp{Q}{P}  
  :=
  \lift{(x)\substp{Q}{P}}{ R \psubstp{Q}{P} } \\
%   (\dropn{x})  \psubstp{Q}{P}       
%   := 
%   \left\{ 
%     \begin{array}{ccc} 
%       \dropn{\quotep{Q}} & & x \nameeq \quotep{P} \\
%       \dropn{x} & & otherwise \\
%     \end{array}
%   \right. 
  (\dropn{x})  \psubstp{Q}{P}       
  := 
  \left\{ 
    \begin{array}{ccc} 
      Q & & x \nameeq \quotep{P} \\
      \dropn{x} & & otherwise \\
    \end{array}
  \right.
\end{mathpar}
 

where

\begin{eqnarray}
  (x)\id{\{} \lpquote Q \rpquote / \lpquote P \rpquote \id{\}}            = 
  \left\{ 
    \begin{array}{ccc}
      \lpquote Q \rpquote & & x \nameeq \lpquote P \rpquote \\
      x & & otherwise \\
    \end{array}
  \right. \nonumber
\end{eqnarray}

and $z$ is chosen distinct from $\quotep{P}$, $\quotep{Q}$, the free
names in $Q$, and all the names in $R$. Our $\alpha$-equivalence will
be built in the standard way from this substitution.

\begin{remark}\label{rem:no_self_referential_names}
  One consequence of these definitions is that $\forall P. \quotep{P}
  \not\in \freenames{P}$.
\end{remark}

\subsection{ Dynamic quote: an example }

Anticipating something of what's to come, consider applying the
substitution, $\widehat{\id{\{}u / z \id{\}}}$, to the following pair
of processes, $\lift{w}{y!(z)}$ and $w[ \lpquote y!(z) \rpquote ]$.

\begin{eqnarray}
	\lift{w}{y!(z)}\widehat{\id{\{}u / z \id{\}}}
		& = &
		\lift{w}{y!(u)} \nonumber\\
	w[ \lpquote y!(z) \rpquote ] \widehat{ \id{\{}u / z \id{\}} }
		& = &
		w[ \lpquote y!(z) \rpquote ] \nonumber
\end{eqnarray}

Because the body of the process between quotes is impervious to
substitution, we get radically different answers. In fact, by
examining the first process in an input context,
e.g. $x?(z).\lift{w}{y!(z)}$, we see that the process under the lift
operator may be shaped by prefixed inputs binding a name inside it. In
this sense, the lift operator will be seen as a way to dynamically
construct processes before reifying them as names.

Finally equipped with these standard features we can present the
dynamics of the calculus.

\subsubsection{Operational semantics} 

Finally, we introduce the computational dynamics. What marks these
algebras as distinct from other more traditionally studied algebraic
structures, e.g. vector spaces or polynomial rings, is the manner in
which dynamics is captured. In traditional structures, dynamics is typically
expressed through morphisms between such structures, as in linear maps
between vector spaces or morphisms between rings. In algebras
associated with the semantics of computation, the dynamics is
expressed as part of the algebraic structure itself, through a
reduction reduction relation typically denoted by $\red$. Below, we
give a recursive presentation of this relation for the calculus used
in the encoding.

$\red \subseteq \pi \times \pi$
$\red : \pi \to \mathcal{P}(\pi)$

\begin{mathpar}
  \inferrule* [lab=Comm] { \textsf{match}( x_{src}, x_{trgt} ) } { x_{trgt}?(y)P \; | \; x_{src}!\langle {Q} \rangle \red P\{\quotep{Q}/y}\} }
  \and \\
  \inferrule* [lab=Par] {{P} \red {P}'} {{{P} | {Q}} \red {{P}' | {Q}}}
  \and
  \inferrule* [lab=Equiv]{{{P} \scong {P}'} \andalso {{P}' \red {Q}'} \andalso {{Q}' \scong {Q}}}{{P} \red {Q}}
\end{mathpar}

\begin{eqnarray*}
  match_{\equiv} (\quotep{P},\quotep{Q}) & := & P \equiv Q \\
  match_{\dagger}(\quotep{P},\quotep{Q}) & := & \forall R. P|Q \red^{*} R => R \red^{*} 0 \\
  match_{K}(\quotep{P},\quotep{Q}) & := & K \mbox{ for some context } K
\end{eqnarray*}

$u?(x)P | u!\langle Q \rangle \red P\{\quotep{Q}/x\}$

%We write $\wred$ for $\red^*$, and $P\red$ if $\exists Q $ such that $ P \red Q$.
We write $P\red$ if $\exists Q $ such that $ P \red Q$ and $P\not\red$, otherwise.

\section{Replication}

As mentioned before, it is known that replication (and hence
recursion) can be implemented in a higher-order process algebra
\cite{SangiorgiWalker}. As our first example of calculation with the
machinery thus far presented we give the construction explicitly in
the {\rhoc}.

\begin{eqnarray}
	D_{x} & := & \prefix{x}{y}{(\binpar{\outputp{x}{y}}{@{y}})} \nonumber\\
	\bangp_{x}{P} & := & \binpar{{x}!\langle{\binpar{D_{x}}{P}}\rangle}{D_{x}} \nonumber
\end{eqnarray}

\begin{eqnarray}
	\bangp_{x}{P} & & \nonumber\\
	=
	& {x}!\langle{(\prefix{x}{y}{(\outputp{x}{y} | @{y})) | P}}\rangle 
	      | \prefix{x}{y}{(\outputp{x}{y} | @{y})} & \nonumber\\
	\red
	& (\outputp{x}{y} | @{y})\substn{\quotep{(\prefix{x}{y}{(@{y} | \outputp{x}{y})) | P}}}{y} & \nonumber\\
	=
	& \outputp{x}{\quotep{(\prefix{x}{y}{(\outputp{x}{y} | @{y})) | P}}}
	  | {(\prefix{x}{y}{(\outputp{x}{y} | @{y})) | P}} & \nonumber\\
	\red
	& \ldots & \nonumber\\
	\red^*
	& P | P | \ldots & \nonumber
\end{eqnarray}

Of course, this encoding, as an implementation, runs away, unfolding
$\bangp{P}$ eagerly. A lazier and more implementable replication
operator, restricted to input-guarded processes, may be obtained as follows.

\begin{eqnarray}
\bangp{\prefix{u}{v}{P}} 
	:= 
	\binpar{\lift{x}{\prefix{u}{v}{(\binpar{D(x)}{P})}}}{D(x)} \nonumber
\end{eqnarray}

\begin{remark}
  Note that the lazier definition still does not deal with summation
  or mixed summation (i.e. sums over input and output). The reader is
  invited to construct definitions of replication that deal with these
  features. 

  Further, the definitions are parameterized in a name, $x$. Can you,
  gentle reader, make a definition that eliminates this parameter and
  guarantees no accidental interaction between the replication
  machinery and the process being replicated -- i.e. no accidental
  sharing of names used by the process to get its work done and the
  name(s) used by the replication to effect copying. This latter
  revision of the definition of replication is crucial to obtaining
  the expected identity $!!P \sim !P$.
\end{remark}

\begin{remark}\label{rem:paradoxical_combinator}
  The reader familiar with the lambda calculus will have noticed the
  similarity between $D$ and the paradoxical combinator.

  [Ed. note: the existence of this seems to suggest we have to be more
  restrictive on the set of processes and names we admit if we are to
  support no-cloning.]
\end{remark}

\subsubsection{Bisimulation}

The computational dynamics gives rise to another kind of equivalence,
the equivalence of computational behavior. As previously mentioned
this is typically captured \emph{via} some form of bisimulation.

% The notion we use in this paper is weak barbed bisimulation
% \cite{milner91polyadicpi}.

The notion we use in this paper is derived from weak barbed
bisimulation \cite{milner91polyadicpi}. 

\begin{definition}
An \emph{observation relation}, $\downarrow_{\mathcal N}$, over a set
of names, $\mathcal N$, is the smallest relation satisfying the rules
below.

\infrule[Out-barb]{y \in {\mathcal N}, \; x \nameeq y}
		  {\outputp{x}{v} \downarrow_{\mathcal N} x}
\infrule[Par-barb]{\mbox{$P\downarrow_{\mathcal N} x$ or $Q\downarrow_{\mathcal N} x$}}
		  {\binpar{P}{Q} \downarrow_{\mathcal N} x}

We write $P \Downarrow_{\mathcal N} x$ if there is $Q$ such that 
$P \wred Q$ and $Q \downarrow_{\mathcal N} x$.
\end{definition}

\begin{definition}
%\label{def.bbisim}
An  ${\mathcal N}$-\emph{barbed bisimulation} over a set of names, ${\mathcal N}$, is a symmetric binary relation 
${\mathcal S}_{\mathcal N}$ between agents such that $P\rel{S}_{\mathcal N}Q$ implies:
\begin{enumerate}
\item If $P \red P'$ then $Q \wred Q'$ and $P'\rel{S}_{\mathcal N} Q'$.
\item If $P\downarrow_{\mathcal N} x$, then $Q\Downarrow_{\mathcal N} x$.
\end{enumerate}
$P$ is ${\mathcal N}$-barbed bisimilar to $Q$, written
$P \wbbisim_{\mathcal N} Q$, if $P \rel{S}_{\mathcal N} Q$ for some ${\mathcal N}$-barbed bisimulation ${\mathcal S}_{\mathcal N}$.
\end{definition}

$\mathcal{R} \subseteq \pi \times \pi$

$P \mathcal{R} Q => \forall P'. P \red P' \Rightarrow \exists Q'. Q \red Q', P' \mathcal{R} Q'$

$P \vdash x \Rightarrow Q \vdash x$

\begin{mathpar}
  \inferrule*[lab=Out-barb]{x \nameeq y}{{y}!\langle{Q}\rangle \vdash x}
  \and
  \inferrule*[lab=Par-barb]{\mbox{$P\vdash x$ or $Q\vdash x$}}{\binpar{P}{Q} \vdash x}
\end{mathpar}

\subsubsection{Contexts}

One of the principle advantages of computational calculi like the
$\pi$-calculus is a well-defined notion of context,
contextual-equivalence and a correlation between
contextual-equivalence and notions of bisimulation. The notion of
context allows the decomposition of a process into (sub-)process and
its syntactic environment, its context. Thus, a context may be
thought of as a process with a ``hole'' (written $\Box$) in it. The
application of a context $M$ to a process $P$, written $M[P]$, is
tantamount to filling the hole in $M$ with $P$. In this paper we do
not need the full weight of this theory, but do make use of the notion
of context in the proof the main theorem. 

\begin{mathpar}
  \inferrule* [lab=summation] {} {{M_{M},M_{N}} \bc \Box \;|\; x.M_{A} \;|\; M_{M}+M_{N}}
  \and
  \inferrule* [lab=agent] {} {{M_{A}} \bc (\vec{x})M_{P} \;| \; \clift{P_0,\ldots,M_{P},\ldots,P_N}}
  \and \\
  \inferrule* [lab=process] {} {{M_{P}} \bc M_{N} \;| \;P|M_{P} }
\end{mathpar} 

\begin{mathpar}
  \inferrule* [lab=sychronization] {} {M_{N} \bc \Box \;|\; x?M_{F} \;|\; x!M_{C}}
  \and
  \inferrule* [lab=abstraction] {} {{M_{F}} \bc (x)M_{P} }
  \and
  \inferrule* [lab=concretion] {} {{M_{C}} \bc \langle M_{P} \rangle }
  \and \\
  \inferrule* [lab=process] {} {{M_{P}} \bc M_{N} \;| \;P|M_{P} }
\end{mathpar}

\begin{definition}[contextual application] Given a context $M$, and
  process $P$, we define the \emph{contextual application}, $M[P] :=
  M\{P/\Box\}$. That is, the contextual application of M to P is the
  substitution of $P$ for $\Box$ in $M$.
\end{definition}

$\meaningof{-} : L \to \mathcal{P}(\pi)$

\begin{mathpar}
  \inferrule* [lab=collection] {} {\meaningof{true} = \pi, \and \meaningof{~E} = \pi \setminus \meaningof{E}, \and \meaningof{E_{1} \& E_{2}} = \meaningof{E_{1}} \cap \meaningof{E_{2}}}
\end{mathpar}

\begin{mathpar}
  \inferrule* [lab=structure] {} {\meaningof{0} = \{ P \in \pi | P \equiv 0 \}, \and \\ \meaningof{E_1 | E_2} = \{ P \in \pi | P \equiv P_{1} | P_{2}, P_{1} \in \meaningof{E_{1}}, P_{2} \in \meaningof{E_2}\} }
\end{mathpar}

\begin{mathpar}
 \inferrule* [lab=behavior] {} {\meaningof{\langle a?b \rangle E} = \{ P \in \pi | P \equiv Q | u?(y)P', \\ \and \\\\ \and \\ \;\;\; u \in \meaningof{a}, \forall z.P'\{z/y\} \in \meaningof{E\{z/b\}}\}, \and \\ \meaningof{a!E} = \{ P \in \pi | P \equiv Q | x!\langle P' \rangle, x \in \meaningof{a} P' \in \meaningof{E}\} }
\end{mathpar}

\begin{mathpar}
 \inferrule* [lab=nominal] {} {\meaningof{\quotep{E}} = \{ \quotep{P} \in \quotep{\pi} | P \in \meaningof{E} \}, \and \meaningof{\quotep{P}} = \{ \quotep{Q} \in \quotep{\pi} | P \equiv Q \} \and \\ \meaningof{@\quotep{E}} = \{ P \in \pi | P \equiv @x, x \in \meaningof{E} \}}
\end{mathpar}

\begin{eqnarray*}
  \\
  \meaningof{-} : TS \to ST
\end{eqnarray*}

\begin{eqnarray*}
  \\
  L : TS \to ST
\end{eqnarray*}

\begin{eqnarray*}
  \\
  P \models E \iff P \in \meaningof{E}
\end{eqnarray*}

\begin{eqnarray*}
  P \approx_{L} Q \iff \forall E \in L. P \models E \iff Q \models E
\end{eqnarray*}

\begin{eqnarray*}
  P \approx_{K} Q
\end{eqnarray*}

\begin{eqnarray*}
  P \approx Q
\end{eqnarray*}

$\approx_{K} = \approx = \approx_{L}$

\subsubsection{Contextual duality}

Note that contexts extend the quotation operation to a family of
operations from processes to names. Given a context, $M$, we can
define a \emph{nominal context}, $\quotep{M}$ by $\quotep{M}[P] :=
\quotep{M[P]}$. To foreshadow what is to come we observe that these
operations enjoy a duality with processes very much like the duality
between vectors and maps from vectors to scalars.

Further, because the calculus is essentially higher-order, we have a
correspondence between contexts and processes. More specifically,
given a name $x$ and a context $M$ we can construct $M^{*}_{x}$ such
that 

\begin{mathpar}
  M^{*}_{x} | \lift{x}{P} \red M[P]
\end{mathpar}

namely,

\begin{mathpar}
  M^{*}_{x} := x?(u).M[\dropn{u}]
\end{mathpar}

The dependence of $M^{*}_{x}$ on a name makes it an abstraction, 

\begin{mathpar}
  M^{*} := (x)x?(u).M[\dropn{u}]
\end{mathpar}

\subsection{Additional notation}

It will sometimes be convenient to denote the process a name
quotes. We already have the notation $x = \quotep{P}$, but it will be
convenient to introduce an alternate notation, $\procn{x}$, when we
want to emphasize the connection to the use of the name. Note that, by
virtue of name equivalence, $\quotep{\procn{x}} \nameeq x$; so, the
notation is consistent with previous definitions.

Further, because names have structure it is possible to effect
substitutions on the basis of that structure. This means we need to
upgrade our notation for substitutions, which we accomplish by
adapting comprehension notation. Thus,

\begin{mathpar}
  P\{ y / x : x \in S \}
\end{mathpar}

is interpreted to mean the process derived from P by replacing (in a
capture-avoiding manner) each occurrence of $x$ in $S$ by $y$. For example,

\begin{mathpar}
  P\{ \quotep{\procn{x}|\procn{x}} / x : x \in \freenames{P} \}
\end{mathpar}

will replace each (occurrence) of a free name $x$ in $P$ by
$\quotep{\procn{x}|\procn{x}}$.

Also, we will avail ourselves of the notation $x^{L}$ and $x^{R}$ to
denote injections of a name into disjoint copies of the name
space. There are numerous ways to accomplish this. One example can be
found in \cite{MeredithR05}. This notation overloads to vectors of
names: $\vec{x}^{\pi} := (x_{i}^{\pi} \; : \; 0 \leq i < |\vec{x}| )$ where $\pi \in \{L,R\}$.

We also use $P^{\Box} := P|\Box$.

In \cite{MeredithR05} an interpretation of the new operator is
given. It turns out that there are several possible interpretations
all enjoying the requisite algebraic properties of the operator (see
\cite{milner91polyadicpi}). We will therefore make liberal use of
$(\nu\; \vec{x})P$.

% subsection the_syntax_and_semantics_of_the_notation_system (end)   

\input{qm2pi.qmops} 

\input{qm2pi.sterngerlach} 

\input{qm2pi.metric} 

% section concurrent_process_calculi (end)

%\input{qm2pi.proofsketch}

% section proof sketch (end)

%\input{qm2pi.slviaknots} 

% section spatial logic via knots (end)

\input{qm2pi.conclusion}

% section conclusion (end)

%\input{qm2pi.dtcodes} 

% section wiring algorithm (end)

\input{qm2pi.ack} 

% section acknowledgments (end)

\newpage


\bibliographystyle{plain}   
\bibliography{../../biblios/main.bib}

\input{qm2pi.rhodetails}

\end{document}



\end{document}



% section front matter (end)

\section{Introduction}\label{sec:introduction} % (fold)
In this draft of the material i am going to have to dispense with the
usual writing conventions adopted in papers on these topics. i'm going
to have adopt whatever tone i need at the time i'm writing up the
calculations. Sometimes this may be very conversational; others it may
be the barest mathematical grunts; others still it may be that i have
lifted text from one of my other papers because the exposition of some
point was better said there. i hope that my readers are not unduly put
out by this decision. i'm not doing this to flout convention or be
rebellious. i find these calculations very technically challenging. To
keep everything going technically, something has to give; i have to
let go of some cognitive burden. So, the academic writing style --
with all of its trade-offs in terms of facilitating technical
communication -- is what i'm letting go of. Perhaps subsequent drafts
can be tightened and polished, but for now, i'm going to speak as if
we were sitting together in a coffee shop with a laptop, wifi and a
pad of paper and a pencil.

So, here's what i have to say. We -- you and i, comfortably ensconced
in our coffee shop and well-equipped with our tools -- can realize and
carry out the calculations of quantum mechanics over a very different
formal theory of dynamics, a formal theory of dynamics that
corresponds to a theory of concurrent computation with
\emph{reflection}. It has the advantage that the underlying theory is
already `quantized', but supports analogues all of the continuuous
operations. Strikingly, this underlying theory has recently been
connected with a notion of metric that we can show, by calculating
together, coincides with the metric induced by the inner product.

There are a lot of reasons why you might be interested in seeing
calculations of this form. Here's why i'm interested. For the past
several centuries there has been no competitor to the ``Newtonian''
account of dynamics. As a result the predominant share of accounts of
dynamical systems and situations have had to be formulated in terms of
the Newtonian machinery. i view this as an intellectually dangerous
position to occupy. Everything, despite it's intrinsic shape, turns
into a nail to be hit with this hammer. Recently, however, the theory
of computation has matured to the point where we have candidates for
theories of dynamics that offer very different perspective on
reasoning about dynamical systems and situations. Testing these
candidates against very successful accounts of dynamical situations,
like quantum mechanics, is going to give us some sense of how mature
they are and some measure of the quality of these accounts of
dynamics.

\subsection{Summary of contributions and outline of paper}

So, we're going to develop an interpretation of the operations of
quantum mechanics normally interpreted by Hilbert spaces and
operators. We're going to do this over a theory of computation. Note
that this is very different than the usual quantum computation program
which develops notions of computation over quantum mechanics. Rather,
we are developing a story that aligns with Wheeler's slogan: It from
Bit. To do this we will first provide an account of the theory of
computation at play here. Then we will dive into a calculation-driven
interpretation of the operations of quantum mechanics.

The reason we take this approach is that -- until very recently --
there hasn't been an axiomatic account of quantum mechanics. As a
result there has been no sharp delineation of the mathematical theory
supporting interpretation of the physical theory and the physical
theory, itself. So, ambient features of the maths are free to be
exploited (or supressed) without a real accounting of their physical
relevance. There is no sharp statement ``here's the physical theory''
qua \emph{theory} and ``here's the mathematical interpretation''
enabling a judgment of how faithful the interpretation is -- apart
from experimental observation. When there is an axiomatic account we
can judge how well a given mathematical formalism supports an
interpretation of the axioms, independent of
experimentation. Likewise, we can judge how well we have captured our
physical evidence and experience with our axiomatics, independent of
any specific mathematical implementation, with accidental detail that
may or may not have physical significance. 

In lieu of a fully fleshed out and vetted axiomatic account of quantum
mechanics, interpreting the operational notions in service of modeling
physical systems will have to suffice. In other words, we are not in
the business of providing a model of Hilbert spaces and operators. We
are in the business of providing a model of quantum mechanics because
we are motivated by testing our notions of dynamics against physical
theory; and, the predictive calculations of the physical theory must
serve as the best formulation -- shy of a fully fleshed out axiomatic
account -- of the physical theory itself (as they have for scientific
theories since time immemorial). Put another way, despite a
whole-hearted commitment to an It-from-Bit ontology, we are firmly
aligned with the shut-up-and-calculate camp as the best way to obtain
results either from the physical perspective or as a quality assurance
measure of our fledgling theory of dynamics.

In detail, we present a reflective process calculus. Then we develop
intuitive correspondences between the notions available in this
calculus and the usual physical notions supporting quantum mechanical
calculations. Thus, 

\begin{table}[htp]
  \center{
    \fbox{
      \begin{tabular}{c|c}
        quantum mechanics & process calculus \\
        \hline
        scalar & name \\
        state vector & process \\
        dual & contextual duals \\
        matrix & formal sums of process-context-dual pairs \\
        orthogonality & process annihilation \\
        inner product & execution-formula + quoting
      \end{tabular}
    }
  }
  \caption{QM - process calculi correspondences}
\end{table}

Then we tighten up these intuitions to operational definitions. We
employ the Dirac notation as the best proxy we can find for an
abstract syntax of the quantum mechanical notions. The definitions we
develop put us in contact with equational constraints coming from the
theory that we demonstrate the definitions and calculations satisfy.

This puts us in a position to shut up and calculate for the
Stern-Gerlach experimental set up, showing how these predictive
calculations become calculations on processes in our theory of a
reflective process calculus.

Penultimately, we demonstrate that the notion of metric coming from
the inner product coincides with the notion of metric available from
the theory of bisimulation. This demonstration gives us the right to
think of space as arising from behavior. Finally, we consider where we
might go from the new vantage point we have obtained.

% section introduction (end) 
 
% section introduction (end)

% \documentclass[12pt]{llncs}
%\documentclass{jktr}

\usepackage[pdftex]{hyperref}                   
\usepackage {listings}
\usepackage {mathpartir}
\usepackage{bcprules}
%\usepackage{listings}
                       
\usepackage{graphicx} 
%\usepackage[margins=2.5cm,nohead,nofoot]{geometry}
%\usepackage{geometry}
\usepackage{amsfonts}
\usepackage{amstext}
\usepackage{latexsym}
\usepackage{amssymb}
\usepackage{color}


%\include{myPreamble}
\documentclass[12pt]{llncs}
%\documentclass{jktr}

\usepackage[pdftex]{hyperref}                   
\usepackage {listings}
\usepackage {mathpartir}
\usepackage{bcprules}
%\usepackage{listings}
                       
\usepackage{graphicx} 
%\usepackage[margins=2.5cm,nohead,nofoot]{geometry}
%\usepackage{geometry}
\usepackage{amsfonts}
\usepackage{amstext}
\usepackage{latexsym}
\usepackage{amssymb}
\usepackage{color}


%\include{myPreamble}
\include{qm2pi.local} 

%\ifpdf
%\usepackage[pdftex]{graphicx}
%\else
%\usepackage{graphicx}
%\fi

 % \ifpdf
%  \usepackage{pdfsync}
%  \if


%\title{Brief Article}
%\author{David F. Snyder}
%\author{L.G. Meredith}

%\address{Dept. of Math., Texas State University--San Marcos, San Marcos, TX 78666}
       
\pagestyle{empty}


\begin{document}

\lstset{language=[Objective]Caml,frame=shadowbox}

\input{qm2pi.front}

% section front matter (end)

\input{qm2pi.intro} 
 
% section introduction (end)

% \input{qm2pi.knotations} 

% section notation (end)

\input{qm2pi.process.calculi} 

% section concurrent_process_calculi_and_spatial_logics_ (end)
    
%\input{qm2pi.knots2pi} 

%\input{qm2pi.trefoil} 

%\input{qm2pi.mainthm} 

% subsection basic_interpretation (end)

%\input{qm2pi.rho.presentation} 
\subsection{The syntax and semantics of the notation system}\label{sub:the_syntax_and_semantics_of_the_notation_system} % (fold)

We now summarize a technical presentation of the calculus that
embodies our theory of dynamics. The typical presentation of such a
calculus follows the style of giving generators and relations on
them. The grammar, below, describing term constructors, freely
generates the set of processes, $\Proc$. This set is then quotiented
by a relation known as structural congruence and it is over this set
that the notion of dynamics is expressed. This presentation is
essentially that of \cite{MeredithR05} with the addition of
polyadicity and summation. For readability we have relegated some of
the technical subtleties to an appendix.

\subsubsection{Process grammar}\label{subsub:process_grammar}

\begin{mathpar}
  \inferrule* [lab=synchronization] {} {{M} \bc \pzero \;|\; x?F \;|\; x!C }
  \and
  \inferrule* [lab=abstraction] {} {{F} \bc (x)P}
  \and
  \inferrule* [lab=concretion] {} {{C} \bc \langle Q \rangle}
  \and
  \inferrule* [lab=process] {} {{P,Q} \bc M \;| \;P|Q \;|\; @{x}}
  \and
  \inferrule* [lab=name] {} {{x} \bc \quotep{P}}
\end{mathpar} 

Note that $\vec{x}$ (resp. $\vec{P}$) denotes a vector of names
(resp. processes) of length $|\vec{x}|$ (resp. $|\vec{P}|$). We adopt
the following useful abbreviations.

\begin{mathpar}
   x?(\vec{y}).P := x.(\vec{y})P \and  x\clift{\vec{P}} := x.\clift{\vec{P}}
   \and x!(y) := \lift{x}{\dropn{y}}
   \and \Pi_{i=0}^{n-1}P_i := P_0 | \ldots | P_{n-1}
\end{mathpar}

\subsubsection{Structural congruence}

\paragraph{Free and bound names and alpha-equivalence.} At the
core of structural equivalence is alpha-equivalence which identifies
process that are the same up to a change of variable. Formally, we
recognize the distinction between free and bound names. The free names
of a process, $\freenames{P}$, may be calculated recursively as
follows:

\begin{mathpar}
\freenames{\pzero} := \emptyset
  \and \\
  \freenames{x?(y).P} := \{ x \} \cup (\freenames{P} \setminus \{ y \})
  \and 
  \freenames{x!\langle P \rangle} := \{ x \} \cup \{ P \} 
  \and \\
  \freenames{P|Q} := \freenames{P} \cup \freenames{Q}
  \and \\
  \freenames{@{x}} := \{ x \}
\end{mathpar}

$\pi$
$\quotep{\pi}$

$\freenames{-} : \pi \to \mathcal{P}(\quotep{\pi})$

\begin{eqnarray*}
  \freenames{\pzero} & := & \emptyset \\
  \freenames{x?(y).P} & := & \{ x \} \cup (\freenames{P} \setminus \{ y \}) \\
  \freenames{x!\langle P \rangle} & := & \{ x \} \cup \{ P \} \\
  \freenames{P|Q} & := & \freenames{P} \cup \freenames{Q} \\
  \freenames{\dropn{x}} & := & \{ x \}
\end{eqnarray*}

The bound names of a process, $\boundnames{P}$, are those names occurring in $P$
that are not free. For example, in $x?(y).0$, the name $x$ is free, while $y$ is bound.

\begin{mathpar}
  \inferrule* [lab=monoidal-laws] {} { P|Q \equiv Q|P \and P|0 \equiv P \and P|(Q|R) \equiv (P|Q)|R }
\end{mathpar}

\begin{mathpar}
  \inferrule* [lab=alpha-equivalence] {} { (x)P \equiv (y)P\{y/x\} \and y \not\in \freenames{P} }
\end{mathpar}

\begin{definition}
Then two processes, $P,Q$, are alpha-equivalent if $P = Q\{\vec{y}/\vec{x}\}$ for
some $\vec{x} \in \boundnames{Q},\vec{y} \in \boundnames{P}$, where $Q\{\vec{y}/\vec{x}\}$
denotes the capture-avoiding substitution of $\vec{y}$ for $\vec{x}$ in $Q$.
\end{definition}

\begin{definition}
  The {\em structural congruence} \cite{SangiorgiWalker} , $\equiv$,
  between processes is the least congruence containing
  alpha-equivalence, satisfying the abelian monoid laws
  (associativity, commutativity and $\pzero$ as identity) for parallel
  composition $|$ and for summation $+$.
\end{definition}

\subsection{Name equivalence}

We take name equivalence, written $\nameeq$, to be the smallest
equivalence relation generated by the following rules.

\begin{mathpar}
\inferrule*[lab=Quote-drop]
{ }
{ \quotep{@{x}} \nameeq x }

\inferrule*[lab=Struct-equiv]
{ P \scong Q }
{ \quotep{P} \nameeq \quotep{Q} }
\end{mathpar}

The astute reader will have noticed that the mutual recursion of names
and processes imposes a mutual recursion on alpha-equivalence and
structural equivalence via name-equivalence. Fortunately, all of this
works out pleasantly and we may calculate in the natural way, free of
concern. The reader interested in the details is referred to the
appendix \ref{appendix:rho_details}.

\subsection{Substitution}

We use $\Proc$ for the set of processes, $\QProc$ for the set of
names, and $\id{\{}\vec{y} / \vec{x} \id{\}}$ to denote partial maps,
$s : \QProc \rightarrow \QProc$. A map, $s$ lifts, uniquely, to a map
on process terms, $\widehat{s} : \Proc \rightarrow \Proc$ by the
following equations.

\begin{mathpar}
  (0) \psubstp{Q}{P} := 0 \\
  (R \juxtap S) \psubstp{Q}{P}
  :=    
  (R)\psubstp{Q}{P} \juxtap (S) \psubstp{Q}{P} \\
  (x?(y).R) \psubstp{Q}{P}    
  :=    
  (x)\substp{Q}{P} (z)\concat( (R \psubstn{z}{y}) \psubstp{Q}{P} ) \\
  (\lift{x}{R}) \psubstp{Q}{P}  
  :=
  \lift{(x)\substp{Q}{P}}{ R \psubstp{Q}{P} } \\
%   (\dropn{x})  \psubstp{Q}{P}       
%   := 
%   \left\{ 
%     \begin{array}{ccc} 
%       \dropn{\quotep{Q}} & & x \nameeq \quotep{P} \\
%       \dropn{x} & & otherwise \\
%     \end{array}
%   \right. 
  (\dropn{x})  \psubstp{Q}{P}       
  := 
  \left\{ 
    \begin{array}{ccc} 
      Q & & x \nameeq \quotep{P} \\
      \dropn{x} & & otherwise \\
    \end{array}
  \right.
\end{mathpar}
 

where

\begin{eqnarray}
  (x)\id{\{} \lpquote Q \rpquote / \lpquote P \rpquote \id{\}}            = 
  \left\{ 
    \begin{array}{ccc}
      \lpquote Q \rpquote & & x \nameeq \lpquote P \rpquote \\
      x & & otherwise \\
    \end{array}
  \right. \nonumber
\end{eqnarray}

and $z$ is chosen distinct from $\quotep{P}$, $\quotep{Q}$, the free
names in $Q$, and all the names in $R$. Our $\alpha$-equivalence will
be built in the standard way from this substitution.

\begin{remark}\label{rem:no_self_referential_names}
  One consequence of these definitions is that $\forall P. \quotep{P}
  \not\in \freenames{P}$.
\end{remark}

\subsection{ Dynamic quote: an example }

Anticipating something of what's to come, consider applying the
substitution, $\widehat{\id{\{}u / z \id{\}}}$, to the following pair
of processes, $\lift{w}{y!(z)}$ and $w[ \lpquote y!(z) \rpquote ]$.

\begin{eqnarray}
	\lift{w}{y!(z)}\widehat{\id{\{}u / z \id{\}}}
		& = &
		\lift{w}{y!(u)} \nonumber\\
	w[ \lpquote y!(z) \rpquote ] \widehat{ \id{\{}u / z \id{\}} }
		& = &
		w[ \lpquote y!(z) \rpquote ] \nonumber
\end{eqnarray}

Because the body of the process between quotes is impervious to
substitution, we get radically different answers. In fact, by
examining the first process in an input context,
e.g. $x?(z).\lift{w}{y!(z)}$, we see that the process under the lift
operator may be shaped by prefixed inputs binding a name inside it. In
this sense, the lift operator will be seen as a way to dynamically
construct processes before reifying them as names.

Finally equipped with these standard features we can present the
dynamics of the calculus.

\subsubsection{Operational semantics} 

Finally, we introduce the computational dynamics. What marks these
algebras as distinct from other more traditionally studied algebraic
structures, e.g. vector spaces or polynomial rings, is the manner in
which dynamics is captured. In traditional structures, dynamics is typically
expressed through morphisms between such structures, as in linear maps
between vector spaces or morphisms between rings. In algebras
associated with the semantics of computation, the dynamics is
expressed as part of the algebraic structure itself, through a
reduction reduction relation typically denoted by $\red$. Below, we
give a recursive presentation of this relation for the calculus used
in the encoding.

$\red \subseteq \pi \times \pi$
$\red : \pi \to \mathcal{P}(\pi)$

\begin{mathpar}
  \inferrule* [lab=Comm] { \textsf{match}( x_{src}, x_{trgt} ) } { x_{trgt}?(y)P \; | \; x_{src}!\langle {Q} \rangle \red P\{\quotep{Q}/y}\} }
  \and \\
  \inferrule* [lab=Par] {{P} \red {P}'} {{{P} | {Q}} \red {{P}' | {Q}}}
  \and
  \inferrule* [lab=Equiv]{{{P} \scong {P}'} \andalso {{P}' \red {Q}'} \andalso {{Q}' \scong {Q}}}{{P} \red {Q}}
\end{mathpar}

\begin{eqnarray*}
  match_{\equiv} (\quotep{P},\quotep{Q}) & := & P \equiv Q \\
  match_{\dagger}(\quotep{P},\quotep{Q}) & := & \forall R. P|Q \red^{*} R => R \red^{*} 0 \\
  match_{K}(\quotep{P},\quotep{Q}) & := & K \mbox{ for some context } K
\end{eqnarray*}

$u?(x)P | u!\langle Q \rangle \red P\{\quotep{Q}/x\}$

%We write $\wred$ for $\red^*$, and $P\red$ if $\exists Q $ such that $ P \red Q$.
We write $P\red$ if $\exists Q $ such that $ P \red Q$ and $P\not\red$, otherwise.

\section{Replication}

As mentioned before, it is known that replication (and hence
recursion) can be implemented in a higher-order process algebra
\cite{SangiorgiWalker}. As our first example of calculation with the
machinery thus far presented we give the construction explicitly in
the {\rhoc}.

\begin{eqnarray}
	D_{x} & := & \prefix{x}{y}{(\binpar{\outputp{x}{y}}{@{y}})} \nonumber\\
	\bangp_{x}{P} & := & \binpar{{x}!\langle{\binpar{D_{x}}{P}}\rangle}{D_{x}} \nonumber
\end{eqnarray}

\begin{eqnarray}
	\bangp_{x}{P} & & \nonumber\\
	=
	& {x}!\langle{(\prefix{x}{y}{(\outputp{x}{y} | @{y})) | P}}\rangle 
	      | \prefix{x}{y}{(\outputp{x}{y} | @{y})} & \nonumber\\
	\red
	& (\outputp{x}{y} | @{y})\substn{\quotep{(\prefix{x}{y}{(@{y} | \outputp{x}{y})) | P}}}{y} & \nonumber\\
	=
	& \outputp{x}{\quotep{(\prefix{x}{y}{(\outputp{x}{y} | @{y})) | P}}}
	  | {(\prefix{x}{y}{(\outputp{x}{y} | @{y})) | P}} & \nonumber\\
	\red
	& \ldots & \nonumber\\
	\red^*
	& P | P | \ldots & \nonumber
\end{eqnarray}

Of course, this encoding, as an implementation, runs away, unfolding
$\bangp{P}$ eagerly. A lazier and more implementable replication
operator, restricted to input-guarded processes, may be obtained as follows.

\begin{eqnarray}
\bangp{\prefix{u}{v}{P}} 
	:= 
	\binpar{\lift{x}{\prefix{u}{v}{(\binpar{D(x)}{P})}}}{D(x)} \nonumber
\end{eqnarray}

\begin{remark}
  Note that the lazier definition still does not deal with summation
  or mixed summation (i.e. sums over input and output). The reader is
  invited to construct definitions of replication that deal with these
  features. 

  Further, the definitions are parameterized in a name, $x$. Can you,
  gentle reader, make a definition that eliminates this parameter and
  guarantees no accidental interaction between the replication
  machinery and the process being replicated -- i.e. no accidental
  sharing of names used by the process to get its work done and the
  name(s) used by the replication to effect copying. This latter
  revision of the definition of replication is crucial to obtaining
  the expected identity $!!P \sim !P$.
\end{remark}

\begin{remark}\label{rem:paradoxical_combinator}
  The reader familiar with the lambda calculus will have noticed the
  similarity between $D$ and the paradoxical combinator.

  [Ed. note: the existence of this seems to suggest we have to be more
  restrictive on the set of processes and names we admit if we are to
  support no-cloning.]
\end{remark}

\subsubsection{Bisimulation}

The computational dynamics gives rise to another kind of equivalence,
the equivalence of computational behavior. As previously mentioned
this is typically captured \emph{via} some form of bisimulation.

% The notion we use in this paper is weak barbed bisimulation
% \cite{milner91polyadicpi}.

The notion we use in this paper is derived from weak barbed
bisimulation \cite{milner91polyadicpi}. 

\begin{definition}
An \emph{observation relation}, $\downarrow_{\mathcal N}$, over a set
of names, $\mathcal N$, is the smallest relation satisfying the rules
below.

\infrule[Out-barb]{y \in {\mathcal N}, \; x \nameeq y}
		  {\outputp{x}{v} \downarrow_{\mathcal N} x}
\infrule[Par-barb]{\mbox{$P\downarrow_{\mathcal N} x$ or $Q\downarrow_{\mathcal N} x$}}
		  {\binpar{P}{Q} \downarrow_{\mathcal N} x}

We write $P \Downarrow_{\mathcal N} x$ if there is $Q$ such that 
$P \wred Q$ and $Q \downarrow_{\mathcal N} x$.
\end{definition}

\begin{definition}
%\label{def.bbisim}
An  ${\mathcal N}$-\emph{barbed bisimulation} over a set of names, ${\mathcal N}$, is a symmetric binary relation 
${\mathcal S}_{\mathcal N}$ between agents such that $P\rel{S}_{\mathcal N}Q$ implies:
\begin{enumerate}
\item If $P \red P'$ then $Q \wred Q'$ and $P'\rel{S}_{\mathcal N} Q'$.
\item If $P\downarrow_{\mathcal N} x$, then $Q\Downarrow_{\mathcal N} x$.
\end{enumerate}
$P$ is ${\mathcal N}$-barbed bisimilar to $Q$, written
$P \wbbisim_{\mathcal N} Q$, if $P \rel{S}_{\mathcal N} Q$ for some ${\mathcal N}$-barbed bisimulation ${\mathcal S}_{\mathcal N}$.
\end{definition}

$\mathcal{R} \subseteq \pi \times \pi$

$P \mathcal{R} Q => \forall P'. P \red P' \Rightarrow \exists Q'. Q \red Q', P' \mathcal{R} Q'$

$P \vdash x \Rightarrow Q \vdash x$

\begin{mathpar}
  \inferrule*[lab=Out-barb]{x \nameeq y}{{y}!\langle{Q}\rangle \vdash x}
  \and
  \inferrule*[lab=Par-barb]{\mbox{$P\vdash x$ or $Q\vdash x$}}{\binpar{P}{Q} \vdash x}
\end{mathpar}

\subsubsection{Contexts}

One of the principle advantages of computational calculi like the
$\pi$-calculus is a well-defined notion of context,
contextual-equivalence and a correlation between
contextual-equivalence and notions of bisimulation. The notion of
context allows the decomposition of a process into (sub-)process and
its syntactic environment, its context. Thus, a context may be
thought of as a process with a ``hole'' (written $\Box$) in it. The
application of a context $M$ to a process $P$, written $M[P]$, is
tantamount to filling the hole in $M$ with $P$. In this paper we do
not need the full weight of this theory, but do make use of the notion
of context in the proof the main theorem. 

\begin{mathpar}
  \inferrule* [lab=summation] {} {{M_{M},M_{N}} \bc \Box \;|\; x.M_{A} \;|\; M_{M}+M_{N}}
  \and
  \inferrule* [lab=agent] {} {{M_{A}} \bc (\vec{x})M_{P} \;| \; \clift{P_0,\ldots,M_{P},\ldots,P_N}}
  \and \\
  \inferrule* [lab=process] {} {{M_{P}} \bc M_{N} \;| \;P|M_{P} }
\end{mathpar} 

\begin{mathpar}
  \inferrule* [lab=sychronization] {} {M_{N} \bc \Box \;|\; x?M_{F} \;|\; x!M_{C}}
  \and
  \inferrule* [lab=abstraction] {} {{M_{F}} \bc (x)M_{P} }
  \and
  \inferrule* [lab=concretion] {} {{M_{C}} \bc \langle M_{P} \rangle }
  \and \\
  \inferrule* [lab=process] {} {{M_{P}} \bc M_{N} \;| \;P|M_{P} }
\end{mathpar}

\begin{definition}[contextual application] Given a context $M$, and
  process $P$, we define the \emph{contextual application}, $M[P] :=
  M\{P/\Box\}$. That is, the contextual application of M to P is the
  substitution of $P$ for $\Box$ in $M$.
\end{definition}

$\meaningof{-} : L \to \mathcal{P}(\pi)$

\begin{mathpar}
  \inferrule* [lab=collection] {} {\meaningof{true} = \pi, \and \meaningof{~E} = \pi \setminus \meaningof{E}, \and \meaningof{E_{1} \& E_{2}} = \meaningof{E_{1}} \cap \meaningof{E_{2}}}
\end{mathpar}

\begin{mathpar}
  \inferrule* [lab=structure] {} {\meaningof{0} = \{ P \in \pi | P \equiv 0 \}, \and \\ \meaningof{E_1 | E_2} = \{ P \in \pi | P \equiv P_{1} | P_{2}, P_{1} \in \meaningof{E_{1}}, P_{2} \in \meaningof{E_2}\} }
\end{mathpar}

\begin{mathpar}
 \inferrule* [lab=behavior] {} {\meaningof{\langle a?b \rangle E} = \{ P \in \pi | P \equiv Q | u?(y)P', \\ \and \\\\ \and \\ \;\;\; u \in \meaningof{a}, \forall z.P'\{z/y\} \in \meaningof{E\{z/b\}}\}, \and \\ \meaningof{a!E} = \{ P \in \pi | P \equiv Q | x!\langle P' \rangle, x \in \meaningof{a} P' \in \meaningof{E}\} }
\end{mathpar}

\begin{mathpar}
 \inferrule* [lab=nominal] {} {\meaningof{\quotep{E}} = \{ \quotep{P} \in \quotep{\pi} | P \in \meaningof{E} \}, \and \meaningof{\quotep{P}} = \{ \quotep{Q} \in \quotep{\pi} | P \equiv Q \} \and \\ \meaningof{@\quotep{E}} = \{ P \in \pi | P \equiv @x, x \in \meaningof{E} \}}
\end{mathpar}

\begin{eqnarray*}
  \\
  \meaningof{-} : TS \to ST
\end{eqnarray*}

\begin{eqnarray*}
  \\
  L : TS \to ST
\end{eqnarray*}

\begin{eqnarray*}
  \\
  P \models E \iff P \in \meaningof{E}
\end{eqnarray*}

\begin{eqnarray*}
  P \approx_{L} Q \iff \forall E \in L. P \models E \iff Q \models E
\end{eqnarray*}

\begin{eqnarray*}
  P \approx_{K} Q
\end{eqnarray*}

\begin{eqnarray*}
  P \approx Q
\end{eqnarray*}

$\approx_{K} = \approx = \approx_{L}$

\subsubsection{Contextual duality}

Note that contexts extend the quotation operation to a family of
operations from processes to names. Given a context, $M$, we can
define a \emph{nominal context}, $\quotep{M}$ by $\quotep{M}[P] :=
\quotep{M[P]}$. To foreshadow what is to come we observe that these
operations enjoy a duality with processes very much like the duality
between vectors and maps from vectors to scalars.

Further, because the calculus is essentially higher-order, we have a
correspondence between contexts and processes. More specifically,
given a name $x$ and a context $M$ we can construct $M^{*}_{x}$ such
that 

\begin{mathpar}
  M^{*}_{x} | \lift{x}{P} \red M[P]
\end{mathpar}

namely,

\begin{mathpar}
  M^{*}_{x} := x?(u).M[\dropn{u}]
\end{mathpar}

The dependence of $M^{*}_{x}$ on a name makes it an abstraction, 

\begin{mathpar}
  M^{*} := (x)x?(u).M[\dropn{u}]
\end{mathpar}

\subsection{Additional notation}

It will sometimes be convenient to denote the process a name
quotes. We already have the notation $x = \quotep{P}$, but it will be
convenient to introduce an alternate notation, $\procn{x}$, when we
want to emphasize the connection to the use of the name. Note that, by
virtue of name equivalence, $\quotep{\procn{x}} \nameeq x$; so, the
notation is consistent with previous definitions.

Further, because names have structure it is possible to effect
substitutions on the basis of that structure. This means we need to
upgrade our notation for substitutions, which we accomplish by
adapting comprehension notation. Thus,

\begin{mathpar}
  P\{ y / x : x \in S \}
\end{mathpar}

is interpreted to mean the process derived from P by replacing (in a
capture-avoiding manner) each occurrence of $x$ in $S$ by $y$. For example,

\begin{mathpar}
  P\{ \quotep{\procn{x}|\procn{x}} / x : x \in \freenames{P} \}
\end{mathpar}

will replace each (occurrence) of a free name $x$ in $P$ by
$\quotep{\procn{x}|\procn{x}}$.

Also, we will avail ourselves of the notation $x^{L}$ and $x^{R}$ to
denote injections of a name into disjoint copies of the name
space. There are numerous ways to accomplish this. One example can be
found in \cite{MeredithR05}. This notation overloads to vectors of
names: $\vec{x}^{\pi} := (x_{i}^{\pi} \; : \; 0 \leq i < |\vec{x}| )$ where $\pi \in \{L,R\}$.

We also use $P^{\Box} := P|\Box$.

In \cite{MeredithR05} an interpretation of the new operator is
given. It turns out that there are several possible interpretations
all enjoying the requisite algebraic properties of the operator (see
\cite{milner91polyadicpi}). We will therefore make liberal use of
$(\nu\; \vec{x})P$.

% subsection the_syntax_and_semantics_of_the_notation_system (end)   

\input{qm2pi.qmops} 

\input{qm2pi.sterngerlach} 

\input{qm2pi.metric} 

% section concurrent_process_calculi (end)

%\input{qm2pi.proofsketch}

% section proof sketch (end)

%\input{qm2pi.slviaknots} 

% section spatial logic via knots (end)

\input{qm2pi.conclusion}

% section conclusion (end)

%\input{qm2pi.dtcodes} 

% section wiring algorithm (end)

\input{qm2pi.ack} 

% section acknowledgments (end)

\newpage


\bibliographystyle{plain}   
\bibliography{../../biblios/main.bib}

\input{qm2pi.rhodetails}

\end{document}

 

%\ifpdf
%\usepackage[pdftex]{graphicx}
%\else
%\usepackage{graphicx}
%\fi

 % \ifpdf
%  \usepackage{pdfsync}
%  \if


%\title{Brief Article}
%\author{David F. Snyder}
%\author{L.G. Meredith}

%\address{Dept. of Math., Texas State University--San Marcos, San Marcos, TX 78666}
       
\pagestyle{empty}


\begin{document}

\lstset{language=[Objective]Caml,frame=shadowbox}

\documentclass[12pt]{llncs}
%\documentclass{jktr}

\usepackage[pdftex]{hyperref}                   
\usepackage {listings}
\usepackage {mathpartir}
\usepackage{bcprules}
%\usepackage{listings}
                       
\usepackage{graphicx} 
%\usepackage[margins=2.5cm,nohead,nofoot]{geometry}
%\usepackage{geometry}
\usepackage{amsfonts}
\usepackage{amstext}
\usepackage{latexsym}
\usepackage{amssymb}
\usepackage{color}


%\include{myPreamble}
\include{qm2pi.local} 

%\ifpdf
%\usepackage[pdftex]{graphicx}
%\else
%\usepackage{graphicx}
%\fi

 % \ifpdf
%  \usepackage{pdfsync}
%  \if


%\title{Brief Article}
%\author{David F. Snyder}
%\author{L.G. Meredith}

%\address{Dept. of Math., Texas State University--San Marcos, San Marcos, TX 78666}
       
\pagestyle{empty}


\begin{document}

\lstset{language=[Objective]Caml,frame=shadowbox}

\input{qm2pi.front}

% section front matter (end)

\input{qm2pi.intro} 
 
% section introduction (end)

% \input{qm2pi.knotations} 

% section notation (end)

\input{qm2pi.process.calculi} 

% section concurrent_process_calculi_and_spatial_logics_ (end)
    
%\input{qm2pi.knots2pi} 

%\input{qm2pi.trefoil} 

%\input{qm2pi.mainthm} 

% subsection basic_interpretation (end)

%\input{qm2pi.rho.presentation} 
\subsection{The syntax and semantics of the notation system}\label{sub:the_syntax_and_semantics_of_the_notation_system} % (fold)

We now summarize a technical presentation of the calculus that
embodies our theory of dynamics. The typical presentation of such a
calculus follows the style of giving generators and relations on
them. The grammar, below, describing term constructors, freely
generates the set of processes, $\Proc$. This set is then quotiented
by a relation known as structural congruence and it is over this set
that the notion of dynamics is expressed. This presentation is
essentially that of \cite{MeredithR05} with the addition of
polyadicity and summation. For readability we have relegated some of
the technical subtleties to an appendix.

\subsubsection{Process grammar}\label{subsub:process_grammar}

\begin{mathpar}
  \inferrule* [lab=synchronization] {} {{M} \bc \pzero \;|\; x?F \;|\; x!C }
  \and
  \inferrule* [lab=abstraction] {} {{F} \bc (x)P}
  \and
  \inferrule* [lab=concretion] {} {{C} \bc \langle Q \rangle}
  \and
  \inferrule* [lab=process] {} {{P,Q} \bc M \;| \;P|Q \;|\; @{x}}
  \and
  \inferrule* [lab=name] {} {{x} \bc \quotep{P}}
\end{mathpar} 

Note that $\vec{x}$ (resp. $\vec{P}$) denotes a vector of names
(resp. processes) of length $|\vec{x}|$ (resp. $|\vec{P}|$). We adopt
the following useful abbreviations.

\begin{mathpar}
   x?(\vec{y}).P := x.(\vec{y})P \and  x\clift{\vec{P}} := x.\clift{\vec{P}}
   \and x!(y) := \lift{x}{\dropn{y}}
   \and \Pi_{i=0}^{n-1}P_i := P_0 | \ldots | P_{n-1}
\end{mathpar}

\subsubsection{Structural congruence}

\paragraph{Free and bound names and alpha-equivalence.} At the
core of structural equivalence is alpha-equivalence which identifies
process that are the same up to a change of variable. Formally, we
recognize the distinction between free and bound names. The free names
of a process, $\freenames{P}$, may be calculated recursively as
follows:

\begin{mathpar}
\freenames{\pzero} := \emptyset
  \and \\
  \freenames{x?(y).P} := \{ x \} \cup (\freenames{P} \setminus \{ y \})
  \and 
  \freenames{x!\langle P \rangle} := \{ x \} \cup \{ P \} 
  \and \\
  \freenames{P|Q} := \freenames{P} \cup \freenames{Q}
  \and \\
  \freenames{@{x}} := \{ x \}
\end{mathpar}

$\pi$
$\quotep{\pi}$

$\freenames{-} : \pi \to \mathcal{P}(\quotep{\pi})$

\begin{eqnarray*}
  \freenames{\pzero} & := & \emptyset \\
  \freenames{x?(y).P} & := & \{ x \} \cup (\freenames{P} \setminus \{ y \}) \\
  \freenames{x!\langle P \rangle} & := & \{ x \} \cup \{ P \} \\
  \freenames{P|Q} & := & \freenames{P} \cup \freenames{Q} \\
  \freenames{\dropn{x}} & := & \{ x \}
\end{eqnarray*}

The bound names of a process, $\boundnames{P}$, are those names occurring in $P$
that are not free. For example, in $x?(y).0$, the name $x$ is free, while $y$ is bound.

\begin{mathpar}
  \inferrule* [lab=monoidal-laws] {} { P|Q \equiv Q|P \and P|0 \equiv P \and P|(Q|R) \equiv (P|Q)|R }
\end{mathpar}

\begin{mathpar}
  \inferrule* [lab=alpha-equivalence] {} { (x)P \equiv (y)P\{y/x\} \and y \not\in \freenames{P} }
\end{mathpar}

\begin{definition}
Then two processes, $P,Q$, are alpha-equivalent if $P = Q\{\vec{y}/\vec{x}\}$ for
some $\vec{x} \in \boundnames{Q},\vec{y} \in \boundnames{P}$, where $Q\{\vec{y}/\vec{x}\}$
denotes the capture-avoiding substitution of $\vec{y}$ for $\vec{x}$ in $Q$.
\end{definition}

\begin{definition}
  The {\em structural congruence} \cite{SangiorgiWalker} , $\equiv$,
  between processes is the least congruence containing
  alpha-equivalence, satisfying the abelian monoid laws
  (associativity, commutativity and $\pzero$ as identity) for parallel
  composition $|$ and for summation $+$.
\end{definition}

\subsection{Name equivalence}

We take name equivalence, written $\nameeq$, to be the smallest
equivalence relation generated by the following rules.

\begin{mathpar}
\inferrule*[lab=Quote-drop]
{ }
{ \quotep{@{x}} \nameeq x }

\inferrule*[lab=Struct-equiv]
{ P \scong Q }
{ \quotep{P} \nameeq \quotep{Q} }
\end{mathpar}

The astute reader will have noticed that the mutual recursion of names
and processes imposes a mutual recursion on alpha-equivalence and
structural equivalence via name-equivalence. Fortunately, all of this
works out pleasantly and we may calculate in the natural way, free of
concern. The reader interested in the details is referred to the
appendix \ref{appendix:rho_details}.

\subsection{Substitution}

We use $\Proc$ for the set of processes, $\QProc$ for the set of
names, and $\id{\{}\vec{y} / \vec{x} \id{\}}$ to denote partial maps,
$s : \QProc \rightarrow \QProc$. A map, $s$ lifts, uniquely, to a map
on process terms, $\widehat{s} : \Proc \rightarrow \Proc$ by the
following equations.

\begin{mathpar}
  (0) \psubstp{Q}{P} := 0 \\
  (R \juxtap S) \psubstp{Q}{P}
  :=    
  (R)\psubstp{Q}{P} \juxtap (S) \psubstp{Q}{P} \\
  (x?(y).R) \psubstp{Q}{P}    
  :=    
  (x)\substp{Q}{P} (z)\concat( (R \psubstn{z}{y}) \psubstp{Q}{P} ) \\
  (\lift{x}{R}) \psubstp{Q}{P}  
  :=
  \lift{(x)\substp{Q}{P}}{ R \psubstp{Q}{P} } \\
%   (\dropn{x})  \psubstp{Q}{P}       
%   := 
%   \left\{ 
%     \begin{array}{ccc} 
%       \dropn{\quotep{Q}} & & x \nameeq \quotep{P} \\
%       \dropn{x} & & otherwise \\
%     \end{array}
%   \right. 
  (\dropn{x})  \psubstp{Q}{P}       
  := 
  \left\{ 
    \begin{array}{ccc} 
      Q & & x \nameeq \quotep{P} \\
      \dropn{x} & & otherwise \\
    \end{array}
  \right.
\end{mathpar}
 

where

\begin{eqnarray}
  (x)\id{\{} \lpquote Q \rpquote / \lpquote P \rpquote \id{\}}            = 
  \left\{ 
    \begin{array}{ccc}
      \lpquote Q \rpquote & & x \nameeq \lpquote P \rpquote \\
      x & & otherwise \\
    \end{array}
  \right. \nonumber
\end{eqnarray}

and $z$ is chosen distinct from $\quotep{P}$, $\quotep{Q}$, the free
names in $Q$, and all the names in $R$. Our $\alpha$-equivalence will
be built in the standard way from this substitution.

\begin{remark}\label{rem:no_self_referential_names}
  One consequence of these definitions is that $\forall P. \quotep{P}
  \not\in \freenames{P}$.
\end{remark}

\subsection{ Dynamic quote: an example }

Anticipating something of what's to come, consider applying the
substitution, $\widehat{\id{\{}u / z \id{\}}}$, to the following pair
of processes, $\lift{w}{y!(z)}$ and $w[ \lpquote y!(z) \rpquote ]$.

\begin{eqnarray}
	\lift{w}{y!(z)}\widehat{\id{\{}u / z \id{\}}}
		& = &
		\lift{w}{y!(u)} \nonumber\\
	w[ \lpquote y!(z) \rpquote ] \widehat{ \id{\{}u / z \id{\}} }
		& = &
		w[ \lpquote y!(z) \rpquote ] \nonumber
\end{eqnarray}

Because the body of the process between quotes is impervious to
substitution, we get radically different answers. In fact, by
examining the first process in an input context,
e.g. $x?(z).\lift{w}{y!(z)}$, we see that the process under the lift
operator may be shaped by prefixed inputs binding a name inside it. In
this sense, the lift operator will be seen as a way to dynamically
construct processes before reifying them as names.

Finally equipped with these standard features we can present the
dynamics of the calculus.

\subsubsection{Operational semantics} 

Finally, we introduce the computational dynamics. What marks these
algebras as distinct from other more traditionally studied algebraic
structures, e.g. vector spaces or polynomial rings, is the manner in
which dynamics is captured. In traditional structures, dynamics is typically
expressed through morphisms between such structures, as in linear maps
between vector spaces or morphisms between rings. In algebras
associated with the semantics of computation, the dynamics is
expressed as part of the algebraic structure itself, through a
reduction reduction relation typically denoted by $\red$. Below, we
give a recursive presentation of this relation for the calculus used
in the encoding.

$\red \subseteq \pi \times \pi$
$\red : \pi \to \mathcal{P}(\pi)$

\begin{mathpar}
  \inferrule* [lab=Comm] { \textsf{match}( x_{src}, x_{trgt} ) } { x_{trgt}?(y)P \; | \; x_{src}!\langle {Q} \rangle \red P\{\quotep{Q}/y}\} }
  \and \\
  \inferrule* [lab=Par] {{P} \red {P}'} {{{P} | {Q}} \red {{P}' | {Q}}}
  \and
  \inferrule* [lab=Equiv]{{{P} \scong {P}'} \andalso {{P}' \red {Q}'} \andalso {{Q}' \scong {Q}}}{{P} \red {Q}}
\end{mathpar}

\begin{eqnarray*}
  match_{\equiv} (\quotep{P},\quotep{Q}) & := & P \equiv Q \\
  match_{\dagger}(\quotep{P},\quotep{Q}) & := & \forall R. P|Q \red^{*} R => R \red^{*} 0 \\
  match_{K}(\quotep{P},\quotep{Q}) & := & K \mbox{ for some context } K
\end{eqnarray*}

$u?(x)P | u!\langle Q \rangle \red P\{\quotep{Q}/x\}$

%We write $\wred$ for $\red^*$, and $P\red$ if $\exists Q $ such that $ P \red Q$.
We write $P\red$ if $\exists Q $ such that $ P \red Q$ and $P\not\red$, otherwise.

\section{Replication}

As mentioned before, it is known that replication (and hence
recursion) can be implemented in a higher-order process algebra
\cite{SangiorgiWalker}. As our first example of calculation with the
machinery thus far presented we give the construction explicitly in
the {\rhoc}.

\begin{eqnarray}
	D_{x} & := & \prefix{x}{y}{(\binpar{\outputp{x}{y}}{@{y}})} \nonumber\\
	\bangp_{x}{P} & := & \binpar{{x}!\langle{\binpar{D_{x}}{P}}\rangle}{D_{x}} \nonumber
\end{eqnarray}

\begin{eqnarray}
	\bangp_{x}{P} & & \nonumber\\
	=
	& {x}!\langle{(\prefix{x}{y}{(\outputp{x}{y} | @{y})) | P}}\rangle 
	      | \prefix{x}{y}{(\outputp{x}{y} | @{y})} & \nonumber\\
	\red
	& (\outputp{x}{y} | @{y})\substn{\quotep{(\prefix{x}{y}{(@{y} | \outputp{x}{y})) | P}}}{y} & \nonumber\\
	=
	& \outputp{x}{\quotep{(\prefix{x}{y}{(\outputp{x}{y} | @{y})) | P}}}
	  | {(\prefix{x}{y}{(\outputp{x}{y} | @{y})) | P}} & \nonumber\\
	\red
	& \ldots & \nonumber\\
	\red^*
	& P | P | \ldots & \nonumber
\end{eqnarray}

Of course, this encoding, as an implementation, runs away, unfolding
$\bangp{P}$ eagerly. A lazier and more implementable replication
operator, restricted to input-guarded processes, may be obtained as follows.

\begin{eqnarray}
\bangp{\prefix{u}{v}{P}} 
	:= 
	\binpar{\lift{x}{\prefix{u}{v}{(\binpar{D(x)}{P})}}}{D(x)} \nonumber
\end{eqnarray}

\begin{remark}
  Note that the lazier definition still does not deal with summation
  or mixed summation (i.e. sums over input and output). The reader is
  invited to construct definitions of replication that deal with these
  features. 

  Further, the definitions are parameterized in a name, $x$. Can you,
  gentle reader, make a definition that eliminates this parameter and
  guarantees no accidental interaction between the replication
  machinery and the process being replicated -- i.e. no accidental
  sharing of names used by the process to get its work done and the
  name(s) used by the replication to effect copying. This latter
  revision of the definition of replication is crucial to obtaining
  the expected identity $!!P \sim !P$.
\end{remark}

\begin{remark}\label{rem:paradoxical_combinator}
  The reader familiar with the lambda calculus will have noticed the
  similarity between $D$ and the paradoxical combinator.

  [Ed. note: the existence of this seems to suggest we have to be more
  restrictive on the set of processes and names we admit if we are to
  support no-cloning.]
\end{remark}

\subsubsection{Bisimulation}

The computational dynamics gives rise to another kind of equivalence,
the equivalence of computational behavior. As previously mentioned
this is typically captured \emph{via} some form of bisimulation.

% The notion we use in this paper is weak barbed bisimulation
% \cite{milner91polyadicpi}.

The notion we use in this paper is derived from weak barbed
bisimulation \cite{milner91polyadicpi}. 

\begin{definition}
An \emph{observation relation}, $\downarrow_{\mathcal N}$, over a set
of names, $\mathcal N$, is the smallest relation satisfying the rules
below.

\infrule[Out-barb]{y \in {\mathcal N}, \; x \nameeq y}
		  {\outputp{x}{v} \downarrow_{\mathcal N} x}
\infrule[Par-barb]{\mbox{$P\downarrow_{\mathcal N} x$ or $Q\downarrow_{\mathcal N} x$}}
		  {\binpar{P}{Q} \downarrow_{\mathcal N} x}

We write $P \Downarrow_{\mathcal N} x$ if there is $Q$ such that 
$P \wred Q$ and $Q \downarrow_{\mathcal N} x$.
\end{definition}

\begin{definition}
%\label{def.bbisim}
An  ${\mathcal N}$-\emph{barbed bisimulation} over a set of names, ${\mathcal N}$, is a symmetric binary relation 
${\mathcal S}_{\mathcal N}$ between agents such that $P\rel{S}_{\mathcal N}Q$ implies:
\begin{enumerate}
\item If $P \red P'$ then $Q \wred Q'$ and $P'\rel{S}_{\mathcal N} Q'$.
\item If $P\downarrow_{\mathcal N} x$, then $Q\Downarrow_{\mathcal N} x$.
\end{enumerate}
$P$ is ${\mathcal N}$-barbed bisimilar to $Q$, written
$P \wbbisim_{\mathcal N} Q$, if $P \rel{S}_{\mathcal N} Q$ for some ${\mathcal N}$-barbed bisimulation ${\mathcal S}_{\mathcal N}$.
\end{definition}

$\mathcal{R} \subseteq \pi \times \pi$

$P \mathcal{R} Q => \forall P'. P \red P' \Rightarrow \exists Q'. Q \red Q', P' \mathcal{R} Q'$

$P \vdash x \Rightarrow Q \vdash x$

\begin{mathpar}
  \inferrule*[lab=Out-barb]{x \nameeq y}{{y}!\langle{Q}\rangle \vdash x}
  \and
  \inferrule*[lab=Par-barb]{\mbox{$P\vdash x$ or $Q\vdash x$}}{\binpar{P}{Q} \vdash x}
\end{mathpar}

\subsubsection{Contexts}

One of the principle advantages of computational calculi like the
$\pi$-calculus is a well-defined notion of context,
contextual-equivalence and a correlation between
contextual-equivalence and notions of bisimulation. The notion of
context allows the decomposition of a process into (sub-)process and
its syntactic environment, its context. Thus, a context may be
thought of as a process with a ``hole'' (written $\Box$) in it. The
application of a context $M$ to a process $P$, written $M[P]$, is
tantamount to filling the hole in $M$ with $P$. In this paper we do
not need the full weight of this theory, but do make use of the notion
of context in the proof the main theorem. 

\begin{mathpar}
  \inferrule* [lab=summation] {} {{M_{M},M_{N}} \bc \Box \;|\; x.M_{A} \;|\; M_{M}+M_{N}}
  \and
  \inferrule* [lab=agent] {} {{M_{A}} \bc (\vec{x})M_{P} \;| \; \clift{P_0,\ldots,M_{P},\ldots,P_N}}
  \and \\
  \inferrule* [lab=process] {} {{M_{P}} \bc M_{N} \;| \;P|M_{P} }
\end{mathpar} 

\begin{mathpar}
  \inferrule* [lab=sychronization] {} {M_{N} \bc \Box \;|\; x?M_{F} \;|\; x!M_{C}}
  \and
  \inferrule* [lab=abstraction] {} {{M_{F}} \bc (x)M_{P} }
  \and
  \inferrule* [lab=concretion] {} {{M_{C}} \bc \langle M_{P} \rangle }
  \and \\
  \inferrule* [lab=process] {} {{M_{P}} \bc M_{N} \;| \;P|M_{P} }
\end{mathpar}

\begin{definition}[contextual application] Given a context $M$, and
  process $P$, we define the \emph{contextual application}, $M[P] :=
  M\{P/\Box\}$. That is, the contextual application of M to P is the
  substitution of $P$ for $\Box$ in $M$.
\end{definition}

$\meaningof{-} : L \to \mathcal{P}(\pi)$

\begin{mathpar}
  \inferrule* [lab=collection] {} {\meaningof{true} = \pi, \and \meaningof{~E} = \pi \setminus \meaningof{E}, \and \meaningof{E_{1} \& E_{2}} = \meaningof{E_{1}} \cap \meaningof{E_{2}}}
\end{mathpar}

\begin{mathpar}
  \inferrule* [lab=structure] {} {\meaningof{0} = \{ P \in \pi | P \equiv 0 \}, \and \\ \meaningof{E_1 | E_2} = \{ P \in \pi | P \equiv P_{1} | P_{2}, P_{1} \in \meaningof{E_{1}}, P_{2} \in \meaningof{E_2}\} }
\end{mathpar}

\begin{mathpar}
 \inferrule* [lab=behavior] {} {\meaningof{\langle a?b \rangle E} = \{ P \in \pi | P \equiv Q | u?(y)P', \\ \and \\\\ \and \\ \;\;\; u \in \meaningof{a}, \forall z.P'\{z/y\} \in \meaningof{E\{z/b\}}\}, \and \\ \meaningof{a!E} = \{ P \in \pi | P \equiv Q | x!\langle P' \rangle, x \in \meaningof{a} P' \in \meaningof{E}\} }
\end{mathpar}

\begin{mathpar}
 \inferrule* [lab=nominal] {} {\meaningof{\quotep{E}} = \{ \quotep{P} \in \quotep{\pi} | P \in \meaningof{E} \}, \and \meaningof{\quotep{P}} = \{ \quotep{Q} \in \quotep{\pi} | P \equiv Q \} \and \\ \meaningof{@\quotep{E}} = \{ P \in \pi | P \equiv @x, x \in \meaningof{E} \}}
\end{mathpar}

\begin{eqnarray*}
  \\
  \meaningof{-} : TS \to ST
\end{eqnarray*}

\begin{eqnarray*}
  \\
  L : TS \to ST
\end{eqnarray*}

\begin{eqnarray*}
  \\
  P \models E \iff P \in \meaningof{E}
\end{eqnarray*}

\begin{eqnarray*}
  P \approx_{L} Q \iff \forall E \in L. P \models E \iff Q \models E
\end{eqnarray*}

\begin{eqnarray*}
  P \approx_{K} Q
\end{eqnarray*}

\begin{eqnarray*}
  P \approx Q
\end{eqnarray*}

$\approx_{K} = \approx = \approx_{L}$

\subsubsection{Contextual duality}

Note that contexts extend the quotation operation to a family of
operations from processes to names. Given a context, $M$, we can
define a \emph{nominal context}, $\quotep{M}$ by $\quotep{M}[P] :=
\quotep{M[P]}$. To foreshadow what is to come we observe that these
operations enjoy a duality with processes very much like the duality
between vectors and maps from vectors to scalars.

Further, because the calculus is essentially higher-order, we have a
correspondence between contexts and processes. More specifically,
given a name $x$ and a context $M$ we can construct $M^{*}_{x}$ such
that 

\begin{mathpar}
  M^{*}_{x} | \lift{x}{P} \red M[P]
\end{mathpar}

namely,

\begin{mathpar}
  M^{*}_{x} := x?(u).M[\dropn{u}]
\end{mathpar}

The dependence of $M^{*}_{x}$ on a name makes it an abstraction, 

\begin{mathpar}
  M^{*} := (x)x?(u).M[\dropn{u}]
\end{mathpar}

\subsection{Additional notation}

It will sometimes be convenient to denote the process a name
quotes. We already have the notation $x = \quotep{P}$, but it will be
convenient to introduce an alternate notation, $\procn{x}$, when we
want to emphasize the connection to the use of the name. Note that, by
virtue of name equivalence, $\quotep{\procn{x}} \nameeq x$; so, the
notation is consistent with previous definitions.

Further, because names have structure it is possible to effect
substitutions on the basis of that structure. This means we need to
upgrade our notation for substitutions, which we accomplish by
adapting comprehension notation. Thus,

\begin{mathpar}
  P\{ y / x : x \in S \}
\end{mathpar}

is interpreted to mean the process derived from P by replacing (in a
capture-avoiding manner) each occurrence of $x$ in $S$ by $y$. For example,

\begin{mathpar}
  P\{ \quotep{\procn{x}|\procn{x}} / x : x \in \freenames{P} \}
\end{mathpar}

will replace each (occurrence) of a free name $x$ in $P$ by
$\quotep{\procn{x}|\procn{x}}$.

Also, we will avail ourselves of the notation $x^{L}$ and $x^{R}$ to
denote injections of a name into disjoint copies of the name
space. There are numerous ways to accomplish this. One example can be
found in \cite{MeredithR05}. This notation overloads to vectors of
names: $\vec{x}^{\pi} := (x_{i}^{\pi} \; : \; 0 \leq i < |\vec{x}| )$ where $\pi \in \{L,R\}$.

We also use $P^{\Box} := P|\Box$.

In \cite{MeredithR05} an interpretation of the new operator is
given. It turns out that there are several possible interpretations
all enjoying the requisite algebraic properties of the operator (see
\cite{milner91polyadicpi}). We will therefore make liberal use of
$(\nu\; \vec{x})P$.

% subsection the_syntax_and_semantics_of_the_notation_system (end)   

\input{qm2pi.qmops} 

\input{qm2pi.sterngerlach} 

\input{qm2pi.metric} 

% section concurrent_process_calculi (end)

%\input{qm2pi.proofsketch}

% section proof sketch (end)

%\input{qm2pi.slviaknots} 

% section spatial logic via knots (end)

\input{qm2pi.conclusion}

% section conclusion (end)

%\input{qm2pi.dtcodes} 

% section wiring algorithm (end)

\input{qm2pi.ack} 

% section acknowledgments (end)

\newpage


\bibliographystyle{plain}   
\bibliography{../../biblios/main.bib}

\input{qm2pi.rhodetails}

\end{document}



% section front matter (end)

\section{Introduction}\label{sec:introduction} % (fold)
In this draft of the material i am going to have to dispense with the
usual writing conventions adopted in papers on these topics. i'm going
to have adopt whatever tone i need at the time i'm writing up the
calculations. Sometimes this may be very conversational; others it may
be the barest mathematical grunts; others still it may be that i have
lifted text from one of my other papers because the exposition of some
point was better said there. i hope that my readers are not unduly put
out by this decision. i'm not doing this to flout convention or be
rebellious. i find these calculations very technically challenging. To
keep everything going technically, something has to give; i have to
let go of some cognitive burden. So, the academic writing style --
with all of its trade-offs in terms of facilitating technical
communication -- is what i'm letting go of. Perhaps subsequent drafts
can be tightened and polished, but for now, i'm going to speak as if
we were sitting together in a coffee shop with a laptop, wifi and a
pad of paper and a pencil.

So, here's what i have to say. We -- you and i, comfortably ensconced
in our coffee shop and well-equipped with our tools -- can realize and
carry out the calculations of quantum mechanics over a very different
formal theory of dynamics, a formal theory of dynamics that
corresponds to a theory of concurrent computation with
\emph{reflection}. It has the advantage that the underlying theory is
already `quantized', but supports analogues all of the continuuous
operations. Strikingly, this underlying theory has recently been
connected with a notion of metric that we can show, by calculating
together, coincides with the metric induced by the inner product.

There are a lot of reasons why you might be interested in seeing
calculations of this form. Here's why i'm interested. For the past
several centuries there has been no competitor to the ``Newtonian''
account of dynamics. As a result the predominant share of accounts of
dynamical systems and situations have had to be formulated in terms of
the Newtonian machinery. i view this as an intellectually dangerous
position to occupy. Everything, despite it's intrinsic shape, turns
into a nail to be hit with this hammer. Recently, however, the theory
of computation has matured to the point where we have candidates for
theories of dynamics that offer very different perspective on
reasoning about dynamical systems and situations. Testing these
candidates against very successful accounts of dynamical situations,
like quantum mechanics, is going to give us some sense of how mature
they are and some measure of the quality of these accounts of
dynamics.

\subsection{Summary of contributions and outline of paper}

So, we're going to develop an interpretation of the operations of
quantum mechanics normally interpreted by Hilbert spaces and
operators. We're going to do this over a theory of computation. Note
that this is very different than the usual quantum computation program
which develops notions of computation over quantum mechanics. Rather,
we are developing a story that aligns with Wheeler's slogan: It from
Bit. To do this we will first provide an account of the theory of
computation at play here. Then we will dive into a calculation-driven
interpretation of the operations of quantum mechanics.

The reason we take this approach is that -- until very recently --
there hasn't been an axiomatic account of quantum mechanics. As a
result there has been no sharp delineation of the mathematical theory
supporting interpretation of the physical theory and the physical
theory, itself. So, ambient features of the maths are free to be
exploited (or supressed) without a real accounting of their physical
relevance. There is no sharp statement ``here's the physical theory''
qua \emph{theory} and ``here's the mathematical interpretation''
enabling a judgment of how faithful the interpretation is -- apart
from experimental observation. When there is an axiomatic account we
can judge how well a given mathematical formalism supports an
interpretation of the axioms, independent of
experimentation. Likewise, we can judge how well we have captured our
physical evidence and experience with our axiomatics, independent of
any specific mathematical implementation, with accidental detail that
may or may not have physical significance. 

In lieu of a fully fleshed out and vetted axiomatic account of quantum
mechanics, interpreting the operational notions in service of modeling
physical systems will have to suffice. In other words, we are not in
the business of providing a model of Hilbert spaces and operators. We
are in the business of providing a model of quantum mechanics because
we are motivated by testing our notions of dynamics against physical
theory; and, the predictive calculations of the physical theory must
serve as the best formulation -- shy of a fully fleshed out axiomatic
account -- of the physical theory itself (as they have for scientific
theories since time immemorial). Put another way, despite a
whole-hearted commitment to an It-from-Bit ontology, we are firmly
aligned with the shut-up-and-calculate camp as the best way to obtain
results either from the physical perspective or as a quality assurance
measure of our fledgling theory of dynamics.

In detail, we present a reflective process calculus. Then we develop
intuitive correspondences between the notions available in this
calculus and the usual physical notions supporting quantum mechanical
calculations. Thus, 

\begin{table}[htp]
  \center{
    \fbox{
      \begin{tabular}{c|c}
        quantum mechanics & process calculus \\
        \hline
        scalar & name \\
        state vector & process \\
        dual & contextual duals \\
        matrix & formal sums of process-context-dual pairs \\
        orthogonality & process annihilation \\
        inner product & execution-formula + quoting
      \end{tabular}
    }
  }
  \caption{QM - process calculi correspondences}
\end{table}

Then we tighten up these intuitions to operational definitions. We
employ the Dirac notation as the best proxy we can find for an
abstract syntax of the quantum mechanical notions. The definitions we
develop put us in contact with equational constraints coming from the
theory that we demonstrate the definitions and calculations satisfy.

This puts us in a position to shut up and calculate for the
Stern-Gerlach experimental set up, showing how these predictive
calculations become calculations on processes in our theory of a
reflective process calculus.

Penultimately, we demonstrate that the notion of metric coming from
the inner product coincides with the notion of metric available from
the theory of bisimulation. This demonstration gives us the right to
think of space as arising from behavior. Finally, we consider where we
might go from the new vantage point we have obtained.

% section introduction (end) 
 
% section introduction (end)

% \documentclass[12pt]{llncs}
%\documentclass{jktr}

\usepackage[pdftex]{hyperref}                   
\usepackage {listings}
\usepackage {mathpartir}
\usepackage{bcprules}
%\usepackage{listings}
                       
\usepackage{graphicx} 
%\usepackage[margins=2.5cm,nohead,nofoot]{geometry}
%\usepackage{geometry}
\usepackage{amsfonts}
\usepackage{amstext}
\usepackage{latexsym}
\usepackage{amssymb}
\usepackage{color}


%\include{myPreamble}
\include{qm2pi.local} 

%\ifpdf
%\usepackage[pdftex]{graphicx}
%\else
%\usepackage{graphicx}
%\fi

 % \ifpdf
%  \usepackage{pdfsync}
%  \if


%\title{Brief Article}
%\author{David F. Snyder}
%\author{L.G. Meredith}

%\address{Dept. of Math., Texas State University--San Marcos, San Marcos, TX 78666}
       
\pagestyle{empty}


\begin{document}

\lstset{language=[Objective]Caml,frame=shadowbox}

\input{qm2pi.front}

% section front matter (end)

\input{qm2pi.intro} 
 
% section introduction (end)

% \input{qm2pi.knotations} 

% section notation (end)

\input{qm2pi.process.calculi} 

% section concurrent_process_calculi_and_spatial_logics_ (end)
    
%\input{qm2pi.knots2pi} 

%\input{qm2pi.trefoil} 

%\input{qm2pi.mainthm} 

% subsection basic_interpretation (end)

%\input{qm2pi.rho.presentation} 
\subsection{The syntax and semantics of the notation system}\label{sub:the_syntax_and_semantics_of_the_notation_system} % (fold)

We now summarize a technical presentation of the calculus that
embodies our theory of dynamics. The typical presentation of such a
calculus follows the style of giving generators and relations on
them. The grammar, below, describing term constructors, freely
generates the set of processes, $\Proc$. This set is then quotiented
by a relation known as structural congruence and it is over this set
that the notion of dynamics is expressed. This presentation is
essentially that of \cite{MeredithR05} with the addition of
polyadicity and summation. For readability we have relegated some of
the technical subtleties to an appendix.

\subsubsection{Process grammar}\label{subsub:process_grammar}

\begin{mathpar}
  \inferrule* [lab=synchronization] {} {{M} \bc \pzero \;|\; x?F \;|\; x!C }
  \and
  \inferrule* [lab=abstraction] {} {{F} \bc (x)P}
  \and
  \inferrule* [lab=concretion] {} {{C} \bc \langle Q \rangle}
  \and
  \inferrule* [lab=process] {} {{P,Q} \bc M \;| \;P|Q \;|\; @{x}}
  \and
  \inferrule* [lab=name] {} {{x} \bc \quotep{P}}
\end{mathpar} 

Note that $\vec{x}$ (resp. $\vec{P}$) denotes a vector of names
(resp. processes) of length $|\vec{x}|$ (resp. $|\vec{P}|$). We adopt
the following useful abbreviations.

\begin{mathpar}
   x?(\vec{y}).P := x.(\vec{y})P \and  x\clift{\vec{P}} := x.\clift{\vec{P}}
   \and x!(y) := \lift{x}{\dropn{y}}
   \and \Pi_{i=0}^{n-1}P_i := P_0 | \ldots | P_{n-1}
\end{mathpar}

\subsubsection{Structural congruence}

\paragraph{Free and bound names and alpha-equivalence.} At the
core of structural equivalence is alpha-equivalence which identifies
process that are the same up to a change of variable. Formally, we
recognize the distinction between free and bound names. The free names
of a process, $\freenames{P}$, may be calculated recursively as
follows:

\begin{mathpar}
\freenames{\pzero} := \emptyset
  \and \\
  \freenames{x?(y).P} := \{ x \} \cup (\freenames{P} \setminus \{ y \})
  \and 
  \freenames{x!\langle P \rangle} := \{ x \} \cup \{ P \} 
  \and \\
  \freenames{P|Q} := \freenames{P} \cup \freenames{Q}
  \and \\
  \freenames{@{x}} := \{ x \}
\end{mathpar}

$\pi$
$\quotep{\pi}$

$\freenames{-} : \pi \to \mathcal{P}(\quotep{\pi})$

\begin{eqnarray*}
  \freenames{\pzero} & := & \emptyset \\
  \freenames{x?(y).P} & := & \{ x \} \cup (\freenames{P} \setminus \{ y \}) \\
  \freenames{x!\langle P \rangle} & := & \{ x \} \cup \{ P \} \\
  \freenames{P|Q} & := & \freenames{P} \cup \freenames{Q} \\
  \freenames{\dropn{x}} & := & \{ x \}
\end{eqnarray*}

The bound names of a process, $\boundnames{P}$, are those names occurring in $P$
that are not free. For example, in $x?(y).0$, the name $x$ is free, while $y$ is bound.

\begin{mathpar}
  \inferrule* [lab=monoidal-laws] {} { P|Q \equiv Q|P \and P|0 \equiv P \and P|(Q|R) \equiv (P|Q)|R }
\end{mathpar}

\begin{mathpar}
  \inferrule* [lab=alpha-equivalence] {} { (x)P \equiv (y)P\{y/x\} \and y \not\in \freenames{P} }
\end{mathpar}

\begin{definition}
Then two processes, $P,Q$, are alpha-equivalent if $P = Q\{\vec{y}/\vec{x}\}$ for
some $\vec{x} \in \boundnames{Q},\vec{y} \in \boundnames{P}$, where $Q\{\vec{y}/\vec{x}\}$
denotes the capture-avoiding substitution of $\vec{y}$ for $\vec{x}$ in $Q$.
\end{definition}

\begin{definition}
  The {\em structural congruence} \cite{SangiorgiWalker} , $\equiv$,
  between processes is the least congruence containing
  alpha-equivalence, satisfying the abelian monoid laws
  (associativity, commutativity and $\pzero$ as identity) for parallel
  composition $|$ and for summation $+$.
\end{definition}

\subsection{Name equivalence}

We take name equivalence, written $\nameeq$, to be the smallest
equivalence relation generated by the following rules.

\begin{mathpar}
\inferrule*[lab=Quote-drop]
{ }
{ \quotep{@{x}} \nameeq x }

\inferrule*[lab=Struct-equiv]
{ P \scong Q }
{ \quotep{P} \nameeq \quotep{Q} }
\end{mathpar}

The astute reader will have noticed that the mutual recursion of names
and processes imposes a mutual recursion on alpha-equivalence and
structural equivalence via name-equivalence. Fortunately, all of this
works out pleasantly and we may calculate in the natural way, free of
concern. The reader interested in the details is referred to the
appendix \ref{appendix:rho_details}.

\subsection{Substitution}

We use $\Proc$ for the set of processes, $\QProc$ for the set of
names, and $\id{\{}\vec{y} / \vec{x} \id{\}}$ to denote partial maps,
$s : \QProc \rightarrow \QProc$. A map, $s$ lifts, uniquely, to a map
on process terms, $\widehat{s} : \Proc \rightarrow \Proc$ by the
following equations.

\begin{mathpar}
  (0) \psubstp{Q}{P} := 0 \\
  (R \juxtap S) \psubstp{Q}{P}
  :=    
  (R)\psubstp{Q}{P} \juxtap (S) \psubstp{Q}{P} \\
  (x?(y).R) \psubstp{Q}{P}    
  :=    
  (x)\substp{Q}{P} (z)\concat( (R \psubstn{z}{y}) \psubstp{Q}{P} ) \\
  (\lift{x}{R}) \psubstp{Q}{P}  
  :=
  \lift{(x)\substp{Q}{P}}{ R \psubstp{Q}{P} } \\
%   (\dropn{x})  \psubstp{Q}{P}       
%   := 
%   \left\{ 
%     \begin{array}{ccc} 
%       \dropn{\quotep{Q}} & & x \nameeq \quotep{P} \\
%       \dropn{x} & & otherwise \\
%     \end{array}
%   \right. 
  (\dropn{x})  \psubstp{Q}{P}       
  := 
  \left\{ 
    \begin{array}{ccc} 
      Q & & x \nameeq \quotep{P} \\
      \dropn{x} & & otherwise \\
    \end{array}
  \right.
\end{mathpar}
 

where

\begin{eqnarray}
  (x)\id{\{} \lpquote Q \rpquote / \lpquote P \rpquote \id{\}}            = 
  \left\{ 
    \begin{array}{ccc}
      \lpquote Q \rpquote & & x \nameeq \lpquote P \rpquote \\
      x & & otherwise \\
    \end{array}
  \right. \nonumber
\end{eqnarray}

and $z$ is chosen distinct from $\quotep{P}$, $\quotep{Q}$, the free
names in $Q$, and all the names in $R$. Our $\alpha$-equivalence will
be built in the standard way from this substitution.

\begin{remark}\label{rem:no_self_referential_names}
  One consequence of these definitions is that $\forall P. \quotep{P}
  \not\in \freenames{P}$.
\end{remark}

\subsection{ Dynamic quote: an example }

Anticipating something of what's to come, consider applying the
substitution, $\widehat{\id{\{}u / z \id{\}}}$, to the following pair
of processes, $\lift{w}{y!(z)}$ and $w[ \lpquote y!(z) \rpquote ]$.

\begin{eqnarray}
	\lift{w}{y!(z)}\widehat{\id{\{}u / z \id{\}}}
		& = &
		\lift{w}{y!(u)} \nonumber\\
	w[ \lpquote y!(z) \rpquote ] \widehat{ \id{\{}u / z \id{\}} }
		& = &
		w[ \lpquote y!(z) \rpquote ] \nonumber
\end{eqnarray}

Because the body of the process between quotes is impervious to
substitution, we get radically different answers. In fact, by
examining the first process in an input context,
e.g. $x?(z).\lift{w}{y!(z)}$, we see that the process under the lift
operator may be shaped by prefixed inputs binding a name inside it. In
this sense, the lift operator will be seen as a way to dynamically
construct processes before reifying them as names.

Finally equipped with these standard features we can present the
dynamics of the calculus.

\subsubsection{Operational semantics} 

Finally, we introduce the computational dynamics. What marks these
algebras as distinct from other more traditionally studied algebraic
structures, e.g. vector spaces or polynomial rings, is the manner in
which dynamics is captured. In traditional structures, dynamics is typically
expressed through morphisms between such structures, as in linear maps
between vector spaces or morphisms between rings. In algebras
associated with the semantics of computation, the dynamics is
expressed as part of the algebraic structure itself, through a
reduction reduction relation typically denoted by $\red$. Below, we
give a recursive presentation of this relation for the calculus used
in the encoding.

$\red \subseteq \pi \times \pi$
$\red : \pi \to \mathcal{P}(\pi)$

\begin{mathpar}
  \inferrule* [lab=Comm] { \textsf{match}( x_{src}, x_{trgt} ) } { x_{trgt}?(y)P \; | \; x_{src}!\langle {Q} \rangle \red P\{\quotep{Q}/y}\} }
  \and \\
  \inferrule* [lab=Par] {{P} \red {P}'} {{{P} | {Q}} \red {{P}' | {Q}}}
  \and
  \inferrule* [lab=Equiv]{{{P} \scong {P}'} \andalso {{P}' \red {Q}'} \andalso {{Q}' \scong {Q}}}{{P} \red {Q}}
\end{mathpar}

\begin{eqnarray*}
  match_{\equiv} (\quotep{P},\quotep{Q}) & := & P \equiv Q \\
  match_{\dagger}(\quotep{P},\quotep{Q}) & := & \forall R. P|Q \red^{*} R => R \red^{*} 0 \\
  match_{K}(\quotep{P},\quotep{Q}) & := & K \mbox{ for some context } K
\end{eqnarray*}

$u?(x)P | u!\langle Q \rangle \red P\{\quotep{Q}/x\}$

%We write $\wred$ for $\red^*$, and $P\red$ if $\exists Q $ such that $ P \red Q$.
We write $P\red$ if $\exists Q $ such that $ P \red Q$ and $P\not\red$, otherwise.

\section{Replication}

As mentioned before, it is known that replication (and hence
recursion) can be implemented in a higher-order process algebra
\cite{SangiorgiWalker}. As our first example of calculation with the
machinery thus far presented we give the construction explicitly in
the {\rhoc}.

\begin{eqnarray}
	D_{x} & := & \prefix{x}{y}{(\binpar{\outputp{x}{y}}{@{y}})} \nonumber\\
	\bangp_{x}{P} & := & \binpar{{x}!\langle{\binpar{D_{x}}{P}}\rangle}{D_{x}} \nonumber
\end{eqnarray}

\begin{eqnarray}
	\bangp_{x}{P} & & \nonumber\\
	=
	& {x}!\langle{(\prefix{x}{y}{(\outputp{x}{y} | @{y})) | P}}\rangle 
	      | \prefix{x}{y}{(\outputp{x}{y} | @{y})} & \nonumber\\
	\red
	& (\outputp{x}{y} | @{y})\substn{\quotep{(\prefix{x}{y}{(@{y} | \outputp{x}{y})) | P}}}{y} & \nonumber\\
	=
	& \outputp{x}{\quotep{(\prefix{x}{y}{(\outputp{x}{y} | @{y})) | P}}}
	  | {(\prefix{x}{y}{(\outputp{x}{y} | @{y})) | P}} & \nonumber\\
	\red
	& \ldots & \nonumber\\
	\red^*
	& P | P | \ldots & \nonumber
\end{eqnarray}

Of course, this encoding, as an implementation, runs away, unfolding
$\bangp{P}$ eagerly. A lazier and more implementable replication
operator, restricted to input-guarded processes, may be obtained as follows.

\begin{eqnarray}
\bangp{\prefix{u}{v}{P}} 
	:= 
	\binpar{\lift{x}{\prefix{u}{v}{(\binpar{D(x)}{P})}}}{D(x)} \nonumber
\end{eqnarray}

\begin{remark}
  Note that the lazier definition still does not deal with summation
  or mixed summation (i.e. sums over input and output). The reader is
  invited to construct definitions of replication that deal with these
  features. 

  Further, the definitions are parameterized in a name, $x$. Can you,
  gentle reader, make a definition that eliminates this parameter and
  guarantees no accidental interaction between the replication
  machinery and the process being replicated -- i.e. no accidental
  sharing of names used by the process to get its work done and the
  name(s) used by the replication to effect copying. This latter
  revision of the definition of replication is crucial to obtaining
  the expected identity $!!P \sim !P$.
\end{remark}

\begin{remark}\label{rem:paradoxical_combinator}
  The reader familiar with the lambda calculus will have noticed the
  similarity between $D$ and the paradoxical combinator.

  [Ed. note: the existence of this seems to suggest we have to be more
  restrictive on the set of processes and names we admit if we are to
  support no-cloning.]
\end{remark}

\subsubsection{Bisimulation}

The computational dynamics gives rise to another kind of equivalence,
the equivalence of computational behavior. As previously mentioned
this is typically captured \emph{via} some form of bisimulation.

% The notion we use in this paper is weak barbed bisimulation
% \cite{milner91polyadicpi}.

The notion we use in this paper is derived from weak barbed
bisimulation \cite{milner91polyadicpi}. 

\begin{definition}
An \emph{observation relation}, $\downarrow_{\mathcal N}$, over a set
of names, $\mathcal N$, is the smallest relation satisfying the rules
below.

\infrule[Out-barb]{y \in {\mathcal N}, \; x \nameeq y}
		  {\outputp{x}{v} \downarrow_{\mathcal N} x}
\infrule[Par-barb]{\mbox{$P\downarrow_{\mathcal N} x$ or $Q\downarrow_{\mathcal N} x$}}
		  {\binpar{P}{Q} \downarrow_{\mathcal N} x}

We write $P \Downarrow_{\mathcal N} x$ if there is $Q$ such that 
$P \wred Q$ and $Q \downarrow_{\mathcal N} x$.
\end{definition}

\begin{definition}
%\label{def.bbisim}
An  ${\mathcal N}$-\emph{barbed bisimulation} over a set of names, ${\mathcal N}$, is a symmetric binary relation 
${\mathcal S}_{\mathcal N}$ between agents such that $P\rel{S}_{\mathcal N}Q$ implies:
\begin{enumerate}
\item If $P \red P'$ then $Q \wred Q'$ and $P'\rel{S}_{\mathcal N} Q'$.
\item If $P\downarrow_{\mathcal N} x$, then $Q\Downarrow_{\mathcal N} x$.
\end{enumerate}
$P$ is ${\mathcal N}$-barbed bisimilar to $Q$, written
$P \wbbisim_{\mathcal N} Q$, if $P \rel{S}_{\mathcal N} Q$ for some ${\mathcal N}$-barbed bisimulation ${\mathcal S}_{\mathcal N}$.
\end{definition}

$\mathcal{R} \subseteq \pi \times \pi$

$P \mathcal{R} Q => \forall P'. P \red P' \Rightarrow \exists Q'. Q \red Q', P' \mathcal{R} Q'$

$P \vdash x \Rightarrow Q \vdash x$

\begin{mathpar}
  \inferrule*[lab=Out-barb]{x \nameeq y}{{y}!\langle{Q}\rangle \vdash x}
  \and
  \inferrule*[lab=Par-barb]{\mbox{$P\vdash x$ or $Q\vdash x$}}{\binpar{P}{Q} \vdash x}
\end{mathpar}

\subsubsection{Contexts}

One of the principle advantages of computational calculi like the
$\pi$-calculus is a well-defined notion of context,
contextual-equivalence and a correlation between
contextual-equivalence and notions of bisimulation. The notion of
context allows the decomposition of a process into (sub-)process and
its syntactic environment, its context. Thus, a context may be
thought of as a process with a ``hole'' (written $\Box$) in it. The
application of a context $M$ to a process $P$, written $M[P]$, is
tantamount to filling the hole in $M$ with $P$. In this paper we do
not need the full weight of this theory, but do make use of the notion
of context in the proof the main theorem. 

\begin{mathpar}
  \inferrule* [lab=summation] {} {{M_{M},M_{N}} \bc \Box \;|\; x.M_{A} \;|\; M_{M}+M_{N}}
  \and
  \inferrule* [lab=agent] {} {{M_{A}} \bc (\vec{x})M_{P} \;| \; \clift{P_0,\ldots,M_{P},\ldots,P_N}}
  \and \\
  \inferrule* [lab=process] {} {{M_{P}} \bc M_{N} \;| \;P|M_{P} }
\end{mathpar} 

\begin{mathpar}
  \inferrule* [lab=sychronization] {} {M_{N} \bc \Box \;|\; x?M_{F} \;|\; x!M_{C}}
  \and
  \inferrule* [lab=abstraction] {} {{M_{F}} \bc (x)M_{P} }
  \and
  \inferrule* [lab=concretion] {} {{M_{C}} \bc \langle M_{P} \rangle }
  \and \\
  \inferrule* [lab=process] {} {{M_{P}} \bc M_{N} \;| \;P|M_{P} }
\end{mathpar}

\begin{definition}[contextual application] Given a context $M$, and
  process $P$, we define the \emph{contextual application}, $M[P] :=
  M\{P/\Box\}$. That is, the contextual application of M to P is the
  substitution of $P$ for $\Box$ in $M$.
\end{definition}

$\meaningof{-} : L \to \mathcal{P}(\pi)$

\begin{mathpar}
  \inferrule* [lab=collection] {} {\meaningof{true} = \pi, \and \meaningof{~E} = \pi \setminus \meaningof{E}, \and \meaningof{E_{1} \& E_{2}} = \meaningof{E_{1}} \cap \meaningof{E_{2}}}
\end{mathpar}

\begin{mathpar}
  \inferrule* [lab=structure] {} {\meaningof{0} = \{ P \in \pi | P \equiv 0 \}, \and \\ \meaningof{E_1 | E_2} = \{ P \in \pi | P \equiv P_{1} | P_{2}, P_{1} \in \meaningof{E_{1}}, P_{2} \in \meaningof{E_2}\} }
\end{mathpar}

\begin{mathpar}
 \inferrule* [lab=behavior] {} {\meaningof{\langle a?b \rangle E} = \{ P \in \pi | P \equiv Q | u?(y)P', \\ \and \\\\ \and \\ \;\;\; u \in \meaningof{a}, \forall z.P'\{z/y\} \in \meaningof{E\{z/b\}}\}, \and \\ \meaningof{a!E} = \{ P \in \pi | P \equiv Q | x!\langle P' \rangle, x \in \meaningof{a} P' \in \meaningof{E}\} }
\end{mathpar}

\begin{mathpar}
 \inferrule* [lab=nominal] {} {\meaningof{\quotep{E}} = \{ \quotep{P} \in \quotep{\pi} | P \in \meaningof{E} \}, \and \meaningof{\quotep{P}} = \{ \quotep{Q} \in \quotep{\pi} | P \equiv Q \} \and \\ \meaningof{@\quotep{E}} = \{ P \in \pi | P \equiv @x, x \in \meaningof{E} \}}
\end{mathpar}

\begin{eqnarray*}
  \\
  \meaningof{-} : TS \to ST
\end{eqnarray*}

\begin{eqnarray*}
  \\
  L : TS \to ST
\end{eqnarray*}

\begin{eqnarray*}
  \\
  P \models E \iff P \in \meaningof{E}
\end{eqnarray*}

\begin{eqnarray*}
  P \approx_{L} Q \iff \forall E \in L. P \models E \iff Q \models E
\end{eqnarray*}

\begin{eqnarray*}
  P \approx_{K} Q
\end{eqnarray*}

\begin{eqnarray*}
  P \approx Q
\end{eqnarray*}

$\approx_{K} = \approx = \approx_{L}$

\subsubsection{Contextual duality}

Note that contexts extend the quotation operation to a family of
operations from processes to names. Given a context, $M$, we can
define a \emph{nominal context}, $\quotep{M}$ by $\quotep{M}[P] :=
\quotep{M[P]}$. To foreshadow what is to come we observe that these
operations enjoy a duality with processes very much like the duality
between vectors and maps from vectors to scalars.

Further, because the calculus is essentially higher-order, we have a
correspondence between contexts and processes. More specifically,
given a name $x$ and a context $M$ we can construct $M^{*}_{x}$ such
that 

\begin{mathpar}
  M^{*}_{x} | \lift{x}{P} \red M[P]
\end{mathpar}

namely,

\begin{mathpar}
  M^{*}_{x} := x?(u).M[\dropn{u}]
\end{mathpar}

The dependence of $M^{*}_{x}$ on a name makes it an abstraction, 

\begin{mathpar}
  M^{*} := (x)x?(u).M[\dropn{u}]
\end{mathpar}

\subsection{Additional notation}

It will sometimes be convenient to denote the process a name
quotes. We already have the notation $x = \quotep{P}$, but it will be
convenient to introduce an alternate notation, $\procn{x}$, when we
want to emphasize the connection to the use of the name. Note that, by
virtue of name equivalence, $\quotep{\procn{x}} \nameeq x$; so, the
notation is consistent with previous definitions.

Further, because names have structure it is possible to effect
substitutions on the basis of that structure. This means we need to
upgrade our notation for substitutions, which we accomplish by
adapting comprehension notation. Thus,

\begin{mathpar}
  P\{ y / x : x \in S \}
\end{mathpar}

is interpreted to mean the process derived from P by replacing (in a
capture-avoiding manner) each occurrence of $x$ in $S$ by $y$. For example,

\begin{mathpar}
  P\{ \quotep{\procn{x}|\procn{x}} / x : x \in \freenames{P} \}
\end{mathpar}

will replace each (occurrence) of a free name $x$ in $P$ by
$\quotep{\procn{x}|\procn{x}}$.

Also, we will avail ourselves of the notation $x^{L}$ and $x^{R}$ to
denote injections of a name into disjoint copies of the name
space. There are numerous ways to accomplish this. One example can be
found in \cite{MeredithR05}. This notation overloads to vectors of
names: $\vec{x}^{\pi} := (x_{i}^{\pi} \; : \; 0 \leq i < |\vec{x}| )$ where $\pi \in \{L,R\}$.

We also use $P^{\Box} := P|\Box$.

In \cite{MeredithR05} an interpretation of the new operator is
given. It turns out that there are several possible interpretations
all enjoying the requisite algebraic properties of the operator (see
\cite{milner91polyadicpi}). We will therefore make liberal use of
$(\nu\; \vec{x})P$.

% subsection the_syntax_and_semantics_of_the_notation_system (end)   

\input{qm2pi.qmops} 

\input{qm2pi.sterngerlach} 

\input{qm2pi.metric} 

% section concurrent_process_calculi (end)

%\input{qm2pi.proofsketch}

% section proof sketch (end)

%\input{qm2pi.slviaknots} 

% section spatial logic via knots (end)

\input{qm2pi.conclusion}

% section conclusion (end)

%\input{qm2pi.dtcodes} 

% section wiring algorithm (end)

\input{qm2pi.ack} 

% section acknowledgments (end)

\newpage


\bibliographystyle{plain}   
\bibliography{../../biblios/main.bib}

\input{qm2pi.rhodetails}

\end{document}

 

% section notation (end)

\input{qm2pi.process.calculi} 

% section concurrent_process_calculi_and_spatial_logics_ (end)
    
%\documentclass[12pt]{llncs}
%\documentclass{jktr}

\usepackage[pdftex]{hyperref}                   
\usepackage {listings}
\usepackage {mathpartir}
\usepackage{bcprules}
%\usepackage{listings}
                       
\usepackage{graphicx} 
%\usepackage[margins=2.5cm,nohead,nofoot]{geometry}
%\usepackage{geometry}
\usepackage{amsfonts}
\usepackage{amstext}
\usepackage{latexsym}
\usepackage{amssymb}
\usepackage{color}


%\include{myPreamble}
\include{qm2pi.local} 

%\ifpdf
%\usepackage[pdftex]{graphicx}
%\else
%\usepackage{graphicx}
%\fi

 % \ifpdf
%  \usepackage{pdfsync}
%  \if


%\title{Brief Article}
%\author{David F. Snyder}
%\author{L.G. Meredith}

%\address{Dept. of Math., Texas State University--San Marcos, San Marcos, TX 78666}
       
\pagestyle{empty}


\begin{document}

\lstset{language=[Objective]Caml,frame=shadowbox}

\input{qm2pi.front}

% section front matter (end)

\input{qm2pi.intro} 
 
% section introduction (end)

% \input{qm2pi.knotations} 

% section notation (end)

\input{qm2pi.process.calculi} 

% section concurrent_process_calculi_and_spatial_logics_ (end)
    
%\input{qm2pi.knots2pi} 

%\input{qm2pi.trefoil} 

%\input{qm2pi.mainthm} 

% subsection basic_interpretation (end)

%\input{qm2pi.rho.presentation} 
\subsection{The syntax and semantics of the notation system}\label{sub:the_syntax_and_semantics_of_the_notation_system} % (fold)

We now summarize a technical presentation of the calculus that
embodies our theory of dynamics. The typical presentation of such a
calculus follows the style of giving generators and relations on
them. The grammar, below, describing term constructors, freely
generates the set of processes, $\Proc$. This set is then quotiented
by a relation known as structural congruence and it is over this set
that the notion of dynamics is expressed. This presentation is
essentially that of \cite{MeredithR05} with the addition of
polyadicity and summation. For readability we have relegated some of
the technical subtleties to an appendix.

\subsubsection{Process grammar}\label{subsub:process_grammar}

\begin{mathpar}
  \inferrule* [lab=synchronization] {} {{M} \bc \pzero \;|\; x?F \;|\; x!C }
  \and
  \inferrule* [lab=abstraction] {} {{F} \bc (x)P}
  \and
  \inferrule* [lab=concretion] {} {{C} \bc \langle Q \rangle}
  \and
  \inferrule* [lab=process] {} {{P,Q} \bc M \;| \;P|Q \;|\; @{x}}
  \and
  \inferrule* [lab=name] {} {{x} \bc \quotep{P}}
\end{mathpar} 

Note that $\vec{x}$ (resp. $\vec{P}$) denotes a vector of names
(resp. processes) of length $|\vec{x}|$ (resp. $|\vec{P}|$). We adopt
the following useful abbreviations.

\begin{mathpar}
   x?(\vec{y}).P := x.(\vec{y})P \and  x\clift{\vec{P}} := x.\clift{\vec{P}}
   \and x!(y) := \lift{x}{\dropn{y}}
   \and \Pi_{i=0}^{n-1}P_i := P_0 | \ldots | P_{n-1}
\end{mathpar}

\subsubsection{Structural congruence}

\paragraph{Free and bound names and alpha-equivalence.} At the
core of structural equivalence is alpha-equivalence which identifies
process that are the same up to a change of variable. Formally, we
recognize the distinction between free and bound names. The free names
of a process, $\freenames{P}$, may be calculated recursively as
follows:

\begin{mathpar}
\freenames{\pzero} := \emptyset
  \and \\
  \freenames{x?(y).P} := \{ x \} \cup (\freenames{P} \setminus \{ y \})
  \and 
  \freenames{x!\langle P \rangle} := \{ x \} \cup \{ P \} 
  \and \\
  \freenames{P|Q} := \freenames{P} \cup \freenames{Q}
  \and \\
  \freenames{@{x}} := \{ x \}
\end{mathpar}

$\pi$
$\quotep{\pi}$

$\freenames{-} : \pi \to \mathcal{P}(\quotep{\pi})$

\begin{eqnarray*}
  \freenames{\pzero} & := & \emptyset \\
  \freenames{x?(y).P} & := & \{ x \} \cup (\freenames{P} \setminus \{ y \}) \\
  \freenames{x!\langle P \rangle} & := & \{ x \} \cup \{ P \} \\
  \freenames{P|Q} & := & \freenames{P} \cup \freenames{Q} \\
  \freenames{\dropn{x}} & := & \{ x \}
\end{eqnarray*}

The bound names of a process, $\boundnames{P}$, are those names occurring in $P$
that are not free. For example, in $x?(y).0$, the name $x$ is free, while $y$ is bound.

\begin{mathpar}
  \inferrule* [lab=monoidal-laws] {} { P|Q \equiv Q|P \and P|0 \equiv P \and P|(Q|R) \equiv (P|Q)|R }
\end{mathpar}

\begin{mathpar}
  \inferrule* [lab=alpha-equivalence] {} { (x)P \equiv (y)P\{y/x\} \and y \not\in \freenames{P} }
\end{mathpar}

\begin{definition}
Then two processes, $P,Q$, are alpha-equivalent if $P = Q\{\vec{y}/\vec{x}\}$ for
some $\vec{x} \in \boundnames{Q},\vec{y} \in \boundnames{P}$, where $Q\{\vec{y}/\vec{x}\}$
denotes the capture-avoiding substitution of $\vec{y}$ for $\vec{x}$ in $Q$.
\end{definition}

\begin{definition}
  The {\em structural congruence} \cite{SangiorgiWalker} , $\equiv$,
  between processes is the least congruence containing
  alpha-equivalence, satisfying the abelian monoid laws
  (associativity, commutativity and $\pzero$ as identity) for parallel
  composition $|$ and for summation $+$.
\end{definition}

\subsection{Name equivalence}

We take name equivalence, written $\nameeq$, to be the smallest
equivalence relation generated by the following rules.

\begin{mathpar}
\inferrule*[lab=Quote-drop]
{ }
{ \quotep{@{x}} \nameeq x }

\inferrule*[lab=Struct-equiv]
{ P \scong Q }
{ \quotep{P} \nameeq \quotep{Q} }
\end{mathpar}

The astute reader will have noticed that the mutual recursion of names
and processes imposes a mutual recursion on alpha-equivalence and
structural equivalence via name-equivalence. Fortunately, all of this
works out pleasantly and we may calculate in the natural way, free of
concern. The reader interested in the details is referred to the
appendix \ref{appendix:rho_details}.

\subsection{Substitution}

We use $\Proc$ for the set of processes, $\QProc$ for the set of
names, and $\id{\{}\vec{y} / \vec{x} \id{\}}$ to denote partial maps,
$s : \QProc \rightarrow \QProc$. A map, $s$ lifts, uniquely, to a map
on process terms, $\widehat{s} : \Proc \rightarrow \Proc$ by the
following equations.

\begin{mathpar}
  (0) \psubstp{Q}{P} := 0 \\
  (R \juxtap S) \psubstp{Q}{P}
  :=    
  (R)\psubstp{Q}{P} \juxtap (S) \psubstp{Q}{P} \\
  (x?(y).R) \psubstp{Q}{P}    
  :=    
  (x)\substp{Q}{P} (z)\concat( (R \psubstn{z}{y}) \psubstp{Q}{P} ) \\
  (\lift{x}{R}) \psubstp{Q}{P}  
  :=
  \lift{(x)\substp{Q}{P}}{ R \psubstp{Q}{P} } \\
%   (\dropn{x})  \psubstp{Q}{P}       
%   := 
%   \left\{ 
%     \begin{array}{ccc} 
%       \dropn{\quotep{Q}} & & x \nameeq \quotep{P} \\
%       \dropn{x} & & otherwise \\
%     \end{array}
%   \right. 
  (\dropn{x})  \psubstp{Q}{P}       
  := 
  \left\{ 
    \begin{array}{ccc} 
      Q & & x \nameeq \quotep{P} \\
      \dropn{x} & & otherwise \\
    \end{array}
  \right.
\end{mathpar}
 

where

\begin{eqnarray}
  (x)\id{\{} \lpquote Q \rpquote / \lpquote P \rpquote \id{\}}            = 
  \left\{ 
    \begin{array}{ccc}
      \lpquote Q \rpquote & & x \nameeq \lpquote P \rpquote \\
      x & & otherwise \\
    \end{array}
  \right. \nonumber
\end{eqnarray}

and $z$ is chosen distinct from $\quotep{P}$, $\quotep{Q}$, the free
names in $Q$, and all the names in $R$. Our $\alpha$-equivalence will
be built in the standard way from this substitution.

\begin{remark}\label{rem:no_self_referential_names}
  One consequence of these definitions is that $\forall P. \quotep{P}
  \not\in \freenames{P}$.
\end{remark}

\subsection{ Dynamic quote: an example }

Anticipating something of what's to come, consider applying the
substitution, $\widehat{\id{\{}u / z \id{\}}}$, to the following pair
of processes, $\lift{w}{y!(z)}$ and $w[ \lpquote y!(z) \rpquote ]$.

\begin{eqnarray}
	\lift{w}{y!(z)}\widehat{\id{\{}u / z \id{\}}}
		& = &
		\lift{w}{y!(u)} \nonumber\\
	w[ \lpquote y!(z) \rpquote ] \widehat{ \id{\{}u / z \id{\}} }
		& = &
		w[ \lpquote y!(z) \rpquote ] \nonumber
\end{eqnarray}

Because the body of the process between quotes is impervious to
substitution, we get radically different answers. In fact, by
examining the first process in an input context,
e.g. $x?(z).\lift{w}{y!(z)}$, we see that the process under the lift
operator may be shaped by prefixed inputs binding a name inside it. In
this sense, the lift operator will be seen as a way to dynamically
construct processes before reifying them as names.

Finally equipped with these standard features we can present the
dynamics of the calculus.

\subsubsection{Operational semantics} 

Finally, we introduce the computational dynamics. What marks these
algebras as distinct from other more traditionally studied algebraic
structures, e.g. vector spaces or polynomial rings, is the manner in
which dynamics is captured. In traditional structures, dynamics is typically
expressed through morphisms between such structures, as in linear maps
between vector spaces or morphisms between rings. In algebras
associated with the semantics of computation, the dynamics is
expressed as part of the algebraic structure itself, through a
reduction reduction relation typically denoted by $\red$. Below, we
give a recursive presentation of this relation for the calculus used
in the encoding.

$\red \subseteq \pi \times \pi$
$\red : \pi \to \mathcal{P}(\pi)$

\begin{mathpar}
  \inferrule* [lab=Comm] { \textsf{match}( x_{src}, x_{trgt} ) } { x_{trgt}?(y)P \; | \; x_{src}!\langle {Q} \rangle \red P\{\quotep{Q}/y}\} }
  \and \\
  \inferrule* [lab=Par] {{P} \red {P}'} {{{P} | {Q}} \red {{P}' | {Q}}}
  \and
  \inferrule* [lab=Equiv]{{{P} \scong {P}'} \andalso {{P}' \red {Q}'} \andalso {{Q}' \scong {Q}}}{{P} \red {Q}}
\end{mathpar}

\begin{eqnarray*}
  match_{\equiv} (\quotep{P},\quotep{Q}) & := & P \equiv Q \\
  match_{\dagger}(\quotep{P},\quotep{Q}) & := & \forall R. P|Q \red^{*} R => R \red^{*} 0 \\
  match_{K}(\quotep{P},\quotep{Q}) & := & K \mbox{ for some context } K
\end{eqnarray*}

$u?(x)P | u!\langle Q \rangle \red P\{\quotep{Q}/x\}$

%We write $\wred$ for $\red^*$, and $P\red$ if $\exists Q $ such that $ P \red Q$.
We write $P\red$ if $\exists Q $ such that $ P \red Q$ and $P\not\red$, otherwise.

\section{Replication}

As mentioned before, it is known that replication (and hence
recursion) can be implemented in a higher-order process algebra
\cite{SangiorgiWalker}. As our first example of calculation with the
machinery thus far presented we give the construction explicitly in
the {\rhoc}.

\begin{eqnarray}
	D_{x} & := & \prefix{x}{y}{(\binpar{\outputp{x}{y}}{@{y}})} \nonumber\\
	\bangp_{x}{P} & := & \binpar{{x}!\langle{\binpar{D_{x}}{P}}\rangle}{D_{x}} \nonumber
\end{eqnarray}

\begin{eqnarray}
	\bangp_{x}{P} & & \nonumber\\
	=
	& {x}!\langle{(\prefix{x}{y}{(\outputp{x}{y} | @{y})) | P}}\rangle 
	      | \prefix{x}{y}{(\outputp{x}{y} | @{y})} & \nonumber\\
	\red
	& (\outputp{x}{y} | @{y})\substn{\quotep{(\prefix{x}{y}{(@{y} | \outputp{x}{y})) | P}}}{y} & \nonumber\\
	=
	& \outputp{x}{\quotep{(\prefix{x}{y}{(\outputp{x}{y} | @{y})) | P}}}
	  | {(\prefix{x}{y}{(\outputp{x}{y} | @{y})) | P}} & \nonumber\\
	\red
	& \ldots & \nonumber\\
	\red^*
	& P | P | \ldots & \nonumber
\end{eqnarray}

Of course, this encoding, as an implementation, runs away, unfolding
$\bangp{P}$ eagerly. A lazier and more implementable replication
operator, restricted to input-guarded processes, may be obtained as follows.

\begin{eqnarray}
\bangp{\prefix{u}{v}{P}} 
	:= 
	\binpar{\lift{x}{\prefix{u}{v}{(\binpar{D(x)}{P})}}}{D(x)} \nonumber
\end{eqnarray}

\begin{remark}
  Note that the lazier definition still does not deal with summation
  or mixed summation (i.e. sums over input and output). The reader is
  invited to construct definitions of replication that deal with these
  features. 

  Further, the definitions are parameterized in a name, $x$. Can you,
  gentle reader, make a definition that eliminates this parameter and
  guarantees no accidental interaction between the replication
  machinery and the process being replicated -- i.e. no accidental
  sharing of names used by the process to get its work done and the
  name(s) used by the replication to effect copying. This latter
  revision of the definition of replication is crucial to obtaining
  the expected identity $!!P \sim !P$.
\end{remark}

\begin{remark}\label{rem:paradoxical_combinator}
  The reader familiar with the lambda calculus will have noticed the
  similarity between $D$ and the paradoxical combinator.

  [Ed. note: the existence of this seems to suggest we have to be more
  restrictive on the set of processes and names we admit if we are to
  support no-cloning.]
\end{remark}

\subsubsection{Bisimulation}

The computational dynamics gives rise to another kind of equivalence,
the equivalence of computational behavior. As previously mentioned
this is typically captured \emph{via} some form of bisimulation.

% The notion we use in this paper is weak barbed bisimulation
% \cite{milner91polyadicpi}.

The notion we use in this paper is derived from weak barbed
bisimulation \cite{milner91polyadicpi}. 

\begin{definition}
An \emph{observation relation}, $\downarrow_{\mathcal N}$, over a set
of names, $\mathcal N$, is the smallest relation satisfying the rules
below.

\infrule[Out-barb]{y \in {\mathcal N}, \; x \nameeq y}
		  {\outputp{x}{v} \downarrow_{\mathcal N} x}
\infrule[Par-barb]{\mbox{$P\downarrow_{\mathcal N} x$ or $Q\downarrow_{\mathcal N} x$}}
		  {\binpar{P}{Q} \downarrow_{\mathcal N} x}

We write $P \Downarrow_{\mathcal N} x$ if there is $Q$ such that 
$P \wred Q$ and $Q \downarrow_{\mathcal N} x$.
\end{definition}

\begin{definition}
%\label{def.bbisim}
An  ${\mathcal N}$-\emph{barbed bisimulation} over a set of names, ${\mathcal N}$, is a symmetric binary relation 
${\mathcal S}_{\mathcal N}$ between agents such that $P\rel{S}_{\mathcal N}Q$ implies:
\begin{enumerate}
\item If $P \red P'$ then $Q \wred Q'$ and $P'\rel{S}_{\mathcal N} Q'$.
\item If $P\downarrow_{\mathcal N} x$, then $Q\Downarrow_{\mathcal N} x$.
\end{enumerate}
$P$ is ${\mathcal N}$-barbed bisimilar to $Q$, written
$P \wbbisim_{\mathcal N} Q$, if $P \rel{S}_{\mathcal N} Q$ for some ${\mathcal N}$-barbed bisimulation ${\mathcal S}_{\mathcal N}$.
\end{definition}

$\mathcal{R} \subseteq \pi \times \pi$

$P \mathcal{R} Q => \forall P'. P \red P' \Rightarrow \exists Q'. Q \red Q', P' \mathcal{R} Q'$

$P \vdash x \Rightarrow Q \vdash x$

\begin{mathpar}
  \inferrule*[lab=Out-barb]{x \nameeq y}{{y}!\langle{Q}\rangle \vdash x}
  \and
  \inferrule*[lab=Par-barb]{\mbox{$P\vdash x$ or $Q\vdash x$}}{\binpar{P}{Q} \vdash x}
\end{mathpar}

\subsubsection{Contexts}

One of the principle advantages of computational calculi like the
$\pi$-calculus is a well-defined notion of context,
contextual-equivalence and a correlation between
contextual-equivalence and notions of bisimulation. The notion of
context allows the decomposition of a process into (sub-)process and
its syntactic environment, its context. Thus, a context may be
thought of as a process with a ``hole'' (written $\Box$) in it. The
application of a context $M$ to a process $P$, written $M[P]$, is
tantamount to filling the hole in $M$ with $P$. In this paper we do
not need the full weight of this theory, but do make use of the notion
of context in the proof the main theorem. 

\begin{mathpar}
  \inferrule* [lab=summation] {} {{M_{M},M_{N}} \bc \Box \;|\; x.M_{A} \;|\; M_{M}+M_{N}}
  \and
  \inferrule* [lab=agent] {} {{M_{A}} \bc (\vec{x})M_{P} \;| \; \clift{P_0,\ldots,M_{P},\ldots,P_N}}
  \and \\
  \inferrule* [lab=process] {} {{M_{P}} \bc M_{N} \;| \;P|M_{P} }
\end{mathpar} 

\begin{mathpar}
  \inferrule* [lab=sychronization] {} {M_{N} \bc \Box \;|\; x?M_{F} \;|\; x!M_{C}}
  \and
  \inferrule* [lab=abstraction] {} {{M_{F}} \bc (x)M_{P} }
  \and
  \inferrule* [lab=concretion] {} {{M_{C}} \bc \langle M_{P} \rangle }
  \and \\
  \inferrule* [lab=process] {} {{M_{P}} \bc M_{N} \;| \;P|M_{P} }
\end{mathpar}

\begin{definition}[contextual application] Given a context $M$, and
  process $P$, we define the \emph{contextual application}, $M[P] :=
  M\{P/\Box\}$. That is, the contextual application of M to P is the
  substitution of $P$ for $\Box$ in $M$.
\end{definition}

$\meaningof{-} : L \to \mathcal{P}(\pi)$

\begin{mathpar}
  \inferrule* [lab=collection] {} {\meaningof{true} = \pi, \and \meaningof{~E} = \pi \setminus \meaningof{E}, \and \meaningof{E_{1} \& E_{2}} = \meaningof{E_{1}} \cap \meaningof{E_{2}}}
\end{mathpar}

\begin{mathpar}
  \inferrule* [lab=structure] {} {\meaningof{0} = \{ P \in \pi | P \equiv 0 \}, \and \\ \meaningof{E_1 | E_2} = \{ P \in \pi | P \equiv P_{1} | P_{2}, P_{1} \in \meaningof{E_{1}}, P_{2} \in \meaningof{E_2}\} }
\end{mathpar}

\begin{mathpar}
 \inferrule* [lab=behavior] {} {\meaningof{\langle a?b \rangle E} = \{ P \in \pi | P \equiv Q | u?(y)P', \\ \and \\\\ \and \\ \;\;\; u \in \meaningof{a}, \forall z.P'\{z/y\} \in \meaningof{E\{z/b\}}\}, \and \\ \meaningof{a!E} = \{ P \in \pi | P \equiv Q | x!\langle P' \rangle, x \in \meaningof{a} P' \in \meaningof{E}\} }
\end{mathpar}

\begin{mathpar}
 \inferrule* [lab=nominal] {} {\meaningof{\quotep{E}} = \{ \quotep{P} \in \quotep{\pi} | P \in \meaningof{E} \}, \and \meaningof{\quotep{P}} = \{ \quotep{Q} \in \quotep{\pi} | P \equiv Q \} \and \\ \meaningof{@\quotep{E}} = \{ P \in \pi | P \equiv @x, x \in \meaningof{E} \}}
\end{mathpar}

\begin{eqnarray*}
  \\
  \meaningof{-} : TS \to ST
\end{eqnarray*}

\begin{eqnarray*}
  \\
  L : TS \to ST
\end{eqnarray*}

\begin{eqnarray*}
  \\
  P \models E \iff P \in \meaningof{E}
\end{eqnarray*}

\begin{eqnarray*}
  P \approx_{L} Q \iff \forall E \in L. P \models E \iff Q \models E
\end{eqnarray*}

\begin{eqnarray*}
  P \approx_{K} Q
\end{eqnarray*}

\begin{eqnarray*}
  P \approx Q
\end{eqnarray*}

$\approx_{K} = \approx = \approx_{L}$

\subsubsection{Contextual duality}

Note that contexts extend the quotation operation to a family of
operations from processes to names. Given a context, $M$, we can
define a \emph{nominal context}, $\quotep{M}$ by $\quotep{M}[P] :=
\quotep{M[P]}$. To foreshadow what is to come we observe that these
operations enjoy a duality with processes very much like the duality
between vectors and maps from vectors to scalars.

Further, because the calculus is essentially higher-order, we have a
correspondence between contexts and processes. More specifically,
given a name $x$ and a context $M$ we can construct $M^{*}_{x}$ such
that 

\begin{mathpar}
  M^{*}_{x} | \lift{x}{P} \red M[P]
\end{mathpar}

namely,

\begin{mathpar}
  M^{*}_{x} := x?(u).M[\dropn{u}]
\end{mathpar}

The dependence of $M^{*}_{x}$ on a name makes it an abstraction, 

\begin{mathpar}
  M^{*} := (x)x?(u).M[\dropn{u}]
\end{mathpar}

\subsection{Additional notation}

It will sometimes be convenient to denote the process a name
quotes. We already have the notation $x = \quotep{P}$, but it will be
convenient to introduce an alternate notation, $\procn{x}$, when we
want to emphasize the connection to the use of the name. Note that, by
virtue of name equivalence, $\quotep{\procn{x}} \nameeq x$; so, the
notation is consistent with previous definitions.

Further, because names have structure it is possible to effect
substitutions on the basis of that structure. This means we need to
upgrade our notation for substitutions, which we accomplish by
adapting comprehension notation. Thus,

\begin{mathpar}
  P\{ y / x : x \in S \}
\end{mathpar}

is interpreted to mean the process derived from P by replacing (in a
capture-avoiding manner) each occurrence of $x$ in $S$ by $y$. For example,

\begin{mathpar}
  P\{ \quotep{\procn{x}|\procn{x}} / x : x \in \freenames{P} \}
\end{mathpar}

will replace each (occurrence) of a free name $x$ in $P$ by
$\quotep{\procn{x}|\procn{x}}$.

Also, we will avail ourselves of the notation $x^{L}$ and $x^{R}$ to
denote injections of a name into disjoint copies of the name
space. There are numerous ways to accomplish this. One example can be
found in \cite{MeredithR05}. This notation overloads to vectors of
names: $\vec{x}^{\pi} := (x_{i}^{\pi} \; : \; 0 \leq i < |\vec{x}| )$ where $\pi \in \{L,R\}$.

We also use $P^{\Box} := P|\Box$.

In \cite{MeredithR05} an interpretation of the new operator is
given. It turns out that there are several possible interpretations
all enjoying the requisite algebraic properties of the operator (see
\cite{milner91polyadicpi}). We will therefore make liberal use of
$(\nu\; \vec{x})P$.

% subsection the_syntax_and_semantics_of_the_notation_system (end)   

\input{qm2pi.qmops} 

\input{qm2pi.sterngerlach} 

\input{qm2pi.metric} 

% section concurrent_process_calculi (end)

%\input{qm2pi.proofsketch}

% section proof sketch (end)

%\input{qm2pi.slviaknots} 

% section spatial logic via knots (end)

\input{qm2pi.conclusion}

% section conclusion (end)

%\input{qm2pi.dtcodes} 

% section wiring algorithm (end)

\input{qm2pi.ack} 

% section acknowledgments (end)

\newpage


\bibliographystyle{plain}   
\bibliography{../../biblios/main.bib}

\input{qm2pi.rhodetails}

\end{document}

 

%\documentclass[12pt]{llncs}
%\documentclass{jktr}

\usepackage[pdftex]{hyperref}                   
\usepackage {listings}
\usepackage {mathpartir}
\usepackage{bcprules}
%\usepackage{listings}
                       
\usepackage{graphicx} 
%\usepackage[margins=2.5cm,nohead,nofoot]{geometry}
%\usepackage{geometry}
\usepackage{amsfonts}
\usepackage{amstext}
\usepackage{latexsym}
\usepackage{amssymb}
\usepackage{color}


%\include{myPreamble}
\include{qm2pi.local} 

%\ifpdf
%\usepackage[pdftex]{graphicx}
%\else
%\usepackage{graphicx}
%\fi

 % \ifpdf
%  \usepackage{pdfsync}
%  \if


%\title{Brief Article}
%\author{David F. Snyder}
%\author{L.G. Meredith}

%\address{Dept. of Math., Texas State University--San Marcos, San Marcos, TX 78666}
       
\pagestyle{empty}


\begin{document}

\lstset{language=[Objective]Caml,frame=shadowbox}

\input{qm2pi.front}

% section front matter (end)

\input{qm2pi.intro} 
 
% section introduction (end)

% \input{qm2pi.knotations} 

% section notation (end)

\input{qm2pi.process.calculi} 

% section concurrent_process_calculi_and_spatial_logics_ (end)
    
%\input{qm2pi.knots2pi} 

%\input{qm2pi.trefoil} 

%\input{qm2pi.mainthm} 

% subsection basic_interpretation (end)

%\input{qm2pi.rho.presentation} 
\subsection{The syntax and semantics of the notation system}\label{sub:the_syntax_and_semantics_of_the_notation_system} % (fold)

We now summarize a technical presentation of the calculus that
embodies our theory of dynamics. The typical presentation of such a
calculus follows the style of giving generators and relations on
them. The grammar, below, describing term constructors, freely
generates the set of processes, $\Proc$. This set is then quotiented
by a relation known as structural congruence and it is over this set
that the notion of dynamics is expressed. This presentation is
essentially that of \cite{MeredithR05} with the addition of
polyadicity and summation. For readability we have relegated some of
the technical subtleties to an appendix.

\subsubsection{Process grammar}\label{subsub:process_grammar}

\begin{mathpar}
  \inferrule* [lab=synchronization] {} {{M} \bc \pzero \;|\; x?F \;|\; x!C }
  \and
  \inferrule* [lab=abstraction] {} {{F} \bc (x)P}
  \and
  \inferrule* [lab=concretion] {} {{C} \bc \langle Q \rangle}
  \and
  \inferrule* [lab=process] {} {{P,Q} \bc M \;| \;P|Q \;|\; @{x}}
  \and
  \inferrule* [lab=name] {} {{x} \bc \quotep{P}}
\end{mathpar} 

Note that $\vec{x}$ (resp. $\vec{P}$) denotes a vector of names
(resp. processes) of length $|\vec{x}|$ (resp. $|\vec{P}|$). We adopt
the following useful abbreviations.

\begin{mathpar}
   x?(\vec{y}).P := x.(\vec{y})P \and  x\clift{\vec{P}} := x.\clift{\vec{P}}
   \and x!(y) := \lift{x}{\dropn{y}}
   \and \Pi_{i=0}^{n-1}P_i := P_0 | \ldots | P_{n-1}
\end{mathpar}

\subsubsection{Structural congruence}

\paragraph{Free and bound names and alpha-equivalence.} At the
core of structural equivalence is alpha-equivalence which identifies
process that are the same up to a change of variable. Formally, we
recognize the distinction between free and bound names. The free names
of a process, $\freenames{P}$, may be calculated recursively as
follows:

\begin{mathpar}
\freenames{\pzero} := \emptyset
  \and \\
  \freenames{x?(y).P} := \{ x \} \cup (\freenames{P} \setminus \{ y \})
  \and 
  \freenames{x!\langle P \rangle} := \{ x \} \cup \{ P \} 
  \and \\
  \freenames{P|Q} := \freenames{P} \cup \freenames{Q}
  \and \\
  \freenames{@{x}} := \{ x \}
\end{mathpar}

$\pi$
$\quotep{\pi}$

$\freenames{-} : \pi \to \mathcal{P}(\quotep{\pi})$

\begin{eqnarray*}
  \freenames{\pzero} & := & \emptyset \\
  \freenames{x?(y).P} & := & \{ x \} \cup (\freenames{P} \setminus \{ y \}) \\
  \freenames{x!\langle P \rangle} & := & \{ x \} \cup \{ P \} \\
  \freenames{P|Q} & := & \freenames{P} \cup \freenames{Q} \\
  \freenames{\dropn{x}} & := & \{ x \}
\end{eqnarray*}

The bound names of a process, $\boundnames{P}$, are those names occurring in $P$
that are not free. For example, in $x?(y).0$, the name $x$ is free, while $y$ is bound.

\begin{mathpar}
  \inferrule* [lab=monoidal-laws] {} { P|Q \equiv Q|P \and P|0 \equiv P \and P|(Q|R) \equiv (P|Q)|R }
\end{mathpar}

\begin{mathpar}
  \inferrule* [lab=alpha-equivalence] {} { (x)P \equiv (y)P\{y/x\} \and y \not\in \freenames{P} }
\end{mathpar}

\begin{definition}
Then two processes, $P,Q$, are alpha-equivalent if $P = Q\{\vec{y}/\vec{x}\}$ for
some $\vec{x} \in \boundnames{Q},\vec{y} \in \boundnames{P}$, where $Q\{\vec{y}/\vec{x}\}$
denotes the capture-avoiding substitution of $\vec{y}$ for $\vec{x}$ in $Q$.
\end{definition}

\begin{definition}
  The {\em structural congruence} \cite{SangiorgiWalker} , $\equiv$,
  between processes is the least congruence containing
  alpha-equivalence, satisfying the abelian monoid laws
  (associativity, commutativity and $\pzero$ as identity) for parallel
  composition $|$ and for summation $+$.
\end{definition}

\subsection{Name equivalence}

We take name equivalence, written $\nameeq$, to be the smallest
equivalence relation generated by the following rules.

\begin{mathpar}
\inferrule*[lab=Quote-drop]
{ }
{ \quotep{@{x}} \nameeq x }

\inferrule*[lab=Struct-equiv]
{ P \scong Q }
{ \quotep{P} \nameeq \quotep{Q} }
\end{mathpar}

The astute reader will have noticed that the mutual recursion of names
and processes imposes a mutual recursion on alpha-equivalence and
structural equivalence via name-equivalence. Fortunately, all of this
works out pleasantly and we may calculate in the natural way, free of
concern. The reader interested in the details is referred to the
appendix \ref{appendix:rho_details}.

\subsection{Substitution}

We use $\Proc$ for the set of processes, $\QProc$ for the set of
names, and $\id{\{}\vec{y} / \vec{x} \id{\}}$ to denote partial maps,
$s : \QProc \rightarrow \QProc$. A map, $s$ lifts, uniquely, to a map
on process terms, $\widehat{s} : \Proc \rightarrow \Proc$ by the
following equations.

\begin{mathpar}
  (0) \psubstp{Q}{P} := 0 \\
  (R \juxtap S) \psubstp{Q}{P}
  :=    
  (R)\psubstp{Q}{P} \juxtap (S) \psubstp{Q}{P} \\
  (x?(y).R) \psubstp{Q}{P}    
  :=    
  (x)\substp{Q}{P} (z)\concat( (R \psubstn{z}{y}) \psubstp{Q}{P} ) \\
  (\lift{x}{R}) \psubstp{Q}{P}  
  :=
  \lift{(x)\substp{Q}{P}}{ R \psubstp{Q}{P} } \\
%   (\dropn{x})  \psubstp{Q}{P}       
%   := 
%   \left\{ 
%     \begin{array}{ccc} 
%       \dropn{\quotep{Q}} & & x \nameeq \quotep{P} \\
%       \dropn{x} & & otherwise \\
%     \end{array}
%   \right. 
  (\dropn{x})  \psubstp{Q}{P}       
  := 
  \left\{ 
    \begin{array}{ccc} 
      Q & & x \nameeq \quotep{P} \\
      \dropn{x} & & otherwise \\
    \end{array}
  \right.
\end{mathpar}
 

where

\begin{eqnarray}
  (x)\id{\{} \lpquote Q \rpquote / \lpquote P \rpquote \id{\}}            = 
  \left\{ 
    \begin{array}{ccc}
      \lpquote Q \rpquote & & x \nameeq \lpquote P \rpquote \\
      x & & otherwise \\
    \end{array}
  \right. \nonumber
\end{eqnarray}

and $z$ is chosen distinct from $\quotep{P}$, $\quotep{Q}$, the free
names in $Q$, and all the names in $R$. Our $\alpha$-equivalence will
be built in the standard way from this substitution.

\begin{remark}\label{rem:no_self_referential_names}
  One consequence of these definitions is that $\forall P. \quotep{P}
  \not\in \freenames{P}$.
\end{remark}

\subsection{ Dynamic quote: an example }

Anticipating something of what's to come, consider applying the
substitution, $\widehat{\id{\{}u / z \id{\}}}$, to the following pair
of processes, $\lift{w}{y!(z)}$ and $w[ \lpquote y!(z) \rpquote ]$.

\begin{eqnarray}
	\lift{w}{y!(z)}\widehat{\id{\{}u / z \id{\}}}
		& = &
		\lift{w}{y!(u)} \nonumber\\
	w[ \lpquote y!(z) \rpquote ] \widehat{ \id{\{}u / z \id{\}} }
		& = &
		w[ \lpquote y!(z) \rpquote ] \nonumber
\end{eqnarray}

Because the body of the process between quotes is impervious to
substitution, we get radically different answers. In fact, by
examining the first process in an input context,
e.g. $x?(z).\lift{w}{y!(z)}$, we see that the process under the lift
operator may be shaped by prefixed inputs binding a name inside it. In
this sense, the lift operator will be seen as a way to dynamically
construct processes before reifying them as names.

Finally equipped with these standard features we can present the
dynamics of the calculus.

\subsubsection{Operational semantics} 

Finally, we introduce the computational dynamics. What marks these
algebras as distinct from other more traditionally studied algebraic
structures, e.g. vector spaces or polynomial rings, is the manner in
which dynamics is captured. In traditional structures, dynamics is typically
expressed through morphisms between such structures, as in linear maps
between vector spaces or morphisms between rings. In algebras
associated with the semantics of computation, the dynamics is
expressed as part of the algebraic structure itself, through a
reduction reduction relation typically denoted by $\red$. Below, we
give a recursive presentation of this relation for the calculus used
in the encoding.

$\red \subseteq \pi \times \pi$
$\red : \pi \to \mathcal{P}(\pi)$

\begin{mathpar}
  \inferrule* [lab=Comm] { \textsf{match}( x_{src}, x_{trgt} ) } { x_{trgt}?(y)P \; | \; x_{src}!\langle {Q} \rangle \red P\{\quotep{Q}/y}\} }
  \and \\
  \inferrule* [lab=Par] {{P} \red {P}'} {{{P} | {Q}} \red {{P}' | {Q}}}
  \and
  \inferrule* [lab=Equiv]{{{P} \scong {P}'} \andalso {{P}' \red {Q}'} \andalso {{Q}' \scong {Q}}}{{P} \red {Q}}
\end{mathpar}

\begin{eqnarray*}
  match_{\equiv} (\quotep{P},\quotep{Q}) & := & P \equiv Q \\
  match_{\dagger}(\quotep{P},\quotep{Q}) & := & \forall R. P|Q \red^{*} R => R \red^{*} 0 \\
  match_{K}(\quotep{P},\quotep{Q}) & := & K \mbox{ for some context } K
\end{eqnarray*}

$u?(x)P | u!\langle Q \rangle \red P\{\quotep{Q}/x\}$

%We write $\wred$ for $\red^*$, and $P\red$ if $\exists Q $ such that $ P \red Q$.
We write $P\red$ if $\exists Q $ such that $ P \red Q$ and $P\not\red$, otherwise.

\section{Replication}

As mentioned before, it is known that replication (and hence
recursion) can be implemented in a higher-order process algebra
\cite{SangiorgiWalker}. As our first example of calculation with the
machinery thus far presented we give the construction explicitly in
the {\rhoc}.

\begin{eqnarray}
	D_{x} & := & \prefix{x}{y}{(\binpar{\outputp{x}{y}}{@{y}})} \nonumber\\
	\bangp_{x}{P} & := & \binpar{{x}!\langle{\binpar{D_{x}}{P}}\rangle}{D_{x}} \nonumber
\end{eqnarray}

\begin{eqnarray}
	\bangp_{x}{P} & & \nonumber\\
	=
	& {x}!\langle{(\prefix{x}{y}{(\outputp{x}{y} | @{y})) | P}}\rangle 
	      | \prefix{x}{y}{(\outputp{x}{y} | @{y})} & \nonumber\\
	\red
	& (\outputp{x}{y} | @{y})\substn{\quotep{(\prefix{x}{y}{(@{y} | \outputp{x}{y})) | P}}}{y} & \nonumber\\
	=
	& \outputp{x}{\quotep{(\prefix{x}{y}{(\outputp{x}{y} | @{y})) | P}}}
	  | {(\prefix{x}{y}{(\outputp{x}{y} | @{y})) | P}} & \nonumber\\
	\red
	& \ldots & \nonumber\\
	\red^*
	& P | P | \ldots & \nonumber
\end{eqnarray}

Of course, this encoding, as an implementation, runs away, unfolding
$\bangp{P}$ eagerly. A lazier and more implementable replication
operator, restricted to input-guarded processes, may be obtained as follows.

\begin{eqnarray}
\bangp{\prefix{u}{v}{P}} 
	:= 
	\binpar{\lift{x}{\prefix{u}{v}{(\binpar{D(x)}{P})}}}{D(x)} \nonumber
\end{eqnarray}

\begin{remark}
  Note that the lazier definition still does not deal with summation
  or mixed summation (i.e. sums over input and output). The reader is
  invited to construct definitions of replication that deal with these
  features. 

  Further, the definitions are parameterized in a name, $x$. Can you,
  gentle reader, make a definition that eliminates this parameter and
  guarantees no accidental interaction between the replication
  machinery and the process being replicated -- i.e. no accidental
  sharing of names used by the process to get its work done and the
  name(s) used by the replication to effect copying. This latter
  revision of the definition of replication is crucial to obtaining
  the expected identity $!!P \sim !P$.
\end{remark}

\begin{remark}\label{rem:paradoxical_combinator}
  The reader familiar with the lambda calculus will have noticed the
  similarity between $D$ and the paradoxical combinator.

  [Ed. note: the existence of this seems to suggest we have to be more
  restrictive on the set of processes and names we admit if we are to
  support no-cloning.]
\end{remark}

\subsubsection{Bisimulation}

The computational dynamics gives rise to another kind of equivalence,
the equivalence of computational behavior. As previously mentioned
this is typically captured \emph{via} some form of bisimulation.

% The notion we use in this paper is weak barbed bisimulation
% \cite{milner91polyadicpi}.

The notion we use in this paper is derived from weak barbed
bisimulation \cite{milner91polyadicpi}. 

\begin{definition}
An \emph{observation relation}, $\downarrow_{\mathcal N}$, over a set
of names, $\mathcal N$, is the smallest relation satisfying the rules
below.

\infrule[Out-barb]{y \in {\mathcal N}, \; x \nameeq y}
		  {\outputp{x}{v} \downarrow_{\mathcal N} x}
\infrule[Par-barb]{\mbox{$P\downarrow_{\mathcal N} x$ or $Q\downarrow_{\mathcal N} x$}}
		  {\binpar{P}{Q} \downarrow_{\mathcal N} x}

We write $P \Downarrow_{\mathcal N} x$ if there is $Q$ such that 
$P \wred Q$ and $Q \downarrow_{\mathcal N} x$.
\end{definition}

\begin{definition}
%\label{def.bbisim}
An  ${\mathcal N}$-\emph{barbed bisimulation} over a set of names, ${\mathcal N}$, is a symmetric binary relation 
${\mathcal S}_{\mathcal N}$ between agents such that $P\rel{S}_{\mathcal N}Q$ implies:
\begin{enumerate}
\item If $P \red P'$ then $Q \wred Q'$ and $P'\rel{S}_{\mathcal N} Q'$.
\item If $P\downarrow_{\mathcal N} x$, then $Q\Downarrow_{\mathcal N} x$.
\end{enumerate}
$P$ is ${\mathcal N}$-barbed bisimilar to $Q$, written
$P \wbbisim_{\mathcal N} Q$, if $P \rel{S}_{\mathcal N} Q$ for some ${\mathcal N}$-barbed bisimulation ${\mathcal S}_{\mathcal N}$.
\end{definition}

$\mathcal{R} \subseteq \pi \times \pi$

$P \mathcal{R} Q => \forall P'. P \red P' \Rightarrow \exists Q'. Q \red Q', P' \mathcal{R} Q'$

$P \vdash x \Rightarrow Q \vdash x$

\begin{mathpar}
  \inferrule*[lab=Out-barb]{x \nameeq y}{{y}!\langle{Q}\rangle \vdash x}
  \and
  \inferrule*[lab=Par-barb]{\mbox{$P\vdash x$ or $Q\vdash x$}}{\binpar{P}{Q} \vdash x}
\end{mathpar}

\subsubsection{Contexts}

One of the principle advantages of computational calculi like the
$\pi$-calculus is a well-defined notion of context,
contextual-equivalence and a correlation between
contextual-equivalence and notions of bisimulation. The notion of
context allows the decomposition of a process into (sub-)process and
its syntactic environment, its context. Thus, a context may be
thought of as a process with a ``hole'' (written $\Box$) in it. The
application of a context $M$ to a process $P$, written $M[P]$, is
tantamount to filling the hole in $M$ with $P$. In this paper we do
not need the full weight of this theory, but do make use of the notion
of context in the proof the main theorem. 

\begin{mathpar}
  \inferrule* [lab=summation] {} {{M_{M},M_{N}} \bc \Box \;|\; x.M_{A} \;|\; M_{M}+M_{N}}
  \and
  \inferrule* [lab=agent] {} {{M_{A}} \bc (\vec{x})M_{P} \;| \; \clift{P_0,\ldots,M_{P},\ldots,P_N}}
  \and \\
  \inferrule* [lab=process] {} {{M_{P}} \bc M_{N} \;| \;P|M_{P} }
\end{mathpar} 

\begin{mathpar}
  \inferrule* [lab=sychronization] {} {M_{N} \bc \Box \;|\; x?M_{F} \;|\; x!M_{C}}
  \and
  \inferrule* [lab=abstraction] {} {{M_{F}} \bc (x)M_{P} }
  \and
  \inferrule* [lab=concretion] {} {{M_{C}} \bc \langle M_{P} \rangle }
  \and \\
  \inferrule* [lab=process] {} {{M_{P}} \bc M_{N} \;| \;P|M_{P} }
\end{mathpar}

\begin{definition}[contextual application] Given a context $M$, and
  process $P$, we define the \emph{contextual application}, $M[P] :=
  M\{P/\Box\}$. That is, the contextual application of M to P is the
  substitution of $P$ for $\Box$ in $M$.
\end{definition}

$\meaningof{-} : L \to \mathcal{P}(\pi)$

\begin{mathpar}
  \inferrule* [lab=collection] {} {\meaningof{true} = \pi, \and \meaningof{~E} = \pi \setminus \meaningof{E}, \and \meaningof{E_{1} \& E_{2}} = \meaningof{E_{1}} \cap \meaningof{E_{2}}}
\end{mathpar}

\begin{mathpar}
  \inferrule* [lab=structure] {} {\meaningof{0} = \{ P \in \pi | P \equiv 0 \}, \and \\ \meaningof{E_1 | E_2} = \{ P \in \pi | P \equiv P_{1} | P_{2}, P_{1} \in \meaningof{E_{1}}, P_{2} \in \meaningof{E_2}\} }
\end{mathpar}

\begin{mathpar}
 \inferrule* [lab=behavior] {} {\meaningof{\langle a?b \rangle E} = \{ P \in \pi | P \equiv Q | u?(y)P', \\ \and \\\\ \and \\ \;\;\; u \in \meaningof{a}, \forall z.P'\{z/y\} \in \meaningof{E\{z/b\}}\}, \and \\ \meaningof{a!E} = \{ P \in \pi | P \equiv Q | x!\langle P' \rangle, x \in \meaningof{a} P' \in \meaningof{E}\} }
\end{mathpar}

\begin{mathpar}
 \inferrule* [lab=nominal] {} {\meaningof{\quotep{E}} = \{ \quotep{P} \in \quotep{\pi} | P \in \meaningof{E} \}, \and \meaningof{\quotep{P}} = \{ \quotep{Q} \in \quotep{\pi} | P \equiv Q \} \and \\ \meaningof{@\quotep{E}} = \{ P \in \pi | P \equiv @x, x \in \meaningof{E} \}}
\end{mathpar}

\begin{eqnarray*}
  \\
  \meaningof{-} : TS \to ST
\end{eqnarray*}

\begin{eqnarray*}
  \\
  L : TS \to ST
\end{eqnarray*}

\begin{eqnarray*}
  \\
  P \models E \iff P \in \meaningof{E}
\end{eqnarray*}

\begin{eqnarray*}
  P \approx_{L} Q \iff \forall E \in L. P \models E \iff Q \models E
\end{eqnarray*}

\begin{eqnarray*}
  P \approx_{K} Q
\end{eqnarray*}

\begin{eqnarray*}
  P \approx Q
\end{eqnarray*}

$\approx_{K} = \approx = \approx_{L}$

\subsubsection{Contextual duality}

Note that contexts extend the quotation operation to a family of
operations from processes to names. Given a context, $M$, we can
define a \emph{nominal context}, $\quotep{M}$ by $\quotep{M}[P] :=
\quotep{M[P]}$. To foreshadow what is to come we observe that these
operations enjoy a duality with processes very much like the duality
between vectors and maps from vectors to scalars.

Further, because the calculus is essentially higher-order, we have a
correspondence between contexts and processes. More specifically,
given a name $x$ and a context $M$ we can construct $M^{*}_{x}$ such
that 

\begin{mathpar}
  M^{*}_{x} | \lift{x}{P} \red M[P]
\end{mathpar}

namely,

\begin{mathpar}
  M^{*}_{x} := x?(u).M[\dropn{u}]
\end{mathpar}

The dependence of $M^{*}_{x}$ on a name makes it an abstraction, 

\begin{mathpar}
  M^{*} := (x)x?(u).M[\dropn{u}]
\end{mathpar}

\subsection{Additional notation}

It will sometimes be convenient to denote the process a name
quotes. We already have the notation $x = \quotep{P}$, but it will be
convenient to introduce an alternate notation, $\procn{x}$, when we
want to emphasize the connection to the use of the name. Note that, by
virtue of name equivalence, $\quotep{\procn{x}} \nameeq x$; so, the
notation is consistent with previous definitions.

Further, because names have structure it is possible to effect
substitutions on the basis of that structure. This means we need to
upgrade our notation for substitutions, which we accomplish by
adapting comprehension notation. Thus,

\begin{mathpar}
  P\{ y / x : x \in S \}
\end{mathpar}

is interpreted to mean the process derived from P by replacing (in a
capture-avoiding manner) each occurrence of $x$ in $S$ by $y$. For example,

\begin{mathpar}
  P\{ \quotep{\procn{x}|\procn{x}} / x : x \in \freenames{P} \}
\end{mathpar}

will replace each (occurrence) of a free name $x$ in $P$ by
$\quotep{\procn{x}|\procn{x}}$.

Also, we will avail ourselves of the notation $x^{L}$ and $x^{R}$ to
denote injections of a name into disjoint copies of the name
space. There are numerous ways to accomplish this. One example can be
found in \cite{MeredithR05}. This notation overloads to vectors of
names: $\vec{x}^{\pi} := (x_{i}^{\pi} \; : \; 0 \leq i < |\vec{x}| )$ where $\pi \in \{L,R\}$.

We also use $P^{\Box} := P|\Box$.

In \cite{MeredithR05} an interpretation of the new operator is
given. It turns out that there are several possible interpretations
all enjoying the requisite algebraic properties of the operator (see
\cite{milner91polyadicpi}). We will therefore make liberal use of
$(\nu\; \vec{x})P$.

% subsection the_syntax_and_semantics_of_the_notation_system (end)   

\input{qm2pi.qmops} 

\input{qm2pi.sterngerlach} 

\input{qm2pi.metric} 

% section concurrent_process_calculi (end)

%\input{qm2pi.proofsketch}

% section proof sketch (end)

%\input{qm2pi.slviaknots} 

% section spatial logic via knots (end)

\input{qm2pi.conclusion}

% section conclusion (end)

%\input{qm2pi.dtcodes} 

% section wiring algorithm (end)

\input{qm2pi.ack} 

% section acknowledgments (end)

\newpage


\bibliographystyle{plain}   
\bibliography{../../biblios/main.bib}

\input{qm2pi.rhodetails}

\end{document}

 

%\documentclass[12pt]{llncs}
%\documentclass{jktr}

\usepackage[pdftex]{hyperref}                   
\usepackage {listings}
\usepackage {mathpartir}
\usepackage{bcprules}
%\usepackage{listings}
                       
\usepackage{graphicx} 
%\usepackage[margins=2.5cm,nohead,nofoot]{geometry}
%\usepackage{geometry}
\usepackage{amsfonts}
\usepackage{amstext}
\usepackage{latexsym}
\usepackage{amssymb}
\usepackage{color}


%\include{myPreamble}
\include{qm2pi.local} 

%\ifpdf
%\usepackage[pdftex]{graphicx}
%\else
%\usepackage{graphicx}
%\fi

 % \ifpdf
%  \usepackage{pdfsync}
%  \if


%\title{Brief Article}
%\author{David F. Snyder}
%\author{L.G. Meredith}

%\address{Dept. of Math., Texas State University--San Marcos, San Marcos, TX 78666}
       
\pagestyle{empty}


\begin{document}

\lstset{language=[Objective]Caml,frame=shadowbox}

\input{qm2pi.front}

% section front matter (end)

\input{qm2pi.intro} 
 
% section introduction (end)

% \input{qm2pi.knotations} 

% section notation (end)

\input{qm2pi.process.calculi} 

% section concurrent_process_calculi_and_spatial_logics_ (end)
    
%\input{qm2pi.knots2pi} 

%\input{qm2pi.trefoil} 

%\input{qm2pi.mainthm} 

% subsection basic_interpretation (end)

%\input{qm2pi.rho.presentation} 
\subsection{The syntax and semantics of the notation system}\label{sub:the_syntax_and_semantics_of_the_notation_system} % (fold)

We now summarize a technical presentation of the calculus that
embodies our theory of dynamics. The typical presentation of such a
calculus follows the style of giving generators and relations on
them. The grammar, below, describing term constructors, freely
generates the set of processes, $\Proc$. This set is then quotiented
by a relation known as structural congruence and it is over this set
that the notion of dynamics is expressed. This presentation is
essentially that of \cite{MeredithR05} with the addition of
polyadicity and summation. For readability we have relegated some of
the technical subtleties to an appendix.

\subsubsection{Process grammar}\label{subsub:process_grammar}

\begin{mathpar}
  \inferrule* [lab=synchronization] {} {{M} \bc \pzero \;|\; x?F \;|\; x!C }
  \and
  \inferrule* [lab=abstraction] {} {{F} \bc (x)P}
  \and
  \inferrule* [lab=concretion] {} {{C} \bc \langle Q \rangle}
  \and
  \inferrule* [lab=process] {} {{P,Q} \bc M \;| \;P|Q \;|\; @{x}}
  \and
  \inferrule* [lab=name] {} {{x} \bc \quotep{P}}
\end{mathpar} 

Note that $\vec{x}$ (resp. $\vec{P}$) denotes a vector of names
(resp. processes) of length $|\vec{x}|$ (resp. $|\vec{P}|$). We adopt
the following useful abbreviations.

\begin{mathpar}
   x?(\vec{y}).P := x.(\vec{y})P \and  x\clift{\vec{P}} := x.\clift{\vec{P}}
   \and x!(y) := \lift{x}{\dropn{y}}
   \and \Pi_{i=0}^{n-1}P_i := P_0 | \ldots | P_{n-1}
\end{mathpar}

\subsubsection{Structural congruence}

\paragraph{Free and bound names and alpha-equivalence.} At the
core of structural equivalence is alpha-equivalence which identifies
process that are the same up to a change of variable. Formally, we
recognize the distinction between free and bound names. The free names
of a process, $\freenames{P}$, may be calculated recursively as
follows:

\begin{mathpar}
\freenames{\pzero} := \emptyset
  \and \\
  \freenames{x?(y).P} := \{ x \} \cup (\freenames{P} \setminus \{ y \})
  \and 
  \freenames{x!\langle P \rangle} := \{ x \} \cup \{ P \} 
  \and \\
  \freenames{P|Q} := \freenames{P} \cup \freenames{Q}
  \and \\
  \freenames{@{x}} := \{ x \}
\end{mathpar}

$\pi$
$\quotep{\pi}$

$\freenames{-} : \pi \to \mathcal{P}(\quotep{\pi})$

\begin{eqnarray*}
  \freenames{\pzero} & := & \emptyset \\
  \freenames{x?(y).P} & := & \{ x \} \cup (\freenames{P} \setminus \{ y \}) \\
  \freenames{x!\langle P \rangle} & := & \{ x \} \cup \{ P \} \\
  \freenames{P|Q} & := & \freenames{P} \cup \freenames{Q} \\
  \freenames{\dropn{x}} & := & \{ x \}
\end{eqnarray*}

The bound names of a process, $\boundnames{P}$, are those names occurring in $P$
that are not free. For example, in $x?(y).0$, the name $x$ is free, while $y$ is bound.

\begin{mathpar}
  \inferrule* [lab=monoidal-laws] {} { P|Q \equiv Q|P \and P|0 \equiv P \and P|(Q|R) \equiv (P|Q)|R }
\end{mathpar}

\begin{mathpar}
  \inferrule* [lab=alpha-equivalence] {} { (x)P \equiv (y)P\{y/x\} \and y \not\in \freenames{P} }
\end{mathpar}

\begin{definition}
Then two processes, $P,Q$, are alpha-equivalent if $P = Q\{\vec{y}/\vec{x}\}$ for
some $\vec{x} \in \boundnames{Q},\vec{y} \in \boundnames{P}$, where $Q\{\vec{y}/\vec{x}\}$
denotes the capture-avoiding substitution of $\vec{y}$ for $\vec{x}$ in $Q$.
\end{definition}

\begin{definition}
  The {\em structural congruence} \cite{SangiorgiWalker} , $\equiv$,
  between processes is the least congruence containing
  alpha-equivalence, satisfying the abelian monoid laws
  (associativity, commutativity and $\pzero$ as identity) for parallel
  composition $|$ and for summation $+$.
\end{definition}

\subsection{Name equivalence}

We take name equivalence, written $\nameeq$, to be the smallest
equivalence relation generated by the following rules.

\begin{mathpar}
\inferrule*[lab=Quote-drop]
{ }
{ \quotep{@{x}} \nameeq x }

\inferrule*[lab=Struct-equiv]
{ P \scong Q }
{ \quotep{P} \nameeq \quotep{Q} }
\end{mathpar}

The astute reader will have noticed that the mutual recursion of names
and processes imposes a mutual recursion on alpha-equivalence and
structural equivalence via name-equivalence. Fortunately, all of this
works out pleasantly and we may calculate in the natural way, free of
concern. The reader interested in the details is referred to the
appendix \ref{appendix:rho_details}.

\subsection{Substitution}

We use $\Proc$ for the set of processes, $\QProc$ for the set of
names, and $\id{\{}\vec{y} / \vec{x} \id{\}}$ to denote partial maps,
$s : \QProc \rightarrow \QProc$. A map, $s$ lifts, uniquely, to a map
on process terms, $\widehat{s} : \Proc \rightarrow \Proc$ by the
following equations.

\begin{mathpar}
  (0) \psubstp{Q}{P} := 0 \\
  (R \juxtap S) \psubstp{Q}{P}
  :=    
  (R)\psubstp{Q}{P} \juxtap (S) \psubstp{Q}{P} \\
  (x?(y).R) \psubstp{Q}{P}    
  :=    
  (x)\substp{Q}{P} (z)\concat( (R \psubstn{z}{y}) \psubstp{Q}{P} ) \\
  (\lift{x}{R}) \psubstp{Q}{P}  
  :=
  \lift{(x)\substp{Q}{P}}{ R \psubstp{Q}{P} } \\
%   (\dropn{x})  \psubstp{Q}{P}       
%   := 
%   \left\{ 
%     \begin{array}{ccc} 
%       \dropn{\quotep{Q}} & & x \nameeq \quotep{P} \\
%       \dropn{x} & & otherwise \\
%     \end{array}
%   \right. 
  (\dropn{x})  \psubstp{Q}{P}       
  := 
  \left\{ 
    \begin{array}{ccc} 
      Q & & x \nameeq \quotep{P} \\
      \dropn{x} & & otherwise \\
    \end{array}
  \right.
\end{mathpar}
 

where

\begin{eqnarray}
  (x)\id{\{} \lpquote Q \rpquote / \lpquote P \rpquote \id{\}}            = 
  \left\{ 
    \begin{array}{ccc}
      \lpquote Q \rpquote & & x \nameeq \lpquote P \rpquote \\
      x & & otherwise \\
    \end{array}
  \right. \nonumber
\end{eqnarray}

and $z$ is chosen distinct from $\quotep{P}$, $\quotep{Q}$, the free
names in $Q$, and all the names in $R$. Our $\alpha$-equivalence will
be built in the standard way from this substitution.

\begin{remark}\label{rem:no_self_referential_names}
  One consequence of these definitions is that $\forall P. \quotep{P}
  \not\in \freenames{P}$.
\end{remark}

\subsection{ Dynamic quote: an example }

Anticipating something of what's to come, consider applying the
substitution, $\widehat{\id{\{}u / z \id{\}}}$, to the following pair
of processes, $\lift{w}{y!(z)}$ and $w[ \lpquote y!(z) \rpquote ]$.

\begin{eqnarray}
	\lift{w}{y!(z)}\widehat{\id{\{}u / z \id{\}}}
		& = &
		\lift{w}{y!(u)} \nonumber\\
	w[ \lpquote y!(z) \rpquote ] \widehat{ \id{\{}u / z \id{\}} }
		& = &
		w[ \lpquote y!(z) \rpquote ] \nonumber
\end{eqnarray}

Because the body of the process between quotes is impervious to
substitution, we get radically different answers. In fact, by
examining the first process in an input context,
e.g. $x?(z).\lift{w}{y!(z)}$, we see that the process under the lift
operator may be shaped by prefixed inputs binding a name inside it. In
this sense, the lift operator will be seen as a way to dynamically
construct processes before reifying them as names.

Finally equipped with these standard features we can present the
dynamics of the calculus.

\subsubsection{Operational semantics} 

Finally, we introduce the computational dynamics. What marks these
algebras as distinct from other more traditionally studied algebraic
structures, e.g. vector spaces or polynomial rings, is the manner in
which dynamics is captured. In traditional structures, dynamics is typically
expressed through morphisms between such structures, as in linear maps
between vector spaces or morphisms between rings. In algebras
associated with the semantics of computation, the dynamics is
expressed as part of the algebraic structure itself, through a
reduction reduction relation typically denoted by $\red$. Below, we
give a recursive presentation of this relation for the calculus used
in the encoding.

$\red \subseteq \pi \times \pi$
$\red : \pi \to \mathcal{P}(\pi)$

\begin{mathpar}
  \inferrule* [lab=Comm] { \textsf{match}( x_{src}, x_{trgt} ) } { x_{trgt}?(y)P \; | \; x_{src}!\langle {Q} \rangle \red P\{\quotep{Q}/y}\} }
  \and \\
  \inferrule* [lab=Par] {{P} \red {P}'} {{{P} | {Q}} \red {{P}' | {Q}}}
  \and
  \inferrule* [lab=Equiv]{{{P} \scong {P}'} \andalso {{P}' \red {Q}'} \andalso {{Q}' \scong {Q}}}{{P} \red {Q}}
\end{mathpar}

\begin{eqnarray*}
  match_{\equiv} (\quotep{P},\quotep{Q}) & := & P \equiv Q \\
  match_{\dagger}(\quotep{P},\quotep{Q}) & := & \forall R. P|Q \red^{*} R => R \red^{*} 0 \\
  match_{K}(\quotep{P},\quotep{Q}) & := & K \mbox{ for some context } K
\end{eqnarray*}

$u?(x)P | u!\langle Q \rangle \red P\{\quotep{Q}/x\}$

%We write $\wred$ for $\red^*$, and $P\red$ if $\exists Q $ such that $ P \red Q$.
We write $P\red$ if $\exists Q $ such that $ P \red Q$ and $P\not\red$, otherwise.

\section{Replication}

As mentioned before, it is known that replication (and hence
recursion) can be implemented in a higher-order process algebra
\cite{SangiorgiWalker}. As our first example of calculation with the
machinery thus far presented we give the construction explicitly in
the {\rhoc}.

\begin{eqnarray}
	D_{x} & := & \prefix{x}{y}{(\binpar{\outputp{x}{y}}{@{y}})} \nonumber\\
	\bangp_{x}{P} & := & \binpar{{x}!\langle{\binpar{D_{x}}{P}}\rangle}{D_{x}} \nonumber
\end{eqnarray}

\begin{eqnarray}
	\bangp_{x}{P} & & \nonumber\\
	=
	& {x}!\langle{(\prefix{x}{y}{(\outputp{x}{y} | @{y})) | P}}\rangle 
	      | \prefix{x}{y}{(\outputp{x}{y} | @{y})} & \nonumber\\
	\red
	& (\outputp{x}{y} | @{y})\substn{\quotep{(\prefix{x}{y}{(@{y} | \outputp{x}{y})) | P}}}{y} & \nonumber\\
	=
	& \outputp{x}{\quotep{(\prefix{x}{y}{(\outputp{x}{y} | @{y})) | P}}}
	  | {(\prefix{x}{y}{(\outputp{x}{y} | @{y})) | P}} & \nonumber\\
	\red
	& \ldots & \nonumber\\
	\red^*
	& P | P | \ldots & \nonumber
\end{eqnarray}

Of course, this encoding, as an implementation, runs away, unfolding
$\bangp{P}$ eagerly. A lazier and more implementable replication
operator, restricted to input-guarded processes, may be obtained as follows.

\begin{eqnarray}
\bangp{\prefix{u}{v}{P}} 
	:= 
	\binpar{\lift{x}{\prefix{u}{v}{(\binpar{D(x)}{P})}}}{D(x)} \nonumber
\end{eqnarray}

\begin{remark}
  Note that the lazier definition still does not deal with summation
  or mixed summation (i.e. sums over input and output). The reader is
  invited to construct definitions of replication that deal with these
  features. 

  Further, the definitions are parameterized in a name, $x$. Can you,
  gentle reader, make a definition that eliminates this parameter and
  guarantees no accidental interaction between the replication
  machinery and the process being replicated -- i.e. no accidental
  sharing of names used by the process to get its work done and the
  name(s) used by the replication to effect copying. This latter
  revision of the definition of replication is crucial to obtaining
  the expected identity $!!P \sim !P$.
\end{remark}

\begin{remark}\label{rem:paradoxical_combinator}
  The reader familiar with the lambda calculus will have noticed the
  similarity between $D$ and the paradoxical combinator.

  [Ed. note: the existence of this seems to suggest we have to be more
  restrictive on the set of processes and names we admit if we are to
  support no-cloning.]
\end{remark}

\subsubsection{Bisimulation}

The computational dynamics gives rise to another kind of equivalence,
the equivalence of computational behavior. As previously mentioned
this is typically captured \emph{via} some form of bisimulation.

% The notion we use in this paper is weak barbed bisimulation
% \cite{milner91polyadicpi}.

The notion we use in this paper is derived from weak barbed
bisimulation \cite{milner91polyadicpi}. 

\begin{definition}
An \emph{observation relation}, $\downarrow_{\mathcal N}$, over a set
of names, $\mathcal N$, is the smallest relation satisfying the rules
below.

\infrule[Out-barb]{y \in {\mathcal N}, \; x \nameeq y}
		  {\outputp{x}{v} \downarrow_{\mathcal N} x}
\infrule[Par-barb]{\mbox{$P\downarrow_{\mathcal N} x$ or $Q\downarrow_{\mathcal N} x$}}
		  {\binpar{P}{Q} \downarrow_{\mathcal N} x}

We write $P \Downarrow_{\mathcal N} x$ if there is $Q$ such that 
$P \wred Q$ and $Q \downarrow_{\mathcal N} x$.
\end{definition}

\begin{definition}
%\label{def.bbisim}
An  ${\mathcal N}$-\emph{barbed bisimulation} over a set of names, ${\mathcal N}$, is a symmetric binary relation 
${\mathcal S}_{\mathcal N}$ between agents such that $P\rel{S}_{\mathcal N}Q$ implies:
\begin{enumerate}
\item If $P \red P'$ then $Q \wred Q'$ and $P'\rel{S}_{\mathcal N} Q'$.
\item If $P\downarrow_{\mathcal N} x$, then $Q\Downarrow_{\mathcal N} x$.
\end{enumerate}
$P$ is ${\mathcal N}$-barbed bisimilar to $Q$, written
$P \wbbisim_{\mathcal N} Q$, if $P \rel{S}_{\mathcal N} Q$ for some ${\mathcal N}$-barbed bisimulation ${\mathcal S}_{\mathcal N}$.
\end{definition}

$\mathcal{R} \subseteq \pi \times \pi$

$P \mathcal{R} Q => \forall P'. P \red P' \Rightarrow \exists Q'. Q \red Q', P' \mathcal{R} Q'$

$P \vdash x \Rightarrow Q \vdash x$

\begin{mathpar}
  \inferrule*[lab=Out-barb]{x \nameeq y}{{y}!\langle{Q}\rangle \vdash x}
  \and
  \inferrule*[lab=Par-barb]{\mbox{$P\vdash x$ or $Q\vdash x$}}{\binpar{P}{Q} \vdash x}
\end{mathpar}

\subsubsection{Contexts}

One of the principle advantages of computational calculi like the
$\pi$-calculus is a well-defined notion of context,
contextual-equivalence and a correlation between
contextual-equivalence and notions of bisimulation. The notion of
context allows the decomposition of a process into (sub-)process and
its syntactic environment, its context. Thus, a context may be
thought of as a process with a ``hole'' (written $\Box$) in it. The
application of a context $M$ to a process $P$, written $M[P]$, is
tantamount to filling the hole in $M$ with $P$. In this paper we do
not need the full weight of this theory, but do make use of the notion
of context in the proof the main theorem. 

\begin{mathpar}
  \inferrule* [lab=summation] {} {{M_{M},M_{N}} \bc \Box \;|\; x.M_{A} \;|\; M_{M}+M_{N}}
  \and
  \inferrule* [lab=agent] {} {{M_{A}} \bc (\vec{x})M_{P} \;| \; \clift{P_0,\ldots,M_{P},\ldots,P_N}}
  \and \\
  \inferrule* [lab=process] {} {{M_{P}} \bc M_{N} \;| \;P|M_{P} }
\end{mathpar} 

\begin{mathpar}
  \inferrule* [lab=sychronization] {} {M_{N} \bc \Box \;|\; x?M_{F} \;|\; x!M_{C}}
  \and
  \inferrule* [lab=abstraction] {} {{M_{F}} \bc (x)M_{P} }
  \and
  \inferrule* [lab=concretion] {} {{M_{C}} \bc \langle M_{P} \rangle }
  \and \\
  \inferrule* [lab=process] {} {{M_{P}} \bc M_{N} \;| \;P|M_{P} }
\end{mathpar}

\begin{definition}[contextual application] Given a context $M$, and
  process $P$, we define the \emph{contextual application}, $M[P] :=
  M\{P/\Box\}$. That is, the contextual application of M to P is the
  substitution of $P$ for $\Box$ in $M$.
\end{definition}

$\meaningof{-} : L \to \mathcal{P}(\pi)$

\begin{mathpar}
  \inferrule* [lab=collection] {} {\meaningof{true} = \pi, \and \meaningof{~E} = \pi \setminus \meaningof{E}, \and \meaningof{E_{1} \& E_{2}} = \meaningof{E_{1}} \cap \meaningof{E_{2}}}
\end{mathpar}

\begin{mathpar}
  \inferrule* [lab=structure] {} {\meaningof{0} = \{ P \in \pi | P \equiv 0 \}, \and \\ \meaningof{E_1 | E_2} = \{ P \in \pi | P \equiv P_{1} | P_{2}, P_{1} \in \meaningof{E_{1}}, P_{2} \in \meaningof{E_2}\} }
\end{mathpar}

\begin{mathpar}
 \inferrule* [lab=behavior] {} {\meaningof{\langle a?b \rangle E} = \{ P \in \pi | P \equiv Q | u?(y)P', \\ \and \\\\ \and \\ \;\;\; u \in \meaningof{a}, \forall z.P'\{z/y\} \in \meaningof{E\{z/b\}}\}, \and \\ \meaningof{a!E} = \{ P \in \pi | P \equiv Q | x!\langle P' \rangle, x \in \meaningof{a} P' \in \meaningof{E}\} }
\end{mathpar}

\begin{mathpar}
 \inferrule* [lab=nominal] {} {\meaningof{\quotep{E}} = \{ \quotep{P} \in \quotep{\pi} | P \in \meaningof{E} \}, \and \meaningof{\quotep{P}} = \{ \quotep{Q} \in \quotep{\pi} | P \equiv Q \} \and \\ \meaningof{@\quotep{E}} = \{ P \in \pi | P \equiv @x, x \in \meaningof{E} \}}
\end{mathpar}

\begin{eqnarray*}
  \\
  \meaningof{-} : TS \to ST
\end{eqnarray*}

\begin{eqnarray*}
  \\
  L : TS \to ST
\end{eqnarray*}

\begin{eqnarray*}
  \\
  P \models E \iff P \in \meaningof{E}
\end{eqnarray*}

\begin{eqnarray*}
  P \approx_{L} Q \iff \forall E \in L. P \models E \iff Q \models E
\end{eqnarray*}

\begin{eqnarray*}
  P \approx_{K} Q
\end{eqnarray*}

\begin{eqnarray*}
  P \approx Q
\end{eqnarray*}

$\approx_{K} = \approx = \approx_{L}$

\subsubsection{Contextual duality}

Note that contexts extend the quotation operation to a family of
operations from processes to names. Given a context, $M$, we can
define a \emph{nominal context}, $\quotep{M}$ by $\quotep{M}[P] :=
\quotep{M[P]}$. To foreshadow what is to come we observe that these
operations enjoy a duality with processes very much like the duality
between vectors and maps from vectors to scalars.

Further, because the calculus is essentially higher-order, we have a
correspondence between contexts and processes. More specifically,
given a name $x$ and a context $M$ we can construct $M^{*}_{x}$ such
that 

\begin{mathpar}
  M^{*}_{x} | \lift{x}{P} \red M[P]
\end{mathpar}

namely,

\begin{mathpar}
  M^{*}_{x} := x?(u).M[\dropn{u}]
\end{mathpar}

The dependence of $M^{*}_{x}$ on a name makes it an abstraction, 

\begin{mathpar}
  M^{*} := (x)x?(u).M[\dropn{u}]
\end{mathpar}

\subsection{Additional notation}

It will sometimes be convenient to denote the process a name
quotes. We already have the notation $x = \quotep{P}$, but it will be
convenient to introduce an alternate notation, $\procn{x}$, when we
want to emphasize the connection to the use of the name. Note that, by
virtue of name equivalence, $\quotep{\procn{x}} \nameeq x$; so, the
notation is consistent with previous definitions.

Further, because names have structure it is possible to effect
substitutions on the basis of that structure. This means we need to
upgrade our notation for substitutions, which we accomplish by
adapting comprehension notation. Thus,

\begin{mathpar}
  P\{ y / x : x \in S \}
\end{mathpar}

is interpreted to mean the process derived from P by replacing (in a
capture-avoiding manner) each occurrence of $x$ in $S$ by $y$. For example,

\begin{mathpar}
  P\{ \quotep{\procn{x}|\procn{x}} / x : x \in \freenames{P} \}
\end{mathpar}

will replace each (occurrence) of a free name $x$ in $P$ by
$\quotep{\procn{x}|\procn{x}}$.

Also, we will avail ourselves of the notation $x^{L}$ and $x^{R}$ to
denote injections of a name into disjoint copies of the name
space. There are numerous ways to accomplish this. One example can be
found in \cite{MeredithR05}. This notation overloads to vectors of
names: $\vec{x}^{\pi} := (x_{i}^{\pi} \; : \; 0 \leq i < |\vec{x}| )$ where $\pi \in \{L,R\}$.

We also use $P^{\Box} := P|\Box$.

In \cite{MeredithR05} an interpretation of the new operator is
given. It turns out that there are several possible interpretations
all enjoying the requisite algebraic properties of the operator (see
\cite{milner91polyadicpi}). We will therefore make liberal use of
$(\nu\; \vec{x})P$.

% subsection the_syntax_and_semantics_of_the_notation_system (end)   

\input{qm2pi.qmops} 

\input{qm2pi.sterngerlach} 

\input{qm2pi.metric} 

% section concurrent_process_calculi (end)

%\input{qm2pi.proofsketch}

% section proof sketch (end)

%\input{qm2pi.slviaknots} 

% section spatial logic via knots (end)

\input{qm2pi.conclusion}

% section conclusion (end)

%\input{qm2pi.dtcodes} 

% section wiring algorithm (end)

\input{qm2pi.ack} 

% section acknowledgments (end)

\newpage


\bibliographystyle{plain}   
\bibliography{../../biblios/main.bib}

\input{qm2pi.rhodetails}

\end{document}

 

% subsection basic_interpretation (end)

%\input{qm2pi.rho.presentation} 
\subsection{The syntax and semantics of the notation system}\label{sub:the_syntax_and_semantics_of_the_notation_system} % (fold)

We now summarize a technical presentation of the calculus that
embodies our theory of dynamics. The typical presentation of such a
calculus follows the style of giving generators and relations on
them. The grammar, below, describing term constructors, freely
generates the set of processes, $\Proc$. This set is then quotiented
by a relation known as structural congruence and it is over this set
that the notion of dynamics is expressed. This presentation is
essentially that of \cite{MeredithR05} with the addition of
polyadicity and summation. For readability we have relegated some of
the technical subtleties to an appendix.

\subsubsection{Process grammar}\label{subsub:process_grammar}

\begin{mathpar}
  \inferrule* [lab=synchronization] {} {{M} \bc \pzero \;|\; x?F \;|\; x!C }
  \and
  \inferrule* [lab=abstraction] {} {{F} \bc (x)P}
  \and
  \inferrule* [lab=concretion] {} {{C} \bc \langle Q \rangle}
  \and
  \inferrule* [lab=process] {} {{P,Q} \bc M \;| \;P|Q \;|\; @{x}}
  \and
  \inferrule* [lab=name] {} {{x} \bc \quotep{P}}
\end{mathpar} 

Note that $\vec{x}$ (resp. $\vec{P}$) denotes a vector of names
(resp. processes) of length $|\vec{x}|$ (resp. $|\vec{P}|$). We adopt
the following useful abbreviations.

\begin{mathpar}
   x?(\vec{y}).P := x.(\vec{y})P \and  x\clift{\vec{P}} := x.\clift{\vec{P}}
   \and x!(y) := \lift{x}{\dropn{y}}
   \and \Pi_{i=0}^{n-1}P_i := P_0 | \ldots | P_{n-1}
\end{mathpar}

\subsubsection{Structural congruence}

\paragraph{Free and bound names and alpha-equivalence.} At the
core of structural equivalence is alpha-equivalence which identifies
process that are the same up to a change of variable. Formally, we
recognize the distinction between free and bound names. The free names
of a process, $\freenames{P}$, may be calculated recursively as
follows:

\begin{mathpar}
\freenames{\pzero} := \emptyset
  \and \\
  \freenames{x?(y).P} := \{ x \} \cup (\freenames{P} \setminus \{ y \})
  \and 
  \freenames{x!\langle P \rangle} := \{ x \} \cup \{ P \} 
  \and \\
  \freenames{P|Q} := \freenames{P} \cup \freenames{Q}
  \and \\
  \freenames{@{x}} := \{ x \}
\end{mathpar}

$\pi$
$\quotep{\pi}$

$\freenames{-} : \pi \to \mathcal{P}(\quotep{\pi})$

\begin{eqnarray*}
  \freenames{\pzero} & := & \emptyset \\
  \freenames{x?(y).P} & := & \{ x \} \cup (\freenames{P} \setminus \{ y \}) \\
  \freenames{x!\langle P \rangle} & := & \{ x \} \cup \{ P \} \\
  \freenames{P|Q} & := & \freenames{P} \cup \freenames{Q} \\
  \freenames{\dropn{x}} & := & \{ x \}
\end{eqnarray*}

The bound names of a process, $\boundnames{P}$, are those names occurring in $P$
that are not free. For example, in $x?(y).0$, the name $x$ is free, while $y$ is bound.

\begin{mathpar}
  \inferrule* [lab=monoidal-laws] {} { P|Q \equiv Q|P \and P|0 \equiv P \and P|(Q|R) \equiv (P|Q)|R }
\end{mathpar}

\begin{mathpar}
  \inferrule* [lab=alpha-equivalence] {} { (x)P \equiv (y)P\{y/x\} \and y \not\in \freenames{P} }
\end{mathpar}

\begin{definition}
Then two processes, $P,Q$, are alpha-equivalent if $P = Q\{\vec{y}/\vec{x}\}$ for
some $\vec{x} \in \boundnames{Q},\vec{y} \in \boundnames{P}$, where $Q\{\vec{y}/\vec{x}\}$
denotes the capture-avoiding substitution of $\vec{y}$ for $\vec{x}$ in $Q$.
\end{definition}

\begin{definition}
  The {\em structural congruence} \cite{SangiorgiWalker} , $\equiv$,
  between processes is the least congruence containing
  alpha-equivalence, satisfying the abelian monoid laws
  (associativity, commutativity and $\pzero$ as identity) for parallel
  composition $|$ and for summation $+$.
\end{definition}

\subsection{Name equivalence}

We take name equivalence, written $\nameeq$, to be the smallest
equivalence relation generated by the following rules.

\begin{mathpar}
\inferrule*[lab=Quote-drop]
{ }
{ \quotep{@{x}} \nameeq x }

\inferrule*[lab=Struct-equiv]
{ P \scong Q }
{ \quotep{P} \nameeq \quotep{Q} }
\end{mathpar}

The astute reader will have noticed that the mutual recursion of names
and processes imposes a mutual recursion on alpha-equivalence and
structural equivalence via name-equivalence. Fortunately, all of this
works out pleasantly and we may calculate in the natural way, free of
concern. The reader interested in the details is referred to the
appendix \ref{appendix:rho_details}.

\subsection{Substitution}

We use $\Proc$ for the set of processes, $\QProc$ for the set of
names, and $\id{\{}\vec{y} / \vec{x} \id{\}}$ to denote partial maps,
$s : \QProc \rightarrow \QProc$. A map, $s$ lifts, uniquely, to a map
on process terms, $\widehat{s} : \Proc \rightarrow \Proc$ by the
following equations.

\begin{mathpar}
  (0) \psubstp{Q}{P} := 0 \\
  (R \juxtap S) \psubstp{Q}{P}
  :=    
  (R)\psubstp{Q}{P} \juxtap (S) \psubstp{Q}{P} \\
  (x?(y).R) \psubstp{Q}{P}    
  :=    
  (x)\substp{Q}{P} (z)\concat( (R \psubstn{z}{y}) \psubstp{Q}{P} ) \\
  (\lift{x}{R}) \psubstp{Q}{P}  
  :=
  \lift{(x)\substp{Q}{P}}{ R \psubstp{Q}{P} } \\
%   (\dropn{x})  \psubstp{Q}{P}       
%   := 
%   \left\{ 
%     \begin{array}{ccc} 
%       \dropn{\quotep{Q}} & & x \nameeq \quotep{P} \\
%       \dropn{x} & & otherwise \\
%     \end{array}
%   \right. 
  (\dropn{x})  \psubstp{Q}{P}       
  := 
  \left\{ 
    \begin{array}{ccc} 
      Q & & x \nameeq \quotep{P} \\
      \dropn{x} & & otherwise \\
    \end{array}
  \right.
\end{mathpar}
 

where

\begin{eqnarray}
  (x)\id{\{} \lpquote Q \rpquote / \lpquote P \rpquote \id{\}}            = 
  \left\{ 
    \begin{array}{ccc}
      \lpquote Q \rpquote & & x \nameeq \lpquote P \rpquote \\
      x & & otherwise \\
    \end{array}
  \right. \nonumber
\end{eqnarray}

and $z$ is chosen distinct from $\quotep{P}$, $\quotep{Q}$, the free
names in $Q$, and all the names in $R$. Our $\alpha$-equivalence will
be built in the standard way from this substitution.

\begin{remark}\label{rem:no_self_referential_names}
  One consequence of these definitions is that $\forall P. \quotep{P}
  \not\in \freenames{P}$.
\end{remark}

\subsection{ Dynamic quote: an example }

Anticipating something of what's to come, consider applying the
substitution, $\widehat{\id{\{}u / z \id{\}}}$, to the following pair
of processes, $\lift{w}{y!(z)}$ and $w[ \lpquote y!(z) \rpquote ]$.

\begin{eqnarray}
	\lift{w}{y!(z)}\widehat{\id{\{}u / z \id{\}}}
		& = &
		\lift{w}{y!(u)} \nonumber\\
	w[ \lpquote y!(z) \rpquote ] \widehat{ \id{\{}u / z \id{\}} }
		& = &
		w[ \lpquote y!(z) \rpquote ] \nonumber
\end{eqnarray}

Because the body of the process between quotes is impervious to
substitution, we get radically different answers. In fact, by
examining the first process in an input context,
e.g. $x?(z).\lift{w}{y!(z)}$, we see that the process under the lift
operator may be shaped by prefixed inputs binding a name inside it. In
this sense, the lift operator will be seen as a way to dynamically
construct processes before reifying them as names.

Finally equipped with these standard features we can present the
dynamics of the calculus.

\subsubsection{Operational semantics} 

Finally, we introduce the computational dynamics. What marks these
algebras as distinct from other more traditionally studied algebraic
structures, e.g. vector spaces or polynomial rings, is the manner in
which dynamics is captured. In traditional structures, dynamics is typically
expressed through morphisms between such structures, as in linear maps
between vector spaces or morphisms between rings. In algebras
associated with the semantics of computation, the dynamics is
expressed as part of the algebraic structure itself, through a
reduction reduction relation typically denoted by $\red$. Below, we
give a recursive presentation of this relation for the calculus used
in the encoding.

$\red \subseteq \pi \times \pi$
$\red : \pi \to \mathcal{P}(\pi)$

\begin{mathpar}
  \inferrule* [lab=Comm] { \textsf{match}( x_{src}, x_{trgt} ) } { x_{trgt}?(y)P \; | \; x_{src}!\langle {Q} \rangle \red P\{\quotep{Q}/y}\} }
  \and \\
  \inferrule* [lab=Par] {{P} \red {P}'} {{{P} | {Q}} \red {{P}' | {Q}}}
  \and
  \inferrule* [lab=Equiv]{{{P} \scong {P}'} \andalso {{P}' \red {Q}'} \andalso {{Q}' \scong {Q}}}{{P} \red {Q}}
\end{mathpar}

\begin{eqnarray*}
  match_{\equiv} (\quotep{P},\quotep{Q}) & := & P \equiv Q \\
  match_{\dagger}(\quotep{P},\quotep{Q}) & := & \forall R. P|Q \red^{*} R => R \red^{*} 0 \\
  match_{K}(\quotep{P},\quotep{Q}) & := & K \mbox{ for some context } K
\end{eqnarray*}

$u?(x)P | u!\langle Q \rangle \red P\{\quotep{Q}/x\}$

%We write $\wred$ for $\red^*$, and $P\red$ if $\exists Q $ such that $ P \red Q$.
We write $P\red$ if $\exists Q $ such that $ P \red Q$ and $P\not\red$, otherwise.

\section{Replication}

As mentioned before, it is known that replication (and hence
recursion) can be implemented in a higher-order process algebra
\cite{SangiorgiWalker}. As our first example of calculation with the
machinery thus far presented we give the construction explicitly in
the {\rhoc}.

\begin{eqnarray}
	D_{x} & := & \prefix{x}{y}{(\binpar{\outputp{x}{y}}{@{y}})} \nonumber\\
	\bangp_{x}{P} & := & \binpar{{x}!\langle{\binpar{D_{x}}{P}}\rangle}{D_{x}} \nonumber
\end{eqnarray}

\begin{eqnarray}
	\bangp_{x}{P} & & \nonumber\\
	=
	& {x}!\langle{(\prefix{x}{y}{(\outputp{x}{y} | @{y})) | P}}\rangle 
	      | \prefix{x}{y}{(\outputp{x}{y} | @{y})} & \nonumber\\
	\red
	& (\outputp{x}{y} | @{y})\substn{\quotep{(\prefix{x}{y}{(@{y} | \outputp{x}{y})) | P}}}{y} & \nonumber\\
	=
	& \outputp{x}{\quotep{(\prefix{x}{y}{(\outputp{x}{y} | @{y})) | P}}}
	  | {(\prefix{x}{y}{(\outputp{x}{y} | @{y})) | P}} & \nonumber\\
	\red
	& \ldots & \nonumber\\
	\red^*
	& P | P | \ldots & \nonumber
\end{eqnarray}

Of course, this encoding, as an implementation, runs away, unfolding
$\bangp{P}$ eagerly. A lazier and more implementable replication
operator, restricted to input-guarded processes, may be obtained as follows.

\begin{eqnarray}
\bangp{\prefix{u}{v}{P}} 
	:= 
	\binpar{\lift{x}{\prefix{u}{v}{(\binpar{D(x)}{P})}}}{D(x)} \nonumber
\end{eqnarray}

\begin{remark}
  Note that the lazier definition still does not deal with summation
  or mixed summation (i.e. sums over input and output). The reader is
  invited to construct definitions of replication that deal with these
  features. 

  Further, the definitions are parameterized in a name, $x$. Can you,
  gentle reader, make a definition that eliminates this parameter and
  guarantees no accidental interaction between the replication
  machinery and the process being replicated -- i.e. no accidental
  sharing of names used by the process to get its work done and the
  name(s) used by the replication to effect copying. This latter
  revision of the definition of replication is crucial to obtaining
  the expected identity $!!P \sim !P$.
\end{remark}

\begin{remark}\label{rem:paradoxical_combinator}
  The reader familiar with the lambda calculus will have noticed the
  similarity between $D$ and the paradoxical combinator.

  [Ed. note: the existence of this seems to suggest we have to be more
  restrictive on the set of processes and names we admit if we are to
  support no-cloning.]
\end{remark}

\subsubsection{Bisimulation}

The computational dynamics gives rise to another kind of equivalence,
the equivalence of computational behavior. As previously mentioned
this is typically captured \emph{via} some form of bisimulation.

% The notion we use in this paper is weak barbed bisimulation
% \cite{milner91polyadicpi}.

The notion we use in this paper is derived from weak barbed
bisimulation \cite{milner91polyadicpi}. 

\begin{definition}
An \emph{observation relation}, $\downarrow_{\mathcal N}$, over a set
of names, $\mathcal N$, is the smallest relation satisfying the rules
below.

\infrule[Out-barb]{y \in {\mathcal N}, \; x \nameeq y}
		  {\outputp{x}{v} \downarrow_{\mathcal N} x}
\infrule[Par-barb]{\mbox{$P\downarrow_{\mathcal N} x$ or $Q\downarrow_{\mathcal N} x$}}
		  {\binpar{P}{Q} \downarrow_{\mathcal N} x}

We write $P \Downarrow_{\mathcal N} x$ if there is $Q$ such that 
$P \wred Q$ and $Q \downarrow_{\mathcal N} x$.
\end{definition}

\begin{definition}
%\label{def.bbisim}
An  ${\mathcal N}$-\emph{barbed bisimulation} over a set of names, ${\mathcal N}$, is a symmetric binary relation 
${\mathcal S}_{\mathcal N}$ between agents such that $P\rel{S}_{\mathcal N}Q$ implies:
\begin{enumerate}
\item If $P \red P'$ then $Q \wred Q'$ and $P'\rel{S}_{\mathcal N} Q'$.
\item If $P\downarrow_{\mathcal N} x$, then $Q\Downarrow_{\mathcal N} x$.
\end{enumerate}
$P$ is ${\mathcal N}$-barbed bisimilar to $Q$, written
$P \wbbisim_{\mathcal N} Q$, if $P \rel{S}_{\mathcal N} Q$ for some ${\mathcal N}$-barbed bisimulation ${\mathcal S}_{\mathcal N}$.
\end{definition}

$\mathcal{R} \subseteq \pi \times \pi$

$P \mathcal{R} Q => \forall P'. P \red P' \Rightarrow \exists Q'. Q \red Q', P' \mathcal{R} Q'$

$P \vdash x \Rightarrow Q \vdash x$

\begin{mathpar}
  \inferrule*[lab=Out-barb]{x \nameeq y}{{y}!\langle{Q}\rangle \vdash x}
  \and
  \inferrule*[lab=Par-barb]{\mbox{$P\vdash x$ or $Q\vdash x$}}{\binpar{P}{Q} \vdash x}
\end{mathpar}

\subsubsection{Contexts}

One of the principle advantages of computational calculi like the
$\pi$-calculus is a well-defined notion of context,
contextual-equivalence and a correlation between
contextual-equivalence and notions of bisimulation. The notion of
context allows the decomposition of a process into (sub-)process and
its syntactic environment, its context. Thus, a context may be
thought of as a process with a ``hole'' (written $\Box$) in it. The
application of a context $M$ to a process $P$, written $M[P]$, is
tantamount to filling the hole in $M$ with $P$. In this paper we do
not need the full weight of this theory, but do make use of the notion
of context in the proof the main theorem. 

\begin{mathpar}
  \inferrule* [lab=summation] {} {{M_{M},M_{N}} \bc \Box \;|\; x.M_{A} \;|\; M_{M}+M_{N}}
  \and
  \inferrule* [lab=agent] {} {{M_{A}} \bc (\vec{x})M_{P} \;| \; \clift{P_0,\ldots,M_{P},\ldots,P_N}}
  \and \\
  \inferrule* [lab=process] {} {{M_{P}} \bc M_{N} \;| \;P|M_{P} }
\end{mathpar} 

\begin{mathpar}
  \inferrule* [lab=sychronization] {} {M_{N} \bc \Box \;|\; x?M_{F} \;|\; x!M_{C}}
  \and
  \inferrule* [lab=abstraction] {} {{M_{F}} \bc (x)M_{P} }
  \and
  \inferrule* [lab=concretion] {} {{M_{C}} \bc \langle M_{P} \rangle }
  \and \\
  \inferrule* [lab=process] {} {{M_{P}} \bc M_{N} \;| \;P|M_{P} }
\end{mathpar}

\begin{definition}[contextual application] Given a context $M$, and
  process $P$, we define the \emph{contextual application}, $M[P] :=
  M\{P/\Box\}$. That is, the contextual application of M to P is the
  substitution of $P$ for $\Box$ in $M$.
\end{definition}

$\meaningof{-} : L \to \mathcal{P}(\pi)$

\begin{mathpar}
  \inferrule* [lab=collection] {} {\meaningof{true} = \pi, \and \meaningof{~E} = \pi \setminus \meaningof{E}, \and \meaningof{E_{1} \& E_{2}} = \meaningof{E_{1}} \cap \meaningof{E_{2}}}
\end{mathpar}

\begin{mathpar}
  \inferrule* [lab=structure] {} {\meaningof{0} = \{ P \in \pi | P \equiv 0 \}, \and \\ \meaningof{E_1 | E_2} = \{ P \in \pi | P \equiv P_{1} | P_{2}, P_{1} \in \meaningof{E_{1}}, P_{2} \in \meaningof{E_2}\} }
\end{mathpar}

\begin{mathpar}
 \inferrule* [lab=behavior] {} {\meaningof{\langle a?b \rangle E} = \{ P \in \pi | P \equiv Q | u?(y)P', \\ \and \\\\ \and \\ \;\;\; u \in \meaningof{a}, \forall z.P'\{z/y\} \in \meaningof{E\{z/b\}}\}, \and \\ \meaningof{a!E} = \{ P \in \pi | P \equiv Q | x!\langle P' \rangle, x \in \meaningof{a} P' \in \meaningof{E}\} }
\end{mathpar}

\begin{mathpar}
 \inferrule* [lab=nominal] {} {\meaningof{\quotep{E}} = \{ \quotep{P} \in \quotep{\pi} | P \in \meaningof{E} \}, \and \meaningof{\quotep{P}} = \{ \quotep{Q} \in \quotep{\pi} | P \equiv Q \} \and \\ \meaningof{@\quotep{E}} = \{ P \in \pi | P \equiv @x, x \in \meaningof{E} \}}
\end{mathpar}

\begin{eqnarray*}
  \\
  \meaningof{-} : TS \to ST
\end{eqnarray*}

\begin{eqnarray*}
  \\
  L : TS \to ST
\end{eqnarray*}

\begin{eqnarray*}
  \\
  P \models E \iff P \in \meaningof{E}
\end{eqnarray*}

\begin{eqnarray*}
  P \approx_{L} Q \iff \forall E \in L. P \models E \iff Q \models E
\end{eqnarray*}

\begin{eqnarray*}
  P \approx_{K} Q
\end{eqnarray*}

\begin{eqnarray*}
  P \approx Q
\end{eqnarray*}

$\approx_{K} = \approx = \approx_{L}$

\subsubsection{Contextual duality}

Note that contexts extend the quotation operation to a family of
operations from processes to names. Given a context, $M$, we can
define a \emph{nominal context}, $\quotep{M}$ by $\quotep{M}[P] :=
\quotep{M[P]}$. To foreshadow what is to come we observe that these
operations enjoy a duality with processes very much like the duality
between vectors and maps from vectors to scalars.

Further, because the calculus is essentially higher-order, we have a
correspondence between contexts and processes. More specifically,
given a name $x$ and a context $M$ we can construct $M^{*}_{x}$ such
that 

\begin{mathpar}
  M^{*}_{x} | \lift{x}{P} \red M[P]
\end{mathpar}

namely,

\begin{mathpar}
  M^{*}_{x} := x?(u).M[\dropn{u}]
\end{mathpar}

The dependence of $M^{*}_{x}$ on a name makes it an abstraction, 

\begin{mathpar}
  M^{*} := (x)x?(u).M[\dropn{u}]
\end{mathpar}

\subsection{Additional notation}

It will sometimes be convenient to denote the process a name
quotes. We already have the notation $x = \quotep{P}$, but it will be
convenient to introduce an alternate notation, $\procn{x}$, when we
want to emphasize the connection to the use of the name. Note that, by
virtue of name equivalence, $\quotep{\procn{x}} \nameeq x$; so, the
notation is consistent with previous definitions.

Further, because names have structure it is possible to effect
substitutions on the basis of that structure. This means we need to
upgrade our notation for substitutions, which we accomplish by
adapting comprehension notation. Thus,

\begin{mathpar}
  P\{ y / x : x \in S \}
\end{mathpar}

is interpreted to mean the process derived from P by replacing (in a
capture-avoiding manner) each occurrence of $x$ in $S$ by $y$. For example,

\begin{mathpar}
  P\{ \quotep{\procn{x}|\procn{x}} / x : x \in \freenames{P} \}
\end{mathpar}

will replace each (occurrence) of a free name $x$ in $P$ by
$\quotep{\procn{x}|\procn{x}}$.

Also, we will avail ourselves of the notation $x^{L}$ and $x^{R}$ to
denote injections of a name into disjoint copies of the name
space. There are numerous ways to accomplish this. One example can be
found in \cite{MeredithR05}. This notation overloads to vectors of
names: $\vec{x}^{\pi} := (x_{i}^{\pi} \; : \; 0 \leq i < |\vec{x}| )$ where $\pi \in \{L,R\}$.

We also use $P^{\Box} := P|\Box$.

In \cite{MeredithR05} an interpretation of the new operator is
given. It turns out that there are several possible interpretations
all enjoying the requisite algebraic properties of the operator (see
\cite{milner91polyadicpi}). We will therefore make liberal use of
$(\nu\; \vec{x})P$.

% subsection the_syntax_and_semantics_of_the_notation_system (end)   

\section{Interpretation of QM}
\subsection{Supporting definitions}
\subsubsection{Multiplication}
\begin{mathpar}
  \quotep{Q} \cdot \quotep{R} := \quotep{Q|R}
  \and \\
  \quotep{Q} \cdot P := P\{ \quotep{Q|R} / \quotep{R} : \quotep{R} \in \freenames{P} \}
\end{mathpar}

\paragraph{Discussion}
The first line needs little explanation. The second line says that
each free name of the process is replaced with the multiplication of
that name by the scalar. Multiplication of a scalar (name) by a state
(process) results in a process all the names of which have been `moved
over' by parallel composition with the process the scalar
quotes. There is a subtlety that the bound names have to be
manipulated so that multiplied names aren't accidentally
captured. There are many ways to achieve this.

\begin{remark}\label{rem:multiplication_identities}
  The reader is invited to verify that for all $x,y,z \in \QProc$ and $P \in \Proc$
  \begin{mathpar}
    x \cdot \quotep{0} \equiv x 
    \and
    x \cdot y \equiv y \cdot x
    \and
    x \cdot (y \cdot z) \equiv (x \cdot y) \cdot z
    \and \\
    \quotep{0} \cdot P \equiv P
    \and \\
    x \cdot (y \cdot P) \equiv (x \cdot y) \cdot P
    \and \\
    x \cdot (P|Q) \equiv (x \cdot P) | (x \cdot Q)
    \and \\    
  \end{mathpar}
\end{remark}

\subsubsection{Tensor product}

We define a tensor product on processes by structural induction.

\paragraph{Tensor of sums} First note that all summations, including
$\pzero$ and sequence, can be written $\Sigma_{i} x_{i}.A_{i} +
\Sigma_{j} x_{j}.C_{j}$, where we have grouped input-guarded processes
together and output-guarded processes together.

Thus, we can define the tensor product of two summations, $N_{1}\otimes N_{2}$, where

\begin{mathpar}
  N_{1} := \Sigma_{i} x_{i}.A_{i} + \Sigma_{j} x_{j}.C_{j}
  \and
  N_{2} := \Sigma_{i'} y_{i'}.B_{i'} + \Sigma_{j'} y_{j'}.D_{j'} 
\end{mathpar}

as follows.

\begin{mathpar}
  \Sigma_{i} x_{i}.A_{i} + \Sigma_{j} x_{j}.C_{j} \otimes \Sigma_{i'}
  y_{i'}.B_{i'} + \Sigma_{j'} y_{j'}.D_{j'} 
  \and \\
  := \; \Sigma_{i} \Sigma_{i'} \quotep{\stackrel{\vee}{x_{i}}| \stackrel{\vee}{y_{i'}}}.(A_{i}\otimes B_{i'}) \; | \; \Sigma_{i'} \Sigma_{i} \quotep{\stackrel{\vee}{y_{i'}}|\stackrel{\vee}{x_{i}}}.(B_{i'}\otimes A_{i})
  \and
  \;\; | \;\; \Sigma_{j} \Sigma_{j'} \quotep{\stackrel{\vee}{x_{j}}|\stackrel{\vee}{y_{j'}}}.(A_{j}\otimes B_{j'}) \; | \; \Sigma_{j'} \Sigma_{j} \quotep{\stackrel{\vee}{y_{j'}}|\stackrel{\vee}{x_{j}}}.(B_{j'}\otimes A_{j})
\end{mathpar}

\begin{remark}
  Do we need to $x^{L}$ and $y^{R}$ for this construction as well?
\end{remark}

\paragraph{Tensor of parallel compositions} Next, we distribute tensor
over par.

\begin{mathpar}
  P_{1}|P_{2} \otimes Q_{1}|Q_{2} := (P_{1} \otimes Q_{1}) | (P_{1}
  \otimes Q_{2}) | (P_{2} \otimes Q_{1}) | (P_{2} \otimes Q_{2})
\end{mathpar}

\paragraph{Tensor with dropped names} We treat tensor of a
process with a dropped name as parallel composition.

\begin{mathpar}
  P \otimes \dropn{x} := P | \dropn{x}
\end{mathpar}

\paragraph{Tensor of agents}

Finally, we need to define tensor on agents. Note that the definition
of tensor on normal products only tensors inputs with inputs and
outputs with outputs. Thus, we only have to define the operation on
``homogeneous'' pairings.

\begin{mathpar}
  (\vec{x})P \otimes (\vec{y})Q
  \and \\
  := (x_{0}^{L}|y_{0}^{R},\ldots,x_{0}^{L}|y_{n}^{R},\ldots,x_{m}^{L}|y_{0}^{R},\ldots,x_{m}^{L}|y_{n}^R)(P\{ \vec{x}^{L}/\vec{x}\} \otimes Q \{ \vec{y}^{R}/\vec{y}\})
  \and \\
  \clift{\vec{P}} \otimes \clift{\vec{Q}}
  \and \\
  := \clift{P_{0}\otimes Q_{0},\ldots,P_{0}\otimes Q_{n},\ldots,P_{m}\otimes Q_{0},\ldots,P_{m}\otimes Q_{n}}
\end{mathpar}

\begin{remark}
  Observe that arities of tensored abstractions matches arities of
  tensored concretions if the original arities matched. Note also that
  the length of the arities corresponds to the increase in dimension
  we see in ordinary vector space tensor product.
\end{remark}

\begin{remark}
  Operationally, this definition distributes the tensor down to
  components ``linked'' by summation. Tensor over summation is
  intriguing in that it mixes names. Moreover, as a consequence of the
  way it mixes names we have the identities for all $x \in \QProc$ and
  $P,Q \in \Proc$

  \begin{mathpar}
    (x \cdot P) \otimes Q \equiv x \cdot (P \otimes Q) \equiv P \otimes (x \cdot Q)
    \and
    P \otimes \pzero \equiv P
  \end{mathpar}

  that the reader is invited to verify.
\end{remark}

\subsubsection{Annihilation}
\begin{mathpar}
  P^{\perp} := \{ Q | \forall R. P|Q \red^{*} R \Rightarrow R \red^{*} \pzero \}
  \and \\
  P^{\underline{\perp}} := \Sigma_{Q \in P^{\perp}} \quotep{Q}?(y).(\dropn{y}|Q) | \Sigma_{Q \in P^{\perp}} \quotep{Q}\clift{\Box}
\end{mathpar}

\paragraph{Discussion} The reader will note that $P^{\perp}$ is a
\emph{set} of processes, while $P^{\underline{\perp}}$ is a
\emph{context}. We call the set $P^{\perp}$ the \emph{annihilators} of
$P$. The parallel composition of a process in the annihilators of $P$
with $P$ will result in a process, the state space of which has all
paths eventually leading to $\pzero$. Execution may endure loops; but
under reasonable conditions of fairness (naturally guaranteed under
most notions of bisimulation) such a composite process cannot get
stuck in such a loop and will, eventually pop out and terminate.

The context $P^{\underline{\perp}}$ is ready and willing to ``take the
$P$ out of'' the process to which it is applied. It will effectively
transmit the code of the process to which it is applied to one of the
annihilators and run the process against it.

\subsubsection{Evaluation}
We fix $M$ a domain of fully abstract interpretation with an equality
coincident with bisimulation. We take $\meaningof{\cdot} : \Proc \to
M$ to be the map interpreting processes and $\nmeaningof{\cdot} : \M
\to Proc$ to be the map running the other way. Then we define

\begin{mathpar}
  \int P := \nmeaningof{\meaningof{P}}
\end{mathpar}

\paragraph{Discussion}
There are many fully abstract interpretations of Milner's
$\pi$-calculus. Any of them can be used as a basis for interpreting
the reflective calculus here. Equipped with such a domain it is
largely a matter of grinding through to check that the Yoneda
construction for the normalization-by-evaluation program can be
extended to this setting.

\begin{remark}
  The reader is invited to verify that $\int (P^{\underline{\perp}}[P]) = 0$.
\end{remark}

\subsection{Quantum mechanics}

Table \ref{tbl:core_qm_op_defns} gives the core operational definitions

\begin{table}[htp]\label{tbl:core_qm_op_defns}
  \center{
    \fbox{
      \begin{tabular}{c|c}
        quantum mechanics & process calculus \\
        \hline
        scalar & $x := \quotep{P}$ \\
        state vector & $\state{P} := P$ \\
        dual & $\state{P}^{*} := \event{P^{\underline{\perp}}} := \quotep{P^{\underline{\perp}}}[-]$ \\
        matrix & $ \Sigma_{\alpha} \state{P_{\alpha}}x_{\alpha}\event{Q_{\alpha}}$ \\
        vector addition & $\state{P} + \state{Q} := \state{P | Q}$ \\
        tensor product & $\state{P} \otimes \state{Q} := \state{P \otimes Q}$ \\
        inner product & $\innerprod{P}{Q} := \quotep{\int P^{\underline{\perp}}[Q]}$ \\
      \end{tabular}
    }
  }
  \caption{QM - operational definitions}
\end{table}

where

\begin{mathpar}
  \prmatrix{P}{Q} := \fprmatrix{P}{\quotep{\pzero}}{Q}
  \and
  \fprmatrix{P}{x}{Q} := (\state{P},x,\event{Q})
  \and
  (\fprmatrix{P}{x}{Q})(\state{R}) := x \cdot \innerprod{Q}{R} \cdot \state{P}
  \and
  (\fprmatrix{P}{x}{Q})(\event{R}) := x \cdot \innerprod{R}{P} \cdot \event{Q}
\end{mathpar}

\paragraph{Discussion}
As promised: vectors (aka states) are represented as processes; duals
as contextual duals; inner product definition should be compared with
standard inner product definition for ....

\begin{remark}
  Assuming $\int (P^{\underline{\perp}}[P]) = 0$, the reader is
  invited to verify that $(\fprmatrix{P}{x}{P})(\state{P}) = x \cdot \state{P}$.
\end{remark}

\begin{remark}
  The reader is invited to verify that $\innerprod{P}{Q}$ could
  equally well have been written $\quotep{\int \stackrel{\vee}{x}}$
  where $x = \event{P^{\underline{\perp}}}(Q)$.

  One of the motivations for this remark is that there is another way
  to factor these operations. We could package up evaluation in the dual:

  \begin{mathpar}
    \state{P}^{*} := \event{\int P^{\underline{\perp}}} := \quotep{\int P^{\underline{\perp}}}[-]
  \end{mathpar}

  and then have inner product defined by
  
  \begin{mathpar}
    \innerprod{P}{Q} := \event{P}(Q)
  \end{mathpar}

  Hopefully, experience with the calculations will provide guidance on
  the best factoring.
\end{remark}

\begin{remark}
  Assuming $\int (P^{\underline{\perp}}[P]) = 0$, the reader is
  invited to verify that $\forall P,Q. (\prmatrix{0}{Q})(\state{0}) =
  \state{0}$ and dually $(\prmatrix{P}{0})(\event{0}) = \event{0}$.
\end{remark}

\begin{remark}
  i'm a little worried that i don't (yet) have proper support for
  complex conjugacy. But, the observation above may give us a
  clue. According to Abramsky, it must be the case that the scalars
  are iso to the homset of the identity for the tensor -- which the
  observation above characterizes. 

  For now, we will simply bookmark the notion with $\overline{x}$.
\end{remark}

\subsubsection{Adjointness}

We need to give a definition of $(\cdot)^{\dagger}$ for matrices. The
obvious candidate definition is
\begin{mathpar}
(\Sigma_{\alpha}\fprmatrix{P_{\alpha}}{x_{\alpha}}{Q_{\alpha}})^{\dagger}
= \Sigma_{\alpha}\fprmatrix{(Q_{\alpha}^{\underline{\perp}})^{*}}{\overline{x}_{\alpha}}{P_{\alpha}^{\underline{\perp}}} 
\end{mathpar}

But, $(Q_{\alpha}^{\underline{\perp}})^{*}$ requires a name along
which to communicate the process to achieve the context application.

\subsubsection{Basis for a basis}
If processes label states and ``addition'' of states (a.k.a. vector
addition) is interpreted as parallel composition, what corresponds to
notions of linear independence and basis? Here, we recall that Yoshida
has developed a set of \emph{combinators} for an asynchronous verison
of Milner's $\pi$-calculus. These are a finite set of processes such
any process can be expressed as parallel composition of these
combinators together with liberal uses of the new operator and
replication. We can simply give a translation of these into the
present calculus and have reasonable expectation that the property
carries over. That is, that the resultant set allows to express all
processes via parallel composition. Note, however, that there is no
new operator or replication in this calculus. As a result, we expect
that the corresponding set is actually infinite. That is, we expect
that the space is actually infinite dimensional.

\begin{remark}
  The attentive reader may be a bit concerned. Certainly, the
  collection $S$, $K$ and $I$ is a finite set of
  combinators. Shouldn't we expect to see a finite set of combinators
  for an effectively equivalent system? i am very sympathetic to this
  critique and feel it warrants full attention. On the other hand, i
  also have in mind the following analogy. The natural numbers, as a
  monoid under addition, has exactly $1$ generator, while the natural
  numbers, as a monoid under multiplication, has countably many
  generators (the primes). We observe that the application of the
  lambda calculus is much less resource sensitive than the parallel
  composition of the $\pi$-calculus. Could it be the case that we have
  an analogy of the form
  
  \begin{mathpar}
    m + n : MN :: m*n : M|N
  \end{mathpar}

  giving a similar blow up in the set of ``primes''?  This is such a
  wonderful thought that, even if it's not true, i think it's worth
  writing down.
\end{remark}
 

\documentclass[12pt]{llncs}
%\documentclass{jktr}

\usepackage[pdftex]{hyperref}                   
\usepackage {listings}
\usepackage {mathpartir}
\usepackage{bcprules}
%\usepackage{listings}
                       
\usepackage{graphicx} 
%\usepackage[margins=2.5cm,nohead,nofoot]{geometry}
%\usepackage{geometry}
\usepackage{amsfonts}
\usepackage{amstext}
\usepackage{latexsym}
\usepackage{amssymb}
\usepackage{color}


%\include{myPreamble}
\include{qm2pi.local} 

%\ifpdf
%\usepackage[pdftex]{graphicx}
%\else
%\usepackage{graphicx}
%\fi

 % \ifpdf
%  \usepackage{pdfsync}
%  \if


%\title{Brief Article}
%\author{David F. Snyder}
%\author{L.G. Meredith}

%\address{Dept. of Math., Texas State University--San Marcos, San Marcos, TX 78666}
       
\pagestyle{empty}


\begin{document}

\lstset{language=[Objective]Caml,frame=shadowbox}

\input{qm2pi.front}

% section front matter (end)

\input{qm2pi.intro} 
 
% section introduction (end)

% \input{qm2pi.knotations} 

% section notation (end)

\input{qm2pi.process.calculi} 

% section concurrent_process_calculi_and_spatial_logics_ (end)
    
%\input{qm2pi.knots2pi} 

%\input{qm2pi.trefoil} 

%\input{qm2pi.mainthm} 

% subsection basic_interpretation (end)

%\input{qm2pi.rho.presentation} 
\subsection{The syntax and semantics of the notation system}\label{sub:the_syntax_and_semantics_of_the_notation_system} % (fold)

We now summarize a technical presentation of the calculus that
embodies our theory of dynamics. The typical presentation of such a
calculus follows the style of giving generators and relations on
them. The grammar, below, describing term constructors, freely
generates the set of processes, $\Proc$. This set is then quotiented
by a relation known as structural congruence and it is over this set
that the notion of dynamics is expressed. This presentation is
essentially that of \cite{MeredithR05} with the addition of
polyadicity and summation. For readability we have relegated some of
the technical subtleties to an appendix.

\subsubsection{Process grammar}\label{subsub:process_grammar}

\begin{mathpar}
  \inferrule* [lab=synchronization] {} {{M} \bc \pzero \;|\; x?F \;|\; x!C }
  \and
  \inferrule* [lab=abstraction] {} {{F} \bc (x)P}
  \and
  \inferrule* [lab=concretion] {} {{C} \bc \langle Q \rangle}
  \and
  \inferrule* [lab=process] {} {{P,Q} \bc M \;| \;P|Q \;|\; @{x}}
  \and
  \inferrule* [lab=name] {} {{x} \bc \quotep{P}}
\end{mathpar} 

Note that $\vec{x}$ (resp. $\vec{P}$) denotes a vector of names
(resp. processes) of length $|\vec{x}|$ (resp. $|\vec{P}|$). We adopt
the following useful abbreviations.

\begin{mathpar}
   x?(\vec{y}).P := x.(\vec{y})P \and  x\clift{\vec{P}} := x.\clift{\vec{P}}
   \and x!(y) := \lift{x}{\dropn{y}}
   \and \Pi_{i=0}^{n-1}P_i := P_0 | \ldots | P_{n-1}
\end{mathpar}

\subsubsection{Structural congruence}

\paragraph{Free and bound names and alpha-equivalence.} At the
core of structural equivalence is alpha-equivalence which identifies
process that are the same up to a change of variable. Formally, we
recognize the distinction between free and bound names. The free names
of a process, $\freenames{P}$, may be calculated recursively as
follows:

\begin{mathpar}
\freenames{\pzero} := \emptyset
  \and \\
  \freenames{x?(y).P} := \{ x \} \cup (\freenames{P} \setminus \{ y \})
  \and 
  \freenames{x!\langle P \rangle} := \{ x \} \cup \{ P \} 
  \and \\
  \freenames{P|Q} := \freenames{P} \cup \freenames{Q}
  \and \\
  \freenames{@{x}} := \{ x \}
\end{mathpar}

$\pi$
$\quotep{\pi}$

$\freenames{-} : \pi \to \mathcal{P}(\quotep{\pi})$

\begin{eqnarray*}
  \freenames{\pzero} & := & \emptyset \\
  \freenames{x?(y).P} & := & \{ x \} \cup (\freenames{P} \setminus \{ y \}) \\
  \freenames{x!\langle P \rangle} & := & \{ x \} \cup \{ P \} \\
  \freenames{P|Q} & := & \freenames{P} \cup \freenames{Q} \\
  \freenames{\dropn{x}} & := & \{ x \}
\end{eqnarray*}

The bound names of a process, $\boundnames{P}$, are those names occurring in $P$
that are not free. For example, in $x?(y).0$, the name $x$ is free, while $y$ is bound.

\begin{mathpar}
  \inferrule* [lab=monoidal-laws] {} { P|Q \equiv Q|P \and P|0 \equiv P \and P|(Q|R) \equiv (P|Q)|R }
\end{mathpar}

\begin{mathpar}
  \inferrule* [lab=alpha-equivalence] {} { (x)P \equiv (y)P\{y/x\} \and y \not\in \freenames{P} }
\end{mathpar}

\begin{definition}
Then two processes, $P,Q$, are alpha-equivalent if $P = Q\{\vec{y}/\vec{x}\}$ for
some $\vec{x} \in \boundnames{Q},\vec{y} \in \boundnames{P}$, where $Q\{\vec{y}/\vec{x}\}$
denotes the capture-avoiding substitution of $\vec{y}$ for $\vec{x}$ in $Q$.
\end{definition}

\begin{definition}
  The {\em structural congruence} \cite{SangiorgiWalker} , $\equiv$,
  between processes is the least congruence containing
  alpha-equivalence, satisfying the abelian monoid laws
  (associativity, commutativity and $\pzero$ as identity) for parallel
  composition $|$ and for summation $+$.
\end{definition}

\subsection{Name equivalence}

We take name equivalence, written $\nameeq$, to be the smallest
equivalence relation generated by the following rules.

\begin{mathpar}
\inferrule*[lab=Quote-drop]
{ }
{ \quotep{@{x}} \nameeq x }

\inferrule*[lab=Struct-equiv]
{ P \scong Q }
{ \quotep{P} \nameeq \quotep{Q} }
\end{mathpar}

The astute reader will have noticed that the mutual recursion of names
and processes imposes a mutual recursion on alpha-equivalence and
structural equivalence via name-equivalence. Fortunately, all of this
works out pleasantly and we may calculate in the natural way, free of
concern. The reader interested in the details is referred to the
appendix \ref{appendix:rho_details}.

\subsection{Substitution}

We use $\Proc$ for the set of processes, $\QProc$ for the set of
names, and $\id{\{}\vec{y} / \vec{x} \id{\}}$ to denote partial maps,
$s : \QProc \rightarrow \QProc$. A map, $s$ lifts, uniquely, to a map
on process terms, $\widehat{s} : \Proc \rightarrow \Proc$ by the
following equations.

\begin{mathpar}
  (0) \psubstp{Q}{P} := 0 \\
  (R \juxtap S) \psubstp{Q}{P}
  :=    
  (R)\psubstp{Q}{P} \juxtap (S) \psubstp{Q}{P} \\
  (x?(y).R) \psubstp{Q}{P}    
  :=    
  (x)\substp{Q}{P} (z)\concat( (R \psubstn{z}{y}) \psubstp{Q}{P} ) \\
  (\lift{x}{R}) \psubstp{Q}{P}  
  :=
  \lift{(x)\substp{Q}{P}}{ R \psubstp{Q}{P} } \\
%   (\dropn{x})  \psubstp{Q}{P}       
%   := 
%   \left\{ 
%     \begin{array}{ccc} 
%       \dropn{\quotep{Q}} & & x \nameeq \quotep{P} \\
%       \dropn{x} & & otherwise \\
%     \end{array}
%   \right. 
  (\dropn{x})  \psubstp{Q}{P}       
  := 
  \left\{ 
    \begin{array}{ccc} 
      Q & & x \nameeq \quotep{P} \\
      \dropn{x} & & otherwise \\
    \end{array}
  \right.
\end{mathpar}
 

where

\begin{eqnarray}
  (x)\id{\{} \lpquote Q \rpquote / \lpquote P \rpquote \id{\}}            = 
  \left\{ 
    \begin{array}{ccc}
      \lpquote Q \rpquote & & x \nameeq \lpquote P \rpquote \\
      x & & otherwise \\
    \end{array}
  \right. \nonumber
\end{eqnarray}

and $z$ is chosen distinct from $\quotep{P}$, $\quotep{Q}$, the free
names in $Q$, and all the names in $R$. Our $\alpha$-equivalence will
be built in the standard way from this substitution.

\begin{remark}\label{rem:no_self_referential_names}
  One consequence of these definitions is that $\forall P. \quotep{P}
  \not\in \freenames{P}$.
\end{remark}

\subsection{ Dynamic quote: an example }

Anticipating something of what's to come, consider applying the
substitution, $\widehat{\id{\{}u / z \id{\}}}$, to the following pair
of processes, $\lift{w}{y!(z)}$ and $w[ \lpquote y!(z) \rpquote ]$.

\begin{eqnarray}
	\lift{w}{y!(z)}\widehat{\id{\{}u / z \id{\}}}
		& = &
		\lift{w}{y!(u)} \nonumber\\
	w[ \lpquote y!(z) \rpquote ] \widehat{ \id{\{}u / z \id{\}} }
		& = &
		w[ \lpquote y!(z) \rpquote ] \nonumber
\end{eqnarray}

Because the body of the process between quotes is impervious to
substitution, we get radically different answers. In fact, by
examining the first process in an input context,
e.g. $x?(z).\lift{w}{y!(z)}$, we see that the process under the lift
operator may be shaped by prefixed inputs binding a name inside it. In
this sense, the lift operator will be seen as a way to dynamically
construct processes before reifying them as names.

Finally equipped with these standard features we can present the
dynamics of the calculus.

\subsubsection{Operational semantics} 

Finally, we introduce the computational dynamics. What marks these
algebras as distinct from other more traditionally studied algebraic
structures, e.g. vector spaces or polynomial rings, is the manner in
which dynamics is captured. In traditional structures, dynamics is typically
expressed through morphisms between such structures, as in linear maps
between vector spaces or morphisms between rings. In algebras
associated with the semantics of computation, the dynamics is
expressed as part of the algebraic structure itself, through a
reduction reduction relation typically denoted by $\red$. Below, we
give a recursive presentation of this relation for the calculus used
in the encoding.

$\red \subseteq \pi \times \pi$
$\red : \pi \to \mathcal{P}(\pi)$

\begin{mathpar}
  \inferrule* [lab=Comm] { \textsf{match}( x_{src}, x_{trgt} ) } { x_{trgt}?(y)P \; | \; x_{src}!\langle {Q} \rangle \red P\{\quotep{Q}/y}\} }
  \and \\
  \inferrule* [lab=Par] {{P} \red {P}'} {{{P} | {Q}} \red {{P}' | {Q}}}
  \and
  \inferrule* [lab=Equiv]{{{P} \scong {P}'} \andalso {{P}' \red {Q}'} \andalso {{Q}' \scong {Q}}}{{P} \red {Q}}
\end{mathpar}

\begin{eqnarray*}
  match_{\equiv} (\quotep{P},\quotep{Q}) & := & P \equiv Q \\
  match_{\dagger}(\quotep{P},\quotep{Q}) & := & \forall R. P|Q \red^{*} R => R \red^{*} 0 \\
  match_{K}(\quotep{P},\quotep{Q}) & := & K \mbox{ for some context } K
\end{eqnarray*}

$u?(x)P | u!\langle Q \rangle \red P\{\quotep{Q}/x\}$

%We write $\wred$ for $\red^*$, and $P\red$ if $\exists Q $ such that $ P \red Q$.
We write $P\red$ if $\exists Q $ such that $ P \red Q$ and $P\not\red$, otherwise.

\section{Replication}

As mentioned before, it is known that replication (and hence
recursion) can be implemented in a higher-order process algebra
\cite{SangiorgiWalker}. As our first example of calculation with the
machinery thus far presented we give the construction explicitly in
the {\rhoc}.

\begin{eqnarray}
	D_{x} & := & \prefix{x}{y}{(\binpar{\outputp{x}{y}}{@{y}})} \nonumber\\
	\bangp_{x}{P} & := & \binpar{{x}!\langle{\binpar{D_{x}}{P}}\rangle}{D_{x}} \nonumber
\end{eqnarray}

\begin{eqnarray}
	\bangp_{x}{P} & & \nonumber\\
	=
	& {x}!\langle{(\prefix{x}{y}{(\outputp{x}{y} | @{y})) | P}}\rangle 
	      | \prefix{x}{y}{(\outputp{x}{y} | @{y})} & \nonumber\\
	\red
	& (\outputp{x}{y} | @{y})\substn{\quotep{(\prefix{x}{y}{(@{y} | \outputp{x}{y})) | P}}}{y} & \nonumber\\
	=
	& \outputp{x}{\quotep{(\prefix{x}{y}{(\outputp{x}{y} | @{y})) | P}}}
	  | {(\prefix{x}{y}{(\outputp{x}{y} | @{y})) | P}} & \nonumber\\
	\red
	& \ldots & \nonumber\\
	\red^*
	& P | P | \ldots & \nonumber
\end{eqnarray}

Of course, this encoding, as an implementation, runs away, unfolding
$\bangp{P}$ eagerly. A lazier and more implementable replication
operator, restricted to input-guarded processes, may be obtained as follows.

\begin{eqnarray}
\bangp{\prefix{u}{v}{P}} 
	:= 
	\binpar{\lift{x}{\prefix{u}{v}{(\binpar{D(x)}{P})}}}{D(x)} \nonumber
\end{eqnarray}

\begin{remark}
  Note that the lazier definition still does not deal with summation
  or mixed summation (i.e. sums over input and output). The reader is
  invited to construct definitions of replication that deal with these
  features. 

  Further, the definitions are parameterized in a name, $x$. Can you,
  gentle reader, make a definition that eliminates this parameter and
  guarantees no accidental interaction between the replication
  machinery and the process being replicated -- i.e. no accidental
  sharing of names used by the process to get its work done and the
  name(s) used by the replication to effect copying. This latter
  revision of the definition of replication is crucial to obtaining
  the expected identity $!!P \sim !P$.
\end{remark}

\begin{remark}\label{rem:paradoxical_combinator}
  The reader familiar with the lambda calculus will have noticed the
  similarity between $D$ and the paradoxical combinator.

  [Ed. note: the existence of this seems to suggest we have to be more
  restrictive on the set of processes and names we admit if we are to
  support no-cloning.]
\end{remark}

\subsubsection{Bisimulation}

The computational dynamics gives rise to another kind of equivalence,
the equivalence of computational behavior. As previously mentioned
this is typically captured \emph{via} some form of bisimulation.

% The notion we use in this paper is weak barbed bisimulation
% \cite{milner91polyadicpi}.

The notion we use in this paper is derived from weak barbed
bisimulation \cite{milner91polyadicpi}. 

\begin{definition}
An \emph{observation relation}, $\downarrow_{\mathcal N}$, over a set
of names, $\mathcal N$, is the smallest relation satisfying the rules
below.

\infrule[Out-barb]{y \in {\mathcal N}, \; x \nameeq y}
		  {\outputp{x}{v} \downarrow_{\mathcal N} x}
\infrule[Par-barb]{\mbox{$P\downarrow_{\mathcal N} x$ or $Q\downarrow_{\mathcal N} x$}}
		  {\binpar{P}{Q} \downarrow_{\mathcal N} x}

We write $P \Downarrow_{\mathcal N} x$ if there is $Q$ such that 
$P \wred Q$ and $Q \downarrow_{\mathcal N} x$.
\end{definition}

\begin{definition}
%\label{def.bbisim}
An  ${\mathcal N}$-\emph{barbed bisimulation} over a set of names, ${\mathcal N}$, is a symmetric binary relation 
${\mathcal S}_{\mathcal N}$ between agents such that $P\rel{S}_{\mathcal N}Q$ implies:
\begin{enumerate}
\item If $P \red P'$ then $Q \wred Q'$ and $P'\rel{S}_{\mathcal N} Q'$.
\item If $P\downarrow_{\mathcal N} x$, then $Q\Downarrow_{\mathcal N} x$.
\end{enumerate}
$P$ is ${\mathcal N}$-barbed bisimilar to $Q$, written
$P \wbbisim_{\mathcal N} Q$, if $P \rel{S}_{\mathcal N} Q$ for some ${\mathcal N}$-barbed bisimulation ${\mathcal S}_{\mathcal N}$.
\end{definition}

$\mathcal{R} \subseteq \pi \times \pi$

$P \mathcal{R} Q => \forall P'. P \red P' \Rightarrow \exists Q'. Q \red Q', P' \mathcal{R} Q'$

$P \vdash x \Rightarrow Q \vdash x$

\begin{mathpar}
  \inferrule*[lab=Out-barb]{x \nameeq y}{{y}!\langle{Q}\rangle \vdash x}
  \and
  \inferrule*[lab=Par-barb]{\mbox{$P\vdash x$ or $Q\vdash x$}}{\binpar{P}{Q} \vdash x}
\end{mathpar}

\subsubsection{Contexts}

One of the principle advantages of computational calculi like the
$\pi$-calculus is a well-defined notion of context,
contextual-equivalence and a correlation between
contextual-equivalence and notions of bisimulation. The notion of
context allows the decomposition of a process into (sub-)process and
its syntactic environment, its context. Thus, a context may be
thought of as a process with a ``hole'' (written $\Box$) in it. The
application of a context $M$ to a process $P$, written $M[P]$, is
tantamount to filling the hole in $M$ with $P$. In this paper we do
not need the full weight of this theory, but do make use of the notion
of context in the proof the main theorem. 

\begin{mathpar}
  \inferrule* [lab=summation] {} {{M_{M},M_{N}} \bc \Box \;|\; x.M_{A} \;|\; M_{M}+M_{N}}
  \and
  \inferrule* [lab=agent] {} {{M_{A}} \bc (\vec{x})M_{P} \;| \; \clift{P_0,\ldots,M_{P},\ldots,P_N}}
  \and \\
  \inferrule* [lab=process] {} {{M_{P}} \bc M_{N} \;| \;P|M_{P} }
\end{mathpar} 

\begin{mathpar}
  \inferrule* [lab=sychronization] {} {M_{N} \bc \Box \;|\; x?M_{F} \;|\; x!M_{C}}
  \and
  \inferrule* [lab=abstraction] {} {{M_{F}} \bc (x)M_{P} }
  \and
  \inferrule* [lab=concretion] {} {{M_{C}} \bc \langle M_{P} \rangle }
  \and \\
  \inferrule* [lab=process] {} {{M_{P}} \bc M_{N} \;| \;P|M_{P} }
\end{mathpar}

\begin{definition}[contextual application] Given a context $M$, and
  process $P$, we define the \emph{contextual application}, $M[P] :=
  M\{P/\Box\}$. That is, the contextual application of M to P is the
  substitution of $P$ for $\Box$ in $M$.
\end{definition}

$\meaningof{-} : L \to \mathcal{P}(\pi)$

\begin{mathpar}
  \inferrule* [lab=collection] {} {\meaningof{true} = \pi, \and \meaningof{~E} = \pi \setminus \meaningof{E}, \and \meaningof{E_{1} \& E_{2}} = \meaningof{E_{1}} \cap \meaningof{E_{2}}}
\end{mathpar}

\begin{mathpar}
  \inferrule* [lab=structure] {} {\meaningof{0} = \{ P \in \pi | P \equiv 0 \}, \and \\ \meaningof{E_1 | E_2} = \{ P \in \pi | P \equiv P_{1} | P_{2}, P_{1} \in \meaningof{E_{1}}, P_{2} \in \meaningof{E_2}\} }
\end{mathpar}

\begin{mathpar}
 \inferrule* [lab=behavior] {} {\meaningof{\langle a?b \rangle E} = \{ P \in \pi | P \equiv Q | u?(y)P', \\ \and \\\\ \and \\ \;\;\; u \in \meaningof{a}, \forall z.P'\{z/y\} \in \meaningof{E\{z/b\}}\}, \and \\ \meaningof{a!E} = \{ P \in \pi | P \equiv Q | x!\langle P' \rangle, x \in \meaningof{a} P' \in \meaningof{E}\} }
\end{mathpar}

\begin{mathpar}
 \inferrule* [lab=nominal] {} {\meaningof{\quotep{E}} = \{ \quotep{P} \in \quotep{\pi} | P \in \meaningof{E} \}, \and \meaningof{\quotep{P}} = \{ \quotep{Q} \in \quotep{\pi} | P \equiv Q \} \and \\ \meaningof{@\quotep{E}} = \{ P \in \pi | P \equiv @x, x \in \meaningof{E} \}}
\end{mathpar}

\begin{eqnarray*}
  \\
  \meaningof{-} : TS \to ST
\end{eqnarray*}

\begin{eqnarray*}
  \\
  L : TS \to ST
\end{eqnarray*}

\begin{eqnarray*}
  \\
  P \models E \iff P \in \meaningof{E}
\end{eqnarray*}

\begin{eqnarray*}
  P \approx_{L} Q \iff \forall E \in L. P \models E \iff Q \models E
\end{eqnarray*}

\begin{eqnarray*}
  P \approx_{K} Q
\end{eqnarray*}

\begin{eqnarray*}
  P \approx Q
\end{eqnarray*}

$\approx_{K} = \approx = \approx_{L}$

\subsubsection{Contextual duality}

Note that contexts extend the quotation operation to a family of
operations from processes to names. Given a context, $M$, we can
define a \emph{nominal context}, $\quotep{M}$ by $\quotep{M}[P] :=
\quotep{M[P]}$. To foreshadow what is to come we observe that these
operations enjoy a duality with processes very much like the duality
between vectors and maps from vectors to scalars.

Further, because the calculus is essentially higher-order, we have a
correspondence between contexts and processes. More specifically,
given a name $x$ and a context $M$ we can construct $M^{*}_{x}$ such
that 

\begin{mathpar}
  M^{*}_{x} | \lift{x}{P} \red M[P]
\end{mathpar}

namely,

\begin{mathpar}
  M^{*}_{x} := x?(u).M[\dropn{u}]
\end{mathpar}

The dependence of $M^{*}_{x}$ on a name makes it an abstraction, 

\begin{mathpar}
  M^{*} := (x)x?(u).M[\dropn{u}]
\end{mathpar}

\subsection{Additional notation}

It will sometimes be convenient to denote the process a name
quotes. We already have the notation $x = \quotep{P}$, but it will be
convenient to introduce an alternate notation, $\procn{x}$, when we
want to emphasize the connection to the use of the name. Note that, by
virtue of name equivalence, $\quotep{\procn{x}} \nameeq x$; so, the
notation is consistent with previous definitions.

Further, because names have structure it is possible to effect
substitutions on the basis of that structure. This means we need to
upgrade our notation for substitutions, which we accomplish by
adapting comprehension notation. Thus,

\begin{mathpar}
  P\{ y / x : x \in S \}
\end{mathpar}

is interpreted to mean the process derived from P by replacing (in a
capture-avoiding manner) each occurrence of $x$ in $S$ by $y$. For example,

\begin{mathpar}
  P\{ \quotep{\procn{x}|\procn{x}} / x : x \in \freenames{P} \}
\end{mathpar}

will replace each (occurrence) of a free name $x$ in $P$ by
$\quotep{\procn{x}|\procn{x}}$.

Also, we will avail ourselves of the notation $x^{L}$ and $x^{R}$ to
denote injections of a name into disjoint copies of the name
space. There are numerous ways to accomplish this. One example can be
found in \cite{MeredithR05}. This notation overloads to vectors of
names: $\vec{x}^{\pi} := (x_{i}^{\pi} \; : \; 0 \leq i < |\vec{x}| )$ where $\pi \in \{L,R\}$.

We also use $P^{\Box} := P|\Box$.

In \cite{MeredithR05} an interpretation of the new operator is
given. It turns out that there are several possible interpretations
all enjoying the requisite algebraic properties of the operator (see
\cite{milner91polyadicpi}). We will therefore make liberal use of
$(\nu\; \vec{x})P$.

% subsection the_syntax_and_semantics_of_the_notation_system (end)   

\input{qm2pi.qmops} 

\input{qm2pi.sterngerlach} 

\input{qm2pi.metric} 

% section concurrent_process_calculi (end)

%\input{qm2pi.proofsketch}

% section proof sketch (end)

%\input{qm2pi.slviaknots} 

% section spatial logic via knots (end)

\input{qm2pi.conclusion}

% section conclusion (end)

%\input{qm2pi.dtcodes} 

% section wiring algorithm (end)

\input{qm2pi.ack} 

% section acknowledgments (end)

\newpage


\bibliographystyle{plain}   
\bibliography{../../biblios/main.bib}

\input{qm2pi.rhodetails}

\end{document}

 

\documentclass[12pt]{llncs}
%\documentclass{jktr}

\usepackage[pdftex]{hyperref}                   
\usepackage {listings}
\usepackage {mathpartir}
\usepackage{bcprules}
%\usepackage{listings}
                       
\usepackage{graphicx} 
%\usepackage[margins=2.5cm,nohead,nofoot]{geometry}
%\usepackage{geometry}
\usepackage{amsfonts}
\usepackage{amstext}
\usepackage{latexsym}
\usepackage{amssymb}
\usepackage{color}


%\include{myPreamble}
\include{qm2pi.local} 

%\ifpdf
%\usepackage[pdftex]{graphicx}
%\else
%\usepackage{graphicx}
%\fi

 % \ifpdf
%  \usepackage{pdfsync}
%  \if


%\title{Brief Article}
%\author{David F. Snyder}
%\author{L.G. Meredith}

%\address{Dept. of Math., Texas State University--San Marcos, San Marcos, TX 78666}
       
\pagestyle{empty}


\begin{document}

\lstset{language=[Objective]Caml,frame=shadowbox}

\input{qm2pi.front}

% section front matter (end)

\input{qm2pi.intro} 
 
% section introduction (end)

% \input{qm2pi.knotations} 

% section notation (end)

\input{qm2pi.process.calculi} 

% section concurrent_process_calculi_and_spatial_logics_ (end)
    
%\input{qm2pi.knots2pi} 

%\input{qm2pi.trefoil} 

%\input{qm2pi.mainthm} 

% subsection basic_interpretation (end)

%\input{qm2pi.rho.presentation} 
\subsection{The syntax and semantics of the notation system}\label{sub:the_syntax_and_semantics_of_the_notation_system} % (fold)

We now summarize a technical presentation of the calculus that
embodies our theory of dynamics. The typical presentation of such a
calculus follows the style of giving generators and relations on
them. The grammar, below, describing term constructors, freely
generates the set of processes, $\Proc$. This set is then quotiented
by a relation known as structural congruence and it is over this set
that the notion of dynamics is expressed. This presentation is
essentially that of \cite{MeredithR05} with the addition of
polyadicity and summation. For readability we have relegated some of
the technical subtleties to an appendix.

\subsubsection{Process grammar}\label{subsub:process_grammar}

\begin{mathpar}
  \inferrule* [lab=synchronization] {} {{M} \bc \pzero \;|\; x?F \;|\; x!C }
  \and
  \inferrule* [lab=abstraction] {} {{F} \bc (x)P}
  \and
  \inferrule* [lab=concretion] {} {{C} \bc \langle Q \rangle}
  \and
  \inferrule* [lab=process] {} {{P,Q} \bc M \;| \;P|Q \;|\; @{x}}
  \and
  \inferrule* [lab=name] {} {{x} \bc \quotep{P}}
\end{mathpar} 

Note that $\vec{x}$ (resp. $\vec{P}$) denotes a vector of names
(resp. processes) of length $|\vec{x}|$ (resp. $|\vec{P}|$). We adopt
the following useful abbreviations.

\begin{mathpar}
   x?(\vec{y}).P := x.(\vec{y})P \and  x\clift{\vec{P}} := x.\clift{\vec{P}}
   \and x!(y) := \lift{x}{\dropn{y}}
   \and \Pi_{i=0}^{n-1}P_i := P_0 | \ldots | P_{n-1}
\end{mathpar}

\subsubsection{Structural congruence}

\paragraph{Free and bound names and alpha-equivalence.} At the
core of structural equivalence is alpha-equivalence which identifies
process that are the same up to a change of variable. Formally, we
recognize the distinction between free and bound names. The free names
of a process, $\freenames{P}$, may be calculated recursively as
follows:

\begin{mathpar}
\freenames{\pzero} := \emptyset
  \and \\
  \freenames{x?(y).P} := \{ x \} \cup (\freenames{P} \setminus \{ y \})
  \and 
  \freenames{x!\langle P \rangle} := \{ x \} \cup \{ P \} 
  \and \\
  \freenames{P|Q} := \freenames{P} \cup \freenames{Q}
  \and \\
  \freenames{@{x}} := \{ x \}
\end{mathpar}

$\pi$
$\quotep{\pi}$

$\freenames{-} : \pi \to \mathcal{P}(\quotep{\pi})$

\begin{eqnarray*}
  \freenames{\pzero} & := & \emptyset \\
  \freenames{x?(y).P} & := & \{ x \} \cup (\freenames{P} \setminus \{ y \}) \\
  \freenames{x!\langle P \rangle} & := & \{ x \} \cup \{ P \} \\
  \freenames{P|Q} & := & \freenames{P} \cup \freenames{Q} \\
  \freenames{\dropn{x}} & := & \{ x \}
\end{eqnarray*}

The bound names of a process, $\boundnames{P}$, are those names occurring in $P$
that are not free. For example, in $x?(y).0$, the name $x$ is free, while $y$ is bound.

\begin{mathpar}
  \inferrule* [lab=monoidal-laws] {} { P|Q \equiv Q|P \and P|0 \equiv P \and P|(Q|R) \equiv (P|Q)|R }
\end{mathpar}

\begin{mathpar}
  \inferrule* [lab=alpha-equivalence] {} { (x)P \equiv (y)P\{y/x\} \and y \not\in \freenames{P} }
\end{mathpar}

\begin{definition}
Then two processes, $P,Q$, are alpha-equivalent if $P = Q\{\vec{y}/\vec{x}\}$ for
some $\vec{x} \in \boundnames{Q},\vec{y} \in \boundnames{P}$, where $Q\{\vec{y}/\vec{x}\}$
denotes the capture-avoiding substitution of $\vec{y}$ for $\vec{x}$ in $Q$.
\end{definition}

\begin{definition}
  The {\em structural congruence} \cite{SangiorgiWalker} , $\equiv$,
  between processes is the least congruence containing
  alpha-equivalence, satisfying the abelian monoid laws
  (associativity, commutativity and $\pzero$ as identity) for parallel
  composition $|$ and for summation $+$.
\end{definition}

\subsection{Name equivalence}

We take name equivalence, written $\nameeq$, to be the smallest
equivalence relation generated by the following rules.

\begin{mathpar}
\inferrule*[lab=Quote-drop]
{ }
{ \quotep{@{x}} \nameeq x }

\inferrule*[lab=Struct-equiv]
{ P \scong Q }
{ \quotep{P} \nameeq \quotep{Q} }
\end{mathpar}

The astute reader will have noticed that the mutual recursion of names
and processes imposes a mutual recursion on alpha-equivalence and
structural equivalence via name-equivalence. Fortunately, all of this
works out pleasantly and we may calculate in the natural way, free of
concern. The reader interested in the details is referred to the
appendix \ref{appendix:rho_details}.

\subsection{Substitution}

We use $\Proc$ for the set of processes, $\QProc$ for the set of
names, and $\id{\{}\vec{y} / \vec{x} \id{\}}$ to denote partial maps,
$s : \QProc \rightarrow \QProc$. A map, $s$ lifts, uniquely, to a map
on process terms, $\widehat{s} : \Proc \rightarrow \Proc$ by the
following equations.

\begin{mathpar}
  (0) \psubstp{Q}{P} := 0 \\
  (R \juxtap S) \psubstp{Q}{P}
  :=    
  (R)\psubstp{Q}{P} \juxtap (S) \psubstp{Q}{P} \\
  (x?(y).R) \psubstp{Q}{P}    
  :=    
  (x)\substp{Q}{P} (z)\concat( (R \psubstn{z}{y}) \psubstp{Q}{P} ) \\
  (\lift{x}{R}) \psubstp{Q}{P}  
  :=
  \lift{(x)\substp{Q}{P}}{ R \psubstp{Q}{P} } \\
%   (\dropn{x})  \psubstp{Q}{P}       
%   := 
%   \left\{ 
%     \begin{array}{ccc} 
%       \dropn{\quotep{Q}} & & x \nameeq \quotep{P} \\
%       \dropn{x} & & otherwise \\
%     \end{array}
%   \right. 
  (\dropn{x})  \psubstp{Q}{P}       
  := 
  \left\{ 
    \begin{array}{ccc} 
      Q & & x \nameeq \quotep{P} \\
      \dropn{x} & & otherwise \\
    \end{array}
  \right.
\end{mathpar}
 

where

\begin{eqnarray}
  (x)\id{\{} \lpquote Q \rpquote / \lpquote P \rpquote \id{\}}            = 
  \left\{ 
    \begin{array}{ccc}
      \lpquote Q \rpquote & & x \nameeq \lpquote P \rpquote \\
      x & & otherwise \\
    \end{array}
  \right. \nonumber
\end{eqnarray}

and $z$ is chosen distinct from $\quotep{P}$, $\quotep{Q}$, the free
names in $Q$, and all the names in $R$. Our $\alpha$-equivalence will
be built in the standard way from this substitution.

\begin{remark}\label{rem:no_self_referential_names}
  One consequence of these definitions is that $\forall P. \quotep{P}
  \not\in \freenames{P}$.
\end{remark}

\subsection{ Dynamic quote: an example }

Anticipating something of what's to come, consider applying the
substitution, $\widehat{\id{\{}u / z \id{\}}}$, to the following pair
of processes, $\lift{w}{y!(z)}$ and $w[ \lpquote y!(z) \rpquote ]$.

\begin{eqnarray}
	\lift{w}{y!(z)}\widehat{\id{\{}u / z \id{\}}}
		& = &
		\lift{w}{y!(u)} \nonumber\\
	w[ \lpquote y!(z) \rpquote ] \widehat{ \id{\{}u / z \id{\}} }
		& = &
		w[ \lpquote y!(z) \rpquote ] \nonumber
\end{eqnarray}

Because the body of the process between quotes is impervious to
substitution, we get radically different answers. In fact, by
examining the first process in an input context,
e.g. $x?(z).\lift{w}{y!(z)}$, we see that the process under the lift
operator may be shaped by prefixed inputs binding a name inside it. In
this sense, the lift operator will be seen as a way to dynamically
construct processes before reifying them as names.

Finally equipped with these standard features we can present the
dynamics of the calculus.

\subsubsection{Operational semantics} 

Finally, we introduce the computational dynamics. What marks these
algebras as distinct from other more traditionally studied algebraic
structures, e.g. vector spaces or polynomial rings, is the manner in
which dynamics is captured. In traditional structures, dynamics is typically
expressed through morphisms between such structures, as in linear maps
between vector spaces or morphisms between rings. In algebras
associated with the semantics of computation, the dynamics is
expressed as part of the algebraic structure itself, through a
reduction reduction relation typically denoted by $\red$. Below, we
give a recursive presentation of this relation for the calculus used
in the encoding.

$\red \subseteq \pi \times \pi$
$\red : \pi \to \mathcal{P}(\pi)$

\begin{mathpar}
  \inferrule* [lab=Comm] { \textsf{match}( x_{src}, x_{trgt} ) } { x_{trgt}?(y)P \; | \; x_{src}!\langle {Q} \rangle \red P\{\quotep{Q}/y}\} }
  \and \\
  \inferrule* [lab=Par] {{P} \red {P}'} {{{P} | {Q}} \red {{P}' | {Q}}}
  \and
  \inferrule* [lab=Equiv]{{{P} \scong {P}'} \andalso {{P}' \red {Q}'} \andalso {{Q}' \scong {Q}}}{{P} \red {Q}}
\end{mathpar}

\begin{eqnarray*}
  match_{\equiv} (\quotep{P},\quotep{Q}) & := & P \equiv Q \\
  match_{\dagger}(\quotep{P},\quotep{Q}) & := & \forall R. P|Q \red^{*} R => R \red^{*} 0 \\
  match_{K}(\quotep{P},\quotep{Q}) & := & K \mbox{ for some context } K
\end{eqnarray*}

$u?(x)P | u!\langle Q \rangle \red P\{\quotep{Q}/x\}$

%We write $\wred$ for $\red^*$, and $P\red$ if $\exists Q $ such that $ P \red Q$.
We write $P\red$ if $\exists Q $ such that $ P \red Q$ and $P\not\red$, otherwise.

\section{Replication}

As mentioned before, it is known that replication (and hence
recursion) can be implemented in a higher-order process algebra
\cite{SangiorgiWalker}. As our first example of calculation with the
machinery thus far presented we give the construction explicitly in
the {\rhoc}.

\begin{eqnarray}
	D_{x} & := & \prefix{x}{y}{(\binpar{\outputp{x}{y}}{@{y}})} \nonumber\\
	\bangp_{x}{P} & := & \binpar{{x}!\langle{\binpar{D_{x}}{P}}\rangle}{D_{x}} \nonumber
\end{eqnarray}

\begin{eqnarray}
	\bangp_{x}{P} & & \nonumber\\
	=
	& {x}!\langle{(\prefix{x}{y}{(\outputp{x}{y} | @{y})) | P}}\rangle 
	      | \prefix{x}{y}{(\outputp{x}{y} | @{y})} & \nonumber\\
	\red
	& (\outputp{x}{y} | @{y})\substn{\quotep{(\prefix{x}{y}{(@{y} | \outputp{x}{y})) | P}}}{y} & \nonumber\\
	=
	& \outputp{x}{\quotep{(\prefix{x}{y}{(\outputp{x}{y} | @{y})) | P}}}
	  | {(\prefix{x}{y}{(\outputp{x}{y} | @{y})) | P}} & \nonumber\\
	\red
	& \ldots & \nonumber\\
	\red^*
	& P | P | \ldots & \nonumber
\end{eqnarray}

Of course, this encoding, as an implementation, runs away, unfolding
$\bangp{P}$ eagerly. A lazier and more implementable replication
operator, restricted to input-guarded processes, may be obtained as follows.

\begin{eqnarray}
\bangp{\prefix{u}{v}{P}} 
	:= 
	\binpar{\lift{x}{\prefix{u}{v}{(\binpar{D(x)}{P})}}}{D(x)} \nonumber
\end{eqnarray}

\begin{remark}
  Note that the lazier definition still does not deal with summation
  or mixed summation (i.e. sums over input and output). The reader is
  invited to construct definitions of replication that deal with these
  features. 

  Further, the definitions are parameterized in a name, $x$. Can you,
  gentle reader, make a definition that eliminates this parameter and
  guarantees no accidental interaction between the replication
  machinery and the process being replicated -- i.e. no accidental
  sharing of names used by the process to get its work done and the
  name(s) used by the replication to effect copying. This latter
  revision of the definition of replication is crucial to obtaining
  the expected identity $!!P \sim !P$.
\end{remark}

\begin{remark}\label{rem:paradoxical_combinator}
  The reader familiar with the lambda calculus will have noticed the
  similarity between $D$ and the paradoxical combinator.

  [Ed. note: the existence of this seems to suggest we have to be more
  restrictive on the set of processes and names we admit if we are to
  support no-cloning.]
\end{remark}

\subsubsection{Bisimulation}

The computational dynamics gives rise to another kind of equivalence,
the equivalence of computational behavior. As previously mentioned
this is typically captured \emph{via} some form of bisimulation.

% The notion we use in this paper is weak barbed bisimulation
% \cite{milner91polyadicpi}.

The notion we use in this paper is derived from weak barbed
bisimulation \cite{milner91polyadicpi}. 

\begin{definition}
An \emph{observation relation}, $\downarrow_{\mathcal N}$, over a set
of names, $\mathcal N$, is the smallest relation satisfying the rules
below.

\infrule[Out-barb]{y \in {\mathcal N}, \; x \nameeq y}
		  {\outputp{x}{v} \downarrow_{\mathcal N} x}
\infrule[Par-barb]{\mbox{$P\downarrow_{\mathcal N} x$ or $Q\downarrow_{\mathcal N} x$}}
		  {\binpar{P}{Q} \downarrow_{\mathcal N} x}

We write $P \Downarrow_{\mathcal N} x$ if there is $Q$ such that 
$P \wred Q$ and $Q \downarrow_{\mathcal N} x$.
\end{definition}

\begin{definition}
%\label{def.bbisim}
An  ${\mathcal N}$-\emph{barbed bisimulation} over a set of names, ${\mathcal N}$, is a symmetric binary relation 
${\mathcal S}_{\mathcal N}$ between agents such that $P\rel{S}_{\mathcal N}Q$ implies:
\begin{enumerate}
\item If $P \red P'$ then $Q \wred Q'$ and $P'\rel{S}_{\mathcal N} Q'$.
\item If $P\downarrow_{\mathcal N} x$, then $Q\Downarrow_{\mathcal N} x$.
\end{enumerate}
$P$ is ${\mathcal N}$-barbed bisimilar to $Q$, written
$P \wbbisim_{\mathcal N} Q$, if $P \rel{S}_{\mathcal N} Q$ for some ${\mathcal N}$-barbed bisimulation ${\mathcal S}_{\mathcal N}$.
\end{definition}

$\mathcal{R} \subseteq \pi \times \pi$

$P \mathcal{R} Q => \forall P'. P \red P' \Rightarrow \exists Q'. Q \red Q', P' \mathcal{R} Q'$

$P \vdash x \Rightarrow Q \vdash x$

\begin{mathpar}
  \inferrule*[lab=Out-barb]{x \nameeq y}{{y}!\langle{Q}\rangle \vdash x}
  \and
  \inferrule*[lab=Par-barb]{\mbox{$P\vdash x$ or $Q\vdash x$}}{\binpar{P}{Q} \vdash x}
\end{mathpar}

\subsubsection{Contexts}

One of the principle advantages of computational calculi like the
$\pi$-calculus is a well-defined notion of context,
contextual-equivalence and a correlation between
contextual-equivalence and notions of bisimulation. The notion of
context allows the decomposition of a process into (sub-)process and
its syntactic environment, its context. Thus, a context may be
thought of as a process with a ``hole'' (written $\Box$) in it. The
application of a context $M$ to a process $P$, written $M[P]$, is
tantamount to filling the hole in $M$ with $P$. In this paper we do
not need the full weight of this theory, but do make use of the notion
of context in the proof the main theorem. 

\begin{mathpar}
  \inferrule* [lab=summation] {} {{M_{M},M_{N}} \bc \Box \;|\; x.M_{A} \;|\; M_{M}+M_{N}}
  \and
  \inferrule* [lab=agent] {} {{M_{A}} \bc (\vec{x})M_{P} \;| \; \clift{P_0,\ldots,M_{P},\ldots,P_N}}
  \and \\
  \inferrule* [lab=process] {} {{M_{P}} \bc M_{N} \;| \;P|M_{P} }
\end{mathpar} 

\begin{mathpar}
  \inferrule* [lab=sychronization] {} {M_{N} \bc \Box \;|\; x?M_{F} \;|\; x!M_{C}}
  \and
  \inferrule* [lab=abstraction] {} {{M_{F}} \bc (x)M_{P} }
  \and
  \inferrule* [lab=concretion] {} {{M_{C}} \bc \langle M_{P} \rangle }
  \and \\
  \inferrule* [lab=process] {} {{M_{P}} \bc M_{N} \;| \;P|M_{P} }
\end{mathpar}

\begin{definition}[contextual application] Given a context $M$, and
  process $P$, we define the \emph{contextual application}, $M[P] :=
  M\{P/\Box\}$. That is, the contextual application of M to P is the
  substitution of $P$ for $\Box$ in $M$.
\end{definition}

$\meaningof{-} : L \to \mathcal{P}(\pi)$

\begin{mathpar}
  \inferrule* [lab=collection] {} {\meaningof{true} = \pi, \and \meaningof{~E} = \pi \setminus \meaningof{E}, \and \meaningof{E_{1} \& E_{2}} = \meaningof{E_{1}} \cap \meaningof{E_{2}}}
\end{mathpar}

\begin{mathpar}
  \inferrule* [lab=structure] {} {\meaningof{0} = \{ P \in \pi | P \equiv 0 \}, \and \\ \meaningof{E_1 | E_2} = \{ P \in \pi | P \equiv P_{1} | P_{2}, P_{1} \in \meaningof{E_{1}}, P_{2} \in \meaningof{E_2}\} }
\end{mathpar}

\begin{mathpar}
 \inferrule* [lab=behavior] {} {\meaningof{\langle a?b \rangle E} = \{ P \in \pi | P \equiv Q | u?(y)P', \\ \and \\\\ \and \\ \;\;\; u \in \meaningof{a}, \forall z.P'\{z/y\} \in \meaningof{E\{z/b\}}\}, \and \\ \meaningof{a!E} = \{ P \in \pi | P \equiv Q | x!\langle P' \rangle, x \in \meaningof{a} P' \in \meaningof{E}\} }
\end{mathpar}

\begin{mathpar}
 \inferrule* [lab=nominal] {} {\meaningof{\quotep{E}} = \{ \quotep{P} \in \quotep{\pi} | P \in \meaningof{E} \}, \and \meaningof{\quotep{P}} = \{ \quotep{Q} \in \quotep{\pi} | P \equiv Q \} \and \\ \meaningof{@\quotep{E}} = \{ P \in \pi | P \equiv @x, x \in \meaningof{E} \}}
\end{mathpar}

\begin{eqnarray*}
  \\
  \meaningof{-} : TS \to ST
\end{eqnarray*}

\begin{eqnarray*}
  \\
  L : TS \to ST
\end{eqnarray*}

\begin{eqnarray*}
  \\
  P \models E \iff P \in \meaningof{E}
\end{eqnarray*}

\begin{eqnarray*}
  P \approx_{L} Q \iff \forall E \in L. P \models E \iff Q \models E
\end{eqnarray*}

\begin{eqnarray*}
  P \approx_{K} Q
\end{eqnarray*}

\begin{eqnarray*}
  P \approx Q
\end{eqnarray*}

$\approx_{K} = \approx = \approx_{L}$

\subsubsection{Contextual duality}

Note that contexts extend the quotation operation to a family of
operations from processes to names. Given a context, $M$, we can
define a \emph{nominal context}, $\quotep{M}$ by $\quotep{M}[P] :=
\quotep{M[P]}$. To foreshadow what is to come we observe that these
operations enjoy a duality with processes very much like the duality
between vectors and maps from vectors to scalars.

Further, because the calculus is essentially higher-order, we have a
correspondence between contexts and processes. More specifically,
given a name $x$ and a context $M$ we can construct $M^{*}_{x}$ such
that 

\begin{mathpar}
  M^{*}_{x} | \lift{x}{P} \red M[P]
\end{mathpar}

namely,

\begin{mathpar}
  M^{*}_{x} := x?(u).M[\dropn{u}]
\end{mathpar}

The dependence of $M^{*}_{x}$ on a name makes it an abstraction, 

\begin{mathpar}
  M^{*} := (x)x?(u).M[\dropn{u}]
\end{mathpar}

\subsection{Additional notation}

It will sometimes be convenient to denote the process a name
quotes. We already have the notation $x = \quotep{P}$, but it will be
convenient to introduce an alternate notation, $\procn{x}$, when we
want to emphasize the connection to the use of the name. Note that, by
virtue of name equivalence, $\quotep{\procn{x}} \nameeq x$; so, the
notation is consistent with previous definitions.

Further, because names have structure it is possible to effect
substitutions on the basis of that structure. This means we need to
upgrade our notation for substitutions, which we accomplish by
adapting comprehension notation. Thus,

\begin{mathpar}
  P\{ y / x : x \in S \}
\end{mathpar}

is interpreted to mean the process derived from P by replacing (in a
capture-avoiding manner) each occurrence of $x$ in $S$ by $y$. For example,

\begin{mathpar}
  P\{ \quotep{\procn{x}|\procn{x}} / x : x \in \freenames{P} \}
\end{mathpar}

will replace each (occurrence) of a free name $x$ in $P$ by
$\quotep{\procn{x}|\procn{x}}$.

Also, we will avail ourselves of the notation $x^{L}$ and $x^{R}$ to
denote injections of a name into disjoint copies of the name
space. There are numerous ways to accomplish this. One example can be
found in \cite{MeredithR05}. This notation overloads to vectors of
names: $\vec{x}^{\pi} := (x_{i}^{\pi} \; : \; 0 \leq i < |\vec{x}| )$ where $\pi \in \{L,R\}$.

We also use $P^{\Box} := P|\Box$.

In \cite{MeredithR05} an interpretation of the new operator is
given. It turns out that there are several possible interpretations
all enjoying the requisite algebraic properties of the operator (see
\cite{milner91polyadicpi}). We will therefore make liberal use of
$(\nu\; \vec{x})P$.

% subsection the_syntax_and_semantics_of_the_notation_system (end)   

\input{qm2pi.qmops} 

\input{qm2pi.sterngerlach} 

\input{qm2pi.metric} 

% section concurrent_process_calculi (end)

%\input{qm2pi.proofsketch}

% section proof sketch (end)

%\input{qm2pi.slviaknots} 

% section spatial logic via knots (end)

\input{qm2pi.conclusion}

% section conclusion (end)

%\input{qm2pi.dtcodes} 

% section wiring algorithm (end)

\input{qm2pi.ack} 

% section acknowledgments (end)

\newpage


\bibliographystyle{plain}   
\bibliography{../../biblios/main.bib}

\input{qm2pi.rhodetails}

\end{document}

 

% section concurrent_process_calculi (end)

%\documentclass[12pt]{llncs}
%\documentclass{jktr}

\usepackage[pdftex]{hyperref}                   
\usepackage {listings}
\usepackage {mathpartir}
\usepackage{bcprules}
%\usepackage{listings}
                       
\usepackage{graphicx} 
%\usepackage[margins=2.5cm,nohead,nofoot]{geometry}
%\usepackage{geometry}
\usepackage{amsfonts}
\usepackage{amstext}
\usepackage{latexsym}
\usepackage{amssymb}
\usepackage{color}


%\include{myPreamble}
\include{qm2pi.local} 

%\ifpdf
%\usepackage[pdftex]{graphicx}
%\else
%\usepackage{graphicx}
%\fi

 % \ifpdf
%  \usepackage{pdfsync}
%  \if


%\title{Brief Article}
%\author{David F. Snyder}
%\author{L.G. Meredith}

%\address{Dept. of Math., Texas State University--San Marcos, San Marcos, TX 78666}
       
\pagestyle{empty}


\begin{document}

\lstset{language=[Objective]Caml,frame=shadowbox}

\input{qm2pi.front}

% section front matter (end)

\input{qm2pi.intro} 
 
% section introduction (end)

% \input{qm2pi.knotations} 

% section notation (end)

\input{qm2pi.process.calculi} 

% section concurrent_process_calculi_and_spatial_logics_ (end)
    
%\input{qm2pi.knots2pi} 

%\input{qm2pi.trefoil} 

%\input{qm2pi.mainthm} 

% subsection basic_interpretation (end)

%\input{qm2pi.rho.presentation} 
\subsection{The syntax and semantics of the notation system}\label{sub:the_syntax_and_semantics_of_the_notation_system} % (fold)

We now summarize a technical presentation of the calculus that
embodies our theory of dynamics. The typical presentation of such a
calculus follows the style of giving generators and relations on
them. The grammar, below, describing term constructors, freely
generates the set of processes, $\Proc$. This set is then quotiented
by a relation known as structural congruence and it is over this set
that the notion of dynamics is expressed. This presentation is
essentially that of \cite{MeredithR05} with the addition of
polyadicity and summation. For readability we have relegated some of
the technical subtleties to an appendix.

\subsubsection{Process grammar}\label{subsub:process_grammar}

\begin{mathpar}
  \inferrule* [lab=synchronization] {} {{M} \bc \pzero \;|\; x?F \;|\; x!C }
  \and
  \inferrule* [lab=abstraction] {} {{F} \bc (x)P}
  \and
  \inferrule* [lab=concretion] {} {{C} \bc \langle Q \rangle}
  \and
  \inferrule* [lab=process] {} {{P,Q} \bc M \;| \;P|Q \;|\; @{x}}
  \and
  \inferrule* [lab=name] {} {{x} \bc \quotep{P}}
\end{mathpar} 

Note that $\vec{x}$ (resp. $\vec{P}$) denotes a vector of names
(resp. processes) of length $|\vec{x}|$ (resp. $|\vec{P}|$). We adopt
the following useful abbreviations.

\begin{mathpar}
   x?(\vec{y}).P := x.(\vec{y})P \and  x\clift{\vec{P}} := x.\clift{\vec{P}}
   \and x!(y) := \lift{x}{\dropn{y}}
   \and \Pi_{i=0}^{n-1}P_i := P_0 | \ldots | P_{n-1}
\end{mathpar}

\subsubsection{Structural congruence}

\paragraph{Free and bound names and alpha-equivalence.} At the
core of structural equivalence is alpha-equivalence which identifies
process that are the same up to a change of variable. Formally, we
recognize the distinction between free and bound names. The free names
of a process, $\freenames{P}$, may be calculated recursively as
follows:

\begin{mathpar}
\freenames{\pzero} := \emptyset
  \and \\
  \freenames{x?(y).P} := \{ x \} \cup (\freenames{P} \setminus \{ y \})
  \and 
  \freenames{x!\langle P \rangle} := \{ x \} \cup \{ P \} 
  \and \\
  \freenames{P|Q} := \freenames{P} \cup \freenames{Q}
  \and \\
  \freenames{@{x}} := \{ x \}
\end{mathpar}

$\pi$
$\quotep{\pi}$

$\freenames{-} : \pi \to \mathcal{P}(\quotep{\pi})$

\begin{eqnarray*}
  \freenames{\pzero} & := & \emptyset \\
  \freenames{x?(y).P} & := & \{ x \} \cup (\freenames{P} \setminus \{ y \}) \\
  \freenames{x!\langle P \rangle} & := & \{ x \} \cup \{ P \} \\
  \freenames{P|Q} & := & \freenames{P} \cup \freenames{Q} \\
  \freenames{\dropn{x}} & := & \{ x \}
\end{eqnarray*}

The bound names of a process, $\boundnames{P}$, are those names occurring in $P$
that are not free. For example, in $x?(y).0$, the name $x$ is free, while $y$ is bound.

\begin{mathpar}
  \inferrule* [lab=monoidal-laws] {} { P|Q \equiv Q|P \and P|0 \equiv P \and P|(Q|R) \equiv (P|Q)|R }
\end{mathpar}

\begin{mathpar}
  \inferrule* [lab=alpha-equivalence] {} { (x)P \equiv (y)P\{y/x\} \and y \not\in \freenames{P} }
\end{mathpar}

\begin{definition}
Then two processes, $P,Q$, are alpha-equivalent if $P = Q\{\vec{y}/\vec{x}\}$ for
some $\vec{x} \in \boundnames{Q},\vec{y} \in \boundnames{P}$, where $Q\{\vec{y}/\vec{x}\}$
denotes the capture-avoiding substitution of $\vec{y}$ for $\vec{x}$ in $Q$.
\end{definition}

\begin{definition}
  The {\em structural congruence} \cite{SangiorgiWalker} , $\equiv$,
  between processes is the least congruence containing
  alpha-equivalence, satisfying the abelian monoid laws
  (associativity, commutativity and $\pzero$ as identity) for parallel
  composition $|$ and for summation $+$.
\end{definition}

\subsection{Name equivalence}

We take name equivalence, written $\nameeq$, to be the smallest
equivalence relation generated by the following rules.

\begin{mathpar}
\inferrule*[lab=Quote-drop]
{ }
{ \quotep{@{x}} \nameeq x }

\inferrule*[lab=Struct-equiv]
{ P \scong Q }
{ \quotep{P} \nameeq \quotep{Q} }
\end{mathpar}

The astute reader will have noticed that the mutual recursion of names
and processes imposes a mutual recursion on alpha-equivalence and
structural equivalence via name-equivalence. Fortunately, all of this
works out pleasantly and we may calculate in the natural way, free of
concern. The reader interested in the details is referred to the
appendix \ref{appendix:rho_details}.

\subsection{Substitution}

We use $\Proc$ for the set of processes, $\QProc$ for the set of
names, and $\id{\{}\vec{y} / \vec{x} \id{\}}$ to denote partial maps,
$s : \QProc \rightarrow \QProc$. A map, $s$ lifts, uniquely, to a map
on process terms, $\widehat{s} : \Proc \rightarrow \Proc$ by the
following equations.

\begin{mathpar}
  (0) \psubstp{Q}{P} := 0 \\
  (R \juxtap S) \psubstp{Q}{P}
  :=    
  (R)\psubstp{Q}{P} \juxtap (S) \psubstp{Q}{P} \\
  (x?(y).R) \psubstp{Q}{P}    
  :=    
  (x)\substp{Q}{P} (z)\concat( (R \psubstn{z}{y}) \psubstp{Q}{P} ) \\
  (\lift{x}{R}) \psubstp{Q}{P}  
  :=
  \lift{(x)\substp{Q}{P}}{ R \psubstp{Q}{P} } \\
%   (\dropn{x})  \psubstp{Q}{P}       
%   := 
%   \left\{ 
%     \begin{array}{ccc} 
%       \dropn{\quotep{Q}} & & x \nameeq \quotep{P} \\
%       \dropn{x} & & otherwise \\
%     \end{array}
%   \right. 
  (\dropn{x})  \psubstp{Q}{P}       
  := 
  \left\{ 
    \begin{array}{ccc} 
      Q & & x \nameeq \quotep{P} \\
      \dropn{x} & & otherwise \\
    \end{array}
  \right.
\end{mathpar}
 

where

\begin{eqnarray}
  (x)\id{\{} \lpquote Q \rpquote / \lpquote P \rpquote \id{\}}            = 
  \left\{ 
    \begin{array}{ccc}
      \lpquote Q \rpquote & & x \nameeq \lpquote P \rpquote \\
      x & & otherwise \\
    \end{array}
  \right. \nonumber
\end{eqnarray}

and $z$ is chosen distinct from $\quotep{P}$, $\quotep{Q}$, the free
names in $Q$, and all the names in $R$. Our $\alpha$-equivalence will
be built in the standard way from this substitution.

\begin{remark}\label{rem:no_self_referential_names}
  One consequence of these definitions is that $\forall P. \quotep{P}
  \not\in \freenames{P}$.
\end{remark}

\subsection{ Dynamic quote: an example }

Anticipating something of what's to come, consider applying the
substitution, $\widehat{\id{\{}u / z \id{\}}}$, to the following pair
of processes, $\lift{w}{y!(z)}$ and $w[ \lpquote y!(z) \rpquote ]$.

\begin{eqnarray}
	\lift{w}{y!(z)}\widehat{\id{\{}u / z \id{\}}}
		& = &
		\lift{w}{y!(u)} \nonumber\\
	w[ \lpquote y!(z) \rpquote ] \widehat{ \id{\{}u / z \id{\}} }
		& = &
		w[ \lpquote y!(z) \rpquote ] \nonumber
\end{eqnarray}

Because the body of the process between quotes is impervious to
substitution, we get radically different answers. In fact, by
examining the first process in an input context,
e.g. $x?(z).\lift{w}{y!(z)}$, we see that the process under the lift
operator may be shaped by prefixed inputs binding a name inside it. In
this sense, the lift operator will be seen as a way to dynamically
construct processes before reifying them as names.

Finally equipped with these standard features we can present the
dynamics of the calculus.

\subsubsection{Operational semantics} 

Finally, we introduce the computational dynamics. What marks these
algebras as distinct from other more traditionally studied algebraic
structures, e.g. vector spaces or polynomial rings, is the manner in
which dynamics is captured. In traditional structures, dynamics is typically
expressed through morphisms between such structures, as in linear maps
between vector spaces or morphisms between rings. In algebras
associated with the semantics of computation, the dynamics is
expressed as part of the algebraic structure itself, through a
reduction reduction relation typically denoted by $\red$. Below, we
give a recursive presentation of this relation for the calculus used
in the encoding.

$\red \subseteq \pi \times \pi$
$\red : \pi \to \mathcal{P}(\pi)$

\begin{mathpar}
  \inferrule* [lab=Comm] { \textsf{match}( x_{src}, x_{trgt} ) } { x_{trgt}?(y)P \; | \; x_{src}!\langle {Q} \rangle \red P\{\quotep{Q}/y}\} }
  \and \\
  \inferrule* [lab=Par] {{P} \red {P}'} {{{P} | {Q}} \red {{P}' | {Q}}}
  \and
  \inferrule* [lab=Equiv]{{{P} \scong {P}'} \andalso {{P}' \red {Q}'} \andalso {{Q}' \scong {Q}}}{{P} \red {Q}}
\end{mathpar}

\begin{eqnarray*}
  match_{\equiv} (\quotep{P},\quotep{Q}) & := & P \equiv Q \\
  match_{\dagger}(\quotep{P},\quotep{Q}) & := & \forall R. P|Q \red^{*} R => R \red^{*} 0 \\
  match_{K}(\quotep{P},\quotep{Q}) & := & K \mbox{ for some context } K
\end{eqnarray*}

$u?(x)P | u!\langle Q \rangle \red P\{\quotep{Q}/x\}$

%We write $\wred$ for $\red^*$, and $P\red$ if $\exists Q $ such that $ P \red Q$.
We write $P\red$ if $\exists Q $ such that $ P \red Q$ and $P\not\red$, otherwise.

\section{Replication}

As mentioned before, it is known that replication (and hence
recursion) can be implemented in a higher-order process algebra
\cite{SangiorgiWalker}. As our first example of calculation with the
machinery thus far presented we give the construction explicitly in
the {\rhoc}.

\begin{eqnarray}
	D_{x} & := & \prefix{x}{y}{(\binpar{\outputp{x}{y}}{@{y}})} \nonumber\\
	\bangp_{x}{P} & := & \binpar{{x}!\langle{\binpar{D_{x}}{P}}\rangle}{D_{x}} \nonumber
\end{eqnarray}

\begin{eqnarray}
	\bangp_{x}{P} & & \nonumber\\
	=
	& {x}!\langle{(\prefix{x}{y}{(\outputp{x}{y} | @{y})) | P}}\rangle 
	      | \prefix{x}{y}{(\outputp{x}{y} | @{y})} & \nonumber\\
	\red
	& (\outputp{x}{y} | @{y})\substn{\quotep{(\prefix{x}{y}{(@{y} | \outputp{x}{y})) | P}}}{y} & \nonumber\\
	=
	& \outputp{x}{\quotep{(\prefix{x}{y}{(\outputp{x}{y} | @{y})) | P}}}
	  | {(\prefix{x}{y}{(\outputp{x}{y} | @{y})) | P}} & \nonumber\\
	\red
	& \ldots & \nonumber\\
	\red^*
	& P | P | \ldots & \nonumber
\end{eqnarray}

Of course, this encoding, as an implementation, runs away, unfolding
$\bangp{P}$ eagerly. A lazier and more implementable replication
operator, restricted to input-guarded processes, may be obtained as follows.

\begin{eqnarray}
\bangp{\prefix{u}{v}{P}} 
	:= 
	\binpar{\lift{x}{\prefix{u}{v}{(\binpar{D(x)}{P})}}}{D(x)} \nonumber
\end{eqnarray}

\begin{remark}
  Note that the lazier definition still does not deal with summation
  or mixed summation (i.e. sums over input and output). The reader is
  invited to construct definitions of replication that deal with these
  features. 

  Further, the definitions are parameterized in a name, $x$. Can you,
  gentle reader, make a definition that eliminates this parameter and
  guarantees no accidental interaction between the replication
  machinery and the process being replicated -- i.e. no accidental
  sharing of names used by the process to get its work done and the
  name(s) used by the replication to effect copying. This latter
  revision of the definition of replication is crucial to obtaining
  the expected identity $!!P \sim !P$.
\end{remark}

\begin{remark}\label{rem:paradoxical_combinator}
  The reader familiar with the lambda calculus will have noticed the
  similarity between $D$ and the paradoxical combinator.

  [Ed. note: the existence of this seems to suggest we have to be more
  restrictive on the set of processes and names we admit if we are to
  support no-cloning.]
\end{remark}

\subsubsection{Bisimulation}

The computational dynamics gives rise to another kind of equivalence,
the equivalence of computational behavior. As previously mentioned
this is typically captured \emph{via} some form of bisimulation.

% The notion we use in this paper is weak barbed bisimulation
% \cite{milner91polyadicpi}.

The notion we use in this paper is derived from weak barbed
bisimulation \cite{milner91polyadicpi}. 

\begin{definition}
An \emph{observation relation}, $\downarrow_{\mathcal N}$, over a set
of names, $\mathcal N$, is the smallest relation satisfying the rules
below.

\infrule[Out-barb]{y \in {\mathcal N}, \; x \nameeq y}
		  {\outputp{x}{v} \downarrow_{\mathcal N} x}
\infrule[Par-barb]{\mbox{$P\downarrow_{\mathcal N} x$ or $Q\downarrow_{\mathcal N} x$}}
		  {\binpar{P}{Q} \downarrow_{\mathcal N} x}

We write $P \Downarrow_{\mathcal N} x$ if there is $Q$ such that 
$P \wred Q$ and $Q \downarrow_{\mathcal N} x$.
\end{definition}

\begin{definition}
%\label{def.bbisim}
An  ${\mathcal N}$-\emph{barbed bisimulation} over a set of names, ${\mathcal N}$, is a symmetric binary relation 
${\mathcal S}_{\mathcal N}$ between agents such that $P\rel{S}_{\mathcal N}Q$ implies:
\begin{enumerate}
\item If $P \red P'$ then $Q \wred Q'$ and $P'\rel{S}_{\mathcal N} Q'$.
\item If $P\downarrow_{\mathcal N} x$, then $Q\Downarrow_{\mathcal N} x$.
\end{enumerate}
$P$ is ${\mathcal N}$-barbed bisimilar to $Q$, written
$P \wbbisim_{\mathcal N} Q$, if $P \rel{S}_{\mathcal N} Q$ for some ${\mathcal N}$-barbed bisimulation ${\mathcal S}_{\mathcal N}$.
\end{definition}

$\mathcal{R} \subseteq \pi \times \pi$

$P \mathcal{R} Q => \forall P'. P \red P' \Rightarrow \exists Q'. Q \red Q', P' \mathcal{R} Q'$

$P \vdash x \Rightarrow Q \vdash x$

\begin{mathpar}
  \inferrule*[lab=Out-barb]{x \nameeq y}{{y}!\langle{Q}\rangle \vdash x}
  \and
  \inferrule*[lab=Par-barb]{\mbox{$P\vdash x$ or $Q\vdash x$}}{\binpar{P}{Q} \vdash x}
\end{mathpar}

\subsubsection{Contexts}

One of the principle advantages of computational calculi like the
$\pi$-calculus is a well-defined notion of context,
contextual-equivalence and a correlation between
contextual-equivalence and notions of bisimulation. The notion of
context allows the decomposition of a process into (sub-)process and
its syntactic environment, its context. Thus, a context may be
thought of as a process with a ``hole'' (written $\Box$) in it. The
application of a context $M$ to a process $P$, written $M[P]$, is
tantamount to filling the hole in $M$ with $P$. In this paper we do
not need the full weight of this theory, but do make use of the notion
of context in the proof the main theorem. 

\begin{mathpar}
  \inferrule* [lab=summation] {} {{M_{M},M_{N}} \bc \Box \;|\; x.M_{A} \;|\; M_{M}+M_{N}}
  \and
  \inferrule* [lab=agent] {} {{M_{A}} \bc (\vec{x})M_{P} \;| \; \clift{P_0,\ldots,M_{P},\ldots,P_N}}
  \and \\
  \inferrule* [lab=process] {} {{M_{P}} \bc M_{N} \;| \;P|M_{P} }
\end{mathpar} 

\begin{mathpar}
  \inferrule* [lab=sychronization] {} {M_{N} \bc \Box \;|\; x?M_{F} \;|\; x!M_{C}}
  \and
  \inferrule* [lab=abstraction] {} {{M_{F}} \bc (x)M_{P} }
  \and
  \inferrule* [lab=concretion] {} {{M_{C}} \bc \langle M_{P} \rangle }
  \and \\
  \inferrule* [lab=process] {} {{M_{P}} \bc M_{N} \;| \;P|M_{P} }
\end{mathpar}

\begin{definition}[contextual application] Given a context $M$, and
  process $P$, we define the \emph{contextual application}, $M[P] :=
  M\{P/\Box\}$. That is, the contextual application of M to P is the
  substitution of $P$ for $\Box$ in $M$.
\end{definition}

$\meaningof{-} : L \to \mathcal{P}(\pi)$

\begin{mathpar}
  \inferrule* [lab=collection] {} {\meaningof{true} = \pi, \and \meaningof{~E} = \pi \setminus \meaningof{E}, \and \meaningof{E_{1} \& E_{2}} = \meaningof{E_{1}} \cap \meaningof{E_{2}}}
\end{mathpar}

\begin{mathpar}
  \inferrule* [lab=structure] {} {\meaningof{0} = \{ P \in \pi | P \equiv 0 \}, \and \\ \meaningof{E_1 | E_2} = \{ P \in \pi | P \equiv P_{1} | P_{2}, P_{1} \in \meaningof{E_{1}}, P_{2} \in \meaningof{E_2}\} }
\end{mathpar}

\begin{mathpar}
 \inferrule* [lab=behavior] {} {\meaningof{\langle a?b \rangle E} = \{ P \in \pi | P \equiv Q | u?(y)P', \\ \and \\\\ \and \\ \;\;\; u \in \meaningof{a}, \forall z.P'\{z/y\} \in \meaningof{E\{z/b\}}\}, \and \\ \meaningof{a!E} = \{ P \in \pi | P \equiv Q | x!\langle P' \rangle, x \in \meaningof{a} P' \in \meaningof{E}\} }
\end{mathpar}

\begin{mathpar}
 \inferrule* [lab=nominal] {} {\meaningof{\quotep{E}} = \{ \quotep{P} \in \quotep{\pi} | P \in \meaningof{E} \}, \and \meaningof{\quotep{P}} = \{ \quotep{Q} \in \quotep{\pi} | P \equiv Q \} \and \\ \meaningof{@\quotep{E}} = \{ P \in \pi | P \equiv @x, x \in \meaningof{E} \}}
\end{mathpar}

\begin{eqnarray*}
  \\
  \meaningof{-} : TS \to ST
\end{eqnarray*}

\begin{eqnarray*}
  \\
  L : TS \to ST
\end{eqnarray*}

\begin{eqnarray*}
  \\
  P \models E \iff P \in \meaningof{E}
\end{eqnarray*}

\begin{eqnarray*}
  P \approx_{L} Q \iff \forall E \in L. P \models E \iff Q \models E
\end{eqnarray*}

\begin{eqnarray*}
  P \approx_{K} Q
\end{eqnarray*}

\begin{eqnarray*}
  P \approx Q
\end{eqnarray*}

$\approx_{K} = \approx = \approx_{L}$

\subsubsection{Contextual duality}

Note that contexts extend the quotation operation to a family of
operations from processes to names. Given a context, $M$, we can
define a \emph{nominal context}, $\quotep{M}$ by $\quotep{M}[P] :=
\quotep{M[P]}$. To foreshadow what is to come we observe that these
operations enjoy a duality with processes very much like the duality
between vectors and maps from vectors to scalars.

Further, because the calculus is essentially higher-order, we have a
correspondence between contexts and processes. More specifically,
given a name $x$ and a context $M$ we can construct $M^{*}_{x}$ such
that 

\begin{mathpar}
  M^{*}_{x} | \lift{x}{P} \red M[P]
\end{mathpar}

namely,

\begin{mathpar}
  M^{*}_{x} := x?(u).M[\dropn{u}]
\end{mathpar}

The dependence of $M^{*}_{x}$ on a name makes it an abstraction, 

\begin{mathpar}
  M^{*} := (x)x?(u).M[\dropn{u}]
\end{mathpar}

\subsection{Additional notation}

It will sometimes be convenient to denote the process a name
quotes. We already have the notation $x = \quotep{P}$, but it will be
convenient to introduce an alternate notation, $\procn{x}$, when we
want to emphasize the connection to the use of the name. Note that, by
virtue of name equivalence, $\quotep{\procn{x}} \nameeq x$; so, the
notation is consistent with previous definitions.

Further, because names have structure it is possible to effect
substitutions on the basis of that structure. This means we need to
upgrade our notation for substitutions, which we accomplish by
adapting comprehension notation. Thus,

\begin{mathpar}
  P\{ y / x : x \in S \}
\end{mathpar}

is interpreted to mean the process derived from P by replacing (in a
capture-avoiding manner) each occurrence of $x$ in $S$ by $y$. For example,

\begin{mathpar}
  P\{ \quotep{\procn{x}|\procn{x}} / x : x \in \freenames{P} \}
\end{mathpar}

will replace each (occurrence) of a free name $x$ in $P$ by
$\quotep{\procn{x}|\procn{x}}$.

Also, we will avail ourselves of the notation $x^{L}$ and $x^{R}$ to
denote injections of a name into disjoint copies of the name
space. There are numerous ways to accomplish this. One example can be
found in \cite{MeredithR05}. This notation overloads to vectors of
names: $\vec{x}^{\pi} := (x_{i}^{\pi} \; : \; 0 \leq i < |\vec{x}| )$ where $\pi \in \{L,R\}$.

We also use $P^{\Box} := P|\Box$.

In \cite{MeredithR05} an interpretation of the new operator is
given. It turns out that there are several possible interpretations
all enjoying the requisite algebraic properties of the operator (see
\cite{milner91polyadicpi}). We will therefore make liberal use of
$(\nu\; \vec{x})P$.

% subsection the_syntax_and_semantics_of_the_notation_system (end)   

\input{qm2pi.qmops} 

\input{qm2pi.sterngerlach} 

\input{qm2pi.metric} 

% section concurrent_process_calculi (end)

%\input{qm2pi.proofsketch}

% section proof sketch (end)

%\input{qm2pi.slviaknots} 

% section spatial logic via knots (end)

\input{qm2pi.conclusion}

% section conclusion (end)

%\input{qm2pi.dtcodes} 

% section wiring algorithm (end)

\input{qm2pi.ack} 

% section acknowledgments (end)

\newpage


\bibliographystyle{plain}   
\bibliography{../../biblios/main.bib}

\input{qm2pi.rhodetails}

\end{document}



% section proof sketch (end)

%\section{Unlikely characters: spatial logic for
  knots}\label{sub:characteristic_formulae} % (fold)

Associated to the mobile process calculi are a family of logics known
as the Hennessy-Milner logics. These logics typically enjoy a
semantics interpreting formulae as sets of processes that when
factored through the encoding outlined above allows an identification
of classes of knots with logical formulae. In the context of this
encoding the sub-family known as the spatial logics \cite{CairesC03}
\cite{CairesC04} \cite{Caires04} are of particular interest providing
several important features for expressing and reasoning about
properties (i.e. classes) of knots. We hint here at how this may be done.

%\begin{description}
%\item [structural connectives] 
\subsubsection{Structural connectives} The spatial logics enjoy
structural connectives corresponding, at the logical level, to the
parallel composition ($P | Q$) and new name ($(\nu \; x)P$)
connectives for processes. As illustrated in the examples below, these
connectives are extremely expressive given the shape of our encoding.
%\item [decideable satisfaction]

\subsubsection{Decideable satisfaction}
In \cite{Caires04} the satisfaction relation is shown to be decideable
for a rich class of processes. It further turns out that the image of
the our encoding is a proper subset of that class. This result
provides the basis for an algorithm by which to search for knots
enjoying a given property.
%\item [characteristic formulae]

\subsubsection{Characteristic formulae}
In the same paper \cite{Caires04} , Caires presents a means of calculating
characteristic formulae, selecting equivalence classes of processes
up to a pre--specified depth limit on the support set of names. Composed with our
encoding, this characteristic formula can be used to select
characteristic formulae for knots.
%\end{description}

\subsubsection{Spatial logic formulae}

The grammar below (segmented for comprehension) summarizes the syntax
of spatial logic formulae. We employ illustrative examples in the
sequel to provide an intuitive understanding of their meaning
referring the reader to \cite{Caires04} for a more detailed explication
of the semantics.

\begin{mathpar}
  \inferrule* [lab=boolean] {} {{A,B} \bc T \;|\; \neg A \;|\; A \wedge B \;|\; \eta = \eta'}
  \and
  \inferrule* [lab=spatial] {} {|\; \pzero \;|\; A | B \;|\; x \text{\textregistered} A \;|\; \forall x . A \;|\;  H x . A}
  \and
  \inferrule* [lab=behavioral] {} {|\; \alpha . A}
  \and 
  \inferrule* [lab=recursion] {} {|\; X(\vec{u}) \;|\; \mu X(\vec{u}) . A}
  \and
  \inferrule* [lab=action] {} {\alpha \bc \langle x?(\vec{y}) \rangle \;|\; \langle x!(\vec{y}) \rangle \;|\; \langle \tau \rangle}
  \and 
  \inferrule* [lab=name] {} {\eta \bc x \;|\; \tau}
\end{mathpar} 

% subsection characteristic_formulae (end)   	 

\subsection{Example formulae}\label{sub:example_formulae_} % (fold)

\subsubsection{Crossing as formula.}
% 
% \begin{align*}
%   \frac{d}{dx} \sin x &= \cos x 
%   & \frac{d}{dx} e^x &= e^x \\
%   \frac{d}{dx} \cos x &= - \sin x 
%   & \frac{d}{dx} \log x &= \frac{1}{x} \\
% \end{align*} 

\begin{align*}
 \mu C(x_{0},x_{1},y_{0},y_{1},u).&(\langle x_{0}?(z) \rangle(\langle u! \rangle\langle y_{1}!z \rangle C(x_{0},x_{1},y_{0},y_{1},u)) & \\
  & \wedge \langle y_{1}?(z) \rangle (\langle u! \rangle \langle x_{0}!z \rangle C(x_{0},x_{1},y_{0},y_{1},u)) & \\
  & \wedge \langle x_{1}?(z) \rangle (\langle u? \rangle \langle y_{0}!z \rangle C(x_{0},x_{1},y_{0},y_{1},u)) & \\
  & \wedge \langle y_{0}?(z) \rangle (\langle u? \rangle \langle x_{1}!z \rangle C(x_{0},x_{1},y_{0},y_{1},u))) &
\end{align*}

The lexicographical similarity between the shape of this formulae and
the shape of definition of the process representing a crossing reveals
the intuitive meaning of this formulae. It describes the capabilities
of a process that has the right to represent a crossing. For example
it picks out processes that may perform an input on the port $x_0$ in
its initial menu of capabilities. What differentiates the formula
from the process, however, is that the crossing process is the
smallest candidate to satisfy the formula. Infinitely many other
processes -- with internal behavior hidden behind this interface, so
to speak -- also satisfy this formula. Even this simple formula,
then, can be seen to open a new view onto knots, providing a
computational interpretation of \emph{virtual} knots.

Note that this formula is derived by hand. A similar formula can be
derived by employing Caires' calculation of characteristic formula
\cite{Caires04} to the process representing a crossing. In light of
this discussion, we let
$\meaningof{C}_{\phi}(x0,x1,y0,y1,u)$ denote a formula specifying the
dynamics we wish to capture of a crossing. To guarantee we preserve
the shape of the interface and minimal semantics we demand that
$\meaningof{C}_{\phi}(x0,x1,y0,y1,u) \Rightarrow
\textbf{C}(x0,x1,y0,y1,u)$ where $\textbf{C}(x0,x1,y0,y1,u)$ denotes
the formula above.
                            
\subsubsection{Crossing number constraints.}
The moral content of the context lemma (Lemma \ref{context}) is that the notion of
``locality'' in the Reidemeister moves is effectively captured by the
parallel composition operator of the process calculus. This intuition
extends through the logic. Given a formula,
$\meaningof{C}_{\phi}(x0,x1,y0,y1,u)$, we can use the structural
connectives to specify constraints on crossing numbers, such as at
least $n$ crossings, or exactly $n$ crossings.
\begin{mathpar}
  \inferrule* [lab=at-least-n] {} { K^{\geq n}_{\phi}(\vec{xs},\vec{ys}) := \Pi_{i=0}^{n-1} Hu . \meaningof{C}_{\phi}(xs_i,ys_i,u) | T }
  \and 
  \inferrule* [lab=exactly-n] {} { K^{= n}_{\phi}(\vec{xs},\vec{ys}) := \Pi_{i=0}^{n-1} Hu . \meaningof{C}_{\phi}(xs_i,ys_i,u) | \neg (\forall x_0,y_0,x_1,y_1,u . \meaningof{C}_{\phi}(x_0,y_0,x_1,y_1,u) | T) }
\end{mathpar}

To round out this section, recall that the encoding of an $n$-crossing
knot decomposes into a parallel composition of $n$ \emph{copies} of a
crossing process together with a wiring harness. To specify different
knot classes with the same crossing number amounts to specifying
logical constraints on the wiring harness. In the interest of space,
we defer examples to a forthcoming paper. Suffice it to say that both
the conditions ``alternating knot'' and ``contains the tangle
corresponding to 5/3'' are expressible. For example, it is possible to
calculate the characteristic formula of a process corresponding to the
tangle 5/3 and conjoin it into the classifying formula via the
composition connective of the logic.

Finally, we wish to observe that it is entirely within reason to
contemplate a more domain-specific version of spatial logic tailored
to the shape of processes in the image of the encoding. Such a
domain-specific logic would have a better claim to the title formal
language of knot properties.

% subsection example_formulae_ (end)

% section knots_as_processes (end) 

% section spatial logic via knots (end)

\section{Conclusions and future work}

\paragraph{Testing physical space}
You, gentle reader, may wonder why of all the theorems to be proved
given this set up we pick the one above. In some sense it's hardly
central to quantum mechanics. We see it as central in the sense that
it firmly establishes a notion of physical space arising from a notion
of the equivalence of behavior. Relating bisimulation to a metric is a
big step forward, but one is faced with interpreting the relationship
of that metric space to something more physical. Quantum mechanical
notions of ``physical'' space are still far from intuitive, but by
relating this idea of distance as testing to calculations that predict
physical circumstances we are making a not insignificant step forward
toward an understanding of the physical space we inhabit as
essentially dynamic.

\paragraph{Effectivity and simulation}
One of the observations we have yet to make is that the entire program
spelled out here is effective. We have built various interpreters for
the reflective calculus at work in this interpretation. In principle,
then, we can simulate quantum mechanics on a computer. The place where
the simulation may lose fidelity is the infinitely branching summation
for the annihilator.

In this connection i also want to point out that the evaluation style
calculation of the inner product puts the non-determinism of the
summation right at the heart of measurement. This suggests that
Milner's original reduction-based formulation of the dynamics of his
calculi in terms of sums was not just notationally suggestive of a
notion of measure-and-continue but captured some significant part of
the physics.

\paragraph{Quantum continuations}
In light of this last observation i want to point out that the
predominant account of quantum mechanics is missing a key aspect of a
truly compositional story of the physical situation. In a real lab,
when a measurement is made the observation can be made to feed into
another device that then makes another measurement conditioned on the
results of the first. This means that after the superposition was
collapsed the entire experimental set up remained in
superposition. While QM offers a means of writing this down it doesn't
quite line up well with the well-trodden formulation of computation
and continuation that we see so succinctly expressed in Milner's
calculi. This suggests that there might be advantages to this account
of dynamics waiting to be explored.

\paragraph{Quantum logic}
In this connection, we also note that by virtue of having the
Hennessy-Milner construction, we can pull the construction through the
interpretation of QM. This gives us a natural candidate for a quantum
logic that enjoys an extremely tight connection with it's domain of
interpretation, making the construction much less ad hoc (rather it is
the image of functor!).

\paragraph{Quantum probabiity}
i have questions about the basis of the interpretation of inner
product as probability amplitude. In particular, using which
axiomatization of probability theory does the notion of probability
amplitude earn the right to be so dubbed? In other words, where is the
proof that the operation for calculating a probability amplitude (and
then squaring) satisfies the axioms of what it means to calculate a
probability? Even if such a proof exists (i have yet to find it in the
literature), i wonder if it might not be possible to turn things on
their heads. Can we view the calculation of the probability amplitude
as an axiomatization of probability? If so, then the definition we
give for calculating probability amplitude may provide the basis for
an \emph{effective} theory of probability.

\paragraph{Quantum vs ``biological'' information}
Finally, i want to conclude with a more philosophical observation. At
a recent workshop in which QM was a predominant topic i noticed
something about quantum information. The speaker was giving a riveting
discussion of axiomatic QM and showing how properties of ``no
cloning'' and ``no deleting'' emerged as consequences of the
axiomatization. Theorems of this form are necessary to give us a sense
of confidence that our axioms characterize the physical theory. What
struck me, though, was that if quantum information is neither erasable
nor replicable it is markedly different from \emph{life}. Two of the
things we know about life is that

\begin{itemize}
  \item it ends;
  \item to gain some measure of persistence, to transcend it's
    finitude it is imminently copyable.
\end{itemize}

Both of these qualities are summarized succinctly in the aphorism: all
flesh is grass. For me these two kinds of ``information'' -- call them
quantum and biological -- are end points on a spectrum of strategies
for persistence. At one end, we have those curious entities that enjoy
uniqueness and permanence; at the other, we have those who in the face
of a certain end and an uncertain present make a go of passing
something on. To me one of the more remarkable aspects of the latter
strategy is that in the presence of noise (and certain features of
copying) we get a kind of dynamism, a chance for improvement against a
given persistent condition.

% subsection other_calculi_other_bisimulations_and_geometry_as_behavior (end)




% section conclusion (end)

%\documentclass[12pt]{llncs}
%\documentclass{jktr}

\usepackage[pdftex]{hyperref}                   
\usepackage {listings}
\usepackage {mathpartir}
\usepackage{bcprules}
%\usepackage{listings}
                       
\usepackage{graphicx} 
%\usepackage[margins=2.5cm,nohead,nofoot]{geometry}
%\usepackage{geometry}
\usepackage{amsfonts}
\usepackage{amstext}
\usepackage{latexsym}
\usepackage{amssymb}
\usepackage{color}


%\include{myPreamble}
\include{qm2pi.local} 

%\ifpdf
%\usepackage[pdftex]{graphicx}
%\else
%\usepackage{graphicx}
%\fi

 % \ifpdf
%  \usepackage{pdfsync}
%  \if


%\title{Brief Article}
%\author{David F. Snyder}
%\author{L.G. Meredith}

%\address{Dept. of Math., Texas State University--San Marcos, San Marcos, TX 78666}
       
\pagestyle{empty}


\begin{document}

\lstset{language=[Objective]Caml,frame=shadowbox}

\input{qm2pi.front}

% section front matter (end)

\input{qm2pi.intro} 
 
% section introduction (end)

% \input{qm2pi.knotations} 

% section notation (end)

\input{qm2pi.process.calculi} 

% section concurrent_process_calculi_and_spatial_logics_ (end)
    
%\input{qm2pi.knots2pi} 

%\input{qm2pi.trefoil} 

%\input{qm2pi.mainthm} 

% subsection basic_interpretation (end)

%\input{qm2pi.rho.presentation} 
\subsection{The syntax and semantics of the notation system}\label{sub:the_syntax_and_semantics_of_the_notation_system} % (fold)

We now summarize a technical presentation of the calculus that
embodies our theory of dynamics. The typical presentation of such a
calculus follows the style of giving generators and relations on
them. The grammar, below, describing term constructors, freely
generates the set of processes, $\Proc$. This set is then quotiented
by a relation known as structural congruence and it is over this set
that the notion of dynamics is expressed. This presentation is
essentially that of \cite{MeredithR05} with the addition of
polyadicity and summation. For readability we have relegated some of
the technical subtleties to an appendix.

\subsubsection{Process grammar}\label{subsub:process_grammar}

\begin{mathpar}
  \inferrule* [lab=synchronization] {} {{M} \bc \pzero \;|\; x?F \;|\; x!C }
  \and
  \inferrule* [lab=abstraction] {} {{F} \bc (x)P}
  \and
  \inferrule* [lab=concretion] {} {{C} \bc \langle Q \rangle}
  \and
  \inferrule* [lab=process] {} {{P,Q} \bc M \;| \;P|Q \;|\; @{x}}
  \and
  \inferrule* [lab=name] {} {{x} \bc \quotep{P}}
\end{mathpar} 

Note that $\vec{x}$ (resp. $\vec{P}$) denotes a vector of names
(resp. processes) of length $|\vec{x}|$ (resp. $|\vec{P}|$). We adopt
the following useful abbreviations.

\begin{mathpar}
   x?(\vec{y}).P := x.(\vec{y})P \and  x\clift{\vec{P}} := x.\clift{\vec{P}}
   \and x!(y) := \lift{x}{\dropn{y}}
   \and \Pi_{i=0}^{n-1}P_i := P_0 | \ldots | P_{n-1}
\end{mathpar}

\subsubsection{Structural congruence}

\paragraph{Free and bound names and alpha-equivalence.} At the
core of structural equivalence is alpha-equivalence which identifies
process that are the same up to a change of variable. Formally, we
recognize the distinction between free and bound names. The free names
of a process, $\freenames{P}$, may be calculated recursively as
follows:

\begin{mathpar}
\freenames{\pzero} := \emptyset
  \and \\
  \freenames{x?(y).P} := \{ x \} \cup (\freenames{P} \setminus \{ y \})
  \and 
  \freenames{x!\langle P \rangle} := \{ x \} \cup \{ P \} 
  \and \\
  \freenames{P|Q} := \freenames{P} \cup \freenames{Q}
  \and \\
  \freenames{@{x}} := \{ x \}
\end{mathpar}

$\pi$
$\quotep{\pi}$

$\freenames{-} : \pi \to \mathcal{P}(\quotep{\pi})$

\begin{eqnarray*}
  \freenames{\pzero} & := & \emptyset \\
  \freenames{x?(y).P} & := & \{ x \} \cup (\freenames{P} \setminus \{ y \}) \\
  \freenames{x!\langle P \rangle} & := & \{ x \} \cup \{ P \} \\
  \freenames{P|Q} & := & \freenames{P} \cup \freenames{Q} \\
  \freenames{\dropn{x}} & := & \{ x \}
\end{eqnarray*}

The bound names of a process, $\boundnames{P}$, are those names occurring in $P$
that are not free. For example, in $x?(y).0$, the name $x$ is free, while $y$ is bound.

\begin{mathpar}
  \inferrule* [lab=monoidal-laws] {} { P|Q \equiv Q|P \and P|0 \equiv P \and P|(Q|R) \equiv (P|Q)|R }
\end{mathpar}

\begin{mathpar}
  \inferrule* [lab=alpha-equivalence] {} { (x)P \equiv (y)P\{y/x\} \and y \not\in \freenames{P} }
\end{mathpar}

\begin{definition}
Then two processes, $P,Q$, are alpha-equivalent if $P = Q\{\vec{y}/\vec{x}\}$ for
some $\vec{x} \in \boundnames{Q},\vec{y} \in \boundnames{P}$, where $Q\{\vec{y}/\vec{x}\}$
denotes the capture-avoiding substitution of $\vec{y}$ for $\vec{x}$ in $Q$.
\end{definition}

\begin{definition}
  The {\em structural congruence} \cite{SangiorgiWalker} , $\equiv$,
  between processes is the least congruence containing
  alpha-equivalence, satisfying the abelian monoid laws
  (associativity, commutativity and $\pzero$ as identity) for parallel
  composition $|$ and for summation $+$.
\end{definition}

\subsection{Name equivalence}

We take name equivalence, written $\nameeq$, to be the smallest
equivalence relation generated by the following rules.

\begin{mathpar}
\inferrule*[lab=Quote-drop]
{ }
{ \quotep{@{x}} \nameeq x }

\inferrule*[lab=Struct-equiv]
{ P \scong Q }
{ \quotep{P} \nameeq \quotep{Q} }
\end{mathpar}

The astute reader will have noticed that the mutual recursion of names
and processes imposes a mutual recursion on alpha-equivalence and
structural equivalence via name-equivalence. Fortunately, all of this
works out pleasantly and we may calculate in the natural way, free of
concern. The reader interested in the details is referred to the
appendix \ref{appendix:rho_details}.

\subsection{Substitution}

We use $\Proc$ for the set of processes, $\QProc$ for the set of
names, and $\id{\{}\vec{y} / \vec{x} \id{\}}$ to denote partial maps,
$s : \QProc \rightarrow \QProc$. A map, $s$ lifts, uniquely, to a map
on process terms, $\widehat{s} : \Proc \rightarrow \Proc$ by the
following equations.

\begin{mathpar}
  (0) \psubstp{Q}{P} := 0 \\
  (R \juxtap S) \psubstp{Q}{P}
  :=    
  (R)\psubstp{Q}{P} \juxtap (S) \psubstp{Q}{P} \\
  (x?(y).R) \psubstp{Q}{P}    
  :=    
  (x)\substp{Q}{P} (z)\concat( (R \psubstn{z}{y}) \psubstp{Q}{P} ) \\
  (\lift{x}{R}) \psubstp{Q}{P}  
  :=
  \lift{(x)\substp{Q}{P}}{ R \psubstp{Q}{P} } \\
%   (\dropn{x})  \psubstp{Q}{P}       
%   := 
%   \left\{ 
%     \begin{array}{ccc} 
%       \dropn{\quotep{Q}} & & x \nameeq \quotep{P} \\
%       \dropn{x} & & otherwise \\
%     \end{array}
%   \right. 
  (\dropn{x})  \psubstp{Q}{P}       
  := 
  \left\{ 
    \begin{array}{ccc} 
      Q & & x \nameeq \quotep{P} \\
      \dropn{x} & & otherwise \\
    \end{array}
  \right.
\end{mathpar}
 

where

\begin{eqnarray}
  (x)\id{\{} \lpquote Q \rpquote / \lpquote P \rpquote \id{\}}            = 
  \left\{ 
    \begin{array}{ccc}
      \lpquote Q \rpquote & & x \nameeq \lpquote P \rpquote \\
      x & & otherwise \\
    \end{array}
  \right. \nonumber
\end{eqnarray}

and $z$ is chosen distinct from $\quotep{P}$, $\quotep{Q}$, the free
names in $Q$, and all the names in $R$. Our $\alpha$-equivalence will
be built in the standard way from this substitution.

\begin{remark}\label{rem:no_self_referential_names}
  One consequence of these definitions is that $\forall P. \quotep{P}
  \not\in \freenames{P}$.
\end{remark}

\subsection{ Dynamic quote: an example }

Anticipating something of what's to come, consider applying the
substitution, $\widehat{\id{\{}u / z \id{\}}}$, to the following pair
of processes, $\lift{w}{y!(z)}$ and $w[ \lpquote y!(z) \rpquote ]$.

\begin{eqnarray}
	\lift{w}{y!(z)}\widehat{\id{\{}u / z \id{\}}}
		& = &
		\lift{w}{y!(u)} \nonumber\\
	w[ \lpquote y!(z) \rpquote ] \widehat{ \id{\{}u / z \id{\}} }
		& = &
		w[ \lpquote y!(z) \rpquote ] \nonumber
\end{eqnarray}

Because the body of the process between quotes is impervious to
substitution, we get radically different answers. In fact, by
examining the first process in an input context,
e.g. $x?(z).\lift{w}{y!(z)}$, we see that the process under the lift
operator may be shaped by prefixed inputs binding a name inside it. In
this sense, the lift operator will be seen as a way to dynamically
construct processes before reifying them as names.

Finally equipped with these standard features we can present the
dynamics of the calculus.

\subsubsection{Operational semantics} 

Finally, we introduce the computational dynamics. What marks these
algebras as distinct from other more traditionally studied algebraic
structures, e.g. vector spaces or polynomial rings, is the manner in
which dynamics is captured. In traditional structures, dynamics is typically
expressed through morphisms between such structures, as in linear maps
between vector spaces or morphisms between rings. In algebras
associated with the semantics of computation, the dynamics is
expressed as part of the algebraic structure itself, through a
reduction reduction relation typically denoted by $\red$. Below, we
give a recursive presentation of this relation for the calculus used
in the encoding.

$\red \subseteq \pi \times \pi$
$\red : \pi \to \mathcal{P}(\pi)$

\begin{mathpar}
  \inferrule* [lab=Comm] { \textsf{match}( x_{src}, x_{trgt} ) } { x_{trgt}?(y)P \; | \; x_{src}!\langle {Q} \rangle \red P\{\quotep{Q}/y}\} }
  \and \\
  \inferrule* [lab=Par] {{P} \red {P}'} {{{P} | {Q}} \red {{P}' | {Q}}}
  \and
  \inferrule* [lab=Equiv]{{{P} \scong {P}'} \andalso {{P}' \red {Q}'} \andalso {{Q}' \scong {Q}}}{{P} \red {Q}}
\end{mathpar}

\begin{eqnarray*}
  match_{\equiv} (\quotep{P},\quotep{Q}) & := & P \equiv Q \\
  match_{\dagger}(\quotep{P},\quotep{Q}) & := & \forall R. P|Q \red^{*} R => R \red^{*} 0 \\
  match_{K}(\quotep{P},\quotep{Q}) & := & K \mbox{ for some context } K
\end{eqnarray*}

$u?(x)P | u!\langle Q \rangle \red P\{\quotep{Q}/x\}$

%We write $\wred$ for $\red^*$, and $P\red$ if $\exists Q $ such that $ P \red Q$.
We write $P\red$ if $\exists Q $ such that $ P \red Q$ and $P\not\red$, otherwise.

\section{Replication}

As mentioned before, it is known that replication (and hence
recursion) can be implemented in a higher-order process algebra
\cite{SangiorgiWalker}. As our first example of calculation with the
machinery thus far presented we give the construction explicitly in
the {\rhoc}.

\begin{eqnarray}
	D_{x} & := & \prefix{x}{y}{(\binpar{\outputp{x}{y}}{@{y}})} \nonumber\\
	\bangp_{x}{P} & := & \binpar{{x}!\langle{\binpar{D_{x}}{P}}\rangle}{D_{x}} \nonumber
\end{eqnarray}

\begin{eqnarray}
	\bangp_{x}{P} & & \nonumber\\
	=
	& {x}!\langle{(\prefix{x}{y}{(\outputp{x}{y} | @{y})) | P}}\rangle 
	      | \prefix{x}{y}{(\outputp{x}{y} | @{y})} & \nonumber\\
	\red
	& (\outputp{x}{y} | @{y})\substn{\quotep{(\prefix{x}{y}{(@{y} | \outputp{x}{y})) | P}}}{y} & \nonumber\\
	=
	& \outputp{x}{\quotep{(\prefix{x}{y}{(\outputp{x}{y} | @{y})) | P}}}
	  | {(\prefix{x}{y}{(\outputp{x}{y} | @{y})) | P}} & \nonumber\\
	\red
	& \ldots & \nonumber\\
	\red^*
	& P | P | \ldots & \nonumber
\end{eqnarray}

Of course, this encoding, as an implementation, runs away, unfolding
$\bangp{P}$ eagerly. A lazier and more implementable replication
operator, restricted to input-guarded processes, may be obtained as follows.

\begin{eqnarray}
\bangp{\prefix{u}{v}{P}} 
	:= 
	\binpar{\lift{x}{\prefix{u}{v}{(\binpar{D(x)}{P})}}}{D(x)} \nonumber
\end{eqnarray}

\begin{remark}
  Note that the lazier definition still does not deal with summation
  or mixed summation (i.e. sums over input and output). The reader is
  invited to construct definitions of replication that deal with these
  features. 

  Further, the definitions are parameterized in a name, $x$. Can you,
  gentle reader, make a definition that eliminates this parameter and
  guarantees no accidental interaction between the replication
  machinery and the process being replicated -- i.e. no accidental
  sharing of names used by the process to get its work done and the
  name(s) used by the replication to effect copying. This latter
  revision of the definition of replication is crucial to obtaining
  the expected identity $!!P \sim !P$.
\end{remark}

\begin{remark}\label{rem:paradoxical_combinator}
  The reader familiar with the lambda calculus will have noticed the
  similarity between $D$ and the paradoxical combinator.

  [Ed. note: the existence of this seems to suggest we have to be more
  restrictive on the set of processes and names we admit if we are to
  support no-cloning.]
\end{remark}

\subsubsection{Bisimulation}

The computational dynamics gives rise to another kind of equivalence,
the equivalence of computational behavior. As previously mentioned
this is typically captured \emph{via} some form of bisimulation.

% The notion we use in this paper is weak barbed bisimulation
% \cite{milner91polyadicpi}.

The notion we use in this paper is derived from weak barbed
bisimulation \cite{milner91polyadicpi}. 

\begin{definition}
An \emph{observation relation}, $\downarrow_{\mathcal N}$, over a set
of names, $\mathcal N$, is the smallest relation satisfying the rules
below.

\infrule[Out-barb]{y \in {\mathcal N}, \; x \nameeq y}
		  {\outputp{x}{v} \downarrow_{\mathcal N} x}
\infrule[Par-barb]{\mbox{$P\downarrow_{\mathcal N} x$ or $Q\downarrow_{\mathcal N} x$}}
		  {\binpar{P}{Q} \downarrow_{\mathcal N} x}

We write $P \Downarrow_{\mathcal N} x$ if there is $Q$ such that 
$P \wred Q$ and $Q \downarrow_{\mathcal N} x$.
\end{definition}

\begin{definition}
%\label{def.bbisim}
An  ${\mathcal N}$-\emph{barbed bisimulation} over a set of names, ${\mathcal N}$, is a symmetric binary relation 
${\mathcal S}_{\mathcal N}$ between agents such that $P\rel{S}_{\mathcal N}Q$ implies:
\begin{enumerate}
\item If $P \red P'$ then $Q \wred Q'$ and $P'\rel{S}_{\mathcal N} Q'$.
\item If $P\downarrow_{\mathcal N} x$, then $Q\Downarrow_{\mathcal N} x$.
\end{enumerate}
$P$ is ${\mathcal N}$-barbed bisimilar to $Q$, written
$P \wbbisim_{\mathcal N} Q$, if $P \rel{S}_{\mathcal N} Q$ for some ${\mathcal N}$-barbed bisimulation ${\mathcal S}_{\mathcal N}$.
\end{definition}

$\mathcal{R} \subseteq \pi \times \pi$

$P \mathcal{R} Q => \forall P'. P \red P' \Rightarrow \exists Q'. Q \red Q', P' \mathcal{R} Q'$

$P \vdash x \Rightarrow Q \vdash x$

\begin{mathpar}
  \inferrule*[lab=Out-barb]{x \nameeq y}{{y}!\langle{Q}\rangle \vdash x}
  \and
  \inferrule*[lab=Par-barb]{\mbox{$P\vdash x$ or $Q\vdash x$}}{\binpar{P}{Q} \vdash x}
\end{mathpar}

\subsubsection{Contexts}

One of the principle advantages of computational calculi like the
$\pi$-calculus is a well-defined notion of context,
contextual-equivalence and a correlation between
contextual-equivalence and notions of bisimulation. The notion of
context allows the decomposition of a process into (sub-)process and
its syntactic environment, its context. Thus, a context may be
thought of as a process with a ``hole'' (written $\Box$) in it. The
application of a context $M$ to a process $P$, written $M[P]$, is
tantamount to filling the hole in $M$ with $P$. In this paper we do
not need the full weight of this theory, but do make use of the notion
of context in the proof the main theorem. 

\begin{mathpar}
  \inferrule* [lab=summation] {} {{M_{M},M_{N}} \bc \Box \;|\; x.M_{A} \;|\; M_{M}+M_{N}}
  \and
  \inferrule* [lab=agent] {} {{M_{A}} \bc (\vec{x})M_{P} \;| \; \clift{P_0,\ldots,M_{P},\ldots,P_N}}
  \and \\
  \inferrule* [lab=process] {} {{M_{P}} \bc M_{N} \;| \;P|M_{P} }
\end{mathpar} 

\begin{mathpar}
  \inferrule* [lab=sychronization] {} {M_{N} \bc \Box \;|\; x?M_{F} \;|\; x!M_{C}}
  \and
  \inferrule* [lab=abstraction] {} {{M_{F}} \bc (x)M_{P} }
  \and
  \inferrule* [lab=concretion] {} {{M_{C}} \bc \langle M_{P} \rangle }
  \and \\
  \inferrule* [lab=process] {} {{M_{P}} \bc M_{N} \;| \;P|M_{P} }
\end{mathpar}

\begin{definition}[contextual application] Given a context $M$, and
  process $P$, we define the \emph{contextual application}, $M[P] :=
  M\{P/\Box\}$. That is, the contextual application of M to P is the
  substitution of $P$ for $\Box$ in $M$.
\end{definition}

$\meaningof{-} : L \to \mathcal{P}(\pi)$

\begin{mathpar}
  \inferrule* [lab=collection] {} {\meaningof{true} = \pi, \and \meaningof{~E} = \pi \setminus \meaningof{E}, \and \meaningof{E_{1} \& E_{2}} = \meaningof{E_{1}} \cap \meaningof{E_{2}}}
\end{mathpar}

\begin{mathpar}
  \inferrule* [lab=structure] {} {\meaningof{0} = \{ P \in \pi | P \equiv 0 \}, \and \\ \meaningof{E_1 | E_2} = \{ P \in \pi | P \equiv P_{1} | P_{2}, P_{1} \in \meaningof{E_{1}}, P_{2} \in \meaningof{E_2}\} }
\end{mathpar}

\begin{mathpar}
 \inferrule* [lab=behavior] {} {\meaningof{\langle a?b \rangle E} = \{ P \in \pi | P \equiv Q | u?(y)P', \\ \and \\\\ \and \\ \;\;\; u \in \meaningof{a}, \forall z.P'\{z/y\} \in \meaningof{E\{z/b\}}\}, \and \\ \meaningof{a!E} = \{ P \in \pi | P \equiv Q | x!\langle P' \rangle, x \in \meaningof{a} P' \in \meaningof{E}\} }
\end{mathpar}

\begin{mathpar}
 \inferrule* [lab=nominal] {} {\meaningof{\quotep{E}} = \{ \quotep{P} \in \quotep{\pi} | P \in \meaningof{E} \}, \and \meaningof{\quotep{P}} = \{ \quotep{Q} \in \quotep{\pi} | P \equiv Q \} \and \\ \meaningof{@\quotep{E}} = \{ P \in \pi | P \equiv @x, x \in \meaningof{E} \}}
\end{mathpar}

\begin{eqnarray*}
  \\
  \meaningof{-} : TS \to ST
\end{eqnarray*}

\begin{eqnarray*}
  \\
  L : TS \to ST
\end{eqnarray*}

\begin{eqnarray*}
  \\
  P \models E \iff P \in \meaningof{E}
\end{eqnarray*}

\begin{eqnarray*}
  P \approx_{L} Q \iff \forall E \in L. P \models E \iff Q \models E
\end{eqnarray*}

\begin{eqnarray*}
  P \approx_{K} Q
\end{eqnarray*}

\begin{eqnarray*}
  P \approx Q
\end{eqnarray*}

$\approx_{K} = \approx = \approx_{L}$

\subsubsection{Contextual duality}

Note that contexts extend the quotation operation to a family of
operations from processes to names. Given a context, $M$, we can
define a \emph{nominal context}, $\quotep{M}$ by $\quotep{M}[P] :=
\quotep{M[P]}$. To foreshadow what is to come we observe that these
operations enjoy a duality with processes very much like the duality
between vectors and maps from vectors to scalars.

Further, because the calculus is essentially higher-order, we have a
correspondence between contexts and processes. More specifically,
given a name $x$ and a context $M$ we can construct $M^{*}_{x}$ such
that 

\begin{mathpar}
  M^{*}_{x} | \lift{x}{P} \red M[P]
\end{mathpar}

namely,

\begin{mathpar}
  M^{*}_{x} := x?(u).M[\dropn{u}]
\end{mathpar}

The dependence of $M^{*}_{x}$ on a name makes it an abstraction, 

\begin{mathpar}
  M^{*} := (x)x?(u).M[\dropn{u}]
\end{mathpar}

\subsection{Additional notation}

It will sometimes be convenient to denote the process a name
quotes. We already have the notation $x = \quotep{P}$, but it will be
convenient to introduce an alternate notation, $\procn{x}$, when we
want to emphasize the connection to the use of the name. Note that, by
virtue of name equivalence, $\quotep{\procn{x}} \nameeq x$; so, the
notation is consistent with previous definitions.

Further, because names have structure it is possible to effect
substitutions on the basis of that structure. This means we need to
upgrade our notation for substitutions, which we accomplish by
adapting comprehension notation. Thus,

\begin{mathpar}
  P\{ y / x : x \in S \}
\end{mathpar}

is interpreted to mean the process derived from P by replacing (in a
capture-avoiding manner) each occurrence of $x$ in $S$ by $y$. For example,

\begin{mathpar}
  P\{ \quotep{\procn{x}|\procn{x}} / x : x \in \freenames{P} \}
\end{mathpar}

will replace each (occurrence) of a free name $x$ in $P$ by
$\quotep{\procn{x}|\procn{x}}$.

Also, we will avail ourselves of the notation $x^{L}$ and $x^{R}$ to
denote injections of a name into disjoint copies of the name
space. There are numerous ways to accomplish this. One example can be
found in \cite{MeredithR05}. This notation overloads to vectors of
names: $\vec{x}^{\pi} := (x_{i}^{\pi} \; : \; 0 \leq i < |\vec{x}| )$ where $\pi \in \{L,R\}$.

We also use $P^{\Box} := P|\Box$.

In \cite{MeredithR05} an interpretation of the new operator is
given. It turns out that there are several possible interpretations
all enjoying the requisite algebraic properties of the operator (see
\cite{milner91polyadicpi}). We will therefore make liberal use of
$(\nu\; \vec{x})P$.

% subsection the_syntax_and_semantics_of_the_notation_system (end)   

\input{qm2pi.qmops} 

\input{qm2pi.sterngerlach} 

\input{qm2pi.metric} 

% section concurrent_process_calculi (end)

%\input{qm2pi.proofsketch}

% section proof sketch (end)

%\input{qm2pi.slviaknots} 

% section spatial logic via knots (end)

\input{qm2pi.conclusion}

% section conclusion (end)

%\input{qm2pi.dtcodes} 

% section wiring algorithm (end)

\input{qm2pi.ack} 

% section acknowledgments (end)

\newpage


\bibliographystyle{plain}   
\bibliography{../../biblios/main.bib}

\input{qm2pi.rhodetails}

\end{document}

 

% section wiring algorithm (end)

\documentclass[12pt]{llncs}
%\documentclass{jktr}

\usepackage[pdftex]{hyperref}                   
\usepackage {listings}
\usepackage {mathpartir}
\usepackage{bcprules}
%\usepackage{listings}
                       
\usepackage{graphicx} 
%\usepackage[margins=2.5cm,nohead,nofoot]{geometry}
%\usepackage{geometry}
\usepackage{amsfonts}
\usepackage{amstext}
\usepackage{latexsym}
\usepackage{amssymb}
\usepackage{color}


%\include{myPreamble}
\include{qm2pi.local} 

%\ifpdf
%\usepackage[pdftex]{graphicx}
%\else
%\usepackage{graphicx}
%\fi

 % \ifpdf
%  \usepackage{pdfsync}
%  \if


%\title{Brief Article}
%\author{David F. Snyder}
%\author{L.G. Meredith}

%\address{Dept. of Math., Texas State University--San Marcos, San Marcos, TX 78666}
       
\pagestyle{empty}


\begin{document}

\lstset{language=[Objective]Caml,frame=shadowbox}

\input{qm2pi.front}

% section front matter (end)

\input{qm2pi.intro} 
 
% section introduction (end)

% \input{qm2pi.knotations} 

% section notation (end)

\input{qm2pi.process.calculi} 

% section concurrent_process_calculi_and_spatial_logics_ (end)
    
%\input{qm2pi.knots2pi} 

%\input{qm2pi.trefoil} 

%\input{qm2pi.mainthm} 

% subsection basic_interpretation (end)

%\input{qm2pi.rho.presentation} 
\subsection{The syntax and semantics of the notation system}\label{sub:the_syntax_and_semantics_of_the_notation_system} % (fold)

We now summarize a technical presentation of the calculus that
embodies our theory of dynamics. The typical presentation of such a
calculus follows the style of giving generators and relations on
them. The grammar, below, describing term constructors, freely
generates the set of processes, $\Proc$. This set is then quotiented
by a relation known as structural congruence and it is over this set
that the notion of dynamics is expressed. This presentation is
essentially that of \cite{MeredithR05} with the addition of
polyadicity and summation. For readability we have relegated some of
the technical subtleties to an appendix.

\subsubsection{Process grammar}\label{subsub:process_grammar}

\begin{mathpar}
  \inferrule* [lab=synchronization] {} {{M} \bc \pzero \;|\; x?F \;|\; x!C }
  \and
  \inferrule* [lab=abstraction] {} {{F} \bc (x)P}
  \and
  \inferrule* [lab=concretion] {} {{C} \bc \langle Q \rangle}
  \and
  \inferrule* [lab=process] {} {{P,Q} \bc M \;| \;P|Q \;|\; @{x}}
  \and
  \inferrule* [lab=name] {} {{x} \bc \quotep{P}}
\end{mathpar} 

Note that $\vec{x}$ (resp. $\vec{P}$) denotes a vector of names
(resp. processes) of length $|\vec{x}|$ (resp. $|\vec{P}|$). We adopt
the following useful abbreviations.

\begin{mathpar}
   x?(\vec{y}).P := x.(\vec{y})P \and  x\clift{\vec{P}} := x.\clift{\vec{P}}
   \and x!(y) := \lift{x}{\dropn{y}}
   \and \Pi_{i=0}^{n-1}P_i := P_0 | \ldots | P_{n-1}
\end{mathpar}

\subsubsection{Structural congruence}

\paragraph{Free and bound names and alpha-equivalence.} At the
core of structural equivalence is alpha-equivalence which identifies
process that are the same up to a change of variable. Formally, we
recognize the distinction between free and bound names. The free names
of a process, $\freenames{P}$, may be calculated recursively as
follows:

\begin{mathpar}
\freenames{\pzero} := \emptyset
  \and \\
  \freenames{x?(y).P} := \{ x \} \cup (\freenames{P} \setminus \{ y \})
  \and 
  \freenames{x!\langle P \rangle} := \{ x \} \cup \{ P \} 
  \and \\
  \freenames{P|Q} := \freenames{P} \cup \freenames{Q}
  \and \\
  \freenames{@{x}} := \{ x \}
\end{mathpar}

$\pi$
$\quotep{\pi}$

$\freenames{-} : \pi \to \mathcal{P}(\quotep{\pi})$

\begin{eqnarray*}
  \freenames{\pzero} & := & \emptyset \\
  \freenames{x?(y).P} & := & \{ x \} \cup (\freenames{P} \setminus \{ y \}) \\
  \freenames{x!\langle P \rangle} & := & \{ x \} \cup \{ P \} \\
  \freenames{P|Q} & := & \freenames{P} \cup \freenames{Q} \\
  \freenames{\dropn{x}} & := & \{ x \}
\end{eqnarray*}

The bound names of a process, $\boundnames{P}$, are those names occurring in $P$
that are not free. For example, in $x?(y).0$, the name $x$ is free, while $y$ is bound.

\begin{mathpar}
  \inferrule* [lab=monoidal-laws] {} { P|Q \equiv Q|P \and P|0 \equiv P \and P|(Q|R) \equiv (P|Q)|R }
\end{mathpar}

\begin{mathpar}
  \inferrule* [lab=alpha-equivalence] {} { (x)P \equiv (y)P\{y/x\} \and y \not\in \freenames{P} }
\end{mathpar}

\begin{definition}
Then two processes, $P,Q$, are alpha-equivalent if $P = Q\{\vec{y}/\vec{x}\}$ for
some $\vec{x} \in \boundnames{Q},\vec{y} \in \boundnames{P}$, where $Q\{\vec{y}/\vec{x}\}$
denotes the capture-avoiding substitution of $\vec{y}$ for $\vec{x}$ in $Q$.
\end{definition}

\begin{definition}
  The {\em structural congruence} \cite{SangiorgiWalker} , $\equiv$,
  between processes is the least congruence containing
  alpha-equivalence, satisfying the abelian monoid laws
  (associativity, commutativity and $\pzero$ as identity) for parallel
  composition $|$ and for summation $+$.
\end{definition}

\subsection{Name equivalence}

We take name equivalence, written $\nameeq$, to be the smallest
equivalence relation generated by the following rules.

\begin{mathpar}
\inferrule*[lab=Quote-drop]
{ }
{ \quotep{@{x}} \nameeq x }

\inferrule*[lab=Struct-equiv]
{ P \scong Q }
{ \quotep{P} \nameeq \quotep{Q} }
\end{mathpar}

The astute reader will have noticed that the mutual recursion of names
and processes imposes a mutual recursion on alpha-equivalence and
structural equivalence via name-equivalence. Fortunately, all of this
works out pleasantly and we may calculate in the natural way, free of
concern. The reader interested in the details is referred to the
appendix \ref{appendix:rho_details}.

\subsection{Substitution}

We use $\Proc$ for the set of processes, $\QProc$ for the set of
names, and $\id{\{}\vec{y} / \vec{x} \id{\}}$ to denote partial maps,
$s : \QProc \rightarrow \QProc$. A map, $s$ lifts, uniquely, to a map
on process terms, $\widehat{s} : \Proc \rightarrow \Proc$ by the
following equations.

\begin{mathpar}
  (0) \psubstp{Q}{P} := 0 \\
  (R \juxtap S) \psubstp{Q}{P}
  :=    
  (R)\psubstp{Q}{P} \juxtap (S) \psubstp{Q}{P} \\
  (x?(y).R) \psubstp{Q}{P}    
  :=    
  (x)\substp{Q}{P} (z)\concat( (R \psubstn{z}{y}) \psubstp{Q}{P} ) \\
  (\lift{x}{R}) \psubstp{Q}{P}  
  :=
  \lift{(x)\substp{Q}{P}}{ R \psubstp{Q}{P} } \\
%   (\dropn{x})  \psubstp{Q}{P}       
%   := 
%   \left\{ 
%     \begin{array}{ccc} 
%       \dropn{\quotep{Q}} & & x \nameeq \quotep{P} \\
%       \dropn{x} & & otherwise \\
%     \end{array}
%   \right. 
  (\dropn{x})  \psubstp{Q}{P}       
  := 
  \left\{ 
    \begin{array}{ccc} 
      Q & & x \nameeq \quotep{P} \\
      \dropn{x} & & otherwise \\
    \end{array}
  \right.
\end{mathpar}
 

where

\begin{eqnarray}
  (x)\id{\{} \lpquote Q \rpquote / \lpquote P \rpquote \id{\}}            = 
  \left\{ 
    \begin{array}{ccc}
      \lpquote Q \rpquote & & x \nameeq \lpquote P \rpquote \\
      x & & otherwise \\
    \end{array}
  \right. \nonumber
\end{eqnarray}

and $z$ is chosen distinct from $\quotep{P}$, $\quotep{Q}$, the free
names in $Q$, and all the names in $R$. Our $\alpha$-equivalence will
be built in the standard way from this substitution.

\begin{remark}\label{rem:no_self_referential_names}
  One consequence of these definitions is that $\forall P. \quotep{P}
  \not\in \freenames{P}$.
\end{remark}

\subsection{ Dynamic quote: an example }

Anticipating something of what's to come, consider applying the
substitution, $\widehat{\id{\{}u / z \id{\}}}$, to the following pair
of processes, $\lift{w}{y!(z)}$ and $w[ \lpquote y!(z) \rpquote ]$.

\begin{eqnarray}
	\lift{w}{y!(z)}\widehat{\id{\{}u / z \id{\}}}
		& = &
		\lift{w}{y!(u)} \nonumber\\
	w[ \lpquote y!(z) \rpquote ] \widehat{ \id{\{}u / z \id{\}} }
		& = &
		w[ \lpquote y!(z) \rpquote ] \nonumber
\end{eqnarray}

Because the body of the process between quotes is impervious to
substitution, we get radically different answers. In fact, by
examining the first process in an input context,
e.g. $x?(z).\lift{w}{y!(z)}$, we see that the process under the lift
operator may be shaped by prefixed inputs binding a name inside it. In
this sense, the lift operator will be seen as a way to dynamically
construct processes before reifying them as names.

Finally equipped with these standard features we can present the
dynamics of the calculus.

\subsubsection{Operational semantics} 

Finally, we introduce the computational dynamics. What marks these
algebras as distinct from other more traditionally studied algebraic
structures, e.g. vector spaces or polynomial rings, is the manner in
which dynamics is captured. In traditional structures, dynamics is typically
expressed through morphisms between such structures, as in linear maps
between vector spaces or morphisms between rings. In algebras
associated with the semantics of computation, the dynamics is
expressed as part of the algebraic structure itself, through a
reduction reduction relation typically denoted by $\red$. Below, we
give a recursive presentation of this relation for the calculus used
in the encoding.

$\red \subseteq \pi \times \pi$
$\red : \pi \to \mathcal{P}(\pi)$

\begin{mathpar}
  \inferrule* [lab=Comm] { \textsf{match}( x_{src}, x_{trgt} ) } { x_{trgt}?(y)P \; | \; x_{src}!\langle {Q} \rangle \red P\{\quotep{Q}/y}\} }
  \and \\
  \inferrule* [lab=Par] {{P} \red {P}'} {{{P} | {Q}} \red {{P}' | {Q}}}
  \and
  \inferrule* [lab=Equiv]{{{P} \scong {P}'} \andalso {{P}' \red {Q}'} \andalso {{Q}' \scong {Q}}}{{P} \red {Q}}
\end{mathpar}

\begin{eqnarray*}
  match_{\equiv} (\quotep{P},\quotep{Q}) & := & P \equiv Q \\
  match_{\dagger}(\quotep{P},\quotep{Q}) & := & \forall R. P|Q \red^{*} R => R \red^{*} 0 \\
  match_{K}(\quotep{P},\quotep{Q}) & := & K \mbox{ for some context } K
\end{eqnarray*}

$u?(x)P | u!\langle Q \rangle \red P\{\quotep{Q}/x\}$

%We write $\wred$ for $\red^*$, and $P\red$ if $\exists Q $ such that $ P \red Q$.
We write $P\red$ if $\exists Q $ such that $ P \red Q$ and $P\not\red$, otherwise.

\section{Replication}

As mentioned before, it is known that replication (and hence
recursion) can be implemented in a higher-order process algebra
\cite{SangiorgiWalker}. As our first example of calculation with the
machinery thus far presented we give the construction explicitly in
the {\rhoc}.

\begin{eqnarray}
	D_{x} & := & \prefix{x}{y}{(\binpar{\outputp{x}{y}}{@{y}})} \nonumber\\
	\bangp_{x}{P} & := & \binpar{{x}!\langle{\binpar{D_{x}}{P}}\rangle}{D_{x}} \nonumber
\end{eqnarray}

\begin{eqnarray}
	\bangp_{x}{P} & & \nonumber\\
	=
	& {x}!\langle{(\prefix{x}{y}{(\outputp{x}{y} | @{y})) | P}}\rangle 
	      | \prefix{x}{y}{(\outputp{x}{y} | @{y})} & \nonumber\\
	\red
	& (\outputp{x}{y} | @{y})\substn{\quotep{(\prefix{x}{y}{(@{y} | \outputp{x}{y})) | P}}}{y} & \nonumber\\
	=
	& \outputp{x}{\quotep{(\prefix{x}{y}{(\outputp{x}{y} | @{y})) | P}}}
	  | {(\prefix{x}{y}{(\outputp{x}{y} | @{y})) | P}} & \nonumber\\
	\red
	& \ldots & \nonumber\\
	\red^*
	& P | P | \ldots & \nonumber
\end{eqnarray}

Of course, this encoding, as an implementation, runs away, unfolding
$\bangp{P}$ eagerly. A lazier and more implementable replication
operator, restricted to input-guarded processes, may be obtained as follows.

\begin{eqnarray}
\bangp{\prefix{u}{v}{P}} 
	:= 
	\binpar{\lift{x}{\prefix{u}{v}{(\binpar{D(x)}{P})}}}{D(x)} \nonumber
\end{eqnarray}

\begin{remark}
  Note that the lazier definition still does not deal with summation
  or mixed summation (i.e. sums over input and output). The reader is
  invited to construct definitions of replication that deal with these
  features. 

  Further, the definitions are parameterized in a name, $x$. Can you,
  gentle reader, make a definition that eliminates this parameter and
  guarantees no accidental interaction between the replication
  machinery and the process being replicated -- i.e. no accidental
  sharing of names used by the process to get its work done and the
  name(s) used by the replication to effect copying. This latter
  revision of the definition of replication is crucial to obtaining
  the expected identity $!!P \sim !P$.
\end{remark}

\begin{remark}\label{rem:paradoxical_combinator}
  The reader familiar with the lambda calculus will have noticed the
  similarity between $D$ and the paradoxical combinator.

  [Ed. note: the existence of this seems to suggest we have to be more
  restrictive on the set of processes and names we admit if we are to
  support no-cloning.]
\end{remark}

\subsubsection{Bisimulation}

The computational dynamics gives rise to another kind of equivalence,
the equivalence of computational behavior. As previously mentioned
this is typically captured \emph{via} some form of bisimulation.

% The notion we use in this paper is weak barbed bisimulation
% \cite{milner91polyadicpi}.

The notion we use in this paper is derived from weak barbed
bisimulation \cite{milner91polyadicpi}. 

\begin{definition}
An \emph{observation relation}, $\downarrow_{\mathcal N}$, over a set
of names, $\mathcal N$, is the smallest relation satisfying the rules
below.

\infrule[Out-barb]{y \in {\mathcal N}, \; x \nameeq y}
		  {\outputp{x}{v} \downarrow_{\mathcal N} x}
\infrule[Par-barb]{\mbox{$P\downarrow_{\mathcal N} x$ or $Q\downarrow_{\mathcal N} x$}}
		  {\binpar{P}{Q} \downarrow_{\mathcal N} x}

We write $P \Downarrow_{\mathcal N} x$ if there is $Q$ such that 
$P \wred Q$ and $Q \downarrow_{\mathcal N} x$.
\end{definition}

\begin{definition}
%\label{def.bbisim}
An  ${\mathcal N}$-\emph{barbed bisimulation} over a set of names, ${\mathcal N}$, is a symmetric binary relation 
${\mathcal S}_{\mathcal N}$ between agents such that $P\rel{S}_{\mathcal N}Q$ implies:
\begin{enumerate}
\item If $P \red P'$ then $Q \wred Q'$ and $P'\rel{S}_{\mathcal N} Q'$.
\item If $P\downarrow_{\mathcal N} x$, then $Q\Downarrow_{\mathcal N} x$.
\end{enumerate}
$P$ is ${\mathcal N}$-barbed bisimilar to $Q$, written
$P \wbbisim_{\mathcal N} Q$, if $P \rel{S}_{\mathcal N} Q$ for some ${\mathcal N}$-barbed bisimulation ${\mathcal S}_{\mathcal N}$.
\end{definition}

$\mathcal{R} \subseteq \pi \times \pi$

$P \mathcal{R} Q => \forall P'. P \red P' \Rightarrow \exists Q'. Q \red Q', P' \mathcal{R} Q'$

$P \vdash x \Rightarrow Q \vdash x$

\begin{mathpar}
  \inferrule*[lab=Out-barb]{x \nameeq y}{{y}!\langle{Q}\rangle \vdash x}
  \and
  \inferrule*[lab=Par-barb]{\mbox{$P\vdash x$ or $Q\vdash x$}}{\binpar{P}{Q} \vdash x}
\end{mathpar}

\subsubsection{Contexts}

One of the principle advantages of computational calculi like the
$\pi$-calculus is a well-defined notion of context,
contextual-equivalence and a correlation between
contextual-equivalence and notions of bisimulation. The notion of
context allows the decomposition of a process into (sub-)process and
its syntactic environment, its context. Thus, a context may be
thought of as a process with a ``hole'' (written $\Box$) in it. The
application of a context $M$ to a process $P$, written $M[P]$, is
tantamount to filling the hole in $M$ with $P$. In this paper we do
not need the full weight of this theory, but do make use of the notion
of context in the proof the main theorem. 

\begin{mathpar}
  \inferrule* [lab=summation] {} {{M_{M},M_{N}} \bc \Box \;|\; x.M_{A} \;|\; M_{M}+M_{N}}
  \and
  \inferrule* [lab=agent] {} {{M_{A}} \bc (\vec{x})M_{P} \;| \; \clift{P_0,\ldots,M_{P},\ldots,P_N}}
  \and \\
  \inferrule* [lab=process] {} {{M_{P}} \bc M_{N} \;| \;P|M_{P} }
\end{mathpar} 

\begin{mathpar}
  \inferrule* [lab=sychronization] {} {M_{N} \bc \Box \;|\; x?M_{F} \;|\; x!M_{C}}
  \and
  \inferrule* [lab=abstraction] {} {{M_{F}} \bc (x)M_{P} }
  \and
  \inferrule* [lab=concretion] {} {{M_{C}} \bc \langle M_{P} \rangle }
  \and \\
  \inferrule* [lab=process] {} {{M_{P}} \bc M_{N} \;| \;P|M_{P} }
\end{mathpar}

\begin{definition}[contextual application] Given a context $M$, and
  process $P$, we define the \emph{contextual application}, $M[P] :=
  M\{P/\Box\}$. That is, the contextual application of M to P is the
  substitution of $P$ for $\Box$ in $M$.
\end{definition}

$\meaningof{-} : L \to \mathcal{P}(\pi)$

\begin{mathpar}
  \inferrule* [lab=collection] {} {\meaningof{true} = \pi, \and \meaningof{~E} = \pi \setminus \meaningof{E}, \and \meaningof{E_{1} \& E_{2}} = \meaningof{E_{1}} \cap \meaningof{E_{2}}}
\end{mathpar}

\begin{mathpar}
  \inferrule* [lab=structure] {} {\meaningof{0} = \{ P \in \pi | P \equiv 0 \}, \and \\ \meaningof{E_1 | E_2} = \{ P \in \pi | P \equiv P_{1} | P_{2}, P_{1} \in \meaningof{E_{1}}, P_{2} \in \meaningof{E_2}\} }
\end{mathpar}

\begin{mathpar}
 \inferrule* [lab=behavior] {} {\meaningof{\langle a?b \rangle E} = \{ P \in \pi | P \equiv Q | u?(y)P', \\ \and \\\\ \and \\ \;\;\; u \in \meaningof{a}, \forall z.P'\{z/y\} \in \meaningof{E\{z/b\}}\}, \and \\ \meaningof{a!E} = \{ P \in \pi | P \equiv Q | x!\langle P' \rangle, x \in \meaningof{a} P' \in \meaningof{E}\} }
\end{mathpar}

\begin{mathpar}
 \inferrule* [lab=nominal] {} {\meaningof{\quotep{E}} = \{ \quotep{P} \in \quotep{\pi} | P \in \meaningof{E} \}, \and \meaningof{\quotep{P}} = \{ \quotep{Q} \in \quotep{\pi} | P \equiv Q \} \and \\ \meaningof{@\quotep{E}} = \{ P \in \pi | P \equiv @x, x \in \meaningof{E} \}}
\end{mathpar}

\begin{eqnarray*}
  \\
  \meaningof{-} : TS \to ST
\end{eqnarray*}

\begin{eqnarray*}
  \\
  L : TS \to ST
\end{eqnarray*}

\begin{eqnarray*}
  \\
  P \models E \iff P \in \meaningof{E}
\end{eqnarray*}

\begin{eqnarray*}
  P \approx_{L} Q \iff \forall E \in L. P \models E \iff Q \models E
\end{eqnarray*}

\begin{eqnarray*}
  P \approx_{K} Q
\end{eqnarray*}

\begin{eqnarray*}
  P \approx Q
\end{eqnarray*}

$\approx_{K} = \approx = \approx_{L}$

\subsubsection{Contextual duality}

Note that contexts extend the quotation operation to a family of
operations from processes to names. Given a context, $M$, we can
define a \emph{nominal context}, $\quotep{M}$ by $\quotep{M}[P] :=
\quotep{M[P]}$. To foreshadow what is to come we observe that these
operations enjoy a duality with processes very much like the duality
between vectors and maps from vectors to scalars.

Further, because the calculus is essentially higher-order, we have a
correspondence between contexts and processes. More specifically,
given a name $x$ and a context $M$ we can construct $M^{*}_{x}$ such
that 

\begin{mathpar}
  M^{*}_{x} | \lift{x}{P} \red M[P]
\end{mathpar}

namely,

\begin{mathpar}
  M^{*}_{x} := x?(u).M[\dropn{u}]
\end{mathpar}

The dependence of $M^{*}_{x}$ on a name makes it an abstraction, 

\begin{mathpar}
  M^{*} := (x)x?(u).M[\dropn{u}]
\end{mathpar}

\subsection{Additional notation}

It will sometimes be convenient to denote the process a name
quotes. We already have the notation $x = \quotep{P}$, but it will be
convenient to introduce an alternate notation, $\procn{x}$, when we
want to emphasize the connection to the use of the name. Note that, by
virtue of name equivalence, $\quotep{\procn{x}} \nameeq x$; so, the
notation is consistent with previous definitions.

Further, because names have structure it is possible to effect
substitutions on the basis of that structure. This means we need to
upgrade our notation for substitutions, which we accomplish by
adapting comprehension notation. Thus,

\begin{mathpar}
  P\{ y / x : x \in S \}
\end{mathpar}

is interpreted to mean the process derived from P by replacing (in a
capture-avoiding manner) each occurrence of $x$ in $S$ by $y$. For example,

\begin{mathpar}
  P\{ \quotep{\procn{x}|\procn{x}} / x : x \in \freenames{P} \}
\end{mathpar}

will replace each (occurrence) of a free name $x$ in $P$ by
$\quotep{\procn{x}|\procn{x}}$.

Also, we will avail ourselves of the notation $x^{L}$ and $x^{R}$ to
denote injections of a name into disjoint copies of the name
space. There are numerous ways to accomplish this. One example can be
found in \cite{MeredithR05}. This notation overloads to vectors of
names: $\vec{x}^{\pi} := (x_{i}^{\pi} \; : \; 0 \leq i < |\vec{x}| )$ where $\pi \in \{L,R\}$.

We also use $P^{\Box} := P|\Box$.

In \cite{MeredithR05} an interpretation of the new operator is
given. It turns out that there are several possible interpretations
all enjoying the requisite algebraic properties of the operator (see
\cite{milner91polyadicpi}). We will therefore make liberal use of
$(\nu\; \vec{x})P$.

% subsection the_syntax_and_semantics_of_the_notation_system (end)   

\input{qm2pi.qmops} 

\input{qm2pi.sterngerlach} 

\input{qm2pi.metric} 

% section concurrent_process_calculi (end)

%\input{qm2pi.proofsketch}

% section proof sketch (end)

%\input{qm2pi.slviaknots} 

% section spatial logic via knots (end)

\input{qm2pi.conclusion}

% section conclusion (end)

%\input{qm2pi.dtcodes} 

% section wiring algorithm (end)

\input{qm2pi.ack} 

% section acknowledgments (end)

\newpage


\bibliographystyle{plain}   
\bibliography{../../biblios/main.bib}

\input{qm2pi.rhodetails}

\end{document}

 

% section acknowledgments (end)

\newpage


\bibliographystyle{plain}   
\bibliography{../../biblios/main.bib}

\documentclass[12pt]{llncs}
%\documentclass{jktr}

\usepackage[pdftex]{hyperref}                   
\usepackage {listings}
\usepackage {mathpartir}
\usepackage{bcprules}
%\usepackage{listings}
                       
\usepackage{graphicx} 
%\usepackage[margins=2.5cm,nohead,nofoot]{geometry}
%\usepackage{geometry}
\usepackage{amsfonts}
\usepackage{amstext}
\usepackage{latexsym}
\usepackage{amssymb}
\usepackage{color}


%\include{myPreamble}
\include{qm2pi.local} 

%\ifpdf
%\usepackage[pdftex]{graphicx}
%\else
%\usepackage{graphicx}
%\fi

 % \ifpdf
%  \usepackage{pdfsync}
%  \if


%\title{Brief Article}
%\author{David F. Snyder}
%\author{L.G. Meredith}

%\address{Dept. of Math., Texas State University--San Marcos, San Marcos, TX 78666}
       
\pagestyle{empty}


\begin{document}

\lstset{language=[Objective]Caml,frame=shadowbox}

\input{qm2pi.front}

% section front matter (end)

\input{qm2pi.intro} 
 
% section introduction (end)

% \input{qm2pi.knotations} 

% section notation (end)

\input{qm2pi.process.calculi} 

% section concurrent_process_calculi_and_spatial_logics_ (end)
    
%\input{qm2pi.knots2pi} 

%\input{qm2pi.trefoil} 

%\input{qm2pi.mainthm} 

% subsection basic_interpretation (end)

%\input{qm2pi.rho.presentation} 
\subsection{The syntax and semantics of the notation system}\label{sub:the_syntax_and_semantics_of_the_notation_system} % (fold)

We now summarize a technical presentation of the calculus that
embodies our theory of dynamics. The typical presentation of such a
calculus follows the style of giving generators and relations on
them. The grammar, below, describing term constructors, freely
generates the set of processes, $\Proc$. This set is then quotiented
by a relation known as structural congruence and it is over this set
that the notion of dynamics is expressed. This presentation is
essentially that of \cite{MeredithR05} with the addition of
polyadicity and summation. For readability we have relegated some of
the technical subtleties to an appendix.

\subsubsection{Process grammar}\label{subsub:process_grammar}

\begin{mathpar}
  \inferrule* [lab=synchronization] {} {{M} \bc \pzero \;|\; x?F \;|\; x!C }
  \and
  \inferrule* [lab=abstraction] {} {{F} \bc (x)P}
  \and
  \inferrule* [lab=concretion] {} {{C} \bc \langle Q \rangle}
  \and
  \inferrule* [lab=process] {} {{P,Q} \bc M \;| \;P|Q \;|\; @{x}}
  \and
  \inferrule* [lab=name] {} {{x} \bc \quotep{P}}
\end{mathpar} 

Note that $\vec{x}$ (resp. $\vec{P}$) denotes a vector of names
(resp. processes) of length $|\vec{x}|$ (resp. $|\vec{P}|$). We adopt
the following useful abbreviations.

\begin{mathpar}
   x?(\vec{y}).P := x.(\vec{y})P \and  x\clift{\vec{P}} := x.\clift{\vec{P}}
   \and x!(y) := \lift{x}{\dropn{y}}
   \and \Pi_{i=0}^{n-1}P_i := P_0 | \ldots | P_{n-1}
\end{mathpar}

\subsubsection{Structural congruence}

\paragraph{Free and bound names and alpha-equivalence.} At the
core of structural equivalence is alpha-equivalence which identifies
process that are the same up to a change of variable. Formally, we
recognize the distinction between free and bound names. The free names
of a process, $\freenames{P}$, may be calculated recursively as
follows:

\begin{mathpar}
\freenames{\pzero} := \emptyset
  \and \\
  \freenames{x?(y).P} := \{ x \} \cup (\freenames{P} \setminus \{ y \})
  \and 
  \freenames{x!\langle P \rangle} := \{ x \} \cup \{ P \} 
  \and \\
  \freenames{P|Q} := \freenames{P} \cup \freenames{Q}
  \and \\
  \freenames{@{x}} := \{ x \}
\end{mathpar}

$\pi$
$\quotep{\pi}$

$\freenames{-} : \pi \to \mathcal{P}(\quotep{\pi})$

\begin{eqnarray*}
  \freenames{\pzero} & := & \emptyset \\
  \freenames{x?(y).P} & := & \{ x \} \cup (\freenames{P} \setminus \{ y \}) \\
  \freenames{x!\langle P \rangle} & := & \{ x \} \cup \{ P \} \\
  \freenames{P|Q} & := & \freenames{P} \cup \freenames{Q} \\
  \freenames{\dropn{x}} & := & \{ x \}
\end{eqnarray*}

The bound names of a process, $\boundnames{P}$, are those names occurring in $P$
that are not free. For example, in $x?(y).0$, the name $x$ is free, while $y$ is bound.

\begin{mathpar}
  \inferrule* [lab=monoidal-laws] {} { P|Q \equiv Q|P \and P|0 \equiv P \and P|(Q|R) \equiv (P|Q)|R }
\end{mathpar}

\begin{mathpar}
  \inferrule* [lab=alpha-equivalence] {} { (x)P \equiv (y)P\{y/x\} \and y \not\in \freenames{P} }
\end{mathpar}

\begin{definition}
Then two processes, $P,Q$, are alpha-equivalent if $P = Q\{\vec{y}/\vec{x}\}$ for
some $\vec{x} \in \boundnames{Q},\vec{y} \in \boundnames{P}$, where $Q\{\vec{y}/\vec{x}\}$
denotes the capture-avoiding substitution of $\vec{y}$ for $\vec{x}$ in $Q$.
\end{definition}

\begin{definition}
  The {\em structural congruence} \cite{SangiorgiWalker} , $\equiv$,
  between processes is the least congruence containing
  alpha-equivalence, satisfying the abelian monoid laws
  (associativity, commutativity and $\pzero$ as identity) for parallel
  composition $|$ and for summation $+$.
\end{definition}

\subsection{Name equivalence}

We take name equivalence, written $\nameeq$, to be the smallest
equivalence relation generated by the following rules.

\begin{mathpar}
\inferrule*[lab=Quote-drop]
{ }
{ \quotep{@{x}} \nameeq x }

\inferrule*[lab=Struct-equiv]
{ P \scong Q }
{ \quotep{P} \nameeq \quotep{Q} }
\end{mathpar}

The astute reader will have noticed that the mutual recursion of names
and processes imposes a mutual recursion on alpha-equivalence and
structural equivalence via name-equivalence. Fortunately, all of this
works out pleasantly and we may calculate in the natural way, free of
concern. The reader interested in the details is referred to the
appendix \ref{appendix:rho_details}.

\subsection{Substitution}

We use $\Proc$ for the set of processes, $\QProc$ for the set of
names, and $\id{\{}\vec{y} / \vec{x} \id{\}}$ to denote partial maps,
$s : \QProc \rightarrow \QProc$. A map, $s$ lifts, uniquely, to a map
on process terms, $\widehat{s} : \Proc \rightarrow \Proc$ by the
following equations.

\begin{mathpar}
  (0) \psubstp{Q}{P} := 0 \\
  (R \juxtap S) \psubstp{Q}{P}
  :=    
  (R)\psubstp{Q}{P} \juxtap (S) \psubstp{Q}{P} \\
  (x?(y).R) \psubstp{Q}{P}    
  :=    
  (x)\substp{Q}{P} (z)\concat( (R \psubstn{z}{y}) \psubstp{Q}{P} ) \\
  (\lift{x}{R}) \psubstp{Q}{P}  
  :=
  \lift{(x)\substp{Q}{P}}{ R \psubstp{Q}{P} } \\
%   (\dropn{x})  \psubstp{Q}{P}       
%   := 
%   \left\{ 
%     \begin{array}{ccc} 
%       \dropn{\quotep{Q}} & & x \nameeq \quotep{P} \\
%       \dropn{x} & & otherwise \\
%     \end{array}
%   \right. 
  (\dropn{x})  \psubstp{Q}{P}       
  := 
  \left\{ 
    \begin{array}{ccc} 
      Q & & x \nameeq \quotep{P} \\
      \dropn{x} & & otherwise \\
    \end{array}
  \right.
\end{mathpar}
 

where

\begin{eqnarray}
  (x)\id{\{} \lpquote Q \rpquote / \lpquote P \rpquote \id{\}}            = 
  \left\{ 
    \begin{array}{ccc}
      \lpquote Q \rpquote & & x \nameeq \lpquote P \rpquote \\
      x & & otherwise \\
    \end{array}
  \right. \nonumber
\end{eqnarray}

and $z$ is chosen distinct from $\quotep{P}$, $\quotep{Q}$, the free
names in $Q$, and all the names in $R$. Our $\alpha$-equivalence will
be built in the standard way from this substitution.

\begin{remark}\label{rem:no_self_referential_names}
  One consequence of these definitions is that $\forall P. \quotep{P}
  \not\in \freenames{P}$.
\end{remark}

\subsection{ Dynamic quote: an example }

Anticipating something of what's to come, consider applying the
substitution, $\widehat{\id{\{}u / z \id{\}}}$, to the following pair
of processes, $\lift{w}{y!(z)}$ and $w[ \lpquote y!(z) \rpquote ]$.

\begin{eqnarray}
	\lift{w}{y!(z)}\widehat{\id{\{}u / z \id{\}}}
		& = &
		\lift{w}{y!(u)} \nonumber\\
	w[ \lpquote y!(z) \rpquote ] \widehat{ \id{\{}u / z \id{\}} }
		& = &
		w[ \lpquote y!(z) \rpquote ] \nonumber
\end{eqnarray}

Because the body of the process between quotes is impervious to
substitution, we get radically different answers. In fact, by
examining the first process in an input context,
e.g. $x?(z).\lift{w}{y!(z)}$, we see that the process under the lift
operator may be shaped by prefixed inputs binding a name inside it. In
this sense, the lift operator will be seen as a way to dynamically
construct processes before reifying them as names.

Finally equipped with these standard features we can present the
dynamics of the calculus.

\subsubsection{Operational semantics} 

Finally, we introduce the computational dynamics. What marks these
algebras as distinct from other more traditionally studied algebraic
structures, e.g. vector spaces or polynomial rings, is the manner in
which dynamics is captured. In traditional structures, dynamics is typically
expressed through morphisms between such structures, as in linear maps
between vector spaces or morphisms between rings. In algebras
associated with the semantics of computation, the dynamics is
expressed as part of the algebraic structure itself, through a
reduction reduction relation typically denoted by $\red$. Below, we
give a recursive presentation of this relation for the calculus used
in the encoding.

$\red \subseteq \pi \times \pi$
$\red : \pi \to \mathcal{P}(\pi)$

\begin{mathpar}
  \inferrule* [lab=Comm] { \textsf{match}( x_{src}, x_{trgt} ) } { x_{trgt}?(y)P \; | \; x_{src}!\langle {Q} \rangle \red P\{\quotep{Q}/y}\} }
  \and \\
  \inferrule* [lab=Par] {{P} \red {P}'} {{{P} | {Q}} \red {{P}' | {Q}}}
  \and
  \inferrule* [lab=Equiv]{{{P} \scong {P}'} \andalso {{P}' \red {Q}'} \andalso {{Q}' \scong {Q}}}{{P} \red {Q}}
\end{mathpar}

\begin{eqnarray*}
  match_{\equiv} (\quotep{P},\quotep{Q}) & := & P \equiv Q \\
  match_{\dagger}(\quotep{P},\quotep{Q}) & := & \forall R. P|Q \red^{*} R => R \red^{*} 0 \\
  match_{K}(\quotep{P},\quotep{Q}) & := & K \mbox{ for some context } K
\end{eqnarray*}

$u?(x)P | u!\langle Q \rangle \red P\{\quotep{Q}/x\}$

%We write $\wred$ for $\red^*$, and $P\red$ if $\exists Q $ such that $ P \red Q$.
We write $P\red$ if $\exists Q $ such that $ P \red Q$ and $P\not\red$, otherwise.

\section{Replication}

As mentioned before, it is known that replication (and hence
recursion) can be implemented in a higher-order process algebra
\cite{SangiorgiWalker}. As our first example of calculation with the
machinery thus far presented we give the construction explicitly in
the {\rhoc}.

\begin{eqnarray}
	D_{x} & := & \prefix{x}{y}{(\binpar{\outputp{x}{y}}{@{y}})} \nonumber\\
	\bangp_{x}{P} & := & \binpar{{x}!\langle{\binpar{D_{x}}{P}}\rangle}{D_{x}} \nonumber
\end{eqnarray}

\begin{eqnarray}
	\bangp_{x}{P} & & \nonumber\\
	=
	& {x}!\langle{(\prefix{x}{y}{(\outputp{x}{y} | @{y})) | P}}\rangle 
	      | \prefix{x}{y}{(\outputp{x}{y} | @{y})} & \nonumber\\
	\red
	& (\outputp{x}{y} | @{y})\substn{\quotep{(\prefix{x}{y}{(@{y} | \outputp{x}{y})) | P}}}{y} & \nonumber\\
	=
	& \outputp{x}{\quotep{(\prefix{x}{y}{(\outputp{x}{y} | @{y})) | P}}}
	  | {(\prefix{x}{y}{(\outputp{x}{y} | @{y})) | P}} & \nonumber\\
	\red
	& \ldots & \nonumber\\
	\red^*
	& P | P | \ldots & \nonumber
\end{eqnarray}

Of course, this encoding, as an implementation, runs away, unfolding
$\bangp{P}$ eagerly. A lazier and more implementable replication
operator, restricted to input-guarded processes, may be obtained as follows.

\begin{eqnarray}
\bangp{\prefix{u}{v}{P}} 
	:= 
	\binpar{\lift{x}{\prefix{u}{v}{(\binpar{D(x)}{P})}}}{D(x)} \nonumber
\end{eqnarray}

\begin{remark}
  Note that the lazier definition still does not deal with summation
  or mixed summation (i.e. sums over input and output). The reader is
  invited to construct definitions of replication that deal with these
  features. 

  Further, the definitions are parameterized in a name, $x$. Can you,
  gentle reader, make a definition that eliminates this parameter and
  guarantees no accidental interaction between the replication
  machinery and the process being replicated -- i.e. no accidental
  sharing of names used by the process to get its work done and the
  name(s) used by the replication to effect copying. This latter
  revision of the definition of replication is crucial to obtaining
  the expected identity $!!P \sim !P$.
\end{remark}

\begin{remark}\label{rem:paradoxical_combinator}
  The reader familiar with the lambda calculus will have noticed the
  similarity between $D$ and the paradoxical combinator.

  [Ed. note: the existence of this seems to suggest we have to be more
  restrictive on the set of processes and names we admit if we are to
  support no-cloning.]
\end{remark}

\subsubsection{Bisimulation}

The computational dynamics gives rise to another kind of equivalence,
the equivalence of computational behavior. As previously mentioned
this is typically captured \emph{via} some form of bisimulation.

% The notion we use in this paper is weak barbed bisimulation
% \cite{milner91polyadicpi}.

The notion we use in this paper is derived from weak barbed
bisimulation \cite{milner91polyadicpi}. 

\begin{definition}
An \emph{observation relation}, $\downarrow_{\mathcal N}$, over a set
of names, $\mathcal N$, is the smallest relation satisfying the rules
below.

\infrule[Out-barb]{y \in {\mathcal N}, \; x \nameeq y}
		  {\outputp{x}{v} \downarrow_{\mathcal N} x}
\infrule[Par-barb]{\mbox{$P\downarrow_{\mathcal N} x$ or $Q\downarrow_{\mathcal N} x$}}
		  {\binpar{P}{Q} \downarrow_{\mathcal N} x}

We write $P \Downarrow_{\mathcal N} x$ if there is $Q$ such that 
$P \wred Q$ and $Q \downarrow_{\mathcal N} x$.
\end{definition}

\begin{definition}
%\label{def.bbisim}
An  ${\mathcal N}$-\emph{barbed bisimulation} over a set of names, ${\mathcal N}$, is a symmetric binary relation 
${\mathcal S}_{\mathcal N}$ between agents such that $P\rel{S}_{\mathcal N}Q$ implies:
\begin{enumerate}
\item If $P \red P'$ then $Q \wred Q'$ and $P'\rel{S}_{\mathcal N} Q'$.
\item If $P\downarrow_{\mathcal N} x$, then $Q\Downarrow_{\mathcal N} x$.
\end{enumerate}
$P$ is ${\mathcal N}$-barbed bisimilar to $Q$, written
$P \wbbisim_{\mathcal N} Q$, if $P \rel{S}_{\mathcal N} Q$ for some ${\mathcal N}$-barbed bisimulation ${\mathcal S}_{\mathcal N}$.
\end{definition}

$\mathcal{R} \subseteq \pi \times \pi$

$P \mathcal{R} Q => \forall P'. P \red P' \Rightarrow \exists Q'. Q \red Q', P' \mathcal{R} Q'$

$P \vdash x \Rightarrow Q \vdash x$

\begin{mathpar}
  \inferrule*[lab=Out-barb]{x \nameeq y}{{y}!\langle{Q}\rangle \vdash x}
  \and
  \inferrule*[lab=Par-barb]{\mbox{$P\vdash x$ or $Q\vdash x$}}{\binpar{P}{Q} \vdash x}
\end{mathpar}

\subsubsection{Contexts}

One of the principle advantages of computational calculi like the
$\pi$-calculus is a well-defined notion of context,
contextual-equivalence and a correlation between
contextual-equivalence and notions of bisimulation. The notion of
context allows the decomposition of a process into (sub-)process and
its syntactic environment, its context. Thus, a context may be
thought of as a process with a ``hole'' (written $\Box$) in it. The
application of a context $M$ to a process $P$, written $M[P]$, is
tantamount to filling the hole in $M$ with $P$. In this paper we do
not need the full weight of this theory, but do make use of the notion
of context in the proof the main theorem. 

\begin{mathpar}
  \inferrule* [lab=summation] {} {{M_{M},M_{N}} \bc \Box \;|\; x.M_{A} \;|\; M_{M}+M_{N}}
  \and
  \inferrule* [lab=agent] {} {{M_{A}} \bc (\vec{x})M_{P} \;| \; \clift{P_0,\ldots,M_{P},\ldots,P_N}}
  \and \\
  \inferrule* [lab=process] {} {{M_{P}} \bc M_{N} \;| \;P|M_{P} }
\end{mathpar} 

\begin{mathpar}
  \inferrule* [lab=sychronization] {} {M_{N} \bc \Box \;|\; x?M_{F} \;|\; x!M_{C}}
  \and
  \inferrule* [lab=abstraction] {} {{M_{F}} \bc (x)M_{P} }
  \and
  \inferrule* [lab=concretion] {} {{M_{C}} \bc \langle M_{P} \rangle }
  \and \\
  \inferrule* [lab=process] {} {{M_{P}} \bc M_{N} \;| \;P|M_{P} }
\end{mathpar}

\begin{definition}[contextual application] Given a context $M$, and
  process $P$, we define the \emph{contextual application}, $M[P] :=
  M\{P/\Box\}$. That is, the contextual application of M to P is the
  substitution of $P$ for $\Box$ in $M$.
\end{definition}

$\meaningof{-} : L \to \mathcal{P}(\pi)$

\begin{mathpar}
  \inferrule* [lab=collection] {} {\meaningof{true} = \pi, \and \meaningof{~E} = \pi \setminus \meaningof{E}, \and \meaningof{E_{1} \& E_{2}} = \meaningof{E_{1}} \cap \meaningof{E_{2}}}
\end{mathpar}

\begin{mathpar}
  \inferrule* [lab=structure] {} {\meaningof{0} = \{ P \in \pi | P \equiv 0 \}, \and \\ \meaningof{E_1 | E_2} = \{ P \in \pi | P \equiv P_{1} | P_{2}, P_{1} \in \meaningof{E_{1}}, P_{2} \in \meaningof{E_2}\} }
\end{mathpar}

\begin{mathpar}
 \inferrule* [lab=behavior] {} {\meaningof{\langle a?b \rangle E} = \{ P \in \pi | P \equiv Q | u?(y)P', \\ \and \\\\ \and \\ \;\;\; u \in \meaningof{a}, \forall z.P'\{z/y\} \in \meaningof{E\{z/b\}}\}, \and \\ \meaningof{a!E} = \{ P \in \pi | P \equiv Q | x!\langle P' \rangle, x \in \meaningof{a} P' \in \meaningof{E}\} }
\end{mathpar}

\begin{mathpar}
 \inferrule* [lab=nominal] {} {\meaningof{\quotep{E}} = \{ \quotep{P} \in \quotep{\pi} | P \in \meaningof{E} \}, \and \meaningof{\quotep{P}} = \{ \quotep{Q} \in \quotep{\pi} | P \equiv Q \} \and \\ \meaningof{@\quotep{E}} = \{ P \in \pi | P \equiv @x, x \in \meaningof{E} \}}
\end{mathpar}

\begin{eqnarray*}
  \\
  \meaningof{-} : TS \to ST
\end{eqnarray*}

\begin{eqnarray*}
  \\
  L : TS \to ST
\end{eqnarray*}

\begin{eqnarray*}
  \\
  P \models E \iff P \in \meaningof{E}
\end{eqnarray*}

\begin{eqnarray*}
  P \approx_{L} Q \iff \forall E \in L. P \models E \iff Q \models E
\end{eqnarray*}

\begin{eqnarray*}
  P \approx_{K} Q
\end{eqnarray*}

\begin{eqnarray*}
  P \approx Q
\end{eqnarray*}

$\approx_{K} = \approx = \approx_{L}$

\subsubsection{Contextual duality}

Note that contexts extend the quotation operation to a family of
operations from processes to names. Given a context, $M$, we can
define a \emph{nominal context}, $\quotep{M}$ by $\quotep{M}[P] :=
\quotep{M[P]}$. To foreshadow what is to come we observe that these
operations enjoy a duality with processes very much like the duality
between vectors and maps from vectors to scalars.

Further, because the calculus is essentially higher-order, we have a
correspondence between contexts and processes. More specifically,
given a name $x$ and a context $M$ we can construct $M^{*}_{x}$ such
that 

\begin{mathpar}
  M^{*}_{x} | \lift{x}{P} \red M[P]
\end{mathpar}

namely,

\begin{mathpar}
  M^{*}_{x} := x?(u).M[\dropn{u}]
\end{mathpar}

The dependence of $M^{*}_{x}$ on a name makes it an abstraction, 

\begin{mathpar}
  M^{*} := (x)x?(u).M[\dropn{u}]
\end{mathpar}

\subsection{Additional notation}

It will sometimes be convenient to denote the process a name
quotes. We already have the notation $x = \quotep{P}$, but it will be
convenient to introduce an alternate notation, $\procn{x}$, when we
want to emphasize the connection to the use of the name. Note that, by
virtue of name equivalence, $\quotep{\procn{x}} \nameeq x$; so, the
notation is consistent with previous definitions.

Further, because names have structure it is possible to effect
substitutions on the basis of that structure. This means we need to
upgrade our notation for substitutions, which we accomplish by
adapting comprehension notation. Thus,

\begin{mathpar}
  P\{ y / x : x \in S \}
\end{mathpar}

is interpreted to mean the process derived from P by replacing (in a
capture-avoiding manner) each occurrence of $x$ in $S$ by $y$. For example,

\begin{mathpar}
  P\{ \quotep{\procn{x}|\procn{x}} / x : x \in \freenames{P} \}
\end{mathpar}

will replace each (occurrence) of a free name $x$ in $P$ by
$\quotep{\procn{x}|\procn{x}}$.

Also, we will avail ourselves of the notation $x^{L}$ and $x^{R}$ to
denote injections of a name into disjoint copies of the name
space. There are numerous ways to accomplish this. One example can be
found in \cite{MeredithR05}. This notation overloads to vectors of
names: $\vec{x}^{\pi} := (x_{i}^{\pi} \; : \; 0 \leq i < |\vec{x}| )$ where $\pi \in \{L,R\}$.

We also use $P^{\Box} := P|\Box$.

In \cite{MeredithR05} an interpretation of the new operator is
given. It turns out that there are several possible interpretations
all enjoying the requisite algebraic properties of the operator (see
\cite{milner91polyadicpi}). We will therefore make liberal use of
$(\nu\; \vec{x})P$.

% subsection the_syntax_and_semantics_of_the_notation_system (end)   

\input{qm2pi.qmops} 

\input{qm2pi.sterngerlach} 

\input{qm2pi.metric} 

% section concurrent_process_calculi (end)

%\input{qm2pi.proofsketch}

% section proof sketch (end)

%\input{qm2pi.slviaknots} 

% section spatial logic via knots (end)

\input{qm2pi.conclusion}

% section conclusion (end)

%\input{qm2pi.dtcodes} 

% section wiring algorithm (end)

\input{qm2pi.ack} 

% section acknowledgments (end)

\newpage


\bibliographystyle{plain}   
\bibliography{../../biblios/main.bib}

\input{qm2pi.rhodetails}

\end{document}



\end{document}

 

% section notation (end)

\input{qm2pi.process.calculi} 

% section concurrent_process_calculi_and_spatial_logics_ (end)
    
%\documentclass[12pt]{llncs}
%\documentclass{jktr}

\usepackage[pdftex]{hyperref}                   
\usepackage {listings}
\usepackage {mathpartir}
\usepackage{bcprules}
%\usepackage{listings}
                       
\usepackage{graphicx} 
%\usepackage[margins=2.5cm,nohead,nofoot]{geometry}
%\usepackage{geometry}
\usepackage{amsfonts}
\usepackage{amstext}
\usepackage{latexsym}
\usepackage{amssymb}
\usepackage{color}


%\include{myPreamble}
\documentclass[12pt]{llncs}
%\documentclass{jktr}

\usepackage[pdftex]{hyperref}                   
\usepackage {listings}
\usepackage {mathpartir}
\usepackage{bcprules}
%\usepackage{listings}
                       
\usepackage{graphicx} 
%\usepackage[margins=2.5cm,nohead,nofoot]{geometry}
%\usepackage{geometry}
\usepackage{amsfonts}
\usepackage{amstext}
\usepackage{latexsym}
\usepackage{amssymb}
\usepackage{color}


%\include{myPreamble}
\include{qm2pi.local} 

%\ifpdf
%\usepackage[pdftex]{graphicx}
%\else
%\usepackage{graphicx}
%\fi

 % \ifpdf
%  \usepackage{pdfsync}
%  \if


%\title{Brief Article}
%\author{David F. Snyder}
%\author{L.G. Meredith}

%\address{Dept. of Math., Texas State University--San Marcos, San Marcos, TX 78666}
       
\pagestyle{empty}


\begin{document}

\lstset{language=[Objective]Caml,frame=shadowbox}

\input{qm2pi.front}

% section front matter (end)

\input{qm2pi.intro} 
 
% section introduction (end)

% \input{qm2pi.knotations} 

% section notation (end)

\input{qm2pi.process.calculi} 

% section concurrent_process_calculi_and_spatial_logics_ (end)
    
%\input{qm2pi.knots2pi} 

%\input{qm2pi.trefoil} 

%\input{qm2pi.mainthm} 

% subsection basic_interpretation (end)

%\input{qm2pi.rho.presentation} 
\subsection{The syntax and semantics of the notation system}\label{sub:the_syntax_and_semantics_of_the_notation_system} % (fold)

We now summarize a technical presentation of the calculus that
embodies our theory of dynamics. The typical presentation of such a
calculus follows the style of giving generators and relations on
them. The grammar, below, describing term constructors, freely
generates the set of processes, $\Proc$. This set is then quotiented
by a relation known as structural congruence and it is over this set
that the notion of dynamics is expressed. This presentation is
essentially that of \cite{MeredithR05} with the addition of
polyadicity and summation. For readability we have relegated some of
the technical subtleties to an appendix.

\subsubsection{Process grammar}\label{subsub:process_grammar}

\begin{mathpar}
  \inferrule* [lab=synchronization] {} {{M} \bc \pzero \;|\; x?F \;|\; x!C }
  \and
  \inferrule* [lab=abstraction] {} {{F} \bc (x)P}
  \and
  \inferrule* [lab=concretion] {} {{C} \bc \langle Q \rangle}
  \and
  \inferrule* [lab=process] {} {{P,Q} \bc M \;| \;P|Q \;|\; @{x}}
  \and
  \inferrule* [lab=name] {} {{x} \bc \quotep{P}}
\end{mathpar} 

Note that $\vec{x}$ (resp. $\vec{P}$) denotes a vector of names
(resp. processes) of length $|\vec{x}|$ (resp. $|\vec{P}|$). We adopt
the following useful abbreviations.

\begin{mathpar}
   x?(\vec{y}).P := x.(\vec{y})P \and  x\clift{\vec{P}} := x.\clift{\vec{P}}
   \and x!(y) := \lift{x}{\dropn{y}}
   \and \Pi_{i=0}^{n-1}P_i := P_0 | \ldots | P_{n-1}
\end{mathpar}

\subsubsection{Structural congruence}

\paragraph{Free and bound names and alpha-equivalence.} At the
core of structural equivalence is alpha-equivalence which identifies
process that are the same up to a change of variable. Formally, we
recognize the distinction between free and bound names. The free names
of a process, $\freenames{P}$, may be calculated recursively as
follows:

\begin{mathpar}
\freenames{\pzero} := \emptyset
  \and \\
  \freenames{x?(y).P} := \{ x \} \cup (\freenames{P} \setminus \{ y \})
  \and 
  \freenames{x!\langle P \rangle} := \{ x \} \cup \{ P \} 
  \and \\
  \freenames{P|Q} := \freenames{P} \cup \freenames{Q}
  \and \\
  \freenames{@{x}} := \{ x \}
\end{mathpar}

$\pi$
$\quotep{\pi}$

$\freenames{-} : \pi \to \mathcal{P}(\quotep{\pi})$

\begin{eqnarray*}
  \freenames{\pzero} & := & \emptyset \\
  \freenames{x?(y).P} & := & \{ x \} \cup (\freenames{P} \setminus \{ y \}) \\
  \freenames{x!\langle P \rangle} & := & \{ x \} \cup \{ P \} \\
  \freenames{P|Q} & := & \freenames{P} \cup \freenames{Q} \\
  \freenames{\dropn{x}} & := & \{ x \}
\end{eqnarray*}

The bound names of a process, $\boundnames{P}$, are those names occurring in $P$
that are not free. For example, in $x?(y).0$, the name $x$ is free, while $y$ is bound.

\begin{mathpar}
  \inferrule* [lab=monoidal-laws] {} { P|Q \equiv Q|P \and P|0 \equiv P \and P|(Q|R) \equiv (P|Q)|R }
\end{mathpar}

\begin{mathpar}
  \inferrule* [lab=alpha-equivalence] {} { (x)P \equiv (y)P\{y/x\} \and y \not\in \freenames{P} }
\end{mathpar}

\begin{definition}
Then two processes, $P,Q$, are alpha-equivalent if $P = Q\{\vec{y}/\vec{x}\}$ for
some $\vec{x} \in \boundnames{Q},\vec{y} \in \boundnames{P}$, where $Q\{\vec{y}/\vec{x}\}$
denotes the capture-avoiding substitution of $\vec{y}$ for $\vec{x}$ in $Q$.
\end{definition}

\begin{definition}
  The {\em structural congruence} \cite{SangiorgiWalker} , $\equiv$,
  between processes is the least congruence containing
  alpha-equivalence, satisfying the abelian monoid laws
  (associativity, commutativity and $\pzero$ as identity) for parallel
  composition $|$ and for summation $+$.
\end{definition}

\subsection{Name equivalence}

We take name equivalence, written $\nameeq$, to be the smallest
equivalence relation generated by the following rules.

\begin{mathpar}
\inferrule*[lab=Quote-drop]
{ }
{ \quotep{@{x}} \nameeq x }

\inferrule*[lab=Struct-equiv]
{ P \scong Q }
{ \quotep{P} \nameeq \quotep{Q} }
\end{mathpar}

The astute reader will have noticed that the mutual recursion of names
and processes imposes a mutual recursion on alpha-equivalence and
structural equivalence via name-equivalence. Fortunately, all of this
works out pleasantly and we may calculate in the natural way, free of
concern. The reader interested in the details is referred to the
appendix \ref{appendix:rho_details}.

\subsection{Substitution}

We use $\Proc$ for the set of processes, $\QProc$ for the set of
names, and $\id{\{}\vec{y} / \vec{x} \id{\}}$ to denote partial maps,
$s : \QProc \rightarrow \QProc$. A map, $s$ lifts, uniquely, to a map
on process terms, $\widehat{s} : \Proc \rightarrow \Proc$ by the
following equations.

\begin{mathpar}
  (0) \psubstp{Q}{P} := 0 \\
  (R \juxtap S) \psubstp{Q}{P}
  :=    
  (R)\psubstp{Q}{P} \juxtap (S) \psubstp{Q}{P} \\
  (x?(y).R) \psubstp{Q}{P}    
  :=    
  (x)\substp{Q}{P} (z)\concat( (R \psubstn{z}{y}) \psubstp{Q}{P} ) \\
  (\lift{x}{R}) \psubstp{Q}{P}  
  :=
  \lift{(x)\substp{Q}{P}}{ R \psubstp{Q}{P} } \\
%   (\dropn{x})  \psubstp{Q}{P}       
%   := 
%   \left\{ 
%     \begin{array}{ccc} 
%       \dropn{\quotep{Q}} & & x \nameeq \quotep{P} \\
%       \dropn{x} & & otherwise \\
%     \end{array}
%   \right. 
  (\dropn{x})  \psubstp{Q}{P}       
  := 
  \left\{ 
    \begin{array}{ccc} 
      Q & & x \nameeq \quotep{P} \\
      \dropn{x} & & otherwise \\
    \end{array}
  \right.
\end{mathpar}
 

where

\begin{eqnarray}
  (x)\id{\{} \lpquote Q \rpquote / \lpquote P \rpquote \id{\}}            = 
  \left\{ 
    \begin{array}{ccc}
      \lpquote Q \rpquote & & x \nameeq \lpquote P \rpquote \\
      x & & otherwise \\
    \end{array}
  \right. \nonumber
\end{eqnarray}

and $z$ is chosen distinct from $\quotep{P}$, $\quotep{Q}$, the free
names in $Q$, and all the names in $R$. Our $\alpha$-equivalence will
be built in the standard way from this substitution.

\begin{remark}\label{rem:no_self_referential_names}
  One consequence of these definitions is that $\forall P. \quotep{P}
  \not\in \freenames{P}$.
\end{remark}

\subsection{ Dynamic quote: an example }

Anticipating something of what's to come, consider applying the
substitution, $\widehat{\id{\{}u / z \id{\}}}$, to the following pair
of processes, $\lift{w}{y!(z)}$ and $w[ \lpquote y!(z) \rpquote ]$.

\begin{eqnarray}
	\lift{w}{y!(z)}\widehat{\id{\{}u / z \id{\}}}
		& = &
		\lift{w}{y!(u)} \nonumber\\
	w[ \lpquote y!(z) \rpquote ] \widehat{ \id{\{}u / z \id{\}} }
		& = &
		w[ \lpquote y!(z) \rpquote ] \nonumber
\end{eqnarray}

Because the body of the process between quotes is impervious to
substitution, we get radically different answers. In fact, by
examining the first process in an input context,
e.g. $x?(z).\lift{w}{y!(z)}$, we see that the process under the lift
operator may be shaped by prefixed inputs binding a name inside it. In
this sense, the lift operator will be seen as a way to dynamically
construct processes before reifying them as names.

Finally equipped with these standard features we can present the
dynamics of the calculus.

\subsubsection{Operational semantics} 

Finally, we introduce the computational dynamics. What marks these
algebras as distinct from other more traditionally studied algebraic
structures, e.g. vector spaces or polynomial rings, is the manner in
which dynamics is captured. In traditional structures, dynamics is typically
expressed through morphisms between such structures, as in linear maps
between vector spaces or morphisms between rings. In algebras
associated with the semantics of computation, the dynamics is
expressed as part of the algebraic structure itself, through a
reduction reduction relation typically denoted by $\red$. Below, we
give a recursive presentation of this relation for the calculus used
in the encoding.

$\red \subseteq \pi \times \pi$
$\red : \pi \to \mathcal{P}(\pi)$

\begin{mathpar}
  \inferrule* [lab=Comm] { \textsf{match}( x_{src}, x_{trgt} ) } { x_{trgt}?(y)P \; | \; x_{src}!\langle {Q} \rangle \red P\{\quotep{Q}/y}\} }
  \and \\
  \inferrule* [lab=Par] {{P} \red {P}'} {{{P} | {Q}} \red {{P}' | {Q}}}
  \and
  \inferrule* [lab=Equiv]{{{P} \scong {P}'} \andalso {{P}' \red {Q}'} \andalso {{Q}' \scong {Q}}}{{P} \red {Q}}
\end{mathpar}

\begin{eqnarray*}
  match_{\equiv} (\quotep{P},\quotep{Q}) & := & P \equiv Q \\
  match_{\dagger}(\quotep{P},\quotep{Q}) & := & \forall R. P|Q \red^{*} R => R \red^{*} 0 \\
  match_{K}(\quotep{P},\quotep{Q}) & := & K \mbox{ for some context } K
\end{eqnarray*}

$u?(x)P | u!\langle Q \rangle \red P\{\quotep{Q}/x\}$

%We write $\wred$ for $\red^*$, and $P\red$ if $\exists Q $ such that $ P \red Q$.
We write $P\red$ if $\exists Q $ such that $ P \red Q$ and $P\not\red$, otherwise.

\section{Replication}

As mentioned before, it is known that replication (and hence
recursion) can be implemented in a higher-order process algebra
\cite{SangiorgiWalker}. As our first example of calculation with the
machinery thus far presented we give the construction explicitly in
the {\rhoc}.

\begin{eqnarray}
	D_{x} & := & \prefix{x}{y}{(\binpar{\outputp{x}{y}}{@{y}})} \nonumber\\
	\bangp_{x}{P} & := & \binpar{{x}!\langle{\binpar{D_{x}}{P}}\rangle}{D_{x}} \nonumber
\end{eqnarray}

\begin{eqnarray}
	\bangp_{x}{P} & & \nonumber\\
	=
	& {x}!\langle{(\prefix{x}{y}{(\outputp{x}{y} | @{y})) | P}}\rangle 
	      | \prefix{x}{y}{(\outputp{x}{y} | @{y})} & \nonumber\\
	\red
	& (\outputp{x}{y} | @{y})\substn{\quotep{(\prefix{x}{y}{(@{y} | \outputp{x}{y})) | P}}}{y} & \nonumber\\
	=
	& \outputp{x}{\quotep{(\prefix{x}{y}{(\outputp{x}{y} | @{y})) | P}}}
	  | {(\prefix{x}{y}{(\outputp{x}{y} | @{y})) | P}} & \nonumber\\
	\red
	& \ldots & \nonumber\\
	\red^*
	& P | P | \ldots & \nonumber
\end{eqnarray}

Of course, this encoding, as an implementation, runs away, unfolding
$\bangp{P}$ eagerly. A lazier and more implementable replication
operator, restricted to input-guarded processes, may be obtained as follows.

\begin{eqnarray}
\bangp{\prefix{u}{v}{P}} 
	:= 
	\binpar{\lift{x}{\prefix{u}{v}{(\binpar{D(x)}{P})}}}{D(x)} \nonumber
\end{eqnarray}

\begin{remark}
  Note that the lazier definition still does not deal with summation
  or mixed summation (i.e. sums over input and output). The reader is
  invited to construct definitions of replication that deal with these
  features. 

  Further, the definitions are parameterized in a name, $x$. Can you,
  gentle reader, make a definition that eliminates this parameter and
  guarantees no accidental interaction between the replication
  machinery and the process being replicated -- i.e. no accidental
  sharing of names used by the process to get its work done and the
  name(s) used by the replication to effect copying. This latter
  revision of the definition of replication is crucial to obtaining
  the expected identity $!!P \sim !P$.
\end{remark}

\begin{remark}\label{rem:paradoxical_combinator}
  The reader familiar with the lambda calculus will have noticed the
  similarity between $D$ and the paradoxical combinator.

  [Ed. note: the existence of this seems to suggest we have to be more
  restrictive on the set of processes and names we admit if we are to
  support no-cloning.]
\end{remark}

\subsubsection{Bisimulation}

The computational dynamics gives rise to another kind of equivalence,
the equivalence of computational behavior. As previously mentioned
this is typically captured \emph{via} some form of bisimulation.

% The notion we use in this paper is weak barbed bisimulation
% \cite{milner91polyadicpi}.

The notion we use in this paper is derived from weak barbed
bisimulation \cite{milner91polyadicpi}. 

\begin{definition}
An \emph{observation relation}, $\downarrow_{\mathcal N}$, over a set
of names, $\mathcal N$, is the smallest relation satisfying the rules
below.

\infrule[Out-barb]{y \in {\mathcal N}, \; x \nameeq y}
		  {\outputp{x}{v} \downarrow_{\mathcal N} x}
\infrule[Par-barb]{\mbox{$P\downarrow_{\mathcal N} x$ or $Q\downarrow_{\mathcal N} x$}}
		  {\binpar{P}{Q} \downarrow_{\mathcal N} x}

We write $P \Downarrow_{\mathcal N} x$ if there is $Q$ such that 
$P \wred Q$ and $Q \downarrow_{\mathcal N} x$.
\end{definition}

\begin{definition}
%\label{def.bbisim}
An  ${\mathcal N}$-\emph{barbed bisimulation} over a set of names, ${\mathcal N}$, is a symmetric binary relation 
${\mathcal S}_{\mathcal N}$ between agents such that $P\rel{S}_{\mathcal N}Q$ implies:
\begin{enumerate}
\item If $P \red P'$ then $Q \wred Q'$ and $P'\rel{S}_{\mathcal N} Q'$.
\item If $P\downarrow_{\mathcal N} x$, then $Q\Downarrow_{\mathcal N} x$.
\end{enumerate}
$P$ is ${\mathcal N}$-barbed bisimilar to $Q$, written
$P \wbbisim_{\mathcal N} Q$, if $P \rel{S}_{\mathcal N} Q$ for some ${\mathcal N}$-barbed bisimulation ${\mathcal S}_{\mathcal N}$.
\end{definition}

$\mathcal{R} \subseteq \pi \times \pi$

$P \mathcal{R} Q => \forall P'. P \red P' \Rightarrow \exists Q'. Q \red Q', P' \mathcal{R} Q'$

$P \vdash x \Rightarrow Q \vdash x$

\begin{mathpar}
  \inferrule*[lab=Out-barb]{x \nameeq y}{{y}!\langle{Q}\rangle \vdash x}
  \and
  \inferrule*[lab=Par-barb]{\mbox{$P\vdash x$ or $Q\vdash x$}}{\binpar{P}{Q} \vdash x}
\end{mathpar}

\subsubsection{Contexts}

One of the principle advantages of computational calculi like the
$\pi$-calculus is a well-defined notion of context,
contextual-equivalence and a correlation between
contextual-equivalence and notions of bisimulation. The notion of
context allows the decomposition of a process into (sub-)process and
its syntactic environment, its context. Thus, a context may be
thought of as a process with a ``hole'' (written $\Box$) in it. The
application of a context $M$ to a process $P$, written $M[P]$, is
tantamount to filling the hole in $M$ with $P$. In this paper we do
not need the full weight of this theory, but do make use of the notion
of context in the proof the main theorem. 

\begin{mathpar}
  \inferrule* [lab=summation] {} {{M_{M},M_{N}} \bc \Box \;|\; x.M_{A} \;|\; M_{M}+M_{N}}
  \and
  \inferrule* [lab=agent] {} {{M_{A}} \bc (\vec{x})M_{P} \;| \; \clift{P_0,\ldots,M_{P},\ldots,P_N}}
  \and \\
  \inferrule* [lab=process] {} {{M_{P}} \bc M_{N} \;| \;P|M_{P} }
\end{mathpar} 

\begin{mathpar}
  \inferrule* [lab=sychronization] {} {M_{N} \bc \Box \;|\; x?M_{F} \;|\; x!M_{C}}
  \and
  \inferrule* [lab=abstraction] {} {{M_{F}} \bc (x)M_{P} }
  \and
  \inferrule* [lab=concretion] {} {{M_{C}} \bc \langle M_{P} \rangle }
  \and \\
  \inferrule* [lab=process] {} {{M_{P}} \bc M_{N} \;| \;P|M_{P} }
\end{mathpar}

\begin{definition}[contextual application] Given a context $M$, and
  process $P$, we define the \emph{contextual application}, $M[P] :=
  M\{P/\Box\}$. That is, the contextual application of M to P is the
  substitution of $P$ for $\Box$ in $M$.
\end{definition}

$\meaningof{-} : L \to \mathcal{P}(\pi)$

\begin{mathpar}
  \inferrule* [lab=collection] {} {\meaningof{true} = \pi, \and \meaningof{~E} = \pi \setminus \meaningof{E}, \and \meaningof{E_{1} \& E_{2}} = \meaningof{E_{1}} \cap \meaningof{E_{2}}}
\end{mathpar}

\begin{mathpar}
  \inferrule* [lab=structure] {} {\meaningof{0} = \{ P \in \pi | P \equiv 0 \}, \and \\ \meaningof{E_1 | E_2} = \{ P \in \pi | P \equiv P_{1} | P_{2}, P_{1} \in \meaningof{E_{1}}, P_{2} \in \meaningof{E_2}\} }
\end{mathpar}

\begin{mathpar}
 \inferrule* [lab=behavior] {} {\meaningof{\langle a?b \rangle E} = \{ P \in \pi | P \equiv Q | u?(y)P', \\ \and \\\\ \and \\ \;\;\; u \in \meaningof{a}, \forall z.P'\{z/y\} \in \meaningof{E\{z/b\}}\}, \and \\ \meaningof{a!E} = \{ P \in \pi | P \equiv Q | x!\langle P' \rangle, x \in \meaningof{a} P' \in \meaningof{E}\} }
\end{mathpar}

\begin{mathpar}
 \inferrule* [lab=nominal] {} {\meaningof{\quotep{E}} = \{ \quotep{P} \in \quotep{\pi} | P \in \meaningof{E} \}, \and \meaningof{\quotep{P}} = \{ \quotep{Q} \in \quotep{\pi} | P \equiv Q \} \and \\ \meaningof{@\quotep{E}} = \{ P \in \pi | P \equiv @x, x \in \meaningof{E} \}}
\end{mathpar}

\begin{eqnarray*}
  \\
  \meaningof{-} : TS \to ST
\end{eqnarray*}

\begin{eqnarray*}
  \\
  L : TS \to ST
\end{eqnarray*}

\begin{eqnarray*}
  \\
  P \models E \iff P \in \meaningof{E}
\end{eqnarray*}

\begin{eqnarray*}
  P \approx_{L} Q \iff \forall E \in L. P \models E \iff Q \models E
\end{eqnarray*}

\begin{eqnarray*}
  P \approx_{K} Q
\end{eqnarray*}

\begin{eqnarray*}
  P \approx Q
\end{eqnarray*}

$\approx_{K} = \approx = \approx_{L}$

\subsubsection{Contextual duality}

Note that contexts extend the quotation operation to a family of
operations from processes to names. Given a context, $M$, we can
define a \emph{nominal context}, $\quotep{M}$ by $\quotep{M}[P] :=
\quotep{M[P]}$. To foreshadow what is to come we observe that these
operations enjoy a duality with processes very much like the duality
between vectors and maps from vectors to scalars.

Further, because the calculus is essentially higher-order, we have a
correspondence between contexts and processes. More specifically,
given a name $x$ and a context $M$ we can construct $M^{*}_{x}$ such
that 

\begin{mathpar}
  M^{*}_{x} | \lift{x}{P} \red M[P]
\end{mathpar}

namely,

\begin{mathpar}
  M^{*}_{x} := x?(u).M[\dropn{u}]
\end{mathpar}

The dependence of $M^{*}_{x}$ on a name makes it an abstraction, 

\begin{mathpar}
  M^{*} := (x)x?(u).M[\dropn{u}]
\end{mathpar}

\subsection{Additional notation}

It will sometimes be convenient to denote the process a name
quotes. We already have the notation $x = \quotep{P}$, but it will be
convenient to introduce an alternate notation, $\procn{x}$, when we
want to emphasize the connection to the use of the name. Note that, by
virtue of name equivalence, $\quotep{\procn{x}} \nameeq x$; so, the
notation is consistent with previous definitions.

Further, because names have structure it is possible to effect
substitutions on the basis of that structure. This means we need to
upgrade our notation for substitutions, which we accomplish by
adapting comprehension notation. Thus,

\begin{mathpar}
  P\{ y / x : x \in S \}
\end{mathpar}

is interpreted to mean the process derived from P by replacing (in a
capture-avoiding manner) each occurrence of $x$ in $S$ by $y$. For example,

\begin{mathpar}
  P\{ \quotep{\procn{x}|\procn{x}} / x : x \in \freenames{P} \}
\end{mathpar}

will replace each (occurrence) of a free name $x$ in $P$ by
$\quotep{\procn{x}|\procn{x}}$.

Also, we will avail ourselves of the notation $x^{L}$ and $x^{R}$ to
denote injections of a name into disjoint copies of the name
space. There are numerous ways to accomplish this. One example can be
found in \cite{MeredithR05}. This notation overloads to vectors of
names: $\vec{x}^{\pi} := (x_{i}^{\pi} \; : \; 0 \leq i < |\vec{x}| )$ where $\pi \in \{L,R\}$.

We also use $P^{\Box} := P|\Box$.

In \cite{MeredithR05} an interpretation of the new operator is
given. It turns out that there are several possible interpretations
all enjoying the requisite algebraic properties of the operator (see
\cite{milner91polyadicpi}). We will therefore make liberal use of
$(\nu\; \vec{x})P$.

% subsection the_syntax_and_semantics_of_the_notation_system (end)   

\input{qm2pi.qmops} 

\input{qm2pi.sterngerlach} 

\input{qm2pi.metric} 

% section concurrent_process_calculi (end)

%\input{qm2pi.proofsketch}

% section proof sketch (end)

%\input{qm2pi.slviaknots} 

% section spatial logic via knots (end)

\input{qm2pi.conclusion}

% section conclusion (end)

%\input{qm2pi.dtcodes} 

% section wiring algorithm (end)

\input{qm2pi.ack} 

% section acknowledgments (end)

\newpage


\bibliographystyle{plain}   
\bibliography{../../biblios/main.bib}

\input{qm2pi.rhodetails}

\end{document}

 

%\ifpdf
%\usepackage[pdftex]{graphicx}
%\else
%\usepackage{graphicx}
%\fi

 % \ifpdf
%  \usepackage{pdfsync}
%  \if


%\title{Brief Article}
%\author{David F. Snyder}
%\author{L.G. Meredith}

%\address{Dept. of Math., Texas State University--San Marcos, San Marcos, TX 78666}
       
\pagestyle{empty}


\begin{document}

\lstset{language=[Objective]Caml,frame=shadowbox}

\documentclass[12pt]{llncs}
%\documentclass{jktr}

\usepackage[pdftex]{hyperref}                   
\usepackage {listings}
\usepackage {mathpartir}
\usepackage{bcprules}
%\usepackage{listings}
                       
\usepackage{graphicx} 
%\usepackage[margins=2.5cm,nohead,nofoot]{geometry}
%\usepackage{geometry}
\usepackage{amsfonts}
\usepackage{amstext}
\usepackage{latexsym}
\usepackage{amssymb}
\usepackage{color}


%\include{myPreamble}
\include{qm2pi.local} 

%\ifpdf
%\usepackage[pdftex]{graphicx}
%\else
%\usepackage{graphicx}
%\fi

 % \ifpdf
%  \usepackage{pdfsync}
%  \if


%\title{Brief Article}
%\author{David F. Snyder}
%\author{L.G. Meredith}

%\address{Dept. of Math., Texas State University--San Marcos, San Marcos, TX 78666}
       
\pagestyle{empty}


\begin{document}

\lstset{language=[Objective]Caml,frame=shadowbox}

\input{qm2pi.front}

% section front matter (end)

\input{qm2pi.intro} 
 
% section introduction (end)

% \input{qm2pi.knotations} 

% section notation (end)

\input{qm2pi.process.calculi} 

% section concurrent_process_calculi_and_spatial_logics_ (end)
    
%\input{qm2pi.knots2pi} 

%\input{qm2pi.trefoil} 

%\input{qm2pi.mainthm} 

% subsection basic_interpretation (end)

%\input{qm2pi.rho.presentation} 
\subsection{The syntax and semantics of the notation system}\label{sub:the_syntax_and_semantics_of_the_notation_system} % (fold)

We now summarize a technical presentation of the calculus that
embodies our theory of dynamics. The typical presentation of such a
calculus follows the style of giving generators and relations on
them. The grammar, below, describing term constructors, freely
generates the set of processes, $\Proc$. This set is then quotiented
by a relation known as structural congruence and it is over this set
that the notion of dynamics is expressed. This presentation is
essentially that of \cite{MeredithR05} with the addition of
polyadicity and summation. For readability we have relegated some of
the technical subtleties to an appendix.

\subsubsection{Process grammar}\label{subsub:process_grammar}

\begin{mathpar}
  \inferrule* [lab=synchronization] {} {{M} \bc \pzero \;|\; x?F \;|\; x!C }
  \and
  \inferrule* [lab=abstraction] {} {{F} \bc (x)P}
  \and
  \inferrule* [lab=concretion] {} {{C} \bc \langle Q \rangle}
  \and
  \inferrule* [lab=process] {} {{P,Q} \bc M \;| \;P|Q \;|\; @{x}}
  \and
  \inferrule* [lab=name] {} {{x} \bc \quotep{P}}
\end{mathpar} 

Note that $\vec{x}$ (resp. $\vec{P}$) denotes a vector of names
(resp. processes) of length $|\vec{x}|$ (resp. $|\vec{P}|$). We adopt
the following useful abbreviations.

\begin{mathpar}
   x?(\vec{y}).P := x.(\vec{y})P \and  x\clift{\vec{P}} := x.\clift{\vec{P}}
   \and x!(y) := \lift{x}{\dropn{y}}
   \and \Pi_{i=0}^{n-1}P_i := P_0 | \ldots | P_{n-1}
\end{mathpar}

\subsubsection{Structural congruence}

\paragraph{Free and bound names and alpha-equivalence.} At the
core of structural equivalence is alpha-equivalence which identifies
process that are the same up to a change of variable. Formally, we
recognize the distinction between free and bound names. The free names
of a process, $\freenames{P}$, may be calculated recursively as
follows:

\begin{mathpar}
\freenames{\pzero} := \emptyset
  \and \\
  \freenames{x?(y).P} := \{ x \} \cup (\freenames{P} \setminus \{ y \})
  \and 
  \freenames{x!\langle P \rangle} := \{ x \} \cup \{ P \} 
  \and \\
  \freenames{P|Q} := \freenames{P} \cup \freenames{Q}
  \and \\
  \freenames{@{x}} := \{ x \}
\end{mathpar}

$\pi$
$\quotep{\pi}$

$\freenames{-} : \pi \to \mathcal{P}(\quotep{\pi})$

\begin{eqnarray*}
  \freenames{\pzero} & := & \emptyset \\
  \freenames{x?(y).P} & := & \{ x \} \cup (\freenames{P} \setminus \{ y \}) \\
  \freenames{x!\langle P \rangle} & := & \{ x \} \cup \{ P \} \\
  \freenames{P|Q} & := & \freenames{P} \cup \freenames{Q} \\
  \freenames{\dropn{x}} & := & \{ x \}
\end{eqnarray*}

The bound names of a process, $\boundnames{P}$, are those names occurring in $P$
that are not free. For example, in $x?(y).0$, the name $x$ is free, while $y$ is bound.

\begin{mathpar}
  \inferrule* [lab=monoidal-laws] {} { P|Q \equiv Q|P \and P|0 \equiv P \and P|(Q|R) \equiv (P|Q)|R }
\end{mathpar}

\begin{mathpar}
  \inferrule* [lab=alpha-equivalence] {} { (x)P \equiv (y)P\{y/x\} \and y \not\in \freenames{P} }
\end{mathpar}

\begin{definition}
Then two processes, $P,Q$, are alpha-equivalent if $P = Q\{\vec{y}/\vec{x}\}$ for
some $\vec{x} \in \boundnames{Q},\vec{y} \in \boundnames{P}$, where $Q\{\vec{y}/\vec{x}\}$
denotes the capture-avoiding substitution of $\vec{y}$ for $\vec{x}$ in $Q$.
\end{definition}

\begin{definition}
  The {\em structural congruence} \cite{SangiorgiWalker} , $\equiv$,
  between processes is the least congruence containing
  alpha-equivalence, satisfying the abelian monoid laws
  (associativity, commutativity and $\pzero$ as identity) for parallel
  composition $|$ and for summation $+$.
\end{definition}

\subsection{Name equivalence}

We take name equivalence, written $\nameeq$, to be the smallest
equivalence relation generated by the following rules.

\begin{mathpar}
\inferrule*[lab=Quote-drop]
{ }
{ \quotep{@{x}} \nameeq x }

\inferrule*[lab=Struct-equiv]
{ P \scong Q }
{ \quotep{P} \nameeq \quotep{Q} }
\end{mathpar}

The astute reader will have noticed that the mutual recursion of names
and processes imposes a mutual recursion on alpha-equivalence and
structural equivalence via name-equivalence. Fortunately, all of this
works out pleasantly and we may calculate in the natural way, free of
concern. The reader interested in the details is referred to the
appendix \ref{appendix:rho_details}.

\subsection{Substitution}

We use $\Proc$ for the set of processes, $\QProc$ for the set of
names, and $\id{\{}\vec{y} / \vec{x} \id{\}}$ to denote partial maps,
$s : \QProc \rightarrow \QProc$. A map, $s$ lifts, uniquely, to a map
on process terms, $\widehat{s} : \Proc \rightarrow \Proc$ by the
following equations.

\begin{mathpar}
  (0) \psubstp{Q}{P} := 0 \\
  (R \juxtap S) \psubstp{Q}{P}
  :=    
  (R)\psubstp{Q}{P} \juxtap (S) \psubstp{Q}{P} \\
  (x?(y).R) \psubstp{Q}{P}    
  :=    
  (x)\substp{Q}{P} (z)\concat( (R \psubstn{z}{y}) \psubstp{Q}{P} ) \\
  (\lift{x}{R}) \psubstp{Q}{P}  
  :=
  \lift{(x)\substp{Q}{P}}{ R \psubstp{Q}{P} } \\
%   (\dropn{x})  \psubstp{Q}{P}       
%   := 
%   \left\{ 
%     \begin{array}{ccc} 
%       \dropn{\quotep{Q}} & & x \nameeq \quotep{P} \\
%       \dropn{x} & & otherwise \\
%     \end{array}
%   \right. 
  (\dropn{x})  \psubstp{Q}{P}       
  := 
  \left\{ 
    \begin{array}{ccc} 
      Q & & x \nameeq \quotep{P} \\
      \dropn{x} & & otherwise \\
    \end{array}
  \right.
\end{mathpar}
 

where

\begin{eqnarray}
  (x)\id{\{} \lpquote Q \rpquote / \lpquote P \rpquote \id{\}}            = 
  \left\{ 
    \begin{array}{ccc}
      \lpquote Q \rpquote & & x \nameeq \lpquote P \rpquote \\
      x & & otherwise \\
    \end{array}
  \right. \nonumber
\end{eqnarray}

and $z$ is chosen distinct from $\quotep{P}$, $\quotep{Q}$, the free
names in $Q$, and all the names in $R$. Our $\alpha$-equivalence will
be built in the standard way from this substitution.

\begin{remark}\label{rem:no_self_referential_names}
  One consequence of these definitions is that $\forall P. \quotep{P}
  \not\in \freenames{P}$.
\end{remark}

\subsection{ Dynamic quote: an example }

Anticipating something of what's to come, consider applying the
substitution, $\widehat{\id{\{}u / z \id{\}}}$, to the following pair
of processes, $\lift{w}{y!(z)}$ and $w[ \lpquote y!(z) \rpquote ]$.

\begin{eqnarray}
	\lift{w}{y!(z)}\widehat{\id{\{}u / z \id{\}}}
		& = &
		\lift{w}{y!(u)} \nonumber\\
	w[ \lpquote y!(z) \rpquote ] \widehat{ \id{\{}u / z \id{\}} }
		& = &
		w[ \lpquote y!(z) \rpquote ] \nonumber
\end{eqnarray}

Because the body of the process between quotes is impervious to
substitution, we get radically different answers. In fact, by
examining the first process in an input context,
e.g. $x?(z).\lift{w}{y!(z)}$, we see that the process under the lift
operator may be shaped by prefixed inputs binding a name inside it. In
this sense, the lift operator will be seen as a way to dynamically
construct processes before reifying them as names.

Finally equipped with these standard features we can present the
dynamics of the calculus.

\subsubsection{Operational semantics} 

Finally, we introduce the computational dynamics. What marks these
algebras as distinct from other more traditionally studied algebraic
structures, e.g. vector spaces or polynomial rings, is the manner in
which dynamics is captured. In traditional structures, dynamics is typically
expressed through morphisms between such structures, as in linear maps
between vector spaces or morphisms between rings. In algebras
associated with the semantics of computation, the dynamics is
expressed as part of the algebraic structure itself, through a
reduction reduction relation typically denoted by $\red$. Below, we
give a recursive presentation of this relation for the calculus used
in the encoding.

$\red \subseteq \pi \times \pi$
$\red : \pi \to \mathcal{P}(\pi)$

\begin{mathpar}
  \inferrule* [lab=Comm] { \textsf{match}( x_{src}, x_{trgt} ) } { x_{trgt}?(y)P \; | \; x_{src}!\langle {Q} \rangle \red P\{\quotep{Q}/y}\} }
  \and \\
  \inferrule* [lab=Par] {{P} \red {P}'} {{{P} | {Q}} \red {{P}' | {Q}}}
  \and
  \inferrule* [lab=Equiv]{{{P} \scong {P}'} \andalso {{P}' \red {Q}'} \andalso {{Q}' \scong {Q}}}{{P} \red {Q}}
\end{mathpar}

\begin{eqnarray*}
  match_{\equiv} (\quotep{P},\quotep{Q}) & := & P \equiv Q \\
  match_{\dagger}(\quotep{P},\quotep{Q}) & := & \forall R. P|Q \red^{*} R => R \red^{*} 0 \\
  match_{K}(\quotep{P},\quotep{Q}) & := & K \mbox{ for some context } K
\end{eqnarray*}

$u?(x)P | u!\langle Q \rangle \red P\{\quotep{Q}/x\}$

%We write $\wred$ for $\red^*$, and $P\red$ if $\exists Q $ such that $ P \red Q$.
We write $P\red$ if $\exists Q $ such that $ P \red Q$ and $P\not\red$, otherwise.

\section{Replication}

As mentioned before, it is known that replication (and hence
recursion) can be implemented in a higher-order process algebra
\cite{SangiorgiWalker}. As our first example of calculation with the
machinery thus far presented we give the construction explicitly in
the {\rhoc}.

\begin{eqnarray}
	D_{x} & := & \prefix{x}{y}{(\binpar{\outputp{x}{y}}{@{y}})} \nonumber\\
	\bangp_{x}{P} & := & \binpar{{x}!\langle{\binpar{D_{x}}{P}}\rangle}{D_{x}} \nonumber
\end{eqnarray}

\begin{eqnarray}
	\bangp_{x}{P} & & \nonumber\\
	=
	& {x}!\langle{(\prefix{x}{y}{(\outputp{x}{y} | @{y})) | P}}\rangle 
	      | \prefix{x}{y}{(\outputp{x}{y} | @{y})} & \nonumber\\
	\red
	& (\outputp{x}{y} | @{y})\substn{\quotep{(\prefix{x}{y}{(@{y} | \outputp{x}{y})) | P}}}{y} & \nonumber\\
	=
	& \outputp{x}{\quotep{(\prefix{x}{y}{(\outputp{x}{y} | @{y})) | P}}}
	  | {(\prefix{x}{y}{(\outputp{x}{y} | @{y})) | P}} & \nonumber\\
	\red
	& \ldots & \nonumber\\
	\red^*
	& P | P | \ldots & \nonumber
\end{eqnarray}

Of course, this encoding, as an implementation, runs away, unfolding
$\bangp{P}$ eagerly. A lazier and more implementable replication
operator, restricted to input-guarded processes, may be obtained as follows.

\begin{eqnarray}
\bangp{\prefix{u}{v}{P}} 
	:= 
	\binpar{\lift{x}{\prefix{u}{v}{(\binpar{D(x)}{P})}}}{D(x)} \nonumber
\end{eqnarray}

\begin{remark}
  Note that the lazier definition still does not deal with summation
  or mixed summation (i.e. sums over input and output). The reader is
  invited to construct definitions of replication that deal with these
  features. 

  Further, the definitions are parameterized in a name, $x$. Can you,
  gentle reader, make a definition that eliminates this parameter and
  guarantees no accidental interaction between the replication
  machinery and the process being replicated -- i.e. no accidental
  sharing of names used by the process to get its work done and the
  name(s) used by the replication to effect copying. This latter
  revision of the definition of replication is crucial to obtaining
  the expected identity $!!P \sim !P$.
\end{remark}

\begin{remark}\label{rem:paradoxical_combinator}
  The reader familiar with the lambda calculus will have noticed the
  similarity between $D$ and the paradoxical combinator.

  [Ed. note: the existence of this seems to suggest we have to be more
  restrictive on the set of processes and names we admit if we are to
  support no-cloning.]
\end{remark}

\subsubsection{Bisimulation}

The computational dynamics gives rise to another kind of equivalence,
the equivalence of computational behavior. As previously mentioned
this is typically captured \emph{via} some form of bisimulation.

% The notion we use in this paper is weak barbed bisimulation
% \cite{milner91polyadicpi}.

The notion we use in this paper is derived from weak barbed
bisimulation \cite{milner91polyadicpi}. 

\begin{definition}
An \emph{observation relation}, $\downarrow_{\mathcal N}$, over a set
of names, $\mathcal N$, is the smallest relation satisfying the rules
below.

\infrule[Out-barb]{y \in {\mathcal N}, \; x \nameeq y}
		  {\outputp{x}{v} \downarrow_{\mathcal N} x}
\infrule[Par-barb]{\mbox{$P\downarrow_{\mathcal N} x$ or $Q\downarrow_{\mathcal N} x$}}
		  {\binpar{P}{Q} \downarrow_{\mathcal N} x}

We write $P \Downarrow_{\mathcal N} x$ if there is $Q$ such that 
$P \wred Q$ and $Q \downarrow_{\mathcal N} x$.
\end{definition}

\begin{definition}
%\label{def.bbisim}
An  ${\mathcal N}$-\emph{barbed bisimulation} over a set of names, ${\mathcal N}$, is a symmetric binary relation 
${\mathcal S}_{\mathcal N}$ between agents such that $P\rel{S}_{\mathcal N}Q$ implies:
\begin{enumerate}
\item If $P \red P'$ then $Q \wred Q'$ and $P'\rel{S}_{\mathcal N} Q'$.
\item If $P\downarrow_{\mathcal N} x$, then $Q\Downarrow_{\mathcal N} x$.
\end{enumerate}
$P$ is ${\mathcal N}$-barbed bisimilar to $Q$, written
$P \wbbisim_{\mathcal N} Q$, if $P \rel{S}_{\mathcal N} Q$ for some ${\mathcal N}$-barbed bisimulation ${\mathcal S}_{\mathcal N}$.
\end{definition}

$\mathcal{R} \subseteq \pi \times \pi$

$P \mathcal{R} Q => \forall P'. P \red P' \Rightarrow \exists Q'. Q \red Q', P' \mathcal{R} Q'$

$P \vdash x \Rightarrow Q \vdash x$

\begin{mathpar}
  \inferrule*[lab=Out-barb]{x \nameeq y}{{y}!\langle{Q}\rangle \vdash x}
  \and
  \inferrule*[lab=Par-barb]{\mbox{$P\vdash x$ or $Q\vdash x$}}{\binpar{P}{Q} \vdash x}
\end{mathpar}

\subsubsection{Contexts}

One of the principle advantages of computational calculi like the
$\pi$-calculus is a well-defined notion of context,
contextual-equivalence and a correlation between
contextual-equivalence and notions of bisimulation. The notion of
context allows the decomposition of a process into (sub-)process and
its syntactic environment, its context. Thus, a context may be
thought of as a process with a ``hole'' (written $\Box$) in it. The
application of a context $M$ to a process $P$, written $M[P]$, is
tantamount to filling the hole in $M$ with $P$. In this paper we do
not need the full weight of this theory, but do make use of the notion
of context in the proof the main theorem. 

\begin{mathpar}
  \inferrule* [lab=summation] {} {{M_{M},M_{N}} \bc \Box \;|\; x.M_{A} \;|\; M_{M}+M_{N}}
  \and
  \inferrule* [lab=agent] {} {{M_{A}} \bc (\vec{x})M_{P} \;| \; \clift{P_0,\ldots,M_{P},\ldots,P_N}}
  \and \\
  \inferrule* [lab=process] {} {{M_{P}} \bc M_{N} \;| \;P|M_{P} }
\end{mathpar} 

\begin{mathpar}
  \inferrule* [lab=sychronization] {} {M_{N} \bc \Box \;|\; x?M_{F} \;|\; x!M_{C}}
  \and
  \inferrule* [lab=abstraction] {} {{M_{F}} \bc (x)M_{P} }
  \and
  \inferrule* [lab=concretion] {} {{M_{C}} \bc \langle M_{P} \rangle }
  \and \\
  \inferrule* [lab=process] {} {{M_{P}} \bc M_{N} \;| \;P|M_{P} }
\end{mathpar}

\begin{definition}[contextual application] Given a context $M$, and
  process $P$, we define the \emph{contextual application}, $M[P] :=
  M\{P/\Box\}$. That is, the contextual application of M to P is the
  substitution of $P$ for $\Box$ in $M$.
\end{definition}

$\meaningof{-} : L \to \mathcal{P}(\pi)$

\begin{mathpar}
  \inferrule* [lab=collection] {} {\meaningof{true} = \pi, \and \meaningof{~E} = \pi \setminus \meaningof{E}, \and \meaningof{E_{1} \& E_{2}} = \meaningof{E_{1}} \cap \meaningof{E_{2}}}
\end{mathpar}

\begin{mathpar}
  \inferrule* [lab=structure] {} {\meaningof{0} = \{ P \in \pi | P \equiv 0 \}, \and \\ \meaningof{E_1 | E_2} = \{ P \in \pi | P \equiv P_{1} | P_{2}, P_{1} \in \meaningof{E_{1}}, P_{2} \in \meaningof{E_2}\} }
\end{mathpar}

\begin{mathpar}
 \inferrule* [lab=behavior] {} {\meaningof{\langle a?b \rangle E} = \{ P \in \pi | P \equiv Q | u?(y)P', \\ \and \\\\ \and \\ \;\;\; u \in \meaningof{a}, \forall z.P'\{z/y\} \in \meaningof{E\{z/b\}}\}, \and \\ \meaningof{a!E} = \{ P \in \pi | P \equiv Q | x!\langle P' \rangle, x \in \meaningof{a} P' \in \meaningof{E}\} }
\end{mathpar}

\begin{mathpar}
 \inferrule* [lab=nominal] {} {\meaningof{\quotep{E}} = \{ \quotep{P} \in \quotep{\pi} | P \in \meaningof{E} \}, \and \meaningof{\quotep{P}} = \{ \quotep{Q} \in \quotep{\pi} | P \equiv Q \} \and \\ \meaningof{@\quotep{E}} = \{ P \in \pi | P \equiv @x, x \in \meaningof{E} \}}
\end{mathpar}

\begin{eqnarray*}
  \\
  \meaningof{-} : TS \to ST
\end{eqnarray*}

\begin{eqnarray*}
  \\
  L : TS \to ST
\end{eqnarray*}

\begin{eqnarray*}
  \\
  P \models E \iff P \in \meaningof{E}
\end{eqnarray*}

\begin{eqnarray*}
  P \approx_{L} Q \iff \forall E \in L. P \models E \iff Q \models E
\end{eqnarray*}

\begin{eqnarray*}
  P \approx_{K} Q
\end{eqnarray*}

\begin{eqnarray*}
  P \approx Q
\end{eqnarray*}

$\approx_{K} = \approx = \approx_{L}$

\subsubsection{Contextual duality}

Note that contexts extend the quotation operation to a family of
operations from processes to names. Given a context, $M$, we can
define a \emph{nominal context}, $\quotep{M}$ by $\quotep{M}[P] :=
\quotep{M[P]}$. To foreshadow what is to come we observe that these
operations enjoy a duality with processes very much like the duality
between vectors and maps from vectors to scalars.

Further, because the calculus is essentially higher-order, we have a
correspondence between contexts and processes. More specifically,
given a name $x$ and a context $M$ we can construct $M^{*}_{x}$ such
that 

\begin{mathpar}
  M^{*}_{x} | \lift{x}{P} \red M[P]
\end{mathpar}

namely,

\begin{mathpar}
  M^{*}_{x} := x?(u).M[\dropn{u}]
\end{mathpar}

The dependence of $M^{*}_{x}$ on a name makes it an abstraction, 

\begin{mathpar}
  M^{*} := (x)x?(u).M[\dropn{u}]
\end{mathpar}

\subsection{Additional notation}

It will sometimes be convenient to denote the process a name
quotes. We already have the notation $x = \quotep{P}$, but it will be
convenient to introduce an alternate notation, $\procn{x}$, when we
want to emphasize the connection to the use of the name. Note that, by
virtue of name equivalence, $\quotep{\procn{x}} \nameeq x$; so, the
notation is consistent with previous definitions.

Further, because names have structure it is possible to effect
substitutions on the basis of that structure. This means we need to
upgrade our notation for substitutions, which we accomplish by
adapting comprehension notation. Thus,

\begin{mathpar}
  P\{ y / x : x \in S \}
\end{mathpar}

is interpreted to mean the process derived from P by replacing (in a
capture-avoiding manner) each occurrence of $x$ in $S$ by $y$. For example,

\begin{mathpar}
  P\{ \quotep{\procn{x}|\procn{x}} / x : x \in \freenames{P} \}
\end{mathpar}

will replace each (occurrence) of a free name $x$ in $P$ by
$\quotep{\procn{x}|\procn{x}}$.

Also, we will avail ourselves of the notation $x^{L}$ and $x^{R}$ to
denote injections of a name into disjoint copies of the name
space. There are numerous ways to accomplish this. One example can be
found in \cite{MeredithR05}. This notation overloads to vectors of
names: $\vec{x}^{\pi} := (x_{i}^{\pi} \; : \; 0 \leq i < |\vec{x}| )$ where $\pi \in \{L,R\}$.

We also use $P^{\Box} := P|\Box$.

In \cite{MeredithR05} an interpretation of the new operator is
given. It turns out that there are several possible interpretations
all enjoying the requisite algebraic properties of the operator (see
\cite{milner91polyadicpi}). We will therefore make liberal use of
$(\nu\; \vec{x})P$.

% subsection the_syntax_and_semantics_of_the_notation_system (end)   

\input{qm2pi.qmops} 

\input{qm2pi.sterngerlach} 

\input{qm2pi.metric} 

% section concurrent_process_calculi (end)

%\input{qm2pi.proofsketch}

% section proof sketch (end)

%\input{qm2pi.slviaknots} 

% section spatial logic via knots (end)

\input{qm2pi.conclusion}

% section conclusion (end)

%\input{qm2pi.dtcodes} 

% section wiring algorithm (end)

\input{qm2pi.ack} 

% section acknowledgments (end)

\newpage


\bibliographystyle{plain}   
\bibliography{../../biblios/main.bib}

\input{qm2pi.rhodetails}

\end{document}



% section front matter (end)

\section{Introduction}\label{sec:introduction} % (fold)
In this draft of the material i am going to have to dispense with the
usual writing conventions adopted in papers on these topics. i'm going
to have adopt whatever tone i need at the time i'm writing up the
calculations. Sometimes this may be very conversational; others it may
be the barest mathematical grunts; others still it may be that i have
lifted text from one of my other papers because the exposition of some
point was better said there. i hope that my readers are not unduly put
out by this decision. i'm not doing this to flout convention or be
rebellious. i find these calculations very technically challenging. To
keep everything going technically, something has to give; i have to
let go of some cognitive burden. So, the academic writing style --
with all of its trade-offs in terms of facilitating technical
communication -- is what i'm letting go of. Perhaps subsequent drafts
can be tightened and polished, but for now, i'm going to speak as if
we were sitting together in a coffee shop with a laptop, wifi and a
pad of paper and a pencil.

So, here's what i have to say. We -- you and i, comfortably ensconced
in our coffee shop and well-equipped with our tools -- can realize and
carry out the calculations of quantum mechanics over a very different
formal theory of dynamics, a formal theory of dynamics that
corresponds to a theory of concurrent computation with
\emph{reflection}. It has the advantage that the underlying theory is
already `quantized', but supports analogues all of the continuuous
operations. Strikingly, this underlying theory has recently been
connected with a notion of metric that we can show, by calculating
together, coincides with the metric induced by the inner product.

There are a lot of reasons why you might be interested in seeing
calculations of this form. Here's why i'm interested. For the past
several centuries there has been no competitor to the ``Newtonian''
account of dynamics. As a result the predominant share of accounts of
dynamical systems and situations have had to be formulated in terms of
the Newtonian machinery. i view this as an intellectually dangerous
position to occupy. Everything, despite it's intrinsic shape, turns
into a nail to be hit with this hammer. Recently, however, the theory
of computation has matured to the point where we have candidates for
theories of dynamics that offer very different perspective on
reasoning about dynamical systems and situations. Testing these
candidates against very successful accounts of dynamical situations,
like quantum mechanics, is going to give us some sense of how mature
they are and some measure of the quality of these accounts of
dynamics.

\subsection{Summary of contributions and outline of paper}

So, we're going to develop an interpretation of the operations of
quantum mechanics normally interpreted by Hilbert spaces and
operators. We're going to do this over a theory of computation. Note
that this is very different than the usual quantum computation program
which develops notions of computation over quantum mechanics. Rather,
we are developing a story that aligns with Wheeler's slogan: It from
Bit. To do this we will first provide an account of the theory of
computation at play here. Then we will dive into a calculation-driven
interpretation of the operations of quantum mechanics.

The reason we take this approach is that -- until very recently --
there hasn't been an axiomatic account of quantum mechanics. As a
result there has been no sharp delineation of the mathematical theory
supporting interpretation of the physical theory and the physical
theory, itself. So, ambient features of the maths are free to be
exploited (or supressed) without a real accounting of their physical
relevance. There is no sharp statement ``here's the physical theory''
qua \emph{theory} and ``here's the mathematical interpretation''
enabling a judgment of how faithful the interpretation is -- apart
from experimental observation. When there is an axiomatic account we
can judge how well a given mathematical formalism supports an
interpretation of the axioms, independent of
experimentation. Likewise, we can judge how well we have captured our
physical evidence and experience with our axiomatics, independent of
any specific mathematical implementation, with accidental detail that
may or may not have physical significance. 

In lieu of a fully fleshed out and vetted axiomatic account of quantum
mechanics, interpreting the operational notions in service of modeling
physical systems will have to suffice. In other words, we are not in
the business of providing a model of Hilbert spaces and operators. We
are in the business of providing a model of quantum mechanics because
we are motivated by testing our notions of dynamics against physical
theory; and, the predictive calculations of the physical theory must
serve as the best formulation -- shy of a fully fleshed out axiomatic
account -- of the physical theory itself (as they have for scientific
theories since time immemorial). Put another way, despite a
whole-hearted commitment to an It-from-Bit ontology, we are firmly
aligned with the shut-up-and-calculate camp as the best way to obtain
results either from the physical perspective or as a quality assurance
measure of our fledgling theory of dynamics.

In detail, we present a reflective process calculus. Then we develop
intuitive correspondences between the notions available in this
calculus and the usual physical notions supporting quantum mechanical
calculations. Thus, 

\begin{table}[htp]
  \center{
    \fbox{
      \begin{tabular}{c|c}
        quantum mechanics & process calculus \\
        \hline
        scalar & name \\
        state vector & process \\
        dual & contextual duals \\
        matrix & formal sums of process-context-dual pairs \\
        orthogonality & process annihilation \\
        inner product & execution-formula + quoting
      \end{tabular}
    }
  }
  \caption{QM - process calculi correspondences}
\end{table}

Then we tighten up these intuitions to operational definitions. We
employ the Dirac notation as the best proxy we can find for an
abstract syntax of the quantum mechanical notions. The definitions we
develop put us in contact with equational constraints coming from the
theory that we demonstrate the definitions and calculations satisfy.

This puts us in a position to shut up and calculate for the
Stern-Gerlach experimental set up, showing how these predictive
calculations become calculations on processes in our theory of a
reflective process calculus.

Penultimately, we demonstrate that the notion of metric coming from
the inner product coincides with the notion of metric available from
the theory of bisimulation. This demonstration gives us the right to
think of space as arising from behavior. Finally, we consider where we
might go from the new vantage point we have obtained.

% section introduction (end) 
 
% section introduction (end)

% \documentclass[12pt]{llncs}
%\documentclass{jktr}

\usepackage[pdftex]{hyperref}                   
\usepackage {listings}
\usepackage {mathpartir}
\usepackage{bcprules}
%\usepackage{listings}
                       
\usepackage{graphicx} 
%\usepackage[margins=2.5cm,nohead,nofoot]{geometry}
%\usepackage{geometry}
\usepackage{amsfonts}
\usepackage{amstext}
\usepackage{latexsym}
\usepackage{amssymb}
\usepackage{color}


%\include{myPreamble}
\include{qm2pi.local} 

%\ifpdf
%\usepackage[pdftex]{graphicx}
%\else
%\usepackage{graphicx}
%\fi

 % \ifpdf
%  \usepackage{pdfsync}
%  \if


%\title{Brief Article}
%\author{David F. Snyder}
%\author{L.G. Meredith}

%\address{Dept. of Math., Texas State University--San Marcos, San Marcos, TX 78666}
       
\pagestyle{empty}


\begin{document}

\lstset{language=[Objective]Caml,frame=shadowbox}

\input{qm2pi.front}

% section front matter (end)

\input{qm2pi.intro} 
 
% section introduction (end)

% \input{qm2pi.knotations} 

% section notation (end)

\input{qm2pi.process.calculi} 

% section concurrent_process_calculi_and_spatial_logics_ (end)
    
%\input{qm2pi.knots2pi} 

%\input{qm2pi.trefoil} 

%\input{qm2pi.mainthm} 

% subsection basic_interpretation (end)

%\input{qm2pi.rho.presentation} 
\subsection{The syntax and semantics of the notation system}\label{sub:the_syntax_and_semantics_of_the_notation_system} % (fold)

We now summarize a technical presentation of the calculus that
embodies our theory of dynamics. The typical presentation of such a
calculus follows the style of giving generators and relations on
them. The grammar, below, describing term constructors, freely
generates the set of processes, $\Proc$. This set is then quotiented
by a relation known as structural congruence and it is over this set
that the notion of dynamics is expressed. This presentation is
essentially that of \cite{MeredithR05} with the addition of
polyadicity and summation. For readability we have relegated some of
the technical subtleties to an appendix.

\subsubsection{Process grammar}\label{subsub:process_grammar}

\begin{mathpar}
  \inferrule* [lab=synchronization] {} {{M} \bc \pzero \;|\; x?F \;|\; x!C }
  \and
  \inferrule* [lab=abstraction] {} {{F} \bc (x)P}
  \and
  \inferrule* [lab=concretion] {} {{C} \bc \langle Q \rangle}
  \and
  \inferrule* [lab=process] {} {{P,Q} \bc M \;| \;P|Q \;|\; @{x}}
  \and
  \inferrule* [lab=name] {} {{x} \bc \quotep{P}}
\end{mathpar} 

Note that $\vec{x}$ (resp. $\vec{P}$) denotes a vector of names
(resp. processes) of length $|\vec{x}|$ (resp. $|\vec{P}|$). We adopt
the following useful abbreviations.

\begin{mathpar}
   x?(\vec{y}).P := x.(\vec{y})P \and  x\clift{\vec{P}} := x.\clift{\vec{P}}
   \and x!(y) := \lift{x}{\dropn{y}}
   \and \Pi_{i=0}^{n-1}P_i := P_0 | \ldots | P_{n-1}
\end{mathpar}

\subsubsection{Structural congruence}

\paragraph{Free and bound names and alpha-equivalence.} At the
core of structural equivalence is alpha-equivalence which identifies
process that are the same up to a change of variable. Formally, we
recognize the distinction between free and bound names. The free names
of a process, $\freenames{P}$, may be calculated recursively as
follows:

\begin{mathpar}
\freenames{\pzero} := \emptyset
  \and \\
  \freenames{x?(y).P} := \{ x \} \cup (\freenames{P} \setminus \{ y \})
  \and 
  \freenames{x!\langle P \rangle} := \{ x \} \cup \{ P \} 
  \and \\
  \freenames{P|Q} := \freenames{P} \cup \freenames{Q}
  \and \\
  \freenames{@{x}} := \{ x \}
\end{mathpar}

$\pi$
$\quotep{\pi}$

$\freenames{-} : \pi \to \mathcal{P}(\quotep{\pi})$

\begin{eqnarray*}
  \freenames{\pzero} & := & \emptyset \\
  \freenames{x?(y).P} & := & \{ x \} \cup (\freenames{P} \setminus \{ y \}) \\
  \freenames{x!\langle P \rangle} & := & \{ x \} \cup \{ P \} \\
  \freenames{P|Q} & := & \freenames{P} \cup \freenames{Q} \\
  \freenames{\dropn{x}} & := & \{ x \}
\end{eqnarray*}

The bound names of a process, $\boundnames{P}$, are those names occurring in $P$
that are not free. For example, in $x?(y).0$, the name $x$ is free, while $y$ is bound.

\begin{mathpar}
  \inferrule* [lab=monoidal-laws] {} { P|Q \equiv Q|P \and P|0 \equiv P \and P|(Q|R) \equiv (P|Q)|R }
\end{mathpar}

\begin{mathpar}
  \inferrule* [lab=alpha-equivalence] {} { (x)P \equiv (y)P\{y/x\} \and y \not\in \freenames{P} }
\end{mathpar}

\begin{definition}
Then two processes, $P,Q$, are alpha-equivalent if $P = Q\{\vec{y}/\vec{x}\}$ for
some $\vec{x} \in \boundnames{Q},\vec{y} \in \boundnames{P}$, where $Q\{\vec{y}/\vec{x}\}$
denotes the capture-avoiding substitution of $\vec{y}$ for $\vec{x}$ in $Q$.
\end{definition}

\begin{definition}
  The {\em structural congruence} \cite{SangiorgiWalker} , $\equiv$,
  between processes is the least congruence containing
  alpha-equivalence, satisfying the abelian monoid laws
  (associativity, commutativity and $\pzero$ as identity) for parallel
  composition $|$ and for summation $+$.
\end{definition}

\subsection{Name equivalence}

We take name equivalence, written $\nameeq$, to be the smallest
equivalence relation generated by the following rules.

\begin{mathpar}
\inferrule*[lab=Quote-drop]
{ }
{ \quotep{@{x}} \nameeq x }

\inferrule*[lab=Struct-equiv]
{ P \scong Q }
{ \quotep{P} \nameeq \quotep{Q} }
\end{mathpar}

The astute reader will have noticed that the mutual recursion of names
and processes imposes a mutual recursion on alpha-equivalence and
structural equivalence via name-equivalence. Fortunately, all of this
works out pleasantly and we may calculate in the natural way, free of
concern. The reader interested in the details is referred to the
appendix \ref{appendix:rho_details}.

\subsection{Substitution}

We use $\Proc$ for the set of processes, $\QProc$ for the set of
names, and $\id{\{}\vec{y} / \vec{x} \id{\}}$ to denote partial maps,
$s : \QProc \rightarrow \QProc$. A map, $s$ lifts, uniquely, to a map
on process terms, $\widehat{s} : \Proc \rightarrow \Proc$ by the
following equations.

\begin{mathpar}
  (0) \psubstp{Q}{P} := 0 \\
  (R \juxtap S) \psubstp{Q}{P}
  :=    
  (R)\psubstp{Q}{P} \juxtap (S) \psubstp{Q}{P} \\
  (x?(y).R) \psubstp{Q}{P}    
  :=    
  (x)\substp{Q}{P} (z)\concat( (R \psubstn{z}{y}) \psubstp{Q}{P} ) \\
  (\lift{x}{R}) \psubstp{Q}{P}  
  :=
  \lift{(x)\substp{Q}{P}}{ R \psubstp{Q}{P} } \\
%   (\dropn{x})  \psubstp{Q}{P}       
%   := 
%   \left\{ 
%     \begin{array}{ccc} 
%       \dropn{\quotep{Q}} & & x \nameeq \quotep{P} \\
%       \dropn{x} & & otherwise \\
%     \end{array}
%   \right. 
  (\dropn{x})  \psubstp{Q}{P}       
  := 
  \left\{ 
    \begin{array}{ccc} 
      Q & & x \nameeq \quotep{P} \\
      \dropn{x} & & otherwise \\
    \end{array}
  \right.
\end{mathpar}
 

where

\begin{eqnarray}
  (x)\id{\{} \lpquote Q \rpquote / \lpquote P \rpquote \id{\}}            = 
  \left\{ 
    \begin{array}{ccc}
      \lpquote Q \rpquote & & x \nameeq \lpquote P \rpquote \\
      x & & otherwise \\
    \end{array}
  \right. \nonumber
\end{eqnarray}

and $z$ is chosen distinct from $\quotep{P}$, $\quotep{Q}$, the free
names in $Q$, and all the names in $R$. Our $\alpha$-equivalence will
be built in the standard way from this substitution.

\begin{remark}\label{rem:no_self_referential_names}
  One consequence of these definitions is that $\forall P. \quotep{P}
  \not\in \freenames{P}$.
\end{remark}

\subsection{ Dynamic quote: an example }

Anticipating something of what's to come, consider applying the
substitution, $\widehat{\id{\{}u / z \id{\}}}$, to the following pair
of processes, $\lift{w}{y!(z)}$ and $w[ \lpquote y!(z) \rpquote ]$.

\begin{eqnarray}
	\lift{w}{y!(z)}\widehat{\id{\{}u / z \id{\}}}
		& = &
		\lift{w}{y!(u)} \nonumber\\
	w[ \lpquote y!(z) \rpquote ] \widehat{ \id{\{}u / z \id{\}} }
		& = &
		w[ \lpquote y!(z) \rpquote ] \nonumber
\end{eqnarray}

Because the body of the process between quotes is impervious to
substitution, we get radically different answers. In fact, by
examining the first process in an input context,
e.g. $x?(z).\lift{w}{y!(z)}$, we see that the process under the lift
operator may be shaped by prefixed inputs binding a name inside it. In
this sense, the lift operator will be seen as a way to dynamically
construct processes before reifying them as names.

Finally equipped with these standard features we can present the
dynamics of the calculus.

\subsubsection{Operational semantics} 

Finally, we introduce the computational dynamics. What marks these
algebras as distinct from other more traditionally studied algebraic
structures, e.g. vector spaces or polynomial rings, is the manner in
which dynamics is captured. In traditional structures, dynamics is typically
expressed through morphisms between such structures, as in linear maps
between vector spaces or morphisms between rings. In algebras
associated with the semantics of computation, the dynamics is
expressed as part of the algebraic structure itself, through a
reduction reduction relation typically denoted by $\red$. Below, we
give a recursive presentation of this relation for the calculus used
in the encoding.

$\red \subseteq \pi \times \pi$
$\red : \pi \to \mathcal{P}(\pi)$

\begin{mathpar}
  \inferrule* [lab=Comm] { \textsf{match}( x_{src}, x_{trgt} ) } { x_{trgt}?(y)P \; | \; x_{src}!\langle {Q} \rangle \red P\{\quotep{Q}/y}\} }
  \and \\
  \inferrule* [lab=Par] {{P} \red {P}'} {{{P} | {Q}} \red {{P}' | {Q}}}
  \and
  \inferrule* [lab=Equiv]{{{P} \scong {P}'} \andalso {{P}' \red {Q}'} \andalso {{Q}' \scong {Q}}}{{P} \red {Q}}
\end{mathpar}

\begin{eqnarray*}
  match_{\equiv} (\quotep{P},\quotep{Q}) & := & P \equiv Q \\
  match_{\dagger}(\quotep{P},\quotep{Q}) & := & \forall R. P|Q \red^{*} R => R \red^{*} 0 \\
  match_{K}(\quotep{P},\quotep{Q}) & := & K \mbox{ for some context } K
\end{eqnarray*}

$u?(x)P | u!\langle Q \rangle \red P\{\quotep{Q}/x\}$

%We write $\wred$ for $\red^*$, and $P\red$ if $\exists Q $ such that $ P \red Q$.
We write $P\red$ if $\exists Q $ such that $ P \red Q$ and $P\not\red$, otherwise.

\section{Replication}

As mentioned before, it is known that replication (and hence
recursion) can be implemented in a higher-order process algebra
\cite{SangiorgiWalker}. As our first example of calculation with the
machinery thus far presented we give the construction explicitly in
the {\rhoc}.

\begin{eqnarray}
	D_{x} & := & \prefix{x}{y}{(\binpar{\outputp{x}{y}}{@{y}})} \nonumber\\
	\bangp_{x}{P} & := & \binpar{{x}!\langle{\binpar{D_{x}}{P}}\rangle}{D_{x}} \nonumber
\end{eqnarray}

\begin{eqnarray}
	\bangp_{x}{P} & & \nonumber\\
	=
	& {x}!\langle{(\prefix{x}{y}{(\outputp{x}{y} | @{y})) | P}}\rangle 
	      | \prefix{x}{y}{(\outputp{x}{y} | @{y})} & \nonumber\\
	\red
	& (\outputp{x}{y} | @{y})\substn{\quotep{(\prefix{x}{y}{(@{y} | \outputp{x}{y})) | P}}}{y} & \nonumber\\
	=
	& \outputp{x}{\quotep{(\prefix{x}{y}{(\outputp{x}{y} | @{y})) | P}}}
	  | {(\prefix{x}{y}{(\outputp{x}{y} | @{y})) | P}} & \nonumber\\
	\red
	& \ldots & \nonumber\\
	\red^*
	& P | P | \ldots & \nonumber
\end{eqnarray}

Of course, this encoding, as an implementation, runs away, unfolding
$\bangp{P}$ eagerly. A lazier and more implementable replication
operator, restricted to input-guarded processes, may be obtained as follows.

\begin{eqnarray}
\bangp{\prefix{u}{v}{P}} 
	:= 
	\binpar{\lift{x}{\prefix{u}{v}{(\binpar{D(x)}{P})}}}{D(x)} \nonumber
\end{eqnarray}

\begin{remark}
  Note that the lazier definition still does not deal with summation
  or mixed summation (i.e. sums over input and output). The reader is
  invited to construct definitions of replication that deal with these
  features. 

  Further, the definitions are parameterized in a name, $x$. Can you,
  gentle reader, make a definition that eliminates this parameter and
  guarantees no accidental interaction between the replication
  machinery and the process being replicated -- i.e. no accidental
  sharing of names used by the process to get its work done and the
  name(s) used by the replication to effect copying. This latter
  revision of the definition of replication is crucial to obtaining
  the expected identity $!!P \sim !P$.
\end{remark}

\begin{remark}\label{rem:paradoxical_combinator}
  The reader familiar with the lambda calculus will have noticed the
  similarity between $D$ and the paradoxical combinator.

  [Ed. note: the existence of this seems to suggest we have to be more
  restrictive on the set of processes and names we admit if we are to
  support no-cloning.]
\end{remark}

\subsubsection{Bisimulation}

The computational dynamics gives rise to another kind of equivalence,
the equivalence of computational behavior. As previously mentioned
this is typically captured \emph{via} some form of bisimulation.

% The notion we use in this paper is weak barbed bisimulation
% \cite{milner91polyadicpi}.

The notion we use in this paper is derived from weak barbed
bisimulation \cite{milner91polyadicpi}. 

\begin{definition}
An \emph{observation relation}, $\downarrow_{\mathcal N}$, over a set
of names, $\mathcal N$, is the smallest relation satisfying the rules
below.

\infrule[Out-barb]{y \in {\mathcal N}, \; x \nameeq y}
		  {\outputp{x}{v} \downarrow_{\mathcal N} x}
\infrule[Par-barb]{\mbox{$P\downarrow_{\mathcal N} x$ or $Q\downarrow_{\mathcal N} x$}}
		  {\binpar{P}{Q} \downarrow_{\mathcal N} x}

We write $P \Downarrow_{\mathcal N} x$ if there is $Q$ such that 
$P \wred Q$ and $Q \downarrow_{\mathcal N} x$.
\end{definition}

\begin{definition}
%\label{def.bbisim}
An  ${\mathcal N}$-\emph{barbed bisimulation} over a set of names, ${\mathcal N}$, is a symmetric binary relation 
${\mathcal S}_{\mathcal N}$ between agents such that $P\rel{S}_{\mathcal N}Q$ implies:
\begin{enumerate}
\item If $P \red P'$ then $Q \wred Q'$ and $P'\rel{S}_{\mathcal N} Q'$.
\item If $P\downarrow_{\mathcal N} x$, then $Q\Downarrow_{\mathcal N} x$.
\end{enumerate}
$P$ is ${\mathcal N}$-barbed bisimilar to $Q$, written
$P \wbbisim_{\mathcal N} Q$, if $P \rel{S}_{\mathcal N} Q$ for some ${\mathcal N}$-barbed bisimulation ${\mathcal S}_{\mathcal N}$.
\end{definition}

$\mathcal{R} \subseteq \pi \times \pi$

$P \mathcal{R} Q => \forall P'. P \red P' \Rightarrow \exists Q'. Q \red Q', P' \mathcal{R} Q'$

$P \vdash x \Rightarrow Q \vdash x$

\begin{mathpar}
  \inferrule*[lab=Out-barb]{x \nameeq y}{{y}!\langle{Q}\rangle \vdash x}
  \and
  \inferrule*[lab=Par-barb]{\mbox{$P\vdash x$ or $Q\vdash x$}}{\binpar{P}{Q} \vdash x}
\end{mathpar}

\subsubsection{Contexts}

One of the principle advantages of computational calculi like the
$\pi$-calculus is a well-defined notion of context,
contextual-equivalence and a correlation between
contextual-equivalence and notions of bisimulation. The notion of
context allows the decomposition of a process into (sub-)process and
its syntactic environment, its context. Thus, a context may be
thought of as a process with a ``hole'' (written $\Box$) in it. The
application of a context $M$ to a process $P$, written $M[P]$, is
tantamount to filling the hole in $M$ with $P$. In this paper we do
not need the full weight of this theory, but do make use of the notion
of context in the proof the main theorem. 

\begin{mathpar}
  \inferrule* [lab=summation] {} {{M_{M},M_{N}} \bc \Box \;|\; x.M_{A} \;|\; M_{M}+M_{N}}
  \and
  \inferrule* [lab=agent] {} {{M_{A}} \bc (\vec{x})M_{P} \;| \; \clift{P_0,\ldots,M_{P},\ldots,P_N}}
  \and \\
  \inferrule* [lab=process] {} {{M_{P}} \bc M_{N} \;| \;P|M_{P} }
\end{mathpar} 

\begin{mathpar}
  \inferrule* [lab=sychronization] {} {M_{N} \bc \Box \;|\; x?M_{F} \;|\; x!M_{C}}
  \and
  \inferrule* [lab=abstraction] {} {{M_{F}} \bc (x)M_{P} }
  \and
  \inferrule* [lab=concretion] {} {{M_{C}} \bc \langle M_{P} \rangle }
  \and \\
  \inferrule* [lab=process] {} {{M_{P}} \bc M_{N} \;| \;P|M_{P} }
\end{mathpar}

\begin{definition}[contextual application] Given a context $M$, and
  process $P$, we define the \emph{contextual application}, $M[P] :=
  M\{P/\Box\}$. That is, the contextual application of M to P is the
  substitution of $P$ for $\Box$ in $M$.
\end{definition}

$\meaningof{-} : L \to \mathcal{P}(\pi)$

\begin{mathpar}
  \inferrule* [lab=collection] {} {\meaningof{true} = \pi, \and \meaningof{~E} = \pi \setminus \meaningof{E}, \and \meaningof{E_{1} \& E_{2}} = \meaningof{E_{1}} \cap \meaningof{E_{2}}}
\end{mathpar}

\begin{mathpar}
  \inferrule* [lab=structure] {} {\meaningof{0} = \{ P \in \pi | P \equiv 0 \}, \and \\ \meaningof{E_1 | E_2} = \{ P \in \pi | P \equiv P_{1} | P_{2}, P_{1} \in \meaningof{E_{1}}, P_{2} \in \meaningof{E_2}\} }
\end{mathpar}

\begin{mathpar}
 \inferrule* [lab=behavior] {} {\meaningof{\langle a?b \rangle E} = \{ P \in \pi | P \equiv Q | u?(y)P', \\ \and \\\\ \and \\ \;\;\; u \in \meaningof{a}, \forall z.P'\{z/y\} \in \meaningof{E\{z/b\}}\}, \and \\ \meaningof{a!E} = \{ P \in \pi | P \equiv Q | x!\langle P' \rangle, x \in \meaningof{a} P' \in \meaningof{E}\} }
\end{mathpar}

\begin{mathpar}
 \inferrule* [lab=nominal] {} {\meaningof{\quotep{E}} = \{ \quotep{P} \in \quotep{\pi} | P \in \meaningof{E} \}, \and \meaningof{\quotep{P}} = \{ \quotep{Q} \in \quotep{\pi} | P \equiv Q \} \and \\ \meaningof{@\quotep{E}} = \{ P \in \pi | P \equiv @x, x \in \meaningof{E} \}}
\end{mathpar}

\begin{eqnarray*}
  \\
  \meaningof{-} : TS \to ST
\end{eqnarray*}

\begin{eqnarray*}
  \\
  L : TS \to ST
\end{eqnarray*}

\begin{eqnarray*}
  \\
  P \models E \iff P \in \meaningof{E}
\end{eqnarray*}

\begin{eqnarray*}
  P \approx_{L} Q \iff \forall E \in L. P \models E \iff Q \models E
\end{eqnarray*}

\begin{eqnarray*}
  P \approx_{K} Q
\end{eqnarray*}

\begin{eqnarray*}
  P \approx Q
\end{eqnarray*}

$\approx_{K} = \approx = \approx_{L}$

\subsubsection{Contextual duality}

Note that contexts extend the quotation operation to a family of
operations from processes to names. Given a context, $M$, we can
define a \emph{nominal context}, $\quotep{M}$ by $\quotep{M}[P] :=
\quotep{M[P]}$. To foreshadow what is to come we observe that these
operations enjoy a duality with processes very much like the duality
between vectors and maps from vectors to scalars.

Further, because the calculus is essentially higher-order, we have a
correspondence between contexts and processes. More specifically,
given a name $x$ and a context $M$ we can construct $M^{*}_{x}$ such
that 

\begin{mathpar}
  M^{*}_{x} | \lift{x}{P} \red M[P]
\end{mathpar}

namely,

\begin{mathpar}
  M^{*}_{x} := x?(u).M[\dropn{u}]
\end{mathpar}

The dependence of $M^{*}_{x}$ on a name makes it an abstraction, 

\begin{mathpar}
  M^{*} := (x)x?(u).M[\dropn{u}]
\end{mathpar}

\subsection{Additional notation}

It will sometimes be convenient to denote the process a name
quotes. We already have the notation $x = \quotep{P}$, but it will be
convenient to introduce an alternate notation, $\procn{x}$, when we
want to emphasize the connection to the use of the name. Note that, by
virtue of name equivalence, $\quotep{\procn{x}} \nameeq x$; so, the
notation is consistent with previous definitions.

Further, because names have structure it is possible to effect
substitutions on the basis of that structure. This means we need to
upgrade our notation for substitutions, which we accomplish by
adapting comprehension notation. Thus,

\begin{mathpar}
  P\{ y / x : x \in S \}
\end{mathpar}

is interpreted to mean the process derived from P by replacing (in a
capture-avoiding manner) each occurrence of $x$ in $S$ by $y$. For example,

\begin{mathpar}
  P\{ \quotep{\procn{x}|\procn{x}} / x : x \in \freenames{P} \}
\end{mathpar}

will replace each (occurrence) of a free name $x$ in $P$ by
$\quotep{\procn{x}|\procn{x}}$.

Also, we will avail ourselves of the notation $x^{L}$ and $x^{R}$ to
denote injections of a name into disjoint copies of the name
space. There are numerous ways to accomplish this. One example can be
found in \cite{MeredithR05}. This notation overloads to vectors of
names: $\vec{x}^{\pi} := (x_{i}^{\pi} \; : \; 0 \leq i < |\vec{x}| )$ where $\pi \in \{L,R\}$.

We also use $P^{\Box} := P|\Box$.

In \cite{MeredithR05} an interpretation of the new operator is
given. It turns out that there are several possible interpretations
all enjoying the requisite algebraic properties of the operator (see
\cite{milner91polyadicpi}). We will therefore make liberal use of
$(\nu\; \vec{x})P$.

% subsection the_syntax_and_semantics_of_the_notation_system (end)   

\input{qm2pi.qmops} 

\input{qm2pi.sterngerlach} 

\input{qm2pi.metric} 

% section concurrent_process_calculi (end)

%\input{qm2pi.proofsketch}

% section proof sketch (end)

%\input{qm2pi.slviaknots} 

% section spatial logic via knots (end)

\input{qm2pi.conclusion}

% section conclusion (end)

%\input{qm2pi.dtcodes} 

% section wiring algorithm (end)

\input{qm2pi.ack} 

% section acknowledgments (end)

\newpage


\bibliographystyle{plain}   
\bibliography{../../biblios/main.bib}

\input{qm2pi.rhodetails}

\end{document}

 

% section notation (end)

\input{qm2pi.process.calculi} 

% section concurrent_process_calculi_and_spatial_logics_ (end)
    
%\documentclass[12pt]{llncs}
%\documentclass{jktr}

\usepackage[pdftex]{hyperref}                   
\usepackage {listings}
\usepackage {mathpartir}
\usepackage{bcprules}
%\usepackage{listings}
                       
\usepackage{graphicx} 
%\usepackage[margins=2.5cm,nohead,nofoot]{geometry}
%\usepackage{geometry}
\usepackage{amsfonts}
\usepackage{amstext}
\usepackage{latexsym}
\usepackage{amssymb}
\usepackage{color}


%\include{myPreamble}
\include{qm2pi.local} 

%\ifpdf
%\usepackage[pdftex]{graphicx}
%\else
%\usepackage{graphicx}
%\fi

 % \ifpdf
%  \usepackage{pdfsync}
%  \if


%\title{Brief Article}
%\author{David F. Snyder}
%\author{L.G. Meredith}

%\address{Dept. of Math., Texas State University--San Marcos, San Marcos, TX 78666}
       
\pagestyle{empty}


\begin{document}

\lstset{language=[Objective]Caml,frame=shadowbox}

\input{qm2pi.front}

% section front matter (end)

\input{qm2pi.intro} 
 
% section introduction (end)

% \input{qm2pi.knotations} 

% section notation (end)

\input{qm2pi.process.calculi} 

% section concurrent_process_calculi_and_spatial_logics_ (end)
    
%\input{qm2pi.knots2pi} 

%\input{qm2pi.trefoil} 

%\input{qm2pi.mainthm} 

% subsection basic_interpretation (end)

%\input{qm2pi.rho.presentation} 
\subsection{The syntax and semantics of the notation system}\label{sub:the_syntax_and_semantics_of_the_notation_system} % (fold)

We now summarize a technical presentation of the calculus that
embodies our theory of dynamics. The typical presentation of such a
calculus follows the style of giving generators and relations on
them. The grammar, below, describing term constructors, freely
generates the set of processes, $\Proc$. This set is then quotiented
by a relation known as structural congruence and it is over this set
that the notion of dynamics is expressed. This presentation is
essentially that of \cite{MeredithR05} with the addition of
polyadicity and summation. For readability we have relegated some of
the technical subtleties to an appendix.

\subsubsection{Process grammar}\label{subsub:process_grammar}

\begin{mathpar}
  \inferrule* [lab=synchronization] {} {{M} \bc \pzero \;|\; x?F \;|\; x!C }
  \and
  \inferrule* [lab=abstraction] {} {{F} \bc (x)P}
  \and
  \inferrule* [lab=concretion] {} {{C} \bc \langle Q \rangle}
  \and
  \inferrule* [lab=process] {} {{P,Q} \bc M \;| \;P|Q \;|\; @{x}}
  \and
  \inferrule* [lab=name] {} {{x} \bc \quotep{P}}
\end{mathpar} 

Note that $\vec{x}$ (resp. $\vec{P}$) denotes a vector of names
(resp. processes) of length $|\vec{x}|$ (resp. $|\vec{P}|$). We adopt
the following useful abbreviations.

\begin{mathpar}
   x?(\vec{y}).P := x.(\vec{y})P \and  x\clift{\vec{P}} := x.\clift{\vec{P}}
   \and x!(y) := \lift{x}{\dropn{y}}
   \and \Pi_{i=0}^{n-1}P_i := P_0 | \ldots | P_{n-1}
\end{mathpar}

\subsubsection{Structural congruence}

\paragraph{Free and bound names and alpha-equivalence.} At the
core of structural equivalence is alpha-equivalence which identifies
process that are the same up to a change of variable. Formally, we
recognize the distinction between free and bound names. The free names
of a process, $\freenames{P}$, may be calculated recursively as
follows:

\begin{mathpar}
\freenames{\pzero} := \emptyset
  \and \\
  \freenames{x?(y).P} := \{ x \} \cup (\freenames{P} \setminus \{ y \})
  \and 
  \freenames{x!\langle P \rangle} := \{ x \} \cup \{ P \} 
  \and \\
  \freenames{P|Q} := \freenames{P} \cup \freenames{Q}
  \and \\
  \freenames{@{x}} := \{ x \}
\end{mathpar}

$\pi$
$\quotep{\pi}$

$\freenames{-} : \pi \to \mathcal{P}(\quotep{\pi})$

\begin{eqnarray*}
  \freenames{\pzero} & := & \emptyset \\
  \freenames{x?(y).P} & := & \{ x \} \cup (\freenames{P} \setminus \{ y \}) \\
  \freenames{x!\langle P \rangle} & := & \{ x \} \cup \{ P \} \\
  \freenames{P|Q} & := & \freenames{P} \cup \freenames{Q} \\
  \freenames{\dropn{x}} & := & \{ x \}
\end{eqnarray*}

The bound names of a process, $\boundnames{P}$, are those names occurring in $P$
that are not free. For example, in $x?(y).0$, the name $x$ is free, while $y$ is bound.

\begin{mathpar}
  \inferrule* [lab=monoidal-laws] {} { P|Q \equiv Q|P \and P|0 \equiv P \and P|(Q|R) \equiv (P|Q)|R }
\end{mathpar}

\begin{mathpar}
  \inferrule* [lab=alpha-equivalence] {} { (x)P \equiv (y)P\{y/x\} \and y \not\in \freenames{P} }
\end{mathpar}

\begin{definition}
Then two processes, $P,Q$, are alpha-equivalent if $P = Q\{\vec{y}/\vec{x}\}$ for
some $\vec{x} \in \boundnames{Q},\vec{y} \in \boundnames{P}$, where $Q\{\vec{y}/\vec{x}\}$
denotes the capture-avoiding substitution of $\vec{y}$ for $\vec{x}$ in $Q$.
\end{definition}

\begin{definition}
  The {\em structural congruence} \cite{SangiorgiWalker} , $\equiv$,
  between processes is the least congruence containing
  alpha-equivalence, satisfying the abelian monoid laws
  (associativity, commutativity and $\pzero$ as identity) for parallel
  composition $|$ and for summation $+$.
\end{definition}

\subsection{Name equivalence}

We take name equivalence, written $\nameeq$, to be the smallest
equivalence relation generated by the following rules.

\begin{mathpar}
\inferrule*[lab=Quote-drop]
{ }
{ \quotep{@{x}} \nameeq x }

\inferrule*[lab=Struct-equiv]
{ P \scong Q }
{ \quotep{P} \nameeq \quotep{Q} }
\end{mathpar}

The astute reader will have noticed that the mutual recursion of names
and processes imposes a mutual recursion on alpha-equivalence and
structural equivalence via name-equivalence. Fortunately, all of this
works out pleasantly and we may calculate in the natural way, free of
concern. The reader interested in the details is referred to the
appendix \ref{appendix:rho_details}.

\subsection{Substitution}

We use $\Proc$ for the set of processes, $\QProc$ for the set of
names, and $\id{\{}\vec{y} / \vec{x} \id{\}}$ to denote partial maps,
$s : \QProc \rightarrow \QProc$. A map, $s$ lifts, uniquely, to a map
on process terms, $\widehat{s} : \Proc \rightarrow \Proc$ by the
following equations.

\begin{mathpar}
  (0) \psubstp{Q}{P} := 0 \\
  (R \juxtap S) \psubstp{Q}{P}
  :=    
  (R)\psubstp{Q}{P} \juxtap (S) \psubstp{Q}{P} \\
  (x?(y).R) \psubstp{Q}{P}    
  :=    
  (x)\substp{Q}{P} (z)\concat( (R \psubstn{z}{y}) \psubstp{Q}{P} ) \\
  (\lift{x}{R}) \psubstp{Q}{P}  
  :=
  \lift{(x)\substp{Q}{P}}{ R \psubstp{Q}{P} } \\
%   (\dropn{x})  \psubstp{Q}{P}       
%   := 
%   \left\{ 
%     \begin{array}{ccc} 
%       \dropn{\quotep{Q}} & & x \nameeq \quotep{P} \\
%       \dropn{x} & & otherwise \\
%     \end{array}
%   \right. 
  (\dropn{x})  \psubstp{Q}{P}       
  := 
  \left\{ 
    \begin{array}{ccc} 
      Q & & x \nameeq \quotep{P} \\
      \dropn{x} & & otherwise \\
    \end{array}
  \right.
\end{mathpar}
 

where

\begin{eqnarray}
  (x)\id{\{} \lpquote Q \rpquote / \lpquote P \rpquote \id{\}}            = 
  \left\{ 
    \begin{array}{ccc}
      \lpquote Q \rpquote & & x \nameeq \lpquote P \rpquote \\
      x & & otherwise \\
    \end{array}
  \right. \nonumber
\end{eqnarray}

and $z$ is chosen distinct from $\quotep{P}$, $\quotep{Q}$, the free
names in $Q$, and all the names in $R$. Our $\alpha$-equivalence will
be built in the standard way from this substitution.

\begin{remark}\label{rem:no_self_referential_names}
  One consequence of these definitions is that $\forall P. \quotep{P}
  \not\in \freenames{P}$.
\end{remark}

\subsection{ Dynamic quote: an example }

Anticipating something of what's to come, consider applying the
substitution, $\widehat{\id{\{}u / z \id{\}}}$, to the following pair
of processes, $\lift{w}{y!(z)}$ and $w[ \lpquote y!(z) \rpquote ]$.

\begin{eqnarray}
	\lift{w}{y!(z)}\widehat{\id{\{}u / z \id{\}}}
		& = &
		\lift{w}{y!(u)} \nonumber\\
	w[ \lpquote y!(z) \rpquote ] \widehat{ \id{\{}u / z \id{\}} }
		& = &
		w[ \lpquote y!(z) \rpquote ] \nonumber
\end{eqnarray}

Because the body of the process between quotes is impervious to
substitution, we get radically different answers. In fact, by
examining the first process in an input context,
e.g. $x?(z).\lift{w}{y!(z)}$, we see that the process under the lift
operator may be shaped by prefixed inputs binding a name inside it. In
this sense, the lift operator will be seen as a way to dynamically
construct processes before reifying them as names.

Finally equipped with these standard features we can present the
dynamics of the calculus.

\subsubsection{Operational semantics} 

Finally, we introduce the computational dynamics. What marks these
algebras as distinct from other more traditionally studied algebraic
structures, e.g. vector spaces or polynomial rings, is the manner in
which dynamics is captured. In traditional structures, dynamics is typically
expressed through morphisms between such structures, as in linear maps
between vector spaces or morphisms between rings. In algebras
associated with the semantics of computation, the dynamics is
expressed as part of the algebraic structure itself, through a
reduction reduction relation typically denoted by $\red$. Below, we
give a recursive presentation of this relation for the calculus used
in the encoding.

$\red \subseteq \pi \times \pi$
$\red : \pi \to \mathcal{P}(\pi)$

\begin{mathpar}
  \inferrule* [lab=Comm] { \textsf{match}( x_{src}, x_{trgt} ) } { x_{trgt}?(y)P \; | \; x_{src}!\langle {Q} \rangle \red P\{\quotep{Q}/y}\} }
  \and \\
  \inferrule* [lab=Par] {{P} \red {P}'} {{{P} | {Q}} \red {{P}' | {Q}}}
  \and
  \inferrule* [lab=Equiv]{{{P} \scong {P}'} \andalso {{P}' \red {Q}'} \andalso {{Q}' \scong {Q}}}{{P} \red {Q}}
\end{mathpar}

\begin{eqnarray*}
  match_{\equiv} (\quotep{P},\quotep{Q}) & := & P \equiv Q \\
  match_{\dagger}(\quotep{P},\quotep{Q}) & := & \forall R. P|Q \red^{*} R => R \red^{*} 0 \\
  match_{K}(\quotep{P},\quotep{Q}) & := & K \mbox{ for some context } K
\end{eqnarray*}

$u?(x)P | u!\langle Q \rangle \red P\{\quotep{Q}/x\}$

%We write $\wred$ for $\red^*$, and $P\red$ if $\exists Q $ such that $ P \red Q$.
We write $P\red$ if $\exists Q $ such that $ P \red Q$ and $P\not\red$, otherwise.

\section{Replication}

As mentioned before, it is known that replication (and hence
recursion) can be implemented in a higher-order process algebra
\cite{SangiorgiWalker}. As our first example of calculation with the
machinery thus far presented we give the construction explicitly in
the {\rhoc}.

\begin{eqnarray}
	D_{x} & := & \prefix{x}{y}{(\binpar{\outputp{x}{y}}{@{y}})} \nonumber\\
	\bangp_{x}{P} & := & \binpar{{x}!\langle{\binpar{D_{x}}{P}}\rangle}{D_{x}} \nonumber
\end{eqnarray}

\begin{eqnarray}
	\bangp_{x}{P} & & \nonumber\\
	=
	& {x}!\langle{(\prefix{x}{y}{(\outputp{x}{y} | @{y})) | P}}\rangle 
	      | \prefix{x}{y}{(\outputp{x}{y} | @{y})} & \nonumber\\
	\red
	& (\outputp{x}{y} | @{y})\substn{\quotep{(\prefix{x}{y}{(@{y} | \outputp{x}{y})) | P}}}{y} & \nonumber\\
	=
	& \outputp{x}{\quotep{(\prefix{x}{y}{(\outputp{x}{y} | @{y})) | P}}}
	  | {(\prefix{x}{y}{(\outputp{x}{y} | @{y})) | P}} & \nonumber\\
	\red
	& \ldots & \nonumber\\
	\red^*
	& P | P | \ldots & \nonumber
\end{eqnarray}

Of course, this encoding, as an implementation, runs away, unfolding
$\bangp{P}$ eagerly. A lazier and more implementable replication
operator, restricted to input-guarded processes, may be obtained as follows.

\begin{eqnarray}
\bangp{\prefix{u}{v}{P}} 
	:= 
	\binpar{\lift{x}{\prefix{u}{v}{(\binpar{D(x)}{P})}}}{D(x)} \nonumber
\end{eqnarray}

\begin{remark}
  Note that the lazier definition still does not deal with summation
  or mixed summation (i.e. sums over input and output). The reader is
  invited to construct definitions of replication that deal with these
  features. 

  Further, the definitions are parameterized in a name, $x$. Can you,
  gentle reader, make a definition that eliminates this parameter and
  guarantees no accidental interaction between the replication
  machinery and the process being replicated -- i.e. no accidental
  sharing of names used by the process to get its work done and the
  name(s) used by the replication to effect copying. This latter
  revision of the definition of replication is crucial to obtaining
  the expected identity $!!P \sim !P$.
\end{remark}

\begin{remark}\label{rem:paradoxical_combinator}
  The reader familiar with the lambda calculus will have noticed the
  similarity between $D$ and the paradoxical combinator.

  [Ed. note: the existence of this seems to suggest we have to be more
  restrictive on the set of processes and names we admit if we are to
  support no-cloning.]
\end{remark}

\subsubsection{Bisimulation}

The computational dynamics gives rise to another kind of equivalence,
the equivalence of computational behavior. As previously mentioned
this is typically captured \emph{via} some form of bisimulation.

% The notion we use in this paper is weak barbed bisimulation
% \cite{milner91polyadicpi}.

The notion we use in this paper is derived from weak barbed
bisimulation \cite{milner91polyadicpi}. 

\begin{definition}
An \emph{observation relation}, $\downarrow_{\mathcal N}$, over a set
of names, $\mathcal N$, is the smallest relation satisfying the rules
below.

\infrule[Out-barb]{y \in {\mathcal N}, \; x \nameeq y}
		  {\outputp{x}{v} \downarrow_{\mathcal N} x}
\infrule[Par-barb]{\mbox{$P\downarrow_{\mathcal N} x$ or $Q\downarrow_{\mathcal N} x$}}
		  {\binpar{P}{Q} \downarrow_{\mathcal N} x}

We write $P \Downarrow_{\mathcal N} x$ if there is $Q$ such that 
$P \wred Q$ and $Q \downarrow_{\mathcal N} x$.
\end{definition}

\begin{definition}
%\label{def.bbisim}
An  ${\mathcal N}$-\emph{barbed bisimulation} over a set of names, ${\mathcal N}$, is a symmetric binary relation 
${\mathcal S}_{\mathcal N}$ between agents such that $P\rel{S}_{\mathcal N}Q$ implies:
\begin{enumerate}
\item If $P \red P'$ then $Q \wred Q'$ and $P'\rel{S}_{\mathcal N} Q'$.
\item If $P\downarrow_{\mathcal N} x$, then $Q\Downarrow_{\mathcal N} x$.
\end{enumerate}
$P$ is ${\mathcal N}$-barbed bisimilar to $Q$, written
$P \wbbisim_{\mathcal N} Q$, if $P \rel{S}_{\mathcal N} Q$ for some ${\mathcal N}$-barbed bisimulation ${\mathcal S}_{\mathcal N}$.
\end{definition}

$\mathcal{R} \subseteq \pi \times \pi$

$P \mathcal{R} Q => \forall P'. P \red P' \Rightarrow \exists Q'. Q \red Q', P' \mathcal{R} Q'$

$P \vdash x \Rightarrow Q \vdash x$

\begin{mathpar}
  \inferrule*[lab=Out-barb]{x \nameeq y}{{y}!\langle{Q}\rangle \vdash x}
  \and
  \inferrule*[lab=Par-barb]{\mbox{$P\vdash x$ or $Q\vdash x$}}{\binpar{P}{Q} \vdash x}
\end{mathpar}

\subsubsection{Contexts}

One of the principle advantages of computational calculi like the
$\pi$-calculus is a well-defined notion of context,
contextual-equivalence and a correlation between
contextual-equivalence and notions of bisimulation. The notion of
context allows the decomposition of a process into (sub-)process and
its syntactic environment, its context. Thus, a context may be
thought of as a process with a ``hole'' (written $\Box$) in it. The
application of a context $M$ to a process $P$, written $M[P]$, is
tantamount to filling the hole in $M$ with $P$. In this paper we do
not need the full weight of this theory, but do make use of the notion
of context in the proof the main theorem. 

\begin{mathpar}
  \inferrule* [lab=summation] {} {{M_{M},M_{N}} \bc \Box \;|\; x.M_{A} \;|\; M_{M}+M_{N}}
  \and
  \inferrule* [lab=agent] {} {{M_{A}} \bc (\vec{x})M_{P} \;| \; \clift{P_0,\ldots,M_{P},\ldots,P_N}}
  \and \\
  \inferrule* [lab=process] {} {{M_{P}} \bc M_{N} \;| \;P|M_{P} }
\end{mathpar} 

\begin{mathpar}
  \inferrule* [lab=sychronization] {} {M_{N} \bc \Box \;|\; x?M_{F} \;|\; x!M_{C}}
  \and
  \inferrule* [lab=abstraction] {} {{M_{F}} \bc (x)M_{P} }
  \and
  \inferrule* [lab=concretion] {} {{M_{C}} \bc \langle M_{P} \rangle }
  \and \\
  \inferrule* [lab=process] {} {{M_{P}} \bc M_{N} \;| \;P|M_{P} }
\end{mathpar}

\begin{definition}[contextual application] Given a context $M$, and
  process $P$, we define the \emph{contextual application}, $M[P] :=
  M\{P/\Box\}$. That is, the contextual application of M to P is the
  substitution of $P$ for $\Box$ in $M$.
\end{definition}

$\meaningof{-} : L \to \mathcal{P}(\pi)$

\begin{mathpar}
  \inferrule* [lab=collection] {} {\meaningof{true} = \pi, \and \meaningof{~E} = \pi \setminus \meaningof{E}, \and \meaningof{E_{1} \& E_{2}} = \meaningof{E_{1}} \cap \meaningof{E_{2}}}
\end{mathpar}

\begin{mathpar}
  \inferrule* [lab=structure] {} {\meaningof{0} = \{ P \in \pi | P \equiv 0 \}, \and \\ \meaningof{E_1 | E_2} = \{ P \in \pi | P \equiv P_{1} | P_{2}, P_{1} \in \meaningof{E_{1}}, P_{2} \in \meaningof{E_2}\} }
\end{mathpar}

\begin{mathpar}
 \inferrule* [lab=behavior] {} {\meaningof{\langle a?b \rangle E} = \{ P \in \pi | P \equiv Q | u?(y)P', \\ \and \\\\ \and \\ \;\;\; u \in \meaningof{a}, \forall z.P'\{z/y\} \in \meaningof{E\{z/b\}}\}, \and \\ \meaningof{a!E} = \{ P \in \pi | P \equiv Q | x!\langle P' \rangle, x \in \meaningof{a} P' \in \meaningof{E}\} }
\end{mathpar}

\begin{mathpar}
 \inferrule* [lab=nominal] {} {\meaningof{\quotep{E}} = \{ \quotep{P} \in \quotep{\pi} | P \in \meaningof{E} \}, \and \meaningof{\quotep{P}} = \{ \quotep{Q} \in \quotep{\pi} | P \equiv Q \} \and \\ \meaningof{@\quotep{E}} = \{ P \in \pi | P \equiv @x, x \in \meaningof{E} \}}
\end{mathpar}

\begin{eqnarray*}
  \\
  \meaningof{-} : TS \to ST
\end{eqnarray*}

\begin{eqnarray*}
  \\
  L : TS \to ST
\end{eqnarray*}

\begin{eqnarray*}
  \\
  P \models E \iff P \in \meaningof{E}
\end{eqnarray*}

\begin{eqnarray*}
  P \approx_{L} Q \iff \forall E \in L. P \models E \iff Q \models E
\end{eqnarray*}

\begin{eqnarray*}
  P \approx_{K} Q
\end{eqnarray*}

\begin{eqnarray*}
  P \approx Q
\end{eqnarray*}

$\approx_{K} = \approx = \approx_{L}$

\subsubsection{Contextual duality}

Note that contexts extend the quotation operation to a family of
operations from processes to names. Given a context, $M$, we can
define a \emph{nominal context}, $\quotep{M}$ by $\quotep{M}[P] :=
\quotep{M[P]}$. To foreshadow what is to come we observe that these
operations enjoy a duality with processes very much like the duality
between vectors and maps from vectors to scalars.

Further, because the calculus is essentially higher-order, we have a
correspondence between contexts and processes. More specifically,
given a name $x$ and a context $M$ we can construct $M^{*}_{x}$ such
that 

\begin{mathpar}
  M^{*}_{x} | \lift{x}{P} \red M[P]
\end{mathpar}

namely,

\begin{mathpar}
  M^{*}_{x} := x?(u).M[\dropn{u}]
\end{mathpar}

The dependence of $M^{*}_{x}$ on a name makes it an abstraction, 

\begin{mathpar}
  M^{*} := (x)x?(u).M[\dropn{u}]
\end{mathpar}

\subsection{Additional notation}

It will sometimes be convenient to denote the process a name
quotes. We already have the notation $x = \quotep{P}$, but it will be
convenient to introduce an alternate notation, $\procn{x}$, when we
want to emphasize the connection to the use of the name. Note that, by
virtue of name equivalence, $\quotep{\procn{x}} \nameeq x$; so, the
notation is consistent with previous definitions.

Further, because names have structure it is possible to effect
substitutions on the basis of that structure. This means we need to
upgrade our notation for substitutions, which we accomplish by
adapting comprehension notation. Thus,

\begin{mathpar}
  P\{ y / x : x \in S \}
\end{mathpar}

is interpreted to mean the process derived from P by replacing (in a
capture-avoiding manner) each occurrence of $x$ in $S$ by $y$. For example,

\begin{mathpar}
  P\{ \quotep{\procn{x}|\procn{x}} / x : x \in \freenames{P} \}
\end{mathpar}

will replace each (occurrence) of a free name $x$ in $P$ by
$\quotep{\procn{x}|\procn{x}}$.

Also, we will avail ourselves of the notation $x^{L}$ and $x^{R}$ to
denote injections of a name into disjoint copies of the name
space. There are numerous ways to accomplish this. One example can be
found in \cite{MeredithR05}. This notation overloads to vectors of
names: $\vec{x}^{\pi} := (x_{i}^{\pi} \; : \; 0 \leq i < |\vec{x}| )$ where $\pi \in \{L,R\}$.

We also use $P^{\Box} := P|\Box$.

In \cite{MeredithR05} an interpretation of the new operator is
given. It turns out that there are several possible interpretations
all enjoying the requisite algebraic properties of the operator (see
\cite{milner91polyadicpi}). We will therefore make liberal use of
$(\nu\; \vec{x})P$.

% subsection the_syntax_and_semantics_of_the_notation_system (end)   

\input{qm2pi.qmops} 

\input{qm2pi.sterngerlach} 

\input{qm2pi.metric} 

% section concurrent_process_calculi (end)

%\input{qm2pi.proofsketch}

% section proof sketch (end)

%\input{qm2pi.slviaknots} 

% section spatial logic via knots (end)

\input{qm2pi.conclusion}

% section conclusion (end)

%\input{qm2pi.dtcodes} 

% section wiring algorithm (end)

\input{qm2pi.ack} 

% section acknowledgments (end)

\newpage


\bibliographystyle{plain}   
\bibliography{../../biblios/main.bib}

\input{qm2pi.rhodetails}

\end{document}

 

%\documentclass[12pt]{llncs}
%\documentclass{jktr}

\usepackage[pdftex]{hyperref}                   
\usepackage {listings}
\usepackage {mathpartir}
\usepackage{bcprules}
%\usepackage{listings}
                       
\usepackage{graphicx} 
%\usepackage[margins=2.5cm,nohead,nofoot]{geometry}
%\usepackage{geometry}
\usepackage{amsfonts}
\usepackage{amstext}
\usepackage{latexsym}
\usepackage{amssymb}
\usepackage{color}


%\include{myPreamble}
\include{qm2pi.local} 

%\ifpdf
%\usepackage[pdftex]{graphicx}
%\else
%\usepackage{graphicx}
%\fi

 % \ifpdf
%  \usepackage{pdfsync}
%  \if


%\title{Brief Article}
%\author{David F. Snyder}
%\author{L.G. Meredith}

%\address{Dept. of Math., Texas State University--San Marcos, San Marcos, TX 78666}
       
\pagestyle{empty}


\begin{document}

\lstset{language=[Objective]Caml,frame=shadowbox}

\input{qm2pi.front}

% section front matter (end)

\input{qm2pi.intro} 
 
% section introduction (end)

% \input{qm2pi.knotations} 

% section notation (end)

\input{qm2pi.process.calculi} 

% section concurrent_process_calculi_and_spatial_logics_ (end)
    
%\input{qm2pi.knots2pi} 

%\input{qm2pi.trefoil} 

%\input{qm2pi.mainthm} 

% subsection basic_interpretation (end)

%\input{qm2pi.rho.presentation} 
\subsection{The syntax and semantics of the notation system}\label{sub:the_syntax_and_semantics_of_the_notation_system} % (fold)

We now summarize a technical presentation of the calculus that
embodies our theory of dynamics. The typical presentation of such a
calculus follows the style of giving generators and relations on
them. The grammar, below, describing term constructors, freely
generates the set of processes, $\Proc$. This set is then quotiented
by a relation known as structural congruence and it is over this set
that the notion of dynamics is expressed. This presentation is
essentially that of \cite{MeredithR05} with the addition of
polyadicity and summation. For readability we have relegated some of
the technical subtleties to an appendix.

\subsubsection{Process grammar}\label{subsub:process_grammar}

\begin{mathpar}
  \inferrule* [lab=synchronization] {} {{M} \bc \pzero \;|\; x?F \;|\; x!C }
  \and
  \inferrule* [lab=abstraction] {} {{F} \bc (x)P}
  \and
  \inferrule* [lab=concretion] {} {{C} \bc \langle Q \rangle}
  \and
  \inferrule* [lab=process] {} {{P,Q} \bc M \;| \;P|Q \;|\; @{x}}
  \and
  \inferrule* [lab=name] {} {{x} \bc \quotep{P}}
\end{mathpar} 

Note that $\vec{x}$ (resp. $\vec{P}$) denotes a vector of names
(resp. processes) of length $|\vec{x}|$ (resp. $|\vec{P}|$). We adopt
the following useful abbreviations.

\begin{mathpar}
   x?(\vec{y}).P := x.(\vec{y})P \and  x\clift{\vec{P}} := x.\clift{\vec{P}}
   \and x!(y) := \lift{x}{\dropn{y}}
   \and \Pi_{i=0}^{n-1}P_i := P_0 | \ldots | P_{n-1}
\end{mathpar}

\subsubsection{Structural congruence}

\paragraph{Free and bound names and alpha-equivalence.} At the
core of structural equivalence is alpha-equivalence which identifies
process that are the same up to a change of variable. Formally, we
recognize the distinction between free and bound names. The free names
of a process, $\freenames{P}$, may be calculated recursively as
follows:

\begin{mathpar}
\freenames{\pzero} := \emptyset
  \and \\
  \freenames{x?(y).P} := \{ x \} \cup (\freenames{P} \setminus \{ y \})
  \and 
  \freenames{x!\langle P \rangle} := \{ x \} \cup \{ P \} 
  \and \\
  \freenames{P|Q} := \freenames{P} \cup \freenames{Q}
  \and \\
  \freenames{@{x}} := \{ x \}
\end{mathpar}

$\pi$
$\quotep{\pi}$

$\freenames{-} : \pi \to \mathcal{P}(\quotep{\pi})$

\begin{eqnarray*}
  \freenames{\pzero} & := & \emptyset \\
  \freenames{x?(y).P} & := & \{ x \} \cup (\freenames{P} \setminus \{ y \}) \\
  \freenames{x!\langle P \rangle} & := & \{ x \} \cup \{ P \} \\
  \freenames{P|Q} & := & \freenames{P} \cup \freenames{Q} \\
  \freenames{\dropn{x}} & := & \{ x \}
\end{eqnarray*}

The bound names of a process, $\boundnames{P}$, are those names occurring in $P$
that are not free. For example, in $x?(y).0$, the name $x$ is free, while $y$ is bound.

\begin{mathpar}
  \inferrule* [lab=monoidal-laws] {} { P|Q \equiv Q|P \and P|0 \equiv P \and P|(Q|R) \equiv (P|Q)|R }
\end{mathpar}

\begin{mathpar}
  \inferrule* [lab=alpha-equivalence] {} { (x)P \equiv (y)P\{y/x\} \and y \not\in \freenames{P} }
\end{mathpar}

\begin{definition}
Then two processes, $P,Q$, are alpha-equivalent if $P = Q\{\vec{y}/\vec{x}\}$ for
some $\vec{x} \in \boundnames{Q},\vec{y} \in \boundnames{P}$, where $Q\{\vec{y}/\vec{x}\}$
denotes the capture-avoiding substitution of $\vec{y}$ for $\vec{x}$ in $Q$.
\end{definition}

\begin{definition}
  The {\em structural congruence} \cite{SangiorgiWalker} , $\equiv$,
  between processes is the least congruence containing
  alpha-equivalence, satisfying the abelian monoid laws
  (associativity, commutativity and $\pzero$ as identity) for parallel
  composition $|$ and for summation $+$.
\end{definition}

\subsection{Name equivalence}

We take name equivalence, written $\nameeq$, to be the smallest
equivalence relation generated by the following rules.

\begin{mathpar}
\inferrule*[lab=Quote-drop]
{ }
{ \quotep{@{x}} \nameeq x }

\inferrule*[lab=Struct-equiv]
{ P \scong Q }
{ \quotep{P} \nameeq \quotep{Q} }
\end{mathpar}

The astute reader will have noticed that the mutual recursion of names
and processes imposes a mutual recursion on alpha-equivalence and
structural equivalence via name-equivalence. Fortunately, all of this
works out pleasantly and we may calculate in the natural way, free of
concern. The reader interested in the details is referred to the
appendix \ref{appendix:rho_details}.

\subsection{Substitution}

We use $\Proc$ for the set of processes, $\QProc$ for the set of
names, and $\id{\{}\vec{y} / \vec{x} \id{\}}$ to denote partial maps,
$s : \QProc \rightarrow \QProc$. A map, $s$ lifts, uniquely, to a map
on process terms, $\widehat{s} : \Proc \rightarrow \Proc$ by the
following equations.

\begin{mathpar}
  (0) \psubstp{Q}{P} := 0 \\
  (R \juxtap S) \psubstp{Q}{P}
  :=    
  (R)\psubstp{Q}{P} \juxtap (S) \psubstp{Q}{P} \\
  (x?(y).R) \psubstp{Q}{P}    
  :=    
  (x)\substp{Q}{P} (z)\concat( (R \psubstn{z}{y}) \psubstp{Q}{P} ) \\
  (\lift{x}{R}) \psubstp{Q}{P}  
  :=
  \lift{(x)\substp{Q}{P}}{ R \psubstp{Q}{P} } \\
%   (\dropn{x})  \psubstp{Q}{P}       
%   := 
%   \left\{ 
%     \begin{array}{ccc} 
%       \dropn{\quotep{Q}} & & x \nameeq \quotep{P} \\
%       \dropn{x} & & otherwise \\
%     \end{array}
%   \right. 
  (\dropn{x})  \psubstp{Q}{P}       
  := 
  \left\{ 
    \begin{array}{ccc} 
      Q & & x \nameeq \quotep{P} \\
      \dropn{x} & & otherwise \\
    \end{array}
  \right.
\end{mathpar}
 

where

\begin{eqnarray}
  (x)\id{\{} \lpquote Q \rpquote / \lpquote P \rpquote \id{\}}            = 
  \left\{ 
    \begin{array}{ccc}
      \lpquote Q \rpquote & & x \nameeq \lpquote P \rpquote \\
      x & & otherwise \\
    \end{array}
  \right. \nonumber
\end{eqnarray}

and $z$ is chosen distinct from $\quotep{P}$, $\quotep{Q}$, the free
names in $Q$, and all the names in $R$. Our $\alpha$-equivalence will
be built in the standard way from this substitution.

\begin{remark}\label{rem:no_self_referential_names}
  One consequence of these definitions is that $\forall P. \quotep{P}
  \not\in \freenames{P}$.
\end{remark}

\subsection{ Dynamic quote: an example }

Anticipating something of what's to come, consider applying the
substitution, $\widehat{\id{\{}u / z \id{\}}}$, to the following pair
of processes, $\lift{w}{y!(z)}$ and $w[ \lpquote y!(z) \rpquote ]$.

\begin{eqnarray}
	\lift{w}{y!(z)}\widehat{\id{\{}u / z \id{\}}}
		& = &
		\lift{w}{y!(u)} \nonumber\\
	w[ \lpquote y!(z) \rpquote ] \widehat{ \id{\{}u / z \id{\}} }
		& = &
		w[ \lpquote y!(z) \rpquote ] \nonumber
\end{eqnarray}

Because the body of the process between quotes is impervious to
substitution, we get radically different answers. In fact, by
examining the first process in an input context,
e.g. $x?(z).\lift{w}{y!(z)}$, we see that the process under the lift
operator may be shaped by prefixed inputs binding a name inside it. In
this sense, the lift operator will be seen as a way to dynamically
construct processes before reifying them as names.

Finally equipped with these standard features we can present the
dynamics of the calculus.

\subsubsection{Operational semantics} 

Finally, we introduce the computational dynamics. What marks these
algebras as distinct from other more traditionally studied algebraic
structures, e.g. vector spaces or polynomial rings, is the manner in
which dynamics is captured. In traditional structures, dynamics is typically
expressed through morphisms between such structures, as in linear maps
between vector spaces or morphisms between rings. In algebras
associated with the semantics of computation, the dynamics is
expressed as part of the algebraic structure itself, through a
reduction reduction relation typically denoted by $\red$. Below, we
give a recursive presentation of this relation for the calculus used
in the encoding.

$\red \subseteq \pi \times \pi$
$\red : \pi \to \mathcal{P}(\pi)$

\begin{mathpar}
  \inferrule* [lab=Comm] { \textsf{match}( x_{src}, x_{trgt} ) } { x_{trgt}?(y)P \; | \; x_{src}!\langle {Q} \rangle \red P\{\quotep{Q}/y}\} }
  \and \\
  \inferrule* [lab=Par] {{P} \red {P}'} {{{P} | {Q}} \red {{P}' | {Q}}}
  \and
  \inferrule* [lab=Equiv]{{{P} \scong {P}'} \andalso {{P}' \red {Q}'} \andalso {{Q}' \scong {Q}}}{{P} \red {Q}}
\end{mathpar}

\begin{eqnarray*}
  match_{\equiv} (\quotep{P},\quotep{Q}) & := & P \equiv Q \\
  match_{\dagger}(\quotep{P},\quotep{Q}) & := & \forall R. P|Q \red^{*} R => R \red^{*} 0 \\
  match_{K}(\quotep{P},\quotep{Q}) & := & K \mbox{ for some context } K
\end{eqnarray*}

$u?(x)P | u!\langle Q \rangle \red P\{\quotep{Q}/x\}$

%We write $\wred$ for $\red^*$, and $P\red$ if $\exists Q $ such that $ P \red Q$.
We write $P\red$ if $\exists Q $ such that $ P \red Q$ and $P\not\red$, otherwise.

\section{Replication}

As mentioned before, it is known that replication (and hence
recursion) can be implemented in a higher-order process algebra
\cite{SangiorgiWalker}. As our first example of calculation with the
machinery thus far presented we give the construction explicitly in
the {\rhoc}.

\begin{eqnarray}
	D_{x} & := & \prefix{x}{y}{(\binpar{\outputp{x}{y}}{@{y}})} \nonumber\\
	\bangp_{x}{P} & := & \binpar{{x}!\langle{\binpar{D_{x}}{P}}\rangle}{D_{x}} \nonumber
\end{eqnarray}

\begin{eqnarray}
	\bangp_{x}{P} & & \nonumber\\
	=
	& {x}!\langle{(\prefix{x}{y}{(\outputp{x}{y} | @{y})) | P}}\rangle 
	      | \prefix{x}{y}{(\outputp{x}{y} | @{y})} & \nonumber\\
	\red
	& (\outputp{x}{y} | @{y})\substn{\quotep{(\prefix{x}{y}{(@{y} | \outputp{x}{y})) | P}}}{y} & \nonumber\\
	=
	& \outputp{x}{\quotep{(\prefix{x}{y}{(\outputp{x}{y} | @{y})) | P}}}
	  | {(\prefix{x}{y}{(\outputp{x}{y} | @{y})) | P}} & \nonumber\\
	\red
	& \ldots & \nonumber\\
	\red^*
	& P | P | \ldots & \nonumber
\end{eqnarray}

Of course, this encoding, as an implementation, runs away, unfolding
$\bangp{P}$ eagerly. A lazier and more implementable replication
operator, restricted to input-guarded processes, may be obtained as follows.

\begin{eqnarray}
\bangp{\prefix{u}{v}{P}} 
	:= 
	\binpar{\lift{x}{\prefix{u}{v}{(\binpar{D(x)}{P})}}}{D(x)} \nonumber
\end{eqnarray}

\begin{remark}
  Note that the lazier definition still does not deal with summation
  or mixed summation (i.e. sums over input and output). The reader is
  invited to construct definitions of replication that deal with these
  features. 

  Further, the definitions are parameterized in a name, $x$. Can you,
  gentle reader, make a definition that eliminates this parameter and
  guarantees no accidental interaction between the replication
  machinery and the process being replicated -- i.e. no accidental
  sharing of names used by the process to get its work done and the
  name(s) used by the replication to effect copying. This latter
  revision of the definition of replication is crucial to obtaining
  the expected identity $!!P \sim !P$.
\end{remark}

\begin{remark}\label{rem:paradoxical_combinator}
  The reader familiar with the lambda calculus will have noticed the
  similarity between $D$ and the paradoxical combinator.

  [Ed. note: the existence of this seems to suggest we have to be more
  restrictive on the set of processes and names we admit if we are to
  support no-cloning.]
\end{remark}

\subsubsection{Bisimulation}

The computational dynamics gives rise to another kind of equivalence,
the equivalence of computational behavior. As previously mentioned
this is typically captured \emph{via} some form of bisimulation.

% The notion we use in this paper is weak barbed bisimulation
% \cite{milner91polyadicpi}.

The notion we use in this paper is derived from weak barbed
bisimulation \cite{milner91polyadicpi}. 

\begin{definition}
An \emph{observation relation}, $\downarrow_{\mathcal N}$, over a set
of names, $\mathcal N$, is the smallest relation satisfying the rules
below.

\infrule[Out-barb]{y \in {\mathcal N}, \; x \nameeq y}
		  {\outputp{x}{v} \downarrow_{\mathcal N} x}
\infrule[Par-barb]{\mbox{$P\downarrow_{\mathcal N} x$ or $Q\downarrow_{\mathcal N} x$}}
		  {\binpar{P}{Q} \downarrow_{\mathcal N} x}

We write $P \Downarrow_{\mathcal N} x$ if there is $Q$ such that 
$P \wred Q$ and $Q \downarrow_{\mathcal N} x$.
\end{definition}

\begin{definition}
%\label{def.bbisim}
An  ${\mathcal N}$-\emph{barbed bisimulation} over a set of names, ${\mathcal N}$, is a symmetric binary relation 
${\mathcal S}_{\mathcal N}$ between agents such that $P\rel{S}_{\mathcal N}Q$ implies:
\begin{enumerate}
\item If $P \red P'$ then $Q \wred Q'$ and $P'\rel{S}_{\mathcal N} Q'$.
\item If $P\downarrow_{\mathcal N} x$, then $Q\Downarrow_{\mathcal N} x$.
\end{enumerate}
$P$ is ${\mathcal N}$-barbed bisimilar to $Q$, written
$P \wbbisim_{\mathcal N} Q$, if $P \rel{S}_{\mathcal N} Q$ for some ${\mathcal N}$-barbed bisimulation ${\mathcal S}_{\mathcal N}$.
\end{definition}

$\mathcal{R} \subseteq \pi \times \pi$

$P \mathcal{R} Q => \forall P'. P \red P' \Rightarrow \exists Q'. Q \red Q', P' \mathcal{R} Q'$

$P \vdash x \Rightarrow Q \vdash x$

\begin{mathpar}
  \inferrule*[lab=Out-barb]{x \nameeq y}{{y}!\langle{Q}\rangle \vdash x}
  \and
  \inferrule*[lab=Par-barb]{\mbox{$P\vdash x$ or $Q\vdash x$}}{\binpar{P}{Q} \vdash x}
\end{mathpar}

\subsubsection{Contexts}

One of the principle advantages of computational calculi like the
$\pi$-calculus is a well-defined notion of context,
contextual-equivalence and a correlation between
contextual-equivalence and notions of bisimulation. The notion of
context allows the decomposition of a process into (sub-)process and
its syntactic environment, its context. Thus, a context may be
thought of as a process with a ``hole'' (written $\Box$) in it. The
application of a context $M$ to a process $P$, written $M[P]$, is
tantamount to filling the hole in $M$ with $P$. In this paper we do
not need the full weight of this theory, but do make use of the notion
of context in the proof the main theorem. 

\begin{mathpar}
  \inferrule* [lab=summation] {} {{M_{M},M_{N}} \bc \Box \;|\; x.M_{A} \;|\; M_{M}+M_{N}}
  \and
  \inferrule* [lab=agent] {} {{M_{A}} \bc (\vec{x})M_{P} \;| \; \clift{P_0,\ldots,M_{P},\ldots,P_N}}
  \and \\
  \inferrule* [lab=process] {} {{M_{P}} \bc M_{N} \;| \;P|M_{P} }
\end{mathpar} 

\begin{mathpar}
  \inferrule* [lab=sychronization] {} {M_{N} \bc \Box \;|\; x?M_{F} \;|\; x!M_{C}}
  \and
  \inferrule* [lab=abstraction] {} {{M_{F}} \bc (x)M_{P} }
  \and
  \inferrule* [lab=concretion] {} {{M_{C}} \bc \langle M_{P} \rangle }
  \and \\
  \inferrule* [lab=process] {} {{M_{P}} \bc M_{N} \;| \;P|M_{P} }
\end{mathpar}

\begin{definition}[contextual application] Given a context $M$, and
  process $P$, we define the \emph{contextual application}, $M[P] :=
  M\{P/\Box\}$. That is, the contextual application of M to P is the
  substitution of $P$ for $\Box$ in $M$.
\end{definition}

$\meaningof{-} : L \to \mathcal{P}(\pi)$

\begin{mathpar}
  \inferrule* [lab=collection] {} {\meaningof{true} = \pi, \and \meaningof{~E} = \pi \setminus \meaningof{E}, \and \meaningof{E_{1} \& E_{2}} = \meaningof{E_{1}} \cap \meaningof{E_{2}}}
\end{mathpar}

\begin{mathpar}
  \inferrule* [lab=structure] {} {\meaningof{0} = \{ P \in \pi | P \equiv 0 \}, \and \\ \meaningof{E_1 | E_2} = \{ P \in \pi | P \equiv P_{1} | P_{2}, P_{1} \in \meaningof{E_{1}}, P_{2} \in \meaningof{E_2}\} }
\end{mathpar}

\begin{mathpar}
 \inferrule* [lab=behavior] {} {\meaningof{\langle a?b \rangle E} = \{ P \in \pi | P \equiv Q | u?(y)P', \\ \and \\\\ \and \\ \;\;\; u \in \meaningof{a}, \forall z.P'\{z/y\} \in \meaningof{E\{z/b\}}\}, \and \\ \meaningof{a!E} = \{ P \in \pi | P \equiv Q | x!\langle P' \rangle, x \in \meaningof{a} P' \in \meaningof{E}\} }
\end{mathpar}

\begin{mathpar}
 \inferrule* [lab=nominal] {} {\meaningof{\quotep{E}} = \{ \quotep{P} \in \quotep{\pi} | P \in \meaningof{E} \}, \and \meaningof{\quotep{P}} = \{ \quotep{Q} \in \quotep{\pi} | P \equiv Q \} \and \\ \meaningof{@\quotep{E}} = \{ P \in \pi | P \equiv @x, x \in \meaningof{E} \}}
\end{mathpar}

\begin{eqnarray*}
  \\
  \meaningof{-} : TS \to ST
\end{eqnarray*}

\begin{eqnarray*}
  \\
  L : TS \to ST
\end{eqnarray*}

\begin{eqnarray*}
  \\
  P \models E \iff P \in \meaningof{E}
\end{eqnarray*}

\begin{eqnarray*}
  P \approx_{L} Q \iff \forall E \in L. P \models E \iff Q \models E
\end{eqnarray*}

\begin{eqnarray*}
  P \approx_{K} Q
\end{eqnarray*}

\begin{eqnarray*}
  P \approx Q
\end{eqnarray*}

$\approx_{K} = \approx = \approx_{L}$

\subsubsection{Contextual duality}

Note that contexts extend the quotation operation to a family of
operations from processes to names. Given a context, $M$, we can
define a \emph{nominal context}, $\quotep{M}$ by $\quotep{M}[P] :=
\quotep{M[P]}$. To foreshadow what is to come we observe that these
operations enjoy a duality with processes very much like the duality
between vectors and maps from vectors to scalars.

Further, because the calculus is essentially higher-order, we have a
correspondence between contexts and processes. More specifically,
given a name $x$ and a context $M$ we can construct $M^{*}_{x}$ such
that 

\begin{mathpar}
  M^{*}_{x} | \lift{x}{P} \red M[P]
\end{mathpar}

namely,

\begin{mathpar}
  M^{*}_{x} := x?(u).M[\dropn{u}]
\end{mathpar}

The dependence of $M^{*}_{x}$ on a name makes it an abstraction, 

\begin{mathpar}
  M^{*} := (x)x?(u).M[\dropn{u}]
\end{mathpar}

\subsection{Additional notation}

It will sometimes be convenient to denote the process a name
quotes. We already have the notation $x = \quotep{P}$, but it will be
convenient to introduce an alternate notation, $\procn{x}$, when we
want to emphasize the connection to the use of the name. Note that, by
virtue of name equivalence, $\quotep{\procn{x}} \nameeq x$; so, the
notation is consistent with previous definitions.

Further, because names have structure it is possible to effect
substitutions on the basis of that structure. This means we need to
upgrade our notation for substitutions, which we accomplish by
adapting comprehension notation. Thus,

\begin{mathpar}
  P\{ y / x : x \in S \}
\end{mathpar}

is interpreted to mean the process derived from P by replacing (in a
capture-avoiding manner) each occurrence of $x$ in $S$ by $y$. For example,

\begin{mathpar}
  P\{ \quotep{\procn{x}|\procn{x}} / x : x \in \freenames{P} \}
\end{mathpar}

will replace each (occurrence) of a free name $x$ in $P$ by
$\quotep{\procn{x}|\procn{x}}$.

Also, we will avail ourselves of the notation $x^{L}$ and $x^{R}$ to
denote injections of a name into disjoint copies of the name
space. There are numerous ways to accomplish this. One example can be
found in \cite{MeredithR05}. This notation overloads to vectors of
names: $\vec{x}^{\pi} := (x_{i}^{\pi} \; : \; 0 \leq i < |\vec{x}| )$ where $\pi \in \{L,R\}$.

We also use $P^{\Box} := P|\Box$.

In \cite{MeredithR05} an interpretation of the new operator is
given. It turns out that there are several possible interpretations
all enjoying the requisite algebraic properties of the operator (see
\cite{milner91polyadicpi}). We will therefore make liberal use of
$(\nu\; \vec{x})P$.

% subsection the_syntax_and_semantics_of_the_notation_system (end)   

\input{qm2pi.qmops} 

\input{qm2pi.sterngerlach} 

\input{qm2pi.metric} 

% section concurrent_process_calculi (end)

%\input{qm2pi.proofsketch}

% section proof sketch (end)

%\input{qm2pi.slviaknots} 

% section spatial logic via knots (end)

\input{qm2pi.conclusion}

% section conclusion (end)

%\input{qm2pi.dtcodes} 

% section wiring algorithm (end)

\input{qm2pi.ack} 

% section acknowledgments (end)

\newpage


\bibliographystyle{plain}   
\bibliography{../../biblios/main.bib}

\input{qm2pi.rhodetails}

\end{document}

 

%\documentclass[12pt]{llncs}
%\documentclass{jktr}

\usepackage[pdftex]{hyperref}                   
\usepackage {listings}
\usepackage {mathpartir}
\usepackage{bcprules}
%\usepackage{listings}
                       
\usepackage{graphicx} 
%\usepackage[margins=2.5cm,nohead,nofoot]{geometry}
%\usepackage{geometry}
\usepackage{amsfonts}
\usepackage{amstext}
\usepackage{latexsym}
\usepackage{amssymb}
\usepackage{color}


%\include{myPreamble}
\include{qm2pi.local} 

%\ifpdf
%\usepackage[pdftex]{graphicx}
%\else
%\usepackage{graphicx}
%\fi

 % \ifpdf
%  \usepackage{pdfsync}
%  \if


%\title{Brief Article}
%\author{David F. Snyder}
%\author{L.G. Meredith}

%\address{Dept. of Math., Texas State University--San Marcos, San Marcos, TX 78666}
       
\pagestyle{empty}


\begin{document}

\lstset{language=[Objective]Caml,frame=shadowbox}

\input{qm2pi.front}

% section front matter (end)

\input{qm2pi.intro} 
 
% section introduction (end)

% \input{qm2pi.knotations} 

% section notation (end)

\input{qm2pi.process.calculi} 

% section concurrent_process_calculi_and_spatial_logics_ (end)
    
%\input{qm2pi.knots2pi} 

%\input{qm2pi.trefoil} 

%\input{qm2pi.mainthm} 

% subsection basic_interpretation (end)

%\input{qm2pi.rho.presentation} 
\subsection{The syntax and semantics of the notation system}\label{sub:the_syntax_and_semantics_of_the_notation_system} % (fold)

We now summarize a technical presentation of the calculus that
embodies our theory of dynamics. The typical presentation of such a
calculus follows the style of giving generators and relations on
them. The grammar, below, describing term constructors, freely
generates the set of processes, $\Proc$. This set is then quotiented
by a relation known as structural congruence and it is over this set
that the notion of dynamics is expressed. This presentation is
essentially that of \cite{MeredithR05} with the addition of
polyadicity and summation. For readability we have relegated some of
the technical subtleties to an appendix.

\subsubsection{Process grammar}\label{subsub:process_grammar}

\begin{mathpar}
  \inferrule* [lab=synchronization] {} {{M} \bc \pzero \;|\; x?F \;|\; x!C }
  \and
  \inferrule* [lab=abstraction] {} {{F} \bc (x)P}
  \and
  \inferrule* [lab=concretion] {} {{C} \bc \langle Q \rangle}
  \and
  \inferrule* [lab=process] {} {{P,Q} \bc M \;| \;P|Q \;|\; @{x}}
  \and
  \inferrule* [lab=name] {} {{x} \bc \quotep{P}}
\end{mathpar} 

Note that $\vec{x}$ (resp. $\vec{P}$) denotes a vector of names
(resp. processes) of length $|\vec{x}|$ (resp. $|\vec{P}|$). We adopt
the following useful abbreviations.

\begin{mathpar}
   x?(\vec{y}).P := x.(\vec{y})P \and  x\clift{\vec{P}} := x.\clift{\vec{P}}
   \and x!(y) := \lift{x}{\dropn{y}}
   \and \Pi_{i=0}^{n-1}P_i := P_0 | \ldots | P_{n-1}
\end{mathpar}

\subsubsection{Structural congruence}

\paragraph{Free and bound names and alpha-equivalence.} At the
core of structural equivalence is alpha-equivalence which identifies
process that are the same up to a change of variable. Formally, we
recognize the distinction between free and bound names. The free names
of a process, $\freenames{P}$, may be calculated recursively as
follows:

\begin{mathpar}
\freenames{\pzero} := \emptyset
  \and \\
  \freenames{x?(y).P} := \{ x \} \cup (\freenames{P} \setminus \{ y \})
  \and 
  \freenames{x!\langle P \rangle} := \{ x \} \cup \{ P \} 
  \and \\
  \freenames{P|Q} := \freenames{P} \cup \freenames{Q}
  \and \\
  \freenames{@{x}} := \{ x \}
\end{mathpar}

$\pi$
$\quotep{\pi}$

$\freenames{-} : \pi \to \mathcal{P}(\quotep{\pi})$

\begin{eqnarray*}
  \freenames{\pzero} & := & \emptyset \\
  \freenames{x?(y).P} & := & \{ x \} \cup (\freenames{P} \setminus \{ y \}) \\
  \freenames{x!\langle P \rangle} & := & \{ x \} \cup \{ P \} \\
  \freenames{P|Q} & := & \freenames{P} \cup \freenames{Q} \\
  \freenames{\dropn{x}} & := & \{ x \}
\end{eqnarray*}

The bound names of a process, $\boundnames{P}$, are those names occurring in $P$
that are not free. For example, in $x?(y).0$, the name $x$ is free, while $y$ is bound.

\begin{mathpar}
  \inferrule* [lab=monoidal-laws] {} { P|Q \equiv Q|P \and P|0 \equiv P \and P|(Q|R) \equiv (P|Q)|R }
\end{mathpar}

\begin{mathpar}
  \inferrule* [lab=alpha-equivalence] {} { (x)P \equiv (y)P\{y/x\} \and y \not\in \freenames{P} }
\end{mathpar}

\begin{definition}
Then two processes, $P,Q$, are alpha-equivalent if $P = Q\{\vec{y}/\vec{x}\}$ for
some $\vec{x} \in \boundnames{Q},\vec{y} \in \boundnames{P}$, where $Q\{\vec{y}/\vec{x}\}$
denotes the capture-avoiding substitution of $\vec{y}$ for $\vec{x}$ in $Q$.
\end{definition}

\begin{definition}
  The {\em structural congruence} \cite{SangiorgiWalker} , $\equiv$,
  between processes is the least congruence containing
  alpha-equivalence, satisfying the abelian monoid laws
  (associativity, commutativity and $\pzero$ as identity) for parallel
  composition $|$ and for summation $+$.
\end{definition}

\subsection{Name equivalence}

We take name equivalence, written $\nameeq$, to be the smallest
equivalence relation generated by the following rules.

\begin{mathpar}
\inferrule*[lab=Quote-drop]
{ }
{ \quotep{@{x}} \nameeq x }

\inferrule*[lab=Struct-equiv]
{ P \scong Q }
{ \quotep{P} \nameeq \quotep{Q} }
\end{mathpar}

The astute reader will have noticed that the mutual recursion of names
and processes imposes a mutual recursion on alpha-equivalence and
structural equivalence via name-equivalence. Fortunately, all of this
works out pleasantly and we may calculate in the natural way, free of
concern. The reader interested in the details is referred to the
appendix \ref{appendix:rho_details}.

\subsection{Substitution}

We use $\Proc$ for the set of processes, $\QProc$ for the set of
names, and $\id{\{}\vec{y} / \vec{x} \id{\}}$ to denote partial maps,
$s : \QProc \rightarrow \QProc$. A map, $s$ lifts, uniquely, to a map
on process terms, $\widehat{s} : \Proc \rightarrow \Proc$ by the
following equations.

\begin{mathpar}
  (0) \psubstp{Q}{P} := 0 \\
  (R \juxtap S) \psubstp{Q}{P}
  :=    
  (R)\psubstp{Q}{P} \juxtap (S) \psubstp{Q}{P} \\
  (x?(y).R) \psubstp{Q}{P}    
  :=    
  (x)\substp{Q}{P} (z)\concat( (R \psubstn{z}{y}) \psubstp{Q}{P} ) \\
  (\lift{x}{R}) \psubstp{Q}{P}  
  :=
  \lift{(x)\substp{Q}{P}}{ R \psubstp{Q}{P} } \\
%   (\dropn{x})  \psubstp{Q}{P}       
%   := 
%   \left\{ 
%     \begin{array}{ccc} 
%       \dropn{\quotep{Q}} & & x \nameeq \quotep{P} \\
%       \dropn{x} & & otherwise \\
%     \end{array}
%   \right. 
  (\dropn{x})  \psubstp{Q}{P}       
  := 
  \left\{ 
    \begin{array}{ccc} 
      Q & & x \nameeq \quotep{P} \\
      \dropn{x} & & otherwise \\
    \end{array}
  \right.
\end{mathpar}
 

where

\begin{eqnarray}
  (x)\id{\{} \lpquote Q \rpquote / \lpquote P \rpquote \id{\}}            = 
  \left\{ 
    \begin{array}{ccc}
      \lpquote Q \rpquote & & x \nameeq \lpquote P \rpquote \\
      x & & otherwise \\
    \end{array}
  \right. \nonumber
\end{eqnarray}

and $z$ is chosen distinct from $\quotep{P}$, $\quotep{Q}$, the free
names in $Q$, and all the names in $R$. Our $\alpha$-equivalence will
be built in the standard way from this substitution.

\begin{remark}\label{rem:no_self_referential_names}
  One consequence of these definitions is that $\forall P. \quotep{P}
  \not\in \freenames{P}$.
\end{remark}

\subsection{ Dynamic quote: an example }

Anticipating something of what's to come, consider applying the
substitution, $\widehat{\id{\{}u / z \id{\}}}$, to the following pair
of processes, $\lift{w}{y!(z)}$ and $w[ \lpquote y!(z) \rpquote ]$.

\begin{eqnarray}
	\lift{w}{y!(z)}\widehat{\id{\{}u / z \id{\}}}
		& = &
		\lift{w}{y!(u)} \nonumber\\
	w[ \lpquote y!(z) \rpquote ] \widehat{ \id{\{}u / z \id{\}} }
		& = &
		w[ \lpquote y!(z) \rpquote ] \nonumber
\end{eqnarray}

Because the body of the process between quotes is impervious to
substitution, we get radically different answers. In fact, by
examining the first process in an input context,
e.g. $x?(z).\lift{w}{y!(z)}$, we see that the process under the lift
operator may be shaped by prefixed inputs binding a name inside it. In
this sense, the lift operator will be seen as a way to dynamically
construct processes before reifying them as names.

Finally equipped with these standard features we can present the
dynamics of the calculus.

\subsubsection{Operational semantics} 

Finally, we introduce the computational dynamics. What marks these
algebras as distinct from other more traditionally studied algebraic
structures, e.g. vector spaces or polynomial rings, is the manner in
which dynamics is captured. In traditional structures, dynamics is typically
expressed through morphisms between such structures, as in linear maps
between vector spaces or morphisms between rings. In algebras
associated with the semantics of computation, the dynamics is
expressed as part of the algebraic structure itself, through a
reduction reduction relation typically denoted by $\red$. Below, we
give a recursive presentation of this relation for the calculus used
in the encoding.

$\red \subseteq \pi \times \pi$
$\red : \pi \to \mathcal{P}(\pi)$

\begin{mathpar}
  \inferrule* [lab=Comm] { \textsf{match}( x_{src}, x_{trgt} ) } { x_{trgt}?(y)P \; | \; x_{src}!\langle {Q} \rangle \red P\{\quotep{Q}/y}\} }
  \and \\
  \inferrule* [lab=Par] {{P} \red {P}'} {{{P} | {Q}} \red {{P}' | {Q}}}
  \and
  \inferrule* [lab=Equiv]{{{P} \scong {P}'} \andalso {{P}' \red {Q}'} \andalso {{Q}' \scong {Q}}}{{P} \red {Q}}
\end{mathpar}

\begin{eqnarray*}
  match_{\equiv} (\quotep{P},\quotep{Q}) & := & P \equiv Q \\
  match_{\dagger}(\quotep{P},\quotep{Q}) & := & \forall R. P|Q \red^{*} R => R \red^{*} 0 \\
  match_{K}(\quotep{P},\quotep{Q}) & := & K \mbox{ for some context } K
\end{eqnarray*}

$u?(x)P | u!\langle Q \rangle \red P\{\quotep{Q}/x\}$

%We write $\wred$ for $\red^*$, and $P\red$ if $\exists Q $ such that $ P \red Q$.
We write $P\red$ if $\exists Q $ such that $ P \red Q$ and $P\not\red$, otherwise.

\section{Replication}

As mentioned before, it is known that replication (and hence
recursion) can be implemented in a higher-order process algebra
\cite{SangiorgiWalker}. As our first example of calculation with the
machinery thus far presented we give the construction explicitly in
the {\rhoc}.

\begin{eqnarray}
	D_{x} & := & \prefix{x}{y}{(\binpar{\outputp{x}{y}}{@{y}})} \nonumber\\
	\bangp_{x}{P} & := & \binpar{{x}!\langle{\binpar{D_{x}}{P}}\rangle}{D_{x}} \nonumber
\end{eqnarray}

\begin{eqnarray}
	\bangp_{x}{P} & & \nonumber\\
	=
	& {x}!\langle{(\prefix{x}{y}{(\outputp{x}{y} | @{y})) | P}}\rangle 
	      | \prefix{x}{y}{(\outputp{x}{y} | @{y})} & \nonumber\\
	\red
	& (\outputp{x}{y} | @{y})\substn{\quotep{(\prefix{x}{y}{(@{y} | \outputp{x}{y})) | P}}}{y} & \nonumber\\
	=
	& \outputp{x}{\quotep{(\prefix{x}{y}{(\outputp{x}{y} | @{y})) | P}}}
	  | {(\prefix{x}{y}{(\outputp{x}{y} | @{y})) | P}} & \nonumber\\
	\red
	& \ldots & \nonumber\\
	\red^*
	& P | P | \ldots & \nonumber
\end{eqnarray}

Of course, this encoding, as an implementation, runs away, unfolding
$\bangp{P}$ eagerly. A lazier and more implementable replication
operator, restricted to input-guarded processes, may be obtained as follows.

\begin{eqnarray}
\bangp{\prefix{u}{v}{P}} 
	:= 
	\binpar{\lift{x}{\prefix{u}{v}{(\binpar{D(x)}{P})}}}{D(x)} \nonumber
\end{eqnarray}

\begin{remark}
  Note that the lazier definition still does not deal with summation
  or mixed summation (i.e. sums over input and output). The reader is
  invited to construct definitions of replication that deal with these
  features. 

  Further, the definitions are parameterized in a name, $x$. Can you,
  gentle reader, make a definition that eliminates this parameter and
  guarantees no accidental interaction between the replication
  machinery and the process being replicated -- i.e. no accidental
  sharing of names used by the process to get its work done and the
  name(s) used by the replication to effect copying. This latter
  revision of the definition of replication is crucial to obtaining
  the expected identity $!!P \sim !P$.
\end{remark}

\begin{remark}\label{rem:paradoxical_combinator}
  The reader familiar with the lambda calculus will have noticed the
  similarity between $D$ and the paradoxical combinator.

  [Ed. note: the existence of this seems to suggest we have to be more
  restrictive on the set of processes and names we admit if we are to
  support no-cloning.]
\end{remark}

\subsubsection{Bisimulation}

The computational dynamics gives rise to another kind of equivalence,
the equivalence of computational behavior. As previously mentioned
this is typically captured \emph{via} some form of bisimulation.

% The notion we use in this paper is weak barbed bisimulation
% \cite{milner91polyadicpi}.

The notion we use in this paper is derived from weak barbed
bisimulation \cite{milner91polyadicpi}. 

\begin{definition}
An \emph{observation relation}, $\downarrow_{\mathcal N}$, over a set
of names, $\mathcal N$, is the smallest relation satisfying the rules
below.

\infrule[Out-barb]{y \in {\mathcal N}, \; x \nameeq y}
		  {\outputp{x}{v} \downarrow_{\mathcal N} x}
\infrule[Par-barb]{\mbox{$P\downarrow_{\mathcal N} x$ or $Q\downarrow_{\mathcal N} x$}}
		  {\binpar{P}{Q} \downarrow_{\mathcal N} x}

We write $P \Downarrow_{\mathcal N} x$ if there is $Q$ such that 
$P \wred Q$ and $Q \downarrow_{\mathcal N} x$.
\end{definition}

\begin{definition}
%\label{def.bbisim}
An  ${\mathcal N}$-\emph{barbed bisimulation} over a set of names, ${\mathcal N}$, is a symmetric binary relation 
${\mathcal S}_{\mathcal N}$ between agents such that $P\rel{S}_{\mathcal N}Q$ implies:
\begin{enumerate}
\item If $P \red P'$ then $Q \wred Q'$ and $P'\rel{S}_{\mathcal N} Q'$.
\item If $P\downarrow_{\mathcal N} x$, then $Q\Downarrow_{\mathcal N} x$.
\end{enumerate}
$P$ is ${\mathcal N}$-barbed bisimilar to $Q$, written
$P \wbbisim_{\mathcal N} Q$, if $P \rel{S}_{\mathcal N} Q$ for some ${\mathcal N}$-barbed bisimulation ${\mathcal S}_{\mathcal N}$.
\end{definition}

$\mathcal{R} \subseteq \pi \times \pi$

$P \mathcal{R} Q => \forall P'. P \red P' \Rightarrow \exists Q'. Q \red Q', P' \mathcal{R} Q'$

$P \vdash x \Rightarrow Q \vdash x$

\begin{mathpar}
  \inferrule*[lab=Out-barb]{x \nameeq y}{{y}!\langle{Q}\rangle \vdash x}
  \and
  \inferrule*[lab=Par-barb]{\mbox{$P\vdash x$ or $Q\vdash x$}}{\binpar{P}{Q} \vdash x}
\end{mathpar}

\subsubsection{Contexts}

One of the principle advantages of computational calculi like the
$\pi$-calculus is a well-defined notion of context,
contextual-equivalence and a correlation between
contextual-equivalence and notions of bisimulation. The notion of
context allows the decomposition of a process into (sub-)process and
its syntactic environment, its context. Thus, a context may be
thought of as a process with a ``hole'' (written $\Box$) in it. The
application of a context $M$ to a process $P$, written $M[P]$, is
tantamount to filling the hole in $M$ with $P$. In this paper we do
not need the full weight of this theory, but do make use of the notion
of context in the proof the main theorem. 

\begin{mathpar}
  \inferrule* [lab=summation] {} {{M_{M},M_{N}} \bc \Box \;|\; x.M_{A} \;|\; M_{M}+M_{N}}
  \and
  \inferrule* [lab=agent] {} {{M_{A}} \bc (\vec{x})M_{P} \;| \; \clift{P_0,\ldots,M_{P},\ldots,P_N}}
  \and \\
  \inferrule* [lab=process] {} {{M_{P}} \bc M_{N} \;| \;P|M_{P} }
\end{mathpar} 

\begin{mathpar}
  \inferrule* [lab=sychronization] {} {M_{N} \bc \Box \;|\; x?M_{F} \;|\; x!M_{C}}
  \and
  \inferrule* [lab=abstraction] {} {{M_{F}} \bc (x)M_{P} }
  \and
  \inferrule* [lab=concretion] {} {{M_{C}} \bc \langle M_{P} \rangle }
  \and \\
  \inferrule* [lab=process] {} {{M_{P}} \bc M_{N} \;| \;P|M_{P} }
\end{mathpar}

\begin{definition}[contextual application] Given a context $M$, and
  process $P$, we define the \emph{contextual application}, $M[P] :=
  M\{P/\Box\}$. That is, the contextual application of M to P is the
  substitution of $P$ for $\Box$ in $M$.
\end{definition}

$\meaningof{-} : L \to \mathcal{P}(\pi)$

\begin{mathpar}
  \inferrule* [lab=collection] {} {\meaningof{true} = \pi, \and \meaningof{~E} = \pi \setminus \meaningof{E}, \and \meaningof{E_{1} \& E_{2}} = \meaningof{E_{1}} \cap \meaningof{E_{2}}}
\end{mathpar}

\begin{mathpar}
  \inferrule* [lab=structure] {} {\meaningof{0} = \{ P \in \pi | P \equiv 0 \}, \and \\ \meaningof{E_1 | E_2} = \{ P \in \pi | P \equiv P_{1} | P_{2}, P_{1} \in \meaningof{E_{1}}, P_{2} \in \meaningof{E_2}\} }
\end{mathpar}

\begin{mathpar}
 \inferrule* [lab=behavior] {} {\meaningof{\langle a?b \rangle E} = \{ P \in \pi | P \equiv Q | u?(y)P', \\ \and \\\\ \and \\ \;\;\; u \in \meaningof{a}, \forall z.P'\{z/y\} \in \meaningof{E\{z/b\}}\}, \and \\ \meaningof{a!E} = \{ P \in \pi | P \equiv Q | x!\langle P' \rangle, x \in \meaningof{a} P' \in \meaningof{E}\} }
\end{mathpar}

\begin{mathpar}
 \inferrule* [lab=nominal] {} {\meaningof{\quotep{E}} = \{ \quotep{P} \in \quotep{\pi} | P \in \meaningof{E} \}, \and \meaningof{\quotep{P}} = \{ \quotep{Q} \in \quotep{\pi} | P \equiv Q \} \and \\ \meaningof{@\quotep{E}} = \{ P \in \pi | P \equiv @x, x \in \meaningof{E} \}}
\end{mathpar}

\begin{eqnarray*}
  \\
  \meaningof{-} : TS \to ST
\end{eqnarray*}

\begin{eqnarray*}
  \\
  L : TS \to ST
\end{eqnarray*}

\begin{eqnarray*}
  \\
  P \models E \iff P \in \meaningof{E}
\end{eqnarray*}

\begin{eqnarray*}
  P \approx_{L} Q \iff \forall E \in L. P \models E \iff Q \models E
\end{eqnarray*}

\begin{eqnarray*}
  P \approx_{K} Q
\end{eqnarray*}

\begin{eqnarray*}
  P \approx Q
\end{eqnarray*}

$\approx_{K} = \approx = \approx_{L}$

\subsubsection{Contextual duality}

Note that contexts extend the quotation operation to a family of
operations from processes to names. Given a context, $M$, we can
define a \emph{nominal context}, $\quotep{M}$ by $\quotep{M}[P] :=
\quotep{M[P]}$. To foreshadow what is to come we observe that these
operations enjoy a duality with processes very much like the duality
between vectors and maps from vectors to scalars.

Further, because the calculus is essentially higher-order, we have a
correspondence between contexts and processes. More specifically,
given a name $x$ and a context $M$ we can construct $M^{*}_{x}$ such
that 

\begin{mathpar}
  M^{*}_{x} | \lift{x}{P} \red M[P]
\end{mathpar}

namely,

\begin{mathpar}
  M^{*}_{x} := x?(u).M[\dropn{u}]
\end{mathpar}

The dependence of $M^{*}_{x}$ on a name makes it an abstraction, 

\begin{mathpar}
  M^{*} := (x)x?(u).M[\dropn{u}]
\end{mathpar}

\subsection{Additional notation}

It will sometimes be convenient to denote the process a name
quotes. We already have the notation $x = \quotep{P}$, but it will be
convenient to introduce an alternate notation, $\procn{x}$, when we
want to emphasize the connection to the use of the name. Note that, by
virtue of name equivalence, $\quotep{\procn{x}} \nameeq x$; so, the
notation is consistent with previous definitions.

Further, because names have structure it is possible to effect
substitutions on the basis of that structure. This means we need to
upgrade our notation for substitutions, which we accomplish by
adapting comprehension notation. Thus,

\begin{mathpar}
  P\{ y / x : x \in S \}
\end{mathpar}

is interpreted to mean the process derived from P by replacing (in a
capture-avoiding manner) each occurrence of $x$ in $S$ by $y$. For example,

\begin{mathpar}
  P\{ \quotep{\procn{x}|\procn{x}} / x : x \in \freenames{P} \}
\end{mathpar}

will replace each (occurrence) of a free name $x$ in $P$ by
$\quotep{\procn{x}|\procn{x}}$.

Also, we will avail ourselves of the notation $x^{L}$ and $x^{R}$ to
denote injections of a name into disjoint copies of the name
space. There are numerous ways to accomplish this. One example can be
found in \cite{MeredithR05}. This notation overloads to vectors of
names: $\vec{x}^{\pi} := (x_{i}^{\pi} \; : \; 0 \leq i < |\vec{x}| )$ where $\pi \in \{L,R\}$.

We also use $P^{\Box} := P|\Box$.

In \cite{MeredithR05} an interpretation of the new operator is
given. It turns out that there are several possible interpretations
all enjoying the requisite algebraic properties of the operator (see
\cite{milner91polyadicpi}). We will therefore make liberal use of
$(\nu\; \vec{x})P$.

% subsection the_syntax_and_semantics_of_the_notation_system (end)   

\input{qm2pi.qmops} 

\input{qm2pi.sterngerlach} 

\input{qm2pi.metric} 

% section concurrent_process_calculi (end)

%\input{qm2pi.proofsketch}

% section proof sketch (end)

%\input{qm2pi.slviaknots} 

% section spatial logic via knots (end)

\input{qm2pi.conclusion}

% section conclusion (end)

%\input{qm2pi.dtcodes} 

% section wiring algorithm (end)

\input{qm2pi.ack} 

% section acknowledgments (end)

\newpage


\bibliographystyle{plain}   
\bibliography{../../biblios/main.bib}

\input{qm2pi.rhodetails}

\end{document}

 

% subsection basic_interpretation (end)

%\input{qm2pi.rho.presentation} 
\subsection{The syntax and semantics of the notation system}\label{sub:the_syntax_and_semantics_of_the_notation_system} % (fold)

We now summarize a technical presentation of the calculus that
embodies our theory of dynamics. The typical presentation of such a
calculus follows the style of giving generators and relations on
them. The grammar, below, describing term constructors, freely
generates the set of processes, $\Proc$. This set is then quotiented
by a relation known as structural congruence and it is over this set
that the notion of dynamics is expressed. This presentation is
essentially that of \cite{MeredithR05} with the addition of
polyadicity and summation. For readability we have relegated some of
the technical subtleties to an appendix.

\subsubsection{Process grammar}\label{subsub:process_grammar}

\begin{mathpar}
  \inferrule* [lab=synchronization] {} {{M} \bc \pzero \;|\; x?F \;|\; x!C }
  \and
  \inferrule* [lab=abstraction] {} {{F} \bc (x)P}
  \and
  \inferrule* [lab=concretion] {} {{C} \bc \langle Q \rangle}
  \and
  \inferrule* [lab=process] {} {{P,Q} \bc M \;| \;P|Q \;|\; @{x}}
  \and
  \inferrule* [lab=name] {} {{x} \bc \quotep{P}}
\end{mathpar} 

Note that $\vec{x}$ (resp. $\vec{P}$) denotes a vector of names
(resp. processes) of length $|\vec{x}|$ (resp. $|\vec{P}|$). We adopt
the following useful abbreviations.

\begin{mathpar}
   x?(\vec{y}).P := x.(\vec{y})P \and  x\clift{\vec{P}} := x.\clift{\vec{P}}
   \and x!(y) := \lift{x}{\dropn{y}}
   \and \Pi_{i=0}^{n-1}P_i := P_0 | \ldots | P_{n-1}
\end{mathpar}

\subsubsection{Structural congruence}

\paragraph{Free and bound names and alpha-equivalence.} At the
core of structural equivalence is alpha-equivalence which identifies
process that are the same up to a change of variable. Formally, we
recognize the distinction between free and bound names. The free names
of a process, $\freenames{P}$, may be calculated recursively as
follows:

\begin{mathpar}
\freenames{\pzero} := \emptyset
  \and \\
  \freenames{x?(y).P} := \{ x \} \cup (\freenames{P} \setminus \{ y \})
  \and 
  \freenames{x!\langle P \rangle} := \{ x \} \cup \{ P \} 
  \and \\
  \freenames{P|Q} := \freenames{P} \cup \freenames{Q}
  \and \\
  \freenames{@{x}} := \{ x \}
\end{mathpar}

$\pi$
$\quotep{\pi}$

$\freenames{-} : \pi \to \mathcal{P}(\quotep{\pi})$

\begin{eqnarray*}
  \freenames{\pzero} & := & \emptyset \\
  \freenames{x?(y).P} & := & \{ x \} \cup (\freenames{P} \setminus \{ y \}) \\
  \freenames{x!\langle P \rangle} & := & \{ x \} \cup \{ P \} \\
  \freenames{P|Q} & := & \freenames{P} \cup \freenames{Q} \\
  \freenames{\dropn{x}} & := & \{ x \}
\end{eqnarray*}

The bound names of a process, $\boundnames{P}$, are those names occurring in $P$
that are not free. For example, in $x?(y).0$, the name $x$ is free, while $y$ is bound.

\begin{mathpar}
  \inferrule* [lab=monoidal-laws] {} { P|Q \equiv Q|P \and P|0 \equiv P \and P|(Q|R) \equiv (P|Q)|R }
\end{mathpar}

\begin{mathpar}
  \inferrule* [lab=alpha-equivalence] {} { (x)P \equiv (y)P\{y/x\} \and y \not\in \freenames{P} }
\end{mathpar}

\begin{definition}
Then two processes, $P,Q$, are alpha-equivalent if $P = Q\{\vec{y}/\vec{x}\}$ for
some $\vec{x} \in \boundnames{Q},\vec{y} \in \boundnames{P}$, where $Q\{\vec{y}/\vec{x}\}$
denotes the capture-avoiding substitution of $\vec{y}$ for $\vec{x}$ in $Q$.
\end{definition}

\begin{definition}
  The {\em structural congruence} \cite{SangiorgiWalker} , $\equiv$,
  between processes is the least congruence containing
  alpha-equivalence, satisfying the abelian monoid laws
  (associativity, commutativity and $\pzero$ as identity) for parallel
  composition $|$ and for summation $+$.
\end{definition}

\subsection{Name equivalence}

We take name equivalence, written $\nameeq$, to be the smallest
equivalence relation generated by the following rules.

\begin{mathpar}
\inferrule*[lab=Quote-drop]
{ }
{ \quotep{@{x}} \nameeq x }

\inferrule*[lab=Struct-equiv]
{ P \scong Q }
{ \quotep{P} \nameeq \quotep{Q} }
\end{mathpar}

The astute reader will have noticed that the mutual recursion of names
and processes imposes a mutual recursion on alpha-equivalence and
structural equivalence via name-equivalence. Fortunately, all of this
works out pleasantly and we may calculate in the natural way, free of
concern. The reader interested in the details is referred to the
appendix \ref{appendix:rho_details}.

\subsection{Substitution}

We use $\Proc$ for the set of processes, $\QProc$ for the set of
names, and $\id{\{}\vec{y} / \vec{x} \id{\}}$ to denote partial maps,
$s : \QProc \rightarrow \QProc$. A map, $s$ lifts, uniquely, to a map
on process terms, $\widehat{s} : \Proc \rightarrow \Proc$ by the
following equations.

\begin{mathpar}
  (0) \psubstp{Q}{P} := 0 \\
  (R \juxtap S) \psubstp{Q}{P}
  :=    
  (R)\psubstp{Q}{P} \juxtap (S) \psubstp{Q}{P} \\
  (x?(y).R) \psubstp{Q}{P}    
  :=    
  (x)\substp{Q}{P} (z)\concat( (R \psubstn{z}{y}) \psubstp{Q}{P} ) \\
  (\lift{x}{R}) \psubstp{Q}{P}  
  :=
  \lift{(x)\substp{Q}{P}}{ R \psubstp{Q}{P} } \\
%   (\dropn{x})  \psubstp{Q}{P}       
%   := 
%   \left\{ 
%     \begin{array}{ccc} 
%       \dropn{\quotep{Q}} & & x \nameeq \quotep{P} \\
%       \dropn{x} & & otherwise \\
%     \end{array}
%   \right. 
  (\dropn{x})  \psubstp{Q}{P}       
  := 
  \left\{ 
    \begin{array}{ccc} 
      Q & & x \nameeq \quotep{P} \\
      \dropn{x} & & otherwise \\
    \end{array}
  \right.
\end{mathpar}
 

where

\begin{eqnarray}
  (x)\id{\{} \lpquote Q \rpquote / \lpquote P \rpquote \id{\}}            = 
  \left\{ 
    \begin{array}{ccc}
      \lpquote Q \rpquote & & x \nameeq \lpquote P \rpquote \\
      x & & otherwise \\
    \end{array}
  \right. \nonumber
\end{eqnarray}

and $z$ is chosen distinct from $\quotep{P}$, $\quotep{Q}$, the free
names in $Q$, and all the names in $R$. Our $\alpha$-equivalence will
be built in the standard way from this substitution.

\begin{remark}\label{rem:no_self_referential_names}
  One consequence of these definitions is that $\forall P. \quotep{P}
  \not\in \freenames{P}$.
\end{remark}

\subsection{ Dynamic quote: an example }

Anticipating something of what's to come, consider applying the
substitution, $\widehat{\id{\{}u / z \id{\}}}$, to the following pair
of processes, $\lift{w}{y!(z)}$ and $w[ \lpquote y!(z) \rpquote ]$.

\begin{eqnarray}
	\lift{w}{y!(z)}\widehat{\id{\{}u / z \id{\}}}
		& = &
		\lift{w}{y!(u)} \nonumber\\
	w[ \lpquote y!(z) \rpquote ] \widehat{ \id{\{}u / z \id{\}} }
		& = &
		w[ \lpquote y!(z) \rpquote ] \nonumber
\end{eqnarray}

Because the body of the process between quotes is impervious to
substitution, we get radically different answers. In fact, by
examining the first process in an input context,
e.g. $x?(z).\lift{w}{y!(z)}$, we see that the process under the lift
operator may be shaped by prefixed inputs binding a name inside it. In
this sense, the lift operator will be seen as a way to dynamically
construct processes before reifying them as names.

Finally equipped with these standard features we can present the
dynamics of the calculus.

\subsubsection{Operational semantics} 

Finally, we introduce the computational dynamics. What marks these
algebras as distinct from other more traditionally studied algebraic
structures, e.g. vector spaces or polynomial rings, is the manner in
which dynamics is captured. In traditional structures, dynamics is typically
expressed through morphisms between such structures, as in linear maps
between vector spaces or morphisms between rings. In algebras
associated with the semantics of computation, the dynamics is
expressed as part of the algebraic structure itself, through a
reduction reduction relation typically denoted by $\red$. Below, we
give a recursive presentation of this relation for the calculus used
in the encoding.

$\red \subseteq \pi \times \pi$
$\red : \pi \to \mathcal{P}(\pi)$

\begin{mathpar}
  \inferrule* [lab=Comm] { \textsf{match}( x_{src}, x_{trgt} ) } { x_{trgt}?(y)P \; | \; x_{src}!\langle {Q} \rangle \red P\{\quotep{Q}/y}\} }
  \and \\
  \inferrule* [lab=Par] {{P} \red {P}'} {{{P} | {Q}} \red {{P}' | {Q}}}
  \and
  \inferrule* [lab=Equiv]{{{P} \scong {P}'} \andalso {{P}' \red {Q}'} \andalso {{Q}' \scong {Q}}}{{P} \red {Q}}
\end{mathpar}

\begin{eqnarray*}
  match_{\equiv} (\quotep{P},\quotep{Q}) & := & P \equiv Q \\
  match_{\dagger}(\quotep{P},\quotep{Q}) & := & \forall R. P|Q \red^{*} R => R \red^{*} 0 \\
  match_{K}(\quotep{P},\quotep{Q}) & := & K \mbox{ for some context } K
\end{eqnarray*}

$u?(x)P | u!\langle Q \rangle \red P\{\quotep{Q}/x\}$

%We write $\wred$ for $\red^*$, and $P\red$ if $\exists Q $ such that $ P \red Q$.
We write $P\red$ if $\exists Q $ such that $ P \red Q$ and $P\not\red$, otherwise.

\section{Replication}

As mentioned before, it is known that replication (and hence
recursion) can be implemented in a higher-order process algebra
\cite{SangiorgiWalker}. As our first example of calculation with the
machinery thus far presented we give the construction explicitly in
the {\rhoc}.

\begin{eqnarray}
	D_{x} & := & \prefix{x}{y}{(\binpar{\outputp{x}{y}}{@{y}})} \nonumber\\
	\bangp_{x}{P} & := & \binpar{{x}!\langle{\binpar{D_{x}}{P}}\rangle}{D_{x}} \nonumber
\end{eqnarray}

\begin{eqnarray}
	\bangp_{x}{P} & & \nonumber\\
	=
	& {x}!\langle{(\prefix{x}{y}{(\outputp{x}{y} | @{y})) | P}}\rangle 
	      | \prefix{x}{y}{(\outputp{x}{y} | @{y})} & \nonumber\\
	\red
	& (\outputp{x}{y} | @{y})\substn{\quotep{(\prefix{x}{y}{(@{y} | \outputp{x}{y})) | P}}}{y} & \nonumber\\
	=
	& \outputp{x}{\quotep{(\prefix{x}{y}{(\outputp{x}{y} | @{y})) | P}}}
	  | {(\prefix{x}{y}{(\outputp{x}{y} | @{y})) | P}} & \nonumber\\
	\red
	& \ldots & \nonumber\\
	\red^*
	& P | P | \ldots & \nonumber
\end{eqnarray}

Of course, this encoding, as an implementation, runs away, unfolding
$\bangp{P}$ eagerly. A lazier and more implementable replication
operator, restricted to input-guarded processes, may be obtained as follows.

\begin{eqnarray}
\bangp{\prefix{u}{v}{P}} 
	:= 
	\binpar{\lift{x}{\prefix{u}{v}{(\binpar{D(x)}{P})}}}{D(x)} \nonumber
\end{eqnarray}

\begin{remark}
  Note that the lazier definition still does not deal with summation
  or mixed summation (i.e. sums over input and output). The reader is
  invited to construct definitions of replication that deal with these
  features. 

  Further, the definitions are parameterized in a name, $x$. Can you,
  gentle reader, make a definition that eliminates this parameter and
  guarantees no accidental interaction between the replication
  machinery and the process being replicated -- i.e. no accidental
  sharing of names used by the process to get its work done and the
  name(s) used by the replication to effect copying. This latter
  revision of the definition of replication is crucial to obtaining
  the expected identity $!!P \sim !P$.
\end{remark}

\begin{remark}\label{rem:paradoxical_combinator}
  The reader familiar with the lambda calculus will have noticed the
  similarity between $D$ and the paradoxical combinator.

  [Ed. note: the existence of this seems to suggest we have to be more
  restrictive on the set of processes and names we admit if we are to
  support no-cloning.]
\end{remark}

\subsubsection{Bisimulation}

The computational dynamics gives rise to another kind of equivalence,
the equivalence of computational behavior. As previously mentioned
this is typically captured \emph{via} some form of bisimulation.

% The notion we use in this paper is weak barbed bisimulation
% \cite{milner91polyadicpi}.

The notion we use in this paper is derived from weak barbed
bisimulation \cite{milner91polyadicpi}. 

\begin{definition}
An \emph{observation relation}, $\downarrow_{\mathcal N}$, over a set
of names, $\mathcal N$, is the smallest relation satisfying the rules
below.

\infrule[Out-barb]{y \in {\mathcal N}, \; x \nameeq y}
		  {\outputp{x}{v} \downarrow_{\mathcal N} x}
\infrule[Par-barb]{\mbox{$P\downarrow_{\mathcal N} x$ or $Q\downarrow_{\mathcal N} x$}}
		  {\binpar{P}{Q} \downarrow_{\mathcal N} x}

We write $P \Downarrow_{\mathcal N} x$ if there is $Q$ such that 
$P \wred Q$ and $Q \downarrow_{\mathcal N} x$.
\end{definition}

\begin{definition}
%\label{def.bbisim}
An  ${\mathcal N}$-\emph{barbed bisimulation} over a set of names, ${\mathcal N}$, is a symmetric binary relation 
${\mathcal S}_{\mathcal N}$ between agents such that $P\rel{S}_{\mathcal N}Q$ implies:
\begin{enumerate}
\item If $P \red P'$ then $Q \wred Q'$ and $P'\rel{S}_{\mathcal N} Q'$.
\item If $P\downarrow_{\mathcal N} x$, then $Q\Downarrow_{\mathcal N} x$.
\end{enumerate}
$P$ is ${\mathcal N}$-barbed bisimilar to $Q$, written
$P \wbbisim_{\mathcal N} Q$, if $P \rel{S}_{\mathcal N} Q$ for some ${\mathcal N}$-barbed bisimulation ${\mathcal S}_{\mathcal N}$.
\end{definition}

$\mathcal{R} \subseteq \pi \times \pi$

$P \mathcal{R} Q => \forall P'. P \red P' \Rightarrow \exists Q'. Q \red Q', P' \mathcal{R} Q'$

$P \vdash x \Rightarrow Q \vdash x$

\begin{mathpar}
  \inferrule*[lab=Out-barb]{x \nameeq y}{{y}!\langle{Q}\rangle \vdash x}
  \and
  \inferrule*[lab=Par-barb]{\mbox{$P\vdash x$ or $Q\vdash x$}}{\binpar{P}{Q} \vdash x}
\end{mathpar}

\subsubsection{Contexts}

One of the principle advantages of computational calculi like the
$\pi$-calculus is a well-defined notion of context,
contextual-equivalence and a correlation between
contextual-equivalence and notions of bisimulation. The notion of
context allows the decomposition of a process into (sub-)process and
its syntactic environment, its context. Thus, a context may be
thought of as a process with a ``hole'' (written $\Box$) in it. The
application of a context $M$ to a process $P$, written $M[P]$, is
tantamount to filling the hole in $M$ with $P$. In this paper we do
not need the full weight of this theory, but do make use of the notion
of context in the proof the main theorem. 

\begin{mathpar}
  \inferrule* [lab=summation] {} {{M_{M},M_{N}} \bc \Box \;|\; x.M_{A} \;|\; M_{M}+M_{N}}
  \and
  \inferrule* [lab=agent] {} {{M_{A}} \bc (\vec{x})M_{P} \;| \; \clift{P_0,\ldots,M_{P},\ldots,P_N}}
  \and \\
  \inferrule* [lab=process] {} {{M_{P}} \bc M_{N} \;| \;P|M_{P} }
\end{mathpar} 

\begin{mathpar}
  \inferrule* [lab=sychronization] {} {M_{N} \bc \Box \;|\; x?M_{F} \;|\; x!M_{C}}
  \and
  \inferrule* [lab=abstraction] {} {{M_{F}} \bc (x)M_{P} }
  \and
  \inferrule* [lab=concretion] {} {{M_{C}} \bc \langle M_{P} \rangle }
  \and \\
  \inferrule* [lab=process] {} {{M_{P}} \bc M_{N} \;| \;P|M_{P} }
\end{mathpar}

\begin{definition}[contextual application] Given a context $M$, and
  process $P$, we define the \emph{contextual application}, $M[P] :=
  M\{P/\Box\}$. That is, the contextual application of M to P is the
  substitution of $P$ for $\Box$ in $M$.
\end{definition}

$\meaningof{-} : L \to \mathcal{P}(\pi)$

\begin{mathpar}
  \inferrule* [lab=collection] {} {\meaningof{true} = \pi, \and \meaningof{~E} = \pi \setminus \meaningof{E}, \and \meaningof{E_{1} \& E_{2}} = \meaningof{E_{1}} \cap \meaningof{E_{2}}}
\end{mathpar}

\begin{mathpar}
  \inferrule* [lab=structure] {} {\meaningof{0} = \{ P \in \pi | P \equiv 0 \}, \and \\ \meaningof{E_1 | E_2} = \{ P \in \pi | P \equiv P_{1} | P_{2}, P_{1} \in \meaningof{E_{1}}, P_{2} \in \meaningof{E_2}\} }
\end{mathpar}

\begin{mathpar}
 \inferrule* [lab=behavior] {} {\meaningof{\langle a?b \rangle E} = \{ P \in \pi | P \equiv Q | u?(y)P', \\ \and \\\\ \and \\ \;\;\; u \in \meaningof{a}, \forall z.P'\{z/y\} \in \meaningof{E\{z/b\}}\}, \and \\ \meaningof{a!E} = \{ P \in \pi | P \equiv Q | x!\langle P' \rangle, x \in \meaningof{a} P' \in \meaningof{E}\} }
\end{mathpar}

\begin{mathpar}
 \inferrule* [lab=nominal] {} {\meaningof{\quotep{E}} = \{ \quotep{P} \in \quotep{\pi} | P \in \meaningof{E} \}, \and \meaningof{\quotep{P}} = \{ \quotep{Q} \in \quotep{\pi} | P \equiv Q \} \and \\ \meaningof{@\quotep{E}} = \{ P \in \pi | P \equiv @x, x \in \meaningof{E} \}}
\end{mathpar}

\begin{eqnarray*}
  \\
  \meaningof{-} : TS \to ST
\end{eqnarray*}

\begin{eqnarray*}
  \\
  L : TS \to ST
\end{eqnarray*}

\begin{eqnarray*}
  \\
  P \models E \iff P \in \meaningof{E}
\end{eqnarray*}

\begin{eqnarray*}
  P \approx_{L} Q \iff \forall E \in L. P \models E \iff Q \models E
\end{eqnarray*}

\begin{eqnarray*}
  P \approx_{K} Q
\end{eqnarray*}

\begin{eqnarray*}
  P \approx Q
\end{eqnarray*}

$\approx_{K} = \approx = \approx_{L}$

\subsubsection{Contextual duality}

Note that contexts extend the quotation operation to a family of
operations from processes to names. Given a context, $M$, we can
define a \emph{nominal context}, $\quotep{M}$ by $\quotep{M}[P] :=
\quotep{M[P]}$. To foreshadow what is to come we observe that these
operations enjoy a duality with processes very much like the duality
between vectors and maps from vectors to scalars.

Further, because the calculus is essentially higher-order, we have a
correspondence between contexts and processes. More specifically,
given a name $x$ and a context $M$ we can construct $M^{*}_{x}$ such
that 

\begin{mathpar}
  M^{*}_{x} | \lift{x}{P} \red M[P]
\end{mathpar}

namely,

\begin{mathpar}
  M^{*}_{x} := x?(u).M[\dropn{u}]
\end{mathpar}

The dependence of $M^{*}_{x}$ on a name makes it an abstraction, 

\begin{mathpar}
  M^{*} := (x)x?(u).M[\dropn{u}]
\end{mathpar}

\subsection{Additional notation}

It will sometimes be convenient to denote the process a name
quotes. We already have the notation $x = \quotep{P}$, but it will be
convenient to introduce an alternate notation, $\procn{x}$, when we
want to emphasize the connection to the use of the name. Note that, by
virtue of name equivalence, $\quotep{\procn{x}} \nameeq x$; so, the
notation is consistent with previous definitions.

Further, because names have structure it is possible to effect
substitutions on the basis of that structure. This means we need to
upgrade our notation for substitutions, which we accomplish by
adapting comprehension notation. Thus,

\begin{mathpar}
  P\{ y / x : x \in S \}
\end{mathpar}

is interpreted to mean the process derived from P by replacing (in a
capture-avoiding manner) each occurrence of $x$ in $S$ by $y$. For example,

\begin{mathpar}
  P\{ \quotep{\procn{x}|\procn{x}} / x : x \in \freenames{P} \}
\end{mathpar}

will replace each (occurrence) of a free name $x$ in $P$ by
$\quotep{\procn{x}|\procn{x}}$.

Also, we will avail ourselves of the notation $x^{L}$ and $x^{R}$ to
denote injections of a name into disjoint copies of the name
space. There are numerous ways to accomplish this. One example can be
found in \cite{MeredithR05}. This notation overloads to vectors of
names: $\vec{x}^{\pi} := (x_{i}^{\pi} \; : \; 0 \leq i < |\vec{x}| )$ where $\pi \in \{L,R\}$.

We also use $P^{\Box} := P|\Box$.

In \cite{MeredithR05} an interpretation of the new operator is
given. It turns out that there are several possible interpretations
all enjoying the requisite algebraic properties of the operator (see
\cite{milner91polyadicpi}). We will therefore make liberal use of
$(\nu\; \vec{x})P$.

% subsection the_syntax_and_semantics_of_the_notation_system (end)   

\section{Interpretation of QM}
\subsection{Supporting definitions}
\subsubsection{Multiplication}
\begin{mathpar}
  \quotep{Q} \cdot \quotep{R} := \quotep{Q|R}
  \and \\
  \quotep{Q} \cdot P := P\{ \quotep{Q|R} / \quotep{R} : \quotep{R} \in \freenames{P} \}
\end{mathpar}

\paragraph{Discussion}
The first line needs little explanation. The second line says that
each free name of the process is replaced with the multiplication of
that name by the scalar. Multiplication of a scalar (name) by a state
(process) results in a process all the names of which have been `moved
over' by parallel composition with the process the scalar
quotes. There is a subtlety that the bound names have to be
manipulated so that multiplied names aren't accidentally
captured. There are many ways to achieve this.

\begin{remark}\label{rem:multiplication_identities}
  The reader is invited to verify that for all $x,y,z \in \QProc$ and $P \in \Proc$
  \begin{mathpar}
    x \cdot \quotep{0} \equiv x 
    \and
    x \cdot y \equiv y \cdot x
    \and
    x \cdot (y \cdot z) \equiv (x \cdot y) \cdot z
    \and \\
    \quotep{0} \cdot P \equiv P
    \and \\
    x \cdot (y \cdot P) \equiv (x \cdot y) \cdot P
    \and \\
    x \cdot (P|Q) \equiv (x \cdot P) | (x \cdot Q)
    \and \\    
  \end{mathpar}
\end{remark}

\subsubsection{Tensor product}

We define a tensor product on processes by structural induction.

\paragraph{Tensor of sums} First note that all summations, including
$\pzero$ and sequence, can be written $\Sigma_{i} x_{i}.A_{i} +
\Sigma_{j} x_{j}.C_{j}$, where we have grouped input-guarded processes
together and output-guarded processes together.

Thus, we can define the tensor product of two summations, $N_{1}\otimes N_{2}$, where

\begin{mathpar}
  N_{1} := \Sigma_{i} x_{i}.A_{i} + \Sigma_{j} x_{j}.C_{j}
  \and
  N_{2} := \Sigma_{i'} y_{i'}.B_{i'} + \Sigma_{j'} y_{j'}.D_{j'} 
\end{mathpar}

as follows.

\begin{mathpar}
  \Sigma_{i} x_{i}.A_{i} + \Sigma_{j} x_{j}.C_{j} \otimes \Sigma_{i'}
  y_{i'}.B_{i'} + \Sigma_{j'} y_{j'}.D_{j'} 
  \and \\
  := \; \Sigma_{i} \Sigma_{i'} \quotep{\stackrel{\vee}{x_{i}}| \stackrel{\vee}{y_{i'}}}.(A_{i}\otimes B_{i'}) \; | \; \Sigma_{i'} \Sigma_{i} \quotep{\stackrel{\vee}{y_{i'}}|\stackrel{\vee}{x_{i}}}.(B_{i'}\otimes A_{i})
  \and
  \;\; | \;\; \Sigma_{j} \Sigma_{j'} \quotep{\stackrel{\vee}{x_{j}}|\stackrel{\vee}{y_{j'}}}.(A_{j}\otimes B_{j'}) \; | \; \Sigma_{j'} \Sigma_{j} \quotep{\stackrel{\vee}{y_{j'}}|\stackrel{\vee}{x_{j}}}.(B_{j'}\otimes A_{j})
\end{mathpar}

\begin{remark}
  Do we need to $x^{L}$ and $y^{R}$ for this construction as well?
\end{remark}

\paragraph{Tensor of parallel compositions} Next, we distribute tensor
over par.

\begin{mathpar}
  P_{1}|P_{2} \otimes Q_{1}|Q_{2} := (P_{1} \otimes Q_{1}) | (P_{1}
  \otimes Q_{2}) | (P_{2} \otimes Q_{1}) | (P_{2} \otimes Q_{2})
\end{mathpar}

\paragraph{Tensor with dropped names} We treat tensor of a
process with a dropped name as parallel composition.

\begin{mathpar}
  P \otimes \dropn{x} := P | \dropn{x}
\end{mathpar}

\paragraph{Tensor of agents}

Finally, we need to define tensor on agents. Note that the definition
of tensor on normal products only tensors inputs with inputs and
outputs with outputs. Thus, we only have to define the operation on
``homogeneous'' pairings.

\begin{mathpar}
  (\vec{x})P \otimes (\vec{y})Q
  \and \\
  := (x_{0}^{L}|y_{0}^{R},\ldots,x_{0}^{L}|y_{n}^{R},\ldots,x_{m}^{L}|y_{0}^{R},\ldots,x_{m}^{L}|y_{n}^R)(P\{ \vec{x}^{L}/\vec{x}\} \otimes Q \{ \vec{y}^{R}/\vec{y}\})
  \and \\
  \clift{\vec{P}} \otimes \clift{\vec{Q}}
  \and \\
  := \clift{P_{0}\otimes Q_{0},\ldots,P_{0}\otimes Q_{n},\ldots,P_{m}\otimes Q_{0},\ldots,P_{m}\otimes Q_{n}}
\end{mathpar}

\begin{remark}
  Observe that arities of tensored abstractions matches arities of
  tensored concretions if the original arities matched. Note also that
  the length of the arities corresponds to the increase in dimension
  we see in ordinary vector space tensor product.
\end{remark}

\begin{remark}
  Operationally, this definition distributes the tensor down to
  components ``linked'' by summation. Tensor over summation is
  intriguing in that it mixes names. Moreover, as a consequence of the
  way it mixes names we have the identities for all $x \in \QProc$ and
  $P,Q \in \Proc$

  \begin{mathpar}
    (x \cdot P) \otimes Q \equiv x \cdot (P \otimes Q) \equiv P \otimes (x \cdot Q)
    \and
    P \otimes \pzero \equiv P
  \end{mathpar}

  that the reader is invited to verify.
\end{remark}

\subsubsection{Annihilation}
\begin{mathpar}
  P^{\perp} := \{ Q | \forall R. P|Q \red^{*} R \Rightarrow R \red^{*} \pzero \}
  \and \\
  P^{\underline{\perp}} := \Sigma_{Q \in P^{\perp}} \quotep{Q}?(y).(\dropn{y}|Q) | \Sigma_{Q \in P^{\perp}} \quotep{Q}\clift{\Box}
\end{mathpar}

\paragraph{Discussion} The reader will note that $P^{\perp}$ is a
\emph{set} of processes, while $P^{\underline{\perp}}$ is a
\emph{context}. We call the set $P^{\perp}$ the \emph{annihilators} of
$P$. The parallel composition of a process in the annihilators of $P$
with $P$ will result in a process, the state space of which has all
paths eventually leading to $\pzero$. Execution may endure loops; but
under reasonable conditions of fairness (naturally guaranteed under
most notions of bisimulation) such a composite process cannot get
stuck in such a loop and will, eventually pop out and terminate.

The context $P^{\underline{\perp}}$ is ready and willing to ``take the
$P$ out of'' the process to which it is applied. It will effectively
transmit the code of the process to which it is applied to one of the
annihilators and run the process against it.

\subsubsection{Evaluation}
We fix $M$ a domain of fully abstract interpretation with an equality
coincident with bisimulation. We take $\meaningof{\cdot} : \Proc \to
M$ to be the map interpreting processes and $\nmeaningof{\cdot} : \M
\to Proc$ to be the map running the other way. Then we define

\begin{mathpar}
  \int P := \nmeaningof{\meaningof{P}}
\end{mathpar}

\paragraph{Discussion}
There are many fully abstract interpretations of Milner's
$\pi$-calculus. Any of them can be used as a basis for interpreting
the reflective calculus here. Equipped with such a domain it is
largely a matter of grinding through to check that the Yoneda
construction for the normalization-by-evaluation program can be
extended to this setting.

\begin{remark}
  The reader is invited to verify that $\int (P^{\underline{\perp}}[P]) = 0$.
\end{remark}

\subsection{Quantum mechanics}

Table \ref{tbl:core_qm_op_defns} gives the core operational definitions

\begin{table}[htp]\label{tbl:core_qm_op_defns}
  \center{
    \fbox{
      \begin{tabular}{c|c}
        quantum mechanics & process calculus \\
        \hline
        scalar & $x := \quotep{P}$ \\
        state vector & $\state{P} := P$ \\
        dual & $\state{P}^{*} := \event{P^{\underline{\perp}}} := \quotep{P^{\underline{\perp}}}[-]$ \\
        matrix & $ \Sigma_{\alpha} \state{P_{\alpha}}x_{\alpha}\event{Q_{\alpha}}$ \\
        vector addition & $\state{P} + \state{Q} := \state{P | Q}$ \\
        tensor product & $\state{P} \otimes \state{Q} := \state{P \otimes Q}$ \\
        inner product & $\innerprod{P}{Q} := \quotep{\int P^{\underline{\perp}}[Q]}$ \\
      \end{tabular}
    }
  }
  \caption{QM - operational definitions}
\end{table}

where

\begin{mathpar}
  \prmatrix{P}{Q} := \fprmatrix{P}{\quotep{\pzero}}{Q}
  \and
  \fprmatrix{P}{x}{Q} := (\state{P},x,\event{Q})
  \and
  (\fprmatrix{P}{x}{Q})(\state{R}) := x \cdot \innerprod{Q}{R} \cdot \state{P}
  \and
  (\fprmatrix{P}{x}{Q})(\event{R}) := x \cdot \innerprod{R}{P} \cdot \event{Q}
\end{mathpar}

\paragraph{Discussion}
As promised: vectors (aka states) are represented as processes; duals
as contextual duals; inner product definition should be compared with
standard inner product definition for ....

\begin{remark}
  Assuming $\int (P^{\underline{\perp}}[P]) = 0$, the reader is
  invited to verify that $(\fprmatrix{P}{x}{P})(\state{P}) = x \cdot \state{P}$.
\end{remark}

\begin{remark}
  The reader is invited to verify that $\innerprod{P}{Q}$ could
  equally well have been written $\quotep{\int \stackrel{\vee}{x}}$
  where $x = \event{P^{\underline{\perp}}}(Q)$.

  One of the motivations for this remark is that there is another way
  to factor these operations. We could package up evaluation in the dual:

  \begin{mathpar}
    \state{P}^{*} := \event{\int P^{\underline{\perp}}} := \quotep{\int P^{\underline{\perp}}}[-]
  \end{mathpar}

  and then have inner product defined by
  
  \begin{mathpar}
    \innerprod{P}{Q} := \event{P}(Q)
  \end{mathpar}

  Hopefully, experience with the calculations will provide guidance on
  the best factoring.
\end{remark}

\begin{remark}
  Assuming $\int (P^{\underline{\perp}}[P]) = 0$, the reader is
  invited to verify that $\forall P,Q. (\prmatrix{0}{Q})(\state{0}) =
  \state{0}$ and dually $(\prmatrix{P}{0})(\event{0}) = \event{0}$.
\end{remark}

\begin{remark}
  i'm a little worried that i don't (yet) have proper support for
  complex conjugacy. But, the observation above may give us a
  clue. According to Abramsky, it must be the case that the scalars
  are iso to the homset of the identity for the tensor -- which the
  observation above characterizes. 

  For now, we will simply bookmark the notion with $\overline{x}$.
\end{remark}

\subsubsection{Adjointness}

We need to give a definition of $(\cdot)^{\dagger}$ for matrices. The
obvious candidate definition is
\begin{mathpar}
(\Sigma_{\alpha}\fprmatrix{P_{\alpha}}{x_{\alpha}}{Q_{\alpha}})^{\dagger}
= \Sigma_{\alpha}\fprmatrix{(Q_{\alpha}^{\underline{\perp}})^{*}}{\overline{x}_{\alpha}}{P_{\alpha}^{\underline{\perp}}} 
\end{mathpar}

But, $(Q_{\alpha}^{\underline{\perp}})^{*}$ requires a name along
which to communicate the process to achieve the context application.

\subsubsection{Basis for a basis}
If processes label states and ``addition'' of states (a.k.a. vector
addition) is interpreted as parallel composition, what corresponds to
notions of linear independence and basis? Here, we recall that Yoshida
has developed a set of \emph{combinators} for an asynchronous verison
of Milner's $\pi$-calculus. These are a finite set of processes such
any process can be expressed as parallel composition of these
combinators together with liberal uses of the new operator and
replication. We can simply give a translation of these into the
present calculus and have reasonable expectation that the property
carries over. That is, that the resultant set allows to express all
processes via parallel composition. Note, however, that there is no
new operator or replication in this calculus. As a result, we expect
that the corresponding set is actually infinite. That is, we expect
that the space is actually infinite dimensional.

\begin{remark}
  The attentive reader may be a bit concerned. Certainly, the
  collection $S$, $K$ and $I$ is a finite set of
  combinators. Shouldn't we expect to see a finite set of combinators
  for an effectively equivalent system? i am very sympathetic to this
  critique and feel it warrants full attention. On the other hand, i
  also have in mind the following analogy. The natural numbers, as a
  monoid under addition, has exactly $1$ generator, while the natural
  numbers, as a monoid under multiplication, has countably many
  generators (the primes). We observe that the application of the
  lambda calculus is much less resource sensitive than the parallel
  composition of the $\pi$-calculus. Could it be the case that we have
  an analogy of the form
  
  \begin{mathpar}
    m + n : MN :: m*n : M|N
  \end{mathpar}

  giving a similar blow up in the set of ``primes''?  This is such a
  wonderful thought that, even if it's not true, i think it's worth
  writing down.
\end{remark}
 

\documentclass[12pt]{llncs}
%\documentclass{jktr}

\usepackage[pdftex]{hyperref}                   
\usepackage {listings}
\usepackage {mathpartir}
\usepackage{bcprules}
%\usepackage{listings}
                       
\usepackage{graphicx} 
%\usepackage[margins=2.5cm,nohead,nofoot]{geometry}
%\usepackage{geometry}
\usepackage{amsfonts}
\usepackage{amstext}
\usepackage{latexsym}
\usepackage{amssymb}
\usepackage{color}


%\include{myPreamble}
\include{qm2pi.local} 

%\ifpdf
%\usepackage[pdftex]{graphicx}
%\else
%\usepackage{graphicx}
%\fi

 % \ifpdf
%  \usepackage{pdfsync}
%  \if


%\title{Brief Article}
%\author{David F. Snyder}
%\author{L.G. Meredith}

%\address{Dept. of Math., Texas State University--San Marcos, San Marcos, TX 78666}
       
\pagestyle{empty}


\begin{document}

\lstset{language=[Objective]Caml,frame=shadowbox}

\input{qm2pi.front}

% section front matter (end)

\input{qm2pi.intro} 
 
% section introduction (end)

% \input{qm2pi.knotations} 

% section notation (end)

\input{qm2pi.process.calculi} 

% section concurrent_process_calculi_and_spatial_logics_ (end)
    
%\input{qm2pi.knots2pi} 

%\input{qm2pi.trefoil} 

%\input{qm2pi.mainthm} 

% subsection basic_interpretation (end)

%\input{qm2pi.rho.presentation} 
\subsection{The syntax and semantics of the notation system}\label{sub:the_syntax_and_semantics_of_the_notation_system} % (fold)

We now summarize a technical presentation of the calculus that
embodies our theory of dynamics. The typical presentation of such a
calculus follows the style of giving generators and relations on
them. The grammar, below, describing term constructors, freely
generates the set of processes, $\Proc$. This set is then quotiented
by a relation known as structural congruence and it is over this set
that the notion of dynamics is expressed. This presentation is
essentially that of \cite{MeredithR05} with the addition of
polyadicity and summation. For readability we have relegated some of
the technical subtleties to an appendix.

\subsubsection{Process grammar}\label{subsub:process_grammar}

\begin{mathpar}
  \inferrule* [lab=synchronization] {} {{M} \bc \pzero \;|\; x?F \;|\; x!C }
  \and
  \inferrule* [lab=abstraction] {} {{F} \bc (x)P}
  \and
  \inferrule* [lab=concretion] {} {{C} \bc \langle Q \rangle}
  \and
  \inferrule* [lab=process] {} {{P,Q} \bc M \;| \;P|Q \;|\; @{x}}
  \and
  \inferrule* [lab=name] {} {{x} \bc \quotep{P}}
\end{mathpar} 

Note that $\vec{x}$ (resp. $\vec{P}$) denotes a vector of names
(resp. processes) of length $|\vec{x}|$ (resp. $|\vec{P}|$). We adopt
the following useful abbreviations.

\begin{mathpar}
   x?(\vec{y}).P := x.(\vec{y})P \and  x\clift{\vec{P}} := x.\clift{\vec{P}}
   \and x!(y) := \lift{x}{\dropn{y}}
   \and \Pi_{i=0}^{n-1}P_i := P_0 | \ldots | P_{n-1}
\end{mathpar}

\subsubsection{Structural congruence}

\paragraph{Free and bound names and alpha-equivalence.} At the
core of structural equivalence is alpha-equivalence which identifies
process that are the same up to a change of variable. Formally, we
recognize the distinction between free and bound names. The free names
of a process, $\freenames{P}$, may be calculated recursively as
follows:

\begin{mathpar}
\freenames{\pzero} := \emptyset
  \and \\
  \freenames{x?(y).P} := \{ x \} \cup (\freenames{P} \setminus \{ y \})
  \and 
  \freenames{x!\langle P \rangle} := \{ x \} \cup \{ P \} 
  \and \\
  \freenames{P|Q} := \freenames{P} \cup \freenames{Q}
  \and \\
  \freenames{@{x}} := \{ x \}
\end{mathpar}

$\pi$
$\quotep{\pi}$

$\freenames{-} : \pi \to \mathcal{P}(\quotep{\pi})$

\begin{eqnarray*}
  \freenames{\pzero} & := & \emptyset \\
  \freenames{x?(y).P} & := & \{ x \} \cup (\freenames{P} \setminus \{ y \}) \\
  \freenames{x!\langle P \rangle} & := & \{ x \} \cup \{ P \} \\
  \freenames{P|Q} & := & \freenames{P} \cup \freenames{Q} \\
  \freenames{\dropn{x}} & := & \{ x \}
\end{eqnarray*}

The bound names of a process, $\boundnames{P}$, are those names occurring in $P$
that are not free. For example, in $x?(y).0$, the name $x$ is free, while $y$ is bound.

\begin{mathpar}
  \inferrule* [lab=monoidal-laws] {} { P|Q \equiv Q|P \and P|0 \equiv P \and P|(Q|R) \equiv (P|Q)|R }
\end{mathpar}

\begin{mathpar}
  \inferrule* [lab=alpha-equivalence] {} { (x)P \equiv (y)P\{y/x\} \and y \not\in \freenames{P} }
\end{mathpar}

\begin{definition}
Then two processes, $P,Q$, are alpha-equivalent if $P = Q\{\vec{y}/\vec{x}\}$ for
some $\vec{x} \in \boundnames{Q},\vec{y} \in \boundnames{P}$, where $Q\{\vec{y}/\vec{x}\}$
denotes the capture-avoiding substitution of $\vec{y}$ for $\vec{x}$ in $Q$.
\end{definition}

\begin{definition}
  The {\em structural congruence} \cite{SangiorgiWalker} , $\equiv$,
  between processes is the least congruence containing
  alpha-equivalence, satisfying the abelian monoid laws
  (associativity, commutativity and $\pzero$ as identity) for parallel
  composition $|$ and for summation $+$.
\end{definition}

\subsection{Name equivalence}

We take name equivalence, written $\nameeq$, to be the smallest
equivalence relation generated by the following rules.

\begin{mathpar}
\inferrule*[lab=Quote-drop]
{ }
{ \quotep{@{x}} \nameeq x }

\inferrule*[lab=Struct-equiv]
{ P \scong Q }
{ \quotep{P} \nameeq \quotep{Q} }
\end{mathpar}

The astute reader will have noticed that the mutual recursion of names
and processes imposes a mutual recursion on alpha-equivalence and
structural equivalence via name-equivalence. Fortunately, all of this
works out pleasantly and we may calculate in the natural way, free of
concern. The reader interested in the details is referred to the
appendix \ref{appendix:rho_details}.

\subsection{Substitution}

We use $\Proc$ for the set of processes, $\QProc$ for the set of
names, and $\id{\{}\vec{y} / \vec{x} \id{\}}$ to denote partial maps,
$s : \QProc \rightarrow \QProc$. A map, $s$ lifts, uniquely, to a map
on process terms, $\widehat{s} : \Proc \rightarrow \Proc$ by the
following equations.

\begin{mathpar}
  (0) \psubstp{Q}{P} := 0 \\
  (R \juxtap S) \psubstp{Q}{P}
  :=    
  (R)\psubstp{Q}{P} \juxtap (S) \psubstp{Q}{P} \\
  (x?(y).R) \psubstp{Q}{P}    
  :=    
  (x)\substp{Q}{P} (z)\concat( (R \psubstn{z}{y}) \psubstp{Q}{P} ) \\
  (\lift{x}{R}) \psubstp{Q}{P}  
  :=
  \lift{(x)\substp{Q}{P}}{ R \psubstp{Q}{P} } \\
%   (\dropn{x})  \psubstp{Q}{P}       
%   := 
%   \left\{ 
%     \begin{array}{ccc} 
%       \dropn{\quotep{Q}} & & x \nameeq \quotep{P} \\
%       \dropn{x} & & otherwise \\
%     \end{array}
%   \right. 
  (\dropn{x})  \psubstp{Q}{P}       
  := 
  \left\{ 
    \begin{array}{ccc} 
      Q & & x \nameeq \quotep{P} \\
      \dropn{x} & & otherwise \\
    \end{array}
  \right.
\end{mathpar}
 

where

\begin{eqnarray}
  (x)\id{\{} \lpquote Q \rpquote / \lpquote P \rpquote \id{\}}            = 
  \left\{ 
    \begin{array}{ccc}
      \lpquote Q \rpquote & & x \nameeq \lpquote P \rpquote \\
      x & & otherwise \\
    \end{array}
  \right. \nonumber
\end{eqnarray}

and $z$ is chosen distinct from $\quotep{P}$, $\quotep{Q}$, the free
names in $Q$, and all the names in $R$. Our $\alpha$-equivalence will
be built in the standard way from this substitution.

\begin{remark}\label{rem:no_self_referential_names}
  One consequence of these definitions is that $\forall P. \quotep{P}
  \not\in \freenames{P}$.
\end{remark}

\subsection{ Dynamic quote: an example }

Anticipating something of what's to come, consider applying the
substitution, $\widehat{\id{\{}u / z \id{\}}}$, to the following pair
of processes, $\lift{w}{y!(z)}$ and $w[ \lpquote y!(z) \rpquote ]$.

\begin{eqnarray}
	\lift{w}{y!(z)}\widehat{\id{\{}u / z \id{\}}}
		& = &
		\lift{w}{y!(u)} \nonumber\\
	w[ \lpquote y!(z) \rpquote ] \widehat{ \id{\{}u / z \id{\}} }
		& = &
		w[ \lpquote y!(z) \rpquote ] \nonumber
\end{eqnarray}

Because the body of the process between quotes is impervious to
substitution, we get radically different answers. In fact, by
examining the first process in an input context,
e.g. $x?(z).\lift{w}{y!(z)}$, we see that the process under the lift
operator may be shaped by prefixed inputs binding a name inside it. In
this sense, the lift operator will be seen as a way to dynamically
construct processes before reifying them as names.

Finally equipped with these standard features we can present the
dynamics of the calculus.

\subsubsection{Operational semantics} 

Finally, we introduce the computational dynamics. What marks these
algebras as distinct from other more traditionally studied algebraic
structures, e.g. vector spaces or polynomial rings, is the manner in
which dynamics is captured. In traditional structures, dynamics is typically
expressed through morphisms between such structures, as in linear maps
between vector spaces or morphisms between rings. In algebras
associated with the semantics of computation, the dynamics is
expressed as part of the algebraic structure itself, through a
reduction reduction relation typically denoted by $\red$. Below, we
give a recursive presentation of this relation for the calculus used
in the encoding.

$\red \subseteq \pi \times \pi$
$\red : \pi \to \mathcal{P}(\pi)$

\begin{mathpar}
  \inferrule* [lab=Comm] { \textsf{match}( x_{src}, x_{trgt} ) } { x_{trgt}?(y)P \; | \; x_{src}!\langle {Q} \rangle \red P\{\quotep{Q}/y}\} }
  \and \\
  \inferrule* [lab=Par] {{P} \red {P}'} {{{P} | {Q}} \red {{P}' | {Q}}}
  \and
  \inferrule* [lab=Equiv]{{{P} \scong {P}'} \andalso {{P}' \red {Q}'} \andalso {{Q}' \scong {Q}}}{{P} \red {Q}}
\end{mathpar}

\begin{eqnarray*}
  match_{\equiv} (\quotep{P},\quotep{Q}) & := & P \equiv Q \\
  match_{\dagger}(\quotep{P},\quotep{Q}) & := & \forall R. P|Q \red^{*} R => R \red^{*} 0 \\
  match_{K}(\quotep{P},\quotep{Q}) & := & K \mbox{ for some context } K
\end{eqnarray*}

$u?(x)P | u!\langle Q \rangle \red P\{\quotep{Q}/x\}$

%We write $\wred$ for $\red^*$, and $P\red$ if $\exists Q $ such that $ P \red Q$.
We write $P\red$ if $\exists Q $ such that $ P \red Q$ and $P\not\red$, otherwise.

\section{Replication}

As mentioned before, it is known that replication (and hence
recursion) can be implemented in a higher-order process algebra
\cite{SangiorgiWalker}. As our first example of calculation with the
machinery thus far presented we give the construction explicitly in
the {\rhoc}.

\begin{eqnarray}
	D_{x} & := & \prefix{x}{y}{(\binpar{\outputp{x}{y}}{@{y}})} \nonumber\\
	\bangp_{x}{P} & := & \binpar{{x}!\langle{\binpar{D_{x}}{P}}\rangle}{D_{x}} \nonumber
\end{eqnarray}

\begin{eqnarray}
	\bangp_{x}{P} & & \nonumber\\
	=
	& {x}!\langle{(\prefix{x}{y}{(\outputp{x}{y} | @{y})) | P}}\rangle 
	      | \prefix{x}{y}{(\outputp{x}{y} | @{y})} & \nonumber\\
	\red
	& (\outputp{x}{y} | @{y})\substn{\quotep{(\prefix{x}{y}{(@{y} | \outputp{x}{y})) | P}}}{y} & \nonumber\\
	=
	& \outputp{x}{\quotep{(\prefix{x}{y}{(\outputp{x}{y} | @{y})) | P}}}
	  | {(\prefix{x}{y}{(\outputp{x}{y} | @{y})) | P}} & \nonumber\\
	\red
	& \ldots & \nonumber\\
	\red^*
	& P | P | \ldots & \nonumber
\end{eqnarray}

Of course, this encoding, as an implementation, runs away, unfolding
$\bangp{P}$ eagerly. A lazier and more implementable replication
operator, restricted to input-guarded processes, may be obtained as follows.

\begin{eqnarray}
\bangp{\prefix{u}{v}{P}} 
	:= 
	\binpar{\lift{x}{\prefix{u}{v}{(\binpar{D(x)}{P})}}}{D(x)} \nonumber
\end{eqnarray}

\begin{remark}
  Note that the lazier definition still does not deal with summation
  or mixed summation (i.e. sums over input and output). The reader is
  invited to construct definitions of replication that deal with these
  features. 

  Further, the definitions are parameterized in a name, $x$. Can you,
  gentle reader, make a definition that eliminates this parameter and
  guarantees no accidental interaction between the replication
  machinery and the process being replicated -- i.e. no accidental
  sharing of names used by the process to get its work done and the
  name(s) used by the replication to effect copying. This latter
  revision of the definition of replication is crucial to obtaining
  the expected identity $!!P \sim !P$.
\end{remark}

\begin{remark}\label{rem:paradoxical_combinator}
  The reader familiar with the lambda calculus will have noticed the
  similarity between $D$ and the paradoxical combinator.

  [Ed. note: the existence of this seems to suggest we have to be more
  restrictive on the set of processes and names we admit if we are to
  support no-cloning.]
\end{remark}

\subsubsection{Bisimulation}

The computational dynamics gives rise to another kind of equivalence,
the equivalence of computational behavior. As previously mentioned
this is typically captured \emph{via} some form of bisimulation.

% The notion we use in this paper is weak barbed bisimulation
% \cite{milner91polyadicpi}.

The notion we use in this paper is derived from weak barbed
bisimulation \cite{milner91polyadicpi}. 

\begin{definition}
An \emph{observation relation}, $\downarrow_{\mathcal N}$, over a set
of names, $\mathcal N$, is the smallest relation satisfying the rules
below.

\infrule[Out-barb]{y \in {\mathcal N}, \; x \nameeq y}
		  {\outputp{x}{v} \downarrow_{\mathcal N} x}
\infrule[Par-barb]{\mbox{$P\downarrow_{\mathcal N} x$ or $Q\downarrow_{\mathcal N} x$}}
		  {\binpar{P}{Q} \downarrow_{\mathcal N} x}

We write $P \Downarrow_{\mathcal N} x$ if there is $Q$ such that 
$P \wred Q$ and $Q \downarrow_{\mathcal N} x$.
\end{definition}

\begin{definition}
%\label{def.bbisim}
An  ${\mathcal N}$-\emph{barbed bisimulation} over a set of names, ${\mathcal N}$, is a symmetric binary relation 
${\mathcal S}_{\mathcal N}$ between agents such that $P\rel{S}_{\mathcal N}Q$ implies:
\begin{enumerate}
\item If $P \red P'$ then $Q \wred Q'$ and $P'\rel{S}_{\mathcal N} Q'$.
\item If $P\downarrow_{\mathcal N} x$, then $Q\Downarrow_{\mathcal N} x$.
\end{enumerate}
$P$ is ${\mathcal N}$-barbed bisimilar to $Q$, written
$P \wbbisim_{\mathcal N} Q$, if $P \rel{S}_{\mathcal N} Q$ for some ${\mathcal N}$-barbed bisimulation ${\mathcal S}_{\mathcal N}$.
\end{definition}

$\mathcal{R} \subseteq \pi \times \pi$

$P \mathcal{R} Q => \forall P'. P \red P' \Rightarrow \exists Q'. Q \red Q', P' \mathcal{R} Q'$

$P \vdash x \Rightarrow Q \vdash x$

\begin{mathpar}
  \inferrule*[lab=Out-barb]{x \nameeq y}{{y}!\langle{Q}\rangle \vdash x}
  \and
  \inferrule*[lab=Par-barb]{\mbox{$P\vdash x$ or $Q\vdash x$}}{\binpar{P}{Q} \vdash x}
\end{mathpar}

\subsubsection{Contexts}

One of the principle advantages of computational calculi like the
$\pi$-calculus is a well-defined notion of context,
contextual-equivalence and a correlation between
contextual-equivalence and notions of bisimulation. The notion of
context allows the decomposition of a process into (sub-)process and
its syntactic environment, its context. Thus, a context may be
thought of as a process with a ``hole'' (written $\Box$) in it. The
application of a context $M$ to a process $P$, written $M[P]$, is
tantamount to filling the hole in $M$ with $P$. In this paper we do
not need the full weight of this theory, but do make use of the notion
of context in the proof the main theorem. 

\begin{mathpar}
  \inferrule* [lab=summation] {} {{M_{M},M_{N}} \bc \Box \;|\; x.M_{A} \;|\; M_{M}+M_{N}}
  \and
  \inferrule* [lab=agent] {} {{M_{A}} \bc (\vec{x})M_{P} \;| \; \clift{P_0,\ldots,M_{P},\ldots,P_N}}
  \and \\
  \inferrule* [lab=process] {} {{M_{P}} \bc M_{N} \;| \;P|M_{P} }
\end{mathpar} 

\begin{mathpar}
  \inferrule* [lab=sychronization] {} {M_{N} \bc \Box \;|\; x?M_{F} \;|\; x!M_{C}}
  \and
  \inferrule* [lab=abstraction] {} {{M_{F}} \bc (x)M_{P} }
  \and
  \inferrule* [lab=concretion] {} {{M_{C}} \bc \langle M_{P} \rangle }
  \and \\
  \inferrule* [lab=process] {} {{M_{P}} \bc M_{N} \;| \;P|M_{P} }
\end{mathpar}

\begin{definition}[contextual application] Given a context $M$, and
  process $P$, we define the \emph{contextual application}, $M[P] :=
  M\{P/\Box\}$. That is, the contextual application of M to P is the
  substitution of $P$ for $\Box$ in $M$.
\end{definition}

$\meaningof{-} : L \to \mathcal{P}(\pi)$

\begin{mathpar}
  \inferrule* [lab=collection] {} {\meaningof{true} = \pi, \and \meaningof{~E} = \pi \setminus \meaningof{E}, \and \meaningof{E_{1} \& E_{2}} = \meaningof{E_{1}} \cap \meaningof{E_{2}}}
\end{mathpar}

\begin{mathpar}
  \inferrule* [lab=structure] {} {\meaningof{0} = \{ P \in \pi | P \equiv 0 \}, \and \\ \meaningof{E_1 | E_2} = \{ P \in \pi | P \equiv P_{1} | P_{2}, P_{1} \in \meaningof{E_{1}}, P_{2} \in \meaningof{E_2}\} }
\end{mathpar}

\begin{mathpar}
 \inferrule* [lab=behavior] {} {\meaningof{\langle a?b \rangle E} = \{ P \in \pi | P \equiv Q | u?(y)P', \\ \and \\\\ \and \\ \;\;\; u \in \meaningof{a}, \forall z.P'\{z/y\} \in \meaningof{E\{z/b\}}\}, \and \\ \meaningof{a!E} = \{ P \in \pi | P \equiv Q | x!\langle P' \rangle, x \in \meaningof{a} P' \in \meaningof{E}\} }
\end{mathpar}

\begin{mathpar}
 \inferrule* [lab=nominal] {} {\meaningof{\quotep{E}} = \{ \quotep{P} \in \quotep{\pi} | P \in \meaningof{E} \}, \and \meaningof{\quotep{P}} = \{ \quotep{Q} \in \quotep{\pi} | P \equiv Q \} \and \\ \meaningof{@\quotep{E}} = \{ P \in \pi | P \equiv @x, x \in \meaningof{E} \}}
\end{mathpar}

\begin{eqnarray*}
  \\
  \meaningof{-} : TS \to ST
\end{eqnarray*}

\begin{eqnarray*}
  \\
  L : TS \to ST
\end{eqnarray*}

\begin{eqnarray*}
  \\
  P \models E \iff P \in \meaningof{E}
\end{eqnarray*}

\begin{eqnarray*}
  P \approx_{L} Q \iff \forall E \in L. P \models E \iff Q \models E
\end{eqnarray*}

\begin{eqnarray*}
  P \approx_{K} Q
\end{eqnarray*}

\begin{eqnarray*}
  P \approx Q
\end{eqnarray*}

$\approx_{K} = \approx = \approx_{L}$

\subsubsection{Contextual duality}

Note that contexts extend the quotation operation to a family of
operations from processes to names. Given a context, $M$, we can
define a \emph{nominal context}, $\quotep{M}$ by $\quotep{M}[P] :=
\quotep{M[P]}$. To foreshadow what is to come we observe that these
operations enjoy a duality with processes very much like the duality
between vectors and maps from vectors to scalars.

Further, because the calculus is essentially higher-order, we have a
correspondence between contexts and processes. More specifically,
given a name $x$ and a context $M$ we can construct $M^{*}_{x}$ such
that 

\begin{mathpar}
  M^{*}_{x} | \lift{x}{P} \red M[P]
\end{mathpar}

namely,

\begin{mathpar}
  M^{*}_{x} := x?(u).M[\dropn{u}]
\end{mathpar}

The dependence of $M^{*}_{x}$ on a name makes it an abstraction, 

\begin{mathpar}
  M^{*} := (x)x?(u).M[\dropn{u}]
\end{mathpar}

\subsection{Additional notation}

It will sometimes be convenient to denote the process a name
quotes. We already have the notation $x = \quotep{P}$, but it will be
convenient to introduce an alternate notation, $\procn{x}$, when we
want to emphasize the connection to the use of the name. Note that, by
virtue of name equivalence, $\quotep{\procn{x}} \nameeq x$; so, the
notation is consistent with previous definitions.

Further, because names have structure it is possible to effect
substitutions on the basis of that structure. This means we need to
upgrade our notation for substitutions, which we accomplish by
adapting comprehension notation. Thus,

\begin{mathpar}
  P\{ y / x : x \in S \}
\end{mathpar}

is interpreted to mean the process derived from P by replacing (in a
capture-avoiding manner) each occurrence of $x$ in $S$ by $y$. For example,

\begin{mathpar}
  P\{ \quotep{\procn{x}|\procn{x}} / x : x \in \freenames{P} \}
\end{mathpar}

will replace each (occurrence) of a free name $x$ in $P$ by
$\quotep{\procn{x}|\procn{x}}$.

Also, we will avail ourselves of the notation $x^{L}$ and $x^{R}$ to
denote injections of a name into disjoint copies of the name
space. There are numerous ways to accomplish this. One example can be
found in \cite{MeredithR05}. This notation overloads to vectors of
names: $\vec{x}^{\pi} := (x_{i}^{\pi} \; : \; 0 \leq i < |\vec{x}| )$ where $\pi \in \{L,R\}$.

We also use $P^{\Box} := P|\Box$.

In \cite{MeredithR05} an interpretation of the new operator is
given. It turns out that there are several possible interpretations
all enjoying the requisite algebraic properties of the operator (see
\cite{milner91polyadicpi}). We will therefore make liberal use of
$(\nu\; \vec{x})P$.

% subsection the_syntax_and_semantics_of_the_notation_system (end)   

\input{qm2pi.qmops} 

\input{qm2pi.sterngerlach} 

\input{qm2pi.metric} 

% section concurrent_process_calculi (end)

%\input{qm2pi.proofsketch}

% section proof sketch (end)

%\input{qm2pi.slviaknots} 

% section spatial logic via knots (end)

\input{qm2pi.conclusion}

% section conclusion (end)

%\input{qm2pi.dtcodes} 

% section wiring algorithm (end)

\input{qm2pi.ack} 

% section acknowledgments (end)

\newpage


\bibliographystyle{plain}   
\bibliography{../../biblios/main.bib}

\input{qm2pi.rhodetails}

\end{document}

 

\documentclass[12pt]{llncs}
%\documentclass{jktr}

\usepackage[pdftex]{hyperref}                   
\usepackage {listings}
\usepackage {mathpartir}
\usepackage{bcprules}
%\usepackage{listings}
                       
\usepackage{graphicx} 
%\usepackage[margins=2.5cm,nohead,nofoot]{geometry}
%\usepackage{geometry}
\usepackage{amsfonts}
\usepackage{amstext}
\usepackage{latexsym}
\usepackage{amssymb}
\usepackage{color}


%\include{myPreamble}
\include{qm2pi.local} 

%\ifpdf
%\usepackage[pdftex]{graphicx}
%\else
%\usepackage{graphicx}
%\fi

 % \ifpdf
%  \usepackage{pdfsync}
%  \if


%\title{Brief Article}
%\author{David F. Snyder}
%\author{L.G. Meredith}

%\address{Dept. of Math., Texas State University--San Marcos, San Marcos, TX 78666}
       
\pagestyle{empty}


\begin{document}

\lstset{language=[Objective]Caml,frame=shadowbox}

\input{qm2pi.front}

% section front matter (end)

\input{qm2pi.intro} 
 
% section introduction (end)

% \input{qm2pi.knotations} 

% section notation (end)

\input{qm2pi.process.calculi} 

% section concurrent_process_calculi_and_spatial_logics_ (end)
    
%\input{qm2pi.knots2pi} 

%\input{qm2pi.trefoil} 

%\input{qm2pi.mainthm} 

% subsection basic_interpretation (end)

%\input{qm2pi.rho.presentation} 
\subsection{The syntax and semantics of the notation system}\label{sub:the_syntax_and_semantics_of_the_notation_system} % (fold)

We now summarize a technical presentation of the calculus that
embodies our theory of dynamics. The typical presentation of such a
calculus follows the style of giving generators and relations on
them. The grammar, below, describing term constructors, freely
generates the set of processes, $\Proc$. This set is then quotiented
by a relation known as structural congruence and it is over this set
that the notion of dynamics is expressed. This presentation is
essentially that of \cite{MeredithR05} with the addition of
polyadicity and summation. For readability we have relegated some of
the technical subtleties to an appendix.

\subsubsection{Process grammar}\label{subsub:process_grammar}

\begin{mathpar}
  \inferrule* [lab=synchronization] {} {{M} \bc \pzero \;|\; x?F \;|\; x!C }
  \and
  \inferrule* [lab=abstraction] {} {{F} \bc (x)P}
  \and
  \inferrule* [lab=concretion] {} {{C} \bc \langle Q \rangle}
  \and
  \inferrule* [lab=process] {} {{P,Q} \bc M \;| \;P|Q \;|\; @{x}}
  \and
  \inferrule* [lab=name] {} {{x} \bc \quotep{P}}
\end{mathpar} 

Note that $\vec{x}$ (resp. $\vec{P}$) denotes a vector of names
(resp. processes) of length $|\vec{x}|$ (resp. $|\vec{P}|$). We adopt
the following useful abbreviations.

\begin{mathpar}
   x?(\vec{y}).P := x.(\vec{y})P \and  x\clift{\vec{P}} := x.\clift{\vec{P}}
   \and x!(y) := \lift{x}{\dropn{y}}
   \and \Pi_{i=0}^{n-1}P_i := P_0 | \ldots | P_{n-1}
\end{mathpar}

\subsubsection{Structural congruence}

\paragraph{Free and bound names and alpha-equivalence.} At the
core of structural equivalence is alpha-equivalence which identifies
process that are the same up to a change of variable. Formally, we
recognize the distinction between free and bound names. The free names
of a process, $\freenames{P}$, may be calculated recursively as
follows:

\begin{mathpar}
\freenames{\pzero} := \emptyset
  \and \\
  \freenames{x?(y).P} := \{ x \} \cup (\freenames{P} \setminus \{ y \})
  \and 
  \freenames{x!\langle P \rangle} := \{ x \} \cup \{ P \} 
  \and \\
  \freenames{P|Q} := \freenames{P} \cup \freenames{Q}
  \and \\
  \freenames{@{x}} := \{ x \}
\end{mathpar}

$\pi$
$\quotep{\pi}$

$\freenames{-} : \pi \to \mathcal{P}(\quotep{\pi})$

\begin{eqnarray*}
  \freenames{\pzero} & := & \emptyset \\
  \freenames{x?(y).P} & := & \{ x \} \cup (\freenames{P} \setminus \{ y \}) \\
  \freenames{x!\langle P \rangle} & := & \{ x \} \cup \{ P \} \\
  \freenames{P|Q} & := & \freenames{P} \cup \freenames{Q} \\
  \freenames{\dropn{x}} & := & \{ x \}
\end{eqnarray*}

The bound names of a process, $\boundnames{P}$, are those names occurring in $P$
that are not free. For example, in $x?(y).0$, the name $x$ is free, while $y$ is bound.

\begin{mathpar}
  \inferrule* [lab=monoidal-laws] {} { P|Q \equiv Q|P \and P|0 \equiv P \and P|(Q|R) \equiv (P|Q)|R }
\end{mathpar}

\begin{mathpar}
  \inferrule* [lab=alpha-equivalence] {} { (x)P \equiv (y)P\{y/x\} \and y \not\in \freenames{P} }
\end{mathpar}

\begin{definition}
Then two processes, $P,Q$, are alpha-equivalent if $P = Q\{\vec{y}/\vec{x}\}$ for
some $\vec{x} \in \boundnames{Q},\vec{y} \in \boundnames{P}$, where $Q\{\vec{y}/\vec{x}\}$
denotes the capture-avoiding substitution of $\vec{y}$ for $\vec{x}$ in $Q$.
\end{definition}

\begin{definition}
  The {\em structural congruence} \cite{SangiorgiWalker} , $\equiv$,
  between processes is the least congruence containing
  alpha-equivalence, satisfying the abelian monoid laws
  (associativity, commutativity and $\pzero$ as identity) for parallel
  composition $|$ and for summation $+$.
\end{definition}

\subsection{Name equivalence}

We take name equivalence, written $\nameeq$, to be the smallest
equivalence relation generated by the following rules.

\begin{mathpar}
\inferrule*[lab=Quote-drop]
{ }
{ \quotep{@{x}} \nameeq x }

\inferrule*[lab=Struct-equiv]
{ P \scong Q }
{ \quotep{P} \nameeq \quotep{Q} }
\end{mathpar}

The astute reader will have noticed that the mutual recursion of names
and processes imposes a mutual recursion on alpha-equivalence and
structural equivalence via name-equivalence. Fortunately, all of this
works out pleasantly and we may calculate in the natural way, free of
concern. The reader interested in the details is referred to the
appendix \ref{appendix:rho_details}.

\subsection{Substitution}

We use $\Proc$ for the set of processes, $\QProc$ for the set of
names, and $\id{\{}\vec{y} / \vec{x} \id{\}}$ to denote partial maps,
$s : \QProc \rightarrow \QProc$. A map, $s$ lifts, uniquely, to a map
on process terms, $\widehat{s} : \Proc \rightarrow \Proc$ by the
following equations.

\begin{mathpar}
  (0) \psubstp{Q}{P} := 0 \\
  (R \juxtap S) \psubstp{Q}{P}
  :=    
  (R)\psubstp{Q}{P} \juxtap (S) \psubstp{Q}{P} \\
  (x?(y).R) \psubstp{Q}{P}    
  :=    
  (x)\substp{Q}{P} (z)\concat( (R \psubstn{z}{y}) \psubstp{Q}{P} ) \\
  (\lift{x}{R}) \psubstp{Q}{P}  
  :=
  \lift{(x)\substp{Q}{P}}{ R \psubstp{Q}{P} } \\
%   (\dropn{x})  \psubstp{Q}{P}       
%   := 
%   \left\{ 
%     \begin{array}{ccc} 
%       \dropn{\quotep{Q}} & & x \nameeq \quotep{P} \\
%       \dropn{x} & & otherwise \\
%     \end{array}
%   \right. 
  (\dropn{x})  \psubstp{Q}{P}       
  := 
  \left\{ 
    \begin{array}{ccc} 
      Q & & x \nameeq \quotep{P} \\
      \dropn{x} & & otherwise \\
    \end{array}
  \right.
\end{mathpar}
 

where

\begin{eqnarray}
  (x)\id{\{} \lpquote Q \rpquote / \lpquote P \rpquote \id{\}}            = 
  \left\{ 
    \begin{array}{ccc}
      \lpquote Q \rpquote & & x \nameeq \lpquote P \rpquote \\
      x & & otherwise \\
    \end{array}
  \right. \nonumber
\end{eqnarray}

and $z$ is chosen distinct from $\quotep{P}$, $\quotep{Q}$, the free
names in $Q$, and all the names in $R$. Our $\alpha$-equivalence will
be built in the standard way from this substitution.

\begin{remark}\label{rem:no_self_referential_names}
  One consequence of these definitions is that $\forall P. \quotep{P}
  \not\in \freenames{P}$.
\end{remark}

\subsection{ Dynamic quote: an example }

Anticipating something of what's to come, consider applying the
substitution, $\widehat{\id{\{}u / z \id{\}}}$, to the following pair
of processes, $\lift{w}{y!(z)}$ and $w[ \lpquote y!(z) \rpquote ]$.

\begin{eqnarray}
	\lift{w}{y!(z)}\widehat{\id{\{}u / z \id{\}}}
		& = &
		\lift{w}{y!(u)} \nonumber\\
	w[ \lpquote y!(z) \rpquote ] \widehat{ \id{\{}u / z \id{\}} }
		& = &
		w[ \lpquote y!(z) \rpquote ] \nonumber
\end{eqnarray}

Because the body of the process between quotes is impervious to
substitution, we get radically different answers. In fact, by
examining the first process in an input context,
e.g. $x?(z).\lift{w}{y!(z)}$, we see that the process under the lift
operator may be shaped by prefixed inputs binding a name inside it. In
this sense, the lift operator will be seen as a way to dynamically
construct processes before reifying them as names.

Finally equipped with these standard features we can present the
dynamics of the calculus.

\subsubsection{Operational semantics} 

Finally, we introduce the computational dynamics. What marks these
algebras as distinct from other more traditionally studied algebraic
structures, e.g. vector spaces or polynomial rings, is the manner in
which dynamics is captured. In traditional structures, dynamics is typically
expressed through morphisms between such structures, as in linear maps
between vector spaces or morphisms between rings. In algebras
associated with the semantics of computation, the dynamics is
expressed as part of the algebraic structure itself, through a
reduction reduction relation typically denoted by $\red$. Below, we
give a recursive presentation of this relation for the calculus used
in the encoding.

$\red \subseteq \pi \times \pi$
$\red : \pi \to \mathcal{P}(\pi)$

\begin{mathpar}
  \inferrule* [lab=Comm] { \textsf{match}( x_{src}, x_{trgt} ) } { x_{trgt}?(y)P \; | \; x_{src}!\langle {Q} \rangle \red P\{\quotep{Q}/y}\} }
  \and \\
  \inferrule* [lab=Par] {{P} \red {P}'} {{{P} | {Q}} \red {{P}' | {Q}}}
  \and
  \inferrule* [lab=Equiv]{{{P} \scong {P}'} \andalso {{P}' \red {Q}'} \andalso {{Q}' \scong {Q}}}{{P} \red {Q}}
\end{mathpar}

\begin{eqnarray*}
  match_{\equiv} (\quotep{P},\quotep{Q}) & := & P \equiv Q \\
  match_{\dagger}(\quotep{P},\quotep{Q}) & := & \forall R. P|Q \red^{*} R => R \red^{*} 0 \\
  match_{K}(\quotep{P},\quotep{Q}) & := & K \mbox{ for some context } K
\end{eqnarray*}

$u?(x)P | u!\langle Q \rangle \red P\{\quotep{Q}/x\}$

%We write $\wred$ for $\red^*$, and $P\red$ if $\exists Q $ such that $ P \red Q$.
We write $P\red$ if $\exists Q $ such that $ P \red Q$ and $P\not\red$, otherwise.

\section{Replication}

As mentioned before, it is known that replication (and hence
recursion) can be implemented in a higher-order process algebra
\cite{SangiorgiWalker}. As our first example of calculation with the
machinery thus far presented we give the construction explicitly in
the {\rhoc}.

\begin{eqnarray}
	D_{x} & := & \prefix{x}{y}{(\binpar{\outputp{x}{y}}{@{y}})} \nonumber\\
	\bangp_{x}{P} & := & \binpar{{x}!\langle{\binpar{D_{x}}{P}}\rangle}{D_{x}} \nonumber
\end{eqnarray}

\begin{eqnarray}
	\bangp_{x}{P} & & \nonumber\\
	=
	& {x}!\langle{(\prefix{x}{y}{(\outputp{x}{y} | @{y})) | P}}\rangle 
	      | \prefix{x}{y}{(\outputp{x}{y} | @{y})} & \nonumber\\
	\red
	& (\outputp{x}{y} | @{y})\substn{\quotep{(\prefix{x}{y}{(@{y} | \outputp{x}{y})) | P}}}{y} & \nonumber\\
	=
	& \outputp{x}{\quotep{(\prefix{x}{y}{(\outputp{x}{y} | @{y})) | P}}}
	  | {(\prefix{x}{y}{(\outputp{x}{y} | @{y})) | P}} & \nonumber\\
	\red
	& \ldots & \nonumber\\
	\red^*
	& P | P | \ldots & \nonumber
\end{eqnarray}

Of course, this encoding, as an implementation, runs away, unfolding
$\bangp{P}$ eagerly. A lazier and more implementable replication
operator, restricted to input-guarded processes, may be obtained as follows.

\begin{eqnarray}
\bangp{\prefix{u}{v}{P}} 
	:= 
	\binpar{\lift{x}{\prefix{u}{v}{(\binpar{D(x)}{P})}}}{D(x)} \nonumber
\end{eqnarray}

\begin{remark}
  Note that the lazier definition still does not deal with summation
  or mixed summation (i.e. sums over input and output). The reader is
  invited to construct definitions of replication that deal with these
  features. 

  Further, the definitions are parameterized in a name, $x$. Can you,
  gentle reader, make a definition that eliminates this parameter and
  guarantees no accidental interaction between the replication
  machinery and the process being replicated -- i.e. no accidental
  sharing of names used by the process to get its work done and the
  name(s) used by the replication to effect copying. This latter
  revision of the definition of replication is crucial to obtaining
  the expected identity $!!P \sim !P$.
\end{remark}

\begin{remark}\label{rem:paradoxical_combinator}
  The reader familiar with the lambda calculus will have noticed the
  similarity between $D$ and the paradoxical combinator.

  [Ed. note: the existence of this seems to suggest we have to be more
  restrictive on the set of processes and names we admit if we are to
  support no-cloning.]
\end{remark}

\subsubsection{Bisimulation}

The computational dynamics gives rise to another kind of equivalence,
the equivalence of computational behavior. As previously mentioned
this is typically captured \emph{via} some form of bisimulation.

% The notion we use in this paper is weak barbed bisimulation
% \cite{milner91polyadicpi}.

The notion we use in this paper is derived from weak barbed
bisimulation \cite{milner91polyadicpi}. 

\begin{definition}
An \emph{observation relation}, $\downarrow_{\mathcal N}$, over a set
of names, $\mathcal N$, is the smallest relation satisfying the rules
below.

\infrule[Out-barb]{y \in {\mathcal N}, \; x \nameeq y}
		  {\outputp{x}{v} \downarrow_{\mathcal N} x}
\infrule[Par-barb]{\mbox{$P\downarrow_{\mathcal N} x$ or $Q\downarrow_{\mathcal N} x$}}
		  {\binpar{P}{Q} \downarrow_{\mathcal N} x}

We write $P \Downarrow_{\mathcal N} x$ if there is $Q$ such that 
$P \wred Q$ and $Q \downarrow_{\mathcal N} x$.
\end{definition}

\begin{definition}
%\label{def.bbisim}
An  ${\mathcal N}$-\emph{barbed bisimulation} over a set of names, ${\mathcal N}$, is a symmetric binary relation 
${\mathcal S}_{\mathcal N}$ between agents such that $P\rel{S}_{\mathcal N}Q$ implies:
\begin{enumerate}
\item If $P \red P'$ then $Q \wred Q'$ and $P'\rel{S}_{\mathcal N} Q'$.
\item If $P\downarrow_{\mathcal N} x$, then $Q\Downarrow_{\mathcal N} x$.
\end{enumerate}
$P$ is ${\mathcal N}$-barbed bisimilar to $Q$, written
$P \wbbisim_{\mathcal N} Q$, if $P \rel{S}_{\mathcal N} Q$ for some ${\mathcal N}$-barbed bisimulation ${\mathcal S}_{\mathcal N}$.
\end{definition}

$\mathcal{R} \subseteq \pi \times \pi$

$P \mathcal{R} Q => \forall P'. P \red P' \Rightarrow \exists Q'. Q \red Q', P' \mathcal{R} Q'$

$P \vdash x \Rightarrow Q \vdash x$

\begin{mathpar}
  \inferrule*[lab=Out-barb]{x \nameeq y}{{y}!\langle{Q}\rangle \vdash x}
  \and
  \inferrule*[lab=Par-barb]{\mbox{$P\vdash x$ or $Q\vdash x$}}{\binpar{P}{Q} \vdash x}
\end{mathpar}

\subsubsection{Contexts}

One of the principle advantages of computational calculi like the
$\pi$-calculus is a well-defined notion of context,
contextual-equivalence and a correlation between
contextual-equivalence and notions of bisimulation. The notion of
context allows the decomposition of a process into (sub-)process and
its syntactic environment, its context. Thus, a context may be
thought of as a process with a ``hole'' (written $\Box$) in it. The
application of a context $M$ to a process $P$, written $M[P]$, is
tantamount to filling the hole in $M$ with $P$. In this paper we do
not need the full weight of this theory, but do make use of the notion
of context in the proof the main theorem. 

\begin{mathpar}
  \inferrule* [lab=summation] {} {{M_{M},M_{N}} \bc \Box \;|\; x.M_{A} \;|\; M_{M}+M_{N}}
  \and
  \inferrule* [lab=agent] {} {{M_{A}} \bc (\vec{x})M_{P} \;| \; \clift{P_0,\ldots,M_{P},\ldots,P_N}}
  \and \\
  \inferrule* [lab=process] {} {{M_{P}} \bc M_{N} \;| \;P|M_{P} }
\end{mathpar} 

\begin{mathpar}
  \inferrule* [lab=sychronization] {} {M_{N} \bc \Box \;|\; x?M_{F} \;|\; x!M_{C}}
  \and
  \inferrule* [lab=abstraction] {} {{M_{F}} \bc (x)M_{P} }
  \and
  \inferrule* [lab=concretion] {} {{M_{C}} \bc \langle M_{P} \rangle }
  \and \\
  \inferrule* [lab=process] {} {{M_{P}} \bc M_{N} \;| \;P|M_{P} }
\end{mathpar}

\begin{definition}[contextual application] Given a context $M$, and
  process $P$, we define the \emph{contextual application}, $M[P] :=
  M\{P/\Box\}$. That is, the contextual application of M to P is the
  substitution of $P$ for $\Box$ in $M$.
\end{definition}

$\meaningof{-} : L \to \mathcal{P}(\pi)$

\begin{mathpar}
  \inferrule* [lab=collection] {} {\meaningof{true} = \pi, \and \meaningof{~E} = \pi \setminus \meaningof{E}, \and \meaningof{E_{1} \& E_{2}} = \meaningof{E_{1}} \cap \meaningof{E_{2}}}
\end{mathpar}

\begin{mathpar}
  \inferrule* [lab=structure] {} {\meaningof{0} = \{ P \in \pi | P \equiv 0 \}, \and \\ \meaningof{E_1 | E_2} = \{ P \in \pi | P \equiv P_{1} | P_{2}, P_{1} \in \meaningof{E_{1}}, P_{2} \in \meaningof{E_2}\} }
\end{mathpar}

\begin{mathpar}
 \inferrule* [lab=behavior] {} {\meaningof{\langle a?b \rangle E} = \{ P \in \pi | P \equiv Q | u?(y)P', \\ \and \\\\ \and \\ \;\;\; u \in \meaningof{a}, \forall z.P'\{z/y\} \in \meaningof{E\{z/b\}}\}, \and \\ \meaningof{a!E} = \{ P \in \pi | P \equiv Q | x!\langle P' \rangle, x \in \meaningof{a} P' \in \meaningof{E}\} }
\end{mathpar}

\begin{mathpar}
 \inferrule* [lab=nominal] {} {\meaningof{\quotep{E}} = \{ \quotep{P} \in \quotep{\pi} | P \in \meaningof{E} \}, \and \meaningof{\quotep{P}} = \{ \quotep{Q} \in \quotep{\pi} | P \equiv Q \} \and \\ \meaningof{@\quotep{E}} = \{ P \in \pi | P \equiv @x, x \in \meaningof{E} \}}
\end{mathpar}

\begin{eqnarray*}
  \\
  \meaningof{-} : TS \to ST
\end{eqnarray*}

\begin{eqnarray*}
  \\
  L : TS \to ST
\end{eqnarray*}

\begin{eqnarray*}
  \\
  P \models E \iff P \in \meaningof{E}
\end{eqnarray*}

\begin{eqnarray*}
  P \approx_{L} Q \iff \forall E \in L. P \models E \iff Q \models E
\end{eqnarray*}

\begin{eqnarray*}
  P \approx_{K} Q
\end{eqnarray*}

\begin{eqnarray*}
  P \approx Q
\end{eqnarray*}

$\approx_{K} = \approx = \approx_{L}$

\subsubsection{Contextual duality}

Note that contexts extend the quotation operation to a family of
operations from processes to names. Given a context, $M$, we can
define a \emph{nominal context}, $\quotep{M}$ by $\quotep{M}[P] :=
\quotep{M[P]}$. To foreshadow what is to come we observe that these
operations enjoy a duality with processes very much like the duality
between vectors and maps from vectors to scalars.

Further, because the calculus is essentially higher-order, we have a
correspondence between contexts and processes. More specifically,
given a name $x$ and a context $M$ we can construct $M^{*}_{x}$ such
that 

\begin{mathpar}
  M^{*}_{x} | \lift{x}{P} \red M[P]
\end{mathpar}

namely,

\begin{mathpar}
  M^{*}_{x} := x?(u).M[\dropn{u}]
\end{mathpar}

The dependence of $M^{*}_{x}$ on a name makes it an abstraction, 

\begin{mathpar}
  M^{*} := (x)x?(u).M[\dropn{u}]
\end{mathpar}

\subsection{Additional notation}

It will sometimes be convenient to denote the process a name
quotes. We already have the notation $x = \quotep{P}$, but it will be
convenient to introduce an alternate notation, $\procn{x}$, when we
want to emphasize the connection to the use of the name. Note that, by
virtue of name equivalence, $\quotep{\procn{x}} \nameeq x$; so, the
notation is consistent with previous definitions.

Further, because names have structure it is possible to effect
substitutions on the basis of that structure. This means we need to
upgrade our notation for substitutions, which we accomplish by
adapting comprehension notation. Thus,

\begin{mathpar}
  P\{ y / x : x \in S \}
\end{mathpar}

is interpreted to mean the process derived from P by replacing (in a
capture-avoiding manner) each occurrence of $x$ in $S$ by $y$. For example,

\begin{mathpar}
  P\{ \quotep{\procn{x}|\procn{x}} / x : x \in \freenames{P} \}
\end{mathpar}

will replace each (occurrence) of a free name $x$ in $P$ by
$\quotep{\procn{x}|\procn{x}}$.

Also, we will avail ourselves of the notation $x^{L}$ and $x^{R}$ to
denote injections of a name into disjoint copies of the name
space. There are numerous ways to accomplish this. One example can be
found in \cite{MeredithR05}. This notation overloads to vectors of
names: $\vec{x}^{\pi} := (x_{i}^{\pi} \; : \; 0 \leq i < |\vec{x}| )$ where $\pi \in \{L,R\}$.

We also use $P^{\Box} := P|\Box$.

In \cite{MeredithR05} an interpretation of the new operator is
given. It turns out that there are several possible interpretations
all enjoying the requisite algebraic properties of the operator (see
\cite{milner91polyadicpi}). We will therefore make liberal use of
$(\nu\; \vec{x})P$.

% subsection the_syntax_and_semantics_of_the_notation_system (end)   

\input{qm2pi.qmops} 

\input{qm2pi.sterngerlach} 

\input{qm2pi.metric} 

% section concurrent_process_calculi (end)

%\input{qm2pi.proofsketch}

% section proof sketch (end)

%\input{qm2pi.slviaknots} 

% section spatial logic via knots (end)

\input{qm2pi.conclusion}

% section conclusion (end)

%\input{qm2pi.dtcodes} 

% section wiring algorithm (end)

\input{qm2pi.ack} 

% section acknowledgments (end)

\newpage


\bibliographystyle{plain}   
\bibliography{../../biblios/main.bib}

\input{qm2pi.rhodetails}

\end{document}

 

% section concurrent_process_calculi (end)

%\documentclass[12pt]{llncs}
%\documentclass{jktr}

\usepackage[pdftex]{hyperref}                   
\usepackage {listings}
\usepackage {mathpartir}
\usepackage{bcprules}
%\usepackage{listings}
                       
\usepackage{graphicx} 
%\usepackage[margins=2.5cm,nohead,nofoot]{geometry}
%\usepackage{geometry}
\usepackage{amsfonts}
\usepackage{amstext}
\usepackage{latexsym}
\usepackage{amssymb}
\usepackage{color}


%\include{myPreamble}
\include{qm2pi.local} 

%\ifpdf
%\usepackage[pdftex]{graphicx}
%\else
%\usepackage{graphicx}
%\fi

 % \ifpdf
%  \usepackage{pdfsync}
%  \if


%\title{Brief Article}
%\author{David F. Snyder}
%\author{L.G. Meredith}

%\address{Dept. of Math., Texas State University--San Marcos, San Marcos, TX 78666}
       
\pagestyle{empty}


\begin{document}

\lstset{language=[Objective]Caml,frame=shadowbox}

\input{qm2pi.front}

% section front matter (end)

\input{qm2pi.intro} 
 
% section introduction (end)

% \input{qm2pi.knotations} 

% section notation (end)

\input{qm2pi.process.calculi} 

% section concurrent_process_calculi_and_spatial_logics_ (end)
    
%\input{qm2pi.knots2pi} 

%\input{qm2pi.trefoil} 

%\input{qm2pi.mainthm} 

% subsection basic_interpretation (end)

%\input{qm2pi.rho.presentation} 
\subsection{The syntax and semantics of the notation system}\label{sub:the_syntax_and_semantics_of_the_notation_system} % (fold)

We now summarize a technical presentation of the calculus that
embodies our theory of dynamics. The typical presentation of such a
calculus follows the style of giving generators and relations on
them. The grammar, below, describing term constructors, freely
generates the set of processes, $\Proc$. This set is then quotiented
by a relation known as structural congruence and it is over this set
that the notion of dynamics is expressed. This presentation is
essentially that of \cite{MeredithR05} with the addition of
polyadicity and summation. For readability we have relegated some of
the technical subtleties to an appendix.

\subsubsection{Process grammar}\label{subsub:process_grammar}

\begin{mathpar}
  \inferrule* [lab=synchronization] {} {{M} \bc \pzero \;|\; x?F \;|\; x!C }
  \and
  \inferrule* [lab=abstraction] {} {{F} \bc (x)P}
  \and
  \inferrule* [lab=concretion] {} {{C} \bc \langle Q \rangle}
  \and
  \inferrule* [lab=process] {} {{P,Q} \bc M \;| \;P|Q \;|\; @{x}}
  \and
  \inferrule* [lab=name] {} {{x} \bc \quotep{P}}
\end{mathpar} 

Note that $\vec{x}$ (resp. $\vec{P}$) denotes a vector of names
(resp. processes) of length $|\vec{x}|$ (resp. $|\vec{P}|$). We adopt
the following useful abbreviations.

\begin{mathpar}
   x?(\vec{y}).P := x.(\vec{y})P \and  x\clift{\vec{P}} := x.\clift{\vec{P}}
   \and x!(y) := \lift{x}{\dropn{y}}
   \and \Pi_{i=0}^{n-1}P_i := P_0 | \ldots | P_{n-1}
\end{mathpar}

\subsubsection{Structural congruence}

\paragraph{Free and bound names and alpha-equivalence.} At the
core of structural equivalence is alpha-equivalence which identifies
process that are the same up to a change of variable. Formally, we
recognize the distinction between free and bound names. The free names
of a process, $\freenames{P}$, may be calculated recursively as
follows:

\begin{mathpar}
\freenames{\pzero} := \emptyset
  \and \\
  \freenames{x?(y).P} := \{ x \} \cup (\freenames{P} \setminus \{ y \})
  \and 
  \freenames{x!\langle P \rangle} := \{ x \} \cup \{ P \} 
  \and \\
  \freenames{P|Q} := \freenames{P} \cup \freenames{Q}
  \and \\
  \freenames{@{x}} := \{ x \}
\end{mathpar}

$\pi$
$\quotep{\pi}$

$\freenames{-} : \pi \to \mathcal{P}(\quotep{\pi})$

\begin{eqnarray*}
  \freenames{\pzero} & := & \emptyset \\
  \freenames{x?(y).P} & := & \{ x \} \cup (\freenames{P} \setminus \{ y \}) \\
  \freenames{x!\langle P \rangle} & := & \{ x \} \cup \{ P \} \\
  \freenames{P|Q} & := & \freenames{P} \cup \freenames{Q} \\
  \freenames{\dropn{x}} & := & \{ x \}
\end{eqnarray*}

The bound names of a process, $\boundnames{P}$, are those names occurring in $P$
that are not free. For example, in $x?(y).0$, the name $x$ is free, while $y$ is bound.

\begin{mathpar}
  \inferrule* [lab=monoidal-laws] {} { P|Q \equiv Q|P \and P|0 \equiv P \and P|(Q|R) \equiv (P|Q)|R }
\end{mathpar}

\begin{mathpar}
  \inferrule* [lab=alpha-equivalence] {} { (x)P \equiv (y)P\{y/x\} \and y \not\in \freenames{P} }
\end{mathpar}

\begin{definition}
Then two processes, $P,Q$, are alpha-equivalent if $P = Q\{\vec{y}/\vec{x}\}$ for
some $\vec{x} \in \boundnames{Q},\vec{y} \in \boundnames{P}$, where $Q\{\vec{y}/\vec{x}\}$
denotes the capture-avoiding substitution of $\vec{y}$ for $\vec{x}$ in $Q$.
\end{definition}

\begin{definition}
  The {\em structural congruence} \cite{SangiorgiWalker} , $\equiv$,
  between processes is the least congruence containing
  alpha-equivalence, satisfying the abelian monoid laws
  (associativity, commutativity and $\pzero$ as identity) for parallel
  composition $|$ and for summation $+$.
\end{definition}

\subsection{Name equivalence}

We take name equivalence, written $\nameeq$, to be the smallest
equivalence relation generated by the following rules.

\begin{mathpar}
\inferrule*[lab=Quote-drop]
{ }
{ \quotep{@{x}} \nameeq x }

\inferrule*[lab=Struct-equiv]
{ P \scong Q }
{ \quotep{P} \nameeq \quotep{Q} }
\end{mathpar}

The astute reader will have noticed that the mutual recursion of names
and processes imposes a mutual recursion on alpha-equivalence and
structural equivalence via name-equivalence. Fortunately, all of this
works out pleasantly and we may calculate in the natural way, free of
concern. The reader interested in the details is referred to the
appendix \ref{appendix:rho_details}.

\subsection{Substitution}

We use $\Proc$ for the set of processes, $\QProc$ for the set of
names, and $\id{\{}\vec{y} / \vec{x} \id{\}}$ to denote partial maps,
$s : \QProc \rightarrow \QProc$. A map, $s$ lifts, uniquely, to a map
on process terms, $\widehat{s} : \Proc \rightarrow \Proc$ by the
following equations.

\begin{mathpar}
  (0) \psubstp{Q}{P} := 0 \\
  (R \juxtap S) \psubstp{Q}{P}
  :=    
  (R)\psubstp{Q}{P} \juxtap (S) \psubstp{Q}{P} \\
  (x?(y).R) \psubstp{Q}{P}    
  :=    
  (x)\substp{Q}{P} (z)\concat( (R \psubstn{z}{y}) \psubstp{Q}{P} ) \\
  (\lift{x}{R}) \psubstp{Q}{P}  
  :=
  \lift{(x)\substp{Q}{P}}{ R \psubstp{Q}{P} } \\
%   (\dropn{x})  \psubstp{Q}{P}       
%   := 
%   \left\{ 
%     \begin{array}{ccc} 
%       \dropn{\quotep{Q}} & & x \nameeq \quotep{P} \\
%       \dropn{x} & & otherwise \\
%     \end{array}
%   \right. 
  (\dropn{x})  \psubstp{Q}{P}       
  := 
  \left\{ 
    \begin{array}{ccc} 
      Q & & x \nameeq \quotep{P} \\
      \dropn{x} & & otherwise \\
    \end{array}
  \right.
\end{mathpar}
 

where

\begin{eqnarray}
  (x)\id{\{} \lpquote Q \rpquote / \lpquote P \rpquote \id{\}}            = 
  \left\{ 
    \begin{array}{ccc}
      \lpquote Q \rpquote & & x \nameeq \lpquote P \rpquote \\
      x & & otherwise \\
    \end{array}
  \right. \nonumber
\end{eqnarray}

and $z$ is chosen distinct from $\quotep{P}$, $\quotep{Q}$, the free
names in $Q$, and all the names in $R$. Our $\alpha$-equivalence will
be built in the standard way from this substitution.

\begin{remark}\label{rem:no_self_referential_names}
  One consequence of these definitions is that $\forall P. \quotep{P}
  \not\in \freenames{P}$.
\end{remark}

\subsection{ Dynamic quote: an example }

Anticipating something of what's to come, consider applying the
substitution, $\widehat{\id{\{}u / z \id{\}}}$, to the following pair
of processes, $\lift{w}{y!(z)}$ and $w[ \lpquote y!(z) \rpquote ]$.

\begin{eqnarray}
	\lift{w}{y!(z)}\widehat{\id{\{}u / z \id{\}}}
		& = &
		\lift{w}{y!(u)} \nonumber\\
	w[ \lpquote y!(z) \rpquote ] \widehat{ \id{\{}u / z \id{\}} }
		& = &
		w[ \lpquote y!(z) \rpquote ] \nonumber
\end{eqnarray}

Because the body of the process between quotes is impervious to
substitution, we get radically different answers. In fact, by
examining the first process in an input context,
e.g. $x?(z).\lift{w}{y!(z)}$, we see that the process under the lift
operator may be shaped by prefixed inputs binding a name inside it. In
this sense, the lift operator will be seen as a way to dynamically
construct processes before reifying them as names.

Finally equipped with these standard features we can present the
dynamics of the calculus.

\subsubsection{Operational semantics} 

Finally, we introduce the computational dynamics. What marks these
algebras as distinct from other more traditionally studied algebraic
structures, e.g. vector spaces or polynomial rings, is the manner in
which dynamics is captured. In traditional structures, dynamics is typically
expressed through morphisms between such structures, as in linear maps
between vector spaces or morphisms between rings. In algebras
associated with the semantics of computation, the dynamics is
expressed as part of the algebraic structure itself, through a
reduction reduction relation typically denoted by $\red$. Below, we
give a recursive presentation of this relation for the calculus used
in the encoding.

$\red \subseteq \pi \times \pi$
$\red : \pi \to \mathcal{P}(\pi)$

\begin{mathpar}
  \inferrule* [lab=Comm] { \textsf{match}( x_{src}, x_{trgt} ) } { x_{trgt}?(y)P \; | \; x_{src}!\langle {Q} \rangle \red P\{\quotep{Q}/y}\} }
  \and \\
  \inferrule* [lab=Par] {{P} \red {P}'} {{{P} | {Q}} \red {{P}' | {Q}}}
  \and
  \inferrule* [lab=Equiv]{{{P} \scong {P}'} \andalso {{P}' \red {Q}'} \andalso {{Q}' \scong {Q}}}{{P} \red {Q}}
\end{mathpar}

\begin{eqnarray*}
  match_{\equiv} (\quotep{P},\quotep{Q}) & := & P \equiv Q \\
  match_{\dagger}(\quotep{P},\quotep{Q}) & := & \forall R. P|Q \red^{*} R => R \red^{*} 0 \\
  match_{K}(\quotep{P},\quotep{Q}) & := & K \mbox{ for some context } K
\end{eqnarray*}

$u?(x)P | u!\langle Q \rangle \red P\{\quotep{Q}/x\}$

%We write $\wred$ for $\red^*$, and $P\red$ if $\exists Q $ such that $ P \red Q$.
We write $P\red$ if $\exists Q $ such that $ P \red Q$ and $P\not\red$, otherwise.

\section{Replication}

As mentioned before, it is known that replication (and hence
recursion) can be implemented in a higher-order process algebra
\cite{SangiorgiWalker}. As our first example of calculation with the
machinery thus far presented we give the construction explicitly in
the {\rhoc}.

\begin{eqnarray}
	D_{x} & := & \prefix{x}{y}{(\binpar{\outputp{x}{y}}{@{y}})} \nonumber\\
	\bangp_{x}{P} & := & \binpar{{x}!\langle{\binpar{D_{x}}{P}}\rangle}{D_{x}} \nonumber
\end{eqnarray}

\begin{eqnarray}
	\bangp_{x}{P} & & \nonumber\\
	=
	& {x}!\langle{(\prefix{x}{y}{(\outputp{x}{y} | @{y})) | P}}\rangle 
	      | \prefix{x}{y}{(\outputp{x}{y} | @{y})} & \nonumber\\
	\red
	& (\outputp{x}{y} | @{y})\substn{\quotep{(\prefix{x}{y}{(@{y} | \outputp{x}{y})) | P}}}{y} & \nonumber\\
	=
	& \outputp{x}{\quotep{(\prefix{x}{y}{(\outputp{x}{y} | @{y})) | P}}}
	  | {(\prefix{x}{y}{(\outputp{x}{y} | @{y})) | P}} & \nonumber\\
	\red
	& \ldots & \nonumber\\
	\red^*
	& P | P | \ldots & \nonumber
\end{eqnarray}

Of course, this encoding, as an implementation, runs away, unfolding
$\bangp{P}$ eagerly. A lazier and more implementable replication
operator, restricted to input-guarded processes, may be obtained as follows.

\begin{eqnarray}
\bangp{\prefix{u}{v}{P}} 
	:= 
	\binpar{\lift{x}{\prefix{u}{v}{(\binpar{D(x)}{P})}}}{D(x)} \nonumber
\end{eqnarray}

\begin{remark}
  Note that the lazier definition still does not deal with summation
  or mixed summation (i.e. sums over input and output). The reader is
  invited to construct definitions of replication that deal with these
  features. 

  Further, the definitions are parameterized in a name, $x$. Can you,
  gentle reader, make a definition that eliminates this parameter and
  guarantees no accidental interaction between the replication
  machinery and the process being replicated -- i.e. no accidental
  sharing of names used by the process to get its work done and the
  name(s) used by the replication to effect copying. This latter
  revision of the definition of replication is crucial to obtaining
  the expected identity $!!P \sim !P$.
\end{remark}

\begin{remark}\label{rem:paradoxical_combinator}
  The reader familiar with the lambda calculus will have noticed the
  similarity between $D$ and the paradoxical combinator.

  [Ed. note: the existence of this seems to suggest we have to be more
  restrictive on the set of processes and names we admit if we are to
  support no-cloning.]
\end{remark}

\subsubsection{Bisimulation}

The computational dynamics gives rise to another kind of equivalence,
the equivalence of computational behavior. As previously mentioned
this is typically captured \emph{via} some form of bisimulation.

% The notion we use in this paper is weak barbed bisimulation
% \cite{milner91polyadicpi}.

The notion we use in this paper is derived from weak barbed
bisimulation \cite{milner91polyadicpi}. 

\begin{definition}
An \emph{observation relation}, $\downarrow_{\mathcal N}$, over a set
of names, $\mathcal N$, is the smallest relation satisfying the rules
below.

\infrule[Out-barb]{y \in {\mathcal N}, \; x \nameeq y}
		  {\outputp{x}{v} \downarrow_{\mathcal N} x}
\infrule[Par-barb]{\mbox{$P\downarrow_{\mathcal N} x$ or $Q\downarrow_{\mathcal N} x$}}
		  {\binpar{P}{Q} \downarrow_{\mathcal N} x}

We write $P \Downarrow_{\mathcal N} x$ if there is $Q$ such that 
$P \wred Q$ and $Q \downarrow_{\mathcal N} x$.
\end{definition}

\begin{definition}
%\label{def.bbisim}
An  ${\mathcal N}$-\emph{barbed bisimulation} over a set of names, ${\mathcal N}$, is a symmetric binary relation 
${\mathcal S}_{\mathcal N}$ between agents such that $P\rel{S}_{\mathcal N}Q$ implies:
\begin{enumerate}
\item If $P \red P'$ then $Q \wred Q'$ and $P'\rel{S}_{\mathcal N} Q'$.
\item If $P\downarrow_{\mathcal N} x$, then $Q\Downarrow_{\mathcal N} x$.
\end{enumerate}
$P$ is ${\mathcal N}$-barbed bisimilar to $Q$, written
$P \wbbisim_{\mathcal N} Q$, if $P \rel{S}_{\mathcal N} Q$ for some ${\mathcal N}$-barbed bisimulation ${\mathcal S}_{\mathcal N}$.
\end{definition}

$\mathcal{R} \subseteq \pi \times \pi$

$P \mathcal{R} Q => \forall P'. P \red P' \Rightarrow \exists Q'. Q \red Q', P' \mathcal{R} Q'$

$P \vdash x \Rightarrow Q \vdash x$

\begin{mathpar}
  \inferrule*[lab=Out-barb]{x \nameeq y}{{y}!\langle{Q}\rangle \vdash x}
  \and
  \inferrule*[lab=Par-barb]{\mbox{$P\vdash x$ or $Q\vdash x$}}{\binpar{P}{Q} \vdash x}
\end{mathpar}

\subsubsection{Contexts}

One of the principle advantages of computational calculi like the
$\pi$-calculus is a well-defined notion of context,
contextual-equivalence and a correlation between
contextual-equivalence and notions of bisimulation. The notion of
context allows the decomposition of a process into (sub-)process and
its syntactic environment, its context. Thus, a context may be
thought of as a process with a ``hole'' (written $\Box$) in it. The
application of a context $M$ to a process $P$, written $M[P]$, is
tantamount to filling the hole in $M$ with $P$. In this paper we do
not need the full weight of this theory, but do make use of the notion
of context in the proof the main theorem. 

\begin{mathpar}
  \inferrule* [lab=summation] {} {{M_{M},M_{N}} \bc \Box \;|\; x.M_{A} \;|\; M_{M}+M_{N}}
  \and
  \inferrule* [lab=agent] {} {{M_{A}} \bc (\vec{x})M_{P} \;| \; \clift{P_0,\ldots,M_{P},\ldots,P_N}}
  \and \\
  \inferrule* [lab=process] {} {{M_{P}} \bc M_{N} \;| \;P|M_{P} }
\end{mathpar} 

\begin{mathpar}
  \inferrule* [lab=sychronization] {} {M_{N} \bc \Box \;|\; x?M_{F} \;|\; x!M_{C}}
  \and
  \inferrule* [lab=abstraction] {} {{M_{F}} \bc (x)M_{P} }
  \and
  \inferrule* [lab=concretion] {} {{M_{C}} \bc \langle M_{P} \rangle }
  \and \\
  \inferrule* [lab=process] {} {{M_{P}} \bc M_{N} \;| \;P|M_{P} }
\end{mathpar}

\begin{definition}[contextual application] Given a context $M$, and
  process $P$, we define the \emph{contextual application}, $M[P] :=
  M\{P/\Box\}$. That is, the contextual application of M to P is the
  substitution of $P$ for $\Box$ in $M$.
\end{definition}

$\meaningof{-} : L \to \mathcal{P}(\pi)$

\begin{mathpar}
  \inferrule* [lab=collection] {} {\meaningof{true} = \pi, \and \meaningof{~E} = \pi \setminus \meaningof{E}, \and \meaningof{E_{1} \& E_{2}} = \meaningof{E_{1}} \cap \meaningof{E_{2}}}
\end{mathpar}

\begin{mathpar}
  \inferrule* [lab=structure] {} {\meaningof{0} = \{ P \in \pi | P \equiv 0 \}, \and \\ \meaningof{E_1 | E_2} = \{ P \in \pi | P \equiv P_{1} | P_{2}, P_{1} \in \meaningof{E_{1}}, P_{2} \in \meaningof{E_2}\} }
\end{mathpar}

\begin{mathpar}
 \inferrule* [lab=behavior] {} {\meaningof{\langle a?b \rangle E} = \{ P \in \pi | P \equiv Q | u?(y)P', \\ \and \\\\ \and \\ \;\;\; u \in \meaningof{a}, \forall z.P'\{z/y\} \in \meaningof{E\{z/b\}}\}, \and \\ \meaningof{a!E} = \{ P \in \pi | P \equiv Q | x!\langle P' \rangle, x \in \meaningof{a} P' \in \meaningof{E}\} }
\end{mathpar}

\begin{mathpar}
 \inferrule* [lab=nominal] {} {\meaningof{\quotep{E}} = \{ \quotep{P} \in \quotep{\pi} | P \in \meaningof{E} \}, \and \meaningof{\quotep{P}} = \{ \quotep{Q} \in \quotep{\pi} | P \equiv Q \} \and \\ \meaningof{@\quotep{E}} = \{ P \in \pi | P \equiv @x, x \in \meaningof{E} \}}
\end{mathpar}

\begin{eqnarray*}
  \\
  \meaningof{-} : TS \to ST
\end{eqnarray*}

\begin{eqnarray*}
  \\
  L : TS \to ST
\end{eqnarray*}

\begin{eqnarray*}
  \\
  P \models E \iff P \in \meaningof{E}
\end{eqnarray*}

\begin{eqnarray*}
  P \approx_{L} Q \iff \forall E \in L. P \models E \iff Q \models E
\end{eqnarray*}

\begin{eqnarray*}
  P \approx_{K} Q
\end{eqnarray*}

\begin{eqnarray*}
  P \approx Q
\end{eqnarray*}

$\approx_{K} = \approx = \approx_{L}$

\subsubsection{Contextual duality}

Note that contexts extend the quotation operation to a family of
operations from processes to names. Given a context, $M$, we can
define a \emph{nominal context}, $\quotep{M}$ by $\quotep{M}[P] :=
\quotep{M[P]}$. To foreshadow what is to come we observe that these
operations enjoy a duality with processes very much like the duality
between vectors and maps from vectors to scalars.

Further, because the calculus is essentially higher-order, we have a
correspondence between contexts and processes. More specifically,
given a name $x$ and a context $M$ we can construct $M^{*}_{x}$ such
that 

\begin{mathpar}
  M^{*}_{x} | \lift{x}{P} \red M[P]
\end{mathpar}

namely,

\begin{mathpar}
  M^{*}_{x} := x?(u).M[\dropn{u}]
\end{mathpar}

The dependence of $M^{*}_{x}$ on a name makes it an abstraction, 

\begin{mathpar}
  M^{*} := (x)x?(u).M[\dropn{u}]
\end{mathpar}

\subsection{Additional notation}

It will sometimes be convenient to denote the process a name
quotes. We already have the notation $x = \quotep{P}$, but it will be
convenient to introduce an alternate notation, $\procn{x}$, when we
want to emphasize the connection to the use of the name. Note that, by
virtue of name equivalence, $\quotep{\procn{x}} \nameeq x$; so, the
notation is consistent with previous definitions.

Further, because names have structure it is possible to effect
substitutions on the basis of that structure. This means we need to
upgrade our notation for substitutions, which we accomplish by
adapting comprehension notation. Thus,

\begin{mathpar}
  P\{ y / x : x \in S \}
\end{mathpar}

is interpreted to mean the process derived from P by replacing (in a
capture-avoiding manner) each occurrence of $x$ in $S$ by $y$. For example,

\begin{mathpar}
  P\{ \quotep{\procn{x}|\procn{x}} / x : x \in \freenames{P} \}
\end{mathpar}

will replace each (occurrence) of a free name $x$ in $P$ by
$\quotep{\procn{x}|\procn{x}}$.

Also, we will avail ourselves of the notation $x^{L}$ and $x^{R}$ to
denote injections of a name into disjoint copies of the name
space. There are numerous ways to accomplish this. One example can be
found in \cite{MeredithR05}. This notation overloads to vectors of
names: $\vec{x}^{\pi} := (x_{i}^{\pi} \; : \; 0 \leq i < |\vec{x}| )$ where $\pi \in \{L,R\}$.

We also use $P^{\Box} := P|\Box$.

In \cite{MeredithR05} an interpretation of the new operator is
given. It turns out that there are several possible interpretations
all enjoying the requisite algebraic properties of the operator (see
\cite{milner91polyadicpi}). We will therefore make liberal use of
$(\nu\; \vec{x})P$.

% subsection the_syntax_and_semantics_of_the_notation_system (end)   

\input{qm2pi.qmops} 

\input{qm2pi.sterngerlach} 

\input{qm2pi.metric} 

% section concurrent_process_calculi (end)

%\input{qm2pi.proofsketch}

% section proof sketch (end)

%\input{qm2pi.slviaknots} 

% section spatial logic via knots (end)

\input{qm2pi.conclusion}

% section conclusion (end)

%\input{qm2pi.dtcodes} 

% section wiring algorithm (end)

\input{qm2pi.ack} 

% section acknowledgments (end)

\newpage


\bibliographystyle{plain}   
\bibliography{../../biblios/main.bib}

\input{qm2pi.rhodetails}

\end{document}



% section proof sketch (end)

%\section{Unlikely characters: spatial logic for
  knots}\label{sub:characteristic_formulae} % (fold)

Associated to the mobile process calculi are a family of logics known
as the Hennessy-Milner logics. These logics typically enjoy a
semantics interpreting formulae as sets of processes that when
factored through the encoding outlined above allows an identification
of classes of knots with logical formulae. In the context of this
encoding the sub-family known as the spatial logics \cite{CairesC03}
\cite{CairesC04} \cite{Caires04} are of particular interest providing
several important features for expressing and reasoning about
properties (i.e. classes) of knots. We hint here at how this may be done.

%\begin{description}
%\item [structural connectives] 
\subsubsection{Structural connectives} The spatial logics enjoy
structural connectives corresponding, at the logical level, to the
parallel composition ($P | Q$) and new name ($(\nu \; x)P$)
connectives for processes. As illustrated in the examples below, these
connectives are extremely expressive given the shape of our encoding.
%\item [decideable satisfaction]

\subsubsection{Decideable satisfaction}
In \cite{Caires04} the satisfaction relation is shown to be decideable
for a rich class of processes. It further turns out that the image of
the our encoding is a proper subset of that class. This result
provides the basis for an algorithm by which to search for knots
enjoying a given property.
%\item [characteristic formulae]

\subsubsection{Characteristic formulae}
In the same paper \cite{Caires04} , Caires presents a means of calculating
characteristic formulae, selecting equivalence classes of processes
up to a pre--specified depth limit on the support set of names. Composed with our
encoding, this characteristic formula can be used to select
characteristic formulae for knots.
%\end{description}

\subsubsection{Spatial logic formulae}

The grammar below (segmented for comprehension) summarizes the syntax
of spatial logic formulae. We employ illustrative examples in the
sequel to provide an intuitive understanding of their meaning
referring the reader to \cite{Caires04} for a more detailed explication
of the semantics.

\begin{mathpar}
  \inferrule* [lab=boolean] {} {{A,B} \bc T \;|\; \neg A \;|\; A \wedge B \;|\; \eta = \eta'}
  \and
  \inferrule* [lab=spatial] {} {|\; \pzero \;|\; A | B \;|\; x \text{\textregistered} A \;|\; \forall x . A \;|\;  H x . A}
  \and
  \inferrule* [lab=behavioral] {} {|\; \alpha . A}
  \and 
  \inferrule* [lab=recursion] {} {|\; X(\vec{u}) \;|\; \mu X(\vec{u}) . A}
  \and
  \inferrule* [lab=action] {} {\alpha \bc \langle x?(\vec{y}) \rangle \;|\; \langle x!(\vec{y}) \rangle \;|\; \langle \tau \rangle}
  \and 
  \inferrule* [lab=name] {} {\eta \bc x \;|\; \tau}
\end{mathpar} 

% subsection characteristic_formulae (end)   	 

\subsection{Example formulae}\label{sub:example_formulae_} % (fold)

\subsubsection{Crossing as formula.}
% 
% \begin{align*}
%   \frac{d}{dx} \sin x &= \cos x 
%   & \frac{d}{dx} e^x &= e^x \\
%   \frac{d}{dx} \cos x &= - \sin x 
%   & \frac{d}{dx} \log x &= \frac{1}{x} \\
% \end{align*} 

\begin{align*}
 \mu C(x_{0},x_{1},y_{0},y_{1},u).&(\langle x_{0}?(z) \rangle(\langle u! \rangle\langle y_{1}!z \rangle C(x_{0},x_{1},y_{0},y_{1},u)) & \\
  & \wedge \langle y_{1}?(z) \rangle (\langle u! \rangle \langle x_{0}!z \rangle C(x_{0},x_{1},y_{0},y_{1},u)) & \\
  & \wedge \langle x_{1}?(z) \rangle (\langle u? \rangle \langle y_{0}!z \rangle C(x_{0},x_{1},y_{0},y_{1},u)) & \\
  & \wedge \langle y_{0}?(z) \rangle (\langle u? \rangle \langle x_{1}!z \rangle C(x_{0},x_{1},y_{0},y_{1},u))) &
\end{align*}

The lexicographical similarity between the shape of this formulae and
the shape of definition of the process representing a crossing reveals
the intuitive meaning of this formulae. It describes the capabilities
of a process that has the right to represent a crossing. For example
it picks out processes that may perform an input on the port $x_0$ in
its initial menu of capabilities. What differentiates the formula
from the process, however, is that the crossing process is the
smallest candidate to satisfy the formula. Infinitely many other
processes -- with internal behavior hidden behind this interface, so
to speak -- also satisfy this formula. Even this simple formula,
then, can be seen to open a new view onto knots, providing a
computational interpretation of \emph{virtual} knots.

Note that this formula is derived by hand. A similar formula can be
derived by employing Caires' calculation of characteristic formula
\cite{Caires04} to the process representing a crossing. In light of
this discussion, we let
$\meaningof{C}_{\phi}(x0,x1,y0,y1,u)$ denote a formula specifying the
dynamics we wish to capture of a crossing. To guarantee we preserve
the shape of the interface and minimal semantics we demand that
$\meaningof{C}_{\phi}(x0,x1,y0,y1,u) \Rightarrow
\textbf{C}(x0,x1,y0,y1,u)$ where $\textbf{C}(x0,x1,y0,y1,u)$ denotes
the formula above.
                            
\subsubsection{Crossing number constraints.}
The moral content of the context lemma (Lemma \ref{context}) is that the notion of
``locality'' in the Reidemeister moves is effectively captured by the
parallel composition operator of the process calculus. This intuition
extends through the logic. Given a formula,
$\meaningof{C}_{\phi}(x0,x1,y0,y1,u)$, we can use the structural
connectives to specify constraints on crossing numbers, such as at
least $n$ crossings, or exactly $n$ crossings.
\begin{mathpar}
  \inferrule* [lab=at-least-n] {} { K^{\geq n}_{\phi}(\vec{xs},\vec{ys}) := \Pi_{i=0}^{n-1} Hu . \meaningof{C}_{\phi}(xs_i,ys_i,u) | T }
  \and 
  \inferrule* [lab=exactly-n] {} { K^{= n}_{\phi}(\vec{xs},\vec{ys}) := \Pi_{i=0}^{n-1} Hu . \meaningof{C}_{\phi}(xs_i,ys_i,u) | \neg (\forall x_0,y_0,x_1,y_1,u . \meaningof{C}_{\phi}(x_0,y_0,x_1,y_1,u) | T) }
\end{mathpar}

To round out this section, recall that the encoding of an $n$-crossing
knot decomposes into a parallel composition of $n$ \emph{copies} of a
crossing process together with a wiring harness. To specify different
knot classes with the same crossing number amounts to specifying
logical constraints on the wiring harness. In the interest of space,
we defer examples to a forthcoming paper. Suffice it to say that both
the conditions ``alternating knot'' and ``contains the tangle
corresponding to 5/3'' are expressible. For example, it is possible to
calculate the characteristic formula of a process corresponding to the
tangle 5/3 and conjoin it into the classifying formula via the
composition connective of the logic.

Finally, we wish to observe that it is entirely within reason to
contemplate a more domain-specific version of spatial logic tailored
to the shape of processes in the image of the encoding. Such a
domain-specific logic would have a better claim to the title formal
language of knot properties.

% subsection example_formulae_ (end)

% section knots_as_processes (end) 

% section spatial logic via knots (end)

\section{Conclusions and future work}

\paragraph{Testing physical space}
You, gentle reader, may wonder why of all the theorems to be proved
given this set up we pick the one above. In some sense it's hardly
central to quantum mechanics. We see it as central in the sense that
it firmly establishes a notion of physical space arising from a notion
of the equivalence of behavior. Relating bisimulation to a metric is a
big step forward, but one is faced with interpreting the relationship
of that metric space to something more physical. Quantum mechanical
notions of ``physical'' space are still far from intuitive, but by
relating this idea of distance as testing to calculations that predict
physical circumstances we are making a not insignificant step forward
toward an understanding of the physical space we inhabit as
essentially dynamic.

\paragraph{Effectivity and simulation}
One of the observations we have yet to make is that the entire program
spelled out here is effective. We have built various interpreters for
the reflective calculus at work in this interpretation. In principle,
then, we can simulate quantum mechanics on a computer. The place where
the simulation may lose fidelity is the infinitely branching summation
for the annihilator.

In this connection i also want to point out that the evaluation style
calculation of the inner product puts the non-determinism of the
summation right at the heart of measurement. This suggests that
Milner's original reduction-based formulation of the dynamics of his
calculi in terms of sums was not just notationally suggestive of a
notion of measure-and-continue but captured some significant part of
the physics.

\paragraph{Quantum continuations}
In light of this last observation i want to point out that the
predominant account of quantum mechanics is missing a key aspect of a
truly compositional story of the physical situation. In a real lab,
when a measurement is made the observation can be made to feed into
another device that then makes another measurement conditioned on the
results of the first. This means that after the superposition was
collapsed the entire experimental set up remained in
superposition. While QM offers a means of writing this down it doesn't
quite line up well with the well-trodden formulation of computation
and continuation that we see so succinctly expressed in Milner's
calculi. This suggests that there might be advantages to this account
of dynamics waiting to be explored.

\paragraph{Quantum logic}
In this connection, we also note that by virtue of having the
Hennessy-Milner construction, we can pull the construction through the
interpretation of QM. This gives us a natural candidate for a quantum
logic that enjoys an extremely tight connection with it's domain of
interpretation, making the construction much less ad hoc (rather it is
the image of functor!).

\paragraph{Quantum probabiity}
i have questions about the basis of the interpretation of inner
product as probability amplitude. In particular, using which
axiomatization of probability theory does the notion of probability
amplitude earn the right to be so dubbed? In other words, where is the
proof that the operation for calculating a probability amplitude (and
then squaring) satisfies the axioms of what it means to calculate a
probability? Even if such a proof exists (i have yet to find it in the
literature), i wonder if it might not be possible to turn things on
their heads. Can we view the calculation of the probability amplitude
as an axiomatization of probability? If so, then the definition we
give for calculating probability amplitude may provide the basis for
an \emph{effective} theory of probability.

\paragraph{Quantum vs ``biological'' information}
Finally, i want to conclude with a more philosophical observation. At
a recent workshop in which QM was a predominant topic i noticed
something about quantum information. The speaker was giving a riveting
discussion of axiomatic QM and showing how properties of ``no
cloning'' and ``no deleting'' emerged as consequences of the
axiomatization. Theorems of this form are necessary to give us a sense
of confidence that our axioms characterize the physical theory. What
struck me, though, was that if quantum information is neither erasable
nor replicable it is markedly different from \emph{life}. Two of the
things we know about life is that

\begin{itemize}
  \item it ends;
  \item to gain some measure of persistence, to transcend it's
    finitude it is imminently copyable.
\end{itemize}

Both of these qualities are summarized succinctly in the aphorism: all
flesh is grass. For me these two kinds of ``information'' -- call them
quantum and biological -- are end points on a spectrum of strategies
for persistence. At one end, we have those curious entities that enjoy
uniqueness and permanence; at the other, we have those who in the face
of a certain end and an uncertain present make a go of passing
something on. To me one of the more remarkable aspects of the latter
strategy is that in the presence of noise (and certain features of
copying) we get a kind of dynamism, a chance for improvement against a
given persistent condition.

% subsection other_calculi_other_bisimulations_and_geometry_as_behavior (end)




% section conclusion (end)

%\documentclass[12pt]{llncs}
%\documentclass{jktr}

\usepackage[pdftex]{hyperref}                   
\usepackage {listings}
\usepackage {mathpartir}
\usepackage{bcprules}
%\usepackage{listings}
                       
\usepackage{graphicx} 
%\usepackage[margins=2.5cm,nohead,nofoot]{geometry}
%\usepackage{geometry}
\usepackage{amsfonts}
\usepackage{amstext}
\usepackage{latexsym}
\usepackage{amssymb}
\usepackage{color}


%\include{myPreamble}
\include{qm2pi.local} 

%\ifpdf
%\usepackage[pdftex]{graphicx}
%\else
%\usepackage{graphicx}
%\fi

 % \ifpdf
%  \usepackage{pdfsync}
%  \if


%\title{Brief Article}
%\author{David F. Snyder}
%\author{L.G. Meredith}

%\address{Dept. of Math., Texas State University--San Marcos, San Marcos, TX 78666}
       
\pagestyle{empty}


\begin{document}

\lstset{language=[Objective]Caml,frame=shadowbox}

\input{qm2pi.front}

% section front matter (end)

\input{qm2pi.intro} 
 
% section introduction (end)

% \input{qm2pi.knotations} 

% section notation (end)

\input{qm2pi.process.calculi} 

% section concurrent_process_calculi_and_spatial_logics_ (end)
    
%\input{qm2pi.knots2pi} 

%\input{qm2pi.trefoil} 

%\input{qm2pi.mainthm} 

% subsection basic_interpretation (end)

%\input{qm2pi.rho.presentation} 
\subsection{The syntax and semantics of the notation system}\label{sub:the_syntax_and_semantics_of_the_notation_system} % (fold)

We now summarize a technical presentation of the calculus that
embodies our theory of dynamics. The typical presentation of such a
calculus follows the style of giving generators and relations on
them. The grammar, below, describing term constructors, freely
generates the set of processes, $\Proc$. This set is then quotiented
by a relation known as structural congruence and it is over this set
that the notion of dynamics is expressed. This presentation is
essentially that of \cite{MeredithR05} with the addition of
polyadicity and summation. For readability we have relegated some of
the technical subtleties to an appendix.

\subsubsection{Process grammar}\label{subsub:process_grammar}

\begin{mathpar}
  \inferrule* [lab=synchronization] {} {{M} \bc \pzero \;|\; x?F \;|\; x!C }
  \and
  \inferrule* [lab=abstraction] {} {{F} \bc (x)P}
  \and
  \inferrule* [lab=concretion] {} {{C} \bc \langle Q \rangle}
  \and
  \inferrule* [lab=process] {} {{P,Q} \bc M \;| \;P|Q \;|\; @{x}}
  \and
  \inferrule* [lab=name] {} {{x} \bc \quotep{P}}
\end{mathpar} 

Note that $\vec{x}$ (resp. $\vec{P}$) denotes a vector of names
(resp. processes) of length $|\vec{x}|$ (resp. $|\vec{P}|$). We adopt
the following useful abbreviations.

\begin{mathpar}
   x?(\vec{y}).P := x.(\vec{y})P \and  x\clift{\vec{P}} := x.\clift{\vec{P}}
   \and x!(y) := \lift{x}{\dropn{y}}
   \and \Pi_{i=0}^{n-1}P_i := P_0 | \ldots | P_{n-1}
\end{mathpar}

\subsubsection{Structural congruence}

\paragraph{Free and bound names and alpha-equivalence.} At the
core of structural equivalence is alpha-equivalence which identifies
process that are the same up to a change of variable. Formally, we
recognize the distinction between free and bound names. The free names
of a process, $\freenames{P}$, may be calculated recursively as
follows:

\begin{mathpar}
\freenames{\pzero} := \emptyset
  \and \\
  \freenames{x?(y).P} := \{ x \} \cup (\freenames{P} \setminus \{ y \})
  \and 
  \freenames{x!\langle P \rangle} := \{ x \} \cup \{ P \} 
  \and \\
  \freenames{P|Q} := \freenames{P} \cup \freenames{Q}
  \and \\
  \freenames{@{x}} := \{ x \}
\end{mathpar}

$\pi$
$\quotep{\pi}$

$\freenames{-} : \pi \to \mathcal{P}(\quotep{\pi})$

\begin{eqnarray*}
  \freenames{\pzero} & := & \emptyset \\
  \freenames{x?(y).P} & := & \{ x \} \cup (\freenames{P} \setminus \{ y \}) \\
  \freenames{x!\langle P \rangle} & := & \{ x \} \cup \{ P \} \\
  \freenames{P|Q} & := & \freenames{P} \cup \freenames{Q} \\
  \freenames{\dropn{x}} & := & \{ x \}
\end{eqnarray*}

The bound names of a process, $\boundnames{P}$, are those names occurring in $P$
that are not free. For example, in $x?(y).0$, the name $x$ is free, while $y$ is bound.

\begin{mathpar}
  \inferrule* [lab=monoidal-laws] {} { P|Q \equiv Q|P \and P|0 \equiv P \and P|(Q|R) \equiv (P|Q)|R }
\end{mathpar}

\begin{mathpar}
  \inferrule* [lab=alpha-equivalence] {} { (x)P \equiv (y)P\{y/x\} \and y \not\in \freenames{P} }
\end{mathpar}

\begin{definition}
Then two processes, $P,Q$, are alpha-equivalent if $P = Q\{\vec{y}/\vec{x}\}$ for
some $\vec{x} \in \boundnames{Q},\vec{y} \in \boundnames{P}$, where $Q\{\vec{y}/\vec{x}\}$
denotes the capture-avoiding substitution of $\vec{y}$ for $\vec{x}$ in $Q$.
\end{definition}

\begin{definition}
  The {\em structural congruence} \cite{SangiorgiWalker} , $\equiv$,
  between processes is the least congruence containing
  alpha-equivalence, satisfying the abelian monoid laws
  (associativity, commutativity and $\pzero$ as identity) for parallel
  composition $|$ and for summation $+$.
\end{definition}

\subsection{Name equivalence}

We take name equivalence, written $\nameeq$, to be the smallest
equivalence relation generated by the following rules.

\begin{mathpar}
\inferrule*[lab=Quote-drop]
{ }
{ \quotep{@{x}} \nameeq x }

\inferrule*[lab=Struct-equiv]
{ P \scong Q }
{ \quotep{P} \nameeq \quotep{Q} }
\end{mathpar}

The astute reader will have noticed that the mutual recursion of names
and processes imposes a mutual recursion on alpha-equivalence and
structural equivalence via name-equivalence. Fortunately, all of this
works out pleasantly and we may calculate in the natural way, free of
concern. The reader interested in the details is referred to the
appendix \ref{appendix:rho_details}.

\subsection{Substitution}

We use $\Proc$ for the set of processes, $\QProc$ for the set of
names, and $\id{\{}\vec{y} / \vec{x} \id{\}}$ to denote partial maps,
$s : \QProc \rightarrow \QProc$. A map, $s$ lifts, uniquely, to a map
on process terms, $\widehat{s} : \Proc \rightarrow \Proc$ by the
following equations.

\begin{mathpar}
  (0) \psubstp{Q}{P} := 0 \\
  (R \juxtap S) \psubstp{Q}{P}
  :=    
  (R)\psubstp{Q}{P} \juxtap (S) \psubstp{Q}{P} \\
  (x?(y).R) \psubstp{Q}{P}    
  :=    
  (x)\substp{Q}{P} (z)\concat( (R \psubstn{z}{y}) \psubstp{Q}{P} ) \\
  (\lift{x}{R}) \psubstp{Q}{P}  
  :=
  \lift{(x)\substp{Q}{P}}{ R \psubstp{Q}{P} } \\
%   (\dropn{x})  \psubstp{Q}{P}       
%   := 
%   \left\{ 
%     \begin{array}{ccc} 
%       \dropn{\quotep{Q}} & & x \nameeq \quotep{P} \\
%       \dropn{x} & & otherwise \\
%     \end{array}
%   \right. 
  (\dropn{x})  \psubstp{Q}{P}       
  := 
  \left\{ 
    \begin{array}{ccc} 
      Q & & x \nameeq \quotep{P} \\
      \dropn{x} & & otherwise \\
    \end{array}
  \right.
\end{mathpar}
 

where

\begin{eqnarray}
  (x)\id{\{} \lpquote Q \rpquote / \lpquote P \rpquote \id{\}}            = 
  \left\{ 
    \begin{array}{ccc}
      \lpquote Q \rpquote & & x \nameeq \lpquote P \rpquote \\
      x & & otherwise \\
    \end{array}
  \right. \nonumber
\end{eqnarray}

and $z$ is chosen distinct from $\quotep{P}$, $\quotep{Q}$, the free
names in $Q$, and all the names in $R$. Our $\alpha$-equivalence will
be built in the standard way from this substitution.

\begin{remark}\label{rem:no_self_referential_names}
  One consequence of these definitions is that $\forall P. \quotep{P}
  \not\in \freenames{P}$.
\end{remark}

\subsection{ Dynamic quote: an example }

Anticipating something of what's to come, consider applying the
substitution, $\widehat{\id{\{}u / z \id{\}}}$, to the following pair
of processes, $\lift{w}{y!(z)}$ and $w[ \lpquote y!(z) \rpquote ]$.

\begin{eqnarray}
	\lift{w}{y!(z)}\widehat{\id{\{}u / z \id{\}}}
		& = &
		\lift{w}{y!(u)} \nonumber\\
	w[ \lpquote y!(z) \rpquote ] \widehat{ \id{\{}u / z \id{\}} }
		& = &
		w[ \lpquote y!(z) \rpquote ] \nonumber
\end{eqnarray}

Because the body of the process between quotes is impervious to
substitution, we get radically different answers. In fact, by
examining the first process in an input context,
e.g. $x?(z).\lift{w}{y!(z)}$, we see that the process under the lift
operator may be shaped by prefixed inputs binding a name inside it. In
this sense, the lift operator will be seen as a way to dynamically
construct processes before reifying them as names.

Finally equipped with these standard features we can present the
dynamics of the calculus.

\subsubsection{Operational semantics} 

Finally, we introduce the computational dynamics. What marks these
algebras as distinct from other more traditionally studied algebraic
structures, e.g. vector spaces or polynomial rings, is the manner in
which dynamics is captured. In traditional structures, dynamics is typically
expressed through morphisms between such structures, as in linear maps
between vector spaces or morphisms between rings. In algebras
associated with the semantics of computation, the dynamics is
expressed as part of the algebraic structure itself, through a
reduction reduction relation typically denoted by $\red$. Below, we
give a recursive presentation of this relation for the calculus used
in the encoding.

$\red \subseteq \pi \times \pi$
$\red : \pi \to \mathcal{P}(\pi)$

\begin{mathpar}
  \inferrule* [lab=Comm] { \textsf{match}( x_{src}, x_{trgt} ) } { x_{trgt}?(y)P \; | \; x_{src}!\langle {Q} \rangle \red P\{\quotep{Q}/y}\} }
  \and \\
  \inferrule* [lab=Par] {{P} \red {P}'} {{{P} | {Q}} \red {{P}' | {Q}}}
  \and
  \inferrule* [lab=Equiv]{{{P} \scong {P}'} \andalso {{P}' \red {Q}'} \andalso {{Q}' \scong {Q}}}{{P} \red {Q}}
\end{mathpar}

\begin{eqnarray*}
  match_{\equiv} (\quotep{P},\quotep{Q}) & := & P \equiv Q \\
  match_{\dagger}(\quotep{P},\quotep{Q}) & := & \forall R. P|Q \red^{*} R => R \red^{*} 0 \\
  match_{K}(\quotep{P},\quotep{Q}) & := & K \mbox{ for some context } K
\end{eqnarray*}

$u?(x)P | u!\langle Q \rangle \red P\{\quotep{Q}/x\}$

%We write $\wred$ for $\red^*$, and $P\red$ if $\exists Q $ such that $ P \red Q$.
We write $P\red$ if $\exists Q $ such that $ P \red Q$ and $P\not\red$, otherwise.

\section{Replication}

As mentioned before, it is known that replication (and hence
recursion) can be implemented in a higher-order process algebra
\cite{SangiorgiWalker}. As our first example of calculation with the
machinery thus far presented we give the construction explicitly in
the {\rhoc}.

\begin{eqnarray}
	D_{x} & := & \prefix{x}{y}{(\binpar{\outputp{x}{y}}{@{y}})} \nonumber\\
	\bangp_{x}{P} & := & \binpar{{x}!\langle{\binpar{D_{x}}{P}}\rangle}{D_{x}} \nonumber
\end{eqnarray}

\begin{eqnarray}
	\bangp_{x}{P} & & \nonumber\\
	=
	& {x}!\langle{(\prefix{x}{y}{(\outputp{x}{y} | @{y})) | P}}\rangle 
	      | \prefix{x}{y}{(\outputp{x}{y} | @{y})} & \nonumber\\
	\red
	& (\outputp{x}{y} | @{y})\substn{\quotep{(\prefix{x}{y}{(@{y} | \outputp{x}{y})) | P}}}{y} & \nonumber\\
	=
	& \outputp{x}{\quotep{(\prefix{x}{y}{(\outputp{x}{y} | @{y})) | P}}}
	  | {(\prefix{x}{y}{(\outputp{x}{y} | @{y})) | P}} & \nonumber\\
	\red
	& \ldots & \nonumber\\
	\red^*
	& P | P | \ldots & \nonumber
\end{eqnarray}

Of course, this encoding, as an implementation, runs away, unfolding
$\bangp{P}$ eagerly. A lazier and more implementable replication
operator, restricted to input-guarded processes, may be obtained as follows.

\begin{eqnarray}
\bangp{\prefix{u}{v}{P}} 
	:= 
	\binpar{\lift{x}{\prefix{u}{v}{(\binpar{D(x)}{P})}}}{D(x)} \nonumber
\end{eqnarray}

\begin{remark}
  Note that the lazier definition still does not deal with summation
  or mixed summation (i.e. sums over input and output). The reader is
  invited to construct definitions of replication that deal with these
  features. 

  Further, the definitions are parameterized in a name, $x$. Can you,
  gentle reader, make a definition that eliminates this parameter and
  guarantees no accidental interaction between the replication
  machinery and the process being replicated -- i.e. no accidental
  sharing of names used by the process to get its work done and the
  name(s) used by the replication to effect copying. This latter
  revision of the definition of replication is crucial to obtaining
  the expected identity $!!P \sim !P$.
\end{remark}

\begin{remark}\label{rem:paradoxical_combinator}
  The reader familiar with the lambda calculus will have noticed the
  similarity between $D$ and the paradoxical combinator.

  [Ed. note: the existence of this seems to suggest we have to be more
  restrictive on the set of processes and names we admit if we are to
  support no-cloning.]
\end{remark}

\subsubsection{Bisimulation}

The computational dynamics gives rise to another kind of equivalence,
the equivalence of computational behavior. As previously mentioned
this is typically captured \emph{via} some form of bisimulation.

% The notion we use in this paper is weak barbed bisimulation
% \cite{milner91polyadicpi}.

The notion we use in this paper is derived from weak barbed
bisimulation \cite{milner91polyadicpi}. 

\begin{definition}
An \emph{observation relation}, $\downarrow_{\mathcal N}$, over a set
of names, $\mathcal N$, is the smallest relation satisfying the rules
below.

\infrule[Out-barb]{y \in {\mathcal N}, \; x \nameeq y}
		  {\outputp{x}{v} \downarrow_{\mathcal N} x}
\infrule[Par-barb]{\mbox{$P\downarrow_{\mathcal N} x$ or $Q\downarrow_{\mathcal N} x$}}
		  {\binpar{P}{Q} \downarrow_{\mathcal N} x}

We write $P \Downarrow_{\mathcal N} x$ if there is $Q$ such that 
$P \wred Q$ and $Q \downarrow_{\mathcal N} x$.
\end{definition}

\begin{definition}
%\label{def.bbisim}
An  ${\mathcal N}$-\emph{barbed bisimulation} over a set of names, ${\mathcal N}$, is a symmetric binary relation 
${\mathcal S}_{\mathcal N}$ between agents such that $P\rel{S}_{\mathcal N}Q$ implies:
\begin{enumerate}
\item If $P \red P'$ then $Q \wred Q'$ and $P'\rel{S}_{\mathcal N} Q'$.
\item If $P\downarrow_{\mathcal N} x$, then $Q\Downarrow_{\mathcal N} x$.
\end{enumerate}
$P$ is ${\mathcal N}$-barbed bisimilar to $Q$, written
$P \wbbisim_{\mathcal N} Q$, if $P \rel{S}_{\mathcal N} Q$ for some ${\mathcal N}$-barbed bisimulation ${\mathcal S}_{\mathcal N}$.
\end{definition}

$\mathcal{R} \subseteq \pi \times \pi$

$P \mathcal{R} Q => \forall P'. P \red P' \Rightarrow \exists Q'. Q \red Q', P' \mathcal{R} Q'$

$P \vdash x \Rightarrow Q \vdash x$

\begin{mathpar}
  \inferrule*[lab=Out-barb]{x \nameeq y}{{y}!\langle{Q}\rangle \vdash x}
  \and
  \inferrule*[lab=Par-barb]{\mbox{$P\vdash x$ or $Q\vdash x$}}{\binpar{P}{Q} \vdash x}
\end{mathpar}

\subsubsection{Contexts}

One of the principle advantages of computational calculi like the
$\pi$-calculus is a well-defined notion of context,
contextual-equivalence and a correlation between
contextual-equivalence and notions of bisimulation. The notion of
context allows the decomposition of a process into (sub-)process and
its syntactic environment, its context. Thus, a context may be
thought of as a process with a ``hole'' (written $\Box$) in it. The
application of a context $M$ to a process $P$, written $M[P]$, is
tantamount to filling the hole in $M$ with $P$. In this paper we do
not need the full weight of this theory, but do make use of the notion
of context in the proof the main theorem. 

\begin{mathpar}
  \inferrule* [lab=summation] {} {{M_{M},M_{N}} \bc \Box \;|\; x.M_{A} \;|\; M_{M}+M_{N}}
  \and
  \inferrule* [lab=agent] {} {{M_{A}} \bc (\vec{x})M_{P} \;| \; \clift{P_0,\ldots,M_{P},\ldots,P_N}}
  \and \\
  \inferrule* [lab=process] {} {{M_{P}} \bc M_{N} \;| \;P|M_{P} }
\end{mathpar} 

\begin{mathpar}
  \inferrule* [lab=sychronization] {} {M_{N} \bc \Box \;|\; x?M_{F} \;|\; x!M_{C}}
  \and
  \inferrule* [lab=abstraction] {} {{M_{F}} \bc (x)M_{P} }
  \and
  \inferrule* [lab=concretion] {} {{M_{C}} \bc \langle M_{P} \rangle }
  \and \\
  \inferrule* [lab=process] {} {{M_{P}} \bc M_{N} \;| \;P|M_{P} }
\end{mathpar}

\begin{definition}[contextual application] Given a context $M$, and
  process $P$, we define the \emph{contextual application}, $M[P] :=
  M\{P/\Box\}$. That is, the contextual application of M to P is the
  substitution of $P$ for $\Box$ in $M$.
\end{definition}

$\meaningof{-} : L \to \mathcal{P}(\pi)$

\begin{mathpar}
  \inferrule* [lab=collection] {} {\meaningof{true} = \pi, \and \meaningof{~E} = \pi \setminus \meaningof{E}, \and \meaningof{E_{1} \& E_{2}} = \meaningof{E_{1}} \cap \meaningof{E_{2}}}
\end{mathpar}

\begin{mathpar}
  \inferrule* [lab=structure] {} {\meaningof{0} = \{ P \in \pi | P \equiv 0 \}, \and \\ \meaningof{E_1 | E_2} = \{ P \in \pi | P \equiv P_{1} | P_{2}, P_{1} \in \meaningof{E_{1}}, P_{2} \in \meaningof{E_2}\} }
\end{mathpar}

\begin{mathpar}
 \inferrule* [lab=behavior] {} {\meaningof{\langle a?b \rangle E} = \{ P \in \pi | P \equiv Q | u?(y)P', \\ \and \\\\ \and \\ \;\;\; u \in \meaningof{a}, \forall z.P'\{z/y\} \in \meaningof{E\{z/b\}}\}, \and \\ \meaningof{a!E} = \{ P \in \pi | P \equiv Q | x!\langle P' \rangle, x \in \meaningof{a} P' \in \meaningof{E}\} }
\end{mathpar}

\begin{mathpar}
 \inferrule* [lab=nominal] {} {\meaningof{\quotep{E}} = \{ \quotep{P} \in \quotep{\pi} | P \in \meaningof{E} \}, \and \meaningof{\quotep{P}} = \{ \quotep{Q} \in \quotep{\pi} | P \equiv Q \} \and \\ \meaningof{@\quotep{E}} = \{ P \in \pi | P \equiv @x, x \in \meaningof{E} \}}
\end{mathpar}

\begin{eqnarray*}
  \\
  \meaningof{-} : TS \to ST
\end{eqnarray*}

\begin{eqnarray*}
  \\
  L : TS \to ST
\end{eqnarray*}

\begin{eqnarray*}
  \\
  P \models E \iff P \in \meaningof{E}
\end{eqnarray*}

\begin{eqnarray*}
  P \approx_{L} Q \iff \forall E \in L. P \models E \iff Q \models E
\end{eqnarray*}

\begin{eqnarray*}
  P \approx_{K} Q
\end{eqnarray*}

\begin{eqnarray*}
  P \approx Q
\end{eqnarray*}

$\approx_{K} = \approx = \approx_{L}$

\subsubsection{Contextual duality}

Note that contexts extend the quotation operation to a family of
operations from processes to names. Given a context, $M$, we can
define a \emph{nominal context}, $\quotep{M}$ by $\quotep{M}[P] :=
\quotep{M[P]}$. To foreshadow what is to come we observe that these
operations enjoy a duality with processes very much like the duality
between vectors and maps from vectors to scalars.

Further, because the calculus is essentially higher-order, we have a
correspondence between contexts and processes. More specifically,
given a name $x$ and a context $M$ we can construct $M^{*}_{x}$ such
that 

\begin{mathpar}
  M^{*}_{x} | \lift{x}{P} \red M[P]
\end{mathpar}

namely,

\begin{mathpar}
  M^{*}_{x} := x?(u).M[\dropn{u}]
\end{mathpar}

The dependence of $M^{*}_{x}$ on a name makes it an abstraction, 

\begin{mathpar}
  M^{*} := (x)x?(u).M[\dropn{u}]
\end{mathpar}

\subsection{Additional notation}

It will sometimes be convenient to denote the process a name
quotes. We already have the notation $x = \quotep{P}$, but it will be
convenient to introduce an alternate notation, $\procn{x}$, when we
want to emphasize the connection to the use of the name. Note that, by
virtue of name equivalence, $\quotep{\procn{x}} \nameeq x$; so, the
notation is consistent with previous definitions.

Further, because names have structure it is possible to effect
substitutions on the basis of that structure. This means we need to
upgrade our notation for substitutions, which we accomplish by
adapting comprehension notation. Thus,

\begin{mathpar}
  P\{ y / x : x \in S \}
\end{mathpar}

is interpreted to mean the process derived from P by replacing (in a
capture-avoiding manner) each occurrence of $x$ in $S$ by $y$. For example,

\begin{mathpar}
  P\{ \quotep{\procn{x}|\procn{x}} / x : x \in \freenames{P} \}
\end{mathpar}

will replace each (occurrence) of a free name $x$ in $P$ by
$\quotep{\procn{x}|\procn{x}}$.

Also, we will avail ourselves of the notation $x^{L}$ and $x^{R}$ to
denote injections of a name into disjoint copies of the name
space. There are numerous ways to accomplish this. One example can be
found in \cite{MeredithR05}. This notation overloads to vectors of
names: $\vec{x}^{\pi} := (x_{i}^{\pi} \; : \; 0 \leq i < |\vec{x}| )$ where $\pi \in \{L,R\}$.

We also use $P^{\Box} := P|\Box$.

In \cite{MeredithR05} an interpretation of the new operator is
given. It turns out that there are several possible interpretations
all enjoying the requisite algebraic properties of the operator (see
\cite{milner91polyadicpi}). We will therefore make liberal use of
$(\nu\; \vec{x})P$.

% subsection the_syntax_and_semantics_of_the_notation_system (end)   

\input{qm2pi.qmops} 

\input{qm2pi.sterngerlach} 

\input{qm2pi.metric} 

% section concurrent_process_calculi (end)

%\input{qm2pi.proofsketch}

% section proof sketch (end)

%\input{qm2pi.slviaknots} 

% section spatial logic via knots (end)

\input{qm2pi.conclusion}

% section conclusion (end)

%\input{qm2pi.dtcodes} 

% section wiring algorithm (end)

\input{qm2pi.ack} 

% section acknowledgments (end)

\newpage


\bibliographystyle{plain}   
\bibliography{../../biblios/main.bib}

\input{qm2pi.rhodetails}

\end{document}

 

% section wiring algorithm (end)

\documentclass[12pt]{llncs}
%\documentclass{jktr}

\usepackage[pdftex]{hyperref}                   
\usepackage {listings}
\usepackage {mathpartir}
\usepackage{bcprules}
%\usepackage{listings}
                       
\usepackage{graphicx} 
%\usepackage[margins=2.5cm,nohead,nofoot]{geometry}
%\usepackage{geometry}
\usepackage{amsfonts}
\usepackage{amstext}
\usepackage{latexsym}
\usepackage{amssymb}
\usepackage{color}


%\include{myPreamble}
\include{qm2pi.local} 

%\ifpdf
%\usepackage[pdftex]{graphicx}
%\else
%\usepackage{graphicx}
%\fi

 % \ifpdf
%  \usepackage{pdfsync}
%  \if


%\title{Brief Article}
%\author{David F. Snyder}
%\author{L.G. Meredith}

%\address{Dept. of Math., Texas State University--San Marcos, San Marcos, TX 78666}
       
\pagestyle{empty}


\begin{document}

\lstset{language=[Objective]Caml,frame=shadowbox}

\input{qm2pi.front}

% section front matter (end)

\input{qm2pi.intro} 
 
% section introduction (end)

% \input{qm2pi.knotations} 

% section notation (end)

\input{qm2pi.process.calculi} 

% section concurrent_process_calculi_and_spatial_logics_ (end)
    
%\input{qm2pi.knots2pi} 

%\input{qm2pi.trefoil} 

%\input{qm2pi.mainthm} 

% subsection basic_interpretation (end)

%\input{qm2pi.rho.presentation} 
\subsection{The syntax and semantics of the notation system}\label{sub:the_syntax_and_semantics_of_the_notation_system} % (fold)

We now summarize a technical presentation of the calculus that
embodies our theory of dynamics. The typical presentation of such a
calculus follows the style of giving generators and relations on
them. The grammar, below, describing term constructors, freely
generates the set of processes, $\Proc$. This set is then quotiented
by a relation known as structural congruence and it is over this set
that the notion of dynamics is expressed. This presentation is
essentially that of \cite{MeredithR05} with the addition of
polyadicity and summation. For readability we have relegated some of
the technical subtleties to an appendix.

\subsubsection{Process grammar}\label{subsub:process_grammar}

\begin{mathpar}
  \inferrule* [lab=synchronization] {} {{M} \bc \pzero \;|\; x?F \;|\; x!C }
  \and
  \inferrule* [lab=abstraction] {} {{F} \bc (x)P}
  \and
  \inferrule* [lab=concretion] {} {{C} \bc \langle Q \rangle}
  \and
  \inferrule* [lab=process] {} {{P,Q} \bc M \;| \;P|Q \;|\; @{x}}
  \and
  \inferrule* [lab=name] {} {{x} \bc \quotep{P}}
\end{mathpar} 

Note that $\vec{x}$ (resp. $\vec{P}$) denotes a vector of names
(resp. processes) of length $|\vec{x}|$ (resp. $|\vec{P}|$). We adopt
the following useful abbreviations.

\begin{mathpar}
   x?(\vec{y}).P := x.(\vec{y})P \and  x\clift{\vec{P}} := x.\clift{\vec{P}}
   \and x!(y) := \lift{x}{\dropn{y}}
   \and \Pi_{i=0}^{n-1}P_i := P_0 | \ldots | P_{n-1}
\end{mathpar}

\subsubsection{Structural congruence}

\paragraph{Free and bound names and alpha-equivalence.} At the
core of structural equivalence is alpha-equivalence which identifies
process that are the same up to a change of variable. Formally, we
recognize the distinction between free and bound names. The free names
of a process, $\freenames{P}$, may be calculated recursively as
follows:

\begin{mathpar}
\freenames{\pzero} := \emptyset
  \and \\
  \freenames{x?(y).P} := \{ x \} \cup (\freenames{P} \setminus \{ y \})
  \and 
  \freenames{x!\langle P \rangle} := \{ x \} \cup \{ P \} 
  \and \\
  \freenames{P|Q} := \freenames{P} \cup \freenames{Q}
  \and \\
  \freenames{@{x}} := \{ x \}
\end{mathpar}

$\pi$
$\quotep{\pi}$

$\freenames{-} : \pi \to \mathcal{P}(\quotep{\pi})$

\begin{eqnarray*}
  \freenames{\pzero} & := & \emptyset \\
  \freenames{x?(y).P} & := & \{ x \} \cup (\freenames{P} \setminus \{ y \}) \\
  \freenames{x!\langle P \rangle} & := & \{ x \} \cup \{ P \} \\
  \freenames{P|Q} & := & \freenames{P} \cup \freenames{Q} \\
  \freenames{\dropn{x}} & := & \{ x \}
\end{eqnarray*}

The bound names of a process, $\boundnames{P}$, are those names occurring in $P$
that are not free. For example, in $x?(y).0$, the name $x$ is free, while $y$ is bound.

\begin{mathpar}
  \inferrule* [lab=monoidal-laws] {} { P|Q \equiv Q|P \and P|0 \equiv P \and P|(Q|R) \equiv (P|Q)|R }
\end{mathpar}

\begin{mathpar}
  \inferrule* [lab=alpha-equivalence] {} { (x)P \equiv (y)P\{y/x\} \and y \not\in \freenames{P} }
\end{mathpar}

\begin{definition}
Then two processes, $P,Q$, are alpha-equivalent if $P = Q\{\vec{y}/\vec{x}\}$ for
some $\vec{x} \in \boundnames{Q},\vec{y} \in \boundnames{P}$, where $Q\{\vec{y}/\vec{x}\}$
denotes the capture-avoiding substitution of $\vec{y}$ for $\vec{x}$ in $Q$.
\end{definition}

\begin{definition}
  The {\em structural congruence} \cite{SangiorgiWalker} , $\equiv$,
  between processes is the least congruence containing
  alpha-equivalence, satisfying the abelian monoid laws
  (associativity, commutativity and $\pzero$ as identity) for parallel
  composition $|$ and for summation $+$.
\end{definition}

\subsection{Name equivalence}

We take name equivalence, written $\nameeq$, to be the smallest
equivalence relation generated by the following rules.

\begin{mathpar}
\inferrule*[lab=Quote-drop]
{ }
{ \quotep{@{x}} \nameeq x }

\inferrule*[lab=Struct-equiv]
{ P \scong Q }
{ \quotep{P} \nameeq \quotep{Q} }
\end{mathpar}

The astute reader will have noticed that the mutual recursion of names
and processes imposes a mutual recursion on alpha-equivalence and
structural equivalence via name-equivalence. Fortunately, all of this
works out pleasantly and we may calculate in the natural way, free of
concern. The reader interested in the details is referred to the
appendix \ref{appendix:rho_details}.

\subsection{Substitution}

We use $\Proc$ for the set of processes, $\QProc$ for the set of
names, and $\id{\{}\vec{y} / \vec{x} \id{\}}$ to denote partial maps,
$s : \QProc \rightarrow \QProc$. A map, $s$ lifts, uniquely, to a map
on process terms, $\widehat{s} : \Proc \rightarrow \Proc$ by the
following equations.

\begin{mathpar}
  (0) \psubstp{Q}{P} := 0 \\
  (R \juxtap S) \psubstp{Q}{P}
  :=    
  (R)\psubstp{Q}{P} \juxtap (S) \psubstp{Q}{P} \\
  (x?(y).R) \psubstp{Q}{P}    
  :=    
  (x)\substp{Q}{P} (z)\concat( (R \psubstn{z}{y}) \psubstp{Q}{P} ) \\
  (\lift{x}{R}) \psubstp{Q}{P}  
  :=
  \lift{(x)\substp{Q}{P}}{ R \psubstp{Q}{P} } \\
%   (\dropn{x})  \psubstp{Q}{P}       
%   := 
%   \left\{ 
%     \begin{array}{ccc} 
%       \dropn{\quotep{Q}} & & x \nameeq \quotep{P} \\
%       \dropn{x} & & otherwise \\
%     \end{array}
%   \right. 
  (\dropn{x})  \psubstp{Q}{P}       
  := 
  \left\{ 
    \begin{array}{ccc} 
      Q & & x \nameeq \quotep{P} \\
      \dropn{x} & & otherwise \\
    \end{array}
  \right.
\end{mathpar}
 

where

\begin{eqnarray}
  (x)\id{\{} \lpquote Q \rpquote / \lpquote P \rpquote \id{\}}            = 
  \left\{ 
    \begin{array}{ccc}
      \lpquote Q \rpquote & & x \nameeq \lpquote P \rpquote \\
      x & & otherwise \\
    \end{array}
  \right. \nonumber
\end{eqnarray}

and $z$ is chosen distinct from $\quotep{P}$, $\quotep{Q}$, the free
names in $Q$, and all the names in $R$. Our $\alpha$-equivalence will
be built in the standard way from this substitution.

\begin{remark}\label{rem:no_self_referential_names}
  One consequence of these definitions is that $\forall P. \quotep{P}
  \not\in \freenames{P}$.
\end{remark}

\subsection{ Dynamic quote: an example }

Anticipating something of what's to come, consider applying the
substitution, $\widehat{\id{\{}u / z \id{\}}}$, to the following pair
of processes, $\lift{w}{y!(z)}$ and $w[ \lpquote y!(z) \rpquote ]$.

\begin{eqnarray}
	\lift{w}{y!(z)}\widehat{\id{\{}u / z \id{\}}}
		& = &
		\lift{w}{y!(u)} \nonumber\\
	w[ \lpquote y!(z) \rpquote ] \widehat{ \id{\{}u / z \id{\}} }
		& = &
		w[ \lpquote y!(z) \rpquote ] \nonumber
\end{eqnarray}

Because the body of the process between quotes is impervious to
substitution, we get radically different answers. In fact, by
examining the first process in an input context,
e.g. $x?(z).\lift{w}{y!(z)}$, we see that the process under the lift
operator may be shaped by prefixed inputs binding a name inside it. In
this sense, the lift operator will be seen as a way to dynamically
construct processes before reifying them as names.

Finally equipped with these standard features we can present the
dynamics of the calculus.

\subsubsection{Operational semantics} 

Finally, we introduce the computational dynamics. What marks these
algebras as distinct from other more traditionally studied algebraic
structures, e.g. vector spaces or polynomial rings, is the manner in
which dynamics is captured. In traditional structures, dynamics is typically
expressed through morphisms between such structures, as in linear maps
between vector spaces or morphisms between rings. In algebras
associated with the semantics of computation, the dynamics is
expressed as part of the algebraic structure itself, through a
reduction reduction relation typically denoted by $\red$. Below, we
give a recursive presentation of this relation for the calculus used
in the encoding.

$\red \subseteq \pi \times \pi$
$\red : \pi \to \mathcal{P}(\pi)$

\begin{mathpar}
  \inferrule* [lab=Comm] { \textsf{match}( x_{src}, x_{trgt} ) } { x_{trgt}?(y)P \; | \; x_{src}!\langle {Q} \rangle \red P\{\quotep{Q}/y}\} }
  \and \\
  \inferrule* [lab=Par] {{P} \red {P}'} {{{P} | {Q}} \red {{P}' | {Q}}}
  \and
  \inferrule* [lab=Equiv]{{{P} \scong {P}'} \andalso {{P}' \red {Q}'} \andalso {{Q}' \scong {Q}}}{{P} \red {Q}}
\end{mathpar}

\begin{eqnarray*}
  match_{\equiv} (\quotep{P},\quotep{Q}) & := & P \equiv Q \\
  match_{\dagger}(\quotep{P},\quotep{Q}) & := & \forall R. P|Q \red^{*} R => R \red^{*} 0 \\
  match_{K}(\quotep{P},\quotep{Q}) & := & K \mbox{ for some context } K
\end{eqnarray*}

$u?(x)P | u!\langle Q \rangle \red P\{\quotep{Q}/x\}$

%We write $\wred$ for $\red^*$, and $P\red$ if $\exists Q $ such that $ P \red Q$.
We write $P\red$ if $\exists Q $ such that $ P \red Q$ and $P\not\red$, otherwise.

\section{Replication}

As mentioned before, it is known that replication (and hence
recursion) can be implemented in a higher-order process algebra
\cite{SangiorgiWalker}. As our first example of calculation with the
machinery thus far presented we give the construction explicitly in
the {\rhoc}.

\begin{eqnarray}
	D_{x} & := & \prefix{x}{y}{(\binpar{\outputp{x}{y}}{@{y}})} \nonumber\\
	\bangp_{x}{P} & := & \binpar{{x}!\langle{\binpar{D_{x}}{P}}\rangle}{D_{x}} \nonumber
\end{eqnarray}

\begin{eqnarray}
	\bangp_{x}{P} & & \nonumber\\
	=
	& {x}!\langle{(\prefix{x}{y}{(\outputp{x}{y} | @{y})) | P}}\rangle 
	      | \prefix{x}{y}{(\outputp{x}{y} | @{y})} & \nonumber\\
	\red
	& (\outputp{x}{y} | @{y})\substn{\quotep{(\prefix{x}{y}{(@{y} | \outputp{x}{y})) | P}}}{y} & \nonumber\\
	=
	& \outputp{x}{\quotep{(\prefix{x}{y}{(\outputp{x}{y} | @{y})) | P}}}
	  | {(\prefix{x}{y}{(\outputp{x}{y} | @{y})) | P}} & \nonumber\\
	\red
	& \ldots & \nonumber\\
	\red^*
	& P | P | \ldots & \nonumber
\end{eqnarray}

Of course, this encoding, as an implementation, runs away, unfolding
$\bangp{P}$ eagerly. A lazier and more implementable replication
operator, restricted to input-guarded processes, may be obtained as follows.

\begin{eqnarray}
\bangp{\prefix{u}{v}{P}} 
	:= 
	\binpar{\lift{x}{\prefix{u}{v}{(\binpar{D(x)}{P})}}}{D(x)} \nonumber
\end{eqnarray}

\begin{remark}
  Note that the lazier definition still does not deal with summation
  or mixed summation (i.e. sums over input and output). The reader is
  invited to construct definitions of replication that deal with these
  features. 

  Further, the definitions are parameterized in a name, $x$. Can you,
  gentle reader, make a definition that eliminates this parameter and
  guarantees no accidental interaction between the replication
  machinery and the process being replicated -- i.e. no accidental
  sharing of names used by the process to get its work done and the
  name(s) used by the replication to effect copying. This latter
  revision of the definition of replication is crucial to obtaining
  the expected identity $!!P \sim !P$.
\end{remark}

\begin{remark}\label{rem:paradoxical_combinator}
  The reader familiar with the lambda calculus will have noticed the
  similarity between $D$ and the paradoxical combinator.

  [Ed. note: the existence of this seems to suggest we have to be more
  restrictive on the set of processes and names we admit if we are to
  support no-cloning.]
\end{remark}

\subsubsection{Bisimulation}

The computational dynamics gives rise to another kind of equivalence,
the equivalence of computational behavior. As previously mentioned
this is typically captured \emph{via} some form of bisimulation.

% The notion we use in this paper is weak barbed bisimulation
% \cite{milner91polyadicpi}.

The notion we use in this paper is derived from weak barbed
bisimulation \cite{milner91polyadicpi}. 

\begin{definition}
An \emph{observation relation}, $\downarrow_{\mathcal N}$, over a set
of names, $\mathcal N$, is the smallest relation satisfying the rules
below.

\infrule[Out-barb]{y \in {\mathcal N}, \; x \nameeq y}
		  {\outputp{x}{v} \downarrow_{\mathcal N} x}
\infrule[Par-barb]{\mbox{$P\downarrow_{\mathcal N} x$ or $Q\downarrow_{\mathcal N} x$}}
		  {\binpar{P}{Q} \downarrow_{\mathcal N} x}

We write $P \Downarrow_{\mathcal N} x$ if there is $Q$ such that 
$P \wred Q$ and $Q \downarrow_{\mathcal N} x$.
\end{definition}

\begin{definition}
%\label{def.bbisim}
An  ${\mathcal N}$-\emph{barbed bisimulation} over a set of names, ${\mathcal N}$, is a symmetric binary relation 
${\mathcal S}_{\mathcal N}$ between agents such that $P\rel{S}_{\mathcal N}Q$ implies:
\begin{enumerate}
\item If $P \red P'$ then $Q \wred Q'$ and $P'\rel{S}_{\mathcal N} Q'$.
\item If $P\downarrow_{\mathcal N} x$, then $Q\Downarrow_{\mathcal N} x$.
\end{enumerate}
$P$ is ${\mathcal N}$-barbed bisimilar to $Q$, written
$P \wbbisim_{\mathcal N} Q$, if $P \rel{S}_{\mathcal N} Q$ for some ${\mathcal N}$-barbed bisimulation ${\mathcal S}_{\mathcal N}$.
\end{definition}

$\mathcal{R} \subseteq \pi \times \pi$

$P \mathcal{R} Q => \forall P'. P \red P' \Rightarrow \exists Q'. Q \red Q', P' \mathcal{R} Q'$

$P \vdash x \Rightarrow Q \vdash x$

\begin{mathpar}
  \inferrule*[lab=Out-barb]{x \nameeq y}{{y}!\langle{Q}\rangle \vdash x}
  \and
  \inferrule*[lab=Par-barb]{\mbox{$P\vdash x$ or $Q\vdash x$}}{\binpar{P}{Q} \vdash x}
\end{mathpar}

\subsubsection{Contexts}

One of the principle advantages of computational calculi like the
$\pi$-calculus is a well-defined notion of context,
contextual-equivalence and a correlation between
contextual-equivalence and notions of bisimulation. The notion of
context allows the decomposition of a process into (sub-)process and
its syntactic environment, its context. Thus, a context may be
thought of as a process with a ``hole'' (written $\Box$) in it. The
application of a context $M$ to a process $P$, written $M[P]$, is
tantamount to filling the hole in $M$ with $P$. In this paper we do
not need the full weight of this theory, but do make use of the notion
of context in the proof the main theorem. 

\begin{mathpar}
  \inferrule* [lab=summation] {} {{M_{M},M_{N}} \bc \Box \;|\; x.M_{A} \;|\; M_{M}+M_{N}}
  \and
  \inferrule* [lab=agent] {} {{M_{A}} \bc (\vec{x})M_{P} \;| \; \clift{P_0,\ldots,M_{P},\ldots,P_N}}
  \and \\
  \inferrule* [lab=process] {} {{M_{P}} \bc M_{N} \;| \;P|M_{P} }
\end{mathpar} 

\begin{mathpar}
  \inferrule* [lab=sychronization] {} {M_{N} \bc \Box \;|\; x?M_{F} \;|\; x!M_{C}}
  \and
  \inferrule* [lab=abstraction] {} {{M_{F}} \bc (x)M_{P} }
  \and
  \inferrule* [lab=concretion] {} {{M_{C}} \bc \langle M_{P} \rangle }
  \and \\
  \inferrule* [lab=process] {} {{M_{P}} \bc M_{N} \;| \;P|M_{P} }
\end{mathpar}

\begin{definition}[contextual application] Given a context $M$, and
  process $P$, we define the \emph{contextual application}, $M[P] :=
  M\{P/\Box\}$. That is, the contextual application of M to P is the
  substitution of $P$ for $\Box$ in $M$.
\end{definition}

$\meaningof{-} : L \to \mathcal{P}(\pi)$

\begin{mathpar}
  \inferrule* [lab=collection] {} {\meaningof{true} = \pi, \and \meaningof{~E} = \pi \setminus \meaningof{E}, \and \meaningof{E_{1} \& E_{2}} = \meaningof{E_{1}} \cap \meaningof{E_{2}}}
\end{mathpar}

\begin{mathpar}
  \inferrule* [lab=structure] {} {\meaningof{0} = \{ P \in \pi | P \equiv 0 \}, \and \\ \meaningof{E_1 | E_2} = \{ P \in \pi | P \equiv P_{1} | P_{2}, P_{1} \in \meaningof{E_{1}}, P_{2} \in \meaningof{E_2}\} }
\end{mathpar}

\begin{mathpar}
 \inferrule* [lab=behavior] {} {\meaningof{\langle a?b \rangle E} = \{ P \in \pi | P \equiv Q | u?(y)P', \\ \and \\\\ \and \\ \;\;\; u \in \meaningof{a}, \forall z.P'\{z/y\} \in \meaningof{E\{z/b\}}\}, \and \\ \meaningof{a!E} = \{ P \in \pi | P \equiv Q | x!\langle P' \rangle, x \in \meaningof{a} P' \in \meaningof{E}\} }
\end{mathpar}

\begin{mathpar}
 \inferrule* [lab=nominal] {} {\meaningof{\quotep{E}} = \{ \quotep{P} \in \quotep{\pi} | P \in \meaningof{E} \}, \and \meaningof{\quotep{P}} = \{ \quotep{Q} \in \quotep{\pi} | P \equiv Q \} \and \\ \meaningof{@\quotep{E}} = \{ P \in \pi | P \equiv @x, x \in \meaningof{E} \}}
\end{mathpar}

\begin{eqnarray*}
  \\
  \meaningof{-} : TS \to ST
\end{eqnarray*}

\begin{eqnarray*}
  \\
  L : TS \to ST
\end{eqnarray*}

\begin{eqnarray*}
  \\
  P \models E \iff P \in \meaningof{E}
\end{eqnarray*}

\begin{eqnarray*}
  P \approx_{L} Q \iff \forall E \in L. P \models E \iff Q \models E
\end{eqnarray*}

\begin{eqnarray*}
  P \approx_{K} Q
\end{eqnarray*}

\begin{eqnarray*}
  P \approx Q
\end{eqnarray*}

$\approx_{K} = \approx = \approx_{L}$

\subsubsection{Contextual duality}

Note that contexts extend the quotation operation to a family of
operations from processes to names. Given a context, $M$, we can
define a \emph{nominal context}, $\quotep{M}$ by $\quotep{M}[P] :=
\quotep{M[P]}$. To foreshadow what is to come we observe that these
operations enjoy a duality with processes very much like the duality
between vectors and maps from vectors to scalars.

Further, because the calculus is essentially higher-order, we have a
correspondence between contexts and processes. More specifically,
given a name $x$ and a context $M$ we can construct $M^{*}_{x}$ such
that 

\begin{mathpar}
  M^{*}_{x} | \lift{x}{P} \red M[P]
\end{mathpar}

namely,

\begin{mathpar}
  M^{*}_{x} := x?(u).M[\dropn{u}]
\end{mathpar}

The dependence of $M^{*}_{x}$ on a name makes it an abstraction, 

\begin{mathpar}
  M^{*} := (x)x?(u).M[\dropn{u}]
\end{mathpar}

\subsection{Additional notation}

It will sometimes be convenient to denote the process a name
quotes. We already have the notation $x = \quotep{P}$, but it will be
convenient to introduce an alternate notation, $\procn{x}$, when we
want to emphasize the connection to the use of the name. Note that, by
virtue of name equivalence, $\quotep{\procn{x}} \nameeq x$; so, the
notation is consistent with previous definitions.

Further, because names have structure it is possible to effect
substitutions on the basis of that structure. This means we need to
upgrade our notation for substitutions, which we accomplish by
adapting comprehension notation. Thus,

\begin{mathpar}
  P\{ y / x : x \in S \}
\end{mathpar}

is interpreted to mean the process derived from P by replacing (in a
capture-avoiding manner) each occurrence of $x$ in $S$ by $y$. For example,

\begin{mathpar}
  P\{ \quotep{\procn{x}|\procn{x}} / x : x \in \freenames{P} \}
\end{mathpar}

will replace each (occurrence) of a free name $x$ in $P$ by
$\quotep{\procn{x}|\procn{x}}$.

Also, we will avail ourselves of the notation $x^{L}$ and $x^{R}$ to
denote injections of a name into disjoint copies of the name
space. There are numerous ways to accomplish this. One example can be
found in \cite{MeredithR05}. This notation overloads to vectors of
names: $\vec{x}^{\pi} := (x_{i}^{\pi} \; : \; 0 \leq i < |\vec{x}| )$ where $\pi \in \{L,R\}$.

We also use $P^{\Box} := P|\Box$.

In \cite{MeredithR05} an interpretation of the new operator is
given. It turns out that there are several possible interpretations
all enjoying the requisite algebraic properties of the operator (see
\cite{milner91polyadicpi}). We will therefore make liberal use of
$(\nu\; \vec{x})P$.

% subsection the_syntax_and_semantics_of_the_notation_system (end)   

\input{qm2pi.qmops} 

\input{qm2pi.sterngerlach} 

\input{qm2pi.metric} 

% section concurrent_process_calculi (end)

%\input{qm2pi.proofsketch}

% section proof sketch (end)

%\input{qm2pi.slviaknots} 

% section spatial logic via knots (end)

\input{qm2pi.conclusion}

% section conclusion (end)

%\input{qm2pi.dtcodes} 

% section wiring algorithm (end)

\input{qm2pi.ack} 

% section acknowledgments (end)

\newpage


\bibliographystyle{plain}   
\bibliography{../../biblios/main.bib}

\input{qm2pi.rhodetails}

\end{document}

 

% section acknowledgments (end)

\newpage


\bibliographystyle{plain}   
\bibliography{../../biblios/main.bib}

\documentclass[12pt]{llncs}
%\documentclass{jktr}

\usepackage[pdftex]{hyperref}                   
\usepackage {listings}
\usepackage {mathpartir}
\usepackage{bcprules}
%\usepackage{listings}
                       
\usepackage{graphicx} 
%\usepackage[margins=2.5cm,nohead,nofoot]{geometry}
%\usepackage{geometry}
\usepackage{amsfonts}
\usepackage{amstext}
\usepackage{latexsym}
\usepackage{amssymb}
\usepackage{color}


%\include{myPreamble}
\include{qm2pi.local} 

%\ifpdf
%\usepackage[pdftex]{graphicx}
%\else
%\usepackage{graphicx}
%\fi

 % \ifpdf
%  \usepackage{pdfsync}
%  \if


%\title{Brief Article}
%\author{David F. Snyder}
%\author{L.G. Meredith}

%\address{Dept. of Math., Texas State University--San Marcos, San Marcos, TX 78666}
       
\pagestyle{empty}


\begin{document}

\lstset{language=[Objective]Caml,frame=shadowbox}

\input{qm2pi.front}

% section front matter (end)

\input{qm2pi.intro} 
 
% section introduction (end)

% \input{qm2pi.knotations} 

% section notation (end)

\input{qm2pi.process.calculi} 

% section concurrent_process_calculi_and_spatial_logics_ (end)
    
%\input{qm2pi.knots2pi} 

%\input{qm2pi.trefoil} 

%\input{qm2pi.mainthm} 

% subsection basic_interpretation (end)

%\input{qm2pi.rho.presentation} 
\subsection{The syntax and semantics of the notation system}\label{sub:the_syntax_and_semantics_of_the_notation_system} % (fold)

We now summarize a technical presentation of the calculus that
embodies our theory of dynamics. The typical presentation of such a
calculus follows the style of giving generators and relations on
them. The grammar, below, describing term constructors, freely
generates the set of processes, $\Proc$. This set is then quotiented
by a relation known as structural congruence and it is over this set
that the notion of dynamics is expressed. This presentation is
essentially that of \cite{MeredithR05} with the addition of
polyadicity and summation. For readability we have relegated some of
the technical subtleties to an appendix.

\subsubsection{Process grammar}\label{subsub:process_grammar}

\begin{mathpar}
  \inferrule* [lab=synchronization] {} {{M} \bc \pzero \;|\; x?F \;|\; x!C }
  \and
  \inferrule* [lab=abstraction] {} {{F} \bc (x)P}
  \and
  \inferrule* [lab=concretion] {} {{C} \bc \langle Q \rangle}
  \and
  \inferrule* [lab=process] {} {{P,Q} \bc M \;| \;P|Q \;|\; @{x}}
  \and
  \inferrule* [lab=name] {} {{x} \bc \quotep{P}}
\end{mathpar} 

Note that $\vec{x}$ (resp. $\vec{P}$) denotes a vector of names
(resp. processes) of length $|\vec{x}|$ (resp. $|\vec{P}|$). We adopt
the following useful abbreviations.

\begin{mathpar}
   x?(\vec{y}).P := x.(\vec{y})P \and  x\clift{\vec{P}} := x.\clift{\vec{P}}
   \and x!(y) := \lift{x}{\dropn{y}}
   \and \Pi_{i=0}^{n-1}P_i := P_0 | \ldots | P_{n-1}
\end{mathpar}

\subsubsection{Structural congruence}

\paragraph{Free and bound names and alpha-equivalence.} At the
core of structural equivalence is alpha-equivalence which identifies
process that are the same up to a change of variable. Formally, we
recognize the distinction between free and bound names. The free names
of a process, $\freenames{P}$, may be calculated recursively as
follows:

\begin{mathpar}
\freenames{\pzero} := \emptyset
  \and \\
  \freenames{x?(y).P} := \{ x \} \cup (\freenames{P} \setminus \{ y \})
  \and 
  \freenames{x!\langle P \rangle} := \{ x \} \cup \{ P \} 
  \and \\
  \freenames{P|Q} := \freenames{P} \cup \freenames{Q}
  \and \\
  \freenames{@{x}} := \{ x \}
\end{mathpar}

$\pi$
$\quotep{\pi}$

$\freenames{-} : \pi \to \mathcal{P}(\quotep{\pi})$

\begin{eqnarray*}
  \freenames{\pzero} & := & \emptyset \\
  \freenames{x?(y).P} & := & \{ x \} \cup (\freenames{P} \setminus \{ y \}) \\
  \freenames{x!\langle P \rangle} & := & \{ x \} \cup \{ P \} \\
  \freenames{P|Q} & := & \freenames{P} \cup \freenames{Q} \\
  \freenames{\dropn{x}} & := & \{ x \}
\end{eqnarray*}

The bound names of a process, $\boundnames{P}$, are those names occurring in $P$
that are not free. For example, in $x?(y).0$, the name $x$ is free, while $y$ is bound.

\begin{mathpar}
  \inferrule* [lab=monoidal-laws] {} { P|Q \equiv Q|P \and P|0 \equiv P \and P|(Q|R) \equiv (P|Q)|R }
\end{mathpar}

\begin{mathpar}
  \inferrule* [lab=alpha-equivalence] {} { (x)P \equiv (y)P\{y/x\} \and y \not\in \freenames{P} }
\end{mathpar}

\begin{definition}
Then two processes, $P,Q$, are alpha-equivalent if $P = Q\{\vec{y}/\vec{x}\}$ for
some $\vec{x} \in \boundnames{Q},\vec{y} \in \boundnames{P}$, where $Q\{\vec{y}/\vec{x}\}$
denotes the capture-avoiding substitution of $\vec{y}$ for $\vec{x}$ in $Q$.
\end{definition}

\begin{definition}
  The {\em structural congruence} \cite{SangiorgiWalker} , $\equiv$,
  between processes is the least congruence containing
  alpha-equivalence, satisfying the abelian monoid laws
  (associativity, commutativity and $\pzero$ as identity) for parallel
  composition $|$ and for summation $+$.
\end{definition}

\subsection{Name equivalence}

We take name equivalence, written $\nameeq$, to be the smallest
equivalence relation generated by the following rules.

\begin{mathpar}
\inferrule*[lab=Quote-drop]
{ }
{ \quotep{@{x}} \nameeq x }

\inferrule*[lab=Struct-equiv]
{ P \scong Q }
{ \quotep{P} \nameeq \quotep{Q} }
\end{mathpar}

The astute reader will have noticed that the mutual recursion of names
and processes imposes a mutual recursion on alpha-equivalence and
structural equivalence via name-equivalence. Fortunately, all of this
works out pleasantly and we may calculate in the natural way, free of
concern. The reader interested in the details is referred to the
appendix \ref{appendix:rho_details}.

\subsection{Substitution}

We use $\Proc$ for the set of processes, $\QProc$ for the set of
names, and $\id{\{}\vec{y} / \vec{x} \id{\}}$ to denote partial maps,
$s : \QProc \rightarrow \QProc$. A map, $s$ lifts, uniquely, to a map
on process terms, $\widehat{s} : \Proc \rightarrow \Proc$ by the
following equations.

\begin{mathpar}
  (0) \psubstp{Q}{P} := 0 \\
  (R \juxtap S) \psubstp{Q}{P}
  :=    
  (R)\psubstp{Q}{P} \juxtap (S) \psubstp{Q}{P} \\
  (x?(y).R) \psubstp{Q}{P}    
  :=    
  (x)\substp{Q}{P} (z)\concat( (R \psubstn{z}{y}) \psubstp{Q}{P} ) \\
  (\lift{x}{R}) \psubstp{Q}{P}  
  :=
  \lift{(x)\substp{Q}{P}}{ R \psubstp{Q}{P} } \\
%   (\dropn{x})  \psubstp{Q}{P}       
%   := 
%   \left\{ 
%     \begin{array}{ccc} 
%       \dropn{\quotep{Q}} & & x \nameeq \quotep{P} \\
%       \dropn{x} & & otherwise \\
%     \end{array}
%   \right. 
  (\dropn{x})  \psubstp{Q}{P}       
  := 
  \left\{ 
    \begin{array}{ccc} 
      Q & & x \nameeq \quotep{P} \\
      \dropn{x} & & otherwise \\
    \end{array}
  \right.
\end{mathpar}
 

where

\begin{eqnarray}
  (x)\id{\{} \lpquote Q \rpquote / \lpquote P \rpquote \id{\}}            = 
  \left\{ 
    \begin{array}{ccc}
      \lpquote Q \rpquote & & x \nameeq \lpquote P \rpquote \\
      x & & otherwise \\
    \end{array}
  \right. \nonumber
\end{eqnarray}

and $z$ is chosen distinct from $\quotep{P}$, $\quotep{Q}$, the free
names in $Q$, and all the names in $R$. Our $\alpha$-equivalence will
be built in the standard way from this substitution.

\begin{remark}\label{rem:no_self_referential_names}
  One consequence of these definitions is that $\forall P. \quotep{P}
  \not\in \freenames{P}$.
\end{remark}

\subsection{ Dynamic quote: an example }

Anticipating something of what's to come, consider applying the
substitution, $\widehat{\id{\{}u / z \id{\}}}$, to the following pair
of processes, $\lift{w}{y!(z)}$ and $w[ \lpquote y!(z) \rpquote ]$.

\begin{eqnarray}
	\lift{w}{y!(z)}\widehat{\id{\{}u / z \id{\}}}
		& = &
		\lift{w}{y!(u)} \nonumber\\
	w[ \lpquote y!(z) \rpquote ] \widehat{ \id{\{}u / z \id{\}} }
		& = &
		w[ \lpquote y!(z) \rpquote ] \nonumber
\end{eqnarray}

Because the body of the process between quotes is impervious to
substitution, we get radically different answers. In fact, by
examining the first process in an input context,
e.g. $x?(z).\lift{w}{y!(z)}$, we see that the process under the lift
operator may be shaped by prefixed inputs binding a name inside it. In
this sense, the lift operator will be seen as a way to dynamically
construct processes before reifying them as names.

Finally equipped with these standard features we can present the
dynamics of the calculus.

\subsubsection{Operational semantics} 

Finally, we introduce the computational dynamics. What marks these
algebras as distinct from other more traditionally studied algebraic
structures, e.g. vector spaces or polynomial rings, is the manner in
which dynamics is captured. In traditional structures, dynamics is typically
expressed through morphisms between such structures, as in linear maps
between vector spaces or morphisms between rings. In algebras
associated with the semantics of computation, the dynamics is
expressed as part of the algebraic structure itself, through a
reduction reduction relation typically denoted by $\red$. Below, we
give a recursive presentation of this relation for the calculus used
in the encoding.

$\red \subseteq \pi \times \pi$
$\red : \pi \to \mathcal{P}(\pi)$

\begin{mathpar}
  \inferrule* [lab=Comm] { \textsf{match}( x_{src}, x_{trgt} ) } { x_{trgt}?(y)P \; | \; x_{src}!\langle {Q} \rangle \red P\{\quotep{Q}/y}\} }
  \and \\
  \inferrule* [lab=Par] {{P} \red {P}'} {{{P} | {Q}} \red {{P}' | {Q}}}
  \and
  \inferrule* [lab=Equiv]{{{P} \scong {P}'} \andalso {{P}' \red {Q}'} \andalso {{Q}' \scong {Q}}}{{P} \red {Q}}
\end{mathpar}

\begin{eqnarray*}
  match_{\equiv} (\quotep{P},\quotep{Q}) & := & P \equiv Q \\
  match_{\dagger}(\quotep{P},\quotep{Q}) & := & \forall R. P|Q \red^{*} R => R \red^{*} 0 \\
  match_{K}(\quotep{P},\quotep{Q}) & := & K \mbox{ for some context } K
\end{eqnarray*}

$u?(x)P | u!\langle Q \rangle \red P\{\quotep{Q}/x\}$

%We write $\wred$ for $\red^*$, and $P\red$ if $\exists Q $ such that $ P \red Q$.
We write $P\red$ if $\exists Q $ such that $ P \red Q$ and $P\not\red$, otherwise.

\section{Replication}

As mentioned before, it is known that replication (and hence
recursion) can be implemented in a higher-order process algebra
\cite{SangiorgiWalker}. As our first example of calculation with the
machinery thus far presented we give the construction explicitly in
the {\rhoc}.

\begin{eqnarray}
	D_{x} & := & \prefix{x}{y}{(\binpar{\outputp{x}{y}}{@{y}})} \nonumber\\
	\bangp_{x}{P} & := & \binpar{{x}!\langle{\binpar{D_{x}}{P}}\rangle}{D_{x}} \nonumber
\end{eqnarray}

\begin{eqnarray}
	\bangp_{x}{P} & & \nonumber\\
	=
	& {x}!\langle{(\prefix{x}{y}{(\outputp{x}{y} | @{y})) | P}}\rangle 
	      | \prefix{x}{y}{(\outputp{x}{y} | @{y})} & \nonumber\\
	\red
	& (\outputp{x}{y} | @{y})\substn{\quotep{(\prefix{x}{y}{(@{y} | \outputp{x}{y})) | P}}}{y} & \nonumber\\
	=
	& \outputp{x}{\quotep{(\prefix{x}{y}{(\outputp{x}{y} | @{y})) | P}}}
	  | {(\prefix{x}{y}{(\outputp{x}{y} | @{y})) | P}} & \nonumber\\
	\red
	& \ldots & \nonumber\\
	\red^*
	& P | P | \ldots & \nonumber
\end{eqnarray}

Of course, this encoding, as an implementation, runs away, unfolding
$\bangp{P}$ eagerly. A lazier and more implementable replication
operator, restricted to input-guarded processes, may be obtained as follows.

\begin{eqnarray}
\bangp{\prefix{u}{v}{P}} 
	:= 
	\binpar{\lift{x}{\prefix{u}{v}{(\binpar{D(x)}{P})}}}{D(x)} \nonumber
\end{eqnarray}

\begin{remark}
  Note that the lazier definition still does not deal with summation
  or mixed summation (i.e. sums over input and output). The reader is
  invited to construct definitions of replication that deal with these
  features. 

  Further, the definitions are parameterized in a name, $x$. Can you,
  gentle reader, make a definition that eliminates this parameter and
  guarantees no accidental interaction between the replication
  machinery and the process being replicated -- i.e. no accidental
  sharing of names used by the process to get its work done and the
  name(s) used by the replication to effect copying. This latter
  revision of the definition of replication is crucial to obtaining
  the expected identity $!!P \sim !P$.
\end{remark}

\begin{remark}\label{rem:paradoxical_combinator}
  The reader familiar with the lambda calculus will have noticed the
  similarity between $D$ and the paradoxical combinator.

  [Ed. note: the existence of this seems to suggest we have to be more
  restrictive on the set of processes and names we admit if we are to
  support no-cloning.]
\end{remark}

\subsubsection{Bisimulation}

The computational dynamics gives rise to another kind of equivalence,
the equivalence of computational behavior. As previously mentioned
this is typically captured \emph{via} some form of bisimulation.

% The notion we use in this paper is weak barbed bisimulation
% \cite{milner91polyadicpi}.

The notion we use in this paper is derived from weak barbed
bisimulation \cite{milner91polyadicpi}. 

\begin{definition}
An \emph{observation relation}, $\downarrow_{\mathcal N}$, over a set
of names, $\mathcal N$, is the smallest relation satisfying the rules
below.

\infrule[Out-barb]{y \in {\mathcal N}, \; x \nameeq y}
		  {\outputp{x}{v} \downarrow_{\mathcal N} x}
\infrule[Par-barb]{\mbox{$P\downarrow_{\mathcal N} x$ or $Q\downarrow_{\mathcal N} x$}}
		  {\binpar{P}{Q} \downarrow_{\mathcal N} x}

We write $P \Downarrow_{\mathcal N} x$ if there is $Q$ such that 
$P \wred Q$ and $Q \downarrow_{\mathcal N} x$.
\end{definition}

\begin{definition}
%\label{def.bbisim}
An  ${\mathcal N}$-\emph{barbed bisimulation} over a set of names, ${\mathcal N}$, is a symmetric binary relation 
${\mathcal S}_{\mathcal N}$ between agents such that $P\rel{S}_{\mathcal N}Q$ implies:
\begin{enumerate}
\item If $P \red P'$ then $Q \wred Q'$ and $P'\rel{S}_{\mathcal N} Q'$.
\item If $P\downarrow_{\mathcal N} x$, then $Q\Downarrow_{\mathcal N} x$.
\end{enumerate}
$P$ is ${\mathcal N}$-barbed bisimilar to $Q$, written
$P \wbbisim_{\mathcal N} Q$, if $P \rel{S}_{\mathcal N} Q$ for some ${\mathcal N}$-barbed bisimulation ${\mathcal S}_{\mathcal N}$.
\end{definition}

$\mathcal{R} \subseteq \pi \times \pi$

$P \mathcal{R} Q => \forall P'. P \red P' \Rightarrow \exists Q'. Q \red Q', P' \mathcal{R} Q'$

$P \vdash x \Rightarrow Q \vdash x$

\begin{mathpar}
  \inferrule*[lab=Out-barb]{x \nameeq y}{{y}!\langle{Q}\rangle \vdash x}
  \and
  \inferrule*[lab=Par-barb]{\mbox{$P\vdash x$ or $Q\vdash x$}}{\binpar{P}{Q} \vdash x}
\end{mathpar}

\subsubsection{Contexts}

One of the principle advantages of computational calculi like the
$\pi$-calculus is a well-defined notion of context,
contextual-equivalence and a correlation between
contextual-equivalence and notions of bisimulation. The notion of
context allows the decomposition of a process into (sub-)process and
its syntactic environment, its context. Thus, a context may be
thought of as a process with a ``hole'' (written $\Box$) in it. The
application of a context $M$ to a process $P$, written $M[P]$, is
tantamount to filling the hole in $M$ with $P$. In this paper we do
not need the full weight of this theory, but do make use of the notion
of context in the proof the main theorem. 

\begin{mathpar}
  \inferrule* [lab=summation] {} {{M_{M},M_{N}} \bc \Box \;|\; x.M_{A} \;|\; M_{M}+M_{N}}
  \and
  \inferrule* [lab=agent] {} {{M_{A}} \bc (\vec{x})M_{P} \;| \; \clift{P_0,\ldots,M_{P},\ldots,P_N}}
  \and \\
  \inferrule* [lab=process] {} {{M_{P}} \bc M_{N} \;| \;P|M_{P} }
\end{mathpar} 

\begin{mathpar}
  \inferrule* [lab=sychronization] {} {M_{N} \bc \Box \;|\; x?M_{F} \;|\; x!M_{C}}
  \and
  \inferrule* [lab=abstraction] {} {{M_{F}} \bc (x)M_{P} }
  \and
  \inferrule* [lab=concretion] {} {{M_{C}} \bc \langle M_{P} \rangle }
  \and \\
  \inferrule* [lab=process] {} {{M_{P}} \bc M_{N} \;| \;P|M_{P} }
\end{mathpar}

\begin{definition}[contextual application] Given a context $M$, and
  process $P$, we define the \emph{contextual application}, $M[P] :=
  M\{P/\Box\}$. That is, the contextual application of M to P is the
  substitution of $P$ for $\Box$ in $M$.
\end{definition}

$\meaningof{-} : L \to \mathcal{P}(\pi)$

\begin{mathpar}
  \inferrule* [lab=collection] {} {\meaningof{true} = \pi, \and \meaningof{~E} = \pi \setminus \meaningof{E}, \and \meaningof{E_{1} \& E_{2}} = \meaningof{E_{1}} \cap \meaningof{E_{2}}}
\end{mathpar}

\begin{mathpar}
  \inferrule* [lab=structure] {} {\meaningof{0} = \{ P \in \pi | P \equiv 0 \}, \and \\ \meaningof{E_1 | E_2} = \{ P \in \pi | P \equiv P_{1} | P_{2}, P_{1} \in \meaningof{E_{1}}, P_{2} \in \meaningof{E_2}\} }
\end{mathpar}

\begin{mathpar}
 \inferrule* [lab=behavior] {} {\meaningof{\langle a?b \rangle E} = \{ P \in \pi | P \equiv Q | u?(y)P', \\ \and \\\\ \and \\ \;\;\; u \in \meaningof{a}, \forall z.P'\{z/y\} \in \meaningof{E\{z/b\}}\}, \and \\ \meaningof{a!E} = \{ P \in \pi | P \equiv Q | x!\langle P' \rangle, x \in \meaningof{a} P' \in \meaningof{E}\} }
\end{mathpar}

\begin{mathpar}
 \inferrule* [lab=nominal] {} {\meaningof{\quotep{E}} = \{ \quotep{P} \in \quotep{\pi} | P \in \meaningof{E} \}, \and \meaningof{\quotep{P}} = \{ \quotep{Q} \in \quotep{\pi} | P \equiv Q \} \and \\ \meaningof{@\quotep{E}} = \{ P \in \pi | P \equiv @x, x \in \meaningof{E} \}}
\end{mathpar}

\begin{eqnarray*}
  \\
  \meaningof{-} : TS \to ST
\end{eqnarray*}

\begin{eqnarray*}
  \\
  L : TS \to ST
\end{eqnarray*}

\begin{eqnarray*}
  \\
  P \models E \iff P \in \meaningof{E}
\end{eqnarray*}

\begin{eqnarray*}
  P \approx_{L} Q \iff \forall E \in L. P \models E \iff Q \models E
\end{eqnarray*}

\begin{eqnarray*}
  P \approx_{K} Q
\end{eqnarray*}

\begin{eqnarray*}
  P \approx Q
\end{eqnarray*}

$\approx_{K} = \approx = \approx_{L}$

\subsubsection{Contextual duality}

Note that contexts extend the quotation operation to a family of
operations from processes to names. Given a context, $M$, we can
define a \emph{nominal context}, $\quotep{M}$ by $\quotep{M}[P] :=
\quotep{M[P]}$. To foreshadow what is to come we observe that these
operations enjoy a duality with processes very much like the duality
between vectors and maps from vectors to scalars.

Further, because the calculus is essentially higher-order, we have a
correspondence between contexts and processes. More specifically,
given a name $x$ and a context $M$ we can construct $M^{*}_{x}$ such
that 

\begin{mathpar}
  M^{*}_{x} | \lift{x}{P} \red M[P]
\end{mathpar}

namely,

\begin{mathpar}
  M^{*}_{x} := x?(u).M[\dropn{u}]
\end{mathpar}

The dependence of $M^{*}_{x}$ on a name makes it an abstraction, 

\begin{mathpar}
  M^{*} := (x)x?(u).M[\dropn{u}]
\end{mathpar}

\subsection{Additional notation}

It will sometimes be convenient to denote the process a name
quotes. We already have the notation $x = \quotep{P}$, but it will be
convenient to introduce an alternate notation, $\procn{x}$, when we
want to emphasize the connection to the use of the name. Note that, by
virtue of name equivalence, $\quotep{\procn{x}} \nameeq x$; so, the
notation is consistent with previous definitions.

Further, because names have structure it is possible to effect
substitutions on the basis of that structure. This means we need to
upgrade our notation for substitutions, which we accomplish by
adapting comprehension notation. Thus,

\begin{mathpar}
  P\{ y / x : x \in S \}
\end{mathpar}

is interpreted to mean the process derived from P by replacing (in a
capture-avoiding manner) each occurrence of $x$ in $S$ by $y$. For example,

\begin{mathpar}
  P\{ \quotep{\procn{x}|\procn{x}} / x : x \in \freenames{P} \}
\end{mathpar}

will replace each (occurrence) of a free name $x$ in $P$ by
$\quotep{\procn{x}|\procn{x}}$.

Also, we will avail ourselves of the notation $x^{L}$ and $x^{R}$ to
denote injections of a name into disjoint copies of the name
space. There are numerous ways to accomplish this. One example can be
found in \cite{MeredithR05}. This notation overloads to vectors of
names: $\vec{x}^{\pi} := (x_{i}^{\pi} \; : \; 0 \leq i < |\vec{x}| )$ where $\pi \in \{L,R\}$.

We also use $P^{\Box} := P|\Box$.

In \cite{MeredithR05} an interpretation of the new operator is
given. It turns out that there are several possible interpretations
all enjoying the requisite algebraic properties of the operator (see
\cite{milner91polyadicpi}). We will therefore make liberal use of
$(\nu\; \vec{x})P$.

% subsection the_syntax_and_semantics_of_the_notation_system (end)   

\input{qm2pi.qmops} 

\input{qm2pi.sterngerlach} 

\input{qm2pi.metric} 

% section concurrent_process_calculi (end)

%\input{qm2pi.proofsketch}

% section proof sketch (end)

%\input{qm2pi.slviaknots} 

% section spatial logic via knots (end)

\input{qm2pi.conclusion}

% section conclusion (end)

%\input{qm2pi.dtcodes} 

% section wiring algorithm (end)

\input{qm2pi.ack} 

% section acknowledgments (end)

\newpage


\bibliographystyle{plain}   
\bibliography{../../biblios/main.bib}

\input{qm2pi.rhodetails}

\end{document}



\end{document}

 

%\documentclass[12pt]{llncs}
%\documentclass{jktr}

\usepackage[pdftex]{hyperref}                   
\usepackage {listings}
\usepackage {mathpartir}
\usepackage{bcprules}
%\usepackage{listings}
                       
\usepackage{graphicx} 
%\usepackage[margins=2.5cm,nohead,nofoot]{geometry}
%\usepackage{geometry}
\usepackage{amsfonts}
\usepackage{amstext}
\usepackage{latexsym}
\usepackage{amssymb}
\usepackage{color}


%\include{myPreamble}
\documentclass[12pt]{llncs}
%\documentclass{jktr}

\usepackage[pdftex]{hyperref}                   
\usepackage {listings}
\usepackage {mathpartir}
\usepackage{bcprules}
%\usepackage{listings}
                       
\usepackage{graphicx} 
%\usepackage[margins=2.5cm,nohead,nofoot]{geometry}
%\usepackage{geometry}
\usepackage{amsfonts}
\usepackage{amstext}
\usepackage{latexsym}
\usepackage{amssymb}
\usepackage{color}


%\include{myPreamble}
\include{qm2pi.local} 

%\ifpdf
%\usepackage[pdftex]{graphicx}
%\else
%\usepackage{graphicx}
%\fi

 % \ifpdf
%  \usepackage{pdfsync}
%  \if


%\title{Brief Article}
%\author{David F. Snyder}
%\author{L.G. Meredith}

%\address{Dept. of Math., Texas State University--San Marcos, San Marcos, TX 78666}
       
\pagestyle{empty}


\begin{document}

\lstset{language=[Objective]Caml,frame=shadowbox}

\input{qm2pi.front}

% section front matter (end)

\input{qm2pi.intro} 
 
% section introduction (end)

% \input{qm2pi.knotations} 

% section notation (end)

\input{qm2pi.process.calculi} 

% section concurrent_process_calculi_and_spatial_logics_ (end)
    
%\input{qm2pi.knots2pi} 

%\input{qm2pi.trefoil} 

%\input{qm2pi.mainthm} 

% subsection basic_interpretation (end)

%\input{qm2pi.rho.presentation} 
\subsection{The syntax and semantics of the notation system}\label{sub:the_syntax_and_semantics_of_the_notation_system} % (fold)

We now summarize a technical presentation of the calculus that
embodies our theory of dynamics. The typical presentation of such a
calculus follows the style of giving generators and relations on
them. The grammar, below, describing term constructors, freely
generates the set of processes, $\Proc$. This set is then quotiented
by a relation known as structural congruence and it is over this set
that the notion of dynamics is expressed. This presentation is
essentially that of \cite{MeredithR05} with the addition of
polyadicity and summation. For readability we have relegated some of
the technical subtleties to an appendix.

\subsubsection{Process grammar}\label{subsub:process_grammar}

\begin{mathpar}
  \inferrule* [lab=synchronization] {} {{M} \bc \pzero \;|\; x?F \;|\; x!C }
  \and
  \inferrule* [lab=abstraction] {} {{F} \bc (x)P}
  \and
  \inferrule* [lab=concretion] {} {{C} \bc \langle Q \rangle}
  \and
  \inferrule* [lab=process] {} {{P,Q} \bc M \;| \;P|Q \;|\; @{x}}
  \and
  \inferrule* [lab=name] {} {{x} \bc \quotep{P}}
\end{mathpar} 

Note that $\vec{x}$ (resp. $\vec{P}$) denotes a vector of names
(resp. processes) of length $|\vec{x}|$ (resp. $|\vec{P}|$). We adopt
the following useful abbreviations.

\begin{mathpar}
   x?(\vec{y}).P := x.(\vec{y})P \and  x\clift{\vec{P}} := x.\clift{\vec{P}}
   \and x!(y) := \lift{x}{\dropn{y}}
   \and \Pi_{i=0}^{n-1}P_i := P_0 | \ldots | P_{n-1}
\end{mathpar}

\subsubsection{Structural congruence}

\paragraph{Free and bound names and alpha-equivalence.} At the
core of structural equivalence is alpha-equivalence which identifies
process that are the same up to a change of variable. Formally, we
recognize the distinction between free and bound names. The free names
of a process, $\freenames{P}$, may be calculated recursively as
follows:

\begin{mathpar}
\freenames{\pzero} := \emptyset
  \and \\
  \freenames{x?(y).P} := \{ x \} \cup (\freenames{P} \setminus \{ y \})
  \and 
  \freenames{x!\langle P \rangle} := \{ x \} \cup \{ P \} 
  \and \\
  \freenames{P|Q} := \freenames{P} \cup \freenames{Q}
  \and \\
  \freenames{@{x}} := \{ x \}
\end{mathpar}

$\pi$
$\quotep{\pi}$

$\freenames{-} : \pi \to \mathcal{P}(\quotep{\pi})$

\begin{eqnarray*}
  \freenames{\pzero} & := & \emptyset \\
  \freenames{x?(y).P} & := & \{ x \} \cup (\freenames{P} \setminus \{ y \}) \\
  \freenames{x!\langle P \rangle} & := & \{ x \} \cup \{ P \} \\
  \freenames{P|Q} & := & \freenames{P} \cup \freenames{Q} \\
  \freenames{\dropn{x}} & := & \{ x \}
\end{eqnarray*}

The bound names of a process, $\boundnames{P}$, are those names occurring in $P$
that are not free. For example, in $x?(y).0$, the name $x$ is free, while $y$ is bound.

\begin{mathpar}
  \inferrule* [lab=monoidal-laws] {} { P|Q \equiv Q|P \and P|0 \equiv P \and P|(Q|R) \equiv (P|Q)|R }
\end{mathpar}

\begin{mathpar}
  \inferrule* [lab=alpha-equivalence] {} { (x)P \equiv (y)P\{y/x\} \and y \not\in \freenames{P} }
\end{mathpar}

\begin{definition}
Then two processes, $P,Q$, are alpha-equivalent if $P = Q\{\vec{y}/\vec{x}\}$ for
some $\vec{x} \in \boundnames{Q},\vec{y} \in \boundnames{P}$, where $Q\{\vec{y}/\vec{x}\}$
denotes the capture-avoiding substitution of $\vec{y}$ for $\vec{x}$ in $Q$.
\end{definition}

\begin{definition}
  The {\em structural congruence} \cite{SangiorgiWalker} , $\equiv$,
  between processes is the least congruence containing
  alpha-equivalence, satisfying the abelian monoid laws
  (associativity, commutativity and $\pzero$ as identity) for parallel
  composition $|$ and for summation $+$.
\end{definition}

\subsection{Name equivalence}

We take name equivalence, written $\nameeq$, to be the smallest
equivalence relation generated by the following rules.

\begin{mathpar}
\inferrule*[lab=Quote-drop]
{ }
{ \quotep{@{x}} \nameeq x }

\inferrule*[lab=Struct-equiv]
{ P \scong Q }
{ \quotep{P} \nameeq \quotep{Q} }
\end{mathpar}

The astute reader will have noticed that the mutual recursion of names
and processes imposes a mutual recursion on alpha-equivalence and
structural equivalence via name-equivalence. Fortunately, all of this
works out pleasantly and we may calculate in the natural way, free of
concern. The reader interested in the details is referred to the
appendix \ref{appendix:rho_details}.

\subsection{Substitution}

We use $\Proc$ for the set of processes, $\QProc$ for the set of
names, and $\id{\{}\vec{y} / \vec{x} \id{\}}$ to denote partial maps,
$s : \QProc \rightarrow \QProc$. A map, $s$ lifts, uniquely, to a map
on process terms, $\widehat{s} : \Proc \rightarrow \Proc$ by the
following equations.

\begin{mathpar}
  (0) \psubstp{Q}{P} := 0 \\
  (R \juxtap S) \psubstp{Q}{P}
  :=    
  (R)\psubstp{Q}{P} \juxtap (S) \psubstp{Q}{P} \\
  (x?(y).R) \psubstp{Q}{P}    
  :=    
  (x)\substp{Q}{P} (z)\concat( (R \psubstn{z}{y}) \psubstp{Q}{P} ) \\
  (\lift{x}{R}) \psubstp{Q}{P}  
  :=
  \lift{(x)\substp{Q}{P}}{ R \psubstp{Q}{P} } \\
%   (\dropn{x})  \psubstp{Q}{P}       
%   := 
%   \left\{ 
%     \begin{array}{ccc} 
%       \dropn{\quotep{Q}} & & x \nameeq \quotep{P} \\
%       \dropn{x} & & otherwise \\
%     \end{array}
%   \right. 
  (\dropn{x})  \psubstp{Q}{P}       
  := 
  \left\{ 
    \begin{array}{ccc} 
      Q & & x \nameeq \quotep{P} \\
      \dropn{x} & & otherwise \\
    \end{array}
  \right.
\end{mathpar}
 

where

\begin{eqnarray}
  (x)\id{\{} \lpquote Q \rpquote / \lpquote P \rpquote \id{\}}            = 
  \left\{ 
    \begin{array}{ccc}
      \lpquote Q \rpquote & & x \nameeq \lpquote P \rpquote \\
      x & & otherwise \\
    \end{array}
  \right. \nonumber
\end{eqnarray}

and $z$ is chosen distinct from $\quotep{P}$, $\quotep{Q}$, the free
names in $Q$, and all the names in $R$. Our $\alpha$-equivalence will
be built in the standard way from this substitution.

\begin{remark}\label{rem:no_self_referential_names}
  One consequence of these definitions is that $\forall P. \quotep{P}
  \not\in \freenames{P}$.
\end{remark}

\subsection{ Dynamic quote: an example }

Anticipating something of what's to come, consider applying the
substitution, $\widehat{\id{\{}u / z \id{\}}}$, to the following pair
of processes, $\lift{w}{y!(z)}$ and $w[ \lpquote y!(z) \rpquote ]$.

\begin{eqnarray}
	\lift{w}{y!(z)}\widehat{\id{\{}u / z \id{\}}}
		& = &
		\lift{w}{y!(u)} \nonumber\\
	w[ \lpquote y!(z) \rpquote ] \widehat{ \id{\{}u / z \id{\}} }
		& = &
		w[ \lpquote y!(z) \rpquote ] \nonumber
\end{eqnarray}

Because the body of the process between quotes is impervious to
substitution, we get radically different answers. In fact, by
examining the first process in an input context,
e.g. $x?(z).\lift{w}{y!(z)}$, we see that the process under the lift
operator may be shaped by prefixed inputs binding a name inside it. In
this sense, the lift operator will be seen as a way to dynamically
construct processes before reifying them as names.

Finally equipped with these standard features we can present the
dynamics of the calculus.

\subsubsection{Operational semantics} 

Finally, we introduce the computational dynamics. What marks these
algebras as distinct from other more traditionally studied algebraic
structures, e.g. vector spaces or polynomial rings, is the manner in
which dynamics is captured. In traditional structures, dynamics is typically
expressed through morphisms between such structures, as in linear maps
between vector spaces or morphisms between rings. In algebras
associated with the semantics of computation, the dynamics is
expressed as part of the algebraic structure itself, through a
reduction reduction relation typically denoted by $\red$. Below, we
give a recursive presentation of this relation for the calculus used
in the encoding.

$\red \subseteq \pi \times \pi$
$\red : \pi \to \mathcal{P}(\pi)$

\begin{mathpar}
  \inferrule* [lab=Comm] { \textsf{match}( x_{src}, x_{trgt} ) } { x_{trgt}?(y)P \; | \; x_{src}!\langle {Q} \rangle \red P\{\quotep{Q}/y}\} }
  \and \\
  \inferrule* [lab=Par] {{P} \red {P}'} {{{P} | {Q}} \red {{P}' | {Q}}}
  \and
  \inferrule* [lab=Equiv]{{{P} \scong {P}'} \andalso {{P}' \red {Q}'} \andalso {{Q}' \scong {Q}}}{{P} \red {Q}}
\end{mathpar}

\begin{eqnarray*}
  match_{\equiv} (\quotep{P},\quotep{Q}) & := & P \equiv Q \\
  match_{\dagger}(\quotep{P},\quotep{Q}) & := & \forall R. P|Q \red^{*} R => R \red^{*} 0 \\
  match_{K}(\quotep{P},\quotep{Q}) & := & K \mbox{ for some context } K
\end{eqnarray*}

$u?(x)P | u!\langle Q \rangle \red P\{\quotep{Q}/x\}$

%We write $\wred$ for $\red^*$, and $P\red$ if $\exists Q $ such that $ P \red Q$.
We write $P\red$ if $\exists Q $ such that $ P \red Q$ and $P\not\red$, otherwise.

\section{Replication}

As mentioned before, it is known that replication (and hence
recursion) can be implemented in a higher-order process algebra
\cite{SangiorgiWalker}. As our first example of calculation with the
machinery thus far presented we give the construction explicitly in
the {\rhoc}.

\begin{eqnarray}
	D_{x} & := & \prefix{x}{y}{(\binpar{\outputp{x}{y}}{@{y}})} \nonumber\\
	\bangp_{x}{P} & := & \binpar{{x}!\langle{\binpar{D_{x}}{P}}\rangle}{D_{x}} \nonumber
\end{eqnarray}

\begin{eqnarray}
	\bangp_{x}{P} & & \nonumber\\
	=
	& {x}!\langle{(\prefix{x}{y}{(\outputp{x}{y} | @{y})) | P}}\rangle 
	      | \prefix{x}{y}{(\outputp{x}{y} | @{y})} & \nonumber\\
	\red
	& (\outputp{x}{y} | @{y})\substn{\quotep{(\prefix{x}{y}{(@{y} | \outputp{x}{y})) | P}}}{y} & \nonumber\\
	=
	& \outputp{x}{\quotep{(\prefix{x}{y}{(\outputp{x}{y} | @{y})) | P}}}
	  | {(\prefix{x}{y}{(\outputp{x}{y} | @{y})) | P}} & \nonumber\\
	\red
	& \ldots & \nonumber\\
	\red^*
	& P | P | \ldots & \nonumber
\end{eqnarray}

Of course, this encoding, as an implementation, runs away, unfolding
$\bangp{P}$ eagerly. A lazier and more implementable replication
operator, restricted to input-guarded processes, may be obtained as follows.

\begin{eqnarray}
\bangp{\prefix{u}{v}{P}} 
	:= 
	\binpar{\lift{x}{\prefix{u}{v}{(\binpar{D(x)}{P})}}}{D(x)} \nonumber
\end{eqnarray}

\begin{remark}
  Note that the lazier definition still does not deal with summation
  or mixed summation (i.e. sums over input and output). The reader is
  invited to construct definitions of replication that deal with these
  features. 

  Further, the definitions are parameterized in a name, $x$. Can you,
  gentle reader, make a definition that eliminates this parameter and
  guarantees no accidental interaction between the replication
  machinery and the process being replicated -- i.e. no accidental
  sharing of names used by the process to get its work done and the
  name(s) used by the replication to effect copying. This latter
  revision of the definition of replication is crucial to obtaining
  the expected identity $!!P \sim !P$.
\end{remark}

\begin{remark}\label{rem:paradoxical_combinator}
  The reader familiar with the lambda calculus will have noticed the
  similarity between $D$ and the paradoxical combinator.

  [Ed. note: the existence of this seems to suggest we have to be more
  restrictive on the set of processes and names we admit if we are to
  support no-cloning.]
\end{remark}

\subsubsection{Bisimulation}

The computational dynamics gives rise to another kind of equivalence,
the equivalence of computational behavior. As previously mentioned
this is typically captured \emph{via} some form of bisimulation.

% The notion we use in this paper is weak barbed bisimulation
% \cite{milner91polyadicpi}.

The notion we use in this paper is derived from weak barbed
bisimulation \cite{milner91polyadicpi}. 

\begin{definition}
An \emph{observation relation}, $\downarrow_{\mathcal N}$, over a set
of names, $\mathcal N$, is the smallest relation satisfying the rules
below.

\infrule[Out-barb]{y \in {\mathcal N}, \; x \nameeq y}
		  {\outputp{x}{v} \downarrow_{\mathcal N} x}
\infrule[Par-barb]{\mbox{$P\downarrow_{\mathcal N} x$ or $Q\downarrow_{\mathcal N} x$}}
		  {\binpar{P}{Q} \downarrow_{\mathcal N} x}

We write $P \Downarrow_{\mathcal N} x$ if there is $Q$ such that 
$P \wred Q$ and $Q \downarrow_{\mathcal N} x$.
\end{definition}

\begin{definition}
%\label{def.bbisim}
An  ${\mathcal N}$-\emph{barbed bisimulation} over a set of names, ${\mathcal N}$, is a symmetric binary relation 
${\mathcal S}_{\mathcal N}$ between agents such that $P\rel{S}_{\mathcal N}Q$ implies:
\begin{enumerate}
\item If $P \red P'$ then $Q \wred Q'$ and $P'\rel{S}_{\mathcal N} Q'$.
\item If $P\downarrow_{\mathcal N} x$, then $Q\Downarrow_{\mathcal N} x$.
\end{enumerate}
$P$ is ${\mathcal N}$-barbed bisimilar to $Q$, written
$P \wbbisim_{\mathcal N} Q$, if $P \rel{S}_{\mathcal N} Q$ for some ${\mathcal N}$-barbed bisimulation ${\mathcal S}_{\mathcal N}$.
\end{definition}

$\mathcal{R} \subseteq \pi \times \pi$

$P \mathcal{R} Q => \forall P'. P \red P' \Rightarrow \exists Q'. Q \red Q', P' \mathcal{R} Q'$

$P \vdash x \Rightarrow Q \vdash x$

\begin{mathpar}
  \inferrule*[lab=Out-barb]{x \nameeq y}{{y}!\langle{Q}\rangle \vdash x}
  \and
  \inferrule*[lab=Par-barb]{\mbox{$P\vdash x$ or $Q\vdash x$}}{\binpar{P}{Q} \vdash x}
\end{mathpar}

\subsubsection{Contexts}

One of the principle advantages of computational calculi like the
$\pi$-calculus is a well-defined notion of context,
contextual-equivalence and a correlation between
contextual-equivalence and notions of bisimulation. The notion of
context allows the decomposition of a process into (sub-)process and
its syntactic environment, its context. Thus, a context may be
thought of as a process with a ``hole'' (written $\Box$) in it. The
application of a context $M$ to a process $P$, written $M[P]$, is
tantamount to filling the hole in $M$ with $P$. In this paper we do
not need the full weight of this theory, but do make use of the notion
of context in the proof the main theorem. 

\begin{mathpar}
  \inferrule* [lab=summation] {} {{M_{M},M_{N}} \bc \Box \;|\; x.M_{A} \;|\; M_{M}+M_{N}}
  \and
  \inferrule* [lab=agent] {} {{M_{A}} \bc (\vec{x})M_{P} \;| \; \clift{P_0,\ldots,M_{P},\ldots,P_N}}
  \and \\
  \inferrule* [lab=process] {} {{M_{P}} \bc M_{N} \;| \;P|M_{P} }
\end{mathpar} 

\begin{mathpar}
  \inferrule* [lab=sychronization] {} {M_{N} \bc \Box \;|\; x?M_{F} \;|\; x!M_{C}}
  \and
  \inferrule* [lab=abstraction] {} {{M_{F}} \bc (x)M_{P} }
  \and
  \inferrule* [lab=concretion] {} {{M_{C}} \bc \langle M_{P} \rangle }
  \and \\
  \inferrule* [lab=process] {} {{M_{P}} \bc M_{N} \;| \;P|M_{P} }
\end{mathpar}

\begin{definition}[contextual application] Given a context $M$, and
  process $P$, we define the \emph{contextual application}, $M[P] :=
  M\{P/\Box\}$. That is, the contextual application of M to P is the
  substitution of $P$ for $\Box$ in $M$.
\end{definition}

$\meaningof{-} : L \to \mathcal{P}(\pi)$

\begin{mathpar}
  \inferrule* [lab=collection] {} {\meaningof{true} = \pi, \and \meaningof{~E} = \pi \setminus \meaningof{E}, \and \meaningof{E_{1} \& E_{2}} = \meaningof{E_{1}} \cap \meaningof{E_{2}}}
\end{mathpar}

\begin{mathpar}
  \inferrule* [lab=structure] {} {\meaningof{0} = \{ P \in \pi | P \equiv 0 \}, \and \\ \meaningof{E_1 | E_2} = \{ P \in \pi | P \equiv P_{1} | P_{2}, P_{1} \in \meaningof{E_{1}}, P_{2} \in \meaningof{E_2}\} }
\end{mathpar}

\begin{mathpar}
 \inferrule* [lab=behavior] {} {\meaningof{\langle a?b \rangle E} = \{ P \in \pi | P \equiv Q | u?(y)P', \\ \and \\\\ \and \\ \;\;\; u \in \meaningof{a}, \forall z.P'\{z/y\} \in \meaningof{E\{z/b\}}\}, \and \\ \meaningof{a!E} = \{ P \in \pi | P \equiv Q | x!\langle P' \rangle, x \in \meaningof{a} P' \in \meaningof{E}\} }
\end{mathpar}

\begin{mathpar}
 \inferrule* [lab=nominal] {} {\meaningof{\quotep{E}} = \{ \quotep{P} \in \quotep{\pi} | P \in \meaningof{E} \}, \and \meaningof{\quotep{P}} = \{ \quotep{Q} \in \quotep{\pi} | P \equiv Q \} \and \\ \meaningof{@\quotep{E}} = \{ P \in \pi | P \equiv @x, x \in \meaningof{E} \}}
\end{mathpar}

\begin{eqnarray*}
  \\
  \meaningof{-} : TS \to ST
\end{eqnarray*}

\begin{eqnarray*}
  \\
  L : TS \to ST
\end{eqnarray*}

\begin{eqnarray*}
  \\
  P \models E \iff P \in \meaningof{E}
\end{eqnarray*}

\begin{eqnarray*}
  P \approx_{L} Q \iff \forall E \in L. P \models E \iff Q \models E
\end{eqnarray*}

\begin{eqnarray*}
  P \approx_{K} Q
\end{eqnarray*}

\begin{eqnarray*}
  P \approx Q
\end{eqnarray*}

$\approx_{K} = \approx = \approx_{L}$

\subsubsection{Contextual duality}

Note that contexts extend the quotation operation to a family of
operations from processes to names. Given a context, $M$, we can
define a \emph{nominal context}, $\quotep{M}$ by $\quotep{M}[P] :=
\quotep{M[P]}$. To foreshadow what is to come we observe that these
operations enjoy a duality with processes very much like the duality
between vectors and maps from vectors to scalars.

Further, because the calculus is essentially higher-order, we have a
correspondence between contexts and processes. More specifically,
given a name $x$ and a context $M$ we can construct $M^{*}_{x}$ such
that 

\begin{mathpar}
  M^{*}_{x} | \lift{x}{P} \red M[P]
\end{mathpar}

namely,

\begin{mathpar}
  M^{*}_{x} := x?(u).M[\dropn{u}]
\end{mathpar}

The dependence of $M^{*}_{x}$ on a name makes it an abstraction, 

\begin{mathpar}
  M^{*} := (x)x?(u).M[\dropn{u}]
\end{mathpar}

\subsection{Additional notation}

It will sometimes be convenient to denote the process a name
quotes. We already have the notation $x = \quotep{P}$, but it will be
convenient to introduce an alternate notation, $\procn{x}$, when we
want to emphasize the connection to the use of the name. Note that, by
virtue of name equivalence, $\quotep{\procn{x}} \nameeq x$; so, the
notation is consistent with previous definitions.

Further, because names have structure it is possible to effect
substitutions on the basis of that structure. This means we need to
upgrade our notation for substitutions, which we accomplish by
adapting comprehension notation. Thus,

\begin{mathpar}
  P\{ y / x : x \in S \}
\end{mathpar}

is interpreted to mean the process derived from P by replacing (in a
capture-avoiding manner) each occurrence of $x$ in $S$ by $y$. For example,

\begin{mathpar}
  P\{ \quotep{\procn{x}|\procn{x}} / x : x \in \freenames{P} \}
\end{mathpar}

will replace each (occurrence) of a free name $x$ in $P$ by
$\quotep{\procn{x}|\procn{x}}$.

Also, we will avail ourselves of the notation $x^{L}$ and $x^{R}$ to
denote injections of a name into disjoint copies of the name
space. There are numerous ways to accomplish this. One example can be
found in \cite{MeredithR05}. This notation overloads to vectors of
names: $\vec{x}^{\pi} := (x_{i}^{\pi} \; : \; 0 \leq i < |\vec{x}| )$ where $\pi \in \{L,R\}$.

We also use $P^{\Box} := P|\Box$.

In \cite{MeredithR05} an interpretation of the new operator is
given. It turns out that there are several possible interpretations
all enjoying the requisite algebraic properties of the operator (see
\cite{milner91polyadicpi}). We will therefore make liberal use of
$(\nu\; \vec{x})P$.

% subsection the_syntax_and_semantics_of_the_notation_system (end)   

\input{qm2pi.qmops} 

\input{qm2pi.sterngerlach} 

\input{qm2pi.metric} 

% section concurrent_process_calculi (end)

%\input{qm2pi.proofsketch}

% section proof sketch (end)

%\input{qm2pi.slviaknots} 

% section spatial logic via knots (end)

\input{qm2pi.conclusion}

% section conclusion (end)

%\input{qm2pi.dtcodes} 

% section wiring algorithm (end)

\input{qm2pi.ack} 

% section acknowledgments (end)

\newpage


\bibliographystyle{plain}   
\bibliography{../../biblios/main.bib}

\input{qm2pi.rhodetails}

\end{document}

 

%\ifpdf
%\usepackage[pdftex]{graphicx}
%\else
%\usepackage{graphicx}
%\fi

 % \ifpdf
%  \usepackage{pdfsync}
%  \if


%\title{Brief Article}
%\author{David F. Snyder}
%\author{L.G. Meredith}

%\address{Dept. of Math., Texas State University--San Marcos, San Marcos, TX 78666}
       
\pagestyle{empty}


\begin{document}

\lstset{language=[Objective]Caml,frame=shadowbox}

\documentclass[12pt]{llncs}
%\documentclass{jktr}

\usepackage[pdftex]{hyperref}                   
\usepackage {listings}
\usepackage {mathpartir}
\usepackage{bcprules}
%\usepackage{listings}
                       
\usepackage{graphicx} 
%\usepackage[margins=2.5cm,nohead,nofoot]{geometry}
%\usepackage{geometry}
\usepackage{amsfonts}
\usepackage{amstext}
\usepackage{latexsym}
\usepackage{amssymb}
\usepackage{color}


%\include{myPreamble}
\include{qm2pi.local} 

%\ifpdf
%\usepackage[pdftex]{graphicx}
%\else
%\usepackage{graphicx}
%\fi

 % \ifpdf
%  \usepackage{pdfsync}
%  \if


%\title{Brief Article}
%\author{David F. Snyder}
%\author{L.G. Meredith}

%\address{Dept. of Math., Texas State University--San Marcos, San Marcos, TX 78666}
       
\pagestyle{empty}


\begin{document}

\lstset{language=[Objective]Caml,frame=shadowbox}

\input{qm2pi.front}

% section front matter (end)

\input{qm2pi.intro} 
 
% section introduction (end)

% \input{qm2pi.knotations} 

% section notation (end)

\input{qm2pi.process.calculi} 

% section concurrent_process_calculi_and_spatial_logics_ (end)
    
%\input{qm2pi.knots2pi} 

%\input{qm2pi.trefoil} 

%\input{qm2pi.mainthm} 

% subsection basic_interpretation (end)

%\input{qm2pi.rho.presentation} 
\subsection{The syntax and semantics of the notation system}\label{sub:the_syntax_and_semantics_of_the_notation_system} % (fold)

We now summarize a technical presentation of the calculus that
embodies our theory of dynamics. The typical presentation of such a
calculus follows the style of giving generators and relations on
them. The grammar, below, describing term constructors, freely
generates the set of processes, $\Proc$. This set is then quotiented
by a relation known as structural congruence and it is over this set
that the notion of dynamics is expressed. This presentation is
essentially that of \cite{MeredithR05} with the addition of
polyadicity and summation. For readability we have relegated some of
the technical subtleties to an appendix.

\subsubsection{Process grammar}\label{subsub:process_grammar}

\begin{mathpar}
  \inferrule* [lab=synchronization] {} {{M} \bc \pzero \;|\; x?F \;|\; x!C }
  \and
  \inferrule* [lab=abstraction] {} {{F} \bc (x)P}
  \and
  \inferrule* [lab=concretion] {} {{C} \bc \langle Q \rangle}
  \and
  \inferrule* [lab=process] {} {{P,Q} \bc M \;| \;P|Q \;|\; @{x}}
  \and
  \inferrule* [lab=name] {} {{x} \bc \quotep{P}}
\end{mathpar} 

Note that $\vec{x}$ (resp. $\vec{P}$) denotes a vector of names
(resp. processes) of length $|\vec{x}|$ (resp. $|\vec{P}|$). We adopt
the following useful abbreviations.

\begin{mathpar}
   x?(\vec{y}).P := x.(\vec{y})P \and  x\clift{\vec{P}} := x.\clift{\vec{P}}
   \and x!(y) := \lift{x}{\dropn{y}}
   \and \Pi_{i=0}^{n-1}P_i := P_0 | \ldots | P_{n-1}
\end{mathpar}

\subsubsection{Structural congruence}

\paragraph{Free and bound names and alpha-equivalence.} At the
core of structural equivalence is alpha-equivalence which identifies
process that are the same up to a change of variable. Formally, we
recognize the distinction between free and bound names. The free names
of a process, $\freenames{P}$, may be calculated recursively as
follows:

\begin{mathpar}
\freenames{\pzero} := \emptyset
  \and \\
  \freenames{x?(y).P} := \{ x \} \cup (\freenames{P} \setminus \{ y \})
  \and 
  \freenames{x!\langle P \rangle} := \{ x \} \cup \{ P \} 
  \and \\
  \freenames{P|Q} := \freenames{P} \cup \freenames{Q}
  \and \\
  \freenames{@{x}} := \{ x \}
\end{mathpar}

$\pi$
$\quotep{\pi}$

$\freenames{-} : \pi \to \mathcal{P}(\quotep{\pi})$

\begin{eqnarray*}
  \freenames{\pzero} & := & \emptyset \\
  \freenames{x?(y).P} & := & \{ x \} \cup (\freenames{P} \setminus \{ y \}) \\
  \freenames{x!\langle P \rangle} & := & \{ x \} \cup \{ P \} \\
  \freenames{P|Q} & := & \freenames{P} \cup \freenames{Q} \\
  \freenames{\dropn{x}} & := & \{ x \}
\end{eqnarray*}

The bound names of a process, $\boundnames{P}$, are those names occurring in $P$
that are not free. For example, in $x?(y).0$, the name $x$ is free, while $y$ is bound.

\begin{mathpar}
  \inferrule* [lab=monoidal-laws] {} { P|Q \equiv Q|P \and P|0 \equiv P \and P|(Q|R) \equiv (P|Q)|R }
\end{mathpar}

\begin{mathpar}
  \inferrule* [lab=alpha-equivalence] {} { (x)P \equiv (y)P\{y/x\} \and y \not\in \freenames{P} }
\end{mathpar}

\begin{definition}
Then two processes, $P,Q$, are alpha-equivalent if $P = Q\{\vec{y}/\vec{x}\}$ for
some $\vec{x} \in \boundnames{Q},\vec{y} \in \boundnames{P}$, where $Q\{\vec{y}/\vec{x}\}$
denotes the capture-avoiding substitution of $\vec{y}$ for $\vec{x}$ in $Q$.
\end{definition}

\begin{definition}
  The {\em structural congruence} \cite{SangiorgiWalker} , $\equiv$,
  between processes is the least congruence containing
  alpha-equivalence, satisfying the abelian monoid laws
  (associativity, commutativity and $\pzero$ as identity) for parallel
  composition $|$ and for summation $+$.
\end{definition}

\subsection{Name equivalence}

We take name equivalence, written $\nameeq$, to be the smallest
equivalence relation generated by the following rules.

\begin{mathpar}
\inferrule*[lab=Quote-drop]
{ }
{ \quotep{@{x}} \nameeq x }

\inferrule*[lab=Struct-equiv]
{ P \scong Q }
{ \quotep{P} \nameeq \quotep{Q} }
\end{mathpar}

The astute reader will have noticed that the mutual recursion of names
and processes imposes a mutual recursion on alpha-equivalence and
structural equivalence via name-equivalence. Fortunately, all of this
works out pleasantly and we may calculate in the natural way, free of
concern. The reader interested in the details is referred to the
appendix \ref{appendix:rho_details}.

\subsection{Substitution}

We use $\Proc$ for the set of processes, $\QProc$ for the set of
names, and $\id{\{}\vec{y} / \vec{x} \id{\}}$ to denote partial maps,
$s : \QProc \rightarrow \QProc$. A map, $s$ lifts, uniquely, to a map
on process terms, $\widehat{s} : \Proc \rightarrow \Proc$ by the
following equations.

\begin{mathpar}
  (0) \psubstp{Q}{P} := 0 \\
  (R \juxtap S) \psubstp{Q}{P}
  :=    
  (R)\psubstp{Q}{P} \juxtap (S) \psubstp{Q}{P} \\
  (x?(y).R) \psubstp{Q}{P}    
  :=    
  (x)\substp{Q}{P} (z)\concat( (R \psubstn{z}{y}) \psubstp{Q}{P} ) \\
  (\lift{x}{R}) \psubstp{Q}{P}  
  :=
  \lift{(x)\substp{Q}{P}}{ R \psubstp{Q}{P} } \\
%   (\dropn{x})  \psubstp{Q}{P}       
%   := 
%   \left\{ 
%     \begin{array}{ccc} 
%       \dropn{\quotep{Q}} & & x \nameeq \quotep{P} \\
%       \dropn{x} & & otherwise \\
%     \end{array}
%   \right. 
  (\dropn{x})  \psubstp{Q}{P}       
  := 
  \left\{ 
    \begin{array}{ccc} 
      Q & & x \nameeq \quotep{P} \\
      \dropn{x} & & otherwise \\
    \end{array}
  \right.
\end{mathpar}
 

where

\begin{eqnarray}
  (x)\id{\{} \lpquote Q \rpquote / \lpquote P \rpquote \id{\}}            = 
  \left\{ 
    \begin{array}{ccc}
      \lpquote Q \rpquote & & x \nameeq \lpquote P \rpquote \\
      x & & otherwise \\
    \end{array}
  \right. \nonumber
\end{eqnarray}

and $z$ is chosen distinct from $\quotep{P}$, $\quotep{Q}$, the free
names in $Q$, and all the names in $R$. Our $\alpha$-equivalence will
be built in the standard way from this substitution.

\begin{remark}\label{rem:no_self_referential_names}
  One consequence of these definitions is that $\forall P. \quotep{P}
  \not\in \freenames{P}$.
\end{remark}

\subsection{ Dynamic quote: an example }

Anticipating something of what's to come, consider applying the
substitution, $\widehat{\id{\{}u / z \id{\}}}$, to the following pair
of processes, $\lift{w}{y!(z)}$ and $w[ \lpquote y!(z) \rpquote ]$.

\begin{eqnarray}
	\lift{w}{y!(z)}\widehat{\id{\{}u / z \id{\}}}
		& = &
		\lift{w}{y!(u)} \nonumber\\
	w[ \lpquote y!(z) \rpquote ] \widehat{ \id{\{}u / z \id{\}} }
		& = &
		w[ \lpquote y!(z) \rpquote ] \nonumber
\end{eqnarray}

Because the body of the process between quotes is impervious to
substitution, we get radically different answers. In fact, by
examining the first process in an input context,
e.g. $x?(z).\lift{w}{y!(z)}$, we see that the process under the lift
operator may be shaped by prefixed inputs binding a name inside it. In
this sense, the lift operator will be seen as a way to dynamically
construct processes before reifying them as names.

Finally equipped with these standard features we can present the
dynamics of the calculus.

\subsubsection{Operational semantics} 

Finally, we introduce the computational dynamics. What marks these
algebras as distinct from other more traditionally studied algebraic
structures, e.g. vector spaces or polynomial rings, is the manner in
which dynamics is captured. In traditional structures, dynamics is typically
expressed through morphisms between such structures, as in linear maps
between vector spaces or morphisms between rings. In algebras
associated with the semantics of computation, the dynamics is
expressed as part of the algebraic structure itself, through a
reduction reduction relation typically denoted by $\red$. Below, we
give a recursive presentation of this relation for the calculus used
in the encoding.

$\red \subseteq \pi \times \pi$
$\red : \pi \to \mathcal{P}(\pi)$

\begin{mathpar}
  \inferrule* [lab=Comm] { \textsf{match}( x_{src}, x_{trgt} ) } { x_{trgt}?(y)P \; | \; x_{src}!\langle {Q} \rangle \red P\{\quotep{Q}/y}\} }
  \and \\
  \inferrule* [lab=Par] {{P} \red {P}'} {{{P} | {Q}} \red {{P}' | {Q}}}
  \and
  \inferrule* [lab=Equiv]{{{P} \scong {P}'} \andalso {{P}' \red {Q}'} \andalso {{Q}' \scong {Q}}}{{P} \red {Q}}
\end{mathpar}

\begin{eqnarray*}
  match_{\equiv} (\quotep{P},\quotep{Q}) & := & P \equiv Q \\
  match_{\dagger}(\quotep{P},\quotep{Q}) & := & \forall R. P|Q \red^{*} R => R \red^{*} 0 \\
  match_{K}(\quotep{P},\quotep{Q}) & := & K \mbox{ for some context } K
\end{eqnarray*}

$u?(x)P | u!\langle Q \rangle \red P\{\quotep{Q}/x\}$

%We write $\wred$ for $\red^*$, and $P\red$ if $\exists Q $ such that $ P \red Q$.
We write $P\red$ if $\exists Q $ such that $ P \red Q$ and $P\not\red$, otherwise.

\section{Replication}

As mentioned before, it is known that replication (and hence
recursion) can be implemented in a higher-order process algebra
\cite{SangiorgiWalker}. As our first example of calculation with the
machinery thus far presented we give the construction explicitly in
the {\rhoc}.

\begin{eqnarray}
	D_{x} & := & \prefix{x}{y}{(\binpar{\outputp{x}{y}}{@{y}})} \nonumber\\
	\bangp_{x}{P} & := & \binpar{{x}!\langle{\binpar{D_{x}}{P}}\rangle}{D_{x}} \nonumber
\end{eqnarray}

\begin{eqnarray}
	\bangp_{x}{P} & & \nonumber\\
	=
	& {x}!\langle{(\prefix{x}{y}{(\outputp{x}{y} | @{y})) | P}}\rangle 
	      | \prefix{x}{y}{(\outputp{x}{y} | @{y})} & \nonumber\\
	\red
	& (\outputp{x}{y} | @{y})\substn{\quotep{(\prefix{x}{y}{(@{y} | \outputp{x}{y})) | P}}}{y} & \nonumber\\
	=
	& \outputp{x}{\quotep{(\prefix{x}{y}{(\outputp{x}{y} | @{y})) | P}}}
	  | {(\prefix{x}{y}{(\outputp{x}{y} | @{y})) | P}} & \nonumber\\
	\red
	& \ldots & \nonumber\\
	\red^*
	& P | P | \ldots & \nonumber
\end{eqnarray}

Of course, this encoding, as an implementation, runs away, unfolding
$\bangp{P}$ eagerly. A lazier and more implementable replication
operator, restricted to input-guarded processes, may be obtained as follows.

\begin{eqnarray}
\bangp{\prefix{u}{v}{P}} 
	:= 
	\binpar{\lift{x}{\prefix{u}{v}{(\binpar{D(x)}{P})}}}{D(x)} \nonumber
\end{eqnarray}

\begin{remark}
  Note that the lazier definition still does not deal with summation
  or mixed summation (i.e. sums over input and output). The reader is
  invited to construct definitions of replication that deal with these
  features. 

  Further, the definitions are parameterized in a name, $x$. Can you,
  gentle reader, make a definition that eliminates this parameter and
  guarantees no accidental interaction between the replication
  machinery and the process being replicated -- i.e. no accidental
  sharing of names used by the process to get its work done and the
  name(s) used by the replication to effect copying. This latter
  revision of the definition of replication is crucial to obtaining
  the expected identity $!!P \sim !P$.
\end{remark}

\begin{remark}\label{rem:paradoxical_combinator}
  The reader familiar with the lambda calculus will have noticed the
  similarity between $D$ and the paradoxical combinator.

  [Ed. note: the existence of this seems to suggest we have to be more
  restrictive on the set of processes and names we admit if we are to
  support no-cloning.]
\end{remark}

\subsubsection{Bisimulation}

The computational dynamics gives rise to another kind of equivalence,
the equivalence of computational behavior. As previously mentioned
this is typically captured \emph{via} some form of bisimulation.

% The notion we use in this paper is weak barbed bisimulation
% \cite{milner91polyadicpi}.

The notion we use in this paper is derived from weak barbed
bisimulation \cite{milner91polyadicpi}. 

\begin{definition}
An \emph{observation relation}, $\downarrow_{\mathcal N}$, over a set
of names, $\mathcal N$, is the smallest relation satisfying the rules
below.

\infrule[Out-barb]{y \in {\mathcal N}, \; x \nameeq y}
		  {\outputp{x}{v} \downarrow_{\mathcal N} x}
\infrule[Par-barb]{\mbox{$P\downarrow_{\mathcal N} x$ or $Q\downarrow_{\mathcal N} x$}}
		  {\binpar{P}{Q} \downarrow_{\mathcal N} x}

We write $P \Downarrow_{\mathcal N} x$ if there is $Q$ such that 
$P \wred Q$ and $Q \downarrow_{\mathcal N} x$.
\end{definition}

\begin{definition}
%\label{def.bbisim}
An  ${\mathcal N}$-\emph{barbed bisimulation} over a set of names, ${\mathcal N}$, is a symmetric binary relation 
${\mathcal S}_{\mathcal N}$ between agents such that $P\rel{S}_{\mathcal N}Q$ implies:
\begin{enumerate}
\item If $P \red P'$ then $Q \wred Q'$ and $P'\rel{S}_{\mathcal N} Q'$.
\item If $P\downarrow_{\mathcal N} x$, then $Q\Downarrow_{\mathcal N} x$.
\end{enumerate}
$P$ is ${\mathcal N}$-barbed bisimilar to $Q$, written
$P \wbbisim_{\mathcal N} Q$, if $P \rel{S}_{\mathcal N} Q$ for some ${\mathcal N}$-barbed bisimulation ${\mathcal S}_{\mathcal N}$.
\end{definition}

$\mathcal{R} \subseteq \pi \times \pi$

$P \mathcal{R} Q => \forall P'. P \red P' \Rightarrow \exists Q'. Q \red Q', P' \mathcal{R} Q'$

$P \vdash x \Rightarrow Q \vdash x$

\begin{mathpar}
  \inferrule*[lab=Out-barb]{x \nameeq y}{{y}!\langle{Q}\rangle \vdash x}
  \and
  \inferrule*[lab=Par-barb]{\mbox{$P\vdash x$ or $Q\vdash x$}}{\binpar{P}{Q} \vdash x}
\end{mathpar}

\subsubsection{Contexts}

One of the principle advantages of computational calculi like the
$\pi$-calculus is a well-defined notion of context,
contextual-equivalence and a correlation between
contextual-equivalence and notions of bisimulation. The notion of
context allows the decomposition of a process into (sub-)process and
its syntactic environment, its context. Thus, a context may be
thought of as a process with a ``hole'' (written $\Box$) in it. The
application of a context $M$ to a process $P$, written $M[P]$, is
tantamount to filling the hole in $M$ with $P$. In this paper we do
not need the full weight of this theory, but do make use of the notion
of context in the proof the main theorem. 

\begin{mathpar}
  \inferrule* [lab=summation] {} {{M_{M},M_{N}} \bc \Box \;|\; x.M_{A} \;|\; M_{M}+M_{N}}
  \and
  \inferrule* [lab=agent] {} {{M_{A}} \bc (\vec{x})M_{P} \;| \; \clift{P_0,\ldots,M_{P},\ldots,P_N}}
  \and \\
  \inferrule* [lab=process] {} {{M_{P}} \bc M_{N} \;| \;P|M_{P} }
\end{mathpar} 

\begin{mathpar}
  \inferrule* [lab=sychronization] {} {M_{N} \bc \Box \;|\; x?M_{F} \;|\; x!M_{C}}
  \and
  \inferrule* [lab=abstraction] {} {{M_{F}} \bc (x)M_{P} }
  \and
  \inferrule* [lab=concretion] {} {{M_{C}} \bc \langle M_{P} \rangle }
  \and \\
  \inferrule* [lab=process] {} {{M_{P}} \bc M_{N} \;| \;P|M_{P} }
\end{mathpar}

\begin{definition}[contextual application] Given a context $M$, and
  process $P$, we define the \emph{contextual application}, $M[P] :=
  M\{P/\Box\}$. That is, the contextual application of M to P is the
  substitution of $P$ for $\Box$ in $M$.
\end{definition}

$\meaningof{-} : L \to \mathcal{P}(\pi)$

\begin{mathpar}
  \inferrule* [lab=collection] {} {\meaningof{true} = \pi, \and \meaningof{~E} = \pi \setminus \meaningof{E}, \and \meaningof{E_{1} \& E_{2}} = \meaningof{E_{1}} \cap \meaningof{E_{2}}}
\end{mathpar}

\begin{mathpar}
  \inferrule* [lab=structure] {} {\meaningof{0} = \{ P \in \pi | P \equiv 0 \}, \and \\ \meaningof{E_1 | E_2} = \{ P \in \pi | P \equiv P_{1} | P_{2}, P_{1} \in \meaningof{E_{1}}, P_{2} \in \meaningof{E_2}\} }
\end{mathpar}

\begin{mathpar}
 \inferrule* [lab=behavior] {} {\meaningof{\langle a?b \rangle E} = \{ P \in \pi | P \equiv Q | u?(y)P', \\ \and \\\\ \and \\ \;\;\; u \in \meaningof{a}, \forall z.P'\{z/y\} \in \meaningof{E\{z/b\}}\}, \and \\ \meaningof{a!E} = \{ P \in \pi | P \equiv Q | x!\langle P' \rangle, x \in \meaningof{a} P' \in \meaningof{E}\} }
\end{mathpar}

\begin{mathpar}
 \inferrule* [lab=nominal] {} {\meaningof{\quotep{E}} = \{ \quotep{P} \in \quotep{\pi} | P \in \meaningof{E} \}, \and \meaningof{\quotep{P}} = \{ \quotep{Q} \in \quotep{\pi} | P \equiv Q \} \and \\ \meaningof{@\quotep{E}} = \{ P \in \pi | P \equiv @x, x \in \meaningof{E} \}}
\end{mathpar}

\begin{eqnarray*}
  \\
  \meaningof{-} : TS \to ST
\end{eqnarray*}

\begin{eqnarray*}
  \\
  L : TS \to ST
\end{eqnarray*}

\begin{eqnarray*}
  \\
  P \models E \iff P \in \meaningof{E}
\end{eqnarray*}

\begin{eqnarray*}
  P \approx_{L} Q \iff \forall E \in L. P \models E \iff Q \models E
\end{eqnarray*}

\begin{eqnarray*}
  P \approx_{K} Q
\end{eqnarray*}

\begin{eqnarray*}
  P \approx Q
\end{eqnarray*}

$\approx_{K} = \approx = \approx_{L}$

\subsubsection{Contextual duality}

Note that contexts extend the quotation operation to a family of
operations from processes to names. Given a context, $M$, we can
define a \emph{nominal context}, $\quotep{M}$ by $\quotep{M}[P] :=
\quotep{M[P]}$. To foreshadow what is to come we observe that these
operations enjoy a duality with processes very much like the duality
between vectors and maps from vectors to scalars.

Further, because the calculus is essentially higher-order, we have a
correspondence between contexts and processes. More specifically,
given a name $x$ and a context $M$ we can construct $M^{*}_{x}$ such
that 

\begin{mathpar}
  M^{*}_{x} | \lift{x}{P} \red M[P]
\end{mathpar}

namely,

\begin{mathpar}
  M^{*}_{x} := x?(u).M[\dropn{u}]
\end{mathpar}

The dependence of $M^{*}_{x}$ on a name makes it an abstraction, 

\begin{mathpar}
  M^{*} := (x)x?(u).M[\dropn{u}]
\end{mathpar}

\subsection{Additional notation}

It will sometimes be convenient to denote the process a name
quotes. We already have the notation $x = \quotep{P}$, but it will be
convenient to introduce an alternate notation, $\procn{x}$, when we
want to emphasize the connection to the use of the name. Note that, by
virtue of name equivalence, $\quotep{\procn{x}} \nameeq x$; so, the
notation is consistent with previous definitions.

Further, because names have structure it is possible to effect
substitutions on the basis of that structure. This means we need to
upgrade our notation for substitutions, which we accomplish by
adapting comprehension notation. Thus,

\begin{mathpar}
  P\{ y / x : x \in S \}
\end{mathpar}

is interpreted to mean the process derived from P by replacing (in a
capture-avoiding manner) each occurrence of $x$ in $S$ by $y$. For example,

\begin{mathpar}
  P\{ \quotep{\procn{x}|\procn{x}} / x : x \in \freenames{P} \}
\end{mathpar}

will replace each (occurrence) of a free name $x$ in $P$ by
$\quotep{\procn{x}|\procn{x}}$.

Also, we will avail ourselves of the notation $x^{L}$ and $x^{R}$ to
denote injections of a name into disjoint copies of the name
space. There are numerous ways to accomplish this. One example can be
found in \cite{MeredithR05}. This notation overloads to vectors of
names: $\vec{x}^{\pi} := (x_{i}^{\pi} \; : \; 0 \leq i < |\vec{x}| )$ where $\pi \in \{L,R\}$.

We also use $P^{\Box} := P|\Box$.

In \cite{MeredithR05} an interpretation of the new operator is
given. It turns out that there are several possible interpretations
all enjoying the requisite algebraic properties of the operator (see
\cite{milner91polyadicpi}). We will therefore make liberal use of
$(\nu\; \vec{x})P$.

% subsection the_syntax_and_semantics_of_the_notation_system (end)   

\input{qm2pi.qmops} 

\input{qm2pi.sterngerlach} 

\input{qm2pi.metric} 

% section concurrent_process_calculi (end)

%\input{qm2pi.proofsketch}

% section proof sketch (end)

%\input{qm2pi.slviaknots} 

% section spatial logic via knots (end)

\input{qm2pi.conclusion}

% section conclusion (end)

%\input{qm2pi.dtcodes} 

% section wiring algorithm (end)

\input{qm2pi.ack} 

% section acknowledgments (end)

\newpage


\bibliographystyle{plain}   
\bibliography{../../biblios/main.bib}

\input{qm2pi.rhodetails}

\end{document}



% section front matter (end)

\section{Introduction}\label{sec:introduction} % (fold)
In this draft of the material i am going to have to dispense with the
usual writing conventions adopted in papers on these topics. i'm going
to have adopt whatever tone i need at the time i'm writing up the
calculations. Sometimes this may be very conversational; others it may
be the barest mathematical grunts; others still it may be that i have
lifted text from one of my other papers because the exposition of some
point was better said there. i hope that my readers are not unduly put
out by this decision. i'm not doing this to flout convention or be
rebellious. i find these calculations very technically challenging. To
keep everything going technically, something has to give; i have to
let go of some cognitive burden. So, the academic writing style --
with all of its trade-offs in terms of facilitating technical
communication -- is what i'm letting go of. Perhaps subsequent drafts
can be tightened and polished, but for now, i'm going to speak as if
we were sitting together in a coffee shop with a laptop, wifi and a
pad of paper and a pencil.

So, here's what i have to say. We -- you and i, comfortably ensconced
in our coffee shop and well-equipped with our tools -- can realize and
carry out the calculations of quantum mechanics over a very different
formal theory of dynamics, a formal theory of dynamics that
corresponds to a theory of concurrent computation with
\emph{reflection}. It has the advantage that the underlying theory is
already `quantized', but supports analogues all of the continuuous
operations. Strikingly, this underlying theory has recently been
connected with a notion of metric that we can show, by calculating
together, coincides with the metric induced by the inner product.

There are a lot of reasons why you might be interested in seeing
calculations of this form. Here's why i'm interested. For the past
several centuries there has been no competitor to the ``Newtonian''
account of dynamics. As a result the predominant share of accounts of
dynamical systems and situations have had to be formulated in terms of
the Newtonian machinery. i view this as an intellectually dangerous
position to occupy. Everything, despite it's intrinsic shape, turns
into a nail to be hit with this hammer. Recently, however, the theory
of computation has matured to the point where we have candidates for
theories of dynamics that offer very different perspective on
reasoning about dynamical systems and situations. Testing these
candidates against very successful accounts of dynamical situations,
like quantum mechanics, is going to give us some sense of how mature
they are and some measure of the quality of these accounts of
dynamics.

\subsection{Summary of contributions and outline of paper}

So, we're going to develop an interpretation of the operations of
quantum mechanics normally interpreted by Hilbert spaces and
operators. We're going to do this over a theory of computation. Note
that this is very different than the usual quantum computation program
which develops notions of computation over quantum mechanics. Rather,
we are developing a story that aligns with Wheeler's slogan: It from
Bit. To do this we will first provide an account of the theory of
computation at play here. Then we will dive into a calculation-driven
interpretation of the operations of quantum mechanics.

The reason we take this approach is that -- until very recently --
there hasn't been an axiomatic account of quantum mechanics. As a
result there has been no sharp delineation of the mathematical theory
supporting interpretation of the physical theory and the physical
theory, itself. So, ambient features of the maths are free to be
exploited (or supressed) without a real accounting of their physical
relevance. There is no sharp statement ``here's the physical theory''
qua \emph{theory} and ``here's the mathematical interpretation''
enabling a judgment of how faithful the interpretation is -- apart
from experimental observation. When there is an axiomatic account we
can judge how well a given mathematical formalism supports an
interpretation of the axioms, independent of
experimentation. Likewise, we can judge how well we have captured our
physical evidence and experience with our axiomatics, independent of
any specific mathematical implementation, with accidental detail that
may or may not have physical significance. 

In lieu of a fully fleshed out and vetted axiomatic account of quantum
mechanics, interpreting the operational notions in service of modeling
physical systems will have to suffice. In other words, we are not in
the business of providing a model of Hilbert spaces and operators. We
are in the business of providing a model of quantum mechanics because
we are motivated by testing our notions of dynamics against physical
theory; and, the predictive calculations of the physical theory must
serve as the best formulation -- shy of a fully fleshed out axiomatic
account -- of the physical theory itself (as they have for scientific
theories since time immemorial). Put another way, despite a
whole-hearted commitment to an It-from-Bit ontology, we are firmly
aligned with the shut-up-and-calculate camp as the best way to obtain
results either from the physical perspective or as a quality assurance
measure of our fledgling theory of dynamics.

In detail, we present a reflective process calculus. Then we develop
intuitive correspondences between the notions available in this
calculus and the usual physical notions supporting quantum mechanical
calculations. Thus, 

\begin{table}[htp]
  \center{
    \fbox{
      \begin{tabular}{c|c}
        quantum mechanics & process calculus \\
        \hline
        scalar & name \\
        state vector & process \\
        dual & contextual duals \\
        matrix & formal sums of process-context-dual pairs \\
        orthogonality & process annihilation \\
        inner product & execution-formula + quoting
      \end{tabular}
    }
  }
  \caption{QM - process calculi correspondences}
\end{table}

Then we tighten up these intuitions to operational definitions. We
employ the Dirac notation as the best proxy we can find for an
abstract syntax of the quantum mechanical notions. The definitions we
develop put us in contact with equational constraints coming from the
theory that we demonstrate the definitions and calculations satisfy.

This puts us in a position to shut up and calculate for the
Stern-Gerlach experimental set up, showing how these predictive
calculations become calculations on processes in our theory of a
reflective process calculus.

Penultimately, we demonstrate that the notion of metric coming from
the inner product coincides with the notion of metric available from
the theory of bisimulation. This demonstration gives us the right to
think of space as arising from behavior. Finally, we consider where we
might go from the new vantage point we have obtained.

% section introduction (end) 
 
% section introduction (end)

% \documentclass[12pt]{llncs}
%\documentclass{jktr}

\usepackage[pdftex]{hyperref}                   
\usepackage {listings}
\usepackage {mathpartir}
\usepackage{bcprules}
%\usepackage{listings}
                       
\usepackage{graphicx} 
%\usepackage[margins=2.5cm,nohead,nofoot]{geometry}
%\usepackage{geometry}
\usepackage{amsfonts}
\usepackage{amstext}
\usepackage{latexsym}
\usepackage{amssymb}
\usepackage{color}


%\include{myPreamble}
\include{qm2pi.local} 

%\ifpdf
%\usepackage[pdftex]{graphicx}
%\else
%\usepackage{graphicx}
%\fi

 % \ifpdf
%  \usepackage{pdfsync}
%  \if


%\title{Brief Article}
%\author{David F. Snyder}
%\author{L.G. Meredith}

%\address{Dept. of Math., Texas State University--San Marcos, San Marcos, TX 78666}
       
\pagestyle{empty}


\begin{document}

\lstset{language=[Objective]Caml,frame=shadowbox}

\input{qm2pi.front}

% section front matter (end)

\input{qm2pi.intro} 
 
% section introduction (end)

% \input{qm2pi.knotations} 

% section notation (end)

\input{qm2pi.process.calculi} 

% section concurrent_process_calculi_and_spatial_logics_ (end)
    
%\input{qm2pi.knots2pi} 

%\input{qm2pi.trefoil} 

%\input{qm2pi.mainthm} 

% subsection basic_interpretation (end)

%\input{qm2pi.rho.presentation} 
\subsection{The syntax and semantics of the notation system}\label{sub:the_syntax_and_semantics_of_the_notation_system} % (fold)

We now summarize a technical presentation of the calculus that
embodies our theory of dynamics. The typical presentation of such a
calculus follows the style of giving generators and relations on
them. The grammar, below, describing term constructors, freely
generates the set of processes, $\Proc$. This set is then quotiented
by a relation known as structural congruence and it is over this set
that the notion of dynamics is expressed. This presentation is
essentially that of \cite{MeredithR05} with the addition of
polyadicity and summation. For readability we have relegated some of
the technical subtleties to an appendix.

\subsubsection{Process grammar}\label{subsub:process_grammar}

\begin{mathpar}
  \inferrule* [lab=synchronization] {} {{M} \bc \pzero \;|\; x?F \;|\; x!C }
  \and
  \inferrule* [lab=abstraction] {} {{F} \bc (x)P}
  \and
  \inferrule* [lab=concretion] {} {{C} \bc \langle Q \rangle}
  \and
  \inferrule* [lab=process] {} {{P,Q} \bc M \;| \;P|Q \;|\; @{x}}
  \and
  \inferrule* [lab=name] {} {{x} \bc \quotep{P}}
\end{mathpar} 

Note that $\vec{x}$ (resp. $\vec{P}$) denotes a vector of names
(resp. processes) of length $|\vec{x}|$ (resp. $|\vec{P}|$). We adopt
the following useful abbreviations.

\begin{mathpar}
   x?(\vec{y}).P := x.(\vec{y})P \and  x\clift{\vec{P}} := x.\clift{\vec{P}}
   \and x!(y) := \lift{x}{\dropn{y}}
   \and \Pi_{i=0}^{n-1}P_i := P_0 | \ldots | P_{n-1}
\end{mathpar}

\subsubsection{Structural congruence}

\paragraph{Free and bound names and alpha-equivalence.} At the
core of structural equivalence is alpha-equivalence which identifies
process that are the same up to a change of variable. Formally, we
recognize the distinction between free and bound names. The free names
of a process, $\freenames{P}$, may be calculated recursively as
follows:

\begin{mathpar}
\freenames{\pzero} := \emptyset
  \and \\
  \freenames{x?(y).P} := \{ x \} \cup (\freenames{P} \setminus \{ y \})
  \and 
  \freenames{x!\langle P \rangle} := \{ x \} \cup \{ P \} 
  \and \\
  \freenames{P|Q} := \freenames{P} \cup \freenames{Q}
  \and \\
  \freenames{@{x}} := \{ x \}
\end{mathpar}

$\pi$
$\quotep{\pi}$

$\freenames{-} : \pi \to \mathcal{P}(\quotep{\pi})$

\begin{eqnarray*}
  \freenames{\pzero} & := & \emptyset \\
  \freenames{x?(y).P} & := & \{ x \} \cup (\freenames{P} \setminus \{ y \}) \\
  \freenames{x!\langle P \rangle} & := & \{ x \} \cup \{ P \} \\
  \freenames{P|Q} & := & \freenames{P} \cup \freenames{Q} \\
  \freenames{\dropn{x}} & := & \{ x \}
\end{eqnarray*}

The bound names of a process, $\boundnames{P}$, are those names occurring in $P$
that are not free. For example, in $x?(y).0$, the name $x$ is free, while $y$ is bound.

\begin{mathpar}
  \inferrule* [lab=monoidal-laws] {} { P|Q \equiv Q|P \and P|0 \equiv P \and P|(Q|R) \equiv (P|Q)|R }
\end{mathpar}

\begin{mathpar}
  \inferrule* [lab=alpha-equivalence] {} { (x)P \equiv (y)P\{y/x\} \and y \not\in \freenames{P} }
\end{mathpar}

\begin{definition}
Then two processes, $P,Q$, are alpha-equivalent if $P = Q\{\vec{y}/\vec{x}\}$ for
some $\vec{x} \in \boundnames{Q},\vec{y} \in \boundnames{P}$, where $Q\{\vec{y}/\vec{x}\}$
denotes the capture-avoiding substitution of $\vec{y}$ for $\vec{x}$ in $Q$.
\end{definition}

\begin{definition}
  The {\em structural congruence} \cite{SangiorgiWalker} , $\equiv$,
  between processes is the least congruence containing
  alpha-equivalence, satisfying the abelian monoid laws
  (associativity, commutativity and $\pzero$ as identity) for parallel
  composition $|$ and for summation $+$.
\end{definition}

\subsection{Name equivalence}

We take name equivalence, written $\nameeq$, to be the smallest
equivalence relation generated by the following rules.

\begin{mathpar}
\inferrule*[lab=Quote-drop]
{ }
{ \quotep{@{x}} \nameeq x }

\inferrule*[lab=Struct-equiv]
{ P \scong Q }
{ \quotep{P} \nameeq \quotep{Q} }
\end{mathpar}

The astute reader will have noticed that the mutual recursion of names
and processes imposes a mutual recursion on alpha-equivalence and
structural equivalence via name-equivalence. Fortunately, all of this
works out pleasantly and we may calculate in the natural way, free of
concern. The reader interested in the details is referred to the
appendix \ref{appendix:rho_details}.

\subsection{Substitution}

We use $\Proc$ for the set of processes, $\QProc$ for the set of
names, and $\id{\{}\vec{y} / \vec{x} \id{\}}$ to denote partial maps,
$s : \QProc \rightarrow \QProc$. A map, $s$ lifts, uniquely, to a map
on process terms, $\widehat{s} : \Proc \rightarrow \Proc$ by the
following equations.

\begin{mathpar}
  (0) \psubstp{Q}{P} := 0 \\
  (R \juxtap S) \psubstp{Q}{P}
  :=    
  (R)\psubstp{Q}{P} \juxtap (S) \psubstp{Q}{P} \\
  (x?(y).R) \psubstp{Q}{P}    
  :=    
  (x)\substp{Q}{P} (z)\concat( (R \psubstn{z}{y}) \psubstp{Q}{P} ) \\
  (\lift{x}{R}) \psubstp{Q}{P}  
  :=
  \lift{(x)\substp{Q}{P}}{ R \psubstp{Q}{P} } \\
%   (\dropn{x})  \psubstp{Q}{P}       
%   := 
%   \left\{ 
%     \begin{array}{ccc} 
%       \dropn{\quotep{Q}} & & x \nameeq \quotep{P} \\
%       \dropn{x} & & otherwise \\
%     \end{array}
%   \right. 
  (\dropn{x})  \psubstp{Q}{P}       
  := 
  \left\{ 
    \begin{array}{ccc} 
      Q & & x \nameeq \quotep{P} \\
      \dropn{x} & & otherwise \\
    \end{array}
  \right.
\end{mathpar}
 

where

\begin{eqnarray}
  (x)\id{\{} \lpquote Q \rpquote / \lpquote P \rpquote \id{\}}            = 
  \left\{ 
    \begin{array}{ccc}
      \lpquote Q \rpquote & & x \nameeq \lpquote P \rpquote \\
      x & & otherwise \\
    \end{array}
  \right. \nonumber
\end{eqnarray}

and $z$ is chosen distinct from $\quotep{P}$, $\quotep{Q}$, the free
names in $Q$, and all the names in $R$. Our $\alpha$-equivalence will
be built in the standard way from this substitution.

\begin{remark}\label{rem:no_self_referential_names}
  One consequence of these definitions is that $\forall P. \quotep{P}
  \not\in \freenames{P}$.
\end{remark}

\subsection{ Dynamic quote: an example }

Anticipating something of what's to come, consider applying the
substitution, $\widehat{\id{\{}u / z \id{\}}}$, to the following pair
of processes, $\lift{w}{y!(z)}$ and $w[ \lpquote y!(z) \rpquote ]$.

\begin{eqnarray}
	\lift{w}{y!(z)}\widehat{\id{\{}u / z \id{\}}}
		& = &
		\lift{w}{y!(u)} \nonumber\\
	w[ \lpquote y!(z) \rpquote ] \widehat{ \id{\{}u / z \id{\}} }
		& = &
		w[ \lpquote y!(z) \rpquote ] \nonumber
\end{eqnarray}

Because the body of the process between quotes is impervious to
substitution, we get radically different answers. In fact, by
examining the first process in an input context,
e.g. $x?(z).\lift{w}{y!(z)}$, we see that the process under the lift
operator may be shaped by prefixed inputs binding a name inside it. In
this sense, the lift operator will be seen as a way to dynamically
construct processes before reifying them as names.

Finally equipped with these standard features we can present the
dynamics of the calculus.

\subsubsection{Operational semantics} 

Finally, we introduce the computational dynamics. What marks these
algebras as distinct from other more traditionally studied algebraic
structures, e.g. vector spaces or polynomial rings, is the manner in
which dynamics is captured. In traditional structures, dynamics is typically
expressed through morphisms between such structures, as in linear maps
between vector spaces or morphisms between rings. In algebras
associated with the semantics of computation, the dynamics is
expressed as part of the algebraic structure itself, through a
reduction reduction relation typically denoted by $\red$. Below, we
give a recursive presentation of this relation for the calculus used
in the encoding.

$\red \subseteq \pi \times \pi$
$\red : \pi \to \mathcal{P}(\pi)$

\begin{mathpar}
  \inferrule* [lab=Comm] { \textsf{match}( x_{src}, x_{trgt} ) } { x_{trgt}?(y)P \; | \; x_{src}!\langle {Q} \rangle \red P\{\quotep{Q}/y}\} }
  \and \\
  \inferrule* [lab=Par] {{P} \red {P}'} {{{P} | {Q}} \red {{P}' | {Q}}}
  \and
  \inferrule* [lab=Equiv]{{{P} \scong {P}'} \andalso {{P}' \red {Q}'} \andalso {{Q}' \scong {Q}}}{{P} \red {Q}}
\end{mathpar}

\begin{eqnarray*}
  match_{\equiv} (\quotep{P},\quotep{Q}) & := & P \equiv Q \\
  match_{\dagger}(\quotep{P},\quotep{Q}) & := & \forall R. P|Q \red^{*} R => R \red^{*} 0 \\
  match_{K}(\quotep{P},\quotep{Q}) & := & K \mbox{ for some context } K
\end{eqnarray*}

$u?(x)P | u!\langle Q \rangle \red P\{\quotep{Q}/x\}$

%We write $\wred$ for $\red^*$, and $P\red$ if $\exists Q $ such that $ P \red Q$.
We write $P\red$ if $\exists Q $ such that $ P \red Q$ and $P\not\red$, otherwise.

\section{Replication}

As mentioned before, it is known that replication (and hence
recursion) can be implemented in a higher-order process algebra
\cite{SangiorgiWalker}. As our first example of calculation with the
machinery thus far presented we give the construction explicitly in
the {\rhoc}.

\begin{eqnarray}
	D_{x} & := & \prefix{x}{y}{(\binpar{\outputp{x}{y}}{@{y}})} \nonumber\\
	\bangp_{x}{P} & := & \binpar{{x}!\langle{\binpar{D_{x}}{P}}\rangle}{D_{x}} \nonumber
\end{eqnarray}

\begin{eqnarray}
	\bangp_{x}{P} & & \nonumber\\
	=
	& {x}!\langle{(\prefix{x}{y}{(\outputp{x}{y} | @{y})) | P}}\rangle 
	      | \prefix{x}{y}{(\outputp{x}{y} | @{y})} & \nonumber\\
	\red
	& (\outputp{x}{y} | @{y})\substn{\quotep{(\prefix{x}{y}{(@{y} | \outputp{x}{y})) | P}}}{y} & \nonumber\\
	=
	& \outputp{x}{\quotep{(\prefix{x}{y}{(\outputp{x}{y} | @{y})) | P}}}
	  | {(\prefix{x}{y}{(\outputp{x}{y} | @{y})) | P}} & \nonumber\\
	\red
	& \ldots & \nonumber\\
	\red^*
	& P | P | \ldots & \nonumber
\end{eqnarray}

Of course, this encoding, as an implementation, runs away, unfolding
$\bangp{P}$ eagerly. A lazier and more implementable replication
operator, restricted to input-guarded processes, may be obtained as follows.

\begin{eqnarray}
\bangp{\prefix{u}{v}{P}} 
	:= 
	\binpar{\lift{x}{\prefix{u}{v}{(\binpar{D(x)}{P})}}}{D(x)} \nonumber
\end{eqnarray}

\begin{remark}
  Note that the lazier definition still does not deal with summation
  or mixed summation (i.e. sums over input and output). The reader is
  invited to construct definitions of replication that deal with these
  features. 

  Further, the definitions are parameterized in a name, $x$. Can you,
  gentle reader, make a definition that eliminates this parameter and
  guarantees no accidental interaction between the replication
  machinery and the process being replicated -- i.e. no accidental
  sharing of names used by the process to get its work done and the
  name(s) used by the replication to effect copying. This latter
  revision of the definition of replication is crucial to obtaining
  the expected identity $!!P \sim !P$.
\end{remark}

\begin{remark}\label{rem:paradoxical_combinator}
  The reader familiar with the lambda calculus will have noticed the
  similarity between $D$ and the paradoxical combinator.

  [Ed. note: the existence of this seems to suggest we have to be more
  restrictive on the set of processes and names we admit if we are to
  support no-cloning.]
\end{remark}

\subsubsection{Bisimulation}

The computational dynamics gives rise to another kind of equivalence,
the equivalence of computational behavior. As previously mentioned
this is typically captured \emph{via} some form of bisimulation.

% The notion we use in this paper is weak barbed bisimulation
% \cite{milner91polyadicpi}.

The notion we use in this paper is derived from weak barbed
bisimulation \cite{milner91polyadicpi}. 

\begin{definition}
An \emph{observation relation}, $\downarrow_{\mathcal N}$, over a set
of names, $\mathcal N$, is the smallest relation satisfying the rules
below.

\infrule[Out-barb]{y \in {\mathcal N}, \; x \nameeq y}
		  {\outputp{x}{v} \downarrow_{\mathcal N} x}
\infrule[Par-barb]{\mbox{$P\downarrow_{\mathcal N} x$ or $Q\downarrow_{\mathcal N} x$}}
		  {\binpar{P}{Q} \downarrow_{\mathcal N} x}

We write $P \Downarrow_{\mathcal N} x$ if there is $Q$ such that 
$P \wred Q$ and $Q \downarrow_{\mathcal N} x$.
\end{definition}

\begin{definition}
%\label{def.bbisim}
An  ${\mathcal N}$-\emph{barbed bisimulation} over a set of names, ${\mathcal N}$, is a symmetric binary relation 
${\mathcal S}_{\mathcal N}$ between agents such that $P\rel{S}_{\mathcal N}Q$ implies:
\begin{enumerate}
\item If $P \red P'$ then $Q \wred Q'$ and $P'\rel{S}_{\mathcal N} Q'$.
\item If $P\downarrow_{\mathcal N} x$, then $Q\Downarrow_{\mathcal N} x$.
\end{enumerate}
$P$ is ${\mathcal N}$-barbed bisimilar to $Q$, written
$P \wbbisim_{\mathcal N} Q$, if $P \rel{S}_{\mathcal N} Q$ for some ${\mathcal N}$-barbed bisimulation ${\mathcal S}_{\mathcal N}$.
\end{definition}

$\mathcal{R} \subseteq \pi \times \pi$

$P \mathcal{R} Q => \forall P'. P \red P' \Rightarrow \exists Q'. Q \red Q', P' \mathcal{R} Q'$

$P \vdash x \Rightarrow Q \vdash x$

\begin{mathpar}
  \inferrule*[lab=Out-barb]{x \nameeq y}{{y}!\langle{Q}\rangle \vdash x}
  \and
  \inferrule*[lab=Par-barb]{\mbox{$P\vdash x$ or $Q\vdash x$}}{\binpar{P}{Q} \vdash x}
\end{mathpar}

\subsubsection{Contexts}

One of the principle advantages of computational calculi like the
$\pi$-calculus is a well-defined notion of context,
contextual-equivalence and a correlation between
contextual-equivalence and notions of bisimulation. The notion of
context allows the decomposition of a process into (sub-)process and
its syntactic environment, its context. Thus, a context may be
thought of as a process with a ``hole'' (written $\Box$) in it. The
application of a context $M$ to a process $P$, written $M[P]$, is
tantamount to filling the hole in $M$ with $P$. In this paper we do
not need the full weight of this theory, but do make use of the notion
of context in the proof the main theorem. 

\begin{mathpar}
  \inferrule* [lab=summation] {} {{M_{M},M_{N}} \bc \Box \;|\; x.M_{A} \;|\; M_{M}+M_{N}}
  \and
  \inferrule* [lab=agent] {} {{M_{A}} \bc (\vec{x})M_{P} \;| \; \clift{P_0,\ldots,M_{P},\ldots,P_N}}
  \and \\
  \inferrule* [lab=process] {} {{M_{P}} \bc M_{N} \;| \;P|M_{P} }
\end{mathpar} 

\begin{mathpar}
  \inferrule* [lab=sychronization] {} {M_{N} \bc \Box \;|\; x?M_{F} \;|\; x!M_{C}}
  \and
  \inferrule* [lab=abstraction] {} {{M_{F}} \bc (x)M_{P} }
  \and
  \inferrule* [lab=concretion] {} {{M_{C}} \bc \langle M_{P} \rangle }
  \and \\
  \inferrule* [lab=process] {} {{M_{P}} \bc M_{N} \;| \;P|M_{P} }
\end{mathpar}

\begin{definition}[contextual application] Given a context $M$, and
  process $P$, we define the \emph{contextual application}, $M[P] :=
  M\{P/\Box\}$. That is, the contextual application of M to P is the
  substitution of $P$ for $\Box$ in $M$.
\end{definition}

$\meaningof{-} : L \to \mathcal{P}(\pi)$

\begin{mathpar}
  \inferrule* [lab=collection] {} {\meaningof{true} = \pi, \and \meaningof{~E} = \pi \setminus \meaningof{E}, \and \meaningof{E_{1} \& E_{2}} = \meaningof{E_{1}} \cap \meaningof{E_{2}}}
\end{mathpar}

\begin{mathpar}
  \inferrule* [lab=structure] {} {\meaningof{0} = \{ P \in \pi | P \equiv 0 \}, \and \\ \meaningof{E_1 | E_2} = \{ P \in \pi | P \equiv P_{1} | P_{2}, P_{1} \in \meaningof{E_{1}}, P_{2} \in \meaningof{E_2}\} }
\end{mathpar}

\begin{mathpar}
 \inferrule* [lab=behavior] {} {\meaningof{\langle a?b \rangle E} = \{ P \in \pi | P \equiv Q | u?(y)P', \\ \and \\\\ \and \\ \;\;\; u \in \meaningof{a}, \forall z.P'\{z/y\} \in \meaningof{E\{z/b\}}\}, \and \\ \meaningof{a!E} = \{ P \in \pi | P \equiv Q | x!\langle P' \rangle, x \in \meaningof{a} P' \in \meaningof{E}\} }
\end{mathpar}

\begin{mathpar}
 \inferrule* [lab=nominal] {} {\meaningof{\quotep{E}} = \{ \quotep{P} \in \quotep{\pi} | P \in \meaningof{E} \}, \and \meaningof{\quotep{P}} = \{ \quotep{Q} \in \quotep{\pi} | P \equiv Q \} \and \\ \meaningof{@\quotep{E}} = \{ P \in \pi | P \equiv @x, x \in \meaningof{E} \}}
\end{mathpar}

\begin{eqnarray*}
  \\
  \meaningof{-} : TS \to ST
\end{eqnarray*}

\begin{eqnarray*}
  \\
  L : TS \to ST
\end{eqnarray*}

\begin{eqnarray*}
  \\
  P \models E \iff P \in \meaningof{E}
\end{eqnarray*}

\begin{eqnarray*}
  P \approx_{L} Q \iff \forall E \in L. P \models E \iff Q \models E
\end{eqnarray*}

\begin{eqnarray*}
  P \approx_{K} Q
\end{eqnarray*}

\begin{eqnarray*}
  P \approx Q
\end{eqnarray*}

$\approx_{K} = \approx = \approx_{L}$

\subsubsection{Contextual duality}

Note that contexts extend the quotation operation to a family of
operations from processes to names. Given a context, $M$, we can
define a \emph{nominal context}, $\quotep{M}$ by $\quotep{M}[P] :=
\quotep{M[P]}$. To foreshadow what is to come we observe that these
operations enjoy a duality with processes very much like the duality
between vectors and maps from vectors to scalars.

Further, because the calculus is essentially higher-order, we have a
correspondence between contexts and processes. More specifically,
given a name $x$ and a context $M$ we can construct $M^{*}_{x}$ such
that 

\begin{mathpar}
  M^{*}_{x} | \lift{x}{P} \red M[P]
\end{mathpar}

namely,

\begin{mathpar}
  M^{*}_{x} := x?(u).M[\dropn{u}]
\end{mathpar}

The dependence of $M^{*}_{x}$ on a name makes it an abstraction, 

\begin{mathpar}
  M^{*} := (x)x?(u).M[\dropn{u}]
\end{mathpar}

\subsection{Additional notation}

It will sometimes be convenient to denote the process a name
quotes. We already have the notation $x = \quotep{P}$, but it will be
convenient to introduce an alternate notation, $\procn{x}$, when we
want to emphasize the connection to the use of the name. Note that, by
virtue of name equivalence, $\quotep{\procn{x}} \nameeq x$; so, the
notation is consistent with previous definitions.

Further, because names have structure it is possible to effect
substitutions on the basis of that structure. This means we need to
upgrade our notation for substitutions, which we accomplish by
adapting comprehension notation. Thus,

\begin{mathpar}
  P\{ y / x : x \in S \}
\end{mathpar}

is interpreted to mean the process derived from P by replacing (in a
capture-avoiding manner) each occurrence of $x$ in $S$ by $y$. For example,

\begin{mathpar}
  P\{ \quotep{\procn{x}|\procn{x}} / x : x \in \freenames{P} \}
\end{mathpar}

will replace each (occurrence) of a free name $x$ in $P$ by
$\quotep{\procn{x}|\procn{x}}$.

Also, we will avail ourselves of the notation $x^{L}$ and $x^{R}$ to
denote injections of a name into disjoint copies of the name
space. There are numerous ways to accomplish this. One example can be
found in \cite{MeredithR05}. This notation overloads to vectors of
names: $\vec{x}^{\pi} := (x_{i}^{\pi} \; : \; 0 \leq i < |\vec{x}| )$ where $\pi \in \{L,R\}$.

We also use $P^{\Box} := P|\Box$.

In \cite{MeredithR05} an interpretation of the new operator is
given. It turns out that there are several possible interpretations
all enjoying the requisite algebraic properties of the operator (see
\cite{milner91polyadicpi}). We will therefore make liberal use of
$(\nu\; \vec{x})P$.

% subsection the_syntax_and_semantics_of_the_notation_system (end)   

\input{qm2pi.qmops} 

\input{qm2pi.sterngerlach} 

\input{qm2pi.metric} 

% section concurrent_process_calculi (end)

%\input{qm2pi.proofsketch}

% section proof sketch (end)

%\input{qm2pi.slviaknots} 

% section spatial logic via knots (end)

\input{qm2pi.conclusion}

% section conclusion (end)

%\input{qm2pi.dtcodes} 

% section wiring algorithm (end)

\input{qm2pi.ack} 

% section acknowledgments (end)

\newpage


\bibliographystyle{plain}   
\bibliography{../../biblios/main.bib}

\input{qm2pi.rhodetails}

\end{document}

 

% section notation (end)

\input{qm2pi.process.calculi} 

% section concurrent_process_calculi_and_spatial_logics_ (end)
    
%\documentclass[12pt]{llncs}
%\documentclass{jktr}

\usepackage[pdftex]{hyperref}                   
\usepackage {listings}
\usepackage {mathpartir}
\usepackage{bcprules}
%\usepackage{listings}
                       
\usepackage{graphicx} 
%\usepackage[margins=2.5cm,nohead,nofoot]{geometry}
%\usepackage{geometry}
\usepackage{amsfonts}
\usepackage{amstext}
\usepackage{latexsym}
\usepackage{amssymb}
\usepackage{color}


%\include{myPreamble}
\include{qm2pi.local} 

%\ifpdf
%\usepackage[pdftex]{graphicx}
%\else
%\usepackage{graphicx}
%\fi

 % \ifpdf
%  \usepackage{pdfsync}
%  \if


%\title{Brief Article}
%\author{David F. Snyder}
%\author{L.G. Meredith}

%\address{Dept. of Math., Texas State University--San Marcos, San Marcos, TX 78666}
       
\pagestyle{empty}


\begin{document}

\lstset{language=[Objective]Caml,frame=shadowbox}

\input{qm2pi.front}

% section front matter (end)

\input{qm2pi.intro} 
 
% section introduction (end)

% \input{qm2pi.knotations} 

% section notation (end)

\input{qm2pi.process.calculi} 

% section concurrent_process_calculi_and_spatial_logics_ (end)
    
%\input{qm2pi.knots2pi} 

%\input{qm2pi.trefoil} 

%\input{qm2pi.mainthm} 

% subsection basic_interpretation (end)

%\input{qm2pi.rho.presentation} 
\subsection{The syntax and semantics of the notation system}\label{sub:the_syntax_and_semantics_of_the_notation_system} % (fold)

We now summarize a technical presentation of the calculus that
embodies our theory of dynamics. The typical presentation of such a
calculus follows the style of giving generators and relations on
them. The grammar, below, describing term constructors, freely
generates the set of processes, $\Proc$. This set is then quotiented
by a relation known as structural congruence and it is over this set
that the notion of dynamics is expressed. This presentation is
essentially that of \cite{MeredithR05} with the addition of
polyadicity and summation. For readability we have relegated some of
the technical subtleties to an appendix.

\subsubsection{Process grammar}\label{subsub:process_grammar}

\begin{mathpar}
  \inferrule* [lab=synchronization] {} {{M} \bc \pzero \;|\; x?F \;|\; x!C }
  \and
  \inferrule* [lab=abstraction] {} {{F} \bc (x)P}
  \and
  \inferrule* [lab=concretion] {} {{C} \bc \langle Q \rangle}
  \and
  \inferrule* [lab=process] {} {{P,Q} \bc M \;| \;P|Q \;|\; @{x}}
  \and
  \inferrule* [lab=name] {} {{x} \bc \quotep{P}}
\end{mathpar} 

Note that $\vec{x}$ (resp. $\vec{P}$) denotes a vector of names
(resp. processes) of length $|\vec{x}|$ (resp. $|\vec{P}|$). We adopt
the following useful abbreviations.

\begin{mathpar}
   x?(\vec{y}).P := x.(\vec{y})P \and  x\clift{\vec{P}} := x.\clift{\vec{P}}
   \and x!(y) := \lift{x}{\dropn{y}}
   \and \Pi_{i=0}^{n-1}P_i := P_0 | \ldots | P_{n-1}
\end{mathpar}

\subsubsection{Structural congruence}

\paragraph{Free and bound names and alpha-equivalence.} At the
core of structural equivalence is alpha-equivalence which identifies
process that are the same up to a change of variable. Formally, we
recognize the distinction between free and bound names. The free names
of a process, $\freenames{P}$, may be calculated recursively as
follows:

\begin{mathpar}
\freenames{\pzero} := \emptyset
  \and \\
  \freenames{x?(y).P} := \{ x \} \cup (\freenames{P} \setminus \{ y \})
  \and 
  \freenames{x!\langle P \rangle} := \{ x \} \cup \{ P \} 
  \and \\
  \freenames{P|Q} := \freenames{P} \cup \freenames{Q}
  \and \\
  \freenames{@{x}} := \{ x \}
\end{mathpar}

$\pi$
$\quotep{\pi}$

$\freenames{-} : \pi \to \mathcal{P}(\quotep{\pi})$

\begin{eqnarray*}
  \freenames{\pzero} & := & \emptyset \\
  \freenames{x?(y).P} & := & \{ x \} \cup (\freenames{P} \setminus \{ y \}) \\
  \freenames{x!\langle P \rangle} & := & \{ x \} \cup \{ P \} \\
  \freenames{P|Q} & := & \freenames{P} \cup \freenames{Q} \\
  \freenames{\dropn{x}} & := & \{ x \}
\end{eqnarray*}

The bound names of a process, $\boundnames{P}$, are those names occurring in $P$
that are not free. For example, in $x?(y).0$, the name $x$ is free, while $y$ is bound.

\begin{mathpar}
  \inferrule* [lab=monoidal-laws] {} { P|Q \equiv Q|P \and P|0 \equiv P \and P|(Q|R) \equiv (P|Q)|R }
\end{mathpar}

\begin{mathpar}
  \inferrule* [lab=alpha-equivalence] {} { (x)P \equiv (y)P\{y/x\} \and y \not\in \freenames{P} }
\end{mathpar}

\begin{definition}
Then two processes, $P,Q$, are alpha-equivalent if $P = Q\{\vec{y}/\vec{x}\}$ for
some $\vec{x} \in \boundnames{Q},\vec{y} \in \boundnames{P}$, where $Q\{\vec{y}/\vec{x}\}$
denotes the capture-avoiding substitution of $\vec{y}$ for $\vec{x}$ in $Q$.
\end{definition}

\begin{definition}
  The {\em structural congruence} \cite{SangiorgiWalker} , $\equiv$,
  between processes is the least congruence containing
  alpha-equivalence, satisfying the abelian monoid laws
  (associativity, commutativity and $\pzero$ as identity) for parallel
  composition $|$ and for summation $+$.
\end{definition}

\subsection{Name equivalence}

We take name equivalence, written $\nameeq$, to be the smallest
equivalence relation generated by the following rules.

\begin{mathpar}
\inferrule*[lab=Quote-drop]
{ }
{ \quotep{@{x}} \nameeq x }

\inferrule*[lab=Struct-equiv]
{ P \scong Q }
{ \quotep{P} \nameeq \quotep{Q} }
\end{mathpar}

The astute reader will have noticed that the mutual recursion of names
and processes imposes a mutual recursion on alpha-equivalence and
structural equivalence via name-equivalence. Fortunately, all of this
works out pleasantly and we may calculate in the natural way, free of
concern. The reader interested in the details is referred to the
appendix \ref{appendix:rho_details}.

\subsection{Substitution}

We use $\Proc$ for the set of processes, $\QProc$ for the set of
names, and $\id{\{}\vec{y} / \vec{x} \id{\}}$ to denote partial maps,
$s : \QProc \rightarrow \QProc$. A map, $s$ lifts, uniquely, to a map
on process terms, $\widehat{s} : \Proc \rightarrow \Proc$ by the
following equations.

\begin{mathpar}
  (0) \psubstp{Q}{P} := 0 \\
  (R \juxtap S) \psubstp{Q}{P}
  :=    
  (R)\psubstp{Q}{P} \juxtap (S) \psubstp{Q}{P} \\
  (x?(y).R) \psubstp{Q}{P}    
  :=    
  (x)\substp{Q}{P} (z)\concat( (R \psubstn{z}{y}) \psubstp{Q}{P} ) \\
  (\lift{x}{R}) \psubstp{Q}{P}  
  :=
  \lift{(x)\substp{Q}{P}}{ R \psubstp{Q}{P} } \\
%   (\dropn{x})  \psubstp{Q}{P}       
%   := 
%   \left\{ 
%     \begin{array}{ccc} 
%       \dropn{\quotep{Q}} & & x \nameeq \quotep{P} \\
%       \dropn{x} & & otherwise \\
%     \end{array}
%   \right. 
  (\dropn{x})  \psubstp{Q}{P}       
  := 
  \left\{ 
    \begin{array}{ccc} 
      Q & & x \nameeq \quotep{P} \\
      \dropn{x} & & otherwise \\
    \end{array}
  \right.
\end{mathpar}
 

where

\begin{eqnarray}
  (x)\id{\{} \lpquote Q \rpquote / \lpquote P \rpquote \id{\}}            = 
  \left\{ 
    \begin{array}{ccc}
      \lpquote Q \rpquote & & x \nameeq \lpquote P \rpquote \\
      x & & otherwise \\
    \end{array}
  \right. \nonumber
\end{eqnarray}

and $z$ is chosen distinct from $\quotep{P}$, $\quotep{Q}$, the free
names in $Q$, and all the names in $R$. Our $\alpha$-equivalence will
be built in the standard way from this substitution.

\begin{remark}\label{rem:no_self_referential_names}
  One consequence of these definitions is that $\forall P. \quotep{P}
  \not\in \freenames{P}$.
\end{remark}

\subsection{ Dynamic quote: an example }

Anticipating something of what's to come, consider applying the
substitution, $\widehat{\id{\{}u / z \id{\}}}$, to the following pair
of processes, $\lift{w}{y!(z)}$ and $w[ \lpquote y!(z) \rpquote ]$.

\begin{eqnarray}
	\lift{w}{y!(z)}\widehat{\id{\{}u / z \id{\}}}
		& = &
		\lift{w}{y!(u)} \nonumber\\
	w[ \lpquote y!(z) \rpquote ] \widehat{ \id{\{}u / z \id{\}} }
		& = &
		w[ \lpquote y!(z) \rpquote ] \nonumber
\end{eqnarray}

Because the body of the process between quotes is impervious to
substitution, we get radically different answers. In fact, by
examining the first process in an input context,
e.g. $x?(z).\lift{w}{y!(z)}$, we see that the process under the lift
operator may be shaped by prefixed inputs binding a name inside it. In
this sense, the lift operator will be seen as a way to dynamically
construct processes before reifying them as names.

Finally equipped with these standard features we can present the
dynamics of the calculus.

\subsubsection{Operational semantics} 

Finally, we introduce the computational dynamics. What marks these
algebras as distinct from other more traditionally studied algebraic
structures, e.g. vector spaces or polynomial rings, is the manner in
which dynamics is captured. In traditional structures, dynamics is typically
expressed through morphisms between such structures, as in linear maps
between vector spaces or morphisms between rings. In algebras
associated with the semantics of computation, the dynamics is
expressed as part of the algebraic structure itself, through a
reduction reduction relation typically denoted by $\red$. Below, we
give a recursive presentation of this relation for the calculus used
in the encoding.

$\red \subseteq \pi \times \pi$
$\red : \pi \to \mathcal{P}(\pi)$

\begin{mathpar}
  \inferrule* [lab=Comm] { \textsf{match}( x_{src}, x_{trgt} ) } { x_{trgt}?(y)P \; | \; x_{src}!\langle {Q} \rangle \red P\{\quotep{Q}/y}\} }
  \and \\
  \inferrule* [lab=Par] {{P} \red {P}'} {{{P} | {Q}} \red {{P}' | {Q}}}
  \and
  \inferrule* [lab=Equiv]{{{P} \scong {P}'} \andalso {{P}' \red {Q}'} \andalso {{Q}' \scong {Q}}}{{P} \red {Q}}
\end{mathpar}

\begin{eqnarray*}
  match_{\equiv} (\quotep{P},\quotep{Q}) & := & P \equiv Q \\
  match_{\dagger}(\quotep{P},\quotep{Q}) & := & \forall R. P|Q \red^{*} R => R \red^{*} 0 \\
  match_{K}(\quotep{P},\quotep{Q}) & := & K \mbox{ for some context } K
\end{eqnarray*}

$u?(x)P | u!\langle Q \rangle \red P\{\quotep{Q}/x\}$

%We write $\wred$ for $\red^*$, and $P\red$ if $\exists Q $ such that $ P \red Q$.
We write $P\red$ if $\exists Q $ such that $ P \red Q$ and $P\not\red$, otherwise.

\section{Replication}

As mentioned before, it is known that replication (and hence
recursion) can be implemented in a higher-order process algebra
\cite{SangiorgiWalker}. As our first example of calculation with the
machinery thus far presented we give the construction explicitly in
the {\rhoc}.

\begin{eqnarray}
	D_{x} & := & \prefix{x}{y}{(\binpar{\outputp{x}{y}}{@{y}})} \nonumber\\
	\bangp_{x}{P} & := & \binpar{{x}!\langle{\binpar{D_{x}}{P}}\rangle}{D_{x}} \nonumber
\end{eqnarray}

\begin{eqnarray}
	\bangp_{x}{P} & & \nonumber\\
	=
	& {x}!\langle{(\prefix{x}{y}{(\outputp{x}{y} | @{y})) | P}}\rangle 
	      | \prefix{x}{y}{(\outputp{x}{y} | @{y})} & \nonumber\\
	\red
	& (\outputp{x}{y} | @{y})\substn{\quotep{(\prefix{x}{y}{(@{y} | \outputp{x}{y})) | P}}}{y} & \nonumber\\
	=
	& \outputp{x}{\quotep{(\prefix{x}{y}{(\outputp{x}{y} | @{y})) | P}}}
	  | {(\prefix{x}{y}{(\outputp{x}{y} | @{y})) | P}} & \nonumber\\
	\red
	& \ldots & \nonumber\\
	\red^*
	& P | P | \ldots & \nonumber
\end{eqnarray}

Of course, this encoding, as an implementation, runs away, unfolding
$\bangp{P}$ eagerly. A lazier and more implementable replication
operator, restricted to input-guarded processes, may be obtained as follows.

\begin{eqnarray}
\bangp{\prefix{u}{v}{P}} 
	:= 
	\binpar{\lift{x}{\prefix{u}{v}{(\binpar{D(x)}{P})}}}{D(x)} \nonumber
\end{eqnarray}

\begin{remark}
  Note that the lazier definition still does not deal with summation
  or mixed summation (i.e. sums over input and output). The reader is
  invited to construct definitions of replication that deal with these
  features. 

  Further, the definitions are parameterized in a name, $x$. Can you,
  gentle reader, make a definition that eliminates this parameter and
  guarantees no accidental interaction between the replication
  machinery and the process being replicated -- i.e. no accidental
  sharing of names used by the process to get its work done and the
  name(s) used by the replication to effect copying. This latter
  revision of the definition of replication is crucial to obtaining
  the expected identity $!!P \sim !P$.
\end{remark}

\begin{remark}\label{rem:paradoxical_combinator}
  The reader familiar with the lambda calculus will have noticed the
  similarity between $D$ and the paradoxical combinator.

  [Ed. note: the existence of this seems to suggest we have to be more
  restrictive on the set of processes and names we admit if we are to
  support no-cloning.]
\end{remark}

\subsubsection{Bisimulation}

The computational dynamics gives rise to another kind of equivalence,
the equivalence of computational behavior. As previously mentioned
this is typically captured \emph{via} some form of bisimulation.

% The notion we use in this paper is weak barbed bisimulation
% \cite{milner91polyadicpi}.

The notion we use in this paper is derived from weak barbed
bisimulation \cite{milner91polyadicpi}. 

\begin{definition}
An \emph{observation relation}, $\downarrow_{\mathcal N}$, over a set
of names, $\mathcal N$, is the smallest relation satisfying the rules
below.

\infrule[Out-barb]{y \in {\mathcal N}, \; x \nameeq y}
		  {\outputp{x}{v} \downarrow_{\mathcal N} x}
\infrule[Par-barb]{\mbox{$P\downarrow_{\mathcal N} x$ or $Q\downarrow_{\mathcal N} x$}}
		  {\binpar{P}{Q} \downarrow_{\mathcal N} x}

We write $P \Downarrow_{\mathcal N} x$ if there is $Q$ such that 
$P \wred Q$ and $Q \downarrow_{\mathcal N} x$.
\end{definition}

\begin{definition}
%\label{def.bbisim}
An  ${\mathcal N}$-\emph{barbed bisimulation} over a set of names, ${\mathcal N}$, is a symmetric binary relation 
${\mathcal S}_{\mathcal N}$ between agents such that $P\rel{S}_{\mathcal N}Q$ implies:
\begin{enumerate}
\item If $P \red P'$ then $Q \wred Q'$ and $P'\rel{S}_{\mathcal N} Q'$.
\item If $P\downarrow_{\mathcal N} x$, then $Q\Downarrow_{\mathcal N} x$.
\end{enumerate}
$P$ is ${\mathcal N}$-barbed bisimilar to $Q$, written
$P \wbbisim_{\mathcal N} Q$, if $P \rel{S}_{\mathcal N} Q$ for some ${\mathcal N}$-barbed bisimulation ${\mathcal S}_{\mathcal N}$.
\end{definition}

$\mathcal{R} \subseteq \pi \times \pi$

$P \mathcal{R} Q => \forall P'. P \red P' \Rightarrow \exists Q'. Q \red Q', P' \mathcal{R} Q'$

$P \vdash x \Rightarrow Q \vdash x$

\begin{mathpar}
  \inferrule*[lab=Out-barb]{x \nameeq y}{{y}!\langle{Q}\rangle \vdash x}
  \and
  \inferrule*[lab=Par-barb]{\mbox{$P\vdash x$ or $Q\vdash x$}}{\binpar{P}{Q} \vdash x}
\end{mathpar}

\subsubsection{Contexts}

One of the principle advantages of computational calculi like the
$\pi$-calculus is a well-defined notion of context,
contextual-equivalence and a correlation between
contextual-equivalence and notions of bisimulation. The notion of
context allows the decomposition of a process into (sub-)process and
its syntactic environment, its context. Thus, a context may be
thought of as a process with a ``hole'' (written $\Box$) in it. The
application of a context $M$ to a process $P$, written $M[P]$, is
tantamount to filling the hole in $M$ with $P$. In this paper we do
not need the full weight of this theory, but do make use of the notion
of context in the proof the main theorem. 

\begin{mathpar}
  \inferrule* [lab=summation] {} {{M_{M},M_{N}} \bc \Box \;|\; x.M_{A} \;|\; M_{M}+M_{N}}
  \and
  \inferrule* [lab=agent] {} {{M_{A}} \bc (\vec{x})M_{P} \;| \; \clift{P_0,\ldots,M_{P},\ldots,P_N}}
  \and \\
  \inferrule* [lab=process] {} {{M_{P}} \bc M_{N} \;| \;P|M_{P} }
\end{mathpar} 

\begin{mathpar}
  \inferrule* [lab=sychronization] {} {M_{N} \bc \Box \;|\; x?M_{F} \;|\; x!M_{C}}
  \and
  \inferrule* [lab=abstraction] {} {{M_{F}} \bc (x)M_{P} }
  \and
  \inferrule* [lab=concretion] {} {{M_{C}} \bc \langle M_{P} \rangle }
  \and \\
  \inferrule* [lab=process] {} {{M_{P}} \bc M_{N} \;| \;P|M_{P} }
\end{mathpar}

\begin{definition}[contextual application] Given a context $M$, and
  process $P$, we define the \emph{contextual application}, $M[P] :=
  M\{P/\Box\}$. That is, the contextual application of M to P is the
  substitution of $P$ for $\Box$ in $M$.
\end{definition}

$\meaningof{-} : L \to \mathcal{P}(\pi)$

\begin{mathpar}
  \inferrule* [lab=collection] {} {\meaningof{true} = \pi, \and \meaningof{~E} = \pi \setminus \meaningof{E}, \and \meaningof{E_{1} \& E_{2}} = \meaningof{E_{1}} \cap \meaningof{E_{2}}}
\end{mathpar}

\begin{mathpar}
  \inferrule* [lab=structure] {} {\meaningof{0} = \{ P \in \pi | P \equiv 0 \}, \and \\ \meaningof{E_1 | E_2} = \{ P \in \pi | P \equiv P_{1} | P_{2}, P_{1} \in \meaningof{E_{1}}, P_{2} \in \meaningof{E_2}\} }
\end{mathpar}

\begin{mathpar}
 \inferrule* [lab=behavior] {} {\meaningof{\langle a?b \rangle E} = \{ P \in \pi | P \equiv Q | u?(y)P', \\ \and \\\\ \and \\ \;\;\; u \in \meaningof{a}, \forall z.P'\{z/y\} \in \meaningof{E\{z/b\}}\}, \and \\ \meaningof{a!E} = \{ P \in \pi | P \equiv Q | x!\langle P' \rangle, x \in \meaningof{a} P' \in \meaningof{E}\} }
\end{mathpar}

\begin{mathpar}
 \inferrule* [lab=nominal] {} {\meaningof{\quotep{E}} = \{ \quotep{P} \in \quotep{\pi} | P \in \meaningof{E} \}, \and \meaningof{\quotep{P}} = \{ \quotep{Q} \in \quotep{\pi} | P \equiv Q \} \and \\ \meaningof{@\quotep{E}} = \{ P \in \pi | P \equiv @x, x \in \meaningof{E} \}}
\end{mathpar}

\begin{eqnarray*}
  \\
  \meaningof{-} : TS \to ST
\end{eqnarray*}

\begin{eqnarray*}
  \\
  L : TS \to ST
\end{eqnarray*}

\begin{eqnarray*}
  \\
  P \models E \iff P \in \meaningof{E}
\end{eqnarray*}

\begin{eqnarray*}
  P \approx_{L} Q \iff \forall E \in L. P \models E \iff Q \models E
\end{eqnarray*}

\begin{eqnarray*}
  P \approx_{K} Q
\end{eqnarray*}

\begin{eqnarray*}
  P \approx Q
\end{eqnarray*}

$\approx_{K} = \approx = \approx_{L}$

\subsubsection{Contextual duality}

Note that contexts extend the quotation operation to a family of
operations from processes to names. Given a context, $M$, we can
define a \emph{nominal context}, $\quotep{M}$ by $\quotep{M}[P] :=
\quotep{M[P]}$. To foreshadow what is to come we observe that these
operations enjoy a duality with processes very much like the duality
between vectors and maps from vectors to scalars.

Further, because the calculus is essentially higher-order, we have a
correspondence between contexts and processes. More specifically,
given a name $x$ and a context $M$ we can construct $M^{*}_{x}$ such
that 

\begin{mathpar}
  M^{*}_{x} | \lift{x}{P} \red M[P]
\end{mathpar}

namely,

\begin{mathpar}
  M^{*}_{x} := x?(u).M[\dropn{u}]
\end{mathpar}

The dependence of $M^{*}_{x}$ on a name makes it an abstraction, 

\begin{mathpar}
  M^{*} := (x)x?(u).M[\dropn{u}]
\end{mathpar}

\subsection{Additional notation}

It will sometimes be convenient to denote the process a name
quotes. We already have the notation $x = \quotep{P}$, but it will be
convenient to introduce an alternate notation, $\procn{x}$, when we
want to emphasize the connection to the use of the name. Note that, by
virtue of name equivalence, $\quotep{\procn{x}} \nameeq x$; so, the
notation is consistent with previous definitions.

Further, because names have structure it is possible to effect
substitutions on the basis of that structure. This means we need to
upgrade our notation for substitutions, which we accomplish by
adapting comprehension notation. Thus,

\begin{mathpar}
  P\{ y / x : x \in S \}
\end{mathpar}

is interpreted to mean the process derived from P by replacing (in a
capture-avoiding manner) each occurrence of $x$ in $S$ by $y$. For example,

\begin{mathpar}
  P\{ \quotep{\procn{x}|\procn{x}} / x : x \in \freenames{P} \}
\end{mathpar}

will replace each (occurrence) of a free name $x$ in $P$ by
$\quotep{\procn{x}|\procn{x}}$.

Also, we will avail ourselves of the notation $x^{L}$ and $x^{R}$ to
denote injections of a name into disjoint copies of the name
space. There are numerous ways to accomplish this. One example can be
found in \cite{MeredithR05}. This notation overloads to vectors of
names: $\vec{x}^{\pi} := (x_{i}^{\pi} \; : \; 0 \leq i < |\vec{x}| )$ where $\pi \in \{L,R\}$.

We also use $P^{\Box} := P|\Box$.

In \cite{MeredithR05} an interpretation of the new operator is
given. It turns out that there are several possible interpretations
all enjoying the requisite algebraic properties of the operator (see
\cite{milner91polyadicpi}). We will therefore make liberal use of
$(\nu\; \vec{x})P$.

% subsection the_syntax_and_semantics_of_the_notation_system (end)   

\input{qm2pi.qmops} 

\input{qm2pi.sterngerlach} 

\input{qm2pi.metric} 

% section concurrent_process_calculi (end)

%\input{qm2pi.proofsketch}

% section proof sketch (end)

%\input{qm2pi.slviaknots} 

% section spatial logic via knots (end)

\input{qm2pi.conclusion}

% section conclusion (end)

%\input{qm2pi.dtcodes} 

% section wiring algorithm (end)

\input{qm2pi.ack} 

% section acknowledgments (end)

\newpage


\bibliographystyle{plain}   
\bibliography{../../biblios/main.bib}

\input{qm2pi.rhodetails}

\end{document}

 

%\documentclass[12pt]{llncs}
%\documentclass{jktr}

\usepackage[pdftex]{hyperref}                   
\usepackage {listings}
\usepackage {mathpartir}
\usepackage{bcprules}
%\usepackage{listings}
                       
\usepackage{graphicx} 
%\usepackage[margins=2.5cm,nohead,nofoot]{geometry}
%\usepackage{geometry}
\usepackage{amsfonts}
\usepackage{amstext}
\usepackage{latexsym}
\usepackage{amssymb}
\usepackage{color}


%\include{myPreamble}
\include{qm2pi.local} 

%\ifpdf
%\usepackage[pdftex]{graphicx}
%\else
%\usepackage{graphicx}
%\fi

 % \ifpdf
%  \usepackage{pdfsync}
%  \if


%\title{Brief Article}
%\author{David F. Snyder}
%\author{L.G. Meredith}

%\address{Dept. of Math., Texas State University--San Marcos, San Marcos, TX 78666}
       
\pagestyle{empty}


\begin{document}

\lstset{language=[Objective]Caml,frame=shadowbox}

\input{qm2pi.front}

% section front matter (end)

\input{qm2pi.intro} 
 
% section introduction (end)

% \input{qm2pi.knotations} 

% section notation (end)

\input{qm2pi.process.calculi} 

% section concurrent_process_calculi_and_spatial_logics_ (end)
    
%\input{qm2pi.knots2pi} 

%\input{qm2pi.trefoil} 

%\input{qm2pi.mainthm} 

% subsection basic_interpretation (end)

%\input{qm2pi.rho.presentation} 
\subsection{The syntax and semantics of the notation system}\label{sub:the_syntax_and_semantics_of_the_notation_system} % (fold)

We now summarize a technical presentation of the calculus that
embodies our theory of dynamics. The typical presentation of such a
calculus follows the style of giving generators and relations on
them. The grammar, below, describing term constructors, freely
generates the set of processes, $\Proc$. This set is then quotiented
by a relation known as structural congruence and it is over this set
that the notion of dynamics is expressed. This presentation is
essentially that of \cite{MeredithR05} with the addition of
polyadicity and summation. For readability we have relegated some of
the technical subtleties to an appendix.

\subsubsection{Process grammar}\label{subsub:process_grammar}

\begin{mathpar}
  \inferrule* [lab=synchronization] {} {{M} \bc \pzero \;|\; x?F \;|\; x!C }
  \and
  \inferrule* [lab=abstraction] {} {{F} \bc (x)P}
  \and
  \inferrule* [lab=concretion] {} {{C} \bc \langle Q \rangle}
  \and
  \inferrule* [lab=process] {} {{P,Q} \bc M \;| \;P|Q \;|\; @{x}}
  \and
  \inferrule* [lab=name] {} {{x} \bc \quotep{P}}
\end{mathpar} 

Note that $\vec{x}$ (resp. $\vec{P}$) denotes a vector of names
(resp. processes) of length $|\vec{x}|$ (resp. $|\vec{P}|$). We adopt
the following useful abbreviations.

\begin{mathpar}
   x?(\vec{y}).P := x.(\vec{y})P \and  x\clift{\vec{P}} := x.\clift{\vec{P}}
   \and x!(y) := \lift{x}{\dropn{y}}
   \and \Pi_{i=0}^{n-1}P_i := P_0 | \ldots | P_{n-1}
\end{mathpar}

\subsubsection{Structural congruence}

\paragraph{Free and bound names and alpha-equivalence.} At the
core of structural equivalence is alpha-equivalence which identifies
process that are the same up to a change of variable. Formally, we
recognize the distinction between free and bound names. The free names
of a process, $\freenames{P}$, may be calculated recursively as
follows:

\begin{mathpar}
\freenames{\pzero} := \emptyset
  \and \\
  \freenames{x?(y).P} := \{ x \} \cup (\freenames{P} \setminus \{ y \})
  \and 
  \freenames{x!\langle P \rangle} := \{ x \} \cup \{ P \} 
  \and \\
  \freenames{P|Q} := \freenames{P} \cup \freenames{Q}
  \and \\
  \freenames{@{x}} := \{ x \}
\end{mathpar}

$\pi$
$\quotep{\pi}$

$\freenames{-} : \pi \to \mathcal{P}(\quotep{\pi})$

\begin{eqnarray*}
  \freenames{\pzero} & := & \emptyset \\
  \freenames{x?(y).P} & := & \{ x \} \cup (\freenames{P} \setminus \{ y \}) \\
  \freenames{x!\langle P \rangle} & := & \{ x \} \cup \{ P \} \\
  \freenames{P|Q} & := & \freenames{P} \cup \freenames{Q} \\
  \freenames{\dropn{x}} & := & \{ x \}
\end{eqnarray*}

The bound names of a process, $\boundnames{P}$, are those names occurring in $P$
that are not free. For example, in $x?(y).0$, the name $x$ is free, while $y$ is bound.

\begin{mathpar}
  \inferrule* [lab=monoidal-laws] {} { P|Q \equiv Q|P \and P|0 \equiv P \and P|(Q|R) \equiv (P|Q)|R }
\end{mathpar}

\begin{mathpar}
  \inferrule* [lab=alpha-equivalence] {} { (x)P \equiv (y)P\{y/x\} \and y \not\in \freenames{P} }
\end{mathpar}

\begin{definition}
Then two processes, $P,Q$, are alpha-equivalent if $P = Q\{\vec{y}/\vec{x}\}$ for
some $\vec{x} \in \boundnames{Q},\vec{y} \in \boundnames{P}$, where $Q\{\vec{y}/\vec{x}\}$
denotes the capture-avoiding substitution of $\vec{y}$ for $\vec{x}$ in $Q$.
\end{definition}

\begin{definition}
  The {\em structural congruence} \cite{SangiorgiWalker} , $\equiv$,
  between processes is the least congruence containing
  alpha-equivalence, satisfying the abelian monoid laws
  (associativity, commutativity and $\pzero$ as identity) for parallel
  composition $|$ and for summation $+$.
\end{definition}

\subsection{Name equivalence}

We take name equivalence, written $\nameeq$, to be the smallest
equivalence relation generated by the following rules.

\begin{mathpar}
\inferrule*[lab=Quote-drop]
{ }
{ \quotep{@{x}} \nameeq x }

\inferrule*[lab=Struct-equiv]
{ P \scong Q }
{ \quotep{P} \nameeq \quotep{Q} }
\end{mathpar}

The astute reader will have noticed that the mutual recursion of names
and processes imposes a mutual recursion on alpha-equivalence and
structural equivalence via name-equivalence. Fortunately, all of this
works out pleasantly and we may calculate in the natural way, free of
concern. The reader interested in the details is referred to the
appendix \ref{appendix:rho_details}.

\subsection{Substitution}

We use $\Proc$ for the set of processes, $\QProc$ for the set of
names, and $\id{\{}\vec{y} / \vec{x} \id{\}}$ to denote partial maps,
$s : \QProc \rightarrow \QProc$. A map, $s$ lifts, uniquely, to a map
on process terms, $\widehat{s} : \Proc \rightarrow \Proc$ by the
following equations.

\begin{mathpar}
  (0) \psubstp{Q}{P} := 0 \\
  (R \juxtap S) \psubstp{Q}{P}
  :=    
  (R)\psubstp{Q}{P} \juxtap (S) \psubstp{Q}{P} \\
  (x?(y).R) \psubstp{Q}{P}    
  :=    
  (x)\substp{Q}{P} (z)\concat( (R \psubstn{z}{y}) \psubstp{Q}{P} ) \\
  (\lift{x}{R}) \psubstp{Q}{P}  
  :=
  \lift{(x)\substp{Q}{P}}{ R \psubstp{Q}{P} } \\
%   (\dropn{x})  \psubstp{Q}{P}       
%   := 
%   \left\{ 
%     \begin{array}{ccc} 
%       \dropn{\quotep{Q}} & & x \nameeq \quotep{P} \\
%       \dropn{x} & & otherwise \\
%     \end{array}
%   \right. 
  (\dropn{x})  \psubstp{Q}{P}       
  := 
  \left\{ 
    \begin{array}{ccc} 
      Q & & x \nameeq \quotep{P} \\
      \dropn{x} & & otherwise \\
    \end{array}
  \right.
\end{mathpar}
 

where

\begin{eqnarray}
  (x)\id{\{} \lpquote Q \rpquote / \lpquote P \rpquote \id{\}}            = 
  \left\{ 
    \begin{array}{ccc}
      \lpquote Q \rpquote & & x \nameeq \lpquote P \rpquote \\
      x & & otherwise \\
    \end{array}
  \right. \nonumber
\end{eqnarray}

and $z$ is chosen distinct from $\quotep{P}$, $\quotep{Q}$, the free
names in $Q$, and all the names in $R$. Our $\alpha$-equivalence will
be built in the standard way from this substitution.

\begin{remark}\label{rem:no_self_referential_names}
  One consequence of these definitions is that $\forall P. \quotep{P}
  \not\in \freenames{P}$.
\end{remark}

\subsection{ Dynamic quote: an example }

Anticipating something of what's to come, consider applying the
substitution, $\widehat{\id{\{}u / z \id{\}}}$, to the following pair
of processes, $\lift{w}{y!(z)}$ and $w[ \lpquote y!(z) \rpquote ]$.

\begin{eqnarray}
	\lift{w}{y!(z)}\widehat{\id{\{}u / z \id{\}}}
		& = &
		\lift{w}{y!(u)} \nonumber\\
	w[ \lpquote y!(z) \rpquote ] \widehat{ \id{\{}u / z \id{\}} }
		& = &
		w[ \lpquote y!(z) \rpquote ] \nonumber
\end{eqnarray}

Because the body of the process between quotes is impervious to
substitution, we get radically different answers. In fact, by
examining the first process in an input context,
e.g. $x?(z).\lift{w}{y!(z)}$, we see that the process under the lift
operator may be shaped by prefixed inputs binding a name inside it. In
this sense, the lift operator will be seen as a way to dynamically
construct processes before reifying them as names.

Finally equipped with these standard features we can present the
dynamics of the calculus.

\subsubsection{Operational semantics} 

Finally, we introduce the computational dynamics. What marks these
algebras as distinct from other more traditionally studied algebraic
structures, e.g. vector spaces or polynomial rings, is the manner in
which dynamics is captured. In traditional structures, dynamics is typically
expressed through morphisms between such structures, as in linear maps
between vector spaces or morphisms between rings. In algebras
associated with the semantics of computation, the dynamics is
expressed as part of the algebraic structure itself, through a
reduction reduction relation typically denoted by $\red$. Below, we
give a recursive presentation of this relation for the calculus used
in the encoding.

$\red \subseteq \pi \times \pi$
$\red : \pi \to \mathcal{P}(\pi)$

\begin{mathpar}
  \inferrule* [lab=Comm] { \textsf{match}( x_{src}, x_{trgt} ) } { x_{trgt}?(y)P \; | \; x_{src}!\langle {Q} \rangle \red P\{\quotep{Q}/y}\} }
  \and \\
  \inferrule* [lab=Par] {{P} \red {P}'} {{{P} | {Q}} \red {{P}' | {Q}}}
  \and
  \inferrule* [lab=Equiv]{{{P} \scong {P}'} \andalso {{P}' \red {Q}'} \andalso {{Q}' \scong {Q}}}{{P} \red {Q}}
\end{mathpar}

\begin{eqnarray*}
  match_{\equiv} (\quotep{P},\quotep{Q}) & := & P \equiv Q \\
  match_{\dagger}(\quotep{P},\quotep{Q}) & := & \forall R. P|Q \red^{*} R => R \red^{*} 0 \\
  match_{K}(\quotep{P},\quotep{Q}) & := & K \mbox{ for some context } K
\end{eqnarray*}

$u?(x)P | u!\langle Q \rangle \red P\{\quotep{Q}/x\}$

%We write $\wred$ for $\red^*$, and $P\red$ if $\exists Q $ such that $ P \red Q$.
We write $P\red$ if $\exists Q $ such that $ P \red Q$ and $P\not\red$, otherwise.

\section{Replication}

As mentioned before, it is known that replication (and hence
recursion) can be implemented in a higher-order process algebra
\cite{SangiorgiWalker}. As our first example of calculation with the
machinery thus far presented we give the construction explicitly in
the {\rhoc}.

\begin{eqnarray}
	D_{x} & := & \prefix{x}{y}{(\binpar{\outputp{x}{y}}{@{y}})} \nonumber\\
	\bangp_{x}{P} & := & \binpar{{x}!\langle{\binpar{D_{x}}{P}}\rangle}{D_{x}} \nonumber
\end{eqnarray}

\begin{eqnarray}
	\bangp_{x}{P} & & \nonumber\\
	=
	& {x}!\langle{(\prefix{x}{y}{(\outputp{x}{y} | @{y})) | P}}\rangle 
	      | \prefix{x}{y}{(\outputp{x}{y} | @{y})} & \nonumber\\
	\red
	& (\outputp{x}{y} | @{y})\substn{\quotep{(\prefix{x}{y}{(@{y} | \outputp{x}{y})) | P}}}{y} & \nonumber\\
	=
	& \outputp{x}{\quotep{(\prefix{x}{y}{(\outputp{x}{y} | @{y})) | P}}}
	  | {(\prefix{x}{y}{(\outputp{x}{y} | @{y})) | P}} & \nonumber\\
	\red
	& \ldots & \nonumber\\
	\red^*
	& P | P | \ldots & \nonumber
\end{eqnarray}

Of course, this encoding, as an implementation, runs away, unfolding
$\bangp{P}$ eagerly. A lazier and more implementable replication
operator, restricted to input-guarded processes, may be obtained as follows.

\begin{eqnarray}
\bangp{\prefix{u}{v}{P}} 
	:= 
	\binpar{\lift{x}{\prefix{u}{v}{(\binpar{D(x)}{P})}}}{D(x)} \nonumber
\end{eqnarray}

\begin{remark}
  Note that the lazier definition still does not deal with summation
  or mixed summation (i.e. sums over input and output). The reader is
  invited to construct definitions of replication that deal with these
  features. 

  Further, the definitions are parameterized in a name, $x$. Can you,
  gentle reader, make a definition that eliminates this parameter and
  guarantees no accidental interaction between the replication
  machinery and the process being replicated -- i.e. no accidental
  sharing of names used by the process to get its work done and the
  name(s) used by the replication to effect copying. This latter
  revision of the definition of replication is crucial to obtaining
  the expected identity $!!P \sim !P$.
\end{remark}

\begin{remark}\label{rem:paradoxical_combinator}
  The reader familiar with the lambda calculus will have noticed the
  similarity between $D$ and the paradoxical combinator.

  [Ed. note: the existence of this seems to suggest we have to be more
  restrictive on the set of processes and names we admit if we are to
  support no-cloning.]
\end{remark}

\subsubsection{Bisimulation}

The computational dynamics gives rise to another kind of equivalence,
the equivalence of computational behavior. As previously mentioned
this is typically captured \emph{via} some form of bisimulation.

% The notion we use in this paper is weak barbed bisimulation
% \cite{milner91polyadicpi}.

The notion we use in this paper is derived from weak barbed
bisimulation \cite{milner91polyadicpi}. 

\begin{definition}
An \emph{observation relation}, $\downarrow_{\mathcal N}$, over a set
of names, $\mathcal N$, is the smallest relation satisfying the rules
below.

\infrule[Out-barb]{y \in {\mathcal N}, \; x \nameeq y}
		  {\outputp{x}{v} \downarrow_{\mathcal N} x}
\infrule[Par-barb]{\mbox{$P\downarrow_{\mathcal N} x$ or $Q\downarrow_{\mathcal N} x$}}
		  {\binpar{P}{Q} \downarrow_{\mathcal N} x}

We write $P \Downarrow_{\mathcal N} x$ if there is $Q$ such that 
$P \wred Q$ and $Q \downarrow_{\mathcal N} x$.
\end{definition}

\begin{definition}
%\label{def.bbisim}
An  ${\mathcal N}$-\emph{barbed bisimulation} over a set of names, ${\mathcal N}$, is a symmetric binary relation 
${\mathcal S}_{\mathcal N}$ between agents such that $P\rel{S}_{\mathcal N}Q$ implies:
\begin{enumerate}
\item If $P \red P'$ then $Q \wred Q'$ and $P'\rel{S}_{\mathcal N} Q'$.
\item If $P\downarrow_{\mathcal N} x$, then $Q\Downarrow_{\mathcal N} x$.
\end{enumerate}
$P$ is ${\mathcal N}$-barbed bisimilar to $Q$, written
$P \wbbisim_{\mathcal N} Q$, if $P \rel{S}_{\mathcal N} Q$ for some ${\mathcal N}$-barbed bisimulation ${\mathcal S}_{\mathcal N}$.
\end{definition}

$\mathcal{R} \subseteq \pi \times \pi$

$P \mathcal{R} Q => \forall P'. P \red P' \Rightarrow \exists Q'. Q \red Q', P' \mathcal{R} Q'$

$P \vdash x \Rightarrow Q \vdash x$

\begin{mathpar}
  \inferrule*[lab=Out-barb]{x \nameeq y}{{y}!\langle{Q}\rangle \vdash x}
  \and
  \inferrule*[lab=Par-barb]{\mbox{$P\vdash x$ or $Q\vdash x$}}{\binpar{P}{Q} \vdash x}
\end{mathpar}

\subsubsection{Contexts}

One of the principle advantages of computational calculi like the
$\pi$-calculus is a well-defined notion of context,
contextual-equivalence and a correlation between
contextual-equivalence and notions of bisimulation. The notion of
context allows the decomposition of a process into (sub-)process and
its syntactic environment, its context. Thus, a context may be
thought of as a process with a ``hole'' (written $\Box$) in it. The
application of a context $M$ to a process $P$, written $M[P]$, is
tantamount to filling the hole in $M$ with $P$. In this paper we do
not need the full weight of this theory, but do make use of the notion
of context in the proof the main theorem. 

\begin{mathpar}
  \inferrule* [lab=summation] {} {{M_{M},M_{N}} \bc \Box \;|\; x.M_{A} \;|\; M_{M}+M_{N}}
  \and
  \inferrule* [lab=agent] {} {{M_{A}} \bc (\vec{x})M_{P} \;| \; \clift{P_0,\ldots,M_{P},\ldots,P_N}}
  \and \\
  \inferrule* [lab=process] {} {{M_{P}} \bc M_{N} \;| \;P|M_{P} }
\end{mathpar} 

\begin{mathpar}
  \inferrule* [lab=sychronization] {} {M_{N} \bc \Box \;|\; x?M_{F} \;|\; x!M_{C}}
  \and
  \inferrule* [lab=abstraction] {} {{M_{F}} \bc (x)M_{P} }
  \and
  \inferrule* [lab=concretion] {} {{M_{C}} \bc \langle M_{P} \rangle }
  \and \\
  \inferrule* [lab=process] {} {{M_{P}} \bc M_{N} \;| \;P|M_{P} }
\end{mathpar}

\begin{definition}[contextual application] Given a context $M$, and
  process $P$, we define the \emph{contextual application}, $M[P] :=
  M\{P/\Box\}$. That is, the contextual application of M to P is the
  substitution of $P$ for $\Box$ in $M$.
\end{definition}

$\meaningof{-} : L \to \mathcal{P}(\pi)$

\begin{mathpar}
  \inferrule* [lab=collection] {} {\meaningof{true} = \pi, \and \meaningof{~E} = \pi \setminus \meaningof{E}, \and \meaningof{E_{1} \& E_{2}} = \meaningof{E_{1}} \cap \meaningof{E_{2}}}
\end{mathpar}

\begin{mathpar}
  \inferrule* [lab=structure] {} {\meaningof{0} = \{ P \in \pi | P \equiv 0 \}, \and \\ \meaningof{E_1 | E_2} = \{ P \in \pi | P \equiv P_{1} | P_{2}, P_{1} \in \meaningof{E_{1}}, P_{2} \in \meaningof{E_2}\} }
\end{mathpar}

\begin{mathpar}
 \inferrule* [lab=behavior] {} {\meaningof{\langle a?b \rangle E} = \{ P \in \pi | P \equiv Q | u?(y)P', \\ \and \\\\ \and \\ \;\;\; u \in \meaningof{a}, \forall z.P'\{z/y\} \in \meaningof{E\{z/b\}}\}, \and \\ \meaningof{a!E} = \{ P \in \pi | P \equiv Q | x!\langle P' \rangle, x \in \meaningof{a} P' \in \meaningof{E}\} }
\end{mathpar}

\begin{mathpar}
 \inferrule* [lab=nominal] {} {\meaningof{\quotep{E}} = \{ \quotep{P} \in \quotep{\pi} | P \in \meaningof{E} \}, \and \meaningof{\quotep{P}} = \{ \quotep{Q} \in \quotep{\pi} | P \equiv Q \} \and \\ \meaningof{@\quotep{E}} = \{ P \in \pi | P \equiv @x, x \in \meaningof{E} \}}
\end{mathpar}

\begin{eqnarray*}
  \\
  \meaningof{-} : TS \to ST
\end{eqnarray*}

\begin{eqnarray*}
  \\
  L : TS \to ST
\end{eqnarray*}

\begin{eqnarray*}
  \\
  P \models E \iff P \in \meaningof{E}
\end{eqnarray*}

\begin{eqnarray*}
  P \approx_{L} Q \iff \forall E \in L. P \models E \iff Q \models E
\end{eqnarray*}

\begin{eqnarray*}
  P \approx_{K} Q
\end{eqnarray*}

\begin{eqnarray*}
  P \approx Q
\end{eqnarray*}

$\approx_{K} = \approx = \approx_{L}$

\subsubsection{Contextual duality}

Note that contexts extend the quotation operation to a family of
operations from processes to names. Given a context, $M$, we can
define a \emph{nominal context}, $\quotep{M}$ by $\quotep{M}[P] :=
\quotep{M[P]}$. To foreshadow what is to come we observe that these
operations enjoy a duality with processes very much like the duality
between vectors and maps from vectors to scalars.

Further, because the calculus is essentially higher-order, we have a
correspondence between contexts and processes. More specifically,
given a name $x$ and a context $M$ we can construct $M^{*}_{x}$ such
that 

\begin{mathpar}
  M^{*}_{x} | \lift{x}{P} \red M[P]
\end{mathpar}

namely,

\begin{mathpar}
  M^{*}_{x} := x?(u).M[\dropn{u}]
\end{mathpar}

The dependence of $M^{*}_{x}$ on a name makes it an abstraction, 

\begin{mathpar}
  M^{*} := (x)x?(u).M[\dropn{u}]
\end{mathpar}

\subsection{Additional notation}

It will sometimes be convenient to denote the process a name
quotes. We already have the notation $x = \quotep{P}$, but it will be
convenient to introduce an alternate notation, $\procn{x}$, when we
want to emphasize the connection to the use of the name. Note that, by
virtue of name equivalence, $\quotep{\procn{x}} \nameeq x$; so, the
notation is consistent with previous definitions.

Further, because names have structure it is possible to effect
substitutions on the basis of that structure. This means we need to
upgrade our notation for substitutions, which we accomplish by
adapting comprehension notation. Thus,

\begin{mathpar}
  P\{ y / x : x \in S \}
\end{mathpar}

is interpreted to mean the process derived from P by replacing (in a
capture-avoiding manner) each occurrence of $x$ in $S$ by $y$. For example,

\begin{mathpar}
  P\{ \quotep{\procn{x}|\procn{x}} / x : x \in \freenames{P} \}
\end{mathpar}

will replace each (occurrence) of a free name $x$ in $P$ by
$\quotep{\procn{x}|\procn{x}}$.

Also, we will avail ourselves of the notation $x^{L}$ and $x^{R}$ to
denote injections of a name into disjoint copies of the name
space. There are numerous ways to accomplish this. One example can be
found in \cite{MeredithR05}. This notation overloads to vectors of
names: $\vec{x}^{\pi} := (x_{i}^{\pi} \; : \; 0 \leq i < |\vec{x}| )$ where $\pi \in \{L,R\}$.

We also use $P^{\Box} := P|\Box$.

In \cite{MeredithR05} an interpretation of the new operator is
given. It turns out that there are several possible interpretations
all enjoying the requisite algebraic properties of the operator (see
\cite{milner91polyadicpi}). We will therefore make liberal use of
$(\nu\; \vec{x})P$.

% subsection the_syntax_and_semantics_of_the_notation_system (end)   

\input{qm2pi.qmops} 

\input{qm2pi.sterngerlach} 

\input{qm2pi.metric} 

% section concurrent_process_calculi (end)

%\input{qm2pi.proofsketch}

% section proof sketch (end)

%\input{qm2pi.slviaknots} 

% section spatial logic via knots (end)

\input{qm2pi.conclusion}

% section conclusion (end)

%\input{qm2pi.dtcodes} 

% section wiring algorithm (end)

\input{qm2pi.ack} 

% section acknowledgments (end)

\newpage


\bibliographystyle{plain}   
\bibliography{../../biblios/main.bib}

\input{qm2pi.rhodetails}

\end{document}

 

%\documentclass[12pt]{llncs}
%\documentclass{jktr}

\usepackage[pdftex]{hyperref}                   
\usepackage {listings}
\usepackage {mathpartir}
\usepackage{bcprules}
%\usepackage{listings}
                       
\usepackage{graphicx} 
%\usepackage[margins=2.5cm,nohead,nofoot]{geometry}
%\usepackage{geometry}
\usepackage{amsfonts}
\usepackage{amstext}
\usepackage{latexsym}
\usepackage{amssymb}
\usepackage{color}


%\include{myPreamble}
\include{qm2pi.local} 

%\ifpdf
%\usepackage[pdftex]{graphicx}
%\else
%\usepackage{graphicx}
%\fi

 % \ifpdf
%  \usepackage{pdfsync}
%  \if


%\title{Brief Article}
%\author{David F. Snyder}
%\author{L.G. Meredith}

%\address{Dept. of Math., Texas State University--San Marcos, San Marcos, TX 78666}
       
\pagestyle{empty}


\begin{document}

\lstset{language=[Objective]Caml,frame=shadowbox}

\input{qm2pi.front}

% section front matter (end)

\input{qm2pi.intro} 
 
% section introduction (end)

% \input{qm2pi.knotations} 

% section notation (end)

\input{qm2pi.process.calculi} 

% section concurrent_process_calculi_and_spatial_logics_ (end)
    
%\input{qm2pi.knots2pi} 

%\input{qm2pi.trefoil} 

%\input{qm2pi.mainthm} 

% subsection basic_interpretation (end)

%\input{qm2pi.rho.presentation} 
\subsection{The syntax and semantics of the notation system}\label{sub:the_syntax_and_semantics_of_the_notation_system} % (fold)

We now summarize a technical presentation of the calculus that
embodies our theory of dynamics. The typical presentation of such a
calculus follows the style of giving generators and relations on
them. The grammar, below, describing term constructors, freely
generates the set of processes, $\Proc$. This set is then quotiented
by a relation known as structural congruence and it is over this set
that the notion of dynamics is expressed. This presentation is
essentially that of \cite{MeredithR05} with the addition of
polyadicity and summation. For readability we have relegated some of
the technical subtleties to an appendix.

\subsubsection{Process grammar}\label{subsub:process_grammar}

\begin{mathpar}
  \inferrule* [lab=synchronization] {} {{M} \bc \pzero \;|\; x?F \;|\; x!C }
  \and
  \inferrule* [lab=abstraction] {} {{F} \bc (x)P}
  \and
  \inferrule* [lab=concretion] {} {{C} \bc \langle Q \rangle}
  \and
  \inferrule* [lab=process] {} {{P,Q} \bc M \;| \;P|Q \;|\; @{x}}
  \and
  \inferrule* [lab=name] {} {{x} \bc \quotep{P}}
\end{mathpar} 

Note that $\vec{x}$ (resp. $\vec{P}$) denotes a vector of names
(resp. processes) of length $|\vec{x}|$ (resp. $|\vec{P}|$). We adopt
the following useful abbreviations.

\begin{mathpar}
   x?(\vec{y}).P := x.(\vec{y})P \and  x\clift{\vec{P}} := x.\clift{\vec{P}}
   \and x!(y) := \lift{x}{\dropn{y}}
   \and \Pi_{i=0}^{n-1}P_i := P_0 | \ldots | P_{n-1}
\end{mathpar}

\subsubsection{Structural congruence}

\paragraph{Free and bound names and alpha-equivalence.} At the
core of structural equivalence is alpha-equivalence which identifies
process that are the same up to a change of variable. Formally, we
recognize the distinction between free and bound names. The free names
of a process, $\freenames{P}$, may be calculated recursively as
follows:

\begin{mathpar}
\freenames{\pzero} := \emptyset
  \and \\
  \freenames{x?(y).P} := \{ x \} \cup (\freenames{P} \setminus \{ y \})
  \and 
  \freenames{x!\langle P \rangle} := \{ x \} \cup \{ P \} 
  \and \\
  \freenames{P|Q} := \freenames{P} \cup \freenames{Q}
  \and \\
  \freenames{@{x}} := \{ x \}
\end{mathpar}

$\pi$
$\quotep{\pi}$

$\freenames{-} : \pi \to \mathcal{P}(\quotep{\pi})$

\begin{eqnarray*}
  \freenames{\pzero} & := & \emptyset \\
  \freenames{x?(y).P} & := & \{ x \} \cup (\freenames{P} \setminus \{ y \}) \\
  \freenames{x!\langle P \rangle} & := & \{ x \} \cup \{ P \} \\
  \freenames{P|Q} & := & \freenames{P} \cup \freenames{Q} \\
  \freenames{\dropn{x}} & := & \{ x \}
\end{eqnarray*}

The bound names of a process, $\boundnames{P}$, are those names occurring in $P$
that are not free. For example, in $x?(y).0$, the name $x$ is free, while $y$ is bound.

\begin{mathpar}
  \inferrule* [lab=monoidal-laws] {} { P|Q \equiv Q|P \and P|0 \equiv P \and P|(Q|R) \equiv (P|Q)|R }
\end{mathpar}

\begin{mathpar}
  \inferrule* [lab=alpha-equivalence] {} { (x)P \equiv (y)P\{y/x\} \and y \not\in \freenames{P} }
\end{mathpar}

\begin{definition}
Then two processes, $P,Q$, are alpha-equivalent if $P = Q\{\vec{y}/\vec{x}\}$ for
some $\vec{x} \in \boundnames{Q},\vec{y} \in \boundnames{P}$, where $Q\{\vec{y}/\vec{x}\}$
denotes the capture-avoiding substitution of $\vec{y}$ for $\vec{x}$ in $Q$.
\end{definition}

\begin{definition}
  The {\em structural congruence} \cite{SangiorgiWalker} , $\equiv$,
  between processes is the least congruence containing
  alpha-equivalence, satisfying the abelian monoid laws
  (associativity, commutativity and $\pzero$ as identity) for parallel
  composition $|$ and for summation $+$.
\end{definition}

\subsection{Name equivalence}

We take name equivalence, written $\nameeq$, to be the smallest
equivalence relation generated by the following rules.

\begin{mathpar}
\inferrule*[lab=Quote-drop]
{ }
{ \quotep{@{x}} \nameeq x }

\inferrule*[lab=Struct-equiv]
{ P \scong Q }
{ \quotep{P} \nameeq \quotep{Q} }
\end{mathpar}

The astute reader will have noticed that the mutual recursion of names
and processes imposes a mutual recursion on alpha-equivalence and
structural equivalence via name-equivalence. Fortunately, all of this
works out pleasantly and we may calculate in the natural way, free of
concern. The reader interested in the details is referred to the
appendix \ref{appendix:rho_details}.

\subsection{Substitution}

We use $\Proc$ for the set of processes, $\QProc$ for the set of
names, and $\id{\{}\vec{y} / \vec{x} \id{\}}$ to denote partial maps,
$s : \QProc \rightarrow \QProc$. A map, $s$ lifts, uniquely, to a map
on process terms, $\widehat{s} : \Proc \rightarrow \Proc$ by the
following equations.

\begin{mathpar}
  (0) \psubstp{Q}{P} := 0 \\
  (R \juxtap S) \psubstp{Q}{P}
  :=    
  (R)\psubstp{Q}{P} \juxtap (S) \psubstp{Q}{P} \\
  (x?(y).R) \psubstp{Q}{P}    
  :=    
  (x)\substp{Q}{P} (z)\concat( (R \psubstn{z}{y}) \psubstp{Q}{P} ) \\
  (\lift{x}{R}) \psubstp{Q}{P}  
  :=
  \lift{(x)\substp{Q}{P}}{ R \psubstp{Q}{P} } \\
%   (\dropn{x})  \psubstp{Q}{P}       
%   := 
%   \left\{ 
%     \begin{array}{ccc} 
%       \dropn{\quotep{Q}} & & x \nameeq \quotep{P} \\
%       \dropn{x} & & otherwise \\
%     \end{array}
%   \right. 
  (\dropn{x})  \psubstp{Q}{P}       
  := 
  \left\{ 
    \begin{array}{ccc} 
      Q & & x \nameeq \quotep{P} \\
      \dropn{x} & & otherwise \\
    \end{array}
  \right.
\end{mathpar}
 

where

\begin{eqnarray}
  (x)\id{\{} \lpquote Q \rpquote / \lpquote P \rpquote \id{\}}            = 
  \left\{ 
    \begin{array}{ccc}
      \lpquote Q \rpquote & & x \nameeq \lpquote P \rpquote \\
      x & & otherwise \\
    \end{array}
  \right. \nonumber
\end{eqnarray}

and $z$ is chosen distinct from $\quotep{P}$, $\quotep{Q}$, the free
names in $Q$, and all the names in $R$. Our $\alpha$-equivalence will
be built in the standard way from this substitution.

\begin{remark}\label{rem:no_self_referential_names}
  One consequence of these definitions is that $\forall P. \quotep{P}
  \not\in \freenames{P}$.
\end{remark}

\subsection{ Dynamic quote: an example }

Anticipating something of what's to come, consider applying the
substitution, $\widehat{\id{\{}u / z \id{\}}}$, to the following pair
of processes, $\lift{w}{y!(z)}$ and $w[ \lpquote y!(z) \rpquote ]$.

\begin{eqnarray}
	\lift{w}{y!(z)}\widehat{\id{\{}u / z \id{\}}}
		& = &
		\lift{w}{y!(u)} \nonumber\\
	w[ \lpquote y!(z) \rpquote ] \widehat{ \id{\{}u / z \id{\}} }
		& = &
		w[ \lpquote y!(z) \rpquote ] \nonumber
\end{eqnarray}

Because the body of the process between quotes is impervious to
substitution, we get radically different answers. In fact, by
examining the first process in an input context,
e.g. $x?(z).\lift{w}{y!(z)}$, we see that the process under the lift
operator may be shaped by prefixed inputs binding a name inside it. In
this sense, the lift operator will be seen as a way to dynamically
construct processes before reifying them as names.

Finally equipped with these standard features we can present the
dynamics of the calculus.

\subsubsection{Operational semantics} 

Finally, we introduce the computational dynamics. What marks these
algebras as distinct from other more traditionally studied algebraic
structures, e.g. vector spaces or polynomial rings, is the manner in
which dynamics is captured. In traditional structures, dynamics is typically
expressed through morphisms between such structures, as in linear maps
between vector spaces or morphisms between rings. In algebras
associated with the semantics of computation, the dynamics is
expressed as part of the algebraic structure itself, through a
reduction reduction relation typically denoted by $\red$. Below, we
give a recursive presentation of this relation for the calculus used
in the encoding.

$\red \subseteq \pi \times \pi$
$\red : \pi \to \mathcal{P}(\pi)$

\begin{mathpar}
  \inferrule* [lab=Comm] { \textsf{match}( x_{src}, x_{trgt} ) } { x_{trgt}?(y)P \; | \; x_{src}!\langle {Q} \rangle \red P\{\quotep{Q}/y}\} }
  \and \\
  \inferrule* [lab=Par] {{P} \red {P}'} {{{P} | {Q}} \red {{P}' | {Q}}}
  \and
  \inferrule* [lab=Equiv]{{{P} \scong {P}'} \andalso {{P}' \red {Q}'} \andalso {{Q}' \scong {Q}}}{{P} \red {Q}}
\end{mathpar}

\begin{eqnarray*}
  match_{\equiv} (\quotep{P},\quotep{Q}) & := & P \equiv Q \\
  match_{\dagger}(\quotep{P},\quotep{Q}) & := & \forall R. P|Q \red^{*} R => R \red^{*} 0 \\
  match_{K}(\quotep{P},\quotep{Q}) & := & K \mbox{ for some context } K
\end{eqnarray*}

$u?(x)P | u!\langle Q \rangle \red P\{\quotep{Q}/x\}$

%We write $\wred$ for $\red^*$, and $P\red$ if $\exists Q $ such that $ P \red Q$.
We write $P\red$ if $\exists Q $ such that $ P \red Q$ and $P\not\red$, otherwise.

\section{Replication}

As mentioned before, it is known that replication (and hence
recursion) can be implemented in a higher-order process algebra
\cite{SangiorgiWalker}. As our first example of calculation with the
machinery thus far presented we give the construction explicitly in
the {\rhoc}.

\begin{eqnarray}
	D_{x} & := & \prefix{x}{y}{(\binpar{\outputp{x}{y}}{@{y}})} \nonumber\\
	\bangp_{x}{P} & := & \binpar{{x}!\langle{\binpar{D_{x}}{P}}\rangle}{D_{x}} \nonumber
\end{eqnarray}

\begin{eqnarray}
	\bangp_{x}{P} & & \nonumber\\
	=
	& {x}!\langle{(\prefix{x}{y}{(\outputp{x}{y} | @{y})) | P}}\rangle 
	      | \prefix{x}{y}{(\outputp{x}{y} | @{y})} & \nonumber\\
	\red
	& (\outputp{x}{y} | @{y})\substn{\quotep{(\prefix{x}{y}{(@{y} | \outputp{x}{y})) | P}}}{y} & \nonumber\\
	=
	& \outputp{x}{\quotep{(\prefix{x}{y}{(\outputp{x}{y} | @{y})) | P}}}
	  | {(\prefix{x}{y}{(\outputp{x}{y} | @{y})) | P}} & \nonumber\\
	\red
	& \ldots & \nonumber\\
	\red^*
	& P | P | \ldots & \nonumber
\end{eqnarray}

Of course, this encoding, as an implementation, runs away, unfolding
$\bangp{P}$ eagerly. A lazier and more implementable replication
operator, restricted to input-guarded processes, may be obtained as follows.

\begin{eqnarray}
\bangp{\prefix{u}{v}{P}} 
	:= 
	\binpar{\lift{x}{\prefix{u}{v}{(\binpar{D(x)}{P})}}}{D(x)} \nonumber
\end{eqnarray}

\begin{remark}
  Note that the lazier definition still does not deal with summation
  or mixed summation (i.e. sums over input and output). The reader is
  invited to construct definitions of replication that deal with these
  features. 

  Further, the definitions are parameterized in a name, $x$. Can you,
  gentle reader, make a definition that eliminates this parameter and
  guarantees no accidental interaction between the replication
  machinery and the process being replicated -- i.e. no accidental
  sharing of names used by the process to get its work done and the
  name(s) used by the replication to effect copying. This latter
  revision of the definition of replication is crucial to obtaining
  the expected identity $!!P \sim !P$.
\end{remark}

\begin{remark}\label{rem:paradoxical_combinator}
  The reader familiar with the lambda calculus will have noticed the
  similarity between $D$ and the paradoxical combinator.

  [Ed. note: the existence of this seems to suggest we have to be more
  restrictive on the set of processes and names we admit if we are to
  support no-cloning.]
\end{remark}

\subsubsection{Bisimulation}

The computational dynamics gives rise to another kind of equivalence,
the equivalence of computational behavior. As previously mentioned
this is typically captured \emph{via} some form of bisimulation.

% The notion we use in this paper is weak barbed bisimulation
% \cite{milner91polyadicpi}.

The notion we use in this paper is derived from weak barbed
bisimulation \cite{milner91polyadicpi}. 

\begin{definition}
An \emph{observation relation}, $\downarrow_{\mathcal N}$, over a set
of names, $\mathcal N$, is the smallest relation satisfying the rules
below.

\infrule[Out-barb]{y \in {\mathcal N}, \; x \nameeq y}
		  {\outputp{x}{v} \downarrow_{\mathcal N} x}
\infrule[Par-barb]{\mbox{$P\downarrow_{\mathcal N} x$ or $Q\downarrow_{\mathcal N} x$}}
		  {\binpar{P}{Q} \downarrow_{\mathcal N} x}

We write $P \Downarrow_{\mathcal N} x$ if there is $Q$ such that 
$P \wred Q$ and $Q \downarrow_{\mathcal N} x$.
\end{definition}

\begin{definition}
%\label{def.bbisim}
An  ${\mathcal N}$-\emph{barbed bisimulation} over a set of names, ${\mathcal N}$, is a symmetric binary relation 
${\mathcal S}_{\mathcal N}$ between agents such that $P\rel{S}_{\mathcal N}Q$ implies:
\begin{enumerate}
\item If $P \red P'$ then $Q \wred Q'$ and $P'\rel{S}_{\mathcal N} Q'$.
\item If $P\downarrow_{\mathcal N} x$, then $Q\Downarrow_{\mathcal N} x$.
\end{enumerate}
$P$ is ${\mathcal N}$-barbed bisimilar to $Q$, written
$P \wbbisim_{\mathcal N} Q$, if $P \rel{S}_{\mathcal N} Q$ for some ${\mathcal N}$-barbed bisimulation ${\mathcal S}_{\mathcal N}$.
\end{definition}

$\mathcal{R} \subseteq \pi \times \pi$

$P \mathcal{R} Q => \forall P'. P \red P' \Rightarrow \exists Q'. Q \red Q', P' \mathcal{R} Q'$

$P \vdash x \Rightarrow Q \vdash x$

\begin{mathpar}
  \inferrule*[lab=Out-barb]{x \nameeq y}{{y}!\langle{Q}\rangle \vdash x}
  \and
  \inferrule*[lab=Par-barb]{\mbox{$P\vdash x$ or $Q\vdash x$}}{\binpar{P}{Q} \vdash x}
\end{mathpar}

\subsubsection{Contexts}

One of the principle advantages of computational calculi like the
$\pi$-calculus is a well-defined notion of context,
contextual-equivalence and a correlation between
contextual-equivalence and notions of bisimulation. The notion of
context allows the decomposition of a process into (sub-)process and
its syntactic environment, its context. Thus, a context may be
thought of as a process with a ``hole'' (written $\Box$) in it. The
application of a context $M$ to a process $P$, written $M[P]$, is
tantamount to filling the hole in $M$ with $P$. In this paper we do
not need the full weight of this theory, but do make use of the notion
of context in the proof the main theorem. 

\begin{mathpar}
  \inferrule* [lab=summation] {} {{M_{M},M_{N}} \bc \Box \;|\; x.M_{A} \;|\; M_{M}+M_{N}}
  \and
  \inferrule* [lab=agent] {} {{M_{A}} \bc (\vec{x})M_{P} \;| \; \clift{P_0,\ldots,M_{P},\ldots,P_N}}
  \and \\
  \inferrule* [lab=process] {} {{M_{P}} \bc M_{N} \;| \;P|M_{P} }
\end{mathpar} 

\begin{mathpar}
  \inferrule* [lab=sychronization] {} {M_{N} \bc \Box \;|\; x?M_{F} \;|\; x!M_{C}}
  \and
  \inferrule* [lab=abstraction] {} {{M_{F}} \bc (x)M_{P} }
  \and
  \inferrule* [lab=concretion] {} {{M_{C}} \bc \langle M_{P} \rangle }
  \and \\
  \inferrule* [lab=process] {} {{M_{P}} \bc M_{N} \;| \;P|M_{P} }
\end{mathpar}

\begin{definition}[contextual application] Given a context $M$, and
  process $P$, we define the \emph{contextual application}, $M[P] :=
  M\{P/\Box\}$. That is, the contextual application of M to P is the
  substitution of $P$ for $\Box$ in $M$.
\end{definition}

$\meaningof{-} : L \to \mathcal{P}(\pi)$

\begin{mathpar}
  \inferrule* [lab=collection] {} {\meaningof{true} = \pi, \and \meaningof{~E} = \pi \setminus \meaningof{E}, \and \meaningof{E_{1} \& E_{2}} = \meaningof{E_{1}} \cap \meaningof{E_{2}}}
\end{mathpar}

\begin{mathpar}
  \inferrule* [lab=structure] {} {\meaningof{0} = \{ P \in \pi | P \equiv 0 \}, \and \\ \meaningof{E_1 | E_2} = \{ P \in \pi | P \equiv P_{1} | P_{2}, P_{1} \in \meaningof{E_{1}}, P_{2} \in \meaningof{E_2}\} }
\end{mathpar}

\begin{mathpar}
 \inferrule* [lab=behavior] {} {\meaningof{\langle a?b \rangle E} = \{ P \in \pi | P \equiv Q | u?(y)P', \\ \and \\\\ \and \\ \;\;\; u \in \meaningof{a}, \forall z.P'\{z/y\} \in \meaningof{E\{z/b\}}\}, \and \\ \meaningof{a!E} = \{ P \in \pi | P \equiv Q | x!\langle P' \rangle, x \in \meaningof{a} P' \in \meaningof{E}\} }
\end{mathpar}

\begin{mathpar}
 \inferrule* [lab=nominal] {} {\meaningof{\quotep{E}} = \{ \quotep{P} \in \quotep{\pi} | P \in \meaningof{E} \}, \and \meaningof{\quotep{P}} = \{ \quotep{Q} \in \quotep{\pi} | P \equiv Q \} \and \\ \meaningof{@\quotep{E}} = \{ P \in \pi | P \equiv @x, x \in \meaningof{E} \}}
\end{mathpar}

\begin{eqnarray*}
  \\
  \meaningof{-} : TS \to ST
\end{eqnarray*}

\begin{eqnarray*}
  \\
  L : TS \to ST
\end{eqnarray*}

\begin{eqnarray*}
  \\
  P \models E \iff P \in \meaningof{E}
\end{eqnarray*}

\begin{eqnarray*}
  P \approx_{L} Q \iff \forall E \in L. P \models E \iff Q \models E
\end{eqnarray*}

\begin{eqnarray*}
  P \approx_{K} Q
\end{eqnarray*}

\begin{eqnarray*}
  P \approx Q
\end{eqnarray*}

$\approx_{K} = \approx = \approx_{L}$

\subsubsection{Contextual duality}

Note that contexts extend the quotation operation to a family of
operations from processes to names. Given a context, $M$, we can
define a \emph{nominal context}, $\quotep{M}$ by $\quotep{M}[P] :=
\quotep{M[P]}$. To foreshadow what is to come we observe that these
operations enjoy a duality with processes very much like the duality
between vectors and maps from vectors to scalars.

Further, because the calculus is essentially higher-order, we have a
correspondence between contexts and processes. More specifically,
given a name $x$ and a context $M$ we can construct $M^{*}_{x}$ such
that 

\begin{mathpar}
  M^{*}_{x} | \lift{x}{P} \red M[P]
\end{mathpar}

namely,

\begin{mathpar}
  M^{*}_{x} := x?(u).M[\dropn{u}]
\end{mathpar}

The dependence of $M^{*}_{x}$ on a name makes it an abstraction, 

\begin{mathpar}
  M^{*} := (x)x?(u).M[\dropn{u}]
\end{mathpar}

\subsection{Additional notation}

It will sometimes be convenient to denote the process a name
quotes. We already have the notation $x = \quotep{P}$, but it will be
convenient to introduce an alternate notation, $\procn{x}$, when we
want to emphasize the connection to the use of the name. Note that, by
virtue of name equivalence, $\quotep{\procn{x}} \nameeq x$; so, the
notation is consistent with previous definitions.

Further, because names have structure it is possible to effect
substitutions on the basis of that structure. This means we need to
upgrade our notation for substitutions, which we accomplish by
adapting comprehension notation. Thus,

\begin{mathpar}
  P\{ y / x : x \in S \}
\end{mathpar}

is interpreted to mean the process derived from P by replacing (in a
capture-avoiding manner) each occurrence of $x$ in $S$ by $y$. For example,

\begin{mathpar}
  P\{ \quotep{\procn{x}|\procn{x}} / x : x \in \freenames{P} \}
\end{mathpar}

will replace each (occurrence) of a free name $x$ in $P$ by
$\quotep{\procn{x}|\procn{x}}$.

Also, we will avail ourselves of the notation $x^{L}$ and $x^{R}$ to
denote injections of a name into disjoint copies of the name
space. There are numerous ways to accomplish this. One example can be
found in \cite{MeredithR05}. This notation overloads to vectors of
names: $\vec{x}^{\pi} := (x_{i}^{\pi} \; : \; 0 \leq i < |\vec{x}| )$ where $\pi \in \{L,R\}$.

We also use $P^{\Box} := P|\Box$.

In \cite{MeredithR05} an interpretation of the new operator is
given. It turns out that there are several possible interpretations
all enjoying the requisite algebraic properties of the operator (see
\cite{milner91polyadicpi}). We will therefore make liberal use of
$(\nu\; \vec{x})P$.

% subsection the_syntax_and_semantics_of_the_notation_system (end)   

\input{qm2pi.qmops} 

\input{qm2pi.sterngerlach} 

\input{qm2pi.metric} 

% section concurrent_process_calculi (end)

%\input{qm2pi.proofsketch}

% section proof sketch (end)

%\input{qm2pi.slviaknots} 

% section spatial logic via knots (end)

\input{qm2pi.conclusion}

% section conclusion (end)

%\input{qm2pi.dtcodes} 

% section wiring algorithm (end)

\input{qm2pi.ack} 

% section acknowledgments (end)

\newpage


\bibliographystyle{plain}   
\bibliography{../../biblios/main.bib}

\input{qm2pi.rhodetails}

\end{document}

 

% subsection basic_interpretation (end)

%\input{qm2pi.rho.presentation} 
\subsection{The syntax and semantics of the notation system}\label{sub:the_syntax_and_semantics_of_the_notation_system} % (fold)

We now summarize a technical presentation of the calculus that
embodies our theory of dynamics. The typical presentation of such a
calculus follows the style of giving generators and relations on
them. The grammar, below, describing term constructors, freely
generates the set of processes, $\Proc$. This set is then quotiented
by a relation known as structural congruence and it is over this set
that the notion of dynamics is expressed. This presentation is
essentially that of \cite{MeredithR05} with the addition of
polyadicity and summation. For readability we have relegated some of
the technical subtleties to an appendix.

\subsubsection{Process grammar}\label{subsub:process_grammar}

\begin{mathpar}
  \inferrule* [lab=synchronization] {} {{M} \bc \pzero \;|\; x?F \;|\; x!C }
  \and
  \inferrule* [lab=abstraction] {} {{F} \bc (x)P}
  \and
  \inferrule* [lab=concretion] {} {{C} \bc \langle Q \rangle}
  \and
  \inferrule* [lab=process] {} {{P,Q} \bc M \;| \;P|Q \;|\; @{x}}
  \and
  \inferrule* [lab=name] {} {{x} \bc \quotep{P}}
\end{mathpar} 

Note that $\vec{x}$ (resp. $\vec{P}$) denotes a vector of names
(resp. processes) of length $|\vec{x}|$ (resp. $|\vec{P}|$). We adopt
the following useful abbreviations.

\begin{mathpar}
   x?(\vec{y}).P := x.(\vec{y})P \and  x\clift{\vec{P}} := x.\clift{\vec{P}}
   \and x!(y) := \lift{x}{\dropn{y}}
   \and \Pi_{i=0}^{n-1}P_i := P_0 | \ldots | P_{n-1}
\end{mathpar}

\subsubsection{Structural congruence}

\paragraph{Free and bound names and alpha-equivalence.} At the
core of structural equivalence is alpha-equivalence which identifies
process that are the same up to a change of variable. Formally, we
recognize the distinction between free and bound names. The free names
of a process, $\freenames{P}$, may be calculated recursively as
follows:

\begin{mathpar}
\freenames{\pzero} := \emptyset
  \and \\
  \freenames{x?(y).P} := \{ x \} \cup (\freenames{P} \setminus \{ y \})
  \and 
  \freenames{x!\langle P \rangle} := \{ x \} \cup \{ P \} 
  \and \\
  \freenames{P|Q} := \freenames{P} \cup \freenames{Q}
  \and \\
  \freenames{@{x}} := \{ x \}
\end{mathpar}

$\pi$
$\quotep{\pi}$

$\freenames{-} : \pi \to \mathcal{P}(\quotep{\pi})$

\begin{eqnarray*}
  \freenames{\pzero} & := & \emptyset \\
  \freenames{x?(y).P} & := & \{ x \} \cup (\freenames{P} \setminus \{ y \}) \\
  \freenames{x!\langle P \rangle} & := & \{ x \} \cup \{ P \} \\
  \freenames{P|Q} & := & \freenames{P} \cup \freenames{Q} \\
  \freenames{\dropn{x}} & := & \{ x \}
\end{eqnarray*}

The bound names of a process, $\boundnames{P}$, are those names occurring in $P$
that are not free. For example, in $x?(y).0$, the name $x$ is free, while $y$ is bound.

\begin{mathpar}
  \inferrule* [lab=monoidal-laws] {} { P|Q \equiv Q|P \and P|0 \equiv P \and P|(Q|R) \equiv (P|Q)|R }
\end{mathpar}

\begin{mathpar}
  \inferrule* [lab=alpha-equivalence] {} { (x)P \equiv (y)P\{y/x\} \and y \not\in \freenames{P} }
\end{mathpar}

\begin{definition}
Then two processes, $P,Q$, are alpha-equivalent if $P = Q\{\vec{y}/\vec{x}\}$ for
some $\vec{x} \in \boundnames{Q},\vec{y} \in \boundnames{P}$, where $Q\{\vec{y}/\vec{x}\}$
denotes the capture-avoiding substitution of $\vec{y}$ for $\vec{x}$ in $Q$.
\end{definition}

\begin{definition}
  The {\em structural congruence} \cite{SangiorgiWalker} , $\equiv$,
  between processes is the least congruence containing
  alpha-equivalence, satisfying the abelian monoid laws
  (associativity, commutativity and $\pzero$ as identity) for parallel
  composition $|$ and for summation $+$.
\end{definition}

\subsection{Name equivalence}

We take name equivalence, written $\nameeq$, to be the smallest
equivalence relation generated by the following rules.

\begin{mathpar}
\inferrule*[lab=Quote-drop]
{ }
{ \quotep{@{x}} \nameeq x }

\inferrule*[lab=Struct-equiv]
{ P \scong Q }
{ \quotep{P} \nameeq \quotep{Q} }
\end{mathpar}

The astute reader will have noticed that the mutual recursion of names
and processes imposes a mutual recursion on alpha-equivalence and
structural equivalence via name-equivalence. Fortunately, all of this
works out pleasantly and we may calculate in the natural way, free of
concern. The reader interested in the details is referred to the
appendix \ref{appendix:rho_details}.

\subsection{Substitution}

We use $\Proc$ for the set of processes, $\QProc$ for the set of
names, and $\id{\{}\vec{y} / \vec{x} \id{\}}$ to denote partial maps,
$s : \QProc \rightarrow \QProc$. A map, $s$ lifts, uniquely, to a map
on process terms, $\widehat{s} : \Proc \rightarrow \Proc$ by the
following equations.

\begin{mathpar}
  (0) \psubstp{Q}{P} := 0 \\
  (R \juxtap S) \psubstp{Q}{P}
  :=    
  (R)\psubstp{Q}{P} \juxtap (S) \psubstp{Q}{P} \\
  (x?(y).R) \psubstp{Q}{P}    
  :=    
  (x)\substp{Q}{P} (z)\concat( (R \psubstn{z}{y}) \psubstp{Q}{P} ) \\
  (\lift{x}{R}) \psubstp{Q}{P}  
  :=
  \lift{(x)\substp{Q}{P}}{ R \psubstp{Q}{P} } \\
%   (\dropn{x})  \psubstp{Q}{P}       
%   := 
%   \left\{ 
%     \begin{array}{ccc} 
%       \dropn{\quotep{Q}} & & x \nameeq \quotep{P} \\
%       \dropn{x} & & otherwise \\
%     \end{array}
%   \right. 
  (\dropn{x})  \psubstp{Q}{P}       
  := 
  \left\{ 
    \begin{array}{ccc} 
      Q & & x \nameeq \quotep{P} \\
      \dropn{x} & & otherwise \\
    \end{array}
  \right.
\end{mathpar}
 

where

\begin{eqnarray}
  (x)\id{\{} \lpquote Q \rpquote / \lpquote P \rpquote \id{\}}            = 
  \left\{ 
    \begin{array}{ccc}
      \lpquote Q \rpquote & & x \nameeq \lpquote P \rpquote \\
      x & & otherwise \\
    \end{array}
  \right. \nonumber
\end{eqnarray}

and $z$ is chosen distinct from $\quotep{P}$, $\quotep{Q}$, the free
names in $Q$, and all the names in $R$. Our $\alpha$-equivalence will
be built in the standard way from this substitution.

\begin{remark}\label{rem:no_self_referential_names}
  One consequence of these definitions is that $\forall P. \quotep{P}
  \not\in \freenames{P}$.
\end{remark}

\subsection{ Dynamic quote: an example }

Anticipating something of what's to come, consider applying the
substitution, $\widehat{\id{\{}u / z \id{\}}}$, to the following pair
of processes, $\lift{w}{y!(z)}$ and $w[ \lpquote y!(z) \rpquote ]$.

\begin{eqnarray}
	\lift{w}{y!(z)}\widehat{\id{\{}u / z \id{\}}}
		& = &
		\lift{w}{y!(u)} \nonumber\\
	w[ \lpquote y!(z) \rpquote ] \widehat{ \id{\{}u / z \id{\}} }
		& = &
		w[ \lpquote y!(z) \rpquote ] \nonumber
\end{eqnarray}

Because the body of the process between quotes is impervious to
substitution, we get radically different answers. In fact, by
examining the first process in an input context,
e.g. $x?(z).\lift{w}{y!(z)}$, we see that the process under the lift
operator may be shaped by prefixed inputs binding a name inside it. In
this sense, the lift operator will be seen as a way to dynamically
construct processes before reifying them as names.

Finally equipped with these standard features we can present the
dynamics of the calculus.

\subsubsection{Operational semantics} 

Finally, we introduce the computational dynamics. What marks these
algebras as distinct from other more traditionally studied algebraic
structures, e.g. vector spaces or polynomial rings, is the manner in
which dynamics is captured. In traditional structures, dynamics is typically
expressed through morphisms between such structures, as in linear maps
between vector spaces or morphisms between rings. In algebras
associated with the semantics of computation, the dynamics is
expressed as part of the algebraic structure itself, through a
reduction reduction relation typically denoted by $\red$. Below, we
give a recursive presentation of this relation for the calculus used
in the encoding.

$\red \subseteq \pi \times \pi$
$\red : \pi \to \mathcal{P}(\pi)$

\begin{mathpar}
  \inferrule* [lab=Comm] { \textsf{match}( x_{src}, x_{trgt} ) } { x_{trgt}?(y)P \; | \; x_{src}!\langle {Q} \rangle \red P\{\quotep{Q}/y}\} }
  \and \\
  \inferrule* [lab=Par] {{P} \red {P}'} {{{P} | {Q}} \red {{P}' | {Q}}}
  \and
  \inferrule* [lab=Equiv]{{{P} \scong {P}'} \andalso {{P}' \red {Q}'} \andalso {{Q}' \scong {Q}}}{{P} \red {Q}}
\end{mathpar}

\begin{eqnarray*}
  match_{\equiv} (\quotep{P},\quotep{Q}) & := & P \equiv Q \\
  match_{\dagger}(\quotep{P},\quotep{Q}) & := & \forall R. P|Q \red^{*} R => R \red^{*} 0 \\
  match_{K}(\quotep{P},\quotep{Q}) & := & K \mbox{ for some context } K
\end{eqnarray*}

$u?(x)P | u!\langle Q \rangle \red P\{\quotep{Q}/x\}$

%We write $\wred$ for $\red^*$, and $P\red$ if $\exists Q $ such that $ P \red Q$.
We write $P\red$ if $\exists Q $ such that $ P \red Q$ and $P\not\red$, otherwise.

\section{Replication}

As mentioned before, it is known that replication (and hence
recursion) can be implemented in a higher-order process algebra
\cite{SangiorgiWalker}. As our first example of calculation with the
machinery thus far presented we give the construction explicitly in
the {\rhoc}.

\begin{eqnarray}
	D_{x} & := & \prefix{x}{y}{(\binpar{\outputp{x}{y}}{@{y}})} \nonumber\\
	\bangp_{x}{P} & := & \binpar{{x}!\langle{\binpar{D_{x}}{P}}\rangle}{D_{x}} \nonumber
\end{eqnarray}

\begin{eqnarray}
	\bangp_{x}{P} & & \nonumber\\
	=
	& {x}!\langle{(\prefix{x}{y}{(\outputp{x}{y} | @{y})) | P}}\rangle 
	      | \prefix{x}{y}{(\outputp{x}{y} | @{y})} & \nonumber\\
	\red
	& (\outputp{x}{y} | @{y})\substn{\quotep{(\prefix{x}{y}{(@{y} | \outputp{x}{y})) | P}}}{y} & \nonumber\\
	=
	& \outputp{x}{\quotep{(\prefix{x}{y}{(\outputp{x}{y} | @{y})) | P}}}
	  | {(\prefix{x}{y}{(\outputp{x}{y} | @{y})) | P}} & \nonumber\\
	\red
	& \ldots & \nonumber\\
	\red^*
	& P | P | \ldots & \nonumber
\end{eqnarray}

Of course, this encoding, as an implementation, runs away, unfolding
$\bangp{P}$ eagerly. A lazier and more implementable replication
operator, restricted to input-guarded processes, may be obtained as follows.

\begin{eqnarray}
\bangp{\prefix{u}{v}{P}} 
	:= 
	\binpar{\lift{x}{\prefix{u}{v}{(\binpar{D(x)}{P})}}}{D(x)} \nonumber
\end{eqnarray}

\begin{remark}
  Note that the lazier definition still does not deal with summation
  or mixed summation (i.e. sums over input and output). The reader is
  invited to construct definitions of replication that deal with these
  features. 

  Further, the definitions are parameterized in a name, $x$. Can you,
  gentle reader, make a definition that eliminates this parameter and
  guarantees no accidental interaction between the replication
  machinery and the process being replicated -- i.e. no accidental
  sharing of names used by the process to get its work done and the
  name(s) used by the replication to effect copying. This latter
  revision of the definition of replication is crucial to obtaining
  the expected identity $!!P \sim !P$.
\end{remark}

\begin{remark}\label{rem:paradoxical_combinator}
  The reader familiar with the lambda calculus will have noticed the
  similarity between $D$ and the paradoxical combinator.

  [Ed. note: the existence of this seems to suggest we have to be more
  restrictive on the set of processes and names we admit if we are to
  support no-cloning.]
\end{remark}

\subsubsection{Bisimulation}

The computational dynamics gives rise to another kind of equivalence,
the equivalence of computational behavior. As previously mentioned
this is typically captured \emph{via} some form of bisimulation.

% The notion we use in this paper is weak barbed bisimulation
% \cite{milner91polyadicpi}.

The notion we use in this paper is derived from weak barbed
bisimulation \cite{milner91polyadicpi}. 

\begin{definition}
An \emph{observation relation}, $\downarrow_{\mathcal N}$, over a set
of names, $\mathcal N$, is the smallest relation satisfying the rules
below.

\infrule[Out-barb]{y \in {\mathcal N}, \; x \nameeq y}
		  {\outputp{x}{v} \downarrow_{\mathcal N} x}
\infrule[Par-barb]{\mbox{$P\downarrow_{\mathcal N} x$ or $Q\downarrow_{\mathcal N} x$}}
		  {\binpar{P}{Q} \downarrow_{\mathcal N} x}

We write $P \Downarrow_{\mathcal N} x$ if there is $Q$ such that 
$P \wred Q$ and $Q \downarrow_{\mathcal N} x$.
\end{definition}

\begin{definition}
%\label{def.bbisim}
An  ${\mathcal N}$-\emph{barbed bisimulation} over a set of names, ${\mathcal N}$, is a symmetric binary relation 
${\mathcal S}_{\mathcal N}$ between agents such that $P\rel{S}_{\mathcal N}Q$ implies:
\begin{enumerate}
\item If $P \red P'$ then $Q \wred Q'$ and $P'\rel{S}_{\mathcal N} Q'$.
\item If $P\downarrow_{\mathcal N} x$, then $Q\Downarrow_{\mathcal N} x$.
\end{enumerate}
$P$ is ${\mathcal N}$-barbed bisimilar to $Q$, written
$P \wbbisim_{\mathcal N} Q$, if $P \rel{S}_{\mathcal N} Q$ for some ${\mathcal N}$-barbed bisimulation ${\mathcal S}_{\mathcal N}$.
\end{definition}

$\mathcal{R} \subseteq \pi \times \pi$

$P \mathcal{R} Q => \forall P'. P \red P' \Rightarrow \exists Q'. Q \red Q', P' \mathcal{R} Q'$

$P \vdash x \Rightarrow Q \vdash x$

\begin{mathpar}
  \inferrule*[lab=Out-barb]{x \nameeq y}{{y}!\langle{Q}\rangle \vdash x}
  \and
  \inferrule*[lab=Par-barb]{\mbox{$P\vdash x$ or $Q\vdash x$}}{\binpar{P}{Q} \vdash x}
\end{mathpar}

\subsubsection{Contexts}

One of the principle advantages of computational calculi like the
$\pi$-calculus is a well-defined notion of context,
contextual-equivalence and a correlation between
contextual-equivalence and notions of bisimulation. The notion of
context allows the decomposition of a process into (sub-)process and
its syntactic environment, its context. Thus, a context may be
thought of as a process with a ``hole'' (written $\Box$) in it. The
application of a context $M$ to a process $P$, written $M[P]$, is
tantamount to filling the hole in $M$ with $P$. In this paper we do
not need the full weight of this theory, but do make use of the notion
of context in the proof the main theorem. 

\begin{mathpar}
  \inferrule* [lab=summation] {} {{M_{M},M_{N}} \bc \Box \;|\; x.M_{A} \;|\; M_{M}+M_{N}}
  \and
  \inferrule* [lab=agent] {} {{M_{A}} \bc (\vec{x})M_{P} \;| \; \clift{P_0,\ldots,M_{P},\ldots,P_N}}
  \and \\
  \inferrule* [lab=process] {} {{M_{P}} \bc M_{N} \;| \;P|M_{P} }
\end{mathpar} 

\begin{mathpar}
  \inferrule* [lab=sychronization] {} {M_{N} \bc \Box \;|\; x?M_{F} \;|\; x!M_{C}}
  \and
  \inferrule* [lab=abstraction] {} {{M_{F}} \bc (x)M_{P} }
  \and
  \inferrule* [lab=concretion] {} {{M_{C}} \bc \langle M_{P} \rangle }
  \and \\
  \inferrule* [lab=process] {} {{M_{P}} \bc M_{N} \;| \;P|M_{P} }
\end{mathpar}

\begin{definition}[contextual application] Given a context $M$, and
  process $P$, we define the \emph{contextual application}, $M[P] :=
  M\{P/\Box\}$. That is, the contextual application of M to P is the
  substitution of $P$ for $\Box$ in $M$.
\end{definition}

$\meaningof{-} : L \to \mathcal{P}(\pi)$

\begin{mathpar}
  \inferrule* [lab=collection] {} {\meaningof{true} = \pi, \and \meaningof{~E} = \pi \setminus \meaningof{E}, \and \meaningof{E_{1} \& E_{2}} = \meaningof{E_{1}} \cap \meaningof{E_{2}}}
\end{mathpar}

\begin{mathpar}
  \inferrule* [lab=structure] {} {\meaningof{0} = \{ P \in \pi | P \equiv 0 \}, \and \\ \meaningof{E_1 | E_2} = \{ P \in \pi | P \equiv P_{1} | P_{2}, P_{1} \in \meaningof{E_{1}}, P_{2} \in \meaningof{E_2}\} }
\end{mathpar}

\begin{mathpar}
 \inferrule* [lab=behavior] {} {\meaningof{\langle a?b \rangle E} = \{ P \in \pi | P \equiv Q | u?(y)P', \\ \and \\\\ \and \\ \;\;\; u \in \meaningof{a}, \forall z.P'\{z/y\} \in \meaningof{E\{z/b\}}\}, \and \\ \meaningof{a!E} = \{ P \in \pi | P \equiv Q | x!\langle P' \rangle, x \in \meaningof{a} P' \in \meaningof{E}\} }
\end{mathpar}

\begin{mathpar}
 \inferrule* [lab=nominal] {} {\meaningof{\quotep{E}} = \{ \quotep{P} \in \quotep{\pi} | P \in \meaningof{E} \}, \and \meaningof{\quotep{P}} = \{ \quotep{Q} \in \quotep{\pi} | P \equiv Q \} \and \\ \meaningof{@\quotep{E}} = \{ P \in \pi | P \equiv @x, x \in \meaningof{E} \}}
\end{mathpar}

\begin{eqnarray*}
  \\
  \meaningof{-} : TS \to ST
\end{eqnarray*}

\begin{eqnarray*}
  \\
  L : TS \to ST
\end{eqnarray*}

\begin{eqnarray*}
  \\
  P \models E \iff P \in \meaningof{E}
\end{eqnarray*}

\begin{eqnarray*}
  P \approx_{L} Q \iff \forall E \in L. P \models E \iff Q \models E
\end{eqnarray*}

\begin{eqnarray*}
  P \approx_{K} Q
\end{eqnarray*}

\begin{eqnarray*}
  P \approx Q
\end{eqnarray*}

$\approx_{K} = \approx = \approx_{L}$

\subsubsection{Contextual duality}

Note that contexts extend the quotation operation to a family of
operations from processes to names. Given a context, $M$, we can
define a \emph{nominal context}, $\quotep{M}$ by $\quotep{M}[P] :=
\quotep{M[P]}$. To foreshadow what is to come we observe that these
operations enjoy a duality with processes very much like the duality
between vectors and maps from vectors to scalars.

Further, because the calculus is essentially higher-order, we have a
correspondence between contexts and processes. More specifically,
given a name $x$ and a context $M$ we can construct $M^{*}_{x}$ such
that 

\begin{mathpar}
  M^{*}_{x} | \lift{x}{P} \red M[P]
\end{mathpar}

namely,

\begin{mathpar}
  M^{*}_{x} := x?(u).M[\dropn{u}]
\end{mathpar}

The dependence of $M^{*}_{x}$ on a name makes it an abstraction, 

\begin{mathpar}
  M^{*} := (x)x?(u).M[\dropn{u}]
\end{mathpar}

\subsection{Additional notation}

It will sometimes be convenient to denote the process a name
quotes. We already have the notation $x = \quotep{P}$, but it will be
convenient to introduce an alternate notation, $\procn{x}$, when we
want to emphasize the connection to the use of the name. Note that, by
virtue of name equivalence, $\quotep{\procn{x}} \nameeq x$; so, the
notation is consistent with previous definitions.

Further, because names have structure it is possible to effect
substitutions on the basis of that structure. This means we need to
upgrade our notation for substitutions, which we accomplish by
adapting comprehension notation. Thus,

\begin{mathpar}
  P\{ y / x : x \in S \}
\end{mathpar}

is interpreted to mean the process derived from P by replacing (in a
capture-avoiding manner) each occurrence of $x$ in $S$ by $y$. For example,

\begin{mathpar}
  P\{ \quotep{\procn{x}|\procn{x}} / x : x \in \freenames{P} \}
\end{mathpar}

will replace each (occurrence) of a free name $x$ in $P$ by
$\quotep{\procn{x}|\procn{x}}$.

Also, we will avail ourselves of the notation $x^{L}$ and $x^{R}$ to
denote injections of a name into disjoint copies of the name
space. There are numerous ways to accomplish this. One example can be
found in \cite{MeredithR05}. This notation overloads to vectors of
names: $\vec{x}^{\pi} := (x_{i}^{\pi} \; : \; 0 \leq i < |\vec{x}| )$ where $\pi \in \{L,R\}$.

We also use $P^{\Box} := P|\Box$.

In \cite{MeredithR05} an interpretation of the new operator is
given. It turns out that there are several possible interpretations
all enjoying the requisite algebraic properties of the operator (see
\cite{milner91polyadicpi}). We will therefore make liberal use of
$(\nu\; \vec{x})P$.

% subsection the_syntax_and_semantics_of_the_notation_system (end)   

\section{Interpretation of QM}
\subsection{Supporting definitions}
\subsubsection{Multiplication}
\begin{mathpar}
  \quotep{Q} \cdot \quotep{R} := \quotep{Q|R}
  \and \\
  \quotep{Q} \cdot P := P\{ \quotep{Q|R} / \quotep{R} : \quotep{R} \in \freenames{P} \}
\end{mathpar}

\paragraph{Discussion}
The first line needs little explanation. The second line says that
each free name of the process is replaced with the multiplication of
that name by the scalar. Multiplication of a scalar (name) by a state
(process) results in a process all the names of which have been `moved
over' by parallel composition with the process the scalar
quotes. There is a subtlety that the bound names have to be
manipulated so that multiplied names aren't accidentally
captured. There are many ways to achieve this.

\begin{remark}\label{rem:multiplication_identities}
  The reader is invited to verify that for all $x,y,z \in \QProc$ and $P \in \Proc$
  \begin{mathpar}
    x \cdot \quotep{0} \equiv x 
    \and
    x \cdot y \equiv y \cdot x
    \and
    x \cdot (y \cdot z) \equiv (x \cdot y) \cdot z
    \and \\
    \quotep{0} \cdot P \equiv P
    \and \\
    x \cdot (y \cdot P) \equiv (x \cdot y) \cdot P
    \and \\
    x \cdot (P|Q) \equiv (x \cdot P) | (x \cdot Q)
    \and \\    
  \end{mathpar}
\end{remark}

\subsubsection{Tensor product}

We define a tensor product on processes by structural induction.

\paragraph{Tensor of sums} First note that all summations, including
$\pzero$ and sequence, can be written $\Sigma_{i} x_{i}.A_{i} +
\Sigma_{j} x_{j}.C_{j}$, where we have grouped input-guarded processes
together and output-guarded processes together.

Thus, we can define the tensor product of two summations, $N_{1}\otimes N_{2}$, where

\begin{mathpar}
  N_{1} := \Sigma_{i} x_{i}.A_{i} + \Sigma_{j} x_{j}.C_{j}
  \and
  N_{2} := \Sigma_{i'} y_{i'}.B_{i'} + \Sigma_{j'} y_{j'}.D_{j'} 
\end{mathpar}

as follows.

\begin{mathpar}
  \Sigma_{i} x_{i}.A_{i} + \Sigma_{j} x_{j}.C_{j} \otimes \Sigma_{i'}
  y_{i'}.B_{i'} + \Sigma_{j'} y_{j'}.D_{j'} 
  \and \\
  := \; \Sigma_{i} \Sigma_{i'} \quotep{\stackrel{\vee}{x_{i}}| \stackrel{\vee}{y_{i'}}}.(A_{i}\otimes B_{i'}) \; | \; \Sigma_{i'} \Sigma_{i} \quotep{\stackrel{\vee}{y_{i'}}|\stackrel{\vee}{x_{i}}}.(B_{i'}\otimes A_{i})
  \and
  \;\; | \;\; \Sigma_{j} \Sigma_{j'} \quotep{\stackrel{\vee}{x_{j}}|\stackrel{\vee}{y_{j'}}}.(A_{j}\otimes B_{j'}) \; | \; \Sigma_{j'} \Sigma_{j} \quotep{\stackrel{\vee}{y_{j'}}|\stackrel{\vee}{x_{j}}}.(B_{j'}\otimes A_{j})
\end{mathpar}

\begin{remark}
  Do we need to $x^{L}$ and $y^{R}$ for this construction as well?
\end{remark}

\paragraph{Tensor of parallel compositions} Next, we distribute tensor
over par.

\begin{mathpar}
  P_{1}|P_{2} \otimes Q_{1}|Q_{2} := (P_{1} \otimes Q_{1}) | (P_{1}
  \otimes Q_{2}) | (P_{2} \otimes Q_{1}) | (P_{2} \otimes Q_{2})
\end{mathpar}

\paragraph{Tensor with dropped names} We treat tensor of a
process with a dropped name as parallel composition.

\begin{mathpar}
  P \otimes \dropn{x} := P | \dropn{x}
\end{mathpar}

\paragraph{Tensor of agents}

Finally, we need to define tensor on agents. Note that the definition
of tensor on normal products only tensors inputs with inputs and
outputs with outputs. Thus, we only have to define the operation on
``homogeneous'' pairings.

\begin{mathpar}
  (\vec{x})P \otimes (\vec{y})Q
  \and \\
  := (x_{0}^{L}|y_{0}^{R},\ldots,x_{0}^{L}|y_{n}^{R},\ldots,x_{m}^{L}|y_{0}^{R},\ldots,x_{m}^{L}|y_{n}^R)(P\{ \vec{x}^{L}/\vec{x}\} \otimes Q \{ \vec{y}^{R}/\vec{y}\})
  \and \\
  \clift{\vec{P}} \otimes \clift{\vec{Q}}
  \and \\
  := \clift{P_{0}\otimes Q_{0},\ldots,P_{0}\otimes Q_{n},\ldots,P_{m}\otimes Q_{0},\ldots,P_{m}\otimes Q_{n}}
\end{mathpar}

\begin{remark}
  Observe that arities of tensored abstractions matches arities of
  tensored concretions if the original arities matched. Note also that
  the length of the arities corresponds to the increase in dimension
  we see in ordinary vector space tensor product.
\end{remark}

\begin{remark}
  Operationally, this definition distributes the tensor down to
  components ``linked'' by summation. Tensor over summation is
  intriguing in that it mixes names. Moreover, as a consequence of the
  way it mixes names we have the identities for all $x \in \QProc$ and
  $P,Q \in \Proc$

  \begin{mathpar}
    (x \cdot P) \otimes Q \equiv x \cdot (P \otimes Q) \equiv P \otimes (x \cdot Q)
    \and
    P \otimes \pzero \equiv P
  \end{mathpar}

  that the reader is invited to verify.
\end{remark}

\subsubsection{Annihilation}
\begin{mathpar}
  P^{\perp} := \{ Q | \forall R. P|Q \red^{*} R \Rightarrow R \red^{*} \pzero \}
  \and \\
  P^{\underline{\perp}} := \Sigma_{Q \in P^{\perp}} \quotep{Q}?(y).(\dropn{y}|Q) | \Sigma_{Q \in P^{\perp}} \quotep{Q}\clift{\Box}
\end{mathpar}

\paragraph{Discussion} The reader will note that $P^{\perp}$ is a
\emph{set} of processes, while $P^{\underline{\perp}}$ is a
\emph{context}. We call the set $P^{\perp}$ the \emph{annihilators} of
$P$. The parallel composition of a process in the annihilators of $P$
with $P$ will result in a process, the state space of which has all
paths eventually leading to $\pzero$. Execution may endure loops; but
under reasonable conditions of fairness (naturally guaranteed under
most notions of bisimulation) such a composite process cannot get
stuck in such a loop and will, eventually pop out and terminate.

The context $P^{\underline{\perp}}$ is ready and willing to ``take the
$P$ out of'' the process to which it is applied. It will effectively
transmit the code of the process to which it is applied to one of the
annihilators and run the process against it.

\subsubsection{Evaluation}
We fix $M$ a domain of fully abstract interpretation with an equality
coincident with bisimulation. We take $\meaningof{\cdot} : \Proc \to
M$ to be the map interpreting processes and $\nmeaningof{\cdot} : \M
\to Proc$ to be the map running the other way. Then we define

\begin{mathpar}
  \int P := \nmeaningof{\meaningof{P}}
\end{mathpar}

\paragraph{Discussion}
There are many fully abstract interpretations of Milner's
$\pi$-calculus. Any of them can be used as a basis for interpreting
the reflective calculus here. Equipped with such a domain it is
largely a matter of grinding through to check that the Yoneda
construction for the normalization-by-evaluation program can be
extended to this setting.

\begin{remark}
  The reader is invited to verify that $\int (P^{\underline{\perp}}[P]) = 0$.
\end{remark}

\subsection{Quantum mechanics}

Table \ref{tbl:core_qm_op_defns} gives the core operational definitions

\begin{table}[htp]\label{tbl:core_qm_op_defns}
  \center{
    \fbox{
      \begin{tabular}{c|c}
        quantum mechanics & process calculus \\
        \hline
        scalar & $x := \quotep{P}$ \\
        state vector & $\state{P} := P$ \\
        dual & $\state{P}^{*} := \event{P^{\underline{\perp}}} := \quotep{P^{\underline{\perp}}}[-]$ \\
        matrix & $ \Sigma_{\alpha} \state{P_{\alpha}}x_{\alpha}\event{Q_{\alpha}}$ \\
        vector addition & $\state{P} + \state{Q} := \state{P | Q}$ \\
        tensor product & $\state{P} \otimes \state{Q} := \state{P \otimes Q}$ \\
        inner product & $\innerprod{P}{Q} := \quotep{\int P^{\underline{\perp}}[Q]}$ \\
      \end{tabular}
    }
  }
  \caption{QM - operational definitions}
\end{table}

where

\begin{mathpar}
  \prmatrix{P}{Q} := \fprmatrix{P}{\quotep{\pzero}}{Q}
  \and
  \fprmatrix{P}{x}{Q} := (\state{P},x,\event{Q})
  \and
  (\fprmatrix{P}{x}{Q})(\state{R}) := x \cdot \innerprod{Q}{R} \cdot \state{P}
  \and
  (\fprmatrix{P}{x}{Q})(\event{R}) := x \cdot \innerprod{R}{P} \cdot \event{Q}
\end{mathpar}

\paragraph{Discussion}
As promised: vectors (aka states) are represented as processes; duals
as contextual duals; inner product definition should be compared with
standard inner product definition for ....

\begin{remark}
  Assuming $\int (P^{\underline{\perp}}[P]) = 0$, the reader is
  invited to verify that $(\fprmatrix{P}{x}{P})(\state{P}) = x \cdot \state{P}$.
\end{remark}

\begin{remark}
  The reader is invited to verify that $\innerprod{P}{Q}$ could
  equally well have been written $\quotep{\int \stackrel{\vee}{x}}$
  where $x = \event{P^{\underline{\perp}}}(Q)$.

  One of the motivations for this remark is that there is another way
  to factor these operations. We could package up evaluation in the dual:

  \begin{mathpar}
    \state{P}^{*} := \event{\int P^{\underline{\perp}}} := \quotep{\int P^{\underline{\perp}}}[-]
  \end{mathpar}

  and then have inner product defined by
  
  \begin{mathpar}
    \innerprod{P}{Q} := \event{P}(Q)
  \end{mathpar}

  Hopefully, experience with the calculations will provide guidance on
  the best factoring.
\end{remark}

\begin{remark}
  Assuming $\int (P^{\underline{\perp}}[P]) = 0$, the reader is
  invited to verify that $\forall P,Q. (\prmatrix{0}{Q})(\state{0}) =
  \state{0}$ and dually $(\prmatrix{P}{0})(\event{0}) = \event{0}$.
\end{remark}

\begin{remark}
  i'm a little worried that i don't (yet) have proper support for
  complex conjugacy. But, the observation above may give us a
  clue. According to Abramsky, it must be the case that the scalars
  are iso to the homset of the identity for the tensor -- which the
  observation above characterizes. 

  For now, we will simply bookmark the notion with $\overline{x}$.
\end{remark}

\subsubsection{Adjointness}

We need to give a definition of $(\cdot)^{\dagger}$ for matrices. The
obvious candidate definition is
\begin{mathpar}
(\Sigma_{\alpha}\fprmatrix{P_{\alpha}}{x_{\alpha}}{Q_{\alpha}})^{\dagger}
= \Sigma_{\alpha}\fprmatrix{(Q_{\alpha}^{\underline{\perp}})^{*}}{\overline{x}_{\alpha}}{P_{\alpha}^{\underline{\perp}}} 
\end{mathpar}

But, $(Q_{\alpha}^{\underline{\perp}})^{*}$ requires a name along
which to communicate the process to achieve the context application.

\subsubsection{Basis for a basis}
If processes label states and ``addition'' of states (a.k.a. vector
addition) is interpreted as parallel composition, what corresponds to
notions of linear independence and basis? Here, we recall that Yoshida
has developed a set of \emph{combinators} for an asynchronous verison
of Milner's $\pi$-calculus. These are a finite set of processes such
any process can be expressed as parallel composition of these
combinators together with liberal uses of the new operator and
replication. We can simply give a translation of these into the
present calculus and have reasonable expectation that the property
carries over. That is, that the resultant set allows to express all
processes via parallel composition. Note, however, that there is no
new operator or replication in this calculus. As a result, we expect
that the corresponding set is actually infinite. That is, we expect
that the space is actually infinite dimensional.

\begin{remark}
  The attentive reader may be a bit concerned. Certainly, the
  collection $S$, $K$ and $I$ is a finite set of
  combinators. Shouldn't we expect to see a finite set of combinators
  for an effectively equivalent system? i am very sympathetic to this
  critique and feel it warrants full attention. On the other hand, i
  also have in mind the following analogy. The natural numbers, as a
  monoid under addition, has exactly $1$ generator, while the natural
  numbers, as a monoid under multiplication, has countably many
  generators (the primes). We observe that the application of the
  lambda calculus is much less resource sensitive than the parallel
  composition of the $\pi$-calculus. Could it be the case that we have
  an analogy of the form
  
  \begin{mathpar}
    m + n : MN :: m*n : M|N
  \end{mathpar}

  giving a similar blow up in the set of ``primes''?  This is such a
  wonderful thought that, even if it's not true, i think it's worth
  writing down.
\end{remark}
 

\documentclass[12pt]{llncs}
%\documentclass{jktr}

\usepackage[pdftex]{hyperref}                   
\usepackage {listings}
\usepackage {mathpartir}
\usepackage{bcprules}
%\usepackage{listings}
                       
\usepackage{graphicx} 
%\usepackage[margins=2.5cm,nohead,nofoot]{geometry}
%\usepackage{geometry}
\usepackage{amsfonts}
\usepackage{amstext}
\usepackage{latexsym}
\usepackage{amssymb}
\usepackage{color}


%\include{myPreamble}
\include{qm2pi.local} 

%\ifpdf
%\usepackage[pdftex]{graphicx}
%\else
%\usepackage{graphicx}
%\fi

 % \ifpdf
%  \usepackage{pdfsync}
%  \if


%\title{Brief Article}
%\author{David F. Snyder}
%\author{L.G. Meredith}

%\address{Dept. of Math., Texas State University--San Marcos, San Marcos, TX 78666}
       
\pagestyle{empty}


\begin{document}

\lstset{language=[Objective]Caml,frame=shadowbox}

\input{qm2pi.front}

% section front matter (end)

\input{qm2pi.intro} 
 
% section introduction (end)

% \input{qm2pi.knotations} 

% section notation (end)

\input{qm2pi.process.calculi} 

% section concurrent_process_calculi_and_spatial_logics_ (end)
    
%\input{qm2pi.knots2pi} 

%\input{qm2pi.trefoil} 

%\input{qm2pi.mainthm} 

% subsection basic_interpretation (end)

%\input{qm2pi.rho.presentation} 
\subsection{The syntax and semantics of the notation system}\label{sub:the_syntax_and_semantics_of_the_notation_system} % (fold)

We now summarize a technical presentation of the calculus that
embodies our theory of dynamics. The typical presentation of such a
calculus follows the style of giving generators and relations on
them. The grammar, below, describing term constructors, freely
generates the set of processes, $\Proc$. This set is then quotiented
by a relation known as structural congruence and it is over this set
that the notion of dynamics is expressed. This presentation is
essentially that of \cite{MeredithR05} with the addition of
polyadicity and summation. For readability we have relegated some of
the technical subtleties to an appendix.

\subsubsection{Process grammar}\label{subsub:process_grammar}

\begin{mathpar}
  \inferrule* [lab=synchronization] {} {{M} \bc \pzero \;|\; x?F \;|\; x!C }
  \and
  \inferrule* [lab=abstraction] {} {{F} \bc (x)P}
  \and
  \inferrule* [lab=concretion] {} {{C} \bc \langle Q \rangle}
  \and
  \inferrule* [lab=process] {} {{P,Q} \bc M \;| \;P|Q \;|\; @{x}}
  \and
  \inferrule* [lab=name] {} {{x} \bc \quotep{P}}
\end{mathpar} 

Note that $\vec{x}$ (resp. $\vec{P}$) denotes a vector of names
(resp. processes) of length $|\vec{x}|$ (resp. $|\vec{P}|$). We adopt
the following useful abbreviations.

\begin{mathpar}
   x?(\vec{y}).P := x.(\vec{y})P \and  x\clift{\vec{P}} := x.\clift{\vec{P}}
   \and x!(y) := \lift{x}{\dropn{y}}
   \and \Pi_{i=0}^{n-1}P_i := P_0 | \ldots | P_{n-1}
\end{mathpar}

\subsubsection{Structural congruence}

\paragraph{Free and bound names and alpha-equivalence.} At the
core of structural equivalence is alpha-equivalence which identifies
process that are the same up to a change of variable. Formally, we
recognize the distinction between free and bound names. The free names
of a process, $\freenames{P}$, may be calculated recursively as
follows:

\begin{mathpar}
\freenames{\pzero} := \emptyset
  \and \\
  \freenames{x?(y).P} := \{ x \} \cup (\freenames{P} \setminus \{ y \})
  \and 
  \freenames{x!\langle P \rangle} := \{ x \} \cup \{ P \} 
  \and \\
  \freenames{P|Q} := \freenames{P} \cup \freenames{Q}
  \and \\
  \freenames{@{x}} := \{ x \}
\end{mathpar}

$\pi$
$\quotep{\pi}$

$\freenames{-} : \pi \to \mathcal{P}(\quotep{\pi})$

\begin{eqnarray*}
  \freenames{\pzero} & := & \emptyset \\
  \freenames{x?(y).P} & := & \{ x \} \cup (\freenames{P} \setminus \{ y \}) \\
  \freenames{x!\langle P \rangle} & := & \{ x \} \cup \{ P \} \\
  \freenames{P|Q} & := & \freenames{P} \cup \freenames{Q} \\
  \freenames{\dropn{x}} & := & \{ x \}
\end{eqnarray*}

The bound names of a process, $\boundnames{P}$, are those names occurring in $P$
that are not free. For example, in $x?(y).0$, the name $x$ is free, while $y$ is bound.

\begin{mathpar}
  \inferrule* [lab=monoidal-laws] {} { P|Q \equiv Q|P \and P|0 \equiv P \and P|(Q|R) \equiv (P|Q)|R }
\end{mathpar}

\begin{mathpar}
  \inferrule* [lab=alpha-equivalence] {} { (x)P \equiv (y)P\{y/x\} \and y \not\in \freenames{P} }
\end{mathpar}

\begin{definition}
Then two processes, $P,Q$, are alpha-equivalent if $P = Q\{\vec{y}/\vec{x}\}$ for
some $\vec{x} \in \boundnames{Q},\vec{y} \in \boundnames{P}$, where $Q\{\vec{y}/\vec{x}\}$
denotes the capture-avoiding substitution of $\vec{y}$ for $\vec{x}$ in $Q$.
\end{definition}

\begin{definition}
  The {\em structural congruence} \cite{SangiorgiWalker} , $\equiv$,
  between processes is the least congruence containing
  alpha-equivalence, satisfying the abelian monoid laws
  (associativity, commutativity and $\pzero$ as identity) for parallel
  composition $|$ and for summation $+$.
\end{definition}

\subsection{Name equivalence}

We take name equivalence, written $\nameeq$, to be the smallest
equivalence relation generated by the following rules.

\begin{mathpar}
\inferrule*[lab=Quote-drop]
{ }
{ \quotep{@{x}} \nameeq x }

\inferrule*[lab=Struct-equiv]
{ P \scong Q }
{ \quotep{P} \nameeq \quotep{Q} }
\end{mathpar}

The astute reader will have noticed that the mutual recursion of names
and processes imposes a mutual recursion on alpha-equivalence and
structural equivalence via name-equivalence. Fortunately, all of this
works out pleasantly and we may calculate in the natural way, free of
concern. The reader interested in the details is referred to the
appendix \ref{appendix:rho_details}.

\subsection{Substitution}

We use $\Proc$ for the set of processes, $\QProc$ for the set of
names, and $\id{\{}\vec{y} / \vec{x} \id{\}}$ to denote partial maps,
$s : \QProc \rightarrow \QProc$. A map, $s$ lifts, uniquely, to a map
on process terms, $\widehat{s} : \Proc \rightarrow \Proc$ by the
following equations.

\begin{mathpar}
  (0) \psubstp{Q}{P} := 0 \\
  (R \juxtap S) \psubstp{Q}{P}
  :=    
  (R)\psubstp{Q}{P} \juxtap (S) \psubstp{Q}{P} \\
  (x?(y).R) \psubstp{Q}{P}    
  :=    
  (x)\substp{Q}{P} (z)\concat( (R \psubstn{z}{y}) \psubstp{Q}{P} ) \\
  (\lift{x}{R}) \psubstp{Q}{P}  
  :=
  \lift{(x)\substp{Q}{P}}{ R \psubstp{Q}{P} } \\
%   (\dropn{x})  \psubstp{Q}{P}       
%   := 
%   \left\{ 
%     \begin{array}{ccc} 
%       \dropn{\quotep{Q}} & & x \nameeq \quotep{P} \\
%       \dropn{x} & & otherwise \\
%     \end{array}
%   \right. 
  (\dropn{x})  \psubstp{Q}{P}       
  := 
  \left\{ 
    \begin{array}{ccc} 
      Q & & x \nameeq \quotep{P} \\
      \dropn{x} & & otherwise \\
    \end{array}
  \right.
\end{mathpar}
 

where

\begin{eqnarray}
  (x)\id{\{} \lpquote Q \rpquote / \lpquote P \rpquote \id{\}}            = 
  \left\{ 
    \begin{array}{ccc}
      \lpquote Q \rpquote & & x \nameeq \lpquote P \rpquote \\
      x & & otherwise \\
    \end{array}
  \right. \nonumber
\end{eqnarray}

and $z$ is chosen distinct from $\quotep{P}$, $\quotep{Q}$, the free
names in $Q$, and all the names in $R$. Our $\alpha$-equivalence will
be built in the standard way from this substitution.

\begin{remark}\label{rem:no_self_referential_names}
  One consequence of these definitions is that $\forall P. \quotep{P}
  \not\in \freenames{P}$.
\end{remark}

\subsection{ Dynamic quote: an example }

Anticipating something of what's to come, consider applying the
substitution, $\widehat{\id{\{}u / z \id{\}}}$, to the following pair
of processes, $\lift{w}{y!(z)}$ and $w[ \lpquote y!(z) \rpquote ]$.

\begin{eqnarray}
	\lift{w}{y!(z)}\widehat{\id{\{}u / z \id{\}}}
		& = &
		\lift{w}{y!(u)} \nonumber\\
	w[ \lpquote y!(z) \rpquote ] \widehat{ \id{\{}u / z \id{\}} }
		& = &
		w[ \lpquote y!(z) \rpquote ] \nonumber
\end{eqnarray}

Because the body of the process between quotes is impervious to
substitution, we get radically different answers. In fact, by
examining the first process in an input context,
e.g. $x?(z).\lift{w}{y!(z)}$, we see that the process under the lift
operator may be shaped by prefixed inputs binding a name inside it. In
this sense, the lift operator will be seen as a way to dynamically
construct processes before reifying them as names.

Finally equipped with these standard features we can present the
dynamics of the calculus.

\subsubsection{Operational semantics} 

Finally, we introduce the computational dynamics. What marks these
algebras as distinct from other more traditionally studied algebraic
structures, e.g. vector spaces or polynomial rings, is the manner in
which dynamics is captured. In traditional structures, dynamics is typically
expressed through morphisms between such structures, as in linear maps
between vector spaces or morphisms between rings. In algebras
associated with the semantics of computation, the dynamics is
expressed as part of the algebraic structure itself, through a
reduction reduction relation typically denoted by $\red$. Below, we
give a recursive presentation of this relation for the calculus used
in the encoding.

$\red \subseteq \pi \times \pi$
$\red : \pi \to \mathcal{P}(\pi)$

\begin{mathpar}
  \inferrule* [lab=Comm] { \textsf{match}( x_{src}, x_{trgt} ) } { x_{trgt}?(y)P \; | \; x_{src}!\langle {Q} \rangle \red P\{\quotep{Q}/y}\} }
  \and \\
  \inferrule* [lab=Par] {{P} \red {P}'} {{{P} | {Q}} \red {{P}' | {Q}}}
  \and
  \inferrule* [lab=Equiv]{{{P} \scong {P}'} \andalso {{P}' \red {Q}'} \andalso {{Q}' \scong {Q}}}{{P} \red {Q}}
\end{mathpar}

\begin{eqnarray*}
  match_{\equiv} (\quotep{P},\quotep{Q}) & := & P \equiv Q \\
  match_{\dagger}(\quotep{P},\quotep{Q}) & := & \forall R. P|Q \red^{*} R => R \red^{*} 0 \\
  match_{K}(\quotep{P},\quotep{Q}) & := & K \mbox{ for some context } K
\end{eqnarray*}

$u?(x)P | u!\langle Q \rangle \red P\{\quotep{Q}/x\}$

%We write $\wred$ for $\red^*$, and $P\red$ if $\exists Q $ such that $ P \red Q$.
We write $P\red$ if $\exists Q $ such that $ P \red Q$ and $P\not\red$, otherwise.

\section{Replication}

As mentioned before, it is known that replication (and hence
recursion) can be implemented in a higher-order process algebra
\cite{SangiorgiWalker}. As our first example of calculation with the
machinery thus far presented we give the construction explicitly in
the {\rhoc}.

\begin{eqnarray}
	D_{x} & := & \prefix{x}{y}{(\binpar{\outputp{x}{y}}{@{y}})} \nonumber\\
	\bangp_{x}{P} & := & \binpar{{x}!\langle{\binpar{D_{x}}{P}}\rangle}{D_{x}} \nonumber
\end{eqnarray}

\begin{eqnarray}
	\bangp_{x}{P} & & \nonumber\\
	=
	& {x}!\langle{(\prefix{x}{y}{(\outputp{x}{y} | @{y})) | P}}\rangle 
	      | \prefix{x}{y}{(\outputp{x}{y} | @{y})} & \nonumber\\
	\red
	& (\outputp{x}{y} | @{y})\substn{\quotep{(\prefix{x}{y}{(@{y} | \outputp{x}{y})) | P}}}{y} & \nonumber\\
	=
	& \outputp{x}{\quotep{(\prefix{x}{y}{(\outputp{x}{y} | @{y})) | P}}}
	  | {(\prefix{x}{y}{(\outputp{x}{y} | @{y})) | P}} & \nonumber\\
	\red
	& \ldots & \nonumber\\
	\red^*
	& P | P | \ldots & \nonumber
\end{eqnarray}

Of course, this encoding, as an implementation, runs away, unfolding
$\bangp{P}$ eagerly. A lazier and more implementable replication
operator, restricted to input-guarded processes, may be obtained as follows.

\begin{eqnarray}
\bangp{\prefix{u}{v}{P}} 
	:= 
	\binpar{\lift{x}{\prefix{u}{v}{(\binpar{D(x)}{P})}}}{D(x)} \nonumber
\end{eqnarray}

\begin{remark}
  Note that the lazier definition still does not deal with summation
  or mixed summation (i.e. sums over input and output). The reader is
  invited to construct definitions of replication that deal with these
  features. 

  Further, the definitions are parameterized in a name, $x$. Can you,
  gentle reader, make a definition that eliminates this parameter and
  guarantees no accidental interaction between the replication
  machinery and the process being replicated -- i.e. no accidental
  sharing of names used by the process to get its work done and the
  name(s) used by the replication to effect copying. This latter
  revision of the definition of replication is crucial to obtaining
  the expected identity $!!P \sim !P$.
\end{remark}

\begin{remark}\label{rem:paradoxical_combinator}
  The reader familiar with the lambda calculus will have noticed the
  similarity between $D$ and the paradoxical combinator.

  [Ed. note: the existence of this seems to suggest we have to be more
  restrictive on the set of processes and names we admit if we are to
  support no-cloning.]
\end{remark}

\subsubsection{Bisimulation}

The computational dynamics gives rise to another kind of equivalence,
the equivalence of computational behavior. As previously mentioned
this is typically captured \emph{via} some form of bisimulation.

% The notion we use in this paper is weak barbed bisimulation
% \cite{milner91polyadicpi}.

The notion we use in this paper is derived from weak barbed
bisimulation \cite{milner91polyadicpi}. 

\begin{definition}
An \emph{observation relation}, $\downarrow_{\mathcal N}$, over a set
of names, $\mathcal N$, is the smallest relation satisfying the rules
below.

\infrule[Out-barb]{y \in {\mathcal N}, \; x \nameeq y}
		  {\outputp{x}{v} \downarrow_{\mathcal N} x}
\infrule[Par-barb]{\mbox{$P\downarrow_{\mathcal N} x$ or $Q\downarrow_{\mathcal N} x$}}
		  {\binpar{P}{Q} \downarrow_{\mathcal N} x}

We write $P \Downarrow_{\mathcal N} x$ if there is $Q$ such that 
$P \wred Q$ and $Q \downarrow_{\mathcal N} x$.
\end{definition}

\begin{definition}
%\label{def.bbisim}
An  ${\mathcal N}$-\emph{barbed bisimulation} over a set of names, ${\mathcal N}$, is a symmetric binary relation 
${\mathcal S}_{\mathcal N}$ between agents such that $P\rel{S}_{\mathcal N}Q$ implies:
\begin{enumerate}
\item If $P \red P'$ then $Q \wred Q'$ and $P'\rel{S}_{\mathcal N} Q'$.
\item If $P\downarrow_{\mathcal N} x$, then $Q\Downarrow_{\mathcal N} x$.
\end{enumerate}
$P$ is ${\mathcal N}$-barbed bisimilar to $Q$, written
$P \wbbisim_{\mathcal N} Q$, if $P \rel{S}_{\mathcal N} Q$ for some ${\mathcal N}$-barbed bisimulation ${\mathcal S}_{\mathcal N}$.
\end{definition}

$\mathcal{R} \subseteq \pi \times \pi$

$P \mathcal{R} Q => \forall P'. P \red P' \Rightarrow \exists Q'. Q \red Q', P' \mathcal{R} Q'$

$P \vdash x \Rightarrow Q \vdash x$

\begin{mathpar}
  \inferrule*[lab=Out-barb]{x \nameeq y}{{y}!\langle{Q}\rangle \vdash x}
  \and
  \inferrule*[lab=Par-barb]{\mbox{$P\vdash x$ or $Q\vdash x$}}{\binpar{P}{Q} \vdash x}
\end{mathpar}

\subsubsection{Contexts}

One of the principle advantages of computational calculi like the
$\pi$-calculus is a well-defined notion of context,
contextual-equivalence and a correlation between
contextual-equivalence and notions of bisimulation. The notion of
context allows the decomposition of a process into (sub-)process and
its syntactic environment, its context. Thus, a context may be
thought of as a process with a ``hole'' (written $\Box$) in it. The
application of a context $M$ to a process $P$, written $M[P]$, is
tantamount to filling the hole in $M$ with $P$. In this paper we do
not need the full weight of this theory, but do make use of the notion
of context in the proof the main theorem. 

\begin{mathpar}
  \inferrule* [lab=summation] {} {{M_{M},M_{N}} \bc \Box \;|\; x.M_{A} \;|\; M_{M}+M_{N}}
  \and
  \inferrule* [lab=agent] {} {{M_{A}} \bc (\vec{x})M_{P} \;| \; \clift{P_0,\ldots,M_{P},\ldots,P_N}}
  \and \\
  \inferrule* [lab=process] {} {{M_{P}} \bc M_{N} \;| \;P|M_{P} }
\end{mathpar} 

\begin{mathpar}
  \inferrule* [lab=sychronization] {} {M_{N} \bc \Box \;|\; x?M_{F} \;|\; x!M_{C}}
  \and
  \inferrule* [lab=abstraction] {} {{M_{F}} \bc (x)M_{P} }
  \and
  \inferrule* [lab=concretion] {} {{M_{C}} \bc \langle M_{P} \rangle }
  \and \\
  \inferrule* [lab=process] {} {{M_{P}} \bc M_{N} \;| \;P|M_{P} }
\end{mathpar}

\begin{definition}[contextual application] Given a context $M$, and
  process $P$, we define the \emph{contextual application}, $M[P] :=
  M\{P/\Box\}$. That is, the contextual application of M to P is the
  substitution of $P$ for $\Box$ in $M$.
\end{definition}

$\meaningof{-} : L \to \mathcal{P}(\pi)$

\begin{mathpar}
  \inferrule* [lab=collection] {} {\meaningof{true} = \pi, \and \meaningof{~E} = \pi \setminus \meaningof{E}, \and \meaningof{E_{1} \& E_{2}} = \meaningof{E_{1}} \cap \meaningof{E_{2}}}
\end{mathpar}

\begin{mathpar}
  \inferrule* [lab=structure] {} {\meaningof{0} = \{ P \in \pi | P \equiv 0 \}, \and \\ \meaningof{E_1 | E_2} = \{ P \in \pi | P \equiv P_{1} | P_{2}, P_{1} \in \meaningof{E_{1}}, P_{2} \in \meaningof{E_2}\} }
\end{mathpar}

\begin{mathpar}
 \inferrule* [lab=behavior] {} {\meaningof{\langle a?b \rangle E} = \{ P \in \pi | P \equiv Q | u?(y)P', \\ \and \\\\ \and \\ \;\;\; u \in \meaningof{a}, \forall z.P'\{z/y\} \in \meaningof{E\{z/b\}}\}, \and \\ \meaningof{a!E} = \{ P \in \pi | P \equiv Q | x!\langle P' \rangle, x \in \meaningof{a} P' \in \meaningof{E}\} }
\end{mathpar}

\begin{mathpar}
 \inferrule* [lab=nominal] {} {\meaningof{\quotep{E}} = \{ \quotep{P} \in \quotep{\pi} | P \in \meaningof{E} \}, \and \meaningof{\quotep{P}} = \{ \quotep{Q} \in \quotep{\pi} | P \equiv Q \} \and \\ \meaningof{@\quotep{E}} = \{ P \in \pi | P \equiv @x, x \in \meaningof{E} \}}
\end{mathpar}

\begin{eqnarray*}
  \\
  \meaningof{-} : TS \to ST
\end{eqnarray*}

\begin{eqnarray*}
  \\
  L : TS \to ST
\end{eqnarray*}

\begin{eqnarray*}
  \\
  P \models E \iff P \in \meaningof{E}
\end{eqnarray*}

\begin{eqnarray*}
  P \approx_{L} Q \iff \forall E \in L. P \models E \iff Q \models E
\end{eqnarray*}

\begin{eqnarray*}
  P \approx_{K} Q
\end{eqnarray*}

\begin{eqnarray*}
  P \approx Q
\end{eqnarray*}

$\approx_{K} = \approx = \approx_{L}$

\subsubsection{Contextual duality}

Note that contexts extend the quotation operation to a family of
operations from processes to names. Given a context, $M$, we can
define a \emph{nominal context}, $\quotep{M}$ by $\quotep{M}[P] :=
\quotep{M[P]}$. To foreshadow what is to come we observe that these
operations enjoy a duality with processes very much like the duality
between vectors and maps from vectors to scalars.

Further, because the calculus is essentially higher-order, we have a
correspondence between contexts and processes. More specifically,
given a name $x$ and a context $M$ we can construct $M^{*}_{x}$ such
that 

\begin{mathpar}
  M^{*}_{x} | \lift{x}{P} \red M[P]
\end{mathpar}

namely,

\begin{mathpar}
  M^{*}_{x} := x?(u).M[\dropn{u}]
\end{mathpar}

The dependence of $M^{*}_{x}$ on a name makes it an abstraction, 

\begin{mathpar}
  M^{*} := (x)x?(u).M[\dropn{u}]
\end{mathpar}

\subsection{Additional notation}

It will sometimes be convenient to denote the process a name
quotes. We already have the notation $x = \quotep{P}$, but it will be
convenient to introduce an alternate notation, $\procn{x}$, when we
want to emphasize the connection to the use of the name. Note that, by
virtue of name equivalence, $\quotep{\procn{x}} \nameeq x$; so, the
notation is consistent with previous definitions.

Further, because names have structure it is possible to effect
substitutions on the basis of that structure. This means we need to
upgrade our notation for substitutions, which we accomplish by
adapting comprehension notation. Thus,

\begin{mathpar}
  P\{ y / x : x \in S \}
\end{mathpar}

is interpreted to mean the process derived from P by replacing (in a
capture-avoiding manner) each occurrence of $x$ in $S$ by $y$. For example,

\begin{mathpar}
  P\{ \quotep{\procn{x}|\procn{x}} / x : x \in \freenames{P} \}
\end{mathpar}

will replace each (occurrence) of a free name $x$ in $P$ by
$\quotep{\procn{x}|\procn{x}}$.

Also, we will avail ourselves of the notation $x^{L}$ and $x^{R}$ to
denote injections of a name into disjoint copies of the name
space. There are numerous ways to accomplish this. One example can be
found in \cite{MeredithR05}. This notation overloads to vectors of
names: $\vec{x}^{\pi} := (x_{i}^{\pi} \; : \; 0 \leq i < |\vec{x}| )$ where $\pi \in \{L,R\}$.

We also use $P^{\Box} := P|\Box$.

In \cite{MeredithR05} an interpretation of the new operator is
given. It turns out that there are several possible interpretations
all enjoying the requisite algebraic properties of the operator (see
\cite{milner91polyadicpi}). We will therefore make liberal use of
$(\nu\; \vec{x})P$.

% subsection the_syntax_and_semantics_of_the_notation_system (end)   

\input{qm2pi.qmops} 

\input{qm2pi.sterngerlach} 

\input{qm2pi.metric} 

% section concurrent_process_calculi (end)

%\input{qm2pi.proofsketch}

% section proof sketch (end)

%\input{qm2pi.slviaknots} 

% section spatial logic via knots (end)

\input{qm2pi.conclusion}

% section conclusion (end)

%\input{qm2pi.dtcodes} 

% section wiring algorithm (end)

\input{qm2pi.ack} 

% section acknowledgments (end)

\newpage


\bibliographystyle{plain}   
\bibliography{../../biblios/main.bib}

\input{qm2pi.rhodetails}

\end{document}

 

\documentclass[12pt]{llncs}
%\documentclass{jktr}

\usepackage[pdftex]{hyperref}                   
\usepackage {listings}
\usepackage {mathpartir}
\usepackage{bcprules}
%\usepackage{listings}
                       
\usepackage{graphicx} 
%\usepackage[margins=2.5cm,nohead,nofoot]{geometry}
%\usepackage{geometry}
\usepackage{amsfonts}
\usepackage{amstext}
\usepackage{latexsym}
\usepackage{amssymb}
\usepackage{color}


%\include{myPreamble}
\include{qm2pi.local} 

%\ifpdf
%\usepackage[pdftex]{graphicx}
%\else
%\usepackage{graphicx}
%\fi

 % \ifpdf
%  \usepackage{pdfsync}
%  \if


%\title{Brief Article}
%\author{David F. Snyder}
%\author{L.G. Meredith}

%\address{Dept. of Math., Texas State University--San Marcos, San Marcos, TX 78666}
       
\pagestyle{empty}


\begin{document}

\lstset{language=[Objective]Caml,frame=shadowbox}

\input{qm2pi.front}

% section front matter (end)

\input{qm2pi.intro} 
 
% section introduction (end)

% \input{qm2pi.knotations} 

% section notation (end)

\input{qm2pi.process.calculi} 

% section concurrent_process_calculi_and_spatial_logics_ (end)
    
%\input{qm2pi.knots2pi} 

%\input{qm2pi.trefoil} 

%\input{qm2pi.mainthm} 

% subsection basic_interpretation (end)

%\input{qm2pi.rho.presentation} 
\subsection{The syntax and semantics of the notation system}\label{sub:the_syntax_and_semantics_of_the_notation_system} % (fold)

We now summarize a technical presentation of the calculus that
embodies our theory of dynamics. The typical presentation of such a
calculus follows the style of giving generators and relations on
them. The grammar, below, describing term constructors, freely
generates the set of processes, $\Proc$. This set is then quotiented
by a relation known as structural congruence and it is over this set
that the notion of dynamics is expressed. This presentation is
essentially that of \cite{MeredithR05} with the addition of
polyadicity and summation. For readability we have relegated some of
the technical subtleties to an appendix.

\subsubsection{Process grammar}\label{subsub:process_grammar}

\begin{mathpar}
  \inferrule* [lab=synchronization] {} {{M} \bc \pzero \;|\; x?F \;|\; x!C }
  \and
  \inferrule* [lab=abstraction] {} {{F} \bc (x)P}
  \and
  \inferrule* [lab=concretion] {} {{C} \bc \langle Q \rangle}
  \and
  \inferrule* [lab=process] {} {{P,Q} \bc M \;| \;P|Q \;|\; @{x}}
  \and
  \inferrule* [lab=name] {} {{x} \bc \quotep{P}}
\end{mathpar} 

Note that $\vec{x}$ (resp. $\vec{P}$) denotes a vector of names
(resp. processes) of length $|\vec{x}|$ (resp. $|\vec{P}|$). We adopt
the following useful abbreviations.

\begin{mathpar}
   x?(\vec{y}).P := x.(\vec{y})P \and  x\clift{\vec{P}} := x.\clift{\vec{P}}
   \and x!(y) := \lift{x}{\dropn{y}}
   \and \Pi_{i=0}^{n-1}P_i := P_0 | \ldots | P_{n-1}
\end{mathpar}

\subsubsection{Structural congruence}

\paragraph{Free and bound names and alpha-equivalence.} At the
core of structural equivalence is alpha-equivalence which identifies
process that are the same up to a change of variable. Formally, we
recognize the distinction between free and bound names. The free names
of a process, $\freenames{P}$, may be calculated recursively as
follows:

\begin{mathpar}
\freenames{\pzero} := \emptyset
  \and \\
  \freenames{x?(y).P} := \{ x \} \cup (\freenames{P} \setminus \{ y \})
  \and 
  \freenames{x!\langle P \rangle} := \{ x \} \cup \{ P \} 
  \and \\
  \freenames{P|Q} := \freenames{P} \cup \freenames{Q}
  \and \\
  \freenames{@{x}} := \{ x \}
\end{mathpar}

$\pi$
$\quotep{\pi}$

$\freenames{-} : \pi \to \mathcal{P}(\quotep{\pi})$

\begin{eqnarray*}
  \freenames{\pzero} & := & \emptyset \\
  \freenames{x?(y).P} & := & \{ x \} \cup (\freenames{P} \setminus \{ y \}) \\
  \freenames{x!\langle P \rangle} & := & \{ x \} \cup \{ P \} \\
  \freenames{P|Q} & := & \freenames{P} \cup \freenames{Q} \\
  \freenames{\dropn{x}} & := & \{ x \}
\end{eqnarray*}

The bound names of a process, $\boundnames{P}$, are those names occurring in $P$
that are not free. For example, in $x?(y).0$, the name $x$ is free, while $y$ is bound.

\begin{mathpar}
  \inferrule* [lab=monoidal-laws] {} { P|Q \equiv Q|P \and P|0 \equiv P \and P|(Q|R) \equiv (P|Q)|R }
\end{mathpar}

\begin{mathpar}
  \inferrule* [lab=alpha-equivalence] {} { (x)P \equiv (y)P\{y/x\} \and y \not\in \freenames{P} }
\end{mathpar}

\begin{definition}
Then two processes, $P,Q$, are alpha-equivalent if $P = Q\{\vec{y}/\vec{x}\}$ for
some $\vec{x} \in \boundnames{Q},\vec{y} \in \boundnames{P}$, where $Q\{\vec{y}/\vec{x}\}$
denotes the capture-avoiding substitution of $\vec{y}$ for $\vec{x}$ in $Q$.
\end{definition}

\begin{definition}
  The {\em structural congruence} \cite{SangiorgiWalker} , $\equiv$,
  between processes is the least congruence containing
  alpha-equivalence, satisfying the abelian monoid laws
  (associativity, commutativity and $\pzero$ as identity) for parallel
  composition $|$ and for summation $+$.
\end{definition}

\subsection{Name equivalence}

We take name equivalence, written $\nameeq$, to be the smallest
equivalence relation generated by the following rules.

\begin{mathpar}
\inferrule*[lab=Quote-drop]
{ }
{ \quotep{@{x}} \nameeq x }

\inferrule*[lab=Struct-equiv]
{ P \scong Q }
{ \quotep{P} \nameeq \quotep{Q} }
\end{mathpar}

The astute reader will have noticed that the mutual recursion of names
and processes imposes a mutual recursion on alpha-equivalence and
structural equivalence via name-equivalence. Fortunately, all of this
works out pleasantly and we may calculate in the natural way, free of
concern. The reader interested in the details is referred to the
appendix \ref{appendix:rho_details}.

\subsection{Substitution}

We use $\Proc$ for the set of processes, $\QProc$ for the set of
names, and $\id{\{}\vec{y} / \vec{x} \id{\}}$ to denote partial maps,
$s : \QProc \rightarrow \QProc$. A map, $s$ lifts, uniquely, to a map
on process terms, $\widehat{s} : \Proc \rightarrow \Proc$ by the
following equations.

\begin{mathpar}
  (0) \psubstp{Q}{P} := 0 \\
  (R \juxtap S) \psubstp{Q}{P}
  :=    
  (R)\psubstp{Q}{P} \juxtap (S) \psubstp{Q}{P} \\
  (x?(y).R) \psubstp{Q}{P}    
  :=    
  (x)\substp{Q}{P} (z)\concat( (R \psubstn{z}{y}) \psubstp{Q}{P} ) \\
  (\lift{x}{R}) \psubstp{Q}{P}  
  :=
  \lift{(x)\substp{Q}{P}}{ R \psubstp{Q}{P} } \\
%   (\dropn{x})  \psubstp{Q}{P}       
%   := 
%   \left\{ 
%     \begin{array}{ccc} 
%       \dropn{\quotep{Q}} & & x \nameeq \quotep{P} \\
%       \dropn{x} & & otherwise \\
%     \end{array}
%   \right. 
  (\dropn{x})  \psubstp{Q}{P}       
  := 
  \left\{ 
    \begin{array}{ccc} 
      Q & & x \nameeq \quotep{P} \\
      \dropn{x} & & otherwise \\
    \end{array}
  \right.
\end{mathpar}
 

where

\begin{eqnarray}
  (x)\id{\{} \lpquote Q \rpquote / \lpquote P \rpquote \id{\}}            = 
  \left\{ 
    \begin{array}{ccc}
      \lpquote Q \rpquote & & x \nameeq \lpquote P \rpquote \\
      x & & otherwise \\
    \end{array}
  \right. \nonumber
\end{eqnarray}

and $z$ is chosen distinct from $\quotep{P}$, $\quotep{Q}$, the free
names in $Q$, and all the names in $R$. Our $\alpha$-equivalence will
be built in the standard way from this substitution.

\begin{remark}\label{rem:no_self_referential_names}
  One consequence of these definitions is that $\forall P. \quotep{P}
  \not\in \freenames{P}$.
\end{remark}

\subsection{ Dynamic quote: an example }

Anticipating something of what's to come, consider applying the
substitution, $\widehat{\id{\{}u / z \id{\}}}$, to the following pair
of processes, $\lift{w}{y!(z)}$ and $w[ \lpquote y!(z) \rpquote ]$.

\begin{eqnarray}
	\lift{w}{y!(z)}\widehat{\id{\{}u / z \id{\}}}
		& = &
		\lift{w}{y!(u)} \nonumber\\
	w[ \lpquote y!(z) \rpquote ] \widehat{ \id{\{}u / z \id{\}} }
		& = &
		w[ \lpquote y!(z) \rpquote ] \nonumber
\end{eqnarray}

Because the body of the process between quotes is impervious to
substitution, we get radically different answers. In fact, by
examining the first process in an input context,
e.g. $x?(z).\lift{w}{y!(z)}$, we see that the process under the lift
operator may be shaped by prefixed inputs binding a name inside it. In
this sense, the lift operator will be seen as a way to dynamically
construct processes before reifying them as names.

Finally equipped with these standard features we can present the
dynamics of the calculus.

\subsubsection{Operational semantics} 

Finally, we introduce the computational dynamics. What marks these
algebras as distinct from other more traditionally studied algebraic
structures, e.g. vector spaces or polynomial rings, is the manner in
which dynamics is captured. In traditional structures, dynamics is typically
expressed through morphisms between such structures, as in linear maps
between vector spaces or morphisms between rings. In algebras
associated with the semantics of computation, the dynamics is
expressed as part of the algebraic structure itself, through a
reduction reduction relation typically denoted by $\red$. Below, we
give a recursive presentation of this relation for the calculus used
in the encoding.

$\red \subseteq \pi \times \pi$
$\red : \pi \to \mathcal{P}(\pi)$

\begin{mathpar}
  \inferrule* [lab=Comm] { \textsf{match}( x_{src}, x_{trgt} ) } { x_{trgt}?(y)P \; | \; x_{src}!\langle {Q} \rangle \red P\{\quotep{Q}/y}\} }
  \and \\
  \inferrule* [lab=Par] {{P} \red {P}'} {{{P} | {Q}} \red {{P}' | {Q}}}
  \and
  \inferrule* [lab=Equiv]{{{P} \scong {P}'} \andalso {{P}' \red {Q}'} \andalso {{Q}' \scong {Q}}}{{P} \red {Q}}
\end{mathpar}

\begin{eqnarray*}
  match_{\equiv} (\quotep{P},\quotep{Q}) & := & P \equiv Q \\
  match_{\dagger}(\quotep{P},\quotep{Q}) & := & \forall R. P|Q \red^{*} R => R \red^{*} 0 \\
  match_{K}(\quotep{P},\quotep{Q}) & := & K \mbox{ for some context } K
\end{eqnarray*}

$u?(x)P | u!\langle Q \rangle \red P\{\quotep{Q}/x\}$

%We write $\wred$ for $\red^*$, and $P\red$ if $\exists Q $ such that $ P \red Q$.
We write $P\red$ if $\exists Q $ such that $ P \red Q$ and $P\not\red$, otherwise.

\section{Replication}

As mentioned before, it is known that replication (and hence
recursion) can be implemented in a higher-order process algebra
\cite{SangiorgiWalker}. As our first example of calculation with the
machinery thus far presented we give the construction explicitly in
the {\rhoc}.

\begin{eqnarray}
	D_{x} & := & \prefix{x}{y}{(\binpar{\outputp{x}{y}}{@{y}})} \nonumber\\
	\bangp_{x}{P} & := & \binpar{{x}!\langle{\binpar{D_{x}}{P}}\rangle}{D_{x}} \nonumber
\end{eqnarray}

\begin{eqnarray}
	\bangp_{x}{P} & & \nonumber\\
	=
	& {x}!\langle{(\prefix{x}{y}{(\outputp{x}{y} | @{y})) | P}}\rangle 
	      | \prefix{x}{y}{(\outputp{x}{y} | @{y})} & \nonumber\\
	\red
	& (\outputp{x}{y} | @{y})\substn{\quotep{(\prefix{x}{y}{(@{y} | \outputp{x}{y})) | P}}}{y} & \nonumber\\
	=
	& \outputp{x}{\quotep{(\prefix{x}{y}{(\outputp{x}{y} | @{y})) | P}}}
	  | {(\prefix{x}{y}{(\outputp{x}{y} | @{y})) | P}} & \nonumber\\
	\red
	& \ldots & \nonumber\\
	\red^*
	& P | P | \ldots & \nonumber
\end{eqnarray}

Of course, this encoding, as an implementation, runs away, unfolding
$\bangp{P}$ eagerly. A lazier and more implementable replication
operator, restricted to input-guarded processes, may be obtained as follows.

\begin{eqnarray}
\bangp{\prefix{u}{v}{P}} 
	:= 
	\binpar{\lift{x}{\prefix{u}{v}{(\binpar{D(x)}{P})}}}{D(x)} \nonumber
\end{eqnarray}

\begin{remark}
  Note that the lazier definition still does not deal with summation
  or mixed summation (i.e. sums over input and output). The reader is
  invited to construct definitions of replication that deal with these
  features. 

  Further, the definitions are parameterized in a name, $x$. Can you,
  gentle reader, make a definition that eliminates this parameter and
  guarantees no accidental interaction between the replication
  machinery and the process being replicated -- i.e. no accidental
  sharing of names used by the process to get its work done and the
  name(s) used by the replication to effect copying. This latter
  revision of the definition of replication is crucial to obtaining
  the expected identity $!!P \sim !P$.
\end{remark}

\begin{remark}\label{rem:paradoxical_combinator}
  The reader familiar with the lambda calculus will have noticed the
  similarity between $D$ and the paradoxical combinator.

  [Ed. note: the existence of this seems to suggest we have to be more
  restrictive on the set of processes and names we admit if we are to
  support no-cloning.]
\end{remark}

\subsubsection{Bisimulation}

The computational dynamics gives rise to another kind of equivalence,
the equivalence of computational behavior. As previously mentioned
this is typically captured \emph{via} some form of bisimulation.

% The notion we use in this paper is weak barbed bisimulation
% \cite{milner91polyadicpi}.

The notion we use in this paper is derived from weak barbed
bisimulation \cite{milner91polyadicpi}. 

\begin{definition}
An \emph{observation relation}, $\downarrow_{\mathcal N}$, over a set
of names, $\mathcal N$, is the smallest relation satisfying the rules
below.

\infrule[Out-barb]{y \in {\mathcal N}, \; x \nameeq y}
		  {\outputp{x}{v} \downarrow_{\mathcal N} x}
\infrule[Par-barb]{\mbox{$P\downarrow_{\mathcal N} x$ or $Q\downarrow_{\mathcal N} x$}}
		  {\binpar{P}{Q} \downarrow_{\mathcal N} x}

We write $P \Downarrow_{\mathcal N} x$ if there is $Q$ such that 
$P \wred Q$ and $Q \downarrow_{\mathcal N} x$.
\end{definition}

\begin{definition}
%\label{def.bbisim}
An  ${\mathcal N}$-\emph{barbed bisimulation} over a set of names, ${\mathcal N}$, is a symmetric binary relation 
${\mathcal S}_{\mathcal N}$ between agents such that $P\rel{S}_{\mathcal N}Q$ implies:
\begin{enumerate}
\item If $P \red P'$ then $Q \wred Q'$ and $P'\rel{S}_{\mathcal N} Q'$.
\item If $P\downarrow_{\mathcal N} x$, then $Q\Downarrow_{\mathcal N} x$.
\end{enumerate}
$P$ is ${\mathcal N}$-barbed bisimilar to $Q$, written
$P \wbbisim_{\mathcal N} Q$, if $P \rel{S}_{\mathcal N} Q$ for some ${\mathcal N}$-barbed bisimulation ${\mathcal S}_{\mathcal N}$.
\end{definition}

$\mathcal{R} \subseteq \pi \times \pi$

$P \mathcal{R} Q => \forall P'. P \red P' \Rightarrow \exists Q'. Q \red Q', P' \mathcal{R} Q'$

$P \vdash x \Rightarrow Q \vdash x$

\begin{mathpar}
  \inferrule*[lab=Out-barb]{x \nameeq y}{{y}!\langle{Q}\rangle \vdash x}
  \and
  \inferrule*[lab=Par-barb]{\mbox{$P\vdash x$ or $Q\vdash x$}}{\binpar{P}{Q} \vdash x}
\end{mathpar}

\subsubsection{Contexts}

One of the principle advantages of computational calculi like the
$\pi$-calculus is a well-defined notion of context,
contextual-equivalence and a correlation between
contextual-equivalence and notions of bisimulation. The notion of
context allows the decomposition of a process into (sub-)process and
its syntactic environment, its context. Thus, a context may be
thought of as a process with a ``hole'' (written $\Box$) in it. The
application of a context $M$ to a process $P$, written $M[P]$, is
tantamount to filling the hole in $M$ with $P$. In this paper we do
not need the full weight of this theory, but do make use of the notion
of context in the proof the main theorem. 

\begin{mathpar}
  \inferrule* [lab=summation] {} {{M_{M},M_{N}} \bc \Box \;|\; x.M_{A} \;|\; M_{M}+M_{N}}
  \and
  \inferrule* [lab=agent] {} {{M_{A}} \bc (\vec{x})M_{P} \;| \; \clift{P_0,\ldots,M_{P},\ldots,P_N}}
  \and \\
  \inferrule* [lab=process] {} {{M_{P}} \bc M_{N} \;| \;P|M_{P} }
\end{mathpar} 

\begin{mathpar}
  \inferrule* [lab=sychronization] {} {M_{N} \bc \Box \;|\; x?M_{F} \;|\; x!M_{C}}
  \and
  \inferrule* [lab=abstraction] {} {{M_{F}} \bc (x)M_{P} }
  \and
  \inferrule* [lab=concretion] {} {{M_{C}} \bc \langle M_{P} \rangle }
  \and \\
  \inferrule* [lab=process] {} {{M_{P}} \bc M_{N} \;| \;P|M_{P} }
\end{mathpar}

\begin{definition}[contextual application] Given a context $M$, and
  process $P$, we define the \emph{contextual application}, $M[P] :=
  M\{P/\Box\}$. That is, the contextual application of M to P is the
  substitution of $P$ for $\Box$ in $M$.
\end{definition}

$\meaningof{-} : L \to \mathcal{P}(\pi)$

\begin{mathpar}
  \inferrule* [lab=collection] {} {\meaningof{true} = \pi, \and \meaningof{~E} = \pi \setminus \meaningof{E}, \and \meaningof{E_{1} \& E_{2}} = \meaningof{E_{1}} \cap \meaningof{E_{2}}}
\end{mathpar}

\begin{mathpar}
  \inferrule* [lab=structure] {} {\meaningof{0} = \{ P \in \pi | P \equiv 0 \}, \and \\ \meaningof{E_1 | E_2} = \{ P \in \pi | P \equiv P_{1} | P_{2}, P_{1} \in \meaningof{E_{1}}, P_{2} \in \meaningof{E_2}\} }
\end{mathpar}

\begin{mathpar}
 \inferrule* [lab=behavior] {} {\meaningof{\langle a?b \rangle E} = \{ P \in \pi | P \equiv Q | u?(y)P', \\ \and \\\\ \and \\ \;\;\; u \in \meaningof{a}, \forall z.P'\{z/y\} \in \meaningof{E\{z/b\}}\}, \and \\ \meaningof{a!E} = \{ P \in \pi | P \equiv Q | x!\langle P' \rangle, x \in \meaningof{a} P' \in \meaningof{E}\} }
\end{mathpar}

\begin{mathpar}
 \inferrule* [lab=nominal] {} {\meaningof{\quotep{E}} = \{ \quotep{P} \in \quotep{\pi} | P \in \meaningof{E} \}, \and \meaningof{\quotep{P}} = \{ \quotep{Q} \in \quotep{\pi} | P \equiv Q \} \and \\ \meaningof{@\quotep{E}} = \{ P \in \pi | P \equiv @x, x \in \meaningof{E} \}}
\end{mathpar}

\begin{eqnarray*}
  \\
  \meaningof{-} : TS \to ST
\end{eqnarray*}

\begin{eqnarray*}
  \\
  L : TS \to ST
\end{eqnarray*}

\begin{eqnarray*}
  \\
  P \models E \iff P \in \meaningof{E}
\end{eqnarray*}

\begin{eqnarray*}
  P \approx_{L} Q \iff \forall E \in L. P \models E \iff Q \models E
\end{eqnarray*}

\begin{eqnarray*}
  P \approx_{K} Q
\end{eqnarray*}

\begin{eqnarray*}
  P \approx Q
\end{eqnarray*}

$\approx_{K} = \approx = \approx_{L}$

\subsubsection{Contextual duality}

Note that contexts extend the quotation operation to a family of
operations from processes to names. Given a context, $M$, we can
define a \emph{nominal context}, $\quotep{M}$ by $\quotep{M}[P] :=
\quotep{M[P]}$. To foreshadow what is to come we observe that these
operations enjoy a duality with processes very much like the duality
between vectors and maps from vectors to scalars.

Further, because the calculus is essentially higher-order, we have a
correspondence between contexts and processes. More specifically,
given a name $x$ and a context $M$ we can construct $M^{*}_{x}$ such
that 

\begin{mathpar}
  M^{*}_{x} | \lift{x}{P} \red M[P]
\end{mathpar}

namely,

\begin{mathpar}
  M^{*}_{x} := x?(u).M[\dropn{u}]
\end{mathpar}

The dependence of $M^{*}_{x}$ on a name makes it an abstraction, 

\begin{mathpar}
  M^{*} := (x)x?(u).M[\dropn{u}]
\end{mathpar}

\subsection{Additional notation}

It will sometimes be convenient to denote the process a name
quotes. We already have the notation $x = \quotep{P}$, but it will be
convenient to introduce an alternate notation, $\procn{x}$, when we
want to emphasize the connection to the use of the name. Note that, by
virtue of name equivalence, $\quotep{\procn{x}} \nameeq x$; so, the
notation is consistent with previous definitions.

Further, because names have structure it is possible to effect
substitutions on the basis of that structure. This means we need to
upgrade our notation for substitutions, which we accomplish by
adapting comprehension notation. Thus,

\begin{mathpar}
  P\{ y / x : x \in S \}
\end{mathpar}

is interpreted to mean the process derived from P by replacing (in a
capture-avoiding manner) each occurrence of $x$ in $S$ by $y$. For example,

\begin{mathpar}
  P\{ \quotep{\procn{x}|\procn{x}} / x : x \in \freenames{P} \}
\end{mathpar}

will replace each (occurrence) of a free name $x$ in $P$ by
$\quotep{\procn{x}|\procn{x}}$.

Also, we will avail ourselves of the notation $x^{L}$ and $x^{R}$ to
denote injections of a name into disjoint copies of the name
space. There are numerous ways to accomplish this. One example can be
found in \cite{MeredithR05}. This notation overloads to vectors of
names: $\vec{x}^{\pi} := (x_{i}^{\pi} \; : \; 0 \leq i < |\vec{x}| )$ where $\pi \in \{L,R\}$.

We also use $P^{\Box} := P|\Box$.

In \cite{MeredithR05} an interpretation of the new operator is
given. It turns out that there are several possible interpretations
all enjoying the requisite algebraic properties of the operator (see
\cite{milner91polyadicpi}). We will therefore make liberal use of
$(\nu\; \vec{x})P$.

% subsection the_syntax_and_semantics_of_the_notation_system (end)   

\input{qm2pi.qmops} 

\input{qm2pi.sterngerlach} 

\input{qm2pi.metric} 

% section concurrent_process_calculi (end)

%\input{qm2pi.proofsketch}

% section proof sketch (end)

%\input{qm2pi.slviaknots} 

% section spatial logic via knots (end)

\input{qm2pi.conclusion}

% section conclusion (end)

%\input{qm2pi.dtcodes} 

% section wiring algorithm (end)

\input{qm2pi.ack} 

% section acknowledgments (end)

\newpage


\bibliographystyle{plain}   
\bibliography{../../biblios/main.bib}

\input{qm2pi.rhodetails}

\end{document}

 

% section concurrent_process_calculi (end)

%\documentclass[12pt]{llncs}
%\documentclass{jktr}

\usepackage[pdftex]{hyperref}                   
\usepackage {listings}
\usepackage {mathpartir}
\usepackage{bcprules}
%\usepackage{listings}
                       
\usepackage{graphicx} 
%\usepackage[margins=2.5cm,nohead,nofoot]{geometry}
%\usepackage{geometry}
\usepackage{amsfonts}
\usepackage{amstext}
\usepackage{latexsym}
\usepackage{amssymb}
\usepackage{color}


%\include{myPreamble}
\include{qm2pi.local} 

%\ifpdf
%\usepackage[pdftex]{graphicx}
%\else
%\usepackage{graphicx}
%\fi

 % \ifpdf
%  \usepackage{pdfsync}
%  \if


%\title{Brief Article}
%\author{David F. Snyder}
%\author{L.G. Meredith}

%\address{Dept. of Math., Texas State University--San Marcos, San Marcos, TX 78666}
       
\pagestyle{empty}


\begin{document}

\lstset{language=[Objective]Caml,frame=shadowbox}

\input{qm2pi.front}

% section front matter (end)

\input{qm2pi.intro} 
 
% section introduction (end)

% \input{qm2pi.knotations} 

% section notation (end)

\input{qm2pi.process.calculi} 

% section concurrent_process_calculi_and_spatial_logics_ (end)
    
%\input{qm2pi.knots2pi} 

%\input{qm2pi.trefoil} 

%\input{qm2pi.mainthm} 

% subsection basic_interpretation (end)

%\input{qm2pi.rho.presentation} 
\subsection{The syntax and semantics of the notation system}\label{sub:the_syntax_and_semantics_of_the_notation_system} % (fold)

We now summarize a technical presentation of the calculus that
embodies our theory of dynamics. The typical presentation of such a
calculus follows the style of giving generators and relations on
them. The grammar, below, describing term constructors, freely
generates the set of processes, $\Proc$. This set is then quotiented
by a relation known as structural congruence and it is over this set
that the notion of dynamics is expressed. This presentation is
essentially that of \cite{MeredithR05} with the addition of
polyadicity and summation. For readability we have relegated some of
the technical subtleties to an appendix.

\subsubsection{Process grammar}\label{subsub:process_grammar}

\begin{mathpar}
  \inferrule* [lab=synchronization] {} {{M} \bc \pzero \;|\; x?F \;|\; x!C }
  \and
  \inferrule* [lab=abstraction] {} {{F} \bc (x)P}
  \and
  \inferrule* [lab=concretion] {} {{C} \bc \langle Q \rangle}
  \and
  \inferrule* [lab=process] {} {{P,Q} \bc M \;| \;P|Q \;|\; @{x}}
  \and
  \inferrule* [lab=name] {} {{x} \bc \quotep{P}}
\end{mathpar} 

Note that $\vec{x}$ (resp. $\vec{P}$) denotes a vector of names
(resp. processes) of length $|\vec{x}|$ (resp. $|\vec{P}|$). We adopt
the following useful abbreviations.

\begin{mathpar}
   x?(\vec{y}).P := x.(\vec{y})P \and  x\clift{\vec{P}} := x.\clift{\vec{P}}
   \and x!(y) := \lift{x}{\dropn{y}}
   \and \Pi_{i=0}^{n-1}P_i := P_0 | \ldots | P_{n-1}
\end{mathpar}

\subsubsection{Structural congruence}

\paragraph{Free and bound names and alpha-equivalence.} At the
core of structural equivalence is alpha-equivalence which identifies
process that are the same up to a change of variable. Formally, we
recognize the distinction between free and bound names. The free names
of a process, $\freenames{P}$, may be calculated recursively as
follows:

\begin{mathpar}
\freenames{\pzero} := \emptyset
  \and \\
  \freenames{x?(y).P} := \{ x \} \cup (\freenames{P} \setminus \{ y \})
  \and 
  \freenames{x!\langle P \rangle} := \{ x \} \cup \{ P \} 
  \and \\
  \freenames{P|Q} := \freenames{P} \cup \freenames{Q}
  \and \\
  \freenames{@{x}} := \{ x \}
\end{mathpar}

$\pi$
$\quotep{\pi}$

$\freenames{-} : \pi \to \mathcal{P}(\quotep{\pi})$

\begin{eqnarray*}
  \freenames{\pzero} & := & \emptyset \\
  \freenames{x?(y).P} & := & \{ x \} \cup (\freenames{P} \setminus \{ y \}) \\
  \freenames{x!\langle P \rangle} & := & \{ x \} \cup \{ P \} \\
  \freenames{P|Q} & := & \freenames{P} \cup \freenames{Q} \\
  \freenames{\dropn{x}} & := & \{ x \}
\end{eqnarray*}

The bound names of a process, $\boundnames{P}$, are those names occurring in $P$
that are not free. For example, in $x?(y).0$, the name $x$ is free, while $y$ is bound.

\begin{mathpar}
  \inferrule* [lab=monoidal-laws] {} { P|Q \equiv Q|P \and P|0 \equiv P \and P|(Q|R) \equiv (P|Q)|R }
\end{mathpar}

\begin{mathpar}
  \inferrule* [lab=alpha-equivalence] {} { (x)P \equiv (y)P\{y/x\} \and y \not\in \freenames{P} }
\end{mathpar}

\begin{definition}
Then two processes, $P,Q$, are alpha-equivalent if $P = Q\{\vec{y}/\vec{x}\}$ for
some $\vec{x} \in \boundnames{Q},\vec{y} \in \boundnames{P}$, where $Q\{\vec{y}/\vec{x}\}$
denotes the capture-avoiding substitution of $\vec{y}$ for $\vec{x}$ in $Q$.
\end{definition}

\begin{definition}
  The {\em structural congruence} \cite{SangiorgiWalker} , $\equiv$,
  between processes is the least congruence containing
  alpha-equivalence, satisfying the abelian monoid laws
  (associativity, commutativity and $\pzero$ as identity) for parallel
  composition $|$ and for summation $+$.
\end{definition}

\subsection{Name equivalence}

We take name equivalence, written $\nameeq$, to be the smallest
equivalence relation generated by the following rules.

\begin{mathpar}
\inferrule*[lab=Quote-drop]
{ }
{ \quotep{@{x}} \nameeq x }

\inferrule*[lab=Struct-equiv]
{ P \scong Q }
{ \quotep{P} \nameeq \quotep{Q} }
\end{mathpar}

The astute reader will have noticed that the mutual recursion of names
and processes imposes a mutual recursion on alpha-equivalence and
structural equivalence via name-equivalence. Fortunately, all of this
works out pleasantly and we may calculate in the natural way, free of
concern. The reader interested in the details is referred to the
appendix \ref{appendix:rho_details}.

\subsection{Substitution}

We use $\Proc$ for the set of processes, $\QProc$ for the set of
names, and $\id{\{}\vec{y} / \vec{x} \id{\}}$ to denote partial maps,
$s : \QProc \rightarrow \QProc$. A map, $s$ lifts, uniquely, to a map
on process terms, $\widehat{s} : \Proc \rightarrow \Proc$ by the
following equations.

\begin{mathpar}
  (0) \psubstp{Q}{P} := 0 \\
  (R \juxtap S) \psubstp{Q}{P}
  :=    
  (R)\psubstp{Q}{P} \juxtap (S) \psubstp{Q}{P} \\
  (x?(y).R) \psubstp{Q}{P}    
  :=    
  (x)\substp{Q}{P} (z)\concat( (R \psubstn{z}{y}) \psubstp{Q}{P} ) \\
  (\lift{x}{R}) \psubstp{Q}{P}  
  :=
  \lift{(x)\substp{Q}{P}}{ R \psubstp{Q}{P} } \\
%   (\dropn{x})  \psubstp{Q}{P}       
%   := 
%   \left\{ 
%     \begin{array}{ccc} 
%       \dropn{\quotep{Q}} & & x \nameeq \quotep{P} \\
%       \dropn{x} & & otherwise \\
%     \end{array}
%   \right. 
  (\dropn{x})  \psubstp{Q}{P}       
  := 
  \left\{ 
    \begin{array}{ccc} 
      Q & & x \nameeq \quotep{P} \\
      \dropn{x} & & otherwise \\
    \end{array}
  \right.
\end{mathpar}
 

where

\begin{eqnarray}
  (x)\id{\{} \lpquote Q \rpquote / \lpquote P \rpquote \id{\}}            = 
  \left\{ 
    \begin{array}{ccc}
      \lpquote Q \rpquote & & x \nameeq \lpquote P \rpquote \\
      x & & otherwise \\
    \end{array}
  \right. \nonumber
\end{eqnarray}

and $z$ is chosen distinct from $\quotep{P}$, $\quotep{Q}$, the free
names in $Q$, and all the names in $R$. Our $\alpha$-equivalence will
be built in the standard way from this substitution.

\begin{remark}\label{rem:no_self_referential_names}
  One consequence of these definitions is that $\forall P. \quotep{P}
  \not\in \freenames{P}$.
\end{remark}

\subsection{ Dynamic quote: an example }

Anticipating something of what's to come, consider applying the
substitution, $\widehat{\id{\{}u / z \id{\}}}$, to the following pair
of processes, $\lift{w}{y!(z)}$ and $w[ \lpquote y!(z) \rpquote ]$.

\begin{eqnarray}
	\lift{w}{y!(z)}\widehat{\id{\{}u / z \id{\}}}
		& = &
		\lift{w}{y!(u)} \nonumber\\
	w[ \lpquote y!(z) \rpquote ] \widehat{ \id{\{}u / z \id{\}} }
		& = &
		w[ \lpquote y!(z) \rpquote ] \nonumber
\end{eqnarray}

Because the body of the process between quotes is impervious to
substitution, we get radically different answers. In fact, by
examining the first process in an input context,
e.g. $x?(z).\lift{w}{y!(z)}$, we see that the process under the lift
operator may be shaped by prefixed inputs binding a name inside it. In
this sense, the lift operator will be seen as a way to dynamically
construct processes before reifying them as names.

Finally equipped with these standard features we can present the
dynamics of the calculus.

\subsubsection{Operational semantics} 

Finally, we introduce the computational dynamics. What marks these
algebras as distinct from other more traditionally studied algebraic
structures, e.g. vector spaces or polynomial rings, is the manner in
which dynamics is captured. In traditional structures, dynamics is typically
expressed through morphisms between such structures, as in linear maps
between vector spaces or morphisms between rings. In algebras
associated with the semantics of computation, the dynamics is
expressed as part of the algebraic structure itself, through a
reduction reduction relation typically denoted by $\red$. Below, we
give a recursive presentation of this relation for the calculus used
in the encoding.

$\red \subseteq \pi \times \pi$
$\red : \pi \to \mathcal{P}(\pi)$

\begin{mathpar}
  \inferrule* [lab=Comm] { \textsf{match}( x_{src}, x_{trgt} ) } { x_{trgt}?(y)P \; | \; x_{src}!\langle {Q} \rangle \red P\{\quotep{Q}/y}\} }
  \and \\
  \inferrule* [lab=Par] {{P} \red {P}'} {{{P} | {Q}} \red {{P}' | {Q}}}
  \and
  \inferrule* [lab=Equiv]{{{P} \scong {P}'} \andalso {{P}' \red {Q}'} \andalso {{Q}' \scong {Q}}}{{P} \red {Q}}
\end{mathpar}

\begin{eqnarray*}
  match_{\equiv} (\quotep{P},\quotep{Q}) & := & P \equiv Q \\
  match_{\dagger}(\quotep{P},\quotep{Q}) & := & \forall R. P|Q \red^{*} R => R \red^{*} 0 \\
  match_{K}(\quotep{P},\quotep{Q}) & := & K \mbox{ for some context } K
\end{eqnarray*}

$u?(x)P | u!\langle Q \rangle \red P\{\quotep{Q}/x\}$

%We write $\wred$ for $\red^*$, and $P\red$ if $\exists Q $ such that $ P \red Q$.
We write $P\red$ if $\exists Q $ such that $ P \red Q$ and $P\not\red$, otherwise.

\section{Replication}

As mentioned before, it is known that replication (and hence
recursion) can be implemented in a higher-order process algebra
\cite{SangiorgiWalker}. As our first example of calculation with the
machinery thus far presented we give the construction explicitly in
the {\rhoc}.

\begin{eqnarray}
	D_{x} & := & \prefix{x}{y}{(\binpar{\outputp{x}{y}}{@{y}})} \nonumber\\
	\bangp_{x}{P} & := & \binpar{{x}!\langle{\binpar{D_{x}}{P}}\rangle}{D_{x}} \nonumber
\end{eqnarray}

\begin{eqnarray}
	\bangp_{x}{P} & & \nonumber\\
	=
	& {x}!\langle{(\prefix{x}{y}{(\outputp{x}{y} | @{y})) | P}}\rangle 
	      | \prefix{x}{y}{(\outputp{x}{y} | @{y})} & \nonumber\\
	\red
	& (\outputp{x}{y} | @{y})\substn{\quotep{(\prefix{x}{y}{(@{y} | \outputp{x}{y})) | P}}}{y} & \nonumber\\
	=
	& \outputp{x}{\quotep{(\prefix{x}{y}{(\outputp{x}{y} | @{y})) | P}}}
	  | {(\prefix{x}{y}{(\outputp{x}{y} | @{y})) | P}} & \nonumber\\
	\red
	& \ldots & \nonumber\\
	\red^*
	& P | P | \ldots & \nonumber
\end{eqnarray}

Of course, this encoding, as an implementation, runs away, unfolding
$\bangp{P}$ eagerly. A lazier and more implementable replication
operator, restricted to input-guarded processes, may be obtained as follows.

\begin{eqnarray}
\bangp{\prefix{u}{v}{P}} 
	:= 
	\binpar{\lift{x}{\prefix{u}{v}{(\binpar{D(x)}{P})}}}{D(x)} \nonumber
\end{eqnarray}

\begin{remark}
  Note that the lazier definition still does not deal with summation
  or mixed summation (i.e. sums over input and output). The reader is
  invited to construct definitions of replication that deal with these
  features. 

  Further, the definitions are parameterized in a name, $x$. Can you,
  gentle reader, make a definition that eliminates this parameter and
  guarantees no accidental interaction between the replication
  machinery and the process being replicated -- i.e. no accidental
  sharing of names used by the process to get its work done and the
  name(s) used by the replication to effect copying. This latter
  revision of the definition of replication is crucial to obtaining
  the expected identity $!!P \sim !P$.
\end{remark}

\begin{remark}\label{rem:paradoxical_combinator}
  The reader familiar with the lambda calculus will have noticed the
  similarity between $D$ and the paradoxical combinator.

  [Ed. note: the existence of this seems to suggest we have to be more
  restrictive on the set of processes and names we admit if we are to
  support no-cloning.]
\end{remark}

\subsubsection{Bisimulation}

The computational dynamics gives rise to another kind of equivalence,
the equivalence of computational behavior. As previously mentioned
this is typically captured \emph{via} some form of bisimulation.

% The notion we use in this paper is weak barbed bisimulation
% \cite{milner91polyadicpi}.

The notion we use in this paper is derived from weak barbed
bisimulation \cite{milner91polyadicpi}. 

\begin{definition}
An \emph{observation relation}, $\downarrow_{\mathcal N}$, over a set
of names, $\mathcal N$, is the smallest relation satisfying the rules
below.

\infrule[Out-barb]{y \in {\mathcal N}, \; x \nameeq y}
		  {\outputp{x}{v} \downarrow_{\mathcal N} x}
\infrule[Par-barb]{\mbox{$P\downarrow_{\mathcal N} x$ or $Q\downarrow_{\mathcal N} x$}}
		  {\binpar{P}{Q} \downarrow_{\mathcal N} x}

We write $P \Downarrow_{\mathcal N} x$ if there is $Q$ such that 
$P \wred Q$ and $Q \downarrow_{\mathcal N} x$.
\end{definition}

\begin{definition}
%\label{def.bbisim}
An  ${\mathcal N}$-\emph{barbed bisimulation} over a set of names, ${\mathcal N}$, is a symmetric binary relation 
${\mathcal S}_{\mathcal N}$ between agents such that $P\rel{S}_{\mathcal N}Q$ implies:
\begin{enumerate}
\item If $P \red P'$ then $Q \wred Q'$ and $P'\rel{S}_{\mathcal N} Q'$.
\item If $P\downarrow_{\mathcal N} x$, then $Q\Downarrow_{\mathcal N} x$.
\end{enumerate}
$P$ is ${\mathcal N}$-barbed bisimilar to $Q$, written
$P \wbbisim_{\mathcal N} Q$, if $P \rel{S}_{\mathcal N} Q$ for some ${\mathcal N}$-barbed bisimulation ${\mathcal S}_{\mathcal N}$.
\end{definition}

$\mathcal{R} \subseteq \pi \times \pi$

$P \mathcal{R} Q => \forall P'. P \red P' \Rightarrow \exists Q'. Q \red Q', P' \mathcal{R} Q'$

$P \vdash x \Rightarrow Q \vdash x$

\begin{mathpar}
  \inferrule*[lab=Out-barb]{x \nameeq y}{{y}!\langle{Q}\rangle \vdash x}
  \and
  \inferrule*[lab=Par-barb]{\mbox{$P\vdash x$ or $Q\vdash x$}}{\binpar{P}{Q} \vdash x}
\end{mathpar}

\subsubsection{Contexts}

One of the principle advantages of computational calculi like the
$\pi$-calculus is a well-defined notion of context,
contextual-equivalence and a correlation between
contextual-equivalence and notions of bisimulation. The notion of
context allows the decomposition of a process into (sub-)process and
its syntactic environment, its context. Thus, a context may be
thought of as a process with a ``hole'' (written $\Box$) in it. The
application of a context $M$ to a process $P$, written $M[P]$, is
tantamount to filling the hole in $M$ with $P$. In this paper we do
not need the full weight of this theory, but do make use of the notion
of context in the proof the main theorem. 

\begin{mathpar}
  \inferrule* [lab=summation] {} {{M_{M},M_{N}} \bc \Box \;|\; x.M_{A} \;|\; M_{M}+M_{N}}
  \and
  \inferrule* [lab=agent] {} {{M_{A}} \bc (\vec{x})M_{P} \;| \; \clift{P_0,\ldots,M_{P},\ldots,P_N}}
  \and \\
  \inferrule* [lab=process] {} {{M_{P}} \bc M_{N} \;| \;P|M_{P} }
\end{mathpar} 

\begin{mathpar}
  \inferrule* [lab=sychronization] {} {M_{N} \bc \Box \;|\; x?M_{F} \;|\; x!M_{C}}
  \and
  \inferrule* [lab=abstraction] {} {{M_{F}} \bc (x)M_{P} }
  \and
  \inferrule* [lab=concretion] {} {{M_{C}} \bc \langle M_{P} \rangle }
  \and \\
  \inferrule* [lab=process] {} {{M_{P}} \bc M_{N} \;| \;P|M_{P} }
\end{mathpar}

\begin{definition}[contextual application] Given a context $M$, and
  process $P$, we define the \emph{contextual application}, $M[P] :=
  M\{P/\Box\}$. That is, the contextual application of M to P is the
  substitution of $P$ for $\Box$ in $M$.
\end{definition}

$\meaningof{-} : L \to \mathcal{P}(\pi)$

\begin{mathpar}
  \inferrule* [lab=collection] {} {\meaningof{true} = \pi, \and \meaningof{~E} = \pi \setminus \meaningof{E}, \and \meaningof{E_{1} \& E_{2}} = \meaningof{E_{1}} \cap \meaningof{E_{2}}}
\end{mathpar}

\begin{mathpar}
  \inferrule* [lab=structure] {} {\meaningof{0} = \{ P \in \pi | P \equiv 0 \}, \and \\ \meaningof{E_1 | E_2} = \{ P \in \pi | P \equiv P_{1} | P_{2}, P_{1} \in \meaningof{E_{1}}, P_{2} \in \meaningof{E_2}\} }
\end{mathpar}

\begin{mathpar}
 \inferrule* [lab=behavior] {} {\meaningof{\langle a?b \rangle E} = \{ P \in \pi | P \equiv Q | u?(y)P', \\ \and \\\\ \and \\ \;\;\; u \in \meaningof{a}, \forall z.P'\{z/y\} \in \meaningof{E\{z/b\}}\}, \and \\ \meaningof{a!E} = \{ P \in \pi | P \equiv Q | x!\langle P' \rangle, x \in \meaningof{a} P' \in \meaningof{E}\} }
\end{mathpar}

\begin{mathpar}
 \inferrule* [lab=nominal] {} {\meaningof{\quotep{E}} = \{ \quotep{P} \in \quotep{\pi} | P \in \meaningof{E} \}, \and \meaningof{\quotep{P}} = \{ \quotep{Q} \in \quotep{\pi} | P \equiv Q \} \and \\ \meaningof{@\quotep{E}} = \{ P \in \pi | P \equiv @x, x \in \meaningof{E} \}}
\end{mathpar}

\begin{eqnarray*}
  \\
  \meaningof{-} : TS \to ST
\end{eqnarray*}

\begin{eqnarray*}
  \\
  L : TS \to ST
\end{eqnarray*}

\begin{eqnarray*}
  \\
  P \models E \iff P \in \meaningof{E}
\end{eqnarray*}

\begin{eqnarray*}
  P \approx_{L} Q \iff \forall E \in L. P \models E \iff Q \models E
\end{eqnarray*}

\begin{eqnarray*}
  P \approx_{K} Q
\end{eqnarray*}

\begin{eqnarray*}
  P \approx Q
\end{eqnarray*}

$\approx_{K} = \approx = \approx_{L}$

\subsubsection{Contextual duality}

Note that contexts extend the quotation operation to a family of
operations from processes to names. Given a context, $M$, we can
define a \emph{nominal context}, $\quotep{M}$ by $\quotep{M}[P] :=
\quotep{M[P]}$. To foreshadow what is to come we observe that these
operations enjoy a duality with processes very much like the duality
between vectors and maps from vectors to scalars.

Further, because the calculus is essentially higher-order, we have a
correspondence between contexts and processes. More specifically,
given a name $x$ and a context $M$ we can construct $M^{*}_{x}$ such
that 

\begin{mathpar}
  M^{*}_{x} | \lift{x}{P} \red M[P]
\end{mathpar}

namely,

\begin{mathpar}
  M^{*}_{x} := x?(u).M[\dropn{u}]
\end{mathpar}

The dependence of $M^{*}_{x}$ on a name makes it an abstraction, 

\begin{mathpar}
  M^{*} := (x)x?(u).M[\dropn{u}]
\end{mathpar}

\subsection{Additional notation}

It will sometimes be convenient to denote the process a name
quotes. We already have the notation $x = \quotep{P}$, but it will be
convenient to introduce an alternate notation, $\procn{x}$, when we
want to emphasize the connection to the use of the name. Note that, by
virtue of name equivalence, $\quotep{\procn{x}} \nameeq x$; so, the
notation is consistent with previous definitions.

Further, because names have structure it is possible to effect
substitutions on the basis of that structure. This means we need to
upgrade our notation for substitutions, which we accomplish by
adapting comprehension notation. Thus,

\begin{mathpar}
  P\{ y / x : x \in S \}
\end{mathpar}

is interpreted to mean the process derived from P by replacing (in a
capture-avoiding manner) each occurrence of $x$ in $S$ by $y$. For example,

\begin{mathpar}
  P\{ \quotep{\procn{x}|\procn{x}} / x : x \in \freenames{P} \}
\end{mathpar}

will replace each (occurrence) of a free name $x$ in $P$ by
$\quotep{\procn{x}|\procn{x}}$.

Also, we will avail ourselves of the notation $x^{L}$ and $x^{R}$ to
denote injections of a name into disjoint copies of the name
space. There are numerous ways to accomplish this. One example can be
found in \cite{MeredithR05}. This notation overloads to vectors of
names: $\vec{x}^{\pi} := (x_{i}^{\pi} \; : \; 0 \leq i < |\vec{x}| )$ where $\pi \in \{L,R\}$.

We also use $P^{\Box} := P|\Box$.

In \cite{MeredithR05} an interpretation of the new operator is
given. It turns out that there are several possible interpretations
all enjoying the requisite algebraic properties of the operator (see
\cite{milner91polyadicpi}). We will therefore make liberal use of
$(\nu\; \vec{x})P$.

% subsection the_syntax_and_semantics_of_the_notation_system (end)   

\input{qm2pi.qmops} 

\input{qm2pi.sterngerlach} 

\input{qm2pi.metric} 

% section concurrent_process_calculi (end)

%\input{qm2pi.proofsketch}

% section proof sketch (end)

%\input{qm2pi.slviaknots} 

% section spatial logic via knots (end)

\input{qm2pi.conclusion}

% section conclusion (end)

%\input{qm2pi.dtcodes} 

% section wiring algorithm (end)

\input{qm2pi.ack} 

% section acknowledgments (end)

\newpage


\bibliographystyle{plain}   
\bibliography{../../biblios/main.bib}

\input{qm2pi.rhodetails}

\end{document}



% section proof sketch (end)

%\section{Unlikely characters: spatial logic for
  knots}\label{sub:characteristic_formulae} % (fold)

Associated to the mobile process calculi are a family of logics known
as the Hennessy-Milner logics. These logics typically enjoy a
semantics interpreting formulae as sets of processes that when
factored through the encoding outlined above allows an identification
of classes of knots with logical formulae. In the context of this
encoding the sub-family known as the spatial logics \cite{CairesC03}
\cite{CairesC04} \cite{Caires04} are of particular interest providing
several important features for expressing and reasoning about
properties (i.e. classes) of knots. We hint here at how this may be done.

%\begin{description}
%\item [structural connectives] 
\subsubsection{Structural connectives} The spatial logics enjoy
structural connectives corresponding, at the logical level, to the
parallel composition ($P | Q$) and new name ($(\nu \; x)P$)
connectives for processes. As illustrated in the examples below, these
connectives are extremely expressive given the shape of our encoding.
%\item [decideable satisfaction]

\subsubsection{Decideable satisfaction}
In \cite{Caires04} the satisfaction relation is shown to be decideable
for a rich class of processes. It further turns out that the image of
the our encoding is a proper subset of that class. This result
provides the basis for an algorithm by which to search for knots
enjoying a given property.
%\item [characteristic formulae]

\subsubsection{Characteristic formulae}
In the same paper \cite{Caires04} , Caires presents a means of calculating
characteristic formulae, selecting equivalence classes of processes
up to a pre--specified depth limit on the support set of names. Composed with our
encoding, this characteristic formula can be used to select
characteristic formulae for knots.
%\end{description}

\subsubsection{Spatial logic formulae}

The grammar below (segmented for comprehension) summarizes the syntax
of spatial logic formulae. We employ illustrative examples in the
sequel to provide an intuitive understanding of their meaning
referring the reader to \cite{Caires04} for a more detailed explication
of the semantics.

\begin{mathpar}
  \inferrule* [lab=boolean] {} {{A,B} \bc T \;|\; \neg A \;|\; A \wedge B \;|\; \eta = \eta'}
  \and
  \inferrule* [lab=spatial] {} {|\; \pzero \;|\; A | B \;|\; x \text{\textregistered} A \;|\; \forall x . A \;|\;  H x . A}
  \and
  \inferrule* [lab=behavioral] {} {|\; \alpha . A}
  \and 
  \inferrule* [lab=recursion] {} {|\; X(\vec{u}) \;|\; \mu X(\vec{u}) . A}
  \and
  \inferrule* [lab=action] {} {\alpha \bc \langle x?(\vec{y}) \rangle \;|\; \langle x!(\vec{y}) \rangle \;|\; \langle \tau \rangle}
  \and 
  \inferrule* [lab=name] {} {\eta \bc x \;|\; \tau}
\end{mathpar} 

% subsection characteristic_formulae (end)   	 

\subsection{Example formulae}\label{sub:example_formulae_} % (fold)

\subsubsection{Crossing as formula.}
% 
% \begin{align*}
%   \frac{d}{dx} \sin x &= \cos x 
%   & \frac{d}{dx} e^x &= e^x \\
%   \frac{d}{dx} \cos x &= - \sin x 
%   & \frac{d}{dx} \log x &= \frac{1}{x} \\
% \end{align*} 

\begin{align*}
 \mu C(x_{0},x_{1},y_{0},y_{1},u).&(\langle x_{0}?(z) \rangle(\langle u! \rangle\langle y_{1}!z \rangle C(x_{0},x_{1},y_{0},y_{1},u)) & \\
  & \wedge \langle y_{1}?(z) \rangle (\langle u! \rangle \langle x_{0}!z \rangle C(x_{0},x_{1},y_{0},y_{1},u)) & \\
  & \wedge \langle x_{1}?(z) \rangle (\langle u? \rangle \langle y_{0}!z \rangle C(x_{0},x_{1},y_{0},y_{1},u)) & \\
  & \wedge \langle y_{0}?(z) \rangle (\langle u? \rangle \langle x_{1}!z \rangle C(x_{0},x_{1},y_{0},y_{1},u))) &
\end{align*}

The lexicographical similarity between the shape of this formulae and
the shape of definition of the process representing a crossing reveals
the intuitive meaning of this formulae. It describes the capabilities
of a process that has the right to represent a crossing. For example
it picks out processes that may perform an input on the port $x_0$ in
its initial menu of capabilities. What differentiates the formula
from the process, however, is that the crossing process is the
smallest candidate to satisfy the formula. Infinitely many other
processes -- with internal behavior hidden behind this interface, so
to speak -- also satisfy this formula. Even this simple formula,
then, can be seen to open a new view onto knots, providing a
computational interpretation of \emph{virtual} knots.

Note that this formula is derived by hand. A similar formula can be
derived by employing Caires' calculation of characteristic formula
\cite{Caires04} to the process representing a crossing. In light of
this discussion, we let
$\meaningof{C}_{\phi}(x0,x1,y0,y1,u)$ denote a formula specifying the
dynamics we wish to capture of a crossing. To guarantee we preserve
the shape of the interface and minimal semantics we demand that
$\meaningof{C}_{\phi}(x0,x1,y0,y1,u) \Rightarrow
\textbf{C}(x0,x1,y0,y1,u)$ where $\textbf{C}(x0,x1,y0,y1,u)$ denotes
the formula above.
                            
\subsubsection{Crossing number constraints.}
The moral content of the context lemma (Lemma \ref{context}) is that the notion of
``locality'' in the Reidemeister moves is effectively captured by the
parallel composition operator of the process calculus. This intuition
extends through the logic. Given a formula,
$\meaningof{C}_{\phi}(x0,x1,y0,y1,u)$, we can use the structural
connectives to specify constraints on crossing numbers, such as at
least $n$ crossings, or exactly $n$ crossings.
\begin{mathpar}
  \inferrule* [lab=at-least-n] {} { K^{\geq n}_{\phi}(\vec{xs},\vec{ys}) := \Pi_{i=0}^{n-1} Hu . \meaningof{C}_{\phi}(xs_i,ys_i,u) | T }
  \and 
  \inferrule* [lab=exactly-n] {} { K^{= n}_{\phi}(\vec{xs},\vec{ys}) := \Pi_{i=0}^{n-1} Hu . \meaningof{C}_{\phi}(xs_i,ys_i,u) | \neg (\forall x_0,y_0,x_1,y_1,u . \meaningof{C}_{\phi}(x_0,y_0,x_1,y_1,u) | T) }
\end{mathpar}

To round out this section, recall that the encoding of an $n$-crossing
knot decomposes into a parallel composition of $n$ \emph{copies} of a
crossing process together with a wiring harness. To specify different
knot classes with the same crossing number amounts to specifying
logical constraints on the wiring harness. In the interest of space,
we defer examples to a forthcoming paper. Suffice it to say that both
the conditions ``alternating knot'' and ``contains the tangle
corresponding to 5/3'' are expressible. For example, it is possible to
calculate the characteristic formula of a process corresponding to the
tangle 5/3 and conjoin it into the classifying formula via the
composition connective of the logic.

Finally, we wish to observe that it is entirely within reason to
contemplate a more domain-specific version of spatial logic tailored
to the shape of processes in the image of the encoding. Such a
domain-specific logic would have a better claim to the title formal
language of knot properties.

% subsection example_formulae_ (end)

% section knots_as_processes (end) 

% section spatial logic via knots (end)

\section{Conclusions and future work}

\paragraph{Testing physical space}
You, gentle reader, may wonder why of all the theorems to be proved
given this set up we pick the one above. In some sense it's hardly
central to quantum mechanics. We see it as central in the sense that
it firmly establishes a notion of physical space arising from a notion
of the equivalence of behavior. Relating bisimulation to a metric is a
big step forward, but one is faced with interpreting the relationship
of that metric space to something more physical. Quantum mechanical
notions of ``physical'' space are still far from intuitive, but by
relating this idea of distance as testing to calculations that predict
physical circumstances we are making a not insignificant step forward
toward an understanding of the physical space we inhabit as
essentially dynamic.

\paragraph{Effectivity and simulation}
One of the observations we have yet to make is that the entire program
spelled out here is effective. We have built various interpreters for
the reflective calculus at work in this interpretation. In principle,
then, we can simulate quantum mechanics on a computer. The place where
the simulation may lose fidelity is the infinitely branching summation
for the annihilator.

In this connection i also want to point out that the evaluation style
calculation of the inner product puts the non-determinism of the
summation right at the heart of measurement. This suggests that
Milner's original reduction-based formulation of the dynamics of his
calculi in terms of sums was not just notationally suggestive of a
notion of measure-and-continue but captured some significant part of
the physics.

\paragraph{Quantum continuations}
In light of this last observation i want to point out that the
predominant account of quantum mechanics is missing a key aspect of a
truly compositional story of the physical situation. In a real lab,
when a measurement is made the observation can be made to feed into
another device that then makes another measurement conditioned on the
results of the first. This means that after the superposition was
collapsed the entire experimental set up remained in
superposition. While QM offers a means of writing this down it doesn't
quite line up well with the well-trodden formulation of computation
and continuation that we see so succinctly expressed in Milner's
calculi. This suggests that there might be advantages to this account
of dynamics waiting to be explored.

\paragraph{Quantum logic}
In this connection, we also note that by virtue of having the
Hennessy-Milner construction, we can pull the construction through the
interpretation of QM. This gives us a natural candidate for a quantum
logic that enjoys an extremely tight connection with it's domain of
interpretation, making the construction much less ad hoc (rather it is
the image of functor!).

\paragraph{Quantum probabiity}
i have questions about the basis of the interpretation of inner
product as probability amplitude. In particular, using which
axiomatization of probability theory does the notion of probability
amplitude earn the right to be so dubbed? In other words, where is the
proof that the operation for calculating a probability amplitude (and
then squaring) satisfies the axioms of what it means to calculate a
probability? Even if such a proof exists (i have yet to find it in the
literature), i wonder if it might not be possible to turn things on
their heads. Can we view the calculation of the probability amplitude
as an axiomatization of probability? If so, then the definition we
give for calculating probability amplitude may provide the basis for
an \emph{effective} theory of probability.

\paragraph{Quantum vs ``biological'' information}
Finally, i want to conclude with a more philosophical observation. At
a recent workshop in which QM was a predominant topic i noticed
something about quantum information. The speaker was giving a riveting
discussion of axiomatic QM and showing how properties of ``no
cloning'' and ``no deleting'' emerged as consequences of the
axiomatization. Theorems of this form are necessary to give us a sense
of confidence that our axioms characterize the physical theory. What
struck me, though, was that if quantum information is neither erasable
nor replicable it is markedly different from \emph{life}. Two of the
things we know about life is that

\begin{itemize}
  \item it ends;
  \item to gain some measure of persistence, to transcend it's
    finitude it is imminently copyable.
\end{itemize}

Both of these qualities are summarized succinctly in the aphorism: all
flesh is grass. For me these two kinds of ``information'' -- call them
quantum and biological -- are end points on a spectrum of strategies
for persistence. At one end, we have those curious entities that enjoy
uniqueness and permanence; at the other, we have those who in the face
of a certain end and an uncertain present make a go of passing
something on. To me one of the more remarkable aspects of the latter
strategy is that in the presence of noise (and certain features of
copying) we get a kind of dynamism, a chance for improvement against a
given persistent condition.

% subsection other_calculi_other_bisimulations_and_geometry_as_behavior (end)




% section conclusion (end)

%\documentclass[12pt]{llncs}
%\documentclass{jktr}

\usepackage[pdftex]{hyperref}                   
\usepackage {listings}
\usepackage {mathpartir}
\usepackage{bcprules}
%\usepackage{listings}
                       
\usepackage{graphicx} 
%\usepackage[margins=2.5cm,nohead,nofoot]{geometry}
%\usepackage{geometry}
\usepackage{amsfonts}
\usepackage{amstext}
\usepackage{latexsym}
\usepackage{amssymb}
\usepackage{color}


%\include{myPreamble}
\include{qm2pi.local} 

%\ifpdf
%\usepackage[pdftex]{graphicx}
%\else
%\usepackage{graphicx}
%\fi

 % \ifpdf
%  \usepackage{pdfsync}
%  \if


%\title{Brief Article}
%\author{David F. Snyder}
%\author{L.G. Meredith}

%\address{Dept. of Math., Texas State University--San Marcos, San Marcos, TX 78666}
       
\pagestyle{empty}


\begin{document}

\lstset{language=[Objective]Caml,frame=shadowbox}

\input{qm2pi.front}

% section front matter (end)

\input{qm2pi.intro} 
 
% section introduction (end)

% \input{qm2pi.knotations} 

% section notation (end)

\input{qm2pi.process.calculi} 

% section concurrent_process_calculi_and_spatial_logics_ (end)
    
%\input{qm2pi.knots2pi} 

%\input{qm2pi.trefoil} 

%\input{qm2pi.mainthm} 

% subsection basic_interpretation (end)

%\input{qm2pi.rho.presentation} 
\subsection{The syntax and semantics of the notation system}\label{sub:the_syntax_and_semantics_of_the_notation_system} % (fold)

We now summarize a technical presentation of the calculus that
embodies our theory of dynamics. The typical presentation of such a
calculus follows the style of giving generators and relations on
them. The grammar, below, describing term constructors, freely
generates the set of processes, $\Proc$. This set is then quotiented
by a relation known as structural congruence and it is over this set
that the notion of dynamics is expressed. This presentation is
essentially that of \cite{MeredithR05} with the addition of
polyadicity and summation. For readability we have relegated some of
the technical subtleties to an appendix.

\subsubsection{Process grammar}\label{subsub:process_grammar}

\begin{mathpar}
  \inferrule* [lab=synchronization] {} {{M} \bc \pzero \;|\; x?F \;|\; x!C }
  \and
  \inferrule* [lab=abstraction] {} {{F} \bc (x)P}
  \and
  \inferrule* [lab=concretion] {} {{C} \bc \langle Q \rangle}
  \and
  \inferrule* [lab=process] {} {{P,Q} \bc M \;| \;P|Q \;|\; @{x}}
  \and
  \inferrule* [lab=name] {} {{x} \bc \quotep{P}}
\end{mathpar} 

Note that $\vec{x}$ (resp. $\vec{P}$) denotes a vector of names
(resp. processes) of length $|\vec{x}|$ (resp. $|\vec{P}|$). We adopt
the following useful abbreviations.

\begin{mathpar}
   x?(\vec{y}).P := x.(\vec{y})P \and  x\clift{\vec{P}} := x.\clift{\vec{P}}
   \and x!(y) := \lift{x}{\dropn{y}}
   \and \Pi_{i=0}^{n-1}P_i := P_0 | \ldots | P_{n-1}
\end{mathpar}

\subsubsection{Structural congruence}

\paragraph{Free and bound names and alpha-equivalence.} At the
core of structural equivalence is alpha-equivalence which identifies
process that are the same up to a change of variable. Formally, we
recognize the distinction between free and bound names. The free names
of a process, $\freenames{P}$, may be calculated recursively as
follows:

\begin{mathpar}
\freenames{\pzero} := \emptyset
  \and \\
  \freenames{x?(y).P} := \{ x \} \cup (\freenames{P} \setminus \{ y \})
  \and 
  \freenames{x!\langle P \rangle} := \{ x \} \cup \{ P \} 
  \and \\
  \freenames{P|Q} := \freenames{P} \cup \freenames{Q}
  \and \\
  \freenames{@{x}} := \{ x \}
\end{mathpar}

$\pi$
$\quotep{\pi}$

$\freenames{-} : \pi \to \mathcal{P}(\quotep{\pi})$

\begin{eqnarray*}
  \freenames{\pzero} & := & \emptyset \\
  \freenames{x?(y).P} & := & \{ x \} \cup (\freenames{P} \setminus \{ y \}) \\
  \freenames{x!\langle P \rangle} & := & \{ x \} \cup \{ P \} \\
  \freenames{P|Q} & := & \freenames{P} \cup \freenames{Q} \\
  \freenames{\dropn{x}} & := & \{ x \}
\end{eqnarray*}

The bound names of a process, $\boundnames{P}$, are those names occurring in $P$
that are not free. For example, in $x?(y).0$, the name $x$ is free, while $y$ is bound.

\begin{mathpar}
  \inferrule* [lab=monoidal-laws] {} { P|Q \equiv Q|P \and P|0 \equiv P \and P|(Q|R) \equiv (P|Q)|R }
\end{mathpar}

\begin{mathpar}
  \inferrule* [lab=alpha-equivalence] {} { (x)P \equiv (y)P\{y/x\} \and y \not\in \freenames{P} }
\end{mathpar}

\begin{definition}
Then two processes, $P,Q$, are alpha-equivalent if $P = Q\{\vec{y}/\vec{x}\}$ for
some $\vec{x} \in \boundnames{Q},\vec{y} \in \boundnames{P}$, where $Q\{\vec{y}/\vec{x}\}$
denotes the capture-avoiding substitution of $\vec{y}$ for $\vec{x}$ in $Q$.
\end{definition}

\begin{definition}
  The {\em structural congruence} \cite{SangiorgiWalker} , $\equiv$,
  between processes is the least congruence containing
  alpha-equivalence, satisfying the abelian monoid laws
  (associativity, commutativity and $\pzero$ as identity) for parallel
  composition $|$ and for summation $+$.
\end{definition}

\subsection{Name equivalence}

We take name equivalence, written $\nameeq$, to be the smallest
equivalence relation generated by the following rules.

\begin{mathpar}
\inferrule*[lab=Quote-drop]
{ }
{ \quotep{@{x}} \nameeq x }

\inferrule*[lab=Struct-equiv]
{ P \scong Q }
{ \quotep{P} \nameeq \quotep{Q} }
\end{mathpar}

The astute reader will have noticed that the mutual recursion of names
and processes imposes a mutual recursion on alpha-equivalence and
structural equivalence via name-equivalence. Fortunately, all of this
works out pleasantly and we may calculate in the natural way, free of
concern. The reader interested in the details is referred to the
appendix \ref{appendix:rho_details}.

\subsection{Substitution}

We use $\Proc$ for the set of processes, $\QProc$ for the set of
names, and $\id{\{}\vec{y} / \vec{x} \id{\}}$ to denote partial maps,
$s : \QProc \rightarrow \QProc$. A map, $s$ lifts, uniquely, to a map
on process terms, $\widehat{s} : \Proc \rightarrow \Proc$ by the
following equations.

\begin{mathpar}
  (0) \psubstp{Q}{P} := 0 \\
  (R \juxtap S) \psubstp{Q}{P}
  :=    
  (R)\psubstp{Q}{P} \juxtap (S) \psubstp{Q}{P} \\
  (x?(y).R) \psubstp{Q}{P}    
  :=    
  (x)\substp{Q}{P} (z)\concat( (R \psubstn{z}{y}) \psubstp{Q}{P} ) \\
  (\lift{x}{R}) \psubstp{Q}{P}  
  :=
  \lift{(x)\substp{Q}{P}}{ R \psubstp{Q}{P} } \\
%   (\dropn{x})  \psubstp{Q}{P}       
%   := 
%   \left\{ 
%     \begin{array}{ccc} 
%       \dropn{\quotep{Q}} & & x \nameeq \quotep{P} \\
%       \dropn{x} & & otherwise \\
%     \end{array}
%   \right. 
  (\dropn{x})  \psubstp{Q}{P}       
  := 
  \left\{ 
    \begin{array}{ccc} 
      Q & & x \nameeq \quotep{P} \\
      \dropn{x} & & otherwise \\
    \end{array}
  \right.
\end{mathpar}
 

where

\begin{eqnarray}
  (x)\id{\{} \lpquote Q \rpquote / \lpquote P \rpquote \id{\}}            = 
  \left\{ 
    \begin{array}{ccc}
      \lpquote Q \rpquote & & x \nameeq \lpquote P \rpquote \\
      x & & otherwise \\
    \end{array}
  \right. \nonumber
\end{eqnarray}

and $z$ is chosen distinct from $\quotep{P}$, $\quotep{Q}$, the free
names in $Q$, and all the names in $R$. Our $\alpha$-equivalence will
be built in the standard way from this substitution.

\begin{remark}\label{rem:no_self_referential_names}
  One consequence of these definitions is that $\forall P. \quotep{P}
  \not\in \freenames{P}$.
\end{remark}

\subsection{ Dynamic quote: an example }

Anticipating something of what's to come, consider applying the
substitution, $\widehat{\id{\{}u / z \id{\}}}$, to the following pair
of processes, $\lift{w}{y!(z)}$ and $w[ \lpquote y!(z) \rpquote ]$.

\begin{eqnarray}
	\lift{w}{y!(z)}\widehat{\id{\{}u / z \id{\}}}
		& = &
		\lift{w}{y!(u)} \nonumber\\
	w[ \lpquote y!(z) \rpquote ] \widehat{ \id{\{}u / z \id{\}} }
		& = &
		w[ \lpquote y!(z) \rpquote ] \nonumber
\end{eqnarray}

Because the body of the process between quotes is impervious to
substitution, we get radically different answers. In fact, by
examining the first process in an input context,
e.g. $x?(z).\lift{w}{y!(z)}$, we see that the process under the lift
operator may be shaped by prefixed inputs binding a name inside it. In
this sense, the lift operator will be seen as a way to dynamically
construct processes before reifying them as names.

Finally equipped with these standard features we can present the
dynamics of the calculus.

\subsubsection{Operational semantics} 

Finally, we introduce the computational dynamics. What marks these
algebras as distinct from other more traditionally studied algebraic
structures, e.g. vector spaces or polynomial rings, is the manner in
which dynamics is captured. In traditional structures, dynamics is typically
expressed through morphisms between such structures, as in linear maps
between vector spaces or morphisms between rings. In algebras
associated with the semantics of computation, the dynamics is
expressed as part of the algebraic structure itself, through a
reduction reduction relation typically denoted by $\red$. Below, we
give a recursive presentation of this relation for the calculus used
in the encoding.

$\red \subseteq \pi \times \pi$
$\red : \pi \to \mathcal{P}(\pi)$

\begin{mathpar}
  \inferrule* [lab=Comm] { \textsf{match}( x_{src}, x_{trgt} ) } { x_{trgt}?(y)P \; | \; x_{src}!\langle {Q} \rangle \red P\{\quotep{Q}/y}\} }
  \and \\
  \inferrule* [lab=Par] {{P} \red {P}'} {{{P} | {Q}} \red {{P}' | {Q}}}
  \and
  \inferrule* [lab=Equiv]{{{P} \scong {P}'} \andalso {{P}' \red {Q}'} \andalso {{Q}' \scong {Q}}}{{P} \red {Q}}
\end{mathpar}

\begin{eqnarray*}
  match_{\equiv} (\quotep{P},\quotep{Q}) & := & P \equiv Q \\
  match_{\dagger}(\quotep{P},\quotep{Q}) & := & \forall R. P|Q \red^{*} R => R \red^{*} 0 \\
  match_{K}(\quotep{P},\quotep{Q}) & := & K \mbox{ for some context } K
\end{eqnarray*}

$u?(x)P | u!\langle Q \rangle \red P\{\quotep{Q}/x\}$

%We write $\wred$ for $\red^*$, and $P\red$ if $\exists Q $ such that $ P \red Q$.
We write $P\red$ if $\exists Q $ such that $ P \red Q$ and $P\not\red$, otherwise.

\section{Replication}

As mentioned before, it is known that replication (and hence
recursion) can be implemented in a higher-order process algebra
\cite{SangiorgiWalker}. As our first example of calculation with the
machinery thus far presented we give the construction explicitly in
the {\rhoc}.

\begin{eqnarray}
	D_{x} & := & \prefix{x}{y}{(\binpar{\outputp{x}{y}}{@{y}})} \nonumber\\
	\bangp_{x}{P} & := & \binpar{{x}!\langle{\binpar{D_{x}}{P}}\rangle}{D_{x}} \nonumber
\end{eqnarray}

\begin{eqnarray}
	\bangp_{x}{P} & & \nonumber\\
	=
	& {x}!\langle{(\prefix{x}{y}{(\outputp{x}{y} | @{y})) | P}}\rangle 
	      | \prefix{x}{y}{(\outputp{x}{y} | @{y})} & \nonumber\\
	\red
	& (\outputp{x}{y} | @{y})\substn{\quotep{(\prefix{x}{y}{(@{y} | \outputp{x}{y})) | P}}}{y} & \nonumber\\
	=
	& \outputp{x}{\quotep{(\prefix{x}{y}{(\outputp{x}{y} | @{y})) | P}}}
	  | {(\prefix{x}{y}{(\outputp{x}{y} | @{y})) | P}} & \nonumber\\
	\red
	& \ldots & \nonumber\\
	\red^*
	& P | P | \ldots & \nonumber
\end{eqnarray}

Of course, this encoding, as an implementation, runs away, unfolding
$\bangp{P}$ eagerly. A lazier and more implementable replication
operator, restricted to input-guarded processes, may be obtained as follows.

\begin{eqnarray}
\bangp{\prefix{u}{v}{P}} 
	:= 
	\binpar{\lift{x}{\prefix{u}{v}{(\binpar{D(x)}{P})}}}{D(x)} \nonumber
\end{eqnarray}

\begin{remark}
  Note that the lazier definition still does not deal with summation
  or mixed summation (i.e. sums over input and output). The reader is
  invited to construct definitions of replication that deal with these
  features. 

  Further, the definitions are parameterized in a name, $x$. Can you,
  gentle reader, make a definition that eliminates this parameter and
  guarantees no accidental interaction between the replication
  machinery and the process being replicated -- i.e. no accidental
  sharing of names used by the process to get its work done and the
  name(s) used by the replication to effect copying. This latter
  revision of the definition of replication is crucial to obtaining
  the expected identity $!!P \sim !P$.
\end{remark}

\begin{remark}\label{rem:paradoxical_combinator}
  The reader familiar with the lambda calculus will have noticed the
  similarity between $D$ and the paradoxical combinator.

  [Ed. note: the existence of this seems to suggest we have to be more
  restrictive on the set of processes and names we admit if we are to
  support no-cloning.]
\end{remark}

\subsubsection{Bisimulation}

The computational dynamics gives rise to another kind of equivalence,
the equivalence of computational behavior. As previously mentioned
this is typically captured \emph{via} some form of bisimulation.

% The notion we use in this paper is weak barbed bisimulation
% \cite{milner91polyadicpi}.

The notion we use in this paper is derived from weak barbed
bisimulation \cite{milner91polyadicpi}. 

\begin{definition}
An \emph{observation relation}, $\downarrow_{\mathcal N}$, over a set
of names, $\mathcal N$, is the smallest relation satisfying the rules
below.

\infrule[Out-barb]{y \in {\mathcal N}, \; x \nameeq y}
		  {\outputp{x}{v} \downarrow_{\mathcal N} x}
\infrule[Par-barb]{\mbox{$P\downarrow_{\mathcal N} x$ or $Q\downarrow_{\mathcal N} x$}}
		  {\binpar{P}{Q} \downarrow_{\mathcal N} x}

We write $P \Downarrow_{\mathcal N} x$ if there is $Q$ such that 
$P \wred Q$ and $Q \downarrow_{\mathcal N} x$.
\end{definition}

\begin{definition}
%\label{def.bbisim}
An  ${\mathcal N}$-\emph{barbed bisimulation} over a set of names, ${\mathcal N}$, is a symmetric binary relation 
${\mathcal S}_{\mathcal N}$ between agents such that $P\rel{S}_{\mathcal N}Q$ implies:
\begin{enumerate}
\item If $P \red P'$ then $Q \wred Q'$ and $P'\rel{S}_{\mathcal N} Q'$.
\item If $P\downarrow_{\mathcal N} x$, then $Q\Downarrow_{\mathcal N} x$.
\end{enumerate}
$P$ is ${\mathcal N}$-barbed bisimilar to $Q$, written
$P \wbbisim_{\mathcal N} Q$, if $P \rel{S}_{\mathcal N} Q$ for some ${\mathcal N}$-barbed bisimulation ${\mathcal S}_{\mathcal N}$.
\end{definition}

$\mathcal{R} \subseteq \pi \times \pi$

$P \mathcal{R} Q => \forall P'. P \red P' \Rightarrow \exists Q'. Q \red Q', P' \mathcal{R} Q'$

$P \vdash x \Rightarrow Q \vdash x$

\begin{mathpar}
  \inferrule*[lab=Out-barb]{x \nameeq y}{{y}!\langle{Q}\rangle \vdash x}
  \and
  \inferrule*[lab=Par-barb]{\mbox{$P\vdash x$ or $Q\vdash x$}}{\binpar{P}{Q} \vdash x}
\end{mathpar}

\subsubsection{Contexts}

One of the principle advantages of computational calculi like the
$\pi$-calculus is a well-defined notion of context,
contextual-equivalence and a correlation between
contextual-equivalence and notions of bisimulation. The notion of
context allows the decomposition of a process into (sub-)process and
its syntactic environment, its context. Thus, a context may be
thought of as a process with a ``hole'' (written $\Box$) in it. The
application of a context $M$ to a process $P$, written $M[P]$, is
tantamount to filling the hole in $M$ with $P$. In this paper we do
not need the full weight of this theory, but do make use of the notion
of context in the proof the main theorem. 

\begin{mathpar}
  \inferrule* [lab=summation] {} {{M_{M},M_{N}} \bc \Box \;|\; x.M_{A} \;|\; M_{M}+M_{N}}
  \and
  \inferrule* [lab=agent] {} {{M_{A}} \bc (\vec{x})M_{P} \;| \; \clift{P_0,\ldots,M_{P},\ldots,P_N}}
  \and \\
  \inferrule* [lab=process] {} {{M_{P}} \bc M_{N} \;| \;P|M_{P} }
\end{mathpar} 

\begin{mathpar}
  \inferrule* [lab=sychronization] {} {M_{N} \bc \Box \;|\; x?M_{F} \;|\; x!M_{C}}
  \and
  \inferrule* [lab=abstraction] {} {{M_{F}} \bc (x)M_{P} }
  \and
  \inferrule* [lab=concretion] {} {{M_{C}} \bc \langle M_{P} \rangle }
  \and \\
  \inferrule* [lab=process] {} {{M_{P}} \bc M_{N} \;| \;P|M_{P} }
\end{mathpar}

\begin{definition}[contextual application] Given a context $M$, and
  process $P$, we define the \emph{contextual application}, $M[P] :=
  M\{P/\Box\}$. That is, the contextual application of M to P is the
  substitution of $P$ for $\Box$ in $M$.
\end{definition}

$\meaningof{-} : L \to \mathcal{P}(\pi)$

\begin{mathpar}
  \inferrule* [lab=collection] {} {\meaningof{true} = \pi, \and \meaningof{~E} = \pi \setminus \meaningof{E}, \and \meaningof{E_{1} \& E_{2}} = \meaningof{E_{1}} \cap \meaningof{E_{2}}}
\end{mathpar}

\begin{mathpar}
  \inferrule* [lab=structure] {} {\meaningof{0} = \{ P \in \pi | P \equiv 0 \}, \and \\ \meaningof{E_1 | E_2} = \{ P \in \pi | P \equiv P_{1} | P_{2}, P_{1} \in \meaningof{E_{1}}, P_{2} \in \meaningof{E_2}\} }
\end{mathpar}

\begin{mathpar}
 \inferrule* [lab=behavior] {} {\meaningof{\langle a?b \rangle E} = \{ P \in \pi | P \equiv Q | u?(y)P', \\ \and \\\\ \and \\ \;\;\; u \in \meaningof{a}, \forall z.P'\{z/y\} \in \meaningof{E\{z/b\}}\}, \and \\ \meaningof{a!E} = \{ P \in \pi | P \equiv Q | x!\langle P' \rangle, x \in \meaningof{a} P' \in \meaningof{E}\} }
\end{mathpar}

\begin{mathpar}
 \inferrule* [lab=nominal] {} {\meaningof{\quotep{E}} = \{ \quotep{P} \in \quotep{\pi} | P \in \meaningof{E} \}, \and \meaningof{\quotep{P}} = \{ \quotep{Q} \in \quotep{\pi} | P \equiv Q \} \and \\ \meaningof{@\quotep{E}} = \{ P \in \pi | P \equiv @x, x \in \meaningof{E} \}}
\end{mathpar}

\begin{eqnarray*}
  \\
  \meaningof{-} : TS \to ST
\end{eqnarray*}

\begin{eqnarray*}
  \\
  L : TS \to ST
\end{eqnarray*}

\begin{eqnarray*}
  \\
  P \models E \iff P \in \meaningof{E}
\end{eqnarray*}

\begin{eqnarray*}
  P \approx_{L} Q \iff \forall E \in L. P \models E \iff Q \models E
\end{eqnarray*}

\begin{eqnarray*}
  P \approx_{K} Q
\end{eqnarray*}

\begin{eqnarray*}
  P \approx Q
\end{eqnarray*}

$\approx_{K} = \approx = \approx_{L}$

\subsubsection{Contextual duality}

Note that contexts extend the quotation operation to a family of
operations from processes to names. Given a context, $M$, we can
define a \emph{nominal context}, $\quotep{M}$ by $\quotep{M}[P] :=
\quotep{M[P]}$. To foreshadow what is to come we observe that these
operations enjoy a duality with processes very much like the duality
between vectors and maps from vectors to scalars.

Further, because the calculus is essentially higher-order, we have a
correspondence between contexts and processes. More specifically,
given a name $x$ and a context $M$ we can construct $M^{*}_{x}$ such
that 

\begin{mathpar}
  M^{*}_{x} | \lift{x}{P} \red M[P]
\end{mathpar}

namely,

\begin{mathpar}
  M^{*}_{x} := x?(u).M[\dropn{u}]
\end{mathpar}

The dependence of $M^{*}_{x}$ on a name makes it an abstraction, 

\begin{mathpar}
  M^{*} := (x)x?(u).M[\dropn{u}]
\end{mathpar}

\subsection{Additional notation}

It will sometimes be convenient to denote the process a name
quotes. We already have the notation $x = \quotep{P}$, but it will be
convenient to introduce an alternate notation, $\procn{x}$, when we
want to emphasize the connection to the use of the name. Note that, by
virtue of name equivalence, $\quotep{\procn{x}} \nameeq x$; so, the
notation is consistent with previous definitions.

Further, because names have structure it is possible to effect
substitutions on the basis of that structure. This means we need to
upgrade our notation for substitutions, which we accomplish by
adapting comprehension notation. Thus,

\begin{mathpar}
  P\{ y / x : x \in S \}
\end{mathpar}

is interpreted to mean the process derived from P by replacing (in a
capture-avoiding manner) each occurrence of $x$ in $S$ by $y$. For example,

\begin{mathpar}
  P\{ \quotep{\procn{x}|\procn{x}} / x : x \in \freenames{P} \}
\end{mathpar}

will replace each (occurrence) of a free name $x$ in $P$ by
$\quotep{\procn{x}|\procn{x}}$.

Also, we will avail ourselves of the notation $x^{L}$ and $x^{R}$ to
denote injections of a name into disjoint copies of the name
space. There are numerous ways to accomplish this. One example can be
found in \cite{MeredithR05}. This notation overloads to vectors of
names: $\vec{x}^{\pi} := (x_{i}^{\pi} \; : \; 0 \leq i < |\vec{x}| )$ where $\pi \in \{L,R\}$.

We also use $P^{\Box} := P|\Box$.

In \cite{MeredithR05} an interpretation of the new operator is
given. It turns out that there are several possible interpretations
all enjoying the requisite algebraic properties of the operator (see
\cite{milner91polyadicpi}). We will therefore make liberal use of
$(\nu\; \vec{x})P$.

% subsection the_syntax_and_semantics_of_the_notation_system (end)   

\input{qm2pi.qmops} 

\input{qm2pi.sterngerlach} 

\input{qm2pi.metric} 

% section concurrent_process_calculi (end)

%\input{qm2pi.proofsketch}

% section proof sketch (end)

%\input{qm2pi.slviaknots} 

% section spatial logic via knots (end)

\input{qm2pi.conclusion}

% section conclusion (end)

%\input{qm2pi.dtcodes} 

% section wiring algorithm (end)

\input{qm2pi.ack} 

% section acknowledgments (end)

\newpage


\bibliographystyle{plain}   
\bibliography{../../biblios/main.bib}

\input{qm2pi.rhodetails}

\end{document}

 

% section wiring algorithm (end)

\documentclass[12pt]{llncs}
%\documentclass{jktr}

\usepackage[pdftex]{hyperref}                   
\usepackage {listings}
\usepackage {mathpartir}
\usepackage{bcprules}
%\usepackage{listings}
                       
\usepackage{graphicx} 
%\usepackage[margins=2.5cm,nohead,nofoot]{geometry}
%\usepackage{geometry}
\usepackage{amsfonts}
\usepackage{amstext}
\usepackage{latexsym}
\usepackage{amssymb}
\usepackage{color}


%\include{myPreamble}
\include{qm2pi.local} 

%\ifpdf
%\usepackage[pdftex]{graphicx}
%\else
%\usepackage{graphicx}
%\fi

 % \ifpdf
%  \usepackage{pdfsync}
%  \if


%\title{Brief Article}
%\author{David F. Snyder}
%\author{L.G. Meredith}

%\address{Dept. of Math., Texas State University--San Marcos, San Marcos, TX 78666}
       
\pagestyle{empty}


\begin{document}

\lstset{language=[Objective]Caml,frame=shadowbox}

\input{qm2pi.front}

% section front matter (end)

\input{qm2pi.intro} 
 
% section introduction (end)

% \input{qm2pi.knotations} 

% section notation (end)

\input{qm2pi.process.calculi} 

% section concurrent_process_calculi_and_spatial_logics_ (end)
    
%\input{qm2pi.knots2pi} 

%\input{qm2pi.trefoil} 

%\input{qm2pi.mainthm} 

% subsection basic_interpretation (end)

%\input{qm2pi.rho.presentation} 
\subsection{The syntax and semantics of the notation system}\label{sub:the_syntax_and_semantics_of_the_notation_system} % (fold)

We now summarize a technical presentation of the calculus that
embodies our theory of dynamics. The typical presentation of such a
calculus follows the style of giving generators and relations on
them. The grammar, below, describing term constructors, freely
generates the set of processes, $\Proc$. This set is then quotiented
by a relation known as structural congruence and it is over this set
that the notion of dynamics is expressed. This presentation is
essentially that of \cite{MeredithR05} with the addition of
polyadicity and summation. For readability we have relegated some of
the technical subtleties to an appendix.

\subsubsection{Process grammar}\label{subsub:process_grammar}

\begin{mathpar}
  \inferrule* [lab=synchronization] {} {{M} \bc \pzero \;|\; x?F \;|\; x!C }
  \and
  \inferrule* [lab=abstraction] {} {{F} \bc (x)P}
  \and
  \inferrule* [lab=concretion] {} {{C} \bc \langle Q \rangle}
  \and
  \inferrule* [lab=process] {} {{P,Q} \bc M \;| \;P|Q \;|\; @{x}}
  \and
  \inferrule* [lab=name] {} {{x} \bc \quotep{P}}
\end{mathpar} 

Note that $\vec{x}$ (resp. $\vec{P}$) denotes a vector of names
(resp. processes) of length $|\vec{x}|$ (resp. $|\vec{P}|$). We adopt
the following useful abbreviations.

\begin{mathpar}
   x?(\vec{y}).P := x.(\vec{y})P \and  x\clift{\vec{P}} := x.\clift{\vec{P}}
   \and x!(y) := \lift{x}{\dropn{y}}
   \and \Pi_{i=0}^{n-1}P_i := P_0 | \ldots | P_{n-1}
\end{mathpar}

\subsubsection{Structural congruence}

\paragraph{Free and bound names and alpha-equivalence.} At the
core of structural equivalence is alpha-equivalence which identifies
process that are the same up to a change of variable. Formally, we
recognize the distinction between free and bound names. The free names
of a process, $\freenames{P}$, may be calculated recursively as
follows:

\begin{mathpar}
\freenames{\pzero} := \emptyset
  \and \\
  \freenames{x?(y).P} := \{ x \} \cup (\freenames{P} \setminus \{ y \})
  \and 
  \freenames{x!\langle P \rangle} := \{ x \} \cup \{ P \} 
  \and \\
  \freenames{P|Q} := \freenames{P} \cup \freenames{Q}
  \and \\
  \freenames{@{x}} := \{ x \}
\end{mathpar}

$\pi$
$\quotep{\pi}$

$\freenames{-} : \pi \to \mathcal{P}(\quotep{\pi})$

\begin{eqnarray*}
  \freenames{\pzero} & := & \emptyset \\
  \freenames{x?(y).P} & := & \{ x \} \cup (\freenames{P} \setminus \{ y \}) \\
  \freenames{x!\langle P \rangle} & := & \{ x \} \cup \{ P \} \\
  \freenames{P|Q} & := & \freenames{P} \cup \freenames{Q} \\
  \freenames{\dropn{x}} & := & \{ x \}
\end{eqnarray*}

The bound names of a process, $\boundnames{P}$, are those names occurring in $P$
that are not free. For example, in $x?(y).0$, the name $x$ is free, while $y$ is bound.

\begin{mathpar}
  \inferrule* [lab=monoidal-laws] {} { P|Q \equiv Q|P \and P|0 \equiv P \and P|(Q|R) \equiv (P|Q)|R }
\end{mathpar}

\begin{mathpar}
  \inferrule* [lab=alpha-equivalence] {} { (x)P \equiv (y)P\{y/x\} \and y \not\in \freenames{P} }
\end{mathpar}

\begin{definition}
Then two processes, $P,Q$, are alpha-equivalent if $P = Q\{\vec{y}/\vec{x}\}$ for
some $\vec{x} \in \boundnames{Q},\vec{y} \in \boundnames{P}$, where $Q\{\vec{y}/\vec{x}\}$
denotes the capture-avoiding substitution of $\vec{y}$ for $\vec{x}$ in $Q$.
\end{definition}

\begin{definition}
  The {\em structural congruence} \cite{SangiorgiWalker} , $\equiv$,
  between processes is the least congruence containing
  alpha-equivalence, satisfying the abelian monoid laws
  (associativity, commutativity and $\pzero$ as identity) for parallel
  composition $|$ and for summation $+$.
\end{definition}

\subsection{Name equivalence}

We take name equivalence, written $\nameeq$, to be the smallest
equivalence relation generated by the following rules.

\begin{mathpar}
\inferrule*[lab=Quote-drop]
{ }
{ \quotep{@{x}} \nameeq x }

\inferrule*[lab=Struct-equiv]
{ P \scong Q }
{ \quotep{P} \nameeq \quotep{Q} }
\end{mathpar}

The astute reader will have noticed that the mutual recursion of names
and processes imposes a mutual recursion on alpha-equivalence and
structural equivalence via name-equivalence. Fortunately, all of this
works out pleasantly and we may calculate in the natural way, free of
concern. The reader interested in the details is referred to the
appendix \ref{appendix:rho_details}.

\subsection{Substitution}

We use $\Proc$ for the set of processes, $\QProc$ for the set of
names, and $\id{\{}\vec{y} / \vec{x} \id{\}}$ to denote partial maps,
$s : \QProc \rightarrow \QProc$. A map, $s$ lifts, uniquely, to a map
on process terms, $\widehat{s} : \Proc \rightarrow \Proc$ by the
following equations.

\begin{mathpar}
  (0) \psubstp{Q}{P} := 0 \\
  (R \juxtap S) \psubstp{Q}{P}
  :=    
  (R)\psubstp{Q}{P} \juxtap (S) \psubstp{Q}{P} \\
  (x?(y).R) \psubstp{Q}{P}    
  :=    
  (x)\substp{Q}{P} (z)\concat( (R \psubstn{z}{y}) \psubstp{Q}{P} ) \\
  (\lift{x}{R}) \psubstp{Q}{P}  
  :=
  \lift{(x)\substp{Q}{P}}{ R \psubstp{Q}{P} } \\
%   (\dropn{x})  \psubstp{Q}{P}       
%   := 
%   \left\{ 
%     \begin{array}{ccc} 
%       \dropn{\quotep{Q}} & & x \nameeq \quotep{P} \\
%       \dropn{x} & & otherwise \\
%     \end{array}
%   \right. 
  (\dropn{x})  \psubstp{Q}{P}       
  := 
  \left\{ 
    \begin{array}{ccc} 
      Q & & x \nameeq \quotep{P} \\
      \dropn{x} & & otherwise \\
    \end{array}
  \right.
\end{mathpar}
 

where

\begin{eqnarray}
  (x)\id{\{} \lpquote Q \rpquote / \lpquote P \rpquote \id{\}}            = 
  \left\{ 
    \begin{array}{ccc}
      \lpquote Q \rpquote & & x \nameeq \lpquote P \rpquote \\
      x & & otherwise \\
    \end{array}
  \right. \nonumber
\end{eqnarray}

and $z$ is chosen distinct from $\quotep{P}$, $\quotep{Q}$, the free
names in $Q$, and all the names in $R$. Our $\alpha$-equivalence will
be built in the standard way from this substitution.

\begin{remark}\label{rem:no_self_referential_names}
  One consequence of these definitions is that $\forall P. \quotep{P}
  \not\in \freenames{P}$.
\end{remark}

\subsection{ Dynamic quote: an example }

Anticipating something of what's to come, consider applying the
substitution, $\widehat{\id{\{}u / z \id{\}}}$, to the following pair
of processes, $\lift{w}{y!(z)}$ and $w[ \lpquote y!(z) \rpquote ]$.

\begin{eqnarray}
	\lift{w}{y!(z)}\widehat{\id{\{}u / z \id{\}}}
		& = &
		\lift{w}{y!(u)} \nonumber\\
	w[ \lpquote y!(z) \rpquote ] \widehat{ \id{\{}u / z \id{\}} }
		& = &
		w[ \lpquote y!(z) \rpquote ] \nonumber
\end{eqnarray}

Because the body of the process between quotes is impervious to
substitution, we get radically different answers. In fact, by
examining the first process in an input context,
e.g. $x?(z).\lift{w}{y!(z)}$, we see that the process under the lift
operator may be shaped by prefixed inputs binding a name inside it. In
this sense, the lift operator will be seen as a way to dynamically
construct processes before reifying them as names.

Finally equipped with these standard features we can present the
dynamics of the calculus.

\subsubsection{Operational semantics} 

Finally, we introduce the computational dynamics. What marks these
algebras as distinct from other more traditionally studied algebraic
structures, e.g. vector spaces or polynomial rings, is the manner in
which dynamics is captured. In traditional structures, dynamics is typically
expressed through morphisms between such structures, as in linear maps
between vector spaces or morphisms between rings. In algebras
associated with the semantics of computation, the dynamics is
expressed as part of the algebraic structure itself, through a
reduction reduction relation typically denoted by $\red$. Below, we
give a recursive presentation of this relation for the calculus used
in the encoding.

$\red \subseteq \pi \times \pi$
$\red : \pi \to \mathcal{P}(\pi)$

\begin{mathpar}
  \inferrule* [lab=Comm] { \textsf{match}( x_{src}, x_{trgt} ) } { x_{trgt}?(y)P \; | \; x_{src}!\langle {Q} \rangle \red P\{\quotep{Q}/y}\} }
  \and \\
  \inferrule* [lab=Par] {{P} \red {P}'} {{{P} | {Q}} \red {{P}' | {Q}}}
  \and
  \inferrule* [lab=Equiv]{{{P} \scong {P}'} \andalso {{P}' \red {Q}'} \andalso {{Q}' \scong {Q}}}{{P} \red {Q}}
\end{mathpar}

\begin{eqnarray*}
  match_{\equiv} (\quotep{P},\quotep{Q}) & := & P \equiv Q \\
  match_{\dagger}(\quotep{P},\quotep{Q}) & := & \forall R. P|Q \red^{*} R => R \red^{*} 0 \\
  match_{K}(\quotep{P},\quotep{Q}) & := & K \mbox{ for some context } K
\end{eqnarray*}

$u?(x)P | u!\langle Q \rangle \red P\{\quotep{Q}/x\}$

%We write $\wred$ for $\red^*$, and $P\red$ if $\exists Q $ such that $ P \red Q$.
We write $P\red$ if $\exists Q $ such that $ P \red Q$ and $P\not\red$, otherwise.

\section{Replication}

As mentioned before, it is known that replication (and hence
recursion) can be implemented in a higher-order process algebra
\cite{SangiorgiWalker}. As our first example of calculation with the
machinery thus far presented we give the construction explicitly in
the {\rhoc}.

\begin{eqnarray}
	D_{x} & := & \prefix{x}{y}{(\binpar{\outputp{x}{y}}{@{y}})} \nonumber\\
	\bangp_{x}{P} & := & \binpar{{x}!\langle{\binpar{D_{x}}{P}}\rangle}{D_{x}} \nonumber
\end{eqnarray}

\begin{eqnarray}
	\bangp_{x}{P} & & \nonumber\\
	=
	& {x}!\langle{(\prefix{x}{y}{(\outputp{x}{y} | @{y})) | P}}\rangle 
	      | \prefix{x}{y}{(\outputp{x}{y} | @{y})} & \nonumber\\
	\red
	& (\outputp{x}{y} | @{y})\substn{\quotep{(\prefix{x}{y}{(@{y} | \outputp{x}{y})) | P}}}{y} & \nonumber\\
	=
	& \outputp{x}{\quotep{(\prefix{x}{y}{(\outputp{x}{y} | @{y})) | P}}}
	  | {(\prefix{x}{y}{(\outputp{x}{y} | @{y})) | P}} & \nonumber\\
	\red
	& \ldots & \nonumber\\
	\red^*
	& P | P | \ldots & \nonumber
\end{eqnarray}

Of course, this encoding, as an implementation, runs away, unfolding
$\bangp{P}$ eagerly. A lazier and more implementable replication
operator, restricted to input-guarded processes, may be obtained as follows.

\begin{eqnarray}
\bangp{\prefix{u}{v}{P}} 
	:= 
	\binpar{\lift{x}{\prefix{u}{v}{(\binpar{D(x)}{P})}}}{D(x)} \nonumber
\end{eqnarray}

\begin{remark}
  Note that the lazier definition still does not deal with summation
  or mixed summation (i.e. sums over input and output). The reader is
  invited to construct definitions of replication that deal with these
  features. 

  Further, the definitions are parameterized in a name, $x$. Can you,
  gentle reader, make a definition that eliminates this parameter and
  guarantees no accidental interaction between the replication
  machinery and the process being replicated -- i.e. no accidental
  sharing of names used by the process to get its work done and the
  name(s) used by the replication to effect copying. This latter
  revision of the definition of replication is crucial to obtaining
  the expected identity $!!P \sim !P$.
\end{remark}

\begin{remark}\label{rem:paradoxical_combinator}
  The reader familiar with the lambda calculus will have noticed the
  similarity between $D$ and the paradoxical combinator.

  [Ed. note: the existence of this seems to suggest we have to be more
  restrictive on the set of processes and names we admit if we are to
  support no-cloning.]
\end{remark}

\subsubsection{Bisimulation}

The computational dynamics gives rise to another kind of equivalence,
the equivalence of computational behavior. As previously mentioned
this is typically captured \emph{via} some form of bisimulation.

% The notion we use in this paper is weak barbed bisimulation
% \cite{milner91polyadicpi}.

The notion we use in this paper is derived from weak barbed
bisimulation \cite{milner91polyadicpi}. 

\begin{definition}
An \emph{observation relation}, $\downarrow_{\mathcal N}$, over a set
of names, $\mathcal N$, is the smallest relation satisfying the rules
below.

\infrule[Out-barb]{y \in {\mathcal N}, \; x \nameeq y}
		  {\outputp{x}{v} \downarrow_{\mathcal N} x}
\infrule[Par-barb]{\mbox{$P\downarrow_{\mathcal N} x$ or $Q\downarrow_{\mathcal N} x$}}
		  {\binpar{P}{Q} \downarrow_{\mathcal N} x}

We write $P \Downarrow_{\mathcal N} x$ if there is $Q$ such that 
$P \wred Q$ and $Q \downarrow_{\mathcal N} x$.
\end{definition}

\begin{definition}
%\label{def.bbisim}
An  ${\mathcal N}$-\emph{barbed bisimulation} over a set of names, ${\mathcal N}$, is a symmetric binary relation 
${\mathcal S}_{\mathcal N}$ between agents such that $P\rel{S}_{\mathcal N}Q$ implies:
\begin{enumerate}
\item If $P \red P'$ then $Q \wred Q'$ and $P'\rel{S}_{\mathcal N} Q'$.
\item If $P\downarrow_{\mathcal N} x$, then $Q\Downarrow_{\mathcal N} x$.
\end{enumerate}
$P$ is ${\mathcal N}$-barbed bisimilar to $Q$, written
$P \wbbisim_{\mathcal N} Q$, if $P \rel{S}_{\mathcal N} Q$ for some ${\mathcal N}$-barbed bisimulation ${\mathcal S}_{\mathcal N}$.
\end{definition}

$\mathcal{R} \subseteq \pi \times \pi$

$P \mathcal{R} Q => \forall P'. P \red P' \Rightarrow \exists Q'. Q \red Q', P' \mathcal{R} Q'$

$P \vdash x \Rightarrow Q \vdash x$

\begin{mathpar}
  \inferrule*[lab=Out-barb]{x \nameeq y}{{y}!\langle{Q}\rangle \vdash x}
  \and
  \inferrule*[lab=Par-barb]{\mbox{$P\vdash x$ or $Q\vdash x$}}{\binpar{P}{Q} \vdash x}
\end{mathpar}

\subsubsection{Contexts}

One of the principle advantages of computational calculi like the
$\pi$-calculus is a well-defined notion of context,
contextual-equivalence and a correlation between
contextual-equivalence and notions of bisimulation. The notion of
context allows the decomposition of a process into (sub-)process and
its syntactic environment, its context. Thus, a context may be
thought of as a process with a ``hole'' (written $\Box$) in it. The
application of a context $M$ to a process $P$, written $M[P]$, is
tantamount to filling the hole in $M$ with $P$. In this paper we do
not need the full weight of this theory, but do make use of the notion
of context in the proof the main theorem. 

\begin{mathpar}
  \inferrule* [lab=summation] {} {{M_{M},M_{N}} \bc \Box \;|\; x.M_{A} \;|\; M_{M}+M_{N}}
  \and
  \inferrule* [lab=agent] {} {{M_{A}} \bc (\vec{x})M_{P} \;| \; \clift{P_0,\ldots,M_{P},\ldots,P_N}}
  \and \\
  \inferrule* [lab=process] {} {{M_{P}} \bc M_{N} \;| \;P|M_{P} }
\end{mathpar} 

\begin{mathpar}
  \inferrule* [lab=sychronization] {} {M_{N} \bc \Box \;|\; x?M_{F} \;|\; x!M_{C}}
  \and
  \inferrule* [lab=abstraction] {} {{M_{F}} \bc (x)M_{P} }
  \and
  \inferrule* [lab=concretion] {} {{M_{C}} \bc \langle M_{P} \rangle }
  \and \\
  \inferrule* [lab=process] {} {{M_{P}} \bc M_{N} \;| \;P|M_{P} }
\end{mathpar}

\begin{definition}[contextual application] Given a context $M$, and
  process $P$, we define the \emph{contextual application}, $M[P] :=
  M\{P/\Box\}$. That is, the contextual application of M to P is the
  substitution of $P$ for $\Box$ in $M$.
\end{definition}

$\meaningof{-} : L \to \mathcal{P}(\pi)$

\begin{mathpar}
  \inferrule* [lab=collection] {} {\meaningof{true} = \pi, \and \meaningof{~E} = \pi \setminus \meaningof{E}, \and \meaningof{E_{1} \& E_{2}} = \meaningof{E_{1}} \cap \meaningof{E_{2}}}
\end{mathpar}

\begin{mathpar}
  \inferrule* [lab=structure] {} {\meaningof{0} = \{ P \in \pi | P \equiv 0 \}, \and \\ \meaningof{E_1 | E_2} = \{ P \in \pi | P \equiv P_{1} | P_{2}, P_{1} \in \meaningof{E_{1}}, P_{2} \in \meaningof{E_2}\} }
\end{mathpar}

\begin{mathpar}
 \inferrule* [lab=behavior] {} {\meaningof{\langle a?b \rangle E} = \{ P \in \pi | P \equiv Q | u?(y)P', \\ \and \\\\ \and \\ \;\;\; u \in \meaningof{a}, \forall z.P'\{z/y\} \in \meaningof{E\{z/b\}}\}, \and \\ \meaningof{a!E} = \{ P \in \pi | P \equiv Q | x!\langle P' \rangle, x \in \meaningof{a} P' \in \meaningof{E}\} }
\end{mathpar}

\begin{mathpar}
 \inferrule* [lab=nominal] {} {\meaningof{\quotep{E}} = \{ \quotep{P} \in \quotep{\pi} | P \in \meaningof{E} \}, \and \meaningof{\quotep{P}} = \{ \quotep{Q} \in \quotep{\pi} | P \equiv Q \} \and \\ \meaningof{@\quotep{E}} = \{ P \in \pi | P \equiv @x, x \in \meaningof{E} \}}
\end{mathpar}

\begin{eqnarray*}
  \\
  \meaningof{-} : TS \to ST
\end{eqnarray*}

\begin{eqnarray*}
  \\
  L : TS \to ST
\end{eqnarray*}

\begin{eqnarray*}
  \\
  P \models E \iff P \in \meaningof{E}
\end{eqnarray*}

\begin{eqnarray*}
  P \approx_{L} Q \iff \forall E \in L. P \models E \iff Q \models E
\end{eqnarray*}

\begin{eqnarray*}
  P \approx_{K} Q
\end{eqnarray*}

\begin{eqnarray*}
  P \approx Q
\end{eqnarray*}

$\approx_{K} = \approx = \approx_{L}$

\subsubsection{Contextual duality}

Note that contexts extend the quotation operation to a family of
operations from processes to names. Given a context, $M$, we can
define a \emph{nominal context}, $\quotep{M}$ by $\quotep{M}[P] :=
\quotep{M[P]}$. To foreshadow what is to come we observe that these
operations enjoy a duality with processes very much like the duality
between vectors and maps from vectors to scalars.

Further, because the calculus is essentially higher-order, we have a
correspondence between contexts and processes. More specifically,
given a name $x$ and a context $M$ we can construct $M^{*}_{x}$ such
that 

\begin{mathpar}
  M^{*}_{x} | \lift{x}{P} \red M[P]
\end{mathpar}

namely,

\begin{mathpar}
  M^{*}_{x} := x?(u).M[\dropn{u}]
\end{mathpar}

The dependence of $M^{*}_{x}$ on a name makes it an abstraction, 

\begin{mathpar}
  M^{*} := (x)x?(u).M[\dropn{u}]
\end{mathpar}

\subsection{Additional notation}

It will sometimes be convenient to denote the process a name
quotes. We already have the notation $x = \quotep{P}$, but it will be
convenient to introduce an alternate notation, $\procn{x}$, when we
want to emphasize the connection to the use of the name. Note that, by
virtue of name equivalence, $\quotep{\procn{x}} \nameeq x$; so, the
notation is consistent with previous definitions.

Further, because names have structure it is possible to effect
substitutions on the basis of that structure. This means we need to
upgrade our notation for substitutions, which we accomplish by
adapting comprehension notation. Thus,

\begin{mathpar}
  P\{ y / x : x \in S \}
\end{mathpar}

is interpreted to mean the process derived from P by replacing (in a
capture-avoiding manner) each occurrence of $x$ in $S$ by $y$. For example,

\begin{mathpar}
  P\{ \quotep{\procn{x}|\procn{x}} / x : x \in \freenames{P} \}
\end{mathpar}

will replace each (occurrence) of a free name $x$ in $P$ by
$\quotep{\procn{x}|\procn{x}}$.

Also, we will avail ourselves of the notation $x^{L}$ and $x^{R}$ to
denote injections of a name into disjoint copies of the name
space. There are numerous ways to accomplish this. One example can be
found in \cite{MeredithR05}. This notation overloads to vectors of
names: $\vec{x}^{\pi} := (x_{i}^{\pi} \; : \; 0 \leq i < |\vec{x}| )$ where $\pi \in \{L,R\}$.

We also use $P^{\Box} := P|\Box$.

In \cite{MeredithR05} an interpretation of the new operator is
given. It turns out that there are several possible interpretations
all enjoying the requisite algebraic properties of the operator (see
\cite{milner91polyadicpi}). We will therefore make liberal use of
$(\nu\; \vec{x})P$.

% subsection the_syntax_and_semantics_of_the_notation_system (end)   

\input{qm2pi.qmops} 

\input{qm2pi.sterngerlach} 

\input{qm2pi.metric} 

% section concurrent_process_calculi (end)

%\input{qm2pi.proofsketch}

% section proof sketch (end)

%\input{qm2pi.slviaknots} 

% section spatial logic via knots (end)

\input{qm2pi.conclusion}

% section conclusion (end)

%\input{qm2pi.dtcodes} 

% section wiring algorithm (end)

\input{qm2pi.ack} 

% section acknowledgments (end)

\newpage


\bibliographystyle{plain}   
\bibliography{../../biblios/main.bib}

\input{qm2pi.rhodetails}

\end{document}

 

% section acknowledgments (end)

\newpage


\bibliographystyle{plain}   
\bibliography{../../biblios/main.bib}

\documentclass[12pt]{llncs}
%\documentclass{jktr}

\usepackage[pdftex]{hyperref}                   
\usepackage {listings}
\usepackage {mathpartir}
\usepackage{bcprules}
%\usepackage{listings}
                       
\usepackage{graphicx} 
%\usepackage[margins=2.5cm,nohead,nofoot]{geometry}
%\usepackage{geometry}
\usepackage{amsfonts}
\usepackage{amstext}
\usepackage{latexsym}
\usepackage{amssymb}
\usepackage{color}


%\include{myPreamble}
\include{qm2pi.local} 

%\ifpdf
%\usepackage[pdftex]{graphicx}
%\else
%\usepackage{graphicx}
%\fi

 % \ifpdf
%  \usepackage{pdfsync}
%  \if


%\title{Brief Article}
%\author{David F. Snyder}
%\author{L.G. Meredith}

%\address{Dept. of Math., Texas State University--San Marcos, San Marcos, TX 78666}
       
\pagestyle{empty}


\begin{document}

\lstset{language=[Objective]Caml,frame=shadowbox}

\input{qm2pi.front}

% section front matter (end)

\input{qm2pi.intro} 
 
% section introduction (end)

% \input{qm2pi.knotations} 

% section notation (end)

\input{qm2pi.process.calculi} 

% section concurrent_process_calculi_and_spatial_logics_ (end)
    
%\input{qm2pi.knots2pi} 

%\input{qm2pi.trefoil} 

%\input{qm2pi.mainthm} 

% subsection basic_interpretation (end)

%\input{qm2pi.rho.presentation} 
\subsection{The syntax and semantics of the notation system}\label{sub:the_syntax_and_semantics_of_the_notation_system} % (fold)

We now summarize a technical presentation of the calculus that
embodies our theory of dynamics. The typical presentation of such a
calculus follows the style of giving generators and relations on
them. The grammar, below, describing term constructors, freely
generates the set of processes, $\Proc$. This set is then quotiented
by a relation known as structural congruence and it is over this set
that the notion of dynamics is expressed. This presentation is
essentially that of \cite{MeredithR05} with the addition of
polyadicity and summation. For readability we have relegated some of
the technical subtleties to an appendix.

\subsubsection{Process grammar}\label{subsub:process_grammar}

\begin{mathpar}
  \inferrule* [lab=synchronization] {} {{M} \bc \pzero \;|\; x?F \;|\; x!C }
  \and
  \inferrule* [lab=abstraction] {} {{F} \bc (x)P}
  \and
  \inferrule* [lab=concretion] {} {{C} \bc \langle Q \rangle}
  \and
  \inferrule* [lab=process] {} {{P,Q} \bc M \;| \;P|Q \;|\; @{x}}
  \and
  \inferrule* [lab=name] {} {{x} \bc \quotep{P}}
\end{mathpar} 

Note that $\vec{x}$ (resp. $\vec{P}$) denotes a vector of names
(resp. processes) of length $|\vec{x}|$ (resp. $|\vec{P}|$). We adopt
the following useful abbreviations.

\begin{mathpar}
   x?(\vec{y}).P := x.(\vec{y})P \and  x\clift{\vec{P}} := x.\clift{\vec{P}}
   \and x!(y) := \lift{x}{\dropn{y}}
   \and \Pi_{i=0}^{n-1}P_i := P_0 | \ldots | P_{n-1}
\end{mathpar}

\subsubsection{Structural congruence}

\paragraph{Free and bound names and alpha-equivalence.} At the
core of structural equivalence is alpha-equivalence which identifies
process that are the same up to a change of variable. Formally, we
recognize the distinction between free and bound names. The free names
of a process, $\freenames{P}$, may be calculated recursively as
follows:

\begin{mathpar}
\freenames{\pzero} := \emptyset
  \and \\
  \freenames{x?(y).P} := \{ x \} \cup (\freenames{P} \setminus \{ y \})
  \and 
  \freenames{x!\langle P \rangle} := \{ x \} \cup \{ P \} 
  \and \\
  \freenames{P|Q} := \freenames{P} \cup \freenames{Q}
  \and \\
  \freenames{@{x}} := \{ x \}
\end{mathpar}

$\pi$
$\quotep{\pi}$

$\freenames{-} : \pi \to \mathcal{P}(\quotep{\pi})$

\begin{eqnarray*}
  \freenames{\pzero} & := & \emptyset \\
  \freenames{x?(y).P} & := & \{ x \} \cup (\freenames{P} \setminus \{ y \}) \\
  \freenames{x!\langle P \rangle} & := & \{ x \} \cup \{ P \} \\
  \freenames{P|Q} & := & \freenames{P} \cup \freenames{Q} \\
  \freenames{\dropn{x}} & := & \{ x \}
\end{eqnarray*}

The bound names of a process, $\boundnames{P}$, are those names occurring in $P$
that are not free. For example, in $x?(y).0$, the name $x$ is free, while $y$ is bound.

\begin{mathpar}
  \inferrule* [lab=monoidal-laws] {} { P|Q \equiv Q|P \and P|0 \equiv P \and P|(Q|R) \equiv (P|Q)|R }
\end{mathpar}

\begin{mathpar}
  \inferrule* [lab=alpha-equivalence] {} { (x)P \equiv (y)P\{y/x\} \and y \not\in \freenames{P} }
\end{mathpar}

\begin{definition}
Then two processes, $P,Q$, are alpha-equivalent if $P = Q\{\vec{y}/\vec{x}\}$ for
some $\vec{x} \in \boundnames{Q},\vec{y} \in \boundnames{P}$, where $Q\{\vec{y}/\vec{x}\}$
denotes the capture-avoiding substitution of $\vec{y}$ for $\vec{x}$ in $Q$.
\end{definition}

\begin{definition}
  The {\em structural congruence} \cite{SangiorgiWalker} , $\equiv$,
  between processes is the least congruence containing
  alpha-equivalence, satisfying the abelian monoid laws
  (associativity, commutativity and $\pzero$ as identity) for parallel
  composition $|$ and for summation $+$.
\end{definition}

\subsection{Name equivalence}

We take name equivalence, written $\nameeq$, to be the smallest
equivalence relation generated by the following rules.

\begin{mathpar}
\inferrule*[lab=Quote-drop]
{ }
{ \quotep{@{x}} \nameeq x }

\inferrule*[lab=Struct-equiv]
{ P \scong Q }
{ \quotep{P} \nameeq \quotep{Q} }
\end{mathpar}

The astute reader will have noticed that the mutual recursion of names
and processes imposes a mutual recursion on alpha-equivalence and
structural equivalence via name-equivalence. Fortunately, all of this
works out pleasantly and we may calculate in the natural way, free of
concern. The reader interested in the details is referred to the
appendix \ref{appendix:rho_details}.

\subsection{Substitution}

We use $\Proc$ for the set of processes, $\QProc$ for the set of
names, and $\id{\{}\vec{y} / \vec{x} \id{\}}$ to denote partial maps,
$s : \QProc \rightarrow \QProc$. A map, $s$ lifts, uniquely, to a map
on process terms, $\widehat{s} : \Proc \rightarrow \Proc$ by the
following equations.

\begin{mathpar}
  (0) \psubstp{Q}{P} := 0 \\
  (R \juxtap S) \psubstp{Q}{P}
  :=    
  (R)\psubstp{Q}{P} \juxtap (S) \psubstp{Q}{P} \\
  (x?(y).R) \psubstp{Q}{P}    
  :=    
  (x)\substp{Q}{P} (z)\concat( (R \psubstn{z}{y}) \psubstp{Q}{P} ) \\
  (\lift{x}{R}) \psubstp{Q}{P}  
  :=
  \lift{(x)\substp{Q}{P}}{ R \psubstp{Q}{P} } \\
%   (\dropn{x})  \psubstp{Q}{P}       
%   := 
%   \left\{ 
%     \begin{array}{ccc} 
%       \dropn{\quotep{Q}} & & x \nameeq \quotep{P} \\
%       \dropn{x} & & otherwise \\
%     \end{array}
%   \right. 
  (\dropn{x})  \psubstp{Q}{P}       
  := 
  \left\{ 
    \begin{array}{ccc} 
      Q & & x \nameeq \quotep{P} \\
      \dropn{x} & & otherwise \\
    \end{array}
  \right.
\end{mathpar}
 

where

\begin{eqnarray}
  (x)\id{\{} \lpquote Q \rpquote / \lpquote P \rpquote \id{\}}            = 
  \left\{ 
    \begin{array}{ccc}
      \lpquote Q \rpquote & & x \nameeq \lpquote P \rpquote \\
      x & & otherwise \\
    \end{array}
  \right. \nonumber
\end{eqnarray}

and $z$ is chosen distinct from $\quotep{P}$, $\quotep{Q}$, the free
names in $Q$, and all the names in $R$. Our $\alpha$-equivalence will
be built in the standard way from this substitution.

\begin{remark}\label{rem:no_self_referential_names}
  One consequence of these definitions is that $\forall P. \quotep{P}
  \not\in \freenames{P}$.
\end{remark}

\subsection{ Dynamic quote: an example }

Anticipating something of what's to come, consider applying the
substitution, $\widehat{\id{\{}u / z \id{\}}}$, to the following pair
of processes, $\lift{w}{y!(z)}$ and $w[ \lpquote y!(z) \rpquote ]$.

\begin{eqnarray}
	\lift{w}{y!(z)}\widehat{\id{\{}u / z \id{\}}}
		& = &
		\lift{w}{y!(u)} \nonumber\\
	w[ \lpquote y!(z) \rpquote ] \widehat{ \id{\{}u / z \id{\}} }
		& = &
		w[ \lpquote y!(z) \rpquote ] \nonumber
\end{eqnarray}

Because the body of the process between quotes is impervious to
substitution, we get radically different answers. In fact, by
examining the first process in an input context,
e.g. $x?(z).\lift{w}{y!(z)}$, we see that the process under the lift
operator may be shaped by prefixed inputs binding a name inside it. In
this sense, the lift operator will be seen as a way to dynamically
construct processes before reifying them as names.

Finally equipped with these standard features we can present the
dynamics of the calculus.

\subsubsection{Operational semantics} 

Finally, we introduce the computational dynamics. What marks these
algebras as distinct from other more traditionally studied algebraic
structures, e.g. vector spaces or polynomial rings, is the manner in
which dynamics is captured. In traditional structures, dynamics is typically
expressed through morphisms between such structures, as in linear maps
between vector spaces or morphisms between rings. In algebras
associated with the semantics of computation, the dynamics is
expressed as part of the algebraic structure itself, through a
reduction reduction relation typically denoted by $\red$. Below, we
give a recursive presentation of this relation for the calculus used
in the encoding.

$\red \subseteq \pi \times \pi$
$\red : \pi \to \mathcal{P}(\pi)$

\begin{mathpar}
  \inferrule* [lab=Comm] { \textsf{match}( x_{src}, x_{trgt} ) } { x_{trgt}?(y)P \; | \; x_{src}!\langle {Q} \rangle \red P\{\quotep{Q}/y}\} }
  \and \\
  \inferrule* [lab=Par] {{P} \red {P}'} {{{P} | {Q}} \red {{P}' | {Q}}}
  \and
  \inferrule* [lab=Equiv]{{{P} \scong {P}'} \andalso {{P}' \red {Q}'} \andalso {{Q}' \scong {Q}}}{{P} \red {Q}}
\end{mathpar}

\begin{eqnarray*}
  match_{\equiv} (\quotep{P},\quotep{Q}) & := & P \equiv Q \\
  match_{\dagger}(\quotep{P},\quotep{Q}) & := & \forall R. P|Q \red^{*} R => R \red^{*} 0 \\
  match_{K}(\quotep{P},\quotep{Q}) & := & K \mbox{ for some context } K
\end{eqnarray*}

$u?(x)P | u!\langle Q \rangle \red P\{\quotep{Q}/x\}$

%We write $\wred$ for $\red^*$, and $P\red$ if $\exists Q $ such that $ P \red Q$.
We write $P\red$ if $\exists Q $ such that $ P \red Q$ and $P\not\red$, otherwise.

\section{Replication}

As mentioned before, it is known that replication (and hence
recursion) can be implemented in a higher-order process algebra
\cite{SangiorgiWalker}. As our first example of calculation with the
machinery thus far presented we give the construction explicitly in
the {\rhoc}.

\begin{eqnarray}
	D_{x} & := & \prefix{x}{y}{(\binpar{\outputp{x}{y}}{@{y}})} \nonumber\\
	\bangp_{x}{P} & := & \binpar{{x}!\langle{\binpar{D_{x}}{P}}\rangle}{D_{x}} \nonumber
\end{eqnarray}

\begin{eqnarray}
	\bangp_{x}{P} & & \nonumber\\
	=
	& {x}!\langle{(\prefix{x}{y}{(\outputp{x}{y} | @{y})) | P}}\rangle 
	      | \prefix{x}{y}{(\outputp{x}{y} | @{y})} & \nonumber\\
	\red
	& (\outputp{x}{y} | @{y})\substn{\quotep{(\prefix{x}{y}{(@{y} | \outputp{x}{y})) | P}}}{y} & \nonumber\\
	=
	& \outputp{x}{\quotep{(\prefix{x}{y}{(\outputp{x}{y} | @{y})) | P}}}
	  | {(\prefix{x}{y}{(\outputp{x}{y} | @{y})) | P}} & \nonumber\\
	\red
	& \ldots & \nonumber\\
	\red^*
	& P | P | \ldots & \nonumber
\end{eqnarray}

Of course, this encoding, as an implementation, runs away, unfolding
$\bangp{P}$ eagerly. A lazier and more implementable replication
operator, restricted to input-guarded processes, may be obtained as follows.

\begin{eqnarray}
\bangp{\prefix{u}{v}{P}} 
	:= 
	\binpar{\lift{x}{\prefix{u}{v}{(\binpar{D(x)}{P})}}}{D(x)} \nonumber
\end{eqnarray}

\begin{remark}
  Note that the lazier definition still does not deal with summation
  or mixed summation (i.e. sums over input and output). The reader is
  invited to construct definitions of replication that deal with these
  features. 

  Further, the definitions are parameterized in a name, $x$. Can you,
  gentle reader, make a definition that eliminates this parameter and
  guarantees no accidental interaction between the replication
  machinery and the process being replicated -- i.e. no accidental
  sharing of names used by the process to get its work done and the
  name(s) used by the replication to effect copying. This latter
  revision of the definition of replication is crucial to obtaining
  the expected identity $!!P \sim !P$.
\end{remark}

\begin{remark}\label{rem:paradoxical_combinator}
  The reader familiar with the lambda calculus will have noticed the
  similarity between $D$ and the paradoxical combinator.

  [Ed. note: the existence of this seems to suggest we have to be more
  restrictive on the set of processes and names we admit if we are to
  support no-cloning.]
\end{remark}

\subsubsection{Bisimulation}

The computational dynamics gives rise to another kind of equivalence,
the equivalence of computational behavior. As previously mentioned
this is typically captured \emph{via} some form of bisimulation.

% The notion we use in this paper is weak barbed bisimulation
% \cite{milner91polyadicpi}.

The notion we use in this paper is derived from weak barbed
bisimulation \cite{milner91polyadicpi}. 

\begin{definition}
An \emph{observation relation}, $\downarrow_{\mathcal N}$, over a set
of names, $\mathcal N$, is the smallest relation satisfying the rules
below.

\infrule[Out-barb]{y \in {\mathcal N}, \; x \nameeq y}
		  {\outputp{x}{v} \downarrow_{\mathcal N} x}
\infrule[Par-barb]{\mbox{$P\downarrow_{\mathcal N} x$ or $Q\downarrow_{\mathcal N} x$}}
		  {\binpar{P}{Q} \downarrow_{\mathcal N} x}

We write $P \Downarrow_{\mathcal N} x$ if there is $Q$ such that 
$P \wred Q$ and $Q \downarrow_{\mathcal N} x$.
\end{definition}

\begin{definition}
%\label{def.bbisim}
An  ${\mathcal N}$-\emph{barbed bisimulation} over a set of names, ${\mathcal N}$, is a symmetric binary relation 
${\mathcal S}_{\mathcal N}$ between agents such that $P\rel{S}_{\mathcal N}Q$ implies:
\begin{enumerate}
\item If $P \red P'$ then $Q \wred Q'$ and $P'\rel{S}_{\mathcal N} Q'$.
\item If $P\downarrow_{\mathcal N} x$, then $Q\Downarrow_{\mathcal N} x$.
\end{enumerate}
$P$ is ${\mathcal N}$-barbed bisimilar to $Q$, written
$P \wbbisim_{\mathcal N} Q$, if $P \rel{S}_{\mathcal N} Q$ for some ${\mathcal N}$-barbed bisimulation ${\mathcal S}_{\mathcal N}$.
\end{definition}

$\mathcal{R} \subseteq \pi \times \pi$

$P \mathcal{R} Q => \forall P'. P \red P' \Rightarrow \exists Q'. Q \red Q', P' \mathcal{R} Q'$

$P \vdash x \Rightarrow Q \vdash x$

\begin{mathpar}
  \inferrule*[lab=Out-barb]{x \nameeq y}{{y}!\langle{Q}\rangle \vdash x}
  \and
  \inferrule*[lab=Par-barb]{\mbox{$P\vdash x$ or $Q\vdash x$}}{\binpar{P}{Q} \vdash x}
\end{mathpar}

\subsubsection{Contexts}

One of the principle advantages of computational calculi like the
$\pi$-calculus is a well-defined notion of context,
contextual-equivalence and a correlation between
contextual-equivalence and notions of bisimulation. The notion of
context allows the decomposition of a process into (sub-)process and
its syntactic environment, its context. Thus, a context may be
thought of as a process with a ``hole'' (written $\Box$) in it. The
application of a context $M$ to a process $P$, written $M[P]$, is
tantamount to filling the hole in $M$ with $P$. In this paper we do
not need the full weight of this theory, but do make use of the notion
of context in the proof the main theorem. 

\begin{mathpar}
  \inferrule* [lab=summation] {} {{M_{M},M_{N}} \bc \Box \;|\; x.M_{A} \;|\; M_{M}+M_{N}}
  \and
  \inferrule* [lab=agent] {} {{M_{A}} \bc (\vec{x})M_{P} \;| \; \clift{P_0,\ldots,M_{P},\ldots,P_N}}
  \and \\
  \inferrule* [lab=process] {} {{M_{P}} \bc M_{N} \;| \;P|M_{P} }
\end{mathpar} 

\begin{mathpar}
  \inferrule* [lab=sychronization] {} {M_{N} \bc \Box \;|\; x?M_{F} \;|\; x!M_{C}}
  \and
  \inferrule* [lab=abstraction] {} {{M_{F}} \bc (x)M_{P} }
  \and
  \inferrule* [lab=concretion] {} {{M_{C}} \bc \langle M_{P} \rangle }
  \and \\
  \inferrule* [lab=process] {} {{M_{P}} \bc M_{N} \;| \;P|M_{P} }
\end{mathpar}

\begin{definition}[contextual application] Given a context $M$, and
  process $P$, we define the \emph{contextual application}, $M[P] :=
  M\{P/\Box\}$. That is, the contextual application of M to P is the
  substitution of $P$ for $\Box$ in $M$.
\end{definition}

$\meaningof{-} : L \to \mathcal{P}(\pi)$

\begin{mathpar}
  \inferrule* [lab=collection] {} {\meaningof{true} = \pi, \and \meaningof{~E} = \pi \setminus \meaningof{E}, \and \meaningof{E_{1} \& E_{2}} = \meaningof{E_{1}} \cap \meaningof{E_{2}}}
\end{mathpar}

\begin{mathpar}
  \inferrule* [lab=structure] {} {\meaningof{0} = \{ P \in \pi | P \equiv 0 \}, \and \\ \meaningof{E_1 | E_2} = \{ P \in \pi | P \equiv P_{1} | P_{2}, P_{1} \in \meaningof{E_{1}}, P_{2} \in \meaningof{E_2}\} }
\end{mathpar}

\begin{mathpar}
 \inferrule* [lab=behavior] {} {\meaningof{\langle a?b \rangle E} = \{ P \in \pi | P \equiv Q | u?(y)P', \\ \and \\\\ \and \\ \;\;\; u \in \meaningof{a}, \forall z.P'\{z/y\} \in \meaningof{E\{z/b\}}\}, \and \\ \meaningof{a!E} = \{ P \in \pi | P \equiv Q | x!\langle P' \rangle, x \in \meaningof{a} P' \in \meaningof{E}\} }
\end{mathpar}

\begin{mathpar}
 \inferrule* [lab=nominal] {} {\meaningof{\quotep{E}} = \{ \quotep{P} \in \quotep{\pi} | P \in \meaningof{E} \}, \and \meaningof{\quotep{P}} = \{ \quotep{Q} \in \quotep{\pi} | P \equiv Q \} \and \\ \meaningof{@\quotep{E}} = \{ P \in \pi | P \equiv @x, x \in \meaningof{E} \}}
\end{mathpar}

\begin{eqnarray*}
  \\
  \meaningof{-} : TS \to ST
\end{eqnarray*}

\begin{eqnarray*}
  \\
  L : TS \to ST
\end{eqnarray*}

\begin{eqnarray*}
  \\
  P \models E \iff P \in \meaningof{E}
\end{eqnarray*}

\begin{eqnarray*}
  P \approx_{L} Q \iff \forall E \in L. P \models E \iff Q \models E
\end{eqnarray*}

\begin{eqnarray*}
  P \approx_{K} Q
\end{eqnarray*}

\begin{eqnarray*}
  P \approx Q
\end{eqnarray*}

$\approx_{K} = \approx = \approx_{L}$

\subsubsection{Contextual duality}

Note that contexts extend the quotation operation to a family of
operations from processes to names. Given a context, $M$, we can
define a \emph{nominal context}, $\quotep{M}$ by $\quotep{M}[P] :=
\quotep{M[P]}$. To foreshadow what is to come we observe that these
operations enjoy a duality with processes very much like the duality
between vectors and maps from vectors to scalars.

Further, because the calculus is essentially higher-order, we have a
correspondence between contexts and processes. More specifically,
given a name $x$ and a context $M$ we can construct $M^{*}_{x}$ such
that 

\begin{mathpar}
  M^{*}_{x} | \lift{x}{P} \red M[P]
\end{mathpar}

namely,

\begin{mathpar}
  M^{*}_{x} := x?(u).M[\dropn{u}]
\end{mathpar}

The dependence of $M^{*}_{x}$ on a name makes it an abstraction, 

\begin{mathpar}
  M^{*} := (x)x?(u).M[\dropn{u}]
\end{mathpar}

\subsection{Additional notation}

It will sometimes be convenient to denote the process a name
quotes. We already have the notation $x = \quotep{P}$, but it will be
convenient to introduce an alternate notation, $\procn{x}$, when we
want to emphasize the connection to the use of the name. Note that, by
virtue of name equivalence, $\quotep{\procn{x}} \nameeq x$; so, the
notation is consistent with previous definitions.

Further, because names have structure it is possible to effect
substitutions on the basis of that structure. This means we need to
upgrade our notation for substitutions, which we accomplish by
adapting comprehension notation. Thus,

\begin{mathpar}
  P\{ y / x : x \in S \}
\end{mathpar}

is interpreted to mean the process derived from P by replacing (in a
capture-avoiding manner) each occurrence of $x$ in $S$ by $y$. For example,

\begin{mathpar}
  P\{ \quotep{\procn{x}|\procn{x}} / x : x \in \freenames{P} \}
\end{mathpar}

will replace each (occurrence) of a free name $x$ in $P$ by
$\quotep{\procn{x}|\procn{x}}$.

Also, we will avail ourselves of the notation $x^{L}$ and $x^{R}$ to
denote injections of a name into disjoint copies of the name
space. There are numerous ways to accomplish this. One example can be
found in \cite{MeredithR05}. This notation overloads to vectors of
names: $\vec{x}^{\pi} := (x_{i}^{\pi} \; : \; 0 \leq i < |\vec{x}| )$ where $\pi \in \{L,R\}$.

We also use $P^{\Box} := P|\Box$.

In \cite{MeredithR05} an interpretation of the new operator is
given. It turns out that there are several possible interpretations
all enjoying the requisite algebraic properties of the operator (see
\cite{milner91polyadicpi}). We will therefore make liberal use of
$(\nu\; \vec{x})P$.

% subsection the_syntax_and_semantics_of_the_notation_system (end)   

\input{qm2pi.qmops} 

\input{qm2pi.sterngerlach} 

\input{qm2pi.metric} 

% section concurrent_process_calculi (end)

%\input{qm2pi.proofsketch}

% section proof sketch (end)

%\input{qm2pi.slviaknots} 

% section spatial logic via knots (end)

\input{qm2pi.conclusion}

% section conclusion (end)

%\input{qm2pi.dtcodes} 

% section wiring algorithm (end)

\input{qm2pi.ack} 

% section acknowledgments (end)

\newpage


\bibliographystyle{plain}   
\bibliography{../../biblios/main.bib}

\input{qm2pi.rhodetails}

\end{document}



\end{document}

 

%\documentclass[12pt]{llncs}
%\documentclass{jktr}

\usepackage[pdftex]{hyperref}                   
\usepackage {listings}
\usepackage {mathpartir}
\usepackage{bcprules}
%\usepackage{listings}
                       
\usepackage{graphicx} 
%\usepackage[margins=2.5cm,nohead,nofoot]{geometry}
%\usepackage{geometry}
\usepackage{amsfonts}
\usepackage{amstext}
\usepackage{latexsym}
\usepackage{amssymb}
\usepackage{color}


%\include{myPreamble}
\documentclass[12pt]{llncs}
%\documentclass{jktr}

\usepackage[pdftex]{hyperref}                   
\usepackage {listings}
\usepackage {mathpartir}
\usepackage{bcprules}
%\usepackage{listings}
                       
\usepackage{graphicx} 
%\usepackage[margins=2.5cm,nohead,nofoot]{geometry}
%\usepackage{geometry}
\usepackage{amsfonts}
\usepackage{amstext}
\usepackage{latexsym}
\usepackage{amssymb}
\usepackage{color}


%\include{myPreamble}
\include{qm2pi.local} 

%\ifpdf
%\usepackage[pdftex]{graphicx}
%\else
%\usepackage{graphicx}
%\fi

 % \ifpdf
%  \usepackage{pdfsync}
%  \if


%\title{Brief Article}
%\author{David F. Snyder}
%\author{L.G. Meredith}

%\address{Dept. of Math., Texas State University--San Marcos, San Marcos, TX 78666}
       
\pagestyle{empty}


\begin{document}

\lstset{language=[Objective]Caml,frame=shadowbox}

\input{qm2pi.front}

% section front matter (end)

\input{qm2pi.intro} 
 
% section introduction (end)

% \input{qm2pi.knotations} 

% section notation (end)

\input{qm2pi.process.calculi} 

% section concurrent_process_calculi_and_spatial_logics_ (end)
    
%\input{qm2pi.knots2pi} 

%\input{qm2pi.trefoil} 

%\input{qm2pi.mainthm} 

% subsection basic_interpretation (end)

%\input{qm2pi.rho.presentation} 
\subsection{The syntax and semantics of the notation system}\label{sub:the_syntax_and_semantics_of_the_notation_system} % (fold)

We now summarize a technical presentation of the calculus that
embodies our theory of dynamics. The typical presentation of such a
calculus follows the style of giving generators and relations on
them. The grammar, below, describing term constructors, freely
generates the set of processes, $\Proc$. This set is then quotiented
by a relation known as structural congruence and it is over this set
that the notion of dynamics is expressed. This presentation is
essentially that of \cite{MeredithR05} with the addition of
polyadicity and summation. For readability we have relegated some of
the technical subtleties to an appendix.

\subsubsection{Process grammar}\label{subsub:process_grammar}

\begin{mathpar}
  \inferrule* [lab=synchronization] {} {{M} \bc \pzero \;|\; x?F \;|\; x!C }
  \and
  \inferrule* [lab=abstraction] {} {{F} \bc (x)P}
  \and
  \inferrule* [lab=concretion] {} {{C} \bc \langle Q \rangle}
  \and
  \inferrule* [lab=process] {} {{P,Q} \bc M \;| \;P|Q \;|\; @{x}}
  \and
  \inferrule* [lab=name] {} {{x} \bc \quotep{P}}
\end{mathpar} 

Note that $\vec{x}$ (resp. $\vec{P}$) denotes a vector of names
(resp. processes) of length $|\vec{x}|$ (resp. $|\vec{P}|$). We adopt
the following useful abbreviations.

\begin{mathpar}
   x?(\vec{y}).P := x.(\vec{y})P \and  x\clift{\vec{P}} := x.\clift{\vec{P}}
   \and x!(y) := \lift{x}{\dropn{y}}
   \and \Pi_{i=0}^{n-1}P_i := P_0 | \ldots | P_{n-1}
\end{mathpar}

\subsubsection{Structural congruence}

\paragraph{Free and bound names and alpha-equivalence.} At the
core of structural equivalence is alpha-equivalence which identifies
process that are the same up to a change of variable. Formally, we
recognize the distinction between free and bound names. The free names
of a process, $\freenames{P}$, may be calculated recursively as
follows:

\begin{mathpar}
\freenames{\pzero} := \emptyset
  \and \\
  \freenames{x?(y).P} := \{ x \} \cup (\freenames{P} \setminus \{ y \})
  \and 
  \freenames{x!\langle P \rangle} := \{ x \} \cup \{ P \} 
  \and \\
  \freenames{P|Q} := \freenames{P} \cup \freenames{Q}
  \and \\
  \freenames{@{x}} := \{ x \}
\end{mathpar}

$\pi$
$\quotep{\pi}$

$\freenames{-} : \pi \to \mathcal{P}(\quotep{\pi})$

\begin{eqnarray*}
  \freenames{\pzero} & := & \emptyset \\
  \freenames{x?(y).P} & := & \{ x \} \cup (\freenames{P} \setminus \{ y \}) \\
  \freenames{x!\langle P \rangle} & := & \{ x \} \cup \{ P \} \\
  \freenames{P|Q} & := & \freenames{P} \cup \freenames{Q} \\
  \freenames{\dropn{x}} & := & \{ x \}
\end{eqnarray*}

The bound names of a process, $\boundnames{P}$, are those names occurring in $P$
that are not free. For example, in $x?(y).0$, the name $x$ is free, while $y$ is bound.

\begin{mathpar}
  \inferrule* [lab=monoidal-laws] {} { P|Q \equiv Q|P \and P|0 \equiv P \and P|(Q|R) \equiv (P|Q)|R }
\end{mathpar}

\begin{mathpar}
  \inferrule* [lab=alpha-equivalence] {} { (x)P \equiv (y)P\{y/x\} \and y \not\in \freenames{P} }
\end{mathpar}

\begin{definition}
Then two processes, $P,Q$, are alpha-equivalent if $P = Q\{\vec{y}/\vec{x}\}$ for
some $\vec{x} \in \boundnames{Q},\vec{y} \in \boundnames{P}$, where $Q\{\vec{y}/\vec{x}\}$
denotes the capture-avoiding substitution of $\vec{y}$ for $\vec{x}$ in $Q$.
\end{definition}

\begin{definition}
  The {\em structural congruence} \cite{SangiorgiWalker} , $\equiv$,
  between processes is the least congruence containing
  alpha-equivalence, satisfying the abelian monoid laws
  (associativity, commutativity and $\pzero$ as identity) for parallel
  composition $|$ and for summation $+$.
\end{definition}

\subsection{Name equivalence}

We take name equivalence, written $\nameeq$, to be the smallest
equivalence relation generated by the following rules.

\begin{mathpar}
\inferrule*[lab=Quote-drop]
{ }
{ \quotep{@{x}} \nameeq x }

\inferrule*[lab=Struct-equiv]
{ P \scong Q }
{ \quotep{P} \nameeq \quotep{Q} }
\end{mathpar}

The astute reader will have noticed that the mutual recursion of names
and processes imposes a mutual recursion on alpha-equivalence and
structural equivalence via name-equivalence. Fortunately, all of this
works out pleasantly and we may calculate in the natural way, free of
concern. The reader interested in the details is referred to the
appendix \ref{appendix:rho_details}.

\subsection{Substitution}

We use $\Proc$ for the set of processes, $\QProc$ for the set of
names, and $\id{\{}\vec{y} / \vec{x} \id{\}}$ to denote partial maps,
$s : \QProc \rightarrow \QProc$. A map, $s$ lifts, uniquely, to a map
on process terms, $\widehat{s} : \Proc \rightarrow \Proc$ by the
following equations.

\begin{mathpar}
  (0) \psubstp{Q}{P} := 0 \\
  (R \juxtap S) \psubstp{Q}{P}
  :=    
  (R)\psubstp{Q}{P} \juxtap (S) \psubstp{Q}{P} \\
  (x?(y).R) \psubstp{Q}{P}    
  :=    
  (x)\substp{Q}{P} (z)\concat( (R \psubstn{z}{y}) \psubstp{Q}{P} ) \\
  (\lift{x}{R}) \psubstp{Q}{P}  
  :=
  \lift{(x)\substp{Q}{P}}{ R \psubstp{Q}{P} } \\
%   (\dropn{x})  \psubstp{Q}{P}       
%   := 
%   \left\{ 
%     \begin{array}{ccc} 
%       \dropn{\quotep{Q}} & & x \nameeq \quotep{P} \\
%       \dropn{x} & & otherwise \\
%     \end{array}
%   \right. 
  (\dropn{x})  \psubstp{Q}{P}       
  := 
  \left\{ 
    \begin{array}{ccc} 
      Q & & x \nameeq \quotep{P} \\
      \dropn{x} & & otherwise \\
    \end{array}
  \right.
\end{mathpar}
 

where

\begin{eqnarray}
  (x)\id{\{} \lpquote Q \rpquote / \lpquote P \rpquote \id{\}}            = 
  \left\{ 
    \begin{array}{ccc}
      \lpquote Q \rpquote & & x \nameeq \lpquote P \rpquote \\
      x & & otherwise \\
    \end{array}
  \right. \nonumber
\end{eqnarray}

and $z$ is chosen distinct from $\quotep{P}$, $\quotep{Q}$, the free
names in $Q$, and all the names in $R$. Our $\alpha$-equivalence will
be built in the standard way from this substitution.

\begin{remark}\label{rem:no_self_referential_names}
  One consequence of these definitions is that $\forall P. \quotep{P}
  \not\in \freenames{P}$.
\end{remark}

\subsection{ Dynamic quote: an example }

Anticipating something of what's to come, consider applying the
substitution, $\widehat{\id{\{}u / z \id{\}}}$, to the following pair
of processes, $\lift{w}{y!(z)}$ and $w[ \lpquote y!(z) \rpquote ]$.

\begin{eqnarray}
	\lift{w}{y!(z)}\widehat{\id{\{}u / z \id{\}}}
		& = &
		\lift{w}{y!(u)} \nonumber\\
	w[ \lpquote y!(z) \rpquote ] \widehat{ \id{\{}u / z \id{\}} }
		& = &
		w[ \lpquote y!(z) \rpquote ] \nonumber
\end{eqnarray}

Because the body of the process between quotes is impervious to
substitution, we get radically different answers. In fact, by
examining the first process in an input context,
e.g. $x?(z).\lift{w}{y!(z)}$, we see that the process under the lift
operator may be shaped by prefixed inputs binding a name inside it. In
this sense, the lift operator will be seen as a way to dynamically
construct processes before reifying them as names.

Finally equipped with these standard features we can present the
dynamics of the calculus.

\subsubsection{Operational semantics} 

Finally, we introduce the computational dynamics. What marks these
algebras as distinct from other more traditionally studied algebraic
structures, e.g. vector spaces or polynomial rings, is the manner in
which dynamics is captured. In traditional structures, dynamics is typically
expressed through morphisms between such structures, as in linear maps
between vector spaces or morphisms between rings. In algebras
associated with the semantics of computation, the dynamics is
expressed as part of the algebraic structure itself, through a
reduction reduction relation typically denoted by $\red$. Below, we
give a recursive presentation of this relation for the calculus used
in the encoding.

$\red \subseteq \pi \times \pi$
$\red : \pi \to \mathcal{P}(\pi)$

\begin{mathpar}
  \inferrule* [lab=Comm] { \textsf{match}( x_{src}, x_{trgt} ) } { x_{trgt}?(y)P \; | \; x_{src}!\langle {Q} \rangle \red P\{\quotep{Q}/y}\} }
  \and \\
  \inferrule* [lab=Par] {{P} \red {P}'} {{{P} | {Q}} \red {{P}' | {Q}}}
  \and
  \inferrule* [lab=Equiv]{{{P} \scong {P}'} \andalso {{P}' \red {Q}'} \andalso {{Q}' \scong {Q}}}{{P} \red {Q}}
\end{mathpar}

\begin{eqnarray*}
  match_{\equiv} (\quotep{P},\quotep{Q}) & := & P \equiv Q \\
  match_{\dagger}(\quotep{P},\quotep{Q}) & := & \forall R. P|Q \red^{*} R => R \red^{*} 0 \\
  match_{K}(\quotep{P},\quotep{Q}) & := & K \mbox{ for some context } K
\end{eqnarray*}

$u?(x)P | u!\langle Q \rangle \red P\{\quotep{Q}/x\}$

%We write $\wred$ for $\red^*$, and $P\red$ if $\exists Q $ such that $ P \red Q$.
We write $P\red$ if $\exists Q $ such that $ P \red Q$ and $P\not\red$, otherwise.

\section{Replication}

As mentioned before, it is known that replication (and hence
recursion) can be implemented in a higher-order process algebra
\cite{SangiorgiWalker}. As our first example of calculation with the
machinery thus far presented we give the construction explicitly in
the {\rhoc}.

\begin{eqnarray}
	D_{x} & := & \prefix{x}{y}{(\binpar{\outputp{x}{y}}{@{y}})} \nonumber\\
	\bangp_{x}{P} & := & \binpar{{x}!\langle{\binpar{D_{x}}{P}}\rangle}{D_{x}} \nonumber
\end{eqnarray}

\begin{eqnarray}
	\bangp_{x}{P} & & \nonumber\\
	=
	& {x}!\langle{(\prefix{x}{y}{(\outputp{x}{y} | @{y})) | P}}\rangle 
	      | \prefix{x}{y}{(\outputp{x}{y} | @{y})} & \nonumber\\
	\red
	& (\outputp{x}{y} | @{y})\substn{\quotep{(\prefix{x}{y}{(@{y} | \outputp{x}{y})) | P}}}{y} & \nonumber\\
	=
	& \outputp{x}{\quotep{(\prefix{x}{y}{(\outputp{x}{y} | @{y})) | P}}}
	  | {(\prefix{x}{y}{(\outputp{x}{y} | @{y})) | P}} & \nonumber\\
	\red
	& \ldots & \nonumber\\
	\red^*
	& P | P | \ldots & \nonumber
\end{eqnarray}

Of course, this encoding, as an implementation, runs away, unfolding
$\bangp{P}$ eagerly. A lazier and more implementable replication
operator, restricted to input-guarded processes, may be obtained as follows.

\begin{eqnarray}
\bangp{\prefix{u}{v}{P}} 
	:= 
	\binpar{\lift{x}{\prefix{u}{v}{(\binpar{D(x)}{P})}}}{D(x)} \nonumber
\end{eqnarray}

\begin{remark}
  Note that the lazier definition still does not deal with summation
  or mixed summation (i.e. sums over input and output). The reader is
  invited to construct definitions of replication that deal with these
  features. 

  Further, the definitions are parameterized in a name, $x$. Can you,
  gentle reader, make a definition that eliminates this parameter and
  guarantees no accidental interaction between the replication
  machinery and the process being replicated -- i.e. no accidental
  sharing of names used by the process to get its work done and the
  name(s) used by the replication to effect copying. This latter
  revision of the definition of replication is crucial to obtaining
  the expected identity $!!P \sim !P$.
\end{remark}

\begin{remark}\label{rem:paradoxical_combinator}
  The reader familiar with the lambda calculus will have noticed the
  similarity between $D$ and the paradoxical combinator.

  [Ed. note: the existence of this seems to suggest we have to be more
  restrictive on the set of processes and names we admit if we are to
  support no-cloning.]
\end{remark}

\subsubsection{Bisimulation}

The computational dynamics gives rise to another kind of equivalence,
the equivalence of computational behavior. As previously mentioned
this is typically captured \emph{via} some form of bisimulation.

% The notion we use in this paper is weak barbed bisimulation
% \cite{milner91polyadicpi}.

The notion we use in this paper is derived from weak barbed
bisimulation \cite{milner91polyadicpi}. 

\begin{definition}
An \emph{observation relation}, $\downarrow_{\mathcal N}$, over a set
of names, $\mathcal N$, is the smallest relation satisfying the rules
below.

\infrule[Out-barb]{y \in {\mathcal N}, \; x \nameeq y}
		  {\outputp{x}{v} \downarrow_{\mathcal N} x}
\infrule[Par-barb]{\mbox{$P\downarrow_{\mathcal N} x$ or $Q\downarrow_{\mathcal N} x$}}
		  {\binpar{P}{Q} \downarrow_{\mathcal N} x}

We write $P \Downarrow_{\mathcal N} x$ if there is $Q$ such that 
$P \wred Q$ and $Q \downarrow_{\mathcal N} x$.
\end{definition}

\begin{definition}
%\label{def.bbisim}
An  ${\mathcal N}$-\emph{barbed bisimulation} over a set of names, ${\mathcal N}$, is a symmetric binary relation 
${\mathcal S}_{\mathcal N}$ between agents such that $P\rel{S}_{\mathcal N}Q$ implies:
\begin{enumerate}
\item If $P \red P'$ then $Q \wred Q'$ and $P'\rel{S}_{\mathcal N} Q'$.
\item If $P\downarrow_{\mathcal N} x$, then $Q\Downarrow_{\mathcal N} x$.
\end{enumerate}
$P$ is ${\mathcal N}$-barbed bisimilar to $Q$, written
$P \wbbisim_{\mathcal N} Q$, if $P \rel{S}_{\mathcal N} Q$ for some ${\mathcal N}$-barbed bisimulation ${\mathcal S}_{\mathcal N}$.
\end{definition}

$\mathcal{R} \subseteq \pi \times \pi$

$P \mathcal{R} Q => \forall P'. P \red P' \Rightarrow \exists Q'. Q \red Q', P' \mathcal{R} Q'$

$P \vdash x \Rightarrow Q \vdash x$

\begin{mathpar}
  \inferrule*[lab=Out-barb]{x \nameeq y}{{y}!\langle{Q}\rangle \vdash x}
  \and
  \inferrule*[lab=Par-barb]{\mbox{$P\vdash x$ or $Q\vdash x$}}{\binpar{P}{Q} \vdash x}
\end{mathpar}

\subsubsection{Contexts}

One of the principle advantages of computational calculi like the
$\pi$-calculus is a well-defined notion of context,
contextual-equivalence and a correlation between
contextual-equivalence and notions of bisimulation. The notion of
context allows the decomposition of a process into (sub-)process and
its syntactic environment, its context. Thus, a context may be
thought of as a process with a ``hole'' (written $\Box$) in it. The
application of a context $M$ to a process $P$, written $M[P]$, is
tantamount to filling the hole in $M$ with $P$. In this paper we do
not need the full weight of this theory, but do make use of the notion
of context in the proof the main theorem. 

\begin{mathpar}
  \inferrule* [lab=summation] {} {{M_{M},M_{N}} \bc \Box \;|\; x.M_{A} \;|\; M_{M}+M_{N}}
  \and
  \inferrule* [lab=agent] {} {{M_{A}} \bc (\vec{x})M_{P} \;| \; \clift{P_0,\ldots,M_{P},\ldots,P_N}}
  \and \\
  \inferrule* [lab=process] {} {{M_{P}} \bc M_{N} \;| \;P|M_{P} }
\end{mathpar} 

\begin{mathpar}
  \inferrule* [lab=sychronization] {} {M_{N} \bc \Box \;|\; x?M_{F} \;|\; x!M_{C}}
  \and
  \inferrule* [lab=abstraction] {} {{M_{F}} \bc (x)M_{P} }
  \and
  \inferrule* [lab=concretion] {} {{M_{C}} \bc \langle M_{P} \rangle }
  \and \\
  \inferrule* [lab=process] {} {{M_{P}} \bc M_{N} \;| \;P|M_{P} }
\end{mathpar}

\begin{definition}[contextual application] Given a context $M$, and
  process $P$, we define the \emph{contextual application}, $M[P] :=
  M\{P/\Box\}$. That is, the contextual application of M to P is the
  substitution of $P$ for $\Box$ in $M$.
\end{definition}

$\meaningof{-} : L \to \mathcal{P}(\pi)$

\begin{mathpar}
  \inferrule* [lab=collection] {} {\meaningof{true} = \pi, \and \meaningof{~E} = \pi \setminus \meaningof{E}, \and \meaningof{E_{1} \& E_{2}} = \meaningof{E_{1}} \cap \meaningof{E_{2}}}
\end{mathpar}

\begin{mathpar}
  \inferrule* [lab=structure] {} {\meaningof{0} = \{ P \in \pi | P \equiv 0 \}, \and \\ \meaningof{E_1 | E_2} = \{ P \in \pi | P \equiv P_{1} | P_{2}, P_{1} \in \meaningof{E_{1}}, P_{2} \in \meaningof{E_2}\} }
\end{mathpar}

\begin{mathpar}
 \inferrule* [lab=behavior] {} {\meaningof{\langle a?b \rangle E} = \{ P \in \pi | P \equiv Q | u?(y)P', \\ \and \\\\ \and \\ \;\;\; u \in \meaningof{a}, \forall z.P'\{z/y\} \in \meaningof{E\{z/b\}}\}, \and \\ \meaningof{a!E} = \{ P \in \pi | P \equiv Q | x!\langle P' \rangle, x \in \meaningof{a} P' \in \meaningof{E}\} }
\end{mathpar}

\begin{mathpar}
 \inferrule* [lab=nominal] {} {\meaningof{\quotep{E}} = \{ \quotep{P} \in \quotep{\pi} | P \in \meaningof{E} \}, \and \meaningof{\quotep{P}} = \{ \quotep{Q} \in \quotep{\pi} | P \equiv Q \} \and \\ \meaningof{@\quotep{E}} = \{ P \in \pi | P \equiv @x, x \in \meaningof{E} \}}
\end{mathpar}

\begin{eqnarray*}
  \\
  \meaningof{-} : TS \to ST
\end{eqnarray*}

\begin{eqnarray*}
  \\
  L : TS \to ST
\end{eqnarray*}

\begin{eqnarray*}
  \\
  P \models E \iff P \in \meaningof{E}
\end{eqnarray*}

\begin{eqnarray*}
  P \approx_{L} Q \iff \forall E \in L. P \models E \iff Q \models E
\end{eqnarray*}

\begin{eqnarray*}
  P \approx_{K} Q
\end{eqnarray*}

\begin{eqnarray*}
  P \approx Q
\end{eqnarray*}

$\approx_{K} = \approx = \approx_{L}$

\subsubsection{Contextual duality}

Note that contexts extend the quotation operation to a family of
operations from processes to names. Given a context, $M$, we can
define a \emph{nominal context}, $\quotep{M}$ by $\quotep{M}[P] :=
\quotep{M[P]}$. To foreshadow what is to come we observe that these
operations enjoy a duality with processes very much like the duality
between vectors and maps from vectors to scalars.

Further, because the calculus is essentially higher-order, we have a
correspondence between contexts and processes. More specifically,
given a name $x$ and a context $M$ we can construct $M^{*}_{x}$ such
that 

\begin{mathpar}
  M^{*}_{x} | \lift{x}{P} \red M[P]
\end{mathpar}

namely,

\begin{mathpar}
  M^{*}_{x} := x?(u).M[\dropn{u}]
\end{mathpar}

The dependence of $M^{*}_{x}$ on a name makes it an abstraction, 

\begin{mathpar}
  M^{*} := (x)x?(u).M[\dropn{u}]
\end{mathpar}

\subsection{Additional notation}

It will sometimes be convenient to denote the process a name
quotes. We already have the notation $x = \quotep{P}$, but it will be
convenient to introduce an alternate notation, $\procn{x}$, when we
want to emphasize the connection to the use of the name. Note that, by
virtue of name equivalence, $\quotep{\procn{x}} \nameeq x$; so, the
notation is consistent with previous definitions.

Further, because names have structure it is possible to effect
substitutions on the basis of that structure. This means we need to
upgrade our notation for substitutions, which we accomplish by
adapting comprehension notation. Thus,

\begin{mathpar}
  P\{ y / x : x \in S \}
\end{mathpar}

is interpreted to mean the process derived from P by replacing (in a
capture-avoiding manner) each occurrence of $x$ in $S$ by $y$. For example,

\begin{mathpar}
  P\{ \quotep{\procn{x}|\procn{x}} / x : x \in \freenames{P} \}
\end{mathpar}

will replace each (occurrence) of a free name $x$ in $P$ by
$\quotep{\procn{x}|\procn{x}}$.

Also, we will avail ourselves of the notation $x^{L}$ and $x^{R}$ to
denote injections of a name into disjoint copies of the name
space. There are numerous ways to accomplish this. One example can be
found in \cite{MeredithR05}. This notation overloads to vectors of
names: $\vec{x}^{\pi} := (x_{i}^{\pi} \; : \; 0 \leq i < |\vec{x}| )$ where $\pi \in \{L,R\}$.

We also use $P^{\Box} := P|\Box$.

In \cite{MeredithR05} an interpretation of the new operator is
given. It turns out that there are several possible interpretations
all enjoying the requisite algebraic properties of the operator (see
\cite{milner91polyadicpi}). We will therefore make liberal use of
$(\nu\; \vec{x})P$.

% subsection the_syntax_and_semantics_of_the_notation_system (end)   

\input{qm2pi.qmops} 

\input{qm2pi.sterngerlach} 

\input{qm2pi.metric} 

% section concurrent_process_calculi (end)

%\input{qm2pi.proofsketch}

% section proof sketch (end)

%\input{qm2pi.slviaknots} 

% section spatial logic via knots (end)

\input{qm2pi.conclusion}

% section conclusion (end)

%\input{qm2pi.dtcodes} 

% section wiring algorithm (end)

\input{qm2pi.ack} 

% section acknowledgments (end)

\newpage


\bibliographystyle{plain}   
\bibliography{../../biblios/main.bib}

\input{qm2pi.rhodetails}

\end{document}

 

%\ifpdf
%\usepackage[pdftex]{graphicx}
%\else
%\usepackage{graphicx}
%\fi

 % \ifpdf
%  \usepackage{pdfsync}
%  \if


%\title{Brief Article}
%\author{David F. Snyder}
%\author{L.G. Meredith}

%\address{Dept. of Math., Texas State University--San Marcos, San Marcos, TX 78666}
       
\pagestyle{empty}


\begin{document}

\lstset{language=[Objective]Caml,frame=shadowbox}

\documentclass[12pt]{llncs}
%\documentclass{jktr}

\usepackage[pdftex]{hyperref}                   
\usepackage {listings}
\usepackage {mathpartir}
\usepackage{bcprules}
%\usepackage{listings}
                       
\usepackage{graphicx} 
%\usepackage[margins=2.5cm,nohead,nofoot]{geometry}
%\usepackage{geometry}
\usepackage{amsfonts}
\usepackage{amstext}
\usepackage{latexsym}
\usepackage{amssymb}
\usepackage{color}


%\include{myPreamble}
\include{qm2pi.local} 

%\ifpdf
%\usepackage[pdftex]{graphicx}
%\else
%\usepackage{graphicx}
%\fi

 % \ifpdf
%  \usepackage{pdfsync}
%  \if


%\title{Brief Article}
%\author{David F. Snyder}
%\author{L.G. Meredith}

%\address{Dept. of Math., Texas State University--San Marcos, San Marcos, TX 78666}
       
\pagestyle{empty}


\begin{document}

\lstset{language=[Objective]Caml,frame=shadowbox}

\input{qm2pi.front}

% section front matter (end)

\input{qm2pi.intro} 
 
% section introduction (end)

% \input{qm2pi.knotations} 

% section notation (end)

\input{qm2pi.process.calculi} 

% section concurrent_process_calculi_and_spatial_logics_ (end)
    
%\input{qm2pi.knots2pi} 

%\input{qm2pi.trefoil} 

%\input{qm2pi.mainthm} 

% subsection basic_interpretation (end)

%\input{qm2pi.rho.presentation} 
\subsection{The syntax and semantics of the notation system}\label{sub:the_syntax_and_semantics_of_the_notation_system} % (fold)

We now summarize a technical presentation of the calculus that
embodies our theory of dynamics. The typical presentation of such a
calculus follows the style of giving generators and relations on
them. The grammar, below, describing term constructors, freely
generates the set of processes, $\Proc$. This set is then quotiented
by a relation known as structural congruence and it is over this set
that the notion of dynamics is expressed. This presentation is
essentially that of \cite{MeredithR05} with the addition of
polyadicity and summation. For readability we have relegated some of
the technical subtleties to an appendix.

\subsubsection{Process grammar}\label{subsub:process_grammar}

\begin{mathpar}
  \inferrule* [lab=synchronization] {} {{M} \bc \pzero \;|\; x?F \;|\; x!C }
  \and
  \inferrule* [lab=abstraction] {} {{F} \bc (x)P}
  \and
  \inferrule* [lab=concretion] {} {{C} \bc \langle Q \rangle}
  \and
  \inferrule* [lab=process] {} {{P,Q} \bc M \;| \;P|Q \;|\; @{x}}
  \and
  \inferrule* [lab=name] {} {{x} \bc \quotep{P}}
\end{mathpar} 

Note that $\vec{x}$ (resp. $\vec{P}$) denotes a vector of names
(resp. processes) of length $|\vec{x}|$ (resp. $|\vec{P}|$). We adopt
the following useful abbreviations.

\begin{mathpar}
   x?(\vec{y}).P := x.(\vec{y})P \and  x\clift{\vec{P}} := x.\clift{\vec{P}}
   \and x!(y) := \lift{x}{\dropn{y}}
   \and \Pi_{i=0}^{n-1}P_i := P_0 | \ldots | P_{n-1}
\end{mathpar}

\subsubsection{Structural congruence}

\paragraph{Free and bound names and alpha-equivalence.} At the
core of structural equivalence is alpha-equivalence which identifies
process that are the same up to a change of variable. Formally, we
recognize the distinction between free and bound names. The free names
of a process, $\freenames{P}$, may be calculated recursively as
follows:

\begin{mathpar}
\freenames{\pzero} := \emptyset
  \and \\
  \freenames{x?(y).P} := \{ x \} \cup (\freenames{P} \setminus \{ y \})
  \and 
  \freenames{x!\langle P \rangle} := \{ x \} \cup \{ P \} 
  \and \\
  \freenames{P|Q} := \freenames{P} \cup \freenames{Q}
  \and \\
  \freenames{@{x}} := \{ x \}
\end{mathpar}

$\pi$
$\quotep{\pi}$

$\freenames{-} : \pi \to \mathcal{P}(\quotep{\pi})$

\begin{eqnarray*}
  \freenames{\pzero} & := & \emptyset \\
  \freenames{x?(y).P} & := & \{ x \} \cup (\freenames{P} \setminus \{ y \}) \\
  \freenames{x!\langle P \rangle} & := & \{ x \} \cup \{ P \} \\
  \freenames{P|Q} & := & \freenames{P} \cup \freenames{Q} \\
  \freenames{\dropn{x}} & := & \{ x \}
\end{eqnarray*}

The bound names of a process, $\boundnames{P}$, are those names occurring in $P$
that are not free. For example, in $x?(y).0$, the name $x$ is free, while $y$ is bound.

\begin{mathpar}
  \inferrule* [lab=monoidal-laws] {} { P|Q \equiv Q|P \and P|0 \equiv P \and P|(Q|R) \equiv (P|Q)|R }
\end{mathpar}

\begin{mathpar}
  \inferrule* [lab=alpha-equivalence] {} { (x)P \equiv (y)P\{y/x\} \and y \not\in \freenames{P} }
\end{mathpar}

\begin{definition}
Then two processes, $P,Q$, are alpha-equivalent if $P = Q\{\vec{y}/\vec{x}\}$ for
some $\vec{x} \in \boundnames{Q},\vec{y} \in \boundnames{P}$, where $Q\{\vec{y}/\vec{x}\}$
denotes the capture-avoiding substitution of $\vec{y}$ for $\vec{x}$ in $Q$.
\end{definition}

\begin{definition}
  The {\em structural congruence} \cite{SangiorgiWalker} , $\equiv$,
  between processes is the least congruence containing
  alpha-equivalence, satisfying the abelian monoid laws
  (associativity, commutativity and $\pzero$ as identity) for parallel
  composition $|$ and for summation $+$.
\end{definition}

\subsection{Name equivalence}

We take name equivalence, written $\nameeq$, to be the smallest
equivalence relation generated by the following rules.

\begin{mathpar}
\inferrule*[lab=Quote-drop]
{ }
{ \quotep{@{x}} \nameeq x }

\inferrule*[lab=Struct-equiv]
{ P \scong Q }
{ \quotep{P} \nameeq \quotep{Q} }
\end{mathpar}

The astute reader will have noticed that the mutual recursion of names
and processes imposes a mutual recursion on alpha-equivalence and
structural equivalence via name-equivalence. Fortunately, all of this
works out pleasantly and we may calculate in the natural way, free of
concern. The reader interested in the details is referred to the
appendix \ref{appendix:rho_details}.

\subsection{Substitution}

We use $\Proc$ for the set of processes, $\QProc$ for the set of
names, and $\id{\{}\vec{y} / \vec{x} \id{\}}$ to denote partial maps,
$s : \QProc \rightarrow \QProc$. A map, $s$ lifts, uniquely, to a map
on process terms, $\widehat{s} : \Proc \rightarrow \Proc$ by the
following equations.

\begin{mathpar}
  (0) \psubstp{Q}{P} := 0 \\
  (R \juxtap S) \psubstp{Q}{P}
  :=    
  (R)\psubstp{Q}{P} \juxtap (S) \psubstp{Q}{P} \\
  (x?(y).R) \psubstp{Q}{P}    
  :=    
  (x)\substp{Q}{P} (z)\concat( (R \psubstn{z}{y}) \psubstp{Q}{P} ) \\
  (\lift{x}{R}) \psubstp{Q}{P}  
  :=
  \lift{(x)\substp{Q}{P}}{ R \psubstp{Q}{P} } \\
%   (\dropn{x})  \psubstp{Q}{P}       
%   := 
%   \left\{ 
%     \begin{array}{ccc} 
%       \dropn{\quotep{Q}} & & x \nameeq \quotep{P} \\
%       \dropn{x} & & otherwise \\
%     \end{array}
%   \right. 
  (\dropn{x})  \psubstp{Q}{P}       
  := 
  \left\{ 
    \begin{array}{ccc} 
      Q & & x \nameeq \quotep{P} \\
      \dropn{x} & & otherwise \\
    \end{array}
  \right.
\end{mathpar}
 

where

\begin{eqnarray}
  (x)\id{\{} \lpquote Q \rpquote / \lpquote P \rpquote \id{\}}            = 
  \left\{ 
    \begin{array}{ccc}
      \lpquote Q \rpquote & & x \nameeq \lpquote P \rpquote \\
      x & & otherwise \\
    \end{array}
  \right. \nonumber
\end{eqnarray}

and $z$ is chosen distinct from $\quotep{P}$, $\quotep{Q}$, the free
names in $Q$, and all the names in $R$. Our $\alpha$-equivalence will
be built in the standard way from this substitution.

\begin{remark}\label{rem:no_self_referential_names}
  One consequence of these definitions is that $\forall P. \quotep{P}
  \not\in \freenames{P}$.
\end{remark}

\subsection{ Dynamic quote: an example }

Anticipating something of what's to come, consider applying the
substitution, $\widehat{\id{\{}u / z \id{\}}}$, to the following pair
of processes, $\lift{w}{y!(z)}$ and $w[ \lpquote y!(z) \rpquote ]$.

\begin{eqnarray}
	\lift{w}{y!(z)}\widehat{\id{\{}u / z \id{\}}}
		& = &
		\lift{w}{y!(u)} \nonumber\\
	w[ \lpquote y!(z) \rpquote ] \widehat{ \id{\{}u / z \id{\}} }
		& = &
		w[ \lpquote y!(z) \rpquote ] \nonumber
\end{eqnarray}

Because the body of the process between quotes is impervious to
substitution, we get radically different answers. In fact, by
examining the first process in an input context,
e.g. $x?(z).\lift{w}{y!(z)}$, we see that the process under the lift
operator may be shaped by prefixed inputs binding a name inside it. In
this sense, the lift operator will be seen as a way to dynamically
construct processes before reifying them as names.

Finally equipped with these standard features we can present the
dynamics of the calculus.

\subsubsection{Operational semantics} 

Finally, we introduce the computational dynamics. What marks these
algebras as distinct from other more traditionally studied algebraic
structures, e.g. vector spaces or polynomial rings, is the manner in
which dynamics is captured. In traditional structures, dynamics is typically
expressed through morphisms between such structures, as in linear maps
between vector spaces or morphisms between rings. In algebras
associated with the semantics of computation, the dynamics is
expressed as part of the algebraic structure itself, through a
reduction reduction relation typically denoted by $\red$. Below, we
give a recursive presentation of this relation for the calculus used
in the encoding.

$\red \subseteq \pi \times \pi$
$\red : \pi \to \mathcal{P}(\pi)$

\begin{mathpar}
  \inferrule* [lab=Comm] { \textsf{match}( x_{src}, x_{trgt} ) } { x_{trgt}?(y)P \; | \; x_{src}!\langle {Q} \rangle \red P\{\quotep{Q}/y}\} }
  \and \\
  \inferrule* [lab=Par] {{P} \red {P}'} {{{P} | {Q}} \red {{P}' | {Q}}}
  \and
  \inferrule* [lab=Equiv]{{{P} \scong {P}'} \andalso {{P}' \red {Q}'} \andalso {{Q}' \scong {Q}}}{{P} \red {Q}}
\end{mathpar}

\begin{eqnarray*}
  match_{\equiv} (\quotep{P},\quotep{Q}) & := & P \equiv Q \\
  match_{\dagger}(\quotep{P},\quotep{Q}) & := & \forall R. P|Q \red^{*} R => R \red^{*} 0 \\
  match_{K}(\quotep{P},\quotep{Q}) & := & K \mbox{ for some context } K
\end{eqnarray*}

$u?(x)P | u!\langle Q \rangle \red P\{\quotep{Q}/x\}$

%We write $\wred$ for $\red^*$, and $P\red$ if $\exists Q $ such that $ P \red Q$.
We write $P\red$ if $\exists Q $ such that $ P \red Q$ and $P\not\red$, otherwise.

\section{Replication}

As mentioned before, it is known that replication (and hence
recursion) can be implemented in a higher-order process algebra
\cite{SangiorgiWalker}. As our first example of calculation with the
machinery thus far presented we give the construction explicitly in
the {\rhoc}.

\begin{eqnarray}
	D_{x} & := & \prefix{x}{y}{(\binpar{\outputp{x}{y}}{@{y}})} \nonumber\\
	\bangp_{x}{P} & := & \binpar{{x}!\langle{\binpar{D_{x}}{P}}\rangle}{D_{x}} \nonumber
\end{eqnarray}

\begin{eqnarray}
	\bangp_{x}{P} & & \nonumber\\
	=
	& {x}!\langle{(\prefix{x}{y}{(\outputp{x}{y} | @{y})) | P}}\rangle 
	      | \prefix{x}{y}{(\outputp{x}{y} | @{y})} & \nonumber\\
	\red
	& (\outputp{x}{y} | @{y})\substn{\quotep{(\prefix{x}{y}{(@{y} | \outputp{x}{y})) | P}}}{y} & \nonumber\\
	=
	& \outputp{x}{\quotep{(\prefix{x}{y}{(\outputp{x}{y} | @{y})) | P}}}
	  | {(\prefix{x}{y}{(\outputp{x}{y} | @{y})) | P}} & \nonumber\\
	\red
	& \ldots & \nonumber\\
	\red^*
	& P | P | \ldots & \nonumber
\end{eqnarray}

Of course, this encoding, as an implementation, runs away, unfolding
$\bangp{P}$ eagerly. A lazier and more implementable replication
operator, restricted to input-guarded processes, may be obtained as follows.

\begin{eqnarray}
\bangp{\prefix{u}{v}{P}} 
	:= 
	\binpar{\lift{x}{\prefix{u}{v}{(\binpar{D(x)}{P})}}}{D(x)} \nonumber
\end{eqnarray}

\begin{remark}
  Note that the lazier definition still does not deal with summation
  or mixed summation (i.e. sums over input and output). The reader is
  invited to construct definitions of replication that deal with these
  features. 

  Further, the definitions are parameterized in a name, $x$. Can you,
  gentle reader, make a definition that eliminates this parameter and
  guarantees no accidental interaction between the replication
  machinery and the process being replicated -- i.e. no accidental
  sharing of names used by the process to get its work done and the
  name(s) used by the replication to effect copying. This latter
  revision of the definition of replication is crucial to obtaining
  the expected identity $!!P \sim !P$.
\end{remark}

\begin{remark}\label{rem:paradoxical_combinator}
  The reader familiar with the lambda calculus will have noticed the
  similarity between $D$ and the paradoxical combinator.

  [Ed. note: the existence of this seems to suggest we have to be more
  restrictive on the set of processes and names we admit if we are to
  support no-cloning.]
\end{remark}

\subsubsection{Bisimulation}

The computational dynamics gives rise to another kind of equivalence,
the equivalence of computational behavior. As previously mentioned
this is typically captured \emph{via} some form of bisimulation.

% The notion we use in this paper is weak barbed bisimulation
% \cite{milner91polyadicpi}.

The notion we use in this paper is derived from weak barbed
bisimulation \cite{milner91polyadicpi}. 

\begin{definition}
An \emph{observation relation}, $\downarrow_{\mathcal N}$, over a set
of names, $\mathcal N$, is the smallest relation satisfying the rules
below.

\infrule[Out-barb]{y \in {\mathcal N}, \; x \nameeq y}
		  {\outputp{x}{v} \downarrow_{\mathcal N} x}
\infrule[Par-barb]{\mbox{$P\downarrow_{\mathcal N} x$ or $Q\downarrow_{\mathcal N} x$}}
		  {\binpar{P}{Q} \downarrow_{\mathcal N} x}

We write $P \Downarrow_{\mathcal N} x$ if there is $Q$ such that 
$P \wred Q$ and $Q \downarrow_{\mathcal N} x$.
\end{definition}

\begin{definition}
%\label{def.bbisim}
An  ${\mathcal N}$-\emph{barbed bisimulation} over a set of names, ${\mathcal N}$, is a symmetric binary relation 
${\mathcal S}_{\mathcal N}$ between agents such that $P\rel{S}_{\mathcal N}Q$ implies:
\begin{enumerate}
\item If $P \red P'$ then $Q \wred Q'$ and $P'\rel{S}_{\mathcal N} Q'$.
\item If $P\downarrow_{\mathcal N} x$, then $Q\Downarrow_{\mathcal N} x$.
\end{enumerate}
$P$ is ${\mathcal N}$-barbed bisimilar to $Q$, written
$P \wbbisim_{\mathcal N} Q$, if $P \rel{S}_{\mathcal N} Q$ for some ${\mathcal N}$-barbed bisimulation ${\mathcal S}_{\mathcal N}$.
\end{definition}

$\mathcal{R} \subseteq \pi \times \pi$

$P \mathcal{R} Q => \forall P'. P \red P' \Rightarrow \exists Q'. Q \red Q', P' \mathcal{R} Q'$

$P \vdash x \Rightarrow Q \vdash x$

\begin{mathpar}
  \inferrule*[lab=Out-barb]{x \nameeq y}{{y}!\langle{Q}\rangle \vdash x}
  \and
  \inferrule*[lab=Par-barb]{\mbox{$P\vdash x$ or $Q\vdash x$}}{\binpar{P}{Q} \vdash x}
\end{mathpar}

\subsubsection{Contexts}

One of the principle advantages of computational calculi like the
$\pi$-calculus is a well-defined notion of context,
contextual-equivalence and a correlation between
contextual-equivalence and notions of bisimulation. The notion of
context allows the decomposition of a process into (sub-)process and
its syntactic environment, its context. Thus, a context may be
thought of as a process with a ``hole'' (written $\Box$) in it. The
application of a context $M$ to a process $P$, written $M[P]$, is
tantamount to filling the hole in $M$ with $P$. In this paper we do
not need the full weight of this theory, but do make use of the notion
of context in the proof the main theorem. 

\begin{mathpar}
  \inferrule* [lab=summation] {} {{M_{M},M_{N}} \bc \Box \;|\; x.M_{A} \;|\; M_{M}+M_{N}}
  \and
  \inferrule* [lab=agent] {} {{M_{A}} \bc (\vec{x})M_{P} \;| \; \clift{P_0,\ldots,M_{P},\ldots,P_N}}
  \and \\
  \inferrule* [lab=process] {} {{M_{P}} \bc M_{N} \;| \;P|M_{P} }
\end{mathpar} 

\begin{mathpar}
  \inferrule* [lab=sychronization] {} {M_{N} \bc \Box \;|\; x?M_{F} \;|\; x!M_{C}}
  \and
  \inferrule* [lab=abstraction] {} {{M_{F}} \bc (x)M_{P} }
  \and
  \inferrule* [lab=concretion] {} {{M_{C}} \bc \langle M_{P} \rangle }
  \and \\
  \inferrule* [lab=process] {} {{M_{P}} \bc M_{N} \;| \;P|M_{P} }
\end{mathpar}

\begin{definition}[contextual application] Given a context $M$, and
  process $P$, we define the \emph{contextual application}, $M[P] :=
  M\{P/\Box\}$. That is, the contextual application of M to P is the
  substitution of $P$ for $\Box$ in $M$.
\end{definition}

$\meaningof{-} : L \to \mathcal{P}(\pi)$

\begin{mathpar}
  \inferrule* [lab=collection] {} {\meaningof{true} = \pi, \and \meaningof{~E} = \pi \setminus \meaningof{E}, \and \meaningof{E_{1} \& E_{2}} = \meaningof{E_{1}} \cap \meaningof{E_{2}}}
\end{mathpar}

\begin{mathpar}
  \inferrule* [lab=structure] {} {\meaningof{0} = \{ P \in \pi | P \equiv 0 \}, \and \\ \meaningof{E_1 | E_2} = \{ P \in \pi | P \equiv P_{1} | P_{2}, P_{1} \in \meaningof{E_{1}}, P_{2} \in \meaningof{E_2}\} }
\end{mathpar}

\begin{mathpar}
 \inferrule* [lab=behavior] {} {\meaningof{\langle a?b \rangle E} = \{ P \in \pi | P \equiv Q | u?(y)P', \\ \and \\\\ \and \\ \;\;\; u \in \meaningof{a}, \forall z.P'\{z/y\} \in \meaningof{E\{z/b\}}\}, \and \\ \meaningof{a!E} = \{ P \in \pi | P \equiv Q | x!\langle P' \rangle, x \in \meaningof{a} P' \in \meaningof{E}\} }
\end{mathpar}

\begin{mathpar}
 \inferrule* [lab=nominal] {} {\meaningof{\quotep{E}} = \{ \quotep{P} \in \quotep{\pi} | P \in \meaningof{E} \}, \and \meaningof{\quotep{P}} = \{ \quotep{Q} \in \quotep{\pi} | P \equiv Q \} \and \\ \meaningof{@\quotep{E}} = \{ P \in \pi | P \equiv @x, x \in \meaningof{E} \}}
\end{mathpar}

\begin{eqnarray*}
  \\
  \meaningof{-} : TS \to ST
\end{eqnarray*}

\begin{eqnarray*}
  \\
  L : TS \to ST
\end{eqnarray*}

\begin{eqnarray*}
  \\
  P \models E \iff P \in \meaningof{E}
\end{eqnarray*}

\begin{eqnarray*}
  P \approx_{L} Q \iff \forall E \in L. P \models E \iff Q \models E
\end{eqnarray*}

\begin{eqnarray*}
  P \approx_{K} Q
\end{eqnarray*}

\begin{eqnarray*}
  P \approx Q
\end{eqnarray*}

$\approx_{K} = \approx = \approx_{L}$

\subsubsection{Contextual duality}

Note that contexts extend the quotation operation to a family of
operations from processes to names. Given a context, $M$, we can
define a \emph{nominal context}, $\quotep{M}$ by $\quotep{M}[P] :=
\quotep{M[P]}$. To foreshadow what is to come we observe that these
operations enjoy a duality with processes very much like the duality
between vectors and maps from vectors to scalars.

Further, because the calculus is essentially higher-order, we have a
correspondence between contexts and processes. More specifically,
given a name $x$ and a context $M$ we can construct $M^{*}_{x}$ such
that 

\begin{mathpar}
  M^{*}_{x} | \lift{x}{P} \red M[P]
\end{mathpar}

namely,

\begin{mathpar}
  M^{*}_{x} := x?(u).M[\dropn{u}]
\end{mathpar}

The dependence of $M^{*}_{x}$ on a name makes it an abstraction, 

\begin{mathpar}
  M^{*} := (x)x?(u).M[\dropn{u}]
\end{mathpar}

\subsection{Additional notation}

It will sometimes be convenient to denote the process a name
quotes. We already have the notation $x = \quotep{P}$, but it will be
convenient to introduce an alternate notation, $\procn{x}$, when we
want to emphasize the connection to the use of the name. Note that, by
virtue of name equivalence, $\quotep{\procn{x}} \nameeq x$; so, the
notation is consistent with previous definitions.

Further, because names have structure it is possible to effect
substitutions on the basis of that structure. This means we need to
upgrade our notation for substitutions, which we accomplish by
adapting comprehension notation. Thus,

\begin{mathpar}
  P\{ y / x : x \in S \}
\end{mathpar}

is interpreted to mean the process derived from P by replacing (in a
capture-avoiding manner) each occurrence of $x$ in $S$ by $y$. For example,

\begin{mathpar}
  P\{ \quotep{\procn{x}|\procn{x}} / x : x \in \freenames{P} \}
\end{mathpar}

will replace each (occurrence) of a free name $x$ in $P$ by
$\quotep{\procn{x}|\procn{x}}$.

Also, we will avail ourselves of the notation $x^{L}$ and $x^{R}$ to
denote injections of a name into disjoint copies of the name
space. There are numerous ways to accomplish this. One example can be
found in \cite{MeredithR05}. This notation overloads to vectors of
names: $\vec{x}^{\pi} := (x_{i}^{\pi} \; : \; 0 \leq i < |\vec{x}| )$ where $\pi \in \{L,R\}$.

We also use $P^{\Box} := P|\Box$.

In \cite{MeredithR05} an interpretation of the new operator is
given. It turns out that there are several possible interpretations
all enjoying the requisite algebraic properties of the operator (see
\cite{milner91polyadicpi}). We will therefore make liberal use of
$(\nu\; \vec{x})P$.

% subsection the_syntax_and_semantics_of_the_notation_system (end)   

\input{qm2pi.qmops} 

\input{qm2pi.sterngerlach} 

\input{qm2pi.metric} 

% section concurrent_process_calculi (end)

%\input{qm2pi.proofsketch}

% section proof sketch (end)

%\input{qm2pi.slviaknots} 

% section spatial logic via knots (end)

\input{qm2pi.conclusion}

% section conclusion (end)

%\input{qm2pi.dtcodes} 

% section wiring algorithm (end)

\input{qm2pi.ack} 

% section acknowledgments (end)

\newpage


\bibliographystyle{plain}   
\bibliography{../../biblios/main.bib}

\input{qm2pi.rhodetails}

\end{document}



% section front matter (end)

\section{Introduction}\label{sec:introduction} % (fold)
In this draft of the material i am going to have to dispense with the
usual writing conventions adopted in papers on these topics. i'm going
to have adopt whatever tone i need at the time i'm writing up the
calculations. Sometimes this may be very conversational; others it may
be the barest mathematical grunts; others still it may be that i have
lifted text from one of my other papers because the exposition of some
point was better said there. i hope that my readers are not unduly put
out by this decision. i'm not doing this to flout convention or be
rebellious. i find these calculations very technically challenging. To
keep everything going technically, something has to give; i have to
let go of some cognitive burden. So, the academic writing style --
with all of its trade-offs in terms of facilitating technical
communication -- is what i'm letting go of. Perhaps subsequent drafts
can be tightened and polished, but for now, i'm going to speak as if
we were sitting together in a coffee shop with a laptop, wifi and a
pad of paper and a pencil.

So, here's what i have to say. We -- you and i, comfortably ensconced
in our coffee shop and well-equipped with our tools -- can realize and
carry out the calculations of quantum mechanics over a very different
formal theory of dynamics, a formal theory of dynamics that
corresponds to a theory of concurrent computation with
\emph{reflection}. It has the advantage that the underlying theory is
already `quantized', but supports analogues all of the continuuous
operations. Strikingly, this underlying theory has recently been
connected with a notion of metric that we can show, by calculating
together, coincides with the metric induced by the inner product.

There are a lot of reasons why you might be interested in seeing
calculations of this form. Here's why i'm interested. For the past
several centuries there has been no competitor to the ``Newtonian''
account of dynamics. As a result the predominant share of accounts of
dynamical systems and situations have had to be formulated in terms of
the Newtonian machinery. i view this as an intellectually dangerous
position to occupy. Everything, despite it's intrinsic shape, turns
into a nail to be hit with this hammer. Recently, however, the theory
of computation has matured to the point where we have candidates for
theories of dynamics that offer very different perspective on
reasoning about dynamical systems and situations. Testing these
candidates against very successful accounts of dynamical situations,
like quantum mechanics, is going to give us some sense of how mature
they are and some measure of the quality of these accounts of
dynamics.

\subsection{Summary of contributions and outline of paper}

So, we're going to develop an interpretation of the operations of
quantum mechanics normally interpreted by Hilbert spaces and
operators. We're going to do this over a theory of computation. Note
that this is very different than the usual quantum computation program
which develops notions of computation over quantum mechanics. Rather,
we are developing a story that aligns with Wheeler's slogan: It from
Bit. To do this we will first provide an account of the theory of
computation at play here. Then we will dive into a calculation-driven
interpretation of the operations of quantum mechanics.

The reason we take this approach is that -- until very recently --
there hasn't been an axiomatic account of quantum mechanics. As a
result there has been no sharp delineation of the mathematical theory
supporting interpretation of the physical theory and the physical
theory, itself. So, ambient features of the maths are free to be
exploited (or supressed) without a real accounting of their physical
relevance. There is no sharp statement ``here's the physical theory''
qua \emph{theory} and ``here's the mathematical interpretation''
enabling a judgment of how faithful the interpretation is -- apart
from experimental observation. When there is an axiomatic account we
can judge how well a given mathematical formalism supports an
interpretation of the axioms, independent of
experimentation. Likewise, we can judge how well we have captured our
physical evidence and experience with our axiomatics, independent of
any specific mathematical implementation, with accidental detail that
may or may not have physical significance. 

In lieu of a fully fleshed out and vetted axiomatic account of quantum
mechanics, interpreting the operational notions in service of modeling
physical systems will have to suffice. In other words, we are not in
the business of providing a model of Hilbert spaces and operators. We
are in the business of providing a model of quantum mechanics because
we are motivated by testing our notions of dynamics against physical
theory; and, the predictive calculations of the physical theory must
serve as the best formulation -- shy of a fully fleshed out axiomatic
account -- of the physical theory itself (as they have for scientific
theories since time immemorial). Put another way, despite a
whole-hearted commitment to an It-from-Bit ontology, we are firmly
aligned with the shut-up-and-calculate camp as the best way to obtain
results either from the physical perspective or as a quality assurance
measure of our fledgling theory of dynamics.

In detail, we present a reflective process calculus. Then we develop
intuitive correspondences between the notions available in this
calculus and the usual physical notions supporting quantum mechanical
calculations. Thus, 

\begin{table}[htp]
  \center{
    \fbox{
      \begin{tabular}{c|c}
        quantum mechanics & process calculus \\
        \hline
        scalar & name \\
        state vector & process \\
        dual & contextual duals \\
        matrix & formal sums of process-context-dual pairs \\
        orthogonality & process annihilation \\
        inner product & execution-formula + quoting
      \end{tabular}
    }
  }
  \caption{QM - process calculi correspondences}
\end{table}

Then we tighten up these intuitions to operational definitions. We
employ the Dirac notation as the best proxy we can find for an
abstract syntax of the quantum mechanical notions. The definitions we
develop put us in contact with equational constraints coming from the
theory that we demonstrate the definitions and calculations satisfy.

This puts us in a position to shut up and calculate for the
Stern-Gerlach experimental set up, showing how these predictive
calculations become calculations on processes in our theory of a
reflective process calculus.

Penultimately, we demonstrate that the notion of metric coming from
the inner product coincides with the notion of metric available from
the theory of bisimulation. This demonstration gives us the right to
think of space as arising from behavior. Finally, we consider where we
might go from the new vantage point we have obtained.

% section introduction (end) 
 
% section introduction (end)

% \documentclass[12pt]{llncs}
%\documentclass{jktr}

\usepackage[pdftex]{hyperref}                   
\usepackage {listings}
\usepackage {mathpartir}
\usepackage{bcprules}
%\usepackage{listings}
                       
\usepackage{graphicx} 
%\usepackage[margins=2.5cm,nohead,nofoot]{geometry}
%\usepackage{geometry}
\usepackage{amsfonts}
\usepackage{amstext}
\usepackage{latexsym}
\usepackage{amssymb}
\usepackage{color}


%\include{myPreamble}
\include{qm2pi.local} 

%\ifpdf
%\usepackage[pdftex]{graphicx}
%\else
%\usepackage{graphicx}
%\fi

 % \ifpdf
%  \usepackage{pdfsync}
%  \if


%\title{Brief Article}
%\author{David F. Snyder}
%\author{L.G. Meredith}

%\address{Dept. of Math., Texas State University--San Marcos, San Marcos, TX 78666}
       
\pagestyle{empty}


\begin{document}

\lstset{language=[Objective]Caml,frame=shadowbox}

\input{qm2pi.front}

% section front matter (end)

\input{qm2pi.intro} 
 
% section introduction (end)

% \input{qm2pi.knotations} 

% section notation (end)

\input{qm2pi.process.calculi} 

% section concurrent_process_calculi_and_spatial_logics_ (end)
    
%\input{qm2pi.knots2pi} 

%\input{qm2pi.trefoil} 

%\input{qm2pi.mainthm} 

% subsection basic_interpretation (end)

%\input{qm2pi.rho.presentation} 
\subsection{The syntax and semantics of the notation system}\label{sub:the_syntax_and_semantics_of_the_notation_system} % (fold)

We now summarize a technical presentation of the calculus that
embodies our theory of dynamics. The typical presentation of such a
calculus follows the style of giving generators and relations on
them. The grammar, below, describing term constructors, freely
generates the set of processes, $\Proc$. This set is then quotiented
by a relation known as structural congruence and it is over this set
that the notion of dynamics is expressed. This presentation is
essentially that of \cite{MeredithR05} with the addition of
polyadicity and summation. For readability we have relegated some of
the technical subtleties to an appendix.

\subsubsection{Process grammar}\label{subsub:process_grammar}

\begin{mathpar}
  \inferrule* [lab=synchronization] {} {{M} \bc \pzero \;|\; x?F \;|\; x!C }
  \and
  \inferrule* [lab=abstraction] {} {{F} \bc (x)P}
  \and
  \inferrule* [lab=concretion] {} {{C} \bc \langle Q \rangle}
  \and
  \inferrule* [lab=process] {} {{P,Q} \bc M \;| \;P|Q \;|\; @{x}}
  \and
  \inferrule* [lab=name] {} {{x} \bc \quotep{P}}
\end{mathpar} 

Note that $\vec{x}$ (resp. $\vec{P}$) denotes a vector of names
(resp. processes) of length $|\vec{x}|$ (resp. $|\vec{P}|$). We adopt
the following useful abbreviations.

\begin{mathpar}
   x?(\vec{y}).P := x.(\vec{y})P \and  x\clift{\vec{P}} := x.\clift{\vec{P}}
   \and x!(y) := \lift{x}{\dropn{y}}
   \and \Pi_{i=0}^{n-1}P_i := P_0 | \ldots | P_{n-1}
\end{mathpar}

\subsubsection{Structural congruence}

\paragraph{Free and bound names and alpha-equivalence.} At the
core of structural equivalence is alpha-equivalence which identifies
process that are the same up to a change of variable. Formally, we
recognize the distinction between free and bound names. The free names
of a process, $\freenames{P}$, may be calculated recursively as
follows:

\begin{mathpar}
\freenames{\pzero} := \emptyset
  \and \\
  \freenames{x?(y).P} := \{ x \} \cup (\freenames{P} \setminus \{ y \})
  \and 
  \freenames{x!\langle P \rangle} := \{ x \} \cup \{ P \} 
  \and \\
  \freenames{P|Q} := \freenames{P} \cup \freenames{Q}
  \and \\
  \freenames{@{x}} := \{ x \}
\end{mathpar}

$\pi$
$\quotep{\pi}$

$\freenames{-} : \pi \to \mathcal{P}(\quotep{\pi})$

\begin{eqnarray*}
  \freenames{\pzero} & := & \emptyset \\
  \freenames{x?(y).P} & := & \{ x \} \cup (\freenames{P} \setminus \{ y \}) \\
  \freenames{x!\langle P \rangle} & := & \{ x \} \cup \{ P \} \\
  \freenames{P|Q} & := & \freenames{P} \cup \freenames{Q} \\
  \freenames{\dropn{x}} & := & \{ x \}
\end{eqnarray*}

The bound names of a process, $\boundnames{P}$, are those names occurring in $P$
that are not free. For example, in $x?(y).0$, the name $x$ is free, while $y$ is bound.

\begin{mathpar}
  \inferrule* [lab=monoidal-laws] {} { P|Q \equiv Q|P \and P|0 \equiv P \and P|(Q|R) \equiv (P|Q)|R }
\end{mathpar}

\begin{mathpar}
  \inferrule* [lab=alpha-equivalence] {} { (x)P \equiv (y)P\{y/x\} \and y \not\in \freenames{P} }
\end{mathpar}

\begin{definition}
Then two processes, $P,Q$, are alpha-equivalent if $P = Q\{\vec{y}/\vec{x}\}$ for
some $\vec{x} \in \boundnames{Q},\vec{y} \in \boundnames{P}$, where $Q\{\vec{y}/\vec{x}\}$
denotes the capture-avoiding substitution of $\vec{y}$ for $\vec{x}$ in $Q$.
\end{definition}

\begin{definition}
  The {\em structural congruence} \cite{SangiorgiWalker} , $\equiv$,
  between processes is the least congruence containing
  alpha-equivalence, satisfying the abelian monoid laws
  (associativity, commutativity and $\pzero$ as identity) for parallel
  composition $|$ and for summation $+$.
\end{definition}

\subsection{Name equivalence}

We take name equivalence, written $\nameeq$, to be the smallest
equivalence relation generated by the following rules.

\begin{mathpar}
\inferrule*[lab=Quote-drop]
{ }
{ \quotep{@{x}} \nameeq x }

\inferrule*[lab=Struct-equiv]
{ P \scong Q }
{ \quotep{P} \nameeq \quotep{Q} }
\end{mathpar}

The astute reader will have noticed that the mutual recursion of names
and processes imposes a mutual recursion on alpha-equivalence and
structural equivalence via name-equivalence. Fortunately, all of this
works out pleasantly and we may calculate in the natural way, free of
concern. The reader interested in the details is referred to the
appendix \ref{appendix:rho_details}.

\subsection{Substitution}

We use $\Proc$ for the set of processes, $\QProc$ for the set of
names, and $\id{\{}\vec{y} / \vec{x} \id{\}}$ to denote partial maps,
$s : \QProc \rightarrow \QProc$. A map, $s$ lifts, uniquely, to a map
on process terms, $\widehat{s} : \Proc \rightarrow \Proc$ by the
following equations.

\begin{mathpar}
  (0) \psubstp{Q}{P} := 0 \\
  (R \juxtap S) \psubstp{Q}{P}
  :=    
  (R)\psubstp{Q}{P} \juxtap (S) \psubstp{Q}{P} \\
  (x?(y).R) \psubstp{Q}{P}    
  :=    
  (x)\substp{Q}{P} (z)\concat( (R \psubstn{z}{y}) \psubstp{Q}{P} ) \\
  (\lift{x}{R}) \psubstp{Q}{P}  
  :=
  \lift{(x)\substp{Q}{P}}{ R \psubstp{Q}{P} } \\
%   (\dropn{x})  \psubstp{Q}{P}       
%   := 
%   \left\{ 
%     \begin{array}{ccc} 
%       \dropn{\quotep{Q}} & & x \nameeq \quotep{P} \\
%       \dropn{x} & & otherwise \\
%     \end{array}
%   \right. 
  (\dropn{x})  \psubstp{Q}{P}       
  := 
  \left\{ 
    \begin{array}{ccc} 
      Q & & x \nameeq \quotep{P} \\
      \dropn{x} & & otherwise \\
    \end{array}
  \right.
\end{mathpar}
 

where

\begin{eqnarray}
  (x)\id{\{} \lpquote Q \rpquote / \lpquote P \rpquote \id{\}}            = 
  \left\{ 
    \begin{array}{ccc}
      \lpquote Q \rpquote & & x \nameeq \lpquote P \rpquote \\
      x & & otherwise \\
    \end{array}
  \right. \nonumber
\end{eqnarray}

and $z$ is chosen distinct from $\quotep{P}$, $\quotep{Q}$, the free
names in $Q$, and all the names in $R$. Our $\alpha$-equivalence will
be built in the standard way from this substitution.

\begin{remark}\label{rem:no_self_referential_names}
  One consequence of these definitions is that $\forall P. \quotep{P}
  \not\in \freenames{P}$.
\end{remark}

\subsection{ Dynamic quote: an example }

Anticipating something of what's to come, consider applying the
substitution, $\widehat{\id{\{}u / z \id{\}}}$, to the following pair
of processes, $\lift{w}{y!(z)}$ and $w[ \lpquote y!(z) \rpquote ]$.

\begin{eqnarray}
	\lift{w}{y!(z)}\widehat{\id{\{}u / z \id{\}}}
		& = &
		\lift{w}{y!(u)} \nonumber\\
	w[ \lpquote y!(z) \rpquote ] \widehat{ \id{\{}u / z \id{\}} }
		& = &
		w[ \lpquote y!(z) \rpquote ] \nonumber
\end{eqnarray}

Because the body of the process between quotes is impervious to
substitution, we get radically different answers. In fact, by
examining the first process in an input context,
e.g. $x?(z).\lift{w}{y!(z)}$, we see that the process under the lift
operator may be shaped by prefixed inputs binding a name inside it. In
this sense, the lift operator will be seen as a way to dynamically
construct processes before reifying them as names.

Finally equipped with these standard features we can present the
dynamics of the calculus.

\subsubsection{Operational semantics} 

Finally, we introduce the computational dynamics. What marks these
algebras as distinct from other more traditionally studied algebraic
structures, e.g. vector spaces or polynomial rings, is the manner in
which dynamics is captured. In traditional structures, dynamics is typically
expressed through morphisms between such structures, as in linear maps
between vector spaces or morphisms between rings. In algebras
associated with the semantics of computation, the dynamics is
expressed as part of the algebraic structure itself, through a
reduction reduction relation typically denoted by $\red$. Below, we
give a recursive presentation of this relation for the calculus used
in the encoding.

$\red \subseteq \pi \times \pi$
$\red : \pi \to \mathcal{P}(\pi)$

\begin{mathpar}
  \inferrule* [lab=Comm] { \textsf{match}( x_{src}, x_{trgt} ) } { x_{trgt}?(y)P \; | \; x_{src}!\langle {Q} \rangle \red P\{\quotep{Q}/y}\} }
  \and \\
  \inferrule* [lab=Par] {{P} \red {P}'} {{{P} | {Q}} \red {{P}' | {Q}}}
  \and
  \inferrule* [lab=Equiv]{{{P} \scong {P}'} \andalso {{P}' \red {Q}'} \andalso {{Q}' \scong {Q}}}{{P} \red {Q}}
\end{mathpar}

\begin{eqnarray*}
  match_{\equiv} (\quotep{P},\quotep{Q}) & := & P \equiv Q \\
  match_{\dagger}(\quotep{P},\quotep{Q}) & := & \forall R. P|Q \red^{*} R => R \red^{*} 0 \\
  match_{K}(\quotep{P},\quotep{Q}) & := & K \mbox{ for some context } K
\end{eqnarray*}

$u?(x)P | u!\langle Q \rangle \red P\{\quotep{Q}/x\}$

%We write $\wred$ for $\red^*$, and $P\red$ if $\exists Q $ such that $ P \red Q$.
We write $P\red$ if $\exists Q $ such that $ P \red Q$ and $P\not\red$, otherwise.

\section{Replication}

As mentioned before, it is known that replication (and hence
recursion) can be implemented in a higher-order process algebra
\cite{SangiorgiWalker}. As our first example of calculation with the
machinery thus far presented we give the construction explicitly in
the {\rhoc}.

\begin{eqnarray}
	D_{x} & := & \prefix{x}{y}{(\binpar{\outputp{x}{y}}{@{y}})} \nonumber\\
	\bangp_{x}{P} & := & \binpar{{x}!\langle{\binpar{D_{x}}{P}}\rangle}{D_{x}} \nonumber
\end{eqnarray}

\begin{eqnarray}
	\bangp_{x}{P} & & \nonumber\\
	=
	& {x}!\langle{(\prefix{x}{y}{(\outputp{x}{y} | @{y})) | P}}\rangle 
	      | \prefix{x}{y}{(\outputp{x}{y} | @{y})} & \nonumber\\
	\red
	& (\outputp{x}{y} | @{y})\substn{\quotep{(\prefix{x}{y}{(@{y} | \outputp{x}{y})) | P}}}{y} & \nonumber\\
	=
	& \outputp{x}{\quotep{(\prefix{x}{y}{(\outputp{x}{y} | @{y})) | P}}}
	  | {(\prefix{x}{y}{(\outputp{x}{y} | @{y})) | P}} & \nonumber\\
	\red
	& \ldots & \nonumber\\
	\red^*
	& P | P | \ldots & \nonumber
\end{eqnarray}

Of course, this encoding, as an implementation, runs away, unfolding
$\bangp{P}$ eagerly. A lazier and more implementable replication
operator, restricted to input-guarded processes, may be obtained as follows.

\begin{eqnarray}
\bangp{\prefix{u}{v}{P}} 
	:= 
	\binpar{\lift{x}{\prefix{u}{v}{(\binpar{D(x)}{P})}}}{D(x)} \nonumber
\end{eqnarray}

\begin{remark}
  Note that the lazier definition still does not deal with summation
  or mixed summation (i.e. sums over input and output). The reader is
  invited to construct definitions of replication that deal with these
  features. 

  Further, the definitions are parameterized in a name, $x$. Can you,
  gentle reader, make a definition that eliminates this parameter and
  guarantees no accidental interaction between the replication
  machinery and the process being replicated -- i.e. no accidental
  sharing of names used by the process to get its work done and the
  name(s) used by the replication to effect copying. This latter
  revision of the definition of replication is crucial to obtaining
  the expected identity $!!P \sim !P$.
\end{remark}

\begin{remark}\label{rem:paradoxical_combinator}
  The reader familiar with the lambda calculus will have noticed the
  similarity between $D$ and the paradoxical combinator.

  [Ed. note: the existence of this seems to suggest we have to be more
  restrictive on the set of processes and names we admit if we are to
  support no-cloning.]
\end{remark}

\subsubsection{Bisimulation}

The computational dynamics gives rise to another kind of equivalence,
the equivalence of computational behavior. As previously mentioned
this is typically captured \emph{via} some form of bisimulation.

% The notion we use in this paper is weak barbed bisimulation
% \cite{milner91polyadicpi}.

The notion we use in this paper is derived from weak barbed
bisimulation \cite{milner91polyadicpi}. 

\begin{definition}
An \emph{observation relation}, $\downarrow_{\mathcal N}$, over a set
of names, $\mathcal N$, is the smallest relation satisfying the rules
below.

\infrule[Out-barb]{y \in {\mathcal N}, \; x \nameeq y}
		  {\outputp{x}{v} \downarrow_{\mathcal N} x}
\infrule[Par-barb]{\mbox{$P\downarrow_{\mathcal N} x$ or $Q\downarrow_{\mathcal N} x$}}
		  {\binpar{P}{Q} \downarrow_{\mathcal N} x}

We write $P \Downarrow_{\mathcal N} x$ if there is $Q$ such that 
$P \wred Q$ and $Q \downarrow_{\mathcal N} x$.
\end{definition}

\begin{definition}
%\label{def.bbisim}
An  ${\mathcal N}$-\emph{barbed bisimulation} over a set of names, ${\mathcal N}$, is a symmetric binary relation 
${\mathcal S}_{\mathcal N}$ between agents such that $P\rel{S}_{\mathcal N}Q$ implies:
\begin{enumerate}
\item If $P \red P'$ then $Q \wred Q'$ and $P'\rel{S}_{\mathcal N} Q'$.
\item If $P\downarrow_{\mathcal N} x$, then $Q\Downarrow_{\mathcal N} x$.
\end{enumerate}
$P$ is ${\mathcal N}$-barbed bisimilar to $Q$, written
$P \wbbisim_{\mathcal N} Q$, if $P \rel{S}_{\mathcal N} Q$ for some ${\mathcal N}$-barbed bisimulation ${\mathcal S}_{\mathcal N}$.
\end{definition}

$\mathcal{R} \subseteq \pi \times \pi$

$P \mathcal{R} Q => \forall P'. P \red P' \Rightarrow \exists Q'. Q \red Q', P' \mathcal{R} Q'$

$P \vdash x \Rightarrow Q \vdash x$

\begin{mathpar}
  \inferrule*[lab=Out-barb]{x \nameeq y}{{y}!\langle{Q}\rangle \vdash x}
  \and
  \inferrule*[lab=Par-barb]{\mbox{$P\vdash x$ or $Q\vdash x$}}{\binpar{P}{Q} \vdash x}
\end{mathpar}

\subsubsection{Contexts}

One of the principle advantages of computational calculi like the
$\pi$-calculus is a well-defined notion of context,
contextual-equivalence and a correlation between
contextual-equivalence and notions of bisimulation. The notion of
context allows the decomposition of a process into (sub-)process and
its syntactic environment, its context. Thus, a context may be
thought of as a process with a ``hole'' (written $\Box$) in it. The
application of a context $M$ to a process $P$, written $M[P]$, is
tantamount to filling the hole in $M$ with $P$. In this paper we do
not need the full weight of this theory, but do make use of the notion
of context in the proof the main theorem. 

\begin{mathpar}
  \inferrule* [lab=summation] {} {{M_{M},M_{N}} \bc \Box \;|\; x.M_{A} \;|\; M_{M}+M_{N}}
  \and
  \inferrule* [lab=agent] {} {{M_{A}} \bc (\vec{x})M_{P} \;| \; \clift{P_0,\ldots,M_{P},\ldots,P_N}}
  \and \\
  \inferrule* [lab=process] {} {{M_{P}} \bc M_{N} \;| \;P|M_{P} }
\end{mathpar} 

\begin{mathpar}
  \inferrule* [lab=sychronization] {} {M_{N} \bc \Box \;|\; x?M_{F} \;|\; x!M_{C}}
  \and
  \inferrule* [lab=abstraction] {} {{M_{F}} \bc (x)M_{P} }
  \and
  \inferrule* [lab=concretion] {} {{M_{C}} \bc \langle M_{P} \rangle }
  \and \\
  \inferrule* [lab=process] {} {{M_{P}} \bc M_{N} \;| \;P|M_{P} }
\end{mathpar}

\begin{definition}[contextual application] Given a context $M$, and
  process $P$, we define the \emph{contextual application}, $M[P] :=
  M\{P/\Box\}$. That is, the contextual application of M to P is the
  substitution of $P$ for $\Box$ in $M$.
\end{definition}

$\meaningof{-} : L \to \mathcal{P}(\pi)$

\begin{mathpar}
  \inferrule* [lab=collection] {} {\meaningof{true} = \pi, \and \meaningof{~E} = \pi \setminus \meaningof{E}, \and \meaningof{E_{1} \& E_{2}} = \meaningof{E_{1}} \cap \meaningof{E_{2}}}
\end{mathpar}

\begin{mathpar}
  \inferrule* [lab=structure] {} {\meaningof{0} = \{ P \in \pi | P \equiv 0 \}, \and \\ \meaningof{E_1 | E_2} = \{ P \in \pi | P \equiv P_{1} | P_{2}, P_{1} \in \meaningof{E_{1}}, P_{2} \in \meaningof{E_2}\} }
\end{mathpar}

\begin{mathpar}
 \inferrule* [lab=behavior] {} {\meaningof{\langle a?b \rangle E} = \{ P \in \pi | P \equiv Q | u?(y)P', \\ \and \\\\ \and \\ \;\;\; u \in \meaningof{a}, \forall z.P'\{z/y\} \in \meaningof{E\{z/b\}}\}, \and \\ \meaningof{a!E} = \{ P \in \pi | P \equiv Q | x!\langle P' \rangle, x \in \meaningof{a} P' \in \meaningof{E}\} }
\end{mathpar}

\begin{mathpar}
 \inferrule* [lab=nominal] {} {\meaningof{\quotep{E}} = \{ \quotep{P} \in \quotep{\pi} | P \in \meaningof{E} \}, \and \meaningof{\quotep{P}} = \{ \quotep{Q} \in \quotep{\pi} | P \equiv Q \} \and \\ \meaningof{@\quotep{E}} = \{ P \in \pi | P \equiv @x, x \in \meaningof{E} \}}
\end{mathpar}

\begin{eqnarray*}
  \\
  \meaningof{-} : TS \to ST
\end{eqnarray*}

\begin{eqnarray*}
  \\
  L : TS \to ST
\end{eqnarray*}

\begin{eqnarray*}
  \\
  P \models E \iff P \in \meaningof{E}
\end{eqnarray*}

\begin{eqnarray*}
  P \approx_{L} Q \iff \forall E \in L. P \models E \iff Q \models E
\end{eqnarray*}

\begin{eqnarray*}
  P \approx_{K} Q
\end{eqnarray*}

\begin{eqnarray*}
  P \approx Q
\end{eqnarray*}

$\approx_{K} = \approx = \approx_{L}$

\subsubsection{Contextual duality}

Note that contexts extend the quotation operation to a family of
operations from processes to names. Given a context, $M$, we can
define a \emph{nominal context}, $\quotep{M}$ by $\quotep{M}[P] :=
\quotep{M[P]}$. To foreshadow what is to come we observe that these
operations enjoy a duality with processes very much like the duality
between vectors and maps from vectors to scalars.

Further, because the calculus is essentially higher-order, we have a
correspondence between contexts and processes. More specifically,
given a name $x$ and a context $M$ we can construct $M^{*}_{x}$ such
that 

\begin{mathpar}
  M^{*}_{x} | \lift{x}{P} \red M[P]
\end{mathpar}

namely,

\begin{mathpar}
  M^{*}_{x} := x?(u).M[\dropn{u}]
\end{mathpar}

The dependence of $M^{*}_{x}$ on a name makes it an abstraction, 

\begin{mathpar}
  M^{*} := (x)x?(u).M[\dropn{u}]
\end{mathpar}

\subsection{Additional notation}

It will sometimes be convenient to denote the process a name
quotes. We already have the notation $x = \quotep{P}$, but it will be
convenient to introduce an alternate notation, $\procn{x}$, when we
want to emphasize the connection to the use of the name. Note that, by
virtue of name equivalence, $\quotep{\procn{x}} \nameeq x$; so, the
notation is consistent with previous definitions.

Further, because names have structure it is possible to effect
substitutions on the basis of that structure. This means we need to
upgrade our notation for substitutions, which we accomplish by
adapting comprehension notation. Thus,

\begin{mathpar}
  P\{ y / x : x \in S \}
\end{mathpar}

is interpreted to mean the process derived from P by replacing (in a
capture-avoiding manner) each occurrence of $x$ in $S$ by $y$. For example,

\begin{mathpar}
  P\{ \quotep{\procn{x}|\procn{x}} / x : x \in \freenames{P} \}
\end{mathpar}

will replace each (occurrence) of a free name $x$ in $P$ by
$\quotep{\procn{x}|\procn{x}}$.

Also, we will avail ourselves of the notation $x^{L}$ and $x^{R}$ to
denote injections of a name into disjoint copies of the name
space. There are numerous ways to accomplish this. One example can be
found in \cite{MeredithR05}. This notation overloads to vectors of
names: $\vec{x}^{\pi} := (x_{i}^{\pi} \; : \; 0 \leq i < |\vec{x}| )$ where $\pi \in \{L,R\}$.

We also use $P^{\Box} := P|\Box$.

In \cite{MeredithR05} an interpretation of the new operator is
given. It turns out that there are several possible interpretations
all enjoying the requisite algebraic properties of the operator (see
\cite{milner91polyadicpi}). We will therefore make liberal use of
$(\nu\; \vec{x})P$.

% subsection the_syntax_and_semantics_of_the_notation_system (end)   

\input{qm2pi.qmops} 

\input{qm2pi.sterngerlach} 

\input{qm2pi.metric} 

% section concurrent_process_calculi (end)

%\input{qm2pi.proofsketch}

% section proof sketch (end)

%\input{qm2pi.slviaknots} 

% section spatial logic via knots (end)

\input{qm2pi.conclusion}

% section conclusion (end)

%\input{qm2pi.dtcodes} 

% section wiring algorithm (end)

\input{qm2pi.ack} 

% section acknowledgments (end)

\newpage


\bibliographystyle{plain}   
\bibliography{../../biblios/main.bib}

\input{qm2pi.rhodetails}

\end{document}

 

% section notation (end)

\input{qm2pi.process.calculi} 

% section concurrent_process_calculi_and_spatial_logics_ (end)
    
%\documentclass[12pt]{llncs}
%\documentclass{jktr}

\usepackage[pdftex]{hyperref}                   
\usepackage {listings}
\usepackage {mathpartir}
\usepackage{bcprules}
%\usepackage{listings}
                       
\usepackage{graphicx} 
%\usepackage[margins=2.5cm,nohead,nofoot]{geometry}
%\usepackage{geometry}
\usepackage{amsfonts}
\usepackage{amstext}
\usepackage{latexsym}
\usepackage{amssymb}
\usepackage{color}


%\include{myPreamble}
\include{qm2pi.local} 

%\ifpdf
%\usepackage[pdftex]{graphicx}
%\else
%\usepackage{graphicx}
%\fi

 % \ifpdf
%  \usepackage{pdfsync}
%  \if


%\title{Brief Article}
%\author{David F. Snyder}
%\author{L.G. Meredith}

%\address{Dept. of Math., Texas State University--San Marcos, San Marcos, TX 78666}
       
\pagestyle{empty}


\begin{document}

\lstset{language=[Objective]Caml,frame=shadowbox}

\input{qm2pi.front}

% section front matter (end)

\input{qm2pi.intro} 
 
% section introduction (end)

% \input{qm2pi.knotations} 

% section notation (end)

\input{qm2pi.process.calculi} 

% section concurrent_process_calculi_and_spatial_logics_ (end)
    
%\input{qm2pi.knots2pi} 

%\input{qm2pi.trefoil} 

%\input{qm2pi.mainthm} 

% subsection basic_interpretation (end)

%\input{qm2pi.rho.presentation} 
\subsection{The syntax and semantics of the notation system}\label{sub:the_syntax_and_semantics_of_the_notation_system} % (fold)

We now summarize a technical presentation of the calculus that
embodies our theory of dynamics. The typical presentation of such a
calculus follows the style of giving generators and relations on
them. The grammar, below, describing term constructors, freely
generates the set of processes, $\Proc$. This set is then quotiented
by a relation known as structural congruence and it is over this set
that the notion of dynamics is expressed. This presentation is
essentially that of \cite{MeredithR05} with the addition of
polyadicity and summation. For readability we have relegated some of
the technical subtleties to an appendix.

\subsubsection{Process grammar}\label{subsub:process_grammar}

\begin{mathpar}
  \inferrule* [lab=synchronization] {} {{M} \bc \pzero \;|\; x?F \;|\; x!C }
  \and
  \inferrule* [lab=abstraction] {} {{F} \bc (x)P}
  \and
  \inferrule* [lab=concretion] {} {{C} \bc \langle Q \rangle}
  \and
  \inferrule* [lab=process] {} {{P,Q} \bc M \;| \;P|Q \;|\; @{x}}
  \and
  \inferrule* [lab=name] {} {{x} \bc \quotep{P}}
\end{mathpar} 

Note that $\vec{x}$ (resp. $\vec{P}$) denotes a vector of names
(resp. processes) of length $|\vec{x}|$ (resp. $|\vec{P}|$). We adopt
the following useful abbreviations.

\begin{mathpar}
   x?(\vec{y}).P := x.(\vec{y})P \and  x\clift{\vec{P}} := x.\clift{\vec{P}}
   \and x!(y) := \lift{x}{\dropn{y}}
   \and \Pi_{i=0}^{n-1}P_i := P_0 | \ldots | P_{n-1}
\end{mathpar}

\subsubsection{Structural congruence}

\paragraph{Free and bound names and alpha-equivalence.} At the
core of structural equivalence is alpha-equivalence which identifies
process that are the same up to a change of variable. Formally, we
recognize the distinction between free and bound names. The free names
of a process, $\freenames{P}$, may be calculated recursively as
follows:

\begin{mathpar}
\freenames{\pzero} := \emptyset
  \and \\
  \freenames{x?(y).P} := \{ x \} \cup (\freenames{P} \setminus \{ y \})
  \and 
  \freenames{x!\langle P \rangle} := \{ x \} \cup \{ P \} 
  \and \\
  \freenames{P|Q} := \freenames{P} \cup \freenames{Q}
  \and \\
  \freenames{@{x}} := \{ x \}
\end{mathpar}

$\pi$
$\quotep{\pi}$

$\freenames{-} : \pi \to \mathcal{P}(\quotep{\pi})$

\begin{eqnarray*}
  \freenames{\pzero} & := & \emptyset \\
  \freenames{x?(y).P} & := & \{ x \} \cup (\freenames{P} \setminus \{ y \}) \\
  \freenames{x!\langle P \rangle} & := & \{ x \} \cup \{ P \} \\
  \freenames{P|Q} & := & \freenames{P} \cup \freenames{Q} \\
  \freenames{\dropn{x}} & := & \{ x \}
\end{eqnarray*}

The bound names of a process, $\boundnames{P}$, are those names occurring in $P$
that are not free. For example, in $x?(y).0$, the name $x$ is free, while $y$ is bound.

\begin{mathpar}
  \inferrule* [lab=monoidal-laws] {} { P|Q \equiv Q|P \and P|0 \equiv P \and P|(Q|R) \equiv (P|Q)|R }
\end{mathpar}

\begin{mathpar}
  \inferrule* [lab=alpha-equivalence] {} { (x)P \equiv (y)P\{y/x\} \and y \not\in \freenames{P} }
\end{mathpar}

\begin{definition}
Then two processes, $P,Q$, are alpha-equivalent if $P = Q\{\vec{y}/\vec{x}\}$ for
some $\vec{x} \in \boundnames{Q},\vec{y} \in \boundnames{P}$, where $Q\{\vec{y}/\vec{x}\}$
denotes the capture-avoiding substitution of $\vec{y}$ for $\vec{x}$ in $Q$.
\end{definition}

\begin{definition}
  The {\em structural congruence} \cite{SangiorgiWalker} , $\equiv$,
  between processes is the least congruence containing
  alpha-equivalence, satisfying the abelian monoid laws
  (associativity, commutativity and $\pzero$ as identity) for parallel
  composition $|$ and for summation $+$.
\end{definition}

\subsection{Name equivalence}

We take name equivalence, written $\nameeq$, to be the smallest
equivalence relation generated by the following rules.

\begin{mathpar}
\inferrule*[lab=Quote-drop]
{ }
{ \quotep{@{x}} \nameeq x }

\inferrule*[lab=Struct-equiv]
{ P \scong Q }
{ \quotep{P} \nameeq \quotep{Q} }
\end{mathpar}

The astute reader will have noticed that the mutual recursion of names
and processes imposes a mutual recursion on alpha-equivalence and
structural equivalence via name-equivalence. Fortunately, all of this
works out pleasantly and we may calculate in the natural way, free of
concern. The reader interested in the details is referred to the
appendix \ref{appendix:rho_details}.

\subsection{Substitution}

We use $\Proc$ for the set of processes, $\QProc$ for the set of
names, and $\id{\{}\vec{y} / \vec{x} \id{\}}$ to denote partial maps,
$s : \QProc \rightarrow \QProc$. A map, $s$ lifts, uniquely, to a map
on process terms, $\widehat{s} : \Proc \rightarrow \Proc$ by the
following equations.

\begin{mathpar}
  (0) \psubstp{Q}{P} := 0 \\
  (R \juxtap S) \psubstp{Q}{P}
  :=    
  (R)\psubstp{Q}{P} \juxtap (S) \psubstp{Q}{P} \\
  (x?(y).R) \psubstp{Q}{P}    
  :=    
  (x)\substp{Q}{P} (z)\concat( (R \psubstn{z}{y}) \psubstp{Q}{P} ) \\
  (\lift{x}{R}) \psubstp{Q}{P}  
  :=
  \lift{(x)\substp{Q}{P}}{ R \psubstp{Q}{P} } \\
%   (\dropn{x})  \psubstp{Q}{P}       
%   := 
%   \left\{ 
%     \begin{array}{ccc} 
%       \dropn{\quotep{Q}} & & x \nameeq \quotep{P} \\
%       \dropn{x} & & otherwise \\
%     \end{array}
%   \right. 
  (\dropn{x})  \psubstp{Q}{P}       
  := 
  \left\{ 
    \begin{array}{ccc} 
      Q & & x \nameeq \quotep{P} \\
      \dropn{x} & & otherwise \\
    \end{array}
  \right.
\end{mathpar}
 

where

\begin{eqnarray}
  (x)\id{\{} \lpquote Q \rpquote / \lpquote P \rpquote \id{\}}            = 
  \left\{ 
    \begin{array}{ccc}
      \lpquote Q \rpquote & & x \nameeq \lpquote P \rpquote \\
      x & & otherwise \\
    \end{array}
  \right. \nonumber
\end{eqnarray}

and $z$ is chosen distinct from $\quotep{P}$, $\quotep{Q}$, the free
names in $Q$, and all the names in $R$. Our $\alpha$-equivalence will
be built in the standard way from this substitution.

\begin{remark}\label{rem:no_self_referential_names}
  One consequence of these definitions is that $\forall P. \quotep{P}
  \not\in \freenames{P}$.
\end{remark}

\subsection{ Dynamic quote: an example }

Anticipating something of what's to come, consider applying the
substitution, $\widehat{\id{\{}u / z \id{\}}}$, to the following pair
of processes, $\lift{w}{y!(z)}$ and $w[ \lpquote y!(z) \rpquote ]$.

\begin{eqnarray}
	\lift{w}{y!(z)}\widehat{\id{\{}u / z \id{\}}}
		& = &
		\lift{w}{y!(u)} \nonumber\\
	w[ \lpquote y!(z) \rpquote ] \widehat{ \id{\{}u / z \id{\}} }
		& = &
		w[ \lpquote y!(z) \rpquote ] \nonumber
\end{eqnarray}

Because the body of the process between quotes is impervious to
substitution, we get radically different answers. In fact, by
examining the first process in an input context,
e.g. $x?(z).\lift{w}{y!(z)}$, we see that the process under the lift
operator may be shaped by prefixed inputs binding a name inside it. In
this sense, the lift operator will be seen as a way to dynamically
construct processes before reifying them as names.

Finally equipped with these standard features we can present the
dynamics of the calculus.

\subsubsection{Operational semantics} 

Finally, we introduce the computational dynamics. What marks these
algebras as distinct from other more traditionally studied algebraic
structures, e.g. vector spaces or polynomial rings, is the manner in
which dynamics is captured. In traditional structures, dynamics is typically
expressed through morphisms between such structures, as in linear maps
between vector spaces or morphisms between rings. In algebras
associated with the semantics of computation, the dynamics is
expressed as part of the algebraic structure itself, through a
reduction reduction relation typically denoted by $\red$. Below, we
give a recursive presentation of this relation for the calculus used
in the encoding.

$\red \subseteq \pi \times \pi$
$\red : \pi \to \mathcal{P}(\pi)$

\begin{mathpar}
  \inferrule* [lab=Comm] { \textsf{match}( x_{src}, x_{trgt} ) } { x_{trgt}?(y)P \; | \; x_{src}!\langle {Q} \rangle \red P\{\quotep{Q}/y}\} }
  \and \\
  \inferrule* [lab=Par] {{P} \red {P}'} {{{P} | {Q}} \red {{P}' | {Q}}}
  \and
  \inferrule* [lab=Equiv]{{{P} \scong {P}'} \andalso {{P}' \red {Q}'} \andalso {{Q}' \scong {Q}}}{{P} \red {Q}}
\end{mathpar}

\begin{eqnarray*}
  match_{\equiv} (\quotep{P},\quotep{Q}) & := & P \equiv Q \\
  match_{\dagger}(\quotep{P},\quotep{Q}) & := & \forall R. P|Q \red^{*} R => R \red^{*} 0 \\
  match_{K}(\quotep{P},\quotep{Q}) & := & K \mbox{ for some context } K
\end{eqnarray*}

$u?(x)P | u!\langle Q \rangle \red P\{\quotep{Q}/x\}$

%We write $\wred$ for $\red^*$, and $P\red$ if $\exists Q $ such that $ P \red Q$.
We write $P\red$ if $\exists Q $ such that $ P \red Q$ and $P\not\red$, otherwise.

\section{Replication}

As mentioned before, it is known that replication (and hence
recursion) can be implemented in a higher-order process algebra
\cite{SangiorgiWalker}. As our first example of calculation with the
machinery thus far presented we give the construction explicitly in
the {\rhoc}.

\begin{eqnarray}
	D_{x} & := & \prefix{x}{y}{(\binpar{\outputp{x}{y}}{@{y}})} \nonumber\\
	\bangp_{x}{P} & := & \binpar{{x}!\langle{\binpar{D_{x}}{P}}\rangle}{D_{x}} \nonumber
\end{eqnarray}

\begin{eqnarray}
	\bangp_{x}{P} & & \nonumber\\
	=
	& {x}!\langle{(\prefix{x}{y}{(\outputp{x}{y} | @{y})) | P}}\rangle 
	      | \prefix{x}{y}{(\outputp{x}{y} | @{y})} & \nonumber\\
	\red
	& (\outputp{x}{y} | @{y})\substn{\quotep{(\prefix{x}{y}{(@{y} | \outputp{x}{y})) | P}}}{y} & \nonumber\\
	=
	& \outputp{x}{\quotep{(\prefix{x}{y}{(\outputp{x}{y} | @{y})) | P}}}
	  | {(\prefix{x}{y}{(\outputp{x}{y} | @{y})) | P}} & \nonumber\\
	\red
	& \ldots & \nonumber\\
	\red^*
	& P | P | \ldots & \nonumber
\end{eqnarray}

Of course, this encoding, as an implementation, runs away, unfolding
$\bangp{P}$ eagerly. A lazier and more implementable replication
operator, restricted to input-guarded processes, may be obtained as follows.

\begin{eqnarray}
\bangp{\prefix{u}{v}{P}} 
	:= 
	\binpar{\lift{x}{\prefix{u}{v}{(\binpar{D(x)}{P})}}}{D(x)} \nonumber
\end{eqnarray}

\begin{remark}
  Note that the lazier definition still does not deal with summation
  or mixed summation (i.e. sums over input and output). The reader is
  invited to construct definitions of replication that deal with these
  features. 

  Further, the definitions are parameterized in a name, $x$. Can you,
  gentle reader, make a definition that eliminates this parameter and
  guarantees no accidental interaction between the replication
  machinery and the process being replicated -- i.e. no accidental
  sharing of names used by the process to get its work done and the
  name(s) used by the replication to effect copying. This latter
  revision of the definition of replication is crucial to obtaining
  the expected identity $!!P \sim !P$.
\end{remark}

\begin{remark}\label{rem:paradoxical_combinator}
  The reader familiar with the lambda calculus will have noticed the
  similarity between $D$ and the paradoxical combinator.

  [Ed. note: the existence of this seems to suggest we have to be more
  restrictive on the set of processes and names we admit if we are to
  support no-cloning.]
\end{remark}

\subsubsection{Bisimulation}

The computational dynamics gives rise to another kind of equivalence,
the equivalence of computational behavior. As previously mentioned
this is typically captured \emph{via} some form of bisimulation.

% The notion we use in this paper is weak barbed bisimulation
% \cite{milner91polyadicpi}.

The notion we use in this paper is derived from weak barbed
bisimulation \cite{milner91polyadicpi}. 

\begin{definition}
An \emph{observation relation}, $\downarrow_{\mathcal N}$, over a set
of names, $\mathcal N$, is the smallest relation satisfying the rules
below.

\infrule[Out-barb]{y \in {\mathcal N}, \; x \nameeq y}
		  {\outputp{x}{v} \downarrow_{\mathcal N} x}
\infrule[Par-barb]{\mbox{$P\downarrow_{\mathcal N} x$ or $Q\downarrow_{\mathcal N} x$}}
		  {\binpar{P}{Q} \downarrow_{\mathcal N} x}

We write $P \Downarrow_{\mathcal N} x$ if there is $Q$ such that 
$P \wred Q$ and $Q \downarrow_{\mathcal N} x$.
\end{definition}

\begin{definition}
%\label{def.bbisim}
An  ${\mathcal N}$-\emph{barbed bisimulation} over a set of names, ${\mathcal N}$, is a symmetric binary relation 
${\mathcal S}_{\mathcal N}$ between agents such that $P\rel{S}_{\mathcal N}Q$ implies:
\begin{enumerate}
\item If $P \red P'$ then $Q \wred Q'$ and $P'\rel{S}_{\mathcal N} Q'$.
\item If $P\downarrow_{\mathcal N} x$, then $Q\Downarrow_{\mathcal N} x$.
\end{enumerate}
$P$ is ${\mathcal N}$-barbed bisimilar to $Q$, written
$P \wbbisim_{\mathcal N} Q$, if $P \rel{S}_{\mathcal N} Q$ for some ${\mathcal N}$-barbed bisimulation ${\mathcal S}_{\mathcal N}$.
\end{definition}

$\mathcal{R} \subseteq \pi \times \pi$

$P \mathcal{R} Q => \forall P'. P \red P' \Rightarrow \exists Q'. Q \red Q', P' \mathcal{R} Q'$

$P \vdash x \Rightarrow Q \vdash x$

\begin{mathpar}
  \inferrule*[lab=Out-barb]{x \nameeq y}{{y}!\langle{Q}\rangle \vdash x}
  \and
  \inferrule*[lab=Par-barb]{\mbox{$P\vdash x$ or $Q\vdash x$}}{\binpar{P}{Q} \vdash x}
\end{mathpar}

\subsubsection{Contexts}

One of the principle advantages of computational calculi like the
$\pi$-calculus is a well-defined notion of context,
contextual-equivalence and a correlation between
contextual-equivalence and notions of bisimulation. The notion of
context allows the decomposition of a process into (sub-)process and
its syntactic environment, its context. Thus, a context may be
thought of as a process with a ``hole'' (written $\Box$) in it. The
application of a context $M$ to a process $P$, written $M[P]$, is
tantamount to filling the hole in $M$ with $P$. In this paper we do
not need the full weight of this theory, but do make use of the notion
of context in the proof the main theorem. 

\begin{mathpar}
  \inferrule* [lab=summation] {} {{M_{M},M_{N}} \bc \Box \;|\; x.M_{A} \;|\; M_{M}+M_{N}}
  \and
  \inferrule* [lab=agent] {} {{M_{A}} \bc (\vec{x})M_{P} \;| \; \clift{P_0,\ldots,M_{P},\ldots,P_N}}
  \and \\
  \inferrule* [lab=process] {} {{M_{P}} \bc M_{N} \;| \;P|M_{P} }
\end{mathpar} 

\begin{mathpar}
  \inferrule* [lab=sychronization] {} {M_{N} \bc \Box \;|\; x?M_{F} \;|\; x!M_{C}}
  \and
  \inferrule* [lab=abstraction] {} {{M_{F}} \bc (x)M_{P} }
  \and
  \inferrule* [lab=concretion] {} {{M_{C}} \bc \langle M_{P} \rangle }
  \and \\
  \inferrule* [lab=process] {} {{M_{P}} \bc M_{N} \;| \;P|M_{P} }
\end{mathpar}

\begin{definition}[contextual application] Given a context $M$, and
  process $P$, we define the \emph{contextual application}, $M[P] :=
  M\{P/\Box\}$. That is, the contextual application of M to P is the
  substitution of $P$ for $\Box$ in $M$.
\end{definition}

$\meaningof{-} : L \to \mathcal{P}(\pi)$

\begin{mathpar}
  \inferrule* [lab=collection] {} {\meaningof{true} = \pi, \and \meaningof{~E} = \pi \setminus \meaningof{E}, \and \meaningof{E_{1} \& E_{2}} = \meaningof{E_{1}} \cap \meaningof{E_{2}}}
\end{mathpar}

\begin{mathpar}
  \inferrule* [lab=structure] {} {\meaningof{0} = \{ P \in \pi | P \equiv 0 \}, \and \\ \meaningof{E_1 | E_2} = \{ P \in \pi | P \equiv P_{1} | P_{2}, P_{1} \in \meaningof{E_{1}}, P_{2} \in \meaningof{E_2}\} }
\end{mathpar}

\begin{mathpar}
 \inferrule* [lab=behavior] {} {\meaningof{\langle a?b \rangle E} = \{ P \in \pi | P \equiv Q | u?(y)P', \\ \and \\\\ \and \\ \;\;\; u \in \meaningof{a}, \forall z.P'\{z/y\} \in \meaningof{E\{z/b\}}\}, \and \\ \meaningof{a!E} = \{ P \in \pi | P \equiv Q | x!\langle P' \rangle, x \in \meaningof{a} P' \in \meaningof{E}\} }
\end{mathpar}

\begin{mathpar}
 \inferrule* [lab=nominal] {} {\meaningof{\quotep{E}} = \{ \quotep{P} \in \quotep{\pi} | P \in \meaningof{E} \}, \and \meaningof{\quotep{P}} = \{ \quotep{Q} \in \quotep{\pi} | P \equiv Q \} \and \\ \meaningof{@\quotep{E}} = \{ P \in \pi | P \equiv @x, x \in \meaningof{E} \}}
\end{mathpar}

\begin{eqnarray*}
  \\
  \meaningof{-} : TS \to ST
\end{eqnarray*}

\begin{eqnarray*}
  \\
  L : TS \to ST
\end{eqnarray*}

\begin{eqnarray*}
  \\
  P \models E \iff P \in \meaningof{E}
\end{eqnarray*}

\begin{eqnarray*}
  P \approx_{L} Q \iff \forall E \in L. P \models E \iff Q \models E
\end{eqnarray*}

\begin{eqnarray*}
  P \approx_{K} Q
\end{eqnarray*}

\begin{eqnarray*}
  P \approx Q
\end{eqnarray*}

$\approx_{K} = \approx = \approx_{L}$

\subsubsection{Contextual duality}

Note that contexts extend the quotation operation to a family of
operations from processes to names. Given a context, $M$, we can
define a \emph{nominal context}, $\quotep{M}$ by $\quotep{M}[P] :=
\quotep{M[P]}$. To foreshadow what is to come we observe that these
operations enjoy a duality with processes very much like the duality
between vectors and maps from vectors to scalars.

Further, because the calculus is essentially higher-order, we have a
correspondence between contexts and processes. More specifically,
given a name $x$ and a context $M$ we can construct $M^{*}_{x}$ such
that 

\begin{mathpar}
  M^{*}_{x} | \lift{x}{P} \red M[P]
\end{mathpar}

namely,

\begin{mathpar}
  M^{*}_{x} := x?(u).M[\dropn{u}]
\end{mathpar}

The dependence of $M^{*}_{x}$ on a name makes it an abstraction, 

\begin{mathpar}
  M^{*} := (x)x?(u).M[\dropn{u}]
\end{mathpar}

\subsection{Additional notation}

It will sometimes be convenient to denote the process a name
quotes. We already have the notation $x = \quotep{P}$, but it will be
convenient to introduce an alternate notation, $\procn{x}$, when we
want to emphasize the connection to the use of the name. Note that, by
virtue of name equivalence, $\quotep{\procn{x}} \nameeq x$; so, the
notation is consistent with previous definitions.

Further, because names have structure it is possible to effect
substitutions on the basis of that structure. This means we need to
upgrade our notation for substitutions, which we accomplish by
adapting comprehension notation. Thus,

\begin{mathpar}
  P\{ y / x : x \in S \}
\end{mathpar}

is interpreted to mean the process derived from P by replacing (in a
capture-avoiding manner) each occurrence of $x$ in $S$ by $y$. For example,

\begin{mathpar}
  P\{ \quotep{\procn{x}|\procn{x}} / x : x \in \freenames{P} \}
\end{mathpar}

will replace each (occurrence) of a free name $x$ in $P$ by
$\quotep{\procn{x}|\procn{x}}$.

Also, we will avail ourselves of the notation $x^{L}$ and $x^{R}$ to
denote injections of a name into disjoint copies of the name
space. There are numerous ways to accomplish this. One example can be
found in \cite{MeredithR05}. This notation overloads to vectors of
names: $\vec{x}^{\pi} := (x_{i}^{\pi} \; : \; 0 \leq i < |\vec{x}| )$ where $\pi \in \{L,R\}$.

We also use $P^{\Box} := P|\Box$.

In \cite{MeredithR05} an interpretation of the new operator is
given. It turns out that there are several possible interpretations
all enjoying the requisite algebraic properties of the operator (see
\cite{milner91polyadicpi}). We will therefore make liberal use of
$(\nu\; \vec{x})P$.

% subsection the_syntax_and_semantics_of_the_notation_system (end)   

\input{qm2pi.qmops} 

\input{qm2pi.sterngerlach} 

\input{qm2pi.metric} 

% section concurrent_process_calculi (end)

%\input{qm2pi.proofsketch}

% section proof sketch (end)

%\input{qm2pi.slviaknots} 

% section spatial logic via knots (end)

\input{qm2pi.conclusion}

% section conclusion (end)

%\input{qm2pi.dtcodes} 

% section wiring algorithm (end)

\input{qm2pi.ack} 

% section acknowledgments (end)

\newpage


\bibliographystyle{plain}   
\bibliography{../../biblios/main.bib}

\input{qm2pi.rhodetails}

\end{document}

 

%\documentclass[12pt]{llncs}
%\documentclass{jktr}

\usepackage[pdftex]{hyperref}                   
\usepackage {listings}
\usepackage {mathpartir}
\usepackage{bcprules}
%\usepackage{listings}
                       
\usepackage{graphicx} 
%\usepackage[margins=2.5cm,nohead,nofoot]{geometry}
%\usepackage{geometry}
\usepackage{amsfonts}
\usepackage{amstext}
\usepackage{latexsym}
\usepackage{amssymb}
\usepackage{color}


%\include{myPreamble}
\include{qm2pi.local} 

%\ifpdf
%\usepackage[pdftex]{graphicx}
%\else
%\usepackage{graphicx}
%\fi

 % \ifpdf
%  \usepackage{pdfsync}
%  \if


%\title{Brief Article}
%\author{David F. Snyder}
%\author{L.G. Meredith}

%\address{Dept. of Math., Texas State University--San Marcos, San Marcos, TX 78666}
       
\pagestyle{empty}


\begin{document}

\lstset{language=[Objective]Caml,frame=shadowbox}

\input{qm2pi.front}

% section front matter (end)

\input{qm2pi.intro} 
 
% section introduction (end)

% \input{qm2pi.knotations} 

% section notation (end)

\input{qm2pi.process.calculi} 

% section concurrent_process_calculi_and_spatial_logics_ (end)
    
%\input{qm2pi.knots2pi} 

%\input{qm2pi.trefoil} 

%\input{qm2pi.mainthm} 

% subsection basic_interpretation (end)

%\input{qm2pi.rho.presentation} 
\subsection{The syntax and semantics of the notation system}\label{sub:the_syntax_and_semantics_of_the_notation_system} % (fold)

We now summarize a technical presentation of the calculus that
embodies our theory of dynamics. The typical presentation of such a
calculus follows the style of giving generators and relations on
them. The grammar, below, describing term constructors, freely
generates the set of processes, $\Proc$. This set is then quotiented
by a relation known as structural congruence and it is over this set
that the notion of dynamics is expressed. This presentation is
essentially that of \cite{MeredithR05} with the addition of
polyadicity and summation. For readability we have relegated some of
the technical subtleties to an appendix.

\subsubsection{Process grammar}\label{subsub:process_grammar}

\begin{mathpar}
  \inferrule* [lab=synchronization] {} {{M} \bc \pzero \;|\; x?F \;|\; x!C }
  \and
  \inferrule* [lab=abstraction] {} {{F} \bc (x)P}
  \and
  \inferrule* [lab=concretion] {} {{C} \bc \langle Q \rangle}
  \and
  \inferrule* [lab=process] {} {{P,Q} \bc M \;| \;P|Q \;|\; @{x}}
  \and
  \inferrule* [lab=name] {} {{x} \bc \quotep{P}}
\end{mathpar} 

Note that $\vec{x}$ (resp. $\vec{P}$) denotes a vector of names
(resp. processes) of length $|\vec{x}|$ (resp. $|\vec{P}|$). We adopt
the following useful abbreviations.

\begin{mathpar}
   x?(\vec{y}).P := x.(\vec{y})P \and  x\clift{\vec{P}} := x.\clift{\vec{P}}
   \and x!(y) := \lift{x}{\dropn{y}}
   \and \Pi_{i=0}^{n-1}P_i := P_0 | \ldots | P_{n-1}
\end{mathpar}

\subsubsection{Structural congruence}

\paragraph{Free and bound names and alpha-equivalence.} At the
core of structural equivalence is alpha-equivalence which identifies
process that are the same up to a change of variable. Formally, we
recognize the distinction between free and bound names. The free names
of a process, $\freenames{P}$, may be calculated recursively as
follows:

\begin{mathpar}
\freenames{\pzero} := \emptyset
  \and \\
  \freenames{x?(y).P} := \{ x \} \cup (\freenames{P} \setminus \{ y \})
  \and 
  \freenames{x!\langle P \rangle} := \{ x \} \cup \{ P \} 
  \and \\
  \freenames{P|Q} := \freenames{P} \cup \freenames{Q}
  \and \\
  \freenames{@{x}} := \{ x \}
\end{mathpar}

$\pi$
$\quotep{\pi}$

$\freenames{-} : \pi \to \mathcal{P}(\quotep{\pi})$

\begin{eqnarray*}
  \freenames{\pzero} & := & \emptyset \\
  \freenames{x?(y).P} & := & \{ x \} \cup (\freenames{P} \setminus \{ y \}) \\
  \freenames{x!\langle P \rangle} & := & \{ x \} \cup \{ P \} \\
  \freenames{P|Q} & := & \freenames{P} \cup \freenames{Q} \\
  \freenames{\dropn{x}} & := & \{ x \}
\end{eqnarray*}

The bound names of a process, $\boundnames{P}$, are those names occurring in $P$
that are not free. For example, in $x?(y).0$, the name $x$ is free, while $y$ is bound.

\begin{mathpar}
  \inferrule* [lab=monoidal-laws] {} { P|Q \equiv Q|P \and P|0 \equiv P \and P|(Q|R) \equiv (P|Q)|R }
\end{mathpar}

\begin{mathpar}
  \inferrule* [lab=alpha-equivalence] {} { (x)P \equiv (y)P\{y/x\} \and y \not\in \freenames{P} }
\end{mathpar}

\begin{definition}
Then two processes, $P,Q$, are alpha-equivalent if $P = Q\{\vec{y}/\vec{x}\}$ for
some $\vec{x} \in \boundnames{Q},\vec{y} \in \boundnames{P}$, where $Q\{\vec{y}/\vec{x}\}$
denotes the capture-avoiding substitution of $\vec{y}$ for $\vec{x}$ in $Q$.
\end{definition}

\begin{definition}
  The {\em structural congruence} \cite{SangiorgiWalker} , $\equiv$,
  between processes is the least congruence containing
  alpha-equivalence, satisfying the abelian monoid laws
  (associativity, commutativity and $\pzero$ as identity) for parallel
  composition $|$ and for summation $+$.
\end{definition}

\subsection{Name equivalence}

We take name equivalence, written $\nameeq$, to be the smallest
equivalence relation generated by the following rules.

\begin{mathpar}
\inferrule*[lab=Quote-drop]
{ }
{ \quotep{@{x}} \nameeq x }

\inferrule*[lab=Struct-equiv]
{ P \scong Q }
{ \quotep{P} \nameeq \quotep{Q} }
\end{mathpar}

The astute reader will have noticed that the mutual recursion of names
and processes imposes a mutual recursion on alpha-equivalence and
structural equivalence via name-equivalence. Fortunately, all of this
works out pleasantly and we may calculate in the natural way, free of
concern. The reader interested in the details is referred to the
appendix \ref{appendix:rho_details}.

\subsection{Substitution}

We use $\Proc$ for the set of processes, $\QProc$ for the set of
names, and $\id{\{}\vec{y} / \vec{x} \id{\}}$ to denote partial maps,
$s : \QProc \rightarrow \QProc$. A map, $s$ lifts, uniquely, to a map
on process terms, $\widehat{s} : \Proc \rightarrow \Proc$ by the
following equations.

\begin{mathpar}
  (0) \psubstp{Q}{P} := 0 \\
  (R \juxtap S) \psubstp{Q}{P}
  :=    
  (R)\psubstp{Q}{P} \juxtap (S) \psubstp{Q}{P} \\
  (x?(y).R) \psubstp{Q}{P}    
  :=    
  (x)\substp{Q}{P} (z)\concat( (R \psubstn{z}{y}) \psubstp{Q}{P} ) \\
  (\lift{x}{R}) \psubstp{Q}{P}  
  :=
  \lift{(x)\substp{Q}{P}}{ R \psubstp{Q}{P} } \\
%   (\dropn{x})  \psubstp{Q}{P}       
%   := 
%   \left\{ 
%     \begin{array}{ccc} 
%       \dropn{\quotep{Q}} & & x \nameeq \quotep{P} \\
%       \dropn{x} & & otherwise \\
%     \end{array}
%   \right. 
  (\dropn{x})  \psubstp{Q}{P}       
  := 
  \left\{ 
    \begin{array}{ccc} 
      Q & & x \nameeq \quotep{P} \\
      \dropn{x} & & otherwise \\
    \end{array}
  \right.
\end{mathpar}
 

where

\begin{eqnarray}
  (x)\id{\{} \lpquote Q \rpquote / \lpquote P \rpquote \id{\}}            = 
  \left\{ 
    \begin{array}{ccc}
      \lpquote Q \rpquote & & x \nameeq \lpquote P \rpquote \\
      x & & otherwise \\
    \end{array}
  \right. \nonumber
\end{eqnarray}

and $z$ is chosen distinct from $\quotep{P}$, $\quotep{Q}$, the free
names in $Q$, and all the names in $R$. Our $\alpha$-equivalence will
be built in the standard way from this substitution.

\begin{remark}\label{rem:no_self_referential_names}
  One consequence of these definitions is that $\forall P. \quotep{P}
  \not\in \freenames{P}$.
\end{remark}

\subsection{ Dynamic quote: an example }

Anticipating something of what's to come, consider applying the
substitution, $\widehat{\id{\{}u / z \id{\}}}$, to the following pair
of processes, $\lift{w}{y!(z)}$ and $w[ \lpquote y!(z) \rpquote ]$.

\begin{eqnarray}
	\lift{w}{y!(z)}\widehat{\id{\{}u / z \id{\}}}
		& = &
		\lift{w}{y!(u)} \nonumber\\
	w[ \lpquote y!(z) \rpquote ] \widehat{ \id{\{}u / z \id{\}} }
		& = &
		w[ \lpquote y!(z) \rpquote ] \nonumber
\end{eqnarray}

Because the body of the process between quotes is impervious to
substitution, we get radically different answers. In fact, by
examining the first process in an input context,
e.g. $x?(z).\lift{w}{y!(z)}$, we see that the process under the lift
operator may be shaped by prefixed inputs binding a name inside it. In
this sense, the lift operator will be seen as a way to dynamically
construct processes before reifying them as names.

Finally equipped with these standard features we can present the
dynamics of the calculus.

\subsubsection{Operational semantics} 

Finally, we introduce the computational dynamics. What marks these
algebras as distinct from other more traditionally studied algebraic
structures, e.g. vector spaces or polynomial rings, is the manner in
which dynamics is captured. In traditional structures, dynamics is typically
expressed through morphisms between such structures, as in linear maps
between vector spaces or morphisms between rings. In algebras
associated with the semantics of computation, the dynamics is
expressed as part of the algebraic structure itself, through a
reduction reduction relation typically denoted by $\red$. Below, we
give a recursive presentation of this relation for the calculus used
in the encoding.

$\red \subseteq \pi \times \pi$
$\red : \pi \to \mathcal{P}(\pi)$

\begin{mathpar}
  \inferrule* [lab=Comm] { \textsf{match}( x_{src}, x_{trgt} ) } { x_{trgt}?(y)P \; | \; x_{src}!\langle {Q} \rangle \red P\{\quotep{Q}/y}\} }
  \and \\
  \inferrule* [lab=Par] {{P} \red {P}'} {{{P} | {Q}} \red {{P}' | {Q}}}
  \and
  \inferrule* [lab=Equiv]{{{P} \scong {P}'} \andalso {{P}' \red {Q}'} \andalso {{Q}' \scong {Q}}}{{P} \red {Q}}
\end{mathpar}

\begin{eqnarray*}
  match_{\equiv} (\quotep{P},\quotep{Q}) & := & P \equiv Q \\
  match_{\dagger}(\quotep{P},\quotep{Q}) & := & \forall R. P|Q \red^{*} R => R \red^{*} 0 \\
  match_{K}(\quotep{P},\quotep{Q}) & := & K \mbox{ for some context } K
\end{eqnarray*}

$u?(x)P | u!\langle Q \rangle \red P\{\quotep{Q}/x\}$

%We write $\wred$ for $\red^*$, and $P\red$ if $\exists Q $ such that $ P \red Q$.
We write $P\red$ if $\exists Q $ such that $ P \red Q$ and $P\not\red$, otherwise.

\section{Replication}

As mentioned before, it is known that replication (and hence
recursion) can be implemented in a higher-order process algebra
\cite{SangiorgiWalker}. As our first example of calculation with the
machinery thus far presented we give the construction explicitly in
the {\rhoc}.

\begin{eqnarray}
	D_{x} & := & \prefix{x}{y}{(\binpar{\outputp{x}{y}}{@{y}})} \nonumber\\
	\bangp_{x}{P} & := & \binpar{{x}!\langle{\binpar{D_{x}}{P}}\rangle}{D_{x}} \nonumber
\end{eqnarray}

\begin{eqnarray}
	\bangp_{x}{P} & & \nonumber\\
	=
	& {x}!\langle{(\prefix{x}{y}{(\outputp{x}{y} | @{y})) | P}}\rangle 
	      | \prefix{x}{y}{(\outputp{x}{y} | @{y})} & \nonumber\\
	\red
	& (\outputp{x}{y} | @{y})\substn{\quotep{(\prefix{x}{y}{(@{y} | \outputp{x}{y})) | P}}}{y} & \nonumber\\
	=
	& \outputp{x}{\quotep{(\prefix{x}{y}{(\outputp{x}{y} | @{y})) | P}}}
	  | {(\prefix{x}{y}{(\outputp{x}{y} | @{y})) | P}} & \nonumber\\
	\red
	& \ldots & \nonumber\\
	\red^*
	& P | P | \ldots & \nonumber
\end{eqnarray}

Of course, this encoding, as an implementation, runs away, unfolding
$\bangp{P}$ eagerly. A lazier and more implementable replication
operator, restricted to input-guarded processes, may be obtained as follows.

\begin{eqnarray}
\bangp{\prefix{u}{v}{P}} 
	:= 
	\binpar{\lift{x}{\prefix{u}{v}{(\binpar{D(x)}{P})}}}{D(x)} \nonumber
\end{eqnarray}

\begin{remark}
  Note that the lazier definition still does not deal with summation
  or mixed summation (i.e. sums over input and output). The reader is
  invited to construct definitions of replication that deal with these
  features. 

  Further, the definitions are parameterized in a name, $x$. Can you,
  gentle reader, make a definition that eliminates this parameter and
  guarantees no accidental interaction between the replication
  machinery and the process being replicated -- i.e. no accidental
  sharing of names used by the process to get its work done and the
  name(s) used by the replication to effect copying. This latter
  revision of the definition of replication is crucial to obtaining
  the expected identity $!!P \sim !P$.
\end{remark}

\begin{remark}\label{rem:paradoxical_combinator}
  The reader familiar with the lambda calculus will have noticed the
  similarity between $D$ and the paradoxical combinator.

  [Ed. note: the existence of this seems to suggest we have to be more
  restrictive on the set of processes and names we admit if we are to
  support no-cloning.]
\end{remark}

\subsubsection{Bisimulation}

The computational dynamics gives rise to another kind of equivalence,
the equivalence of computational behavior. As previously mentioned
this is typically captured \emph{via} some form of bisimulation.

% The notion we use in this paper is weak barbed bisimulation
% \cite{milner91polyadicpi}.

The notion we use in this paper is derived from weak barbed
bisimulation \cite{milner91polyadicpi}. 

\begin{definition}
An \emph{observation relation}, $\downarrow_{\mathcal N}$, over a set
of names, $\mathcal N$, is the smallest relation satisfying the rules
below.

\infrule[Out-barb]{y \in {\mathcal N}, \; x \nameeq y}
		  {\outputp{x}{v} \downarrow_{\mathcal N} x}
\infrule[Par-barb]{\mbox{$P\downarrow_{\mathcal N} x$ or $Q\downarrow_{\mathcal N} x$}}
		  {\binpar{P}{Q} \downarrow_{\mathcal N} x}

We write $P \Downarrow_{\mathcal N} x$ if there is $Q$ such that 
$P \wred Q$ and $Q \downarrow_{\mathcal N} x$.
\end{definition}

\begin{definition}
%\label{def.bbisim}
An  ${\mathcal N}$-\emph{barbed bisimulation} over a set of names, ${\mathcal N}$, is a symmetric binary relation 
${\mathcal S}_{\mathcal N}$ between agents such that $P\rel{S}_{\mathcal N}Q$ implies:
\begin{enumerate}
\item If $P \red P'$ then $Q \wred Q'$ and $P'\rel{S}_{\mathcal N} Q'$.
\item If $P\downarrow_{\mathcal N} x$, then $Q\Downarrow_{\mathcal N} x$.
\end{enumerate}
$P$ is ${\mathcal N}$-barbed bisimilar to $Q$, written
$P \wbbisim_{\mathcal N} Q$, if $P \rel{S}_{\mathcal N} Q$ for some ${\mathcal N}$-barbed bisimulation ${\mathcal S}_{\mathcal N}$.
\end{definition}

$\mathcal{R} \subseteq \pi \times \pi$

$P \mathcal{R} Q => \forall P'. P \red P' \Rightarrow \exists Q'. Q \red Q', P' \mathcal{R} Q'$

$P \vdash x \Rightarrow Q \vdash x$

\begin{mathpar}
  \inferrule*[lab=Out-barb]{x \nameeq y}{{y}!\langle{Q}\rangle \vdash x}
  \and
  \inferrule*[lab=Par-barb]{\mbox{$P\vdash x$ or $Q\vdash x$}}{\binpar{P}{Q} \vdash x}
\end{mathpar}

\subsubsection{Contexts}

One of the principle advantages of computational calculi like the
$\pi$-calculus is a well-defined notion of context,
contextual-equivalence and a correlation between
contextual-equivalence and notions of bisimulation. The notion of
context allows the decomposition of a process into (sub-)process and
its syntactic environment, its context. Thus, a context may be
thought of as a process with a ``hole'' (written $\Box$) in it. The
application of a context $M$ to a process $P$, written $M[P]$, is
tantamount to filling the hole in $M$ with $P$. In this paper we do
not need the full weight of this theory, but do make use of the notion
of context in the proof the main theorem. 

\begin{mathpar}
  \inferrule* [lab=summation] {} {{M_{M},M_{N}} \bc \Box \;|\; x.M_{A} \;|\; M_{M}+M_{N}}
  \and
  \inferrule* [lab=agent] {} {{M_{A}} \bc (\vec{x})M_{P} \;| \; \clift{P_0,\ldots,M_{P},\ldots,P_N}}
  \and \\
  \inferrule* [lab=process] {} {{M_{P}} \bc M_{N} \;| \;P|M_{P} }
\end{mathpar} 

\begin{mathpar}
  \inferrule* [lab=sychronization] {} {M_{N} \bc \Box \;|\; x?M_{F} \;|\; x!M_{C}}
  \and
  \inferrule* [lab=abstraction] {} {{M_{F}} \bc (x)M_{P} }
  \and
  \inferrule* [lab=concretion] {} {{M_{C}} \bc \langle M_{P} \rangle }
  \and \\
  \inferrule* [lab=process] {} {{M_{P}} \bc M_{N} \;| \;P|M_{P} }
\end{mathpar}

\begin{definition}[contextual application] Given a context $M$, and
  process $P$, we define the \emph{contextual application}, $M[P] :=
  M\{P/\Box\}$. That is, the contextual application of M to P is the
  substitution of $P$ for $\Box$ in $M$.
\end{definition}

$\meaningof{-} : L \to \mathcal{P}(\pi)$

\begin{mathpar}
  \inferrule* [lab=collection] {} {\meaningof{true} = \pi, \and \meaningof{~E} = \pi \setminus \meaningof{E}, \and \meaningof{E_{1} \& E_{2}} = \meaningof{E_{1}} \cap \meaningof{E_{2}}}
\end{mathpar}

\begin{mathpar}
  \inferrule* [lab=structure] {} {\meaningof{0} = \{ P \in \pi | P \equiv 0 \}, \and \\ \meaningof{E_1 | E_2} = \{ P \in \pi | P \equiv P_{1} | P_{2}, P_{1} \in \meaningof{E_{1}}, P_{2} \in \meaningof{E_2}\} }
\end{mathpar}

\begin{mathpar}
 \inferrule* [lab=behavior] {} {\meaningof{\langle a?b \rangle E} = \{ P \in \pi | P \equiv Q | u?(y)P', \\ \and \\\\ \and \\ \;\;\; u \in \meaningof{a}, \forall z.P'\{z/y\} \in \meaningof{E\{z/b\}}\}, \and \\ \meaningof{a!E} = \{ P \in \pi | P \equiv Q | x!\langle P' \rangle, x \in \meaningof{a} P' \in \meaningof{E}\} }
\end{mathpar}

\begin{mathpar}
 \inferrule* [lab=nominal] {} {\meaningof{\quotep{E}} = \{ \quotep{P} \in \quotep{\pi} | P \in \meaningof{E} \}, \and \meaningof{\quotep{P}} = \{ \quotep{Q} \in \quotep{\pi} | P \equiv Q \} \and \\ \meaningof{@\quotep{E}} = \{ P \in \pi | P \equiv @x, x \in \meaningof{E} \}}
\end{mathpar}

\begin{eqnarray*}
  \\
  \meaningof{-} : TS \to ST
\end{eqnarray*}

\begin{eqnarray*}
  \\
  L : TS \to ST
\end{eqnarray*}

\begin{eqnarray*}
  \\
  P \models E \iff P \in \meaningof{E}
\end{eqnarray*}

\begin{eqnarray*}
  P \approx_{L} Q \iff \forall E \in L. P \models E \iff Q \models E
\end{eqnarray*}

\begin{eqnarray*}
  P \approx_{K} Q
\end{eqnarray*}

\begin{eqnarray*}
  P \approx Q
\end{eqnarray*}

$\approx_{K} = \approx = \approx_{L}$

\subsubsection{Contextual duality}

Note that contexts extend the quotation operation to a family of
operations from processes to names. Given a context, $M$, we can
define a \emph{nominal context}, $\quotep{M}$ by $\quotep{M}[P] :=
\quotep{M[P]}$. To foreshadow what is to come we observe that these
operations enjoy a duality with processes very much like the duality
between vectors and maps from vectors to scalars.

Further, because the calculus is essentially higher-order, we have a
correspondence between contexts and processes. More specifically,
given a name $x$ and a context $M$ we can construct $M^{*}_{x}$ such
that 

\begin{mathpar}
  M^{*}_{x} | \lift{x}{P} \red M[P]
\end{mathpar}

namely,

\begin{mathpar}
  M^{*}_{x} := x?(u).M[\dropn{u}]
\end{mathpar}

The dependence of $M^{*}_{x}$ on a name makes it an abstraction, 

\begin{mathpar}
  M^{*} := (x)x?(u).M[\dropn{u}]
\end{mathpar}

\subsection{Additional notation}

It will sometimes be convenient to denote the process a name
quotes. We already have the notation $x = \quotep{P}$, but it will be
convenient to introduce an alternate notation, $\procn{x}$, when we
want to emphasize the connection to the use of the name. Note that, by
virtue of name equivalence, $\quotep{\procn{x}} \nameeq x$; so, the
notation is consistent with previous definitions.

Further, because names have structure it is possible to effect
substitutions on the basis of that structure. This means we need to
upgrade our notation for substitutions, which we accomplish by
adapting comprehension notation. Thus,

\begin{mathpar}
  P\{ y / x : x \in S \}
\end{mathpar}

is interpreted to mean the process derived from P by replacing (in a
capture-avoiding manner) each occurrence of $x$ in $S$ by $y$. For example,

\begin{mathpar}
  P\{ \quotep{\procn{x}|\procn{x}} / x : x \in \freenames{P} \}
\end{mathpar}

will replace each (occurrence) of a free name $x$ in $P$ by
$\quotep{\procn{x}|\procn{x}}$.

Also, we will avail ourselves of the notation $x^{L}$ and $x^{R}$ to
denote injections of a name into disjoint copies of the name
space. There are numerous ways to accomplish this. One example can be
found in \cite{MeredithR05}. This notation overloads to vectors of
names: $\vec{x}^{\pi} := (x_{i}^{\pi} \; : \; 0 \leq i < |\vec{x}| )$ where $\pi \in \{L,R\}$.

We also use $P^{\Box} := P|\Box$.

In \cite{MeredithR05} an interpretation of the new operator is
given. It turns out that there are several possible interpretations
all enjoying the requisite algebraic properties of the operator (see
\cite{milner91polyadicpi}). We will therefore make liberal use of
$(\nu\; \vec{x})P$.

% subsection the_syntax_and_semantics_of_the_notation_system (end)   

\input{qm2pi.qmops} 

\input{qm2pi.sterngerlach} 

\input{qm2pi.metric} 

% section concurrent_process_calculi (end)

%\input{qm2pi.proofsketch}

% section proof sketch (end)

%\input{qm2pi.slviaknots} 

% section spatial logic via knots (end)

\input{qm2pi.conclusion}

% section conclusion (end)

%\input{qm2pi.dtcodes} 

% section wiring algorithm (end)

\input{qm2pi.ack} 

% section acknowledgments (end)

\newpage


\bibliographystyle{plain}   
\bibliography{../../biblios/main.bib}

\input{qm2pi.rhodetails}

\end{document}

 

%\documentclass[12pt]{llncs}
%\documentclass{jktr}

\usepackage[pdftex]{hyperref}                   
\usepackage {listings}
\usepackage {mathpartir}
\usepackage{bcprules}
%\usepackage{listings}
                       
\usepackage{graphicx} 
%\usepackage[margins=2.5cm,nohead,nofoot]{geometry}
%\usepackage{geometry}
\usepackage{amsfonts}
\usepackage{amstext}
\usepackage{latexsym}
\usepackage{amssymb}
\usepackage{color}


%\include{myPreamble}
\include{qm2pi.local} 

%\ifpdf
%\usepackage[pdftex]{graphicx}
%\else
%\usepackage{graphicx}
%\fi

 % \ifpdf
%  \usepackage{pdfsync}
%  \if


%\title{Brief Article}
%\author{David F. Snyder}
%\author{L.G. Meredith}

%\address{Dept. of Math., Texas State University--San Marcos, San Marcos, TX 78666}
       
\pagestyle{empty}


\begin{document}

\lstset{language=[Objective]Caml,frame=shadowbox}

\input{qm2pi.front}

% section front matter (end)

\input{qm2pi.intro} 
 
% section introduction (end)

% \input{qm2pi.knotations} 

% section notation (end)

\input{qm2pi.process.calculi} 

% section concurrent_process_calculi_and_spatial_logics_ (end)
    
%\input{qm2pi.knots2pi} 

%\input{qm2pi.trefoil} 

%\input{qm2pi.mainthm} 

% subsection basic_interpretation (end)

%\input{qm2pi.rho.presentation} 
\subsection{The syntax and semantics of the notation system}\label{sub:the_syntax_and_semantics_of_the_notation_system} % (fold)

We now summarize a technical presentation of the calculus that
embodies our theory of dynamics. The typical presentation of such a
calculus follows the style of giving generators and relations on
them. The grammar, below, describing term constructors, freely
generates the set of processes, $\Proc$. This set is then quotiented
by a relation known as structural congruence and it is over this set
that the notion of dynamics is expressed. This presentation is
essentially that of \cite{MeredithR05} with the addition of
polyadicity and summation. For readability we have relegated some of
the technical subtleties to an appendix.

\subsubsection{Process grammar}\label{subsub:process_grammar}

\begin{mathpar}
  \inferrule* [lab=synchronization] {} {{M} \bc \pzero \;|\; x?F \;|\; x!C }
  \and
  \inferrule* [lab=abstraction] {} {{F} \bc (x)P}
  \and
  \inferrule* [lab=concretion] {} {{C} \bc \langle Q \rangle}
  \and
  \inferrule* [lab=process] {} {{P,Q} \bc M \;| \;P|Q \;|\; @{x}}
  \and
  \inferrule* [lab=name] {} {{x} \bc \quotep{P}}
\end{mathpar} 

Note that $\vec{x}$ (resp. $\vec{P}$) denotes a vector of names
(resp. processes) of length $|\vec{x}|$ (resp. $|\vec{P}|$). We adopt
the following useful abbreviations.

\begin{mathpar}
   x?(\vec{y}).P := x.(\vec{y})P \and  x\clift{\vec{P}} := x.\clift{\vec{P}}
   \and x!(y) := \lift{x}{\dropn{y}}
   \and \Pi_{i=0}^{n-1}P_i := P_0 | \ldots | P_{n-1}
\end{mathpar}

\subsubsection{Structural congruence}

\paragraph{Free and bound names and alpha-equivalence.} At the
core of structural equivalence is alpha-equivalence which identifies
process that are the same up to a change of variable. Formally, we
recognize the distinction between free and bound names. The free names
of a process, $\freenames{P}$, may be calculated recursively as
follows:

\begin{mathpar}
\freenames{\pzero} := \emptyset
  \and \\
  \freenames{x?(y).P} := \{ x \} \cup (\freenames{P} \setminus \{ y \})
  \and 
  \freenames{x!\langle P \rangle} := \{ x \} \cup \{ P \} 
  \and \\
  \freenames{P|Q} := \freenames{P} \cup \freenames{Q}
  \and \\
  \freenames{@{x}} := \{ x \}
\end{mathpar}

$\pi$
$\quotep{\pi}$

$\freenames{-} : \pi \to \mathcal{P}(\quotep{\pi})$

\begin{eqnarray*}
  \freenames{\pzero} & := & \emptyset \\
  \freenames{x?(y).P} & := & \{ x \} \cup (\freenames{P} \setminus \{ y \}) \\
  \freenames{x!\langle P \rangle} & := & \{ x \} \cup \{ P \} \\
  \freenames{P|Q} & := & \freenames{P} \cup \freenames{Q} \\
  \freenames{\dropn{x}} & := & \{ x \}
\end{eqnarray*}

The bound names of a process, $\boundnames{P}$, are those names occurring in $P$
that are not free. For example, in $x?(y).0$, the name $x$ is free, while $y$ is bound.

\begin{mathpar}
  \inferrule* [lab=monoidal-laws] {} { P|Q \equiv Q|P \and P|0 \equiv P \and P|(Q|R) \equiv (P|Q)|R }
\end{mathpar}

\begin{mathpar}
  \inferrule* [lab=alpha-equivalence] {} { (x)P \equiv (y)P\{y/x\} \and y \not\in \freenames{P} }
\end{mathpar}

\begin{definition}
Then two processes, $P,Q$, are alpha-equivalent if $P = Q\{\vec{y}/\vec{x}\}$ for
some $\vec{x} \in \boundnames{Q},\vec{y} \in \boundnames{P}$, where $Q\{\vec{y}/\vec{x}\}$
denotes the capture-avoiding substitution of $\vec{y}$ for $\vec{x}$ in $Q$.
\end{definition}

\begin{definition}
  The {\em structural congruence} \cite{SangiorgiWalker} , $\equiv$,
  between processes is the least congruence containing
  alpha-equivalence, satisfying the abelian monoid laws
  (associativity, commutativity and $\pzero$ as identity) for parallel
  composition $|$ and for summation $+$.
\end{definition}

\subsection{Name equivalence}

We take name equivalence, written $\nameeq$, to be the smallest
equivalence relation generated by the following rules.

\begin{mathpar}
\inferrule*[lab=Quote-drop]
{ }
{ \quotep{@{x}} \nameeq x }

\inferrule*[lab=Struct-equiv]
{ P \scong Q }
{ \quotep{P} \nameeq \quotep{Q} }
\end{mathpar}

The astute reader will have noticed that the mutual recursion of names
and processes imposes a mutual recursion on alpha-equivalence and
structural equivalence via name-equivalence. Fortunately, all of this
works out pleasantly and we may calculate in the natural way, free of
concern. The reader interested in the details is referred to the
appendix \ref{appendix:rho_details}.

\subsection{Substitution}

We use $\Proc$ for the set of processes, $\QProc$ for the set of
names, and $\id{\{}\vec{y} / \vec{x} \id{\}}$ to denote partial maps,
$s : \QProc \rightarrow \QProc$. A map, $s$ lifts, uniquely, to a map
on process terms, $\widehat{s} : \Proc \rightarrow \Proc$ by the
following equations.

\begin{mathpar}
  (0) \psubstp{Q}{P} := 0 \\
  (R \juxtap S) \psubstp{Q}{P}
  :=    
  (R)\psubstp{Q}{P} \juxtap (S) \psubstp{Q}{P} \\
  (x?(y).R) \psubstp{Q}{P}    
  :=    
  (x)\substp{Q}{P} (z)\concat( (R \psubstn{z}{y}) \psubstp{Q}{P} ) \\
  (\lift{x}{R}) \psubstp{Q}{P}  
  :=
  \lift{(x)\substp{Q}{P}}{ R \psubstp{Q}{P} } \\
%   (\dropn{x})  \psubstp{Q}{P}       
%   := 
%   \left\{ 
%     \begin{array}{ccc} 
%       \dropn{\quotep{Q}} & & x \nameeq \quotep{P} \\
%       \dropn{x} & & otherwise \\
%     \end{array}
%   \right. 
  (\dropn{x})  \psubstp{Q}{P}       
  := 
  \left\{ 
    \begin{array}{ccc} 
      Q & & x \nameeq \quotep{P} \\
      \dropn{x} & & otherwise \\
    \end{array}
  \right.
\end{mathpar}
 

where

\begin{eqnarray}
  (x)\id{\{} \lpquote Q \rpquote / \lpquote P \rpquote \id{\}}            = 
  \left\{ 
    \begin{array}{ccc}
      \lpquote Q \rpquote & & x \nameeq \lpquote P \rpquote \\
      x & & otherwise \\
    \end{array}
  \right. \nonumber
\end{eqnarray}

and $z$ is chosen distinct from $\quotep{P}$, $\quotep{Q}$, the free
names in $Q$, and all the names in $R$. Our $\alpha$-equivalence will
be built in the standard way from this substitution.

\begin{remark}\label{rem:no_self_referential_names}
  One consequence of these definitions is that $\forall P. \quotep{P}
  \not\in \freenames{P}$.
\end{remark}

\subsection{ Dynamic quote: an example }

Anticipating something of what's to come, consider applying the
substitution, $\widehat{\id{\{}u / z \id{\}}}$, to the following pair
of processes, $\lift{w}{y!(z)}$ and $w[ \lpquote y!(z) \rpquote ]$.

\begin{eqnarray}
	\lift{w}{y!(z)}\widehat{\id{\{}u / z \id{\}}}
		& = &
		\lift{w}{y!(u)} \nonumber\\
	w[ \lpquote y!(z) \rpquote ] \widehat{ \id{\{}u / z \id{\}} }
		& = &
		w[ \lpquote y!(z) \rpquote ] \nonumber
\end{eqnarray}

Because the body of the process between quotes is impervious to
substitution, we get radically different answers. In fact, by
examining the first process in an input context,
e.g. $x?(z).\lift{w}{y!(z)}$, we see that the process under the lift
operator may be shaped by prefixed inputs binding a name inside it. In
this sense, the lift operator will be seen as a way to dynamically
construct processes before reifying them as names.

Finally equipped with these standard features we can present the
dynamics of the calculus.

\subsubsection{Operational semantics} 

Finally, we introduce the computational dynamics. What marks these
algebras as distinct from other more traditionally studied algebraic
structures, e.g. vector spaces or polynomial rings, is the manner in
which dynamics is captured. In traditional structures, dynamics is typically
expressed through morphisms between such structures, as in linear maps
between vector spaces or morphisms between rings. In algebras
associated with the semantics of computation, the dynamics is
expressed as part of the algebraic structure itself, through a
reduction reduction relation typically denoted by $\red$. Below, we
give a recursive presentation of this relation for the calculus used
in the encoding.

$\red \subseteq \pi \times \pi$
$\red : \pi \to \mathcal{P}(\pi)$

\begin{mathpar}
  \inferrule* [lab=Comm] { \textsf{match}( x_{src}, x_{trgt} ) } { x_{trgt}?(y)P \; | \; x_{src}!\langle {Q} \rangle \red P\{\quotep{Q}/y}\} }
  \and \\
  \inferrule* [lab=Par] {{P} \red {P}'} {{{P} | {Q}} \red {{P}' | {Q}}}
  \and
  \inferrule* [lab=Equiv]{{{P} \scong {P}'} \andalso {{P}' \red {Q}'} \andalso {{Q}' \scong {Q}}}{{P} \red {Q}}
\end{mathpar}

\begin{eqnarray*}
  match_{\equiv} (\quotep{P},\quotep{Q}) & := & P \equiv Q \\
  match_{\dagger}(\quotep{P},\quotep{Q}) & := & \forall R. P|Q \red^{*} R => R \red^{*} 0 \\
  match_{K}(\quotep{P},\quotep{Q}) & := & K \mbox{ for some context } K
\end{eqnarray*}

$u?(x)P | u!\langle Q \rangle \red P\{\quotep{Q}/x\}$

%We write $\wred$ for $\red^*$, and $P\red$ if $\exists Q $ such that $ P \red Q$.
We write $P\red$ if $\exists Q $ such that $ P \red Q$ and $P\not\red$, otherwise.

\section{Replication}

As mentioned before, it is known that replication (and hence
recursion) can be implemented in a higher-order process algebra
\cite{SangiorgiWalker}. As our first example of calculation with the
machinery thus far presented we give the construction explicitly in
the {\rhoc}.

\begin{eqnarray}
	D_{x} & := & \prefix{x}{y}{(\binpar{\outputp{x}{y}}{@{y}})} \nonumber\\
	\bangp_{x}{P} & := & \binpar{{x}!\langle{\binpar{D_{x}}{P}}\rangle}{D_{x}} \nonumber
\end{eqnarray}

\begin{eqnarray}
	\bangp_{x}{P} & & \nonumber\\
	=
	& {x}!\langle{(\prefix{x}{y}{(\outputp{x}{y} | @{y})) | P}}\rangle 
	      | \prefix{x}{y}{(\outputp{x}{y} | @{y})} & \nonumber\\
	\red
	& (\outputp{x}{y} | @{y})\substn{\quotep{(\prefix{x}{y}{(@{y} | \outputp{x}{y})) | P}}}{y} & \nonumber\\
	=
	& \outputp{x}{\quotep{(\prefix{x}{y}{(\outputp{x}{y} | @{y})) | P}}}
	  | {(\prefix{x}{y}{(\outputp{x}{y} | @{y})) | P}} & \nonumber\\
	\red
	& \ldots & \nonumber\\
	\red^*
	& P | P | \ldots & \nonumber
\end{eqnarray}

Of course, this encoding, as an implementation, runs away, unfolding
$\bangp{P}$ eagerly. A lazier and more implementable replication
operator, restricted to input-guarded processes, may be obtained as follows.

\begin{eqnarray}
\bangp{\prefix{u}{v}{P}} 
	:= 
	\binpar{\lift{x}{\prefix{u}{v}{(\binpar{D(x)}{P})}}}{D(x)} \nonumber
\end{eqnarray}

\begin{remark}
  Note that the lazier definition still does not deal with summation
  or mixed summation (i.e. sums over input and output). The reader is
  invited to construct definitions of replication that deal with these
  features. 

  Further, the definitions are parameterized in a name, $x$. Can you,
  gentle reader, make a definition that eliminates this parameter and
  guarantees no accidental interaction between the replication
  machinery and the process being replicated -- i.e. no accidental
  sharing of names used by the process to get its work done and the
  name(s) used by the replication to effect copying. This latter
  revision of the definition of replication is crucial to obtaining
  the expected identity $!!P \sim !P$.
\end{remark}

\begin{remark}\label{rem:paradoxical_combinator}
  The reader familiar with the lambda calculus will have noticed the
  similarity between $D$ and the paradoxical combinator.

  [Ed. note: the existence of this seems to suggest we have to be more
  restrictive on the set of processes and names we admit if we are to
  support no-cloning.]
\end{remark}

\subsubsection{Bisimulation}

The computational dynamics gives rise to another kind of equivalence,
the equivalence of computational behavior. As previously mentioned
this is typically captured \emph{via} some form of bisimulation.

% The notion we use in this paper is weak barbed bisimulation
% \cite{milner91polyadicpi}.

The notion we use in this paper is derived from weak barbed
bisimulation \cite{milner91polyadicpi}. 

\begin{definition}
An \emph{observation relation}, $\downarrow_{\mathcal N}$, over a set
of names, $\mathcal N$, is the smallest relation satisfying the rules
below.

\infrule[Out-barb]{y \in {\mathcal N}, \; x \nameeq y}
		  {\outputp{x}{v} \downarrow_{\mathcal N} x}
\infrule[Par-barb]{\mbox{$P\downarrow_{\mathcal N} x$ or $Q\downarrow_{\mathcal N} x$}}
		  {\binpar{P}{Q} \downarrow_{\mathcal N} x}

We write $P \Downarrow_{\mathcal N} x$ if there is $Q$ such that 
$P \wred Q$ and $Q \downarrow_{\mathcal N} x$.
\end{definition}

\begin{definition}
%\label{def.bbisim}
An  ${\mathcal N}$-\emph{barbed bisimulation} over a set of names, ${\mathcal N}$, is a symmetric binary relation 
${\mathcal S}_{\mathcal N}$ between agents such that $P\rel{S}_{\mathcal N}Q$ implies:
\begin{enumerate}
\item If $P \red P'$ then $Q \wred Q'$ and $P'\rel{S}_{\mathcal N} Q'$.
\item If $P\downarrow_{\mathcal N} x$, then $Q\Downarrow_{\mathcal N} x$.
\end{enumerate}
$P$ is ${\mathcal N}$-barbed bisimilar to $Q$, written
$P \wbbisim_{\mathcal N} Q$, if $P \rel{S}_{\mathcal N} Q$ for some ${\mathcal N}$-barbed bisimulation ${\mathcal S}_{\mathcal N}$.
\end{definition}

$\mathcal{R} \subseteq \pi \times \pi$

$P \mathcal{R} Q => \forall P'. P \red P' \Rightarrow \exists Q'. Q \red Q', P' \mathcal{R} Q'$

$P \vdash x \Rightarrow Q \vdash x$

\begin{mathpar}
  \inferrule*[lab=Out-barb]{x \nameeq y}{{y}!\langle{Q}\rangle \vdash x}
  \and
  \inferrule*[lab=Par-barb]{\mbox{$P\vdash x$ or $Q\vdash x$}}{\binpar{P}{Q} \vdash x}
\end{mathpar}

\subsubsection{Contexts}

One of the principle advantages of computational calculi like the
$\pi$-calculus is a well-defined notion of context,
contextual-equivalence and a correlation between
contextual-equivalence and notions of bisimulation. The notion of
context allows the decomposition of a process into (sub-)process and
its syntactic environment, its context. Thus, a context may be
thought of as a process with a ``hole'' (written $\Box$) in it. The
application of a context $M$ to a process $P$, written $M[P]$, is
tantamount to filling the hole in $M$ with $P$. In this paper we do
not need the full weight of this theory, but do make use of the notion
of context in the proof the main theorem. 

\begin{mathpar}
  \inferrule* [lab=summation] {} {{M_{M},M_{N}} \bc \Box \;|\; x.M_{A} \;|\; M_{M}+M_{N}}
  \and
  \inferrule* [lab=agent] {} {{M_{A}} \bc (\vec{x})M_{P} \;| \; \clift{P_0,\ldots,M_{P},\ldots,P_N}}
  \and \\
  \inferrule* [lab=process] {} {{M_{P}} \bc M_{N} \;| \;P|M_{P} }
\end{mathpar} 

\begin{mathpar}
  \inferrule* [lab=sychronization] {} {M_{N} \bc \Box \;|\; x?M_{F} \;|\; x!M_{C}}
  \and
  \inferrule* [lab=abstraction] {} {{M_{F}} \bc (x)M_{P} }
  \and
  \inferrule* [lab=concretion] {} {{M_{C}} \bc \langle M_{P} \rangle }
  \and \\
  \inferrule* [lab=process] {} {{M_{P}} \bc M_{N} \;| \;P|M_{P} }
\end{mathpar}

\begin{definition}[contextual application] Given a context $M$, and
  process $P$, we define the \emph{contextual application}, $M[P] :=
  M\{P/\Box\}$. That is, the contextual application of M to P is the
  substitution of $P$ for $\Box$ in $M$.
\end{definition}

$\meaningof{-} : L \to \mathcal{P}(\pi)$

\begin{mathpar}
  \inferrule* [lab=collection] {} {\meaningof{true} = \pi, \and \meaningof{~E} = \pi \setminus \meaningof{E}, \and \meaningof{E_{1} \& E_{2}} = \meaningof{E_{1}} \cap \meaningof{E_{2}}}
\end{mathpar}

\begin{mathpar}
  \inferrule* [lab=structure] {} {\meaningof{0} = \{ P \in \pi | P \equiv 0 \}, \and \\ \meaningof{E_1 | E_2} = \{ P \in \pi | P \equiv P_{1} | P_{2}, P_{1} \in \meaningof{E_{1}}, P_{2} \in \meaningof{E_2}\} }
\end{mathpar}

\begin{mathpar}
 \inferrule* [lab=behavior] {} {\meaningof{\langle a?b \rangle E} = \{ P \in \pi | P \equiv Q | u?(y)P', \\ \and \\\\ \and \\ \;\;\; u \in \meaningof{a}, \forall z.P'\{z/y\} \in \meaningof{E\{z/b\}}\}, \and \\ \meaningof{a!E} = \{ P \in \pi | P \equiv Q | x!\langle P' \rangle, x \in \meaningof{a} P' \in \meaningof{E}\} }
\end{mathpar}

\begin{mathpar}
 \inferrule* [lab=nominal] {} {\meaningof{\quotep{E}} = \{ \quotep{P} \in \quotep{\pi} | P \in \meaningof{E} \}, \and \meaningof{\quotep{P}} = \{ \quotep{Q} \in \quotep{\pi} | P \equiv Q \} \and \\ \meaningof{@\quotep{E}} = \{ P \in \pi | P \equiv @x, x \in \meaningof{E} \}}
\end{mathpar}

\begin{eqnarray*}
  \\
  \meaningof{-} : TS \to ST
\end{eqnarray*}

\begin{eqnarray*}
  \\
  L : TS \to ST
\end{eqnarray*}

\begin{eqnarray*}
  \\
  P \models E \iff P \in \meaningof{E}
\end{eqnarray*}

\begin{eqnarray*}
  P \approx_{L} Q \iff \forall E \in L. P \models E \iff Q \models E
\end{eqnarray*}

\begin{eqnarray*}
  P \approx_{K} Q
\end{eqnarray*}

\begin{eqnarray*}
  P \approx Q
\end{eqnarray*}

$\approx_{K} = \approx = \approx_{L}$

\subsubsection{Contextual duality}

Note that contexts extend the quotation operation to a family of
operations from processes to names. Given a context, $M$, we can
define a \emph{nominal context}, $\quotep{M}$ by $\quotep{M}[P] :=
\quotep{M[P]}$. To foreshadow what is to come we observe that these
operations enjoy a duality with processes very much like the duality
between vectors and maps from vectors to scalars.

Further, because the calculus is essentially higher-order, we have a
correspondence between contexts and processes. More specifically,
given a name $x$ and a context $M$ we can construct $M^{*}_{x}$ such
that 

\begin{mathpar}
  M^{*}_{x} | \lift{x}{P} \red M[P]
\end{mathpar}

namely,

\begin{mathpar}
  M^{*}_{x} := x?(u).M[\dropn{u}]
\end{mathpar}

The dependence of $M^{*}_{x}$ on a name makes it an abstraction, 

\begin{mathpar}
  M^{*} := (x)x?(u).M[\dropn{u}]
\end{mathpar}

\subsection{Additional notation}

It will sometimes be convenient to denote the process a name
quotes. We already have the notation $x = \quotep{P}$, but it will be
convenient to introduce an alternate notation, $\procn{x}$, when we
want to emphasize the connection to the use of the name. Note that, by
virtue of name equivalence, $\quotep{\procn{x}} \nameeq x$; so, the
notation is consistent with previous definitions.

Further, because names have structure it is possible to effect
substitutions on the basis of that structure. This means we need to
upgrade our notation for substitutions, which we accomplish by
adapting comprehension notation. Thus,

\begin{mathpar}
  P\{ y / x : x \in S \}
\end{mathpar}

is interpreted to mean the process derived from P by replacing (in a
capture-avoiding manner) each occurrence of $x$ in $S$ by $y$. For example,

\begin{mathpar}
  P\{ \quotep{\procn{x}|\procn{x}} / x : x \in \freenames{P} \}
\end{mathpar}

will replace each (occurrence) of a free name $x$ in $P$ by
$\quotep{\procn{x}|\procn{x}}$.

Also, we will avail ourselves of the notation $x^{L}$ and $x^{R}$ to
denote injections of a name into disjoint copies of the name
space. There are numerous ways to accomplish this. One example can be
found in \cite{MeredithR05}. This notation overloads to vectors of
names: $\vec{x}^{\pi} := (x_{i}^{\pi} \; : \; 0 \leq i < |\vec{x}| )$ where $\pi \in \{L,R\}$.

We also use $P^{\Box} := P|\Box$.

In \cite{MeredithR05} an interpretation of the new operator is
given. It turns out that there are several possible interpretations
all enjoying the requisite algebraic properties of the operator (see
\cite{milner91polyadicpi}). We will therefore make liberal use of
$(\nu\; \vec{x})P$.

% subsection the_syntax_and_semantics_of_the_notation_system (end)   

\input{qm2pi.qmops} 

\input{qm2pi.sterngerlach} 

\input{qm2pi.metric} 

% section concurrent_process_calculi (end)

%\input{qm2pi.proofsketch}

% section proof sketch (end)

%\input{qm2pi.slviaknots} 

% section spatial logic via knots (end)

\input{qm2pi.conclusion}

% section conclusion (end)

%\input{qm2pi.dtcodes} 

% section wiring algorithm (end)

\input{qm2pi.ack} 

% section acknowledgments (end)

\newpage


\bibliographystyle{plain}   
\bibliography{../../biblios/main.bib}

\input{qm2pi.rhodetails}

\end{document}

 

% subsection basic_interpretation (end)

%\input{qm2pi.rho.presentation} 
\subsection{The syntax and semantics of the notation system}\label{sub:the_syntax_and_semantics_of_the_notation_system} % (fold)

We now summarize a technical presentation of the calculus that
embodies our theory of dynamics. The typical presentation of such a
calculus follows the style of giving generators and relations on
them. The grammar, below, describing term constructors, freely
generates the set of processes, $\Proc$. This set is then quotiented
by a relation known as structural congruence and it is over this set
that the notion of dynamics is expressed. This presentation is
essentially that of \cite{MeredithR05} with the addition of
polyadicity and summation. For readability we have relegated some of
the technical subtleties to an appendix.

\subsubsection{Process grammar}\label{subsub:process_grammar}

\begin{mathpar}
  \inferrule* [lab=synchronization] {} {{M} \bc \pzero \;|\; x?F \;|\; x!C }
  \and
  \inferrule* [lab=abstraction] {} {{F} \bc (x)P}
  \and
  \inferrule* [lab=concretion] {} {{C} \bc \langle Q \rangle}
  \and
  \inferrule* [lab=process] {} {{P,Q} \bc M \;| \;P|Q \;|\; @{x}}
  \and
  \inferrule* [lab=name] {} {{x} \bc \quotep{P}}
\end{mathpar} 

Note that $\vec{x}$ (resp. $\vec{P}$) denotes a vector of names
(resp. processes) of length $|\vec{x}|$ (resp. $|\vec{P}|$). We adopt
the following useful abbreviations.

\begin{mathpar}
   x?(\vec{y}).P := x.(\vec{y})P \and  x\clift{\vec{P}} := x.\clift{\vec{P}}
   \and x!(y) := \lift{x}{\dropn{y}}
   \and \Pi_{i=0}^{n-1}P_i := P_0 | \ldots | P_{n-1}
\end{mathpar}

\subsubsection{Structural congruence}

\paragraph{Free and bound names and alpha-equivalence.} At the
core of structural equivalence is alpha-equivalence which identifies
process that are the same up to a change of variable. Formally, we
recognize the distinction between free and bound names. The free names
of a process, $\freenames{P}$, may be calculated recursively as
follows:

\begin{mathpar}
\freenames{\pzero} := \emptyset
  \and \\
  \freenames{x?(y).P} := \{ x \} \cup (\freenames{P} \setminus \{ y \})
  \and 
  \freenames{x!\langle P \rangle} := \{ x \} \cup \{ P \} 
  \and \\
  \freenames{P|Q} := \freenames{P} \cup \freenames{Q}
  \and \\
  \freenames{@{x}} := \{ x \}
\end{mathpar}

$\pi$
$\quotep{\pi}$

$\freenames{-} : \pi \to \mathcal{P}(\quotep{\pi})$

\begin{eqnarray*}
  \freenames{\pzero} & := & \emptyset \\
  \freenames{x?(y).P} & := & \{ x \} \cup (\freenames{P} \setminus \{ y \}) \\
  \freenames{x!\langle P \rangle} & := & \{ x \} \cup \{ P \} \\
  \freenames{P|Q} & := & \freenames{P} \cup \freenames{Q} \\
  \freenames{\dropn{x}} & := & \{ x \}
\end{eqnarray*}

The bound names of a process, $\boundnames{P}$, are those names occurring in $P$
that are not free. For example, in $x?(y).0$, the name $x$ is free, while $y$ is bound.

\begin{mathpar}
  \inferrule* [lab=monoidal-laws] {} { P|Q \equiv Q|P \and P|0 \equiv P \and P|(Q|R) \equiv (P|Q)|R }
\end{mathpar}

\begin{mathpar}
  \inferrule* [lab=alpha-equivalence] {} { (x)P \equiv (y)P\{y/x\} \and y \not\in \freenames{P} }
\end{mathpar}

\begin{definition}
Then two processes, $P,Q$, are alpha-equivalent if $P = Q\{\vec{y}/\vec{x}\}$ for
some $\vec{x} \in \boundnames{Q},\vec{y} \in \boundnames{P}$, where $Q\{\vec{y}/\vec{x}\}$
denotes the capture-avoiding substitution of $\vec{y}$ for $\vec{x}$ in $Q$.
\end{definition}

\begin{definition}
  The {\em structural congruence} \cite{SangiorgiWalker} , $\equiv$,
  between processes is the least congruence containing
  alpha-equivalence, satisfying the abelian monoid laws
  (associativity, commutativity and $\pzero$ as identity) for parallel
  composition $|$ and for summation $+$.
\end{definition}

\subsection{Name equivalence}

We take name equivalence, written $\nameeq$, to be the smallest
equivalence relation generated by the following rules.

\begin{mathpar}
\inferrule*[lab=Quote-drop]
{ }
{ \quotep{@{x}} \nameeq x }

\inferrule*[lab=Struct-equiv]
{ P \scong Q }
{ \quotep{P} \nameeq \quotep{Q} }
\end{mathpar}

The astute reader will have noticed that the mutual recursion of names
and processes imposes a mutual recursion on alpha-equivalence and
structural equivalence via name-equivalence. Fortunately, all of this
works out pleasantly and we may calculate in the natural way, free of
concern. The reader interested in the details is referred to the
appendix \ref{appendix:rho_details}.

\subsection{Substitution}

We use $\Proc$ for the set of processes, $\QProc$ for the set of
names, and $\id{\{}\vec{y} / \vec{x} \id{\}}$ to denote partial maps,
$s : \QProc \rightarrow \QProc$. A map, $s$ lifts, uniquely, to a map
on process terms, $\widehat{s} : \Proc \rightarrow \Proc$ by the
following equations.

\begin{mathpar}
  (0) \psubstp{Q}{P} := 0 \\
  (R \juxtap S) \psubstp{Q}{P}
  :=    
  (R)\psubstp{Q}{P} \juxtap (S) \psubstp{Q}{P} \\
  (x?(y).R) \psubstp{Q}{P}    
  :=    
  (x)\substp{Q}{P} (z)\concat( (R \psubstn{z}{y}) \psubstp{Q}{P} ) \\
  (\lift{x}{R}) \psubstp{Q}{P}  
  :=
  \lift{(x)\substp{Q}{P}}{ R \psubstp{Q}{P} } \\
%   (\dropn{x})  \psubstp{Q}{P}       
%   := 
%   \left\{ 
%     \begin{array}{ccc} 
%       \dropn{\quotep{Q}} & & x \nameeq \quotep{P} \\
%       \dropn{x} & & otherwise \\
%     \end{array}
%   \right. 
  (\dropn{x})  \psubstp{Q}{P}       
  := 
  \left\{ 
    \begin{array}{ccc} 
      Q & & x \nameeq \quotep{P} \\
      \dropn{x} & & otherwise \\
    \end{array}
  \right.
\end{mathpar}
 

where

\begin{eqnarray}
  (x)\id{\{} \lpquote Q \rpquote / \lpquote P \rpquote \id{\}}            = 
  \left\{ 
    \begin{array}{ccc}
      \lpquote Q \rpquote & & x \nameeq \lpquote P \rpquote \\
      x & & otherwise \\
    \end{array}
  \right. \nonumber
\end{eqnarray}

and $z$ is chosen distinct from $\quotep{P}$, $\quotep{Q}$, the free
names in $Q$, and all the names in $R$. Our $\alpha$-equivalence will
be built in the standard way from this substitution.

\begin{remark}\label{rem:no_self_referential_names}
  One consequence of these definitions is that $\forall P. \quotep{P}
  \not\in \freenames{P}$.
\end{remark}

\subsection{ Dynamic quote: an example }

Anticipating something of what's to come, consider applying the
substitution, $\widehat{\id{\{}u / z \id{\}}}$, to the following pair
of processes, $\lift{w}{y!(z)}$ and $w[ \lpquote y!(z) \rpquote ]$.

\begin{eqnarray}
	\lift{w}{y!(z)}\widehat{\id{\{}u / z \id{\}}}
		& = &
		\lift{w}{y!(u)} \nonumber\\
	w[ \lpquote y!(z) \rpquote ] \widehat{ \id{\{}u / z \id{\}} }
		& = &
		w[ \lpquote y!(z) \rpquote ] \nonumber
\end{eqnarray}

Because the body of the process between quotes is impervious to
substitution, we get radically different answers. In fact, by
examining the first process in an input context,
e.g. $x?(z).\lift{w}{y!(z)}$, we see that the process under the lift
operator may be shaped by prefixed inputs binding a name inside it. In
this sense, the lift operator will be seen as a way to dynamically
construct processes before reifying them as names.

Finally equipped with these standard features we can present the
dynamics of the calculus.

\subsubsection{Operational semantics} 

Finally, we introduce the computational dynamics. What marks these
algebras as distinct from other more traditionally studied algebraic
structures, e.g. vector spaces or polynomial rings, is the manner in
which dynamics is captured. In traditional structures, dynamics is typically
expressed through morphisms between such structures, as in linear maps
between vector spaces or morphisms between rings. In algebras
associated with the semantics of computation, the dynamics is
expressed as part of the algebraic structure itself, through a
reduction reduction relation typically denoted by $\red$. Below, we
give a recursive presentation of this relation for the calculus used
in the encoding.

$\red \subseteq \pi \times \pi$
$\red : \pi \to \mathcal{P}(\pi)$

\begin{mathpar}
  \inferrule* [lab=Comm] { \textsf{match}( x_{src}, x_{trgt} ) } { x_{trgt}?(y)P \; | \; x_{src}!\langle {Q} \rangle \red P\{\quotep{Q}/y}\} }
  \and \\
  \inferrule* [lab=Par] {{P} \red {P}'} {{{P} | {Q}} \red {{P}' | {Q}}}
  \and
  \inferrule* [lab=Equiv]{{{P} \scong {P}'} \andalso {{P}' \red {Q}'} \andalso {{Q}' \scong {Q}}}{{P} \red {Q}}
\end{mathpar}

\begin{eqnarray*}
  match_{\equiv} (\quotep{P},\quotep{Q}) & := & P \equiv Q \\
  match_{\dagger}(\quotep{P},\quotep{Q}) & := & \forall R. P|Q \red^{*} R => R \red^{*} 0 \\
  match_{K}(\quotep{P},\quotep{Q}) & := & K \mbox{ for some context } K
\end{eqnarray*}

$u?(x)P | u!\langle Q \rangle \red P\{\quotep{Q}/x\}$

%We write $\wred$ for $\red^*$, and $P\red$ if $\exists Q $ such that $ P \red Q$.
We write $P\red$ if $\exists Q $ such that $ P \red Q$ and $P\not\red$, otherwise.

\section{Replication}

As mentioned before, it is known that replication (and hence
recursion) can be implemented in a higher-order process algebra
\cite{SangiorgiWalker}. As our first example of calculation with the
machinery thus far presented we give the construction explicitly in
the {\rhoc}.

\begin{eqnarray}
	D_{x} & := & \prefix{x}{y}{(\binpar{\outputp{x}{y}}{@{y}})} \nonumber\\
	\bangp_{x}{P} & := & \binpar{{x}!\langle{\binpar{D_{x}}{P}}\rangle}{D_{x}} \nonumber
\end{eqnarray}

\begin{eqnarray}
	\bangp_{x}{P} & & \nonumber\\
	=
	& {x}!\langle{(\prefix{x}{y}{(\outputp{x}{y} | @{y})) | P}}\rangle 
	      | \prefix{x}{y}{(\outputp{x}{y} | @{y})} & \nonumber\\
	\red
	& (\outputp{x}{y} | @{y})\substn{\quotep{(\prefix{x}{y}{(@{y} | \outputp{x}{y})) | P}}}{y} & \nonumber\\
	=
	& \outputp{x}{\quotep{(\prefix{x}{y}{(\outputp{x}{y} | @{y})) | P}}}
	  | {(\prefix{x}{y}{(\outputp{x}{y} | @{y})) | P}} & \nonumber\\
	\red
	& \ldots & \nonumber\\
	\red^*
	& P | P | \ldots & \nonumber
\end{eqnarray}

Of course, this encoding, as an implementation, runs away, unfolding
$\bangp{P}$ eagerly. A lazier and more implementable replication
operator, restricted to input-guarded processes, may be obtained as follows.

\begin{eqnarray}
\bangp{\prefix{u}{v}{P}} 
	:= 
	\binpar{\lift{x}{\prefix{u}{v}{(\binpar{D(x)}{P})}}}{D(x)} \nonumber
\end{eqnarray}

\begin{remark}
  Note that the lazier definition still does not deal with summation
  or mixed summation (i.e. sums over input and output). The reader is
  invited to construct definitions of replication that deal with these
  features. 

  Further, the definitions are parameterized in a name, $x$. Can you,
  gentle reader, make a definition that eliminates this parameter and
  guarantees no accidental interaction between the replication
  machinery and the process being replicated -- i.e. no accidental
  sharing of names used by the process to get its work done and the
  name(s) used by the replication to effect copying. This latter
  revision of the definition of replication is crucial to obtaining
  the expected identity $!!P \sim !P$.
\end{remark}

\begin{remark}\label{rem:paradoxical_combinator}
  The reader familiar with the lambda calculus will have noticed the
  similarity between $D$ and the paradoxical combinator.

  [Ed. note: the existence of this seems to suggest we have to be more
  restrictive on the set of processes and names we admit if we are to
  support no-cloning.]
\end{remark}

\subsubsection{Bisimulation}

The computational dynamics gives rise to another kind of equivalence,
the equivalence of computational behavior. As previously mentioned
this is typically captured \emph{via} some form of bisimulation.

% The notion we use in this paper is weak barbed bisimulation
% \cite{milner91polyadicpi}.

The notion we use in this paper is derived from weak barbed
bisimulation \cite{milner91polyadicpi}. 

\begin{definition}
An \emph{observation relation}, $\downarrow_{\mathcal N}$, over a set
of names, $\mathcal N$, is the smallest relation satisfying the rules
below.

\infrule[Out-barb]{y \in {\mathcal N}, \; x \nameeq y}
		  {\outputp{x}{v} \downarrow_{\mathcal N} x}
\infrule[Par-barb]{\mbox{$P\downarrow_{\mathcal N} x$ or $Q\downarrow_{\mathcal N} x$}}
		  {\binpar{P}{Q} \downarrow_{\mathcal N} x}

We write $P \Downarrow_{\mathcal N} x$ if there is $Q$ such that 
$P \wred Q$ and $Q \downarrow_{\mathcal N} x$.
\end{definition}

\begin{definition}
%\label{def.bbisim}
An  ${\mathcal N}$-\emph{barbed bisimulation} over a set of names, ${\mathcal N}$, is a symmetric binary relation 
${\mathcal S}_{\mathcal N}$ between agents such that $P\rel{S}_{\mathcal N}Q$ implies:
\begin{enumerate}
\item If $P \red P'$ then $Q \wred Q'$ and $P'\rel{S}_{\mathcal N} Q'$.
\item If $P\downarrow_{\mathcal N} x$, then $Q\Downarrow_{\mathcal N} x$.
\end{enumerate}
$P$ is ${\mathcal N}$-barbed bisimilar to $Q$, written
$P \wbbisim_{\mathcal N} Q$, if $P \rel{S}_{\mathcal N} Q$ for some ${\mathcal N}$-barbed bisimulation ${\mathcal S}_{\mathcal N}$.
\end{definition}

$\mathcal{R} \subseteq \pi \times \pi$

$P \mathcal{R} Q => \forall P'. P \red P' \Rightarrow \exists Q'. Q \red Q', P' \mathcal{R} Q'$

$P \vdash x \Rightarrow Q \vdash x$

\begin{mathpar}
  \inferrule*[lab=Out-barb]{x \nameeq y}{{y}!\langle{Q}\rangle \vdash x}
  \and
  \inferrule*[lab=Par-barb]{\mbox{$P\vdash x$ or $Q\vdash x$}}{\binpar{P}{Q} \vdash x}
\end{mathpar}

\subsubsection{Contexts}

One of the principle advantages of computational calculi like the
$\pi$-calculus is a well-defined notion of context,
contextual-equivalence and a correlation between
contextual-equivalence and notions of bisimulation. The notion of
context allows the decomposition of a process into (sub-)process and
its syntactic environment, its context. Thus, a context may be
thought of as a process with a ``hole'' (written $\Box$) in it. The
application of a context $M$ to a process $P$, written $M[P]$, is
tantamount to filling the hole in $M$ with $P$. In this paper we do
not need the full weight of this theory, but do make use of the notion
of context in the proof the main theorem. 

\begin{mathpar}
  \inferrule* [lab=summation] {} {{M_{M},M_{N}} \bc \Box \;|\; x.M_{A} \;|\; M_{M}+M_{N}}
  \and
  \inferrule* [lab=agent] {} {{M_{A}} \bc (\vec{x})M_{P} \;| \; \clift{P_0,\ldots,M_{P},\ldots,P_N}}
  \and \\
  \inferrule* [lab=process] {} {{M_{P}} \bc M_{N} \;| \;P|M_{P} }
\end{mathpar} 

\begin{mathpar}
  \inferrule* [lab=sychronization] {} {M_{N} \bc \Box \;|\; x?M_{F} \;|\; x!M_{C}}
  \and
  \inferrule* [lab=abstraction] {} {{M_{F}} \bc (x)M_{P} }
  \and
  \inferrule* [lab=concretion] {} {{M_{C}} \bc \langle M_{P} \rangle }
  \and \\
  \inferrule* [lab=process] {} {{M_{P}} \bc M_{N} \;| \;P|M_{P} }
\end{mathpar}

\begin{definition}[contextual application] Given a context $M$, and
  process $P$, we define the \emph{contextual application}, $M[P] :=
  M\{P/\Box\}$. That is, the contextual application of M to P is the
  substitution of $P$ for $\Box$ in $M$.
\end{definition}

$\meaningof{-} : L \to \mathcal{P}(\pi)$

\begin{mathpar}
  \inferrule* [lab=collection] {} {\meaningof{true} = \pi, \and \meaningof{~E} = \pi \setminus \meaningof{E}, \and \meaningof{E_{1} \& E_{2}} = \meaningof{E_{1}} \cap \meaningof{E_{2}}}
\end{mathpar}

\begin{mathpar}
  \inferrule* [lab=structure] {} {\meaningof{0} = \{ P \in \pi | P \equiv 0 \}, \and \\ \meaningof{E_1 | E_2} = \{ P \in \pi | P \equiv P_{1} | P_{2}, P_{1} \in \meaningof{E_{1}}, P_{2} \in \meaningof{E_2}\} }
\end{mathpar}

\begin{mathpar}
 \inferrule* [lab=behavior] {} {\meaningof{\langle a?b \rangle E} = \{ P \in \pi | P \equiv Q | u?(y)P', \\ \and \\\\ \and \\ \;\;\; u \in \meaningof{a}, \forall z.P'\{z/y\} \in \meaningof{E\{z/b\}}\}, \and \\ \meaningof{a!E} = \{ P \in \pi | P \equiv Q | x!\langle P' \rangle, x \in \meaningof{a} P' \in \meaningof{E}\} }
\end{mathpar}

\begin{mathpar}
 \inferrule* [lab=nominal] {} {\meaningof{\quotep{E}} = \{ \quotep{P} \in \quotep{\pi} | P \in \meaningof{E} \}, \and \meaningof{\quotep{P}} = \{ \quotep{Q} \in \quotep{\pi} | P \equiv Q \} \and \\ \meaningof{@\quotep{E}} = \{ P \in \pi | P \equiv @x, x \in \meaningof{E} \}}
\end{mathpar}

\begin{eqnarray*}
  \\
  \meaningof{-} : TS \to ST
\end{eqnarray*}

\begin{eqnarray*}
  \\
  L : TS \to ST
\end{eqnarray*}

\begin{eqnarray*}
  \\
  P \models E \iff P \in \meaningof{E}
\end{eqnarray*}

\begin{eqnarray*}
  P \approx_{L} Q \iff \forall E \in L. P \models E \iff Q \models E
\end{eqnarray*}

\begin{eqnarray*}
  P \approx_{K} Q
\end{eqnarray*}

\begin{eqnarray*}
  P \approx Q
\end{eqnarray*}

$\approx_{K} = \approx = \approx_{L}$

\subsubsection{Contextual duality}

Note that contexts extend the quotation operation to a family of
operations from processes to names. Given a context, $M$, we can
define a \emph{nominal context}, $\quotep{M}$ by $\quotep{M}[P] :=
\quotep{M[P]}$. To foreshadow what is to come we observe that these
operations enjoy a duality with processes very much like the duality
between vectors and maps from vectors to scalars.

Further, because the calculus is essentially higher-order, we have a
correspondence between contexts and processes. More specifically,
given a name $x$ and a context $M$ we can construct $M^{*}_{x}$ such
that 

\begin{mathpar}
  M^{*}_{x} | \lift{x}{P} \red M[P]
\end{mathpar}

namely,

\begin{mathpar}
  M^{*}_{x} := x?(u).M[\dropn{u}]
\end{mathpar}

The dependence of $M^{*}_{x}$ on a name makes it an abstraction, 

\begin{mathpar}
  M^{*} := (x)x?(u).M[\dropn{u}]
\end{mathpar}

\subsection{Additional notation}

It will sometimes be convenient to denote the process a name
quotes. We already have the notation $x = \quotep{P}$, but it will be
convenient to introduce an alternate notation, $\procn{x}$, when we
want to emphasize the connection to the use of the name. Note that, by
virtue of name equivalence, $\quotep{\procn{x}} \nameeq x$; so, the
notation is consistent with previous definitions.

Further, because names have structure it is possible to effect
substitutions on the basis of that structure. This means we need to
upgrade our notation for substitutions, which we accomplish by
adapting comprehension notation. Thus,

\begin{mathpar}
  P\{ y / x : x \in S \}
\end{mathpar}

is interpreted to mean the process derived from P by replacing (in a
capture-avoiding manner) each occurrence of $x$ in $S$ by $y$. For example,

\begin{mathpar}
  P\{ \quotep{\procn{x}|\procn{x}} / x : x \in \freenames{P} \}
\end{mathpar}

will replace each (occurrence) of a free name $x$ in $P$ by
$\quotep{\procn{x}|\procn{x}}$.

Also, we will avail ourselves of the notation $x^{L}$ and $x^{R}$ to
denote injections of a name into disjoint copies of the name
space. There are numerous ways to accomplish this. One example can be
found in \cite{MeredithR05}. This notation overloads to vectors of
names: $\vec{x}^{\pi} := (x_{i}^{\pi} \; : \; 0 \leq i < |\vec{x}| )$ where $\pi \in \{L,R\}$.

We also use $P^{\Box} := P|\Box$.

In \cite{MeredithR05} an interpretation of the new operator is
given. It turns out that there are several possible interpretations
all enjoying the requisite algebraic properties of the operator (see
\cite{milner91polyadicpi}). We will therefore make liberal use of
$(\nu\; \vec{x})P$.

% subsection the_syntax_and_semantics_of_the_notation_system (end)   

\section{Interpretation of QM}
\subsection{Supporting definitions}
\subsubsection{Multiplication}
\begin{mathpar}
  \quotep{Q} \cdot \quotep{R} := \quotep{Q|R}
  \and \\
  \quotep{Q} \cdot P := P\{ \quotep{Q|R} / \quotep{R} : \quotep{R} \in \freenames{P} \}
\end{mathpar}

\paragraph{Discussion}
The first line needs little explanation. The second line says that
each free name of the process is replaced with the multiplication of
that name by the scalar. Multiplication of a scalar (name) by a state
(process) results in a process all the names of which have been `moved
over' by parallel composition with the process the scalar
quotes. There is a subtlety that the bound names have to be
manipulated so that multiplied names aren't accidentally
captured. There are many ways to achieve this.

\begin{remark}\label{rem:multiplication_identities}
  The reader is invited to verify that for all $x,y,z \in \QProc$ and $P \in \Proc$
  \begin{mathpar}
    x \cdot \quotep{0} \equiv x 
    \and
    x \cdot y \equiv y \cdot x
    \and
    x \cdot (y \cdot z) \equiv (x \cdot y) \cdot z
    \and \\
    \quotep{0} \cdot P \equiv P
    \and \\
    x \cdot (y \cdot P) \equiv (x \cdot y) \cdot P
    \and \\
    x \cdot (P|Q) \equiv (x \cdot P) | (x \cdot Q)
    \and \\    
  \end{mathpar}
\end{remark}

\subsubsection{Tensor product}

We define a tensor product on processes by structural induction.

\paragraph{Tensor of sums} First note that all summations, including
$\pzero$ and sequence, can be written $\Sigma_{i} x_{i}.A_{i} +
\Sigma_{j} x_{j}.C_{j}$, where we have grouped input-guarded processes
together and output-guarded processes together.

Thus, we can define the tensor product of two summations, $N_{1}\otimes N_{2}$, where

\begin{mathpar}
  N_{1} := \Sigma_{i} x_{i}.A_{i} + \Sigma_{j} x_{j}.C_{j}
  \and
  N_{2} := \Sigma_{i'} y_{i'}.B_{i'} + \Sigma_{j'} y_{j'}.D_{j'} 
\end{mathpar}

as follows.

\begin{mathpar}
  \Sigma_{i} x_{i}.A_{i} + \Sigma_{j} x_{j}.C_{j} \otimes \Sigma_{i'}
  y_{i'}.B_{i'} + \Sigma_{j'} y_{j'}.D_{j'} 
  \and \\
  := \; \Sigma_{i} \Sigma_{i'} \quotep{\stackrel{\vee}{x_{i}}| \stackrel{\vee}{y_{i'}}}.(A_{i}\otimes B_{i'}) \; | \; \Sigma_{i'} \Sigma_{i} \quotep{\stackrel{\vee}{y_{i'}}|\stackrel{\vee}{x_{i}}}.(B_{i'}\otimes A_{i})
  \and
  \;\; | \;\; \Sigma_{j} \Sigma_{j'} \quotep{\stackrel{\vee}{x_{j}}|\stackrel{\vee}{y_{j'}}}.(A_{j}\otimes B_{j'}) \; | \; \Sigma_{j'} \Sigma_{j} \quotep{\stackrel{\vee}{y_{j'}}|\stackrel{\vee}{x_{j}}}.(B_{j'}\otimes A_{j})
\end{mathpar}

\begin{remark}
  Do we need to $x^{L}$ and $y^{R}$ for this construction as well?
\end{remark}

\paragraph{Tensor of parallel compositions} Next, we distribute tensor
over par.

\begin{mathpar}
  P_{1}|P_{2} \otimes Q_{1}|Q_{2} := (P_{1} \otimes Q_{1}) | (P_{1}
  \otimes Q_{2}) | (P_{2} \otimes Q_{1}) | (P_{2} \otimes Q_{2})
\end{mathpar}

\paragraph{Tensor with dropped names} We treat tensor of a
process with a dropped name as parallel composition.

\begin{mathpar}
  P \otimes \dropn{x} := P | \dropn{x}
\end{mathpar}

\paragraph{Tensor of agents}

Finally, we need to define tensor on agents. Note that the definition
of tensor on normal products only tensors inputs with inputs and
outputs with outputs. Thus, we only have to define the operation on
``homogeneous'' pairings.

\begin{mathpar}
  (\vec{x})P \otimes (\vec{y})Q
  \and \\
  := (x_{0}^{L}|y_{0}^{R},\ldots,x_{0}^{L}|y_{n}^{R},\ldots,x_{m}^{L}|y_{0}^{R},\ldots,x_{m}^{L}|y_{n}^R)(P\{ \vec{x}^{L}/\vec{x}\} \otimes Q \{ \vec{y}^{R}/\vec{y}\})
  \and \\
  \clift{\vec{P}} \otimes \clift{\vec{Q}}
  \and \\
  := \clift{P_{0}\otimes Q_{0},\ldots,P_{0}\otimes Q_{n},\ldots,P_{m}\otimes Q_{0},\ldots,P_{m}\otimes Q_{n}}
\end{mathpar}

\begin{remark}
  Observe that arities of tensored abstractions matches arities of
  tensored concretions if the original arities matched. Note also that
  the length of the arities corresponds to the increase in dimension
  we see in ordinary vector space tensor product.
\end{remark}

\begin{remark}
  Operationally, this definition distributes the tensor down to
  components ``linked'' by summation. Tensor over summation is
  intriguing in that it mixes names. Moreover, as a consequence of the
  way it mixes names we have the identities for all $x \in \QProc$ and
  $P,Q \in \Proc$

  \begin{mathpar}
    (x \cdot P) \otimes Q \equiv x \cdot (P \otimes Q) \equiv P \otimes (x \cdot Q)
    \and
    P \otimes \pzero \equiv P
  \end{mathpar}

  that the reader is invited to verify.
\end{remark}

\subsubsection{Annihilation}
\begin{mathpar}
  P^{\perp} := \{ Q | \forall R. P|Q \red^{*} R \Rightarrow R \red^{*} \pzero \}
  \and \\
  P^{\underline{\perp}} := \Sigma_{Q \in P^{\perp}} \quotep{Q}?(y).(\dropn{y}|Q) | \Sigma_{Q \in P^{\perp}} \quotep{Q}\clift{\Box}
\end{mathpar}

\paragraph{Discussion} The reader will note that $P^{\perp}$ is a
\emph{set} of processes, while $P^{\underline{\perp}}$ is a
\emph{context}. We call the set $P^{\perp}$ the \emph{annihilators} of
$P$. The parallel composition of a process in the annihilators of $P$
with $P$ will result in a process, the state space of which has all
paths eventually leading to $\pzero$. Execution may endure loops; but
under reasonable conditions of fairness (naturally guaranteed under
most notions of bisimulation) such a composite process cannot get
stuck in such a loop and will, eventually pop out and terminate.

The context $P^{\underline{\perp}}$ is ready and willing to ``take the
$P$ out of'' the process to which it is applied. It will effectively
transmit the code of the process to which it is applied to one of the
annihilators and run the process against it.

\subsubsection{Evaluation}
We fix $M$ a domain of fully abstract interpretation with an equality
coincident with bisimulation. We take $\meaningof{\cdot} : \Proc \to
M$ to be the map interpreting processes and $\nmeaningof{\cdot} : \M
\to Proc$ to be the map running the other way. Then we define

\begin{mathpar}
  \int P := \nmeaningof{\meaningof{P}}
\end{mathpar}

\paragraph{Discussion}
There are many fully abstract interpretations of Milner's
$\pi$-calculus. Any of them can be used as a basis for interpreting
the reflective calculus here. Equipped with such a domain it is
largely a matter of grinding through to check that the Yoneda
construction for the normalization-by-evaluation program can be
extended to this setting.

\begin{remark}
  The reader is invited to verify that $\int (P^{\underline{\perp}}[P]) = 0$.
\end{remark}

\subsection{Quantum mechanics}

Table \ref{tbl:core_qm_op_defns} gives the core operational definitions

\begin{table}[htp]\label{tbl:core_qm_op_defns}
  \center{
    \fbox{
      \begin{tabular}{c|c}
        quantum mechanics & process calculus \\
        \hline
        scalar & $x := \quotep{P}$ \\
        state vector & $\state{P} := P$ \\
        dual & $\state{P}^{*} := \event{P^{\underline{\perp}}} := \quotep{P^{\underline{\perp}}}[-]$ \\
        matrix & $ \Sigma_{\alpha} \state{P_{\alpha}}x_{\alpha}\event{Q_{\alpha}}$ \\
        vector addition & $\state{P} + \state{Q} := \state{P | Q}$ \\
        tensor product & $\state{P} \otimes \state{Q} := \state{P \otimes Q}$ \\
        inner product & $\innerprod{P}{Q} := \quotep{\int P^{\underline{\perp}}[Q]}$ \\
      \end{tabular}
    }
  }
  \caption{QM - operational definitions}
\end{table}

where

\begin{mathpar}
  \prmatrix{P}{Q} := \fprmatrix{P}{\quotep{\pzero}}{Q}
  \and
  \fprmatrix{P}{x}{Q} := (\state{P},x,\event{Q})
  \and
  (\fprmatrix{P}{x}{Q})(\state{R}) := x \cdot \innerprod{Q}{R} \cdot \state{P}
  \and
  (\fprmatrix{P}{x}{Q})(\event{R}) := x \cdot \innerprod{R}{P} \cdot \event{Q}
\end{mathpar}

\paragraph{Discussion}
As promised: vectors (aka states) are represented as processes; duals
as contextual duals; inner product definition should be compared with
standard inner product definition for ....

\begin{remark}
  Assuming $\int (P^{\underline{\perp}}[P]) = 0$, the reader is
  invited to verify that $(\fprmatrix{P}{x}{P})(\state{P}) = x \cdot \state{P}$.
\end{remark}

\begin{remark}
  The reader is invited to verify that $\innerprod{P}{Q}$ could
  equally well have been written $\quotep{\int \stackrel{\vee}{x}}$
  where $x = \event{P^{\underline{\perp}}}(Q)$.

  One of the motivations for this remark is that there is another way
  to factor these operations. We could package up evaluation in the dual:

  \begin{mathpar}
    \state{P}^{*} := \event{\int P^{\underline{\perp}}} := \quotep{\int P^{\underline{\perp}}}[-]
  \end{mathpar}

  and then have inner product defined by
  
  \begin{mathpar}
    \innerprod{P}{Q} := \event{P}(Q)
  \end{mathpar}

  Hopefully, experience with the calculations will provide guidance on
  the best factoring.
\end{remark}

\begin{remark}
  Assuming $\int (P^{\underline{\perp}}[P]) = 0$, the reader is
  invited to verify that $\forall P,Q. (\prmatrix{0}{Q})(\state{0}) =
  \state{0}$ and dually $(\prmatrix{P}{0})(\event{0}) = \event{0}$.
\end{remark}

\begin{remark}
  i'm a little worried that i don't (yet) have proper support for
  complex conjugacy. But, the observation above may give us a
  clue. According to Abramsky, it must be the case that the scalars
  are iso to the homset of the identity for the tensor -- which the
  observation above characterizes. 

  For now, we will simply bookmark the notion with $\overline{x}$.
\end{remark}

\subsubsection{Adjointness}

We need to give a definition of $(\cdot)^{\dagger}$ for matrices. The
obvious candidate definition is
\begin{mathpar}
(\Sigma_{\alpha}\fprmatrix{P_{\alpha}}{x_{\alpha}}{Q_{\alpha}})^{\dagger}
= \Sigma_{\alpha}\fprmatrix{(Q_{\alpha}^{\underline{\perp}})^{*}}{\overline{x}_{\alpha}}{P_{\alpha}^{\underline{\perp}}} 
\end{mathpar}

But, $(Q_{\alpha}^{\underline{\perp}})^{*}$ requires a name along
which to communicate the process to achieve the context application.

\subsubsection{Basis for a basis}
If processes label states and ``addition'' of states (a.k.a. vector
addition) is interpreted as parallel composition, what corresponds to
notions of linear independence and basis? Here, we recall that Yoshida
has developed a set of \emph{combinators} for an asynchronous verison
of Milner's $\pi$-calculus. These are a finite set of processes such
any process can be expressed as parallel composition of these
combinators together with liberal uses of the new operator and
replication. We can simply give a translation of these into the
present calculus and have reasonable expectation that the property
carries over. That is, that the resultant set allows to express all
processes via parallel composition. Note, however, that there is no
new operator or replication in this calculus. As a result, we expect
that the corresponding set is actually infinite. That is, we expect
that the space is actually infinite dimensional.

\begin{remark}
  The attentive reader may be a bit concerned. Certainly, the
  collection $S$, $K$ and $I$ is a finite set of
  combinators. Shouldn't we expect to see a finite set of combinators
  for an effectively equivalent system? i am very sympathetic to this
  critique and feel it warrants full attention. On the other hand, i
  also have in mind the following analogy. The natural numbers, as a
  monoid under addition, has exactly $1$ generator, while the natural
  numbers, as a monoid under multiplication, has countably many
  generators (the primes). We observe that the application of the
  lambda calculus is much less resource sensitive than the parallel
  composition of the $\pi$-calculus. Could it be the case that we have
  an analogy of the form
  
  \begin{mathpar}
    m + n : MN :: m*n : M|N
  \end{mathpar}

  giving a similar blow up in the set of ``primes''?  This is such a
  wonderful thought that, even if it's not true, i think it's worth
  writing down.
\end{remark}
 

\documentclass[12pt]{llncs}
%\documentclass{jktr}

\usepackage[pdftex]{hyperref}                   
\usepackage {listings}
\usepackage {mathpartir}
\usepackage{bcprules}
%\usepackage{listings}
                       
\usepackage{graphicx} 
%\usepackage[margins=2.5cm,nohead,nofoot]{geometry}
%\usepackage{geometry}
\usepackage{amsfonts}
\usepackage{amstext}
\usepackage{latexsym}
\usepackage{amssymb}
\usepackage{color}


%\include{myPreamble}
\include{qm2pi.local} 

%\ifpdf
%\usepackage[pdftex]{graphicx}
%\else
%\usepackage{graphicx}
%\fi

 % \ifpdf
%  \usepackage{pdfsync}
%  \if


%\title{Brief Article}
%\author{David F. Snyder}
%\author{L.G. Meredith}

%\address{Dept. of Math., Texas State University--San Marcos, San Marcos, TX 78666}
       
\pagestyle{empty}


\begin{document}

\lstset{language=[Objective]Caml,frame=shadowbox}

\input{qm2pi.front}

% section front matter (end)

\input{qm2pi.intro} 
 
% section introduction (end)

% \input{qm2pi.knotations} 

% section notation (end)

\input{qm2pi.process.calculi} 

% section concurrent_process_calculi_and_spatial_logics_ (end)
    
%\input{qm2pi.knots2pi} 

%\input{qm2pi.trefoil} 

%\input{qm2pi.mainthm} 

% subsection basic_interpretation (end)

%\input{qm2pi.rho.presentation} 
\subsection{The syntax and semantics of the notation system}\label{sub:the_syntax_and_semantics_of_the_notation_system} % (fold)

We now summarize a technical presentation of the calculus that
embodies our theory of dynamics. The typical presentation of such a
calculus follows the style of giving generators and relations on
them. The grammar, below, describing term constructors, freely
generates the set of processes, $\Proc$. This set is then quotiented
by a relation known as structural congruence and it is over this set
that the notion of dynamics is expressed. This presentation is
essentially that of \cite{MeredithR05} with the addition of
polyadicity and summation. For readability we have relegated some of
the technical subtleties to an appendix.

\subsubsection{Process grammar}\label{subsub:process_grammar}

\begin{mathpar}
  \inferrule* [lab=synchronization] {} {{M} \bc \pzero \;|\; x?F \;|\; x!C }
  \and
  \inferrule* [lab=abstraction] {} {{F} \bc (x)P}
  \and
  \inferrule* [lab=concretion] {} {{C} \bc \langle Q \rangle}
  \and
  \inferrule* [lab=process] {} {{P,Q} \bc M \;| \;P|Q \;|\; @{x}}
  \and
  \inferrule* [lab=name] {} {{x} \bc \quotep{P}}
\end{mathpar} 

Note that $\vec{x}$ (resp. $\vec{P}$) denotes a vector of names
(resp. processes) of length $|\vec{x}|$ (resp. $|\vec{P}|$). We adopt
the following useful abbreviations.

\begin{mathpar}
   x?(\vec{y}).P := x.(\vec{y})P \and  x\clift{\vec{P}} := x.\clift{\vec{P}}
   \and x!(y) := \lift{x}{\dropn{y}}
   \and \Pi_{i=0}^{n-1}P_i := P_0 | \ldots | P_{n-1}
\end{mathpar}

\subsubsection{Structural congruence}

\paragraph{Free and bound names and alpha-equivalence.} At the
core of structural equivalence is alpha-equivalence which identifies
process that are the same up to a change of variable. Formally, we
recognize the distinction between free and bound names. The free names
of a process, $\freenames{P}$, may be calculated recursively as
follows:

\begin{mathpar}
\freenames{\pzero} := \emptyset
  \and \\
  \freenames{x?(y).P} := \{ x \} \cup (\freenames{P} \setminus \{ y \})
  \and 
  \freenames{x!\langle P \rangle} := \{ x \} \cup \{ P \} 
  \and \\
  \freenames{P|Q} := \freenames{P} \cup \freenames{Q}
  \and \\
  \freenames{@{x}} := \{ x \}
\end{mathpar}

$\pi$
$\quotep{\pi}$

$\freenames{-} : \pi \to \mathcal{P}(\quotep{\pi})$

\begin{eqnarray*}
  \freenames{\pzero} & := & \emptyset \\
  \freenames{x?(y).P} & := & \{ x \} \cup (\freenames{P} \setminus \{ y \}) \\
  \freenames{x!\langle P \rangle} & := & \{ x \} \cup \{ P \} \\
  \freenames{P|Q} & := & \freenames{P} \cup \freenames{Q} \\
  \freenames{\dropn{x}} & := & \{ x \}
\end{eqnarray*}

The bound names of a process, $\boundnames{P}$, are those names occurring in $P$
that are not free. For example, in $x?(y).0$, the name $x$ is free, while $y$ is bound.

\begin{mathpar}
  \inferrule* [lab=monoidal-laws] {} { P|Q \equiv Q|P \and P|0 \equiv P \and P|(Q|R) \equiv (P|Q)|R }
\end{mathpar}

\begin{mathpar}
  \inferrule* [lab=alpha-equivalence] {} { (x)P \equiv (y)P\{y/x\} \and y \not\in \freenames{P} }
\end{mathpar}

\begin{definition}
Then two processes, $P,Q$, are alpha-equivalent if $P = Q\{\vec{y}/\vec{x}\}$ for
some $\vec{x} \in \boundnames{Q},\vec{y} \in \boundnames{P}$, where $Q\{\vec{y}/\vec{x}\}$
denotes the capture-avoiding substitution of $\vec{y}$ for $\vec{x}$ in $Q$.
\end{definition}

\begin{definition}
  The {\em structural congruence} \cite{SangiorgiWalker} , $\equiv$,
  between processes is the least congruence containing
  alpha-equivalence, satisfying the abelian monoid laws
  (associativity, commutativity and $\pzero$ as identity) for parallel
  composition $|$ and for summation $+$.
\end{definition}

\subsection{Name equivalence}

We take name equivalence, written $\nameeq$, to be the smallest
equivalence relation generated by the following rules.

\begin{mathpar}
\inferrule*[lab=Quote-drop]
{ }
{ \quotep{@{x}} \nameeq x }

\inferrule*[lab=Struct-equiv]
{ P \scong Q }
{ \quotep{P} \nameeq \quotep{Q} }
\end{mathpar}

The astute reader will have noticed that the mutual recursion of names
and processes imposes a mutual recursion on alpha-equivalence and
structural equivalence via name-equivalence. Fortunately, all of this
works out pleasantly and we may calculate in the natural way, free of
concern. The reader interested in the details is referred to the
appendix \ref{appendix:rho_details}.

\subsection{Substitution}

We use $\Proc$ for the set of processes, $\QProc$ for the set of
names, and $\id{\{}\vec{y} / \vec{x} \id{\}}$ to denote partial maps,
$s : \QProc \rightarrow \QProc$. A map, $s$ lifts, uniquely, to a map
on process terms, $\widehat{s} : \Proc \rightarrow \Proc$ by the
following equations.

\begin{mathpar}
  (0) \psubstp{Q}{P} := 0 \\
  (R \juxtap S) \psubstp{Q}{P}
  :=    
  (R)\psubstp{Q}{P} \juxtap (S) \psubstp{Q}{P} \\
  (x?(y).R) \psubstp{Q}{P}    
  :=    
  (x)\substp{Q}{P} (z)\concat( (R \psubstn{z}{y}) \psubstp{Q}{P} ) \\
  (\lift{x}{R}) \psubstp{Q}{P}  
  :=
  \lift{(x)\substp{Q}{P}}{ R \psubstp{Q}{P} } \\
%   (\dropn{x})  \psubstp{Q}{P}       
%   := 
%   \left\{ 
%     \begin{array}{ccc} 
%       \dropn{\quotep{Q}} & & x \nameeq \quotep{P} \\
%       \dropn{x} & & otherwise \\
%     \end{array}
%   \right. 
  (\dropn{x})  \psubstp{Q}{P}       
  := 
  \left\{ 
    \begin{array}{ccc} 
      Q & & x \nameeq \quotep{P} \\
      \dropn{x} & & otherwise \\
    \end{array}
  \right.
\end{mathpar}
 

where

\begin{eqnarray}
  (x)\id{\{} \lpquote Q \rpquote / \lpquote P \rpquote \id{\}}            = 
  \left\{ 
    \begin{array}{ccc}
      \lpquote Q \rpquote & & x \nameeq \lpquote P \rpquote \\
      x & & otherwise \\
    \end{array}
  \right. \nonumber
\end{eqnarray}

and $z$ is chosen distinct from $\quotep{P}$, $\quotep{Q}$, the free
names in $Q$, and all the names in $R$. Our $\alpha$-equivalence will
be built in the standard way from this substitution.

\begin{remark}\label{rem:no_self_referential_names}
  One consequence of these definitions is that $\forall P. \quotep{P}
  \not\in \freenames{P}$.
\end{remark}

\subsection{ Dynamic quote: an example }

Anticipating something of what's to come, consider applying the
substitution, $\widehat{\id{\{}u / z \id{\}}}$, to the following pair
of processes, $\lift{w}{y!(z)}$ and $w[ \lpquote y!(z) \rpquote ]$.

\begin{eqnarray}
	\lift{w}{y!(z)}\widehat{\id{\{}u / z \id{\}}}
		& = &
		\lift{w}{y!(u)} \nonumber\\
	w[ \lpquote y!(z) \rpquote ] \widehat{ \id{\{}u / z \id{\}} }
		& = &
		w[ \lpquote y!(z) \rpquote ] \nonumber
\end{eqnarray}

Because the body of the process between quotes is impervious to
substitution, we get radically different answers. In fact, by
examining the first process in an input context,
e.g. $x?(z).\lift{w}{y!(z)}$, we see that the process under the lift
operator may be shaped by prefixed inputs binding a name inside it. In
this sense, the lift operator will be seen as a way to dynamically
construct processes before reifying them as names.

Finally equipped with these standard features we can present the
dynamics of the calculus.

\subsubsection{Operational semantics} 

Finally, we introduce the computational dynamics. What marks these
algebras as distinct from other more traditionally studied algebraic
structures, e.g. vector spaces or polynomial rings, is the manner in
which dynamics is captured. In traditional structures, dynamics is typically
expressed through morphisms between such structures, as in linear maps
between vector spaces or morphisms between rings. In algebras
associated with the semantics of computation, the dynamics is
expressed as part of the algebraic structure itself, through a
reduction reduction relation typically denoted by $\red$. Below, we
give a recursive presentation of this relation for the calculus used
in the encoding.

$\red \subseteq \pi \times \pi$
$\red : \pi \to \mathcal{P}(\pi)$

\begin{mathpar}
  \inferrule* [lab=Comm] { \textsf{match}( x_{src}, x_{trgt} ) } { x_{trgt}?(y)P \; | \; x_{src}!\langle {Q} \rangle \red P\{\quotep{Q}/y}\} }
  \and \\
  \inferrule* [lab=Par] {{P} \red {P}'} {{{P} | {Q}} \red {{P}' | {Q}}}
  \and
  \inferrule* [lab=Equiv]{{{P} \scong {P}'} \andalso {{P}' \red {Q}'} \andalso {{Q}' \scong {Q}}}{{P} \red {Q}}
\end{mathpar}

\begin{eqnarray*}
  match_{\equiv} (\quotep{P},\quotep{Q}) & := & P \equiv Q \\
  match_{\dagger}(\quotep{P},\quotep{Q}) & := & \forall R. P|Q \red^{*} R => R \red^{*} 0 \\
  match_{K}(\quotep{P},\quotep{Q}) & := & K \mbox{ for some context } K
\end{eqnarray*}

$u?(x)P | u!\langle Q \rangle \red P\{\quotep{Q}/x\}$

%We write $\wred$ for $\red^*$, and $P\red$ if $\exists Q $ such that $ P \red Q$.
We write $P\red$ if $\exists Q $ such that $ P \red Q$ and $P\not\red$, otherwise.

\section{Replication}

As mentioned before, it is known that replication (and hence
recursion) can be implemented in a higher-order process algebra
\cite{SangiorgiWalker}. As our first example of calculation with the
machinery thus far presented we give the construction explicitly in
the {\rhoc}.

\begin{eqnarray}
	D_{x} & := & \prefix{x}{y}{(\binpar{\outputp{x}{y}}{@{y}})} \nonumber\\
	\bangp_{x}{P} & := & \binpar{{x}!\langle{\binpar{D_{x}}{P}}\rangle}{D_{x}} \nonumber
\end{eqnarray}

\begin{eqnarray}
	\bangp_{x}{P} & & \nonumber\\
	=
	& {x}!\langle{(\prefix{x}{y}{(\outputp{x}{y} | @{y})) | P}}\rangle 
	      | \prefix{x}{y}{(\outputp{x}{y} | @{y})} & \nonumber\\
	\red
	& (\outputp{x}{y} | @{y})\substn{\quotep{(\prefix{x}{y}{(@{y} | \outputp{x}{y})) | P}}}{y} & \nonumber\\
	=
	& \outputp{x}{\quotep{(\prefix{x}{y}{(\outputp{x}{y} | @{y})) | P}}}
	  | {(\prefix{x}{y}{(\outputp{x}{y} | @{y})) | P}} & \nonumber\\
	\red
	& \ldots & \nonumber\\
	\red^*
	& P | P | \ldots & \nonumber
\end{eqnarray}

Of course, this encoding, as an implementation, runs away, unfolding
$\bangp{P}$ eagerly. A lazier and more implementable replication
operator, restricted to input-guarded processes, may be obtained as follows.

\begin{eqnarray}
\bangp{\prefix{u}{v}{P}} 
	:= 
	\binpar{\lift{x}{\prefix{u}{v}{(\binpar{D(x)}{P})}}}{D(x)} \nonumber
\end{eqnarray}

\begin{remark}
  Note that the lazier definition still does not deal with summation
  or mixed summation (i.e. sums over input and output). The reader is
  invited to construct definitions of replication that deal with these
  features. 

  Further, the definitions are parameterized in a name, $x$. Can you,
  gentle reader, make a definition that eliminates this parameter and
  guarantees no accidental interaction between the replication
  machinery and the process being replicated -- i.e. no accidental
  sharing of names used by the process to get its work done and the
  name(s) used by the replication to effect copying. This latter
  revision of the definition of replication is crucial to obtaining
  the expected identity $!!P \sim !P$.
\end{remark}

\begin{remark}\label{rem:paradoxical_combinator}
  The reader familiar with the lambda calculus will have noticed the
  similarity between $D$ and the paradoxical combinator.

  [Ed. note: the existence of this seems to suggest we have to be more
  restrictive on the set of processes and names we admit if we are to
  support no-cloning.]
\end{remark}

\subsubsection{Bisimulation}

The computational dynamics gives rise to another kind of equivalence,
the equivalence of computational behavior. As previously mentioned
this is typically captured \emph{via} some form of bisimulation.

% The notion we use in this paper is weak barbed bisimulation
% \cite{milner91polyadicpi}.

The notion we use in this paper is derived from weak barbed
bisimulation \cite{milner91polyadicpi}. 

\begin{definition}
An \emph{observation relation}, $\downarrow_{\mathcal N}$, over a set
of names, $\mathcal N$, is the smallest relation satisfying the rules
below.

\infrule[Out-barb]{y \in {\mathcal N}, \; x \nameeq y}
		  {\outputp{x}{v} \downarrow_{\mathcal N} x}
\infrule[Par-barb]{\mbox{$P\downarrow_{\mathcal N} x$ or $Q\downarrow_{\mathcal N} x$}}
		  {\binpar{P}{Q} \downarrow_{\mathcal N} x}

We write $P \Downarrow_{\mathcal N} x$ if there is $Q$ such that 
$P \wred Q$ and $Q \downarrow_{\mathcal N} x$.
\end{definition}

\begin{definition}
%\label{def.bbisim}
An  ${\mathcal N}$-\emph{barbed bisimulation} over a set of names, ${\mathcal N}$, is a symmetric binary relation 
${\mathcal S}_{\mathcal N}$ between agents such that $P\rel{S}_{\mathcal N}Q$ implies:
\begin{enumerate}
\item If $P \red P'$ then $Q \wred Q'$ and $P'\rel{S}_{\mathcal N} Q'$.
\item If $P\downarrow_{\mathcal N} x$, then $Q\Downarrow_{\mathcal N} x$.
\end{enumerate}
$P$ is ${\mathcal N}$-barbed bisimilar to $Q$, written
$P \wbbisim_{\mathcal N} Q$, if $P \rel{S}_{\mathcal N} Q$ for some ${\mathcal N}$-barbed bisimulation ${\mathcal S}_{\mathcal N}$.
\end{definition}

$\mathcal{R} \subseteq \pi \times \pi$

$P \mathcal{R} Q => \forall P'. P \red P' \Rightarrow \exists Q'. Q \red Q', P' \mathcal{R} Q'$

$P \vdash x \Rightarrow Q \vdash x$

\begin{mathpar}
  \inferrule*[lab=Out-barb]{x \nameeq y}{{y}!\langle{Q}\rangle \vdash x}
  \and
  \inferrule*[lab=Par-barb]{\mbox{$P\vdash x$ or $Q\vdash x$}}{\binpar{P}{Q} \vdash x}
\end{mathpar}

\subsubsection{Contexts}

One of the principle advantages of computational calculi like the
$\pi$-calculus is a well-defined notion of context,
contextual-equivalence and a correlation between
contextual-equivalence and notions of bisimulation. The notion of
context allows the decomposition of a process into (sub-)process and
its syntactic environment, its context. Thus, a context may be
thought of as a process with a ``hole'' (written $\Box$) in it. The
application of a context $M$ to a process $P$, written $M[P]$, is
tantamount to filling the hole in $M$ with $P$. In this paper we do
not need the full weight of this theory, but do make use of the notion
of context in the proof the main theorem. 

\begin{mathpar}
  \inferrule* [lab=summation] {} {{M_{M},M_{N}} \bc \Box \;|\; x.M_{A} \;|\; M_{M}+M_{N}}
  \and
  \inferrule* [lab=agent] {} {{M_{A}} \bc (\vec{x})M_{P} \;| \; \clift{P_0,\ldots,M_{P},\ldots,P_N}}
  \and \\
  \inferrule* [lab=process] {} {{M_{P}} \bc M_{N} \;| \;P|M_{P} }
\end{mathpar} 

\begin{mathpar}
  \inferrule* [lab=sychronization] {} {M_{N} \bc \Box \;|\; x?M_{F} \;|\; x!M_{C}}
  \and
  \inferrule* [lab=abstraction] {} {{M_{F}} \bc (x)M_{P} }
  \and
  \inferrule* [lab=concretion] {} {{M_{C}} \bc \langle M_{P} \rangle }
  \and \\
  \inferrule* [lab=process] {} {{M_{P}} \bc M_{N} \;| \;P|M_{P} }
\end{mathpar}

\begin{definition}[contextual application] Given a context $M$, and
  process $P$, we define the \emph{contextual application}, $M[P] :=
  M\{P/\Box\}$. That is, the contextual application of M to P is the
  substitution of $P$ for $\Box$ in $M$.
\end{definition}

$\meaningof{-} : L \to \mathcal{P}(\pi)$

\begin{mathpar}
  \inferrule* [lab=collection] {} {\meaningof{true} = \pi, \and \meaningof{~E} = \pi \setminus \meaningof{E}, \and \meaningof{E_{1} \& E_{2}} = \meaningof{E_{1}} \cap \meaningof{E_{2}}}
\end{mathpar}

\begin{mathpar}
  \inferrule* [lab=structure] {} {\meaningof{0} = \{ P \in \pi | P \equiv 0 \}, \and \\ \meaningof{E_1 | E_2} = \{ P \in \pi | P \equiv P_{1} | P_{2}, P_{1} \in \meaningof{E_{1}}, P_{2} \in \meaningof{E_2}\} }
\end{mathpar}

\begin{mathpar}
 \inferrule* [lab=behavior] {} {\meaningof{\langle a?b \rangle E} = \{ P \in \pi | P \equiv Q | u?(y)P', \\ \and \\\\ \and \\ \;\;\; u \in \meaningof{a}, \forall z.P'\{z/y\} \in \meaningof{E\{z/b\}}\}, \and \\ \meaningof{a!E} = \{ P \in \pi | P \equiv Q | x!\langle P' \rangle, x \in \meaningof{a} P' \in \meaningof{E}\} }
\end{mathpar}

\begin{mathpar}
 \inferrule* [lab=nominal] {} {\meaningof{\quotep{E}} = \{ \quotep{P} \in \quotep{\pi} | P \in \meaningof{E} \}, \and \meaningof{\quotep{P}} = \{ \quotep{Q} \in \quotep{\pi} | P \equiv Q \} \and \\ \meaningof{@\quotep{E}} = \{ P \in \pi | P \equiv @x, x \in \meaningof{E} \}}
\end{mathpar}

\begin{eqnarray*}
  \\
  \meaningof{-} : TS \to ST
\end{eqnarray*}

\begin{eqnarray*}
  \\
  L : TS \to ST
\end{eqnarray*}

\begin{eqnarray*}
  \\
  P \models E \iff P \in \meaningof{E}
\end{eqnarray*}

\begin{eqnarray*}
  P \approx_{L} Q \iff \forall E \in L. P \models E \iff Q \models E
\end{eqnarray*}

\begin{eqnarray*}
  P \approx_{K} Q
\end{eqnarray*}

\begin{eqnarray*}
  P \approx Q
\end{eqnarray*}

$\approx_{K} = \approx = \approx_{L}$

\subsubsection{Contextual duality}

Note that contexts extend the quotation operation to a family of
operations from processes to names. Given a context, $M$, we can
define a \emph{nominal context}, $\quotep{M}$ by $\quotep{M}[P] :=
\quotep{M[P]}$. To foreshadow what is to come we observe that these
operations enjoy a duality with processes very much like the duality
between vectors and maps from vectors to scalars.

Further, because the calculus is essentially higher-order, we have a
correspondence between contexts and processes. More specifically,
given a name $x$ and a context $M$ we can construct $M^{*}_{x}$ such
that 

\begin{mathpar}
  M^{*}_{x} | \lift{x}{P} \red M[P]
\end{mathpar}

namely,

\begin{mathpar}
  M^{*}_{x} := x?(u).M[\dropn{u}]
\end{mathpar}

The dependence of $M^{*}_{x}$ on a name makes it an abstraction, 

\begin{mathpar}
  M^{*} := (x)x?(u).M[\dropn{u}]
\end{mathpar}

\subsection{Additional notation}

It will sometimes be convenient to denote the process a name
quotes. We already have the notation $x = \quotep{P}$, but it will be
convenient to introduce an alternate notation, $\procn{x}$, when we
want to emphasize the connection to the use of the name. Note that, by
virtue of name equivalence, $\quotep{\procn{x}} \nameeq x$; so, the
notation is consistent with previous definitions.

Further, because names have structure it is possible to effect
substitutions on the basis of that structure. This means we need to
upgrade our notation for substitutions, which we accomplish by
adapting comprehension notation. Thus,

\begin{mathpar}
  P\{ y / x : x \in S \}
\end{mathpar}

is interpreted to mean the process derived from P by replacing (in a
capture-avoiding manner) each occurrence of $x$ in $S$ by $y$. For example,

\begin{mathpar}
  P\{ \quotep{\procn{x}|\procn{x}} / x : x \in \freenames{P} \}
\end{mathpar}

will replace each (occurrence) of a free name $x$ in $P$ by
$\quotep{\procn{x}|\procn{x}}$.

Also, we will avail ourselves of the notation $x^{L}$ and $x^{R}$ to
denote injections of a name into disjoint copies of the name
space. There are numerous ways to accomplish this. One example can be
found in \cite{MeredithR05}. This notation overloads to vectors of
names: $\vec{x}^{\pi} := (x_{i}^{\pi} \; : \; 0 \leq i < |\vec{x}| )$ where $\pi \in \{L,R\}$.

We also use $P^{\Box} := P|\Box$.

In \cite{MeredithR05} an interpretation of the new operator is
given. It turns out that there are several possible interpretations
all enjoying the requisite algebraic properties of the operator (see
\cite{milner91polyadicpi}). We will therefore make liberal use of
$(\nu\; \vec{x})P$.

% subsection the_syntax_and_semantics_of_the_notation_system (end)   

\input{qm2pi.qmops} 

\input{qm2pi.sterngerlach} 

\input{qm2pi.metric} 

% section concurrent_process_calculi (end)

%\input{qm2pi.proofsketch}

% section proof sketch (end)

%\input{qm2pi.slviaknots} 

% section spatial logic via knots (end)

\input{qm2pi.conclusion}

% section conclusion (end)

%\input{qm2pi.dtcodes} 

% section wiring algorithm (end)

\input{qm2pi.ack} 

% section acknowledgments (end)

\newpage


\bibliographystyle{plain}   
\bibliography{../../biblios/main.bib}

\input{qm2pi.rhodetails}

\end{document}

 

\documentclass[12pt]{llncs}
%\documentclass{jktr}

\usepackage[pdftex]{hyperref}                   
\usepackage {listings}
\usepackage {mathpartir}
\usepackage{bcprules}
%\usepackage{listings}
                       
\usepackage{graphicx} 
%\usepackage[margins=2.5cm,nohead,nofoot]{geometry}
%\usepackage{geometry}
\usepackage{amsfonts}
\usepackage{amstext}
\usepackage{latexsym}
\usepackage{amssymb}
\usepackage{color}


%\include{myPreamble}
\include{qm2pi.local} 

%\ifpdf
%\usepackage[pdftex]{graphicx}
%\else
%\usepackage{graphicx}
%\fi

 % \ifpdf
%  \usepackage{pdfsync}
%  \if


%\title{Brief Article}
%\author{David F. Snyder}
%\author{L.G. Meredith}

%\address{Dept. of Math., Texas State University--San Marcos, San Marcos, TX 78666}
       
\pagestyle{empty}


\begin{document}

\lstset{language=[Objective]Caml,frame=shadowbox}

\input{qm2pi.front}

% section front matter (end)

\input{qm2pi.intro} 
 
% section introduction (end)

% \input{qm2pi.knotations} 

% section notation (end)

\input{qm2pi.process.calculi} 

% section concurrent_process_calculi_and_spatial_logics_ (end)
    
%\input{qm2pi.knots2pi} 

%\input{qm2pi.trefoil} 

%\input{qm2pi.mainthm} 

% subsection basic_interpretation (end)

%\input{qm2pi.rho.presentation} 
\subsection{The syntax and semantics of the notation system}\label{sub:the_syntax_and_semantics_of_the_notation_system} % (fold)

We now summarize a technical presentation of the calculus that
embodies our theory of dynamics. The typical presentation of such a
calculus follows the style of giving generators and relations on
them. The grammar, below, describing term constructors, freely
generates the set of processes, $\Proc$. This set is then quotiented
by a relation known as structural congruence and it is over this set
that the notion of dynamics is expressed. This presentation is
essentially that of \cite{MeredithR05} with the addition of
polyadicity and summation. For readability we have relegated some of
the technical subtleties to an appendix.

\subsubsection{Process grammar}\label{subsub:process_grammar}

\begin{mathpar}
  \inferrule* [lab=synchronization] {} {{M} \bc \pzero \;|\; x?F \;|\; x!C }
  \and
  \inferrule* [lab=abstraction] {} {{F} \bc (x)P}
  \and
  \inferrule* [lab=concretion] {} {{C} \bc \langle Q \rangle}
  \and
  \inferrule* [lab=process] {} {{P,Q} \bc M \;| \;P|Q \;|\; @{x}}
  \and
  \inferrule* [lab=name] {} {{x} \bc \quotep{P}}
\end{mathpar} 

Note that $\vec{x}$ (resp. $\vec{P}$) denotes a vector of names
(resp. processes) of length $|\vec{x}|$ (resp. $|\vec{P}|$). We adopt
the following useful abbreviations.

\begin{mathpar}
   x?(\vec{y}).P := x.(\vec{y})P \and  x\clift{\vec{P}} := x.\clift{\vec{P}}
   \and x!(y) := \lift{x}{\dropn{y}}
   \and \Pi_{i=0}^{n-1}P_i := P_0 | \ldots | P_{n-1}
\end{mathpar}

\subsubsection{Structural congruence}

\paragraph{Free and bound names and alpha-equivalence.} At the
core of structural equivalence is alpha-equivalence which identifies
process that are the same up to a change of variable. Formally, we
recognize the distinction between free and bound names. The free names
of a process, $\freenames{P}$, may be calculated recursively as
follows:

\begin{mathpar}
\freenames{\pzero} := \emptyset
  \and \\
  \freenames{x?(y).P} := \{ x \} \cup (\freenames{P} \setminus \{ y \})
  \and 
  \freenames{x!\langle P \rangle} := \{ x \} \cup \{ P \} 
  \and \\
  \freenames{P|Q} := \freenames{P} \cup \freenames{Q}
  \and \\
  \freenames{@{x}} := \{ x \}
\end{mathpar}

$\pi$
$\quotep{\pi}$

$\freenames{-} : \pi \to \mathcal{P}(\quotep{\pi})$

\begin{eqnarray*}
  \freenames{\pzero} & := & \emptyset \\
  \freenames{x?(y).P} & := & \{ x \} \cup (\freenames{P} \setminus \{ y \}) \\
  \freenames{x!\langle P \rangle} & := & \{ x \} \cup \{ P \} \\
  \freenames{P|Q} & := & \freenames{P} \cup \freenames{Q} \\
  \freenames{\dropn{x}} & := & \{ x \}
\end{eqnarray*}

The bound names of a process, $\boundnames{P}$, are those names occurring in $P$
that are not free. For example, in $x?(y).0$, the name $x$ is free, while $y$ is bound.

\begin{mathpar}
  \inferrule* [lab=monoidal-laws] {} { P|Q \equiv Q|P \and P|0 \equiv P \and P|(Q|R) \equiv (P|Q)|R }
\end{mathpar}

\begin{mathpar}
  \inferrule* [lab=alpha-equivalence] {} { (x)P \equiv (y)P\{y/x\} \and y \not\in \freenames{P} }
\end{mathpar}

\begin{definition}
Then two processes, $P,Q$, are alpha-equivalent if $P = Q\{\vec{y}/\vec{x}\}$ for
some $\vec{x} \in \boundnames{Q},\vec{y} \in \boundnames{P}$, where $Q\{\vec{y}/\vec{x}\}$
denotes the capture-avoiding substitution of $\vec{y}$ for $\vec{x}$ in $Q$.
\end{definition}

\begin{definition}
  The {\em structural congruence} \cite{SangiorgiWalker} , $\equiv$,
  between processes is the least congruence containing
  alpha-equivalence, satisfying the abelian monoid laws
  (associativity, commutativity and $\pzero$ as identity) for parallel
  composition $|$ and for summation $+$.
\end{definition}

\subsection{Name equivalence}

We take name equivalence, written $\nameeq$, to be the smallest
equivalence relation generated by the following rules.

\begin{mathpar}
\inferrule*[lab=Quote-drop]
{ }
{ \quotep{@{x}} \nameeq x }

\inferrule*[lab=Struct-equiv]
{ P \scong Q }
{ \quotep{P} \nameeq \quotep{Q} }
\end{mathpar}

The astute reader will have noticed that the mutual recursion of names
and processes imposes a mutual recursion on alpha-equivalence and
structural equivalence via name-equivalence. Fortunately, all of this
works out pleasantly and we may calculate in the natural way, free of
concern. The reader interested in the details is referred to the
appendix \ref{appendix:rho_details}.

\subsection{Substitution}

We use $\Proc$ for the set of processes, $\QProc$ for the set of
names, and $\id{\{}\vec{y} / \vec{x} \id{\}}$ to denote partial maps,
$s : \QProc \rightarrow \QProc$. A map, $s$ lifts, uniquely, to a map
on process terms, $\widehat{s} : \Proc \rightarrow \Proc$ by the
following equations.

\begin{mathpar}
  (0) \psubstp{Q}{P} := 0 \\
  (R \juxtap S) \psubstp{Q}{P}
  :=    
  (R)\psubstp{Q}{P} \juxtap (S) \psubstp{Q}{P} \\
  (x?(y).R) \psubstp{Q}{P}    
  :=    
  (x)\substp{Q}{P} (z)\concat( (R \psubstn{z}{y}) \psubstp{Q}{P} ) \\
  (\lift{x}{R}) \psubstp{Q}{P}  
  :=
  \lift{(x)\substp{Q}{P}}{ R \psubstp{Q}{P} } \\
%   (\dropn{x})  \psubstp{Q}{P}       
%   := 
%   \left\{ 
%     \begin{array}{ccc} 
%       \dropn{\quotep{Q}} & & x \nameeq \quotep{P} \\
%       \dropn{x} & & otherwise \\
%     \end{array}
%   \right. 
  (\dropn{x})  \psubstp{Q}{P}       
  := 
  \left\{ 
    \begin{array}{ccc} 
      Q & & x \nameeq \quotep{P} \\
      \dropn{x} & & otherwise \\
    \end{array}
  \right.
\end{mathpar}
 

where

\begin{eqnarray}
  (x)\id{\{} \lpquote Q \rpquote / \lpquote P \rpquote \id{\}}            = 
  \left\{ 
    \begin{array}{ccc}
      \lpquote Q \rpquote & & x \nameeq \lpquote P \rpquote \\
      x & & otherwise \\
    \end{array}
  \right. \nonumber
\end{eqnarray}

and $z$ is chosen distinct from $\quotep{P}$, $\quotep{Q}$, the free
names in $Q$, and all the names in $R$. Our $\alpha$-equivalence will
be built in the standard way from this substitution.

\begin{remark}\label{rem:no_self_referential_names}
  One consequence of these definitions is that $\forall P. \quotep{P}
  \not\in \freenames{P}$.
\end{remark}

\subsection{ Dynamic quote: an example }

Anticipating something of what's to come, consider applying the
substitution, $\widehat{\id{\{}u / z \id{\}}}$, to the following pair
of processes, $\lift{w}{y!(z)}$ and $w[ \lpquote y!(z) \rpquote ]$.

\begin{eqnarray}
	\lift{w}{y!(z)}\widehat{\id{\{}u / z \id{\}}}
		& = &
		\lift{w}{y!(u)} \nonumber\\
	w[ \lpquote y!(z) \rpquote ] \widehat{ \id{\{}u / z \id{\}} }
		& = &
		w[ \lpquote y!(z) \rpquote ] \nonumber
\end{eqnarray}

Because the body of the process between quotes is impervious to
substitution, we get radically different answers. In fact, by
examining the first process in an input context,
e.g. $x?(z).\lift{w}{y!(z)}$, we see that the process under the lift
operator may be shaped by prefixed inputs binding a name inside it. In
this sense, the lift operator will be seen as a way to dynamically
construct processes before reifying them as names.

Finally equipped with these standard features we can present the
dynamics of the calculus.

\subsubsection{Operational semantics} 

Finally, we introduce the computational dynamics. What marks these
algebras as distinct from other more traditionally studied algebraic
structures, e.g. vector spaces or polynomial rings, is the manner in
which dynamics is captured. In traditional structures, dynamics is typically
expressed through morphisms between such structures, as in linear maps
between vector spaces or morphisms between rings. In algebras
associated with the semantics of computation, the dynamics is
expressed as part of the algebraic structure itself, through a
reduction reduction relation typically denoted by $\red$. Below, we
give a recursive presentation of this relation for the calculus used
in the encoding.

$\red \subseteq \pi \times \pi$
$\red : \pi \to \mathcal{P}(\pi)$

\begin{mathpar}
  \inferrule* [lab=Comm] { \textsf{match}( x_{src}, x_{trgt} ) } { x_{trgt}?(y)P \; | \; x_{src}!\langle {Q} \rangle \red P\{\quotep{Q}/y}\} }
  \and \\
  \inferrule* [lab=Par] {{P} \red {P}'} {{{P} | {Q}} \red {{P}' | {Q}}}
  \and
  \inferrule* [lab=Equiv]{{{P} \scong {P}'} \andalso {{P}' \red {Q}'} \andalso {{Q}' \scong {Q}}}{{P} \red {Q}}
\end{mathpar}

\begin{eqnarray*}
  match_{\equiv} (\quotep{P},\quotep{Q}) & := & P \equiv Q \\
  match_{\dagger}(\quotep{P},\quotep{Q}) & := & \forall R. P|Q \red^{*} R => R \red^{*} 0 \\
  match_{K}(\quotep{P},\quotep{Q}) & := & K \mbox{ for some context } K
\end{eqnarray*}

$u?(x)P | u!\langle Q \rangle \red P\{\quotep{Q}/x\}$

%We write $\wred$ for $\red^*$, and $P\red$ if $\exists Q $ such that $ P \red Q$.
We write $P\red$ if $\exists Q $ such that $ P \red Q$ and $P\not\red$, otherwise.

\section{Replication}

As mentioned before, it is known that replication (and hence
recursion) can be implemented in a higher-order process algebra
\cite{SangiorgiWalker}. As our first example of calculation with the
machinery thus far presented we give the construction explicitly in
the {\rhoc}.

\begin{eqnarray}
	D_{x} & := & \prefix{x}{y}{(\binpar{\outputp{x}{y}}{@{y}})} \nonumber\\
	\bangp_{x}{P} & := & \binpar{{x}!\langle{\binpar{D_{x}}{P}}\rangle}{D_{x}} \nonumber
\end{eqnarray}

\begin{eqnarray}
	\bangp_{x}{P} & & \nonumber\\
	=
	& {x}!\langle{(\prefix{x}{y}{(\outputp{x}{y} | @{y})) | P}}\rangle 
	      | \prefix{x}{y}{(\outputp{x}{y} | @{y})} & \nonumber\\
	\red
	& (\outputp{x}{y} | @{y})\substn{\quotep{(\prefix{x}{y}{(@{y} | \outputp{x}{y})) | P}}}{y} & \nonumber\\
	=
	& \outputp{x}{\quotep{(\prefix{x}{y}{(\outputp{x}{y} | @{y})) | P}}}
	  | {(\prefix{x}{y}{(\outputp{x}{y} | @{y})) | P}} & \nonumber\\
	\red
	& \ldots & \nonumber\\
	\red^*
	& P | P | \ldots & \nonumber
\end{eqnarray}

Of course, this encoding, as an implementation, runs away, unfolding
$\bangp{P}$ eagerly. A lazier and more implementable replication
operator, restricted to input-guarded processes, may be obtained as follows.

\begin{eqnarray}
\bangp{\prefix{u}{v}{P}} 
	:= 
	\binpar{\lift{x}{\prefix{u}{v}{(\binpar{D(x)}{P})}}}{D(x)} \nonumber
\end{eqnarray}

\begin{remark}
  Note that the lazier definition still does not deal with summation
  or mixed summation (i.e. sums over input and output). The reader is
  invited to construct definitions of replication that deal with these
  features. 

  Further, the definitions are parameterized in a name, $x$. Can you,
  gentle reader, make a definition that eliminates this parameter and
  guarantees no accidental interaction between the replication
  machinery and the process being replicated -- i.e. no accidental
  sharing of names used by the process to get its work done and the
  name(s) used by the replication to effect copying. This latter
  revision of the definition of replication is crucial to obtaining
  the expected identity $!!P \sim !P$.
\end{remark}

\begin{remark}\label{rem:paradoxical_combinator}
  The reader familiar with the lambda calculus will have noticed the
  similarity between $D$ and the paradoxical combinator.

  [Ed. note: the existence of this seems to suggest we have to be more
  restrictive on the set of processes and names we admit if we are to
  support no-cloning.]
\end{remark}

\subsubsection{Bisimulation}

The computational dynamics gives rise to another kind of equivalence,
the equivalence of computational behavior. As previously mentioned
this is typically captured \emph{via} some form of bisimulation.

% The notion we use in this paper is weak barbed bisimulation
% \cite{milner91polyadicpi}.

The notion we use in this paper is derived from weak barbed
bisimulation \cite{milner91polyadicpi}. 

\begin{definition}
An \emph{observation relation}, $\downarrow_{\mathcal N}$, over a set
of names, $\mathcal N$, is the smallest relation satisfying the rules
below.

\infrule[Out-barb]{y \in {\mathcal N}, \; x \nameeq y}
		  {\outputp{x}{v} \downarrow_{\mathcal N} x}
\infrule[Par-barb]{\mbox{$P\downarrow_{\mathcal N} x$ or $Q\downarrow_{\mathcal N} x$}}
		  {\binpar{P}{Q} \downarrow_{\mathcal N} x}

We write $P \Downarrow_{\mathcal N} x$ if there is $Q$ such that 
$P \wred Q$ and $Q \downarrow_{\mathcal N} x$.
\end{definition}

\begin{definition}
%\label{def.bbisim}
An  ${\mathcal N}$-\emph{barbed bisimulation} over a set of names, ${\mathcal N}$, is a symmetric binary relation 
${\mathcal S}_{\mathcal N}$ between agents such that $P\rel{S}_{\mathcal N}Q$ implies:
\begin{enumerate}
\item If $P \red P'$ then $Q \wred Q'$ and $P'\rel{S}_{\mathcal N} Q'$.
\item If $P\downarrow_{\mathcal N} x$, then $Q\Downarrow_{\mathcal N} x$.
\end{enumerate}
$P$ is ${\mathcal N}$-barbed bisimilar to $Q$, written
$P \wbbisim_{\mathcal N} Q$, if $P \rel{S}_{\mathcal N} Q$ for some ${\mathcal N}$-barbed bisimulation ${\mathcal S}_{\mathcal N}$.
\end{definition}

$\mathcal{R} \subseteq \pi \times \pi$

$P \mathcal{R} Q => \forall P'. P \red P' \Rightarrow \exists Q'. Q \red Q', P' \mathcal{R} Q'$

$P \vdash x \Rightarrow Q \vdash x$

\begin{mathpar}
  \inferrule*[lab=Out-barb]{x \nameeq y}{{y}!\langle{Q}\rangle \vdash x}
  \and
  \inferrule*[lab=Par-barb]{\mbox{$P\vdash x$ or $Q\vdash x$}}{\binpar{P}{Q} \vdash x}
\end{mathpar}

\subsubsection{Contexts}

One of the principle advantages of computational calculi like the
$\pi$-calculus is a well-defined notion of context,
contextual-equivalence and a correlation between
contextual-equivalence and notions of bisimulation. The notion of
context allows the decomposition of a process into (sub-)process and
its syntactic environment, its context. Thus, a context may be
thought of as a process with a ``hole'' (written $\Box$) in it. The
application of a context $M$ to a process $P$, written $M[P]$, is
tantamount to filling the hole in $M$ with $P$. In this paper we do
not need the full weight of this theory, but do make use of the notion
of context in the proof the main theorem. 

\begin{mathpar}
  \inferrule* [lab=summation] {} {{M_{M},M_{N}} \bc \Box \;|\; x.M_{A} \;|\; M_{M}+M_{N}}
  \and
  \inferrule* [lab=agent] {} {{M_{A}} \bc (\vec{x})M_{P} \;| \; \clift{P_0,\ldots,M_{P},\ldots,P_N}}
  \and \\
  \inferrule* [lab=process] {} {{M_{P}} \bc M_{N} \;| \;P|M_{P} }
\end{mathpar} 

\begin{mathpar}
  \inferrule* [lab=sychronization] {} {M_{N} \bc \Box \;|\; x?M_{F} \;|\; x!M_{C}}
  \and
  \inferrule* [lab=abstraction] {} {{M_{F}} \bc (x)M_{P} }
  \and
  \inferrule* [lab=concretion] {} {{M_{C}} \bc \langle M_{P} \rangle }
  \and \\
  \inferrule* [lab=process] {} {{M_{P}} \bc M_{N} \;| \;P|M_{P} }
\end{mathpar}

\begin{definition}[contextual application] Given a context $M$, and
  process $P$, we define the \emph{contextual application}, $M[P] :=
  M\{P/\Box\}$. That is, the contextual application of M to P is the
  substitution of $P$ for $\Box$ in $M$.
\end{definition}

$\meaningof{-} : L \to \mathcal{P}(\pi)$

\begin{mathpar}
  \inferrule* [lab=collection] {} {\meaningof{true} = \pi, \and \meaningof{~E} = \pi \setminus \meaningof{E}, \and \meaningof{E_{1} \& E_{2}} = \meaningof{E_{1}} \cap \meaningof{E_{2}}}
\end{mathpar}

\begin{mathpar}
  \inferrule* [lab=structure] {} {\meaningof{0} = \{ P \in \pi | P \equiv 0 \}, \and \\ \meaningof{E_1 | E_2} = \{ P \in \pi | P \equiv P_{1} | P_{2}, P_{1} \in \meaningof{E_{1}}, P_{2} \in \meaningof{E_2}\} }
\end{mathpar}

\begin{mathpar}
 \inferrule* [lab=behavior] {} {\meaningof{\langle a?b \rangle E} = \{ P \in \pi | P \equiv Q | u?(y)P', \\ \and \\\\ \and \\ \;\;\; u \in \meaningof{a}, \forall z.P'\{z/y\} \in \meaningof{E\{z/b\}}\}, \and \\ \meaningof{a!E} = \{ P \in \pi | P \equiv Q | x!\langle P' \rangle, x \in \meaningof{a} P' \in \meaningof{E}\} }
\end{mathpar}

\begin{mathpar}
 \inferrule* [lab=nominal] {} {\meaningof{\quotep{E}} = \{ \quotep{P} \in \quotep{\pi} | P \in \meaningof{E} \}, \and \meaningof{\quotep{P}} = \{ \quotep{Q} \in \quotep{\pi} | P \equiv Q \} \and \\ \meaningof{@\quotep{E}} = \{ P \in \pi | P \equiv @x, x \in \meaningof{E} \}}
\end{mathpar}

\begin{eqnarray*}
  \\
  \meaningof{-} : TS \to ST
\end{eqnarray*}

\begin{eqnarray*}
  \\
  L : TS \to ST
\end{eqnarray*}

\begin{eqnarray*}
  \\
  P \models E \iff P \in \meaningof{E}
\end{eqnarray*}

\begin{eqnarray*}
  P \approx_{L} Q \iff \forall E \in L. P \models E \iff Q \models E
\end{eqnarray*}

\begin{eqnarray*}
  P \approx_{K} Q
\end{eqnarray*}

\begin{eqnarray*}
  P \approx Q
\end{eqnarray*}

$\approx_{K} = \approx = \approx_{L}$

\subsubsection{Contextual duality}

Note that contexts extend the quotation operation to a family of
operations from processes to names. Given a context, $M$, we can
define a \emph{nominal context}, $\quotep{M}$ by $\quotep{M}[P] :=
\quotep{M[P]}$. To foreshadow what is to come we observe that these
operations enjoy a duality with processes very much like the duality
between vectors and maps from vectors to scalars.

Further, because the calculus is essentially higher-order, we have a
correspondence between contexts and processes. More specifically,
given a name $x$ and a context $M$ we can construct $M^{*}_{x}$ such
that 

\begin{mathpar}
  M^{*}_{x} | \lift{x}{P} \red M[P]
\end{mathpar}

namely,

\begin{mathpar}
  M^{*}_{x} := x?(u).M[\dropn{u}]
\end{mathpar}

The dependence of $M^{*}_{x}$ on a name makes it an abstraction, 

\begin{mathpar}
  M^{*} := (x)x?(u).M[\dropn{u}]
\end{mathpar}

\subsection{Additional notation}

It will sometimes be convenient to denote the process a name
quotes. We already have the notation $x = \quotep{P}$, but it will be
convenient to introduce an alternate notation, $\procn{x}$, when we
want to emphasize the connection to the use of the name. Note that, by
virtue of name equivalence, $\quotep{\procn{x}} \nameeq x$; so, the
notation is consistent with previous definitions.

Further, because names have structure it is possible to effect
substitutions on the basis of that structure. This means we need to
upgrade our notation for substitutions, which we accomplish by
adapting comprehension notation. Thus,

\begin{mathpar}
  P\{ y / x : x \in S \}
\end{mathpar}

is interpreted to mean the process derived from P by replacing (in a
capture-avoiding manner) each occurrence of $x$ in $S$ by $y$. For example,

\begin{mathpar}
  P\{ \quotep{\procn{x}|\procn{x}} / x : x \in \freenames{P} \}
\end{mathpar}

will replace each (occurrence) of a free name $x$ in $P$ by
$\quotep{\procn{x}|\procn{x}}$.

Also, we will avail ourselves of the notation $x^{L}$ and $x^{R}$ to
denote injections of a name into disjoint copies of the name
space. There are numerous ways to accomplish this. One example can be
found in \cite{MeredithR05}. This notation overloads to vectors of
names: $\vec{x}^{\pi} := (x_{i}^{\pi} \; : \; 0 \leq i < |\vec{x}| )$ where $\pi \in \{L,R\}$.

We also use $P^{\Box} := P|\Box$.

In \cite{MeredithR05} an interpretation of the new operator is
given. It turns out that there are several possible interpretations
all enjoying the requisite algebraic properties of the operator (see
\cite{milner91polyadicpi}). We will therefore make liberal use of
$(\nu\; \vec{x})P$.

% subsection the_syntax_and_semantics_of_the_notation_system (end)   

\input{qm2pi.qmops} 

\input{qm2pi.sterngerlach} 

\input{qm2pi.metric} 

% section concurrent_process_calculi (end)

%\input{qm2pi.proofsketch}

% section proof sketch (end)

%\input{qm2pi.slviaknots} 

% section spatial logic via knots (end)

\input{qm2pi.conclusion}

% section conclusion (end)

%\input{qm2pi.dtcodes} 

% section wiring algorithm (end)

\input{qm2pi.ack} 

% section acknowledgments (end)

\newpage


\bibliographystyle{plain}   
\bibliography{../../biblios/main.bib}

\input{qm2pi.rhodetails}

\end{document}

 

% section concurrent_process_calculi (end)

%\documentclass[12pt]{llncs}
%\documentclass{jktr}

\usepackage[pdftex]{hyperref}                   
\usepackage {listings}
\usepackage {mathpartir}
\usepackage{bcprules}
%\usepackage{listings}
                       
\usepackage{graphicx} 
%\usepackage[margins=2.5cm,nohead,nofoot]{geometry}
%\usepackage{geometry}
\usepackage{amsfonts}
\usepackage{amstext}
\usepackage{latexsym}
\usepackage{amssymb}
\usepackage{color}


%\include{myPreamble}
\include{qm2pi.local} 

%\ifpdf
%\usepackage[pdftex]{graphicx}
%\else
%\usepackage{graphicx}
%\fi

 % \ifpdf
%  \usepackage{pdfsync}
%  \if


%\title{Brief Article}
%\author{David F. Snyder}
%\author{L.G. Meredith}

%\address{Dept. of Math., Texas State University--San Marcos, San Marcos, TX 78666}
       
\pagestyle{empty}


\begin{document}

\lstset{language=[Objective]Caml,frame=shadowbox}

\input{qm2pi.front}

% section front matter (end)

\input{qm2pi.intro} 
 
% section introduction (end)

% \input{qm2pi.knotations} 

% section notation (end)

\input{qm2pi.process.calculi} 

% section concurrent_process_calculi_and_spatial_logics_ (end)
    
%\input{qm2pi.knots2pi} 

%\input{qm2pi.trefoil} 

%\input{qm2pi.mainthm} 

% subsection basic_interpretation (end)

%\input{qm2pi.rho.presentation} 
\subsection{The syntax and semantics of the notation system}\label{sub:the_syntax_and_semantics_of_the_notation_system} % (fold)

We now summarize a technical presentation of the calculus that
embodies our theory of dynamics. The typical presentation of such a
calculus follows the style of giving generators and relations on
them. The grammar, below, describing term constructors, freely
generates the set of processes, $\Proc$. This set is then quotiented
by a relation known as structural congruence and it is over this set
that the notion of dynamics is expressed. This presentation is
essentially that of \cite{MeredithR05} with the addition of
polyadicity and summation. For readability we have relegated some of
the technical subtleties to an appendix.

\subsubsection{Process grammar}\label{subsub:process_grammar}

\begin{mathpar}
  \inferrule* [lab=synchronization] {} {{M} \bc \pzero \;|\; x?F \;|\; x!C }
  \and
  \inferrule* [lab=abstraction] {} {{F} \bc (x)P}
  \and
  \inferrule* [lab=concretion] {} {{C} \bc \langle Q \rangle}
  \and
  \inferrule* [lab=process] {} {{P,Q} \bc M \;| \;P|Q \;|\; @{x}}
  \and
  \inferrule* [lab=name] {} {{x} \bc \quotep{P}}
\end{mathpar} 

Note that $\vec{x}$ (resp. $\vec{P}$) denotes a vector of names
(resp. processes) of length $|\vec{x}|$ (resp. $|\vec{P}|$). We adopt
the following useful abbreviations.

\begin{mathpar}
   x?(\vec{y}).P := x.(\vec{y})P \and  x\clift{\vec{P}} := x.\clift{\vec{P}}
   \and x!(y) := \lift{x}{\dropn{y}}
   \and \Pi_{i=0}^{n-1}P_i := P_0 | \ldots | P_{n-1}
\end{mathpar}

\subsubsection{Structural congruence}

\paragraph{Free and bound names and alpha-equivalence.} At the
core of structural equivalence is alpha-equivalence which identifies
process that are the same up to a change of variable. Formally, we
recognize the distinction between free and bound names. The free names
of a process, $\freenames{P}$, may be calculated recursively as
follows:

\begin{mathpar}
\freenames{\pzero} := \emptyset
  \and \\
  \freenames{x?(y).P} := \{ x \} \cup (\freenames{P} \setminus \{ y \})
  \and 
  \freenames{x!\langle P \rangle} := \{ x \} \cup \{ P \} 
  \and \\
  \freenames{P|Q} := \freenames{P} \cup \freenames{Q}
  \and \\
  \freenames{@{x}} := \{ x \}
\end{mathpar}

$\pi$
$\quotep{\pi}$

$\freenames{-} : \pi \to \mathcal{P}(\quotep{\pi})$

\begin{eqnarray*}
  \freenames{\pzero} & := & \emptyset \\
  \freenames{x?(y).P} & := & \{ x \} \cup (\freenames{P} \setminus \{ y \}) \\
  \freenames{x!\langle P \rangle} & := & \{ x \} \cup \{ P \} \\
  \freenames{P|Q} & := & \freenames{P} \cup \freenames{Q} \\
  \freenames{\dropn{x}} & := & \{ x \}
\end{eqnarray*}

The bound names of a process, $\boundnames{P}$, are those names occurring in $P$
that are not free. For example, in $x?(y).0$, the name $x$ is free, while $y$ is bound.

\begin{mathpar}
  \inferrule* [lab=monoidal-laws] {} { P|Q \equiv Q|P \and P|0 \equiv P \and P|(Q|R) \equiv (P|Q)|R }
\end{mathpar}

\begin{mathpar}
  \inferrule* [lab=alpha-equivalence] {} { (x)P \equiv (y)P\{y/x\} \and y \not\in \freenames{P} }
\end{mathpar}

\begin{definition}
Then two processes, $P,Q$, are alpha-equivalent if $P = Q\{\vec{y}/\vec{x}\}$ for
some $\vec{x} \in \boundnames{Q},\vec{y} \in \boundnames{P}$, where $Q\{\vec{y}/\vec{x}\}$
denotes the capture-avoiding substitution of $\vec{y}$ for $\vec{x}$ in $Q$.
\end{definition}

\begin{definition}
  The {\em structural congruence} \cite{SangiorgiWalker} , $\equiv$,
  between processes is the least congruence containing
  alpha-equivalence, satisfying the abelian monoid laws
  (associativity, commutativity and $\pzero$ as identity) for parallel
  composition $|$ and for summation $+$.
\end{definition}

\subsection{Name equivalence}

We take name equivalence, written $\nameeq$, to be the smallest
equivalence relation generated by the following rules.

\begin{mathpar}
\inferrule*[lab=Quote-drop]
{ }
{ \quotep{@{x}} \nameeq x }

\inferrule*[lab=Struct-equiv]
{ P \scong Q }
{ \quotep{P} \nameeq \quotep{Q} }
\end{mathpar}

The astute reader will have noticed that the mutual recursion of names
and processes imposes a mutual recursion on alpha-equivalence and
structural equivalence via name-equivalence. Fortunately, all of this
works out pleasantly and we may calculate in the natural way, free of
concern. The reader interested in the details is referred to the
appendix \ref{appendix:rho_details}.

\subsection{Substitution}

We use $\Proc$ for the set of processes, $\QProc$ for the set of
names, and $\id{\{}\vec{y} / \vec{x} \id{\}}$ to denote partial maps,
$s : \QProc \rightarrow \QProc$. A map, $s$ lifts, uniquely, to a map
on process terms, $\widehat{s} : \Proc \rightarrow \Proc$ by the
following equations.

\begin{mathpar}
  (0) \psubstp{Q}{P} := 0 \\
  (R \juxtap S) \psubstp{Q}{P}
  :=    
  (R)\psubstp{Q}{P} \juxtap (S) \psubstp{Q}{P} \\
  (x?(y).R) \psubstp{Q}{P}    
  :=    
  (x)\substp{Q}{P} (z)\concat( (R \psubstn{z}{y}) \psubstp{Q}{P} ) \\
  (\lift{x}{R}) \psubstp{Q}{P}  
  :=
  \lift{(x)\substp{Q}{P}}{ R \psubstp{Q}{P} } \\
%   (\dropn{x})  \psubstp{Q}{P}       
%   := 
%   \left\{ 
%     \begin{array}{ccc} 
%       \dropn{\quotep{Q}} & & x \nameeq \quotep{P} \\
%       \dropn{x} & & otherwise \\
%     \end{array}
%   \right. 
  (\dropn{x})  \psubstp{Q}{P}       
  := 
  \left\{ 
    \begin{array}{ccc} 
      Q & & x \nameeq \quotep{P} \\
      \dropn{x} & & otherwise \\
    \end{array}
  \right.
\end{mathpar}
 

where

\begin{eqnarray}
  (x)\id{\{} \lpquote Q \rpquote / \lpquote P \rpquote \id{\}}            = 
  \left\{ 
    \begin{array}{ccc}
      \lpquote Q \rpquote & & x \nameeq \lpquote P \rpquote \\
      x & & otherwise \\
    \end{array}
  \right. \nonumber
\end{eqnarray}

and $z$ is chosen distinct from $\quotep{P}$, $\quotep{Q}$, the free
names in $Q$, and all the names in $R$. Our $\alpha$-equivalence will
be built in the standard way from this substitution.

\begin{remark}\label{rem:no_self_referential_names}
  One consequence of these definitions is that $\forall P. \quotep{P}
  \not\in \freenames{P}$.
\end{remark}

\subsection{ Dynamic quote: an example }

Anticipating something of what's to come, consider applying the
substitution, $\widehat{\id{\{}u / z \id{\}}}$, to the following pair
of processes, $\lift{w}{y!(z)}$ and $w[ \lpquote y!(z) \rpquote ]$.

\begin{eqnarray}
	\lift{w}{y!(z)}\widehat{\id{\{}u / z \id{\}}}
		& = &
		\lift{w}{y!(u)} \nonumber\\
	w[ \lpquote y!(z) \rpquote ] \widehat{ \id{\{}u / z \id{\}} }
		& = &
		w[ \lpquote y!(z) \rpquote ] \nonumber
\end{eqnarray}

Because the body of the process between quotes is impervious to
substitution, we get radically different answers. In fact, by
examining the first process in an input context,
e.g. $x?(z).\lift{w}{y!(z)}$, we see that the process under the lift
operator may be shaped by prefixed inputs binding a name inside it. In
this sense, the lift operator will be seen as a way to dynamically
construct processes before reifying them as names.

Finally equipped with these standard features we can present the
dynamics of the calculus.

\subsubsection{Operational semantics} 

Finally, we introduce the computational dynamics. What marks these
algebras as distinct from other more traditionally studied algebraic
structures, e.g. vector spaces or polynomial rings, is the manner in
which dynamics is captured. In traditional structures, dynamics is typically
expressed through morphisms between such structures, as in linear maps
between vector spaces or morphisms between rings. In algebras
associated with the semantics of computation, the dynamics is
expressed as part of the algebraic structure itself, through a
reduction reduction relation typically denoted by $\red$. Below, we
give a recursive presentation of this relation for the calculus used
in the encoding.

$\red \subseteq \pi \times \pi$
$\red : \pi \to \mathcal{P}(\pi)$

\begin{mathpar}
  \inferrule* [lab=Comm] { \textsf{match}( x_{src}, x_{trgt} ) } { x_{trgt}?(y)P \; | \; x_{src}!\langle {Q} \rangle \red P\{\quotep{Q}/y}\} }
  \and \\
  \inferrule* [lab=Par] {{P} \red {P}'} {{{P} | {Q}} \red {{P}' | {Q}}}
  \and
  \inferrule* [lab=Equiv]{{{P} \scong {P}'} \andalso {{P}' \red {Q}'} \andalso {{Q}' \scong {Q}}}{{P} \red {Q}}
\end{mathpar}

\begin{eqnarray*}
  match_{\equiv} (\quotep{P},\quotep{Q}) & := & P \equiv Q \\
  match_{\dagger}(\quotep{P},\quotep{Q}) & := & \forall R. P|Q \red^{*} R => R \red^{*} 0 \\
  match_{K}(\quotep{P},\quotep{Q}) & := & K \mbox{ for some context } K
\end{eqnarray*}

$u?(x)P | u!\langle Q \rangle \red P\{\quotep{Q}/x\}$

%We write $\wred$ for $\red^*$, and $P\red$ if $\exists Q $ such that $ P \red Q$.
We write $P\red$ if $\exists Q $ such that $ P \red Q$ and $P\not\red$, otherwise.

\section{Replication}

As mentioned before, it is known that replication (and hence
recursion) can be implemented in a higher-order process algebra
\cite{SangiorgiWalker}. As our first example of calculation with the
machinery thus far presented we give the construction explicitly in
the {\rhoc}.

\begin{eqnarray}
	D_{x} & := & \prefix{x}{y}{(\binpar{\outputp{x}{y}}{@{y}})} \nonumber\\
	\bangp_{x}{P} & := & \binpar{{x}!\langle{\binpar{D_{x}}{P}}\rangle}{D_{x}} \nonumber
\end{eqnarray}

\begin{eqnarray}
	\bangp_{x}{P} & & \nonumber\\
	=
	& {x}!\langle{(\prefix{x}{y}{(\outputp{x}{y} | @{y})) | P}}\rangle 
	      | \prefix{x}{y}{(\outputp{x}{y} | @{y})} & \nonumber\\
	\red
	& (\outputp{x}{y} | @{y})\substn{\quotep{(\prefix{x}{y}{(@{y} | \outputp{x}{y})) | P}}}{y} & \nonumber\\
	=
	& \outputp{x}{\quotep{(\prefix{x}{y}{(\outputp{x}{y} | @{y})) | P}}}
	  | {(\prefix{x}{y}{(\outputp{x}{y} | @{y})) | P}} & \nonumber\\
	\red
	& \ldots & \nonumber\\
	\red^*
	& P | P | \ldots & \nonumber
\end{eqnarray}

Of course, this encoding, as an implementation, runs away, unfolding
$\bangp{P}$ eagerly. A lazier and more implementable replication
operator, restricted to input-guarded processes, may be obtained as follows.

\begin{eqnarray}
\bangp{\prefix{u}{v}{P}} 
	:= 
	\binpar{\lift{x}{\prefix{u}{v}{(\binpar{D(x)}{P})}}}{D(x)} \nonumber
\end{eqnarray}

\begin{remark}
  Note that the lazier definition still does not deal with summation
  or mixed summation (i.e. sums over input and output). The reader is
  invited to construct definitions of replication that deal with these
  features. 

  Further, the definitions are parameterized in a name, $x$. Can you,
  gentle reader, make a definition that eliminates this parameter and
  guarantees no accidental interaction between the replication
  machinery and the process being replicated -- i.e. no accidental
  sharing of names used by the process to get its work done and the
  name(s) used by the replication to effect copying. This latter
  revision of the definition of replication is crucial to obtaining
  the expected identity $!!P \sim !P$.
\end{remark}

\begin{remark}\label{rem:paradoxical_combinator}
  The reader familiar with the lambda calculus will have noticed the
  similarity between $D$ and the paradoxical combinator.

  [Ed. note: the existence of this seems to suggest we have to be more
  restrictive on the set of processes and names we admit if we are to
  support no-cloning.]
\end{remark}

\subsubsection{Bisimulation}

The computational dynamics gives rise to another kind of equivalence,
the equivalence of computational behavior. As previously mentioned
this is typically captured \emph{via} some form of bisimulation.

% The notion we use in this paper is weak barbed bisimulation
% \cite{milner91polyadicpi}.

The notion we use in this paper is derived from weak barbed
bisimulation \cite{milner91polyadicpi}. 

\begin{definition}
An \emph{observation relation}, $\downarrow_{\mathcal N}$, over a set
of names, $\mathcal N$, is the smallest relation satisfying the rules
below.

\infrule[Out-barb]{y \in {\mathcal N}, \; x \nameeq y}
		  {\outputp{x}{v} \downarrow_{\mathcal N} x}
\infrule[Par-barb]{\mbox{$P\downarrow_{\mathcal N} x$ or $Q\downarrow_{\mathcal N} x$}}
		  {\binpar{P}{Q} \downarrow_{\mathcal N} x}

We write $P \Downarrow_{\mathcal N} x$ if there is $Q$ such that 
$P \wred Q$ and $Q \downarrow_{\mathcal N} x$.
\end{definition}

\begin{definition}
%\label{def.bbisim}
An  ${\mathcal N}$-\emph{barbed bisimulation} over a set of names, ${\mathcal N}$, is a symmetric binary relation 
${\mathcal S}_{\mathcal N}$ between agents such that $P\rel{S}_{\mathcal N}Q$ implies:
\begin{enumerate}
\item If $P \red P'$ then $Q \wred Q'$ and $P'\rel{S}_{\mathcal N} Q'$.
\item If $P\downarrow_{\mathcal N} x$, then $Q\Downarrow_{\mathcal N} x$.
\end{enumerate}
$P$ is ${\mathcal N}$-barbed bisimilar to $Q$, written
$P \wbbisim_{\mathcal N} Q$, if $P \rel{S}_{\mathcal N} Q$ for some ${\mathcal N}$-barbed bisimulation ${\mathcal S}_{\mathcal N}$.
\end{definition}

$\mathcal{R} \subseteq \pi \times \pi$

$P \mathcal{R} Q => \forall P'. P \red P' \Rightarrow \exists Q'. Q \red Q', P' \mathcal{R} Q'$

$P \vdash x \Rightarrow Q \vdash x$

\begin{mathpar}
  \inferrule*[lab=Out-barb]{x \nameeq y}{{y}!\langle{Q}\rangle \vdash x}
  \and
  \inferrule*[lab=Par-barb]{\mbox{$P\vdash x$ or $Q\vdash x$}}{\binpar{P}{Q} \vdash x}
\end{mathpar}

\subsubsection{Contexts}

One of the principle advantages of computational calculi like the
$\pi$-calculus is a well-defined notion of context,
contextual-equivalence and a correlation between
contextual-equivalence and notions of bisimulation. The notion of
context allows the decomposition of a process into (sub-)process and
its syntactic environment, its context. Thus, a context may be
thought of as a process with a ``hole'' (written $\Box$) in it. The
application of a context $M$ to a process $P$, written $M[P]$, is
tantamount to filling the hole in $M$ with $P$. In this paper we do
not need the full weight of this theory, but do make use of the notion
of context in the proof the main theorem. 

\begin{mathpar}
  \inferrule* [lab=summation] {} {{M_{M},M_{N}} \bc \Box \;|\; x.M_{A} \;|\; M_{M}+M_{N}}
  \and
  \inferrule* [lab=agent] {} {{M_{A}} \bc (\vec{x})M_{P} \;| \; \clift{P_0,\ldots,M_{P},\ldots,P_N}}
  \and \\
  \inferrule* [lab=process] {} {{M_{P}} \bc M_{N} \;| \;P|M_{P} }
\end{mathpar} 

\begin{mathpar}
  \inferrule* [lab=sychronization] {} {M_{N} \bc \Box \;|\; x?M_{F} \;|\; x!M_{C}}
  \and
  \inferrule* [lab=abstraction] {} {{M_{F}} \bc (x)M_{P} }
  \and
  \inferrule* [lab=concretion] {} {{M_{C}} \bc \langle M_{P} \rangle }
  \and \\
  \inferrule* [lab=process] {} {{M_{P}} \bc M_{N} \;| \;P|M_{P} }
\end{mathpar}

\begin{definition}[contextual application] Given a context $M$, and
  process $P$, we define the \emph{contextual application}, $M[P] :=
  M\{P/\Box\}$. That is, the contextual application of M to P is the
  substitution of $P$ for $\Box$ in $M$.
\end{definition}

$\meaningof{-} : L \to \mathcal{P}(\pi)$

\begin{mathpar}
  \inferrule* [lab=collection] {} {\meaningof{true} = \pi, \and \meaningof{~E} = \pi \setminus \meaningof{E}, \and \meaningof{E_{1} \& E_{2}} = \meaningof{E_{1}} \cap \meaningof{E_{2}}}
\end{mathpar}

\begin{mathpar}
  \inferrule* [lab=structure] {} {\meaningof{0} = \{ P \in \pi | P \equiv 0 \}, \and \\ \meaningof{E_1 | E_2} = \{ P \in \pi | P \equiv P_{1} | P_{2}, P_{1} \in \meaningof{E_{1}}, P_{2} \in \meaningof{E_2}\} }
\end{mathpar}

\begin{mathpar}
 \inferrule* [lab=behavior] {} {\meaningof{\langle a?b \rangle E} = \{ P \in \pi | P \equiv Q | u?(y)P', \\ \and \\\\ \and \\ \;\;\; u \in \meaningof{a}, \forall z.P'\{z/y\} \in \meaningof{E\{z/b\}}\}, \and \\ \meaningof{a!E} = \{ P \in \pi | P \equiv Q | x!\langle P' \rangle, x \in \meaningof{a} P' \in \meaningof{E}\} }
\end{mathpar}

\begin{mathpar}
 \inferrule* [lab=nominal] {} {\meaningof{\quotep{E}} = \{ \quotep{P} \in \quotep{\pi} | P \in \meaningof{E} \}, \and \meaningof{\quotep{P}} = \{ \quotep{Q} \in \quotep{\pi} | P \equiv Q \} \and \\ \meaningof{@\quotep{E}} = \{ P \in \pi | P \equiv @x, x \in \meaningof{E} \}}
\end{mathpar}

\begin{eqnarray*}
  \\
  \meaningof{-} : TS \to ST
\end{eqnarray*}

\begin{eqnarray*}
  \\
  L : TS \to ST
\end{eqnarray*}

\begin{eqnarray*}
  \\
  P \models E \iff P \in \meaningof{E}
\end{eqnarray*}

\begin{eqnarray*}
  P \approx_{L} Q \iff \forall E \in L. P \models E \iff Q \models E
\end{eqnarray*}

\begin{eqnarray*}
  P \approx_{K} Q
\end{eqnarray*}

\begin{eqnarray*}
  P \approx Q
\end{eqnarray*}

$\approx_{K} = \approx = \approx_{L}$

\subsubsection{Contextual duality}

Note that contexts extend the quotation operation to a family of
operations from processes to names. Given a context, $M$, we can
define a \emph{nominal context}, $\quotep{M}$ by $\quotep{M}[P] :=
\quotep{M[P]}$. To foreshadow what is to come we observe that these
operations enjoy a duality with processes very much like the duality
between vectors and maps from vectors to scalars.

Further, because the calculus is essentially higher-order, we have a
correspondence between contexts and processes. More specifically,
given a name $x$ and a context $M$ we can construct $M^{*}_{x}$ such
that 

\begin{mathpar}
  M^{*}_{x} | \lift{x}{P} \red M[P]
\end{mathpar}

namely,

\begin{mathpar}
  M^{*}_{x} := x?(u).M[\dropn{u}]
\end{mathpar}

The dependence of $M^{*}_{x}$ on a name makes it an abstraction, 

\begin{mathpar}
  M^{*} := (x)x?(u).M[\dropn{u}]
\end{mathpar}

\subsection{Additional notation}

It will sometimes be convenient to denote the process a name
quotes. We already have the notation $x = \quotep{P}$, but it will be
convenient to introduce an alternate notation, $\procn{x}$, when we
want to emphasize the connection to the use of the name. Note that, by
virtue of name equivalence, $\quotep{\procn{x}} \nameeq x$; so, the
notation is consistent with previous definitions.

Further, because names have structure it is possible to effect
substitutions on the basis of that structure. This means we need to
upgrade our notation for substitutions, which we accomplish by
adapting comprehension notation. Thus,

\begin{mathpar}
  P\{ y / x : x \in S \}
\end{mathpar}

is interpreted to mean the process derived from P by replacing (in a
capture-avoiding manner) each occurrence of $x$ in $S$ by $y$. For example,

\begin{mathpar}
  P\{ \quotep{\procn{x}|\procn{x}} / x : x \in \freenames{P} \}
\end{mathpar}

will replace each (occurrence) of a free name $x$ in $P$ by
$\quotep{\procn{x}|\procn{x}}$.

Also, we will avail ourselves of the notation $x^{L}$ and $x^{R}$ to
denote injections of a name into disjoint copies of the name
space. There are numerous ways to accomplish this. One example can be
found in \cite{MeredithR05}. This notation overloads to vectors of
names: $\vec{x}^{\pi} := (x_{i}^{\pi} \; : \; 0 \leq i < |\vec{x}| )$ where $\pi \in \{L,R\}$.

We also use $P^{\Box} := P|\Box$.

In \cite{MeredithR05} an interpretation of the new operator is
given. It turns out that there are several possible interpretations
all enjoying the requisite algebraic properties of the operator (see
\cite{milner91polyadicpi}). We will therefore make liberal use of
$(\nu\; \vec{x})P$.

% subsection the_syntax_and_semantics_of_the_notation_system (end)   

\input{qm2pi.qmops} 

\input{qm2pi.sterngerlach} 

\input{qm2pi.metric} 

% section concurrent_process_calculi (end)

%\input{qm2pi.proofsketch}

% section proof sketch (end)

%\input{qm2pi.slviaknots} 

% section spatial logic via knots (end)

\input{qm2pi.conclusion}

% section conclusion (end)

%\input{qm2pi.dtcodes} 

% section wiring algorithm (end)

\input{qm2pi.ack} 

% section acknowledgments (end)

\newpage


\bibliographystyle{plain}   
\bibliography{../../biblios/main.bib}

\input{qm2pi.rhodetails}

\end{document}



% section proof sketch (end)

%\section{Unlikely characters: spatial logic for
  knots}\label{sub:characteristic_formulae} % (fold)

Associated to the mobile process calculi are a family of logics known
as the Hennessy-Milner logics. These logics typically enjoy a
semantics interpreting formulae as sets of processes that when
factored through the encoding outlined above allows an identification
of classes of knots with logical formulae. In the context of this
encoding the sub-family known as the spatial logics \cite{CairesC03}
\cite{CairesC04} \cite{Caires04} are of particular interest providing
several important features for expressing and reasoning about
properties (i.e. classes) of knots. We hint here at how this may be done.

%\begin{description}
%\item [structural connectives] 
\subsubsection{Structural connectives} The spatial logics enjoy
structural connectives corresponding, at the logical level, to the
parallel composition ($P | Q$) and new name ($(\nu \; x)P$)
connectives for processes. As illustrated in the examples below, these
connectives are extremely expressive given the shape of our encoding.
%\item [decideable satisfaction]

\subsubsection{Decideable satisfaction}
In \cite{Caires04} the satisfaction relation is shown to be decideable
for a rich class of processes. It further turns out that the image of
the our encoding is a proper subset of that class. This result
provides the basis for an algorithm by which to search for knots
enjoying a given property.
%\item [characteristic formulae]

\subsubsection{Characteristic formulae}
In the same paper \cite{Caires04} , Caires presents a means of calculating
characteristic formulae, selecting equivalence classes of processes
up to a pre--specified depth limit on the support set of names. Composed with our
encoding, this characteristic formula can be used to select
characteristic formulae for knots.
%\end{description}

\subsubsection{Spatial logic formulae}

The grammar below (segmented for comprehension) summarizes the syntax
of spatial logic formulae. We employ illustrative examples in the
sequel to provide an intuitive understanding of their meaning
referring the reader to \cite{Caires04} for a more detailed explication
of the semantics.

\begin{mathpar}
  \inferrule* [lab=boolean] {} {{A,B} \bc T \;|\; \neg A \;|\; A \wedge B \;|\; \eta = \eta'}
  \and
  \inferrule* [lab=spatial] {} {|\; \pzero \;|\; A | B \;|\; x \text{\textregistered} A \;|\; \forall x . A \;|\;  H x . A}
  \and
  \inferrule* [lab=behavioral] {} {|\; \alpha . A}
  \and 
  \inferrule* [lab=recursion] {} {|\; X(\vec{u}) \;|\; \mu X(\vec{u}) . A}
  \and
  \inferrule* [lab=action] {} {\alpha \bc \langle x?(\vec{y}) \rangle \;|\; \langle x!(\vec{y}) \rangle \;|\; \langle \tau \rangle}
  \and 
  \inferrule* [lab=name] {} {\eta \bc x \;|\; \tau}
\end{mathpar} 

% subsection characteristic_formulae (end)   	 

\subsection{Example formulae}\label{sub:example_formulae_} % (fold)

\subsubsection{Crossing as formula.}
% 
% \begin{align*}
%   \frac{d}{dx} \sin x &= \cos x 
%   & \frac{d}{dx} e^x &= e^x \\
%   \frac{d}{dx} \cos x &= - \sin x 
%   & \frac{d}{dx} \log x &= \frac{1}{x} \\
% \end{align*} 

\begin{align*}
 \mu C(x_{0},x_{1},y_{0},y_{1},u).&(\langle x_{0}?(z) \rangle(\langle u! \rangle\langle y_{1}!z \rangle C(x_{0},x_{1},y_{0},y_{1},u)) & \\
  & \wedge \langle y_{1}?(z) \rangle (\langle u! \rangle \langle x_{0}!z \rangle C(x_{0},x_{1},y_{0},y_{1},u)) & \\
  & \wedge \langle x_{1}?(z) \rangle (\langle u? \rangle \langle y_{0}!z \rangle C(x_{0},x_{1},y_{0},y_{1},u)) & \\
  & \wedge \langle y_{0}?(z) \rangle (\langle u? \rangle \langle x_{1}!z \rangle C(x_{0},x_{1},y_{0},y_{1},u))) &
\end{align*}

The lexicographical similarity between the shape of this formulae and
the shape of definition of the process representing a crossing reveals
the intuitive meaning of this formulae. It describes the capabilities
of a process that has the right to represent a crossing. For example
it picks out processes that may perform an input on the port $x_0$ in
its initial menu of capabilities. What differentiates the formula
from the process, however, is that the crossing process is the
smallest candidate to satisfy the formula. Infinitely many other
processes -- with internal behavior hidden behind this interface, so
to speak -- also satisfy this formula. Even this simple formula,
then, can be seen to open a new view onto knots, providing a
computational interpretation of \emph{virtual} knots.

Note that this formula is derived by hand. A similar formula can be
derived by employing Caires' calculation of characteristic formula
\cite{Caires04} to the process representing a crossing. In light of
this discussion, we let
$\meaningof{C}_{\phi}(x0,x1,y0,y1,u)$ denote a formula specifying the
dynamics we wish to capture of a crossing. To guarantee we preserve
the shape of the interface and minimal semantics we demand that
$\meaningof{C}_{\phi}(x0,x1,y0,y1,u) \Rightarrow
\textbf{C}(x0,x1,y0,y1,u)$ where $\textbf{C}(x0,x1,y0,y1,u)$ denotes
the formula above.
                            
\subsubsection{Crossing number constraints.}
The moral content of the context lemma (Lemma \ref{context}) is that the notion of
``locality'' in the Reidemeister moves is effectively captured by the
parallel composition operator of the process calculus. This intuition
extends through the logic. Given a formula,
$\meaningof{C}_{\phi}(x0,x1,y0,y1,u)$, we can use the structural
connectives to specify constraints on crossing numbers, such as at
least $n$ crossings, or exactly $n$ crossings.
\begin{mathpar}
  \inferrule* [lab=at-least-n] {} { K^{\geq n}_{\phi}(\vec{xs},\vec{ys}) := \Pi_{i=0}^{n-1} Hu . \meaningof{C}_{\phi}(xs_i,ys_i,u) | T }
  \and 
  \inferrule* [lab=exactly-n] {} { K^{= n}_{\phi}(\vec{xs},\vec{ys}) := \Pi_{i=0}^{n-1} Hu . \meaningof{C}_{\phi}(xs_i,ys_i,u) | \neg (\forall x_0,y_0,x_1,y_1,u . \meaningof{C}_{\phi}(x_0,y_0,x_1,y_1,u) | T) }
\end{mathpar}

To round out this section, recall that the encoding of an $n$-crossing
knot decomposes into a parallel composition of $n$ \emph{copies} of a
crossing process together with a wiring harness. To specify different
knot classes with the same crossing number amounts to specifying
logical constraints on the wiring harness. In the interest of space,
we defer examples to a forthcoming paper. Suffice it to say that both
the conditions ``alternating knot'' and ``contains the tangle
corresponding to 5/3'' are expressible. For example, it is possible to
calculate the characteristic formula of a process corresponding to the
tangle 5/3 and conjoin it into the classifying formula via the
composition connective of the logic.

Finally, we wish to observe that it is entirely within reason to
contemplate a more domain-specific version of spatial logic tailored
to the shape of processes in the image of the encoding. Such a
domain-specific logic would have a better claim to the title formal
language of knot properties.

% subsection example_formulae_ (end)

% section knots_as_processes (end) 

% section spatial logic via knots (end)

\section{Conclusions and future work}

\paragraph{Testing physical space}
You, gentle reader, may wonder why of all the theorems to be proved
given this set up we pick the one above. In some sense it's hardly
central to quantum mechanics. We see it as central in the sense that
it firmly establishes a notion of physical space arising from a notion
of the equivalence of behavior. Relating bisimulation to a metric is a
big step forward, but one is faced with interpreting the relationship
of that metric space to something more physical. Quantum mechanical
notions of ``physical'' space are still far from intuitive, but by
relating this idea of distance as testing to calculations that predict
physical circumstances we are making a not insignificant step forward
toward an understanding of the physical space we inhabit as
essentially dynamic.

\paragraph{Effectivity and simulation}
One of the observations we have yet to make is that the entire program
spelled out here is effective. We have built various interpreters for
the reflective calculus at work in this interpretation. In principle,
then, we can simulate quantum mechanics on a computer. The place where
the simulation may lose fidelity is the infinitely branching summation
for the annihilator.

In this connection i also want to point out that the evaluation style
calculation of the inner product puts the non-determinism of the
summation right at the heart of measurement. This suggests that
Milner's original reduction-based formulation of the dynamics of his
calculi in terms of sums was not just notationally suggestive of a
notion of measure-and-continue but captured some significant part of
the physics.

\paragraph{Quantum continuations}
In light of this last observation i want to point out that the
predominant account of quantum mechanics is missing a key aspect of a
truly compositional story of the physical situation. In a real lab,
when a measurement is made the observation can be made to feed into
another device that then makes another measurement conditioned on the
results of the first. This means that after the superposition was
collapsed the entire experimental set up remained in
superposition. While QM offers a means of writing this down it doesn't
quite line up well with the well-trodden formulation of computation
and continuation that we see so succinctly expressed in Milner's
calculi. This suggests that there might be advantages to this account
of dynamics waiting to be explored.

\paragraph{Quantum logic}
In this connection, we also note that by virtue of having the
Hennessy-Milner construction, we can pull the construction through the
interpretation of QM. This gives us a natural candidate for a quantum
logic that enjoys an extremely tight connection with it's domain of
interpretation, making the construction much less ad hoc (rather it is
the image of functor!).

\paragraph{Quantum probabiity}
i have questions about the basis of the interpretation of inner
product as probability amplitude. In particular, using which
axiomatization of probability theory does the notion of probability
amplitude earn the right to be so dubbed? In other words, where is the
proof that the operation for calculating a probability amplitude (and
then squaring) satisfies the axioms of what it means to calculate a
probability? Even if such a proof exists (i have yet to find it in the
literature), i wonder if it might not be possible to turn things on
their heads. Can we view the calculation of the probability amplitude
as an axiomatization of probability? If so, then the definition we
give for calculating probability amplitude may provide the basis for
an \emph{effective} theory of probability.

\paragraph{Quantum vs ``biological'' information}
Finally, i want to conclude with a more philosophical observation. At
a recent workshop in which QM was a predominant topic i noticed
something about quantum information. The speaker was giving a riveting
discussion of axiomatic QM and showing how properties of ``no
cloning'' and ``no deleting'' emerged as consequences of the
axiomatization. Theorems of this form are necessary to give us a sense
of confidence that our axioms characterize the physical theory. What
struck me, though, was that if quantum information is neither erasable
nor replicable it is markedly different from \emph{life}. Two of the
things we know about life is that

\begin{itemize}
  \item it ends;
  \item to gain some measure of persistence, to transcend it's
    finitude it is imminently copyable.
\end{itemize}

Both of these qualities are summarized succinctly in the aphorism: all
flesh is grass. For me these two kinds of ``information'' -- call them
quantum and biological -- are end points on a spectrum of strategies
for persistence. At one end, we have those curious entities that enjoy
uniqueness and permanence; at the other, we have those who in the face
of a certain end and an uncertain present make a go of passing
something on. To me one of the more remarkable aspects of the latter
strategy is that in the presence of noise (and certain features of
copying) we get a kind of dynamism, a chance for improvement against a
given persistent condition.

% subsection other_calculi_other_bisimulations_and_geometry_as_behavior (end)




% section conclusion (end)

%\documentclass[12pt]{llncs}
%\documentclass{jktr}

\usepackage[pdftex]{hyperref}                   
\usepackage {listings}
\usepackage {mathpartir}
\usepackage{bcprules}
%\usepackage{listings}
                       
\usepackage{graphicx} 
%\usepackage[margins=2.5cm,nohead,nofoot]{geometry}
%\usepackage{geometry}
\usepackage{amsfonts}
\usepackage{amstext}
\usepackage{latexsym}
\usepackage{amssymb}
\usepackage{color}


%\include{myPreamble}
\include{qm2pi.local} 

%\ifpdf
%\usepackage[pdftex]{graphicx}
%\else
%\usepackage{graphicx}
%\fi

 % \ifpdf
%  \usepackage{pdfsync}
%  \if


%\title{Brief Article}
%\author{David F. Snyder}
%\author{L.G. Meredith}

%\address{Dept. of Math., Texas State University--San Marcos, San Marcos, TX 78666}
       
\pagestyle{empty}


\begin{document}

\lstset{language=[Objective]Caml,frame=shadowbox}

\input{qm2pi.front}

% section front matter (end)

\input{qm2pi.intro} 
 
% section introduction (end)

% \input{qm2pi.knotations} 

% section notation (end)

\input{qm2pi.process.calculi} 

% section concurrent_process_calculi_and_spatial_logics_ (end)
    
%\input{qm2pi.knots2pi} 

%\input{qm2pi.trefoil} 

%\input{qm2pi.mainthm} 

% subsection basic_interpretation (end)

%\input{qm2pi.rho.presentation} 
\subsection{The syntax and semantics of the notation system}\label{sub:the_syntax_and_semantics_of_the_notation_system} % (fold)

We now summarize a technical presentation of the calculus that
embodies our theory of dynamics. The typical presentation of such a
calculus follows the style of giving generators and relations on
them. The grammar, below, describing term constructors, freely
generates the set of processes, $\Proc$. This set is then quotiented
by a relation known as structural congruence and it is over this set
that the notion of dynamics is expressed. This presentation is
essentially that of \cite{MeredithR05} with the addition of
polyadicity and summation. For readability we have relegated some of
the technical subtleties to an appendix.

\subsubsection{Process grammar}\label{subsub:process_grammar}

\begin{mathpar}
  \inferrule* [lab=synchronization] {} {{M} \bc \pzero \;|\; x?F \;|\; x!C }
  \and
  \inferrule* [lab=abstraction] {} {{F} \bc (x)P}
  \and
  \inferrule* [lab=concretion] {} {{C} \bc \langle Q \rangle}
  \and
  \inferrule* [lab=process] {} {{P,Q} \bc M \;| \;P|Q \;|\; @{x}}
  \and
  \inferrule* [lab=name] {} {{x} \bc \quotep{P}}
\end{mathpar} 

Note that $\vec{x}$ (resp. $\vec{P}$) denotes a vector of names
(resp. processes) of length $|\vec{x}|$ (resp. $|\vec{P}|$). We adopt
the following useful abbreviations.

\begin{mathpar}
   x?(\vec{y}).P := x.(\vec{y})P \and  x\clift{\vec{P}} := x.\clift{\vec{P}}
   \and x!(y) := \lift{x}{\dropn{y}}
   \and \Pi_{i=0}^{n-1}P_i := P_0 | \ldots | P_{n-1}
\end{mathpar}

\subsubsection{Structural congruence}

\paragraph{Free and bound names and alpha-equivalence.} At the
core of structural equivalence is alpha-equivalence which identifies
process that are the same up to a change of variable. Formally, we
recognize the distinction between free and bound names. The free names
of a process, $\freenames{P}$, may be calculated recursively as
follows:

\begin{mathpar}
\freenames{\pzero} := \emptyset
  \and \\
  \freenames{x?(y).P} := \{ x \} \cup (\freenames{P} \setminus \{ y \})
  \and 
  \freenames{x!\langle P \rangle} := \{ x \} \cup \{ P \} 
  \and \\
  \freenames{P|Q} := \freenames{P} \cup \freenames{Q}
  \and \\
  \freenames{@{x}} := \{ x \}
\end{mathpar}

$\pi$
$\quotep{\pi}$

$\freenames{-} : \pi \to \mathcal{P}(\quotep{\pi})$

\begin{eqnarray*}
  \freenames{\pzero} & := & \emptyset \\
  \freenames{x?(y).P} & := & \{ x \} \cup (\freenames{P} \setminus \{ y \}) \\
  \freenames{x!\langle P \rangle} & := & \{ x \} \cup \{ P \} \\
  \freenames{P|Q} & := & \freenames{P} \cup \freenames{Q} \\
  \freenames{\dropn{x}} & := & \{ x \}
\end{eqnarray*}

The bound names of a process, $\boundnames{P}$, are those names occurring in $P$
that are not free. For example, in $x?(y).0$, the name $x$ is free, while $y$ is bound.

\begin{mathpar}
  \inferrule* [lab=monoidal-laws] {} { P|Q \equiv Q|P \and P|0 \equiv P \and P|(Q|R) \equiv (P|Q)|R }
\end{mathpar}

\begin{mathpar}
  \inferrule* [lab=alpha-equivalence] {} { (x)P \equiv (y)P\{y/x\} \and y \not\in \freenames{P} }
\end{mathpar}

\begin{definition}
Then two processes, $P,Q$, are alpha-equivalent if $P = Q\{\vec{y}/\vec{x}\}$ for
some $\vec{x} \in \boundnames{Q},\vec{y} \in \boundnames{P}$, where $Q\{\vec{y}/\vec{x}\}$
denotes the capture-avoiding substitution of $\vec{y}$ for $\vec{x}$ in $Q$.
\end{definition}

\begin{definition}
  The {\em structural congruence} \cite{SangiorgiWalker} , $\equiv$,
  between processes is the least congruence containing
  alpha-equivalence, satisfying the abelian monoid laws
  (associativity, commutativity and $\pzero$ as identity) for parallel
  composition $|$ and for summation $+$.
\end{definition}

\subsection{Name equivalence}

We take name equivalence, written $\nameeq$, to be the smallest
equivalence relation generated by the following rules.

\begin{mathpar}
\inferrule*[lab=Quote-drop]
{ }
{ \quotep{@{x}} \nameeq x }

\inferrule*[lab=Struct-equiv]
{ P \scong Q }
{ \quotep{P} \nameeq \quotep{Q} }
\end{mathpar}

The astute reader will have noticed that the mutual recursion of names
and processes imposes a mutual recursion on alpha-equivalence and
structural equivalence via name-equivalence. Fortunately, all of this
works out pleasantly and we may calculate in the natural way, free of
concern. The reader interested in the details is referred to the
appendix \ref{appendix:rho_details}.

\subsection{Substitution}

We use $\Proc$ for the set of processes, $\QProc$ for the set of
names, and $\id{\{}\vec{y} / \vec{x} \id{\}}$ to denote partial maps,
$s : \QProc \rightarrow \QProc$. A map, $s$ lifts, uniquely, to a map
on process terms, $\widehat{s} : \Proc \rightarrow \Proc$ by the
following equations.

\begin{mathpar}
  (0) \psubstp{Q}{P} := 0 \\
  (R \juxtap S) \psubstp{Q}{P}
  :=    
  (R)\psubstp{Q}{P} \juxtap (S) \psubstp{Q}{P} \\
  (x?(y).R) \psubstp{Q}{P}    
  :=    
  (x)\substp{Q}{P} (z)\concat( (R \psubstn{z}{y}) \psubstp{Q}{P} ) \\
  (\lift{x}{R}) \psubstp{Q}{P}  
  :=
  \lift{(x)\substp{Q}{P}}{ R \psubstp{Q}{P} } \\
%   (\dropn{x})  \psubstp{Q}{P}       
%   := 
%   \left\{ 
%     \begin{array}{ccc} 
%       \dropn{\quotep{Q}} & & x \nameeq \quotep{P} \\
%       \dropn{x} & & otherwise \\
%     \end{array}
%   \right. 
  (\dropn{x})  \psubstp{Q}{P}       
  := 
  \left\{ 
    \begin{array}{ccc} 
      Q & & x \nameeq \quotep{P} \\
      \dropn{x} & & otherwise \\
    \end{array}
  \right.
\end{mathpar}
 

where

\begin{eqnarray}
  (x)\id{\{} \lpquote Q \rpquote / \lpquote P \rpquote \id{\}}            = 
  \left\{ 
    \begin{array}{ccc}
      \lpquote Q \rpquote & & x \nameeq \lpquote P \rpquote \\
      x & & otherwise \\
    \end{array}
  \right. \nonumber
\end{eqnarray}

and $z$ is chosen distinct from $\quotep{P}$, $\quotep{Q}$, the free
names in $Q$, and all the names in $R$. Our $\alpha$-equivalence will
be built in the standard way from this substitution.

\begin{remark}\label{rem:no_self_referential_names}
  One consequence of these definitions is that $\forall P. \quotep{P}
  \not\in \freenames{P}$.
\end{remark}

\subsection{ Dynamic quote: an example }

Anticipating something of what's to come, consider applying the
substitution, $\widehat{\id{\{}u / z \id{\}}}$, to the following pair
of processes, $\lift{w}{y!(z)}$ and $w[ \lpquote y!(z) \rpquote ]$.

\begin{eqnarray}
	\lift{w}{y!(z)}\widehat{\id{\{}u / z \id{\}}}
		& = &
		\lift{w}{y!(u)} \nonumber\\
	w[ \lpquote y!(z) \rpquote ] \widehat{ \id{\{}u / z \id{\}} }
		& = &
		w[ \lpquote y!(z) \rpquote ] \nonumber
\end{eqnarray}

Because the body of the process between quotes is impervious to
substitution, we get radically different answers. In fact, by
examining the first process in an input context,
e.g. $x?(z).\lift{w}{y!(z)}$, we see that the process under the lift
operator may be shaped by prefixed inputs binding a name inside it. In
this sense, the lift operator will be seen as a way to dynamically
construct processes before reifying them as names.

Finally equipped with these standard features we can present the
dynamics of the calculus.

\subsubsection{Operational semantics} 

Finally, we introduce the computational dynamics. What marks these
algebras as distinct from other more traditionally studied algebraic
structures, e.g. vector spaces or polynomial rings, is the manner in
which dynamics is captured. In traditional structures, dynamics is typically
expressed through morphisms between such structures, as in linear maps
between vector spaces or morphisms between rings. In algebras
associated with the semantics of computation, the dynamics is
expressed as part of the algebraic structure itself, through a
reduction reduction relation typically denoted by $\red$. Below, we
give a recursive presentation of this relation for the calculus used
in the encoding.

$\red \subseteq \pi \times \pi$
$\red : \pi \to \mathcal{P}(\pi)$

\begin{mathpar}
  \inferrule* [lab=Comm] { \textsf{match}( x_{src}, x_{trgt} ) } { x_{trgt}?(y)P \; | \; x_{src}!\langle {Q} \rangle \red P\{\quotep{Q}/y}\} }
  \and \\
  \inferrule* [lab=Par] {{P} \red {P}'} {{{P} | {Q}} \red {{P}' | {Q}}}
  \and
  \inferrule* [lab=Equiv]{{{P} \scong {P}'} \andalso {{P}' \red {Q}'} \andalso {{Q}' \scong {Q}}}{{P} \red {Q}}
\end{mathpar}

\begin{eqnarray*}
  match_{\equiv} (\quotep{P},\quotep{Q}) & := & P \equiv Q \\
  match_{\dagger}(\quotep{P},\quotep{Q}) & := & \forall R. P|Q \red^{*} R => R \red^{*} 0 \\
  match_{K}(\quotep{P},\quotep{Q}) & := & K \mbox{ for some context } K
\end{eqnarray*}

$u?(x)P | u!\langle Q \rangle \red P\{\quotep{Q}/x\}$

%We write $\wred$ for $\red^*$, and $P\red$ if $\exists Q $ such that $ P \red Q$.
We write $P\red$ if $\exists Q $ such that $ P \red Q$ and $P\not\red$, otherwise.

\section{Replication}

As mentioned before, it is known that replication (and hence
recursion) can be implemented in a higher-order process algebra
\cite{SangiorgiWalker}. As our first example of calculation with the
machinery thus far presented we give the construction explicitly in
the {\rhoc}.

\begin{eqnarray}
	D_{x} & := & \prefix{x}{y}{(\binpar{\outputp{x}{y}}{@{y}})} \nonumber\\
	\bangp_{x}{P} & := & \binpar{{x}!\langle{\binpar{D_{x}}{P}}\rangle}{D_{x}} \nonumber
\end{eqnarray}

\begin{eqnarray}
	\bangp_{x}{P} & & \nonumber\\
	=
	& {x}!\langle{(\prefix{x}{y}{(\outputp{x}{y} | @{y})) | P}}\rangle 
	      | \prefix{x}{y}{(\outputp{x}{y} | @{y})} & \nonumber\\
	\red
	& (\outputp{x}{y} | @{y})\substn{\quotep{(\prefix{x}{y}{(@{y} | \outputp{x}{y})) | P}}}{y} & \nonumber\\
	=
	& \outputp{x}{\quotep{(\prefix{x}{y}{(\outputp{x}{y} | @{y})) | P}}}
	  | {(\prefix{x}{y}{(\outputp{x}{y} | @{y})) | P}} & \nonumber\\
	\red
	& \ldots & \nonumber\\
	\red^*
	& P | P | \ldots & \nonumber
\end{eqnarray}

Of course, this encoding, as an implementation, runs away, unfolding
$\bangp{P}$ eagerly. A lazier and more implementable replication
operator, restricted to input-guarded processes, may be obtained as follows.

\begin{eqnarray}
\bangp{\prefix{u}{v}{P}} 
	:= 
	\binpar{\lift{x}{\prefix{u}{v}{(\binpar{D(x)}{P})}}}{D(x)} \nonumber
\end{eqnarray}

\begin{remark}
  Note that the lazier definition still does not deal with summation
  or mixed summation (i.e. sums over input and output). The reader is
  invited to construct definitions of replication that deal with these
  features. 

  Further, the definitions are parameterized in a name, $x$. Can you,
  gentle reader, make a definition that eliminates this parameter and
  guarantees no accidental interaction between the replication
  machinery and the process being replicated -- i.e. no accidental
  sharing of names used by the process to get its work done and the
  name(s) used by the replication to effect copying. This latter
  revision of the definition of replication is crucial to obtaining
  the expected identity $!!P \sim !P$.
\end{remark}

\begin{remark}\label{rem:paradoxical_combinator}
  The reader familiar with the lambda calculus will have noticed the
  similarity between $D$ and the paradoxical combinator.

  [Ed. note: the existence of this seems to suggest we have to be more
  restrictive on the set of processes and names we admit if we are to
  support no-cloning.]
\end{remark}

\subsubsection{Bisimulation}

The computational dynamics gives rise to another kind of equivalence,
the equivalence of computational behavior. As previously mentioned
this is typically captured \emph{via} some form of bisimulation.

% The notion we use in this paper is weak barbed bisimulation
% \cite{milner91polyadicpi}.

The notion we use in this paper is derived from weak barbed
bisimulation \cite{milner91polyadicpi}. 

\begin{definition}
An \emph{observation relation}, $\downarrow_{\mathcal N}$, over a set
of names, $\mathcal N$, is the smallest relation satisfying the rules
below.

\infrule[Out-barb]{y \in {\mathcal N}, \; x \nameeq y}
		  {\outputp{x}{v} \downarrow_{\mathcal N} x}
\infrule[Par-barb]{\mbox{$P\downarrow_{\mathcal N} x$ or $Q\downarrow_{\mathcal N} x$}}
		  {\binpar{P}{Q} \downarrow_{\mathcal N} x}

We write $P \Downarrow_{\mathcal N} x$ if there is $Q$ such that 
$P \wred Q$ and $Q \downarrow_{\mathcal N} x$.
\end{definition}

\begin{definition}
%\label{def.bbisim}
An  ${\mathcal N}$-\emph{barbed bisimulation} over a set of names, ${\mathcal N}$, is a symmetric binary relation 
${\mathcal S}_{\mathcal N}$ between agents such that $P\rel{S}_{\mathcal N}Q$ implies:
\begin{enumerate}
\item If $P \red P'$ then $Q \wred Q'$ and $P'\rel{S}_{\mathcal N} Q'$.
\item If $P\downarrow_{\mathcal N} x$, then $Q\Downarrow_{\mathcal N} x$.
\end{enumerate}
$P$ is ${\mathcal N}$-barbed bisimilar to $Q$, written
$P \wbbisim_{\mathcal N} Q$, if $P \rel{S}_{\mathcal N} Q$ for some ${\mathcal N}$-barbed bisimulation ${\mathcal S}_{\mathcal N}$.
\end{definition}

$\mathcal{R} \subseteq \pi \times \pi$

$P \mathcal{R} Q => \forall P'. P \red P' \Rightarrow \exists Q'. Q \red Q', P' \mathcal{R} Q'$

$P \vdash x \Rightarrow Q \vdash x$

\begin{mathpar}
  \inferrule*[lab=Out-barb]{x \nameeq y}{{y}!\langle{Q}\rangle \vdash x}
  \and
  \inferrule*[lab=Par-barb]{\mbox{$P\vdash x$ or $Q\vdash x$}}{\binpar{P}{Q} \vdash x}
\end{mathpar}

\subsubsection{Contexts}

One of the principle advantages of computational calculi like the
$\pi$-calculus is a well-defined notion of context,
contextual-equivalence and a correlation between
contextual-equivalence and notions of bisimulation. The notion of
context allows the decomposition of a process into (sub-)process and
its syntactic environment, its context. Thus, a context may be
thought of as a process with a ``hole'' (written $\Box$) in it. The
application of a context $M$ to a process $P$, written $M[P]$, is
tantamount to filling the hole in $M$ with $P$. In this paper we do
not need the full weight of this theory, but do make use of the notion
of context in the proof the main theorem. 

\begin{mathpar}
  \inferrule* [lab=summation] {} {{M_{M},M_{N}} \bc \Box \;|\; x.M_{A} \;|\; M_{M}+M_{N}}
  \and
  \inferrule* [lab=agent] {} {{M_{A}} \bc (\vec{x})M_{P} \;| \; \clift{P_0,\ldots,M_{P},\ldots,P_N}}
  \and \\
  \inferrule* [lab=process] {} {{M_{P}} \bc M_{N} \;| \;P|M_{P} }
\end{mathpar} 

\begin{mathpar}
  \inferrule* [lab=sychronization] {} {M_{N} \bc \Box \;|\; x?M_{F} \;|\; x!M_{C}}
  \and
  \inferrule* [lab=abstraction] {} {{M_{F}} \bc (x)M_{P} }
  \and
  \inferrule* [lab=concretion] {} {{M_{C}} \bc \langle M_{P} \rangle }
  \and \\
  \inferrule* [lab=process] {} {{M_{P}} \bc M_{N} \;| \;P|M_{P} }
\end{mathpar}

\begin{definition}[contextual application] Given a context $M$, and
  process $P$, we define the \emph{contextual application}, $M[P] :=
  M\{P/\Box\}$. That is, the contextual application of M to P is the
  substitution of $P$ for $\Box$ in $M$.
\end{definition}

$\meaningof{-} : L \to \mathcal{P}(\pi)$

\begin{mathpar}
  \inferrule* [lab=collection] {} {\meaningof{true} = \pi, \and \meaningof{~E} = \pi \setminus \meaningof{E}, \and \meaningof{E_{1} \& E_{2}} = \meaningof{E_{1}} \cap \meaningof{E_{2}}}
\end{mathpar}

\begin{mathpar}
  \inferrule* [lab=structure] {} {\meaningof{0} = \{ P \in \pi | P \equiv 0 \}, \and \\ \meaningof{E_1 | E_2} = \{ P \in \pi | P \equiv P_{1} | P_{2}, P_{1} \in \meaningof{E_{1}}, P_{2} \in \meaningof{E_2}\} }
\end{mathpar}

\begin{mathpar}
 \inferrule* [lab=behavior] {} {\meaningof{\langle a?b \rangle E} = \{ P \in \pi | P \equiv Q | u?(y)P', \\ \and \\\\ \and \\ \;\;\; u \in \meaningof{a}, \forall z.P'\{z/y\} \in \meaningof{E\{z/b\}}\}, \and \\ \meaningof{a!E} = \{ P \in \pi | P \equiv Q | x!\langle P' \rangle, x \in \meaningof{a} P' \in \meaningof{E}\} }
\end{mathpar}

\begin{mathpar}
 \inferrule* [lab=nominal] {} {\meaningof{\quotep{E}} = \{ \quotep{P} \in \quotep{\pi} | P \in \meaningof{E} \}, \and \meaningof{\quotep{P}} = \{ \quotep{Q} \in \quotep{\pi} | P \equiv Q \} \and \\ \meaningof{@\quotep{E}} = \{ P \in \pi | P \equiv @x, x \in \meaningof{E} \}}
\end{mathpar}

\begin{eqnarray*}
  \\
  \meaningof{-} : TS \to ST
\end{eqnarray*}

\begin{eqnarray*}
  \\
  L : TS \to ST
\end{eqnarray*}

\begin{eqnarray*}
  \\
  P \models E \iff P \in \meaningof{E}
\end{eqnarray*}

\begin{eqnarray*}
  P \approx_{L} Q \iff \forall E \in L. P \models E \iff Q \models E
\end{eqnarray*}

\begin{eqnarray*}
  P \approx_{K} Q
\end{eqnarray*}

\begin{eqnarray*}
  P \approx Q
\end{eqnarray*}

$\approx_{K} = \approx = \approx_{L}$

\subsubsection{Contextual duality}

Note that contexts extend the quotation operation to a family of
operations from processes to names. Given a context, $M$, we can
define a \emph{nominal context}, $\quotep{M}$ by $\quotep{M}[P] :=
\quotep{M[P]}$. To foreshadow what is to come we observe that these
operations enjoy a duality with processes very much like the duality
between vectors and maps from vectors to scalars.

Further, because the calculus is essentially higher-order, we have a
correspondence between contexts and processes. More specifically,
given a name $x$ and a context $M$ we can construct $M^{*}_{x}$ such
that 

\begin{mathpar}
  M^{*}_{x} | \lift{x}{P} \red M[P]
\end{mathpar}

namely,

\begin{mathpar}
  M^{*}_{x} := x?(u).M[\dropn{u}]
\end{mathpar}

The dependence of $M^{*}_{x}$ on a name makes it an abstraction, 

\begin{mathpar}
  M^{*} := (x)x?(u).M[\dropn{u}]
\end{mathpar}

\subsection{Additional notation}

It will sometimes be convenient to denote the process a name
quotes. We already have the notation $x = \quotep{P}$, but it will be
convenient to introduce an alternate notation, $\procn{x}$, when we
want to emphasize the connection to the use of the name. Note that, by
virtue of name equivalence, $\quotep{\procn{x}} \nameeq x$; so, the
notation is consistent with previous definitions.

Further, because names have structure it is possible to effect
substitutions on the basis of that structure. This means we need to
upgrade our notation for substitutions, which we accomplish by
adapting comprehension notation. Thus,

\begin{mathpar}
  P\{ y / x : x \in S \}
\end{mathpar}

is interpreted to mean the process derived from P by replacing (in a
capture-avoiding manner) each occurrence of $x$ in $S$ by $y$. For example,

\begin{mathpar}
  P\{ \quotep{\procn{x}|\procn{x}} / x : x \in \freenames{P} \}
\end{mathpar}

will replace each (occurrence) of a free name $x$ in $P$ by
$\quotep{\procn{x}|\procn{x}}$.

Also, we will avail ourselves of the notation $x^{L}$ and $x^{R}$ to
denote injections of a name into disjoint copies of the name
space. There are numerous ways to accomplish this. One example can be
found in \cite{MeredithR05}. This notation overloads to vectors of
names: $\vec{x}^{\pi} := (x_{i}^{\pi} \; : \; 0 \leq i < |\vec{x}| )$ where $\pi \in \{L,R\}$.

We also use $P^{\Box} := P|\Box$.

In \cite{MeredithR05} an interpretation of the new operator is
given. It turns out that there are several possible interpretations
all enjoying the requisite algebraic properties of the operator (see
\cite{milner91polyadicpi}). We will therefore make liberal use of
$(\nu\; \vec{x})P$.

% subsection the_syntax_and_semantics_of_the_notation_system (end)   

\input{qm2pi.qmops} 

\input{qm2pi.sterngerlach} 

\input{qm2pi.metric} 

% section concurrent_process_calculi (end)

%\input{qm2pi.proofsketch}

% section proof sketch (end)

%\input{qm2pi.slviaknots} 

% section spatial logic via knots (end)

\input{qm2pi.conclusion}

% section conclusion (end)

%\input{qm2pi.dtcodes} 

% section wiring algorithm (end)

\input{qm2pi.ack} 

% section acknowledgments (end)

\newpage


\bibliographystyle{plain}   
\bibliography{../../biblios/main.bib}

\input{qm2pi.rhodetails}

\end{document}

 

% section wiring algorithm (end)

\documentclass[12pt]{llncs}
%\documentclass{jktr}

\usepackage[pdftex]{hyperref}                   
\usepackage {listings}
\usepackage {mathpartir}
\usepackage{bcprules}
%\usepackage{listings}
                       
\usepackage{graphicx} 
%\usepackage[margins=2.5cm,nohead,nofoot]{geometry}
%\usepackage{geometry}
\usepackage{amsfonts}
\usepackage{amstext}
\usepackage{latexsym}
\usepackage{amssymb}
\usepackage{color}


%\include{myPreamble}
\include{qm2pi.local} 

%\ifpdf
%\usepackage[pdftex]{graphicx}
%\else
%\usepackage{graphicx}
%\fi

 % \ifpdf
%  \usepackage{pdfsync}
%  \if


%\title{Brief Article}
%\author{David F. Snyder}
%\author{L.G. Meredith}

%\address{Dept. of Math., Texas State University--San Marcos, San Marcos, TX 78666}
       
\pagestyle{empty}


\begin{document}

\lstset{language=[Objective]Caml,frame=shadowbox}

\input{qm2pi.front}

% section front matter (end)

\input{qm2pi.intro} 
 
% section introduction (end)

% \input{qm2pi.knotations} 

% section notation (end)

\input{qm2pi.process.calculi} 

% section concurrent_process_calculi_and_spatial_logics_ (end)
    
%\input{qm2pi.knots2pi} 

%\input{qm2pi.trefoil} 

%\input{qm2pi.mainthm} 

% subsection basic_interpretation (end)

%\input{qm2pi.rho.presentation} 
\subsection{The syntax and semantics of the notation system}\label{sub:the_syntax_and_semantics_of_the_notation_system} % (fold)

We now summarize a technical presentation of the calculus that
embodies our theory of dynamics. The typical presentation of such a
calculus follows the style of giving generators and relations on
them. The grammar, below, describing term constructors, freely
generates the set of processes, $\Proc$. This set is then quotiented
by a relation known as structural congruence and it is over this set
that the notion of dynamics is expressed. This presentation is
essentially that of \cite{MeredithR05} with the addition of
polyadicity and summation. For readability we have relegated some of
the technical subtleties to an appendix.

\subsubsection{Process grammar}\label{subsub:process_grammar}

\begin{mathpar}
  \inferrule* [lab=synchronization] {} {{M} \bc \pzero \;|\; x?F \;|\; x!C }
  \and
  \inferrule* [lab=abstraction] {} {{F} \bc (x)P}
  \and
  \inferrule* [lab=concretion] {} {{C} \bc \langle Q \rangle}
  \and
  \inferrule* [lab=process] {} {{P,Q} \bc M \;| \;P|Q \;|\; @{x}}
  \and
  \inferrule* [lab=name] {} {{x} \bc \quotep{P}}
\end{mathpar} 

Note that $\vec{x}$ (resp. $\vec{P}$) denotes a vector of names
(resp. processes) of length $|\vec{x}|$ (resp. $|\vec{P}|$). We adopt
the following useful abbreviations.

\begin{mathpar}
   x?(\vec{y}).P := x.(\vec{y})P \and  x\clift{\vec{P}} := x.\clift{\vec{P}}
   \and x!(y) := \lift{x}{\dropn{y}}
   \and \Pi_{i=0}^{n-1}P_i := P_0 | \ldots | P_{n-1}
\end{mathpar}

\subsubsection{Structural congruence}

\paragraph{Free and bound names and alpha-equivalence.} At the
core of structural equivalence is alpha-equivalence which identifies
process that are the same up to a change of variable. Formally, we
recognize the distinction between free and bound names. The free names
of a process, $\freenames{P}$, may be calculated recursively as
follows:

\begin{mathpar}
\freenames{\pzero} := \emptyset
  \and \\
  \freenames{x?(y).P} := \{ x \} \cup (\freenames{P} \setminus \{ y \})
  \and 
  \freenames{x!\langle P \rangle} := \{ x \} \cup \{ P \} 
  \and \\
  \freenames{P|Q} := \freenames{P} \cup \freenames{Q}
  \and \\
  \freenames{@{x}} := \{ x \}
\end{mathpar}

$\pi$
$\quotep{\pi}$

$\freenames{-} : \pi \to \mathcal{P}(\quotep{\pi})$

\begin{eqnarray*}
  \freenames{\pzero} & := & \emptyset \\
  \freenames{x?(y).P} & := & \{ x \} \cup (\freenames{P} \setminus \{ y \}) \\
  \freenames{x!\langle P \rangle} & := & \{ x \} \cup \{ P \} \\
  \freenames{P|Q} & := & \freenames{P} \cup \freenames{Q} \\
  \freenames{\dropn{x}} & := & \{ x \}
\end{eqnarray*}

The bound names of a process, $\boundnames{P}$, are those names occurring in $P$
that are not free. For example, in $x?(y).0$, the name $x$ is free, while $y$ is bound.

\begin{mathpar}
  \inferrule* [lab=monoidal-laws] {} { P|Q \equiv Q|P \and P|0 \equiv P \and P|(Q|R) \equiv (P|Q)|R }
\end{mathpar}

\begin{mathpar}
  \inferrule* [lab=alpha-equivalence] {} { (x)P \equiv (y)P\{y/x\} \and y \not\in \freenames{P} }
\end{mathpar}

\begin{definition}
Then two processes, $P,Q$, are alpha-equivalent if $P = Q\{\vec{y}/\vec{x}\}$ for
some $\vec{x} \in \boundnames{Q},\vec{y} \in \boundnames{P}$, where $Q\{\vec{y}/\vec{x}\}$
denotes the capture-avoiding substitution of $\vec{y}$ for $\vec{x}$ in $Q$.
\end{definition}

\begin{definition}
  The {\em structural congruence} \cite{SangiorgiWalker} , $\equiv$,
  between processes is the least congruence containing
  alpha-equivalence, satisfying the abelian monoid laws
  (associativity, commutativity and $\pzero$ as identity) for parallel
  composition $|$ and for summation $+$.
\end{definition}

\subsection{Name equivalence}

We take name equivalence, written $\nameeq$, to be the smallest
equivalence relation generated by the following rules.

\begin{mathpar}
\inferrule*[lab=Quote-drop]
{ }
{ \quotep{@{x}} \nameeq x }

\inferrule*[lab=Struct-equiv]
{ P \scong Q }
{ \quotep{P} \nameeq \quotep{Q} }
\end{mathpar}

The astute reader will have noticed that the mutual recursion of names
and processes imposes a mutual recursion on alpha-equivalence and
structural equivalence via name-equivalence. Fortunately, all of this
works out pleasantly and we may calculate in the natural way, free of
concern. The reader interested in the details is referred to the
appendix \ref{appendix:rho_details}.

\subsection{Substitution}

We use $\Proc$ for the set of processes, $\QProc$ for the set of
names, and $\id{\{}\vec{y} / \vec{x} \id{\}}$ to denote partial maps,
$s : \QProc \rightarrow \QProc$. A map, $s$ lifts, uniquely, to a map
on process terms, $\widehat{s} : \Proc \rightarrow \Proc$ by the
following equations.

\begin{mathpar}
  (0) \psubstp{Q}{P} := 0 \\
  (R \juxtap S) \psubstp{Q}{P}
  :=    
  (R)\psubstp{Q}{P} \juxtap (S) \psubstp{Q}{P} \\
  (x?(y).R) \psubstp{Q}{P}    
  :=    
  (x)\substp{Q}{P} (z)\concat( (R \psubstn{z}{y}) \psubstp{Q}{P} ) \\
  (\lift{x}{R}) \psubstp{Q}{P}  
  :=
  \lift{(x)\substp{Q}{P}}{ R \psubstp{Q}{P} } \\
%   (\dropn{x})  \psubstp{Q}{P}       
%   := 
%   \left\{ 
%     \begin{array}{ccc} 
%       \dropn{\quotep{Q}} & & x \nameeq \quotep{P} \\
%       \dropn{x} & & otherwise \\
%     \end{array}
%   \right. 
  (\dropn{x})  \psubstp{Q}{P}       
  := 
  \left\{ 
    \begin{array}{ccc} 
      Q & & x \nameeq \quotep{P} \\
      \dropn{x} & & otherwise \\
    \end{array}
  \right.
\end{mathpar}
 

where

\begin{eqnarray}
  (x)\id{\{} \lpquote Q \rpquote / \lpquote P \rpquote \id{\}}            = 
  \left\{ 
    \begin{array}{ccc}
      \lpquote Q \rpquote & & x \nameeq \lpquote P \rpquote \\
      x & & otherwise \\
    \end{array}
  \right. \nonumber
\end{eqnarray}

and $z$ is chosen distinct from $\quotep{P}$, $\quotep{Q}$, the free
names in $Q$, and all the names in $R$. Our $\alpha$-equivalence will
be built in the standard way from this substitution.

\begin{remark}\label{rem:no_self_referential_names}
  One consequence of these definitions is that $\forall P. \quotep{P}
  \not\in \freenames{P}$.
\end{remark}

\subsection{ Dynamic quote: an example }

Anticipating something of what's to come, consider applying the
substitution, $\widehat{\id{\{}u / z \id{\}}}$, to the following pair
of processes, $\lift{w}{y!(z)}$ and $w[ \lpquote y!(z) \rpquote ]$.

\begin{eqnarray}
	\lift{w}{y!(z)}\widehat{\id{\{}u / z \id{\}}}
		& = &
		\lift{w}{y!(u)} \nonumber\\
	w[ \lpquote y!(z) \rpquote ] \widehat{ \id{\{}u / z \id{\}} }
		& = &
		w[ \lpquote y!(z) \rpquote ] \nonumber
\end{eqnarray}

Because the body of the process between quotes is impervious to
substitution, we get radically different answers. In fact, by
examining the first process in an input context,
e.g. $x?(z).\lift{w}{y!(z)}$, we see that the process under the lift
operator may be shaped by prefixed inputs binding a name inside it. In
this sense, the lift operator will be seen as a way to dynamically
construct processes before reifying them as names.

Finally equipped with these standard features we can present the
dynamics of the calculus.

\subsubsection{Operational semantics} 

Finally, we introduce the computational dynamics. What marks these
algebras as distinct from other more traditionally studied algebraic
structures, e.g. vector spaces or polynomial rings, is the manner in
which dynamics is captured. In traditional structures, dynamics is typically
expressed through morphisms between such structures, as in linear maps
between vector spaces or morphisms between rings. In algebras
associated with the semantics of computation, the dynamics is
expressed as part of the algebraic structure itself, through a
reduction reduction relation typically denoted by $\red$. Below, we
give a recursive presentation of this relation for the calculus used
in the encoding.

$\red \subseteq \pi \times \pi$
$\red : \pi \to \mathcal{P}(\pi)$

\begin{mathpar}
  \inferrule* [lab=Comm] { \textsf{match}( x_{src}, x_{trgt} ) } { x_{trgt}?(y)P \; | \; x_{src}!\langle {Q} \rangle \red P\{\quotep{Q}/y}\} }
  \and \\
  \inferrule* [lab=Par] {{P} \red {P}'} {{{P} | {Q}} \red {{P}' | {Q}}}
  \and
  \inferrule* [lab=Equiv]{{{P} \scong {P}'} \andalso {{P}' \red {Q}'} \andalso {{Q}' \scong {Q}}}{{P} \red {Q}}
\end{mathpar}

\begin{eqnarray*}
  match_{\equiv} (\quotep{P},\quotep{Q}) & := & P \equiv Q \\
  match_{\dagger}(\quotep{P},\quotep{Q}) & := & \forall R. P|Q \red^{*} R => R \red^{*} 0 \\
  match_{K}(\quotep{P},\quotep{Q}) & := & K \mbox{ for some context } K
\end{eqnarray*}

$u?(x)P | u!\langle Q \rangle \red P\{\quotep{Q}/x\}$

%We write $\wred$ for $\red^*$, and $P\red$ if $\exists Q $ such that $ P \red Q$.
We write $P\red$ if $\exists Q $ such that $ P \red Q$ and $P\not\red$, otherwise.

\section{Replication}

As mentioned before, it is known that replication (and hence
recursion) can be implemented in a higher-order process algebra
\cite{SangiorgiWalker}. As our first example of calculation with the
machinery thus far presented we give the construction explicitly in
the {\rhoc}.

\begin{eqnarray}
	D_{x} & := & \prefix{x}{y}{(\binpar{\outputp{x}{y}}{@{y}})} \nonumber\\
	\bangp_{x}{P} & := & \binpar{{x}!\langle{\binpar{D_{x}}{P}}\rangle}{D_{x}} \nonumber
\end{eqnarray}

\begin{eqnarray}
	\bangp_{x}{P} & & \nonumber\\
	=
	& {x}!\langle{(\prefix{x}{y}{(\outputp{x}{y} | @{y})) | P}}\rangle 
	      | \prefix{x}{y}{(\outputp{x}{y} | @{y})} & \nonumber\\
	\red
	& (\outputp{x}{y} | @{y})\substn{\quotep{(\prefix{x}{y}{(@{y} | \outputp{x}{y})) | P}}}{y} & \nonumber\\
	=
	& \outputp{x}{\quotep{(\prefix{x}{y}{(\outputp{x}{y} | @{y})) | P}}}
	  | {(\prefix{x}{y}{(\outputp{x}{y} | @{y})) | P}} & \nonumber\\
	\red
	& \ldots & \nonumber\\
	\red^*
	& P | P | \ldots & \nonumber
\end{eqnarray}

Of course, this encoding, as an implementation, runs away, unfolding
$\bangp{P}$ eagerly. A lazier and more implementable replication
operator, restricted to input-guarded processes, may be obtained as follows.

\begin{eqnarray}
\bangp{\prefix{u}{v}{P}} 
	:= 
	\binpar{\lift{x}{\prefix{u}{v}{(\binpar{D(x)}{P})}}}{D(x)} \nonumber
\end{eqnarray}

\begin{remark}
  Note that the lazier definition still does not deal with summation
  or mixed summation (i.e. sums over input and output). The reader is
  invited to construct definitions of replication that deal with these
  features. 

  Further, the definitions are parameterized in a name, $x$. Can you,
  gentle reader, make a definition that eliminates this parameter and
  guarantees no accidental interaction between the replication
  machinery and the process being replicated -- i.e. no accidental
  sharing of names used by the process to get its work done and the
  name(s) used by the replication to effect copying. This latter
  revision of the definition of replication is crucial to obtaining
  the expected identity $!!P \sim !P$.
\end{remark}

\begin{remark}\label{rem:paradoxical_combinator}
  The reader familiar with the lambda calculus will have noticed the
  similarity between $D$ and the paradoxical combinator.

  [Ed. note: the existence of this seems to suggest we have to be more
  restrictive on the set of processes and names we admit if we are to
  support no-cloning.]
\end{remark}

\subsubsection{Bisimulation}

The computational dynamics gives rise to another kind of equivalence,
the equivalence of computational behavior. As previously mentioned
this is typically captured \emph{via} some form of bisimulation.

% The notion we use in this paper is weak barbed bisimulation
% \cite{milner91polyadicpi}.

The notion we use in this paper is derived from weak barbed
bisimulation \cite{milner91polyadicpi}. 

\begin{definition}
An \emph{observation relation}, $\downarrow_{\mathcal N}$, over a set
of names, $\mathcal N$, is the smallest relation satisfying the rules
below.

\infrule[Out-barb]{y \in {\mathcal N}, \; x \nameeq y}
		  {\outputp{x}{v} \downarrow_{\mathcal N} x}
\infrule[Par-barb]{\mbox{$P\downarrow_{\mathcal N} x$ or $Q\downarrow_{\mathcal N} x$}}
		  {\binpar{P}{Q} \downarrow_{\mathcal N} x}

We write $P \Downarrow_{\mathcal N} x$ if there is $Q$ such that 
$P \wred Q$ and $Q \downarrow_{\mathcal N} x$.
\end{definition}

\begin{definition}
%\label{def.bbisim}
An  ${\mathcal N}$-\emph{barbed bisimulation} over a set of names, ${\mathcal N}$, is a symmetric binary relation 
${\mathcal S}_{\mathcal N}$ between agents such that $P\rel{S}_{\mathcal N}Q$ implies:
\begin{enumerate}
\item If $P \red P'$ then $Q \wred Q'$ and $P'\rel{S}_{\mathcal N} Q'$.
\item If $P\downarrow_{\mathcal N} x$, then $Q\Downarrow_{\mathcal N} x$.
\end{enumerate}
$P$ is ${\mathcal N}$-barbed bisimilar to $Q$, written
$P \wbbisim_{\mathcal N} Q$, if $P \rel{S}_{\mathcal N} Q$ for some ${\mathcal N}$-barbed bisimulation ${\mathcal S}_{\mathcal N}$.
\end{definition}

$\mathcal{R} \subseteq \pi \times \pi$

$P \mathcal{R} Q => \forall P'. P \red P' \Rightarrow \exists Q'. Q \red Q', P' \mathcal{R} Q'$

$P \vdash x \Rightarrow Q \vdash x$

\begin{mathpar}
  \inferrule*[lab=Out-barb]{x \nameeq y}{{y}!\langle{Q}\rangle \vdash x}
  \and
  \inferrule*[lab=Par-barb]{\mbox{$P\vdash x$ or $Q\vdash x$}}{\binpar{P}{Q} \vdash x}
\end{mathpar}

\subsubsection{Contexts}

One of the principle advantages of computational calculi like the
$\pi$-calculus is a well-defined notion of context,
contextual-equivalence and a correlation between
contextual-equivalence and notions of bisimulation. The notion of
context allows the decomposition of a process into (sub-)process and
its syntactic environment, its context. Thus, a context may be
thought of as a process with a ``hole'' (written $\Box$) in it. The
application of a context $M$ to a process $P$, written $M[P]$, is
tantamount to filling the hole in $M$ with $P$. In this paper we do
not need the full weight of this theory, but do make use of the notion
of context in the proof the main theorem. 

\begin{mathpar}
  \inferrule* [lab=summation] {} {{M_{M},M_{N}} \bc \Box \;|\; x.M_{A} \;|\; M_{M}+M_{N}}
  \and
  \inferrule* [lab=agent] {} {{M_{A}} \bc (\vec{x})M_{P} \;| \; \clift{P_0,\ldots,M_{P},\ldots,P_N}}
  \and \\
  \inferrule* [lab=process] {} {{M_{P}} \bc M_{N} \;| \;P|M_{P} }
\end{mathpar} 

\begin{mathpar}
  \inferrule* [lab=sychronization] {} {M_{N} \bc \Box \;|\; x?M_{F} \;|\; x!M_{C}}
  \and
  \inferrule* [lab=abstraction] {} {{M_{F}} \bc (x)M_{P} }
  \and
  \inferrule* [lab=concretion] {} {{M_{C}} \bc \langle M_{P} \rangle }
  \and \\
  \inferrule* [lab=process] {} {{M_{P}} \bc M_{N} \;| \;P|M_{P} }
\end{mathpar}

\begin{definition}[contextual application] Given a context $M$, and
  process $P$, we define the \emph{contextual application}, $M[P] :=
  M\{P/\Box\}$. That is, the contextual application of M to P is the
  substitution of $P$ for $\Box$ in $M$.
\end{definition}

$\meaningof{-} : L \to \mathcal{P}(\pi)$

\begin{mathpar}
  \inferrule* [lab=collection] {} {\meaningof{true} = \pi, \and \meaningof{~E} = \pi \setminus \meaningof{E}, \and \meaningof{E_{1} \& E_{2}} = \meaningof{E_{1}} \cap \meaningof{E_{2}}}
\end{mathpar}

\begin{mathpar}
  \inferrule* [lab=structure] {} {\meaningof{0} = \{ P \in \pi | P \equiv 0 \}, \and \\ \meaningof{E_1 | E_2} = \{ P \in \pi | P \equiv P_{1} | P_{2}, P_{1} \in \meaningof{E_{1}}, P_{2} \in \meaningof{E_2}\} }
\end{mathpar}

\begin{mathpar}
 \inferrule* [lab=behavior] {} {\meaningof{\langle a?b \rangle E} = \{ P \in \pi | P \equiv Q | u?(y)P', \\ \and \\\\ \and \\ \;\;\; u \in \meaningof{a}, \forall z.P'\{z/y\} \in \meaningof{E\{z/b\}}\}, \and \\ \meaningof{a!E} = \{ P \in \pi | P \equiv Q | x!\langle P' \rangle, x \in \meaningof{a} P' \in \meaningof{E}\} }
\end{mathpar}

\begin{mathpar}
 \inferrule* [lab=nominal] {} {\meaningof{\quotep{E}} = \{ \quotep{P} \in \quotep{\pi} | P \in \meaningof{E} \}, \and \meaningof{\quotep{P}} = \{ \quotep{Q} \in \quotep{\pi} | P \equiv Q \} \and \\ \meaningof{@\quotep{E}} = \{ P \in \pi | P \equiv @x, x \in \meaningof{E} \}}
\end{mathpar}

\begin{eqnarray*}
  \\
  \meaningof{-} : TS \to ST
\end{eqnarray*}

\begin{eqnarray*}
  \\
  L : TS \to ST
\end{eqnarray*}

\begin{eqnarray*}
  \\
  P \models E \iff P \in \meaningof{E}
\end{eqnarray*}

\begin{eqnarray*}
  P \approx_{L} Q \iff \forall E \in L. P \models E \iff Q \models E
\end{eqnarray*}

\begin{eqnarray*}
  P \approx_{K} Q
\end{eqnarray*}

\begin{eqnarray*}
  P \approx Q
\end{eqnarray*}

$\approx_{K} = \approx = \approx_{L}$

\subsubsection{Contextual duality}

Note that contexts extend the quotation operation to a family of
operations from processes to names. Given a context, $M$, we can
define a \emph{nominal context}, $\quotep{M}$ by $\quotep{M}[P] :=
\quotep{M[P]}$. To foreshadow what is to come we observe that these
operations enjoy a duality with processes very much like the duality
between vectors and maps from vectors to scalars.

Further, because the calculus is essentially higher-order, we have a
correspondence between contexts and processes. More specifically,
given a name $x$ and a context $M$ we can construct $M^{*}_{x}$ such
that 

\begin{mathpar}
  M^{*}_{x} | \lift{x}{P} \red M[P]
\end{mathpar}

namely,

\begin{mathpar}
  M^{*}_{x} := x?(u).M[\dropn{u}]
\end{mathpar}

The dependence of $M^{*}_{x}$ on a name makes it an abstraction, 

\begin{mathpar}
  M^{*} := (x)x?(u).M[\dropn{u}]
\end{mathpar}

\subsection{Additional notation}

It will sometimes be convenient to denote the process a name
quotes. We already have the notation $x = \quotep{P}$, but it will be
convenient to introduce an alternate notation, $\procn{x}$, when we
want to emphasize the connection to the use of the name. Note that, by
virtue of name equivalence, $\quotep{\procn{x}} \nameeq x$; so, the
notation is consistent with previous definitions.

Further, because names have structure it is possible to effect
substitutions on the basis of that structure. This means we need to
upgrade our notation for substitutions, which we accomplish by
adapting comprehension notation. Thus,

\begin{mathpar}
  P\{ y / x : x \in S \}
\end{mathpar}

is interpreted to mean the process derived from P by replacing (in a
capture-avoiding manner) each occurrence of $x$ in $S$ by $y$. For example,

\begin{mathpar}
  P\{ \quotep{\procn{x}|\procn{x}} / x : x \in \freenames{P} \}
\end{mathpar}

will replace each (occurrence) of a free name $x$ in $P$ by
$\quotep{\procn{x}|\procn{x}}$.

Also, we will avail ourselves of the notation $x^{L}$ and $x^{R}$ to
denote injections of a name into disjoint copies of the name
space. There are numerous ways to accomplish this. One example can be
found in \cite{MeredithR05}. This notation overloads to vectors of
names: $\vec{x}^{\pi} := (x_{i}^{\pi} \; : \; 0 \leq i < |\vec{x}| )$ where $\pi \in \{L,R\}$.

We also use $P^{\Box} := P|\Box$.

In \cite{MeredithR05} an interpretation of the new operator is
given. It turns out that there are several possible interpretations
all enjoying the requisite algebraic properties of the operator (see
\cite{milner91polyadicpi}). We will therefore make liberal use of
$(\nu\; \vec{x})P$.

% subsection the_syntax_and_semantics_of_the_notation_system (end)   

\input{qm2pi.qmops} 

\input{qm2pi.sterngerlach} 

\input{qm2pi.metric} 

% section concurrent_process_calculi (end)

%\input{qm2pi.proofsketch}

% section proof sketch (end)

%\input{qm2pi.slviaknots} 

% section spatial logic via knots (end)

\input{qm2pi.conclusion}

% section conclusion (end)

%\input{qm2pi.dtcodes} 

% section wiring algorithm (end)

\input{qm2pi.ack} 

% section acknowledgments (end)

\newpage


\bibliographystyle{plain}   
\bibliography{../../biblios/main.bib}

\input{qm2pi.rhodetails}

\end{document}

 

% section acknowledgments (end)

\newpage


\bibliographystyle{plain}   
\bibliography{../../biblios/main.bib}

\documentclass[12pt]{llncs}
%\documentclass{jktr}

\usepackage[pdftex]{hyperref}                   
\usepackage {listings}
\usepackage {mathpartir}
\usepackage{bcprules}
%\usepackage{listings}
                       
\usepackage{graphicx} 
%\usepackage[margins=2.5cm,nohead,nofoot]{geometry}
%\usepackage{geometry}
\usepackage{amsfonts}
\usepackage{amstext}
\usepackage{latexsym}
\usepackage{amssymb}
\usepackage{color}


%\include{myPreamble}
\include{qm2pi.local} 

%\ifpdf
%\usepackage[pdftex]{graphicx}
%\else
%\usepackage{graphicx}
%\fi

 % \ifpdf
%  \usepackage{pdfsync}
%  \if


%\title{Brief Article}
%\author{David F. Snyder}
%\author{L.G. Meredith}

%\address{Dept. of Math., Texas State University--San Marcos, San Marcos, TX 78666}
       
\pagestyle{empty}


\begin{document}

\lstset{language=[Objective]Caml,frame=shadowbox}

\input{qm2pi.front}

% section front matter (end)

\input{qm2pi.intro} 
 
% section introduction (end)

% \input{qm2pi.knotations} 

% section notation (end)

\input{qm2pi.process.calculi} 

% section concurrent_process_calculi_and_spatial_logics_ (end)
    
%\input{qm2pi.knots2pi} 

%\input{qm2pi.trefoil} 

%\input{qm2pi.mainthm} 

% subsection basic_interpretation (end)

%\input{qm2pi.rho.presentation} 
\subsection{The syntax and semantics of the notation system}\label{sub:the_syntax_and_semantics_of_the_notation_system} % (fold)

We now summarize a technical presentation of the calculus that
embodies our theory of dynamics. The typical presentation of such a
calculus follows the style of giving generators and relations on
them. The grammar, below, describing term constructors, freely
generates the set of processes, $\Proc$. This set is then quotiented
by a relation known as structural congruence and it is over this set
that the notion of dynamics is expressed. This presentation is
essentially that of \cite{MeredithR05} with the addition of
polyadicity and summation. For readability we have relegated some of
the technical subtleties to an appendix.

\subsubsection{Process grammar}\label{subsub:process_grammar}

\begin{mathpar}
  \inferrule* [lab=synchronization] {} {{M} \bc \pzero \;|\; x?F \;|\; x!C }
  \and
  \inferrule* [lab=abstraction] {} {{F} \bc (x)P}
  \and
  \inferrule* [lab=concretion] {} {{C} \bc \langle Q \rangle}
  \and
  \inferrule* [lab=process] {} {{P,Q} \bc M \;| \;P|Q \;|\; @{x}}
  \and
  \inferrule* [lab=name] {} {{x} \bc \quotep{P}}
\end{mathpar} 

Note that $\vec{x}$ (resp. $\vec{P}$) denotes a vector of names
(resp. processes) of length $|\vec{x}|$ (resp. $|\vec{P}|$). We adopt
the following useful abbreviations.

\begin{mathpar}
   x?(\vec{y}).P := x.(\vec{y})P \and  x\clift{\vec{P}} := x.\clift{\vec{P}}
   \and x!(y) := \lift{x}{\dropn{y}}
   \and \Pi_{i=0}^{n-1}P_i := P_0 | \ldots | P_{n-1}
\end{mathpar}

\subsubsection{Structural congruence}

\paragraph{Free and bound names and alpha-equivalence.} At the
core of structural equivalence is alpha-equivalence which identifies
process that are the same up to a change of variable. Formally, we
recognize the distinction between free and bound names. The free names
of a process, $\freenames{P}$, may be calculated recursively as
follows:

\begin{mathpar}
\freenames{\pzero} := \emptyset
  \and \\
  \freenames{x?(y).P} := \{ x \} \cup (\freenames{P} \setminus \{ y \})
  \and 
  \freenames{x!\langle P \rangle} := \{ x \} \cup \{ P \} 
  \and \\
  \freenames{P|Q} := \freenames{P} \cup \freenames{Q}
  \and \\
  \freenames{@{x}} := \{ x \}
\end{mathpar}

$\pi$
$\quotep{\pi}$

$\freenames{-} : \pi \to \mathcal{P}(\quotep{\pi})$

\begin{eqnarray*}
  \freenames{\pzero} & := & \emptyset \\
  \freenames{x?(y).P} & := & \{ x \} \cup (\freenames{P} \setminus \{ y \}) \\
  \freenames{x!\langle P \rangle} & := & \{ x \} \cup \{ P \} \\
  \freenames{P|Q} & := & \freenames{P} \cup \freenames{Q} \\
  \freenames{\dropn{x}} & := & \{ x \}
\end{eqnarray*}

The bound names of a process, $\boundnames{P}$, are those names occurring in $P$
that are not free. For example, in $x?(y).0$, the name $x$ is free, while $y$ is bound.

\begin{mathpar}
  \inferrule* [lab=monoidal-laws] {} { P|Q \equiv Q|P \and P|0 \equiv P \and P|(Q|R) \equiv (P|Q)|R }
\end{mathpar}

\begin{mathpar}
  \inferrule* [lab=alpha-equivalence] {} { (x)P \equiv (y)P\{y/x\} \and y \not\in \freenames{P} }
\end{mathpar}

\begin{definition}
Then two processes, $P,Q$, are alpha-equivalent if $P = Q\{\vec{y}/\vec{x}\}$ for
some $\vec{x} \in \boundnames{Q},\vec{y} \in \boundnames{P}$, where $Q\{\vec{y}/\vec{x}\}$
denotes the capture-avoiding substitution of $\vec{y}$ for $\vec{x}$ in $Q$.
\end{definition}

\begin{definition}
  The {\em structural congruence} \cite{SangiorgiWalker} , $\equiv$,
  between processes is the least congruence containing
  alpha-equivalence, satisfying the abelian monoid laws
  (associativity, commutativity and $\pzero$ as identity) for parallel
  composition $|$ and for summation $+$.
\end{definition}

\subsection{Name equivalence}

We take name equivalence, written $\nameeq$, to be the smallest
equivalence relation generated by the following rules.

\begin{mathpar}
\inferrule*[lab=Quote-drop]
{ }
{ \quotep{@{x}} \nameeq x }

\inferrule*[lab=Struct-equiv]
{ P \scong Q }
{ \quotep{P} \nameeq \quotep{Q} }
\end{mathpar}

The astute reader will have noticed that the mutual recursion of names
and processes imposes a mutual recursion on alpha-equivalence and
structural equivalence via name-equivalence. Fortunately, all of this
works out pleasantly and we may calculate in the natural way, free of
concern. The reader interested in the details is referred to the
appendix \ref{appendix:rho_details}.

\subsection{Substitution}

We use $\Proc$ for the set of processes, $\QProc$ for the set of
names, and $\id{\{}\vec{y} / \vec{x} \id{\}}$ to denote partial maps,
$s : \QProc \rightarrow \QProc$. A map, $s$ lifts, uniquely, to a map
on process terms, $\widehat{s} : \Proc \rightarrow \Proc$ by the
following equations.

\begin{mathpar}
  (0) \psubstp{Q}{P} := 0 \\
  (R \juxtap S) \psubstp{Q}{P}
  :=    
  (R)\psubstp{Q}{P} \juxtap (S) \psubstp{Q}{P} \\
  (x?(y).R) \psubstp{Q}{P}    
  :=    
  (x)\substp{Q}{P} (z)\concat( (R \psubstn{z}{y}) \psubstp{Q}{P} ) \\
  (\lift{x}{R}) \psubstp{Q}{P}  
  :=
  \lift{(x)\substp{Q}{P}}{ R \psubstp{Q}{P} } \\
%   (\dropn{x})  \psubstp{Q}{P}       
%   := 
%   \left\{ 
%     \begin{array}{ccc} 
%       \dropn{\quotep{Q}} & & x \nameeq \quotep{P} \\
%       \dropn{x} & & otherwise \\
%     \end{array}
%   \right. 
  (\dropn{x})  \psubstp{Q}{P}       
  := 
  \left\{ 
    \begin{array}{ccc} 
      Q & & x \nameeq \quotep{P} \\
      \dropn{x} & & otherwise \\
    \end{array}
  \right.
\end{mathpar}
 

where

\begin{eqnarray}
  (x)\id{\{} \lpquote Q \rpquote / \lpquote P \rpquote \id{\}}            = 
  \left\{ 
    \begin{array}{ccc}
      \lpquote Q \rpquote & & x \nameeq \lpquote P \rpquote \\
      x & & otherwise \\
    \end{array}
  \right. \nonumber
\end{eqnarray}

and $z$ is chosen distinct from $\quotep{P}$, $\quotep{Q}$, the free
names in $Q$, and all the names in $R$. Our $\alpha$-equivalence will
be built in the standard way from this substitution.

\begin{remark}\label{rem:no_self_referential_names}
  One consequence of these definitions is that $\forall P. \quotep{P}
  \not\in \freenames{P}$.
\end{remark}

\subsection{ Dynamic quote: an example }

Anticipating something of what's to come, consider applying the
substitution, $\widehat{\id{\{}u / z \id{\}}}$, to the following pair
of processes, $\lift{w}{y!(z)}$ and $w[ \lpquote y!(z) \rpquote ]$.

\begin{eqnarray}
	\lift{w}{y!(z)}\widehat{\id{\{}u / z \id{\}}}
		& = &
		\lift{w}{y!(u)} \nonumber\\
	w[ \lpquote y!(z) \rpquote ] \widehat{ \id{\{}u / z \id{\}} }
		& = &
		w[ \lpquote y!(z) \rpquote ] \nonumber
\end{eqnarray}

Because the body of the process between quotes is impervious to
substitution, we get radically different answers. In fact, by
examining the first process in an input context,
e.g. $x?(z).\lift{w}{y!(z)}$, we see that the process under the lift
operator may be shaped by prefixed inputs binding a name inside it. In
this sense, the lift operator will be seen as a way to dynamically
construct processes before reifying them as names.

Finally equipped with these standard features we can present the
dynamics of the calculus.

\subsubsection{Operational semantics} 

Finally, we introduce the computational dynamics. What marks these
algebras as distinct from other more traditionally studied algebraic
structures, e.g. vector spaces or polynomial rings, is the manner in
which dynamics is captured. In traditional structures, dynamics is typically
expressed through morphisms between such structures, as in linear maps
between vector spaces or morphisms between rings. In algebras
associated with the semantics of computation, the dynamics is
expressed as part of the algebraic structure itself, through a
reduction reduction relation typically denoted by $\red$. Below, we
give a recursive presentation of this relation for the calculus used
in the encoding.

$\red \subseteq \pi \times \pi$
$\red : \pi \to \mathcal{P}(\pi)$

\begin{mathpar}
  \inferrule* [lab=Comm] { \textsf{match}( x_{src}, x_{trgt} ) } { x_{trgt}?(y)P \; | \; x_{src}!\langle {Q} \rangle \red P\{\quotep{Q}/y}\} }
  \and \\
  \inferrule* [lab=Par] {{P} \red {P}'} {{{P} | {Q}} \red {{P}' | {Q}}}
  \and
  \inferrule* [lab=Equiv]{{{P} \scong {P}'} \andalso {{P}' \red {Q}'} \andalso {{Q}' \scong {Q}}}{{P} \red {Q}}
\end{mathpar}

\begin{eqnarray*}
  match_{\equiv} (\quotep{P},\quotep{Q}) & := & P \equiv Q \\
  match_{\dagger}(\quotep{P},\quotep{Q}) & := & \forall R. P|Q \red^{*} R => R \red^{*} 0 \\
  match_{K}(\quotep{P},\quotep{Q}) & := & K \mbox{ for some context } K
\end{eqnarray*}

$u?(x)P | u!\langle Q \rangle \red P\{\quotep{Q}/x\}$

%We write $\wred$ for $\red^*$, and $P\red$ if $\exists Q $ such that $ P \red Q$.
We write $P\red$ if $\exists Q $ such that $ P \red Q$ and $P\not\red$, otherwise.

\section{Replication}

As mentioned before, it is known that replication (and hence
recursion) can be implemented in a higher-order process algebra
\cite{SangiorgiWalker}. As our first example of calculation with the
machinery thus far presented we give the construction explicitly in
the {\rhoc}.

\begin{eqnarray}
	D_{x} & := & \prefix{x}{y}{(\binpar{\outputp{x}{y}}{@{y}})} \nonumber\\
	\bangp_{x}{P} & := & \binpar{{x}!\langle{\binpar{D_{x}}{P}}\rangle}{D_{x}} \nonumber
\end{eqnarray}

\begin{eqnarray}
	\bangp_{x}{P} & & \nonumber\\
	=
	& {x}!\langle{(\prefix{x}{y}{(\outputp{x}{y} | @{y})) | P}}\rangle 
	      | \prefix{x}{y}{(\outputp{x}{y} | @{y})} & \nonumber\\
	\red
	& (\outputp{x}{y} | @{y})\substn{\quotep{(\prefix{x}{y}{(@{y} | \outputp{x}{y})) | P}}}{y} & \nonumber\\
	=
	& \outputp{x}{\quotep{(\prefix{x}{y}{(\outputp{x}{y} | @{y})) | P}}}
	  | {(\prefix{x}{y}{(\outputp{x}{y} | @{y})) | P}} & \nonumber\\
	\red
	& \ldots & \nonumber\\
	\red^*
	& P | P | \ldots & \nonumber
\end{eqnarray}

Of course, this encoding, as an implementation, runs away, unfolding
$\bangp{P}$ eagerly. A lazier and more implementable replication
operator, restricted to input-guarded processes, may be obtained as follows.

\begin{eqnarray}
\bangp{\prefix{u}{v}{P}} 
	:= 
	\binpar{\lift{x}{\prefix{u}{v}{(\binpar{D(x)}{P})}}}{D(x)} \nonumber
\end{eqnarray}

\begin{remark}
  Note that the lazier definition still does not deal with summation
  or mixed summation (i.e. sums over input and output). The reader is
  invited to construct definitions of replication that deal with these
  features. 

  Further, the definitions are parameterized in a name, $x$. Can you,
  gentle reader, make a definition that eliminates this parameter and
  guarantees no accidental interaction between the replication
  machinery and the process being replicated -- i.e. no accidental
  sharing of names used by the process to get its work done and the
  name(s) used by the replication to effect copying. This latter
  revision of the definition of replication is crucial to obtaining
  the expected identity $!!P \sim !P$.
\end{remark}

\begin{remark}\label{rem:paradoxical_combinator}
  The reader familiar with the lambda calculus will have noticed the
  similarity between $D$ and the paradoxical combinator.

  [Ed. note: the existence of this seems to suggest we have to be more
  restrictive on the set of processes and names we admit if we are to
  support no-cloning.]
\end{remark}

\subsubsection{Bisimulation}

The computational dynamics gives rise to another kind of equivalence,
the equivalence of computational behavior. As previously mentioned
this is typically captured \emph{via} some form of bisimulation.

% The notion we use in this paper is weak barbed bisimulation
% \cite{milner91polyadicpi}.

The notion we use in this paper is derived from weak barbed
bisimulation \cite{milner91polyadicpi}. 

\begin{definition}
An \emph{observation relation}, $\downarrow_{\mathcal N}$, over a set
of names, $\mathcal N$, is the smallest relation satisfying the rules
below.

\infrule[Out-barb]{y \in {\mathcal N}, \; x \nameeq y}
		  {\outputp{x}{v} \downarrow_{\mathcal N} x}
\infrule[Par-barb]{\mbox{$P\downarrow_{\mathcal N} x$ or $Q\downarrow_{\mathcal N} x$}}
		  {\binpar{P}{Q} \downarrow_{\mathcal N} x}

We write $P \Downarrow_{\mathcal N} x$ if there is $Q$ such that 
$P \wred Q$ and $Q \downarrow_{\mathcal N} x$.
\end{definition}

\begin{definition}
%\label{def.bbisim}
An  ${\mathcal N}$-\emph{barbed bisimulation} over a set of names, ${\mathcal N}$, is a symmetric binary relation 
${\mathcal S}_{\mathcal N}$ between agents such that $P\rel{S}_{\mathcal N}Q$ implies:
\begin{enumerate}
\item If $P \red P'$ then $Q \wred Q'$ and $P'\rel{S}_{\mathcal N} Q'$.
\item If $P\downarrow_{\mathcal N} x$, then $Q\Downarrow_{\mathcal N} x$.
\end{enumerate}
$P$ is ${\mathcal N}$-barbed bisimilar to $Q$, written
$P \wbbisim_{\mathcal N} Q$, if $P \rel{S}_{\mathcal N} Q$ for some ${\mathcal N}$-barbed bisimulation ${\mathcal S}_{\mathcal N}$.
\end{definition}

$\mathcal{R} \subseteq \pi \times \pi$

$P \mathcal{R} Q => \forall P'. P \red P' \Rightarrow \exists Q'. Q \red Q', P' \mathcal{R} Q'$

$P \vdash x \Rightarrow Q \vdash x$

\begin{mathpar}
  \inferrule*[lab=Out-barb]{x \nameeq y}{{y}!\langle{Q}\rangle \vdash x}
  \and
  \inferrule*[lab=Par-barb]{\mbox{$P\vdash x$ or $Q\vdash x$}}{\binpar{P}{Q} \vdash x}
\end{mathpar}

\subsubsection{Contexts}

One of the principle advantages of computational calculi like the
$\pi$-calculus is a well-defined notion of context,
contextual-equivalence and a correlation between
contextual-equivalence and notions of bisimulation. The notion of
context allows the decomposition of a process into (sub-)process and
its syntactic environment, its context. Thus, a context may be
thought of as a process with a ``hole'' (written $\Box$) in it. The
application of a context $M$ to a process $P$, written $M[P]$, is
tantamount to filling the hole in $M$ with $P$. In this paper we do
not need the full weight of this theory, but do make use of the notion
of context in the proof the main theorem. 

\begin{mathpar}
  \inferrule* [lab=summation] {} {{M_{M},M_{N}} \bc \Box \;|\; x.M_{A} \;|\; M_{M}+M_{N}}
  \and
  \inferrule* [lab=agent] {} {{M_{A}} \bc (\vec{x})M_{P} \;| \; \clift{P_0,\ldots,M_{P},\ldots,P_N}}
  \and \\
  \inferrule* [lab=process] {} {{M_{P}} \bc M_{N} \;| \;P|M_{P} }
\end{mathpar} 

\begin{mathpar}
  \inferrule* [lab=sychronization] {} {M_{N} \bc \Box \;|\; x?M_{F} \;|\; x!M_{C}}
  \and
  \inferrule* [lab=abstraction] {} {{M_{F}} \bc (x)M_{P} }
  \and
  \inferrule* [lab=concretion] {} {{M_{C}} \bc \langle M_{P} \rangle }
  \and \\
  \inferrule* [lab=process] {} {{M_{P}} \bc M_{N} \;| \;P|M_{P} }
\end{mathpar}

\begin{definition}[contextual application] Given a context $M$, and
  process $P$, we define the \emph{contextual application}, $M[P] :=
  M\{P/\Box\}$. That is, the contextual application of M to P is the
  substitution of $P$ for $\Box$ in $M$.
\end{definition}

$\meaningof{-} : L \to \mathcal{P}(\pi)$

\begin{mathpar}
  \inferrule* [lab=collection] {} {\meaningof{true} = \pi, \and \meaningof{~E} = \pi \setminus \meaningof{E}, \and \meaningof{E_{1} \& E_{2}} = \meaningof{E_{1}} \cap \meaningof{E_{2}}}
\end{mathpar}

\begin{mathpar}
  \inferrule* [lab=structure] {} {\meaningof{0} = \{ P \in \pi | P \equiv 0 \}, \and \\ \meaningof{E_1 | E_2} = \{ P \in \pi | P \equiv P_{1} | P_{2}, P_{1} \in \meaningof{E_{1}}, P_{2} \in \meaningof{E_2}\} }
\end{mathpar}

\begin{mathpar}
 \inferrule* [lab=behavior] {} {\meaningof{\langle a?b \rangle E} = \{ P \in \pi | P \equiv Q | u?(y)P', \\ \and \\\\ \and \\ \;\;\; u \in \meaningof{a}, \forall z.P'\{z/y\} \in \meaningof{E\{z/b\}}\}, \and \\ \meaningof{a!E} = \{ P \in \pi | P \equiv Q | x!\langle P' \rangle, x \in \meaningof{a} P' \in \meaningof{E}\} }
\end{mathpar}

\begin{mathpar}
 \inferrule* [lab=nominal] {} {\meaningof{\quotep{E}} = \{ \quotep{P} \in \quotep{\pi} | P \in \meaningof{E} \}, \and \meaningof{\quotep{P}} = \{ \quotep{Q} \in \quotep{\pi} | P \equiv Q \} \and \\ \meaningof{@\quotep{E}} = \{ P \in \pi | P \equiv @x, x \in \meaningof{E} \}}
\end{mathpar}

\begin{eqnarray*}
  \\
  \meaningof{-} : TS \to ST
\end{eqnarray*}

\begin{eqnarray*}
  \\
  L : TS \to ST
\end{eqnarray*}

\begin{eqnarray*}
  \\
  P \models E \iff P \in \meaningof{E}
\end{eqnarray*}

\begin{eqnarray*}
  P \approx_{L} Q \iff \forall E \in L. P \models E \iff Q \models E
\end{eqnarray*}

\begin{eqnarray*}
  P \approx_{K} Q
\end{eqnarray*}

\begin{eqnarray*}
  P \approx Q
\end{eqnarray*}

$\approx_{K} = \approx = \approx_{L}$

\subsubsection{Contextual duality}

Note that contexts extend the quotation operation to a family of
operations from processes to names. Given a context, $M$, we can
define a \emph{nominal context}, $\quotep{M}$ by $\quotep{M}[P] :=
\quotep{M[P]}$. To foreshadow what is to come we observe that these
operations enjoy a duality with processes very much like the duality
between vectors and maps from vectors to scalars.

Further, because the calculus is essentially higher-order, we have a
correspondence between contexts and processes. More specifically,
given a name $x$ and a context $M$ we can construct $M^{*}_{x}$ such
that 

\begin{mathpar}
  M^{*}_{x} | \lift{x}{P} \red M[P]
\end{mathpar}

namely,

\begin{mathpar}
  M^{*}_{x} := x?(u).M[\dropn{u}]
\end{mathpar}

The dependence of $M^{*}_{x}$ on a name makes it an abstraction, 

\begin{mathpar}
  M^{*} := (x)x?(u).M[\dropn{u}]
\end{mathpar}

\subsection{Additional notation}

It will sometimes be convenient to denote the process a name
quotes. We already have the notation $x = \quotep{P}$, but it will be
convenient to introduce an alternate notation, $\procn{x}$, when we
want to emphasize the connection to the use of the name. Note that, by
virtue of name equivalence, $\quotep{\procn{x}} \nameeq x$; so, the
notation is consistent with previous definitions.

Further, because names have structure it is possible to effect
substitutions on the basis of that structure. This means we need to
upgrade our notation for substitutions, which we accomplish by
adapting comprehension notation. Thus,

\begin{mathpar}
  P\{ y / x : x \in S \}
\end{mathpar}

is interpreted to mean the process derived from P by replacing (in a
capture-avoiding manner) each occurrence of $x$ in $S$ by $y$. For example,

\begin{mathpar}
  P\{ \quotep{\procn{x}|\procn{x}} / x : x \in \freenames{P} \}
\end{mathpar}

will replace each (occurrence) of a free name $x$ in $P$ by
$\quotep{\procn{x}|\procn{x}}$.

Also, we will avail ourselves of the notation $x^{L}$ and $x^{R}$ to
denote injections of a name into disjoint copies of the name
space. There are numerous ways to accomplish this. One example can be
found in \cite{MeredithR05}. This notation overloads to vectors of
names: $\vec{x}^{\pi} := (x_{i}^{\pi} \; : \; 0 \leq i < |\vec{x}| )$ where $\pi \in \{L,R\}$.

We also use $P^{\Box} := P|\Box$.

In \cite{MeredithR05} an interpretation of the new operator is
given. It turns out that there are several possible interpretations
all enjoying the requisite algebraic properties of the operator (see
\cite{milner91polyadicpi}). We will therefore make liberal use of
$(\nu\; \vec{x})P$.

% subsection the_syntax_and_semantics_of_the_notation_system (end)   

\input{qm2pi.qmops} 

\input{qm2pi.sterngerlach} 

\input{qm2pi.metric} 

% section concurrent_process_calculi (end)

%\input{qm2pi.proofsketch}

% section proof sketch (end)

%\input{qm2pi.slviaknots} 

% section spatial logic via knots (end)

\input{qm2pi.conclusion}

% section conclusion (end)

%\input{qm2pi.dtcodes} 

% section wiring algorithm (end)

\input{qm2pi.ack} 

% section acknowledgments (end)

\newpage


\bibliographystyle{plain}   
\bibliography{../../biblios/main.bib}

\input{qm2pi.rhodetails}

\end{document}



\end{document}

 

% subsection basic_interpretation (end)

%\input{qm2pi.rho.presentation} 
\subsection{The syntax and semantics of the notation system}\label{sub:the_syntax_and_semantics_of_the_notation_system} % (fold)

We now summarize a technical presentation of the calculus that
embodies our theory of dynamics. The typical presentation of such a
calculus follows the style of giving generators and relations on
them. The grammar, below, describing term constructors, freely
generates the set of processes, $\Proc$. This set is then quotiented
by a relation known as structural congruence and it is over this set
that the notion of dynamics is expressed. This presentation is
essentially that of \cite{MeredithR05} with the addition of
polyadicity and summation. For readability we have relegated some of
the technical subtleties to an appendix.

\subsubsection{Process grammar}\label{subsub:process_grammar}

\begin{mathpar}
  \inferrule* [lab=synchronization] {} {{M} \bc \pzero \;|\; x?F \;|\; x!C }
  \and
  \inferrule* [lab=abstraction] {} {{F} \bc (x)P}
  \and
  \inferrule* [lab=concretion] {} {{C} \bc \langle Q \rangle}
  \and
  \inferrule* [lab=process] {} {{P,Q} \bc M \;| \;P|Q \;|\; @{x}}
  \and
  \inferrule* [lab=name] {} {{x} \bc \quotep{P}}
\end{mathpar} 

Note that $\vec{x}$ (resp. $\vec{P}$) denotes a vector of names
(resp. processes) of length $|\vec{x}|$ (resp. $|\vec{P}|$). We adopt
the following useful abbreviations.

\begin{mathpar}
   x?(\vec{y}).P := x.(\vec{y})P \and  x\clift{\vec{P}} := x.\clift{\vec{P}}
   \and x!(y) := \lift{x}{\dropn{y}}
   \and \Pi_{i=0}^{n-1}P_i := P_0 | \ldots | P_{n-1}
\end{mathpar}

\subsubsection{Structural congruence}

\paragraph{Free and bound names and alpha-equivalence.} At the
core of structural equivalence is alpha-equivalence which identifies
process that are the same up to a change of variable. Formally, we
recognize the distinction between free and bound names. The free names
of a process, $\freenames{P}$, may be calculated recursively as
follows:

\begin{mathpar}
\freenames{\pzero} := \emptyset
  \and \\
  \freenames{x?(y).P} := \{ x \} \cup (\freenames{P} \setminus \{ y \})
  \and 
  \freenames{x!\langle P \rangle} := \{ x \} \cup \{ P \} 
  \and \\
  \freenames{P|Q} := \freenames{P} \cup \freenames{Q}
  \and \\
  \freenames{@{x}} := \{ x \}
\end{mathpar}

$\pi$
$\quotep{\pi}$

$\freenames{-} : \pi \to \mathcal{P}(\quotep{\pi})$

\begin{eqnarray*}
  \freenames{\pzero} & := & \emptyset \\
  \freenames{x?(y).P} & := & \{ x \} \cup (\freenames{P} \setminus \{ y \}) \\
  \freenames{x!\langle P \rangle} & := & \{ x \} \cup \{ P \} \\
  \freenames{P|Q} & := & \freenames{P} \cup \freenames{Q} \\
  \freenames{\dropn{x}} & := & \{ x \}
\end{eqnarray*}

The bound names of a process, $\boundnames{P}$, are those names occurring in $P$
that are not free. For example, in $x?(y).0$, the name $x$ is free, while $y$ is bound.

\begin{mathpar}
  \inferrule* [lab=monoidal-laws] {} { P|Q \equiv Q|P \and P|0 \equiv P \and P|(Q|R) \equiv (P|Q)|R }
\end{mathpar}

\begin{mathpar}
  \inferrule* [lab=alpha-equivalence] {} { (x)P \equiv (y)P\{y/x\} \and y \not\in \freenames{P} }
\end{mathpar}

\begin{definition}
Then two processes, $P,Q$, are alpha-equivalent if $P = Q\{\vec{y}/\vec{x}\}$ for
some $\vec{x} \in \boundnames{Q},\vec{y} \in \boundnames{P}$, where $Q\{\vec{y}/\vec{x}\}$
denotes the capture-avoiding substitution of $\vec{y}$ for $\vec{x}$ in $Q$.
\end{definition}

\begin{definition}
  The {\em structural congruence} \cite{SangiorgiWalker} , $\equiv$,
  between processes is the least congruence containing
  alpha-equivalence, satisfying the abelian monoid laws
  (associativity, commutativity and $\pzero$ as identity) for parallel
  composition $|$ and for summation $+$.
\end{definition}

\subsection{Name equivalence}

We take name equivalence, written $\nameeq$, to be the smallest
equivalence relation generated by the following rules.

\begin{mathpar}
\inferrule*[lab=Quote-drop]
{ }
{ \quotep{@{x}} \nameeq x }

\inferrule*[lab=Struct-equiv]
{ P \scong Q }
{ \quotep{P} \nameeq \quotep{Q} }
\end{mathpar}

The astute reader will have noticed that the mutual recursion of names
and processes imposes a mutual recursion on alpha-equivalence and
structural equivalence via name-equivalence. Fortunately, all of this
works out pleasantly and we may calculate in the natural way, free of
concern. The reader interested in the details is referred to the
appendix \ref{appendix:rho_details}.

\subsection{Substitution}

We use $\Proc$ for the set of processes, $\QProc$ for the set of
names, and $\id{\{}\vec{y} / \vec{x} \id{\}}$ to denote partial maps,
$s : \QProc \rightarrow \QProc$. A map, $s$ lifts, uniquely, to a map
on process terms, $\widehat{s} : \Proc \rightarrow \Proc$ by the
following equations.

\begin{mathpar}
  (0) \psubstp{Q}{P} := 0 \\
  (R \juxtap S) \psubstp{Q}{P}
  :=    
  (R)\psubstp{Q}{P} \juxtap (S) \psubstp{Q}{P} \\
  (x?(y).R) \psubstp{Q}{P}    
  :=    
  (x)\substp{Q}{P} (z)\concat( (R \psubstn{z}{y}) \psubstp{Q}{P} ) \\
  (\lift{x}{R}) \psubstp{Q}{P}  
  :=
  \lift{(x)\substp{Q}{P}}{ R \psubstp{Q}{P} } \\
%   (\dropn{x})  \psubstp{Q}{P}       
%   := 
%   \left\{ 
%     \begin{array}{ccc} 
%       \dropn{\quotep{Q}} & & x \nameeq \quotep{P} \\
%       \dropn{x} & & otherwise \\
%     \end{array}
%   \right. 
  (\dropn{x})  \psubstp{Q}{P}       
  := 
  \left\{ 
    \begin{array}{ccc} 
      Q & & x \nameeq \quotep{P} \\
      \dropn{x} & & otherwise \\
    \end{array}
  \right.
\end{mathpar}
 

where

\begin{eqnarray}
  (x)\id{\{} \lpquote Q \rpquote / \lpquote P \rpquote \id{\}}            = 
  \left\{ 
    \begin{array}{ccc}
      \lpquote Q \rpquote & & x \nameeq \lpquote P \rpquote \\
      x & & otherwise \\
    \end{array}
  \right. \nonumber
\end{eqnarray}

and $z$ is chosen distinct from $\quotep{P}$, $\quotep{Q}$, the free
names in $Q$, and all the names in $R$. Our $\alpha$-equivalence will
be built in the standard way from this substitution.

\begin{remark}\label{rem:no_self_referential_names}
  One consequence of these definitions is that $\forall P. \quotep{P}
  \not\in \freenames{P}$.
\end{remark}

\subsection{ Dynamic quote: an example }

Anticipating something of what's to come, consider applying the
substitution, $\widehat{\id{\{}u / z \id{\}}}$, to the following pair
of processes, $\lift{w}{y!(z)}$ and $w[ \lpquote y!(z) \rpquote ]$.

\begin{eqnarray}
	\lift{w}{y!(z)}\widehat{\id{\{}u / z \id{\}}}
		& = &
		\lift{w}{y!(u)} \nonumber\\
	w[ \lpquote y!(z) \rpquote ] \widehat{ \id{\{}u / z \id{\}} }
		& = &
		w[ \lpquote y!(z) \rpquote ] \nonumber
\end{eqnarray}

Because the body of the process between quotes is impervious to
substitution, we get radically different answers. In fact, by
examining the first process in an input context,
e.g. $x?(z).\lift{w}{y!(z)}$, we see that the process under the lift
operator may be shaped by prefixed inputs binding a name inside it. In
this sense, the lift operator will be seen as a way to dynamically
construct processes before reifying them as names.

Finally equipped with these standard features we can present the
dynamics of the calculus.

\subsubsection{Operational semantics} 

Finally, we introduce the computational dynamics. What marks these
algebras as distinct from other more traditionally studied algebraic
structures, e.g. vector spaces or polynomial rings, is the manner in
which dynamics is captured. In traditional structures, dynamics is typically
expressed through morphisms between such structures, as in linear maps
between vector spaces or morphisms between rings. In algebras
associated with the semantics of computation, the dynamics is
expressed as part of the algebraic structure itself, through a
reduction reduction relation typically denoted by $\red$. Below, we
give a recursive presentation of this relation for the calculus used
in the encoding.

$\red \subseteq \pi \times \pi$
$\red : \pi \to \mathcal{P}(\pi)$

\begin{mathpar}
  \inferrule* [lab=Comm] { \textsf{match}( x_{src}, x_{trgt} ) } { x_{trgt}?(y)P \; | \; x_{src}!\langle {Q} \rangle \red P\{\quotep{Q}/y}\} }
  \and \\
  \inferrule* [lab=Par] {{P} \red {P}'} {{{P} | {Q}} \red {{P}' | {Q}}}
  \and
  \inferrule* [lab=Equiv]{{{P} \scong {P}'} \andalso {{P}' \red {Q}'} \andalso {{Q}' \scong {Q}}}{{P} \red {Q}}
\end{mathpar}

\begin{eqnarray*}
  match_{\equiv} (\quotep{P},\quotep{Q}) & := & P \equiv Q \\
  match_{\dagger}(\quotep{P},\quotep{Q}) & := & \forall R. P|Q \red^{*} R => R \red^{*} 0 \\
  match_{K}(\quotep{P},\quotep{Q}) & := & K \mbox{ for some context } K
\end{eqnarray*}

$u?(x)P | u!\langle Q \rangle \red P\{\quotep{Q}/x\}$

%We write $\wred$ for $\red^*$, and $P\red$ if $\exists Q $ such that $ P \red Q$.
We write $P\red$ if $\exists Q $ such that $ P \red Q$ and $P\not\red$, otherwise.

\section{Replication}

As mentioned before, it is known that replication (and hence
recursion) can be implemented in a higher-order process algebra
\cite{SangiorgiWalker}. As our first example of calculation with the
machinery thus far presented we give the construction explicitly in
the {\rhoc}.

\begin{eqnarray}
	D_{x} & := & \prefix{x}{y}{(\binpar{\outputp{x}{y}}{@{y}})} \nonumber\\
	\bangp_{x}{P} & := & \binpar{{x}!\langle{\binpar{D_{x}}{P}}\rangle}{D_{x}} \nonumber
\end{eqnarray}

\begin{eqnarray}
	\bangp_{x}{P} & & \nonumber\\
	=
	& {x}!\langle{(\prefix{x}{y}{(\outputp{x}{y} | @{y})) | P}}\rangle 
	      | \prefix{x}{y}{(\outputp{x}{y} | @{y})} & \nonumber\\
	\red
	& (\outputp{x}{y} | @{y})\substn{\quotep{(\prefix{x}{y}{(@{y} | \outputp{x}{y})) | P}}}{y} & \nonumber\\
	=
	& \outputp{x}{\quotep{(\prefix{x}{y}{(\outputp{x}{y} | @{y})) | P}}}
	  | {(\prefix{x}{y}{(\outputp{x}{y} | @{y})) | P}} & \nonumber\\
	\red
	& \ldots & \nonumber\\
	\red^*
	& P | P | \ldots & \nonumber
\end{eqnarray}

Of course, this encoding, as an implementation, runs away, unfolding
$\bangp{P}$ eagerly. A lazier and more implementable replication
operator, restricted to input-guarded processes, may be obtained as follows.

\begin{eqnarray}
\bangp{\prefix{u}{v}{P}} 
	:= 
	\binpar{\lift{x}{\prefix{u}{v}{(\binpar{D(x)}{P})}}}{D(x)} \nonumber
\end{eqnarray}

\begin{remark}
  Note that the lazier definition still does not deal with summation
  or mixed summation (i.e. sums over input and output). The reader is
  invited to construct definitions of replication that deal with these
  features. 

  Further, the definitions are parameterized in a name, $x$. Can you,
  gentle reader, make a definition that eliminates this parameter and
  guarantees no accidental interaction between the replication
  machinery and the process being replicated -- i.e. no accidental
  sharing of names used by the process to get its work done and the
  name(s) used by the replication to effect copying. This latter
  revision of the definition of replication is crucial to obtaining
  the expected identity $!!P \sim !P$.
\end{remark}

\begin{remark}\label{rem:paradoxical_combinator}
  The reader familiar with the lambda calculus will have noticed the
  similarity between $D$ and the paradoxical combinator.

  [Ed. note: the existence of this seems to suggest we have to be more
  restrictive on the set of processes and names we admit if we are to
  support no-cloning.]
\end{remark}

\subsubsection{Bisimulation}

The computational dynamics gives rise to another kind of equivalence,
the equivalence of computational behavior. As previously mentioned
this is typically captured \emph{via} some form of bisimulation.

% The notion we use in this paper is weak barbed bisimulation
% \cite{milner91polyadicpi}.

The notion we use in this paper is derived from weak barbed
bisimulation \cite{milner91polyadicpi}. 

\begin{definition}
An \emph{observation relation}, $\downarrow_{\mathcal N}$, over a set
of names, $\mathcal N$, is the smallest relation satisfying the rules
below.

\infrule[Out-barb]{y \in {\mathcal N}, \; x \nameeq y}
		  {\outputp{x}{v} \downarrow_{\mathcal N} x}
\infrule[Par-barb]{\mbox{$P\downarrow_{\mathcal N} x$ or $Q\downarrow_{\mathcal N} x$}}
		  {\binpar{P}{Q} \downarrow_{\mathcal N} x}

We write $P \Downarrow_{\mathcal N} x$ if there is $Q$ such that 
$P \wred Q$ and $Q \downarrow_{\mathcal N} x$.
\end{definition}

\begin{definition}
%\label{def.bbisim}
An  ${\mathcal N}$-\emph{barbed bisimulation} over a set of names, ${\mathcal N}$, is a symmetric binary relation 
${\mathcal S}_{\mathcal N}$ between agents such that $P\rel{S}_{\mathcal N}Q$ implies:
\begin{enumerate}
\item If $P \red P'$ then $Q \wred Q'$ and $P'\rel{S}_{\mathcal N} Q'$.
\item If $P\downarrow_{\mathcal N} x$, then $Q\Downarrow_{\mathcal N} x$.
\end{enumerate}
$P$ is ${\mathcal N}$-barbed bisimilar to $Q$, written
$P \wbbisim_{\mathcal N} Q$, if $P \rel{S}_{\mathcal N} Q$ for some ${\mathcal N}$-barbed bisimulation ${\mathcal S}_{\mathcal N}$.
\end{definition}

$\mathcal{R} \subseteq \pi \times \pi$

$P \mathcal{R} Q => \forall P'. P \red P' \Rightarrow \exists Q'. Q \red Q', P' \mathcal{R} Q'$

$P \vdash x \Rightarrow Q \vdash x$

\begin{mathpar}
  \inferrule*[lab=Out-barb]{x \nameeq y}{{y}!\langle{Q}\rangle \vdash x}
  \and
  \inferrule*[lab=Par-barb]{\mbox{$P\vdash x$ or $Q\vdash x$}}{\binpar{P}{Q} \vdash x}
\end{mathpar}

\subsubsection{Contexts}

One of the principle advantages of computational calculi like the
$\pi$-calculus is a well-defined notion of context,
contextual-equivalence and a correlation between
contextual-equivalence and notions of bisimulation. The notion of
context allows the decomposition of a process into (sub-)process and
its syntactic environment, its context. Thus, a context may be
thought of as a process with a ``hole'' (written $\Box$) in it. The
application of a context $M$ to a process $P$, written $M[P]$, is
tantamount to filling the hole in $M$ with $P$. In this paper we do
not need the full weight of this theory, but do make use of the notion
of context in the proof the main theorem. 

\begin{mathpar}
  \inferrule* [lab=summation] {} {{M_{M},M_{N}} \bc \Box \;|\; x.M_{A} \;|\; M_{M}+M_{N}}
  \and
  \inferrule* [lab=agent] {} {{M_{A}} \bc (\vec{x})M_{P} \;| \; \clift{P_0,\ldots,M_{P},\ldots,P_N}}
  \and \\
  \inferrule* [lab=process] {} {{M_{P}} \bc M_{N} \;| \;P|M_{P} }
\end{mathpar} 

\begin{mathpar}
  \inferrule* [lab=sychronization] {} {M_{N} \bc \Box \;|\; x?M_{F} \;|\; x!M_{C}}
  \and
  \inferrule* [lab=abstraction] {} {{M_{F}} \bc (x)M_{P} }
  \and
  \inferrule* [lab=concretion] {} {{M_{C}} \bc \langle M_{P} \rangle }
  \and \\
  \inferrule* [lab=process] {} {{M_{P}} \bc M_{N} \;| \;P|M_{P} }
\end{mathpar}

\begin{definition}[contextual application] Given a context $M$, and
  process $P$, we define the \emph{contextual application}, $M[P] :=
  M\{P/\Box\}$. That is, the contextual application of M to P is the
  substitution of $P$ for $\Box$ in $M$.
\end{definition}

$\meaningof{-} : L \to \mathcal{P}(\pi)$

\begin{mathpar}
  \inferrule* [lab=collection] {} {\meaningof{true} = \pi, \and \meaningof{~E} = \pi \setminus \meaningof{E}, \and \meaningof{E_{1} \& E_{2}} = \meaningof{E_{1}} \cap \meaningof{E_{2}}}
\end{mathpar}

\begin{mathpar}
  \inferrule* [lab=structure] {} {\meaningof{0} = \{ P \in \pi | P \equiv 0 \}, \and \\ \meaningof{E_1 | E_2} = \{ P \in \pi | P \equiv P_{1} | P_{2}, P_{1} \in \meaningof{E_{1}}, P_{2} \in \meaningof{E_2}\} }
\end{mathpar}

\begin{mathpar}
 \inferrule* [lab=behavior] {} {\meaningof{\langle a?b \rangle E} = \{ P \in \pi | P \equiv Q | u?(y)P', \\ \and \\\\ \and \\ \;\;\; u \in \meaningof{a}, \forall z.P'\{z/y\} \in \meaningof{E\{z/b\}}\}, \and \\ \meaningof{a!E} = \{ P \in \pi | P \equiv Q | x!\langle P' \rangle, x \in \meaningof{a} P' \in \meaningof{E}\} }
\end{mathpar}

\begin{mathpar}
 \inferrule* [lab=nominal] {} {\meaningof{\quotep{E}} = \{ \quotep{P} \in \quotep{\pi} | P \in \meaningof{E} \}, \and \meaningof{\quotep{P}} = \{ \quotep{Q} \in \quotep{\pi} | P \equiv Q \} \and \\ \meaningof{@\quotep{E}} = \{ P \in \pi | P \equiv @x, x \in \meaningof{E} \}}
\end{mathpar}

\begin{eqnarray*}
  \\
  \meaningof{-} : TS \to ST
\end{eqnarray*}

\begin{eqnarray*}
  \\
  L : TS \to ST
\end{eqnarray*}

\begin{eqnarray*}
  \\
  P \models E \iff P \in \meaningof{E}
\end{eqnarray*}

\begin{eqnarray*}
  P \approx_{L} Q \iff \forall E \in L. P \models E \iff Q \models E
\end{eqnarray*}

\begin{eqnarray*}
  P \approx_{K} Q
\end{eqnarray*}

\begin{eqnarray*}
  P \approx Q
\end{eqnarray*}

$\approx_{K} = \approx = \approx_{L}$

\subsubsection{Contextual duality}

Note that contexts extend the quotation operation to a family of
operations from processes to names. Given a context, $M$, we can
define a \emph{nominal context}, $\quotep{M}$ by $\quotep{M}[P] :=
\quotep{M[P]}$. To foreshadow what is to come we observe that these
operations enjoy a duality with processes very much like the duality
between vectors and maps from vectors to scalars.

Further, because the calculus is essentially higher-order, we have a
correspondence between contexts and processes. More specifically,
given a name $x$ and a context $M$ we can construct $M^{*}_{x}$ such
that 

\begin{mathpar}
  M^{*}_{x} | \lift{x}{P} \red M[P]
\end{mathpar}

namely,

\begin{mathpar}
  M^{*}_{x} := x?(u).M[\dropn{u}]
\end{mathpar}

The dependence of $M^{*}_{x}$ on a name makes it an abstraction, 

\begin{mathpar}
  M^{*} := (x)x?(u).M[\dropn{u}]
\end{mathpar}

\subsection{Additional notation}

It will sometimes be convenient to denote the process a name
quotes. We already have the notation $x = \quotep{P}$, but it will be
convenient to introduce an alternate notation, $\procn{x}$, when we
want to emphasize the connection to the use of the name. Note that, by
virtue of name equivalence, $\quotep{\procn{x}} \nameeq x$; so, the
notation is consistent with previous definitions.

Further, because names have structure it is possible to effect
substitutions on the basis of that structure. This means we need to
upgrade our notation for substitutions, which we accomplish by
adapting comprehension notation. Thus,

\begin{mathpar}
  P\{ y / x : x \in S \}
\end{mathpar}

is interpreted to mean the process derived from P by replacing (in a
capture-avoiding manner) each occurrence of $x$ in $S$ by $y$. For example,

\begin{mathpar}
  P\{ \quotep{\procn{x}|\procn{x}} / x : x \in \freenames{P} \}
\end{mathpar}

will replace each (occurrence) of a free name $x$ in $P$ by
$\quotep{\procn{x}|\procn{x}}$.

Also, we will avail ourselves of the notation $x^{L}$ and $x^{R}$ to
denote injections of a name into disjoint copies of the name
space. There are numerous ways to accomplish this. One example can be
found in \cite{MeredithR05}. This notation overloads to vectors of
names: $\vec{x}^{\pi} := (x_{i}^{\pi} \; : \; 0 \leq i < |\vec{x}| )$ where $\pi \in \{L,R\}$.

We also use $P^{\Box} := P|\Box$.

In \cite{MeredithR05} an interpretation of the new operator is
given. It turns out that there are several possible interpretations
all enjoying the requisite algebraic properties of the operator (see
\cite{milner91polyadicpi}). We will therefore make liberal use of
$(\nu\; \vec{x})P$.

% subsection the_syntax_and_semantics_of_the_notation_system (end)   

\section{Interpretation of QM}
\subsection{Supporting definitions}
\subsubsection{Multiplication}
\begin{mathpar}
  \quotep{Q} \cdot \quotep{R} := \quotep{Q|R}
  \and \\
  \quotep{Q} \cdot P := P\{ \quotep{Q|R} / \quotep{R} : \quotep{R} \in \freenames{P} \}
\end{mathpar}

\paragraph{Discussion}
The first line needs little explanation. The second line says that
each free name of the process is replaced with the multiplication of
that name by the scalar. Multiplication of a scalar (name) by a state
(process) results in a process all the names of which have been `moved
over' by parallel composition with the process the scalar
quotes. There is a subtlety that the bound names have to be
manipulated so that multiplied names aren't accidentally
captured. There are many ways to achieve this.

\begin{remark}\label{rem:multiplication_identities}
  The reader is invited to verify that for all $x,y,z \in \QProc$ and $P \in \Proc$
  \begin{mathpar}
    x \cdot \quotep{0} \equiv x 
    \and
    x \cdot y \equiv y \cdot x
    \and
    x \cdot (y \cdot z) \equiv (x \cdot y) \cdot z
    \and \\
    \quotep{0} \cdot P \equiv P
    \and \\
    x \cdot (y \cdot P) \equiv (x \cdot y) \cdot P
    \and \\
    x \cdot (P|Q) \equiv (x \cdot P) | (x \cdot Q)
    \and \\    
  \end{mathpar}
\end{remark}

\subsubsection{Tensor product}

We define a tensor product on processes by structural induction.

\paragraph{Tensor of sums} First note that all summations, including
$\pzero$ and sequence, can be written $\Sigma_{i} x_{i}.A_{i} +
\Sigma_{j} x_{j}.C_{j}$, where we have grouped input-guarded processes
together and output-guarded processes together.

Thus, we can define the tensor product of two summations, $N_{1}\otimes N_{2}$, where

\begin{mathpar}
  N_{1} := \Sigma_{i} x_{i}.A_{i} + \Sigma_{j} x_{j}.C_{j}
  \and
  N_{2} := \Sigma_{i'} y_{i'}.B_{i'} + \Sigma_{j'} y_{j'}.D_{j'} 
\end{mathpar}

as follows.

\begin{mathpar}
  \Sigma_{i} x_{i}.A_{i} + \Sigma_{j} x_{j}.C_{j} \otimes \Sigma_{i'}
  y_{i'}.B_{i'} + \Sigma_{j'} y_{j'}.D_{j'} 
  \and \\
  := \; \Sigma_{i} \Sigma_{i'} \quotep{\stackrel{\vee}{x_{i}}| \stackrel{\vee}{y_{i'}}}.(A_{i}\otimes B_{i'}) \; | \; \Sigma_{i'} \Sigma_{i} \quotep{\stackrel{\vee}{y_{i'}}|\stackrel{\vee}{x_{i}}}.(B_{i'}\otimes A_{i})
  \and
  \;\; | \;\; \Sigma_{j} \Sigma_{j'} \quotep{\stackrel{\vee}{x_{j}}|\stackrel{\vee}{y_{j'}}}.(A_{j}\otimes B_{j'}) \; | \; \Sigma_{j'} \Sigma_{j} \quotep{\stackrel{\vee}{y_{j'}}|\stackrel{\vee}{x_{j}}}.(B_{j'}\otimes A_{j})
\end{mathpar}

\begin{remark}
  Do we need to $x^{L}$ and $y^{R}$ for this construction as well?
\end{remark}

\paragraph{Tensor of parallel compositions} Next, we distribute tensor
over par.

\begin{mathpar}
  P_{1}|P_{2} \otimes Q_{1}|Q_{2} := (P_{1} \otimes Q_{1}) | (P_{1}
  \otimes Q_{2}) | (P_{2} \otimes Q_{1}) | (P_{2} \otimes Q_{2})
\end{mathpar}

\paragraph{Tensor with dropped names} We treat tensor of a
process with a dropped name as parallel composition.

\begin{mathpar}
  P \otimes \dropn{x} := P | \dropn{x}
\end{mathpar}

\paragraph{Tensor of agents}

Finally, we need to define tensor on agents. Note that the definition
of tensor on normal products only tensors inputs with inputs and
outputs with outputs. Thus, we only have to define the operation on
``homogeneous'' pairings.

\begin{mathpar}
  (\vec{x})P \otimes (\vec{y})Q
  \and \\
  := (x_{0}^{L}|y_{0}^{R},\ldots,x_{0}^{L}|y_{n}^{R},\ldots,x_{m}^{L}|y_{0}^{R},\ldots,x_{m}^{L}|y_{n}^R)(P\{ \vec{x}^{L}/\vec{x}\} \otimes Q \{ \vec{y}^{R}/\vec{y}\})
  \and \\
  \clift{\vec{P}} \otimes \clift{\vec{Q}}
  \and \\
  := \clift{P_{0}\otimes Q_{0},\ldots,P_{0}\otimes Q_{n},\ldots,P_{m}\otimes Q_{0},\ldots,P_{m}\otimes Q_{n}}
\end{mathpar}

\begin{remark}
  Observe that arities of tensored abstractions matches arities of
  tensored concretions if the original arities matched. Note also that
  the length of the arities corresponds to the increase in dimension
  we see in ordinary vector space tensor product.
\end{remark}

\begin{remark}
  Operationally, this definition distributes the tensor down to
  components ``linked'' by summation. Tensor over summation is
  intriguing in that it mixes names. Moreover, as a consequence of the
  way it mixes names we have the identities for all $x \in \QProc$ and
  $P,Q \in \Proc$

  \begin{mathpar}
    (x \cdot P) \otimes Q \equiv x \cdot (P \otimes Q) \equiv P \otimes (x \cdot Q)
    \and
    P \otimes \pzero \equiv P
  \end{mathpar}

  that the reader is invited to verify.
\end{remark}

\subsubsection{Annihilation}
\begin{mathpar}
  P^{\perp} := \{ Q | \forall R. P|Q \red^{*} R \Rightarrow R \red^{*} \pzero \}
  \and \\
  P^{\underline{\perp}} := \Sigma_{Q \in P^{\perp}} \quotep{Q}?(y).(\dropn{y}|Q) | \Sigma_{Q \in P^{\perp}} \quotep{Q}\clift{\Box}
\end{mathpar}

\paragraph{Discussion} The reader will note that $P^{\perp}$ is a
\emph{set} of processes, while $P^{\underline{\perp}}$ is a
\emph{context}. We call the set $P^{\perp}$ the \emph{annihilators} of
$P$. The parallel composition of a process in the annihilators of $P$
with $P$ will result in a process, the state space of which has all
paths eventually leading to $\pzero$. Execution may endure loops; but
under reasonable conditions of fairness (naturally guaranteed under
most notions of bisimulation) such a composite process cannot get
stuck in such a loop and will, eventually pop out and terminate.

The context $P^{\underline{\perp}}$ is ready and willing to ``take the
$P$ out of'' the process to which it is applied. It will effectively
transmit the code of the process to which it is applied to one of the
annihilators and run the process against it.

\subsubsection{Evaluation}
We fix $M$ a domain of fully abstract interpretation with an equality
coincident with bisimulation. We take $\meaningof{\cdot} : \Proc \to
M$ to be the map interpreting processes and $\nmeaningof{\cdot} : \M
\to Proc$ to be the map running the other way. Then we define

\begin{mathpar}
  \int P := \nmeaningof{\meaningof{P}}
\end{mathpar}

\paragraph{Discussion}
There are many fully abstract interpretations of Milner's
$\pi$-calculus. Any of them can be used as a basis for interpreting
the reflective calculus here. Equipped with such a domain it is
largely a matter of grinding through to check that the Yoneda
construction for the normalization-by-evaluation program can be
extended to this setting.

\begin{remark}
  The reader is invited to verify that $\int (P^{\underline{\perp}}[P]) = 0$.
\end{remark}

\subsection{Quantum mechanics}

Table \ref{tbl:core_qm_op_defns} gives the core operational definitions

\begin{table}[htp]\label{tbl:core_qm_op_defns}
  \center{
    \fbox{
      \begin{tabular}{c|c}
        quantum mechanics & process calculus \\
        \hline
        scalar & $x := \quotep{P}$ \\
        state vector & $\state{P} := P$ \\
        dual & $\state{P}^{*} := \event{P^{\underline{\perp}}} := \quotep{P^{\underline{\perp}}}[-]$ \\
        matrix & $ \Sigma_{\alpha} \state{P_{\alpha}}x_{\alpha}\event{Q_{\alpha}}$ \\
        vector addition & $\state{P} + \state{Q} := \state{P | Q}$ \\
        tensor product & $\state{P} \otimes \state{Q} := \state{P \otimes Q}$ \\
        inner product & $\innerprod{P}{Q} := \quotep{\int P^{\underline{\perp}}[Q]}$ \\
      \end{tabular}
    }
  }
  \caption{QM - operational definitions}
\end{table}

where

\begin{mathpar}
  \prmatrix{P}{Q} := \fprmatrix{P}{\quotep{\pzero}}{Q}
  \and
  \fprmatrix{P}{x}{Q} := (\state{P},x,\event{Q})
  \and
  (\fprmatrix{P}{x}{Q})(\state{R}) := x \cdot \innerprod{Q}{R} \cdot \state{P}
  \and
  (\fprmatrix{P}{x}{Q})(\event{R}) := x \cdot \innerprod{R}{P} \cdot \event{Q}
\end{mathpar}

\paragraph{Discussion}
As promised: vectors (aka states) are represented as processes; duals
as contextual duals; inner product definition should be compared with
standard inner product definition for ....

\begin{remark}
  Assuming $\int (P^{\underline{\perp}}[P]) = 0$, the reader is
  invited to verify that $(\fprmatrix{P}{x}{P})(\state{P}) = x \cdot \state{P}$.
\end{remark}

\begin{remark}
  The reader is invited to verify that $\innerprod{P}{Q}$ could
  equally well have been written $\quotep{\int \stackrel{\vee}{x}}$
  where $x = \event{P^{\underline{\perp}}}(Q)$.

  One of the motivations for this remark is that there is another way
  to factor these operations. We could package up evaluation in the dual:

  \begin{mathpar}
    \state{P}^{*} := \event{\int P^{\underline{\perp}}} := \quotep{\int P^{\underline{\perp}}}[-]
  \end{mathpar}

  and then have inner product defined by
  
  \begin{mathpar}
    \innerprod{P}{Q} := \event{P}(Q)
  \end{mathpar}

  Hopefully, experience with the calculations will provide guidance on
  the best factoring.
\end{remark}

\begin{remark}
  Assuming $\int (P^{\underline{\perp}}[P]) = 0$, the reader is
  invited to verify that $\forall P,Q. (\prmatrix{0}{Q})(\state{0}) =
  \state{0}$ and dually $(\prmatrix{P}{0})(\event{0}) = \event{0}$.
\end{remark}

\begin{remark}
  i'm a little worried that i don't (yet) have proper support for
  complex conjugacy. But, the observation above may give us a
  clue. According to Abramsky, it must be the case that the scalars
  are iso to the homset of the identity for the tensor -- which the
  observation above characterizes. 

  For now, we will simply bookmark the notion with $\overline{x}$.
\end{remark}

\subsubsection{Adjointness}

We need to give a definition of $(\cdot)^{\dagger}$ for matrices. The
obvious candidate definition is
\begin{mathpar}
(\Sigma_{\alpha}\fprmatrix{P_{\alpha}}{x_{\alpha}}{Q_{\alpha}})^{\dagger}
= \Sigma_{\alpha}\fprmatrix{(Q_{\alpha}^{\underline{\perp}})^{*}}{\overline{x}_{\alpha}}{P_{\alpha}^{\underline{\perp}}} 
\end{mathpar}

But, $(Q_{\alpha}^{\underline{\perp}})^{*}$ requires a name along
which to communicate the process to achieve the context application.

\subsubsection{Basis for a basis}
If processes label states and ``addition'' of states (a.k.a. vector
addition) is interpreted as parallel composition, what corresponds to
notions of linear independence and basis? Here, we recall that Yoshida
has developed a set of \emph{combinators} for an asynchronous verison
of Milner's $\pi$-calculus. These are a finite set of processes such
any process can be expressed as parallel composition of these
combinators together with liberal uses of the new operator and
replication. We can simply give a translation of these into the
present calculus and have reasonable expectation that the property
carries over. That is, that the resultant set allows to express all
processes via parallel composition. Note, however, that there is no
new operator or replication in this calculus. As a result, we expect
that the corresponding set is actually infinite. That is, we expect
that the space is actually infinite dimensional.

\begin{remark}
  The attentive reader may be a bit concerned. Certainly, the
  collection $S$, $K$ and $I$ is a finite set of
  combinators. Shouldn't we expect to see a finite set of combinators
  for an effectively equivalent system? i am very sympathetic to this
  critique and feel it warrants full attention. On the other hand, i
  also have in mind the following analogy. The natural numbers, as a
  monoid under addition, has exactly $1$ generator, while the natural
  numbers, as a monoid under multiplication, has countably many
  generators (the primes). We observe that the application of the
  lambda calculus is much less resource sensitive than the parallel
  composition of the $\pi$-calculus. Could it be the case that we have
  an analogy of the form
  
  \begin{mathpar}
    m + n : MN :: m*n : M|N
  \end{mathpar}

  giving a similar blow up in the set of ``primes''?  This is such a
  wonderful thought that, even if it's not true, i think it's worth
  writing down.
\end{remark}
 

\documentclass[12pt]{llncs}
%\documentclass{jktr}

\usepackage[pdftex]{hyperref}                   
\usepackage {listings}
\usepackage {mathpartir}
\usepackage{bcprules}
%\usepackage{listings}
                       
\usepackage{graphicx} 
%\usepackage[margins=2.5cm,nohead,nofoot]{geometry}
%\usepackage{geometry}
\usepackage{amsfonts}
\usepackage{amstext}
\usepackage{latexsym}
\usepackage{amssymb}
\usepackage{color}


%\include{myPreamble}
\documentclass[12pt]{llncs}
%\documentclass{jktr}

\usepackage[pdftex]{hyperref}                   
\usepackage {listings}
\usepackage {mathpartir}
\usepackage{bcprules}
%\usepackage{listings}
                       
\usepackage{graphicx} 
%\usepackage[margins=2.5cm,nohead,nofoot]{geometry}
%\usepackage{geometry}
\usepackage{amsfonts}
\usepackage{amstext}
\usepackage{latexsym}
\usepackage{amssymb}
\usepackage{color}


%\include{myPreamble}
\include{qm2pi.local} 

%\ifpdf
%\usepackage[pdftex]{graphicx}
%\else
%\usepackage{graphicx}
%\fi

 % \ifpdf
%  \usepackage{pdfsync}
%  \if


%\title{Brief Article}
%\author{David F. Snyder}
%\author{L.G. Meredith}

%\address{Dept. of Math., Texas State University--San Marcos, San Marcos, TX 78666}
       
\pagestyle{empty}


\begin{document}

\lstset{language=[Objective]Caml,frame=shadowbox}

\input{qm2pi.front}

% section front matter (end)

\input{qm2pi.intro} 
 
% section introduction (end)

% \input{qm2pi.knotations} 

% section notation (end)

\input{qm2pi.process.calculi} 

% section concurrent_process_calculi_and_spatial_logics_ (end)
    
%\input{qm2pi.knots2pi} 

%\input{qm2pi.trefoil} 

%\input{qm2pi.mainthm} 

% subsection basic_interpretation (end)

%\input{qm2pi.rho.presentation} 
\subsection{The syntax and semantics of the notation system}\label{sub:the_syntax_and_semantics_of_the_notation_system} % (fold)

We now summarize a technical presentation of the calculus that
embodies our theory of dynamics. The typical presentation of such a
calculus follows the style of giving generators and relations on
them. The grammar, below, describing term constructors, freely
generates the set of processes, $\Proc$. This set is then quotiented
by a relation known as structural congruence and it is over this set
that the notion of dynamics is expressed. This presentation is
essentially that of \cite{MeredithR05} with the addition of
polyadicity and summation. For readability we have relegated some of
the technical subtleties to an appendix.

\subsubsection{Process grammar}\label{subsub:process_grammar}

\begin{mathpar}
  \inferrule* [lab=synchronization] {} {{M} \bc \pzero \;|\; x?F \;|\; x!C }
  \and
  \inferrule* [lab=abstraction] {} {{F} \bc (x)P}
  \and
  \inferrule* [lab=concretion] {} {{C} \bc \langle Q \rangle}
  \and
  \inferrule* [lab=process] {} {{P,Q} \bc M \;| \;P|Q \;|\; @{x}}
  \and
  \inferrule* [lab=name] {} {{x} \bc \quotep{P}}
\end{mathpar} 

Note that $\vec{x}$ (resp. $\vec{P}$) denotes a vector of names
(resp. processes) of length $|\vec{x}|$ (resp. $|\vec{P}|$). We adopt
the following useful abbreviations.

\begin{mathpar}
   x?(\vec{y}).P := x.(\vec{y})P \and  x\clift{\vec{P}} := x.\clift{\vec{P}}
   \and x!(y) := \lift{x}{\dropn{y}}
   \and \Pi_{i=0}^{n-1}P_i := P_0 | \ldots | P_{n-1}
\end{mathpar}

\subsubsection{Structural congruence}

\paragraph{Free and bound names and alpha-equivalence.} At the
core of structural equivalence is alpha-equivalence which identifies
process that are the same up to a change of variable. Formally, we
recognize the distinction between free and bound names. The free names
of a process, $\freenames{P}$, may be calculated recursively as
follows:

\begin{mathpar}
\freenames{\pzero} := \emptyset
  \and \\
  \freenames{x?(y).P} := \{ x \} \cup (\freenames{P} \setminus \{ y \})
  \and 
  \freenames{x!\langle P \rangle} := \{ x \} \cup \{ P \} 
  \and \\
  \freenames{P|Q} := \freenames{P} \cup \freenames{Q}
  \and \\
  \freenames{@{x}} := \{ x \}
\end{mathpar}

$\pi$
$\quotep{\pi}$

$\freenames{-} : \pi \to \mathcal{P}(\quotep{\pi})$

\begin{eqnarray*}
  \freenames{\pzero} & := & \emptyset \\
  \freenames{x?(y).P} & := & \{ x \} \cup (\freenames{P} \setminus \{ y \}) \\
  \freenames{x!\langle P \rangle} & := & \{ x \} \cup \{ P \} \\
  \freenames{P|Q} & := & \freenames{P} \cup \freenames{Q} \\
  \freenames{\dropn{x}} & := & \{ x \}
\end{eqnarray*}

The bound names of a process, $\boundnames{P}$, are those names occurring in $P$
that are not free. For example, in $x?(y).0$, the name $x$ is free, while $y$ is bound.

\begin{mathpar}
  \inferrule* [lab=monoidal-laws] {} { P|Q \equiv Q|P \and P|0 \equiv P \and P|(Q|R) \equiv (P|Q)|R }
\end{mathpar}

\begin{mathpar}
  \inferrule* [lab=alpha-equivalence] {} { (x)P \equiv (y)P\{y/x\} \and y \not\in \freenames{P} }
\end{mathpar}

\begin{definition}
Then two processes, $P,Q$, are alpha-equivalent if $P = Q\{\vec{y}/\vec{x}\}$ for
some $\vec{x} \in \boundnames{Q},\vec{y} \in \boundnames{P}$, where $Q\{\vec{y}/\vec{x}\}$
denotes the capture-avoiding substitution of $\vec{y}$ for $\vec{x}$ in $Q$.
\end{definition}

\begin{definition}
  The {\em structural congruence} \cite{SangiorgiWalker} , $\equiv$,
  between processes is the least congruence containing
  alpha-equivalence, satisfying the abelian monoid laws
  (associativity, commutativity and $\pzero$ as identity) for parallel
  composition $|$ and for summation $+$.
\end{definition}

\subsection{Name equivalence}

We take name equivalence, written $\nameeq$, to be the smallest
equivalence relation generated by the following rules.

\begin{mathpar}
\inferrule*[lab=Quote-drop]
{ }
{ \quotep{@{x}} \nameeq x }

\inferrule*[lab=Struct-equiv]
{ P \scong Q }
{ \quotep{P} \nameeq \quotep{Q} }
\end{mathpar}

The astute reader will have noticed that the mutual recursion of names
and processes imposes a mutual recursion on alpha-equivalence and
structural equivalence via name-equivalence. Fortunately, all of this
works out pleasantly and we may calculate in the natural way, free of
concern. The reader interested in the details is referred to the
appendix \ref{appendix:rho_details}.

\subsection{Substitution}

We use $\Proc$ for the set of processes, $\QProc$ for the set of
names, and $\id{\{}\vec{y} / \vec{x} \id{\}}$ to denote partial maps,
$s : \QProc \rightarrow \QProc$. A map, $s$ lifts, uniquely, to a map
on process terms, $\widehat{s} : \Proc \rightarrow \Proc$ by the
following equations.

\begin{mathpar}
  (0) \psubstp{Q}{P} := 0 \\
  (R \juxtap S) \psubstp{Q}{P}
  :=    
  (R)\psubstp{Q}{P} \juxtap (S) \psubstp{Q}{P} \\
  (x?(y).R) \psubstp{Q}{P}    
  :=    
  (x)\substp{Q}{P} (z)\concat( (R \psubstn{z}{y}) \psubstp{Q}{P} ) \\
  (\lift{x}{R}) \psubstp{Q}{P}  
  :=
  \lift{(x)\substp{Q}{P}}{ R \psubstp{Q}{P} } \\
%   (\dropn{x})  \psubstp{Q}{P}       
%   := 
%   \left\{ 
%     \begin{array}{ccc} 
%       \dropn{\quotep{Q}} & & x \nameeq \quotep{P} \\
%       \dropn{x} & & otherwise \\
%     \end{array}
%   \right. 
  (\dropn{x})  \psubstp{Q}{P}       
  := 
  \left\{ 
    \begin{array}{ccc} 
      Q & & x \nameeq \quotep{P} \\
      \dropn{x} & & otherwise \\
    \end{array}
  \right.
\end{mathpar}
 

where

\begin{eqnarray}
  (x)\id{\{} \lpquote Q \rpquote / \lpquote P \rpquote \id{\}}            = 
  \left\{ 
    \begin{array}{ccc}
      \lpquote Q \rpquote & & x \nameeq \lpquote P \rpquote \\
      x & & otherwise \\
    \end{array}
  \right. \nonumber
\end{eqnarray}

and $z$ is chosen distinct from $\quotep{P}$, $\quotep{Q}$, the free
names in $Q$, and all the names in $R$. Our $\alpha$-equivalence will
be built in the standard way from this substitution.

\begin{remark}\label{rem:no_self_referential_names}
  One consequence of these definitions is that $\forall P. \quotep{P}
  \not\in \freenames{P}$.
\end{remark}

\subsection{ Dynamic quote: an example }

Anticipating something of what's to come, consider applying the
substitution, $\widehat{\id{\{}u / z \id{\}}}$, to the following pair
of processes, $\lift{w}{y!(z)}$ and $w[ \lpquote y!(z) \rpquote ]$.

\begin{eqnarray}
	\lift{w}{y!(z)}\widehat{\id{\{}u / z \id{\}}}
		& = &
		\lift{w}{y!(u)} \nonumber\\
	w[ \lpquote y!(z) \rpquote ] \widehat{ \id{\{}u / z \id{\}} }
		& = &
		w[ \lpquote y!(z) \rpquote ] \nonumber
\end{eqnarray}

Because the body of the process between quotes is impervious to
substitution, we get radically different answers. In fact, by
examining the first process in an input context,
e.g. $x?(z).\lift{w}{y!(z)}$, we see that the process under the lift
operator may be shaped by prefixed inputs binding a name inside it. In
this sense, the lift operator will be seen as a way to dynamically
construct processes before reifying them as names.

Finally equipped with these standard features we can present the
dynamics of the calculus.

\subsubsection{Operational semantics} 

Finally, we introduce the computational dynamics. What marks these
algebras as distinct from other more traditionally studied algebraic
structures, e.g. vector spaces or polynomial rings, is the manner in
which dynamics is captured. In traditional structures, dynamics is typically
expressed through morphisms between such structures, as in linear maps
between vector spaces or morphisms between rings. In algebras
associated with the semantics of computation, the dynamics is
expressed as part of the algebraic structure itself, through a
reduction reduction relation typically denoted by $\red$. Below, we
give a recursive presentation of this relation for the calculus used
in the encoding.

$\red \subseteq \pi \times \pi$
$\red : \pi \to \mathcal{P}(\pi)$

\begin{mathpar}
  \inferrule* [lab=Comm] { \textsf{match}( x_{src}, x_{trgt} ) } { x_{trgt}?(y)P \; | \; x_{src}!\langle {Q} \rangle \red P\{\quotep{Q}/y}\} }
  \and \\
  \inferrule* [lab=Par] {{P} \red {P}'} {{{P} | {Q}} \red {{P}' | {Q}}}
  \and
  \inferrule* [lab=Equiv]{{{P} \scong {P}'} \andalso {{P}' \red {Q}'} \andalso {{Q}' \scong {Q}}}{{P} \red {Q}}
\end{mathpar}

\begin{eqnarray*}
  match_{\equiv} (\quotep{P},\quotep{Q}) & := & P \equiv Q \\
  match_{\dagger}(\quotep{P},\quotep{Q}) & := & \forall R. P|Q \red^{*} R => R \red^{*} 0 \\
  match_{K}(\quotep{P},\quotep{Q}) & := & K \mbox{ for some context } K
\end{eqnarray*}

$u?(x)P | u!\langle Q \rangle \red P\{\quotep{Q}/x\}$

%We write $\wred$ for $\red^*$, and $P\red$ if $\exists Q $ such that $ P \red Q$.
We write $P\red$ if $\exists Q $ such that $ P \red Q$ and $P\not\red$, otherwise.

\section{Replication}

As mentioned before, it is known that replication (and hence
recursion) can be implemented in a higher-order process algebra
\cite{SangiorgiWalker}. As our first example of calculation with the
machinery thus far presented we give the construction explicitly in
the {\rhoc}.

\begin{eqnarray}
	D_{x} & := & \prefix{x}{y}{(\binpar{\outputp{x}{y}}{@{y}})} \nonumber\\
	\bangp_{x}{P} & := & \binpar{{x}!\langle{\binpar{D_{x}}{P}}\rangle}{D_{x}} \nonumber
\end{eqnarray}

\begin{eqnarray}
	\bangp_{x}{P} & & \nonumber\\
	=
	& {x}!\langle{(\prefix{x}{y}{(\outputp{x}{y} | @{y})) | P}}\rangle 
	      | \prefix{x}{y}{(\outputp{x}{y} | @{y})} & \nonumber\\
	\red
	& (\outputp{x}{y} | @{y})\substn{\quotep{(\prefix{x}{y}{(@{y} | \outputp{x}{y})) | P}}}{y} & \nonumber\\
	=
	& \outputp{x}{\quotep{(\prefix{x}{y}{(\outputp{x}{y} | @{y})) | P}}}
	  | {(\prefix{x}{y}{(\outputp{x}{y} | @{y})) | P}} & \nonumber\\
	\red
	& \ldots & \nonumber\\
	\red^*
	& P | P | \ldots & \nonumber
\end{eqnarray}

Of course, this encoding, as an implementation, runs away, unfolding
$\bangp{P}$ eagerly. A lazier and more implementable replication
operator, restricted to input-guarded processes, may be obtained as follows.

\begin{eqnarray}
\bangp{\prefix{u}{v}{P}} 
	:= 
	\binpar{\lift{x}{\prefix{u}{v}{(\binpar{D(x)}{P})}}}{D(x)} \nonumber
\end{eqnarray}

\begin{remark}
  Note that the lazier definition still does not deal with summation
  or mixed summation (i.e. sums over input and output). The reader is
  invited to construct definitions of replication that deal with these
  features. 

  Further, the definitions are parameterized in a name, $x$. Can you,
  gentle reader, make a definition that eliminates this parameter and
  guarantees no accidental interaction between the replication
  machinery and the process being replicated -- i.e. no accidental
  sharing of names used by the process to get its work done and the
  name(s) used by the replication to effect copying. This latter
  revision of the definition of replication is crucial to obtaining
  the expected identity $!!P \sim !P$.
\end{remark}

\begin{remark}\label{rem:paradoxical_combinator}
  The reader familiar with the lambda calculus will have noticed the
  similarity between $D$ and the paradoxical combinator.

  [Ed. note: the existence of this seems to suggest we have to be more
  restrictive on the set of processes and names we admit if we are to
  support no-cloning.]
\end{remark}

\subsubsection{Bisimulation}

The computational dynamics gives rise to another kind of equivalence,
the equivalence of computational behavior. As previously mentioned
this is typically captured \emph{via} some form of bisimulation.

% The notion we use in this paper is weak barbed bisimulation
% \cite{milner91polyadicpi}.

The notion we use in this paper is derived from weak barbed
bisimulation \cite{milner91polyadicpi}. 

\begin{definition}
An \emph{observation relation}, $\downarrow_{\mathcal N}$, over a set
of names, $\mathcal N$, is the smallest relation satisfying the rules
below.

\infrule[Out-barb]{y \in {\mathcal N}, \; x \nameeq y}
		  {\outputp{x}{v} \downarrow_{\mathcal N} x}
\infrule[Par-barb]{\mbox{$P\downarrow_{\mathcal N} x$ or $Q\downarrow_{\mathcal N} x$}}
		  {\binpar{P}{Q} \downarrow_{\mathcal N} x}

We write $P \Downarrow_{\mathcal N} x$ if there is $Q$ such that 
$P \wred Q$ and $Q \downarrow_{\mathcal N} x$.
\end{definition}

\begin{definition}
%\label{def.bbisim}
An  ${\mathcal N}$-\emph{barbed bisimulation} over a set of names, ${\mathcal N}$, is a symmetric binary relation 
${\mathcal S}_{\mathcal N}$ between agents such that $P\rel{S}_{\mathcal N}Q$ implies:
\begin{enumerate}
\item If $P \red P'$ then $Q \wred Q'$ and $P'\rel{S}_{\mathcal N} Q'$.
\item If $P\downarrow_{\mathcal N} x$, then $Q\Downarrow_{\mathcal N} x$.
\end{enumerate}
$P$ is ${\mathcal N}$-barbed bisimilar to $Q$, written
$P \wbbisim_{\mathcal N} Q$, if $P \rel{S}_{\mathcal N} Q$ for some ${\mathcal N}$-barbed bisimulation ${\mathcal S}_{\mathcal N}$.
\end{definition}

$\mathcal{R} \subseteq \pi \times \pi$

$P \mathcal{R} Q => \forall P'. P \red P' \Rightarrow \exists Q'. Q \red Q', P' \mathcal{R} Q'$

$P \vdash x \Rightarrow Q \vdash x$

\begin{mathpar}
  \inferrule*[lab=Out-barb]{x \nameeq y}{{y}!\langle{Q}\rangle \vdash x}
  \and
  \inferrule*[lab=Par-barb]{\mbox{$P\vdash x$ or $Q\vdash x$}}{\binpar{P}{Q} \vdash x}
\end{mathpar}

\subsubsection{Contexts}

One of the principle advantages of computational calculi like the
$\pi$-calculus is a well-defined notion of context,
contextual-equivalence and a correlation between
contextual-equivalence and notions of bisimulation. The notion of
context allows the decomposition of a process into (sub-)process and
its syntactic environment, its context. Thus, a context may be
thought of as a process with a ``hole'' (written $\Box$) in it. The
application of a context $M$ to a process $P$, written $M[P]$, is
tantamount to filling the hole in $M$ with $P$. In this paper we do
not need the full weight of this theory, but do make use of the notion
of context in the proof the main theorem. 

\begin{mathpar}
  \inferrule* [lab=summation] {} {{M_{M},M_{N}} \bc \Box \;|\; x.M_{A} \;|\; M_{M}+M_{N}}
  \and
  \inferrule* [lab=agent] {} {{M_{A}} \bc (\vec{x})M_{P} \;| \; \clift{P_0,\ldots,M_{P},\ldots,P_N}}
  \and \\
  \inferrule* [lab=process] {} {{M_{P}} \bc M_{N} \;| \;P|M_{P} }
\end{mathpar} 

\begin{mathpar}
  \inferrule* [lab=sychronization] {} {M_{N} \bc \Box \;|\; x?M_{F} \;|\; x!M_{C}}
  \and
  \inferrule* [lab=abstraction] {} {{M_{F}} \bc (x)M_{P} }
  \and
  \inferrule* [lab=concretion] {} {{M_{C}} \bc \langle M_{P} \rangle }
  \and \\
  \inferrule* [lab=process] {} {{M_{P}} \bc M_{N} \;| \;P|M_{P} }
\end{mathpar}

\begin{definition}[contextual application] Given a context $M$, and
  process $P$, we define the \emph{contextual application}, $M[P] :=
  M\{P/\Box\}$. That is, the contextual application of M to P is the
  substitution of $P$ for $\Box$ in $M$.
\end{definition}

$\meaningof{-} : L \to \mathcal{P}(\pi)$

\begin{mathpar}
  \inferrule* [lab=collection] {} {\meaningof{true} = \pi, \and \meaningof{~E} = \pi \setminus \meaningof{E}, \and \meaningof{E_{1} \& E_{2}} = \meaningof{E_{1}} \cap \meaningof{E_{2}}}
\end{mathpar}

\begin{mathpar}
  \inferrule* [lab=structure] {} {\meaningof{0} = \{ P \in \pi | P \equiv 0 \}, \and \\ \meaningof{E_1 | E_2} = \{ P \in \pi | P \equiv P_{1} | P_{2}, P_{1} \in \meaningof{E_{1}}, P_{2} \in \meaningof{E_2}\} }
\end{mathpar}

\begin{mathpar}
 \inferrule* [lab=behavior] {} {\meaningof{\langle a?b \rangle E} = \{ P \in \pi | P \equiv Q | u?(y)P', \\ \and \\\\ \and \\ \;\;\; u \in \meaningof{a}, \forall z.P'\{z/y\} \in \meaningof{E\{z/b\}}\}, \and \\ \meaningof{a!E} = \{ P \in \pi | P \equiv Q | x!\langle P' \rangle, x \in \meaningof{a} P' \in \meaningof{E}\} }
\end{mathpar}

\begin{mathpar}
 \inferrule* [lab=nominal] {} {\meaningof{\quotep{E}} = \{ \quotep{P} \in \quotep{\pi} | P \in \meaningof{E} \}, \and \meaningof{\quotep{P}} = \{ \quotep{Q} \in \quotep{\pi} | P \equiv Q \} \and \\ \meaningof{@\quotep{E}} = \{ P \in \pi | P \equiv @x, x \in \meaningof{E} \}}
\end{mathpar}

\begin{eqnarray*}
  \\
  \meaningof{-} : TS \to ST
\end{eqnarray*}

\begin{eqnarray*}
  \\
  L : TS \to ST
\end{eqnarray*}

\begin{eqnarray*}
  \\
  P \models E \iff P \in \meaningof{E}
\end{eqnarray*}

\begin{eqnarray*}
  P \approx_{L} Q \iff \forall E \in L. P \models E \iff Q \models E
\end{eqnarray*}

\begin{eqnarray*}
  P \approx_{K} Q
\end{eqnarray*}

\begin{eqnarray*}
  P \approx Q
\end{eqnarray*}

$\approx_{K} = \approx = \approx_{L}$

\subsubsection{Contextual duality}

Note that contexts extend the quotation operation to a family of
operations from processes to names. Given a context, $M$, we can
define a \emph{nominal context}, $\quotep{M}$ by $\quotep{M}[P] :=
\quotep{M[P]}$. To foreshadow what is to come we observe that these
operations enjoy a duality with processes very much like the duality
between vectors and maps from vectors to scalars.

Further, because the calculus is essentially higher-order, we have a
correspondence between contexts and processes. More specifically,
given a name $x$ and a context $M$ we can construct $M^{*}_{x}$ such
that 

\begin{mathpar}
  M^{*}_{x} | \lift{x}{P} \red M[P]
\end{mathpar}

namely,

\begin{mathpar}
  M^{*}_{x} := x?(u).M[\dropn{u}]
\end{mathpar}

The dependence of $M^{*}_{x}$ on a name makes it an abstraction, 

\begin{mathpar}
  M^{*} := (x)x?(u).M[\dropn{u}]
\end{mathpar}

\subsection{Additional notation}

It will sometimes be convenient to denote the process a name
quotes. We already have the notation $x = \quotep{P}$, but it will be
convenient to introduce an alternate notation, $\procn{x}$, when we
want to emphasize the connection to the use of the name. Note that, by
virtue of name equivalence, $\quotep{\procn{x}} \nameeq x$; so, the
notation is consistent with previous definitions.

Further, because names have structure it is possible to effect
substitutions on the basis of that structure. This means we need to
upgrade our notation for substitutions, which we accomplish by
adapting comprehension notation. Thus,

\begin{mathpar}
  P\{ y / x : x \in S \}
\end{mathpar}

is interpreted to mean the process derived from P by replacing (in a
capture-avoiding manner) each occurrence of $x$ in $S$ by $y$. For example,

\begin{mathpar}
  P\{ \quotep{\procn{x}|\procn{x}} / x : x \in \freenames{P} \}
\end{mathpar}

will replace each (occurrence) of a free name $x$ in $P$ by
$\quotep{\procn{x}|\procn{x}}$.

Also, we will avail ourselves of the notation $x^{L}$ and $x^{R}$ to
denote injections of a name into disjoint copies of the name
space. There are numerous ways to accomplish this. One example can be
found in \cite{MeredithR05}. This notation overloads to vectors of
names: $\vec{x}^{\pi} := (x_{i}^{\pi} \; : \; 0 \leq i < |\vec{x}| )$ where $\pi \in \{L,R\}$.

We also use $P^{\Box} := P|\Box$.

In \cite{MeredithR05} an interpretation of the new operator is
given. It turns out that there are several possible interpretations
all enjoying the requisite algebraic properties of the operator (see
\cite{milner91polyadicpi}). We will therefore make liberal use of
$(\nu\; \vec{x})P$.

% subsection the_syntax_and_semantics_of_the_notation_system (end)   

\input{qm2pi.qmops} 

\input{qm2pi.sterngerlach} 

\input{qm2pi.metric} 

% section concurrent_process_calculi (end)

%\input{qm2pi.proofsketch}

% section proof sketch (end)

%\input{qm2pi.slviaknots} 

% section spatial logic via knots (end)

\input{qm2pi.conclusion}

% section conclusion (end)

%\input{qm2pi.dtcodes} 

% section wiring algorithm (end)

\input{qm2pi.ack} 

% section acknowledgments (end)

\newpage


\bibliographystyle{plain}   
\bibliography{../../biblios/main.bib}

\input{qm2pi.rhodetails}

\end{document}

 

%\ifpdf
%\usepackage[pdftex]{graphicx}
%\else
%\usepackage{graphicx}
%\fi

 % \ifpdf
%  \usepackage{pdfsync}
%  \if


%\title{Brief Article}
%\author{David F. Snyder}
%\author{L.G. Meredith}

%\address{Dept. of Math., Texas State University--San Marcos, San Marcos, TX 78666}
       
\pagestyle{empty}


\begin{document}

\lstset{language=[Objective]Caml,frame=shadowbox}

\documentclass[12pt]{llncs}
%\documentclass{jktr}

\usepackage[pdftex]{hyperref}                   
\usepackage {listings}
\usepackage {mathpartir}
\usepackage{bcprules}
%\usepackage{listings}
                       
\usepackage{graphicx} 
%\usepackage[margins=2.5cm,nohead,nofoot]{geometry}
%\usepackage{geometry}
\usepackage{amsfonts}
\usepackage{amstext}
\usepackage{latexsym}
\usepackage{amssymb}
\usepackage{color}


%\include{myPreamble}
\include{qm2pi.local} 

%\ifpdf
%\usepackage[pdftex]{graphicx}
%\else
%\usepackage{graphicx}
%\fi

 % \ifpdf
%  \usepackage{pdfsync}
%  \if


%\title{Brief Article}
%\author{David F. Snyder}
%\author{L.G. Meredith}

%\address{Dept. of Math., Texas State University--San Marcos, San Marcos, TX 78666}
       
\pagestyle{empty}


\begin{document}

\lstset{language=[Objective]Caml,frame=shadowbox}

\input{qm2pi.front}

% section front matter (end)

\input{qm2pi.intro} 
 
% section introduction (end)

% \input{qm2pi.knotations} 

% section notation (end)

\input{qm2pi.process.calculi} 

% section concurrent_process_calculi_and_spatial_logics_ (end)
    
%\input{qm2pi.knots2pi} 

%\input{qm2pi.trefoil} 

%\input{qm2pi.mainthm} 

% subsection basic_interpretation (end)

%\input{qm2pi.rho.presentation} 
\subsection{The syntax and semantics of the notation system}\label{sub:the_syntax_and_semantics_of_the_notation_system} % (fold)

We now summarize a technical presentation of the calculus that
embodies our theory of dynamics. The typical presentation of such a
calculus follows the style of giving generators and relations on
them. The grammar, below, describing term constructors, freely
generates the set of processes, $\Proc$. This set is then quotiented
by a relation known as structural congruence and it is over this set
that the notion of dynamics is expressed. This presentation is
essentially that of \cite{MeredithR05} with the addition of
polyadicity and summation. For readability we have relegated some of
the technical subtleties to an appendix.

\subsubsection{Process grammar}\label{subsub:process_grammar}

\begin{mathpar}
  \inferrule* [lab=synchronization] {} {{M} \bc \pzero \;|\; x?F \;|\; x!C }
  \and
  \inferrule* [lab=abstraction] {} {{F} \bc (x)P}
  \and
  \inferrule* [lab=concretion] {} {{C} \bc \langle Q \rangle}
  \and
  \inferrule* [lab=process] {} {{P,Q} \bc M \;| \;P|Q \;|\; @{x}}
  \and
  \inferrule* [lab=name] {} {{x} \bc \quotep{P}}
\end{mathpar} 

Note that $\vec{x}$ (resp. $\vec{P}$) denotes a vector of names
(resp. processes) of length $|\vec{x}|$ (resp. $|\vec{P}|$). We adopt
the following useful abbreviations.

\begin{mathpar}
   x?(\vec{y}).P := x.(\vec{y})P \and  x\clift{\vec{P}} := x.\clift{\vec{P}}
   \and x!(y) := \lift{x}{\dropn{y}}
   \and \Pi_{i=0}^{n-1}P_i := P_0 | \ldots | P_{n-1}
\end{mathpar}

\subsubsection{Structural congruence}

\paragraph{Free and bound names and alpha-equivalence.} At the
core of structural equivalence is alpha-equivalence which identifies
process that are the same up to a change of variable. Formally, we
recognize the distinction between free and bound names. The free names
of a process, $\freenames{P}$, may be calculated recursively as
follows:

\begin{mathpar}
\freenames{\pzero} := \emptyset
  \and \\
  \freenames{x?(y).P} := \{ x \} \cup (\freenames{P} \setminus \{ y \})
  \and 
  \freenames{x!\langle P \rangle} := \{ x \} \cup \{ P \} 
  \and \\
  \freenames{P|Q} := \freenames{P} \cup \freenames{Q}
  \and \\
  \freenames{@{x}} := \{ x \}
\end{mathpar}

$\pi$
$\quotep{\pi}$

$\freenames{-} : \pi \to \mathcal{P}(\quotep{\pi})$

\begin{eqnarray*}
  \freenames{\pzero} & := & \emptyset \\
  \freenames{x?(y).P} & := & \{ x \} \cup (\freenames{P} \setminus \{ y \}) \\
  \freenames{x!\langle P \rangle} & := & \{ x \} \cup \{ P \} \\
  \freenames{P|Q} & := & \freenames{P} \cup \freenames{Q} \\
  \freenames{\dropn{x}} & := & \{ x \}
\end{eqnarray*}

The bound names of a process, $\boundnames{P}$, are those names occurring in $P$
that are not free. For example, in $x?(y).0$, the name $x$ is free, while $y$ is bound.

\begin{mathpar}
  \inferrule* [lab=monoidal-laws] {} { P|Q \equiv Q|P \and P|0 \equiv P \and P|(Q|R) \equiv (P|Q)|R }
\end{mathpar}

\begin{mathpar}
  \inferrule* [lab=alpha-equivalence] {} { (x)P \equiv (y)P\{y/x\} \and y \not\in \freenames{P} }
\end{mathpar}

\begin{definition}
Then two processes, $P,Q$, are alpha-equivalent if $P = Q\{\vec{y}/\vec{x}\}$ for
some $\vec{x} \in \boundnames{Q},\vec{y} \in \boundnames{P}$, where $Q\{\vec{y}/\vec{x}\}$
denotes the capture-avoiding substitution of $\vec{y}$ for $\vec{x}$ in $Q$.
\end{definition}

\begin{definition}
  The {\em structural congruence} \cite{SangiorgiWalker} , $\equiv$,
  between processes is the least congruence containing
  alpha-equivalence, satisfying the abelian monoid laws
  (associativity, commutativity and $\pzero$ as identity) for parallel
  composition $|$ and for summation $+$.
\end{definition}

\subsection{Name equivalence}

We take name equivalence, written $\nameeq$, to be the smallest
equivalence relation generated by the following rules.

\begin{mathpar}
\inferrule*[lab=Quote-drop]
{ }
{ \quotep{@{x}} \nameeq x }

\inferrule*[lab=Struct-equiv]
{ P \scong Q }
{ \quotep{P} \nameeq \quotep{Q} }
\end{mathpar}

The astute reader will have noticed that the mutual recursion of names
and processes imposes a mutual recursion on alpha-equivalence and
structural equivalence via name-equivalence. Fortunately, all of this
works out pleasantly and we may calculate in the natural way, free of
concern. The reader interested in the details is referred to the
appendix \ref{appendix:rho_details}.

\subsection{Substitution}

We use $\Proc$ for the set of processes, $\QProc$ for the set of
names, and $\id{\{}\vec{y} / \vec{x} \id{\}}$ to denote partial maps,
$s : \QProc \rightarrow \QProc$. A map, $s$ lifts, uniquely, to a map
on process terms, $\widehat{s} : \Proc \rightarrow \Proc$ by the
following equations.

\begin{mathpar}
  (0) \psubstp{Q}{P} := 0 \\
  (R \juxtap S) \psubstp{Q}{P}
  :=    
  (R)\psubstp{Q}{P} \juxtap (S) \psubstp{Q}{P} \\
  (x?(y).R) \psubstp{Q}{P}    
  :=    
  (x)\substp{Q}{P} (z)\concat( (R \psubstn{z}{y}) \psubstp{Q}{P} ) \\
  (\lift{x}{R}) \psubstp{Q}{P}  
  :=
  \lift{(x)\substp{Q}{P}}{ R \psubstp{Q}{P} } \\
%   (\dropn{x})  \psubstp{Q}{P}       
%   := 
%   \left\{ 
%     \begin{array}{ccc} 
%       \dropn{\quotep{Q}} & & x \nameeq \quotep{P} \\
%       \dropn{x} & & otherwise \\
%     \end{array}
%   \right. 
  (\dropn{x})  \psubstp{Q}{P}       
  := 
  \left\{ 
    \begin{array}{ccc} 
      Q & & x \nameeq \quotep{P} \\
      \dropn{x} & & otherwise \\
    \end{array}
  \right.
\end{mathpar}
 

where

\begin{eqnarray}
  (x)\id{\{} \lpquote Q \rpquote / \lpquote P \rpquote \id{\}}            = 
  \left\{ 
    \begin{array}{ccc}
      \lpquote Q \rpquote & & x \nameeq \lpquote P \rpquote \\
      x & & otherwise \\
    \end{array}
  \right. \nonumber
\end{eqnarray}

and $z$ is chosen distinct from $\quotep{P}$, $\quotep{Q}$, the free
names in $Q$, and all the names in $R$. Our $\alpha$-equivalence will
be built in the standard way from this substitution.

\begin{remark}\label{rem:no_self_referential_names}
  One consequence of these definitions is that $\forall P. \quotep{P}
  \not\in \freenames{P}$.
\end{remark}

\subsection{ Dynamic quote: an example }

Anticipating something of what's to come, consider applying the
substitution, $\widehat{\id{\{}u / z \id{\}}}$, to the following pair
of processes, $\lift{w}{y!(z)}$ and $w[ \lpquote y!(z) \rpquote ]$.

\begin{eqnarray}
	\lift{w}{y!(z)}\widehat{\id{\{}u / z \id{\}}}
		& = &
		\lift{w}{y!(u)} \nonumber\\
	w[ \lpquote y!(z) \rpquote ] \widehat{ \id{\{}u / z \id{\}} }
		& = &
		w[ \lpquote y!(z) \rpquote ] \nonumber
\end{eqnarray}

Because the body of the process between quotes is impervious to
substitution, we get radically different answers. In fact, by
examining the first process in an input context,
e.g. $x?(z).\lift{w}{y!(z)}$, we see that the process under the lift
operator may be shaped by prefixed inputs binding a name inside it. In
this sense, the lift operator will be seen as a way to dynamically
construct processes before reifying them as names.

Finally equipped with these standard features we can present the
dynamics of the calculus.

\subsubsection{Operational semantics} 

Finally, we introduce the computational dynamics. What marks these
algebras as distinct from other more traditionally studied algebraic
structures, e.g. vector spaces or polynomial rings, is the manner in
which dynamics is captured. In traditional structures, dynamics is typically
expressed through morphisms between such structures, as in linear maps
between vector spaces or morphisms between rings. In algebras
associated with the semantics of computation, the dynamics is
expressed as part of the algebraic structure itself, through a
reduction reduction relation typically denoted by $\red$. Below, we
give a recursive presentation of this relation for the calculus used
in the encoding.

$\red \subseteq \pi \times \pi$
$\red : \pi \to \mathcal{P}(\pi)$

\begin{mathpar}
  \inferrule* [lab=Comm] { \textsf{match}( x_{src}, x_{trgt} ) } { x_{trgt}?(y)P \; | \; x_{src}!\langle {Q} \rangle \red P\{\quotep{Q}/y}\} }
  \and \\
  \inferrule* [lab=Par] {{P} \red {P}'} {{{P} | {Q}} \red {{P}' | {Q}}}
  \and
  \inferrule* [lab=Equiv]{{{P} \scong {P}'} \andalso {{P}' \red {Q}'} \andalso {{Q}' \scong {Q}}}{{P} \red {Q}}
\end{mathpar}

\begin{eqnarray*}
  match_{\equiv} (\quotep{P},\quotep{Q}) & := & P \equiv Q \\
  match_{\dagger}(\quotep{P},\quotep{Q}) & := & \forall R. P|Q \red^{*} R => R \red^{*} 0 \\
  match_{K}(\quotep{P},\quotep{Q}) & := & K \mbox{ for some context } K
\end{eqnarray*}

$u?(x)P | u!\langle Q \rangle \red P\{\quotep{Q}/x\}$

%We write $\wred$ for $\red^*$, and $P\red$ if $\exists Q $ such that $ P \red Q$.
We write $P\red$ if $\exists Q $ such that $ P \red Q$ and $P\not\red$, otherwise.

\section{Replication}

As mentioned before, it is known that replication (and hence
recursion) can be implemented in a higher-order process algebra
\cite{SangiorgiWalker}. As our first example of calculation with the
machinery thus far presented we give the construction explicitly in
the {\rhoc}.

\begin{eqnarray}
	D_{x} & := & \prefix{x}{y}{(\binpar{\outputp{x}{y}}{@{y}})} \nonumber\\
	\bangp_{x}{P} & := & \binpar{{x}!\langle{\binpar{D_{x}}{P}}\rangle}{D_{x}} \nonumber
\end{eqnarray}

\begin{eqnarray}
	\bangp_{x}{P} & & \nonumber\\
	=
	& {x}!\langle{(\prefix{x}{y}{(\outputp{x}{y} | @{y})) | P}}\rangle 
	      | \prefix{x}{y}{(\outputp{x}{y} | @{y})} & \nonumber\\
	\red
	& (\outputp{x}{y} | @{y})\substn{\quotep{(\prefix{x}{y}{(@{y} | \outputp{x}{y})) | P}}}{y} & \nonumber\\
	=
	& \outputp{x}{\quotep{(\prefix{x}{y}{(\outputp{x}{y} | @{y})) | P}}}
	  | {(\prefix{x}{y}{(\outputp{x}{y} | @{y})) | P}} & \nonumber\\
	\red
	& \ldots & \nonumber\\
	\red^*
	& P | P | \ldots & \nonumber
\end{eqnarray}

Of course, this encoding, as an implementation, runs away, unfolding
$\bangp{P}$ eagerly. A lazier and more implementable replication
operator, restricted to input-guarded processes, may be obtained as follows.

\begin{eqnarray}
\bangp{\prefix{u}{v}{P}} 
	:= 
	\binpar{\lift{x}{\prefix{u}{v}{(\binpar{D(x)}{P})}}}{D(x)} \nonumber
\end{eqnarray}

\begin{remark}
  Note that the lazier definition still does not deal with summation
  or mixed summation (i.e. sums over input and output). The reader is
  invited to construct definitions of replication that deal with these
  features. 

  Further, the definitions are parameterized in a name, $x$. Can you,
  gentle reader, make a definition that eliminates this parameter and
  guarantees no accidental interaction between the replication
  machinery and the process being replicated -- i.e. no accidental
  sharing of names used by the process to get its work done and the
  name(s) used by the replication to effect copying. This latter
  revision of the definition of replication is crucial to obtaining
  the expected identity $!!P \sim !P$.
\end{remark}

\begin{remark}\label{rem:paradoxical_combinator}
  The reader familiar with the lambda calculus will have noticed the
  similarity between $D$ and the paradoxical combinator.

  [Ed. note: the existence of this seems to suggest we have to be more
  restrictive on the set of processes and names we admit if we are to
  support no-cloning.]
\end{remark}

\subsubsection{Bisimulation}

The computational dynamics gives rise to another kind of equivalence,
the equivalence of computational behavior. As previously mentioned
this is typically captured \emph{via} some form of bisimulation.

% The notion we use in this paper is weak barbed bisimulation
% \cite{milner91polyadicpi}.

The notion we use in this paper is derived from weak barbed
bisimulation \cite{milner91polyadicpi}. 

\begin{definition}
An \emph{observation relation}, $\downarrow_{\mathcal N}$, over a set
of names, $\mathcal N$, is the smallest relation satisfying the rules
below.

\infrule[Out-barb]{y \in {\mathcal N}, \; x \nameeq y}
		  {\outputp{x}{v} \downarrow_{\mathcal N} x}
\infrule[Par-barb]{\mbox{$P\downarrow_{\mathcal N} x$ or $Q\downarrow_{\mathcal N} x$}}
		  {\binpar{P}{Q} \downarrow_{\mathcal N} x}

We write $P \Downarrow_{\mathcal N} x$ if there is $Q$ such that 
$P \wred Q$ and $Q \downarrow_{\mathcal N} x$.
\end{definition}

\begin{definition}
%\label{def.bbisim}
An  ${\mathcal N}$-\emph{barbed bisimulation} over a set of names, ${\mathcal N}$, is a symmetric binary relation 
${\mathcal S}_{\mathcal N}$ between agents such that $P\rel{S}_{\mathcal N}Q$ implies:
\begin{enumerate}
\item If $P \red P'$ then $Q \wred Q'$ and $P'\rel{S}_{\mathcal N} Q'$.
\item If $P\downarrow_{\mathcal N} x$, then $Q\Downarrow_{\mathcal N} x$.
\end{enumerate}
$P$ is ${\mathcal N}$-barbed bisimilar to $Q$, written
$P \wbbisim_{\mathcal N} Q$, if $P \rel{S}_{\mathcal N} Q$ for some ${\mathcal N}$-barbed bisimulation ${\mathcal S}_{\mathcal N}$.
\end{definition}

$\mathcal{R} \subseteq \pi \times \pi$

$P \mathcal{R} Q => \forall P'. P \red P' \Rightarrow \exists Q'. Q \red Q', P' \mathcal{R} Q'$

$P \vdash x \Rightarrow Q \vdash x$

\begin{mathpar}
  \inferrule*[lab=Out-barb]{x \nameeq y}{{y}!\langle{Q}\rangle \vdash x}
  \and
  \inferrule*[lab=Par-barb]{\mbox{$P\vdash x$ or $Q\vdash x$}}{\binpar{P}{Q} \vdash x}
\end{mathpar}

\subsubsection{Contexts}

One of the principle advantages of computational calculi like the
$\pi$-calculus is a well-defined notion of context,
contextual-equivalence and a correlation between
contextual-equivalence and notions of bisimulation. The notion of
context allows the decomposition of a process into (sub-)process and
its syntactic environment, its context. Thus, a context may be
thought of as a process with a ``hole'' (written $\Box$) in it. The
application of a context $M$ to a process $P$, written $M[P]$, is
tantamount to filling the hole in $M$ with $P$. In this paper we do
not need the full weight of this theory, but do make use of the notion
of context in the proof the main theorem. 

\begin{mathpar}
  \inferrule* [lab=summation] {} {{M_{M},M_{N}} \bc \Box \;|\; x.M_{A} \;|\; M_{M}+M_{N}}
  \and
  \inferrule* [lab=agent] {} {{M_{A}} \bc (\vec{x})M_{P} \;| \; \clift{P_0,\ldots,M_{P},\ldots,P_N}}
  \and \\
  \inferrule* [lab=process] {} {{M_{P}} \bc M_{N} \;| \;P|M_{P} }
\end{mathpar} 

\begin{mathpar}
  \inferrule* [lab=sychronization] {} {M_{N} \bc \Box \;|\; x?M_{F} \;|\; x!M_{C}}
  \and
  \inferrule* [lab=abstraction] {} {{M_{F}} \bc (x)M_{P} }
  \and
  \inferrule* [lab=concretion] {} {{M_{C}} \bc \langle M_{P} \rangle }
  \and \\
  \inferrule* [lab=process] {} {{M_{P}} \bc M_{N} \;| \;P|M_{P} }
\end{mathpar}

\begin{definition}[contextual application] Given a context $M$, and
  process $P$, we define the \emph{contextual application}, $M[P] :=
  M\{P/\Box\}$. That is, the contextual application of M to P is the
  substitution of $P$ for $\Box$ in $M$.
\end{definition}

$\meaningof{-} : L \to \mathcal{P}(\pi)$

\begin{mathpar}
  \inferrule* [lab=collection] {} {\meaningof{true} = \pi, \and \meaningof{~E} = \pi \setminus \meaningof{E}, \and \meaningof{E_{1} \& E_{2}} = \meaningof{E_{1}} \cap \meaningof{E_{2}}}
\end{mathpar}

\begin{mathpar}
  \inferrule* [lab=structure] {} {\meaningof{0} = \{ P \in \pi | P \equiv 0 \}, \and \\ \meaningof{E_1 | E_2} = \{ P \in \pi | P \equiv P_{1} | P_{2}, P_{1} \in \meaningof{E_{1}}, P_{2} \in \meaningof{E_2}\} }
\end{mathpar}

\begin{mathpar}
 \inferrule* [lab=behavior] {} {\meaningof{\langle a?b \rangle E} = \{ P \in \pi | P \equiv Q | u?(y)P', \\ \and \\\\ \and \\ \;\;\; u \in \meaningof{a}, \forall z.P'\{z/y\} \in \meaningof{E\{z/b\}}\}, \and \\ \meaningof{a!E} = \{ P \in \pi | P \equiv Q | x!\langle P' \rangle, x \in \meaningof{a} P' \in \meaningof{E}\} }
\end{mathpar}

\begin{mathpar}
 \inferrule* [lab=nominal] {} {\meaningof{\quotep{E}} = \{ \quotep{P} \in \quotep{\pi} | P \in \meaningof{E} \}, \and \meaningof{\quotep{P}} = \{ \quotep{Q} \in \quotep{\pi} | P \equiv Q \} \and \\ \meaningof{@\quotep{E}} = \{ P \in \pi | P \equiv @x, x \in \meaningof{E} \}}
\end{mathpar}

\begin{eqnarray*}
  \\
  \meaningof{-} : TS \to ST
\end{eqnarray*}

\begin{eqnarray*}
  \\
  L : TS \to ST
\end{eqnarray*}

\begin{eqnarray*}
  \\
  P \models E \iff P \in \meaningof{E}
\end{eqnarray*}

\begin{eqnarray*}
  P \approx_{L} Q \iff \forall E \in L. P \models E \iff Q \models E
\end{eqnarray*}

\begin{eqnarray*}
  P \approx_{K} Q
\end{eqnarray*}

\begin{eqnarray*}
  P \approx Q
\end{eqnarray*}

$\approx_{K} = \approx = \approx_{L}$

\subsubsection{Contextual duality}

Note that contexts extend the quotation operation to a family of
operations from processes to names. Given a context, $M$, we can
define a \emph{nominal context}, $\quotep{M}$ by $\quotep{M}[P] :=
\quotep{M[P]}$. To foreshadow what is to come we observe that these
operations enjoy a duality with processes very much like the duality
between vectors and maps from vectors to scalars.

Further, because the calculus is essentially higher-order, we have a
correspondence between contexts and processes. More specifically,
given a name $x$ and a context $M$ we can construct $M^{*}_{x}$ such
that 

\begin{mathpar}
  M^{*}_{x} | \lift{x}{P} \red M[P]
\end{mathpar}

namely,

\begin{mathpar}
  M^{*}_{x} := x?(u).M[\dropn{u}]
\end{mathpar}

The dependence of $M^{*}_{x}$ on a name makes it an abstraction, 

\begin{mathpar}
  M^{*} := (x)x?(u).M[\dropn{u}]
\end{mathpar}

\subsection{Additional notation}

It will sometimes be convenient to denote the process a name
quotes. We already have the notation $x = \quotep{P}$, but it will be
convenient to introduce an alternate notation, $\procn{x}$, when we
want to emphasize the connection to the use of the name. Note that, by
virtue of name equivalence, $\quotep{\procn{x}} \nameeq x$; so, the
notation is consistent with previous definitions.

Further, because names have structure it is possible to effect
substitutions on the basis of that structure. This means we need to
upgrade our notation for substitutions, which we accomplish by
adapting comprehension notation. Thus,

\begin{mathpar}
  P\{ y / x : x \in S \}
\end{mathpar}

is interpreted to mean the process derived from P by replacing (in a
capture-avoiding manner) each occurrence of $x$ in $S$ by $y$. For example,

\begin{mathpar}
  P\{ \quotep{\procn{x}|\procn{x}} / x : x \in \freenames{P} \}
\end{mathpar}

will replace each (occurrence) of a free name $x$ in $P$ by
$\quotep{\procn{x}|\procn{x}}$.

Also, we will avail ourselves of the notation $x^{L}$ and $x^{R}$ to
denote injections of a name into disjoint copies of the name
space. There are numerous ways to accomplish this. One example can be
found in \cite{MeredithR05}. This notation overloads to vectors of
names: $\vec{x}^{\pi} := (x_{i}^{\pi} \; : \; 0 \leq i < |\vec{x}| )$ where $\pi \in \{L,R\}$.

We also use $P^{\Box} := P|\Box$.

In \cite{MeredithR05} an interpretation of the new operator is
given. It turns out that there are several possible interpretations
all enjoying the requisite algebraic properties of the operator (see
\cite{milner91polyadicpi}). We will therefore make liberal use of
$(\nu\; \vec{x})P$.

% subsection the_syntax_and_semantics_of_the_notation_system (end)   

\input{qm2pi.qmops} 

\input{qm2pi.sterngerlach} 

\input{qm2pi.metric} 

% section concurrent_process_calculi (end)

%\input{qm2pi.proofsketch}

% section proof sketch (end)

%\input{qm2pi.slviaknots} 

% section spatial logic via knots (end)

\input{qm2pi.conclusion}

% section conclusion (end)

%\input{qm2pi.dtcodes} 

% section wiring algorithm (end)

\input{qm2pi.ack} 

% section acknowledgments (end)

\newpage


\bibliographystyle{plain}   
\bibliography{../../biblios/main.bib}

\input{qm2pi.rhodetails}

\end{document}



% section front matter (end)

\section{Introduction}\label{sec:introduction} % (fold)
In this draft of the material i am going to have to dispense with the
usual writing conventions adopted in papers on these topics. i'm going
to have adopt whatever tone i need at the time i'm writing up the
calculations. Sometimes this may be very conversational; others it may
be the barest mathematical grunts; others still it may be that i have
lifted text from one of my other papers because the exposition of some
point was better said there. i hope that my readers are not unduly put
out by this decision. i'm not doing this to flout convention or be
rebellious. i find these calculations very technically challenging. To
keep everything going technically, something has to give; i have to
let go of some cognitive burden. So, the academic writing style --
with all of its trade-offs in terms of facilitating technical
communication -- is what i'm letting go of. Perhaps subsequent drafts
can be tightened and polished, but for now, i'm going to speak as if
we were sitting together in a coffee shop with a laptop, wifi and a
pad of paper and a pencil.

So, here's what i have to say. We -- you and i, comfortably ensconced
in our coffee shop and well-equipped with our tools -- can realize and
carry out the calculations of quantum mechanics over a very different
formal theory of dynamics, a formal theory of dynamics that
corresponds to a theory of concurrent computation with
\emph{reflection}. It has the advantage that the underlying theory is
already `quantized', but supports analogues all of the continuuous
operations. Strikingly, this underlying theory has recently been
connected with a notion of metric that we can show, by calculating
together, coincides with the metric induced by the inner product.

There are a lot of reasons why you might be interested in seeing
calculations of this form. Here's why i'm interested. For the past
several centuries there has been no competitor to the ``Newtonian''
account of dynamics. As a result the predominant share of accounts of
dynamical systems and situations have had to be formulated in terms of
the Newtonian machinery. i view this as an intellectually dangerous
position to occupy. Everything, despite it's intrinsic shape, turns
into a nail to be hit with this hammer. Recently, however, the theory
of computation has matured to the point where we have candidates for
theories of dynamics that offer very different perspective on
reasoning about dynamical systems and situations. Testing these
candidates against very successful accounts of dynamical situations,
like quantum mechanics, is going to give us some sense of how mature
they are and some measure of the quality of these accounts of
dynamics.

\subsection{Summary of contributions and outline of paper}

So, we're going to develop an interpretation of the operations of
quantum mechanics normally interpreted by Hilbert spaces and
operators. We're going to do this over a theory of computation. Note
that this is very different than the usual quantum computation program
which develops notions of computation over quantum mechanics. Rather,
we are developing a story that aligns with Wheeler's slogan: It from
Bit. To do this we will first provide an account of the theory of
computation at play here. Then we will dive into a calculation-driven
interpretation of the operations of quantum mechanics.

The reason we take this approach is that -- until very recently --
there hasn't been an axiomatic account of quantum mechanics. As a
result there has been no sharp delineation of the mathematical theory
supporting interpretation of the physical theory and the physical
theory, itself. So, ambient features of the maths are free to be
exploited (or supressed) without a real accounting of their physical
relevance. There is no sharp statement ``here's the physical theory''
qua \emph{theory} and ``here's the mathematical interpretation''
enabling a judgment of how faithful the interpretation is -- apart
from experimental observation. When there is an axiomatic account we
can judge how well a given mathematical formalism supports an
interpretation of the axioms, independent of
experimentation. Likewise, we can judge how well we have captured our
physical evidence and experience with our axiomatics, independent of
any specific mathematical implementation, with accidental detail that
may or may not have physical significance. 

In lieu of a fully fleshed out and vetted axiomatic account of quantum
mechanics, interpreting the operational notions in service of modeling
physical systems will have to suffice. In other words, we are not in
the business of providing a model of Hilbert spaces and operators. We
are in the business of providing a model of quantum mechanics because
we are motivated by testing our notions of dynamics against physical
theory; and, the predictive calculations of the physical theory must
serve as the best formulation -- shy of a fully fleshed out axiomatic
account -- of the physical theory itself (as they have for scientific
theories since time immemorial). Put another way, despite a
whole-hearted commitment to an It-from-Bit ontology, we are firmly
aligned with the shut-up-and-calculate camp as the best way to obtain
results either from the physical perspective or as a quality assurance
measure of our fledgling theory of dynamics.

In detail, we present a reflective process calculus. Then we develop
intuitive correspondences between the notions available in this
calculus and the usual physical notions supporting quantum mechanical
calculations. Thus, 

\begin{table}[htp]
  \center{
    \fbox{
      \begin{tabular}{c|c}
        quantum mechanics & process calculus \\
        \hline
        scalar & name \\
        state vector & process \\
        dual & contextual duals \\
        matrix & formal sums of process-context-dual pairs \\
        orthogonality & process annihilation \\
        inner product & execution-formula + quoting
      \end{tabular}
    }
  }
  \caption{QM - process calculi correspondences}
\end{table}

Then we tighten up these intuitions to operational definitions. We
employ the Dirac notation as the best proxy we can find for an
abstract syntax of the quantum mechanical notions. The definitions we
develop put us in contact with equational constraints coming from the
theory that we demonstrate the definitions and calculations satisfy.

This puts us in a position to shut up and calculate for the
Stern-Gerlach experimental set up, showing how these predictive
calculations become calculations on processes in our theory of a
reflective process calculus.

Penultimately, we demonstrate that the notion of metric coming from
the inner product coincides with the notion of metric available from
the theory of bisimulation. This demonstration gives us the right to
think of space as arising from behavior. Finally, we consider where we
might go from the new vantage point we have obtained.

% section introduction (end) 
 
% section introduction (end)

% \documentclass[12pt]{llncs}
%\documentclass{jktr}

\usepackage[pdftex]{hyperref}                   
\usepackage {listings}
\usepackage {mathpartir}
\usepackage{bcprules}
%\usepackage{listings}
                       
\usepackage{graphicx} 
%\usepackage[margins=2.5cm,nohead,nofoot]{geometry}
%\usepackage{geometry}
\usepackage{amsfonts}
\usepackage{amstext}
\usepackage{latexsym}
\usepackage{amssymb}
\usepackage{color}


%\include{myPreamble}
\include{qm2pi.local} 

%\ifpdf
%\usepackage[pdftex]{graphicx}
%\else
%\usepackage{graphicx}
%\fi

 % \ifpdf
%  \usepackage{pdfsync}
%  \if


%\title{Brief Article}
%\author{David F. Snyder}
%\author{L.G. Meredith}

%\address{Dept. of Math., Texas State University--San Marcos, San Marcos, TX 78666}
       
\pagestyle{empty}


\begin{document}

\lstset{language=[Objective]Caml,frame=shadowbox}

\input{qm2pi.front}

% section front matter (end)

\input{qm2pi.intro} 
 
% section introduction (end)

% \input{qm2pi.knotations} 

% section notation (end)

\input{qm2pi.process.calculi} 

% section concurrent_process_calculi_and_spatial_logics_ (end)
    
%\input{qm2pi.knots2pi} 

%\input{qm2pi.trefoil} 

%\input{qm2pi.mainthm} 

% subsection basic_interpretation (end)

%\input{qm2pi.rho.presentation} 
\subsection{The syntax and semantics of the notation system}\label{sub:the_syntax_and_semantics_of_the_notation_system} % (fold)

We now summarize a technical presentation of the calculus that
embodies our theory of dynamics. The typical presentation of such a
calculus follows the style of giving generators and relations on
them. The grammar, below, describing term constructors, freely
generates the set of processes, $\Proc$. This set is then quotiented
by a relation known as structural congruence and it is over this set
that the notion of dynamics is expressed. This presentation is
essentially that of \cite{MeredithR05} with the addition of
polyadicity and summation. For readability we have relegated some of
the technical subtleties to an appendix.

\subsubsection{Process grammar}\label{subsub:process_grammar}

\begin{mathpar}
  \inferrule* [lab=synchronization] {} {{M} \bc \pzero \;|\; x?F \;|\; x!C }
  \and
  \inferrule* [lab=abstraction] {} {{F} \bc (x)P}
  \and
  \inferrule* [lab=concretion] {} {{C} \bc \langle Q \rangle}
  \and
  \inferrule* [lab=process] {} {{P,Q} \bc M \;| \;P|Q \;|\; @{x}}
  \and
  \inferrule* [lab=name] {} {{x} \bc \quotep{P}}
\end{mathpar} 

Note that $\vec{x}$ (resp. $\vec{P}$) denotes a vector of names
(resp. processes) of length $|\vec{x}|$ (resp. $|\vec{P}|$). We adopt
the following useful abbreviations.

\begin{mathpar}
   x?(\vec{y}).P := x.(\vec{y})P \and  x\clift{\vec{P}} := x.\clift{\vec{P}}
   \and x!(y) := \lift{x}{\dropn{y}}
   \and \Pi_{i=0}^{n-1}P_i := P_0 | \ldots | P_{n-1}
\end{mathpar}

\subsubsection{Structural congruence}

\paragraph{Free and bound names and alpha-equivalence.} At the
core of structural equivalence is alpha-equivalence which identifies
process that are the same up to a change of variable. Formally, we
recognize the distinction between free and bound names. The free names
of a process, $\freenames{P}$, may be calculated recursively as
follows:

\begin{mathpar}
\freenames{\pzero} := \emptyset
  \and \\
  \freenames{x?(y).P} := \{ x \} \cup (\freenames{P} \setminus \{ y \})
  \and 
  \freenames{x!\langle P \rangle} := \{ x \} \cup \{ P \} 
  \and \\
  \freenames{P|Q} := \freenames{P} \cup \freenames{Q}
  \and \\
  \freenames{@{x}} := \{ x \}
\end{mathpar}

$\pi$
$\quotep{\pi}$

$\freenames{-} : \pi \to \mathcal{P}(\quotep{\pi})$

\begin{eqnarray*}
  \freenames{\pzero} & := & \emptyset \\
  \freenames{x?(y).P} & := & \{ x \} \cup (\freenames{P} \setminus \{ y \}) \\
  \freenames{x!\langle P \rangle} & := & \{ x \} \cup \{ P \} \\
  \freenames{P|Q} & := & \freenames{P} \cup \freenames{Q} \\
  \freenames{\dropn{x}} & := & \{ x \}
\end{eqnarray*}

The bound names of a process, $\boundnames{P}$, are those names occurring in $P$
that are not free. For example, in $x?(y).0$, the name $x$ is free, while $y$ is bound.

\begin{mathpar}
  \inferrule* [lab=monoidal-laws] {} { P|Q \equiv Q|P \and P|0 \equiv P \and P|(Q|R) \equiv (P|Q)|R }
\end{mathpar}

\begin{mathpar}
  \inferrule* [lab=alpha-equivalence] {} { (x)P \equiv (y)P\{y/x\} \and y \not\in \freenames{P} }
\end{mathpar}

\begin{definition}
Then two processes, $P,Q$, are alpha-equivalent if $P = Q\{\vec{y}/\vec{x}\}$ for
some $\vec{x} \in \boundnames{Q},\vec{y} \in \boundnames{P}$, where $Q\{\vec{y}/\vec{x}\}$
denotes the capture-avoiding substitution of $\vec{y}$ for $\vec{x}$ in $Q$.
\end{definition}

\begin{definition}
  The {\em structural congruence} \cite{SangiorgiWalker} , $\equiv$,
  between processes is the least congruence containing
  alpha-equivalence, satisfying the abelian monoid laws
  (associativity, commutativity and $\pzero$ as identity) for parallel
  composition $|$ and for summation $+$.
\end{definition}

\subsection{Name equivalence}

We take name equivalence, written $\nameeq$, to be the smallest
equivalence relation generated by the following rules.

\begin{mathpar}
\inferrule*[lab=Quote-drop]
{ }
{ \quotep{@{x}} \nameeq x }

\inferrule*[lab=Struct-equiv]
{ P \scong Q }
{ \quotep{P} \nameeq \quotep{Q} }
\end{mathpar}

The astute reader will have noticed that the mutual recursion of names
and processes imposes a mutual recursion on alpha-equivalence and
structural equivalence via name-equivalence. Fortunately, all of this
works out pleasantly and we may calculate in the natural way, free of
concern. The reader interested in the details is referred to the
appendix \ref{appendix:rho_details}.

\subsection{Substitution}

We use $\Proc$ for the set of processes, $\QProc$ for the set of
names, and $\id{\{}\vec{y} / \vec{x} \id{\}}$ to denote partial maps,
$s : \QProc \rightarrow \QProc$. A map, $s$ lifts, uniquely, to a map
on process terms, $\widehat{s} : \Proc \rightarrow \Proc$ by the
following equations.

\begin{mathpar}
  (0) \psubstp{Q}{P} := 0 \\
  (R \juxtap S) \psubstp{Q}{P}
  :=    
  (R)\psubstp{Q}{P} \juxtap (S) \psubstp{Q}{P} \\
  (x?(y).R) \psubstp{Q}{P}    
  :=    
  (x)\substp{Q}{P} (z)\concat( (R \psubstn{z}{y}) \psubstp{Q}{P} ) \\
  (\lift{x}{R}) \psubstp{Q}{P}  
  :=
  \lift{(x)\substp{Q}{P}}{ R \psubstp{Q}{P} } \\
%   (\dropn{x})  \psubstp{Q}{P}       
%   := 
%   \left\{ 
%     \begin{array}{ccc} 
%       \dropn{\quotep{Q}} & & x \nameeq \quotep{P} \\
%       \dropn{x} & & otherwise \\
%     \end{array}
%   \right. 
  (\dropn{x})  \psubstp{Q}{P}       
  := 
  \left\{ 
    \begin{array}{ccc} 
      Q & & x \nameeq \quotep{P} \\
      \dropn{x} & & otherwise \\
    \end{array}
  \right.
\end{mathpar}
 

where

\begin{eqnarray}
  (x)\id{\{} \lpquote Q \rpquote / \lpquote P \rpquote \id{\}}            = 
  \left\{ 
    \begin{array}{ccc}
      \lpquote Q \rpquote & & x \nameeq \lpquote P \rpquote \\
      x & & otherwise \\
    \end{array}
  \right. \nonumber
\end{eqnarray}

and $z$ is chosen distinct from $\quotep{P}$, $\quotep{Q}$, the free
names in $Q$, and all the names in $R$. Our $\alpha$-equivalence will
be built in the standard way from this substitution.

\begin{remark}\label{rem:no_self_referential_names}
  One consequence of these definitions is that $\forall P. \quotep{P}
  \not\in \freenames{P}$.
\end{remark}

\subsection{ Dynamic quote: an example }

Anticipating something of what's to come, consider applying the
substitution, $\widehat{\id{\{}u / z \id{\}}}$, to the following pair
of processes, $\lift{w}{y!(z)}$ and $w[ \lpquote y!(z) \rpquote ]$.

\begin{eqnarray}
	\lift{w}{y!(z)}\widehat{\id{\{}u / z \id{\}}}
		& = &
		\lift{w}{y!(u)} \nonumber\\
	w[ \lpquote y!(z) \rpquote ] \widehat{ \id{\{}u / z \id{\}} }
		& = &
		w[ \lpquote y!(z) \rpquote ] \nonumber
\end{eqnarray}

Because the body of the process between quotes is impervious to
substitution, we get radically different answers. In fact, by
examining the first process in an input context,
e.g. $x?(z).\lift{w}{y!(z)}$, we see that the process under the lift
operator may be shaped by prefixed inputs binding a name inside it. In
this sense, the lift operator will be seen as a way to dynamically
construct processes before reifying them as names.

Finally equipped with these standard features we can present the
dynamics of the calculus.

\subsubsection{Operational semantics} 

Finally, we introduce the computational dynamics. What marks these
algebras as distinct from other more traditionally studied algebraic
structures, e.g. vector spaces or polynomial rings, is the manner in
which dynamics is captured. In traditional structures, dynamics is typically
expressed through morphisms between such structures, as in linear maps
between vector spaces or morphisms between rings. In algebras
associated with the semantics of computation, the dynamics is
expressed as part of the algebraic structure itself, through a
reduction reduction relation typically denoted by $\red$. Below, we
give a recursive presentation of this relation for the calculus used
in the encoding.

$\red \subseteq \pi \times \pi$
$\red : \pi \to \mathcal{P}(\pi)$

\begin{mathpar}
  \inferrule* [lab=Comm] { \textsf{match}( x_{src}, x_{trgt} ) } { x_{trgt}?(y)P \; | \; x_{src}!\langle {Q} \rangle \red P\{\quotep{Q}/y}\} }
  \and \\
  \inferrule* [lab=Par] {{P} \red {P}'} {{{P} | {Q}} \red {{P}' | {Q}}}
  \and
  \inferrule* [lab=Equiv]{{{P} \scong {P}'} \andalso {{P}' \red {Q}'} \andalso {{Q}' \scong {Q}}}{{P} \red {Q}}
\end{mathpar}

\begin{eqnarray*}
  match_{\equiv} (\quotep{P},\quotep{Q}) & := & P \equiv Q \\
  match_{\dagger}(\quotep{P},\quotep{Q}) & := & \forall R. P|Q \red^{*} R => R \red^{*} 0 \\
  match_{K}(\quotep{P},\quotep{Q}) & := & K \mbox{ for some context } K
\end{eqnarray*}

$u?(x)P | u!\langle Q \rangle \red P\{\quotep{Q}/x\}$

%We write $\wred$ for $\red^*$, and $P\red$ if $\exists Q $ such that $ P \red Q$.
We write $P\red$ if $\exists Q $ such that $ P \red Q$ and $P\not\red$, otherwise.

\section{Replication}

As mentioned before, it is known that replication (and hence
recursion) can be implemented in a higher-order process algebra
\cite{SangiorgiWalker}. As our first example of calculation with the
machinery thus far presented we give the construction explicitly in
the {\rhoc}.

\begin{eqnarray}
	D_{x} & := & \prefix{x}{y}{(\binpar{\outputp{x}{y}}{@{y}})} \nonumber\\
	\bangp_{x}{P} & := & \binpar{{x}!\langle{\binpar{D_{x}}{P}}\rangle}{D_{x}} \nonumber
\end{eqnarray}

\begin{eqnarray}
	\bangp_{x}{P} & & \nonumber\\
	=
	& {x}!\langle{(\prefix{x}{y}{(\outputp{x}{y} | @{y})) | P}}\rangle 
	      | \prefix{x}{y}{(\outputp{x}{y} | @{y})} & \nonumber\\
	\red
	& (\outputp{x}{y} | @{y})\substn{\quotep{(\prefix{x}{y}{(@{y} | \outputp{x}{y})) | P}}}{y} & \nonumber\\
	=
	& \outputp{x}{\quotep{(\prefix{x}{y}{(\outputp{x}{y} | @{y})) | P}}}
	  | {(\prefix{x}{y}{(\outputp{x}{y} | @{y})) | P}} & \nonumber\\
	\red
	& \ldots & \nonumber\\
	\red^*
	& P | P | \ldots & \nonumber
\end{eqnarray}

Of course, this encoding, as an implementation, runs away, unfolding
$\bangp{P}$ eagerly. A lazier and more implementable replication
operator, restricted to input-guarded processes, may be obtained as follows.

\begin{eqnarray}
\bangp{\prefix{u}{v}{P}} 
	:= 
	\binpar{\lift{x}{\prefix{u}{v}{(\binpar{D(x)}{P})}}}{D(x)} \nonumber
\end{eqnarray}

\begin{remark}
  Note that the lazier definition still does not deal with summation
  or mixed summation (i.e. sums over input and output). The reader is
  invited to construct definitions of replication that deal with these
  features. 

  Further, the definitions are parameterized in a name, $x$. Can you,
  gentle reader, make a definition that eliminates this parameter and
  guarantees no accidental interaction between the replication
  machinery and the process being replicated -- i.e. no accidental
  sharing of names used by the process to get its work done and the
  name(s) used by the replication to effect copying. This latter
  revision of the definition of replication is crucial to obtaining
  the expected identity $!!P \sim !P$.
\end{remark}

\begin{remark}\label{rem:paradoxical_combinator}
  The reader familiar with the lambda calculus will have noticed the
  similarity between $D$ and the paradoxical combinator.

  [Ed. note: the existence of this seems to suggest we have to be more
  restrictive on the set of processes and names we admit if we are to
  support no-cloning.]
\end{remark}

\subsubsection{Bisimulation}

The computational dynamics gives rise to another kind of equivalence,
the equivalence of computational behavior. As previously mentioned
this is typically captured \emph{via} some form of bisimulation.

% The notion we use in this paper is weak barbed bisimulation
% \cite{milner91polyadicpi}.

The notion we use in this paper is derived from weak barbed
bisimulation \cite{milner91polyadicpi}. 

\begin{definition}
An \emph{observation relation}, $\downarrow_{\mathcal N}$, over a set
of names, $\mathcal N$, is the smallest relation satisfying the rules
below.

\infrule[Out-barb]{y \in {\mathcal N}, \; x \nameeq y}
		  {\outputp{x}{v} \downarrow_{\mathcal N} x}
\infrule[Par-barb]{\mbox{$P\downarrow_{\mathcal N} x$ or $Q\downarrow_{\mathcal N} x$}}
		  {\binpar{P}{Q} \downarrow_{\mathcal N} x}

We write $P \Downarrow_{\mathcal N} x$ if there is $Q$ such that 
$P \wred Q$ and $Q \downarrow_{\mathcal N} x$.
\end{definition}

\begin{definition}
%\label{def.bbisim}
An  ${\mathcal N}$-\emph{barbed bisimulation} over a set of names, ${\mathcal N}$, is a symmetric binary relation 
${\mathcal S}_{\mathcal N}$ between agents such that $P\rel{S}_{\mathcal N}Q$ implies:
\begin{enumerate}
\item If $P \red P'$ then $Q \wred Q'$ and $P'\rel{S}_{\mathcal N} Q'$.
\item If $P\downarrow_{\mathcal N} x$, then $Q\Downarrow_{\mathcal N} x$.
\end{enumerate}
$P$ is ${\mathcal N}$-barbed bisimilar to $Q$, written
$P \wbbisim_{\mathcal N} Q$, if $P \rel{S}_{\mathcal N} Q$ for some ${\mathcal N}$-barbed bisimulation ${\mathcal S}_{\mathcal N}$.
\end{definition}

$\mathcal{R} \subseteq \pi \times \pi$

$P \mathcal{R} Q => \forall P'. P \red P' \Rightarrow \exists Q'. Q \red Q', P' \mathcal{R} Q'$

$P \vdash x \Rightarrow Q \vdash x$

\begin{mathpar}
  \inferrule*[lab=Out-barb]{x \nameeq y}{{y}!\langle{Q}\rangle \vdash x}
  \and
  \inferrule*[lab=Par-barb]{\mbox{$P\vdash x$ or $Q\vdash x$}}{\binpar{P}{Q} \vdash x}
\end{mathpar}

\subsubsection{Contexts}

One of the principle advantages of computational calculi like the
$\pi$-calculus is a well-defined notion of context,
contextual-equivalence and a correlation between
contextual-equivalence and notions of bisimulation. The notion of
context allows the decomposition of a process into (sub-)process and
its syntactic environment, its context. Thus, a context may be
thought of as a process with a ``hole'' (written $\Box$) in it. The
application of a context $M$ to a process $P$, written $M[P]$, is
tantamount to filling the hole in $M$ with $P$. In this paper we do
not need the full weight of this theory, but do make use of the notion
of context in the proof the main theorem. 

\begin{mathpar}
  \inferrule* [lab=summation] {} {{M_{M},M_{N}} \bc \Box \;|\; x.M_{A} \;|\; M_{M}+M_{N}}
  \and
  \inferrule* [lab=agent] {} {{M_{A}} \bc (\vec{x})M_{P} \;| \; \clift{P_0,\ldots,M_{P},\ldots,P_N}}
  \and \\
  \inferrule* [lab=process] {} {{M_{P}} \bc M_{N} \;| \;P|M_{P} }
\end{mathpar} 

\begin{mathpar}
  \inferrule* [lab=sychronization] {} {M_{N} \bc \Box \;|\; x?M_{F} \;|\; x!M_{C}}
  \and
  \inferrule* [lab=abstraction] {} {{M_{F}} \bc (x)M_{P} }
  \and
  \inferrule* [lab=concretion] {} {{M_{C}} \bc \langle M_{P} \rangle }
  \and \\
  \inferrule* [lab=process] {} {{M_{P}} \bc M_{N} \;| \;P|M_{P} }
\end{mathpar}

\begin{definition}[contextual application] Given a context $M$, and
  process $P$, we define the \emph{contextual application}, $M[P] :=
  M\{P/\Box\}$. That is, the contextual application of M to P is the
  substitution of $P$ for $\Box$ in $M$.
\end{definition}

$\meaningof{-} : L \to \mathcal{P}(\pi)$

\begin{mathpar}
  \inferrule* [lab=collection] {} {\meaningof{true} = \pi, \and \meaningof{~E} = \pi \setminus \meaningof{E}, \and \meaningof{E_{1} \& E_{2}} = \meaningof{E_{1}} \cap \meaningof{E_{2}}}
\end{mathpar}

\begin{mathpar}
  \inferrule* [lab=structure] {} {\meaningof{0} = \{ P \in \pi | P \equiv 0 \}, \and \\ \meaningof{E_1 | E_2} = \{ P \in \pi | P \equiv P_{1} | P_{2}, P_{1} \in \meaningof{E_{1}}, P_{2} \in \meaningof{E_2}\} }
\end{mathpar}

\begin{mathpar}
 \inferrule* [lab=behavior] {} {\meaningof{\langle a?b \rangle E} = \{ P \in \pi | P \equiv Q | u?(y)P', \\ \and \\\\ \and \\ \;\;\; u \in \meaningof{a}, \forall z.P'\{z/y\} \in \meaningof{E\{z/b\}}\}, \and \\ \meaningof{a!E} = \{ P \in \pi | P \equiv Q | x!\langle P' \rangle, x \in \meaningof{a} P' \in \meaningof{E}\} }
\end{mathpar}

\begin{mathpar}
 \inferrule* [lab=nominal] {} {\meaningof{\quotep{E}} = \{ \quotep{P} \in \quotep{\pi} | P \in \meaningof{E} \}, \and \meaningof{\quotep{P}} = \{ \quotep{Q} \in \quotep{\pi} | P \equiv Q \} \and \\ \meaningof{@\quotep{E}} = \{ P \in \pi | P \equiv @x, x \in \meaningof{E} \}}
\end{mathpar}

\begin{eqnarray*}
  \\
  \meaningof{-} : TS \to ST
\end{eqnarray*}

\begin{eqnarray*}
  \\
  L : TS \to ST
\end{eqnarray*}

\begin{eqnarray*}
  \\
  P \models E \iff P \in \meaningof{E}
\end{eqnarray*}

\begin{eqnarray*}
  P \approx_{L} Q \iff \forall E \in L. P \models E \iff Q \models E
\end{eqnarray*}

\begin{eqnarray*}
  P \approx_{K} Q
\end{eqnarray*}

\begin{eqnarray*}
  P \approx Q
\end{eqnarray*}

$\approx_{K} = \approx = \approx_{L}$

\subsubsection{Contextual duality}

Note that contexts extend the quotation operation to a family of
operations from processes to names. Given a context, $M$, we can
define a \emph{nominal context}, $\quotep{M}$ by $\quotep{M}[P] :=
\quotep{M[P]}$. To foreshadow what is to come we observe that these
operations enjoy a duality with processes very much like the duality
between vectors and maps from vectors to scalars.

Further, because the calculus is essentially higher-order, we have a
correspondence between contexts and processes. More specifically,
given a name $x$ and a context $M$ we can construct $M^{*}_{x}$ such
that 

\begin{mathpar}
  M^{*}_{x} | \lift{x}{P} \red M[P]
\end{mathpar}

namely,

\begin{mathpar}
  M^{*}_{x} := x?(u).M[\dropn{u}]
\end{mathpar}

The dependence of $M^{*}_{x}$ on a name makes it an abstraction, 

\begin{mathpar}
  M^{*} := (x)x?(u).M[\dropn{u}]
\end{mathpar}

\subsection{Additional notation}

It will sometimes be convenient to denote the process a name
quotes. We already have the notation $x = \quotep{P}$, but it will be
convenient to introduce an alternate notation, $\procn{x}$, when we
want to emphasize the connection to the use of the name. Note that, by
virtue of name equivalence, $\quotep{\procn{x}} \nameeq x$; so, the
notation is consistent with previous definitions.

Further, because names have structure it is possible to effect
substitutions on the basis of that structure. This means we need to
upgrade our notation for substitutions, which we accomplish by
adapting comprehension notation. Thus,

\begin{mathpar}
  P\{ y / x : x \in S \}
\end{mathpar}

is interpreted to mean the process derived from P by replacing (in a
capture-avoiding manner) each occurrence of $x$ in $S$ by $y$. For example,

\begin{mathpar}
  P\{ \quotep{\procn{x}|\procn{x}} / x : x \in \freenames{P} \}
\end{mathpar}

will replace each (occurrence) of a free name $x$ in $P$ by
$\quotep{\procn{x}|\procn{x}}$.

Also, we will avail ourselves of the notation $x^{L}$ and $x^{R}$ to
denote injections of a name into disjoint copies of the name
space. There are numerous ways to accomplish this. One example can be
found in \cite{MeredithR05}. This notation overloads to vectors of
names: $\vec{x}^{\pi} := (x_{i}^{\pi} \; : \; 0 \leq i < |\vec{x}| )$ where $\pi \in \{L,R\}$.

We also use $P^{\Box} := P|\Box$.

In \cite{MeredithR05} an interpretation of the new operator is
given. It turns out that there are several possible interpretations
all enjoying the requisite algebraic properties of the operator (see
\cite{milner91polyadicpi}). We will therefore make liberal use of
$(\nu\; \vec{x})P$.

% subsection the_syntax_and_semantics_of_the_notation_system (end)   

\input{qm2pi.qmops} 

\input{qm2pi.sterngerlach} 

\input{qm2pi.metric} 

% section concurrent_process_calculi (end)

%\input{qm2pi.proofsketch}

% section proof sketch (end)

%\input{qm2pi.slviaknots} 

% section spatial logic via knots (end)

\input{qm2pi.conclusion}

% section conclusion (end)

%\input{qm2pi.dtcodes} 

% section wiring algorithm (end)

\input{qm2pi.ack} 

% section acknowledgments (end)

\newpage


\bibliographystyle{plain}   
\bibliography{../../biblios/main.bib}

\input{qm2pi.rhodetails}

\end{document}

 

% section notation (end)

\input{qm2pi.process.calculi} 

% section concurrent_process_calculi_and_spatial_logics_ (end)
    
%\documentclass[12pt]{llncs}
%\documentclass{jktr}

\usepackage[pdftex]{hyperref}                   
\usepackage {listings}
\usepackage {mathpartir}
\usepackage{bcprules}
%\usepackage{listings}
                       
\usepackage{graphicx} 
%\usepackage[margins=2.5cm,nohead,nofoot]{geometry}
%\usepackage{geometry}
\usepackage{amsfonts}
\usepackage{amstext}
\usepackage{latexsym}
\usepackage{amssymb}
\usepackage{color}


%\include{myPreamble}
\include{qm2pi.local} 

%\ifpdf
%\usepackage[pdftex]{graphicx}
%\else
%\usepackage{graphicx}
%\fi

 % \ifpdf
%  \usepackage{pdfsync}
%  \if


%\title{Brief Article}
%\author{David F. Snyder}
%\author{L.G. Meredith}

%\address{Dept. of Math., Texas State University--San Marcos, San Marcos, TX 78666}
       
\pagestyle{empty}


\begin{document}

\lstset{language=[Objective]Caml,frame=shadowbox}

\input{qm2pi.front}

% section front matter (end)

\input{qm2pi.intro} 
 
% section introduction (end)

% \input{qm2pi.knotations} 

% section notation (end)

\input{qm2pi.process.calculi} 

% section concurrent_process_calculi_and_spatial_logics_ (end)
    
%\input{qm2pi.knots2pi} 

%\input{qm2pi.trefoil} 

%\input{qm2pi.mainthm} 

% subsection basic_interpretation (end)

%\input{qm2pi.rho.presentation} 
\subsection{The syntax and semantics of the notation system}\label{sub:the_syntax_and_semantics_of_the_notation_system} % (fold)

We now summarize a technical presentation of the calculus that
embodies our theory of dynamics. The typical presentation of such a
calculus follows the style of giving generators and relations on
them. The grammar, below, describing term constructors, freely
generates the set of processes, $\Proc$. This set is then quotiented
by a relation known as structural congruence and it is over this set
that the notion of dynamics is expressed. This presentation is
essentially that of \cite{MeredithR05} with the addition of
polyadicity and summation. For readability we have relegated some of
the technical subtleties to an appendix.

\subsubsection{Process grammar}\label{subsub:process_grammar}

\begin{mathpar}
  \inferrule* [lab=synchronization] {} {{M} \bc \pzero \;|\; x?F \;|\; x!C }
  \and
  \inferrule* [lab=abstraction] {} {{F} \bc (x)P}
  \and
  \inferrule* [lab=concretion] {} {{C} \bc \langle Q \rangle}
  \and
  \inferrule* [lab=process] {} {{P,Q} \bc M \;| \;P|Q \;|\; @{x}}
  \and
  \inferrule* [lab=name] {} {{x} \bc \quotep{P}}
\end{mathpar} 

Note that $\vec{x}$ (resp. $\vec{P}$) denotes a vector of names
(resp. processes) of length $|\vec{x}|$ (resp. $|\vec{P}|$). We adopt
the following useful abbreviations.

\begin{mathpar}
   x?(\vec{y}).P := x.(\vec{y})P \and  x\clift{\vec{P}} := x.\clift{\vec{P}}
   \and x!(y) := \lift{x}{\dropn{y}}
   \and \Pi_{i=0}^{n-1}P_i := P_0 | \ldots | P_{n-1}
\end{mathpar}

\subsubsection{Structural congruence}

\paragraph{Free and bound names and alpha-equivalence.} At the
core of structural equivalence is alpha-equivalence which identifies
process that are the same up to a change of variable. Formally, we
recognize the distinction between free and bound names. The free names
of a process, $\freenames{P}$, may be calculated recursively as
follows:

\begin{mathpar}
\freenames{\pzero} := \emptyset
  \and \\
  \freenames{x?(y).P} := \{ x \} \cup (\freenames{P} \setminus \{ y \})
  \and 
  \freenames{x!\langle P \rangle} := \{ x \} \cup \{ P \} 
  \and \\
  \freenames{P|Q} := \freenames{P} \cup \freenames{Q}
  \and \\
  \freenames{@{x}} := \{ x \}
\end{mathpar}

$\pi$
$\quotep{\pi}$

$\freenames{-} : \pi \to \mathcal{P}(\quotep{\pi})$

\begin{eqnarray*}
  \freenames{\pzero} & := & \emptyset \\
  \freenames{x?(y).P} & := & \{ x \} \cup (\freenames{P} \setminus \{ y \}) \\
  \freenames{x!\langle P \rangle} & := & \{ x \} \cup \{ P \} \\
  \freenames{P|Q} & := & \freenames{P} \cup \freenames{Q} \\
  \freenames{\dropn{x}} & := & \{ x \}
\end{eqnarray*}

The bound names of a process, $\boundnames{P}$, are those names occurring in $P$
that are not free. For example, in $x?(y).0$, the name $x$ is free, while $y$ is bound.

\begin{mathpar}
  \inferrule* [lab=monoidal-laws] {} { P|Q \equiv Q|P \and P|0 \equiv P \and P|(Q|R) \equiv (P|Q)|R }
\end{mathpar}

\begin{mathpar}
  \inferrule* [lab=alpha-equivalence] {} { (x)P \equiv (y)P\{y/x\} \and y \not\in \freenames{P} }
\end{mathpar}

\begin{definition}
Then two processes, $P,Q$, are alpha-equivalent if $P = Q\{\vec{y}/\vec{x}\}$ for
some $\vec{x} \in \boundnames{Q},\vec{y} \in \boundnames{P}$, where $Q\{\vec{y}/\vec{x}\}$
denotes the capture-avoiding substitution of $\vec{y}$ for $\vec{x}$ in $Q$.
\end{definition}

\begin{definition}
  The {\em structural congruence} \cite{SangiorgiWalker} , $\equiv$,
  between processes is the least congruence containing
  alpha-equivalence, satisfying the abelian monoid laws
  (associativity, commutativity and $\pzero$ as identity) for parallel
  composition $|$ and for summation $+$.
\end{definition}

\subsection{Name equivalence}

We take name equivalence, written $\nameeq$, to be the smallest
equivalence relation generated by the following rules.

\begin{mathpar}
\inferrule*[lab=Quote-drop]
{ }
{ \quotep{@{x}} \nameeq x }

\inferrule*[lab=Struct-equiv]
{ P \scong Q }
{ \quotep{P} \nameeq \quotep{Q} }
\end{mathpar}

The astute reader will have noticed that the mutual recursion of names
and processes imposes a mutual recursion on alpha-equivalence and
structural equivalence via name-equivalence. Fortunately, all of this
works out pleasantly and we may calculate in the natural way, free of
concern. The reader interested in the details is referred to the
appendix \ref{appendix:rho_details}.

\subsection{Substitution}

We use $\Proc$ for the set of processes, $\QProc$ for the set of
names, and $\id{\{}\vec{y} / \vec{x} \id{\}}$ to denote partial maps,
$s : \QProc \rightarrow \QProc$. A map, $s$ lifts, uniquely, to a map
on process terms, $\widehat{s} : \Proc \rightarrow \Proc$ by the
following equations.

\begin{mathpar}
  (0) \psubstp{Q}{P} := 0 \\
  (R \juxtap S) \psubstp{Q}{P}
  :=    
  (R)\psubstp{Q}{P} \juxtap (S) \psubstp{Q}{P} \\
  (x?(y).R) \psubstp{Q}{P}    
  :=    
  (x)\substp{Q}{P} (z)\concat( (R \psubstn{z}{y}) \psubstp{Q}{P} ) \\
  (\lift{x}{R}) \psubstp{Q}{P}  
  :=
  \lift{(x)\substp{Q}{P}}{ R \psubstp{Q}{P} } \\
%   (\dropn{x})  \psubstp{Q}{P}       
%   := 
%   \left\{ 
%     \begin{array}{ccc} 
%       \dropn{\quotep{Q}} & & x \nameeq \quotep{P} \\
%       \dropn{x} & & otherwise \\
%     \end{array}
%   \right. 
  (\dropn{x})  \psubstp{Q}{P}       
  := 
  \left\{ 
    \begin{array}{ccc} 
      Q & & x \nameeq \quotep{P} \\
      \dropn{x} & & otherwise \\
    \end{array}
  \right.
\end{mathpar}
 

where

\begin{eqnarray}
  (x)\id{\{} \lpquote Q \rpquote / \lpquote P \rpquote \id{\}}            = 
  \left\{ 
    \begin{array}{ccc}
      \lpquote Q \rpquote & & x \nameeq \lpquote P \rpquote \\
      x & & otherwise \\
    \end{array}
  \right. \nonumber
\end{eqnarray}

and $z$ is chosen distinct from $\quotep{P}$, $\quotep{Q}$, the free
names in $Q$, and all the names in $R$. Our $\alpha$-equivalence will
be built in the standard way from this substitution.

\begin{remark}\label{rem:no_self_referential_names}
  One consequence of these definitions is that $\forall P. \quotep{P}
  \not\in \freenames{P}$.
\end{remark}

\subsection{ Dynamic quote: an example }

Anticipating something of what's to come, consider applying the
substitution, $\widehat{\id{\{}u / z \id{\}}}$, to the following pair
of processes, $\lift{w}{y!(z)}$ and $w[ \lpquote y!(z) \rpquote ]$.

\begin{eqnarray}
	\lift{w}{y!(z)}\widehat{\id{\{}u / z \id{\}}}
		& = &
		\lift{w}{y!(u)} \nonumber\\
	w[ \lpquote y!(z) \rpquote ] \widehat{ \id{\{}u / z \id{\}} }
		& = &
		w[ \lpquote y!(z) \rpquote ] \nonumber
\end{eqnarray}

Because the body of the process between quotes is impervious to
substitution, we get radically different answers. In fact, by
examining the first process in an input context,
e.g. $x?(z).\lift{w}{y!(z)}$, we see that the process under the lift
operator may be shaped by prefixed inputs binding a name inside it. In
this sense, the lift operator will be seen as a way to dynamically
construct processes before reifying them as names.

Finally equipped with these standard features we can present the
dynamics of the calculus.

\subsubsection{Operational semantics} 

Finally, we introduce the computational dynamics. What marks these
algebras as distinct from other more traditionally studied algebraic
structures, e.g. vector spaces or polynomial rings, is the manner in
which dynamics is captured. In traditional structures, dynamics is typically
expressed through morphisms between such structures, as in linear maps
between vector spaces or morphisms between rings. In algebras
associated with the semantics of computation, the dynamics is
expressed as part of the algebraic structure itself, through a
reduction reduction relation typically denoted by $\red$. Below, we
give a recursive presentation of this relation for the calculus used
in the encoding.

$\red \subseteq \pi \times \pi$
$\red : \pi \to \mathcal{P}(\pi)$

\begin{mathpar}
  \inferrule* [lab=Comm] { \textsf{match}( x_{src}, x_{trgt} ) } { x_{trgt}?(y)P \; | \; x_{src}!\langle {Q} \rangle \red P\{\quotep{Q}/y}\} }
  \and \\
  \inferrule* [lab=Par] {{P} \red {P}'} {{{P} | {Q}} \red {{P}' | {Q}}}
  \and
  \inferrule* [lab=Equiv]{{{P} \scong {P}'} \andalso {{P}' \red {Q}'} \andalso {{Q}' \scong {Q}}}{{P} \red {Q}}
\end{mathpar}

\begin{eqnarray*}
  match_{\equiv} (\quotep{P},\quotep{Q}) & := & P \equiv Q \\
  match_{\dagger}(\quotep{P},\quotep{Q}) & := & \forall R. P|Q \red^{*} R => R \red^{*} 0 \\
  match_{K}(\quotep{P},\quotep{Q}) & := & K \mbox{ for some context } K
\end{eqnarray*}

$u?(x)P | u!\langle Q \rangle \red P\{\quotep{Q}/x\}$

%We write $\wred$ for $\red^*$, and $P\red$ if $\exists Q $ such that $ P \red Q$.
We write $P\red$ if $\exists Q $ such that $ P \red Q$ and $P\not\red$, otherwise.

\section{Replication}

As mentioned before, it is known that replication (and hence
recursion) can be implemented in a higher-order process algebra
\cite{SangiorgiWalker}. As our first example of calculation with the
machinery thus far presented we give the construction explicitly in
the {\rhoc}.

\begin{eqnarray}
	D_{x} & := & \prefix{x}{y}{(\binpar{\outputp{x}{y}}{@{y}})} \nonumber\\
	\bangp_{x}{P} & := & \binpar{{x}!\langle{\binpar{D_{x}}{P}}\rangle}{D_{x}} \nonumber
\end{eqnarray}

\begin{eqnarray}
	\bangp_{x}{P} & & \nonumber\\
	=
	& {x}!\langle{(\prefix{x}{y}{(\outputp{x}{y} | @{y})) | P}}\rangle 
	      | \prefix{x}{y}{(\outputp{x}{y} | @{y})} & \nonumber\\
	\red
	& (\outputp{x}{y} | @{y})\substn{\quotep{(\prefix{x}{y}{(@{y} | \outputp{x}{y})) | P}}}{y} & \nonumber\\
	=
	& \outputp{x}{\quotep{(\prefix{x}{y}{(\outputp{x}{y} | @{y})) | P}}}
	  | {(\prefix{x}{y}{(\outputp{x}{y} | @{y})) | P}} & \nonumber\\
	\red
	& \ldots & \nonumber\\
	\red^*
	& P | P | \ldots & \nonumber
\end{eqnarray}

Of course, this encoding, as an implementation, runs away, unfolding
$\bangp{P}$ eagerly. A lazier and more implementable replication
operator, restricted to input-guarded processes, may be obtained as follows.

\begin{eqnarray}
\bangp{\prefix{u}{v}{P}} 
	:= 
	\binpar{\lift{x}{\prefix{u}{v}{(\binpar{D(x)}{P})}}}{D(x)} \nonumber
\end{eqnarray}

\begin{remark}
  Note that the lazier definition still does not deal with summation
  or mixed summation (i.e. sums over input and output). The reader is
  invited to construct definitions of replication that deal with these
  features. 

  Further, the definitions are parameterized in a name, $x$. Can you,
  gentle reader, make a definition that eliminates this parameter and
  guarantees no accidental interaction between the replication
  machinery and the process being replicated -- i.e. no accidental
  sharing of names used by the process to get its work done and the
  name(s) used by the replication to effect copying. This latter
  revision of the definition of replication is crucial to obtaining
  the expected identity $!!P \sim !P$.
\end{remark}

\begin{remark}\label{rem:paradoxical_combinator}
  The reader familiar with the lambda calculus will have noticed the
  similarity between $D$ and the paradoxical combinator.

  [Ed. note: the existence of this seems to suggest we have to be more
  restrictive on the set of processes and names we admit if we are to
  support no-cloning.]
\end{remark}

\subsubsection{Bisimulation}

The computational dynamics gives rise to another kind of equivalence,
the equivalence of computational behavior. As previously mentioned
this is typically captured \emph{via} some form of bisimulation.

% The notion we use in this paper is weak barbed bisimulation
% \cite{milner91polyadicpi}.

The notion we use in this paper is derived from weak barbed
bisimulation \cite{milner91polyadicpi}. 

\begin{definition}
An \emph{observation relation}, $\downarrow_{\mathcal N}$, over a set
of names, $\mathcal N$, is the smallest relation satisfying the rules
below.

\infrule[Out-barb]{y \in {\mathcal N}, \; x \nameeq y}
		  {\outputp{x}{v} \downarrow_{\mathcal N} x}
\infrule[Par-barb]{\mbox{$P\downarrow_{\mathcal N} x$ or $Q\downarrow_{\mathcal N} x$}}
		  {\binpar{P}{Q} \downarrow_{\mathcal N} x}

We write $P \Downarrow_{\mathcal N} x$ if there is $Q$ such that 
$P \wred Q$ and $Q \downarrow_{\mathcal N} x$.
\end{definition}

\begin{definition}
%\label{def.bbisim}
An  ${\mathcal N}$-\emph{barbed bisimulation} over a set of names, ${\mathcal N}$, is a symmetric binary relation 
${\mathcal S}_{\mathcal N}$ between agents such that $P\rel{S}_{\mathcal N}Q$ implies:
\begin{enumerate}
\item If $P \red P'$ then $Q \wred Q'$ and $P'\rel{S}_{\mathcal N} Q'$.
\item If $P\downarrow_{\mathcal N} x$, then $Q\Downarrow_{\mathcal N} x$.
\end{enumerate}
$P$ is ${\mathcal N}$-barbed bisimilar to $Q$, written
$P \wbbisim_{\mathcal N} Q$, if $P \rel{S}_{\mathcal N} Q$ for some ${\mathcal N}$-barbed bisimulation ${\mathcal S}_{\mathcal N}$.
\end{definition}

$\mathcal{R} \subseteq \pi \times \pi$

$P \mathcal{R} Q => \forall P'. P \red P' \Rightarrow \exists Q'. Q \red Q', P' \mathcal{R} Q'$

$P \vdash x \Rightarrow Q \vdash x$

\begin{mathpar}
  \inferrule*[lab=Out-barb]{x \nameeq y}{{y}!\langle{Q}\rangle \vdash x}
  \and
  \inferrule*[lab=Par-barb]{\mbox{$P\vdash x$ or $Q\vdash x$}}{\binpar{P}{Q} \vdash x}
\end{mathpar}

\subsubsection{Contexts}

One of the principle advantages of computational calculi like the
$\pi$-calculus is a well-defined notion of context,
contextual-equivalence and a correlation between
contextual-equivalence and notions of bisimulation. The notion of
context allows the decomposition of a process into (sub-)process and
its syntactic environment, its context. Thus, a context may be
thought of as a process with a ``hole'' (written $\Box$) in it. The
application of a context $M$ to a process $P$, written $M[P]$, is
tantamount to filling the hole in $M$ with $P$. In this paper we do
not need the full weight of this theory, but do make use of the notion
of context in the proof the main theorem. 

\begin{mathpar}
  \inferrule* [lab=summation] {} {{M_{M},M_{N}} \bc \Box \;|\; x.M_{A} \;|\; M_{M}+M_{N}}
  \and
  \inferrule* [lab=agent] {} {{M_{A}} \bc (\vec{x})M_{P} \;| \; \clift{P_0,\ldots,M_{P},\ldots,P_N}}
  \and \\
  \inferrule* [lab=process] {} {{M_{P}} \bc M_{N} \;| \;P|M_{P} }
\end{mathpar} 

\begin{mathpar}
  \inferrule* [lab=sychronization] {} {M_{N} \bc \Box \;|\; x?M_{F} \;|\; x!M_{C}}
  \and
  \inferrule* [lab=abstraction] {} {{M_{F}} \bc (x)M_{P} }
  \and
  \inferrule* [lab=concretion] {} {{M_{C}} \bc \langle M_{P} \rangle }
  \and \\
  \inferrule* [lab=process] {} {{M_{P}} \bc M_{N} \;| \;P|M_{P} }
\end{mathpar}

\begin{definition}[contextual application] Given a context $M$, and
  process $P$, we define the \emph{contextual application}, $M[P] :=
  M\{P/\Box\}$. That is, the contextual application of M to P is the
  substitution of $P$ for $\Box$ in $M$.
\end{definition}

$\meaningof{-} : L \to \mathcal{P}(\pi)$

\begin{mathpar}
  \inferrule* [lab=collection] {} {\meaningof{true} = \pi, \and \meaningof{~E} = \pi \setminus \meaningof{E}, \and \meaningof{E_{1} \& E_{2}} = \meaningof{E_{1}} \cap \meaningof{E_{2}}}
\end{mathpar}

\begin{mathpar}
  \inferrule* [lab=structure] {} {\meaningof{0} = \{ P \in \pi | P \equiv 0 \}, \and \\ \meaningof{E_1 | E_2} = \{ P \in \pi | P \equiv P_{1} | P_{2}, P_{1} \in \meaningof{E_{1}}, P_{2} \in \meaningof{E_2}\} }
\end{mathpar}

\begin{mathpar}
 \inferrule* [lab=behavior] {} {\meaningof{\langle a?b \rangle E} = \{ P \in \pi | P \equiv Q | u?(y)P', \\ \and \\\\ \and \\ \;\;\; u \in \meaningof{a}, \forall z.P'\{z/y\} \in \meaningof{E\{z/b\}}\}, \and \\ \meaningof{a!E} = \{ P \in \pi | P \equiv Q | x!\langle P' \rangle, x \in \meaningof{a} P' \in \meaningof{E}\} }
\end{mathpar}

\begin{mathpar}
 \inferrule* [lab=nominal] {} {\meaningof{\quotep{E}} = \{ \quotep{P} \in \quotep{\pi} | P \in \meaningof{E} \}, \and \meaningof{\quotep{P}} = \{ \quotep{Q} \in \quotep{\pi} | P \equiv Q \} \and \\ \meaningof{@\quotep{E}} = \{ P \in \pi | P \equiv @x, x \in \meaningof{E} \}}
\end{mathpar}

\begin{eqnarray*}
  \\
  \meaningof{-} : TS \to ST
\end{eqnarray*}

\begin{eqnarray*}
  \\
  L : TS \to ST
\end{eqnarray*}

\begin{eqnarray*}
  \\
  P \models E \iff P \in \meaningof{E}
\end{eqnarray*}

\begin{eqnarray*}
  P \approx_{L} Q \iff \forall E \in L. P \models E \iff Q \models E
\end{eqnarray*}

\begin{eqnarray*}
  P \approx_{K} Q
\end{eqnarray*}

\begin{eqnarray*}
  P \approx Q
\end{eqnarray*}

$\approx_{K} = \approx = \approx_{L}$

\subsubsection{Contextual duality}

Note that contexts extend the quotation operation to a family of
operations from processes to names. Given a context, $M$, we can
define a \emph{nominal context}, $\quotep{M}$ by $\quotep{M}[P] :=
\quotep{M[P]}$. To foreshadow what is to come we observe that these
operations enjoy a duality with processes very much like the duality
between vectors and maps from vectors to scalars.

Further, because the calculus is essentially higher-order, we have a
correspondence between contexts and processes. More specifically,
given a name $x$ and a context $M$ we can construct $M^{*}_{x}$ such
that 

\begin{mathpar}
  M^{*}_{x} | \lift{x}{P} \red M[P]
\end{mathpar}

namely,

\begin{mathpar}
  M^{*}_{x} := x?(u).M[\dropn{u}]
\end{mathpar}

The dependence of $M^{*}_{x}$ on a name makes it an abstraction, 

\begin{mathpar}
  M^{*} := (x)x?(u).M[\dropn{u}]
\end{mathpar}

\subsection{Additional notation}

It will sometimes be convenient to denote the process a name
quotes. We already have the notation $x = \quotep{P}$, but it will be
convenient to introduce an alternate notation, $\procn{x}$, when we
want to emphasize the connection to the use of the name. Note that, by
virtue of name equivalence, $\quotep{\procn{x}} \nameeq x$; so, the
notation is consistent with previous definitions.

Further, because names have structure it is possible to effect
substitutions on the basis of that structure. This means we need to
upgrade our notation for substitutions, which we accomplish by
adapting comprehension notation. Thus,

\begin{mathpar}
  P\{ y / x : x \in S \}
\end{mathpar}

is interpreted to mean the process derived from P by replacing (in a
capture-avoiding manner) each occurrence of $x$ in $S$ by $y$. For example,

\begin{mathpar}
  P\{ \quotep{\procn{x}|\procn{x}} / x : x \in \freenames{P} \}
\end{mathpar}

will replace each (occurrence) of a free name $x$ in $P$ by
$\quotep{\procn{x}|\procn{x}}$.

Also, we will avail ourselves of the notation $x^{L}$ and $x^{R}$ to
denote injections of a name into disjoint copies of the name
space. There are numerous ways to accomplish this. One example can be
found in \cite{MeredithR05}. This notation overloads to vectors of
names: $\vec{x}^{\pi} := (x_{i}^{\pi} \; : \; 0 \leq i < |\vec{x}| )$ where $\pi \in \{L,R\}$.

We also use $P^{\Box} := P|\Box$.

In \cite{MeredithR05} an interpretation of the new operator is
given. It turns out that there are several possible interpretations
all enjoying the requisite algebraic properties of the operator (see
\cite{milner91polyadicpi}). We will therefore make liberal use of
$(\nu\; \vec{x})P$.

% subsection the_syntax_and_semantics_of_the_notation_system (end)   

\input{qm2pi.qmops} 

\input{qm2pi.sterngerlach} 

\input{qm2pi.metric} 

% section concurrent_process_calculi (end)

%\input{qm2pi.proofsketch}

% section proof sketch (end)

%\input{qm2pi.slviaknots} 

% section spatial logic via knots (end)

\input{qm2pi.conclusion}

% section conclusion (end)

%\input{qm2pi.dtcodes} 

% section wiring algorithm (end)

\input{qm2pi.ack} 

% section acknowledgments (end)

\newpage


\bibliographystyle{plain}   
\bibliography{../../biblios/main.bib}

\input{qm2pi.rhodetails}

\end{document}

 

%\documentclass[12pt]{llncs}
%\documentclass{jktr}

\usepackage[pdftex]{hyperref}                   
\usepackage {listings}
\usepackage {mathpartir}
\usepackage{bcprules}
%\usepackage{listings}
                       
\usepackage{graphicx} 
%\usepackage[margins=2.5cm,nohead,nofoot]{geometry}
%\usepackage{geometry}
\usepackage{amsfonts}
\usepackage{amstext}
\usepackage{latexsym}
\usepackage{amssymb}
\usepackage{color}


%\include{myPreamble}
\include{qm2pi.local} 

%\ifpdf
%\usepackage[pdftex]{graphicx}
%\else
%\usepackage{graphicx}
%\fi

 % \ifpdf
%  \usepackage{pdfsync}
%  \if


%\title{Brief Article}
%\author{David F. Snyder}
%\author{L.G. Meredith}

%\address{Dept. of Math., Texas State University--San Marcos, San Marcos, TX 78666}
       
\pagestyle{empty}


\begin{document}

\lstset{language=[Objective]Caml,frame=shadowbox}

\input{qm2pi.front}

% section front matter (end)

\input{qm2pi.intro} 
 
% section introduction (end)

% \input{qm2pi.knotations} 

% section notation (end)

\input{qm2pi.process.calculi} 

% section concurrent_process_calculi_and_spatial_logics_ (end)
    
%\input{qm2pi.knots2pi} 

%\input{qm2pi.trefoil} 

%\input{qm2pi.mainthm} 

% subsection basic_interpretation (end)

%\input{qm2pi.rho.presentation} 
\subsection{The syntax and semantics of the notation system}\label{sub:the_syntax_and_semantics_of_the_notation_system} % (fold)

We now summarize a technical presentation of the calculus that
embodies our theory of dynamics. The typical presentation of such a
calculus follows the style of giving generators and relations on
them. The grammar, below, describing term constructors, freely
generates the set of processes, $\Proc$. This set is then quotiented
by a relation known as structural congruence and it is over this set
that the notion of dynamics is expressed. This presentation is
essentially that of \cite{MeredithR05} with the addition of
polyadicity and summation. For readability we have relegated some of
the technical subtleties to an appendix.

\subsubsection{Process grammar}\label{subsub:process_grammar}

\begin{mathpar}
  \inferrule* [lab=synchronization] {} {{M} \bc \pzero \;|\; x?F \;|\; x!C }
  \and
  \inferrule* [lab=abstraction] {} {{F} \bc (x)P}
  \and
  \inferrule* [lab=concretion] {} {{C} \bc \langle Q \rangle}
  \and
  \inferrule* [lab=process] {} {{P,Q} \bc M \;| \;P|Q \;|\; @{x}}
  \and
  \inferrule* [lab=name] {} {{x} \bc \quotep{P}}
\end{mathpar} 

Note that $\vec{x}$ (resp. $\vec{P}$) denotes a vector of names
(resp. processes) of length $|\vec{x}|$ (resp. $|\vec{P}|$). We adopt
the following useful abbreviations.

\begin{mathpar}
   x?(\vec{y}).P := x.(\vec{y})P \and  x\clift{\vec{P}} := x.\clift{\vec{P}}
   \and x!(y) := \lift{x}{\dropn{y}}
   \and \Pi_{i=0}^{n-1}P_i := P_0 | \ldots | P_{n-1}
\end{mathpar}

\subsubsection{Structural congruence}

\paragraph{Free and bound names and alpha-equivalence.} At the
core of structural equivalence is alpha-equivalence which identifies
process that are the same up to a change of variable. Formally, we
recognize the distinction between free and bound names. The free names
of a process, $\freenames{P}$, may be calculated recursively as
follows:

\begin{mathpar}
\freenames{\pzero} := \emptyset
  \and \\
  \freenames{x?(y).P} := \{ x \} \cup (\freenames{P} \setminus \{ y \})
  \and 
  \freenames{x!\langle P \rangle} := \{ x \} \cup \{ P \} 
  \and \\
  \freenames{P|Q} := \freenames{P} \cup \freenames{Q}
  \and \\
  \freenames{@{x}} := \{ x \}
\end{mathpar}

$\pi$
$\quotep{\pi}$

$\freenames{-} : \pi \to \mathcal{P}(\quotep{\pi})$

\begin{eqnarray*}
  \freenames{\pzero} & := & \emptyset \\
  \freenames{x?(y).P} & := & \{ x \} \cup (\freenames{P} \setminus \{ y \}) \\
  \freenames{x!\langle P \rangle} & := & \{ x \} \cup \{ P \} \\
  \freenames{P|Q} & := & \freenames{P} \cup \freenames{Q} \\
  \freenames{\dropn{x}} & := & \{ x \}
\end{eqnarray*}

The bound names of a process, $\boundnames{P}$, are those names occurring in $P$
that are not free. For example, in $x?(y).0$, the name $x$ is free, while $y$ is bound.

\begin{mathpar}
  \inferrule* [lab=monoidal-laws] {} { P|Q \equiv Q|P \and P|0 \equiv P \and P|(Q|R) \equiv (P|Q)|R }
\end{mathpar}

\begin{mathpar}
  \inferrule* [lab=alpha-equivalence] {} { (x)P \equiv (y)P\{y/x\} \and y \not\in \freenames{P} }
\end{mathpar}

\begin{definition}
Then two processes, $P,Q$, are alpha-equivalent if $P = Q\{\vec{y}/\vec{x}\}$ for
some $\vec{x} \in \boundnames{Q},\vec{y} \in \boundnames{P}$, where $Q\{\vec{y}/\vec{x}\}$
denotes the capture-avoiding substitution of $\vec{y}$ for $\vec{x}$ in $Q$.
\end{definition}

\begin{definition}
  The {\em structural congruence} \cite{SangiorgiWalker} , $\equiv$,
  between processes is the least congruence containing
  alpha-equivalence, satisfying the abelian monoid laws
  (associativity, commutativity and $\pzero$ as identity) for parallel
  composition $|$ and for summation $+$.
\end{definition}

\subsection{Name equivalence}

We take name equivalence, written $\nameeq$, to be the smallest
equivalence relation generated by the following rules.

\begin{mathpar}
\inferrule*[lab=Quote-drop]
{ }
{ \quotep{@{x}} \nameeq x }

\inferrule*[lab=Struct-equiv]
{ P \scong Q }
{ \quotep{P} \nameeq \quotep{Q} }
\end{mathpar}

The astute reader will have noticed that the mutual recursion of names
and processes imposes a mutual recursion on alpha-equivalence and
structural equivalence via name-equivalence. Fortunately, all of this
works out pleasantly and we may calculate in the natural way, free of
concern. The reader interested in the details is referred to the
appendix \ref{appendix:rho_details}.

\subsection{Substitution}

We use $\Proc$ for the set of processes, $\QProc$ for the set of
names, and $\id{\{}\vec{y} / \vec{x} \id{\}}$ to denote partial maps,
$s : \QProc \rightarrow \QProc$. A map, $s$ lifts, uniquely, to a map
on process terms, $\widehat{s} : \Proc \rightarrow \Proc$ by the
following equations.

\begin{mathpar}
  (0) \psubstp{Q}{P} := 0 \\
  (R \juxtap S) \psubstp{Q}{P}
  :=    
  (R)\psubstp{Q}{P} \juxtap (S) \psubstp{Q}{P} \\
  (x?(y).R) \psubstp{Q}{P}    
  :=    
  (x)\substp{Q}{P} (z)\concat( (R \psubstn{z}{y}) \psubstp{Q}{P} ) \\
  (\lift{x}{R}) \psubstp{Q}{P}  
  :=
  \lift{(x)\substp{Q}{P}}{ R \psubstp{Q}{P} } \\
%   (\dropn{x})  \psubstp{Q}{P}       
%   := 
%   \left\{ 
%     \begin{array}{ccc} 
%       \dropn{\quotep{Q}} & & x \nameeq \quotep{P} \\
%       \dropn{x} & & otherwise \\
%     \end{array}
%   \right. 
  (\dropn{x})  \psubstp{Q}{P}       
  := 
  \left\{ 
    \begin{array}{ccc} 
      Q & & x \nameeq \quotep{P} \\
      \dropn{x} & & otherwise \\
    \end{array}
  \right.
\end{mathpar}
 

where

\begin{eqnarray}
  (x)\id{\{} \lpquote Q \rpquote / \lpquote P \rpquote \id{\}}            = 
  \left\{ 
    \begin{array}{ccc}
      \lpquote Q \rpquote & & x \nameeq \lpquote P \rpquote \\
      x & & otherwise \\
    \end{array}
  \right. \nonumber
\end{eqnarray}

and $z$ is chosen distinct from $\quotep{P}$, $\quotep{Q}$, the free
names in $Q$, and all the names in $R$. Our $\alpha$-equivalence will
be built in the standard way from this substitution.

\begin{remark}\label{rem:no_self_referential_names}
  One consequence of these definitions is that $\forall P. \quotep{P}
  \not\in \freenames{P}$.
\end{remark}

\subsection{ Dynamic quote: an example }

Anticipating something of what's to come, consider applying the
substitution, $\widehat{\id{\{}u / z \id{\}}}$, to the following pair
of processes, $\lift{w}{y!(z)}$ and $w[ \lpquote y!(z) \rpquote ]$.

\begin{eqnarray}
	\lift{w}{y!(z)}\widehat{\id{\{}u / z \id{\}}}
		& = &
		\lift{w}{y!(u)} \nonumber\\
	w[ \lpquote y!(z) \rpquote ] \widehat{ \id{\{}u / z \id{\}} }
		& = &
		w[ \lpquote y!(z) \rpquote ] \nonumber
\end{eqnarray}

Because the body of the process between quotes is impervious to
substitution, we get radically different answers. In fact, by
examining the first process in an input context,
e.g. $x?(z).\lift{w}{y!(z)}$, we see that the process under the lift
operator may be shaped by prefixed inputs binding a name inside it. In
this sense, the lift operator will be seen as a way to dynamically
construct processes before reifying them as names.

Finally equipped with these standard features we can present the
dynamics of the calculus.

\subsubsection{Operational semantics} 

Finally, we introduce the computational dynamics. What marks these
algebras as distinct from other more traditionally studied algebraic
structures, e.g. vector spaces or polynomial rings, is the manner in
which dynamics is captured. In traditional structures, dynamics is typically
expressed through morphisms between such structures, as in linear maps
between vector spaces or morphisms between rings. In algebras
associated with the semantics of computation, the dynamics is
expressed as part of the algebraic structure itself, through a
reduction reduction relation typically denoted by $\red$. Below, we
give a recursive presentation of this relation for the calculus used
in the encoding.

$\red \subseteq \pi \times \pi$
$\red : \pi \to \mathcal{P}(\pi)$

\begin{mathpar}
  \inferrule* [lab=Comm] { \textsf{match}( x_{src}, x_{trgt} ) } { x_{trgt}?(y)P \; | \; x_{src}!\langle {Q} \rangle \red P\{\quotep{Q}/y}\} }
  \and \\
  \inferrule* [lab=Par] {{P} \red {P}'} {{{P} | {Q}} \red {{P}' | {Q}}}
  \and
  \inferrule* [lab=Equiv]{{{P} \scong {P}'} \andalso {{P}' \red {Q}'} \andalso {{Q}' \scong {Q}}}{{P} \red {Q}}
\end{mathpar}

\begin{eqnarray*}
  match_{\equiv} (\quotep{P},\quotep{Q}) & := & P \equiv Q \\
  match_{\dagger}(\quotep{P},\quotep{Q}) & := & \forall R. P|Q \red^{*} R => R \red^{*} 0 \\
  match_{K}(\quotep{P},\quotep{Q}) & := & K \mbox{ for some context } K
\end{eqnarray*}

$u?(x)P | u!\langle Q \rangle \red P\{\quotep{Q}/x\}$

%We write $\wred$ for $\red^*$, and $P\red$ if $\exists Q $ such that $ P \red Q$.
We write $P\red$ if $\exists Q $ such that $ P \red Q$ and $P\not\red$, otherwise.

\section{Replication}

As mentioned before, it is known that replication (and hence
recursion) can be implemented in a higher-order process algebra
\cite{SangiorgiWalker}. As our first example of calculation with the
machinery thus far presented we give the construction explicitly in
the {\rhoc}.

\begin{eqnarray}
	D_{x} & := & \prefix{x}{y}{(\binpar{\outputp{x}{y}}{@{y}})} \nonumber\\
	\bangp_{x}{P} & := & \binpar{{x}!\langle{\binpar{D_{x}}{P}}\rangle}{D_{x}} \nonumber
\end{eqnarray}

\begin{eqnarray}
	\bangp_{x}{P} & & \nonumber\\
	=
	& {x}!\langle{(\prefix{x}{y}{(\outputp{x}{y} | @{y})) | P}}\rangle 
	      | \prefix{x}{y}{(\outputp{x}{y} | @{y})} & \nonumber\\
	\red
	& (\outputp{x}{y} | @{y})\substn{\quotep{(\prefix{x}{y}{(@{y} | \outputp{x}{y})) | P}}}{y} & \nonumber\\
	=
	& \outputp{x}{\quotep{(\prefix{x}{y}{(\outputp{x}{y} | @{y})) | P}}}
	  | {(\prefix{x}{y}{(\outputp{x}{y} | @{y})) | P}} & \nonumber\\
	\red
	& \ldots & \nonumber\\
	\red^*
	& P | P | \ldots & \nonumber
\end{eqnarray}

Of course, this encoding, as an implementation, runs away, unfolding
$\bangp{P}$ eagerly. A lazier and more implementable replication
operator, restricted to input-guarded processes, may be obtained as follows.

\begin{eqnarray}
\bangp{\prefix{u}{v}{P}} 
	:= 
	\binpar{\lift{x}{\prefix{u}{v}{(\binpar{D(x)}{P})}}}{D(x)} \nonumber
\end{eqnarray}

\begin{remark}
  Note that the lazier definition still does not deal with summation
  or mixed summation (i.e. sums over input and output). The reader is
  invited to construct definitions of replication that deal with these
  features. 

  Further, the definitions are parameterized in a name, $x$. Can you,
  gentle reader, make a definition that eliminates this parameter and
  guarantees no accidental interaction between the replication
  machinery and the process being replicated -- i.e. no accidental
  sharing of names used by the process to get its work done and the
  name(s) used by the replication to effect copying. This latter
  revision of the definition of replication is crucial to obtaining
  the expected identity $!!P \sim !P$.
\end{remark}

\begin{remark}\label{rem:paradoxical_combinator}
  The reader familiar with the lambda calculus will have noticed the
  similarity between $D$ and the paradoxical combinator.

  [Ed. note: the existence of this seems to suggest we have to be more
  restrictive on the set of processes and names we admit if we are to
  support no-cloning.]
\end{remark}

\subsubsection{Bisimulation}

The computational dynamics gives rise to another kind of equivalence,
the equivalence of computational behavior. As previously mentioned
this is typically captured \emph{via} some form of bisimulation.

% The notion we use in this paper is weak barbed bisimulation
% \cite{milner91polyadicpi}.

The notion we use in this paper is derived from weak barbed
bisimulation \cite{milner91polyadicpi}. 

\begin{definition}
An \emph{observation relation}, $\downarrow_{\mathcal N}$, over a set
of names, $\mathcal N$, is the smallest relation satisfying the rules
below.

\infrule[Out-barb]{y \in {\mathcal N}, \; x \nameeq y}
		  {\outputp{x}{v} \downarrow_{\mathcal N} x}
\infrule[Par-barb]{\mbox{$P\downarrow_{\mathcal N} x$ or $Q\downarrow_{\mathcal N} x$}}
		  {\binpar{P}{Q} \downarrow_{\mathcal N} x}

We write $P \Downarrow_{\mathcal N} x$ if there is $Q$ such that 
$P \wred Q$ and $Q \downarrow_{\mathcal N} x$.
\end{definition}

\begin{definition}
%\label{def.bbisim}
An  ${\mathcal N}$-\emph{barbed bisimulation} over a set of names, ${\mathcal N}$, is a symmetric binary relation 
${\mathcal S}_{\mathcal N}$ between agents such that $P\rel{S}_{\mathcal N}Q$ implies:
\begin{enumerate}
\item If $P \red P'$ then $Q \wred Q'$ and $P'\rel{S}_{\mathcal N} Q'$.
\item If $P\downarrow_{\mathcal N} x$, then $Q\Downarrow_{\mathcal N} x$.
\end{enumerate}
$P$ is ${\mathcal N}$-barbed bisimilar to $Q$, written
$P \wbbisim_{\mathcal N} Q$, if $P \rel{S}_{\mathcal N} Q$ for some ${\mathcal N}$-barbed bisimulation ${\mathcal S}_{\mathcal N}$.
\end{definition}

$\mathcal{R} \subseteq \pi \times \pi$

$P \mathcal{R} Q => \forall P'. P \red P' \Rightarrow \exists Q'. Q \red Q', P' \mathcal{R} Q'$

$P \vdash x \Rightarrow Q \vdash x$

\begin{mathpar}
  \inferrule*[lab=Out-barb]{x \nameeq y}{{y}!\langle{Q}\rangle \vdash x}
  \and
  \inferrule*[lab=Par-barb]{\mbox{$P\vdash x$ or $Q\vdash x$}}{\binpar{P}{Q} \vdash x}
\end{mathpar}

\subsubsection{Contexts}

One of the principle advantages of computational calculi like the
$\pi$-calculus is a well-defined notion of context,
contextual-equivalence and a correlation between
contextual-equivalence and notions of bisimulation. The notion of
context allows the decomposition of a process into (sub-)process and
its syntactic environment, its context. Thus, a context may be
thought of as a process with a ``hole'' (written $\Box$) in it. The
application of a context $M$ to a process $P$, written $M[P]$, is
tantamount to filling the hole in $M$ with $P$. In this paper we do
not need the full weight of this theory, but do make use of the notion
of context in the proof the main theorem. 

\begin{mathpar}
  \inferrule* [lab=summation] {} {{M_{M},M_{N}} \bc \Box \;|\; x.M_{A} \;|\; M_{M}+M_{N}}
  \and
  \inferrule* [lab=agent] {} {{M_{A}} \bc (\vec{x})M_{P} \;| \; \clift{P_0,\ldots,M_{P},\ldots,P_N}}
  \and \\
  \inferrule* [lab=process] {} {{M_{P}} \bc M_{N} \;| \;P|M_{P} }
\end{mathpar} 

\begin{mathpar}
  \inferrule* [lab=sychronization] {} {M_{N} \bc \Box \;|\; x?M_{F} \;|\; x!M_{C}}
  \and
  \inferrule* [lab=abstraction] {} {{M_{F}} \bc (x)M_{P} }
  \and
  \inferrule* [lab=concretion] {} {{M_{C}} \bc \langle M_{P} \rangle }
  \and \\
  \inferrule* [lab=process] {} {{M_{P}} \bc M_{N} \;| \;P|M_{P} }
\end{mathpar}

\begin{definition}[contextual application] Given a context $M$, and
  process $P$, we define the \emph{contextual application}, $M[P] :=
  M\{P/\Box\}$. That is, the contextual application of M to P is the
  substitution of $P$ for $\Box$ in $M$.
\end{definition}

$\meaningof{-} : L \to \mathcal{P}(\pi)$

\begin{mathpar}
  \inferrule* [lab=collection] {} {\meaningof{true} = \pi, \and \meaningof{~E} = \pi \setminus \meaningof{E}, \and \meaningof{E_{1} \& E_{2}} = \meaningof{E_{1}} \cap \meaningof{E_{2}}}
\end{mathpar}

\begin{mathpar}
  \inferrule* [lab=structure] {} {\meaningof{0} = \{ P \in \pi | P \equiv 0 \}, \and \\ \meaningof{E_1 | E_2} = \{ P \in \pi | P \equiv P_{1} | P_{2}, P_{1} \in \meaningof{E_{1}}, P_{2} \in \meaningof{E_2}\} }
\end{mathpar}

\begin{mathpar}
 \inferrule* [lab=behavior] {} {\meaningof{\langle a?b \rangle E} = \{ P \in \pi | P \equiv Q | u?(y)P', \\ \and \\\\ \and \\ \;\;\; u \in \meaningof{a}, \forall z.P'\{z/y\} \in \meaningof{E\{z/b\}}\}, \and \\ \meaningof{a!E} = \{ P \in \pi | P \equiv Q | x!\langle P' \rangle, x \in \meaningof{a} P' \in \meaningof{E}\} }
\end{mathpar}

\begin{mathpar}
 \inferrule* [lab=nominal] {} {\meaningof{\quotep{E}} = \{ \quotep{P} \in \quotep{\pi} | P \in \meaningof{E} \}, \and \meaningof{\quotep{P}} = \{ \quotep{Q} \in \quotep{\pi} | P \equiv Q \} \and \\ \meaningof{@\quotep{E}} = \{ P \in \pi | P \equiv @x, x \in \meaningof{E} \}}
\end{mathpar}

\begin{eqnarray*}
  \\
  \meaningof{-} : TS \to ST
\end{eqnarray*}

\begin{eqnarray*}
  \\
  L : TS \to ST
\end{eqnarray*}

\begin{eqnarray*}
  \\
  P \models E \iff P \in \meaningof{E}
\end{eqnarray*}

\begin{eqnarray*}
  P \approx_{L} Q \iff \forall E \in L. P \models E \iff Q \models E
\end{eqnarray*}

\begin{eqnarray*}
  P \approx_{K} Q
\end{eqnarray*}

\begin{eqnarray*}
  P \approx Q
\end{eqnarray*}

$\approx_{K} = \approx = \approx_{L}$

\subsubsection{Contextual duality}

Note that contexts extend the quotation operation to a family of
operations from processes to names. Given a context, $M$, we can
define a \emph{nominal context}, $\quotep{M}$ by $\quotep{M}[P] :=
\quotep{M[P]}$. To foreshadow what is to come we observe that these
operations enjoy a duality with processes very much like the duality
between vectors and maps from vectors to scalars.

Further, because the calculus is essentially higher-order, we have a
correspondence between contexts and processes. More specifically,
given a name $x$ and a context $M$ we can construct $M^{*}_{x}$ such
that 

\begin{mathpar}
  M^{*}_{x} | \lift{x}{P} \red M[P]
\end{mathpar}

namely,

\begin{mathpar}
  M^{*}_{x} := x?(u).M[\dropn{u}]
\end{mathpar}

The dependence of $M^{*}_{x}$ on a name makes it an abstraction, 

\begin{mathpar}
  M^{*} := (x)x?(u).M[\dropn{u}]
\end{mathpar}

\subsection{Additional notation}

It will sometimes be convenient to denote the process a name
quotes. We already have the notation $x = \quotep{P}$, but it will be
convenient to introduce an alternate notation, $\procn{x}$, when we
want to emphasize the connection to the use of the name. Note that, by
virtue of name equivalence, $\quotep{\procn{x}} \nameeq x$; so, the
notation is consistent with previous definitions.

Further, because names have structure it is possible to effect
substitutions on the basis of that structure. This means we need to
upgrade our notation for substitutions, which we accomplish by
adapting comprehension notation. Thus,

\begin{mathpar}
  P\{ y / x : x \in S \}
\end{mathpar}

is interpreted to mean the process derived from P by replacing (in a
capture-avoiding manner) each occurrence of $x$ in $S$ by $y$. For example,

\begin{mathpar}
  P\{ \quotep{\procn{x}|\procn{x}} / x : x \in \freenames{P} \}
\end{mathpar}

will replace each (occurrence) of a free name $x$ in $P$ by
$\quotep{\procn{x}|\procn{x}}$.

Also, we will avail ourselves of the notation $x^{L}$ and $x^{R}$ to
denote injections of a name into disjoint copies of the name
space. There are numerous ways to accomplish this. One example can be
found in \cite{MeredithR05}. This notation overloads to vectors of
names: $\vec{x}^{\pi} := (x_{i}^{\pi} \; : \; 0 \leq i < |\vec{x}| )$ where $\pi \in \{L,R\}$.

We also use $P^{\Box} := P|\Box$.

In \cite{MeredithR05} an interpretation of the new operator is
given. It turns out that there are several possible interpretations
all enjoying the requisite algebraic properties of the operator (see
\cite{milner91polyadicpi}). We will therefore make liberal use of
$(\nu\; \vec{x})P$.

% subsection the_syntax_and_semantics_of_the_notation_system (end)   

\input{qm2pi.qmops} 

\input{qm2pi.sterngerlach} 

\input{qm2pi.metric} 

% section concurrent_process_calculi (end)

%\input{qm2pi.proofsketch}

% section proof sketch (end)

%\input{qm2pi.slviaknots} 

% section spatial logic via knots (end)

\input{qm2pi.conclusion}

% section conclusion (end)

%\input{qm2pi.dtcodes} 

% section wiring algorithm (end)

\input{qm2pi.ack} 

% section acknowledgments (end)

\newpage


\bibliographystyle{plain}   
\bibliography{../../biblios/main.bib}

\input{qm2pi.rhodetails}

\end{document}

 

%\documentclass[12pt]{llncs}
%\documentclass{jktr}

\usepackage[pdftex]{hyperref}                   
\usepackage {listings}
\usepackage {mathpartir}
\usepackage{bcprules}
%\usepackage{listings}
                       
\usepackage{graphicx} 
%\usepackage[margins=2.5cm,nohead,nofoot]{geometry}
%\usepackage{geometry}
\usepackage{amsfonts}
\usepackage{amstext}
\usepackage{latexsym}
\usepackage{amssymb}
\usepackage{color}


%\include{myPreamble}
\include{qm2pi.local} 

%\ifpdf
%\usepackage[pdftex]{graphicx}
%\else
%\usepackage{graphicx}
%\fi

 % \ifpdf
%  \usepackage{pdfsync}
%  \if


%\title{Brief Article}
%\author{David F. Snyder}
%\author{L.G. Meredith}

%\address{Dept. of Math., Texas State University--San Marcos, San Marcos, TX 78666}
       
\pagestyle{empty}


\begin{document}

\lstset{language=[Objective]Caml,frame=shadowbox}

\input{qm2pi.front}

% section front matter (end)

\input{qm2pi.intro} 
 
% section introduction (end)

% \input{qm2pi.knotations} 

% section notation (end)

\input{qm2pi.process.calculi} 

% section concurrent_process_calculi_and_spatial_logics_ (end)
    
%\input{qm2pi.knots2pi} 

%\input{qm2pi.trefoil} 

%\input{qm2pi.mainthm} 

% subsection basic_interpretation (end)

%\input{qm2pi.rho.presentation} 
\subsection{The syntax and semantics of the notation system}\label{sub:the_syntax_and_semantics_of_the_notation_system} % (fold)

We now summarize a technical presentation of the calculus that
embodies our theory of dynamics. The typical presentation of such a
calculus follows the style of giving generators and relations on
them. The grammar, below, describing term constructors, freely
generates the set of processes, $\Proc$. This set is then quotiented
by a relation known as structural congruence and it is over this set
that the notion of dynamics is expressed. This presentation is
essentially that of \cite{MeredithR05} with the addition of
polyadicity and summation. For readability we have relegated some of
the technical subtleties to an appendix.

\subsubsection{Process grammar}\label{subsub:process_grammar}

\begin{mathpar}
  \inferrule* [lab=synchronization] {} {{M} \bc \pzero \;|\; x?F \;|\; x!C }
  \and
  \inferrule* [lab=abstraction] {} {{F} \bc (x)P}
  \and
  \inferrule* [lab=concretion] {} {{C} \bc \langle Q \rangle}
  \and
  \inferrule* [lab=process] {} {{P,Q} \bc M \;| \;P|Q \;|\; @{x}}
  \and
  \inferrule* [lab=name] {} {{x} \bc \quotep{P}}
\end{mathpar} 

Note that $\vec{x}$ (resp. $\vec{P}$) denotes a vector of names
(resp. processes) of length $|\vec{x}|$ (resp. $|\vec{P}|$). We adopt
the following useful abbreviations.

\begin{mathpar}
   x?(\vec{y}).P := x.(\vec{y})P \and  x\clift{\vec{P}} := x.\clift{\vec{P}}
   \and x!(y) := \lift{x}{\dropn{y}}
   \and \Pi_{i=0}^{n-1}P_i := P_0 | \ldots | P_{n-1}
\end{mathpar}

\subsubsection{Structural congruence}

\paragraph{Free and bound names and alpha-equivalence.} At the
core of structural equivalence is alpha-equivalence which identifies
process that are the same up to a change of variable. Formally, we
recognize the distinction between free and bound names. The free names
of a process, $\freenames{P}$, may be calculated recursively as
follows:

\begin{mathpar}
\freenames{\pzero} := \emptyset
  \and \\
  \freenames{x?(y).P} := \{ x \} \cup (\freenames{P} \setminus \{ y \})
  \and 
  \freenames{x!\langle P \rangle} := \{ x \} \cup \{ P \} 
  \and \\
  \freenames{P|Q} := \freenames{P} \cup \freenames{Q}
  \and \\
  \freenames{@{x}} := \{ x \}
\end{mathpar}

$\pi$
$\quotep{\pi}$

$\freenames{-} : \pi \to \mathcal{P}(\quotep{\pi})$

\begin{eqnarray*}
  \freenames{\pzero} & := & \emptyset \\
  \freenames{x?(y).P} & := & \{ x \} \cup (\freenames{P} \setminus \{ y \}) \\
  \freenames{x!\langle P \rangle} & := & \{ x \} \cup \{ P \} \\
  \freenames{P|Q} & := & \freenames{P} \cup \freenames{Q} \\
  \freenames{\dropn{x}} & := & \{ x \}
\end{eqnarray*}

The bound names of a process, $\boundnames{P}$, are those names occurring in $P$
that are not free. For example, in $x?(y).0$, the name $x$ is free, while $y$ is bound.

\begin{mathpar}
  \inferrule* [lab=monoidal-laws] {} { P|Q \equiv Q|P \and P|0 \equiv P \and P|(Q|R) \equiv (P|Q)|R }
\end{mathpar}

\begin{mathpar}
  \inferrule* [lab=alpha-equivalence] {} { (x)P \equiv (y)P\{y/x\} \and y \not\in \freenames{P} }
\end{mathpar}

\begin{definition}
Then two processes, $P,Q$, are alpha-equivalent if $P = Q\{\vec{y}/\vec{x}\}$ for
some $\vec{x} \in \boundnames{Q},\vec{y} \in \boundnames{P}$, where $Q\{\vec{y}/\vec{x}\}$
denotes the capture-avoiding substitution of $\vec{y}$ for $\vec{x}$ in $Q$.
\end{definition}

\begin{definition}
  The {\em structural congruence} \cite{SangiorgiWalker} , $\equiv$,
  between processes is the least congruence containing
  alpha-equivalence, satisfying the abelian monoid laws
  (associativity, commutativity and $\pzero$ as identity) for parallel
  composition $|$ and for summation $+$.
\end{definition}

\subsection{Name equivalence}

We take name equivalence, written $\nameeq$, to be the smallest
equivalence relation generated by the following rules.

\begin{mathpar}
\inferrule*[lab=Quote-drop]
{ }
{ \quotep{@{x}} \nameeq x }

\inferrule*[lab=Struct-equiv]
{ P \scong Q }
{ \quotep{P} \nameeq \quotep{Q} }
\end{mathpar}

The astute reader will have noticed that the mutual recursion of names
and processes imposes a mutual recursion on alpha-equivalence and
structural equivalence via name-equivalence. Fortunately, all of this
works out pleasantly and we may calculate in the natural way, free of
concern. The reader interested in the details is referred to the
appendix \ref{appendix:rho_details}.

\subsection{Substitution}

We use $\Proc$ for the set of processes, $\QProc$ for the set of
names, and $\id{\{}\vec{y} / \vec{x} \id{\}}$ to denote partial maps,
$s : \QProc \rightarrow \QProc$. A map, $s$ lifts, uniquely, to a map
on process terms, $\widehat{s} : \Proc \rightarrow \Proc$ by the
following equations.

\begin{mathpar}
  (0) \psubstp{Q}{P} := 0 \\
  (R \juxtap S) \psubstp{Q}{P}
  :=    
  (R)\psubstp{Q}{P} \juxtap (S) \psubstp{Q}{P} \\
  (x?(y).R) \psubstp{Q}{P}    
  :=    
  (x)\substp{Q}{P} (z)\concat( (R \psubstn{z}{y}) \psubstp{Q}{P} ) \\
  (\lift{x}{R}) \psubstp{Q}{P}  
  :=
  \lift{(x)\substp{Q}{P}}{ R \psubstp{Q}{P} } \\
%   (\dropn{x})  \psubstp{Q}{P}       
%   := 
%   \left\{ 
%     \begin{array}{ccc} 
%       \dropn{\quotep{Q}} & & x \nameeq \quotep{P} \\
%       \dropn{x} & & otherwise \\
%     \end{array}
%   \right. 
  (\dropn{x})  \psubstp{Q}{P}       
  := 
  \left\{ 
    \begin{array}{ccc} 
      Q & & x \nameeq \quotep{P} \\
      \dropn{x} & & otherwise \\
    \end{array}
  \right.
\end{mathpar}
 

where

\begin{eqnarray}
  (x)\id{\{} \lpquote Q \rpquote / \lpquote P \rpquote \id{\}}            = 
  \left\{ 
    \begin{array}{ccc}
      \lpquote Q \rpquote & & x \nameeq \lpquote P \rpquote \\
      x & & otherwise \\
    \end{array}
  \right. \nonumber
\end{eqnarray}

and $z$ is chosen distinct from $\quotep{P}$, $\quotep{Q}$, the free
names in $Q$, and all the names in $R$. Our $\alpha$-equivalence will
be built in the standard way from this substitution.

\begin{remark}\label{rem:no_self_referential_names}
  One consequence of these definitions is that $\forall P. \quotep{P}
  \not\in \freenames{P}$.
\end{remark}

\subsection{ Dynamic quote: an example }

Anticipating something of what's to come, consider applying the
substitution, $\widehat{\id{\{}u / z \id{\}}}$, to the following pair
of processes, $\lift{w}{y!(z)}$ and $w[ \lpquote y!(z) \rpquote ]$.

\begin{eqnarray}
	\lift{w}{y!(z)}\widehat{\id{\{}u / z \id{\}}}
		& = &
		\lift{w}{y!(u)} \nonumber\\
	w[ \lpquote y!(z) \rpquote ] \widehat{ \id{\{}u / z \id{\}} }
		& = &
		w[ \lpquote y!(z) \rpquote ] \nonumber
\end{eqnarray}

Because the body of the process between quotes is impervious to
substitution, we get radically different answers. In fact, by
examining the first process in an input context,
e.g. $x?(z).\lift{w}{y!(z)}$, we see that the process under the lift
operator may be shaped by prefixed inputs binding a name inside it. In
this sense, the lift operator will be seen as a way to dynamically
construct processes before reifying them as names.

Finally equipped with these standard features we can present the
dynamics of the calculus.

\subsubsection{Operational semantics} 

Finally, we introduce the computational dynamics. What marks these
algebras as distinct from other more traditionally studied algebraic
structures, e.g. vector spaces or polynomial rings, is the manner in
which dynamics is captured. In traditional structures, dynamics is typically
expressed through morphisms between such structures, as in linear maps
between vector spaces or morphisms between rings. In algebras
associated with the semantics of computation, the dynamics is
expressed as part of the algebraic structure itself, through a
reduction reduction relation typically denoted by $\red$. Below, we
give a recursive presentation of this relation for the calculus used
in the encoding.

$\red \subseteq \pi \times \pi$
$\red : \pi \to \mathcal{P}(\pi)$

\begin{mathpar}
  \inferrule* [lab=Comm] { \textsf{match}( x_{src}, x_{trgt} ) } { x_{trgt}?(y)P \; | \; x_{src}!\langle {Q} \rangle \red P\{\quotep{Q}/y}\} }
  \and \\
  \inferrule* [lab=Par] {{P} \red {P}'} {{{P} | {Q}} \red {{P}' | {Q}}}
  \and
  \inferrule* [lab=Equiv]{{{P} \scong {P}'} \andalso {{P}' \red {Q}'} \andalso {{Q}' \scong {Q}}}{{P} \red {Q}}
\end{mathpar}

\begin{eqnarray*}
  match_{\equiv} (\quotep{P},\quotep{Q}) & := & P \equiv Q \\
  match_{\dagger}(\quotep{P},\quotep{Q}) & := & \forall R. P|Q \red^{*} R => R \red^{*} 0 \\
  match_{K}(\quotep{P},\quotep{Q}) & := & K \mbox{ for some context } K
\end{eqnarray*}

$u?(x)P | u!\langle Q \rangle \red P\{\quotep{Q}/x\}$

%We write $\wred$ for $\red^*$, and $P\red$ if $\exists Q $ such that $ P \red Q$.
We write $P\red$ if $\exists Q $ such that $ P \red Q$ and $P\not\red$, otherwise.

\section{Replication}

As mentioned before, it is known that replication (and hence
recursion) can be implemented in a higher-order process algebra
\cite{SangiorgiWalker}. As our first example of calculation with the
machinery thus far presented we give the construction explicitly in
the {\rhoc}.

\begin{eqnarray}
	D_{x} & := & \prefix{x}{y}{(\binpar{\outputp{x}{y}}{@{y}})} \nonumber\\
	\bangp_{x}{P} & := & \binpar{{x}!\langle{\binpar{D_{x}}{P}}\rangle}{D_{x}} \nonumber
\end{eqnarray}

\begin{eqnarray}
	\bangp_{x}{P} & & \nonumber\\
	=
	& {x}!\langle{(\prefix{x}{y}{(\outputp{x}{y} | @{y})) | P}}\rangle 
	      | \prefix{x}{y}{(\outputp{x}{y} | @{y})} & \nonumber\\
	\red
	& (\outputp{x}{y} | @{y})\substn{\quotep{(\prefix{x}{y}{(@{y} | \outputp{x}{y})) | P}}}{y} & \nonumber\\
	=
	& \outputp{x}{\quotep{(\prefix{x}{y}{(\outputp{x}{y} | @{y})) | P}}}
	  | {(\prefix{x}{y}{(\outputp{x}{y} | @{y})) | P}} & \nonumber\\
	\red
	& \ldots & \nonumber\\
	\red^*
	& P | P | \ldots & \nonumber
\end{eqnarray}

Of course, this encoding, as an implementation, runs away, unfolding
$\bangp{P}$ eagerly. A lazier and more implementable replication
operator, restricted to input-guarded processes, may be obtained as follows.

\begin{eqnarray}
\bangp{\prefix{u}{v}{P}} 
	:= 
	\binpar{\lift{x}{\prefix{u}{v}{(\binpar{D(x)}{P})}}}{D(x)} \nonumber
\end{eqnarray}

\begin{remark}
  Note that the lazier definition still does not deal with summation
  or mixed summation (i.e. sums over input and output). The reader is
  invited to construct definitions of replication that deal with these
  features. 

  Further, the definitions are parameterized in a name, $x$. Can you,
  gentle reader, make a definition that eliminates this parameter and
  guarantees no accidental interaction between the replication
  machinery and the process being replicated -- i.e. no accidental
  sharing of names used by the process to get its work done and the
  name(s) used by the replication to effect copying. This latter
  revision of the definition of replication is crucial to obtaining
  the expected identity $!!P \sim !P$.
\end{remark}

\begin{remark}\label{rem:paradoxical_combinator}
  The reader familiar with the lambda calculus will have noticed the
  similarity between $D$ and the paradoxical combinator.

  [Ed. note: the existence of this seems to suggest we have to be more
  restrictive on the set of processes and names we admit if we are to
  support no-cloning.]
\end{remark}

\subsubsection{Bisimulation}

The computational dynamics gives rise to another kind of equivalence,
the equivalence of computational behavior. As previously mentioned
this is typically captured \emph{via} some form of bisimulation.

% The notion we use in this paper is weak barbed bisimulation
% \cite{milner91polyadicpi}.

The notion we use in this paper is derived from weak barbed
bisimulation \cite{milner91polyadicpi}. 

\begin{definition}
An \emph{observation relation}, $\downarrow_{\mathcal N}$, over a set
of names, $\mathcal N$, is the smallest relation satisfying the rules
below.

\infrule[Out-barb]{y \in {\mathcal N}, \; x \nameeq y}
		  {\outputp{x}{v} \downarrow_{\mathcal N} x}
\infrule[Par-barb]{\mbox{$P\downarrow_{\mathcal N} x$ or $Q\downarrow_{\mathcal N} x$}}
		  {\binpar{P}{Q} \downarrow_{\mathcal N} x}

We write $P \Downarrow_{\mathcal N} x$ if there is $Q$ such that 
$P \wred Q$ and $Q \downarrow_{\mathcal N} x$.
\end{definition}

\begin{definition}
%\label{def.bbisim}
An  ${\mathcal N}$-\emph{barbed bisimulation} over a set of names, ${\mathcal N}$, is a symmetric binary relation 
${\mathcal S}_{\mathcal N}$ between agents such that $P\rel{S}_{\mathcal N}Q$ implies:
\begin{enumerate}
\item If $P \red P'$ then $Q \wred Q'$ and $P'\rel{S}_{\mathcal N} Q'$.
\item If $P\downarrow_{\mathcal N} x$, then $Q\Downarrow_{\mathcal N} x$.
\end{enumerate}
$P$ is ${\mathcal N}$-barbed bisimilar to $Q$, written
$P \wbbisim_{\mathcal N} Q$, if $P \rel{S}_{\mathcal N} Q$ for some ${\mathcal N}$-barbed bisimulation ${\mathcal S}_{\mathcal N}$.
\end{definition}

$\mathcal{R} \subseteq \pi \times \pi$

$P \mathcal{R} Q => \forall P'. P \red P' \Rightarrow \exists Q'. Q \red Q', P' \mathcal{R} Q'$

$P \vdash x \Rightarrow Q \vdash x$

\begin{mathpar}
  \inferrule*[lab=Out-barb]{x \nameeq y}{{y}!\langle{Q}\rangle \vdash x}
  \and
  \inferrule*[lab=Par-barb]{\mbox{$P\vdash x$ or $Q\vdash x$}}{\binpar{P}{Q} \vdash x}
\end{mathpar}

\subsubsection{Contexts}

One of the principle advantages of computational calculi like the
$\pi$-calculus is a well-defined notion of context,
contextual-equivalence and a correlation between
contextual-equivalence and notions of bisimulation. The notion of
context allows the decomposition of a process into (sub-)process and
its syntactic environment, its context. Thus, a context may be
thought of as a process with a ``hole'' (written $\Box$) in it. The
application of a context $M$ to a process $P$, written $M[P]$, is
tantamount to filling the hole in $M$ with $P$. In this paper we do
not need the full weight of this theory, but do make use of the notion
of context in the proof the main theorem. 

\begin{mathpar}
  \inferrule* [lab=summation] {} {{M_{M},M_{N}} \bc \Box \;|\; x.M_{A} \;|\; M_{M}+M_{N}}
  \and
  \inferrule* [lab=agent] {} {{M_{A}} \bc (\vec{x})M_{P} \;| \; \clift{P_0,\ldots,M_{P},\ldots,P_N}}
  \and \\
  \inferrule* [lab=process] {} {{M_{P}} \bc M_{N} \;| \;P|M_{P} }
\end{mathpar} 

\begin{mathpar}
  \inferrule* [lab=sychronization] {} {M_{N} \bc \Box \;|\; x?M_{F} \;|\; x!M_{C}}
  \and
  \inferrule* [lab=abstraction] {} {{M_{F}} \bc (x)M_{P} }
  \and
  \inferrule* [lab=concretion] {} {{M_{C}} \bc \langle M_{P} \rangle }
  \and \\
  \inferrule* [lab=process] {} {{M_{P}} \bc M_{N} \;| \;P|M_{P} }
\end{mathpar}

\begin{definition}[contextual application] Given a context $M$, and
  process $P$, we define the \emph{contextual application}, $M[P] :=
  M\{P/\Box\}$. That is, the contextual application of M to P is the
  substitution of $P$ for $\Box$ in $M$.
\end{definition}

$\meaningof{-} : L \to \mathcal{P}(\pi)$

\begin{mathpar}
  \inferrule* [lab=collection] {} {\meaningof{true} = \pi, \and \meaningof{~E} = \pi \setminus \meaningof{E}, \and \meaningof{E_{1} \& E_{2}} = \meaningof{E_{1}} \cap \meaningof{E_{2}}}
\end{mathpar}

\begin{mathpar}
  \inferrule* [lab=structure] {} {\meaningof{0} = \{ P \in \pi | P \equiv 0 \}, \and \\ \meaningof{E_1 | E_2} = \{ P \in \pi | P \equiv P_{1} | P_{2}, P_{1} \in \meaningof{E_{1}}, P_{2} \in \meaningof{E_2}\} }
\end{mathpar}

\begin{mathpar}
 \inferrule* [lab=behavior] {} {\meaningof{\langle a?b \rangle E} = \{ P \in \pi | P \equiv Q | u?(y)P', \\ \and \\\\ \and \\ \;\;\; u \in \meaningof{a}, \forall z.P'\{z/y\} \in \meaningof{E\{z/b\}}\}, \and \\ \meaningof{a!E} = \{ P \in \pi | P \equiv Q | x!\langle P' \rangle, x \in \meaningof{a} P' \in \meaningof{E}\} }
\end{mathpar}

\begin{mathpar}
 \inferrule* [lab=nominal] {} {\meaningof{\quotep{E}} = \{ \quotep{P} \in \quotep{\pi} | P \in \meaningof{E} \}, \and \meaningof{\quotep{P}} = \{ \quotep{Q} \in \quotep{\pi} | P \equiv Q \} \and \\ \meaningof{@\quotep{E}} = \{ P \in \pi | P \equiv @x, x \in \meaningof{E} \}}
\end{mathpar}

\begin{eqnarray*}
  \\
  \meaningof{-} : TS \to ST
\end{eqnarray*}

\begin{eqnarray*}
  \\
  L : TS \to ST
\end{eqnarray*}

\begin{eqnarray*}
  \\
  P \models E \iff P \in \meaningof{E}
\end{eqnarray*}

\begin{eqnarray*}
  P \approx_{L} Q \iff \forall E \in L. P \models E \iff Q \models E
\end{eqnarray*}

\begin{eqnarray*}
  P \approx_{K} Q
\end{eqnarray*}

\begin{eqnarray*}
  P \approx Q
\end{eqnarray*}

$\approx_{K} = \approx = \approx_{L}$

\subsubsection{Contextual duality}

Note that contexts extend the quotation operation to a family of
operations from processes to names. Given a context, $M$, we can
define a \emph{nominal context}, $\quotep{M}$ by $\quotep{M}[P] :=
\quotep{M[P]}$. To foreshadow what is to come we observe that these
operations enjoy a duality with processes very much like the duality
between vectors and maps from vectors to scalars.

Further, because the calculus is essentially higher-order, we have a
correspondence between contexts and processes. More specifically,
given a name $x$ and a context $M$ we can construct $M^{*}_{x}$ such
that 

\begin{mathpar}
  M^{*}_{x} | \lift{x}{P} \red M[P]
\end{mathpar}

namely,

\begin{mathpar}
  M^{*}_{x} := x?(u).M[\dropn{u}]
\end{mathpar}

The dependence of $M^{*}_{x}$ on a name makes it an abstraction, 

\begin{mathpar}
  M^{*} := (x)x?(u).M[\dropn{u}]
\end{mathpar}

\subsection{Additional notation}

It will sometimes be convenient to denote the process a name
quotes. We already have the notation $x = \quotep{P}$, but it will be
convenient to introduce an alternate notation, $\procn{x}$, when we
want to emphasize the connection to the use of the name. Note that, by
virtue of name equivalence, $\quotep{\procn{x}} \nameeq x$; so, the
notation is consistent with previous definitions.

Further, because names have structure it is possible to effect
substitutions on the basis of that structure. This means we need to
upgrade our notation for substitutions, which we accomplish by
adapting comprehension notation. Thus,

\begin{mathpar}
  P\{ y / x : x \in S \}
\end{mathpar}

is interpreted to mean the process derived from P by replacing (in a
capture-avoiding manner) each occurrence of $x$ in $S$ by $y$. For example,

\begin{mathpar}
  P\{ \quotep{\procn{x}|\procn{x}} / x : x \in \freenames{P} \}
\end{mathpar}

will replace each (occurrence) of a free name $x$ in $P$ by
$\quotep{\procn{x}|\procn{x}}$.

Also, we will avail ourselves of the notation $x^{L}$ and $x^{R}$ to
denote injections of a name into disjoint copies of the name
space. There are numerous ways to accomplish this. One example can be
found in \cite{MeredithR05}. This notation overloads to vectors of
names: $\vec{x}^{\pi} := (x_{i}^{\pi} \; : \; 0 \leq i < |\vec{x}| )$ where $\pi \in \{L,R\}$.

We also use $P^{\Box} := P|\Box$.

In \cite{MeredithR05} an interpretation of the new operator is
given. It turns out that there are several possible interpretations
all enjoying the requisite algebraic properties of the operator (see
\cite{milner91polyadicpi}). We will therefore make liberal use of
$(\nu\; \vec{x})P$.

% subsection the_syntax_and_semantics_of_the_notation_system (end)   

\input{qm2pi.qmops} 

\input{qm2pi.sterngerlach} 

\input{qm2pi.metric} 

% section concurrent_process_calculi (end)

%\input{qm2pi.proofsketch}

% section proof sketch (end)

%\input{qm2pi.slviaknots} 

% section spatial logic via knots (end)

\input{qm2pi.conclusion}

% section conclusion (end)

%\input{qm2pi.dtcodes} 

% section wiring algorithm (end)

\input{qm2pi.ack} 

% section acknowledgments (end)

\newpage


\bibliographystyle{plain}   
\bibliography{../../biblios/main.bib}

\input{qm2pi.rhodetails}

\end{document}

 

% subsection basic_interpretation (end)

%\input{qm2pi.rho.presentation} 
\subsection{The syntax and semantics of the notation system}\label{sub:the_syntax_and_semantics_of_the_notation_system} % (fold)

We now summarize a technical presentation of the calculus that
embodies our theory of dynamics. The typical presentation of such a
calculus follows the style of giving generators and relations on
them. The grammar, below, describing term constructors, freely
generates the set of processes, $\Proc$. This set is then quotiented
by a relation known as structural congruence and it is over this set
that the notion of dynamics is expressed. This presentation is
essentially that of \cite{MeredithR05} with the addition of
polyadicity and summation. For readability we have relegated some of
the technical subtleties to an appendix.

\subsubsection{Process grammar}\label{subsub:process_grammar}

\begin{mathpar}
  \inferrule* [lab=synchronization] {} {{M} \bc \pzero \;|\; x?F \;|\; x!C }
  \and
  \inferrule* [lab=abstraction] {} {{F} \bc (x)P}
  \and
  \inferrule* [lab=concretion] {} {{C} \bc \langle Q \rangle}
  \and
  \inferrule* [lab=process] {} {{P,Q} \bc M \;| \;P|Q \;|\; @{x}}
  \and
  \inferrule* [lab=name] {} {{x} \bc \quotep{P}}
\end{mathpar} 

Note that $\vec{x}$ (resp. $\vec{P}$) denotes a vector of names
(resp. processes) of length $|\vec{x}|$ (resp. $|\vec{P}|$). We adopt
the following useful abbreviations.

\begin{mathpar}
   x?(\vec{y}).P := x.(\vec{y})P \and  x\clift{\vec{P}} := x.\clift{\vec{P}}
   \and x!(y) := \lift{x}{\dropn{y}}
   \and \Pi_{i=0}^{n-1}P_i := P_0 | \ldots | P_{n-1}
\end{mathpar}

\subsubsection{Structural congruence}

\paragraph{Free and bound names and alpha-equivalence.} At the
core of structural equivalence is alpha-equivalence which identifies
process that are the same up to a change of variable. Formally, we
recognize the distinction between free and bound names. The free names
of a process, $\freenames{P}$, may be calculated recursively as
follows:

\begin{mathpar}
\freenames{\pzero} := \emptyset
  \and \\
  \freenames{x?(y).P} := \{ x \} \cup (\freenames{P} \setminus \{ y \})
  \and 
  \freenames{x!\langle P \rangle} := \{ x \} \cup \{ P \} 
  \and \\
  \freenames{P|Q} := \freenames{P} \cup \freenames{Q}
  \and \\
  \freenames{@{x}} := \{ x \}
\end{mathpar}

$\pi$
$\quotep{\pi}$

$\freenames{-} : \pi \to \mathcal{P}(\quotep{\pi})$

\begin{eqnarray*}
  \freenames{\pzero} & := & \emptyset \\
  \freenames{x?(y).P} & := & \{ x \} \cup (\freenames{P} \setminus \{ y \}) \\
  \freenames{x!\langle P \rangle} & := & \{ x \} \cup \{ P \} \\
  \freenames{P|Q} & := & \freenames{P} \cup \freenames{Q} \\
  \freenames{\dropn{x}} & := & \{ x \}
\end{eqnarray*}

The bound names of a process, $\boundnames{P}$, are those names occurring in $P$
that are not free. For example, in $x?(y).0$, the name $x$ is free, while $y$ is bound.

\begin{mathpar}
  \inferrule* [lab=monoidal-laws] {} { P|Q \equiv Q|P \and P|0 \equiv P \and P|(Q|R) \equiv (P|Q)|R }
\end{mathpar}

\begin{mathpar}
  \inferrule* [lab=alpha-equivalence] {} { (x)P \equiv (y)P\{y/x\} \and y \not\in \freenames{P} }
\end{mathpar}

\begin{definition}
Then two processes, $P,Q$, are alpha-equivalent if $P = Q\{\vec{y}/\vec{x}\}$ for
some $\vec{x} \in \boundnames{Q},\vec{y} \in \boundnames{P}$, where $Q\{\vec{y}/\vec{x}\}$
denotes the capture-avoiding substitution of $\vec{y}$ for $\vec{x}$ in $Q$.
\end{definition}

\begin{definition}
  The {\em structural congruence} \cite{SangiorgiWalker} , $\equiv$,
  between processes is the least congruence containing
  alpha-equivalence, satisfying the abelian monoid laws
  (associativity, commutativity and $\pzero$ as identity) for parallel
  composition $|$ and for summation $+$.
\end{definition}

\subsection{Name equivalence}

We take name equivalence, written $\nameeq$, to be the smallest
equivalence relation generated by the following rules.

\begin{mathpar}
\inferrule*[lab=Quote-drop]
{ }
{ \quotep{@{x}} \nameeq x }

\inferrule*[lab=Struct-equiv]
{ P \scong Q }
{ \quotep{P} \nameeq \quotep{Q} }
\end{mathpar}

The astute reader will have noticed that the mutual recursion of names
and processes imposes a mutual recursion on alpha-equivalence and
structural equivalence via name-equivalence. Fortunately, all of this
works out pleasantly and we may calculate in the natural way, free of
concern. The reader interested in the details is referred to the
appendix \ref{appendix:rho_details}.

\subsection{Substitution}

We use $\Proc$ for the set of processes, $\QProc$ for the set of
names, and $\id{\{}\vec{y} / \vec{x} \id{\}}$ to denote partial maps,
$s : \QProc \rightarrow \QProc$. A map, $s$ lifts, uniquely, to a map
on process terms, $\widehat{s} : \Proc \rightarrow \Proc$ by the
following equations.

\begin{mathpar}
  (0) \psubstp{Q}{P} := 0 \\
  (R \juxtap S) \psubstp{Q}{P}
  :=    
  (R)\psubstp{Q}{P} \juxtap (S) \psubstp{Q}{P} \\
  (x?(y).R) \psubstp{Q}{P}    
  :=    
  (x)\substp{Q}{P} (z)\concat( (R \psubstn{z}{y}) \psubstp{Q}{P} ) \\
  (\lift{x}{R}) \psubstp{Q}{P}  
  :=
  \lift{(x)\substp{Q}{P}}{ R \psubstp{Q}{P} } \\
%   (\dropn{x})  \psubstp{Q}{P}       
%   := 
%   \left\{ 
%     \begin{array}{ccc} 
%       \dropn{\quotep{Q}} & & x \nameeq \quotep{P} \\
%       \dropn{x} & & otherwise \\
%     \end{array}
%   \right. 
  (\dropn{x})  \psubstp{Q}{P}       
  := 
  \left\{ 
    \begin{array}{ccc} 
      Q & & x \nameeq \quotep{P} \\
      \dropn{x} & & otherwise \\
    \end{array}
  \right.
\end{mathpar}
 

where

\begin{eqnarray}
  (x)\id{\{} \lpquote Q \rpquote / \lpquote P \rpquote \id{\}}            = 
  \left\{ 
    \begin{array}{ccc}
      \lpquote Q \rpquote & & x \nameeq \lpquote P \rpquote \\
      x & & otherwise \\
    \end{array}
  \right. \nonumber
\end{eqnarray}

and $z$ is chosen distinct from $\quotep{P}$, $\quotep{Q}$, the free
names in $Q$, and all the names in $R$. Our $\alpha$-equivalence will
be built in the standard way from this substitution.

\begin{remark}\label{rem:no_self_referential_names}
  One consequence of these definitions is that $\forall P. \quotep{P}
  \not\in \freenames{P}$.
\end{remark}

\subsection{ Dynamic quote: an example }

Anticipating something of what's to come, consider applying the
substitution, $\widehat{\id{\{}u / z \id{\}}}$, to the following pair
of processes, $\lift{w}{y!(z)}$ and $w[ \lpquote y!(z) \rpquote ]$.

\begin{eqnarray}
	\lift{w}{y!(z)}\widehat{\id{\{}u / z \id{\}}}
		& = &
		\lift{w}{y!(u)} \nonumber\\
	w[ \lpquote y!(z) \rpquote ] \widehat{ \id{\{}u / z \id{\}} }
		& = &
		w[ \lpquote y!(z) \rpquote ] \nonumber
\end{eqnarray}

Because the body of the process between quotes is impervious to
substitution, we get radically different answers. In fact, by
examining the first process in an input context,
e.g. $x?(z).\lift{w}{y!(z)}$, we see that the process under the lift
operator may be shaped by prefixed inputs binding a name inside it. In
this sense, the lift operator will be seen as a way to dynamically
construct processes before reifying them as names.

Finally equipped with these standard features we can present the
dynamics of the calculus.

\subsubsection{Operational semantics} 

Finally, we introduce the computational dynamics. What marks these
algebras as distinct from other more traditionally studied algebraic
structures, e.g. vector spaces or polynomial rings, is the manner in
which dynamics is captured. In traditional structures, dynamics is typically
expressed through morphisms between such structures, as in linear maps
between vector spaces or morphisms between rings. In algebras
associated with the semantics of computation, the dynamics is
expressed as part of the algebraic structure itself, through a
reduction reduction relation typically denoted by $\red$. Below, we
give a recursive presentation of this relation for the calculus used
in the encoding.

$\red \subseteq \pi \times \pi$
$\red : \pi \to \mathcal{P}(\pi)$

\begin{mathpar}
  \inferrule* [lab=Comm] { \textsf{match}( x_{src}, x_{trgt} ) } { x_{trgt}?(y)P \; | \; x_{src}!\langle {Q} \rangle \red P\{\quotep{Q}/y}\} }
  \and \\
  \inferrule* [lab=Par] {{P} \red {P}'} {{{P} | {Q}} \red {{P}' | {Q}}}
  \and
  \inferrule* [lab=Equiv]{{{P} \scong {P}'} \andalso {{P}' \red {Q}'} \andalso {{Q}' \scong {Q}}}{{P} \red {Q}}
\end{mathpar}

\begin{eqnarray*}
  match_{\equiv} (\quotep{P},\quotep{Q}) & := & P \equiv Q \\
  match_{\dagger}(\quotep{P},\quotep{Q}) & := & \forall R. P|Q \red^{*} R => R \red^{*} 0 \\
  match_{K}(\quotep{P},\quotep{Q}) & := & K \mbox{ for some context } K
\end{eqnarray*}

$u?(x)P | u!\langle Q \rangle \red P\{\quotep{Q}/x\}$

%We write $\wred$ for $\red^*$, and $P\red$ if $\exists Q $ such that $ P \red Q$.
We write $P\red$ if $\exists Q $ such that $ P \red Q$ and $P\not\red$, otherwise.

\section{Replication}

As mentioned before, it is known that replication (and hence
recursion) can be implemented in a higher-order process algebra
\cite{SangiorgiWalker}. As our first example of calculation with the
machinery thus far presented we give the construction explicitly in
the {\rhoc}.

\begin{eqnarray}
	D_{x} & := & \prefix{x}{y}{(\binpar{\outputp{x}{y}}{@{y}})} \nonumber\\
	\bangp_{x}{P} & := & \binpar{{x}!\langle{\binpar{D_{x}}{P}}\rangle}{D_{x}} \nonumber
\end{eqnarray}

\begin{eqnarray}
	\bangp_{x}{P} & & \nonumber\\
	=
	& {x}!\langle{(\prefix{x}{y}{(\outputp{x}{y} | @{y})) | P}}\rangle 
	      | \prefix{x}{y}{(\outputp{x}{y} | @{y})} & \nonumber\\
	\red
	& (\outputp{x}{y} | @{y})\substn{\quotep{(\prefix{x}{y}{(@{y} | \outputp{x}{y})) | P}}}{y} & \nonumber\\
	=
	& \outputp{x}{\quotep{(\prefix{x}{y}{(\outputp{x}{y} | @{y})) | P}}}
	  | {(\prefix{x}{y}{(\outputp{x}{y} | @{y})) | P}} & \nonumber\\
	\red
	& \ldots & \nonumber\\
	\red^*
	& P | P | \ldots & \nonumber
\end{eqnarray}

Of course, this encoding, as an implementation, runs away, unfolding
$\bangp{P}$ eagerly. A lazier and more implementable replication
operator, restricted to input-guarded processes, may be obtained as follows.

\begin{eqnarray}
\bangp{\prefix{u}{v}{P}} 
	:= 
	\binpar{\lift{x}{\prefix{u}{v}{(\binpar{D(x)}{P})}}}{D(x)} \nonumber
\end{eqnarray}

\begin{remark}
  Note that the lazier definition still does not deal with summation
  or mixed summation (i.e. sums over input and output). The reader is
  invited to construct definitions of replication that deal with these
  features. 

  Further, the definitions are parameterized in a name, $x$. Can you,
  gentle reader, make a definition that eliminates this parameter and
  guarantees no accidental interaction between the replication
  machinery and the process being replicated -- i.e. no accidental
  sharing of names used by the process to get its work done and the
  name(s) used by the replication to effect copying. This latter
  revision of the definition of replication is crucial to obtaining
  the expected identity $!!P \sim !P$.
\end{remark}

\begin{remark}\label{rem:paradoxical_combinator}
  The reader familiar with the lambda calculus will have noticed the
  similarity between $D$ and the paradoxical combinator.

  [Ed. note: the existence of this seems to suggest we have to be more
  restrictive on the set of processes and names we admit if we are to
  support no-cloning.]
\end{remark}

\subsubsection{Bisimulation}

The computational dynamics gives rise to another kind of equivalence,
the equivalence of computational behavior. As previously mentioned
this is typically captured \emph{via} some form of bisimulation.

% The notion we use in this paper is weak barbed bisimulation
% \cite{milner91polyadicpi}.

The notion we use in this paper is derived from weak barbed
bisimulation \cite{milner91polyadicpi}. 

\begin{definition}
An \emph{observation relation}, $\downarrow_{\mathcal N}$, over a set
of names, $\mathcal N$, is the smallest relation satisfying the rules
below.

\infrule[Out-barb]{y \in {\mathcal N}, \; x \nameeq y}
		  {\outputp{x}{v} \downarrow_{\mathcal N} x}
\infrule[Par-barb]{\mbox{$P\downarrow_{\mathcal N} x$ or $Q\downarrow_{\mathcal N} x$}}
		  {\binpar{P}{Q} \downarrow_{\mathcal N} x}

We write $P \Downarrow_{\mathcal N} x$ if there is $Q$ such that 
$P \wred Q$ and $Q \downarrow_{\mathcal N} x$.
\end{definition}

\begin{definition}
%\label{def.bbisim}
An  ${\mathcal N}$-\emph{barbed bisimulation} over a set of names, ${\mathcal N}$, is a symmetric binary relation 
${\mathcal S}_{\mathcal N}$ between agents such that $P\rel{S}_{\mathcal N}Q$ implies:
\begin{enumerate}
\item If $P \red P'$ then $Q \wred Q'$ and $P'\rel{S}_{\mathcal N} Q'$.
\item If $P\downarrow_{\mathcal N} x$, then $Q\Downarrow_{\mathcal N} x$.
\end{enumerate}
$P$ is ${\mathcal N}$-barbed bisimilar to $Q$, written
$P \wbbisim_{\mathcal N} Q$, if $P \rel{S}_{\mathcal N} Q$ for some ${\mathcal N}$-barbed bisimulation ${\mathcal S}_{\mathcal N}$.
\end{definition}

$\mathcal{R} \subseteq \pi \times \pi$

$P \mathcal{R} Q => \forall P'. P \red P' \Rightarrow \exists Q'. Q \red Q', P' \mathcal{R} Q'$

$P \vdash x \Rightarrow Q \vdash x$

\begin{mathpar}
  \inferrule*[lab=Out-barb]{x \nameeq y}{{y}!\langle{Q}\rangle \vdash x}
  \and
  \inferrule*[lab=Par-barb]{\mbox{$P\vdash x$ or $Q\vdash x$}}{\binpar{P}{Q} \vdash x}
\end{mathpar}

\subsubsection{Contexts}

One of the principle advantages of computational calculi like the
$\pi$-calculus is a well-defined notion of context,
contextual-equivalence and a correlation between
contextual-equivalence and notions of bisimulation. The notion of
context allows the decomposition of a process into (sub-)process and
its syntactic environment, its context. Thus, a context may be
thought of as a process with a ``hole'' (written $\Box$) in it. The
application of a context $M$ to a process $P$, written $M[P]$, is
tantamount to filling the hole in $M$ with $P$. In this paper we do
not need the full weight of this theory, but do make use of the notion
of context in the proof the main theorem. 

\begin{mathpar}
  \inferrule* [lab=summation] {} {{M_{M},M_{N}} \bc \Box \;|\; x.M_{A} \;|\; M_{M}+M_{N}}
  \and
  \inferrule* [lab=agent] {} {{M_{A}} \bc (\vec{x})M_{P} \;| \; \clift{P_0,\ldots,M_{P},\ldots,P_N}}
  \and \\
  \inferrule* [lab=process] {} {{M_{P}} \bc M_{N} \;| \;P|M_{P} }
\end{mathpar} 

\begin{mathpar}
  \inferrule* [lab=sychronization] {} {M_{N} \bc \Box \;|\; x?M_{F} \;|\; x!M_{C}}
  \and
  \inferrule* [lab=abstraction] {} {{M_{F}} \bc (x)M_{P} }
  \and
  \inferrule* [lab=concretion] {} {{M_{C}} \bc \langle M_{P} \rangle }
  \and \\
  \inferrule* [lab=process] {} {{M_{P}} \bc M_{N} \;| \;P|M_{P} }
\end{mathpar}

\begin{definition}[contextual application] Given a context $M$, and
  process $P$, we define the \emph{contextual application}, $M[P] :=
  M\{P/\Box\}$. That is, the contextual application of M to P is the
  substitution of $P$ for $\Box$ in $M$.
\end{definition}

$\meaningof{-} : L \to \mathcal{P}(\pi)$

\begin{mathpar}
  \inferrule* [lab=collection] {} {\meaningof{true} = \pi, \and \meaningof{~E} = \pi \setminus \meaningof{E}, \and \meaningof{E_{1} \& E_{2}} = \meaningof{E_{1}} \cap \meaningof{E_{2}}}
\end{mathpar}

\begin{mathpar}
  \inferrule* [lab=structure] {} {\meaningof{0} = \{ P \in \pi | P \equiv 0 \}, \and \\ \meaningof{E_1 | E_2} = \{ P \in \pi | P \equiv P_{1} | P_{2}, P_{1} \in \meaningof{E_{1}}, P_{2} \in \meaningof{E_2}\} }
\end{mathpar}

\begin{mathpar}
 \inferrule* [lab=behavior] {} {\meaningof{\langle a?b \rangle E} = \{ P \in \pi | P \equiv Q | u?(y)P', \\ \and \\\\ \and \\ \;\;\; u \in \meaningof{a}, \forall z.P'\{z/y\} \in \meaningof{E\{z/b\}}\}, \and \\ \meaningof{a!E} = \{ P \in \pi | P \equiv Q | x!\langle P' \rangle, x \in \meaningof{a} P' \in \meaningof{E}\} }
\end{mathpar}

\begin{mathpar}
 \inferrule* [lab=nominal] {} {\meaningof{\quotep{E}} = \{ \quotep{P} \in \quotep{\pi} | P \in \meaningof{E} \}, \and \meaningof{\quotep{P}} = \{ \quotep{Q} \in \quotep{\pi} | P \equiv Q \} \and \\ \meaningof{@\quotep{E}} = \{ P \in \pi | P \equiv @x, x \in \meaningof{E} \}}
\end{mathpar}

\begin{eqnarray*}
  \\
  \meaningof{-} : TS \to ST
\end{eqnarray*}

\begin{eqnarray*}
  \\
  L : TS \to ST
\end{eqnarray*}

\begin{eqnarray*}
  \\
  P \models E \iff P \in \meaningof{E}
\end{eqnarray*}

\begin{eqnarray*}
  P \approx_{L} Q \iff \forall E \in L. P \models E \iff Q \models E
\end{eqnarray*}

\begin{eqnarray*}
  P \approx_{K} Q
\end{eqnarray*}

\begin{eqnarray*}
  P \approx Q
\end{eqnarray*}

$\approx_{K} = \approx = \approx_{L}$

\subsubsection{Contextual duality}

Note that contexts extend the quotation operation to a family of
operations from processes to names. Given a context, $M$, we can
define a \emph{nominal context}, $\quotep{M}$ by $\quotep{M}[P] :=
\quotep{M[P]}$. To foreshadow what is to come we observe that these
operations enjoy a duality with processes very much like the duality
between vectors and maps from vectors to scalars.

Further, because the calculus is essentially higher-order, we have a
correspondence between contexts and processes. More specifically,
given a name $x$ and a context $M$ we can construct $M^{*}_{x}$ such
that 

\begin{mathpar}
  M^{*}_{x} | \lift{x}{P} \red M[P]
\end{mathpar}

namely,

\begin{mathpar}
  M^{*}_{x} := x?(u).M[\dropn{u}]
\end{mathpar}

The dependence of $M^{*}_{x}$ on a name makes it an abstraction, 

\begin{mathpar}
  M^{*} := (x)x?(u).M[\dropn{u}]
\end{mathpar}

\subsection{Additional notation}

It will sometimes be convenient to denote the process a name
quotes. We already have the notation $x = \quotep{P}$, but it will be
convenient to introduce an alternate notation, $\procn{x}$, when we
want to emphasize the connection to the use of the name. Note that, by
virtue of name equivalence, $\quotep{\procn{x}} \nameeq x$; so, the
notation is consistent with previous definitions.

Further, because names have structure it is possible to effect
substitutions on the basis of that structure. This means we need to
upgrade our notation for substitutions, which we accomplish by
adapting comprehension notation. Thus,

\begin{mathpar}
  P\{ y / x : x \in S \}
\end{mathpar}

is interpreted to mean the process derived from P by replacing (in a
capture-avoiding manner) each occurrence of $x$ in $S$ by $y$. For example,

\begin{mathpar}
  P\{ \quotep{\procn{x}|\procn{x}} / x : x \in \freenames{P} \}
\end{mathpar}

will replace each (occurrence) of a free name $x$ in $P$ by
$\quotep{\procn{x}|\procn{x}}$.

Also, we will avail ourselves of the notation $x^{L}$ and $x^{R}$ to
denote injections of a name into disjoint copies of the name
space. There are numerous ways to accomplish this. One example can be
found in \cite{MeredithR05}. This notation overloads to vectors of
names: $\vec{x}^{\pi} := (x_{i}^{\pi} \; : \; 0 \leq i < |\vec{x}| )$ where $\pi \in \{L,R\}$.

We also use $P^{\Box} := P|\Box$.

In \cite{MeredithR05} an interpretation of the new operator is
given. It turns out that there are several possible interpretations
all enjoying the requisite algebraic properties of the operator (see
\cite{milner91polyadicpi}). We will therefore make liberal use of
$(\nu\; \vec{x})P$.

% subsection the_syntax_and_semantics_of_the_notation_system (end)   

\section{Interpretation of QM}
\subsection{Supporting definitions}
\subsubsection{Multiplication}
\begin{mathpar}
  \quotep{Q} \cdot \quotep{R} := \quotep{Q|R}
  \and \\
  \quotep{Q} \cdot P := P\{ \quotep{Q|R} / \quotep{R} : \quotep{R} \in \freenames{P} \}
\end{mathpar}

\paragraph{Discussion}
The first line needs little explanation. The second line says that
each free name of the process is replaced with the multiplication of
that name by the scalar. Multiplication of a scalar (name) by a state
(process) results in a process all the names of which have been `moved
over' by parallel composition with the process the scalar
quotes. There is a subtlety that the bound names have to be
manipulated so that multiplied names aren't accidentally
captured. There are many ways to achieve this.

\begin{remark}\label{rem:multiplication_identities}
  The reader is invited to verify that for all $x,y,z \in \QProc$ and $P \in \Proc$
  \begin{mathpar}
    x \cdot \quotep{0} \equiv x 
    \and
    x \cdot y \equiv y \cdot x
    \and
    x \cdot (y \cdot z) \equiv (x \cdot y) \cdot z
    \and \\
    \quotep{0} \cdot P \equiv P
    \and \\
    x \cdot (y \cdot P) \equiv (x \cdot y) \cdot P
    \and \\
    x \cdot (P|Q) \equiv (x \cdot P) | (x \cdot Q)
    \and \\    
  \end{mathpar}
\end{remark}

\subsubsection{Tensor product}

We define a tensor product on processes by structural induction.

\paragraph{Tensor of sums} First note that all summations, including
$\pzero$ and sequence, can be written $\Sigma_{i} x_{i}.A_{i} +
\Sigma_{j} x_{j}.C_{j}$, where we have grouped input-guarded processes
together and output-guarded processes together.

Thus, we can define the tensor product of two summations, $N_{1}\otimes N_{2}$, where

\begin{mathpar}
  N_{1} := \Sigma_{i} x_{i}.A_{i} + \Sigma_{j} x_{j}.C_{j}
  \and
  N_{2} := \Sigma_{i'} y_{i'}.B_{i'} + \Sigma_{j'} y_{j'}.D_{j'} 
\end{mathpar}

as follows.

\begin{mathpar}
  \Sigma_{i} x_{i}.A_{i} + \Sigma_{j} x_{j}.C_{j} \otimes \Sigma_{i'}
  y_{i'}.B_{i'} + \Sigma_{j'} y_{j'}.D_{j'} 
  \and \\
  := \; \Sigma_{i} \Sigma_{i'} \quotep{\stackrel{\vee}{x_{i}}| \stackrel{\vee}{y_{i'}}}.(A_{i}\otimes B_{i'}) \; | \; \Sigma_{i'} \Sigma_{i} \quotep{\stackrel{\vee}{y_{i'}}|\stackrel{\vee}{x_{i}}}.(B_{i'}\otimes A_{i})
  \and
  \;\; | \;\; \Sigma_{j} \Sigma_{j'} \quotep{\stackrel{\vee}{x_{j}}|\stackrel{\vee}{y_{j'}}}.(A_{j}\otimes B_{j'}) \; | \; \Sigma_{j'} \Sigma_{j} \quotep{\stackrel{\vee}{y_{j'}}|\stackrel{\vee}{x_{j}}}.(B_{j'}\otimes A_{j})
\end{mathpar}

\begin{remark}
  Do we need to $x^{L}$ and $y^{R}$ for this construction as well?
\end{remark}

\paragraph{Tensor of parallel compositions} Next, we distribute tensor
over par.

\begin{mathpar}
  P_{1}|P_{2} \otimes Q_{1}|Q_{2} := (P_{1} \otimes Q_{1}) | (P_{1}
  \otimes Q_{2}) | (P_{2} \otimes Q_{1}) | (P_{2} \otimes Q_{2})
\end{mathpar}

\paragraph{Tensor with dropped names} We treat tensor of a
process with a dropped name as parallel composition.

\begin{mathpar}
  P \otimes \dropn{x} := P | \dropn{x}
\end{mathpar}

\paragraph{Tensor of agents}

Finally, we need to define tensor on agents. Note that the definition
of tensor on normal products only tensors inputs with inputs and
outputs with outputs. Thus, we only have to define the operation on
``homogeneous'' pairings.

\begin{mathpar}
  (\vec{x})P \otimes (\vec{y})Q
  \and \\
  := (x_{0}^{L}|y_{0}^{R},\ldots,x_{0}^{L}|y_{n}^{R},\ldots,x_{m}^{L}|y_{0}^{R},\ldots,x_{m}^{L}|y_{n}^R)(P\{ \vec{x}^{L}/\vec{x}\} \otimes Q \{ \vec{y}^{R}/\vec{y}\})
  \and \\
  \clift{\vec{P}} \otimes \clift{\vec{Q}}
  \and \\
  := \clift{P_{0}\otimes Q_{0},\ldots,P_{0}\otimes Q_{n},\ldots,P_{m}\otimes Q_{0},\ldots,P_{m}\otimes Q_{n}}
\end{mathpar}

\begin{remark}
  Observe that arities of tensored abstractions matches arities of
  tensored concretions if the original arities matched. Note also that
  the length of the arities corresponds to the increase in dimension
  we see in ordinary vector space tensor product.
\end{remark}

\begin{remark}
  Operationally, this definition distributes the tensor down to
  components ``linked'' by summation. Tensor over summation is
  intriguing in that it mixes names. Moreover, as a consequence of the
  way it mixes names we have the identities for all $x \in \QProc$ and
  $P,Q \in \Proc$

  \begin{mathpar}
    (x \cdot P) \otimes Q \equiv x \cdot (P \otimes Q) \equiv P \otimes (x \cdot Q)
    \and
    P \otimes \pzero \equiv P
  \end{mathpar}

  that the reader is invited to verify.
\end{remark}

\subsubsection{Annihilation}
\begin{mathpar}
  P^{\perp} := \{ Q | \forall R. P|Q \red^{*} R \Rightarrow R \red^{*} \pzero \}
  \and \\
  P^{\underline{\perp}} := \Sigma_{Q \in P^{\perp}} \quotep{Q}?(y).(\dropn{y}|Q) | \Sigma_{Q \in P^{\perp}} \quotep{Q}\clift{\Box}
\end{mathpar}

\paragraph{Discussion} The reader will note that $P^{\perp}$ is a
\emph{set} of processes, while $P^{\underline{\perp}}$ is a
\emph{context}. We call the set $P^{\perp}$ the \emph{annihilators} of
$P$. The parallel composition of a process in the annihilators of $P$
with $P$ will result in a process, the state space of which has all
paths eventually leading to $\pzero$. Execution may endure loops; but
under reasonable conditions of fairness (naturally guaranteed under
most notions of bisimulation) such a composite process cannot get
stuck in such a loop and will, eventually pop out and terminate.

The context $P^{\underline{\perp}}$ is ready and willing to ``take the
$P$ out of'' the process to which it is applied. It will effectively
transmit the code of the process to which it is applied to one of the
annihilators and run the process against it.

\subsubsection{Evaluation}
We fix $M$ a domain of fully abstract interpretation with an equality
coincident with bisimulation. We take $\meaningof{\cdot} : \Proc \to
M$ to be the map interpreting processes and $\nmeaningof{\cdot} : \M
\to Proc$ to be the map running the other way. Then we define

\begin{mathpar}
  \int P := \nmeaningof{\meaningof{P}}
\end{mathpar}

\paragraph{Discussion}
There are many fully abstract interpretations of Milner's
$\pi$-calculus. Any of them can be used as a basis for interpreting
the reflective calculus here. Equipped with such a domain it is
largely a matter of grinding through to check that the Yoneda
construction for the normalization-by-evaluation program can be
extended to this setting.

\begin{remark}
  The reader is invited to verify that $\int (P^{\underline{\perp}}[P]) = 0$.
\end{remark}

\subsection{Quantum mechanics}

Table \ref{tbl:core_qm_op_defns} gives the core operational definitions

\begin{table}[htp]\label{tbl:core_qm_op_defns}
  \center{
    \fbox{
      \begin{tabular}{c|c}
        quantum mechanics & process calculus \\
        \hline
        scalar & $x := \quotep{P}$ \\
        state vector & $\state{P} := P$ \\
        dual & $\state{P}^{*} := \event{P^{\underline{\perp}}} := \quotep{P^{\underline{\perp}}}[-]$ \\
        matrix & $ \Sigma_{\alpha} \state{P_{\alpha}}x_{\alpha}\event{Q_{\alpha}}$ \\
        vector addition & $\state{P} + \state{Q} := \state{P | Q}$ \\
        tensor product & $\state{P} \otimes \state{Q} := \state{P \otimes Q}$ \\
        inner product & $\innerprod{P}{Q} := \quotep{\int P^{\underline{\perp}}[Q]}$ \\
      \end{tabular}
    }
  }
  \caption{QM - operational definitions}
\end{table}

where

\begin{mathpar}
  \prmatrix{P}{Q} := \fprmatrix{P}{\quotep{\pzero}}{Q}
  \and
  \fprmatrix{P}{x}{Q} := (\state{P},x,\event{Q})
  \and
  (\fprmatrix{P}{x}{Q})(\state{R}) := x \cdot \innerprod{Q}{R} \cdot \state{P}
  \and
  (\fprmatrix{P}{x}{Q})(\event{R}) := x \cdot \innerprod{R}{P} \cdot \event{Q}
\end{mathpar}

\paragraph{Discussion}
As promised: vectors (aka states) are represented as processes; duals
as contextual duals; inner product definition should be compared with
standard inner product definition for ....

\begin{remark}
  Assuming $\int (P^{\underline{\perp}}[P]) = 0$, the reader is
  invited to verify that $(\fprmatrix{P}{x}{P})(\state{P}) = x \cdot \state{P}$.
\end{remark}

\begin{remark}
  The reader is invited to verify that $\innerprod{P}{Q}$ could
  equally well have been written $\quotep{\int \stackrel{\vee}{x}}$
  where $x = \event{P^{\underline{\perp}}}(Q)$.

  One of the motivations for this remark is that there is another way
  to factor these operations. We could package up evaluation in the dual:

  \begin{mathpar}
    \state{P}^{*} := \event{\int P^{\underline{\perp}}} := \quotep{\int P^{\underline{\perp}}}[-]
  \end{mathpar}

  and then have inner product defined by
  
  \begin{mathpar}
    \innerprod{P}{Q} := \event{P}(Q)
  \end{mathpar}

  Hopefully, experience with the calculations will provide guidance on
  the best factoring.
\end{remark}

\begin{remark}
  Assuming $\int (P^{\underline{\perp}}[P]) = 0$, the reader is
  invited to verify that $\forall P,Q. (\prmatrix{0}{Q})(\state{0}) =
  \state{0}$ and dually $(\prmatrix{P}{0})(\event{0}) = \event{0}$.
\end{remark}

\begin{remark}
  i'm a little worried that i don't (yet) have proper support for
  complex conjugacy. But, the observation above may give us a
  clue. According to Abramsky, it must be the case that the scalars
  are iso to the homset of the identity for the tensor -- which the
  observation above characterizes. 

  For now, we will simply bookmark the notion with $\overline{x}$.
\end{remark}

\subsubsection{Adjointness}

We need to give a definition of $(\cdot)^{\dagger}$ for matrices. The
obvious candidate definition is
\begin{mathpar}
(\Sigma_{\alpha}\fprmatrix{P_{\alpha}}{x_{\alpha}}{Q_{\alpha}})^{\dagger}
= \Sigma_{\alpha}\fprmatrix{(Q_{\alpha}^{\underline{\perp}})^{*}}{\overline{x}_{\alpha}}{P_{\alpha}^{\underline{\perp}}} 
\end{mathpar}

But, $(Q_{\alpha}^{\underline{\perp}})^{*}$ requires a name along
which to communicate the process to achieve the context application.

\subsubsection{Basis for a basis}
If processes label states and ``addition'' of states (a.k.a. vector
addition) is interpreted as parallel composition, what corresponds to
notions of linear independence and basis? Here, we recall that Yoshida
has developed a set of \emph{combinators} for an asynchronous verison
of Milner's $\pi$-calculus. These are a finite set of processes such
any process can be expressed as parallel composition of these
combinators together with liberal uses of the new operator and
replication. We can simply give a translation of these into the
present calculus and have reasonable expectation that the property
carries over. That is, that the resultant set allows to express all
processes via parallel composition. Note, however, that there is no
new operator or replication in this calculus. As a result, we expect
that the corresponding set is actually infinite. That is, we expect
that the space is actually infinite dimensional.

\begin{remark}
  The attentive reader may be a bit concerned. Certainly, the
  collection $S$, $K$ and $I$ is a finite set of
  combinators. Shouldn't we expect to see a finite set of combinators
  for an effectively equivalent system? i am very sympathetic to this
  critique and feel it warrants full attention. On the other hand, i
  also have in mind the following analogy. The natural numbers, as a
  monoid under addition, has exactly $1$ generator, while the natural
  numbers, as a monoid under multiplication, has countably many
  generators (the primes). We observe that the application of the
  lambda calculus is much less resource sensitive than the parallel
  composition of the $\pi$-calculus. Could it be the case that we have
  an analogy of the form
  
  \begin{mathpar}
    m + n : MN :: m*n : M|N
  \end{mathpar}

  giving a similar blow up in the set of ``primes''?  This is such a
  wonderful thought that, even if it's not true, i think it's worth
  writing down.
\end{remark}
 

\documentclass[12pt]{llncs}
%\documentclass{jktr}

\usepackage[pdftex]{hyperref}                   
\usepackage {listings}
\usepackage {mathpartir}
\usepackage{bcprules}
%\usepackage{listings}
                       
\usepackage{graphicx} 
%\usepackage[margins=2.5cm,nohead,nofoot]{geometry}
%\usepackage{geometry}
\usepackage{amsfonts}
\usepackage{amstext}
\usepackage{latexsym}
\usepackage{amssymb}
\usepackage{color}


%\include{myPreamble}
\include{qm2pi.local} 

%\ifpdf
%\usepackage[pdftex]{graphicx}
%\else
%\usepackage{graphicx}
%\fi

 % \ifpdf
%  \usepackage{pdfsync}
%  \if


%\title{Brief Article}
%\author{David F. Snyder}
%\author{L.G. Meredith}

%\address{Dept. of Math., Texas State University--San Marcos, San Marcos, TX 78666}
       
\pagestyle{empty}


\begin{document}

\lstset{language=[Objective]Caml,frame=shadowbox}

\input{qm2pi.front}

% section front matter (end)

\input{qm2pi.intro} 
 
% section introduction (end)

% \input{qm2pi.knotations} 

% section notation (end)

\input{qm2pi.process.calculi} 

% section concurrent_process_calculi_and_spatial_logics_ (end)
    
%\input{qm2pi.knots2pi} 

%\input{qm2pi.trefoil} 

%\input{qm2pi.mainthm} 

% subsection basic_interpretation (end)

%\input{qm2pi.rho.presentation} 
\subsection{The syntax and semantics of the notation system}\label{sub:the_syntax_and_semantics_of_the_notation_system} % (fold)

We now summarize a technical presentation of the calculus that
embodies our theory of dynamics. The typical presentation of such a
calculus follows the style of giving generators and relations on
them. The grammar, below, describing term constructors, freely
generates the set of processes, $\Proc$. This set is then quotiented
by a relation known as structural congruence and it is over this set
that the notion of dynamics is expressed. This presentation is
essentially that of \cite{MeredithR05} with the addition of
polyadicity and summation. For readability we have relegated some of
the technical subtleties to an appendix.

\subsubsection{Process grammar}\label{subsub:process_grammar}

\begin{mathpar}
  \inferrule* [lab=synchronization] {} {{M} \bc \pzero \;|\; x?F \;|\; x!C }
  \and
  \inferrule* [lab=abstraction] {} {{F} \bc (x)P}
  \and
  \inferrule* [lab=concretion] {} {{C} \bc \langle Q \rangle}
  \and
  \inferrule* [lab=process] {} {{P,Q} \bc M \;| \;P|Q \;|\; @{x}}
  \and
  \inferrule* [lab=name] {} {{x} \bc \quotep{P}}
\end{mathpar} 

Note that $\vec{x}$ (resp. $\vec{P}$) denotes a vector of names
(resp. processes) of length $|\vec{x}|$ (resp. $|\vec{P}|$). We adopt
the following useful abbreviations.

\begin{mathpar}
   x?(\vec{y}).P := x.(\vec{y})P \and  x\clift{\vec{P}} := x.\clift{\vec{P}}
   \and x!(y) := \lift{x}{\dropn{y}}
   \and \Pi_{i=0}^{n-1}P_i := P_0 | \ldots | P_{n-1}
\end{mathpar}

\subsubsection{Structural congruence}

\paragraph{Free and bound names and alpha-equivalence.} At the
core of structural equivalence is alpha-equivalence which identifies
process that are the same up to a change of variable. Formally, we
recognize the distinction between free and bound names. The free names
of a process, $\freenames{P}$, may be calculated recursively as
follows:

\begin{mathpar}
\freenames{\pzero} := \emptyset
  \and \\
  \freenames{x?(y).P} := \{ x \} \cup (\freenames{P} \setminus \{ y \})
  \and 
  \freenames{x!\langle P \rangle} := \{ x \} \cup \{ P \} 
  \and \\
  \freenames{P|Q} := \freenames{P} \cup \freenames{Q}
  \and \\
  \freenames{@{x}} := \{ x \}
\end{mathpar}

$\pi$
$\quotep{\pi}$

$\freenames{-} : \pi \to \mathcal{P}(\quotep{\pi})$

\begin{eqnarray*}
  \freenames{\pzero} & := & \emptyset \\
  \freenames{x?(y).P} & := & \{ x \} \cup (\freenames{P} \setminus \{ y \}) \\
  \freenames{x!\langle P \rangle} & := & \{ x \} \cup \{ P \} \\
  \freenames{P|Q} & := & \freenames{P} \cup \freenames{Q} \\
  \freenames{\dropn{x}} & := & \{ x \}
\end{eqnarray*}

The bound names of a process, $\boundnames{P}$, are those names occurring in $P$
that are not free. For example, in $x?(y).0$, the name $x$ is free, while $y$ is bound.

\begin{mathpar}
  \inferrule* [lab=monoidal-laws] {} { P|Q \equiv Q|P \and P|0 \equiv P \and P|(Q|R) \equiv (P|Q)|R }
\end{mathpar}

\begin{mathpar}
  \inferrule* [lab=alpha-equivalence] {} { (x)P \equiv (y)P\{y/x\} \and y \not\in \freenames{P} }
\end{mathpar}

\begin{definition}
Then two processes, $P,Q$, are alpha-equivalent if $P = Q\{\vec{y}/\vec{x}\}$ for
some $\vec{x} \in \boundnames{Q},\vec{y} \in \boundnames{P}$, where $Q\{\vec{y}/\vec{x}\}$
denotes the capture-avoiding substitution of $\vec{y}$ for $\vec{x}$ in $Q$.
\end{definition}

\begin{definition}
  The {\em structural congruence} \cite{SangiorgiWalker} , $\equiv$,
  between processes is the least congruence containing
  alpha-equivalence, satisfying the abelian monoid laws
  (associativity, commutativity and $\pzero$ as identity) for parallel
  composition $|$ and for summation $+$.
\end{definition}

\subsection{Name equivalence}

We take name equivalence, written $\nameeq$, to be the smallest
equivalence relation generated by the following rules.

\begin{mathpar}
\inferrule*[lab=Quote-drop]
{ }
{ \quotep{@{x}} \nameeq x }

\inferrule*[lab=Struct-equiv]
{ P \scong Q }
{ \quotep{P} \nameeq \quotep{Q} }
\end{mathpar}

The astute reader will have noticed that the mutual recursion of names
and processes imposes a mutual recursion on alpha-equivalence and
structural equivalence via name-equivalence. Fortunately, all of this
works out pleasantly and we may calculate in the natural way, free of
concern. The reader interested in the details is referred to the
appendix \ref{appendix:rho_details}.

\subsection{Substitution}

We use $\Proc$ for the set of processes, $\QProc$ for the set of
names, and $\id{\{}\vec{y} / \vec{x} \id{\}}$ to denote partial maps,
$s : \QProc \rightarrow \QProc$. A map, $s$ lifts, uniquely, to a map
on process terms, $\widehat{s} : \Proc \rightarrow \Proc$ by the
following equations.

\begin{mathpar}
  (0) \psubstp{Q}{P} := 0 \\
  (R \juxtap S) \psubstp{Q}{P}
  :=    
  (R)\psubstp{Q}{P} \juxtap (S) \psubstp{Q}{P} \\
  (x?(y).R) \psubstp{Q}{P}    
  :=    
  (x)\substp{Q}{P} (z)\concat( (R \psubstn{z}{y}) \psubstp{Q}{P} ) \\
  (\lift{x}{R}) \psubstp{Q}{P}  
  :=
  \lift{(x)\substp{Q}{P}}{ R \psubstp{Q}{P} } \\
%   (\dropn{x})  \psubstp{Q}{P}       
%   := 
%   \left\{ 
%     \begin{array}{ccc} 
%       \dropn{\quotep{Q}} & & x \nameeq \quotep{P} \\
%       \dropn{x} & & otherwise \\
%     \end{array}
%   \right. 
  (\dropn{x})  \psubstp{Q}{P}       
  := 
  \left\{ 
    \begin{array}{ccc} 
      Q & & x \nameeq \quotep{P} \\
      \dropn{x} & & otherwise \\
    \end{array}
  \right.
\end{mathpar}
 

where

\begin{eqnarray}
  (x)\id{\{} \lpquote Q \rpquote / \lpquote P \rpquote \id{\}}            = 
  \left\{ 
    \begin{array}{ccc}
      \lpquote Q \rpquote & & x \nameeq \lpquote P \rpquote \\
      x & & otherwise \\
    \end{array}
  \right. \nonumber
\end{eqnarray}

and $z$ is chosen distinct from $\quotep{P}$, $\quotep{Q}$, the free
names in $Q$, and all the names in $R$. Our $\alpha$-equivalence will
be built in the standard way from this substitution.

\begin{remark}\label{rem:no_self_referential_names}
  One consequence of these definitions is that $\forall P. \quotep{P}
  \not\in \freenames{P}$.
\end{remark}

\subsection{ Dynamic quote: an example }

Anticipating something of what's to come, consider applying the
substitution, $\widehat{\id{\{}u / z \id{\}}}$, to the following pair
of processes, $\lift{w}{y!(z)}$ and $w[ \lpquote y!(z) \rpquote ]$.

\begin{eqnarray}
	\lift{w}{y!(z)}\widehat{\id{\{}u / z \id{\}}}
		& = &
		\lift{w}{y!(u)} \nonumber\\
	w[ \lpquote y!(z) \rpquote ] \widehat{ \id{\{}u / z \id{\}} }
		& = &
		w[ \lpquote y!(z) \rpquote ] \nonumber
\end{eqnarray}

Because the body of the process between quotes is impervious to
substitution, we get radically different answers. In fact, by
examining the first process in an input context,
e.g. $x?(z).\lift{w}{y!(z)}$, we see that the process under the lift
operator may be shaped by prefixed inputs binding a name inside it. In
this sense, the lift operator will be seen as a way to dynamically
construct processes before reifying them as names.

Finally equipped with these standard features we can present the
dynamics of the calculus.

\subsubsection{Operational semantics} 

Finally, we introduce the computational dynamics. What marks these
algebras as distinct from other more traditionally studied algebraic
structures, e.g. vector spaces or polynomial rings, is the manner in
which dynamics is captured. In traditional structures, dynamics is typically
expressed through morphisms between such structures, as in linear maps
between vector spaces or morphisms between rings. In algebras
associated with the semantics of computation, the dynamics is
expressed as part of the algebraic structure itself, through a
reduction reduction relation typically denoted by $\red$. Below, we
give a recursive presentation of this relation for the calculus used
in the encoding.

$\red \subseteq \pi \times \pi$
$\red : \pi \to \mathcal{P}(\pi)$

\begin{mathpar}
  \inferrule* [lab=Comm] { \textsf{match}( x_{src}, x_{trgt} ) } { x_{trgt}?(y)P \; | \; x_{src}!\langle {Q} \rangle \red P\{\quotep{Q}/y}\} }
  \and \\
  \inferrule* [lab=Par] {{P} \red {P}'} {{{P} | {Q}} \red {{P}' | {Q}}}
  \and
  \inferrule* [lab=Equiv]{{{P} \scong {P}'} \andalso {{P}' \red {Q}'} \andalso {{Q}' \scong {Q}}}{{P} \red {Q}}
\end{mathpar}

\begin{eqnarray*}
  match_{\equiv} (\quotep{P},\quotep{Q}) & := & P \equiv Q \\
  match_{\dagger}(\quotep{P},\quotep{Q}) & := & \forall R. P|Q \red^{*} R => R \red^{*} 0 \\
  match_{K}(\quotep{P},\quotep{Q}) & := & K \mbox{ for some context } K
\end{eqnarray*}

$u?(x)P | u!\langle Q \rangle \red P\{\quotep{Q}/x\}$

%We write $\wred$ for $\red^*$, and $P\red$ if $\exists Q $ such that $ P \red Q$.
We write $P\red$ if $\exists Q $ such that $ P \red Q$ and $P\not\red$, otherwise.

\section{Replication}

As mentioned before, it is known that replication (and hence
recursion) can be implemented in a higher-order process algebra
\cite{SangiorgiWalker}. As our first example of calculation with the
machinery thus far presented we give the construction explicitly in
the {\rhoc}.

\begin{eqnarray}
	D_{x} & := & \prefix{x}{y}{(\binpar{\outputp{x}{y}}{@{y}})} \nonumber\\
	\bangp_{x}{P} & := & \binpar{{x}!\langle{\binpar{D_{x}}{P}}\rangle}{D_{x}} \nonumber
\end{eqnarray}

\begin{eqnarray}
	\bangp_{x}{P} & & \nonumber\\
	=
	& {x}!\langle{(\prefix{x}{y}{(\outputp{x}{y} | @{y})) | P}}\rangle 
	      | \prefix{x}{y}{(\outputp{x}{y} | @{y})} & \nonumber\\
	\red
	& (\outputp{x}{y} | @{y})\substn{\quotep{(\prefix{x}{y}{(@{y} | \outputp{x}{y})) | P}}}{y} & \nonumber\\
	=
	& \outputp{x}{\quotep{(\prefix{x}{y}{(\outputp{x}{y} | @{y})) | P}}}
	  | {(\prefix{x}{y}{(\outputp{x}{y} | @{y})) | P}} & \nonumber\\
	\red
	& \ldots & \nonumber\\
	\red^*
	& P | P | \ldots & \nonumber
\end{eqnarray}

Of course, this encoding, as an implementation, runs away, unfolding
$\bangp{P}$ eagerly. A lazier and more implementable replication
operator, restricted to input-guarded processes, may be obtained as follows.

\begin{eqnarray}
\bangp{\prefix{u}{v}{P}} 
	:= 
	\binpar{\lift{x}{\prefix{u}{v}{(\binpar{D(x)}{P})}}}{D(x)} \nonumber
\end{eqnarray}

\begin{remark}
  Note that the lazier definition still does not deal with summation
  or mixed summation (i.e. sums over input and output). The reader is
  invited to construct definitions of replication that deal with these
  features. 

  Further, the definitions are parameterized in a name, $x$. Can you,
  gentle reader, make a definition that eliminates this parameter and
  guarantees no accidental interaction between the replication
  machinery and the process being replicated -- i.e. no accidental
  sharing of names used by the process to get its work done and the
  name(s) used by the replication to effect copying. This latter
  revision of the definition of replication is crucial to obtaining
  the expected identity $!!P \sim !P$.
\end{remark}

\begin{remark}\label{rem:paradoxical_combinator}
  The reader familiar with the lambda calculus will have noticed the
  similarity between $D$ and the paradoxical combinator.

  [Ed. note: the existence of this seems to suggest we have to be more
  restrictive on the set of processes and names we admit if we are to
  support no-cloning.]
\end{remark}

\subsubsection{Bisimulation}

The computational dynamics gives rise to another kind of equivalence,
the equivalence of computational behavior. As previously mentioned
this is typically captured \emph{via} some form of bisimulation.

% The notion we use in this paper is weak barbed bisimulation
% \cite{milner91polyadicpi}.

The notion we use in this paper is derived from weak barbed
bisimulation \cite{milner91polyadicpi}. 

\begin{definition}
An \emph{observation relation}, $\downarrow_{\mathcal N}$, over a set
of names, $\mathcal N$, is the smallest relation satisfying the rules
below.

\infrule[Out-barb]{y \in {\mathcal N}, \; x \nameeq y}
		  {\outputp{x}{v} \downarrow_{\mathcal N} x}
\infrule[Par-barb]{\mbox{$P\downarrow_{\mathcal N} x$ or $Q\downarrow_{\mathcal N} x$}}
		  {\binpar{P}{Q} \downarrow_{\mathcal N} x}

We write $P \Downarrow_{\mathcal N} x$ if there is $Q$ such that 
$P \wred Q$ and $Q \downarrow_{\mathcal N} x$.
\end{definition}

\begin{definition}
%\label{def.bbisim}
An  ${\mathcal N}$-\emph{barbed bisimulation} over a set of names, ${\mathcal N}$, is a symmetric binary relation 
${\mathcal S}_{\mathcal N}$ between agents such that $P\rel{S}_{\mathcal N}Q$ implies:
\begin{enumerate}
\item If $P \red P'$ then $Q \wred Q'$ and $P'\rel{S}_{\mathcal N} Q'$.
\item If $P\downarrow_{\mathcal N} x$, then $Q\Downarrow_{\mathcal N} x$.
\end{enumerate}
$P$ is ${\mathcal N}$-barbed bisimilar to $Q$, written
$P \wbbisim_{\mathcal N} Q$, if $P \rel{S}_{\mathcal N} Q$ for some ${\mathcal N}$-barbed bisimulation ${\mathcal S}_{\mathcal N}$.
\end{definition}

$\mathcal{R} \subseteq \pi \times \pi$

$P \mathcal{R} Q => \forall P'. P \red P' \Rightarrow \exists Q'. Q \red Q', P' \mathcal{R} Q'$

$P \vdash x \Rightarrow Q \vdash x$

\begin{mathpar}
  \inferrule*[lab=Out-barb]{x \nameeq y}{{y}!\langle{Q}\rangle \vdash x}
  \and
  \inferrule*[lab=Par-barb]{\mbox{$P\vdash x$ or $Q\vdash x$}}{\binpar{P}{Q} \vdash x}
\end{mathpar}

\subsubsection{Contexts}

One of the principle advantages of computational calculi like the
$\pi$-calculus is a well-defined notion of context,
contextual-equivalence and a correlation between
contextual-equivalence and notions of bisimulation. The notion of
context allows the decomposition of a process into (sub-)process and
its syntactic environment, its context. Thus, a context may be
thought of as a process with a ``hole'' (written $\Box$) in it. The
application of a context $M$ to a process $P$, written $M[P]$, is
tantamount to filling the hole in $M$ with $P$. In this paper we do
not need the full weight of this theory, but do make use of the notion
of context in the proof the main theorem. 

\begin{mathpar}
  \inferrule* [lab=summation] {} {{M_{M},M_{N}} \bc \Box \;|\; x.M_{A} \;|\; M_{M}+M_{N}}
  \and
  \inferrule* [lab=agent] {} {{M_{A}} \bc (\vec{x})M_{P} \;| \; \clift{P_0,\ldots,M_{P},\ldots,P_N}}
  \and \\
  \inferrule* [lab=process] {} {{M_{P}} \bc M_{N} \;| \;P|M_{P} }
\end{mathpar} 

\begin{mathpar}
  \inferrule* [lab=sychronization] {} {M_{N} \bc \Box \;|\; x?M_{F} \;|\; x!M_{C}}
  \and
  \inferrule* [lab=abstraction] {} {{M_{F}} \bc (x)M_{P} }
  \and
  \inferrule* [lab=concretion] {} {{M_{C}} \bc \langle M_{P} \rangle }
  \and \\
  \inferrule* [lab=process] {} {{M_{P}} \bc M_{N} \;| \;P|M_{P} }
\end{mathpar}

\begin{definition}[contextual application] Given a context $M$, and
  process $P$, we define the \emph{contextual application}, $M[P] :=
  M\{P/\Box\}$. That is, the contextual application of M to P is the
  substitution of $P$ for $\Box$ in $M$.
\end{definition}

$\meaningof{-} : L \to \mathcal{P}(\pi)$

\begin{mathpar}
  \inferrule* [lab=collection] {} {\meaningof{true} = \pi, \and \meaningof{~E} = \pi \setminus \meaningof{E}, \and \meaningof{E_{1} \& E_{2}} = \meaningof{E_{1}} \cap \meaningof{E_{2}}}
\end{mathpar}

\begin{mathpar}
  \inferrule* [lab=structure] {} {\meaningof{0} = \{ P \in \pi | P \equiv 0 \}, \and \\ \meaningof{E_1 | E_2} = \{ P \in \pi | P \equiv P_{1} | P_{2}, P_{1} \in \meaningof{E_{1}}, P_{2} \in \meaningof{E_2}\} }
\end{mathpar}

\begin{mathpar}
 \inferrule* [lab=behavior] {} {\meaningof{\langle a?b \rangle E} = \{ P \in \pi | P \equiv Q | u?(y)P', \\ \and \\\\ \and \\ \;\;\; u \in \meaningof{a}, \forall z.P'\{z/y\} \in \meaningof{E\{z/b\}}\}, \and \\ \meaningof{a!E} = \{ P \in \pi | P \equiv Q | x!\langle P' \rangle, x \in \meaningof{a} P' \in \meaningof{E}\} }
\end{mathpar}

\begin{mathpar}
 \inferrule* [lab=nominal] {} {\meaningof{\quotep{E}} = \{ \quotep{P} \in \quotep{\pi} | P \in \meaningof{E} \}, \and \meaningof{\quotep{P}} = \{ \quotep{Q} \in \quotep{\pi} | P \equiv Q \} \and \\ \meaningof{@\quotep{E}} = \{ P \in \pi | P \equiv @x, x \in \meaningof{E} \}}
\end{mathpar}

\begin{eqnarray*}
  \\
  \meaningof{-} : TS \to ST
\end{eqnarray*}

\begin{eqnarray*}
  \\
  L : TS \to ST
\end{eqnarray*}

\begin{eqnarray*}
  \\
  P \models E \iff P \in \meaningof{E}
\end{eqnarray*}

\begin{eqnarray*}
  P \approx_{L} Q \iff \forall E \in L. P \models E \iff Q \models E
\end{eqnarray*}

\begin{eqnarray*}
  P \approx_{K} Q
\end{eqnarray*}

\begin{eqnarray*}
  P \approx Q
\end{eqnarray*}

$\approx_{K} = \approx = \approx_{L}$

\subsubsection{Contextual duality}

Note that contexts extend the quotation operation to a family of
operations from processes to names. Given a context, $M$, we can
define a \emph{nominal context}, $\quotep{M}$ by $\quotep{M}[P] :=
\quotep{M[P]}$. To foreshadow what is to come we observe that these
operations enjoy a duality with processes very much like the duality
between vectors and maps from vectors to scalars.

Further, because the calculus is essentially higher-order, we have a
correspondence between contexts and processes. More specifically,
given a name $x$ and a context $M$ we can construct $M^{*}_{x}$ such
that 

\begin{mathpar}
  M^{*}_{x} | \lift{x}{P} \red M[P]
\end{mathpar}

namely,

\begin{mathpar}
  M^{*}_{x} := x?(u).M[\dropn{u}]
\end{mathpar}

The dependence of $M^{*}_{x}$ on a name makes it an abstraction, 

\begin{mathpar}
  M^{*} := (x)x?(u).M[\dropn{u}]
\end{mathpar}

\subsection{Additional notation}

It will sometimes be convenient to denote the process a name
quotes. We already have the notation $x = \quotep{P}$, but it will be
convenient to introduce an alternate notation, $\procn{x}$, when we
want to emphasize the connection to the use of the name. Note that, by
virtue of name equivalence, $\quotep{\procn{x}} \nameeq x$; so, the
notation is consistent with previous definitions.

Further, because names have structure it is possible to effect
substitutions on the basis of that structure. This means we need to
upgrade our notation for substitutions, which we accomplish by
adapting comprehension notation. Thus,

\begin{mathpar}
  P\{ y / x : x \in S \}
\end{mathpar}

is interpreted to mean the process derived from P by replacing (in a
capture-avoiding manner) each occurrence of $x$ in $S$ by $y$. For example,

\begin{mathpar}
  P\{ \quotep{\procn{x}|\procn{x}} / x : x \in \freenames{P} \}
\end{mathpar}

will replace each (occurrence) of a free name $x$ in $P$ by
$\quotep{\procn{x}|\procn{x}}$.

Also, we will avail ourselves of the notation $x^{L}$ and $x^{R}$ to
denote injections of a name into disjoint copies of the name
space. There are numerous ways to accomplish this. One example can be
found in \cite{MeredithR05}. This notation overloads to vectors of
names: $\vec{x}^{\pi} := (x_{i}^{\pi} \; : \; 0 \leq i < |\vec{x}| )$ where $\pi \in \{L,R\}$.

We also use $P^{\Box} := P|\Box$.

In \cite{MeredithR05} an interpretation of the new operator is
given. It turns out that there are several possible interpretations
all enjoying the requisite algebraic properties of the operator (see
\cite{milner91polyadicpi}). We will therefore make liberal use of
$(\nu\; \vec{x})P$.

% subsection the_syntax_and_semantics_of_the_notation_system (end)   

\input{qm2pi.qmops} 

\input{qm2pi.sterngerlach} 

\input{qm2pi.metric} 

% section concurrent_process_calculi (end)

%\input{qm2pi.proofsketch}

% section proof sketch (end)

%\input{qm2pi.slviaknots} 

% section spatial logic via knots (end)

\input{qm2pi.conclusion}

% section conclusion (end)

%\input{qm2pi.dtcodes} 

% section wiring algorithm (end)

\input{qm2pi.ack} 

% section acknowledgments (end)

\newpage


\bibliographystyle{plain}   
\bibliography{../../biblios/main.bib}

\input{qm2pi.rhodetails}

\end{document}

 

\documentclass[12pt]{llncs}
%\documentclass{jktr}

\usepackage[pdftex]{hyperref}                   
\usepackage {listings}
\usepackage {mathpartir}
\usepackage{bcprules}
%\usepackage{listings}
                       
\usepackage{graphicx} 
%\usepackage[margins=2.5cm,nohead,nofoot]{geometry}
%\usepackage{geometry}
\usepackage{amsfonts}
\usepackage{amstext}
\usepackage{latexsym}
\usepackage{amssymb}
\usepackage{color}


%\include{myPreamble}
\include{qm2pi.local} 

%\ifpdf
%\usepackage[pdftex]{graphicx}
%\else
%\usepackage{graphicx}
%\fi

 % \ifpdf
%  \usepackage{pdfsync}
%  \if


%\title{Brief Article}
%\author{David F. Snyder}
%\author{L.G. Meredith}

%\address{Dept. of Math., Texas State University--San Marcos, San Marcos, TX 78666}
       
\pagestyle{empty}


\begin{document}

\lstset{language=[Objective]Caml,frame=shadowbox}

\input{qm2pi.front}

% section front matter (end)

\input{qm2pi.intro} 
 
% section introduction (end)

% \input{qm2pi.knotations} 

% section notation (end)

\input{qm2pi.process.calculi} 

% section concurrent_process_calculi_and_spatial_logics_ (end)
    
%\input{qm2pi.knots2pi} 

%\input{qm2pi.trefoil} 

%\input{qm2pi.mainthm} 

% subsection basic_interpretation (end)

%\input{qm2pi.rho.presentation} 
\subsection{The syntax and semantics of the notation system}\label{sub:the_syntax_and_semantics_of_the_notation_system} % (fold)

We now summarize a technical presentation of the calculus that
embodies our theory of dynamics. The typical presentation of such a
calculus follows the style of giving generators and relations on
them. The grammar, below, describing term constructors, freely
generates the set of processes, $\Proc$. This set is then quotiented
by a relation known as structural congruence and it is over this set
that the notion of dynamics is expressed. This presentation is
essentially that of \cite{MeredithR05} with the addition of
polyadicity and summation. For readability we have relegated some of
the technical subtleties to an appendix.

\subsubsection{Process grammar}\label{subsub:process_grammar}

\begin{mathpar}
  \inferrule* [lab=synchronization] {} {{M} \bc \pzero \;|\; x?F \;|\; x!C }
  \and
  \inferrule* [lab=abstraction] {} {{F} \bc (x)P}
  \and
  \inferrule* [lab=concretion] {} {{C} \bc \langle Q \rangle}
  \and
  \inferrule* [lab=process] {} {{P,Q} \bc M \;| \;P|Q \;|\; @{x}}
  \and
  \inferrule* [lab=name] {} {{x} \bc \quotep{P}}
\end{mathpar} 

Note that $\vec{x}$ (resp. $\vec{P}$) denotes a vector of names
(resp. processes) of length $|\vec{x}|$ (resp. $|\vec{P}|$). We adopt
the following useful abbreviations.

\begin{mathpar}
   x?(\vec{y}).P := x.(\vec{y})P \and  x\clift{\vec{P}} := x.\clift{\vec{P}}
   \and x!(y) := \lift{x}{\dropn{y}}
   \and \Pi_{i=0}^{n-1}P_i := P_0 | \ldots | P_{n-1}
\end{mathpar}

\subsubsection{Structural congruence}

\paragraph{Free and bound names and alpha-equivalence.} At the
core of structural equivalence is alpha-equivalence which identifies
process that are the same up to a change of variable. Formally, we
recognize the distinction between free and bound names. The free names
of a process, $\freenames{P}$, may be calculated recursively as
follows:

\begin{mathpar}
\freenames{\pzero} := \emptyset
  \and \\
  \freenames{x?(y).P} := \{ x \} \cup (\freenames{P} \setminus \{ y \})
  \and 
  \freenames{x!\langle P \rangle} := \{ x \} \cup \{ P \} 
  \and \\
  \freenames{P|Q} := \freenames{P} \cup \freenames{Q}
  \and \\
  \freenames{@{x}} := \{ x \}
\end{mathpar}

$\pi$
$\quotep{\pi}$

$\freenames{-} : \pi \to \mathcal{P}(\quotep{\pi})$

\begin{eqnarray*}
  \freenames{\pzero} & := & \emptyset \\
  \freenames{x?(y).P} & := & \{ x \} \cup (\freenames{P} \setminus \{ y \}) \\
  \freenames{x!\langle P \rangle} & := & \{ x \} \cup \{ P \} \\
  \freenames{P|Q} & := & \freenames{P} \cup \freenames{Q} \\
  \freenames{\dropn{x}} & := & \{ x \}
\end{eqnarray*}

The bound names of a process, $\boundnames{P}$, are those names occurring in $P$
that are not free. For example, in $x?(y).0$, the name $x$ is free, while $y$ is bound.

\begin{mathpar}
  \inferrule* [lab=monoidal-laws] {} { P|Q \equiv Q|P \and P|0 \equiv P \and P|(Q|R) \equiv (P|Q)|R }
\end{mathpar}

\begin{mathpar}
  \inferrule* [lab=alpha-equivalence] {} { (x)P \equiv (y)P\{y/x\} \and y \not\in \freenames{P} }
\end{mathpar}

\begin{definition}
Then two processes, $P,Q$, are alpha-equivalent if $P = Q\{\vec{y}/\vec{x}\}$ for
some $\vec{x} \in \boundnames{Q},\vec{y} \in \boundnames{P}$, where $Q\{\vec{y}/\vec{x}\}$
denotes the capture-avoiding substitution of $\vec{y}$ for $\vec{x}$ in $Q$.
\end{definition}

\begin{definition}
  The {\em structural congruence} \cite{SangiorgiWalker} , $\equiv$,
  between processes is the least congruence containing
  alpha-equivalence, satisfying the abelian monoid laws
  (associativity, commutativity and $\pzero$ as identity) for parallel
  composition $|$ and for summation $+$.
\end{definition}

\subsection{Name equivalence}

We take name equivalence, written $\nameeq$, to be the smallest
equivalence relation generated by the following rules.

\begin{mathpar}
\inferrule*[lab=Quote-drop]
{ }
{ \quotep{@{x}} \nameeq x }

\inferrule*[lab=Struct-equiv]
{ P \scong Q }
{ \quotep{P} \nameeq \quotep{Q} }
\end{mathpar}

The astute reader will have noticed that the mutual recursion of names
and processes imposes a mutual recursion on alpha-equivalence and
structural equivalence via name-equivalence. Fortunately, all of this
works out pleasantly and we may calculate in the natural way, free of
concern. The reader interested in the details is referred to the
appendix \ref{appendix:rho_details}.

\subsection{Substitution}

We use $\Proc$ for the set of processes, $\QProc$ for the set of
names, and $\id{\{}\vec{y} / \vec{x} \id{\}}$ to denote partial maps,
$s : \QProc \rightarrow \QProc$. A map, $s$ lifts, uniquely, to a map
on process terms, $\widehat{s} : \Proc \rightarrow \Proc$ by the
following equations.

\begin{mathpar}
  (0) \psubstp{Q}{P} := 0 \\
  (R \juxtap S) \psubstp{Q}{P}
  :=    
  (R)\psubstp{Q}{P} \juxtap (S) \psubstp{Q}{P} \\
  (x?(y).R) \psubstp{Q}{P}    
  :=    
  (x)\substp{Q}{P} (z)\concat( (R \psubstn{z}{y}) \psubstp{Q}{P} ) \\
  (\lift{x}{R}) \psubstp{Q}{P}  
  :=
  \lift{(x)\substp{Q}{P}}{ R \psubstp{Q}{P} } \\
%   (\dropn{x})  \psubstp{Q}{P}       
%   := 
%   \left\{ 
%     \begin{array}{ccc} 
%       \dropn{\quotep{Q}} & & x \nameeq \quotep{P} \\
%       \dropn{x} & & otherwise \\
%     \end{array}
%   \right. 
  (\dropn{x})  \psubstp{Q}{P}       
  := 
  \left\{ 
    \begin{array}{ccc} 
      Q & & x \nameeq \quotep{P} \\
      \dropn{x} & & otherwise \\
    \end{array}
  \right.
\end{mathpar}
 

where

\begin{eqnarray}
  (x)\id{\{} \lpquote Q \rpquote / \lpquote P \rpquote \id{\}}            = 
  \left\{ 
    \begin{array}{ccc}
      \lpquote Q \rpquote & & x \nameeq \lpquote P \rpquote \\
      x & & otherwise \\
    \end{array}
  \right. \nonumber
\end{eqnarray}

and $z$ is chosen distinct from $\quotep{P}$, $\quotep{Q}$, the free
names in $Q$, and all the names in $R$. Our $\alpha$-equivalence will
be built in the standard way from this substitution.

\begin{remark}\label{rem:no_self_referential_names}
  One consequence of these definitions is that $\forall P. \quotep{P}
  \not\in \freenames{P}$.
\end{remark}

\subsection{ Dynamic quote: an example }

Anticipating something of what's to come, consider applying the
substitution, $\widehat{\id{\{}u / z \id{\}}}$, to the following pair
of processes, $\lift{w}{y!(z)}$ and $w[ \lpquote y!(z) \rpquote ]$.

\begin{eqnarray}
	\lift{w}{y!(z)}\widehat{\id{\{}u / z \id{\}}}
		& = &
		\lift{w}{y!(u)} \nonumber\\
	w[ \lpquote y!(z) \rpquote ] \widehat{ \id{\{}u / z \id{\}} }
		& = &
		w[ \lpquote y!(z) \rpquote ] \nonumber
\end{eqnarray}

Because the body of the process between quotes is impervious to
substitution, we get radically different answers. In fact, by
examining the first process in an input context,
e.g. $x?(z).\lift{w}{y!(z)}$, we see that the process under the lift
operator may be shaped by prefixed inputs binding a name inside it. In
this sense, the lift operator will be seen as a way to dynamically
construct processes before reifying them as names.

Finally equipped with these standard features we can present the
dynamics of the calculus.

\subsubsection{Operational semantics} 

Finally, we introduce the computational dynamics. What marks these
algebras as distinct from other more traditionally studied algebraic
structures, e.g. vector spaces or polynomial rings, is the manner in
which dynamics is captured. In traditional structures, dynamics is typically
expressed through morphisms between such structures, as in linear maps
between vector spaces or morphisms between rings. In algebras
associated with the semantics of computation, the dynamics is
expressed as part of the algebraic structure itself, through a
reduction reduction relation typically denoted by $\red$. Below, we
give a recursive presentation of this relation for the calculus used
in the encoding.

$\red \subseteq \pi \times \pi$
$\red : \pi \to \mathcal{P}(\pi)$

\begin{mathpar}
  \inferrule* [lab=Comm] { \textsf{match}( x_{src}, x_{trgt} ) } { x_{trgt}?(y)P \; | \; x_{src}!\langle {Q} \rangle \red P\{\quotep{Q}/y}\} }
  \and \\
  \inferrule* [lab=Par] {{P} \red {P}'} {{{P} | {Q}} \red {{P}' | {Q}}}
  \and
  \inferrule* [lab=Equiv]{{{P} \scong {P}'} \andalso {{P}' \red {Q}'} \andalso {{Q}' \scong {Q}}}{{P} \red {Q}}
\end{mathpar}

\begin{eqnarray*}
  match_{\equiv} (\quotep{P},\quotep{Q}) & := & P \equiv Q \\
  match_{\dagger}(\quotep{P},\quotep{Q}) & := & \forall R. P|Q \red^{*} R => R \red^{*} 0 \\
  match_{K}(\quotep{P},\quotep{Q}) & := & K \mbox{ for some context } K
\end{eqnarray*}

$u?(x)P | u!\langle Q \rangle \red P\{\quotep{Q}/x\}$

%We write $\wred$ for $\red^*$, and $P\red$ if $\exists Q $ such that $ P \red Q$.
We write $P\red$ if $\exists Q $ such that $ P \red Q$ and $P\not\red$, otherwise.

\section{Replication}

As mentioned before, it is known that replication (and hence
recursion) can be implemented in a higher-order process algebra
\cite{SangiorgiWalker}. As our first example of calculation with the
machinery thus far presented we give the construction explicitly in
the {\rhoc}.

\begin{eqnarray}
	D_{x} & := & \prefix{x}{y}{(\binpar{\outputp{x}{y}}{@{y}})} \nonumber\\
	\bangp_{x}{P} & := & \binpar{{x}!\langle{\binpar{D_{x}}{P}}\rangle}{D_{x}} \nonumber
\end{eqnarray}

\begin{eqnarray}
	\bangp_{x}{P} & & \nonumber\\
	=
	& {x}!\langle{(\prefix{x}{y}{(\outputp{x}{y} | @{y})) | P}}\rangle 
	      | \prefix{x}{y}{(\outputp{x}{y} | @{y})} & \nonumber\\
	\red
	& (\outputp{x}{y} | @{y})\substn{\quotep{(\prefix{x}{y}{(@{y} | \outputp{x}{y})) | P}}}{y} & \nonumber\\
	=
	& \outputp{x}{\quotep{(\prefix{x}{y}{(\outputp{x}{y} | @{y})) | P}}}
	  | {(\prefix{x}{y}{(\outputp{x}{y} | @{y})) | P}} & \nonumber\\
	\red
	& \ldots & \nonumber\\
	\red^*
	& P | P | \ldots & \nonumber
\end{eqnarray}

Of course, this encoding, as an implementation, runs away, unfolding
$\bangp{P}$ eagerly. A lazier and more implementable replication
operator, restricted to input-guarded processes, may be obtained as follows.

\begin{eqnarray}
\bangp{\prefix{u}{v}{P}} 
	:= 
	\binpar{\lift{x}{\prefix{u}{v}{(\binpar{D(x)}{P})}}}{D(x)} \nonumber
\end{eqnarray}

\begin{remark}
  Note that the lazier definition still does not deal with summation
  or mixed summation (i.e. sums over input and output). The reader is
  invited to construct definitions of replication that deal with these
  features. 

  Further, the definitions are parameterized in a name, $x$. Can you,
  gentle reader, make a definition that eliminates this parameter and
  guarantees no accidental interaction between the replication
  machinery and the process being replicated -- i.e. no accidental
  sharing of names used by the process to get its work done and the
  name(s) used by the replication to effect copying. This latter
  revision of the definition of replication is crucial to obtaining
  the expected identity $!!P \sim !P$.
\end{remark}

\begin{remark}\label{rem:paradoxical_combinator}
  The reader familiar with the lambda calculus will have noticed the
  similarity between $D$ and the paradoxical combinator.

  [Ed. note: the existence of this seems to suggest we have to be more
  restrictive on the set of processes and names we admit if we are to
  support no-cloning.]
\end{remark}

\subsubsection{Bisimulation}

The computational dynamics gives rise to another kind of equivalence,
the equivalence of computational behavior. As previously mentioned
this is typically captured \emph{via} some form of bisimulation.

% The notion we use in this paper is weak barbed bisimulation
% \cite{milner91polyadicpi}.

The notion we use in this paper is derived from weak barbed
bisimulation \cite{milner91polyadicpi}. 

\begin{definition}
An \emph{observation relation}, $\downarrow_{\mathcal N}$, over a set
of names, $\mathcal N$, is the smallest relation satisfying the rules
below.

\infrule[Out-barb]{y \in {\mathcal N}, \; x \nameeq y}
		  {\outputp{x}{v} \downarrow_{\mathcal N} x}
\infrule[Par-barb]{\mbox{$P\downarrow_{\mathcal N} x$ or $Q\downarrow_{\mathcal N} x$}}
		  {\binpar{P}{Q} \downarrow_{\mathcal N} x}

We write $P \Downarrow_{\mathcal N} x$ if there is $Q$ such that 
$P \wred Q$ and $Q \downarrow_{\mathcal N} x$.
\end{definition}

\begin{definition}
%\label{def.bbisim}
An  ${\mathcal N}$-\emph{barbed bisimulation} over a set of names, ${\mathcal N}$, is a symmetric binary relation 
${\mathcal S}_{\mathcal N}$ between agents such that $P\rel{S}_{\mathcal N}Q$ implies:
\begin{enumerate}
\item If $P \red P'$ then $Q \wred Q'$ and $P'\rel{S}_{\mathcal N} Q'$.
\item If $P\downarrow_{\mathcal N} x$, then $Q\Downarrow_{\mathcal N} x$.
\end{enumerate}
$P$ is ${\mathcal N}$-barbed bisimilar to $Q$, written
$P \wbbisim_{\mathcal N} Q$, if $P \rel{S}_{\mathcal N} Q$ for some ${\mathcal N}$-barbed bisimulation ${\mathcal S}_{\mathcal N}$.
\end{definition}

$\mathcal{R} \subseteq \pi \times \pi$

$P \mathcal{R} Q => \forall P'. P \red P' \Rightarrow \exists Q'. Q \red Q', P' \mathcal{R} Q'$

$P \vdash x \Rightarrow Q \vdash x$

\begin{mathpar}
  \inferrule*[lab=Out-barb]{x \nameeq y}{{y}!\langle{Q}\rangle \vdash x}
  \and
  \inferrule*[lab=Par-barb]{\mbox{$P\vdash x$ or $Q\vdash x$}}{\binpar{P}{Q} \vdash x}
\end{mathpar}

\subsubsection{Contexts}

One of the principle advantages of computational calculi like the
$\pi$-calculus is a well-defined notion of context,
contextual-equivalence and a correlation between
contextual-equivalence and notions of bisimulation. The notion of
context allows the decomposition of a process into (sub-)process and
its syntactic environment, its context. Thus, a context may be
thought of as a process with a ``hole'' (written $\Box$) in it. The
application of a context $M$ to a process $P$, written $M[P]$, is
tantamount to filling the hole in $M$ with $P$. In this paper we do
not need the full weight of this theory, but do make use of the notion
of context in the proof the main theorem. 

\begin{mathpar}
  \inferrule* [lab=summation] {} {{M_{M},M_{N}} \bc \Box \;|\; x.M_{A} \;|\; M_{M}+M_{N}}
  \and
  \inferrule* [lab=agent] {} {{M_{A}} \bc (\vec{x})M_{P} \;| \; \clift{P_0,\ldots,M_{P},\ldots,P_N}}
  \and \\
  \inferrule* [lab=process] {} {{M_{P}} \bc M_{N} \;| \;P|M_{P} }
\end{mathpar} 

\begin{mathpar}
  \inferrule* [lab=sychronization] {} {M_{N} \bc \Box \;|\; x?M_{F} \;|\; x!M_{C}}
  \and
  \inferrule* [lab=abstraction] {} {{M_{F}} \bc (x)M_{P} }
  \and
  \inferrule* [lab=concretion] {} {{M_{C}} \bc \langle M_{P} \rangle }
  \and \\
  \inferrule* [lab=process] {} {{M_{P}} \bc M_{N} \;| \;P|M_{P} }
\end{mathpar}

\begin{definition}[contextual application] Given a context $M$, and
  process $P$, we define the \emph{contextual application}, $M[P] :=
  M\{P/\Box\}$. That is, the contextual application of M to P is the
  substitution of $P$ for $\Box$ in $M$.
\end{definition}

$\meaningof{-} : L \to \mathcal{P}(\pi)$

\begin{mathpar}
  \inferrule* [lab=collection] {} {\meaningof{true} = \pi, \and \meaningof{~E} = \pi \setminus \meaningof{E}, \and \meaningof{E_{1} \& E_{2}} = \meaningof{E_{1}} \cap \meaningof{E_{2}}}
\end{mathpar}

\begin{mathpar}
  \inferrule* [lab=structure] {} {\meaningof{0} = \{ P \in \pi | P \equiv 0 \}, \and \\ \meaningof{E_1 | E_2} = \{ P \in \pi | P \equiv P_{1} | P_{2}, P_{1} \in \meaningof{E_{1}}, P_{2} \in \meaningof{E_2}\} }
\end{mathpar}

\begin{mathpar}
 \inferrule* [lab=behavior] {} {\meaningof{\langle a?b \rangle E} = \{ P \in \pi | P \equiv Q | u?(y)P', \\ \and \\\\ \and \\ \;\;\; u \in \meaningof{a}, \forall z.P'\{z/y\} \in \meaningof{E\{z/b\}}\}, \and \\ \meaningof{a!E} = \{ P \in \pi | P \equiv Q | x!\langle P' \rangle, x \in \meaningof{a} P' \in \meaningof{E}\} }
\end{mathpar}

\begin{mathpar}
 \inferrule* [lab=nominal] {} {\meaningof{\quotep{E}} = \{ \quotep{P} \in \quotep{\pi} | P \in \meaningof{E} \}, \and \meaningof{\quotep{P}} = \{ \quotep{Q} \in \quotep{\pi} | P \equiv Q \} \and \\ \meaningof{@\quotep{E}} = \{ P \in \pi | P \equiv @x, x \in \meaningof{E} \}}
\end{mathpar}

\begin{eqnarray*}
  \\
  \meaningof{-} : TS \to ST
\end{eqnarray*}

\begin{eqnarray*}
  \\
  L : TS \to ST
\end{eqnarray*}

\begin{eqnarray*}
  \\
  P \models E \iff P \in \meaningof{E}
\end{eqnarray*}

\begin{eqnarray*}
  P \approx_{L} Q \iff \forall E \in L. P \models E \iff Q \models E
\end{eqnarray*}

\begin{eqnarray*}
  P \approx_{K} Q
\end{eqnarray*}

\begin{eqnarray*}
  P \approx Q
\end{eqnarray*}

$\approx_{K} = \approx = \approx_{L}$

\subsubsection{Contextual duality}

Note that contexts extend the quotation operation to a family of
operations from processes to names. Given a context, $M$, we can
define a \emph{nominal context}, $\quotep{M}$ by $\quotep{M}[P] :=
\quotep{M[P]}$. To foreshadow what is to come we observe that these
operations enjoy a duality with processes very much like the duality
between vectors and maps from vectors to scalars.

Further, because the calculus is essentially higher-order, we have a
correspondence between contexts and processes. More specifically,
given a name $x$ and a context $M$ we can construct $M^{*}_{x}$ such
that 

\begin{mathpar}
  M^{*}_{x} | \lift{x}{P} \red M[P]
\end{mathpar}

namely,

\begin{mathpar}
  M^{*}_{x} := x?(u).M[\dropn{u}]
\end{mathpar}

The dependence of $M^{*}_{x}$ on a name makes it an abstraction, 

\begin{mathpar}
  M^{*} := (x)x?(u).M[\dropn{u}]
\end{mathpar}

\subsection{Additional notation}

It will sometimes be convenient to denote the process a name
quotes. We already have the notation $x = \quotep{P}$, but it will be
convenient to introduce an alternate notation, $\procn{x}$, when we
want to emphasize the connection to the use of the name. Note that, by
virtue of name equivalence, $\quotep{\procn{x}} \nameeq x$; so, the
notation is consistent with previous definitions.

Further, because names have structure it is possible to effect
substitutions on the basis of that structure. This means we need to
upgrade our notation for substitutions, which we accomplish by
adapting comprehension notation. Thus,

\begin{mathpar}
  P\{ y / x : x \in S \}
\end{mathpar}

is interpreted to mean the process derived from P by replacing (in a
capture-avoiding manner) each occurrence of $x$ in $S$ by $y$. For example,

\begin{mathpar}
  P\{ \quotep{\procn{x}|\procn{x}} / x : x \in \freenames{P} \}
\end{mathpar}

will replace each (occurrence) of a free name $x$ in $P$ by
$\quotep{\procn{x}|\procn{x}}$.

Also, we will avail ourselves of the notation $x^{L}$ and $x^{R}$ to
denote injections of a name into disjoint copies of the name
space. There are numerous ways to accomplish this. One example can be
found in \cite{MeredithR05}. This notation overloads to vectors of
names: $\vec{x}^{\pi} := (x_{i}^{\pi} \; : \; 0 \leq i < |\vec{x}| )$ where $\pi \in \{L,R\}$.

We also use $P^{\Box} := P|\Box$.

In \cite{MeredithR05} an interpretation of the new operator is
given. It turns out that there are several possible interpretations
all enjoying the requisite algebraic properties of the operator (see
\cite{milner91polyadicpi}). We will therefore make liberal use of
$(\nu\; \vec{x})P$.

% subsection the_syntax_and_semantics_of_the_notation_system (end)   

\input{qm2pi.qmops} 

\input{qm2pi.sterngerlach} 

\input{qm2pi.metric} 

% section concurrent_process_calculi (end)

%\input{qm2pi.proofsketch}

% section proof sketch (end)

%\input{qm2pi.slviaknots} 

% section spatial logic via knots (end)

\input{qm2pi.conclusion}

% section conclusion (end)

%\input{qm2pi.dtcodes} 

% section wiring algorithm (end)

\input{qm2pi.ack} 

% section acknowledgments (end)

\newpage


\bibliographystyle{plain}   
\bibliography{../../biblios/main.bib}

\input{qm2pi.rhodetails}

\end{document}

 

% section concurrent_process_calculi (end)

%\documentclass[12pt]{llncs}
%\documentclass{jktr}

\usepackage[pdftex]{hyperref}                   
\usepackage {listings}
\usepackage {mathpartir}
\usepackage{bcprules}
%\usepackage{listings}
                       
\usepackage{graphicx} 
%\usepackage[margins=2.5cm,nohead,nofoot]{geometry}
%\usepackage{geometry}
\usepackage{amsfonts}
\usepackage{amstext}
\usepackage{latexsym}
\usepackage{amssymb}
\usepackage{color}


%\include{myPreamble}
\include{qm2pi.local} 

%\ifpdf
%\usepackage[pdftex]{graphicx}
%\else
%\usepackage{graphicx}
%\fi

 % \ifpdf
%  \usepackage{pdfsync}
%  \if


%\title{Brief Article}
%\author{David F. Snyder}
%\author{L.G. Meredith}

%\address{Dept. of Math., Texas State University--San Marcos, San Marcos, TX 78666}
       
\pagestyle{empty}


\begin{document}

\lstset{language=[Objective]Caml,frame=shadowbox}

\input{qm2pi.front}

% section front matter (end)

\input{qm2pi.intro} 
 
% section introduction (end)

% \input{qm2pi.knotations} 

% section notation (end)

\input{qm2pi.process.calculi} 

% section concurrent_process_calculi_and_spatial_logics_ (end)
    
%\input{qm2pi.knots2pi} 

%\input{qm2pi.trefoil} 

%\input{qm2pi.mainthm} 

% subsection basic_interpretation (end)

%\input{qm2pi.rho.presentation} 
\subsection{The syntax and semantics of the notation system}\label{sub:the_syntax_and_semantics_of_the_notation_system} % (fold)

We now summarize a technical presentation of the calculus that
embodies our theory of dynamics. The typical presentation of such a
calculus follows the style of giving generators and relations on
them. The grammar, below, describing term constructors, freely
generates the set of processes, $\Proc$. This set is then quotiented
by a relation known as structural congruence and it is over this set
that the notion of dynamics is expressed. This presentation is
essentially that of \cite{MeredithR05} with the addition of
polyadicity and summation. For readability we have relegated some of
the technical subtleties to an appendix.

\subsubsection{Process grammar}\label{subsub:process_grammar}

\begin{mathpar}
  \inferrule* [lab=synchronization] {} {{M} \bc \pzero \;|\; x?F \;|\; x!C }
  \and
  \inferrule* [lab=abstraction] {} {{F} \bc (x)P}
  \and
  \inferrule* [lab=concretion] {} {{C} \bc \langle Q \rangle}
  \and
  \inferrule* [lab=process] {} {{P,Q} \bc M \;| \;P|Q \;|\; @{x}}
  \and
  \inferrule* [lab=name] {} {{x} \bc \quotep{P}}
\end{mathpar} 

Note that $\vec{x}$ (resp. $\vec{P}$) denotes a vector of names
(resp. processes) of length $|\vec{x}|$ (resp. $|\vec{P}|$). We adopt
the following useful abbreviations.

\begin{mathpar}
   x?(\vec{y}).P := x.(\vec{y})P \and  x\clift{\vec{P}} := x.\clift{\vec{P}}
   \and x!(y) := \lift{x}{\dropn{y}}
   \and \Pi_{i=0}^{n-1}P_i := P_0 | \ldots | P_{n-1}
\end{mathpar}

\subsubsection{Structural congruence}

\paragraph{Free and bound names and alpha-equivalence.} At the
core of structural equivalence is alpha-equivalence which identifies
process that are the same up to a change of variable. Formally, we
recognize the distinction between free and bound names. The free names
of a process, $\freenames{P}$, may be calculated recursively as
follows:

\begin{mathpar}
\freenames{\pzero} := \emptyset
  \and \\
  \freenames{x?(y).P} := \{ x \} \cup (\freenames{P} \setminus \{ y \})
  \and 
  \freenames{x!\langle P \rangle} := \{ x \} \cup \{ P \} 
  \and \\
  \freenames{P|Q} := \freenames{P} \cup \freenames{Q}
  \and \\
  \freenames{@{x}} := \{ x \}
\end{mathpar}

$\pi$
$\quotep{\pi}$

$\freenames{-} : \pi \to \mathcal{P}(\quotep{\pi})$

\begin{eqnarray*}
  \freenames{\pzero} & := & \emptyset \\
  \freenames{x?(y).P} & := & \{ x \} \cup (\freenames{P} \setminus \{ y \}) \\
  \freenames{x!\langle P \rangle} & := & \{ x \} \cup \{ P \} \\
  \freenames{P|Q} & := & \freenames{P} \cup \freenames{Q} \\
  \freenames{\dropn{x}} & := & \{ x \}
\end{eqnarray*}

The bound names of a process, $\boundnames{P}$, are those names occurring in $P$
that are not free. For example, in $x?(y).0$, the name $x$ is free, while $y$ is bound.

\begin{mathpar}
  \inferrule* [lab=monoidal-laws] {} { P|Q \equiv Q|P \and P|0 \equiv P \and P|(Q|R) \equiv (P|Q)|R }
\end{mathpar}

\begin{mathpar}
  \inferrule* [lab=alpha-equivalence] {} { (x)P \equiv (y)P\{y/x\} \and y \not\in \freenames{P} }
\end{mathpar}

\begin{definition}
Then two processes, $P,Q$, are alpha-equivalent if $P = Q\{\vec{y}/\vec{x}\}$ for
some $\vec{x} \in \boundnames{Q},\vec{y} \in \boundnames{P}$, where $Q\{\vec{y}/\vec{x}\}$
denotes the capture-avoiding substitution of $\vec{y}$ for $\vec{x}$ in $Q$.
\end{definition}

\begin{definition}
  The {\em structural congruence} \cite{SangiorgiWalker} , $\equiv$,
  between processes is the least congruence containing
  alpha-equivalence, satisfying the abelian monoid laws
  (associativity, commutativity and $\pzero$ as identity) for parallel
  composition $|$ and for summation $+$.
\end{definition}

\subsection{Name equivalence}

We take name equivalence, written $\nameeq$, to be the smallest
equivalence relation generated by the following rules.

\begin{mathpar}
\inferrule*[lab=Quote-drop]
{ }
{ \quotep{@{x}} \nameeq x }

\inferrule*[lab=Struct-equiv]
{ P \scong Q }
{ \quotep{P} \nameeq \quotep{Q} }
\end{mathpar}

The astute reader will have noticed that the mutual recursion of names
and processes imposes a mutual recursion on alpha-equivalence and
structural equivalence via name-equivalence. Fortunately, all of this
works out pleasantly and we may calculate in the natural way, free of
concern. The reader interested in the details is referred to the
appendix \ref{appendix:rho_details}.

\subsection{Substitution}

We use $\Proc$ for the set of processes, $\QProc$ for the set of
names, and $\id{\{}\vec{y} / \vec{x} \id{\}}$ to denote partial maps,
$s : \QProc \rightarrow \QProc$. A map, $s$ lifts, uniquely, to a map
on process terms, $\widehat{s} : \Proc \rightarrow \Proc$ by the
following equations.

\begin{mathpar}
  (0) \psubstp{Q}{P} := 0 \\
  (R \juxtap S) \psubstp{Q}{P}
  :=    
  (R)\psubstp{Q}{P} \juxtap (S) \psubstp{Q}{P} \\
  (x?(y).R) \psubstp{Q}{P}    
  :=    
  (x)\substp{Q}{P} (z)\concat( (R \psubstn{z}{y}) \psubstp{Q}{P} ) \\
  (\lift{x}{R}) \psubstp{Q}{P}  
  :=
  \lift{(x)\substp{Q}{P}}{ R \psubstp{Q}{P} } \\
%   (\dropn{x})  \psubstp{Q}{P}       
%   := 
%   \left\{ 
%     \begin{array}{ccc} 
%       \dropn{\quotep{Q}} & & x \nameeq \quotep{P} \\
%       \dropn{x} & & otherwise \\
%     \end{array}
%   \right. 
  (\dropn{x})  \psubstp{Q}{P}       
  := 
  \left\{ 
    \begin{array}{ccc} 
      Q & & x \nameeq \quotep{P} \\
      \dropn{x} & & otherwise \\
    \end{array}
  \right.
\end{mathpar}
 

where

\begin{eqnarray}
  (x)\id{\{} \lpquote Q \rpquote / \lpquote P \rpquote \id{\}}            = 
  \left\{ 
    \begin{array}{ccc}
      \lpquote Q \rpquote & & x \nameeq \lpquote P \rpquote \\
      x & & otherwise \\
    \end{array}
  \right. \nonumber
\end{eqnarray}

and $z$ is chosen distinct from $\quotep{P}$, $\quotep{Q}$, the free
names in $Q$, and all the names in $R$. Our $\alpha$-equivalence will
be built in the standard way from this substitution.

\begin{remark}\label{rem:no_self_referential_names}
  One consequence of these definitions is that $\forall P. \quotep{P}
  \not\in \freenames{P}$.
\end{remark}

\subsection{ Dynamic quote: an example }

Anticipating something of what's to come, consider applying the
substitution, $\widehat{\id{\{}u / z \id{\}}}$, to the following pair
of processes, $\lift{w}{y!(z)}$ and $w[ \lpquote y!(z) \rpquote ]$.

\begin{eqnarray}
	\lift{w}{y!(z)}\widehat{\id{\{}u / z \id{\}}}
		& = &
		\lift{w}{y!(u)} \nonumber\\
	w[ \lpquote y!(z) \rpquote ] \widehat{ \id{\{}u / z \id{\}} }
		& = &
		w[ \lpquote y!(z) \rpquote ] \nonumber
\end{eqnarray}

Because the body of the process between quotes is impervious to
substitution, we get radically different answers. In fact, by
examining the first process in an input context,
e.g. $x?(z).\lift{w}{y!(z)}$, we see that the process under the lift
operator may be shaped by prefixed inputs binding a name inside it. In
this sense, the lift operator will be seen as a way to dynamically
construct processes before reifying them as names.

Finally equipped with these standard features we can present the
dynamics of the calculus.

\subsubsection{Operational semantics} 

Finally, we introduce the computational dynamics. What marks these
algebras as distinct from other more traditionally studied algebraic
structures, e.g. vector spaces or polynomial rings, is the manner in
which dynamics is captured. In traditional structures, dynamics is typically
expressed through morphisms between such structures, as in linear maps
between vector spaces or morphisms between rings. In algebras
associated with the semantics of computation, the dynamics is
expressed as part of the algebraic structure itself, through a
reduction reduction relation typically denoted by $\red$. Below, we
give a recursive presentation of this relation for the calculus used
in the encoding.

$\red \subseteq \pi \times \pi$
$\red : \pi \to \mathcal{P}(\pi)$

\begin{mathpar}
  \inferrule* [lab=Comm] { \textsf{match}( x_{src}, x_{trgt} ) } { x_{trgt}?(y)P \; | \; x_{src}!\langle {Q} \rangle \red P\{\quotep{Q}/y}\} }
  \and \\
  \inferrule* [lab=Par] {{P} \red {P}'} {{{P} | {Q}} \red {{P}' | {Q}}}
  \and
  \inferrule* [lab=Equiv]{{{P} \scong {P}'} \andalso {{P}' \red {Q}'} \andalso {{Q}' \scong {Q}}}{{P} \red {Q}}
\end{mathpar}

\begin{eqnarray*}
  match_{\equiv} (\quotep{P},\quotep{Q}) & := & P \equiv Q \\
  match_{\dagger}(\quotep{P},\quotep{Q}) & := & \forall R. P|Q \red^{*} R => R \red^{*} 0 \\
  match_{K}(\quotep{P},\quotep{Q}) & := & K \mbox{ for some context } K
\end{eqnarray*}

$u?(x)P | u!\langle Q \rangle \red P\{\quotep{Q}/x\}$

%We write $\wred$ for $\red^*$, and $P\red$ if $\exists Q $ such that $ P \red Q$.
We write $P\red$ if $\exists Q $ such that $ P \red Q$ and $P\not\red$, otherwise.

\section{Replication}

As mentioned before, it is known that replication (and hence
recursion) can be implemented in a higher-order process algebra
\cite{SangiorgiWalker}. As our first example of calculation with the
machinery thus far presented we give the construction explicitly in
the {\rhoc}.

\begin{eqnarray}
	D_{x} & := & \prefix{x}{y}{(\binpar{\outputp{x}{y}}{@{y}})} \nonumber\\
	\bangp_{x}{P} & := & \binpar{{x}!\langle{\binpar{D_{x}}{P}}\rangle}{D_{x}} \nonumber
\end{eqnarray}

\begin{eqnarray}
	\bangp_{x}{P} & & \nonumber\\
	=
	& {x}!\langle{(\prefix{x}{y}{(\outputp{x}{y} | @{y})) | P}}\rangle 
	      | \prefix{x}{y}{(\outputp{x}{y} | @{y})} & \nonumber\\
	\red
	& (\outputp{x}{y} | @{y})\substn{\quotep{(\prefix{x}{y}{(@{y} | \outputp{x}{y})) | P}}}{y} & \nonumber\\
	=
	& \outputp{x}{\quotep{(\prefix{x}{y}{(\outputp{x}{y} | @{y})) | P}}}
	  | {(\prefix{x}{y}{(\outputp{x}{y} | @{y})) | P}} & \nonumber\\
	\red
	& \ldots & \nonumber\\
	\red^*
	& P | P | \ldots & \nonumber
\end{eqnarray}

Of course, this encoding, as an implementation, runs away, unfolding
$\bangp{P}$ eagerly. A lazier and more implementable replication
operator, restricted to input-guarded processes, may be obtained as follows.

\begin{eqnarray}
\bangp{\prefix{u}{v}{P}} 
	:= 
	\binpar{\lift{x}{\prefix{u}{v}{(\binpar{D(x)}{P})}}}{D(x)} \nonumber
\end{eqnarray}

\begin{remark}
  Note that the lazier definition still does not deal with summation
  or mixed summation (i.e. sums over input and output). The reader is
  invited to construct definitions of replication that deal with these
  features. 

  Further, the definitions are parameterized in a name, $x$. Can you,
  gentle reader, make a definition that eliminates this parameter and
  guarantees no accidental interaction between the replication
  machinery and the process being replicated -- i.e. no accidental
  sharing of names used by the process to get its work done and the
  name(s) used by the replication to effect copying. This latter
  revision of the definition of replication is crucial to obtaining
  the expected identity $!!P \sim !P$.
\end{remark}

\begin{remark}\label{rem:paradoxical_combinator}
  The reader familiar with the lambda calculus will have noticed the
  similarity between $D$ and the paradoxical combinator.

  [Ed. note: the existence of this seems to suggest we have to be more
  restrictive on the set of processes and names we admit if we are to
  support no-cloning.]
\end{remark}

\subsubsection{Bisimulation}

The computational dynamics gives rise to another kind of equivalence,
the equivalence of computational behavior. As previously mentioned
this is typically captured \emph{via} some form of bisimulation.

% The notion we use in this paper is weak barbed bisimulation
% \cite{milner91polyadicpi}.

The notion we use in this paper is derived from weak barbed
bisimulation \cite{milner91polyadicpi}. 

\begin{definition}
An \emph{observation relation}, $\downarrow_{\mathcal N}$, over a set
of names, $\mathcal N$, is the smallest relation satisfying the rules
below.

\infrule[Out-barb]{y \in {\mathcal N}, \; x \nameeq y}
		  {\outputp{x}{v} \downarrow_{\mathcal N} x}
\infrule[Par-barb]{\mbox{$P\downarrow_{\mathcal N} x$ or $Q\downarrow_{\mathcal N} x$}}
		  {\binpar{P}{Q} \downarrow_{\mathcal N} x}

We write $P \Downarrow_{\mathcal N} x$ if there is $Q$ such that 
$P \wred Q$ and $Q \downarrow_{\mathcal N} x$.
\end{definition}

\begin{definition}
%\label{def.bbisim}
An  ${\mathcal N}$-\emph{barbed bisimulation} over a set of names, ${\mathcal N}$, is a symmetric binary relation 
${\mathcal S}_{\mathcal N}$ between agents such that $P\rel{S}_{\mathcal N}Q$ implies:
\begin{enumerate}
\item If $P \red P'$ then $Q \wred Q'$ and $P'\rel{S}_{\mathcal N} Q'$.
\item If $P\downarrow_{\mathcal N} x$, then $Q\Downarrow_{\mathcal N} x$.
\end{enumerate}
$P$ is ${\mathcal N}$-barbed bisimilar to $Q$, written
$P \wbbisim_{\mathcal N} Q$, if $P \rel{S}_{\mathcal N} Q$ for some ${\mathcal N}$-barbed bisimulation ${\mathcal S}_{\mathcal N}$.
\end{definition}

$\mathcal{R} \subseteq \pi \times \pi$

$P \mathcal{R} Q => \forall P'. P \red P' \Rightarrow \exists Q'. Q \red Q', P' \mathcal{R} Q'$

$P \vdash x \Rightarrow Q \vdash x$

\begin{mathpar}
  \inferrule*[lab=Out-barb]{x \nameeq y}{{y}!\langle{Q}\rangle \vdash x}
  \and
  \inferrule*[lab=Par-barb]{\mbox{$P\vdash x$ or $Q\vdash x$}}{\binpar{P}{Q} \vdash x}
\end{mathpar}

\subsubsection{Contexts}

One of the principle advantages of computational calculi like the
$\pi$-calculus is a well-defined notion of context,
contextual-equivalence and a correlation between
contextual-equivalence and notions of bisimulation. The notion of
context allows the decomposition of a process into (sub-)process and
its syntactic environment, its context. Thus, a context may be
thought of as a process with a ``hole'' (written $\Box$) in it. The
application of a context $M$ to a process $P$, written $M[P]$, is
tantamount to filling the hole in $M$ with $P$. In this paper we do
not need the full weight of this theory, but do make use of the notion
of context in the proof the main theorem. 

\begin{mathpar}
  \inferrule* [lab=summation] {} {{M_{M},M_{N}} \bc \Box \;|\; x.M_{A} \;|\; M_{M}+M_{N}}
  \and
  \inferrule* [lab=agent] {} {{M_{A}} \bc (\vec{x})M_{P} \;| \; \clift{P_0,\ldots,M_{P},\ldots,P_N}}
  \and \\
  \inferrule* [lab=process] {} {{M_{P}} \bc M_{N} \;| \;P|M_{P} }
\end{mathpar} 

\begin{mathpar}
  \inferrule* [lab=sychronization] {} {M_{N} \bc \Box \;|\; x?M_{F} \;|\; x!M_{C}}
  \and
  \inferrule* [lab=abstraction] {} {{M_{F}} \bc (x)M_{P} }
  \and
  \inferrule* [lab=concretion] {} {{M_{C}} \bc \langle M_{P} \rangle }
  \and \\
  \inferrule* [lab=process] {} {{M_{P}} \bc M_{N} \;| \;P|M_{P} }
\end{mathpar}

\begin{definition}[contextual application] Given a context $M$, and
  process $P$, we define the \emph{contextual application}, $M[P] :=
  M\{P/\Box\}$. That is, the contextual application of M to P is the
  substitution of $P$ for $\Box$ in $M$.
\end{definition}

$\meaningof{-} : L \to \mathcal{P}(\pi)$

\begin{mathpar}
  \inferrule* [lab=collection] {} {\meaningof{true} = \pi, \and \meaningof{~E} = \pi \setminus \meaningof{E}, \and \meaningof{E_{1} \& E_{2}} = \meaningof{E_{1}} \cap \meaningof{E_{2}}}
\end{mathpar}

\begin{mathpar}
  \inferrule* [lab=structure] {} {\meaningof{0} = \{ P \in \pi | P \equiv 0 \}, \and \\ \meaningof{E_1 | E_2} = \{ P \in \pi | P \equiv P_{1} | P_{2}, P_{1} \in \meaningof{E_{1}}, P_{2} \in \meaningof{E_2}\} }
\end{mathpar}

\begin{mathpar}
 \inferrule* [lab=behavior] {} {\meaningof{\langle a?b \rangle E} = \{ P \in \pi | P \equiv Q | u?(y)P', \\ \and \\\\ \and \\ \;\;\; u \in \meaningof{a}, \forall z.P'\{z/y\} \in \meaningof{E\{z/b\}}\}, \and \\ \meaningof{a!E} = \{ P \in \pi | P \equiv Q | x!\langle P' \rangle, x \in \meaningof{a} P' \in \meaningof{E}\} }
\end{mathpar}

\begin{mathpar}
 \inferrule* [lab=nominal] {} {\meaningof{\quotep{E}} = \{ \quotep{P} \in \quotep{\pi} | P \in \meaningof{E} \}, \and \meaningof{\quotep{P}} = \{ \quotep{Q} \in \quotep{\pi} | P \equiv Q \} \and \\ \meaningof{@\quotep{E}} = \{ P \in \pi | P \equiv @x, x \in \meaningof{E} \}}
\end{mathpar}

\begin{eqnarray*}
  \\
  \meaningof{-} : TS \to ST
\end{eqnarray*}

\begin{eqnarray*}
  \\
  L : TS \to ST
\end{eqnarray*}

\begin{eqnarray*}
  \\
  P \models E \iff P \in \meaningof{E}
\end{eqnarray*}

\begin{eqnarray*}
  P \approx_{L} Q \iff \forall E \in L. P \models E \iff Q \models E
\end{eqnarray*}

\begin{eqnarray*}
  P \approx_{K} Q
\end{eqnarray*}

\begin{eqnarray*}
  P \approx Q
\end{eqnarray*}

$\approx_{K} = \approx = \approx_{L}$

\subsubsection{Contextual duality}

Note that contexts extend the quotation operation to a family of
operations from processes to names. Given a context, $M$, we can
define a \emph{nominal context}, $\quotep{M}$ by $\quotep{M}[P] :=
\quotep{M[P]}$. To foreshadow what is to come we observe that these
operations enjoy a duality with processes very much like the duality
between vectors and maps from vectors to scalars.

Further, because the calculus is essentially higher-order, we have a
correspondence between contexts and processes. More specifically,
given a name $x$ and a context $M$ we can construct $M^{*}_{x}$ such
that 

\begin{mathpar}
  M^{*}_{x} | \lift{x}{P} \red M[P]
\end{mathpar}

namely,

\begin{mathpar}
  M^{*}_{x} := x?(u).M[\dropn{u}]
\end{mathpar}

The dependence of $M^{*}_{x}$ on a name makes it an abstraction, 

\begin{mathpar}
  M^{*} := (x)x?(u).M[\dropn{u}]
\end{mathpar}

\subsection{Additional notation}

It will sometimes be convenient to denote the process a name
quotes. We already have the notation $x = \quotep{P}$, but it will be
convenient to introduce an alternate notation, $\procn{x}$, when we
want to emphasize the connection to the use of the name. Note that, by
virtue of name equivalence, $\quotep{\procn{x}} \nameeq x$; so, the
notation is consistent with previous definitions.

Further, because names have structure it is possible to effect
substitutions on the basis of that structure. This means we need to
upgrade our notation for substitutions, which we accomplish by
adapting comprehension notation. Thus,

\begin{mathpar}
  P\{ y / x : x \in S \}
\end{mathpar}

is interpreted to mean the process derived from P by replacing (in a
capture-avoiding manner) each occurrence of $x$ in $S$ by $y$. For example,

\begin{mathpar}
  P\{ \quotep{\procn{x}|\procn{x}} / x : x \in \freenames{P} \}
\end{mathpar}

will replace each (occurrence) of a free name $x$ in $P$ by
$\quotep{\procn{x}|\procn{x}}$.

Also, we will avail ourselves of the notation $x^{L}$ and $x^{R}$ to
denote injections of a name into disjoint copies of the name
space. There are numerous ways to accomplish this. One example can be
found in \cite{MeredithR05}. This notation overloads to vectors of
names: $\vec{x}^{\pi} := (x_{i}^{\pi} \; : \; 0 \leq i < |\vec{x}| )$ where $\pi \in \{L,R\}$.

We also use $P^{\Box} := P|\Box$.

In \cite{MeredithR05} an interpretation of the new operator is
given. It turns out that there are several possible interpretations
all enjoying the requisite algebraic properties of the operator (see
\cite{milner91polyadicpi}). We will therefore make liberal use of
$(\nu\; \vec{x})P$.

% subsection the_syntax_and_semantics_of_the_notation_system (end)   

\input{qm2pi.qmops} 

\input{qm2pi.sterngerlach} 

\input{qm2pi.metric} 

% section concurrent_process_calculi (end)

%\input{qm2pi.proofsketch}

% section proof sketch (end)

%\input{qm2pi.slviaknots} 

% section spatial logic via knots (end)

\input{qm2pi.conclusion}

% section conclusion (end)

%\input{qm2pi.dtcodes} 

% section wiring algorithm (end)

\input{qm2pi.ack} 

% section acknowledgments (end)

\newpage


\bibliographystyle{plain}   
\bibliography{../../biblios/main.bib}

\input{qm2pi.rhodetails}

\end{document}



% section proof sketch (end)

%\section{Unlikely characters: spatial logic for
  knots}\label{sub:characteristic_formulae} % (fold)

Associated to the mobile process calculi are a family of logics known
as the Hennessy-Milner logics. These logics typically enjoy a
semantics interpreting formulae as sets of processes that when
factored through the encoding outlined above allows an identification
of classes of knots with logical formulae. In the context of this
encoding the sub-family known as the spatial logics \cite{CairesC03}
\cite{CairesC04} \cite{Caires04} are of particular interest providing
several important features for expressing and reasoning about
properties (i.e. classes) of knots. We hint here at how this may be done.

%\begin{description}
%\item [structural connectives] 
\subsubsection{Structural connectives} The spatial logics enjoy
structural connectives corresponding, at the logical level, to the
parallel composition ($P | Q$) and new name ($(\nu \; x)P$)
connectives for processes. As illustrated in the examples below, these
connectives are extremely expressive given the shape of our encoding.
%\item [decideable satisfaction]

\subsubsection{Decideable satisfaction}
In \cite{Caires04} the satisfaction relation is shown to be decideable
for a rich class of processes. It further turns out that the image of
the our encoding is a proper subset of that class. This result
provides the basis for an algorithm by which to search for knots
enjoying a given property.
%\item [characteristic formulae]

\subsubsection{Characteristic formulae}
In the same paper \cite{Caires04} , Caires presents a means of calculating
characteristic formulae, selecting equivalence classes of processes
up to a pre--specified depth limit on the support set of names. Composed with our
encoding, this characteristic formula can be used to select
characteristic formulae for knots.
%\end{description}

\subsubsection{Spatial logic formulae}

The grammar below (segmented for comprehension) summarizes the syntax
of spatial logic formulae. We employ illustrative examples in the
sequel to provide an intuitive understanding of their meaning
referring the reader to \cite{Caires04} for a more detailed explication
of the semantics.

\begin{mathpar}
  \inferrule* [lab=boolean] {} {{A,B} \bc T \;|\; \neg A \;|\; A \wedge B \;|\; \eta = \eta'}
  \and
  \inferrule* [lab=spatial] {} {|\; \pzero \;|\; A | B \;|\; x \text{\textregistered} A \;|\; \forall x . A \;|\;  H x . A}
  \and
  \inferrule* [lab=behavioral] {} {|\; \alpha . A}
  \and 
  \inferrule* [lab=recursion] {} {|\; X(\vec{u}) \;|\; \mu X(\vec{u}) . A}
  \and
  \inferrule* [lab=action] {} {\alpha \bc \langle x?(\vec{y}) \rangle \;|\; \langle x!(\vec{y}) \rangle \;|\; \langle \tau \rangle}
  \and 
  \inferrule* [lab=name] {} {\eta \bc x \;|\; \tau}
\end{mathpar} 

% subsection characteristic_formulae (end)   	 

\subsection{Example formulae}\label{sub:example_formulae_} % (fold)

\subsubsection{Crossing as formula.}
% 
% \begin{align*}
%   \frac{d}{dx} \sin x &= \cos x 
%   & \frac{d}{dx} e^x &= e^x \\
%   \frac{d}{dx} \cos x &= - \sin x 
%   & \frac{d}{dx} \log x &= \frac{1}{x} \\
% \end{align*} 

\begin{align*}
 \mu C(x_{0},x_{1},y_{0},y_{1},u).&(\langle x_{0}?(z) \rangle(\langle u! \rangle\langle y_{1}!z \rangle C(x_{0},x_{1},y_{0},y_{1},u)) & \\
  & \wedge \langle y_{1}?(z) \rangle (\langle u! \rangle \langle x_{0}!z \rangle C(x_{0},x_{1},y_{0},y_{1},u)) & \\
  & \wedge \langle x_{1}?(z) \rangle (\langle u? \rangle \langle y_{0}!z \rangle C(x_{0},x_{1},y_{0},y_{1},u)) & \\
  & \wedge \langle y_{0}?(z) \rangle (\langle u? \rangle \langle x_{1}!z \rangle C(x_{0},x_{1},y_{0},y_{1},u))) &
\end{align*}

The lexicographical similarity between the shape of this formulae and
the shape of definition of the process representing a crossing reveals
the intuitive meaning of this formulae. It describes the capabilities
of a process that has the right to represent a crossing. For example
it picks out processes that may perform an input on the port $x_0$ in
its initial menu of capabilities. What differentiates the formula
from the process, however, is that the crossing process is the
smallest candidate to satisfy the formula. Infinitely many other
processes -- with internal behavior hidden behind this interface, so
to speak -- also satisfy this formula. Even this simple formula,
then, can be seen to open a new view onto knots, providing a
computational interpretation of \emph{virtual} knots.

Note that this formula is derived by hand. A similar formula can be
derived by employing Caires' calculation of characteristic formula
\cite{Caires04} to the process representing a crossing. In light of
this discussion, we let
$\meaningof{C}_{\phi}(x0,x1,y0,y1,u)$ denote a formula specifying the
dynamics we wish to capture of a crossing. To guarantee we preserve
the shape of the interface and minimal semantics we demand that
$\meaningof{C}_{\phi}(x0,x1,y0,y1,u) \Rightarrow
\textbf{C}(x0,x1,y0,y1,u)$ where $\textbf{C}(x0,x1,y0,y1,u)$ denotes
the formula above.
                            
\subsubsection{Crossing number constraints.}
The moral content of the context lemma (Lemma \ref{context}) is that the notion of
``locality'' in the Reidemeister moves is effectively captured by the
parallel composition operator of the process calculus. This intuition
extends through the logic. Given a formula,
$\meaningof{C}_{\phi}(x0,x1,y0,y1,u)$, we can use the structural
connectives to specify constraints on crossing numbers, such as at
least $n$ crossings, or exactly $n$ crossings.
\begin{mathpar}
  \inferrule* [lab=at-least-n] {} { K^{\geq n}_{\phi}(\vec{xs},\vec{ys}) := \Pi_{i=0}^{n-1} Hu . \meaningof{C}_{\phi}(xs_i,ys_i,u) | T }
  \and 
  \inferrule* [lab=exactly-n] {} { K^{= n}_{\phi}(\vec{xs},\vec{ys}) := \Pi_{i=0}^{n-1} Hu . \meaningof{C}_{\phi}(xs_i,ys_i,u) | \neg (\forall x_0,y_0,x_1,y_1,u . \meaningof{C}_{\phi}(x_0,y_0,x_1,y_1,u) | T) }
\end{mathpar}

To round out this section, recall that the encoding of an $n$-crossing
knot decomposes into a parallel composition of $n$ \emph{copies} of a
crossing process together with a wiring harness. To specify different
knot classes with the same crossing number amounts to specifying
logical constraints on the wiring harness. In the interest of space,
we defer examples to a forthcoming paper. Suffice it to say that both
the conditions ``alternating knot'' and ``contains the tangle
corresponding to 5/3'' are expressible. For example, it is possible to
calculate the characteristic formula of a process corresponding to the
tangle 5/3 and conjoin it into the classifying formula via the
composition connective of the logic.

Finally, we wish to observe that it is entirely within reason to
contemplate a more domain-specific version of spatial logic tailored
to the shape of processes in the image of the encoding. Such a
domain-specific logic would have a better claim to the title formal
language of knot properties.

% subsection example_formulae_ (end)

% section knots_as_processes (end) 

% section spatial logic via knots (end)

\section{Conclusions and future work}

\paragraph{Testing physical space}
You, gentle reader, may wonder why of all the theorems to be proved
given this set up we pick the one above. In some sense it's hardly
central to quantum mechanics. We see it as central in the sense that
it firmly establishes a notion of physical space arising from a notion
of the equivalence of behavior. Relating bisimulation to a metric is a
big step forward, but one is faced with interpreting the relationship
of that metric space to something more physical. Quantum mechanical
notions of ``physical'' space are still far from intuitive, but by
relating this idea of distance as testing to calculations that predict
physical circumstances we are making a not insignificant step forward
toward an understanding of the physical space we inhabit as
essentially dynamic.

\paragraph{Effectivity and simulation}
One of the observations we have yet to make is that the entire program
spelled out here is effective. We have built various interpreters for
the reflective calculus at work in this interpretation. In principle,
then, we can simulate quantum mechanics on a computer. The place where
the simulation may lose fidelity is the infinitely branching summation
for the annihilator.

In this connection i also want to point out that the evaluation style
calculation of the inner product puts the non-determinism of the
summation right at the heart of measurement. This suggests that
Milner's original reduction-based formulation of the dynamics of his
calculi in terms of sums was not just notationally suggestive of a
notion of measure-and-continue but captured some significant part of
the physics.

\paragraph{Quantum continuations}
In light of this last observation i want to point out that the
predominant account of quantum mechanics is missing a key aspect of a
truly compositional story of the physical situation. In a real lab,
when a measurement is made the observation can be made to feed into
another device that then makes another measurement conditioned on the
results of the first. This means that after the superposition was
collapsed the entire experimental set up remained in
superposition. While QM offers a means of writing this down it doesn't
quite line up well with the well-trodden formulation of computation
and continuation that we see so succinctly expressed in Milner's
calculi. This suggests that there might be advantages to this account
of dynamics waiting to be explored.

\paragraph{Quantum logic}
In this connection, we also note that by virtue of having the
Hennessy-Milner construction, we can pull the construction through the
interpretation of QM. This gives us a natural candidate for a quantum
logic that enjoys an extremely tight connection with it's domain of
interpretation, making the construction much less ad hoc (rather it is
the image of functor!).

\paragraph{Quantum probabiity}
i have questions about the basis of the interpretation of inner
product as probability amplitude. In particular, using which
axiomatization of probability theory does the notion of probability
amplitude earn the right to be so dubbed? In other words, where is the
proof that the operation for calculating a probability amplitude (and
then squaring) satisfies the axioms of what it means to calculate a
probability? Even if such a proof exists (i have yet to find it in the
literature), i wonder if it might not be possible to turn things on
their heads. Can we view the calculation of the probability amplitude
as an axiomatization of probability? If so, then the definition we
give for calculating probability amplitude may provide the basis for
an \emph{effective} theory of probability.

\paragraph{Quantum vs ``biological'' information}
Finally, i want to conclude with a more philosophical observation. At
a recent workshop in which QM was a predominant topic i noticed
something about quantum information. The speaker was giving a riveting
discussion of axiomatic QM and showing how properties of ``no
cloning'' and ``no deleting'' emerged as consequences of the
axiomatization. Theorems of this form are necessary to give us a sense
of confidence that our axioms characterize the physical theory. What
struck me, though, was that if quantum information is neither erasable
nor replicable it is markedly different from \emph{life}. Two of the
things we know about life is that

\begin{itemize}
  \item it ends;
  \item to gain some measure of persistence, to transcend it's
    finitude it is imminently copyable.
\end{itemize}

Both of these qualities are summarized succinctly in the aphorism: all
flesh is grass. For me these two kinds of ``information'' -- call them
quantum and biological -- are end points on a spectrum of strategies
for persistence. At one end, we have those curious entities that enjoy
uniqueness and permanence; at the other, we have those who in the face
of a certain end and an uncertain present make a go of passing
something on. To me one of the more remarkable aspects of the latter
strategy is that in the presence of noise (and certain features of
copying) we get a kind of dynamism, a chance for improvement against a
given persistent condition.

% subsection other_calculi_other_bisimulations_and_geometry_as_behavior (end)




% section conclusion (end)

%\documentclass[12pt]{llncs}
%\documentclass{jktr}

\usepackage[pdftex]{hyperref}                   
\usepackage {listings}
\usepackage {mathpartir}
\usepackage{bcprules}
%\usepackage{listings}
                       
\usepackage{graphicx} 
%\usepackage[margins=2.5cm,nohead,nofoot]{geometry}
%\usepackage{geometry}
\usepackage{amsfonts}
\usepackage{amstext}
\usepackage{latexsym}
\usepackage{amssymb}
\usepackage{color}


%\include{myPreamble}
\include{qm2pi.local} 

%\ifpdf
%\usepackage[pdftex]{graphicx}
%\else
%\usepackage{graphicx}
%\fi

 % \ifpdf
%  \usepackage{pdfsync}
%  \if


%\title{Brief Article}
%\author{David F. Snyder}
%\author{L.G. Meredith}

%\address{Dept. of Math., Texas State University--San Marcos, San Marcos, TX 78666}
       
\pagestyle{empty}


\begin{document}

\lstset{language=[Objective]Caml,frame=shadowbox}

\input{qm2pi.front}

% section front matter (end)

\input{qm2pi.intro} 
 
% section introduction (end)

% \input{qm2pi.knotations} 

% section notation (end)

\input{qm2pi.process.calculi} 

% section concurrent_process_calculi_and_spatial_logics_ (end)
    
%\input{qm2pi.knots2pi} 

%\input{qm2pi.trefoil} 

%\input{qm2pi.mainthm} 

% subsection basic_interpretation (end)

%\input{qm2pi.rho.presentation} 
\subsection{The syntax and semantics of the notation system}\label{sub:the_syntax_and_semantics_of_the_notation_system} % (fold)

We now summarize a technical presentation of the calculus that
embodies our theory of dynamics. The typical presentation of such a
calculus follows the style of giving generators and relations on
them. The grammar, below, describing term constructors, freely
generates the set of processes, $\Proc$. This set is then quotiented
by a relation known as structural congruence and it is over this set
that the notion of dynamics is expressed. This presentation is
essentially that of \cite{MeredithR05} with the addition of
polyadicity and summation. For readability we have relegated some of
the technical subtleties to an appendix.

\subsubsection{Process grammar}\label{subsub:process_grammar}

\begin{mathpar}
  \inferrule* [lab=synchronization] {} {{M} \bc \pzero \;|\; x?F \;|\; x!C }
  \and
  \inferrule* [lab=abstraction] {} {{F} \bc (x)P}
  \and
  \inferrule* [lab=concretion] {} {{C} \bc \langle Q \rangle}
  \and
  \inferrule* [lab=process] {} {{P,Q} \bc M \;| \;P|Q \;|\; @{x}}
  \and
  \inferrule* [lab=name] {} {{x} \bc \quotep{P}}
\end{mathpar} 

Note that $\vec{x}$ (resp. $\vec{P}$) denotes a vector of names
(resp. processes) of length $|\vec{x}|$ (resp. $|\vec{P}|$). We adopt
the following useful abbreviations.

\begin{mathpar}
   x?(\vec{y}).P := x.(\vec{y})P \and  x\clift{\vec{P}} := x.\clift{\vec{P}}
   \and x!(y) := \lift{x}{\dropn{y}}
   \and \Pi_{i=0}^{n-1}P_i := P_0 | \ldots | P_{n-1}
\end{mathpar}

\subsubsection{Structural congruence}

\paragraph{Free and bound names and alpha-equivalence.} At the
core of structural equivalence is alpha-equivalence which identifies
process that are the same up to a change of variable. Formally, we
recognize the distinction between free and bound names. The free names
of a process, $\freenames{P}$, may be calculated recursively as
follows:

\begin{mathpar}
\freenames{\pzero} := \emptyset
  \and \\
  \freenames{x?(y).P} := \{ x \} \cup (\freenames{P} \setminus \{ y \})
  \and 
  \freenames{x!\langle P \rangle} := \{ x \} \cup \{ P \} 
  \and \\
  \freenames{P|Q} := \freenames{P} \cup \freenames{Q}
  \and \\
  \freenames{@{x}} := \{ x \}
\end{mathpar}

$\pi$
$\quotep{\pi}$

$\freenames{-} : \pi \to \mathcal{P}(\quotep{\pi})$

\begin{eqnarray*}
  \freenames{\pzero} & := & \emptyset \\
  \freenames{x?(y).P} & := & \{ x \} \cup (\freenames{P} \setminus \{ y \}) \\
  \freenames{x!\langle P \rangle} & := & \{ x \} \cup \{ P \} \\
  \freenames{P|Q} & := & \freenames{P} \cup \freenames{Q} \\
  \freenames{\dropn{x}} & := & \{ x \}
\end{eqnarray*}

The bound names of a process, $\boundnames{P}$, are those names occurring in $P$
that are not free. For example, in $x?(y).0$, the name $x$ is free, while $y$ is bound.

\begin{mathpar}
  \inferrule* [lab=monoidal-laws] {} { P|Q \equiv Q|P \and P|0 \equiv P \and P|(Q|R) \equiv (P|Q)|R }
\end{mathpar}

\begin{mathpar}
  \inferrule* [lab=alpha-equivalence] {} { (x)P \equiv (y)P\{y/x\} \and y \not\in \freenames{P} }
\end{mathpar}

\begin{definition}
Then two processes, $P,Q$, are alpha-equivalent if $P = Q\{\vec{y}/\vec{x}\}$ for
some $\vec{x} \in \boundnames{Q},\vec{y} \in \boundnames{P}$, where $Q\{\vec{y}/\vec{x}\}$
denotes the capture-avoiding substitution of $\vec{y}$ for $\vec{x}$ in $Q$.
\end{definition}

\begin{definition}
  The {\em structural congruence} \cite{SangiorgiWalker} , $\equiv$,
  between processes is the least congruence containing
  alpha-equivalence, satisfying the abelian monoid laws
  (associativity, commutativity and $\pzero$ as identity) for parallel
  composition $|$ and for summation $+$.
\end{definition}

\subsection{Name equivalence}

We take name equivalence, written $\nameeq$, to be the smallest
equivalence relation generated by the following rules.

\begin{mathpar}
\inferrule*[lab=Quote-drop]
{ }
{ \quotep{@{x}} \nameeq x }

\inferrule*[lab=Struct-equiv]
{ P \scong Q }
{ \quotep{P} \nameeq \quotep{Q} }
\end{mathpar}

The astute reader will have noticed that the mutual recursion of names
and processes imposes a mutual recursion on alpha-equivalence and
structural equivalence via name-equivalence. Fortunately, all of this
works out pleasantly and we may calculate in the natural way, free of
concern. The reader interested in the details is referred to the
appendix \ref{appendix:rho_details}.

\subsection{Substitution}

We use $\Proc$ for the set of processes, $\QProc$ for the set of
names, and $\id{\{}\vec{y} / \vec{x} \id{\}}$ to denote partial maps,
$s : \QProc \rightarrow \QProc$. A map, $s$ lifts, uniquely, to a map
on process terms, $\widehat{s} : \Proc \rightarrow \Proc$ by the
following equations.

\begin{mathpar}
  (0) \psubstp{Q}{P} := 0 \\
  (R \juxtap S) \psubstp{Q}{P}
  :=    
  (R)\psubstp{Q}{P} \juxtap (S) \psubstp{Q}{P} \\
  (x?(y).R) \psubstp{Q}{P}    
  :=    
  (x)\substp{Q}{P} (z)\concat( (R \psubstn{z}{y}) \psubstp{Q}{P} ) \\
  (\lift{x}{R}) \psubstp{Q}{P}  
  :=
  \lift{(x)\substp{Q}{P}}{ R \psubstp{Q}{P} } \\
%   (\dropn{x})  \psubstp{Q}{P}       
%   := 
%   \left\{ 
%     \begin{array}{ccc} 
%       \dropn{\quotep{Q}} & & x \nameeq \quotep{P} \\
%       \dropn{x} & & otherwise \\
%     \end{array}
%   \right. 
  (\dropn{x})  \psubstp{Q}{P}       
  := 
  \left\{ 
    \begin{array}{ccc} 
      Q & & x \nameeq \quotep{P} \\
      \dropn{x} & & otherwise \\
    \end{array}
  \right.
\end{mathpar}
 

where

\begin{eqnarray}
  (x)\id{\{} \lpquote Q \rpquote / \lpquote P \rpquote \id{\}}            = 
  \left\{ 
    \begin{array}{ccc}
      \lpquote Q \rpquote & & x \nameeq \lpquote P \rpquote \\
      x & & otherwise \\
    \end{array}
  \right. \nonumber
\end{eqnarray}

and $z$ is chosen distinct from $\quotep{P}$, $\quotep{Q}$, the free
names in $Q$, and all the names in $R$. Our $\alpha$-equivalence will
be built in the standard way from this substitution.

\begin{remark}\label{rem:no_self_referential_names}
  One consequence of these definitions is that $\forall P. \quotep{P}
  \not\in \freenames{P}$.
\end{remark}

\subsection{ Dynamic quote: an example }

Anticipating something of what's to come, consider applying the
substitution, $\widehat{\id{\{}u / z \id{\}}}$, to the following pair
of processes, $\lift{w}{y!(z)}$ and $w[ \lpquote y!(z) \rpquote ]$.

\begin{eqnarray}
	\lift{w}{y!(z)}\widehat{\id{\{}u / z \id{\}}}
		& = &
		\lift{w}{y!(u)} \nonumber\\
	w[ \lpquote y!(z) \rpquote ] \widehat{ \id{\{}u / z \id{\}} }
		& = &
		w[ \lpquote y!(z) \rpquote ] \nonumber
\end{eqnarray}

Because the body of the process between quotes is impervious to
substitution, we get radically different answers. In fact, by
examining the first process in an input context,
e.g. $x?(z).\lift{w}{y!(z)}$, we see that the process under the lift
operator may be shaped by prefixed inputs binding a name inside it. In
this sense, the lift operator will be seen as a way to dynamically
construct processes before reifying them as names.

Finally equipped with these standard features we can present the
dynamics of the calculus.

\subsubsection{Operational semantics} 

Finally, we introduce the computational dynamics. What marks these
algebras as distinct from other more traditionally studied algebraic
structures, e.g. vector spaces or polynomial rings, is the manner in
which dynamics is captured. In traditional structures, dynamics is typically
expressed through morphisms between such structures, as in linear maps
between vector spaces or morphisms between rings. In algebras
associated with the semantics of computation, the dynamics is
expressed as part of the algebraic structure itself, through a
reduction reduction relation typically denoted by $\red$. Below, we
give a recursive presentation of this relation for the calculus used
in the encoding.

$\red \subseteq \pi \times \pi$
$\red : \pi \to \mathcal{P}(\pi)$

\begin{mathpar}
  \inferrule* [lab=Comm] { \textsf{match}( x_{src}, x_{trgt} ) } { x_{trgt}?(y)P \; | \; x_{src}!\langle {Q} \rangle \red P\{\quotep{Q}/y}\} }
  \and \\
  \inferrule* [lab=Par] {{P} \red {P}'} {{{P} | {Q}} \red {{P}' | {Q}}}
  \and
  \inferrule* [lab=Equiv]{{{P} \scong {P}'} \andalso {{P}' \red {Q}'} \andalso {{Q}' \scong {Q}}}{{P} \red {Q}}
\end{mathpar}

\begin{eqnarray*}
  match_{\equiv} (\quotep{P},\quotep{Q}) & := & P \equiv Q \\
  match_{\dagger}(\quotep{P},\quotep{Q}) & := & \forall R. P|Q \red^{*} R => R \red^{*} 0 \\
  match_{K}(\quotep{P},\quotep{Q}) & := & K \mbox{ for some context } K
\end{eqnarray*}

$u?(x)P | u!\langle Q \rangle \red P\{\quotep{Q}/x\}$

%We write $\wred$ for $\red^*$, and $P\red$ if $\exists Q $ such that $ P \red Q$.
We write $P\red$ if $\exists Q $ such that $ P \red Q$ and $P\not\red$, otherwise.

\section{Replication}

As mentioned before, it is known that replication (and hence
recursion) can be implemented in a higher-order process algebra
\cite{SangiorgiWalker}. As our first example of calculation with the
machinery thus far presented we give the construction explicitly in
the {\rhoc}.

\begin{eqnarray}
	D_{x} & := & \prefix{x}{y}{(\binpar{\outputp{x}{y}}{@{y}})} \nonumber\\
	\bangp_{x}{P} & := & \binpar{{x}!\langle{\binpar{D_{x}}{P}}\rangle}{D_{x}} \nonumber
\end{eqnarray}

\begin{eqnarray}
	\bangp_{x}{P} & & \nonumber\\
	=
	& {x}!\langle{(\prefix{x}{y}{(\outputp{x}{y} | @{y})) | P}}\rangle 
	      | \prefix{x}{y}{(\outputp{x}{y} | @{y})} & \nonumber\\
	\red
	& (\outputp{x}{y} | @{y})\substn{\quotep{(\prefix{x}{y}{(@{y} | \outputp{x}{y})) | P}}}{y} & \nonumber\\
	=
	& \outputp{x}{\quotep{(\prefix{x}{y}{(\outputp{x}{y} | @{y})) | P}}}
	  | {(\prefix{x}{y}{(\outputp{x}{y} | @{y})) | P}} & \nonumber\\
	\red
	& \ldots & \nonumber\\
	\red^*
	& P | P | \ldots & \nonumber
\end{eqnarray}

Of course, this encoding, as an implementation, runs away, unfolding
$\bangp{P}$ eagerly. A lazier and more implementable replication
operator, restricted to input-guarded processes, may be obtained as follows.

\begin{eqnarray}
\bangp{\prefix{u}{v}{P}} 
	:= 
	\binpar{\lift{x}{\prefix{u}{v}{(\binpar{D(x)}{P})}}}{D(x)} \nonumber
\end{eqnarray}

\begin{remark}
  Note that the lazier definition still does not deal with summation
  or mixed summation (i.e. sums over input and output). The reader is
  invited to construct definitions of replication that deal with these
  features. 

  Further, the definitions are parameterized in a name, $x$. Can you,
  gentle reader, make a definition that eliminates this parameter and
  guarantees no accidental interaction between the replication
  machinery and the process being replicated -- i.e. no accidental
  sharing of names used by the process to get its work done and the
  name(s) used by the replication to effect copying. This latter
  revision of the definition of replication is crucial to obtaining
  the expected identity $!!P \sim !P$.
\end{remark}

\begin{remark}\label{rem:paradoxical_combinator}
  The reader familiar with the lambda calculus will have noticed the
  similarity between $D$ and the paradoxical combinator.

  [Ed. note: the existence of this seems to suggest we have to be more
  restrictive on the set of processes and names we admit if we are to
  support no-cloning.]
\end{remark}

\subsubsection{Bisimulation}

The computational dynamics gives rise to another kind of equivalence,
the equivalence of computational behavior. As previously mentioned
this is typically captured \emph{via} some form of bisimulation.

% The notion we use in this paper is weak barbed bisimulation
% \cite{milner91polyadicpi}.

The notion we use in this paper is derived from weak barbed
bisimulation \cite{milner91polyadicpi}. 

\begin{definition}
An \emph{observation relation}, $\downarrow_{\mathcal N}$, over a set
of names, $\mathcal N$, is the smallest relation satisfying the rules
below.

\infrule[Out-barb]{y \in {\mathcal N}, \; x \nameeq y}
		  {\outputp{x}{v} \downarrow_{\mathcal N} x}
\infrule[Par-barb]{\mbox{$P\downarrow_{\mathcal N} x$ or $Q\downarrow_{\mathcal N} x$}}
		  {\binpar{P}{Q} \downarrow_{\mathcal N} x}

We write $P \Downarrow_{\mathcal N} x$ if there is $Q$ such that 
$P \wred Q$ and $Q \downarrow_{\mathcal N} x$.
\end{definition}

\begin{definition}
%\label{def.bbisim}
An  ${\mathcal N}$-\emph{barbed bisimulation} over a set of names, ${\mathcal N}$, is a symmetric binary relation 
${\mathcal S}_{\mathcal N}$ between agents such that $P\rel{S}_{\mathcal N}Q$ implies:
\begin{enumerate}
\item If $P \red P'$ then $Q \wred Q'$ and $P'\rel{S}_{\mathcal N} Q'$.
\item If $P\downarrow_{\mathcal N} x$, then $Q\Downarrow_{\mathcal N} x$.
\end{enumerate}
$P$ is ${\mathcal N}$-barbed bisimilar to $Q$, written
$P \wbbisim_{\mathcal N} Q$, if $P \rel{S}_{\mathcal N} Q$ for some ${\mathcal N}$-barbed bisimulation ${\mathcal S}_{\mathcal N}$.
\end{definition}

$\mathcal{R} \subseteq \pi \times \pi$

$P \mathcal{R} Q => \forall P'. P \red P' \Rightarrow \exists Q'. Q \red Q', P' \mathcal{R} Q'$

$P \vdash x \Rightarrow Q \vdash x$

\begin{mathpar}
  \inferrule*[lab=Out-barb]{x \nameeq y}{{y}!\langle{Q}\rangle \vdash x}
  \and
  \inferrule*[lab=Par-barb]{\mbox{$P\vdash x$ or $Q\vdash x$}}{\binpar{P}{Q} \vdash x}
\end{mathpar}

\subsubsection{Contexts}

One of the principle advantages of computational calculi like the
$\pi$-calculus is a well-defined notion of context,
contextual-equivalence and a correlation between
contextual-equivalence and notions of bisimulation. The notion of
context allows the decomposition of a process into (sub-)process and
its syntactic environment, its context. Thus, a context may be
thought of as a process with a ``hole'' (written $\Box$) in it. The
application of a context $M$ to a process $P$, written $M[P]$, is
tantamount to filling the hole in $M$ with $P$. In this paper we do
not need the full weight of this theory, but do make use of the notion
of context in the proof the main theorem. 

\begin{mathpar}
  \inferrule* [lab=summation] {} {{M_{M},M_{N}} \bc \Box \;|\; x.M_{A} \;|\; M_{M}+M_{N}}
  \and
  \inferrule* [lab=agent] {} {{M_{A}} \bc (\vec{x})M_{P} \;| \; \clift{P_0,\ldots,M_{P},\ldots,P_N}}
  \and \\
  \inferrule* [lab=process] {} {{M_{P}} \bc M_{N} \;| \;P|M_{P} }
\end{mathpar} 

\begin{mathpar}
  \inferrule* [lab=sychronization] {} {M_{N} \bc \Box \;|\; x?M_{F} \;|\; x!M_{C}}
  \and
  \inferrule* [lab=abstraction] {} {{M_{F}} \bc (x)M_{P} }
  \and
  \inferrule* [lab=concretion] {} {{M_{C}} \bc \langle M_{P} \rangle }
  \and \\
  \inferrule* [lab=process] {} {{M_{P}} \bc M_{N} \;| \;P|M_{P} }
\end{mathpar}

\begin{definition}[contextual application] Given a context $M$, and
  process $P$, we define the \emph{contextual application}, $M[P] :=
  M\{P/\Box\}$. That is, the contextual application of M to P is the
  substitution of $P$ for $\Box$ in $M$.
\end{definition}

$\meaningof{-} : L \to \mathcal{P}(\pi)$

\begin{mathpar}
  \inferrule* [lab=collection] {} {\meaningof{true} = \pi, \and \meaningof{~E} = \pi \setminus \meaningof{E}, \and \meaningof{E_{1} \& E_{2}} = \meaningof{E_{1}} \cap \meaningof{E_{2}}}
\end{mathpar}

\begin{mathpar}
  \inferrule* [lab=structure] {} {\meaningof{0} = \{ P \in \pi | P \equiv 0 \}, \and \\ \meaningof{E_1 | E_2} = \{ P \in \pi | P \equiv P_{1} | P_{2}, P_{1} \in \meaningof{E_{1}}, P_{2} \in \meaningof{E_2}\} }
\end{mathpar}

\begin{mathpar}
 \inferrule* [lab=behavior] {} {\meaningof{\langle a?b \rangle E} = \{ P \in \pi | P \equiv Q | u?(y)P', \\ \and \\\\ \and \\ \;\;\; u \in \meaningof{a}, \forall z.P'\{z/y\} \in \meaningof{E\{z/b\}}\}, \and \\ \meaningof{a!E} = \{ P \in \pi | P \equiv Q | x!\langle P' \rangle, x \in \meaningof{a} P' \in \meaningof{E}\} }
\end{mathpar}

\begin{mathpar}
 \inferrule* [lab=nominal] {} {\meaningof{\quotep{E}} = \{ \quotep{P} \in \quotep{\pi} | P \in \meaningof{E} \}, \and \meaningof{\quotep{P}} = \{ \quotep{Q} \in \quotep{\pi} | P \equiv Q \} \and \\ \meaningof{@\quotep{E}} = \{ P \in \pi | P \equiv @x, x \in \meaningof{E} \}}
\end{mathpar}

\begin{eqnarray*}
  \\
  \meaningof{-} : TS \to ST
\end{eqnarray*}

\begin{eqnarray*}
  \\
  L : TS \to ST
\end{eqnarray*}

\begin{eqnarray*}
  \\
  P \models E \iff P \in \meaningof{E}
\end{eqnarray*}

\begin{eqnarray*}
  P \approx_{L} Q \iff \forall E \in L. P \models E \iff Q \models E
\end{eqnarray*}

\begin{eqnarray*}
  P \approx_{K} Q
\end{eqnarray*}

\begin{eqnarray*}
  P \approx Q
\end{eqnarray*}

$\approx_{K} = \approx = \approx_{L}$

\subsubsection{Contextual duality}

Note that contexts extend the quotation operation to a family of
operations from processes to names. Given a context, $M$, we can
define a \emph{nominal context}, $\quotep{M}$ by $\quotep{M}[P] :=
\quotep{M[P]}$. To foreshadow what is to come we observe that these
operations enjoy a duality with processes very much like the duality
between vectors and maps from vectors to scalars.

Further, because the calculus is essentially higher-order, we have a
correspondence between contexts and processes. More specifically,
given a name $x$ and a context $M$ we can construct $M^{*}_{x}$ such
that 

\begin{mathpar}
  M^{*}_{x} | \lift{x}{P} \red M[P]
\end{mathpar}

namely,

\begin{mathpar}
  M^{*}_{x} := x?(u).M[\dropn{u}]
\end{mathpar}

The dependence of $M^{*}_{x}$ on a name makes it an abstraction, 

\begin{mathpar}
  M^{*} := (x)x?(u).M[\dropn{u}]
\end{mathpar}

\subsection{Additional notation}

It will sometimes be convenient to denote the process a name
quotes. We already have the notation $x = \quotep{P}$, but it will be
convenient to introduce an alternate notation, $\procn{x}$, when we
want to emphasize the connection to the use of the name. Note that, by
virtue of name equivalence, $\quotep{\procn{x}} \nameeq x$; so, the
notation is consistent with previous definitions.

Further, because names have structure it is possible to effect
substitutions on the basis of that structure. This means we need to
upgrade our notation for substitutions, which we accomplish by
adapting comprehension notation. Thus,

\begin{mathpar}
  P\{ y / x : x \in S \}
\end{mathpar}

is interpreted to mean the process derived from P by replacing (in a
capture-avoiding manner) each occurrence of $x$ in $S$ by $y$. For example,

\begin{mathpar}
  P\{ \quotep{\procn{x}|\procn{x}} / x : x \in \freenames{P} \}
\end{mathpar}

will replace each (occurrence) of a free name $x$ in $P$ by
$\quotep{\procn{x}|\procn{x}}$.

Also, we will avail ourselves of the notation $x^{L}$ and $x^{R}$ to
denote injections of a name into disjoint copies of the name
space. There are numerous ways to accomplish this. One example can be
found in \cite{MeredithR05}. This notation overloads to vectors of
names: $\vec{x}^{\pi} := (x_{i}^{\pi} \; : \; 0 \leq i < |\vec{x}| )$ where $\pi \in \{L,R\}$.

We also use $P^{\Box} := P|\Box$.

In \cite{MeredithR05} an interpretation of the new operator is
given. It turns out that there are several possible interpretations
all enjoying the requisite algebraic properties of the operator (see
\cite{milner91polyadicpi}). We will therefore make liberal use of
$(\nu\; \vec{x})P$.

% subsection the_syntax_and_semantics_of_the_notation_system (end)   

\input{qm2pi.qmops} 

\input{qm2pi.sterngerlach} 

\input{qm2pi.metric} 

% section concurrent_process_calculi (end)

%\input{qm2pi.proofsketch}

% section proof sketch (end)

%\input{qm2pi.slviaknots} 

% section spatial logic via knots (end)

\input{qm2pi.conclusion}

% section conclusion (end)

%\input{qm2pi.dtcodes} 

% section wiring algorithm (end)

\input{qm2pi.ack} 

% section acknowledgments (end)

\newpage


\bibliographystyle{plain}   
\bibliography{../../biblios/main.bib}

\input{qm2pi.rhodetails}

\end{document}

 

% section wiring algorithm (end)

\documentclass[12pt]{llncs}
%\documentclass{jktr}

\usepackage[pdftex]{hyperref}                   
\usepackage {listings}
\usepackage {mathpartir}
\usepackage{bcprules}
%\usepackage{listings}
                       
\usepackage{graphicx} 
%\usepackage[margins=2.5cm,nohead,nofoot]{geometry}
%\usepackage{geometry}
\usepackage{amsfonts}
\usepackage{amstext}
\usepackage{latexsym}
\usepackage{amssymb}
\usepackage{color}


%\include{myPreamble}
\include{qm2pi.local} 

%\ifpdf
%\usepackage[pdftex]{graphicx}
%\else
%\usepackage{graphicx}
%\fi

 % \ifpdf
%  \usepackage{pdfsync}
%  \if


%\title{Brief Article}
%\author{David F. Snyder}
%\author{L.G. Meredith}

%\address{Dept. of Math., Texas State University--San Marcos, San Marcos, TX 78666}
       
\pagestyle{empty}


\begin{document}

\lstset{language=[Objective]Caml,frame=shadowbox}

\input{qm2pi.front}

% section front matter (end)

\input{qm2pi.intro} 
 
% section introduction (end)

% \input{qm2pi.knotations} 

% section notation (end)

\input{qm2pi.process.calculi} 

% section concurrent_process_calculi_and_spatial_logics_ (end)
    
%\input{qm2pi.knots2pi} 

%\input{qm2pi.trefoil} 

%\input{qm2pi.mainthm} 

% subsection basic_interpretation (end)

%\input{qm2pi.rho.presentation} 
\subsection{The syntax and semantics of the notation system}\label{sub:the_syntax_and_semantics_of_the_notation_system} % (fold)

We now summarize a technical presentation of the calculus that
embodies our theory of dynamics. The typical presentation of such a
calculus follows the style of giving generators and relations on
them. The grammar, below, describing term constructors, freely
generates the set of processes, $\Proc$. This set is then quotiented
by a relation known as structural congruence and it is over this set
that the notion of dynamics is expressed. This presentation is
essentially that of \cite{MeredithR05} with the addition of
polyadicity and summation. For readability we have relegated some of
the technical subtleties to an appendix.

\subsubsection{Process grammar}\label{subsub:process_grammar}

\begin{mathpar}
  \inferrule* [lab=synchronization] {} {{M} \bc \pzero \;|\; x?F \;|\; x!C }
  \and
  \inferrule* [lab=abstraction] {} {{F} \bc (x)P}
  \and
  \inferrule* [lab=concretion] {} {{C} \bc \langle Q \rangle}
  \and
  \inferrule* [lab=process] {} {{P,Q} \bc M \;| \;P|Q \;|\; @{x}}
  \and
  \inferrule* [lab=name] {} {{x} \bc \quotep{P}}
\end{mathpar} 

Note that $\vec{x}$ (resp. $\vec{P}$) denotes a vector of names
(resp. processes) of length $|\vec{x}|$ (resp. $|\vec{P}|$). We adopt
the following useful abbreviations.

\begin{mathpar}
   x?(\vec{y}).P := x.(\vec{y})P \and  x\clift{\vec{P}} := x.\clift{\vec{P}}
   \and x!(y) := \lift{x}{\dropn{y}}
   \and \Pi_{i=0}^{n-1}P_i := P_0 | \ldots | P_{n-1}
\end{mathpar}

\subsubsection{Structural congruence}

\paragraph{Free and bound names and alpha-equivalence.} At the
core of structural equivalence is alpha-equivalence which identifies
process that are the same up to a change of variable. Formally, we
recognize the distinction between free and bound names. The free names
of a process, $\freenames{P}$, may be calculated recursively as
follows:

\begin{mathpar}
\freenames{\pzero} := \emptyset
  \and \\
  \freenames{x?(y).P} := \{ x \} \cup (\freenames{P} \setminus \{ y \})
  \and 
  \freenames{x!\langle P \rangle} := \{ x \} \cup \{ P \} 
  \and \\
  \freenames{P|Q} := \freenames{P} \cup \freenames{Q}
  \and \\
  \freenames{@{x}} := \{ x \}
\end{mathpar}

$\pi$
$\quotep{\pi}$

$\freenames{-} : \pi \to \mathcal{P}(\quotep{\pi})$

\begin{eqnarray*}
  \freenames{\pzero} & := & \emptyset \\
  \freenames{x?(y).P} & := & \{ x \} \cup (\freenames{P} \setminus \{ y \}) \\
  \freenames{x!\langle P \rangle} & := & \{ x \} \cup \{ P \} \\
  \freenames{P|Q} & := & \freenames{P} \cup \freenames{Q} \\
  \freenames{\dropn{x}} & := & \{ x \}
\end{eqnarray*}

The bound names of a process, $\boundnames{P}$, are those names occurring in $P$
that are not free. For example, in $x?(y).0$, the name $x$ is free, while $y$ is bound.

\begin{mathpar}
  \inferrule* [lab=monoidal-laws] {} { P|Q \equiv Q|P \and P|0 \equiv P \and P|(Q|R) \equiv (P|Q)|R }
\end{mathpar}

\begin{mathpar}
  \inferrule* [lab=alpha-equivalence] {} { (x)P \equiv (y)P\{y/x\} \and y \not\in \freenames{P} }
\end{mathpar}

\begin{definition}
Then two processes, $P,Q$, are alpha-equivalent if $P = Q\{\vec{y}/\vec{x}\}$ for
some $\vec{x} \in \boundnames{Q},\vec{y} \in \boundnames{P}$, where $Q\{\vec{y}/\vec{x}\}$
denotes the capture-avoiding substitution of $\vec{y}$ for $\vec{x}$ in $Q$.
\end{definition}

\begin{definition}
  The {\em structural congruence} \cite{SangiorgiWalker} , $\equiv$,
  between processes is the least congruence containing
  alpha-equivalence, satisfying the abelian monoid laws
  (associativity, commutativity and $\pzero$ as identity) for parallel
  composition $|$ and for summation $+$.
\end{definition}

\subsection{Name equivalence}

We take name equivalence, written $\nameeq$, to be the smallest
equivalence relation generated by the following rules.

\begin{mathpar}
\inferrule*[lab=Quote-drop]
{ }
{ \quotep{@{x}} \nameeq x }

\inferrule*[lab=Struct-equiv]
{ P \scong Q }
{ \quotep{P} \nameeq \quotep{Q} }
\end{mathpar}

The astute reader will have noticed that the mutual recursion of names
and processes imposes a mutual recursion on alpha-equivalence and
structural equivalence via name-equivalence. Fortunately, all of this
works out pleasantly and we may calculate in the natural way, free of
concern. The reader interested in the details is referred to the
appendix \ref{appendix:rho_details}.

\subsection{Substitution}

We use $\Proc$ for the set of processes, $\QProc$ for the set of
names, and $\id{\{}\vec{y} / \vec{x} \id{\}}$ to denote partial maps,
$s : \QProc \rightarrow \QProc$. A map, $s$ lifts, uniquely, to a map
on process terms, $\widehat{s} : \Proc \rightarrow \Proc$ by the
following equations.

\begin{mathpar}
  (0) \psubstp{Q}{P} := 0 \\
  (R \juxtap S) \psubstp{Q}{P}
  :=    
  (R)\psubstp{Q}{P} \juxtap (S) \psubstp{Q}{P} \\
  (x?(y).R) \psubstp{Q}{P}    
  :=    
  (x)\substp{Q}{P} (z)\concat( (R \psubstn{z}{y}) \psubstp{Q}{P} ) \\
  (\lift{x}{R}) \psubstp{Q}{P}  
  :=
  \lift{(x)\substp{Q}{P}}{ R \psubstp{Q}{P} } \\
%   (\dropn{x})  \psubstp{Q}{P}       
%   := 
%   \left\{ 
%     \begin{array}{ccc} 
%       \dropn{\quotep{Q}} & & x \nameeq \quotep{P} \\
%       \dropn{x} & & otherwise \\
%     \end{array}
%   \right. 
  (\dropn{x})  \psubstp{Q}{P}       
  := 
  \left\{ 
    \begin{array}{ccc} 
      Q & & x \nameeq \quotep{P} \\
      \dropn{x} & & otherwise \\
    \end{array}
  \right.
\end{mathpar}
 

where

\begin{eqnarray}
  (x)\id{\{} \lpquote Q \rpquote / \lpquote P \rpquote \id{\}}            = 
  \left\{ 
    \begin{array}{ccc}
      \lpquote Q \rpquote & & x \nameeq \lpquote P \rpquote \\
      x & & otherwise \\
    \end{array}
  \right. \nonumber
\end{eqnarray}

and $z$ is chosen distinct from $\quotep{P}$, $\quotep{Q}$, the free
names in $Q$, and all the names in $R$. Our $\alpha$-equivalence will
be built in the standard way from this substitution.

\begin{remark}\label{rem:no_self_referential_names}
  One consequence of these definitions is that $\forall P. \quotep{P}
  \not\in \freenames{P}$.
\end{remark}

\subsection{ Dynamic quote: an example }

Anticipating something of what's to come, consider applying the
substitution, $\widehat{\id{\{}u / z \id{\}}}$, to the following pair
of processes, $\lift{w}{y!(z)}$ and $w[ \lpquote y!(z) \rpquote ]$.

\begin{eqnarray}
	\lift{w}{y!(z)}\widehat{\id{\{}u / z \id{\}}}
		& = &
		\lift{w}{y!(u)} \nonumber\\
	w[ \lpquote y!(z) \rpquote ] \widehat{ \id{\{}u / z \id{\}} }
		& = &
		w[ \lpquote y!(z) \rpquote ] \nonumber
\end{eqnarray}

Because the body of the process between quotes is impervious to
substitution, we get radically different answers. In fact, by
examining the first process in an input context,
e.g. $x?(z).\lift{w}{y!(z)}$, we see that the process under the lift
operator may be shaped by prefixed inputs binding a name inside it. In
this sense, the lift operator will be seen as a way to dynamically
construct processes before reifying them as names.

Finally equipped with these standard features we can present the
dynamics of the calculus.

\subsubsection{Operational semantics} 

Finally, we introduce the computational dynamics. What marks these
algebras as distinct from other more traditionally studied algebraic
structures, e.g. vector spaces or polynomial rings, is the manner in
which dynamics is captured. In traditional structures, dynamics is typically
expressed through morphisms between such structures, as in linear maps
between vector spaces or morphisms between rings. In algebras
associated with the semantics of computation, the dynamics is
expressed as part of the algebraic structure itself, through a
reduction reduction relation typically denoted by $\red$. Below, we
give a recursive presentation of this relation for the calculus used
in the encoding.

$\red \subseteq \pi \times \pi$
$\red : \pi \to \mathcal{P}(\pi)$

\begin{mathpar}
  \inferrule* [lab=Comm] { \textsf{match}( x_{src}, x_{trgt} ) } { x_{trgt}?(y)P \; | \; x_{src}!\langle {Q} \rangle \red P\{\quotep{Q}/y}\} }
  \and \\
  \inferrule* [lab=Par] {{P} \red {P}'} {{{P} | {Q}} \red {{P}' | {Q}}}
  \and
  \inferrule* [lab=Equiv]{{{P} \scong {P}'} \andalso {{P}' \red {Q}'} \andalso {{Q}' \scong {Q}}}{{P} \red {Q}}
\end{mathpar}

\begin{eqnarray*}
  match_{\equiv} (\quotep{P},\quotep{Q}) & := & P \equiv Q \\
  match_{\dagger}(\quotep{P},\quotep{Q}) & := & \forall R. P|Q \red^{*} R => R \red^{*} 0 \\
  match_{K}(\quotep{P},\quotep{Q}) & := & K \mbox{ for some context } K
\end{eqnarray*}

$u?(x)P | u!\langle Q \rangle \red P\{\quotep{Q}/x\}$

%We write $\wred$ for $\red^*$, and $P\red$ if $\exists Q $ such that $ P \red Q$.
We write $P\red$ if $\exists Q $ such that $ P \red Q$ and $P\not\red$, otherwise.

\section{Replication}

As mentioned before, it is known that replication (and hence
recursion) can be implemented in a higher-order process algebra
\cite{SangiorgiWalker}. As our first example of calculation with the
machinery thus far presented we give the construction explicitly in
the {\rhoc}.

\begin{eqnarray}
	D_{x} & := & \prefix{x}{y}{(\binpar{\outputp{x}{y}}{@{y}})} \nonumber\\
	\bangp_{x}{P} & := & \binpar{{x}!\langle{\binpar{D_{x}}{P}}\rangle}{D_{x}} \nonumber
\end{eqnarray}

\begin{eqnarray}
	\bangp_{x}{P} & & \nonumber\\
	=
	& {x}!\langle{(\prefix{x}{y}{(\outputp{x}{y} | @{y})) | P}}\rangle 
	      | \prefix{x}{y}{(\outputp{x}{y} | @{y})} & \nonumber\\
	\red
	& (\outputp{x}{y} | @{y})\substn{\quotep{(\prefix{x}{y}{(@{y} | \outputp{x}{y})) | P}}}{y} & \nonumber\\
	=
	& \outputp{x}{\quotep{(\prefix{x}{y}{(\outputp{x}{y} | @{y})) | P}}}
	  | {(\prefix{x}{y}{(\outputp{x}{y} | @{y})) | P}} & \nonumber\\
	\red
	& \ldots & \nonumber\\
	\red^*
	& P | P | \ldots & \nonumber
\end{eqnarray}

Of course, this encoding, as an implementation, runs away, unfolding
$\bangp{P}$ eagerly. A lazier and more implementable replication
operator, restricted to input-guarded processes, may be obtained as follows.

\begin{eqnarray}
\bangp{\prefix{u}{v}{P}} 
	:= 
	\binpar{\lift{x}{\prefix{u}{v}{(\binpar{D(x)}{P})}}}{D(x)} \nonumber
\end{eqnarray}

\begin{remark}
  Note that the lazier definition still does not deal with summation
  or mixed summation (i.e. sums over input and output). The reader is
  invited to construct definitions of replication that deal with these
  features. 

  Further, the definitions are parameterized in a name, $x$. Can you,
  gentle reader, make a definition that eliminates this parameter and
  guarantees no accidental interaction between the replication
  machinery and the process being replicated -- i.e. no accidental
  sharing of names used by the process to get its work done and the
  name(s) used by the replication to effect copying. This latter
  revision of the definition of replication is crucial to obtaining
  the expected identity $!!P \sim !P$.
\end{remark}

\begin{remark}\label{rem:paradoxical_combinator}
  The reader familiar with the lambda calculus will have noticed the
  similarity between $D$ and the paradoxical combinator.

  [Ed. note: the existence of this seems to suggest we have to be more
  restrictive on the set of processes and names we admit if we are to
  support no-cloning.]
\end{remark}

\subsubsection{Bisimulation}

The computational dynamics gives rise to another kind of equivalence,
the equivalence of computational behavior. As previously mentioned
this is typically captured \emph{via} some form of bisimulation.

% The notion we use in this paper is weak barbed bisimulation
% \cite{milner91polyadicpi}.

The notion we use in this paper is derived from weak barbed
bisimulation \cite{milner91polyadicpi}. 

\begin{definition}
An \emph{observation relation}, $\downarrow_{\mathcal N}$, over a set
of names, $\mathcal N$, is the smallest relation satisfying the rules
below.

\infrule[Out-barb]{y \in {\mathcal N}, \; x \nameeq y}
		  {\outputp{x}{v} \downarrow_{\mathcal N} x}
\infrule[Par-barb]{\mbox{$P\downarrow_{\mathcal N} x$ or $Q\downarrow_{\mathcal N} x$}}
		  {\binpar{P}{Q} \downarrow_{\mathcal N} x}

We write $P \Downarrow_{\mathcal N} x$ if there is $Q$ such that 
$P \wred Q$ and $Q \downarrow_{\mathcal N} x$.
\end{definition}

\begin{definition}
%\label{def.bbisim}
An  ${\mathcal N}$-\emph{barbed bisimulation} over a set of names, ${\mathcal N}$, is a symmetric binary relation 
${\mathcal S}_{\mathcal N}$ between agents such that $P\rel{S}_{\mathcal N}Q$ implies:
\begin{enumerate}
\item If $P \red P'$ then $Q \wred Q'$ and $P'\rel{S}_{\mathcal N} Q'$.
\item If $P\downarrow_{\mathcal N} x$, then $Q\Downarrow_{\mathcal N} x$.
\end{enumerate}
$P$ is ${\mathcal N}$-barbed bisimilar to $Q$, written
$P \wbbisim_{\mathcal N} Q$, if $P \rel{S}_{\mathcal N} Q$ for some ${\mathcal N}$-barbed bisimulation ${\mathcal S}_{\mathcal N}$.
\end{definition}

$\mathcal{R} \subseteq \pi \times \pi$

$P \mathcal{R} Q => \forall P'. P \red P' \Rightarrow \exists Q'. Q \red Q', P' \mathcal{R} Q'$

$P \vdash x \Rightarrow Q \vdash x$

\begin{mathpar}
  \inferrule*[lab=Out-barb]{x \nameeq y}{{y}!\langle{Q}\rangle \vdash x}
  \and
  \inferrule*[lab=Par-barb]{\mbox{$P\vdash x$ or $Q\vdash x$}}{\binpar{P}{Q} \vdash x}
\end{mathpar}

\subsubsection{Contexts}

One of the principle advantages of computational calculi like the
$\pi$-calculus is a well-defined notion of context,
contextual-equivalence and a correlation between
contextual-equivalence and notions of bisimulation. The notion of
context allows the decomposition of a process into (sub-)process and
its syntactic environment, its context. Thus, a context may be
thought of as a process with a ``hole'' (written $\Box$) in it. The
application of a context $M$ to a process $P$, written $M[P]$, is
tantamount to filling the hole in $M$ with $P$. In this paper we do
not need the full weight of this theory, but do make use of the notion
of context in the proof the main theorem. 

\begin{mathpar}
  \inferrule* [lab=summation] {} {{M_{M},M_{N}} \bc \Box \;|\; x.M_{A} \;|\; M_{M}+M_{N}}
  \and
  \inferrule* [lab=agent] {} {{M_{A}} \bc (\vec{x})M_{P} \;| \; \clift{P_0,\ldots,M_{P},\ldots,P_N}}
  \and \\
  \inferrule* [lab=process] {} {{M_{P}} \bc M_{N} \;| \;P|M_{P} }
\end{mathpar} 

\begin{mathpar}
  \inferrule* [lab=sychronization] {} {M_{N} \bc \Box \;|\; x?M_{F} \;|\; x!M_{C}}
  \and
  \inferrule* [lab=abstraction] {} {{M_{F}} \bc (x)M_{P} }
  \and
  \inferrule* [lab=concretion] {} {{M_{C}} \bc \langle M_{P} \rangle }
  \and \\
  \inferrule* [lab=process] {} {{M_{P}} \bc M_{N} \;| \;P|M_{P} }
\end{mathpar}

\begin{definition}[contextual application] Given a context $M$, and
  process $P$, we define the \emph{contextual application}, $M[P] :=
  M\{P/\Box\}$. That is, the contextual application of M to P is the
  substitution of $P$ for $\Box$ in $M$.
\end{definition}

$\meaningof{-} : L \to \mathcal{P}(\pi)$

\begin{mathpar}
  \inferrule* [lab=collection] {} {\meaningof{true} = \pi, \and \meaningof{~E} = \pi \setminus \meaningof{E}, \and \meaningof{E_{1} \& E_{2}} = \meaningof{E_{1}} \cap \meaningof{E_{2}}}
\end{mathpar}

\begin{mathpar}
  \inferrule* [lab=structure] {} {\meaningof{0} = \{ P \in \pi | P \equiv 0 \}, \and \\ \meaningof{E_1 | E_2} = \{ P \in \pi | P \equiv P_{1} | P_{2}, P_{1} \in \meaningof{E_{1}}, P_{2} \in \meaningof{E_2}\} }
\end{mathpar}

\begin{mathpar}
 \inferrule* [lab=behavior] {} {\meaningof{\langle a?b \rangle E} = \{ P \in \pi | P \equiv Q | u?(y)P', \\ \and \\\\ \and \\ \;\;\; u \in \meaningof{a}, \forall z.P'\{z/y\} \in \meaningof{E\{z/b\}}\}, \and \\ \meaningof{a!E} = \{ P \in \pi | P \equiv Q | x!\langle P' \rangle, x \in \meaningof{a} P' \in \meaningof{E}\} }
\end{mathpar}

\begin{mathpar}
 \inferrule* [lab=nominal] {} {\meaningof{\quotep{E}} = \{ \quotep{P} \in \quotep{\pi} | P \in \meaningof{E} \}, \and \meaningof{\quotep{P}} = \{ \quotep{Q} \in \quotep{\pi} | P \equiv Q \} \and \\ \meaningof{@\quotep{E}} = \{ P \in \pi | P \equiv @x, x \in \meaningof{E} \}}
\end{mathpar}

\begin{eqnarray*}
  \\
  \meaningof{-} : TS \to ST
\end{eqnarray*}

\begin{eqnarray*}
  \\
  L : TS \to ST
\end{eqnarray*}

\begin{eqnarray*}
  \\
  P \models E \iff P \in \meaningof{E}
\end{eqnarray*}

\begin{eqnarray*}
  P \approx_{L} Q \iff \forall E \in L. P \models E \iff Q \models E
\end{eqnarray*}

\begin{eqnarray*}
  P \approx_{K} Q
\end{eqnarray*}

\begin{eqnarray*}
  P \approx Q
\end{eqnarray*}

$\approx_{K} = \approx = \approx_{L}$

\subsubsection{Contextual duality}

Note that contexts extend the quotation operation to a family of
operations from processes to names. Given a context, $M$, we can
define a \emph{nominal context}, $\quotep{M}$ by $\quotep{M}[P] :=
\quotep{M[P]}$. To foreshadow what is to come we observe that these
operations enjoy a duality with processes very much like the duality
between vectors and maps from vectors to scalars.

Further, because the calculus is essentially higher-order, we have a
correspondence between contexts and processes. More specifically,
given a name $x$ and a context $M$ we can construct $M^{*}_{x}$ such
that 

\begin{mathpar}
  M^{*}_{x} | \lift{x}{P} \red M[P]
\end{mathpar}

namely,

\begin{mathpar}
  M^{*}_{x} := x?(u).M[\dropn{u}]
\end{mathpar}

The dependence of $M^{*}_{x}$ on a name makes it an abstraction, 

\begin{mathpar}
  M^{*} := (x)x?(u).M[\dropn{u}]
\end{mathpar}

\subsection{Additional notation}

It will sometimes be convenient to denote the process a name
quotes. We already have the notation $x = \quotep{P}$, but it will be
convenient to introduce an alternate notation, $\procn{x}$, when we
want to emphasize the connection to the use of the name. Note that, by
virtue of name equivalence, $\quotep{\procn{x}} \nameeq x$; so, the
notation is consistent with previous definitions.

Further, because names have structure it is possible to effect
substitutions on the basis of that structure. This means we need to
upgrade our notation for substitutions, which we accomplish by
adapting comprehension notation. Thus,

\begin{mathpar}
  P\{ y / x : x \in S \}
\end{mathpar}

is interpreted to mean the process derived from P by replacing (in a
capture-avoiding manner) each occurrence of $x$ in $S$ by $y$. For example,

\begin{mathpar}
  P\{ \quotep{\procn{x}|\procn{x}} / x : x \in \freenames{P} \}
\end{mathpar}

will replace each (occurrence) of a free name $x$ in $P$ by
$\quotep{\procn{x}|\procn{x}}$.

Also, we will avail ourselves of the notation $x^{L}$ and $x^{R}$ to
denote injections of a name into disjoint copies of the name
space. There are numerous ways to accomplish this. One example can be
found in \cite{MeredithR05}. This notation overloads to vectors of
names: $\vec{x}^{\pi} := (x_{i}^{\pi} \; : \; 0 \leq i < |\vec{x}| )$ where $\pi \in \{L,R\}$.

We also use $P^{\Box} := P|\Box$.

In \cite{MeredithR05} an interpretation of the new operator is
given. It turns out that there are several possible interpretations
all enjoying the requisite algebraic properties of the operator (see
\cite{milner91polyadicpi}). We will therefore make liberal use of
$(\nu\; \vec{x})P$.

% subsection the_syntax_and_semantics_of_the_notation_system (end)   

\input{qm2pi.qmops} 

\input{qm2pi.sterngerlach} 

\input{qm2pi.metric} 

% section concurrent_process_calculi (end)

%\input{qm2pi.proofsketch}

% section proof sketch (end)

%\input{qm2pi.slviaknots} 

% section spatial logic via knots (end)

\input{qm2pi.conclusion}

% section conclusion (end)

%\input{qm2pi.dtcodes} 

% section wiring algorithm (end)

\input{qm2pi.ack} 

% section acknowledgments (end)

\newpage


\bibliographystyle{plain}   
\bibliography{../../biblios/main.bib}

\input{qm2pi.rhodetails}

\end{document}

 

% section acknowledgments (end)

\newpage


\bibliographystyle{plain}   
\bibliography{../../biblios/main.bib}

\documentclass[12pt]{llncs}
%\documentclass{jktr}

\usepackage[pdftex]{hyperref}                   
\usepackage {listings}
\usepackage {mathpartir}
\usepackage{bcprules}
%\usepackage{listings}
                       
\usepackage{graphicx} 
%\usepackage[margins=2.5cm,nohead,nofoot]{geometry}
%\usepackage{geometry}
\usepackage{amsfonts}
\usepackage{amstext}
\usepackage{latexsym}
\usepackage{amssymb}
\usepackage{color}


%\include{myPreamble}
\include{qm2pi.local} 

%\ifpdf
%\usepackage[pdftex]{graphicx}
%\else
%\usepackage{graphicx}
%\fi

 % \ifpdf
%  \usepackage{pdfsync}
%  \if


%\title{Brief Article}
%\author{David F. Snyder}
%\author{L.G. Meredith}

%\address{Dept. of Math., Texas State University--San Marcos, San Marcos, TX 78666}
       
\pagestyle{empty}


\begin{document}

\lstset{language=[Objective]Caml,frame=shadowbox}

\input{qm2pi.front}

% section front matter (end)

\input{qm2pi.intro} 
 
% section introduction (end)

% \input{qm2pi.knotations} 

% section notation (end)

\input{qm2pi.process.calculi} 

% section concurrent_process_calculi_and_spatial_logics_ (end)
    
%\input{qm2pi.knots2pi} 

%\input{qm2pi.trefoil} 

%\input{qm2pi.mainthm} 

% subsection basic_interpretation (end)

%\input{qm2pi.rho.presentation} 
\subsection{The syntax and semantics of the notation system}\label{sub:the_syntax_and_semantics_of_the_notation_system} % (fold)

We now summarize a technical presentation of the calculus that
embodies our theory of dynamics. The typical presentation of such a
calculus follows the style of giving generators and relations on
them. The grammar, below, describing term constructors, freely
generates the set of processes, $\Proc$. This set is then quotiented
by a relation known as structural congruence and it is over this set
that the notion of dynamics is expressed. This presentation is
essentially that of \cite{MeredithR05} with the addition of
polyadicity and summation. For readability we have relegated some of
the technical subtleties to an appendix.

\subsubsection{Process grammar}\label{subsub:process_grammar}

\begin{mathpar}
  \inferrule* [lab=synchronization] {} {{M} \bc \pzero \;|\; x?F \;|\; x!C }
  \and
  \inferrule* [lab=abstraction] {} {{F} \bc (x)P}
  \and
  \inferrule* [lab=concretion] {} {{C} \bc \langle Q \rangle}
  \and
  \inferrule* [lab=process] {} {{P,Q} \bc M \;| \;P|Q \;|\; @{x}}
  \and
  \inferrule* [lab=name] {} {{x} \bc \quotep{P}}
\end{mathpar} 

Note that $\vec{x}$ (resp. $\vec{P}$) denotes a vector of names
(resp. processes) of length $|\vec{x}|$ (resp. $|\vec{P}|$). We adopt
the following useful abbreviations.

\begin{mathpar}
   x?(\vec{y}).P := x.(\vec{y})P \and  x\clift{\vec{P}} := x.\clift{\vec{P}}
   \and x!(y) := \lift{x}{\dropn{y}}
   \and \Pi_{i=0}^{n-1}P_i := P_0 | \ldots | P_{n-1}
\end{mathpar}

\subsubsection{Structural congruence}

\paragraph{Free and bound names and alpha-equivalence.} At the
core of structural equivalence is alpha-equivalence which identifies
process that are the same up to a change of variable. Formally, we
recognize the distinction between free and bound names. The free names
of a process, $\freenames{P}$, may be calculated recursively as
follows:

\begin{mathpar}
\freenames{\pzero} := \emptyset
  \and \\
  \freenames{x?(y).P} := \{ x \} \cup (\freenames{P} \setminus \{ y \})
  \and 
  \freenames{x!\langle P \rangle} := \{ x \} \cup \{ P \} 
  \and \\
  \freenames{P|Q} := \freenames{P} \cup \freenames{Q}
  \and \\
  \freenames{@{x}} := \{ x \}
\end{mathpar}

$\pi$
$\quotep{\pi}$

$\freenames{-} : \pi \to \mathcal{P}(\quotep{\pi})$

\begin{eqnarray*}
  \freenames{\pzero} & := & \emptyset \\
  \freenames{x?(y).P} & := & \{ x \} \cup (\freenames{P} \setminus \{ y \}) \\
  \freenames{x!\langle P \rangle} & := & \{ x \} \cup \{ P \} \\
  \freenames{P|Q} & := & \freenames{P} \cup \freenames{Q} \\
  \freenames{\dropn{x}} & := & \{ x \}
\end{eqnarray*}

The bound names of a process, $\boundnames{P}$, are those names occurring in $P$
that are not free. For example, in $x?(y).0$, the name $x$ is free, while $y$ is bound.

\begin{mathpar}
  \inferrule* [lab=monoidal-laws] {} { P|Q \equiv Q|P \and P|0 \equiv P \and P|(Q|R) \equiv (P|Q)|R }
\end{mathpar}

\begin{mathpar}
  \inferrule* [lab=alpha-equivalence] {} { (x)P \equiv (y)P\{y/x\} \and y \not\in \freenames{P} }
\end{mathpar}

\begin{definition}
Then two processes, $P,Q$, are alpha-equivalent if $P = Q\{\vec{y}/\vec{x}\}$ for
some $\vec{x} \in \boundnames{Q},\vec{y} \in \boundnames{P}$, where $Q\{\vec{y}/\vec{x}\}$
denotes the capture-avoiding substitution of $\vec{y}$ for $\vec{x}$ in $Q$.
\end{definition}

\begin{definition}
  The {\em structural congruence} \cite{SangiorgiWalker} , $\equiv$,
  between processes is the least congruence containing
  alpha-equivalence, satisfying the abelian monoid laws
  (associativity, commutativity and $\pzero$ as identity) for parallel
  composition $|$ and for summation $+$.
\end{definition}

\subsection{Name equivalence}

We take name equivalence, written $\nameeq$, to be the smallest
equivalence relation generated by the following rules.

\begin{mathpar}
\inferrule*[lab=Quote-drop]
{ }
{ \quotep{@{x}} \nameeq x }

\inferrule*[lab=Struct-equiv]
{ P \scong Q }
{ \quotep{P} \nameeq \quotep{Q} }
\end{mathpar}

The astute reader will have noticed that the mutual recursion of names
and processes imposes a mutual recursion on alpha-equivalence and
structural equivalence via name-equivalence. Fortunately, all of this
works out pleasantly and we may calculate in the natural way, free of
concern. The reader interested in the details is referred to the
appendix \ref{appendix:rho_details}.

\subsection{Substitution}

We use $\Proc$ for the set of processes, $\QProc$ for the set of
names, and $\id{\{}\vec{y} / \vec{x} \id{\}}$ to denote partial maps,
$s : \QProc \rightarrow \QProc$. A map, $s$ lifts, uniquely, to a map
on process terms, $\widehat{s} : \Proc \rightarrow \Proc$ by the
following equations.

\begin{mathpar}
  (0) \psubstp{Q}{P} := 0 \\
  (R \juxtap S) \psubstp{Q}{P}
  :=    
  (R)\psubstp{Q}{P} \juxtap (S) \psubstp{Q}{P} \\
  (x?(y).R) \psubstp{Q}{P}    
  :=    
  (x)\substp{Q}{P} (z)\concat( (R \psubstn{z}{y}) \psubstp{Q}{P} ) \\
  (\lift{x}{R}) \psubstp{Q}{P}  
  :=
  \lift{(x)\substp{Q}{P}}{ R \psubstp{Q}{P} } \\
%   (\dropn{x})  \psubstp{Q}{P}       
%   := 
%   \left\{ 
%     \begin{array}{ccc} 
%       \dropn{\quotep{Q}} & & x \nameeq \quotep{P} \\
%       \dropn{x} & & otherwise \\
%     \end{array}
%   \right. 
  (\dropn{x})  \psubstp{Q}{P}       
  := 
  \left\{ 
    \begin{array}{ccc} 
      Q & & x \nameeq \quotep{P} \\
      \dropn{x} & & otherwise \\
    \end{array}
  \right.
\end{mathpar}
 

where

\begin{eqnarray}
  (x)\id{\{} \lpquote Q \rpquote / \lpquote P \rpquote \id{\}}            = 
  \left\{ 
    \begin{array}{ccc}
      \lpquote Q \rpquote & & x \nameeq \lpquote P \rpquote \\
      x & & otherwise \\
    \end{array}
  \right. \nonumber
\end{eqnarray}

and $z$ is chosen distinct from $\quotep{P}$, $\quotep{Q}$, the free
names in $Q$, and all the names in $R$. Our $\alpha$-equivalence will
be built in the standard way from this substitution.

\begin{remark}\label{rem:no_self_referential_names}
  One consequence of these definitions is that $\forall P. \quotep{P}
  \not\in \freenames{P}$.
\end{remark}

\subsection{ Dynamic quote: an example }

Anticipating something of what's to come, consider applying the
substitution, $\widehat{\id{\{}u / z \id{\}}}$, to the following pair
of processes, $\lift{w}{y!(z)}$ and $w[ \lpquote y!(z) \rpquote ]$.

\begin{eqnarray}
	\lift{w}{y!(z)}\widehat{\id{\{}u / z \id{\}}}
		& = &
		\lift{w}{y!(u)} \nonumber\\
	w[ \lpquote y!(z) \rpquote ] \widehat{ \id{\{}u / z \id{\}} }
		& = &
		w[ \lpquote y!(z) \rpquote ] \nonumber
\end{eqnarray}

Because the body of the process between quotes is impervious to
substitution, we get radically different answers. In fact, by
examining the first process in an input context,
e.g. $x?(z).\lift{w}{y!(z)}$, we see that the process under the lift
operator may be shaped by prefixed inputs binding a name inside it. In
this sense, the lift operator will be seen as a way to dynamically
construct processes before reifying them as names.

Finally equipped with these standard features we can present the
dynamics of the calculus.

\subsubsection{Operational semantics} 

Finally, we introduce the computational dynamics. What marks these
algebras as distinct from other more traditionally studied algebraic
structures, e.g. vector spaces or polynomial rings, is the manner in
which dynamics is captured. In traditional structures, dynamics is typically
expressed through morphisms between such structures, as in linear maps
between vector spaces or morphisms between rings. In algebras
associated with the semantics of computation, the dynamics is
expressed as part of the algebraic structure itself, through a
reduction reduction relation typically denoted by $\red$. Below, we
give a recursive presentation of this relation for the calculus used
in the encoding.

$\red \subseteq \pi \times \pi$
$\red : \pi \to \mathcal{P}(\pi)$

\begin{mathpar}
  \inferrule* [lab=Comm] { \textsf{match}( x_{src}, x_{trgt} ) } { x_{trgt}?(y)P \; | \; x_{src}!\langle {Q} \rangle \red P\{\quotep{Q}/y}\} }
  \and \\
  \inferrule* [lab=Par] {{P} \red {P}'} {{{P} | {Q}} \red {{P}' | {Q}}}
  \and
  \inferrule* [lab=Equiv]{{{P} \scong {P}'} \andalso {{P}' \red {Q}'} \andalso {{Q}' \scong {Q}}}{{P} \red {Q}}
\end{mathpar}

\begin{eqnarray*}
  match_{\equiv} (\quotep{P},\quotep{Q}) & := & P \equiv Q \\
  match_{\dagger}(\quotep{P},\quotep{Q}) & := & \forall R. P|Q \red^{*} R => R \red^{*} 0 \\
  match_{K}(\quotep{P},\quotep{Q}) & := & K \mbox{ for some context } K
\end{eqnarray*}

$u?(x)P | u!\langle Q \rangle \red P\{\quotep{Q}/x\}$

%We write $\wred$ for $\red^*$, and $P\red$ if $\exists Q $ such that $ P \red Q$.
We write $P\red$ if $\exists Q $ such that $ P \red Q$ and $P\not\red$, otherwise.

\section{Replication}

As mentioned before, it is known that replication (and hence
recursion) can be implemented in a higher-order process algebra
\cite{SangiorgiWalker}. As our first example of calculation with the
machinery thus far presented we give the construction explicitly in
the {\rhoc}.

\begin{eqnarray}
	D_{x} & := & \prefix{x}{y}{(\binpar{\outputp{x}{y}}{@{y}})} \nonumber\\
	\bangp_{x}{P} & := & \binpar{{x}!\langle{\binpar{D_{x}}{P}}\rangle}{D_{x}} \nonumber
\end{eqnarray}

\begin{eqnarray}
	\bangp_{x}{P} & & \nonumber\\
	=
	& {x}!\langle{(\prefix{x}{y}{(\outputp{x}{y} | @{y})) | P}}\rangle 
	      | \prefix{x}{y}{(\outputp{x}{y} | @{y})} & \nonumber\\
	\red
	& (\outputp{x}{y} | @{y})\substn{\quotep{(\prefix{x}{y}{(@{y} | \outputp{x}{y})) | P}}}{y} & \nonumber\\
	=
	& \outputp{x}{\quotep{(\prefix{x}{y}{(\outputp{x}{y} | @{y})) | P}}}
	  | {(\prefix{x}{y}{(\outputp{x}{y} | @{y})) | P}} & \nonumber\\
	\red
	& \ldots & \nonumber\\
	\red^*
	& P | P | \ldots & \nonumber
\end{eqnarray}

Of course, this encoding, as an implementation, runs away, unfolding
$\bangp{P}$ eagerly. A lazier and more implementable replication
operator, restricted to input-guarded processes, may be obtained as follows.

\begin{eqnarray}
\bangp{\prefix{u}{v}{P}} 
	:= 
	\binpar{\lift{x}{\prefix{u}{v}{(\binpar{D(x)}{P})}}}{D(x)} \nonumber
\end{eqnarray}

\begin{remark}
  Note that the lazier definition still does not deal with summation
  or mixed summation (i.e. sums over input and output). The reader is
  invited to construct definitions of replication that deal with these
  features. 

  Further, the definitions are parameterized in a name, $x$. Can you,
  gentle reader, make a definition that eliminates this parameter and
  guarantees no accidental interaction between the replication
  machinery and the process being replicated -- i.e. no accidental
  sharing of names used by the process to get its work done and the
  name(s) used by the replication to effect copying. This latter
  revision of the definition of replication is crucial to obtaining
  the expected identity $!!P \sim !P$.
\end{remark}

\begin{remark}\label{rem:paradoxical_combinator}
  The reader familiar with the lambda calculus will have noticed the
  similarity between $D$ and the paradoxical combinator.

  [Ed. note: the existence of this seems to suggest we have to be more
  restrictive on the set of processes and names we admit if we are to
  support no-cloning.]
\end{remark}

\subsubsection{Bisimulation}

The computational dynamics gives rise to another kind of equivalence,
the equivalence of computational behavior. As previously mentioned
this is typically captured \emph{via} some form of bisimulation.

% The notion we use in this paper is weak barbed bisimulation
% \cite{milner91polyadicpi}.

The notion we use in this paper is derived from weak barbed
bisimulation \cite{milner91polyadicpi}. 

\begin{definition}
An \emph{observation relation}, $\downarrow_{\mathcal N}$, over a set
of names, $\mathcal N$, is the smallest relation satisfying the rules
below.

\infrule[Out-barb]{y \in {\mathcal N}, \; x \nameeq y}
		  {\outputp{x}{v} \downarrow_{\mathcal N} x}
\infrule[Par-barb]{\mbox{$P\downarrow_{\mathcal N} x$ or $Q\downarrow_{\mathcal N} x$}}
		  {\binpar{P}{Q} \downarrow_{\mathcal N} x}

We write $P \Downarrow_{\mathcal N} x$ if there is $Q$ such that 
$P \wred Q$ and $Q \downarrow_{\mathcal N} x$.
\end{definition}

\begin{definition}
%\label{def.bbisim}
An  ${\mathcal N}$-\emph{barbed bisimulation} over a set of names, ${\mathcal N}$, is a symmetric binary relation 
${\mathcal S}_{\mathcal N}$ between agents such that $P\rel{S}_{\mathcal N}Q$ implies:
\begin{enumerate}
\item If $P \red P'$ then $Q \wred Q'$ and $P'\rel{S}_{\mathcal N} Q'$.
\item If $P\downarrow_{\mathcal N} x$, then $Q\Downarrow_{\mathcal N} x$.
\end{enumerate}
$P$ is ${\mathcal N}$-barbed bisimilar to $Q$, written
$P \wbbisim_{\mathcal N} Q$, if $P \rel{S}_{\mathcal N} Q$ for some ${\mathcal N}$-barbed bisimulation ${\mathcal S}_{\mathcal N}$.
\end{definition}

$\mathcal{R} \subseteq \pi \times \pi$

$P \mathcal{R} Q => \forall P'. P \red P' \Rightarrow \exists Q'. Q \red Q', P' \mathcal{R} Q'$

$P \vdash x \Rightarrow Q \vdash x$

\begin{mathpar}
  \inferrule*[lab=Out-barb]{x \nameeq y}{{y}!\langle{Q}\rangle \vdash x}
  \and
  \inferrule*[lab=Par-barb]{\mbox{$P\vdash x$ or $Q\vdash x$}}{\binpar{P}{Q} \vdash x}
\end{mathpar}

\subsubsection{Contexts}

One of the principle advantages of computational calculi like the
$\pi$-calculus is a well-defined notion of context,
contextual-equivalence and a correlation between
contextual-equivalence and notions of bisimulation. The notion of
context allows the decomposition of a process into (sub-)process and
its syntactic environment, its context. Thus, a context may be
thought of as a process with a ``hole'' (written $\Box$) in it. The
application of a context $M$ to a process $P$, written $M[P]$, is
tantamount to filling the hole in $M$ with $P$. In this paper we do
not need the full weight of this theory, but do make use of the notion
of context in the proof the main theorem. 

\begin{mathpar}
  \inferrule* [lab=summation] {} {{M_{M},M_{N}} \bc \Box \;|\; x.M_{A} \;|\; M_{M}+M_{N}}
  \and
  \inferrule* [lab=agent] {} {{M_{A}} \bc (\vec{x})M_{P} \;| \; \clift{P_0,\ldots,M_{P},\ldots,P_N}}
  \and \\
  \inferrule* [lab=process] {} {{M_{P}} \bc M_{N} \;| \;P|M_{P} }
\end{mathpar} 

\begin{mathpar}
  \inferrule* [lab=sychronization] {} {M_{N} \bc \Box \;|\; x?M_{F} \;|\; x!M_{C}}
  \and
  \inferrule* [lab=abstraction] {} {{M_{F}} \bc (x)M_{P} }
  \and
  \inferrule* [lab=concretion] {} {{M_{C}} \bc \langle M_{P} \rangle }
  \and \\
  \inferrule* [lab=process] {} {{M_{P}} \bc M_{N} \;| \;P|M_{P} }
\end{mathpar}

\begin{definition}[contextual application] Given a context $M$, and
  process $P$, we define the \emph{contextual application}, $M[P] :=
  M\{P/\Box\}$. That is, the contextual application of M to P is the
  substitution of $P$ for $\Box$ in $M$.
\end{definition}

$\meaningof{-} : L \to \mathcal{P}(\pi)$

\begin{mathpar}
  \inferrule* [lab=collection] {} {\meaningof{true} = \pi, \and \meaningof{~E} = \pi \setminus \meaningof{E}, \and \meaningof{E_{1} \& E_{2}} = \meaningof{E_{1}} \cap \meaningof{E_{2}}}
\end{mathpar}

\begin{mathpar}
  \inferrule* [lab=structure] {} {\meaningof{0} = \{ P \in \pi | P \equiv 0 \}, \and \\ \meaningof{E_1 | E_2} = \{ P \in \pi | P \equiv P_{1} | P_{2}, P_{1} \in \meaningof{E_{1}}, P_{2} \in \meaningof{E_2}\} }
\end{mathpar}

\begin{mathpar}
 \inferrule* [lab=behavior] {} {\meaningof{\langle a?b \rangle E} = \{ P \in \pi | P \equiv Q | u?(y)P', \\ \and \\\\ \and \\ \;\;\; u \in \meaningof{a}, \forall z.P'\{z/y\} \in \meaningof{E\{z/b\}}\}, \and \\ \meaningof{a!E} = \{ P \in \pi | P \equiv Q | x!\langle P' \rangle, x \in \meaningof{a} P' \in \meaningof{E}\} }
\end{mathpar}

\begin{mathpar}
 \inferrule* [lab=nominal] {} {\meaningof{\quotep{E}} = \{ \quotep{P} \in \quotep{\pi} | P \in \meaningof{E} \}, \and \meaningof{\quotep{P}} = \{ \quotep{Q} \in \quotep{\pi} | P \equiv Q \} \and \\ \meaningof{@\quotep{E}} = \{ P \in \pi | P \equiv @x, x \in \meaningof{E} \}}
\end{mathpar}

\begin{eqnarray*}
  \\
  \meaningof{-} : TS \to ST
\end{eqnarray*}

\begin{eqnarray*}
  \\
  L : TS \to ST
\end{eqnarray*}

\begin{eqnarray*}
  \\
  P \models E \iff P \in \meaningof{E}
\end{eqnarray*}

\begin{eqnarray*}
  P \approx_{L} Q \iff \forall E \in L. P \models E \iff Q \models E
\end{eqnarray*}

\begin{eqnarray*}
  P \approx_{K} Q
\end{eqnarray*}

\begin{eqnarray*}
  P \approx Q
\end{eqnarray*}

$\approx_{K} = \approx = \approx_{L}$

\subsubsection{Contextual duality}

Note that contexts extend the quotation operation to a family of
operations from processes to names. Given a context, $M$, we can
define a \emph{nominal context}, $\quotep{M}$ by $\quotep{M}[P] :=
\quotep{M[P]}$. To foreshadow what is to come we observe that these
operations enjoy a duality with processes very much like the duality
between vectors and maps from vectors to scalars.

Further, because the calculus is essentially higher-order, we have a
correspondence between contexts and processes. More specifically,
given a name $x$ and a context $M$ we can construct $M^{*}_{x}$ such
that 

\begin{mathpar}
  M^{*}_{x} | \lift{x}{P} \red M[P]
\end{mathpar}

namely,

\begin{mathpar}
  M^{*}_{x} := x?(u).M[\dropn{u}]
\end{mathpar}

The dependence of $M^{*}_{x}$ on a name makes it an abstraction, 

\begin{mathpar}
  M^{*} := (x)x?(u).M[\dropn{u}]
\end{mathpar}

\subsection{Additional notation}

It will sometimes be convenient to denote the process a name
quotes. We already have the notation $x = \quotep{P}$, but it will be
convenient to introduce an alternate notation, $\procn{x}$, when we
want to emphasize the connection to the use of the name. Note that, by
virtue of name equivalence, $\quotep{\procn{x}} \nameeq x$; so, the
notation is consistent with previous definitions.

Further, because names have structure it is possible to effect
substitutions on the basis of that structure. This means we need to
upgrade our notation for substitutions, which we accomplish by
adapting comprehension notation. Thus,

\begin{mathpar}
  P\{ y / x : x \in S \}
\end{mathpar}

is interpreted to mean the process derived from P by replacing (in a
capture-avoiding manner) each occurrence of $x$ in $S$ by $y$. For example,

\begin{mathpar}
  P\{ \quotep{\procn{x}|\procn{x}} / x : x \in \freenames{P} \}
\end{mathpar}

will replace each (occurrence) of a free name $x$ in $P$ by
$\quotep{\procn{x}|\procn{x}}$.

Also, we will avail ourselves of the notation $x^{L}$ and $x^{R}$ to
denote injections of a name into disjoint copies of the name
space. There are numerous ways to accomplish this. One example can be
found in \cite{MeredithR05}. This notation overloads to vectors of
names: $\vec{x}^{\pi} := (x_{i}^{\pi} \; : \; 0 \leq i < |\vec{x}| )$ where $\pi \in \{L,R\}$.

We also use $P^{\Box} := P|\Box$.

In \cite{MeredithR05} an interpretation of the new operator is
given. It turns out that there are several possible interpretations
all enjoying the requisite algebraic properties of the operator (see
\cite{milner91polyadicpi}). We will therefore make liberal use of
$(\nu\; \vec{x})P$.

% subsection the_syntax_and_semantics_of_the_notation_system (end)   

\input{qm2pi.qmops} 

\input{qm2pi.sterngerlach} 

\input{qm2pi.metric} 

% section concurrent_process_calculi (end)

%\input{qm2pi.proofsketch}

% section proof sketch (end)

%\input{qm2pi.slviaknots} 

% section spatial logic via knots (end)

\input{qm2pi.conclusion}

% section conclusion (end)

%\input{qm2pi.dtcodes} 

% section wiring algorithm (end)

\input{qm2pi.ack} 

% section acknowledgments (end)

\newpage


\bibliographystyle{plain}   
\bibliography{../../biblios/main.bib}

\input{qm2pi.rhodetails}

\end{document}



\end{document}

 

\documentclass[12pt]{llncs}
%\documentclass{jktr}

\usepackage[pdftex]{hyperref}                   
\usepackage {listings}
\usepackage {mathpartir}
\usepackage{bcprules}
%\usepackage{listings}
                       
\usepackage{graphicx} 
%\usepackage[margins=2.5cm,nohead,nofoot]{geometry}
%\usepackage{geometry}
\usepackage{amsfonts}
\usepackage{amstext}
\usepackage{latexsym}
\usepackage{amssymb}
\usepackage{color}


%\include{myPreamble}
\documentclass[12pt]{llncs}
%\documentclass{jktr}

\usepackage[pdftex]{hyperref}                   
\usepackage {listings}
\usepackage {mathpartir}
\usepackage{bcprules}
%\usepackage{listings}
                       
\usepackage{graphicx} 
%\usepackage[margins=2.5cm,nohead,nofoot]{geometry}
%\usepackage{geometry}
\usepackage{amsfonts}
\usepackage{amstext}
\usepackage{latexsym}
\usepackage{amssymb}
\usepackage{color}


%\include{myPreamble}
\include{qm2pi.local} 

%\ifpdf
%\usepackage[pdftex]{graphicx}
%\else
%\usepackage{graphicx}
%\fi

 % \ifpdf
%  \usepackage{pdfsync}
%  \if


%\title{Brief Article}
%\author{David F. Snyder}
%\author{L.G. Meredith}

%\address{Dept. of Math., Texas State University--San Marcos, San Marcos, TX 78666}
       
\pagestyle{empty}


\begin{document}

\lstset{language=[Objective]Caml,frame=shadowbox}

\input{qm2pi.front}

% section front matter (end)

\input{qm2pi.intro} 
 
% section introduction (end)

% \input{qm2pi.knotations} 

% section notation (end)

\input{qm2pi.process.calculi} 

% section concurrent_process_calculi_and_spatial_logics_ (end)
    
%\input{qm2pi.knots2pi} 

%\input{qm2pi.trefoil} 

%\input{qm2pi.mainthm} 

% subsection basic_interpretation (end)

%\input{qm2pi.rho.presentation} 
\subsection{The syntax and semantics of the notation system}\label{sub:the_syntax_and_semantics_of_the_notation_system} % (fold)

We now summarize a technical presentation of the calculus that
embodies our theory of dynamics. The typical presentation of such a
calculus follows the style of giving generators and relations on
them. The grammar, below, describing term constructors, freely
generates the set of processes, $\Proc$. This set is then quotiented
by a relation known as structural congruence and it is over this set
that the notion of dynamics is expressed. This presentation is
essentially that of \cite{MeredithR05} with the addition of
polyadicity and summation. For readability we have relegated some of
the technical subtleties to an appendix.

\subsubsection{Process grammar}\label{subsub:process_grammar}

\begin{mathpar}
  \inferrule* [lab=synchronization] {} {{M} \bc \pzero \;|\; x?F \;|\; x!C }
  \and
  \inferrule* [lab=abstraction] {} {{F} \bc (x)P}
  \and
  \inferrule* [lab=concretion] {} {{C} \bc \langle Q \rangle}
  \and
  \inferrule* [lab=process] {} {{P,Q} \bc M \;| \;P|Q \;|\; @{x}}
  \and
  \inferrule* [lab=name] {} {{x} \bc \quotep{P}}
\end{mathpar} 

Note that $\vec{x}$ (resp. $\vec{P}$) denotes a vector of names
(resp. processes) of length $|\vec{x}|$ (resp. $|\vec{P}|$). We adopt
the following useful abbreviations.

\begin{mathpar}
   x?(\vec{y}).P := x.(\vec{y})P \and  x\clift{\vec{P}} := x.\clift{\vec{P}}
   \and x!(y) := \lift{x}{\dropn{y}}
   \and \Pi_{i=0}^{n-1}P_i := P_0 | \ldots | P_{n-1}
\end{mathpar}

\subsubsection{Structural congruence}

\paragraph{Free and bound names and alpha-equivalence.} At the
core of structural equivalence is alpha-equivalence which identifies
process that are the same up to a change of variable. Formally, we
recognize the distinction between free and bound names. The free names
of a process, $\freenames{P}$, may be calculated recursively as
follows:

\begin{mathpar}
\freenames{\pzero} := \emptyset
  \and \\
  \freenames{x?(y).P} := \{ x \} \cup (\freenames{P} \setminus \{ y \})
  \and 
  \freenames{x!\langle P \rangle} := \{ x \} \cup \{ P \} 
  \and \\
  \freenames{P|Q} := \freenames{P} \cup \freenames{Q}
  \and \\
  \freenames{@{x}} := \{ x \}
\end{mathpar}

$\pi$
$\quotep{\pi}$

$\freenames{-} : \pi \to \mathcal{P}(\quotep{\pi})$

\begin{eqnarray*}
  \freenames{\pzero} & := & \emptyset \\
  \freenames{x?(y).P} & := & \{ x \} \cup (\freenames{P} \setminus \{ y \}) \\
  \freenames{x!\langle P \rangle} & := & \{ x \} \cup \{ P \} \\
  \freenames{P|Q} & := & \freenames{P} \cup \freenames{Q} \\
  \freenames{\dropn{x}} & := & \{ x \}
\end{eqnarray*}

The bound names of a process, $\boundnames{P}$, are those names occurring in $P$
that are not free. For example, in $x?(y).0$, the name $x$ is free, while $y$ is bound.

\begin{mathpar}
  \inferrule* [lab=monoidal-laws] {} { P|Q \equiv Q|P \and P|0 \equiv P \and P|(Q|R) \equiv (P|Q)|R }
\end{mathpar}

\begin{mathpar}
  \inferrule* [lab=alpha-equivalence] {} { (x)P \equiv (y)P\{y/x\} \and y \not\in \freenames{P} }
\end{mathpar}

\begin{definition}
Then two processes, $P,Q$, are alpha-equivalent if $P = Q\{\vec{y}/\vec{x}\}$ for
some $\vec{x} \in \boundnames{Q},\vec{y} \in \boundnames{P}$, where $Q\{\vec{y}/\vec{x}\}$
denotes the capture-avoiding substitution of $\vec{y}$ for $\vec{x}$ in $Q$.
\end{definition}

\begin{definition}
  The {\em structural congruence} \cite{SangiorgiWalker} , $\equiv$,
  between processes is the least congruence containing
  alpha-equivalence, satisfying the abelian monoid laws
  (associativity, commutativity and $\pzero$ as identity) for parallel
  composition $|$ and for summation $+$.
\end{definition}

\subsection{Name equivalence}

We take name equivalence, written $\nameeq$, to be the smallest
equivalence relation generated by the following rules.

\begin{mathpar}
\inferrule*[lab=Quote-drop]
{ }
{ \quotep{@{x}} \nameeq x }

\inferrule*[lab=Struct-equiv]
{ P \scong Q }
{ \quotep{P} \nameeq \quotep{Q} }
\end{mathpar}

The astute reader will have noticed that the mutual recursion of names
and processes imposes a mutual recursion on alpha-equivalence and
structural equivalence via name-equivalence. Fortunately, all of this
works out pleasantly and we may calculate in the natural way, free of
concern. The reader interested in the details is referred to the
appendix \ref{appendix:rho_details}.

\subsection{Substitution}

We use $\Proc$ for the set of processes, $\QProc$ for the set of
names, and $\id{\{}\vec{y} / \vec{x} \id{\}}$ to denote partial maps,
$s : \QProc \rightarrow \QProc$. A map, $s$ lifts, uniquely, to a map
on process terms, $\widehat{s} : \Proc \rightarrow \Proc$ by the
following equations.

\begin{mathpar}
  (0) \psubstp{Q}{P} := 0 \\
  (R \juxtap S) \psubstp{Q}{P}
  :=    
  (R)\psubstp{Q}{P} \juxtap (S) \psubstp{Q}{P} \\
  (x?(y).R) \psubstp{Q}{P}    
  :=    
  (x)\substp{Q}{P} (z)\concat( (R \psubstn{z}{y}) \psubstp{Q}{P} ) \\
  (\lift{x}{R}) \psubstp{Q}{P}  
  :=
  \lift{(x)\substp{Q}{P}}{ R \psubstp{Q}{P} } \\
%   (\dropn{x})  \psubstp{Q}{P}       
%   := 
%   \left\{ 
%     \begin{array}{ccc} 
%       \dropn{\quotep{Q}} & & x \nameeq \quotep{P} \\
%       \dropn{x} & & otherwise \\
%     \end{array}
%   \right. 
  (\dropn{x})  \psubstp{Q}{P}       
  := 
  \left\{ 
    \begin{array}{ccc} 
      Q & & x \nameeq \quotep{P} \\
      \dropn{x} & & otherwise \\
    \end{array}
  \right.
\end{mathpar}
 

where

\begin{eqnarray}
  (x)\id{\{} \lpquote Q \rpquote / \lpquote P \rpquote \id{\}}            = 
  \left\{ 
    \begin{array}{ccc}
      \lpquote Q \rpquote & & x \nameeq \lpquote P \rpquote \\
      x & & otherwise \\
    \end{array}
  \right. \nonumber
\end{eqnarray}

and $z$ is chosen distinct from $\quotep{P}$, $\quotep{Q}$, the free
names in $Q$, and all the names in $R$. Our $\alpha$-equivalence will
be built in the standard way from this substitution.

\begin{remark}\label{rem:no_self_referential_names}
  One consequence of these definitions is that $\forall P. \quotep{P}
  \not\in \freenames{P}$.
\end{remark}

\subsection{ Dynamic quote: an example }

Anticipating something of what's to come, consider applying the
substitution, $\widehat{\id{\{}u / z \id{\}}}$, to the following pair
of processes, $\lift{w}{y!(z)}$ and $w[ \lpquote y!(z) \rpquote ]$.

\begin{eqnarray}
	\lift{w}{y!(z)}\widehat{\id{\{}u / z \id{\}}}
		& = &
		\lift{w}{y!(u)} \nonumber\\
	w[ \lpquote y!(z) \rpquote ] \widehat{ \id{\{}u / z \id{\}} }
		& = &
		w[ \lpquote y!(z) \rpquote ] \nonumber
\end{eqnarray}

Because the body of the process between quotes is impervious to
substitution, we get radically different answers. In fact, by
examining the first process in an input context,
e.g. $x?(z).\lift{w}{y!(z)}$, we see that the process under the lift
operator may be shaped by prefixed inputs binding a name inside it. In
this sense, the lift operator will be seen as a way to dynamically
construct processes before reifying them as names.

Finally equipped with these standard features we can present the
dynamics of the calculus.

\subsubsection{Operational semantics} 

Finally, we introduce the computational dynamics. What marks these
algebras as distinct from other more traditionally studied algebraic
structures, e.g. vector spaces or polynomial rings, is the manner in
which dynamics is captured. In traditional structures, dynamics is typically
expressed through morphisms between such structures, as in linear maps
between vector spaces or morphisms between rings. In algebras
associated with the semantics of computation, the dynamics is
expressed as part of the algebraic structure itself, through a
reduction reduction relation typically denoted by $\red$. Below, we
give a recursive presentation of this relation for the calculus used
in the encoding.

$\red \subseteq \pi \times \pi$
$\red : \pi \to \mathcal{P}(\pi)$

\begin{mathpar}
  \inferrule* [lab=Comm] { \textsf{match}( x_{src}, x_{trgt} ) } { x_{trgt}?(y)P \; | \; x_{src}!\langle {Q} \rangle \red P\{\quotep{Q}/y}\} }
  \and \\
  \inferrule* [lab=Par] {{P} \red {P}'} {{{P} | {Q}} \red {{P}' | {Q}}}
  \and
  \inferrule* [lab=Equiv]{{{P} \scong {P}'} \andalso {{P}' \red {Q}'} \andalso {{Q}' \scong {Q}}}{{P} \red {Q}}
\end{mathpar}

\begin{eqnarray*}
  match_{\equiv} (\quotep{P},\quotep{Q}) & := & P \equiv Q \\
  match_{\dagger}(\quotep{P},\quotep{Q}) & := & \forall R. P|Q \red^{*} R => R \red^{*} 0 \\
  match_{K}(\quotep{P},\quotep{Q}) & := & K \mbox{ for some context } K
\end{eqnarray*}

$u?(x)P | u!\langle Q \rangle \red P\{\quotep{Q}/x\}$

%We write $\wred$ for $\red^*$, and $P\red$ if $\exists Q $ such that $ P \red Q$.
We write $P\red$ if $\exists Q $ such that $ P \red Q$ and $P\not\red$, otherwise.

\section{Replication}

As mentioned before, it is known that replication (and hence
recursion) can be implemented in a higher-order process algebra
\cite{SangiorgiWalker}. As our first example of calculation with the
machinery thus far presented we give the construction explicitly in
the {\rhoc}.

\begin{eqnarray}
	D_{x} & := & \prefix{x}{y}{(\binpar{\outputp{x}{y}}{@{y}})} \nonumber\\
	\bangp_{x}{P} & := & \binpar{{x}!\langle{\binpar{D_{x}}{P}}\rangle}{D_{x}} \nonumber
\end{eqnarray}

\begin{eqnarray}
	\bangp_{x}{P} & & \nonumber\\
	=
	& {x}!\langle{(\prefix{x}{y}{(\outputp{x}{y} | @{y})) | P}}\rangle 
	      | \prefix{x}{y}{(\outputp{x}{y} | @{y})} & \nonumber\\
	\red
	& (\outputp{x}{y} | @{y})\substn{\quotep{(\prefix{x}{y}{(@{y} | \outputp{x}{y})) | P}}}{y} & \nonumber\\
	=
	& \outputp{x}{\quotep{(\prefix{x}{y}{(\outputp{x}{y} | @{y})) | P}}}
	  | {(\prefix{x}{y}{(\outputp{x}{y} | @{y})) | P}} & \nonumber\\
	\red
	& \ldots & \nonumber\\
	\red^*
	& P | P | \ldots & \nonumber
\end{eqnarray}

Of course, this encoding, as an implementation, runs away, unfolding
$\bangp{P}$ eagerly. A lazier and more implementable replication
operator, restricted to input-guarded processes, may be obtained as follows.

\begin{eqnarray}
\bangp{\prefix{u}{v}{P}} 
	:= 
	\binpar{\lift{x}{\prefix{u}{v}{(\binpar{D(x)}{P})}}}{D(x)} \nonumber
\end{eqnarray}

\begin{remark}
  Note that the lazier definition still does not deal with summation
  or mixed summation (i.e. sums over input and output). The reader is
  invited to construct definitions of replication that deal with these
  features. 

  Further, the definitions are parameterized in a name, $x$. Can you,
  gentle reader, make a definition that eliminates this parameter and
  guarantees no accidental interaction between the replication
  machinery and the process being replicated -- i.e. no accidental
  sharing of names used by the process to get its work done and the
  name(s) used by the replication to effect copying. This latter
  revision of the definition of replication is crucial to obtaining
  the expected identity $!!P \sim !P$.
\end{remark}

\begin{remark}\label{rem:paradoxical_combinator}
  The reader familiar with the lambda calculus will have noticed the
  similarity between $D$ and the paradoxical combinator.

  [Ed. note: the existence of this seems to suggest we have to be more
  restrictive on the set of processes and names we admit if we are to
  support no-cloning.]
\end{remark}

\subsubsection{Bisimulation}

The computational dynamics gives rise to another kind of equivalence,
the equivalence of computational behavior. As previously mentioned
this is typically captured \emph{via} some form of bisimulation.

% The notion we use in this paper is weak barbed bisimulation
% \cite{milner91polyadicpi}.

The notion we use in this paper is derived from weak barbed
bisimulation \cite{milner91polyadicpi}. 

\begin{definition}
An \emph{observation relation}, $\downarrow_{\mathcal N}$, over a set
of names, $\mathcal N$, is the smallest relation satisfying the rules
below.

\infrule[Out-barb]{y \in {\mathcal N}, \; x \nameeq y}
		  {\outputp{x}{v} \downarrow_{\mathcal N} x}
\infrule[Par-barb]{\mbox{$P\downarrow_{\mathcal N} x$ or $Q\downarrow_{\mathcal N} x$}}
		  {\binpar{P}{Q} \downarrow_{\mathcal N} x}

We write $P \Downarrow_{\mathcal N} x$ if there is $Q$ such that 
$P \wred Q$ and $Q \downarrow_{\mathcal N} x$.
\end{definition}

\begin{definition}
%\label{def.bbisim}
An  ${\mathcal N}$-\emph{barbed bisimulation} over a set of names, ${\mathcal N}$, is a symmetric binary relation 
${\mathcal S}_{\mathcal N}$ between agents such that $P\rel{S}_{\mathcal N}Q$ implies:
\begin{enumerate}
\item If $P \red P'$ then $Q \wred Q'$ and $P'\rel{S}_{\mathcal N} Q'$.
\item If $P\downarrow_{\mathcal N} x$, then $Q\Downarrow_{\mathcal N} x$.
\end{enumerate}
$P$ is ${\mathcal N}$-barbed bisimilar to $Q$, written
$P \wbbisim_{\mathcal N} Q$, if $P \rel{S}_{\mathcal N} Q$ for some ${\mathcal N}$-barbed bisimulation ${\mathcal S}_{\mathcal N}$.
\end{definition}

$\mathcal{R} \subseteq \pi \times \pi$

$P \mathcal{R} Q => \forall P'. P \red P' \Rightarrow \exists Q'. Q \red Q', P' \mathcal{R} Q'$

$P \vdash x \Rightarrow Q \vdash x$

\begin{mathpar}
  \inferrule*[lab=Out-barb]{x \nameeq y}{{y}!\langle{Q}\rangle \vdash x}
  \and
  \inferrule*[lab=Par-barb]{\mbox{$P\vdash x$ or $Q\vdash x$}}{\binpar{P}{Q} \vdash x}
\end{mathpar}

\subsubsection{Contexts}

One of the principle advantages of computational calculi like the
$\pi$-calculus is a well-defined notion of context,
contextual-equivalence and a correlation between
contextual-equivalence and notions of bisimulation. The notion of
context allows the decomposition of a process into (sub-)process and
its syntactic environment, its context. Thus, a context may be
thought of as a process with a ``hole'' (written $\Box$) in it. The
application of a context $M$ to a process $P$, written $M[P]$, is
tantamount to filling the hole in $M$ with $P$. In this paper we do
not need the full weight of this theory, but do make use of the notion
of context in the proof the main theorem. 

\begin{mathpar}
  \inferrule* [lab=summation] {} {{M_{M},M_{N}} \bc \Box \;|\; x.M_{A} \;|\; M_{M}+M_{N}}
  \and
  \inferrule* [lab=agent] {} {{M_{A}} \bc (\vec{x})M_{P} \;| \; \clift{P_0,\ldots,M_{P},\ldots,P_N}}
  \and \\
  \inferrule* [lab=process] {} {{M_{P}} \bc M_{N} \;| \;P|M_{P} }
\end{mathpar} 

\begin{mathpar}
  \inferrule* [lab=sychronization] {} {M_{N} \bc \Box \;|\; x?M_{F} \;|\; x!M_{C}}
  \and
  \inferrule* [lab=abstraction] {} {{M_{F}} \bc (x)M_{P} }
  \and
  \inferrule* [lab=concretion] {} {{M_{C}} \bc \langle M_{P} \rangle }
  \and \\
  \inferrule* [lab=process] {} {{M_{P}} \bc M_{N} \;| \;P|M_{P} }
\end{mathpar}

\begin{definition}[contextual application] Given a context $M$, and
  process $P$, we define the \emph{contextual application}, $M[P] :=
  M\{P/\Box\}$. That is, the contextual application of M to P is the
  substitution of $P$ for $\Box$ in $M$.
\end{definition}

$\meaningof{-} : L \to \mathcal{P}(\pi)$

\begin{mathpar}
  \inferrule* [lab=collection] {} {\meaningof{true} = \pi, \and \meaningof{~E} = \pi \setminus \meaningof{E}, \and \meaningof{E_{1} \& E_{2}} = \meaningof{E_{1}} \cap \meaningof{E_{2}}}
\end{mathpar}

\begin{mathpar}
  \inferrule* [lab=structure] {} {\meaningof{0} = \{ P \in \pi | P \equiv 0 \}, \and \\ \meaningof{E_1 | E_2} = \{ P \in \pi | P \equiv P_{1} | P_{2}, P_{1} \in \meaningof{E_{1}}, P_{2} \in \meaningof{E_2}\} }
\end{mathpar}

\begin{mathpar}
 \inferrule* [lab=behavior] {} {\meaningof{\langle a?b \rangle E} = \{ P \in \pi | P \equiv Q | u?(y)P', \\ \and \\\\ \and \\ \;\;\; u \in \meaningof{a}, \forall z.P'\{z/y\} \in \meaningof{E\{z/b\}}\}, \and \\ \meaningof{a!E} = \{ P \in \pi | P \equiv Q | x!\langle P' \rangle, x \in \meaningof{a} P' \in \meaningof{E}\} }
\end{mathpar}

\begin{mathpar}
 \inferrule* [lab=nominal] {} {\meaningof{\quotep{E}} = \{ \quotep{P} \in \quotep{\pi} | P \in \meaningof{E} \}, \and \meaningof{\quotep{P}} = \{ \quotep{Q} \in \quotep{\pi} | P \equiv Q \} \and \\ \meaningof{@\quotep{E}} = \{ P \in \pi | P \equiv @x, x \in \meaningof{E} \}}
\end{mathpar}

\begin{eqnarray*}
  \\
  \meaningof{-} : TS \to ST
\end{eqnarray*}

\begin{eqnarray*}
  \\
  L : TS \to ST
\end{eqnarray*}

\begin{eqnarray*}
  \\
  P \models E \iff P \in \meaningof{E}
\end{eqnarray*}

\begin{eqnarray*}
  P \approx_{L} Q \iff \forall E \in L. P \models E \iff Q \models E
\end{eqnarray*}

\begin{eqnarray*}
  P \approx_{K} Q
\end{eqnarray*}

\begin{eqnarray*}
  P \approx Q
\end{eqnarray*}

$\approx_{K} = \approx = \approx_{L}$

\subsubsection{Contextual duality}

Note that contexts extend the quotation operation to a family of
operations from processes to names. Given a context, $M$, we can
define a \emph{nominal context}, $\quotep{M}$ by $\quotep{M}[P] :=
\quotep{M[P]}$. To foreshadow what is to come we observe that these
operations enjoy a duality with processes very much like the duality
between vectors and maps from vectors to scalars.

Further, because the calculus is essentially higher-order, we have a
correspondence between contexts and processes. More specifically,
given a name $x$ and a context $M$ we can construct $M^{*}_{x}$ such
that 

\begin{mathpar}
  M^{*}_{x} | \lift{x}{P} \red M[P]
\end{mathpar}

namely,

\begin{mathpar}
  M^{*}_{x} := x?(u).M[\dropn{u}]
\end{mathpar}

The dependence of $M^{*}_{x}$ on a name makes it an abstraction, 

\begin{mathpar}
  M^{*} := (x)x?(u).M[\dropn{u}]
\end{mathpar}

\subsection{Additional notation}

It will sometimes be convenient to denote the process a name
quotes. We already have the notation $x = \quotep{P}$, but it will be
convenient to introduce an alternate notation, $\procn{x}$, when we
want to emphasize the connection to the use of the name. Note that, by
virtue of name equivalence, $\quotep{\procn{x}} \nameeq x$; so, the
notation is consistent with previous definitions.

Further, because names have structure it is possible to effect
substitutions on the basis of that structure. This means we need to
upgrade our notation for substitutions, which we accomplish by
adapting comprehension notation. Thus,

\begin{mathpar}
  P\{ y / x : x \in S \}
\end{mathpar}

is interpreted to mean the process derived from P by replacing (in a
capture-avoiding manner) each occurrence of $x$ in $S$ by $y$. For example,

\begin{mathpar}
  P\{ \quotep{\procn{x}|\procn{x}} / x : x \in \freenames{P} \}
\end{mathpar}

will replace each (occurrence) of a free name $x$ in $P$ by
$\quotep{\procn{x}|\procn{x}}$.

Also, we will avail ourselves of the notation $x^{L}$ and $x^{R}$ to
denote injections of a name into disjoint copies of the name
space. There are numerous ways to accomplish this. One example can be
found in \cite{MeredithR05}. This notation overloads to vectors of
names: $\vec{x}^{\pi} := (x_{i}^{\pi} \; : \; 0 \leq i < |\vec{x}| )$ where $\pi \in \{L,R\}$.

We also use $P^{\Box} := P|\Box$.

In \cite{MeredithR05} an interpretation of the new operator is
given. It turns out that there are several possible interpretations
all enjoying the requisite algebraic properties of the operator (see
\cite{milner91polyadicpi}). We will therefore make liberal use of
$(\nu\; \vec{x})P$.

% subsection the_syntax_and_semantics_of_the_notation_system (end)   

\input{qm2pi.qmops} 

\input{qm2pi.sterngerlach} 

\input{qm2pi.metric} 

% section concurrent_process_calculi (end)

%\input{qm2pi.proofsketch}

% section proof sketch (end)

%\input{qm2pi.slviaknots} 

% section spatial logic via knots (end)

\input{qm2pi.conclusion}

% section conclusion (end)

%\input{qm2pi.dtcodes} 

% section wiring algorithm (end)

\input{qm2pi.ack} 

% section acknowledgments (end)

\newpage


\bibliographystyle{plain}   
\bibliography{../../biblios/main.bib}

\input{qm2pi.rhodetails}

\end{document}

 

%\ifpdf
%\usepackage[pdftex]{graphicx}
%\else
%\usepackage{graphicx}
%\fi

 % \ifpdf
%  \usepackage{pdfsync}
%  \if


%\title{Brief Article}
%\author{David F. Snyder}
%\author{L.G. Meredith}

%\address{Dept. of Math., Texas State University--San Marcos, San Marcos, TX 78666}
       
\pagestyle{empty}


\begin{document}

\lstset{language=[Objective]Caml,frame=shadowbox}

\documentclass[12pt]{llncs}
%\documentclass{jktr}

\usepackage[pdftex]{hyperref}                   
\usepackage {listings}
\usepackage {mathpartir}
\usepackage{bcprules}
%\usepackage{listings}
                       
\usepackage{graphicx} 
%\usepackage[margins=2.5cm,nohead,nofoot]{geometry}
%\usepackage{geometry}
\usepackage{amsfonts}
\usepackage{amstext}
\usepackage{latexsym}
\usepackage{amssymb}
\usepackage{color}


%\include{myPreamble}
\include{qm2pi.local} 

%\ifpdf
%\usepackage[pdftex]{graphicx}
%\else
%\usepackage{graphicx}
%\fi

 % \ifpdf
%  \usepackage{pdfsync}
%  \if


%\title{Brief Article}
%\author{David F. Snyder}
%\author{L.G. Meredith}

%\address{Dept. of Math., Texas State University--San Marcos, San Marcos, TX 78666}
       
\pagestyle{empty}


\begin{document}

\lstset{language=[Objective]Caml,frame=shadowbox}

\input{qm2pi.front}

% section front matter (end)

\input{qm2pi.intro} 
 
% section introduction (end)

% \input{qm2pi.knotations} 

% section notation (end)

\input{qm2pi.process.calculi} 

% section concurrent_process_calculi_and_spatial_logics_ (end)
    
%\input{qm2pi.knots2pi} 

%\input{qm2pi.trefoil} 

%\input{qm2pi.mainthm} 

% subsection basic_interpretation (end)

%\input{qm2pi.rho.presentation} 
\subsection{The syntax and semantics of the notation system}\label{sub:the_syntax_and_semantics_of_the_notation_system} % (fold)

We now summarize a technical presentation of the calculus that
embodies our theory of dynamics. The typical presentation of such a
calculus follows the style of giving generators and relations on
them. The grammar, below, describing term constructors, freely
generates the set of processes, $\Proc$. This set is then quotiented
by a relation known as structural congruence and it is over this set
that the notion of dynamics is expressed. This presentation is
essentially that of \cite{MeredithR05} with the addition of
polyadicity and summation. For readability we have relegated some of
the technical subtleties to an appendix.

\subsubsection{Process grammar}\label{subsub:process_grammar}

\begin{mathpar}
  \inferrule* [lab=synchronization] {} {{M} \bc \pzero \;|\; x?F \;|\; x!C }
  \and
  \inferrule* [lab=abstraction] {} {{F} \bc (x)P}
  \and
  \inferrule* [lab=concretion] {} {{C} \bc \langle Q \rangle}
  \and
  \inferrule* [lab=process] {} {{P,Q} \bc M \;| \;P|Q \;|\; @{x}}
  \and
  \inferrule* [lab=name] {} {{x} \bc \quotep{P}}
\end{mathpar} 

Note that $\vec{x}$ (resp. $\vec{P}$) denotes a vector of names
(resp. processes) of length $|\vec{x}|$ (resp. $|\vec{P}|$). We adopt
the following useful abbreviations.

\begin{mathpar}
   x?(\vec{y}).P := x.(\vec{y})P \and  x\clift{\vec{P}} := x.\clift{\vec{P}}
   \and x!(y) := \lift{x}{\dropn{y}}
   \and \Pi_{i=0}^{n-1}P_i := P_0 | \ldots | P_{n-1}
\end{mathpar}

\subsubsection{Structural congruence}

\paragraph{Free and bound names and alpha-equivalence.} At the
core of structural equivalence is alpha-equivalence which identifies
process that are the same up to a change of variable. Formally, we
recognize the distinction between free and bound names. The free names
of a process, $\freenames{P}$, may be calculated recursively as
follows:

\begin{mathpar}
\freenames{\pzero} := \emptyset
  \and \\
  \freenames{x?(y).P} := \{ x \} \cup (\freenames{P} \setminus \{ y \})
  \and 
  \freenames{x!\langle P \rangle} := \{ x \} \cup \{ P \} 
  \and \\
  \freenames{P|Q} := \freenames{P} \cup \freenames{Q}
  \and \\
  \freenames{@{x}} := \{ x \}
\end{mathpar}

$\pi$
$\quotep{\pi}$

$\freenames{-} : \pi \to \mathcal{P}(\quotep{\pi})$

\begin{eqnarray*}
  \freenames{\pzero} & := & \emptyset \\
  \freenames{x?(y).P} & := & \{ x \} \cup (\freenames{P} \setminus \{ y \}) \\
  \freenames{x!\langle P \rangle} & := & \{ x \} \cup \{ P \} \\
  \freenames{P|Q} & := & \freenames{P} \cup \freenames{Q} \\
  \freenames{\dropn{x}} & := & \{ x \}
\end{eqnarray*}

The bound names of a process, $\boundnames{P}$, are those names occurring in $P$
that are not free. For example, in $x?(y).0$, the name $x$ is free, while $y$ is bound.

\begin{mathpar}
  \inferrule* [lab=monoidal-laws] {} { P|Q \equiv Q|P \and P|0 \equiv P \and P|(Q|R) \equiv (P|Q)|R }
\end{mathpar}

\begin{mathpar}
  \inferrule* [lab=alpha-equivalence] {} { (x)P \equiv (y)P\{y/x\} \and y \not\in \freenames{P} }
\end{mathpar}

\begin{definition}
Then two processes, $P,Q$, are alpha-equivalent if $P = Q\{\vec{y}/\vec{x}\}$ for
some $\vec{x} \in \boundnames{Q},\vec{y} \in \boundnames{P}$, where $Q\{\vec{y}/\vec{x}\}$
denotes the capture-avoiding substitution of $\vec{y}$ for $\vec{x}$ in $Q$.
\end{definition}

\begin{definition}
  The {\em structural congruence} \cite{SangiorgiWalker} , $\equiv$,
  between processes is the least congruence containing
  alpha-equivalence, satisfying the abelian monoid laws
  (associativity, commutativity and $\pzero$ as identity) for parallel
  composition $|$ and for summation $+$.
\end{definition}

\subsection{Name equivalence}

We take name equivalence, written $\nameeq$, to be the smallest
equivalence relation generated by the following rules.

\begin{mathpar}
\inferrule*[lab=Quote-drop]
{ }
{ \quotep{@{x}} \nameeq x }

\inferrule*[lab=Struct-equiv]
{ P \scong Q }
{ \quotep{P} \nameeq \quotep{Q} }
\end{mathpar}

The astute reader will have noticed that the mutual recursion of names
and processes imposes a mutual recursion on alpha-equivalence and
structural equivalence via name-equivalence. Fortunately, all of this
works out pleasantly and we may calculate in the natural way, free of
concern. The reader interested in the details is referred to the
appendix \ref{appendix:rho_details}.

\subsection{Substitution}

We use $\Proc$ for the set of processes, $\QProc$ for the set of
names, and $\id{\{}\vec{y} / \vec{x} \id{\}}$ to denote partial maps,
$s : \QProc \rightarrow \QProc$. A map, $s$ lifts, uniquely, to a map
on process terms, $\widehat{s} : \Proc \rightarrow \Proc$ by the
following equations.

\begin{mathpar}
  (0) \psubstp{Q}{P} := 0 \\
  (R \juxtap S) \psubstp{Q}{P}
  :=    
  (R)\psubstp{Q}{P} \juxtap (S) \psubstp{Q}{P} \\
  (x?(y).R) \psubstp{Q}{P}    
  :=    
  (x)\substp{Q}{P} (z)\concat( (R \psubstn{z}{y}) \psubstp{Q}{P} ) \\
  (\lift{x}{R}) \psubstp{Q}{P}  
  :=
  \lift{(x)\substp{Q}{P}}{ R \psubstp{Q}{P} } \\
%   (\dropn{x})  \psubstp{Q}{P}       
%   := 
%   \left\{ 
%     \begin{array}{ccc} 
%       \dropn{\quotep{Q}} & & x \nameeq \quotep{P} \\
%       \dropn{x} & & otherwise \\
%     \end{array}
%   \right. 
  (\dropn{x})  \psubstp{Q}{P}       
  := 
  \left\{ 
    \begin{array}{ccc} 
      Q & & x \nameeq \quotep{P} \\
      \dropn{x} & & otherwise \\
    \end{array}
  \right.
\end{mathpar}
 

where

\begin{eqnarray}
  (x)\id{\{} \lpquote Q \rpquote / \lpquote P \rpquote \id{\}}            = 
  \left\{ 
    \begin{array}{ccc}
      \lpquote Q \rpquote & & x \nameeq \lpquote P \rpquote \\
      x & & otherwise \\
    \end{array}
  \right. \nonumber
\end{eqnarray}

and $z$ is chosen distinct from $\quotep{P}$, $\quotep{Q}$, the free
names in $Q$, and all the names in $R$. Our $\alpha$-equivalence will
be built in the standard way from this substitution.

\begin{remark}\label{rem:no_self_referential_names}
  One consequence of these definitions is that $\forall P. \quotep{P}
  \not\in \freenames{P}$.
\end{remark}

\subsection{ Dynamic quote: an example }

Anticipating something of what's to come, consider applying the
substitution, $\widehat{\id{\{}u / z \id{\}}}$, to the following pair
of processes, $\lift{w}{y!(z)}$ and $w[ \lpquote y!(z) \rpquote ]$.

\begin{eqnarray}
	\lift{w}{y!(z)}\widehat{\id{\{}u / z \id{\}}}
		& = &
		\lift{w}{y!(u)} \nonumber\\
	w[ \lpquote y!(z) \rpquote ] \widehat{ \id{\{}u / z \id{\}} }
		& = &
		w[ \lpquote y!(z) \rpquote ] \nonumber
\end{eqnarray}

Because the body of the process between quotes is impervious to
substitution, we get radically different answers. In fact, by
examining the first process in an input context,
e.g. $x?(z).\lift{w}{y!(z)}$, we see that the process under the lift
operator may be shaped by prefixed inputs binding a name inside it. In
this sense, the lift operator will be seen as a way to dynamically
construct processes before reifying them as names.

Finally equipped with these standard features we can present the
dynamics of the calculus.

\subsubsection{Operational semantics} 

Finally, we introduce the computational dynamics. What marks these
algebras as distinct from other more traditionally studied algebraic
structures, e.g. vector spaces or polynomial rings, is the manner in
which dynamics is captured. In traditional structures, dynamics is typically
expressed through morphisms between such structures, as in linear maps
between vector spaces or morphisms between rings. In algebras
associated with the semantics of computation, the dynamics is
expressed as part of the algebraic structure itself, through a
reduction reduction relation typically denoted by $\red$. Below, we
give a recursive presentation of this relation for the calculus used
in the encoding.

$\red \subseteq \pi \times \pi$
$\red : \pi \to \mathcal{P}(\pi)$

\begin{mathpar}
  \inferrule* [lab=Comm] { \textsf{match}( x_{src}, x_{trgt} ) } { x_{trgt}?(y)P \; | \; x_{src}!\langle {Q} \rangle \red P\{\quotep{Q}/y}\} }
  \and \\
  \inferrule* [lab=Par] {{P} \red {P}'} {{{P} | {Q}} \red {{P}' | {Q}}}
  \and
  \inferrule* [lab=Equiv]{{{P} \scong {P}'} \andalso {{P}' \red {Q}'} \andalso {{Q}' \scong {Q}}}{{P} \red {Q}}
\end{mathpar}

\begin{eqnarray*}
  match_{\equiv} (\quotep{P},\quotep{Q}) & := & P \equiv Q \\
  match_{\dagger}(\quotep{P},\quotep{Q}) & := & \forall R. P|Q \red^{*} R => R \red^{*} 0 \\
  match_{K}(\quotep{P},\quotep{Q}) & := & K \mbox{ for some context } K
\end{eqnarray*}

$u?(x)P | u!\langle Q \rangle \red P\{\quotep{Q}/x\}$

%We write $\wred$ for $\red^*$, and $P\red$ if $\exists Q $ such that $ P \red Q$.
We write $P\red$ if $\exists Q $ such that $ P \red Q$ and $P\not\red$, otherwise.

\section{Replication}

As mentioned before, it is known that replication (and hence
recursion) can be implemented in a higher-order process algebra
\cite{SangiorgiWalker}. As our first example of calculation with the
machinery thus far presented we give the construction explicitly in
the {\rhoc}.

\begin{eqnarray}
	D_{x} & := & \prefix{x}{y}{(\binpar{\outputp{x}{y}}{@{y}})} \nonumber\\
	\bangp_{x}{P} & := & \binpar{{x}!\langle{\binpar{D_{x}}{P}}\rangle}{D_{x}} \nonumber
\end{eqnarray}

\begin{eqnarray}
	\bangp_{x}{P} & & \nonumber\\
	=
	& {x}!\langle{(\prefix{x}{y}{(\outputp{x}{y} | @{y})) | P}}\rangle 
	      | \prefix{x}{y}{(\outputp{x}{y} | @{y})} & \nonumber\\
	\red
	& (\outputp{x}{y} | @{y})\substn{\quotep{(\prefix{x}{y}{(@{y} | \outputp{x}{y})) | P}}}{y} & \nonumber\\
	=
	& \outputp{x}{\quotep{(\prefix{x}{y}{(\outputp{x}{y} | @{y})) | P}}}
	  | {(\prefix{x}{y}{(\outputp{x}{y} | @{y})) | P}} & \nonumber\\
	\red
	& \ldots & \nonumber\\
	\red^*
	& P | P | \ldots & \nonumber
\end{eqnarray}

Of course, this encoding, as an implementation, runs away, unfolding
$\bangp{P}$ eagerly. A lazier and more implementable replication
operator, restricted to input-guarded processes, may be obtained as follows.

\begin{eqnarray}
\bangp{\prefix{u}{v}{P}} 
	:= 
	\binpar{\lift{x}{\prefix{u}{v}{(\binpar{D(x)}{P})}}}{D(x)} \nonumber
\end{eqnarray}

\begin{remark}
  Note that the lazier definition still does not deal with summation
  or mixed summation (i.e. sums over input and output). The reader is
  invited to construct definitions of replication that deal with these
  features. 

  Further, the definitions are parameterized in a name, $x$. Can you,
  gentle reader, make a definition that eliminates this parameter and
  guarantees no accidental interaction between the replication
  machinery and the process being replicated -- i.e. no accidental
  sharing of names used by the process to get its work done and the
  name(s) used by the replication to effect copying. This latter
  revision of the definition of replication is crucial to obtaining
  the expected identity $!!P \sim !P$.
\end{remark}

\begin{remark}\label{rem:paradoxical_combinator}
  The reader familiar with the lambda calculus will have noticed the
  similarity between $D$ and the paradoxical combinator.

  [Ed. note: the existence of this seems to suggest we have to be more
  restrictive on the set of processes and names we admit if we are to
  support no-cloning.]
\end{remark}

\subsubsection{Bisimulation}

The computational dynamics gives rise to another kind of equivalence,
the equivalence of computational behavior. As previously mentioned
this is typically captured \emph{via} some form of bisimulation.

% The notion we use in this paper is weak barbed bisimulation
% \cite{milner91polyadicpi}.

The notion we use in this paper is derived from weak barbed
bisimulation \cite{milner91polyadicpi}. 

\begin{definition}
An \emph{observation relation}, $\downarrow_{\mathcal N}$, over a set
of names, $\mathcal N$, is the smallest relation satisfying the rules
below.

\infrule[Out-barb]{y \in {\mathcal N}, \; x \nameeq y}
		  {\outputp{x}{v} \downarrow_{\mathcal N} x}
\infrule[Par-barb]{\mbox{$P\downarrow_{\mathcal N} x$ or $Q\downarrow_{\mathcal N} x$}}
		  {\binpar{P}{Q} \downarrow_{\mathcal N} x}

We write $P \Downarrow_{\mathcal N} x$ if there is $Q$ such that 
$P \wred Q$ and $Q \downarrow_{\mathcal N} x$.
\end{definition}

\begin{definition}
%\label{def.bbisim}
An  ${\mathcal N}$-\emph{barbed bisimulation} over a set of names, ${\mathcal N}$, is a symmetric binary relation 
${\mathcal S}_{\mathcal N}$ between agents such that $P\rel{S}_{\mathcal N}Q$ implies:
\begin{enumerate}
\item If $P \red P'$ then $Q \wred Q'$ and $P'\rel{S}_{\mathcal N} Q'$.
\item If $P\downarrow_{\mathcal N} x$, then $Q\Downarrow_{\mathcal N} x$.
\end{enumerate}
$P$ is ${\mathcal N}$-barbed bisimilar to $Q$, written
$P \wbbisim_{\mathcal N} Q$, if $P \rel{S}_{\mathcal N} Q$ for some ${\mathcal N}$-barbed bisimulation ${\mathcal S}_{\mathcal N}$.
\end{definition}

$\mathcal{R} \subseteq \pi \times \pi$

$P \mathcal{R} Q => \forall P'. P \red P' \Rightarrow \exists Q'. Q \red Q', P' \mathcal{R} Q'$

$P \vdash x \Rightarrow Q \vdash x$

\begin{mathpar}
  \inferrule*[lab=Out-barb]{x \nameeq y}{{y}!\langle{Q}\rangle \vdash x}
  \and
  \inferrule*[lab=Par-barb]{\mbox{$P\vdash x$ or $Q\vdash x$}}{\binpar{P}{Q} \vdash x}
\end{mathpar}

\subsubsection{Contexts}

One of the principle advantages of computational calculi like the
$\pi$-calculus is a well-defined notion of context,
contextual-equivalence and a correlation between
contextual-equivalence and notions of bisimulation. The notion of
context allows the decomposition of a process into (sub-)process and
its syntactic environment, its context. Thus, a context may be
thought of as a process with a ``hole'' (written $\Box$) in it. The
application of a context $M$ to a process $P$, written $M[P]$, is
tantamount to filling the hole in $M$ with $P$. In this paper we do
not need the full weight of this theory, but do make use of the notion
of context in the proof the main theorem. 

\begin{mathpar}
  \inferrule* [lab=summation] {} {{M_{M},M_{N}} \bc \Box \;|\; x.M_{A} \;|\; M_{M}+M_{N}}
  \and
  \inferrule* [lab=agent] {} {{M_{A}} \bc (\vec{x})M_{P} \;| \; \clift{P_0,\ldots,M_{P},\ldots,P_N}}
  \and \\
  \inferrule* [lab=process] {} {{M_{P}} \bc M_{N} \;| \;P|M_{P} }
\end{mathpar} 

\begin{mathpar}
  \inferrule* [lab=sychronization] {} {M_{N} \bc \Box \;|\; x?M_{F} \;|\; x!M_{C}}
  \and
  \inferrule* [lab=abstraction] {} {{M_{F}} \bc (x)M_{P} }
  \and
  \inferrule* [lab=concretion] {} {{M_{C}} \bc \langle M_{P} \rangle }
  \and \\
  \inferrule* [lab=process] {} {{M_{P}} \bc M_{N} \;| \;P|M_{P} }
\end{mathpar}

\begin{definition}[contextual application] Given a context $M$, and
  process $P$, we define the \emph{contextual application}, $M[P] :=
  M\{P/\Box\}$. That is, the contextual application of M to P is the
  substitution of $P$ for $\Box$ in $M$.
\end{definition}

$\meaningof{-} : L \to \mathcal{P}(\pi)$

\begin{mathpar}
  \inferrule* [lab=collection] {} {\meaningof{true} = \pi, \and \meaningof{~E} = \pi \setminus \meaningof{E}, \and \meaningof{E_{1} \& E_{2}} = \meaningof{E_{1}} \cap \meaningof{E_{2}}}
\end{mathpar}

\begin{mathpar}
  \inferrule* [lab=structure] {} {\meaningof{0} = \{ P \in \pi | P \equiv 0 \}, \and \\ \meaningof{E_1 | E_2} = \{ P \in \pi | P \equiv P_{1} | P_{2}, P_{1} \in \meaningof{E_{1}}, P_{2} \in \meaningof{E_2}\} }
\end{mathpar}

\begin{mathpar}
 \inferrule* [lab=behavior] {} {\meaningof{\langle a?b \rangle E} = \{ P \in \pi | P \equiv Q | u?(y)P', \\ \and \\\\ \and \\ \;\;\; u \in \meaningof{a}, \forall z.P'\{z/y\} \in \meaningof{E\{z/b\}}\}, \and \\ \meaningof{a!E} = \{ P \in \pi | P \equiv Q | x!\langle P' \rangle, x \in \meaningof{a} P' \in \meaningof{E}\} }
\end{mathpar}

\begin{mathpar}
 \inferrule* [lab=nominal] {} {\meaningof{\quotep{E}} = \{ \quotep{P} \in \quotep{\pi} | P \in \meaningof{E} \}, \and \meaningof{\quotep{P}} = \{ \quotep{Q} \in \quotep{\pi} | P \equiv Q \} \and \\ \meaningof{@\quotep{E}} = \{ P \in \pi | P \equiv @x, x \in \meaningof{E} \}}
\end{mathpar}

\begin{eqnarray*}
  \\
  \meaningof{-} : TS \to ST
\end{eqnarray*}

\begin{eqnarray*}
  \\
  L : TS \to ST
\end{eqnarray*}

\begin{eqnarray*}
  \\
  P \models E \iff P \in \meaningof{E}
\end{eqnarray*}

\begin{eqnarray*}
  P \approx_{L} Q \iff \forall E \in L. P \models E \iff Q \models E
\end{eqnarray*}

\begin{eqnarray*}
  P \approx_{K} Q
\end{eqnarray*}

\begin{eqnarray*}
  P \approx Q
\end{eqnarray*}

$\approx_{K} = \approx = \approx_{L}$

\subsubsection{Contextual duality}

Note that contexts extend the quotation operation to a family of
operations from processes to names. Given a context, $M$, we can
define a \emph{nominal context}, $\quotep{M}$ by $\quotep{M}[P] :=
\quotep{M[P]}$. To foreshadow what is to come we observe that these
operations enjoy a duality with processes very much like the duality
between vectors and maps from vectors to scalars.

Further, because the calculus is essentially higher-order, we have a
correspondence between contexts and processes. More specifically,
given a name $x$ and a context $M$ we can construct $M^{*}_{x}$ such
that 

\begin{mathpar}
  M^{*}_{x} | \lift{x}{P} \red M[P]
\end{mathpar}

namely,

\begin{mathpar}
  M^{*}_{x} := x?(u).M[\dropn{u}]
\end{mathpar}

The dependence of $M^{*}_{x}$ on a name makes it an abstraction, 

\begin{mathpar}
  M^{*} := (x)x?(u).M[\dropn{u}]
\end{mathpar}

\subsection{Additional notation}

It will sometimes be convenient to denote the process a name
quotes. We already have the notation $x = \quotep{P}$, but it will be
convenient to introduce an alternate notation, $\procn{x}$, when we
want to emphasize the connection to the use of the name. Note that, by
virtue of name equivalence, $\quotep{\procn{x}} \nameeq x$; so, the
notation is consistent with previous definitions.

Further, because names have structure it is possible to effect
substitutions on the basis of that structure. This means we need to
upgrade our notation for substitutions, which we accomplish by
adapting comprehension notation. Thus,

\begin{mathpar}
  P\{ y / x : x \in S \}
\end{mathpar}

is interpreted to mean the process derived from P by replacing (in a
capture-avoiding manner) each occurrence of $x$ in $S$ by $y$. For example,

\begin{mathpar}
  P\{ \quotep{\procn{x}|\procn{x}} / x : x \in \freenames{P} \}
\end{mathpar}

will replace each (occurrence) of a free name $x$ in $P$ by
$\quotep{\procn{x}|\procn{x}}$.

Also, we will avail ourselves of the notation $x^{L}$ and $x^{R}$ to
denote injections of a name into disjoint copies of the name
space. There are numerous ways to accomplish this. One example can be
found in \cite{MeredithR05}. This notation overloads to vectors of
names: $\vec{x}^{\pi} := (x_{i}^{\pi} \; : \; 0 \leq i < |\vec{x}| )$ where $\pi \in \{L,R\}$.

We also use $P^{\Box} := P|\Box$.

In \cite{MeredithR05} an interpretation of the new operator is
given. It turns out that there are several possible interpretations
all enjoying the requisite algebraic properties of the operator (see
\cite{milner91polyadicpi}). We will therefore make liberal use of
$(\nu\; \vec{x})P$.

% subsection the_syntax_and_semantics_of_the_notation_system (end)   

\input{qm2pi.qmops} 

\input{qm2pi.sterngerlach} 

\input{qm2pi.metric} 

% section concurrent_process_calculi (end)

%\input{qm2pi.proofsketch}

% section proof sketch (end)

%\input{qm2pi.slviaknots} 

% section spatial logic via knots (end)

\input{qm2pi.conclusion}

% section conclusion (end)

%\input{qm2pi.dtcodes} 

% section wiring algorithm (end)

\input{qm2pi.ack} 

% section acknowledgments (end)

\newpage


\bibliographystyle{plain}   
\bibliography{../../biblios/main.bib}

\input{qm2pi.rhodetails}

\end{document}



% section front matter (end)

\section{Introduction}\label{sec:introduction} % (fold)
In this draft of the material i am going to have to dispense with the
usual writing conventions adopted in papers on these topics. i'm going
to have adopt whatever tone i need at the time i'm writing up the
calculations. Sometimes this may be very conversational; others it may
be the barest mathematical grunts; others still it may be that i have
lifted text from one of my other papers because the exposition of some
point was better said there. i hope that my readers are not unduly put
out by this decision. i'm not doing this to flout convention or be
rebellious. i find these calculations very technically challenging. To
keep everything going technically, something has to give; i have to
let go of some cognitive burden. So, the academic writing style --
with all of its trade-offs in terms of facilitating technical
communication -- is what i'm letting go of. Perhaps subsequent drafts
can be tightened and polished, but for now, i'm going to speak as if
we were sitting together in a coffee shop with a laptop, wifi and a
pad of paper and a pencil.

So, here's what i have to say. We -- you and i, comfortably ensconced
in our coffee shop and well-equipped with our tools -- can realize and
carry out the calculations of quantum mechanics over a very different
formal theory of dynamics, a formal theory of dynamics that
corresponds to a theory of concurrent computation with
\emph{reflection}. It has the advantage that the underlying theory is
already `quantized', but supports analogues all of the continuuous
operations. Strikingly, this underlying theory has recently been
connected with a notion of metric that we can show, by calculating
together, coincides with the metric induced by the inner product.

There are a lot of reasons why you might be interested in seeing
calculations of this form. Here's why i'm interested. For the past
several centuries there has been no competitor to the ``Newtonian''
account of dynamics. As a result the predominant share of accounts of
dynamical systems and situations have had to be formulated in terms of
the Newtonian machinery. i view this as an intellectually dangerous
position to occupy. Everything, despite it's intrinsic shape, turns
into a nail to be hit with this hammer. Recently, however, the theory
of computation has matured to the point where we have candidates for
theories of dynamics that offer very different perspective on
reasoning about dynamical systems and situations. Testing these
candidates against very successful accounts of dynamical situations,
like quantum mechanics, is going to give us some sense of how mature
they are and some measure of the quality of these accounts of
dynamics.

\subsection{Summary of contributions and outline of paper}

So, we're going to develop an interpretation of the operations of
quantum mechanics normally interpreted by Hilbert spaces and
operators. We're going to do this over a theory of computation. Note
that this is very different than the usual quantum computation program
which develops notions of computation over quantum mechanics. Rather,
we are developing a story that aligns with Wheeler's slogan: It from
Bit. To do this we will first provide an account of the theory of
computation at play here. Then we will dive into a calculation-driven
interpretation of the operations of quantum mechanics.

The reason we take this approach is that -- until very recently --
there hasn't been an axiomatic account of quantum mechanics. As a
result there has been no sharp delineation of the mathematical theory
supporting interpretation of the physical theory and the physical
theory, itself. So, ambient features of the maths are free to be
exploited (or supressed) without a real accounting of their physical
relevance. There is no sharp statement ``here's the physical theory''
qua \emph{theory} and ``here's the mathematical interpretation''
enabling a judgment of how faithful the interpretation is -- apart
from experimental observation. When there is an axiomatic account we
can judge how well a given mathematical formalism supports an
interpretation of the axioms, independent of
experimentation. Likewise, we can judge how well we have captured our
physical evidence and experience with our axiomatics, independent of
any specific mathematical implementation, with accidental detail that
may or may not have physical significance. 

In lieu of a fully fleshed out and vetted axiomatic account of quantum
mechanics, interpreting the operational notions in service of modeling
physical systems will have to suffice. In other words, we are not in
the business of providing a model of Hilbert spaces and operators. We
are in the business of providing a model of quantum mechanics because
we are motivated by testing our notions of dynamics against physical
theory; and, the predictive calculations of the physical theory must
serve as the best formulation -- shy of a fully fleshed out axiomatic
account -- of the physical theory itself (as they have for scientific
theories since time immemorial). Put another way, despite a
whole-hearted commitment to an It-from-Bit ontology, we are firmly
aligned with the shut-up-and-calculate camp as the best way to obtain
results either from the physical perspective or as a quality assurance
measure of our fledgling theory of dynamics.

In detail, we present a reflective process calculus. Then we develop
intuitive correspondences between the notions available in this
calculus and the usual physical notions supporting quantum mechanical
calculations. Thus, 

\begin{table}[htp]
  \center{
    \fbox{
      \begin{tabular}{c|c}
        quantum mechanics & process calculus \\
        \hline
        scalar & name \\
        state vector & process \\
        dual & contextual duals \\
        matrix & formal sums of process-context-dual pairs \\
        orthogonality & process annihilation \\
        inner product & execution-formula + quoting
      \end{tabular}
    }
  }
  \caption{QM - process calculi correspondences}
\end{table}

Then we tighten up these intuitions to operational definitions. We
employ the Dirac notation as the best proxy we can find for an
abstract syntax of the quantum mechanical notions. The definitions we
develop put us in contact with equational constraints coming from the
theory that we demonstrate the definitions and calculations satisfy.

This puts us in a position to shut up and calculate for the
Stern-Gerlach experimental set up, showing how these predictive
calculations become calculations on processes in our theory of a
reflective process calculus.

Penultimately, we demonstrate that the notion of metric coming from
the inner product coincides with the notion of metric available from
the theory of bisimulation. This demonstration gives us the right to
think of space as arising from behavior. Finally, we consider where we
might go from the new vantage point we have obtained.

% section introduction (end) 
 
% section introduction (end)

% \documentclass[12pt]{llncs}
%\documentclass{jktr}

\usepackage[pdftex]{hyperref}                   
\usepackage {listings}
\usepackage {mathpartir}
\usepackage{bcprules}
%\usepackage{listings}
                       
\usepackage{graphicx} 
%\usepackage[margins=2.5cm,nohead,nofoot]{geometry}
%\usepackage{geometry}
\usepackage{amsfonts}
\usepackage{amstext}
\usepackage{latexsym}
\usepackage{amssymb}
\usepackage{color}


%\include{myPreamble}
\include{qm2pi.local} 

%\ifpdf
%\usepackage[pdftex]{graphicx}
%\else
%\usepackage{graphicx}
%\fi

 % \ifpdf
%  \usepackage{pdfsync}
%  \if


%\title{Brief Article}
%\author{David F. Snyder}
%\author{L.G. Meredith}

%\address{Dept. of Math., Texas State University--San Marcos, San Marcos, TX 78666}
       
\pagestyle{empty}


\begin{document}

\lstset{language=[Objective]Caml,frame=shadowbox}

\input{qm2pi.front}

% section front matter (end)

\input{qm2pi.intro} 
 
% section introduction (end)

% \input{qm2pi.knotations} 

% section notation (end)

\input{qm2pi.process.calculi} 

% section concurrent_process_calculi_and_spatial_logics_ (end)
    
%\input{qm2pi.knots2pi} 

%\input{qm2pi.trefoil} 

%\input{qm2pi.mainthm} 

% subsection basic_interpretation (end)

%\input{qm2pi.rho.presentation} 
\subsection{The syntax and semantics of the notation system}\label{sub:the_syntax_and_semantics_of_the_notation_system} % (fold)

We now summarize a technical presentation of the calculus that
embodies our theory of dynamics. The typical presentation of such a
calculus follows the style of giving generators and relations on
them. The grammar, below, describing term constructors, freely
generates the set of processes, $\Proc$. This set is then quotiented
by a relation known as structural congruence and it is over this set
that the notion of dynamics is expressed. This presentation is
essentially that of \cite{MeredithR05} with the addition of
polyadicity and summation. For readability we have relegated some of
the technical subtleties to an appendix.

\subsubsection{Process grammar}\label{subsub:process_grammar}

\begin{mathpar}
  \inferrule* [lab=synchronization] {} {{M} \bc \pzero \;|\; x?F \;|\; x!C }
  \and
  \inferrule* [lab=abstraction] {} {{F} \bc (x)P}
  \and
  \inferrule* [lab=concretion] {} {{C} \bc \langle Q \rangle}
  \and
  \inferrule* [lab=process] {} {{P,Q} \bc M \;| \;P|Q \;|\; @{x}}
  \and
  \inferrule* [lab=name] {} {{x} \bc \quotep{P}}
\end{mathpar} 

Note that $\vec{x}$ (resp. $\vec{P}$) denotes a vector of names
(resp. processes) of length $|\vec{x}|$ (resp. $|\vec{P}|$). We adopt
the following useful abbreviations.

\begin{mathpar}
   x?(\vec{y}).P := x.(\vec{y})P \and  x\clift{\vec{P}} := x.\clift{\vec{P}}
   \and x!(y) := \lift{x}{\dropn{y}}
   \and \Pi_{i=0}^{n-1}P_i := P_0 | \ldots | P_{n-1}
\end{mathpar}

\subsubsection{Structural congruence}

\paragraph{Free and bound names and alpha-equivalence.} At the
core of structural equivalence is alpha-equivalence which identifies
process that are the same up to a change of variable. Formally, we
recognize the distinction between free and bound names. The free names
of a process, $\freenames{P}$, may be calculated recursively as
follows:

\begin{mathpar}
\freenames{\pzero} := \emptyset
  \and \\
  \freenames{x?(y).P} := \{ x \} \cup (\freenames{P} \setminus \{ y \})
  \and 
  \freenames{x!\langle P \rangle} := \{ x \} \cup \{ P \} 
  \and \\
  \freenames{P|Q} := \freenames{P} \cup \freenames{Q}
  \and \\
  \freenames{@{x}} := \{ x \}
\end{mathpar}

$\pi$
$\quotep{\pi}$

$\freenames{-} : \pi \to \mathcal{P}(\quotep{\pi})$

\begin{eqnarray*}
  \freenames{\pzero} & := & \emptyset \\
  \freenames{x?(y).P} & := & \{ x \} \cup (\freenames{P} \setminus \{ y \}) \\
  \freenames{x!\langle P \rangle} & := & \{ x \} \cup \{ P \} \\
  \freenames{P|Q} & := & \freenames{P} \cup \freenames{Q} \\
  \freenames{\dropn{x}} & := & \{ x \}
\end{eqnarray*}

The bound names of a process, $\boundnames{P}$, are those names occurring in $P$
that are not free. For example, in $x?(y).0$, the name $x$ is free, while $y$ is bound.

\begin{mathpar}
  \inferrule* [lab=monoidal-laws] {} { P|Q \equiv Q|P \and P|0 \equiv P \and P|(Q|R) \equiv (P|Q)|R }
\end{mathpar}

\begin{mathpar}
  \inferrule* [lab=alpha-equivalence] {} { (x)P \equiv (y)P\{y/x\} \and y \not\in \freenames{P} }
\end{mathpar}

\begin{definition}
Then two processes, $P,Q$, are alpha-equivalent if $P = Q\{\vec{y}/\vec{x}\}$ for
some $\vec{x} \in \boundnames{Q},\vec{y} \in \boundnames{P}$, where $Q\{\vec{y}/\vec{x}\}$
denotes the capture-avoiding substitution of $\vec{y}$ for $\vec{x}$ in $Q$.
\end{definition}

\begin{definition}
  The {\em structural congruence} \cite{SangiorgiWalker} , $\equiv$,
  between processes is the least congruence containing
  alpha-equivalence, satisfying the abelian monoid laws
  (associativity, commutativity and $\pzero$ as identity) for parallel
  composition $|$ and for summation $+$.
\end{definition}

\subsection{Name equivalence}

We take name equivalence, written $\nameeq$, to be the smallest
equivalence relation generated by the following rules.

\begin{mathpar}
\inferrule*[lab=Quote-drop]
{ }
{ \quotep{@{x}} \nameeq x }

\inferrule*[lab=Struct-equiv]
{ P \scong Q }
{ \quotep{P} \nameeq \quotep{Q} }
\end{mathpar}

The astute reader will have noticed that the mutual recursion of names
and processes imposes a mutual recursion on alpha-equivalence and
structural equivalence via name-equivalence. Fortunately, all of this
works out pleasantly and we may calculate in the natural way, free of
concern. The reader interested in the details is referred to the
appendix \ref{appendix:rho_details}.

\subsection{Substitution}

We use $\Proc$ for the set of processes, $\QProc$ for the set of
names, and $\id{\{}\vec{y} / \vec{x} \id{\}}$ to denote partial maps,
$s : \QProc \rightarrow \QProc$. A map, $s$ lifts, uniquely, to a map
on process terms, $\widehat{s} : \Proc \rightarrow \Proc$ by the
following equations.

\begin{mathpar}
  (0) \psubstp{Q}{P} := 0 \\
  (R \juxtap S) \psubstp{Q}{P}
  :=    
  (R)\psubstp{Q}{P} \juxtap (S) \psubstp{Q}{P} \\
  (x?(y).R) \psubstp{Q}{P}    
  :=    
  (x)\substp{Q}{P} (z)\concat( (R \psubstn{z}{y}) \psubstp{Q}{P} ) \\
  (\lift{x}{R}) \psubstp{Q}{P}  
  :=
  \lift{(x)\substp{Q}{P}}{ R \psubstp{Q}{P} } \\
%   (\dropn{x})  \psubstp{Q}{P}       
%   := 
%   \left\{ 
%     \begin{array}{ccc} 
%       \dropn{\quotep{Q}} & & x \nameeq \quotep{P} \\
%       \dropn{x} & & otherwise \\
%     \end{array}
%   \right. 
  (\dropn{x})  \psubstp{Q}{P}       
  := 
  \left\{ 
    \begin{array}{ccc} 
      Q & & x \nameeq \quotep{P} \\
      \dropn{x} & & otherwise \\
    \end{array}
  \right.
\end{mathpar}
 

where

\begin{eqnarray}
  (x)\id{\{} \lpquote Q \rpquote / \lpquote P \rpquote \id{\}}            = 
  \left\{ 
    \begin{array}{ccc}
      \lpquote Q \rpquote & & x \nameeq \lpquote P \rpquote \\
      x & & otherwise \\
    \end{array}
  \right. \nonumber
\end{eqnarray}

and $z$ is chosen distinct from $\quotep{P}$, $\quotep{Q}$, the free
names in $Q$, and all the names in $R$. Our $\alpha$-equivalence will
be built in the standard way from this substitution.

\begin{remark}\label{rem:no_self_referential_names}
  One consequence of these definitions is that $\forall P. \quotep{P}
  \not\in \freenames{P}$.
\end{remark}

\subsection{ Dynamic quote: an example }

Anticipating something of what's to come, consider applying the
substitution, $\widehat{\id{\{}u / z \id{\}}}$, to the following pair
of processes, $\lift{w}{y!(z)}$ and $w[ \lpquote y!(z) \rpquote ]$.

\begin{eqnarray}
	\lift{w}{y!(z)}\widehat{\id{\{}u / z \id{\}}}
		& = &
		\lift{w}{y!(u)} \nonumber\\
	w[ \lpquote y!(z) \rpquote ] \widehat{ \id{\{}u / z \id{\}} }
		& = &
		w[ \lpquote y!(z) \rpquote ] \nonumber
\end{eqnarray}

Because the body of the process between quotes is impervious to
substitution, we get radically different answers. In fact, by
examining the first process in an input context,
e.g. $x?(z).\lift{w}{y!(z)}$, we see that the process under the lift
operator may be shaped by prefixed inputs binding a name inside it. In
this sense, the lift operator will be seen as a way to dynamically
construct processes before reifying them as names.

Finally equipped with these standard features we can present the
dynamics of the calculus.

\subsubsection{Operational semantics} 

Finally, we introduce the computational dynamics. What marks these
algebras as distinct from other more traditionally studied algebraic
structures, e.g. vector spaces or polynomial rings, is the manner in
which dynamics is captured. In traditional structures, dynamics is typically
expressed through morphisms between such structures, as in linear maps
between vector spaces or morphisms between rings. In algebras
associated with the semantics of computation, the dynamics is
expressed as part of the algebraic structure itself, through a
reduction reduction relation typically denoted by $\red$. Below, we
give a recursive presentation of this relation for the calculus used
in the encoding.

$\red \subseteq \pi \times \pi$
$\red : \pi \to \mathcal{P}(\pi)$

\begin{mathpar}
  \inferrule* [lab=Comm] { \textsf{match}( x_{src}, x_{trgt} ) } { x_{trgt}?(y)P \; | \; x_{src}!\langle {Q} \rangle \red P\{\quotep{Q}/y}\} }
  \and \\
  \inferrule* [lab=Par] {{P} \red {P}'} {{{P} | {Q}} \red {{P}' | {Q}}}
  \and
  \inferrule* [lab=Equiv]{{{P} \scong {P}'} \andalso {{P}' \red {Q}'} \andalso {{Q}' \scong {Q}}}{{P} \red {Q}}
\end{mathpar}

\begin{eqnarray*}
  match_{\equiv} (\quotep{P},\quotep{Q}) & := & P \equiv Q \\
  match_{\dagger}(\quotep{P},\quotep{Q}) & := & \forall R. P|Q \red^{*} R => R \red^{*} 0 \\
  match_{K}(\quotep{P},\quotep{Q}) & := & K \mbox{ for some context } K
\end{eqnarray*}

$u?(x)P | u!\langle Q \rangle \red P\{\quotep{Q}/x\}$

%We write $\wred$ for $\red^*$, and $P\red$ if $\exists Q $ such that $ P \red Q$.
We write $P\red$ if $\exists Q $ such that $ P \red Q$ and $P\not\red$, otherwise.

\section{Replication}

As mentioned before, it is known that replication (and hence
recursion) can be implemented in a higher-order process algebra
\cite{SangiorgiWalker}. As our first example of calculation with the
machinery thus far presented we give the construction explicitly in
the {\rhoc}.

\begin{eqnarray}
	D_{x} & := & \prefix{x}{y}{(\binpar{\outputp{x}{y}}{@{y}})} \nonumber\\
	\bangp_{x}{P} & := & \binpar{{x}!\langle{\binpar{D_{x}}{P}}\rangle}{D_{x}} \nonumber
\end{eqnarray}

\begin{eqnarray}
	\bangp_{x}{P} & & \nonumber\\
	=
	& {x}!\langle{(\prefix{x}{y}{(\outputp{x}{y} | @{y})) | P}}\rangle 
	      | \prefix{x}{y}{(\outputp{x}{y} | @{y})} & \nonumber\\
	\red
	& (\outputp{x}{y} | @{y})\substn{\quotep{(\prefix{x}{y}{(@{y} | \outputp{x}{y})) | P}}}{y} & \nonumber\\
	=
	& \outputp{x}{\quotep{(\prefix{x}{y}{(\outputp{x}{y} | @{y})) | P}}}
	  | {(\prefix{x}{y}{(\outputp{x}{y} | @{y})) | P}} & \nonumber\\
	\red
	& \ldots & \nonumber\\
	\red^*
	& P | P | \ldots & \nonumber
\end{eqnarray}

Of course, this encoding, as an implementation, runs away, unfolding
$\bangp{P}$ eagerly. A lazier and more implementable replication
operator, restricted to input-guarded processes, may be obtained as follows.

\begin{eqnarray}
\bangp{\prefix{u}{v}{P}} 
	:= 
	\binpar{\lift{x}{\prefix{u}{v}{(\binpar{D(x)}{P})}}}{D(x)} \nonumber
\end{eqnarray}

\begin{remark}
  Note that the lazier definition still does not deal with summation
  or mixed summation (i.e. sums over input and output). The reader is
  invited to construct definitions of replication that deal with these
  features. 

  Further, the definitions are parameterized in a name, $x$. Can you,
  gentle reader, make a definition that eliminates this parameter and
  guarantees no accidental interaction between the replication
  machinery and the process being replicated -- i.e. no accidental
  sharing of names used by the process to get its work done and the
  name(s) used by the replication to effect copying. This latter
  revision of the definition of replication is crucial to obtaining
  the expected identity $!!P \sim !P$.
\end{remark}

\begin{remark}\label{rem:paradoxical_combinator}
  The reader familiar with the lambda calculus will have noticed the
  similarity between $D$ and the paradoxical combinator.

  [Ed. note: the existence of this seems to suggest we have to be more
  restrictive on the set of processes and names we admit if we are to
  support no-cloning.]
\end{remark}

\subsubsection{Bisimulation}

The computational dynamics gives rise to another kind of equivalence,
the equivalence of computational behavior. As previously mentioned
this is typically captured \emph{via} some form of bisimulation.

% The notion we use in this paper is weak barbed bisimulation
% \cite{milner91polyadicpi}.

The notion we use in this paper is derived from weak barbed
bisimulation \cite{milner91polyadicpi}. 

\begin{definition}
An \emph{observation relation}, $\downarrow_{\mathcal N}$, over a set
of names, $\mathcal N$, is the smallest relation satisfying the rules
below.

\infrule[Out-barb]{y \in {\mathcal N}, \; x \nameeq y}
		  {\outputp{x}{v} \downarrow_{\mathcal N} x}
\infrule[Par-barb]{\mbox{$P\downarrow_{\mathcal N} x$ or $Q\downarrow_{\mathcal N} x$}}
		  {\binpar{P}{Q} \downarrow_{\mathcal N} x}

We write $P \Downarrow_{\mathcal N} x$ if there is $Q$ such that 
$P \wred Q$ and $Q \downarrow_{\mathcal N} x$.
\end{definition}

\begin{definition}
%\label{def.bbisim}
An  ${\mathcal N}$-\emph{barbed bisimulation} over a set of names, ${\mathcal N}$, is a symmetric binary relation 
${\mathcal S}_{\mathcal N}$ between agents such that $P\rel{S}_{\mathcal N}Q$ implies:
\begin{enumerate}
\item If $P \red P'$ then $Q \wred Q'$ and $P'\rel{S}_{\mathcal N} Q'$.
\item If $P\downarrow_{\mathcal N} x$, then $Q\Downarrow_{\mathcal N} x$.
\end{enumerate}
$P$ is ${\mathcal N}$-barbed bisimilar to $Q$, written
$P \wbbisim_{\mathcal N} Q$, if $P \rel{S}_{\mathcal N} Q$ for some ${\mathcal N}$-barbed bisimulation ${\mathcal S}_{\mathcal N}$.
\end{definition}

$\mathcal{R} \subseteq \pi \times \pi$

$P \mathcal{R} Q => \forall P'. P \red P' \Rightarrow \exists Q'. Q \red Q', P' \mathcal{R} Q'$

$P \vdash x \Rightarrow Q \vdash x$

\begin{mathpar}
  \inferrule*[lab=Out-barb]{x \nameeq y}{{y}!\langle{Q}\rangle \vdash x}
  \and
  \inferrule*[lab=Par-barb]{\mbox{$P\vdash x$ or $Q\vdash x$}}{\binpar{P}{Q} \vdash x}
\end{mathpar}

\subsubsection{Contexts}

One of the principle advantages of computational calculi like the
$\pi$-calculus is a well-defined notion of context,
contextual-equivalence and a correlation between
contextual-equivalence and notions of bisimulation. The notion of
context allows the decomposition of a process into (sub-)process and
its syntactic environment, its context. Thus, a context may be
thought of as a process with a ``hole'' (written $\Box$) in it. The
application of a context $M$ to a process $P$, written $M[P]$, is
tantamount to filling the hole in $M$ with $P$. In this paper we do
not need the full weight of this theory, but do make use of the notion
of context in the proof the main theorem. 

\begin{mathpar}
  \inferrule* [lab=summation] {} {{M_{M},M_{N}} \bc \Box \;|\; x.M_{A} \;|\; M_{M}+M_{N}}
  \and
  \inferrule* [lab=agent] {} {{M_{A}} \bc (\vec{x})M_{P} \;| \; \clift{P_0,\ldots,M_{P},\ldots,P_N}}
  \and \\
  \inferrule* [lab=process] {} {{M_{P}} \bc M_{N} \;| \;P|M_{P} }
\end{mathpar} 

\begin{mathpar}
  \inferrule* [lab=sychronization] {} {M_{N} \bc \Box \;|\; x?M_{F} \;|\; x!M_{C}}
  \and
  \inferrule* [lab=abstraction] {} {{M_{F}} \bc (x)M_{P} }
  \and
  \inferrule* [lab=concretion] {} {{M_{C}} \bc \langle M_{P} \rangle }
  \and \\
  \inferrule* [lab=process] {} {{M_{P}} \bc M_{N} \;| \;P|M_{P} }
\end{mathpar}

\begin{definition}[contextual application] Given a context $M$, and
  process $P$, we define the \emph{contextual application}, $M[P] :=
  M\{P/\Box\}$. That is, the contextual application of M to P is the
  substitution of $P$ for $\Box$ in $M$.
\end{definition}

$\meaningof{-} : L \to \mathcal{P}(\pi)$

\begin{mathpar}
  \inferrule* [lab=collection] {} {\meaningof{true} = \pi, \and \meaningof{~E} = \pi \setminus \meaningof{E}, \and \meaningof{E_{1} \& E_{2}} = \meaningof{E_{1}} \cap \meaningof{E_{2}}}
\end{mathpar}

\begin{mathpar}
  \inferrule* [lab=structure] {} {\meaningof{0} = \{ P \in \pi | P \equiv 0 \}, \and \\ \meaningof{E_1 | E_2} = \{ P \in \pi | P \equiv P_{1} | P_{2}, P_{1} \in \meaningof{E_{1}}, P_{2} \in \meaningof{E_2}\} }
\end{mathpar}

\begin{mathpar}
 \inferrule* [lab=behavior] {} {\meaningof{\langle a?b \rangle E} = \{ P \in \pi | P \equiv Q | u?(y)P', \\ \and \\\\ \and \\ \;\;\; u \in \meaningof{a}, \forall z.P'\{z/y\} \in \meaningof{E\{z/b\}}\}, \and \\ \meaningof{a!E} = \{ P \in \pi | P \equiv Q | x!\langle P' \rangle, x \in \meaningof{a} P' \in \meaningof{E}\} }
\end{mathpar}

\begin{mathpar}
 \inferrule* [lab=nominal] {} {\meaningof{\quotep{E}} = \{ \quotep{P} \in \quotep{\pi} | P \in \meaningof{E} \}, \and \meaningof{\quotep{P}} = \{ \quotep{Q} \in \quotep{\pi} | P \equiv Q \} \and \\ \meaningof{@\quotep{E}} = \{ P \in \pi | P \equiv @x, x \in \meaningof{E} \}}
\end{mathpar}

\begin{eqnarray*}
  \\
  \meaningof{-} : TS \to ST
\end{eqnarray*}

\begin{eqnarray*}
  \\
  L : TS \to ST
\end{eqnarray*}

\begin{eqnarray*}
  \\
  P \models E \iff P \in \meaningof{E}
\end{eqnarray*}

\begin{eqnarray*}
  P \approx_{L} Q \iff \forall E \in L. P \models E \iff Q \models E
\end{eqnarray*}

\begin{eqnarray*}
  P \approx_{K} Q
\end{eqnarray*}

\begin{eqnarray*}
  P \approx Q
\end{eqnarray*}

$\approx_{K} = \approx = \approx_{L}$

\subsubsection{Contextual duality}

Note that contexts extend the quotation operation to a family of
operations from processes to names. Given a context, $M$, we can
define a \emph{nominal context}, $\quotep{M}$ by $\quotep{M}[P] :=
\quotep{M[P]}$. To foreshadow what is to come we observe that these
operations enjoy a duality with processes very much like the duality
between vectors and maps from vectors to scalars.

Further, because the calculus is essentially higher-order, we have a
correspondence between contexts and processes. More specifically,
given a name $x$ and a context $M$ we can construct $M^{*}_{x}$ such
that 

\begin{mathpar}
  M^{*}_{x} | \lift{x}{P} \red M[P]
\end{mathpar}

namely,

\begin{mathpar}
  M^{*}_{x} := x?(u).M[\dropn{u}]
\end{mathpar}

The dependence of $M^{*}_{x}$ on a name makes it an abstraction, 

\begin{mathpar}
  M^{*} := (x)x?(u).M[\dropn{u}]
\end{mathpar}

\subsection{Additional notation}

It will sometimes be convenient to denote the process a name
quotes. We already have the notation $x = \quotep{P}$, but it will be
convenient to introduce an alternate notation, $\procn{x}$, when we
want to emphasize the connection to the use of the name. Note that, by
virtue of name equivalence, $\quotep{\procn{x}} \nameeq x$; so, the
notation is consistent with previous definitions.

Further, because names have structure it is possible to effect
substitutions on the basis of that structure. This means we need to
upgrade our notation for substitutions, which we accomplish by
adapting comprehension notation. Thus,

\begin{mathpar}
  P\{ y / x : x \in S \}
\end{mathpar}

is interpreted to mean the process derived from P by replacing (in a
capture-avoiding manner) each occurrence of $x$ in $S$ by $y$. For example,

\begin{mathpar}
  P\{ \quotep{\procn{x}|\procn{x}} / x : x \in \freenames{P} \}
\end{mathpar}

will replace each (occurrence) of a free name $x$ in $P$ by
$\quotep{\procn{x}|\procn{x}}$.

Also, we will avail ourselves of the notation $x^{L}$ and $x^{R}$ to
denote injections of a name into disjoint copies of the name
space. There are numerous ways to accomplish this. One example can be
found in \cite{MeredithR05}. This notation overloads to vectors of
names: $\vec{x}^{\pi} := (x_{i}^{\pi} \; : \; 0 \leq i < |\vec{x}| )$ where $\pi \in \{L,R\}$.

We also use $P^{\Box} := P|\Box$.

In \cite{MeredithR05} an interpretation of the new operator is
given. It turns out that there are several possible interpretations
all enjoying the requisite algebraic properties of the operator (see
\cite{milner91polyadicpi}). We will therefore make liberal use of
$(\nu\; \vec{x})P$.

% subsection the_syntax_and_semantics_of_the_notation_system (end)   

\input{qm2pi.qmops} 

\input{qm2pi.sterngerlach} 

\input{qm2pi.metric} 

% section concurrent_process_calculi (end)

%\input{qm2pi.proofsketch}

% section proof sketch (end)

%\input{qm2pi.slviaknots} 

% section spatial logic via knots (end)

\input{qm2pi.conclusion}

% section conclusion (end)

%\input{qm2pi.dtcodes} 

% section wiring algorithm (end)

\input{qm2pi.ack} 

% section acknowledgments (end)

\newpage


\bibliographystyle{plain}   
\bibliography{../../biblios/main.bib}

\input{qm2pi.rhodetails}

\end{document}

 

% section notation (end)

\input{qm2pi.process.calculi} 

% section concurrent_process_calculi_and_spatial_logics_ (end)
    
%\documentclass[12pt]{llncs}
%\documentclass{jktr}

\usepackage[pdftex]{hyperref}                   
\usepackage {listings}
\usepackage {mathpartir}
\usepackage{bcprules}
%\usepackage{listings}
                       
\usepackage{graphicx} 
%\usepackage[margins=2.5cm,nohead,nofoot]{geometry}
%\usepackage{geometry}
\usepackage{amsfonts}
\usepackage{amstext}
\usepackage{latexsym}
\usepackage{amssymb}
\usepackage{color}


%\include{myPreamble}
\include{qm2pi.local} 

%\ifpdf
%\usepackage[pdftex]{graphicx}
%\else
%\usepackage{graphicx}
%\fi

 % \ifpdf
%  \usepackage{pdfsync}
%  \if


%\title{Brief Article}
%\author{David F. Snyder}
%\author{L.G. Meredith}

%\address{Dept. of Math., Texas State University--San Marcos, San Marcos, TX 78666}
       
\pagestyle{empty}


\begin{document}

\lstset{language=[Objective]Caml,frame=shadowbox}

\input{qm2pi.front}

% section front matter (end)

\input{qm2pi.intro} 
 
% section introduction (end)

% \input{qm2pi.knotations} 

% section notation (end)

\input{qm2pi.process.calculi} 

% section concurrent_process_calculi_and_spatial_logics_ (end)
    
%\input{qm2pi.knots2pi} 

%\input{qm2pi.trefoil} 

%\input{qm2pi.mainthm} 

% subsection basic_interpretation (end)

%\input{qm2pi.rho.presentation} 
\subsection{The syntax and semantics of the notation system}\label{sub:the_syntax_and_semantics_of_the_notation_system} % (fold)

We now summarize a technical presentation of the calculus that
embodies our theory of dynamics. The typical presentation of such a
calculus follows the style of giving generators and relations on
them. The grammar, below, describing term constructors, freely
generates the set of processes, $\Proc$. This set is then quotiented
by a relation known as structural congruence and it is over this set
that the notion of dynamics is expressed. This presentation is
essentially that of \cite{MeredithR05} with the addition of
polyadicity and summation. For readability we have relegated some of
the technical subtleties to an appendix.

\subsubsection{Process grammar}\label{subsub:process_grammar}

\begin{mathpar}
  \inferrule* [lab=synchronization] {} {{M} \bc \pzero \;|\; x?F \;|\; x!C }
  \and
  \inferrule* [lab=abstraction] {} {{F} \bc (x)P}
  \and
  \inferrule* [lab=concretion] {} {{C} \bc \langle Q \rangle}
  \and
  \inferrule* [lab=process] {} {{P,Q} \bc M \;| \;P|Q \;|\; @{x}}
  \and
  \inferrule* [lab=name] {} {{x} \bc \quotep{P}}
\end{mathpar} 

Note that $\vec{x}$ (resp. $\vec{P}$) denotes a vector of names
(resp. processes) of length $|\vec{x}|$ (resp. $|\vec{P}|$). We adopt
the following useful abbreviations.

\begin{mathpar}
   x?(\vec{y}).P := x.(\vec{y})P \and  x\clift{\vec{P}} := x.\clift{\vec{P}}
   \and x!(y) := \lift{x}{\dropn{y}}
   \and \Pi_{i=0}^{n-1}P_i := P_0 | \ldots | P_{n-1}
\end{mathpar}

\subsubsection{Structural congruence}

\paragraph{Free and bound names and alpha-equivalence.} At the
core of structural equivalence is alpha-equivalence which identifies
process that are the same up to a change of variable. Formally, we
recognize the distinction between free and bound names. The free names
of a process, $\freenames{P}$, may be calculated recursively as
follows:

\begin{mathpar}
\freenames{\pzero} := \emptyset
  \and \\
  \freenames{x?(y).P} := \{ x \} \cup (\freenames{P} \setminus \{ y \})
  \and 
  \freenames{x!\langle P \rangle} := \{ x \} \cup \{ P \} 
  \and \\
  \freenames{P|Q} := \freenames{P} \cup \freenames{Q}
  \and \\
  \freenames{@{x}} := \{ x \}
\end{mathpar}

$\pi$
$\quotep{\pi}$

$\freenames{-} : \pi \to \mathcal{P}(\quotep{\pi})$

\begin{eqnarray*}
  \freenames{\pzero} & := & \emptyset \\
  \freenames{x?(y).P} & := & \{ x \} \cup (\freenames{P} \setminus \{ y \}) \\
  \freenames{x!\langle P \rangle} & := & \{ x \} \cup \{ P \} \\
  \freenames{P|Q} & := & \freenames{P} \cup \freenames{Q} \\
  \freenames{\dropn{x}} & := & \{ x \}
\end{eqnarray*}

The bound names of a process, $\boundnames{P}$, are those names occurring in $P$
that are not free. For example, in $x?(y).0$, the name $x$ is free, while $y$ is bound.

\begin{mathpar}
  \inferrule* [lab=monoidal-laws] {} { P|Q \equiv Q|P \and P|0 \equiv P \and P|(Q|R) \equiv (P|Q)|R }
\end{mathpar}

\begin{mathpar}
  \inferrule* [lab=alpha-equivalence] {} { (x)P \equiv (y)P\{y/x\} \and y \not\in \freenames{P} }
\end{mathpar}

\begin{definition}
Then two processes, $P,Q$, are alpha-equivalent if $P = Q\{\vec{y}/\vec{x}\}$ for
some $\vec{x} \in \boundnames{Q},\vec{y} \in \boundnames{P}$, where $Q\{\vec{y}/\vec{x}\}$
denotes the capture-avoiding substitution of $\vec{y}$ for $\vec{x}$ in $Q$.
\end{definition}

\begin{definition}
  The {\em structural congruence} \cite{SangiorgiWalker} , $\equiv$,
  between processes is the least congruence containing
  alpha-equivalence, satisfying the abelian monoid laws
  (associativity, commutativity and $\pzero$ as identity) for parallel
  composition $|$ and for summation $+$.
\end{definition}

\subsection{Name equivalence}

We take name equivalence, written $\nameeq$, to be the smallest
equivalence relation generated by the following rules.

\begin{mathpar}
\inferrule*[lab=Quote-drop]
{ }
{ \quotep{@{x}} \nameeq x }

\inferrule*[lab=Struct-equiv]
{ P \scong Q }
{ \quotep{P} \nameeq \quotep{Q} }
\end{mathpar}

The astute reader will have noticed that the mutual recursion of names
and processes imposes a mutual recursion on alpha-equivalence and
structural equivalence via name-equivalence. Fortunately, all of this
works out pleasantly and we may calculate in the natural way, free of
concern. The reader interested in the details is referred to the
appendix \ref{appendix:rho_details}.

\subsection{Substitution}

We use $\Proc$ for the set of processes, $\QProc$ for the set of
names, and $\id{\{}\vec{y} / \vec{x} \id{\}}$ to denote partial maps,
$s : \QProc \rightarrow \QProc$. A map, $s$ lifts, uniquely, to a map
on process terms, $\widehat{s} : \Proc \rightarrow \Proc$ by the
following equations.

\begin{mathpar}
  (0) \psubstp{Q}{P} := 0 \\
  (R \juxtap S) \psubstp{Q}{P}
  :=    
  (R)\psubstp{Q}{P} \juxtap (S) \psubstp{Q}{P} \\
  (x?(y).R) \psubstp{Q}{P}    
  :=    
  (x)\substp{Q}{P} (z)\concat( (R \psubstn{z}{y}) \psubstp{Q}{P} ) \\
  (\lift{x}{R}) \psubstp{Q}{P}  
  :=
  \lift{(x)\substp{Q}{P}}{ R \psubstp{Q}{P} } \\
%   (\dropn{x})  \psubstp{Q}{P}       
%   := 
%   \left\{ 
%     \begin{array}{ccc} 
%       \dropn{\quotep{Q}} & & x \nameeq \quotep{P} \\
%       \dropn{x} & & otherwise \\
%     \end{array}
%   \right. 
  (\dropn{x})  \psubstp{Q}{P}       
  := 
  \left\{ 
    \begin{array}{ccc} 
      Q & & x \nameeq \quotep{P} \\
      \dropn{x} & & otherwise \\
    \end{array}
  \right.
\end{mathpar}
 

where

\begin{eqnarray}
  (x)\id{\{} \lpquote Q \rpquote / \lpquote P \rpquote \id{\}}            = 
  \left\{ 
    \begin{array}{ccc}
      \lpquote Q \rpquote & & x \nameeq \lpquote P \rpquote \\
      x & & otherwise \\
    \end{array}
  \right. \nonumber
\end{eqnarray}

and $z$ is chosen distinct from $\quotep{P}$, $\quotep{Q}$, the free
names in $Q$, and all the names in $R$. Our $\alpha$-equivalence will
be built in the standard way from this substitution.

\begin{remark}\label{rem:no_self_referential_names}
  One consequence of these definitions is that $\forall P. \quotep{P}
  \not\in \freenames{P}$.
\end{remark}

\subsection{ Dynamic quote: an example }

Anticipating something of what's to come, consider applying the
substitution, $\widehat{\id{\{}u / z \id{\}}}$, to the following pair
of processes, $\lift{w}{y!(z)}$ and $w[ \lpquote y!(z) \rpquote ]$.

\begin{eqnarray}
	\lift{w}{y!(z)}\widehat{\id{\{}u / z \id{\}}}
		& = &
		\lift{w}{y!(u)} \nonumber\\
	w[ \lpquote y!(z) \rpquote ] \widehat{ \id{\{}u / z \id{\}} }
		& = &
		w[ \lpquote y!(z) \rpquote ] \nonumber
\end{eqnarray}

Because the body of the process between quotes is impervious to
substitution, we get radically different answers. In fact, by
examining the first process in an input context,
e.g. $x?(z).\lift{w}{y!(z)}$, we see that the process under the lift
operator may be shaped by prefixed inputs binding a name inside it. In
this sense, the lift operator will be seen as a way to dynamically
construct processes before reifying them as names.

Finally equipped with these standard features we can present the
dynamics of the calculus.

\subsubsection{Operational semantics} 

Finally, we introduce the computational dynamics. What marks these
algebras as distinct from other more traditionally studied algebraic
structures, e.g. vector spaces or polynomial rings, is the manner in
which dynamics is captured. In traditional structures, dynamics is typically
expressed through morphisms between such structures, as in linear maps
between vector spaces or morphisms between rings. In algebras
associated with the semantics of computation, the dynamics is
expressed as part of the algebraic structure itself, through a
reduction reduction relation typically denoted by $\red$. Below, we
give a recursive presentation of this relation for the calculus used
in the encoding.

$\red \subseteq \pi \times \pi$
$\red : \pi \to \mathcal{P}(\pi)$

\begin{mathpar}
  \inferrule* [lab=Comm] { \textsf{match}( x_{src}, x_{trgt} ) } { x_{trgt}?(y)P \; | \; x_{src}!\langle {Q} \rangle \red P\{\quotep{Q}/y}\} }
  \and \\
  \inferrule* [lab=Par] {{P} \red {P}'} {{{P} | {Q}} \red {{P}' | {Q}}}
  \and
  \inferrule* [lab=Equiv]{{{P} \scong {P}'} \andalso {{P}' \red {Q}'} \andalso {{Q}' \scong {Q}}}{{P} \red {Q}}
\end{mathpar}

\begin{eqnarray*}
  match_{\equiv} (\quotep{P},\quotep{Q}) & := & P \equiv Q \\
  match_{\dagger}(\quotep{P},\quotep{Q}) & := & \forall R. P|Q \red^{*} R => R \red^{*} 0 \\
  match_{K}(\quotep{P},\quotep{Q}) & := & K \mbox{ for some context } K
\end{eqnarray*}

$u?(x)P | u!\langle Q \rangle \red P\{\quotep{Q}/x\}$

%We write $\wred$ for $\red^*$, and $P\red$ if $\exists Q $ such that $ P \red Q$.
We write $P\red$ if $\exists Q $ such that $ P \red Q$ and $P\not\red$, otherwise.

\section{Replication}

As mentioned before, it is known that replication (and hence
recursion) can be implemented in a higher-order process algebra
\cite{SangiorgiWalker}. As our first example of calculation with the
machinery thus far presented we give the construction explicitly in
the {\rhoc}.

\begin{eqnarray}
	D_{x} & := & \prefix{x}{y}{(\binpar{\outputp{x}{y}}{@{y}})} \nonumber\\
	\bangp_{x}{P} & := & \binpar{{x}!\langle{\binpar{D_{x}}{P}}\rangle}{D_{x}} \nonumber
\end{eqnarray}

\begin{eqnarray}
	\bangp_{x}{P} & & \nonumber\\
	=
	& {x}!\langle{(\prefix{x}{y}{(\outputp{x}{y} | @{y})) | P}}\rangle 
	      | \prefix{x}{y}{(\outputp{x}{y} | @{y})} & \nonumber\\
	\red
	& (\outputp{x}{y} | @{y})\substn{\quotep{(\prefix{x}{y}{(@{y} | \outputp{x}{y})) | P}}}{y} & \nonumber\\
	=
	& \outputp{x}{\quotep{(\prefix{x}{y}{(\outputp{x}{y} | @{y})) | P}}}
	  | {(\prefix{x}{y}{(\outputp{x}{y} | @{y})) | P}} & \nonumber\\
	\red
	& \ldots & \nonumber\\
	\red^*
	& P | P | \ldots & \nonumber
\end{eqnarray}

Of course, this encoding, as an implementation, runs away, unfolding
$\bangp{P}$ eagerly. A lazier and more implementable replication
operator, restricted to input-guarded processes, may be obtained as follows.

\begin{eqnarray}
\bangp{\prefix{u}{v}{P}} 
	:= 
	\binpar{\lift{x}{\prefix{u}{v}{(\binpar{D(x)}{P})}}}{D(x)} \nonumber
\end{eqnarray}

\begin{remark}
  Note that the lazier definition still does not deal with summation
  or mixed summation (i.e. sums over input and output). The reader is
  invited to construct definitions of replication that deal with these
  features. 

  Further, the definitions are parameterized in a name, $x$. Can you,
  gentle reader, make a definition that eliminates this parameter and
  guarantees no accidental interaction between the replication
  machinery and the process being replicated -- i.e. no accidental
  sharing of names used by the process to get its work done and the
  name(s) used by the replication to effect copying. This latter
  revision of the definition of replication is crucial to obtaining
  the expected identity $!!P \sim !P$.
\end{remark}

\begin{remark}\label{rem:paradoxical_combinator}
  The reader familiar with the lambda calculus will have noticed the
  similarity between $D$ and the paradoxical combinator.

  [Ed. note: the existence of this seems to suggest we have to be more
  restrictive on the set of processes and names we admit if we are to
  support no-cloning.]
\end{remark}

\subsubsection{Bisimulation}

The computational dynamics gives rise to another kind of equivalence,
the equivalence of computational behavior. As previously mentioned
this is typically captured \emph{via} some form of bisimulation.

% The notion we use in this paper is weak barbed bisimulation
% \cite{milner91polyadicpi}.

The notion we use in this paper is derived from weak barbed
bisimulation \cite{milner91polyadicpi}. 

\begin{definition}
An \emph{observation relation}, $\downarrow_{\mathcal N}$, over a set
of names, $\mathcal N$, is the smallest relation satisfying the rules
below.

\infrule[Out-barb]{y \in {\mathcal N}, \; x \nameeq y}
		  {\outputp{x}{v} \downarrow_{\mathcal N} x}
\infrule[Par-barb]{\mbox{$P\downarrow_{\mathcal N} x$ or $Q\downarrow_{\mathcal N} x$}}
		  {\binpar{P}{Q} \downarrow_{\mathcal N} x}

We write $P \Downarrow_{\mathcal N} x$ if there is $Q$ such that 
$P \wred Q$ and $Q \downarrow_{\mathcal N} x$.
\end{definition}

\begin{definition}
%\label{def.bbisim}
An  ${\mathcal N}$-\emph{barbed bisimulation} over a set of names, ${\mathcal N}$, is a symmetric binary relation 
${\mathcal S}_{\mathcal N}$ between agents such that $P\rel{S}_{\mathcal N}Q$ implies:
\begin{enumerate}
\item If $P \red P'$ then $Q \wred Q'$ and $P'\rel{S}_{\mathcal N} Q'$.
\item If $P\downarrow_{\mathcal N} x$, then $Q\Downarrow_{\mathcal N} x$.
\end{enumerate}
$P$ is ${\mathcal N}$-barbed bisimilar to $Q$, written
$P \wbbisim_{\mathcal N} Q$, if $P \rel{S}_{\mathcal N} Q$ for some ${\mathcal N}$-barbed bisimulation ${\mathcal S}_{\mathcal N}$.
\end{definition}

$\mathcal{R} \subseteq \pi \times \pi$

$P \mathcal{R} Q => \forall P'. P \red P' \Rightarrow \exists Q'. Q \red Q', P' \mathcal{R} Q'$

$P \vdash x \Rightarrow Q \vdash x$

\begin{mathpar}
  \inferrule*[lab=Out-barb]{x \nameeq y}{{y}!\langle{Q}\rangle \vdash x}
  \and
  \inferrule*[lab=Par-barb]{\mbox{$P\vdash x$ or $Q\vdash x$}}{\binpar{P}{Q} \vdash x}
\end{mathpar}

\subsubsection{Contexts}

One of the principle advantages of computational calculi like the
$\pi$-calculus is a well-defined notion of context,
contextual-equivalence and a correlation between
contextual-equivalence and notions of bisimulation. The notion of
context allows the decomposition of a process into (sub-)process and
its syntactic environment, its context. Thus, a context may be
thought of as a process with a ``hole'' (written $\Box$) in it. The
application of a context $M$ to a process $P$, written $M[P]$, is
tantamount to filling the hole in $M$ with $P$. In this paper we do
not need the full weight of this theory, but do make use of the notion
of context in the proof the main theorem. 

\begin{mathpar}
  \inferrule* [lab=summation] {} {{M_{M},M_{N}} \bc \Box \;|\; x.M_{A} \;|\; M_{M}+M_{N}}
  \and
  \inferrule* [lab=agent] {} {{M_{A}} \bc (\vec{x})M_{P} \;| \; \clift{P_0,\ldots,M_{P},\ldots,P_N}}
  \and \\
  \inferrule* [lab=process] {} {{M_{P}} \bc M_{N} \;| \;P|M_{P} }
\end{mathpar} 

\begin{mathpar}
  \inferrule* [lab=sychronization] {} {M_{N} \bc \Box \;|\; x?M_{F} \;|\; x!M_{C}}
  \and
  \inferrule* [lab=abstraction] {} {{M_{F}} \bc (x)M_{P} }
  \and
  \inferrule* [lab=concretion] {} {{M_{C}} \bc \langle M_{P} \rangle }
  \and \\
  \inferrule* [lab=process] {} {{M_{P}} \bc M_{N} \;| \;P|M_{P} }
\end{mathpar}

\begin{definition}[contextual application] Given a context $M$, and
  process $P$, we define the \emph{contextual application}, $M[P] :=
  M\{P/\Box\}$. That is, the contextual application of M to P is the
  substitution of $P$ for $\Box$ in $M$.
\end{definition}

$\meaningof{-} : L \to \mathcal{P}(\pi)$

\begin{mathpar}
  \inferrule* [lab=collection] {} {\meaningof{true} = \pi, \and \meaningof{~E} = \pi \setminus \meaningof{E}, \and \meaningof{E_{1} \& E_{2}} = \meaningof{E_{1}} \cap \meaningof{E_{2}}}
\end{mathpar}

\begin{mathpar}
  \inferrule* [lab=structure] {} {\meaningof{0} = \{ P \in \pi | P \equiv 0 \}, \and \\ \meaningof{E_1 | E_2} = \{ P \in \pi | P \equiv P_{1} | P_{2}, P_{1} \in \meaningof{E_{1}}, P_{2} \in \meaningof{E_2}\} }
\end{mathpar}

\begin{mathpar}
 \inferrule* [lab=behavior] {} {\meaningof{\langle a?b \rangle E} = \{ P \in \pi | P \equiv Q | u?(y)P', \\ \and \\\\ \and \\ \;\;\; u \in \meaningof{a}, \forall z.P'\{z/y\} \in \meaningof{E\{z/b\}}\}, \and \\ \meaningof{a!E} = \{ P \in \pi | P \equiv Q | x!\langle P' \rangle, x \in \meaningof{a} P' \in \meaningof{E}\} }
\end{mathpar}

\begin{mathpar}
 \inferrule* [lab=nominal] {} {\meaningof{\quotep{E}} = \{ \quotep{P} \in \quotep{\pi} | P \in \meaningof{E} \}, \and \meaningof{\quotep{P}} = \{ \quotep{Q} \in \quotep{\pi} | P \equiv Q \} \and \\ \meaningof{@\quotep{E}} = \{ P \in \pi | P \equiv @x, x \in \meaningof{E} \}}
\end{mathpar}

\begin{eqnarray*}
  \\
  \meaningof{-} : TS \to ST
\end{eqnarray*}

\begin{eqnarray*}
  \\
  L : TS \to ST
\end{eqnarray*}

\begin{eqnarray*}
  \\
  P \models E \iff P \in \meaningof{E}
\end{eqnarray*}

\begin{eqnarray*}
  P \approx_{L} Q \iff \forall E \in L. P \models E \iff Q \models E
\end{eqnarray*}

\begin{eqnarray*}
  P \approx_{K} Q
\end{eqnarray*}

\begin{eqnarray*}
  P \approx Q
\end{eqnarray*}

$\approx_{K} = \approx = \approx_{L}$

\subsubsection{Contextual duality}

Note that contexts extend the quotation operation to a family of
operations from processes to names. Given a context, $M$, we can
define a \emph{nominal context}, $\quotep{M}$ by $\quotep{M}[P] :=
\quotep{M[P]}$. To foreshadow what is to come we observe that these
operations enjoy a duality with processes very much like the duality
between vectors and maps from vectors to scalars.

Further, because the calculus is essentially higher-order, we have a
correspondence between contexts and processes. More specifically,
given a name $x$ and a context $M$ we can construct $M^{*}_{x}$ such
that 

\begin{mathpar}
  M^{*}_{x} | \lift{x}{P} \red M[P]
\end{mathpar}

namely,

\begin{mathpar}
  M^{*}_{x} := x?(u).M[\dropn{u}]
\end{mathpar}

The dependence of $M^{*}_{x}$ on a name makes it an abstraction, 

\begin{mathpar}
  M^{*} := (x)x?(u).M[\dropn{u}]
\end{mathpar}

\subsection{Additional notation}

It will sometimes be convenient to denote the process a name
quotes. We already have the notation $x = \quotep{P}$, but it will be
convenient to introduce an alternate notation, $\procn{x}$, when we
want to emphasize the connection to the use of the name. Note that, by
virtue of name equivalence, $\quotep{\procn{x}} \nameeq x$; so, the
notation is consistent with previous definitions.

Further, because names have structure it is possible to effect
substitutions on the basis of that structure. This means we need to
upgrade our notation for substitutions, which we accomplish by
adapting comprehension notation. Thus,

\begin{mathpar}
  P\{ y / x : x \in S \}
\end{mathpar}

is interpreted to mean the process derived from P by replacing (in a
capture-avoiding manner) each occurrence of $x$ in $S$ by $y$. For example,

\begin{mathpar}
  P\{ \quotep{\procn{x}|\procn{x}} / x : x \in \freenames{P} \}
\end{mathpar}

will replace each (occurrence) of a free name $x$ in $P$ by
$\quotep{\procn{x}|\procn{x}}$.

Also, we will avail ourselves of the notation $x^{L}$ and $x^{R}$ to
denote injections of a name into disjoint copies of the name
space. There are numerous ways to accomplish this. One example can be
found in \cite{MeredithR05}. This notation overloads to vectors of
names: $\vec{x}^{\pi} := (x_{i}^{\pi} \; : \; 0 \leq i < |\vec{x}| )$ where $\pi \in \{L,R\}$.

We also use $P^{\Box} := P|\Box$.

In \cite{MeredithR05} an interpretation of the new operator is
given. It turns out that there are several possible interpretations
all enjoying the requisite algebraic properties of the operator (see
\cite{milner91polyadicpi}). We will therefore make liberal use of
$(\nu\; \vec{x})P$.

% subsection the_syntax_and_semantics_of_the_notation_system (end)   

\input{qm2pi.qmops} 

\input{qm2pi.sterngerlach} 

\input{qm2pi.metric} 

% section concurrent_process_calculi (end)

%\input{qm2pi.proofsketch}

% section proof sketch (end)

%\input{qm2pi.slviaknots} 

% section spatial logic via knots (end)

\input{qm2pi.conclusion}

% section conclusion (end)

%\input{qm2pi.dtcodes} 

% section wiring algorithm (end)

\input{qm2pi.ack} 

% section acknowledgments (end)

\newpage


\bibliographystyle{plain}   
\bibliography{../../biblios/main.bib}

\input{qm2pi.rhodetails}

\end{document}

 

%\documentclass[12pt]{llncs}
%\documentclass{jktr}

\usepackage[pdftex]{hyperref}                   
\usepackage {listings}
\usepackage {mathpartir}
\usepackage{bcprules}
%\usepackage{listings}
                       
\usepackage{graphicx} 
%\usepackage[margins=2.5cm,nohead,nofoot]{geometry}
%\usepackage{geometry}
\usepackage{amsfonts}
\usepackage{amstext}
\usepackage{latexsym}
\usepackage{amssymb}
\usepackage{color}


%\include{myPreamble}
\include{qm2pi.local} 

%\ifpdf
%\usepackage[pdftex]{graphicx}
%\else
%\usepackage{graphicx}
%\fi

 % \ifpdf
%  \usepackage{pdfsync}
%  \if


%\title{Brief Article}
%\author{David F. Snyder}
%\author{L.G. Meredith}

%\address{Dept. of Math., Texas State University--San Marcos, San Marcos, TX 78666}
       
\pagestyle{empty}


\begin{document}

\lstset{language=[Objective]Caml,frame=shadowbox}

\input{qm2pi.front}

% section front matter (end)

\input{qm2pi.intro} 
 
% section introduction (end)

% \input{qm2pi.knotations} 

% section notation (end)

\input{qm2pi.process.calculi} 

% section concurrent_process_calculi_and_spatial_logics_ (end)
    
%\input{qm2pi.knots2pi} 

%\input{qm2pi.trefoil} 

%\input{qm2pi.mainthm} 

% subsection basic_interpretation (end)

%\input{qm2pi.rho.presentation} 
\subsection{The syntax and semantics of the notation system}\label{sub:the_syntax_and_semantics_of_the_notation_system} % (fold)

We now summarize a technical presentation of the calculus that
embodies our theory of dynamics. The typical presentation of such a
calculus follows the style of giving generators and relations on
them. The grammar, below, describing term constructors, freely
generates the set of processes, $\Proc$. This set is then quotiented
by a relation known as structural congruence and it is over this set
that the notion of dynamics is expressed. This presentation is
essentially that of \cite{MeredithR05} with the addition of
polyadicity and summation. For readability we have relegated some of
the technical subtleties to an appendix.

\subsubsection{Process grammar}\label{subsub:process_grammar}

\begin{mathpar}
  \inferrule* [lab=synchronization] {} {{M} \bc \pzero \;|\; x?F \;|\; x!C }
  \and
  \inferrule* [lab=abstraction] {} {{F} \bc (x)P}
  \and
  \inferrule* [lab=concretion] {} {{C} \bc \langle Q \rangle}
  \and
  \inferrule* [lab=process] {} {{P,Q} \bc M \;| \;P|Q \;|\; @{x}}
  \and
  \inferrule* [lab=name] {} {{x} \bc \quotep{P}}
\end{mathpar} 

Note that $\vec{x}$ (resp. $\vec{P}$) denotes a vector of names
(resp. processes) of length $|\vec{x}|$ (resp. $|\vec{P}|$). We adopt
the following useful abbreviations.

\begin{mathpar}
   x?(\vec{y}).P := x.(\vec{y})P \and  x\clift{\vec{P}} := x.\clift{\vec{P}}
   \and x!(y) := \lift{x}{\dropn{y}}
   \and \Pi_{i=0}^{n-1}P_i := P_0 | \ldots | P_{n-1}
\end{mathpar}

\subsubsection{Structural congruence}

\paragraph{Free and bound names and alpha-equivalence.} At the
core of structural equivalence is alpha-equivalence which identifies
process that are the same up to a change of variable. Formally, we
recognize the distinction between free and bound names. The free names
of a process, $\freenames{P}$, may be calculated recursively as
follows:

\begin{mathpar}
\freenames{\pzero} := \emptyset
  \and \\
  \freenames{x?(y).P} := \{ x \} \cup (\freenames{P} \setminus \{ y \})
  \and 
  \freenames{x!\langle P \rangle} := \{ x \} \cup \{ P \} 
  \and \\
  \freenames{P|Q} := \freenames{P} \cup \freenames{Q}
  \and \\
  \freenames{@{x}} := \{ x \}
\end{mathpar}

$\pi$
$\quotep{\pi}$

$\freenames{-} : \pi \to \mathcal{P}(\quotep{\pi})$

\begin{eqnarray*}
  \freenames{\pzero} & := & \emptyset \\
  \freenames{x?(y).P} & := & \{ x \} \cup (\freenames{P} \setminus \{ y \}) \\
  \freenames{x!\langle P \rangle} & := & \{ x \} \cup \{ P \} \\
  \freenames{P|Q} & := & \freenames{P} \cup \freenames{Q} \\
  \freenames{\dropn{x}} & := & \{ x \}
\end{eqnarray*}

The bound names of a process, $\boundnames{P}$, are those names occurring in $P$
that are not free. For example, in $x?(y).0$, the name $x$ is free, while $y$ is bound.

\begin{mathpar}
  \inferrule* [lab=monoidal-laws] {} { P|Q \equiv Q|P \and P|0 \equiv P \and P|(Q|R) \equiv (P|Q)|R }
\end{mathpar}

\begin{mathpar}
  \inferrule* [lab=alpha-equivalence] {} { (x)P \equiv (y)P\{y/x\} \and y \not\in \freenames{P} }
\end{mathpar}

\begin{definition}
Then two processes, $P,Q$, are alpha-equivalent if $P = Q\{\vec{y}/\vec{x}\}$ for
some $\vec{x} \in \boundnames{Q},\vec{y} \in \boundnames{P}$, where $Q\{\vec{y}/\vec{x}\}$
denotes the capture-avoiding substitution of $\vec{y}$ for $\vec{x}$ in $Q$.
\end{definition}

\begin{definition}
  The {\em structural congruence} \cite{SangiorgiWalker} , $\equiv$,
  between processes is the least congruence containing
  alpha-equivalence, satisfying the abelian monoid laws
  (associativity, commutativity and $\pzero$ as identity) for parallel
  composition $|$ and for summation $+$.
\end{definition}

\subsection{Name equivalence}

We take name equivalence, written $\nameeq$, to be the smallest
equivalence relation generated by the following rules.

\begin{mathpar}
\inferrule*[lab=Quote-drop]
{ }
{ \quotep{@{x}} \nameeq x }

\inferrule*[lab=Struct-equiv]
{ P \scong Q }
{ \quotep{P} \nameeq \quotep{Q} }
\end{mathpar}

The astute reader will have noticed that the mutual recursion of names
and processes imposes a mutual recursion on alpha-equivalence and
structural equivalence via name-equivalence. Fortunately, all of this
works out pleasantly and we may calculate in the natural way, free of
concern. The reader interested in the details is referred to the
appendix \ref{appendix:rho_details}.

\subsection{Substitution}

We use $\Proc$ for the set of processes, $\QProc$ for the set of
names, and $\id{\{}\vec{y} / \vec{x} \id{\}}$ to denote partial maps,
$s : \QProc \rightarrow \QProc$. A map, $s$ lifts, uniquely, to a map
on process terms, $\widehat{s} : \Proc \rightarrow \Proc$ by the
following equations.

\begin{mathpar}
  (0) \psubstp{Q}{P} := 0 \\
  (R \juxtap S) \psubstp{Q}{P}
  :=    
  (R)\psubstp{Q}{P} \juxtap (S) \psubstp{Q}{P} \\
  (x?(y).R) \psubstp{Q}{P}    
  :=    
  (x)\substp{Q}{P} (z)\concat( (R \psubstn{z}{y}) \psubstp{Q}{P} ) \\
  (\lift{x}{R}) \psubstp{Q}{P}  
  :=
  \lift{(x)\substp{Q}{P}}{ R \psubstp{Q}{P} } \\
%   (\dropn{x})  \psubstp{Q}{P}       
%   := 
%   \left\{ 
%     \begin{array}{ccc} 
%       \dropn{\quotep{Q}} & & x \nameeq \quotep{P} \\
%       \dropn{x} & & otherwise \\
%     \end{array}
%   \right. 
  (\dropn{x})  \psubstp{Q}{P}       
  := 
  \left\{ 
    \begin{array}{ccc} 
      Q & & x \nameeq \quotep{P} \\
      \dropn{x} & & otherwise \\
    \end{array}
  \right.
\end{mathpar}
 

where

\begin{eqnarray}
  (x)\id{\{} \lpquote Q \rpquote / \lpquote P \rpquote \id{\}}            = 
  \left\{ 
    \begin{array}{ccc}
      \lpquote Q \rpquote & & x \nameeq \lpquote P \rpquote \\
      x & & otherwise \\
    \end{array}
  \right. \nonumber
\end{eqnarray}

and $z$ is chosen distinct from $\quotep{P}$, $\quotep{Q}$, the free
names in $Q$, and all the names in $R$. Our $\alpha$-equivalence will
be built in the standard way from this substitution.

\begin{remark}\label{rem:no_self_referential_names}
  One consequence of these definitions is that $\forall P. \quotep{P}
  \not\in \freenames{P}$.
\end{remark}

\subsection{ Dynamic quote: an example }

Anticipating something of what's to come, consider applying the
substitution, $\widehat{\id{\{}u / z \id{\}}}$, to the following pair
of processes, $\lift{w}{y!(z)}$ and $w[ \lpquote y!(z) \rpquote ]$.

\begin{eqnarray}
	\lift{w}{y!(z)}\widehat{\id{\{}u / z \id{\}}}
		& = &
		\lift{w}{y!(u)} \nonumber\\
	w[ \lpquote y!(z) \rpquote ] \widehat{ \id{\{}u / z \id{\}} }
		& = &
		w[ \lpquote y!(z) \rpquote ] \nonumber
\end{eqnarray}

Because the body of the process between quotes is impervious to
substitution, we get radically different answers. In fact, by
examining the first process in an input context,
e.g. $x?(z).\lift{w}{y!(z)}$, we see that the process under the lift
operator may be shaped by prefixed inputs binding a name inside it. In
this sense, the lift operator will be seen as a way to dynamically
construct processes before reifying them as names.

Finally equipped with these standard features we can present the
dynamics of the calculus.

\subsubsection{Operational semantics} 

Finally, we introduce the computational dynamics. What marks these
algebras as distinct from other more traditionally studied algebraic
structures, e.g. vector spaces or polynomial rings, is the manner in
which dynamics is captured. In traditional structures, dynamics is typically
expressed through morphisms between such structures, as in linear maps
between vector spaces or morphisms between rings. In algebras
associated with the semantics of computation, the dynamics is
expressed as part of the algebraic structure itself, through a
reduction reduction relation typically denoted by $\red$. Below, we
give a recursive presentation of this relation for the calculus used
in the encoding.

$\red \subseteq \pi \times \pi$
$\red : \pi \to \mathcal{P}(\pi)$

\begin{mathpar}
  \inferrule* [lab=Comm] { \textsf{match}( x_{src}, x_{trgt} ) } { x_{trgt}?(y)P \; | \; x_{src}!\langle {Q} \rangle \red P\{\quotep{Q}/y}\} }
  \and \\
  \inferrule* [lab=Par] {{P} \red {P}'} {{{P} | {Q}} \red {{P}' | {Q}}}
  \and
  \inferrule* [lab=Equiv]{{{P} \scong {P}'} \andalso {{P}' \red {Q}'} \andalso {{Q}' \scong {Q}}}{{P} \red {Q}}
\end{mathpar}

\begin{eqnarray*}
  match_{\equiv} (\quotep{P},\quotep{Q}) & := & P \equiv Q \\
  match_{\dagger}(\quotep{P},\quotep{Q}) & := & \forall R. P|Q \red^{*} R => R \red^{*} 0 \\
  match_{K}(\quotep{P},\quotep{Q}) & := & K \mbox{ for some context } K
\end{eqnarray*}

$u?(x)P | u!\langle Q \rangle \red P\{\quotep{Q}/x\}$

%We write $\wred$ for $\red^*$, and $P\red$ if $\exists Q $ such that $ P \red Q$.
We write $P\red$ if $\exists Q $ such that $ P \red Q$ and $P\not\red$, otherwise.

\section{Replication}

As mentioned before, it is known that replication (and hence
recursion) can be implemented in a higher-order process algebra
\cite{SangiorgiWalker}. As our first example of calculation with the
machinery thus far presented we give the construction explicitly in
the {\rhoc}.

\begin{eqnarray}
	D_{x} & := & \prefix{x}{y}{(\binpar{\outputp{x}{y}}{@{y}})} \nonumber\\
	\bangp_{x}{P} & := & \binpar{{x}!\langle{\binpar{D_{x}}{P}}\rangle}{D_{x}} \nonumber
\end{eqnarray}

\begin{eqnarray}
	\bangp_{x}{P} & & \nonumber\\
	=
	& {x}!\langle{(\prefix{x}{y}{(\outputp{x}{y} | @{y})) | P}}\rangle 
	      | \prefix{x}{y}{(\outputp{x}{y} | @{y})} & \nonumber\\
	\red
	& (\outputp{x}{y} | @{y})\substn{\quotep{(\prefix{x}{y}{(@{y} | \outputp{x}{y})) | P}}}{y} & \nonumber\\
	=
	& \outputp{x}{\quotep{(\prefix{x}{y}{(\outputp{x}{y} | @{y})) | P}}}
	  | {(\prefix{x}{y}{(\outputp{x}{y} | @{y})) | P}} & \nonumber\\
	\red
	& \ldots & \nonumber\\
	\red^*
	& P | P | \ldots & \nonumber
\end{eqnarray}

Of course, this encoding, as an implementation, runs away, unfolding
$\bangp{P}$ eagerly. A lazier and more implementable replication
operator, restricted to input-guarded processes, may be obtained as follows.

\begin{eqnarray}
\bangp{\prefix{u}{v}{P}} 
	:= 
	\binpar{\lift{x}{\prefix{u}{v}{(\binpar{D(x)}{P})}}}{D(x)} \nonumber
\end{eqnarray}

\begin{remark}
  Note that the lazier definition still does not deal with summation
  or mixed summation (i.e. sums over input and output). The reader is
  invited to construct definitions of replication that deal with these
  features. 

  Further, the definitions are parameterized in a name, $x$. Can you,
  gentle reader, make a definition that eliminates this parameter and
  guarantees no accidental interaction between the replication
  machinery and the process being replicated -- i.e. no accidental
  sharing of names used by the process to get its work done and the
  name(s) used by the replication to effect copying. This latter
  revision of the definition of replication is crucial to obtaining
  the expected identity $!!P \sim !P$.
\end{remark}

\begin{remark}\label{rem:paradoxical_combinator}
  The reader familiar with the lambda calculus will have noticed the
  similarity between $D$ and the paradoxical combinator.

  [Ed. note: the existence of this seems to suggest we have to be more
  restrictive on the set of processes and names we admit if we are to
  support no-cloning.]
\end{remark}

\subsubsection{Bisimulation}

The computational dynamics gives rise to another kind of equivalence,
the equivalence of computational behavior. As previously mentioned
this is typically captured \emph{via} some form of bisimulation.

% The notion we use in this paper is weak barbed bisimulation
% \cite{milner91polyadicpi}.

The notion we use in this paper is derived from weak barbed
bisimulation \cite{milner91polyadicpi}. 

\begin{definition}
An \emph{observation relation}, $\downarrow_{\mathcal N}$, over a set
of names, $\mathcal N$, is the smallest relation satisfying the rules
below.

\infrule[Out-barb]{y \in {\mathcal N}, \; x \nameeq y}
		  {\outputp{x}{v} \downarrow_{\mathcal N} x}
\infrule[Par-barb]{\mbox{$P\downarrow_{\mathcal N} x$ or $Q\downarrow_{\mathcal N} x$}}
		  {\binpar{P}{Q} \downarrow_{\mathcal N} x}

We write $P \Downarrow_{\mathcal N} x$ if there is $Q$ such that 
$P \wred Q$ and $Q \downarrow_{\mathcal N} x$.
\end{definition}

\begin{definition}
%\label{def.bbisim}
An  ${\mathcal N}$-\emph{barbed bisimulation} over a set of names, ${\mathcal N}$, is a symmetric binary relation 
${\mathcal S}_{\mathcal N}$ between agents such that $P\rel{S}_{\mathcal N}Q$ implies:
\begin{enumerate}
\item If $P \red P'$ then $Q \wred Q'$ and $P'\rel{S}_{\mathcal N} Q'$.
\item If $P\downarrow_{\mathcal N} x$, then $Q\Downarrow_{\mathcal N} x$.
\end{enumerate}
$P$ is ${\mathcal N}$-barbed bisimilar to $Q$, written
$P \wbbisim_{\mathcal N} Q$, if $P \rel{S}_{\mathcal N} Q$ for some ${\mathcal N}$-barbed bisimulation ${\mathcal S}_{\mathcal N}$.
\end{definition}

$\mathcal{R} \subseteq \pi \times \pi$

$P \mathcal{R} Q => \forall P'. P \red P' \Rightarrow \exists Q'. Q \red Q', P' \mathcal{R} Q'$

$P \vdash x \Rightarrow Q \vdash x$

\begin{mathpar}
  \inferrule*[lab=Out-barb]{x \nameeq y}{{y}!\langle{Q}\rangle \vdash x}
  \and
  \inferrule*[lab=Par-barb]{\mbox{$P\vdash x$ or $Q\vdash x$}}{\binpar{P}{Q} \vdash x}
\end{mathpar}

\subsubsection{Contexts}

One of the principle advantages of computational calculi like the
$\pi$-calculus is a well-defined notion of context,
contextual-equivalence and a correlation between
contextual-equivalence and notions of bisimulation. The notion of
context allows the decomposition of a process into (sub-)process and
its syntactic environment, its context. Thus, a context may be
thought of as a process with a ``hole'' (written $\Box$) in it. The
application of a context $M$ to a process $P$, written $M[P]$, is
tantamount to filling the hole in $M$ with $P$. In this paper we do
not need the full weight of this theory, but do make use of the notion
of context in the proof the main theorem. 

\begin{mathpar}
  \inferrule* [lab=summation] {} {{M_{M},M_{N}} \bc \Box \;|\; x.M_{A} \;|\; M_{M}+M_{N}}
  \and
  \inferrule* [lab=agent] {} {{M_{A}} \bc (\vec{x})M_{P} \;| \; \clift{P_0,\ldots,M_{P},\ldots,P_N}}
  \and \\
  \inferrule* [lab=process] {} {{M_{P}} \bc M_{N} \;| \;P|M_{P} }
\end{mathpar} 

\begin{mathpar}
  \inferrule* [lab=sychronization] {} {M_{N} \bc \Box \;|\; x?M_{F} \;|\; x!M_{C}}
  \and
  \inferrule* [lab=abstraction] {} {{M_{F}} \bc (x)M_{P} }
  \and
  \inferrule* [lab=concretion] {} {{M_{C}} \bc \langle M_{P} \rangle }
  \and \\
  \inferrule* [lab=process] {} {{M_{P}} \bc M_{N} \;| \;P|M_{P} }
\end{mathpar}

\begin{definition}[contextual application] Given a context $M$, and
  process $P$, we define the \emph{contextual application}, $M[P] :=
  M\{P/\Box\}$. That is, the contextual application of M to P is the
  substitution of $P$ for $\Box$ in $M$.
\end{definition}

$\meaningof{-} : L \to \mathcal{P}(\pi)$

\begin{mathpar}
  \inferrule* [lab=collection] {} {\meaningof{true} = \pi, \and \meaningof{~E} = \pi \setminus \meaningof{E}, \and \meaningof{E_{1} \& E_{2}} = \meaningof{E_{1}} \cap \meaningof{E_{2}}}
\end{mathpar}

\begin{mathpar}
  \inferrule* [lab=structure] {} {\meaningof{0} = \{ P \in \pi | P \equiv 0 \}, \and \\ \meaningof{E_1 | E_2} = \{ P \in \pi | P \equiv P_{1} | P_{2}, P_{1} \in \meaningof{E_{1}}, P_{2} \in \meaningof{E_2}\} }
\end{mathpar}

\begin{mathpar}
 \inferrule* [lab=behavior] {} {\meaningof{\langle a?b \rangle E} = \{ P \in \pi | P \equiv Q | u?(y)P', \\ \and \\\\ \and \\ \;\;\; u \in \meaningof{a}, \forall z.P'\{z/y\} \in \meaningof{E\{z/b\}}\}, \and \\ \meaningof{a!E} = \{ P \in \pi | P \equiv Q | x!\langle P' \rangle, x \in \meaningof{a} P' \in \meaningof{E}\} }
\end{mathpar}

\begin{mathpar}
 \inferrule* [lab=nominal] {} {\meaningof{\quotep{E}} = \{ \quotep{P} \in \quotep{\pi} | P \in \meaningof{E} \}, \and \meaningof{\quotep{P}} = \{ \quotep{Q} \in \quotep{\pi} | P \equiv Q \} \and \\ \meaningof{@\quotep{E}} = \{ P \in \pi | P \equiv @x, x \in \meaningof{E} \}}
\end{mathpar}

\begin{eqnarray*}
  \\
  \meaningof{-} : TS \to ST
\end{eqnarray*}

\begin{eqnarray*}
  \\
  L : TS \to ST
\end{eqnarray*}

\begin{eqnarray*}
  \\
  P \models E \iff P \in \meaningof{E}
\end{eqnarray*}

\begin{eqnarray*}
  P \approx_{L} Q \iff \forall E \in L. P \models E \iff Q \models E
\end{eqnarray*}

\begin{eqnarray*}
  P \approx_{K} Q
\end{eqnarray*}

\begin{eqnarray*}
  P \approx Q
\end{eqnarray*}

$\approx_{K} = \approx = \approx_{L}$

\subsubsection{Contextual duality}

Note that contexts extend the quotation operation to a family of
operations from processes to names. Given a context, $M$, we can
define a \emph{nominal context}, $\quotep{M}$ by $\quotep{M}[P] :=
\quotep{M[P]}$. To foreshadow what is to come we observe that these
operations enjoy a duality with processes very much like the duality
between vectors and maps from vectors to scalars.

Further, because the calculus is essentially higher-order, we have a
correspondence between contexts and processes. More specifically,
given a name $x$ and a context $M$ we can construct $M^{*}_{x}$ such
that 

\begin{mathpar}
  M^{*}_{x} | \lift{x}{P} \red M[P]
\end{mathpar}

namely,

\begin{mathpar}
  M^{*}_{x} := x?(u).M[\dropn{u}]
\end{mathpar}

The dependence of $M^{*}_{x}$ on a name makes it an abstraction, 

\begin{mathpar}
  M^{*} := (x)x?(u).M[\dropn{u}]
\end{mathpar}

\subsection{Additional notation}

It will sometimes be convenient to denote the process a name
quotes. We already have the notation $x = \quotep{P}$, but it will be
convenient to introduce an alternate notation, $\procn{x}$, when we
want to emphasize the connection to the use of the name. Note that, by
virtue of name equivalence, $\quotep{\procn{x}} \nameeq x$; so, the
notation is consistent with previous definitions.

Further, because names have structure it is possible to effect
substitutions on the basis of that structure. This means we need to
upgrade our notation for substitutions, which we accomplish by
adapting comprehension notation. Thus,

\begin{mathpar}
  P\{ y / x : x \in S \}
\end{mathpar}

is interpreted to mean the process derived from P by replacing (in a
capture-avoiding manner) each occurrence of $x$ in $S$ by $y$. For example,

\begin{mathpar}
  P\{ \quotep{\procn{x}|\procn{x}} / x : x \in \freenames{P} \}
\end{mathpar}

will replace each (occurrence) of a free name $x$ in $P$ by
$\quotep{\procn{x}|\procn{x}}$.

Also, we will avail ourselves of the notation $x^{L}$ and $x^{R}$ to
denote injections of a name into disjoint copies of the name
space. There are numerous ways to accomplish this. One example can be
found in \cite{MeredithR05}. This notation overloads to vectors of
names: $\vec{x}^{\pi} := (x_{i}^{\pi} \; : \; 0 \leq i < |\vec{x}| )$ where $\pi \in \{L,R\}$.

We also use $P^{\Box} := P|\Box$.

In \cite{MeredithR05} an interpretation of the new operator is
given. It turns out that there are several possible interpretations
all enjoying the requisite algebraic properties of the operator (see
\cite{milner91polyadicpi}). We will therefore make liberal use of
$(\nu\; \vec{x})P$.

% subsection the_syntax_and_semantics_of_the_notation_system (end)   

\input{qm2pi.qmops} 

\input{qm2pi.sterngerlach} 

\input{qm2pi.metric} 

% section concurrent_process_calculi (end)

%\input{qm2pi.proofsketch}

% section proof sketch (end)

%\input{qm2pi.slviaknots} 

% section spatial logic via knots (end)

\input{qm2pi.conclusion}

% section conclusion (end)

%\input{qm2pi.dtcodes} 

% section wiring algorithm (end)

\input{qm2pi.ack} 

% section acknowledgments (end)

\newpage


\bibliographystyle{plain}   
\bibliography{../../biblios/main.bib}

\input{qm2pi.rhodetails}

\end{document}

 

%\documentclass[12pt]{llncs}
%\documentclass{jktr}

\usepackage[pdftex]{hyperref}                   
\usepackage {listings}
\usepackage {mathpartir}
\usepackage{bcprules}
%\usepackage{listings}
                       
\usepackage{graphicx} 
%\usepackage[margins=2.5cm,nohead,nofoot]{geometry}
%\usepackage{geometry}
\usepackage{amsfonts}
\usepackage{amstext}
\usepackage{latexsym}
\usepackage{amssymb}
\usepackage{color}


%\include{myPreamble}
\include{qm2pi.local} 

%\ifpdf
%\usepackage[pdftex]{graphicx}
%\else
%\usepackage{graphicx}
%\fi

 % \ifpdf
%  \usepackage{pdfsync}
%  \if


%\title{Brief Article}
%\author{David F. Snyder}
%\author{L.G. Meredith}

%\address{Dept. of Math., Texas State University--San Marcos, San Marcos, TX 78666}
       
\pagestyle{empty}


\begin{document}

\lstset{language=[Objective]Caml,frame=shadowbox}

\input{qm2pi.front}

% section front matter (end)

\input{qm2pi.intro} 
 
% section introduction (end)

% \input{qm2pi.knotations} 

% section notation (end)

\input{qm2pi.process.calculi} 

% section concurrent_process_calculi_and_spatial_logics_ (end)
    
%\input{qm2pi.knots2pi} 

%\input{qm2pi.trefoil} 

%\input{qm2pi.mainthm} 

% subsection basic_interpretation (end)

%\input{qm2pi.rho.presentation} 
\subsection{The syntax and semantics of the notation system}\label{sub:the_syntax_and_semantics_of_the_notation_system} % (fold)

We now summarize a technical presentation of the calculus that
embodies our theory of dynamics. The typical presentation of such a
calculus follows the style of giving generators and relations on
them. The grammar, below, describing term constructors, freely
generates the set of processes, $\Proc$. This set is then quotiented
by a relation known as structural congruence and it is over this set
that the notion of dynamics is expressed. This presentation is
essentially that of \cite{MeredithR05} with the addition of
polyadicity and summation. For readability we have relegated some of
the technical subtleties to an appendix.

\subsubsection{Process grammar}\label{subsub:process_grammar}

\begin{mathpar}
  \inferrule* [lab=synchronization] {} {{M} \bc \pzero \;|\; x?F \;|\; x!C }
  \and
  \inferrule* [lab=abstraction] {} {{F} \bc (x)P}
  \and
  \inferrule* [lab=concretion] {} {{C} \bc \langle Q \rangle}
  \and
  \inferrule* [lab=process] {} {{P,Q} \bc M \;| \;P|Q \;|\; @{x}}
  \and
  \inferrule* [lab=name] {} {{x} \bc \quotep{P}}
\end{mathpar} 

Note that $\vec{x}$ (resp. $\vec{P}$) denotes a vector of names
(resp. processes) of length $|\vec{x}|$ (resp. $|\vec{P}|$). We adopt
the following useful abbreviations.

\begin{mathpar}
   x?(\vec{y}).P := x.(\vec{y})P \and  x\clift{\vec{P}} := x.\clift{\vec{P}}
   \and x!(y) := \lift{x}{\dropn{y}}
   \and \Pi_{i=0}^{n-1}P_i := P_0 | \ldots | P_{n-1}
\end{mathpar}

\subsubsection{Structural congruence}

\paragraph{Free and bound names and alpha-equivalence.} At the
core of structural equivalence is alpha-equivalence which identifies
process that are the same up to a change of variable. Formally, we
recognize the distinction between free and bound names. The free names
of a process, $\freenames{P}$, may be calculated recursively as
follows:

\begin{mathpar}
\freenames{\pzero} := \emptyset
  \and \\
  \freenames{x?(y).P} := \{ x \} \cup (\freenames{P} \setminus \{ y \})
  \and 
  \freenames{x!\langle P \rangle} := \{ x \} \cup \{ P \} 
  \and \\
  \freenames{P|Q} := \freenames{P} \cup \freenames{Q}
  \and \\
  \freenames{@{x}} := \{ x \}
\end{mathpar}

$\pi$
$\quotep{\pi}$

$\freenames{-} : \pi \to \mathcal{P}(\quotep{\pi})$

\begin{eqnarray*}
  \freenames{\pzero} & := & \emptyset \\
  \freenames{x?(y).P} & := & \{ x \} \cup (\freenames{P} \setminus \{ y \}) \\
  \freenames{x!\langle P \rangle} & := & \{ x \} \cup \{ P \} \\
  \freenames{P|Q} & := & \freenames{P} \cup \freenames{Q} \\
  \freenames{\dropn{x}} & := & \{ x \}
\end{eqnarray*}

The bound names of a process, $\boundnames{P}$, are those names occurring in $P$
that are not free. For example, in $x?(y).0$, the name $x$ is free, while $y$ is bound.

\begin{mathpar}
  \inferrule* [lab=monoidal-laws] {} { P|Q \equiv Q|P \and P|0 \equiv P \and P|(Q|R) \equiv (P|Q)|R }
\end{mathpar}

\begin{mathpar}
  \inferrule* [lab=alpha-equivalence] {} { (x)P \equiv (y)P\{y/x\} \and y \not\in \freenames{P} }
\end{mathpar}

\begin{definition}
Then two processes, $P,Q$, are alpha-equivalent if $P = Q\{\vec{y}/\vec{x}\}$ for
some $\vec{x} \in \boundnames{Q},\vec{y} \in \boundnames{P}$, where $Q\{\vec{y}/\vec{x}\}$
denotes the capture-avoiding substitution of $\vec{y}$ for $\vec{x}$ in $Q$.
\end{definition}

\begin{definition}
  The {\em structural congruence} \cite{SangiorgiWalker} , $\equiv$,
  between processes is the least congruence containing
  alpha-equivalence, satisfying the abelian monoid laws
  (associativity, commutativity and $\pzero$ as identity) for parallel
  composition $|$ and for summation $+$.
\end{definition}

\subsection{Name equivalence}

We take name equivalence, written $\nameeq$, to be the smallest
equivalence relation generated by the following rules.

\begin{mathpar}
\inferrule*[lab=Quote-drop]
{ }
{ \quotep{@{x}} \nameeq x }

\inferrule*[lab=Struct-equiv]
{ P \scong Q }
{ \quotep{P} \nameeq \quotep{Q} }
\end{mathpar}

The astute reader will have noticed that the mutual recursion of names
and processes imposes a mutual recursion on alpha-equivalence and
structural equivalence via name-equivalence. Fortunately, all of this
works out pleasantly and we may calculate in the natural way, free of
concern. The reader interested in the details is referred to the
appendix \ref{appendix:rho_details}.

\subsection{Substitution}

We use $\Proc$ for the set of processes, $\QProc$ for the set of
names, and $\id{\{}\vec{y} / \vec{x} \id{\}}$ to denote partial maps,
$s : \QProc \rightarrow \QProc$. A map, $s$ lifts, uniquely, to a map
on process terms, $\widehat{s} : \Proc \rightarrow \Proc$ by the
following equations.

\begin{mathpar}
  (0) \psubstp{Q}{P} := 0 \\
  (R \juxtap S) \psubstp{Q}{P}
  :=    
  (R)\psubstp{Q}{P} \juxtap (S) \psubstp{Q}{P} \\
  (x?(y).R) \psubstp{Q}{P}    
  :=    
  (x)\substp{Q}{P} (z)\concat( (R \psubstn{z}{y}) \psubstp{Q}{P} ) \\
  (\lift{x}{R}) \psubstp{Q}{P}  
  :=
  \lift{(x)\substp{Q}{P}}{ R \psubstp{Q}{P} } \\
%   (\dropn{x})  \psubstp{Q}{P}       
%   := 
%   \left\{ 
%     \begin{array}{ccc} 
%       \dropn{\quotep{Q}} & & x \nameeq \quotep{P} \\
%       \dropn{x} & & otherwise \\
%     \end{array}
%   \right. 
  (\dropn{x})  \psubstp{Q}{P}       
  := 
  \left\{ 
    \begin{array}{ccc} 
      Q & & x \nameeq \quotep{P} \\
      \dropn{x} & & otherwise \\
    \end{array}
  \right.
\end{mathpar}
 

where

\begin{eqnarray}
  (x)\id{\{} \lpquote Q \rpquote / \lpquote P \rpquote \id{\}}            = 
  \left\{ 
    \begin{array}{ccc}
      \lpquote Q \rpquote & & x \nameeq \lpquote P \rpquote \\
      x & & otherwise \\
    \end{array}
  \right. \nonumber
\end{eqnarray}

and $z$ is chosen distinct from $\quotep{P}$, $\quotep{Q}$, the free
names in $Q$, and all the names in $R$. Our $\alpha$-equivalence will
be built in the standard way from this substitution.

\begin{remark}\label{rem:no_self_referential_names}
  One consequence of these definitions is that $\forall P. \quotep{P}
  \not\in \freenames{P}$.
\end{remark}

\subsection{ Dynamic quote: an example }

Anticipating something of what's to come, consider applying the
substitution, $\widehat{\id{\{}u / z \id{\}}}$, to the following pair
of processes, $\lift{w}{y!(z)}$ and $w[ \lpquote y!(z) \rpquote ]$.

\begin{eqnarray}
	\lift{w}{y!(z)}\widehat{\id{\{}u / z \id{\}}}
		& = &
		\lift{w}{y!(u)} \nonumber\\
	w[ \lpquote y!(z) \rpquote ] \widehat{ \id{\{}u / z \id{\}} }
		& = &
		w[ \lpquote y!(z) \rpquote ] \nonumber
\end{eqnarray}

Because the body of the process between quotes is impervious to
substitution, we get radically different answers. In fact, by
examining the first process in an input context,
e.g. $x?(z).\lift{w}{y!(z)}$, we see that the process under the lift
operator may be shaped by prefixed inputs binding a name inside it. In
this sense, the lift operator will be seen as a way to dynamically
construct processes before reifying them as names.

Finally equipped with these standard features we can present the
dynamics of the calculus.

\subsubsection{Operational semantics} 

Finally, we introduce the computational dynamics. What marks these
algebras as distinct from other more traditionally studied algebraic
structures, e.g. vector spaces or polynomial rings, is the manner in
which dynamics is captured. In traditional structures, dynamics is typically
expressed through morphisms between such structures, as in linear maps
between vector spaces or morphisms between rings. In algebras
associated with the semantics of computation, the dynamics is
expressed as part of the algebraic structure itself, through a
reduction reduction relation typically denoted by $\red$. Below, we
give a recursive presentation of this relation for the calculus used
in the encoding.

$\red \subseteq \pi \times \pi$
$\red : \pi \to \mathcal{P}(\pi)$

\begin{mathpar}
  \inferrule* [lab=Comm] { \textsf{match}( x_{src}, x_{trgt} ) } { x_{trgt}?(y)P \; | \; x_{src}!\langle {Q} \rangle \red P\{\quotep{Q}/y}\} }
  \and \\
  \inferrule* [lab=Par] {{P} \red {P}'} {{{P} | {Q}} \red {{P}' | {Q}}}
  \and
  \inferrule* [lab=Equiv]{{{P} \scong {P}'} \andalso {{P}' \red {Q}'} \andalso {{Q}' \scong {Q}}}{{P} \red {Q}}
\end{mathpar}

\begin{eqnarray*}
  match_{\equiv} (\quotep{P},\quotep{Q}) & := & P \equiv Q \\
  match_{\dagger}(\quotep{P},\quotep{Q}) & := & \forall R. P|Q \red^{*} R => R \red^{*} 0 \\
  match_{K}(\quotep{P},\quotep{Q}) & := & K \mbox{ for some context } K
\end{eqnarray*}

$u?(x)P | u!\langle Q \rangle \red P\{\quotep{Q}/x\}$

%We write $\wred$ for $\red^*$, and $P\red$ if $\exists Q $ such that $ P \red Q$.
We write $P\red$ if $\exists Q $ such that $ P \red Q$ and $P\not\red$, otherwise.

\section{Replication}

As mentioned before, it is known that replication (and hence
recursion) can be implemented in a higher-order process algebra
\cite{SangiorgiWalker}. As our first example of calculation with the
machinery thus far presented we give the construction explicitly in
the {\rhoc}.

\begin{eqnarray}
	D_{x} & := & \prefix{x}{y}{(\binpar{\outputp{x}{y}}{@{y}})} \nonumber\\
	\bangp_{x}{P} & := & \binpar{{x}!\langle{\binpar{D_{x}}{P}}\rangle}{D_{x}} \nonumber
\end{eqnarray}

\begin{eqnarray}
	\bangp_{x}{P} & & \nonumber\\
	=
	& {x}!\langle{(\prefix{x}{y}{(\outputp{x}{y} | @{y})) | P}}\rangle 
	      | \prefix{x}{y}{(\outputp{x}{y} | @{y})} & \nonumber\\
	\red
	& (\outputp{x}{y} | @{y})\substn{\quotep{(\prefix{x}{y}{(@{y} | \outputp{x}{y})) | P}}}{y} & \nonumber\\
	=
	& \outputp{x}{\quotep{(\prefix{x}{y}{(\outputp{x}{y} | @{y})) | P}}}
	  | {(\prefix{x}{y}{(\outputp{x}{y} | @{y})) | P}} & \nonumber\\
	\red
	& \ldots & \nonumber\\
	\red^*
	& P | P | \ldots & \nonumber
\end{eqnarray}

Of course, this encoding, as an implementation, runs away, unfolding
$\bangp{P}$ eagerly. A lazier and more implementable replication
operator, restricted to input-guarded processes, may be obtained as follows.

\begin{eqnarray}
\bangp{\prefix{u}{v}{P}} 
	:= 
	\binpar{\lift{x}{\prefix{u}{v}{(\binpar{D(x)}{P})}}}{D(x)} \nonumber
\end{eqnarray}

\begin{remark}
  Note that the lazier definition still does not deal with summation
  or mixed summation (i.e. sums over input and output). The reader is
  invited to construct definitions of replication that deal with these
  features. 

  Further, the definitions are parameterized in a name, $x$. Can you,
  gentle reader, make a definition that eliminates this parameter and
  guarantees no accidental interaction between the replication
  machinery and the process being replicated -- i.e. no accidental
  sharing of names used by the process to get its work done and the
  name(s) used by the replication to effect copying. This latter
  revision of the definition of replication is crucial to obtaining
  the expected identity $!!P \sim !P$.
\end{remark}

\begin{remark}\label{rem:paradoxical_combinator}
  The reader familiar with the lambda calculus will have noticed the
  similarity between $D$ and the paradoxical combinator.

  [Ed. note: the existence of this seems to suggest we have to be more
  restrictive on the set of processes and names we admit if we are to
  support no-cloning.]
\end{remark}

\subsubsection{Bisimulation}

The computational dynamics gives rise to another kind of equivalence,
the equivalence of computational behavior. As previously mentioned
this is typically captured \emph{via} some form of bisimulation.

% The notion we use in this paper is weak barbed bisimulation
% \cite{milner91polyadicpi}.

The notion we use in this paper is derived from weak barbed
bisimulation \cite{milner91polyadicpi}. 

\begin{definition}
An \emph{observation relation}, $\downarrow_{\mathcal N}$, over a set
of names, $\mathcal N$, is the smallest relation satisfying the rules
below.

\infrule[Out-barb]{y \in {\mathcal N}, \; x \nameeq y}
		  {\outputp{x}{v} \downarrow_{\mathcal N} x}
\infrule[Par-barb]{\mbox{$P\downarrow_{\mathcal N} x$ or $Q\downarrow_{\mathcal N} x$}}
		  {\binpar{P}{Q} \downarrow_{\mathcal N} x}

We write $P \Downarrow_{\mathcal N} x$ if there is $Q$ such that 
$P \wred Q$ and $Q \downarrow_{\mathcal N} x$.
\end{definition}

\begin{definition}
%\label{def.bbisim}
An  ${\mathcal N}$-\emph{barbed bisimulation} over a set of names, ${\mathcal N}$, is a symmetric binary relation 
${\mathcal S}_{\mathcal N}$ between agents such that $P\rel{S}_{\mathcal N}Q$ implies:
\begin{enumerate}
\item If $P \red P'$ then $Q \wred Q'$ and $P'\rel{S}_{\mathcal N} Q'$.
\item If $P\downarrow_{\mathcal N} x$, then $Q\Downarrow_{\mathcal N} x$.
\end{enumerate}
$P$ is ${\mathcal N}$-barbed bisimilar to $Q$, written
$P \wbbisim_{\mathcal N} Q$, if $P \rel{S}_{\mathcal N} Q$ for some ${\mathcal N}$-barbed bisimulation ${\mathcal S}_{\mathcal N}$.
\end{definition}

$\mathcal{R} \subseteq \pi \times \pi$

$P \mathcal{R} Q => \forall P'. P \red P' \Rightarrow \exists Q'. Q \red Q', P' \mathcal{R} Q'$

$P \vdash x \Rightarrow Q \vdash x$

\begin{mathpar}
  \inferrule*[lab=Out-barb]{x \nameeq y}{{y}!\langle{Q}\rangle \vdash x}
  \and
  \inferrule*[lab=Par-barb]{\mbox{$P\vdash x$ or $Q\vdash x$}}{\binpar{P}{Q} \vdash x}
\end{mathpar}

\subsubsection{Contexts}

One of the principle advantages of computational calculi like the
$\pi$-calculus is a well-defined notion of context,
contextual-equivalence and a correlation between
contextual-equivalence and notions of bisimulation. The notion of
context allows the decomposition of a process into (sub-)process and
its syntactic environment, its context. Thus, a context may be
thought of as a process with a ``hole'' (written $\Box$) in it. The
application of a context $M$ to a process $P$, written $M[P]$, is
tantamount to filling the hole in $M$ with $P$. In this paper we do
not need the full weight of this theory, but do make use of the notion
of context in the proof the main theorem. 

\begin{mathpar}
  \inferrule* [lab=summation] {} {{M_{M},M_{N}} \bc \Box \;|\; x.M_{A} \;|\; M_{M}+M_{N}}
  \and
  \inferrule* [lab=agent] {} {{M_{A}} \bc (\vec{x})M_{P} \;| \; \clift{P_0,\ldots,M_{P},\ldots,P_N}}
  \and \\
  \inferrule* [lab=process] {} {{M_{P}} \bc M_{N} \;| \;P|M_{P} }
\end{mathpar} 

\begin{mathpar}
  \inferrule* [lab=sychronization] {} {M_{N} \bc \Box \;|\; x?M_{F} \;|\; x!M_{C}}
  \and
  \inferrule* [lab=abstraction] {} {{M_{F}} \bc (x)M_{P} }
  \and
  \inferrule* [lab=concretion] {} {{M_{C}} \bc \langle M_{P} \rangle }
  \and \\
  \inferrule* [lab=process] {} {{M_{P}} \bc M_{N} \;| \;P|M_{P} }
\end{mathpar}

\begin{definition}[contextual application] Given a context $M$, and
  process $P$, we define the \emph{contextual application}, $M[P] :=
  M\{P/\Box\}$. That is, the contextual application of M to P is the
  substitution of $P$ for $\Box$ in $M$.
\end{definition}

$\meaningof{-} : L \to \mathcal{P}(\pi)$

\begin{mathpar}
  \inferrule* [lab=collection] {} {\meaningof{true} = \pi, \and \meaningof{~E} = \pi \setminus \meaningof{E}, \and \meaningof{E_{1} \& E_{2}} = \meaningof{E_{1}} \cap \meaningof{E_{2}}}
\end{mathpar}

\begin{mathpar}
  \inferrule* [lab=structure] {} {\meaningof{0} = \{ P \in \pi | P \equiv 0 \}, \and \\ \meaningof{E_1 | E_2} = \{ P \in \pi | P \equiv P_{1} | P_{2}, P_{1} \in \meaningof{E_{1}}, P_{2} \in \meaningof{E_2}\} }
\end{mathpar}

\begin{mathpar}
 \inferrule* [lab=behavior] {} {\meaningof{\langle a?b \rangle E} = \{ P \in \pi | P \equiv Q | u?(y)P', \\ \and \\\\ \and \\ \;\;\; u \in \meaningof{a}, \forall z.P'\{z/y\} \in \meaningof{E\{z/b\}}\}, \and \\ \meaningof{a!E} = \{ P \in \pi | P \equiv Q | x!\langle P' \rangle, x \in \meaningof{a} P' \in \meaningof{E}\} }
\end{mathpar}

\begin{mathpar}
 \inferrule* [lab=nominal] {} {\meaningof{\quotep{E}} = \{ \quotep{P} \in \quotep{\pi} | P \in \meaningof{E} \}, \and \meaningof{\quotep{P}} = \{ \quotep{Q} \in \quotep{\pi} | P \equiv Q \} \and \\ \meaningof{@\quotep{E}} = \{ P \in \pi | P \equiv @x, x \in \meaningof{E} \}}
\end{mathpar}

\begin{eqnarray*}
  \\
  \meaningof{-} : TS \to ST
\end{eqnarray*}

\begin{eqnarray*}
  \\
  L : TS \to ST
\end{eqnarray*}

\begin{eqnarray*}
  \\
  P \models E \iff P \in \meaningof{E}
\end{eqnarray*}

\begin{eqnarray*}
  P \approx_{L} Q \iff \forall E \in L. P \models E \iff Q \models E
\end{eqnarray*}

\begin{eqnarray*}
  P \approx_{K} Q
\end{eqnarray*}

\begin{eqnarray*}
  P \approx Q
\end{eqnarray*}

$\approx_{K} = \approx = \approx_{L}$

\subsubsection{Contextual duality}

Note that contexts extend the quotation operation to a family of
operations from processes to names. Given a context, $M$, we can
define a \emph{nominal context}, $\quotep{M}$ by $\quotep{M}[P] :=
\quotep{M[P]}$. To foreshadow what is to come we observe that these
operations enjoy a duality with processes very much like the duality
between vectors and maps from vectors to scalars.

Further, because the calculus is essentially higher-order, we have a
correspondence between contexts and processes. More specifically,
given a name $x$ and a context $M$ we can construct $M^{*}_{x}$ such
that 

\begin{mathpar}
  M^{*}_{x} | \lift{x}{P} \red M[P]
\end{mathpar}

namely,

\begin{mathpar}
  M^{*}_{x} := x?(u).M[\dropn{u}]
\end{mathpar}

The dependence of $M^{*}_{x}$ on a name makes it an abstraction, 

\begin{mathpar}
  M^{*} := (x)x?(u).M[\dropn{u}]
\end{mathpar}

\subsection{Additional notation}

It will sometimes be convenient to denote the process a name
quotes. We already have the notation $x = \quotep{P}$, but it will be
convenient to introduce an alternate notation, $\procn{x}$, when we
want to emphasize the connection to the use of the name. Note that, by
virtue of name equivalence, $\quotep{\procn{x}} \nameeq x$; so, the
notation is consistent with previous definitions.

Further, because names have structure it is possible to effect
substitutions on the basis of that structure. This means we need to
upgrade our notation for substitutions, which we accomplish by
adapting comprehension notation. Thus,

\begin{mathpar}
  P\{ y / x : x \in S \}
\end{mathpar}

is interpreted to mean the process derived from P by replacing (in a
capture-avoiding manner) each occurrence of $x$ in $S$ by $y$. For example,

\begin{mathpar}
  P\{ \quotep{\procn{x}|\procn{x}} / x : x \in \freenames{P} \}
\end{mathpar}

will replace each (occurrence) of a free name $x$ in $P$ by
$\quotep{\procn{x}|\procn{x}}$.

Also, we will avail ourselves of the notation $x^{L}$ and $x^{R}$ to
denote injections of a name into disjoint copies of the name
space. There are numerous ways to accomplish this. One example can be
found in \cite{MeredithR05}. This notation overloads to vectors of
names: $\vec{x}^{\pi} := (x_{i}^{\pi} \; : \; 0 \leq i < |\vec{x}| )$ where $\pi \in \{L,R\}$.

We also use $P^{\Box} := P|\Box$.

In \cite{MeredithR05} an interpretation of the new operator is
given. It turns out that there are several possible interpretations
all enjoying the requisite algebraic properties of the operator (see
\cite{milner91polyadicpi}). We will therefore make liberal use of
$(\nu\; \vec{x})P$.

% subsection the_syntax_and_semantics_of_the_notation_system (end)   

\input{qm2pi.qmops} 

\input{qm2pi.sterngerlach} 

\input{qm2pi.metric} 

% section concurrent_process_calculi (end)

%\input{qm2pi.proofsketch}

% section proof sketch (end)

%\input{qm2pi.slviaknots} 

% section spatial logic via knots (end)

\input{qm2pi.conclusion}

% section conclusion (end)

%\input{qm2pi.dtcodes} 

% section wiring algorithm (end)

\input{qm2pi.ack} 

% section acknowledgments (end)

\newpage


\bibliographystyle{plain}   
\bibliography{../../biblios/main.bib}

\input{qm2pi.rhodetails}

\end{document}

 

% subsection basic_interpretation (end)

%\input{qm2pi.rho.presentation} 
\subsection{The syntax and semantics of the notation system}\label{sub:the_syntax_and_semantics_of_the_notation_system} % (fold)

We now summarize a technical presentation of the calculus that
embodies our theory of dynamics. The typical presentation of such a
calculus follows the style of giving generators and relations on
them. The grammar, below, describing term constructors, freely
generates the set of processes, $\Proc$. This set is then quotiented
by a relation known as structural congruence and it is over this set
that the notion of dynamics is expressed. This presentation is
essentially that of \cite{MeredithR05} with the addition of
polyadicity and summation. For readability we have relegated some of
the technical subtleties to an appendix.

\subsubsection{Process grammar}\label{subsub:process_grammar}

\begin{mathpar}
  \inferrule* [lab=synchronization] {} {{M} \bc \pzero \;|\; x?F \;|\; x!C }
  \and
  \inferrule* [lab=abstraction] {} {{F} \bc (x)P}
  \and
  \inferrule* [lab=concretion] {} {{C} \bc \langle Q \rangle}
  \and
  \inferrule* [lab=process] {} {{P,Q} \bc M \;| \;P|Q \;|\; @{x}}
  \and
  \inferrule* [lab=name] {} {{x} \bc \quotep{P}}
\end{mathpar} 

Note that $\vec{x}$ (resp. $\vec{P}$) denotes a vector of names
(resp. processes) of length $|\vec{x}|$ (resp. $|\vec{P}|$). We adopt
the following useful abbreviations.

\begin{mathpar}
   x?(\vec{y}).P := x.(\vec{y})P \and  x\clift{\vec{P}} := x.\clift{\vec{P}}
   \and x!(y) := \lift{x}{\dropn{y}}
   \and \Pi_{i=0}^{n-1}P_i := P_0 | \ldots | P_{n-1}
\end{mathpar}

\subsubsection{Structural congruence}

\paragraph{Free and bound names and alpha-equivalence.} At the
core of structural equivalence is alpha-equivalence which identifies
process that are the same up to a change of variable. Formally, we
recognize the distinction between free and bound names. The free names
of a process, $\freenames{P}$, may be calculated recursively as
follows:

\begin{mathpar}
\freenames{\pzero} := \emptyset
  \and \\
  \freenames{x?(y).P} := \{ x \} \cup (\freenames{P} \setminus \{ y \})
  \and 
  \freenames{x!\langle P \rangle} := \{ x \} \cup \{ P \} 
  \and \\
  \freenames{P|Q} := \freenames{P} \cup \freenames{Q}
  \and \\
  \freenames{@{x}} := \{ x \}
\end{mathpar}

$\pi$
$\quotep{\pi}$

$\freenames{-} : \pi \to \mathcal{P}(\quotep{\pi})$

\begin{eqnarray*}
  \freenames{\pzero} & := & \emptyset \\
  \freenames{x?(y).P} & := & \{ x \} \cup (\freenames{P} \setminus \{ y \}) \\
  \freenames{x!\langle P \rangle} & := & \{ x \} \cup \{ P \} \\
  \freenames{P|Q} & := & \freenames{P} \cup \freenames{Q} \\
  \freenames{\dropn{x}} & := & \{ x \}
\end{eqnarray*}

The bound names of a process, $\boundnames{P}$, are those names occurring in $P$
that are not free. For example, in $x?(y).0$, the name $x$ is free, while $y$ is bound.

\begin{mathpar}
  \inferrule* [lab=monoidal-laws] {} { P|Q \equiv Q|P \and P|0 \equiv P \and P|(Q|R) \equiv (P|Q)|R }
\end{mathpar}

\begin{mathpar}
  \inferrule* [lab=alpha-equivalence] {} { (x)P \equiv (y)P\{y/x\} \and y \not\in \freenames{P} }
\end{mathpar}

\begin{definition}
Then two processes, $P,Q$, are alpha-equivalent if $P = Q\{\vec{y}/\vec{x}\}$ for
some $\vec{x} \in \boundnames{Q},\vec{y} \in \boundnames{P}$, where $Q\{\vec{y}/\vec{x}\}$
denotes the capture-avoiding substitution of $\vec{y}$ for $\vec{x}$ in $Q$.
\end{definition}

\begin{definition}
  The {\em structural congruence} \cite{SangiorgiWalker} , $\equiv$,
  between processes is the least congruence containing
  alpha-equivalence, satisfying the abelian monoid laws
  (associativity, commutativity and $\pzero$ as identity) for parallel
  composition $|$ and for summation $+$.
\end{definition}

\subsection{Name equivalence}

We take name equivalence, written $\nameeq$, to be the smallest
equivalence relation generated by the following rules.

\begin{mathpar}
\inferrule*[lab=Quote-drop]
{ }
{ \quotep{@{x}} \nameeq x }

\inferrule*[lab=Struct-equiv]
{ P \scong Q }
{ \quotep{P} \nameeq \quotep{Q} }
\end{mathpar}

The astute reader will have noticed that the mutual recursion of names
and processes imposes a mutual recursion on alpha-equivalence and
structural equivalence via name-equivalence. Fortunately, all of this
works out pleasantly and we may calculate in the natural way, free of
concern. The reader interested in the details is referred to the
appendix \ref{appendix:rho_details}.

\subsection{Substitution}

We use $\Proc$ for the set of processes, $\QProc$ for the set of
names, and $\id{\{}\vec{y} / \vec{x} \id{\}}$ to denote partial maps,
$s : \QProc \rightarrow \QProc$. A map, $s$ lifts, uniquely, to a map
on process terms, $\widehat{s} : \Proc \rightarrow \Proc$ by the
following equations.

\begin{mathpar}
  (0) \psubstp{Q}{P} := 0 \\
  (R \juxtap S) \psubstp{Q}{P}
  :=    
  (R)\psubstp{Q}{P} \juxtap (S) \psubstp{Q}{P} \\
  (x?(y).R) \psubstp{Q}{P}    
  :=    
  (x)\substp{Q}{P} (z)\concat( (R \psubstn{z}{y}) \psubstp{Q}{P} ) \\
  (\lift{x}{R}) \psubstp{Q}{P}  
  :=
  \lift{(x)\substp{Q}{P}}{ R \psubstp{Q}{P} } \\
%   (\dropn{x})  \psubstp{Q}{P}       
%   := 
%   \left\{ 
%     \begin{array}{ccc} 
%       \dropn{\quotep{Q}} & & x \nameeq \quotep{P} \\
%       \dropn{x} & & otherwise \\
%     \end{array}
%   \right. 
  (\dropn{x})  \psubstp{Q}{P}       
  := 
  \left\{ 
    \begin{array}{ccc} 
      Q & & x \nameeq \quotep{P} \\
      \dropn{x} & & otherwise \\
    \end{array}
  \right.
\end{mathpar}
 

where

\begin{eqnarray}
  (x)\id{\{} \lpquote Q \rpquote / \lpquote P \rpquote \id{\}}            = 
  \left\{ 
    \begin{array}{ccc}
      \lpquote Q \rpquote & & x \nameeq \lpquote P \rpquote \\
      x & & otherwise \\
    \end{array}
  \right. \nonumber
\end{eqnarray}

and $z$ is chosen distinct from $\quotep{P}$, $\quotep{Q}$, the free
names in $Q$, and all the names in $R$. Our $\alpha$-equivalence will
be built in the standard way from this substitution.

\begin{remark}\label{rem:no_self_referential_names}
  One consequence of these definitions is that $\forall P. \quotep{P}
  \not\in \freenames{P}$.
\end{remark}

\subsection{ Dynamic quote: an example }

Anticipating something of what's to come, consider applying the
substitution, $\widehat{\id{\{}u / z \id{\}}}$, to the following pair
of processes, $\lift{w}{y!(z)}$ and $w[ \lpquote y!(z) \rpquote ]$.

\begin{eqnarray}
	\lift{w}{y!(z)}\widehat{\id{\{}u / z \id{\}}}
		& = &
		\lift{w}{y!(u)} \nonumber\\
	w[ \lpquote y!(z) \rpquote ] \widehat{ \id{\{}u / z \id{\}} }
		& = &
		w[ \lpquote y!(z) \rpquote ] \nonumber
\end{eqnarray}

Because the body of the process between quotes is impervious to
substitution, we get radically different answers. In fact, by
examining the first process in an input context,
e.g. $x?(z).\lift{w}{y!(z)}$, we see that the process under the lift
operator may be shaped by prefixed inputs binding a name inside it. In
this sense, the lift operator will be seen as a way to dynamically
construct processes before reifying them as names.

Finally equipped with these standard features we can present the
dynamics of the calculus.

\subsubsection{Operational semantics} 

Finally, we introduce the computational dynamics. What marks these
algebras as distinct from other more traditionally studied algebraic
structures, e.g. vector spaces or polynomial rings, is the manner in
which dynamics is captured. In traditional structures, dynamics is typically
expressed through morphisms between such structures, as in linear maps
between vector spaces or morphisms between rings. In algebras
associated with the semantics of computation, the dynamics is
expressed as part of the algebraic structure itself, through a
reduction reduction relation typically denoted by $\red$. Below, we
give a recursive presentation of this relation for the calculus used
in the encoding.

$\red \subseteq \pi \times \pi$
$\red : \pi \to \mathcal{P}(\pi)$

\begin{mathpar}
  \inferrule* [lab=Comm] { \textsf{match}( x_{src}, x_{trgt} ) } { x_{trgt}?(y)P \; | \; x_{src}!\langle {Q} \rangle \red P\{\quotep{Q}/y}\} }
  \and \\
  \inferrule* [lab=Par] {{P} \red {P}'} {{{P} | {Q}} \red {{P}' | {Q}}}
  \and
  \inferrule* [lab=Equiv]{{{P} \scong {P}'} \andalso {{P}' \red {Q}'} \andalso {{Q}' \scong {Q}}}{{P} \red {Q}}
\end{mathpar}

\begin{eqnarray*}
  match_{\equiv} (\quotep{P},\quotep{Q}) & := & P \equiv Q \\
  match_{\dagger}(\quotep{P},\quotep{Q}) & := & \forall R. P|Q \red^{*} R => R \red^{*} 0 \\
  match_{K}(\quotep{P},\quotep{Q}) & := & K \mbox{ for some context } K
\end{eqnarray*}

$u?(x)P | u!\langle Q \rangle \red P\{\quotep{Q}/x\}$

%We write $\wred$ for $\red^*$, and $P\red$ if $\exists Q $ such that $ P \red Q$.
We write $P\red$ if $\exists Q $ such that $ P \red Q$ and $P\not\red$, otherwise.

\section{Replication}

As mentioned before, it is known that replication (and hence
recursion) can be implemented in a higher-order process algebra
\cite{SangiorgiWalker}. As our first example of calculation with the
machinery thus far presented we give the construction explicitly in
the {\rhoc}.

\begin{eqnarray}
	D_{x} & := & \prefix{x}{y}{(\binpar{\outputp{x}{y}}{@{y}})} \nonumber\\
	\bangp_{x}{P} & := & \binpar{{x}!\langle{\binpar{D_{x}}{P}}\rangle}{D_{x}} \nonumber
\end{eqnarray}

\begin{eqnarray}
	\bangp_{x}{P} & & \nonumber\\
	=
	& {x}!\langle{(\prefix{x}{y}{(\outputp{x}{y} | @{y})) | P}}\rangle 
	      | \prefix{x}{y}{(\outputp{x}{y} | @{y})} & \nonumber\\
	\red
	& (\outputp{x}{y} | @{y})\substn{\quotep{(\prefix{x}{y}{(@{y} | \outputp{x}{y})) | P}}}{y} & \nonumber\\
	=
	& \outputp{x}{\quotep{(\prefix{x}{y}{(\outputp{x}{y} | @{y})) | P}}}
	  | {(\prefix{x}{y}{(\outputp{x}{y} | @{y})) | P}} & \nonumber\\
	\red
	& \ldots & \nonumber\\
	\red^*
	& P | P | \ldots & \nonumber
\end{eqnarray}

Of course, this encoding, as an implementation, runs away, unfolding
$\bangp{P}$ eagerly. A lazier and more implementable replication
operator, restricted to input-guarded processes, may be obtained as follows.

\begin{eqnarray}
\bangp{\prefix{u}{v}{P}} 
	:= 
	\binpar{\lift{x}{\prefix{u}{v}{(\binpar{D(x)}{P})}}}{D(x)} \nonumber
\end{eqnarray}

\begin{remark}
  Note that the lazier definition still does not deal with summation
  or mixed summation (i.e. sums over input and output). The reader is
  invited to construct definitions of replication that deal with these
  features. 

  Further, the definitions are parameterized in a name, $x$. Can you,
  gentle reader, make a definition that eliminates this parameter and
  guarantees no accidental interaction between the replication
  machinery and the process being replicated -- i.e. no accidental
  sharing of names used by the process to get its work done and the
  name(s) used by the replication to effect copying. This latter
  revision of the definition of replication is crucial to obtaining
  the expected identity $!!P \sim !P$.
\end{remark}

\begin{remark}\label{rem:paradoxical_combinator}
  The reader familiar with the lambda calculus will have noticed the
  similarity between $D$ and the paradoxical combinator.

  [Ed. note: the existence of this seems to suggest we have to be more
  restrictive on the set of processes and names we admit if we are to
  support no-cloning.]
\end{remark}

\subsubsection{Bisimulation}

The computational dynamics gives rise to another kind of equivalence,
the equivalence of computational behavior. As previously mentioned
this is typically captured \emph{via} some form of bisimulation.

% The notion we use in this paper is weak barbed bisimulation
% \cite{milner91polyadicpi}.

The notion we use in this paper is derived from weak barbed
bisimulation \cite{milner91polyadicpi}. 

\begin{definition}
An \emph{observation relation}, $\downarrow_{\mathcal N}$, over a set
of names, $\mathcal N$, is the smallest relation satisfying the rules
below.

\infrule[Out-barb]{y \in {\mathcal N}, \; x \nameeq y}
		  {\outputp{x}{v} \downarrow_{\mathcal N} x}
\infrule[Par-barb]{\mbox{$P\downarrow_{\mathcal N} x$ or $Q\downarrow_{\mathcal N} x$}}
		  {\binpar{P}{Q} \downarrow_{\mathcal N} x}

We write $P \Downarrow_{\mathcal N} x$ if there is $Q$ such that 
$P \wred Q$ and $Q \downarrow_{\mathcal N} x$.
\end{definition}

\begin{definition}
%\label{def.bbisim}
An  ${\mathcal N}$-\emph{barbed bisimulation} over a set of names, ${\mathcal N}$, is a symmetric binary relation 
${\mathcal S}_{\mathcal N}$ between agents such that $P\rel{S}_{\mathcal N}Q$ implies:
\begin{enumerate}
\item If $P \red P'$ then $Q \wred Q'$ and $P'\rel{S}_{\mathcal N} Q'$.
\item If $P\downarrow_{\mathcal N} x$, then $Q\Downarrow_{\mathcal N} x$.
\end{enumerate}
$P$ is ${\mathcal N}$-barbed bisimilar to $Q$, written
$P \wbbisim_{\mathcal N} Q$, if $P \rel{S}_{\mathcal N} Q$ for some ${\mathcal N}$-barbed bisimulation ${\mathcal S}_{\mathcal N}$.
\end{definition}

$\mathcal{R} \subseteq \pi \times \pi$

$P \mathcal{R} Q => \forall P'. P \red P' \Rightarrow \exists Q'. Q \red Q', P' \mathcal{R} Q'$

$P \vdash x \Rightarrow Q \vdash x$

\begin{mathpar}
  \inferrule*[lab=Out-barb]{x \nameeq y}{{y}!\langle{Q}\rangle \vdash x}
  \and
  \inferrule*[lab=Par-barb]{\mbox{$P\vdash x$ or $Q\vdash x$}}{\binpar{P}{Q} \vdash x}
\end{mathpar}

\subsubsection{Contexts}

One of the principle advantages of computational calculi like the
$\pi$-calculus is a well-defined notion of context,
contextual-equivalence and a correlation between
contextual-equivalence and notions of bisimulation. The notion of
context allows the decomposition of a process into (sub-)process and
its syntactic environment, its context. Thus, a context may be
thought of as a process with a ``hole'' (written $\Box$) in it. The
application of a context $M$ to a process $P$, written $M[P]$, is
tantamount to filling the hole in $M$ with $P$. In this paper we do
not need the full weight of this theory, but do make use of the notion
of context in the proof the main theorem. 

\begin{mathpar}
  \inferrule* [lab=summation] {} {{M_{M},M_{N}} \bc \Box \;|\; x.M_{A} \;|\; M_{M}+M_{N}}
  \and
  \inferrule* [lab=agent] {} {{M_{A}} \bc (\vec{x})M_{P} \;| \; \clift{P_0,\ldots,M_{P},\ldots,P_N}}
  \and \\
  \inferrule* [lab=process] {} {{M_{P}} \bc M_{N} \;| \;P|M_{P} }
\end{mathpar} 

\begin{mathpar}
  \inferrule* [lab=sychronization] {} {M_{N} \bc \Box \;|\; x?M_{F} \;|\; x!M_{C}}
  \and
  \inferrule* [lab=abstraction] {} {{M_{F}} \bc (x)M_{P} }
  \and
  \inferrule* [lab=concretion] {} {{M_{C}} \bc \langle M_{P} \rangle }
  \and \\
  \inferrule* [lab=process] {} {{M_{P}} \bc M_{N} \;| \;P|M_{P} }
\end{mathpar}

\begin{definition}[contextual application] Given a context $M$, and
  process $P$, we define the \emph{contextual application}, $M[P] :=
  M\{P/\Box\}$. That is, the contextual application of M to P is the
  substitution of $P$ for $\Box$ in $M$.
\end{definition}

$\meaningof{-} : L \to \mathcal{P}(\pi)$

\begin{mathpar}
  \inferrule* [lab=collection] {} {\meaningof{true} = \pi, \and \meaningof{~E} = \pi \setminus \meaningof{E}, \and \meaningof{E_{1} \& E_{2}} = \meaningof{E_{1}} \cap \meaningof{E_{2}}}
\end{mathpar}

\begin{mathpar}
  \inferrule* [lab=structure] {} {\meaningof{0} = \{ P \in \pi | P \equiv 0 \}, \and \\ \meaningof{E_1 | E_2} = \{ P \in \pi | P \equiv P_{1} | P_{2}, P_{1} \in \meaningof{E_{1}}, P_{2} \in \meaningof{E_2}\} }
\end{mathpar}

\begin{mathpar}
 \inferrule* [lab=behavior] {} {\meaningof{\langle a?b \rangle E} = \{ P \in \pi | P \equiv Q | u?(y)P', \\ \and \\\\ \and \\ \;\;\; u \in \meaningof{a}, \forall z.P'\{z/y\} \in \meaningof{E\{z/b\}}\}, \and \\ \meaningof{a!E} = \{ P \in \pi | P \equiv Q | x!\langle P' \rangle, x \in \meaningof{a} P' \in \meaningof{E}\} }
\end{mathpar}

\begin{mathpar}
 \inferrule* [lab=nominal] {} {\meaningof{\quotep{E}} = \{ \quotep{P} \in \quotep{\pi} | P \in \meaningof{E} \}, \and \meaningof{\quotep{P}} = \{ \quotep{Q} \in \quotep{\pi} | P \equiv Q \} \and \\ \meaningof{@\quotep{E}} = \{ P \in \pi | P \equiv @x, x \in \meaningof{E} \}}
\end{mathpar}

\begin{eqnarray*}
  \\
  \meaningof{-} : TS \to ST
\end{eqnarray*}

\begin{eqnarray*}
  \\
  L : TS \to ST
\end{eqnarray*}

\begin{eqnarray*}
  \\
  P \models E \iff P \in \meaningof{E}
\end{eqnarray*}

\begin{eqnarray*}
  P \approx_{L} Q \iff \forall E \in L. P \models E \iff Q \models E
\end{eqnarray*}

\begin{eqnarray*}
  P \approx_{K} Q
\end{eqnarray*}

\begin{eqnarray*}
  P \approx Q
\end{eqnarray*}

$\approx_{K} = \approx = \approx_{L}$

\subsubsection{Contextual duality}

Note that contexts extend the quotation operation to a family of
operations from processes to names. Given a context, $M$, we can
define a \emph{nominal context}, $\quotep{M}$ by $\quotep{M}[P] :=
\quotep{M[P]}$. To foreshadow what is to come we observe that these
operations enjoy a duality with processes very much like the duality
between vectors and maps from vectors to scalars.

Further, because the calculus is essentially higher-order, we have a
correspondence between contexts and processes. More specifically,
given a name $x$ and a context $M$ we can construct $M^{*}_{x}$ such
that 

\begin{mathpar}
  M^{*}_{x} | \lift{x}{P} \red M[P]
\end{mathpar}

namely,

\begin{mathpar}
  M^{*}_{x} := x?(u).M[\dropn{u}]
\end{mathpar}

The dependence of $M^{*}_{x}$ on a name makes it an abstraction, 

\begin{mathpar}
  M^{*} := (x)x?(u).M[\dropn{u}]
\end{mathpar}

\subsection{Additional notation}

It will sometimes be convenient to denote the process a name
quotes. We already have the notation $x = \quotep{P}$, but it will be
convenient to introduce an alternate notation, $\procn{x}$, when we
want to emphasize the connection to the use of the name. Note that, by
virtue of name equivalence, $\quotep{\procn{x}} \nameeq x$; so, the
notation is consistent with previous definitions.

Further, because names have structure it is possible to effect
substitutions on the basis of that structure. This means we need to
upgrade our notation for substitutions, which we accomplish by
adapting comprehension notation. Thus,

\begin{mathpar}
  P\{ y / x : x \in S \}
\end{mathpar}

is interpreted to mean the process derived from P by replacing (in a
capture-avoiding manner) each occurrence of $x$ in $S$ by $y$. For example,

\begin{mathpar}
  P\{ \quotep{\procn{x}|\procn{x}} / x : x \in \freenames{P} \}
\end{mathpar}

will replace each (occurrence) of a free name $x$ in $P$ by
$\quotep{\procn{x}|\procn{x}}$.

Also, we will avail ourselves of the notation $x^{L}$ and $x^{R}$ to
denote injections of a name into disjoint copies of the name
space. There are numerous ways to accomplish this. One example can be
found in \cite{MeredithR05}. This notation overloads to vectors of
names: $\vec{x}^{\pi} := (x_{i}^{\pi} \; : \; 0 \leq i < |\vec{x}| )$ where $\pi \in \{L,R\}$.

We also use $P^{\Box} := P|\Box$.

In \cite{MeredithR05} an interpretation of the new operator is
given. It turns out that there are several possible interpretations
all enjoying the requisite algebraic properties of the operator (see
\cite{milner91polyadicpi}). We will therefore make liberal use of
$(\nu\; \vec{x})P$.

% subsection the_syntax_and_semantics_of_the_notation_system (end)   

\section{Interpretation of QM}
\subsection{Supporting definitions}
\subsubsection{Multiplication}
\begin{mathpar}
  \quotep{Q} \cdot \quotep{R} := \quotep{Q|R}
  \and \\
  \quotep{Q} \cdot P := P\{ \quotep{Q|R} / \quotep{R} : \quotep{R} \in \freenames{P} \}
\end{mathpar}

\paragraph{Discussion}
The first line needs little explanation. The second line says that
each free name of the process is replaced with the multiplication of
that name by the scalar. Multiplication of a scalar (name) by a state
(process) results in a process all the names of which have been `moved
over' by parallel composition with the process the scalar
quotes. There is a subtlety that the bound names have to be
manipulated so that multiplied names aren't accidentally
captured. There are many ways to achieve this.

\begin{remark}\label{rem:multiplication_identities}
  The reader is invited to verify that for all $x,y,z \in \QProc$ and $P \in \Proc$
  \begin{mathpar}
    x \cdot \quotep{0} \equiv x 
    \and
    x \cdot y \equiv y \cdot x
    \and
    x \cdot (y \cdot z) \equiv (x \cdot y) \cdot z
    \and \\
    \quotep{0} \cdot P \equiv P
    \and \\
    x \cdot (y \cdot P) \equiv (x \cdot y) \cdot P
    \and \\
    x \cdot (P|Q) \equiv (x \cdot P) | (x \cdot Q)
    \and \\    
  \end{mathpar}
\end{remark}

\subsubsection{Tensor product}

We define a tensor product on processes by structural induction.

\paragraph{Tensor of sums} First note that all summations, including
$\pzero$ and sequence, can be written $\Sigma_{i} x_{i}.A_{i} +
\Sigma_{j} x_{j}.C_{j}$, where we have grouped input-guarded processes
together and output-guarded processes together.

Thus, we can define the tensor product of two summations, $N_{1}\otimes N_{2}$, where

\begin{mathpar}
  N_{1} := \Sigma_{i} x_{i}.A_{i} + \Sigma_{j} x_{j}.C_{j}
  \and
  N_{2} := \Sigma_{i'} y_{i'}.B_{i'} + \Sigma_{j'} y_{j'}.D_{j'} 
\end{mathpar}

as follows.

\begin{mathpar}
  \Sigma_{i} x_{i}.A_{i} + \Sigma_{j} x_{j}.C_{j} \otimes \Sigma_{i'}
  y_{i'}.B_{i'} + \Sigma_{j'} y_{j'}.D_{j'} 
  \and \\
  := \; \Sigma_{i} \Sigma_{i'} \quotep{\stackrel{\vee}{x_{i}}| \stackrel{\vee}{y_{i'}}}.(A_{i}\otimes B_{i'}) \; | \; \Sigma_{i'} \Sigma_{i} \quotep{\stackrel{\vee}{y_{i'}}|\stackrel{\vee}{x_{i}}}.(B_{i'}\otimes A_{i})
  \and
  \;\; | \;\; \Sigma_{j} \Sigma_{j'} \quotep{\stackrel{\vee}{x_{j}}|\stackrel{\vee}{y_{j'}}}.(A_{j}\otimes B_{j'}) \; | \; \Sigma_{j'} \Sigma_{j} \quotep{\stackrel{\vee}{y_{j'}}|\stackrel{\vee}{x_{j}}}.(B_{j'}\otimes A_{j})
\end{mathpar}

\begin{remark}
  Do we need to $x^{L}$ and $y^{R}$ for this construction as well?
\end{remark}

\paragraph{Tensor of parallel compositions} Next, we distribute tensor
over par.

\begin{mathpar}
  P_{1}|P_{2} \otimes Q_{1}|Q_{2} := (P_{1} \otimes Q_{1}) | (P_{1}
  \otimes Q_{2}) | (P_{2} \otimes Q_{1}) | (P_{2} \otimes Q_{2})
\end{mathpar}

\paragraph{Tensor with dropped names} We treat tensor of a
process with a dropped name as parallel composition.

\begin{mathpar}
  P \otimes \dropn{x} := P | \dropn{x}
\end{mathpar}

\paragraph{Tensor of agents}

Finally, we need to define tensor on agents. Note that the definition
of tensor on normal products only tensors inputs with inputs and
outputs with outputs. Thus, we only have to define the operation on
``homogeneous'' pairings.

\begin{mathpar}
  (\vec{x})P \otimes (\vec{y})Q
  \and \\
  := (x_{0}^{L}|y_{0}^{R},\ldots,x_{0}^{L}|y_{n}^{R},\ldots,x_{m}^{L}|y_{0}^{R},\ldots,x_{m}^{L}|y_{n}^R)(P\{ \vec{x}^{L}/\vec{x}\} \otimes Q \{ \vec{y}^{R}/\vec{y}\})
  \and \\
  \clift{\vec{P}} \otimes \clift{\vec{Q}}
  \and \\
  := \clift{P_{0}\otimes Q_{0},\ldots,P_{0}\otimes Q_{n},\ldots,P_{m}\otimes Q_{0},\ldots,P_{m}\otimes Q_{n}}
\end{mathpar}

\begin{remark}
  Observe that arities of tensored abstractions matches arities of
  tensored concretions if the original arities matched. Note also that
  the length of the arities corresponds to the increase in dimension
  we see in ordinary vector space tensor product.
\end{remark}

\begin{remark}
  Operationally, this definition distributes the tensor down to
  components ``linked'' by summation. Tensor over summation is
  intriguing in that it mixes names. Moreover, as a consequence of the
  way it mixes names we have the identities for all $x \in \QProc$ and
  $P,Q \in \Proc$

  \begin{mathpar}
    (x \cdot P) \otimes Q \equiv x \cdot (P \otimes Q) \equiv P \otimes (x \cdot Q)
    \and
    P \otimes \pzero \equiv P
  \end{mathpar}

  that the reader is invited to verify.
\end{remark}

\subsubsection{Annihilation}
\begin{mathpar}
  P^{\perp} := \{ Q | \forall R. P|Q \red^{*} R \Rightarrow R \red^{*} \pzero \}
  \and \\
  P^{\underline{\perp}} := \Sigma_{Q \in P^{\perp}} \quotep{Q}?(y).(\dropn{y}|Q) | \Sigma_{Q \in P^{\perp}} \quotep{Q}\clift{\Box}
\end{mathpar}

\paragraph{Discussion} The reader will note that $P^{\perp}$ is a
\emph{set} of processes, while $P^{\underline{\perp}}$ is a
\emph{context}. We call the set $P^{\perp}$ the \emph{annihilators} of
$P$. The parallel composition of a process in the annihilators of $P$
with $P$ will result in a process, the state space of which has all
paths eventually leading to $\pzero$. Execution may endure loops; but
under reasonable conditions of fairness (naturally guaranteed under
most notions of bisimulation) such a composite process cannot get
stuck in such a loop and will, eventually pop out and terminate.

The context $P^{\underline{\perp}}$ is ready and willing to ``take the
$P$ out of'' the process to which it is applied. It will effectively
transmit the code of the process to which it is applied to one of the
annihilators and run the process against it.

\subsubsection{Evaluation}
We fix $M$ a domain of fully abstract interpretation with an equality
coincident with bisimulation. We take $\meaningof{\cdot} : \Proc \to
M$ to be the map interpreting processes and $\nmeaningof{\cdot} : \M
\to Proc$ to be the map running the other way. Then we define

\begin{mathpar}
  \int P := \nmeaningof{\meaningof{P}}
\end{mathpar}

\paragraph{Discussion}
There are many fully abstract interpretations of Milner's
$\pi$-calculus. Any of them can be used as a basis for interpreting
the reflective calculus here. Equipped with such a domain it is
largely a matter of grinding through to check that the Yoneda
construction for the normalization-by-evaluation program can be
extended to this setting.

\begin{remark}
  The reader is invited to verify that $\int (P^{\underline{\perp}}[P]) = 0$.
\end{remark}

\subsection{Quantum mechanics}

Table \ref{tbl:core_qm_op_defns} gives the core operational definitions

\begin{table}[htp]\label{tbl:core_qm_op_defns}
  \center{
    \fbox{
      \begin{tabular}{c|c}
        quantum mechanics & process calculus \\
        \hline
        scalar & $x := \quotep{P}$ \\
        state vector & $\state{P} := P$ \\
        dual & $\state{P}^{*} := \event{P^{\underline{\perp}}} := \quotep{P^{\underline{\perp}}}[-]$ \\
        matrix & $ \Sigma_{\alpha} \state{P_{\alpha}}x_{\alpha}\event{Q_{\alpha}}$ \\
        vector addition & $\state{P} + \state{Q} := \state{P | Q}$ \\
        tensor product & $\state{P} \otimes \state{Q} := \state{P \otimes Q}$ \\
        inner product & $\innerprod{P}{Q} := \quotep{\int P^{\underline{\perp}}[Q]}$ \\
      \end{tabular}
    }
  }
  \caption{QM - operational definitions}
\end{table}

where

\begin{mathpar}
  \prmatrix{P}{Q} := \fprmatrix{P}{\quotep{\pzero}}{Q}
  \and
  \fprmatrix{P}{x}{Q} := (\state{P},x,\event{Q})
  \and
  (\fprmatrix{P}{x}{Q})(\state{R}) := x \cdot \innerprod{Q}{R} \cdot \state{P}
  \and
  (\fprmatrix{P}{x}{Q})(\event{R}) := x \cdot \innerprod{R}{P} \cdot \event{Q}
\end{mathpar}

\paragraph{Discussion}
As promised: vectors (aka states) are represented as processes; duals
as contextual duals; inner product definition should be compared with
standard inner product definition for ....

\begin{remark}
  Assuming $\int (P^{\underline{\perp}}[P]) = 0$, the reader is
  invited to verify that $(\fprmatrix{P}{x}{P})(\state{P}) = x \cdot \state{P}$.
\end{remark}

\begin{remark}
  The reader is invited to verify that $\innerprod{P}{Q}$ could
  equally well have been written $\quotep{\int \stackrel{\vee}{x}}$
  where $x = \event{P^{\underline{\perp}}}(Q)$.

  One of the motivations for this remark is that there is another way
  to factor these operations. We could package up evaluation in the dual:

  \begin{mathpar}
    \state{P}^{*} := \event{\int P^{\underline{\perp}}} := \quotep{\int P^{\underline{\perp}}}[-]
  \end{mathpar}

  and then have inner product defined by
  
  \begin{mathpar}
    \innerprod{P}{Q} := \event{P}(Q)
  \end{mathpar}

  Hopefully, experience with the calculations will provide guidance on
  the best factoring.
\end{remark}

\begin{remark}
  Assuming $\int (P^{\underline{\perp}}[P]) = 0$, the reader is
  invited to verify that $\forall P,Q. (\prmatrix{0}{Q})(\state{0}) =
  \state{0}$ and dually $(\prmatrix{P}{0})(\event{0}) = \event{0}$.
\end{remark}

\begin{remark}
  i'm a little worried that i don't (yet) have proper support for
  complex conjugacy. But, the observation above may give us a
  clue. According to Abramsky, it must be the case that the scalars
  are iso to the homset of the identity for the tensor -- which the
  observation above characterizes. 

  For now, we will simply bookmark the notion with $\overline{x}$.
\end{remark}

\subsubsection{Adjointness}

We need to give a definition of $(\cdot)^{\dagger}$ for matrices. The
obvious candidate definition is
\begin{mathpar}
(\Sigma_{\alpha}\fprmatrix{P_{\alpha}}{x_{\alpha}}{Q_{\alpha}})^{\dagger}
= \Sigma_{\alpha}\fprmatrix{(Q_{\alpha}^{\underline{\perp}})^{*}}{\overline{x}_{\alpha}}{P_{\alpha}^{\underline{\perp}}} 
\end{mathpar}

But, $(Q_{\alpha}^{\underline{\perp}})^{*}$ requires a name along
which to communicate the process to achieve the context application.

\subsubsection{Basis for a basis}
If processes label states and ``addition'' of states (a.k.a. vector
addition) is interpreted as parallel composition, what corresponds to
notions of linear independence and basis? Here, we recall that Yoshida
has developed a set of \emph{combinators} for an asynchronous verison
of Milner's $\pi$-calculus. These are a finite set of processes such
any process can be expressed as parallel composition of these
combinators together with liberal uses of the new operator and
replication. We can simply give a translation of these into the
present calculus and have reasonable expectation that the property
carries over. That is, that the resultant set allows to express all
processes via parallel composition. Note, however, that there is no
new operator or replication in this calculus. As a result, we expect
that the corresponding set is actually infinite. That is, we expect
that the space is actually infinite dimensional.

\begin{remark}
  The attentive reader may be a bit concerned. Certainly, the
  collection $S$, $K$ and $I$ is a finite set of
  combinators. Shouldn't we expect to see a finite set of combinators
  for an effectively equivalent system? i am very sympathetic to this
  critique and feel it warrants full attention. On the other hand, i
  also have in mind the following analogy. The natural numbers, as a
  monoid under addition, has exactly $1$ generator, while the natural
  numbers, as a monoid under multiplication, has countably many
  generators (the primes). We observe that the application of the
  lambda calculus is much less resource sensitive than the parallel
  composition of the $\pi$-calculus. Could it be the case that we have
  an analogy of the form
  
  \begin{mathpar}
    m + n : MN :: m*n : M|N
  \end{mathpar}

  giving a similar blow up in the set of ``primes''?  This is such a
  wonderful thought that, even if it's not true, i think it's worth
  writing down.
\end{remark}
 

\documentclass[12pt]{llncs}
%\documentclass{jktr}

\usepackage[pdftex]{hyperref}                   
\usepackage {listings}
\usepackage {mathpartir}
\usepackage{bcprules}
%\usepackage{listings}
                       
\usepackage{graphicx} 
%\usepackage[margins=2.5cm,nohead,nofoot]{geometry}
%\usepackage{geometry}
\usepackage{amsfonts}
\usepackage{amstext}
\usepackage{latexsym}
\usepackage{amssymb}
\usepackage{color}


%\include{myPreamble}
\include{qm2pi.local} 

%\ifpdf
%\usepackage[pdftex]{graphicx}
%\else
%\usepackage{graphicx}
%\fi

 % \ifpdf
%  \usepackage{pdfsync}
%  \if


%\title{Brief Article}
%\author{David F. Snyder}
%\author{L.G. Meredith}

%\address{Dept. of Math., Texas State University--San Marcos, San Marcos, TX 78666}
       
\pagestyle{empty}


\begin{document}

\lstset{language=[Objective]Caml,frame=shadowbox}

\input{qm2pi.front}

% section front matter (end)

\input{qm2pi.intro} 
 
% section introduction (end)

% \input{qm2pi.knotations} 

% section notation (end)

\input{qm2pi.process.calculi} 

% section concurrent_process_calculi_and_spatial_logics_ (end)
    
%\input{qm2pi.knots2pi} 

%\input{qm2pi.trefoil} 

%\input{qm2pi.mainthm} 

% subsection basic_interpretation (end)

%\input{qm2pi.rho.presentation} 
\subsection{The syntax and semantics of the notation system}\label{sub:the_syntax_and_semantics_of_the_notation_system} % (fold)

We now summarize a technical presentation of the calculus that
embodies our theory of dynamics. The typical presentation of such a
calculus follows the style of giving generators and relations on
them. The grammar, below, describing term constructors, freely
generates the set of processes, $\Proc$. This set is then quotiented
by a relation known as structural congruence and it is over this set
that the notion of dynamics is expressed. This presentation is
essentially that of \cite{MeredithR05} with the addition of
polyadicity and summation. For readability we have relegated some of
the technical subtleties to an appendix.

\subsubsection{Process grammar}\label{subsub:process_grammar}

\begin{mathpar}
  \inferrule* [lab=synchronization] {} {{M} \bc \pzero \;|\; x?F \;|\; x!C }
  \and
  \inferrule* [lab=abstraction] {} {{F} \bc (x)P}
  \and
  \inferrule* [lab=concretion] {} {{C} \bc \langle Q \rangle}
  \and
  \inferrule* [lab=process] {} {{P,Q} \bc M \;| \;P|Q \;|\; @{x}}
  \and
  \inferrule* [lab=name] {} {{x} \bc \quotep{P}}
\end{mathpar} 

Note that $\vec{x}$ (resp. $\vec{P}$) denotes a vector of names
(resp. processes) of length $|\vec{x}|$ (resp. $|\vec{P}|$). We adopt
the following useful abbreviations.

\begin{mathpar}
   x?(\vec{y}).P := x.(\vec{y})P \and  x\clift{\vec{P}} := x.\clift{\vec{P}}
   \and x!(y) := \lift{x}{\dropn{y}}
   \and \Pi_{i=0}^{n-1}P_i := P_0 | \ldots | P_{n-1}
\end{mathpar}

\subsubsection{Structural congruence}

\paragraph{Free and bound names and alpha-equivalence.} At the
core of structural equivalence is alpha-equivalence which identifies
process that are the same up to a change of variable. Formally, we
recognize the distinction between free and bound names. The free names
of a process, $\freenames{P}$, may be calculated recursively as
follows:

\begin{mathpar}
\freenames{\pzero} := \emptyset
  \and \\
  \freenames{x?(y).P} := \{ x \} \cup (\freenames{P} \setminus \{ y \})
  \and 
  \freenames{x!\langle P \rangle} := \{ x \} \cup \{ P \} 
  \and \\
  \freenames{P|Q} := \freenames{P} \cup \freenames{Q}
  \and \\
  \freenames{@{x}} := \{ x \}
\end{mathpar}

$\pi$
$\quotep{\pi}$

$\freenames{-} : \pi \to \mathcal{P}(\quotep{\pi})$

\begin{eqnarray*}
  \freenames{\pzero} & := & \emptyset \\
  \freenames{x?(y).P} & := & \{ x \} \cup (\freenames{P} \setminus \{ y \}) \\
  \freenames{x!\langle P \rangle} & := & \{ x \} \cup \{ P \} \\
  \freenames{P|Q} & := & \freenames{P} \cup \freenames{Q} \\
  \freenames{\dropn{x}} & := & \{ x \}
\end{eqnarray*}

The bound names of a process, $\boundnames{P}$, are those names occurring in $P$
that are not free. For example, in $x?(y).0$, the name $x$ is free, while $y$ is bound.

\begin{mathpar}
  \inferrule* [lab=monoidal-laws] {} { P|Q \equiv Q|P \and P|0 \equiv P \and P|(Q|R) \equiv (P|Q)|R }
\end{mathpar}

\begin{mathpar}
  \inferrule* [lab=alpha-equivalence] {} { (x)P \equiv (y)P\{y/x\} \and y \not\in \freenames{P} }
\end{mathpar}

\begin{definition}
Then two processes, $P,Q$, are alpha-equivalent if $P = Q\{\vec{y}/\vec{x}\}$ for
some $\vec{x} \in \boundnames{Q},\vec{y} \in \boundnames{P}$, where $Q\{\vec{y}/\vec{x}\}$
denotes the capture-avoiding substitution of $\vec{y}$ for $\vec{x}$ in $Q$.
\end{definition}

\begin{definition}
  The {\em structural congruence} \cite{SangiorgiWalker} , $\equiv$,
  between processes is the least congruence containing
  alpha-equivalence, satisfying the abelian monoid laws
  (associativity, commutativity and $\pzero$ as identity) for parallel
  composition $|$ and for summation $+$.
\end{definition}

\subsection{Name equivalence}

We take name equivalence, written $\nameeq$, to be the smallest
equivalence relation generated by the following rules.

\begin{mathpar}
\inferrule*[lab=Quote-drop]
{ }
{ \quotep{@{x}} \nameeq x }

\inferrule*[lab=Struct-equiv]
{ P \scong Q }
{ \quotep{P} \nameeq \quotep{Q} }
\end{mathpar}

The astute reader will have noticed that the mutual recursion of names
and processes imposes a mutual recursion on alpha-equivalence and
structural equivalence via name-equivalence. Fortunately, all of this
works out pleasantly and we may calculate in the natural way, free of
concern. The reader interested in the details is referred to the
appendix \ref{appendix:rho_details}.

\subsection{Substitution}

We use $\Proc$ for the set of processes, $\QProc$ for the set of
names, and $\id{\{}\vec{y} / \vec{x} \id{\}}$ to denote partial maps,
$s : \QProc \rightarrow \QProc$. A map, $s$ lifts, uniquely, to a map
on process terms, $\widehat{s} : \Proc \rightarrow \Proc$ by the
following equations.

\begin{mathpar}
  (0) \psubstp{Q}{P} := 0 \\
  (R \juxtap S) \psubstp{Q}{P}
  :=    
  (R)\psubstp{Q}{P} \juxtap (S) \psubstp{Q}{P} \\
  (x?(y).R) \psubstp{Q}{P}    
  :=    
  (x)\substp{Q}{P} (z)\concat( (R \psubstn{z}{y}) \psubstp{Q}{P} ) \\
  (\lift{x}{R}) \psubstp{Q}{P}  
  :=
  \lift{(x)\substp{Q}{P}}{ R \psubstp{Q}{P} } \\
%   (\dropn{x})  \psubstp{Q}{P}       
%   := 
%   \left\{ 
%     \begin{array}{ccc} 
%       \dropn{\quotep{Q}} & & x \nameeq \quotep{P} \\
%       \dropn{x} & & otherwise \\
%     \end{array}
%   \right. 
  (\dropn{x})  \psubstp{Q}{P}       
  := 
  \left\{ 
    \begin{array}{ccc} 
      Q & & x \nameeq \quotep{P} \\
      \dropn{x} & & otherwise \\
    \end{array}
  \right.
\end{mathpar}
 

where

\begin{eqnarray}
  (x)\id{\{} \lpquote Q \rpquote / \lpquote P \rpquote \id{\}}            = 
  \left\{ 
    \begin{array}{ccc}
      \lpquote Q \rpquote & & x \nameeq \lpquote P \rpquote \\
      x & & otherwise \\
    \end{array}
  \right. \nonumber
\end{eqnarray}

and $z$ is chosen distinct from $\quotep{P}$, $\quotep{Q}$, the free
names in $Q$, and all the names in $R$. Our $\alpha$-equivalence will
be built in the standard way from this substitution.

\begin{remark}\label{rem:no_self_referential_names}
  One consequence of these definitions is that $\forall P. \quotep{P}
  \not\in \freenames{P}$.
\end{remark}

\subsection{ Dynamic quote: an example }

Anticipating something of what's to come, consider applying the
substitution, $\widehat{\id{\{}u / z \id{\}}}$, to the following pair
of processes, $\lift{w}{y!(z)}$ and $w[ \lpquote y!(z) \rpquote ]$.

\begin{eqnarray}
	\lift{w}{y!(z)}\widehat{\id{\{}u / z \id{\}}}
		& = &
		\lift{w}{y!(u)} \nonumber\\
	w[ \lpquote y!(z) \rpquote ] \widehat{ \id{\{}u / z \id{\}} }
		& = &
		w[ \lpquote y!(z) \rpquote ] \nonumber
\end{eqnarray}

Because the body of the process between quotes is impervious to
substitution, we get radically different answers. In fact, by
examining the first process in an input context,
e.g. $x?(z).\lift{w}{y!(z)}$, we see that the process under the lift
operator may be shaped by prefixed inputs binding a name inside it. In
this sense, the lift operator will be seen as a way to dynamically
construct processes before reifying them as names.

Finally equipped with these standard features we can present the
dynamics of the calculus.

\subsubsection{Operational semantics} 

Finally, we introduce the computational dynamics. What marks these
algebras as distinct from other more traditionally studied algebraic
structures, e.g. vector spaces or polynomial rings, is the manner in
which dynamics is captured. In traditional structures, dynamics is typically
expressed through morphisms between such structures, as in linear maps
between vector spaces or morphisms between rings. In algebras
associated with the semantics of computation, the dynamics is
expressed as part of the algebraic structure itself, through a
reduction reduction relation typically denoted by $\red$. Below, we
give a recursive presentation of this relation for the calculus used
in the encoding.

$\red \subseteq \pi \times \pi$
$\red : \pi \to \mathcal{P}(\pi)$

\begin{mathpar}
  \inferrule* [lab=Comm] { \textsf{match}( x_{src}, x_{trgt} ) } { x_{trgt}?(y)P \; | \; x_{src}!\langle {Q} \rangle \red P\{\quotep{Q}/y}\} }
  \and \\
  \inferrule* [lab=Par] {{P} \red {P}'} {{{P} | {Q}} \red {{P}' | {Q}}}
  \and
  \inferrule* [lab=Equiv]{{{P} \scong {P}'} \andalso {{P}' \red {Q}'} \andalso {{Q}' \scong {Q}}}{{P} \red {Q}}
\end{mathpar}

\begin{eqnarray*}
  match_{\equiv} (\quotep{P},\quotep{Q}) & := & P \equiv Q \\
  match_{\dagger}(\quotep{P},\quotep{Q}) & := & \forall R. P|Q \red^{*} R => R \red^{*} 0 \\
  match_{K}(\quotep{P},\quotep{Q}) & := & K \mbox{ for some context } K
\end{eqnarray*}

$u?(x)P | u!\langle Q \rangle \red P\{\quotep{Q}/x\}$

%We write $\wred$ for $\red^*$, and $P\red$ if $\exists Q $ such that $ P \red Q$.
We write $P\red$ if $\exists Q $ such that $ P \red Q$ and $P\not\red$, otherwise.

\section{Replication}

As mentioned before, it is known that replication (and hence
recursion) can be implemented in a higher-order process algebra
\cite{SangiorgiWalker}. As our first example of calculation with the
machinery thus far presented we give the construction explicitly in
the {\rhoc}.

\begin{eqnarray}
	D_{x} & := & \prefix{x}{y}{(\binpar{\outputp{x}{y}}{@{y}})} \nonumber\\
	\bangp_{x}{P} & := & \binpar{{x}!\langle{\binpar{D_{x}}{P}}\rangle}{D_{x}} \nonumber
\end{eqnarray}

\begin{eqnarray}
	\bangp_{x}{P} & & \nonumber\\
	=
	& {x}!\langle{(\prefix{x}{y}{(\outputp{x}{y} | @{y})) | P}}\rangle 
	      | \prefix{x}{y}{(\outputp{x}{y} | @{y})} & \nonumber\\
	\red
	& (\outputp{x}{y} | @{y})\substn{\quotep{(\prefix{x}{y}{(@{y} | \outputp{x}{y})) | P}}}{y} & \nonumber\\
	=
	& \outputp{x}{\quotep{(\prefix{x}{y}{(\outputp{x}{y} | @{y})) | P}}}
	  | {(\prefix{x}{y}{(\outputp{x}{y} | @{y})) | P}} & \nonumber\\
	\red
	& \ldots & \nonumber\\
	\red^*
	& P | P | \ldots & \nonumber
\end{eqnarray}

Of course, this encoding, as an implementation, runs away, unfolding
$\bangp{P}$ eagerly. A lazier and more implementable replication
operator, restricted to input-guarded processes, may be obtained as follows.

\begin{eqnarray}
\bangp{\prefix{u}{v}{P}} 
	:= 
	\binpar{\lift{x}{\prefix{u}{v}{(\binpar{D(x)}{P})}}}{D(x)} \nonumber
\end{eqnarray}

\begin{remark}
  Note that the lazier definition still does not deal with summation
  or mixed summation (i.e. sums over input and output). The reader is
  invited to construct definitions of replication that deal with these
  features. 

  Further, the definitions are parameterized in a name, $x$. Can you,
  gentle reader, make a definition that eliminates this parameter and
  guarantees no accidental interaction between the replication
  machinery and the process being replicated -- i.e. no accidental
  sharing of names used by the process to get its work done and the
  name(s) used by the replication to effect copying. This latter
  revision of the definition of replication is crucial to obtaining
  the expected identity $!!P \sim !P$.
\end{remark}

\begin{remark}\label{rem:paradoxical_combinator}
  The reader familiar with the lambda calculus will have noticed the
  similarity between $D$ and the paradoxical combinator.

  [Ed. note: the existence of this seems to suggest we have to be more
  restrictive on the set of processes and names we admit if we are to
  support no-cloning.]
\end{remark}

\subsubsection{Bisimulation}

The computational dynamics gives rise to another kind of equivalence,
the equivalence of computational behavior. As previously mentioned
this is typically captured \emph{via} some form of bisimulation.

% The notion we use in this paper is weak barbed bisimulation
% \cite{milner91polyadicpi}.

The notion we use in this paper is derived from weak barbed
bisimulation \cite{milner91polyadicpi}. 

\begin{definition}
An \emph{observation relation}, $\downarrow_{\mathcal N}$, over a set
of names, $\mathcal N$, is the smallest relation satisfying the rules
below.

\infrule[Out-barb]{y \in {\mathcal N}, \; x \nameeq y}
		  {\outputp{x}{v} \downarrow_{\mathcal N} x}
\infrule[Par-barb]{\mbox{$P\downarrow_{\mathcal N} x$ or $Q\downarrow_{\mathcal N} x$}}
		  {\binpar{P}{Q} \downarrow_{\mathcal N} x}

We write $P \Downarrow_{\mathcal N} x$ if there is $Q$ such that 
$P \wred Q$ and $Q \downarrow_{\mathcal N} x$.
\end{definition}

\begin{definition}
%\label{def.bbisim}
An  ${\mathcal N}$-\emph{barbed bisimulation} over a set of names, ${\mathcal N}$, is a symmetric binary relation 
${\mathcal S}_{\mathcal N}$ between agents such that $P\rel{S}_{\mathcal N}Q$ implies:
\begin{enumerate}
\item If $P \red P'$ then $Q \wred Q'$ and $P'\rel{S}_{\mathcal N} Q'$.
\item If $P\downarrow_{\mathcal N} x$, then $Q\Downarrow_{\mathcal N} x$.
\end{enumerate}
$P$ is ${\mathcal N}$-barbed bisimilar to $Q$, written
$P \wbbisim_{\mathcal N} Q$, if $P \rel{S}_{\mathcal N} Q$ for some ${\mathcal N}$-barbed bisimulation ${\mathcal S}_{\mathcal N}$.
\end{definition}

$\mathcal{R} \subseteq \pi \times \pi$

$P \mathcal{R} Q => \forall P'. P \red P' \Rightarrow \exists Q'. Q \red Q', P' \mathcal{R} Q'$

$P \vdash x \Rightarrow Q \vdash x$

\begin{mathpar}
  \inferrule*[lab=Out-barb]{x \nameeq y}{{y}!\langle{Q}\rangle \vdash x}
  \and
  \inferrule*[lab=Par-barb]{\mbox{$P\vdash x$ or $Q\vdash x$}}{\binpar{P}{Q} \vdash x}
\end{mathpar}

\subsubsection{Contexts}

One of the principle advantages of computational calculi like the
$\pi$-calculus is a well-defined notion of context,
contextual-equivalence and a correlation between
contextual-equivalence and notions of bisimulation. The notion of
context allows the decomposition of a process into (sub-)process and
its syntactic environment, its context. Thus, a context may be
thought of as a process with a ``hole'' (written $\Box$) in it. The
application of a context $M$ to a process $P$, written $M[P]$, is
tantamount to filling the hole in $M$ with $P$. In this paper we do
not need the full weight of this theory, but do make use of the notion
of context in the proof the main theorem. 

\begin{mathpar}
  \inferrule* [lab=summation] {} {{M_{M},M_{N}} \bc \Box \;|\; x.M_{A} \;|\; M_{M}+M_{N}}
  \and
  \inferrule* [lab=agent] {} {{M_{A}} \bc (\vec{x})M_{P} \;| \; \clift{P_0,\ldots,M_{P},\ldots,P_N}}
  \and \\
  \inferrule* [lab=process] {} {{M_{P}} \bc M_{N} \;| \;P|M_{P} }
\end{mathpar} 

\begin{mathpar}
  \inferrule* [lab=sychronization] {} {M_{N} \bc \Box \;|\; x?M_{F} \;|\; x!M_{C}}
  \and
  \inferrule* [lab=abstraction] {} {{M_{F}} \bc (x)M_{P} }
  \and
  \inferrule* [lab=concretion] {} {{M_{C}} \bc \langle M_{P} \rangle }
  \and \\
  \inferrule* [lab=process] {} {{M_{P}} \bc M_{N} \;| \;P|M_{P} }
\end{mathpar}

\begin{definition}[contextual application] Given a context $M$, and
  process $P$, we define the \emph{contextual application}, $M[P] :=
  M\{P/\Box\}$. That is, the contextual application of M to P is the
  substitution of $P$ for $\Box$ in $M$.
\end{definition}

$\meaningof{-} : L \to \mathcal{P}(\pi)$

\begin{mathpar}
  \inferrule* [lab=collection] {} {\meaningof{true} = \pi, \and \meaningof{~E} = \pi \setminus \meaningof{E}, \and \meaningof{E_{1} \& E_{2}} = \meaningof{E_{1}} \cap \meaningof{E_{2}}}
\end{mathpar}

\begin{mathpar}
  \inferrule* [lab=structure] {} {\meaningof{0} = \{ P \in \pi | P \equiv 0 \}, \and \\ \meaningof{E_1 | E_2} = \{ P \in \pi | P \equiv P_{1} | P_{2}, P_{1} \in \meaningof{E_{1}}, P_{2} \in \meaningof{E_2}\} }
\end{mathpar}

\begin{mathpar}
 \inferrule* [lab=behavior] {} {\meaningof{\langle a?b \rangle E} = \{ P \in \pi | P \equiv Q | u?(y)P', \\ \and \\\\ \and \\ \;\;\; u \in \meaningof{a}, \forall z.P'\{z/y\} \in \meaningof{E\{z/b\}}\}, \and \\ \meaningof{a!E} = \{ P \in \pi | P \equiv Q | x!\langle P' \rangle, x \in \meaningof{a} P' \in \meaningof{E}\} }
\end{mathpar}

\begin{mathpar}
 \inferrule* [lab=nominal] {} {\meaningof{\quotep{E}} = \{ \quotep{P} \in \quotep{\pi} | P \in \meaningof{E} \}, \and \meaningof{\quotep{P}} = \{ \quotep{Q} \in \quotep{\pi} | P \equiv Q \} \and \\ \meaningof{@\quotep{E}} = \{ P \in \pi | P \equiv @x, x \in \meaningof{E} \}}
\end{mathpar}

\begin{eqnarray*}
  \\
  \meaningof{-} : TS \to ST
\end{eqnarray*}

\begin{eqnarray*}
  \\
  L : TS \to ST
\end{eqnarray*}

\begin{eqnarray*}
  \\
  P \models E \iff P \in \meaningof{E}
\end{eqnarray*}

\begin{eqnarray*}
  P \approx_{L} Q \iff \forall E \in L. P \models E \iff Q \models E
\end{eqnarray*}

\begin{eqnarray*}
  P \approx_{K} Q
\end{eqnarray*}

\begin{eqnarray*}
  P \approx Q
\end{eqnarray*}

$\approx_{K} = \approx = \approx_{L}$

\subsubsection{Contextual duality}

Note that contexts extend the quotation operation to a family of
operations from processes to names. Given a context, $M$, we can
define a \emph{nominal context}, $\quotep{M}$ by $\quotep{M}[P] :=
\quotep{M[P]}$. To foreshadow what is to come we observe that these
operations enjoy a duality with processes very much like the duality
between vectors and maps from vectors to scalars.

Further, because the calculus is essentially higher-order, we have a
correspondence between contexts and processes. More specifically,
given a name $x$ and a context $M$ we can construct $M^{*}_{x}$ such
that 

\begin{mathpar}
  M^{*}_{x} | \lift{x}{P} \red M[P]
\end{mathpar}

namely,

\begin{mathpar}
  M^{*}_{x} := x?(u).M[\dropn{u}]
\end{mathpar}

The dependence of $M^{*}_{x}$ on a name makes it an abstraction, 

\begin{mathpar}
  M^{*} := (x)x?(u).M[\dropn{u}]
\end{mathpar}

\subsection{Additional notation}

It will sometimes be convenient to denote the process a name
quotes. We already have the notation $x = \quotep{P}$, but it will be
convenient to introduce an alternate notation, $\procn{x}$, when we
want to emphasize the connection to the use of the name. Note that, by
virtue of name equivalence, $\quotep{\procn{x}} \nameeq x$; so, the
notation is consistent with previous definitions.

Further, because names have structure it is possible to effect
substitutions on the basis of that structure. This means we need to
upgrade our notation for substitutions, which we accomplish by
adapting comprehension notation. Thus,

\begin{mathpar}
  P\{ y / x : x \in S \}
\end{mathpar}

is interpreted to mean the process derived from P by replacing (in a
capture-avoiding manner) each occurrence of $x$ in $S$ by $y$. For example,

\begin{mathpar}
  P\{ \quotep{\procn{x}|\procn{x}} / x : x \in \freenames{P} \}
\end{mathpar}

will replace each (occurrence) of a free name $x$ in $P$ by
$\quotep{\procn{x}|\procn{x}}$.

Also, we will avail ourselves of the notation $x^{L}$ and $x^{R}$ to
denote injections of a name into disjoint copies of the name
space. There are numerous ways to accomplish this. One example can be
found in \cite{MeredithR05}. This notation overloads to vectors of
names: $\vec{x}^{\pi} := (x_{i}^{\pi} \; : \; 0 \leq i < |\vec{x}| )$ where $\pi \in \{L,R\}$.

We also use $P^{\Box} := P|\Box$.

In \cite{MeredithR05} an interpretation of the new operator is
given. It turns out that there are several possible interpretations
all enjoying the requisite algebraic properties of the operator (see
\cite{milner91polyadicpi}). We will therefore make liberal use of
$(\nu\; \vec{x})P$.

% subsection the_syntax_and_semantics_of_the_notation_system (end)   

\input{qm2pi.qmops} 

\input{qm2pi.sterngerlach} 

\input{qm2pi.metric} 

% section concurrent_process_calculi (end)

%\input{qm2pi.proofsketch}

% section proof sketch (end)

%\input{qm2pi.slviaknots} 

% section spatial logic via knots (end)

\input{qm2pi.conclusion}

% section conclusion (end)

%\input{qm2pi.dtcodes} 

% section wiring algorithm (end)

\input{qm2pi.ack} 

% section acknowledgments (end)

\newpage


\bibliographystyle{plain}   
\bibliography{../../biblios/main.bib}

\input{qm2pi.rhodetails}

\end{document}

 

\documentclass[12pt]{llncs}
%\documentclass{jktr}

\usepackage[pdftex]{hyperref}                   
\usepackage {listings}
\usepackage {mathpartir}
\usepackage{bcprules}
%\usepackage{listings}
                       
\usepackage{graphicx} 
%\usepackage[margins=2.5cm,nohead,nofoot]{geometry}
%\usepackage{geometry}
\usepackage{amsfonts}
\usepackage{amstext}
\usepackage{latexsym}
\usepackage{amssymb}
\usepackage{color}


%\include{myPreamble}
\include{qm2pi.local} 

%\ifpdf
%\usepackage[pdftex]{graphicx}
%\else
%\usepackage{graphicx}
%\fi

 % \ifpdf
%  \usepackage{pdfsync}
%  \if


%\title{Brief Article}
%\author{David F. Snyder}
%\author{L.G. Meredith}

%\address{Dept. of Math., Texas State University--San Marcos, San Marcos, TX 78666}
       
\pagestyle{empty}


\begin{document}

\lstset{language=[Objective]Caml,frame=shadowbox}

\input{qm2pi.front}

% section front matter (end)

\input{qm2pi.intro} 
 
% section introduction (end)

% \input{qm2pi.knotations} 

% section notation (end)

\input{qm2pi.process.calculi} 

% section concurrent_process_calculi_and_spatial_logics_ (end)
    
%\input{qm2pi.knots2pi} 

%\input{qm2pi.trefoil} 

%\input{qm2pi.mainthm} 

% subsection basic_interpretation (end)

%\input{qm2pi.rho.presentation} 
\subsection{The syntax and semantics of the notation system}\label{sub:the_syntax_and_semantics_of_the_notation_system} % (fold)

We now summarize a technical presentation of the calculus that
embodies our theory of dynamics. The typical presentation of such a
calculus follows the style of giving generators and relations on
them. The grammar, below, describing term constructors, freely
generates the set of processes, $\Proc$. This set is then quotiented
by a relation known as structural congruence and it is over this set
that the notion of dynamics is expressed. This presentation is
essentially that of \cite{MeredithR05} with the addition of
polyadicity and summation. For readability we have relegated some of
the technical subtleties to an appendix.

\subsubsection{Process grammar}\label{subsub:process_grammar}

\begin{mathpar}
  \inferrule* [lab=synchronization] {} {{M} \bc \pzero \;|\; x?F \;|\; x!C }
  \and
  \inferrule* [lab=abstraction] {} {{F} \bc (x)P}
  \and
  \inferrule* [lab=concretion] {} {{C} \bc \langle Q \rangle}
  \and
  \inferrule* [lab=process] {} {{P,Q} \bc M \;| \;P|Q \;|\; @{x}}
  \and
  \inferrule* [lab=name] {} {{x} \bc \quotep{P}}
\end{mathpar} 

Note that $\vec{x}$ (resp. $\vec{P}$) denotes a vector of names
(resp. processes) of length $|\vec{x}|$ (resp. $|\vec{P}|$). We adopt
the following useful abbreviations.

\begin{mathpar}
   x?(\vec{y}).P := x.(\vec{y})P \and  x\clift{\vec{P}} := x.\clift{\vec{P}}
   \and x!(y) := \lift{x}{\dropn{y}}
   \and \Pi_{i=0}^{n-1}P_i := P_0 | \ldots | P_{n-1}
\end{mathpar}

\subsubsection{Structural congruence}

\paragraph{Free and bound names and alpha-equivalence.} At the
core of structural equivalence is alpha-equivalence which identifies
process that are the same up to a change of variable. Formally, we
recognize the distinction between free and bound names. The free names
of a process, $\freenames{P}$, may be calculated recursively as
follows:

\begin{mathpar}
\freenames{\pzero} := \emptyset
  \and \\
  \freenames{x?(y).P} := \{ x \} \cup (\freenames{P} \setminus \{ y \})
  \and 
  \freenames{x!\langle P \rangle} := \{ x \} \cup \{ P \} 
  \and \\
  \freenames{P|Q} := \freenames{P} \cup \freenames{Q}
  \and \\
  \freenames{@{x}} := \{ x \}
\end{mathpar}

$\pi$
$\quotep{\pi}$

$\freenames{-} : \pi \to \mathcal{P}(\quotep{\pi})$

\begin{eqnarray*}
  \freenames{\pzero} & := & \emptyset \\
  \freenames{x?(y).P} & := & \{ x \} \cup (\freenames{P} \setminus \{ y \}) \\
  \freenames{x!\langle P \rangle} & := & \{ x \} \cup \{ P \} \\
  \freenames{P|Q} & := & \freenames{P} \cup \freenames{Q} \\
  \freenames{\dropn{x}} & := & \{ x \}
\end{eqnarray*}

The bound names of a process, $\boundnames{P}$, are those names occurring in $P$
that are not free. For example, in $x?(y).0$, the name $x$ is free, while $y$ is bound.

\begin{mathpar}
  \inferrule* [lab=monoidal-laws] {} { P|Q \equiv Q|P \and P|0 \equiv P \and P|(Q|R) \equiv (P|Q)|R }
\end{mathpar}

\begin{mathpar}
  \inferrule* [lab=alpha-equivalence] {} { (x)P \equiv (y)P\{y/x\} \and y \not\in \freenames{P} }
\end{mathpar}

\begin{definition}
Then two processes, $P,Q$, are alpha-equivalent if $P = Q\{\vec{y}/\vec{x}\}$ for
some $\vec{x} \in \boundnames{Q},\vec{y} \in \boundnames{P}$, where $Q\{\vec{y}/\vec{x}\}$
denotes the capture-avoiding substitution of $\vec{y}$ for $\vec{x}$ in $Q$.
\end{definition}

\begin{definition}
  The {\em structural congruence} \cite{SangiorgiWalker} , $\equiv$,
  between processes is the least congruence containing
  alpha-equivalence, satisfying the abelian monoid laws
  (associativity, commutativity and $\pzero$ as identity) for parallel
  composition $|$ and for summation $+$.
\end{definition}

\subsection{Name equivalence}

We take name equivalence, written $\nameeq$, to be the smallest
equivalence relation generated by the following rules.

\begin{mathpar}
\inferrule*[lab=Quote-drop]
{ }
{ \quotep{@{x}} \nameeq x }

\inferrule*[lab=Struct-equiv]
{ P \scong Q }
{ \quotep{P} \nameeq \quotep{Q} }
\end{mathpar}

The astute reader will have noticed that the mutual recursion of names
and processes imposes a mutual recursion on alpha-equivalence and
structural equivalence via name-equivalence. Fortunately, all of this
works out pleasantly and we may calculate in the natural way, free of
concern. The reader interested in the details is referred to the
appendix \ref{appendix:rho_details}.

\subsection{Substitution}

We use $\Proc$ for the set of processes, $\QProc$ for the set of
names, and $\id{\{}\vec{y} / \vec{x} \id{\}}$ to denote partial maps,
$s : \QProc \rightarrow \QProc$. A map, $s$ lifts, uniquely, to a map
on process terms, $\widehat{s} : \Proc \rightarrow \Proc$ by the
following equations.

\begin{mathpar}
  (0) \psubstp{Q}{P} := 0 \\
  (R \juxtap S) \psubstp{Q}{P}
  :=    
  (R)\psubstp{Q}{P} \juxtap (S) \psubstp{Q}{P} \\
  (x?(y).R) \psubstp{Q}{P}    
  :=    
  (x)\substp{Q}{P} (z)\concat( (R \psubstn{z}{y}) \psubstp{Q}{P} ) \\
  (\lift{x}{R}) \psubstp{Q}{P}  
  :=
  \lift{(x)\substp{Q}{P}}{ R \psubstp{Q}{P} } \\
%   (\dropn{x})  \psubstp{Q}{P}       
%   := 
%   \left\{ 
%     \begin{array}{ccc} 
%       \dropn{\quotep{Q}} & & x \nameeq \quotep{P} \\
%       \dropn{x} & & otherwise \\
%     \end{array}
%   \right. 
  (\dropn{x})  \psubstp{Q}{P}       
  := 
  \left\{ 
    \begin{array}{ccc} 
      Q & & x \nameeq \quotep{P} \\
      \dropn{x} & & otherwise \\
    \end{array}
  \right.
\end{mathpar}
 

where

\begin{eqnarray}
  (x)\id{\{} \lpquote Q \rpquote / \lpquote P \rpquote \id{\}}            = 
  \left\{ 
    \begin{array}{ccc}
      \lpquote Q \rpquote & & x \nameeq \lpquote P \rpquote \\
      x & & otherwise \\
    \end{array}
  \right. \nonumber
\end{eqnarray}

and $z$ is chosen distinct from $\quotep{P}$, $\quotep{Q}$, the free
names in $Q$, and all the names in $R$. Our $\alpha$-equivalence will
be built in the standard way from this substitution.

\begin{remark}\label{rem:no_self_referential_names}
  One consequence of these definitions is that $\forall P. \quotep{P}
  \not\in \freenames{P}$.
\end{remark}

\subsection{ Dynamic quote: an example }

Anticipating something of what's to come, consider applying the
substitution, $\widehat{\id{\{}u / z \id{\}}}$, to the following pair
of processes, $\lift{w}{y!(z)}$ and $w[ \lpquote y!(z) \rpquote ]$.

\begin{eqnarray}
	\lift{w}{y!(z)}\widehat{\id{\{}u / z \id{\}}}
		& = &
		\lift{w}{y!(u)} \nonumber\\
	w[ \lpquote y!(z) \rpquote ] \widehat{ \id{\{}u / z \id{\}} }
		& = &
		w[ \lpquote y!(z) \rpquote ] \nonumber
\end{eqnarray}

Because the body of the process between quotes is impervious to
substitution, we get radically different answers. In fact, by
examining the first process in an input context,
e.g. $x?(z).\lift{w}{y!(z)}$, we see that the process under the lift
operator may be shaped by prefixed inputs binding a name inside it. In
this sense, the lift operator will be seen as a way to dynamically
construct processes before reifying them as names.

Finally equipped with these standard features we can present the
dynamics of the calculus.

\subsubsection{Operational semantics} 

Finally, we introduce the computational dynamics. What marks these
algebras as distinct from other more traditionally studied algebraic
structures, e.g. vector spaces or polynomial rings, is the manner in
which dynamics is captured. In traditional structures, dynamics is typically
expressed through morphisms between such structures, as in linear maps
between vector spaces or morphisms between rings. In algebras
associated with the semantics of computation, the dynamics is
expressed as part of the algebraic structure itself, through a
reduction reduction relation typically denoted by $\red$. Below, we
give a recursive presentation of this relation for the calculus used
in the encoding.

$\red \subseteq \pi \times \pi$
$\red : \pi \to \mathcal{P}(\pi)$

\begin{mathpar}
  \inferrule* [lab=Comm] { \textsf{match}( x_{src}, x_{trgt} ) } { x_{trgt}?(y)P \; | \; x_{src}!\langle {Q} \rangle \red P\{\quotep{Q}/y}\} }
  \and \\
  \inferrule* [lab=Par] {{P} \red {P}'} {{{P} | {Q}} \red {{P}' | {Q}}}
  \and
  \inferrule* [lab=Equiv]{{{P} \scong {P}'} \andalso {{P}' \red {Q}'} \andalso {{Q}' \scong {Q}}}{{P} \red {Q}}
\end{mathpar}

\begin{eqnarray*}
  match_{\equiv} (\quotep{P},\quotep{Q}) & := & P \equiv Q \\
  match_{\dagger}(\quotep{P},\quotep{Q}) & := & \forall R. P|Q \red^{*} R => R \red^{*} 0 \\
  match_{K}(\quotep{P},\quotep{Q}) & := & K \mbox{ for some context } K
\end{eqnarray*}

$u?(x)P | u!\langle Q \rangle \red P\{\quotep{Q}/x\}$

%We write $\wred$ for $\red^*$, and $P\red$ if $\exists Q $ such that $ P \red Q$.
We write $P\red$ if $\exists Q $ such that $ P \red Q$ and $P\not\red$, otherwise.

\section{Replication}

As mentioned before, it is known that replication (and hence
recursion) can be implemented in a higher-order process algebra
\cite{SangiorgiWalker}. As our first example of calculation with the
machinery thus far presented we give the construction explicitly in
the {\rhoc}.

\begin{eqnarray}
	D_{x} & := & \prefix{x}{y}{(\binpar{\outputp{x}{y}}{@{y}})} \nonumber\\
	\bangp_{x}{P} & := & \binpar{{x}!\langle{\binpar{D_{x}}{P}}\rangle}{D_{x}} \nonumber
\end{eqnarray}

\begin{eqnarray}
	\bangp_{x}{P} & & \nonumber\\
	=
	& {x}!\langle{(\prefix{x}{y}{(\outputp{x}{y} | @{y})) | P}}\rangle 
	      | \prefix{x}{y}{(\outputp{x}{y} | @{y})} & \nonumber\\
	\red
	& (\outputp{x}{y} | @{y})\substn{\quotep{(\prefix{x}{y}{(@{y} | \outputp{x}{y})) | P}}}{y} & \nonumber\\
	=
	& \outputp{x}{\quotep{(\prefix{x}{y}{(\outputp{x}{y} | @{y})) | P}}}
	  | {(\prefix{x}{y}{(\outputp{x}{y} | @{y})) | P}} & \nonumber\\
	\red
	& \ldots & \nonumber\\
	\red^*
	& P | P | \ldots & \nonumber
\end{eqnarray}

Of course, this encoding, as an implementation, runs away, unfolding
$\bangp{P}$ eagerly. A lazier and more implementable replication
operator, restricted to input-guarded processes, may be obtained as follows.

\begin{eqnarray}
\bangp{\prefix{u}{v}{P}} 
	:= 
	\binpar{\lift{x}{\prefix{u}{v}{(\binpar{D(x)}{P})}}}{D(x)} \nonumber
\end{eqnarray}

\begin{remark}
  Note that the lazier definition still does not deal with summation
  or mixed summation (i.e. sums over input and output). The reader is
  invited to construct definitions of replication that deal with these
  features. 

  Further, the definitions are parameterized in a name, $x$. Can you,
  gentle reader, make a definition that eliminates this parameter and
  guarantees no accidental interaction between the replication
  machinery and the process being replicated -- i.e. no accidental
  sharing of names used by the process to get its work done and the
  name(s) used by the replication to effect copying. This latter
  revision of the definition of replication is crucial to obtaining
  the expected identity $!!P \sim !P$.
\end{remark}

\begin{remark}\label{rem:paradoxical_combinator}
  The reader familiar with the lambda calculus will have noticed the
  similarity between $D$ and the paradoxical combinator.

  [Ed. note: the existence of this seems to suggest we have to be more
  restrictive on the set of processes and names we admit if we are to
  support no-cloning.]
\end{remark}

\subsubsection{Bisimulation}

The computational dynamics gives rise to another kind of equivalence,
the equivalence of computational behavior. As previously mentioned
this is typically captured \emph{via} some form of bisimulation.

% The notion we use in this paper is weak barbed bisimulation
% \cite{milner91polyadicpi}.

The notion we use in this paper is derived from weak barbed
bisimulation \cite{milner91polyadicpi}. 

\begin{definition}
An \emph{observation relation}, $\downarrow_{\mathcal N}$, over a set
of names, $\mathcal N$, is the smallest relation satisfying the rules
below.

\infrule[Out-barb]{y \in {\mathcal N}, \; x \nameeq y}
		  {\outputp{x}{v} \downarrow_{\mathcal N} x}
\infrule[Par-barb]{\mbox{$P\downarrow_{\mathcal N} x$ or $Q\downarrow_{\mathcal N} x$}}
		  {\binpar{P}{Q} \downarrow_{\mathcal N} x}

We write $P \Downarrow_{\mathcal N} x$ if there is $Q$ such that 
$P \wred Q$ and $Q \downarrow_{\mathcal N} x$.
\end{definition}

\begin{definition}
%\label{def.bbisim}
An  ${\mathcal N}$-\emph{barbed bisimulation} over a set of names, ${\mathcal N}$, is a symmetric binary relation 
${\mathcal S}_{\mathcal N}$ between agents such that $P\rel{S}_{\mathcal N}Q$ implies:
\begin{enumerate}
\item If $P \red P'$ then $Q \wred Q'$ and $P'\rel{S}_{\mathcal N} Q'$.
\item If $P\downarrow_{\mathcal N} x$, then $Q\Downarrow_{\mathcal N} x$.
\end{enumerate}
$P$ is ${\mathcal N}$-barbed bisimilar to $Q$, written
$P \wbbisim_{\mathcal N} Q$, if $P \rel{S}_{\mathcal N} Q$ for some ${\mathcal N}$-barbed bisimulation ${\mathcal S}_{\mathcal N}$.
\end{definition}

$\mathcal{R} \subseteq \pi \times \pi$

$P \mathcal{R} Q => \forall P'. P \red P' \Rightarrow \exists Q'. Q \red Q', P' \mathcal{R} Q'$

$P \vdash x \Rightarrow Q \vdash x$

\begin{mathpar}
  \inferrule*[lab=Out-barb]{x \nameeq y}{{y}!\langle{Q}\rangle \vdash x}
  \and
  \inferrule*[lab=Par-barb]{\mbox{$P\vdash x$ or $Q\vdash x$}}{\binpar{P}{Q} \vdash x}
\end{mathpar}

\subsubsection{Contexts}

One of the principle advantages of computational calculi like the
$\pi$-calculus is a well-defined notion of context,
contextual-equivalence and a correlation between
contextual-equivalence and notions of bisimulation. The notion of
context allows the decomposition of a process into (sub-)process and
its syntactic environment, its context. Thus, a context may be
thought of as a process with a ``hole'' (written $\Box$) in it. The
application of a context $M$ to a process $P$, written $M[P]$, is
tantamount to filling the hole in $M$ with $P$. In this paper we do
not need the full weight of this theory, but do make use of the notion
of context in the proof the main theorem. 

\begin{mathpar}
  \inferrule* [lab=summation] {} {{M_{M},M_{N}} \bc \Box \;|\; x.M_{A} \;|\; M_{M}+M_{N}}
  \and
  \inferrule* [lab=agent] {} {{M_{A}} \bc (\vec{x})M_{P} \;| \; \clift{P_0,\ldots,M_{P},\ldots,P_N}}
  \and \\
  \inferrule* [lab=process] {} {{M_{P}} \bc M_{N} \;| \;P|M_{P} }
\end{mathpar} 

\begin{mathpar}
  \inferrule* [lab=sychronization] {} {M_{N} \bc \Box \;|\; x?M_{F} \;|\; x!M_{C}}
  \and
  \inferrule* [lab=abstraction] {} {{M_{F}} \bc (x)M_{P} }
  \and
  \inferrule* [lab=concretion] {} {{M_{C}} \bc \langle M_{P} \rangle }
  \and \\
  \inferrule* [lab=process] {} {{M_{P}} \bc M_{N} \;| \;P|M_{P} }
\end{mathpar}

\begin{definition}[contextual application] Given a context $M$, and
  process $P$, we define the \emph{contextual application}, $M[P] :=
  M\{P/\Box\}$. That is, the contextual application of M to P is the
  substitution of $P$ for $\Box$ in $M$.
\end{definition}

$\meaningof{-} : L \to \mathcal{P}(\pi)$

\begin{mathpar}
  \inferrule* [lab=collection] {} {\meaningof{true} = \pi, \and \meaningof{~E} = \pi \setminus \meaningof{E}, \and \meaningof{E_{1} \& E_{2}} = \meaningof{E_{1}} \cap \meaningof{E_{2}}}
\end{mathpar}

\begin{mathpar}
  \inferrule* [lab=structure] {} {\meaningof{0} = \{ P \in \pi | P \equiv 0 \}, \and \\ \meaningof{E_1 | E_2} = \{ P \in \pi | P \equiv P_{1} | P_{2}, P_{1} \in \meaningof{E_{1}}, P_{2} \in \meaningof{E_2}\} }
\end{mathpar}

\begin{mathpar}
 \inferrule* [lab=behavior] {} {\meaningof{\langle a?b \rangle E} = \{ P \in \pi | P \equiv Q | u?(y)P', \\ \and \\\\ \and \\ \;\;\; u \in \meaningof{a}, \forall z.P'\{z/y\} \in \meaningof{E\{z/b\}}\}, \and \\ \meaningof{a!E} = \{ P \in \pi | P \equiv Q | x!\langle P' \rangle, x \in \meaningof{a} P' \in \meaningof{E}\} }
\end{mathpar}

\begin{mathpar}
 \inferrule* [lab=nominal] {} {\meaningof{\quotep{E}} = \{ \quotep{P} \in \quotep{\pi} | P \in \meaningof{E} \}, \and \meaningof{\quotep{P}} = \{ \quotep{Q} \in \quotep{\pi} | P \equiv Q \} \and \\ \meaningof{@\quotep{E}} = \{ P \in \pi | P \equiv @x, x \in \meaningof{E} \}}
\end{mathpar}

\begin{eqnarray*}
  \\
  \meaningof{-} : TS \to ST
\end{eqnarray*}

\begin{eqnarray*}
  \\
  L : TS \to ST
\end{eqnarray*}

\begin{eqnarray*}
  \\
  P \models E \iff P \in \meaningof{E}
\end{eqnarray*}

\begin{eqnarray*}
  P \approx_{L} Q \iff \forall E \in L. P \models E \iff Q \models E
\end{eqnarray*}

\begin{eqnarray*}
  P \approx_{K} Q
\end{eqnarray*}

\begin{eqnarray*}
  P \approx Q
\end{eqnarray*}

$\approx_{K} = \approx = \approx_{L}$

\subsubsection{Contextual duality}

Note that contexts extend the quotation operation to a family of
operations from processes to names. Given a context, $M$, we can
define a \emph{nominal context}, $\quotep{M}$ by $\quotep{M}[P] :=
\quotep{M[P]}$. To foreshadow what is to come we observe that these
operations enjoy a duality with processes very much like the duality
between vectors and maps from vectors to scalars.

Further, because the calculus is essentially higher-order, we have a
correspondence between contexts and processes. More specifically,
given a name $x$ and a context $M$ we can construct $M^{*}_{x}$ such
that 

\begin{mathpar}
  M^{*}_{x} | \lift{x}{P} \red M[P]
\end{mathpar}

namely,

\begin{mathpar}
  M^{*}_{x} := x?(u).M[\dropn{u}]
\end{mathpar}

The dependence of $M^{*}_{x}$ on a name makes it an abstraction, 

\begin{mathpar}
  M^{*} := (x)x?(u).M[\dropn{u}]
\end{mathpar}

\subsection{Additional notation}

It will sometimes be convenient to denote the process a name
quotes. We already have the notation $x = \quotep{P}$, but it will be
convenient to introduce an alternate notation, $\procn{x}$, when we
want to emphasize the connection to the use of the name. Note that, by
virtue of name equivalence, $\quotep{\procn{x}} \nameeq x$; so, the
notation is consistent with previous definitions.

Further, because names have structure it is possible to effect
substitutions on the basis of that structure. This means we need to
upgrade our notation for substitutions, which we accomplish by
adapting comprehension notation. Thus,

\begin{mathpar}
  P\{ y / x : x \in S \}
\end{mathpar}

is interpreted to mean the process derived from P by replacing (in a
capture-avoiding manner) each occurrence of $x$ in $S$ by $y$. For example,

\begin{mathpar}
  P\{ \quotep{\procn{x}|\procn{x}} / x : x \in \freenames{P} \}
\end{mathpar}

will replace each (occurrence) of a free name $x$ in $P$ by
$\quotep{\procn{x}|\procn{x}}$.

Also, we will avail ourselves of the notation $x^{L}$ and $x^{R}$ to
denote injections of a name into disjoint copies of the name
space. There are numerous ways to accomplish this. One example can be
found in \cite{MeredithR05}. This notation overloads to vectors of
names: $\vec{x}^{\pi} := (x_{i}^{\pi} \; : \; 0 \leq i < |\vec{x}| )$ where $\pi \in \{L,R\}$.

We also use $P^{\Box} := P|\Box$.

In \cite{MeredithR05} an interpretation of the new operator is
given. It turns out that there are several possible interpretations
all enjoying the requisite algebraic properties of the operator (see
\cite{milner91polyadicpi}). We will therefore make liberal use of
$(\nu\; \vec{x})P$.

% subsection the_syntax_and_semantics_of_the_notation_system (end)   

\input{qm2pi.qmops} 

\input{qm2pi.sterngerlach} 

\input{qm2pi.metric} 

% section concurrent_process_calculi (end)

%\input{qm2pi.proofsketch}

% section proof sketch (end)

%\input{qm2pi.slviaknots} 

% section spatial logic via knots (end)

\input{qm2pi.conclusion}

% section conclusion (end)

%\input{qm2pi.dtcodes} 

% section wiring algorithm (end)

\input{qm2pi.ack} 

% section acknowledgments (end)

\newpage


\bibliographystyle{plain}   
\bibliography{../../biblios/main.bib}

\input{qm2pi.rhodetails}

\end{document}

 

% section concurrent_process_calculi (end)

%\documentclass[12pt]{llncs}
%\documentclass{jktr}

\usepackage[pdftex]{hyperref}                   
\usepackage {listings}
\usepackage {mathpartir}
\usepackage{bcprules}
%\usepackage{listings}
                       
\usepackage{graphicx} 
%\usepackage[margins=2.5cm,nohead,nofoot]{geometry}
%\usepackage{geometry}
\usepackage{amsfonts}
\usepackage{amstext}
\usepackage{latexsym}
\usepackage{amssymb}
\usepackage{color}


%\include{myPreamble}
\include{qm2pi.local} 

%\ifpdf
%\usepackage[pdftex]{graphicx}
%\else
%\usepackage{graphicx}
%\fi

 % \ifpdf
%  \usepackage{pdfsync}
%  \if


%\title{Brief Article}
%\author{David F. Snyder}
%\author{L.G. Meredith}

%\address{Dept. of Math., Texas State University--San Marcos, San Marcos, TX 78666}
       
\pagestyle{empty}


\begin{document}

\lstset{language=[Objective]Caml,frame=shadowbox}

\input{qm2pi.front}

% section front matter (end)

\input{qm2pi.intro} 
 
% section introduction (end)

% \input{qm2pi.knotations} 

% section notation (end)

\input{qm2pi.process.calculi} 

% section concurrent_process_calculi_and_spatial_logics_ (end)
    
%\input{qm2pi.knots2pi} 

%\input{qm2pi.trefoil} 

%\input{qm2pi.mainthm} 

% subsection basic_interpretation (end)

%\input{qm2pi.rho.presentation} 
\subsection{The syntax and semantics of the notation system}\label{sub:the_syntax_and_semantics_of_the_notation_system} % (fold)

We now summarize a technical presentation of the calculus that
embodies our theory of dynamics. The typical presentation of such a
calculus follows the style of giving generators and relations on
them. The grammar, below, describing term constructors, freely
generates the set of processes, $\Proc$. This set is then quotiented
by a relation known as structural congruence and it is over this set
that the notion of dynamics is expressed. This presentation is
essentially that of \cite{MeredithR05} with the addition of
polyadicity and summation. For readability we have relegated some of
the technical subtleties to an appendix.

\subsubsection{Process grammar}\label{subsub:process_grammar}

\begin{mathpar}
  \inferrule* [lab=synchronization] {} {{M} \bc \pzero \;|\; x?F \;|\; x!C }
  \and
  \inferrule* [lab=abstraction] {} {{F} \bc (x)P}
  \and
  \inferrule* [lab=concretion] {} {{C} \bc \langle Q \rangle}
  \and
  \inferrule* [lab=process] {} {{P,Q} \bc M \;| \;P|Q \;|\; @{x}}
  \and
  \inferrule* [lab=name] {} {{x} \bc \quotep{P}}
\end{mathpar} 

Note that $\vec{x}$ (resp. $\vec{P}$) denotes a vector of names
(resp. processes) of length $|\vec{x}|$ (resp. $|\vec{P}|$). We adopt
the following useful abbreviations.

\begin{mathpar}
   x?(\vec{y}).P := x.(\vec{y})P \and  x\clift{\vec{P}} := x.\clift{\vec{P}}
   \and x!(y) := \lift{x}{\dropn{y}}
   \and \Pi_{i=0}^{n-1}P_i := P_0 | \ldots | P_{n-1}
\end{mathpar}

\subsubsection{Structural congruence}

\paragraph{Free and bound names and alpha-equivalence.} At the
core of structural equivalence is alpha-equivalence which identifies
process that are the same up to a change of variable. Formally, we
recognize the distinction between free and bound names. The free names
of a process, $\freenames{P}$, may be calculated recursively as
follows:

\begin{mathpar}
\freenames{\pzero} := \emptyset
  \and \\
  \freenames{x?(y).P} := \{ x \} \cup (\freenames{P} \setminus \{ y \})
  \and 
  \freenames{x!\langle P \rangle} := \{ x \} \cup \{ P \} 
  \and \\
  \freenames{P|Q} := \freenames{P} \cup \freenames{Q}
  \and \\
  \freenames{@{x}} := \{ x \}
\end{mathpar}

$\pi$
$\quotep{\pi}$

$\freenames{-} : \pi \to \mathcal{P}(\quotep{\pi})$

\begin{eqnarray*}
  \freenames{\pzero} & := & \emptyset \\
  \freenames{x?(y).P} & := & \{ x \} \cup (\freenames{P} \setminus \{ y \}) \\
  \freenames{x!\langle P \rangle} & := & \{ x \} \cup \{ P \} \\
  \freenames{P|Q} & := & \freenames{P} \cup \freenames{Q} \\
  \freenames{\dropn{x}} & := & \{ x \}
\end{eqnarray*}

The bound names of a process, $\boundnames{P}$, are those names occurring in $P$
that are not free. For example, in $x?(y).0$, the name $x$ is free, while $y$ is bound.

\begin{mathpar}
  \inferrule* [lab=monoidal-laws] {} { P|Q \equiv Q|P \and P|0 \equiv P \and P|(Q|R) \equiv (P|Q)|R }
\end{mathpar}

\begin{mathpar}
  \inferrule* [lab=alpha-equivalence] {} { (x)P \equiv (y)P\{y/x\} \and y \not\in \freenames{P} }
\end{mathpar}

\begin{definition}
Then two processes, $P,Q$, are alpha-equivalent if $P = Q\{\vec{y}/\vec{x}\}$ for
some $\vec{x} \in \boundnames{Q},\vec{y} \in \boundnames{P}$, where $Q\{\vec{y}/\vec{x}\}$
denotes the capture-avoiding substitution of $\vec{y}$ for $\vec{x}$ in $Q$.
\end{definition}

\begin{definition}
  The {\em structural congruence} \cite{SangiorgiWalker} , $\equiv$,
  between processes is the least congruence containing
  alpha-equivalence, satisfying the abelian monoid laws
  (associativity, commutativity and $\pzero$ as identity) for parallel
  composition $|$ and for summation $+$.
\end{definition}

\subsection{Name equivalence}

We take name equivalence, written $\nameeq$, to be the smallest
equivalence relation generated by the following rules.

\begin{mathpar}
\inferrule*[lab=Quote-drop]
{ }
{ \quotep{@{x}} \nameeq x }

\inferrule*[lab=Struct-equiv]
{ P \scong Q }
{ \quotep{P} \nameeq \quotep{Q} }
\end{mathpar}

The astute reader will have noticed that the mutual recursion of names
and processes imposes a mutual recursion on alpha-equivalence and
structural equivalence via name-equivalence. Fortunately, all of this
works out pleasantly and we may calculate in the natural way, free of
concern. The reader interested in the details is referred to the
appendix \ref{appendix:rho_details}.

\subsection{Substitution}

We use $\Proc$ for the set of processes, $\QProc$ for the set of
names, and $\id{\{}\vec{y} / \vec{x} \id{\}}$ to denote partial maps,
$s : \QProc \rightarrow \QProc$. A map, $s$ lifts, uniquely, to a map
on process terms, $\widehat{s} : \Proc \rightarrow \Proc$ by the
following equations.

\begin{mathpar}
  (0) \psubstp{Q}{P} := 0 \\
  (R \juxtap S) \psubstp{Q}{P}
  :=    
  (R)\psubstp{Q}{P} \juxtap (S) \psubstp{Q}{P} \\
  (x?(y).R) \psubstp{Q}{P}    
  :=    
  (x)\substp{Q}{P} (z)\concat( (R \psubstn{z}{y}) \psubstp{Q}{P} ) \\
  (\lift{x}{R}) \psubstp{Q}{P}  
  :=
  \lift{(x)\substp{Q}{P}}{ R \psubstp{Q}{P} } \\
%   (\dropn{x})  \psubstp{Q}{P}       
%   := 
%   \left\{ 
%     \begin{array}{ccc} 
%       \dropn{\quotep{Q}} & & x \nameeq \quotep{P} \\
%       \dropn{x} & & otherwise \\
%     \end{array}
%   \right. 
  (\dropn{x})  \psubstp{Q}{P}       
  := 
  \left\{ 
    \begin{array}{ccc} 
      Q & & x \nameeq \quotep{P} \\
      \dropn{x} & & otherwise \\
    \end{array}
  \right.
\end{mathpar}
 

where

\begin{eqnarray}
  (x)\id{\{} \lpquote Q \rpquote / \lpquote P \rpquote \id{\}}            = 
  \left\{ 
    \begin{array}{ccc}
      \lpquote Q \rpquote & & x \nameeq \lpquote P \rpquote \\
      x & & otherwise \\
    \end{array}
  \right. \nonumber
\end{eqnarray}

and $z$ is chosen distinct from $\quotep{P}$, $\quotep{Q}$, the free
names in $Q$, and all the names in $R$. Our $\alpha$-equivalence will
be built in the standard way from this substitution.

\begin{remark}\label{rem:no_self_referential_names}
  One consequence of these definitions is that $\forall P. \quotep{P}
  \not\in \freenames{P}$.
\end{remark}

\subsection{ Dynamic quote: an example }

Anticipating something of what's to come, consider applying the
substitution, $\widehat{\id{\{}u / z \id{\}}}$, to the following pair
of processes, $\lift{w}{y!(z)}$ and $w[ \lpquote y!(z) \rpquote ]$.

\begin{eqnarray}
	\lift{w}{y!(z)}\widehat{\id{\{}u / z \id{\}}}
		& = &
		\lift{w}{y!(u)} \nonumber\\
	w[ \lpquote y!(z) \rpquote ] \widehat{ \id{\{}u / z \id{\}} }
		& = &
		w[ \lpquote y!(z) \rpquote ] \nonumber
\end{eqnarray}

Because the body of the process between quotes is impervious to
substitution, we get radically different answers. In fact, by
examining the first process in an input context,
e.g. $x?(z).\lift{w}{y!(z)}$, we see that the process under the lift
operator may be shaped by prefixed inputs binding a name inside it. In
this sense, the lift operator will be seen as a way to dynamically
construct processes before reifying them as names.

Finally equipped with these standard features we can present the
dynamics of the calculus.

\subsubsection{Operational semantics} 

Finally, we introduce the computational dynamics. What marks these
algebras as distinct from other more traditionally studied algebraic
structures, e.g. vector spaces or polynomial rings, is the manner in
which dynamics is captured. In traditional structures, dynamics is typically
expressed through morphisms between such structures, as in linear maps
between vector spaces or morphisms between rings. In algebras
associated with the semantics of computation, the dynamics is
expressed as part of the algebraic structure itself, through a
reduction reduction relation typically denoted by $\red$. Below, we
give a recursive presentation of this relation for the calculus used
in the encoding.

$\red \subseteq \pi \times \pi$
$\red : \pi \to \mathcal{P}(\pi)$

\begin{mathpar}
  \inferrule* [lab=Comm] { \textsf{match}( x_{src}, x_{trgt} ) } { x_{trgt}?(y)P \; | \; x_{src}!\langle {Q} \rangle \red P\{\quotep{Q}/y}\} }
  \and \\
  \inferrule* [lab=Par] {{P} \red {P}'} {{{P} | {Q}} \red {{P}' | {Q}}}
  \and
  \inferrule* [lab=Equiv]{{{P} \scong {P}'} \andalso {{P}' \red {Q}'} \andalso {{Q}' \scong {Q}}}{{P} \red {Q}}
\end{mathpar}

\begin{eqnarray*}
  match_{\equiv} (\quotep{P},\quotep{Q}) & := & P \equiv Q \\
  match_{\dagger}(\quotep{P},\quotep{Q}) & := & \forall R. P|Q \red^{*} R => R \red^{*} 0 \\
  match_{K}(\quotep{P},\quotep{Q}) & := & K \mbox{ for some context } K
\end{eqnarray*}

$u?(x)P | u!\langle Q \rangle \red P\{\quotep{Q}/x\}$

%We write $\wred$ for $\red^*$, and $P\red$ if $\exists Q $ such that $ P \red Q$.
We write $P\red$ if $\exists Q $ such that $ P \red Q$ and $P\not\red$, otherwise.

\section{Replication}

As mentioned before, it is known that replication (and hence
recursion) can be implemented in a higher-order process algebra
\cite{SangiorgiWalker}. As our first example of calculation with the
machinery thus far presented we give the construction explicitly in
the {\rhoc}.

\begin{eqnarray}
	D_{x} & := & \prefix{x}{y}{(\binpar{\outputp{x}{y}}{@{y}})} \nonumber\\
	\bangp_{x}{P} & := & \binpar{{x}!\langle{\binpar{D_{x}}{P}}\rangle}{D_{x}} \nonumber
\end{eqnarray}

\begin{eqnarray}
	\bangp_{x}{P} & & \nonumber\\
	=
	& {x}!\langle{(\prefix{x}{y}{(\outputp{x}{y} | @{y})) | P}}\rangle 
	      | \prefix{x}{y}{(\outputp{x}{y} | @{y})} & \nonumber\\
	\red
	& (\outputp{x}{y} | @{y})\substn{\quotep{(\prefix{x}{y}{(@{y} | \outputp{x}{y})) | P}}}{y} & \nonumber\\
	=
	& \outputp{x}{\quotep{(\prefix{x}{y}{(\outputp{x}{y} | @{y})) | P}}}
	  | {(\prefix{x}{y}{(\outputp{x}{y} | @{y})) | P}} & \nonumber\\
	\red
	& \ldots & \nonumber\\
	\red^*
	& P | P | \ldots & \nonumber
\end{eqnarray}

Of course, this encoding, as an implementation, runs away, unfolding
$\bangp{P}$ eagerly. A lazier and more implementable replication
operator, restricted to input-guarded processes, may be obtained as follows.

\begin{eqnarray}
\bangp{\prefix{u}{v}{P}} 
	:= 
	\binpar{\lift{x}{\prefix{u}{v}{(\binpar{D(x)}{P})}}}{D(x)} \nonumber
\end{eqnarray}

\begin{remark}
  Note that the lazier definition still does not deal with summation
  or mixed summation (i.e. sums over input and output). The reader is
  invited to construct definitions of replication that deal with these
  features. 

  Further, the definitions are parameterized in a name, $x$. Can you,
  gentle reader, make a definition that eliminates this parameter and
  guarantees no accidental interaction between the replication
  machinery and the process being replicated -- i.e. no accidental
  sharing of names used by the process to get its work done and the
  name(s) used by the replication to effect copying. This latter
  revision of the definition of replication is crucial to obtaining
  the expected identity $!!P \sim !P$.
\end{remark}

\begin{remark}\label{rem:paradoxical_combinator}
  The reader familiar with the lambda calculus will have noticed the
  similarity between $D$ and the paradoxical combinator.

  [Ed. note: the existence of this seems to suggest we have to be more
  restrictive on the set of processes and names we admit if we are to
  support no-cloning.]
\end{remark}

\subsubsection{Bisimulation}

The computational dynamics gives rise to another kind of equivalence,
the equivalence of computational behavior. As previously mentioned
this is typically captured \emph{via} some form of bisimulation.

% The notion we use in this paper is weak barbed bisimulation
% \cite{milner91polyadicpi}.

The notion we use in this paper is derived from weak barbed
bisimulation \cite{milner91polyadicpi}. 

\begin{definition}
An \emph{observation relation}, $\downarrow_{\mathcal N}$, over a set
of names, $\mathcal N$, is the smallest relation satisfying the rules
below.

\infrule[Out-barb]{y \in {\mathcal N}, \; x \nameeq y}
		  {\outputp{x}{v} \downarrow_{\mathcal N} x}
\infrule[Par-barb]{\mbox{$P\downarrow_{\mathcal N} x$ or $Q\downarrow_{\mathcal N} x$}}
		  {\binpar{P}{Q} \downarrow_{\mathcal N} x}

We write $P \Downarrow_{\mathcal N} x$ if there is $Q$ such that 
$P \wred Q$ and $Q \downarrow_{\mathcal N} x$.
\end{definition}

\begin{definition}
%\label{def.bbisim}
An  ${\mathcal N}$-\emph{barbed bisimulation} over a set of names, ${\mathcal N}$, is a symmetric binary relation 
${\mathcal S}_{\mathcal N}$ between agents such that $P\rel{S}_{\mathcal N}Q$ implies:
\begin{enumerate}
\item If $P \red P'$ then $Q \wred Q'$ and $P'\rel{S}_{\mathcal N} Q'$.
\item If $P\downarrow_{\mathcal N} x$, then $Q\Downarrow_{\mathcal N} x$.
\end{enumerate}
$P$ is ${\mathcal N}$-barbed bisimilar to $Q$, written
$P \wbbisim_{\mathcal N} Q$, if $P \rel{S}_{\mathcal N} Q$ for some ${\mathcal N}$-barbed bisimulation ${\mathcal S}_{\mathcal N}$.
\end{definition}

$\mathcal{R} \subseteq \pi \times \pi$

$P \mathcal{R} Q => \forall P'. P \red P' \Rightarrow \exists Q'. Q \red Q', P' \mathcal{R} Q'$

$P \vdash x \Rightarrow Q \vdash x$

\begin{mathpar}
  \inferrule*[lab=Out-barb]{x \nameeq y}{{y}!\langle{Q}\rangle \vdash x}
  \and
  \inferrule*[lab=Par-barb]{\mbox{$P\vdash x$ or $Q\vdash x$}}{\binpar{P}{Q} \vdash x}
\end{mathpar}

\subsubsection{Contexts}

One of the principle advantages of computational calculi like the
$\pi$-calculus is a well-defined notion of context,
contextual-equivalence and a correlation between
contextual-equivalence and notions of bisimulation. The notion of
context allows the decomposition of a process into (sub-)process and
its syntactic environment, its context. Thus, a context may be
thought of as a process with a ``hole'' (written $\Box$) in it. The
application of a context $M$ to a process $P$, written $M[P]$, is
tantamount to filling the hole in $M$ with $P$. In this paper we do
not need the full weight of this theory, but do make use of the notion
of context in the proof the main theorem. 

\begin{mathpar}
  \inferrule* [lab=summation] {} {{M_{M},M_{N}} \bc \Box \;|\; x.M_{A} \;|\; M_{M}+M_{N}}
  \and
  \inferrule* [lab=agent] {} {{M_{A}} \bc (\vec{x})M_{P} \;| \; \clift{P_0,\ldots,M_{P},\ldots,P_N}}
  \and \\
  \inferrule* [lab=process] {} {{M_{P}} \bc M_{N} \;| \;P|M_{P} }
\end{mathpar} 

\begin{mathpar}
  \inferrule* [lab=sychronization] {} {M_{N} \bc \Box \;|\; x?M_{F} \;|\; x!M_{C}}
  \and
  \inferrule* [lab=abstraction] {} {{M_{F}} \bc (x)M_{P} }
  \and
  \inferrule* [lab=concretion] {} {{M_{C}} \bc \langle M_{P} \rangle }
  \and \\
  \inferrule* [lab=process] {} {{M_{P}} \bc M_{N} \;| \;P|M_{P} }
\end{mathpar}

\begin{definition}[contextual application] Given a context $M$, and
  process $P$, we define the \emph{contextual application}, $M[P] :=
  M\{P/\Box\}$. That is, the contextual application of M to P is the
  substitution of $P$ for $\Box$ in $M$.
\end{definition}

$\meaningof{-} : L \to \mathcal{P}(\pi)$

\begin{mathpar}
  \inferrule* [lab=collection] {} {\meaningof{true} = \pi, \and \meaningof{~E} = \pi \setminus \meaningof{E}, \and \meaningof{E_{1} \& E_{2}} = \meaningof{E_{1}} \cap \meaningof{E_{2}}}
\end{mathpar}

\begin{mathpar}
  \inferrule* [lab=structure] {} {\meaningof{0} = \{ P \in \pi | P \equiv 0 \}, \and \\ \meaningof{E_1 | E_2} = \{ P \in \pi | P \equiv P_{1} | P_{2}, P_{1} \in \meaningof{E_{1}}, P_{2} \in \meaningof{E_2}\} }
\end{mathpar}

\begin{mathpar}
 \inferrule* [lab=behavior] {} {\meaningof{\langle a?b \rangle E} = \{ P \in \pi | P \equiv Q | u?(y)P', \\ \and \\\\ \and \\ \;\;\; u \in \meaningof{a}, \forall z.P'\{z/y\} \in \meaningof{E\{z/b\}}\}, \and \\ \meaningof{a!E} = \{ P \in \pi | P \equiv Q | x!\langle P' \rangle, x \in \meaningof{a} P' \in \meaningof{E}\} }
\end{mathpar}

\begin{mathpar}
 \inferrule* [lab=nominal] {} {\meaningof{\quotep{E}} = \{ \quotep{P} \in \quotep{\pi} | P \in \meaningof{E} \}, \and \meaningof{\quotep{P}} = \{ \quotep{Q} \in \quotep{\pi} | P \equiv Q \} \and \\ \meaningof{@\quotep{E}} = \{ P \in \pi | P \equiv @x, x \in \meaningof{E} \}}
\end{mathpar}

\begin{eqnarray*}
  \\
  \meaningof{-} : TS \to ST
\end{eqnarray*}

\begin{eqnarray*}
  \\
  L : TS \to ST
\end{eqnarray*}

\begin{eqnarray*}
  \\
  P \models E \iff P \in \meaningof{E}
\end{eqnarray*}

\begin{eqnarray*}
  P \approx_{L} Q \iff \forall E \in L. P \models E \iff Q \models E
\end{eqnarray*}

\begin{eqnarray*}
  P \approx_{K} Q
\end{eqnarray*}

\begin{eqnarray*}
  P \approx Q
\end{eqnarray*}

$\approx_{K} = \approx = \approx_{L}$

\subsubsection{Contextual duality}

Note that contexts extend the quotation operation to a family of
operations from processes to names. Given a context, $M$, we can
define a \emph{nominal context}, $\quotep{M}$ by $\quotep{M}[P] :=
\quotep{M[P]}$. To foreshadow what is to come we observe that these
operations enjoy a duality with processes very much like the duality
between vectors and maps from vectors to scalars.

Further, because the calculus is essentially higher-order, we have a
correspondence between contexts and processes. More specifically,
given a name $x$ and a context $M$ we can construct $M^{*}_{x}$ such
that 

\begin{mathpar}
  M^{*}_{x} | \lift{x}{P} \red M[P]
\end{mathpar}

namely,

\begin{mathpar}
  M^{*}_{x} := x?(u).M[\dropn{u}]
\end{mathpar}

The dependence of $M^{*}_{x}$ on a name makes it an abstraction, 

\begin{mathpar}
  M^{*} := (x)x?(u).M[\dropn{u}]
\end{mathpar}

\subsection{Additional notation}

It will sometimes be convenient to denote the process a name
quotes. We already have the notation $x = \quotep{P}$, but it will be
convenient to introduce an alternate notation, $\procn{x}$, when we
want to emphasize the connection to the use of the name. Note that, by
virtue of name equivalence, $\quotep{\procn{x}} \nameeq x$; so, the
notation is consistent with previous definitions.

Further, because names have structure it is possible to effect
substitutions on the basis of that structure. This means we need to
upgrade our notation for substitutions, which we accomplish by
adapting comprehension notation. Thus,

\begin{mathpar}
  P\{ y / x : x \in S \}
\end{mathpar}

is interpreted to mean the process derived from P by replacing (in a
capture-avoiding manner) each occurrence of $x$ in $S$ by $y$. For example,

\begin{mathpar}
  P\{ \quotep{\procn{x}|\procn{x}} / x : x \in \freenames{P} \}
\end{mathpar}

will replace each (occurrence) of a free name $x$ in $P$ by
$\quotep{\procn{x}|\procn{x}}$.

Also, we will avail ourselves of the notation $x^{L}$ and $x^{R}$ to
denote injections of a name into disjoint copies of the name
space. There are numerous ways to accomplish this. One example can be
found in \cite{MeredithR05}. This notation overloads to vectors of
names: $\vec{x}^{\pi} := (x_{i}^{\pi} \; : \; 0 \leq i < |\vec{x}| )$ where $\pi \in \{L,R\}$.

We also use $P^{\Box} := P|\Box$.

In \cite{MeredithR05} an interpretation of the new operator is
given. It turns out that there are several possible interpretations
all enjoying the requisite algebraic properties of the operator (see
\cite{milner91polyadicpi}). We will therefore make liberal use of
$(\nu\; \vec{x})P$.

% subsection the_syntax_and_semantics_of_the_notation_system (end)   

\input{qm2pi.qmops} 

\input{qm2pi.sterngerlach} 

\input{qm2pi.metric} 

% section concurrent_process_calculi (end)

%\input{qm2pi.proofsketch}

% section proof sketch (end)

%\input{qm2pi.slviaknots} 

% section spatial logic via knots (end)

\input{qm2pi.conclusion}

% section conclusion (end)

%\input{qm2pi.dtcodes} 

% section wiring algorithm (end)

\input{qm2pi.ack} 

% section acknowledgments (end)

\newpage


\bibliographystyle{plain}   
\bibliography{../../biblios/main.bib}

\input{qm2pi.rhodetails}

\end{document}



% section proof sketch (end)

%\section{Unlikely characters: spatial logic for
  knots}\label{sub:characteristic_formulae} % (fold)

Associated to the mobile process calculi are a family of logics known
as the Hennessy-Milner logics. These logics typically enjoy a
semantics interpreting formulae as sets of processes that when
factored through the encoding outlined above allows an identification
of classes of knots with logical formulae. In the context of this
encoding the sub-family known as the spatial logics \cite{CairesC03}
\cite{CairesC04} \cite{Caires04} are of particular interest providing
several important features for expressing and reasoning about
properties (i.e. classes) of knots. We hint here at how this may be done.

%\begin{description}
%\item [structural connectives] 
\subsubsection{Structural connectives} The spatial logics enjoy
structural connectives corresponding, at the logical level, to the
parallel composition ($P | Q$) and new name ($(\nu \; x)P$)
connectives for processes. As illustrated in the examples below, these
connectives are extremely expressive given the shape of our encoding.
%\item [decideable satisfaction]

\subsubsection{Decideable satisfaction}
In \cite{Caires04} the satisfaction relation is shown to be decideable
for a rich class of processes. It further turns out that the image of
the our encoding is a proper subset of that class. This result
provides the basis for an algorithm by which to search for knots
enjoying a given property.
%\item [characteristic formulae]

\subsubsection{Characteristic formulae}
In the same paper \cite{Caires04} , Caires presents a means of calculating
characteristic formulae, selecting equivalence classes of processes
up to a pre--specified depth limit on the support set of names. Composed with our
encoding, this characteristic formula can be used to select
characteristic formulae for knots.
%\end{description}

\subsubsection{Spatial logic formulae}

The grammar below (segmented for comprehension) summarizes the syntax
of spatial logic formulae. We employ illustrative examples in the
sequel to provide an intuitive understanding of their meaning
referring the reader to \cite{Caires04} for a more detailed explication
of the semantics.

\begin{mathpar}
  \inferrule* [lab=boolean] {} {{A,B} \bc T \;|\; \neg A \;|\; A \wedge B \;|\; \eta = \eta'}
  \and
  \inferrule* [lab=spatial] {} {|\; \pzero \;|\; A | B \;|\; x \text{\textregistered} A \;|\; \forall x . A \;|\;  H x . A}
  \and
  \inferrule* [lab=behavioral] {} {|\; \alpha . A}
  \and 
  \inferrule* [lab=recursion] {} {|\; X(\vec{u}) \;|\; \mu X(\vec{u}) . A}
  \and
  \inferrule* [lab=action] {} {\alpha \bc \langle x?(\vec{y}) \rangle \;|\; \langle x!(\vec{y}) \rangle \;|\; \langle \tau \rangle}
  \and 
  \inferrule* [lab=name] {} {\eta \bc x \;|\; \tau}
\end{mathpar} 

% subsection characteristic_formulae (end)   	 

\subsection{Example formulae}\label{sub:example_formulae_} % (fold)

\subsubsection{Crossing as formula.}
% 
% \begin{align*}
%   \frac{d}{dx} \sin x &= \cos x 
%   & \frac{d}{dx} e^x &= e^x \\
%   \frac{d}{dx} \cos x &= - \sin x 
%   & \frac{d}{dx} \log x &= \frac{1}{x} \\
% \end{align*} 

\begin{align*}
 \mu C(x_{0},x_{1},y_{0},y_{1},u).&(\langle x_{0}?(z) \rangle(\langle u! \rangle\langle y_{1}!z \rangle C(x_{0},x_{1},y_{0},y_{1},u)) & \\
  & \wedge \langle y_{1}?(z) \rangle (\langle u! \rangle \langle x_{0}!z \rangle C(x_{0},x_{1},y_{0},y_{1},u)) & \\
  & \wedge \langle x_{1}?(z) \rangle (\langle u? \rangle \langle y_{0}!z \rangle C(x_{0},x_{1},y_{0},y_{1},u)) & \\
  & \wedge \langle y_{0}?(z) \rangle (\langle u? \rangle \langle x_{1}!z \rangle C(x_{0},x_{1},y_{0},y_{1},u))) &
\end{align*}

The lexicographical similarity between the shape of this formulae and
the shape of definition of the process representing a crossing reveals
the intuitive meaning of this formulae. It describes the capabilities
of a process that has the right to represent a crossing. For example
it picks out processes that may perform an input on the port $x_0$ in
its initial menu of capabilities. What differentiates the formula
from the process, however, is that the crossing process is the
smallest candidate to satisfy the formula. Infinitely many other
processes -- with internal behavior hidden behind this interface, so
to speak -- also satisfy this formula. Even this simple formula,
then, can be seen to open a new view onto knots, providing a
computational interpretation of \emph{virtual} knots.

Note that this formula is derived by hand. A similar formula can be
derived by employing Caires' calculation of characteristic formula
\cite{Caires04} to the process representing a crossing. In light of
this discussion, we let
$\meaningof{C}_{\phi}(x0,x1,y0,y1,u)$ denote a formula specifying the
dynamics we wish to capture of a crossing. To guarantee we preserve
the shape of the interface and minimal semantics we demand that
$\meaningof{C}_{\phi}(x0,x1,y0,y1,u) \Rightarrow
\textbf{C}(x0,x1,y0,y1,u)$ where $\textbf{C}(x0,x1,y0,y1,u)$ denotes
the formula above.
                            
\subsubsection{Crossing number constraints.}
The moral content of the context lemma (Lemma \ref{context}) is that the notion of
``locality'' in the Reidemeister moves is effectively captured by the
parallel composition operator of the process calculus. This intuition
extends through the logic. Given a formula,
$\meaningof{C}_{\phi}(x0,x1,y0,y1,u)$, we can use the structural
connectives to specify constraints on crossing numbers, such as at
least $n$ crossings, or exactly $n$ crossings.
\begin{mathpar}
  \inferrule* [lab=at-least-n] {} { K^{\geq n}_{\phi}(\vec{xs},\vec{ys}) := \Pi_{i=0}^{n-1} Hu . \meaningof{C}_{\phi}(xs_i,ys_i,u) | T }
  \and 
  \inferrule* [lab=exactly-n] {} { K^{= n}_{\phi}(\vec{xs},\vec{ys}) := \Pi_{i=0}^{n-1} Hu . \meaningof{C}_{\phi}(xs_i,ys_i,u) | \neg (\forall x_0,y_0,x_1,y_1,u . \meaningof{C}_{\phi}(x_0,y_0,x_1,y_1,u) | T) }
\end{mathpar}

To round out this section, recall that the encoding of an $n$-crossing
knot decomposes into a parallel composition of $n$ \emph{copies} of a
crossing process together with a wiring harness. To specify different
knot classes with the same crossing number amounts to specifying
logical constraints on the wiring harness. In the interest of space,
we defer examples to a forthcoming paper. Suffice it to say that both
the conditions ``alternating knot'' and ``contains the tangle
corresponding to 5/3'' are expressible. For example, it is possible to
calculate the characteristic formula of a process corresponding to the
tangle 5/3 and conjoin it into the classifying formula via the
composition connective of the logic.

Finally, we wish to observe that it is entirely within reason to
contemplate a more domain-specific version of spatial logic tailored
to the shape of processes in the image of the encoding. Such a
domain-specific logic would have a better claim to the title formal
language of knot properties.

% subsection example_formulae_ (end)

% section knots_as_processes (end) 

% section spatial logic via knots (end)

\section{Conclusions and future work}

\paragraph{Testing physical space}
You, gentle reader, may wonder why of all the theorems to be proved
given this set up we pick the one above. In some sense it's hardly
central to quantum mechanics. We see it as central in the sense that
it firmly establishes a notion of physical space arising from a notion
of the equivalence of behavior. Relating bisimulation to a metric is a
big step forward, but one is faced with interpreting the relationship
of that metric space to something more physical. Quantum mechanical
notions of ``physical'' space are still far from intuitive, but by
relating this idea of distance as testing to calculations that predict
physical circumstances we are making a not insignificant step forward
toward an understanding of the physical space we inhabit as
essentially dynamic.

\paragraph{Effectivity and simulation}
One of the observations we have yet to make is that the entire program
spelled out here is effective. We have built various interpreters for
the reflective calculus at work in this interpretation. In principle,
then, we can simulate quantum mechanics on a computer. The place where
the simulation may lose fidelity is the infinitely branching summation
for the annihilator.

In this connection i also want to point out that the evaluation style
calculation of the inner product puts the non-determinism of the
summation right at the heart of measurement. This suggests that
Milner's original reduction-based formulation of the dynamics of his
calculi in terms of sums was not just notationally suggestive of a
notion of measure-and-continue but captured some significant part of
the physics.

\paragraph{Quantum continuations}
In light of this last observation i want to point out that the
predominant account of quantum mechanics is missing a key aspect of a
truly compositional story of the physical situation. In a real lab,
when a measurement is made the observation can be made to feed into
another device that then makes another measurement conditioned on the
results of the first. This means that after the superposition was
collapsed the entire experimental set up remained in
superposition. While QM offers a means of writing this down it doesn't
quite line up well with the well-trodden formulation of computation
and continuation that we see so succinctly expressed in Milner's
calculi. This suggests that there might be advantages to this account
of dynamics waiting to be explored.

\paragraph{Quantum logic}
In this connection, we also note that by virtue of having the
Hennessy-Milner construction, we can pull the construction through the
interpretation of QM. This gives us a natural candidate for a quantum
logic that enjoys an extremely tight connection with it's domain of
interpretation, making the construction much less ad hoc (rather it is
the image of functor!).

\paragraph{Quantum probabiity}
i have questions about the basis of the interpretation of inner
product as probability amplitude. In particular, using which
axiomatization of probability theory does the notion of probability
amplitude earn the right to be so dubbed? In other words, where is the
proof that the operation for calculating a probability amplitude (and
then squaring) satisfies the axioms of what it means to calculate a
probability? Even if such a proof exists (i have yet to find it in the
literature), i wonder if it might not be possible to turn things on
their heads. Can we view the calculation of the probability amplitude
as an axiomatization of probability? If so, then the definition we
give for calculating probability amplitude may provide the basis for
an \emph{effective} theory of probability.

\paragraph{Quantum vs ``biological'' information}
Finally, i want to conclude with a more philosophical observation. At
a recent workshop in which QM was a predominant topic i noticed
something about quantum information. The speaker was giving a riveting
discussion of axiomatic QM and showing how properties of ``no
cloning'' and ``no deleting'' emerged as consequences of the
axiomatization. Theorems of this form are necessary to give us a sense
of confidence that our axioms characterize the physical theory. What
struck me, though, was that if quantum information is neither erasable
nor replicable it is markedly different from \emph{life}. Two of the
things we know about life is that

\begin{itemize}
  \item it ends;
  \item to gain some measure of persistence, to transcend it's
    finitude it is imminently copyable.
\end{itemize}

Both of these qualities are summarized succinctly in the aphorism: all
flesh is grass. For me these two kinds of ``information'' -- call them
quantum and biological -- are end points on a spectrum of strategies
for persistence. At one end, we have those curious entities that enjoy
uniqueness and permanence; at the other, we have those who in the face
of a certain end and an uncertain present make a go of passing
something on. To me one of the more remarkable aspects of the latter
strategy is that in the presence of noise (and certain features of
copying) we get a kind of dynamism, a chance for improvement against a
given persistent condition.

% subsection other_calculi_other_bisimulations_and_geometry_as_behavior (end)




% section conclusion (end)

%\documentclass[12pt]{llncs}
%\documentclass{jktr}

\usepackage[pdftex]{hyperref}                   
\usepackage {listings}
\usepackage {mathpartir}
\usepackage{bcprules}
%\usepackage{listings}
                       
\usepackage{graphicx} 
%\usepackage[margins=2.5cm,nohead,nofoot]{geometry}
%\usepackage{geometry}
\usepackage{amsfonts}
\usepackage{amstext}
\usepackage{latexsym}
\usepackage{amssymb}
\usepackage{color}


%\include{myPreamble}
\include{qm2pi.local} 

%\ifpdf
%\usepackage[pdftex]{graphicx}
%\else
%\usepackage{graphicx}
%\fi

 % \ifpdf
%  \usepackage{pdfsync}
%  \if


%\title{Brief Article}
%\author{David F. Snyder}
%\author{L.G. Meredith}

%\address{Dept. of Math., Texas State University--San Marcos, San Marcos, TX 78666}
       
\pagestyle{empty}


\begin{document}

\lstset{language=[Objective]Caml,frame=shadowbox}

\input{qm2pi.front}

% section front matter (end)

\input{qm2pi.intro} 
 
% section introduction (end)

% \input{qm2pi.knotations} 

% section notation (end)

\input{qm2pi.process.calculi} 

% section concurrent_process_calculi_and_spatial_logics_ (end)
    
%\input{qm2pi.knots2pi} 

%\input{qm2pi.trefoil} 

%\input{qm2pi.mainthm} 

% subsection basic_interpretation (end)

%\input{qm2pi.rho.presentation} 
\subsection{The syntax and semantics of the notation system}\label{sub:the_syntax_and_semantics_of_the_notation_system} % (fold)

We now summarize a technical presentation of the calculus that
embodies our theory of dynamics. The typical presentation of such a
calculus follows the style of giving generators and relations on
them. The grammar, below, describing term constructors, freely
generates the set of processes, $\Proc$. This set is then quotiented
by a relation known as structural congruence and it is over this set
that the notion of dynamics is expressed. This presentation is
essentially that of \cite{MeredithR05} with the addition of
polyadicity and summation. For readability we have relegated some of
the technical subtleties to an appendix.

\subsubsection{Process grammar}\label{subsub:process_grammar}

\begin{mathpar}
  \inferrule* [lab=synchronization] {} {{M} \bc \pzero \;|\; x?F \;|\; x!C }
  \and
  \inferrule* [lab=abstraction] {} {{F} \bc (x)P}
  \and
  \inferrule* [lab=concretion] {} {{C} \bc \langle Q \rangle}
  \and
  \inferrule* [lab=process] {} {{P,Q} \bc M \;| \;P|Q \;|\; @{x}}
  \and
  \inferrule* [lab=name] {} {{x} \bc \quotep{P}}
\end{mathpar} 

Note that $\vec{x}$ (resp. $\vec{P}$) denotes a vector of names
(resp. processes) of length $|\vec{x}|$ (resp. $|\vec{P}|$). We adopt
the following useful abbreviations.

\begin{mathpar}
   x?(\vec{y}).P := x.(\vec{y})P \and  x\clift{\vec{P}} := x.\clift{\vec{P}}
   \and x!(y) := \lift{x}{\dropn{y}}
   \and \Pi_{i=0}^{n-1}P_i := P_0 | \ldots | P_{n-1}
\end{mathpar}

\subsubsection{Structural congruence}

\paragraph{Free and bound names and alpha-equivalence.} At the
core of structural equivalence is alpha-equivalence which identifies
process that are the same up to a change of variable. Formally, we
recognize the distinction between free and bound names. The free names
of a process, $\freenames{P}$, may be calculated recursively as
follows:

\begin{mathpar}
\freenames{\pzero} := \emptyset
  \and \\
  \freenames{x?(y).P} := \{ x \} \cup (\freenames{P} \setminus \{ y \})
  \and 
  \freenames{x!\langle P \rangle} := \{ x \} \cup \{ P \} 
  \and \\
  \freenames{P|Q} := \freenames{P} \cup \freenames{Q}
  \and \\
  \freenames{@{x}} := \{ x \}
\end{mathpar}

$\pi$
$\quotep{\pi}$

$\freenames{-} : \pi \to \mathcal{P}(\quotep{\pi})$

\begin{eqnarray*}
  \freenames{\pzero} & := & \emptyset \\
  \freenames{x?(y).P} & := & \{ x \} \cup (\freenames{P} \setminus \{ y \}) \\
  \freenames{x!\langle P \rangle} & := & \{ x \} \cup \{ P \} \\
  \freenames{P|Q} & := & \freenames{P} \cup \freenames{Q} \\
  \freenames{\dropn{x}} & := & \{ x \}
\end{eqnarray*}

The bound names of a process, $\boundnames{P}$, are those names occurring in $P$
that are not free. For example, in $x?(y).0$, the name $x$ is free, while $y$ is bound.

\begin{mathpar}
  \inferrule* [lab=monoidal-laws] {} { P|Q \equiv Q|P \and P|0 \equiv P \and P|(Q|R) \equiv (P|Q)|R }
\end{mathpar}

\begin{mathpar}
  \inferrule* [lab=alpha-equivalence] {} { (x)P \equiv (y)P\{y/x\} \and y \not\in \freenames{P} }
\end{mathpar}

\begin{definition}
Then two processes, $P,Q$, are alpha-equivalent if $P = Q\{\vec{y}/\vec{x}\}$ for
some $\vec{x} \in \boundnames{Q},\vec{y} \in \boundnames{P}$, where $Q\{\vec{y}/\vec{x}\}$
denotes the capture-avoiding substitution of $\vec{y}$ for $\vec{x}$ in $Q$.
\end{definition}

\begin{definition}
  The {\em structural congruence} \cite{SangiorgiWalker} , $\equiv$,
  between processes is the least congruence containing
  alpha-equivalence, satisfying the abelian monoid laws
  (associativity, commutativity and $\pzero$ as identity) for parallel
  composition $|$ and for summation $+$.
\end{definition}

\subsection{Name equivalence}

We take name equivalence, written $\nameeq$, to be the smallest
equivalence relation generated by the following rules.

\begin{mathpar}
\inferrule*[lab=Quote-drop]
{ }
{ \quotep{@{x}} \nameeq x }

\inferrule*[lab=Struct-equiv]
{ P \scong Q }
{ \quotep{P} \nameeq \quotep{Q} }
\end{mathpar}

The astute reader will have noticed that the mutual recursion of names
and processes imposes a mutual recursion on alpha-equivalence and
structural equivalence via name-equivalence. Fortunately, all of this
works out pleasantly and we may calculate in the natural way, free of
concern. The reader interested in the details is referred to the
appendix \ref{appendix:rho_details}.

\subsection{Substitution}

We use $\Proc$ for the set of processes, $\QProc$ for the set of
names, and $\id{\{}\vec{y} / \vec{x} \id{\}}$ to denote partial maps,
$s : \QProc \rightarrow \QProc$. A map, $s$ lifts, uniquely, to a map
on process terms, $\widehat{s} : \Proc \rightarrow \Proc$ by the
following equations.

\begin{mathpar}
  (0) \psubstp{Q}{P} := 0 \\
  (R \juxtap S) \psubstp{Q}{P}
  :=    
  (R)\psubstp{Q}{P} \juxtap (S) \psubstp{Q}{P} \\
  (x?(y).R) \psubstp{Q}{P}    
  :=    
  (x)\substp{Q}{P} (z)\concat( (R \psubstn{z}{y}) \psubstp{Q}{P} ) \\
  (\lift{x}{R}) \psubstp{Q}{P}  
  :=
  \lift{(x)\substp{Q}{P}}{ R \psubstp{Q}{P} } \\
%   (\dropn{x})  \psubstp{Q}{P}       
%   := 
%   \left\{ 
%     \begin{array}{ccc} 
%       \dropn{\quotep{Q}} & & x \nameeq \quotep{P} \\
%       \dropn{x} & & otherwise \\
%     \end{array}
%   \right. 
  (\dropn{x})  \psubstp{Q}{P}       
  := 
  \left\{ 
    \begin{array}{ccc} 
      Q & & x \nameeq \quotep{P} \\
      \dropn{x} & & otherwise \\
    \end{array}
  \right.
\end{mathpar}
 

where

\begin{eqnarray}
  (x)\id{\{} \lpquote Q \rpquote / \lpquote P \rpquote \id{\}}            = 
  \left\{ 
    \begin{array}{ccc}
      \lpquote Q \rpquote & & x \nameeq \lpquote P \rpquote \\
      x & & otherwise \\
    \end{array}
  \right. \nonumber
\end{eqnarray}

and $z$ is chosen distinct from $\quotep{P}$, $\quotep{Q}$, the free
names in $Q$, and all the names in $R$. Our $\alpha$-equivalence will
be built in the standard way from this substitution.

\begin{remark}\label{rem:no_self_referential_names}
  One consequence of these definitions is that $\forall P. \quotep{P}
  \not\in \freenames{P}$.
\end{remark}

\subsection{ Dynamic quote: an example }

Anticipating something of what's to come, consider applying the
substitution, $\widehat{\id{\{}u / z \id{\}}}$, to the following pair
of processes, $\lift{w}{y!(z)}$ and $w[ \lpquote y!(z) \rpquote ]$.

\begin{eqnarray}
	\lift{w}{y!(z)}\widehat{\id{\{}u / z \id{\}}}
		& = &
		\lift{w}{y!(u)} \nonumber\\
	w[ \lpquote y!(z) \rpquote ] \widehat{ \id{\{}u / z \id{\}} }
		& = &
		w[ \lpquote y!(z) \rpquote ] \nonumber
\end{eqnarray}

Because the body of the process between quotes is impervious to
substitution, we get radically different answers. In fact, by
examining the first process in an input context,
e.g. $x?(z).\lift{w}{y!(z)}$, we see that the process under the lift
operator may be shaped by prefixed inputs binding a name inside it. In
this sense, the lift operator will be seen as a way to dynamically
construct processes before reifying them as names.

Finally equipped with these standard features we can present the
dynamics of the calculus.

\subsubsection{Operational semantics} 

Finally, we introduce the computational dynamics. What marks these
algebras as distinct from other more traditionally studied algebraic
structures, e.g. vector spaces or polynomial rings, is the manner in
which dynamics is captured. In traditional structures, dynamics is typically
expressed through morphisms between such structures, as in linear maps
between vector spaces or morphisms between rings. In algebras
associated with the semantics of computation, the dynamics is
expressed as part of the algebraic structure itself, through a
reduction reduction relation typically denoted by $\red$. Below, we
give a recursive presentation of this relation for the calculus used
in the encoding.

$\red \subseteq \pi \times \pi$
$\red : \pi \to \mathcal{P}(\pi)$

\begin{mathpar}
  \inferrule* [lab=Comm] { \textsf{match}( x_{src}, x_{trgt} ) } { x_{trgt}?(y)P \; | \; x_{src}!\langle {Q} \rangle \red P\{\quotep{Q}/y}\} }
  \and \\
  \inferrule* [lab=Par] {{P} \red {P}'} {{{P} | {Q}} \red {{P}' | {Q}}}
  \and
  \inferrule* [lab=Equiv]{{{P} \scong {P}'} \andalso {{P}' \red {Q}'} \andalso {{Q}' \scong {Q}}}{{P} \red {Q}}
\end{mathpar}

\begin{eqnarray*}
  match_{\equiv} (\quotep{P},\quotep{Q}) & := & P \equiv Q \\
  match_{\dagger}(\quotep{P},\quotep{Q}) & := & \forall R. P|Q \red^{*} R => R \red^{*} 0 \\
  match_{K}(\quotep{P},\quotep{Q}) & := & K \mbox{ for some context } K
\end{eqnarray*}

$u?(x)P | u!\langle Q \rangle \red P\{\quotep{Q}/x\}$

%We write $\wred$ for $\red^*$, and $P\red$ if $\exists Q $ such that $ P \red Q$.
We write $P\red$ if $\exists Q $ such that $ P \red Q$ and $P\not\red$, otherwise.

\section{Replication}

As mentioned before, it is known that replication (and hence
recursion) can be implemented in a higher-order process algebra
\cite{SangiorgiWalker}. As our first example of calculation with the
machinery thus far presented we give the construction explicitly in
the {\rhoc}.

\begin{eqnarray}
	D_{x} & := & \prefix{x}{y}{(\binpar{\outputp{x}{y}}{@{y}})} \nonumber\\
	\bangp_{x}{P} & := & \binpar{{x}!\langle{\binpar{D_{x}}{P}}\rangle}{D_{x}} \nonumber
\end{eqnarray}

\begin{eqnarray}
	\bangp_{x}{P} & & \nonumber\\
	=
	& {x}!\langle{(\prefix{x}{y}{(\outputp{x}{y} | @{y})) | P}}\rangle 
	      | \prefix{x}{y}{(\outputp{x}{y} | @{y})} & \nonumber\\
	\red
	& (\outputp{x}{y} | @{y})\substn{\quotep{(\prefix{x}{y}{(@{y} | \outputp{x}{y})) | P}}}{y} & \nonumber\\
	=
	& \outputp{x}{\quotep{(\prefix{x}{y}{(\outputp{x}{y} | @{y})) | P}}}
	  | {(\prefix{x}{y}{(\outputp{x}{y} | @{y})) | P}} & \nonumber\\
	\red
	& \ldots & \nonumber\\
	\red^*
	& P | P | \ldots & \nonumber
\end{eqnarray}

Of course, this encoding, as an implementation, runs away, unfolding
$\bangp{P}$ eagerly. A lazier and more implementable replication
operator, restricted to input-guarded processes, may be obtained as follows.

\begin{eqnarray}
\bangp{\prefix{u}{v}{P}} 
	:= 
	\binpar{\lift{x}{\prefix{u}{v}{(\binpar{D(x)}{P})}}}{D(x)} \nonumber
\end{eqnarray}

\begin{remark}
  Note that the lazier definition still does not deal with summation
  or mixed summation (i.e. sums over input and output). The reader is
  invited to construct definitions of replication that deal with these
  features. 

  Further, the definitions are parameterized in a name, $x$. Can you,
  gentle reader, make a definition that eliminates this parameter and
  guarantees no accidental interaction between the replication
  machinery and the process being replicated -- i.e. no accidental
  sharing of names used by the process to get its work done and the
  name(s) used by the replication to effect copying. This latter
  revision of the definition of replication is crucial to obtaining
  the expected identity $!!P \sim !P$.
\end{remark}

\begin{remark}\label{rem:paradoxical_combinator}
  The reader familiar with the lambda calculus will have noticed the
  similarity between $D$ and the paradoxical combinator.

  [Ed. note: the existence of this seems to suggest we have to be more
  restrictive on the set of processes and names we admit if we are to
  support no-cloning.]
\end{remark}

\subsubsection{Bisimulation}

The computational dynamics gives rise to another kind of equivalence,
the equivalence of computational behavior. As previously mentioned
this is typically captured \emph{via} some form of bisimulation.

% The notion we use in this paper is weak barbed bisimulation
% \cite{milner91polyadicpi}.

The notion we use in this paper is derived from weak barbed
bisimulation \cite{milner91polyadicpi}. 

\begin{definition}
An \emph{observation relation}, $\downarrow_{\mathcal N}$, over a set
of names, $\mathcal N$, is the smallest relation satisfying the rules
below.

\infrule[Out-barb]{y \in {\mathcal N}, \; x \nameeq y}
		  {\outputp{x}{v} \downarrow_{\mathcal N} x}
\infrule[Par-barb]{\mbox{$P\downarrow_{\mathcal N} x$ or $Q\downarrow_{\mathcal N} x$}}
		  {\binpar{P}{Q} \downarrow_{\mathcal N} x}

We write $P \Downarrow_{\mathcal N} x$ if there is $Q$ such that 
$P \wred Q$ and $Q \downarrow_{\mathcal N} x$.
\end{definition}

\begin{definition}
%\label{def.bbisim}
An  ${\mathcal N}$-\emph{barbed bisimulation} over a set of names, ${\mathcal N}$, is a symmetric binary relation 
${\mathcal S}_{\mathcal N}$ between agents such that $P\rel{S}_{\mathcal N}Q$ implies:
\begin{enumerate}
\item If $P \red P'$ then $Q \wred Q'$ and $P'\rel{S}_{\mathcal N} Q'$.
\item If $P\downarrow_{\mathcal N} x$, then $Q\Downarrow_{\mathcal N} x$.
\end{enumerate}
$P$ is ${\mathcal N}$-barbed bisimilar to $Q$, written
$P \wbbisim_{\mathcal N} Q$, if $P \rel{S}_{\mathcal N} Q$ for some ${\mathcal N}$-barbed bisimulation ${\mathcal S}_{\mathcal N}$.
\end{definition}

$\mathcal{R} \subseteq \pi \times \pi$

$P \mathcal{R} Q => \forall P'. P \red P' \Rightarrow \exists Q'. Q \red Q', P' \mathcal{R} Q'$

$P \vdash x \Rightarrow Q \vdash x$

\begin{mathpar}
  \inferrule*[lab=Out-barb]{x \nameeq y}{{y}!\langle{Q}\rangle \vdash x}
  \and
  \inferrule*[lab=Par-barb]{\mbox{$P\vdash x$ or $Q\vdash x$}}{\binpar{P}{Q} \vdash x}
\end{mathpar}

\subsubsection{Contexts}

One of the principle advantages of computational calculi like the
$\pi$-calculus is a well-defined notion of context,
contextual-equivalence and a correlation between
contextual-equivalence and notions of bisimulation. The notion of
context allows the decomposition of a process into (sub-)process and
its syntactic environment, its context. Thus, a context may be
thought of as a process with a ``hole'' (written $\Box$) in it. The
application of a context $M$ to a process $P$, written $M[P]$, is
tantamount to filling the hole in $M$ with $P$. In this paper we do
not need the full weight of this theory, but do make use of the notion
of context in the proof the main theorem. 

\begin{mathpar}
  \inferrule* [lab=summation] {} {{M_{M},M_{N}} \bc \Box \;|\; x.M_{A} \;|\; M_{M}+M_{N}}
  \and
  \inferrule* [lab=agent] {} {{M_{A}} \bc (\vec{x})M_{P} \;| \; \clift{P_0,\ldots,M_{P},\ldots,P_N}}
  \and \\
  \inferrule* [lab=process] {} {{M_{P}} \bc M_{N} \;| \;P|M_{P} }
\end{mathpar} 

\begin{mathpar}
  \inferrule* [lab=sychronization] {} {M_{N} \bc \Box \;|\; x?M_{F} \;|\; x!M_{C}}
  \and
  \inferrule* [lab=abstraction] {} {{M_{F}} \bc (x)M_{P} }
  \and
  \inferrule* [lab=concretion] {} {{M_{C}} \bc \langle M_{P} \rangle }
  \and \\
  \inferrule* [lab=process] {} {{M_{P}} \bc M_{N} \;| \;P|M_{P} }
\end{mathpar}

\begin{definition}[contextual application] Given a context $M$, and
  process $P$, we define the \emph{contextual application}, $M[P] :=
  M\{P/\Box\}$. That is, the contextual application of M to P is the
  substitution of $P$ for $\Box$ in $M$.
\end{definition}

$\meaningof{-} : L \to \mathcal{P}(\pi)$

\begin{mathpar}
  \inferrule* [lab=collection] {} {\meaningof{true} = \pi, \and \meaningof{~E} = \pi \setminus \meaningof{E}, \and \meaningof{E_{1} \& E_{2}} = \meaningof{E_{1}} \cap \meaningof{E_{2}}}
\end{mathpar}

\begin{mathpar}
  \inferrule* [lab=structure] {} {\meaningof{0} = \{ P \in \pi | P \equiv 0 \}, \and \\ \meaningof{E_1 | E_2} = \{ P \in \pi | P \equiv P_{1} | P_{2}, P_{1} \in \meaningof{E_{1}}, P_{2} \in \meaningof{E_2}\} }
\end{mathpar}

\begin{mathpar}
 \inferrule* [lab=behavior] {} {\meaningof{\langle a?b \rangle E} = \{ P \in \pi | P \equiv Q | u?(y)P', \\ \and \\\\ \and \\ \;\;\; u \in \meaningof{a}, \forall z.P'\{z/y\} \in \meaningof{E\{z/b\}}\}, \and \\ \meaningof{a!E} = \{ P \in \pi | P \equiv Q | x!\langle P' \rangle, x \in \meaningof{a} P' \in \meaningof{E}\} }
\end{mathpar}

\begin{mathpar}
 \inferrule* [lab=nominal] {} {\meaningof{\quotep{E}} = \{ \quotep{P} \in \quotep{\pi} | P \in \meaningof{E} \}, \and \meaningof{\quotep{P}} = \{ \quotep{Q} \in \quotep{\pi} | P \equiv Q \} \and \\ \meaningof{@\quotep{E}} = \{ P \in \pi | P \equiv @x, x \in \meaningof{E} \}}
\end{mathpar}

\begin{eqnarray*}
  \\
  \meaningof{-} : TS \to ST
\end{eqnarray*}

\begin{eqnarray*}
  \\
  L : TS \to ST
\end{eqnarray*}

\begin{eqnarray*}
  \\
  P \models E \iff P \in \meaningof{E}
\end{eqnarray*}

\begin{eqnarray*}
  P \approx_{L} Q \iff \forall E \in L. P \models E \iff Q \models E
\end{eqnarray*}

\begin{eqnarray*}
  P \approx_{K} Q
\end{eqnarray*}

\begin{eqnarray*}
  P \approx Q
\end{eqnarray*}

$\approx_{K} = \approx = \approx_{L}$

\subsubsection{Contextual duality}

Note that contexts extend the quotation operation to a family of
operations from processes to names. Given a context, $M$, we can
define a \emph{nominal context}, $\quotep{M}$ by $\quotep{M}[P] :=
\quotep{M[P]}$. To foreshadow what is to come we observe that these
operations enjoy a duality with processes very much like the duality
between vectors and maps from vectors to scalars.

Further, because the calculus is essentially higher-order, we have a
correspondence between contexts and processes. More specifically,
given a name $x$ and a context $M$ we can construct $M^{*}_{x}$ such
that 

\begin{mathpar}
  M^{*}_{x} | \lift{x}{P} \red M[P]
\end{mathpar}

namely,

\begin{mathpar}
  M^{*}_{x} := x?(u).M[\dropn{u}]
\end{mathpar}

The dependence of $M^{*}_{x}$ on a name makes it an abstraction, 

\begin{mathpar}
  M^{*} := (x)x?(u).M[\dropn{u}]
\end{mathpar}

\subsection{Additional notation}

It will sometimes be convenient to denote the process a name
quotes. We already have the notation $x = \quotep{P}$, but it will be
convenient to introduce an alternate notation, $\procn{x}$, when we
want to emphasize the connection to the use of the name. Note that, by
virtue of name equivalence, $\quotep{\procn{x}} \nameeq x$; so, the
notation is consistent with previous definitions.

Further, because names have structure it is possible to effect
substitutions on the basis of that structure. This means we need to
upgrade our notation for substitutions, which we accomplish by
adapting comprehension notation. Thus,

\begin{mathpar}
  P\{ y / x : x \in S \}
\end{mathpar}

is interpreted to mean the process derived from P by replacing (in a
capture-avoiding manner) each occurrence of $x$ in $S$ by $y$. For example,

\begin{mathpar}
  P\{ \quotep{\procn{x}|\procn{x}} / x : x \in \freenames{P} \}
\end{mathpar}

will replace each (occurrence) of a free name $x$ in $P$ by
$\quotep{\procn{x}|\procn{x}}$.

Also, we will avail ourselves of the notation $x^{L}$ and $x^{R}$ to
denote injections of a name into disjoint copies of the name
space. There are numerous ways to accomplish this. One example can be
found in \cite{MeredithR05}. This notation overloads to vectors of
names: $\vec{x}^{\pi} := (x_{i}^{\pi} \; : \; 0 \leq i < |\vec{x}| )$ where $\pi \in \{L,R\}$.

We also use $P^{\Box} := P|\Box$.

In \cite{MeredithR05} an interpretation of the new operator is
given. It turns out that there are several possible interpretations
all enjoying the requisite algebraic properties of the operator (see
\cite{milner91polyadicpi}). We will therefore make liberal use of
$(\nu\; \vec{x})P$.

% subsection the_syntax_and_semantics_of_the_notation_system (end)   

\input{qm2pi.qmops} 

\input{qm2pi.sterngerlach} 

\input{qm2pi.metric} 

% section concurrent_process_calculi (end)

%\input{qm2pi.proofsketch}

% section proof sketch (end)

%\input{qm2pi.slviaknots} 

% section spatial logic via knots (end)

\input{qm2pi.conclusion}

% section conclusion (end)

%\input{qm2pi.dtcodes} 

% section wiring algorithm (end)

\input{qm2pi.ack} 

% section acknowledgments (end)

\newpage


\bibliographystyle{plain}   
\bibliography{../../biblios/main.bib}

\input{qm2pi.rhodetails}

\end{document}

 

% section wiring algorithm (end)

\documentclass[12pt]{llncs}
%\documentclass{jktr}

\usepackage[pdftex]{hyperref}                   
\usepackage {listings}
\usepackage {mathpartir}
\usepackage{bcprules}
%\usepackage{listings}
                       
\usepackage{graphicx} 
%\usepackage[margins=2.5cm,nohead,nofoot]{geometry}
%\usepackage{geometry}
\usepackage{amsfonts}
\usepackage{amstext}
\usepackage{latexsym}
\usepackage{amssymb}
\usepackage{color}


%\include{myPreamble}
\include{qm2pi.local} 

%\ifpdf
%\usepackage[pdftex]{graphicx}
%\else
%\usepackage{graphicx}
%\fi

 % \ifpdf
%  \usepackage{pdfsync}
%  \if


%\title{Brief Article}
%\author{David F. Snyder}
%\author{L.G. Meredith}

%\address{Dept. of Math., Texas State University--San Marcos, San Marcos, TX 78666}
       
\pagestyle{empty}


\begin{document}

\lstset{language=[Objective]Caml,frame=shadowbox}

\input{qm2pi.front}

% section front matter (end)

\input{qm2pi.intro} 
 
% section introduction (end)

% \input{qm2pi.knotations} 

% section notation (end)

\input{qm2pi.process.calculi} 

% section concurrent_process_calculi_and_spatial_logics_ (end)
    
%\input{qm2pi.knots2pi} 

%\input{qm2pi.trefoil} 

%\input{qm2pi.mainthm} 

% subsection basic_interpretation (end)

%\input{qm2pi.rho.presentation} 
\subsection{The syntax and semantics of the notation system}\label{sub:the_syntax_and_semantics_of_the_notation_system} % (fold)

We now summarize a technical presentation of the calculus that
embodies our theory of dynamics. The typical presentation of such a
calculus follows the style of giving generators and relations on
them. The grammar, below, describing term constructors, freely
generates the set of processes, $\Proc$. This set is then quotiented
by a relation known as structural congruence and it is over this set
that the notion of dynamics is expressed. This presentation is
essentially that of \cite{MeredithR05} with the addition of
polyadicity and summation. For readability we have relegated some of
the technical subtleties to an appendix.

\subsubsection{Process grammar}\label{subsub:process_grammar}

\begin{mathpar}
  \inferrule* [lab=synchronization] {} {{M} \bc \pzero \;|\; x?F \;|\; x!C }
  \and
  \inferrule* [lab=abstraction] {} {{F} \bc (x)P}
  \and
  \inferrule* [lab=concretion] {} {{C} \bc \langle Q \rangle}
  \and
  \inferrule* [lab=process] {} {{P,Q} \bc M \;| \;P|Q \;|\; @{x}}
  \and
  \inferrule* [lab=name] {} {{x} \bc \quotep{P}}
\end{mathpar} 

Note that $\vec{x}$ (resp. $\vec{P}$) denotes a vector of names
(resp. processes) of length $|\vec{x}|$ (resp. $|\vec{P}|$). We adopt
the following useful abbreviations.

\begin{mathpar}
   x?(\vec{y}).P := x.(\vec{y})P \and  x\clift{\vec{P}} := x.\clift{\vec{P}}
   \and x!(y) := \lift{x}{\dropn{y}}
   \and \Pi_{i=0}^{n-1}P_i := P_0 | \ldots | P_{n-1}
\end{mathpar}

\subsubsection{Structural congruence}

\paragraph{Free and bound names and alpha-equivalence.} At the
core of structural equivalence is alpha-equivalence which identifies
process that are the same up to a change of variable. Formally, we
recognize the distinction between free and bound names. The free names
of a process, $\freenames{P}$, may be calculated recursively as
follows:

\begin{mathpar}
\freenames{\pzero} := \emptyset
  \and \\
  \freenames{x?(y).P} := \{ x \} \cup (\freenames{P} \setminus \{ y \})
  \and 
  \freenames{x!\langle P \rangle} := \{ x \} \cup \{ P \} 
  \and \\
  \freenames{P|Q} := \freenames{P} \cup \freenames{Q}
  \and \\
  \freenames{@{x}} := \{ x \}
\end{mathpar}

$\pi$
$\quotep{\pi}$

$\freenames{-} : \pi \to \mathcal{P}(\quotep{\pi})$

\begin{eqnarray*}
  \freenames{\pzero} & := & \emptyset \\
  \freenames{x?(y).P} & := & \{ x \} \cup (\freenames{P} \setminus \{ y \}) \\
  \freenames{x!\langle P \rangle} & := & \{ x \} \cup \{ P \} \\
  \freenames{P|Q} & := & \freenames{P} \cup \freenames{Q} \\
  \freenames{\dropn{x}} & := & \{ x \}
\end{eqnarray*}

The bound names of a process, $\boundnames{P}$, are those names occurring in $P$
that are not free. For example, in $x?(y).0$, the name $x$ is free, while $y$ is bound.

\begin{mathpar}
  \inferrule* [lab=monoidal-laws] {} { P|Q \equiv Q|P \and P|0 \equiv P \and P|(Q|R) \equiv (P|Q)|R }
\end{mathpar}

\begin{mathpar}
  \inferrule* [lab=alpha-equivalence] {} { (x)P \equiv (y)P\{y/x\} \and y \not\in \freenames{P} }
\end{mathpar}

\begin{definition}
Then two processes, $P,Q$, are alpha-equivalent if $P = Q\{\vec{y}/\vec{x}\}$ for
some $\vec{x} \in \boundnames{Q},\vec{y} \in \boundnames{P}$, where $Q\{\vec{y}/\vec{x}\}$
denotes the capture-avoiding substitution of $\vec{y}$ for $\vec{x}$ in $Q$.
\end{definition}

\begin{definition}
  The {\em structural congruence} \cite{SangiorgiWalker} , $\equiv$,
  between processes is the least congruence containing
  alpha-equivalence, satisfying the abelian monoid laws
  (associativity, commutativity and $\pzero$ as identity) for parallel
  composition $|$ and for summation $+$.
\end{definition}

\subsection{Name equivalence}

We take name equivalence, written $\nameeq$, to be the smallest
equivalence relation generated by the following rules.

\begin{mathpar}
\inferrule*[lab=Quote-drop]
{ }
{ \quotep{@{x}} \nameeq x }

\inferrule*[lab=Struct-equiv]
{ P \scong Q }
{ \quotep{P} \nameeq \quotep{Q} }
\end{mathpar}

The astute reader will have noticed that the mutual recursion of names
and processes imposes a mutual recursion on alpha-equivalence and
structural equivalence via name-equivalence. Fortunately, all of this
works out pleasantly and we may calculate in the natural way, free of
concern. The reader interested in the details is referred to the
appendix \ref{appendix:rho_details}.

\subsection{Substitution}

We use $\Proc$ for the set of processes, $\QProc$ for the set of
names, and $\id{\{}\vec{y} / \vec{x} \id{\}}$ to denote partial maps,
$s : \QProc \rightarrow \QProc$. A map, $s$ lifts, uniquely, to a map
on process terms, $\widehat{s} : \Proc \rightarrow \Proc$ by the
following equations.

\begin{mathpar}
  (0) \psubstp{Q}{P} := 0 \\
  (R \juxtap S) \psubstp{Q}{P}
  :=    
  (R)\psubstp{Q}{P} \juxtap (S) \psubstp{Q}{P} \\
  (x?(y).R) \psubstp{Q}{P}    
  :=    
  (x)\substp{Q}{P} (z)\concat( (R \psubstn{z}{y}) \psubstp{Q}{P} ) \\
  (\lift{x}{R}) \psubstp{Q}{P}  
  :=
  \lift{(x)\substp{Q}{P}}{ R \psubstp{Q}{P} } \\
%   (\dropn{x})  \psubstp{Q}{P}       
%   := 
%   \left\{ 
%     \begin{array}{ccc} 
%       \dropn{\quotep{Q}} & & x \nameeq \quotep{P} \\
%       \dropn{x} & & otherwise \\
%     \end{array}
%   \right. 
  (\dropn{x})  \psubstp{Q}{P}       
  := 
  \left\{ 
    \begin{array}{ccc} 
      Q & & x \nameeq \quotep{P} \\
      \dropn{x} & & otherwise \\
    \end{array}
  \right.
\end{mathpar}
 

where

\begin{eqnarray}
  (x)\id{\{} \lpquote Q \rpquote / \lpquote P \rpquote \id{\}}            = 
  \left\{ 
    \begin{array}{ccc}
      \lpquote Q \rpquote & & x \nameeq \lpquote P \rpquote \\
      x & & otherwise \\
    \end{array}
  \right. \nonumber
\end{eqnarray}

and $z$ is chosen distinct from $\quotep{P}$, $\quotep{Q}$, the free
names in $Q$, and all the names in $R$. Our $\alpha$-equivalence will
be built in the standard way from this substitution.

\begin{remark}\label{rem:no_self_referential_names}
  One consequence of these definitions is that $\forall P. \quotep{P}
  \not\in \freenames{P}$.
\end{remark}

\subsection{ Dynamic quote: an example }

Anticipating something of what's to come, consider applying the
substitution, $\widehat{\id{\{}u / z \id{\}}}$, to the following pair
of processes, $\lift{w}{y!(z)}$ and $w[ \lpquote y!(z) \rpquote ]$.

\begin{eqnarray}
	\lift{w}{y!(z)}\widehat{\id{\{}u / z \id{\}}}
		& = &
		\lift{w}{y!(u)} \nonumber\\
	w[ \lpquote y!(z) \rpquote ] \widehat{ \id{\{}u / z \id{\}} }
		& = &
		w[ \lpquote y!(z) \rpquote ] \nonumber
\end{eqnarray}

Because the body of the process between quotes is impervious to
substitution, we get radically different answers. In fact, by
examining the first process in an input context,
e.g. $x?(z).\lift{w}{y!(z)}$, we see that the process under the lift
operator may be shaped by prefixed inputs binding a name inside it. In
this sense, the lift operator will be seen as a way to dynamically
construct processes before reifying them as names.

Finally equipped with these standard features we can present the
dynamics of the calculus.

\subsubsection{Operational semantics} 

Finally, we introduce the computational dynamics. What marks these
algebras as distinct from other more traditionally studied algebraic
structures, e.g. vector spaces or polynomial rings, is the manner in
which dynamics is captured. In traditional structures, dynamics is typically
expressed through morphisms between such structures, as in linear maps
between vector spaces or morphisms between rings. In algebras
associated with the semantics of computation, the dynamics is
expressed as part of the algebraic structure itself, through a
reduction reduction relation typically denoted by $\red$. Below, we
give a recursive presentation of this relation for the calculus used
in the encoding.

$\red \subseteq \pi \times \pi$
$\red : \pi \to \mathcal{P}(\pi)$

\begin{mathpar}
  \inferrule* [lab=Comm] { \textsf{match}( x_{src}, x_{trgt} ) } { x_{trgt}?(y)P \; | \; x_{src}!\langle {Q} \rangle \red P\{\quotep{Q}/y}\} }
  \and \\
  \inferrule* [lab=Par] {{P} \red {P}'} {{{P} | {Q}} \red {{P}' | {Q}}}
  \and
  \inferrule* [lab=Equiv]{{{P} \scong {P}'} \andalso {{P}' \red {Q}'} \andalso {{Q}' \scong {Q}}}{{P} \red {Q}}
\end{mathpar}

\begin{eqnarray*}
  match_{\equiv} (\quotep{P},\quotep{Q}) & := & P \equiv Q \\
  match_{\dagger}(\quotep{P},\quotep{Q}) & := & \forall R. P|Q \red^{*} R => R \red^{*} 0 \\
  match_{K}(\quotep{P},\quotep{Q}) & := & K \mbox{ for some context } K
\end{eqnarray*}

$u?(x)P | u!\langle Q \rangle \red P\{\quotep{Q}/x\}$

%We write $\wred$ for $\red^*$, and $P\red$ if $\exists Q $ such that $ P \red Q$.
We write $P\red$ if $\exists Q $ such that $ P \red Q$ and $P\not\red$, otherwise.

\section{Replication}

As mentioned before, it is known that replication (and hence
recursion) can be implemented in a higher-order process algebra
\cite{SangiorgiWalker}. As our first example of calculation with the
machinery thus far presented we give the construction explicitly in
the {\rhoc}.

\begin{eqnarray}
	D_{x} & := & \prefix{x}{y}{(\binpar{\outputp{x}{y}}{@{y}})} \nonumber\\
	\bangp_{x}{P} & := & \binpar{{x}!\langle{\binpar{D_{x}}{P}}\rangle}{D_{x}} \nonumber
\end{eqnarray}

\begin{eqnarray}
	\bangp_{x}{P} & & \nonumber\\
	=
	& {x}!\langle{(\prefix{x}{y}{(\outputp{x}{y} | @{y})) | P}}\rangle 
	      | \prefix{x}{y}{(\outputp{x}{y} | @{y})} & \nonumber\\
	\red
	& (\outputp{x}{y} | @{y})\substn{\quotep{(\prefix{x}{y}{(@{y} | \outputp{x}{y})) | P}}}{y} & \nonumber\\
	=
	& \outputp{x}{\quotep{(\prefix{x}{y}{(\outputp{x}{y} | @{y})) | P}}}
	  | {(\prefix{x}{y}{(\outputp{x}{y} | @{y})) | P}} & \nonumber\\
	\red
	& \ldots & \nonumber\\
	\red^*
	& P | P | \ldots & \nonumber
\end{eqnarray}

Of course, this encoding, as an implementation, runs away, unfolding
$\bangp{P}$ eagerly. A lazier and more implementable replication
operator, restricted to input-guarded processes, may be obtained as follows.

\begin{eqnarray}
\bangp{\prefix{u}{v}{P}} 
	:= 
	\binpar{\lift{x}{\prefix{u}{v}{(\binpar{D(x)}{P})}}}{D(x)} \nonumber
\end{eqnarray}

\begin{remark}
  Note that the lazier definition still does not deal with summation
  or mixed summation (i.e. sums over input and output). The reader is
  invited to construct definitions of replication that deal with these
  features. 

  Further, the definitions are parameterized in a name, $x$. Can you,
  gentle reader, make a definition that eliminates this parameter and
  guarantees no accidental interaction between the replication
  machinery and the process being replicated -- i.e. no accidental
  sharing of names used by the process to get its work done and the
  name(s) used by the replication to effect copying. This latter
  revision of the definition of replication is crucial to obtaining
  the expected identity $!!P \sim !P$.
\end{remark}

\begin{remark}\label{rem:paradoxical_combinator}
  The reader familiar with the lambda calculus will have noticed the
  similarity between $D$ and the paradoxical combinator.

  [Ed. note: the existence of this seems to suggest we have to be more
  restrictive on the set of processes and names we admit if we are to
  support no-cloning.]
\end{remark}

\subsubsection{Bisimulation}

The computational dynamics gives rise to another kind of equivalence,
the equivalence of computational behavior. As previously mentioned
this is typically captured \emph{via} some form of bisimulation.

% The notion we use in this paper is weak barbed bisimulation
% \cite{milner91polyadicpi}.

The notion we use in this paper is derived from weak barbed
bisimulation \cite{milner91polyadicpi}. 

\begin{definition}
An \emph{observation relation}, $\downarrow_{\mathcal N}$, over a set
of names, $\mathcal N$, is the smallest relation satisfying the rules
below.

\infrule[Out-barb]{y \in {\mathcal N}, \; x \nameeq y}
		  {\outputp{x}{v} \downarrow_{\mathcal N} x}
\infrule[Par-barb]{\mbox{$P\downarrow_{\mathcal N} x$ or $Q\downarrow_{\mathcal N} x$}}
		  {\binpar{P}{Q} \downarrow_{\mathcal N} x}

We write $P \Downarrow_{\mathcal N} x$ if there is $Q$ such that 
$P \wred Q$ and $Q \downarrow_{\mathcal N} x$.
\end{definition}

\begin{definition}
%\label{def.bbisim}
An  ${\mathcal N}$-\emph{barbed bisimulation} over a set of names, ${\mathcal N}$, is a symmetric binary relation 
${\mathcal S}_{\mathcal N}$ between agents such that $P\rel{S}_{\mathcal N}Q$ implies:
\begin{enumerate}
\item If $P \red P'$ then $Q \wred Q'$ and $P'\rel{S}_{\mathcal N} Q'$.
\item If $P\downarrow_{\mathcal N} x$, then $Q\Downarrow_{\mathcal N} x$.
\end{enumerate}
$P$ is ${\mathcal N}$-barbed bisimilar to $Q$, written
$P \wbbisim_{\mathcal N} Q$, if $P \rel{S}_{\mathcal N} Q$ for some ${\mathcal N}$-barbed bisimulation ${\mathcal S}_{\mathcal N}$.
\end{definition}

$\mathcal{R} \subseteq \pi \times \pi$

$P \mathcal{R} Q => \forall P'. P \red P' \Rightarrow \exists Q'. Q \red Q', P' \mathcal{R} Q'$

$P \vdash x \Rightarrow Q \vdash x$

\begin{mathpar}
  \inferrule*[lab=Out-barb]{x \nameeq y}{{y}!\langle{Q}\rangle \vdash x}
  \and
  \inferrule*[lab=Par-barb]{\mbox{$P\vdash x$ or $Q\vdash x$}}{\binpar{P}{Q} \vdash x}
\end{mathpar}

\subsubsection{Contexts}

One of the principle advantages of computational calculi like the
$\pi$-calculus is a well-defined notion of context,
contextual-equivalence and a correlation between
contextual-equivalence and notions of bisimulation. The notion of
context allows the decomposition of a process into (sub-)process and
its syntactic environment, its context. Thus, a context may be
thought of as a process with a ``hole'' (written $\Box$) in it. The
application of a context $M$ to a process $P$, written $M[P]$, is
tantamount to filling the hole in $M$ with $P$. In this paper we do
not need the full weight of this theory, but do make use of the notion
of context in the proof the main theorem. 

\begin{mathpar}
  \inferrule* [lab=summation] {} {{M_{M},M_{N}} \bc \Box \;|\; x.M_{A} \;|\; M_{M}+M_{N}}
  \and
  \inferrule* [lab=agent] {} {{M_{A}} \bc (\vec{x})M_{P} \;| \; \clift{P_0,\ldots,M_{P},\ldots,P_N}}
  \and \\
  \inferrule* [lab=process] {} {{M_{P}} \bc M_{N} \;| \;P|M_{P} }
\end{mathpar} 

\begin{mathpar}
  \inferrule* [lab=sychronization] {} {M_{N} \bc \Box \;|\; x?M_{F} \;|\; x!M_{C}}
  \and
  \inferrule* [lab=abstraction] {} {{M_{F}} \bc (x)M_{P} }
  \and
  \inferrule* [lab=concretion] {} {{M_{C}} \bc \langle M_{P} \rangle }
  \and \\
  \inferrule* [lab=process] {} {{M_{P}} \bc M_{N} \;| \;P|M_{P} }
\end{mathpar}

\begin{definition}[contextual application] Given a context $M$, and
  process $P$, we define the \emph{contextual application}, $M[P] :=
  M\{P/\Box\}$. That is, the contextual application of M to P is the
  substitution of $P$ for $\Box$ in $M$.
\end{definition}

$\meaningof{-} : L \to \mathcal{P}(\pi)$

\begin{mathpar}
  \inferrule* [lab=collection] {} {\meaningof{true} = \pi, \and \meaningof{~E} = \pi \setminus \meaningof{E}, \and \meaningof{E_{1} \& E_{2}} = \meaningof{E_{1}} \cap \meaningof{E_{2}}}
\end{mathpar}

\begin{mathpar}
  \inferrule* [lab=structure] {} {\meaningof{0} = \{ P \in \pi | P \equiv 0 \}, \and \\ \meaningof{E_1 | E_2} = \{ P \in \pi | P \equiv P_{1} | P_{2}, P_{1} \in \meaningof{E_{1}}, P_{2} \in \meaningof{E_2}\} }
\end{mathpar}

\begin{mathpar}
 \inferrule* [lab=behavior] {} {\meaningof{\langle a?b \rangle E} = \{ P \in \pi | P \equiv Q | u?(y)P', \\ \and \\\\ \and \\ \;\;\; u \in \meaningof{a}, \forall z.P'\{z/y\} \in \meaningof{E\{z/b\}}\}, \and \\ \meaningof{a!E} = \{ P \in \pi | P \equiv Q | x!\langle P' \rangle, x \in \meaningof{a} P' \in \meaningof{E}\} }
\end{mathpar}

\begin{mathpar}
 \inferrule* [lab=nominal] {} {\meaningof{\quotep{E}} = \{ \quotep{P} \in \quotep{\pi} | P \in \meaningof{E} \}, \and \meaningof{\quotep{P}} = \{ \quotep{Q} \in \quotep{\pi} | P \equiv Q \} \and \\ \meaningof{@\quotep{E}} = \{ P \in \pi | P \equiv @x, x \in \meaningof{E} \}}
\end{mathpar}

\begin{eqnarray*}
  \\
  \meaningof{-} : TS \to ST
\end{eqnarray*}

\begin{eqnarray*}
  \\
  L : TS \to ST
\end{eqnarray*}

\begin{eqnarray*}
  \\
  P \models E \iff P \in \meaningof{E}
\end{eqnarray*}

\begin{eqnarray*}
  P \approx_{L} Q \iff \forall E \in L. P \models E \iff Q \models E
\end{eqnarray*}

\begin{eqnarray*}
  P \approx_{K} Q
\end{eqnarray*}

\begin{eqnarray*}
  P \approx Q
\end{eqnarray*}

$\approx_{K} = \approx = \approx_{L}$

\subsubsection{Contextual duality}

Note that contexts extend the quotation operation to a family of
operations from processes to names. Given a context, $M$, we can
define a \emph{nominal context}, $\quotep{M}$ by $\quotep{M}[P] :=
\quotep{M[P]}$. To foreshadow what is to come we observe that these
operations enjoy a duality with processes very much like the duality
between vectors and maps from vectors to scalars.

Further, because the calculus is essentially higher-order, we have a
correspondence between contexts and processes. More specifically,
given a name $x$ and a context $M$ we can construct $M^{*}_{x}$ such
that 

\begin{mathpar}
  M^{*}_{x} | \lift{x}{P} \red M[P]
\end{mathpar}

namely,

\begin{mathpar}
  M^{*}_{x} := x?(u).M[\dropn{u}]
\end{mathpar}

The dependence of $M^{*}_{x}$ on a name makes it an abstraction, 

\begin{mathpar}
  M^{*} := (x)x?(u).M[\dropn{u}]
\end{mathpar}

\subsection{Additional notation}

It will sometimes be convenient to denote the process a name
quotes. We already have the notation $x = \quotep{P}$, but it will be
convenient to introduce an alternate notation, $\procn{x}$, when we
want to emphasize the connection to the use of the name. Note that, by
virtue of name equivalence, $\quotep{\procn{x}} \nameeq x$; so, the
notation is consistent with previous definitions.

Further, because names have structure it is possible to effect
substitutions on the basis of that structure. This means we need to
upgrade our notation for substitutions, which we accomplish by
adapting comprehension notation. Thus,

\begin{mathpar}
  P\{ y / x : x \in S \}
\end{mathpar}

is interpreted to mean the process derived from P by replacing (in a
capture-avoiding manner) each occurrence of $x$ in $S$ by $y$. For example,

\begin{mathpar}
  P\{ \quotep{\procn{x}|\procn{x}} / x : x \in \freenames{P} \}
\end{mathpar}

will replace each (occurrence) of a free name $x$ in $P$ by
$\quotep{\procn{x}|\procn{x}}$.

Also, we will avail ourselves of the notation $x^{L}$ and $x^{R}$ to
denote injections of a name into disjoint copies of the name
space. There are numerous ways to accomplish this. One example can be
found in \cite{MeredithR05}. This notation overloads to vectors of
names: $\vec{x}^{\pi} := (x_{i}^{\pi} \; : \; 0 \leq i < |\vec{x}| )$ where $\pi \in \{L,R\}$.

We also use $P^{\Box} := P|\Box$.

In \cite{MeredithR05} an interpretation of the new operator is
given. It turns out that there are several possible interpretations
all enjoying the requisite algebraic properties of the operator (see
\cite{milner91polyadicpi}). We will therefore make liberal use of
$(\nu\; \vec{x})P$.

% subsection the_syntax_and_semantics_of_the_notation_system (end)   

\input{qm2pi.qmops} 

\input{qm2pi.sterngerlach} 

\input{qm2pi.metric} 

% section concurrent_process_calculi (end)

%\input{qm2pi.proofsketch}

% section proof sketch (end)

%\input{qm2pi.slviaknots} 

% section spatial logic via knots (end)

\input{qm2pi.conclusion}

% section conclusion (end)

%\input{qm2pi.dtcodes} 

% section wiring algorithm (end)

\input{qm2pi.ack} 

% section acknowledgments (end)

\newpage


\bibliographystyle{plain}   
\bibliography{../../biblios/main.bib}

\input{qm2pi.rhodetails}

\end{document}

 

% section acknowledgments (end)

\newpage


\bibliographystyle{plain}   
\bibliography{../../biblios/main.bib}

\documentclass[12pt]{llncs}
%\documentclass{jktr}

\usepackage[pdftex]{hyperref}                   
\usepackage {listings}
\usepackage {mathpartir}
\usepackage{bcprules}
%\usepackage{listings}
                       
\usepackage{graphicx} 
%\usepackage[margins=2.5cm,nohead,nofoot]{geometry}
%\usepackage{geometry}
\usepackage{amsfonts}
\usepackage{amstext}
\usepackage{latexsym}
\usepackage{amssymb}
\usepackage{color}


%\include{myPreamble}
\include{qm2pi.local} 

%\ifpdf
%\usepackage[pdftex]{graphicx}
%\else
%\usepackage{graphicx}
%\fi

 % \ifpdf
%  \usepackage{pdfsync}
%  \if


%\title{Brief Article}
%\author{David F. Snyder}
%\author{L.G. Meredith}

%\address{Dept. of Math., Texas State University--San Marcos, San Marcos, TX 78666}
       
\pagestyle{empty}


\begin{document}

\lstset{language=[Objective]Caml,frame=shadowbox}

\input{qm2pi.front}

% section front matter (end)

\input{qm2pi.intro} 
 
% section introduction (end)

% \input{qm2pi.knotations} 

% section notation (end)

\input{qm2pi.process.calculi} 

% section concurrent_process_calculi_and_spatial_logics_ (end)
    
%\input{qm2pi.knots2pi} 

%\input{qm2pi.trefoil} 

%\input{qm2pi.mainthm} 

% subsection basic_interpretation (end)

%\input{qm2pi.rho.presentation} 
\subsection{The syntax and semantics of the notation system}\label{sub:the_syntax_and_semantics_of_the_notation_system} % (fold)

We now summarize a technical presentation of the calculus that
embodies our theory of dynamics. The typical presentation of such a
calculus follows the style of giving generators and relations on
them. The grammar, below, describing term constructors, freely
generates the set of processes, $\Proc$. This set is then quotiented
by a relation known as structural congruence and it is over this set
that the notion of dynamics is expressed. This presentation is
essentially that of \cite{MeredithR05} with the addition of
polyadicity and summation. For readability we have relegated some of
the technical subtleties to an appendix.

\subsubsection{Process grammar}\label{subsub:process_grammar}

\begin{mathpar}
  \inferrule* [lab=synchronization] {} {{M} \bc \pzero \;|\; x?F \;|\; x!C }
  \and
  \inferrule* [lab=abstraction] {} {{F} \bc (x)P}
  \and
  \inferrule* [lab=concretion] {} {{C} \bc \langle Q \rangle}
  \and
  \inferrule* [lab=process] {} {{P,Q} \bc M \;| \;P|Q \;|\; @{x}}
  \and
  \inferrule* [lab=name] {} {{x} \bc \quotep{P}}
\end{mathpar} 

Note that $\vec{x}$ (resp. $\vec{P}$) denotes a vector of names
(resp. processes) of length $|\vec{x}|$ (resp. $|\vec{P}|$). We adopt
the following useful abbreviations.

\begin{mathpar}
   x?(\vec{y}).P := x.(\vec{y})P \and  x\clift{\vec{P}} := x.\clift{\vec{P}}
   \and x!(y) := \lift{x}{\dropn{y}}
   \and \Pi_{i=0}^{n-1}P_i := P_0 | \ldots | P_{n-1}
\end{mathpar}

\subsubsection{Structural congruence}

\paragraph{Free and bound names and alpha-equivalence.} At the
core of structural equivalence is alpha-equivalence which identifies
process that are the same up to a change of variable. Formally, we
recognize the distinction between free and bound names. The free names
of a process, $\freenames{P}$, may be calculated recursively as
follows:

\begin{mathpar}
\freenames{\pzero} := \emptyset
  \and \\
  \freenames{x?(y).P} := \{ x \} \cup (\freenames{P} \setminus \{ y \})
  \and 
  \freenames{x!\langle P \rangle} := \{ x \} \cup \{ P \} 
  \and \\
  \freenames{P|Q} := \freenames{P} \cup \freenames{Q}
  \and \\
  \freenames{@{x}} := \{ x \}
\end{mathpar}

$\pi$
$\quotep{\pi}$

$\freenames{-} : \pi \to \mathcal{P}(\quotep{\pi})$

\begin{eqnarray*}
  \freenames{\pzero} & := & \emptyset \\
  \freenames{x?(y).P} & := & \{ x \} \cup (\freenames{P} \setminus \{ y \}) \\
  \freenames{x!\langle P \rangle} & := & \{ x \} \cup \{ P \} \\
  \freenames{P|Q} & := & \freenames{P} \cup \freenames{Q} \\
  \freenames{\dropn{x}} & := & \{ x \}
\end{eqnarray*}

The bound names of a process, $\boundnames{P}$, are those names occurring in $P$
that are not free. For example, in $x?(y).0$, the name $x$ is free, while $y$ is bound.

\begin{mathpar}
  \inferrule* [lab=monoidal-laws] {} { P|Q \equiv Q|P \and P|0 \equiv P \and P|(Q|R) \equiv (P|Q)|R }
\end{mathpar}

\begin{mathpar}
  \inferrule* [lab=alpha-equivalence] {} { (x)P \equiv (y)P\{y/x\} \and y \not\in \freenames{P} }
\end{mathpar}

\begin{definition}
Then two processes, $P,Q$, are alpha-equivalent if $P = Q\{\vec{y}/\vec{x}\}$ for
some $\vec{x} \in \boundnames{Q},\vec{y} \in \boundnames{P}$, where $Q\{\vec{y}/\vec{x}\}$
denotes the capture-avoiding substitution of $\vec{y}$ for $\vec{x}$ in $Q$.
\end{definition}

\begin{definition}
  The {\em structural congruence} \cite{SangiorgiWalker} , $\equiv$,
  between processes is the least congruence containing
  alpha-equivalence, satisfying the abelian monoid laws
  (associativity, commutativity and $\pzero$ as identity) for parallel
  composition $|$ and for summation $+$.
\end{definition}

\subsection{Name equivalence}

We take name equivalence, written $\nameeq$, to be the smallest
equivalence relation generated by the following rules.

\begin{mathpar}
\inferrule*[lab=Quote-drop]
{ }
{ \quotep{@{x}} \nameeq x }

\inferrule*[lab=Struct-equiv]
{ P \scong Q }
{ \quotep{P} \nameeq \quotep{Q} }
\end{mathpar}

The astute reader will have noticed that the mutual recursion of names
and processes imposes a mutual recursion on alpha-equivalence and
structural equivalence via name-equivalence. Fortunately, all of this
works out pleasantly and we may calculate in the natural way, free of
concern. The reader interested in the details is referred to the
appendix \ref{appendix:rho_details}.

\subsection{Substitution}

We use $\Proc$ for the set of processes, $\QProc$ for the set of
names, and $\id{\{}\vec{y} / \vec{x} \id{\}}$ to denote partial maps,
$s : \QProc \rightarrow \QProc$. A map, $s$ lifts, uniquely, to a map
on process terms, $\widehat{s} : \Proc \rightarrow \Proc$ by the
following equations.

\begin{mathpar}
  (0) \psubstp{Q}{P} := 0 \\
  (R \juxtap S) \psubstp{Q}{P}
  :=    
  (R)\psubstp{Q}{P} \juxtap (S) \psubstp{Q}{P} \\
  (x?(y).R) \psubstp{Q}{P}    
  :=    
  (x)\substp{Q}{P} (z)\concat( (R \psubstn{z}{y}) \psubstp{Q}{P} ) \\
  (\lift{x}{R}) \psubstp{Q}{P}  
  :=
  \lift{(x)\substp{Q}{P}}{ R \psubstp{Q}{P} } \\
%   (\dropn{x})  \psubstp{Q}{P}       
%   := 
%   \left\{ 
%     \begin{array}{ccc} 
%       \dropn{\quotep{Q}} & & x \nameeq \quotep{P} \\
%       \dropn{x} & & otherwise \\
%     \end{array}
%   \right. 
  (\dropn{x})  \psubstp{Q}{P}       
  := 
  \left\{ 
    \begin{array}{ccc} 
      Q & & x \nameeq \quotep{P} \\
      \dropn{x} & & otherwise \\
    \end{array}
  \right.
\end{mathpar}
 

where

\begin{eqnarray}
  (x)\id{\{} \lpquote Q \rpquote / \lpquote P \rpquote \id{\}}            = 
  \left\{ 
    \begin{array}{ccc}
      \lpquote Q \rpquote & & x \nameeq \lpquote P \rpquote \\
      x & & otherwise \\
    \end{array}
  \right. \nonumber
\end{eqnarray}

and $z$ is chosen distinct from $\quotep{P}$, $\quotep{Q}$, the free
names in $Q$, and all the names in $R$. Our $\alpha$-equivalence will
be built in the standard way from this substitution.

\begin{remark}\label{rem:no_self_referential_names}
  One consequence of these definitions is that $\forall P. \quotep{P}
  \not\in \freenames{P}$.
\end{remark}

\subsection{ Dynamic quote: an example }

Anticipating something of what's to come, consider applying the
substitution, $\widehat{\id{\{}u / z \id{\}}}$, to the following pair
of processes, $\lift{w}{y!(z)}$ and $w[ \lpquote y!(z) \rpquote ]$.

\begin{eqnarray}
	\lift{w}{y!(z)}\widehat{\id{\{}u / z \id{\}}}
		& = &
		\lift{w}{y!(u)} \nonumber\\
	w[ \lpquote y!(z) \rpquote ] \widehat{ \id{\{}u / z \id{\}} }
		& = &
		w[ \lpquote y!(z) \rpquote ] \nonumber
\end{eqnarray}

Because the body of the process between quotes is impervious to
substitution, we get radically different answers. In fact, by
examining the first process in an input context,
e.g. $x?(z).\lift{w}{y!(z)}$, we see that the process under the lift
operator may be shaped by prefixed inputs binding a name inside it. In
this sense, the lift operator will be seen as a way to dynamically
construct processes before reifying them as names.

Finally equipped with these standard features we can present the
dynamics of the calculus.

\subsubsection{Operational semantics} 

Finally, we introduce the computational dynamics. What marks these
algebras as distinct from other more traditionally studied algebraic
structures, e.g. vector spaces or polynomial rings, is the manner in
which dynamics is captured. In traditional structures, dynamics is typically
expressed through morphisms between such structures, as in linear maps
between vector spaces or morphisms between rings. In algebras
associated with the semantics of computation, the dynamics is
expressed as part of the algebraic structure itself, through a
reduction reduction relation typically denoted by $\red$. Below, we
give a recursive presentation of this relation for the calculus used
in the encoding.

$\red \subseteq \pi \times \pi$
$\red : \pi \to \mathcal{P}(\pi)$

\begin{mathpar}
  \inferrule* [lab=Comm] { \textsf{match}( x_{src}, x_{trgt} ) } { x_{trgt}?(y)P \; | \; x_{src}!\langle {Q} \rangle \red P\{\quotep{Q}/y}\} }
  \and \\
  \inferrule* [lab=Par] {{P} \red {P}'} {{{P} | {Q}} \red {{P}' | {Q}}}
  \and
  \inferrule* [lab=Equiv]{{{P} \scong {P}'} \andalso {{P}' \red {Q}'} \andalso {{Q}' \scong {Q}}}{{P} \red {Q}}
\end{mathpar}

\begin{eqnarray*}
  match_{\equiv} (\quotep{P},\quotep{Q}) & := & P \equiv Q \\
  match_{\dagger}(\quotep{P},\quotep{Q}) & := & \forall R. P|Q \red^{*} R => R \red^{*} 0 \\
  match_{K}(\quotep{P},\quotep{Q}) & := & K \mbox{ for some context } K
\end{eqnarray*}

$u?(x)P | u!\langle Q \rangle \red P\{\quotep{Q}/x\}$

%We write $\wred$ for $\red^*$, and $P\red$ if $\exists Q $ such that $ P \red Q$.
We write $P\red$ if $\exists Q $ such that $ P \red Q$ and $P\not\red$, otherwise.

\section{Replication}

As mentioned before, it is known that replication (and hence
recursion) can be implemented in a higher-order process algebra
\cite{SangiorgiWalker}. As our first example of calculation with the
machinery thus far presented we give the construction explicitly in
the {\rhoc}.

\begin{eqnarray}
	D_{x} & := & \prefix{x}{y}{(\binpar{\outputp{x}{y}}{@{y}})} \nonumber\\
	\bangp_{x}{P} & := & \binpar{{x}!\langle{\binpar{D_{x}}{P}}\rangle}{D_{x}} \nonumber
\end{eqnarray}

\begin{eqnarray}
	\bangp_{x}{P} & & \nonumber\\
	=
	& {x}!\langle{(\prefix{x}{y}{(\outputp{x}{y} | @{y})) | P}}\rangle 
	      | \prefix{x}{y}{(\outputp{x}{y} | @{y})} & \nonumber\\
	\red
	& (\outputp{x}{y} | @{y})\substn{\quotep{(\prefix{x}{y}{(@{y} | \outputp{x}{y})) | P}}}{y} & \nonumber\\
	=
	& \outputp{x}{\quotep{(\prefix{x}{y}{(\outputp{x}{y} | @{y})) | P}}}
	  | {(\prefix{x}{y}{(\outputp{x}{y} | @{y})) | P}} & \nonumber\\
	\red
	& \ldots & \nonumber\\
	\red^*
	& P | P | \ldots & \nonumber
\end{eqnarray}

Of course, this encoding, as an implementation, runs away, unfolding
$\bangp{P}$ eagerly. A lazier and more implementable replication
operator, restricted to input-guarded processes, may be obtained as follows.

\begin{eqnarray}
\bangp{\prefix{u}{v}{P}} 
	:= 
	\binpar{\lift{x}{\prefix{u}{v}{(\binpar{D(x)}{P})}}}{D(x)} \nonumber
\end{eqnarray}

\begin{remark}
  Note that the lazier definition still does not deal with summation
  or mixed summation (i.e. sums over input and output). The reader is
  invited to construct definitions of replication that deal with these
  features. 

  Further, the definitions are parameterized in a name, $x$. Can you,
  gentle reader, make a definition that eliminates this parameter and
  guarantees no accidental interaction between the replication
  machinery and the process being replicated -- i.e. no accidental
  sharing of names used by the process to get its work done and the
  name(s) used by the replication to effect copying. This latter
  revision of the definition of replication is crucial to obtaining
  the expected identity $!!P \sim !P$.
\end{remark}

\begin{remark}\label{rem:paradoxical_combinator}
  The reader familiar with the lambda calculus will have noticed the
  similarity between $D$ and the paradoxical combinator.

  [Ed. note: the existence of this seems to suggest we have to be more
  restrictive on the set of processes and names we admit if we are to
  support no-cloning.]
\end{remark}

\subsubsection{Bisimulation}

The computational dynamics gives rise to another kind of equivalence,
the equivalence of computational behavior. As previously mentioned
this is typically captured \emph{via} some form of bisimulation.

% The notion we use in this paper is weak barbed bisimulation
% \cite{milner91polyadicpi}.

The notion we use in this paper is derived from weak barbed
bisimulation \cite{milner91polyadicpi}. 

\begin{definition}
An \emph{observation relation}, $\downarrow_{\mathcal N}$, over a set
of names, $\mathcal N$, is the smallest relation satisfying the rules
below.

\infrule[Out-barb]{y \in {\mathcal N}, \; x \nameeq y}
		  {\outputp{x}{v} \downarrow_{\mathcal N} x}
\infrule[Par-barb]{\mbox{$P\downarrow_{\mathcal N} x$ or $Q\downarrow_{\mathcal N} x$}}
		  {\binpar{P}{Q} \downarrow_{\mathcal N} x}

We write $P \Downarrow_{\mathcal N} x$ if there is $Q$ such that 
$P \wred Q$ and $Q \downarrow_{\mathcal N} x$.
\end{definition}

\begin{definition}
%\label{def.bbisim}
An  ${\mathcal N}$-\emph{barbed bisimulation} over a set of names, ${\mathcal N}$, is a symmetric binary relation 
${\mathcal S}_{\mathcal N}$ between agents such that $P\rel{S}_{\mathcal N}Q$ implies:
\begin{enumerate}
\item If $P \red P'$ then $Q \wred Q'$ and $P'\rel{S}_{\mathcal N} Q'$.
\item If $P\downarrow_{\mathcal N} x$, then $Q\Downarrow_{\mathcal N} x$.
\end{enumerate}
$P$ is ${\mathcal N}$-barbed bisimilar to $Q$, written
$P \wbbisim_{\mathcal N} Q$, if $P \rel{S}_{\mathcal N} Q$ for some ${\mathcal N}$-barbed bisimulation ${\mathcal S}_{\mathcal N}$.
\end{definition}

$\mathcal{R} \subseteq \pi \times \pi$

$P \mathcal{R} Q => \forall P'. P \red P' \Rightarrow \exists Q'. Q \red Q', P' \mathcal{R} Q'$

$P \vdash x \Rightarrow Q \vdash x$

\begin{mathpar}
  \inferrule*[lab=Out-barb]{x \nameeq y}{{y}!\langle{Q}\rangle \vdash x}
  \and
  \inferrule*[lab=Par-barb]{\mbox{$P\vdash x$ or $Q\vdash x$}}{\binpar{P}{Q} \vdash x}
\end{mathpar}

\subsubsection{Contexts}

One of the principle advantages of computational calculi like the
$\pi$-calculus is a well-defined notion of context,
contextual-equivalence and a correlation between
contextual-equivalence and notions of bisimulation. The notion of
context allows the decomposition of a process into (sub-)process and
its syntactic environment, its context. Thus, a context may be
thought of as a process with a ``hole'' (written $\Box$) in it. The
application of a context $M$ to a process $P$, written $M[P]$, is
tantamount to filling the hole in $M$ with $P$. In this paper we do
not need the full weight of this theory, but do make use of the notion
of context in the proof the main theorem. 

\begin{mathpar}
  \inferrule* [lab=summation] {} {{M_{M},M_{N}} \bc \Box \;|\; x.M_{A} \;|\; M_{M}+M_{N}}
  \and
  \inferrule* [lab=agent] {} {{M_{A}} \bc (\vec{x})M_{P} \;| \; \clift{P_0,\ldots,M_{P},\ldots,P_N}}
  \and \\
  \inferrule* [lab=process] {} {{M_{P}} \bc M_{N} \;| \;P|M_{P} }
\end{mathpar} 

\begin{mathpar}
  \inferrule* [lab=sychronization] {} {M_{N} \bc \Box \;|\; x?M_{F} \;|\; x!M_{C}}
  \and
  \inferrule* [lab=abstraction] {} {{M_{F}} \bc (x)M_{P} }
  \and
  \inferrule* [lab=concretion] {} {{M_{C}} \bc \langle M_{P} \rangle }
  \and \\
  \inferrule* [lab=process] {} {{M_{P}} \bc M_{N} \;| \;P|M_{P} }
\end{mathpar}

\begin{definition}[contextual application] Given a context $M$, and
  process $P$, we define the \emph{contextual application}, $M[P] :=
  M\{P/\Box\}$. That is, the contextual application of M to P is the
  substitution of $P$ for $\Box$ in $M$.
\end{definition}

$\meaningof{-} : L \to \mathcal{P}(\pi)$

\begin{mathpar}
  \inferrule* [lab=collection] {} {\meaningof{true} = \pi, \and \meaningof{~E} = \pi \setminus \meaningof{E}, \and \meaningof{E_{1} \& E_{2}} = \meaningof{E_{1}} \cap \meaningof{E_{2}}}
\end{mathpar}

\begin{mathpar}
  \inferrule* [lab=structure] {} {\meaningof{0} = \{ P \in \pi | P \equiv 0 \}, \and \\ \meaningof{E_1 | E_2} = \{ P \in \pi | P \equiv P_{1} | P_{2}, P_{1} \in \meaningof{E_{1}}, P_{2} \in \meaningof{E_2}\} }
\end{mathpar}

\begin{mathpar}
 \inferrule* [lab=behavior] {} {\meaningof{\langle a?b \rangle E} = \{ P \in \pi | P \equiv Q | u?(y)P', \\ \and \\\\ \and \\ \;\;\; u \in \meaningof{a}, \forall z.P'\{z/y\} \in \meaningof{E\{z/b\}}\}, \and \\ \meaningof{a!E} = \{ P \in \pi | P \equiv Q | x!\langle P' \rangle, x \in \meaningof{a} P' \in \meaningof{E}\} }
\end{mathpar}

\begin{mathpar}
 \inferrule* [lab=nominal] {} {\meaningof{\quotep{E}} = \{ \quotep{P} \in \quotep{\pi} | P \in \meaningof{E} \}, \and \meaningof{\quotep{P}} = \{ \quotep{Q} \in \quotep{\pi} | P \equiv Q \} \and \\ \meaningof{@\quotep{E}} = \{ P \in \pi | P \equiv @x, x \in \meaningof{E} \}}
\end{mathpar}

\begin{eqnarray*}
  \\
  \meaningof{-} : TS \to ST
\end{eqnarray*}

\begin{eqnarray*}
  \\
  L : TS \to ST
\end{eqnarray*}

\begin{eqnarray*}
  \\
  P \models E \iff P \in \meaningof{E}
\end{eqnarray*}

\begin{eqnarray*}
  P \approx_{L} Q \iff \forall E \in L. P \models E \iff Q \models E
\end{eqnarray*}

\begin{eqnarray*}
  P \approx_{K} Q
\end{eqnarray*}

\begin{eqnarray*}
  P \approx Q
\end{eqnarray*}

$\approx_{K} = \approx = \approx_{L}$

\subsubsection{Contextual duality}

Note that contexts extend the quotation operation to a family of
operations from processes to names. Given a context, $M$, we can
define a \emph{nominal context}, $\quotep{M}$ by $\quotep{M}[P] :=
\quotep{M[P]}$. To foreshadow what is to come we observe that these
operations enjoy a duality with processes very much like the duality
between vectors and maps from vectors to scalars.

Further, because the calculus is essentially higher-order, we have a
correspondence between contexts and processes. More specifically,
given a name $x$ and a context $M$ we can construct $M^{*}_{x}$ such
that 

\begin{mathpar}
  M^{*}_{x} | \lift{x}{P} \red M[P]
\end{mathpar}

namely,

\begin{mathpar}
  M^{*}_{x} := x?(u).M[\dropn{u}]
\end{mathpar}

The dependence of $M^{*}_{x}$ on a name makes it an abstraction, 

\begin{mathpar}
  M^{*} := (x)x?(u).M[\dropn{u}]
\end{mathpar}

\subsection{Additional notation}

It will sometimes be convenient to denote the process a name
quotes. We already have the notation $x = \quotep{P}$, but it will be
convenient to introduce an alternate notation, $\procn{x}$, when we
want to emphasize the connection to the use of the name. Note that, by
virtue of name equivalence, $\quotep{\procn{x}} \nameeq x$; so, the
notation is consistent with previous definitions.

Further, because names have structure it is possible to effect
substitutions on the basis of that structure. This means we need to
upgrade our notation for substitutions, which we accomplish by
adapting comprehension notation. Thus,

\begin{mathpar}
  P\{ y / x : x \in S \}
\end{mathpar}

is interpreted to mean the process derived from P by replacing (in a
capture-avoiding manner) each occurrence of $x$ in $S$ by $y$. For example,

\begin{mathpar}
  P\{ \quotep{\procn{x}|\procn{x}} / x : x \in \freenames{P} \}
\end{mathpar}

will replace each (occurrence) of a free name $x$ in $P$ by
$\quotep{\procn{x}|\procn{x}}$.

Also, we will avail ourselves of the notation $x^{L}$ and $x^{R}$ to
denote injections of a name into disjoint copies of the name
space. There are numerous ways to accomplish this. One example can be
found in \cite{MeredithR05}. This notation overloads to vectors of
names: $\vec{x}^{\pi} := (x_{i}^{\pi} \; : \; 0 \leq i < |\vec{x}| )$ where $\pi \in \{L,R\}$.

We also use $P^{\Box} := P|\Box$.

In \cite{MeredithR05} an interpretation of the new operator is
given. It turns out that there are several possible interpretations
all enjoying the requisite algebraic properties of the operator (see
\cite{milner91polyadicpi}). We will therefore make liberal use of
$(\nu\; \vec{x})P$.

% subsection the_syntax_and_semantics_of_the_notation_system (end)   

\input{qm2pi.qmops} 

\input{qm2pi.sterngerlach} 

\input{qm2pi.metric} 

% section concurrent_process_calculi (end)

%\input{qm2pi.proofsketch}

% section proof sketch (end)

%\input{qm2pi.slviaknots} 

% section spatial logic via knots (end)

\input{qm2pi.conclusion}

% section conclusion (end)

%\input{qm2pi.dtcodes} 

% section wiring algorithm (end)

\input{qm2pi.ack} 

% section acknowledgments (end)

\newpage


\bibliographystyle{plain}   
\bibliography{../../biblios/main.bib}

\input{qm2pi.rhodetails}

\end{document}



\end{document}

 

% section concurrent_process_calculi (end)

%\documentclass[12pt]{llncs}
%\documentclass{jktr}

\usepackage[pdftex]{hyperref}                   
\usepackage {listings}
\usepackage {mathpartir}
\usepackage{bcprules}
%\usepackage{listings}
                       
\usepackage{graphicx} 
%\usepackage[margins=2.5cm,nohead,nofoot]{geometry}
%\usepackage{geometry}
\usepackage{amsfonts}
\usepackage{amstext}
\usepackage{latexsym}
\usepackage{amssymb}
\usepackage{color}


%\include{myPreamble}
\documentclass[12pt]{llncs}
%\documentclass{jktr}

\usepackage[pdftex]{hyperref}                   
\usepackage {listings}
\usepackage {mathpartir}
\usepackage{bcprules}
%\usepackage{listings}
                       
\usepackage{graphicx} 
%\usepackage[margins=2.5cm,nohead,nofoot]{geometry}
%\usepackage{geometry}
\usepackage{amsfonts}
\usepackage{amstext}
\usepackage{latexsym}
\usepackage{amssymb}
\usepackage{color}


%\include{myPreamble}
\include{qm2pi.local} 

%\ifpdf
%\usepackage[pdftex]{graphicx}
%\else
%\usepackage{graphicx}
%\fi

 % \ifpdf
%  \usepackage{pdfsync}
%  \if


%\title{Brief Article}
%\author{David F. Snyder}
%\author{L.G. Meredith}

%\address{Dept. of Math., Texas State University--San Marcos, San Marcos, TX 78666}
       
\pagestyle{empty}


\begin{document}

\lstset{language=[Objective]Caml,frame=shadowbox}

\input{qm2pi.front}

% section front matter (end)

\input{qm2pi.intro} 
 
% section introduction (end)

% \input{qm2pi.knotations} 

% section notation (end)

\input{qm2pi.process.calculi} 

% section concurrent_process_calculi_and_spatial_logics_ (end)
    
%\input{qm2pi.knots2pi} 

%\input{qm2pi.trefoil} 

%\input{qm2pi.mainthm} 

% subsection basic_interpretation (end)

%\input{qm2pi.rho.presentation} 
\subsection{The syntax and semantics of the notation system}\label{sub:the_syntax_and_semantics_of_the_notation_system} % (fold)

We now summarize a technical presentation of the calculus that
embodies our theory of dynamics. The typical presentation of such a
calculus follows the style of giving generators and relations on
them. The grammar, below, describing term constructors, freely
generates the set of processes, $\Proc$. This set is then quotiented
by a relation known as structural congruence and it is over this set
that the notion of dynamics is expressed. This presentation is
essentially that of \cite{MeredithR05} with the addition of
polyadicity and summation. For readability we have relegated some of
the technical subtleties to an appendix.

\subsubsection{Process grammar}\label{subsub:process_grammar}

\begin{mathpar}
  \inferrule* [lab=synchronization] {} {{M} \bc \pzero \;|\; x?F \;|\; x!C }
  \and
  \inferrule* [lab=abstraction] {} {{F} \bc (x)P}
  \and
  \inferrule* [lab=concretion] {} {{C} \bc \langle Q \rangle}
  \and
  \inferrule* [lab=process] {} {{P,Q} \bc M \;| \;P|Q \;|\; @{x}}
  \and
  \inferrule* [lab=name] {} {{x} \bc \quotep{P}}
\end{mathpar} 

Note that $\vec{x}$ (resp. $\vec{P}$) denotes a vector of names
(resp. processes) of length $|\vec{x}|$ (resp. $|\vec{P}|$). We adopt
the following useful abbreviations.

\begin{mathpar}
   x?(\vec{y}).P := x.(\vec{y})P \and  x\clift{\vec{P}} := x.\clift{\vec{P}}
   \and x!(y) := \lift{x}{\dropn{y}}
   \and \Pi_{i=0}^{n-1}P_i := P_0 | \ldots | P_{n-1}
\end{mathpar}

\subsubsection{Structural congruence}

\paragraph{Free and bound names and alpha-equivalence.} At the
core of structural equivalence is alpha-equivalence which identifies
process that are the same up to a change of variable. Formally, we
recognize the distinction between free and bound names. The free names
of a process, $\freenames{P}$, may be calculated recursively as
follows:

\begin{mathpar}
\freenames{\pzero} := \emptyset
  \and \\
  \freenames{x?(y).P} := \{ x \} \cup (\freenames{P} \setminus \{ y \})
  \and 
  \freenames{x!\langle P \rangle} := \{ x \} \cup \{ P \} 
  \and \\
  \freenames{P|Q} := \freenames{P} \cup \freenames{Q}
  \and \\
  \freenames{@{x}} := \{ x \}
\end{mathpar}

$\pi$
$\quotep{\pi}$

$\freenames{-} : \pi \to \mathcal{P}(\quotep{\pi})$

\begin{eqnarray*}
  \freenames{\pzero} & := & \emptyset \\
  \freenames{x?(y).P} & := & \{ x \} \cup (\freenames{P} \setminus \{ y \}) \\
  \freenames{x!\langle P \rangle} & := & \{ x \} \cup \{ P \} \\
  \freenames{P|Q} & := & \freenames{P} \cup \freenames{Q} \\
  \freenames{\dropn{x}} & := & \{ x \}
\end{eqnarray*}

The bound names of a process, $\boundnames{P}$, are those names occurring in $P$
that are not free. For example, in $x?(y).0$, the name $x$ is free, while $y$ is bound.

\begin{mathpar}
  \inferrule* [lab=monoidal-laws] {} { P|Q \equiv Q|P \and P|0 \equiv P \and P|(Q|R) \equiv (P|Q)|R }
\end{mathpar}

\begin{mathpar}
  \inferrule* [lab=alpha-equivalence] {} { (x)P \equiv (y)P\{y/x\} \and y \not\in \freenames{P} }
\end{mathpar}

\begin{definition}
Then two processes, $P,Q$, are alpha-equivalent if $P = Q\{\vec{y}/\vec{x}\}$ for
some $\vec{x} \in \boundnames{Q},\vec{y} \in \boundnames{P}$, where $Q\{\vec{y}/\vec{x}\}$
denotes the capture-avoiding substitution of $\vec{y}$ for $\vec{x}$ in $Q$.
\end{definition}

\begin{definition}
  The {\em structural congruence} \cite{SangiorgiWalker} , $\equiv$,
  between processes is the least congruence containing
  alpha-equivalence, satisfying the abelian monoid laws
  (associativity, commutativity and $\pzero$ as identity) for parallel
  composition $|$ and for summation $+$.
\end{definition}

\subsection{Name equivalence}

We take name equivalence, written $\nameeq$, to be the smallest
equivalence relation generated by the following rules.

\begin{mathpar}
\inferrule*[lab=Quote-drop]
{ }
{ \quotep{@{x}} \nameeq x }

\inferrule*[lab=Struct-equiv]
{ P \scong Q }
{ \quotep{P} \nameeq \quotep{Q} }
\end{mathpar}

The astute reader will have noticed that the mutual recursion of names
and processes imposes a mutual recursion on alpha-equivalence and
structural equivalence via name-equivalence. Fortunately, all of this
works out pleasantly and we may calculate in the natural way, free of
concern. The reader interested in the details is referred to the
appendix \ref{appendix:rho_details}.

\subsection{Substitution}

We use $\Proc$ for the set of processes, $\QProc$ for the set of
names, and $\id{\{}\vec{y} / \vec{x} \id{\}}$ to denote partial maps,
$s : \QProc \rightarrow \QProc$. A map, $s$ lifts, uniquely, to a map
on process terms, $\widehat{s} : \Proc \rightarrow \Proc$ by the
following equations.

\begin{mathpar}
  (0) \psubstp{Q}{P} := 0 \\
  (R \juxtap S) \psubstp{Q}{P}
  :=    
  (R)\psubstp{Q}{P} \juxtap (S) \psubstp{Q}{P} \\
  (x?(y).R) \psubstp{Q}{P}    
  :=    
  (x)\substp{Q}{P} (z)\concat( (R \psubstn{z}{y}) \psubstp{Q}{P} ) \\
  (\lift{x}{R}) \psubstp{Q}{P}  
  :=
  \lift{(x)\substp{Q}{P}}{ R \psubstp{Q}{P} } \\
%   (\dropn{x})  \psubstp{Q}{P}       
%   := 
%   \left\{ 
%     \begin{array}{ccc} 
%       \dropn{\quotep{Q}} & & x \nameeq \quotep{P} \\
%       \dropn{x} & & otherwise \\
%     \end{array}
%   \right. 
  (\dropn{x})  \psubstp{Q}{P}       
  := 
  \left\{ 
    \begin{array}{ccc} 
      Q & & x \nameeq \quotep{P} \\
      \dropn{x} & & otherwise \\
    \end{array}
  \right.
\end{mathpar}
 

where

\begin{eqnarray}
  (x)\id{\{} \lpquote Q \rpquote / \lpquote P \rpquote \id{\}}            = 
  \left\{ 
    \begin{array}{ccc}
      \lpquote Q \rpquote & & x \nameeq \lpquote P \rpquote \\
      x & & otherwise \\
    \end{array}
  \right. \nonumber
\end{eqnarray}

and $z$ is chosen distinct from $\quotep{P}$, $\quotep{Q}$, the free
names in $Q$, and all the names in $R$. Our $\alpha$-equivalence will
be built in the standard way from this substitution.

\begin{remark}\label{rem:no_self_referential_names}
  One consequence of these definitions is that $\forall P. \quotep{P}
  \not\in \freenames{P}$.
\end{remark}

\subsection{ Dynamic quote: an example }

Anticipating something of what's to come, consider applying the
substitution, $\widehat{\id{\{}u / z \id{\}}}$, to the following pair
of processes, $\lift{w}{y!(z)}$ and $w[ \lpquote y!(z) \rpquote ]$.

\begin{eqnarray}
	\lift{w}{y!(z)}\widehat{\id{\{}u / z \id{\}}}
		& = &
		\lift{w}{y!(u)} \nonumber\\
	w[ \lpquote y!(z) \rpquote ] \widehat{ \id{\{}u / z \id{\}} }
		& = &
		w[ \lpquote y!(z) \rpquote ] \nonumber
\end{eqnarray}

Because the body of the process between quotes is impervious to
substitution, we get radically different answers. In fact, by
examining the first process in an input context,
e.g. $x?(z).\lift{w}{y!(z)}$, we see that the process under the lift
operator may be shaped by prefixed inputs binding a name inside it. In
this sense, the lift operator will be seen as a way to dynamically
construct processes before reifying them as names.

Finally equipped with these standard features we can present the
dynamics of the calculus.

\subsubsection{Operational semantics} 

Finally, we introduce the computational dynamics. What marks these
algebras as distinct from other more traditionally studied algebraic
structures, e.g. vector spaces or polynomial rings, is the manner in
which dynamics is captured. In traditional structures, dynamics is typically
expressed through morphisms between such structures, as in linear maps
between vector spaces or morphisms between rings. In algebras
associated with the semantics of computation, the dynamics is
expressed as part of the algebraic structure itself, through a
reduction reduction relation typically denoted by $\red$. Below, we
give a recursive presentation of this relation for the calculus used
in the encoding.

$\red \subseteq \pi \times \pi$
$\red : \pi \to \mathcal{P}(\pi)$

\begin{mathpar}
  \inferrule* [lab=Comm] { \textsf{match}( x_{src}, x_{trgt} ) } { x_{trgt}?(y)P \; | \; x_{src}!\langle {Q} \rangle \red P\{\quotep{Q}/y}\} }
  \and \\
  \inferrule* [lab=Par] {{P} \red {P}'} {{{P} | {Q}} \red {{P}' | {Q}}}
  \and
  \inferrule* [lab=Equiv]{{{P} \scong {P}'} \andalso {{P}' \red {Q}'} \andalso {{Q}' \scong {Q}}}{{P} \red {Q}}
\end{mathpar}

\begin{eqnarray*}
  match_{\equiv} (\quotep{P},\quotep{Q}) & := & P \equiv Q \\
  match_{\dagger}(\quotep{P},\quotep{Q}) & := & \forall R. P|Q \red^{*} R => R \red^{*} 0 \\
  match_{K}(\quotep{P},\quotep{Q}) & := & K \mbox{ for some context } K
\end{eqnarray*}

$u?(x)P | u!\langle Q \rangle \red P\{\quotep{Q}/x\}$

%We write $\wred$ for $\red^*$, and $P\red$ if $\exists Q $ such that $ P \red Q$.
We write $P\red$ if $\exists Q $ such that $ P \red Q$ and $P\not\red$, otherwise.

\section{Replication}

As mentioned before, it is known that replication (and hence
recursion) can be implemented in a higher-order process algebra
\cite{SangiorgiWalker}. As our first example of calculation with the
machinery thus far presented we give the construction explicitly in
the {\rhoc}.

\begin{eqnarray}
	D_{x} & := & \prefix{x}{y}{(\binpar{\outputp{x}{y}}{@{y}})} \nonumber\\
	\bangp_{x}{P} & := & \binpar{{x}!\langle{\binpar{D_{x}}{P}}\rangle}{D_{x}} \nonumber
\end{eqnarray}

\begin{eqnarray}
	\bangp_{x}{P} & & \nonumber\\
	=
	& {x}!\langle{(\prefix{x}{y}{(\outputp{x}{y} | @{y})) | P}}\rangle 
	      | \prefix{x}{y}{(\outputp{x}{y} | @{y})} & \nonumber\\
	\red
	& (\outputp{x}{y} | @{y})\substn{\quotep{(\prefix{x}{y}{(@{y} | \outputp{x}{y})) | P}}}{y} & \nonumber\\
	=
	& \outputp{x}{\quotep{(\prefix{x}{y}{(\outputp{x}{y} | @{y})) | P}}}
	  | {(\prefix{x}{y}{(\outputp{x}{y} | @{y})) | P}} & \nonumber\\
	\red
	& \ldots & \nonumber\\
	\red^*
	& P | P | \ldots & \nonumber
\end{eqnarray}

Of course, this encoding, as an implementation, runs away, unfolding
$\bangp{P}$ eagerly. A lazier and more implementable replication
operator, restricted to input-guarded processes, may be obtained as follows.

\begin{eqnarray}
\bangp{\prefix{u}{v}{P}} 
	:= 
	\binpar{\lift{x}{\prefix{u}{v}{(\binpar{D(x)}{P})}}}{D(x)} \nonumber
\end{eqnarray}

\begin{remark}
  Note that the lazier definition still does not deal with summation
  or mixed summation (i.e. sums over input and output). The reader is
  invited to construct definitions of replication that deal with these
  features. 

  Further, the definitions are parameterized in a name, $x$. Can you,
  gentle reader, make a definition that eliminates this parameter and
  guarantees no accidental interaction between the replication
  machinery and the process being replicated -- i.e. no accidental
  sharing of names used by the process to get its work done and the
  name(s) used by the replication to effect copying. This latter
  revision of the definition of replication is crucial to obtaining
  the expected identity $!!P \sim !P$.
\end{remark}

\begin{remark}\label{rem:paradoxical_combinator}
  The reader familiar with the lambda calculus will have noticed the
  similarity between $D$ and the paradoxical combinator.

  [Ed. note: the existence of this seems to suggest we have to be more
  restrictive on the set of processes and names we admit if we are to
  support no-cloning.]
\end{remark}

\subsubsection{Bisimulation}

The computational dynamics gives rise to another kind of equivalence,
the equivalence of computational behavior. As previously mentioned
this is typically captured \emph{via} some form of bisimulation.

% The notion we use in this paper is weak barbed bisimulation
% \cite{milner91polyadicpi}.

The notion we use in this paper is derived from weak barbed
bisimulation \cite{milner91polyadicpi}. 

\begin{definition}
An \emph{observation relation}, $\downarrow_{\mathcal N}$, over a set
of names, $\mathcal N$, is the smallest relation satisfying the rules
below.

\infrule[Out-barb]{y \in {\mathcal N}, \; x \nameeq y}
		  {\outputp{x}{v} \downarrow_{\mathcal N} x}
\infrule[Par-barb]{\mbox{$P\downarrow_{\mathcal N} x$ or $Q\downarrow_{\mathcal N} x$}}
		  {\binpar{P}{Q} \downarrow_{\mathcal N} x}

We write $P \Downarrow_{\mathcal N} x$ if there is $Q$ such that 
$P \wred Q$ and $Q \downarrow_{\mathcal N} x$.
\end{definition}

\begin{definition}
%\label{def.bbisim}
An  ${\mathcal N}$-\emph{barbed bisimulation} over a set of names, ${\mathcal N}$, is a symmetric binary relation 
${\mathcal S}_{\mathcal N}$ between agents such that $P\rel{S}_{\mathcal N}Q$ implies:
\begin{enumerate}
\item If $P \red P'$ then $Q \wred Q'$ and $P'\rel{S}_{\mathcal N} Q'$.
\item If $P\downarrow_{\mathcal N} x$, then $Q\Downarrow_{\mathcal N} x$.
\end{enumerate}
$P$ is ${\mathcal N}$-barbed bisimilar to $Q$, written
$P \wbbisim_{\mathcal N} Q$, if $P \rel{S}_{\mathcal N} Q$ for some ${\mathcal N}$-barbed bisimulation ${\mathcal S}_{\mathcal N}$.
\end{definition}

$\mathcal{R} \subseteq \pi \times \pi$

$P \mathcal{R} Q => \forall P'. P \red P' \Rightarrow \exists Q'. Q \red Q', P' \mathcal{R} Q'$

$P \vdash x \Rightarrow Q \vdash x$

\begin{mathpar}
  \inferrule*[lab=Out-barb]{x \nameeq y}{{y}!\langle{Q}\rangle \vdash x}
  \and
  \inferrule*[lab=Par-barb]{\mbox{$P\vdash x$ or $Q\vdash x$}}{\binpar{P}{Q} \vdash x}
\end{mathpar}

\subsubsection{Contexts}

One of the principle advantages of computational calculi like the
$\pi$-calculus is a well-defined notion of context,
contextual-equivalence and a correlation between
contextual-equivalence and notions of bisimulation. The notion of
context allows the decomposition of a process into (sub-)process and
its syntactic environment, its context. Thus, a context may be
thought of as a process with a ``hole'' (written $\Box$) in it. The
application of a context $M$ to a process $P$, written $M[P]$, is
tantamount to filling the hole in $M$ with $P$. In this paper we do
not need the full weight of this theory, but do make use of the notion
of context in the proof the main theorem. 

\begin{mathpar}
  \inferrule* [lab=summation] {} {{M_{M},M_{N}} \bc \Box \;|\; x.M_{A} \;|\; M_{M}+M_{N}}
  \and
  \inferrule* [lab=agent] {} {{M_{A}} \bc (\vec{x})M_{P} \;| \; \clift{P_0,\ldots,M_{P},\ldots,P_N}}
  \and \\
  \inferrule* [lab=process] {} {{M_{P}} \bc M_{N} \;| \;P|M_{P} }
\end{mathpar} 

\begin{mathpar}
  \inferrule* [lab=sychronization] {} {M_{N} \bc \Box \;|\; x?M_{F} \;|\; x!M_{C}}
  \and
  \inferrule* [lab=abstraction] {} {{M_{F}} \bc (x)M_{P} }
  \and
  \inferrule* [lab=concretion] {} {{M_{C}} \bc \langle M_{P} \rangle }
  \and \\
  \inferrule* [lab=process] {} {{M_{P}} \bc M_{N} \;| \;P|M_{P} }
\end{mathpar}

\begin{definition}[contextual application] Given a context $M$, and
  process $P$, we define the \emph{contextual application}, $M[P] :=
  M\{P/\Box\}$. That is, the contextual application of M to P is the
  substitution of $P$ for $\Box$ in $M$.
\end{definition}

$\meaningof{-} : L \to \mathcal{P}(\pi)$

\begin{mathpar}
  \inferrule* [lab=collection] {} {\meaningof{true} = \pi, \and \meaningof{~E} = \pi \setminus \meaningof{E}, \and \meaningof{E_{1} \& E_{2}} = \meaningof{E_{1}} \cap \meaningof{E_{2}}}
\end{mathpar}

\begin{mathpar}
  \inferrule* [lab=structure] {} {\meaningof{0} = \{ P \in \pi | P \equiv 0 \}, \and \\ \meaningof{E_1 | E_2} = \{ P \in \pi | P \equiv P_{1} | P_{2}, P_{1} \in \meaningof{E_{1}}, P_{2} \in \meaningof{E_2}\} }
\end{mathpar}

\begin{mathpar}
 \inferrule* [lab=behavior] {} {\meaningof{\langle a?b \rangle E} = \{ P \in \pi | P \equiv Q | u?(y)P', \\ \and \\\\ \and \\ \;\;\; u \in \meaningof{a}, \forall z.P'\{z/y\} \in \meaningof{E\{z/b\}}\}, \and \\ \meaningof{a!E} = \{ P \in \pi | P \equiv Q | x!\langle P' \rangle, x \in \meaningof{a} P' \in \meaningof{E}\} }
\end{mathpar}

\begin{mathpar}
 \inferrule* [lab=nominal] {} {\meaningof{\quotep{E}} = \{ \quotep{P} \in \quotep{\pi} | P \in \meaningof{E} \}, \and \meaningof{\quotep{P}} = \{ \quotep{Q} \in \quotep{\pi} | P \equiv Q \} \and \\ \meaningof{@\quotep{E}} = \{ P \in \pi | P \equiv @x, x \in \meaningof{E} \}}
\end{mathpar}

\begin{eqnarray*}
  \\
  \meaningof{-} : TS \to ST
\end{eqnarray*}

\begin{eqnarray*}
  \\
  L : TS \to ST
\end{eqnarray*}

\begin{eqnarray*}
  \\
  P \models E \iff P \in \meaningof{E}
\end{eqnarray*}

\begin{eqnarray*}
  P \approx_{L} Q \iff \forall E \in L. P \models E \iff Q \models E
\end{eqnarray*}

\begin{eqnarray*}
  P \approx_{K} Q
\end{eqnarray*}

\begin{eqnarray*}
  P \approx Q
\end{eqnarray*}

$\approx_{K} = \approx = \approx_{L}$

\subsubsection{Contextual duality}

Note that contexts extend the quotation operation to a family of
operations from processes to names. Given a context, $M$, we can
define a \emph{nominal context}, $\quotep{M}$ by $\quotep{M}[P] :=
\quotep{M[P]}$. To foreshadow what is to come we observe that these
operations enjoy a duality with processes very much like the duality
between vectors and maps from vectors to scalars.

Further, because the calculus is essentially higher-order, we have a
correspondence between contexts and processes. More specifically,
given a name $x$ and a context $M$ we can construct $M^{*}_{x}$ such
that 

\begin{mathpar}
  M^{*}_{x} | \lift{x}{P} \red M[P]
\end{mathpar}

namely,

\begin{mathpar}
  M^{*}_{x} := x?(u).M[\dropn{u}]
\end{mathpar}

The dependence of $M^{*}_{x}$ on a name makes it an abstraction, 

\begin{mathpar}
  M^{*} := (x)x?(u).M[\dropn{u}]
\end{mathpar}

\subsection{Additional notation}

It will sometimes be convenient to denote the process a name
quotes. We already have the notation $x = \quotep{P}$, but it will be
convenient to introduce an alternate notation, $\procn{x}$, when we
want to emphasize the connection to the use of the name. Note that, by
virtue of name equivalence, $\quotep{\procn{x}} \nameeq x$; so, the
notation is consistent with previous definitions.

Further, because names have structure it is possible to effect
substitutions on the basis of that structure. This means we need to
upgrade our notation for substitutions, which we accomplish by
adapting comprehension notation. Thus,

\begin{mathpar}
  P\{ y / x : x \in S \}
\end{mathpar}

is interpreted to mean the process derived from P by replacing (in a
capture-avoiding manner) each occurrence of $x$ in $S$ by $y$. For example,

\begin{mathpar}
  P\{ \quotep{\procn{x}|\procn{x}} / x : x \in \freenames{P} \}
\end{mathpar}

will replace each (occurrence) of a free name $x$ in $P$ by
$\quotep{\procn{x}|\procn{x}}$.

Also, we will avail ourselves of the notation $x^{L}$ and $x^{R}$ to
denote injections of a name into disjoint copies of the name
space. There are numerous ways to accomplish this. One example can be
found in \cite{MeredithR05}. This notation overloads to vectors of
names: $\vec{x}^{\pi} := (x_{i}^{\pi} \; : \; 0 \leq i < |\vec{x}| )$ where $\pi \in \{L,R\}$.

We also use $P^{\Box} := P|\Box$.

In \cite{MeredithR05} an interpretation of the new operator is
given. It turns out that there are several possible interpretations
all enjoying the requisite algebraic properties of the operator (see
\cite{milner91polyadicpi}). We will therefore make liberal use of
$(\nu\; \vec{x})P$.

% subsection the_syntax_and_semantics_of_the_notation_system (end)   

\input{qm2pi.qmops} 

\input{qm2pi.sterngerlach} 

\input{qm2pi.metric} 

% section concurrent_process_calculi (end)

%\input{qm2pi.proofsketch}

% section proof sketch (end)

%\input{qm2pi.slviaknots} 

% section spatial logic via knots (end)

\input{qm2pi.conclusion}

% section conclusion (end)

%\input{qm2pi.dtcodes} 

% section wiring algorithm (end)

\input{qm2pi.ack} 

% section acknowledgments (end)

\newpage


\bibliographystyle{plain}   
\bibliography{../../biblios/main.bib}

\input{qm2pi.rhodetails}

\end{document}

 

%\ifpdf
%\usepackage[pdftex]{graphicx}
%\else
%\usepackage{graphicx}
%\fi

 % \ifpdf
%  \usepackage{pdfsync}
%  \if


%\title{Brief Article}
%\author{David F. Snyder}
%\author{L.G. Meredith}

%\address{Dept. of Math., Texas State University--San Marcos, San Marcos, TX 78666}
       
\pagestyle{empty}


\begin{document}

\lstset{language=[Objective]Caml,frame=shadowbox}

\documentclass[12pt]{llncs}
%\documentclass{jktr}

\usepackage[pdftex]{hyperref}                   
\usepackage {listings}
\usepackage {mathpartir}
\usepackage{bcprules}
%\usepackage{listings}
                       
\usepackage{graphicx} 
%\usepackage[margins=2.5cm,nohead,nofoot]{geometry}
%\usepackage{geometry}
\usepackage{amsfonts}
\usepackage{amstext}
\usepackage{latexsym}
\usepackage{amssymb}
\usepackage{color}


%\include{myPreamble}
\include{qm2pi.local} 

%\ifpdf
%\usepackage[pdftex]{graphicx}
%\else
%\usepackage{graphicx}
%\fi

 % \ifpdf
%  \usepackage{pdfsync}
%  \if


%\title{Brief Article}
%\author{David F. Snyder}
%\author{L.G. Meredith}

%\address{Dept. of Math., Texas State University--San Marcos, San Marcos, TX 78666}
       
\pagestyle{empty}


\begin{document}

\lstset{language=[Objective]Caml,frame=shadowbox}

\input{qm2pi.front}

% section front matter (end)

\input{qm2pi.intro} 
 
% section introduction (end)

% \input{qm2pi.knotations} 

% section notation (end)

\input{qm2pi.process.calculi} 

% section concurrent_process_calculi_and_spatial_logics_ (end)
    
%\input{qm2pi.knots2pi} 

%\input{qm2pi.trefoil} 

%\input{qm2pi.mainthm} 

% subsection basic_interpretation (end)

%\input{qm2pi.rho.presentation} 
\subsection{The syntax and semantics of the notation system}\label{sub:the_syntax_and_semantics_of_the_notation_system} % (fold)

We now summarize a technical presentation of the calculus that
embodies our theory of dynamics. The typical presentation of such a
calculus follows the style of giving generators and relations on
them. The grammar, below, describing term constructors, freely
generates the set of processes, $\Proc$. This set is then quotiented
by a relation known as structural congruence and it is over this set
that the notion of dynamics is expressed. This presentation is
essentially that of \cite{MeredithR05} with the addition of
polyadicity and summation. For readability we have relegated some of
the technical subtleties to an appendix.

\subsubsection{Process grammar}\label{subsub:process_grammar}

\begin{mathpar}
  \inferrule* [lab=synchronization] {} {{M} \bc \pzero \;|\; x?F \;|\; x!C }
  \and
  \inferrule* [lab=abstraction] {} {{F} \bc (x)P}
  \and
  \inferrule* [lab=concretion] {} {{C} \bc \langle Q \rangle}
  \and
  \inferrule* [lab=process] {} {{P,Q} \bc M \;| \;P|Q \;|\; @{x}}
  \and
  \inferrule* [lab=name] {} {{x} \bc \quotep{P}}
\end{mathpar} 

Note that $\vec{x}$ (resp. $\vec{P}$) denotes a vector of names
(resp. processes) of length $|\vec{x}|$ (resp. $|\vec{P}|$). We adopt
the following useful abbreviations.

\begin{mathpar}
   x?(\vec{y}).P := x.(\vec{y})P \and  x\clift{\vec{P}} := x.\clift{\vec{P}}
   \and x!(y) := \lift{x}{\dropn{y}}
   \and \Pi_{i=0}^{n-1}P_i := P_0 | \ldots | P_{n-1}
\end{mathpar}

\subsubsection{Structural congruence}

\paragraph{Free and bound names and alpha-equivalence.} At the
core of structural equivalence is alpha-equivalence which identifies
process that are the same up to a change of variable. Formally, we
recognize the distinction between free and bound names. The free names
of a process, $\freenames{P}$, may be calculated recursively as
follows:

\begin{mathpar}
\freenames{\pzero} := \emptyset
  \and \\
  \freenames{x?(y).P} := \{ x \} \cup (\freenames{P} \setminus \{ y \})
  \and 
  \freenames{x!\langle P \rangle} := \{ x \} \cup \{ P \} 
  \and \\
  \freenames{P|Q} := \freenames{P} \cup \freenames{Q}
  \and \\
  \freenames{@{x}} := \{ x \}
\end{mathpar}

$\pi$
$\quotep{\pi}$

$\freenames{-} : \pi \to \mathcal{P}(\quotep{\pi})$

\begin{eqnarray*}
  \freenames{\pzero} & := & \emptyset \\
  \freenames{x?(y).P} & := & \{ x \} \cup (\freenames{P} \setminus \{ y \}) \\
  \freenames{x!\langle P \rangle} & := & \{ x \} \cup \{ P \} \\
  \freenames{P|Q} & := & \freenames{P} \cup \freenames{Q} \\
  \freenames{\dropn{x}} & := & \{ x \}
\end{eqnarray*}

The bound names of a process, $\boundnames{P}$, are those names occurring in $P$
that are not free. For example, in $x?(y).0$, the name $x$ is free, while $y$ is bound.

\begin{mathpar}
  \inferrule* [lab=monoidal-laws] {} { P|Q \equiv Q|P \and P|0 \equiv P \and P|(Q|R) \equiv (P|Q)|R }
\end{mathpar}

\begin{mathpar}
  \inferrule* [lab=alpha-equivalence] {} { (x)P \equiv (y)P\{y/x\} \and y \not\in \freenames{P} }
\end{mathpar}

\begin{definition}
Then two processes, $P,Q$, are alpha-equivalent if $P = Q\{\vec{y}/\vec{x}\}$ for
some $\vec{x} \in \boundnames{Q},\vec{y} \in \boundnames{P}$, where $Q\{\vec{y}/\vec{x}\}$
denotes the capture-avoiding substitution of $\vec{y}$ for $\vec{x}$ in $Q$.
\end{definition}

\begin{definition}
  The {\em structural congruence} \cite{SangiorgiWalker} , $\equiv$,
  between processes is the least congruence containing
  alpha-equivalence, satisfying the abelian monoid laws
  (associativity, commutativity and $\pzero$ as identity) for parallel
  composition $|$ and for summation $+$.
\end{definition}

\subsection{Name equivalence}

We take name equivalence, written $\nameeq$, to be the smallest
equivalence relation generated by the following rules.

\begin{mathpar}
\inferrule*[lab=Quote-drop]
{ }
{ \quotep{@{x}} \nameeq x }

\inferrule*[lab=Struct-equiv]
{ P \scong Q }
{ \quotep{P} \nameeq \quotep{Q} }
\end{mathpar}

The astute reader will have noticed that the mutual recursion of names
and processes imposes a mutual recursion on alpha-equivalence and
structural equivalence via name-equivalence. Fortunately, all of this
works out pleasantly and we may calculate in the natural way, free of
concern. The reader interested in the details is referred to the
appendix \ref{appendix:rho_details}.

\subsection{Substitution}

We use $\Proc$ for the set of processes, $\QProc$ for the set of
names, and $\id{\{}\vec{y} / \vec{x} \id{\}}$ to denote partial maps,
$s : \QProc \rightarrow \QProc$. A map, $s$ lifts, uniquely, to a map
on process terms, $\widehat{s} : \Proc \rightarrow \Proc$ by the
following equations.

\begin{mathpar}
  (0) \psubstp{Q}{P} := 0 \\
  (R \juxtap S) \psubstp{Q}{P}
  :=    
  (R)\psubstp{Q}{P} \juxtap (S) \psubstp{Q}{P} \\
  (x?(y).R) \psubstp{Q}{P}    
  :=    
  (x)\substp{Q}{P} (z)\concat( (R \psubstn{z}{y}) \psubstp{Q}{P} ) \\
  (\lift{x}{R}) \psubstp{Q}{P}  
  :=
  \lift{(x)\substp{Q}{P}}{ R \psubstp{Q}{P} } \\
%   (\dropn{x})  \psubstp{Q}{P}       
%   := 
%   \left\{ 
%     \begin{array}{ccc} 
%       \dropn{\quotep{Q}} & & x \nameeq \quotep{P} \\
%       \dropn{x} & & otherwise \\
%     \end{array}
%   \right. 
  (\dropn{x})  \psubstp{Q}{P}       
  := 
  \left\{ 
    \begin{array}{ccc} 
      Q & & x \nameeq \quotep{P} \\
      \dropn{x} & & otherwise \\
    \end{array}
  \right.
\end{mathpar}
 

where

\begin{eqnarray}
  (x)\id{\{} \lpquote Q \rpquote / \lpquote P \rpquote \id{\}}            = 
  \left\{ 
    \begin{array}{ccc}
      \lpquote Q \rpquote & & x \nameeq \lpquote P \rpquote \\
      x & & otherwise \\
    \end{array}
  \right. \nonumber
\end{eqnarray}

and $z$ is chosen distinct from $\quotep{P}$, $\quotep{Q}$, the free
names in $Q$, and all the names in $R$. Our $\alpha$-equivalence will
be built in the standard way from this substitution.

\begin{remark}\label{rem:no_self_referential_names}
  One consequence of these definitions is that $\forall P. \quotep{P}
  \not\in \freenames{P}$.
\end{remark}

\subsection{ Dynamic quote: an example }

Anticipating something of what's to come, consider applying the
substitution, $\widehat{\id{\{}u / z \id{\}}}$, to the following pair
of processes, $\lift{w}{y!(z)}$ and $w[ \lpquote y!(z) \rpquote ]$.

\begin{eqnarray}
	\lift{w}{y!(z)}\widehat{\id{\{}u / z \id{\}}}
		& = &
		\lift{w}{y!(u)} \nonumber\\
	w[ \lpquote y!(z) \rpquote ] \widehat{ \id{\{}u / z \id{\}} }
		& = &
		w[ \lpquote y!(z) \rpquote ] \nonumber
\end{eqnarray}

Because the body of the process between quotes is impervious to
substitution, we get radically different answers. In fact, by
examining the first process in an input context,
e.g. $x?(z).\lift{w}{y!(z)}$, we see that the process under the lift
operator may be shaped by prefixed inputs binding a name inside it. In
this sense, the lift operator will be seen as a way to dynamically
construct processes before reifying them as names.

Finally equipped with these standard features we can present the
dynamics of the calculus.

\subsubsection{Operational semantics} 

Finally, we introduce the computational dynamics. What marks these
algebras as distinct from other more traditionally studied algebraic
structures, e.g. vector spaces or polynomial rings, is the manner in
which dynamics is captured. In traditional structures, dynamics is typically
expressed through morphisms between such structures, as in linear maps
between vector spaces or morphisms between rings. In algebras
associated with the semantics of computation, the dynamics is
expressed as part of the algebraic structure itself, through a
reduction reduction relation typically denoted by $\red$. Below, we
give a recursive presentation of this relation for the calculus used
in the encoding.

$\red \subseteq \pi \times \pi$
$\red : \pi \to \mathcal{P}(\pi)$

\begin{mathpar}
  \inferrule* [lab=Comm] { \textsf{match}( x_{src}, x_{trgt} ) } { x_{trgt}?(y)P \; | \; x_{src}!\langle {Q} \rangle \red P\{\quotep{Q}/y}\} }
  \and \\
  \inferrule* [lab=Par] {{P} \red {P}'} {{{P} | {Q}} \red {{P}' | {Q}}}
  \and
  \inferrule* [lab=Equiv]{{{P} \scong {P}'} \andalso {{P}' \red {Q}'} \andalso {{Q}' \scong {Q}}}{{P} \red {Q}}
\end{mathpar}

\begin{eqnarray*}
  match_{\equiv} (\quotep{P},\quotep{Q}) & := & P \equiv Q \\
  match_{\dagger}(\quotep{P},\quotep{Q}) & := & \forall R. P|Q \red^{*} R => R \red^{*} 0 \\
  match_{K}(\quotep{P},\quotep{Q}) & := & K \mbox{ for some context } K
\end{eqnarray*}

$u?(x)P | u!\langle Q \rangle \red P\{\quotep{Q}/x\}$

%We write $\wred$ for $\red^*$, and $P\red$ if $\exists Q $ such that $ P \red Q$.
We write $P\red$ if $\exists Q $ such that $ P \red Q$ and $P\not\red$, otherwise.

\section{Replication}

As mentioned before, it is known that replication (and hence
recursion) can be implemented in a higher-order process algebra
\cite{SangiorgiWalker}. As our first example of calculation with the
machinery thus far presented we give the construction explicitly in
the {\rhoc}.

\begin{eqnarray}
	D_{x} & := & \prefix{x}{y}{(\binpar{\outputp{x}{y}}{@{y}})} \nonumber\\
	\bangp_{x}{P} & := & \binpar{{x}!\langle{\binpar{D_{x}}{P}}\rangle}{D_{x}} \nonumber
\end{eqnarray}

\begin{eqnarray}
	\bangp_{x}{P} & & \nonumber\\
	=
	& {x}!\langle{(\prefix{x}{y}{(\outputp{x}{y} | @{y})) | P}}\rangle 
	      | \prefix{x}{y}{(\outputp{x}{y} | @{y})} & \nonumber\\
	\red
	& (\outputp{x}{y} | @{y})\substn{\quotep{(\prefix{x}{y}{(@{y} | \outputp{x}{y})) | P}}}{y} & \nonumber\\
	=
	& \outputp{x}{\quotep{(\prefix{x}{y}{(\outputp{x}{y} | @{y})) | P}}}
	  | {(\prefix{x}{y}{(\outputp{x}{y} | @{y})) | P}} & \nonumber\\
	\red
	& \ldots & \nonumber\\
	\red^*
	& P | P | \ldots & \nonumber
\end{eqnarray}

Of course, this encoding, as an implementation, runs away, unfolding
$\bangp{P}$ eagerly. A lazier and more implementable replication
operator, restricted to input-guarded processes, may be obtained as follows.

\begin{eqnarray}
\bangp{\prefix{u}{v}{P}} 
	:= 
	\binpar{\lift{x}{\prefix{u}{v}{(\binpar{D(x)}{P})}}}{D(x)} \nonumber
\end{eqnarray}

\begin{remark}
  Note that the lazier definition still does not deal with summation
  or mixed summation (i.e. sums over input and output). The reader is
  invited to construct definitions of replication that deal with these
  features. 

  Further, the definitions are parameterized in a name, $x$. Can you,
  gentle reader, make a definition that eliminates this parameter and
  guarantees no accidental interaction between the replication
  machinery and the process being replicated -- i.e. no accidental
  sharing of names used by the process to get its work done and the
  name(s) used by the replication to effect copying. This latter
  revision of the definition of replication is crucial to obtaining
  the expected identity $!!P \sim !P$.
\end{remark}

\begin{remark}\label{rem:paradoxical_combinator}
  The reader familiar with the lambda calculus will have noticed the
  similarity between $D$ and the paradoxical combinator.

  [Ed. note: the existence of this seems to suggest we have to be more
  restrictive on the set of processes and names we admit if we are to
  support no-cloning.]
\end{remark}

\subsubsection{Bisimulation}

The computational dynamics gives rise to another kind of equivalence,
the equivalence of computational behavior. As previously mentioned
this is typically captured \emph{via} some form of bisimulation.

% The notion we use in this paper is weak barbed bisimulation
% \cite{milner91polyadicpi}.

The notion we use in this paper is derived from weak barbed
bisimulation \cite{milner91polyadicpi}. 

\begin{definition}
An \emph{observation relation}, $\downarrow_{\mathcal N}$, over a set
of names, $\mathcal N$, is the smallest relation satisfying the rules
below.

\infrule[Out-barb]{y \in {\mathcal N}, \; x \nameeq y}
		  {\outputp{x}{v} \downarrow_{\mathcal N} x}
\infrule[Par-barb]{\mbox{$P\downarrow_{\mathcal N} x$ or $Q\downarrow_{\mathcal N} x$}}
		  {\binpar{P}{Q} \downarrow_{\mathcal N} x}

We write $P \Downarrow_{\mathcal N} x$ if there is $Q$ such that 
$P \wred Q$ and $Q \downarrow_{\mathcal N} x$.
\end{definition}

\begin{definition}
%\label{def.bbisim}
An  ${\mathcal N}$-\emph{barbed bisimulation} over a set of names, ${\mathcal N}$, is a symmetric binary relation 
${\mathcal S}_{\mathcal N}$ between agents such that $P\rel{S}_{\mathcal N}Q$ implies:
\begin{enumerate}
\item If $P \red P'$ then $Q \wred Q'$ and $P'\rel{S}_{\mathcal N} Q'$.
\item If $P\downarrow_{\mathcal N} x$, then $Q\Downarrow_{\mathcal N} x$.
\end{enumerate}
$P$ is ${\mathcal N}$-barbed bisimilar to $Q$, written
$P \wbbisim_{\mathcal N} Q$, if $P \rel{S}_{\mathcal N} Q$ for some ${\mathcal N}$-barbed bisimulation ${\mathcal S}_{\mathcal N}$.
\end{definition}

$\mathcal{R} \subseteq \pi \times \pi$

$P \mathcal{R} Q => \forall P'. P \red P' \Rightarrow \exists Q'. Q \red Q', P' \mathcal{R} Q'$

$P \vdash x \Rightarrow Q \vdash x$

\begin{mathpar}
  \inferrule*[lab=Out-barb]{x \nameeq y}{{y}!\langle{Q}\rangle \vdash x}
  \and
  \inferrule*[lab=Par-barb]{\mbox{$P\vdash x$ or $Q\vdash x$}}{\binpar{P}{Q} \vdash x}
\end{mathpar}

\subsubsection{Contexts}

One of the principle advantages of computational calculi like the
$\pi$-calculus is a well-defined notion of context,
contextual-equivalence and a correlation between
contextual-equivalence and notions of bisimulation. The notion of
context allows the decomposition of a process into (sub-)process and
its syntactic environment, its context. Thus, a context may be
thought of as a process with a ``hole'' (written $\Box$) in it. The
application of a context $M$ to a process $P$, written $M[P]$, is
tantamount to filling the hole in $M$ with $P$. In this paper we do
not need the full weight of this theory, but do make use of the notion
of context in the proof the main theorem. 

\begin{mathpar}
  \inferrule* [lab=summation] {} {{M_{M},M_{N}} \bc \Box \;|\; x.M_{A} \;|\; M_{M}+M_{N}}
  \and
  \inferrule* [lab=agent] {} {{M_{A}} \bc (\vec{x})M_{P} \;| \; \clift{P_0,\ldots,M_{P},\ldots,P_N}}
  \and \\
  \inferrule* [lab=process] {} {{M_{P}} \bc M_{N} \;| \;P|M_{P} }
\end{mathpar} 

\begin{mathpar}
  \inferrule* [lab=sychronization] {} {M_{N} \bc \Box \;|\; x?M_{F} \;|\; x!M_{C}}
  \and
  \inferrule* [lab=abstraction] {} {{M_{F}} \bc (x)M_{P} }
  \and
  \inferrule* [lab=concretion] {} {{M_{C}} \bc \langle M_{P} \rangle }
  \and \\
  \inferrule* [lab=process] {} {{M_{P}} \bc M_{N} \;| \;P|M_{P} }
\end{mathpar}

\begin{definition}[contextual application] Given a context $M$, and
  process $P$, we define the \emph{contextual application}, $M[P] :=
  M\{P/\Box\}$. That is, the contextual application of M to P is the
  substitution of $P$ for $\Box$ in $M$.
\end{definition}

$\meaningof{-} : L \to \mathcal{P}(\pi)$

\begin{mathpar}
  \inferrule* [lab=collection] {} {\meaningof{true} = \pi, \and \meaningof{~E} = \pi \setminus \meaningof{E}, \and \meaningof{E_{1} \& E_{2}} = \meaningof{E_{1}} \cap \meaningof{E_{2}}}
\end{mathpar}

\begin{mathpar}
  \inferrule* [lab=structure] {} {\meaningof{0} = \{ P \in \pi | P \equiv 0 \}, \and \\ \meaningof{E_1 | E_2} = \{ P \in \pi | P \equiv P_{1} | P_{2}, P_{1} \in \meaningof{E_{1}}, P_{2} \in \meaningof{E_2}\} }
\end{mathpar}

\begin{mathpar}
 \inferrule* [lab=behavior] {} {\meaningof{\langle a?b \rangle E} = \{ P \in \pi | P \equiv Q | u?(y)P', \\ \and \\\\ \and \\ \;\;\; u \in \meaningof{a}, \forall z.P'\{z/y\} \in \meaningof{E\{z/b\}}\}, \and \\ \meaningof{a!E} = \{ P \in \pi | P \equiv Q | x!\langle P' \rangle, x \in \meaningof{a} P' \in \meaningof{E}\} }
\end{mathpar}

\begin{mathpar}
 \inferrule* [lab=nominal] {} {\meaningof{\quotep{E}} = \{ \quotep{P} \in \quotep{\pi} | P \in \meaningof{E} \}, \and \meaningof{\quotep{P}} = \{ \quotep{Q} \in \quotep{\pi} | P \equiv Q \} \and \\ \meaningof{@\quotep{E}} = \{ P \in \pi | P \equiv @x, x \in \meaningof{E} \}}
\end{mathpar}

\begin{eqnarray*}
  \\
  \meaningof{-} : TS \to ST
\end{eqnarray*}

\begin{eqnarray*}
  \\
  L : TS \to ST
\end{eqnarray*}

\begin{eqnarray*}
  \\
  P \models E \iff P \in \meaningof{E}
\end{eqnarray*}

\begin{eqnarray*}
  P \approx_{L} Q \iff \forall E \in L. P \models E \iff Q \models E
\end{eqnarray*}

\begin{eqnarray*}
  P \approx_{K} Q
\end{eqnarray*}

\begin{eqnarray*}
  P \approx Q
\end{eqnarray*}

$\approx_{K} = \approx = \approx_{L}$

\subsubsection{Contextual duality}

Note that contexts extend the quotation operation to a family of
operations from processes to names. Given a context, $M$, we can
define a \emph{nominal context}, $\quotep{M}$ by $\quotep{M}[P] :=
\quotep{M[P]}$. To foreshadow what is to come we observe that these
operations enjoy a duality with processes very much like the duality
between vectors and maps from vectors to scalars.

Further, because the calculus is essentially higher-order, we have a
correspondence between contexts and processes. More specifically,
given a name $x$ and a context $M$ we can construct $M^{*}_{x}$ such
that 

\begin{mathpar}
  M^{*}_{x} | \lift{x}{P} \red M[P]
\end{mathpar}

namely,

\begin{mathpar}
  M^{*}_{x} := x?(u).M[\dropn{u}]
\end{mathpar}

The dependence of $M^{*}_{x}$ on a name makes it an abstraction, 

\begin{mathpar}
  M^{*} := (x)x?(u).M[\dropn{u}]
\end{mathpar}

\subsection{Additional notation}

It will sometimes be convenient to denote the process a name
quotes. We already have the notation $x = \quotep{P}$, but it will be
convenient to introduce an alternate notation, $\procn{x}$, when we
want to emphasize the connection to the use of the name. Note that, by
virtue of name equivalence, $\quotep{\procn{x}} \nameeq x$; so, the
notation is consistent with previous definitions.

Further, because names have structure it is possible to effect
substitutions on the basis of that structure. This means we need to
upgrade our notation for substitutions, which we accomplish by
adapting comprehension notation. Thus,

\begin{mathpar}
  P\{ y / x : x \in S \}
\end{mathpar}

is interpreted to mean the process derived from P by replacing (in a
capture-avoiding manner) each occurrence of $x$ in $S$ by $y$. For example,

\begin{mathpar}
  P\{ \quotep{\procn{x}|\procn{x}} / x : x \in \freenames{P} \}
\end{mathpar}

will replace each (occurrence) of a free name $x$ in $P$ by
$\quotep{\procn{x}|\procn{x}}$.

Also, we will avail ourselves of the notation $x^{L}$ and $x^{R}$ to
denote injections of a name into disjoint copies of the name
space. There are numerous ways to accomplish this. One example can be
found in \cite{MeredithR05}. This notation overloads to vectors of
names: $\vec{x}^{\pi} := (x_{i}^{\pi} \; : \; 0 \leq i < |\vec{x}| )$ where $\pi \in \{L,R\}$.

We also use $P^{\Box} := P|\Box$.

In \cite{MeredithR05} an interpretation of the new operator is
given. It turns out that there are several possible interpretations
all enjoying the requisite algebraic properties of the operator (see
\cite{milner91polyadicpi}). We will therefore make liberal use of
$(\nu\; \vec{x})P$.

% subsection the_syntax_and_semantics_of_the_notation_system (end)   

\input{qm2pi.qmops} 

\input{qm2pi.sterngerlach} 

\input{qm2pi.metric} 

% section concurrent_process_calculi (end)

%\input{qm2pi.proofsketch}

% section proof sketch (end)

%\input{qm2pi.slviaknots} 

% section spatial logic via knots (end)

\input{qm2pi.conclusion}

% section conclusion (end)

%\input{qm2pi.dtcodes} 

% section wiring algorithm (end)

\input{qm2pi.ack} 

% section acknowledgments (end)

\newpage


\bibliographystyle{plain}   
\bibliography{../../biblios/main.bib}

\input{qm2pi.rhodetails}

\end{document}



% section front matter (end)

\section{Introduction}\label{sec:introduction} % (fold)
In this draft of the material i am going to have to dispense with the
usual writing conventions adopted in papers on these topics. i'm going
to have adopt whatever tone i need at the time i'm writing up the
calculations. Sometimes this may be very conversational; others it may
be the barest mathematical grunts; others still it may be that i have
lifted text from one of my other papers because the exposition of some
point was better said there. i hope that my readers are not unduly put
out by this decision. i'm not doing this to flout convention or be
rebellious. i find these calculations very technically challenging. To
keep everything going technically, something has to give; i have to
let go of some cognitive burden. So, the academic writing style --
with all of its trade-offs in terms of facilitating technical
communication -- is what i'm letting go of. Perhaps subsequent drafts
can be tightened and polished, but for now, i'm going to speak as if
we were sitting together in a coffee shop with a laptop, wifi and a
pad of paper and a pencil.

So, here's what i have to say. We -- you and i, comfortably ensconced
in our coffee shop and well-equipped with our tools -- can realize and
carry out the calculations of quantum mechanics over a very different
formal theory of dynamics, a formal theory of dynamics that
corresponds to a theory of concurrent computation with
\emph{reflection}. It has the advantage that the underlying theory is
already `quantized', but supports analogues all of the continuuous
operations. Strikingly, this underlying theory has recently been
connected with a notion of metric that we can show, by calculating
together, coincides with the metric induced by the inner product.

There are a lot of reasons why you might be interested in seeing
calculations of this form. Here's why i'm interested. For the past
several centuries there has been no competitor to the ``Newtonian''
account of dynamics. As a result the predominant share of accounts of
dynamical systems and situations have had to be formulated in terms of
the Newtonian machinery. i view this as an intellectually dangerous
position to occupy. Everything, despite it's intrinsic shape, turns
into a nail to be hit with this hammer. Recently, however, the theory
of computation has matured to the point where we have candidates for
theories of dynamics that offer very different perspective on
reasoning about dynamical systems and situations. Testing these
candidates against very successful accounts of dynamical situations,
like quantum mechanics, is going to give us some sense of how mature
they are and some measure of the quality of these accounts of
dynamics.

\subsection{Summary of contributions and outline of paper}

So, we're going to develop an interpretation of the operations of
quantum mechanics normally interpreted by Hilbert spaces and
operators. We're going to do this over a theory of computation. Note
that this is very different than the usual quantum computation program
which develops notions of computation over quantum mechanics. Rather,
we are developing a story that aligns with Wheeler's slogan: It from
Bit. To do this we will first provide an account of the theory of
computation at play here. Then we will dive into a calculation-driven
interpretation of the operations of quantum mechanics.

The reason we take this approach is that -- until very recently --
there hasn't been an axiomatic account of quantum mechanics. As a
result there has been no sharp delineation of the mathematical theory
supporting interpretation of the physical theory and the physical
theory, itself. So, ambient features of the maths are free to be
exploited (or supressed) without a real accounting of their physical
relevance. There is no sharp statement ``here's the physical theory''
qua \emph{theory} and ``here's the mathematical interpretation''
enabling a judgment of how faithful the interpretation is -- apart
from experimental observation. When there is an axiomatic account we
can judge how well a given mathematical formalism supports an
interpretation of the axioms, independent of
experimentation. Likewise, we can judge how well we have captured our
physical evidence and experience with our axiomatics, independent of
any specific mathematical implementation, with accidental detail that
may or may not have physical significance. 

In lieu of a fully fleshed out and vetted axiomatic account of quantum
mechanics, interpreting the operational notions in service of modeling
physical systems will have to suffice. In other words, we are not in
the business of providing a model of Hilbert spaces and operators. We
are in the business of providing a model of quantum mechanics because
we are motivated by testing our notions of dynamics against physical
theory; and, the predictive calculations of the physical theory must
serve as the best formulation -- shy of a fully fleshed out axiomatic
account -- of the physical theory itself (as they have for scientific
theories since time immemorial). Put another way, despite a
whole-hearted commitment to an It-from-Bit ontology, we are firmly
aligned with the shut-up-and-calculate camp as the best way to obtain
results either from the physical perspective or as a quality assurance
measure of our fledgling theory of dynamics.

In detail, we present a reflective process calculus. Then we develop
intuitive correspondences between the notions available in this
calculus and the usual physical notions supporting quantum mechanical
calculations. Thus, 

\begin{table}[htp]
  \center{
    \fbox{
      \begin{tabular}{c|c}
        quantum mechanics & process calculus \\
        \hline
        scalar & name \\
        state vector & process \\
        dual & contextual duals \\
        matrix & formal sums of process-context-dual pairs \\
        orthogonality & process annihilation \\
        inner product & execution-formula + quoting
      \end{tabular}
    }
  }
  \caption{QM - process calculi correspondences}
\end{table}

Then we tighten up these intuitions to operational definitions. We
employ the Dirac notation as the best proxy we can find for an
abstract syntax of the quantum mechanical notions. The definitions we
develop put us in contact with equational constraints coming from the
theory that we demonstrate the definitions and calculations satisfy.

This puts us in a position to shut up and calculate for the
Stern-Gerlach experimental set up, showing how these predictive
calculations become calculations on processes in our theory of a
reflective process calculus.

Penultimately, we demonstrate that the notion of metric coming from
the inner product coincides with the notion of metric available from
the theory of bisimulation. This demonstration gives us the right to
think of space as arising from behavior. Finally, we consider where we
might go from the new vantage point we have obtained.

% section introduction (end) 
 
% section introduction (end)

% \documentclass[12pt]{llncs}
%\documentclass{jktr}

\usepackage[pdftex]{hyperref}                   
\usepackage {listings}
\usepackage {mathpartir}
\usepackage{bcprules}
%\usepackage{listings}
                       
\usepackage{graphicx} 
%\usepackage[margins=2.5cm,nohead,nofoot]{geometry}
%\usepackage{geometry}
\usepackage{amsfonts}
\usepackage{amstext}
\usepackage{latexsym}
\usepackage{amssymb}
\usepackage{color}


%\include{myPreamble}
\include{qm2pi.local} 

%\ifpdf
%\usepackage[pdftex]{graphicx}
%\else
%\usepackage{graphicx}
%\fi

 % \ifpdf
%  \usepackage{pdfsync}
%  \if


%\title{Brief Article}
%\author{David F. Snyder}
%\author{L.G. Meredith}

%\address{Dept. of Math., Texas State University--San Marcos, San Marcos, TX 78666}
       
\pagestyle{empty}


\begin{document}

\lstset{language=[Objective]Caml,frame=shadowbox}

\input{qm2pi.front}

% section front matter (end)

\input{qm2pi.intro} 
 
% section introduction (end)

% \input{qm2pi.knotations} 

% section notation (end)

\input{qm2pi.process.calculi} 

% section concurrent_process_calculi_and_spatial_logics_ (end)
    
%\input{qm2pi.knots2pi} 

%\input{qm2pi.trefoil} 

%\input{qm2pi.mainthm} 

% subsection basic_interpretation (end)

%\input{qm2pi.rho.presentation} 
\subsection{The syntax and semantics of the notation system}\label{sub:the_syntax_and_semantics_of_the_notation_system} % (fold)

We now summarize a technical presentation of the calculus that
embodies our theory of dynamics. The typical presentation of such a
calculus follows the style of giving generators and relations on
them. The grammar, below, describing term constructors, freely
generates the set of processes, $\Proc$. This set is then quotiented
by a relation known as structural congruence and it is over this set
that the notion of dynamics is expressed. This presentation is
essentially that of \cite{MeredithR05} with the addition of
polyadicity and summation. For readability we have relegated some of
the technical subtleties to an appendix.

\subsubsection{Process grammar}\label{subsub:process_grammar}

\begin{mathpar}
  \inferrule* [lab=synchronization] {} {{M} \bc \pzero \;|\; x?F \;|\; x!C }
  \and
  \inferrule* [lab=abstraction] {} {{F} \bc (x)P}
  \and
  \inferrule* [lab=concretion] {} {{C} \bc \langle Q \rangle}
  \and
  \inferrule* [lab=process] {} {{P,Q} \bc M \;| \;P|Q \;|\; @{x}}
  \and
  \inferrule* [lab=name] {} {{x} \bc \quotep{P}}
\end{mathpar} 

Note that $\vec{x}$ (resp. $\vec{P}$) denotes a vector of names
(resp. processes) of length $|\vec{x}|$ (resp. $|\vec{P}|$). We adopt
the following useful abbreviations.

\begin{mathpar}
   x?(\vec{y}).P := x.(\vec{y})P \and  x\clift{\vec{P}} := x.\clift{\vec{P}}
   \and x!(y) := \lift{x}{\dropn{y}}
   \and \Pi_{i=0}^{n-1}P_i := P_0 | \ldots | P_{n-1}
\end{mathpar}

\subsubsection{Structural congruence}

\paragraph{Free and bound names and alpha-equivalence.} At the
core of structural equivalence is alpha-equivalence which identifies
process that are the same up to a change of variable. Formally, we
recognize the distinction between free and bound names. The free names
of a process, $\freenames{P}$, may be calculated recursively as
follows:

\begin{mathpar}
\freenames{\pzero} := \emptyset
  \and \\
  \freenames{x?(y).P} := \{ x \} \cup (\freenames{P} \setminus \{ y \})
  \and 
  \freenames{x!\langle P \rangle} := \{ x \} \cup \{ P \} 
  \and \\
  \freenames{P|Q} := \freenames{P} \cup \freenames{Q}
  \and \\
  \freenames{@{x}} := \{ x \}
\end{mathpar}

$\pi$
$\quotep{\pi}$

$\freenames{-} : \pi \to \mathcal{P}(\quotep{\pi})$

\begin{eqnarray*}
  \freenames{\pzero} & := & \emptyset \\
  \freenames{x?(y).P} & := & \{ x \} \cup (\freenames{P} \setminus \{ y \}) \\
  \freenames{x!\langle P \rangle} & := & \{ x \} \cup \{ P \} \\
  \freenames{P|Q} & := & \freenames{P} \cup \freenames{Q} \\
  \freenames{\dropn{x}} & := & \{ x \}
\end{eqnarray*}

The bound names of a process, $\boundnames{P}$, are those names occurring in $P$
that are not free. For example, in $x?(y).0$, the name $x$ is free, while $y$ is bound.

\begin{mathpar}
  \inferrule* [lab=monoidal-laws] {} { P|Q \equiv Q|P \and P|0 \equiv P \and P|(Q|R) \equiv (P|Q)|R }
\end{mathpar}

\begin{mathpar}
  \inferrule* [lab=alpha-equivalence] {} { (x)P \equiv (y)P\{y/x\} \and y \not\in \freenames{P} }
\end{mathpar}

\begin{definition}
Then two processes, $P,Q$, are alpha-equivalent if $P = Q\{\vec{y}/\vec{x}\}$ for
some $\vec{x} \in \boundnames{Q},\vec{y} \in \boundnames{P}$, where $Q\{\vec{y}/\vec{x}\}$
denotes the capture-avoiding substitution of $\vec{y}$ for $\vec{x}$ in $Q$.
\end{definition}

\begin{definition}
  The {\em structural congruence} \cite{SangiorgiWalker} , $\equiv$,
  between processes is the least congruence containing
  alpha-equivalence, satisfying the abelian monoid laws
  (associativity, commutativity and $\pzero$ as identity) for parallel
  composition $|$ and for summation $+$.
\end{definition}

\subsection{Name equivalence}

We take name equivalence, written $\nameeq$, to be the smallest
equivalence relation generated by the following rules.

\begin{mathpar}
\inferrule*[lab=Quote-drop]
{ }
{ \quotep{@{x}} \nameeq x }

\inferrule*[lab=Struct-equiv]
{ P \scong Q }
{ \quotep{P} \nameeq \quotep{Q} }
\end{mathpar}

The astute reader will have noticed that the mutual recursion of names
and processes imposes a mutual recursion on alpha-equivalence and
structural equivalence via name-equivalence. Fortunately, all of this
works out pleasantly and we may calculate in the natural way, free of
concern. The reader interested in the details is referred to the
appendix \ref{appendix:rho_details}.

\subsection{Substitution}

We use $\Proc$ for the set of processes, $\QProc$ for the set of
names, and $\id{\{}\vec{y} / \vec{x} \id{\}}$ to denote partial maps,
$s : \QProc \rightarrow \QProc$. A map, $s$ lifts, uniquely, to a map
on process terms, $\widehat{s} : \Proc \rightarrow \Proc$ by the
following equations.

\begin{mathpar}
  (0) \psubstp{Q}{P} := 0 \\
  (R \juxtap S) \psubstp{Q}{P}
  :=    
  (R)\psubstp{Q}{P} \juxtap (S) \psubstp{Q}{P} \\
  (x?(y).R) \psubstp{Q}{P}    
  :=    
  (x)\substp{Q}{P} (z)\concat( (R \psubstn{z}{y}) \psubstp{Q}{P} ) \\
  (\lift{x}{R}) \psubstp{Q}{P}  
  :=
  \lift{(x)\substp{Q}{P}}{ R \psubstp{Q}{P} } \\
%   (\dropn{x})  \psubstp{Q}{P}       
%   := 
%   \left\{ 
%     \begin{array}{ccc} 
%       \dropn{\quotep{Q}} & & x \nameeq \quotep{P} \\
%       \dropn{x} & & otherwise \\
%     \end{array}
%   \right. 
  (\dropn{x})  \psubstp{Q}{P}       
  := 
  \left\{ 
    \begin{array}{ccc} 
      Q & & x \nameeq \quotep{P} \\
      \dropn{x} & & otherwise \\
    \end{array}
  \right.
\end{mathpar}
 

where

\begin{eqnarray}
  (x)\id{\{} \lpquote Q \rpquote / \lpquote P \rpquote \id{\}}            = 
  \left\{ 
    \begin{array}{ccc}
      \lpquote Q \rpquote & & x \nameeq \lpquote P \rpquote \\
      x & & otherwise \\
    \end{array}
  \right. \nonumber
\end{eqnarray}

and $z$ is chosen distinct from $\quotep{P}$, $\quotep{Q}$, the free
names in $Q$, and all the names in $R$. Our $\alpha$-equivalence will
be built in the standard way from this substitution.

\begin{remark}\label{rem:no_self_referential_names}
  One consequence of these definitions is that $\forall P. \quotep{P}
  \not\in \freenames{P}$.
\end{remark}

\subsection{ Dynamic quote: an example }

Anticipating something of what's to come, consider applying the
substitution, $\widehat{\id{\{}u / z \id{\}}}$, to the following pair
of processes, $\lift{w}{y!(z)}$ and $w[ \lpquote y!(z) \rpquote ]$.

\begin{eqnarray}
	\lift{w}{y!(z)}\widehat{\id{\{}u / z \id{\}}}
		& = &
		\lift{w}{y!(u)} \nonumber\\
	w[ \lpquote y!(z) \rpquote ] \widehat{ \id{\{}u / z \id{\}} }
		& = &
		w[ \lpquote y!(z) \rpquote ] \nonumber
\end{eqnarray}

Because the body of the process between quotes is impervious to
substitution, we get radically different answers. In fact, by
examining the first process in an input context,
e.g. $x?(z).\lift{w}{y!(z)}$, we see that the process under the lift
operator may be shaped by prefixed inputs binding a name inside it. In
this sense, the lift operator will be seen as a way to dynamically
construct processes before reifying them as names.

Finally equipped with these standard features we can present the
dynamics of the calculus.

\subsubsection{Operational semantics} 

Finally, we introduce the computational dynamics. What marks these
algebras as distinct from other more traditionally studied algebraic
structures, e.g. vector spaces or polynomial rings, is the manner in
which dynamics is captured. In traditional structures, dynamics is typically
expressed through morphisms between such structures, as in linear maps
between vector spaces or morphisms between rings. In algebras
associated with the semantics of computation, the dynamics is
expressed as part of the algebraic structure itself, through a
reduction reduction relation typically denoted by $\red$. Below, we
give a recursive presentation of this relation for the calculus used
in the encoding.

$\red \subseteq \pi \times \pi$
$\red : \pi \to \mathcal{P}(\pi)$

\begin{mathpar}
  \inferrule* [lab=Comm] { \textsf{match}( x_{src}, x_{trgt} ) } { x_{trgt}?(y)P \; | \; x_{src}!\langle {Q} \rangle \red P\{\quotep{Q}/y}\} }
  \and \\
  \inferrule* [lab=Par] {{P} \red {P}'} {{{P} | {Q}} \red {{P}' | {Q}}}
  \and
  \inferrule* [lab=Equiv]{{{P} \scong {P}'} \andalso {{P}' \red {Q}'} \andalso {{Q}' \scong {Q}}}{{P} \red {Q}}
\end{mathpar}

\begin{eqnarray*}
  match_{\equiv} (\quotep{P},\quotep{Q}) & := & P \equiv Q \\
  match_{\dagger}(\quotep{P},\quotep{Q}) & := & \forall R. P|Q \red^{*} R => R \red^{*} 0 \\
  match_{K}(\quotep{P},\quotep{Q}) & := & K \mbox{ for some context } K
\end{eqnarray*}

$u?(x)P | u!\langle Q \rangle \red P\{\quotep{Q}/x\}$

%We write $\wred$ for $\red^*$, and $P\red$ if $\exists Q $ such that $ P \red Q$.
We write $P\red$ if $\exists Q $ such that $ P \red Q$ and $P\not\red$, otherwise.

\section{Replication}

As mentioned before, it is known that replication (and hence
recursion) can be implemented in a higher-order process algebra
\cite{SangiorgiWalker}. As our first example of calculation with the
machinery thus far presented we give the construction explicitly in
the {\rhoc}.

\begin{eqnarray}
	D_{x} & := & \prefix{x}{y}{(\binpar{\outputp{x}{y}}{@{y}})} \nonumber\\
	\bangp_{x}{P} & := & \binpar{{x}!\langle{\binpar{D_{x}}{P}}\rangle}{D_{x}} \nonumber
\end{eqnarray}

\begin{eqnarray}
	\bangp_{x}{P} & & \nonumber\\
	=
	& {x}!\langle{(\prefix{x}{y}{(\outputp{x}{y} | @{y})) | P}}\rangle 
	      | \prefix{x}{y}{(\outputp{x}{y} | @{y})} & \nonumber\\
	\red
	& (\outputp{x}{y} | @{y})\substn{\quotep{(\prefix{x}{y}{(@{y} | \outputp{x}{y})) | P}}}{y} & \nonumber\\
	=
	& \outputp{x}{\quotep{(\prefix{x}{y}{(\outputp{x}{y} | @{y})) | P}}}
	  | {(\prefix{x}{y}{(\outputp{x}{y} | @{y})) | P}} & \nonumber\\
	\red
	& \ldots & \nonumber\\
	\red^*
	& P | P | \ldots & \nonumber
\end{eqnarray}

Of course, this encoding, as an implementation, runs away, unfolding
$\bangp{P}$ eagerly. A lazier and more implementable replication
operator, restricted to input-guarded processes, may be obtained as follows.

\begin{eqnarray}
\bangp{\prefix{u}{v}{P}} 
	:= 
	\binpar{\lift{x}{\prefix{u}{v}{(\binpar{D(x)}{P})}}}{D(x)} \nonumber
\end{eqnarray}

\begin{remark}
  Note that the lazier definition still does not deal with summation
  or mixed summation (i.e. sums over input and output). The reader is
  invited to construct definitions of replication that deal with these
  features. 

  Further, the definitions are parameterized in a name, $x$. Can you,
  gentle reader, make a definition that eliminates this parameter and
  guarantees no accidental interaction between the replication
  machinery and the process being replicated -- i.e. no accidental
  sharing of names used by the process to get its work done and the
  name(s) used by the replication to effect copying. This latter
  revision of the definition of replication is crucial to obtaining
  the expected identity $!!P \sim !P$.
\end{remark}

\begin{remark}\label{rem:paradoxical_combinator}
  The reader familiar with the lambda calculus will have noticed the
  similarity between $D$ and the paradoxical combinator.

  [Ed. note: the existence of this seems to suggest we have to be more
  restrictive on the set of processes and names we admit if we are to
  support no-cloning.]
\end{remark}

\subsubsection{Bisimulation}

The computational dynamics gives rise to another kind of equivalence,
the equivalence of computational behavior. As previously mentioned
this is typically captured \emph{via} some form of bisimulation.

% The notion we use in this paper is weak barbed bisimulation
% \cite{milner91polyadicpi}.

The notion we use in this paper is derived from weak barbed
bisimulation \cite{milner91polyadicpi}. 

\begin{definition}
An \emph{observation relation}, $\downarrow_{\mathcal N}$, over a set
of names, $\mathcal N$, is the smallest relation satisfying the rules
below.

\infrule[Out-barb]{y \in {\mathcal N}, \; x \nameeq y}
		  {\outputp{x}{v} \downarrow_{\mathcal N} x}
\infrule[Par-barb]{\mbox{$P\downarrow_{\mathcal N} x$ or $Q\downarrow_{\mathcal N} x$}}
		  {\binpar{P}{Q} \downarrow_{\mathcal N} x}

We write $P \Downarrow_{\mathcal N} x$ if there is $Q$ such that 
$P \wred Q$ and $Q \downarrow_{\mathcal N} x$.
\end{definition}

\begin{definition}
%\label{def.bbisim}
An  ${\mathcal N}$-\emph{barbed bisimulation} over a set of names, ${\mathcal N}$, is a symmetric binary relation 
${\mathcal S}_{\mathcal N}$ between agents such that $P\rel{S}_{\mathcal N}Q$ implies:
\begin{enumerate}
\item If $P \red P'$ then $Q \wred Q'$ and $P'\rel{S}_{\mathcal N} Q'$.
\item If $P\downarrow_{\mathcal N} x$, then $Q\Downarrow_{\mathcal N} x$.
\end{enumerate}
$P$ is ${\mathcal N}$-barbed bisimilar to $Q$, written
$P \wbbisim_{\mathcal N} Q$, if $P \rel{S}_{\mathcal N} Q$ for some ${\mathcal N}$-barbed bisimulation ${\mathcal S}_{\mathcal N}$.
\end{definition}

$\mathcal{R} \subseteq \pi \times \pi$

$P \mathcal{R} Q => \forall P'. P \red P' \Rightarrow \exists Q'. Q \red Q', P' \mathcal{R} Q'$

$P \vdash x \Rightarrow Q \vdash x$

\begin{mathpar}
  \inferrule*[lab=Out-barb]{x \nameeq y}{{y}!\langle{Q}\rangle \vdash x}
  \and
  \inferrule*[lab=Par-barb]{\mbox{$P\vdash x$ or $Q\vdash x$}}{\binpar{P}{Q} \vdash x}
\end{mathpar}

\subsubsection{Contexts}

One of the principle advantages of computational calculi like the
$\pi$-calculus is a well-defined notion of context,
contextual-equivalence and a correlation between
contextual-equivalence and notions of bisimulation. The notion of
context allows the decomposition of a process into (sub-)process and
its syntactic environment, its context. Thus, a context may be
thought of as a process with a ``hole'' (written $\Box$) in it. The
application of a context $M$ to a process $P$, written $M[P]$, is
tantamount to filling the hole in $M$ with $P$. In this paper we do
not need the full weight of this theory, but do make use of the notion
of context in the proof the main theorem. 

\begin{mathpar}
  \inferrule* [lab=summation] {} {{M_{M},M_{N}} \bc \Box \;|\; x.M_{A} \;|\; M_{M}+M_{N}}
  \and
  \inferrule* [lab=agent] {} {{M_{A}} \bc (\vec{x})M_{P} \;| \; \clift{P_0,\ldots,M_{P},\ldots,P_N}}
  \and \\
  \inferrule* [lab=process] {} {{M_{P}} \bc M_{N} \;| \;P|M_{P} }
\end{mathpar} 

\begin{mathpar}
  \inferrule* [lab=sychronization] {} {M_{N} \bc \Box \;|\; x?M_{F} \;|\; x!M_{C}}
  \and
  \inferrule* [lab=abstraction] {} {{M_{F}} \bc (x)M_{P} }
  \and
  \inferrule* [lab=concretion] {} {{M_{C}} \bc \langle M_{P} \rangle }
  \and \\
  \inferrule* [lab=process] {} {{M_{P}} \bc M_{N} \;| \;P|M_{P} }
\end{mathpar}

\begin{definition}[contextual application] Given a context $M$, and
  process $P$, we define the \emph{contextual application}, $M[P] :=
  M\{P/\Box\}$. That is, the contextual application of M to P is the
  substitution of $P$ for $\Box$ in $M$.
\end{definition}

$\meaningof{-} : L \to \mathcal{P}(\pi)$

\begin{mathpar}
  \inferrule* [lab=collection] {} {\meaningof{true} = \pi, \and \meaningof{~E} = \pi \setminus \meaningof{E}, \and \meaningof{E_{1} \& E_{2}} = \meaningof{E_{1}} \cap \meaningof{E_{2}}}
\end{mathpar}

\begin{mathpar}
  \inferrule* [lab=structure] {} {\meaningof{0} = \{ P \in \pi | P \equiv 0 \}, \and \\ \meaningof{E_1 | E_2} = \{ P \in \pi | P \equiv P_{1} | P_{2}, P_{1} \in \meaningof{E_{1}}, P_{2} \in \meaningof{E_2}\} }
\end{mathpar}

\begin{mathpar}
 \inferrule* [lab=behavior] {} {\meaningof{\langle a?b \rangle E} = \{ P \in \pi | P \equiv Q | u?(y)P', \\ \and \\\\ \and \\ \;\;\; u \in \meaningof{a}, \forall z.P'\{z/y\} \in \meaningof{E\{z/b\}}\}, \and \\ \meaningof{a!E} = \{ P \in \pi | P \equiv Q | x!\langle P' \rangle, x \in \meaningof{a} P' \in \meaningof{E}\} }
\end{mathpar}

\begin{mathpar}
 \inferrule* [lab=nominal] {} {\meaningof{\quotep{E}} = \{ \quotep{P} \in \quotep{\pi} | P \in \meaningof{E} \}, \and \meaningof{\quotep{P}} = \{ \quotep{Q} \in \quotep{\pi} | P \equiv Q \} \and \\ \meaningof{@\quotep{E}} = \{ P \in \pi | P \equiv @x, x \in \meaningof{E} \}}
\end{mathpar}

\begin{eqnarray*}
  \\
  \meaningof{-} : TS \to ST
\end{eqnarray*}

\begin{eqnarray*}
  \\
  L : TS \to ST
\end{eqnarray*}

\begin{eqnarray*}
  \\
  P \models E \iff P \in \meaningof{E}
\end{eqnarray*}

\begin{eqnarray*}
  P \approx_{L} Q \iff \forall E \in L. P \models E \iff Q \models E
\end{eqnarray*}

\begin{eqnarray*}
  P \approx_{K} Q
\end{eqnarray*}

\begin{eqnarray*}
  P \approx Q
\end{eqnarray*}

$\approx_{K} = \approx = \approx_{L}$

\subsubsection{Contextual duality}

Note that contexts extend the quotation operation to a family of
operations from processes to names. Given a context, $M$, we can
define a \emph{nominal context}, $\quotep{M}$ by $\quotep{M}[P] :=
\quotep{M[P]}$. To foreshadow what is to come we observe that these
operations enjoy a duality with processes very much like the duality
between vectors and maps from vectors to scalars.

Further, because the calculus is essentially higher-order, we have a
correspondence between contexts and processes. More specifically,
given a name $x$ and a context $M$ we can construct $M^{*}_{x}$ such
that 

\begin{mathpar}
  M^{*}_{x} | \lift{x}{P} \red M[P]
\end{mathpar}

namely,

\begin{mathpar}
  M^{*}_{x} := x?(u).M[\dropn{u}]
\end{mathpar}

The dependence of $M^{*}_{x}$ on a name makes it an abstraction, 

\begin{mathpar}
  M^{*} := (x)x?(u).M[\dropn{u}]
\end{mathpar}

\subsection{Additional notation}

It will sometimes be convenient to denote the process a name
quotes. We already have the notation $x = \quotep{P}$, but it will be
convenient to introduce an alternate notation, $\procn{x}$, when we
want to emphasize the connection to the use of the name. Note that, by
virtue of name equivalence, $\quotep{\procn{x}} \nameeq x$; so, the
notation is consistent with previous definitions.

Further, because names have structure it is possible to effect
substitutions on the basis of that structure. This means we need to
upgrade our notation for substitutions, which we accomplish by
adapting comprehension notation. Thus,

\begin{mathpar}
  P\{ y / x : x \in S \}
\end{mathpar}

is interpreted to mean the process derived from P by replacing (in a
capture-avoiding manner) each occurrence of $x$ in $S$ by $y$. For example,

\begin{mathpar}
  P\{ \quotep{\procn{x}|\procn{x}} / x : x \in \freenames{P} \}
\end{mathpar}

will replace each (occurrence) of a free name $x$ in $P$ by
$\quotep{\procn{x}|\procn{x}}$.

Also, we will avail ourselves of the notation $x^{L}$ and $x^{R}$ to
denote injections of a name into disjoint copies of the name
space. There are numerous ways to accomplish this. One example can be
found in \cite{MeredithR05}. This notation overloads to vectors of
names: $\vec{x}^{\pi} := (x_{i}^{\pi} \; : \; 0 \leq i < |\vec{x}| )$ where $\pi \in \{L,R\}$.

We also use $P^{\Box} := P|\Box$.

In \cite{MeredithR05} an interpretation of the new operator is
given. It turns out that there are several possible interpretations
all enjoying the requisite algebraic properties of the operator (see
\cite{milner91polyadicpi}). We will therefore make liberal use of
$(\nu\; \vec{x})P$.

% subsection the_syntax_and_semantics_of_the_notation_system (end)   

\input{qm2pi.qmops} 

\input{qm2pi.sterngerlach} 

\input{qm2pi.metric} 

% section concurrent_process_calculi (end)

%\input{qm2pi.proofsketch}

% section proof sketch (end)

%\input{qm2pi.slviaknots} 

% section spatial logic via knots (end)

\input{qm2pi.conclusion}

% section conclusion (end)

%\input{qm2pi.dtcodes} 

% section wiring algorithm (end)

\input{qm2pi.ack} 

% section acknowledgments (end)

\newpage


\bibliographystyle{plain}   
\bibliography{../../biblios/main.bib}

\input{qm2pi.rhodetails}

\end{document}

 

% section notation (end)

\input{qm2pi.process.calculi} 

% section concurrent_process_calculi_and_spatial_logics_ (end)
    
%\documentclass[12pt]{llncs}
%\documentclass{jktr}

\usepackage[pdftex]{hyperref}                   
\usepackage {listings}
\usepackage {mathpartir}
\usepackage{bcprules}
%\usepackage{listings}
                       
\usepackage{graphicx} 
%\usepackage[margins=2.5cm,nohead,nofoot]{geometry}
%\usepackage{geometry}
\usepackage{amsfonts}
\usepackage{amstext}
\usepackage{latexsym}
\usepackage{amssymb}
\usepackage{color}


%\include{myPreamble}
\include{qm2pi.local} 

%\ifpdf
%\usepackage[pdftex]{graphicx}
%\else
%\usepackage{graphicx}
%\fi

 % \ifpdf
%  \usepackage{pdfsync}
%  \if


%\title{Brief Article}
%\author{David F. Snyder}
%\author{L.G. Meredith}

%\address{Dept. of Math., Texas State University--San Marcos, San Marcos, TX 78666}
       
\pagestyle{empty}


\begin{document}

\lstset{language=[Objective]Caml,frame=shadowbox}

\input{qm2pi.front}

% section front matter (end)

\input{qm2pi.intro} 
 
% section introduction (end)

% \input{qm2pi.knotations} 

% section notation (end)

\input{qm2pi.process.calculi} 

% section concurrent_process_calculi_and_spatial_logics_ (end)
    
%\input{qm2pi.knots2pi} 

%\input{qm2pi.trefoil} 

%\input{qm2pi.mainthm} 

% subsection basic_interpretation (end)

%\input{qm2pi.rho.presentation} 
\subsection{The syntax and semantics of the notation system}\label{sub:the_syntax_and_semantics_of_the_notation_system} % (fold)

We now summarize a technical presentation of the calculus that
embodies our theory of dynamics. The typical presentation of such a
calculus follows the style of giving generators and relations on
them. The grammar, below, describing term constructors, freely
generates the set of processes, $\Proc$. This set is then quotiented
by a relation known as structural congruence and it is over this set
that the notion of dynamics is expressed. This presentation is
essentially that of \cite{MeredithR05} with the addition of
polyadicity and summation. For readability we have relegated some of
the technical subtleties to an appendix.

\subsubsection{Process grammar}\label{subsub:process_grammar}

\begin{mathpar}
  \inferrule* [lab=synchronization] {} {{M} \bc \pzero \;|\; x?F \;|\; x!C }
  \and
  \inferrule* [lab=abstraction] {} {{F} \bc (x)P}
  \and
  \inferrule* [lab=concretion] {} {{C} \bc \langle Q \rangle}
  \and
  \inferrule* [lab=process] {} {{P,Q} \bc M \;| \;P|Q \;|\; @{x}}
  \and
  \inferrule* [lab=name] {} {{x} \bc \quotep{P}}
\end{mathpar} 

Note that $\vec{x}$ (resp. $\vec{P}$) denotes a vector of names
(resp. processes) of length $|\vec{x}|$ (resp. $|\vec{P}|$). We adopt
the following useful abbreviations.

\begin{mathpar}
   x?(\vec{y}).P := x.(\vec{y})P \and  x\clift{\vec{P}} := x.\clift{\vec{P}}
   \and x!(y) := \lift{x}{\dropn{y}}
   \and \Pi_{i=0}^{n-1}P_i := P_0 | \ldots | P_{n-1}
\end{mathpar}

\subsubsection{Structural congruence}

\paragraph{Free and bound names and alpha-equivalence.} At the
core of structural equivalence is alpha-equivalence which identifies
process that are the same up to a change of variable. Formally, we
recognize the distinction between free and bound names. The free names
of a process, $\freenames{P}$, may be calculated recursively as
follows:

\begin{mathpar}
\freenames{\pzero} := \emptyset
  \and \\
  \freenames{x?(y).P} := \{ x \} \cup (\freenames{P} \setminus \{ y \})
  \and 
  \freenames{x!\langle P \rangle} := \{ x \} \cup \{ P \} 
  \and \\
  \freenames{P|Q} := \freenames{P} \cup \freenames{Q}
  \and \\
  \freenames{@{x}} := \{ x \}
\end{mathpar}

$\pi$
$\quotep{\pi}$

$\freenames{-} : \pi \to \mathcal{P}(\quotep{\pi})$

\begin{eqnarray*}
  \freenames{\pzero} & := & \emptyset \\
  \freenames{x?(y).P} & := & \{ x \} \cup (\freenames{P} \setminus \{ y \}) \\
  \freenames{x!\langle P \rangle} & := & \{ x \} \cup \{ P \} \\
  \freenames{P|Q} & := & \freenames{P} \cup \freenames{Q} \\
  \freenames{\dropn{x}} & := & \{ x \}
\end{eqnarray*}

The bound names of a process, $\boundnames{P}$, are those names occurring in $P$
that are not free. For example, in $x?(y).0$, the name $x$ is free, while $y$ is bound.

\begin{mathpar}
  \inferrule* [lab=monoidal-laws] {} { P|Q \equiv Q|P \and P|0 \equiv P \and P|(Q|R) \equiv (P|Q)|R }
\end{mathpar}

\begin{mathpar}
  \inferrule* [lab=alpha-equivalence] {} { (x)P \equiv (y)P\{y/x\} \and y \not\in \freenames{P} }
\end{mathpar}

\begin{definition}
Then two processes, $P,Q$, are alpha-equivalent if $P = Q\{\vec{y}/\vec{x}\}$ for
some $\vec{x} \in \boundnames{Q},\vec{y} \in \boundnames{P}$, where $Q\{\vec{y}/\vec{x}\}$
denotes the capture-avoiding substitution of $\vec{y}$ for $\vec{x}$ in $Q$.
\end{definition}

\begin{definition}
  The {\em structural congruence} \cite{SangiorgiWalker} , $\equiv$,
  between processes is the least congruence containing
  alpha-equivalence, satisfying the abelian monoid laws
  (associativity, commutativity and $\pzero$ as identity) for parallel
  composition $|$ and for summation $+$.
\end{definition}

\subsection{Name equivalence}

We take name equivalence, written $\nameeq$, to be the smallest
equivalence relation generated by the following rules.

\begin{mathpar}
\inferrule*[lab=Quote-drop]
{ }
{ \quotep{@{x}} \nameeq x }

\inferrule*[lab=Struct-equiv]
{ P \scong Q }
{ \quotep{P} \nameeq \quotep{Q} }
\end{mathpar}

The astute reader will have noticed that the mutual recursion of names
and processes imposes a mutual recursion on alpha-equivalence and
structural equivalence via name-equivalence. Fortunately, all of this
works out pleasantly and we may calculate in the natural way, free of
concern. The reader interested in the details is referred to the
appendix \ref{appendix:rho_details}.

\subsection{Substitution}

We use $\Proc$ for the set of processes, $\QProc$ for the set of
names, and $\id{\{}\vec{y} / \vec{x} \id{\}}$ to denote partial maps,
$s : \QProc \rightarrow \QProc$. A map, $s$ lifts, uniquely, to a map
on process terms, $\widehat{s} : \Proc \rightarrow \Proc$ by the
following equations.

\begin{mathpar}
  (0) \psubstp{Q}{P} := 0 \\
  (R \juxtap S) \psubstp{Q}{P}
  :=    
  (R)\psubstp{Q}{P} \juxtap (S) \psubstp{Q}{P} \\
  (x?(y).R) \psubstp{Q}{P}    
  :=    
  (x)\substp{Q}{P} (z)\concat( (R \psubstn{z}{y}) \psubstp{Q}{P} ) \\
  (\lift{x}{R}) \psubstp{Q}{P}  
  :=
  \lift{(x)\substp{Q}{P}}{ R \psubstp{Q}{P} } \\
%   (\dropn{x})  \psubstp{Q}{P}       
%   := 
%   \left\{ 
%     \begin{array}{ccc} 
%       \dropn{\quotep{Q}} & & x \nameeq \quotep{P} \\
%       \dropn{x} & & otherwise \\
%     \end{array}
%   \right. 
  (\dropn{x})  \psubstp{Q}{P}       
  := 
  \left\{ 
    \begin{array}{ccc} 
      Q & & x \nameeq \quotep{P} \\
      \dropn{x} & & otherwise \\
    \end{array}
  \right.
\end{mathpar}
 

where

\begin{eqnarray}
  (x)\id{\{} \lpquote Q \rpquote / \lpquote P \rpquote \id{\}}            = 
  \left\{ 
    \begin{array}{ccc}
      \lpquote Q \rpquote & & x \nameeq \lpquote P \rpquote \\
      x & & otherwise \\
    \end{array}
  \right. \nonumber
\end{eqnarray}

and $z$ is chosen distinct from $\quotep{P}$, $\quotep{Q}$, the free
names in $Q$, and all the names in $R$. Our $\alpha$-equivalence will
be built in the standard way from this substitution.

\begin{remark}\label{rem:no_self_referential_names}
  One consequence of these definitions is that $\forall P. \quotep{P}
  \not\in \freenames{P}$.
\end{remark}

\subsection{ Dynamic quote: an example }

Anticipating something of what's to come, consider applying the
substitution, $\widehat{\id{\{}u / z \id{\}}}$, to the following pair
of processes, $\lift{w}{y!(z)}$ and $w[ \lpquote y!(z) \rpquote ]$.

\begin{eqnarray}
	\lift{w}{y!(z)}\widehat{\id{\{}u / z \id{\}}}
		& = &
		\lift{w}{y!(u)} \nonumber\\
	w[ \lpquote y!(z) \rpquote ] \widehat{ \id{\{}u / z \id{\}} }
		& = &
		w[ \lpquote y!(z) \rpquote ] \nonumber
\end{eqnarray}

Because the body of the process between quotes is impervious to
substitution, we get radically different answers. In fact, by
examining the first process in an input context,
e.g. $x?(z).\lift{w}{y!(z)}$, we see that the process under the lift
operator may be shaped by prefixed inputs binding a name inside it. In
this sense, the lift operator will be seen as a way to dynamically
construct processes before reifying them as names.

Finally equipped with these standard features we can present the
dynamics of the calculus.

\subsubsection{Operational semantics} 

Finally, we introduce the computational dynamics. What marks these
algebras as distinct from other more traditionally studied algebraic
structures, e.g. vector spaces or polynomial rings, is the manner in
which dynamics is captured. In traditional structures, dynamics is typically
expressed through morphisms between such structures, as in linear maps
between vector spaces or morphisms between rings. In algebras
associated with the semantics of computation, the dynamics is
expressed as part of the algebraic structure itself, through a
reduction reduction relation typically denoted by $\red$. Below, we
give a recursive presentation of this relation for the calculus used
in the encoding.

$\red \subseteq \pi \times \pi$
$\red : \pi \to \mathcal{P}(\pi)$

\begin{mathpar}
  \inferrule* [lab=Comm] { \textsf{match}( x_{src}, x_{trgt} ) } { x_{trgt}?(y)P \; | \; x_{src}!\langle {Q} \rangle \red P\{\quotep{Q}/y}\} }
  \and \\
  \inferrule* [lab=Par] {{P} \red {P}'} {{{P} | {Q}} \red {{P}' | {Q}}}
  \and
  \inferrule* [lab=Equiv]{{{P} \scong {P}'} \andalso {{P}' \red {Q}'} \andalso {{Q}' \scong {Q}}}{{P} \red {Q}}
\end{mathpar}

\begin{eqnarray*}
  match_{\equiv} (\quotep{P},\quotep{Q}) & := & P \equiv Q \\
  match_{\dagger}(\quotep{P},\quotep{Q}) & := & \forall R. P|Q \red^{*} R => R \red^{*} 0 \\
  match_{K}(\quotep{P},\quotep{Q}) & := & K \mbox{ for some context } K
\end{eqnarray*}

$u?(x)P | u!\langle Q \rangle \red P\{\quotep{Q}/x\}$

%We write $\wred$ for $\red^*$, and $P\red$ if $\exists Q $ such that $ P \red Q$.
We write $P\red$ if $\exists Q $ such that $ P \red Q$ and $P\not\red$, otherwise.

\section{Replication}

As mentioned before, it is known that replication (and hence
recursion) can be implemented in a higher-order process algebra
\cite{SangiorgiWalker}. As our first example of calculation with the
machinery thus far presented we give the construction explicitly in
the {\rhoc}.

\begin{eqnarray}
	D_{x} & := & \prefix{x}{y}{(\binpar{\outputp{x}{y}}{@{y}})} \nonumber\\
	\bangp_{x}{P} & := & \binpar{{x}!\langle{\binpar{D_{x}}{P}}\rangle}{D_{x}} \nonumber
\end{eqnarray}

\begin{eqnarray}
	\bangp_{x}{P} & & \nonumber\\
	=
	& {x}!\langle{(\prefix{x}{y}{(\outputp{x}{y} | @{y})) | P}}\rangle 
	      | \prefix{x}{y}{(\outputp{x}{y} | @{y})} & \nonumber\\
	\red
	& (\outputp{x}{y} | @{y})\substn{\quotep{(\prefix{x}{y}{(@{y} | \outputp{x}{y})) | P}}}{y} & \nonumber\\
	=
	& \outputp{x}{\quotep{(\prefix{x}{y}{(\outputp{x}{y} | @{y})) | P}}}
	  | {(\prefix{x}{y}{(\outputp{x}{y} | @{y})) | P}} & \nonumber\\
	\red
	& \ldots & \nonumber\\
	\red^*
	& P | P | \ldots & \nonumber
\end{eqnarray}

Of course, this encoding, as an implementation, runs away, unfolding
$\bangp{P}$ eagerly. A lazier and more implementable replication
operator, restricted to input-guarded processes, may be obtained as follows.

\begin{eqnarray}
\bangp{\prefix{u}{v}{P}} 
	:= 
	\binpar{\lift{x}{\prefix{u}{v}{(\binpar{D(x)}{P})}}}{D(x)} \nonumber
\end{eqnarray}

\begin{remark}
  Note that the lazier definition still does not deal with summation
  or mixed summation (i.e. sums over input and output). The reader is
  invited to construct definitions of replication that deal with these
  features. 

  Further, the definitions are parameterized in a name, $x$. Can you,
  gentle reader, make a definition that eliminates this parameter and
  guarantees no accidental interaction between the replication
  machinery and the process being replicated -- i.e. no accidental
  sharing of names used by the process to get its work done and the
  name(s) used by the replication to effect copying. This latter
  revision of the definition of replication is crucial to obtaining
  the expected identity $!!P \sim !P$.
\end{remark}

\begin{remark}\label{rem:paradoxical_combinator}
  The reader familiar with the lambda calculus will have noticed the
  similarity between $D$ and the paradoxical combinator.

  [Ed. note: the existence of this seems to suggest we have to be more
  restrictive on the set of processes and names we admit if we are to
  support no-cloning.]
\end{remark}

\subsubsection{Bisimulation}

The computational dynamics gives rise to another kind of equivalence,
the equivalence of computational behavior. As previously mentioned
this is typically captured \emph{via} some form of bisimulation.

% The notion we use in this paper is weak barbed bisimulation
% \cite{milner91polyadicpi}.

The notion we use in this paper is derived from weak barbed
bisimulation \cite{milner91polyadicpi}. 

\begin{definition}
An \emph{observation relation}, $\downarrow_{\mathcal N}$, over a set
of names, $\mathcal N$, is the smallest relation satisfying the rules
below.

\infrule[Out-barb]{y \in {\mathcal N}, \; x \nameeq y}
		  {\outputp{x}{v} \downarrow_{\mathcal N} x}
\infrule[Par-barb]{\mbox{$P\downarrow_{\mathcal N} x$ or $Q\downarrow_{\mathcal N} x$}}
		  {\binpar{P}{Q} \downarrow_{\mathcal N} x}

We write $P \Downarrow_{\mathcal N} x$ if there is $Q$ such that 
$P \wred Q$ and $Q \downarrow_{\mathcal N} x$.
\end{definition}

\begin{definition}
%\label{def.bbisim}
An  ${\mathcal N}$-\emph{barbed bisimulation} over a set of names, ${\mathcal N}$, is a symmetric binary relation 
${\mathcal S}_{\mathcal N}$ between agents such that $P\rel{S}_{\mathcal N}Q$ implies:
\begin{enumerate}
\item If $P \red P'$ then $Q \wred Q'$ and $P'\rel{S}_{\mathcal N} Q'$.
\item If $P\downarrow_{\mathcal N} x$, then $Q\Downarrow_{\mathcal N} x$.
\end{enumerate}
$P$ is ${\mathcal N}$-barbed bisimilar to $Q$, written
$P \wbbisim_{\mathcal N} Q$, if $P \rel{S}_{\mathcal N} Q$ for some ${\mathcal N}$-barbed bisimulation ${\mathcal S}_{\mathcal N}$.
\end{definition}

$\mathcal{R} \subseteq \pi \times \pi$

$P \mathcal{R} Q => \forall P'. P \red P' \Rightarrow \exists Q'. Q \red Q', P' \mathcal{R} Q'$

$P \vdash x \Rightarrow Q \vdash x$

\begin{mathpar}
  \inferrule*[lab=Out-barb]{x \nameeq y}{{y}!\langle{Q}\rangle \vdash x}
  \and
  \inferrule*[lab=Par-barb]{\mbox{$P\vdash x$ or $Q\vdash x$}}{\binpar{P}{Q} \vdash x}
\end{mathpar}

\subsubsection{Contexts}

One of the principle advantages of computational calculi like the
$\pi$-calculus is a well-defined notion of context,
contextual-equivalence and a correlation between
contextual-equivalence and notions of bisimulation. The notion of
context allows the decomposition of a process into (sub-)process and
its syntactic environment, its context. Thus, a context may be
thought of as a process with a ``hole'' (written $\Box$) in it. The
application of a context $M$ to a process $P$, written $M[P]$, is
tantamount to filling the hole in $M$ with $P$. In this paper we do
not need the full weight of this theory, but do make use of the notion
of context in the proof the main theorem. 

\begin{mathpar}
  \inferrule* [lab=summation] {} {{M_{M},M_{N}} \bc \Box \;|\; x.M_{A} \;|\; M_{M}+M_{N}}
  \and
  \inferrule* [lab=agent] {} {{M_{A}} \bc (\vec{x})M_{P} \;| \; \clift{P_0,\ldots,M_{P},\ldots,P_N}}
  \and \\
  \inferrule* [lab=process] {} {{M_{P}} \bc M_{N} \;| \;P|M_{P} }
\end{mathpar} 

\begin{mathpar}
  \inferrule* [lab=sychronization] {} {M_{N} \bc \Box \;|\; x?M_{F} \;|\; x!M_{C}}
  \and
  \inferrule* [lab=abstraction] {} {{M_{F}} \bc (x)M_{P} }
  \and
  \inferrule* [lab=concretion] {} {{M_{C}} \bc \langle M_{P} \rangle }
  \and \\
  \inferrule* [lab=process] {} {{M_{P}} \bc M_{N} \;| \;P|M_{P} }
\end{mathpar}

\begin{definition}[contextual application] Given a context $M$, and
  process $P$, we define the \emph{contextual application}, $M[P] :=
  M\{P/\Box\}$. That is, the contextual application of M to P is the
  substitution of $P$ for $\Box$ in $M$.
\end{definition}

$\meaningof{-} : L \to \mathcal{P}(\pi)$

\begin{mathpar}
  \inferrule* [lab=collection] {} {\meaningof{true} = \pi, \and \meaningof{~E} = \pi \setminus \meaningof{E}, \and \meaningof{E_{1} \& E_{2}} = \meaningof{E_{1}} \cap \meaningof{E_{2}}}
\end{mathpar}

\begin{mathpar}
  \inferrule* [lab=structure] {} {\meaningof{0} = \{ P \in \pi | P \equiv 0 \}, \and \\ \meaningof{E_1 | E_2} = \{ P \in \pi | P \equiv P_{1} | P_{2}, P_{1} \in \meaningof{E_{1}}, P_{2} \in \meaningof{E_2}\} }
\end{mathpar}

\begin{mathpar}
 \inferrule* [lab=behavior] {} {\meaningof{\langle a?b \rangle E} = \{ P \in \pi | P \equiv Q | u?(y)P', \\ \and \\\\ \and \\ \;\;\; u \in \meaningof{a}, \forall z.P'\{z/y\} \in \meaningof{E\{z/b\}}\}, \and \\ \meaningof{a!E} = \{ P \in \pi | P \equiv Q | x!\langle P' \rangle, x \in \meaningof{a} P' \in \meaningof{E}\} }
\end{mathpar}

\begin{mathpar}
 \inferrule* [lab=nominal] {} {\meaningof{\quotep{E}} = \{ \quotep{P} \in \quotep{\pi} | P \in \meaningof{E} \}, \and \meaningof{\quotep{P}} = \{ \quotep{Q} \in \quotep{\pi} | P \equiv Q \} \and \\ \meaningof{@\quotep{E}} = \{ P \in \pi | P \equiv @x, x \in \meaningof{E} \}}
\end{mathpar}

\begin{eqnarray*}
  \\
  \meaningof{-} : TS \to ST
\end{eqnarray*}

\begin{eqnarray*}
  \\
  L : TS \to ST
\end{eqnarray*}

\begin{eqnarray*}
  \\
  P \models E \iff P \in \meaningof{E}
\end{eqnarray*}

\begin{eqnarray*}
  P \approx_{L} Q \iff \forall E \in L. P \models E \iff Q \models E
\end{eqnarray*}

\begin{eqnarray*}
  P \approx_{K} Q
\end{eqnarray*}

\begin{eqnarray*}
  P \approx Q
\end{eqnarray*}

$\approx_{K} = \approx = \approx_{L}$

\subsubsection{Contextual duality}

Note that contexts extend the quotation operation to a family of
operations from processes to names. Given a context, $M$, we can
define a \emph{nominal context}, $\quotep{M}$ by $\quotep{M}[P] :=
\quotep{M[P]}$. To foreshadow what is to come we observe that these
operations enjoy a duality with processes very much like the duality
between vectors and maps from vectors to scalars.

Further, because the calculus is essentially higher-order, we have a
correspondence between contexts and processes. More specifically,
given a name $x$ and a context $M$ we can construct $M^{*}_{x}$ such
that 

\begin{mathpar}
  M^{*}_{x} | \lift{x}{P} \red M[P]
\end{mathpar}

namely,

\begin{mathpar}
  M^{*}_{x} := x?(u).M[\dropn{u}]
\end{mathpar}

The dependence of $M^{*}_{x}$ on a name makes it an abstraction, 

\begin{mathpar}
  M^{*} := (x)x?(u).M[\dropn{u}]
\end{mathpar}

\subsection{Additional notation}

It will sometimes be convenient to denote the process a name
quotes. We already have the notation $x = \quotep{P}$, but it will be
convenient to introduce an alternate notation, $\procn{x}$, when we
want to emphasize the connection to the use of the name. Note that, by
virtue of name equivalence, $\quotep{\procn{x}} \nameeq x$; so, the
notation is consistent with previous definitions.

Further, because names have structure it is possible to effect
substitutions on the basis of that structure. This means we need to
upgrade our notation for substitutions, which we accomplish by
adapting comprehension notation. Thus,

\begin{mathpar}
  P\{ y / x : x \in S \}
\end{mathpar}

is interpreted to mean the process derived from P by replacing (in a
capture-avoiding manner) each occurrence of $x$ in $S$ by $y$. For example,

\begin{mathpar}
  P\{ \quotep{\procn{x}|\procn{x}} / x : x \in \freenames{P} \}
\end{mathpar}

will replace each (occurrence) of a free name $x$ in $P$ by
$\quotep{\procn{x}|\procn{x}}$.

Also, we will avail ourselves of the notation $x^{L}$ and $x^{R}$ to
denote injections of a name into disjoint copies of the name
space. There are numerous ways to accomplish this. One example can be
found in \cite{MeredithR05}. This notation overloads to vectors of
names: $\vec{x}^{\pi} := (x_{i}^{\pi} \; : \; 0 \leq i < |\vec{x}| )$ where $\pi \in \{L,R\}$.

We also use $P^{\Box} := P|\Box$.

In \cite{MeredithR05} an interpretation of the new operator is
given. It turns out that there are several possible interpretations
all enjoying the requisite algebraic properties of the operator (see
\cite{milner91polyadicpi}). We will therefore make liberal use of
$(\nu\; \vec{x})P$.

% subsection the_syntax_and_semantics_of_the_notation_system (end)   

\input{qm2pi.qmops} 

\input{qm2pi.sterngerlach} 

\input{qm2pi.metric} 

% section concurrent_process_calculi (end)

%\input{qm2pi.proofsketch}

% section proof sketch (end)

%\input{qm2pi.slviaknots} 

% section spatial logic via knots (end)

\input{qm2pi.conclusion}

% section conclusion (end)

%\input{qm2pi.dtcodes} 

% section wiring algorithm (end)

\input{qm2pi.ack} 

% section acknowledgments (end)

\newpage


\bibliographystyle{plain}   
\bibliography{../../biblios/main.bib}

\input{qm2pi.rhodetails}

\end{document}

 

%\documentclass[12pt]{llncs}
%\documentclass{jktr}

\usepackage[pdftex]{hyperref}                   
\usepackage {listings}
\usepackage {mathpartir}
\usepackage{bcprules}
%\usepackage{listings}
                       
\usepackage{graphicx} 
%\usepackage[margins=2.5cm,nohead,nofoot]{geometry}
%\usepackage{geometry}
\usepackage{amsfonts}
\usepackage{amstext}
\usepackage{latexsym}
\usepackage{amssymb}
\usepackage{color}


%\include{myPreamble}
\include{qm2pi.local} 

%\ifpdf
%\usepackage[pdftex]{graphicx}
%\else
%\usepackage{graphicx}
%\fi

 % \ifpdf
%  \usepackage{pdfsync}
%  \if


%\title{Brief Article}
%\author{David F. Snyder}
%\author{L.G. Meredith}

%\address{Dept. of Math., Texas State University--San Marcos, San Marcos, TX 78666}
       
\pagestyle{empty}


\begin{document}

\lstset{language=[Objective]Caml,frame=shadowbox}

\input{qm2pi.front}

% section front matter (end)

\input{qm2pi.intro} 
 
% section introduction (end)

% \input{qm2pi.knotations} 

% section notation (end)

\input{qm2pi.process.calculi} 

% section concurrent_process_calculi_and_spatial_logics_ (end)
    
%\input{qm2pi.knots2pi} 

%\input{qm2pi.trefoil} 

%\input{qm2pi.mainthm} 

% subsection basic_interpretation (end)

%\input{qm2pi.rho.presentation} 
\subsection{The syntax and semantics of the notation system}\label{sub:the_syntax_and_semantics_of_the_notation_system} % (fold)

We now summarize a technical presentation of the calculus that
embodies our theory of dynamics. The typical presentation of such a
calculus follows the style of giving generators and relations on
them. The grammar, below, describing term constructors, freely
generates the set of processes, $\Proc$. This set is then quotiented
by a relation known as structural congruence and it is over this set
that the notion of dynamics is expressed. This presentation is
essentially that of \cite{MeredithR05} with the addition of
polyadicity and summation. For readability we have relegated some of
the technical subtleties to an appendix.

\subsubsection{Process grammar}\label{subsub:process_grammar}

\begin{mathpar}
  \inferrule* [lab=synchronization] {} {{M} \bc \pzero \;|\; x?F \;|\; x!C }
  \and
  \inferrule* [lab=abstraction] {} {{F} \bc (x)P}
  \and
  \inferrule* [lab=concretion] {} {{C} \bc \langle Q \rangle}
  \and
  \inferrule* [lab=process] {} {{P,Q} \bc M \;| \;P|Q \;|\; @{x}}
  \and
  \inferrule* [lab=name] {} {{x} \bc \quotep{P}}
\end{mathpar} 

Note that $\vec{x}$ (resp. $\vec{P}$) denotes a vector of names
(resp. processes) of length $|\vec{x}|$ (resp. $|\vec{P}|$). We adopt
the following useful abbreviations.

\begin{mathpar}
   x?(\vec{y}).P := x.(\vec{y})P \and  x\clift{\vec{P}} := x.\clift{\vec{P}}
   \and x!(y) := \lift{x}{\dropn{y}}
   \and \Pi_{i=0}^{n-1}P_i := P_0 | \ldots | P_{n-1}
\end{mathpar}

\subsubsection{Structural congruence}

\paragraph{Free and bound names and alpha-equivalence.} At the
core of structural equivalence is alpha-equivalence which identifies
process that are the same up to a change of variable. Formally, we
recognize the distinction between free and bound names. The free names
of a process, $\freenames{P}$, may be calculated recursively as
follows:

\begin{mathpar}
\freenames{\pzero} := \emptyset
  \and \\
  \freenames{x?(y).P} := \{ x \} \cup (\freenames{P} \setminus \{ y \})
  \and 
  \freenames{x!\langle P \rangle} := \{ x \} \cup \{ P \} 
  \and \\
  \freenames{P|Q} := \freenames{P} \cup \freenames{Q}
  \and \\
  \freenames{@{x}} := \{ x \}
\end{mathpar}

$\pi$
$\quotep{\pi}$

$\freenames{-} : \pi \to \mathcal{P}(\quotep{\pi})$

\begin{eqnarray*}
  \freenames{\pzero} & := & \emptyset \\
  \freenames{x?(y).P} & := & \{ x \} \cup (\freenames{P} \setminus \{ y \}) \\
  \freenames{x!\langle P \rangle} & := & \{ x \} \cup \{ P \} \\
  \freenames{P|Q} & := & \freenames{P} \cup \freenames{Q} \\
  \freenames{\dropn{x}} & := & \{ x \}
\end{eqnarray*}

The bound names of a process, $\boundnames{P}$, are those names occurring in $P$
that are not free. For example, in $x?(y).0$, the name $x$ is free, while $y$ is bound.

\begin{mathpar}
  \inferrule* [lab=monoidal-laws] {} { P|Q \equiv Q|P \and P|0 \equiv P \and P|(Q|R) \equiv (P|Q)|R }
\end{mathpar}

\begin{mathpar}
  \inferrule* [lab=alpha-equivalence] {} { (x)P \equiv (y)P\{y/x\} \and y \not\in \freenames{P} }
\end{mathpar}

\begin{definition}
Then two processes, $P,Q$, are alpha-equivalent if $P = Q\{\vec{y}/\vec{x}\}$ for
some $\vec{x} \in \boundnames{Q},\vec{y} \in \boundnames{P}$, where $Q\{\vec{y}/\vec{x}\}$
denotes the capture-avoiding substitution of $\vec{y}$ for $\vec{x}$ in $Q$.
\end{definition}

\begin{definition}
  The {\em structural congruence} \cite{SangiorgiWalker} , $\equiv$,
  between processes is the least congruence containing
  alpha-equivalence, satisfying the abelian monoid laws
  (associativity, commutativity and $\pzero$ as identity) for parallel
  composition $|$ and for summation $+$.
\end{definition}

\subsection{Name equivalence}

We take name equivalence, written $\nameeq$, to be the smallest
equivalence relation generated by the following rules.

\begin{mathpar}
\inferrule*[lab=Quote-drop]
{ }
{ \quotep{@{x}} \nameeq x }

\inferrule*[lab=Struct-equiv]
{ P \scong Q }
{ \quotep{P} \nameeq \quotep{Q} }
\end{mathpar}

The astute reader will have noticed that the mutual recursion of names
and processes imposes a mutual recursion on alpha-equivalence and
structural equivalence via name-equivalence. Fortunately, all of this
works out pleasantly and we may calculate in the natural way, free of
concern. The reader interested in the details is referred to the
appendix \ref{appendix:rho_details}.

\subsection{Substitution}

We use $\Proc$ for the set of processes, $\QProc$ for the set of
names, and $\id{\{}\vec{y} / \vec{x} \id{\}}$ to denote partial maps,
$s : \QProc \rightarrow \QProc$. A map, $s$ lifts, uniquely, to a map
on process terms, $\widehat{s} : \Proc \rightarrow \Proc$ by the
following equations.

\begin{mathpar}
  (0) \psubstp{Q}{P} := 0 \\
  (R \juxtap S) \psubstp{Q}{P}
  :=    
  (R)\psubstp{Q}{P} \juxtap (S) \psubstp{Q}{P} \\
  (x?(y).R) \psubstp{Q}{P}    
  :=    
  (x)\substp{Q}{P} (z)\concat( (R \psubstn{z}{y}) \psubstp{Q}{P} ) \\
  (\lift{x}{R}) \psubstp{Q}{P}  
  :=
  \lift{(x)\substp{Q}{P}}{ R \psubstp{Q}{P} } \\
%   (\dropn{x})  \psubstp{Q}{P}       
%   := 
%   \left\{ 
%     \begin{array}{ccc} 
%       \dropn{\quotep{Q}} & & x \nameeq \quotep{P} \\
%       \dropn{x} & & otherwise \\
%     \end{array}
%   \right. 
  (\dropn{x})  \psubstp{Q}{P}       
  := 
  \left\{ 
    \begin{array}{ccc} 
      Q & & x \nameeq \quotep{P} \\
      \dropn{x} & & otherwise \\
    \end{array}
  \right.
\end{mathpar}
 

where

\begin{eqnarray}
  (x)\id{\{} \lpquote Q \rpquote / \lpquote P \rpquote \id{\}}            = 
  \left\{ 
    \begin{array}{ccc}
      \lpquote Q \rpquote & & x \nameeq \lpquote P \rpquote \\
      x & & otherwise \\
    \end{array}
  \right. \nonumber
\end{eqnarray}

and $z$ is chosen distinct from $\quotep{P}$, $\quotep{Q}$, the free
names in $Q$, and all the names in $R$. Our $\alpha$-equivalence will
be built in the standard way from this substitution.

\begin{remark}\label{rem:no_self_referential_names}
  One consequence of these definitions is that $\forall P. \quotep{P}
  \not\in \freenames{P}$.
\end{remark}

\subsection{ Dynamic quote: an example }

Anticipating something of what's to come, consider applying the
substitution, $\widehat{\id{\{}u / z \id{\}}}$, to the following pair
of processes, $\lift{w}{y!(z)}$ and $w[ \lpquote y!(z) \rpquote ]$.

\begin{eqnarray}
	\lift{w}{y!(z)}\widehat{\id{\{}u / z \id{\}}}
		& = &
		\lift{w}{y!(u)} \nonumber\\
	w[ \lpquote y!(z) \rpquote ] \widehat{ \id{\{}u / z \id{\}} }
		& = &
		w[ \lpquote y!(z) \rpquote ] \nonumber
\end{eqnarray}

Because the body of the process between quotes is impervious to
substitution, we get radically different answers. In fact, by
examining the first process in an input context,
e.g. $x?(z).\lift{w}{y!(z)}$, we see that the process under the lift
operator may be shaped by prefixed inputs binding a name inside it. In
this sense, the lift operator will be seen as a way to dynamically
construct processes before reifying them as names.

Finally equipped with these standard features we can present the
dynamics of the calculus.

\subsubsection{Operational semantics} 

Finally, we introduce the computational dynamics. What marks these
algebras as distinct from other more traditionally studied algebraic
structures, e.g. vector spaces or polynomial rings, is the manner in
which dynamics is captured. In traditional structures, dynamics is typically
expressed through morphisms between such structures, as in linear maps
between vector spaces or morphisms between rings. In algebras
associated with the semantics of computation, the dynamics is
expressed as part of the algebraic structure itself, through a
reduction reduction relation typically denoted by $\red$. Below, we
give a recursive presentation of this relation for the calculus used
in the encoding.

$\red \subseteq \pi \times \pi$
$\red : \pi \to \mathcal{P}(\pi)$

\begin{mathpar}
  \inferrule* [lab=Comm] { \textsf{match}( x_{src}, x_{trgt} ) } { x_{trgt}?(y)P \; | \; x_{src}!\langle {Q} \rangle \red P\{\quotep{Q}/y}\} }
  \and \\
  \inferrule* [lab=Par] {{P} \red {P}'} {{{P} | {Q}} \red {{P}' | {Q}}}
  \and
  \inferrule* [lab=Equiv]{{{P} \scong {P}'} \andalso {{P}' \red {Q}'} \andalso {{Q}' \scong {Q}}}{{P} \red {Q}}
\end{mathpar}

\begin{eqnarray*}
  match_{\equiv} (\quotep{P},\quotep{Q}) & := & P \equiv Q \\
  match_{\dagger}(\quotep{P},\quotep{Q}) & := & \forall R. P|Q \red^{*} R => R \red^{*} 0 \\
  match_{K}(\quotep{P},\quotep{Q}) & := & K \mbox{ for some context } K
\end{eqnarray*}

$u?(x)P | u!\langle Q \rangle \red P\{\quotep{Q}/x\}$

%We write $\wred$ for $\red^*$, and $P\red$ if $\exists Q $ such that $ P \red Q$.
We write $P\red$ if $\exists Q $ such that $ P \red Q$ and $P\not\red$, otherwise.

\section{Replication}

As mentioned before, it is known that replication (and hence
recursion) can be implemented in a higher-order process algebra
\cite{SangiorgiWalker}. As our first example of calculation with the
machinery thus far presented we give the construction explicitly in
the {\rhoc}.

\begin{eqnarray}
	D_{x} & := & \prefix{x}{y}{(\binpar{\outputp{x}{y}}{@{y}})} \nonumber\\
	\bangp_{x}{P} & := & \binpar{{x}!\langle{\binpar{D_{x}}{P}}\rangle}{D_{x}} \nonumber
\end{eqnarray}

\begin{eqnarray}
	\bangp_{x}{P} & & \nonumber\\
	=
	& {x}!\langle{(\prefix{x}{y}{(\outputp{x}{y} | @{y})) | P}}\rangle 
	      | \prefix{x}{y}{(\outputp{x}{y} | @{y})} & \nonumber\\
	\red
	& (\outputp{x}{y} | @{y})\substn{\quotep{(\prefix{x}{y}{(@{y} | \outputp{x}{y})) | P}}}{y} & \nonumber\\
	=
	& \outputp{x}{\quotep{(\prefix{x}{y}{(\outputp{x}{y} | @{y})) | P}}}
	  | {(\prefix{x}{y}{(\outputp{x}{y} | @{y})) | P}} & \nonumber\\
	\red
	& \ldots & \nonumber\\
	\red^*
	& P | P | \ldots & \nonumber
\end{eqnarray}

Of course, this encoding, as an implementation, runs away, unfolding
$\bangp{P}$ eagerly. A lazier and more implementable replication
operator, restricted to input-guarded processes, may be obtained as follows.

\begin{eqnarray}
\bangp{\prefix{u}{v}{P}} 
	:= 
	\binpar{\lift{x}{\prefix{u}{v}{(\binpar{D(x)}{P})}}}{D(x)} \nonumber
\end{eqnarray}

\begin{remark}
  Note that the lazier definition still does not deal with summation
  or mixed summation (i.e. sums over input and output). The reader is
  invited to construct definitions of replication that deal with these
  features. 

  Further, the definitions are parameterized in a name, $x$. Can you,
  gentle reader, make a definition that eliminates this parameter and
  guarantees no accidental interaction between the replication
  machinery and the process being replicated -- i.e. no accidental
  sharing of names used by the process to get its work done and the
  name(s) used by the replication to effect copying. This latter
  revision of the definition of replication is crucial to obtaining
  the expected identity $!!P \sim !P$.
\end{remark}

\begin{remark}\label{rem:paradoxical_combinator}
  The reader familiar with the lambda calculus will have noticed the
  similarity between $D$ and the paradoxical combinator.

  [Ed. note: the existence of this seems to suggest we have to be more
  restrictive on the set of processes and names we admit if we are to
  support no-cloning.]
\end{remark}

\subsubsection{Bisimulation}

The computational dynamics gives rise to another kind of equivalence,
the equivalence of computational behavior. As previously mentioned
this is typically captured \emph{via} some form of bisimulation.

% The notion we use in this paper is weak barbed bisimulation
% \cite{milner91polyadicpi}.

The notion we use in this paper is derived from weak barbed
bisimulation \cite{milner91polyadicpi}. 

\begin{definition}
An \emph{observation relation}, $\downarrow_{\mathcal N}$, over a set
of names, $\mathcal N$, is the smallest relation satisfying the rules
below.

\infrule[Out-barb]{y \in {\mathcal N}, \; x \nameeq y}
		  {\outputp{x}{v} \downarrow_{\mathcal N} x}
\infrule[Par-barb]{\mbox{$P\downarrow_{\mathcal N} x$ or $Q\downarrow_{\mathcal N} x$}}
		  {\binpar{P}{Q} \downarrow_{\mathcal N} x}

We write $P \Downarrow_{\mathcal N} x$ if there is $Q$ such that 
$P \wred Q$ and $Q \downarrow_{\mathcal N} x$.
\end{definition}

\begin{definition}
%\label{def.bbisim}
An  ${\mathcal N}$-\emph{barbed bisimulation} over a set of names, ${\mathcal N}$, is a symmetric binary relation 
${\mathcal S}_{\mathcal N}$ between agents such that $P\rel{S}_{\mathcal N}Q$ implies:
\begin{enumerate}
\item If $P \red P'$ then $Q \wred Q'$ and $P'\rel{S}_{\mathcal N} Q'$.
\item If $P\downarrow_{\mathcal N} x$, then $Q\Downarrow_{\mathcal N} x$.
\end{enumerate}
$P$ is ${\mathcal N}$-barbed bisimilar to $Q$, written
$P \wbbisim_{\mathcal N} Q$, if $P \rel{S}_{\mathcal N} Q$ for some ${\mathcal N}$-barbed bisimulation ${\mathcal S}_{\mathcal N}$.
\end{definition}

$\mathcal{R} \subseteq \pi \times \pi$

$P \mathcal{R} Q => \forall P'. P \red P' \Rightarrow \exists Q'. Q \red Q', P' \mathcal{R} Q'$

$P \vdash x \Rightarrow Q \vdash x$

\begin{mathpar}
  \inferrule*[lab=Out-barb]{x \nameeq y}{{y}!\langle{Q}\rangle \vdash x}
  \and
  \inferrule*[lab=Par-barb]{\mbox{$P\vdash x$ or $Q\vdash x$}}{\binpar{P}{Q} \vdash x}
\end{mathpar}

\subsubsection{Contexts}

One of the principle advantages of computational calculi like the
$\pi$-calculus is a well-defined notion of context,
contextual-equivalence and a correlation between
contextual-equivalence and notions of bisimulation. The notion of
context allows the decomposition of a process into (sub-)process and
its syntactic environment, its context. Thus, a context may be
thought of as a process with a ``hole'' (written $\Box$) in it. The
application of a context $M$ to a process $P$, written $M[P]$, is
tantamount to filling the hole in $M$ with $P$. In this paper we do
not need the full weight of this theory, but do make use of the notion
of context in the proof the main theorem. 

\begin{mathpar}
  \inferrule* [lab=summation] {} {{M_{M},M_{N}} \bc \Box \;|\; x.M_{A} \;|\; M_{M}+M_{N}}
  \and
  \inferrule* [lab=agent] {} {{M_{A}} \bc (\vec{x})M_{P} \;| \; \clift{P_0,\ldots,M_{P},\ldots,P_N}}
  \and \\
  \inferrule* [lab=process] {} {{M_{P}} \bc M_{N} \;| \;P|M_{P} }
\end{mathpar} 

\begin{mathpar}
  \inferrule* [lab=sychronization] {} {M_{N} \bc \Box \;|\; x?M_{F} \;|\; x!M_{C}}
  \and
  \inferrule* [lab=abstraction] {} {{M_{F}} \bc (x)M_{P} }
  \and
  \inferrule* [lab=concretion] {} {{M_{C}} \bc \langle M_{P} \rangle }
  \and \\
  \inferrule* [lab=process] {} {{M_{P}} \bc M_{N} \;| \;P|M_{P} }
\end{mathpar}

\begin{definition}[contextual application] Given a context $M$, and
  process $P$, we define the \emph{contextual application}, $M[P] :=
  M\{P/\Box\}$. That is, the contextual application of M to P is the
  substitution of $P$ for $\Box$ in $M$.
\end{definition}

$\meaningof{-} : L \to \mathcal{P}(\pi)$

\begin{mathpar}
  \inferrule* [lab=collection] {} {\meaningof{true} = \pi, \and \meaningof{~E} = \pi \setminus \meaningof{E}, \and \meaningof{E_{1} \& E_{2}} = \meaningof{E_{1}} \cap \meaningof{E_{2}}}
\end{mathpar}

\begin{mathpar}
  \inferrule* [lab=structure] {} {\meaningof{0} = \{ P \in \pi | P \equiv 0 \}, \and \\ \meaningof{E_1 | E_2} = \{ P \in \pi | P \equiv P_{1} | P_{2}, P_{1} \in \meaningof{E_{1}}, P_{2} \in \meaningof{E_2}\} }
\end{mathpar}

\begin{mathpar}
 \inferrule* [lab=behavior] {} {\meaningof{\langle a?b \rangle E} = \{ P \in \pi | P \equiv Q | u?(y)P', \\ \and \\\\ \and \\ \;\;\; u \in \meaningof{a}, \forall z.P'\{z/y\} \in \meaningof{E\{z/b\}}\}, \and \\ \meaningof{a!E} = \{ P \in \pi | P \equiv Q | x!\langle P' \rangle, x \in \meaningof{a} P' \in \meaningof{E}\} }
\end{mathpar}

\begin{mathpar}
 \inferrule* [lab=nominal] {} {\meaningof{\quotep{E}} = \{ \quotep{P} \in \quotep{\pi} | P \in \meaningof{E} \}, \and \meaningof{\quotep{P}} = \{ \quotep{Q} \in \quotep{\pi} | P \equiv Q \} \and \\ \meaningof{@\quotep{E}} = \{ P \in \pi | P \equiv @x, x \in \meaningof{E} \}}
\end{mathpar}

\begin{eqnarray*}
  \\
  \meaningof{-} : TS \to ST
\end{eqnarray*}

\begin{eqnarray*}
  \\
  L : TS \to ST
\end{eqnarray*}

\begin{eqnarray*}
  \\
  P \models E \iff P \in \meaningof{E}
\end{eqnarray*}

\begin{eqnarray*}
  P \approx_{L} Q \iff \forall E \in L. P \models E \iff Q \models E
\end{eqnarray*}

\begin{eqnarray*}
  P \approx_{K} Q
\end{eqnarray*}

\begin{eqnarray*}
  P \approx Q
\end{eqnarray*}

$\approx_{K} = \approx = \approx_{L}$

\subsubsection{Contextual duality}

Note that contexts extend the quotation operation to a family of
operations from processes to names. Given a context, $M$, we can
define a \emph{nominal context}, $\quotep{M}$ by $\quotep{M}[P] :=
\quotep{M[P]}$. To foreshadow what is to come we observe that these
operations enjoy a duality with processes very much like the duality
between vectors and maps from vectors to scalars.

Further, because the calculus is essentially higher-order, we have a
correspondence between contexts and processes. More specifically,
given a name $x$ and a context $M$ we can construct $M^{*}_{x}$ such
that 

\begin{mathpar}
  M^{*}_{x} | \lift{x}{P} \red M[P]
\end{mathpar}

namely,

\begin{mathpar}
  M^{*}_{x} := x?(u).M[\dropn{u}]
\end{mathpar}

The dependence of $M^{*}_{x}$ on a name makes it an abstraction, 

\begin{mathpar}
  M^{*} := (x)x?(u).M[\dropn{u}]
\end{mathpar}

\subsection{Additional notation}

It will sometimes be convenient to denote the process a name
quotes. We already have the notation $x = \quotep{P}$, but it will be
convenient to introduce an alternate notation, $\procn{x}$, when we
want to emphasize the connection to the use of the name. Note that, by
virtue of name equivalence, $\quotep{\procn{x}} \nameeq x$; so, the
notation is consistent with previous definitions.

Further, because names have structure it is possible to effect
substitutions on the basis of that structure. This means we need to
upgrade our notation for substitutions, which we accomplish by
adapting comprehension notation. Thus,

\begin{mathpar}
  P\{ y / x : x \in S \}
\end{mathpar}

is interpreted to mean the process derived from P by replacing (in a
capture-avoiding manner) each occurrence of $x$ in $S$ by $y$. For example,

\begin{mathpar}
  P\{ \quotep{\procn{x}|\procn{x}} / x : x \in \freenames{P} \}
\end{mathpar}

will replace each (occurrence) of a free name $x$ in $P$ by
$\quotep{\procn{x}|\procn{x}}$.

Also, we will avail ourselves of the notation $x^{L}$ and $x^{R}$ to
denote injections of a name into disjoint copies of the name
space. There are numerous ways to accomplish this. One example can be
found in \cite{MeredithR05}. This notation overloads to vectors of
names: $\vec{x}^{\pi} := (x_{i}^{\pi} \; : \; 0 \leq i < |\vec{x}| )$ where $\pi \in \{L,R\}$.

We also use $P^{\Box} := P|\Box$.

In \cite{MeredithR05} an interpretation of the new operator is
given. It turns out that there are several possible interpretations
all enjoying the requisite algebraic properties of the operator (see
\cite{milner91polyadicpi}). We will therefore make liberal use of
$(\nu\; \vec{x})P$.

% subsection the_syntax_and_semantics_of_the_notation_system (end)   

\input{qm2pi.qmops} 

\input{qm2pi.sterngerlach} 

\input{qm2pi.metric} 

% section concurrent_process_calculi (end)

%\input{qm2pi.proofsketch}

% section proof sketch (end)

%\input{qm2pi.slviaknots} 

% section spatial logic via knots (end)

\input{qm2pi.conclusion}

% section conclusion (end)

%\input{qm2pi.dtcodes} 

% section wiring algorithm (end)

\input{qm2pi.ack} 

% section acknowledgments (end)

\newpage


\bibliographystyle{plain}   
\bibliography{../../biblios/main.bib}

\input{qm2pi.rhodetails}

\end{document}

 

%\documentclass[12pt]{llncs}
%\documentclass{jktr}

\usepackage[pdftex]{hyperref}                   
\usepackage {listings}
\usepackage {mathpartir}
\usepackage{bcprules}
%\usepackage{listings}
                       
\usepackage{graphicx} 
%\usepackage[margins=2.5cm,nohead,nofoot]{geometry}
%\usepackage{geometry}
\usepackage{amsfonts}
\usepackage{amstext}
\usepackage{latexsym}
\usepackage{amssymb}
\usepackage{color}


%\include{myPreamble}
\include{qm2pi.local} 

%\ifpdf
%\usepackage[pdftex]{graphicx}
%\else
%\usepackage{graphicx}
%\fi

 % \ifpdf
%  \usepackage{pdfsync}
%  \if


%\title{Brief Article}
%\author{David F. Snyder}
%\author{L.G. Meredith}

%\address{Dept. of Math., Texas State University--San Marcos, San Marcos, TX 78666}
       
\pagestyle{empty}


\begin{document}

\lstset{language=[Objective]Caml,frame=shadowbox}

\input{qm2pi.front}

% section front matter (end)

\input{qm2pi.intro} 
 
% section introduction (end)

% \input{qm2pi.knotations} 

% section notation (end)

\input{qm2pi.process.calculi} 

% section concurrent_process_calculi_and_spatial_logics_ (end)
    
%\input{qm2pi.knots2pi} 

%\input{qm2pi.trefoil} 

%\input{qm2pi.mainthm} 

% subsection basic_interpretation (end)

%\input{qm2pi.rho.presentation} 
\subsection{The syntax and semantics of the notation system}\label{sub:the_syntax_and_semantics_of_the_notation_system} % (fold)

We now summarize a technical presentation of the calculus that
embodies our theory of dynamics. The typical presentation of such a
calculus follows the style of giving generators and relations on
them. The grammar, below, describing term constructors, freely
generates the set of processes, $\Proc$. This set is then quotiented
by a relation known as structural congruence and it is over this set
that the notion of dynamics is expressed. This presentation is
essentially that of \cite{MeredithR05} with the addition of
polyadicity and summation. For readability we have relegated some of
the technical subtleties to an appendix.

\subsubsection{Process grammar}\label{subsub:process_grammar}

\begin{mathpar}
  \inferrule* [lab=synchronization] {} {{M} \bc \pzero \;|\; x?F \;|\; x!C }
  \and
  \inferrule* [lab=abstraction] {} {{F} \bc (x)P}
  \and
  \inferrule* [lab=concretion] {} {{C} \bc \langle Q \rangle}
  \and
  \inferrule* [lab=process] {} {{P,Q} \bc M \;| \;P|Q \;|\; @{x}}
  \and
  \inferrule* [lab=name] {} {{x} \bc \quotep{P}}
\end{mathpar} 

Note that $\vec{x}$ (resp. $\vec{P}$) denotes a vector of names
(resp. processes) of length $|\vec{x}|$ (resp. $|\vec{P}|$). We adopt
the following useful abbreviations.

\begin{mathpar}
   x?(\vec{y}).P := x.(\vec{y})P \and  x\clift{\vec{P}} := x.\clift{\vec{P}}
   \and x!(y) := \lift{x}{\dropn{y}}
   \and \Pi_{i=0}^{n-1}P_i := P_0 | \ldots | P_{n-1}
\end{mathpar}

\subsubsection{Structural congruence}

\paragraph{Free and bound names and alpha-equivalence.} At the
core of structural equivalence is alpha-equivalence which identifies
process that are the same up to a change of variable. Formally, we
recognize the distinction between free and bound names. The free names
of a process, $\freenames{P}$, may be calculated recursively as
follows:

\begin{mathpar}
\freenames{\pzero} := \emptyset
  \and \\
  \freenames{x?(y).P} := \{ x \} \cup (\freenames{P} \setminus \{ y \})
  \and 
  \freenames{x!\langle P \rangle} := \{ x \} \cup \{ P \} 
  \and \\
  \freenames{P|Q} := \freenames{P} \cup \freenames{Q}
  \and \\
  \freenames{@{x}} := \{ x \}
\end{mathpar}

$\pi$
$\quotep{\pi}$

$\freenames{-} : \pi \to \mathcal{P}(\quotep{\pi})$

\begin{eqnarray*}
  \freenames{\pzero} & := & \emptyset \\
  \freenames{x?(y).P} & := & \{ x \} \cup (\freenames{P} \setminus \{ y \}) \\
  \freenames{x!\langle P \rangle} & := & \{ x \} \cup \{ P \} \\
  \freenames{P|Q} & := & \freenames{P} \cup \freenames{Q} \\
  \freenames{\dropn{x}} & := & \{ x \}
\end{eqnarray*}

The bound names of a process, $\boundnames{P}$, are those names occurring in $P$
that are not free. For example, in $x?(y).0$, the name $x$ is free, while $y$ is bound.

\begin{mathpar}
  \inferrule* [lab=monoidal-laws] {} { P|Q \equiv Q|P \and P|0 \equiv P \and P|(Q|R) \equiv (P|Q)|R }
\end{mathpar}

\begin{mathpar}
  \inferrule* [lab=alpha-equivalence] {} { (x)P \equiv (y)P\{y/x\} \and y \not\in \freenames{P} }
\end{mathpar}

\begin{definition}
Then two processes, $P,Q$, are alpha-equivalent if $P = Q\{\vec{y}/\vec{x}\}$ for
some $\vec{x} \in \boundnames{Q},\vec{y} \in \boundnames{P}$, where $Q\{\vec{y}/\vec{x}\}$
denotes the capture-avoiding substitution of $\vec{y}$ for $\vec{x}$ in $Q$.
\end{definition}

\begin{definition}
  The {\em structural congruence} \cite{SangiorgiWalker} , $\equiv$,
  between processes is the least congruence containing
  alpha-equivalence, satisfying the abelian monoid laws
  (associativity, commutativity and $\pzero$ as identity) for parallel
  composition $|$ and for summation $+$.
\end{definition}

\subsection{Name equivalence}

We take name equivalence, written $\nameeq$, to be the smallest
equivalence relation generated by the following rules.

\begin{mathpar}
\inferrule*[lab=Quote-drop]
{ }
{ \quotep{@{x}} \nameeq x }

\inferrule*[lab=Struct-equiv]
{ P \scong Q }
{ \quotep{P} \nameeq \quotep{Q} }
\end{mathpar}

The astute reader will have noticed that the mutual recursion of names
and processes imposes a mutual recursion on alpha-equivalence and
structural equivalence via name-equivalence. Fortunately, all of this
works out pleasantly and we may calculate in the natural way, free of
concern. The reader interested in the details is referred to the
appendix \ref{appendix:rho_details}.

\subsection{Substitution}

We use $\Proc$ for the set of processes, $\QProc$ for the set of
names, and $\id{\{}\vec{y} / \vec{x} \id{\}}$ to denote partial maps,
$s : \QProc \rightarrow \QProc$. A map, $s$ lifts, uniquely, to a map
on process terms, $\widehat{s} : \Proc \rightarrow \Proc$ by the
following equations.

\begin{mathpar}
  (0) \psubstp{Q}{P} := 0 \\
  (R \juxtap S) \psubstp{Q}{P}
  :=    
  (R)\psubstp{Q}{P} \juxtap (S) \psubstp{Q}{P} \\
  (x?(y).R) \psubstp{Q}{P}    
  :=    
  (x)\substp{Q}{P} (z)\concat( (R \psubstn{z}{y}) \psubstp{Q}{P} ) \\
  (\lift{x}{R}) \psubstp{Q}{P}  
  :=
  \lift{(x)\substp{Q}{P}}{ R \psubstp{Q}{P} } \\
%   (\dropn{x})  \psubstp{Q}{P}       
%   := 
%   \left\{ 
%     \begin{array}{ccc} 
%       \dropn{\quotep{Q}} & & x \nameeq \quotep{P} \\
%       \dropn{x} & & otherwise \\
%     \end{array}
%   \right. 
  (\dropn{x})  \psubstp{Q}{P}       
  := 
  \left\{ 
    \begin{array}{ccc} 
      Q & & x \nameeq \quotep{P} \\
      \dropn{x} & & otherwise \\
    \end{array}
  \right.
\end{mathpar}
 

where

\begin{eqnarray}
  (x)\id{\{} \lpquote Q \rpquote / \lpquote P \rpquote \id{\}}            = 
  \left\{ 
    \begin{array}{ccc}
      \lpquote Q \rpquote & & x \nameeq \lpquote P \rpquote \\
      x & & otherwise \\
    \end{array}
  \right. \nonumber
\end{eqnarray}

and $z$ is chosen distinct from $\quotep{P}$, $\quotep{Q}$, the free
names in $Q$, and all the names in $R$. Our $\alpha$-equivalence will
be built in the standard way from this substitution.

\begin{remark}\label{rem:no_self_referential_names}
  One consequence of these definitions is that $\forall P. \quotep{P}
  \not\in \freenames{P}$.
\end{remark}

\subsection{ Dynamic quote: an example }

Anticipating something of what's to come, consider applying the
substitution, $\widehat{\id{\{}u / z \id{\}}}$, to the following pair
of processes, $\lift{w}{y!(z)}$ and $w[ \lpquote y!(z) \rpquote ]$.

\begin{eqnarray}
	\lift{w}{y!(z)}\widehat{\id{\{}u / z \id{\}}}
		& = &
		\lift{w}{y!(u)} \nonumber\\
	w[ \lpquote y!(z) \rpquote ] \widehat{ \id{\{}u / z \id{\}} }
		& = &
		w[ \lpquote y!(z) \rpquote ] \nonumber
\end{eqnarray}

Because the body of the process between quotes is impervious to
substitution, we get radically different answers. In fact, by
examining the first process in an input context,
e.g. $x?(z).\lift{w}{y!(z)}$, we see that the process under the lift
operator may be shaped by prefixed inputs binding a name inside it. In
this sense, the lift operator will be seen as a way to dynamically
construct processes before reifying them as names.

Finally equipped with these standard features we can present the
dynamics of the calculus.

\subsubsection{Operational semantics} 

Finally, we introduce the computational dynamics. What marks these
algebras as distinct from other more traditionally studied algebraic
structures, e.g. vector spaces or polynomial rings, is the manner in
which dynamics is captured. In traditional structures, dynamics is typically
expressed through morphisms between such structures, as in linear maps
between vector spaces or morphisms between rings. In algebras
associated with the semantics of computation, the dynamics is
expressed as part of the algebraic structure itself, through a
reduction reduction relation typically denoted by $\red$. Below, we
give a recursive presentation of this relation for the calculus used
in the encoding.

$\red \subseteq \pi \times \pi$
$\red : \pi \to \mathcal{P}(\pi)$

\begin{mathpar}
  \inferrule* [lab=Comm] { \textsf{match}( x_{src}, x_{trgt} ) } { x_{trgt}?(y)P \; | \; x_{src}!\langle {Q} \rangle \red P\{\quotep{Q}/y}\} }
  \and \\
  \inferrule* [lab=Par] {{P} \red {P}'} {{{P} | {Q}} \red {{P}' | {Q}}}
  \and
  \inferrule* [lab=Equiv]{{{P} \scong {P}'} \andalso {{P}' \red {Q}'} \andalso {{Q}' \scong {Q}}}{{P} \red {Q}}
\end{mathpar}

\begin{eqnarray*}
  match_{\equiv} (\quotep{P},\quotep{Q}) & := & P \equiv Q \\
  match_{\dagger}(\quotep{P},\quotep{Q}) & := & \forall R. P|Q \red^{*} R => R \red^{*} 0 \\
  match_{K}(\quotep{P},\quotep{Q}) & := & K \mbox{ for some context } K
\end{eqnarray*}

$u?(x)P | u!\langle Q \rangle \red P\{\quotep{Q}/x\}$

%We write $\wred$ for $\red^*$, and $P\red$ if $\exists Q $ such that $ P \red Q$.
We write $P\red$ if $\exists Q $ such that $ P \red Q$ and $P\not\red$, otherwise.

\section{Replication}

As mentioned before, it is known that replication (and hence
recursion) can be implemented in a higher-order process algebra
\cite{SangiorgiWalker}. As our first example of calculation with the
machinery thus far presented we give the construction explicitly in
the {\rhoc}.

\begin{eqnarray}
	D_{x} & := & \prefix{x}{y}{(\binpar{\outputp{x}{y}}{@{y}})} \nonumber\\
	\bangp_{x}{P} & := & \binpar{{x}!\langle{\binpar{D_{x}}{P}}\rangle}{D_{x}} \nonumber
\end{eqnarray}

\begin{eqnarray}
	\bangp_{x}{P} & & \nonumber\\
	=
	& {x}!\langle{(\prefix{x}{y}{(\outputp{x}{y} | @{y})) | P}}\rangle 
	      | \prefix{x}{y}{(\outputp{x}{y} | @{y})} & \nonumber\\
	\red
	& (\outputp{x}{y} | @{y})\substn{\quotep{(\prefix{x}{y}{(@{y} | \outputp{x}{y})) | P}}}{y} & \nonumber\\
	=
	& \outputp{x}{\quotep{(\prefix{x}{y}{(\outputp{x}{y} | @{y})) | P}}}
	  | {(\prefix{x}{y}{(\outputp{x}{y} | @{y})) | P}} & \nonumber\\
	\red
	& \ldots & \nonumber\\
	\red^*
	& P | P | \ldots & \nonumber
\end{eqnarray}

Of course, this encoding, as an implementation, runs away, unfolding
$\bangp{P}$ eagerly. A lazier and more implementable replication
operator, restricted to input-guarded processes, may be obtained as follows.

\begin{eqnarray}
\bangp{\prefix{u}{v}{P}} 
	:= 
	\binpar{\lift{x}{\prefix{u}{v}{(\binpar{D(x)}{P})}}}{D(x)} \nonumber
\end{eqnarray}

\begin{remark}
  Note that the lazier definition still does not deal with summation
  or mixed summation (i.e. sums over input and output). The reader is
  invited to construct definitions of replication that deal with these
  features. 

  Further, the definitions are parameterized in a name, $x$. Can you,
  gentle reader, make a definition that eliminates this parameter and
  guarantees no accidental interaction between the replication
  machinery and the process being replicated -- i.e. no accidental
  sharing of names used by the process to get its work done and the
  name(s) used by the replication to effect copying. This latter
  revision of the definition of replication is crucial to obtaining
  the expected identity $!!P \sim !P$.
\end{remark}

\begin{remark}\label{rem:paradoxical_combinator}
  The reader familiar with the lambda calculus will have noticed the
  similarity between $D$ and the paradoxical combinator.

  [Ed. note: the existence of this seems to suggest we have to be more
  restrictive on the set of processes and names we admit if we are to
  support no-cloning.]
\end{remark}

\subsubsection{Bisimulation}

The computational dynamics gives rise to another kind of equivalence,
the equivalence of computational behavior. As previously mentioned
this is typically captured \emph{via} some form of bisimulation.

% The notion we use in this paper is weak barbed bisimulation
% \cite{milner91polyadicpi}.

The notion we use in this paper is derived from weak barbed
bisimulation \cite{milner91polyadicpi}. 

\begin{definition}
An \emph{observation relation}, $\downarrow_{\mathcal N}$, over a set
of names, $\mathcal N$, is the smallest relation satisfying the rules
below.

\infrule[Out-barb]{y \in {\mathcal N}, \; x \nameeq y}
		  {\outputp{x}{v} \downarrow_{\mathcal N} x}
\infrule[Par-barb]{\mbox{$P\downarrow_{\mathcal N} x$ or $Q\downarrow_{\mathcal N} x$}}
		  {\binpar{P}{Q} \downarrow_{\mathcal N} x}

We write $P \Downarrow_{\mathcal N} x$ if there is $Q$ such that 
$P \wred Q$ and $Q \downarrow_{\mathcal N} x$.
\end{definition}

\begin{definition}
%\label{def.bbisim}
An  ${\mathcal N}$-\emph{barbed bisimulation} over a set of names, ${\mathcal N}$, is a symmetric binary relation 
${\mathcal S}_{\mathcal N}$ between agents such that $P\rel{S}_{\mathcal N}Q$ implies:
\begin{enumerate}
\item If $P \red P'$ then $Q \wred Q'$ and $P'\rel{S}_{\mathcal N} Q'$.
\item If $P\downarrow_{\mathcal N} x$, then $Q\Downarrow_{\mathcal N} x$.
\end{enumerate}
$P$ is ${\mathcal N}$-barbed bisimilar to $Q$, written
$P \wbbisim_{\mathcal N} Q$, if $P \rel{S}_{\mathcal N} Q$ for some ${\mathcal N}$-barbed bisimulation ${\mathcal S}_{\mathcal N}$.
\end{definition}

$\mathcal{R} \subseteq \pi \times \pi$

$P \mathcal{R} Q => \forall P'. P \red P' \Rightarrow \exists Q'. Q \red Q', P' \mathcal{R} Q'$

$P \vdash x \Rightarrow Q \vdash x$

\begin{mathpar}
  \inferrule*[lab=Out-barb]{x \nameeq y}{{y}!\langle{Q}\rangle \vdash x}
  \and
  \inferrule*[lab=Par-barb]{\mbox{$P\vdash x$ or $Q\vdash x$}}{\binpar{P}{Q} \vdash x}
\end{mathpar}

\subsubsection{Contexts}

One of the principle advantages of computational calculi like the
$\pi$-calculus is a well-defined notion of context,
contextual-equivalence and a correlation between
contextual-equivalence and notions of bisimulation. The notion of
context allows the decomposition of a process into (sub-)process and
its syntactic environment, its context. Thus, a context may be
thought of as a process with a ``hole'' (written $\Box$) in it. The
application of a context $M$ to a process $P$, written $M[P]$, is
tantamount to filling the hole in $M$ with $P$. In this paper we do
not need the full weight of this theory, but do make use of the notion
of context in the proof the main theorem. 

\begin{mathpar}
  \inferrule* [lab=summation] {} {{M_{M},M_{N}} \bc \Box \;|\; x.M_{A} \;|\; M_{M}+M_{N}}
  \and
  \inferrule* [lab=agent] {} {{M_{A}} \bc (\vec{x})M_{P} \;| \; \clift{P_0,\ldots,M_{P},\ldots,P_N}}
  \and \\
  \inferrule* [lab=process] {} {{M_{P}} \bc M_{N} \;| \;P|M_{P} }
\end{mathpar} 

\begin{mathpar}
  \inferrule* [lab=sychronization] {} {M_{N} \bc \Box \;|\; x?M_{F} \;|\; x!M_{C}}
  \and
  \inferrule* [lab=abstraction] {} {{M_{F}} \bc (x)M_{P} }
  \and
  \inferrule* [lab=concretion] {} {{M_{C}} \bc \langle M_{P} \rangle }
  \and \\
  \inferrule* [lab=process] {} {{M_{P}} \bc M_{N} \;| \;P|M_{P} }
\end{mathpar}

\begin{definition}[contextual application] Given a context $M$, and
  process $P$, we define the \emph{contextual application}, $M[P] :=
  M\{P/\Box\}$. That is, the contextual application of M to P is the
  substitution of $P$ for $\Box$ in $M$.
\end{definition}

$\meaningof{-} : L \to \mathcal{P}(\pi)$

\begin{mathpar}
  \inferrule* [lab=collection] {} {\meaningof{true} = \pi, \and \meaningof{~E} = \pi \setminus \meaningof{E}, \and \meaningof{E_{1} \& E_{2}} = \meaningof{E_{1}} \cap \meaningof{E_{2}}}
\end{mathpar}

\begin{mathpar}
  \inferrule* [lab=structure] {} {\meaningof{0} = \{ P \in \pi | P \equiv 0 \}, \and \\ \meaningof{E_1 | E_2} = \{ P \in \pi | P \equiv P_{1} | P_{2}, P_{1} \in \meaningof{E_{1}}, P_{2} \in \meaningof{E_2}\} }
\end{mathpar}

\begin{mathpar}
 \inferrule* [lab=behavior] {} {\meaningof{\langle a?b \rangle E} = \{ P \in \pi | P \equiv Q | u?(y)P', \\ \and \\\\ \and \\ \;\;\; u \in \meaningof{a}, \forall z.P'\{z/y\} \in \meaningof{E\{z/b\}}\}, \and \\ \meaningof{a!E} = \{ P \in \pi | P \equiv Q | x!\langle P' \rangle, x \in \meaningof{a} P' \in \meaningof{E}\} }
\end{mathpar}

\begin{mathpar}
 \inferrule* [lab=nominal] {} {\meaningof{\quotep{E}} = \{ \quotep{P} \in \quotep{\pi} | P \in \meaningof{E} \}, \and \meaningof{\quotep{P}} = \{ \quotep{Q} \in \quotep{\pi} | P \equiv Q \} \and \\ \meaningof{@\quotep{E}} = \{ P \in \pi | P \equiv @x, x \in \meaningof{E} \}}
\end{mathpar}

\begin{eqnarray*}
  \\
  \meaningof{-} : TS \to ST
\end{eqnarray*}

\begin{eqnarray*}
  \\
  L : TS \to ST
\end{eqnarray*}

\begin{eqnarray*}
  \\
  P \models E \iff P \in \meaningof{E}
\end{eqnarray*}

\begin{eqnarray*}
  P \approx_{L} Q \iff \forall E \in L. P \models E \iff Q \models E
\end{eqnarray*}

\begin{eqnarray*}
  P \approx_{K} Q
\end{eqnarray*}

\begin{eqnarray*}
  P \approx Q
\end{eqnarray*}

$\approx_{K} = \approx = \approx_{L}$

\subsubsection{Contextual duality}

Note that contexts extend the quotation operation to a family of
operations from processes to names. Given a context, $M$, we can
define a \emph{nominal context}, $\quotep{M}$ by $\quotep{M}[P] :=
\quotep{M[P]}$. To foreshadow what is to come we observe that these
operations enjoy a duality with processes very much like the duality
between vectors and maps from vectors to scalars.

Further, because the calculus is essentially higher-order, we have a
correspondence between contexts and processes. More specifically,
given a name $x$ and a context $M$ we can construct $M^{*}_{x}$ such
that 

\begin{mathpar}
  M^{*}_{x} | \lift{x}{P} \red M[P]
\end{mathpar}

namely,

\begin{mathpar}
  M^{*}_{x} := x?(u).M[\dropn{u}]
\end{mathpar}

The dependence of $M^{*}_{x}$ on a name makes it an abstraction, 

\begin{mathpar}
  M^{*} := (x)x?(u).M[\dropn{u}]
\end{mathpar}

\subsection{Additional notation}

It will sometimes be convenient to denote the process a name
quotes. We already have the notation $x = \quotep{P}$, but it will be
convenient to introduce an alternate notation, $\procn{x}$, when we
want to emphasize the connection to the use of the name. Note that, by
virtue of name equivalence, $\quotep{\procn{x}} \nameeq x$; so, the
notation is consistent with previous definitions.

Further, because names have structure it is possible to effect
substitutions on the basis of that structure. This means we need to
upgrade our notation for substitutions, which we accomplish by
adapting comprehension notation. Thus,

\begin{mathpar}
  P\{ y / x : x \in S \}
\end{mathpar}

is interpreted to mean the process derived from P by replacing (in a
capture-avoiding manner) each occurrence of $x$ in $S$ by $y$. For example,

\begin{mathpar}
  P\{ \quotep{\procn{x}|\procn{x}} / x : x \in \freenames{P} \}
\end{mathpar}

will replace each (occurrence) of a free name $x$ in $P$ by
$\quotep{\procn{x}|\procn{x}}$.

Also, we will avail ourselves of the notation $x^{L}$ and $x^{R}$ to
denote injections of a name into disjoint copies of the name
space. There are numerous ways to accomplish this. One example can be
found in \cite{MeredithR05}. This notation overloads to vectors of
names: $\vec{x}^{\pi} := (x_{i}^{\pi} \; : \; 0 \leq i < |\vec{x}| )$ where $\pi \in \{L,R\}$.

We also use $P^{\Box} := P|\Box$.

In \cite{MeredithR05} an interpretation of the new operator is
given. It turns out that there are several possible interpretations
all enjoying the requisite algebraic properties of the operator (see
\cite{milner91polyadicpi}). We will therefore make liberal use of
$(\nu\; \vec{x})P$.

% subsection the_syntax_and_semantics_of_the_notation_system (end)   

\input{qm2pi.qmops} 

\input{qm2pi.sterngerlach} 

\input{qm2pi.metric} 

% section concurrent_process_calculi (end)

%\input{qm2pi.proofsketch}

% section proof sketch (end)

%\input{qm2pi.slviaknots} 

% section spatial logic via knots (end)

\input{qm2pi.conclusion}

% section conclusion (end)

%\input{qm2pi.dtcodes} 

% section wiring algorithm (end)

\input{qm2pi.ack} 

% section acknowledgments (end)

\newpage


\bibliographystyle{plain}   
\bibliography{../../biblios/main.bib}

\input{qm2pi.rhodetails}

\end{document}

 

% subsection basic_interpretation (end)

%\input{qm2pi.rho.presentation} 
\subsection{The syntax and semantics of the notation system}\label{sub:the_syntax_and_semantics_of_the_notation_system} % (fold)

We now summarize a technical presentation of the calculus that
embodies our theory of dynamics. The typical presentation of such a
calculus follows the style of giving generators and relations on
them. The grammar, below, describing term constructors, freely
generates the set of processes, $\Proc$. This set is then quotiented
by a relation known as structural congruence and it is over this set
that the notion of dynamics is expressed. This presentation is
essentially that of \cite{MeredithR05} with the addition of
polyadicity and summation. For readability we have relegated some of
the technical subtleties to an appendix.

\subsubsection{Process grammar}\label{subsub:process_grammar}

\begin{mathpar}
  \inferrule* [lab=synchronization] {} {{M} \bc \pzero \;|\; x?F \;|\; x!C }
  \and
  \inferrule* [lab=abstraction] {} {{F} \bc (x)P}
  \and
  \inferrule* [lab=concretion] {} {{C} \bc \langle Q \rangle}
  \and
  \inferrule* [lab=process] {} {{P,Q} \bc M \;| \;P|Q \;|\; @{x}}
  \and
  \inferrule* [lab=name] {} {{x} \bc \quotep{P}}
\end{mathpar} 

Note that $\vec{x}$ (resp. $\vec{P}$) denotes a vector of names
(resp. processes) of length $|\vec{x}|$ (resp. $|\vec{P}|$). We adopt
the following useful abbreviations.

\begin{mathpar}
   x?(\vec{y}).P := x.(\vec{y})P \and  x\clift{\vec{P}} := x.\clift{\vec{P}}
   \and x!(y) := \lift{x}{\dropn{y}}
   \and \Pi_{i=0}^{n-1}P_i := P_0 | \ldots | P_{n-1}
\end{mathpar}

\subsubsection{Structural congruence}

\paragraph{Free and bound names and alpha-equivalence.} At the
core of structural equivalence is alpha-equivalence which identifies
process that are the same up to a change of variable. Formally, we
recognize the distinction between free and bound names. The free names
of a process, $\freenames{P}$, may be calculated recursively as
follows:

\begin{mathpar}
\freenames{\pzero} := \emptyset
  \and \\
  \freenames{x?(y).P} := \{ x \} \cup (\freenames{P} \setminus \{ y \})
  \and 
  \freenames{x!\langle P \rangle} := \{ x \} \cup \{ P \} 
  \and \\
  \freenames{P|Q} := \freenames{P} \cup \freenames{Q}
  \and \\
  \freenames{@{x}} := \{ x \}
\end{mathpar}

$\pi$
$\quotep{\pi}$

$\freenames{-} : \pi \to \mathcal{P}(\quotep{\pi})$

\begin{eqnarray*}
  \freenames{\pzero} & := & \emptyset \\
  \freenames{x?(y).P} & := & \{ x \} \cup (\freenames{P} \setminus \{ y \}) \\
  \freenames{x!\langle P \rangle} & := & \{ x \} \cup \{ P \} \\
  \freenames{P|Q} & := & \freenames{P} \cup \freenames{Q} \\
  \freenames{\dropn{x}} & := & \{ x \}
\end{eqnarray*}

The bound names of a process, $\boundnames{P}$, are those names occurring in $P$
that are not free. For example, in $x?(y).0$, the name $x$ is free, while $y$ is bound.

\begin{mathpar}
  \inferrule* [lab=monoidal-laws] {} { P|Q \equiv Q|P \and P|0 \equiv P \and P|(Q|R) \equiv (P|Q)|R }
\end{mathpar}

\begin{mathpar}
  \inferrule* [lab=alpha-equivalence] {} { (x)P \equiv (y)P\{y/x\} \and y \not\in \freenames{P} }
\end{mathpar}

\begin{definition}
Then two processes, $P,Q$, are alpha-equivalent if $P = Q\{\vec{y}/\vec{x}\}$ for
some $\vec{x} \in \boundnames{Q},\vec{y} \in \boundnames{P}$, where $Q\{\vec{y}/\vec{x}\}$
denotes the capture-avoiding substitution of $\vec{y}$ for $\vec{x}$ in $Q$.
\end{definition}

\begin{definition}
  The {\em structural congruence} \cite{SangiorgiWalker} , $\equiv$,
  between processes is the least congruence containing
  alpha-equivalence, satisfying the abelian monoid laws
  (associativity, commutativity and $\pzero$ as identity) for parallel
  composition $|$ and for summation $+$.
\end{definition}

\subsection{Name equivalence}

We take name equivalence, written $\nameeq$, to be the smallest
equivalence relation generated by the following rules.

\begin{mathpar}
\inferrule*[lab=Quote-drop]
{ }
{ \quotep{@{x}} \nameeq x }

\inferrule*[lab=Struct-equiv]
{ P \scong Q }
{ \quotep{P} \nameeq \quotep{Q} }
\end{mathpar}

The astute reader will have noticed that the mutual recursion of names
and processes imposes a mutual recursion on alpha-equivalence and
structural equivalence via name-equivalence. Fortunately, all of this
works out pleasantly and we may calculate in the natural way, free of
concern. The reader interested in the details is referred to the
appendix \ref{appendix:rho_details}.

\subsection{Substitution}

We use $\Proc$ for the set of processes, $\QProc$ for the set of
names, and $\id{\{}\vec{y} / \vec{x} \id{\}}$ to denote partial maps,
$s : \QProc \rightarrow \QProc$. A map, $s$ lifts, uniquely, to a map
on process terms, $\widehat{s} : \Proc \rightarrow \Proc$ by the
following equations.

\begin{mathpar}
  (0) \psubstp{Q}{P} := 0 \\
  (R \juxtap S) \psubstp{Q}{P}
  :=    
  (R)\psubstp{Q}{P} \juxtap (S) \psubstp{Q}{P} \\
  (x?(y).R) \psubstp{Q}{P}    
  :=    
  (x)\substp{Q}{P} (z)\concat( (R \psubstn{z}{y}) \psubstp{Q}{P} ) \\
  (\lift{x}{R}) \psubstp{Q}{P}  
  :=
  \lift{(x)\substp{Q}{P}}{ R \psubstp{Q}{P} } \\
%   (\dropn{x})  \psubstp{Q}{P}       
%   := 
%   \left\{ 
%     \begin{array}{ccc} 
%       \dropn{\quotep{Q}} & & x \nameeq \quotep{P} \\
%       \dropn{x} & & otherwise \\
%     \end{array}
%   \right. 
  (\dropn{x})  \psubstp{Q}{P}       
  := 
  \left\{ 
    \begin{array}{ccc} 
      Q & & x \nameeq \quotep{P} \\
      \dropn{x} & & otherwise \\
    \end{array}
  \right.
\end{mathpar}
 

where

\begin{eqnarray}
  (x)\id{\{} \lpquote Q \rpquote / \lpquote P \rpquote \id{\}}            = 
  \left\{ 
    \begin{array}{ccc}
      \lpquote Q \rpquote & & x \nameeq \lpquote P \rpquote \\
      x & & otherwise \\
    \end{array}
  \right. \nonumber
\end{eqnarray}

and $z$ is chosen distinct from $\quotep{P}$, $\quotep{Q}$, the free
names in $Q$, and all the names in $R$. Our $\alpha$-equivalence will
be built in the standard way from this substitution.

\begin{remark}\label{rem:no_self_referential_names}
  One consequence of these definitions is that $\forall P. \quotep{P}
  \not\in \freenames{P}$.
\end{remark}

\subsection{ Dynamic quote: an example }

Anticipating something of what's to come, consider applying the
substitution, $\widehat{\id{\{}u / z \id{\}}}$, to the following pair
of processes, $\lift{w}{y!(z)}$ and $w[ \lpquote y!(z) \rpquote ]$.

\begin{eqnarray}
	\lift{w}{y!(z)}\widehat{\id{\{}u / z \id{\}}}
		& = &
		\lift{w}{y!(u)} \nonumber\\
	w[ \lpquote y!(z) \rpquote ] \widehat{ \id{\{}u / z \id{\}} }
		& = &
		w[ \lpquote y!(z) \rpquote ] \nonumber
\end{eqnarray}

Because the body of the process between quotes is impervious to
substitution, we get radically different answers. In fact, by
examining the first process in an input context,
e.g. $x?(z).\lift{w}{y!(z)}$, we see that the process under the lift
operator may be shaped by prefixed inputs binding a name inside it. In
this sense, the lift operator will be seen as a way to dynamically
construct processes before reifying them as names.

Finally equipped with these standard features we can present the
dynamics of the calculus.

\subsubsection{Operational semantics} 

Finally, we introduce the computational dynamics. What marks these
algebras as distinct from other more traditionally studied algebraic
structures, e.g. vector spaces or polynomial rings, is the manner in
which dynamics is captured. In traditional structures, dynamics is typically
expressed through morphisms between such structures, as in linear maps
between vector spaces or morphisms between rings. In algebras
associated with the semantics of computation, the dynamics is
expressed as part of the algebraic structure itself, through a
reduction reduction relation typically denoted by $\red$. Below, we
give a recursive presentation of this relation for the calculus used
in the encoding.

$\red \subseteq \pi \times \pi$
$\red : \pi \to \mathcal{P}(\pi)$

\begin{mathpar}
  \inferrule* [lab=Comm] { \textsf{match}( x_{src}, x_{trgt} ) } { x_{trgt}?(y)P \; | \; x_{src}!\langle {Q} \rangle \red P\{\quotep{Q}/y}\} }
  \and \\
  \inferrule* [lab=Par] {{P} \red {P}'} {{{P} | {Q}} \red {{P}' | {Q}}}
  \and
  \inferrule* [lab=Equiv]{{{P} \scong {P}'} \andalso {{P}' \red {Q}'} \andalso {{Q}' \scong {Q}}}{{P} \red {Q}}
\end{mathpar}

\begin{eqnarray*}
  match_{\equiv} (\quotep{P},\quotep{Q}) & := & P \equiv Q \\
  match_{\dagger}(\quotep{P},\quotep{Q}) & := & \forall R. P|Q \red^{*} R => R \red^{*} 0 \\
  match_{K}(\quotep{P},\quotep{Q}) & := & K \mbox{ for some context } K
\end{eqnarray*}

$u?(x)P | u!\langle Q \rangle \red P\{\quotep{Q}/x\}$

%We write $\wred$ for $\red^*$, and $P\red$ if $\exists Q $ such that $ P \red Q$.
We write $P\red$ if $\exists Q $ such that $ P \red Q$ and $P\not\red$, otherwise.

\section{Replication}

As mentioned before, it is known that replication (and hence
recursion) can be implemented in a higher-order process algebra
\cite{SangiorgiWalker}. As our first example of calculation with the
machinery thus far presented we give the construction explicitly in
the {\rhoc}.

\begin{eqnarray}
	D_{x} & := & \prefix{x}{y}{(\binpar{\outputp{x}{y}}{@{y}})} \nonumber\\
	\bangp_{x}{P} & := & \binpar{{x}!\langle{\binpar{D_{x}}{P}}\rangle}{D_{x}} \nonumber
\end{eqnarray}

\begin{eqnarray}
	\bangp_{x}{P} & & \nonumber\\
	=
	& {x}!\langle{(\prefix{x}{y}{(\outputp{x}{y} | @{y})) | P}}\rangle 
	      | \prefix{x}{y}{(\outputp{x}{y} | @{y})} & \nonumber\\
	\red
	& (\outputp{x}{y} | @{y})\substn{\quotep{(\prefix{x}{y}{(@{y} | \outputp{x}{y})) | P}}}{y} & \nonumber\\
	=
	& \outputp{x}{\quotep{(\prefix{x}{y}{(\outputp{x}{y} | @{y})) | P}}}
	  | {(\prefix{x}{y}{(\outputp{x}{y} | @{y})) | P}} & \nonumber\\
	\red
	& \ldots & \nonumber\\
	\red^*
	& P | P | \ldots & \nonumber
\end{eqnarray}

Of course, this encoding, as an implementation, runs away, unfolding
$\bangp{P}$ eagerly. A lazier and more implementable replication
operator, restricted to input-guarded processes, may be obtained as follows.

\begin{eqnarray}
\bangp{\prefix{u}{v}{P}} 
	:= 
	\binpar{\lift{x}{\prefix{u}{v}{(\binpar{D(x)}{P})}}}{D(x)} \nonumber
\end{eqnarray}

\begin{remark}
  Note that the lazier definition still does not deal with summation
  or mixed summation (i.e. sums over input and output). The reader is
  invited to construct definitions of replication that deal with these
  features. 

  Further, the definitions are parameterized in a name, $x$. Can you,
  gentle reader, make a definition that eliminates this parameter and
  guarantees no accidental interaction between the replication
  machinery and the process being replicated -- i.e. no accidental
  sharing of names used by the process to get its work done and the
  name(s) used by the replication to effect copying. This latter
  revision of the definition of replication is crucial to obtaining
  the expected identity $!!P \sim !P$.
\end{remark}

\begin{remark}\label{rem:paradoxical_combinator}
  The reader familiar with the lambda calculus will have noticed the
  similarity between $D$ and the paradoxical combinator.

  [Ed. note: the existence of this seems to suggest we have to be more
  restrictive on the set of processes and names we admit if we are to
  support no-cloning.]
\end{remark}

\subsubsection{Bisimulation}

The computational dynamics gives rise to another kind of equivalence,
the equivalence of computational behavior. As previously mentioned
this is typically captured \emph{via} some form of bisimulation.

% The notion we use in this paper is weak barbed bisimulation
% \cite{milner91polyadicpi}.

The notion we use in this paper is derived from weak barbed
bisimulation \cite{milner91polyadicpi}. 

\begin{definition}
An \emph{observation relation}, $\downarrow_{\mathcal N}$, over a set
of names, $\mathcal N$, is the smallest relation satisfying the rules
below.

\infrule[Out-barb]{y \in {\mathcal N}, \; x \nameeq y}
		  {\outputp{x}{v} \downarrow_{\mathcal N} x}
\infrule[Par-barb]{\mbox{$P\downarrow_{\mathcal N} x$ or $Q\downarrow_{\mathcal N} x$}}
		  {\binpar{P}{Q} \downarrow_{\mathcal N} x}

We write $P \Downarrow_{\mathcal N} x$ if there is $Q$ such that 
$P \wred Q$ and $Q \downarrow_{\mathcal N} x$.
\end{definition}

\begin{definition}
%\label{def.bbisim}
An  ${\mathcal N}$-\emph{barbed bisimulation} over a set of names, ${\mathcal N}$, is a symmetric binary relation 
${\mathcal S}_{\mathcal N}$ between agents such that $P\rel{S}_{\mathcal N}Q$ implies:
\begin{enumerate}
\item If $P \red P'$ then $Q \wred Q'$ and $P'\rel{S}_{\mathcal N} Q'$.
\item If $P\downarrow_{\mathcal N} x$, then $Q\Downarrow_{\mathcal N} x$.
\end{enumerate}
$P$ is ${\mathcal N}$-barbed bisimilar to $Q$, written
$P \wbbisim_{\mathcal N} Q$, if $P \rel{S}_{\mathcal N} Q$ for some ${\mathcal N}$-barbed bisimulation ${\mathcal S}_{\mathcal N}$.
\end{definition}

$\mathcal{R} \subseteq \pi \times \pi$

$P \mathcal{R} Q => \forall P'. P \red P' \Rightarrow \exists Q'. Q \red Q', P' \mathcal{R} Q'$

$P \vdash x \Rightarrow Q \vdash x$

\begin{mathpar}
  \inferrule*[lab=Out-barb]{x \nameeq y}{{y}!\langle{Q}\rangle \vdash x}
  \and
  \inferrule*[lab=Par-barb]{\mbox{$P\vdash x$ or $Q\vdash x$}}{\binpar{P}{Q} \vdash x}
\end{mathpar}

\subsubsection{Contexts}

One of the principle advantages of computational calculi like the
$\pi$-calculus is a well-defined notion of context,
contextual-equivalence and a correlation between
contextual-equivalence and notions of bisimulation. The notion of
context allows the decomposition of a process into (sub-)process and
its syntactic environment, its context. Thus, a context may be
thought of as a process with a ``hole'' (written $\Box$) in it. The
application of a context $M$ to a process $P$, written $M[P]$, is
tantamount to filling the hole in $M$ with $P$. In this paper we do
not need the full weight of this theory, but do make use of the notion
of context in the proof the main theorem. 

\begin{mathpar}
  \inferrule* [lab=summation] {} {{M_{M},M_{N}} \bc \Box \;|\; x.M_{A} \;|\; M_{M}+M_{N}}
  \and
  \inferrule* [lab=agent] {} {{M_{A}} \bc (\vec{x})M_{P} \;| \; \clift{P_0,\ldots,M_{P},\ldots,P_N}}
  \and \\
  \inferrule* [lab=process] {} {{M_{P}} \bc M_{N} \;| \;P|M_{P} }
\end{mathpar} 

\begin{mathpar}
  \inferrule* [lab=sychronization] {} {M_{N} \bc \Box \;|\; x?M_{F} \;|\; x!M_{C}}
  \and
  \inferrule* [lab=abstraction] {} {{M_{F}} \bc (x)M_{P} }
  \and
  \inferrule* [lab=concretion] {} {{M_{C}} \bc \langle M_{P} \rangle }
  \and \\
  \inferrule* [lab=process] {} {{M_{P}} \bc M_{N} \;| \;P|M_{P} }
\end{mathpar}

\begin{definition}[contextual application] Given a context $M$, and
  process $P$, we define the \emph{contextual application}, $M[P] :=
  M\{P/\Box\}$. That is, the contextual application of M to P is the
  substitution of $P$ for $\Box$ in $M$.
\end{definition}

$\meaningof{-} : L \to \mathcal{P}(\pi)$

\begin{mathpar}
  \inferrule* [lab=collection] {} {\meaningof{true} = \pi, \and \meaningof{~E} = \pi \setminus \meaningof{E}, \and \meaningof{E_{1} \& E_{2}} = \meaningof{E_{1}} \cap \meaningof{E_{2}}}
\end{mathpar}

\begin{mathpar}
  \inferrule* [lab=structure] {} {\meaningof{0} = \{ P \in \pi | P \equiv 0 \}, \and \\ \meaningof{E_1 | E_2} = \{ P \in \pi | P \equiv P_{1} | P_{2}, P_{1} \in \meaningof{E_{1}}, P_{2} \in \meaningof{E_2}\} }
\end{mathpar}

\begin{mathpar}
 \inferrule* [lab=behavior] {} {\meaningof{\langle a?b \rangle E} = \{ P \in \pi | P \equiv Q | u?(y)P', \\ \and \\\\ \and \\ \;\;\; u \in \meaningof{a}, \forall z.P'\{z/y\} \in \meaningof{E\{z/b\}}\}, \and \\ \meaningof{a!E} = \{ P \in \pi | P \equiv Q | x!\langle P' \rangle, x \in \meaningof{a} P' \in \meaningof{E}\} }
\end{mathpar}

\begin{mathpar}
 \inferrule* [lab=nominal] {} {\meaningof{\quotep{E}} = \{ \quotep{P} \in \quotep{\pi} | P \in \meaningof{E} \}, \and \meaningof{\quotep{P}} = \{ \quotep{Q} \in \quotep{\pi} | P \equiv Q \} \and \\ \meaningof{@\quotep{E}} = \{ P \in \pi | P \equiv @x, x \in \meaningof{E} \}}
\end{mathpar}

\begin{eqnarray*}
  \\
  \meaningof{-} : TS \to ST
\end{eqnarray*}

\begin{eqnarray*}
  \\
  L : TS \to ST
\end{eqnarray*}

\begin{eqnarray*}
  \\
  P \models E \iff P \in \meaningof{E}
\end{eqnarray*}

\begin{eqnarray*}
  P \approx_{L} Q \iff \forall E \in L. P \models E \iff Q \models E
\end{eqnarray*}

\begin{eqnarray*}
  P \approx_{K} Q
\end{eqnarray*}

\begin{eqnarray*}
  P \approx Q
\end{eqnarray*}

$\approx_{K} = \approx = \approx_{L}$

\subsubsection{Contextual duality}

Note that contexts extend the quotation operation to a family of
operations from processes to names. Given a context, $M$, we can
define a \emph{nominal context}, $\quotep{M}$ by $\quotep{M}[P] :=
\quotep{M[P]}$. To foreshadow what is to come we observe that these
operations enjoy a duality with processes very much like the duality
between vectors and maps from vectors to scalars.

Further, because the calculus is essentially higher-order, we have a
correspondence between contexts and processes. More specifically,
given a name $x$ and a context $M$ we can construct $M^{*}_{x}$ such
that 

\begin{mathpar}
  M^{*}_{x} | \lift{x}{P} \red M[P]
\end{mathpar}

namely,

\begin{mathpar}
  M^{*}_{x} := x?(u).M[\dropn{u}]
\end{mathpar}

The dependence of $M^{*}_{x}$ on a name makes it an abstraction, 

\begin{mathpar}
  M^{*} := (x)x?(u).M[\dropn{u}]
\end{mathpar}

\subsection{Additional notation}

It will sometimes be convenient to denote the process a name
quotes. We already have the notation $x = \quotep{P}$, but it will be
convenient to introduce an alternate notation, $\procn{x}$, when we
want to emphasize the connection to the use of the name. Note that, by
virtue of name equivalence, $\quotep{\procn{x}} \nameeq x$; so, the
notation is consistent with previous definitions.

Further, because names have structure it is possible to effect
substitutions on the basis of that structure. This means we need to
upgrade our notation for substitutions, which we accomplish by
adapting comprehension notation. Thus,

\begin{mathpar}
  P\{ y / x : x \in S \}
\end{mathpar}

is interpreted to mean the process derived from P by replacing (in a
capture-avoiding manner) each occurrence of $x$ in $S$ by $y$. For example,

\begin{mathpar}
  P\{ \quotep{\procn{x}|\procn{x}} / x : x \in \freenames{P} \}
\end{mathpar}

will replace each (occurrence) of a free name $x$ in $P$ by
$\quotep{\procn{x}|\procn{x}}$.

Also, we will avail ourselves of the notation $x^{L}$ and $x^{R}$ to
denote injections of a name into disjoint copies of the name
space. There are numerous ways to accomplish this. One example can be
found in \cite{MeredithR05}. This notation overloads to vectors of
names: $\vec{x}^{\pi} := (x_{i}^{\pi} \; : \; 0 \leq i < |\vec{x}| )$ where $\pi \in \{L,R\}$.

We also use $P^{\Box} := P|\Box$.

In \cite{MeredithR05} an interpretation of the new operator is
given. It turns out that there are several possible interpretations
all enjoying the requisite algebraic properties of the operator (see
\cite{milner91polyadicpi}). We will therefore make liberal use of
$(\nu\; \vec{x})P$.

% subsection the_syntax_and_semantics_of_the_notation_system (end)   

\section{Interpretation of QM}
\subsection{Supporting definitions}
\subsubsection{Multiplication}
\begin{mathpar}
  \quotep{Q} \cdot \quotep{R} := \quotep{Q|R}
  \and \\
  \quotep{Q} \cdot P := P\{ \quotep{Q|R} / \quotep{R} : \quotep{R} \in \freenames{P} \}
\end{mathpar}

\paragraph{Discussion}
The first line needs little explanation. The second line says that
each free name of the process is replaced with the multiplication of
that name by the scalar. Multiplication of a scalar (name) by a state
(process) results in a process all the names of which have been `moved
over' by parallel composition with the process the scalar
quotes. There is a subtlety that the bound names have to be
manipulated so that multiplied names aren't accidentally
captured. There are many ways to achieve this.

\begin{remark}\label{rem:multiplication_identities}
  The reader is invited to verify that for all $x,y,z \in \QProc$ and $P \in \Proc$
  \begin{mathpar}
    x \cdot \quotep{0} \equiv x 
    \and
    x \cdot y \equiv y \cdot x
    \and
    x \cdot (y \cdot z) \equiv (x \cdot y) \cdot z
    \and \\
    \quotep{0} \cdot P \equiv P
    \and \\
    x \cdot (y \cdot P) \equiv (x \cdot y) \cdot P
    \and \\
    x \cdot (P|Q) \equiv (x \cdot P) | (x \cdot Q)
    \and \\    
  \end{mathpar}
\end{remark}

\subsubsection{Tensor product}

We define a tensor product on processes by structural induction.

\paragraph{Tensor of sums} First note that all summations, including
$\pzero$ and sequence, can be written $\Sigma_{i} x_{i}.A_{i} +
\Sigma_{j} x_{j}.C_{j}$, where we have grouped input-guarded processes
together and output-guarded processes together.

Thus, we can define the tensor product of two summations, $N_{1}\otimes N_{2}$, where

\begin{mathpar}
  N_{1} := \Sigma_{i} x_{i}.A_{i} + \Sigma_{j} x_{j}.C_{j}
  \and
  N_{2} := \Sigma_{i'} y_{i'}.B_{i'} + \Sigma_{j'} y_{j'}.D_{j'} 
\end{mathpar}

as follows.

\begin{mathpar}
  \Sigma_{i} x_{i}.A_{i} + \Sigma_{j} x_{j}.C_{j} \otimes \Sigma_{i'}
  y_{i'}.B_{i'} + \Sigma_{j'} y_{j'}.D_{j'} 
  \and \\
  := \; \Sigma_{i} \Sigma_{i'} \quotep{\stackrel{\vee}{x_{i}}| \stackrel{\vee}{y_{i'}}}.(A_{i}\otimes B_{i'}) \; | \; \Sigma_{i'} \Sigma_{i} \quotep{\stackrel{\vee}{y_{i'}}|\stackrel{\vee}{x_{i}}}.(B_{i'}\otimes A_{i})
  \and
  \;\; | \;\; \Sigma_{j} \Sigma_{j'} \quotep{\stackrel{\vee}{x_{j}}|\stackrel{\vee}{y_{j'}}}.(A_{j}\otimes B_{j'}) \; | \; \Sigma_{j'} \Sigma_{j} \quotep{\stackrel{\vee}{y_{j'}}|\stackrel{\vee}{x_{j}}}.(B_{j'}\otimes A_{j})
\end{mathpar}

\begin{remark}
  Do we need to $x^{L}$ and $y^{R}$ for this construction as well?
\end{remark}

\paragraph{Tensor of parallel compositions} Next, we distribute tensor
over par.

\begin{mathpar}
  P_{1}|P_{2} \otimes Q_{1}|Q_{2} := (P_{1} \otimes Q_{1}) | (P_{1}
  \otimes Q_{2}) | (P_{2} \otimes Q_{1}) | (P_{2} \otimes Q_{2})
\end{mathpar}

\paragraph{Tensor with dropped names} We treat tensor of a
process with a dropped name as parallel composition.

\begin{mathpar}
  P \otimes \dropn{x} := P | \dropn{x}
\end{mathpar}

\paragraph{Tensor of agents}

Finally, we need to define tensor on agents. Note that the definition
of tensor on normal products only tensors inputs with inputs and
outputs with outputs. Thus, we only have to define the operation on
``homogeneous'' pairings.

\begin{mathpar}
  (\vec{x})P \otimes (\vec{y})Q
  \and \\
  := (x_{0}^{L}|y_{0}^{R},\ldots,x_{0}^{L}|y_{n}^{R},\ldots,x_{m}^{L}|y_{0}^{R},\ldots,x_{m}^{L}|y_{n}^R)(P\{ \vec{x}^{L}/\vec{x}\} \otimes Q \{ \vec{y}^{R}/\vec{y}\})
  \and \\
  \clift{\vec{P}} \otimes \clift{\vec{Q}}
  \and \\
  := \clift{P_{0}\otimes Q_{0},\ldots,P_{0}\otimes Q_{n},\ldots,P_{m}\otimes Q_{0},\ldots,P_{m}\otimes Q_{n}}
\end{mathpar}

\begin{remark}
  Observe that arities of tensored abstractions matches arities of
  tensored concretions if the original arities matched. Note also that
  the length of the arities corresponds to the increase in dimension
  we see in ordinary vector space tensor product.
\end{remark}

\begin{remark}
  Operationally, this definition distributes the tensor down to
  components ``linked'' by summation. Tensor over summation is
  intriguing in that it mixes names. Moreover, as a consequence of the
  way it mixes names we have the identities for all $x \in \QProc$ and
  $P,Q \in \Proc$

  \begin{mathpar}
    (x \cdot P) \otimes Q \equiv x \cdot (P \otimes Q) \equiv P \otimes (x \cdot Q)
    \and
    P \otimes \pzero \equiv P
  \end{mathpar}

  that the reader is invited to verify.
\end{remark}

\subsubsection{Annihilation}
\begin{mathpar}
  P^{\perp} := \{ Q | \forall R. P|Q \red^{*} R \Rightarrow R \red^{*} \pzero \}
  \and \\
  P^{\underline{\perp}} := \Sigma_{Q \in P^{\perp}} \quotep{Q}?(y).(\dropn{y}|Q) | \Sigma_{Q \in P^{\perp}} \quotep{Q}\clift{\Box}
\end{mathpar}

\paragraph{Discussion} The reader will note that $P^{\perp}$ is a
\emph{set} of processes, while $P^{\underline{\perp}}$ is a
\emph{context}. We call the set $P^{\perp}$ the \emph{annihilators} of
$P$. The parallel composition of a process in the annihilators of $P$
with $P$ will result in a process, the state space of which has all
paths eventually leading to $\pzero$. Execution may endure loops; but
under reasonable conditions of fairness (naturally guaranteed under
most notions of bisimulation) such a composite process cannot get
stuck in such a loop and will, eventually pop out and terminate.

The context $P^{\underline{\perp}}$ is ready and willing to ``take the
$P$ out of'' the process to which it is applied. It will effectively
transmit the code of the process to which it is applied to one of the
annihilators and run the process against it.

\subsubsection{Evaluation}
We fix $M$ a domain of fully abstract interpretation with an equality
coincident with bisimulation. We take $\meaningof{\cdot} : \Proc \to
M$ to be the map interpreting processes and $\nmeaningof{\cdot} : \M
\to Proc$ to be the map running the other way. Then we define

\begin{mathpar}
  \int P := \nmeaningof{\meaningof{P}}
\end{mathpar}

\paragraph{Discussion}
There are many fully abstract interpretations of Milner's
$\pi$-calculus. Any of them can be used as a basis for interpreting
the reflective calculus here. Equipped with such a domain it is
largely a matter of grinding through to check that the Yoneda
construction for the normalization-by-evaluation program can be
extended to this setting.

\begin{remark}
  The reader is invited to verify that $\int (P^{\underline{\perp}}[P]) = 0$.
\end{remark}

\subsection{Quantum mechanics}

Table \ref{tbl:core_qm_op_defns} gives the core operational definitions

\begin{table}[htp]\label{tbl:core_qm_op_defns}
  \center{
    \fbox{
      \begin{tabular}{c|c}
        quantum mechanics & process calculus \\
        \hline
        scalar & $x := \quotep{P}$ \\
        state vector & $\state{P} := P$ \\
        dual & $\state{P}^{*} := \event{P^{\underline{\perp}}} := \quotep{P^{\underline{\perp}}}[-]$ \\
        matrix & $ \Sigma_{\alpha} \state{P_{\alpha}}x_{\alpha}\event{Q_{\alpha}}$ \\
        vector addition & $\state{P} + \state{Q} := \state{P | Q}$ \\
        tensor product & $\state{P} \otimes \state{Q} := \state{P \otimes Q}$ \\
        inner product & $\innerprod{P}{Q} := \quotep{\int P^{\underline{\perp}}[Q]}$ \\
      \end{tabular}
    }
  }
  \caption{QM - operational definitions}
\end{table}

where

\begin{mathpar}
  \prmatrix{P}{Q} := \fprmatrix{P}{\quotep{\pzero}}{Q}
  \and
  \fprmatrix{P}{x}{Q} := (\state{P},x,\event{Q})
  \and
  (\fprmatrix{P}{x}{Q})(\state{R}) := x \cdot \innerprod{Q}{R} \cdot \state{P}
  \and
  (\fprmatrix{P}{x}{Q})(\event{R}) := x \cdot \innerprod{R}{P} \cdot \event{Q}
\end{mathpar}

\paragraph{Discussion}
As promised: vectors (aka states) are represented as processes; duals
as contextual duals; inner product definition should be compared with
standard inner product definition for ....

\begin{remark}
  Assuming $\int (P^{\underline{\perp}}[P]) = 0$, the reader is
  invited to verify that $(\fprmatrix{P}{x}{P})(\state{P}) = x \cdot \state{P}$.
\end{remark}

\begin{remark}
  The reader is invited to verify that $\innerprod{P}{Q}$ could
  equally well have been written $\quotep{\int \stackrel{\vee}{x}}$
  where $x = \event{P^{\underline{\perp}}}(Q)$.

  One of the motivations for this remark is that there is another way
  to factor these operations. We could package up evaluation in the dual:

  \begin{mathpar}
    \state{P}^{*} := \event{\int P^{\underline{\perp}}} := \quotep{\int P^{\underline{\perp}}}[-]
  \end{mathpar}

  and then have inner product defined by
  
  \begin{mathpar}
    \innerprod{P}{Q} := \event{P}(Q)
  \end{mathpar}

  Hopefully, experience with the calculations will provide guidance on
  the best factoring.
\end{remark}

\begin{remark}
  Assuming $\int (P^{\underline{\perp}}[P]) = 0$, the reader is
  invited to verify that $\forall P,Q. (\prmatrix{0}{Q})(\state{0}) =
  \state{0}$ and dually $(\prmatrix{P}{0})(\event{0}) = \event{0}$.
\end{remark}

\begin{remark}
  i'm a little worried that i don't (yet) have proper support for
  complex conjugacy. But, the observation above may give us a
  clue. According to Abramsky, it must be the case that the scalars
  are iso to the homset of the identity for the tensor -- which the
  observation above characterizes. 

  For now, we will simply bookmark the notion with $\overline{x}$.
\end{remark}

\subsubsection{Adjointness}

We need to give a definition of $(\cdot)^{\dagger}$ for matrices. The
obvious candidate definition is
\begin{mathpar}
(\Sigma_{\alpha}\fprmatrix{P_{\alpha}}{x_{\alpha}}{Q_{\alpha}})^{\dagger}
= \Sigma_{\alpha}\fprmatrix{(Q_{\alpha}^{\underline{\perp}})^{*}}{\overline{x}_{\alpha}}{P_{\alpha}^{\underline{\perp}}} 
\end{mathpar}

But, $(Q_{\alpha}^{\underline{\perp}})^{*}$ requires a name along
which to communicate the process to achieve the context application.

\subsubsection{Basis for a basis}
If processes label states and ``addition'' of states (a.k.a. vector
addition) is interpreted as parallel composition, what corresponds to
notions of linear independence and basis? Here, we recall that Yoshida
has developed a set of \emph{combinators} for an asynchronous verison
of Milner's $\pi$-calculus. These are a finite set of processes such
any process can be expressed as parallel composition of these
combinators together with liberal uses of the new operator and
replication. We can simply give a translation of these into the
present calculus and have reasonable expectation that the property
carries over. That is, that the resultant set allows to express all
processes via parallel composition. Note, however, that there is no
new operator or replication in this calculus. As a result, we expect
that the corresponding set is actually infinite. That is, we expect
that the space is actually infinite dimensional.

\begin{remark}
  The attentive reader may be a bit concerned. Certainly, the
  collection $S$, $K$ and $I$ is a finite set of
  combinators. Shouldn't we expect to see a finite set of combinators
  for an effectively equivalent system? i am very sympathetic to this
  critique and feel it warrants full attention. On the other hand, i
  also have in mind the following analogy. The natural numbers, as a
  monoid under addition, has exactly $1$ generator, while the natural
  numbers, as a monoid under multiplication, has countably many
  generators (the primes). We observe that the application of the
  lambda calculus is much less resource sensitive than the parallel
  composition of the $\pi$-calculus. Could it be the case that we have
  an analogy of the form
  
  \begin{mathpar}
    m + n : MN :: m*n : M|N
  \end{mathpar}

  giving a similar blow up in the set of ``primes''?  This is such a
  wonderful thought that, even if it's not true, i think it's worth
  writing down.
\end{remark}
 

\documentclass[12pt]{llncs}
%\documentclass{jktr}

\usepackage[pdftex]{hyperref}                   
\usepackage {listings}
\usepackage {mathpartir}
\usepackage{bcprules}
%\usepackage{listings}
                       
\usepackage{graphicx} 
%\usepackage[margins=2.5cm,nohead,nofoot]{geometry}
%\usepackage{geometry}
\usepackage{amsfonts}
\usepackage{amstext}
\usepackage{latexsym}
\usepackage{amssymb}
\usepackage{color}


%\include{myPreamble}
\include{qm2pi.local} 

%\ifpdf
%\usepackage[pdftex]{graphicx}
%\else
%\usepackage{graphicx}
%\fi

 % \ifpdf
%  \usepackage{pdfsync}
%  \if


%\title{Brief Article}
%\author{David F. Snyder}
%\author{L.G. Meredith}

%\address{Dept. of Math., Texas State University--San Marcos, San Marcos, TX 78666}
       
\pagestyle{empty}


\begin{document}

\lstset{language=[Objective]Caml,frame=shadowbox}

\input{qm2pi.front}

% section front matter (end)

\input{qm2pi.intro} 
 
% section introduction (end)

% \input{qm2pi.knotations} 

% section notation (end)

\input{qm2pi.process.calculi} 

% section concurrent_process_calculi_and_spatial_logics_ (end)
    
%\input{qm2pi.knots2pi} 

%\input{qm2pi.trefoil} 

%\input{qm2pi.mainthm} 

% subsection basic_interpretation (end)

%\input{qm2pi.rho.presentation} 
\subsection{The syntax and semantics of the notation system}\label{sub:the_syntax_and_semantics_of_the_notation_system} % (fold)

We now summarize a technical presentation of the calculus that
embodies our theory of dynamics. The typical presentation of such a
calculus follows the style of giving generators and relations on
them. The grammar, below, describing term constructors, freely
generates the set of processes, $\Proc$. This set is then quotiented
by a relation known as structural congruence and it is over this set
that the notion of dynamics is expressed. This presentation is
essentially that of \cite{MeredithR05} with the addition of
polyadicity and summation. For readability we have relegated some of
the technical subtleties to an appendix.

\subsubsection{Process grammar}\label{subsub:process_grammar}

\begin{mathpar}
  \inferrule* [lab=synchronization] {} {{M} \bc \pzero \;|\; x?F \;|\; x!C }
  \and
  \inferrule* [lab=abstraction] {} {{F} \bc (x)P}
  \and
  \inferrule* [lab=concretion] {} {{C} \bc \langle Q \rangle}
  \and
  \inferrule* [lab=process] {} {{P,Q} \bc M \;| \;P|Q \;|\; @{x}}
  \and
  \inferrule* [lab=name] {} {{x} \bc \quotep{P}}
\end{mathpar} 

Note that $\vec{x}$ (resp. $\vec{P}$) denotes a vector of names
(resp. processes) of length $|\vec{x}|$ (resp. $|\vec{P}|$). We adopt
the following useful abbreviations.

\begin{mathpar}
   x?(\vec{y}).P := x.(\vec{y})P \and  x\clift{\vec{P}} := x.\clift{\vec{P}}
   \and x!(y) := \lift{x}{\dropn{y}}
   \and \Pi_{i=0}^{n-1}P_i := P_0 | \ldots | P_{n-1}
\end{mathpar}

\subsubsection{Structural congruence}

\paragraph{Free and bound names and alpha-equivalence.} At the
core of structural equivalence is alpha-equivalence which identifies
process that are the same up to a change of variable. Formally, we
recognize the distinction between free and bound names. The free names
of a process, $\freenames{P}$, may be calculated recursively as
follows:

\begin{mathpar}
\freenames{\pzero} := \emptyset
  \and \\
  \freenames{x?(y).P} := \{ x \} \cup (\freenames{P} \setminus \{ y \})
  \and 
  \freenames{x!\langle P \rangle} := \{ x \} \cup \{ P \} 
  \and \\
  \freenames{P|Q} := \freenames{P} \cup \freenames{Q}
  \and \\
  \freenames{@{x}} := \{ x \}
\end{mathpar}

$\pi$
$\quotep{\pi}$

$\freenames{-} : \pi \to \mathcal{P}(\quotep{\pi})$

\begin{eqnarray*}
  \freenames{\pzero} & := & \emptyset \\
  \freenames{x?(y).P} & := & \{ x \} \cup (\freenames{P} \setminus \{ y \}) \\
  \freenames{x!\langle P \rangle} & := & \{ x \} \cup \{ P \} \\
  \freenames{P|Q} & := & \freenames{P} \cup \freenames{Q} \\
  \freenames{\dropn{x}} & := & \{ x \}
\end{eqnarray*}

The bound names of a process, $\boundnames{P}$, are those names occurring in $P$
that are not free. For example, in $x?(y).0$, the name $x$ is free, while $y$ is bound.

\begin{mathpar}
  \inferrule* [lab=monoidal-laws] {} { P|Q \equiv Q|P \and P|0 \equiv P \and P|(Q|R) \equiv (P|Q)|R }
\end{mathpar}

\begin{mathpar}
  \inferrule* [lab=alpha-equivalence] {} { (x)P \equiv (y)P\{y/x\} \and y \not\in \freenames{P} }
\end{mathpar}

\begin{definition}
Then two processes, $P,Q$, are alpha-equivalent if $P = Q\{\vec{y}/\vec{x}\}$ for
some $\vec{x} \in \boundnames{Q},\vec{y} \in \boundnames{P}$, where $Q\{\vec{y}/\vec{x}\}$
denotes the capture-avoiding substitution of $\vec{y}$ for $\vec{x}$ in $Q$.
\end{definition}

\begin{definition}
  The {\em structural congruence} \cite{SangiorgiWalker} , $\equiv$,
  between processes is the least congruence containing
  alpha-equivalence, satisfying the abelian monoid laws
  (associativity, commutativity and $\pzero$ as identity) for parallel
  composition $|$ and for summation $+$.
\end{definition}

\subsection{Name equivalence}

We take name equivalence, written $\nameeq$, to be the smallest
equivalence relation generated by the following rules.

\begin{mathpar}
\inferrule*[lab=Quote-drop]
{ }
{ \quotep{@{x}} \nameeq x }

\inferrule*[lab=Struct-equiv]
{ P \scong Q }
{ \quotep{P} \nameeq \quotep{Q} }
\end{mathpar}

The astute reader will have noticed that the mutual recursion of names
and processes imposes a mutual recursion on alpha-equivalence and
structural equivalence via name-equivalence. Fortunately, all of this
works out pleasantly and we may calculate in the natural way, free of
concern. The reader interested in the details is referred to the
appendix \ref{appendix:rho_details}.

\subsection{Substitution}

We use $\Proc$ for the set of processes, $\QProc$ for the set of
names, and $\id{\{}\vec{y} / \vec{x} \id{\}}$ to denote partial maps,
$s : \QProc \rightarrow \QProc$. A map, $s$ lifts, uniquely, to a map
on process terms, $\widehat{s} : \Proc \rightarrow \Proc$ by the
following equations.

\begin{mathpar}
  (0) \psubstp{Q}{P} := 0 \\
  (R \juxtap S) \psubstp{Q}{P}
  :=    
  (R)\psubstp{Q}{P} \juxtap (S) \psubstp{Q}{P} \\
  (x?(y).R) \psubstp{Q}{P}    
  :=    
  (x)\substp{Q}{P} (z)\concat( (R \psubstn{z}{y}) \psubstp{Q}{P} ) \\
  (\lift{x}{R}) \psubstp{Q}{P}  
  :=
  \lift{(x)\substp{Q}{P}}{ R \psubstp{Q}{P} } \\
%   (\dropn{x})  \psubstp{Q}{P}       
%   := 
%   \left\{ 
%     \begin{array}{ccc} 
%       \dropn{\quotep{Q}} & & x \nameeq \quotep{P} \\
%       \dropn{x} & & otherwise \\
%     \end{array}
%   \right. 
  (\dropn{x})  \psubstp{Q}{P}       
  := 
  \left\{ 
    \begin{array}{ccc} 
      Q & & x \nameeq \quotep{P} \\
      \dropn{x} & & otherwise \\
    \end{array}
  \right.
\end{mathpar}
 

where

\begin{eqnarray}
  (x)\id{\{} \lpquote Q \rpquote / \lpquote P \rpquote \id{\}}            = 
  \left\{ 
    \begin{array}{ccc}
      \lpquote Q \rpquote & & x \nameeq \lpquote P \rpquote \\
      x & & otherwise \\
    \end{array}
  \right. \nonumber
\end{eqnarray}

and $z$ is chosen distinct from $\quotep{P}$, $\quotep{Q}$, the free
names in $Q$, and all the names in $R$. Our $\alpha$-equivalence will
be built in the standard way from this substitution.

\begin{remark}\label{rem:no_self_referential_names}
  One consequence of these definitions is that $\forall P. \quotep{P}
  \not\in \freenames{P}$.
\end{remark}

\subsection{ Dynamic quote: an example }

Anticipating something of what's to come, consider applying the
substitution, $\widehat{\id{\{}u / z \id{\}}}$, to the following pair
of processes, $\lift{w}{y!(z)}$ and $w[ \lpquote y!(z) \rpquote ]$.

\begin{eqnarray}
	\lift{w}{y!(z)}\widehat{\id{\{}u / z \id{\}}}
		& = &
		\lift{w}{y!(u)} \nonumber\\
	w[ \lpquote y!(z) \rpquote ] \widehat{ \id{\{}u / z \id{\}} }
		& = &
		w[ \lpquote y!(z) \rpquote ] \nonumber
\end{eqnarray}

Because the body of the process between quotes is impervious to
substitution, we get radically different answers. In fact, by
examining the first process in an input context,
e.g. $x?(z).\lift{w}{y!(z)}$, we see that the process under the lift
operator may be shaped by prefixed inputs binding a name inside it. In
this sense, the lift operator will be seen as a way to dynamically
construct processes before reifying them as names.

Finally equipped with these standard features we can present the
dynamics of the calculus.

\subsubsection{Operational semantics} 

Finally, we introduce the computational dynamics. What marks these
algebras as distinct from other more traditionally studied algebraic
structures, e.g. vector spaces or polynomial rings, is the manner in
which dynamics is captured. In traditional structures, dynamics is typically
expressed through morphisms between such structures, as in linear maps
between vector spaces or morphisms between rings. In algebras
associated with the semantics of computation, the dynamics is
expressed as part of the algebraic structure itself, through a
reduction reduction relation typically denoted by $\red$. Below, we
give a recursive presentation of this relation for the calculus used
in the encoding.

$\red \subseteq \pi \times \pi$
$\red : \pi \to \mathcal{P}(\pi)$

\begin{mathpar}
  \inferrule* [lab=Comm] { \textsf{match}( x_{src}, x_{trgt} ) } { x_{trgt}?(y)P \; | \; x_{src}!\langle {Q} \rangle \red P\{\quotep{Q}/y}\} }
  \and \\
  \inferrule* [lab=Par] {{P} \red {P}'} {{{P} | {Q}} \red {{P}' | {Q}}}
  \and
  \inferrule* [lab=Equiv]{{{P} \scong {P}'} \andalso {{P}' \red {Q}'} \andalso {{Q}' \scong {Q}}}{{P} \red {Q}}
\end{mathpar}

\begin{eqnarray*}
  match_{\equiv} (\quotep{P},\quotep{Q}) & := & P \equiv Q \\
  match_{\dagger}(\quotep{P},\quotep{Q}) & := & \forall R. P|Q \red^{*} R => R \red^{*} 0 \\
  match_{K}(\quotep{P},\quotep{Q}) & := & K \mbox{ for some context } K
\end{eqnarray*}

$u?(x)P | u!\langle Q \rangle \red P\{\quotep{Q}/x\}$

%We write $\wred$ for $\red^*$, and $P\red$ if $\exists Q $ such that $ P \red Q$.
We write $P\red$ if $\exists Q $ such that $ P \red Q$ and $P\not\red$, otherwise.

\section{Replication}

As mentioned before, it is known that replication (and hence
recursion) can be implemented in a higher-order process algebra
\cite{SangiorgiWalker}. As our first example of calculation with the
machinery thus far presented we give the construction explicitly in
the {\rhoc}.

\begin{eqnarray}
	D_{x} & := & \prefix{x}{y}{(\binpar{\outputp{x}{y}}{@{y}})} \nonumber\\
	\bangp_{x}{P} & := & \binpar{{x}!\langle{\binpar{D_{x}}{P}}\rangle}{D_{x}} \nonumber
\end{eqnarray}

\begin{eqnarray}
	\bangp_{x}{P} & & \nonumber\\
	=
	& {x}!\langle{(\prefix{x}{y}{(\outputp{x}{y} | @{y})) | P}}\rangle 
	      | \prefix{x}{y}{(\outputp{x}{y} | @{y})} & \nonumber\\
	\red
	& (\outputp{x}{y} | @{y})\substn{\quotep{(\prefix{x}{y}{(@{y} | \outputp{x}{y})) | P}}}{y} & \nonumber\\
	=
	& \outputp{x}{\quotep{(\prefix{x}{y}{(\outputp{x}{y} | @{y})) | P}}}
	  | {(\prefix{x}{y}{(\outputp{x}{y} | @{y})) | P}} & \nonumber\\
	\red
	& \ldots & \nonumber\\
	\red^*
	& P | P | \ldots & \nonumber
\end{eqnarray}

Of course, this encoding, as an implementation, runs away, unfolding
$\bangp{P}$ eagerly. A lazier and more implementable replication
operator, restricted to input-guarded processes, may be obtained as follows.

\begin{eqnarray}
\bangp{\prefix{u}{v}{P}} 
	:= 
	\binpar{\lift{x}{\prefix{u}{v}{(\binpar{D(x)}{P})}}}{D(x)} \nonumber
\end{eqnarray}

\begin{remark}
  Note that the lazier definition still does not deal with summation
  or mixed summation (i.e. sums over input and output). The reader is
  invited to construct definitions of replication that deal with these
  features. 

  Further, the definitions are parameterized in a name, $x$. Can you,
  gentle reader, make a definition that eliminates this parameter and
  guarantees no accidental interaction between the replication
  machinery and the process being replicated -- i.e. no accidental
  sharing of names used by the process to get its work done and the
  name(s) used by the replication to effect copying. This latter
  revision of the definition of replication is crucial to obtaining
  the expected identity $!!P \sim !P$.
\end{remark}

\begin{remark}\label{rem:paradoxical_combinator}
  The reader familiar with the lambda calculus will have noticed the
  similarity between $D$ and the paradoxical combinator.

  [Ed. note: the existence of this seems to suggest we have to be more
  restrictive on the set of processes and names we admit if we are to
  support no-cloning.]
\end{remark}

\subsubsection{Bisimulation}

The computational dynamics gives rise to another kind of equivalence,
the equivalence of computational behavior. As previously mentioned
this is typically captured \emph{via} some form of bisimulation.

% The notion we use in this paper is weak barbed bisimulation
% \cite{milner91polyadicpi}.

The notion we use in this paper is derived from weak barbed
bisimulation \cite{milner91polyadicpi}. 

\begin{definition}
An \emph{observation relation}, $\downarrow_{\mathcal N}$, over a set
of names, $\mathcal N$, is the smallest relation satisfying the rules
below.

\infrule[Out-barb]{y \in {\mathcal N}, \; x \nameeq y}
		  {\outputp{x}{v} \downarrow_{\mathcal N} x}
\infrule[Par-barb]{\mbox{$P\downarrow_{\mathcal N} x$ or $Q\downarrow_{\mathcal N} x$}}
		  {\binpar{P}{Q} \downarrow_{\mathcal N} x}

We write $P \Downarrow_{\mathcal N} x$ if there is $Q$ such that 
$P \wred Q$ and $Q \downarrow_{\mathcal N} x$.
\end{definition}

\begin{definition}
%\label{def.bbisim}
An  ${\mathcal N}$-\emph{barbed bisimulation} over a set of names, ${\mathcal N}$, is a symmetric binary relation 
${\mathcal S}_{\mathcal N}$ between agents such that $P\rel{S}_{\mathcal N}Q$ implies:
\begin{enumerate}
\item If $P \red P'$ then $Q \wred Q'$ and $P'\rel{S}_{\mathcal N} Q'$.
\item If $P\downarrow_{\mathcal N} x$, then $Q\Downarrow_{\mathcal N} x$.
\end{enumerate}
$P$ is ${\mathcal N}$-barbed bisimilar to $Q$, written
$P \wbbisim_{\mathcal N} Q$, if $P \rel{S}_{\mathcal N} Q$ for some ${\mathcal N}$-barbed bisimulation ${\mathcal S}_{\mathcal N}$.
\end{definition}

$\mathcal{R} \subseteq \pi \times \pi$

$P \mathcal{R} Q => \forall P'. P \red P' \Rightarrow \exists Q'. Q \red Q', P' \mathcal{R} Q'$

$P \vdash x \Rightarrow Q \vdash x$

\begin{mathpar}
  \inferrule*[lab=Out-barb]{x \nameeq y}{{y}!\langle{Q}\rangle \vdash x}
  \and
  \inferrule*[lab=Par-barb]{\mbox{$P\vdash x$ or $Q\vdash x$}}{\binpar{P}{Q} \vdash x}
\end{mathpar}

\subsubsection{Contexts}

One of the principle advantages of computational calculi like the
$\pi$-calculus is a well-defined notion of context,
contextual-equivalence and a correlation between
contextual-equivalence and notions of bisimulation. The notion of
context allows the decomposition of a process into (sub-)process and
its syntactic environment, its context. Thus, a context may be
thought of as a process with a ``hole'' (written $\Box$) in it. The
application of a context $M$ to a process $P$, written $M[P]$, is
tantamount to filling the hole in $M$ with $P$. In this paper we do
not need the full weight of this theory, but do make use of the notion
of context in the proof the main theorem. 

\begin{mathpar}
  \inferrule* [lab=summation] {} {{M_{M},M_{N}} \bc \Box \;|\; x.M_{A} \;|\; M_{M}+M_{N}}
  \and
  \inferrule* [lab=agent] {} {{M_{A}} \bc (\vec{x})M_{P} \;| \; \clift{P_0,\ldots,M_{P},\ldots,P_N}}
  \and \\
  \inferrule* [lab=process] {} {{M_{P}} \bc M_{N} \;| \;P|M_{P} }
\end{mathpar} 

\begin{mathpar}
  \inferrule* [lab=sychronization] {} {M_{N} \bc \Box \;|\; x?M_{F} \;|\; x!M_{C}}
  \and
  \inferrule* [lab=abstraction] {} {{M_{F}} \bc (x)M_{P} }
  \and
  \inferrule* [lab=concretion] {} {{M_{C}} \bc \langle M_{P} \rangle }
  \and \\
  \inferrule* [lab=process] {} {{M_{P}} \bc M_{N} \;| \;P|M_{P} }
\end{mathpar}

\begin{definition}[contextual application] Given a context $M$, and
  process $P$, we define the \emph{contextual application}, $M[P] :=
  M\{P/\Box\}$. That is, the contextual application of M to P is the
  substitution of $P$ for $\Box$ in $M$.
\end{definition}

$\meaningof{-} : L \to \mathcal{P}(\pi)$

\begin{mathpar}
  \inferrule* [lab=collection] {} {\meaningof{true} = \pi, \and \meaningof{~E} = \pi \setminus \meaningof{E}, \and \meaningof{E_{1} \& E_{2}} = \meaningof{E_{1}} \cap \meaningof{E_{2}}}
\end{mathpar}

\begin{mathpar}
  \inferrule* [lab=structure] {} {\meaningof{0} = \{ P \in \pi | P \equiv 0 \}, \and \\ \meaningof{E_1 | E_2} = \{ P \in \pi | P \equiv P_{1} | P_{2}, P_{1} \in \meaningof{E_{1}}, P_{2} \in \meaningof{E_2}\} }
\end{mathpar}

\begin{mathpar}
 \inferrule* [lab=behavior] {} {\meaningof{\langle a?b \rangle E} = \{ P \in \pi | P \equiv Q | u?(y)P', \\ \and \\\\ \and \\ \;\;\; u \in \meaningof{a}, \forall z.P'\{z/y\} \in \meaningof{E\{z/b\}}\}, \and \\ \meaningof{a!E} = \{ P \in \pi | P \equiv Q | x!\langle P' \rangle, x \in \meaningof{a} P' \in \meaningof{E}\} }
\end{mathpar}

\begin{mathpar}
 \inferrule* [lab=nominal] {} {\meaningof{\quotep{E}} = \{ \quotep{P} \in \quotep{\pi} | P \in \meaningof{E} \}, \and \meaningof{\quotep{P}} = \{ \quotep{Q} \in \quotep{\pi} | P \equiv Q \} \and \\ \meaningof{@\quotep{E}} = \{ P \in \pi | P \equiv @x, x \in \meaningof{E} \}}
\end{mathpar}

\begin{eqnarray*}
  \\
  \meaningof{-} : TS \to ST
\end{eqnarray*}

\begin{eqnarray*}
  \\
  L : TS \to ST
\end{eqnarray*}

\begin{eqnarray*}
  \\
  P \models E \iff P \in \meaningof{E}
\end{eqnarray*}

\begin{eqnarray*}
  P \approx_{L} Q \iff \forall E \in L. P \models E \iff Q \models E
\end{eqnarray*}

\begin{eqnarray*}
  P \approx_{K} Q
\end{eqnarray*}

\begin{eqnarray*}
  P \approx Q
\end{eqnarray*}

$\approx_{K} = \approx = \approx_{L}$

\subsubsection{Contextual duality}

Note that contexts extend the quotation operation to a family of
operations from processes to names. Given a context, $M$, we can
define a \emph{nominal context}, $\quotep{M}$ by $\quotep{M}[P] :=
\quotep{M[P]}$. To foreshadow what is to come we observe that these
operations enjoy a duality with processes very much like the duality
between vectors and maps from vectors to scalars.

Further, because the calculus is essentially higher-order, we have a
correspondence between contexts and processes. More specifically,
given a name $x$ and a context $M$ we can construct $M^{*}_{x}$ such
that 

\begin{mathpar}
  M^{*}_{x} | \lift{x}{P} \red M[P]
\end{mathpar}

namely,

\begin{mathpar}
  M^{*}_{x} := x?(u).M[\dropn{u}]
\end{mathpar}

The dependence of $M^{*}_{x}$ on a name makes it an abstraction, 

\begin{mathpar}
  M^{*} := (x)x?(u).M[\dropn{u}]
\end{mathpar}

\subsection{Additional notation}

It will sometimes be convenient to denote the process a name
quotes. We already have the notation $x = \quotep{P}$, but it will be
convenient to introduce an alternate notation, $\procn{x}$, when we
want to emphasize the connection to the use of the name. Note that, by
virtue of name equivalence, $\quotep{\procn{x}} \nameeq x$; so, the
notation is consistent with previous definitions.

Further, because names have structure it is possible to effect
substitutions on the basis of that structure. This means we need to
upgrade our notation for substitutions, which we accomplish by
adapting comprehension notation. Thus,

\begin{mathpar}
  P\{ y / x : x \in S \}
\end{mathpar}

is interpreted to mean the process derived from P by replacing (in a
capture-avoiding manner) each occurrence of $x$ in $S$ by $y$. For example,

\begin{mathpar}
  P\{ \quotep{\procn{x}|\procn{x}} / x : x \in \freenames{P} \}
\end{mathpar}

will replace each (occurrence) of a free name $x$ in $P$ by
$\quotep{\procn{x}|\procn{x}}$.

Also, we will avail ourselves of the notation $x^{L}$ and $x^{R}$ to
denote injections of a name into disjoint copies of the name
space. There are numerous ways to accomplish this. One example can be
found in \cite{MeredithR05}. This notation overloads to vectors of
names: $\vec{x}^{\pi} := (x_{i}^{\pi} \; : \; 0 \leq i < |\vec{x}| )$ where $\pi \in \{L,R\}$.

We also use $P^{\Box} := P|\Box$.

In \cite{MeredithR05} an interpretation of the new operator is
given. It turns out that there are several possible interpretations
all enjoying the requisite algebraic properties of the operator (see
\cite{milner91polyadicpi}). We will therefore make liberal use of
$(\nu\; \vec{x})P$.

% subsection the_syntax_and_semantics_of_the_notation_system (end)   

\input{qm2pi.qmops} 

\input{qm2pi.sterngerlach} 

\input{qm2pi.metric} 

% section concurrent_process_calculi (end)

%\input{qm2pi.proofsketch}

% section proof sketch (end)

%\input{qm2pi.slviaknots} 

% section spatial logic via knots (end)

\input{qm2pi.conclusion}

% section conclusion (end)

%\input{qm2pi.dtcodes} 

% section wiring algorithm (end)

\input{qm2pi.ack} 

% section acknowledgments (end)

\newpage


\bibliographystyle{plain}   
\bibliography{../../biblios/main.bib}

\input{qm2pi.rhodetails}

\end{document}

 

\documentclass[12pt]{llncs}
%\documentclass{jktr}

\usepackage[pdftex]{hyperref}                   
\usepackage {listings}
\usepackage {mathpartir}
\usepackage{bcprules}
%\usepackage{listings}
                       
\usepackage{graphicx} 
%\usepackage[margins=2.5cm,nohead,nofoot]{geometry}
%\usepackage{geometry}
\usepackage{amsfonts}
\usepackage{amstext}
\usepackage{latexsym}
\usepackage{amssymb}
\usepackage{color}


%\include{myPreamble}
\include{qm2pi.local} 

%\ifpdf
%\usepackage[pdftex]{graphicx}
%\else
%\usepackage{graphicx}
%\fi

 % \ifpdf
%  \usepackage{pdfsync}
%  \if


%\title{Brief Article}
%\author{David F. Snyder}
%\author{L.G. Meredith}

%\address{Dept. of Math., Texas State University--San Marcos, San Marcos, TX 78666}
       
\pagestyle{empty}


\begin{document}

\lstset{language=[Objective]Caml,frame=shadowbox}

\input{qm2pi.front}

% section front matter (end)

\input{qm2pi.intro} 
 
% section introduction (end)

% \input{qm2pi.knotations} 

% section notation (end)

\input{qm2pi.process.calculi} 

% section concurrent_process_calculi_and_spatial_logics_ (end)
    
%\input{qm2pi.knots2pi} 

%\input{qm2pi.trefoil} 

%\input{qm2pi.mainthm} 

% subsection basic_interpretation (end)

%\input{qm2pi.rho.presentation} 
\subsection{The syntax and semantics of the notation system}\label{sub:the_syntax_and_semantics_of_the_notation_system} % (fold)

We now summarize a technical presentation of the calculus that
embodies our theory of dynamics. The typical presentation of such a
calculus follows the style of giving generators and relations on
them. The grammar, below, describing term constructors, freely
generates the set of processes, $\Proc$. This set is then quotiented
by a relation known as structural congruence and it is over this set
that the notion of dynamics is expressed. This presentation is
essentially that of \cite{MeredithR05} with the addition of
polyadicity and summation. For readability we have relegated some of
the technical subtleties to an appendix.

\subsubsection{Process grammar}\label{subsub:process_grammar}

\begin{mathpar}
  \inferrule* [lab=synchronization] {} {{M} \bc \pzero \;|\; x?F \;|\; x!C }
  \and
  \inferrule* [lab=abstraction] {} {{F} \bc (x)P}
  \and
  \inferrule* [lab=concretion] {} {{C} \bc \langle Q \rangle}
  \and
  \inferrule* [lab=process] {} {{P,Q} \bc M \;| \;P|Q \;|\; @{x}}
  \and
  \inferrule* [lab=name] {} {{x} \bc \quotep{P}}
\end{mathpar} 

Note that $\vec{x}$ (resp. $\vec{P}$) denotes a vector of names
(resp. processes) of length $|\vec{x}|$ (resp. $|\vec{P}|$). We adopt
the following useful abbreviations.

\begin{mathpar}
   x?(\vec{y}).P := x.(\vec{y})P \and  x\clift{\vec{P}} := x.\clift{\vec{P}}
   \and x!(y) := \lift{x}{\dropn{y}}
   \and \Pi_{i=0}^{n-1}P_i := P_0 | \ldots | P_{n-1}
\end{mathpar}

\subsubsection{Structural congruence}

\paragraph{Free and bound names and alpha-equivalence.} At the
core of structural equivalence is alpha-equivalence which identifies
process that are the same up to a change of variable. Formally, we
recognize the distinction between free and bound names. The free names
of a process, $\freenames{P}$, may be calculated recursively as
follows:

\begin{mathpar}
\freenames{\pzero} := \emptyset
  \and \\
  \freenames{x?(y).P} := \{ x \} \cup (\freenames{P} \setminus \{ y \})
  \and 
  \freenames{x!\langle P \rangle} := \{ x \} \cup \{ P \} 
  \and \\
  \freenames{P|Q} := \freenames{P} \cup \freenames{Q}
  \and \\
  \freenames{@{x}} := \{ x \}
\end{mathpar}

$\pi$
$\quotep{\pi}$

$\freenames{-} : \pi \to \mathcal{P}(\quotep{\pi})$

\begin{eqnarray*}
  \freenames{\pzero} & := & \emptyset \\
  \freenames{x?(y).P} & := & \{ x \} \cup (\freenames{P} \setminus \{ y \}) \\
  \freenames{x!\langle P \rangle} & := & \{ x \} \cup \{ P \} \\
  \freenames{P|Q} & := & \freenames{P} \cup \freenames{Q} \\
  \freenames{\dropn{x}} & := & \{ x \}
\end{eqnarray*}

The bound names of a process, $\boundnames{P}$, are those names occurring in $P$
that are not free. For example, in $x?(y).0$, the name $x$ is free, while $y$ is bound.

\begin{mathpar}
  \inferrule* [lab=monoidal-laws] {} { P|Q \equiv Q|P \and P|0 \equiv P \and P|(Q|R) \equiv (P|Q)|R }
\end{mathpar}

\begin{mathpar}
  \inferrule* [lab=alpha-equivalence] {} { (x)P \equiv (y)P\{y/x\} \and y \not\in \freenames{P} }
\end{mathpar}

\begin{definition}
Then two processes, $P,Q$, are alpha-equivalent if $P = Q\{\vec{y}/\vec{x}\}$ for
some $\vec{x} \in \boundnames{Q},\vec{y} \in \boundnames{P}$, where $Q\{\vec{y}/\vec{x}\}$
denotes the capture-avoiding substitution of $\vec{y}$ for $\vec{x}$ in $Q$.
\end{definition}

\begin{definition}
  The {\em structural congruence} \cite{SangiorgiWalker} , $\equiv$,
  between processes is the least congruence containing
  alpha-equivalence, satisfying the abelian monoid laws
  (associativity, commutativity and $\pzero$ as identity) for parallel
  composition $|$ and for summation $+$.
\end{definition}

\subsection{Name equivalence}

We take name equivalence, written $\nameeq$, to be the smallest
equivalence relation generated by the following rules.

\begin{mathpar}
\inferrule*[lab=Quote-drop]
{ }
{ \quotep{@{x}} \nameeq x }

\inferrule*[lab=Struct-equiv]
{ P \scong Q }
{ \quotep{P} \nameeq \quotep{Q} }
\end{mathpar}

The astute reader will have noticed that the mutual recursion of names
and processes imposes a mutual recursion on alpha-equivalence and
structural equivalence via name-equivalence. Fortunately, all of this
works out pleasantly and we may calculate in the natural way, free of
concern. The reader interested in the details is referred to the
appendix \ref{appendix:rho_details}.

\subsection{Substitution}

We use $\Proc$ for the set of processes, $\QProc$ for the set of
names, and $\id{\{}\vec{y} / \vec{x} \id{\}}$ to denote partial maps,
$s : \QProc \rightarrow \QProc$. A map, $s$ lifts, uniquely, to a map
on process terms, $\widehat{s} : \Proc \rightarrow \Proc$ by the
following equations.

\begin{mathpar}
  (0) \psubstp{Q}{P} := 0 \\
  (R \juxtap S) \psubstp{Q}{P}
  :=    
  (R)\psubstp{Q}{P} \juxtap (S) \psubstp{Q}{P} \\
  (x?(y).R) \psubstp{Q}{P}    
  :=    
  (x)\substp{Q}{P} (z)\concat( (R \psubstn{z}{y}) \psubstp{Q}{P} ) \\
  (\lift{x}{R}) \psubstp{Q}{P}  
  :=
  \lift{(x)\substp{Q}{P}}{ R \psubstp{Q}{P} } \\
%   (\dropn{x})  \psubstp{Q}{P}       
%   := 
%   \left\{ 
%     \begin{array}{ccc} 
%       \dropn{\quotep{Q}} & & x \nameeq \quotep{P} \\
%       \dropn{x} & & otherwise \\
%     \end{array}
%   \right. 
  (\dropn{x})  \psubstp{Q}{P}       
  := 
  \left\{ 
    \begin{array}{ccc} 
      Q & & x \nameeq \quotep{P} \\
      \dropn{x} & & otherwise \\
    \end{array}
  \right.
\end{mathpar}
 

where

\begin{eqnarray}
  (x)\id{\{} \lpquote Q \rpquote / \lpquote P \rpquote \id{\}}            = 
  \left\{ 
    \begin{array}{ccc}
      \lpquote Q \rpquote & & x \nameeq \lpquote P \rpquote \\
      x & & otherwise \\
    \end{array}
  \right. \nonumber
\end{eqnarray}

and $z$ is chosen distinct from $\quotep{P}$, $\quotep{Q}$, the free
names in $Q$, and all the names in $R$. Our $\alpha$-equivalence will
be built in the standard way from this substitution.

\begin{remark}\label{rem:no_self_referential_names}
  One consequence of these definitions is that $\forall P. \quotep{P}
  \not\in \freenames{P}$.
\end{remark}

\subsection{ Dynamic quote: an example }

Anticipating something of what's to come, consider applying the
substitution, $\widehat{\id{\{}u / z \id{\}}}$, to the following pair
of processes, $\lift{w}{y!(z)}$ and $w[ \lpquote y!(z) \rpquote ]$.

\begin{eqnarray}
	\lift{w}{y!(z)}\widehat{\id{\{}u / z \id{\}}}
		& = &
		\lift{w}{y!(u)} \nonumber\\
	w[ \lpquote y!(z) \rpquote ] \widehat{ \id{\{}u / z \id{\}} }
		& = &
		w[ \lpquote y!(z) \rpquote ] \nonumber
\end{eqnarray}

Because the body of the process between quotes is impervious to
substitution, we get radically different answers. In fact, by
examining the first process in an input context,
e.g. $x?(z).\lift{w}{y!(z)}$, we see that the process under the lift
operator may be shaped by prefixed inputs binding a name inside it. In
this sense, the lift operator will be seen as a way to dynamically
construct processes before reifying them as names.

Finally equipped with these standard features we can present the
dynamics of the calculus.

\subsubsection{Operational semantics} 

Finally, we introduce the computational dynamics. What marks these
algebras as distinct from other more traditionally studied algebraic
structures, e.g. vector spaces or polynomial rings, is the manner in
which dynamics is captured. In traditional structures, dynamics is typically
expressed through morphisms between such structures, as in linear maps
between vector spaces or morphisms between rings. In algebras
associated with the semantics of computation, the dynamics is
expressed as part of the algebraic structure itself, through a
reduction reduction relation typically denoted by $\red$. Below, we
give a recursive presentation of this relation for the calculus used
in the encoding.

$\red \subseteq \pi \times \pi$
$\red : \pi \to \mathcal{P}(\pi)$

\begin{mathpar}
  \inferrule* [lab=Comm] { \textsf{match}( x_{src}, x_{trgt} ) } { x_{trgt}?(y)P \; | \; x_{src}!\langle {Q} \rangle \red P\{\quotep{Q}/y}\} }
  \and \\
  \inferrule* [lab=Par] {{P} \red {P}'} {{{P} | {Q}} \red {{P}' | {Q}}}
  \and
  \inferrule* [lab=Equiv]{{{P} \scong {P}'} \andalso {{P}' \red {Q}'} \andalso {{Q}' \scong {Q}}}{{P} \red {Q}}
\end{mathpar}

\begin{eqnarray*}
  match_{\equiv} (\quotep{P},\quotep{Q}) & := & P \equiv Q \\
  match_{\dagger}(\quotep{P},\quotep{Q}) & := & \forall R. P|Q \red^{*} R => R \red^{*} 0 \\
  match_{K}(\quotep{P},\quotep{Q}) & := & K \mbox{ for some context } K
\end{eqnarray*}

$u?(x)P | u!\langle Q \rangle \red P\{\quotep{Q}/x\}$

%We write $\wred$ for $\red^*$, and $P\red$ if $\exists Q $ such that $ P \red Q$.
We write $P\red$ if $\exists Q $ such that $ P \red Q$ and $P\not\red$, otherwise.

\section{Replication}

As mentioned before, it is known that replication (and hence
recursion) can be implemented in a higher-order process algebra
\cite{SangiorgiWalker}. As our first example of calculation with the
machinery thus far presented we give the construction explicitly in
the {\rhoc}.

\begin{eqnarray}
	D_{x} & := & \prefix{x}{y}{(\binpar{\outputp{x}{y}}{@{y}})} \nonumber\\
	\bangp_{x}{P} & := & \binpar{{x}!\langle{\binpar{D_{x}}{P}}\rangle}{D_{x}} \nonumber
\end{eqnarray}

\begin{eqnarray}
	\bangp_{x}{P} & & \nonumber\\
	=
	& {x}!\langle{(\prefix{x}{y}{(\outputp{x}{y} | @{y})) | P}}\rangle 
	      | \prefix{x}{y}{(\outputp{x}{y} | @{y})} & \nonumber\\
	\red
	& (\outputp{x}{y} | @{y})\substn{\quotep{(\prefix{x}{y}{(@{y} | \outputp{x}{y})) | P}}}{y} & \nonumber\\
	=
	& \outputp{x}{\quotep{(\prefix{x}{y}{(\outputp{x}{y} | @{y})) | P}}}
	  | {(\prefix{x}{y}{(\outputp{x}{y} | @{y})) | P}} & \nonumber\\
	\red
	& \ldots & \nonumber\\
	\red^*
	& P | P | \ldots & \nonumber
\end{eqnarray}

Of course, this encoding, as an implementation, runs away, unfolding
$\bangp{P}$ eagerly. A lazier and more implementable replication
operator, restricted to input-guarded processes, may be obtained as follows.

\begin{eqnarray}
\bangp{\prefix{u}{v}{P}} 
	:= 
	\binpar{\lift{x}{\prefix{u}{v}{(\binpar{D(x)}{P})}}}{D(x)} \nonumber
\end{eqnarray}

\begin{remark}
  Note that the lazier definition still does not deal with summation
  or mixed summation (i.e. sums over input and output). The reader is
  invited to construct definitions of replication that deal with these
  features. 

  Further, the definitions are parameterized in a name, $x$. Can you,
  gentle reader, make a definition that eliminates this parameter and
  guarantees no accidental interaction between the replication
  machinery and the process being replicated -- i.e. no accidental
  sharing of names used by the process to get its work done and the
  name(s) used by the replication to effect copying. This latter
  revision of the definition of replication is crucial to obtaining
  the expected identity $!!P \sim !P$.
\end{remark}

\begin{remark}\label{rem:paradoxical_combinator}
  The reader familiar with the lambda calculus will have noticed the
  similarity between $D$ and the paradoxical combinator.

  [Ed. note: the existence of this seems to suggest we have to be more
  restrictive on the set of processes and names we admit if we are to
  support no-cloning.]
\end{remark}

\subsubsection{Bisimulation}

The computational dynamics gives rise to another kind of equivalence,
the equivalence of computational behavior. As previously mentioned
this is typically captured \emph{via} some form of bisimulation.

% The notion we use in this paper is weak barbed bisimulation
% \cite{milner91polyadicpi}.

The notion we use in this paper is derived from weak barbed
bisimulation \cite{milner91polyadicpi}. 

\begin{definition}
An \emph{observation relation}, $\downarrow_{\mathcal N}$, over a set
of names, $\mathcal N$, is the smallest relation satisfying the rules
below.

\infrule[Out-barb]{y \in {\mathcal N}, \; x \nameeq y}
		  {\outputp{x}{v} \downarrow_{\mathcal N} x}
\infrule[Par-barb]{\mbox{$P\downarrow_{\mathcal N} x$ or $Q\downarrow_{\mathcal N} x$}}
		  {\binpar{P}{Q} \downarrow_{\mathcal N} x}

We write $P \Downarrow_{\mathcal N} x$ if there is $Q$ such that 
$P \wred Q$ and $Q \downarrow_{\mathcal N} x$.
\end{definition}

\begin{definition}
%\label{def.bbisim}
An  ${\mathcal N}$-\emph{barbed bisimulation} over a set of names, ${\mathcal N}$, is a symmetric binary relation 
${\mathcal S}_{\mathcal N}$ between agents such that $P\rel{S}_{\mathcal N}Q$ implies:
\begin{enumerate}
\item If $P \red P'$ then $Q \wred Q'$ and $P'\rel{S}_{\mathcal N} Q'$.
\item If $P\downarrow_{\mathcal N} x$, then $Q\Downarrow_{\mathcal N} x$.
\end{enumerate}
$P$ is ${\mathcal N}$-barbed bisimilar to $Q$, written
$P \wbbisim_{\mathcal N} Q$, if $P \rel{S}_{\mathcal N} Q$ for some ${\mathcal N}$-barbed bisimulation ${\mathcal S}_{\mathcal N}$.
\end{definition}

$\mathcal{R} \subseteq \pi \times \pi$

$P \mathcal{R} Q => \forall P'. P \red P' \Rightarrow \exists Q'. Q \red Q', P' \mathcal{R} Q'$

$P \vdash x \Rightarrow Q \vdash x$

\begin{mathpar}
  \inferrule*[lab=Out-barb]{x \nameeq y}{{y}!\langle{Q}\rangle \vdash x}
  \and
  \inferrule*[lab=Par-barb]{\mbox{$P\vdash x$ or $Q\vdash x$}}{\binpar{P}{Q} \vdash x}
\end{mathpar}

\subsubsection{Contexts}

One of the principle advantages of computational calculi like the
$\pi$-calculus is a well-defined notion of context,
contextual-equivalence and a correlation between
contextual-equivalence and notions of bisimulation. The notion of
context allows the decomposition of a process into (sub-)process and
its syntactic environment, its context. Thus, a context may be
thought of as a process with a ``hole'' (written $\Box$) in it. The
application of a context $M$ to a process $P$, written $M[P]$, is
tantamount to filling the hole in $M$ with $P$. In this paper we do
not need the full weight of this theory, but do make use of the notion
of context in the proof the main theorem. 

\begin{mathpar}
  \inferrule* [lab=summation] {} {{M_{M},M_{N}} \bc \Box \;|\; x.M_{A} \;|\; M_{M}+M_{N}}
  \and
  \inferrule* [lab=agent] {} {{M_{A}} \bc (\vec{x})M_{P} \;| \; \clift{P_0,\ldots,M_{P},\ldots,P_N}}
  \and \\
  \inferrule* [lab=process] {} {{M_{P}} \bc M_{N} \;| \;P|M_{P} }
\end{mathpar} 

\begin{mathpar}
  \inferrule* [lab=sychronization] {} {M_{N} \bc \Box \;|\; x?M_{F} \;|\; x!M_{C}}
  \and
  \inferrule* [lab=abstraction] {} {{M_{F}} \bc (x)M_{P} }
  \and
  \inferrule* [lab=concretion] {} {{M_{C}} \bc \langle M_{P} \rangle }
  \and \\
  \inferrule* [lab=process] {} {{M_{P}} \bc M_{N} \;| \;P|M_{P} }
\end{mathpar}

\begin{definition}[contextual application] Given a context $M$, and
  process $P$, we define the \emph{contextual application}, $M[P] :=
  M\{P/\Box\}$. That is, the contextual application of M to P is the
  substitution of $P$ for $\Box$ in $M$.
\end{definition}

$\meaningof{-} : L \to \mathcal{P}(\pi)$

\begin{mathpar}
  \inferrule* [lab=collection] {} {\meaningof{true} = \pi, \and \meaningof{~E} = \pi \setminus \meaningof{E}, \and \meaningof{E_{1} \& E_{2}} = \meaningof{E_{1}} \cap \meaningof{E_{2}}}
\end{mathpar}

\begin{mathpar}
  \inferrule* [lab=structure] {} {\meaningof{0} = \{ P \in \pi | P \equiv 0 \}, \and \\ \meaningof{E_1 | E_2} = \{ P \in \pi | P \equiv P_{1} | P_{2}, P_{1} \in \meaningof{E_{1}}, P_{2} \in \meaningof{E_2}\} }
\end{mathpar}

\begin{mathpar}
 \inferrule* [lab=behavior] {} {\meaningof{\langle a?b \rangle E} = \{ P \in \pi | P \equiv Q | u?(y)P', \\ \and \\\\ \and \\ \;\;\; u \in \meaningof{a}, \forall z.P'\{z/y\} \in \meaningof{E\{z/b\}}\}, \and \\ \meaningof{a!E} = \{ P \in \pi | P \equiv Q | x!\langle P' \rangle, x \in \meaningof{a} P' \in \meaningof{E}\} }
\end{mathpar}

\begin{mathpar}
 \inferrule* [lab=nominal] {} {\meaningof{\quotep{E}} = \{ \quotep{P} \in \quotep{\pi} | P \in \meaningof{E} \}, \and \meaningof{\quotep{P}} = \{ \quotep{Q} \in \quotep{\pi} | P \equiv Q \} \and \\ \meaningof{@\quotep{E}} = \{ P \in \pi | P \equiv @x, x \in \meaningof{E} \}}
\end{mathpar}

\begin{eqnarray*}
  \\
  \meaningof{-} : TS \to ST
\end{eqnarray*}

\begin{eqnarray*}
  \\
  L : TS \to ST
\end{eqnarray*}

\begin{eqnarray*}
  \\
  P \models E \iff P \in \meaningof{E}
\end{eqnarray*}

\begin{eqnarray*}
  P \approx_{L} Q \iff \forall E \in L. P \models E \iff Q \models E
\end{eqnarray*}

\begin{eqnarray*}
  P \approx_{K} Q
\end{eqnarray*}

\begin{eqnarray*}
  P \approx Q
\end{eqnarray*}

$\approx_{K} = \approx = \approx_{L}$

\subsubsection{Contextual duality}

Note that contexts extend the quotation operation to a family of
operations from processes to names. Given a context, $M$, we can
define a \emph{nominal context}, $\quotep{M}$ by $\quotep{M}[P] :=
\quotep{M[P]}$. To foreshadow what is to come we observe that these
operations enjoy a duality with processes very much like the duality
between vectors and maps from vectors to scalars.

Further, because the calculus is essentially higher-order, we have a
correspondence between contexts and processes. More specifically,
given a name $x$ and a context $M$ we can construct $M^{*}_{x}$ such
that 

\begin{mathpar}
  M^{*}_{x} | \lift{x}{P} \red M[P]
\end{mathpar}

namely,

\begin{mathpar}
  M^{*}_{x} := x?(u).M[\dropn{u}]
\end{mathpar}

The dependence of $M^{*}_{x}$ on a name makes it an abstraction, 

\begin{mathpar}
  M^{*} := (x)x?(u).M[\dropn{u}]
\end{mathpar}

\subsection{Additional notation}

It will sometimes be convenient to denote the process a name
quotes. We already have the notation $x = \quotep{P}$, but it will be
convenient to introduce an alternate notation, $\procn{x}$, when we
want to emphasize the connection to the use of the name. Note that, by
virtue of name equivalence, $\quotep{\procn{x}} \nameeq x$; so, the
notation is consistent with previous definitions.

Further, because names have structure it is possible to effect
substitutions on the basis of that structure. This means we need to
upgrade our notation for substitutions, which we accomplish by
adapting comprehension notation. Thus,

\begin{mathpar}
  P\{ y / x : x \in S \}
\end{mathpar}

is interpreted to mean the process derived from P by replacing (in a
capture-avoiding manner) each occurrence of $x$ in $S$ by $y$. For example,

\begin{mathpar}
  P\{ \quotep{\procn{x}|\procn{x}} / x : x \in \freenames{P} \}
\end{mathpar}

will replace each (occurrence) of a free name $x$ in $P$ by
$\quotep{\procn{x}|\procn{x}}$.

Also, we will avail ourselves of the notation $x^{L}$ and $x^{R}$ to
denote injections of a name into disjoint copies of the name
space. There are numerous ways to accomplish this. One example can be
found in \cite{MeredithR05}. This notation overloads to vectors of
names: $\vec{x}^{\pi} := (x_{i}^{\pi} \; : \; 0 \leq i < |\vec{x}| )$ where $\pi \in \{L,R\}$.

We also use $P^{\Box} := P|\Box$.

In \cite{MeredithR05} an interpretation of the new operator is
given. It turns out that there are several possible interpretations
all enjoying the requisite algebraic properties of the operator (see
\cite{milner91polyadicpi}). We will therefore make liberal use of
$(\nu\; \vec{x})P$.

% subsection the_syntax_and_semantics_of_the_notation_system (end)   

\input{qm2pi.qmops} 

\input{qm2pi.sterngerlach} 

\input{qm2pi.metric} 

% section concurrent_process_calculi (end)

%\input{qm2pi.proofsketch}

% section proof sketch (end)

%\input{qm2pi.slviaknots} 

% section spatial logic via knots (end)

\input{qm2pi.conclusion}

% section conclusion (end)

%\input{qm2pi.dtcodes} 

% section wiring algorithm (end)

\input{qm2pi.ack} 

% section acknowledgments (end)

\newpage


\bibliographystyle{plain}   
\bibliography{../../biblios/main.bib}

\input{qm2pi.rhodetails}

\end{document}

 

% section concurrent_process_calculi (end)

%\documentclass[12pt]{llncs}
%\documentclass{jktr}

\usepackage[pdftex]{hyperref}                   
\usepackage {listings}
\usepackage {mathpartir}
\usepackage{bcprules}
%\usepackage{listings}
                       
\usepackage{graphicx} 
%\usepackage[margins=2.5cm,nohead,nofoot]{geometry}
%\usepackage{geometry}
\usepackage{amsfonts}
\usepackage{amstext}
\usepackage{latexsym}
\usepackage{amssymb}
\usepackage{color}


%\include{myPreamble}
\include{qm2pi.local} 

%\ifpdf
%\usepackage[pdftex]{graphicx}
%\else
%\usepackage{graphicx}
%\fi

 % \ifpdf
%  \usepackage{pdfsync}
%  \if


%\title{Brief Article}
%\author{David F. Snyder}
%\author{L.G. Meredith}

%\address{Dept. of Math., Texas State University--San Marcos, San Marcos, TX 78666}
       
\pagestyle{empty}


\begin{document}

\lstset{language=[Objective]Caml,frame=shadowbox}

\input{qm2pi.front}

% section front matter (end)

\input{qm2pi.intro} 
 
% section introduction (end)

% \input{qm2pi.knotations} 

% section notation (end)

\input{qm2pi.process.calculi} 

% section concurrent_process_calculi_and_spatial_logics_ (end)
    
%\input{qm2pi.knots2pi} 

%\input{qm2pi.trefoil} 

%\input{qm2pi.mainthm} 

% subsection basic_interpretation (end)

%\input{qm2pi.rho.presentation} 
\subsection{The syntax and semantics of the notation system}\label{sub:the_syntax_and_semantics_of_the_notation_system} % (fold)

We now summarize a technical presentation of the calculus that
embodies our theory of dynamics. The typical presentation of such a
calculus follows the style of giving generators and relations on
them. The grammar, below, describing term constructors, freely
generates the set of processes, $\Proc$. This set is then quotiented
by a relation known as structural congruence and it is over this set
that the notion of dynamics is expressed. This presentation is
essentially that of \cite{MeredithR05} with the addition of
polyadicity and summation. For readability we have relegated some of
the technical subtleties to an appendix.

\subsubsection{Process grammar}\label{subsub:process_grammar}

\begin{mathpar}
  \inferrule* [lab=synchronization] {} {{M} \bc \pzero \;|\; x?F \;|\; x!C }
  \and
  \inferrule* [lab=abstraction] {} {{F} \bc (x)P}
  \and
  \inferrule* [lab=concretion] {} {{C} \bc \langle Q \rangle}
  \and
  \inferrule* [lab=process] {} {{P,Q} \bc M \;| \;P|Q \;|\; @{x}}
  \and
  \inferrule* [lab=name] {} {{x} \bc \quotep{P}}
\end{mathpar} 

Note that $\vec{x}$ (resp. $\vec{P}$) denotes a vector of names
(resp. processes) of length $|\vec{x}|$ (resp. $|\vec{P}|$). We adopt
the following useful abbreviations.

\begin{mathpar}
   x?(\vec{y}).P := x.(\vec{y})P \and  x\clift{\vec{P}} := x.\clift{\vec{P}}
   \and x!(y) := \lift{x}{\dropn{y}}
   \and \Pi_{i=0}^{n-1}P_i := P_0 | \ldots | P_{n-1}
\end{mathpar}

\subsubsection{Structural congruence}

\paragraph{Free and bound names and alpha-equivalence.} At the
core of structural equivalence is alpha-equivalence which identifies
process that are the same up to a change of variable. Formally, we
recognize the distinction between free and bound names. The free names
of a process, $\freenames{P}$, may be calculated recursively as
follows:

\begin{mathpar}
\freenames{\pzero} := \emptyset
  \and \\
  \freenames{x?(y).P} := \{ x \} \cup (\freenames{P} \setminus \{ y \})
  \and 
  \freenames{x!\langle P \rangle} := \{ x \} \cup \{ P \} 
  \and \\
  \freenames{P|Q} := \freenames{P} \cup \freenames{Q}
  \and \\
  \freenames{@{x}} := \{ x \}
\end{mathpar}

$\pi$
$\quotep{\pi}$

$\freenames{-} : \pi \to \mathcal{P}(\quotep{\pi})$

\begin{eqnarray*}
  \freenames{\pzero} & := & \emptyset \\
  \freenames{x?(y).P} & := & \{ x \} \cup (\freenames{P} \setminus \{ y \}) \\
  \freenames{x!\langle P \rangle} & := & \{ x \} \cup \{ P \} \\
  \freenames{P|Q} & := & \freenames{P} \cup \freenames{Q} \\
  \freenames{\dropn{x}} & := & \{ x \}
\end{eqnarray*}

The bound names of a process, $\boundnames{P}$, are those names occurring in $P$
that are not free. For example, in $x?(y).0$, the name $x$ is free, while $y$ is bound.

\begin{mathpar}
  \inferrule* [lab=monoidal-laws] {} { P|Q \equiv Q|P \and P|0 \equiv P \and P|(Q|R) \equiv (P|Q)|R }
\end{mathpar}

\begin{mathpar}
  \inferrule* [lab=alpha-equivalence] {} { (x)P \equiv (y)P\{y/x\} \and y \not\in \freenames{P} }
\end{mathpar}

\begin{definition}
Then two processes, $P,Q$, are alpha-equivalent if $P = Q\{\vec{y}/\vec{x}\}$ for
some $\vec{x} \in \boundnames{Q},\vec{y} \in \boundnames{P}$, where $Q\{\vec{y}/\vec{x}\}$
denotes the capture-avoiding substitution of $\vec{y}$ for $\vec{x}$ in $Q$.
\end{definition}

\begin{definition}
  The {\em structural congruence} \cite{SangiorgiWalker} , $\equiv$,
  between processes is the least congruence containing
  alpha-equivalence, satisfying the abelian monoid laws
  (associativity, commutativity and $\pzero$ as identity) for parallel
  composition $|$ and for summation $+$.
\end{definition}

\subsection{Name equivalence}

We take name equivalence, written $\nameeq$, to be the smallest
equivalence relation generated by the following rules.

\begin{mathpar}
\inferrule*[lab=Quote-drop]
{ }
{ \quotep{@{x}} \nameeq x }

\inferrule*[lab=Struct-equiv]
{ P \scong Q }
{ \quotep{P} \nameeq \quotep{Q} }
\end{mathpar}

The astute reader will have noticed that the mutual recursion of names
and processes imposes a mutual recursion on alpha-equivalence and
structural equivalence via name-equivalence. Fortunately, all of this
works out pleasantly and we may calculate in the natural way, free of
concern. The reader interested in the details is referred to the
appendix \ref{appendix:rho_details}.

\subsection{Substitution}

We use $\Proc$ for the set of processes, $\QProc$ for the set of
names, and $\id{\{}\vec{y} / \vec{x} \id{\}}$ to denote partial maps,
$s : \QProc \rightarrow \QProc$. A map, $s$ lifts, uniquely, to a map
on process terms, $\widehat{s} : \Proc \rightarrow \Proc$ by the
following equations.

\begin{mathpar}
  (0) \psubstp{Q}{P} := 0 \\
  (R \juxtap S) \psubstp{Q}{P}
  :=    
  (R)\psubstp{Q}{P} \juxtap (S) \psubstp{Q}{P} \\
  (x?(y).R) \psubstp{Q}{P}    
  :=    
  (x)\substp{Q}{P} (z)\concat( (R \psubstn{z}{y}) \psubstp{Q}{P} ) \\
  (\lift{x}{R}) \psubstp{Q}{P}  
  :=
  \lift{(x)\substp{Q}{P}}{ R \psubstp{Q}{P} } \\
%   (\dropn{x})  \psubstp{Q}{P}       
%   := 
%   \left\{ 
%     \begin{array}{ccc} 
%       \dropn{\quotep{Q}} & & x \nameeq \quotep{P} \\
%       \dropn{x} & & otherwise \\
%     \end{array}
%   \right. 
  (\dropn{x})  \psubstp{Q}{P}       
  := 
  \left\{ 
    \begin{array}{ccc} 
      Q & & x \nameeq \quotep{P} \\
      \dropn{x} & & otherwise \\
    \end{array}
  \right.
\end{mathpar}
 

where

\begin{eqnarray}
  (x)\id{\{} \lpquote Q \rpquote / \lpquote P \rpquote \id{\}}            = 
  \left\{ 
    \begin{array}{ccc}
      \lpquote Q \rpquote & & x \nameeq \lpquote P \rpquote \\
      x & & otherwise \\
    \end{array}
  \right. \nonumber
\end{eqnarray}

and $z$ is chosen distinct from $\quotep{P}$, $\quotep{Q}$, the free
names in $Q$, and all the names in $R$. Our $\alpha$-equivalence will
be built in the standard way from this substitution.

\begin{remark}\label{rem:no_self_referential_names}
  One consequence of these definitions is that $\forall P. \quotep{P}
  \not\in \freenames{P}$.
\end{remark}

\subsection{ Dynamic quote: an example }

Anticipating something of what's to come, consider applying the
substitution, $\widehat{\id{\{}u / z \id{\}}}$, to the following pair
of processes, $\lift{w}{y!(z)}$ and $w[ \lpquote y!(z) \rpquote ]$.

\begin{eqnarray}
	\lift{w}{y!(z)}\widehat{\id{\{}u / z \id{\}}}
		& = &
		\lift{w}{y!(u)} \nonumber\\
	w[ \lpquote y!(z) \rpquote ] \widehat{ \id{\{}u / z \id{\}} }
		& = &
		w[ \lpquote y!(z) \rpquote ] \nonumber
\end{eqnarray}

Because the body of the process between quotes is impervious to
substitution, we get radically different answers. In fact, by
examining the first process in an input context,
e.g. $x?(z).\lift{w}{y!(z)}$, we see that the process under the lift
operator may be shaped by prefixed inputs binding a name inside it. In
this sense, the lift operator will be seen as a way to dynamically
construct processes before reifying them as names.

Finally equipped with these standard features we can present the
dynamics of the calculus.

\subsubsection{Operational semantics} 

Finally, we introduce the computational dynamics. What marks these
algebras as distinct from other more traditionally studied algebraic
structures, e.g. vector spaces or polynomial rings, is the manner in
which dynamics is captured. In traditional structures, dynamics is typically
expressed through morphisms between such structures, as in linear maps
between vector spaces or morphisms between rings. In algebras
associated with the semantics of computation, the dynamics is
expressed as part of the algebraic structure itself, through a
reduction reduction relation typically denoted by $\red$. Below, we
give a recursive presentation of this relation for the calculus used
in the encoding.

$\red \subseteq \pi \times \pi$
$\red : \pi \to \mathcal{P}(\pi)$

\begin{mathpar}
  \inferrule* [lab=Comm] { \textsf{match}( x_{src}, x_{trgt} ) } { x_{trgt}?(y)P \; | \; x_{src}!\langle {Q} \rangle \red P\{\quotep{Q}/y}\} }
  \and \\
  \inferrule* [lab=Par] {{P} \red {P}'} {{{P} | {Q}} \red {{P}' | {Q}}}
  \and
  \inferrule* [lab=Equiv]{{{P} \scong {P}'} \andalso {{P}' \red {Q}'} \andalso {{Q}' \scong {Q}}}{{P} \red {Q}}
\end{mathpar}

\begin{eqnarray*}
  match_{\equiv} (\quotep{P},\quotep{Q}) & := & P \equiv Q \\
  match_{\dagger}(\quotep{P},\quotep{Q}) & := & \forall R. P|Q \red^{*} R => R \red^{*} 0 \\
  match_{K}(\quotep{P},\quotep{Q}) & := & K \mbox{ for some context } K
\end{eqnarray*}

$u?(x)P | u!\langle Q \rangle \red P\{\quotep{Q}/x\}$

%We write $\wred$ for $\red^*$, and $P\red$ if $\exists Q $ such that $ P \red Q$.
We write $P\red$ if $\exists Q $ such that $ P \red Q$ and $P\not\red$, otherwise.

\section{Replication}

As mentioned before, it is known that replication (and hence
recursion) can be implemented in a higher-order process algebra
\cite{SangiorgiWalker}. As our first example of calculation with the
machinery thus far presented we give the construction explicitly in
the {\rhoc}.

\begin{eqnarray}
	D_{x} & := & \prefix{x}{y}{(\binpar{\outputp{x}{y}}{@{y}})} \nonumber\\
	\bangp_{x}{P} & := & \binpar{{x}!\langle{\binpar{D_{x}}{P}}\rangle}{D_{x}} \nonumber
\end{eqnarray}

\begin{eqnarray}
	\bangp_{x}{P} & & \nonumber\\
	=
	& {x}!\langle{(\prefix{x}{y}{(\outputp{x}{y} | @{y})) | P}}\rangle 
	      | \prefix{x}{y}{(\outputp{x}{y} | @{y})} & \nonumber\\
	\red
	& (\outputp{x}{y} | @{y})\substn{\quotep{(\prefix{x}{y}{(@{y} | \outputp{x}{y})) | P}}}{y} & \nonumber\\
	=
	& \outputp{x}{\quotep{(\prefix{x}{y}{(\outputp{x}{y} | @{y})) | P}}}
	  | {(\prefix{x}{y}{(\outputp{x}{y} | @{y})) | P}} & \nonumber\\
	\red
	& \ldots & \nonumber\\
	\red^*
	& P | P | \ldots & \nonumber
\end{eqnarray}

Of course, this encoding, as an implementation, runs away, unfolding
$\bangp{P}$ eagerly. A lazier and more implementable replication
operator, restricted to input-guarded processes, may be obtained as follows.

\begin{eqnarray}
\bangp{\prefix{u}{v}{P}} 
	:= 
	\binpar{\lift{x}{\prefix{u}{v}{(\binpar{D(x)}{P})}}}{D(x)} \nonumber
\end{eqnarray}

\begin{remark}
  Note that the lazier definition still does not deal with summation
  or mixed summation (i.e. sums over input and output). The reader is
  invited to construct definitions of replication that deal with these
  features. 

  Further, the definitions are parameterized in a name, $x$. Can you,
  gentle reader, make a definition that eliminates this parameter and
  guarantees no accidental interaction between the replication
  machinery and the process being replicated -- i.e. no accidental
  sharing of names used by the process to get its work done and the
  name(s) used by the replication to effect copying. This latter
  revision of the definition of replication is crucial to obtaining
  the expected identity $!!P \sim !P$.
\end{remark}

\begin{remark}\label{rem:paradoxical_combinator}
  The reader familiar with the lambda calculus will have noticed the
  similarity between $D$ and the paradoxical combinator.

  [Ed. note: the existence of this seems to suggest we have to be more
  restrictive on the set of processes and names we admit if we are to
  support no-cloning.]
\end{remark}

\subsubsection{Bisimulation}

The computational dynamics gives rise to another kind of equivalence,
the equivalence of computational behavior. As previously mentioned
this is typically captured \emph{via} some form of bisimulation.

% The notion we use in this paper is weak barbed bisimulation
% \cite{milner91polyadicpi}.

The notion we use in this paper is derived from weak barbed
bisimulation \cite{milner91polyadicpi}. 

\begin{definition}
An \emph{observation relation}, $\downarrow_{\mathcal N}$, over a set
of names, $\mathcal N$, is the smallest relation satisfying the rules
below.

\infrule[Out-barb]{y \in {\mathcal N}, \; x \nameeq y}
		  {\outputp{x}{v} \downarrow_{\mathcal N} x}
\infrule[Par-barb]{\mbox{$P\downarrow_{\mathcal N} x$ or $Q\downarrow_{\mathcal N} x$}}
		  {\binpar{P}{Q} \downarrow_{\mathcal N} x}

We write $P \Downarrow_{\mathcal N} x$ if there is $Q$ such that 
$P \wred Q$ and $Q \downarrow_{\mathcal N} x$.
\end{definition}

\begin{definition}
%\label{def.bbisim}
An  ${\mathcal N}$-\emph{barbed bisimulation} over a set of names, ${\mathcal N}$, is a symmetric binary relation 
${\mathcal S}_{\mathcal N}$ between agents such that $P\rel{S}_{\mathcal N}Q$ implies:
\begin{enumerate}
\item If $P \red P'$ then $Q \wred Q'$ and $P'\rel{S}_{\mathcal N} Q'$.
\item If $P\downarrow_{\mathcal N} x$, then $Q\Downarrow_{\mathcal N} x$.
\end{enumerate}
$P$ is ${\mathcal N}$-barbed bisimilar to $Q$, written
$P \wbbisim_{\mathcal N} Q$, if $P \rel{S}_{\mathcal N} Q$ for some ${\mathcal N}$-barbed bisimulation ${\mathcal S}_{\mathcal N}$.
\end{definition}

$\mathcal{R} \subseteq \pi \times \pi$

$P \mathcal{R} Q => \forall P'. P \red P' \Rightarrow \exists Q'. Q \red Q', P' \mathcal{R} Q'$

$P \vdash x \Rightarrow Q \vdash x$

\begin{mathpar}
  \inferrule*[lab=Out-barb]{x \nameeq y}{{y}!\langle{Q}\rangle \vdash x}
  \and
  \inferrule*[lab=Par-barb]{\mbox{$P\vdash x$ or $Q\vdash x$}}{\binpar{P}{Q} \vdash x}
\end{mathpar}

\subsubsection{Contexts}

One of the principle advantages of computational calculi like the
$\pi$-calculus is a well-defined notion of context,
contextual-equivalence and a correlation between
contextual-equivalence and notions of bisimulation. The notion of
context allows the decomposition of a process into (sub-)process and
its syntactic environment, its context. Thus, a context may be
thought of as a process with a ``hole'' (written $\Box$) in it. The
application of a context $M$ to a process $P$, written $M[P]$, is
tantamount to filling the hole in $M$ with $P$. In this paper we do
not need the full weight of this theory, but do make use of the notion
of context in the proof the main theorem. 

\begin{mathpar}
  \inferrule* [lab=summation] {} {{M_{M},M_{N}} \bc \Box \;|\; x.M_{A} \;|\; M_{M}+M_{N}}
  \and
  \inferrule* [lab=agent] {} {{M_{A}} \bc (\vec{x})M_{P} \;| \; \clift{P_0,\ldots,M_{P},\ldots,P_N}}
  \and \\
  \inferrule* [lab=process] {} {{M_{P}} \bc M_{N} \;| \;P|M_{P} }
\end{mathpar} 

\begin{mathpar}
  \inferrule* [lab=sychronization] {} {M_{N} \bc \Box \;|\; x?M_{F} \;|\; x!M_{C}}
  \and
  \inferrule* [lab=abstraction] {} {{M_{F}} \bc (x)M_{P} }
  \and
  \inferrule* [lab=concretion] {} {{M_{C}} \bc \langle M_{P} \rangle }
  \and \\
  \inferrule* [lab=process] {} {{M_{P}} \bc M_{N} \;| \;P|M_{P} }
\end{mathpar}

\begin{definition}[contextual application] Given a context $M$, and
  process $P$, we define the \emph{contextual application}, $M[P] :=
  M\{P/\Box\}$. That is, the contextual application of M to P is the
  substitution of $P$ for $\Box$ in $M$.
\end{definition}

$\meaningof{-} : L \to \mathcal{P}(\pi)$

\begin{mathpar}
  \inferrule* [lab=collection] {} {\meaningof{true} = \pi, \and \meaningof{~E} = \pi \setminus \meaningof{E}, \and \meaningof{E_{1} \& E_{2}} = \meaningof{E_{1}} \cap \meaningof{E_{2}}}
\end{mathpar}

\begin{mathpar}
  \inferrule* [lab=structure] {} {\meaningof{0} = \{ P \in \pi | P \equiv 0 \}, \and \\ \meaningof{E_1 | E_2} = \{ P \in \pi | P \equiv P_{1} | P_{2}, P_{1} \in \meaningof{E_{1}}, P_{2} \in \meaningof{E_2}\} }
\end{mathpar}

\begin{mathpar}
 \inferrule* [lab=behavior] {} {\meaningof{\langle a?b \rangle E} = \{ P \in \pi | P \equiv Q | u?(y)P', \\ \and \\\\ \and \\ \;\;\; u \in \meaningof{a}, \forall z.P'\{z/y\} \in \meaningof{E\{z/b\}}\}, \and \\ \meaningof{a!E} = \{ P \in \pi | P \equiv Q | x!\langle P' \rangle, x \in \meaningof{a} P' \in \meaningof{E}\} }
\end{mathpar}

\begin{mathpar}
 \inferrule* [lab=nominal] {} {\meaningof{\quotep{E}} = \{ \quotep{P} \in \quotep{\pi} | P \in \meaningof{E} \}, \and \meaningof{\quotep{P}} = \{ \quotep{Q} \in \quotep{\pi} | P \equiv Q \} \and \\ \meaningof{@\quotep{E}} = \{ P \in \pi | P \equiv @x, x \in \meaningof{E} \}}
\end{mathpar}

\begin{eqnarray*}
  \\
  \meaningof{-} : TS \to ST
\end{eqnarray*}

\begin{eqnarray*}
  \\
  L : TS \to ST
\end{eqnarray*}

\begin{eqnarray*}
  \\
  P \models E \iff P \in \meaningof{E}
\end{eqnarray*}

\begin{eqnarray*}
  P \approx_{L} Q \iff \forall E \in L. P \models E \iff Q \models E
\end{eqnarray*}

\begin{eqnarray*}
  P \approx_{K} Q
\end{eqnarray*}

\begin{eqnarray*}
  P \approx Q
\end{eqnarray*}

$\approx_{K} = \approx = \approx_{L}$

\subsubsection{Contextual duality}

Note that contexts extend the quotation operation to a family of
operations from processes to names. Given a context, $M$, we can
define a \emph{nominal context}, $\quotep{M}$ by $\quotep{M}[P] :=
\quotep{M[P]}$. To foreshadow what is to come we observe that these
operations enjoy a duality with processes very much like the duality
between vectors and maps from vectors to scalars.

Further, because the calculus is essentially higher-order, we have a
correspondence between contexts and processes. More specifically,
given a name $x$ and a context $M$ we can construct $M^{*}_{x}$ such
that 

\begin{mathpar}
  M^{*}_{x} | \lift{x}{P} \red M[P]
\end{mathpar}

namely,

\begin{mathpar}
  M^{*}_{x} := x?(u).M[\dropn{u}]
\end{mathpar}

The dependence of $M^{*}_{x}$ on a name makes it an abstraction, 

\begin{mathpar}
  M^{*} := (x)x?(u).M[\dropn{u}]
\end{mathpar}

\subsection{Additional notation}

It will sometimes be convenient to denote the process a name
quotes. We already have the notation $x = \quotep{P}$, but it will be
convenient to introduce an alternate notation, $\procn{x}$, when we
want to emphasize the connection to the use of the name. Note that, by
virtue of name equivalence, $\quotep{\procn{x}} \nameeq x$; so, the
notation is consistent with previous definitions.

Further, because names have structure it is possible to effect
substitutions on the basis of that structure. This means we need to
upgrade our notation for substitutions, which we accomplish by
adapting comprehension notation. Thus,

\begin{mathpar}
  P\{ y / x : x \in S \}
\end{mathpar}

is interpreted to mean the process derived from P by replacing (in a
capture-avoiding manner) each occurrence of $x$ in $S$ by $y$. For example,

\begin{mathpar}
  P\{ \quotep{\procn{x}|\procn{x}} / x : x \in \freenames{P} \}
\end{mathpar}

will replace each (occurrence) of a free name $x$ in $P$ by
$\quotep{\procn{x}|\procn{x}}$.

Also, we will avail ourselves of the notation $x^{L}$ and $x^{R}$ to
denote injections of a name into disjoint copies of the name
space. There are numerous ways to accomplish this. One example can be
found in \cite{MeredithR05}. This notation overloads to vectors of
names: $\vec{x}^{\pi} := (x_{i}^{\pi} \; : \; 0 \leq i < |\vec{x}| )$ where $\pi \in \{L,R\}$.

We also use $P^{\Box} := P|\Box$.

In \cite{MeredithR05} an interpretation of the new operator is
given. It turns out that there are several possible interpretations
all enjoying the requisite algebraic properties of the operator (see
\cite{milner91polyadicpi}). We will therefore make liberal use of
$(\nu\; \vec{x})P$.

% subsection the_syntax_and_semantics_of_the_notation_system (end)   

\input{qm2pi.qmops} 

\input{qm2pi.sterngerlach} 

\input{qm2pi.metric} 

% section concurrent_process_calculi (end)

%\input{qm2pi.proofsketch}

% section proof sketch (end)

%\input{qm2pi.slviaknots} 

% section spatial logic via knots (end)

\input{qm2pi.conclusion}

% section conclusion (end)

%\input{qm2pi.dtcodes} 

% section wiring algorithm (end)

\input{qm2pi.ack} 

% section acknowledgments (end)

\newpage


\bibliographystyle{plain}   
\bibliography{../../biblios/main.bib}

\input{qm2pi.rhodetails}

\end{document}



% section proof sketch (end)

%\section{Unlikely characters: spatial logic for
  knots}\label{sub:characteristic_formulae} % (fold)

Associated to the mobile process calculi are a family of logics known
as the Hennessy-Milner logics. These logics typically enjoy a
semantics interpreting formulae as sets of processes that when
factored through the encoding outlined above allows an identification
of classes of knots with logical formulae. In the context of this
encoding the sub-family known as the spatial logics \cite{CairesC03}
\cite{CairesC04} \cite{Caires04} are of particular interest providing
several important features for expressing and reasoning about
properties (i.e. classes) of knots. We hint here at how this may be done.

%\begin{description}
%\item [structural connectives] 
\subsubsection{Structural connectives} The spatial logics enjoy
structural connectives corresponding, at the logical level, to the
parallel composition ($P | Q$) and new name ($(\nu \; x)P$)
connectives for processes. As illustrated in the examples below, these
connectives are extremely expressive given the shape of our encoding.
%\item [decideable satisfaction]

\subsubsection{Decideable satisfaction}
In \cite{Caires04} the satisfaction relation is shown to be decideable
for a rich class of processes. It further turns out that the image of
the our encoding is a proper subset of that class. This result
provides the basis for an algorithm by which to search for knots
enjoying a given property.
%\item [characteristic formulae]

\subsubsection{Characteristic formulae}
In the same paper \cite{Caires04} , Caires presents a means of calculating
characteristic formulae, selecting equivalence classes of processes
up to a pre--specified depth limit on the support set of names. Composed with our
encoding, this characteristic formula can be used to select
characteristic formulae for knots.
%\end{description}

\subsubsection{Spatial logic formulae}

The grammar below (segmented for comprehension) summarizes the syntax
of spatial logic formulae. We employ illustrative examples in the
sequel to provide an intuitive understanding of their meaning
referring the reader to \cite{Caires04} for a more detailed explication
of the semantics.

\begin{mathpar}
  \inferrule* [lab=boolean] {} {{A,B} \bc T \;|\; \neg A \;|\; A \wedge B \;|\; \eta = \eta'}
  \and
  \inferrule* [lab=spatial] {} {|\; \pzero \;|\; A | B \;|\; x \text{\textregistered} A \;|\; \forall x . A \;|\;  H x . A}
  \and
  \inferrule* [lab=behavioral] {} {|\; \alpha . A}
  \and 
  \inferrule* [lab=recursion] {} {|\; X(\vec{u}) \;|\; \mu X(\vec{u}) . A}
  \and
  \inferrule* [lab=action] {} {\alpha \bc \langle x?(\vec{y}) \rangle \;|\; \langle x!(\vec{y}) \rangle \;|\; \langle \tau \rangle}
  \and 
  \inferrule* [lab=name] {} {\eta \bc x \;|\; \tau}
\end{mathpar} 

% subsection characteristic_formulae (end)   	 

\subsection{Example formulae}\label{sub:example_formulae_} % (fold)

\subsubsection{Crossing as formula.}
% 
% \begin{align*}
%   \frac{d}{dx} \sin x &= \cos x 
%   & \frac{d}{dx} e^x &= e^x \\
%   \frac{d}{dx} \cos x &= - \sin x 
%   & \frac{d}{dx} \log x &= \frac{1}{x} \\
% \end{align*} 

\begin{align*}
 \mu C(x_{0},x_{1},y_{0},y_{1},u).&(\langle x_{0}?(z) \rangle(\langle u! \rangle\langle y_{1}!z \rangle C(x_{0},x_{1},y_{0},y_{1},u)) & \\
  & \wedge \langle y_{1}?(z) \rangle (\langle u! \rangle \langle x_{0}!z \rangle C(x_{0},x_{1},y_{0},y_{1},u)) & \\
  & \wedge \langle x_{1}?(z) \rangle (\langle u? \rangle \langle y_{0}!z \rangle C(x_{0},x_{1},y_{0},y_{1},u)) & \\
  & \wedge \langle y_{0}?(z) \rangle (\langle u? \rangle \langle x_{1}!z \rangle C(x_{0},x_{1},y_{0},y_{1},u))) &
\end{align*}

The lexicographical similarity between the shape of this formulae and
the shape of definition of the process representing a crossing reveals
the intuitive meaning of this formulae. It describes the capabilities
of a process that has the right to represent a crossing. For example
it picks out processes that may perform an input on the port $x_0$ in
its initial menu of capabilities. What differentiates the formula
from the process, however, is that the crossing process is the
smallest candidate to satisfy the formula. Infinitely many other
processes -- with internal behavior hidden behind this interface, so
to speak -- also satisfy this formula. Even this simple formula,
then, can be seen to open a new view onto knots, providing a
computational interpretation of \emph{virtual} knots.

Note that this formula is derived by hand. A similar formula can be
derived by employing Caires' calculation of characteristic formula
\cite{Caires04} to the process representing a crossing. In light of
this discussion, we let
$\meaningof{C}_{\phi}(x0,x1,y0,y1,u)$ denote a formula specifying the
dynamics we wish to capture of a crossing. To guarantee we preserve
the shape of the interface and minimal semantics we demand that
$\meaningof{C}_{\phi}(x0,x1,y0,y1,u) \Rightarrow
\textbf{C}(x0,x1,y0,y1,u)$ where $\textbf{C}(x0,x1,y0,y1,u)$ denotes
the formula above.
                            
\subsubsection{Crossing number constraints.}
The moral content of the context lemma (Lemma \ref{context}) is that the notion of
``locality'' in the Reidemeister moves is effectively captured by the
parallel composition operator of the process calculus. This intuition
extends through the logic. Given a formula,
$\meaningof{C}_{\phi}(x0,x1,y0,y1,u)$, we can use the structural
connectives to specify constraints on crossing numbers, such as at
least $n$ crossings, or exactly $n$ crossings.
\begin{mathpar}
  \inferrule* [lab=at-least-n] {} { K^{\geq n}_{\phi}(\vec{xs},\vec{ys}) := \Pi_{i=0}^{n-1} Hu . \meaningof{C}_{\phi}(xs_i,ys_i,u) | T }
  \and 
  \inferrule* [lab=exactly-n] {} { K^{= n}_{\phi}(\vec{xs},\vec{ys}) := \Pi_{i=0}^{n-1} Hu . \meaningof{C}_{\phi}(xs_i,ys_i,u) | \neg (\forall x_0,y_0,x_1,y_1,u . \meaningof{C}_{\phi}(x_0,y_0,x_1,y_1,u) | T) }
\end{mathpar}

To round out this section, recall that the encoding of an $n$-crossing
knot decomposes into a parallel composition of $n$ \emph{copies} of a
crossing process together with a wiring harness. To specify different
knot classes with the same crossing number amounts to specifying
logical constraints on the wiring harness. In the interest of space,
we defer examples to a forthcoming paper. Suffice it to say that both
the conditions ``alternating knot'' and ``contains the tangle
corresponding to 5/3'' are expressible. For example, it is possible to
calculate the characteristic formula of a process corresponding to the
tangle 5/3 and conjoin it into the classifying formula via the
composition connective of the logic.

Finally, we wish to observe that it is entirely within reason to
contemplate a more domain-specific version of spatial logic tailored
to the shape of processes in the image of the encoding. Such a
domain-specific logic would have a better claim to the title formal
language of knot properties.

% subsection example_formulae_ (end)

% section knots_as_processes (end) 

% section spatial logic via knots (end)

\section{Conclusions and future work}

\paragraph{Testing physical space}
You, gentle reader, may wonder why of all the theorems to be proved
given this set up we pick the one above. In some sense it's hardly
central to quantum mechanics. We see it as central in the sense that
it firmly establishes a notion of physical space arising from a notion
of the equivalence of behavior. Relating bisimulation to a metric is a
big step forward, but one is faced with interpreting the relationship
of that metric space to something more physical. Quantum mechanical
notions of ``physical'' space are still far from intuitive, but by
relating this idea of distance as testing to calculations that predict
physical circumstances we are making a not insignificant step forward
toward an understanding of the physical space we inhabit as
essentially dynamic.

\paragraph{Effectivity and simulation}
One of the observations we have yet to make is that the entire program
spelled out here is effective. We have built various interpreters for
the reflective calculus at work in this interpretation. In principle,
then, we can simulate quantum mechanics on a computer. The place where
the simulation may lose fidelity is the infinitely branching summation
for the annihilator.

In this connection i also want to point out that the evaluation style
calculation of the inner product puts the non-determinism of the
summation right at the heart of measurement. This suggests that
Milner's original reduction-based formulation of the dynamics of his
calculi in terms of sums was not just notationally suggestive of a
notion of measure-and-continue but captured some significant part of
the physics.

\paragraph{Quantum continuations}
In light of this last observation i want to point out that the
predominant account of quantum mechanics is missing a key aspect of a
truly compositional story of the physical situation. In a real lab,
when a measurement is made the observation can be made to feed into
another device that then makes another measurement conditioned on the
results of the first. This means that after the superposition was
collapsed the entire experimental set up remained in
superposition. While QM offers a means of writing this down it doesn't
quite line up well with the well-trodden formulation of computation
and continuation that we see so succinctly expressed in Milner's
calculi. This suggests that there might be advantages to this account
of dynamics waiting to be explored.

\paragraph{Quantum logic}
In this connection, we also note that by virtue of having the
Hennessy-Milner construction, we can pull the construction through the
interpretation of QM. This gives us a natural candidate for a quantum
logic that enjoys an extremely tight connection with it's domain of
interpretation, making the construction much less ad hoc (rather it is
the image of functor!).

\paragraph{Quantum probabiity}
i have questions about the basis of the interpretation of inner
product as probability amplitude. In particular, using which
axiomatization of probability theory does the notion of probability
amplitude earn the right to be so dubbed? In other words, where is the
proof that the operation for calculating a probability amplitude (and
then squaring) satisfies the axioms of what it means to calculate a
probability? Even if such a proof exists (i have yet to find it in the
literature), i wonder if it might not be possible to turn things on
their heads. Can we view the calculation of the probability amplitude
as an axiomatization of probability? If so, then the definition we
give for calculating probability amplitude may provide the basis for
an \emph{effective} theory of probability.

\paragraph{Quantum vs ``biological'' information}
Finally, i want to conclude with a more philosophical observation. At
a recent workshop in which QM was a predominant topic i noticed
something about quantum information. The speaker was giving a riveting
discussion of axiomatic QM and showing how properties of ``no
cloning'' and ``no deleting'' emerged as consequences of the
axiomatization. Theorems of this form are necessary to give us a sense
of confidence that our axioms characterize the physical theory. What
struck me, though, was that if quantum information is neither erasable
nor replicable it is markedly different from \emph{life}. Two of the
things we know about life is that

\begin{itemize}
  \item it ends;
  \item to gain some measure of persistence, to transcend it's
    finitude it is imminently copyable.
\end{itemize}

Both of these qualities are summarized succinctly in the aphorism: all
flesh is grass. For me these two kinds of ``information'' -- call them
quantum and biological -- are end points on a spectrum of strategies
for persistence. At one end, we have those curious entities that enjoy
uniqueness and permanence; at the other, we have those who in the face
of a certain end and an uncertain present make a go of passing
something on. To me one of the more remarkable aspects of the latter
strategy is that in the presence of noise (and certain features of
copying) we get a kind of dynamism, a chance for improvement against a
given persistent condition.

% subsection other_calculi_other_bisimulations_and_geometry_as_behavior (end)




% section conclusion (end)

%\documentclass[12pt]{llncs}
%\documentclass{jktr}

\usepackage[pdftex]{hyperref}                   
\usepackage {listings}
\usepackage {mathpartir}
\usepackage{bcprules}
%\usepackage{listings}
                       
\usepackage{graphicx} 
%\usepackage[margins=2.5cm,nohead,nofoot]{geometry}
%\usepackage{geometry}
\usepackage{amsfonts}
\usepackage{amstext}
\usepackage{latexsym}
\usepackage{amssymb}
\usepackage{color}


%\include{myPreamble}
\include{qm2pi.local} 

%\ifpdf
%\usepackage[pdftex]{graphicx}
%\else
%\usepackage{graphicx}
%\fi

 % \ifpdf
%  \usepackage{pdfsync}
%  \if


%\title{Brief Article}
%\author{David F. Snyder}
%\author{L.G. Meredith}

%\address{Dept. of Math., Texas State University--San Marcos, San Marcos, TX 78666}
       
\pagestyle{empty}


\begin{document}

\lstset{language=[Objective]Caml,frame=shadowbox}

\input{qm2pi.front}

% section front matter (end)

\input{qm2pi.intro} 
 
% section introduction (end)

% \input{qm2pi.knotations} 

% section notation (end)

\input{qm2pi.process.calculi} 

% section concurrent_process_calculi_and_spatial_logics_ (end)
    
%\input{qm2pi.knots2pi} 

%\input{qm2pi.trefoil} 

%\input{qm2pi.mainthm} 

% subsection basic_interpretation (end)

%\input{qm2pi.rho.presentation} 
\subsection{The syntax and semantics of the notation system}\label{sub:the_syntax_and_semantics_of_the_notation_system} % (fold)

We now summarize a technical presentation of the calculus that
embodies our theory of dynamics. The typical presentation of such a
calculus follows the style of giving generators and relations on
them. The grammar, below, describing term constructors, freely
generates the set of processes, $\Proc$. This set is then quotiented
by a relation known as structural congruence and it is over this set
that the notion of dynamics is expressed. This presentation is
essentially that of \cite{MeredithR05} with the addition of
polyadicity and summation. For readability we have relegated some of
the technical subtleties to an appendix.

\subsubsection{Process grammar}\label{subsub:process_grammar}

\begin{mathpar}
  \inferrule* [lab=synchronization] {} {{M} \bc \pzero \;|\; x?F \;|\; x!C }
  \and
  \inferrule* [lab=abstraction] {} {{F} \bc (x)P}
  \and
  \inferrule* [lab=concretion] {} {{C} \bc \langle Q \rangle}
  \and
  \inferrule* [lab=process] {} {{P,Q} \bc M \;| \;P|Q \;|\; @{x}}
  \and
  \inferrule* [lab=name] {} {{x} \bc \quotep{P}}
\end{mathpar} 

Note that $\vec{x}$ (resp. $\vec{P}$) denotes a vector of names
(resp. processes) of length $|\vec{x}|$ (resp. $|\vec{P}|$). We adopt
the following useful abbreviations.

\begin{mathpar}
   x?(\vec{y}).P := x.(\vec{y})P \and  x\clift{\vec{P}} := x.\clift{\vec{P}}
   \and x!(y) := \lift{x}{\dropn{y}}
   \and \Pi_{i=0}^{n-1}P_i := P_0 | \ldots | P_{n-1}
\end{mathpar}

\subsubsection{Structural congruence}

\paragraph{Free and bound names and alpha-equivalence.} At the
core of structural equivalence is alpha-equivalence which identifies
process that are the same up to a change of variable. Formally, we
recognize the distinction between free and bound names. The free names
of a process, $\freenames{P}$, may be calculated recursively as
follows:

\begin{mathpar}
\freenames{\pzero} := \emptyset
  \and \\
  \freenames{x?(y).P} := \{ x \} \cup (\freenames{P} \setminus \{ y \})
  \and 
  \freenames{x!\langle P \rangle} := \{ x \} \cup \{ P \} 
  \and \\
  \freenames{P|Q} := \freenames{P} \cup \freenames{Q}
  \and \\
  \freenames{@{x}} := \{ x \}
\end{mathpar}

$\pi$
$\quotep{\pi}$

$\freenames{-} : \pi \to \mathcal{P}(\quotep{\pi})$

\begin{eqnarray*}
  \freenames{\pzero} & := & \emptyset \\
  \freenames{x?(y).P} & := & \{ x \} \cup (\freenames{P} \setminus \{ y \}) \\
  \freenames{x!\langle P \rangle} & := & \{ x \} \cup \{ P \} \\
  \freenames{P|Q} & := & \freenames{P} \cup \freenames{Q} \\
  \freenames{\dropn{x}} & := & \{ x \}
\end{eqnarray*}

The bound names of a process, $\boundnames{P}$, are those names occurring in $P$
that are not free. For example, in $x?(y).0$, the name $x$ is free, while $y$ is bound.

\begin{mathpar}
  \inferrule* [lab=monoidal-laws] {} { P|Q \equiv Q|P \and P|0 \equiv P \and P|(Q|R) \equiv (P|Q)|R }
\end{mathpar}

\begin{mathpar}
  \inferrule* [lab=alpha-equivalence] {} { (x)P \equiv (y)P\{y/x\} \and y \not\in \freenames{P} }
\end{mathpar}

\begin{definition}
Then two processes, $P,Q$, are alpha-equivalent if $P = Q\{\vec{y}/\vec{x}\}$ for
some $\vec{x} \in \boundnames{Q},\vec{y} \in \boundnames{P}$, where $Q\{\vec{y}/\vec{x}\}$
denotes the capture-avoiding substitution of $\vec{y}$ for $\vec{x}$ in $Q$.
\end{definition}

\begin{definition}
  The {\em structural congruence} \cite{SangiorgiWalker} , $\equiv$,
  between processes is the least congruence containing
  alpha-equivalence, satisfying the abelian monoid laws
  (associativity, commutativity and $\pzero$ as identity) for parallel
  composition $|$ and for summation $+$.
\end{definition}

\subsection{Name equivalence}

We take name equivalence, written $\nameeq$, to be the smallest
equivalence relation generated by the following rules.

\begin{mathpar}
\inferrule*[lab=Quote-drop]
{ }
{ \quotep{@{x}} \nameeq x }

\inferrule*[lab=Struct-equiv]
{ P \scong Q }
{ \quotep{P} \nameeq \quotep{Q} }
\end{mathpar}

The astute reader will have noticed that the mutual recursion of names
and processes imposes a mutual recursion on alpha-equivalence and
structural equivalence via name-equivalence. Fortunately, all of this
works out pleasantly and we may calculate in the natural way, free of
concern. The reader interested in the details is referred to the
appendix \ref{appendix:rho_details}.

\subsection{Substitution}

We use $\Proc$ for the set of processes, $\QProc$ for the set of
names, and $\id{\{}\vec{y} / \vec{x} \id{\}}$ to denote partial maps,
$s : \QProc \rightarrow \QProc$. A map, $s$ lifts, uniquely, to a map
on process terms, $\widehat{s} : \Proc \rightarrow \Proc$ by the
following equations.

\begin{mathpar}
  (0) \psubstp{Q}{P} := 0 \\
  (R \juxtap S) \psubstp{Q}{P}
  :=    
  (R)\psubstp{Q}{P} \juxtap (S) \psubstp{Q}{P} \\
  (x?(y).R) \psubstp{Q}{P}    
  :=    
  (x)\substp{Q}{P} (z)\concat( (R \psubstn{z}{y}) \psubstp{Q}{P} ) \\
  (\lift{x}{R}) \psubstp{Q}{P}  
  :=
  \lift{(x)\substp{Q}{P}}{ R \psubstp{Q}{P} } \\
%   (\dropn{x})  \psubstp{Q}{P}       
%   := 
%   \left\{ 
%     \begin{array}{ccc} 
%       \dropn{\quotep{Q}} & & x \nameeq \quotep{P} \\
%       \dropn{x} & & otherwise \\
%     \end{array}
%   \right. 
  (\dropn{x})  \psubstp{Q}{P}       
  := 
  \left\{ 
    \begin{array}{ccc} 
      Q & & x \nameeq \quotep{P} \\
      \dropn{x} & & otherwise \\
    \end{array}
  \right.
\end{mathpar}
 

where

\begin{eqnarray}
  (x)\id{\{} \lpquote Q \rpquote / \lpquote P \rpquote \id{\}}            = 
  \left\{ 
    \begin{array}{ccc}
      \lpquote Q \rpquote & & x \nameeq \lpquote P \rpquote \\
      x & & otherwise \\
    \end{array}
  \right. \nonumber
\end{eqnarray}

and $z$ is chosen distinct from $\quotep{P}$, $\quotep{Q}$, the free
names in $Q$, and all the names in $R$. Our $\alpha$-equivalence will
be built in the standard way from this substitution.

\begin{remark}\label{rem:no_self_referential_names}
  One consequence of these definitions is that $\forall P. \quotep{P}
  \not\in \freenames{P}$.
\end{remark}

\subsection{ Dynamic quote: an example }

Anticipating something of what's to come, consider applying the
substitution, $\widehat{\id{\{}u / z \id{\}}}$, to the following pair
of processes, $\lift{w}{y!(z)}$ and $w[ \lpquote y!(z) \rpquote ]$.

\begin{eqnarray}
	\lift{w}{y!(z)}\widehat{\id{\{}u / z \id{\}}}
		& = &
		\lift{w}{y!(u)} \nonumber\\
	w[ \lpquote y!(z) \rpquote ] \widehat{ \id{\{}u / z \id{\}} }
		& = &
		w[ \lpquote y!(z) \rpquote ] \nonumber
\end{eqnarray}

Because the body of the process between quotes is impervious to
substitution, we get radically different answers. In fact, by
examining the first process in an input context,
e.g. $x?(z).\lift{w}{y!(z)}$, we see that the process under the lift
operator may be shaped by prefixed inputs binding a name inside it. In
this sense, the lift operator will be seen as a way to dynamically
construct processes before reifying them as names.

Finally equipped with these standard features we can present the
dynamics of the calculus.

\subsubsection{Operational semantics} 

Finally, we introduce the computational dynamics. What marks these
algebras as distinct from other more traditionally studied algebraic
structures, e.g. vector spaces or polynomial rings, is the manner in
which dynamics is captured. In traditional structures, dynamics is typically
expressed through morphisms between such structures, as in linear maps
between vector spaces or morphisms between rings. In algebras
associated with the semantics of computation, the dynamics is
expressed as part of the algebraic structure itself, through a
reduction reduction relation typically denoted by $\red$. Below, we
give a recursive presentation of this relation for the calculus used
in the encoding.

$\red \subseteq \pi \times \pi$
$\red : \pi \to \mathcal{P}(\pi)$

\begin{mathpar}
  \inferrule* [lab=Comm] { \textsf{match}( x_{src}, x_{trgt} ) } { x_{trgt}?(y)P \; | \; x_{src}!\langle {Q} \rangle \red P\{\quotep{Q}/y}\} }
  \and \\
  \inferrule* [lab=Par] {{P} \red {P}'} {{{P} | {Q}} \red {{P}' | {Q}}}
  \and
  \inferrule* [lab=Equiv]{{{P} \scong {P}'} \andalso {{P}' \red {Q}'} \andalso {{Q}' \scong {Q}}}{{P} \red {Q}}
\end{mathpar}

\begin{eqnarray*}
  match_{\equiv} (\quotep{P},\quotep{Q}) & := & P \equiv Q \\
  match_{\dagger}(\quotep{P},\quotep{Q}) & := & \forall R. P|Q \red^{*} R => R \red^{*} 0 \\
  match_{K}(\quotep{P},\quotep{Q}) & := & K \mbox{ for some context } K
\end{eqnarray*}

$u?(x)P | u!\langle Q \rangle \red P\{\quotep{Q}/x\}$

%We write $\wred$ for $\red^*$, and $P\red$ if $\exists Q $ such that $ P \red Q$.
We write $P\red$ if $\exists Q $ such that $ P \red Q$ and $P\not\red$, otherwise.

\section{Replication}

As mentioned before, it is known that replication (and hence
recursion) can be implemented in a higher-order process algebra
\cite{SangiorgiWalker}. As our first example of calculation with the
machinery thus far presented we give the construction explicitly in
the {\rhoc}.

\begin{eqnarray}
	D_{x} & := & \prefix{x}{y}{(\binpar{\outputp{x}{y}}{@{y}})} \nonumber\\
	\bangp_{x}{P} & := & \binpar{{x}!\langle{\binpar{D_{x}}{P}}\rangle}{D_{x}} \nonumber
\end{eqnarray}

\begin{eqnarray}
	\bangp_{x}{P} & & \nonumber\\
	=
	& {x}!\langle{(\prefix{x}{y}{(\outputp{x}{y} | @{y})) | P}}\rangle 
	      | \prefix{x}{y}{(\outputp{x}{y} | @{y})} & \nonumber\\
	\red
	& (\outputp{x}{y} | @{y})\substn{\quotep{(\prefix{x}{y}{(@{y} | \outputp{x}{y})) | P}}}{y} & \nonumber\\
	=
	& \outputp{x}{\quotep{(\prefix{x}{y}{(\outputp{x}{y} | @{y})) | P}}}
	  | {(\prefix{x}{y}{(\outputp{x}{y} | @{y})) | P}} & \nonumber\\
	\red
	& \ldots & \nonumber\\
	\red^*
	& P | P | \ldots & \nonumber
\end{eqnarray}

Of course, this encoding, as an implementation, runs away, unfolding
$\bangp{P}$ eagerly. A lazier and more implementable replication
operator, restricted to input-guarded processes, may be obtained as follows.

\begin{eqnarray}
\bangp{\prefix{u}{v}{P}} 
	:= 
	\binpar{\lift{x}{\prefix{u}{v}{(\binpar{D(x)}{P})}}}{D(x)} \nonumber
\end{eqnarray}

\begin{remark}
  Note that the lazier definition still does not deal with summation
  or mixed summation (i.e. sums over input and output). The reader is
  invited to construct definitions of replication that deal with these
  features. 

  Further, the definitions are parameterized in a name, $x$. Can you,
  gentle reader, make a definition that eliminates this parameter and
  guarantees no accidental interaction between the replication
  machinery and the process being replicated -- i.e. no accidental
  sharing of names used by the process to get its work done and the
  name(s) used by the replication to effect copying. This latter
  revision of the definition of replication is crucial to obtaining
  the expected identity $!!P \sim !P$.
\end{remark}

\begin{remark}\label{rem:paradoxical_combinator}
  The reader familiar with the lambda calculus will have noticed the
  similarity between $D$ and the paradoxical combinator.

  [Ed. note: the existence of this seems to suggest we have to be more
  restrictive on the set of processes and names we admit if we are to
  support no-cloning.]
\end{remark}

\subsubsection{Bisimulation}

The computational dynamics gives rise to another kind of equivalence,
the equivalence of computational behavior. As previously mentioned
this is typically captured \emph{via} some form of bisimulation.

% The notion we use in this paper is weak barbed bisimulation
% \cite{milner91polyadicpi}.

The notion we use in this paper is derived from weak barbed
bisimulation \cite{milner91polyadicpi}. 

\begin{definition}
An \emph{observation relation}, $\downarrow_{\mathcal N}$, over a set
of names, $\mathcal N$, is the smallest relation satisfying the rules
below.

\infrule[Out-barb]{y \in {\mathcal N}, \; x \nameeq y}
		  {\outputp{x}{v} \downarrow_{\mathcal N} x}
\infrule[Par-barb]{\mbox{$P\downarrow_{\mathcal N} x$ or $Q\downarrow_{\mathcal N} x$}}
		  {\binpar{P}{Q} \downarrow_{\mathcal N} x}

We write $P \Downarrow_{\mathcal N} x$ if there is $Q$ such that 
$P \wred Q$ and $Q \downarrow_{\mathcal N} x$.
\end{definition}

\begin{definition}
%\label{def.bbisim}
An  ${\mathcal N}$-\emph{barbed bisimulation} over a set of names, ${\mathcal N}$, is a symmetric binary relation 
${\mathcal S}_{\mathcal N}$ between agents such that $P\rel{S}_{\mathcal N}Q$ implies:
\begin{enumerate}
\item If $P \red P'$ then $Q \wred Q'$ and $P'\rel{S}_{\mathcal N} Q'$.
\item If $P\downarrow_{\mathcal N} x$, then $Q\Downarrow_{\mathcal N} x$.
\end{enumerate}
$P$ is ${\mathcal N}$-barbed bisimilar to $Q$, written
$P \wbbisim_{\mathcal N} Q$, if $P \rel{S}_{\mathcal N} Q$ for some ${\mathcal N}$-barbed bisimulation ${\mathcal S}_{\mathcal N}$.
\end{definition}

$\mathcal{R} \subseteq \pi \times \pi$

$P \mathcal{R} Q => \forall P'. P \red P' \Rightarrow \exists Q'. Q \red Q', P' \mathcal{R} Q'$

$P \vdash x \Rightarrow Q \vdash x$

\begin{mathpar}
  \inferrule*[lab=Out-barb]{x \nameeq y}{{y}!\langle{Q}\rangle \vdash x}
  \and
  \inferrule*[lab=Par-barb]{\mbox{$P\vdash x$ or $Q\vdash x$}}{\binpar{P}{Q} \vdash x}
\end{mathpar}

\subsubsection{Contexts}

One of the principle advantages of computational calculi like the
$\pi$-calculus is a well-defined notion of context,
contextual-equivalence and a correlation between
contextual-equivalence and notions of bisimulation. The notion of
context allows the decomposition of a process into (sub-)process and
its syntactic environment, its context. Thus, a context may be
thought of as a process with a ``hole'' (written $\Box$) in it. The
application of a context $M$ to a process $P$, written $M[P]$, is
tantamount to filling the hole in $M$ with $P$. In this paper we do
not need the full weight of this theory, but do make use of the notion
of context in the proof the main theorem. 

\begin{mathpar}
  \inferrule* [lab=summation] {} {{M_{M},M_{N}} \bc \Box \;|\; x.M_{A} \;|\; M_{M}+M_{N}}
  \and
  \inferrule* [lab=agent] {} {{M_{A}} \bc (\vec{x})M_{P} \;| \; \clift{P_0,\ldots,M_{P},\ldots,P_N}}
  \and \\
  \inferrule* [lab=process] {} {{M_{P}} \bc M_{N} \;| \;P|M_{P} }
\end{mathpar} 

\begin{mathpar}
  \inferrule* [lab=sychronization] {} {M_{N} \bc \Box \;|\; x?M_{F} \;|\; x!M_{C}}
  \and
  \inferrule* [lab=abstraction] {} {{M_{F}} \bc (x)M_{P} }
  \and
  \inferrule* [lab=concretion] {} {{M_{C}} \bc \langle M_{P} \rangle }
  \and \\
  \inferrule* [lab=process] {} {{M_{P}} \bc M_{N} \;| \;P|M_{P} }
\end{mathpar}

\begin{definition}[contextual application] Given a context $M$, and
  process $P$, we define the \emph{contextual application}, $M[P] :=
  M\{P/\Box\}$. That is, the contextual application of M to P is the
  substitution of $P$ for $\Box$ in $M$.
\end{definition}

$\meaningof{-} : L \to \mathcal{P}(\pi)$

\begin{mathpar}
  \inferrule* [lab=collection] {} {\meaningof{true} = \pi, \and \meaningof{~E} = \pi \setminus \meaningof{E}, \and \meaningof{E_{1} \& E_{2}} = \meaningof{E_{1}} \cap \meaningof{E_{2}}}
\end{mathpar}

\begin{mathpar}
  \inferrule* [lab=structure] {} {\meaningof{0} = \{ P \in \pi | P \equiv 0 \}, \and \\ \meaningof{E_1 | E_2} = \{ P \in \pi | P \equiv P_{1} | P_{2}, P_{1} \in \meaningof{E_{1}}, P_{2} \in \meaningof{E_2}\} }
\end{mathpar}

\begin{mathpar}
 \inferrule* [lab=behavior] {} {\meaningof{\langle a?b \rangle E} = \{ P \in \pi | P \equiv Q | u?(y)P', \\ \and \\\\ \and \\ \;\;\; u \in \meaningof{a}, \forall z.P'\{z/y\} \in \meaningof{E\{z/b\}}\}, \and \\ \meaningof{a!E} = \{ P \in \pi | P \equiv Q | x!\langle P' \rangle, x \in \meaningof{a} P' \in \meaningof{E}\} }
\end{mathpar}

\begin{mathpar}
 \inferrule* [lab=nominal] {} {\meaningof{\quotep{E}} = \{ \quotep{P} \in \quotep{\pi} | P \in \meaningof{E} \}, \and \meaningof{\quotep{P}} = \{ \quotep{Q} \in \quotep{\pi} | P \equiv Q \} \and \\ \meaningof{@\quotep{E}} = \{ P \in \pi | P \equiv @x, x \in \meaningof{E} \}}
\end{mathpar}

\begin{eqnarray*}
  \\
  \meaningof{-} : TS \to ST
\end{eqnarray*}

\begin{eqnarray*}
  \\
  L : TS \to ST
\end{eqnarray*}

\begin{eqnarray*}
  \\
  P \models E \iff P \in \meaningof{E}
\end{eqnarray*}

\begin{eqnarray*}
  P \approx_{L} Q \iff \forall E \in L. P \models E \iff Q \models E
\end{eqnarray*}

\begin{eqnarray*}
  P \approx_{K} Q
\end{eqnarray*}

\begin{eqnarray*}
  P \approx Q
\end{eqnarray*}

$\approx_{K} = \approx = \approx_{L}$

\subsubsection{Contextual duality}

Note that contexts extend the quotation operation to a family of
operations from processes to names. Given a context, $M$, we can
define a \emph{nominal context}, $\quotep{M}$ by $\quotep{M}[P] :=
\quotep{M[P]}$. To foreshadow what is to come we observe that these
operations enjoy a duality with processes very much like the duality
between vectors and maps from vectors to scalars.

Further, because the calculus is essentially higher-order, we have a
correspondence between contexts and processes. More specifically,
given a name $x$ and a context $M$ we can construct $M^{*}_{x}$ such
that 

\begin{mathpar}
  M^{*}_{x} | \lift{x}{P} \red M[P]
\end{mathpar}

namely,

\begin{mathpar}
  M^{*}_{x} := x?(u).M[\dropn{u}]
\end{mathpar}

The dependence of $M^{*}_{x}$ on a name makes it an abstraction, 

\begin{mathpar}
  M^{*} := (x)x?(u).M[\dropn{u}]
\end{mathpar}

\subsection{Additional notation}

It will sometimes be convenient to denote the process a name
quotes. We already have the notation $x = \quotep{P}$, but it will be
convenient to introduce an alternate notation, $\procn{x}$, when we
want to emphasize the connection to the use of the name. Note that, by
virtue of name equivalence, $\quotep{\procn{x}} \nameeq x$; so, the
notation is consistent with previous definitions.

Further, because names have structure it is possible to effect
substitutions on the basis of that structure. This means we need to
upgrade our notation for substitutions, which we accomplish by
adapting comprehension notation. Thus,

\begin{mathpar}
  P\{ y / x : x \in S \}
\end{mathpar}

is interpreted to mean the process derived from P by replacing (in a
capture-avoiding manner) each occurrence of $x$ in $S$ by $y$. For example,

\begin{mathpar}
  P\{ \quotep{\procn{x}|\procn{x}} / x : x \in \freenames{P} \}
\end{mathpar}

will replace each (occurrence) of a free name $x$ in $P$ by
$\quotep{\procn{x}|\procn{x}}$.

Also, we will avail ourselves of the notation $x^{L}$ and $x^{R}$ to
denote injections of a name into disjoint copies of the name
space. There are numerous ways to accomplish this. One example can be
found in \cite{MeredithR05}. This notation overloads to vectors of
names: $\vec{x}^{\pi} := (x_{i}^{\pi} \; : \; 0 \leq i < |\vec{x}| )$ where $\pi \in \{L,R\}$.

We also use $P^{\Box} := P|\Box$.

In \cite{MeredithR05} an interpretation of the new operator is
given. It turns out that there are several possible interpretations
all enjoying the requisite algebraic properties of the operator (see
\cite{milner91polyadicpi}). We will therefore make liberal use of
$(\nu\; \vec{x})P$.

% subsection the_syntax_and_semantics_of_the_notation_system (end)   

\input{qm2pi.qmops} 

\input{qm2pi.sterngerlach} 

\input{qm2pi.metric} 

% section concurrent_process_calculi (end)

%\input{qm2pi.proofsketch}

% section proof sketch (end)

%\input{qm2pi.slviaknots} 

% section spatial logic via knots (end)

\input{qm2pi.conclusion}

% section conclusion (end)

%\input{qm2pi.dtcodes} 

% section wiring algorithm (end)

\input{qm2pi.ack} 

% section acknowledgments (end)

\newpage


\bibliographystyle{plain}   
\bibliography{../../biblios/main.bib}

\input{qm2pi.rhodetails}

\end{document}

 

% section wiring algorithm (end)

\documentclass[12pt]{llncs}
%\documentclass{jktr}

\usepackage[pdftex]{hyperref}                   
\usepackage {listings}
\usepackage {mathpartir}
\usepackage{bcprules}
%\usepackage{listings}
                       
\usepackage{graphicx} 
%\usepackage[margins=2.5cm,nohead,nofoot]{geometry}
%\usepackage{geometry}
\usepackage{amsfonts}
\usepackage{amstext}
\usepackage{latexsym}
\usepackage{amssymb}
\usepackage{color}


%\include{myPreamble}
\include{qm2pi.local} 

%\ifpdf
%\usepackage[pdftex]{graphicx}
%\else
%\usepackage{graphicx}
%\fi

 % \ifpdf
%  \usepackage{pdfsync}
%  \if


%\title{Brief Article}
%\author{David F. Snyder}
%\author{L.G. Meredith}

%\address{Dept. of Math., Texas State University--San Marcos, San Marcos, TX 78666}
       
\pagestyle{empty}


\begin{document}

\lstset{language=[Objective]Caml,frame=shadowbox}

\input{qm2pi.front}

% section front matter (end)

\input{qm2pi.intro} 
 
% section introduction (end)

% \input{qm2pi.knotations} 

% section notation (end)

\input{qm2pi.process.calculi} 

% section concurrent_process_calculi_and_spatial_logics_ (end)
    
%\input{qm2pi.knots2pi} 

%\input{qm2pi.trefoil} 

%\input{qm2pi.mainthm} 

% subsection basic_interpretation (end)

%\input{qm2pi.rho.presentation} 
\subsection{The syntax and semantics of the notation system}\label{sub:the_syntax_and_semantics_of_the_notation_system} % (fold)

We now summarize a technical presentation of the calculus that
embodies our theory of dynamics. The typical presentation of such a
calculus follows the style of giving generators and relations on
them. The grammar, below, describing term constructors, freely
generates the set of processes, $\Proc$. This set is then quotiented
by a relation known as structural congruence and it is over this set
that the notion of dynamics is expressed. This presentation is
essentially that of \cite{MeredithR05} with the addition of
polyadicity and summation. For readability we have relegated some of
the technical subtleties to an appendix.

\subsubsection{Process grammar}\label{subsub:process_grammar}

\begin{mathpar}
  \inferrule* [lab=synchronization] {} {{M} \bc \pzero \;|\; x?F \;|\; x!C }
  \and
  \inferrule* [lab=abstraction] {} {{F} \bc (x)P}
  \and
  \inferrule* [lab=concretion] {} {{C} \bc \langle Q \rangle}
  \and
  \inferrule* [lab=process] {} {{P,Q} \bc M \;| \;P|Q \;|\; @{x}}
  \and
  \inferrule* [lab=name] {} {{x} \bc \quotep{P}}
\end{mathpar} 

Note that $\vec{x}$ (resp. $\vec{P}$) denotes a vector of names
(resp. processes) of length $|\vec{x}|$ (resp. $|\vec{P}|$). We adopt
the following useful abbreviations.

\begin{mathpar}
   x?(\vec{y}).P := x.(\vec{y})P \and  x\clift{\vec{P}} := x.\clift{\vec{P}}
   \and x!(y) := \lift{x}{\dropn{y}}
   \and \Pi_{i=0}^{n-1}P_i := P_0 | \ldots | P_{n-1}
\end{mathpar}

\subsubsection{Structural congruence}

\paragraph{Free and bound names and alpha-equivalence.} At the
core of structural equivalence is alpha-equivalence which identifies
process that are the same up to a change of variable. Formally, we
recognize the distinction between free and bound names. The free names
of a process, $\freenames{P}$, may be calculated recursively as
follows:

\begin{mathpar}
\freenames{\pzero} := \emptyset
  \and \\
  \freenames{x?(y).P} := \{ x \} \cup (\freenames{P} \setminus \{ y \})
  \and 
  \freenames{x!\langle P \rangle} := \{ x \} \cup \{ P \} 
  \and \\
  \freenames{P|Q} := \freenames{P} \cup \freenames{Q}
  \and \\
  \freenames{@{x}} := \{ x \}
\end{mathpar}

$\pi$
$\quotep{\pi}$

$\freenames{-} : \pi \to \mathcal{P}(\quotep{\pi})$

\begin{eqnarray*}
  \freenames{\pzero} & := & \emptyset \\
  \freenames{x?(y).P} & := & \{ x \} \cup (\freenames{P} \setminus \{ y \}) \\
  \freenames{x!\langle P \rangle} & := & \{ x \} \cup \{ P \} \\
  \freenames{P|Q} & := & \freenames{P} \cup \freenames{Q} \\
  \freenames{\dropn{x}} & := & \{ x \}
\end{eqnarray*}

The bound names of a process, $\boundnames{P}$, are those names occurring in $P$
that are not free. For example, in $x?(y).0$, the name $x$ is free, while $y$ is bound.

\begin{mathpar}
  \inferrule* [lab=monoidal-laws] {} { P|Q \equiv Q|P \and P|0 \equiv P \and P|(Q|R) \equiv (P|Q)|R }
\end{mathpar}

\begin{mathpar}
  \inferrule* [lab=alpha-equivalence] {} { (x)P \equiv (y)P\{y/x\} \and y \not\in \freenames{P} }
\end{mathpar}

\begin{definition}
Then two processes, $P,Q$, are alpha-equivalent if $P = Q\{\vec{y}/\vec{x}\}$ for
some $\vec{x} \in \boundnames{Q},\vec{y} \in \boundnames{P}$, where $Q\{\vec{y}/\vec{x}\}$
denotes the capture-avoiding substitution of $\vec{y}$ for $\vec{x}$ in $Q$.
\end{definition}

\begin{definition}
  The {\em structural congruence} \cite{SangiorgiWalker} , $\equiv$,
  between processes is the least congruence containing
  alpha-equivalence, satisfying the abelian monoid laws
  (associativity, commutativity and $\pzero$ as identity) for parallel
  composition $|$ and for summation $+$.
\end{definition}

\subsection{Name equivalence}

We take name equivalence, written $\nameeq$, to be the smallest
equivalence relation generated by the following rules.

\begin{mathpar}
\inferrule*[lab=Quote-drop]
{ }
{ \quotep{@{x}} \nameeq x }

\inferrule*[lab=Struct-equiv]
{ P \scong Q }
{ \quotep{P} \nameeq \quotep{Q} }
\end{mathpar}

The astute reader will have noticed that the mutual recursion of names
and processes imposes a mutual recursion on alpha-equivalence and
structural equivalence via name-equivalence. Fortunately, all of this
works out pleasantly and we may calculate in the natural way, free of
concern. The reader interested in the details is referred to the
appendix \ref{appendix:rho_details}.

\subsection{Substitution}

We use $\Proc$ for the set of processes, $\QProc$ for the set of
names, and $\id{\{}\vec{y} / \vec{x} \id{\}}$ to denote partial maps,
$s : \QProc \rightarrow \QProc$. A map, $s$ lifts, uniquely, to a map
on process terms, $\widehat{s} : \Proc \rightarrow \Proc$ by the
following equations.

\begin{mathpar}
  (0) \psubstp{Q}{P} := 0 \\
  (R \juxtap S) \psubstp{Q}{P}
  :=    
  (R)\psubstp{Q}{P} \juxtap (S) \psubstp{Q}{P} \\
  (x?(y).R) \psubstp{Q}{P}    
  :=    
  (x)\substp{Q}{P} (z)\concat( (R \psubstn{z}{y}) \psubstp{Q}{P} ) \\
  (\lift{x}{R}) \psubstp{Q}{P}  
  :=
  \lift{(x)\substp{Q}{P}}{ R \psubstp{Q}{P} } \\
%   (\dropn{x})  \psubstp{Q}{P}       
%   := 
%   \left\{ 
%     \begin{array}{ccc} 
%       \dropn{\quotep{Q}} & & x \nameeq \quotep{P} \\
%       \dropn{x} & & otherwise \\
%     \end{array}
%   \right. 
  (\dropn{x})  \psubstp{Q}{P}       
  := 
  \left\{ 
    \begin{array}{ccc} 
      Q & & x \nameeq \quotep{P} \\
      \dropn{x} & & otherwise \\
    \end{array}
  \right.
\end{mathpar}
 

where

\begin{eqnarray}
  (x)\id{\{} \lpquote Q \rpquote / \lpquote P \rpquote \id{\}}            = 
  \left\{ 
    \begin{array}{ccc}
      \lpquote Q \rpquote & & x \nameeq \lpquote P \rpquote \\
      x & & otherwise \\
    \end{array}
  \right. \nonumber
\end{eqnarray}

and $z$ is chosen distinct from $\quotep{P}$, $\quotep{Q}$, the free
names in $Q$, and all the names in $R$. Our $\alpha$-equivalence will
be built in the standard way from this substitution.

\begin{remark}\label{rem:no_self_referential_names}
  One consequence of these definitions is that $\forall P. \quotep{P}
  \not\in \freenames{P}$.
\end{remark}

\subsection{ Dynamic quote: an example }

Anticipating something of what's to come, consider applying the
substitution, $\widehat{\id{\{}u / z \id{\}}}$, to the following pair
of processes, $\lift{w}{y!(z)}$ and $w[ \lpquote y!(z) \rpquote ]$.

\begin{eqnarray}
	\lift{w}{y!(z)}\widehat{\id{\{}u / z \id{\}}}
		& = &
		\lift{w}{y!(u)} \nonumber\\
	w[ \lpquote y!(z) \rpquote ] \widehat{ \id{\{}u / z \id{\}} }
		& = &
		w[ \lpquote y!(z) \rpquote ] \nonumber
\end{eqnarray}

Because the body of the process between quotes is impervious to
substitution, we get radically different answers. In fact, by
examining the first process in an input context,
e.g. $x?(z).\lift{w}{y!(z)}$, we see that the process under the lift
operator may be shaped by prefixed inputs binding a name inside it. In
this sense, the lift operator will be seen as a way to dynamically
construct processes before reifying them as names.

Finally equipped with these standard features we can present the
dynamics of the calculus.

\subsubsection{Operational semantics} 

Finally, we introduce the computational dynamics. What marks these
algebras as distinct from other more traditionally studied algebraic
structures, e.g. vector spaces or polynomial rings, is the manner in
which dynamics is captured. In traditional structures, dynamics is typically
expressed through morphisms between such structures, as in linear maps
between vector spaces or morphisms between rings. In algebras
associated with the semantics of computation, the dynamics is
expressed as part of the algebraic structure itself, through a
reduction reduction relation typically denoted by $\red$. Below, we
give a recursive presentation of this relation for the calculus used
in the encoding.

$\red \subseteq \pi \times \pi$
$\red : \pi \to \mathcal{P}(\pi)$

\begin{mathpar}
  \inferrule* [lab=Comm] { \textsf{match}( x_{src}, x_{trgt} ) } { x_{trgt}?(y)P \; | \; x_{src}!\langle {Q} \rangle \red P\{\quotep{Q}/y}\} }
  \and \\
  \inferrule* [lab=Par] {{P} \red {P}'} {{{P} | {Q}} \red {{P}' | {Q}}}
  \and
  \inferrule* [lab=Equiv]{{{P} \scong {P}'} \andalso {{P}' \red {Q}'} \andalso {{Q}' \scong {Q}}}{{P} \red {Q}}
\end{mathpar}

\begin{eqnarray*}
  match_{\equiv} (\quotep{P},\quotep{Q}) & := & P \equiv Q \\
  match_{\dagger}(\quotep{P},\quotep{Q}) & := & \forall R. P|Q \red^{*} R => R \red^{*} 0 \\
  match_{K}(\quotep{P},\quotep{Q}) & := & K \mbox{ for some context } K
\end{eqnarray*}

$u?(x)P | u!\langle Q \rangle \red P\{\quotep{Q}/x\}$

%We write $\wred$ for $\red^*$, and $P\red$ if $\exists Q $ such that $ P \red Q$.
We write $P\red$ if $\exists Q $ such that $ P \red Q$ and $P\not\red$, otherwise.

\section{Replication}

As mentioned before, it is known that replication (and hence
recursion) can be implemented in a higher-order process algebra
\cite{SangiorgiWalker}. As our first example of calculation with the
machinery thus far presented we give the construction explicitly in
the {\rhoc}.

\begin{eqnarray}
	D_{x} & := & \prefix{x}{y}{(\binpar{\outputp{x}{y}}{@{y}})} \nonumber\\
	\bangp_{x}{P} & := & \binpar{{x}!\langle{\binpar{D_{x}}{P}}\rangle}{D_{x}} \nonumber
\end{eqnarray}

\begin{eqnarray}
	\bangp_{x}{P} & & \nonumber\\
	=
	& {x}!\langle{(\prefix{x}{y}{(\outputp{x}{y} | @{y})) | P}}\rangle 
	      | \prefix{x}{y}{(\outputp{x}{y} | @{y})} & \nonumber\\
	\red
	& (\outputp{x}{y} | @{y})\substn{\quotep{(\prefix{x}{y}{(@{y} | \outputp{x}{y})) | P}}}{y} & \nonumber\\
	=
	& \outputp{x}{\quotep{(\prefix{x}{y}{(\outputp{x}{y} | @{y})) | P}}}
	  | {(\prefix{x}{y}{(\outputp{x}{y} | @{y})) | P}} & \nonumber\\
	\red
	& \ldots & \nonumber\\
	\red^*
	& P | P | \ldots & \nonumber
\end{eqnarray}

Of course, this encoding, as an implementation, runs away, unfolding
$\bangp{P}$ eagerly. A lazier and more implementable replication
operator, restricted to input-guarded processes, may be obtained as follows.

\begin{eqnarray}
\bangp{\prefix{u}{v}{P}} 
	:= 
	\binpar{\lift{x}{\prefix{u}{v}{(\binpar{D(x)}{P})}}}{D(x)} \nonumber
\end{eqnarray}

\begin{remark}
  Note that the lazier definition still does not deal with summation
  or mixed summation (i.e. sums over input and output). The reader is
  invited to construct definitions of replication that deal with these
  features. 

  Further, the definitions are parameterized in a name, $x$. Can you,
  gentle reader, make a definition that eliminates this parameter and
  guarantees no accidental interaction between the replication
  machinery and the process being replicated -- i.e. no accidental
  sharing of names used by the process to get its work done and the
  name(s) used by the replication to effect copying. This latter
  revision of the definition of replication is crucial to obtaining
  the expected identity $!!P \sim !P$.
\end{remark}

\begin{remark}\label{rem:paradoxical_combinator}
  The reader familiar with the lambda calculus will have noticed the
  similarity between $D$ and the paradoxical combinator.

  [Ed. note: the existence of this seems to suggest we have to be more
  restrictive on the set of processes and names we admit if we are to
  support no-cloning.]
\end{remark}

\subsubsection{Bisimulation}

The computational dynamics gives rise to another kind of equivalence,
the equivalence of computational behavior. As previously mentioned
this is typically captured \emph{via} some form of bisimulation.

% The notion we use in this paper is weak barbed bisimulation
% \cite{milner91polyadicpi}.

The notion we use in this paper is derived from weak barbed
bisimulation \cite{milner91polyadicpi}. 

\begin{definition}
An \emph{observation relation}, $\downarrow_{\mathcal N}$, over a set
of names, $\mathcal N$, is the smallest relation satisfying the rules
below.

\infrule[Out-barb]{y \in {\mathcal N}, \; x \nameeq y}
		  {\outputp{x}{v} \downarrow_{\mathcal N} x}
\infrule[Par-barb]{\mbox{$P\downarrow_{\mathcal N} x$ or $Q\downarrow_{\mathcal N} x$}}
		  {\binpar{P}{Q} \downarrow_{\mathcal N} x}

We write $P \Downarrow_{\mathcal N} x$ if there is $Q$ such that 
$P \wred Q$ and $Q \downarrow_{\mathcal N} x$.
\end{definition}

\begin{definition}
%\label{def.bbisim}
An  ${\mathcal N}$-\emph{barbed bisimulation} over a set of names, ${\mathcal N}$, is a symmetric binary relation 
${\mathcal S}_{\mathcal N}$ between agents such that $P\rel{S}_{\mathcal N}Q$ implies:
\begin{enumerate}
\item If $P \red P'$ then $Q \wred Q'$ and $P'\rel{S}_{\mathcal N} Q'$.
\item If $P\downarrow_{\mathcal N} x$, then $Q\Downarrow_{\mathcal N} x$.
\end{enumerate}
$P$ is ${\mathcal N}$-barbed bisimilar to $Q$, written
$P \wbbisim_{\mathcal N} Q$, if $P \rel{S}_{\mathcal N} Q$ for some ${\mathcal N}$-barbed bisimulation ${\mathcal S}_{\mathcal N}$.
\end{definition}

$\mathcal{R} \subseteq \pi \times \pi$

$P \mathcal{R} Q => \forall P'. P \red P' \Rightarrow \exists Q'. Q \red Q', P' \mathcal{R} Q'$

$P \vdash x \Rightarrow Q \vdash x$

\begin{mathpar}
  \inferrule*[lab=Out-barb]{x \nameeq y}{{y}!\langle{Q}\rangle \vdash x}
  \and
  \inferrule*[lab=Par-barb]{\mbox{$P\vdash x$ or $Q\vdash x$}}{\binpar{P}{Q} \vdash x}
\end{mathpar}

\subsubsection{Contexts}

One of the principle advantages of computational calculi like the
$\pi$-calculus is a well-defined notion of context,
contextual-equivalence and a correlation between
contextual-equivalence and notions of bisimulation. The notion of
context allows the decomposition of a process into (sub-)process and
its syntactic environment, its context. Thus, a context may be
thought of as a process with a ``hole'' (written $\Box$) in it. The
application of a context $M$ to a process $P$, written $M[P]$, is
tantamount to filling the hole in $M$ with $P$. In this paper we do
not need the full weight of this theory, but do make use of the notion
of context in the proof the main theorem. 

\begin{mathpar}
  \inferrule* [lab=summation] {} {{M_{M},M_{N}} \bc \Box \;|\; x.M_{A} \;|\; M_{M}+M_{N}}
  \and
  \inferrule* [lab=agent] {} {{M_{A}} \bc (\vec{x})M_{P} \;| \; \clift{P_0,\ldots,M_{P},\ldots,P_N}}
  \and \\
  \inferrule* [lab=process] {} {{M_{P}} \bc M_{N} \;| \;P|M_{P} }
\end{mathpar} 

\begin{mathpar}
  \inferrule* [lab=sychronization] {} {M_{N} \bc \Box \;|\; x?M_{F} \;|\; x!M_{C}}
  \and
  \inferrule* [lab=abstraction] {} {{M_{F}} \bc (x)M_{P} }
  \and
  \inferrule* [lab=concretion] {} {{M_{C}} \bc \langle M_{P} \rangle }
  \and \\
  \inferrule* [lab=process] {} {{M_{P}} \bc M_{N} \;| \;P|M_{P} }
\end{mathpar}

\begin{definition}[contextual application] Given a context $M$, and
  process $P$, we define the \emph{contextual application}, $M[P] :=
  M\{P/\Box\}$. That is, the contextual application of M to P is the
  substitution of $P$ for $\Box$ in $M$.
\end{definition}

$\meaningof{-} : L \to \mathcal{P}(\pi)$

\begin{mathpar}
  \inferrule* [lab=collection] {} {\meaningof{true} = \pi, \and \meaningof{~E} = \pi \setminus \meaningof{E}, \and \meaningof{E_{1} \& E_{2}} = \meaningof{E_{1}} \cap \meaningof{E_{2}}}
\end{mathpar}

\begin{mathpar}
  \inferrule* [lab=structure] {} {\meaningof{0} = \{ P \in \pi | P \equiv 0 \}, \and \\ \meaningof{E_1 | E_2} = \{ P \in \pi | P \equiv P_{1} | P_{2}, P_{1} \in \meaningof{E_{1}}, P_{2} \in \meaningof{E_2}\} }
\end{mathpar}

\begin{mathpar}
 \inferrule* [lab=behavior] {} {\meaningof{\langle a?b \rangle E} = \{ P \in \pi | P \equiv Q | u?(y)P', \\ \and \\\\ \and \\ \;\;\; u \in \meaningof{a}, \forall z.P'\{z/y\} \in \meaningof{E\{z/b\}}\}, \and \\ \meaningof{a!E} = \{ P \in \pi | P \equiv Q | x!\langle P' \rangle, x \in \meaningof{a} P' \in \meaningof{E}\} }
\end{mathpar}

\begin{mathpar}
 \inferrule* [lab=nominal] {} {\meaningof{\quotep{E}} = \{ \quotep{P} \in \quotep{\pi} | P \in \meaningof{E} \}, \and \meaningof{\quotep{P}} = \{ \quotep{Q} \in \quotep{\pi} | P \equiv Q \} \and \\ \meaningof{@\quotep{E}} = \{ P \in \pi | P \equiv @x, x \in \meaningof{E} \}}
\end{mathpar}

\begin{eqnarray*}
  \\
  \meaningof{-} : TS \to ST
\end{eqnarray*}

\begin{eqnarray*}
  \\
  L : TS \to ST
\end{eqnarray*}

\begin{eqnarray*}
  \\
  P \models E \iff P \in \meaningof{E}
\end{eqnarray*}

\begin{eqnarray*}
  P \approx_{L} Q \iff \forall E \in L. P \models E \iff Q \models E
\end{eqnarray*}

\begin{eqnarray*}
  P \approx_{K} Q
\end{eqnarray*}

\begin{eqnarray*}
  P \approx Q
\end{eqnarray*}

$\approx_{K} = \approx = \approx_{L}$

\subsubsection{Contextual duality}

Note that contexts extend the quotation operation to a family of
operations from processes to names. Given a context, $M$, we can
define a \emph{nominal context}, $\quotep{M}$ by $\quotep{M}[P] :=
\quotep{M[P]}$. To foreshadow what is to come we observe that these
operations enjoy a duality with processes very much like the duality
between vectors and maps from vectors to scalars.

Further, because the calculus is essentially higher-order, we have a
correspondence between contexts and processes. More specifically,
given a name $x$ and a context $M$ we can construct $M^{*}_{x}$ such
that 

\begin{mathpar}
  M^{*}_{x} | \lift{x}{P} \red M[P]
\end{mathpar}

namely,

\begin{mathpar}
  M^{*}_{x} := x?(u).M[\dropn{u}]
\end{mathpar}

The dependence of $M^{*}_{x}$ on a name makes it an abstraction, 

\begin{mathpar}
  M^{*} := (x)x?(u).M[\dropn{u}]
\end{mathpar}

\subsection{Additional notation}

It will sometimes be convenient to denote the process a name
quotes. We already have the notation $x = \quotep{P}$, but it will be
convenient to introduce an alternate notation, $\procn{x}$, when we
want to emphasize the connection to the use of the name. Note that, by
virtue of name equivalence, $\quotep{\procn{x}} \nameeq x$; so, the
notation is consistent with previous definitions.

Further, because names have structure it is possible to effect
substitutions on the basis of that structure. This means we need to
upgrade our notation for substitutions, which we accomplish by
adapting comprehension notation. Thus,

\begin{mathpar}
  P\{ y / x : x \in S \}
\end{mathpar}

is interpreted to mean the process derived from P by replacing (in a
capture-avoiding manner) each occurrence of $x$ in $S$ by $y$. For example,

\begin{mathpar}
  P\{ \quotep{\procn{x}|\procn{x}} / x : x \in \freenames{P} \}
\end{mathpar}

will replace each (occurrence) of a free name $x$ in $P$ by
$\quotep{\procn{x}|\procn{x}}$.

Also, we will avail ourselves of the notation $x^{L}$ and $x^{R}$ to
denote injections of a name into disjoint copies of the name
space. There are numerous ways to accomplish this. One example can be
found in \cite{MeredithR05}. This notation overloads to vectors of
names: $\vec{x}^{\pi} := (x_{i}^{\pi} \; : \; 0 \leq i < |\vec{x}| )$ where $\pi \in \{L,R\}$.

We also use $P^{\Box} := P|\Box$.

In \cite{MeredithR05} an interpretation of the new operator is
given. It turns out that there are several possible interpretations
all enjoying the requisite algebraic properties of the operator (see
\cite{milner91polyadicpi}). We will therefore make liberal use of
$(\nu\; \vec{x})P$.

% subsection the_syntax_and_semantics_of_the_notation_system (end)   

\input{qm2pi.qmops} 

\input{qm2pi.sterngerlach} 

\input{qm2pi.metric} 

% section concurrent_process_calculi (end)

%\input{qm2pi.proofsketch}

% section proof sketch (end)

%\input{qm2pi.slviaknots} 

% section spatial logic via knots (end)

\input{qm2pi.conclusion}

% section conclusion (end)

%\input{qm2pi.dtcodes} 

% section wiring algorithm (end)

\input{qm2pi.ack} 

% section acknowledgments (end)

\newpage


\bibliographystyle{plain}   
\bibliography{../../biblios/main.bib}

\input{qm2pi.rhodetails}

\end{document}

 

% section acknowledgments (end)

\newpage


\bibliographystyle{plain}   
\bibliography{../../biblios/main.bib}

\documentclass[12pt]{llncs}
%\documentclass{jktr}

\usepackage[pdftex]{hyperref}                   
\usepackage {listings}
\usepackage {mathpartir}
\usepackage{bcprules}
%\usepackage{listings}
                       
\usepackage{graphicx} 
%\usepackage[margins=2.5cm,nohead,nofoot]{geometry}
%\usepackage{geometry}
\usepackage{amsfonts}
\usepackage{amstext}
\usepackage{latexsym}
\usepackage{amssymb}
\usepackage{color}


%\include{myPreamble}
\include{qm2pi.local} 

%\ifpdf
%\usepackage[pdftex]{graphicx}
%\else
%\usepackage{graphicx}
%\fi

 % \ifpdf
%  \usepackage{pdfsync}
%  \if


%\title{Brief Article}
%\author{David F. Snyder}
%\author{L.G. Meredith}

%\address{Dept. of Math., Texas State University--San Marcos, San Marcos, TX 78666}
       
\pagestyle{empty}


\begin{document}

\lstset{language=[Objective]Caml,frame=shadowbox}

\input{qm2pi.front}

% section front matter (end)

\input{qm2pi.intro} 
 
% section introduction (end)

% \input{qm2pi.knotations} 

% section notation (end)

\input{qm2pi.process.calculi} 

% section concurrent_process_calculi_and_spatial_logics_ (end)
    
%\input{qm2pi.knots2pi} 

%\input{qm2pi.trefoil} 

%\input{qm2pi.mainthm} 

% subsection basic_interpretation (end)

%\input{qm2pi.rho.presentation} 
\subsection{The syntax and semantics of the notation system}\label{sub:the_syntax_and_semantics_of_the_notation_system} % (fold)

We now summarize a technical presentation of the calculus that
embodies our theory of dynamics. The typical presentation of such a
calculus follows the style of giving generators and relations on
them. The grammar, below, describing term constructors, freely
generates the set of processes, $\Proc$. This set is then quotiented
by a relation known as structural congruence and it is over this set
that the notion of dynamics is expressed. This presentation is
essentially that of \cite{MeredithR05} with the addition of
polyadicity and summation. For readability we have relegated some of
the technical subtleties to an appendix.

\subsubsection{Process grammar}\label{subsub:process_grammar}

\begin{mathpar}
  \inferrule* [lab=synchronization] {} {{M} \bc \pzero \;|\; x?F \;|\; x!C }
  \and
  \inferrule* [lab=abstraction] {} {{F} \bc (x)P}
  \and
  \inferrule* [lab=concretion] {} {{C} \bc \langle Q \rangle}
  \and
  \inferrule* [lab=process] {} {{P,Q} \bc M \;| \;P|Q \;|\; @{x}}
  \and
  \inferrule* [lab=name] {} {{x} \bc \quotep{P}}
\end{mathpar} 

Note that $\vec{x}$ (resp. $\vec{P}$) denotes a vector of names
(resp. processes) of length $|\vec{x}|$ (resp. $|\vec{P}|$). We adopt
the following useful abbreviations.

\begin{mathpar}
   x?(\vec{y}).P := x.(\vec{y})P \and  x\clift{\vec{P}} := x.\clift{\vec{P}}
   \and x!(y) := \lift{x}{\dropn{y}}
   \and \Pi_{i=0}^{n-1}P_i := P_0 | \ldots | P_{n-1}
\end{mathpar}

\subsubsection{Structural congruence}

\paragraph{Free and bound names and alpha-equivalence.} At the
core of structural equivalence is alpha-equivalence which identifies
process that are the same up to a change of variable. Formally, we
recognize the distinction between free and bound names. The free names
of a process, $\freenames{P}$, may be calculated recursively as
follows:

\begin{mathpar}
\freenames{\pzero} := \emptyset
  \and \\
  \freenames{x?(y).P} := \{ x \} \cup (\freenames{P} \setminus \{ y \})
  \and 
  \freenames{x!\langle P \rangle} := \{ x \} \cup \{ P \} 
  \and \\
  \freenames{P|Q} := \freenames{P} \cup \freenames{Q}
  \and \\
  \freenames{@{x}} := \{ x \}
\end{mathpar}

$\pi$
$\quotep{\pi}$

$\freenames{-} : \pi \to \mathcal{P}(\quotep{\pi})$

\begin{eqnarray*}
  \freenames{\pzero} & := & \emptyset \\
  \freenames{x?(y).P} & := & \{ x \} \cup (\freenames{P} \setminus \{ y \}) \\
  \freenames{x!\langle P \rangle} & := & \{ x \} \cup \{ P \} \\
  \freenames{P|Q} & := & \freenames{P} \cup \freenames{Q} \\
  \freenames{\dropn{x}} & := & \{ x \}
\end{eqnarray*}

The bound names of a process, $\boundnames{P}$, are those names occurring in $P$
that are not free. For example, in $x?(y).0$, the name $x$ is free, while $y$ is bound.

\begin{mathpar}
  \inferrule* [lab=monoidal-laws] {} { P|Q \equiv Q|P \and P|0 \equiv P \and P|(Q|R) \equiv (P|Q)|R }
\end{mathpar}

\begin{mathpar}
  \inferrule* [lab=alpha-equivalence] {} { (x)P \equiv (y)P\{y/x\} \and y \not\in \freenames{P} }
\end{mathpar}

\begin{definition}
Then two processes, $P,Q$, are alpha-equivalent if $P = Q\{\vec{y}/\vec{x}\}$ for
some $\vec{x} \in \boundnames{Q},\vec{y} \in \boundnames{P}$, where $Q\{\vec{y}/\vec{x}\}$
denotes the capture-avoiding substitution of $\vec{y}$ for $\vec{x}$ in $Q$.
\end{definition}

\begin{definition}
  The {\em structural congruence} \cite{SangiorgiWalker} , $\equiv$,
  between processes is the least congruence containing
  alpha-equivalence, satisfying the abelian monoid laws
  (associativity, commutativity and $\pzero$ as identity) for parallel
  composition $|$ and for summation $+$.
\end{definition}

\subsection{Name equivalence}

We take name equivalence, written $\nameeq$, to be the smallest
equivalence relation generated by the following rules.

\begin{mathpar}
\inferrule*[lab=Quote-drop]
{ }
{ \quotep{@{x}} \nameeq x }

\inferrule*[lab=Struct-equiv]
{ P \scong Q }
{ \quotep{P} \nameeq \quotep{Q} }
\end{mathpar}

The astute reader will have noticed that the mutual recursion of names
and processes imposes a mutual recursion on alpha-equivalence and
structural equivalence via name-equivalence. Fortunately, all of this
works out pleasantly and we may calculate in the natural way, free of
concern. The reader interested in the details is referred to the
appendix \ref{appendix:rho_details}.

\subsection{Substitution}

We use $\Proc$ for the set of processes, $\QProc$ for the set of
names, and $\id{\{}\vec{y} / \vec{x} \id{\}}$ to denote partial maps,
$s : \QProc \rightarrow \QProc$. A map, $s$ lifts, uniquely, to a map
on process terms, $\widehat{s} : \Proc \rightarrow \Proc$ by the
following equations.

\begin{mathpar}
  (0) \psubstp{Q}{P} := 0 \\
  (R \juxtap S) \psubstp{Q}{P}
  :=    
  (R)\psubstp{Q}{P} \juxtap (S) \psubstp{Q}{P} \\
  (x?(y).R) \psubstp{Q}{P}    
  :=    
  (x)\substp{Q}{P} (z)\concat( (R \psubstn{z}{y}) \psubstp{Q}{P} ) \\
  (\lift{x}{R}) \psubstp{Q}{P}  
  :=
  \lift{(x)\substp{Q}{P}}{ R \psubstp{Q}{P} } \\
%   (\dropn{x})  \psubstp{Q}{P}       
%   := 
%   \left\{ 
%     \begin{array}{ccc} 
%       \dropn{\quotep{Q}} & & x \nameeq \quotep{P} \\
%       \dropn{x} & & otherwise \\
%     \end{array}
%   \right. 
  (\dropn{x})  \psubstp{Q}{P}       
  := 
  \left\{ 
    \begin{array}{ccc} 
      Q & & x \nameeq \quotep{P} \\
      \dropn{x} & & otherwise \\
    \end{array}
  \right.
\end{mathpar}
 

where

\begin{eqnarray}
  (x)\id{\{} \lpquote Q \rpquote / \lpquote P \rpquote \id{\}}            = 
  \left\{ 
    \begin{array}{ccc}
      \lpquote Q \rpquote & & x \nameeq \lpquote P \rpquote \\
      x & & otherwise \\
    \end{array}
  \right. \nonumber
\end{eqnarray}

and $z$ is chosen distinct from $\quotep{P}$, $\quotep{Q}$, the free
names in $Q$, and all the names in $R$. Our $\alpha$-equivalence will
be built in the standard way from this substitution.

\begin{remark}\label{rem:no_self_referential_names}
  One consequence of these definitions is that $\forall P. \quotep{P}
  \not\in \freenames{P}$.
\end{remark}

\subsection{ Dynamic quote: an example }

Anticipating something of what's to come, consider applying the
substitution, $\widehat{\id{\{}u / z \id{\}}}$, to the following pair
of processes, $\lift{w}{y!(z)}$ and $w[ \lpquote y!(z) \rpquote ]$.

\begin{eqnarray}
	\lift{w}{y!(z)}\widehat{\id{\{}u / z \id{\}}}
		& = &
		\lift{w}{y!(u)} \nonumber\\
	w[ \lpquote y!(z) \rpquote ] \widehat{ \id{\{}u / z \id{\}} }
		& = &
		w[ \lpquote y!(z) \rpquote ] \nonumber
\end{eqnarray}

Because the body of the process between quotes is impervious to
substitution, we get radically different answers. In fact, by
examining the first process in an input context,
e.g. $x?(z).\lift{w}{y!(z)}$, we see that the process under the lift
operator may be shaped by prefixed inputs binding a name inside it. In
this sense, the lift operator will be seen as a way to dynamically
construct processes before reifying them as names.

Finally equipped with these standard features we can present the
dynamics of the calculus.

\subsubsection{Operational semantics} 

Finally, we introduce the computational dynamics. What marks these
algebras as distinct from other more traditionally studied algebraic
structures, e.g. vector spaces or polynomial rings, is the manner in
which dynamics is captured. In traditional structures, dynamics is typically
expressed through morphisms between such structures, as in linear maps
between vector spaces or morphisms between rings. In algebras
associated with the semantics of computation, the dynamics is
expressed as part of the algebraic structure itself, through a
reduction reduction relation typically denoted by $\red$. Below, we
give a recursive presentation of this relation for the calculus used
in the encoding.

$\red \subseteq \pi \times \pi$
$\red : \pi \to \mathcal{P}(\pi)$

\begin{mathpar}
  \inferrule* [lab=Comm] { \textsf{match}( x_{src}, x_{trgt} ) } { x_{trgt}?(y)P \; | \; x_{src}!\langle {Q} \rangle \red P\{\quotep{Q}/y}\} }
  \and \\
  \inferrule* [lab=Par] {{P} \red {P}'} {{{P} | {Q}} \red {{P}' | {Q}}}
  \and
  \inferrule* [lab=Equiv]{{{P} \scong {P}'} \andalso {{P}' \red {Q}'} \andalso {{Q}' \scong {Q}}}{{P} \red {Q}}
\end{mathpar}

\begin{eqnarray*}
  match_{\equiv} (\quotep{P},\quotep{Q}) & := & P \equiv Q \\
  match_{\dagger}(\quotep{P},\quotep{Q}) & := & \forall R. P|Q \red^{*} R => R \red^{*} 0 \\
  match_{K}(\quotep{P},\quotep{Q}) & := & K \mbox{ for some context } K
\end{eqnarray*}

$u?(x)P | u!\langle Q \rangle \red P\{\quotep{Q}/x\}$

%We write $\wred$ for $\red^*$, and $P\red$ if $\exists Q $ such that $ P \red Q$.
We write $P\red$ if $\exists Q $ such that $ P \red Q$ and $P\not\red$, otherwise.

\section{Replication}

As mentioned before, it is known that replication (and hence
recursion) can be implemented in a higher-order process algebra
\cite{SangiorgiWalker}. As our first example of calculation with the
machinery thus far presented we give the construction explicitly in
the {\rhoc}.

\begin{eqnarray}
	D_{x} & := & \prefix{x}{y}{(\binpar{\outputp{x}{y}}{@{y}})} \nonumber\\
	\bangp_{x}{P} & := & \binpar{{x}!\langle{\binpar{D_{x}}{P}}\rangle}{D_{x}} \nonumber
\end{eqnarray}

\begin{eqnarray}
	\bangp_{x}{P} & & \nonumber\\
	=
	& {x}!\langle{(\prefix{x}{y}{(\outputp{x}{y} | @{y})) | P}}\rangle 
	      | \prefix{x}{y}{(\outputp{x}{y} | @{y})} & \nonumber\\
	\red
	& (\outputp{x}{y} | @{y})\substn{\quotep{(\prefix{x}{y}{(@{y} | \outputp{x}{y})) | P}}}{y} & \nonumber\\
	=
	& \outputp{x}{\quotep{(\prefix{x}{y}{(\outputp{x}{y} | @{y})) | P}}}
	  | {(\prefix{x}{y}{(\outputp{x}{y} | @{y})) | P}} & \nonumber\\
	\red
	& \ldots & \nonumber\\
	\red^*
	& P | P | \ldots & \nonumber
\end{eqnarray}

Of course, this encoding, as an implementation, runs away, unfolding
$\bangp{P}$ eagerly. A lazier and more implementable replication
operator, restricted to input-guarded processes, may be obtained as follows.

\begin{eqnarray}
\bangp{\prefix{u}{v}{P}} 
	:= 
	\binpar{\lift{x}{\prefix{u}{v}{(\binpar{D(x)}{P})}}}{D(x)} \nonumber
\end{eqnarray}

\begin{remark}
  Note that the lazier definition still does not deal with summation
  or mixed summation (i.e. sums over input and output). The reader is
  invited to construct definitions of replication that deal with these
  features. 

  Further, the definitions are parameterized in a name, $x$. Can you,
  gentle reader, make a definition that eliminates this parameter and
  guarantees no accidental interaction between the replication
  machinery and the process being replicated -- i.e. no accidental
  sharing of names used by the process to get its work done and the
  name(s) used by the replication to effect copying. This latter
  revision of the definition of replication is crucial to obtaining
  the expected identity $!!P \sim !P$.
\end{remark}

\begin{remark}\label{rem:paradoxical_combinator}
  The reader familiar with the lambda calculus will have noticed the
  similarity between $D$ and the paradoxical combinator.

  [Ed. note: the existence of this seems to suggest we have to be more
  restrictive on the set of processes and names we admit if we are to
  support no-cloning.]
\end{remark}

\subsubsection{Bisimulation}

The computational dynamics gives rise to another kind of equivalence,
the equivalence of computational behavior. As previously mentioned
this is typically captured \emph{via} some form of bisimulation.

% The notion we use in this paper is weak barbed bisimulation
% \cite{milner91polyadicpi}.

The notion we use in this paper is derived from weak barbed
bisimulation \cite{milner91polyadicpi}. 

\begin{definition}
An \emph{observation relation}, $\downarrow_{\mathcal N}$, over a set
of names, $\mathcal N$, is the smallest relation satisfying the rules
below.

\infrule[Out-barb]{y \in {\mathcal N}, \; x \nameeq y}
		  {\outputp{x}{v} \downarrow_{\mathcal N} x}
\infrule[Par-barb]{\mbox{$P\downarrow_{\mathcal N} x$ or $Q\downarrow_{\mathcal N} x$}}
		  {\binpar{P}{Q} \downarrow_{\mathcal N} x}

We write $P \Downarrow_{\mathcal N} x$ if there is $Q$ such that 
$P \wred Q$ and $Q \downarrow_{\mathcal N} x$.
\end{definition}

\begin{definition}
%\label{def.bbisim}
An  ${\mathcal N}$-\emph{barbed bisimulation} over a set of names, ${\mathcal N}$, is a symmetric binary relation 
${\mathcal S}_{\mathcal N}$ between agents such that $P\rel{S}_{\mathcal N}Q$ implies:
\begin{enumerate}
\item If $P \red P'$ then $Q \wred Q'$ and $P'\rel{S}_{\mathcal N} Q'$.
\item If $P\downarrow_{\mathcal N} x$, then $Q\Downarrow_{\mathcal N} x$.
\end{enumerate}
$P$ is ${\mathcal N}$-barbed bisimilar to $Q$, written
$P \wbbisim_{\mathcal N} Q$, if $P \rel{S}_{\mathcal N} Q$ for some ${\mathcal N}$-barbed bisimulation ${\mathcal S}_{\mathcal N}$.
\end{definition}

$\mathcal{R} \subseteq \pi \times \pi$

$P \mathcal{R} Q => \forall P'. P \red P' \Rightarrow \exists Q'. Q \red Q', P' \mathcal{R} Q'$

$P \vdash x \Rightarrow Q \vdash x$

\begin{mathpar}
  \inferrule*[lab=Out-barb]{x \nameeq y}{{y}!\langle{Q}\rangle \vdash x}
  \and
  \inferrule*[lab=Par-barb]{\mbox{$P\vdash x$ or $Q\vdash x$}}{\binpar{P}{Q} \vdash x}
\end{mathpar}

\subsubsection{Contexts}

One of the principle advantages of computational calculi like the
$\pi$-calculus is a well-defined notion of context,
contextual-equivalence and a correlation between
contextual-equivalence and notions of bisimulation. The notion of
context allows the decomposition of a process into (sub-)process and
its syntactic environment, its context. Thus, a context may be
thought of as a process with a ``hole'' (written $\Box$) in it. The
application of a context $M$ to a process $P$, written $M[P]$, is
tantamount to filling the hole in $M$ with $P$. In this paper we do
not need the full weight of this theory, but do make use of the notion
of context in the proof the main theorem. 

\begin{mathpar}
  \inferrule* [lab=summation] {} {{M_{M},M_{N}} \bc \Box \;|\; x.M_{A} \;|\; M_{M}+M_{N}}
  \and
  \inferrule* [lab=agent] {} {{M_{A}} \bc (\vec{x})M_{P} \;| \; \clift{P_0,\ldots,M_{P},\ldots,P_N}}
  \and \\
  \inferrule* [lab=process] {} {{M_{P}} \bc M_{N} \;| \;P|M_{P} }
\end{mathpar} 

\begin{mathpar}
  \inferrule* [lab=sychronization] {} {M_{N} \bc \Box \;|\; x?M_{F} \;|\; x!M_{C}}
  \and
  \inferrule* [lab=abstraction] {} {{M_{F}} \bc (x)M_{P} }
  \and
  \inferrule* [lab=concretion] {} {{M_{C}} \bc \langle M_{P} \rangle }
  \and \\
  \inferrule* [lab=process] {} {{M_{P}} \bc M_{N} \;| \;P|M_{P} }
\end{mathpar}

\begin{definition}[contextual application] Given a context $M$, and
  process $P$, we define the \emph{contextual application}, $M[P] :=
  M\{P/\Box\}$. That is, the contextual application of M to P is the
  substitution of $P$ for $\Box$ in $M$.
\end{definition}

$\meaningof{-} : L \to \mathcal{P}(\pi)$

\begin{mathpar}
  \inferrule* [lab=collection] {} {\meaningof{true} = \pi, \and \meaningof{~E} = \pi \setminus \meaningof{E}, \and \meaningof{E_{1} \& E_{2}} = \meaningof{E_{1}} \cap \meaningof{E_{2}}}
\end{mathpar}

\begin{mathpar}
  \inferrule* [lab=structure] {} {\meaningof{0} = \{ P \in \pi | P \equiv 0 \}, \and \\ \meaningof{E_1 | E_2} = \{ P \in \pi | P \equiv P_{1} | P_{2}, P_{1} \in \meaningof{E_{1}}, P_{2} \in \meaningof{E_2}\} }
\end{mathpar}

\begin{mathpar}
 \inferrule* [lab=behavior] {} {\meaningof{\langle a?b \rangle E} = \{ P \in \pi | P \equiv Q | u?(y)P', \\ \and \\\\ \and \\ \;\;\; u \in \meaningof{a}, \forall z.P'\{z/y\} \in \meaningof{E\{z/b\}}\}, \and \\ \meaningof{a!E} = \{ P \in \pi | P \equiv Q | x!\langle P' \rangle, x \in \meaningof{a} P' \in \meaningof{E}\} }
\end{mathpar}

\begin{mathpar}
 \inferrule* [lab=nominal] {} {\meaningof{\quotep{E}} = \{ \quotep{P} \in \quotep{\pi} | P \in \meaningof{E} \}, \and \meaningof{\quotep{P}} = \{ \quotep{Q} \in \quotep{\pi} | P \equiv Q \} \and \\ \meaningof{@\quotep{E}} = \{ P \in \pi | P \equiv @x, x \in \meaningof{E} \}}
\end{mathpar}

\begin{eqnarray*}
  \\
  \meaningof{-} : TS \to ST
\end{eqnarray*}

\begin{eqnarray*}
  \\
  L : TS \to ST
\end{eqnarray*}

\begin{eqnarray*}
  \\
  P \models E \iff P \in \meaningof{E}
\end{eqnarray*}

\begin{eqnarray*}
  P \approx_{L} Q \iff \forall E \in L. P \models E \iff Q \models E
\end{eqnarray*}

\begin{eqnarray*}
  P \approx_{K} Q
\end{eqnarray*}

\begin{eqnarray*}
  P \approx Q
\end{eqnarray*}

$\approx_{K} = \approx = \approx_{L}$

\subsubsection{Contextual duality}

Note that contexts extend the quotation operation to a family of
operations from processes to names. Given a context, $M$, we can
define a \emph{nominal context}, $\quotep{M}$ by $\quotep{M}[P] :=
\quotep{M[P]}$. To foreshadow what is to come we observe that these
operations enjoy a duality with processes very much like the duality
between vectors and maps from vectors to scalars.

Further, because the calculus is essentially higher-order, we have a
correspondence between contexts and processes. More specifically,
given a name $x$ and a context $M$ we can construct $M^{*}_{x}$ such
that 

\begin{mathpar}
  M^{*}_{x} | \lift{x}{P} \red M[P]
\end{mathpar}

namely,

\begin{mathpar}
  M^{*}_{x} := x?(u).M[\dropn{u}]
\end{mathpar}

The dependence of $M^{*}_{x}$ on a name makes it an abstraction, 

\begin{mathpar}
  M^{*} := (x)x?(u).M[\dropn{u}]
\end{mathpar}

\subsection{Additional notation}

It will sometimes be convenient to denote the process a name
quotes. We already have the notation $x = \quotep{P}$, but it will be
convenient to introduce an alternate notation, $\procn{x}$, when we
want to emphasize the connection to the use of the name. Note that, by
virtue of name equivalence, $\quotep{\procn{x}} \nameeq x$; so, the
notation is consistent with previous definitions.

Further, because names have structure it is possible to effect
substitutions on the basis of that structure. This means we need to
upgrade our notation for substitutions, which we accomplish by
adapting comprehension notation. Thus,

\begin{mathpar}
  P\{ y / x : x \in S \}
\end{mathpar}

is interpreted to mean the process derived from P by replacing (in a
capture-avoiding manner) each occurrence of $x$ in $S$ by $y$. For example,

\begin{mathpar}
  P\{ \quotep{\procn{x}|\procn{x}} / x : x \in \freenames{P} \}
\end{mathpar}

will replace each (occurrence) of a free name $x$ in $P$ by
$\quotep{\procn{x}|\procn{x}}$.

Also, we will avail ourselves of the notation $x^{L}$ and $x^{R}$ to
denote injections of a name into disjoint copies of the name
space. There are numerous ways to accomplish this. One example can be
found in \cite{MeredithR05}. This notation overloads to vectors of
names: $\vec{x}^{\pi} := (x_{i}^{\pi} \; : \; 0 \leq i < |\vec{x}| )$ where $\pi \in \{L,R\}$.

We also use $P^{\Box} := P|\Box$.

In \cite{MeredithR05} an interpretation of the new operator is
given. It turns out that there are several possible interpretations
all enjoying the requisite algebraic properties of the operator (see
\cite{milner91polyadicpi}). We will therefore make liberal use of
$(\nu\; \vec{x})P$.

% subsection the_syntax_and_semantics_of_the_notation_system (end)   

\input{qm2pi.qmops} 

\input{qm2pi.sterngerlach} 

\input{qm2pi.metric} 

% section concurrent_process_calculi (end)

%\input{qm2pi.proofsketch}

% section proof sketch (end)

%\input{qm2pi.slviaknots} 

% section spatial logic via knots (end)

\input{qm2pi.conclusion}

% section conclusion (end)

%\input{qm2pi.dtcodes} 

% section wiring algorithm (end)

\input{qm2pi.ack} 

% section acknowledgments (end)

\newpage


\bibliographystyle{plain}   
\bibliography{../../biblios/main.bib}

\input{qm2pi.rhodetails}

\end{document}



\end{document}



% section proof sketch (end)

%\section{Unlikely characters: spatial logic for
  knots}\label{sub:characteristic_formulae} % (fold)

Associated to the mobile process calculi are a family of logics known
as the Hennessy-Milner logics. These logics typically enjoy a
semantics interpreting formulae as sets of processes that when
factored through the encoding outlined above allows an identification
of classes of knots with logical formulae. In the context of this
encoding the sub-family known as the spatial logics \cite{CairesC03}
\cite{CairesC04} \cite{Caires04} are of particular interest providing
several important features for expressing and reasoning about
properties (i.e. classes) of knots. We hint here at how this may be done.

%\begin{description}
%\item [structural connectives] 
\subsubsection{Structural connectives} The spatial logics enjoy
structural connectives corresponding, at the logical level, to the
parallel composition ($P | Q$) and new name ($(\nu \; x)P$)
connectives for processes. As illustrated in the examples below, these
connectives are extremely expressive given the shape of our encoding.
%\item [decideable satisfaction]

\subsubsection{Decideable satisfaction}
In \cite{Caires04} the satisfaction relation is shown to be decideable
for a rich class of processes. It further turns out that the image of
the our encoding is a proper subset of that class. This result
provides the basis for an algorithm by which to search for knots
enjoying a given property.
%\item [characteristic formulae]

\subsubsection{Characteristic formulae}
In the same paper \cite{Caires04} , Caires presents a means of calculating
characteristic formulae, selecting equivalence classes of processes
up to a pre--specified depth limit on the support set of names. Composed with our
encoding, this characteristic formula can be used to select
characteristic formulae for knots.
%\end{description}

\subsubsection{Spatial logic formulae}

The grammar below (segmented for comprehension) summarizes the syntax
of spatial logic formulae. We employ illustrative examples in the
sequel to provide an intuitive understanding of their meaning
referring the reader to \cite{Caires04} for a more detailed explication
of the semantics.

\begin{mathpar}
  \inferrule* [lab=boolean] {} {{A,B} \bc T \;|\; \neg A \;|\; A \wedge B \;|\; \eta = \eta'}
  \and
  \inferrule* [lab=spatial] {} {|\; \pzero \;|\; A | B \;|\; x \text{\textregistered} A \;|\; \forall x . A \;|\;  H x . A}
  \and
  \inferrule* [lab=behavioral] {} {|\; \alpha . A}
  \and 
  \inferrule* [lab=recursion] {} {|\; X(\vec{u}) \;|\; \mu X(\vec{u}) . A}
  \and
  \inferrule* [lab=action] {} {\alpha \bc \langle x?(\vec{y}) \rangle \;|\; \langle x!(\vec{y}) \rangle \;|\; \langle \tau \rangle}
  \and 
  \inferrule* [lab=name] {} {\eta \bc x \;|\; \tau}
\end{mathpar} 

% subsection characteristic_formulae (end)   	 

\subsection{Example formulae}\label{sub:example_formulae_} % (fold)

\subsubsection{Crossing as formula.}
% 
% \begin{align*}
%   \frac{d}{dx} \sin x &= \cos x 
%   & \frac{d}{dx} e^x &= e^x \\
%   \frac{d}{dx} \cos x &= - \sin x 
%   & \frac{d}{dx} \log x &= \frac{1}{x} \\
% \end{align*} 

\begin{align*}
 \mu C(x_{0},x_{1},y_{0},y_{1},u).&(\langle x_{0}?(z) \rangle(\langle u! \rangle\langle y_{1}!z \rangle C(x_{0},x_{1},y_{0},y_{1},u)) & \\
  & \wedge \langle y_{1}?(z) \rangle (\langle u! \rangle \langle x_{0}!z \rangle C(x_{0},x_{1},y_{0},y_{1},u)) & \\
  & \wedge \langle x_{1}?(z) \rangle (\langle u? \rangle \langle y_{0}!z \rangle C(x_{0},x_{1},y_{0},y_{1},u)) & \\
  & \wedge \langle y_{0}?(z) \rangle (\langle u? \rangle \langle x_{1}!z \rangle C(x_{0},x_{1},y_{0},y_{1},u))) &
\end{align*}

The lexicographical similarity between the shape of this formulae and
the shape of definition of the process representing a crossing reveals
the intuitive meaning of this formulae. It describes the capabilities
of a process that has the right to represent a crossing. For example
it picks out processes that may perform an input on the port $x_0$ in
its initial menu of capabilities. What differentiates the formula
from the process, however, is that the crossing process is the
smallest candidate to satisfy the formula. Infinitely many other
processes -- with internal behavior hidden behind this interface, so
to speak -- also satisfy this formula. Even this simple formula,
then, can be seen to open a new view onto knots, providing a
computational interpretation of \emph{virtual} knots.

Note that this formula is derived by hand. A similar formula can be
derived by employing Caires' calculation of characteristic formula
\cite{Caires04} to the process representing a crossing. In light of
this discussion, we let
$\meaningof{C}_{\phi}(x0,x1,y0,y1,u)$ denote a formula specifying the
dynamics we wish to capture of a crossing. To guarantee we preserve
the shape of the interface and minimal semantics we demand that
$\meaningof{C}_{\phi}(x0,x1,y0,y1,u) \Rightarrow
\textbf{C}(x0,x1,y0,y1,u)$ where $\textbf{C}(x0,x1,y0,y1,u)$ denotes
the formula above.
                            
\subsubsection{Crossing number constraints.}
The moral content of the context lemma (Lemma \ref{context}) is that the notion of
``locality'' in the Reidemeister moves is effectively captured by the
parallel composition operator of the process calculus. This intuition
extends through the logic. Given a formula,
$\meaningof{C}_{\phi}(x0,x1,y0,y1,u)$, we can use the structural
connectives to specify constraints on crossing numbers, such as at
least $n$ crossings, or exactly $n$ crossings.
\begin{mathpar}
  \inferrule* [lab=at-least-n] {} { K^{\geq n}_{\phi}(\vec{xs},\vec{ys}) := \Pi_{i=0}^{n-1} Hu . \meaningof{C}_{\phi}(xs_i,ys_i,u) | T }
  \and 
  \inferrule* [lab=exactly-n] {} { K^{= n}_{\phi}(\vec{xs},\vec{ys}) := \Pi_{i=0}^{n-1} Hu . \meaningof{C}_{\phi}(xs_i,ys_i,u) | \neg (\forall x_0,y_0,x_1,y_1,u . \meaningof{C}_{\phi}(x_0,y_0,x_1,y_1,u) | T) }
\end{mathpar}

To round out this section, recall that the encoding of an $n$-crossing
knot decomposes into a parallel composition of $n$ \emph{copies} of a
crossing process together with a wiring harness. To specify different
knot classes with the same crossing number amounts to specifying
logical constraints on the wiring harness. In the interest of space,
we defer examples to a forthcoming paper. Suffice it to say that both
the conditions ``alternating knot'' and ``contains the tangle
corresponding to 5/3'' are expressible. For example, it is possible to
calculate the characteristic formula of a process corresponding to the
tangle 5/3 and conjoin it into the classifying formula via the
composition connective of the logic.

Finally, we wish to observe that it is entirely within reason to
contemplate a more domain-specific version of spatial logic tailored
to the shape of processes in the image of the encoding. Such a
domain-specific logic would have a better claim to the title formal
language of knot properties.

% subsection example_formulae_ (end)

% section knots_as_processes (end) 

% section spatial logic via knots (end)

\section{Conclusions and future work}

\paragraph{Testing physical space}
You, gentle reader, may wonder why of all the theorems to be proved
given this set up we pick the one above. In some sense it's hardly
central to quantum mechanics. We see it as central in the sense that
it firmly establishes a notion of physical space arising from a notion
of the equivalence of behavior. Relating bisimulation to a metric is a
big step forward, but one is faced with interpreting the relationship
of that metric space to something more physical. Quantum mechanical
notions of ``physical'' space are still far from intuitive, but by
relating this idea of distance as testing to calculations that predict
physical circumstances we are making a not insignificant step forward
toward an understanding of the physical space we inhabit as
essentially dynamic.

\paragraph{Effectivity and simulation}
One of the observations we have yet to make is that the entire program
spelled out here is effective. We have built various interpreters for
the reflective calculus at work in this interpretation. In principle,
then, we can simulate quantum mechanics on a computer. The place where
the simulation may lose fidelity is the infinitely branching summation
for the annihilator.

In this connection i also want to point out that the evaluation style
calculation of the inner product puts the non-determinism of the
summation right at the heart of measurement. This suggests that
Milner's original reduction-based formulation of the dynamics of his
calculi in terms of sums was not just notationally suggestive of a
notion of measure-and-continue but captured some significant part of
the physics.

\paragraph{Quantum continuations}
In light of this last observation i want to point out that the
predominant account of quantum mechanics is missing a key aspect of a
truly compositional story of the physical situation. In a real lab,
when a measurement is made the observation can be made to feed into
another device that then makes another measurement conditioned on the
results of the first. This means that after the superposition was
collapsed the entire experimental set up remained in
superposition. While QM offers a means of writing this down it doesn't
quite line up well with the well-trodden formulation of computation
and continuation that we see so succinctly expressed in Milner's
calculi. This suggests that there might be advantages to this account
of dynamics waiting to be explored.

\paragraph{Quantum logic}
In this connection, we also note that by virtue of having the
Hennessy-Milner construction, we can pull the construction through the
interpretation of QM. This gives us a natural candidate for a quantum
logic that enjoys an extremely tight connection with it's domain of
interpretation, making the construction much less ad hoc (rather it is
the image of functor!).

\paragraph{Quantum probabiity}
i have questions about the basis of the interpretation of inner
product as probability amplitude. In particular, using which
axiomatization of probability theory does the notion of probability
amplitude earn the right to be so dubbed? In other words, where is the
proof that the operation for calculating a probability amplitude (and
then squaring) satisfies the axioms of what it means to calculate a
probability? Even if such a proof exists (i have yet to find it in the
literature), i wonder if it might not be possible to turn things on
their heads. Can we view the calculation of the probability amplitude
as an axiomatization of probability? If so, then the definition we
give for calculating probability amplitude may provide the basis for
an \emph{effective} theory of probability.

\paragraph{Quantum vs ``biological'' information}
Finally, i want to conclude with a more philosophical observation. At
a recent workshop in which QM was a predominant topic i noticed
something about quantum information. The speaker was giving a riveting
discussion of axiomatic QM and showing how properties of ``no
cloning'' and ``no deleting'' emerged as consequences of the
axiomatization. Theorems of this form are necessary to give us a sense
of confidence that our axioms characterize the physical theory. What
struck me, though, was that if quantum information is neither erasable
nor replicable it is markedly different from \emph{life}. Two of the
things we know about life is that

\begin{itemize}
  \item it ends;
  \item to gain some measure of persistence, to transcend it's
    finitude it is imminently copyable.
\end{itemize}

Both of these qualities are summarized succinctly in the aphorism: all
flesh is grass. For me these two kinds of ``information'' -- call them
quantum and biological -- are end points on a spectrum of strategies
for persistence. At one end, we have those curious entities that enjoy
uniqueness and permanence; at the other, we have those who in the face
of a certain end and an uncertain present make a go of passing
something on. To me one of the more remarkable aspects of the latter
strategy is that in the presence of noise (and certain features of
copying) we get a kind of dynamism, a chance for improvement against a
given persistent condition.

% subsection other_calculi_other_bisimulations_and_geometry_as_behavior (end)




% section conclusion (end)

%\documentclass[12pt]{llncs}
%\documentclass{jktr}

\usepackage[pdftex]{hyperref}                   
\usepackage {listings}
\usepackage {mathpartir}
\usepackage{bcprules}
%\usepackage{listings}
                       
\usepackage{graphicx} 
%\usepackage[margins=2.5cm,nohead,nofoot]{geometry}
%\usepackage{geometry}
\usepackage{amsfonts}
\usepackage{amstext}
\usepackage{latexsym}
\usepackage{amssymb}
\usepackage{color}


%\include{myPreamble}
\documentclass[12pt]{llncs}
%\documentclass{jktr}

\usepackage[pdftex]{hyperref}                   
\usepackage {listings}
\usepackage {mathpartir}
\usepackage{bcprules}
%\usepackage{listings}
                       
\usepackage{graphicx} 
%\usepackage[margins=2.5cm,nohead,nofoot]{geometry}
%\usepackage{geometry}
\usepackage{amsfonts}
\usepackage{amstext}
\usepackage{latexsym}
\usepackage{amssymb}
\usepackage{color}


%\include{myPreamble}
\include{qm2pi.local} 

%\ifpdf
%\usepackage[pdftex]{graphicx}
%\else
%\usepackage{graphicx}
%\fi

 % \ifpdf
%  \usepackage{pdfsync}
%  \if


%\title{Brief Article}
%\author{David F. Snyder}
%\author{L.G. Meredith}

%\address{Dept. of Math., Texas State University--San Marcos, San Marcos, TX 78666}
       
\pagestyle{empty}


\begin{document}

\lstset{language=[Objective]Caml,frame=shadowbox}

\input{qm2pi.front}

% section front matter (end)

\input{qm2pi.intro} 
 
% section introduction (end)

% \input{qm2pi.knotations} 

% section notation (end)

\input{qm2pi.process.calculi} 

% section concurrent_process_calculi_and_spatial_logics_ (end)
    
%\input{qm2pi.knots2pi} 

%\input{qm2pi.trefoil} 

%\input{qm2pi.mainthm} 

% subsection basic_interpretation (end)

%\input{qm2pi.rho.presentation} 
\subsection{The syntax and semantics of the notation system}\label{sub:the_syntax_and_semantics_of_the_notation_system} % (fold)

We now summarize a technical presentation of the calculus that
embodies our theory of dynamics. The typical presentation of such a
calculus follows the style of giving generators and relations on
them. The grammar, below, describing term constructors, freely
generates the set of processes, $\Proc$. This set is then quotiented
by a relation known as structural congruence and it is over this set
that the notion of dynamics is expressed. This presentation is
essentially that of \cite{MeredithR05} with the addition of
polyadicity and summation. For readability we have relegated some of
the technical subtleties to an appendix.

\subsubsection{Process grammar}\label{subsub:process_grammar}

\begin{mathpar}
  \inferrule* [lab=synchronization] {} {{M} \bc \pzero \;|\; x?F \;|\; x!C }
  \and
  \inferrule* [lab=abstraction] {} {{F} \bc (x)P}
  \and
  \inferrule* [lab=concretion] {} {{C} \bc \langle Q \rangle}
  \and
  \inferrule* [lab=process] {} {{P,Q} \bc M \;| \;P|Q \;|\; @{x}}
  \and
  \inferrule* [lab=name] {} {{x} \bc \quotep{P}}
\end{mathpar} 

Note that $\vec{x}$ (resp. $\vec{P}$) denotes a vector of names
(resp. processes) of length $|\vec{x}|$ (resp. $|\vec{P}|$). We adopt
the following useful abbreviations.

\begin{mathpar}
   x?(\vec{y}).P := x.(\vec{y})P \and  x\clift{\vec{P}} := x.\clift{\vec{P}}
   \and x!(y) := \lift{x}{\dropn{y}}
   \and \Pi_{i=0}^{n-1}P_i := P_0 | \ldots | P_{n-1}
\end{mathpar}

\subsubsection{Structural congruence}

\paragraph{Free and bound names and alpha-equivalence.} At the
core of structural equivalence is alpha-equivalence which identifies
process that are the same up to a change of variable. Formally, we
recognize the distinction between free and bound names. The free names
of a process, $\freenames{P}$, may be calculated recursively as
follows:

\begin{mathpar}
\freenames{\pzero} := \emptyset
  \and \\
  \freenames{x?(y).P} := \{ x \} \cup (\freenames{P} \setminus \{ y \})
  \and 
  \freenames{x!\langle P \rangle} := \{ x \} \cup \{ P \} 
  \and \\
  \freenames{P|Q} := \freenames{P} \cup \freenames{Q}
  \and \\
  \freenames{@{x}} := \{ x \}
\end{mathpar}

$\pi$
$\quotep{\pi}$

$\freenames{-} : \pi \to \mathcal{P}(\quotep{\pi})$

\begin{eqnarray*}
  \freenames{\pzero} & := & \emptyset \\
  \freenames{x?(y).P} & := & \{ x \} \cup (\freenames{P} \setminus \{ y \}) \\
  \freenames{x!\langle P \rangle} & := & \{ x \} \cup \{ P \} \\
  \freenames{P|Q} & := & \freenames{P} \cup \freenames{Q} \\
  \freenames{\dropn{x}} & := & \{ x \}
\end{eqnarray*}

The bound names of a process, $\boundnames{P}$, are those names occurring in $P$
that are not free. For example, in $x?(y).0$, the name $x$ is free, while $y$ is bound.

\begin{mathpar}
  \inferrule* [lab=monoidal-laws] {} { P|Q \equiv Q|P \and P|0 \equiv P \and P|(Q|R) \equiv (P|Q)|R }
\end{mathpar}

\begin{mathpar}
  \inferrule* [lab=alpha-equivalence] {} { (x)P \equiv (y)P\{y/x\} \and y \not\in \freenames{P} }
\end{mathpar}

\begin{definition}
Then two processes, $P,Q$, are alpha-equivalent if $P = Q\{\vec{y}/\vec{x}\}$ for
some $\vec{x} \in \boundnames{Q},\vec{y} \in \boundnames{P}$, where $Q\{\vec{y}/\vec{x}\}$
denotes the capture-avoiding substitution of $\vec{y}$ for $\vec{x}$ in $Q$.
\end{definition}

\begin{definition}
  The {\em structural congruence} \cite{SangiorgiWalker} , $\equiv$,
  between processes is the least congruence containing
  alpha-equivalence, satisfying the abelian monoid laws
  (associativity, commutativity and $\pzero$ as identity) for parallel
  composition $|$ and for summation $+$.
\end{definition}

\subsection{Name equivalence}

We take name equivalence, written $\nameeq$, to be the smallest
equivalence relation generated by the following rules.

\begin{mathpar}
\inferrule*[lab=Quote-drop]
{ }
{ \quotep{@{x}} \nameeq x }

\inferrule*[lab=Struct-equiv]
{ P \scong Q }
{ \quotep{P} \nameeq \quotep{Q} }
\end{mathpar}

The astute reader will have noticed that the mutual recursion of names
and processes imposes a mutual recursion on alpha-equivalence and
structural equivalence via name-equivalence. Fortunately, all of this
works out pleasantly and we may calculate in the natural way, free of
concern. The reader interested in the details is referred to the
appendix \ref{appendix:rho_details}.

\subsection{Substitution}

We use $\Proc$ for the set of processes, $\QProc$ for the set of
names, and $\id{\{}\vec{y} / \vec{x} \id{\}}$ to denote partial maps,
$s : \QProc \rightarrow \QProc$. A map, $s$ lifts, uniquely, to a map
on process terms, $\widehat{s} : \Proc \rightarrow \Proc$ by the
following equations.

\begin{mathpar}
  (0) \psubstp{Q}{P} := 0 \\
  (R \juxtap S) \psubstp{Q}{P}
  :=    
  (R)\psubstp{Q}{P} \juxtap (S) \psubstp{Q}{P} \\
  (x?(y).R) \psubstp{Q}{P}    
  :=    
  (x)\substp{Q}{P} (z)\concat( (R \psubstn{z}{y}) \psubstp{Q}{P} ) \\
  (\lift{x}{R}) \psubstp{Q}{P}  
  :=
  \lift{(x)\substp{Q}{P}}{ R \psubstp{Q}{P} } \\
%   (\dropn{x})  \psubstp{Q}{P}       
%   := 
%   \left\{ 
%     \begin{array}{ccc} 
%       \dropn{\quotep{Q}} & & x \nameeq \quotep{P} \\
%       \dropn{x} & & otherwise \\
%     \end{array}
%   \right. 
  (\dropn{x})  \psubstp{Q}{P}       
  := 
  \left\{ 
    \begin{array}{ccc} 
      Q & & x \nameeq \quotep{P} \\
      \dropn{x} & & otherwise \\
    \end{array}
  \right.
\end{mathpar}
 

where

\begin{eqnarray}
  (x)\id{\{} \lpquote Q \rpquote / \lpquote P \rpquote \id{\}}            = 
  \left\{ 
    \begin{array}{ccc}
      \lpquote Q \rpquote & & x \nameeq \lpquote P \rpquote \\
      x & & otherwise \\
    \end{array}
  \right. \nonumber
\end{eqnarray}

and $z$ is chosen distinct from $\quotep{P}$, $\quotep{Q}$, the free
names in $Q$, and all the names in $R$. Our $\alpha$-equivalence will
be built in the standard way from this substitution.

\begin{remark}\label{rem:no_self_referential_names}
  One consequence of these definitions is that $\forall P. \quotep{P}
  \not\in \freenames{P}$.
\end{remark}

\subsection{ Dynamic quote: an example }

Anticipating something of what's to come, consider applying the
substitution, $\widehat{\id{\{}u / z \id{\}}}$, to the following pair
of processes, $\lift{w}{y!(z)}$ and $w[ \lpquote y!(z) \rpquote ]$.

\begin{eqnarray}
	\lift{w}{y!(z)}\widehat{\id{\{}u / z \id{\}}}
		& = &
		\lift{w}{y!(u)} \nonumber\\
	w[ \lpquote y!(z) \rpquote ] \widehat{ \id{\{}u / z \id{\}} }
		& = &
		w[ \lpquote y!(z) \rpquote ] \nonumber
\end{eqnarray}

Because the body of the process between quotes is impervious to
substitution, we get radically different answers. In fact, by
examining the first process in an input context,
e.g. $x?(z).\lift{w}{y!(z)}$, we see that the process under the lift
operator may be shaped by prefixed inputs binding a name inside it. In
this sense, the lift operator will be seen as a way to dynamically
construct processes before reifying them as names.

Finally equipped with these standard features we can present the
dynamics of the calculus.

\subsubsection{Operational semantics} 

Finally, we introduce the computational dynamics. What marks these
algebras as distinct from other more traditionally studied algebraic
structures, e.g. vector spaces or polynomial rings, is the manner in
which dynamics is captured. In traditional structures, dynamics is typically
expressed through morphisms between such structures, as in linear maps
between vector spaces or morphisms between rings. In algebras
associated with the semantics of computation, the dynamics is
expressed as part of the algebraic structure itself, through a
reduction reduction relation typically denoted by $\red$. Below, we
give a recursive presentation of this relation for the calculus used
in the encoding.

$\red \subseteq \pi \times \pi$
$\red : \pi \to \mathcal{P}(\pi)$

\begin{mathpar}
  \inferrule* [lab=Comm] { \textsf{match}( x_{src}, x_{trgt} ) } { x_{trgt}?(y)P \; | \; x_{src}!\langle {Q} \rangle \red P\{\quotep{Q}/y}\} }
  \and \\
  \inferrule* [lab=Par] {{P} \red {P}'} {{{P} | {Q}} \red {{P}' | {Q}}}
  \and
  \inferrule* [lab=Equiv]{{{P} \scong {P}'} \andalso {{P}' \red {Q}'} \andalso {{Q}' \scong {Q}}}{{P} \red {Q}}
\end{mathpar}

\begin{eqnarray*}
  match_{\equiv} (\quotep{P},\quotep{Q}) & := & P \equiv Q \\
  match_{\dagger}(\quotep{P},\quotep{Q}) & := & \forall R. P|Q \red^{*} R => R \red^{*} 0 \\
  match_{K}(\quotep{P},\quotep{Q}) & := & K \mbox{ for some context } K
\end{eqnarray*}

$u?(x)P | u!\langle Q \rangle \red P\{\quotep{Q}/x\}$

%We write $\wred$ for $\red^*$, and $P\red$ if $\exists Q $ such that $ P \red Q$.
We write $P\red$ if $\exists Q $ such that $ P \red Q$ and $P\not\red$, otherwise.

\section{Replication}

As mentioned before, it is known that replication (and hence
recursion) can be implemented in a higher-order process algebra
\cite{SangiorgiWalker}. As our first example of calculation with the
machinery thus far presented we give the construction explicitly in
the {\rhoc}.

\begin{eqnarray}
	D_{x} & := & \prefix{x}{y}{(\binpar{\outputp{x}{y}}{@{y}})} \nonumber\\
	\bangp_{x}{P} & := & \binpar{{x}!\langle{\binpar{D_{x}}{P}}\rangle}{D_{x}} \nonumber
\end{eqnarray}

\begin{eqnarray}
	\bangp_{x}{P} & & \nonumber\\
	=
	& {x}!\langle{(\prefix{x}{y}{(\outputp{x}{y} | @{y})) | P}}\rangle 
	      | \prefix{x}{y}{(\outputp{x}{y} | @{y})} & \nonumber\\
	\red
	& (\outputp{x}{y} | @{y})\substn{\quotep{(\prefix{x}{y}{(@{y} | \outputp{x}{y})) | P}}}{y} & \nonumber\\
	=
	& \outputp{x}{\quotep{(\prefix{x}{y}{(\outputp{x}{y} | @{y})) | P}}}
	  | {(\prefix{x}{y}{(\outputp{x}{y} | @{y})) | P}} & \nonumber\\
	\red
	& \ldots & \nonumber\\
	\red^*
	& P | P | \ldots & \nonumber
\end{eqnarray}

Of course, this encoding, as an implementation, runs away, unfolding
$\bangp{P}$ eagerly. A lazier and more implementable replication
operator, restricted to input-guarded processes, may be obtained as follows.

\begin{eqnarray}
\bangp{\prefix{u}{v}{P}} 
	:= 
	\binpar{\lift{x}{\prefix{u}{v}{(\binpar{D(x)}{P})}}}{D(x)} \nonumber
\end{eqnarray}

\begin{remark}
  Note that the lazier definition still does not deal with summation
  or mixed summation (i.e. sums over input and output). The reader is
  invited to construct definitions of replication that deal with these
  features. 

  Further, the definitions are parameterized in a name, $x$. Can you,
  gentle reader, make a definition that eliminates this parameter and
  guarantees no accidental interaction between the replication
  machinery and the process being replicated -- i.e. no accidental
  sharing of names used by the process to get its work done and the
  name(s) used by the replication to effect copying. This latter
  revision of the definition of replication is crucial to obtaining
  the expected identity $!!P \sim !P$.
\end{remark}

\begin{remark}\label{rem:paradoxical_combinator}
  The reader familiar with the lambda calculus will have noticed the
  similarity between $D$ and the paradoxical combinator.

  [Ed. note: the existence of this seems to suggest we have to be more
  restrictive on the set of processes and names we admit if we are to
  support no-cloning.]
\end{remark}

\subsubsection{Bisimulation}

The computational dynamics gives rise to another kind of equivalence,
the equivalence of computational behavior. As previously mentioned
this is typically captured \emph{via} some form of bisimulation.

% The notion we use in this paper is weak barbed bisimulation
% \cite{milner91polyadicpi}.

The notion we use in this paper is derived from weak barbed
bisimulation \cite{milner91polyadicpi}. 

\begin{definition}
An \emph{observation relation}, $\downarrow_{\mathcal N}$, over a set
of names, $\mathcal N$, is the smallest relation satisfying the rules
below.

\infrule[Out-barb]{y \in {\mathcal N}, \; x \nameeq y}
		  {\outputp{x}{v} \downarrow_{\mathcal N} x}
\infrule[Par-barb]{\mbox{$P\downarrow_{\mathcal N} x$ or $Q\downarrow_{\mathcal N} x$}}
		  {\binpar{P}{Q} \downarrow_{\mathcal N} x}

We write $P \Downarrow_{\mathcal N} x$ if there is $Q$ such that 
$P \wred Q$ and $Q \downarrow_{\mathcal N} x$.
\end{definition}

\begin{definition}
%\label{def.bbisim}
An  ${\mathcal N}$-\emph{barbed bisimulation} over a set of names, ${\mathcal N}$, is a symmetric binary relation 
${\mathcal S}_{\mathcal N}$ between agents such that $P\rel{S}_{\mathcal N}Q$ implies:
\begin{enumerate}
\item If $P \red P'$ then $Q \wred Q'$ and $P'\rel{S}_{\mathcal N} Q'$.
\item If $P\downarrow_{\mathcal N} x$, then $Q\Downarrow_{\mathcal N} x$.
\end{enumerate}
$P$ is ${\mathcal N}$-barbed bisimilar to $Q$, written
$P \wbbisim_{\mathcal N} Q$, if $P \rel{S}_{\mathcal N} Q$ for some ${\mathcal N}$-barbed bisimulation ${\mathcal S}_{\mathcal N}$.
\end{definition}

$\mathcal{R} \subseteq \pi \times \pi$

$P \mathcal{R} Q => \forall P'. P \red P' \Rightarrow \exists Q'. Q \red Q', P' \mathcal{R} Q'$

$P \vdash x \Rightarrow Q \vdash x$

\begin{mathpar}
  \inferrule*[lab=Out-barb]{x \nameeq y}{{y}!\langle{Q}\rangle \vdash x}
  \and
  \inferrule*[lab=Par-barb]{\mbox{$P\vdash x$ or $Q\vdash x$}}{\binpar{P}{Q} \vdash x}
\end{mathpar}

\subsubsection{Contexts}

One of the principle advantages of computational calculi like the
$\pi$-calculus is a well-defined notion of context,
contextual-equivalence and a correlation between
contextual-equivalence and notions of bisimulation. The notion of
context allows the decomposition of a process into (sub-)process and
its syntactic environment, its context. Thus, a context may be
thought of as a process with a ``hole'' (written $\Box$) in it. The
application of a context $M$ to a process $P$, written $M[P]$, is
tantamount to filling the hole in $M$ with $P$. In this paper we do
not need the full weight of this theory, but do make use of the notion
of context in the proof the main theorem. 

\begin{mathpar}
  \inferrule* [lab=summation] {} {{M_{M},M_{N}} \bc \Box \;|\; x.M_{A} \;|\; M_{M}+M_{N}}
  \and
  \inferrule* [lab=agent] {} {{M_{A}} \bc (\vec{x})M_{P} \;| \; \clift{P_0,\ldots,M_{P},\ldots,P_N}}
  \and \\
  \inferrule* [lab=process] {} {{M_{P}} \bc M_{N} \;| \;P|M_{P} }
\end{mathpar} 

\begin{mathpar}
  \inferrule* [lab=sychronization] {} {M_{N} \bc \Box \;|\; x?M_{F} \;|\; x!M_{C}}
  \and
  \inferrule* [lab=abstraction] {} {{M_{F}} \bc (x)M_{P} }
  \and
  \inferrule* [lab=concretion] {} {{M_{C}} \bc \langle M_{P} \rangle }
  \and \\
  \inferrule* [lab=process] {} {{M_{P}} \bc M_{N} \;| \;P|M_{P} }
\end{mathpar}

\begin{definition}[contextual application] Given a context $M$, and
  process $P$, we define the \emph{contextual application}, $M[P] :=
  M\{P/\Box\}$. That is, the contextual application of M to P is the
  substitution of $P$ for $\Box$ in $M$.
\end{definition}

$\meaningof{-} : L \to \mathcal{P}(\pi)$

\begin{mathpar}
  \inferrule* [lab=collection] {} {\meaningof{true} = \pi, \and \meaningof{~E} = \pi \setminus \meaningof{E}, \and \meaningof{E_{1} \& E_{2}} = \meaningof{E_{1}} \cap \meaningof{E_{2}}}
\end{mathpar}

\begin{mathpar}
  \inferrule* [lab=structure] {} {\meaningof{0} = \{ P \in \pi | P \equiv 0 \}, \and \\ \meaningof{E_1 | E_2} = \{ P \in \pi | P \equiv P_{1} | P_{2}, P_{1} \in \meaningof{E_{1}}, P_{2} \in \meaningof{E_2}\} }
\end{mathpar}

\begin{mathpar}
 \inferrule* [lab=behavior] {} {\meaningof{\langle a?b \rangle E} = \{ P \in \pi | P \equiv Q | u?(y)P', \\ \and \\\\ \and \\ \;\;\; u \in \meaningof{a}, \forall z.P'\{z/y\} \in \meaningof{E\{z/b\}}\}, \and \\ \meaningof{a!E} = \{ P \in \pi | P \equiv Q | x!\langle P' \rangle, x \in \meaningof{a} P' \in \meaningof{E}\} }
\end{mathpar}

\begin{mathpar}
 \inferrule* [lab=nominal] {} {\meaningof{\quotep{E}} = \{ \quotep{P} \in \quotep{\pi} | P \in \meaningof{E} \}, \and \meaningof{\quotep{P}} = \{ \quotep{Q} \in \quotep{\pi} | P \equiv Q \} \and \\ \meaningof{@\quotep{E}} = \{ P \in \pi | P \equiv @x, x \in \meaningof{E} \}}
\end{mathpar}

\begin{eqnarray*}
  \\
  \meaningof{-} : TS \to ST
\end{eqnarray*}

\begin{eqnarray*}
  \\
  L : TS \to ST
\end{eqnarray*}

\begin{eqnarray*}
  \\
  P \models E \iff P \in \meaningof{E}
\end{eqnarray*}

\begin{eqnarray*}
  P \approx_{L} Q \iff \forall E \in L. P \models E \iff Q \models E
\end{eqnarray*}

\begin{eqnarray*}
  P \approx_{K} Q
\end{eqnarray*}

\begin{eqnarray*}
  P \approx Q
\end{eqnarray*}

$\approx_{K} = \approx = \approx_{L}$

\subsubsection{Contextual duality}

Note that contexts extend the quotation operation to a family of
operations from processes to names. Given a context, $M$, we can
define a \emph{nominal context}, $\quotep{M}$ by $\quotep{M}[P] :=
\quotep{M[P]}$. To foreshadow what is to come we observe that these
operations enjoy a duality with processes very much like the duality
between vectors and maps from vectors to scalars.

Further, because the calculus is essentially higher-order, we have a
correspondence between contexts and processes. More specifically,
given a name $x$ and a context $M$ we can construct $M^{*}_{x}$ such
that 

\begin{mathpar}
  M^{*}_{x} | \lift{x}{P} \red M[P]
\end{mathpar}

namely,

\begin{mathpar}
  M^{*}_{x} := x?(u).M[\dropn{u}]
\end{mathpar}

The dependence of $M^{*}_{x}$ on a name makes it an abstraction, 

\begin{mathpar}
  M^{*} := (x)x?(u).M[\dropn{u}]
\end{mathpar}

\subsection{Additional notation}

It will sometimes be convenient to denote the process a name
quotes. We already have the notation $x = \quotep{P}$, but it will be
convenient to introduce an alternate notation, $\procn{x}$, when we
want to emphasize the connection to the use of the name. Note that, by
virtue of name equivalence, $\quotep{\procn{x}} \nameeq x$; so, the
notation is consistent with previous definitions.

Further, because names have structure it is possible to effect
substitutions on the basis of that structure. This means we need to
upgrade our notation for substitutions, which we accomplish by
adapting comprehension notation. Thus,

\begin{mathpar}
  P\{ y / x : x \in S \}
\end{mathpar}

is interpreted to mean the process derived from P by replacing (in a
capture-avoiding manner) each occurrence of $x$ in $S$ by $y$. For example,

\begin{mathpar}
  P\{ \quotep{\procn{x}|\procn{x}} / x : x \in \freenames{P} \}
\end{mathpar}

will replace each (occurrence) of a free name $x$ in $P$ by
$\quotep{\procn{x}|\procn{x}}$.

Also, we will avail ourselves of the notation $x^{L}$ and $x^{R}$ to
denote injections of a name into disjoint copies of the name
space. There are numerous ways to accomplish this. One example can be
found in \cite{MeredithR05}. This notation overloads to vectors of
names: $\vec{x}^{\pi} := (x_{i}^{\pi} \; : \; 0 \leq i < |\vec{x}| )$ where $\pi \in \{L,R\}$.

We also use $P^{\Box} := P|\Box$.

In \cite{MeredithR05} an interpretation of the new operator is
given. It turns out that there are several possible interpretations
all enjoying the requisite algebraic properties of the operator (see
\cite{milner91polyadicpi}). We will therefore make liberal use of
$(\nu\; \vec{x})P$.

% subsection the_syntax_and_semantics_of_the_notation_system (end)   

\input{qm2pi.qmops} 

\input{qm2pi.sterngerlach} 

\input{qm2pi.metric} 

% section concurrent_process_calculi (end)

%\input{qm2pi.proofsketch}

% section proof sketch (end)

%\input{qm2pi.slviaknots} 

% section spatial logic via knots (end)

\input{qm2pi.conclusion}

% section conclusion (end)

%\input{qm2pi.dtcodes} 

% section wiring algorithm (end)

\input{qm2pi.ack} 

% section acknowledgments (end)

\newpage


\bibliographystyle{plain}   
\bibliography{../../biblios/main.bib}

\input{qm2pi.rhodetails}

\end{document}

 

%\ifpdf
%\usepackage[pdftex]{graphicx}
%\else
%\usepackage{graphicx}
%\fi

 % \ifpdf
%  \usepackage{pdfsync}
%  \if


%\title{Brief Article}
%\author{David F. Snyder}
%\author{L.G. Meredith}

%\address{Dept. of Math., Texas State University--San Marcos, San Marcos, TX 78666}
       
\pagestyle{empty}


\begin{document}

\lstset{language=[Objective]Caml,frame=shadowbox}

\documentclass[12pt]{llncs}
%\documentclass{jktr}

\usepackage[pdftex]{hyperref}                   
\usepackage {listings}
\usepackage {mathpartir}
\usepackage{bcprules}
%\usepackage{listings}
                       
\usepackage{graphicx} 
%\usepackage[margins=2.5cm,nohead,nofoot]{geometry}
%\usepackage{geometry}
\usepackage{amsfonts}
\usepackage{amstext}
\usepackage{latexsym}
\usepackage{amssymb}
\usepackage{color}


%\include{myPreamble}
\include{qm2pi.local} 

%\ifpdf
%\usepackage[pdftex]{graphicx}
%\else
%\usepackage{graphicx}
%\fi

 % \ifpdf
%  \usepackage{pdfsync}
%  \if


%\title{Brief Article}
%\author{David F. Snyder}
%\author{L.G. Meredith}

%\address{Dept. of Math., Texas State University--San Marcos, San Marcos, TX 78666}
       
\pagestyle{empty}


\begin{document}

\lstset{language=[Objective]Caml,frame=shadowbox}

\input{qm2pi.front}

% section front matter (end)

\input{qm2pi.intro} 
 
% section introduction (end)

% \input{qm2pi.knotations} 

% section notation (end)

\input{qm2pi.process.calculi} 

% section concurrent_process_calculi_and_spatial_logics_ (end)
    
%\input{qm2pi.knots2pi} 

%\input{qm2pi.trefoil} 

%\input{qm2pi.mainthm} 

% subsection basic_interpretation (end)

%\input{qm2pi.rho.presentation} 
\subsection{The syntax and semantics of the notation system}\label{sub:the_syntax_and_semantics_of_the_notation_system} % (fold)

We now summarize a technical presentation of the calculus that
embodies our theory of dynamics. The typical presentation of such a
calculus follows the style of giving generators and relations on
them. The grammar, below, describing term constructors, freely
generates the set of processes, $\Proc$. This set is then quotiented
by a relation known as structural congruence and it is over this set
that the notion of dynamics is expressed. This presentation is
essentially that of \cite{MeredithR05} with the addition of
polyadicity and summation. For readability we have relegated some of
the technical subtleties to an appendix.

\subsubsection{Process grammar}\label{subsub:process_grammar}

\begin{mathpar}
  \inferrule* [lab=synchronization] {} {{M} \bc \pzero \;|\; x?F \;|\; x!C }
  \and
  \inferrule* [lab=abstraction] {} {{F} \bc (x)P}
  \and
  \inferrule* [lab=concretion] {} {{C} \bc \langle Q \rangle}
  \and
  \inferrule* [lab=process] {} {{P,Q} \bc M \;| \;P|Q \;|\; @{x}}
  \and
  \inferrule* [lab=name] {} {{x} \bc \quotep{P}}
\end{mathpar} 

Note that $\vec{x}$ (resp. $\vec{P}$) denotes a vector of names
(resp. processes) of length $|\vec{x}|$ (resp. $|\vec{P}|$). We adopt
the following useful abbreviations.

\begin{mathpar}
   x?(\vec{y}).P := x.(\vec{y})P \and  x\clift{\vec{P}} := x.\clift{\vec{P}}
   \and x!(y) := \lift{x}{\dropn{y}}
   \and \Pi_{i=0}^{n-1}P_i := P_0 | \ldots | P_{n-1}
\end{mathpar}

\subsubsection{Structural congruence}

\paragraph{Free and bound names and alpha-equivalence.} At the
core of structural equivalence is alpha-equivalence which identifies
process that are the same up to a change of variable. Formally, we
recognize the distinction between free and bound names. The free names
of a process, $\freenames{P}$, may be calculated recursively as
follows:

\begin{mathpar}
\freenames{\pzero} := \emptyset
  \and \\
  \freenames{x?(y).P} := \{ x \} \cup (\freenames{P} \setminus \{ y \})
  \and 
  \freenames{x!\langle P \rangle} := \{ x \} \cup \{ P \} 
  \and \\
  \freenames{P|Q} := \freenames{P} \cup \freenames{Q}
  \and \\
  \freenames{@{x}} := \{ x \}
\end{mathpar}

$\pi$
$\quotep{\pi}$

$\freenames{-} : \pi \to \mathcal{P}(\quotep{\pi})$

\begin{eqnarray*}
  \freenames{\pzero} & := & \emptyset \\
  \freenames{x?(y).P} & := & \{ x \} \cup (\freenames{P} \setminus \{ y \}) \\
  \freenames{x!\langle P \rangle} & := & \{ x \} \cup \{ P \} \\
  \freenames{P|Q} & := & \freenames{P} \cup \freenames{Q} \\
  \freenames{\dropn{x}} & := & \{ x \}
\end{eqnarray*}

The bound names of a process, $\boundnames{P}$, are those names occurring in $P$
that are not free. For example, in $x?(y).0$, the name $x$ is free, while $y$ is bound.

\begin{mathpar}
  \inferrule* [lab=monoidal-laws] {} { P|Q \equiv Q|P \and P|0 \equiv P \and P|(Q|R) \equiv (P|Q)|R }
\end{mathpar}

\begin{mathpar}
  \inferrule* [lab=alpha-equivalence] {} { (x)P \equiv (y)P\{y/x\} \and y \not\in \freenames{P} }
\end{mathpar}

\begin{definition}
Then two processes, $P,Q$, are alpha-equivalent if $P = Q\{\vec{y}/\vec{x}\}$ for
some $\vec{x} \in \boundnames{Q},\vec{y} \in \boundnames{P}$, where $Q\{\vec{y}/\vec{x}\}$
denotes the capture-avoiding substitution of $\vec{y}$ for $\vec{x}$ in $Q$.
\end{definition}

\begin{definition}
  The {\em structural congruence} \cite{SangiorgiWalker} , $\equiv$,
  between processes is the least congruence containing
  alpha-equivalence, satisfying the abelian monoid laws
  (associativity, commutativity and $\pzero$ as identity) for parallel
  composition $|$ and for summation $+$.
\end{definition}

\subsection{Name equivalence}

We take name equivalence, written $\nameeq$, to be the smallest
equivalence relation generated by the following rules.

\begin{mathpar}
\inferrule*[lab=Quote-drop]
{ }
{ \quotep{@{x}} \nameeq x }

\inferrule*[lab=Struct-equiv]
{ P \scong Q }
{ \quotep{P} \nameeq \quotep{Q} }
\end{mathpar}

The astute reader will have noticed that the mutual recursion of names
and processes imposes a mutual recursion on alpha-equivalence and
structural equivalence via name-equivalence. Fortunately, all of this
works out pleasantly and we may calculate in the natural way, free of
concern. The reader interested in the details is referred to the
appendix \ref{appendix:rho_details}.

\subsection{Substitution}

We use $\Proc$ for the set of processes, $\QProc$ for the set of
names, and $\id{\{}\vec{y} / \vec{x} \id{\}}$ to denote partial maps,
$s : \QProc \rightarrow \QProc$. A map, $s$ lifts, uniquely, to a map
on process terms, $\widehat{s} : \Proc \rightarrow \Proc$ by the
following equations.

\begin{mathpar}
  (0) \psubstp{Q}{P} := 0 \\
  (R \juxtap S) \psubstp{Q}{P}
  :=    
  (R)\psubstp{Q}{P} \juxtap (S) \psubstp{Q}{P} \\
  (x?(y).R) \psubstp{Q}{P}    
  :=    
  (x)\substp{Q}{P} (z)\concat( (R \psubstn{z}{y}) \psubstp{Q}{P} ) \\
  (\lift{x}{R}) \psubstp{Q}{P}  
  :=
  \lift{(x)\substp{Q}{P}}{ R \psubstp{Q}{P} } \\
%   (\dropn{x})  \psubstp{Q}{P}       
%   := 
%   \left\{ 
%     \begin{array}{ccc} 
%       \dropn{\quotep{Q}} & & x \nameeq \quotep{P} \\
%       \dropn{x} & & otherwise \\
%     \end{array}
%   \right. 
  (\dropn{x})  \psubstp{Q}{P}       
  := 
  \left\{ 
    \begin{array}{ccc} 
      Q & & x \nameeq \quotep{P} \\
      \dropn{x} & & otherwise \\
    \end{array}
  \right.
\end{mathpar}
 

where

\begin{eqnarray}
  (x)\id{\{} \lpquote Q \rpquote / \lpquote P \rpquote \id{\}}            = 
  \left\{ 
    \begin{array}{ccc}
      \lpquote Q \rpquote & & x \nameeq \lpquote P \rpquote \\
      x & & otherwise \\
    \end{array}
  \right. \nonumber
\end{eqnarray}

and $z$ is chosen distinct from $\quotep{P}$, $\quotep{Q}$, the free
names in $Q$, and all the names in $R$. Our $\alpha$-equivalence will
be built in the standard way from this substitution.

\begin{remark}\label{rem:no_self_referential_names}
  One consequence of these definitions is that $\forall P. \quotep{P}
  \not\in \freenames{P}$.
\end{remark}

\subsection{ Dynamic quote: an example }

Anticipating something of what's to come, consider applying the
substitution, $\widehat{\id{\{}u / z \id{\}}}$, to the following pair
of processes, $\lift{w}{y!(z)}$ and $w[ \lpquote y!(z) \rpquote ]$.

\begin{eqnarray}
	\lift{w}{y!(z)}\widehat{\id{\{}u / z \id{\}}}
		& = &
		\lift{w}{y!(u)} \nonumber\\
	w[ \lpquote y!(z) \rpquote ] \widehat{ \id{\{}u / z \id{\}} }
		& = &
		w[ \lpquote y!(z) \rpquote ] \nonumber
\end{eqnarray}

Because the body of the process between quotes is impervious to
substitution, we get radically different answers. In fact, by
examining the first process in an input context,
e.g. $x?(z).\lift{w}{y!(z)}$, we see that the process under the lift
operator may be shaped by prefixed inputs binding a name inside it. In
this sense, the lift operator will be seen as a way to dynamically
construct processes before reifying them as names.

Finally equipped with these standard features we can present the
dynamics of the calculus.

\subsubsection{Operational semantics} 

Finally, we introduce the computational dynamics. What marks these
algebras as distinct from other more traditionally studied algebraic
structures, e.g. vector spaces or polynomial rings, is the manner in
which dynamics is captured. In traditional structures, dynamics is typically
expressed through morphisms between such structures, as in linear maps
between vector spaces or morphisms between rings. In algebras
associated with the semantics of computation, the dynamics is
expressed as part of the algebraic structure itself, through a
reduction reduction relation typically denoted by $\red$. Below, we
give a recursive presentation of this relation for the calculus used
in the encoding.

$\red \subseteq \pi \times \pi$
$\red : \pi \to \mathcal{P}(\pi)$

\begin{mathpar}
  \inferrule* [lab=Comm] { \textsf{match}( x_{src}, x_{trgt} ) } { x_{trgt}?(y)P \; | \; x_{src}!\langle {Q} \rangle \red P\{\quotep{Q}/y}\} }
  \and \\
  \inferrule* [lab=Par] {{P} \red {P}'} {{{P} | {Q}} \red {{P}' | {Q}}}
  \and
  \inferrule* [lab=Equiv]{{{P} \scong {P}'} \andalso {{P}' \red {Q}'} \andalso {{Q}' \scong {Q}}}{{P} \red {Q}}
\end{mathpar}

\begin{eqnarray*}
  match_{\equiv} (\quotep{P},\quotep{Q}) & := & P \equiv Q \\
  match_{\dagger}(\quotep{P},\quotep{Q}) & := & \forall R. P|Q \red^{*} R => R \red^{*} 0 \\
  match_{K}(\quotep{P},\quotep{Q}) & := & K \mbox{ for some context } K
\end{eqnarray*}

$u?(x)P | u!\langle Q \rangle \red P\{\quotep{Q}/x\}$

%We write $\wred$ for $\red^*$, and $P\red$ if $\exists Q $ such that $ P \red Q$.
We write $P\red$ if $\exists Q $ such that $ P \red Q$ and $P\not\red$, otherwise.

\section{Replication}

As mentioned before, it is known that replication (and hence
recursion) can be implemented in a higher-order process algebra
\cite{SangiorgiWalker}. As our first example of calculation with the
machinery thus far presented we give the construction explicitly in
the {\rhoc}.

\begin{eqnarray}
	D_{x} & := & \prefix{x}{y}{(\binpar{\outputp{x}{y}}{@{y}})} \nonumber\\
	\bangp_{x}{P} & := & \binpar{{x}!\langle{\binpar{D_{x}}{P}}\rangle}{D_{x}} \nonumber
\end{eqnarray}

\begin{eqnarray}
	\bangp_{x}{P} & & \nonumber\\
	=
	& {x}!\langle{(\prefix{x}{y}{(\outputp{x}{y} | @{y})) | P}}\rangle 
	      | \prefix{x}{y}{(\outputp{x}{y} | @{y})} & \nonumber\\
	\red
	& (\outputp{x}{y} | @{y})\substn{\quotep{(\prefix{x}{y}{(@{y} | \outputp{x}{y})) | P}}}{y} & \nonumber\\
	=
	& \outputp{x}{\quotep{(\prefix{x}{y}{(\outputp{x}{y} | @{y})) | P}}}
	  | {(\prefix{x}{y}{(\outputp{x}{y} | @{y})) | P}} & \nonumber\\
	\red
	& \ldots & \nonumber\\
	\red^*
	& P | P | \ldots & \nonumber
\end{eqnarray}

Of course, this encoding, as an implementation, runs away, unfolding
$\bangp{P}$ eagerly. A lazier and more implementable replication
operator, restricted to input-guarded processes, may be obtained as follows.

\begin{eqnarray}
\bangp{\prefix{u}{v}{P}} 
	:= 
	\binpar{\lift{x}{\prefix{u}{v}{(\binpar{D(x)}{P})}}}{D(x)} \nonumber
\end{eqnarray}

\begin{remark}
  Note that the lazier definition still does not deal with summation
  or mixed summation (i.e. sums over input and output). The reader is
  invited to construct definitions of replication that deal with these
  features. 

  Further, the definitions are parameterized in a name, $x$. Can you,
  gentle reader, make a definition that eliminates this parameter and
  guarantees no accidental interaction between the replication
  machinery and the process being replicated -- i.e. no accidental
  sharing of names used by the process to get its work done and the
  name(s) used by the replication to effect copying. This latter
  revision of the definition of replication is crucial to obtaining
  the expected identity $!!P \sim !P$.
\end{remark}

\begin{remark}\label{rem:paradoxical_combinator}
  The reader familiar with the lambda calculus will have noticed the
  similarity between $D$ and the paradoxical combinator.

  [Ed. note: the existence of this seems to suggest we have to be more
  restrictive on the set of processes and names we admit if we are to
  support no-cloning.]
\end{remark}

\subsubsection{Bisimulation}

The computational dynamics gives rise to another kind of equivalence,
the equivalence of computational behavior. As previously mentioned
this is typically captured \emph{via} some form of bisimulation.

% The notion we use in this paper is weak barbed bisimulation
% \cite{milner91polyadicpi}.

The notion we use in this paper is derived from weak barbed
bisimulation \cite{milner91polyadicpi}. 

\begin{definition}
An \emph{observation relation}, $\downarrow_{\mathcal N}$, over a set
of names, $\mathcal N$, is the smallest relation satisfying the rules
below.

\infrule[Out-barb]{y \in {\mathcal N}, \; x \nameeq y}
		  {\outputp{x}{v} \downarrow_{\mathcal N} x}
\infrule[Par-barb]{\mbox{$P\downarrow_{\mathcal N} x$ or $Q\downarrow_{\mathcal N} x$}}
		  {\binpar{P}{Q} \downarrow_{\mathcal N} x}

We write $P \Downarrow_{\mathcal N} x$ if there is $Q$ such that 
$P \wred Q$ and $Q \downarrow_{\mathcal N} x$.
\end{definition}

\begin{definition}
%\label{def.bbisim}
An  ${\mathcal N}$-\emph{barbed bisimulation} over a set of names, ${\mathcal N}$, is a symmetric binary relation 
${\mathcal S}_{\mathcal N}$ between agents such that $P\rel{S}_{\mathcal N}Q$ implies:
\begin{enumerate}
\item If $P \red P'$ then $Q \wred Q'$ and $P'\rel{S}_{\mathcal N} Q'$.
\item If $P\downarrow_{\mathcal N} x$, then $Q\Downarrow_{\mathcal N} x$.
\end{enumerate}
$P$ is ${\mathcal N}$-barbed bisimilar to $Q$, written
$P \wbbisim_{\mathcal N} Q$, if $P \rel{S}_{\mathcal N} Q$ for some ${\mathcal N}$-barbed bisimulation ${\mathcal S}_{\mathcal N}$.
\end{definition}

$\mathcal{R} \subseteq \pi \times \pi$

$P \mathcal{R} Q => \forall P'. P \red P' \Rightarrow \exists Q'. Q \red Q', P' \mathcal{R} Q'$

$P \vdash x \Rightarrow Q \vdash x$

\begin{mathpar}
  \inferrule*[lab=Out-barb]{x \nameeq y}{{y}!\langle{Q}\rangle \vdash x}
  \and
  \inferrule*[lab=Par-barb]{\mbox{$P\vdash x$ or $Q\vdash x$}}{\binpar{P}{Q} \vdash x}
\end{mathpar}

\subsubsection{Contexts}

One of the principle advantages of computational calculi like the
$\pi$-calculus is a well-defined notion of context,
contextual-equivalence and a correlation between
contextual-equivalence and notions of bisimulation. The notion of
context allows the decomposition of a process into (sub-)process and
its syntactic environment, its context. Thus, a context may be
thought of as a process with a ``hole'' (written $\Box$) in it. The
application of a context $M$ to a process $P$, written $M[P]$, is
tantamount to filling the hole in $M$ with $P$. In this paper we do
not need the full weight of this theory, but do make use of the notion
of context in the proof the main theorem. 

\begin{mathpar}
  \inferrule* [lab=summation] {} {{M_{M},M_{N}} \bc \Box \;|\; x.M_{A} \;|\; M_{M}+M_{N}}
  \and
  \inferrule* [lab=agent] {} {{M_{A}} \bc (\vec{x})M_{P} \;| \; \clift{P_0,\ldots,M_{P},\ldots,P_N}}
  \and \\
  \inferrule* [lab=process] {} {{M_{P}} \bc M_{N} \;| \;P|M_{P} }
\end{mathpar} 

\begin{mathpar}
  \inferrule* [lab=sychronization] {} {M_{N} \bc \Box \;|\; x?M_{F} \;|\; x!M_{C}}
  \and
  \inferrule* [lab=abstraction] {} {{M_{F}} \bc (x)M_{P} }
  \and
  \inferrule* [lab=concretion] {} {{M_{C}} \bc \langle M_{P} \rangle }
  \and \\
  \inferrule* [lab=process] {} {{M_{P}} \bc M_{N} \;| \;P|M_{P} }
\end{mathpar}

\begin{definition}[contextual application] Given a context $M$, and
  process $P$, we define the \emph{contextual application}, $M[P] :=
  M\{P/\Box\}$. That is, the contextual application of M to P is the
  substitution of $P$ for $\Box$ in $M$.
\end{definition}

$\meaningof{-} : L \to \mathcal{P}(\pi)$

\begin{mathpar}
  \inferrule* [lab=collection] {} {\meaningof{true} = \pi, \and \meaningof{~E} = \pi \setminus \meaningof{E}, \and \meaningof{E_{1} \& E_{2}} = \meaningof{E_{1}} \cap \meaningof{E_{2}}}
\end{mathpar}

\begin{mathpar}
  \inferrule* [lab=structure] {} {\meaningof{0} = \{ P \in \pi | P \equiv 0 \}, \and \\ \meaningof{E_1 | E_2} = \{ P \in \pi | P \equiv P_{1} | P_{2}, P_{1} \in \meaningof{E_{1}}, P_{2} \in \meaningof{E_2}\} }
\end{mathpar}

\begin{mathpar}
 \inferrule* [lab=behavior] {} {\meaningof{\langle a?b \rangle E} = \{ P \in \pi | P \equiv Q | u?(y)P', \\ \and \\\\ \and \\ \;\;\; u \in \meaningof{a}, \forall z.P'\{z/y\} \in \meaningof{E\{z/b\}}\}, \and \\ \meaningof{a!E} = \{ P \in \pi | P \equiv Q | x!\langle P' \rangle, x \in \meaningof{a} P' \in \meaningof{E}\} }
\end{mathpar}

\begin{mathpar}
 \inferrule* [lab=nominal] {} {\meaningof{\quotep{E}} = \{ \quotep{P} \in \quotep{\pi} | P \in \meaningof{E} \}, \and \meaningof{\quotep{P}} = \{ \quotep{Q} \in \quotep{\pi} | P \equiv Q \} \and \\ \meaningof{@\quotep{E}} = \{ P \in \pi | P \equiv @x, x \in \meaningof{E} \}}
\end{mathpar}

\begin{eqnarray*}
  \\
  \meaningof{-} : TS \to ST
\end{eqnarray*}

\begin{eqnarray*}
  \\
  L : TS \to ST
\end{eqnarray*}

\begin{eqnarray*}
  \\
  P \models E \iff P \in \meaningof{E}
\end{eqnarray*}

\begin{eqnarray*}
  P \approx_{L} Q \iff \forall E \in L. P \models E \iff Q \models E
\end{eqnarray*}

\begin{eqnarray*}
  P \approx_{K} Q
\end{eqnarray*}

\begin{eqnarray*}
  P \approx Q
\end{eqnarray*}

$\approx_{K} = \approx = \approx_{L}$

\subsubsection{Contextual duality}

Note that contexts extend the quotation operation to a family of
operations from processes to names. Given a context, $M$, we can
define a \emph{nominal context}, $\quotep{M}$ by $\quotep{M}[P] :=
\quotep{M[P]}$. To foreshadow what is to come we observe that these
operations enjoy a duality with processes very much like the duality
between vectors and maps from vectors to scalars.

Further, because the calculus is essentially higher-order, we have a
correspondence between contexts and processes. More specifically,
given a name $x$ and a context $M$ we can construct $M^{*}_{x}$ such
that 

\begin{mathpar}
  M^{*}_{x} | \lift{x}{P} \red M[P]
\end{mathpar}

namely,

\begin{mathpar}
  M^{*}_{x} := x?(u).M[\dropn{u}]
\end{mathpar}

The dependence of $M^{*}_{x}$ on a name makes it an abstraction, 

\begin{mathpar}
  M^{*} := (x)x?(u).M[\dropn{u}]
\end{mathpar}

\subsection{Additional notation}

It will sometimes be convenient to denote the process a name
quotes. We already have the notation $x = \quotep{P}$, but it will be
convenient to introduce an alternate notation, $\procn{x}$, when we
want to emphasize the connection to the use of the name. Note that, by
virtue of name equivalence, $\quotep{\procn{x}} \nameeq x$; so, the
notation is consistent with previous definitions.

Further, because names have structure it is possible to effect
substitutions on the basis of that structure. This means we need to
upgrade our notation for substitutions, which we accomplish by
adapting comprehension notation. Thus,

\begin{mathpar}
  P\{ y / x : x \in S \}
\end{mathpar}

is interpreted to mean the process derived from P by replacing (in a
capture-avoiding manner) each occurrence of $x$ in $S$ by $y$. For example,

\begin{mathpar}
  P\{ \quotep{\procn{x}|\procn{x}} / x : x \in \freenames{P} \}
\end{mathpar}

will replace each (occurrence) of a free name $x$ in $P$ by
$\quotep{\procn{x}|\procn{x}}$.

Also, we will avail ourselves of the notation $x^{L}$ and $x^{R}$ to
denote injections of a name into disjoint copies of the name
space. There are numerous ways to accomplish this. One example can be
found in \cite{MeredithR05}. This notation overloads to vectors of
names: $\vec{x}^{\pi} := (x_{i}^{\pi} \; : \; 0 \leq i < |\vec{x}| )$ where $\pi \in \{L,R\}$.

We also use $P^{\Box} := P|\Box$.

In \cite{MeredithR05} an interpretation of the new operator is
given. It turns out that there are several possible interpretations
all enjoying the requisite algebraic properties of the operator (see
\cite{milner91polyadicpi}). We will therefore make liberal use of
$(\nu\; \vec{x})P$.

% subsection the_syntax_and_semantics_of_the_notation_system (end)   

\input{qm2pi.qmops} 

\input{qm2pi.sterngerlach} 

\input{qm2pi.metric} 

% section concurrent_process_calculi (end)

%\input{qm2pi.proofsketch}

% section proof sketch (end)

%\input{qm2pi.slviaknots} 

% section spatial logic via knots (end)

\input{qm2pi.conclusion}

% section conclusion (end)

%\input{qm2pi.dtcodes} 

% section wiring algorithm (end)

\input{qm2pi.ack} 

% section acknowledgments (end)

\newpage


\bibliographystyle{plain}   
\bibliography{../../biblios/main.bib}

\input{qm2pi.rhodetails}

\end{document}



% section front matter (end)

\section{Introduction}\label{sec:introduction} % (fold)
In this draft of the material i am going to have to dispense with the
usual writing conventions adopted in papers on these topics. i'm going
to have adopt whatever tone i need at the time i'm writing up the
calculations. Sometimes this may be very conversational; others it may
be the barest mathematical grunts; others still it may be that i have
lifted text from one of my other papers because the exposition of some
point was better said there. i hope that my readers are not unduly put
out by this decision. i'm not doing this to flout convention or be
rebellious. i find these calculations very technically challenging. To
keep everything going technically, something has to give; i have to
let go of some cognitive burden. So, the academic writing style --
with all of its trade-offs in terms of facilitating technical
communication -- is what i'm letting go of. Perhaps subsequent drafts
can be tightened and polished, but for now, i'm going to speak as if
we were sitting together in a coffee shop with a laptop, wifi and a
pad of paper and a pencil.

So, here's what i have to say. We -- you and i, comfortably ensconced
in our coffee shop and well-equipped with our tools -- can realize and
carry out the calculations of quantum mechanics over a very different
formal theory of dynamics, a formal theory of dynamics that
corresponds to a theory of concurrent computation with
\emph{reflection}. It has the advantage that the underlying theory is
already `quantized', but supports analogues all of the continuuous
operations. Strikingly, this underlying theory has recently been
connected with a notion of metric that we can show, by calculating
together, coincides with the metric induced by the inner product.

There are a lot of reasons why you might be interested in seeing
calculations of this form. Here's why i'm interested. For the past
several centuries there has been no competitor to the ``Newtonian''
account of dynamics. As a result the predominant share of accounts of
dynamical systems and situations have had to be formulated in terms of
the Newtonian machinery. i view this as an intellectually dangerous
position to occupy. Everything, despite it's intrinsic shape, turns
into a nail to be hit with this hammer. Recently, however, the theory
of computation has matured to the point where we have candidates for
theories of dynamics that offer very different perspective on
reasoning about dynamical systems and situations. Testing these
candidates against very successful accounts of dynamical situations,
like quantum mechanics, is going to give us some sense of how mature
they are and some measure of the quality of these accounts of
dynamics.

\subsection{Summary of contributions and outline of paper}

So, we're going to develop an interpretation of the operations of
quantum mechanics normally interpreted by Hilbert spaces and
operators. We're going to do this over a theory of computation. Note
that this is very different than the usual quantum computation program
which develops notions of computation over quantum mechanics. Rather,
we are developing a story that aligns with Wheeler's slogan: It from
Bit. To do this we will first provide an account of the theory of
computation at play here. Then we will dive into a calculation-driven
interpretation of the operations of quantum mechanics.

The reason we take this approach is that -- until very recently --
there hasn't been an axiomatic account of quantum mechanics. As a
result there has been no sharp delineation of the mathematical theory
supporting interpretation of the physical theory and the physical
theory, itself. So, ambient features of the maths are free to be
exploited (or supressed) without a real accounting of their physical
relevance. There is no sharp statement ``here's the physical theory''
qua \emph{theory} and ``here's the mathematical interpretation''
enabling a judgment of how faithful the interpretation is -- apart
from experimental observation. When there is an axiomatic account we
can judge how well a given mathematical formalism supports an
interpretation of the axioms, independent of
experimentation. Likewise, we can judge how well we have captured our
physical evidence and experience with our axiomatics, independent of
any specific mathematical implementation, with accidental detail that
may or may not have physical significance. 

In lieu of a fully fleshed out and vetted axiomatic account of quantum
mechanics, interpreting the operational notions in service of modeling
physical systems will have to suffice. In other words, we are not in
the business of providing a model of Hilbert spaces and operators. We
are in the business of providing a model of quantum mechanics because
we are motivated by testing our notions of dynamics against physical
theory; and, the predictive calculations of the physical theory must
serve as the best formulation -- shy of a fully fleshed out axiomatic
account -- of the physical theory itself (as they have for scientific
theories since time immemorial). Put another way, despite a
whole-hearted commitment to an It-from-Bit ontology, we are firmly
aligned with the shut-up-and-calculate camp as the best way to obtain
results either from the physical perspective or as a quality assurance
measure of our fledgling theory of dynamics.

In detail, we present a reflective process calculus. Then we develop
intuitive correspondences between the notions available in this
calculus and the usual physical notions supporting quantum mechanical
calculations. Thus, 

\begin{table}[htp]
  \center{
    \fbox{
      \begin{tabular}{c|c}
        quantum mechanics & process calculus \\
        \hline
        scalar & name \\
        state vector & process \\
        dual & contextual duals \\
        matrix & formal sums of process-context-dual pairs \\
        orthogonality & process annihilation \\
        inner product & execution-formula + quoting
      \end{tabular}
    }
  }
  \caption{QM - process calculi correspondences}
\end{table}

Then we tighten up these intuitions to operational definitions. We
employ the Dirac notation as the best proxy we can find for an
abstract syntax of the quantum mechanical notions. The definitions we
develop put us in contact with equational constraints coming from the
theory that we demonstrate the definitions and calculations satisfy.

This puts us in a position to shut up and calculate for the
Stern-Gerlach experimental set up, showing how these predictive
calculations become calculations on processes in our theory of a
reflective process calculus.

Penultimately, we demonstrate that the notion of metric coming from
the inner product coincides with the notion of metric available from
the theory of bisimulation. This demonstration gives us the right to
think of space as arising from behavior. Finally, we consider where we
might go from the new vantage point we have obtained.

% section introduction (end) 
 
% section introduction (end)

% \documentclass[12pt]{llncs}
%\documentclass{jktr}

\usepackage[pdftex]{hyperref}                   
\usepackage {listings}
\usepackage {mathpartir}
\usepackage{bcprules}
%\usepackage{listings}
                       
\usepackage{graphicx} 
%\usepackage[margins=2.5cm,nohead,nofoot]{geometry}
%\usepackage{geometry}
\usepackage{amsfonts}
\usepackage{amstext}
\usepackage{latexsym}
\usepackage{amssymb}
\usepackage{color}


%\include{myPreamble}
\include{qm2pi.local} 

%\ifpdf
%\usepackage[pdftex]{graphicx}
%\else
%\usepackage{graphicx}
%\fi

 % \ifpdf
%  \usepackage{pdfsync}
%  \if


%\title{Brief Article}
%\author{David F. Snyder}
%\author{L.G. Meredith}

%\address{Dept. of Math., Texas State University--San Marcos, San Marcos, TX 78666}
       
\pagestyle{empty}


\begin{document}

\lstset{language=[Objective]Caml,frame=shadowbox}

\input{qm2pi.front}

% section front matter (end)

\input{qm2pi.intro} 
 
% section introduction (end)

% \input{qm2pi.knotations} 

% section notation (end)

\input{qm2pi.process.calculi} 

% section concurrent_process_calculi_and_spatial_logics_ (end)
    
%\input{qm2pi.knots2pi} 

%\input{qm2pi.trefoil} 

%\input{qm2pi.mainthm} 

% subsection basic_interpretation (end)

%\input{qm2pi.rho.presentation} 
\subsection{The syntax and semantics of the notation system}\label{sub:the_syntax_and_semantics_of_the_notation_system} % (fold)

We now summarize a technical presentation of the calculus that
embodies our theory of dynamics. The typical presentation of such a
calculus follows the style of giving generators and relations on
them. The grammar, below, describing term constructors, freely
generates the set of processes, $\Proc$. This set is then quotiented
by a relation known as structural congruence and it is over this set
that the notion of dynamics is expressed. This presentation is
essentially that of \cite{MeredithR05} with the addition of
polyadicity and summation. For readability we have relegated some of
the technical subtleties to an appendix.

\subsubsection{Process grammar}\label{subsub:process_grammar}

\begin{mathpar}
  \inferrule* [lab=synchronization] {} {{M} \bc \pzero \;|\; x?F \;|\; x!C }
  \and
  \inferrule* [lab=abstraction] {} {{F} \bc (x)P}
  \and
  \inferrule* [lab=concretion] {} {{C} \bc \langle Q \rangle}
  \and
  \inferrule* [lab=process] {} {{P,Q} \bc M \;| \;P|Q \;|\; @{x}}
  \and
  \inferrule* [lab=name] {} {{x} \bc \quotep{P}}
\end{mathpar} 

Note that $\vec{x}$ (resp. $\vec{P}$) denotes a vector of names
(resp. processes) of length $|\vec{x}|$ (resp. $|\vec{P}|$). We adopt
the following useful abbreviations.

\begin{mathpar}
   x?(\vec{y}).P := x.(\vec{y})P \and  x\clift{\vec{P}} := x.\clift{\vec{P}}
   \and x!(y) := \lift{x}{\dropn{y}}
   \and \Pi_{i=0}^{n-1}P_i := P_0 | \ldots | P_{n-1}
\end{mathpar}

\subsubsection{Structural congruence}

\paragraph{Free and bound names and alpha-equivalence.} At the
core of structural equivalence is alpha-equivalence which identifies
process that are the same up to a change of variable. Formally, we
recognize the distinction between free and bound names. The free names
of a process, $\freenames{P}$, may be calculated recursively as
follows:

\begin{mathpar}
\freenames{\pzero} := \emptyset
  \and \\
  \freenames{x?(y).P} := \{ x \} \cup (\freenames{P} \setminus \{ y \})
  \and 
  \freenames{x!\langle P \rangle} := \{ x \} \cup \{ P \} 
  \and \\
  \freenames{P|Q} := \freenames{P} \cup \freenames{Q}
  \and \\
  \freenames{@{x}} := \{ x \}
\end{mathpar}

$\pi$
$\quotep{\pi}$

$\freenames{-} : \pi \to \mathcal{P}(\quotep{\pi})$

\begin{eqnarray*}
  \freenames{\pzero} & := & \emptyset \\
  \freenames{x?(y).P} & := & \{ x \} \cup (\freenames{P} \setminus \{ y \}) \\
  \freenames{x!\langle P \rangle} & := & \{ x \} \cup \{ P \} \\
  \freenames{P|Q} & := & \freenames{P} \cup \freenames{Q} \\
  \freenames{\dropn{x}} & := & \{ x \}
\end{eqnarray*}

The bound names of a process, $\boundnames{P}$, are those names occurring in $P$
that are not free. For example, in $x?(y).0$, the name $x$ is free, while $y$ is bound.

\begin{mathpar}
  \inferrule* [lab=monoidal-laws] {} { P|Q \equiv Q|P \and P|0 \equiv P \and P|(Q|R) \equiv (P|Q)|R }
\end{mathpar}

\begin{mathpar}
  \inferrule* [lab=alpha-equivalence] {} { (x)P \equiv (y)P\{y/x\} \and y \not\in \freenames{P} }
\end{mathpar}

\begin{definition}
Then two processes, $P,Q$, are alpha-equivalent if $P = Q\{\vec{y}/\vec{x}\}$ for
some $\vec{x} \in \boundnames{Q},\vec{y} \in \boundnames{P}$, where $Q\{\vec{y}/\vec{x}\}$
denotes the capture-avoiding substitution of $\vec{y}$ for $\vec{x}$ in $Q$.
\end{definition}

\begin{definition}
  The {\em structural congruence} \cite{SangiorgiWalker} , $\equiv$,
  between processes is the least congruence containing
  alpha-equivalence, satisfying the abelian monoid laws
  (associativity, commutativity and $\pzero$ as identity) for parallel
  composition $|$ and for summation $+$.
\end{definition}

\subsection{Name equivalence}

We take name equivalence, written $\nameeq$, to be the smallest
equivalence relation generated by the following rules.

\begin{mathpar}
\inferrule*[lab=Quote-drop]
{ }
{ \quotep{@{x}} \nameeq x }

\inferrule*[lab=Struct-equiv]
{ P \scong Q }
{ \quotep{P} \nameeq \quotep{Q} }
\end{mathpar}

The astute reader will have noticed that the mutual recursion of names
and processes imposes a mutual recursion on alpha-equivalence and
structural equivalence via name-equivalence. Fortunately, all of this
works out pleasantly and we may calculate in the natural way, free of
concern. The reader interested in the details is referred to the
appendix \ref{appendix:rho_details}.

\subsection{Substitution}

We use $\Proc$ for the set of processes, $\QProc$ for the set of
names, and $\id{\{}\vec{y} / \vec{x} \id{\}}$ to denote partial maps,
$s : \QProc \rightarrow \QProc$. A map, $s$ lifts, uniquely, to a map
on process terms, $\widehat{s} : \Proc \rightarrow \Proc$ by the
following equations.

\begin{mathpar}
  (0) \psubstp{Q}{P} := 0 \\
  (R \juxtap S) \psubstp{Q}{P}
  :=    
  (R)\psubstp{Q}{P} \juxtap (S) \psubstp{Q}{P} \\
  (x?(y).R) \psubstp{Q}{P}    
  :=    
  (x)\substp{Q}{P} (z)\concat( (R \psubstn{z}{y}) \psubstp{Q}{P} ) \\
  (\lift{x}{R}) \psubstp{Q}{P}  
  :=
  \lift{(x)\substp{Q}{P}}{ R \psubstp{Q}{P} } \\
%   (\dropn{x})  \psubstp{Q}{P}       
%   := 
%   \left\{ 
%     \begin{array}{ccc} 
%       \dropn{\quotep{Q}} & & x \nameeq \quotep{P} \\
%       \dropn{x} & & otherwise \\
%     \end{array}
%   \right. 
  (\dropn{x})  \psubstp{Q}{P}       
  := 
  \left\{ 
    \begin{array}{ccc} 
      Q & & x \nameeq \quotep{P} \\
      \dropn{x} & & otherwise \\
    \end{array}
  \right.
\end{mathpar}
 

where

\begin{eqnarray}
  (x)\id{\{} \lpquote Q \rpquote / \lpquote P \rpquote \id{\}}            = 
  \left\{ 
    \begin{array}{ccc}
      \lpquote Q \rpquote & & x \nameeq \lpquote P \rpquote \\
      x & & otherwise \\
    \end{array}
  \right. \nonumber
\end{eqnarray}

and $z$ is chosen distinct from $\quotep{P}$, $\quotep{Q}$, the free
names in $Q$, and all the names in $R$. Our $\alpha$-equivalence will
be built in the standard way from this substitution.

\begin{remark}\label{rem:no_self_referential_names}
  One consequence of these definitions is that $\forall P. \quotep{P}
  \not\in \freenames{P}$.
\end{remark}

\subsection{ Dynamic quote: an example }

Anticipating something of what's to come, consider applying the
substitution, $\widehat{\id{\{}u / z \id{\}}}$, to the following pair
of processes, $\lift{w}{y!(z)}$ and $w[ \lpquote y!(z) \rpquote ]$.

\begin{eqnarray}
	\lift{w}{y!(z)}\widehat{\id{\{}u / z \id{\}}}
		& = &
		\lift{w}{y!(u)} \nonumber\\
	w[ \lpquote y!(z) \rpquote ] \widehat{ \id{\{}u / z \id{\}} }
		& = &
		w[ \lpquote y!(z) \rpquote ] \nonumber
\end{eqnarray}

Because the body of the process between quotes is impervious to
substitution, we get radically different answers. In fact, by
examining the first process in an input context,
e.g. $x?(z).\lift{w}{y!(z)}$, we see that the process under the lift
operator may be shaped by prefixed inputs binding a name inside it. In
this sense, the lift operator will be seen as a way to dynamically
construct processes before reifying them as names.

Finally equipped with these standard features we can present the
dynamics of the calculus.

\subsubsection{Operational semantics} 

Finally, we introduce the computational dynamics. What marks these
algebras as distinct from other more traditionally studied algebraic
structures, e.g. vector spaces or polynomial rings, is the manner in
which dynamics is captured. In traditional structures, dynamics is typically
expressed through morphisms between such structures, as in linear maps
between vector spaces or morphisms between rings. In algebras
associated with the semantics of computation, the dynamics is
expressed as part of the algebraic structure itself, through a
reduction reduction relation typically denoted by $\red$. Below, we
give a recursive presentation of this relation for the calculus used
in the encoding.

$\red \subseteq \pi \times \pi$
$\red : \pi \to \mathcal{P}(\pi)$

\begin{mathpar}
  \inferrule* [lab=Comm] { \textsf{match}( x_{src}, x_{trgt} ) } { x_{trgt}?(y)P \; | \; x_{src}!\langle {Q} \rangle \red P\{\quotep{Q}/y}\} }
  \and \\
  \inferrule* [lab=Par] {{P} \red {P}'} {{{P} | {Q}} \red {{P}' | {Q}}}
  \and
  \inferrule* [lab=Equiv]{{{P} \scong {P}'} \andalso {{P}' \red {Q}'} \andalso {{Q}' \scong {Q}}}{{P} \red {Q}}
\end{mathpar}

\begin{eqnarray*}
  match_{\equiv} (\quotep{P},\quotep{Q}) & := & P \equiv Q \\
  match_{\dagger}(\quotep{P},\quotep{Q}) & := & \forall R. P|Q \red^{*} R => R \red^{*} 0 \\
  match_{K}(\quotep{P},\quotep{Q}) & := & K \mbox{ for some context } K
\end{eqnarray*}

$u?(x)P | u!\langle Q \rangle \red P\{\quotep{Q}/x\}$

%We write $\wred$ for $\red^*$, and $P\red$ if $\exists Q $ such that $ P \red Q$.
We write $P\red$ if $\exists Q $ such that $ P \red Q$ and $P\not\red$, otherwise.

\section{Replication}

As mentioned before, it is known that replication (and hence
recursion) can be implemented in a higher-order process algebra
\cite{SangiorgiWalker}. As our first example of calculation with the
machinery thus far presented we give the construction explicitly in
the {\rhoc}.

\begin{eqnarray}
	D_{x} & := & \prefix{x}{y}{(\binpar{\outputp{x}{y}}{@{y}})} \nonumber\\
	\bangp_{x}{P} & := & \binpar{{x}!\langle{\binpar{D_{x}}{P}}\rangle}{D_{x}} \nonumber
\end{eqnarray}

\begin{eqnarray}
	\bangp_{x}{P} & & \nonumber\\
	=
	& {x}!\langle{(\prefix{x}{y}{(\outputp{x}{y} | @{y})) | P}}\rangle 
	      | \prefix{x}{y}{(\outputp{x}{y} | @{y})} & \nonumber\\
	\red
	& (\outputp{x}{y} | @{y})\substn{\quotep{(\prefix{x}{y}{(@{y} | \outputp{x}{y})) | P}}}{y} & \nonumber\\
	=
	& \outputp{x}{\quotep{(\prefix{x}{y}{(\outputp{x}{y} | @{y})) | P}}}
	  | {(\prefix{x}{y}{(\outputp{x}{y} | @{y})) | P}} & \nonumber\\
	\red
	& \ldots & \nonumber\\
	\red^*
	& P | P | \ldots & \nonumber
\end{eqnarray}

Of course, this encoding, as an implementation, runs away, unfolding
$\bangp{P}$ eagerly. A lazier and more implementable replication
operator, restricted to input-guarded processes, may be obtained as follows.

\begin{eqnarray}
\bangp{\prefix{u}{v}{P}} 
	:= 
	\binpar{\lift{x}{\prefix{u}{v}{(\binpar{D(x)}{P})}}}{D(x)} \nonumber
\end{eqnarray}

\begin{remark}
  Note that the lazier definition still does not deal with summation
  or mixed summation (i.e. sums over input and output). The reader is
  invited to construct definitions of replication that deal with these
  features. 

  Further, the definitions are parameterized in a name, $x$. Can you,
  gentle reader, make a definition that eliminates this parameter and
  guarantees no accidental interaction between the replication
  machinery and the process being replicated -- i.e. no accidental
  sharing of names used by the process to get its work done and the
  name(s) used by the replication to effect copying. This latter
  revision of the definition of replication is crucial to obtaining
  the expected identity $!!P \sim !P$.
\end{remark}

\begin{remark}\label{rem:paradoxical_combinator}
  The reader familiar with the lambda calculus will have noticed the
  similarity between $D$ and the paradoxical combinator.

  [Ed. note: the existence of this seems to suggest we have to be more
  restrictive on the set of processes and names we admit if we are to
  support no-cloning.]
\end{remark}

\subsubsection{Bisimulation}

The computational dynamics gives rise to another kind of equivalence,
the equivalence of computational behavior. As previously mentioned
this is typically captured \emph{via} some form of bisimulation.

% The notion we use in this paper is weak barbed bisimulation
% \cite{milner91polyadicpi}.

The notion we use in this paper is derived from weak barbed
bisimulation \cite{milner91polyadicpi}. 

\begin{definition}
An \emph{observation relation}, $\downarrow_{\mathcal N}$, over a set
of names, $\mathcal N$, is the smallest relation satisfying the rules
below.

\infrule[Out-barb]{y \in {\mathcal N}, \; x \nameeq y}
		  {\outputp{x}{v} \downarrow_{\mathcal N} x}
\infrule[Par-barb]{\mbox{$P\downarrow_{\mathcal N} x$ or $Q\downarrow_{\mathcal N} x$}}
		  {\binpar{P}{Q} \downarrow_{\mathcal N} x}

We write $P \Downarrow_{\mathcal N} x$ if there is $Q$ such that 
$P \wred Q$ and $Q \downarrow_{\mathcal N} x$.
\end{definition}

\begin{definition}
%\label{def.bbisim}
An  ${\mathcal N}$-\emph{barbed bisimulation} over a set of names, ${\mathcal N}$, is a symmetric binary relation 
${\mathcal S}_{\mathcal N}$ between agents such that $P\rel{S}_{\mathcal N}Q$ implies:
\begin{enumerate}
\item If $P \red P'$ then $Q \wred Q'$ and $P'\rel{S}_{\mathcal N} Q'$.
\item If $P\downarrow_{\mathcal N} x$, then $Q\Downarrow_{\mathcal N} x$.
\end{enumerate}
$P$ is ${\mathcal N}$-barbed bisimilar to $Q$, written
$P \wbbisim_{\mathcal N} Q$, if $P \rel{S}_{\mathcal N} Q$ for some ${\mathcal N}$-barbed bisimulation ${\mathcal S}_{\mathcal N}$.
\end{definition}

$\mathcal{R} \subseteq \pi \times \pi$

$P \mathcal{R} Q => \forall P'. P \red P' \Rightarrow \exists Q'. Q \red Q', P' \mathcal{R} Q'$

$P \vdash x \Rightarrow Q \vdash x$

\begin{mathpar}
  \inferrule*[lab=Out-barb]{x \nameeq y}{{y}!\langle{Q}\rangle \vdash x}
  \and
  \inferrule*[lab=Par-barb]{\mbox{$P\vdash x$ or $Q\vdash x$}}{\binpar{P}{Q} \vdash x}
\end{mathpar}

\subsubsection{Contexts}

One of the principle advantages of computational calculi like the
$\pi$-calculus is a well-defined notion of context,
contextual-equivalence and a correlation between
contextual-equivalence and notions of bisimulation. The notion of
context allows the decomposition of a process into (sub-)process and
its syntactic environment, its context. Thus, a context may be
thought of as a process with a ``hole'' (written $\Box$) in it. The
application of a context $M$ to a process $P$, written $M[P]$, is
tantamount to filling the hole in $M$ with $P$. In this paper we do
not need the full weight of this theory, but do make use of the notion
of context in the proof the main theorem. 

\begin{mathpar}
  \inferrule* [lab=summation] {} {{M_{M},M_{N}} \bc \Box \;|\; x.M_{A} \;|\; M_{M}+M_{N}}
  \and
  \inferrule* [lab=agent] {} {{M_{A}} \bc (\vec{x})M_{P} \;| \; \clift{P_0,\ldots,M_{P},\ldots,P_N}}
  \and \\
  \inferrule* [lab=process] {} {{M_{P}} \bc M_{N} \;| \;P|M_{P} }
\end{mathpar} 

\begin{mathpar}
  \inferrule* [lab=sychronization] {} {M_{N} \bc \Box \;|\; x?M_{F} \;|\; x!M_{C}}
  \and
  \inferrule* [lab=abstraction] {} {{M_{F}} \bc (x)M_{P} }
  \and
  \inferrule* [lab=concretion] {} {{M_{C}} \bc \langle M_{P} \rangle }
  \and \\
  \inferrule* [lab=process] {} {{M_{P}} \bc M_{N} \;| \;P|M_{P} }
\end{mathpar}

\begin{definition}[contextual application] Given a context $M$, and
  process $P$, we define the \emph{contextual application}, $M[P] :=
  M\{P/\Box\}$. That is, the contextual application of M to P is the
  substitution of $P$ for $\Box$ in $M$.
\end{definition}

$\meaningof{-} : L \to \mathcal{P}(\pi)$

\begin{mathpar}
  \inferrule* [lab=collection] {} {\meaningof{true} = \pi, \and \meaningof{~E} = \pi \setminus \meaningof{E}, \and \meaningof{E_{1} \& E_{2}} = \meaningof{E_{1}} \cap \meaningof{E_{2}}}
\end{mathpar}

\begin{mathpar}
  \inferrule* [lab=structure] {} {\meaningof{0} = \{ P \in \pi | P \equiv 0 \}, \and \\ \meaningof{E_1 | E_2} = \{ P \in \pi | P \equiv P_{1} | P_{2}, P_{1} \in \meaningof{E_{1}}, P_{2} \in \meaningof{E_2}\} }
\end{mathpar}

\begin{mathpar}
 \inferrule* [lab=behavior] {} {\meaningof{\langle a?b \rangle E} = \{ P \in \pi | P \equiv Q | u?(y)P', \\ \and \\\\ \and \\ \;\;\; u \in \meaningof{a}, \forall z.P'\{z/y\} \in \meaningof{E\{z/b\}}\}, \and \\ \meaningof{a!E} = \{ P \in \pi | P \equiv Q | x!\langle P' \rangle, x \in \meaningof{a} P' \in \meaningof{E}\} }
\end{mathpar}

\begin{mathpar}
 \inferrule* [lab=nominal] {} {\meaningof{\quotep{E}} = \{ \quotep{P} \in \quotep{\pi} | P \in \meaningof{E} \}, \and \meaningof{\quotep{P}} = \{ \quotep{Q} \in \quotep{\pi} | P \equiv Q \} \and \\ \meaningof{@\quotep{E}} = \{ P \in \pi | P \equiv @x, x \in \meaningof{E} \}}
\end{mathpar}

\begin{eqnarray*}
  \\
  \meaningof{-} : TS \to ST
\end{eqnarray*}

\begin{eqnarray*}
  \\
  L : TS \to ST
\end{eqnarray*}

\begin{eqnarray*}
  \\
  P \models E \iff P \in \meaningof{E}
\end{eqnarray*}

\begin{eqnarray*}
  P \approx_{L} Q \iff \forall E \in L. P \models E \iff Q \models E
\end{eqnarray*}

\begin{eqnarray*}
  P \approx_{K} Q
\end{eqnarray*}

\begin{eqnarray*}
  P \approx Q
\end{eqnarray*}

$\approx_{K} = \approx = \approx_{L}$

\subsubsection{Contextual duality}

Note that contexts extend the quotation operation to a family of
operations from processes to names. Given a context, $M$, we can
define a \emph{nominal context}, $\quotep{M}$ by $\quotep{M}[P] :=
\quotep{M[P]}$. To foreshadow what is to come we observe that these
operations enjoy a duality with processes very much like the duality
between vectors and maps from vectors to scalars.

Further, because the calculus is essentially higher-order, we have a
correspondence between contexts and processes. More specifically,
given a name $x$ and a context $M$ we can construct $M^{*}_{x}$ such
that 

\begin{mathpar}
  M^{*}_{x} | \lift{x}{P} \red M[P]
\end{mathpar}

namely,

\begin{mathpar}
  M^{*}_{x} := x?(u).M[\dropn{u}]
\end{mathpar}

The dependence of $M^{*}_{x}$ on a name makes it an abstraction, 

\begin{mathpar}
  M^{*} := (x)x?(u).M[\dropn{u}]
\end{mathpar}

\subsection{Additional notation}

It will sometimes be convenient to denote the process a name
quotes. We already have the notation $x = \quotep{P}$, but it will be
convenient to introduce an alternate notation, $\procn{x}$, when we
want to emphasize the connection to the use of the name. Note that, by
virtue of name equivalence, $\quotep{\procn{x}} \nameeq x$; so, the
notation is consistent with previous definitions.

Further, because names have structure it is possible to effect
substitutions on the basis of that structure. This means we need to
upgrade our notation for substitutions, which we accomplish by
adapting comprehension notation. Thus,

\begin{mathpar}
  P\{ y / x : x \in S \}
\end{mathpar}

is interpreted to mean the process derived from P by replacing (in a
capture-avoiding manner) each occurrence of $x$ in $S$ by $y$. For example,

\begin{mathpar}
  P\{ \quotep{\procn{x}|\procn{x}} / x : x \in \freenames{P} \}
\end{mathpar}

will replace each (occurrence) of a free name $x$ in $P$ by
$\quotep{\procn{x}|\procn{x}}$.

Also, we will avail ourselves of the notation $x^{L}$ and $x^{R}$ to
denote injections of a name into disjoint copies of the name
space. There are numerous ways to accomplish this. One example can be
found in \cite{MeredithR05}. This notation overloads to vectors of
names: $\vec{x}^{\pi} := (x_{i}^{\pi} \; : \; 0 \leq i < |\vec{x}| )$ where $\pi \in \{L,R\}$.

We also use $P^{\Box} := P|\Box$.

In \cite{MeredithR05} an interpretation of the new operator is
given. It turns out that there are several possible interpretations
all enjoying the requisite algebraic properties of the operator (see
\cite{milner91polyadicpi}). We will therefore make liberal use of
$(\nu\; \vec{x})P$.

% subsection the_syntax_and_semantics_of_the_notation_system (end)   

\input{qm2pi.qmops} 

\input{qm2pi.sterngerlach} 

\input{qm2pi.metric} 

% section concurrent_process_calculi (end)

%\input{qm2pi.proofsketch}

% section proof sketch (end)

%\input{qm2pi.slviaknots} 

% section spatial logic via knots (end)

\input{qm2pi.conclusion}

% section conclusion (end)

%\input{qm2pi.dtcodes} 

% section wiring algorithm (end)

\input{qm2pi.ack} 

% section acknowledgments (end)

\newpage


\bibliographystyle{plain}   
\bibliography{../../biblios/main.bib}

\input{qm2pi.rhodetails}

\end{document}

 

% section notation (end)

\input{qm2pi.process.calculi} 

% section concurrent_process_calculi_and_spatial_logics_ (end)
    
%\documentclass[12pt]{llncs}
%\documentclass{jktr}

\usepackage[pdftex]{hyperref}                   
\usepackage {listings}
\usepackage {mathpartir}
\usepackage{bcprules}
%\usepackage{listings}
                       
\usepackage{graphicx} 
%\usepackage[margins=2.5cm,nohead,nofoot]{geometry}
%\usepackage{geometry}
\usepackage{amsfonts}
\usepackage{amstext}
\usepackage{latexsym}
\usepackage{amssymb}
\usepackage{color}


%\include{myPreamble}
\include{qm2pi.local} 

%\ifpdf
%\usepackage[pdftex]{graphicx}
%\else
%\usepackage{graphicx}
%\fi

 % \ifpdf
%  \usepackage{pdfsync}
%  \if


%\title{Brief Article}
%\author{David F. Snyder}
%\author{L.G. Meredith}

%\address{Dept. of Math., Texas State University--San Marcos, San Marcos, TX 78666}
       
\pagestyle{empty}


\begin{document}

\lstset{language=[Objective]Caml,frame=shadowbox}

\input{qm2pi.front}

% section front matter (end)

\input{qm2pi.intro} 
 
% section introduction (end)

% \input{qm2pi.knotations} 

% section notation (end)

\input{qm2pi.process.calculi} 

% section concurrent_process_calculi_and_spatial_logics_ (end)
    
%\input{qm2pi.knots2pi} 

%\input{qm2pi.trefoil} 

%\input{qm2pi.mainthm} 

% subsection basic_interpretation (end)

%\input{qm2pi.rho.presentation} 
\subsection{The syntax and semantics of the notation system}\label{sub:the_syntax_and_semantics_of_the_notation_system} % (fold)

We now summarize a technical presentation of the calculus that
embodies our theory of dynamics. The typical presentation of such a
calculus follows the style of giving generators and relations on
them. The grammar, below, describing term constructors, freely
generates the set of processes, $\Proc$. This set is then quotiented
by a relation known as structural congruence and it is over this set
that the notion of dynamics is expressed. This presentation is
essentially that of \cite{MeredithR05} with the addition of
polyadicity and summation. For readability we have relegated some of
the technical subtleties to an appendix.

\subsubsection{Process grammar}\label{subsub:process_grammar}

\begin{mathpar}
  \inferrule* [lab=synchronization] {} {{M} \bc \pzero \;|\; x?F \;|\; x!C }
  \and
  \inferrule* [lab=abstraction] {} {{F} \bc (x)P}
  \and
  \inferrule* [lab=concretion] {} {{C} \bc \langle Q \rangle}
  \and
  \inferrule* [lab=process] {} {{P,Q} \bc M \;| \;P|Q \;|\; @{x}}
  \and
  \inferrule* [lab=name] {} {{x} \bc \quotep{P}}
\end{mathpar} 

Note that $\vec{x}$ (resp. $\vec{P}$) denotes a vector of names
(resp. processes) of length $|\vec{x}|$ (resp. $|\vec{P}|$). We adopt
the following useful abbreviations.

\begin{mathpar}
   x?(\vec{y}).P := x.(\vec{y})P \and  x\clift{\vec{P}} := x.\clift{\vec{P}}
   \and x!(y) := \lift{x}{\dropn{y}}
   \and \Pi_{i=0}^{n-1}P_i := P_0 | \ldots | P_{n-1}
\end{mathpar}

\subsubsection{Structural congruence}

\paragraph{Free and bound names and alpha-equivalence.} At the
core of structural equivalence is alpha-equivalence which identifies
process that are the same up to a change of variable. Formally, we
recognize the distinction between free and bound names. The free names
of a process, $\freenames{P}$, may be calculated recursively as
follows:

\begin{mathpar}
\freenames{\pzero} := \emptyset
  \and \\
  \freenames{x?(y).P} := \{ x \} \cup (\freenames{P} \setminus \{ y \})
  \and 
  \freenames{x!\langle P \rangle} := \{ x \} \cup \{ P \} 
  \and \\
  \freenames{P|Q} := \freenames{P} \cup \freenames{Q}
  \and \\
  \freenames{@{x}} := \{ x \}
\end{mathpar}

$\pi$
$\quotep{\pi}$

$\freenames{-} : \pi \to \mathcal{P}(\quotep{\pi})$

\begin{eqnarray*}
  \freenames{\pzero} & := & \emptyset \\
  \freenames{x?(y).P} & := & \{ x \} \cup (\freenames{P} \setminus \{ y \}) \\
  \freenames{x!\langle P \rangle} & := & \{ x \} \cup \{ P \} \\
  \freenames{P|Q} & := & \freenames{P} \cup \freenames{Q} \\
  \freenames{\dropn{x}} & := & \{ x \}
\end{eqnarray*}

The bound names of a process, $\boundnames{P}$, are those names occurring in $P$
that are not free. For example, in $x?(y).0$, the name $x$ is free, while $y$ is bound.

\begin{mathpar}
  \inferrule* [lab=monoidal-laws] {} { P|Q \equiv Q|P \and P|0 \equiv P \and P|(Q|R) \equiv (P|Q)|R }
\end{mathpar}

\begin{mathpar}
  \inferrule* [lab=alpha-equivalence] {} { (x)P \equiv (y)P\{y/x\} \and y \not\in \freenames{P} }
\end{mathpar}

\begin{definition}
Then two processes, $P,Q$, are alpha-equivalent if $P = Q\{\vec{y}/\vec{x}\}$ for
some $\vec{x} \in \boundnames{Q},\vec{y} \in \boundnames{P}$, where $Q\{\vec{y}/\vec{x}\}$
denotes the capture-avoiding substitution of $\vec{y}$ for $\vec{x}$ in $Q$.
\end{definition}

\begin{definition}
  The {\em structural congruence} \cite{SangiorgiWalker} , $\equiv$,
  between processes is the least congruence containing
  alpha-equivalence, satisfying the abelian monoid laws
  (associativity, commutativity and $\pzero$ as identity) for parallel
  composition $|$ and for summation $+$.
\end{definition}

\subsection{Name equivalence}

We take name equivalence, written $\nameeq$, to be the smallest
equivalence relation generated by the following rules.

\begin{mathpar}
\inferrule*[lab=Quote-drop]
{ }
{ \quotep{@{x}} \nameeq x }

\inferrule*[lab=Struct-equiv]
{ P \scong Q }
{ \quotep{P} \nameeq \quotep{Q} }
\end{mathpar}

The astute reader will have noticed that the mutual recursion of names
and processes imposes a mutual recursion on alpha-equivalence and
structural equivalence via name-equivalence. Fortunately, all of this
works out pleasantly and we may calculate in the natural way, free of
concern. The reader interested in the details is referred to the
appendix \ref{appendix:rho_details}.

\subsection{Substitution}

We use $\Proc$ for the set of processes, $\QProc$ for the set of
names, and $\id{\{}\vec{y} / \vec{x} \id{\}}$ to denote partial maps,
$s : \QProc \rightarrow \QProc$. A map, $s$ lifts, uniquely, to a map
on process terms, $\widehat{s} : \Proc \rightarrow \Proc$ by the
following equations.

\begin{mathpar}
  (0) \psubstp{Q}{P} := 0 \\
  (R \juxtap S) \psubstp{Q}{P}
  :=    
  (R)\psubstp{Q}{P} \juxtap (S) \psubstp{Q}{P} \\
  (x?(y).R) \psubstp{Q}{P}    
  :=    
  (x)\substp{Q}{P} (z)\concat( (R \psubstn{z}{y}) \psubstp{Q}{P} ) \\
  (\lift{x}{R}) \psubstp{Q}{P}  
  :=
  \lift{(x)\substp{Q}{P}}{ R \psubstp{Q}{P} } \\
%   (\dropn{x})  \psubstp{Q}{P}       
%   := 
%   \left\{ 
%     \begin{array}{ccc} 
%       \dropn{\quotep{Q}} & & x \nameeq \quotep{P} \\
%       \dropn{x} & & otherwise \\
%     \end{array}
%   \right. 
  (\dropn{x})  \psubstp{Q}{P}       
  := 
  \left\{ 
    \begin{array}{ccc} 
      Q & & x \nameeq \quotep{P} \\
      \dropn{x} & & otherwise \\
    \end{array}
  \right.
\end{mathpar}
 

where

\begin{eqnarray}
  (x)\id{\{} \lpquote Q \rpquote / \lpquote P \rpquote \id{\}}            = 
  \left\{ 
    \begin{array}{ccc}
      \lpquote Q \rpquote & & x \nameeq \lpquote P \rpquote \\
      x & & otherwise \\
    \end{array}
  \right. \nonumber
\end{eqnarray}

and $z$ is chosen distinct from $\quotep{P}$, $\quotep{Q}$, the free
names in $Q$, and all the names in $R$. Our $\alpha$-equivalence will
be built in the standard way from this substitution.

\begin{remark}\label{rem:no_self_referential_names}
  One consequence of these definitions is that $\forall P. \quotep{P}
  \not\in \freenames{P}$.
\end{remark}

\subsection{ Dynamic quote: an example }

Anticipating something of what's to come, consider applying the
substitution, $\widehat{\id{\{}u / z \id{\}}}$, to the following pair
of processes, $\lift{w}{y!(z)}$ and $w[ \lpquote y!(z) \rpquote ]$.

\begin{eqnarray}
	\lift{w}{y!(z)}\widehat{\id{\{}u / z \id{\}}}
		& = &
		\lift{w}{y!(u)} \nonumber\\
	w[ \lpquote y!(z) \rpquote ] \widehat{ \id{\{}u / z \id{\}} }
		& = &
		w[ \lpquote y!(z) \rpquote ] \nonumber
\end{eqnarray}

Because the body of the process between quotes is impervious to
substitution, we get radically different answers. In fact, by
examining the first process in an input context,
e.g. $x?(z).\lift{w}{y!(z)}$, we see that the process under the lift
operator may be shaped by prefixed inputs binding a name inside it. In
this sense, the lift operator will be seen as a way to dynamically
construct processes before reifying them as names.

Finally equipped with these standard features we can present the
dynamics of the calculus.

\subsubsection{Operational semantics} 

Finally, we introduce the computational dynamics. What marks these
algebras as distinct from other more traditionally studied algebraic
structures, e.g. vector spaces or polynomial rings, is the manner in
which dynamics is captured. In traditional structures, dynamics is typically
expressed through morphisms between such structures, as in linear maps
between vector spaces or morphisms between rings. In algebras
associated with the semantics of computation, the dynamics is
expressed as part of the algebraic structure itself, through a
reduction reduction relation typically denoted by $\red$. Below, we
give a recursive presentation of this relation for the calculus used
in the encoding.

$\red \subseteq \pi \times \pi$
$\red : \pi \to \mathcal{P}(\pi)$

\begin{mathpar}
  \inferrule* [lab=Comm] { \textsf{match}( x_{src}, x_{trgt} ) } { x_{trgt}?(y)P \; | \; x_{src}!\langle {Q} \rangle \red P\{\quotep{Q}/y}\} }
  \and \\
  \inferrule* [lab=Par] {{P} \red {P}'} {{{P} | {Q}} \red {{P}' | {Q}}}
  \and
  \inferrule* [lab=Equiv]{{{P} \scong {P}'} \andalso {{P}' \red {Q}'} \andalso {{Q}' \scong {Q}}}{{P} \red {Q}}
\end{mathpar}

\begin{eqnarray*}
  match_{\equiv} (\quotep{P},\quotep{Q}) & := & P \equiv Q \\
  match_{\dagger}(\quotep{P},\quotep{Q}) & := & \forall R. P|Q \red^{*} R => R \red^{*} 0 \\
  match_{K}(\quotep{P},\quotep{Q}) & := & K \mbox{ for some context } K
\end{eqnarray*}

$u?(x)P | u!\langle Q \rangle \red P\{\quotep{Q}/x\}$

%We write $\wred$ for $\red^*$, and $P\red$ if $\exists Q $ such that $ P \red Q$.
We write $P\red$ if $\exists Q $ such that $ P \red Q$ and $P\not\red$, otherwise.

\section{Replication}

As mentioned before, it is known that replication (and hence
recursion) can be implemented in a higher-order process algebra
\cite{SangiorgiWalker}. As our first example of calculation with the
machinery thus far presented we give the construction explicitly in
the {\rhoc}.

\begin{eqnarray}
	D_{x} & := & \prefix{x}{y}{(\binpar{\outputp{x}{y}}{@{y}})} \nonumber\\
	\bangp_{x}{P} & := & \binpar{{x}!\langle{\binpar{D_{x}}{P}}\rangle}{D_{x}} \nonumber
\end{eqnarray}

\begin{eqnarray}
	\bangp_{x}{P} & & \nonumber\\
	=
	& {x}!\langle{(\prefix{x}{y}{(\outputp{x}{y} | @{y})) | P}}\rangle 
	      | \prefix{x}{y}{(\outputp{x}{y} | @{y})} & \nonumber\\
	\red
	& (\outputp{x}{y} | @{y})\substn{\quotep{(\prefix{x}{y}{(@{y} | \outputp{x}{y})) | P}}}{y} & \nonumber\\
	=
	& \outputp{x}{\quotep{(\prefix{x}{y}{(\outputp{x}{y} | @{y})) | P}}}
	  | {(\prefix{x}{y}{(\outputp{x}{y} | @{y})) | P}} & \nonumber\\
	\red
	& \ldots & \nonumber\\
	\red^*
	& P | P | \ldots & \nonumber
\end{eqnarray}

Of course, this encoding, as an implementation, runs away, unfolding
$\bangp{P}$ eagerly. A lazier and more implementable replication
operator, restricted to input-guarded processes, may be obtained as follows.

\begin{eqnarray}
\bangp{\prefix{u}{v}{P}} 
	:= 
	\binpar{\lift{x}{\prefix{u}{v}{(\binpar{D(x)}{P})}}}{D(x)} \nonumber
\end{eqnarray}

\begin{remark}
  Note that the lazier definition still does not deal with summation
  or mixed summation (i.e. sums over input and output). The reader is
  invited to construct definitions of replication that deal with these
  features. 

  Further, the definitions are parameterized in a name, $x$. Can you,
  gentle reader, make a definition that eliminates this parameter and
  guarantees no accidental interaction between the replication
  machinery and the process being replicated -- i.e. no accidental
  sharing of names used by the process to get its work done and the
  name(s) used by the replication to effect copying. This latter
  revision of the definition of replication is crucial to obtaining
  the expected identity $!!P \sim !P$.
\end{remark}

\begin{remark}\label{rem:paradoxical_combinator}
  The reader familiar with the lambda calculus will have noticed the
  similarity between $D$ and the paradoxical combinator.

  [Ed. note: the existence of this seems to suggest we have to be more
  restrictive on the set of processes and names we admit if we are to
  support no-cloning.]
\end{remark}

\subsubsection{Bisimulation}

The computational dynamics gives rise to another kind of equivalence,
the equivalence of computational behavior. As previously mentioned
this is typically captured \emph{via} some form of bisimulation.

% The notion we use in this paper is weak barbed bisimulation
% \cite{milner91polyadicpi}.

The notion we use in this paper is derived from weak barbed
bisimulation \cite{milner91polyadicpi}. 

\begin{definition}
An \emph{observation relation}, $\downarrow_{\mathcal N}$, over a set
of names, $\mathcal N$, is the smallest relation satisfying the rules
below.

\infrule[Out-barb]{y \in {\mathcal N}, \; x \nameeq y}
		  {\outputp{x}{v} \downarrow_{\mathcal N} x}
\infrule[Par-barb]{\mbox{$P\downarrow_{\mathcal N} x$ or $Q\downarrow_{\mathcal N} x$}}
		  {\binpar{P}{Q} \downarrow_{\mathcal N} x}

We write $P \Downarrow_{\mathcal N} x$ if there is $Q$ such that 
$P \wred Q$ and $Q \downarrow_{\mathcal N} x$.
\end{definition}

\begin{definition}
%\label{def.bbisim}
An  ${\mathcal N}$-\emph{barbed bisimulation} over a set of names, ${\mathcal N}$, is a symmetric binary relation 
${\mathcal S}_{\mathcal N}$ between agents such that $P\rel{S}_{\mathcal N}Q$ implies:
\begin{enumerate}
\item If $P \red P'$ then $Q \wred Q'$ and $P'\rel{S}_{\mathcal N} Q'$.
\item If $P\downarrow_{\mathcal N} x$, then $Q\Downarrow_{\mathcal N} x$.
\end{enumerate}
$P$ is ${\mathcal N}$-barbed bisimilar to $Q$, written
$P \wbbisim_{\mathcal N} Q$, if $P \rel{S}_{\mathcal N} Q$ for some ${\mathcal N}$-barbed bisimulation ${\mathcal S}_{\mathcal N}$.
\end{definition}

$\mathcal{R} \subseteq \pi \times \pi$

$P \mathcal{R} Q => \forall P'. P \red P' \Rightarrow \exists Q'. Q \red Q', P' \mathcal{R} Q'$

$P \vdash x \Rightarrow Q \vdash x$

\begin{mathpar}
  \inferrule*[lab=Out-barb]{x \nameeq y}{{y}!\langle{Q}\rangle \vdash x}
  \and
  \inferrule*[lab=Par-barb]{\mbox{$P\vdash x$ or $Q\vdash x$}}{\binpar{P}{Q} \vdash x}
\end{mathpar}

\subsubsection{Contexts}

One of the principle advantages of computational calculi like the
$\pi$-calculus is a well-defined notion of context,
contextual-equivalence and a correlation between
contextual-equivalence and notions of bisimulation. The notion of
context allows the decomposition of a process into (sub-)process and
its syntactic environment, its context. Thus, a context may be
thought of as a process with a ``hole'' (written $\Box$) in it. The
application of a context $M$ to a process $P$, written $M[P]$, is
tantamount to filling the hole in $M$ with $P$. In this paper we do
not need the full weight of this theory, but do make use of the notion
of context in the proof the main theorem. 

\begin{mathpar}
  \inferrule* [lab=summation] {} {{M_{M},M_{N}} \bc \Box \;|\; x.M_{A} \;|\; M_{M}+M_{N}}
  \and
  \inferrule* [lab=agent] {} {{M_{A}} \bc (\vec{x})M_{P} \;| \; \clift{P_0,\ldots,M_{P},\ldots,P_N}}
  \and \\
  \inferrule* [lab=process] {} {{M_{P}} \bc M_{N} \;| \;P|M_{P} }
\end{mathpar} 

\begin{mathpar}
  \inferrule* [lab=sychronization] {} {M_{N} \bc \Box \;|\; x?M_{F} \;|\; x!M_{C}}
  \and
  \inferrule* [lab=abstraction] {} {{M_{F}} \bc (x)M_{P} }
  \and
  \inferrule* [lab=concretion] {} {{M_{C}} \bc \langle M_{P} \rangle }
  \and \\
  \inferrule* [lab=process] {} {{M_{P}} \bc M_{N} \;| \;P|M_{P} }
\end{mathpar}

\begin{definition}[contextual application] Given a context $M$, and
  process $P$, we define the \emph{contextual application}, $M[P] :=
  M\{P/\Box\}$. That is, the contextual application of M to P is the
  substitution of $P$ for $\Box$ in $M$.
\end{definition}

$\meaningof{-} : L \to \mathcal{P}(\pi)$

\begin{mathpar}
  \inferrule* [lab=collection] {} {\meaningof{true} = \pi, \and \meaningof{~E} = \pi \setminus \meaningof{E}, \and \meaningof{E_{1} \& E_{2}} = \meaningof{E_{1}} \cap \meaningof{E_{2}}}
\end{mathpar}

\begin{mathpar}
  \inferrule* [lab=structure] {} {\meaningof{0} = \{ P \in \pi | P \equiv 0 \}, \and \\ \meaningof{E_1 | E_2} = \{ P \in \pi | P \equiv P_{1} | P_{2}, P_{1} \in \meaningof{E_{1}}, P_{2} \in \meaningof{E_2}\} }
\end{mathpar}

\begin{mathpar}
 \inferrule* [lab=behavior] {} {\meaningof{\langle a?b \rangle E} = \{ P \in \pi | P \equiv Q | u?(y)P', \\ \and \\\\ \and \\ \;\;\; u \in \meaningof{a}, \forall z.P'\{z/y\} \in \meaningof{E\{z/b\}}\}, \and \\ \meaningof{a!E} = \{ P \in \pi | P \equiv Q | x!\langle P' \rangle, x \in \meaningof{a} P' \in \meaningof{E}\} }
\end{mathpar}

\begin{mathpar}
 \inferrule* [lab=nominal] {} {\meaningof{\quotep{E}} = \{ \quotep{P} \in \quotep{\pi} | P \in \meaningof{E} \}, \and \meaningof{\quotep{P}} = \{ \quotep{Q} \in \quotep{\pi} | P \equiv Q \} \and \\ \meaningof{@\quotep{E}} = \{ P \in \pi | P \equiv @x, x \in \meaningof{E} \}}
\end{mathpar}

\begin{eqnarray*}
  \\
  \meaningof{-} : TS \to ST
\end{eqnarray*}

\begin{eqnarray*}
  \\
  L : TS \to ST
\end{eqnarray*}

\begin{eqnarray*}
  \\
  P \models E \iff P \in \meaningof{E}
\end{eqnarray*}

\begin{eqnarray*}
  P \approx_{L} Q \iff \forall E \in L. P \models E \iff Q \models E
\end{eqnarray*}

\begin{eqnarray*}
  P \approx_{K} Q
\end{eqnarray*}

\begin{eqnarray*}
  P \approx Q
\end{eqnarray*}

$\approx_{K} = \approx = \approx_{L}$

\subsubsection{Contextual duality}

Note that contexts extend the quotation operation to a family of
operations from processes to names. Given a context, $M$, we can
define a \emph{nominal context}, $\quotep{M}$ by $\quotep{M}[P] :=
\quotep{M[P]}$. To foreshadow what is to come we observe that these
operations enjoy a duality with processes very much like the duality
between vectors and maps from vectors to scalars.

Further, because the calculus is essentially higher-order, we have a
correspondence between contexts and processes. More specifically,
given a name $x$ and a context $M$ we can construct $M^{*}_{x}$ such
that 

\begin{mathpar}
  M^{*}_{x} | \lift{x}{P} \red M[P]
\end{mathpar}

namely,

\begin{mathpar}
  M^{*}_{x} := x?(u).M[\dropn{u}]
\end{mathpar}

The dependence of $M^{*}_{x}$ on a name makes it an abstraction, 

\begin{mathpar}
  M^{*} := (x)x?(u).M[\dropn{u}]
\end{mathpar}

\subsection{Additional notation}

It will sometimes be convenient to denote the process a name
quotes. We already have the notation $x = \quotep{P}$, but it will be
convenient to introduce an alternate notation, $\procn{x}$, when we
want to emphasize the connection to the use of the name. Note that, by
virtue of name equivalence, $\quotep{\procn{x}} \nameeq x$; so, the
notation is consistent with previous definitions.

Further, because names have structure it is possible to effect
substitutions on the basis of that structure. This means we need to
upgrade our notation for substitutions, which we accomplish by
adapting comprehension notation. Thus,

\begin{mathpar}
  P\{ y / x : x \in S \}
\end{mathpar}

is interpreted to mean the process derived from P by replacing (in a
capture-avoiding manner) each occurrence of $x$ in $S$ by $y$. For example,

\begin{mathpar}
  P\{ \quotep{\procn{x}|\procn{x}} / x : x \in \freenames{P} \}
\end{mathpar}

will replace each (occurrence) of a free name $x$ in $P$ by
$\quotep{\procn{x}|\procn{x}}$.

Also, we will avail ourselves of the notation $x^{L}$ and $x^{R}$ to
denote injections of a name into disjoint copies of the name
space. There are numerous ways to accomplish this. One example can be
found in \cite{MeredithR05}. This notation overloads to vectors of
names: $\vec{x}^{\pi} := (x_{i}^{\pi} \; : \; 0 \leq i < |\vec{x}| )$ where $\pi \in \{L,R\}$.

We also use $P^{\Box} := P|\Box$.

In \cite{MeredithR05} an interpretation of the new operator is
given. It turns out that there are several possible interpretations
all enjoying the requisite algebraic properties of the operator (see
\cite{milner91polyadicpi}). We will therefore make liberal use of
$(\nu\; \vec{x})P$.

% subsection the_syntax_and_semantics_of_the_notation_system (end)   

\input{qm2pi.qmops} 

\input{qm2pi.sterngerlach} 

\input{qm2pi.metric} 

% section concurrent_process_calculi (end)

%\input{qm2pi.proofsketch}

% section proof sketch (end)

%\input{qm2pi.slviaknots} 

% section spatial logic via knots (end)

\input{qm2pi.conclusion}

% section conclusion (end)

%\input{qm2pi.dtcodes} 

% section wiring algorithm (end)

\input{qm2pi.ack} 

% section acknowledgments (end)

\newpage


\bibliographystyle{plain}   
\bibliography{../../biblios/main.bib}

\input{qm2pi.rhodetails}

\end{document}

 

%\documentclass[12pt]{llncs}
%\documentclass{jktr}

\usepackage[pdftex]{hyperref}                   
\usepackage {listings}
\usepackage {mathpartir}
\usepackage{bcprules}
%\usepackage{listings}
                       
\usepackage{graphicx} 
%\usepackage[margins=2.5cm,nohead,nofoot]{geometry}
%\usepackage{geometry}
\usepackage{amsfonts}
\usepackage{amstext}
\usepackage{latexsym}
\usepackage{amssymb}
\usepackage{color}


%\include{myPreamble}
\include{qm2pi.local} 

%\ifpdf
%\usepackage[pdftex]{graphicx}
%\else
%\usepackage{graphicx}
%\fi

 % \ifpdf
%  \usepackage{pdfsync}
%  \if


%\title{Brief Article}
%\author{David F. Snyder}
%\author{L.G. Meredith}

%\address{Dept. of Math., Texas State University--San Marcos, San Marcos, TX 78666}
       
\pagestyle{empty}


\begin{document}

\lstset{language=[Objective]Caml,frame=shadowbox}

\input{qm2pi.front}

% section front matter (end)

\input{qm2pi.intro} 
 
% section introduction (end)

% \input{qm2pi.knotations} 

% section notation (end)

\input{qm2pi.process.calculi} 

% section concurrent_process_calculi_and_spatial_logics_ (end)
    
%\input{qm2pi.knots2pi} 

%\input{qm2pi.trefoil} 

%\input{qm2pi.mainthm} 

% subsection basic_interpretation (end)

%\input{qm2pi.rho.presentation} 
\subsection{The syntax and semantics of the notation system}\label{sub:the_syntax_and_semantics_of_the_notation_system} % (fold)

We now summarize a technical presentation of the calculus that
embodies our theory of dynamics. The typical presentation of such a
calculus follows the style of giving generators and relations on
them. The grammar, below, describing term constructors, freely
generates the set of processes, $\Proc$. This set is then quotiented
by a relation known as structural congruence and it is over this set
that the notion of dynamics is expressed. This presentation is
essentially that of \cite{MeredithR05} with the addition of
polyadicity and summation. For readability we have relegated some of
the technical subtleties to an appendix.

\subsubsection{Process grammar}\label{subsub:process_grammar}

\begin{mathpar}
  \inferrule* [lab=synchronization] {} {{M} \bc \pzero \;|\; x?F \;|\; x!C }
  \and
  \inferrule* [lab=abstraction] {} {{F} \bc (x)P}
  \and
  \inferrule* [lab=concretion] {} {{C} \bc \langle Q \rangle}
  \and
  \inferrule* [lab=process] {} {{P,Q} \bc M \;| \;P|Q \;|\; @{x}}
  \and
  \inferrule* [lab=name] {} {{x} \bc \quotep{P}}
\end{mathpar} 

Note that $\vec{x}$ (resp. $\vec{P}$) denotes a vector of names
(resp. processes) of length $|\vec{x}|$ (resp. $|\vec{P}|$). We adopt
the following useful abbreviations.

\begin{mathpar}
   x?(\vec{y}).P := x.(\vec{y})P \and  x\clift{\vec{P}} := x.\clift{\vec{P}}
   \and x!(y) := \lift{x}{\dropn{y}}
   \and \Pi_{i=0}^{n-1}P_i := P_0 | \ldots | P_{n-1}
\end{mathpar}

\subsubsection{Structural congruence}

\paragraph{Free and bound names and alpha-equivalence.} At the
core of structural equivalence is alpha-equivalence which identifies
process that are the same up to a change of variable. Formally, we
recognize the distinction between free and bound names. The free names
of a process, $\freenames{P}$, may be calculated recursively as
follows:

\begin{mathpar}
\freenames{\pzero} := \emptyset
  \and \\
  \freenames{x?(y).P} := \{ x \} \cup (\freenames{P} \setminus \{ y \})
  \and 
  \freenames{x!\langle P \rangle} := \{ x \} \cup \{ P \} 
  \and \\
  \freenames{P|Q} := \freenames{P} \cup \freenames{Q}
  \and \\
  \freenames{@{x}} := \{ x \}
\end{mathpar}

$\pi$
$\quotep{\pi}$

$\freenames{-} : \pi \to \mathcal{P}(\quotep{\pi})$

\begin{eqnarray*}
  \freenames{\pzero} & := & \emptyset \\
  \freenames{x?(y).P} & := & \{ x \} \cup (\freenames{P} \setminus \{ y \}) \\
  \freenames{x!\langle P \rangle} & := & \{ x \} \cup \{ P \} \\
  \freenames{P|Q} & := & \freenames{P} \cup \freenames{Q} \\
  \freenames{\dropn{x}} & := & \{ x \}
\end{eqnarray*}

The bound names of a process, $\boundnames{P}$, are those names occurring in $P$
that are not free. For example, in $x?(y).0$, the name $x$ is free, while $y$ is bound.

\begin{mathpar}
  \inferrule* [lab=monoidal-laws] {} { P|Q \equiv Q|P \and P|0 \equiv P \and P|(Q|R) \equiv (P|Q)|R }
\end{mathpar}

\begin{mathpar}
  \inferrule* [lab=alpha-equivalence] {} { (x)P \equiv (y)P\{y/x\} \and y \not\in \freenames{P} }
\end{mathpar}

\begin{definition}
Then two processes, $P,Q$, are alpha-equivalent if $P = Q\{\vec{y}/\vec{x}\}$ for
some $\vec{x} \in \boundnames{Q},\vec{y} \in \boundnames{P}$, where $Q\{\vec{y}/\vec{x}\}$
denotes the capture-avoiding substitution of $\vec{y}$ for $\vec{x}$ in $Q$.
\end{definition}

\begin{definition}
  The {\em structural congruence} \cite{SangiorgiWalker} , $\equiv$,
  between processes is the least congruence containing
  alpha-equivalence, satisfying the abelian monoid laws
  (associativity, commutativity and $\pzero$ as identity) for parallel
  composition $|$ and for summation $+$.
\end{definition}

\subsection{Name equivalence}

We take name equivalence, written $\nameeq$, to be the smallest
equivalence relation generated by the following rules.

\begin{mathpar}
\inferrule*[lab=Quote-drop]
{ }
{ \quotep{@{x}} \nameeq x }

\inferrule*[lab=Struct-equiv]
{ P \scong Q }
{ \quotep{P} \nameeq \quotep{Q} }
\end{mathpar}

The astute reader will have noticed that the mutual recursion of names
and processes imposes a mutual recursion on alpha-equivalence and
structural equivalence via name-equivalence. Fortunately, all of this
works out pleasantly and we may calculate in the natural way, free of
concern. The reader interested in the details is referred to the
appendix \ref{appendix:rho_details}.

\subsection{Substitution}

We use $\Proc$ for the set of processes, $\QProc$ for the set of
names, and $\id{\{}\vec{y} / \vec{x} \id{\}}$ to denote partial maps,
$s : \QProc \rightarrow \QProc$. A map, $s$ lifts, uniquely, to a map
on process terms, $\widehat{s} : \Proc \rightarrow \Proc$ by the
following equations.

\begin{mathpar}
  (0) \psubstp{Q}{P} := 0 \\
  (R \juxtap S) \psubstp{Q}{P}
  :=    
  (R)\psubstp{Q}{P} \juxtap (S) \psubstp{Q}{P} \\
  (x?(y).R) \psubstp{Q}{P}    
  :=    
  (x)\substp{Q}{P} (z)\concat( (R \psubstn{z}{y}) \psubstp{Q}{P} ) \\
  (\lift{x}{R}) \psubstp{Q}{P}  
  :=
  \lift{(x)\substp{Q}{P}}{ R \psubstp{Q}{P} } \\
%   (\dropn{x})  \psubstp{Q}{P}       
%   := 
%   \left\{ 
%     \begin{array}{ccc} 
%       \dropn{\quotep{Q}} & & x \nameeq \quotep{P} \\
%       \dropn{x} & & otherwise \\
%     \end{array}
%   \right. 
  (\dropn{x})  \psubstp{Q}{P}       
  := 
  \left\{ 
    \begin{array}{ccc} 
      Q & & x \nameeq \quotep{P} \\
      \dropn{x} & & otherwise \\
    \end{array}
  \right.
\end{mathpar}
 

where

\begin{eqnarray}
  (x)\id{\{} \lpquote Q \rpquote / \lpquote P \rpquote \id{\}}            = 
  \left\{ 
    \begin{array}{ccc}
      \lpquote Q \rpquote & & x \nameeq \lpquote P \rpquote \\
      x & & otherwise \\
    \end{array}
  \right. \nonumber
\end{eqnarray}

and $z$ is chosen distinct from $\quotep{P}$, $\quotep{Q}$, the free
names in $Q$, and all the names in $R$. Our $\alpha$-equivalence will
be built in the standard way from this substitution.

\begin{remark}\label{rem:no_self_referential_names}
  One consequence of these definitions is that $\forall P. \quotep{P}
  \not\in \freenames{P}$.
\end{remark}

\subsection{ Dynamic quote: an example }

Anticipating something of what's to come, consider applying the
substitution, $\widehat{\id{\{}u / z \id{\}}}$, to the following pair
of processes, $\lift{w}{y!(z)}$ and $w[ \lpquote y!(z) \rpquote ]$.

\begin{eqnarray}
	\lift{w}{y!(z)}\widehat{\id{\{}u / z \id{\}}}
		& = &
		\lift{w}{y!(u)} \nonumber\\
	w[ \lpquote y!(z) \rpquote ] \widehat{ \id{\{}u / z \id{\}} }
		& = &
		w[ \lpquote y!(z) \rpquote ] \nonumber
\end{eqnarray}

Because the body of the process between quotes is impervious to
substitution, we get radically different answers. In fact, by
examining the first process in an input context,
e.g. $x?(z).\lift{w}{y!(z)}$, we see that the process under the lift
operator may be shaped by prefixed inputs binding a name inside it. In
this sense, the lift operator will be seen as a way to dynamically
construct processes before reifying them as names.

Finally equipped with these standard features we can present the
dynamics of the calculus.

\subsubsection{Operational semantics} 

Finally, we introduce the computational dynamics. What marks these
algebras as distinct from other more traditionally studied algebraic
structures, e.g. vector spaces or polynomial rings, is the manner in
which dynamics is captured. In traditional structures, dynamics is typically
expressed through morphisms between such structures, as in linear maps
between vector spaces or morphisms between rings. In algebras
associated with the semantics of computation, the dynamics is
expressed as part of the algebraic structure itself, through a
reduction reduction relation typically denoted by $\red$. Below, we
give a recursive presentation of this relation for the calculus used
in the encoding.

$\red \subseteq \pi \times \pi$
$\red : \pi \to \mathcal{P}(\pi)$

\begin{mathpar}
  \inferrule* [lab=Comm] { \textsf{match}( x_{src}, x_{trgt} ) } { x_{trgt}?(y)P \; | \; x_{src}!\langle {Q} \rangle \red P\{\quotep{Q}/y}\} }
  \and \\
  \inferrule* [lab=Par] {{P} \red {P}'} {{{P} | {Q}} \red {{P}' | {Q}}}
  \and
  \inferrule* [lab=Equiv]{{{P} \scong {P}'} \andalso {{P}' \red {Q}'} \andalso {{Q}' \scong {Q}}}{{P} \red {Q}}
\end{mathpar}

\begin{eqnarray*}
  match_{\equiv} (\quotep{P},\quotep{Q}) & := & P \equiv Q \\
  match_{\dagger}(\quotep{P},\quotep{Q}) & := & \forall R. P|Q \red^{*} R => R \red^{*} 0 \\
  match_{K}(\quotep{P},\quotep{Q}) & := & K \mbox{ for some context } K
\end{eqnarray*}

$u?(x)P | u!\langle Q \rangle \red P\{\quotep{Q}/x\}$

%We write $\wred$ for $\red^*$, and $P\red$ if $\exists Q $ such that $ P \red Q$.
We write $P\red$ if $\exists Q $ such that $ P \red Q$ and $P\not\red$, otherwise.

\section{Replication}

As mentioned before, it is known that replication (and hence
recursion) can be implemented in a higher-order process algebra
\cite{SangiorgiWalker}. As our first example of calculation with the
machinery thus far presented we give the construction explicitly in
the {\rhoc}.

\begin{eqnarray}
	D_{x} & := & \prefix{x}{y}{(\binpar{\outputp{x}{y}}{@{y}})} \nonumber\\
	\bangp_{x}{P} & := & \binpar{{x}!\langle{\binpar{D_{x}}{P}}\rangle}{D_{x}} \nonumber
\end{eqnarray}

\begin{eqnarray}
	\bangp_{x}{P} & & \nonumber\\
	=
	& {x}!\langle{(\prefix{x}{y}{(\outputp{x}{y} | @{y})) | P}}\rangle 
	      | \prefix{x}{y}{(\outputp{x}{y} | @{y})} & \nonumber\\
	\red
	& (\outputp{x}{y} | @{y})\substn{\quotep{(\prefix{x}{y}{(@{y} | \outputp{x}{y})) | P}}}{y} & \nonumber\\
	=
	& \outputp{x}{\quotep{(\prefix{x}{y}{(\outputp{x}{y} | @{y})) | P}}}
	  | {(\prefix{x}{y}{(\outputp{x}{y} | @{y})) | P}} & \nonumber\\
	\red
	& \ldots & \nonumber\\
	\red^*
	& P | P | \ldots & \nonumber
\end{eqnarray}

Of course, this encoding, as an implementation, runs away, unfolding
$\bangp{P}$ eagerly. A lazier and more implementable replication
operator, restricted to input-guarded processes, may be obtained as follows.

\begin{eqnarray}
\bangp{\prefix{u}{v}{P}} 
	:= 
	\binpar{\lift{x}{\prefix{u}{v}{(\binpar{D(x)}{P})}}}{D(x)} \nonumber
\end{eqnarray}

\begin{remark}
  Note that the lazier definition still does not deal with summation
  or mixed summation (i.e. sums over input and output). The reader is
  invited to construct definitions of replication that deal with these
  features. 

  Further, the definitions are parameterized in a name, $x$. Can you,
  gentle reader, make a definition that eliminates this parameter and
  guarantees no accidental interaction between the replication
  machinery and the process being replicated -- i.e. no accidental
  sharing of names used by the process to get its work done and the
  name(s) used by the replication to effect copying. This latter
  revision of the definition of replication is crucial to obtaining
  the expected identity $!!P \sim !P$.
\end{remark}

\begin{remark}\label{rem:paradoxical_combinator}
  The reader familiar with the lambda calculus will have noticed the
  similarity between $D$ and the paradoxical combinator.

  [Ed. note: the existence of this seems to suggest we have to be more
  restrictive on the set of processes and names we admit if we are to
  support no-cloning.]
\end{remark}

\subsubsection{Bisimulation}

The computational dynamics gives rise to another kind of equivalence,
the equivalence of computational behavior. As previously mentioned
this is typically captured \emph{via} some form of bisimulation.

% The notion we use in this paper is weak barbed bisimulation
% \cite{milner91polyadicpi}.

The notion we use in this paper is derived from weak barbed
bisimulation \cite{milner91polyadicpi}. 

\begin{definition}
An \emph{observation relation}, $\downarrow_{\mathcal N}$, over a set
of names, $\mathcal N$, is the smallest relation satisfying the rules
below.

\infrule[Out-barb]{y \in {\mathcal N}, \; x \nameeq y}
		  {\outputp{x}{v} \downarrow_{\mathcal N} x}
\infrule[Par-barb]{\mbox{$P\downarrow_{\mathcal N} x$ or $Q\downarrow_{\mathcal N} x$}}
		  {\binpar{P}{Q} \downarrow_{\mathcal N} x}

We write $P \Downarrow_{\mathcal N} x$ if there is $Q$ such that 
$P \wred Q$ and $Q \downarrow_{\mathcal N} x$.
\end{definition}

\begin{definition}
%\label{def.bbisim}
An  ${\mathcal N}$-\emph{barbed bisimulation} over a set of names, ${\mathcal N}$, is a symmetric binary relation 
${\mathcal S}_{\mathcal N}$ between agents such that $P\rel{S}_{\mathcal N}Q$ implies:
\begin{enumerate}
\item If $P \red P'$ then $Q \wred Q'$ and $P'\rel{S}_{\mathcal N} Q'$.
\item If $P\downarrow_{\mathcal N} x$, then $Q\Downarrow_{\mathcal N} x$.
\end{enumerate}
$P$ is ${\mathcal N}$-barbed bisimilar to $Q$, written
$P \wbbisim_{\mathcal N} Q$, if $P \rel{S}_{\mathcal N} Q$ for some ${\mathcal N}$-barbed bisimulation ${\mathcal S}_{\mathcal N}$.
\end{definition}

$\mathcal{R} \subseteq \pi \times \pi$

$P \mathcal{R} Q => \forall P'. P \red P' \Rightarrow \exists Q'. Q \red Q', P' \mathcal{R} Q'$

$P \vdash x \Rightarrow Q \vdash x$

\begin{mathpar}
  \inferrule*[lab=Out-barb]{x \nameeq y}{{y}!\langle{Q}\rangle \vdash x}
  \and
  \inferrule*[lab=Par-barb]{\mbox{$P\vdash x$ or $Q\vdash x$}}{\binpar{P}{Q} \vdash x}
\end{mathpar}

\subsubsection{Contexts}

One of the principle advantages of computational calculi like the
$\pi$-calculus is a well-defined notion of context,
contextual-equivalence and a correlation between
contextual-equivalence and notions of bisimulation. The notion of
context allows the decomposition of a process into (sub-)process and
its syntactic environment, its context. Thus, a context may be
thought of as a process with a ``hole'' (written $\Box$) in it. The
application of a context $M$ to a process $P$, written $M[P]$, is
tantamount to filling the hole in $M$ with $P$. In this paper we do
not need the full weight of this theory, but do make use of the notion
of context in the proof the main theorem. 

\begin{mathpar}
  \inferrule* [lab=summation] {} {{M_{M},M_{N}} \bc \Box \;|\; x.M_{A} \;|\; M_{M}+M_{N}}
  \and
  \inferrule* [lab=agent] {} {{M_{A}} \bc (\vec{x})M_{P} \;| \; \clift{P_0,\ldots,M_{P},\ldots,P_N}}
  \and \\
  \inferrule* [lab=process] {} {{M_{P}} \bc M_{N} \;| \;P|M_{P} }
\end{mathpar} 

\begin{mathpar}
  \inferrule* [lab=sychronization] {} {M_{N} \bc \Box \;|\; x?M_{F} \;|\; x!M_{C}}
  \and
  \inferrule* [lab=abstraction] {} {{M_{F}} \bc (x)M_{P} }
  \and
  \inferrule* [lab=concretion] {} {{M_{C}} \bc \langle M_{P} \rangle }
  \and \\
  \inferrule* [lab=process] {} {{M_{P}} \bc M_{N} \;| \;P|M_{P} }
\end{mathpar}

\begin{definition}[contextual application] Given a context $M$, and
  process $P$, we define the \emph{contextual application}, $M[P] :=
  M\{P/\Box\}$. That is, the contextual application of M to P is the
  substitution of $P$ for $\Box$ in $M$.
\end{definition}

$\meaningof{-} : L \to \mathcal{P}(\pi)$

\begin{mathpar}
  \inferrule* [lab=collection] {} {\meaningof{true} = \pi, \and \meaningof{~E} = \pi \setminus \meaningof{E}, \and \meaningof{E_{1} \& E_{2}} = \meaningof{E_{1}} \cap \meaningof{E_{2}}}
\end{mathpar}

\begin{mathpar}
  \inferrule* [lab=structure] {} {\meaningof{0} = \{ P \in \pi | P \equiv 0 \}, \and \\ \meaningof{E_1 | E_2} = \{ P \in \pi | P \equiv P_{1} | P_{2}, P_{1} \in \meaningof{E_{1}}, P_{2} \in \meaningof{E_2}\} }
\end{mathpar}

\begin{mathpar}
 \inferrule* [lab=behavior] {} {\meaningof{\langle a?b \rangle E} = \{ P \in \pi | P \equiv Q | u?(y)P', \\ \and \\\\ \and \\ \;\;\; u \in \meaningof{a}, \forall z.P'\{z/y\} \in \meaningof{E\{z/b\}}\}, \and \\ \meaningof{a!E} = \{ P \in \pi | P \equiv Q | x!\langle P' \rangle, x \in \meaningof{a} P' \in \meaningof{E}\} }
\end{mathpar}

\begin{mathpar}
 \inferrule* [lab=nominal] {} {\meaningof{\quotep{E}} = \{ \quotep{P} \in \quotep{\pi} | P \in \meaningof{E} \}, \and \meaningof{\quotep{P}} = \{ \quotep{Q} \in \quotep{\pi} | P \equiv Q \} \and \\ \meaningof{@\quotep{E}} = \{ P \in \pi | P \equiv @x, x \in \meaningof{E} \}}
\end{mathpar}

\begin{eqnarray*}
  \\
  \meaningof{-} : TS \to ST
\end{eqnarray*}

\begin{eqnarray*}
  \\
  L : TS \to ST
\end{eqnarray*}

\begin{eqnarray*}
  \\
  P \models E \iff P \in \meaningof{E}
\end{eqnarray*}

\begin{eqnarray*}
  P \approx_{L} Q \iff \forall E \in L. P \models E \iff Q \models E
\end{eqnarray*}

\begin{eqnarray*}
  P \approx_{K} Q
\end{eqnarray*}

\begin{eqnarray*}
  P \approx Q
\end{eqnarray*}

$\approx_{K} = \approx = \approx_{L}$

\subsubsection{Contextual duality}

Note that contexts extend the quotation operation to a family of
operations from processes to names. Given a context, $M$, we can
define a \emph{nominal context}, $\quotep{M}$ by $\quotep{M}[P] :=
\quotep{M[P]}$. To foreshadow what is to come we observe that these
operations enjoy a duality with processes very much like the duality
between vectors and maps from vectors to scalars.

Further, because the calculus is essentially higher-order, we have a
correspondence between contexts and processes. More specifically,
given a name $x$ and a context $M$ we can construct $M^{*}_{x}$ such
that 

\begin{mathpar}
  M^{*}_{x} | \lift{x}{P} \red M[P]
\end{mathpar}

namely,

\begin{mathpar}
  M^{*}_{x} := x?(u).M[\dropn{u}]
\end{mathpar}

The dependence of $M^{*}_{x}$ on a name makes it an abstraction, 

\begin{mathpar}
  M^{*} := (x)x?(u).M[\dropn{u}]
\end{mathpar}

\subsection{Additional notation}

It will sometimes be convenient to denote the process a name
quotes. We already have the notation $x = \quotep{P}$, but it will be
convenient to introduce an alternate notation, $\procn{x}$, when we
want to emphasize the connection to the use of the name. Note that, by
virtue of name equivalence, $\quotep{\procn{x}} \nameeq x$; so, the
notation is consistent with previous definitions.

Further, because names have structure it is possible to effect
substitutions on the basis of that structure. This means we need to
upgrade our notation for substitutions, which we accomplish by
adapting comprehension notation. Thus,

\begin{mathpar}
  P\{ y / x : x \in S \}
\end{mathpar}

is interpreted to mean the process derived from P by replacing (in a
capture-avoiding manner) each occurrence of $x$ in $S$ by $y$. For example,

\begin{mathpar}
  P\{ \quotep{\procn{x}|\procn{x}} / x : x \in \freenames{P} \}
\end{mathpar}

will replace each (occurrence) of a free name $x$ in $P$ by
$\quotep{\procn{x}|\procn{x}}$.

Also, we will avail ourselves of the notation $x^{L}$ and $x^{R}$ to
denote injections of a name into disjoint copies of the name
space. There are numerous ways to accomplish this. One example can be
found in \cite{MeredithR05}. This notation overloads to vectors of
names: $\vec{x}^{\pi} := (x_{i}^{\pi} \; : \; 0 \leq i < |\vec{x}| )$ where $\pi \in \{L,R\}$.

We also use $P^{\Box} := P|\Box$.

In \cite{MeredithR05} an interpretation of the new operator is
given. It turns out that there are several possible interpretations
all enjoying the requisite algebraic properties of the operator (see
\cite{milner91polyadicpi}). We will therefore make liberal use of
$(\nu\; \vec{x})P$.

% subsection the_syntax_and_semantics_of_the_notation_system (end)   

\input{qm2pi.qmops} 

\input{qm2pi.sterngerlach} 

\input{qm2pi.metric} 

% section concurrent_process_calculi (end)

%\input{qm2pi.proofsketch}

% section proof sketch (end)

%\input{qm2pi.slviaknots} 

% section spatial logic via knots (end)

\input{qm2pi.conclusion}

% section conclusion (end)

%\input{qm2pi.dtcodes} 

% section wiring algorithm (end)

\input{qm2pi.ack} 

% section acknowledgments (end)

\newpage


\bibliographystyle{plain}   
\bibliography{../../biblios/main.bib}

\input{qm2pi.rhodetails}

\end{document}

 

%\documentclass[12pt]{llncs}
%\documentclass{jktr}

\usepackage[pdftex]{hyperref}                   
\usepackage {listings}
\usepackage {mathpartir}
\usepackage{bcprules}
%\usepackage{listings}
                       
\usepackage{graphicx} 
%\usepackage[margins=2.5cm,nohead,nofoot]{geometry}
%\usepackage{geometry}
\usepackage{amsfonts}
\usepackage{amstext}
\usepackage{latexsym}
\usepackage{amssymb}
\usepackage{color}


%\include{myPreamble}
\include{qm2pi.local} 

%\ifpdf
%\usepackage[pdftex]{graphicx}
%\else
%\usepackage{graphicx}
%\fi

 % \ifpdf
%  \usepackage{pdfsync}
%  \if


%\title{Brief Article}
%\author{David F. Snyder}
%\author{L.G. Meredith}

%\address{Dept. of Math., Texas State University--San Marcos, San Marcos, TX 78666}
       
\pagestyle{empty}


\begin{document}

\lstset{language=[Objective]Caml,frame=shadowbox}

\input{qm2pi.front}

% section front matter (end)

\input{qm2pi.intro} 
 
% section introduction (end)

% \input{qm2pi.knotations} 

% section notation (end)

\input{qm2pi.process.calculi} 

% section concurrent_process_calculi_and_spatial_logics_ (end)
    
%\input{qm2pi.knots2pi} 

%\input{qm2pi.trefoil} 

%\input{qm2pi.mainthm} 

% subsection basic_interpretation (end)

%\input{qm2pi.rho.presentation} 
\subsection{The syntax and semantics of the notation system}\label{sub:the_syntax_and_semantics_of_the_notation_system} % (fold)

We now summarize a technical presentation of the calculus that
embodies our theory of dynamics. The typical presentation of such a
calculus follows the style of giving generators and relations on
them. The grammar, below, describing term constructors, freely
generates the set of processes, $\Proc$. This set is then quotiented
by a relation known as structural congruence and it is over this set
that the notion of dynamics is expressed. This presentation is
essentially that of \cite{MeredithR05} with the addition of
polyadicity and summation. For readability we have relegated some of
the technical subtleties to an appendix.

\subsubsection{Process grammar}\label{subsub:process_grammar}

\begin{mathpar}
  \inferrule* [lab=synchronization] {} {{M} \bc \pzero \;|\; x?F \;|\; x!C }
  \and
  \inferrule* [lab=abstraction] {} {{F} \bc (x)P}
  \and
  \inferrule* [lab=concretion] {} {{C} \bc \langle Q \rangle}
  \and
  \inferrule* [lab=process] {} {{P,Q} \bc M \;| \;P|Q \;|\; @{x}}
  \and
  \inferrule* [lab=name] {} {{x} \bc \quotep{P}}
\end{mathpar} 

Note that $\vec{x}$ (resp. $\vec{P}$) denotes a vector of names
(resp. processes) of length $|\vec{x}|$ (resp. $|\vec{P}|$). We adopt
the following useful abbreviations.

\begin{mathpar}
   x?(\vec{y}).P := x.(\vec{y})P \and  x\clift{\vec{P}} := x.\clift{\vec{P}}
   \and x!(y) := \lift{x}{\dropn{y}}
   \and \Pi_{i=0}^{n-1}P_i := P_0 | \ldots | P_{n-1}
\end{mathpar}

\subsubsection{Structural congruence}

\paragraph{Free and bound names and alpha-equivalence.} At the
core of structural equivalence is alpha-equivalence which identifies
process that are the same up to a change of variable. Formally, we
recognize the distinction between free and bound names. The free names
of a process, $\freenames{P}$, may be calculated recursively as
follows:

\begin{mathpar}
\freenames{\pzero} := \emptyset
  \and \\
  \freenames{x?(y).P} := \{ x \} \cup (\freenames{P} \setminus \{ y \})
  \and 
  \freenames{x!\langle P \rangle} := \{ x \} \cup \{ P \} 
  \and \\
  \freenames{P|Q} := \freenames{P} \cup \freenames{Q}
  \and \\
  \freenames{@{x}} := \{ x \}
\end{mathpar}

$\pi$
$\quotep{\pi}$

$\freenames{-} : \pi \to \mathcal{P}(\quotep{\pi})$

\begin{eqnarray*}
  \freenames{\pzero} & := & \emptyset \\
  \freenames{x?(y).P} & := & \{ x \} \cup (\freenames{P} \setminus \{ y \}) \\
  \freenames{x!\langle P \rangle} & := & \{ x \} \cup \{ P \} \\
  \freenames{P|Q} & := & \freenames{P} \cup \freenames{Q} \\
  \freenames{\dropn{x}} & := & \{ x \}
\end{eqnarray*}

The bound names of a process, $\boundnames{P}$, are those names occurring in $P$
that are not free. For example, in $x?(y).0$, the name $x$ is free, while $y$ is bound.

\begin{mathpar}
  \inferrule* [lab=monoidal-laws] {} { P|Q \equiv Q|P \and P|0 \equiv P \and P|(Q|R) \equiv (P|Q)|R }
\end{mathpar}

\begin{mathpar}
  \inferrule* [lab=alpha-equivalence] {} { (x)P \equiv (y)P\{y/x\} \and y \not\in \freenames{P} }
\end{mathpar}

\begin{definition}
Then two processes, $P,Q$, are alpha-equivalent if $P = Q\{\vec{y}/\vec{x}\}$ for
some $\vec{x} \in \boundnames{Q},\vec{y} \in \boundnames{P}$, where $Q\{\vec{y}/\vec{x}\}$
denotes the capture-avoiding substitution of $\vec{y}$ for $\vec{x}$ in $Q$.
\end{definition}

\begin{definition}
  The {\em structural congruence} \cite{SangiorgiWalker} , $\equiv$,
  between processes is the least congruence containing
  alpha-equivalence, satisfying the abelian monoid laws
  (associativity, commutativity and $\pzero$ as identity) for parallel
  composition $|$ and for summation $+$.
\end{definition}

\subsection{Name equivalence}

We take name equivalence, written $\nameeq$, to be the smallest
equivalence relation generated by the following rules.

\begin{mathpar}
\inferrule*[lab=Quote-drop]
{ }
{ \quotep{@{x}} \nameeq x }

\inferrule*[lab=Struct-equiv]
{ P \scong Q }
{ \quotep{P} \nameeq \quotep{Q} }
\end{mathpar}

The astute reader will have noticed that the mutual recursion of names
and processes imposes a mutual recursion on alpha-equivalence and
structural equivalence via name-equivalence. Fortunately, all of this
works out pleasantly and we may calculate in the natural way, free of
concern. The reader interested in the details is referred to the
appendix \ref{appendix:rho_details}.

\subsection{Substitution}

We use $\Proc$ for the set of processes, $\QProc$ for the set of
names, and $\id{\{}\vec{y} / \vec{x} \id{\}}$ to denote partial maps,
$s : \QProc \rightarrow \QProc$. A map, $s$ lifts, uniquely, to a map
on process terms, $\widehat{s} : \Proc \rightarrow \Proc$ by the
following equations.

\begin{mathpar}
  (0) \psubstp{Q}{P} := 0 \\
  (R \juxtap S) \psubstp{Q}{P}
  :=    
  (R)\psubstp{Q}{P} \juxtap (S) \psubstp{Q}{P} \\
  (x?(y).R) \psubstp{Q}{P}    
  :=    
  (x)\substp{Q}{P} (z)\concat( (R \psubstn{z}{y}) \psubstp{Q}{P} ) \\
  (\lift{x}{R}) \psubstp{Q}{P}  
  :=
  \lift{(x)\substp{Q}{P}}{ R \psubstp{Q}{P} } \\
%   (\dropn{x})  \psubstp{Q}{P}       
%   := 
%   \left\{ 
%     \begin{array}{ccc} 
%       \dropn{\quotep{Q}} & & x \nameeq \quotep{P} \\
%       \dropn{x} & & otherwise \\
%     \end{array}
%   \right. 
  (\dropn{x})  \psubstp{Q}{P}       
  := 
  \left\{ 
    \begin{array}{ccc} 
      Q & & x \nameeq \quotep{P} \\
      \dropn{x} & & otherwise \\
    \end{array}
  \right.
\end{mathpar}
 

where

\begin{eqnarray}
  (x)\id{\{} \lpquote Q \rpquote / \lpquote P \rpquote \id{\}}            = 
  \left\{ 
    \begin{array}{ccc}
      \lpquote Q \rpquote & & x \nameeq \lpquote P \rpquote \\
      x & & otherwise \\
    \end{array}
  \right. \nonumber
\end{eqnarray}

and $z$ is chosen distinct from $\quotep{P}$, $\quotep{Q}$, the free
names in $Q$, and all the names in $R$. Our $\alpha$-equivalence will
be built in the standard way from this substitution.

\begin{remark}\label{rem:no_self_referential_names}
  One consequence of these definitions is that $\forall P. \quotep{P}
  \not\in \freenames{P}$.
\end{remark}

\subsection{ Dynamic quote: an example }

Anticipating something of what's to come, consider applying the
substitution, $\widehat{\id{\{}u / z \id{\}}}$, to the following pair
of processes, $\lift{w}{y!(z)}$ and $w[ \lpquote y!(z) \rpquote ]$.

\begin{eqnarray}
	\lift{w}{y!(z)}\widehat{\id{\{}u / z \id{\}}}
		& = &
		\lift{w}{y!(u)} \nonumber\\
	w[ \lpquote y!(z) \rpquote ] \widehat{ \id{\{}u / z \id{\}} }
		& = &
		w[ \lpquote y!(z) \rpquote ] \nonumber
\end{eqnarray}

Because the body of the process between quotes is impervious to
substitution, we get radically different answers. In fact, by
examining the first process in an input context,
e.g. $x?(z).\lift{w}{y!(z)}$, we see that the process under the lift
operator may be shaped by prefixed inputs binding a name inside it. In
this sense, the lift operator will be seen as a way to dynamically
construct processes before reifying them as names.

Finally equipped with these standard features we can present the
dynamics of the calculus.

\subsubsection{Operational semantics} 

Finally, we introduce the computational dynamics. What marks these
algebras as distinct from other more traditionally studied algebraic
structures, e.g. vector spaces or polynomial rings, is the manner in
which dynamics is captured. In traditional structures, dynamics is typically
expressed through morphisms between such structures, as in linear maps
between vector spaces or morphisms between rings. In algebras
associated with the semantics of computation, the dynamics is
expressed as part of the algebraic structure itself, through a
reduction reduction relation typically denoted by $\red$. Below, we
give a recursive presentation of this relation for the calculus used
in the encoding.

$\red \subseteq \pi \times \pi$
$\red : \pi \to \mathcal{P}(\pi)$

\begin{mathpar}
  \inferrule* [lab=Comm] { \textsf{match}( x_{src}, x_{trgt} ) } { x_{trgt}?(y)P \; | \; x_{src}!\langle {Q} \rangle \red P\{\quotep{Q}/y}\} }
  \and \\
  \inferrule* [lab=Par] {{P} \red {P}'} {{{P} | {Q}} \red {{P}' | {Q}}}
  \and
  \inferrule* [lab=Equiv]{{{P} \scong {P}'} \andalso {{P}' \red {Q}'} \andalso {{Q}' \scong {Q}}}{{P} \red {Q}}
\end{mathpar}

\begin{eqnarray*}
  match_{\equiv} (\quotep{P},\quotep{Q}) & := & P \equiv Q \\
  match_{\dagger}(\quotep{P},\quotep{Q}) & := & \forall R. P|Q \red^{*} R => R \red^{*} 0 \\
  match_{K}(\quotep{P},\quotep{Q}) & := & K \mbox{ for some context } K
\end{eqnarray*}

$u?(x)P | u!\langle Q \rangle \red P\{\quotep{Q}/x\}$

%We write $\wred$ for $\red^*$, and $P\red$ if $\exists Q $ such that $ P \red Q$.
We write $P\red$ if $\exists Q $ such that $ P \red Q$ and $P\not\red$, otherwise.

\section{Replication}

As mentioned before, it is known that replication (and hence
recursion) can be implemented in a higher-order process algebra
\cite{SangiorgiWalker}. As our first example of calculation with the
machinery thus far presented we give the construction explicitly in
the {\rhoc}.

\begin{eqnarray}
	D_{x} & := & \prefix{x}{y}{(\binpar{\outputp{x}{y}}{@{y}})} \nonumber\\
	\bangp_{x}{P} & := & \binpar{{x}!\langle{\binpar{D_{x}}{P}}\rangle}{D_{x}} \nonumber
\end{eqnarray}

\begin{eqnarray}
	\bangp_{x}{P} & & \nonumber\\
	=
	& {x}!\langle{(\prefix{x}{y}{(\outputp{x}{y} | @{y})) | P}}\rangle 
	      | \prefix{x}{y}{(\outputp{x}{y} | @{y})} & \nonumber\\
	\red
	& (\outputp{x}{y} | @{y})\substn{\quotep{(\prefix{x}{y}{(@{y} | \outputp{x}{y})) | P}}}{y} & \nonumber\\
	=
	& \outputp{x}{\quotep{(\prefix{x}{y}{(\outputp{x}{y} | @{y})) | P}}}
	  | {(\prefix{x}{y}{(\outputp{x}{y} | @{y})) | P}} & \nonumber\\
	\red
	& \ldots & \nonumber\\
	\red^*
	& P | P | \ldots & \nonumber
\end{eqnarray}

Of course, this encoding, as an implementation, runs away, unfolding
$\bangp{P}$ eagerly. A lazier and more implementable replication
operator, restricted to input-guarded processes, may be obtained as follows.

\begin{eqnarray}
\bangp{\prefix{u}{v}{P}} 
	:= 
	\binpar{\lift{x}{\prefix{u}{v}{(\binpar{D(x)}{P})}}}{D(x)} \nonumber
\end{eqnarray}

\begin{remark}
  Note that the lazier definition still does not deal with summation
  or mixed summation (i.e. sums over input and output). The reader is
  invited to construct definitions of replication that deal with these
  features. 

  Further, the definitions are parameterized in a name, $x$. Can you,
  gentle reader, make a definition that eliminates this parameter and
  guarantees no accidental interaction between the replication
  machinery and the process being replicated -- i.e. no accidental
  sharing of names used by the process to get its work done and the
  name(s) used by the replication to effect copying. This latter
  revision of the definition of replication is crucial to obtaining
  the expected identity $!!P \sim !P$.
\end{remark}

\begin{remark}\label{rem:paradoxical_combinator}
  The reader familiar with the lambda calculus will have noticed the
  similarity between $D$ and the paradoxical combinator.

  [Ed. note: the existence of this seems to suggest we have to be more
  restrictive on the set of processes and names we admit if we are to
  support no-cloning.]
\end{remark}

\subsubsection{Bisimulation}

The computational dynamics gives rise to another kind of equivalence,
the equivalence of computational behavior. As previously mentioned
this is typically captured \emph{via} some form of bisimulation.

% The notion we use in this paper is weak barbed bisimulation
% \cite{milner91polyadicpi}.

The notion we use in this paper is derived from weak barbed
bisimulation \cite{milner91polyadicpi}. 

\begin{definition}
An \emph{observation relation}, $\downarrow_{\mathcal N}$, over a set
of names, $\mathcal N$, is the smallest relation satisfying the rules
below.

\infrule[Out-barb]{y \in {\mathcal N}, \; x \nameeq y}
		  {\outputp{x}{v} \downarrow_{\mathcal N} x}
\infrule[Par-barb]{\mbox{$P\downarrow_{\mathcal N} x$ or $Q\downarrow_{\mathcal N} x$}}
		  {\binpar{P}{Q} \downarrow_{\mathcal N} x}

We write $P \Downarrow_{\mathcal N} x$ if there is $Q$ such that 
$P \wred Q$ and $Q \downarrow_{\mathcal N} x$.
\end{definition}

\begin{definition}
%\label{def.bbisim}
An  ${\mathcal N}$-\emph{barbed bisimulation} over a set of names, ${\mathcal N}$, is a symmetric binary relation 
${\mathcal S}_{\mathcal N}$ between agents such that $P\rel{S}_{\mathcal N}Q$ implies:
\begin{enumerate}
\item If $P \red P'$ then $Q \wred Q'$ and $P'\rel{S}_{\mathcal N} Q'$.
\item If $P\downarrow_{\mathcal N} x$, then $Q\Downarrow_{\mathcal N} x$.
\end{enumerate}
$P$ is ${\mathcal N}$-barbed bisimilar to $Q$, written
$P \wbbisim_{\mathcal N} Q$, if $P \rel{S}_{\mathcal N} Q$ for some ${\mathcal N}$-barbed bisimulation ${\mathcal S}_{\mathcal N}$.
\end{definition}

$\mathcal{R} \subseteq \pi \times \pi$

$P \mathcal{R} Q => \forall P'. P \red P' \Rightarrow \exists Q'. Q \red Q', P' \mathcal{R} Q'$

$P \vdash x \Rightarrow Q \vdash x$

\begin{mathpar}
  \inferrule*[lab=Out-barb]{x \nameeq y}{{y}!\langle{Q}\rangle \vdash x}
  \and
  \inferrule*[lab=Par-barb]{\mbox{$P\vdash x$ or $Q\vdash x$}}{\binpar{P}{Q} \vdash x}
\end{mathpar}

\subsubsection{Contexts}

One of the principle advantages of computational calculi like the
$\pi$-calculus is a well-defined notion of context,
contextual-equivalence and a correlation between
contextual-equivalence and notions of bisimulation. The notion of
context allows the decomposition of a process into (sub-)process and
its syntactic environment, its context. Thus, a context may be
thought of as a process with a ``hole'' (written $\Box$) in it. The
application of a context $M$ to a process $P$, written $M[P]$, is
tantamount to filling the hole in $M$ with $P$. In this paper we do
not need the full weight of this theory, but do make use of the notion
of context in the proof the main theorem. 

\begin{mathpar}
  \inferrule* [lab=summation] {} {{M_{M},M_{N}} \bc \Box \;|\; x.M_{A} \;|\; M_{M}+M_{N}}
  \and
  \inferrule* [lab=agent] {} {{M_{A}} \bc (\vec{x})M_{P} \;| \; \clift{P_0,\ldots,M_{P},\ldots,P_N}}
  \and \\
  \inferrule* [lab=process] {} {{M_{P}} \bc M_{N} \;| \;P|M_{P} }
\end{mathpar} 

\begin{mathpar}
  \inferrule* [lab=sychronization] {} {M_{N} \bc \Box \;|\; x?M_{F} \;|\; x!M_{C}}
  \and
  \inferrule* [lab=abstraction] {} {{M_{F}} \bc (x)M_{P} }
  \and
  \inferrule* [lab=concretion] {} {{M_{C}} \bc \langle M_{P} \rangle }
  \and \\
  \inferrule* [lab=process] {} {{M_{P}} \bc M_{N} \;| \;P|M_{P} }
\end{mathpar}

\begin{definition}[contextual application] Given a context $M$, and
  process $P$, we define the \emph{contextual application}, $M[P] :=
  M\{P/\Box\}$. That is, the contextual application of M to P is the
  substitution of $P$ for $\Box$ in $M$.
\end{definition}

$\meaningof{-} : L \to \mathcal{P}(\pi)$

\begin{mathpar}
  \inferrule* [lab=collection] {} {\meaningof{true} = \pi, \and \meaningof{~E} = \pi \setminus \meaningof{E}, \and \meaningof{E_{1} \& E_{2}} = \meaningof{E_{1}} \cap \meaningof{E_{2}}}
\end{mathpar}

\begin{mathpar}
  \inferrule* [lab=structure] {} {\meaningof{0} = \{ P \in \pi | P \equiv 0 \}, \and \\ \meaningof{E_1 | E_2} = \{ P \in \pi | P \equiv P_{1} | P_{2}, P_{1} \in \meaningof{E_{1}}, P_{2} \in \meaningof{E_2}\} }
\end{mathpar}

\begin{mathpar}
 \inferrule* [lab=behavior] {} {\meaningof{\langle a?b \rangle E} = \{ P \in \pi | P \equiv Q | u?(y)P', \\ \and \\\\ \and \\ \;\;\; u \in \meaningof{a}, \forall z.P'\{z/y\} \in \meaningof{E\{z/b\}}\}, \and \\ \meaningof{a!E} = \{ P \in \pi | P \equiv Q | x!\langle P' \rangle, x \in \meaningof{a} P' \in \meaningof{E}\} }
\end{mathpar}

\begin{mathpar}
 \inferrule* [lab=nominal] {} {\meaningof{\quotep{E}} = \{ \quotep{P} \in \quotep{\pi} | P \in \meaningof{E} \}, \and \meaningof{\quotep{P}} = \{ \quotep{Q} \in \quotep{\pi} | P \equiv Q \} \and \\ \meaningof{@\quotep{E}} = \{ P \in \pi | P \equiv @x, x \in \meaningof{E} \}}
\end{mathpar}

\begin{eqnarray*}
  \\
  \meaningof{-} : TS \to ST
\end{eqnarray*}

\begin{eqnarray*}
  \\
  L : TS \to ST
\end{eqnarray*}

\begin{eqnarray*}
  \\
  P \models E \iff P \in \meaningof{E}
\end{eqnarray*}

\begin{eqnarray*}
  P \approx_{L} Q \iff \forall E \in L. P \models E \iff Q \models E
\end{eqnarray*}

\begin{eqnarray*}
  P \approx_{K} Q
\end{eqnarray*}

\begin{eqnarray*}
  P \approx Q
\end{eqnarray*}

$\approx_{K} = \approx = \approx_{L}$

\subsubsection{Contextual duality}

Note that contexts extend the quotation operation to a family of
operations from processes to names. Given a context, $M$, we can
define a \emph{nominal context}, $\quotep{M}$ by $\quotep{M}[P] :=
\quotep{M[P]}$. To foreshadow what is to come we observe that these
operations enjoy a duality with processes very much like the duality
between vectors and maps from vectors to scalars.

Further, because the calculus is essentially higher-order, we have a
correspondence between contexts and processes. More specifically,
given a name $x$ and a context $M$ we can construct $M^{*}_{x}$ such
that 

\begin{mathpar}
  M^{*}_{x} | \lift{x}{P} \red M[P]
\end{mathpar}

namely,

\begin{mathpar}
  M^{*}_{x} := x?(u).M[\dropn{u}]
\end{mathpar}

The dependence of $M^{*}_{x}$ on a name makes it an abstraction, 

\begin{mathpar}
  M^{*} := (x)x?(u).M[\dropn{u}]
\end{mathpar}

\subsection{Additional notation}

It will sometimes be convenient to denote the process a name
quotes. We already have the notation $x = \quotep{P}$, but it will be
convenient to introduce an alternate notation, $\procn{x}$, when we
want to emphasize the connection to the use of the name. Note that, by
virtue of name equivalence, $\quotep{\procn{x}} \nameeq x$; so, the
notation is consistent with previous definitions.

Further, because names have structure it is possible to effect
substitutions on the basis of that structure. This means we need to
upgrade our notation for substitutions, which we accomplish by
adapting comprehension notation. Thus,

\begin{mathpar}
  P\{ y / x : x \in S \}
\end{mathpar}

is interpreted to mean the process derived from P by replacing (in a
capture-avoiding manner) each occurrence of $x$ in $S$ by $y$. For example,

\begin{mathpar}
  P\{ \quotep{\procn{x}|\procn{x}} / x : x \in \freenames{P} \}
\end{mathpar}

will replace each (occurrence) of a free name $x$ in $P$ by
$\quotep{\procn{x}|\procn{x}}$.

Also, we will avail ourselves of the notation $x^{L}$ and $x^{R}$ to
denote injections of a name into disjoint copies of the name
space. There are numerous ways to accomplish this. One example can be
found in \cite{MeredithR05}. This notation overloads to vectors of
names: $\vec{x}^{\pi} := (x_{i}^{\pi} \; : \; 0 \leq i < |\vec{x}| )$ where $\pi \in \{L,R\}$.

We also use $P^{\Box} := P|\Box$.

In \cite{MeredithR05} an interpretation of the new operator is
given. It turns out that there are several possible interpretations
all enjoying the requisite algebraic properties of the operator (see
\cite{milner91polyadicpi}). We will therefore make liberal use of
$(\nu\; \vec{x})P$.

% subsection the_syntax_and_semantics_of_the_notation_system (end)   

\input{qm2pi.qmops} 

\input{qm2pi.sterngerlach} 

\input{qm2pi.metric} 

% section concurrent_process_calculi (end)

%\input{qm2pi.proofsketch}

% section proof sketch (end)

%\input{qm2pi.slviaknots} 

% section spatial logic via knots (end)

\input{qm2pi.conclusion}

% section conclusion (end)

%\input{qm2pi.dtcodes} 

% section wiring algorithm (end)

\input{qm2pi.ack} 

% section acknowledgments (end)

\newpage


\bibliographystyle{plain}   
\bibliography{../../biblios/main.bib}

\input{qm2pi.rhodetails}

\end{document}

 

% subsection basic_interpretation (end)

%\input{qm2pi.rho.presentation} 
\subsection{The syntax and semantics of the notation system}\label{sub:the_syntax_and_semantics_of_the_notation_system} % (fold)

We now summarize a technical presentation of the calculus that
embodies our theory of dynamics. The typical presentation of such a
calculus follows the style of giving generators and relations on
them. The grammar, below, describing term constructors, freely
generates the set of processes, $\Proc$. This set is then quotiented
by a relation known as structural congruence and it is over this set
that the notion of dynamics is expressed. This presentation is
essentially that of \cite{MeredithR05} with the addition of
polyadicity and summation. For readability we have relegated some of
the technical subtleties to an appendix.

\subsubsection{Process grammar}\label{subsub:process_grammar}

\begin{mathpar}
  \inferrule* [lab=synchronization] {} {{M} \bc \pzero \;|\; x?F \;|\; x!C }
  \and
  \inferrule* [lab=abstraction] {} {{F} \bc (x)P}
  \and
  \inferrule* [lab=concretion] {} {{C} \bc \langle Q \rangle}
  \and
  \inferrule* [lab=process] {} {{P,Q} \bc M \;| \;P|Q \;|\; @{x}}
  \and
  \inferrule* [lab=name] {} {{x} \bc \quotep{P}}
\end{mathpar} 

Note that $\vec{x}$ (resp. $\vec{P}$) denotes a vector of names
(resp. processes) of length $|\vec{x}|$ (resp. $|\vec{P}|$). We adopt
the following useful abbreviations.

\begin{mathpar}
   x?(\vec{y}).P := x.(\vec{y})P \and  x\clift{\vec{P}} := x.\clift{\vec{P}}
   \and x!(y) := \lift{x}{\dropn{y}}
   \and \Pi_{i=0}^{n-1}P_i := P_0 | \ldots | P_{n-1}
\end{mathpar}

\subsubsection{Structural congruence}

\paragraph{Free and bound names and alpha-equivalence.} At the
core of structural equivalence is alpha-equivalence which identifies
process that are the same up to a change of variable. Formally, we
recognize the distinction between free and bound names. The free names
of a process, $\freenames{P}$, may be calculated recursively as
follows:

\begin{mathpar}
\freenames{\pzero} := \emptyset
  \and \\
  \freenames{x?(y).P} := \{ x \} \cup (\freenames{P} \setminus \{ y \})
  \and 
  \freenames{x!\langle P \rangle} := \{ x \} \cup \{ P \} 
  \and \\
  \freenames{P|Q} := \freenames{P} \cup \freenames{Q}
  \and \\
  \freenames{@{x}} := \{ x \}
\end{mathpar}

$\pi$
$\quotep{\pi}$

$\freenames{-} : \pi \to \mathcal{P}(\quotep{\pi})$

\begin{eqnarray*}
  \freenames{\pzero} & := & \emptyset \\
  \freenames{x?(y).P} & := & \{ x \} \cup (\freenames{P} \setminus \{ y \}) \\
  \freenames{x!\langle P \rangle} & := & \{ x \} \cup \{ P \} \\
  \freenames{P|Q} & := & \freenames{P} \cup \freenames{Q} \\
  \freenames{\dropn{x}} & := & \{ x \}
\end{eqnarray*}

The bound names of a process, $\boundnames{P}$, are those names occurring in $P$
that are not free. For example, in $x?(y).0$, the name $x$ is free, while $y$ is bound.

\begin{mathpar}
  \inferrule* [lab=monoidal-laws] {} { P|Q \equiv Q|P \and P|0 \equiv P \and P|(Q|R) \equiv (P|Q)|R }
\end{mathpar}

\begin{mathpar}
  \inferrule* [lab=alpha-equivalence] {} { (x)P \equiv (y)P\{y/x\} \and y \not\in \freenames{P} }
\end{mathpar}

\begin{definition}
Then two processes, $P,Q$, are alpha-equivalent if $P = Q\{\vec{y}/\vec{x}\}$ for
some $\vec{x} \in \boundnames{Q},\vec{y} \in \boundnames{P}$, where $Q\{\vec{y}/\vec{x}\}$
denotes the capture-avoiding substitution of $\vec{y}$ for $\vec{x}$ in $Q$.
\end{definition}

\begin{definition}
  The {\em structural congruence} \cite{SangiorgiWalker} , $\equiv$,
  between processes is the least congruence containing
  alpha-equivalence, satisfying the abelian monoid laws
  (associativity, commutativity and $\pzero$ as identity) for parallel
  composition $|$ and for summation $+$.
\end{definition}

\subsection{Name equivalence}

We take name equivalence, written $\nameeq$, to be the smallest
equivalence relation generated by the following rules.

\begin{mathpar}
\inferrule*[lab=Quote-drop]
{ }
{ \quotep{@{x}} \nameeq x }

\inferrule*[lab=Struct-equiv]
{ P \scong Q }
{ \quotep{P} \nameeq \quotep{Q} }
\end{mathpar}

The astute reader will have noticed that the mutual recursion of names
and processes imposes a mutual recursion on alpha-equivalence and
structural equivalence via name-equivalence. Fortunately, all of this
works out pleasantly and we may calculate in the natural way, free of
concern. The reader interested in the details is referred to the
appendix \ref{appendix:rho_details}.

\subsection{Substitution}

We use $\Proc$ for the set of processes, $\QProc$ for the set of
names, and $\id{\{}\vec{y} / \vec{x} \id{\}}$ to denote partial maps,
$s : \QProc \rightarrow \QProc$. A map, $s$ lifts, uniquely, to a map
on process terms, $\widehat{s} : \Proc \rightarrow \Proc$ by the
following equations.

\begin{mathpar}
  (0) \psubstp{Q}{P} := 0 \\
  (R \juxtap S) \psubstp{Q}{P}
  :=    
  (R)\psubstp{Q}{P} \juxtap (S) \psubstp{Q}{P} \\
  (x?(y).R) \psubstp{Q}{P}    
  :=    
  (x)\substp{Q}{P} (z)\concat( (R \psubstn{z}{y}) \psubstp{Q}{P} ) \\
  (\lift{x}{R}) \psubstp{Q}{P}  
  :=
  \lift{(x)\substp{Q}{P}}{ R \psubstp{Q}{P} } \\
%   (\dropn{x})  \psubstp{Q}{P}       
%   := 
%   \left\{ 
%     \begin{array}{ccc} 
%       \dropn{\quotep{Q}} & & x \nameeq \quotep{P} \\
%       \dropn{x} & & otherwise \\
%     \end{array}
%   \right. 
  (\dropn{x})  \psubstp{Q}{P}       
  := 
  \left\{ 
    \begin{array}{ccc} 
      Q & & x \nameeq \quotep{P} \\
      \dropn{x} & & otherwise \\
    \end{array}
  \right.
\end{mathpar}
 

where

\begin{eqnarray}
  (x)\id{\{} \lpquote Q \rpquote / \lpquote P \rpquote \id{\}}            = 
  \left\{ 
    \begin{array}{ccc}
      \lpquote Q \rpquote & & x \nameeq \lpquote P \rpquote \\
      x & & otherwise \\
    \end{array}
  \right. \nonumber
\end{eqnarray}

and $z$ is chosen distinct from $\quotep{P}$, $\quotep{Q}$, the free
names in $Q$, and all the names in $R$. Our $\alpha$-equivalence will
be built in the standard way from this substitution.

\begin{remark}\label{rem:no_self_referential_names}
  One consequence of these definitions is that $\forall P. \quotep{P}
  \not\in \freenames{P}$.
\end{remark}

\subsection{ Dynamic quote: an example }

Anticipating something of what's to come, consider applying the
substitution, $\widehat{\id{\{}u / z \id{\}}}$, to the following pair
of processes, $\lift{w}{y!(z)}$ and $w[ \lpquote y!(z) \rpquote ]$.

\begin{eqnarray}
	\lift{w}{y!(z)}\widehat{\id{\{}u / z \id{\}}}
		& = &
		\lift{w}{y!(u)} \nonumber\\
	w[ \lpquote y!(z) \rpquote ] \widehat{ \id{\{}u / z \id{\}} }
		& = &
		w[ \lpquote y!(z) \rpquote ] \nonumber
\end{eqnarray}

Because the body of the process between quotes is impervious to
substitution, we get radically different answers. In fact, by
examining the first process in an input context,
e.g. $x?(z).\lift{w}{y!(z)}$, we see that the process under the lift
operator may be shaped by prefixed inputs binding a name inside it. In
this sense, the lift operator will be seen as a way to dynamically
construct processes before reifying them as names.

Finally equipped with these standard features we can present the
dynamics of the calculus.

\subsubsection{Operational semantics} 

Finally, we introduce the computational dynamics. What marks these
algebras as distinct from other more traditionally studied algebraic
structures, e.g. vector spaces or polynomial rings, is the manner in
which dynamics is captured. In traditional structures, dynamics is typically
expressed through morphisms between such structures, as in linear maps
between vector spaces or morphisms between rings. In algebras
associated with the semantics of computation, the dynamics is
expressed as part of the algebraic structure itself, through a
reduction reduction relation typically denoted by $\red$. Below, we
give a recursive presentation of this relation for the calculus used
in the encoding.

$\red \subseteq \pi \times \pi$
$\red : \pi \to \mathcal{P}(\pi)$

\begin{mathpar}
  \inferrule* [lab=Comm] { \textsf{match}( x_{src}, x_{trgt} ) } { x_{trgt}?(y)P \; | \; x_{src}!\langle {Q} \rangle \red P\{\quotep{Q}/y}\} }
  \and \\
  \inferrule* [lab=Par] {{P} \red {P}'} {{{P} | {Q}} \red {{P}' | {Q}}}
  \and
  \inferrule* [lab=Equiv]{{{P} \scong {P}'} \andalso {{P}' \red {Q}'} \andalso {{Q}' \scong {Q}}}{{P} \red {Q}}
\end{mathpar}

\begin{eqnarray*}
  match_{\equiv} (\quotep{P},\quotep{Q}) & := & P \equiv Q \\
  match_{\dagger}(\quotep{P},\quotep{Q}) & := & \forall R. P|Q \red^{*} R => R \red^{*} 0 \\
  match_{K}(\quotep{P},\quotep{Q}) & := & K \mbox{ for some context } K
\end{eqnarray*}

$u?(x)P | u!\langle Q \rangle \red P\{\quotep{Q}/x\}$

%We write $\wred$ for $\red^*$, and $P\red$ if $\exists Q $ such that $ P \red Q$.
We write $P\red$ if $\exists Q $ such that $ P \red Q$ and $P\not\red$, otherwise.

\section{Replication}

As mentioned before, it is known that replication (and hence
recursion) can be implemented in a higher-order process algebra
\cite{SangiorgiWalker}. As our first example of calculation with the
machinery thus far presented we give the construction explicitly in
the {\rhoc}.

\begin{eqnarray}
	D_{x} & := & \prefix{x}{y}{(\binpar{\outputp{x}{y}}{@{y}})} \nonumber\\
	\bangp_{x}{P} & := & \binpar{{x}!\langle{\binpar{D_{x}}{P}}\rangle}{D_{x}} \nonumber
\end{eqnarray}

\begin{eqnarray}
	\bangp_{x}{P} & & \nonumber\\
	=
	& {x}!\langle{(\prefix{x}{y}{(\outputp{x}{y} | @{y})) | P}}\rangle 
	      | \prefix{x}{y}{(\outputp{x}{y} | @{y})} & \nonumber\\
	\red
	& (\outputp{x}{y} | @{y})\substn{\quotep{(\prefix{x}{y}{(@{y} | \outputp{x}{y})) | P}}}{y} & \nonumber\\
	=
	& \outputp{x}{\quotep{(\prefix{x}{y}{(\outputp{x}{y} | @{y})) | P}}}
	  | {(\prefix{x}{y}{(\outputp{x}{y} | @{y})) | P}} & \nonumber\\
	\red
	& \ldots & \nonumber\\
	\red^*
	& P | P | \ldots & \nonumber
\end{eqnarray}

Of course, this encoding, as an implementation, runs away, unfolding
$\bangp{P}$ eagerly. A lazier and more implementable replication
operator, restricted to input-guarded processes, may be obtained as follows.

\begin{eqnarray}
\bangp{\prefix{u}{v}{P}} 
	:= 
	\binpar{\lift{x}{\prefix{u}{v}{(\binpar{D(x)}{P})}}}{D(x)} \nonumber
\end{eqnarray}

\begin{remark}
  Note that the lazier definition still does not deal with summation
  or mixed summation (i.e. sums over input and output). The reader is
  invited to construct definitions of replication that deal with these
  features. 

  Further, the definitions are parameterized in a name, $x$. Can you,
  gentle reader, make a definition that eliminates this parameter and
  guarantees no accidental interaction between the replication
  machinery and the process being replicated -- i.e. no accidental
  sharing of names used by the process to get its work done and the
  name(s) used by the replication to effect copying. This latter
  revision of the definition of replication is crucial to obtaining
  the expected identity $!!P \sim !P$.
\end{remark}

\begin{remark}\label{rem:paradoxical_combinator}
  The reader familiar with the lambda calculus will have noticed the
  similarity between $D$ and the paradoxical combinator.

  [Ed. note: the existence of this seems to suggest we have to be more
  restrictive on the set of processes and names we admit if we are to
  support no-cloning.]
\end{remark}

\subsubsection{Bisimulation}

The computational dynamics gives rise to another kind of equivalence,
the equivalence of computational behavior. As previously mentioned
this is typically captured \emph{via} some form of bisimulation.

% The notion we use in this paper is weak barbed bisimulation
% \cite{milner91polyadicpi}.

The notion we use in this paper is derived from weak barbed
bisimulation \cite{milner91polyadicpi}. 

\begin{definition}
An \emph{observation relation}, $\downarrow_{\mathcal N}$, over a set
of names, $\mathcal N$, is the smallest relation satisfying the rules
below.

\infrule[Out-barb]{y \in {\mathcal N}, \; x \nameeq y}
		  {\outputp{x}{v} \downarrow_{\mathcal N} x}
\infrule[Par-barb]{\mbox{$P\downarrow_{\mathcal N} x$ or $Q\downarrow_{\mathcal N} x$}}
		  {\binpar{P}{Q} \downarrow_{\mathcal N} x}

We write $P \Downarrow_{\mathcal N} x$ if there is $Q$ such that 
$P \wred Q$ and $Q \downarrow_{\mathcal N} x$.
\end{definition}

\begin{definition}
%\label{def.bbisim}
An  ${\mathcal N}$-\emph{barbed bisimulation} over a set of names, ${\mathcal N}$, is a symmetric binary relation 
${\mathcal S}_{\mathcal N}$ between agents such that $P\rel{S}_{\mathcal N}Q$ implies:
\begin{enumerate}
\item If $P \red P'$ then $Q \wred Q'$ and $P'\rel{S}_{\mathcal N} Q'$.
\item If $P\downarrow_{\mathcal N} x$, then $Q\Downarrow_{\mathcal N} x$.
\end{enumerate}
$P$ is ${\mathcal N}$-barbed bisimilar to $Q$, written
$P \wbbisim_{\mathcal N} Q$, if $P \rel{S}_{\mathcal N} Q$ for some ${\mathcal N}$-barbed bisimulation ${\mathcal S}_{\mathcal N}$.
\end{definition}

$\mathcal{R} \subseteq \pi \times \pi$

$P \mathcal{R} Q => \forall P'. P \red P' \Rightarrow \exists Q'. Q \red Q', P' \mathcal{R} Q'$

$P \vdash x \Rightarrow Q \vdash x$

\begin{mathpar}
  \inferrule*[lab=Out-barb]{x \nameeq y}{{y}!\langle{Q}\rangle \vdash x}
  \and
  \inferrule*[lab=Par-barb]{\mbox{$P\vdash x$ or $Q\vdash x$}}{\binpar{P}{Q} \vdash x}
\end{mathpar}

\subsubsection{Contexts}

One of the principle advantages of computational calculi like the
$\pi$-calculus is a well-defined notion of context,
contextual-equivalence and a correlation between
contextual-equivalence and notions of bisimulation. The notion of
context allows the decomposition of a process into (sub-)process and
its syntactic environment, its context. Thus, a context may be
thought of as a process with a ``hole'' (written $\Box$) in it. The
application of a context $M$ to a process $P$, written $M[P]$, is
tantamount to filling the hole in $M$ with $P$. In this paper we do
not need the full weight of this theory, but do make use of the notion
of context in the proof the main theorem. 

\begin{mathpar}
  \inferrule* [lab=summation] {} {{M_{M},M_{N}} \bc \Box \;|\; x.M_{A} \;|\; M_{M}+M_{N}}
  \and
  \inferrule* [lab=agent] {} {{M_{A}} \bc (\vec{x})M_{P} \;| \; \clift{P_0,\ldots,M_{P},\ldots,P_N}}
  \and \\
  \inferrule* [lab=process] {} {{M_{P}} \bc M_{N} \;| \;P|M_{P} }
\end{mathpar} 

\begin{mathpar}
  \inferrule* [lab=sychronization] {} {M_{N} \bc \Box \;|\; x?M_{F} \;|\; x!M_{C}}
  \and
  \inferrule* [lab=abstraction] {} {{M_{F}} \bc (x)M_{P} }
  \and
  \inferrule* [lab=concretion] {} {{M_{C}} \bc \langle M_{P} \rangle }
  \and \\
  \inferrule* [lab=process] {} {{M_{P}} \bc M_{N} \;| \;P|M_{P} }
\end{mathpar}

\begin{definition}[contextual application] Given a context $M$, and
  process $P$, we define the \emph{contextual application}, $M[P] :=
  M\{P/\Box\}$. That is, the contextual application of M to P is the
  substitution of $P$ for $\Box$ in $M$.
\end{definition}

$\meaningof{-} : L \to \mathcal{P}(\pi)$

\begin{mathpar}
  \inferrule* [lab=collection] {} {\meaningof{true} = \pi, \and \meaningof{~E} = \pi \setminus \meaningof{E}, \and \meaningof{E_{1} \& E_{2}} = \meaningof{E_{1}} \cap \meaningof{E_{2}}}
\end{mathpar}

\begin{mathpar}
  \inferrule* [lab=structure] {} {\meaningof{0} = \{ P \in \pi | P \equiv 0 \}, \and \\ \meaningof{E_1 | E_2} = \{ P \in \pi | P \equiv P_{1} | P_{2}, P_{1} \in \meaningof{E_{1}}, P_{2} \in \meaningof{E_2}\} }
\end{mathpar}

\begin{mathpar}
 \inferrule* [lab=behavior] {} {\meaningof{\langle a?b \rangle E} = \{ P \in \pi | P \equiv Q | u?(y)P', \\ \and \\\\ \and \\ \;\;\; u \in \meaningof{a}, \forall z.P'\{z/y\} \in \meaningof{E\{z/b\}}\}, \and \\ \meaningof{a!E} = \{ P \in \pi | P \equiv Q | x!\langle P' \rangle, x \in \meaningof{a} P' \in \meaningof{E}\} }
\end{mathpar}

\begin{mathpar}
 \inferrule* [lab=nominal] {} {\meaningof{\quotep{E}} = \{ \quotep{P} \in \quotep{\pi} | P \in \meaningof{E} \}, \and \meaningof{\quotep{P}} = \{ \quotep{Q} \in \quotep{\pi} | P \equiv Q \} \and \\ \meaningof{@\quotep{E}} = \{ P \in \pi | P \equiv @x, x \in \meaningof{E} \}}
\end{mathpar}

\begin{eqnarray*}
  \\
  \meaningof{-} : TS \to ST
\end{eqnarray*}

\begin{eqnarray*}
  \\
  L : TS \to ST
\end{eqnarray*}

\begin{eqnarray*}
  \\
  P \models E \iff P \in \meaningof{E}
\end{eqnarray*}

\begin{eqnarray*}
  P \approx_{L} Q \iff \forall E \in L. P \models E \iff Q \models E
\end{eqnarray*}

\begin{eqnarray*}
  P \approx_{K} Q
\end{eqnarray*}

\begin{eqnarray*}
  P \approx Q
\end{eqnarray*}

$\approx_{K} = \approx = \approx_{L}$

\subsubsection{Contextual duality}

Note that contexts extend the quotation operation to a family of
operations from processes to names. Given a context, $M$, we can
define a \emph{nominal context}, $\quotep{M}$ by $\quotep{M}[P] :=
\quotep{M[P]}$. To foreshadow what is to come we observe that these
operations enjoy a duality with processes very much like the duality
between vectors and maps from vectors to scalars.

Further, because the calculus is essentially higher-order, we have a
correspondence between contexts and processes. More specifically,
given a name $x$ and a context $M$ we can construct $M^{*}_{x}$ such
that 

\begin{mathpar}
  M^{*}_{x} | \lift{x}{P} \red M[P]
\end{mathpar}

namely,

\begin{mathpar}
  M^{*}_{x} := x?(u).M[\dropn{u}]
\end{mathpar}

The dependence of $M^{*}_{x}$ on a name makes it an abstraction, 

\begin{mathpar}
  M^{*} := (x)x?(u).M[\dropn{u}]
\end{mathpar}

\subsection{Additional notation}

It will sometimes be convenient to denote the process a name
quotes. We already have the notation $x = \quotep{P}$, but it will be
convenient to introduce an alternate notation, $\procn{x}$, when we
want to emphasize the connection to the use of the name. Note that, by
virtue of name equivalence, $\quotep{\procn{x}} \nameeq x$; so, the
notation is consistent with previous definitions.

Further, because names have structure it is possible to effect
substitutions on the basis of that structure. This means we need to
upgrade our notation for substitutions, which we accomplish by
adapting comprehension notation. Thus,

\begin{mathpar}
  P\{ y / x : x \in S \}
\end{mathpar}

is interpreted to mean the process derived from P by replacing (in a
capture-avoiding manner) each occurrence of $x$ in $S$ by $y$. For example,

\begin{mathpar}
  P\{ \quotep{\procn{x}|\procn{x}} / x : x \in \freenames{P} \}
\end{mathpar}

will replace each (occurrence) of a free name $x$ in $P$ by
$\quotep{\procn{x}|\procn{x}}$.

Also, we will avail ourselves of the notation $x^{L}$ and $x^{R}$ to
denote injections of a name into disjoint copies of the name
space. There are numerous ways to accomplish this. One example can be
found in \cite{MeredithR05}. This notation overloads to vectors of
names: $\vec{x}^{\pi} := (x_{i}^{\pi} \; : \; 0 \leq i < |\vec{x}| )$ where $\pi \in \{L,R\}$.

We also use $P^{\Box} := P|\Box$.

In \cite{MeredithR05} an interpretation of the new operator is
given. It turns out that there are several possible interpretations
all enjoying the requisite algebraic properties of the operator (see
\cite{milner91polyadicpi}). We will therefore make liberal use of
$(\nu\; \vec{x})P$.

% subsection the_syntax_and_semantics_of_the_notation_system (end)   

\section{Interpretation of QM}
\subsection{Supporting definitions}
\subsubsection{Multiplication}
\begin{mathpar}
  \quotep{Q} \cdot \quotep{R} := \quotep{Q|R}
  \and \\
  \quotep{Q} \cdot P := P\{ \quotep{Q|R} / \quotep{R} : \quotep{R} \in \freenames{P} \}
\end{mathpar}

\paragraph{Discussion}
The first line needs little explanation. The second line says that
each free name of the process is replaced with the multiplication of
that name by the scalar. Multiplication of a scalar (name) by a state
(process) results in a process all the names of which have been `moved
over' by parallel composition with the process the scalar
quotes. There is a subtlety that the bound names have to be
manipulated so that multiplied names aren't accidentally
captured. There are many ways to achieve this.

\begin{remark}\label{rem:multiplication_identities}
  The reader is invited to verify that for all $x,y,z \in \QProc$ and $P \in \Proc$
  \begin{mathpar}
    x \cdot \quotep{0} \equiv x 
    \and
    x \cdot y \equiv y \cdot x
    \and
    x \cdot (y \cdot z) \equiv (x \cdot y) \cdot z
    \and \\
    \quotep{0} \cdot P \equiv P
    \and \\
    x \cdot (y \cdot P) \equiv (x \cdot y) \cdot P
    \and \\
    x \cdot (P|Q) \equiv (x \cdot P) | (x \cdot Q)
    \and \\    
  \end{mathpar}
\end{remark}

\subsubsection{Tensor product}

We define a tensor product on processes by structural induction.

\paragraph{Tensor of sums} First note that all summations, including
$\pzero$ and sequence, can be written $\Sigma_{i} x_{i}.A_{i} +
\Sigma_{j} x_{j}.C_{j}$, where we have grouped input-guarded processes
together and output-guarded processes together.

Thus, we can define the tensor product of two summations, $N_{1}\otimes N_{2}$, where

\begin{mathpar}
  N_{1} := \Sigma_{i} x_{i}.A_{i} + \Sigma_{j} x_{j}.C_{j}
  \and
  N_{2} := \Sigma_{i'} y_{i'}.B_{i'} + \Sigma_{j'} y_{j'}.D_{j'} 
\end{mathpar}

as follows.

\begin{mathpar}
  \Sigma_{i} x_{i}.A_{i} + \Sigma_{j} x_{j}.C_{j} \otimes \Sigma_{i'}
  y_{i'}.B_{i'} + \Sigma_{j'} y_{j'}.D_{j'} 
  \and \\
  := \; \Sigma_{i} \Sigma_{i'} \quotep{\stackrel{\vee}{x_{i}}| \stackrel{\vee}{y_{i'}}}.(A_{i}\otimes B_{i'}) \; | \; \Sigma_{i'} \Sigma_{i} \quotep{\stackrel{\vee}{y_{i'}}|\stackrel{\vee}{x_{i}}}.(B_{i'}\otimes A_{i})
  \and
  \;\; | \;\; \Sigma_{j} \Sigma_{j'} \quotep{\stackrel{\vee}{x_{j}}|\stackrel{\vee}{y_{j'}}}.(A_{j}\otimes B_{j'}) \; | \; \Sigma_{j'} \Sigma_{j} \quotep{\stackrel{\vee}{y_{j'}}|\stackrel{\vee}{x_{j}}}.(B_{j'}\otimes A_{j})
\end{mathpar}

\begin{remark}
  Do we need to $x^{L}$ and $y^{R}$ for this construction as well?
\end{remark}

\paragraph{Tensor of parallel compositions} Next, we distribute tensor
over par.

\begin{mathpar}
  P_{1}|P_{2} \otimes Q_{1}|Q_{2} := (P_{1} \otimes Q_{1}) | (P_{1}
  \otimes Q_{2}) | (P_{2} \otimes Q_{1}) | (P_{2} \otimes Q_{2})
\end{mathpar}

\paragraph{Tensor with dropped names} We treat tensor of a
process with a dropped name as parallel composition.

\begin{mathpar}
  P \otimes \dropn{x} := P | \dropn{x}
\end{mathpar}

\paragraph{Tensor of agents}

Finally, we need to define tensor on agents. Note that the definition
of tensor on normal products only tensors inputs with inputs and
outputs with outputs. Thus, we only have to define the operation on
``homogeneous'' pairings.

\begin{mathpar}
  (\vec{x})P \otimes (\vec{y})Q
  \and \\
  := (x_{0}^{L}|y_{0}^{R},\ldots,x_{0}^{L}|y_{n}^{R},\ldots,x_{m}^{L}|y_{0}^{R},\ldots,x_{m}^{L}|y_{n}^R)(P\{ \vec{x}^{L}/\vec{x}\} \otimes Q \{ \vec{y}^{R}/\vec{y}\})
  \and \\
  \clift{\vec{P}} \otimes \clift{\vec{Q}}
  \and \\
  := \clift{P_{0}\otimes Q_{0},\ldots,P_{0}\otimes Q_{n},\ldots,P_{m}\otimes Q_{0},\ldots,P_{m}\otimes Q_{n}}
\end{mathpar}

\begin{remark}
  Observe that arities of tensored abstractions matches arities of
  tensored concretions if the original arities matched. Note also that
  the length of the arities corresponds to the increase in dimension
  we see in ordinary vector space tensor product.
\end{remark}

\begin{remark}
  Operationally, this definition distributes the tensor down to
  components ``linked'' by summation. Tensor over summation is
  intriguing in that it mixes names. Moreover, as a consequence of the
  way it mixes names we have the identities for all $x \in \QProc$ and
  $P,Q \in \Proc$

  \begin{mathpar}
    (x \cdot P) \otimes Q \equiv x \cdot (P \otimes Q) \equiv P \otimes (x \cdot Q)
    \and
    P \otimes \pzero \equiv P
  \end{mathpar}

  that the reader is invited to verify.
\end{remark}

\subsubsection{Annihilation}
\begin{mathpar}
  P^{\perp} := \{ Q | \forall R. P|Q \red^{*} R \Rightarrow R \red^{*} \pzero \}
  \and \\
  P^{\underline{\perp}} := \Sigma_{Q \in P^{\perp}} \quotep{Q}?(y).(\dropn{y}|Q) | \Sigma_{Q \in P^{\perp}} \quotep{Q}\clift{\Box}
\end{mathpar}

\paragraph{Discussion} The reader will note that $P^{\perp}$ is a
\emph{set} of processes, while $P^{\underline{\perp}}$ is a
\emph{context}. We call the set $P^{\perp}$ the \emph{annihilators} of
$P$. The parallel composition of a process in the annihilators of $P$
with $P$ will result in a process, the state space of which has all
paths eventually leading to $\pzero$. Execution may endure loops; but
under reasonable conditions of fairness (naturally guaranteed under
most notions of bisimulation) such a composite process cannot get
stuck in such a loop and will, eventually pop out and terminate.

The context $P^{\underline{\perp}}$ is ready and willing to ``take the
$P$ out of'' the process to which it is applied. It will effectively
transmit the code of the process to which it is applied to one of the
annihilators and run the process against it.

\subsubsection{Evaluation}
We fix $M$ a domain of fully abstract interpretation with an equality
coincident with bisimulation. We take $\meaningof{\cdot} : \Proc \to
M$ to be the map interpreting processes and $\nmeaningof{\cdot} : \M
\to Proc$ to be the map running the other way. Then we define

\begin{mathpar}
  \int P := \nmeaningof{\meaningof{P}}
\end{mathpar}

\paragraph{Discussion}
There are many fully abstract interpretations of Milner's
$\pi$-calculus. Any of them can be used as a basis for interpreting
the reflective calculus here. Equipped with such a domain it is
largely a matter of grinding through to check that the Yoneda
construction for the normalization-by-evaluation program can be
extended to this setting.

\begin{remark}
  The reader is invited to verify that $\int (P^{\underline{\perp}}[P]) = 0$.
\end{remark}

\subsection{Quantum mechanics}

Table \ref{tbl:core_qm_op_defns} gives the core operational definitions

\begin{table}[htp]\label{tbl:core_qm_op_defns}
  \center{
    \fbox{
      \begin{tabular}{c|c}
        quantum mechanics & process calculus \\
        \hline
        scalar & $x := \quotep{P}$ \\
        state vector & $\state{P} := P$ \\
        dual & $\state{P}^{*} := \event{P^{\underline{\perp}}} := \quotep{P^{\underline{\perp}}}[-]$ \\
        matrix & $ \Sigma_{\alpha} \state{P_{\alpha}}x_{\alpha}\event{Q_{\alpha}}$ \\
        vector addition & $\state{P} + \state{Q} := \state{P | Q}$ \\
        tensor product & $\state{P} \otimes \state{Q} := \state{P \otimes Q}$ \\
        inner product & $\innerprod{P}{Q} := \quotep{\int P^{\underline{\perp}}[Q]}$ \\
      \end{tabular}
    }
  }
  \caption{QM - operational definitions}
\end{table}

where

\begin{mathpar}
  \prmatrix{P}{Q} := \fprmatrix{P}{\quotep{\pzero}}{Q}
  \and
  \fprmatrix{P}{x}{Q} := (\state{P},x,\event{Q})
  \and
  (\fprmatrix{P}{x}{Q})(\state{R}) := x \cdot \innerprod{Q}{R} \cdot \state{P}
  \and
  (\fprmatrix{P}{x}{Q})(\event{R}) := x \cdot \innerprod{R}{P} \cdot \event{Q}
\end{mathpar}

\paragraph{Discussion}
As promised: vectors (aka states) are represented as processes; duals
as contextual duals; inner product definition should be compared with
standard inner product definition for ....

\begin{remark}
  Assuming $\int (P^{\underline{\perp}}[P]) = 0$, the reader is
  invited to verify that $(\fprmatrix{P}{x}{P})(\state{P}) = x \cdot \state{P}$.
\end{remark}

\begin{remark}
  The reader is invited to verify that $\innerprod{P}{Q}$ could
  equally well have been written $\quotep{\int \stackrel{\vee}{x}}$
  where $x = \event{P^{\underline{\perp}}}(Q)$.

  One of the motivations for this remark is that there is another way
  to factor these operations. We could package up evaluation in the dual:

  \begin{mathpar}
    \state{P}^{*} := \event{\int P^{\underline{\perp}}} := \quotep{\int P^{\underline{\perp}}}[-]
  \end{mathpar}

  and then have inner product defined by
  
  \begin{mathpar}
    \innerprod{P}{Q} := \event{P}(Q)
  \end{mathpar}

  Hopefully, experience with the calculations will provide guidance on
  the best factoring.
\end{remark}

\begin{remark}
  Assuming $\int (P^{\underline{\perp}}[P]) = 0$, the reader is
  invited to verify that $\forall P,Q. (\prmatrix{0}{Q})(\state{0}) =
  \state{0}$ and dually $(\prmatrix{P}{0})(\event{0}) = \event{0}$.
\end{remark}

\begin{remark}
  i'm a little worried that i don't (yet) have proper support for
  complex conjugacy. But, the observation above may give us a
  clue. According to Abramsky, it must be the case that the scalars
  are iso to the homset of the identity for the tensor -- which the
  observation above characterizes. 

  For now, we will simply bookmark the notion with $\overline{x}$.
\end{remark}

\subsubsection{Adjointness}

We need to give a definition of $(\cdot)^{\dagger}$ for matrices. The
obvious candidate definition is
\begin{mathpar}
(\Sigma_{\alpha}\fprmatrix{P_{\alpha}}{x_{\alpha}}{Q_{\alpha}})^{\dagger}
= \Sigma_{\alpha}\fprmatrix{(Q_{\alpha}^{\underline{\perp}})^{*}}{\overline{x}_{\alpha}}{P_{\alpha}^{\underline{\perp}}} 
\end{mathpar}

But, $(Q_{\alpha}^{\underline{\perp}})^{*}$ requires a name along
which to communicate the process to achieve the context application.

\subsubsection{Basis for a basis}
If processes label states and ``addition'' of states (a.k.a. vector
addition) is interpreted as parallel composition, what corresponds to
notions of linear independence and basis? Here, we recall that Yoshida
has developed a set of \emph{combinators} for an asynchronous verison
of Milner's $\pi$-calculus. These are a finite set of processes such
any process can be expressed as parallel composition of these
combinators together with liberal uses of the new operator and
replication. We can simply give a translation of these into the
present calculus and have reasonable expectation that the property
carries over. That is, that the resultant set allows to express all
processes via parallel composition. Note, however, that there is no
new operator or replication in this calculus. As a result, we expect
that the corresponding set is actually infinite. That is, we expect
that the space is actually infinite dimensional.

\begin{remark}
  The attentive reader may be a bit concerned. Certainly, the
  collection $S$, $K$ and $I$ is a finite set of
  combinators. Shouldn't we expect to see a finite set of combinators
  for an effectively equivalent system? i am very sympathetic to this
  critique and feel it warrants full attention. On the other hand, i
  also have in mind the following analogy. The natural numbers, as a
  monoid under addition, has exactly $1$ generator, while the natural
  numbers, as a monoid under multiplication, has countably many
  generators (the primes). We observe that the application of the
  lambda calculus is much less resource sensitive than the parallel
  composition of the $\pi$-calculus. Could it be the case that we have
  an analogy of the form
  
  \begin{mathpar}
    m + n : MN :: m*n : M|N
  \end{mathpar}

  giving a similar blow up in the set of ``primes''?  This is such a
  wonderful thought that, even if it's not true, i think it's worth
  writing down.
\end{remark}
 

\documentclass[12pt]{llncs}
%\documentclass{jktr}

\usepackage[pdftex]{hyperref}                   
\usepackage {listings}
\usepackage {mathpartir}
\usepackage{bcprules}
%\usepackage{listings}
                       
\usepackage{graphicx} 
%\usepackage[margins=2.5cm,nohead,nofoot]{geometry}
%\usepackage{geometry}
\usepackage{amsfonts}
\usepackage{amstext}
\usepackage{latexsym}
\usepackage{amssymb}
\usepackage{color}


%\include{myPreamble}
\include{qm2pi.local} 

%\ifpdf
%\usepackage[pdftex]{graphicx}
%\else
%\usepackage{graphicx}
%\fi

 % \ifpdf
%  \usepackage{pdfsync}
%  \if


%\title{Brief Article}
%\author{David F. Snyder}
%\author{L.G. Meredith}

%\address{Dept. of Math., Texas State University--San Marcos, San Marcos, TX 78666}
       
\pagestyle{empty}


\begin{document}

\lstset{language=[Objective]Caml,frame=shadowbox}

\input{qm2pi.front}

% section front matter (end)

\input{qm2pi.intro} 
 
% section introduction (end)

% \input{qm2pi.knotations} 

% section notation (end)

\input{qm2pi.process.calculi} 

% section concurrent_process_calculi_and_spatial_logics_ (end)
    
%\input{qm2pi.knots2pi} 

%\input{qm2pi.trefoil} 

%\input{qm2pi.mainthm} 

% subsection basic_interpretation (end)

%\input{qm2pi.rho.presentation} 
\subsection{The syntax and semantics of the notation system}\label{sub:the_syntax_and_semantics_of_the_notation_system} % (fold)

We now summarize a technical presentation of the calculus that
embodies our theory of dynamics. The typical presentation of such a
calculus follows the style of giving generators and relations on
them. The grammar, below, describing term constructors, freely
generates the set of processes, $\Proc$. This set is then quotiented
by a relation known as structural congruence and it is over this set
that the notion of dynamics is expressed. This presentation is
essentially that of \cite{MeredithR05} with the addition of
polyadicity and summation. For readability we have relegated some of
the technical subtleties to an appendix.

\subsubsection{Process grammar}\label{subsub:process_grammar}

\begin{mathpar}
  \inferrule* [lab=synchronization] {} {{M} \bc \pzero \;|\; x?F \;|\; x!C }
  \and
  \inferrule* [lab=abstraction] {} {{F} \bc (x)P}
  \and
  \inferrule* [lab=concretion] {} {{C} \bc \langle Q \rangle}
  \and
  \inferrule* [lab=process] {} {{P,Q} \bc M \;| \;P|Q \;|\; @{x}}
  \and
  \inferrule* [lab=name] {} {{x} \bc \quotep{P}}
\end{mathpar} 

Note that $\vec{x}$ (resp. $\vec{P}$) denotes a vector of names
(resp. processes) of length $|\vec{x}|$ (resp. $|\vec{P}|$). We adopt
the following useful abbreviations.

\begin{mathpar}
   x?(\vec{y}).P := x.(\vec{y})P \and  x\clift{\vec{P}} := x.\clift{\vec{P}}
   \and x!(y) := \lift{x}{\dropn{y}}
   \and \Pi_{i=0}^{n-1}P_i := P_0 | \ldots | P_{n-1}
\end{mathpar}

\subsubsection{Structural congruence}

\paragraph{Free and bound names and alpha-equivalence.} At the
core of structural equivalence is alpha-equivalence which identifies
process that are the same up to a change of variable. Formally, we
recognize the distinction between free and bound names. The free names
of a process, $\freenames{P}$, may be calculated recursively as
follows:

\begin{mathpar}
\freenames{\pzero} := \emptyset
  \and \\
  \freenames{x?(y).P} := \{ x \} \cup (\freenames{P} \setminus \{ y \})
  \and 
  \freenames{x!\langle P \rangle} := \{ x \} \cup \{ P \} 
  \and \\
  \freenames{P|Q} := \freenames{P} \cup \freenames{Q}
  \and \\
  \freenames{@{x}} := \{ x \}
\end{mathpar}

$\pi$
$\quotep{\pi}$

$\freenames{-} : \pi \to \mathcal{P}(\quotep{\pi})$

\begin{eqnarray*}
  \freenames{\pzero} & := & \emptyset \\
  \freenames{x?(y).P} & := & \{ x \} \cup (\freenames{P} \setminus \{ y \}) \\
  \freenames{x!\langle P \rangle} & := & \{ x \} \cup \{ P \} \\
  \freenames{P|Q} & := & \freenames{P} \cup \freenames{Q} \\
  \freenames{\dropn{x}} & := & \{ x \}
\end{eqnarray*}

The bound names of a process, $\boundnames{P}$, are those names occurring in $P$
that are not free. For example, in $x?(y).0$, the name $x$ is free, while $y$ is bound.

\begin{mathpar}
  \inferrule* [lab=monoidal-laws] {} { P|Q \equiv Q|P \and P|0 \equiv P \and P|(Q|R) \equiv (P|Q)|R }
\end{mathpar}

\begin{mathpar}
  \inferrule* [lab=alpha-equivalence] {} { (x)P \equiv (y)P\{y/x\} \and y \not\in \freenames{P} }
\end{mathpar}

\begin{definition}
Then two processes, $P,Q$, are alpha-equivalent if $P = Q\{\vec{y}/\vec{x}\}$ for
some $\vec{x} \in \boundnames{Q},\vec{y} \in \boundnames{P}$, where $Q\{\vec{y}/\vec{x}\}$
denotes the capture-avoiding substitution of $\vec{y}$ for $\vec{x}$ in $Q$.
\end{definition}

\begin{definition}
  The {\em structural congruence} \cite{SangiorgiWalker} , $\equiv$,
  between processes is the least congruence containing
  alpha-equivalence, satisfying the abelian monoid laws
  (associativity, commutativity and $\pzero$ as identity) for parallel
  composition $|$ and for summation $+$.
\end{definition}

\subsection{Name equivalence}

We take name equivalence, written $\nameeq$, to be the smallest
equivalence relation generated by the following rules.

\begin{mathpar}
\inferrule*[lab=Quote-drop]
{ }
{ \quotep{@{x}} \nameeq x }

\inferrule*[lab=Struct-equiv]
{ P \scong Q }
{ \quotep{P} \nameeq \quotep{Q} }
\end{mathpar}

The astute reader will have noticed that the mutual recursion of names
and processes imposes a mutual recursion on alpha-equivalence and
structural equivalence via name-equivalence. Fortunately, all of this
works out pleasantly and we may calculate in the natural way, free of
concern. The reader interested in the details is referred to the
appendix \ref{appendix:rho_details}.

\subsection{Substitution}

We use $\Proc$ for the set of processes, $\QProc$ for the set of
names, and $\id{\{}\vec{y} / \vec{x} \id{\}}$ to denote partial maps,
$s : \QProc \rightarrow \QProc$. A map, $s$ lifts, uniquely, to a map
on process terms, $\widehat{s} : \Proc \rightarrow \Proc$ by the
following equations.

\begin{mathpar}
  (0) \psubstp{Q}{P} := 0 \\
  (R \juxtap S) \psubstp{Q}{P}
  :=    
  (R)\psubstp{Q}{P} \juxtap (S) \psubstp{Q}{P} \\
  (x?(y).R) \psubstp{Q}{P}    
  :=    
  (x)\substp{Q}{P} (z)\concat( (R \psubstn{z}{y}) \psubstp{Q}{P} ) \\
  (\lift{x}{R}) \psubstp{Q}{P}  
  :=
  \lift{(x)\substp{Q}{P}}{ R \psubstp{Q}{P} } \\
%   (\dropn{x})  \psubstp{Q}{P}       
%   := 
%   \left\{ 
%     \begin{array}{ccc} 
%       \dropn{\quotep{Q}} & & x \nameeq \quotep{P} \\
%       \dropn{x} & & otherwise \\
%     \end{array}
%   \right. 
  (\dropn{x})  \psubstp{Q}{P}       
  := 
  \left\{ 
    \begin{array}{ccc} 
      Q & & x \nameeq \quotep{P} \\
      \dropn{x} & & otherwise \\
    \end{array}
  \right.
\end{mathpar}
 

where

\begin{eqnarray}
  (x)\id{\{} \lpquote Q \rpquote / \lpquote P \rpquote \id{\}}            = 
  \left\{ 
    \begin{array}{ccc}
      \lpquote Q \rpquote & & x \nameeq \lpquote P \rpquote \\
      x & & otherwise \\
    \end{array}
  \right. \nonumber
\end{eqnarray}

and $z$ is chosen distinct from $\quotep{P}$, $\quotep{Q}$, the free
names in $Q$, and all the names in $R$. Our $\alpha$-equivalence will
be built in the standard way from this substitution.

\begin{remark}\label{rem:no_self_referential_names}
  One consequence of these definitions is that $\forall P. \quotep{P}
  \not\in \freenames{P}$.
\end{remark}

\subsection{ Dynamic quote: an example }

Anticipating something of what's to come, consider applying the
substitution, $\widehat{\id{\{}u / z \id{\}}}$, to the following pair
of processes, $\lift{w}{y!(z)}$ and $w[ \lpquote y!(z) \rpquote ]$.

\begin{eqnarray}
	\lift{w}{y!(z)}\widehat{\id{\{}u / z \id{\}}}
		& = &
		\lift{w}{y!(u)} \nonumber\\
	w[ \lpquote y!(z) \rpquote ] \widehat{ \id{\{}u / z \id{\}} }
		& = &
		w[ \lpquote y!(z) \rpquote ] \nonumber
\end{eqnarray}

Because the body of the process between quotes is impervious to
substitution, we get radically different answers. In fact, by
examining the first process in an input context,
e.g. $x?(z).\lift{w}{y!(z)}$, we see that the process under the lift
operator may be shaped by prefixed inputs binding a name inside it. In
this sense, the lift operator will be seen as a way to dynamically
construct processes before reifying them as names.

Finally equipped with these standard features we can present the
dynamics of the calculus.

\subsubsection{Operational semantics} 

Finally, we introduce the computational dynamics. What marks these
algebras as distinct from other more traditionally studied algebraic
structures, e.g. vector spaces or polynomial rings, is the manner in
which dynamics is captured. In traditional structures, dynamics is typically
expressed through morphisms between such structures, as in linear maps
between vector spaces or morphisms between rings. In algebras
associated with the semantics of computation, the dynamics is
expressed as part of the algebraic structure itself, through a
reduction reduction relation typically denoted by $\red$. Below, we
give a recursive presentation of this relation for the calculus used
in the encoding.

$\red \subseteq \pi \times \pi$
$\red : \pi \to \mathcal{P}(\pi)$

\begin{mathpar}
  \inferrule* [lab=Comm] { \textsf{match}( x_{src}, x_{trgt} ) } { x_{trgt}?(y)P \; | \; x_{src}!\langle {Q} \rangle \red P\{\quotep{Q}/y}\} }
  \and \\
  \inferrule* [lab=Par] {{P} \red {P}'} {{{P} | {Q}} \red {{P}' | {Q}}}
  \and
  \inferrule* [lab=Equiv]{{{P} \scong {P}'} \andalso {{P}' \red {Q}'} \andalso {{Q}' \scong {Q}}}{{P} \red {Q}}
\end{mathpar}

\begin{eqnarray*}
  match_{\equiv} (\quotep{P},\quotep{Q}) & := & P \equiv Q \\
  match_{\dagger}(\quotep{P},\quotep{Q}) & := & \forall R. P|Q \red^{*} R => R \red^{*} 0 \\
  match_{K}(\quotep{P},\quotep{Q}) & := & K \mbox{ for some context } K
\end{eqnarray*}

$u?(x)P | u!\langle Q \rangle \red P\{\quotep{Q}/x\}$

%We write $\wred$ for $\red^*$, and $P\red$ if $\exists Q $ such that $ P \red Q$.
We write $P\red$ if $\exists Q $ such that $ P \red Q$ and $P\not\red$, otherwise.

\section{Replication}

As mentioned before, it is known that replication (and hence
recursion) can be implemented in a higher-order process algebra
\cite{SangiorgiWalker}. As our first example of calculation with the
machinery thus far presented we give the construction explicitly in
the {\rhoc}.

\begin{eqnarray}
	D_{x} & := & \prefix{x}{y}{(\binpar{\outputp{x}{y}}{@{y}})} \nonumber\\
	\bangp_{x}{P} & := & \binpar{{x}!\langle{\binpar{D_{x}}{P}}\rangle}{D_{x}} \nonumber
\end{eqnarray}

\begin{eqnarray}
	\bangp_{x}{P} & & \nonumber\\
	=
	& {x}!\langle{(\prefix{x}{y}{(\outputp{x}{y} | @{y})) | P}}\rangle 
	      | \prefix{x}{y}{(\outputp{x}{y} | @{y})} & \nonumber\\
	\red
	& (\outputp{x}{y} | @{y})\substn{\quotep{(\prefix{x}{y}{(@{y} | \outputp{x}{y})) | P}}}{y} & \nonumber\\
	=
	& \outputp{x}{\quotep{(\prefix{x}{y}{(\outputp{x}{y} | @{y})) | P}}}
	  | {(\prefix{x}{y}{(\outputp{x}{y} | @{y})) | P}} & \nonumber\\
	\red
	& \ldots & \nonumber\\
	\red^*
	& P | P | \ldots & \nonumber
\end{eqnarray}

Of course, this encoding, as an implementation, runs away, unfolding
$\bangp{P}$ eagerly. A lazier and more implementable replication
operator, restricted to input-guarded processes, may be obtained as follows.

\begin{eqnarray}
\bangp{\prefix{u}{v}{P}} 
	:= 
	\binpar{\lift{x}{\prefix{u}{v}{(\binpar{D(x)}{P})}}}{D(x)} \nonumber
\end{eqnarray}

\begin{remark}
  Note that the lazier definition still does not deal with summation
  or mixed summation (i.e. sums over input and output). The reader is
  invited to construct definitions of replication that deal with these
  features. 

  Further, the definitions are parameterized in a name, $x$. Can you,
  gentle reader, make a definition that eliminates this parameter and
  guarantees no accidental interaction between the replication
  machinery and the process being replicated -- i.e. no accidental
  sharing of names used by the process to get its work done and the
  name(s) used by the replication to effect copying. This latter
  revision of the definition of replication is crucial to obtaining
  the expected identity $!!P \sim !P$.
\end{remark}

\begin{remark}\label{rem:paradoxical_combinator}
  The reader familiar with the lambda calculus will have noticed the
  similarity between $D$ and the paradoxical combinator.

  [Ed. note: the existence of this seems to suggest we have to be more
  restrictive on the set of processes and names we admit if we are to
  support no-cloning.]
\end{remark}

\subsubsection{Bisimulation}

The computational dynamics gives rise to another kind of equivalence,
the equivalence of computational behavior. As previously mentioned
this is typically captured \emph{via} some form of bisimulation.

% The notion we use in this paper is weak barbed bisimulation
% \cite{milner91polyadicpi}.

The notion we use in this paper is derived from weak barbed
bisimulation \cite{milner91polyadicpi}. 

\begin{definition}
An \emph{observation relation}, $\downarrow_{\mathcal N}$, over a set
of names, $\mathcal N$, is the smallest relation satisfying the rules
below.

\infrule[Out-barb]{y \in {\mathcal N}, \; x \nameeq y}
		  {\outputp{x}{v} \downarrow_{\mathcal N} x}
\infrule[Par-barb]{\mbox{$P\downarrow_{\mathcal N} x$ or $Q\downarrow_{\mathcal N} x$}}
		  {\binpar{P}{Q} \downarrow_{\mathcal N} x}

We write $P \Downarrow_{\mathcal N} x$ if there is $Q$ such that 
$P \wred Q$ and $Q \downarrow_{\mathcal N} x$.
\end{definition}

\begin{definition}
%\label{def.bbisim}
An  ${\mathcal N}$-\emph{barbed bisimulation} over a set of names, ${\mathcal N}$, is a symmetric binary relation 
${\mathcal S}_{\mathcal N}$ between agents such that $P\rel{S}_{\mathcal N}Q$ implies:
\begin{enumerate}
\item If $P \red P'$ then $Q \wred Q'$ and $P'\rel{S}_{\mathcal N} Q'$.
\item If $P\downarrow_{\mathcal N} x$, then $Q\Downarrow_{\mathcal N} x$.
\end{enumerate}
$P$ is ${\mathcal N}$-barbed bisimilar to $Q$, written
$P \wbbisim_{\mathcal N} Q$, if $P \rel{S}_{\mathcal N} Q$ for some ${\mathcal N}$-barbed bisimulation ${\mathcal S}_{\mathcal N}$.
\end{definition}

$\mathcal{R} \subseteq \pi \times \pi$

$P \mathcal{R} Q => \forall P'. P \red P' \Rightarrow \exists Q'. Q \red Q', P' \mathcal{R} Q'$

$P \vdash x \Rightarrow Q \vdash x$

\begin{mathpar}
  \inferrule*[lab=Out-barb]{x \nameeq y}{{y}!\langle{Q}\rangle \vdash x}
  \and
  \inferrule*[lab=Par-barb]{\mbox{$P\vdash x$ or $Q\vdash x$}}{\binpar{P}{Q} \vdash x}
\end{mathpar}

\subsubsection{Contexts}

One of the principle advantages of computational calculi like the
$\pi$-calculus is a well-defined notion of context,
contextual-equivalence and a correlation between
contextual-equivalence and notions of bisimulation. The notion of
context allows the decomposition of a process into (sub-)process and
its syntactic environment, its context. Thus, a context may be
thought of as a process with a ``hole'' (written $\Box$) in it. The
application of a context $M$ to a process $P$, written $M[P]$, is
tantamount to filling the hole in $M$ with $P$. In this paper we do
not need the full weight of this theory, but do make use of the notion
of context in the proof the main theorem. 

\begin{mathpar}
  \inferrule* [lab=summation] {} {{M_{M},M_{N}} \bc \Box \;|\; x.M_{A} \;|\; M_{M}+M_{N}}
  \and
  \inferrule* [lab=agent] {} {{M_{A}} \bc (\vec{x})M_{P} \;| \; \clift{P_0,\ldots,M_{P},\ldots,P_N}}
  \and \\
  \inferrule* [lab=process] {} {{M_{P}} \bc M_{N} \;| \;P|M_{P} }
\end{mathpar} 

\begin{mathpar}
  \inferrule* [lab=sychronization] {} {M_{N} \bc \Box \;|\; x?M_{F} \;|\; x!M_{C}}
  \and
  \inferrule* [lab=abstraction] {} {{M_{F}} \bc (x)M_{P} }
  \and
  \inferrule* [lab=concretion] {} {{M_{C}} \bc \langle M_{P} \rangle }
  \and \\
  \inferrule* [lab=process] {} {{M_{P}} \bc M_{N} \;| \;P|M_{P} }
\end{mathpar}

\begin{definition}[contextual application] Given a context $M$, and
  process $P$, we define the \emph{contextual application}, $M[P] :=
  M\{P/\Box\}$. That is, the contextual application of M to P is the
  substitution of $P$ for $\Box$ in $M$.
\end{definition}

$\meaningof{-} : L \to \mathcal{P}(\pi)$

\begin{mathpar}
  \inferrule* [lab=collection] {} {\meaningof{true} = \pi, \and \meaningof{~E} = \pi \setminus \meaningof{E}, \and \meaningof{E_{1} \& E_{2}} = \meaningof{E_{1}} \cap \meaningof{E_{2}}}
\end{mathpar}

\begin{mathpar}
  \inferrule* [lab=structure] {} {\meaningof{0} = \{ P \in \pi | P \equiv 0 \}, \and \\ \meaningof{E_1 | E_2} = \{ P \in \pi | P \equiv P_{1} | P_{2}, P_{1} \in \meaningof{E_{1}}, P_{2} \in \meaningof{E_2}\} }
\end{mathpar}

\begin{mathpar}
 \inferrule* [lab=behavior] {} {\meaningof{\langle a?b \rangle E} = \{ P \in \pi | P \equiv Q | u?(y)P', \\ \and \\\\ \and \\ \;\;\; u \in \meaningof{a}, \forall z.P'\{z/y\} \in \meaningof{E\{z/b\}}\}, \and \\ \meaningof{a!E} = \{ P \in \pi | P \equiv Q | x!\langle P' \rangle, x \in \meaningof{a} P' \in \meaningof{E}\} }
\end{mathpar}

\begin{mathpar}
 \inferrule* [lab=nominal] {} {\meaningof{\quotep{E}} = \{ \quotep{P} \in \quotep{\pi} | P \in \meaningof{E} \}, \and \meaningof{\quotep{P}} = \{ \quotep{Q} \in \quotep{\pi} | P \equiv Q \} \and \\ \meaningof{@\quotep{E}} = \{ P \in \pi | P \equiv @x, x \in \meaningof{E} \}}
\end{mathpar}

\begin{eqnarray*}
  \\
  \meaningof{-} : TS \to ST
\end{eqnarray*}

\begin{eqnarray*}
  \\
  L : TS \to ST
\end{eqnarray*}

\begin{eqnarray*}
  \\
  P \models E \iff P \in \meaningof{E}
\end{eqnarray*}

\begin{eqnarray*}
  P \approx_{L} Q \iff \forall E \in L. P \models E \iff Q \models E
\end{eqnarray*}

\begin{eqnarray*}
  P \approx_{K} Q
\end{eqnarray*}

\begin{eqnarray*}
  P \approx Q
\end{eqnarray*}

$\approx_{K} = \approx = \approx_{L}$

\subsubsection{Contextual duality}

Note that contexts extend the quotation operation to a family of
operations from processes to names. Given a context, $M$, we can
define a \emph{nominal context}, $\quotep{M}$ by $\quotep{M}[P] :=
\quotep{M[P]}$. To foreshadow what is to come we observe that these
operations enjoy a duality with processes very much like the duality
between vectors and maps from vectors to scalars.

Further, because the calculus is essentially higher-order, we have a
correspondence between contexts and processes. More specifically,
given a name $x$ and a context $M$ we can construct $M^{*}_{x}$ such
that 

\begin{mathpar}
  M^{*}_{x} | \lift{x}{P} \red M[P]
\end{mathpar}

namely,

\begin{mathpar}
  M^{*}_{x} := x?(u).M[\dropn{u}]
\end{mathpar}

The dependence of $M^{*}_{x}$ on a name makes it an abstraction, 

\begin{mathpar}
  M^{*} := (x)x?(u).M[\dropn{u}]
\end{mathpar}

\subsection{Additional notation}

It will sometimes be convenient to denote the process a name
quotes. We already have the notation $x = \quotep{P}$, but it will be
convenient to introduce an alternate notation, $\procn{x}$, when we
want to emphasize the connection to the use of the name. Note that, by
virtue of name equivalence, $\quotep{\procn{x}} \nameeq x$; so, the
notation is consistent with previous definitions.

Further, because names have structure it is possible to effect
substitutions on the basis of that structure. This means we need to
upgrade our notation for substitutions, which we accomplish by
adapting comprehension notation. Thus,

\begin{mathpar}
  P\{ y / x : x \in S \}
\end{mathpar}

is interpreted to mean the process derived from P by replacing (in a
capture-avoiding manner) each occurrence of $x$ in $S$ by $y$. For example,

\begin{mathpar}
  P\{ \quotep{\procn{x}|\procn{x}} / x : x \in \freenames{P} \}
\end{mathpar}

will replace each (occurrence) of a free name $x$ in $P$ by
$\quotep{\procn{x}|\procn{x}}$.

Also, we will avail ourselves of the notation $x^{L}$ and $x^{R}$ to
denote injections of a name into disjoint copies of the name
space. There are numerous ways to accomplish this. One example can be
found in \cite{MeredithR05}. This notation overloads to vectors of
names: $\vec{x}^{\pi} := (x_{i}^{\pi} \; : \; 0 \leq i < |\vec{x}| )$ where $\pi \in \{L,R\}$.

We also use $P^{\Box} := P|\Box$.

In \cite{MeredithR05} an interpretation of the new operator is
given. It turns out that there are several possible interpretations
all enjoying the requisite algebraic properties of the operator (see
\cite{milner91polyadicpi}). We will therefore make liberal use of
$(\nu\; \vec{x})P$.

% subsection the_syntax_and_semantics_of_the_notation_system (end)   

\input{qm2pi.qmops} 

\input{qm2pi.sterngerlach} 

\input{qm2pi.metric} 

% section concurrent_process_calculi (end)

%\input{qm2pi.proofsketch}

% section proof sketch (end)

%\input{qm2pi.slviaknots} 

% section spatial logic via knots (end)

\input{qm2pi.conclusion}

% section conclusion (end)

%\input{qm2pi.dtcodes} 

% section wiring algorithm (end)

\input{qm2pi.ack} 

% section acknowledgments (end)

\newpage


\bibliographystyle{plain}   
\bibliography{../../biblios/main.bib}

\input{qm2pi.rhodetails}

\end{document}

 

\documentclass[12pt]{llncs}
%\documentclass{jktr}

\usepackage[pdftex]{hyperref}                   
\usepackage {listings}
\usepackage {mathpartir}
\usepackage{bcprules}
%\usepackage{listings}
                       
\usepackage{graphicx} 
%\usepackage[margins=2.5cm,nohead,nofoot]{geometry}
%\usepackage{geometry}
\usepackage{amsfonts}
\usepackage{amstext}
\usepackage{latexsym}
\usepackage{amssymb}
\usepackage{color}


%\include{myPreamble}
\include{qm2pi.local} 

%\ifpdf
%\usepackage[pdftex]{graphicx}
%\else
%\usepackage{graphicx}
%\fi

 % \ifpdf
%  \usepackage{pdfsync}
%  \if


%\title{Brief Article}
%\author{David F. Snyder}
%\author{L.G. Meredith}

%\address{Dept. of Math., Texas State University--San Marcos, San Marcos, TX 78666}
       
\pagestyle{empty}


\begin{document}

\lstset{language=[Objective]Caml,frame=shadowbox}

\input{qm2pi.front}

% section front matter (end)

\input{qm2pi.intro} 
 
% section introduction (end)

% \input{qm2pi.knotations} 

% section notation (end)

\input{qm2pi.process.calculi} 

% section concurrent_process_calculi_and_spatial_logics_ (end)
    
%\input{qm2pi.knots2pi} 

%\input{qm2pi.trefoil} 

%\input{qm2pi.mainthm} 

% subsection basic_interpretation (end)

%\input{qm2pi.rho.presentation} 
\subsection{The syntax and semantics of the notation system}\label{sub:the_syntax_and_semantics_of_the_notation_system} % (fold)

We now summarize a technical presentation of the calculus that
embodies our theory of dynamics. The typical presentation of such a
calculus follows the style of giving generators and relations on
them. The grammar, below, describing term constructors, freely
generates the set of processes, $\Proc$. This set is then quotiented
by a relation known as structural congruence and it is over this set
that the notion of dynamics is expressed. This presentation is
essentially that of \cite{MeredithR05} with the addition of
polyadicity and summation. For readability we have relegated some of
the technical subtleties to an appendix.

\subsubsection{Process grammar}\label{subsub:process_grammar}

\begin{mathpar}
  \inferrule* [lab=synchronization] {} {{M} \bc \pzero \;|\; x?F \;|\; x!C }
  \and
  \inferrule* [lab=abstraction] {} {{F} \bc (x)P}
  \and
  \inferrule* [lab=concretion] {} {{C} \bc \langle Q \rangle}
  \and
  \inferrule* [lab=process] {} {{P,Q} \bc M \;| \;P|Q \;|\; @{x}}
  \and
  \inferrule* [lab=name] {} {{x} \bc \quotep{P}}
\end{mathpar} 

Note that $\vec{x}$ (resp. $\vec{P}$) denotes a vector of names
(resp. processes) of length $|\vec{x}|$ (resp. $|\vec{P}|$). We adopt
the following useful abbreviations.

\begin{mathpar}
   x?(\vec{y}).P := x.(\vec{y})P \and  x\clift{\vec{P}} := x.\clift{\vec{P}}
   \and x!(y) := \lift{x}{\dropn{y}}
   \and \Pi_{i=0}^{n-1}P_i := P_0 | \ldots | P_{n-1}
\end{mathpar}

\subsubsection{Structural congruence}

\paragraph{Free and bound names and alpha-equivalence.} At the
core of structural equivalence is alpha-equivalence which identifies
process that are the same up to a change of variable. Formally, we
recognize the distinction between free and bound names. The free names
of a process, $\freenames{P}$, may be calculated recursively as
follows:

\begin{mathpar}
\freenames{\pzero} := \emptyset
  \and \\
  \freenames{x?(y).P} := \{ x \} \cup (\freenames{P} \setminus \{ y \})
  \and 
  \freenames{x!\langle P \rangle} := \{ x \} \cup \{ P \} 
  \and \\
  \freenames{P|Q} := \freenames{P} \cup \freenames{Q}
  \and \\
  \freenames{@{x}} := \{ x \}
\end{mathpar}

$\pi$
$\quotep{\pi}$

$\freenames{-} : \pi \to \mathcal{P}(\quotep{\pi})$

\begin{eqnarray*}
  \freenames{\pzero} & := & \emptyset \\
  \freenames{x?(y).P} & := & \{ x \} \cup (\freenames{P} \setminus \{ y \}) \\
  \freenames{x!\langle P \rangle} & := & \{ x \} \cup \{ P \} \\
  \freenames{P|Q} & := & \freenames{P} \cup \freenames{Q} \\
  \freenames{\dropn{x}} & := & \{ x \}
\end{eqnarray*}

The bound names of a process, $\boundnames{P}$, are those names occurring in $P$
that are not free. For example, in $x?(y).0$, the name $x$ is free, while $y$ is bound.

\begin{mathpar}
  \inferrule* [lab=monoidal-laws] {} { P|Q \equiv Q|P \and P|0 \equiv P \and P|(Q|R) \equiv (P|Q)|R }
\end{mathpar}

\begin{mathpar}
  \inferrule* [lab=alpha-equivalence] {} { (x)P \equiv (y)P\{y/x\} \and y \not\in \freenames{P} }
\end{mathpar}

\begin{definition}
Then two processes, $P,Q$, are alpha-equivalent if $P = Q\{\vec{y}/\vec{x}\}$ for
some $\vec{x} \in \boundnames{Q},\vec{y} \in \boundnames{P}$, where $Q\{\vec{y}/\vec{x}\}$
denotes the capture-avoiding substitution of $\vec{y}$ for $\vec{x}$ in $Q$.
\end{definition}

\begin{definition}
  The {\em structural congruence} \cite{SangiorgiWalker} , $\equiv$,
  between processes is the least congruence containing
  alpha-equivalence, satisfying the abelian monoid laws
  (associativity, commutativity and $\pzero$ as identity) for parallel
  composition $|$ and for summation $+$.
\end{definition}

\subsection{Name equivalence}

We take name equivalence, written $\nameeq$, to be the smallest
equivalence relation generated by the following rules.

\begin{mathpar}
\inferrule*[lab=Quote-drop]
{ }
{ \quotep{@{x}} \nameeq x }

\inferrule*[lab=Struct-equiv]
{ P \scong Q }
{ \quotep{P} \nameeq \quotep{Q} }
\end{mathpar}

The astute reader will have noticed that the mutual recursion of names
and processes imposes a mutual recursion on alpha-equivalence and
structural equivalence via name-equivalence. Fortunately, all of this
works out pleasantly and we may calculate in the natural way, free of
concern. The reader interested in the details is referred to the
appendix \ref{appendix:rho_details}.

\subsection{Substitution}

We use $\Proc$ for the set of processes, $\QProc$ for the set of
names, and $\id{\{}\vec{y} / \vec{x} \id{\}}$ to denote partial maps,
$s : \QProc \rightarrow \QProc$. A map, $s$ lifts, uniquely, to a map
on process terms, $\widehat{s} : \Proc \rightarrow \Proc$ by the
following equations.

\begin{mathpar}
  (0) \psubstp{Q}{P} := 0 \\
  (R \juxtap S) \psubstp{Q}{P}
  :=    
  (R)\psubstp{Q}{P} \juxtap (S) \psubstp{Q}{P} \\
  (x?(y).R) \psubstp{Q}{P}    
  :=    
  (x)\substp{Q}{P} (z)\concat( (R \psubstn{z}{y}) \psubstp{Q}{P} ) \\
  (\lift{x}{R}) \psubstp{Q}{P}  
  :=
  \lift{(x)\substp{Q}{P}}{ R \psubstp{Q}{P} } \\
%   (\dropn{x})  \psubstp{Q}{P}       
%   := 
%   \left\{ 
%     \begin{array}{ccc} 
%       \dropn{\quotep{Q}} & & x \nameeq \quotep{P} \\
%       \dropn{x} & & otherwise \\
%     \end{array}
%   \right. 
  (\dropn{x})  \psubstp{Q}{P}       
  := 
  \left\{ 
    \begin{array}{ccc} 
      Q & & x \nameeq \quotep{P} \\
      \dropn{x} & & otherwise \\
    \end{array}
  \right.
\end{mathpar}
 

where

\begin{eqnarray}
  (x)\id{\{} \lpquote Q \rpquote / \lpquote P \rpquote \id{\}}            = 
  \left\{ 
    \begin{array}{ccc}
      \lpquote Q \rpquote & & x \nameeq \lpquote P \rpquote \\
      x & & otherwise \\
    \end{array}
  \right. \nonumber
\end{eqnarray}

and $z$ is chosen distinct from $\quotep{P}$, $\quotep{Q}$, the free
names in $Q$, and all the names in $R$. Our $\alpha$-equivalence will
be built in the standard way from this substitution.

\begin{remark}\label{rem:no_self_referential_names}
  One consequence of these definitions is that $\forall P. \quotep{P}
  \not\in \freenames{P}$.
\end{remark}

\subsection{ Dynamic quote: an example }

Anticipating something of what's to come, consider applying the
substitution, $\widehat{\id{\{}u / z \id{\}}}$, to the following pair
of processes, $\lift{w}{y!(z)}$ and $w[ \lpquote y!(z) \rpquote ]$.

\begin{eqnarray}
	\lift{w}{y!(z)}\widehat{\id{\{}u / z \id{\}}}
		& = &
		\lift{w}{y!(u)} \nonumber\\
	w[ \lpquote y!(z) \rpquote ] \widehat{ \id{\{}u / z \id{\}} }
		& = &
		w[ \lpquote y!(z) \rpquote ] \nonumber
\end{eqnarray}

Because the body of the process between quotes is impervious to
substitution, we get radically different answers. In fact, by
examining the first process in an input context,
e.g. $x?(z).\lift{w}{y!(z)}$, we see that the process under the lift
operator may be shaped by prefixed inputs binding a name inside it. In
this sense, the lift operator will be seen as a way to dynamically
construct processes before reifying them as names.

Finally equipped with these standard features we can present the
dynamics of the calculus.

\subsubsection{Operational semantics} 

Finally, we introduce the computational dynamics. What marks these
algebras as distinct from other more traditionally studied algebraic
structures, e.g. vector spaces or polynomial rings, is the manner in
which dynamics is captured. In traditional structures, dynamics is typically
expressed through morphisms between such structures, as in linear maps
between vector spaces or morphisms between rings. In algebras
associated with the semantics of computation, the dynamics is
expressed as part of the algebraic structure itself, through a
reduction reduction relation typically denoted by $\red$. Below, we
give a recursive presentation of this relation for the calculus used
in the encoding.

$\red \subseteq \pi \times \pi$
$\red : \pi \to \mathcal{P}(\pi)$

\begin{mathpar}
  \inferrule* [lab=Comm] { \textsf{match}( x_{src}, x_{trgt} ) } { x_{trgt}?(y)P \; | \; x_{src}!\langle {Q} \rangle \red P\{\quotep{Q}/y}\} }
  \and \\
  \inferrule* [lab=Par] {{P} \red {P}'} {{{P} | {Q}} \red {{P}' | {Q}}}
  \and
  \inferrule* [lab=Equiv]{{{P} \scong {P}'} \andalso {{P}' \red {Q}'} \andalso {{Q}' \scong {Q}}}{{P} \red {Q}}
\end{mathpar}

\begin{eqnarray*}
  match_{\equiv} (\quotep{P},\quotep{Q}) & := & P \equiv Q \\
  match_{\dagger}(\quotep{P},\quotep{Q}) & := & \forall R. P|Q \red^{*} R => R \red^{*} 0 \\
  match_{K}(\quotep{P},\quotep{Q}) & := & K \mbox{ for some context } K
\end{eqnarray*}

$u?(x)P | u!\langle Q \rangle \red P\{\quotep{Q}/x\}$

%We write $\wred$ for $\red^*$, and $P\red$ if $\exists Q $ such that $ P \red Q$.
We write $P\red$ if $\exists Q $ such that $ P \red Q$ and $P\not\red$, otherwise.

\section{Replication}

As mentioned before, it is known that replication (and hence
recursion) can be implemented in a higher-order process algebra
\cite{SangiorgiWalker}. As our first example of calculation with the
machinery thus far presented we give the construction explicitly in
the {\rhoc}.

\begin{eqnarray}
	D_{x} & := & \prefix{x}{y}{(\binpar{\outputp{x}{y}}{@{y}})} \nonumber\\
	\bangp_{x}{P} & := & \binpar{{x}!\langle{\binpar{D_{x}}{P}}\rangle}{D_{x}} \nonumber
\end{eqnarray}

\begin{eqnarray}
	\bangp_{x}{P} & & \nonumber\\
	=
	& {x}!\langle{(\prefix{x}{y}{(\outputp{x}{y} | @{y})) | P}}\rangle 
	      | \prefix{x}{y}{(\outputp{x}{y} | @{y})} & \nonumber\\
	\red
	& (\outputp{x}{y} | @{y})\substn{\quotep{(\prefix{x}{y}{(@{y} | \outputp{x}{y})) | P}}}{y} & \nonumber\\
	=
	& \outputp{x}{\quotep{(\prefix{x}{y}{(\outputp{x}{y} | @{y})) | P}}}
	  | {(\prefix{x}{y}{(\outputp{x}{y} | @{y})) | P}} & \nonumber\\
	\red
	& \ldots & \nonumber\\
	\red^*
	& P | P | \ldots & \nonumber
\end{eqnarray}

Of course, this encoding, as an implementation, runs away, unfolding
$\bangp{P}$ eagerly. A lazier and more implementable replication
operator, restricted to input-guarded processes, may be obtained as follows.

\begin{eqnarray}
\bangp{\prefix{u}{v}{P}} 
	:= 
	\binpar{\lift{x}{\prefix{u}{v}{(\binpar{D(x)}{P})}}}{D(x)} \nonumber
\end{eqnarray}

\begin{remark}
  Note that the lazier definition still does not deal with summation
  or mixed summation (i.e. sums over input and output). The reader is
  invited to construct definitions of replication that deal with these
  features. 

  Further, the definitions are parameterized in a name, $x$. Can you,
  gentle reader, make a definition that eliminates this parameter and
  guarantees no accidental interaction between the replication
  machinery and the process being replicated -- i.e. no accidental
  sharing of names used by the process to get its work done and the
  name(s) used by the replication to effect copying. This latter
  revision of the definition of replication is crucial to obtaining
  the expected identity $!!P \sim !P$.
\end{remark}

\begin{remark}\label{rem:paradoxical_combinator}
  The reader familiar with the lambda calculus will have noticed the
  similarity between $D$ and the paradoxical combinator.

  [Ed. note: the existence of this seems to suggest we have to be more
  restrictive on the set of processes and names we admit if we are to
  support no-cloning.]
\end{remark}

\subsubsection{Bisimulation}

The computational dynamics gives rise to another kind of equivalence,
the equivalence of computational behavior. As previously mentioned
this is typically captured \emph{via} some form of bisimulation.

% The notion we use in this paper is weak barbed bisimulation
% \cite{milner91polyadicpi}.

The notion we use in this paper is derived from weak barbed
bisimulation \cite{milner91polyadicpi}. 

\begin{definition}
An \emph{observation relation}, $\downarrow_{\mathcal N}$, over a set
of names, $\mathcal N$, is the smallest relation satisfying the rules
below.

\infrule[Out-barb]{y \in {\mathcal N}, \; x \nameeq y}
		  {\outputp{x}{v} \downarrow_{\mathcal N} x}
\infrule[Par-barb]{\mbox{$P\downarrow_{\mathcal N} x$ or $Q\downarrow_{\mathcal N} x$}}
		  {\binpar{P}{Q} \downarrow_{\mathcal N} x}

We write $P \Downarrow_{\mathcal N} x$ if there is $Q$ such that 
$P \wred Q$ and $Q \downarrow_{\mathcal N} x$.
\end{definition}

\begin{definition}
%\label{def.bbisim}
An  ${\mathcal N}$-\emph{barbed bisimulation} over a set of names, ${\mathcal N}$, is a symmetric binary relation 
${\mathcal S}_{\mathcal N}$ between agents such that $P\rel{S}_{\mathcal N}Q$ implies:
\begin{enumerate}
\item If $P \red P'$ then $Q \wred Q'$ and $P'\rel{S}_{\mathcal N} Q'$.
\item If $P\downarrow_{\mathcal N} x$, then $Q\Downarrow_{\mathcal N} x$.
\end{enumerate}
$P$ is ${\mathcal N}$-barbed bisimilar to $Q$, written
$P \wbbisim_{\mathcal N} Q$, if $P \rel{S}_{\mathcal N} Q$ for some ${\mathcal N}$-barbed bisimulation ${\mathcal S}_{\mathcal N}$.
\end{definition}

$\mathcal{R} \subseteq \pi \times \pi$

$P \mathcal{R} Q => \forall P'. P \red P' \Rightarrow \exists Q'. Q \red Q', P' \mathcal{R} Q'$

$P \vdash x \Rightarrow Q \vdash x$

\begin{mathpar}
  \inferrule*[lab=Out-barb]{x \nameeq y}{{y}!\langle{Q}\rangle \vdash x}
  \and
  \inferrule*[lab=Par-barb]{\mbox{$P\vdash x$ or $Q\vdash x$}}{\binpar{P}{Q} \vdash x}
\end{mathpar}

\subsubsection{Contexts}

One of the principle advantages of computational calculi like the
$\pi$-calculus is a well-defined notion of context,
contextual-equivalence and a correlation between
contextual-equivalence and notions of bisimulation. The notion of
context allows the decomposition of a process into (sub-)process and
its syntactic environment, its context. Thus, a context may be
thought of as a process with a ``hole'' (written $\Box$) in it. The
application of a context $M$ to a process $P$, written $M[P]$, is
tantamount to filling the hole in $M$ with $P$. In this paper we do
not need the full weight of this theory, but do make use of the notion
of context in the proof the main theorem. 

\begin{mathpar}
  \inferrule* [lab=summation] {} {{M_{M},M_{N}} \bc \Box \;|\; x.M_{A} \;|\; M_{M}+M_{N}}
  \and
  \inferrule* [lab=agent] {} {{M_{A}} \bc (\vec{x})M_{P} \;| \; \clift{P_0,\ldots,M_{P},\ldots,P_N}}
  \and \\
  \inferrule* [lab=process] {} {{M_{P}} \bc M_{N} \;| \;P|M_{P} }
\end{mathpar} 

\begin{mathpar}
  \inferrule* [lab=sychronization] {} {M_{N} \bc \Box \;|\; x?M_{F} \;|\; x!M_{C}}
  \and
  \inferrule* [lab=abstraction] {} {{M_{F}} \bc (x)M_{P} }
  \and
  \inferrule* [lab=concretion] {} {{M_{C}} \bc \langle M_{P} \rangle }
  \and \\
  \inferrule* [lab=process] {} {{M_{P}} \bc M_{N} \;| \;P|M_{P} }
\end{mathpar}

\begin{definition}[contextual application] Given a context $M$, and
  process $P$, we define the \emph{contextual application}, $M[P] :=
  M\{P/\Box\}$. That is, the contextual application of M to P is the
  substitution of $P$ for $\Box$ in $M$.
\end{definition}

$\meaningof{-} : L \to \mathcal{P}(\pi)$

\begin{mathpar}
  \inferrule* [lab=collection] {} {\meaningof{true} = \pi, \and \meaningof{~E} = \pi \setminus \meaningof{E}, \and \meaningof{E_{1} \& E_{2}} = \meaningof{E_{1}} \cap \meaningof{E_{2}}}
\end{mathpar}

\begin{mathpar}
  \inferrule* [lab=structure] {} {\meaningof{0} = \{ P \in \pi | P \equiv 0 \}, \and \\ \meaningof{E_1 | E_2} = \{ P \in \pi | P \equiv P_{1} | P_{2}, P_{1} \in \meaningof{E_{1}}, P_{2} \in \meaningof{E_2}\} }
\end{mathpar}

\begin{mathpar}
 \inferrule* [lab=behavior] {} {\meaningof{\langle a?b \rangle E} = \{ P \in \pi | P \equiv Q | u?(y)P', \\ \and \\\\ \and \\ \;\;\; u \in \meaningof{a}, \forall z.P'\{z/y\} \in \meaningof{E\{z/b\}}\}, \and \\ \meaningof{a!E} = \{ P \in \pi | P \equiv Q | x!\langle P' \rangle, x \in \meaningof{a} P' \in \meaningof{E}\} }
\end{mathpar}

\begin{mathpar}
 \inferrule* [lab=nominal] {} {\meaningof{\quotep{E}} = \{ \quotep{P} \in \quotep{\pi} | P \in \meaningof{E} \}, \and \meaningof{\quotep{P}} = \{ \quotep{Q} \in \quotep{\pi} | P \equiv Q \} \and \\ \meaningof{@\quotep{E}} = \{ P \in \pi | P \equiv @x, x \in \meaningof{E} \}}
\end{mathpar}

\begin{eqnarray*}
  \\
  \meaningof{-} : TS \to ST
\end{eqnarray*}

\begin{eqnarray*}
  \\
  L : TS \to ST
\end{eqnarray*}

\begin{eqnarray*}
  \\
  P \models E \iff P \in \meaningof{E}
\end{eqnarray*}

\begin{eqnarray*}
  P \approx_{L} Q \iff \forall E \in L. P \models E \iff Q \models E
\end{eqnarray*}

\begin{eqnarray*}
  P \approx_{K} Q
\end{eqnarray*}

\begin{eqnarray*}
  P \approx Q
\end{eqnarray*}

$\approx_{K} = \approx = \approx_{L}$

\subsubsection{Contextual duality}

Note that contexts extend the quotation operation to a family of
operations from processes to names. Given a context, $M$, we can
define a \emph{nominal context}, $\quotep{M}$ by $\quotep{M}[P] :=
\quotep{M[P]}$. To foreshadow what is to come we observe that these
operations enjoy a duality with processes very much like the duality
between vectors and maps from vectors to scalars.

Further, because the calculus is essentially higher-order, we have a
correspondence between contexts and processes. More specifically,
given a name $x$ and a context $M$ we can construct $M^{*}_{x}$ such
that 

\begin{mathpar}
  M^{*}_{x} | \lift{x}{P} \red M[P]
\end{mathpar}

namely,

\begin{mathpar}
  M^{*}_{x} := x?(u).M[\dropn{u}]
\end{mathpar}

The dependence of $M^{*}_{x}$ on a name makes it an abstraction, 

\begin{mathpar}
  M^{*} := (x)x?(u).M[\dropn{u}]
\end{mathpar}

\subsection{Additional notation}

It will sometimes be convenient to denote the process a name
quotes. We already have the notation $x = \quotep{P}$, but it will be
convenient to introduce an alternate notation, $\procn{x}$, when we
want to emphasize the connection to the use of the name. Note that, by
virtue of name equivalence, $\quotep{\procn{x}} \nameeq x$; so, the
notation is consistent with previous definitions.

Further, because names have structure it is possible to effect
substitutions on the basis of that structure. This means we need to
upgrade our notation for substitutions, which we accomplish by
adapting comprehension notation. Thus,

\begin{mathpar}
  P\{ y / x : x \in S \}
\end{mathpar}

is interpreted to mean the process derived from P by replacing (in a
capture-avoiding manner) each occurrence of $x$ in $S$ by $y$. For example,

\begin{mathpar}
  P\{ \quotep{\procn{x}|\procn{x}} / x : x \in \freenames{P} \}
\end{mathpar}

will replace each (occurrence) of a free name $x$ in $P$ by
$\quotep{\procn{x}|\procn{x}}$.

Also, we will avail ourselves of the notation $x^{L}$ and $x^{R}$ to
denote injections of a name into disjoint copies of the name
space. There are numerous ways to accomplish this. One example can be
found in \cite{MeredithR05}. This notation overloads to vectors of
names: $\vec{x}^{\pi} := (x_{i}^{\pi} \; : \; 0 \leq i < |\vec{x}| )$ where $\pi \in \{L,R\}$.

We also use $P^{\Box} := P|\Box$.

In \cite{MeredithR05} an interpretation of the new operator is
given. It turns out that there are several possible interpretations
all enjoying the requisite algebraic properties of the operator (see
\cite{milner91polyadicpi}). We will therefore make liberal use of
$(\nu\; \vec{x})P$.

% subsection the_syntax_and_semantics_of_the_notation_system (end)   

\input{qm2pi.qmops} 

\input{qm2pi.sterngerlach} 

\input{qm2pi.metric} 

% section concurrent_process_calculi (end)

%\input{qm2pi.proofsketch}

% section proof sketch (end)

%\input{qm2pi.slviaknots} 

% section spatial logic via knots (end)

\input{qm2pi.conclusion}

% section conclusion (end)

%\input{qm2pi.dtcodes} 

% section wiring algorithm (end)

\input{qm2pi.ack} 

% section acknowledgments (end)

\newpage


\bibliographystyle{plain}   
\bibliography{../../biblios/main.bib}

\input{qm2pi.rhodetails}

\end{document}

 

% section concurrent_process_calculi (end)

%\documentclass[12pt]{llncs}
%\documentclass{jktr}

\usepackage[pdftex]{hyperref}                   
\usepackage {listings}
\usepackage {mathpartir}
\usepackage{bcprules}
%\usepackage{listings}
                       
\usepackage{graphicx} 
%\usepackage[margins=2.5cm,nohead,nofoot]{geometry}
%\usepackage{geometry}
\usepackage{amsfonts}
\usepackage{amstext}
\usepackage{latexsym}
\usepackage{amssymb}
\usepackage{color}


%\include{myPreamble}
\include{qm2pi.local} 

%\ifpdf
%\usepackage[pdftex]{graphicx}
%\else
%\usepackage{graphicx}
%\fi

 % \ifpdf
%  \usepackage{pdfsync}
%  \if


%\title{Brief Article}
%\author{David F. Snyder}
%\author{L.G. Meredith}

%\address{Dept. of Math., Texas State University--San Marcos, San Marcos, TX 78666}
       
\pagestyle{empty}


\begin{document}

\lstset{language=[Objective]Caml,frame=shadowbox}

\input{qm2pi.front}

% section front matter (end)

\input{qm2pi.intro} 
 
% section introduction (end)

% \input{qm2pi.knotations} 

% section notation (end)

\input{qm2pi.process.calculi} 

% section concurrent_process_calculi_and_spatial_logics_ (end)
    
%\input{qm2pi.knots2pi} 

%\input{qm2pi.trefoil} 

%\input{qm2pi.mainthm} 

% subsection basic_interpretation (end)

%\input{qm2pi.rho.presentation} 
\subsection{The syntax and semantics of the notation system}\label{sub:the_syntax_and_semantics_of_the_notation_system} % (fold)

We now summarize a technical presentation of the calculus that
embodies our theory of dynamics. The typical presentation of such a
calculus follows the style of giving generators and relations on
them. The grammar, below, describing term constructors, freely
generates the set of processes, $\Proc$. This set is then quotiented
by a relation known as structural congruence and it is over this set
that the notion of dynamics is expressed. This presentation is
essentially that of \cite{MeredithR05} with the addition of
polyadicity and summation. For readability we have relegated some of
the technical subtleties to an appendix.

\subsubsection{Process grammar}\label{subsub:process_grammar}

\begin{mathpar}
  \inferrule* [lab=synchronization] {} {{M} \bc \pzero \;|\; x?F \;|\; x!C }
  \and
  \inferrule* [lab=abstraction] {} {{F} \bc (x)P}
  \and
  \inferrule* [lab=concretion] {} {{C} \bc \langle Q \rangle}
  \and
  \inferrule* [lab=process] {} {{P,Q} \bc M \;| \;P|Q \;|\; @{x}}
  \and
  \inferrule* [lab=name] {} {{x} \bc \quotep{P}}
\end{mathpar} 

Note that $\vec{x}$ (resp. $\vec{P}$) denotes a vector of names
(resp. processes) of length $|\vec{x}|$ (resp. $|\vec{P}|$). We adopt
the following useful abbreviations.

\begin{mathpar}
   x?(\vec{y}).P := x.(\vec{y})P \and  x\clift{\vec{P}} := x.\clift{\vec{P}}
   \and x!(y) := \lift{x}{\dropn{y}}
   \and \Pi_{i=0}^{n-1}P_i := P_0 | \ldots | P_{n-1}
\end{mathpar}

\subsubsection{Structural congruence}

\paragraph{Free and bound names and alpha-equivalence.} At the
core of structural equivalence is alpha-equivalence which identifies
process that are the same up to a change of variable. Formally, we
recognize the distinction between free and bound names. The free names
of a process, $\freenames{P}$, may be calculated recursively as
follows:

\begin{mathpar}
\freenames{\pzero} := \emptyset
  \and \\
  \freenames{x?(y).P} := \{ x \} \cup (\freenames{P} \setminus \{ y \})
  \and 
  \freenames{x!\langle P \rangle} := \{ x \} \cup \{ P \} 
  \and \\
  \freenames{P|Q} := \freenames{P} \cup \freenames{Q}
  \and \\
  \freenames{@{x}} := \{ x \}
\end{mathpar}

$\pi$
$\quotep{\pi}$

$\freenames{-} : \pi \to \mathcal{P}(\quotep{\pi})$

\begin{eqnarray*}
  \freenames{\pzero} & := & \emptyset \\
  \freenames{x?(y).P} & := & \{ x \} \cup (\freenames{P} \setminus \{ y \}) \\
  \freenames{x!\langle P \rangle} & := & \{ x \} \cup \{ P \} \\
  \freenames{P|Q} & := & \freenames{P} \cup \freenames{Q} \\
  \freenames{\dropn{x}} & := & \{ x \}
\end{eqnarray*}

The bound names of a process, $\boundnames{P}$, are those names occurring in $P$
that are not free. For example, in $x?(y).0$, the name $x$ is free, while $y$ is bound.

\begin{mathpar}
  \inferrule* [lab=monoidal-laws] {} { P|Q \equiv Q|P \and P|0 \equiv P \and P|(Q|R) \equiv (P|Q)|R }
\end{mathpar}

\begin{mathpar}
  \inferrule* [lab=alpha-equivalence] {} { (x)P \equiv (y)P\{y/x\} \and y \not\in \freenames{P} }
\end{mathpar}

\begin{definition}
Then two processes, $P,Q$, are alpha-equivalent if $P = Q\{\vec{y}/\vec{x}\}$ for
some $\vec{x} \in \boundnames{Q},\vec{y} \in \boundnames{P}$, where $Q\{\vec{y}/\vec{x}\}$
denotes the capture-avoiding substitution of $\vec{y}$ for $\vec{x}$ in $Q$.
\end{definition}

\begin{definition}
  The {\em structural congruence} \cite{SangiorgiWalker} , $\equiv$,
  between processes is the least congruence containing
  alpha-equivalence, satisfying the abelian monoid laws
  (associativity, commutativity and $\pzero$ as identity) for parallel
  composition $|$ and for summation $+$.
\end{definition}

\subsection{Name equivalence}

We take name equivalence, written $\nameeq$, to be the smallest
equivalence relation generated by the following rules.

\begin{mathpar}
\inferrule*[lab=Quote-drop]
{ }
{ \quotep{@{x}} \nameeq x }

\inferrule*[lab=Struct-equiv]
{ P \scong Q }
{ \quotep{P} \nameeq \quotep{Q} }
\end{mathpar}

The astute reader will have noticed that the mutual recursion of names
and processes imposes a mutual recursion on alpha-equivalence and
structural equivalence via name-equivalence. Fortunately, all of this
works out pleasantly and we may calculate in the natural way, free of
concern. The reader interested in the details is referred to the
appendix \ref{appendix:rho_details}.

\subsection{Substitution}

We use $\Proc$ for the set of processes, $\QProc$ for the set of
names, and $\id{\{}\vec{y} / \vec{x} \id{\}}$ to denote partial maps,
$s : \QProc \rightarrow \QProc$. A map, $s$ lifts, uniquely, to a map
on process terms, $\widehat{s} : \Proc \rightarrow \Proc$ by the
following equations.

\begin{mathpar}
  (0) \psubstp{Q}{P} := 0 \\
  (R \juxtap S) \psubstp{Q}{P}
  :=    
  (R)\psubstp{Q}{P} \juxtap (S) \psubstp{Q}{P} \\
  (x?(y).R) \psubstp{Q}{P}    
  :=    
  (x)\substp{Q}{P} (z)\concat( (R \psubstn{z}{y}) \psubstp{Q}{P} ) \\
  (\lift{x}{R}) \psubstp{Q}{P}  
  :=
  \lift{(x)\substp{Q}{P}}{ R \psubstp{Q}{P} } \\
%   (\dropn{x})  \psubstp{Q}{P}       
%   := 
%   \left\{ 
%     \begin{array}{ccc} 
%       \dropn{\quotep{Q}} & & x \nameeq \quotep{P} \\
%       \dropn{x} & & otherwise \\
%     \end{array}
%   \right. 
  (\dropn{x})  \psubstp{Q}{P}       
  := 
  \left\{ 
    \begin{array}{ccc} 
      Q & & x \nameeq \quotep{P} \\
      \dropn{x} & & otherwise \\
    \end{array}
  \right.
\end{mathpar}
 

where

\begin{eqnarray}
  (x)\id{\{} \lpquote Q \rpquote / \lpquote P \rpquote \id{\}}            = 
  \left\{ 
    \begin{array}{ccc}
      \lpquote Q \rpquote & & x \nameeq \lpquote P \rpquote \\
      x & & otherwise \\
    \end{array}
  \right. \nonumber
\end{eqnarray}

and $z$ is chosen distinct from $\quotep{P}$, $\quotep{Q}$, the free
names in $Q$, and all the names in $R$. Our $\alpha$-equivalence will
be built in the standard way from this substitution.

\begin{remark}\label{rem:no_self_referential_names}
  One consequence of these definitions is that $\forall P. \quotep{P}
  \not\in \freenames{P}$.
\end{remark}

\subsection{ Dynamic quote: an example }

Anticipating something of what's to come, consider applying the
substitution, $\widehat{\id{\{}u / z \id{\}}}$, to the following pair
of processes, $\lift{w}{y!(z)}$ and $w[ \lpquote y!(z) \rpquote ]$.

\begin{eqnarray}
	\lift{w}{y!(z)}\widehat{\id{\{}u / z \id{\}}}
		& = &
		\lift{w}{y!(u)} \nonumber\\
	w[ \lpquote y!(z) \rpquote ] \widehat{ \id{\{}u / z \id{\}} }
		& = &
		w[ \lpquote y!(z) \rpquote ] \nonumber
\end{eqnarray}

Because the body of the process between quotes is impervious to
substitution, we get radically different answers. In fact, by
examining the first process in an input context,
e.g. $x?(z).\lift{w}{y!(z)}$, we see that the process under the lift
operator may be shaped by prefixed inputs binding a name inside it. In
this sense, the lift operator will be seen as a way to dynamically
construct processes before reifying them as names.

Finally equipped with these standard features we can present the
dynamics of the calculus.

\subsubsection{Operational semantics} 

Finally, we introduce the computational dynamics. What marks these
algebras as distinct from other more traditionally studied algebraic
structures, e.g. vector spaces or polynomial rings, is the manner in
which dynamics is captured. In traditional structures, dynamics is typically
expressed through morphisms between such structures, as in linear maps
between vector spaces or morphisms between rings. In algebras
associated with the semantics of computation, the dynamics is
expressed as part of the algebraic structure itself, through a
reduction reduction relation typically denoted by $\red$. Below, we
give a recursive presentation of this relation for the calculus used
in the encoding.

$\red \subseteq \pi \times \pi$
$\red : \pi \to \mathcal{P}(\pi)$

\begin{mathpar}
  \inferrule* [lab=Comm] { \textsf{match}( x_{src}, x_{trgt} ) } { x_{trgt}?(y)P \; | \; x_{src}!\langle {Q} \rangle \red P\{\quotep{Q}/y}\} }
  \and \\
  \inferrule* [lab=Par] {{P} \red {P}'} {{{P} | {Q}} \red {{P}' | {Q}}}
  \and
  \inferrule* [lab=Equiv]{{{P} \scong {P}'} \andalso {{P}' \red {Q}'} \andalso {{Q}' \scong {Q}}}{{P} \red {Q}}
\end{mathpar}

\begin{eqnarray*}
  match_{\equiv} (\quotep{P},\quotep{Q}) & := & P \equiv Q \\
  match_{\dagger}(\quotep{P},\quotep{Q}) & := & \forall R. P|Q \red^{*} R => R \red^{*} 0 \\
  match_{K}(\quotep{P},\quotep{Q}) & := & K \mbox{ for some context } K
\end{eqnarray*}

$u?(x)P | u!\langle Q \rangle \red P\{\quotep{Q}/x\}$

%We write $\wred$ for $\red^*$, and $P\red$ if $\exists Q $ such that $ P \red Q$.
We write $P\red$ if $\exists Q $ such that $ P \red Q$ and $P\not\red$, otherwise.

\section{Replication}

As mentioned before, it is known that replication (and hence
recursion) can be implemented in a higher-order process algebra
\cite{SangiorgiWalker}. As our first example of calculation with the
machinery thus far presented we give the construction explicitly in
the {\rhoc}.

\begin{eqnarray}
	D_{x} & := & \prefix{x}{y}{(\binpar{\outputp{x}{y}}{@{y}})} \nonumber\\
	\bangp_{x}{P} & := & \binpar{{x}!\langle{\binpar{D_{x}}{P}}\rangle}{D_{x}} \nonumber
\end{eqnarray}

\begin{eqnarray}
	\bangp_{x}{P} & & \nonumber\\
	=
	& {x}!\langle{(\prefix{x}{y}{(\outputp{x}{y} | @{y})) | P}}\rangle 
	      | \prefix{x}{y}{(\outputp{x}{y} | @{y})} & \nonumber\\
	\red
	& (\outputp{x}{y} | @{y})\substn{\quotep{(\prefix{x}{y}{(@{y} | \outputp{x}{y})) | P}}}{y} & \nonumber\\
	=
	& \outputp{x}{\quotep{(\prefix{x}{y}{(\outputp{x}{y} | @{y})) | P}}}
	  | {(\prefix{x}{y}{(\outputp{x}{y} | @{y})) | P}} & \nonumber\\
	\red
	& \ldots & \nonumber\\
	\red^*
	& P | P | \ldots & \nonumber
\end{eqnarray}

Of course, this encoding, as an implementation, runs away, unfolding
$\bangp{P}$ eagerly. A lazier and more implementable replication
operator, restricted to input-guarded processes, may be obtained as follows.

\begin{eqnarray}
\bangp{\prefix{u}{v}{P}} 
	:= 
	\binpar{\lift{x}{\prefix{u}{v}{(\binpar{D(x)}{P})}}}{D(x)} \nonumber
\end{eqnarray}

\begin{remark}
  Note that the lazier definition still does not deal with summation
  or mixed summation (i.e. sums over input and output). The reader is
  invited to construct definitions of replication that deal with these
  features. 

  Further, the definitions are parameterized in a name, $x$. Can you,
  gentle reader, make a definition that eliminates this parameter and
  guarantees no accidental interaction between the replication
  machinery and the process being replicated -- i.e. no accidental
  sharing of names used by the process to get its work done and the
  name(s) used by the replication to effect copying. This latter
  revision of the definition of replication is crucial to obtaining
  the expected identity $!!P \sim !P$.
\end{remark}

\begin{remark}\label{rem:paradoxical_combinator}
  The reader familiar with the lambda calculus will have noticed the
  similarity between $D$ and the paradoxical combinator.

  [Ed. note: the existence of this seems to suggest we have to be more
  restrictive on the set of processes and names we admit if we are to
  support no-cloning.]
\end{remark}

\subsubsection{Bisimulation}

The computational dynamics gives rise to another kind of equivalence,
the equivalence of computational behavior. As previously mentioned
this is typically captured \emph{via} some form of bisimulation.

% The notion we use in this paper is weak barbed bisimulation
% \cite{milner91polyadicpi}.

The notion we use in this paper is derived from weak barbed
bisimulation \cite{milner91polyadicpi}. 

\begin{definition}
An \emph{observation relation}, $\downarrow_{\mathcal N}$, over a set
of names, $\mathcal N$, is the smallest relation satisfying the rules
below.

\infrule[Out-barb]{y \in {\mathcal N}, \; x \nameeq y}
		  {\outputp{x}{v} \downarrow_{\mathcal N} x}
\infrule[Par-barb]{\mbox{$P\downarrow_{\mathcal N} x$ or $Q\downarrow_{\mathcal N} x$}}
		  {\binpar{P}{Q} \downarrow_{\mathcal N} x}

We write $P \Downarrow_{\mathcal N} x$ if there is $Q$ such that 
$P \wred Q$ and $Q \downarrow_{\mathcal N} x$.
\end{definition}

\begin{definition}
%\label{def.bbisim}
An  ${\mathcal N}$-\emph{barbed bisimulation} over a set of names, ${\mathcal N}$, is a symmetric binary relation 
${\mathcal S}_{\mathcal N}$ between agents such that $P\rel{S}_{\mathcal N}Q$ implies:
\begin{enumerate}
\item If $P \red P'$ then $Q \wred Q'$ and $P'\rel{S}_{\mathcal N} Q'$.
\item If $P\downarrow_{\mathcal N} x$, then $Q\Downarrow_{\mathcal N} x$.
\end{enumerate}
$P$ is ${\mathcal N}$-barbed bisimilar to $Q$, written
$P \wbbisim_{\mathcal N} Q$, if $P \rel{S}_{\mathcal N} Q$ for some ${\mathcal N}$-barbed bisimulation ${\mathcal S}_{\mathcal N}$.
\end{definition}

$\mathcal{R} \subseteq \pi \times \pi$

$P \mathcal{R} Q => \forall P'. P \red P' \Rightarrow \exists Q'. Q \red Q', P' \mathcal{R} Q'$

$P \vdash x \Rightarrow Q \vdash x$

\begin{mathpar}
  \inferrule*[lab=Out-barb]{x \nameeq y}{{y}!\langle{Q}\rangle \vdash x}
  \and
  \inferrule*[lab=Par-barb]{\mbox{$P\vdash x$ or $Q\vdash x$}}{\binpar{P}{Q} \vdash x}
\end{mathpar}

\subsubsection{Contexts}

One of the principle advantages of computational calculi like the
$\pi$-calculus is a well-defined notion of context,
contextual-equivalence and a correlation between
contextual-equivalence and notions of bisimulation. The notion of
context allows the decomposition of a process into (sub-)process and
its syntactic environment, its context. Thus, a context may be
thought of as a process with a ``hole'' (written $\Box$) in it. The
application of a context $M$ to a process $P$, written $M[P]$, is
tantamount to filling the hole in $M$ with $P$. In this paper we do
not need the full weight of this theory, but do make use of the notion
of context in the proof the main theorem. 

\begin{mathpar}
  \inferrule* [lab=summation] {} {{M_{M},M_{N}} \bc \Box \;|\; x.M_{A} \;|\; M_{M}+M_{N}}
  \and
  \inferrule* [lab=agent] {} {{M_{A}} \bc (\vec{x})M_{P} \;| \; \clift{P_0,\ldots,M_{P},\ldots,P_N}}
  \and \\
  \inferrule* [lab=process] {} {{M_{P}} \bc M_{N} \;| \;P|M_{P} }
\end{mathpar} 

\begin{mathpar}
  \inferrule* [lab=sychronization] {} {M_{N} \bc \Box \;|\; x?M_{F} \;|\; x!M_{C}}
  \and
  \inferrule* [lab=abstraction] {} {{M_{F}} \bc (x)M_{P} }
  \and
  \inferrule* [lab=concretion] {} {{M_{C}} \bc \langle M_{P} \rangle }
  \and \\
  \inferrule* [lab=process] {} {{M_{P}} \bc M_{N} \;| \;P|M_{P} }
\end{mathpar}

\begin{definition}[contextual application] Given a context $M$, and
  process $P$, we define the \emph{contextual application}, $M[P] :=
  M\{P/\Box\}$. That is, the contextual application of M to P is the
  substitution of $P$ for $\Box$ in $M$.
\end{definition}

$\meaningof{-} : L \to \mathcal{P}(\pi)$

\begin{mathpar}
  \inferrule* [lab=collection] {} {\meaningof{true} = \pi, \and \meaningof{~E} = \pi \setminus \meaningof{E}, \and \meaningof{E_{1} \& E_{2}} = \meaningof{E_{1}} \cap \meaningof{E_{2}}}
\end{mathpar}

\begin{mathpar}
  \inferrule* [lab=structure] {} {\meaningof{0} = \{ P \in \pi | P \equiv 0 \}, \and \\ \meaningof{E_1 | E_2} = \{ P \in \pi | P \equiv P_{1} | P_{2}, P_{1} \in \meaningof{E_{1}}, P_{2} \in \meaningof{E_2}\} }
\end{mathpar}

\begin{mathpar}
 \inferrule* [lab=behavior] {} {\meaningof{\langle a?b \rangle E} = \{ P \in \pi | P \equiv Q | u?(y)P', \\ \and \\\\ \and \\ \;\;\; u \in \meaningof{a}, \forall z.P'\{z/y\} \in \meaningof{E\{z/b\}}\}, \and \\ \meaningof{a!E} = \{ P \in \pi | P \equiv Q | x!\langle P' \rangle, x \in \meaningof{a} P' \in \meaningof{E}\} }
\end{mathpar}

\begin{mathpar}
 \inferrule* [lab=nominal] {} {\meaningof{\quotep{E}} = \{ \quotep{P} \in \quotep{\pi} | P \in \meaningof{E} \}, \and \meaningof{\quotep{P}} = \{ \quotep{Q} \in \quotep{\pi} | P \equiv Q \} \and \\ \meaningof{@\quotep{E}} = \{ P \in \pi | P \equiv @x, x \in \meaningof{E} \}}
\end{mathpar}

\begin{eqnarray*}
  \\
  \meaningof{-} : TS \to ST
\end{eqnarray*}

\begin{eqnarray*}
  \\
  L : TS \to ST
\end{eqnarray*}

\begin{eqnarray*}
  \\
  P \models E \iff P \in \meaningof{E}
\end{eqnarray*}

\begin{eqnarray*}
  P \approx_{L} Q \iff \forall E \in L. P \models E \iff Q \models E
\end{eqnarray*}

\begin{eqnarray*}
  P \approx_{K} Q
\end{eqnarray*}

\begin{eqnarray*}
  P \approx Q
\end{eqnarray*}

$\approx_{K} = \approx = \approx_{L}$

\subsubsection{Contextual duality}

Note that contexts extend the quotation operation to a family of
operations from processes to names. Given a context, $M$, we can
define a \emph{nominal context}, $\quotep{M}$ by $\quotep{M}[P] :=
\quotep{M[P]}$. To foreshadow what is to come we observe that these
operations enjoy a duality with processes very much like the duality
between vectors and maps from vectors to scalars.

Further, because the calculus is essentially higher-order, we have a
correspondence between contexts and processes. More specifically,
given a name $x$ and a context $M$ we can construct $M^{*}_{x}$ such
that 

\begin{mathpar}
  M^{*}_{x} | \lift{x}{P} \red M[P]
\end{mathpar}

namely,

\begin{mathpar}
  M^{*}_{x} := x?(u).M[\dropn{u}]
\end{mathpar}

The dependence of $M^{*}_{x}$ on a name makes it an abstraction, 

\begin{mathpar}
  M^{*} := (x)x?(u).M[\dropn{u}]
\end{mathpar}

\subsection{Additional notation}

It will sometimes be convenient to denote the process a name
quotes. We already have the notation $x = \quotep{P}$, but it will be
convenient to introduce an alternate notation, $\procn{x}$, when we
want to emphasize the connection to the use of the name. Note that, by
virtue of name equivalence, $\quotep{\procn{x}} \nameeq x$; so, the
notation is consistent with previous definitions.

Further, because names have structure it is possible to effect
substitutions on the basis of that structure. This means we need to
upgrade our notation for substitutions, which we accomplish by
adapting comprehension notation. Thus,

\begin{mathpar}
  P\{ y / x : x \in S \}
\end{mathpar}

is interpreted to mean the process derived from P by replacing (in a
capture-avoiding manner) each occurrence of $x$ in $S$ by $y$. For example,

\begin{mathpar}
  P\{ \quotep{\procn{x}|\procn{x}} / x : x \in \freenames{P} \}
\end{mathpar}

will replace each (occurrence) of a free name $x$ in $P$ by
$\quotep{\procn{x}|\procn{x}}$.

Also, we will avail ourselves of the notation $x^{L}$ and $x^{R}$ to
denote injections of a name into disjoint copies of the name
space. There are numerous ways to accomplish this. One example can be
found in \cite{MeredithR05}. This notation overloads to vectors of
names: $\vec{x}^{\pi} := (x_{i}^{\pi} \; : \; 0 \leq i < |\vec{x}| )$ where $\pi \in \{L,R\}$.

We also use $P^{\Box} := P|\Box$.

In \cite{MeredithR05} an interpretation of the new operator is
given. It turns out that there are several possible interpretations
all enjoying the requisite algebraic properties of the operator (see
\cite{milner91polyadicpi}). We will therefore make liberal use of
$(\nu\; \vec{x})P$.

% subsection the_syntax_and_semantics_of_the_notation_system (end)   

\input{qm2pi.qmops} 

\input{qm2pi.sterngerlach} 

\input{qm2pi.metric} 

% section concurrent_process_calculi (end)

%\input{qm2pi.proofsketch}

% section proof sketch (end)

%\input{qm2pi.slviaknots} 

% section spatial logic via knots (end)

\input{qm2pi.conclusion}

% section conclusion (end)

%\input{qm2pi.dtcodes} 

% section wiring algorithm (end)

\input{qm2pi.ack} 

% section acknowledgments (end)

\newpage


\bibliographystyle{plain}   
\bibliography{../../biblios/main.bib}

\input{qm2pi.rhodetails}

\end{document}



% section proof sketch (end)

%\section{Unlikely characters: spatial logic for
  knots}\label{sub:characteristic_formulae} % (fold)

Associated to the mobile process calculi are a family of logics known
as the Hennessy-Milner logics. These logics typically enjoy a
semantics interpreting formulae as sets of processes that when
factored through the encoding outlined above allows an identification
of classes of knots with logical formulae. In the context of this
encoding the sub-family known as the spatial logics \cite{CairesC03}
\cite{CairesC04} \cite{Caires04} are of particular interest providing
several important features for expressing and reasoning about
properties (i.e. classes) of knots. We hint here at how this may be done.

%\begin{description}
%\item [structural connectives] 
\subsubsection{Structural connectives} The spatial logics enjoy
structural connectives corresponding, at the logical level, to the
parallel composition ($P | Q$) and new name ($(\nu \; x)P$)
connectives for processes. As illustrated in the examples below, these
connectives are extremely expressive given the shape of our encoding.
%\item [decideable satisfaction]

\subsubsection{Decideable satisfaction}
In \cite{Caires04} the satisfaction relation is shown to be decideable
for a rich class of processes. It further turns out that the image of
the our encoding is a proper subset of that class. This result
provides the basis for an algorithm by which to search for knots
enjoying a given property.
%\item [characteristic formulae]

\subsubsection{Characteristic formulae}
In the same paper \cite{Caires04} , Caires presents a means of calculating
characteristic formulae, selecting equivalence classes of processes
up to a pre--specified depth limit on the support set of names. Composed with our
encoding, this characteristic formula can be used to select
characteristic formulae for knots.
%\end{description}

\subsubsection{Spatial logic formulae}

The grammar below (segmented for comprehension) summarizes the syntax
of spatial logic formulae. We employ illustrative examples in the
sequel to provide an intuitive understanding of their meaning
referring the reader to \cite{Caires04} for a more detailed explication
of the semantics.

\begin{mathpar}
  \inferrule* [lab=boolean] {} {{A,B} \bc T \;|\; \neg A \;|\; A \wedge B \;|\; \eta = \eta'}
  \and
  \inferrule* [lab=spatial] {} {|\; \pzero \;|\; A | B \;|\; x \text{\textregistered} A \;|\; \forall x . A \;|\;  H x . A}
  \and
  \inferrule* [lab=behavioral] {} {|\; \alpha . A}
  \and 
  \inferrule* [lab=recursion] {} {|\; X(\vec{u}) \;|\; \mu X(\vec{u}) . A}
  \and
  \inferrule* [lab=action] {} {\alpha \bc \langle x?(\vec{y}) \rangle \;|\; \langle x!(\vec{y}) \rangle \;|\; \langle \tau \rangle}
  \and 
  \inferrule* [lab=name] {} {\eta \bc x \;|\; \tau}
\end{mathpar} 

% subsection characteristic_formulae (end)   	 

\subsection{Example formulae}\label{sub:example_formulae_} % (fold)

\subsubsection{Crossing as formula.}
% 
% \begin{align*}
%   \frac{d}{dx} \sin x &= \cos x 
%   & \frac{d}{dx} e^x &= e^x \\
%   \frac{d}{dx} \cos x &= - \sin x 
%   & \frac{d}{dx} \log x &= \frac{1}{x} \\
% \end{align*} 

\begin{align*}
 \mu C(x_{0},x_{1},y_{0},y_{1},u).&(\langle x_{0}?(z) \rangle(\langle u! \rangle\langle y_{1}!z \rangle C(x_{0},x_{1},y_{0},y_{1},u)) & \\
  & \wedge \langle y_{1}?(z) \rangle (\langle u! \rangle \langle x_{0}!z \rangle C(x_{0},x_{1},y_{0},y_{1},u)) & \\
  & \wedge \langle x_{1}?(z) \rangle (\langle u? \rangle \langle y_{0}!z \rangle C(x_{0},x_{1},y_{0},y_{1},u)) & \\
  & \wedge \langle y_{0}?(z) \rangle (\langle u? \rangle \langle x_{1}!z \rangle C(x_{0},x_{1},y_{0},y_{1},u))) &
\end{align*}

The lexicographical similarity between the shape of this formulae and
the shape of definition of the process representing a crossing reveals
the intuitive meaning of this formulae. It describes the capabilities
of a process that has the right to represent a crossing. For example
it picks out processes that may perform an input on the port $x_0$ in
its initial menu of capabilities. What differentiates the formula
from the process, however, is that the crossing process is the
smallest candidate to satisfy the formula. Infinitely many other
processes -- with internal behavior hidden behind this interface, so
to speak -- also satisfy this formula. Even this simple formula,
then, can be seen to open a new view onto knots, providing a
computational interpretation of \emph{virtual} knots.

Note that this formula is derived by hand. A similar formula can be
derived by employing Caires' calculation of characteristic formula
\cite{Caires04} to the process representing a crossing. In light of
this discussion, we let
$\meaningof{C}_{\phi}(x0,x1,y0,y1,u)$ denote a formula specifying the
dynamics we wish to capture of a crossing. To guarantee we preserve
the shape of the interface and minimal semantics we demand that
$\meaningof{C}_{\phi}(x0,x1,y0,y1,u) \Rightarrow
\textbf{C}(x0,x1,y0,y1,u)$ where $\textbf{C}(x0,x1,y0,y1,u)$ denotes
the formula above.
                            
\subsubsection{Crossing number constraints.}
The moral content of the context lemma (Lemma \ref{context}) is that the notion of
``locality'' in the Reidemeister moves is effectively captured by the
parallel composition operator of the process calculus. This intuition
extends through the logic. Given a formula,
$\meaningof{C}_{\phi}(x0,x1,y0,y1,u)$, we can use the structural
connectives to specify constraints on crossing numbers, such as at
least $n$ crossings, or exactly $n$ crossings.
\begin{mathpar}
  \inferrule* [lab=at-least-n] {} { K^{\geq n}_{\phi}(\vec{xs},\vec{ys}) := \Pi_{i=0}^{n-1} Hu . \meaningof{C}_{\phi}(xs_i,ys_i,u) | T }
  \and 
  \inferrule* [lab=exactly-n] {} { K^{= n}_{\phi}(\vec{xs},\vec{ys}) := \Pi_{i=0}^{n-1} Hu . \meaningof{C}_{\phi}(xs_i,ys_i,u) | \neg (\forall x_0,y_0,x_1,y_1,u . \meaningof{C}_{\phi}(x_0,y_0,x_1,y_1,u) | T) }
\end{mathpar}

To round out this section, recall that the encoding of an $n$-crossing
knot decomposes into a parallel composition of $n$ \emph{copies} of a
crossing process together with a wiring harness. To specify different
knot classes with the same crossing number amounts to specifying
logical constraints on the wiring harness. In the interest of space,
we defer examples to a forthcoming paper. Suffice it to say that both
the conditions ``alternating knot'' and ``contains the tangle
corresponding to 5/3'' are expressible. For example, it is possible to
calculate the characteristic formula of a process corresponding to the
tangle 5/3 and conjoin it into the classifying formula via the
composition connective of the logic.

Finally, we wish to observe that it is entirely within reason to
contemplate a more domain-specific version of spatial logic tailored
to the shape of processes in the image of the encoding. Such a
domain-specific logic would have a better claim to the title formal
language of knot properties.

% subsection example_formulae_ (end)

% section knots_as_processes (end) 

% section spatial logic via knots (end)

\section{Conclusions and future work}

\paragraph{Testing physical space}
You, gentle reader, may wonder why of all the theorems to be proved
given this set up we pick the one above. In some sense it's hardly
central to quantum mechanics. We see it as central in the sense that
it firmly establishes a notion of physical space arising from a notion
of the equivalence of behavior. Relating bisimulation to a metric is a
big step forward, but one is faced with interpreting the relationship
of that metric space to something more physical. Quantum mechanical
notions of ``physical'' space are still far from intuitive, but by
relating this idea of distance as testing to calculations that predict
physical circumstances we are making a not insignificant step forward
toward an understanding of the physical space we inhabit as
essentially dynamic.

\paragraph{Effectivity and simulation}
One of the observations we have yet to make is that the entire program
spelled out here is effective. We have built various interpreters for
the reflective calculus at work in this interpretation. In principle,
then, we can simulate quantum mechanics on a computer. The place where
the simulation may lose fidelity is the infinitely branching summation
for the annihilator.

In this connection i also want to point out that the evaluation style
calculation of the inner product puts the non-determinism of the
summation right at the heart of measurement. This suggests that
Milner's original reduction-based formulation of the dynamics of his
calculi in terms of sums was not just notationally suggestive of a
notion of measure-and-continue but captured some significant part of
the physics.

\paragraph{Quantum continuations}
In light of this last observation i want to point out that the
predominant account of quantum mechanics is missing a key aspect of a
truly compositional story of the physical situation. In a real lab,
when a measurement is made the observation can be made to feed into
another device that then makes another measurement conditioned on the
results of the first. This means that after the superposition was
collapsed the entire experimental set up remained in
superposition. While QM offers a means of writing this down it doesn't
quite line up well with the well-trodden formulation of computation
and continuation that we see so succinctly expressed in Milner's
calculi. This suggests that there might be advantages to this account
of dynamics waiting to be explored.

\paragraph{Quantum logic}
In this connection, we also note that by virtue of having the
Hennessy-Milner construction, we can pull the construction through the
interpretation of QM. This gives us a natural candidate for a quantum
logic that enjoys an extremely tight connection with it's domain of
interpretation, making the construction much less ad hoc (rather it is
the image of functor!).

\paragraph{Quantum probabiity}
i have questions about the basis of the interpretation of inner
product as probability amplitude. In particular, using which
axiomatization of probability theory does the notion of probability
amplitude earn the right to be so dubbed? In other words, where is the
proof that the operation for calculating a probability amplitude (and
then squaring) satisfies the axioms of what it means to calculate a
probability? Even if such a proof exists (i have yet to find it in the
literature), i wonder if it might not be possible to turn things on
their heads. Can we view the calculation of the probability amplitude
as an axiomatization of probability? If so, then the definition we
give for calculating probability amplitude may provide the basis for
an \emph{effective} theory of probability.

\paragraph{Quantum vs ``biological'' information}
Finally, i want to conclude with a more philosophical observation. At
a recent workshop in which QM was a predominant topic i noticed
something about quantum information. The speaker was giving a riveting
discussion of axiomatic QM and showing how properties of ``no
cloning'' and ``no deleting'' emerged as consequences of the
axiomatization. Theorems of this form are necessary to give us a sense
of confidence that our axioms characterize the physical theory. What
struck me, though, was that if quantum information is neither erasable
nor replicable it is markedly different from \emph{life}. Two of the
things we know about life is that

\begin{itemize}
  \item it ends;
  \item to gain some measure of persistence, to transcend it's
    finitude it is imminently copyable.
\end{itemize}

Both of these qualities are summarized succinctly in the aphorism: all
flesh is grass. For me these two kinds of ``information'' -- call them
quantum and biological -- are end points on a spectrum of strategies
for persistence. At one end, we have those curious entities that enjoy
uniqueness and permanence; at the other, we have those who in the face
of a certain end and an uncertain present make a go of passing
something on. To me one of the more remarkable aspects of the latter
strategy is that in the presence of noise (and certain features of
copying) we get a kind of dynamism, a chance for improvement against a
given persistent condition.

% subsection other_calculi_other_bisimulations_and_geometry_as_behavior (end)




% section conclusion (end)

%\documentclass[12pt]{llncs}
%\documentclass{jktr}

\usepackage[pdftex]{hyperref}                   
\usepackage {listings}
\usepackage {mathpartir}
\usepackage{bcprules}
%\usepackage{listings}
                       
\usepackage{graphicx} 
%\usepackage[margins=2.5cm,nohead,nofoot]{geometry}
%\usepackage{geometry}
\usepackage{amsfonts}
\usepackage{amstext}
\usepackage{latexsym}
\usepackage{amssymb}
\usepackage{color}


%\include{myPreamble}
\include{qm2pi.local} 

%\ifpdf
%\usepackage[pdftex]{graphicx}
%\else
%\usepackage{graphicx}
%\fi

 % \ifpdf
%  \usepackage{pdfsync}
%  \if


%\title{Brief Article}
%\author{David F. Snyder}
%\author{L.G. Meredith}

%\address{Dept. of Math., Texas State University--San Marcos, San Marcos, TX 78666}
       
\pagestyle{empty}


\begin{document}

\lstset{language=[Objective]Caml,frame=shadowbox}

\input{qm2pi.front}

% section front matter (end)

\input{qm2pi.intro} 
 
% section introduction (end)

% \input{qm2pi.knotations} 

% section notation (end)

\input{qm2pi.process.calculi} 

% section concurrent_process_calculi_and_spatial_logics_ (end)
    
%\input{qm2pi.knots2pi} 

%\input{qm2pi.trefoil} 

%\input{qm2pi.mainthm} 

% subsection basic_interpretation (end)

%\input{qm2pi.rho.presentation} 
\subsection{The syntax and semantics of the notation system}\label{sub:the_syntax_and_semantics_of_the_notation_system} % (fold)

We now summarize a technical presentation of the calculus that
embodies our theory of dynamics. The typical presentation of such a
calculus follows the style of giving generators and relations on
them. The grammar, below, describing term constructors, freely
generates the set of processes, $\Proc$. This set is then quotiented
by a relation known as structural congruence and it is over this set
that the notion of dynamics is expressed. This presentation is
essentially that of \cite{MeredithR05} with the addition of
polyadicity and summation. For readability we have relegated some of
the technical subtleties to an appendix.

\subsubsection{Process grammar}\label{subsub:process_grammar}

\begin{mathpar}
  \inferrule* [lab=synchronization] {} {{M} \bc \pzero \;|\; x?F \;|\; x!C }
  \and
  \inferrule* [lab=abstraction] {} {{F} \bc (x)P}
  \and
  \inferrule* [lab=concretion] {} {{C} \bc \langle Q \rangle}
  \and
  \inferrule* [lab=process] {} {{P,Q} \bc M \;| \;P|Q \;|\; @{x}}
  \and
  \inferrule* [lab=name] {} {{x} \bc \quotep{P}}
\end{mathpar} 

Note that $\vec{x}$ (resp. $\vec{P}$) denotes a vector of names
(resp. processes) of length $|\vec{x}|$ (resp. $|\vec{P}|$). We adopt
the following useful abbreviations.

\begin{mathpar}
   x?(\vec{y}).P := x.(\vec{y})P \and  x\clift{\vec{P}} := x.\clift{\vec{P}}
   \and x!(y) := \lift{x}{\dropn{y}}
   \and \Pi_{i=0}^{n-1}P_i := P_0 | \ldots | P_{n-1}
\end{mathpar}

\subsubsection{Structural congruence}

\paragraph{Free and bound names and alpha-equivalence.} At the
core of structural equivalence is alpha-equivalence which identifies
process that are the same up to a change of variable. Formally, we
recognize the distinction between free and bound names. The free names
of a process, $\freenames{P}$, may be calculated recursively as
follows:

\begin{mathpar}
\freenames{\pzero} := \emptyset
  \and \\
  \freenames{x?(y).P} := \{ x \} \cup (\freenames{P} \setminus \{ y \})
  \and 
  \freenames{x!\langle P \rangle} := \{ x \} \cup \{ P \} 
  \and \\
  \freenames{P|Q} := \freenames{P} \cup \freenames{Q}
  \and \\
  \freenames{@{x}} := \{ x \}
\end{mathpar}

$\pi$
$\quotep{\pi}$

$\freenames{-} : \pi \to \mathcal{P}(\quotep{\pi})$

\begin{eqnarray*}
  \freenames{\pzero} & := & \emptyset \\
  \freenames{x?(y).P} & := & \{ x \} \cup (\freenames{P} \setminus \{ y \}) \\
  \freenames{x!\langle P \rangle} & := & \{ x \} \cup \{ P \} \\
  \freenames{P|Q} & := & \freenames{P} \cup \freenames{Q} \\
  \freenames{\dropn{x}} & := & \{ x \}
\end{eqnarray*}

The bound names of a process, $\boundnames{P}$, are those names occurring in $P$
that are not free. For example, in $x?(y).0$, the name $x$ is free, while $y$ is bound.

\begin{mathpar}
  \inferrule* [lab=monoidal-laws] {} { P|Q \equiv Q|P \and P|0 \equiv P \and P|(Q|R) \equiv (P|Q)|R }
\end{mathpar}

\begin{mathpar}
  \inferrule* [lab=alpha-equivalence] {} { (x)P \equiv (y)P\{y/x\} \and y \not\in \freenames{P} }
\end{mathpar}

\begin{definition}
Then two processes, $P,Q$, are alpha-equivalent if $P = Q\{\vec{y}/\vec{x}\}$ for
some $\vec{x} \in \boundnames{Q},\vec{y} \in \boundnames{P}$, where $Q\{\vec{y}/\vec{x}\}$
denotes the capture-avoiding substitution of $\vec{y}$ for $\vec{x}$ in $Q$.
\end{definition}

\begin{definition}
  The {\em structural congruence} \cite{SangiorgiWalker} , $\equiv$,
  between processes is the least congruence containing
  alpha-equivalence, satisfying the abelian monoid laws
  (associativity, commutativity and $\pzero$ as identity) for parallel
  composition $|$ and for summation $+$.
\end{definition}

\subsection{Name equivalence}

We take name equivalence, written $\nameeq$, to be the smallest
equivalence relation generated by the following rules.

\begin{mathpar}
\inferrule*[lab=Quote-drop]
{ }
{ \quotep{@{x}} \nameeq x }

\inferrule*[lab=Struct-equiv]
{ P \scong Q }
{ \quotep{P} \nameeq \quotep{Q} }
\end{mathpar}

The astute reader will have noticed that the mutual recursion of names
and processes imposes a mutual recursion on alpha-equivalence and
structural equivalence via name-equivalence. Fortunately, all of this
works out pleasantly and we may calculate in the natural way, free of
concern. The reader interested in the details is referred to the
appendix \ref{appendix:rho_details}.

\subsection{Substitution}

We use $\Proc$ for the set of processes, $\QProc$ for the set of
names, and $\id{\{}\vec{y} / \vec{x} \id{\}}$ to denote partial maps,
$s : \QProc \rightarrow \QProc$. A map, $s$ lifts, uniquely, to a map
on process terms, $\widehat{s} : \Proc \rightarrow \Proc$ by the
following equations.

\begin{mathpar}
  (0) \psubstp{Q}{P} := 0 \\
  (R \juxtap S) \psubstp{Q}{P}
  :=    
  (R)\psubstp{Q}{P} \juxtap (S) \psubstp{Q}{P} \\
  (x?(y).R) \psubstp{Q}{P}    
  :=    
  (x)\substp{Q}{P} (z)\concat( (R \psubstn{z}{y}) \psubstp{Q}{P} ) \\
  (\lift{x}{R}) \psubstp{Q}{P}  
  :=
  \lift{(x)\substp{Q}{P}}{ R \psubstp{Q}{P} } \\
%   (\dropn{x})  \psubstp{Q}{P}       
%   := 
%   \left\{ 
%     \begin{array}{ccc} 
%       \dropn{\quotep{Q}} & & x \nameeq \quotep{P} \\
%       \dropn{x} & & otherwise \\
%     \end{array}
%   \right. 
  (\dropn{x})  \psubstp{Q}{P}       
  := 
  \left\{ 
    \begin{array}{ccc} 
      Q & & x \nameeq \quotep{P} \\
      \dropn{x} & & otherwise \\
    \end{array}
  \right.
\end{mathpar}
 

where

\begin{eqnarray}
  (x)\id{\{} \lpquote Q \rpquote / \lpquote P \rpquote \id{\}}            = 
  \left\{ 
    \begin{array}{ccc}
      \lpquote Q \rpquote & & x \nameeq \lpquote P \rpquote \\
      x & & otherwise \\
    \end{array}
  \right. \nonumber
\end{eqnarray}

and $z$ is chosen distinct from $\quotep{P}$, $\quotep{Q}$, the free
names in $Q$, and all the names in $R$. Our $\alpha$-equivalence will
be built in the standard way from this substitution.

\begin{remark}\label{rem:no_self_referential_names}
  One consequence of these definitions is that $\forall P. \quotep{P}
  \not\in \freenames{P}$.
\end{remark}

\subsection{ Dynamic quote: an example }

Anticipating something of what's to come, consider applying the
substitution, $\widehat{\id{\{}u / z \id{\}}}$, to the following pair
of processes, $\lift{w}{y!(z)}$ and $w[ \lpquote y!(z) \rpquote ]$.

\begin{eqnarray}
	\lift{w}{y!(z)}\widehat{\id{\{}u / z \id{\}}}
		& = &
		\lift{w}{y!(u)} \nonumber\\
	w[ \lpquote y!(z) \rpquote ] \widehat{ \id{\{}u / z \id{\}} }
		& = &
		w[ \lpquote y!(z) \rpquote ] \nonumber
\end{eqnarray}

Because the body of the process between quotes is impervious to
substitution, we get radically different answers. In fact, by
examining the first process in an input context,
e.g. $x?(z).\lift{w}{y!(z)}$, we see that the process under the lift
operator may be shaped by prefixed inputs binding a name inside it. In
this sense, the lift operator will be seen as a way to dynamically
construct processes before reifying them as names.

Finally equipped with these standard features we can present the
dynamics of the calculus.

\subsubsection{Operational semantics} 

Finally, we introduce the computational dynamics. What marks these
algebras as distinct from other more traditionally studied algebraic
structures, e.g. vector spaces or polynomial rings, is the manner in
which dynamics is captured. In traditional structures, dynamics is typically
expressed through morphisms between such structures, as in linear maps
between vector spaces or morphisms between rings. In algebras
associated with the semantics of computation, the dynamics is
expressed as part of the algebraic structure itself, through a
reduction reduction relation typically denoted by $\red$. Below, we
give a recursive presentation of this relation for the calculus used
in the encoding.

$\red \subseteq \pi \times \pi$
$\red : \pi \to \mathcal{P}(\pi)$

\begin{mathpar}
  \inferrule* [lab=Comm] { \textsf{match}( x_{src}, x_{trgt} ) } { x_{trgt}?(y)P \; | \; x_{src}!\langle {Q} \rangle \red P\{\quotep{Q}/y}\} }
  \and \\
  \inferrule* [lab=Par] {{P} \red {P}'} {{{P} | {Q}} \red {{P}' | {Q}}}
  \and
  \inferrule* [lab=Equiv]{{{P} \scong {P}'} \andalso {{P}' \red {Q}'} \andalso {{Q}' \scong {Q}}}{{P} \red {Q}}
\end{mathpar}

\begin{eqnarray*}
  match_{\equiv} (\quotep{P},\quotep{Q}) & := & P \equiv Q \\
  match_{\dagger}(\quotep{P},\quotep{Q}) & := & \forall R. P|Q \red^{*} R => R \red^{*} 0 \\
  match_{K}(\quotep{P},\quotep{Q}) & := & K \mbox{ for some context } K
\end{eqnarray*}

$u?(x)P | u!\langle Q \rangle \red P\{\quotep{Q}/x\}$

%We write $\wred$ for $\red^*$, and $P\red$ if $\exists Q $ such that $ P \red Q$.
We write $P\red$ if $\exists Q $ such that $ P \red Q$ and $P\not\red$, otherwise.

\section{Replication}

As mentioned before, it is known that replication (and hence
recursion) can be implemented in a higher-order process algebra
\cite{SangiorgiWalker}. As our first example of calculation with the
machinery thus far presented we give the construction explicitly in
the {\rhoc}.

\begin{eqnarray}
	D_{x} & := & \prefix{x}{y}{(\binpar{\outputp{x}{y}}{@{y}})} \nonumber\\
	\bangp_{x}{P} & := & \binpar{{x}!\langle{\binpar{D_{x}}{P}}\rangle}{D_{x}} \nonumber
\end{eqnarray}

\begin{eqnarray}
	\bangp_{x}{P} & & \nonumber\\
	=
	& {x}!\langle{(\prefix{x}{y}{(\outputp{x}{y} | @{y})) | P}}\rangle 
	      | \prefix{x}{y}{(\outputp{x}{y} | @{y})} & \nonumber\\
	\red
	& (\outputp{x}{y} | @{y})\substn{\quotep{(\prefix{x}{y}{(@{y} | \outputp{x}{y})) | P}}}{y} & \nonumber\\
	=
	& \outputp{x}{\quotep{(\prefix{x}{y}{(\outputp{x}{y} | @{y})) | P}}}
	  | {(\prefix{x}{y}{(\outputp{x}{y} | @{y})) | P}} & \nonumber\\
	\red
	& \ldots & \nonumber\\
	\red^*
	& P | P | \ldots & \nonumber
\end{eqnarray}

Of course, this encoding, as an implementation, runs away, unfolding
$\bangp{P}$ eagerly. A lazier and more implementable replication
operator, restricted to input-guarded processes, may be obtained as follows.

\begin{eqnarray}
\bangp{\prefix{u}{v}{P}} 
	:= 
	\binpar{\lift{x}{\prefix{u}{v}{(\binpar{D(x)}{P})}}}{D(x)} \nonumber
\end{eqnarray}

\begin{remark}
  Note that the lazier definition still does not deal with summation
  or mixed summation (i.e. sums over input and output). The reader is
  invited to construct definitions of replication that deal with these
  features. 

  Further, the definitions are parameterized in a name, $x$. Can you,
  gentle reader, make a definition that eliminates this parameter and
  guarantees no accidental interaction between the replication
  machinery and the process being replicated -- i.e. no accidental
  sharing of names used by the process to get its work done and the
  name(s) used by the replication to effect copying. This latter
  revision of the definition of replication is crucial to obtaining
  the expected identity $!!P \sim !P$.
\end{remark}

\begin{remark}\label{rem:paradoxical_combinator}
  The reader familiar with the lambda calculus will have noticed the
  similarity between $D$ and the paradoxical combinator.

  [Ed. note: the existence of this seems to suggest we have to be more
  restrictive on the set of processes and names we admit if we are to
  support no-cloning.]
\end{remark}

\subsubsection{Bisimulation}

The computational dynamics gives rise to another kind of equivalence,
the equivalence of computational behavior. As previously mentioned
this is typically captured \emph{via} some form of bisimulation.

% The notion we use in this paper is weak barbed bisimulation
% \cite{milner91polyadicpi}.

The notion we use in this paper is derived from weak barbed
bisimulation \cite{milner91polyadicpi}. 

\begin{definition}
An \emph{observation relation}, $\downarrow_{\mathcal N}$, over a set
of names, $\mathcal N$, is the smallest relation satisfying the rules
below.

\infrule[Out-barb]{y \in {\mathcal N}, \; x \nameeq y}
		  {\outputp{x}{v} \downarrow_{\mathcal N} x}
\infrule[Par-barb]{\mbox{$P\downarrow_{\mathcal N} x$ or $Q\downarrow_{\mathcal N} x$}}
		  {\binpar{P}{Q} \downarrow_{\mathcal N} x}

We write $P \Downarrow_{\mathcal N} x$ if there is $Q$ such that 
$P \wred Q$ and $Q \downarrow_{\mathcal N} x$.
\end{definition}

\begin{definition}
%\label{def.bbisim}
An  ${\mathcal N}$-\emph{barbed bisimulation} over a set of names, ${\mathcal N}$, is a symmetric binary relation 
${\mathcal S}_{\mathcal N}$ between agents such that $P\rel{S}_{\mathcal N}Q$ implies:
\begin{enumerate}
\item If $P \red P'$ then $Q \wred Q'$ and $P'\rel{S}_{\mathcal N} Q'$.
\item If $P\downarrow_{\mathcal N} x$, then $Q\Downarrow_{\mathcal N} x$.
\end{enumerate}
$P$ is ${\mathcal N}$-barbed bisimilar to $Q$, written
$P \wbbisim_{\mathcal N} Q$, if $P \rel{S}_{\mathcal N} Q$ for some ${\mathcal N}$-barbed bisimulation ${\mathcal S}_{\mathcal N}$.
\end{definition}

$\mathcal{R} \subseteq \pi \times \pi$

$P \mathcal{R} Q => \forall P'. P \red P' \Rightarrow \exists Q'. Q \red Q', P' \mathcal{R} Q'$

$P \vdash x \Rightarrow Q \vdash x$

\begin{mathpar}
  \inferrule*[lab=Out-barb]{x \nameeq y}{{y}!\langle{Q}\rangle \vdash x}
  \and
  \inferrule*[lab=Par-barb]{\mbox{$P\vdash x$ or $Q\vdash x$}}{\binpar{P}{Q} \vdash x}
\end{mathpar}

\subsubsection{Contexts}

One of the principle advantages of computational calculi like the
$\pi$-calculus is a well-defined notion of context,
contextual-equivalence and a correlation between
contextual-equivalence and notions of bisimulation. The notion of
context allows the decomposition of a process into (sub-)process and
its syntactic environment, its context. Thus, a context may be
thought of as a process with a ``hole'' (written $\Box$) in it. The
application of a context $M$ to a process $P$, written $M[P]$, is
tantamount to filling the hole in $M$ with $P$. In this paper we do
not need the full weight of this theory, but do make use of the notion
of context in the proof the main theorem. 

\begin{mathpar}
  \inferrule* [lab=summation] {} {{M_{M},M_{N}} \bc \Box \;|\; x.M_{A} \;|\; M_{M}+M_{N}}
  \and
  \inferrule* [lab=agent] {} {{M_{A}} \bc (\vec{x})M_{P} \;| \; \clift{P_0,\ldots,M_{P},\ldots,P_N}}
  \and \\
  \inferrule* [lab=process] {} {{M_{P}} \bc M_{N} \;| \;P|M_{P} }
\end{mathpar} 

\begin{mathpar}
  \inferrule* [lab=sychronization] {} {M_{N} \bc \Box \;|\; x?M_{F} \;|\; x!M_{C}}
  \and
  \inferrule* [lab=abstraction] {} {{M_{F}} \bc (x)M_{P} }
  \and
  \inferrule* [lab=concretion] {} {{M_{C}} \bc \langle M_{P} \rangle }
  \and \\
  \inferrule* [lab=process] {} {{M_{P}} \bc M_{N} \;| \;P|M_{P} }
\end{mathpar}

\begin{definition}[contextual application] Given a context $M$, and
  process $P$, we define the \emph{contextual application}, $M[P] :=
  M\{P/\Box\}$. That is, the contextual application of M to P is the
  substitution of $P$ for $\Box$ in $M$.
\end{definition}

$\meaningof{-} : L \to \mathcal{P}(\pi)$

\begin{mathpar}
  \inferrule* [lab=collection] {} {\meaningof{true} = \pi, \and \meaningof{~E} = \pi \setminus \meaningof{E}, \and \meaningof{E_{1} \& E_{2}} = \meaningof{E_{1}} \cap \meaningof{E_{2}}}
\end{mathpar}

\begin{mathpar}
  \inferrule* [lab=structure] {} {\meaningof{0} = \{ P \in \pi | P \equiv 0 \}, \and \\ \meaningof{E_1 | E_2} = \{ P \in \pi | P \equiv P_{1} | P_{2}, P_{1} \in \meaningof{E_{1}}, P_{2} \in \meaningof{E_2}\} }
\end{mathpar}

\begin{mathpar}
 \inferrule* [lab=behavior] {} {\meaningof{\langle a?b \rangle E} = \{ P \in \pi | P \equiv Q | u?(y)P', \\ \and \\\\ \and \\ \;\;\; u \in \meaningof{a}, \forall z.P'\{z/y\} \in \meaningof{E\{z/b\}}\}, \and \\ \meaningof{a!E} = \{ P \in \pi | P \equiv Q | x!\langle P' \rangle, x \in \meaningof{a} P' \in \meaningof{E}\} }
\end{mathpar}

\begin{mathpar}
 \inferrule* [lab=nominal] {} {\meaningof{\quotep{E}} = \{ \quotep{P} \in \quotep{\pi} | P \in \meaningof{E} \}, \and \meaningof{\quotep{P}} = \{ \quotep{Q} \in \quotep{\pi} | P \equiv Q \} \and \\ \meaningof{@\quotep{E}} = \{ P \in \pi | P \equiv @x, x \in \meaningof{E} \}}
\end{mathpar}

\begin{eqnarray*}
  \\
  \meaningof{-} : TS \to ST
\end{eqnarray*}

\begin{eqnarray*}
  \\
  L : TS \to ST
\end{eqnarray*}

\begin{eqnarray*}
  \\
  P \models E \iff P \in \meaningof{E}
\end{eqnarray*}

\begin{eqnarray*}
  P \approx_{L} Q \iff \forall E \in L. P \models E \iff Q \models E
\end{eqnarray*}

\begin{eqnarray*}
  P \approx_{K} Q
\end{eqnarray*}

\begin{eqnarray*}
  P \approx Q
\end{eqnarray*}

$\approx_{K} = \approx = \approx_{L}$

\subsubsection{Contextual duality}

Note that contexts extend the quotation operation to a family of
operations from processes to names. Given a context, $M$, we can
define a \emph{nominal context}, $\quotep{M}$ by $\quotep{M}[P] :=
\quotep{M[P]}$. To foreshadow what is to come we observe that these
operations enjoy a duality with processes very much like the duality
between vectors and maps from vectors to scalars.

Further, because the calculus is essentially higher-order, we have a
correspondence between contexts and processes. More specifically,
given a name $x$ and a context $M$ we can construct $M^{*}_{x}$ such
that 

\begin{mathpar}
  M^{*}_{x} | \lift{x}{P} \red M[P]
\end{mathpar}

namely,

\begin{mathpar}
  M^{*}_{x} := x?(u).M[\dropn{u}]
\end{mathpar}

The dependence of $M^{*}_{x}$ on a name makes it an abstraction, 

\begin{mathpar}
  M^{*} := (x)x?(u).M[\dropn{u}]
\end{mathpar}

\subsection{Additional notation}

It will sometimes be convenient to denote the process a name
quotes. We already have the notation $x = \quotep{P}$, but it will be
convenient to introduce an alternate notation, $\procn{x}$, when we
want to emphasize the connection to the use of the name. Note that, by
virtue of name equivalence, $\quotep{\procn{x}} \nameeq x$; so, the
notation is consistent with previous definitions.

Further, because names have structure it is possible to effect
substitutions on the basis of that structure. This means we need to
upgrade our notation for substitutions, which we accomplish by
adapting comprehension notation. Thus,

\begin{mathpar}
  P\{ y / x : x \in S \}
\end{mathpar}

is interpreted to mean the process derived from P by replacing (in a
capture-avoiding manner) each occurrence of $x$ in $S$ by $y$. For example,

\begin{mathpar}
  P\{ \quotep{\procn{x}|\procn{x}} / x : x \in \freenames{P} \}
\end{mathpar}

will replace each (occurrence) of a free name $x$ in $P$ by
$\quotep{\procn{x}|\procn{x}}$.

Also, we will avail ourselves of the notation $x^{L}$ and $x^{R}$ to
denote injections of a name into disjoint copies of the name
space. There are numerous ways to accomplish this. One example can be
found in \cite{MeredithR05}. This notation overloads to vectors of
names: $\vec{x}^{\pi} := (x_{i}^{\pi} \; : \; 0 \leq i < |\vec{x}| )$ where $\pi \in \{L,R\}$.

We also use $P^{\Box} := P|\Box$.

In \cite{MeredithR05} an interpretation of the new operator is
given. It turns out that there are several possible interpretations
all enjoying the requisite algebraic properties of the operator (see
\cite{milner91polyadicpi}). We will therefore make liberal use of
$(\nu\; \vec{x})P$.

% subsection the_syntax_and_semantics_of_the_notation_system (end)   

\input{qm2pi.qmops} 

\input{qm2pi.sterngerlach} 

\input{qm2pi.metric} 

% section concurrent_process_calculi (end)

%\input{qm2pi.proofsketch}

% section proof sketch (end)

%\input{qm2pi.slviaknots} 

% section spatial logic via knots (end)

\input{qm2pi.conclusion}

% section conclusion (end)

%\input{qm2pi.dtcodes} 

% section wiring algorithm (end)

\input{qm2pi.ack} 

% section acknowledgments (end)

\newpage


\bibliographystyle{plain}   
\bibliography{../../biblios/main.bib}

\input{qm2pi.rhodetails}

\end{document}

 

% section wiring algorithm (end)

\documentclass[12pt]{llncs}
%\documentclass{jktr}

\usepackage[pdftex]{hyperref}                   
\usepackage {listings}
\usepackage {mathpartir}
\usepackage{bcprules}
%\usepackage{listings}
                       
\usepackage{graphicx} 
%\usepackage[margins=2.5cm,nohead,nofoot]{geometry}
%\usepackage{geometry}
\usepackage{amsfonts}
\usepackage{amstext}
\usepackage{latexsym}
\usepackage{amssymb}
\usepackage{color}


%\include{myPreamble}
\include{qm2pi.local} 

%\ifpdf
%\usepackage[pdftex]{graphicx}
%\else
%\usepackage{graphicx}
%\fi

 % \ifpdf
%  \usepackage{pdfsync}
%  \if


%\title{Brief Article}
%\author{David F. Snyder}
%\author{L.G. Meredith}

%\address{Dept. of Math., Texas State University--San Marcos, San Marcos, TX 78666}
       
\pagestyle{empty}


\begin{document}

\lstset{language=[Objective]Caml,frame=shadowbox}

\input{qm2pi.front}

% section front matter (end)

\input{qm2pi.intro} 
 
% section introduction (end)

% \input{qm2pi.knotations} 

% section notation (end)

\input{qm2pi.process.calculi} 

% section concurrent_process_calculi_and_spatial_logics_ (end)
    
%\input{qm2pi.knots2pi} 

%\input{qm2pi.trefoil} 

%\input{qm2pi.mainthm} 

% subsection basic_interpretation (end)

%\input{qm2pi.rho.presentation} 
\subsection{The syntax and semantics of the notation system}\label{sub:the_syntax_and_semantics_of_the_notation_system} % (fold)

We now summarize a technical presentation of the calculus that
embodies our theory of dynamics. The typical presentation of such a
calculus follows the style of giving generators and relations on
them. The grammar, below, describing term constructors, freely
generates the set of processes, $\Proc$. This set is then quotiented
by a relation known as structural congruence and it is over this set
that the notion of dynamics is expressed. This presentation is
essentially that of \cite{MeredithR05} with the addition of
polyadicity and summation. For readability we have relegated some of
the technical subtleties to an appendix.

\subsubsection{Process grammar}\label{subsub:process_grammar}

\begin{mathpar}
  \inferrule* [lab=synchronization] {} {{M} \bc \pzero \;|\; x?F \;|\; x!C }
  \and
  \inferrule* [lab=abstraction] {} {{F} \bc (x)P}
  \and
  \inferrule* [lab=concretion] {} {{C} \bc \langle Q \rangle}
  \and
  \inferrule* [lab=process] {} {{P,Q} \bc M \;| \;P|Q \;|\; @{x}}
  \and
  \inferrule* [lab=name] {} {{x} \bc \quotep{P}}
\end{mathpar} 

Note that $\vec{x}$ (resp. $\vec{P}$) denotes a vector of names
(resp. processes) of length $|\vec{x}|$ (resp. $|\vec{P}|$). We adopt
the following useful abbreviations.

\begin{mathpar}
   x?(\vec{y}).P := x.(\vec{y})P \and  x\clift{\vec{P}} := x.\clift{\vec{P}}
   \and x!(y) := \lift{x}{\dropn{y}}
   \and \Pi_{i=0}^{n-1}P_i := P_0 | \ldots | P_{n-1}
\end{mathpar}

\subsubsection{Structural congruence}

\paragraph{Free and bound names and alpha-equivalence.} At the
core of structural equivalence is alpha-equivalence which identifies
process that are the same up to a change of variable. Formally, we
recognize the distinction between free and bound names. The free names
of a process, $\freenames{P}$, may be calculated recursively as
follows:

\begin{mathpar}
\freenames{\pzero} := \emptyset
  \and \\
  \freenames{x?(y).P} := \{ x \} \cup (\freenames{P} \setminus \{ y \})
  \and 
  \freenames{x!\langle P \rangle} := \{ x \} \cup \{ P \} 
  \and \\
  \freenames{P|Q} := \freenames{P} \cup \freenames{Q}
  \and \\
  \freenames{@{x}} := \{ x \}
\end{mathpar}

$\pi$
$\quotep{\pi}$

$\freenames{-} : \pi \to \mathcal{P}(\quotep{\pi})$

\begin{eqnarray*}
  \freenames{\pzero} & := & \emptyset \\
  \freenames{x?(y).P} & := & \{ x \} \cup (\freenames{P} \setminus \{ y \}) \\
  \freenames{x!\langle P \rangle} & := & \{ x \} \cup \{ P \} \\
  \freenames{P|Q} & := & \freenames{P} \cup \freenames{Q} \\
  \freenames{\dropn{x}} & := & \{ x \}
\end{eqnarray*}

The bound names of a process, $\boundnames{P}$, are those names occurring in $P$
that are not free. For example, in $x?(y).0$, the name $x$ is free, while $y$ is bound.

\begin{mathpar}
  \inferrule* [lab=monoidal-laws] {} { P|Q \equiv Q|P \and P|0 \equiv P \and P|(Q|R) \equiv (P|Q)|R }
\end{mathpar}

\begin{mathpar}
  \inferrule* [lab=alpha-equivalence] {} { (x)P \equiv (y)P\{y/x\} \and y \not\in \freenames{P} }
\end{mathpar}

\begin{definition}
Then two processes, $P,Q$, are alpha-equivalent if $P = Q\{\vec{y}/\vec{x}\}$ for
some $\vec{x} \in \boundnames{Q},\vec{y} \in \boundnames{P}$, where $Q\{\vec{y}/\vec{x}\}$
denotes the capture-avoiding substitution of $\vec{y}$ for $\vec{x}$ in $Q$.
\end{definition}

\begin{definition}
  The {\em structural congruence} \cite{SangiorgiWalker} , $\equiv$,
  between processes is the least congruence containing
  alpha-equivalence, satisfying the abelian monoid laws
  (associativity, commutativity and $\pzero$ as identity) for parallel
  composition $|$ and for summation $+$.
\end{definition}

\subsection{Name equivalence}

We take name equivalence, written $\nameeq$, to be the smallest
equivalence relation generated by the following rules.

\begin{mathpar}
\inferrule*[lab=Quote-drop]
{ }
{ \quotep{@{x}} \nameeq x }

\inferrule*[lab=Struct-equiv]
{ P \scong Q }
{ \quotep{P} \nameeq \quotep{Q} }
\end{mathpar}

The astute reader will have noticed that the mutual recursion of names
and processes imposes a mutual recursion on alpha-equivalence and
structural equivalence via name-equivalence. Fortunately, all of this
works out pleasantly and we may calculate in the natural way, free of
concern. The reader interested in the details is referred to the
appendix \ref{appendix:rho_details}.

\subsection{Substitution}

We use $\Proc$ for the set of processes, $\QProc$ for the set of
names, and $\id{\{}\vec{y} / \vec{x} \id{\}}$ to denote partial maps,
$s : \QProc \rightarrow \QProc$. A map, $s$ lifts, uniquely, to a map
on process terms, $\widehat{s} : \Proc \rightarrow \Proc$ by the
following equations.

\begin{mathpar}
  (0) \psubstp{Q}{P} := 0 \\
  (R \juxtap S) \psubstp{Q}{P}
  :=    
  (R)\psubstp{Q}{P} \juxtap (S) \psubstp{Q}{P} \\
  (x?(y).R) \psubstp{Q}{P}    
  :=    
  (x)\substp{Q}{P} (z)\concat( (R \psubstn{z}{y}) \psubstp{Q}{P} ) \\
  (\lift{x}{R}) \psubstp{Q}{P}  
  :=
  \lift{(x)\substp{Q}{P}}{ R \psubstp{Q}{P} } \\
%   (\dropn{x})  \psubstp{Q}{P}       
%   := 
%   \left\{ 
%     \begin{array}{ccc} 
%       \dropn{\quotep{Q}} & & x \nameeq \quotep{P} \\
%       \dropn{x} & & otherwise \\
%     \end{array}
%   \right. 
  (\dropn{x})  \psubstp{Q}{P}       
  := 
  \left\{ 
    \begin{array}{ccc} 
      Q & & x \nameeq \quotep{P} \\
      \dropn{x} & & otherwise \\
    \end{array}
  \right.
\end{mathpar}
 

where

\begin{eqnarray}
  (x)\id{\{} \lpquote Q \rpquote / \lpquote P \rpquote \id{\}}            = 
  \left\{ 
    \begin{array}{ccc}
      \lpquote Q \rpquote & & x \nameeq \lpquote P \rpquote \\
      x & & otherwise \\
    \end{array}
  \right. \nonumber
\end{eqnarray}

and $z$ is chosen distinct from $\quotep{P}$, $\quotep{Q}$, the free
names in $Q$, and all the names in $R$. Our $\alpha$-equivalence will
be built in the standard way from this substitution.

\begin{remark}\label{rem:no_self_referential_names}
  One consequence of these definitions is that $\forall P. \quotep{P}
  \not\in \freenames{P}$.
\end{remark}

\subsection{ Dynamic quote: an example }

Anticipating something of what's to come, consider applying the
substitution, $\widehat{\id{\{}u / z \id{\}}}$, to the following pair
of processes, $\lift{w}{y!(z)}$ and $w[ \lpquote y!(z) \rpquote ]$.

\begin{eqnarray}
	\lift{w}{y!(z)}\widehat{\id{\{}u / z \id{\}}}
		& = &
		\lift{w}{y!(u)} \nonumber\\
	w[ \lpquote y!(z) \rpquote ] \widehat{ \id{\{}u / z \id{\}} }
		& = &
		w[ \lpquote y!(z) \rpquote ] \nonumber
\end{eqnarray}

Because the body of the process between quotes is impervious to
substitution, we get radically different answers. In fact, by
examining the first process in an input context,
e.g. $x?(z).\lift{w}{y!(z)}$, we see that the process under the lift
operator may be shaped by prefixed inputs binding a name inside it. In
this sense, the lift operator will be seen as a way to dynamically
construct processes before reifying them as names.

Finally equipped with these standard features we can present the
dynamics of the calculus.

\subsubsection{Operational semantics} 

Finally, we introduce the computational dynamics. What marks these
algebras as distinct from other more traditionally studied algebraic
structures, e.g. vector spaces or polynomial rings, is the manner in
which dynamics is captured. In traditional structures, dynamics is typically
expressed through morphisms between such structures, as in linear maps
between vector spaces or morphisms between rings. In algebras
associated with the semantics of computation, the dynamics is
expressed as part of the algebraic structure itself, through a
reduction reduction relation typically denoted by $\red$. Below, we
give a recursive presentation of this relation for the calculus used
in the encoding.

$\red \subseteq \pi \times \pi$
$\red : \pi \to \mathcal{P}(\pi)$

\begin{mathpar}
  \inferrule* [lab=Comm] { \textsf{match}( x_{src}, x_{trgt} ) } { x_{trgt}?(y)P \; | \; x_{src}!\langle {Q} \rangle \red P\{\quotep{Q}/y}\} }
  \and \\
  \inferrule* [lab=Par] {{P} \red {P}'} {{{P} | {Q}} \red {{P}' | {Q}}}
  \and
  \inferrule* [lab=Equiv]{{{P} \scong {P}'} \andalso {{P}' \red {Q}'} \andalso {{Q}' \scong {Q}}}{{P} \red {Q}}
\end{mathpar}

\begin{eqnarray*}
  match_{\equiv} (\quotep{P},\quotep{Q}) & := & P \equiv Q \\
  match_{\dagger}(\quotep{P},\quotep{Q}) & := & \forall R. P|Q \red^{*} R => R \red^{*} 0 \\
  match_{K}(\quotep{P},\quotep{Q}) & := & K \mbox{ for some context } K
\end{eqnarray*}

$u?(x)P | u!\langle Q \rangle \red P\{\quotep{Q}/x\}$

%We write $\wred$ for $\red^*$, and $P\red$ if $\exists Q $ such that $ P \red Q$.
We write $P\red$ if $\exists Q $ such that $ P \red Q$ and $P\not\red$, otherwise.

\section{Replication}

As mentioned before, it is known that replication (and hence
recursion) can be implemented in a higher-order process algebra
\cite{SangiorgiWalker}. As our first example of calculation with the
machinery thus far presented we give the construction explicitly in
the {\rhoc}.

\begin{eqnarray}
	D_{x} & := & \prefix{x}{y}{(\binpar{\outputp{x}{y}}{@{y}})} \nonumber\\
	\bangp_{x}{P} & := & \binpar{{x}!\langle{\binpar{D_{x}}{P}}\rangle}{D_{x}} \nonumber
\end{eqnarray}

\begin{eqnarray}
	\bangp_{x}{P} & & \nonumber\\
	=
	& {x}!\langle{(\prefix{x}{y}{(\outputp{x}{y} | @{y})) | P}}\rangle 
	      | \prefix{x}{y}{(\outputp{x}{y} | @{y})} & \nonumber\\
	\red
	& (\outputp{x}{y} | @{y})\substn{\quotep{(\prefix{x}{y}{(@{y} | \outputp{x}{y})) | P}}}{y} & \nonumber\\
	=
	& \outputp{x}{\quotep{(\prefix{x}{y}{(\outputp{x}{y} | @{y})) | P}}}
	  | {(\prefix{x}{y}{(\outputp{x}{y} | @{y})) | P}} & \nonumber\\
	\red
	& \ldots & \nonumber\\
	\red^*
	& P | P | \ldots & \nonumber
\end{eqnarray}

Of course, this encoding, as an implementation, runs away, unfolding
$\bangp{P}$ eagerly. A lazier and more implementable replication
operator, restricted to input-guarded processes, may be obtained as follows.

\begin{eqnarray}
\bangp{\prefix{u}{v}{P}} 
	:= 
	\binpar{\lift{x}{\prefix{u}{v}{(\binpar{D(x)}{P})}}}{D(x)} \nonumber
\end{eqnarray}

\begin{remark}
  Note that the lazier definition still does not deal with summation
  or mixed summation (i.e. sums over input and output). The reader is
  invited to construct definitions of replication that deal with these
  features. 

  Further, the definitions are parameterized in a name, $x$. Can you,
  gentle reader, make a definition that eliminates this parameter and
  guarantees no accidental interaction between the replication
  machinery and the process being replicated -- i.e. no accidental
  sharing of names used by the process to get its work done and the
  name(s) used by the replication to effect copying. This latter
  revision of the definition of replication is crucial to obtaining
  the expected identity $!!P \sim !P$.
\end{remark}

\begin{remark}\label{rem:paradoxical_combinator}
  The reader familiar with the lambda calculus will have noticed the
  similarity between $D$ and the paradoxical combinator.

  [Ed. note: the existence of this seems to suggest we have to be more
  restrictive on the set of processes and names we admit if we are to
  support no-cloning.]
\end{remark}

\subsubsection{Bisimulation}

The computational dynamics gives rise to another kind of equivalence,
the equivalence of computational behavior. As previously mentioned
this is typically captured \emph{via} some form of bisimulation.

% The notion we use in this paper is weak barbed bisimulation
% \cite{milner91polyadicpi}.

The notion we use in this paper is derived from weak barbed
bisimulation \cite{milner91polyadicpi}. 

\begin{definition}
An \emph{observation relation}, $\downarrow_{\mathcal N}$, over a set
of names, $\mathcal N$, is the smallest relation satisfying the rules
below.

\infrule[Out-barb]{y \in {\mathcal N}, \; x \nameeq y}
		  {\outputp{x}{v} \downarrow_{\mathcal N} x}
\infrule[Par-barb]{\mbox{$P\downarrow_{\mathcal N} x$ or $Q\downarrow_{\mathcal N} x$}}
		  {\binpar{P}{Q} \downarrow_{\mathcal N} x}

We write $P \Downarrow_{\mathcal N} x$ if there is $Q$ such that 
$P \wred Q$ and $Q \downarrow_{\mathcal N} x$.
\end{definition}

\begin{definition}
%\label{def.bbisim}
An  ${\mathcal N}$-\emph{barbed bisimulation} over a set of names, ${\mathcal N}$, is a symmetric binary relation 
${\mathcal S}_{\mathcal N}$ between agents such that $P\rel{S}_{\mathcal N}Q$ implies:
\begin{enumerate}
\item If $P \red P'$ then $Q \wred Q'$ and $P'\rel{S}_{\mathcal N} Q'$.
\item If $P\downarrow_{\mathcal N} x$, then $Q\Downarrow_{\mathcal N} x$.
\end{enumerate}
$P$ is ${\mathcal N}$-barbed bisimilar to $Q$, written
$P \wbbisim_{\mathcal N} Q$, if $P \rel{S}_{\mathcal N} Q$ for some ${\mathcal N}$-barbed bisimulation ${\mathcal S}_{\mathcal N}$.
\end{definition}

$\mathcal{R} \subseteq \pi \times \pi$

$P \mathcal{R} Q => \forall P'. P \red P' \Rightarrow \exists Q'. Q \red Q', P' \mathcal{R} Q'$

$P \vdash x \Rightarrow Q \vdash x$

\begin{mathpar}
  \inferrule*[lab=Out-barb]{x \nameeq y}{{y}!\langle{Q}\rangle \vdash x}
  \and
  \inferrule*[lab=Par-barb]{\mbox{$P\vdash x$ or $Q\vdash x$}}{\binpar{P}{Q} \vdash x}
\end{mathpar}

\subsubsection{Contexts}

One of the principle advantages of computational calculi like the
$\pi$-calculus is a well-defined notion of context,
contextual-equivalence and a correlation between
contextual-equivalence and notions of bisimulation. The notion of
context allows the decomposition of a process into (sub-)process and
its syntactic environment, its context. Thus, a context may be
thought of as a process with a ``hole'' (written $\Box$) in it. The
application of a context $M$ to a process $P$, written $M[P]$, is
tantamount to filling the hole in $M$ with $P$. In this paper we do
not need the full weight of this theory, but do make use of the notion
of context in the proof the main theorem. 

\begin{mathpar}
  \inferrule* [lab=summation] {} {{M_{M},M_{N}} \bc \Box \;|\; x.M_{A} \;|\; M_{M}+M_{N}}
  \and
  \inferrule* [lab=agent] {} {{M_{A}} \bc (\vec{x})M_{P} \;| \; \clift{P_0,\ldots,M_{P},\ldots,P_N}}
  \and \\
  \inferrule* [lab=process] {} {{M_{P}} \bc M_{N} \;| \;P|M_{P} }
\end{mathpar} 

\begin{mathpar}
  \inferrule* [lab=sychronization] {} {M_{N} \bc \Box \;|\; x?M_{F} \;|\; x!M_{C}}
  \and
  \inferrule* [lab=abstraction] {} {{M_{F}} \bc (x)M_{P} }
  \and
  \inferrule* [lab=concretion] {} {{M_{C}} \bc \langle M_{P} \rangle }
  \and \\
  \inferrule* [lab=process] {} {{M_{P}} \bc M_{N} \;| \;P|M_{P} }
\end{mathpar}

\begin{definition}[contextual application] Given a context $M$, and
  process $P$, we define the \emph{contextual application}, $M[P] :=
  M\{P/\Box\}$. That is, the contextual application of M to P is the
  substitution of $P$ for $\Box$ in $M$.
\end{definition}

$\meaningof{-} : L \to \mathcal{P}(\pi)$

\begin{mathpar}
  \inferrule* [lab=collection] {} {\meaningof{true} = \pi, \and \meaningof{~E} = \pi \setminus \meaningof{E}, \and \meaningof{E_{1} \& E_{2}} = \meaningof{E_{1}} \cap \meaningof{E_{2}}}
\end{mathpar}

\begin{mathpar}
  \inferrule* [lab=structure] {} {\meaningof{0} = \{ P \in \pi | P \equiv 0 \}, \and \\ \meaningof{E_1 | E_2} = \{ P \in \pi | P \equiv P_{1} | P_{2}, P_{1} \in \meaningof{E_{1}}, P_{2} \in \meaningof{E_2}\} }
\end{mathpar}

\begin{mathpar}
 \inferrule* [lab=behavior] {} {\meaningof{\langle a?b \rangle E} = \{ P \in \pi | P \equiv Q | u?(y)P', \\ \and \\\\ \and \\ \;\;\; u \in \meaningof{a}, \forall z.P'\{z/y\} \in \meaningof{E\{z/b\}}\}, \and \\ \meaningof{a!E} = \{ P \in \pi | P \equiv Q | x!\langle P' \rangle, x \in \meaningof{a} P' \in \meaningof{E}\} }
\end{mathpar}

\begin{mathpar}
 \inferrule* [lab=nominal] {} {\meaningof{\quotep{E}} = \{ \quotep{P} \in \quotep{\pi} | P \in \meaningof{E} \}, \and \meaningof{\quotep{P}} = \{ \quotep{Q} \in \quotep{\pi} | P \equiv Q \} \and \\ \meaningof{@\quotep{E}} = \{ P \in \pi | P \equiv @x, x \in \meaningof{E} \}}
\end{mathpar}

\begin{eqnarray*}
  \\
  \meaningof{-} : TS \to ST
\end{eqnarray*}

\begin{eqnarray*}
  \\
  L : TS \to ST
\end{eqnarray*}

\begin{eqnarray*}
  \\
  P \models E \iff P \in \meaningof{E}
\end{eqnarray*}

\begin{eqnarray*}
  P \approx_{L} Q \iff \forall E \in L. P \models E \iff Q \models E
\end{eqnarray*}

\begin{eqnarray*}
  P \approx_{K} Q
\end{eqnarray*}

\begin{eqnarray*}
  P \approx Q
\end{eqnarray*}

$\approx_{K} = \approx = \approx_{L}$

\subsubsection{Contextual duality}

Note that contexts extend the quotation operation to a family of
operations from processes to names. Given a context, $M$, we can
define a \emph{nominal context}, $\quotep{M}$ by $\quotep{M}[P] :=
\quotep{M[P]}$. To foreshadow what is to come we observe that these
operations enjoy a duality with processes very much like the duality
between vectors and maps from vectors to scalars.

Further, because the calculus is essentially higher-order, we have a
correspondence between contexts and processes. More specifically,
given a name $x$ and a context $M$ we can construct $M^{*}_{x}$ such
that 

\begin{mathpar}
  M^{*}_{x} | \lift{x}{P} \red M[P]
\end{mathpar}

namely,

\begin{mathpar}
  M^{*}_{x} := x?(u).M[\dropn{u}]
\end{mathpar}

The dependence of $M^{*}_{x}$ on a name makes it an abstraction, 

\begin{mathpar}
  M^{*} := (x)x?(u).M[\dropn{u}]
\end{mathpar}

\subsection{Additional notation}

It will sometimes be convenient to denote the process a name
quotes. We already have the notation $x = \quotep{P}$, but it will be
convenient to introduce an alternate notation, $\procn{x}$, when we
want to emphasize the connection to the use of the name. Note that, by
virtue of name equivalence, $\quotep{\procn{x}} \nameeq x$; so, the
notation is consistent with previous definitions.

Further, because names have structure it is possible to effect
substitutions on the basis of that structure. This means we need to
upgrade our notation for substitutions, which we accomplish by
adapting comprehension notation. Thus,

\begin{mathpar}
  P\{ y / x : x \in S \}
\end{mathpar}

is interpreted to mean the process derived from P by replacing (in a
capture-avoiding manner) each occurrence of $x$ in $S$ by $y$. For example,

\begin{mathpar}
  P\{ \quotep{\procn{x}|\procn{x}} / x : x \in \freenames{P} \}
\end{mathpar}

will replace each (occurrence) of a free name $x$ in $P$ by
$\quotep{\procn{x}|\procn{x}}$.

Also, we will avail ourselves of the notation $x^{L}$ and $x^{R}$ to
denote injections of a name into disjoint copies of the name
space. There are numerous ways to accomplish this. One example can be
found in \cite{MeredithR05}. This notation overloads to vectors of
names: $\vec{x}^{\pi} := (x_{i}^{\pi} \; : \; 0 \leq i < |\vec{x}| )$ where $\pi \in \{L,R\}$.

We also use $P^{\Box} := P|\Box$.

In \cite{MeredithR05} an interpretation of the new operator is
given. It turns out that there are several possible interpretations
all enjoying the requisite algebraic properties of the operator (see
\cite{milner91polyadicpi}). We will therefore make liberal use of
$(\nu\; \vec{x})P$.

% subsection the_syntax_and_semantics_of_the_notation_system (end)   

\input{qm2pi.qmops} 

\input{qm2pi.sterngerlach} 

\input{qm2pi.metric} 

% section concurrent_process_calculi (end)

%\input{qm2pi.proofsketch}

% section proof sketch (end)

%\input{qm2pi.slviaknots} 

% section spatial logic via knots (end)

\input{qm2pi.conclusion}

% section conclusion (end)

%\input{qm2pi.dtcodes} 

% section wiring algorithm (end)

\input{qm2pi.ack} 

% section acknowledgments (end)

\newpage


\bibliographystyle{plain}   
\bibliography{../../biblios/main.bib}

\input{qm2pi.rhodetails}

\end{document}

 

% section acknowledgments (end)

\newpage


\bibliographystyle{plain}   
\bibliography{../../biblios/main.bib}

\documentclass[12pt]{llncs}
%\documentclass{jktr}

\usepackage[pdftex]{hyperref}                   
\usepackage {listings}
\usepackage {mathpartir}
\usepackage{bcprules}
%\usepackage{listings}
                       
\usepackage{graphicx} 
%\usepackage[margins=2.5cm,nohead,nofoot]{geometry}
%\usepackage{geometry}
\usepackage{amsfonts}
\usepackage{amstext}
\usepackage{latexsym}
\usepackage{amssymb}
\usepackage{color}


%\include{myPreamble}
\include{qm2pi.local} 

%\ifpdf
%\usepackage[pdftex]{graphicx}
%\else
%\usepackage{graphicx}
%\fi

 % \ifpdf
%  \usepackage{pdfsync}
%  \if


%\title{Brief Article}
%\author{David F. Snyder}
%\author{L.G. Meredith}

%\address{Dept. of Math., Texas State University--San Marcos, San Marcos, TX 78666}
       
\pagestyle{empty}


\begin{document}

\lstset{language=[Objective]Caml,frame=shadowbox}

\input{qm2pi.front}

% section front matter (end)

\input{qm2pi.intro} 
 
% section introduction (end)

% \input{qm2pi.knotations} 

% section notation (end)

\input{qm2pi.process.calculi} 

% section concurrent_process_calculi_and_spatial_logics_ (end)
    
%\input{qm2pi.knots2pi} 

%\input{qm2pi.trefoil} 

%\input{qm2pi.mainthm} 

% subsection basic_interpretation (end)

%\input{qm2pi.rho.presentation} 
\subsection{The syntax and semantics of the notation system}\label{sub:the_syntax_and_semantics_of_the_notation_system} % (fold)

We now summarize a technical presentation of the calculus that
embodies our theory of dynamics. The typical presentation of such a
calculus follows the style of giving generators and relations on
them. The grammar, below, describing term constructors, freely
generates the set of processes, $\Proc$. This set is then quotiented
by a relation known as structural congruence and it is over this set
that the notion of dynamics is expressed. This presentation is
essentially that of \cite{MeredithR05} with the addition of
polyadicity and summation. For readability we have relegated some of
the technical subtleties to an appendix.

\subsubsection{Process grammar}\label{subsub:process_grammar}

\begin{mathpar}
  \inferrule* [lab=synchronization] {} {{M} \bc \pzero \;|\; x?F \;|\; x!C }
  \and
  \inferrule* [lab=abstraction] {} {{F} \bc (x)P}
  \and
  \inferrule* [lab=concretion] {} {{C} \bc \langle Q \rangle}
  \and
  \inferrule* [lab=process] {} {{P,Q} \bc M \;| \;P|Q \;|\; @{x}}
  \and
  \inferrule* [lab=name] {} {{x} \bc \quotep{P}}
\end{mathpar} 

Note that $\vec{x}$ (resp. $\vec{P}$) denotes a vector of names
(resp. processes) of length $|\vec{x}|$ (resp. $|\vec{P}|$). We adopt
the following useful abbreviations.

\begin{mathpar}
   x?(\vec{y}).P := x.(\vec{y})P \and  x\clift{\vec{P}} := x.\clift{\vec{P}}
   \and x!(y) := \lift{x}{\dropn{y}}
   \and \Pi_{i=0}^{n-1}P_i := P_0 | \ldots | P_{n-1}
\end{mathpar}

\subsubsection{Structural congruence}

\paragraph{Free and bound names and alpha-equivalence.} At the
core of structural equivalence is alpha-equivalence which identifies
process that are the same up to a change of variable. Formally, we
recognize the distinction between free and bound names. The free names
of a process, $\freenames{P}$, may be calculated recursively as
follows:

\begin{mathpar}
\freenames{\pzero} := \emptyset
  \and \\
  \freenames{x?(y).P} := \{ x \} \cup (\freenames{P} \setminus \{ y \})
  \and 
  \freenames{x!\langle P \rangle} := \{ x \} \cup \{ P \} 
  \and \\
  \freenames{P|Q} := \freenames{P} \cup \freenames{Q}
  \and \\
  \freenames{@{x}} := \{ x \}
\end{mathpar}

$\pi$
$\quotep{\pi}$

$\freenames{-} : \pi \to \mathcal{P}(\quotep{\pi})$

\begin{eqnarray*}
  \freenames{\pzero} & := & \emptyset \\
  \freenames{x?(y).P} & := & \{ x \} \cup (\freenames{P} \setminus \{ y \}) \\
  \freenames{x!\langle P \rangle} & := & \{ x \} \cup \{ P \} \\
  \freenames{P|Q} & := & \freenames{P} \cup \freenames{Q} \\
  \freenames{\dropn{x}} & := & \{ x \}
\end{eqnarray*}

The bound names of a process, $\boundnames{P}$, are those names occurring in $P$
that are not free. For example, in $x?(y).0$, the name $x$ is free, while $y$ is bound.

\begin{mathpar}
  \inferrule* [lab=monoidal-laws] {} { P|Q \equiv Q|P \and P|0 \equiv P \and P|(Q|R) \equiv (P|Q)|R }
\end{mathpar}

\begin{mathpar}
  \inferrule* [lab=alpha-equivalence] {} { (x)P \equiv (y)P\{y/x\} \and y \not\in \freenames{P} }
\end{mathpar}

\begin{definition}
Then two processes, $P,Q$, are alpha-equivalent if $P = Q\{\vec{y}/\vec{x}\}$ for
some $\vec{x} \in \boundnames{Q},\vec{y} \in \boundnames{P}$, where $Q\{\vec{y}/\vec{x}\}$
denotes the capture-avoiding substitution of $\vec{y}$ for $\vec{x}$ in $Q$.
\end{definition}

\begin{definition}
  The {\em structural congruence} \cite{SangiorgiWalker} , $\equiv$,
  between processes is the least congruence containing
  alpha-equivalence, satisfying the abelian monoid laws
  (associativity, commutativity and $\pzero$ as identity) for parallel
  composition $|$ and for summation $+$.
\end{definition}

\subsection{Name equivalence}

We take name equivalence, written $\nameeq$, to be the smallest
equivalence relation generated by the following rules.

\begin{mathpar}
\inferrule*[lab=Quote-drop]
{ }
{ \quotep{@{x}} \nameeq x }

\inferrule*[lab=Struct-equiv]
{ P \scong Q }
{ \quotep{P} \nameeq \quotep{Q} }
\end{mathpar}

The astute reader will have noticed that the mutual recursion of names
and processes imposes a mutual recursion on alpha-equivalence and
structural equivalence via name-equivalence. Fortunately, all of this
works out pleasantly and we may calculate in the natural way, free of
concern. The reader interested in the details is referred to the
appendix \ref{appendix:rho_details}.

\subsection{Substitution}

We use $\Proc$ for the set of processes, $\QProc$ for the set of
names, and $\id{\{}\vec{y} / \vec{x} \id{\}}$ to denote partial maps,
$s : \QProc \rightarrow \QProc$. A map, $s$ lifts, uniquely, to a map
on process terms, $\widehat{s} : \Proc \rightarrow \Proc$ by the
following equations.

\begin{mathpar}
  (0) \psubstp{Q}{P} := 0 \\
  (R \juxtap S) \psubstp{Q}{P}
  :=    
  (R)\psubstp{Q}{P} \juxtap (S) \psubstp{Q}{P} \\
  (x?(y).R) \psubstp{Q}{P}    
  :=    
  (x)\substp{Q}{P} (z)\concat( (R \psubstn{z}{y}) \psubstp{Q}{P} ) \\
  (\lift{x}{R}) \psubstp{Q}{P}  
  :=
  \lift{(x)\substp{Q}{P}}{ R \psubstp{Q}{P} } \\
%   (\dropn{x})  \psubstp{Q}{P}       
%   := 
%   \left\{ 
%     \begin{array}{ccc} 
%       \dropn{\quotep{Q}} & & x \nameeq \quotep{P} \\
%       \dropn{x} & & otherwise \\
%     \end{array}
%   \right. 
  (\dropn{x})  \psubstp{Q}{P}       
  := 
  \left\{ 
    \begin{array}{ccc} 
      Q & & x \nameeq \quotep{P} \\
      \dropn{x} & & otherwise \\
    \end{array}
  \right.
\end{mathpar}
 

where

\begin{eqnarray}
  (x)\id{\{} \lpquote Q \rpquote / \lpquote P \rpquote \id{\}}            = 
  \left\{ 
    \begin{array}{ccc}
      \lpquote Q \rpquote & & x \nameeq \lpquote P \rpquote \\
      x & & otherwise \\
    \end{array}
  \right. \nonumber
\end{eqnarray}

and $z$ is chosen distinct from $\quotep{P}$, $\quotep{Q}$, the free
names in $Q$, and all the names in $R$. Our $\alpha$-equivalence will
be built in the standard way from this substitution.

\begin{remark}\label{rem:no_self_referential_names}
  One consequence of these definitions is that $\forall P. \quotep{P}
  \not\in \freenames{P}$.
\end{remark}

\subsection{ Dynamic quote: an example }

Anticipating something of what's to come, consider applying the
substitution, $\widehat{\id{\{}u / z \id{\}}}$, to the following pair
of processes, $\lift{w}{y!(z)}$ and $w[ \lpquote y!(z) \rpquote ]$.

\begin{eqnarray}
	\lift{w}{y!(z)}\widehat{\id{\{}u / z \id{\}}}
		& = &
		\lift{w}{y!(u)} \nonumber\\
	w[ \lpquote y!(z) \rpquote ] \widehat{ \id{\{}u / z \id{\}} }
		& = &
		w[ \lpquote y!(z) \rpquote ] \nonumber
\end{eqnarray}

Because the body of the process between quotes is impervious to
substitution, we get radically different answers. In fact, by
examining the first process in an input context,
e.g. $x?(z).\lift{w}{y!(z)}$, we see that the process under the lift
operator may be shaped by prefixed inputs binding a name inside it. In
this sense, the lift operator will be seen as a way to dynamically
construct processes before reifying them as names.

Finally equipped with these standard features we can present the
dynamics of the calculus.

\subsubsection{Operational semantics} 

Finally, we introduce the computational dynamics. What marks these
algebras as distinct from other more traditionally studied algebraic
structures, e.g. vector spaces or polynomial rings, is the manner in
which dynamics is captured. In traditional structures, dynamics is typically
expressed through morphisms between such structures, as in linear maps
between vector spaces or morphisms between rings. In algebras
associated with the semantics of computation, the dynamics is
expressed as part of the algebraic structure itself, through a
reduction reduction relation typically denoted by $\red$. Below, we
give a recursive presentation of this relation for the calculus used
in the encoding.

$\red \subseteq \pi \times \pi$
$\red : \pi \to \mathcal{P}(\pi)$

\begin{mathpar}
  \inferrule* [lab=Comm] { \textsf{match}( x_{src}, x_{trgt} ) } { x_{trgt}?(y)P \; | \; x_{src}!\langle {Q} \rangle \red P\{\quotep{Q}/y}\} }
  \and \\
  \inferrule* [lab=Par] {{P} \red {P}'} {{{P} | {Q}} \red {{P}' | {Q}}}
  \and
  \inferrule* [lab=Equiv]{{{P} \scong {P}'} \andalso {{P}' \red {Q}'} \andalso {{Q}' \scong {Q}}}{{P} \red {Q}}
\end{mathpar}

\begin{eqnarray*}
  match_{\equiv} (\quotep{P},\quotep{Q}) & := & P \equiv Q \\
  match_{\dagger}(\quotep{P},\quotep{Q}) & := & \forall R. P|Q \red^{*} R => R \red^{*} 0 \\
  match_{K}(\quotep{P},\quotep{Q}) & := & K \mbox{ for some context } K
\end{eqnarray*}

$u?(x)P | u!\langle Q \rangle \red P\{\quotep{Q}/x\}$

%We write $\wred$ for $\red^*$, and $P\red$ if $\exists Q $ such that $ P \red Q$.
We write $P\red$ if $\exists Q $ such that $ P \red Q$ and $P\not\red$, otherwise.

\section{Replication}

As mentioned before, it is known that replication (and hence
recursion) can be implemented in a higher-order process algebra
\cite{SangiorgiWalker}. As our first example of calculation with the
machinery thus far presented we give the construction explicitly in
the {\rhoc}.

\begin{eqnarray}
	D_{x} & := & \prefix{x}{y}{(\binpar{\outputp{x}{y}}{@{y}})} \nonumber\\
	\bangp_{x}{P} & := & \binpar{{x}!\langle{\binpar{D_{x}}{P}}\rangle}{D_{x}} \nonumber
\end{eqnarray}

\begin{eqnarray}
	\bangp_{x}{P} & & \nonumber\\
	=
	& {x}!\langle{(\prefix{x}{y}{(\outputp{x}{y} | @{y})) | P}}\rangle 
	      | \prefix{x}{y}{(\outputp{x}{y} | @{y})} & \nonumber\\
	\red
	& (\outputp{x}{y} | @{y})\substn{\quotep{(\prefix{x}{y}{(@{y} | \outputp{x}{y})) | P}}}{y} & \nonumber\\
	=
	& \outputp{x}{\quotep{(\prefix{x}{y}{(\outputp{x}{y} | @{y})) | P}}}
	  | {(\prefix{x}{y}{(\outputp{x}{y} | @{y})) | P}} & \nonumber\\
	\red
	& \ldots & \nonumber\\
	\red^*
	& P | P | \ldots & \nonumber
\end{eqnarray}

Of course, this encoding, as an implementation, runs away, unfolding
$\bangp{P}$ eagerly. A lazier and more implementable replication
operator, restricted to input-guarded processes, may be obtained as follows.

\begin{eqnarray}
\bangp{\prefix{u}{v}{P}} 
	:= 
	\binpar{\lift{x}{\prefix{u}{v}{(\binpar{D(x)}{P})}}}{D(x)} \nonumber
\end{eqnarray}

\begin{remark}
  Note that the lazier definition still does not deal with summation
  or mixed summation (i.e. sums over input and output). The reader is
  invited to construct definitions of replication that deal with these
  features. 

  Further, the definitions are parameterized in a name, $x$. Can you,
  gentle reader, make a definition that eliminates this parameter and
  guarantees no accidental interaction between the replication
  machinery and the process being replicated -- i.e. no accidental
  sharing of names used by the process to get its work done and the
  name(s) used by the replication to effect copying. This latter
  revision of the definition of replication is crucial to obtaining
  the expected identity $!!P \sim !P$.
\end{remark}

\begin{remark}\label{rem:paradoxical_combinator}
  The reader familiar with the lambda calculus will have noticed the
  similarity between $D$ and the paradoxical combinator.

  [Ed. note: the existence of this seems to suggest we have to be more
  restrictive on the set of processes and names we admit if we are to
  support no-cloning.]
\end{remark}

\subsubsection{Bisimulation}

The computational dynamics gives rise to another kind of equivalence,
the equivalence of computational behavior. As previously mentioned
this is typically captured \emph{via} some form of bisimulation.

% The notion we use in this paper is weak barbed bisimulation
% \cite{milner91polyadicpi}.

The notion we use in this paper is derived from weak barbed
bisimulation \cite{milner91polyadicpi}. 

\begin{definition}
An \emph{observation relation}, $\downarrow_{\mathcal N}$, over a set
of names, $\mathcal N$, is the smallest relation satisfying the rules
below.

\infrule[Out-barb]{y \in {\mathcal N}, \; x \nameeq y}
		  {\outputp{x}{v} \downarrow_{\mathcal N} x}
\infrule[Par-barb]{\mbox{$P\downarrow_{\mathcal N} x$ or $Q\downarrow_{\mathcal N} x$}}
		  {\binpar{P}{Q} \downarrow_{\mathcal N} x}

We write $P \Downarrow_{\mathcal N} x$ if there is $Q$ such that 
$P \wred Q$ and $Q \downarrow_{\mathcal N} x$.
\end{definition}

\begin{definition}
%\label{def.bbisim}
An  ${\mathcal N}$-\emph{barbed bisimulation} over a set of names, ${\mathcal N}$, is a symmetric binary relation 
${\mathcal S}_{\mathcal N}$ between agents such that $P\rel{S}_{\mathcal N}Q$ implies:
\begin{enumerate}
\item If $P \red P'$ then $Q \wred Q'$ and $P'\rel{S}_{\mathcal N} Q'$.
\item If $P\downarrow_{\mathcal N} x$, then $Q\Downarrow_{\mathcal N} x$.
\end{enumerate}
$P$ is ${\mathcal N}$-barbed bisimilar to $Q$, written
$P \wbbisim_{\mathcal N} Q$, if $P \rel{S}_{\mathcal N} Q$ for some ${\mathcal N}$-barbed bisimulation ${\mathcal S}_{\mathcal N}$.
\end{definition}

$\mathcal{R} \subseteq \pi \times \pi$

$P \mathcal{R} Q => \forall P'. P \red P' \Rightarrow \exists Q'. Q \red Q', P' \mathcal{R} Q'$

$P \vdash x \Rightarrow Q \vdash x$

\begin{mathpar}
  \inferrule*[lab=Out-barb]{x \nameeq y}{{y}!\langle{Q}\rangle \vdash x}
  \and
  \inferrule*[lab=Par-barb]{\mbox{$P\vdash x$ or $Q\vdash x$}}{\binpar{P}{Q} \vdash x}
\end{mathpar}

\subsubsection{Contexts}

One of the principle advantages of computational calculi like the
$\pi$-calculus is a well-defined notion of context,
contextual-equivalence and a correlation between
contextual-equivalence and notions of bisimulation. The notion of
context allows the decomposition of a process into (sub-)process and
its syntactic environment, its context. Thus, a context may be
thought of as a process with a ``hole'' (written $\Box$) in it. The
application of a context $M$ to a process $P$, written $M[P]$, is
tantamount to filling the hole in $M$ with $P$. In this paper we do
not need the full weight of this theory, but do make use of the notion
of context in the proof the main theorem. 

\begin{mathpar}
  \inferrule* [lab=summation] {} {{M_{M},M_{N}} \bc \Box \;|\; x.M_{A} \;|\; M_{M}+M_{N}}
  \and
  \inferrule* [lab=agent] {} {{M_{A}} \bc (\vec{x})M_{P} \;| \; \clift{P_0,\ldots,M_{P},\ldots,P_N}}
  \and \\
  \inferrule* [lab=process] {} {{M_{P}} \bc M_{N} \;| \;P|M_{P} }
\end{mathpar} 

\begin{mathpar}
  \inferrule* [lab=sychronization] {} {M_{N} \bc \Box \;|\; x?M_{F} \;|\; x!M_{C}}
  \and
  \inferrule* [lab=abstraction] {} {{M_{F}} \bc (x)M_{P} }
  \and
  \inferrule* [lab=concretion] {} {{M_{C}} \bc \langle M_{P} \rangle }
  \and \\
  \inferrule* [lab=process] {} {{M_{P}} \bc M_{N} \;| \;P|M_{P} }
\end{mathpar}

\begin{definition}[contextual application] Given a context $M$, and
  process $P$, we define the \emph{contextual application}, $M[P] :=
  M\{P/\Box\}$. That is, the contextual application of M to P is the
  substitution of $P$ for $\Box$ in $M$.
\end{definition}

$\meaningof{-} : L \to \mathcal{P}(\pi)$

\begin{mathpar}
  \inferrule* [lab=collection] {} {\meaningof{true} = \pi, \and \meaningof{~E} = \pi \setminus \meaningof{E}, \and \meaningof{E_{1} \& E_{2}} = \meaningof{E_{1}} \cap \meaningof{E_{2}}}
\end{mathpar}

\begin{mathpar}
  \inferrule* [lab=structure] {} {\meaningof{0} = \{ P \in \pi | P \equiv 0 \}, \and \\ \meaningof{E_1 | E_2} = \{ P \in \pi | P \equiv P_{1} | P_{2}, P_{1} \in \meaningof{E_{1}}, P_{2} \in \meaningof{E_2}\} }
\end{mathpar}

\begin{mathpar}
 \inferrule* [lab=behavior] {} {\meaningof{\langle a?b \rangle E} = \{ P \in \pi | P \equiv Q | u?(y)P', \\ \and \\\\ \and \\ \;\;\; u \in \meaningof{a}, \forall z.P'\{z/y\} \in \meaningof{E\{z/b\}}\}, \and \\ \meaningof{a!E} = \{ P \in \pi | P \equiv Q | x!\langle P' \rangle, x \in \meaningof{a} P' \in \meaningof{E}\} }
\end{mathpar}

\begin{mathpar}
 \inferrule* [lab=nominal] {} {\meaningof{\quotep{E}} = \{ \quotep{P} \in \quotep{\pi} | P \in \meaningof{E} \}, \and \meaningof{\quotep{P}} = \{ \quotep{Q} \in \quotep{\pi} | P \equiv Q \} \and \\ \meaningof{@\quotep{E}} = \{ P \in \pi | P \equiv @x, x \in \meaningof{E} \}}
\end{mathpar}

\begin{eqnarray*}
  \\
  \meaningof{-} : TS \to ST
\end{eqnarray*}

\begin{eqnarray*}
  \\
  L : TS \to ST
\end{eqnarray*}

\begin{eqnarray*}
  \\
  P \models E \iff P \in \meaningof{E}
\end{eqnarray*}

\begin{eqnarray*}
  P \approx_{L} Q \iff \forall E \in L. P \models E \iff Q \models E
\end{eqnarray*}

\begin{eqnarray*}
  P \approx_{K} Q
\end{eqnarray*}

\begin{eqnarray*}
  P \approx Q
\end{eqnarray*}

$\approx_{K} = \approx = \approx_{L}$

\subsubsection{Contextual duality}

Note that contexts extend the quotation operation to a family of
operations from processes to names. Given a context, $M$, we can
define a \emph{nominal context}, $\quotep{M}$ by $\quotep{M}[P] :=
\quotep{M[P]}$. To foreshadow what is to come we observe that these
operations enjoy a duality with processes very much like the duality
between vectors and maps from vectors to scalars.

Further, because the calculus is essentially higher-order, we have a
correspondence between contexts and processes. More specifically,
given a name $x$ and a context $M$ we can construct $M^{*}_{x}$ such
that 

\begin{mathpar}
  M^{*}_{x} | \lift{x}{P} \red M[P]
\end{mathpar}

namely,

\begin{mathpar}
  M^{*}_{x} := x?(u).M[\dropn{u}]
\end{mathpar}

The dependence of $M^{*}_{x}$ on a name makes it an abstraction, 

\begin{mathpar}
  M^{*} := (x)x?(u).M[\dropn{u}]
\end{mathpar}

\subsection{Additional notation}

It will sometimes be convenient to denote the process a name
quotes. We already have the notation $x = \quotep{P}$, but it will be
convenient to introduce an alternate notation, $\procn{x}$, when we
want to emphasize the connection to the use of the name. Note that, by
virtue of name equivalence, $\quotep{\procn{x}} \nameeq x$; so, the
notation is consistent with previous definitions.

Further, because names have structure it is possible to effect
substitutions on the basis of that structure. This means we need to
upgrade our notation for substitutions, which we accomplish by
adapting comprehension notation. Thus,

\begin{mathpar}
  P\{ y / x : x \in S \}
\end{mathpar}

is interpreted to mean the process derived from P by replacing (in a
capture-avoiding manner) each occurrence of $x$ in $S$ by $y$. For example,

\begin{mathpar}
  P\{ \quotep{\procn{x}|\procn{x}} / x : x \in \freenames{P} \}
\end{mathpar}

will replace each (occurrence) of a free name $x$ in $P$ by
$\quotep{\procn{x}|\procn{x}}$.

Also, we will avail ourselves of the notation $x^{L}$ and $x^{R}$ to
denote injections of a name into disjoint copies of the name
space. There are numerous ways to accomplish this. One example can be
found in \cite{MeredithR05}. This notation overloads to vectors of
names: $\vec{x}^{\pi} := (x_{i}^{\pi} \; : \; 0 \leq i < |\vec{x}| )$ where $\pi \in \{L,R\}$.

We also use $P^{\Box} := P|\Box$.

In \cite{MeredithR05} an interpretation of the new operator is
given. It turns out that there are several possible interpretations
all enjoying the requisite algebraic properties of the operator (see
\cite{milner91polyadicpi}). We will therefore make liberal use of
$(\nu\; \vec{x})P$.

% subsection the_syntax_and_semantics_of_the_notation_system (end)   

\input{qm2pi.qmops} 

\input{qm2pi.sterngerlach} 

\input{qm2pi.metric} 

% section concurrent_process_calculi (end)

%\input{qm2pi.proofsketch}

% section proof sketch (end)

%\input{qm2pi.slviaknots} 

% section spatial logic via knots (end)

\input{qm2pi.conclusion}

% section conclusion (end)

%\input{qm2pi.dtcodes} 

% section wiring algorithm (end)

\input{qm2pi.ack} 

% section acknowledgments (end)

\newpage


\bibliographystyle{plain}   
\bibliography{../../biblios/main.bib}

\input{qm2pi.rhodetails}

\end{document}



\end{document}

 

% section wiring algorithm (end)

\documentclass[12pt]{llncs}
%\documentclass{jktr}

\usepackage[pdftex]{hyperref}                   
\usepackage {listings}
\usepackage {mathpartir}
\usepackage{bcprules}
%\usepackage{listings}
                       
\usepackage{graphicx} 
%\usepackage[margins=2.5cm,nohead,nofoot]{geometry}
%\usepackage{geometry}
\usepackage{amsfonts}
\usepackage{amstext}
\usepackage{latexsym}
\usepackage{amssymb}
\usepackage{color}


%\include{myPreamble}
\documentclass[12pt]{llncs}
%\documentclass{jktr}

\usepackage[pdftex]{hyperref}                   
\usepackage {listings}
\usepackage {mathpartir}
\usepackage{bcprules}
%\usepackage{listings}
                       
\usepackage{graphicx} 
%\usepackage[margins=2.5cm,nohead,nofoot]{geometry}
%\usepackage{geometry}
\usepackage{amsfonts}
\usepackage{amstext}
\usepackage{latexsym}
\usepackage{amssymb}
\usepackage{color}


%\include{myPreamble}
\include{qm2pi.local} 

%\ifpdf
%\usepackage[pdftex]{graphicx}
%\else
%\usepackage{graphicx}
%\fi

 % \ifpdf
%  \usepackage{pdfsync}
%  \if


%\title{Brief Article}
%\author{David F. Snyder}
%\author{L.G. Meredith}

%\address{Dept. of Math., Texas State University--San Marcos, San Marcos, TX 78666}
       
\pagestyle{empty}


\begin{document}

\lstset{language=[Objective]Caml,frame=shadowbox}

\input{qm2pi.front}

% section front matter (end)

\input{qm2pi.intro} 
 
% section introduction (end)

% \input{qm2pi.knotations} 

% section notation (end)

\input{qm2pi.process.calculi} 

% section concurrent_process_calculi_and_spatial_logics_ (end)
    
%\input{qm2pi.knots2pi} 

%\input{qm2pi.trefoil} 

%\input{qm2pi.mainthm} 

% subsection basic_interpretation (end)

%\input{qm2pi.rho.presentation} 
\subsection{The syntax and semantics of the notation system}\label{sub:the_syntax_and_semantics_of_the_notation_system} % (fold)

We now summarize a technical presentation of the calculus that
embodies our theory of dynamics. The typical presentation of such a
calculus follows the style of giving generators and relations on
them. The grammar, below, describing term constructors, freely
generates the set of processes, $\Proc$. This set is then quotiented
by a relation known as structural congruence and it is over this set
that the notion of dynamics is expressed. This presentation is
essentially that of \cite{MeredithR05} with the addition of
polyadicity and summation. For readability we have relegated some of
the technical subtleties to an appendix.

\subsubsection{Process grammar}\label{subsub:process_grammar}

\begin{mathpar}
  \inferrule* [lab=synchronization] {} {{M} \bc \pzero \;|\; x?F \;|\; x!C }
  \and
  \inferrule* [lab=abstraction] {} {{F} \bc (x)P}
  \and
  \inferrule* [lab=concretion] {} {{C} \bc \langle Q \rangle}
  \and
  \inferrule* [lab=process] {} {{P,Q} \bc M \;| \;P|Q \;|\; @{x}}
  \and
  \inferrule* [lab=name] {} {{x} \bc \quotep{P}}
\end{mathpar} 

Note that $\vec{x}$ (resp. $\vec{P}$) denotes a vector of names
(resp. processes) of length $|\vec{x}|$ (resp. $|\vec{P}|$). We adopt
the following useful abbreviations.

\begin{mathpar}
   x?(\vec{y}).P := x.(\vec{y})P \and  x\clift{\vec{P}} := x.\clift{\vec{P}}
   \and x!(y) := \lift{x}{\dropn{y}}
   \and \Pi_{i=0}^{n-1}P_i := P_0 | \ldots | P_{n-1}
\end{mathpar}

\subsubsection{Structural congruence}

\paragraph{Free and bound names and alpha-equivalence.} At the
core of structural equivalence is alpha-equivalence which identifies
process that are the same up to a change of variable. Formally, we
recognize the distinction between free and bound names. The free names
of a process, $\freenames{P}$, may be calculated recursively as
follows:

\begin{mathpar}
\freenames{\pzero} := \emptyset
  \and \\
  \freenames{x?(y).P} := \{ x \} \cup (\freenames{P} \setminus \{ y \})
  \and 
  \freenames{x!\langle P \rangle} := \{ x \} \cup \{ P \} 
  \and \\
  \freenames{P|Q} := \freenames{P} \cup \freenames{Q}
  \and \\
  \freenames{@{x}} := \{ x \}
\end{mathpar}

$\pi$
$\quotep{\pi}$

$\freenames{-} : \pi \to \mathcal{P}(\quotep{\pi})$

\begin{eqnarray*}
  \freenames{\pzero} & := & \emptyset \\
  \freenames{x?(y).P} & := & \{ x \} \cup (\freenames{P} \setminus \{ y \}) \\
  \freenames{x!\langle P \rangle} & := & \{ x \} \cup \{ P \} \\
  \freenames{P|Q} & := & \freenames{P} \cup \freenames{Q} \\
  \freenames{\dropn{x}} & := & \{ x \}
\end{eqnarray*}

The bound names of a process, $\boundnames{P}$, are those names occurring in $P$
that are not free. For example, in $x?(y).0$, the name $x$ is free, while $y$ is bound.

\begin{mathpar}
  \inferrule* [lab=monoidal-laws] {} { P|Q \equiv Q|P \and P|0 \equiv P \and P|(Q|R) \equiv (P|Q)|R }
\end{mathpar}

\begin{mathpar}
  \inferrule* [lab=alpha-equivalence] {} { (x)P \equiv (y)P\{y/x\} \and y \not\in \freenames{P} }
\end{mathpar}

\begin{definition}
Then two processes, $P,Q$, are alpha-equivalent if $P = Q\{\vec{y}/\vec{x}\}$ for
some $\vec{x} \in \boundnames{Q},\vec{y} \in \boundnames{P}$, where $Q\{\vec{y}/\vec{x}\}$
denotes the capture-avoiding substitution of $\vec{y}$ for $\vec{x}$ in $Q$.
\end{definition}

\begin{definition}
  The {\em structural congruence} \cite{SangiorgiWalker} , $\equiv$,
  between processes is the least congruence containing
  alpha-equivalence, satisfying the abelian monoid laws
  (associativity, commutativity and $\pzero$ as identity) for parallel
  composition $|$ and for summation $+$.
\end{definition}

\subsection{Name equivalence}

We take name equivalence, written $\nameeq$, to be the smallest
equivalence relation generated by the following rules.

\begin{mathpar}
\inferrule*[lab=Quote-drop]
{ }
{ \quotep{@{x}} \nameeq x }

\inferrule*[lab=Struct-equiv]
{ P \scong Q }
{ \quotep{P} \nameeq \quotep{Q} }
\end{mathpar}

The astute reader will have noticed that the mutual recursion of names
and processes imposes a mutual recursion on alpha-equivalence and
structural equivalence via name-equivalence. Fortunately, all of this
works out pleasantly and we may calculate in the natural way, free of
concern. The reader interested in the details is referred to the
appendix \ref{appendix:rho_details}.

\subsection{Substitution}

We use $\Proc$ for the set of processes, $\QProc$ for the set of
names, and $\id{\{}\vec{y} / \vec{x} \id{\}}$ to denote partial maps,
$s : \QProc \rightarrow \QProc$. A map, $s$ lifts, uniquely, to a map
on process terms, $\widehat{s} : \Proc \rightarrow \Proc$ by the
following equations.

\begin{mathpar}
  (0) \psubstp{Q}{P} := 0 \\
  (R \juxtap S) \psubstp{Q}{P}
  :=    
  (R)\psubstp{Q}{P} \juxtap (S) \psubstp{Q}{P} \\
  (x?(y).R) \psubstp{Q}{P}    
  :=    
  (x)\substp{Q}{P} (z)\concat( (R \psubstn{z}{y}) \psubstp{Q}{P} ) \\
  (\lift{x}{R}) \psubstp{Q}{P}  
  :=
  \lift{(x)\substp{Q}{P}}{ R \psubstp{Q}{P} } \\
%   (\dropn{x})  \psubstp{Q}{P}       
%   := 
%   \left\{ 
%     \begin{array}{ccc} 
%       \dropn{\quotep{Q}} & & x \nameeq \quotep{P} \\
%       \dropn{x} & & otherwise \\
%     \end{array}
%   \right. 
  (\dropn{x})  \psubstp{Q}{P}       
  := 
  \left\{ 
    \begin{array}{ccc} 
      Q & & x \nameeq \quotep{P} \\
      \dropn{x} & & otherwise \\
    \end{array}
  \right.
\end{mathpar}
 

where

\begin{eqnarray}
  (x)\id{\{} \lpquote Q \rpquote / \lpquote P \rpquote \id{\}}            = 
  \left\{ 
    \begin{array}{ccc}
      \lpquote Q \rpquote & & x \nameeq \lpquote P \rpquote \\
      x & & otherwise \\
    \end{array}
  \right. \nonumber
\end{eqnarray}

and $z$ is chosen distinct from $\quotep{P}$, $\quotep{Q}$, the free
names in $Q$, and all the names in $R$. Our $\alpha$-equivalence will
be built in the standard way from this substitution.

\begin{remark}\label{rem:no_self_referential_names}
  One consequence of these definitions is that $\forall P. \quotep{P}
  \not\in \freenames{P}$.
\end{remark}

\subsection{ Dynamic quote: an example }

Anticipating something of what's to come, consider applying the
substitution, $\widehat{\id{\{}u / z \id{\}}}$, to the following pair
of processes, $\lift{w}{y!(z)}$ and $w[ \lpquote y!(z) \rpquote ]$.

\begin{eqnarray}
	\lift{w}{y!(z)}\widehat{\id{\{}u / z \id{\}}}
		& = &
		\lift{w}{y!(u)} \nonumber\\
	w[ \lpquote y!(z) \rpquote ] \widehat{ \id{\{}u / z \id{\}} }
		& = &
		w[ \lpquote y!(z) \rpquote ] \nonumber
\end{eqnarray}

Because the body of the process between quotes is impervious to
substitution, we get radically different answers. In fact, by
examining the first process in an input context,
e.g. $x?(z).\lift{w}{y!(z)}$, we see that the process under the lift
operator may be shaped by prefixed inputs binding a name inside it. In
this sense, the lift operator will be seen as a way to dynamically
construct processes before reifying them as names.

Finally equipped with these standard features we can present the
dynamics of the calculus.

\subsubsection{Operational semantics} 

Finally, we introduce the computational dynamics. What marks these
algebras as distinct from other more traditionally studied algebraic
structures, e.g. vector spaces or polynomial rings, is the manner in
which dynamics is captured. In traditional structures, dynamics is typically
expressed through morphisms between such structures, as in linear maps
between vector spaces or morphisms between rings. In algebras
associated with the semantics of computation, the dynamics is
expressed as part of the algebraic structure itself, through a
reduction reduction relation typically denoted by $\red$. Below, we
give a recursive presentation of this relation for the calculus used
in the encoding.

$\red \subseteq \pi \times \pi$
$\red : \pi \to \mathcal{P}(\pi)$

\begin{mathpar}
  \inferrule* [lab=Comm] { \textsf{match}( x_{src}, x_{trgt} ) } { x_{trgt}?(y)P \; | \; x_{src}!\langle {Q} \rangle \red P\{\quotep{Q}/y}\} }
  \and \\
  \inferrule* [lab=Par] {{P} \red {P}'} {{{P} | {Q}} \red {{P}' | {Q}}}
  \and
  \inferrule* [lab=Equiv]{{{P} \scong {P}'} \andalso {{P}' \red {Q}'} \andalso {{Q}' \scong {Q}}}{{P} \red {Q}}
\end{mathpar}

\begin{eqnarray*}
  match_{\equiv} (\quotep{P},\quotep{Q}) & := & P \equiv Q \\
  match_{\dagger}(\quotep{P},\quotep{Q}) & := & \forall R. P|Q \red^{*} R => R \red^{*} 0 \\
  match_{K}(\quotep{P},\quotep{Q}) & := & K \mbox{ for some context } K
\end{eqnarray*}

$u?(x)P | u!\langle Q \rangle \red P\{\quotep{Q}/x\}$

%We write $\wred$ for $\red^*$, and $P\red$ if $\exists Q $ such that $ P \red Q$.
We write $P\red$ if $\exists Q $ such that $ P \red Q$ and $P\not\red$, otherwise.

\section{Replication}

As mentioned before, it is known that replication (and hence
recursion) can be implemented in a higher-order process algebra
\cite{SangiorgiWalker}. As our first example of calculation with the
machinery thus far presented we give the construction explicitly in
the {\rhoc}.

\begin{eqnarray}
	D_{x} & := & \prefix{x}{y}{(\binpar{\outputp{x}{y}}{@{y}})} \nonumber\\
	\bangp_{x}{P} & := & \binpar{{x}!\langle{\binpar{D_{x}}{P}}\rangle}{D_{x}} \nonumber
\end{eqnarray}

\begin{eqnarray}
	\bangp_{x}{P} & & \nonumber\\
	=
	& {x}!\langle{(\prefix{x}{y}{(\outputp{x}{y} | @{y})) | P}}\rangle 
	      | \prefix{x}{y}{(\outputp{x}{y} | @{y})} & \nonumber\\
	\red
	& (\outputp{x}{y} | @{y})\substn{\quotep{(\prefix{x}{y}{(@{y} | \outputp{x}{y})) | P}}}{y} & \nonumber\\
	=
	& \outputp{x}{\quotep{(\prefix{x}{y}{(\outputp{x}{y} | @{y})) | P}}}
	  | {(\prefix{x}{y}{(\outputp{x}{y} | @{y})) | P}} & \nonumber\\
	\red
	& \ldots & \nonumber\\
	\red^*
	& P | P | \ldots & \nonumber
\end{eqnarray}

Of course, this encoding, as an implementation, runs away, unfolding
$\bangp{P}$ eagerly. A lazier and more implementable replication
operator, restricted to input-guarded processes, may be obtained as follows.

\begin{eqnarray}
\bangp{\prefix{u}{v}{P}} 
	:= 
	\binpar{\lift{x}{\prefix{u}{v}{(\binpar{D(x)}{P})}}}{D(x)} \nonumber
\end{eqnarray}

\begin{remark}
  Note that the lazier definition still does not deal with summation
  or mixed summation (i.e. sums over input and output). The reader is
  invited to construct definitions of replication that deal with these
  features. 

  Further, the definitions are parameterized in a name, $x$. Can you,
  gentle reader, make a definition that eliminates this parameter and
  guarantees no accidental interaction between the replication
  machinery and the process being replicated -- i.e. no accidental
  sharing of names used by the process to get its work done and the
  name(s) used by the replication to effect copying. This latter
  revision of the definition of replication is crucial to obtaining
  the expected identity $!!P \sim !P$.
\end{remark}

\begin{remark}\label{rem:paradoxical_combinator}
  The reader familiar with the lambda calculus will have noticed the
  similarity between $D$ and the paradoxical combinator.

  [Ed. note: the existence of this seems to suggest we have to be more
  restrictive on the set of processes and names we admit if we are to
  support no-cloning.]
\end{remark}

\subsubsection{Bisimulation}

The computational dynamics gives rise to another kind of equivalence,
the equivalence of computational behavior. As previously mentioned
this is typically captured \emph{via} some form of bisimulation.

% The notion we use in this paper is weak barbed bisimulation
% \cite{milner91polyadicpi}.

The notion we use in this paper is derived from weak barbed
bisimulation \cite{milner91polyadicpi}. 

\begin{definition}
An \emph{observation relation}, $\downarrow_{\mathcal N}$, over a set
of names, $\mathcal N$, is the smallest relation satisfying the rules
below.

\infrule[Out-barb]{y \in {\mathcal N}, \; x \nameeq y}
		  {\outputp{x}{v} \downarrow_{\mathcal N} x}
\infrule[Par-barb]{\mbox{$P\downarrow_{\mathcal N} x$ or $Q\downarrow_{\mathcal N} x$}}
		  {\binpar{P}{Q} \downarrow_{\mathcal N} x}

We write $P \Downarrow_{\mathcal N} x$ if there is $Q$ such that 
$P \wred Q$ and $Q \downarrow_{\mathcal N} x$.
\end{definition}

\begin{definition}
%\label{def.bbisim}
An  ${\mathcal N}$-\emph{barbed bisimulation} over a set of names, ${\mathcal N}$, is a symmetric binary relation 
${\mathcal S}_{\mathcal N}$ between agents such that $P\rel{S}_{\mathcal N}Q$ implies:
\begin{enumerate}
\item If $P \red P'$ then $Q \wred Q'$ and $P'\rel{S}_{\mathcal N} Q'$.
\item If $P\downarrow_{\mathcal N} x$, then $Q\Downarrow_{\mathcal N} x$.
\end{enumerate}
$P$ is ${\mathcal N}$-barbed bisimilar to $Q$, written
$P \wbbisim_{\mathcal N} Q$, if $P \rel{S}_{\mathcal N} Q$ for some ${\mathcal N}$-barbed bisimulation ${\mathcal S}_{\mathcal N}$.
\end{definition}

$\mathcal{R} \subseteq \pi \times \pi$

$P \mathcal{R} Q => \forall P'. P \red P' \Rightarrow \exists Q'. Q \red Q', P' \mathcal{R} Q'$

$P \vdash x \Rightarrow Q \vdash x$

\begin{mathpar}
  \inferrule*[lab=Out-barb]{x \nameeq y}{{y}!\langle{Q}\rangle \vdash x}
  \and
  \inferrule*[lab=Par-barb]{\mbox{$P\vdash x$ or $Q\vdash x$}}{\binpar{P}{Q} \vdash x}
\end{mathpar}

\subsubsection{Contexts}

One of the principle advantages of computational calculi like the
$\pi$-calculus is a well-defined notion of context,
contextual-equivalence and a correlation between
contextual-equivalence and notions of bisimulation. The notion of
context allows the decomposition of a process into (sub-)process and
its syntactic environment, its context. Thus, a context may be
thought of as a process with a ``hole'' (written $\Box$) in it. The
application of a context $M$ to a process $P$, written $M[P]$, is
tantamount to filling the hole in $M$ with $P$. In this paper we do
not need the full weight of this theory, but do make use of the notion
of context in the proof the main theorem. 

\begin{mathpar}
  \inferrule* [lab=summation] {} {{M_{M},M_{N}} \bc \Box \;|\; x.M_{A} \;|\; M_{M}+M_{N}}
  \and
  \inferrule* [lab=agent] {} {{M_{A}} \bc (\vec{x})M_{P} \;| \; \clift{P_0,\ldots,M_{P},\ldots,P_N}}
  \and \\
  \inferrule* [lab=process] {} {{M_{P}} \bc M_{N} \;| \;P|M_{P} }
\end{mathpar} 

\begin{mathpar}
  \inferrule* [lab=sychronization] {} {M_{N} \bc \Box \;|\; x?M_{F} \;|\; x!M_{C}}
  \and
  \inferrule* [lab=abstraction] {} {{M_{F}} \bc (x)M_{P} }
  \and
  \inferrule* [lab=concretion] {} {{M_{C}} \bc \langle M_{P} \rangle }
  \and \\
  \inferrule* [lab=process] {} {{M_{P}} \bc M_{N} \;| \;P|M_{P} }
\end{mathpar}

\begin{definition}[contextual application] Given a context $M$, and
  process $P$, we define the \emph{contextual application}, $M[P] :=
  M\{P/\Box\}$. That is, the contextual application of M to P is the
  substitution of $P$ for $\Box$ in $M$.
\end{definition}

$\meaningof{-} : L \to \mathcal{P}(\pi)$

\begin{mathpar}
  \inferrule* [lab=collection] {} {\meaningof{true} = \pi, \and \meaningof{~E} = \pi \setminus \meaningof{E}, \and \meaningof{E_{1} \& E_{2}} = \meaningof{E_{1}} \cap \meaningof{E_{2}}}
\end{mathpar}

\begin{mathpar}
  \inferrule* [lab=structure] {} {\meaningof{0} = \{ P \in \pi | P \equiv 0 \}, \and \\ \meaningof{E_1 | E_2} = \{ P \in \pi | P \equiv P_{1} | P_{2}, P_{1} \in \meaningof{E_{1}}, P_{2} \in \meaningof{E_2}\} }
\end{mathpar}

\begin{mathpar}
 \inferrule* [lab=behavior] {} {\meaningof{\langle a?b \rangle E} = \{ P \in \pi | P \equiv Q | u?(y)P', \\ \and \\\\ \and \\ \;\;\; u \in \meaningof{a}, \forall z.P'\{z/y\} \in \meaningof{E\{z/b\}}\}, \and \\ \meaningof{a!E} = \{ P \in \pi | P \equiv Q | x!\langle P' \rangle, x \in \meaningof{a} P' \in \meaningof{E}\} }
\end{mathpar}

\begin{mathpar}
 \inferrule* [lab=nominal] {} {\meaningof{\quotep{E}} = \{ \quotep{P} \in \quotep{\pi} | P \in \meaningof{E} \}, \and \meaningof{\quotep{P}} = \{ \quotep{Q} \in \quotep{\pi} | P \equiv Q \} \and \\ \meaningof{@\quotep{E}} = \{ P \in \pi | P \equiv @x, x \in \meaningof{E} \}}
\end{mathpar}

\begin{eqnarray*}
  \\
  \meaningof{-} : TS \to ST
\end{eqnarray*}

\begin{eqnarray*}
  \\
  L : TS \to ST
\end{eqnarray*}

\begin{eqnarray*}
  \\
  P \models E \iff P \in \meaningof{E}
\end{eqnarray*}

\begin{eqnarray*}
  P \approx_{L} Q \iff \forall E \in L. P \models E \iff Q \models E
\end{eqnarray*}

\begin{eqnarray*}
  P \approx_{K} Q
\end{eqnarray*}

\begin{eqnarray*}
  P \approx Q
\end{eqnarray*}

$\approx_{K} = \approx = \approx_{L}$

\subsubsection{Contextual duality}

Note that contexts extend the quotation operation to a family of
operations from processes to names. Given a context, $M$, we can
define a \emph{nominal context}, $\quotep{M}$ by $\quotep{M}[P] :=
\quotep{M[P]}$. To foreshadow what is to come we observe that these
operations enjoy a duality with processes very much like the duality
between vectors and maps from vectors to scalars.

Further, because the calculus is essentially higher-order, we have a
correspondence between contexts and processes. More specifically,
given a name $x$ and a context $M$ we can construct $M^{*}_{x}$ such
that 

\begin{mathpar}
  M^{*}_{x} | \lift{x}{P} \red M[P]
\end{mathpar}

namely,

\begin{mathpar}
  M^{*}_{x} := x?(u).M[\dropn{u}]
\end{mathpar}

The dependence of $M^{*}_{x}$ on a name makes it an abstraction, 

\begin{mathpar}
  M^{*} := (x)x?(u).M[\dropn{u}]
\end{mathpar}

\subsection{Additional notation}

It will sometimes be convenient to denote the process a name
quotes. We already have the notation $x = \quotep{P}$, but it will be
convenient to introduce an alternate notation, $\procn{x}$, when we
want to emphasize the connection to the use of the name. Note that, by
virtue of name equivalence, $\quotep{\procn{x}} \nameeq x$; so, the
notation is consistent with previous definitions.

Further, because names have structure it is possible to effect
substitutions on the basis of that structure. This means we need to
upgrade our notation for substitutions, which we accomplish by
adapting comprehension notation. Thus,

\begin{mathpar}
  P\{ y / x : x \in S \}
\end{mathpar}

is interpreted to mean the process derived from P by replacing (in a
capture-avoiding manner) each occurrence of $x$ in $S$ by $y$. For example,

\begin{mathpar}
  P\{ \quotep{\procn{x}|\procn{x}} / x : x \in \freenames{P} \}
\end{mathpar}

will replace each (occurrence) of a free name $x$ in $P$ by
$\quotep{\procn{x}|\procn{x}}$.

Also, we will avail ourselves of the notation $x^{L}$ and $x^{R}$ to
denote injections of a name into disjoint copies of the name
space. There are numerous ways to accomplish this. One example can be
found in \cite{MeredithR05}. This notation overloads to vectors of
names: $\vec{x}^{\pi} := (x_{i}^{\pi} \; : \; 0 \leq i < |\vec{x}| )$ where $\pi \in \{L,R\}$.

We also use $P^{\Box} := P|\Box$.

In \cite{MeredithR05} an interpretation of the new operator is
given. It turns out that there are several possible interpretations
all enjoying the requisite algebraic properties of the operator (see
\cite{milner91polyadicpi}). We will therefore make liberal use of
$(\nu\; \vec{x})P$.

% subsection the_syntax_and_semantics_of_the_notation_system (end)   

\input{qm2pi.qmops} 

\input{qm2pi.sterngerlach} 

\input{qm2pi.metric} 

% section concurrent_process_calculi (end)

%\input{qm2pi.proofsketch}

% section proof sketch (end)

%\input{qm2pi.slviaknots} 

% section spatial logic via knots (end)

\input{qm2pi.conclusion}

% section conclusion (end)

%\input{qm2pi.dtcodes} 

% section wiring algorithm (end)

\input{qm2pi.ack} 

% section acknowledgments (end)

\newpage


\bibliographystyle{plain}   
\bibliography{../../biblios/main.bib}

\input{qm2pi.rhodetails}

\end{document}

 

%\ifpdf
%\usepackage[pdftex]{graphicx}
%\else
%\usepackage{graphicx}
%\fi

 % \ifpdf
%  \usepackage{pdfsync}
%  \if


%\title{Brief Article}
%\author{David F. Snyder}
%\author{L.G. Meredith}

%\address{Dept. of Math., Texas State University--San Marcos, San Marcos, TX 78666}
       
\pagestyle{empty}


\begin{document}

\lstset{language=[Objective]Caml,frame=shadowbox}

\documentclass[12pt]{llncs}
%\documentclass{jktr}

\usepackage[pdftex]{hyperref}                   
\usepackage {listings}
\usepackage {mathpartir}
\usepackage{bcprules}
%\usepackage{listings}
                       
\usepackage{graphicx} 
%\usepackage[margins=2.5cm,nohead,nofoot]{geometry}
%\usepackage{geometry}
\usepackage{amsfonts}
\usepackage{amstext}
\usepackage{latexsym}
\usepackage{amssymb}
\usepackage{color}


%\include{myPreamble}
\include{qm2pi.local} 

%\ifpdf
%\usepackage[pdftex]{graphicx}
%\else
%\usepackage{graphicx}
%\fi

 % \ifpdf
%  \usepackage{pdfsync}
%  \if


%\title{Brief Article}
%\author{David F. Snyder}
%\author{L.G. Meredith}

%\address{Dept. of Math., Texas State University--San Marcos, San Marcos, TX 78666}
       
\pagestyle{empty}


\begin{document}

\lstset{language=[Objective]Caml,frame=shadowbox}

\input{qm2pi.front}

% section front matter (end)

\input{qm2pi.intro} 
 
% section introduction (end)

% \input{qm2pi.knotations} 

% section notation (end)

\input{qm2pi.process.calculi} 

% section concurrent_process_calculi_and_spatial_logics_ (end)
    
%\input{qm2pi.knots2pi} 

%\input{qm2pi.trefoil} 

%\input{qm2pi.mainthm} 

% subsection basic_interpretation (end)

%\input{qm2pi.rho.presentation} 
\subsection{The syntax and semantics of the notation system}\label{sub:the_syntax_and_semantics_of_the_notation_system} % (fold)

We now summarize a technical presentation of the calculus that
embodies our theory of dynamics. The typical presentation of such a
calculus follows the style of giving generators and relations on
them. The grammar, below, describing term constructors, freely
generates the set of processes, $\Proc$. This set is then quotiented
by a relation known as structural congruence and it is over this set
that the notion of dynamics is expressed. This presentation is
essentially that of \cite{MeredithR05} with the addition of
polyadicity and summation. For readability we have relegated some of
the technical subtleties to an appendix.

\subsubsection{Process grammar}\label{subsub:process_grammar}

\begin{mathpar}
  \inferrule* [lab=synchronization] {} {{M} \bc \pzero \;|\; x?F \;|\; x!C }
  \and
  \inferrule* [lab=abstraction] {} {{F} \bc (x)P}
  \and
  \inferrule* [lab=concretion] {} {{C} \bc \langle Q \rangle}
  \and
  \inferrule* [lab=process] {} {{P,Q} \bc M \;| \;P|Q \;|\; @{x}}
  \and
  \inferrule* [lab=name] {} {{x} \bc \quotep{P}}
\end{mathpar} 

Note that $\vec{x}$ (resp. $\vec{P}$) denotes a vector of names
(resp. processes) of length $|\vec{x}|$ (resp. $|\vec{P}|$). We adopt
the following useful abbreviations.

\begin{mathpar}
   x?(\vec{y}).P := x.(\vec{y})P \and  x\clift{\vec{P}} := x.\clift{\vec{P}}
   \and x!(y) := \lift{x}{\dropn{y}}
   \and \Pi_{i=0}^{n-1}P_i := P_0 | \ldots | P_{n-1}
\end{mathpar}

\subsubsection{Structural congruence}

\paragraph{Free and bound names and alpha-equivalence.} At the
core of structural equivalence is alpha-equivalence which identifies
process that are the same up to a change of variable. Formally, we
recognize the distinction between free and bound names. The free names
of a process, $\freenames{P}$, may be calculated recursively as
follows:

\begin{mathpar}
\freenames{\pzero} := \emptyset
  \and \\
  \freenames{x?(y).P} := \{ x \} \cup (\freenames{P} \setminus \{ y \})
  \and 
  \freenames{x!\langle P \rangle} := \{ x \} \cup \{ P \} 
  \and \\
  \freenames{P|Q} := \freenames{P} \cup \freenames{Q}
  \and \\
  \freenames{@{x}} := \{ x \}
\end{mathpar}

$\pi$
$\quotep{\pi}$

$\freenames{-} : \pi \to \mathcal{P}(\quotep{\pi})$

\begin{eqnarray*}
  \freenames{\pzero} & := & \emptyset \\
  \freenames{x?(y).P} & := & \{ x \} \cup (\freenames{P} \setminus \{ y \}) \\
  \freenames{x!\langle P \rangle} & := & \{ x \} \cup \{ P \} \\
  \freenames{P|Q} & := & \freenames{P} \cup \freenames{Q} \\
  \freenames{\dropn{x}} & := & \{ x \}
\end{eqnarray*}

The bound names of a process, $\boundnames{P}$, are those names occurring in $P$
that are not free. For example, in $x?(y).0$, the name $x$ is free, while $y$ is bound.

\begin{mathpar}
  \inferrule* [lab=monoidal-laws] {} { P|Q \equiv Q|P \and P|0 \equiv P \and P|(Q|R) \equiv (P|Q)|R }
\end{mathpar}

\begin{mathpar}
  \inferrule* [lab=alpha-equivalence] {} { (x)P \equiv (y)P\{y/x\} \and y \not\in \freenames{P} }
\end{mathpar}

\begin{definition}
Then two processes, $P,Q$, are alpha-equivalent if $P = Q\{\vec{y}/\vec{x}\}$ for
some $\vec{x} \in \boundnames{Q},\vec{y} \in \boundnames{P}$, where $Q\{\vec{y}/\vec{x}\}$
denotes the capture-avoiding substitution of $\vec{y}$ for $\vec{x}$ in $Q$.
\end{definition}

\begin{definition}
  The {\em structural congruence} \cite{SangiorgiWalker} , $\equiv$,
  between processes is the least congruence containing
  alpha-equivalence, satisfying the abelian monoid laws
  (associativity, commutativity and $\pzero$ as identity) for parallel
  composition $|$ and for summation $+$.
\end{definition}

\subsection{Name equivalence}

We take name equivalence, written $\nameeq$, to be the smallest
equivalence relation generated by the following rules.

\begin{mathpar}
\inferrule*[lab=Quote-drop]
{ }
{ \quotep{@{x}} \nameeq x }

\inferrule*[lab=Struct-equiv]
{ P \scong Q }
{ \quotep{P} \nameeq \quotep{Q} }
\end{mathpar}

The astute reader will have noticed that the mutual recursion of names
and processes imposes a mutual recursion on alpha-equivalence and
structural equivalence via name-equivalence. Fortunately, all of this
works out pleasantly and we may calculate in the natural way, free of
concern. The reader interested in the details is referred to the
appendix \ref{appendix:rho_details}.

\subsection{Substitution}

We use $\Proc$ for the set of processes, $\QProc$ for the set of
names, and $\id{\{}\vec{y} / \vec{x} \id{\}}$ to denote partial maps,
$s : \QProc \rightarrow \QProc$. A map, $s$ lifts, uniquely, to a map
on process terms, $\widehat{s} : \Proc \rightarrow \Proc$ by the
following equations.

\begin{mathpar}
  (0) \psubstp{Q}{P} := 0 \\
  (R \juxtap S) \psubstp{Q}{P}
  :=    
  (R)\psubstp{Q}{P} \juxtap (S) \psubstp{Q}{P} \\
  (x?(y).R) \psubstp{Q}{P}    
  :=    
  (x)\substp{Q}{P} (z)\concat( (R \psubstn{z}{y}) \psubstp{Q}{P} ) \\
  (\lift{x}{R}) \psubstp{Q}{P}  
  :=
  \lift{(x)\substp{Q}{P}}{ R \psubstp{Q}{P} } \\
%   (\dropn{x})  \psubstp{Q}{P}       
%   := 
%   \left\{ 
%     \begin{array}{ccc} 
%       \dropn{\quotep{Q}} & & x \nameeq \quotep{P} \\
%       \dropn{x} & & otherwise \\
%     \end{array}
%   \right. 
  (\dropn{x})  \psubstp{Q}{P}       
  := 
  \left\{ 
    \begin{array}{ccc} 
      Q & & x \nameeq \quotep{P} \\
      \dropn{x} & & otherwise \\
    \end{array}
  \right.
\end{mathpar}
 

where

\begin{eqnarray}
  (x)\id{\{} \lpquote Q \rpquote / \lpquote P \rpquote \id{\}}            = 
  \left\{ 
    \begin{array}{ccc}
      \lpquote Q \rpquote & & x \nameeq \lpquote P \rpquote \\
      x & & otherwise \\
    \end{array}
  \right. \nonumber
\end{eqnarray}

and $z$ is chosen distinct from $\quotep{P}$, $\quotep{Q}$, the free
names in $Q$, and all the names in $R$. Our $\alpha$-equivalence will
be built in the standard way from this substitution.

\begin{remark}\label{rem:no_self_referential_names}
  One consequence of these definitions is that $\forall P. \quotep{P}
  \not\in \freenames{P}$.
\end{remark}

\subsection{ Dynamic quote: an example }

Anticipating something of what's to come, consider applying the
substitution, $\widehat{\id{\{}u / z \id{\}}}$, to the following pair
of processes, $\lift{w}{y!(z)}$ and $w[ \lpquote y!(z) \rpquote ]$.

\begin{eqnarray}
	\lift{w}{y!(z)}\widehat{\id{\{}u / z \id{\}}}
		& = &
		\lift{w}{y!(u)} \nonumber\\
	w[ \lpquote y!(z) \rpquote ] \widehat{ \id{\{}u / z \id{\}} }
		& = &
		w[ \lpquote y!(z) \rpquote ] \nonumber
\end{eqnarray}

Because the body of the process between quotes is impervious to
substitution, we get radically different answers. In fact, by
examining the first process in an input context,
e.g. $x?(z).\lift{w}{y!(z)}$, we see that the process under the lift
operator may be shaped by prefixed inputs binding a name inside it. In
this sense, the lift operator will be seen as a way to dynamically
construct processes before reifying them as names.

Finally equipped with these standard features we can present the
dynamics of the calculus.

\subsubsection{Operational semantics} 

Finally, we introduce the computational dynamics. What marks these
algebras as distinct from other more traditionally studied algebraic
structures, e.g. vector spaces or polynomial rings, is the manner in
which dynamics is captured. In traditional structures, dynamics is typically
expressed through morphisms between such structures, as in linear maps
between vector spaces or morphisms between rings. In algebras
associated with the semantics of computation, the dynamics is
expressed as part of the algebraic structure itself, through a
reduction reduction relation typically denoted by $\red$. Below, we
give a recursive presentation of this relation for the calculus used
in the encoding.

$\red \subseteq \pi \times \pi$
$\red : \pi \to \mathcal{P}(\pi)$

\begin{mathpar}
  \inferrule* [lab=Comm] { \textsf{match}( x_{src}, x_{trgt} ) } { x_{trgt}?(y)P \; | \; x_{src}!\langle {Q} \rangle \red P\{\quotep{Q}/y}\} }
  \and \\
  \inferrule* [lab=Par] {{P} \red {P}'} {{{P} | {Q}} \red {{P}' | {Q}}}
  \and
  \inferrule* [lab=Equiv]{{{P} \scong {P}'} \andalso {{P}' \red {Q}'} \andalso {{Q}' \scong {Q}}}{{P} \red {Q}}
\end{mathpar}

\begin{eqnarray*}
  match_{\equiv} (\quotep{P},\quotep{Q}) & := & P \equiv Q \\
  match_{\dagger}(\quotep{P},\quotep{Q}) & := & \forall R. P|Q \red^{*} R => R \red^{*} 0 \\
  match_{K}(\quotep{P},\quotep{Q}) & := & K \mbox{ for some context } K
\end{eqnarray*}

$u?(x)P | u!\langle Q \rangle \red P\{\quotep{Q}/x\}$

%We write $\wred$ for $\red^*$, and $P\red$ if $\exists Q $ such that $ P \red Q$.
We write $P\red$ if $\exists Q $ such that $ P \red Q$ and $P\not\red$, otherwise.

\section{Replication}

As mentioned before, it is known that replication (and hence
recursion) can be implemented in a higher-order process algebra
\cite{SangiorgiWalker}. As our first example of calculation with the
machinery thus far presented we give the construction explicitly in
the {\rhoc}.

\begin{eqnarray}
	D_{x} & := & \prefix{x}{y}{(\binpar{\outputp{x}{y}}{@{y}})} \nonumber\\
	\bangp_{x}{P} & := & \binpar{{x}!\langle{\binpar{D_{x}}{P}}\rangle}{D_{x}} \nonumber
\end{eqnarray}

\begin{eqnarray}
	\bangp_{x}{P} & & \nonumber\\
	=
	& {x}!\langle{(\prefix{x}{y}{(\outputp{x}{y} | @{y})) | P}}\rangle 
	      | \prefix{x}{y}{(\outputp{x}{y} | @{y})} & \nonumber\\
	\red
	& (\outputp{x}{y} | @{y})\substn{\quotep{(\prefix{x}{y}{(@{y} | \outputp{x}{y})) | P}}}{y} & \nonumber\\
	=
	& \outputp{x}{\quotep{(\prefix{x}{y}{(\outputp{x}{y} | @{y})) | P}}}
	  | {(\prefix{x}{y}{(\outputp{x}{y} | @{y})) | P}} & \nonumber\\
	\red
	& \ldots & \nonumber\\
	\red^*
	& P | P | \ldots & \nonumber
\end{eqnarray}

Of course, this encoding, as an implementation, runs away, unfolding
$\bangp{P}$ eagerly. A lazier and more implementable replication
operator, restricted to input-guarded processes, may be obtained as follows.

\begin{eqnarray}
\bangp{\prefix{u}{v}{P}} 
	:= 
	\binpar{\lift{x}{\prefix{u}{v}{(\binpar{D(x)}{P})}}}{D(x)} \nonumber
\end{eqnarray}

\begin{remark}
  Note that the lazier definition still does not deal with summation
  or mixed summation (i.e. sums over input and output). The reader is
  invited to construct definitions of replication that deal with these
  features. 

  Further, the definitions are parameterized in a name, $x$. Can you,
  gentle reader, make a definition that eliminates this parameter and
  guarantees no accidental interaction between the replication
  machinery and the process being replicated -- i.e. no accidental
  sharing of names used by the process to get its work done and the
  name(s) used by the replication to effect copying. This latter
  revision of the definition of replication is crucial to obtaining
  the expected identity $!!P \sim !P$.
\end{remark}

\begin{remark}\label{rem:paradoxical_combinator}
  The reader familiar with the lambda calculus will have noticed the
  similarity between $D$ and the paradoxical combinator.

  [Ed. note: the existence of this seems to suggest we have to be more
  restrictive on the set of processes and names we admit if we are to
  support no-cloning.]
\end{remark}

\subsubsection{Bisimulation}

The computational dynamics gives rise to another kind of equivalence,
the equivalence of computational behavior. As previously mentioned
this is typically captured \emph{via} some form of bisimulation.

% The notion we use in this paper is weak barbed bisimulation
% \cite{milner91polyadicpi}.

The notion we use in this paper is derived from weak barbed
bisimulation \cite{milner91polyadicpi}. 

\begin{definition}
An \emph{observation relation}, $\downarrow_{\mathcal N}$, over a set
of names, $\mathcal N$, is the smallest relation satisfying the rules
below.

\infrule[Out-barb]{y \in {\mathcal N}, \; x \nameeq y}
		  {\outputp{x}{v} \downarrow_{\mathcal N} x}
\infrule[Par-barb]{\mbox{$P\downarrow_{\mathcal N} x$ or $Q\downarrow_{\mathcal N} x$}}
		  {\binpar{P}{Q} \downarrow_{\mathcal N} x}

We write $P \Downarrow_{\mathcal N} x$ if there is $Q$ such that 
$P \wred Q$ and $Q \downarrow_{\mathcal N} x$.
\end{definition}

\begin{definition}
%\label{def.bbisim}
An  ${\mathcal N}$-\emph{barbed bisimulation} over a set of names, ${\mathcal N}$, is a symmetric binary relation 
${\mathcal S}_{\mathcal N}$ between agents such that $P\rel{S}_{\mathcal N}Q$ implies:
\begin{enumerate}
\item If $P \red P'$ then $Q \wred Q'$ and $P'\rel{S}_{\mathcal N} Q'$.
\item If $P\downarrow_{\mathcal N} x$, then $Q\Downarrow_{\mathcal N} x$.
\end{enumerate}
$P$ is ${\mathcal N}$-barbed bisimilar to $Q$, written
$P \wbbisim_{\mathcal N} Q$, if $P \rel{S}_{\mathcal N} Q$ for some ${\mathcal N}$-barbed bisimulation ${\mathcal S}_{\mathcal N}$.
\end{definition}

$\mathcal{R} \subseteq \pi \times \pi$

$P \mathcal{R} Q => \forall P'. P \red P' \Rightarrow \exists Q'. Q \red Q', P' \mathcal{R} Q'$

$P \vdash x \Rightarrow Q \vdash x$

\begin{mathpar}
  \inferrule*[lab=Out-barb]{x \nameeq y}{{y}!\langle{Q}\rangle \vdash x}
  \and
  \inferrule*[lab=Par-barb]{\mbox{$P\vdash x$ or $Q\vdash x$}}{\binpar{P}{Q} \vdash x}
\end{mathpar}

\subsubsection{Contexts}

One of the principle advantages of computational calculi like the
$\pi$-calculus is a well-defined notion of context,
contextual-equivalence and a correlation between
contextual-equivalence and notions of bisimulation. The notion of
context allows the decomposition of a process into (sub-)process and
its syntactic environment, its context. Thus, a context may be
thought of as a process with a ``hole'' (written $\Box$) in it. The
application of a context $M$ to a process $P$, written $M[P]$, is
tantamount to filling the hole in $M$ with $P$. In this paper we do
not need the full weight of this theory, but do make use of the notion
of context in the proof the main theorem. 

\begin{mathpar}
  \inferrule* [lab=summation] {} {{M_{M},M_{N}} \bc \Box \;|\; x.M_{A} \;|\; M_{M}+M_{N}}
  \and
  \inferrule* [lab=agent] {} {{M_{A}} \bc (\vec{x})M_{P} \;| \; \clift{P_0,\ldots,M_{P},\ldots,P_N}}
  \and \\
  \inferrule* [lab=process] {} {{M_{P}} \bc M_{N} \;| \;P|M_{P} }
\end{mathpar} 

\begin{mathpar}
  \inferrule* [lab=sychronization] {} {M_{N} \bc \Box \;|\; x?M_{F} \;|\; x!M_{C}}
  \and
  \inferrule* [lab=abstraction] {} {{M_{F}} \bc (x)M_{P} }
  \and
  \inferrule* [lab=concretion] {} {{M_{C}} \bc \langle M_{P} \rangle }
  \and \\
  \inferrule* [lab=process] {} {{M_{P}} \bc M_{N} \;| \;P|M_{P} }
\end{mathpar}

\begin{definition}[contextual application] Given a context $M$, and
  process $P$, we define the \emph{contextual application}, $M[P] :=
  M\{P/\Box\}$. That is, the contextual application of M to P is the
  substitution of $P$ for $\Box$ in $M$.
\end{definition}

$\meaningof{-} : L \to \mathcal{P}(\pi)$

\begin{mathpar}
  \inferrule* [lab=collection] {} {\meaningof{true} = \pi, \and \meaningof{~E} = \pi \setminus \meaningof{E}, \and \meaningof{E_{1} \& E_{2}} = \meaningof{E_{1}} \cap \meaningof{E_{2}}}
\end{mathpar}

\begin{mathpar}
  \inferrule* [lab=structure] {} {\meaningof{0} = \{ P \in \pi | P \equiv 0 \}, \and \\ \meaningof{E_1 | E_2} = \{ P \in \pi | P \equiv P_{1} | P_{2}, P_{1} \in \meaningof{E_{1}}, P_{2} \in \meaningof{E_2}\} }
\end{mathpar}

\begin{mathpar}
 \inferrule* [lab=behavior] {} {\meaningof{\langle a?b \rangle E} = \{ P \in \pi | P \equiv Q | u?(y)P', \\ \and \\\\ \and \\ \;\;\; u \in \meaningof{a}, \forall z.P'\{z/y\} \in \meaningof{E\{z/b\}}\}, \and \\ \meaningof{a!E} = \{ P \in \pi | P \equiv Q | x!\langle P' \rangle, x \in \meaningof{a} P' \in \meaningof{E}\} }
\end{mathpar}

\begin{mathpar}
 \inferrule* [lab=nominal] {} {\meaningof{\quotep{E}} = \{ \quotep{P} \in \quotep{\pi} | P \in \meaningof{E} \}, \and \meaningof{\quotep{P}} = \{ \quotep{Q} \in \quotep{\pi} | P \equiv Q \} \and \\ \meaningof{@\quotep{E}} = \{ P \in \pi | P \equiv @x, x \in \meaningof{E} \}}
\end{mathpar}

\begin{eqnarray*}
  \\
  \meaningof{-} : TS \to ST
\end{eqnarray*}

\begin{eqnarray*}
  \\
  L : TS \to ST
\end{eqnarray*}

\begin{eqnarray*}
  \\
  P \models E \iff P \in \meaningof{E}
\end{eqnarray*}

\begin{eqnarray*}
  P \approx_{L} Q \iff \forall E \in L. P \models E \iff Q \models E
\end{eqnarray*}

\begin{eqnarray*}
  P \approx_{K} Q
\end{eqnarray*}

\begin{eqnarray*}
  P \approx Q
\end{eqnarray*}

$\approx_{K} = \approx = \approx_{L}$

\subsubsection{Contextual duality}

Note that contexts extend the quotation operation to a family of
operations from processes to names. Given a context, $M$, we can
define a \emph{nominal context}, $\quotep{M}$ by $\quotep{M}[P] :=
\quotep{M[P]}$. To foreshadow what is to come we observe that these
operations enjoy a duality with processes very much like the duality
between vectors and maps from vectors to scalars.

Further, because the calculus is essentially higher-order, we have a
correspondence between contexts and processes. More specifically,
given a name $x$ and a context $M$ we can construct $M^{*}_{x}$ such
that 

\begin{mathpar}
  M^{*}_{x} | \lift{x}{P} \red M[P]
\end{mathpar}

namely,

\begin{mathpar}
  M^{*}_{x} := x?(u).M[\dropn{u}]
\end{mathpar}

The dependence of $M^{*}_{x}$ on a name makes it an abstraction, 

\begin{mathpar}
  M^{*} := (x)x?(u).M[\dropn{u}]
\end{mathpar}

\subsection{Additional notation}

It will sometimes be convenient to denote the process a name
quotes. We already have the notation $x = \quotep{P}$, but it will be
convenient to introduce an alternate notation, $\procn{x}$, when we
want to emphasize the connection to the use of the name. Note that, by
virtue of name equivalence, $\quotep{\procn{x}} \nameeq x$; so, the
notation is consistent with previous definitions.

Further, because names have structure it is possible to effect
substitutions on the basis of that structure. This means we need to
upgrade our notation for substitutions, which we accomplish by
adapting comprehension notation. Thus,

\begin{mathpar}
  P\{ y / x : x \in S \}
\end{mathpar}

is interpreted to mean the process derived from P by replacing (in a
capture-avoiding manner) each occurrence of $x$ in $S$ by $y$. For example,

\begin{mathpar}
  P\{ \quotep{\procn{x}|\procn{x}} / x : x \in \freenames{P} \}
\end{mathpar}

will replace each (occurrence) of a free name $x$ in $P$ by
$\quotep{\procn{x}|\procn{x}}$.

Also, we will avail ourselves of the notation $x^{L}$ and $x^{R}$ to
denote injections of a name into disjoint copies of the name
space. There are numerous ways to accomplish this. One example can be
found in \cite{MeredithR05}. This notation overloads to vectors of
names: $\vec{x}^{\pi} := (x_{i}^{\pi} \; : \; 0 \leq i < |\vec{x}| )$ where $\pi \in \{L,R\}$.

We also use $P^{\Box} := P|\Box$.

In \cite{MeredithR05} an interpretation of the new operator is
given. It turns out that there are several possible interpretations
all enjoying the requisite algebraic properties of the operator (see
\cite{milner91polyadicpi}). We will therefore make liberal use of
$(\nu\; \vec{x})P$.

% subsection the_syntax_and_semantics_of_the_notation_system (end)   

\input{qm2pi.qmops} 

\input{qm2pi.sterngerlach} 

\input{qm2pi.metric} 

% section concurrent_process_calculi (end)

%\input{qm2pi.proofsketch}

% section proof sketch (end)

%\input{qm2pi.slviaknots} 

% section spatial logic via knots (end)

\input{qm2pi.conclusion}

% section conclusion (end)

%\input{qm2pi.dtcodes} 

% section wiring algorithm (end)

\input{qm2pi.ack} 

% section acknowledgments (end)

\newpage


\bibliographystyle{plain}   
\bibliography{../../biblios/main.bib}

\input{qm2pi.rhodetails}

\end{document}



% section front matter (end)

\section{Introduction}\label{sec:introduction} % (fold)
In this draft of the material i am going to have to dispense with the
usual writing conventions adopted in papers on these topics. i'm going
to have adopt whatever tone i need at the time i'm writing up the
calculations. Sometimes this may be very conversational; others it may
be the barest mathematical grunts; others still it may be that i have
lifted text from one of my other papers because the exposition of some
point was better said there. i hope that my readers are not unduly put
out by this decision. i'm not doing this to flout convention or be
rebellious. i find these calculations very technically challenging. To
keep everything going technically, something has to give; i have to
let go of some cognitive burden. So, the academic writing style --
with all of its trade-offs in terms of facilitating technical
communication -- is what i'm letting go of. Perhaps subsequent drafts
can be tightened and polished, but for now, i'm going to speak as if
we were sitting together in a coffee shop with a laptop, wifi and a
pad of paper and a pencil.

So, here's what i have to say. We -- you and i, comfortably ensconced
in our coffee shop and well-equipped with our tools -- can realize and
carry out the calculations of quantum mechanics over a very different
formal theory of dynamics, a formal theory of dynamics that
corresponds to a theory of concurrent computation with
\emph{reflection}. It has the advantage that the underlying theory is
already `quantized', but supports analogues all of the continuuous
operations. Strikingly, this underlying theory has recently been
connected with a notion of metric that we can show, by calculating
together, coincides with the metric induced by the inner product.

There are a lot of reasons why you might be interested in seeing
calculations of this form. Here's why i'm interested. For the past
several centuries there has been no competitor to the ``Newtonian''
account of dynamics. As a result the predominant share of accounts of
dynamical systems and situations have had to be formulated in terms of
the Newtonian machinery. i view this as an intellectually dangerous
position to occupy. Everything, despite it's intrinsic shape, turns
into a nail to be hit with this hammer. Recently, however, the theory
of computation has matured to the point where we have candidates for
theories of dynamics that offer very different perspective on
reasoning about dynamical systems and situations. Testing these
candidates against very successful accounts of dynamical situations,
like quantum mechanics, is going to give us some sense of how mature
they are and some measure of the quality of these accounts of
dynamics.

\subsection{Summary of contributions and outline of paper}

So, we're going to develop an interpretation of the operations of
quantum mechanics normally interpreted by Hilbert spaces and
operators. We're going to do this over a theory of computation. Note
that this is very different than the usual quantum computation program
which develops notions of computation over quantum mechanics. Rather,
we are developing a story that aligns with Wheeler's slogan: It from
Bit. To do this we will first provide an account of the theory of
computation at play here. Then we will dive into a calculation-driven
interpretation of the operations of quantum mechanics.

The reason we take this approach is that -- until very recently --
there hasn't been an axiomatic account of quantum mechanics. As a
result there has been no sharp delineation of the mathematical theory
supporting interpretation of the physical theory and the physical
theory, itself. So, ambient features of the maths are free to be
exploited (or supressed) without a real accounting of their physical
relevance. There is no sharp statement ``here's the physical theory''
qua \emph{theory} and ``here's the mathematical interpretation''
enabling a judgment of how faithful the interpretation is -- apart
from experimental observation. When there is an axiomatic account we
can judge how well a given mathematical formalism supports an
interpretation of the axioms, independent of
experimentation. Likewise, we can judge how well we have captured our
physical evidence and experience with our axiomatics, independent of
any specific mathematical implementation, with accidental detail that
may or may not have physical significance. 

In lieu of a fully fleshed out and vetted axiomatic account of quantum
mechanics, interpreting the operational notions in service of modeling
physical systems will have to suffice. In other words, we are not in
the business of providing a model of Hilbert spaces and operators. We
are in the business of providing a model of quantum mechanics because
we are motivated by testing our notions of dynamics against physical
theory; and, the predictive calculations of the physical theory must
serve as the best formulation -- shy of a fully fleshed out axiomatic
account -- of the physical theory itself (as they have for scientific
theories since time immemorial). Put another way, despite a
whole-hearted commitment to an It-from-Bit ontology, we are firmly
aligned with the shut-up-and-calculate camp as the best way to obtain
results either from the physical perspective or as a quality assurance
measure of our fledgling theory of dynamics.

In detail, we present a reflective process calculus. Then we develop
intuitive correspondences between the notions available in this
calculus and the usual physical notions supporting quantum mechanical
calculations. Thus, 

\begin{table}[htp]
  \center{
    \fbox{
      \begin{tabular}{c|c}
        quantum mechanics & process calculus \\
        \hline
        scalar & name \\
        state vector & process \\
        dual & contextual duals \\
        matrix & formal sums of process-context-dual pairs \\
        orthogonality & process annihilation \\
        inner product & execution-formula + quoting
      \end{tabular}
    }
  }
  \caption{QM - process calculi correspondences}
\end{table}

Then we tighten up these intuitions to operational definitions. We
employ the Dirac notation as the best proxy we can find for an
abstract syntax of the quantum mechanical notions. The definitions we
develop put us in contact with equational constraints coming from the
theory that we demonstrate the definitions and calculations satisfy.

This puts us in a position to shut up and calculate for the
Stern-Gerlach experimental set up, showing how these predictive
calculations become calculations on processes in our theory of a
reflective process calculus.

Penultimately, we demonstrate that the notion of metric coming from
the inner product coincides with the notion of metric available from
the theory of bisimulation. This demonstration gives us the right to
think of space as arising from behavior. Finally, we consider where we
might go from the new vantage point we have obtained.

% section introduction (end) 
 
% section introduction (end)

% \documentclass[12pt]{llncs}
%\documentclass{jktr}

\usepackage[pdftex]{hyperref}                   
\usepackage {listings}
\usepackage {mathpartir}
\usepackage{bcprules}
%\usepackage{listings}
                       
\usepackage{graphicx} 
%\usepackage[margins=2.5cm,nohead,nofoot]{geometry}
%\usepackage{geometry}
\usepackage{amsfonts}
\usepackage{amstext}
\usepackage{latexsym}
\usepackage{amssymb}
\usepackage{color}


%\include{myPreamble}
\include{qm2pi.local} 

%\ifpdf
%\usepackage[pdftex]{graphicx}
%\else
%\usepackage{graphicx}
%\fi

 % \ifpdf
%  \usepackage{pdfsync}
%  \if


%\title{Brief Article}
%\author{David F. Snyder}
%\author{L.G. Meredith}

%\address{Dept. of Math., Texas State University--San Marcos, San Marcos, TX 78666}
       
\pagestyle{empty}


\begin{document}

\lstset{language=[Objective]Caml,frame=shadowbox}

\input{qm2pi.front}

% section front matter (end)

\input{qm2pi.intro} 
 
% section introduction (end)

% \input{qm2pi.knotations} 

% section notation (end)

\input{qm2pi.process.calculi} 

% section concurrent_process_calculi_and_spatial_logics_ (end)
    
%\input{qm2pi.knots2pi} 

%\input{qm2pi.trefoil} 

%\input{qm2pi.mainthm} 

% subsection basic_interpretation (end)

%\input{qm2pi.rho.presentation} 
\subsection{The syntax and semantics of the notation system}\label{sub:the_syntax_and_semantics_of_the_notation_system} % (fold)

We now summarize a technical presentation of the calculus that
embodies our theory of dynamics. The typical presentation of such a
calculus follows the style of giving generators and relations on
them. The grammar, below, describing term constructors, freely
generates the set of processes, $\Proc$. This set is then quotiented
by a relation known as structural congruence and it is over this set
that the notion of dynamics is expressed. This presentation is
essentially that of \cite{MeredithR05} with the addition of
polyadicity and summation. For readability we have relegated some of
the technical subtleties to an appendix.

\subsubsection{Process grammar}\label{subsub:process_grammar}

\begin{mathpar}
  \inferrule* [lab=synchronization] {} {{M} \bc \pzero \;|\; x?F \;|\; x!C }
  \and
  \inferrule* [lab=abstraction] {} {{F} \bc (x)P}
  \and
  \inferrule* [lab=concretion] {} {{C} \bc \langle Q \rangle}
  \and
  \inferrule* [lab=process] {} {{P,Q} \bc M \;| \;P|Q \;|\; @{x}}
  \and
  \inferrule* [lab=name] {} {{x} \bc \quotep{P}}
\end{mathpar} 

Note that $\vec{x}$ (resp. $\vec{P}$) denotes a vector of names
(resp. processes) of length $|\vec{x}|$ (resp. $|\vec{P}|$). We adopt
the following useful abbreviations.

\begin{mathpar}
   x?(\vec{y}).P := x.(\vec{y})P \and  x\clift{\vec{P}} := x.\clift{\vec{P}}
   \and x!(y) := \lift{x}{\dropn{y}}
   \and \Pi_{i=0}^{n-1}P_i := P_0 | \ldots | P_{n-1}
\end{mathpar}

\subsubsection{Structural congruence}

\paragraph{Free and bound names and alpha-equivalence.} At the
core of structural equivalence is alpha-equivalence which identifies
process that are the same up to a change of variable. Formally, we
recognize the distinction between free and bound names. The free names
of a process, $\freenames{P}$, may be calculated recursively as
follows:

\begin{mathpar}
\freenames{\pzero} := \emptyset
  \and \\
  \freenames{x?(y).P} := \{ x \} \cup (\freenames{P} \setminus \{ y \})
  \and 
  \freenames{x!\langle P \rangle} := \{ x \} \cup \{ P \} 
  \and \\
  \freenames{P|Q} := \freenames{P} \cup \freenames{Q}
  \and \\
  \freenames{@{x}} := \{ x \}
\end{mathpar}

$\pi$
$\quotep{\pi}$

$\freenames{-} : \pi \to \mathcal{P}(\quotep{\pi})$

\begin{eqnarray*}
  \freenames{\pzero} & := & \emptyset \\
  \freenames{x?(y).P} & := & \{ x \} \cup (\freenames{P} \setminus \{ y \}) \\
  \freenames{x!\langle P \rangle} & := & \{ x \} \cup \{ P \} \\
  \freenames{P|Q} & := & \freenames{P} \cup \freenames{Q} \\
  \freenames{\dropn{x}} & := & \{ x \}
\end{eqnarray*}

The bound names of a process, $\boundnames{P}$, are those names occurring in $P$
that are not free. For example, in $x?(y).0$, the name $x$ is free, while $y$ is bound.

\begin{mathpar}
  \inferrule* [lab=monoidal-laws] {} { P|Q \equiv Q|P \and P|0 \equiv P \and P|(Q|R) \equiv (P|Q)|R }
\end{mathpar}

\begin{mathpar}
  \inferrule* [lab=alpha-equivalence] {} { (x)P \equiv (y)P\{y/x\} \and y \not\in \freenames{P} }
\end{mathpar}

\begin{definition}
Then two processes, $P,Q$, are alpha-equivalent if $P = Q\{\vec{y}/\vec{x}\}$ for
some $\vec{x} \in \boundnames{Q},\vec{y} \in \boundnames{P}$, where $Q\{\vec{y}/\vec{x}\}$
denotes the capture-avoiding substitution of $\vec{y}$ for $\vec{x}$ in $Q$.
\end{definition}

\begin{definition}
  The {\em structural congruence} \cite{SangiorgiWalker} , $\equiv$,
  between processes is the least congruence containing
  alpha-equivalence, satisfying the abelian monoid laws
  (associativity, commutativity and $\pzero$ as identity) for parallel
  composition $|$ and for summation $+$.
\end{definition}

\subsection{Name equivalence}

We take name equivalence, written $\nameeq$, to be the smallest
equivalence relation generated by the following rules.

\begin{mathpar}
\inferrule*[lab=Quote-drop]
{ }
{ \quotep{@{x}} \nameeq x }

\inferrule*[lab=Struct-equiv]
{ P \scong Q }
{ \quotep{P} \nameeq \quotep{Q} }
\end{mathpar}

The astute reader will have noticed that the mutual recursion of names
and processes imposes a mutual recursion on alpha-equivalence and
structural equivalence via name-equivalence. Fortunately, all of this
works out pleasantly and we may calculate in the natural way, free of
concern. The reader interested in the details is referred to the
appendix \ref{appendix:rho_details}.

\subsection{Substitution}

We use $\Proc$ for the set of processes, $\QProc$ for the set of
names, and $\id{\{}\vec{y} / \vec{x} \id{\}}$ to denote partial maps,
$s : \QProc \rightarrow \QProc$. A map, $s$ lifts, uniquely, to a map
on process terms, $\widehat{s} : \Proc \rightarrow \Proc$ by the
following equations.

\begin{mathpar}
  (0) \psubstp{Q}{P} := 0 \\
  (R \juxtap S) \psubstp{Q}{P}
  :=    
  (R)\psubstp{Q}{P} \juxtap (S) \psubstp{Q}{P} \\
  (x?(y).R) \psubstp{Q}{P}    
  :=    
  (x)\substp{Q}{P} (z)\concat( (R \psubstn{z}{y}) \psubstp{Q}{P} ) \\
  (\lift{x}{R}) \psubstp{Q}{P}  
  :=
  \lift{(x)\substp{Q}{P}}{ R \psubstp{Q}{P} } \\
%   (\dropn{x})  \psubstp{Q}{P}       
%   := 
%   \left\{ 
%     \begin{array}{ccc} 
%       \dropn{\quotep{Q}} & & x \nameeq \quotep{P} \\
%       \dropn{x} & & otherwise \\
%     \end{array}
%   \right. 
  (\dropn{x})  \psubstp{Q}{P}       
  := 
  \left\{ 
    \begin{array}{ccc} 
      Q & & x \nameeq \quotep{P} \\
      \dropn{x} & & otherwise \\
    \end{array}
  \right.
\end{mathpar}
 

where

\begin{eqnarray}
  (x)\id{\{} \lpquote Q \rpquote / \lpquote P \rpquote \id{\}}            = 
  \left\{ 
    \begin{array}{ccc}
      \lpquote Q \rpquote & & x \nameeq \lpquote P \rpquote \\
      x & & otherwise \\
    \end{array}
  \right. \nonumber
\end{eqnarray}

and $z$ is chosen distinct from $\quotep{P}$, $\quotep{Q}$, the free
names in $Q$, and all the names in $R$. Our $\alpha$-equivalence will
be built in the standard way from this substitution.

\begin{remark}\label{rem:no_self_referential_names}
  One consequence of these definitions is that $\forall P. \quotep{P}
  \not\in \freenames{P}$.
\end{remark}

\subsection{ Dynamic quote: an example }

Anticipating something of what's to come, consider applying the
substitution, $\widehat{\id{\{}u / z \id{\}}}$, to the following pair
of processes, $\lift{w}{y!(z)}$ and $w[ \lpquote y!(z) \rpquote ]$.

\begin{eqnarray}
	\lift{w}{y!(z)}\widehat{\id{\{}u / z \id{\}}}
		& = &
		\lift{w}{y!(u)} \nonumber\\
	w[ \lpquote y!(z) \rpquote ] \widehat{ \id{\{}u / z \id{\}} }
		& = &
		w[ \lpquote y!(z) \rpquote ] \nonumber
\end{eqnarray}

Because the body of the process between quotes is impervious to
substitution, we get radically different answers. In fact, by
examining the first process in an input context,
e.g. $x?(z).\lift{w}{y!(z)}$, we see that the process under the lift
operator may be shaped by prefixed inputs binding a name inside it. In
this sense, the lift operator will be seen as a way to dynamically
construct processes before reifying them as names.

Finally equipped with these standard features we can present the
dynamics of the calculus.

\subsubsection{Operational semantics} 

Finally, we introduce the computational dynamics. What marks these
algebras as distinct from other more traditionally studied algebraic
structures, e.g. vector spaces or polynomial rings, is the manner in
which dynamics is captured. In traditional structures, dynamics is typically
expressed through morphisms between such structures, as in linear maps
between vector spaces or morphisms between rings. In algebras
associated with the semantics of computation, the dynamics is
expressed as part of the algebraic structure itself, through a
reduction reduction relation typically denoted by $\red$. Below, we
give a recursive presentation of this relation for the calculus used
in the encoding.

$\red \subseteq \pi \times \pi$
$\red : \pi \to \mathcal{P}(\pi)$

\begin{mathpar}
  \inferrule* [lab=Comm] { \textsf{match}( x_{src}, x_{trgt} ) } { x_{trgt}?(y)P \; | \; x_{src}!\langle {Q} \rangle \red P\{\quotep{Q}/y}\} }
  \and \\
  \inferrule* [lab=Par] {{P} \red {P}'} {{{P} | {Q}} \red {{P}' | {Q}}}
  \and
  \inferrule* [lab=Equiv]{{{P} \scong {P}'} \andalso {{P}' \red {Q}'} \andalso {{Q}' \scong {Q}}}{{P} \red {Q}}
\end{mathpar}

\begin{eqnarray*}
  match_{\equiv} (\quotep{P},\quotep{Q}) & := & P \equiv Q \\
  match_{\dagger}(\quotep{P},\quotep{Q}) & := & \forall R. P|Q \red^{*} R => R \red^{*} 0 \\
  match_{K}(\quotep{P},\quotep{Q}) & := & K \mbox{ for some context } K
\end{eqnarray*}

$u?(x)P | u!\langle Q \rangle \red P\{\quotep{Q}/x\}$

%We write $\wred$ for $\red^*$, and $P\red$ if $\exists Q $ such that $ P \red Q$.
We write $P\red$ if $\exists Q $ such that $ P \red Q$ and $P\not\red$, otherwise.

\section{Replication}

As mentioned before, it is known that replication (and hence
recursion) can be implemented in a higher-order process algebra
\cite{SangiorgiWalker}. As our first example of calculation with the
machinery thus far presented we give the construction explicitly in
the {\rhoc}.

\begin{eqnarray}
	D_{x} & := & \prefix{x}{y}{(\binpar{\outputp{x}{y}}{@{y}})} \nonumber\\
	\bangp_{x}{P} & := & \binpar{{x}!\langle{\binpar{D_{x}}{P}}\rangle}{D_{x}} \nonumber
\end{eqnarray}

\begin{eqnarray}
	\bangp_{x}{P} & & \nonumber\\
	=
	& {x}!\langle{(\prefix{x}{y}{(\outputp{x}{y} | @{y})) | P}}\rangle 
	      | \prefix{x}{y}{(\outputp{x}{y} | @{y})} & \nonumber\\
	\red
	& (\outputp{x}{y} | @{y})\substn{\quotep{(\prefix{x}{y}{(@{y} | \outputp{x}{y})) | P}}}{y} & \nonumber\\
	=
	& \outputp{x}{\quotep{(\prefix{x}{y}{(\outputp{x}{y} | @{y})) | P}}}
	  | {(\prefix{x}{y}{(\outputp{x}{y} | @{y})) | P}} & \nonumber\\
	\red
	& \ldots & \nonumber\\
	\red^*
	& P | P | \ldots & \nonumber
\end{eqnarray}

Of course, this encoding, as an implementation, runs away, unfolding
$\bangp{P}$ eagerly. A lazier and more implementable replication
operator, restricted to input-guarded processes, may be obtained as follows.

\begin{eqnarray}
\bangp{\prefix{u}{v}{P}} 
	:= 
	\binpar{\lift{x}{\prefix{u}{v}{(\binpar{D(x)}{P})}}}{D(x)} \nonumber
\end{eqnarray}

\begin{remark}
  Note that the lazier definition still does not deal with summation
  or mixed summation (i.e. sums over input and output). The reader is
  invited to construct definitions of replication that deal with these
  features. 

  Further, the definitions are parameterized in a name, $x$. Can you,
  gentle reader, make a definition that eliminates this parameter and
  guarantees no accidental interaction between the replication
  machinery and the process being replicated -- i.e. no accidental
  sharing of names used by the process to get its work done and the
  name(s) used by the replication to effect copying. This latter
  revision of the definition of replication is crucial to obtaining
  the expected identity $!!P \sim !P$.
\end{remark}

\begin{remark}\label{rem:paradoxical_combinator}
  The reader familiar with the lambda calculus will have noticed the
  similarity between $D$ and the paradoxical combinator.

  [Ed. note: the existence of this seems to suggest we have to be more
  restrictive on the set of processes and names we admit if we are to
  support no-cloning.]
\end{remark}

\subsubsection{Bisimulation}

The computational dynamics gives rise to another kind of equivalence,
the equivalence of computational behavior. As previously mentioned
this is typically captured \emph{via} some form of bisimulation.

% The notion we use in this paper is weak barbed bisimulation
% \cite{milner91polyadicpi}.

The notion we use in this paper is derived from weak barbed
bisimulation \cite{milner91polyadicpi}. 

\begin{definition}
An \emph{observation relation}, $\downarrow_{\mathcal N}$, over a set
of names, $\mathcal N$, is the smallest relation satisfying the rules
below.

\infrule[Out-barb]{y \in {\mathcal N}, \; x \nameeq y}
		  {\outputp{x}{v} \downarrow_{\mathcal N} x}
\infrule[Par-barb]{\mbox{$P\downarrow_{\mathcal N} x$ or $Q\downarrow_{\mathcal N} x$}}
		  {\binpar{P}{Q} \downarrow_{\mathcal N} x}

We write $P \Downarrow_{\mathcal N} x$ if there is $Q$ such that 
$P \wred Q$ and $Q \downarrow_{\mathcal N} x$.
\end{definition}

\begin{definition}
%\label{def.bbisim}
An  ${\mathcal N}$-\emph{barbed bisimulation} over a set of names, ${\mathcal N}$, is a symmetric binary relation 
${\mathcal S}_{\mathcal N}$ between agents such that $P\rel{S}_{\mathcal N}Q$ implies:
\begin{enumerate}
\item If $P \red P'$ then $Q \wred Q'$ and $P'\rel{S}_{\mathcal N} Q'$.
\item If $P\downarrow_{\mathcal N} x$, then $Q\Downarrow_{\mathcal N} x$.
\end{enumerate}
$P$ is ${\mathcal N}$-barbed bisimilar to $Q$, written
$P \wbbisim_{\mathcal N} Q$, if $P \rel{S}_{\mathcal N} Q$ for some ${\mathcal N}$-barbed bisimulation ${\mathcal S}_{\mathcal N}$.
\end{definition}

$\mathcal{R} \subseteq \pi \times \pi$

$P \mathcal{R} Q => \forall P'. P \red P' \Rightarrow \exists Q'. Q \red Q', P' \mathcal{R} Q'$

$P \vdash x \Rightarrow Q \vdash x$

\begin{mathpar}
  \inferrule*[lab=Out-barb]{x \nameeq y}{{y}!\langle{Q}\rangle \vdash x}
  \and
  \inferrule*[lab=Par-barb]{\mbox{$P\vdash x$ or $Q\vdash x$}}{\binpar{P}{Q} \vdash x}
\end{mathpar}

\subsubsection{Contexts}

One of the principle advantages of computational calculi like the
$\pi$-calculus is a well-defined notion of context,
contextual-equivalence and a correlation between
contextual-equivalence and notions of bisimulation. The notion of
context allows the decomposition of a process into (sub-)process and
its syntactic environment, its context. Thus, a context may be
thought of as a process with a ``hole'' (written $\Box$) in it. The
application of a context $M$ to a process $P$, written $M[P]$, is
tantamount to filling the hole in $M$ with $P$. In this paper we do
not need the full weight of this theory, but do make use of the notion
of context in the proof the main theorem. 

\begin{mathpar}
  \inferrule* [lab=summation] {} {{M_{M},M_{N}} \bc \Box \;|\; x.M_{A} \;|\; M_{M}+M_{N}}
  \and
  \inferrule* [lab=agent] {} {{M_{A}} \bc (\vec{x})M_{P} \;| \; \clift{P_0,\ldots,M_{P},\ldots,P_N}}
  \and \\
  \inferrule* [lab=process] {} {{M_{P}} \bc M_{N} \;| \;P|M_{P} }
\end{mathpar} 

\begin{mathpar}
  \inferrule* [lab=sychronization] {} {M_{N} \bc \Box \;|\; x?M_{F} \;|\; x!M_{C}}
  \and
  \inferrule* [lab=abstraction] {} {{M_{F}} \bc (x)M_{P} }
  \and
  \inferrule* [lab=concretion] {} {{M_{C}} \bc \langle M_{P} \rangle }
  \and \\
  \inferrule* [lab=process] {} {{M_{P}} \bc M_{N} \;| \;P|M_{P} }
\end{mathpar}

\begin{definition}[contextual application] Given a context $M$, and
  process $P$, we define the \emph{contextual application}, $M[P] :=
  M\{P/\Box\}$. That is, the contextual application of M to P is the
  substitution of $P$ for $\Box$ in $M$.
\end{definition}

$\meaningof{-} : L \to \mathcal{P}(\pi)$

\begin{mathpar}
  \inferrule* [lab=collection] {} {\meaningof{true} = \pi, \and \meaningof{~E} = \pi \setminus \meaningof{E}, \and \meaningof{E_{1} \& E_{2}} = \meaningof{E_{1}} \cap \meaningof{E_{2}}}
\end{mathpar}

\begin{mathpar}
  \inferrule* [lab=structure] {} {\meaningof{0} = \{ P \in \pi | P \equiv 0 \}, \and \\ \meaningof{E_1 | E_2} = \{ P \in \pi | P \equiv P_{1} | P_{2}, P_{1} \in \meaningof{E_{1}}, P_{2} \in \meaningof{E_2}\} }
\end{mathpar}

\begin{mathpar}
 \inferrule* [lab=behavior] {} {\meaningof{\langle a?b \rangle E} = \{ P \in \pi | P \equiv Q | u?(y)P', \\ \and \\\\ \and \\ \;\;\; u \in \meaningof{a}, \forall z.P'\{z/y\} \in \meaningof{E\{z/b\}}\}, \and \\ \meaningof{a!E} = \{ P \in \pi | P \equiv Q | x!\langle P' \rangle, x \in \meaningof{a} P' \in \meaningof{E}\} }
\end{mathpar}

\begin{mathpar}
 \inferrule* [lab=nominal] {} {\meaningof{\quotep{E}} = \{ \quotep{P} \in \quotep{\pi} | P \in \meaningof{E} \}, \and \meaningof{\quotep{P}} = \{ \quotep{Q} \in \quotep{\pi} | P \equiv Q \} \and \\ \meaningof{@\quotep{E}} = \{ P \in \pi | P \equiv @x, x \in \meaningof{E} \}}
\end{mathpar}

\begin{eqnarray*}
  \\
  \meaningof{-} : TS \to ST
\end{eqnarray*}

\begin{eqnarray*}
  \\
  L : TS \to ST
\end{eqnarray*}

\begin{eqnarray*}
  \\
  P \models E \iff P \in \meaningof{E}
\end{eqnarray*}

\begin{eqnarray*}
  P \approx_{L} Q \iff \forall E \in L. P \models E \iff Q \models E
\end{eqnarray*}

\begin{eqnarray*}
  P \approx_{K} Q
\end{eqnarray*}

\begin{eqnarray*}
  P \approx Q
\end{eqnarray*}

$\approx_{K} = \approx = \approx_{L}$

\subsubsection{Contextual duality}

Note that contexts extend the quotation operation to a family of
operations from processes to names. Given a context, $M$, we can
define a \emph{nominal context}, $\quotep{M}$ by $\quotep{M}[P] :=
\quotep{M[P]}$. To foreshadow what is to come we observe that these
operations enjoy a duality with processes very much like the duality
between vectors and maps from vectors to scalars.

Further, because the calculus is essentially higher-order, we have a
correspondence between contexts and processes. More specifically,
given a name $x$ and a context $M$ we can construct $M^{*}_{x}$ such
that 

\begin{mathpar}
  M^{*}_{x} | \lift{x}{P} \red M[P]
\end{mathpar}

namely,

\begin{mathpar}
  M^{*}_{x} := x?(u).M[\dropn{u}]
\end{mathpar}

The dependence of $M^{*}_{x}$ on a name makes it an abstraction, 

\begin{mathpar}
  M^{*} := (x)x?(u).M[\dropn{u}]
\end{mathpar}

\subsection{Additional notation}

It will sometimes be convenient to denote the process a name
quotes. We already have the notation $x = \quotep{P}$, but it will be
convenient to introduce an alternate notation, $\procn{x}$, when we
want to emphasize the connection to the use of the name. Note that, by
virtue of name equivalence, $\quotep{\procn{x}} \nameeq x$; so, the
notation is consistent with previous definitions.

Further, because names have structure it is possible to effect
substitutions on the basis of that structure. This means we need to
upgrade our notation for substitutions, which we accomplish by
adapting comprehension notation. Thus,

\begin{mathpar}
  P\{ y / x : x \in S \}
\end{mathpar}

is interpreted to mean the process derived from P by replacing (in a
capture-avoiding manner) each occurrence of $x$ in $S$ by $y$. For example,

\begin{mathpar}
  P\{ \quotep{\procn{x}|\procn{x}} / x : x \in \freenames{P} \}
\end{mathpar}

will replace each (occurrence) of a free name $x$ in $P$ by
$\quotep{\procn{x}|\procn{x}}$.

Also, we will avail ourselves of the notation $x^{L}$ and $x^{R}$ to
denote injections of a name into disjoint copies of the name
space. There are numerous ways to accomplish this. One example can be
found in \cite{MeredithR05}. This notation overloads to vectors of
names: $\vec{x}^{\pi} := (x_{i}^{\pi} \; : \; 0 \leq i < |\vec{x}| )$ where $\pi \in \{L,R\}$.

We also use $P^{\Box} := P|\Box$.

In \cite{MeredithR05} an interpretation of the new operator is
given. It turns out that there are several possible interpretations
all enjoying the requisite algebraic properties of the operator (see
\cite{milner91polyadicpi}). We will therefore make liberal use of
$(\nu\; \vec{x})P$.

% subsection the_syntax_and_semantics_of_the_notation_system (end)   

\input{qm2pi.qmops} 

\input{qm2pi.sterngerlach} 

\input{qm2pi.metric} 

% section concurrent_process_calculi (end)

%\input{qm2pi.proofsketch}

% section proof sketch (end)

%\input{qm2pi.slviaknots} 

% section spatial logic via knots (end)

\input{qm2pi.conclusion}

% section conclusion (end)

%\input{qm2pi.dtcodes} 

% section wiring algorithm (end)

\input{qm2pi.ack} 

% section acknowledgments (end)

\newpage


\bibliographystyle{plain}   
\bibliography{../../biblios/main.bib}

\input{qm2pi.rhodetails}

\end{document}

 

% section notation (end)

\input{qm2pi.process.calculi} 

% section concurrent_process_calculi_and_spatial_logics_ (end)
    
%\documentclass[12pt]{llncs}
%\documentclass{jktr}

\usepackage[pdftex]{hyperref}                   
\usepackage {listings}
\usepackage {mathpartir}
\usepackage{bcprules}
%\usepackage{listings}
                       
\usepackage{graphicx} 
%\usepackage[margins=2.5cm,nohead,nofoot]{geometry}
%\usepackage{geometry}
\usepackage{amsfonts}
\usepackage{amstext}
\usepackage{latexsym}
\usepackage{amssymb}
\usepackage{color}


%\include{myPreamble}
\include{qm2pi.local} 

%\ifpdf
%\usepackage[pdftex]{graphicx}
%\else
%\usepackage{graphicx}
%\fi

 % \ifpdf
%  \usepackage{pdfsync}
%  \if


%\title{Brief Article}
%\author{David F. Snyder}
%\author{L.G. Meredith}

%\address{Dept. of Math., Texas State University--San Marcos, San Marcos, TX 78666}
       
\pagestyle{empty}


\begin{document}

\lstset{language=[Objective]Caml,frame=shadowbox}

\input{qm2pi.front}

% section front matter (end)

\input{qm2pi.intro} 
 
% section introduction (end)

% \input{qm2pi.knotations} 

% section notation (end)

\input{qm2pi.process.calculi} 

% section concurrent_process_calculi_and_spatial_logics_ (end)
    
%\input{qm2pi.knots2pi} 

%\input{qm2pi.trefoil} 

%\input{qm2pi.mainthm} 

% subsection basic_interpretation (end)

%\input{qm2pi.rho.presentation} 
\subsection{The syntax and semantics of the notation system}\label{sub:the_syntax_and_semantics_of_the_notation_system} % (fold)

We now summarize a technical presentation of the calculus that
embodies our theory of dynamics. The typical presentation of such a
calculus follows the style of giving generators and relations on
them. The grammar, below, describing term constructors, freely
generates the set of processes, $\Proc$. This set is then quotiented
by a relation known as structural congruence and it is over this set
that the notion of dynamics is expressed. This presentation is
essentially that of \cite{MeredithR05} with the addition of
polyadicity and summation. For readability we have relegated some of
the technical subtleties to an appendix.

\subsubsection{Process grammar}\label{subsub:process_grammar}

\begin{mathpar}
  \inferrule* [lab=synchronization] {} {{M} \bc \pzero \;|\; x?F \;|\; x!C }
  \and
  \inferrule* [lab=abstraction] {} {{F} \bc (x)P}
  \and
  \inferrule* [lab=concretion] {} {{C} \bc \langle Q \rangle}
  \and
  \inferrule* [lab=process] {} {{P,Q} \bc M \;| \;P|Q \;|\; @{x}}
  \and
  \inferrule* [lab=name] {} {{x} \bc \quotep{P}}
\end{mathpar} 

Note that $\vec{x}$ (resp. $\vec{P}$) denotes a vector of names
(resp. processes) of length $|\vec{x}|$ (resp. $|\vec{P}|$). We adopt
the following useful abbreviations.

\begin{mathpar}
   x?(\vec{y}).P := x.(\vec{y})P \and  x\clift{\vec{P}} := x.\clift{\vec{P}}
   \and x!(y) := \lift{x}{\dropn{y}}
   \and \Pi_{i=0}^{n-1}P_i := P_0 | \ldots | P_{n-1}
\end{mathpar}

\subsubsection{Structural congruence}

\paragraph{Free and bound names and alpha-equivalence.} At the
core of structural equivalence is alpha-equivalence which identifies
process that are the same up to a change of variable. Formally, we
recognize the distinction between free and bound names. The free names
of a process, $\freenames{P}$, may be calculated recursively as
follows:

\begin{mathpar}
\freenames{\pzero} := \emptyset
  \and \\
  \freenames{x?(y).P} := \{ x \} \cup (\freenames{P} \setminus \{ y \})
  \and 
  \freenames{x!\langle P \rangle} := \{ x \} \cup \{ P \} 
  \and \\
  \freenames{P|Q} := \freenames{P} \cup \freenames{Q}
  \and \\
  \freenames{@{x}} := \{ x \}
\end{mathpar}

$\pi$
$\quotep{\pi}$

$\freenames{-} : \pi \to \mathcal{P}(\quotep{\pi})$

\begin{eqnarray*}
  \freenames{\pzero} & := & \emptyset \\
  \freenames{x?(y).P} & := & \{ x \} \cup (\freenames{P} \setminus \{ y \}) \\
  \freenames{x!\langle P \rangle} & := & \{ x \} \cup \{ P \} \\
  \freenames{P|Q} & := & \freenames{P} \cup \freenames{Q} \\
  \freenames{\dropn{x}} & := & \{ x \}
\end{eqnarray*}

The bound names of a process, $\boundnames{P}$, are those names occurring in $P$
that are not free. For example, in $x?(y).0$, the name $x$ is free, while $y$ is bound.

\begin{mathpar}
  \inferrule* [lab=monoidal-laws] {} { P|Q \equiv Q|P \and P|0 \equiv P \and P|(Q|R) \equiv (P|Q)|R }
\end{mathpar}

\begin{mathpar}
  \inferrule* [lab=alpha-equivalence] {} { (x)P \equiv (y)P\{y/x\} \and y \not\in \freenames{P} }
\end{mathpar}

\begin{definition}
Then two processes, $P,Q$, are alpha-equivalent if $P = Q\{\vec{y}/\vec{x}\}$ for
some $\vec{x} \in \boundnames{Q},\vec{y} \in \boundnames{P}$, where $Q\{\vec{y}/\vec{x}\}$
denotes the capture-avoiding substitution of $\vec{y}$ for $\vec{x}$ in $Q$.
\end{definition}

\begin{definition}
  The {\em structural congruence} \cite{SangiorgiWalker} , $\equiv$,
  between processes is the least congruence containing
  alpha-equivalence, satisfying the abelian monoid laws
  (associativity, commutativity and $\pzero$ as identity) for parallel
  composition $|$ and for summation $+$.
\end{definition}

\subsection{Name equivalence}

We take name equivalence, written $\nameeq$, to be the smallest
equivalence relation generated by the following rules.

\begin{mathpar}
\inferrule*[lab=Quote-drop]
{ }
{ \quotep{@{x}} \nameeq x }

\inferrule*[lab=Struct-equiv]
{ P \scong Q }
{ \quotep{P} \nameeq \quotep{Q} }
\end{mathpar}

The astute reader will have noticed that the mutual recursion of names
and processes imposes a mutual recursion on alpha-equivalence and
structural equivalence via name-equivalence. Fortunately, all of this
works out pleasantly and we may calculate in the natural way, free of
concern. The reader interested in the details is referred to the
appendix \ref{appendix:rho_details}.

\subsection{Substitution}

We use $\Proc$ for the set of processes, $\QProc$ for the set of
names, and $\id{\{}\vec{y} / \vec{x} \id{\}}$ to denote partial maps,
$s : \QProc \rightarrow \QProc$. A map, $s$ lifts, uniquely, to a map
on process terms, $\widehat{s} : \Proc \rightarrow \Proc$ by the
following equations.

\begin{mathpar}
  (0) \psubstp{Q}{P} := 0 \\
  (R \juxtap S) \psubstp{Q}{P}
  :=    
  (R)\psubstp{Q}{P} \juxtap (S) \psubstp{Q}{P} \\
  (x?(y).R) \psubstp{Q}{P}    
  :=    
  (x)\substp{Q}{P} (z)\concat( (R \psubstn{z}{y}) \psubstp{Q}{P} ) \\
  (\lift{x}{R}) \psubstp{Q}{P}  
  :=
  \lift{(x)\substp{Q}{P}}{ R \psubstp{Q}{P} } \\
%   (\dropn{x})  \psubstp{Q}{P}       
%   := 
%   \left\{ 
%     \begin{array}{ccc} 
%       \dropn{\quotep{Q}} & & x \nameeq \quotep{P} \\
%       \dropn{x} & & otherwise \\
%     \end{array}
%   \right. 
  (\dropn{x})  \psubstp{Q}{P}       
  := 
  \left\{ 
    \begin{array}{ccc} 
      Q & & x \nameeq \quotep{P} \\
      \dropn{x} & & otherwise \\
    \end{array}
  \right.
\end{mathpar}
 

where

\begin{eqnarray}
  (x)\id{\{} \lpquote Q \rpquote / \lpquote P \rpquote \id{\}}            = 
  \left\{ 
    \begin{array}{ccc}
      \lpquote Q \rpquote & & x \nameeq \lpquote P \rpquote \\
      x & & otherwise \\
    \end{array}
  \right. \nonumber
\end{eqnarray}

and $z$ is chosen distinct from $\quotep{P}$, $\quotep{Q}$, the free
names in $Q$, and all the names in $R$. Our $\alpha$-equivalence will
be built in the standard way from this substitution.

\begin{remark}\label{rem:no_self_referential_names}
  One consequence of these definitions is that $\forall P. \quotep{P}
  \not\in \freenames{P}$.
\end{remark}

\subsection{ Dynamic quote: an example }

Anticipating something of what's to come, consider applying the
substitution, $\widehat{\id{\{}u / z \id{\}}}$, to the following pair
of processes, $\lift{w}{y!(z)}$ and $w[ \lpquote y!(z) \rpquote ]$.

\begin{eqnarray}
	\lift{w}{y!(z)}\widehat{\id{\{}u / z \id{\}}}
		& = &
		\lift{w}{y!(u)} \nonumber\\
	w[ \lpquote y!(z) \rpquote ] \widehat{ \id{\{}u / z \id{\}} }
		& = &
		w[ \lpquote y!(z) \rpquote ] \nonumber
\end{eqnarray}

Because the body of the process between quotes is impervious to
substitution, we get radically different answers. In fact, by
examining the first process in an input context,
e.g. $x?(z).\lift{w}{y!(z)}$, we see that the process under the lift
operator may be shaped by prefixed inputs binding a name inside it. In
this sense, the lift operator will be seen as a way to dynamically
construct processes before reifying them as names.

Finally equipped with these standard features we can present the
dynamics of the calculus.

\subsubsection{Operational semantics} 

Finally, we introduce the computational dynamics. What marks these
algebras as distinct from other more traditionally studied algebraic
structures, e.g. vector spaces or polynomial rings, is the manner in
which dynamics is captured. In traditional structures, dynamics is typically
expressed through morphisms between such structures, as in linear maps
between vector spaces or morphisms between rings. In algebras
associated with the semantics of computation, the dynamics is
expressed as part of the algebraic structure itself, through a
reduction reduction relation typically denoted by $\red$. Below, we
give a recursive presentation of this relation for the calculus used
in the encoding.

$\red \subseteq \pi \times \pi$
$\red : \pi \to \mathcal{P}(\pi)$

\begin{mathpar}
  \inferrule* [lab=Comm] { \textsf{match}( x_{src}, x_{trgt} ) } { x_{trgt}?(y)P \; | \; x_{src}!\langle {Q} \rangle \red P\{\quotep{Q}/y}\} }
  \and \\
  \inferrule* [lab=Par] {{P} \red {P}'} {{{P} | {Q}} \red {{P}' | {Q}}}
  \and
  \inferrule* [lab=Equiv]{{{P} \scong {P}'} \andalso {{P}' \red {Q}'} \andalso {{Q}' \scong {Q}}}{{P} \red {Q}}
\end{mathpar}

\begin{eqnarray*}
  match_{\equiv} (\quotep{P},\quotep{Q}) & := & P \equiv Q \\
  match_{\dagger}(\quotep{P},\quotep{Q}) & := & \forall R. P|Q \red^{*} R => R \red^{*} 0 \\
  match_{K}(\quotep{P},\quotep{Q}) & := & K \mbox{ for some context } K
\end{eqnarray*}

$u?(x)P | u!\langle Q \rangle \red P\{\quotep{Q}/x\}$

%We write $\wred$ for $\red^*$, and $P\red$ if $\exists Q $ such that $ P \red Q$.
We write $P\red$ if $\exists Q $ such that $ P \red Q$ and $P\not\red$, otherwise.

\section{Replication}

As mentioned before, it is known that replication (and hence
recursion) can be implemented in a higher-order process algebra
\cite{SangiorgiWalker}. As our first example of calculation with the
machinery thus far presented we give the construction explicitly in
the {\rhoc}.

\begin{eqnarray}
	D_{x} & := & \prefix{x}{y}{(\binpar{\outputp{x}{y}}{@{y}})} \nonumber\\
	\bangp_{x}{P} & := & \binpar{{x}!\langle{\binpar{D_{x}}{P}}\rangle}{D_{x}} \nonumber
\end{eqnarray}

\begin{eqnarray}
	\bangp_{x}{P} & & \nonumber\\
	=
	& {x}!\langle{(\prefix{x}{y}{(\outputp{x}{y} | @{y})) | P}}\rangle 
	      | \prefix{x}{y}{(\outputp{x}{y} | @{y})} & \nonumber\\
	\red
	& (\outputp{x}{y} | @{y})\substn{\quotep{(\prefix{x}{y}{(@{y} | \outputp{x}{y})) | P}}}{y} & \nonumber\\
	=
	& \outputp{x}{\quotep{(\prefix{x}{y}{(\outputp{x}{y} | @{y})) | P}}}
	  | {(\prefix{x}{y}{(\outputp{x}{y} | @{y})) | P}} & \nonumber\\
	\red
	& \ldots & \nonumber\\
	\red^*
	& P | P | \ldots & \nonumber
\end{eqnarray}

Of course, this encoding, as an implementation, runs away, unfolding
$\bangp{P}$ eagerly. A lazier and more implementable replication
operator, restricted to input-guarded processes, may be obtained as follows.

\begin{eqnarray}
\bangp{\prefix{u}{v}{P}} 
	:= 
	\binpar{\lift{x}{\prefix{u}{v}{(\binpar{D(x)}{P})}}}{D(x)} \nonumber
\end{eqnarray}

\begin{remark}
  Note that the lazier definition still does not deal with summation
  or mixed summation (i.e. sums over input and output). The reader is
  invited to construct definitions of replication that deal with these
  features. 

  Further, the definitions are parameterized in a name, $x$. Can you,
  gentle reader, make a definition that eliminates this parameter and
  guarantees no accidental interaction between the replication
  machinery and the process being replicated -- i.e. no accidental
  sharing of names used by the process to get its work done and the
  name(s) used by the replication to effect copying. This latter
  revision of the definition of replication is crucial to obtaining
  the expected identity $!!P \sim !P$.
\end{remark}

\begin{remark}\label{rem:paradoxical_combinator}
  The reader familiar with the lambda calculus will have noticed the
  similarity between $D$ and the paradoxical combinator.

  [Ed. note: the existence of this seems to suggest we have to be more
  restrictive on the set of processes and names we admit if we are to
  support no-cloning.]
\end{remark}

\subsubsection{Bisimulation}

The computational dynamics gives rise to another kind of equivalence,
the equivalence of computational behavior. As previously mentioned
this is typically captured \emph{via} some form of bisimulation.

% The notion we use in this paper is weak barbed bisimulation
% \cite{milner91polyadicpi}.

The notion we use in this paper is derived from weak barbed
bisimulation \cite{milner91polyadicpi}. 

\begin{definition}
An \emph{observation relation}, $\downarrow_{\mathcal N}$, over a set
of names, $\mathcal N$, is the smallest relation satisfying the rules
below.

\infrule[Out-barb]{y \in {\mathcal N}, \; x \nameeq y}
		  {\outputp{x}{v} \downarrow_{\mathcal N} x}
\infrule[Par-barb]{\mbox{$P\downarrow_{\mathcal N} x$ or $Q\downarrow_{\mathcal N} x$}}
		  {\binpar{P}{Q} \downarrow_{\mathcal N} x}

We write $P \Downarrow_{\mathcal N} x$ if there is $Q$ such that 
$P \wred Q$ and $Q \downarrow_{\mathcal N} x$.
\end{definition}

\begin{definition}
%\label{def.bbisim}
An  ${\mathcal N}$-\emph{barbed bisimulation} over a set of names, ${\mathcal N}$, is a symmetric binary relation 
${\mathcal S}_{\mathcal N}$ between agents such that $P\rel{S}_{\mathcal N}Q$ implies:
\begin{enumerate}
\item If $P \red P'$ then $Q \wred Q'$ and $P'\rel{S}_{\mathcal N} Q'$.
\item If $P\downarrow_{\mathcal N} x$, then $Q\Downarrow_{\mathcal N} x$.
\end{enumerate}
$P$ is ${\mathcal N}$-barbed bisimilar to $Q$, written
$P \wbbisim_{\mathcal N} Q$, if $P \rel{S}_{\mathcal N} Q$ for some ${\mathcal N}$-barbed bisimulation ${\mathcal S}_{\mathcal N}$.
\end{definition}

$\mathcal{R} \subseteq \pi \times \pi$

$P \mathcal{R} Q => \forall P'. P \red P' \Rightarrow \exists Q'. Q \red Q', P' \mathcal{R} Q'$

$P \vdash x \Rightarrow Q \vdash x$

\begin{mathpar}
  \inferrule*[lab=Out-barb]{x \nameeq y}{{y}!\langle{Q}\rangle \vdash x}
  \and
  \inferrule*[lab=Par-barb]{\mbox{$P\vdash x$ or $Q\vdash x$}}{\binpar{P}{Q} \vdash x}
\end{mathpar}

\subsubsection{Contexts}

One of the principle advantages of computational calculi like the
$\pi$-calculus is a well-defined notion of context,
contextual-equivalence and a correlation between
contextual-equivalence and notions of bisimulation. The notion of
context allows the decomposition of a process into (sub-)process and
its syntactic environment, its context. Thus, a context may be
thought of as a process with a ``hole'' (written $\Box$) in it. The
application of a context $M$ to a process $P$, written $M[P]$, is
tantamount to filling the hole in $M$ with $P$. In this paper we do
not need the full weight of this theory, but do make use of the notion
of context in the proof the main theorem. 

\begin{mathpar}
  \inferrule* [lab=summation] {} {{M_{M},M_{N}} \bc \Box \;|\; x.M_{A} \;|\; M_{M}+M_{N}}
  \and
  \inferrule* [lab=agent] {} {{M_{A}} \bc (\vec{x})M_{P} \;| \; \clift{P_0,\ldots,M_{P},\ldots,P_N}}
  \and \\
  \inferrule* [lab=process] {} {{M_{P}} \bc M_{N} \;| \;P|M_{P} }
\end{mathpar} 

\begin{mathpar}
  \inferrule* [lab=sychronization] {} {M_{N} \bc \Box \;|\; x?M_{F} \;|\; x!M_{C}}
  \and
  \inferrule* [lab=abstraction] {} {{M_{F}} \bc (x)M_{P} }
  \and
  \inferrule* [lab=concretion] {} {{M_{C}} \bc \langle M_{P} \rangle }
  \and \\
  \inferrule* [lab=process] {} {{M_{P}} \bc M_{N} \;| \;P|M_{P} }
\end{mathpar}

\begin{definition}[contextual application] Given a context $M$, and
  process $P$, we define the \emph{contextual application}, $M[P] :=
  M\{P/\Box\}$. That is, the contextual application of M to P is the
  substitution of $P$ for $\Box$ in $M$.
\end{definition}

$\meaningof{-} : L \to \mathcal{P}(\pi)$

\begin{mathpar}
  \inferrule* [lab=collection] {} {\meaningof{true} = \pi, \and \meaningof{~E} = \pi \setminus \meaningof{E}, \and \meaningof{E_{1} \& E_{2}} = \meaningof{E_{1}} \cap \meaningof{E_{2}}}
\end{mathpar}

\begin{mathpar}
  \inferrule* [lab=structure] {} {\meaningof{0} = \{ P \in \pi | P \equiv 0 \}, \and \\ \meaningof{E_1 | E_2} = \{ P \in \pi | P \equiv P_{1} | P_{2}, P_{1} \in \meaningof{E_{1}}, P_{2} \in \meaningof{E_2}\} }
\end{mathpar}

\begin{mathpar}
 \inferrule* [lab=behavior] {} {\meaningof{\langle a?b \rangle E} = \{ P \in \pi | P \equiv Q | u?(y)P', \\ \and \\\\ \and \\ \;\;\; u \in \meaningof{a}, \forall z.P'\{z/y\} \in \meaningof{E\{z/b\}}\}, \and \\ \meaningof{a!E} = \{ P \in \pi | P \equiv Q | x!\langle P' \rangle, x \in \meaningof{a} P' \in \meaningof{E}\} }
\end{mathpar}

\begin{mathpar}
 \inferrule* [lab=nominal] {} {\meaningof{\quotep{E}} = \{ \quotep{P} \in \quotep{\pi} | P \in \meaningof{E} \}, \and \meaningof{\quotep{P}} = \{ \quotep{Q} \in \quotep{\pi} | P \equiv Q \} \and \\ \meaningof{@\quotep{E}} = \{ P \in \pi | P \equiv @x, x \in \meaningof{E} \}}
\end{mathpar}

\begin{eqnarray*}
  \\
  \meaningof{-} : TS \to ST
\end{eqnarray*}

\begin{eqnarray*}
  \\
  L : TS \to ST
\end{eqnarray*}

\begin{eqnarray*}
  \\
  P \models E \iff P \in \meaningof{E}
\end{eqnarray*}

\begin{eqnarray*}
  P \approx_{L} Q \iff \forall E \in L. P \models E \iff Q \models E
\end{eqnarray*}

\begin{eqnarray*}
  P \approx_{K} Q
\end{eqnarray*}

\begin{eqnarray*}
  P \approx Q
\end{eqnarray*}

$\approx_{K} = \approx = \approx_{L}$

\subsubsection{Contextual duality}

Note that contexts extend the quotation operation to a family of
operations from processes to names. Given a context, $M$, we can
define a \emph{nominal context}, $\quotep{M}$ by $\quotep{M}[P] :=
\quotep{M[P]}$. To foreshadow what is to come we observe that these
operations enjoy a duality with processes very much like the duality
between vectors and maps from vectors to scalars.

Further, because the calculus is essentially higher-order, we have a
correspondence between contexts and processes. More specifically,
given a name $x$ and a context $M$ we can construct $M^{*}_{x}$ such
that 

\begin{mathpar}
  M^{*}_{x} | \lift{x}{P} \red M[P]
\end{mathpar}

namely,

\begin{mathpar}
  M^{*}_{x} := x?(u).M[\dropn{u}]
\end{mathpar}

The dependence of $M^{*}_{x}$ on a name makes it an abstraction, 

\begin{mathpar}
  M^{*} := (x)x?(u).M[\dropn{u}]
\end{mathpar}

\subsection{Additional notation}

It will sometimes be convenient to denote the process a name
quotes. We already have the notation $x = \quotep{P}$, but it will be
convenient to introduce an alternate notation, $\procn{x}$, when we
want to emphasize the connection to the use of the name. Note that, by
virtue of name equivalence, $\quotep{\procn{x}} \nameeq x$; so, the
notation is consistent with previous definitions.

Further, because names have structure it is possible to effect
substitutions on the basis of that structure. This means we need to
upgrade our notation for substitutions, which we accomplish by
adapting comprehension notation. Thus,

\begin{mathpar}
  P\{ y / x : x \in S \}
\end{mathpar}

is interpreted to mean the process derived from P by replacing (in a
capture-avoiding manner) each occurrence of $x$ in $S$ by $y$. For example,

\begin{mathpar}
  P\{ \quotep{\procn{x}|\procn{x}} / x : x \in \freenames{P} \}
\end{mathpar}

will replace each (occurrence) of a free name $x$ in $P$ by
$\quotep{\procn{x}|\procn{x}}$.

Also, we will avail ourselves of the notation $x^{L}$ and $x^{R}$ to
denote injections of a name into disjoint copies of the name
space. There are numerous ways to accomplish this. One example can be
found in \cite{MeredithR05}. This notation overloads to vectors of
names: $\vec{x}^{\pi} := (x_{i}^{\pi} \; : \; 0 \leq i < |\vec{x}| )$ where $\pi \in \{L,R\}$.

We also use $P^{\Box} := P|\Box$.

In \cite{MeredithR05} an interpretation of the new operator is
given. It turns out that there are several possible interpretations
all enjoying the requisite algebraic properties of the operator (see
\cite{milner91polyadicpi}). We will therefore make liberal use of
$(\nu\; \vec{x})P$.

% subsection the_syntax_and_semantics_of_the_notation_system (end)   

\input{qm2pi.qmops} 

\input{qm2pi.sterngerlach} 

\input{qm2pi.metric} 

% section concurrent_process_calculi (end)

%\input{qm2pi.proofsketch}

% section proof sketch (end)

%\input{qm2pi.slviaknots} 

% section spatial logic via knots (end)

\input{qm2pi.conclusion}

% section conclusion (end)

%\input{qm2pi.dtcodes} 

% section wiring algorithm (end)

\input{qm2pi.ack} 

% section acknowledgments (end)

\newpage


\bibliographystyle{plain}   
\bibliography{../../biblios/main.bib}

\input{qm2pi.rhodetails}

\end{document}

 

%\documentclass[12pt]{llncs}
%\documentclass{jktr}

\usepackage[pdftex]{hyperref}                   
\usepackage {listings}
\usepackage {mathpartir}
\usepackage{bcprules}
%\usepackage{listings}
                       
\usepackage{graphicx} 
%\usepackage[margins=2.5cm,nohead,nofoot]{geometry}
%\usepackage{geometry}
\usepackage{amsfonts}
\usepackage{amstext}
\usepackage{latexsym}
\usepackage{amssymb}
\usepackage{color}


%\include{myPreamble}
\include{qm2pi.local} 

%\ifpdf
%\usepackage[pdftex]{graphicx}
%\else
%\usepackage{graphicx}
%\fi

 % \ifpdf
%  \usepackage{pdfsync}
%  \if


%\title{Brief Article}
%\author{David F. Snyder}
%\author{L.G. Meredith}

%\address{Dept. of Math., Texas State University--San Marcos, San Marcos, TX 78666}
       
\pagestyle{empty}


\begin{document}

\lstset{language=[Objective]Caml,frame=shadowbox}

\input{qm2pi.front}

% section front matter (end)

\input{qm2pi.intro} 
 
% section introduction (end)

% \input{qm2pi.knotations} 

% section notation (end)

\input{qm2pi.process.calculi} 

% section concurrent_process_calculi_and_spatial_logics_ (end)
    
%\input{qm2pi.knots2pi} 

%\input{qm2pi.trefoil} 

%\input{qm2pi.mainthm} 

% subsection basic_interpretation (end)

%\input{qm2pi.rho.presentation} 
\subsection{The syntax and semantics of the notation system}\label{sub:the_syntax_and_semantics_of_the_notation_system} % (fold)

We now summarize a technical presentation of the calculus that
embodies our theory of dynamics. The typical presentation of such a
calculus follows the style of giving generators and relations on
them. The grammar, below, describing term constructors, freely
generates the set of processes, $\Proc$. This set is then quotiented
by a relation known as structural congruence and it is over this set
that the notion of dynamics is expressed. This presentation is
essentially that of \cite{MeredithR05} with the addition of
polyadicity and summation. For readability we have relegated some of
the technical subtleties to an appendix.

\subsubsection{Process grammar}\label{subsub:process_grammar}

\begin{mathpar}
  \inferrule* [lab=synchronization] {} {{M} \bc \pzero \;|\; x?F \;|\; x!C }
  \and
  \inferrule* [lab=abstraction] {} {{F} \bc (x)P}
  \and
  \inferrule* [lab=concretion] {} {{C} \bc \langle Q \rangle}
  \and
  \inferrule* [lab=process] {} {{P,Q} \bc M \;| \;P|Q \;|\; @{x}}
  \and
  \inferrule* [lab=name] {} {{x} \bc \quotep{P}}
\end{mathpar} 

Note that $\vec{x}$ (resp. $\vec{P}$) denotes a vector of names
(resp. processes) of length $|\vec{x}|$ (resp. $|\vec{P}|$). We adopt
the following useful abbreviations.

\begin{mathpar}
   x?(\vec{y}).P := x.(\vec{y})P \and  x\clift{\vec{P}} := x.\clift{\vec{P}}
   \and x!(y) := \lift{x}{\dropn{y}}
   \and \Pi_{i=0}^{n-1}P_i := P_0 | \ldots | P_{n-1}
\end{mathpar}

\subsubsection{Structural congruence}

\paragraph{Free and bound names and alpha-equivalence.} At the
core of structural equivalence is alpha-equivalence which identifies
process that are the same up to a change of variable. Formally, we
recognize the distinction between free and bound names. The free names
of a process, $\freenames{P}$, may be calculated recursively as
follows:

\begin{mathpar}
\freenames{\pzero} := \emptyset
  \and \\
  \freenames{x?(y).P} := \{ x \} \cup (\freenames{P} \setminus \{ y \})
  \and 
  \freenames{x!\langle P \rangle} := \{ x \} \cup \{ P \} 
  \and \\
  \freenames{P|Q} := \freenames{P} \cup \freenames{Q}
  \and \\
  \freenames{@{x}} := \{ x \}
\end{mathpar}

$\pi$
$\quotep{\pi}$

$\freenames{-} : \pi \to \mathcal{P}(\quotep{\pi})$

\begin{eqnarray*}
  \freenames{\pzero} & := & \emptyset \\
  \freenames{x?(y).P} & := & \{ x \} \cup (\freenames{P} \setminus \{ y \}) \\
  \freenames{x!\langle P \rangle} & := & \{ x \} \cup \{ P \} \\
  \freenames{P|Q} & := & \freenames{P} \cup \freenames{Q} \\
  \freenames{\dropn{x}} & := & \{ x \}
\end{eqnarray*}

The bound names of a process, $\boundnames{P}$, are those names occurring in $P$
that are not free. For example, in $x?(y).0$, the name $x$ is free, while $y$ is bound.

\begin{mathpar}
  \inferrule* [lab=monoidal-laws] {} { P|Q \equiv Q|P \and P|0 \equiv P \and P|(Q|R) \equiv (P|Q)|R }
\end{mathpar}

\begin{mathpar}
  \inferrule* [lab=alpha-equivalence] {} { (x)P \equiv (y)P\{y/x\} \and y \not\in \freenames{P} }
\end{mathpar}

\begin{definition}
Then two processes, $P,Q$, are alpha-equivalent if $P = Q\{\vec{y}/\vec{x}\}$ for
some $\vec{x} \in \boundnames{Q},\vec{y} \in \boundnames{P}$, where $Q\{\vec{y}/\vec{x}\}$
denotes the capture-avoiding substitution of $\vec{y}$ for $\vec{x}$ in $Q$.
\end{definition}

\begin{definition}
  The {\em structural congruence} \cite{SangiorgiWalker} , $\equiv$,
  between processes is the least congruence containing
  alpha-equivalence, satisfying the abelian monoid laws
  (associativity, commutativity and $\pzero$ as identity) for parallel
  composition $|$ and for summation $+$.
\end{definition}

\subsection{Name equivalence}

We take name equivalence, written $\nameeq$, to be the smallest
equivalence relation generated by the following rules.

\begin{mathpar}
\inferrule*[lab=Quote-drop]
{ }
{ \quotep{@{x}} \nameeq x }

\inferrule*[lab=Struct-equiv]
{ P \scong Q }
{ \quotep{P} \nameeq \quotep{Q} }
\end{mathpar}

The astute reader will have noticed that the mutual recursion of names
and processes imposes a mutual recursion on alpha-equivalence and
structural equivalence via name-equivalence. Fortunately, all of this
works out pleasantly and we may calculate in the natural way, free of
concern. The reader interested in the details is referred to the
appendix \ref{appendix:rho_details}.

\subsection{Substitution}

We use $\Proc$ for the set of processes, $\QProc$ for the set of
names, and $\id{\{}\vec{y} / \vec{x} \id{\}}$ to denote partial maps,
$s : \QProc \rightarrow \QProc$. A map, $s$ lifts, uniquely, to a map
on process terms, $\widehat{s} : \Proc \rightarrow \Proc$ by the
following equations.

\begin{mathpar}
  (0) \psubstp{Q}{P} := 0 \\
  (R \juxtap S) \psubstp{Q}{P}
  :=    
  (R)\psubstp{Q}{P} \juxtap (S) \psubstp{Q}{P} \\
  (x?(y).R) \psubstp{Q}{P}    
  :=    
  (x)\substp{Q}{P} (z)\concat( (R \psubstn{z}{y}) \psubstp{Q}{P} ) \\
  (\lift{x}{R}) \psubstp{Q}{P}  
  :=
  \lift{(x)\substp{Q}{P}}{ R \psubstp{Q}{P} } \\
%   (\dropn{x})  \psubstp{Q}{P}       
%   := 
%   \left\{ 
%     \begin{array}{ccc} 
%       \dropn{\quotep{Q}} & & x \nameeq \quotep{P} \\
%       \dropn{x} & & otherwise \\
%     \end{array}
%   \right. 
  (\dropn{x})  \psubstp{Q}{P}       
  := 
  \left\{ 
    \begin{array}{ccc} 
      Q & & x \nameeq \quotep{P} \\
      \dropn{x} & & otherwise \\
    \end{array}
  \right.
\end{mathpar}
 

where

\begin{eqnarray}
  (x)\id{\{} \lpquote Q \rpquote / \lpquote P \rpquote \id{\}}            = 
  \left\{ 
    \begin{array}{ccc}
      \lpquote Q \rpquote & & x \nameeq \lpquote P \rpquote \\
      x & & otherwise \\
    \end{array}
  \right. \nonumber
\end{eqnarray}

and $z$ is chosen distinct from $\quotep{P}$, $\quotep{Q}$, the free
names in $Q$, and all the names in $R$. Our $\alpha$-equivalence will
be built in the standard way from this substitution.

\begin{remark}\label{rem:no_self_referential_names}
  One consequence of these definitions is that $\forall P. \quotep{P}
  \not\in \freenames{P}$.
\end{remark}

\subsection{ Dynamic quote: an example }

Anticipating something of what's to come, consider applying the
substitution, $\widehat{\id{\{}u / z \id{\}}}$, to the following pair
of processes, $\lift{w}{y!(z)}$ and $w[ \lpquote y!(z) \rpquote ]$.

\begin{eqnarray}
	\lift{w}{y!(z)}\widehat{\id{\{}u / z \id{\}}}
		& = &
		\lift{w}{y!(u)} \nonumber\\
	w[ \lpquote y!(z) \rpquote ] \widehat{ \id{\{}u / z \id{\}} }
		& = &
		w[ \lpquote y!(z) \rpquote ] \nonumber
\end{eqnarray}

Because the body of the process between quotes is impervious to
substitution, we get radically different answers. In fact, by
examining the first process in an input context,
e.g. $x?(z).\lift{w}{y!(z)}$, we see that the process under the lift
operator may be shaped by prefixed inputs binding a name inside it. In
this sense, the lift operator will be seen as a way to dynamically
construct processes before reifying them as names.

Finally equipped with these standard features we can present the
dynamics of the calculus.

\subsubsection{Operational semantics} 

Finally, we introduce the computational dynamics. What marks these
algebras as distinct from other more traditionally studied algebraic
structures, e.g. vector spaces or polynomial rings, is the manner in
which dynamics is captured. In traditional structures, dynamics is typically
expressed through morphisms between such structures, as in linear maps
between vector spaces or morphisms between rings. In algebras
associated with the semantics of computation, the dynamics is
expressed as part of the algebraic structure itself, through a
reduction reduction relation typically denoted by $\red$. Below, we
give a recursive presentation of this relation for the calculus used
in the encoding.

$\red \subseteq \pi \times \pi$
$\red : \pi \to \mathcal{P}(\pi)$

\begin{mathpar}
  \inferrule* [lab=Comm] { \textsf{match}( x_{src}, x_{trgt} ) } { x_{trgt}?(y)P \; | \; x_{src}!\langle {Q} \rangle \red P\{\quotep{Q}/y}\} }
  \and \\
  \inferrule* [lab=Par] {{P} \red {P}'} {{{P} | {Q}} \red {{P}' | {Q}}}
  \and
  \inferrule* [lab=Equiv]{{{P} \scong {P}'} \andalso {{P}' \red {Q}'} \andalso {{Q}' \scong {Q}}}{{P} \red {Q}}
\end{mathpar}

\begin{eqnarray*}
  match_{\equiv} (\quotep{P},\quotep{Q}) & := & P \equiv Q \\
  match_{\dagger}(\quotep{P},\quotep{Q}) & := & \forall R. P|Q \red^{*} R => R \red^{*} 0 \\
  match_{K}(\quotep{P},\quotep{Q}) & := & K \mbox{ for some context } K
\end{eqnarray*}

$u?(x)P | u!\langle Q \rangle \red P\{\quotep{Q}/x\}$

%We write $\wred$ for $\red^*$, and $P\red$ if $\exists Q $ such that $ P \red Q$.
We write $P\red$ if $\exists Q $ such that $ P \red Q$ and $P\not\red$, otherwise.

\section{Replication}

As mentioned before, it is known that replication (and hence
recursion) can be implemented in a higher-order process algebra
\cite{SangiorgiWalker}. As our first example of calculation with the
machinery thus far presented we give the construction explicitly in
the {\rhoc}.

\begin{eqnarray}
	D_{x} & := & \prefix{x}{y}{(\binpar{\outputp{x}{y}}{@{y}})} \nonumber\\
	\bangp_{x}{P} & := & \binpar{{x}!\langle{\binpar{D_{x}}{P}}\rangle}{D_{x}} \nonumber
\end{eqnarray}

\begin{eqnarray}
	\bangp_{x}{P} & & \nonumber\\
	=
	& {x}!\langle{(\prefix{x}{y}{(\outputp{x}{y} | @{y})) | P}}\rangle 
	      | \prefix{x}{y}{(\outputp{x}{y} | @{y})} & \nonumber\\
	\red
	& (\outputp{x}{y} | @{y})\substn{\quotep{(\prefix{x}{y}{(@{y} | \outputp{x}{y})) | P}}}{y} & \nonumber\\
	=
	& \outputp{x}{\quotep{(\prefix{x}{y}{(\outputp{x}{y} | @{y})) | P}}}
	  | {(\prefix{x}{y}{(\outputp{x}{y} | @{y})) | P}} & \nonumber\\
	\red
	& \ldots & \nonumber\\
	\red^*
	& P | P | \ldots & \nonumber
\end{eqnarray}

Of course, this encoding, as an implementation, runs away, unfolding
$\bangp{P}$ eagerly. A lazier and more implementable replication
operator, restricted to input-guarded processes, may be obtained as follows.

\begin{eqnarray}
\bangp{\prefix{u}{v}{P}} 
	:= 
	\binpar{\lift{x}{\prefix{u}{v}{(\binpar{D(x)}{P})}}}{D(x)} \nonumber
\end{eqnarray}

\begin{remark}
  Note that the lazier definition still does not deal with summation
  or mixed summation (i.e. sums over input and output). The reader is
  invited to construct definitions of replication that deal with these
  features. 

  Further, the definitions are parameterized in a name, $x$. Can you,
  gentle reader, make a definition that eliminates this parameter and
  guarantees no accidental interaction between the replication
  machinery and the process being replicated -- i.e. no accidental
  sharing of names used by the process to get its work done and the
  name(s) used by the replication to effect copying. This latter
  revision of the definition of replication is crucial to obtaining
  the expected identity $!!P \sim !P$.
\end{remark}

\begin{remark}\label{rem:paradoxical_combinator}
  The reader familiar with the lambda calculus will have noticed the
  similarity between $D$ and the paradoxical combinator.

  [Ed. note: the existence of this seems to suggest we have to be more
  restrictive on the set of processes and names we admit if we are to
  support no-cloning.]
\end{remark}

\subsubsection{Bisimulation}

The computational dynamics gives rise to another kind of equivalence,
the equivalence of computational behavior. As previously mentioned
this is typically captured \emph{via} some form of bisimulation.

% The notion we use in this paper is weak barbed bisimulation
% \cite{milner91polyadicpi}.

The notion we use in this paper is derived from weak barbed
bisimulation \cite{milner91polyadicpi}. 

\begin{definition}
An \emph{observation relation}, $\downarrow_{\mathcal N}$, over a set
of names, $\mathcal N$, is the smallest relation satisfying the rules
below.

\infrule[Out-barb]{y \in {\mathcal N}, \; x \nameeq y}
		  {\outputp{x}{v} \downarrow_{\mathcal N} x}
\infrule[Par-barb]{\mbox{$P\downarrow_{\mathcal N} x$ or $Q\downarrow_{\mathcal N} x$}}
		  {\binpar{P}{Q} \downarrow_{\mathcal N} x}

We write $P \Downarrow_{\mathcal N} x$ if there is $Q$ such that 
$P \wred Q$ and $Q \downarrow_{\mathcal N} x$.
\end{definition}

\begin{definition}
%\label{def.bbisim}
An  ${\mathcal N}$-\emph{barbed bisimulation} over a set of names, ${\mathcal N}$, is a symmetric binary relation 
${\mathcal S}_{\mathcal N}$ between agents such that $P\rel{S}_{\mathcal N}Q$ implies:
\begin{enumerate}
\item If $P \red P'$ then $Q \wred Q'$ and $P'\rel{S}_{\mathcal N} Q'$.
\item If $P\downarrow_{\mathcal N} x$, then $Q\Downarrow_{\mathcal N} x$.
\end{enumerate}
$P$ is ${\mathcal N}$-barbed bisimilar to $Q$, written
$P \wbbisim_{\mathcal N} Q$, if $P \rel{S}_{\mathcal N} Q$ for some ${\mathcal N}$-barbed bisimulation ${\mathcal S}_{\mathcal N}$.
\end{definition}

$\mathcal{R} \subseteq \pi \times \pi$

$P \mathcal{R} Q => \forall P'. P \red P' \Rightarrow \exists Q'. Q \red Q', P' \mathcal{R} Q'$

$P \vdash x \Rightarrow Q \vdash x$

\begin{mathpar}
  \inferrule*[lab=Out-barb]{x \nameeq y}{{y}!\langle{Q}\rangle \vdash x}
  \and
  \inferrule*[lab=Par-barb]{\mbox{$P\vdash x$ or $Q\vdash x$}}{\binpar{P}{Q} \vdash x}
\end{mathpar}

\subsubsection{Contexts}

One of the principle advantages of computational calculi like the
$\pi$-calculus is a well-defined notion of context,
contextual-equivalence and a correlation between
contextual-equivalence and notions of bisimulation. The notion of
context allows the decomposition of a process into (sub-)process and
its syntactic environment, its context. Thus, a context may be
thought of as a process with a ``hole'' (written $\Box$) in it. The
application of a context $M$ to a process $P$, written $M[P]$, is
tantamount to filling the hole in $M$ with $P$. In this paper we do
not need the full weight of this theory, but do make use of the notion
of context in the proof the main theorem. 

\begin{mathpar}
  \inferrule* [lab=summation] {} {{M_{M},M_{N}} \bc \Box \;|\; x.M_{A} \;|\; M_{M}+M_{N}}
  \and
  \inferrule* [lab=agent] {} {{M_{A}} \bc (\vec{x})M_{P} \;| \; \clift{P_0,\ldots,M_{P},\ldots,P_N}}
  \and \\
  \inferrule* [lab=process] {} {{M_{P}} \bc M_{N} \;| \;P|M_{P} }
\end{mathpar} 

\begin{mathpar}
  \inferrule* [lab=sychronization] {} {M_{N} \bc \Box \;|\; x?M_{F} \;|\; x!M_{C}}
  \and
  \inferrule* [lab=abstraction] {} {{M_{F}} \bc (x)M_{P} }
  \and
  \inferrule* [lab=concretion] {} {{M_{C}} \bc \langle M_{P} \rangle }
  \and \\
  \inferrule* [lab=process] {} {{M_{P}} \bc M_{N} \;| \;P|M_{P} }
\end{mathpar}

\begin{definition}[contextual application] Given a context $M$, and
  process $P$, we define the \emph{contextual application}, $M[P] :=
  M\{P/\Box\}$. That is, the contextual application of M to P is the
  substitution of $P$ for $\Box$ in $M$.
\end{definition}

$\meaningof{-} : L \to \mathcal{P}(\pi)$

\begin{mathpar}
  \inferrule* [lab=collection] {} {\meaningof{true} = \pi, \and \meaningof{~E} = \pi \setminus \meaningof{E}, \and \meaningof{E_{1} \& E_{2}} = \meaningof{E_{1}} \cap \meaningof{E_{2}}}
\end{mathpar}

\begin{mathpar}
  \inferrule* [lab=structure] {} {\meaningof{0} = \{ P \in \pi | P \equiv 0 \}, \and \\ \meaningof{E_1 | E_2} = \{ P \in \pi | P \equiv P_{1} | P_{2}, P_{1} \in \meaningof{E_{1}}, P_{2} \in \meaningof{E_2}\} }
\end{mathpar}

\begin{mathpar}
 \inferrule* [lab=behavior] {} {\meaningof{\langle a?b \rangle E} = \{ P \in \pi | P \equiv Q | u?(y)P', \\ \and \\\\ \and \\ \;\;\; u \in \meaningof{a}, \forall z.P'\{z/y\} \in \meaningof{E\{z/b\}}\}, \and \\ \meaningof{a!E} = \{ P \in \pi | P \equiv Q | x!\langle P' \rangle, x \in \meaningof{a} P' \in \meaningof{E}\} }
\end{mathpar}

\begin{mathpar}
 \inferrule* [lab=nominal] {} {\meaningof{\quotep{E}} = \{ \quotep{P} \in \quotep{\pi} | P \in \meaningof{E} \}, \and \meaningof{\quotep{P}} = \{ \quotep{Q} \in \quotep{\pi} | P \equiv Q \} \and \\ \meaningof{@\quotep{E}} = \{ P \in \pi | P \equiv @x, x \in \meaningof{E} \}}
\end{mathpar}

\begin{eqnarray*}
  \\
  \meaningof{-} : TS \to ST
\end{eqnarray*}

\begin{eqnarray*}
  \\
  L : TS \to ST
\end{eqnarray*}

\begin{eqnarray*}
  \\
  P \models E \iff P \in \meaningof{E}
\end{eqnarray*}

\begin{eqnarray*}
  P \approx_{L} Q \iff \forall E \in L. P \models E \iff Q \models E
\end{eqnarray*}

\begin{eqnarray*}
  P \approx_{K} Q
\end{eqnarray*}

\begin{eqnarray*}
  P \approx Q
\end{eqnarray*}

$\approx_{K} = \approx = \approx_{L}$

\subsubsection{Contextual duality}

Note that contexts extend the quotation operation to a family of
operations from processes to names. Given a context, $M$, we can
define a \emph{nominal context}, $\quotep{M}$ by $\quotep{M}[P] :=
\quotep{M[P]}$. To foreshadow what is to come we observe that these
operations enjoy a duality with processes very much like the duality
between vectors and maps from vectors to scalars.

Further, because the calculus is essentially higher-order, we have a
correspondence between contexts and processes. More specifically,
given a name $x$ and a context $M$ we can construct $M^{*}_{x}$ such
that 

\begin{mathpar}
  M^{*}_{x} | \lift{x}{P} \red M[P]
\end{mathpar}

namely,

\begin{mathpar}
  M^{*}_{x} := x?(u).M[\dropn{u}]
\end{mathpar}

The dependence of $M^{*}_{x}$ on a name makes it an abstraction, 

\begin{mathpar}
  M^{*} := (x)x?(u).M[\dropn{u}]
\end{mathpar}

\subsection{Additional notation}

It will sometimes be convenient to denote the process a name
quotes. We already have the notation $x = \quotep{P}$, but it will be
convenient to introduce an alternate notation, $\procn{x}$, when we
want to emphasize the connection to the use of the name. Note that, by
virtue of name equivalence, $\quotep{\procn{x}} \nameeq x$; so, the
notation is consistent with previous definitions.

Further, because names have structure it is possible to effect
substitutions on the basis of that structure. This means we need to
upgrade our notation for substitutions, which we accomplish by
adapting comprehension notation. Thus,

\begin{mathpar}
  P\{ y / x : x \in S \}
\end{mathpar}

is interpreted to mean the process derived from P by replacing (in a
capture-avoiding manner) each occurrence of $x$ in $S$ by $y$. For example,

\begin{mathpar}
  P\{ \quotep{\procn{x}|\procn{x}} / x : x \in \freenames{P} \}
\end{mathpar}

will replace each (occurrence) of a free name $x$ in $P$ by
$\quotep{\procn{x}|\procn{x}}$.

Also, we will avail ourselves of the notation $x^{L}$ and $x^{R}$ to
denote injections of a name into disjoint copies of the name
space. There are numerous ways to accomplish this. One example can be
found in \cite{MeredithR05}. This notation overloads to vectors of
names: $\vec{x}^{\pi} := (x_{i}^{\pi} \; : \; 0 \leq i < |\vec{x}| )$ where $\pi \in \{L,R\}$.

We also use $P^{\Box} := P|\Box$.

In \cite{MeredithR05} an interpretation of the new operator is
given. It turns out that there are several possible interpretations
all enjoying the requisite algebraic properties of the operator (see
\cite{milner91polyadicpi}). We will therefore make liberal use of
$(\nu\; \vec{x})P$.

% subsection the_syntax_and_semantics_of_the_notation_system (end)   

\input{qm2pi.qmops} 

\input{qm2pi.sterngerlach} 

\input{qm2pi.metric} 

% section concurrent_process_calculi (end)

%\input{qm2pi.proofsketch}

% section proof sketch (end)

%\input{qm2pi.slviaknots} 

% section spatial logic via knots (end)

\input{qm2pi.conclusion}

% section conclusion (end)

%\input{qm2pi.dtcodes} 

% section wiring algorithm (end)

\input{qm2pi.ack} 

% section acknowledgments (end)

\newpage


\bibliographystyle{plain}   
\bibliography{../../biblios/main.bib}

\input{qm2pi.rhodetails}

\end{document}

 

%\documentclass[12pt]{llncs}
%\documentclass{jktr}

\usepackage[pdftex]{hyperref}                   
\usepackage {listings}
\usepackage {mathpartir}
\usepackage{bcprules}
%\usepackage{listings}
                       
\usepackage{graphicx} 
%\usepackage[margins=2.5cm,nohead,nofoot]{geometry}
%\usepackage{geometry}
\usepackage{amsfonts}
\usepackage{amstext}
\usepackage{latexsym}
\usepackage{amssymb}
\usepackage{color}


%\include{myPreamble}
\include{qm2pi.local} 

%\ifpdf
%\usepackage[pdftex]{graphicx}
%\else
%\usepackage{graphicx}
%\fi

 % \ifpdf
%  \usepackage{pdfsync}
%  \if


%\title{Brief Article}
%\author{David F. Snyder}
%\author{L.G. Meredith}

%\address{Dept. of Math., Texas State University--San Marcos, San Marcos, TX 78666}
       
\pagestyle{empty}


\begin{document}

\lstset{language=[Objective]Caml,frame=shadowbox}

\input{qm2pi.front}

% section front matter (end)

\input{qm2pi.intro} 
 
% section introduction (end)

% \input{qm2pi.knotations} 

% section notation (end)

\input{qm2pi.process.calculi} 

% section concurrent_process_calculi_and_spatial_logics_ (end)
    
%\input{qm2pi.knots2pi} 

%\input{qm2pi.trefoil} 

%\input{qm2pi.mainthm} 

% subsection basic_interpretation (end)

%\input{qm2pi.rho.presentation} 
\subsection{The syntax and semantics of the notation system}\label{sub:the_syntax_and_semantics_of_the_notation_system} % (fold)

We now summarize a technical presentation of the calculus that
embodies our theory of dynamics. The typical presentation of such a
calculus follows the style of giving generators and relations on
them. The grammar, below, describing term constructors, freely
generates the set of processes, $\Proc$. This set is then quotiented
by a relation known as structural congruence and it is over this set
that the notion of dynamics is expressed. This presentation is
essentially that of \cite{MeredithR05} with the addition of
polyadicity and summation. For readability we have relegated some of
the technical subtleties to an appendix.

\subsubsection{Process grammar}\label{subsub:process_grammar}

\begin{mathpar}
  \inferrule* [lab=synchronization] {} {{M} \bc \pzero \;|\; x?F \;|\; x!C }
  \and
  \inferrule* [lab=abstraction] {} {{F} \bc (x)P}
  \and
  \inferrule* [lab=concretion] {} {{C} \bc \langle Q \rangle}
  \and
  \inferrule* [lab=process] {} {{P,Q} \bc M \;| \;P|Q \;|\; @{x}}
  \and
  \inferrule* [lab=name] {} {{x} \bc \quotep{P}}
\end{mathpar} 

Note that $\vec{x}$ (resp. $\vec{P}$) denotes a vector of names
(resp. processes) of length $|\vec{x}|$ (resp. $|\vec{P}|$). We adopt
the following useful abbreviations.

\begin{mathpar}
   x?(\vec{y}).P := x.(\vec{y})P \and  x\clift{\vec{P}} := x.\clift{\vec{P}}
   \and x!(y) := \lift{x}{\dropn{y}}
   \and \Pi_{i=0}^{n-1}P_i := P_0 | \ldots | P_{n-1}
\end{mathpar}

\subsubsection{Structural congruence}

\paragraph{Free and bound names and alpha-equivalence.} At the
core of structural equivalence is alpha-equivalence which identifies
process that are the same up to a change of variable. Formally, we
recognize the distinction between free and bound names. The free names
of a process, $\freenames{P}$, may be calculated recursively as
follows:

\begin{mathpar}
\freenames{\pzero} := \emptyset
  \and \\
  \freenames{x?(y).P} := \{ x \} \cup (\freenames{P} \setminus \{ y \})
  \and 
  \freenames{x!\langle P \rangle} := \{ x \} \cup \{ P \} 
  \and \\
  \freenames{P|Q} := \freenames{P} \cup \freenames{Q}
  \and \\
  \freenames{@{x}} := \{ x \}
\end{mathpar}

$\pi$
$\quotep{\pi}$

$\freenames{-} : \pi \to \mathcal{P}(\quotep{\pi})$

\begin{eqnarray*}
  \freenames{\pzero} & := & \emptyset \\
  \freenames{x?(y).P} & := & \{ x \} \cup (\freenames{P} \setminus \{ y \}) \\
  \freenames{x!\langle P \rangle} & := & \{ x \} \cup \{ P \} \\
  \freenames{P|Q} & := & \freenames{P} \cup \freenames{Q} \\
  \freenames{\dropn{x}} & := & \{ x \}
\end{eqnarray*}

The bound names of a process, $\boundnames{P}$, are those names occurring in $P$
that are not free. For example, in $x?(y).0$, the name $x$ is free, while $y$ is bound.

\begin{mathpar}
  \inferrule* [lab=monoidal-laws] {} { P|Q \equiv Q|P \and P|0 \equiv P \and P|(Q|R) \equiv (P|Q)|R }
\end{mathpar}

\begin{mathpar}
  \inferrule* [lab=alpha-equivalence] {} { (x)P \equiv (y)P\{y/x\} \and y \not\in \freenames{P} }
\end{mathpar}

\begin{definition}
Then two processes, $P,Q$, are alpha-equivalent if $P = Q\{\vec{y}/\vec{x}\}$ for
some $\vec{x} \in \boundnames{Q},\vec{y} \in \boundnames{P}$, where $Q\{\vec{y}/\vec{x}\}$
denotes the capture-avoiding substitution of $\vec{y}$ for $\vec{x}$ in $Q$.
\end{definition}

\begin{definition}
  The {\em structural congruence} \cite{SangiorgiWalker} , $\equiv$,
  between processes is the least congruence containing
  alpha-equivalence, satisfying the abelian monoid laws
  (associativity, commutativity and $\pzero$ as identity) for parallel
  composition $|$ and for summation $+$.
\end{definition}

\subsection{Name equivalence}

We take name equivalence, written $\nameeq$, to be the smallest
equivalence relation generated by the following rules.

\begin{mathpar}
\inferrule*[lab=Quote-drop]
{ }
{ \quotep{@{x}} \nameeq x }

\inferrule*[lab=Struct-equiv]
{ P \scong Q }
{ \quotep{P} \nameeq \quotep{Q} }
\end{mathpar}

The astute reader will have noticed that the mutual recursion of names
and processes imposes a mutual recursion on alpha-equivalence and
structural equivalence via name-equivalence. Fortunately, all of this
works out pleasantly and we may calculate in the natural way, free of
concern. The reader interested in the details is referred to the
appendix \ref{appendix:rho_details}.

\subsection{Substitution}

We use $\Proc$ for the set of processes, $\QProc$ for the set of
names, and $\id{\{}\vec{y} / \vec{x} \id{\}}$ to denote partial maps,
$s : \QProc \rightarrow \QProc$. A map, $s$ lifts, uniquely, to a map
on process terms, $\widehat{s} : \Proc \rightarrow \Proc$ by the
following equations.

\begin{mathpar}
  (0) \psubstp{Q}{P} := 0 \\
  (R \juxtap S) \psubstp{Q}{P}
  :=    
  (R)\psubstp{Q}{P} \juxtap (S) \psubstp{Q}{P} \\
  (x?(y).R) \psubstp{Q}{P}    
  :=    
  (x)\substp{Q}{P} (z)\concat( (R \psubstn{z}{y}) \psubstp{Q}{P} ) \\
  (\lift{x}{R}) \psubstp{Q}{P}  
  :=
  \lift{(x)\substp{Q}{P}}{ R \psubstp{Q}{P} } \\
%   (\dropn{x})  \psubstp{Q}{P}       
%   := 
%   \left\{ 
%     \begin{array}{ccc} 
%       \dropn{\quotep{Q}} & & x \nameeq \quotep{P} \\
%       \dropn{x} & & otherwise \\
%     \end{array}
%   \right. 
  (\dropn{x})  \psubstp{Q}{P}       
  := 
  \left\{ 
    \begin{array}{ccc} 
      Q & & x \nameeq \quotep{P} \\
      \dropn{x} & & otherwise \\
    \end{array}
  \right.
\end{mathpar}
 

where

\begin{eqnarray}
  (x)\id{\{} \lpquote Q \rpquote / \lpquote P \rpquote \id{\}}            = 
  \left\{ 
    \begin{array}{ccc}
      \lpquote Q \rpquote & & x \nameeq \lpquote P \rpquote \\
      x & & otherwise \\
    \end{array}
  \right. \nonumber
\end{eqnarray}

and $z$ is chosen distinct from $\quotep{P}$, $\quotep{Q}$, the free
names in $Q$, and all the names in $R$. Our $\alpha$-equivalence will
be built in the standard way from this substitution.

\begin{remark}\label{rem:no_self_referential_names}
  One consequence of these definitions is that $\forall P. \quotep{P}
  \not\in \freenames{P}$.
\end{remark}

\subsection{ Dynamic quote: an example }

Anticipating something of what's to come, consider applying the
substitution, $\widehat{\id{\{}u / z \id{\}}}$, to the following pair
of processes, $\lift{w}{y!(z)}$ and $w[ \lpquote y!(z) \rpquote ]$.

\begin{eqnarray}
	\lift{w}{y!(z)}\widehat{\id{\{}u / z \id{\}}}
		& = &
		\lift{w}{y!(u)} \nonumber\\
	w[ \lpquote y!(z) \rpquote ] \widehat{ \id{\{}u / z \id{\}} }
		& = &
		w[ \lpquote y!(z) \rpquote ] \nonumber
\end{eqnarray}

Because the body of the process between quotes is impervious to
substitution, we get radically different answers. In fact, by
examining the first process in an input context,
e.g. $x?(z).\lift{w}{y!(z)}$, we see that the process under the lift
operator may be shaped by prefixed inputs binding a name inside it. In
this sense, the lift operator will be seen as a way to dynamically
construct processes before reifying them as names.

Finally equipped with these standard features we can present the
dynamics of the calculus.

\subsubsection{Operational semantics} 

Finally, we introduce the computational dynamics. What marks these
algebras as distinct from other more traditionally studied algebraic
structures, e.g. vector spaces or polynomial rings, is the manner in
which dynamics is captured. In traditional structures, dynamics is typically
expressed through morphisms between such structures, as in linear maps
between vector spaces or morphisms between rings. In algebras
associated with the semantics of computation, the dynamics is
expressed as part of the algebraic structure itself, through a
reduction reduction relation typically denoted by $\red$. Below, we
give a recursive presentation of this relation for the calculus used
in the encoding.

$\red \subseteq \pi \times \pi$
$\red : \pi \to \mathcal{P}(\pi)$

\begin{mathpar}
  \inferrule* [lab=Comm] { \textsf{match}( x_{src}, x_{trgt} ) } { x_{trgt}?(y)P \; | \; x_{src}!\langle {Q} \rangle \red P\{\quotep{Q}/y}\} }
  \and \\
  \inferrule* [lab=Par] {{P} \red {P}'} {{{P} | {Q}} \red {{P}' | {Q}}}
  \and
  \inferrule* [lab=Equiv]{{{P} \scong {P}'} \andalso {{P}' \red {Q}'} \andalso {{Q}' \scong {Q}}}{{P} \red {Q}}
\end{mathpar}

\begin{eqnarray*}
  match_{\equiv} (\quotep{P},\quotep{Q}) & := & P \equiv Q \\
  match_{\dagger}(\quotep{P},\quotep{Q}) & := & \forall R. P|Q \red^{*} R => R \red^{*} 0 \\
  match_{K}(\quotep{P},\quotep{Q}) & := & K \mbox{ for some context } K
\end{eqnarray*}

$u?(x)P | u!\langle Q \rangle \red P\{\quotep{Q}/x\}$

%We write $\wred$ for $\red^*$, and $P\red$ if $\exists Q $ such that $ P \red Q$.
We write $P\red$ if $\exists Q $ such that $ P \red Q$ and $P\not\red$, otherwise.

\section{Replication}

As mentioned before, it is known that replication (and hence
recursion) can be implemented in a higher-order process algebra
\cite{SangiorgiWalker}. As our first example of calculation with the
machinery thus far presented we give the construction explicitly in
the {\rhoc}.

\begin{eqnarray}
	D_{x} & := & \prefix{x}{y}{(\binpar{\outputp{x}{y}}{@{y}})} \nonumber\\
	\bangp_{x}{P} & := & \binpar{{x}!\langle{\binpar{D_{x}}{P}}\rangle}{D_{x}} \nonumber
\end{eqnarray}

\begin{eqnarray}
	\bangp_{x}{P} & & \nonumber\\
	=
	& {x}!\langle{(\prefix{x}{y}{(\outputp{x}{y} | @{y})) | P}}\rangle 
	      | \prefix{x}{y}{(\outputp{x}{y} | @{y})} & \nonumber\\
	\red
	& (\outputp{x}{y} | @{y})\substn{\quotep{(\prefix{x}{y}{(@{y} | \outputp{x}{y})) | P}}}{y} & \nonumber\\
	=
	& \outputp{x}{\quotep{(\prefix{x}{y}{(\outputp{x}{y} | @{y})) | P}}}
	  | {(\prefix{x}{y}{(\outputp{x}{y} | @{y})) | P}} & \nonumber\\
	\red
	& \ldots & \nonumber\\
	\red^*
	& P | P | \ldots & \nonumber
\end{eqnarray}

Of course, this encoding, as an implementation, runs away, unfolding
$\bangp{P}$ eagerly. A lazier and more implementable replication
operator, restricted to input-guarded processes, may be obtained as follows.

\begin{eqnarray}
\bangp{\prefix{u}{v}{P}} 
	:= 
	\binpar{\lift{x}{\prefix{u}{v}{(\binpar{D(x)}{P})}}}{D(x)} \nonumber
\end{eqnarray}

\begin{remark}
  Note that the lazier definition still does not deal with summation
  or mixed summation (i.e. sums over input and output). The reader is
  invited to construct definitions of replication that deal with these
  features. 

  Further, the definitions are parameterized in a name, $x$. Can you,
  gentle reader, make a definition that eliminates this parameter and
  guarantees no accidental interaction between the replication
  machinery and the process being replicated -- i.e. no accidental
  sharing of names used by the process to get its work done and the
  name(s) used by the replication to effect copying. This latter
  revision of the definition of replication is crucial to obtaining
  the expected identity $!!P \sim !P$.
\end{remark}

\begin{remark}\label{rem:paradoxical_combinator}
  The reader familiar with the lambda calculus will have noticed the
  similarity between $D$ and the paradoxical combinator.

  [Ed. note: the existence of this seems to suggest we have to be more
  restrictive on the set of processes and names we admit if we are to
  support no-cloning.]
\end{remark}

\subsubsection{Bisimulation}

The computational dynamics gives rise to another kind of equivalence,
the equivalence of computational behavior. As previously mentioned
this is typically captured \emph{via} some form of bisimulation.

% The notion we use in this paper is weak barbed bisimulation
% \cite{milner91polyadicpi}.

The notion we use in this paper is derived from weak barbed
bisimulation \cite{milner91polyadicpi}. 

\begin{definition}
An \emph{observation relation}, $\downarrow_{\mathcal N}$, over a set
of names, $\mathcal N$, is the smallest relation satisfying the rules
below.

\infrule[Out-barb]{y \in {\mathcal N}, \; x \nameeq y}
		  {\outputp{x}{v} \downarrow_{\mathcal N} x}
\infrule[Par-barb]{\mbox{$P\downarrow_{\mathcal N} x$ or $Q\downarrow_{\mathcal N} x$}}
		  {\binpar{P}{Q} \downarrow_{\mathcal N} x}

We write $P \Downarrow_{\mathcal N} x$ if there is $Q$ such that 
$P \wred Q$ and $Q \downarrow_{\mathcal N} x$.
\end{definition}

\begin{definition}
%\label{def.bbisim}
An  ${\mathcal N}$-\emph{barbed bisimulation} over a set of names, ${\mathcal N}$, is a symmetric binary relation 
${\mathcal S}_{\mathcal N}$ between agents such that $P\rel{S}_{\mathcal N}Q$ implies:
\begin{enumerate}
\item If $P \red P'$ then $Q \wred Q'$ and $P'\rel{S}_{\mathcal N} Q'$.
\item If $P\downarrow_{\mathcal N} x$, then $Q\Downarrow_{\mathcal N} x$.
\end{enumerate}
$P$ is ${\mathcal N}$-barbed bisimilar to $Q$, written
$P \wbbisim_{\mathcal N} Q$, if $P \rel{S}_{\mathcal N} Q$ for some ${\mathcal N}$-barbed bisimulation ${\mathcal S}_{\mathcal N}$.
\end{definition}

$\mathcal{R} \subseteq \pi \times \pi$

$P \mathcal{R} Q => \forall P'. P \red P' \Rightarrow \exists Q'. Q \red Q', P' \mathcal{R} Q'$

$P \vdash x \Rightarrow Q \vdash x$

\begin{mathpar}
  \inferrule*[lab=Out-barb]{x \nameeq y}{{y}!\langle{Q}\rangle \vdash x}
  \and
  \inferrule*[lab=Par-barb]{\mbox{$P\vdash x$ or $Q\vdash x$}}{\binpar{P}{Q} \vdash x}
\end{mathpar}

\subsubsection{Contexts}

One of the principle advantages of computational calculi like the
$\pi$-calculus is a well-defined notion of context,
contextual-equivalence and a correlation between
contextual-equivalence and notions of bisimulation. The notion of
context allows the decomposition of a process into (sub-)process and
its syntactic environment, its context. Thus, a context may be
thought of as a process with a ``hole'' (written $\Box$) in it. The
application of a context $M$ to a process $P$, written $M[P]$, is
tantamount to filling the hole in $M$ with $P$. In this paper we do
not need the full weight of this theory, but do make use of the notion
of context in the proof the main theorem. 

\begin{mathpar}
  \inferrule* [lab=summation] {} {{M_{M},M_{N}} \bc \Box \;|\; x.M_{A} \;|\; M_{M}+M_{N}}
  \and
  \inferrule* [lab=agent] {} {{M_{A}} \bc (\vec{x})M_{P} \;| \; \clift{P_0,\ldots,M_{P},\ldots,P_N}}
  \and \\
  \inferrule* [lab=process] {} {{M_{P}} \bc M_{N} \;| \;P|M_{P} }
\end{mathpar} 

\begin{mathpar}
  \inferrule* [lab=sychronization] {} {M_{N} \bc \Box \;|\; x?M_{F} \;|\; x!M_{C}}
  \and
  \inferrule* [lab=abstraction] {} {{M_{F}} \bc (x)M_{P} }
  \and
  \inferrule* [lab=concretion] {} {{M_{C}} \bc \langle M_{P} \rangle }
  \and \\
  \inferrule* [lab=process] {} {{M_{P}} \bc M_{N} \;| \;P|M_{P} }
\end{mathpar}

\begin{definition}[contextual application] Given a context $M$, and
  process $P$, we define the \emph{contextual application}, $M[P] :=
  M\{P/\Box\}$. That is, the contextual application of M to P is the
  substitution of $P$ for $\Box$ in $M$.
\end{definition}

$\meaningof{-} : L \to \mathcal{P}(\pi)$

\begin{mathpar}
  \inferrule* [lab=collection] {} {\meaningof{true} = \pi, \and \meaningof{~E} = \pi \setminus \meaningof{E}, \and \meaningof{E_{1} \& E_{2}} = \meaningof{E_{1}} \cap \meaningof{E_{2}}}
\end{mathpar}

\begin{mathpar}
  \inferrule* [lab=structure] {} {\meaningof{0} = \{ P \in \pi | P \equiv 0 \}, \and \\ \meaningof{E_1 | E_2} = \{ P \in \pi | P \equiv P_{1} | P_{2}, P_{1} \in \meaningof{E_{1}}, P_{2} \in \meaningof{E_2}\} }
\end{mathpar}

\begin{mathpar}
 \inferrule* [lab=behavior] {} {\meaningof{\langle a?b \rangle E} = \{ P \in \pi | P \equiv Q | u?(y)P', \\ \and \\\\ \and \\ \;\;\; u \in \meaningof{a}, \forall z.P'\{z/y\} \in \meaningof{E\{z/b\}}\}, \and \\ \meaningof{a!E} = \{ P \in \pi | P \equiv Q | x!\langle P' \rangle, x \in \meaningof{a} P' \in \meaningof{E}\} }
\end{mathpar}

\begin{mathpar}
 \inferrule* [lab=nominal] {} {\meaningof{\quotep{E}} = \{ \quotep{P} \in \quotep{\pi} | P \in \meaningof{E} \}, \and \meaningof{\quotep{P}} = \{ \quotep{Q} \in \quotep{\pi} | P \equiv Q \} \and \\ \meaningof{@\quotep{E}} = \{ P \in \pi | P \equiv @x, x \in \meaningof{E} \}}
\end{mathpar}

\begin{eqnarray*}
  \\
  \meaningof{-} : TS \to ST
\end{eqnarray*}

\begin{eqnarray*}
  \\
  L : TS \to ST
\end{eqnarray*}

\begin{eqnarray*}
  \\
  P \models E \iff P \in \meaningof{E}
\end{eqnarray*}

\begin{eqnarray*}
  P \approx_{L} Q \iff \forall E \in L. P \models E \iff Q \models E
\end{eqnarray*}

\begin{eqnarray*}
  P \approx_{K} Q
\end{eqnarray*}

\begin{eqnarray*}
  P \approx Q
\end{eqnarray*}

$\approx_{K} = \approx = \approx_{L}$

\subsubsection{Contextual duality}

Note that contexts extend the quotation operation to a family of
operations from processes to names. Given a context, $M$, we can
define a \emph{nominal context}, $\quotep{M}$ by $\quotep{M}[P] :=
\quotep{M[P]}$. To foreshadow what is to come we observe that these
operations enjoy a duality with processes very much like the duality
between vectors and maps from vectors to scalars.

Further, because the calculus is essentially higher-order, we have a
correspondence between contexts and processes. More specifically,
given a name $x$ and a context $M$ we can construct $M^{*}_{x}$ such
that 

\begin{mathpar}
  M^{*}_{x} | \lift{x}{P} \red M[P]
\end{mathpar}

namely,

\begin{mathpar}
  M^{*}_{x} := x?(u).M[\dropn{u}]
\end{mathpar}

The dependence of $M^{*}_{x}$ on a name makes it an abstraction, 

\begin{mathpar}
  M^{*} := (x)x?(u).M[\dropn{u}]
\end{mathpar}

\subsection{Additional notation}

It will sometimes be convenient to denote the process a name
quotes. We already have the notation $x = \quotep{P}$, but it will be
convenient to introduce an alternate notation, $\procn{x}$, when we
want to emphasize the connection to the use of the name. Note that, by
virtue of name equivalence, $\quotep{\procn{x}} \nameeq x$; so, the
notation is consistent with previous definitions.

Further, because names have structure it is possible to effect
substitutions on the basis of that structure. This means we need to
upgrade our notation for substitutions, which we accomplish by
adapting comprehension notation. Thus,

\begin{mathpar}
  P\{ y / x : x \in S \}
\end{mathpar}

is interpreted to mean the process derived from P by replacing (in a
capture-avoiding manner) each occurrence of $x$ in $S$ by $y$. For example,

\begin{mathpar}
  P\{ \quotep{\procn{x}|\procn{x}} / x : x \in \freenames{P} \}
\end{mathpar}

will replace each (occurrence) of a free name $x$ in $P$ by
$\quotep{\procn{x}|\procn{x}}$.

Also, we will avail ourselves of the notation $x^{L}$ and $x^{R}$ to
denote injections of a name into disjoint copies of the name
space. There are numerous ways to accomplish this. One example can be
found in \cite{MeredithR05}. This notation overloads to vectors of
names: $\vec{x}^{\pi} := (x_{i}^{\pi} \; : \; 0 \leq i < |\vec{x}| )$ where $\pi \in \{L,R\}$.

We also use $P^{\Box} := P|\Box$.

In \cite{MeredithR05} an interpretation of the new operator is
given. It turns out that there are several possible interpretations
all enjoying the requisite algebraic properties of the operator (see
\cite{milner91polyadicpi}). We will therefore make liberal use of
$(\nu\; \vec{x})P$.

% subsection the_syntax_and_semantics_of_the_notation_system (end)   

\input{qm2pi.qmops} 

\input{qm2pi.sterngerlach} 

\input{qm2pi.metric} 

% section concurrent_process_calculi (end)

%\input{qm2pi.proofsketch}

% section proof sketch (end)

%\input{qm2pi.slviaknots} 

% section spatial logic via knots (end)

\input{qm2pi.conclusion}

% section conclusion (end)

%\input{qm2pi.dtcodes} 

% section wiring algorithm (end)

\input{qm2pi.ack} 

% section acknowledgments (end)

\newpage


\bibliographystyle{plain}   
\bibliography{../../biblios/main.bib}

\input{qm2pi.rhodetails}

\end{document}

 

% subsection basic_interpretation (end)

%\input{qm2pi.rho.presentation} 
\subsection{The syntax and semantics of the notation system}\label{sub:the_syntax_and_semantics_of_the_notation_system} % (fold)

We now summarize a technical presentation of the calculus that
embodies our theory of dynamics. The typical presentation of such a
calculus follows the style of giving generators and relations on
them. The grammar, below, describing term constructors, freely
generates the set of processes, $\Proc$. This set is then quotiented
by a relation known as structural congruence and it is over this set
that the notion of dynamics is expressed. This presentation is
essentially that of \cite{MeredithR05} with the addition of
polyadicity and summation. For readability we have relegated some of
the technical subtleties to an appendix.

\subsubsection{Process grammar}\label{subsub:process_grammar}

\begin{mathpar}
  \inferrule* [lab=synchronization] {} {{M} \bc \pzero \;|\; x?F \;|\; x!C }
  \and
  \inferrule* [lab=abstraction] {} {{F} \bc (x)P}
  \and
  \inferrule* [lab=concretion] {} {{C} \bc \langle Q \rangle}
  \and
  \inferrule* [lab=process] {} {{P,Q} \bc M \;| \;P|Q \;|\; @{x}}
  \and
  \inferrule* [lab=name] {} {{x} \bc \quotep{P}}
\end{mathpar} 

Note that $\vec{x}$ (resp. $\vec{P}$) denotes a vector of names
(resp. processes) of length $|\vec{x}|$ (resp. $|\vec{P}|$). We adopt
the following useful abbreviations.

\begin{mathpar}
   x?(\vec{y}).P := x.(\vec{y})P \and  x\clift{\vec{P}} := x.\clift{\vec{P}}
   \and x!(y) := \lift{x}{\dropn{y}}
   \and \Pi_{i=0}^{n-1}P_i := P_0 | \ldots | P_{n-1}
\end{mathpar}

\subsubsection{Structural congruence}

\paragraph{Free and bound names and alpha-equivalence.} At the
core of structural equivalence is alpha-equivalence which identifies
process that are the same up to a change of variable. Formally, we
recognize the distinction between free and bound names. The free names
of a process, $\freenames{P}$, may be calculated recursively as
follows:

\begin{mathpar}
\freenames{\pzero} := \emptyset
  \and \\
  \freenames{x?(y).P} := \{ x \} \cup (\freenames{P} \setminus \{ y \})
  \and 
  \freenames{x!\langle P \rangle} := \{ x \} \cup \{ P \} 
  \and \\
  \freenames{P|Q} := \freenames{P} \cup \freenames{Q}
  \and \\
  \freenames{@{x}} := \{ x \}
\end{mathpar}

$\pi$
$\quotep{\pi}$

$\freenames{-} : \pi \to \mathcal{P}(\quotep{\pi})$

\begin{eqnarray*}
  \freenames{\pzero} & := & \emptyset \\
  \freenames{x?(y).P} & := & \{ x \} \cup (\freenames{P} \setminus \{ y \}) \\
  \freenames{x!\langle P \rangle} & := & \{ x \} \cup \{ P \} \\
  \freenames{P|Q} & := & \freenames{P} \cup \freenames{Q} \\
  \freenames{\dropn{x}} & := & \{ x \}
\end{eqnarray*}

The bound names of a process, $\boundnames{P}$, are those names occurring in $P$
that are not free. For example, in $x?(y).0$, the name $x$ is free, while $y$ is bound.

\begin{mathpar}
  \inferrule* [lab=monoidal-laws] {} { P|Q \equiv Q|P \and P|0 \equiv P \and P|(Q|R) \equiv (P|Q)|R }
\end{mathpar}

\begin{mathpar}
  \inferrule* [lab=alpha-equivalence] {} { (x)P \equiv (y)P\{y/x\} \and y \not\in \freenames{P} }
\end{mathpar}

\begin{definition}
Then two processes, $P,Q$, are alpha-equivalent if $P = Q\{\vec{y}/\vec{x}\}$ for
some $\vec{x} \in \boundnames{Q},\vec{y} \in \boundnames{P}$, where $Q\{\vec{y}/\vec{x}\}$
denotes the capture-avoiding substitution of $\vec{y}$ for $\vec{x}$ in $Q$.
\end{definition}

\begin{definition}
  The {\em structural congruence} \cite{SangiorgiWalker} , $\equiv$,
  between processes is the least congruence containing
  alpha-equivalence, satisfying the abelian monoid laws
  (associativity, commutativity and $\pzero$ as identity) for parallel
  composition $|$ and for summation $+$.
\end{definition}

\subsection{Name equivalence}

We take name equivalence, written $\nameeq$, to be the smallest
equivalence relation generated by the following rules.

\begin{mathpar}
\inferrule*[lab=Quote-drop]
{ }
{ \quotep{@{x}} \nameeq x }

\inferrule*[lab=Struct-equiv]
{ P \scong Q }
{ \quotep{P} \nameeq \quotep{Q} }
\end{mathpar}

The astute reader will have noticed that the mutual recursion of names
and processes imposes a mutual recursion on alpha-equivalence and
structural equivalence via name-equivalence. Fortunately, all of this
works out pleasantly and we may calculate in the natural way, free of
concern. The reader interested in the details is referred to the
appendix \ref{appendix:rho_details}.

\subsection{Substitution}

We use $\Proc$ for the set of processes, $\QProc$ for the set of
names, and $\id{\{}\vec{y} / \vec{x} \id{\}}$ to denote partial maps,
$s : \QProc \rightarrow \QProc$. A map, $s$ lifts, uniquely, to a map
on process terms, $\widehat{s} : \Proc \rightarrow \Proc$ by the
following equations.

\begin{mathpar}
  (0) \psubstp{Q}{P} := 0 \\
  (R \juxtap S) \psubstp{Q}{P}
  :=    
  (R)\psubstp{Q}{P} \juxtap (S) \psubstp{Q}{P} \\
  (x?(y).R) \psubstp{Q}{P}    
  :=    
  (x)\substp{Q}{P} (z)\concat( (R \psubstn{z}{y}) \psubstp{Q}{P} ) \\
  (\lift{x}{R}) \psubstp{Q}{P}  
  :=
  \lift{(x)\substp{Q}{P}}{ R \psubstp{Q}{P} } \\
%   (\dropn{x})  \psubstp{Q}{P}       
%   := 
%   \left\{ 
%     \begin{array}{ccc} 
%       \dropn{\quotep{Q}} & & x \nameeq \quotep{P} \\
%       \dropn{x} & & otherwise \\
%     \end{array}
%   \right. 
  (\dropn{x})  \psubstp{Q}{P}       
  := 
  \left\{ 
    \begin{array}{ccc} 
      Q & & x \nameeq \quotep{P} \\
      \dropn{x} & & otherwise \\
    \end{array}
  \right.
\end{mathpar}
 

where

\begin{eqnarray}
  (x)\id{\{} \lpquote Q \rpquote / \lpquote P \rpquote \id{\}}            = 
  \left\{ 
    \begin{array}{ccc}
      \lpquote Q \rpquote & & x \nameeq \lpquote P \rpquote \\
      x & & otherwise \\
    \end{array}
  \right. \nonumber
\end{eqnarray}

and $z$ is chosen distinct from $\quotep{P}$, $\quotep{Q}$, the free
names in $Q$, and all the names in $R$. Our $\alpha$-equivalence will
be built in the standard way from this substitution.

\begin{remark}\label{rem:no_self_referential_names}
  One consequence of these definitions is that $\forall P. \quotep{P}
  \not\in \freenames{P}$.
\end{remark}

\subsection{ Dynamic quote: an example }

Anticipating something of what's to come, consider applying the
substitution, $\widehat{\id{\{}u / z \id{\}}}$, to the following pair
of processes, $\lift{w}{y!(z)}$ and $w[ \lpquote y!(z) \rpquote ]$.

\begin{eqnarray}
	\lift{w}{y!(z)}\widehat{\id{\{}u / z \id{\}}}
		& = &
		\lift{w}{y!(u)} \nonumber\\
	w[ \lpquote y!(z) \rpquote ] \widehat{ \id{\{}u / z \id{\}} }
		& = &
		w[ \lpquote y!(z) \rpquote ] \nonumber
\end{eqnarray}

Because the body of the process between quotes is impervious to
substitution, we get radically different answers. In fact, by
examining the first process in an input context,
e.g. $x?(z).\lift{w}{y!(z)}$, we see that the process under the lift
operator may be shaped by prefixed inputs binding a name inside it. In
this sense, the lift operator will be seen as a way to dynamically
construct processes before reifying them as names.

Finally equipped with these standard features we can present the
dynamics of the calculus.

\subsubsection{Operational semantics} 

Finally, we introduce the computational dynamics. What marks these
algebras as distinct from other more traditionally studied algebraic
structures, e.g. vector spaces or polynomial rings, is the manner in
which dynamics is captured. In traditional structures, dynamics is typically
expressed through morphisms between such structures, as in linear maps
between vector spaces or morphisms between rings. In algebras
associated with the semantics of computation, the dynamics is
expressed as part of the algebraic structure itself, through a
reduction reduction relation typically denoted by $\red$. Below, we
give a recursive presentation of this relation for the calculus used
in the encoding.

$\red \subseteq \pi \times \pi$
$\red : \pi \to \mathcal{P}(\pi)$

\begin{mathpar}
  \inferrule* [lab=Comm] { \textsf{match}( x_{src}, x_{trgt} ) } { x_{trgt}?(y)P \; | \; x_{src}!\langle {Q} \rangle \red P\{\quotep{Q}/y}\} }
  \and \\
  \inferrule* [lab=Par] {{P} \red {P}'} {{{P} | {Q}} \red {{P}' | {Q}}}
  \and
  \inferrule* [lab=Equiv]{{{P} \scong {P}'} \andalso {{P}' \red {Q}'} \andalso {{Q}' \scong {Q}}}{{P} \red {Q}}
\end{mathpar}

\begin{eqnarray*}
  match_{\equiv} (\quotep{P},\quotep{Q}) & := & P \equiv Q \\
  match_{\dagger}(\quotep{P},\quotep{Q}) & := & \forall R. P|Q \red^{*} R => R \red^{*} 0 \\
  match_{K}(\quotep{P},\quotep{Q}) & := & K \mbox{ for some context } K
\end{eqnarray*}

$u?(x)P | u!\langle Q \rangle \red P\{\quotep{Q}/x\}$

%We write $\wred$ for $\red^*$, and $P\red$ if $\exists Q $ such that $ P \red Q$.
We write $P\red$ if $\exists Q $ such that $ P \red Q$ and $P\not\red$, otherwise.

\section{Replication}

As mentioned before, it is known that replication (and hence
recursion) can be implemented in a higher-order process algebra
\cite{SangiorgiWalker}. As our first example of calculation with the
machinery thus far presented we give the construction explicitly in
the {\rhoc}.

\begin{eqnarray}
	D_{x} & := & \prefix{x}{y}{(\binpar{\outputp{x}{y}}{@{y}})} \nonumber\\
	\bangp_{x}{P} & := & \binpar{{x}!\langle{\binpar{D_{x}}{P}}\rangle}{D_{x}} \nonumber
\end{eqnarray}

\begin{eqnarray}
	\bangp_{x}{P} & & \nonumber\\
	=
	& {x}!\langle{(\prefix{x}{y}{(\outputp{x}{y} | @{y})) | P}}\rangle 
	      | \prefix{x}{y}{(\outputp{x}{y} | @{y})} & \nonumber\\
	\red
	& (\outputp{x}{y} | @{y})\substn{\quotep{(\prefix{x}{y}{(@{y} | \outputp{x}{y})) | P}}}{y} & \nonumber\\
	=
	& \outputp{x}{\quotep{(\prefix{x}{y}{(\outputp{x}{y} | @{y})) | P}}}
	  | {(\prefix{x}{y}{(\outputp{x}{y} | @{y})) | P}} & \nonumber\\
	\red
	& \ldots & \nonumber\\
	\red^*
	& P | P | \ldots & \nonumber
\end{eqnarray}

Of course, this encoding, as an implementation, runs away, unfolding
$\bangp{P}$ eagerly. A lazier and more implementable replication
operator, restricted to input-guarded processes, may be obtained as follows.

\begin{eqnarray}
\bangp{\prefix{u}{v}{P}} 
	:= 
	\binpar{\lift{x}{\prefix{u}{v}{(\binpar{D(x)}{P})}}}{D(x)} \nonumber
\end{eqnarray}

\begin{remark}
  Note that the lazier definition still does not deal with summation
  or mixed summation (i.e. sums over input and output). The reader is
  invited to construct definitions of replication that deal with these
  features. 

  Further, the definitions are parameterized in a name, $x$. Can you,
  gentle reader, make a definition that eliminates this parameter and
  guarantees no accidental interaction between the replication
  machinery and the process being replicated -- i.e. no accidental
  sharing of names used by the process to get its work done and the
  name(s) used by the replication to effect copying. This latter
  revision of the definition of replication is crucial to obtaining
  the expected identity $!!P \sim !P$.
\end{remark}

\begin{remark}\label{rem:paradoxical_combinator}
  The reader familiar with the lambda calculus will have noticed the
  similarity between $D$ and the paradoxical combinator.

  [Ed. note: the existence of this seems to suggest we have to be more
  restrictive on the set of processes and names we admit if we are to
  support no-cloning.]
\end{remark}

\subsubsection{Bisimulation}

The computational dynamics gives rise to another kind of equivalence,
the equivalence of computational behavior. As previously mentioned
this is typically captured \emph{via} some form of bisimulation.

% The notion we use in this paper is weak barbed bisimulation
% \cite{milner91polyadicpi}.

The notion we use in this paper is derived from weak barbed
bisimulation \cite{milner91polyadicpi}. 

\begin{definition}
An \emph{observation relation}, $\downarrow_{\mathcal N}$, over a set
of names, $\mathcal N$, is the smallest relation satisfying the rules
below.

\infrule[Out-barb]{y \in {\mathcal N}, \; x \nameeq y}
		  {\outputp{x}{v} \downarrow_{\mathcal N} x}
\infrule[Par-barb]{\mbox{$P\downarrow_{\mathcal N} x$ or $Q\downarrow_{\mathcal N} x$}}
		  {\binpar{P}{Q} \downarrow_{\mathcal N} x}

We write $P \Downarrow_{\mathcal N} x$ if there is $Q$ such that 
$P \wred Q$ and $Q \downarrow_{\mathcal N} x$.
\end{definition}

\begin{definition}
%\label{def.bbisim}
An  ${\mathcal N}$-\emph{barbed bisimulation} over a set of names, ${\mathcal N}$, is a symmetric binary relation 
${\mathcal S}_{\mathcal N}$ between agents such that $P\rel{S}_{\mathcal N}Q$ implies:
\begin{enumerate}
\item If $P \red P'$ then $Q \wred Q'$ and $P'\rel{S}_{\mathcal N} Q'$.
\item If $P\downarrow_{\mathcal N} x$, then $Q\Downarrow_{\mathcal N} x$.
\end{enumerate}
$P$ is ${\mathcal N}$-barbed bisimilar to $Q$, written
$P \wbbisim_{\mathcal N} Q$, if $P \rel{S}_{\mathcal N} Q$ for some ${\mathcal N}$-barbed bisimulation ${\mathcal S}_{\mathcal N}$.
\end{definition}

$\mathcal{R} \subseteq \pi \times \pi$

$P \mathcal{R} Q => \forall P'. P \red P' \Rightarrow \exists Q'. Q \red Q', P' \mathcal{R} Q'$

$P \vdash x \Rightarrow Q \vdash x$

\begin{mathpar}
  \inferrule*[lab=Out-barb]{x \nameeq y}{{y}!\langle{Q}\rangle \vdash x}
  \and
  \inferrule*[lab=Par-barb]{\mbox{$P\vdash x$ or $Q\vdash x$}}{\binpar{P}{Q} \vdash x}
\end{mathpar}

\subsubsection{Contexts}

One of the principle advantages of computational calculi like the
$\pi$-calculus is a well-defined notion of context,
contextual-equivalence and a correlation between
contextual-equivalence and notions of bisimulation. The notion of
context allows the decomposition of a process into (sub-)process and
its syntactic environment, its context. Thus, a context may be
thought of as a process with a ``hole'' (written $\Box$) in it. The
application of a context $M$ to a process $P$, written $M[P]$, is
tantamount to filling the hole in $M$ with $P$. In this paper we do
not need the full weight of this theory, but do make use of the notion
of context in the proof the main theorem. 

\begin{mathpar}
  \inferrule* [lab=summation] {} {{M_{M},M_{N}} \bc \Box \;|\; x.M_{A} \;|\; M_{M}+M_{N}}
  \and
  \inferrule* [lab=agent] {} {{M_{A}} \bc (\vec{x})M_{P} \;| \; \clift{P_0,\ldots,M_{P},\ldots,P_N}}
  \and \\
  \inferrule* [lab=process] {} {{M_{P}} \bc M_{N} \;| \;P|M_{P} }
\end{mathpar} 

\begin{mathpar}
  \inferrule* [lab=sychronization] {} {M_{N} \bc \Box \;|\; x?M_{F} \;|\; x!M_{C}}
  \and
  \inferrule* [lab=abstraction] {} {{M_{F}} \bc (x)M_{P} }
  \and
  \inferrule* [lab=concretion] {} {{M_{C}} \bc \langle M_{P} \rangle }
  \and \\
  \inferrule* [lab=process] {} {{M_{P}} \bc M_{N} \;| \;P|M_{P} }
\end{mathpar}

\begin{definition}[contextual application] Given a context $M$, and
  process $P$, we define the \emph{contextual application}, $M[P] :=
  M\{P/\Box\}$. That is, the contextual application of M to P is the
  substitution of $P$ for $\Box$ in $M$.
\end{definition}

$\meaningof{-} : L \to \mathcal{P}(\pi)$

\begin{mathpar}
  \inferrule* [lab=collection] {} {\meaningof{true} = \pi, \and \meaningof{~E} = \pi \setminus \meaningof{E}, \and \meaningof{E_{1} \& E_{2}} = \meaningof{E_{1}} \cap \meaningof{E_{2}}}
\end{mathpar}

\begin{mathpar}
  \inferrule* [lab=structure] {} {\meaningof{0} = \{ P \in \pi | P \equiv 0 \}, \and \\ \meaningof{E_1 | E_2} = \{ P \in \pi | P \equiv P_{1} | P_{2}, P_{1} \in \meaningof{E_{1}}, P_{2} \in \meaningof{E_2}\} }
\end{mathpar}

\begin{mathpar}
 \inferrule* [lab=behavior] {} {\meaningof{\langle a?b \rangle E} = \{ P \in \pi | P \equiv Q | u?(y)P', \\ \and \\\\ \and \\ \;\;\; u \in \meaningof{a}, \forall z.P'\{z/y\} \in \meaningof{E\{z/b\}}\}, \and \\ \meaningof{a!E} = \{ P \in \pi | P \equiv Q | x!\langle P' \rangle, x \in \meaningof{a} P' \in \meaningof{E}\} }
\end{mathpar}

\begin{mathpar}
 \inferrule* [lab=nominal] {} {\meaningof{\quotep{E}} = \{ \quotep{P} \in \quotep{\pi} | P \in \meaningof{E} \}, \and \meaningof{\quotep{P}} = \{ \quotep{Q} \in \quotep{\pi} | P \equiv Q \} \and \\ \meaningof{@\quotep{E}} = \{ P \in \pi | P \equiv @x, x \in \meaningof{E} \}}
\end{mathpar}

\begin{eqnarray*}
  \\
  \meaningof{-} : TS \to ST
\end{eqnarray*}

\begin{eqnarray*}
  \\
  L : TS \to ST
\end{eqnarray*}

\begin{eqnarray*}
  \\
  P \models E \iff P \in \meaningof{E}
\end{eqnarray*}

\begin{eqnarray*}
  P \approx_{L} Q \iff \forall E \in L. P \models E \iff Q \models E
\end{eqnarray*}

\begin{eqnarray*}
  P \approx_{K} Q
\end{eqnarray*}

\begin{eqnarray*}
  P \approx Q
\end{eqnarray*}

$\approx_{K} = \approx = \approx_{L}$

\subsubsection{Contextual duality}

Note that contexts extend the quotation operation to a family of
operations from processes to names. Given a context, $M$, we can
define a \emph{nominal context}, $\quotep{M}$ by $\quotep{M}[P] :=
\quotep{M[P]}$. To foreshadow what is to come we observe that these
operations enjoy a duality with processes very much like the duality
between vectors and maps from vectors to scalars.

Further, because the calculus is essentially higher-order, we have a
correspondence between contexts and processes. More specifically,
given a name $x$ and a context $M$ we can construct $M^{*}_{x}$ such
that 

\begin{mathpar}
  M^{*}_{x} | \lift{x}{P} \red M[P]
\end{mathpar}

namely,

\begin{mathpar}
  M^{*}_{x} := x?(u).M[\dropn{u}]
\end{mathpar}

The dependence of $M^{*}_{x}$ on a name makes it an abstraction, 

\begin{mathpar}
  M^{*} := (x)x?(u).M[\dropn{u}]
\end{mathpar}

\subsection{Additional notation}

It will sometimes be convenient to denote the process a name
quotes. We already have the notation $x = \quotep{P}$, but it will be
convenient to introduce an alternate notation, $\procn{x}$, when we
want to emphasize the connection to the use of the name. Note that, by
virtue of name equivalence, $\quotep{\procn{x}} \nameeq x$; so, the
notation is consistent with previous definitions.

Further, because names have structure it is possible to effect
substitutions on the basis of that structure. This means we need to
upgrade our notation for substitutions, which we accomplish by
adapting comprehension notation. Thus,

\begin{mathpar}
  P\{ y / x : x \in S \}
\end{mathpar}

is interpreted to mean the process derived from P by replacing (in a
capture-avoiding manner) each occurrence of $x$ in $S$ by $y$. For example,

\begin{mathpar}
  P\{ \quotep{\procn{x}|\procn{x}} / x : x \in \freenames{P} \}
\end{mathpar}

will replace each (occurrence) of a free name $x$ in $P$ by
$\quotep{\procn{x}|\procn{x}}$.

Also, we will avail ourselves of the notation $x^{L}$ and $x^{R}$ to
denote injections of a name into disjoint copies of the name
space. There are numerous ways to accomplish this. One example can be
found in \cite{MeredithR05}. This notation overloads to vectors of
names: $\vec{x}^{\pi} := (x_{i}^{\pi} \; : \; 0 \leq i < |\vec{x}| )$ where $\pi \in \{L,R\}$.

We also use $P^{\Box} := P|\Box$.

In \cite{MeredithR05} an interpretation of the new operator is
given. It turns out that there are several possible interpretations
all enjoying the requisite algebraic properties of the operator (see
\cite{milner91polyadicpi}). We will therefore make liberal use of
$(\nu\; \vec{x})P$.

% subsection the_syntax_and_semantics_of_the_notation_system (end)   

\section{Interpretation of QM}
\subsection{Supporting definitions}
\subsubsection{Multiplication}
\begin{mathpar}
  \quotep{Q} \cdot \quotep{R} := \quotep{Q|R}
  \and \\
  \quotep{Q} \cdot P := P\{ \quotep{Q|R} / \quotep{R} : \quotep{R} \in \freenames{P} \}
\end{mathpar}

\paragraph{Discussion}
The first line needs little explanation. The second line says that
each free name of the process is replaced with the multiplication of
that name by the scalar. Multiplication of a scalar (name) by a state
(process) results in a process all the names of which have been `moved
over' by parallel composition with the process the scalar
quotes. There is a subtlety that the bound names have to be
manipulated so that multiplied names aren't accidentally
captured. There are many ways to achieve this.

\begin{remark}\label{rem:multiplication_identities}
  The reader is invited to verify that for all $x,y,z \in \QProc$ and $P \in \Proc$
  \begin{mathpar}
    x \cdot \quotep{0} \equiv x 
    \and
    x \cdot y \equiv y \cdot x
    \and
    x \cdot (y \cdot z) \equiv (x \cdot y) \cdot z
    \and \\
    \quotep{0} \cdot P \equiv P
    \and \\
    x \cdot (y \cdot P) \equiv (x \cdot y) \cdot P
    \and \\
    x \cdot (P|Q) \equiv (x \cdot P) | (x \cdot Q)
    \and \\    
  \end{mathpar}
\end{remark}

\subsubsection{Tensor product}

We define a tensor product on processes by structural induction.

\paragraph{Tensor of sums} First note that all summations, including
$\pzero$ and sequence, can be written $\Sigma_{i} x_{i}.A_{i} +
\Sigma_{j} x_{j}.C_{j}$, where we have grouped input-guarded processes
together and output-guarded processes together.

Thus, we can define the tensor product of two summations, $N_{1}\otimes N_{2}$, where

\begin{mathpar}
  N_{1} := \Sigma_{i} x_{i}.A_{i} + \Sigma_{j} x_{j}.C_{j}
  \and
  N_{2} := \Sigma_{i'} y_{i'}.B_{i'} + \Sigma_{j'} y_{j'}.D_{j'} 
\end{mathpar}

as follows.

\begin{mathpar}
  \Sigma_{i} x_{i}.A_{i} + \Sigma_{j} x_{j}.C_{j} \otimes \Sigma_{i'}
  y_{i'}.B_{i'} + \Sigma_{j'} y_{j'}.D_{j'} 
  \and \\
  := \; \Sigma_{i} \Sigma_{i'} \quotep{\stackrel{\vee}{x_{i}}| \stackrel{\vee}{y_{i'}}}.(A_{i}\otimes B_{i'}) \; | \; \Sigma_{i'} \Sigma_{i} \quotep{\stackrel{\vee}{y_{i'}}|\stackrel{\vee}{x_{i}}}.(B_{i'}\otimes A_{i})
  \and
  \;\; | \;\; \Sigma_{j} \Sigma_{j'} \quotep{\stackrel{\vee}{x_{j}}|\stackrel{\vee}{y_{j'}}}.(A_{j}\otimes B_{j'}) \; | \; \Sigma_{j'} \Sigma_{j} \quotep{\stackrel{\vee}{y_{j'}}|\stackrel{\vee}{x_{j}}}.(B_{j'}\otimes A_{j})
\end{mathpar}

\begin{remark}
  Do we need to $x^{L}$ and $y^{R}$ for this construction as well?
\end{remark}

\paragraph{Tensor of parallel compositions} Next, we distribute tensor
over par.

\begin{mathpar}
  P_{1}|P_{2} \otimes Q_{1}|Q_{2} := (P_{1} \otimes Q_{1}) | (P_{1}
  \otimes Q_{2}) | (P_{2} \otimes Q_{1}) | (P_{2} \otimes Q_{2})
\end{mathpar}

\paragraph{Tensor with dropped names} We treat tensor of a
process with a dropped name as parallel composition.

\begin{mathpar}
  P \otimes \dropn{x} := P | \dropn{x}
\end{mathpar}

\paragraph{Tensor of agents}

Finally, we need to define tensor on agents. Note that the definition
of tensor on normal products only tensors inputs with inputs and
outputs with outputs. Thus, we only have to define the operation on
``homogeneous'' pairings.

\begin{mathpar}
  (\vec{x})P \otimes (\vec{y})Q
  \and \\
  := (x_{0}^{L}|y_{0}^{R},\ldots,x_{0}^{L}|y_{n}^{R},\ldots,x_{m}^{L}|y_{0}^{R},\ldots,x_{m}^{L}|y_{n}^R)(P\{ \vec{x}^{L}/\vec{x}\} \otimes Q \{ \vec{y}^{R}/\vec{y}\})
  \and \\
  \clift{\vec{P}} \otimes \clift{\vec{Q}}
  \and \\
  := \clift{P_{0}\otimes Q_{0},\ldots,P_{0}\otimes Q_{n},\ldots,P_{m}\otimes Q_{0},\ldots,P_{m}\otimes Q_{n}}
\end{mathpar}

\begin{remark}
  Observe that arities of tensored abstractions matches arities of
  tensored concretions if the original arities matched. Note also that
  the length of the arities corresponds to the increase in dimension
  we see in ordinary vector space tensor product.
\end{remark}

\begin{remark}
  Operationally, this definition distributes the tensor down to
  components ``linked'' by summation. Tensor over summation is
  intriguing in that it mixes names. Moreover, as a consequence of the
  way it mixes names we have the identities for all $x \in \QProc$ and
  $P,Q \in \Proc$

  \begin{mathpar}
    (x \cdot P) \otimes Q \equiv x \cdot (P \otimes Q) \equiv P \otimes (x \cdot Q)
    \and
    P \otimes \pzero \equiv P
  \end{mathpar}

  that the reader is invited to verify.
\end{remark}

\subsubsection{Annihilation}
\begin{mathpar}
  P^{\perp} := \{ Q | \forall R. P|Q \red^{*} R \Rightarrow R \red^{*} \pzero \}
  \and \\
  P^{\underline{\perp}} := \Sigma_{Q \in P^{\perp}} \quotep{Q}?(y).(\dropn{y}|Q) | \Sigma_{Q \in P^{\perp}} \quotep{Q}\clift{\Box}
\end{mathpar}

\paragraph{Discussion} The reader will note that $P^{\perp}$ is a
\emph{set} of processes, while $P^{\underline{\perp}}$ is a
\emph{context}. We call the set $P^{\perp}$ the \emph{annihilators} of
$P$. The parallel composition of a process in the annihilators of $P$
with $P$ will result in a process, the state space of which has all
paths eventually leading to $\pzero$. Execution may endure loops; but
under reasonable conditions of fairness (naturally guaranteed under
most notions of bisimulation) such a composite process cannot get
stuck in such a loop and will, eventually pop out and terminate.

The context $P^{\underline{\perp}}$ is ready and willing to ``take the
$P$ out of'' the process to which it is applied. It will effectively
transmit the code of the process to which it is applied to one of the
annihilators and run the process against it.

\subsubsection{Evaluation}
We fix $M$ a domain of fully abstract interpretation with an equality
coincident with bisimulation. We take $\meaningof{\cdot} : \Proc \to
M$ to be the map interpreting processes and $\nmeaningof{\cdot} : \M
\to Proc$ to be the map running the other way. Then we define

\begin{mathpar}
  \int P := \nmeaningof{\meaningof{P}}
\end{mathpar}

\paragraph{Discussion}
There are many fully abstract interpretations of Milner's
$\pi$-calculus. Any of them can be used as a basis for interpreting
the reflective calculus here. Equipped with such a domain it is
largely a matter of grinding through to check that the Yoneda
construction for the normalization-by-evaluation program can be
extended to this setting.

\begin{remark}
  The reader is invited to verify that $\int (P^{\underline{\perp}}[P]) = 0$.
\end{remark}

\subsection{Quantum mechanics}

Table \ref{tbl:core_qm_op_defns} gives the core operational definitions

\begin{table}[htp]\label{tbl:core_qm_op_defns}
  \center{
    \fbox{
      \begin{tabular}{c|c}
        quantum mechanics & process calculus \\
        \hline
        scalar & $x := \quotep{P}$ \\
        state vector & $\state{P} := P$ \\
        dual & $\state{P}^{*} := \event{P^{\underline{\perp}}} := \quotep{P^{\underline{\perp}}}[-]$ \\
        matrix & $ \Sigma_{\alpha} \state{P_{\alpha}}x_{\alpha}\event{Q_{\alpha}}$ \\
        vector addition & $\state{P} + \state{Q} := \state{P | Q}$ \\
        tensor product & $\state{P} \otimes \state{Q} := \state{P \otimes Q}$ \\
        inner product & $\innerprod{P}{Q} := \quotep{\int P^{\underline{\perp}}[Q]}$ \\
      \end{tabular}
    }
  }
  \caption{QM - operational definitions}
\end{table}

where

\begin{mathpar}
  \prmatrix{P}{Q} := \fprmatrix{P}{\quotep{\pzero}}{Q}
  \and
  \fprmatrix{P}{x}{Q} := (\state{P},x,\event{Q})
  \and
  (\fprmatrix{P}{x}{Q})(\state{R}) := x \cdot \innerprod{Q}{R} \cdot \state{P}
  \and
  (\fprmatrix{P}{x}{Q})(\event{R}) := x \cdot \innerprod{R}{P} \cdot \event{Q}
\end{mathpar}

\paragraph{Discussion}
As promised: vectors (aka states) are represented as processes; duals
as contextual duals; inner product definition should be compared with
standard inner product definition for ....

\begin{remark}
  Assuming $\int (P^{\underline{\perp}}[P]) = 0$, the reader is
  invited to verify that $(\fprmatrix{P}{x}{P})(\state{P}) = x \cdot \state{P}$.
\end{remark}

\begin{remark}
  The reader is invited to verify that $\innerprod{P}{Q}$ could
  equally well have been written $\quotep{\int \stackrel{\vee}{x}}$
  where $x = \event{P^{\underline{\perp}}}(Q)$.

  One of the motivations for this remark is that there is another way
  to factor these operations. We could package up evaluation in the dual:

  \begin{mathpar}
    \state{P}^{*} := \event{\int P^{\underline{\perp}}} := \quotep{\int P^{\underline{\perp}}}[-]
  \end{mathpar}

  and then have inner product defined by
  
  \begin{mathpar}
    \innerprod{P}{Q} := \event{P}(Q)
  \end{mathpar}

  Hopefully, experience with the calculations will provide guidance on
  the best factoring.
\end{remark}

\begin{remark}
  Assuming $\int (P^{\underline{\perp}}[P]) = 0$, the reader is
  invited to verify that $\forall P,Q. (\prmatrix{0}{Q})(\state{0}) =
  \state{0}$ and dually $(\prmatrix{P}{0})(\event{0}) = \event{0}$.
\end{remark}

\begin{remark}
  i'm a little worried that i don't (yet) have proper support for
  complex conjugacy. But, the observation above may give us a
  clue. According to Abramsky, it must be the case that the scalars
  are iso to the homset of the identity for the tensor -- which the
  observation above characterizes. 

  For now, we will simply bookmark the notion with $\overline{x}$.
\end{remark}

\subsubsection{Adjointness}

We need to give a definition of $(\cdot)^{\dagger}$ for matrices. The
obvious candidate definition is
\begin{mathpar}
(\Sigma_{\alpha}\fprmatrix{P_{\alpha}}{x_{\alpha}}{Q_{\alpha}})^{\dagger}
= \Sigma_{\alpha}\fprmatrix{(Q_{\alpha}^{\underline{\perp}})^{*}}{\overline{x}_{\alpha}}{P_{\alpha}^{\underline{\perp}}} 
\end{mathpar}

But, $(Q_{\alpha}^{\underline{\perp}})^{*}$ requires a name along
which to communicate the process to achieve the context application.

\subsubsection{Basis for a basis}
If processes label states and ``addition'' of states (a.k.a. vector
addition) is interpreted as parallel composition, what corresponds to
notions of linear independence and basis? Here, we recall that Yoshida
has developed a set of \emph{combinators} for an asynchronous verison
of Milner's $\pi$-calculus. These are a finite set of processes such
any process can be expressed as parallel composition of these
combinators together with liberal uses of the new operator and
replication. We can simply give a translation of these into the
present calculus and have reasonable expectation that the property
carries over. That is, that the resultant set allows to express all
processes via parallel composition. Note, however, that there is no
new operator or replication in this calculus. As a result, we expect
that the corresponding set is actually infinite. That is, we expect
that the space is actually infinite dimensional.

\begin{remark}
  The attentive reader may be a bit concerned. Certainly, the
  collection $S$, $K$ and $I$ is a finite set of
  combinators. Shouldn't we expect to see a finite set of combinators
  for an effectively equivalent system? i am very sympathetic to this
  critique and feel it warrants full attention. On the other hand, i
  also have in mind the following analogy. The natural numbers, as a
  monoid under addition, has exactly $1$ generator, while the natural
  numbers, as a monoid under multiplication, has countably many
  generators (the primes). We observe that the application of the
  lambda calculus is much less resource sensitive than the parallel
  composition of the $\pi$-calculus. Could it be the case that we have
  an analogy of the form
  
  \begin{mathpar}
    m + n : MN :: m*n : M|N
  \end{mathpar}

  giving a similar blow up in the set of ``primes''?  This is such a
  wonderful thought that, even if it's not true, i think it's worth
  writing down.
\end{remark}
 

\documentclass[12pt]{llncs}
%\documentclass{jktr}

\usepackage[pdftex]{hyperref}                   
\usepackage {listings}
\usepackage {mathpartir}
\usepackage{bcprules}
%\usepackage{listings}
                       
\usepackage{graphicx} 
%\usepackage[margins=2.5cm,nohead,nofoot]{geometry}
%\usepackage{geometry}
\usepackage{amsfonts}
\usepackage{amstext}
\usepackage{latexsym}
\usepackage{amssymb}
\usepackage{color}


%\include{myPreamble}
\include{qm2pi.local} 

%\ifpdf
%\usepackage[pdftex]{graphicx}
%\else
%\usepackage{graphicx}
%\fi

 % \ifpdf
%  \usepackage{pdfsync}
%  \if


%\title{Brief Article}
%\author{David F. Snyder}
%\author{L.G. Meredith}

%\address{Dept. of Math., Texas State University--San Marcos, San Marcos, TX 78666}
       
\pagestyle{empty}


\begin{document}

\lstset{language=[Objective]Caml,frame=shadowbox}

\input{qm2pi.front}

% section front matter (end)

\input{qm2pi.intro} 
 
% section introduction (end)

% \input{qm2pi.knotations} 

% section notation (end)

\input{qm2pi.process.calculi} 

% section concurrent_process_calculi_and_spatial_logics_ (end)
    
%\input{qm2pi.knots2pi} 

%\input{qm2pi.trefoil} 

%\input{qm2pi.mainthm} 

% subsection basic_interpretation (end)

%\input{qm2pi.rho.presentation} 
\subsection{The syntax and semantics of the notation system}\label{sub:the_syntax_and_semantics_of_the_notation_system} % (fold)

We now summarize a technical presentation of the calculus that
embodies our theory of dynamics. The typical presentation of such a
calculus follows the style of giving generators and relations on
them. The grammar, below, describing term constructors, freely
generates the set of processes, $\Proc$. This set is then quotiented
by a relation known as structural congruence and it is over this set
that the notion of dynamics is expressed. This presentation is
essentially that of \cite{MeredithR05} with the addition of
polyadicity and summation. For readability we have relegated some of
the technical subtleties to an appendix.

\subsubsection{Process grammar}\label{subsub:process_grammar}

\begin{mathpar}
  \inferrule* [lab=synchronization] {} {{M} \bc \pzero \;|\; x?F \;|\; x!C }
  \and
  \inferrule* [lab=abstraction] {} {{F} \bc (x)P}
  \and
  \inferrule* [lab=concretion] {} {{C} \bc \langle Q \rangle}
  \and
  \inferrule* [lab=process] {} {{P,Q} \bc M \;| \;P|Q \;|\; @{x}}
  \and
  \inferrule* [lab=name] {} {{x} \bc \quotep{P}}
\end{mathpar} 

Note that $\vec{x}$ (resp. $\vec{P}$) denotes a vector of names
(resp. processes) of length $|\vec{x}|$ (resp. $|\vec{P}|$). We adopt
the following useful abbreviations.

\begin{mathpar}
   x?(\vec{y}).P := x.(\vec{y})P \and  x\clift{\vec{P}} := x.\clift{\vec{P}}
   \and x!(y) := \lift{x}{\dropn{y}}
   \and \Pi_{i=0}^{n-1}P_i := P_0 | \ldots | P_{n-1}
\end{mathpar}

\subsubsection{Structural congruence}

\paragraph{Free and bound names and alpha-equivalence.} At the
core of structural equivalence is alpha-equivalence which identifies
process that are the same up to a change of variable. Formally, we
recognize the distinction between free and bound names. The free names
of a process, $\freenames{P}$, may be calculated recursively as
follows:

\begin{mathpar}
\freenames{\pzero} := \emptyset
  \and \\
  \freenames{x?(y).P} := \{ x \} \cup (\freenames{P} \setminus \{ y \})
  \and 
  \freenames{x!\langle P \rangle} := \{ x \} \cup \{ P \} 
  \and \\
  \freenames{P|Q} := \freenames{P} \cup \freenames{Q}
  \and \\
  \freenames{@{x}} := \{ x \}
\end{mathpar}

$\pi$
$\quotep{\pi}$

$\freenames{-} : \pi \to \mathcal{P}(\quotep{\pi})$

\begin{eqnarray*}
  \freenames{\pzero} & := & \emptyset \\
  \freenames{x?(y).P} & := & \{ x \} \cup (\freenames{P} \setminus \{ y \}) \\
  \freenames{x!\langle P \rangle} & := & \{ x \} \cup \{ P \} \\
  \freenames{P|Q} & := & \freenames{P} \cup \freenames{Q} \\
  \freenames{\dropn{x}} & := & \{ x \}
\end{eqnarray*}

The bound names of a process, $\boundnames{P}$, are those names occurring in $P$
that are not free. For example, in $x?(y).0$, the name $x$ is free, while $y$ is bound.

\begin{mathpar}
  \inferrule* [lab=monoidal-laws] {} { P|Q \equiv Q|P \and P|0 \equiv P \and P|(Q|R) \equiv (P|Q)|R }
\end{mathpar}

\begin{mathpar}
  \inferrule* [lab=alpha-equivalence] {} { (x)P \equiv (y)P\{y/x\} \and y \not\in \freenames{P} }
\end{mathpar}

\begin{definition}
Then two processes, $P,Q$, are alpha-equivalent if $P = Q\{\vec{y}/\vec{x}\}$ for
some $\vec{x} \in \boundnames{Q},\vec{y} \in \boundnames{P}$, where $Q\{\vec{y}/\vec{x}\}$
denotes the capture-avoiding substitution of $\vec{y}$ for $\vec{x}$ in $Q$.
\end{definition}

\begin{definition}
  The {\em structural congruence} \cite{SangiorgiWalker} , $\equiv$,
  between processes is the least congruence containing
  alpha-equivalence, satisfying the abelian monoid laws
  (associativity, commutativity and $\pzero$ as identity) for parallel
  composition $|$ and for summation $+$.
\end{definition}

\subsection{Name equivalence}

We take name equivalence, written $\nameeq$, to be the smallest
equivalence relation generated by the following rules.

\begin{mathpar}
\inferrule*[lab=Quote-drop]
{ }
{ \quotep{@{x}} \nameeq x }

\inferrule*[lab=Struct-equiv]
{ P \scong Q }
{ \quotep{P} \nameeq \quotep{Q} }
\end{mathpar}

The astute reader will have noticed that the mutual recursion of names
and processes imposes a mutual recursion on alpha-equivalence and
structural equivalence via name-equivalence. Fortunately, all of this
works out pleasantly and we may calculate in the natural way, free of
concern. The reader interested in the details is referred to the
appendix \ref{appendix:rho_details}.

\subsection{Substitution}

We use $\Proc$ for the set of processes, $\QProc$ for the set of
names, and $\id{\{}\vec{y} / \vec{x} \id{\}}$ to denote partial maps,
$s : \QProc \rightarrow \QProc$. A map, $s$ lifts, uniquely, to a map
on process terms, $\widehat{s} : \Proc \rightarrow \Proc$ by the
following equations.

\begin{mathpar}
  (0) \psubstp{Q}{P} := 0 \\
  (R \juxtap S) \psubstp{Q}{P}
  :=    
  (R)\psubstp{Q}{P} \juxtap (S) \psubstp{Q}{P} \\
  (x?(y).R) \psubstp{Q}{P}    
  :=    
  (x)\substp{Q}{P} (z)\concat( (R \psubstn{z}{y}) \psubstp{Q}{P} ) \\
  (\lift{x}{R}) \psubstp{Q}{P}  
  :=
  \lift{(x)\substp{Q}{P}}{ R \psubstp{Q}{P} } \\
%   (\dropn{x})  \psubstp{Q}{P}       
%   := 
%   \left\{ 
%     \begin{array}{ccc} 
%       \dropn{\quotep{Q}} & & x \nameeq \quotep{P} \\
%       \dropn{x} & & otherwise \\
%     \end{array}
%   \right. 
  (\dropn{x})  \psubstp{Q}{P}       
  := 
  \left\{ 
    \begin{array}{ccc} 
      Q & & x \nameeq \quotep{P} \\
      \dropn{x} & & otherwise \\
    \end{array}
  \right.
\end{mathpar}
 

where

\begin{eqnarray}
  (x)\id{\{} \lpquote Q \rpquote / \lpquote P \rpquote \id{\}}            = 
  \left\{ 
    \begin{array}{ccc}
      \lpquote Q \rpquote & & x \nameeq \lpquote P \rpquote \\
      x & & otherwise \\
    \end{array}
  \right. \nonumber
\end{eqnarray}

and $z$ is chosen distinct from $\quotep{P}$, $\quotep{Q}$, the free
names in $Q$, and all the names in $R$. Our $\alpha$-equivalence will
be built in the standard way from this substitution.

\begin{remark}\label{rem:no_self_referential_names}
  One consequence of these definitions is that $\forall P. \quotep{P}
  \not\in \freenames{P}$.
\end{remark}

\subsection{ Dynamic quote: an example }

Anticipating something of what's to come, consider applying the
substitution, $\widehat{\id{\{}u / z \id{\}}}$, to the following pair
of processes, $\lift{w}{y!(z)}$ and $w[ \lpquote y!(z) \rpquote ]$.

\begin{eqnarray}
	\lift{w}{y!(z)}\widehat{\id{\{}u / z \id{\}}}
		& = &
		\lift{w}{y!(u)} \nonumber\\
	w[ \lpquote y!(z) \rpquote ] \widehat{ \id{\{}u / z \id{\}} }
		& = &
		w[ \lpquote y!(z) \rpquote ] \nonumber
\end{eqnarray}

Because the body of the process between quotes is impervious to
substitution, we get radically different answers. In fact, by
examining the first process in an input context,
e.g. $x?(z).\lift{w}{y!(z)}$, we see that the process under the lift
operator may be shaped by prefixed inputs binding a name inside it. In
this sense, the lift operator will be seen as a way to dynamically
construct processes before reifying them as names.

Finally equipped with these standard features we can present the
dynamics of the calculus.

\subsubsection{Operational semantics} 

Finally, we introduce the computational dynamics. What marks these
algebras as distinct from other more traditionally studied algebraic
structures, e.g. vector spaces or polynomial rings, is the manner in
which dynamics is captured. In traditional structures, dynamics is typically
expressed through morphisms between such structures, as in linear maps
between vector spaces or morphisms between rings. In algebras
associated with the semantics of computation, the dynamics is
expressed as part of the algebraic structure itself, through a
reduction reduction relation typically denoted by $\red$. Below, we
give a recursive presentation of this relation for the calculus used
in the encoding.

$\red \subseteq \pi \times \pi$
$\red : \pi \to \mathcal{P}(\pi)$

\begin{mathpar}
  \inferrule* [lab=Comm] { \textsf{match}( x_{src}, x_{trgt} ) } { x_{trgt}?(y)P \; | \; x_{src}!\langle {Q} \rangle \red P\{\quotep{Q}/y}\} }
  \and \\
  \inferrule* [lab=Par] {{P} \red {P}'} {{{P} | {Q}} \red {{P}' | {Q}}}
  \and
  \inferrule* [lab=Equiv]{{{P} \scong {P}'} \andalso {{P}' \red {Q}'} \andalso {{Q}' \scong {Q}}}{{P} \red {Q}}
\end{mathpar}

\begin{eqnarray*}
  match_{\equiv} (\quotep{P},\quotep{Q}) & := & P \equiv Q \\
  match_{\dagger}(\quotep{P},\quotep{Q}) & := & \forall R. P|Q \red^{*} R => R \red^{*} 0 \\
  match_{K}(\quotep{P},\quotep{Q}) & := & K \mbox{ for some context } K
\end{eqnarray*}

$u?(x)P | u!\langle Q \rangle \red P\{\quotep{Q}/x\}$

%We write $\wred$ for $\red^*$, and $P\red$ if $\exists Q $ such that $ P \red Q$.
We write $P\red$ if $\exists Q $ such that $ P \red Q$ and $P\not\red$, otherwise.

\section{Replication}

As mentioned before, it is known that replication (and hence
recursion) can be implemented in a higher-order process algebra
\cite{SangiorgiWalker}. As our first example of calculation with the
machinery thus far presented we give the construction explicitly in
the {\rhoc}.

\begin{eqnarray}
	D_{x} & := & \prefix{x}{y}{(\binpar{\outputp{x}{y}}{@{y}})} \nonumber\\
	\bangp_{x}{P} & := & \binpar{{x}!\langle{\binpar{D_{x}}{P}}\rangle}{D_{x}} \nonumber
\end{eqnarray}

\begin{eqnarray}
	\bangp_{x}{P} & & \nonumber\\
	=
	& {x}!\langle{(\prefix{x}{y}{(\outputp{x}{y} | @{y})) | P}}\rangle 
	      | \prefix{x}{y}{(\outputp{x}{y} | @{y})} & \nonumber\\
	\red
	& (\outputp{x}{y} | @{y})\substn{\quotep{(\prefix{x}{y}{(@{y} | \outputp{x}{y})) | P}}}{y} & \nonumber\\
	=
	& \outputp{x}{\quotep{(\prefix{x}{y}{(\outputp{x}{y} | @{y})) | P}}}
	  | {(\prefix{x}{y}{(\outputp{x}{y} | @{y})) | P}} & \nonumber\\
	\red
	& \ldots & \nonumber\\
	\red^*
	& P | P | \ldots & \nonumber
\end{eqnarray}

Of course, this encoding, as an implementation, runs away, unfolding
$\bangp{P}$ eagerly. A lazier and more implementable replication
operator, restricted to input-guarded processes, may be obtained as follows.

\begin{eqnarray}
\bangp{\prefix{u}{v}{P}} 
	:= 
	\binpar{\lift{x}{\prefix{u}{v}{(\binpar{D(x)}{P})}}}{D(x)} \nonumber
\end{eqnarray}

\begin{remark}
  Note that the lazier definition still does not deal with summation
  or mixed summation (i.e. sums over input and output). The reader is
  invited to construct definitions of replication that deal with these
  features. 

  Further, the definitions are parameterized in a name, $x$. Can you,
  gentle reader, make a definition that eliminates this parameter and
  guarantees no accidental interaction between the replication
  machinery and the process being replicated -- i.e. no accidental
  sharing of names used by the process to get its work done and the
  name(s) used by the replication to effect copying. This latter
  revision of the definition of replication is crucial to obtaining
  the expected identity $!!P \sim !P$.
\end{remark}

\begin{remark}\label{rem:paradoxical_combinator}
  The reader familiar with the lambda calculus will have noticed the
  similarity between $D$ and the paradoxical combinator.

  [Ed. note: the existence of this seems to suggest we have to be more
  restrictive on the set of processes and names we admit if we are to
  support no-cloning.]
\end{remark}

\subsubsection{Bisimulation}

The computational dynamics gives rise to another kind of equivalence,
the equivalence of computational behavior. As previously mentioned
this is typically captured \emph{via} some form of bisimulation.

% The notion we use in this paper is weak barbed bisimulation
% \cite{milner91polyadicpi}.

The notion we use in this paper is derived from weak barbed
bisimulation \cite{milner91polyadicpi}. 

\begin{definition}
An \emph{observation relation}, $\downarrow_{\mathcal N}$, over a set
of names, $\mathcal N$, is the smallest relation satisfying the rules
below.

\infrule[Out-barb]{y \in {\mathcal N}, \; x \nameeq y}
		  {\outputp{x}{v} \downarrow_{\mathcal N} x}
\infrule[Par-barb]{\mbox{$P\downarrow_{\mathcal N} x$ or $Q\downarrow_{\mathcal N} x$}}
		  {\binpar{P}{Q} \downarrow_{\mathcal N} x}

We write $P \Downarrow_{\mathcal N} x$ if there is $Q$ such that 
$P \wred Q$ and $Q \downarrow_{\mathcal N} x$.
\end{definition}

\begin{definition}
%\label{def.bbisim}
An  ${\mathcal N}$-\emph{barbed bisimulation} over a set of names, ${\mathcal N}$, is a symmetric binary relation 
${\mathcal S}_{\mathcal N}$ between agents such that $P\rel{S}_{\mathcal N}Q$ implies:
\begin{enumerate}
\item If $P \red P'$ then $Q \wred Q'$ and $P'\rel{S}_{\mathcal N} Q'$.
\item If $P\downarrow_{\mathcal N} x$, then $Q\Downarrow_{\mathcal N} x$.
\end{enumerate}
$P$ is ${\mathcal N}$-barbed bisimilar to $Q$, written
$P \wbbisim_{\mathcal N} Q$, if $P \rel{S}_{\mathcal N} Q$ for some ${\mathcal N}$-barbed bisimulation ${\mathcal S}_{\mathcal N}$.
\end{definition}

$\mathcal{R} \subseteq \pi \times \pi$

$P \mathcal{R} Q => \forall P'. P \red P' \Rightarrow \exists Q'. Q \red Q', P' \mathcal{R} Q'$

$P \vdash x \Rightarrow Q \vdash x$

\begin{mathpar}
  \inferrule*[lab=Out-barb]{x \nameeq y}{{y}!\langle{Q}\rangle \vdash x}
  \and
  \inferrule*[lab=Par-barb]{\mbox{$P\vdash x$ or $Q\vdash x$}}{\binpar{P}{Q} \vdash x}
\end{mathpar}

\subsubsection{Contexts}

One of the principle advantages of computational calculi like the
$\pi$-calculus is a well-defined notion of context,
contextual-equivalence and a correlation between
contextual-equivalence and notions of bisimulation. The notion of
context allows the decomposition of a process into (sub-)process and
its syntactic environment, its context. Thus, a context may be
thought of as a process with a ``hole'' (written $\Box$) in it. The
application of a context $M$ to a process $P$, written $M[P]$, is
tantamount to filling the hole in $M$ with $P$. In this paper we do
not need the full weight of this theory, but do make use of the notion
of context in the proof the main theorem. 

\begin{mathpar}
  \inferrule* [lab=summation] {} {{M_{M},M_{N}} \bc \Box \;|\; x.M_{A} \;|\; M_{M}+M_{N}}
  \and
  \inferrule* [lab=agent] {} {{M_{A}} \bc (\vec{x})M_{P} \;| \; \clift{P_0,\ldots,M_{P},\ldots,P_N}}
  \and \\
  \inferrule* [lab=process] {} {{M_{P}} \bc M_{N} \;| \;P|M_{P} }
\end{mathpar} 

\begin{mathpar}
  \inferrule* [lab=sychronization] {} {M_{N} \bc \Box \;|\; x?M_{F} \;|\; x!M_{C}}
  \and
  \inferrule* [lab=abstraction] {} {{M_{F}} \bc (x)M_{P} }
  \and
  \inferrule* [lab=concretion] {} {{M_{C}} \bc \langle M_{P} \rangle }
  \and \\
  \inferrule* [lab=process] {} {{M_{P}} \bc M_{N} \;| \;P|M_{P} }
\end{mathpar}

\begin{definition}[contextual application] Given a context $M$, and
  process $P$, we define the \emph{contextual application}, $M[P] :=
  M\{P/\Box\}$. That is, the contextual application of M to P is the
  substitution of $P$ for $\Box$ in $M$.
\end{definition}

$\meaningof{-} : L \to \mathcal{P}(\pi)$

\begin{mathpar}
  \inferrule* [lab=collection] {} {\meaningof{true} = \pi, \and \meaningof{~E} = \pi \setminus \meaningof{E}, \and \meaningof{E_{1} \& E_{2}} = \meaningof{E_{1}} \cap \meaningof{E_{2}}}
\end{mathpar}

\begin{mathpar}
  \inferrule* [lab=structure] {} {\meaningof{0} = \{ P \in \pi | P \equiv 0 \}, \and \\ \meaningof{E_1 | E_2} = \{ P \in \pi | P \equiv P_{1} | P_{2}, P_{1} \in \meaningof{E_{1}}, P_{2} \in \meaningof{E_2}\} }
\end{mathpar}

\begin{mathpar}
 \inferrule* [lab=behavior] {} {\meaningof{\langle a?b \rangle E} = \{ P \in \pi | P \equiv Q | u?(y)P', \\ \and \\\\ \and \\ \;\;\; u \in \meaningof{a}, \forall z.P'\{z/y\} \in \meaningof{E\{z/b\}}\}, \and \\ \meaningof{a!E} = \{ P \in \pi | P \equiv Q | x!\langle P' \rangle, x \in \meaningof{a} P' \in \meaningof{E}\} }
\end{mathpar}

\begin{mathpar}
 \inferrule* [lab=nominal] {} {\meaningof{\quotep{E}} = \{ \quotep{P} \in \quotep{\pi} | P \in \meaningof{E} \}, \and \meaningof{\quotep{P}} = \{ \quotep{Q} \in \quotep{\pi} | P \equiv Q \} \and \\ \meaningof{@\quotep{E}} = \{ P \in \pi | P \equiv @x, x \in \meaningof{E} \}}
\end{mathpar}

\begin{eqnarray*}
  \\
  \meaningof{-} : TS \to ST
\end{eqnarray*}

\begin{eqnarray*}
  \\
  L : TS \to ST
\end{eqnarray*}

\begin{eqnarray*}
  \\
  P \models E \iff P \in \meaningof{E}
\end{eqnarray*}

\begin{eqnarray*}
  P \approx_{L} Q \iff \forall E \in L. P \models E \iff Q \models E
\end{eqnarray*}

\begin{eqnarray*}
  P \approx_{K} Q
\end{eqnarray*}

\begin{eqnarray*}
  P \approx Q
\end{eqnarray*}

$\approx_{K} = \approx = \approx_{L}$

\subsubsection{Contextual duality}

Note that contexts extend the quotation operation to a family of
operations from processes to names. Given a context, $M$, we can
define a \emph{nominal context}, $\quotep{M}$ by $\quotep{M}[P] :=
\quotep{M[P]}$. To foreshadow what is to come we observe that these
operations enjoy a duality with processes very much like the duality
between vectors and maps from vectors to scalars.

Further, because the calculus is essentially higher-order, we have a
correspondence between contexts and processes. More specifically,
given a name $x$ and a context $M$ we can construct $M^{*}_{x}$ such
that 

\begin{mathpar}
  M^{*}_{x} | \lift{x}{P} \red M[P]
\end{mathpar}

namely,

\begin{mathpar}
  M^{*}_{x} := x?(u).M[\dropn{u}]
\end{mathpar}

The dependence of $M^{*}_{x}$ on a name makes it an abstraction, 

\begin{mathpar}
  M^{*} := (x)x?(u).M[\dropn{u}]
\end{mathpar}

\subsection{Additional notation}

It will sometimes be convenient to denote the process a name
quotes. We already have the notation $x = \quotep{P}$, but it will be
convenient to introduce an alternate notation, $\procn{x}$, when we
want to emphasize the connection to the use of the name. Note that, by
virtue of name equivalence, $\quotep{\procn{x}} \nameeq x$; so, the
notation is consistent with previous definitions.

Further, because names have structure it is possible to effect
substitutions on the basis of that structure. This means we need to
upgrade our notation for substitutions, which we accomplish by
adapting comprehension notation. Thus,

\begin{mathpar}
  P\{ y / x : x \in S \}
\end{mathpar}

is interpreted to mean the process derived from P by replacing (in a
capture-avoiding manner) each occurrence of $x$ in $S$ by $y$. For example,

\begin{mathpar}
  P\{ \quotep{\procn{x}|\procn{x}} / x : x \in \freenames{P} \}
\end{mathpar}

will replace each (occurrence) of a free name $x$ in $P$ by
$\quotep{\procn{x}|\procn{x}}$.

Also, we will avail ourselves of the notation $x^{L}$ and $x^{R}$ to
denote injections of a name into disjoint copies of the name
space. There are numerous ways to accomplish this. One example can be
found in \cite{MeredithR05}. This notation overloads to vectors of
names: $\vec{x}^{\pi} := (x_{i}^{\pi} \; : \; 0 \leq i < |\vec{x}| )$ where $\pi \in \{L,R\}$.

We also use $P^{\Box} := P|\Box$.

In \cite{MeredithR05} an interpretation of the new operator is
given. It turns out that there are several possible interpretations
all enjoying the requisite algebraic properties of the operator (see
\cite{milner91polyadicpi}). We will therefore make liberal use of
$(\nu\; \vec{x})P$.

% subsection the_syntax_and_semantics_of_the_notation_system (end)   

\input{qm2pi.qmops} 

\input{qm2pi.sterngerlach} 

\input{qm2pi.metric} 

% section concurrent_process_calculi (end)

%\input{qm2pi.proofsketch}

% section proof sketch (end)

%\input{qm2pi.slviaknots} 

% section spatial logic via knots (end)

\input{qm2pi.conclusion}

% section conclusion (end)

%\input{qm2pi.dtcodes} 

% section wiring algorithm (end)

\input{qm2pi.ack} 

% section acknowledgments (end)

\newpage


\bibliographystyle{plain}   
\bibliography{../../biblios/main.bib}

\input{qm2pi.rhodetails}

\end{document}

 

\documentclass[12pt]{llncs}
%\documentclass{jktr}

\usepackage[pdftex]{hyperref}                   
\usepackage {listings}
\usepackage {mathpartir}
\usepackage{bcprules}
%\usepackage{listings}
                       
\usepackage{graphicx} 
%\usepackage[margins=2.5cm,nohead,nofoot]{geometry}
%\usepackage{geometry}
\usepackage{amsfonts}
\usepackage{amstext}
\usepackage{latexsym}
\usepackage{amssymb}
\usepackage{color}


%\include{myPreamble}
\include{qm2pi.local} 

%\ifpdf
%\usepackage[pdftex]{graphicx}
%\else
%\usepackage{graphicx}
%\fi

 % \ifpdf
%  \usepackage{pdfsync}
%  \if


%\title{Brief Article}
%\author{David F. Snyder}
%\author{L.G. Meredith}

%\address{Dept. of Math., Texas State University--San Marcos, San Marcos, TX 78666}
       
\pagestyle{empty}


\begin{document}

\lstset{language=[Objective]Caml,frame=shadowbox}

\input{qm2pi.front}

% section front matter (end)

\input{qm2pi.intro} 
 
% section introduction (end)

% \input{qm2pi.knotations} 

% section notation (end)

\input{qm2pi.process.calculi} 

% section concurrent_process_calculi_and_spatial_logics_ (end)
    
%\input{qm2pi.knots2pi} 

%\input{qm2pi.trefoil} 

%\input{qm2pi.mainthm} 

% subsection basic_interpretation (end)

%\input{qm2pi.rho.presentation} 
\subsection{The syntax and semantics of the notation system}\label{sub:the_syntax_and_semantics_of_the_notation_system} % (fold)

We now summarize a technical presentation of the calculus that
embodies our theory of dynamics. The typical presentation of such a
calculus follows the style of giving generators and relations on
them. The grammar, below, describing term constructors, freely
generates the set of processes, $\Proc$. This set is then quotiented
by a relation known as structural congruence and it is over this set
that the notion of dynamics is expressed. This presentation is
essentially that of \cite{MeredithR05} with the addition of
polyadicity and summation. For readability we have relegated some of
the technical subtleties to an appendix.

\subsubsection{Process grammar}\label{subsub:process_grammar}

\begin{mathpar}
  \inferrule* [lab=synchronization] {} {{M} \bc \pzero \;|\; x?F \;|\; x!C }
  \and
  \inferrule* [lab=abstraction] {} {{F} \bc (x)P}
  \and
  \inferrule* [lab=concretion] {} {{C} \bc \langle Q \rangle}
  \and
  \inferrule* [lab=process] {} {{P,Q} \bc M \;| \;P|Q \;|\; @{x}}
  \and
  \inferrule* [lab=name] {} {{x} \bc \quotep{P}}
\end{mathpar} 

Note that $\vec{x}$ (resp. $\vec{P}$) denotes a vector of names
(resp. processes) of length $|\vec{x}|$ (resp. $|\vec{P}|$). We adopt
the following useful abbreviations.

\begin{mathpar}
   x?(\vec{y}).P := x.(\vec{y})P \and  x\clift{\vec{P}} := x.\clift{\vec{P}}
   \and x!(y) := \lift{x}{\dropn{y}}
   \and \Pi_{i=0}^{n-1}P_i := P_0 | \ldots | P_{n-1}
\end{mathpar}

\subsubsection{Structural congruence}

\paragraph{Free and bound names and alpha-equivalence.} At the
core of structural equivalence is alpha-equivalence which identifies
process that are the same up to a change of variable. Formally, we
recognize the distinction between free and bound names. The free names
of a process, $\freenames{P}$, may be calculated recursively as
follows:

\begin{mathpar}
\freenames{\pzero} := \emptyset
  \and \\
  \freenames{x?(y).P} := \{ x \} \cup (\freenames{P} \setminus \{ y \})
  \and 
  \freenames{x!\langle P \rangle} := \{ x \} \cup \{ P \} 
  \and \\
  \freenames{P|Q} := \freenames{P} \cup \freenames{Q}
  \and \\
  \freenames{@{x}} := \{ x \}
\end{mathpar}

$\pi$
$\quotep{\pi}$

$\freenames{-} : \pi \to \mathcal{P}(\quotep{\pi})$

\begin{eqnarray*}
  \freenames{\pzero} & := & \emptyset \\
  \freenames{x?(y).P} & := & \{ x \} \cup (\freenames{P} \setminus \{ y \}) \\
  \freenames{x!\langle P \rangle} & := & \{ x \} \cup \{ P \} \\
  \freenames{P|Q} & := & \freenames{P} \cup \freenames{Q} \\
  \freenames{\dropn{x}} & := & \{ x \}
\end{eqnarray*}

The bound names of a process, $\boundnames{P}$, are those names occurring in $P$
that are not free. For example, in $x?(y).0$, the name $x$ is free, while $y$ is bound.

\begin{mathpar}
  \inferrule* [lab=monoidal-laws] {} { P|Q \equiv Q|P \and P|0 \equiv P \and P|(Q|R) \equiv (P|Q)|R }
\end{mathpar}

\begin{mathpar}
  \inferrule* [lab=alpha-equivalence] {} { (x)P \equiv (y)P\{y/x\} \and y \not\in \freenames{P} }
\end{mathpar}

\begin{definition}
Then two processes, $P,Q$, are alpha-equivalent if $P = Q\{\vec{y}/\vec{x}\}$ for
some $\vec{x} \in \boundnames{Q},\vec{y} \in \boundnames{P}$, where $Q\{\vec{y}/\vec{x}\}$
denotes the capture-avoiding substitution of $\vec{y}$ for $\vec{x}$ in $Q$.
\end{definition}

\begin{definition}
  The {\em structural congruence} \cite{SangiorgiWalker} , $\equiv$,
  between processes is the least congruence containing
  alpha-equivalence, satisfying the abelian monoid laws
  (associativity, commutativity and $\pzero$ as identity) for parallel
  composition $|$ and for summation $+$.
\end{definition}

\subsection{Name equivalence}

We take name equivalence, written $\nameeq$, to be the smallest
equivalence relation generated by the following rules.

\begin{mathpar}
\inferrule*[lab=Quote-drop]
{ }
{ \quotep{@{x}} \nameeq x }

\inferrule*[lab=Struct-equiv]
{ P \scong Q }
{ \quotep{P} \nameeq \quotep{Q} }
\end{mathpar}

The astute reader will have noticed that the mutual recursion of names
and processes imposes a mutual recursion on alpha-equivalence and
structural equivalence via name-equivalence. Fortunately, all of this
works out pleasantly and we may calculate in the natural way, free of
concern. The reader interested in the details is referred to the
appendix \ref{appendix:rho_details}.

\subsection{Substitution}

We use $\Proc$ for the set of processes, $\QProc$ for the set of
names, and $\id{\{}\vec{y} / \vec{x} \id{\}}$ to denote partial maps,
$s : \QProc \rightarrow \QProc$. A map, $s$ lifts, uniquely, to a map
on process terms, $\widehat{s} : \Proc \rightarrow \Proc$ by the
following equations.

\begin{mathpar}
  (0) \psubstp{Q}{P} := 0 \\
  (R \juxtap S) \psubstp{Q}{P}
  :=    
  (R)\psubstp{Q}{P} \juxtap (S) \psubstp{Q}{P} \\
  (x?(y).R) \psubstp{Q}{P}    
  :=    
  (x)\substp{Q}{P} (z)\concat( (R \psubstn{z}{y}) \psubstp{Q}{P} ) \\
  (\lift{x}{R}) \psubstp{Q}{P}  
  :=
  \lift{(x)\substp{Q}{P}}{ R \psubstp{Q}{P} } \\
%   (\dropn{x})  \psubstp{Q}{P}       
%   := 
%   \left\{ 
%     \begin{array}{ccc} 
%       \dropn{\quotep{Q}} & & x \nameeq \quotep{P} \\
%       \dropn{x} & & otherwise \\
%     \end{array}
%   \right. 
  (\dropn{x})  \psubstp{Q}{P}       
  := 
  \left\{ 
    \begin{array}{ccc} 
      Q & & x \nameeq \quotep{P} \\
      \dropn{x} & & otherwise \\
    \end{array}
  \right.
\end{mathpar}
 

where

\begin{eqnarray}
  (x)\id{\{} \lpquote Q \rpquote / \lpquote P \rpquote \id{\}}            = 
  \left\{ 
    \begin{array}{ccc}
      \lpquote Q \rpquote & & x \nameeq \lpquote P \rpquote \\
      x & & otherwise \\
    \end{array}
  \right. \nonumber
\end{eqnarray}

and $z$ is chosen distinct from $\quotep{P}$, $\quotep{Q}$, the free
names in $Q$, and all the names in $R$. Our $\alpha$-equivalence will
be built in the standard way from this substitution.

\begin{remark}\label{rem:no_self_referential_names}
  One consequence of these definitions is that $\forall P. \quotep{P}
  \not\in \freenames{P}$.
\end{remark}

\subsection{ Dynamic quote: an example }

Anticipating something of what's to come, consider applying the
substitution, $\widehat{\id{\{}u / z \id{\}}}$, to the following pair
of processes, $\lift{w}{y!(z)}$ and $w[ \lpquote y!(z) \rpquote ]$.

\begin{eqnarray}
	\lift{w}{y!(z)}\widehat{\id{\{}u / z \id{\}}}
		& = &
		\lift{w}{y!(u)} \nonumber\\
	w[ \lpquote y!(z) \rpquote ] \widehat{ \id{\{}u / z \id{\}} }
		& = &
		w[ \lpquote y!(z) \rpquote ] \nonumber
\end{eqnarray}

Because the body of the process between quotes is impervious to
substitution, we get radically different answers. In fact, by
examining the first process in an input context,
e.g. $x?(z).\lift{w}{y!(z)}$, we see that the process under the lift
operator may be shaped by prefixed inputs binding a name inside it. In
this sense, the lift operator will be seen as a way to dynamically
construct processes before reifying them as names.

Finally equipped with these standard features we can present the
dynamics of the calculus.

\subsubsection{Operational semantics} 

Finally, we introduce the computational dynamics. What marks these
algebras as distinct from other more traditionally studied algebraic
structures, e.g. vector spaces or polynomial rings, is the manner in
which dynamics is captured. In traditional structures, dynamics is typically
expressed through morphisms between such structures, as in linear maps
between vector spaces or morphisms between rings. In algebras
associated with the semantics of computation, the dynamics is
expressed as part of the algebraic structure itself, through a
reduction reduction relation typically denoted by $\red$. Below, we
give a recursive presentation of this relation for the calculus used
in the encoding.

$\red \subseteq \pi \times \pi$
$\red : \pi \to \mathcal{P}(\pi)$

\begin{mathpar}
  \inferrule* [lab=Comm] { \textsf{match}( x_{src}, x_{trgt} ) } { x_{trgt}?(y)P \; | \; x_{src}!\langle {Q} \rangle \red P\{\quotep{Q}/y}\} }
  \and \\
  \inferrule* [lab=Par] {{P} \red {P}'} {{{P} | {Q}} \red {{P}' | {Q}}}
  \and
  \inferrule* [lab=Equiv]{{{P} \scong {P}'} \andalso {{P}' \red {Q}'} \andalso {{Q}' \scong {Q}}}{{P} \red {Q}}
\end{mathpar}

\begin{eqnarray*}
  match_{\equiv} (\quotep{P},\quotep{Q}) & := & P \equiv Q \\
  match_{\dagger}(\quotep{P},\quotep{Q}) & := & \forall R. P|Q \red^{*} R => R \red^{*} 0 \\
  match_{K}(\quotep{P},\quotep{Q}) & := & K \mbox{ for some context } K
\end{eqnarray*}

$u?(x)P | u!\langle Q \rangle \red P\{\quotep{Q}/x\}$

%We write $\wred$ for $\red^*$, and $P\red$ if $\exists Q $ such that $ P \red Q$.
We write $P\red$ if $\exists Q $ such that $ P \red Q$ and $P\not\red$, otherwise.

\section{Replication}

As mentioned before, it is known that replication (and hence
recursion) can be implemented in a higher-order process algebra
\cite{SangiorgiWalker}. As our first example of calculation with the
machinery thus far presented we give the construction explicitly in
the {\rhoc}.

\begin{eqnarray}
	D_{x} & := & \prefix{x}{y}{(\binpar{\outputp{x}{y}}{@{y}})} \nonumber\\
	\bangp_{x}{P} & := & \binpar{{x}!\langle{\binpar{D_{x}}{P}}\rangle}{D_{x}} \nonumber
\end{eqnarray}

\begin{eqnarray}
	\bangp_{x}{P} & & \nonumber\\
	=
	& {x}!\langle{(\prefix{x}{y}{(\outputp{x}{y} | @{y})) | P}}\rangle 
	      | \prefix{x}{y}{(\outputp{x}{y} | @{y})} & \nonumber\\
	\red
	& (\outputp{x}{y} | @{y})\substn{\quotep{(\prefix{x}{y}{(@{y} | \outputp{x}{y})) | P}}}{y} & \nonumber\\
	=
	& \outputp{x}{\quotep{(\prefix{x}{y}{(\outputp{x}{y} | @{y})) | P}}}
	  | {(\prefix{x}{y}{(\outputp{x}{y} | @{y})) | P}} & \nonumber\\
	\red
	& \ldots & \nonumber\\
	\red^*
	& P | P | \ldots & \nonumber
\end{eqnarray}

Of course, this encoding, as an implementation, runs away, unfolding
$\bangp{P}$ eagerly. A lazier and more implementable replication
operator, restricted to input-guarded processes, may be obtained as follows.

\begin{eqnarray}
\bangp{\prefix{u}{v}{P}} 
	:= 
	\binpar{\lift{x}{\prefix{u}{v}{(\binpar{D(x)}{P})}}}{D(x)} \nonumber
\end{eqnarray}

\begin{remark}
  Note that the lazier definition still does not deal with summation
  or mixed summation (i.e. sums over input and output). The reader is
  invited to construct definitions of replication that deal with these
  features. 

  Further, the definitions are parameterized in a name, $x$. Can you,
  gentle reader, make a definition that eliminates this parameter and
  guarantees no accidental interaction between the replication
  machinery and the process being replicated -- i.e. no accidental
  sharing of names used by the process to get its work done and the
  name(s) used by the replication to effect copying. This latter
  revision of the definition of replication is crucial to obtaining
  the expected identity $!!P \sim !P$.
\end{remark}

\begin{remark}\label{rem:paradoxical_combinator}
  The reader familiar with the lambda calculus will have noticed the
  similarity between $D$ and the paradoxical combinator.

  [Ed. note: the existence of this seems to suggest we have to be more
  restrictive on the set of processes and names we admit if we are to
  support no-cloning.]
\end{remark}

\subsubsection{Bisimulation}

The computational dynamics gives rise to another kind of equivalence,
the equivalence of computational behavior. As previously mentioned
this is typically captured \emph{via} some form of bisimulation.

% The notion we use in this paper is weak barbed bisimulation
% \cite{milner91polyadicpi}.

The notion we use in this paper is derived from weak barbed
bisimulation \cite{milner91polyadicpi}. 

\begin{definition}
An \emph{observation relation}, $\downarrow_{\mathcal N}$, over a set
of names, $\mathcal N$, is the smallest relation satisfying the rules
below.

\infrule[Out-barb]{y \in {\mathcal N}, \; x \nameeq y}
		  {\outputp{x}{v} \downarrow_{\mathcal N} x}
\infrule[Par-barb]{\mbox{$P\downarrow_{\mathcal N} x$ or $Q\downarrow_{\mathcal N} x$}}
		  {\binpar{P}{Q} \downarrow_{\mathcal N} x}

We write $P \Downarrow_{\mathcal N} x$ if there is $Q$ such that 
$P \wred Q$ and $Q \downarrow_{\mathcal N} x$.
\end{definition}

\begin{definition}
%\label{def.bbisim}
An  ${\mathcal N}$-\emph{barbed bisimulation} over a set of names, ${\mathcal N}$, is a symmetric binary relation 
${\mathcal S}_{\mathcal N}$ between agents such that $P\rel{S}_{\mathcal N}Q$ implies:
\begin{enumerate}
\item If $P \red P'$ then $Q \wred Q'$ and $P'\rel{S}_{\mathcal N} Q'$.
\item If $P\downarrow_{\mathcal N} x$, then $Q\Downarrow_{\mathcal N} x$.
\end{enumerate}
$P$ is ${\mathcal N}$-barbed bisimilar to $Q$, written
$P \wbbisim_{\mathcal N} Q$, if $P \rel{S}_{\mathcal N} Q$ for some ${\mathcal N}$-barbed bisimulation ${\mathcal S}_{\mathcal N}$.
\end{definition}

$\mathcal{R} \subseteq \pi \times \pi$

$P \mathcal{R} Q => \forall P'. P \red P' \Rightarrow \exists Q'. Q \red Q', P' \mathcal{R} Q'$

$P \vdash x \Rightarrow Q \vdash x$

\begin{mathpar}
  \inferrule*[lab=Out-barb]{x \nameeq y}{{y}!\langle{Q}\rangle \vdash x}
  \and
  \inferrule*[lab=Par-barb]{\mbox{$P\vdash x$ or $Q\vdash x$}}{\binpar{P}{Q} \vdash x}
\end{mathpar}

\subsubsection{Contexts}

One of the principle advantages of computational calculi like the
$\pi$-calculus is a well-defined notion of context,
contextual-equivalence and a correlation between
contextual-equivalence and notions of bisimulation. The notion of
context allows the decomposition of a process into (sub-)process and
its syntactic environment, its context. Thus, a context may be
thought of as a process with a ``hole'' (written $\Box$) in it. The
application of a context $M$ to a process $P$, written $M[P]$, is
tantamount to filling the hole in $M$ with $P$. In this paper we do
not need the full weight of this theory, but do make use of the notion
of context in the proof the main theorem. 

\begin{mathpar}
  \inferrule* [lab=summation] {} {{M_{M},M_{N}} \bc \Box \;|\; x.M_{A} \;|\; M_{M}+M_{N}}
  \and
  \inferrule* [lab=agent] {} {{M_{A}} \bc (\vec{x})M_{P} \;| \; \clift{P_0,\ldots,M_{P},\ldots,P_N}}
  \and \\
  \inferrule* [lab=process] {} {{M_{P}} \bc M_{N} \;| \;P|M_{P} }
\end{mathpar} 

\begin{mathpar}
  \inferrule* [lab=sychronization] {} {M_{N} \bc \Box \;|\; x?M_{F} \;|\; x!M_{C}}
  \and
  \inferrule* [lab=abstraction] {} {{M_{F}} \bc (x)M_{P} }
  \and
  \inferrule* [lab=concretion] {} {{M_{C}} \bc \langle M_{P} \rangle }
  \and \\
  \inferrule* [lab=process] {} {{M_{P}} \bc M_{N} \;| \;P|M_{P} }
\end{mathpar}

\begin{definition}[contextual application] Given a context $M$, and
  process $P$, we define the \emph{contextual application}, $M[P] :=
  M\{P/\Box\}$. That is, the contextual application of M to P is the
  substitution of $P$ for $\Box$ in $M$.
\end{definition}

$\meaningof{-} : L \to \mathcal{P}(\pi)$

\begin{mathpar}
  \inferrule* [lab=collection] {} {\meaningof{true} = \pi, \and \meaningof{~E} = \pi \setminus \meaningof{E}, \and \meaningof{E_{1} \& E_{2}} = \meaningof{E_{1}} \cap \meaningof{E_{2}}}
\end{mathpar}

\begin{mathpar}
  \inferrule* [lab=structure] {} {\meaningof{0} = \{ P \in \pi | P \equiv 0 \}, \and \\ \meaningof{E_1 | E_2} = \{ P \in \pi | P \equiv P_{1} | P_{2}, P_{1} \in \meaningof{E_{1}}, P_{2} \in \meaningof{E_2}\} }
\end{mathpar}

\begin{mathpar}
 \inferrule* [lab=behavior] {} {\meaningof{\langle a?b \rangle E} = \{ P \in \pi | P \equiv Q | u?(y)P', \\ \and \\\\ \and \\ \;\;\; u \in \meaningof{a}, \forall z.P'\{z/y\} \in \meaningof{E\{z/b\}}\}, \and \\ \meaningof{a!E} = \{ P \in \pi | P \equiv Q | x!\langle P' \rangle, x \in \meaningof{a} P' \in \meaningof{E}\} }
\end{mathpar}

\begin{mathpar}
 \inferrule* [lab=nominal] {} {\meaningof{\quotep{E}} = \{ \quotep{P} \in \quotep{\pi} | P \in \meaningof{E} \}, \and \meaningof{\quotep{P}} = \{ \quotep{Q} \in \quotep{\pi} | P \equiv Q \} \and \\ \meaningof{@\quotep{E}} = \{ P \in \pi | P \equiv @x, x \in \meaningof{E} \}}
\end{mathpar}

\begin{eqnarray*}
  \\
  \meaningof{-} : TS \to ST
\end{eqnarray*}

\begin{eqnarray*}
  \\
  L : TS \to ST
\end{eqnarray*}

\begin{eqnarray*}
  \\
  P \models E \iff P \in \meaningof{E}
\end{eqnarray*}

\begin{eqnarray*}
  P \approx_{L} Q \iff \forall E \in L. P \models E \iff Q \models E
\end{eqnarray*}

\begin{eqnarray*}
  P \approx_{K} Q
\end{eqnarray*}

\begin{eqnarray*}
  P \approx Q
\end{eqnarray*}

$\approx_{K} = \approx = \approx_{L}$

\subsubsection{Contextual duality}

Note that contexts extend the quotation operation to a family of
operations from processes to names. Given a context, $M$, we can
define a \emph{nominal context}, $\quotep{M}$ by $\quotep{M}[P] :=
\quotep{M[P]}$. To foreshadow what is to come we observe that these
operations enjoy a duality with processes very much like the duality
between vectors and maps from vectors to scalars.

Further, because the calculus is essentially higher-order, we have a
correspondence between contexts and processes. More specifically,
given a name $x$ and a context $M$ we can construct $M^{*}_{x}$ such
that 

\begin{mathpar}
  M^{*}_{x} | \lift{x}{P} \red M[P]
\end{mathpar}

namely,

\begin{mathpar}
  M^{*}_{x} := x?(u).M[\dropn{u}]
\end{mathpar}

The dependence of $M^{*}_{x}$ on a name makes it an abstraction, 

\begin{mathpar}
  M^{*} := (x)x?(u).M[\dropn{u}]
\end{mathpar}

\subsection{Additional notation}

It will sometimes be convenient to denote the process a name
quotes. We already have the notation $x = \quotep{P}$, but it will be
convenient to introduce an alternate notation, $\procn{x}$, when we
want to emphasize the connection to the use of the name. Note that, by
virtue of name equivalence, $\quotep{\procn{x}} \nameeq x$; so, the
notation is consistent with previous definitions.

Further, because names have structure it is possible to effect
substitutions on the basis of that structure. This means we need to
upgrade our notation for substitutions, which we accomplish by
adapting comprehension notation. Thus,

\begin{mathpar}
  P\{ y / x : x \in S \}
\end{mathpar}

is interpreted to mean the process derived from P by replacing (in a
capture-avoiding manner) each occurrence of $x$ in $S$ by $y$. For example,

\begin{mathpar}
  P\{ \quotep{\procn{x}|\procn{x}} / x : x \in \freenames{P} \}
\end{mathpar}

will replace each (occurrence) of a free name $x$ in $P$ by
$\quotep{\procn{x}|\procn{x}}$.

Also, we will avail ourselves of the notation $x^{L}$ and $x^{R}$ to
denote injections of a name into disjoint copies of the name
space. There are numerous ways to accomplish this. One example can be
found in \cite{MeredithR05}. This notation overloads to vectors of
names: $\vec{x}^{\pi} := (x_{i}^{\pi} \; : \; 0 \leq i < |\vec{x}| )$ where $\pi \in \{L,R\}$.

We also use $P^{\Box} := P|\Box$.

In \cite{MeredithR05} an interpretation of the new operator is
given. It turns out that there are several possible interpretations
all enjoying the requisite algebraic properties of the operator (see
\cite{milner91polyadicpi}). We will therefore make liberal use of
$(\nu\; \vec{x})P$.

% subsection the_syntax_and_semantics_of_the_notation_system (end)   

\input{qm2pi.qmops} 

\input{qm2pi.sterngerlach} 

\input{qm2pi.metric} 

% section concurrent_process_calculi (end)

%\input{qm2pi.proofsketch}

% section proof sketch (end)

%\input{qm2pi.slviaknots} 

% section spatial logic via knots (end)

\input{qm2pi.conclusion}

% section conclusion (end)

%\input{qm2pi.dtcodes} 

% section wiring algorithm (end)

\input{qm2pi.ack} 

% section acknowledgments (end)

\newpage


\bibliographystyle{plain}   
\bibliography{../../biblios/main.bib}

\input{qm2pi.rhodetails}

\end{document}

 

% section concurrent_process_calculi (end)

%\documentclass[12pt]{llncs}
%\documentclass{jktr}

\usepackage[pdftex]{hyperref}                   
\usepackage {listings}
\usepackage {mathpartir}
\usepackage{bcprules}
%\usepackage{listings}
                       
\usepackage{graphicx} 
%\usepackage[margins=2.5cm,nohead,nofoot]{geometry}
%\usepackage{geometry}
\usepackage{amsfonts}
\usepackage{amstext}
\usepackage{latexsym}
\usepackage{amssymb}
\usepackage{color}


%\include{myPreamble}
\include{qm2pi.local} 

%\ifpdf
%\usepackage[pdftex]{graphicx}
%\else
%\usepackage{graphicx}
%\fi

 % \ifpdf
%  \usepackage{pdfsync}
%  \if


%\title{Brief Article}
%\author{David F. Snyder}
%\author{L.G. Meredith}

%\address{Dept. of Math., Texas State University--San Marcos, San Marcos, TX 78666}
       
\pagestyle{empty}


\begin{document}

\lstset{language=[Objective]Caml,frame=shadowbox}

\input{qm2pi.front}

% section front matter (end)

\input{qm2pi.intro} 
 
% section introduction (end)

% \input{qm2pi.knotations} 

% section notation (end)

\input{qm2pi.process.calculi} 

% section concurrent_process_calculi_and_spatial_logics_ (end)
    
%\input{qm2pi.knots2pi} 

%\input{qm2pi.trefoil} 

%\input{qm2pi.mainthm} 

% subsection basic_interpretation (end)

%\input{qm2pi.rho.presentation} 
\subsection{The syntax and semantics of the notation system}\label{sub:the_syntax_and_semantics_of_the_notation_system} % (fold)

We now summarize a technical presentation of the calculus that
embodies our theory of dynamics. The typical presentation of such a
calculus follows the style of giving generators and relations on
them. The grammar, below, describing term constructors, freely
generates the set of processes, $\Proc$. This set is then quotiented
by a relation known as structural congruence and it is over this set
that the notion of dynamics is expressed. This presentation is
essentially that of \cite{MeredithR05} with the addition of
polyadicity and summation. For readability we have relegated some of
the technical subtleties to an appendix.

\subsubsection{Process grammar}\label{subsub:process_grammar}

\begin{mathpar}
  \inferrule* [lab=synchronization] {} {{M} \bc \pzero \;|\; x?F \;|\; x!C }
  \and
  \inferrule* [lab=abstraction] {} {{F} \bc (x)P}
  \and
  \inferrule* [lab=concretion] {} {{C} \bc \langle Q \rangle}
  \and
  \inferrule* [lab=process] {} {{P,Q} \bc M \;| \;P|Q \;|\; @{x}}
  \and
  \inferrule* [lab=name] {} {{x} \bc \quotep{P}}
\end{mathpar} 

Note that $\vec{x}$ (resp. $\vec{P}$) denotes a vector of names
(resp. processes) of length $|\vec{x}|$ (resp. $|\vec{P}|$). We adopt
the following useful abbreviations.

\begin{mathpar}
   x?(\vec{y}).P := x.(\vec{y})P \and  x\clift{\vec{P}} := x.\clift{\vec{P}}
   \and x!(y) := \lift{x}{\dropn{y}}
   \and \Pi_{i=0}^{n-1}P_i := P_0 | \ldots | P_{n-1}
\end{mathpar}

\subsubsection{Structural congruence}

\paragraph{Free and bound names and alpha-equivalence.} At the
core of structural equivalence is alpha-equivalence which identifies
process that are the same up to a change of variable. Formally, we
recognize the distinction between free and bound names. The free names
of a process, $\freenames{P}$, may be calculated recursively as
follows:

\begin{mathpar}
\freenames{\pzero} := \emptyset
  \and \\
  \freenames{x?(y).P} := \{ x \} \cup (\freenames{P} \setminus \{ y \})
  \and 
  \freenames{x!\langle P \rangle} := \{ x \} \cup \{ P \} 
  \and \\
  \freenames{P|Q} := \freenames{P} \cup \freenames{Q}
  \and \\
  \freenames{@{x}} := \{ x \}
\end{mathpar}

$\pi$
$\quotep{\pi}$

$\freenames{-} : \pi \to \mathcal{P}(\quotep{\pi})$

\begin{eqnarray*}
  \freenames{\pzero} & := & \emptyset \\
  \freenames{x?(y).P} & := & \{ x \} \cup (\freenames{P} \setminus \{ y \}) \\
  \freenames{x!\langle P \rangle} & := & \{ x \} \cup \{ P \} \\
  \freenames{P|Q} & := & \freenames{P} \cup \freenames{Q} \\
  \freenames{\dropn{x}} & := & \{ x \}
\end{eqnarray*}

The bound names of a process, $\boundnames{P}$, are those names occurring in $P$
that are not free. For example, in $x?(y).0$, the name $x$ is free, while $y$ is bound.

\begin{mathpar}
  \inferrule* [lab=monoidal-laws] {} { P|Q \equiv Q|P \and P|0 \equiv P \and P|(Q|R) \equiv (P|Q)|R }
\end{mathpar}

\begin{mathpar}
  \inferrule* [lab=alpha-equivalence] {} { (x)P \equiv (y)P\{y/x\} \and y \not\in \freenames{P} }
\end{mathpar}

\begin{definition}
Then two processes, $P,Q$, are alpha-equivalent if $P = Q\{\vec{y}/\vec{x}\}$ for
some $\vec{x} \in \boundnames{Q},\vec{y} \in \boundnames{P}$, where $Q\{\vec{y}/\vec{x}\}$
denotes the capture-avoiding substitution of $\vec{y}$ for $\vec{x}$ in $Q$.
\end{definition}

\begin{definition}
  The {\em structural congruence} \cite{SangiorgiWalker} , $\equiv$,
  between processes is the least congruence containing
  alpha-equivalence, satisfying the abelian monoid laws
  (associativity, commutativity and $\pzero$ as identity) for parallel
  composition $|$ and for summation $+$.
\end{definition}

\subsection{Name equivalence}

We take name equivalence, written $\nameeq$, to be the smallest
equivalence relation generated by the following rules.

\begin{mathpar}
\inferrule*[lab=Quote-drop]
{ }
{ \quotep{@{x}} \nameeq x }

\inferrule*[lab=Struct-equiv]
{ P \scong Q }
{ \quotep{P} \nameeq \quotep{Q} }
\end{mathpar}

The astute reader will have noticed that the mutual recursion of names
and processes imposes a mutual recursion on alpha-equivalence and
structural equivalence via name-equivalence. Fortunately, all of this
works out pleasantly and we may calculate in the natural way, free of
concern. The reader interested in the details is referred to the
appendix \ref{appendix:rho_details}.

\subsection{Substitution}

We use $\Proc$ for the set of processes, $\QProc$ for the set of
names, and $\id{\{}\vec{y} / \vec{x} \id{\}}$ to denote partial maps,
$s : \QProc \rightarrow \QProc$. A map, $s$ lifts, uniquely, to a map
on process terms, $\widehat{s} : \Proc \rightarrow \Proc$ by the
following equations.

\begin{mathpar}
  (0) \psubstp{Q}{P} := 0 \\
  (R \juxtap S) \psubstp{Q}{P}
  :=    
  (R)\psubstp{Q}{P} \juxtap (S) \psubstp{Q}{P} \\
  (x?(y).R) \psubstp{Q}{P}    
  :=    
  (x)\substp{Q}{P} (z)\concat( (R \psubstn{z}{y}) \psubstp{Q}{P} ) \\
  (\lift{x}{R}) \psubstp{Q}{P}  
  :=
  \lift{(x)\substp{Q}{P}}{ R \psubstp{Q}{P} } \\
%   (\dropn{x})  \psubstp{Q}{P}       
%   := 
%   \left\{ 
%     \begin{array}{ccc} 
%       \dropn{\quotep{Q}} & & x \nameeq \quotep{P} \\
%       \dropn{x} & & otherwise \\
%     \end{array}
%   \right. 
  (\dropn{x})  \psubstp{Q}{P}       
  := 
  \left\{ 
    \begin{array}{ccc} 
      Q & & x \nameeq \quotep{P} \\
      \dropn{x} & & otherwise \\
    \end{array}
  \right.
\end{mathpar}
 

where

\begin{eqnarray}
  (x)\id{\{} \lpquote Q \rpquote / \lpquote P \rpquote \id{\}}            = 
  \left\{ 
    \begin{array}{ccc}
      \lpquote Q \rpquote & & x \nameeq \lpquote P \rpquote \\
      x & & otherwise \\
    \end{array}
  \right. \nonumber
\end{eqnarray}

and $z$ is chosen distinct from $\quotep{P}$, $\quotep{Q}$, the free
names in $Q$, and all the names in $R$. Our $\alpha$-equivalence will
be built in the standard way from this substitution.

\begin{remark}\label{rem:no_self_referential_names}
  One consequence of these definitions is that $\forall P. \quotep{P}
  \not\in \freenames{P}$.
\end{remark}

\subsection{ Dynamic quote: an example }

Anticipating something of what's to come, consider applying the
substitution, $\widehat{\id{\{}u / z \id{\}}}$, to the following pair
of processes, $\lift{w}{y!(z)}$ and $w[ \lpquote y!(z) \rpquote ]$.

\begin{eqnarray}
	\lift{w}{y!(z)}\widehat{\id{\{}u / z \id{\}}}
		& = &
		\lift{w}{y!(u)} \nonumber\\
	w[ \lpquote y!(z) \rpquote ] \widehat{ \id{\{}u / z \id{\}} }
		& = &
		w[ \lpquote y!(z) \rpquote ] \nonumber
\end{eqnarray}

Because the body of the process between quotes is impervious to
substitution, we get radically different answers. In fact, by
examining the first process in an input context,
e.g. $x?(z).\lift{w}{y!(z)}$, we see that the process under the lift
operator may be shaped by prefixed inputs binding a name inside it. In
this sense, the lift operator will be seen as a way to dynamically
construct processes before reifying them as names.

Finally equipped with these standard features we can present the
dynamics of the calculus.

\subsubsection{Operational semantics} 

Finally, we introduce the computational dynamics. What marks these
algebras as distinct from other more traditionally studied algebraic
structures, e.g. vector spaces or polynomial rings, is the manner in
which dynamics is captured. In traditional structures, dynamics is typically
expressed through morphisms between such structures, as in linear maps
between vector spaces or morphisms between rings. In algebras
associated with the semantics of computation, the dynamics is
expressed as part of the algebraic structure itself, through a
reduction reduction relation typically denoted by $\red$. Below, we
give a recursive presentation of this relation for the calculus used
in the encoding.

$\red \subseteq \pi \times \pi$
$\red : \pi \to \mathcal{P}(\pi)$

\begin{mathpar}
  \inferrule* [lab=Comm] { \textsf{match}( x_{src}, x_{trgt} ) } { x_{trgt}?(y)P \; | \; x_{src}!\langle {Q} \rangle \red P\{\quotep{Q}/y}\} }
  \and \\
  \inferrule* [lab=Par] {{P} \red {P}'} {{{P} | {Q}} \red {{P}' | {Q}}}
  \and
  \inferrule* [lab=Equiv]{{{P} \scong {P}'} \andalso {{P}' \red {Q}'} \andalso {{Q}' \scong {Q}}}{{P} \red {Q}}
\end{mathpar}

\begin{eqnarray*}
  match_{\equiv} (\quotep{P},\quotep{Q}) & := & P \equiv Q \\
  match_{\dagger}(\quotep{P},\quotep{Q}) & := & \forall R. P|Q \red^{*} R => R \red^{*} 0 \\
  match_{K}(\quotep{P},\quotep{Q}) & := & K \mbox{ for some context } K
\end{eqnarray*}

$u?(x)P | u!\langle Q \rangle \red P\{\quotep{Q}/x\}$

%We write $\wred$ for $\red^*$, and $P\red$ if $\exists Q $ such that $ P \red Q$.
We write $P\red$ if $\exists Q $ such that $ P \red Q$ and $P\not\red$, otherwise.

\section{Replication}

As mentioned before, it is known that replication (and hence
recursion) can be implemented in a higher-order process algebra
\cite{SangiorgiWalker}. As our first example of calculation with the
machinery thus far presented we give the construction explicitly in
the {\rhoc}.

\begin{eqnarray}
	D_{x} & := & \prefix{x}{y}{(\binpar{\outputp{x}{y}}{@{y}})} \nonumber\\
	\bangp_{x}{P} & := & \binpar{{x}!\langle{\binpar{D_{x}}{P}}\rangle}{D_{x}} \nonumber
\end{eqnarray}

\begin{eqnarray}
	\bangp_{x}{P} & & \nonumber\\
	=
	& {x}!\langle{(\prefix{x}{y}{(\outputp{x}{y} | @{y})) | P}}\rangle 
	      | \prefix{x}{y}{(\outputp{x}{y} | @{y})} & \nonumber\\
	\red
	& (\outputp{x}{y} | @{y})\substn{\quotep{(\prefix{x}{y}{(@{y} | \outputp{x}{y})) | P}}}{y} & \nonumber\\
	=
	& \outputp{x}{\quotep{(\prefix{x}{y}{(\outputp{x}{y} | @{y})) | P}}}
	  | {(\prefix{x}{y}{(\outputp{x}{y} | @{y})) | P}} & \nonumber\\
	\red
	& \ldots & \nonumber\\
	\red^*
	& P | P | \ldots & \nonumber
\end{eqnarray}

Of course, this encoding, as an implementation, runs away, unfolding
$\bangp{P}$ eagerly. A lazier and more implementable replication
operator, restricted to input-guarded processes, may be obtained as follows.

\begin{eqnarray}
\bangp{\prefix{u}{v}{P}} 
	:= 
	\binpar{\lift{x}{\prefix{u}{v}{(\binpar{D(x)}{P})}}}{D(x)} \nonumber
\end{eqnarray}

\begin{remark}
  Note that the lazier definition still does not deal with summation
  or mixed summation (i.e. sums over input and output). The reader is
  invited to construct definitions of replication that deal with these
  features. 

  Further, the definitions are parameterized in a name, $x$. Can you,
  gentle reader, make a definition that eliminates this parameter and
  guarantees no accidental interaction between the replication
  machinery and the process being replicated -- i.e. no accidental
  sharing of names used by the process to get its work done and the
  name(s) used by the replication to effect copying. This latter
  revision of the definition of replication is crucial to obtaining
  the expected identity $!!P \sim !P$.
\end{remark}

\begin{remark}\label{rem:paradoxical_combinator}
  The reader familiar with the lambda calculus will have noticed the
  similarity between $D$ and the paradoxical combinator.

  [Ed. note: the existence of this seems to suggest we have to be more
  restrictive on the set of processes and names we admit if we are to
  support no-cloning.]
\end{remark}

\subsubsection{Bisimulation}

The computational dynamics gives rise to another kind of equivalence,
the equivalence of computational behavior. As previously mentioned
this is typically captured \emph{via} some form of bisimulation.

% The notion we use in this paper is weak barbed bisimulation
% \cite{milner91polyadicpi}.

The notion we use in this paper is derived from weak barbed
bisimulation \cite{milner91polyadicpi}. 

\begin{definition}
An \emph{observation relation}, $\downarrow_{\mathcal N}$, over a set
of names, $\mathcal N$, is the smallest relation satisfying the rules
below.

\infrule[Out-barb]{y \in {\mathcal N}, \; x \nameeq y}
		  {\outputp{x}{v} \downarrow_{\mathcal N} x}
\infrule[Par-barb]{\mbox{$P\downarrow_{\mathcal N} x$ or $Q\downarrow_{\mathcal N} x$}}
		  {\binpar{P}{Q} \downarrow_{\mathcal N} x}

We write $P \Downarrow_{\mathcal N} x$ if there is $Q$ such that 
$P \wred Q$ and $Q \downarrow_{\mathcal N} x$.
\end{definition}

\begin{definition}
%\label{def.bbisim}
An  ${\mathcal N}$-\emph{barbed bisimulation} over a set of names, ${\mathcal N}$, is a symmetric binary relation 
${\mathcal S}_{\mathcal N}$ between agents such that $P\rel{S}_{\mathcal N}Q$ implies:
\begin{enumerate}
\item If $P \red P'$ then $Q \wred Q'$ and $P'\rel{S}_{\mathcal N} Q'$.
\item If $P\downarrow_{\mathcal N} x$, then $Q\Downarrow_{\mathcal N} x$.
\end{enumerate}
$P$ is ${\mathcal N}$-barbed bisimilar to $Q$, written
$P \wbbisim_{\mathcal N} Q$, if $P \rel{S}_{\mathcal N} Q$ for some ${\mathcal N}$-barbed bisimulation ${\mathcal S}_{\mathcal N}$.
\end{definition}

$\mathcal{R} \subseteq \pi \times \pi$

$P \mathcal{R} Q => \forall P'. P \red P' \Rightarrow \exists Q'. Q \red Q', P' \mathcal{R} Q'$

$P \vdash x \Rightarrow Q \vdash x$

\begin{mathpar}
  \inferrule*[lab=Out-barb]{x \nameeq y}{{y}!\langle{Q}\rangle \vdash x}
  \and
  \inferrule*[lab=Par-barb]{\mbox{$P\vdash x$ or $Q\vdash x$}}{\binpar{P}{Q} \vdash x}
\end{mathpar}

\subsubsection{Contexts}

One of the principle advantages of computational calculi like the
$\pi$-calculus is a well-defined notion of context,
contextual-equivalence and a correlation between
contextual-equivalence and notions of bisimulation. The notion of
context allows the decomposition of a process into (sub-)process and
its syntactic environment, its context. Thus, a context may be
thought of as a process with a ``hole'' (written $\Box$) in it. The
application of a context $M$ to a process $P$, written $M[P]$, is
tantamount to filling the hole in $M$ with $P$. In this paper we do
not need the full weight of this theory, but do make use of the notion
of context in the proof the main theorem. 

\begin{mathpar}
  \inferrule* [lab=summation] {} {{M_{M},M_{N}} \bc \Box \;|\; x.M_{A} \;|\; M_{M}+M_{N}}
  \and
  \inferrule* [lab=agent] {} {{M_{A}} \bc (\vec{x})M_{P} \;| \; \clift{P_0,\ldots,M_{P},\ldots,P_N}}
  \and \\
  \inferrule* [lab=process] {} {{M_{P}} \bc M_{N} \;| \;P|M_{P} }
\end{mathpar} 

\begin{mathpar}
  \inferrule* [lab=sychronization] {} {M_{N} \bc \Box \;|\; x?M_{F} \;|\; x!M_{C}}
  \and
  \inferrule* [lab=abstraction] {} {{M_{F}} \bc (x)M_{P} }
  \and
  \inferrule* [lab=concretion] {} {{M_{C}} \bc \langle M_{P} \rangle }
  \and \\
  \inferrule* [lab=process] {} {{M_{P}} \bc M_{N} \;| \;P|M_{P} }
\end{mathpar}

\begin{definition}[contextual application] Given a context $M$, and
  process $P$, we define the \emph{contextual application}, $M[P] :=
  M\{P/\Box\}$. That is, the contextual application of M to P is the
  substitution of $P$ for $\Box$ in $M$.
\end{definition}

$\meaningof{-} : L \to \mathcal{P}(\pi)$

\begin{mathpar}
  \inferrule* [lab=collection] {} {\meaningof{true} = \pi, \and \meaningof{~E} = \pi \setminus \meaningof{E}, \and \meaningof{E_{1} \& E_{2}} = \meaningof{E_{1}} \cap \meaningof{E_{2}}}
\end{mathpar}

\begin{mathpar}
  \inferrule* [lab=structure] {} {\meaningof{0} = \{ P \in \pi | P \equiv 0 \}, \and \\ \meaningof{E_1 | E_2} = \{ P \in \pi | P \equiv P_{1} | P_{2}, P_{1} \in \meaningof{E_{1}}, P_{2} \in \meaningof{E_2}\} }
\end{mathpar}

\begin{mathpar}
 \inferrule* [lab=behavior] {} {\meaningof{\langle a?b \rangle E} = \{ P \in \pi | P \equiv Q | u?(y)P', \\ \and \\\\ \and \\ \;\;\; u \in \meaningof{a}, \forall z.P'\{z/y\} \in \meaningof{E\{z/b\}}\}, \and \\ \meaningof{a!E} = \{ P \in \pi | P \equiv Q | x!\langle P' \rangle, x \in \meaningof{a} P' \in \meaningof{E}\} }
\end{mathpar}

\begin{mathpar}
 \inferrule* [lab=nominal] {} {\meaningof{\quotep{E}} = \{ \quotep{P} \in \quotep{\pi} | P \in \meaningof{E} \}, \and \meaningof{\quotep{P}} = \{ \quotep{Q} \in \quotep{\pi} | P \equiv Q \} \and \\ \meaningof{@\quotep{E}} = \{ P \in \pi | P \equiv @x, x \in \meaningof{E} \}}
\end{mathpar}

\begin{eqnarray*}
  \\
  \meaningof{-} : TS \to ST
\end{eqnarray*}

\begin{eqnarray*}
  \\
  L : TS \to ST
\end{eqnarray*}

\begin{eqnarray*}
  \\
  P \models E \iff P \in \meaningof{E}
\end{eqnarray*}

\begin{eqnarray*}
  P \approx_{L} Q \iff \forall E \in L. P \models E \iff Q \models E
\end{eqnarray*}

\begin{eqnarray*}
  P \approx_{K} Q
\end{eqnarray*}

\begin{eqnarray*}
  P \approx Q
\end{eqnarray*}

$\approx_{K} = \approx = \approx_{L}$

\subsubsection{Contextual duality}

Note that contexts extend the quotation operation to a family of
operations from processes to names. Given a context, $M$, we can
define a \emph{nominal context}, $\quotep{M}$ by $\quotep{M}[P] :=
\quotep{M[P]}$. To foreshadow what is to come we observe that these
operations enjoy a duality with processes very much like the duality
between vectors and maps from vectors to scalars.

Further, because the calculus is essentially higher-order, we have a
correspondence between contexts and processes. More specifically,
given a name $x$ and a context $M$ we can construct $M^{*}_{x}$ such
that 

\begin{mathpar}
  M^{*}_{x} | \lift{x}{P} \red M[P]
\end{mathpar}

namely,

\begin{mathpar}
  M^{*}_{x} := x?(u).M[\dropn{u}]
\end{mathpar}

The dependence of $M^{*}_{x}$ on a name makes it an abstraction, 

\begin{mathpar}
  M^{*} := (x)x?(u).M[\dropn{u}]
\end{mathpar}

\subsection{Additional notation}

It will sometimes be convenient to denote the process a name
quotes. We already have the notation $x = \quotep{P}$, but it will be
convenient to introduce an alternate notation, $\procn{x}$, when we
want to emphasize the connection to the use of the name. Note that, by
virtue of name equivalence, $\quotep{\procn{x}} \nameeq x$; so, the
notation is consistent with previous definitions.

Further, because names have structure it is possible to effect
substitutions on the basis of that structure. This means we need to
upgrade our notation for substitutions, which we accomplish by
adapting comprehension notation. Thus,

\begin{mathpar}
  P\{ y / x : x \in S \}
\end{mathpar}

is interpreted to mean the process derived from P by replacing (in a
capture-avoiding manner) each occurrence of $x$ in $S$ by $y$. For example,

\begin{mathpar}
  P\{ \quotep{\procn{x}|\procn{x}} / x : x \in \freenames{P} \}
\end{mathpar}

will replace each (occurrence) of a free name $x$ in $P$ by
$\quotep{\procn{x}|\procn{x}}$.

Also, we will avail ourselves of the notation $x^{L}$ and $x^{R}$ to
denote injections of a name into disjoint copies of the name
space. There are numerous ways to accomplish this. One example can be
found in \cite{MeredithR05}. This notation overloads to vectors of
names: $\vec{x}^{\pi} := (x_{i}^{\pi} \; : \; 0 \leq i < |\vec{x}| )$ where $\pi \in \{L,R\}$.

We also use $P^{\Box} := P|\Box$.

In \cite{MeredithR05} an interpretation of the new operator is
given. It turns out that there are several possible interpretations
all enjoying the requisite algebraic properties of the operator (see
\cite{milner91polyadicpi}). We will therefore make liberal use of
$(\nu\; \vec{x})P$.

% subsection the_syntax_and_semantics_of_the_notation_system (end)   

\input{qm2pi.qmops} 

\input{qm2pi.sterngerlach} 

\input{qm2pi.metric} 

% section concurrent_process_calculi (end)

%\input{qm2pi.proofsketch}

% section proof sketch (end)

%\input{qm2pi.slviaknots} 

% section spatial logic via knots (end)

\input{qm2pi.conclusion}

% section conclusion (end)

%\input{qm2pi.dtcodes} 

% section wiring algorithm (end)

\input{qm2pi.ack} 

% section acknowledgments (end)

\newpage


\bibliographystyle{plain}   
\bibliography{../../biblios/main.bib}

\input{qm2pi.rhodetails}

\end{document}



% section proof sketch (end)

%\section{Unlikely characters: spatial logic for
  knots}\label{sub:characteristic_formulae} % (fold)

Associated to the mobile process calculi are a family of logics known
as the Hennessy-Milner logics. These logics typically enjoy a
semantics interpreting formulae as sets of processes that when
factored through the encoding outlined above allows an identification
of classes of knots with logical formulae. In the context of this
encoding the sub-family known as the spatial logics \cite{CairesC03}
\cite{CairesC04} \cite{Caires04} are of particular interest providing
several important features for expressing and reasoning about
properties (i.e. classes) of knots. We hint here at how this may be done.

%\begin{description}
%\item [structural connectives] 
\subsubsection{Structural connectives} The spatial logics enjoy
structural connectives corresponding, at the logical level, to the
parallel composition ($P | Q$) and new name ($(\nu \; x)P$)
connectives for processes. As illustrated in the examples below, these
connectives are extremely expressive given the shape of our encoding.
%\item [decideable satisfaction]

\subsubsection{Decideable satisfaction}
In \cite{Caires04} the satisfaction relation is shown to be decideable
for a rich class of processes. It further turns out that the image of
the our encoding is a proper subset of that class. This result
provides the basis for an algorithm by which to search for knots
enjoying a given property.
%\item [characteristic formulae]

\subsubsection{Characteristic formulae}
In the same paper \cite{Caires04} , Caires presents a means of calculating
characteristic formulae, selecting equivalence classes of processes
up to a pre--specified depth limit on the support set of names. Composed with our
encoding, this characteristic formula can be used to select
characteristic formulae for knots.
%\end{description}

\subsubsection{Spatial logic formulae}

The grammar below (segmented for comprehension) summarizes the syntax
of spatial logic formulae. We employ illustrative examples in the
sequel to provide an intuitive understanding of their meaning
referring the reader to \cite{Caires04} for a more detailed explication
of the semantics.

\begin{mathpar}
  \inferrule* [lab=boolean] {} {{A,B} \bc T \;|\; \neg A \;|\; A \wedge B \;|\; \eta = \eta'}
  \and
  \inferrule* [lab=spatial] {} {|\; \pzero \;|\; A | B \;|\; x \text{\textregistered} A \;|\; \forall x . A \;|\;  H x . A}
  \and
  \inferrule* [lab=behavioral] {} {|\; \alpha . A}
  \and 
  \inferrule* [lab=recursion] {} {|\; X(\vec{u}) \;|\; \mu X(\vec{u}) . A}
  \and
  \inferrule* [lab=action] {} {\alpha \bc \langle x?(\vec{y}) \rangle \;|\; \langle x!(\vec{y}) \rangle \;|\; \langle \tau \rangle}
  \and 
  \inferrule* [lab=name] {} {\eta \bc x \;|\; \tau}
\end{mathpar} 

% subsection characteristic_formulae (end)   	 

\subsection{Example formulae}\label{sub:example_formulae_} % (fold)

\subsubsection{Crossing as formula.}
% 
% \begin{align*}
%   \frac{d}{dx} \sin x &= \cos x 
%   & \frac{d}{dx} e^x &= e^x \\
%   \frac{d}{dx} \cos x &= - \sin x 
%   & \frac{d}{dx} \log x &= \frac{1}{x} \\
% \end{align*} 

\begin{align*}
 \mu C(x_{0},x_{1},y_{0},y_{1},u).&(\langle x_{0}?(z) \rangle(\langle u! \rangle\langle y_{1}!z \rangle C(x_{0},x_{1},y_{0},y_{1},u)) & \\
  & \wedge \langle y_{1}?(z) \rangle (\langle u! \rangle \langle x_{0}!z \rangle C(x_{0},x_{1},y_{0},y_{1},u)) & \\
  & \wedge \langle x_{1}?(z) \rangle (\langle u? \rangle \langle y_{0}!z \rangle C(x_{0},x_{1},y_{0},y_{1},u)) & \\
  & \wedge \langle y_{0}?(z) \rangle (\langle u? \rangle \langle x_{1}!z \rangle C(x_{0},x_{1},y_{0},y_{1},u))) &
\end{align*}

The lexicographical similarity between the shape of this formulae and
the shape of definition of the process representing a crossing reveals
the intuitive meaning of this formulae. It describes the capabilities
of a process that has the right to represent a crossing. For example
it picks out processes that may perform an input on the port $x_0$ in
its initial menu of capabilities. What differentiates the formula
from the process, however, is that the crossing process is the
smallest candidate to satisfy the formula. Infinitely many other
processes -- with internal behavior hidden behind this interface, so
to speak -- also satisfy this formula. Even this simple formula,
then, can be seen to open a new view onto knots, providing a
computational interpretation of \emph{virtual} knots.

Note that this formula is derived by hand. A similar formula can be
derived by employing Caires' calculation of characteristic formula
\cite{Caires04} to the process representing a crossing. In light of
this discussion, we let
$\meaningof{C}_{\phi}(x0,x1,y0,y1,u)$ denote a formula specifying the
dynamics we wish to capture of a crossing. To guarantee we preserve
the shape of the interface and minimal semantics we demand that
$\meaningof{C}_{\phi}(x0,x1,y0,y1,u) \Rightarrow
\textbf{C}(x0,x1,y0,y1,u)$ where $\textbf{C}(x0,x1,y0,y1,u)$ denotes
the formula above.
                            
\subsubsection{Crossing number constraints.}
The moral content of the context lemma (Lemma \ref{context}) is that the notion of
``locality'' in the Reidemeister moves is effectively captured by the
parallel composition operator of the process calculus. This intuition
extends through the logic. Given a formula,
$\meaningof{C}_{\phi}(x0,x1,y0,y1,u)$, we can use the structural
connectives to specify constraints on crossing numbers, such as at
least $n$ crossings, or exactly $n$ crossings.
\begin{mathpar}
  \inferrule* [lab=at-least-n] {} { K^{\geq n}_{\phi}(\vec{xs},\vec{ys}) := \Pi_{i=0}^{n-1} Hu . \meaningof{C}_{\phi}(xs_i,ys_i,u) | T }
  \and 
  \inferrule* [lab=exactly-n] {} { K^{= n}_{\phi}(\vec{xs},\vec{ys}) := \Pi_{i=0}^{n-1} Hu . \meaningof{C}_{\phi}(xs_i,ys_i,u) | \neg (\forall x_0,y_0,x_1,y_1,u . \meaningof{C}_{\phi}(x_0,y_0,x_1,y_1,u) | T) }
\end{mathpar}

To round out this section, recall that the encoding of an $n$-crossing
knot decomposes into a parallel composition of $n$ \emph{copies} of a
crossing process together with a wiring harness. To specify different
knot classes with the same crossing number amounts to specifying
logical constraints on the wiring harness. In the interest of space,
we defer examples to a forthcoming paper. Suffice it to say that both
the conditions ``alternating knot'' and ``contains the tangle
corresponding to 5/3'' are expressible. For example, it is possible to
calculate the characteristic formula of a process corresponding to the
tangle 5/3 and conjoin it into the classifying formula via the
composition connective of the logic.

Finally, we wish to observe that it is entirely within reason to
contemplate a more domain-specific version of spatial logic tailored
to the shape of processes in the image of the encoding. Such a
domain-specific logic would have a better claim to the title formal
language of knot properties.

% subsection example_formulae_ (end)

% section knots_as_processes (end) 

% section spatial logic via knots (end)

\section{Conclusions and future work}

\paragraph{Testing physical space}
You, gentle reader, may wonder why of all the theorems to be proved
given this set up we pick the one above. In some sense it's hardly
central to quantum mechanics. We see it as central in the sense that
it firmly establishes a notion of physical space arising from a notion
of the equivalence of behavior. Relating bisimulation to a metric is a
big step forward, but one is faced with interpreting the relationship
of that metric space to something more physical. Quantum mechanical
notions of ``physical'' space are still far from intuitive, but by
relating this idea of distance as testing to calculations that predict
physical circumstances we are making a not insignificant step forward
toward an understanding of the physical space we inhabit as
essentially dynamic.

\paragraph{Effectivity and simulation}
One of the observations we have yet to make is that the entire program
spelled out here is effective. We have built various interpreters for
the reflective calculus at work in this interpretation. In principle,
then, we can simulate quantum mechanics on a computer. The place where
the simulation may lose fidelity is the infinitely branching summation
for the annihilator.

In this connection i also want to point out that the evaluation style
calculation of the inner product puts the non-determinism of the
summation right at the heart of measurement. This suggests that
Milner's original reduction-based formulation of the dynamics of his
calculi in terms of sums was not just notationally suggestive of a
notion of measure-and-continue but captured some significant part of
the physics.

\paragraph{Quantum continuations}
In light of this last observation i want to point out that the
predominant account of quantum mechanics is missing a key aspect of a
truly compositional story of the physical situation. In a real lab,
when a measurement is made the observation can be made to feed into
another device that then makes another measurement conditioned on the
results of the first. This means that after the superposition was
collapsed the entire experimental set up remained in
superposition. While QM offers a means of writing this down it doesn't
quite line up well with the well-trodden formulation of computation
and continuation that we see so succinctly expressed in Milner's
calculi. This suggests that there might be advantages to this account
of dynamics waiting to be explored.

\paragraph{Quantum logic}
In this connection, we also note that by virtue of having the
Hennessy-Milner construction, we can pull the construction through the
interpretation of QM. This gives us a natural candidate for a quantum
logic that enjoys an extremely tight connection with it's domain of
interpretation, making the construction much less ad hoc (rather it is
the image of functor!).

\paragraph{Quantum probabiity}
i have questions about the basis of the interpretation of inner
product as probability amplitude. In particular, using which
axiomatization of probability theory does the notion of probability
amplitude earn the right to be so dubbed? In other words, where is the
proof that the operation for calculating a probability amplitude (and
then squaring) satisfies the axioms of what it means to calculate a
probability? Even if such a proof exists (i have yet to find it in the
literature), i wonder if it might not be possible to turn things on
their heads. Can we view the calculation of the probability amplitude
as an axiomatization of probability? If so, then the definition we
give for calculating probability amplitude may provide the basis for
an \emph{effective} theory of probability.

\paragraph{Quantum vs ``biological'' information}
Finally, i want to conclude with a more philosophical observation. At
a recent workshop in which QM was a predominant topic i noticed
something about quantum information. The speaker was giving a riveting
discussion of axiomatic QM and showing how properties of ``no
cloning'' and ``no deleting'' emerged as consequences of the
axiomatization. Theorems of this form are necessary to give us a sense
of confidence that our axioms characterize the physical theory. What
struck me, though, was that if quantum information is neither erasable
nor replicable it is markedly different from \emph{life}. Two of the
things we know about life is that

\begin{itemize}
  \item it ends;
  \item to gain some measure of persistence, to transcend it's
    finitude it is imminently copyable.
\end{itemize}

Both of these qualities are summarized succinctly in the aphorism: all
flesh is grass. For me these two kinds of ``information'' -- call them
quantum and biological -- are end points on a spectrum of strategies
for persistence. At one end, we have those curious entities that enjoy
uniqueness and permanence; at the other, we have those who in the face
of a certain end and an uncertain present make a go of passing
something on. To me one of the more remarkable aspects of the latter
strategy is that in the presence of noise (and certain features of
copying) we get a kind of dynamism, a chance for improvement against a
given persistent condition.

% subsection other_calculi_other_bisimulations_and_geometry_as_behavior (end)




% section conclusion (end)

%\documentclass[12pt]{llncs}
%\documentclass{jktr}

\usepackage[pdftex]{hyperref}                   
\usepackage {listings}
\usepackage {mathpartir}
\usepackage{bcprules}
%\usepackage{listings}
                       
\usepackage{graphicx} 
%\usepackage[margins=2.5cm,nohead,nofoot]{geometry}
%\usepackage{geometry}
\usepackage{amsfonts}
\usepackage{amstext}
\usepackage{latexsym}
\usepackage{amssymb}
\usepackage{color}


%\include{myPreamble}
\include{qm2pi.local} 

%\ifpdf
%\usepackage[pdftex]{graphicx}
%\else
%\usepackage{graphicx}
%\fi

 % \ifpdf
%  \usepackage{pdfsync}
%  \if


%\title{Brief Article}
%\author{David F. Snyder}
%\author{L.G. Meredith}

%\address{Dept. of Math., Texas State University--San Marcos, San Marcos, TX 78666}
       
\pagestyle{empty}


\begin{document}

\lstset{language=[Objective]Caml,frame=shadowbox}

\input{qm2pi.front}

% section front matter (end)

\input{qm2pi.intro} 
 
% section introduction (end)

% \input{qm2pi.knotations} 

% section notation (end)

\input{qm2pi.process.calculi} 

% section concurrent_process_calculi_and_spatial_logics_ (end)
    
%\input{qm2pi.knots2pi} 

%\input{qm2pi.trefoil} 

%\input{qm2pi.mainthm} 

% subsection basic_interpretation (end)

%\input{qm2pi.rho.presentation} 
\subsection{The syntax and semantics of the notation system}\label{sub:the_syntax_and_semantics_of_the_notation_system} % (fold)

We now summarize a technical presentation of the calculus that
embodies our theory of dynamics. The typical presentation of such a
calculus follows the style of giving generators and relations on
them. The grammar, below, describing term constructors, freely
generates the set of processes, $\Proc$. This set is then quotiented
by a relation known as structural congruence and it is over this set
that the notion of dynamics is expressed. This presentation is
essentially that of \cite{MeredithR05} with the addition of
polyadicity and summation. For readability we have relegated some of
the technical subtleties to an appendix.

\subsubsection{Process grammar}\label{subsub:process_grammar}

\begin{mathpar}
  \inferrule* [lab=synchronization] {} {{M} \bc \pzero \;|\; x?F \;|\; x!C }
  \and
  \inferrule* [lab=abstraction] {} {{F} \bc (x)P}
  \and
  \inferrule* [lab=concretion] {} {{C} \bc \langle Q \rangle}
  \and
  \inferrule* [lab=process] {} {{P,Q} \bc M \;| \;P|Q \;|\; @{x}}
  \and
  \inferrule* [lab=name] {} {{x} \bc \quotep{P}}
\end{mathpar} 

Note that $\vec{x}$ (resp. $\vec{P}$) denotes a vector of names
(resp. processes) of length $|\vec{x}|$ (resp. $|\vec{P}|$). We adopt
the following useful abbreviations.

\begin{mathpar}
   x?(\vec{y}).P := x.(\vec{y})P \and  x\clift{\vec{P}} := x.\clift{\vec{P}}
   \and x!(y) := \lift{x}{\dropn{y}}
   \and \Pi_{i=0}^{n-1}P_i := P_0 | \ldots | P_{n-1}
\end{mathpar}

\subsubsection{Structural congruence}

\paragraph{Free and bound names and alpha-equivalence.} At the
core of structural equivalence is alpha-equivalence which identifies
process that are the same up to a change of variable. Formally, we
recognize the distinction between free and bound names. The free names
of a process, $\freenames{P}$, may be calculated recursively as
follows:

\begin{mathpar}
\freenames{\pzero} := \emptyset
  \and \\
  \freenames{x?(y).P} := \{ x \} \cup (\freenames{P} \setminus \{ y \})
  \and 
  \freenames{x!\langle P \rangle} := \{ x \} \cup \{ P \} 
  \and \\
  \freenames{P|Q} := \freenames{P} \cup \freenames{Q}
  \and \\
  \freenames{@{x}} := \{ x \}
\end{mathpar}

$\pi$
$\quotep{\pi}$

$\freenames{-} : \pi \to \mathcal{P}(\quotep{\pi})$

\begin{eqnarray*}
  \freenames{\pzero} & := & \emptyset \\
  \freenames{x?(y).P} & := & \{ x \} \cup (\freenames{P} \setminus \{ y \}) \\
  \freenames{x!\langle P \rangle} & := & \{ x \} \cup \{ P \} \\
  \freenames{P|Q} & := & \freenames{P} \cup \freenames{Q} \\
  \freenames{\dropn{x}} & := & \{ x \}
\end{eqnarray*}

The bound names of a process, $\boundnames{P}$, are those names occurring in $P$
that are not free. For example, in $x?(y).0$, the name $x$ is free, while $y$ is bound.

\begin{mathpar}
  \inferrule* [lab=monoidal-laws] {} { P|Q \equiv Q|P \and P|0 \equiv P \and P|(Q|R) \equiv (P|Q)|R }
\end{mathpar}

\begin{mathpar}
  \inferrule* [lab=alpha-equivalence] {} { (x)P \equiv (y)P\{y/x\} \and y \not\in \freenames{P} }
\end{mathpar}

\begin{definition}
Then two processes, $P,Q$, are alpha-equivalent if $P = Q\{\vec{y}/\vec{x}\}$ for
some $\vec{x} \in \boundnames{Q},\vec{y} \in \boundnames{P}$, where $Q\{\vec{y}/\vec{x}\}$
denotes the capture-avoiding substitution of $\vec{y}$ for $\vec{x}$ in $Q$.
\end{definition}

\begin{definition}
  The {\em structural congruence} \cite{SangiorgiWalker} , $\equiv$,
  between processes is the least congruence containing
  alpha-equivalence, satisfying the abelian monoid laws
  (associativity, commutativity and $\pzero$ as identity) for parallel
  composition $|$ and for summation $+$.
\end{definition}

\subsection{Name equivalence}

We take name equivalence, written $\nameeq$, to be the smallest
equivalence relation generated by the following rules.

\begin{mathpar}
\inferrule*[lab=Quote-drop]
{ }
{ \quotep{@{x}} \nameeq x }

\inferrule*[lab=Struct-equiv]
{ P \scong Q }
{ \quotep{P} \nameeq \quotep{Q} }
\end{mathpar}

The astute reader will have noticed that the mutual recursion of names
and processes imposes a mutual recursion on alpha-equivalence and
structural equivalence via name-equivalence. Fortunately, all of this
works out pleasantly and we may calculate in the natural way, free of
concern. The reader interested in the details is referred to the
appendix \ref{appendix:rho_details}.

\subsection{Substitution}

We use $\Proc$ for the set of processes, $\QProc$ for the set of
names, and $\id{\{}\vec{y} / \vec{x} \id{\}}$ to denote partial maps,
$s : \QProc \rightarrow \QProc$. A map, $s$ lifts, uniquely, to a map
on process terms, $\widehat{s} : \Proc \rightarrow \Proc$ by the
following equations.

\begin{mathpar}
  (0) \psubstp{Q}{P} := 0 \\
  (R \juxtap S) \psubstp{Q}{P}
  :=    
  (R)\psubstp{Q}{P} \juxtap (S) \psubstp{Q}{P} \\
  (x?(y).R) \psubstp{Q}{P}    
  :=    
  (x)\substp{Q}{P} (z)\concat( (R \psubstn{z}{y}) \psubstp{Q}{P} ) \\
  (\lift{x}{R}) \psubstp{Q}{P}  
  :=
  \lift{(x)\substp{Q}{P}}{ R \psubstp{Q}{P} } \\
%   (\dropn{x})  \psubstp{Q}{P}       
%   := 
%   \left\{ 
%     \begin{array}{ccc} 
%       \dropn{\quotep{Q}} & & x \nameeq \quotep{P} \\
%       \dropn{x} & & otherwise \\
%     \end{array}
%   \right. 
  (\dropn{x})  \psubstp{Q}{P}       
  := 
  \left\{ 
    \begin{array}{ccc} 
      Q & & x \nameeq \quotep{P} \\
      \dropn{x} & & otherwise \\
    \end{array}
  \right.
\end{mathpar}
 

where

\begin{eqnarray}
  (x)\id{\{} \lpquote Q \rpquote / \lpquote P \rpquote \id{\}}            = 
  \left\{ 
    \begin{array}{ccc}
      \lpquote Q \rpquote & & x \nameeq \lpquote P \rpquote \\
      x & & otherwise \\
    \end{array}
  \right. \nonumber
\end{eqnarray}

and $z$ is chosen distinct from $\quotep{P}$, $\quotep{Q}$, the free
names in $Q$, and all the names in $R$. Our $\alpha$-equivalence will
be built in the standard way from this substitution.

\begin{remark}\label{rem:no_self_referential_names}
  One consequence of these definitions is that $\forall P. \quotep{P}
  \not\in \freenames{P}$.
\end{remark}

\subsection{ Dynamic quote: an example }

Anticipating something of what's to come, consider applying the
substitution, $\widehat{\id{\{}u / z \id{\}}}$, to the following pair
of processes, $\lift{w}{y!(z)}$ and $w[ \lpquote y!(z) \rpquote ]$.

\begin{eqnarray}
	\lift{w}{y!(z)}\widehat{\id{\{}u / z \id{\}}}
		& = &
		\lift{w}{y!(u)} \nonumber\\
	w[ \lpquote y!(z) \rpquote ] \widehat{ \id{\{}u / z \id{\}} }
		& = &
		w[ \lpquote y!(z) \rpquote ] \nonumber
\end{eqnarray}

Because the body of the process between quotes is impervious to
substitution, we get radically different answers. In fact, by
examining the first process in an input context,
e.g. $x?(z).\lift{w}{y!(z)}$, we see that the process under the lift
operator may be shaped by prefixed inputs binding a name inside it. In
this sense, the lift operator will be seen as a way to dynamically
construct processes before reifying them as names.

Finally equipped with these standard features we can present the
dynamics of the calculus.

\subsubsection{Operational semantics} 

Finally, we introduce the computational dynamics. What marks these
algebras as distinct from other more traditionally studied algebraic
structures, e.g. vector spaces or polynomial rings, is the manner in
which dynamics is captured. In traditional structures, dynamics is typically
expressed through morphisms between such structures, as in linear maps
between vector spaces or morphisms between rings. In algebras
associated with the semantics of computation, the dynamics is
expressed as part of the algebraic structure itself, through a
reduction reduction relation typically denoted by $\red$. Below, we
give a recursive presentation of this relation for the calculus used
in the encoding.

$\red \subseteq \pi \times \pi$
$\red : \pi \to \mathcal{P}(\pi)$

\begin{mathpar}
  \inferrule* [lab=Comm] { \textsf{match}( x_{src}, x_{trgt} ) } { x_{trgt}?(y)P \; | \; x_{src}!\langle {Q} \rangle \red P\{\quotep{Q}/y}\} }
  \and \\
  \inferrule* [lab=Par] {{P} \red {P}'} {{{P} | {Q}} \red {{P}' | {Q}}}
  \and
  \inferrule* [lab=Equiv]{{{P} \scong {P}'} \andalso {{P}' \red {Q}'} \andalso {{Q}' \scong {Q}}}{{P} \red {Q}}
\end{mathpar}

\begin{eqnarray*}
  match_{\equiv} (\quotep{P},\quotep{Q}) & := & P \equiv Q \\
  match_{\dagger}(\quotep{P},\quotep{Q}) & := & \forall R. P|Q \red^{*} R => R \red^{*} 0 \\
  match_{K}(\quotep{P},\quotep{Q}) & := & K \mbox{ for some context } K
\end{eqnarray*}

$u?(x)P | u!\langle Q \rangle \red P\{\quotep{Q}/x\}$

%We write $\wred$ for $\red^*$, and $P\red$ if $\exists Q $ such that $ P \red Q$.
We write $P\red$ if $\exists Q $ such that $ P \red Q$ and $P\not\red$, otherwise.

\section{Replication}

As mentioned before, it is known that replication (and hence
recursion) can be implemented in a higher-order process algebra
\cite{SangiorgiWalker}. As our first example of calculation with the
machinery thus far presented we give the construction explicitly in
the {\rhoc}.

\begin{eqnarray}
	D_{x} & := & \prefix{x}{y}{(\binpar{\outputp{x}{y}}{@{y}})} \nonumber\\
	\bangp_{x}{P} & := & \binpar{{x}!\langle{\binpar{D_{x}}{P}}\rangle}{D_{x}} \nonumber
\end{eqnarray}

\begin{eqnarray}
	\bangp_{x}{P} & & \nonumber\\
	=
	& {x}!\langle{(\prefix{x}{y}{(\outputp{x}{y} | @{y})) | P}}\rangle 
	      | \prefix{x}{y}{(\outputp{x}{y} | @{y})} & \nonumber\\
	\red
	& (\outputp{x}{y} | @{y})\substn{\quotep{(\prefix{x}{y}{(@{y} | \outputp{x}{y})) | P}}}{y} & \nonumber\\
	=
	& \outputp{x}{\quotep{(\prefix{x}{y}{(\outputp{x}{y} | @{y})) | P}}}
	  | {(\prefix{x}{y}{(\outputp{x}{y} | @{y})) | P}} & \nonumber\\
	\red
	& \ldots & \nonumber\\
	\red^*
	& P | P | \ldots & \nonumber
\end{eqnarray}

Of course, this encoding, as an implementation, runs away, unfolding
$\bangp{P}$ eagerly. A lazier and more implementable replication
operator, restricted to input-guarded processes, may be obtained as follows.

\begin{eqnarray}
\bangp{\prefix{u}{v}{P}} 
	:= 
	\binpar{\lift{x}{\prefix{u}{v}{(\binpar{D(x)}{P})}}}{D(x)} \nonumber
\end{eqnarray}

\begin{remark}
  Note that the lazier definition still does not deal with summation
  or mixed summation (i.e. sums over input and output). The reader is
  invited to construct definitions of replication that deal with these
  features. 

  Further, the definitions are parameterized in a name, $x$. Can you,
  gentle reader, make a definition that eliminates this parameter and
  guarantees no accidental interaction between the replication
  machinery and the process being replicated -- i.e. no accidental
  sharing of names used by the process to get its work done and the
  name(s) used by the replication to effect copying. This latter
  revision of the definition of replication is crucial to obtaining
  the expected identity $!!P \sim !P$.
\end{remark}

\begin{remark}\label{rem:paradoxical_combinator}
  The reader familiar with the lambda calculus will have noticed the
  similarity between $D$ and the paradoxical combinator.

  [Ed. note: the existence of this seems to suggest we have to be more
  restrictive on the set of processes and names we admit if we are to
  support no-cloning.]
\end{remark}

\subsubsection{Bisimulation}

The computational dynamics gives rise to another kind of equivalence,
the equivalence of computational behavior. As previously mentioned
this is typically captured \emph{via} some form of bisimulation.

% The notion we use in this paper is weak barbed bisimulation
% \cite{milner91polyadicpi}.

The notion we use in this paper is derived from weak barbed
bisimulation \cite{milner91polyadicpi}. 

\begin{definition}
An \emph{observation relation}, $\downarrow_{\mathcal N}$, over a set
of names, $\mathcal N$, is the smallest relation satisfying the rules
below.

\infrule[Out-barb]{y \in {\mathcal N}, \; x \nameeq y}
		  {\outputp{x}{v} \downarrow_{\mathcal N} x}
\infrule[Par-barb]{\mbox{$P\downarrow_{\mathcal N} x$ or $Q\downarrow_{\mathcal N} x$}}
		  {\binpar{P}{Q} \downarrow_{\mathcal N} x}

We write $P \Downarrow_{\mathcal N} x$ if there is $Q$ such that 
$P \wred Q$ and $Q \downarrow_{\mathcal N} x$.
\end{definition}

\begin{definition}
%\label{def.bbisim}
An  ${\mathcal N}$-\emph{barbed bisimulation} over a set of names, ${\mathcal N}$, is a symmetric binary relation 
${\mathcal S}_{\mathcal N}$ between agents such that $P\rel{S}_{\mathcal N}Q$ implies:
\begin{enumerate}
\item If $P \red P'$ then $Q \wred Q'$ and $P'\rel{S}_{\mathcal N} Q'$.
\item If $P\downarrow_{\mathcal N} x$, then $Q\Downarrow_{\mathcal N} x$.
\end{enumerate}
$P$ is ${\mathcal N}$-barbed bisimilar to $Q$, written
$P \wbbisim_{\mathcal N} Q$, if $P \rel{S}_{\mathcal N} Q$ for some ${\mathcal N}$-barbed bisimulation ${\mathcal S}_{\mathcal N}$.
\end{definition}

$\mathcal{R} \subseteq \pi \times \pi$

$P \mathcal{R} Q => \forall P'. P \red P' \Rightarrow \exists Q'. Q \red Q', P' \mathcal{R} Q'$

$P \vdash x \Rightarrow Q \vdash x$

\begin{mathpar}
  \inferrule*[lab=Out-barb]{x \nameeq y}{{y}!\langle{Q}\rangle \vdash x}
  \and
  \inferrule*[lab=Par-barb]{\mbox{$P\vdash x$ or $Q\vdash x$}}{\binpar{P}{Q} \vdash x}
\end{mathpar}

\subsubsection{Contexts}

One of the principle advantages of computational calculi like the
$\pi$-calculus is a well-defined notion of context,
contextual-equivalence and a correlation between
contextual-equivalence and notions of bisimulation. The notion of
context allows the decomposition of a process into (sub-)process and
its syntactic environment, its context. Thus, a context may be
thought of as a process with a ``hole'' (written $\Box$) in it. The
application of a context $M$ to a process $P$, written $M[P]$, is
tantamount to filling the hole in $M$ with $P$. In this paper we do
not need the full weight of this theory, but do make use of the notion
of context in the proof the main theorem. 

\begin{mathpar}
  \inferrule* [lab=summation] {} {{M_{M},M_{N}} \bc \Box \;|\; x.M_{A} \;|\; M_{M}+M_{N}}
  \and
  \inferrule* [lab=agent] {} {{M_{A}} \bc (\vec{x})M_{P} \;| \; \clift{P_0,\ldots,M_{P},\ldots,P_N}}
  \and \\
  \inferrule* [lab=process] {} {{M_{P}} \bc M_{N} \;| \;P|M_{P} }
\end{mathpar} 

\begin{mathpar}
  \inferrule* [lab=sychronization] {} {M_{N} \bc \Box \;|\; x?M_{F} \;|\; x!M_{C}}
  \and
  \inferrule* [lab=abstraction] {} {{M_{F}} \bc (x)M_{P} }
  \and
  \inferrule* [lab=concretion] {} {{M_{C}} \bc \langle M_{P} \rangle }
  \and \\
  \inferrule* [lab=process] {} {{M_{P}} \bc M_{N} \;| \;P|M_{P} }
\end{mathpar}

\begin{definition}[contextual application] Given a context $M$, and
  process $P$, we define the \emph{contextual application}, $M[P] :=
  M\{P/\Box\}$. That is, the contextual application of M to P is the
  substitution of $P$ for $\Box$ in $M$.
\end{definition}

$\meaningof{-} : L \to \mathcal{P}(\pi)$

\begin{mathpar}
  \inferrule* [lab=collection] {} {\meaningof{true} = \pi, \and \meaningof{~E} = \pi \setminus \meaningof{E}, \and \meaningof{E_{1} \& E_{2}} = \meaningof{E_{1}} \cap \meaningof{E_{2}}}
\end{mathpar}

\begin{mathpar}
  \inferrule* [lab=structure] {} {\meaningof{0} = \{ P \in \pi | P \equiv 0 \}, \and \\ \meaningof{E_1 | E_2} = \{ P \in \pi | P \equiv P_{1} | P_{2}, P_{1} \in \meaningof{E_{1}}, P_{2} \in \meaningof{E_2}\} }
\end{mathpar}

\begin{mathpar}
 \inferrule* [lab=behavior] {} {\meaningof{\langle a?b \rangle E} = \{ P \in \pi | P \equiv Q | u?(y)P', \\ \and \\\\ \and \\ \;\;\; u \in \meaningof{a}, \forall z.P'\{z/y\} \in \meaningof{E\{z/b\}}\}, \and \\ \meaningof{a!E} = \{ P \in \pi | P \equiv Q | x!\langle P' \rangle, x \in \meaningof{a} P' \in \meaningof{E}\} }
\end{mathpar}

\begin{mathpar}
 \inferrule* [lab=nominal] {} {\meaningof{\quotep{E}} = \{ \quotep{P} \in \quotep{\pi} | P \in \meaningof{E} \}, \and \meaningof{\quotep{P}} = \{ \quotep{Q} \in \quotep{\pi} | P \equiv Q \} \and \\ \meaningof{@\quotep{E}} = \{ P \in \pi | P \equiv @x, x \in \meaningof{E} \}}
\end{mathpar}

\begin{eqnarray*}
  \\
  \meaningof{-} : TS \to ST
\end{eqnarray*}

\begin{eqnarray*}
  \\
  L : TS \to ST
\end{eqnarray*}

\begin{eqnarray*}
  \\
  P \models E \iff P \in \meaningof{E}
\end{eqnarray*}

\begin{eqnarray*}
  P \approx_{L} Q \iff \forall E \in L. P \models E \iff Q \models E
\end{eqnarray*}

\begin{eqnarray*}
  P \approx_{K} Q
\end{eqnarray*}

\begin{eqnarray*}
  P \approx Q
\end{eqnarray*}

$\approx_{K} = \approx = \approx_{L}$

\subsubsection{Contextual duality}

Note that contexts extend the quotation operation to a family of
operations from processes to names. Given a context, $M$, we can
define a \emph{nominal context}, $\quotep{M}$ by $\quotep{M}[P] :=
\quotep{M[P]}$. To foreshadow what is to come we observe that these
operations enjoy a duality with processes very much like the duality
between vectors and maps from vectors to scalars.

Further, because the calculus is essentially higher-order, we have a
correspondence between contexts and processes. More specifically,
given a name $x$ and a context $M$ we can construct $M^{*}_{x}$ such
that 

\begin{mathpar}
  M^{*}_{x} | \lift{x}{P} \red M[P]
\end{mathpar}

namely,

\begin{mathpar}
  M^{*}_{x} := x?(u).M[\dropn{u}]
\end{mathpar}

The dependence of $M^{*}_{x}$ on a name makes it an abstraction, 

\begin{mathpar}
  M^{*} := (x)x?(u).M[\dropn{u}]
\end{mathpar}

\subsection{Additional notation}

It will sometimes be convenient to denote the process a name
quotes. We already have the notation $x = \quotep{P}$, but it will be
convenient to introduce an alternate notation, $\procn{x}$, when we
want to emphasize the connection to the use of the name. Note that, by
virtue of name equivalence, $\quotep{\procn{x}} \nameeq x$; so, the
notation is consistent with previous definitions.

Further, because names have structure it is possible to effect
substitutions on the basis of that structure. This means we need to
upgrade our notation for substitutions, which we accomplish by
adapting comprehension notation. Thus,

\begin{mathpar}
  P\{ y / x : x \in S \}
\end{mathpar}

is interpreted to mean the process derived from P by replacing (in a
capture-avoiding manner) each occurrence of $x$ in $S$ by $y$. For example,

\begin{mathpar}
  P\{ \quotep{\procn{x}|\procn{x}} / x : x \in \freenames{P} \}
\end{mathpar}

will replace each (occurrence) of a free name $x$ in $P$ by
$\quotep{\procn{x}|\procn{x}}$.

Also, we will avail ourselves of the notation $x^{L}$ and $x^{R}$ to
denote injections of a name into disjoint copies of the name
space. There are numerous ways to accomplish this. One example can be
found in \cite{MeredithR05}. This notation overloads to vectors of
names: $\vec{x}^{\pi} := (x_{i}^{\pi} \; : \; 0 \leq i < |\vec{x}| )$ where $\pi \in \{L,R\}$.

We also use $P^{\Box} := P|\Box$.

In \cite{MeredithR05} an interpretation of the new operator is
given. It turns out that there are several possible interpretations
all enjoying the requisite algebraic properties of the operator (see
\cite{milner91polyadicpi}). We will therefore make liberal use of
$(\nu\; \vec{x})P$.

% subsection the_syntax_and_semantics_of_the_notation_system (end)   

\input{qm2pi.qmops} 

\input{qm2pi.sterngerlach} 

\input{qm2pi.metric} 

% section concurrent_process_calculi (end)

%\input{qm2pi.proofsketch}

% section proof sketch (end)

%\input{qm2pi.slviaknots} 

% section spatial logic via knots (end)

\input{qm2pi.conclusion}

% section conclusion (end)

%\input{qm2pi.dtcodes} 

% section wiring algorithm (end)

\input{qm2pi.ack} 

% section acknowledgments (end)

\newpage


\bibliographystyle{plain}   
\bibliography{../../biblios/main.bib}

\input{qm2pi.rhodetails}

\end{document}

 

% section wiring algorithm (end)

\documentclass[12pt]{llncs}
%\documentclass{jktr}

\usepackage[pdftex]{hyperref}                   
\usepackage {listings}
\usepackage {mathpartir}
\usepackage{bcprules}
%\usepackage{listings}
                       
\usepackage{graphicx} 
%\usepackage[margins=2.5cm,nohead,nofoot]{geometry}
%\usepackage{geometry}
\usepackage{amsfonts}
\usepackage{amstext}
\usepackage{latexsym}
\usepackage{amssymb}
\usepackage{color}


%\include{myPreamble}
\include{qm2pi.local} 

%\ifpdf
%\usepackage[pdftex]{graphicx}
%\else
%\usepackage{graphicx}
%\fi

 % \ifpdf
%  \usepackage{pdfsync}
%  \if


%\title{Brief Article}
%\author{David F. Snyder}
%\author{L.G. Meredith}

%\address{Dept. of Math., Texas State University--San Marcos, San Marcos, TX 78666}
       
\pagestyle{empty}


\begin{document}

\lstset{language=[Objective]Caml,frame=shadowbox}

\input{qm2pi.front}

% section front matter (end)

\input{qm2pi.intro} 
 
% section introduction (end)

% \input{qm2pi.knotations} 

% section notation (end)

\input{qm2pi.process.calculi} 

% section concurrent_process_calculi_and_spatial_logics_ (end)
    
%\input{qm2pi.knots2pi} 

%\input{qm2pi.trefoil} 

%\input{qm2pi.mainthm} 

% subsection basic_interpretation (end)

%\input{qm2pi.rho.presentation} 
\subsection{The syntax and semantics of the notation system}\label{sub:the_syntax_and_semantics_of_the_notation_system} % (fold)

We now summarize a technical presentation of the calculus that
embodies our theory of dynamics. The typical presentation of such a
calculus follows the style of giving generators and relations on
them. The grammar, below, describing term constructors, freely
generates the set of processes, $\Proc$. This set is then quotiented
by a relation known as structural congruence and it is over this set
that the notion of dynamics is expressed. This presentation is
essentially that of \cite{MeredithR05} with the addition of
polyadicity and summation. For readability we have relegated some of
the technical subtleties to an appendix.

\subsubsection{Process grammar}\label{subsub:process_grammar}

\begin{mathpar}
  \inferrule* [lab=synchronization] {} {{M} \bc \pzero \;|\; x?F \;|\; x!C }
  \and
  \inferrule* [lab=abstraction] {} {{F} \bc (x)P}
  \and
  \inferrule* [lab=concretion] {} {{C} \bc \langle Q \rangle}
  \and
  \inferrule* [lab=process] {} {{P,Q} \bc M \;| \;P|Q \;|\; @{x}}
  \and
  \inferrule* [lab=name] {} {{x} \bc \quotep{P}}
\end{mathpar} 

Note that $\vec{x}$ (resp. $\vec{P}$) denotes a vector of names
(resp. processes) of length $|\vec{x}|$ (resp. $|\vec{P}|$). We adopt
the following useful abbreviations.

\begin{mathpar}
   x?(\vec{y}).P := x.(\vec{y})P \and  x\clift{\vec{P}} := x.\clift{\vec{P}}
   \and x!(y) := \lift{x}{\dropn{y}}
   \and \Pi_{i=0}^{n-1}P_i := P_0 | \ldots | P_{n-1}
\end{mathpar}

\subsubsection{Structural congruence}

\paragraph{Free and bound names and alpha-equivalence.} At the
core of structural equivalence is alpha-equivalence which identifies
process that are the same up to a change of variable. Formally, we
recognize the distinction between free and bound names. The free names
of a process, $\freenames{P}$, may be calculated recursively as
follows:

\begin{mathpar}
\freenames{\pzero} := \emptyset
  \and \\
  \freenames{x?(y).P} := \{ x \} \cup (\freenames{P} \setminus \{ y \})
  \and 
  \freenames{x!\langle P \rangle} := \{ x \} \cup \{ P \} 
  \and \\
  \freenames{P|Q} := \freenames{P} \cup \freenames{Q}
  \and \\
  \freenames{@{x}} := \{ x \}
\end{mathpar}

$\pi$
$\quotep{\pi}$

$\freenames{-} : \pi \to \mathcal{P}(\quotep{\pi})$

\begin{eqnarray*}
  \freenames{\pzero} & := & \emptyset \\
  \freenames{x?(y).P} & := & \{ x \} \cup (\freenames{P} \setminus \{ y \}) \\
  \freenames{x!\langle P \rangle} & := & \{ x \} \cup \{ P \} \\
  \freenames{P|Q} & := & \freenames{P} \cup \freenames{Q} \\
  \freenames{\dropn{x}} & := & \{ x \}
\end{eqnarray*}

The bound names of a process, $\boundnames{P}$, are those names occurring in $P$
that are not free. For example, in $x?(y).0$, the name $x$ is free, while $y$ is bound.

\begin{mathpar}
  \inferrule* [lab=monoidal-laws] {} { P|Q \equiv Q|P \and P|0 \equiv P \and P|(Q|R) \equiv (P|Q)|R }
\end{mathpar}

\begin{mathpar}
  \inferrule* [lab=alpha-equivalence] {} { (x)P \equiv (y)P\{y/x\} \and y \not\in \freenames{P} }
\end{mathpar}

\begin{definition}
Then two processes, $P,Q$, are alpha-equivalent if $P = Q\{\vec{y}/\vec{x}\}$ for
some $\vec{x} \in \boundnames{Q},\vec{y} \in \boundnames{P}$, where $Q\{\vec{y}/\vec{x}\}$
denotes the capture-avoiding substitution of $\vec{y}$ for $\vec{x}$ in $Q$.
\end{definition}

\begin{definition}
  The {\em structural congruence} \cite{SangiorgiWalker} , $\equiv$,
  between processes is the least congruence containing
  alpha-equivalence, satisfying the abelian monoid laws
  (associativity, commutativity and $\pzero$ as identity) for parallel
  composition $|$ and for summation $+$.
\end{definition}

\subsection{Name equivalence}

We take name equivalence, written $\nameeq$, to be the smallest
equivalence relation generated by the following rules.

\begin{mathpar}
\inferrule*[lab=Quote-drop]
{ }
{ \quotep{@{x}} \nameeq x }

\inferrule*[lab=Struct-equiv]
{ P \scong Q }
{ \quotep{P} \nameeq \quotep{Q} }
\end{mathpar}

The astute reader will have noticed that the mutual recursion of names
and processes imposes a mutual recursion on alpha-equivalence and
structural equivalence via name-equivalence. Fortunately, all of this
works out pleasantly and we may calculate in the natural way, free of
concern. The reader interested in the details is referred to the
appendix \ref{appendix:rho_details}.

\subsection{Substitution}

We use $\Proc$ for the set of processes, $\QProc$ for the set of
names, and $\id{\{}\vec{y} / \vec{x} \id{\}}$ to denote partial maps,
$s : \QProc \rightarrow \QProc$. A map, $s$ lifts, uniquely, to a map
on process terms, $\widehat{s} : \Proc \rightarrow \Proc$ by the
following equations.

\begin{mathpar}
  (0) \psubstp{Q}{P} := 0 \\
  (R \juxtap S) \psubstp{Q}{P}
  :=    
  (R)\psubstp{Q}{P} \juxtap (S) \psubstp{Q}{P} \\
  (x?(y).R) \psubstp{Q}{P}    
  :=    
  (x)\substp{Q}{P} (z)\concat( (R \psubstn{z}{y}) \psubstp{Q}{P} ) \\
  (\lift{x}{R}) \psubstp{Q}{P}  
  :=
  \lift{(x)\substp{Q}{P}}{ R \psubstp{Q}{P} } \\
%   (\dropn{x})  \psubstp{Q}{P}       
%   := 
%   \left\{ 
%     \begin{array}{ccc} 
%       \dropn{\quotep{Q}} & & x \nameeq \quotep{P} \\
%       \dropn{x} & & otherwise \\
%     \end{array}
%   \right. 
  (\dropn{x})  \psubstp{Q}{P}       
  := 
  \left\{ 
    \begin{array}{ccc} 
      Q & & x \nameeq \quotep{P} \\
      \dropn{x} & & otherwise \\
    \end{array}
  \right.
\end{mathpar}
 

where

\begin{eqnarray}
  (x)\id{\{} \lpquote Q \rpquote / \lpquote P \rpquote \id{\}}            = 
  \left\{ 
    \begin{array}{ccc}
      \lpquote Q \rpquote & & x \nameeq \lpquote P \rpquote \\
      x & & otherwise \\
    \end{array}
  \right. \nonumber
\end{eqnarray}

and $z$ is chosen distinct from $\quotep{P}$, $\quotep{Q}$, the free
names in $Q$, and all the names in $R$. Our $\alpha$-equivalence will
be built in the standard way from this substitution.

\begin{remark}\label{rem:no_self_referential_names}
  One consequence of these definitions is that $\forall P. \quotep{P}
  \not\in \freenames{P}$.
\end{remark}

\subsection{ Dynamic quote: an example }

Anticipating something of what's to come, consider applying the
substitution, $\widehat{\id{\{}u / z \id{\}}}$, to the following pair
of processes, $\lift{w}{y!(z)}$ and $w[ \lpquote y!(z) \rpquote ]$.

\begin{eqnarray}
	\lift{w}{y!(z)}\widehat{\id{\{}u / z \id{\}}}
		& = &
		\lift{w}{y!(u)} \nonumber\\
	w[ \lpquote y!(z) \rpquote ] \widehat{ \id{\{}u / z \id{\}} }
		& = &
		w[ \lpquote y!(z) \rpquote ] \nonumber
\end{eqnarray}

Because the body of the process between quotes is impervious to
substitution, we get radically different answers. In fact, by
examining the first process in an input context,
e.g. $x?(z).\lift{w}{y!(z)}$, we see that the process under the lift
operator may be shaped by prefixed inputs binding a name inside it. In
this sense, the lift operator will be seen as a way to dynamically
construct processes before reifying them as names.

Finally equipped with these standard features we can present the
dynamics of the calculus.

\subsubsection{Operational semantics} 

Finally, we introduce the computational dynamics. What marks these
algebras as distinct from other more traditionally studied algebraic
structures, e.g. vector spaces or polynomial rings, is the manner in
which dynamics is captured. In traditional structures, dynamics is typically
expressed through morphisms between such structures, as in linear maps
between vector spaces or morphisms between rings. In algebras
associated with the semantics of computation, the dynamics is
expressed as part of the algebraic structure itself, through a
reduction reduction relation typically denoted by $\red$. Below, we
give a recursive presentation of this relation for the calculus used
in the encoding.

$\red \subseteq \pi \times \pi$
$\red : \pi \to \mathcal{P}(\pi)$

\begin{mathpar}
  \inferrule* [lab=Comm] { \textsf{match}( x_{src}, x_{trgt} ) } { x_{trgt}?(y)P \; | \; x_{src}!\langle {Q} \rangle \red P\{\quotep{Q}/y}\} }
  \and \\
  \inferrule* [lab=Par] {{P} \red {P}'} {{{P} | {Q}} \red {{P}' | {Q}}}
  \and
  \inferrule* [lab=Equiv]{{{P} \scong {P}'} \andalso {{P}' \red {Q}'} \andalso {{Q}' \scong {Q}}}{{P} \red {Q}}
\end{mathpar}

\begin{eqnarray*}
  match_{\equiv} (\quotep{P},\quotep{Q}) & := & P \equiv Q \\
  match_{\dagger}(\quotep{P},\quotep{Q}) & := & \forall R. P|Q \red^{*} R => R \red^{*} 0 \\
  match_{K}(\quotep{P},\quotep{Q}) & := & K \mbox{ for some context } K
\end{eqnarray*}

$u?(x)P | u!\langle Q \rangle \red P\{\quotep{Q}/x\}$

%We write $\wred$ for $\red^*$, and $P\red$ if $\exists Q $ such that $ P \red Q$.
We write $P\red$ if $\exists Q $ such that $ P \red Q$ and $P\not\red$, otherwise.

\section{Replication}

As mentioned before, it is known that replication (and hence
recursion) can be implemented in a higher-order process algebra
\cite{SangiorgiWalker}. As our first example of calculation with the
machinery thus far presented we give the construction explicitly in
the {\rhoc}.

\begin{eqnarray}
	D_{x} & := & \prefix{x}{y}{(\binpar{\outputp{x}{y}}{@{y}})} \nonumber\\
	\bangp_{x}{P} & := & \binpar{{x}!\langle{\binpar{D_{x}}{P}}\rangle}{D_{x}} \nonumber
\end{eqnarray}

\begin{eqnarray}
	\bangp_{x}{P} & & \nonumber\\
	=
	& {x}!\langle{(\prefix{x}{y}{(\outputp{x}{y} | @{y})) | P}}\rangle 
	      | \prefix{x}{y}{(\outputp{x}{y} | @{y})} & \nonumber\\
	\red
	& (\outputp{x}{y} | @{y})\substn{\quotep{(\prefix{x}{y}{(@{y} | \outputp{x}{y})) | P}}}{y} & \nonumber\\
	=
	& \outputp{x}{\quotep{(\prefix{x}{y}{(\outputp{x}{y} | @{y})) | P}}}
	  | {(\prefix{x}{y}{(\outputp{x}{y} | @{y})) | P}} & \nonumber\\
	\red
	& \ldots & \nonumber\\
	\red^*
	& P | P | \ldots & \nonumber
\end{eqnarray}

Of course, this encoding, as an implementation, runs away, unfolding
$\bangp{P}$ eagerly. A lazier and more implementable replication
operator, restricted to input-guarded processes, may be obtained as follows.

\begin{eqnarray}
\bangp{\prefix{u}{v}{P}} 
	:= 
	\binpar{\lift{x}{\prefix{u}{v}{(\binpar{D(x)}{P})}}}{D(x)} \nonumber
\end{eqnarray}

\begin{remark}
  Note that the lazier definition still does not deal with summation
  or mixed summation (i.e. sums over input and output). The reader is
  invited to construct definitions of replication that deal with these
  features. 

  Further, the definitions are parameterized in a name, $x$. Can you,
  gentle reader, make a definition that eliminates this parameter and
  guarantees no accidental interaction between the replication
  machinery and the process being replicated -- i.e. no accidental
  sharing of names used by the process to get its work done and the
  name(s) used by the replication to effect copying. This latter
  revision of the definition of replication is crucial to obtaining
  the expected identity $!!P \sim !P$.
\end{remark}

\begin{remark}\label{rem:paradoxical_combinator}
  The reader familiar with the lambda calculus will have noticed the
  similarity between $D$ and the paradoxical combinator.

  [Ed. note: the existence of this seems to suggest we have to be more
  restrictive on the set of processes and names we admit if we are to
  support no-cloning.]
\end{remark}

\subsubsection{Bisimulation}

The computational dynamics gives rise to another kind of equivalence,
the equivalence of computational behavior. As previously mentioned
this is typically captured \emph{via} some form of bisimulation.

% The notion we use in this paper is weak barbed bisimulation
% \cite{milner91polyadicpi}.

The notion we use in this paper is derived from weak barbed
bisimulation \cite{milner91polyadicpi}. 

\begin{definition}
An \emph{observation relation}, $\downarrow_{\mathcal N}$, over a set
of names, $\mathcal N$, is the smallest relation satisfying the rules
below.

\infrule[Out-barb]{y \in {\mathcal N}, \; x \nameeq y}
		  {\outputp{x}{v} \downarrow_{\mathcal N} x}
\infrule[Par-barb]{\mbox{$P\downarrow_{\mathcal N} x$ or $Q\downarrow_{\mathcal N} x$}}
		  {\binpar{P}{Q} \downarrow_{\mathcal N} x}

We write $P \Downarrow_{\mathcal N} x$ if there is $Q$ such that 
$P \wred Q$ and $Q \downarrow_{\mathcal N} x$.
\end{definition}

\begin{definition}
%\label{def.bbisim}
An  ${\mathcal N}$-\emph{barbed bisimulation} over a set of names, ${\mathcal N}$, is a symmetric binary relation 
${\mathcal S}_{\mathcal N}$ between agents such that $P\rel{S}_{\mathcal N}Q$ implies:
\begin{enumerate}
\item If $P \red P'$ then $Q \wred Q'$ and $P'\rel{S}_{\mathcal N} Q'$.
\item If $P\downarrow_{\mathcal N} x$, then $Q\Downarrow_{\mathcal N} x$.
\end{enumerate}
$P$ is ${\mathcal N}$-barbed bisimilar to $Q$, written
$P \wbbisim_{\mathcal N} Q$, if $P \rel{S}_{\mathcal N} Q$ for some ${\mathcal N}$-barbed bisimulation ${\mathcal S}_{\mathcal N}$.
\end{definition}

$\mathcal{R} \subseteq \pi \times \pi$

$P \mathcal{R} Q => \forall P'. P \red P' \Rightarrow \exists Q'. Q \red Q', P' \mathcal{R} Q'$

$P \vdash x \Rightarrow Q \vdash x$

\begin{mathpar}
  \inferrule*[lab=Out-barb]{x \nameeq y}{{y}!\langle{Q}\rangle \vdash x}
  \and
  \inferrule*[lab=Par-barb]{\mbox{$P\vdash x$ or $Q\vdash x$}}{\binpar{P}{Q} \vdash x}
\end{mathpar}

\subsubsection{Contexts}

One of the principle advantages of computational calculi like the
$\pi$-calculus is a well-defined notion of context,
contextual-equivalence and a correlation between
contextual-equivalence and notions of bisimulation. The notion of
context allows the decomposition of a process into (sub-)process and
its syntactic environment, its context. Thus, a context may be
thought of as a process with a ``hole'' (written $\Box$) in it. The
application of a context $M$ to a process $P$, written $M[P]$, is
tantamount to filling the hole in $M$ with $P$. In this paper we do
not need the full weight of this theory, but do make use of the notion
of context in the proof the main theorem. 

\begin{mathpar}
  \inferrule* [lab=summation] {} {{M_{M},M_{N}} \bc \Box \;|\; x.M_{A} \;|\; M_{M}+M_{N}}
  \and
  \inferrule* [lab=agent] {} {{M_{A}} \bc (\vec{x})M_{P} \;| \; \clift{P_0,\ldots,M_{P},\ldots,P_N}}
  \and \\
  \inferrule* [lab=process] {} {{M_{P}} \bc M_{N} \;| \;P|M_{P} }
\end{mathpar} 

\begin{mathpar}
  \inferrule* [lab=sychronization] {} {M_{N} \bc \Box \;|\; x?M_{F} \;|\; x!M_{C}}
  \and
  \inferrule* [lab=abstraction] {} {{M_{F}} \bc (x)M_{P} }
  \and
  \inferrule* [lab=concretion] {} {{M_{C}} \bc \langle M_{P} \rangle }
  \and \\
  \inferrule* [lab=process] {} {{M_{P}} \bc M_{N} \;| \;P|M_{P} }
\end{mathpar}

\begin{definition}[contextual application] Given a context $M$, and
  process $P$, we define the \emph{contextual application}, $M[P] :=
  M\{P/\Box\}$. That is, the contextual application of M to P is the
  substitution of $P$ for $\Box$ in $M$.
\end{definition}

$\meaningof{-} : L \to \mathcal{P}(\pi)$

\begin{mathpar}
  \inferrule* [lab=collection] {} {\meaningof{true} = \pi, \and \meaningof{~E} = \pi \setminus \meaningof{E}, \and \meaningof{E_{1} \& E_{2}} = \meaningof{E_{1}} \cap \meaningof{E_{2}}}
\end{mathpar}

\begin{mathpar}
  \inferrule* [lab=structure] {} {\meaningof{0} = \{ P \in \pi | P \equiv 0 \}, \and \\ \meaningof{E_1 | E_2} = \{ P \in \pi | P \equiv P_{1} | P_{2}, P_{1} \in \meaningof{E_{1}}, P_{2} \in \meaningof{E_2}\} }
\end{mathpar}

\begin{mathpar}
 \inferrule* [lab=behavior] {} {\meaningof{\langle a?b \rangle E} = \{ P \in \pi | P \equiv Q | u?(y)P', \\ \and \\\\ \and \\ \;\;\; u \in \meaningof{a}, \forall z.P'\{z/y\} \in \meaningof{E\{z/b\}}\}, \and \\ \meaningof{a!E} = \{ P \in \pi | P \equiv Q | x!\langle P' \rangle, x \in \meaningof{a} P' \in \meaningof{E}\} }
\end{mathpar}

\begin{mathpar}
 \inferrule* [lab=nominal] {} {\meaningof{\quotep{E}} = \{ \quotep{P} \in \quotep{\pi} | P \in \meaningof{E} \}, \and \meaningof{\quotep{P}} = \{ \quotep{Q} \in \quotep{\pi} | P \equiv Q \} \and \\ \meaningof{@\quotep{E}} = \{ P \in \pi | P \equiv @x, x \in \meaningof{E} \}}
\end{mathpar}

\begin{eqnarray*}
  \\
  \meaningof{-} : TS \to ST
\end{eqnarray*}

\begin{eqnarray*}
  \\
  L : TS \to ST
\end{eqnarray*}

\begin{eqnarray*}
  \\
  P \models E \iff P \in \meaningof{E}
\end{eqnarray*}

\begin{eqnarray*}
  P \approx_{L} Q \iff \forall E \in L. P \models E \iff Q \models E
\end{eqnarray*}

\begin{eqnarray*}
  P \approx_{K} Q
\end{eqnarray*}

\begin{eqnarray*}
  P \approx Q
\end{eqnarray*}

$\approx_{K} = \approx = \approx_{L}$

\subsubsection{Contextual duality}

Note that contexts extend the quotation operation to a family of
operations from processes to names. Given a context, $M$, we can
define a \emph{nominal context}, $\quotep{M}$ by $\quotep{M}[P] :=
\quotep{M[P]}$. To foreshadow what is to come we observe that these
operations enjoy a duality with processes very much like the duality
between vectors and maps from vectors to scalars.

Further, because the calculus is essentially higher-order, we have a
correspondence between contexts and processes. More specifically,
given a name $x$ and a context $M$ we can construct $M^{*}_{x}$ such
that 

\begin{mathpar}
  M^{*}_{x} | \lift{x}{P} \red M[P]
\end{mathpar}

namely,

\begin{mathpar}
  M^{*}_{x} := x?(u).M[\dropn{u}]
\end{mathpar}

The dependence of $M^{*}_{x}$ on a name makes it an abstraction, 

\begin{mathpar}
  M^{*} := (x)x?(u).M[\dropn{u}]
\end{mathpar}

\subsection{Additional notation}

It will sometimes be convenient to denote the process a name
quotes. We already have the notation $x = \quotep{P}$, but it will be
convenient to introduce an alternate notation, $\procn{x}$, when we
want to emphasize the connection to the use of the name. Note that, by
virtue of name equivalence, $\quotep{\procn{x}} \nameeq x$; so, the
notation is consistent with previous definitions.

Further, because names have structure it is possible to effect
substitutions on the basis of that structure. This means we need to
upgrade our notation for substitutions, which we accomplish by
adapting comprehension notation. Thus,

\begin{mathpar}
  P\{ y / x : x \in S \}
\end{mathpar}

is interpreted to mean the process derived from P by replacing (in a
capture-avoiding manner) each occurrence of $x$ in $S$ by $y$. For example,

\begin{mathpar}
  P\{ \quotep{\procn{x}|\procn{x}} / x : x \in \freenames{P} \}
\end{mathpar}

will replace each (occurrence) of a free name $x$ in $P$ by
$\quotep{\procn{x}|\procn{x}}$.

Also, we will avail ourselves of the notation $x^{L}$ and $x^{R}$ to
denote injections of a name into disjoint copies of the name
space. There are numerous ways to accomplish this. One example can be
found in \cite{MeredithR05}. This notation overloads to vectors of
names: $\vec{x}^{\pi} := (x_{i}^{\pi} \; : \; 0 \leq i < |\vec{x}| )$ where $\pi \in \{L,R\}$.

We also use $P^{\Box} := P|\Box$.

In \cite{MeredithR05} an interpretation of the new operator is
given. It turns out that there are several possible interpretations
all enjoying the requisite algebraic properties of the operator (see
\cite{milner91polyadicpi}). We will therefore make liberal use of
$(\nu\; \vec{x})P$.

% subsection the_syntax_and_semantics_of_the_notation_system (end)   

\input{qm2pi.qmops} 

\input{qm2pi.sterngerlach} 

\input{qm2pi.metric} 

% section concurrent_process_calculi (end)

%\input{qm2pi.proofsketch}

% section proof sketch (end)

%\input{qm2pi.slviaknots} 

% section spatial logic via knots (end)

\input{qm2pi.conclusion}

% section conclusion (end)

%\input{qm2pi.dtcodes} 

% section wiring algorithm (end)

\input{qm2pi.ack} 

% section acknowledgments (end)

\newpage


\bibliographystyle{plain}   
\bibliography{../../biblios/main.bib}

\input{qm2pi.rhodetails}

\end{document}

 

% section acknowledgments (end)

\newpage


\bibliographystyle{plain}   
\bibliography{../../biblios/main.bib}

\documentclass[12pt]{llncs}
%\documentclass{jktr}

\usepackage[pdftex]{hyperref}                   
\usepackage {listings}
\usepackage {mathpartir}
\usepackage{bcprules}
%\usepackage{listings}
                       
\usepackage{graphicx} 
%\usepackage[margins=2.5cm,nohead,nofoot]{geometry}
%\usepackage{geometry}
\usepackage{amsfonts}
\usepackage{amstext}
\usepackage{latexsym}
\usepackage{amssymb}
\usepackage{color}


%\include{myPreamble}
\include{qm2pi.local} 

%\ifpdf
%\usepackage[pdftex]{graphicx}
%\else
%\usepackage{graphicx}
%\fi

 % \ifpdf
%  \usepackage{pdfsync}
%  \if


%\title{Brief Article}
%\author{David F. Snyder}
%\author{L.G. Meredith}

%\address{Dept. of Math., Texas State University--San Marcos, San Marcos, TX 78666}
       
\pagestyle{empty}


\begin{document}

\lstset{language=[Objective]Caml,frame=shadowbox}

\input{qm2pi.front}

% section front matter (end)

\input{qm2pi.intro} 
 
% section introduction (end)

% \input{qm2pi.knotations} 

% section notation (end)

\input{qm2pi.process.calculi} 

% section concurrent_process_calculi_and_spatial_logics_ (end)
    
%\input{qm2pi.knots2pi} 

%\input{qm2pi.trefoil} 

%\input{qm2pi.mainthm} 

% subsection basic_interpretation (end)

%\input{qm2pi.rho.presentation} 
\subsection{The syntax and semantics of the notation system}\label{sub:the_syntax_and_semantics_of_the_notation_system} % (fold)

We now summarize a technical presentation of the calculus that
embodies our theory of dynamics. The typical presentation of such a
calculus follows the style of giving generators and relations on
them. The grammar, below, describing term constructors, freely
generates the set of processes, $\Proc$. This set is then quotiented
by a relation known as structural congruence and it is over this set
that the notion of dynamics is expressed. This presentation is
essentially that of \cite{MeredithR05} with the addition of
polyadicity and summation. For readability we have relegated some of
the technical subtleties to an appendix.

\subsubsection{Process grammar}\label{subsub:process_grammar}

\begin{mathpar}
  \inferrule* [lab=synchronization] {} {{M} \bc \pzero \;|\; x?F \;|\; x!C }
  \and
  \inferrule* [lab=abstraction] {} {{F} \bc (x)P}
  \and
  \inferrule* [lab=concretion] {} {{C} \bc \langle Q \rangle}
  \and
  \inferrule* [lab=process] {} {{P,Q} \bc M \;| \;P|Q \;|\; @{x}}
  \and
  \inferrule* [lab=name] {} {{x} \bc \quotep{P}}
\end{mathpar} 

Note that $\vec{x}$ (resp. $\vec{P}$) denotes a vector of names
(resp. processes) of length $|\vec{x}|$ (resp. $|\vec{P}|$). We adopt
the following useful abbreviations.

\begin{mathpar}
   x?(\vec{y}).P := x.(\vec{y})P \and  x\clift{\vec{P}} := x.\clift{\vec{P}}
   \and x!(y) := \lift{x}{\dropn{y}}
   \and \Pi_{i=0}^{n-1}P_i := P_0 | \ldots | P_{n-1}
\end{mathpar}

\subsubsection{Structural congruence}

\paragraph{Free and bound names and alpha-equivalence.} At the
core of structural equivalence is alpha-equivalence which identifies
process that are the same up to a change of variable. Formally, we
recognize the distinction between free and bound names. The free names
of a process, $\freenames{P}$, may be calculated recursively as
follows:

\begin{mathpar}
\freenames{\pzero} := \emptyset
  \and \\
  \freenames{x?(y).P} := \{ x \} \cup (\freenames{P} \setminus \{ y \})
  \and 
  \freenames{x!\langle P \rangle} := \{ x \} \cup \{ P \} 
  \and \\
  \freenames{P|Q} := \freenames{P} \cup \freenames{Q}
  \and \\
  \freenames{@{x}} := \{ x \}
\end{mathpar}

$\pi$
$\quotep{\pi}$

$\freenames{-} : \pi \to \mathcal{P}(\quotep{\pi})$

\begin{eqnarray*}
  \freenames{\pzero} & := & \emptyset \\
  \freenames{x?(y).P} & := & \{ x \} \cup (\freenames{P} \setminus \{ y \}) \\
  \freenames{x!\langle P \rangle} & := & \{ x \} \cup \{ P \} \\
  \freenames{P|Q} & := & \freenames{P} \cup \freenames{Q} \\
  \freenames{\dropn{x}} & := & \{ x \}
\end{eqnarray*}

The bound names of a process, $\boundnames{P}$, are those names occurring in $P$
that are not free. For example, in $x?(y).0$, the name $x$ is free, while $y$ is bound.

\begin{mathpar}
  \inferrule* [lab=monoidal-laws] {} { P|Q \equiv Q|P \and P|0 \equiv P \and P|(Q|R) \equiv (P|Q)|R }
\end{mathpar}

\begin{mathpar}
  \inferrule* [lab=alpha-equivalence] {} { (x)P \equiv (y)P\{y/x\} \and y \not\in \freenames{P} }
\end{mathpar}

\begin{definition}
Then two processes, $P,Q$, are alpha-equivalent if $P = Q\{\vec{y}/\vec{x}\}$ for
some $\vec{x} \in \boundnames{Q},\vec{y} \in \boundnames{P}$, where $Q\{\vec{y}/\vec{x}\}$
denotes the capture-avoiding substitution of $\vec{y}$ for $\vec{x}$ in $Q$.
\end{definition}

\begin{definition}
  The {\em structural congruence} \cite{SangiorgiWalker} , $\equiv$,
  between processes is the least congruence containing
  alpha-equivalence, satisfying the abelian monoid laws
  (associativity, commutativity and $\pzero$ as identity) for parallel
  composition $|$ and for summation $+$.
\end{definition}

\subsection{Name equivalence}

We take name equivalence, written $\nameeq$, to be the smallest
equivalence relation generated by the following rules.

\begin{mathpar}
\inferrule*[lab=Quote-drop]
{ }
{ \quotep{@{x}} \nameeq x }

\inferrule*[lab=Struct-equiv]
{ P \scong Q }
{ \quotep{P} \nameeq \quotep{Q} }
\end{mathpar}

The astute reader will have noticed that the mutual recursion of names
and processes imposes a mutual recursion on alpha-equivalence and
structural equivalence via name-equivalence. Fortunately, all of this
works out pleasantly and we may calculate in the natural way, free of
concern. The reader interested in the details is referred to the
appendix \ref{appendix:rho_details}.

\subsection{Substitution}

We use $\Proc$ for the set of processes, $\QProc$ for the set of
names, and $\id{\{}\vec{y} / \vec{x} \id{\}}$ to denote partial maps,
$s : \QProc \rightarrow \QProc$. A map, $s$ lifts, uniquely, to a map
on process terms, $\widehat{s} : \Proc \rightarrow \Proc$ by the
following equations.

\begin{mathpar}
  (0) \psubstp{Q}{P} := 0 \\
  (R \juxtap S) \psubstp{Q}{P}
  :=    
  (R)\psubstp{Q}{P} \juxtap (S) \psubstp{Q}{P} \\
  (x?(y).R) \psubstp{Q}{P}    
  :=    
  (x)\substp{Q}{P} (z)\concat( (R \psubstn{z}{y}) \psubstp{Q}{P} ) \\
  (\lift{x}{R}) \psubstp{Q}{P}  
  :=
  \lift{(x)\substp{Q}{P}}{ R \psubstp{Q}{P} } \\
%   (\dropn{x})  \psubstp{Q}{P}       
%   := 
%   \left\{ 
%     \begin{array}{ccc} 
%       \dropn{\quotep{Q}} & & x \nameeq \quotep{P} \\
%       \dropn{x} & & otherwise \\
%     \end{array}
%   \right. 
  (\dropn{x})  \psubstp{Q}{P}       
  := 
  \left\{ 
    \begin{array}{ccc} 
      Q & & x \nameeq \quotep{P} \\
      \dropn{x} & & otherwise \\
    \end{array}
  \right.
\end{mathpar}
 

where

\begin{eqnarray}
  (x)\id{\{} \lpquote Q \rpquote / \lpquote P \rpquote \id{\}}            = 
  \left\{ 
    \begin{array}{ccc}
      \lpquote Q \rpquote & & x \nameeq \lpquote P \rpquote \\
      x & & otherwise \\
    \end{array}
  \right. \nonumber
\end{eqnarray}

and $z$ is chosen distinct from $\quotep{P}$, $\quotep{Q}$, the free
names in $Q$, and all the names in $R$. Our $\alpha$-equivalence will
be built in the standard way from this substitution.

\begin{remark}\label{rem:no_self_referential_names}
  One consequence of these definitions is that $\forall P. \quotep{P}
  \not\in \freenames{P}$.
\end{remark}

\subsection{ Dynamic quote: an example }

Anticipating something of what's to come, consider applying the
substitution, $\widehat{\id{\{}u / z \id{\}}}$, to the following pair
of processes, $\lift{w}{y!(z)}$ and $w[ \lpquote y!(z) \rpquote ]$.

\begin{eqnarray}
	\lift{w}{y!(z)}\widehat{\id{\{}u / z \id{\}}}
		& = &
		\lift{w}{y!(u)} \nonumber\\
	w[ \lpquote y!(z) \rpquote ] \widehat{ \id{\{}u / z \id{\}} }
		& = &
		w[ \lpquote y!(z) \rpquote ] \nonumber
\end{eqnarray}

Because the body of the process between quotes is impervious to
substitution, we get radically different answers. In fact, by
examining the first process in an input context,
e.g. $x?(z).\lift{w}{y!(z)}$, we see that the process under the lift
operator may be shaped by prefixed inputs binding a name inside it. In
this sense, the lift operator will be seen as a way to dynamically
construct processes before reifying them as names.

Finally equipped with these standard features we can present the
dynamics of the calculus.

\subsubsection{Operational semantics} 

Finally, we introduce the computational dynamics. What marks these
algebras as distinct from other more traditionally studied algebraic
structures, e.g. vector spaces or polynomial rings, is the manner in
which dynamics is captured. In traditional structures, dynamics is typically
expressed through morphisms between such structures, as in linear maps
between vector spaces or morphisms between rings. In algebras
associated with the semantics of computation, the dynamics is
expressed as part of the algebraic structure itself, through a
reduction reduction relation typically denoted by $\red$. Below, we
give a recursive presentation of this relation for the calculus used
in the encoding.

$\red \subseteq \pi \times \pi$
$\red : \pi \to \mathcal{P}(\pi)$

\begin{mathpar}
  \inferrule* [lab=Comm] { \textsf{match}( x_{src}, x_{trgt} ) } { x_{trgt}?(y)P \; | \; x_{src}!\langle {Q} \rangle \red P\{\quotep{Q}/y}\} }
  \and \\
  \inferrule* [lab=Par] {{P} \red {P}'} {{{P} | {Q}} \red {{P}' | {Q}}}
  \and
  \inferrule* [lab=Equiv]{{{P} \scong {P}'} \andalso {{P}' \red {Q}'} \andalso {{Q}' \scong {Q}}}{{P} \red {Q}}
\end{mathpar}

\begin{eqnarray*}
  match_{\equiv} (\quotep{P},\quotep{Q}) & := & P \equiv Q \\
  match_{\dagger}(\quotep{P},\quotep{Q}) & := & \forall R. P|Q \red^{*} R => R \red^{*} 0 \\
  match_{K}(\quotep{P},\quotep{Q}) & := & K \mbox{ for some context } K
\end{eqnarray*}

$u?(x)P | u!\langle Q \rangle \red P\{\quotep{Q}/x\}$

%We write $\wred$ for $\red^*$, and $P\red$ if $\exists Q $ such that $ P \red Q$.
We write $P\red$ if $\exists Q $ such that $ P \red Q$ and $P\not\red$, otherwise.

\section{Replication}

As mentioned before, it is known that replication (and hence
recursion) can be implemented in a higher-order process algebra
\cite{SangiorgiWalker}. As our first example of calculation with the
machinery thus far presented we give the construction explicitly in
the {\rhoc}.

\begin{eqnarray}
	D_{x} & := & \prefix{x}{y}{(\binpar{\outputp{x}{y}}{@{y}})} \nonumber\\
	\bangp_{x}{P} & := & \binpar{{x}!\langle{\binpar{D_{x}}{P}}\rangle}{D_{x}} \nonumber
\end{eqnarray}

\begin{eqnarray}
	\bangp_{x}{P} & & \nonumber\\
	=
	& {x}!\langle{(\prefix{x}{y}{(\outputp{x}{y} | @{y})) | P}}\rangle 
	      | \prefix{x}{y}{(\outputp{x}{y} | @{y})} & \nonumber\\
	\red
	& (\outputp{x}{y} | @{y})\substn{\quotep{(\prefix{x}{y}{(@{y} | \outputp{x}{y})) | P}}}{y} & \nonumber\\
	=
	& \outputp{x}{\quotep{(\prefix{x}{y}{(\outputp{x}{y} | @{y})) | P}}}
	  | {(\prefix{x}{y}{(\outputp{x}{y} | @{y})) | P}} & \nonumber\\
	\red
	& \ldots & \nonumber\\
	\red^*
	& P | P | \ldots & \nonumber
\end{eqnarray}

Of course, this encoding, as an implementation, runs away, unfolding
$\bangp{P}$ eagerly. A lazier and more implementable replication
operator, restricted to input-guarded processes, may be obtained as follows.

\begin{eqnarray}
\bangp{\prefix{u}{v}{P}} 
	:= 
	\binpar{\lift{x}{\prefix{u}{v}{(\binpar{D(x)}{P})}}}{D(x)} \nonumber
\end{eqnarray}

\begin{remark}
  Note that the lazier definition still does not deal with summation
  or mixed summation (i.e. sums over input and output). The reader is
  invited to construct definitions of replication that deal with these
  features. 

  Further, the definitions are parameterized in a name, $x$. Can you,
  gentle reader, make a definition that eliminates this parameter and
  guarantees no accidental interaction between the replication
  machinery and the process being replicated -- i.e. no accidental
  sharing of names used by the process to get its work done and the
  name(s) used by the replication to effect copying. This latter
  revision of the definition of replication is crucial to obtaining
  the expected identity $!!P \sim !P$.
\end{remark}

\begin{remark}\label{rem:paradoxical_combinator}
  The reader familiar with the lambda calculus will have noticed the
  similarity between $D$ and the paradoxical combinator.

  [Ed. note: the existence of this seems to suggest we have to be more
  restrictive on the set of processes and names we admit if we are to
  support no-cloning.]
\end{remark}

\subsubsection{Bisimulation}

The computational dynamics gives rise to another kind of equivalence,
the equivalence of computational behavior. As previously mentioned
this is typically captured \emph{via} some form of bisimulation.

% The notion we use in this paper is weak barbed bisimulation
% \cite{milner91polyadicpi}.

The notion we use in this paper is derived from weak barbed
bisimulation \cite{milner91polyadicpi}. 

\begin{definition}
An \emph{observation relation}, $\downarrow_{\mathcal N}$, over a set
of names, $\mathcal N$, is the smallest relation satisfying the rules
below.

\infrule[Out-barb]{y \in {\mathcal N}, \; x \nameeq y}
		  {\outputp{x}{v} \downarrow_{\mathcal N} x}
\infrule[Par-barb]{\mbox{$P\downarrow_{\mathcal N} x$ or $Q\downarrow_{\mathcal N} x$}}
		  {\binpar{P}{Q} \downarrow_{\mathcal N} x}

We write $P \Downarrow_{\mathcal N} x$ if there is $Q$ such that 
$P \wred Q$ and $Q \downarrow_{\mathcal N} x$.
\end{definition}

\begin{definition}
%\label{def.bbisim}
An  ${\mathcal N}$-\emph{barbed bisimulation} over a set of names, ${\mathcal N}$, is a symmetric binary relation 
${\mathcal S}_{\mathcal N}$ between agents such that $P\rel{S}_{\mathcal N}Q$ implies:
\begin{enumerate}
\item If $P \red P'$ then $Q \wred Q'$ and $P'\rel{S}_{\mathcal N} Q'$.
\item If $P\downarrow_{\mathcal N} x$, then $Q\Downarrow_{\mathcal N} x$.
\end{enumerate}
$P$ is ${\mathcal N}$-barbed bisimilar to $Q$, written
$P \wbbisim_{\mathcal N} Q$, if $P \rel{S}_{\mathcal N} Q$ for some ${\mathcal N}$-barbed bisimulation ${\mathcal S}_{\mathcal N}$.
\end{definition}

$\mathcal{R} \subseteq \pi \times \pi$

$P \mathcal{R} Q => \forall P'. P \red P' \Rightarrow \exists Q'. Q \red Q', P' \mathcal{R} Q'$

$P \vdash x \Rightarrow Q \vdash x$

\begin{mathpar}
  \inferrule*[lab=Out-barb]{x \nameeq y}{{y}!\langle{Q}\rangle \vdash x}
  \and
  \inferrule*[lab=Par-barb]{\mbox{$P\vdash x$ or $Q\vdash x$}}{\binpar{P}{Q} \vdash x}
\end{mathpar}

\subsubsection{Contexts}

One of the principle advantages of computational calculi like the
$\pi$-calculus is a well-defined notion of context,
contextual-equivalence and a correlation between
contextual-equivalence and notions of bisimulation. The notion of
context allows the decomposition of a process into (sub-)process and
its syntactic environment, its context. Thus, a context may be
thought of as a process with a ``hole'' (written $\Box$) in it. The
application of a context $M$ to a process $P$, written $M[P]$, is
tantamount to filling the hole in $M$ with $P$. In this paper we do
not need the full weight of this theory, but do make use of the notion
of context in the proof the main theorem. 

\begin{mathpar}
  \inferrule* [lab=summation] {} {{M_{M},M_{N}} \bc \Box \;|\; x.M_{A} \;|\; M_{M}+M_{N}}
  \and
  \inferrule* [lab=agent] {} {{M_{A}} \bc (\vec{x})M_{P} \;| \; \clift{P_0,\ldots,M_{P},\ldots,P_N}}
  \and \\
  \inferrule* [lab=process] {} {{M_{P}} \bc M_{N} \;| \;P|M_{P} }
\end{mathpar} 

\begin{mathpar}
  \inferrule* [lab=sychronization] {} {M_{N} \bc \Box \;|\; x?M_{F} \;|\; x!M_{C}}
  \and
  \inferrule* [lab=abstraction] {} {{M_{F}} \bc (x)M_{P} }
  \and
  \inferrule* [lab=concretion] {} {{M_{C}} \bc \langle M_{P} \rangle }
  \and \\
  \inferrule* [lab=process] {} {{M_{P}} \bc M_{N} \;| \;P|M_{P} }
\end{mathpar}

\begin{definition}[contextual application] Given a context $M$, and
  process $P$, we define the \emph{contextual application}, $M[P] :=
  M\{P/\Box\}$. That is, the contextual application of M to P is the
  substitution of $P$ for $\Box$ in $M$.
\end{definition}

$\meaningof{-} : L \to \mathcal{P}(\pi)$

\begin{mathpar}
  \inferrule* [lab=collection] {} {\meaningof{true} = \pi, \and \meaningof{~E} = \pi \setminus \meaningof{E}, \and \meaningof{E_{1} \& E_{2}} = \meaningof{E_{1}} \cap \meaningof{E_{2}}}
\end{mathpar}

\begin{mathpar}
  \inferrule* [lab=structure] {} {\meaningof{0} = \{ P \in \pi | P \equiv 0 \}, \and \\ \meaningof{E_1 | E_2} = \{ P \in \pi | P \equiv P_{1} | P_{2}, P_{1} \in \meaningof{E_{1}}, P_{2} \in \meaningof{E_2}\} }
\end{mathpar}

\begin{mathpar}
 \inferrule* [lab=behavior] {} {\meaningof{\langle a?b \rangle E} = \{ P \in \pi | P \equiv Q | u?(y)P', \\ \and \\\\ \and \\ \;\;\; u \in \meaningof{a}, \forall z.P'\{z/y\} \in \meaningof{E\{z/b\}}\}, \and \\ \meaningof{a!E} = \{ P \in \pi | P \equiv Q | x!\langle P' \rangle, x \in \meaningof{a} P' \in \meaningof{E}\} }
\end{mathpar}

\begin{mathpar}
 \inferrule* [lab=nominal] {} {\meaningof{\quotep{E}} = \{ \quotep{P} \in \quotep{\pi} | P \in \meaningof{E} \}, \and \meaningof{\quotep{P}} = \{ \quotep{Q} \in \quotep{\pi} | P \equiv Q \} \and \\ \meaningof{@\quotep{E}} = \{ P \in \pi | P \equiv @x, x \in \meaningof{E} \}}
\end{mathpar}

\begin{eqnarray*}
  \\
  \meaningof{-} : TS \to ST
\end{eqnarray*}

\begin{eqnarray*}
  \\
  L : TS \to ST
\end{eqnarray*}

\begin{eqnarray*}
  \\
  P \models E \iff P \in \meaningof{E}
\end{eqnarray*}

\begin{eqnarray*}
  P \approx_{L} Q \iff \forall E \in L. P \models E \iff Q \models E
\end{eqnarray*}

\begin{eqnarray*}
  P \approx_{K} Q
\end{eqnarray*}

\begin{eqnarray*}
  P \approx Q
\end{eqnarray*}

$\approx_{K} = \approx = \approx_{L}$

\subsubsection{Contextual duality}

Note that contexts extend the quotation operation to a family of
operations from processes to names. Given a context, $M$, we can
define a \emph{nominal context}, $\quotep{M}$ by $\quotep{M}[P] :=
\quotep{M[P]}$. To foreshadow what is to come we observe that these
operations enjoy a duality with processes very much like the duality
between vectors and maps from vectors to scalars.

Further, because the calculus is essentially higher-order, we have a
correspondence between contexts and processes. More specifically,
given a name $x$ and a context $M$ we can construct $M^{*}_{x}$ such
that 

\begin{mathpar}
  M^{*}_{x} | \lift{x}{P} \red M[P]
\end{mathpar}

namely,

\begin{mathpar}
  M^{*}_{x} := x?(u).M[\dropn{u}]
\end{mathpar}

The dependence of $M^{*}_{x}$ on a name makes it an abstraction, 

\begin{mathpar}
  M^{*} := (x)x?(u).M[\dropn{u}]
\end{mathpar}

\subsection{Additional notation}

It will sometimes be convenient to denote the process a name
quotes. We already have the notation $x = \quotep{P}$, but it will be
convenient to introduce an alternate notation, $\procn{x}$, when we
want to emphasize the connection to the use of the name. Note that, by
virtue of name equivalence, $\quotep{\procn{x}} \nameeq x$; so, the
notation is consistent with previous definitions.

Further, because names have structure it is possible to effect
substitutions on the basis of that structure. This means we need to
upgrade our notation for substitutions, which we accomplish by
adapting comprehension notation. Thus,

\begin{mathpar}
  P\{ y / x : x \in S \}
\end{mathpar}

is interpreted to mean the process derived from P by replacing (in a
capture-avoiding manner) each occurrence of $x$ in $S$ by $y$. For example,

\begin{mathpar}
  P\{ \quotep{\procn{x}|\procn{x}} / x : x \in \freenames{P} \}
\end{mathpar}

will replace each (occurrence) of a free name $x$ in $P$ by
$\quotep{\procn{x}|\procn{x}}$.

Also, we will avail ourselves of the notation $x^{L}$ and $x^{R}$ to
denote injections of a name into disjoint copies of the name
space. There are numerous ways to accomplish this. One example can be
found in \cite{MeredithR05}. This notation overloads to vectors of
names: $\vec{x}^{\pi} := (x_{i}^{\pi} \; : \; 0 \leq i < |\vec{x}| )$ where $\pi \in \{L,R\}$.

We also use $P^{\Box} := P|\Box$.

In \cite{MeredithR05} an interpretation of the new operator is
given. It turns out that there are several possible interpretations
all enjoying the requisite algebraic properties of the operator (see
\cite{milner91polyadicpi}). We will therefore make liberal use of
$(\nu\; \vec{x})P$.

% subsection the_syntax_and_semantics_of_the_notation_system (end)   

\input{qm2pi.qmops} 

\input{qm2pi.sterngerlach} 

\input{qm2pi.metric} 

% section concurrent_process_calculi (end)

%\input{qm2pi.proofsketch}

% section proof sketch (end)

%\input{qm2pi.slviaknots} 

% section spatial logic via knots (end)

\input{qm2pi.conclusion}

% section conclusion (end)

%\input{qm2pi.dtcodes} 

% section wiring algorithm (end)

\input{qm2pi.ack} 

% section acknowledgments (end)

\newpage


\bibliographystyle{plain}   
\bibliography{../../biblios/main.bib}

\input{qm2pi.rhodetails}

\end{document}



\end{document}

 

% section acknowledgments (end)

\newpage


\bibliographystyle{plain}   
\bibliography{../../biblios/main.bib}

\documentclass[12pt]{llncs}
%\documentclass{jktr}

\usepackage[pdftex]{hyperref}                   
\usepackage {listings}
\usepackage {mathpartir}
\usepackage{bcprules}
%\usepackage{listings}
                       
\usepackage{graphicx} 
%\usepackage[margins=2.5cm,nohead,nofoot]{geometry}
%\usepackage{geometry}
\usepackage{amsfonts}
\usepackage{amstext}
\usepackage{latexsym}
\usepackage{amssymb}
\usepackage{color}


%\include{myPreamble}
\documentclass[12pt]{llncs}
%\documentclass{jktr}

\usepackage[pdftex]{hyperref}                   
\usepackage {listings}
\usepackage {mathpartir}
\usepackage{bcprules}
%\usepackage{listings}
                       
\usepackage{graphicx} 
%\usepackage[margins=2.5cm,nohead,nofoot]{geometry}
%\usepackage{geometry}
\usepackage{amsfonts}
\usepackage{amstext}
\usepackage{latexsym}
\usepackage{amssymb}
\usepackage{color}


%\include{myPreamble}
\include{qm2pi.local} 

%\ifpdf
%\usepackage[pdftex]{graphicx}
%\else
%\usepackage{graphicx}
%\fi

 % \ifpdf
%  \usepackage{pdfsync}
%  \if


%\title{Brief Article}
%\author{David F. Snyder}
%\author{L.G. Meredith}

%\address{Dept. of Math., Texas State University--San Marcos, San Marcos, TX 78666}
       
\pagestyle{empty}


\begin{document}

\lstset{language=[Objective]Caml,frame=shadowbox}

\input{qm2pi.front}

% section front matter (end)

\input{qm2pi.intro} 
 
% section introduction (end)

% \input{qm2pi.knotations} 

% section notation (end)

\input{qm2pi.process.calculi} 

% section concurrent_process_calculi_and_spatial_logics_ (end)
    
%\input{qm2pi.knots2pi} 

%\input{qm2pi.trefoil} 

%\input{qm2pi.mainthm} 

% subsection basic_interpretation (end)

%\input{qm2pi.rho.presentation} 
\subsection{The syntax and semantics of the notation system}\label{sub:the_syntax_and_semantics_of_the_notation_system} % (fold)

We now summarize a technical presentation of the calculus that
embodies our theory of dynamics. The typical presentation of such a
calculus follows the style of giving generators and relations on
them. The grammar, below, describing term constructors, freely
generates the set of processes, $\Proc$. This set is then quotiented
by a relation known as structural congruence and it is over this set
that the notion of dynamics is expressed. This presentation is
essentially that of \cite{MeredithR05} with the addition of
polyadicity and summation. For readability we have relegated some of
the technical subtleties to an appendix.

\subsubsection{Process grammar}\label{subsub:process_grammar}

\begin{mathpar}
  \inferrule* [lab=synchronization] {} {{M} \bc \pzero \;|\; x?F \;|\; x!C }
  \and
  \inferrule* [lab=abstraction] {} {{F} \bc (x)P}
  \and
  \inferrule* [lab=concretion] {} {{C} \bc \langle Q \rangle}
  \and
  \inferrule* [lab=process] {} {{P,Q} \bc M \;| \;P|Q \;|\; @{x}}
  \and
  \inferrule* [lab=name] {} {{x} \bc \quotep{P}}
\end{mathpar} 

Note that $\vec{x}$ (resp. $\vec{P}$) denotes a vector of names
(resp. processes) of length $|\vec{x}|$ (resp. $|\vec{P}|$). We adopt
the following useful abbreviations.

\begin{mathpar}
   x?(\vec{y}).P := x.(\vec{y})P \and  x\clift{\vec{P}} := x.\clift{\vec{P}}
   \and x!(y) := \lift{x}{\dropn{y}}
   \and \Pi_{i=0}^{n-1}P_i := P_0 | \ldots | P_{n-1}
\end{mathpar}

\subsubsection{Structural congruence}

\paragraph{Free and bound names and alpha-equivalence.} At the
core of structural equivalence is alpha-equivalence which identifies
process that are the same up to a change of variable. Formally, we
recognize the distinction between free and bound names. The free names
of a process, $\freenames{P}$, may be calculated recursively as
follows:

\begin{mathpar}
\freenames{\pzero} := \emptyset
  \and \\
  \freenames{x?(y).P} := \{ x \} \cup (\freenames{P} \setminus \{ y \})
  \and 
  \freenames{x!\langle P \rangle} := \{ x \} \cup \{ P \} 
  \and \\
  \freenames{P|Q} := \freenames{P} \cup \freenames{Q}
  \and \\
  \freenames{@{x}} := \{ x \}
\end{mathpar}

$\pi$
$\quotep{\pi}$

$\freenames{-} : \pi \to \mathcal{P}(\quotep{\pi})$

\begin{eqnarray*}
  \freenames{\pzero} & := & \emptyset \\
  \freenames{x?(y).P} & := & \{ x \} \cup (\freenames{P} \setminus \{ y \}) \\
  \freenames{x!\langle P \rangle} & := & \{ x \} \cup \{ P \} \\
  \freenames{P|Q} & := & \freenames{P} \cup \freenames{Q} \\
  \freenames{\dropn{x}} & := & \{ x \}
\end{eqnarray*}

The bound names of a process, $\boundnames{P}$, are those names occurring in $P$
that are not free. For example, in $x?(y).0$, the name $x$ is free, while $y$ is bound.

\begin{mathpar}
  \inferrule* [lab=monoidal-laws] {} { P|Q \equiv Q|P \and P|0 \equiv P \and P|(Q|R) \equiv (P|Q)|R }
\end{mathpar}

\begin{mathpar}
  \inferrule* [lab=alpha-equivalence] {} { (x)P \equiv (y)P\{y/x\} \and y \not\in \freenames{P} }
\end{mathpar}

\begin{definition}
Then two processes, $P,Q$, are alpha-equivalent if $P = Q\{\vec{y}/\vec{x}\}$ for
some $\vec{x} \in \boundnames{Q},\vec{y} \in \boundnames{P}$, where $Q\{\vec{y}/\vec{x}\}$
denotes the capture-avoiding substitution of $\vec{y}$ for $\vec{x}$ in $Q$.
\end{definition}

\begin{definition}
  The {\em structural congruence} \cite{SangiorgiWalker} , $\equiv$,
  between processes is the least congruence containing
  alpha-equivalence, satisfying the abelian monoid laws
  (associativity, commutativity and $\pzero$ as identity) for parallel
  composition $|$ and for summation $+$.
\end{definition}

\subsection{Name equivalence}

We take name equivalence, written $\nameeq$, to be the smallest
equivalence relation generated by the following rules.

\begin{mathpar}
\inferrule*[lab=Quote-drop]
{ }
{ \quotep{@{x}} \nameeq x }

\inferrule*[lab=Struct-equiv]
{ P \scong Q }
{ \quotep{P} \nameeq \quotep{Q} }
\end{mathpar}

The astute reader will have noticed that the mutual recursion of names
and processes imposes a mutual recursion on alpha-equivalence and
structural equivalence via name-equivalence. Fortunately, all of this
works out pleasantly and we may calculate in the natural way, free of
concern. The reader interested in the details is referred to the
appendix \ref{appendix:rho_details}.

\subsection{Substitution}

We use $\Proc$ for the set of processes, $\QProc$ for the set of
names, and $\id{\{}\vec{y} / \vec{x} \id{\}}$ to denote partial maps,
$s : \QProc \rightarrow \QProc$. A map, $s$ lifts, uniquely, to a map
on process terms, $\widehat{s} : \Proc \rightarrow \Proc$ by the
following equations.

\begin{mathpar}
  (0) \psubstp{Q}{P} := 0 \\
  (R \juxtap S) \psubstp{Q}{P}
  :=    
  (R)\psubstp{Q}{P} \juxtap (S) \psubstp{Q}{P} \\
  (x?(y).R) \psubstp{Q}{P}    
  :=    
  (x)\substp{Q}{P} (z)\concat( (R \psubstn{z}{y}) \psubstp{Q}{P} ) \\
  (\lift{x}{R}) \psubstp{Q}{P}  
  :=
  \lift{(x)\substp{Q}{P}}{ R \psubstp{Q}{P} } \\
%   (\dropn{x})  \psubstp{Q}{P}       
%   := 
%   \left\{ 
%     \begin{array}{ccc} 
%       \dropn{\quotep{Q}} & & x \nameeq \quotep{P} \\
%       \dropn{x} & & otherwise \\
%     \end{array}
%   \right. 
  (\dropn{x})  \psubstp{Q}{P}       
  := 
  \left\{ 
    \begin{array}{ccc} 
      Q & & x \nameeq \quotep{P} \\
      \dropn{x} & & otherwise \\
    \end{array}
  \right.
\end{mathpar}
 

where

\begin{eqnarray}
  (x)\id{\{} \lpquote Q \rpquote / \lpquote P \rpquote \id{\}}            = 
  \left\{ 
    \begin{array}{ccc}
      \lpquote Q \rpquote & & x \nameeq \lpquote P \rpquote \\
      x & & otherwise \\
    \end{array}
  \right. \nonumber
\end{eqnarray}

and $z$ is chosen distinct from $\quotep{P}$, $\quotep{Q}$, the free
names in $Q$, and all the names in $R$. Our $\alpha$-equivalence will
be built in the standard way from this substitution.

\begin{remark}\label{rem:no_self_referential_names}
  One consequence of these definitions is that $\forall P. \quotep{P}
  \not\in \freenames{P}$.
\end{remark}

\subsection{ Dynamic quote: an example }

Anticipating something of what's to come, consider applying the
substitution, $\widehat{\id{\{}u / z \id{\}}}$, to the following pair
of processes, $\lift{w}{y!(z)}$ and $w[ \lpquote y!(z) \rpquote ]$.

\begin{eqnarray}
	\lift{w}{y!(z)}\widehat{\id{\{}u / z \id{\}}}
		& = &
		\lift{w}{y!(u)} \nonumber\\
	w[ \lpquote y!(z) \rpquote ] \widehat{ \id{\{}u / z \id{\}} }
		& = &
		w[ \lpquote y!(z) \rpquote ] \nonumber
\end{eqnarray}

Because the body of the process between quotes is impervious to
substitution, we get radically different answers. In fact, by
examining the first process in an input context,
e.g. $x?(z).\lift{w}{y!(z)}$, we see that the process under the lift
operator may be shaped by prefixed inputs binding a name inside it. In
this sense, the lift operator will be seen as a way to dynamically
construct processes before reifying them as names.

Finally equipped with these standard features we can present the
dynamics of the calculus.

\subsubsection{Operational semantics} 

Finally, we introduce the computational dynamics. What marks these
algebras as distinct from other more traditionally studied algebraic
structures, e.g. vector spaces or polynomial rings, is the manner in
which dynamics is captured. In traditional structures, dynamics is typically
expressed through morphisms between such structures, as in linear maps
between vector spaces or morphisms between rings. In algebras
associated with the semantics of computation, the dynamics is
expressed as part of the algebraic structure itself, through a
reduction reduction relation typically denoted by $\red$. Below, we
give a recursive presentation of this relation for the calculus used
in the encoding.

$\red \subseteq \pi \times \pi$
$\red : \pi \to \mathcal{P}(\pi)$

\begin{mathpar}
  \inferrule* [lab=Comm] { \textsf{match}( x_{src}, x_{trgt} ) } { x_{trgt}?(y)P \; | \; x_{src}!\langle {Q} \rangle \red P\{\quotep{Q}/y}\} }
  \and \\
  \inferrule* [lab=Par] {{P} \red {P}'} {{{P} | {Q}} \red {{P}' | {Q}}}
  \and
  \inferrule* [lab=Equiv]{{{P} \scong {P}'} \andalso {{P}' \red {Q}'} \andalso {{Q}' \scong {Q}}}{{P} \red {Q}}
\end{mathpar}

\begin{eqnarray*}
  match_{\equiv} (\quotep{P},\quotep{Q}) & := & P \equiv Q \\
  match_{\dagger}(\quotep{P},\quotep{Q}) & := & \forall R. P|Q \red^{*} R => R \red^{*} 0 \\
  match_{K}(\quotep{P},\quotep{Q}) & := & K \mbox{ for some context } K
\end{eqnarray*}

$u?(x)P | u!\langle Q \rangle \red P\{\quotep{Q}/x\}$

%We write $\wred$ for $\red^*$, and $P\red$ if $\exists Q $ such that $ P \red Q$.
We write $P\red$ if $\exists Q $ such that $ P \red Q$ and $P\not\red$, otherwise.

\section{Replication}

As mentioned before, it is known that replication (and hence
recursion) can be implemented in a higher-order process algebra
\cite{SangiorgiWalker}. As our first example of calculation with the
machinery thus far presented we give the construction explicitly in
the {\rhoc}.

\begin{eqnarray}
	D_{x} & := & \prefix{x}{y}{(\binpar{\outputp{x}{y}}{@{y}})} \nonumber\\
	\bangp_{x}{P} & := & \binpar{{x}!\langle{\binpar{D_{x}}{P}}\rangle}{D_{x}} \nonumber
\end{eqnarray}

\begin{eqnarray}
	\bangp_{x}{P} & & \nonumber\\
	=
	& {x}!\langle{(\prefix{x}{y}{(\outputp{x}{y} | @{y})) | P}}\rangle 
	      | \prefix{x}{y}{(\outputp{x}{y} | @{y})} & \nonumber\\
	\red
	& (\outputp{x}{y} | @{y})\substn{\quotep{(\prefix{x}{y}{(@{y} | \outputp{x}{y})) | P}}}{y} & \nonumber\\
	=
	& \outputp{x}{\quotep{(\prefix{x}{y}{(\outputp{x}{y} | @{y})) | P}}}
	  | {(\prefix{x}{y}{(\outputp{x}{y} | @{y})) | P}} & \nonumber\\
	\red
	& \ldots & \nonumber\\
	\red^*
	& P | P | \ldots & \nonumber
\end{eqnarray}

Of course, this encoding, as an implementation, runs away, unfolding
$\bangp{P}$ eagerly. A lazier and more implementable replication
operator, restricted to input-guarded processes, may be obtained as follows.

\begin{eqnarray}
\bangp{\prefix{u}{v}{P}} 
	:= 
	\binpar{\lift{x}{\prefix{u}{v}{(\binpar{D(x)}{P})}}}{D(x)} \nonumber
\end{eqnarray}

\begin{remark}
  Note that the lazier definition still does not deal with summation
  or mixed summation (i.e. sums over input and output). The reader is
  invited to construct definitions of replication that deal with these
  features. 

  Further, the definitions are parameterized in a name, $x$. Can you,
  gentle reader, make a definition that eliminates this parameter and
  guarantees no accidental interaction between the replication
  machinery and the process being replicated -- i.e. no accidental
  sharing of names used by the process to get its work done and the
  name(s) used by the replication to effect copying. This latter
  revision of the definition of replication is crucial to obtaining
  the expected identity $!!P \sim !P$.
\end{remark}

\begin{remark}\label{rem:paradoxical_combinator}
  The reader familiar with the lambda calculus will have noticed the
  similarity between $D$ and the paradoxical combinator.

  [Ed. note: the existence of this seems to suggest we have to be more
  restrictive on the set of processes and names we admit if we are to
  support no-cloning.]
\end{remark}

\subsubsection{Bisimulation}

The computational dynamics gives rise to another kind of equivalence,
the equivalence of computational behavior. As previously mentioned
this is typically captured \emph{via} some form of bisimulation.

% The notion we use in this paper is weak barbed bisimulation
% \cite{milner91polyadicpi}.

The notion we use in this paper is derived from weak barbed
bisimulation \cite{milner91polyadicpi}. 

\begin{definition}
An \emph{observation relation}, $\downarrow_{\mathcal N}$, over a set
of names, $\mathcal N$, is the smallest relation satisfying the rules
below.

\infrule[Out-barb]{y \in {\mathcal N}, \; x \nameeq y}
		  {\outputp{x}{v} \downarrow_{\mathcal N} x}
\infrule[Par-barb]{\mbox{$P\downarrow_{\mathcal N} x$ or $Q\downarrow_{\mathcal N} x$}}
		  {\binpar{P}{Q} \downarrow_{\mathcal N} x}

We write $P \Downarrow_{\mathcal N} x$ if there is $Q$ such that 
$P \wred Q$ and $Q \downarrow_{\mathcal N} x$.
\end{definition}

\begin{definition}
%\label{def.bbisim}
An  ${\mathcal N}$-\emph{barbed bisimulation} over a set of names, ${\mathcal N}$, is a symmetric binary relation 
${\mathcal S}_{\mathcal N}$ between agents such that $P\rel{S}_{\mathcal N}Q$ implies:
\begin{enumerate}
\item If $P \red P'$ then $Q \wred Q'$ and $P'\rel{S}_{\mathcal N} Q'$.
\item If $P\downarrow_{\mathcal N} x$, then $Q\Downarrow_{\mathcal N} x$.
\end{enumerate}
$P$ is ${\mathcal N}$-barbed bisimilar to $Q$, written
$P \wbbisim_{\mathcal N} Q$, if $P \rel{S}_{\mathcal N} Q$ for some ${\mathcal N}$-barbed bisimulation ${\mathcal S}_{\mathcal N}$.
\end{definition}

$\mathcal{R} \subseteq \pi \times \pi$

$P \mathcal{R} Q => \forall P'. P \red P' \Rightarrow \exists Q'. Q \red Q', P' \mathcal{R} Q'$

$P \vdash x \Rightarrow Q \vdash x$

\begin{mathpar}
  \inferrule*[lab=Out-barb]{x \nameeq y}{{y}!\langle{Q}\rangle \vdash x}
  \and
  \inferrule*[lab=Par-barb]{\mbox{$P\vdash x$ or $Q\vdash x$}}{\binpar{P}{Q} \vdash x}
\end{mathpar}

\subsubsection{Contexts}

One of the principle advantages of computational calculi like the
$\pi$-calculus is a well-defined notion of context,
contextual-equivalence and a correlation between
contextual-equivalence and notions of bisimulation. The notion of
context allows the decomposition of a process into (sub-)process and
its syntactic environment, its context. Thus, a context may be
thought of as a process with a ``hole'' (written $\Box$) in it. The
application of a context $M$ to a process $P$, written $M[P]$, is
tantamount to filling the hole in $M$ with $P$. In this paper we do
not need the full weight of this theory, but do make use of the notion
of context in the proof the main theorem. 

\begin{mathpar}
  \inferrule* [lab=summation] {} {{M_{M},M_{N}} \bc \Box \;|\; x.M_{A} \;|\; M_{M}+M_{N}}
  \and
  \inferrule* [lab=agent] {} {{M_{A}} \bc (\vec{x})M_{P} \;| \; \clift{P_0,\ldots,M_{P},\ldots,P_N}}
  \and \\
  \inferrule* [lab=process] {} {{M_{P}} \bc M_{N} \;| \;P|M_{P} }
\end{mathpar} 

\begin{mathpar}
  \inferrule* [lab=sychronization] {} {M_{N} \bc \Box \;|\; x?M_{F} \;|\; x!M_{C}}
  \and
  \inferrule* [lab=abstraction] {} {{M_{F}} \bc (x)M_{P} }
  \and
  \inferrule* [lab=concretion] {} {{M_{C}} \bc \langle M_{P} \rangle }
  \and \\
  \inferrule* [lab=process] {} {{M_{P}} \bc M_{N} \;| \;P|M_{P} }
\end{mathpar}

\begin{definition}[contextual application] Given a context $M$, and
  process $P$, we define the \emph{contextual application}, $M[P] :=
  M\{P/\Box\}$. That is, the contextual application of M to P is the
  substitution of $P$ for $\Box$ in $M$.
\end{definition}

$\meaningof{-} : L \to \mathcal{P}(\pi)$

\begin{mathpar}
  \inferrule* [lab=collection] {} {\meaningof{true} = \pi, \and \meaningof{~E} = \pi \setminus \meaningof{E}, \and \meaningof{E_{1} \& E_{2}} = \meaningof{E_{1}} \cap \meaningof{E_{2}}}
\end{mathpar}

\begin{mathpar}
  \inferrule* [lab=structure] {} {\meaningof{0} = \{ P \in \pi | P \equiv 0 \}, \and \\ \meaningof{E_1 | E_2} = \{ P \in \pi | P \equiv P_{1} | P_{2}, P_{1} \in \meaningof{E_{1}}, P_{2} \in \meaningof{E_2}\} }
\end{mathpar}

\begin{mathpar}
 \inferrule* [lab=behavior] {} {\meaningof{\langle a?b \rangle E} = \{ P \in \pi | P \equiv Q | u?(y)P', \\ \and \\\\ \and \\ \;\;\; u \in \meaningof{a}, \forall z.P'\{z/y\} \in \meaningof{E\{z/b\}}\}, \and \\ \meaningof{a!E} = \{ P \in \pi | P \equiv Q | x!\langle P' \rangle, x \in \meaningof{a} P' \in \meaningof{E}\} }
\end{mathpar}

\begin{mathpar}
 \inferrule* [lab=nominal] {} {\meaningof{\quotep{E}} = \{ \quotep{P} \in \quotep{\pi} | P \in \meaningof{E} \}, \and \meaningof{\quotep{P}} = \{ \quotep{Q} \in \quotep{\pi} | P \equiv Q \} \and \\ \meaningof{@\quotep{E}} = \{ P \in \pi | P \equiv @x, x \in \meaningof{E} \}}
\end{mathpar}

\begin{eqnarray*}
  \\
  \meaningof{-} : TS \to ST
\end{eqnarray*}

\begin{eqnarray*}
  \\
  L : TS \to ST
\end{eqnarray*}

\begin{eqnarray*}
  \\
  P \models E \iff P \in \meaningof{E}
\end{eqnarray*}

\begin{eqnarray*}
  P \approx_{L} Q \iff \forall E \in L. P \models E \iff Q \models E
\end{eqnarray*}

\begin{eqnarray*}
  P \approx_{K} Q
\end{eqnarray*}

\begin{eqnarray*}
  P \approx Q
\end{eqnarray*}

$\approx_{K} = \approx = \approx_{L}$

\subsubsection{Contextual duality}

Note that contexts extend the quotation operation to a family of
operations from processes to names. Given a context, $M$, we can
define a \emph{nominal context}, $\quotep{M}$ by $\quotep{M}[P] :=
\quotep{M[P]}$. To foreshadow what is to come we observe that these
operations enjoy a duality with processes very much like the duality
between vectors and maps from vectors to scalars.

Further, because the calculus is essentially higher-order, we have a
correspondence between contexts and processes. More specifically,
given a name $x$ and a context $M$ we can construct $M^{*}_{x}$ such
that 

\begin{mathpar}
  M^{*}_{x} | \lift{x}{P} \red M[P]
\end{mathpar}

namely,

\begin{mathpar}
  M^{*}_{x} := x?(u).M[\dropn{u}]
\end{mathpar}

The dependence of $M^{*}_{x}$ on a name makes it an abstraction, 

\begin{mathpar}
  M^{*} := (x)x?(u).M[\dropn{u}]
\end{mathpar}

\subsection{Additional notation}

It will sometimes be convenient to denote the process a name
quotes. We already have the notation $x = \quotep{P}$, but it will be
convenient to introduce an alternate notation, $\procn{x}$, when we
want to emphasize the connection to the use of the name. Note that, by
virtue of name equivalence, $\quotep{\procn{x}} \nameeq x$; so, the
notation is consistent with previous definitions.

Further, because names have structure it is possible to effect
substitutions on the basis of that structure. This means we need to
upgrade our notation for substitutions, which we accomplish by
adapting comprehension notation. Thus,

\begin{mathpar}
  P\{ y / x : x \in S \}
\end{mathpar}

is interpreted to mean the process derived from P by replacing (in a
capture-avoiding manner) each occurrence of $x$ in $S$ by $y$. For example,

\begin{mathpar}
  P\{ \quotep{\procn{x}|\procn{x}} / x : x \in \freenames{P} \}
\end{mathpar}

will replace each (occurrence) of a free name $x$ in $P$ by
$\quotep{\procn{x}|\procn{x}}$.

Also, we will avail ourselves of the notation $x^{L}$ and $x^{R}$ to
denote injections of a name into disjoint copies of the name
space. There are numerous ways to accomplish this. One example can be
found in \cite{MeredithR05}. This notation overloads to vectors of
names: $\vec{x}^{\pi} := (x_{i}^{\pi} \; : \; 0 \leq i < |\vec{x}| )$ where $\pi \in \{L,R\}$.

We also use $P^{\Box} := P|\Box$.

In \cite{MeredithR05} an interpretation of the new operator is
given. It turns out that there are several possible interpretations
all enjoying the requisite algebraic properties of the operator (see
\cite{milner91polyadicpi}). We will therefore make liberal use of
$(\nu\; \vec{x})P$.

% subsection the_syntax_and_semantics_of_the_notation_system (end)   

\input{qm2pi.qmops} 

\input{qm2pi.sterngerlach} 

\input{qm2pi.metric} 

% section concurrent_process_calculi (end)

%\input{qm2pi.proofsketch}

% section proof sketch (end)

%\input{qm2pi.slviaknots} 

% section spatial logic via knots (end)

\input{qm2pi.conclusion}

% section conclusion (end)

%\input{qm2pi.dtcodes} 

% section wiring algorithm (end)

\input{qm2pi.ack} 

% section acknowledgments (end)

\newpage


\bibliographystyle{plain}   
\bibliography{../../biblios/main.bib}

\input{qm2pi.rhodetails}

\end{document}

 

%\ifpdf
%\usepackage[pdftex]{graphicx}
%\else
%\usepackage{graphicx}
%\fi

 % \ifpdf
%  \usepackage{pdfsync}
%  \if


%\title{Brief Article}
%\author{David F. Snyder}
%\author{L.G. Meredith}

%\address{Dept. of Math., Texas State University--San Marcos, San Marcos, TX 78666}
       
\pagestyle{empty}


\begin{document}

\lstset{language=[Objective]Caml,frame=shadowbox}

\documentclass[12pt]{llncs}
%\documentclass{jktr}

\usepackage[pdftex]{hyperref}                   
\usepackage {listings}
\usepackage {mathpartir}
\usepackage{bcprules}
%\usepackage{listings}
                       
\usepackage{graphicx} 
%\usepackage[margins=2.5cm,nohead,nofoot]{geometry}
%\usepackage{geometry}
\usepackage{amsfonts}
\usepackage{amstext}
\usepackage{latexsym}
\usepackage{amssymb}
\usepackage{color}


%\include{myPreamble}
\include{qm2pi.local} 

%\ifpdf
%\usepackage[pdftex]{graphicx}
%\else
%\usepackage{graphicx}
%\fi

 % \ifpdf
%  \usepackage{pdfsync}
%  \if


%\title{Brief Article}
%\author{David F. Snyder}
%\author{L.G. Meredith}

%\address{Dept. of Math., Texas State University--San Marcos, San Marcos, TX 78666}
       
\pagestyle{empty}


\begin{document}

\lstset{language=[Objective]Caml,frame=shadowbox}

\input{qm2pi.front}

% section front matter (end)

\input{qm2pi.intro} 
 
% section introduction (end)

% \input{qm2pi.knotations} 

% section notation (end)

\input{qm2pi.process.calculi} 

% section concurrent_process_calculi_and_spatial_logics_ (end)
    
%\input{qm2pi.knots2pi} 

%\input{qm2pi.trefoil} 

%\input{qm2pi.mainthm} 

% subsection basic_interpretation (end)

%\input{qm2pi.rho.presentation} 
\subsection{The syntax and semantics of the notation system}\label{sub:the_syntax_and_semantics_of_the_notation_system} % (fold)

We now summarize a technical presentation of the calculus that
embodies our theory of dynamics. The typical presentation of such a
calculus follows the style of giving generators and relations on
them. The grammar, below, describing term constructors, freely
generates the set of processes, $\Proc$. This set is then quotiented
by a relation known as structural congruence and it is over this set
that the notion of dynamics is expressed. This presentation is
essentially that of \cite{MeredithR05} with the addition of
polyadicity and summation. For readability we have relegated some of
the technical subtleties to an appendix.

\subsubsection{Process grammar}\label{subsub:process_grammar}

\begin{mathpar}
  \inferrule* [lab=synchronization] {} {{M} \bc \pzero \;|\; x?F \;|\; x!C }
  \and
  \inferrule* [lab=abstraction] {} {{F} \bc (x)P}
  \and
  \inferrule* [lab=concretion] {} {{C} \bc \langle Q \rangle}
  \and
  \inferrule* [lab=process] {} {{P,Q} \bc M \;| \;P|Q \;|\; @{x}}
  \and
  \inferrule* [lab=name] {} {{x} \bc \quotep{P}}
\end{mathpar} 

Note that $\vec{x}$ (resp. $\vec{P}$) denotes a vector of names
(resp. processes) of length $|\vec{x}|$ (resp. $|\vec{P}|$). We adopt
the following useful abbreviations.

\begin{mathpar}
   x?(\vec{y}).P := x.(\vec{y})P \and  x\clift{\vec{P}} := x.\clift{\vec{P}}
   \and x!(y) := \lift{x}{\dropn{y}}
   \and \Pi_{i=0}^{n-1}P_i := P_0 | \ldots | P_{n-1}
\end{mathpar}

\subsubsection{Structural congruence}

\paragraph{Free and bound names and alpha-equivalence.} At the
core of structural equivalence is alpha-equivalence which identifies
process that are the same up to a change of variable. Formally, we
recognize the distinction between free and bound names. The free names
of a process, $\freenames{P}$, may be calculated recursively as
follows:

\begin{mathpar}
\freenames{\pzero} := \emptyset
  \and \\
  \freenames{x?(y).P} := \{ x \} \cup (\freenames{P} \setminus \{ y \})
  \and 
  \freenames{x!\langle P \rangle} := \{ x \} \cup \{ P \} 
  \and \\
  \freenames{P|Q} := \freenames{P} \cup \freenames{Q}
  \and \\
  \freenames{@{x}} := \{ x \}
\end{mathpar}

$\pi$
$\quotep{\pi}$

$\freenames{-} : \pi \to \mathcal{P}(\quotep{\pi})$

\begin{eqnarray*}
  \freenames{\pzero} & := & \emptyset \\
  \freenames{x?(y).P} & := & \{ x \} \cup (\freenames{P} \setminus \{ y \}) \\
  \freenames{x!\langle P \rangle} & := & \{ x \} \cup \{ P \} \\
  \freenames{P|Q} & := & \freenames{P} \cup \freenames{Q} \\
  \freenames{\dropn{x}} & := & \{ x \}
\end{eqnarray*}

The bound names of a process, $\boundnames{P}$, are those names occurring in $P$
that are not free. For example, in $x?(y).0$, the name $x$ is free, while $y$ is bound.

\begin{mathpar}
  \inferrule* [lab=monoidal-laws] {} { P|Q \equiv Q|P \and P|0 \equiv P \and P|(Q|R) \equiv (P|Q)|R }
\end{mathpar}

\begin{mathpar}
  \inferrule* [lab=alpha-equivalence] {} { (x)P \equiv (y)P\{y/x\} \and y \not\in \freenames{P} }
\end{mathpar}

\begin{definition}
Then two processes, $P,Q$, are alpha-equivalent if $P = Q\{\vec{y}/\vec{x}\}$ for
some $\vec{x} \in \boundnames{Q},\vec{y} \in \boundnames{P}$, where $Q\{\vec{y}/\vec{x}\}$
denotes the capture-avoiding substitution of $\vec{y}$ for $\vec{x}$ in $Q$.
\end{definition}

\begin{definition}
  The {\em structural congruence} \cite{SangiorgiWalker} , $\equiv$,
  between processes is the least congruence containing
  alpha-equivalence, satisfying the abelian monoid laws
  (associativity, commutativity and $\pzero$ as identity) for parallel
  composition $|$ and for summation $+$.
\end{definition}

\subsection{Name equivalence}

We take name equivalence, written $\nameeq$, to be the smallest
equivalence relation generated by the following rules.

\begin{mathpar}
\inferrule*[lab=Quote-drop]
{ }
{ \quotep{@{x}} \nameeq x }

\inferrule*[lab=Struct-equiv]
{ P \scong Q }
{ \quotep{P} \nameeq \quotep{Q} }
\end{mathpar}

The astute reader will have noticed that the mutual recursion of names
and processes imposes a mutual recursion on alpha-equivalence and
structural equivalence via name-equivalence. Fortunately, all of this
works out pleasantly and we may calculate in the natural way, free of
concern. The reader interested in the details is referred to the
appendix \ref{appendix:rho_details}.

\subsection{Substitution}

We use $\Proc$ for the set of processes, $\QProc$ for the set of
names, and $\id{\{}\vec{y} / \vec{x} \id{\}}$ to denote partial maps,
$s : \QProc \rightarrow \QProc$. A map, $s$ lifts, uniquely, to a map
on process terms, $\widehat{s} : \Proc \rightarrow \Proc$ by the
following equations.

\begin{mathpar}
  (0) \psubstp{Q}{P} := 0 \\
  (R \juxtap S) \psubstp{Q}{P}
  :=    
  (R)\psubstp{Q}{P} \juxtap (S) \psubstp{Q}{P} \\
  (x?(y).R) \psubstp{Q}{P}    
  :=    
  (x)\substp{Q}{P} (z)\concat( (R \psubstn{z}{y}) \psubstp{Q}{P} ) \\
  (\lift{x}{R}) \psubstp{Q}{P}  
  :=
  \lift{(x)\substp{Q}{P}}{ R \psubstp{Q}{P} } \\
%   (\dropn{x})  \psubstp{Q}{P}       
%   := 
%   \left\{ 
%     \begin{array}{ccc} 
%       \dropn{\quotep{Q}} & & x \nameeq \quotep{P} \\
%       \dropn{x} & & otherwise \\
%     \end{array}
%   \right. 
  (\dropn{x})  \psubstp{Q}{P}       
  := 
  \left\{ 
    \begin{array}{ccc} 
      Q & & x \nameeq \quotep{P} \\
      \dropn{x} & & otherwise \\
    \end{array}
  \right.
\end{mathpar}
 

where

\begin{eqnarray}
  (x)\id{\{} \lpquote Q \rpquote / \lpquote P \rpquote \id{\}}            = 
  \left\{ 
    \begin{array}{ccc}
      \lpquote Q \rpquote & & x \nameeq \lpquote P \rpquote \\
      x & & otherwise \\
    \end{array}
  \right. \nonumber
\end{eqnarray}

and $z$ is chosen distinct from $\quotep{P}$, $\quotep{Q}$, the free
names in $Q$, and all the names in $R$. Our $\alpha$-equivalence will
be built in the standard way from this substitution.

\begin{remark}\label{rem:no_self_referential_names}
  One consequence of these definitions is that $\forall P. \quotep{P}
  \not\in \freenames{P}$.
\end{remark}

\subsection{ Dynamic quote: an example }

Anticipating something of what's to come, consider applying the
substitution, $\widehat{\id{\{}u / z \id{\}}}$, to the following pair
of processes, $\lift{w}{y!(z)}$ and $w[ \lpquote y!(z) \rpquote ]$.

\begin{eqnarray}
	\lift{w}{y!(z)}\widehat{\id{\{}u / z \id{\}}}
		& = &
		\lift{w}{y!(u)} \nonumber\\
	w[ \lpquote y!(z) \rpquote ] \widehat{ \id{\{}u / z \id{\}} }
		& = &
		w[ \lpquote y!(z) \rpquote ] \nonumber
\end{eqnarray}

Because the body of the process between quotes is impervious to
substitution, we get radically different answers. In fact, by
examining the first process in an input context,
e.g. $x?(z).\lift{w}{y!(z)}$, we see that the process under the lift
operator may be shaped by prefixed inputs binding a name inside it. In
this sense, the lift operator will be seen as a way to dynamically
construct processes before reifying them as names.

Finally equipped with these standard features we can present the
dynamics of the calculus.

\subsubsection{Operational semantics} 

Finally, we introduce the computational dynamics. What marks these
algebras as distinct from other more traditionally studied algebraic
structures, e.g. vector spaces or polynomial rings, is the manner in
which dynamics is captured. In traditional structures, dynamics is typically
expressed through morphisms between such structures, as in linear maps
between vector spaces or morphisms between rings. In algebras
associated with the semantics of computation, the dynamics is
expressed as part of the algebraic structure itself, through a
reduction reduction relation typically denoted by $\red$. Below, we
give a recursive presentation of this relation for the calculus used
in the encoding.

$\red \subseteq \pi \times \pi$
$\red : \pi \to \mathcal{P}(\pi)$

\begin{mathpar}
  \inferrule* [lab=Comm] { \textsf{match}( x_{src}, x_{trgt} ) } { x_{trgt}?(y)P \; | \; x_{src}!\langle {Q} \rangle \red P\{\quotep{Q}/y}\} }
  \and \\
  \inferrule* [lab=Par] {{P} \red {P}'} {{{P} | {Q}} \red {{P}' | {Q}}}
  \and
  \inferrule* [lab=Equiv]{{{P} \scong {P}'} \andalso {{P}' \red {Q}'} \andalso {{Q}' \scong {Q}}}{{P} \red {Q}}
\end{mathpar}

\begin{eqnarray*}
  match_{\equiv} (\quotep{P},\quotep{Q}) & := & P \equiv Q \\
  match_{\dagger}(\quotep{P},\quotep{Q}) & := & \forall R. P|Q \red^{*} R => R \red^{*} 0 \\
  match_{K}(\quotep{P},\quotep{Q}) & := & K \mbox{ for some context } K
\end{eqnarray*}

$u?(x)P | u!\langle Q \rangle \red P\{\quotep{Q}/x\}$

%We write $\wred$ for $\red^*$, and $P\red$ if $\exists Q $ such that $ P \red Q$.
We write $P\red$ if $\exists Q $ such that $ P \red Q$ and $P\not\red$, otherwise.

\section{Replication}

As mentioned before, it is known that replication (and hence
recursion) can be implemented in a higher-order process algebra
\cite{SangiorgiWalker}. As our first example of calculation with the
machinery thus far presented we give the construction explicitly in
the {\rhoc}.

\begin{eqnarray}
	D_{x} & := & \prefix{x}{y}{(\binpar{\outputp{x}{y}}{@{y}})} \nonumber\\
	\bangp_{x}{P} & := & \binpar{{x}!\langle{\binpar{D_{x}}{P}}\rangle}{D_{x}} \nonumber
\end{eqnarray}

\begin{eqnarray}
	\bangp_{x}{P} & & \nonumber\\
	=
	& {x}!\langle{(\prefix{x}{y}{(\outputp{x}{y} | @{y})) | P}}\rangle 
	      | \prefix{x}{y}{(\outputp{x}{y} | @{y})} & \nonumber\\
	\red
	& (\outputp{x}{y} | @{y})\substn{\quotep{(\prefix{x}{y}{(@{y} | \outputp{x}{y})) | P}}}{y} & \nonumber\\
	=
	& \outputp{x}{\quotep{(\prefix{x}{y}{(\outputp{x}{y} | @{y})) | P}}}
	  | {(\prefix{x}{y}{(\outputp{x}{y} | @{y})) | P}} & \nonumber\\
	\red
	& \ldots & \nonumber\\
	\red^*
	& P | P | \ldots & \nonumber
\end{eqnarray}

Of course, this encoding, as an implementation, runs away, unfolding
$\bangp{P}$ eagerly. A lazier and more implementable replication
operator, restricted to input-guarded processes, may be obtained as follows.

\begin{eqnarray}
\bangp{\prefix{u}{v}{P}} 
	:= 
	\binpar{\lift{x}{\prefix{u}{v}{(\binpar{D(x)}{P})}}}{D(x)} \nonumber
\end{eqnarray}

\begin{remark}
  Note that the lazier definition still does not deal with summation
  or mixed summation (i.e. sums over input and output). The reader is
  invited to construct definitions of replication that deal with these
  features. 

  Further, the definitions are parameterized in a name, $x$. Can you,
  gentle reader, make a definition that eliminates this parameter and
  guarantees no accidental interaction between the replication
  machinery and the process being replicated -- i.e. no accidental
  sharing of names used by the process to get its work done and the
  name(s) used by the replication to effect copying. This latter
  revision of the definition of replication is crucial to obtaining
  the expected identity $!!P \sim !P$.
\end{remark}

\begin{remark}\label{rem:paradoxical_combinator}
  The reader familiar with the lambda calculus will have noticed the
  similarity between $D$ and the paradoxical combinator.

  [Ed. note: the existence of this seems to suggest we have to be more
  restrictive on the set of processes and names we admit if we are to
  support no-cloning.]
\end{remark}

\subsubsection{Bisimulation}

The computational dynamics gives rise to another kind of equivalence,
the equivalence of computational behavior. As previously mentioned
this is typically captured \emph{via} some form of bisimulation.

% The notion we use in this paper is weak barbed bisimulation
% \cite{milner91polyadicpi}.

The notion we use in this paper is derived from weak barbed
bisimulation \cite{milner91polyadicpi}. 

\begin{definition}
An \emph{observation relation}, $\downarrow_{\mathcal N}$, over a set
of names, $\mathcal N$, is the smallest relation satisfying the rules
below.

\infrule[Out-barb]{y \in {\mathcal N}, \; x \nameeq y}
		  {\outputp{x}{v} \downarrow_{\mathcal N} x}
\infrule[Par-barb]{\mbox{$P\downarrow_{\mathcal N} x$ or $Q\downarrow_{\mathcal N} x$}}
		  {\binpar{P}{Q} \downarrow_{\mathcal N} x}

We write $P \Downarrow_{\mathcal N} x$ if there is $Q$ such that 
$P \wred Q$ and $Q \downarrow_{\mathcal N} x$.
\end{definition}

\begin{definition}
%\label{def.bbisim}
An  ${\mathcal N}$-\emph{barbed bisimulation} over a set of names, ${\mathcal N}$, is a symmetric binary relation 
${\mathcal S}_{\mathcal N}$ between agents such that $P\rel{S}_{\mathcal N}Q$ implies:
\begin{enumerate}
\item If $P \red P'$ then $Q \wred Q'$ and $P'\rel{S}_{\mathcal N} Q'$.
\item If $P\downarrow_{\mathcal N} x$, then $Q\Downarrow_{\mathcal N} x$.
\end{enumerate}
$P$ is ${\mathcal N}$-barbed bisimilar to $Q$, written
$P \wbbisim_{\mathcal N} Q$, if $P \rel{S}_{\mathcal N} Q$ for some ${\mathcal N}$-barbed bisimulation ${\mathcal S}_{\mathcal N}$.
\end{definition}

$\mathcal{R} \subseteq \pi \times \pi$

$P \mathcal{R} Q => \forall P'. P \red P' \Rightarrow \exists Q'. Q \red Q', P' \mathcal{R} Q'$

$P \vdash x \Rightarrow Q \vdash x$

\begin{mathpar}
  \inferrule*[lab=Out-barb]{x \nameeq y}{{y}!\langle{Q}\rangle \vdash x}
  \and
  \inferrule*[lab=Par-barb]{\mbox{$P\vdash x$ or $Q\vdash x$}}{\binpar{P}{Q} \vdash x}
\end{mathpar}

\subsubsection{Contexts}

One of the principle advantages of computational calculi like the
$\pi$-calculus is a well-defined notion of context,
contextual-equivalence and a correlation between
contextual-equivalence and notions of bisimulation. The notion of
context allows the decomposition of a process into (sub-)process and
its syntactic environment, its context. Thus, a context may be
thought of as a process with a ``hole'' (written $\Box$) in it. The
application of a context $M$ to a process $P$, written $M[P]$, is
tantamount to filling the hole in $M$ with $P$. In this paper we do
not need the full weight of this theory, but do make use of the notion
of context in the proof the main theorem. 

\begin{mathpar}
  \inferrule* [lab=summation] {} {{M_{M},M_{N}} \bc \Box \;|\; x.M_{A} \;|\; M_{M}+M_{N}}
  \and
  \inferrule* [lab=agent] {} {{M_{A}} \bc (\vec{x})M_{P} \;| \; \clift{P_0,\ldots,M_{P},\ldots,P_N}}
  \and \\
  \inferrule* [lab=process] {} {{M_{P}} \bc M_{N} \;| \;P|M_{P} }
\end{mathpar} 

\begin{mathpar}
  \inferrule* [lab=sychronization] {} {M_{N} \bc \Box \;|\; x?M_{F} \;|\; x!M_{C}}
  \and
  \inferrule* [lab=abstraction] {} {{M_{F}} \bc (x)M_{P} }
  \and
  \inferrule* [lab=concretion] {} {{M_{C}} \bc \langle M_{P} \rangle }
  \and \\
  \inferrule* [lab=process] {} {{M_{P}} \bc M_{N} \;| \;P|M_{P} }
\end{mathpar}

\begin{definition}[contextual application] Given a context $M$, and
  process $P$, we define the \emph{contextual application}, $M[P] :=
  M\{P/\Box\}$. That is, the contextual application of M to P is the
  substitution of $P$ for $\Box$ in $M$.
\end{definition}

$\meaningof{-} : L \to \mathcal{P}(\pi)$

\begin{mathpar}
  \inferrule* [lab=collection] {} {\meaningof{true} = \pi, \and \meaningof{~E} = \pi \setminus \meaningof{E}, \and \meaningof{E_{1} \& E_{2}} = \meaningof{E_{1}} \cap \meaningof{E_{2}}}
\end{mathpar}

\begin{mathpar}
  \inferrule* [lab=structure] {} {\meaningof{0} = \{ P \in \pi | P \equiv 0 \}, \and \\ \meaningof{E_1 | E_2} = \{ P \in \pi | P \equiv P_{1} | P_{2}, P_{1} \in \meaningof{E_{1}}, P_{2} \in \meaningof{E_2}\} }
\end{mathpar}

\begin{mathpar}
 \inferrule* [lab=behavior] {} {\meaningof{\langle a?b \rangle E} = \{ P \in \pi | P \equiv Q | u?(y)P', \\ \and \\\\ \and \\ \;\;\; u \in \meaningof{a}, \forall z.P'\{z/y\} \in \meaningof{E\{z/b\}}\}, \and \\ \meaningof{a!E} = \{ P \in \pi | P \equiv Q | x!\langle P' \rangle, x \in \meaningof{a} P' \in \meaningof{E}\} }
\end{mathpar}

\begin{mathpar}
 \inferrule* [lab=nominal] {} {\meaningof{\quotep{E}} = \{ \quotep{P} \in \quotep{\pi} | P \in \meaningof{E} \}, \and \meaningof{\quotep{P}} = \{ \quotep{Q} \in \quotep{\pi} | P \equiv Q \} \and \\ \meaningof{@\quotep{E}} = \{ P \in \pi | P \equiv @x, x \in \meaningof{E} \}}
\end{mathpar}

\begin{eqnarray*}
  \\
  \meaningof{-} : TS \to ST
\end{eqnarray*}

\begin{eqnarray*}
  \\
  L : TS \to ST
\end{eqnarray*}

\begin{eqnarray*}
  \\
  P \models E \iff P \in \meaningof{E}
\end{eqnarray*}

\begin{eqnarray*}
  P \approx_{L} Q \iff \forall E \in L. P \models E \iff Q \models E
\end{eqnarray*}

\begin{eqnarray*}
  P \approx_{K} Q
\end{eqnarray*}

\begin{eqnarray*}
  P \approx Q
\end{eqnarray*}

$\approx_{K} = \approx = \approx_{L}$

\subsubsection{Contextual duality}

Note that contexts extend the quotation operation to a family of
operations from processes to names. Given a context, $M$, we can
define a \emph{nominal context}, $\quotep{M}$ by $\quotep{M}[P] :=
\quotep{M[P]}$. To foreshadow what is to come we observe that these
operations enjoy a duality with processes very much like the duality
between vectors and maps from vectors to scalars.

Further, because the calculus is essentially higher-order, we have a
correspondence between contexts and processes. More specifically,
given a name $x$ and a context $M$ we can construct $M^{*}_{x}$ such
that 

\begin{mathpar}
  M^{*}_{x} | \lift{x}{P} \red M[P]
\end{mathpar}

namely,

\begin{mathpar}
  M^{*}_{x} := x?(u).M[\dropn{u}]
\end{mathpar}

The dependence of $M^{*}_{x}$ on a name makes it an abstraction, 

\begin{mathpar}
  M^{*} := (x)x?(u).M[\dropn{u}]
\end{mathpar}

\subsection{Additional notation}

It will sometimes be convenient to denote the process a name
quotes. We already have the notation $x = \quotep{P}$, but it will be
convenient to introduce an alternate notation, $\procn{x}$, when we
want to emphasize the connection to the use of the name. Note that, by
virtue of name equivalence, $\quotep{\procn{x}} \nameeq x$; so, the
notation is consistent with previous definitions.

Further, because names have structure it is possible to effect
substitutions on the basis of that structure. This means we need to
upgrade our notation for substitutions, which we accomplish by
adapting comprehension notation. Thus,

\begin{mathpar}
  P\{ y / x : x \in S \}
\end{mathpar}

is interpreted to mean the process derived from P by replacing (in a
capture-avoiding manner) each occurrence of $x$ in $S$ by $y$. For example,

\begin{mathpar}
  P\{ \quotep{\procn{x}|\procn{x}} / x : x \in \freenames{P} \}
\end{mathpar}

will replace each (occurrence) of a free name $x$ in $P$ by
$\quotep{\procn{x}|\procn{x}}$.

Also, we will avail ourselves of the notation $x^{L}$ and $x^{R}$ to
denote injections of a name into disjoint copies of the name
space. There are numerous ways to accomplish this. One example can be
found in \cite{MeredithR05}. This notation overloads to vectors of
names: $\vec{x}^{\pi} := (x_{i}^{\pi} \; : \; 0 \leq i < |\vec{x}| )$ where $\pi \in \{L,R\}$.

We also use $P^{\Box} := P|\Box$.

In \cite{MeredithR05} an interpretation of the new operator is
given. It turns out that there are several possible interpretations
all enjoying the requisite algebraic properties of the operator (see
\cite{milner91polyadicpi}). We will therefore make liberal use of
$(\nu\; \vec{x})P$.

% subsection the_syntax_and_semantics_of_the_notation_system (end)   

\input{qm2pi.qmops} 

\input{qm2pi.sterngerlach} 

\input{qm2pi.metric} 

% section concurrent_process_calculi (end)

%\input{qm2pi.proofsketch}

% section proof sketch (end)

%\input{qm2pi.slviaknots} 

% section spatial logic via knots (end)

\input{qm2pi.conclusion}

% section conclusion (end)

%\input{qm2pi.dtcodes} 

% section wiring algorithm (end)

\input{qm2pi.ack} 

% section acknowledgments (end)

\newpage


\bibliographystyle{plain}   
\bibliography{../../biblios/main.bib}

\input{qm2pi.rhodetails}

\end{document}



% section front matter (end)

\section{Introduction}\label{sec:introduction} % (fold)
In this draft of the material i am going to have to dispense with the
usual writing conventions adopted in papers on these topics. i'm going
to have adopt whatever tone i need at the time i'm writing up the
calculations. Sometimes this may be very conversational; others it may
be the barest mathematical grunts; others still it may be that i have
lifted text from one of my other papers because the exposition of some
point was better said there. i hope that my readers are not unduly put
out by this decision. i'm not doing this to flout convention or be
rebellious. i find these calculations very technically challenging. To
keep everything going technically, something has to give; i have to
let go of some cognitive burden. So, the academic writing style --
with all of its trade-offs in terms of facilitating technical
communication -- is what i'm letting go of. Perhaps subsequent drafts
can be tightened and polished, but for now, i'm going to speak as if
we were sitting together in a coffee shop with a laptop, wifi and a
pad of paper and a pencil.

So, here's what i have to say. We -- you and i, comfortably ensconced
in our coffee shop and well-equipped with our tools -- can realize and
carry out the calculations of quantum mechanics over a very different
formal theory of dynamics, a formal theory of dynamics that
corresponds to a theory of concurrent computation with
\emph{reflection}. It has the advantage that the underlying theory is
already `quantized', but supports analogues all of the continuuous
operations. Strikingly, this underlying theory has recently been
connected with a notion of metric that we can show, by calculating
together, coincides with the metric induced by the inner product.

There are a lot of reasons why you might be interested in seeing
calculations of this form. Here's why i'm interested. For the past
several centuries there has been no competitor to the ``Newtonian''
account of dynamics. As a result the predominant share of accounts of
dynamical systems and situations have had to be formulated in terms of
the Newtonian machinery. i view this as an intellectually dangerous
position to occupy. Everything, despite it's intrinsic shape, turns
into a nail to be hit with this hammer. Recently, however, the theory
of computation has matured to the point where we have candidates for
theories of dynamics that offer very different perspective on
reasoning about dynamical systems and situations. Testing these
candidates against very successful accounts of dynamical situations,
like quantum mechanics, is going to give us some sense of how mature
they are and some measure of the quality of these accounts of
dynamics.

\subsection{Summary of contributions and outline of paper}

So, we're going to develop an interpretation of the operations of
quantum mechanics normally interpreted by Hilbert spaces and
operators. We're going to do this over a theory of computation. Note
that this is very different than the usual quantum computation program
which develops notions of computation over quantum mechanics. Rather,
we are developing a story that aligns with Wheeler's slogan: It from
Bit. To do this we will first provide an account of the theory of
computation at play here. Then we will dive into a calculation-driven
interpretation of the operations of quantum mechanics.

The reason we take this approach is that -- until very recently --
there hasn't been an axiomatic account of quantum mechanics. As a
result there has been no sharp delineation of the mathematical theory
supporting interpretation of the physical theory and the physical
theory, itself. So, ambient features of the maths are free to be
exploited (or supressed) without a real accounting of their physical
relevance. There is no sharp statement ``here's the physical theory''
qua \emph{theory} and ``here's the mathematical interpretation''
enabling a judgment of how faithful the interpretation is -- apart
from experimental observation. When there is an axiomatic account we
can judge how well a given mathematical formalism supports an
interpretation of the axioms, independent of
experimentation. Likewise, we can judge how well we have captured our
physical evidence and experience with our axiomatics, independent of
any specific mathematical implementation, with accidental detail that
may or may not have physical significance. 

In lieu of a fully fleshed out and vetted axiomatic account of quantum
mechanics, interpreting the operational notions in service of modeling
physical systems will have to suffice. In other words, we are not in
the business of providing a model of Hilbert spaces and operators. We
are in the business of providing a model of quantum mechanics because
we are motivated by testing our notions of dynamics against physical
theory; and, the predictive calculations of the physical theory must
serve as the best formulation -- shy of a fully fleshed out axiomatic
account -- of the physical theory itself (as they have for scientific
theories since time immemorial). Put another way, despite a
whole-hearted commitment to an It-from-Bit ontology, we are firmly
aligned with the shut-up-and-calculate camp as the best way to obtain
results either from the physical perspective or as a quality assurance
measure of our fledgling theory of dynamics.

In detail, we present a reflective process calculus. Then we develop
intuitive correspondences between the notions available in this
calculus and the usual physical notions supporting quantum mechanical
calculations. Thus, 

\begin{table}[htp]
  \center{
    \fbox{
      \begin{tabular}{c|c}
        quantum mechanics & process calculus \\
        \hline
        scalar & name \\
        state vector & process \\
        dual & contextual duals \\
        matrix & formal sums of process-context-dual pairs \\
        orthogonality & process annihilation \\
        inner product & execution-formula + quoting
      \end{tabular}
    }
  }
  \caption{QM - process calculi correspondences}
\end{table}

Then we tighten up these intuitions to operational definitions. We
employ the Dirac notation as the best proxy we can find for an
abstract syntax of the quantum mechanical notions. The definitions we
develop put us in contact with equational constraints coming from the
theory that we demonstrate the definitions and calculations satisfy.

This puts us in a position to shut up and calculate for the
Stern-Gerlach experimental set up, showing how these predictive
calculations become calculations on processes in our theory of a
reflective process calculus.

Penultimately, we demonstrate that the notion of metric coming from
the inner product coincides with the notion of metric available from
the theory of bisimulation. This demonstration gives us the right to
think of space as arising from behavior. Finally, we consider where we
might go from the new vantage point we have obtained.

% section introduction (end) 
 
% section introduction (end)

% \documentclass[12pt]{llncs}
%\documentclass{jktr}

\usepackage[pdftex]{hyperref}                   
\usepackage {listings}
\usepackage {mathpartir}
\usepackage{bcprules}
%\usepackage{listings}
                       
\usepackage{graphicx} 
%\usepackage[margins=2.5cm,nohead,nofoot]{geometry}
%\usepackage{geometry}
\usepackage{amsfonts}
\usepackage{amstext}
\usepackage{latexsym}
\usepackage{amssymb}
\usepackage{color}


%\include{myPreamble}
\include{qm2pi.local} 

%\ifpdf
%\usepackage[pdftex]{graphicx}
%\else
%\usepackage{graphicx}
%\fi

 % \ifpdf
%  \usepackage{pdfsync}
%  \if


%\title{Brief Article}
%\author{David F. Snyder}
%\author{L.G. Meredith}

%\address{Dept. of Math., Texas State University--San Marcos, San Marcos, TX 78666}
       
\pagestyle{empty}


\begin{document}

\lstset{language=[Objective]Caml,frame=shadowbox}

\input{qm2pi.front}

% section front matter (end)

\input{qm2pi.intro} 
 
% section introduction (end)

% \input{qm2pi.knotations} 

% section notation (end)

\input{qm2pi.process.calculi} 

% section concurrent_process_calculi_and_spatial_logics_ (end)
    
%\input{qm2pi.knots2pi} 

%\input{qm2pi.trefoil} 

%\input{qm2pi.mainthm} 

% subsection basic_interpretation (end)

%\input{qm2pi.rho.presentation} 
\subsection{The syntax and semantics of the notation system}\label{sub:the_syntax_and_semantics_of_the_notation_system} % (fold)

We now summarize a technical presentation of the calculus that
embodies our theory of dynamics. The typical presentation of such a
calculus follows the style of giving generators and relations on
them. The grammar, below, describing term constructors, freely
generates the set of processes, $\Proc$. This set is then quotiented
by a relation known as structural congruence and it is over this set
that the notion of dynamics is expressed. This presentation is
essentially that of \cite{MeredithR05} with the addition of
polyadicity and summation. For readability we have relegated some of
the technical subtleties to an appendix.

\subsubsection{Process grammar}\label{subsub:process_grammar}

\begin{mathpar}
  \inferrule* [lab=synchronization] {} {{M} \bc \pzero \;|\; x?F \;|\; x!C }
  \and
  \inferrule* [lab=abstraction] {} {{F} \bc (x)P}
  \and
  \inferrule* [lab=concretion] {} {{C} \bc \langle Q \rangle}
  \and
  \inferrule* [lab=process] {} {{P,Q} \bc M \;| \;P|Q \;|\; @{x}}
  \and
  \inferrule* [lab=name] {} {{x} \bc \quotep{P}}
\end{mathpar} 

Note that $\vec{x}$ (resp. $\vec{P}$) denotes a vector of names
(resp. processes) of length $|\vec{x}|$ (resp. $|\vec{P}|$). We adopt
the following useful abbreviations.

\begin{mathpar}
   x?(\vec{y}).P := x.(\vec{y})P \and  x\clift{\vec{P}} := x.\clift{\vec{P}}
   \and x!(y) := \lift{x}{\dropn{y}}
   \and \Pi_{i=0}^{n-1}P_i := P_0 | \ldots | P_{n-1}
\end{mathpar}

\subsubsection{Structural congruence}

\paragraph{Free and bound names and alpha-equivalence.} At the
core of structural equivalence is alpha-equivalence which identifies
process that are the same up to a change of variable. Formally, we
recognize the distinction between free and bound names. The free names
of a process, $\freenames{P}$, may be calculated recursively as
follows:

\begin{mathpar}
\freenames{\pzero} := \emptyset
  \and \\
  \freenames{x?(y).P} := \{ x \} \cup (\freenames{P} \setminus \{ y \})
  \and 
  \freenames{x!\langle P \rangle} := \{ x \} \cup \{ P \} 
  \and \\
  \freenames{P|Q} := \freenames{P} \cup \freenames{Q}
  \and \\
  \freenames{@{x}} := \{ x \}
\end{mathpar}

$\pi$
$\quotep{\pi}$

$\freenames{-} : \pi \to \mathcal{P}(\quotep{\pi})$

\begin{eqnarray*}
  \freenames{\pzero} & := & \emptyset \\
  \freenames{x?(y).P} & := & \{ x \} \cup (\freenames{P} \setminus \{ y \}) \\
  \freenames{x!\langle P \rangle} & := & \{ x \} \cup \{ P \} \\
  \freenames{P|Q} & := & \freenames{P} \cup \freenames{Q} \\
  \freenames{\dropn{x}} & := & \{ x \}
\end{eqnarray*}

The bound names of a process, $\boundnames{P}$, are those names occurring in $P$
that are not free. For example, in $x?(y).0$, the name $x$ is free, while $y$ is bound.

\begin{mathpar}
  \inferrule* [lab=monoidal-laws] {} { P|Q \equiv Q|P \and P|0 \equiv P \and P|(Q|R) \equiv (P|Q)|R }
\end{mathpar}

\begin{mathpar}
  \inferrule* [lab=alpha-equivalence] {} { (x)P \equiv (y)P\{y/x\} \and y \not\in \freenames{P} }
\end{mathpar}

\begin{definition}
Then two processes, $P,Q$, are alpha-equivalent if $P = Q\{\vec{y}/\vec{x}\}$ for
some $\vec{x} \in \boundnames{Q},\vec{y} \in \boundnames{P}$, where $Q\{\vec{y}/\vec{x}\}$
denotes the capture-avoiding substitution of $\vec{y}$ for $\vec{x}$ in $Q$.
\end{definition}

\begin{definition}
  The {\em structural congruence} \cite{SangiorgiWalker} , $\equiv$,
  between processes is the least congruence containing
  alpha-equivalence, satisfying the abelian monoid laws
  (associativity, commutativity and $\pzero$ as identity) for parallel
  composition $|$ and for summation $+$.
\end{definition}

\subsection{Name equivalence}

We take name equivalence, written $\nameeq$, to be the smallest
equivalence relation generated by the following rules.

\begin{mathpar}
\inferrule*[lab=Quote-drop]
{ }
{ \quotep{@{x}} \nameeq x }

\inferrule*[lab=Struct-equiv]
{ P \scong Q }
{ \quotep{P} \nameeq \quotep{Q} }
\end{mathpar}

The astute reader will have noticed that the mutual recursion of names
and processes imposes a mutual recursion on alpha-equivalence and
structural equivalence via name-equivalence. Fortunately, all of this
works out pleasantly and we may calculate in the natural way, free of
concern. The reader interested in the details is referred to the
appendix \ref{appendix:rho_details}.

\subsection{Substitution}

We use $\Proc$ for the set of processes, $\QProc$ for the set of
names, and $\id{\{}\vec{y} / \vec{x} \id{\}}$ to denote partial maps,
$s : \QProc \rightarrow \QProc$. A map, $s$ lifts, uniquely, to a map
on process terms, $\widehat{s} : \Proc \rightarrow \Proc$ by the
following equations.

\begin{mathpar}
  (0) \psubstp{Q}{P} := 0 \\
  (R \juxtap S) \psubstp{Q}{P}
  :=    
  (R)\psubstp{Q}{P} \juxtap (S) \psubstp{Q}{P} \\
  (x?(y).R) \psubstp{Q}{P}    
  :=    
  (x)\substp{Q}{P} (z)\concat( (R \psubstn{z}{y}) \psubstp{Q}{P} ) \\
  (\lift{x}{R}) \psubstp{Q}{P}  
  :=
  \lift{(x)\substp{Q}{P}}{ R \psubstp{Q}{P} } \\
%   (\dropn{x})  \psubstp{Q}{P}       
%   := 
%   \left\{ 
%     \begin{array}{ccc} 
%       \dropn{\quotep{Q}} & & x \nameeq \quotep{P} \\
%       \dropn{x} & & otherwise \\
%     \end{array}
%   \right. 
  (\dropn{x})  \psubstp{Q}{P}       
  := 
  \left\{ 
    \begin{array}{ccc} 
      Q & & x \nameeq \quotep{P} \\
      \dropn{x} & & otherwise \\
    \end{array}
  \right.
\end{mathpar}
 

where

\begin{eqnarray}
  (x)\id{\{} \lpquote Q \rpquote / \lpquote P \rpquote \id{\}}            = 
  \left\{ 
    \begin{array}{ccc}
      \lpquote Q \rpquote & & x \nameeq \lpquote P \rpquote \\
      x & & otherwise \\
    \end{array}
  \right. \nonumber
\end{eqnarray}

and $z$ is chosen distinct from $\quotep{P}$, $\quotep{Q}$, the free
names in $Q$, and all the names in $R$. Our $\alpha$-equivalence will
be built in the standard way from this substitution.

\begin{remark}\label{rem:no_self_referential_names}
  One consequence of these definitions is that $\forall P. \quotep{P}
  \not\in \freenames{P}$.
\end{remark}

\subsection{ Dynamic quote: an example }

Anticipating something of what's to come, consider applying the
substitution, $\widehat{\id{\{}u / z \id{\}}}$, to the following pair
of processes, $\lift{w}{y!(z)}$ and $w[ \lpquote y!(z) \rpquote ]$.

\begin{eqnarray}
	\lift{w}{y!(z)}\widehat{\id{\{}u / z \id{\}}}
		& = &
		\lift{w}{y!(u)} \nonumber\\
	w[ \lpquote y!(z) \rpquote ] \widehat{ \id{\{}u / z \id{\}} }
		& = &
		w[ \lpquote y!(z) \rpquote ] \nonumber
\end{eqnarray}

Because the body of the process between quotes is impervious to
substitution, we get radically different answers. In fact, by
examining the first process in an input context,
e.g. $x?(z).\lift{w}{y!(z)}$, we see that the process under the lift
operator may be shaped by prefixed inputs binding a name inside it. In
this sense, the lift operator will be seen as a way to dynamically
construct processes before reifying them as names.

Finally equipped with these standard features we can present the
dynamics of the calculus.

\subsubsection{Operational semantics} 

Finally, we introduce the computational dynamics. What marks these
algebras as distinct from other more traditionally studied algebraic
structures, e.g. vector spaces or polynomial rings, is the manner in
which dynamics is captured. In traditional structures, dynamics is typically
expressed through morphisms between such structures, as in linear maps
between vector spaces or morphisms between rings. In algebras
associated with the semantics of computation, the dynamics is
expressed as part of the algebraic structure itself, through a
reduction reduction relation typically denoted by $\red$. Below, we
give a recursive presentation of this relation for the calculus used
in the encoding.

$\red \subseteq \pi \times \pi$
$\red : \pi \to \mathcal{P}(\pi)$

\begin{mathpar}
  \inferrule* [lab=Comm] { \textsf{match}( x_{src}, x_{trgt} ) } { x_{trgt}?(y)P \; | \; x_{src}!\langle {Q} \rangle \red P\{\quotep{Q}/y}\} }
  \and \\
  \inferrule* [lab=Par] {{P} \red {P}'} {{{P} | {Q}} \red {{P}' | {Q}}}
  \and
  \inferrule* [lab=Equiv]{{{P} \scong {P}'} \andalso {{P}' \red {Q}'} \andalso {{Q}' \scong {Q}}}{{P} \red {Q}}
\end{mathpar}

\begin{eqnarray*}
  match_{\equiv} (\quotep{P},\quotep{Q}) & := & P \equiv Q \\
  match_{\dagger}(\quotep{P},\quotep{Q}) & := & \forall R. P|Q \red^{*} R => R \red^{*} 0 \\
  match_{K}(\quotep{P},\quotep{Q}) & := & K \mbox{ for some context } K
\end{eqnarray*}

$u?(x)P | u!\langle Q \rangle \red P\{\quotep{Q}/x\}$

%We write $\wred$ for $\red^*$, and $P\red$ if $\exists Q $ such that $ P \red Q$.
We write $P\red$ if $\exists Q $ such that $ P \red Q$ and $P\not\red$, otherwise.

\section{Replication}

As mentioned before, it is known that replication (and hence
recursion) can be implemented in a higher-order process algebra
\cite{SangiorgiWalker}. As our first example of calculation with the
machinery thus far presented we give the construction explicitly in
the {\rhoc}.

\begin{eqnarray}
	D_{x} & := & \prefix{x}{y}{(\binpar{\outputp{x}{y}}{@{y}})} \nonumber\\
	\bangp_{x}{P} & := & \binpar{{x}!\langle{\binpar{D_{x}}{P}}\rangle}{D_{x}} \nonumber
\end{eqnarray}

\begin{eqnarray}
	\bangp_{x}{P} & & \nonumber\\
	=
	& {x}!\langle{(\prefix{x}{y}{(\outputp{x}{y} | @{y})) | P}}\rangle 
	      | \prefix{x}{y}{(\outputp{x}{y} | @{y})} & \nonumber\\
	\red
	& (\outputp{x}{y} | @{y})\substn{\quotep{(\prefix{x}{y}{(@{y} | \outputp{x}{y})) | P}}}{y} & \nonumber\\
	=
	& \outputp{x}{\quotep{(\prefix{x}{y}{(\outputp{x}{y} | @{y})) | P}}}
	  | {(\prefix{x}{y}{(\outputp{x}{y} | @{y})) | P}} & \nonumber\\
	\red
	& \ldots & \nonumber\\
	\red^*
	& P | P | \ldots & \nonumber
\end{eqnarray}

Of course, this encoding, as an implementation, runs away, unfolding
$\bangp{P}$ eagerly. A lazier and more implementable replication
operator, restricted to input-guarded processes, may be obtained as follows.

\begin{eqnarray}
\bangp{\prefix{u}{v}{P}} 
	:= 
	\binpar{\lift{x}{\prefix{u}{v}{(\binpar{D(x)}{P})}}}{D(x)} \nonumber
\end{eqnarray}

\begin{remark}
  Note that the lazier definition still does not deal with summation
  or mixed summation (i.e. sums over input and output). The reader is
  invited to construct definitions of replication that deal with these
  features. 

  Further, the definitions are parameterized in a name, $x$. Can you,
  gentle reader, make a definition that eliminates this parameter and
  guarantees no accidental interaction between the replication
  machinery and the process being replicated -- i.e. no accidental
  sharing of names used by the process to get its work done and the
  name(s) used by the replication to effect copying. This latter
  revision of the definition of replication is crucial to obtaining
  the expected identity $!!P \sim !P$.
\end{remark}

\begin{remark}\label{rem:paradoxical_combinator}
  The reader familiar with the lambda calculus will have noticed the
  similarity between $D$ and the paradoxical combinator.

  [Ed. note: the existence of this seems to suggest we have to be more
  restrictive on the set of processes and names we admit if we are to
  support no-cloning.]
\end{remark}

\subsubsection{Bisimulation}

The computational dynamics gives rise to another kind of equivalence,
the equivalence of computational behavior. As previously mentioned
this is typically captured \emph{via} some form of bisimulation.

% The notion we use in this paper is weak barbed bisimulation
% \cite{milner91polyadicpi}.

The notion we use in this paper is derived from weak barbed
bisimulation \cite{milner91polyadicpi}. 

\begin{definition}
An \emph{observation relation}, $\downarrow_{\mathcal N}$, over a set
of names, $\mathcal N$, is the smallest relation satisfying the rules
below.

\infrule[Out-barb]{y \in {\mathcal N}, \; x \nameeq y}
		  {\outputp{x}{v} \downarrow_{\mathcal N} x}
\infrule[Par-barb]{\mbox{$P\downarrow_{\mathcal N} x$ or $Q\downarrow_{\mathcal N} x$}}
		  {\binpar{P}{Q} \downarrow_{\mathcal N} x}

We write $P \Downarrow_{\mathcal N} x$ if there is $Q$ such that 
$P \wred Q$ and $Q \downarrow_{\mathcal N} x$.
\end{definition}

\begin{definition}
%\label{def.bbisim}
An  ${\mathcal N}$-\emph{barbed bisimulation} over a set of names, ${\mathcal N}$, is a symmetric binary relation 
${\mathcal S}_{\mathcal N}$ between agents such that $P\rel{S}_{\mathcal N}Q$ implies:
\begin{enumerate}
\item If $P \red P'$ then $Q \wred Q'$ and $P'\rel{S}_{\mathcal N} Q'$.
\item If $P\downarrow_{\mathcal N} x$, then $Q\Downarrow_{\mathcal N} x$.
\end{enumerate}
$P$ is ${\mathcal N}$-barbed bisimilar to $Q$, written
$P \wbbisim_{\mathcal N} Q$, if $P \rel{S}_{\mathcal N} Q$ for some ${\mathcal N}$-barbed bisimulation ${\mathcal S}_{\mathcal N}$.
\end{definition}

$\mathcal{R} \subseteq \pi \times \pi$

$P \mathcal{R} Q => \forall P'. P \red P' \Rightarrow \exists Q'. Q \red Q', P' \mathcal{R} Q'$

$P \vdash x \Rightarrow Q \vdash x$

\begin{mathpar}
  \inferrule*[lab=Out-barb]{x \nameeq y}{{y}!\langle{Q}\rangle \vdash x}
  \and
  \inferrule*[lab=Par-barb]{\mbox{$P\vdash x$ or $Q\vdash x$}}{\binpar{P}{Q} \vdash x}
\end{mathpar}

\subsubsection{Contexts}

One of the principle advantages of computational calculi like the
$\pi$-calculus is a well-defined notion of context,
contextual-equivalence and a correlation between
contextual-equivalence and notions of bisimulation. The notion of
context allows the decomposition of a process into (sub-)process and
its syntactic environment, its context. Thus, a context may be
thought of as a process with a ``hole'' (written $\Box$) in it. The
application of a context $M$ to a process $P$, written $M[P]$, is
tantamount to filling the hole in $M$ with $P$. In this paper we do
not need the full weight of this theory, but do make use of the notion
of context in the proof the main theorem. 

\begin{mathpar}
  \inferrule* [lab=summation] {} {{M_{M},M_{N}} \bc \Box \;|\; x.M_{A} \;|\; M_{M}+M_{N}}
  \and
  \inferrule* [lab=agent] {} {{M_{A}} \bc (\vec{x})M_{P} \;| \; \clift{P_0,\ldots,M_{P},\ldots,P_N}}
  \and \\
  \inferrule* [lab=process] {} {{M_{P}} \bc M_{N} \;| \;P|M_{P} }
\end{mathpar} 

\begin{mathpar}
  \inferrule* [lab=sychronization] {} {M_{N} \bc \Box \;|\; x?M_{F} \;|\; x!M_{C}}
  \and
  \inferrule* [lab=abstraction] {} {{M_{F}} \bc (x)M_{P} }
  \and
  \inferrule* [lab=concretion] {} {{M_{C}} \bc \langle M_{P} \rangle }
  \and \\
  \inferrule* [lab=process] {} {{M_{P}} \bc M_{N} \;| \;P|M_{P} }
\end{mathpar}

\begin{definition}[contextual application] Given a context $M$, and
  process $P$, we define the \emph{contextual application}, $M[P] :=
  M\{P/\Box\}$. That is, the contextual application of M to P is the
  substitution of $P$ for $\Box$ in $M$.
\end{definition}

$\meaningof{-} : L \to \mathcal{P}(\pi)$

\begin{mathpar}
  \inferrule* [lab=collection] {} {\meaningof{true} = \pi, \and \meaningof{~E} = \pi \setminus \meaningof{E}, \and \meaningof{E_{1} \& E_{2}} = \meaningof{E_{1}} \cap \meaningof{E_{2}}}
\end{mathpar}

\begin{mathpar}
  \inferrule* [lab=structure] {} {\meaningof{0} = \{ P \in \pi | P \equiv 0 \}, \and \\ \meaningof{E_1 | E_2} = \{ P \in \pi | P \equiv P_{1} | P_{2}, P_{1} \in \meaningof{E_{1}}, P_{2} \in \meaningof{E_2}\} }
\end{mathpar}

\begin{mathpar}
 \inferrule* [lab=behavior] {} {\meaningof{\langle a?b \rangle E} = \{ P \in \pi | P \equiv Q | u?(y)P', \\ \and \\\\ \and \\ \;\;\; u \in \meaningof{a}, \forall z.P'\{z/y\} \in \meaningof{E\{z/b\}}\}, \and \\ \meaningof{a!E} = \{ P \in \pi | P \equiv Q | x!\langle P' \rangle, x \in \meaningof{a} P' \in \meaningof{E}\} }
\end{mathpar}

\begin{mathpar}
 \inferrule* [lab=nominal] {} {\meaningof{\quotep{E}} = \{ \quotep{P} \in \quotep{\pi} | P \in \meaningof{E} \}, \and \meaningof{\quotep{P}} = \{ \quotep{Q} \in \quotep{\pi} | P \equiv Q \} \and \\ \meaningof{@\quotep{E}} = \{ P \in \pi | P \equiv @x, x \in \meaningof{E} \}}
\end{mathpar}

\begin{eqnarray*}
  \\
  \meaningof{-} : TS \to ST
\end{eqnarray*}

\begin{eqnarray*}
  \\
  L : TS \to ST
\end{eqnarray*}

\begin{eqnarray*}
  \\
  P \models E \iff P \in \meaningof{E}
\end{eqnarray*}

\begin{eqnarray*}
  P \approx_{L} Q \iff \forall E \in L. P \models E \iff Q \models E
\end{eqnarray*}

\begin{eqnarray*}
  P \approx_{K} Q
\end{eqnarray*}

\begin{eqnarray*}
  P \approx Q
\end{eqnarray*}

$\approx_{K} = \approx = \approx_{L}$

\subsubsection{Contextual duality}

Note that contexts extend the quotation operation to a family of
operations from processes to names. Given a context, $M$, we can
define a \emph{nominal context}, $\quotep{M}$ by $\quotep{M}[P] :=
\quotep{M[P]}$. To foreshadow what is to come we observe that these
operations enjoy a duality with processes very much like the duality
between vectors and maps from vectors to scalars.

Further, because the calculus is essentially higher-order, we have a
correspondence between contexts and processes. More specifically,
given a name $x$ and a context $M$ we can construct $M^{*}_{x}$ such
that 

\begin{mathpar}
  M^{*}_{x} | \lift{x}{P} \red M[P]
\end{mathpar}

namely,

\begin{mathpar}
  M^{*}_{x} := x?(u).M[\dropn{u}]
\end{mathpar}

The dependence of $M^{*}_{x}$ on a name makes it an abstraction, 

\begin{mathpar}
  M^{*} := (x)x?(u).M[\dropn{u}]
\end{mathpar}

\subsection{Additional notation}

It will sometimes be convenient to denote the process a name
quotes. We already have the notation $x = \quotep{P}$, but it will be
convenient to introduce an alternate notation, $\procn{x}$, when we
want to emphasize the connection to the use of the name. Note that, by
virtue of name equivalence, $\quotep{\procn{x}} \nameeq x$; so, the
notation is consistent with previous definitions.

Further, because names have structure it is possible to effect
substitutions on the basis of that structure. This means we need to
upgrade our notation for substitutions, which we accomplish by
adapting comprehension notation. Thus,

\begin{mathpar}
  P\{ y / x : x \in S \}
\end{mathpar}

is interpreted to mean the process derived from P by replacing (in a
capture-avoiding manner) each occurrence of $x$ in $S$ by $y$. For example,

\begin{mathpar}
  P\{ \quotep{\procn{x}|\procn{x}} / x : x \in \freenames{P} \}
\end{mathpar}

will replace each (occurrence) of a free name $x$ in $P$ by
$\quotep{\procn{x}|\procn{x}}$.

Also, we will avail ourselves of the notation $x^{L}$ and $x^{R}$ to
denote injections of a name into disjoint copies of the name
space. There are numerous ways to accomplish this. One example can be
found in \cite{MeredithR05}. This notation overloads to vectors of
names: $\vec{x}^{\pi} := (x_{i}^{\pi} \; : \; 0 \leq i < |\vec{x}| )$ where $\pi \in \{L,R\}$.

We also use $P^{\Box} := P|\Box$.

In \cite{MeredithR05} an interpretation of the new operator is
given. It turns out that there are several possible interpretations
all enjoying the requisite algebraic properties of the operator (see
\cite{milner91polyadicpi}). We will therefore make liberal use of
$(\nu\; \vec{x})P$.

% subsection the_syntax_and_semantics_of_the_notation_system (end)   

\input{qm2pi.qmops} 

\input{qm2pi.sterngerlach} 

\input{qm2pi.metric} 

% section concurrent_process_calculi (end)

%\input{qm2pi.proofsketch}

% section proof sketch (end)

%\input{qm2pi.slviaknots} 

% section spatial logic via knots (end)

\input{qm2pi.conclusion}

% section conclusion (end)

%\input{qm2pi.dtcodes} 

% section wiring algorithm (end)

\input{qm2pi.ack} 

% section acknowledgments (end)

\newpage


\bibliographystyle{plain}   
\bibliography{../../biblios/main.bib}

\input{qm2pi.rhodetails}

\end{document}

 

% section notation (end)

\input{qm2pi.process.calculi} 

% section concurrent_process_calculi_and_spatial_logics_ (end)
    
%\documentclass[12pt]{llncs}
%\documentclass{jktr}

\usepackage[pdftex]{hyperref}                   
\usepackage {listings}
\usepackage {mathpartir}
\usepackage{bcprules}
%\usepackage{listings}
                       
\usepackage{graphicx} 
%\usepackage[margins=2.5cm,nohead,nofoot]{geometry}
%\usepackage{geometry}
\usepackage{amsfonts}
\usepackage{amstext}
\usepackage{latexsym}
\usepackage{amssymb}
\usepackage{color}


%\include{myPreamble}
\include{qm2pi.local} 

%\ifpdf
%\usepackage[pdftex]{graphicx}
%\else
%\usepackage{graphicx}
%\fi

 % \ifpdf
%  \usepackage{pdfsync}
%  \if


%\title{Brief Article}
%\author{David F. Snyder}
%\author{L.G. Meredith}

%\address{Dept. of Math., Texas State University--San Marcos, San Marcos, TX 78666}
       
\pagestyle{empty}


\begin{document}

\lstset{language=[Objective]Caml,frame=shadowbox}

\input{qm2pi.front}

% section front matter (end)

\input{qm2pi.intro} 
 
% section introduction (end)

% \input{qm2pi.knotations} 

% section notation (end)

\input{qm2pi.process.calculi} 

% section concurrent_process_calculi_and_spatial_logics_ (end)
    
%\input{qm2pi.knots2pi} 

%\input{qm2pi.trefoil} 

%\input{qm2pi.mainthm} 

% subsection basic_interpretation (end)

%\input{qm2pi.rho.presentation} 
\subsection{The syntax and semantics of the notation system}\label{sub:the_syntax_and_semantics_of_the_notation_system} % (fold)

We now summarize a technical presentation of the calculus that
embodies our theory of dynamics. The typical presentation of such a
calculus follows the style of giving generators and relations on
them. The grammar, below, describing term constructors, freely
generates the set of processes, $\Proc$. This set is then quotiented
by a relation known as structural congruence and it is over this set
that the notion of dynamics is expressed. This presentation is
essentially that of \cite{MeredithR05} with the addition of
polyadicity and summation. For readability we have relegated some of
the technical subtleties to an appendix.

\subsubsection{Process grammar}\label{subsub:process_grammar}

\begin{mathpar}
  \inferrule* [lab=synchronization] {} {{M} \bc \pzero \;|\; x?F \;|\; x!C }
  \and
  \inferrule* [lab=abstraction] {} {{F} \bc (x)P}
  \and
  \inferrule* [lab=concretion] {} {{C} \bc \langle Q \rangle}
  \and
  \inferrule* [lab=process] {} {{P,Q} \bc M \;| \;P|Q \;|\; @{x}}
  \and
  \inferrule* [lab=name] {} {{x} \bc \quotep{P}}
\end{mathpar} 

Note that $\vec{x}$ (resp. $\vec{P}$) denotes a vector of names
(resp. processes) of length $|\vec{x}|$ (resp. $|\vec{P}|$). We adopt
the following useful abbreviations.

\begin{mathpar}
   x?(\vec{y}).P := x.(\vec{y})P \and  x\clift{\vec{P}} := x.\clift{\vec{P}}
   \and x!(y) := \lift{x}{\dropn{y}}
   \and \Pi_{i=0}^{n-1}P_i := P_0 | \ldots | P_{n-1}
\end{mathpar}

\subsubsection{Structural congruence}

\paragraph{Free and bound names and alpha-equivalence.} At the
core of structural equivalence is alpha-equivalence which identifies
process that are the same up to a change of variable. Formally, we
recognize the distinction between free and bound names. The free names
of a process, $\freenames{P}$, may be calculated recursively as
follows:

\begin{mathpar}
\freenames{\pzero} := \emptyset
  \and \\
  \freenames{x?(y).P} := \{ x \} \cup (\freenames{P} \setminus \{ y \})
  \and 
  \freenames{x!\langle P \rangle} := \{ x \} \cup \{ P \} 
  \and \\
  \freenames{P|Q} := \freenames{P} \cup \freenames{Q}
  \and \\
  \freenames{@{x}} := \{ x \}
\end{mathpar}

$\pi$
$\quotep{\pi}$

$\freenames{-} : \pi \to \mathcal{P}(\quotep{\pi})$

\begin{eqnarray*}
  \freenames{\pzero} & := & \emptyset \\
  \freenames{x?(y).P} & := & \{ x \} \cup (\freenames{P} \setminus \{ y \}) \\
  \freenames{x!\langle P \rangle} & := & \{ x \} \cup \{ P \} \\
  \freenames{P|Q} & := & \freenames{P} \cup \freenames{Q} \\
  \freenames{\dropn{x}} & := & \{ x \}
\end{eqnarray*}

The bound names of a process, $\boundnames{P}$, are those names occurring in $P$
that are not free. For example, in $x?(y).0$, the name $x$ is free, while $y$ is bound.

\begin{mathpar}
  \inferrule* [lab=monoidal-laws] {} { P|Q \equiv Q|P \and P|0 \equiv P \and P|(Q|R) \equiv (P|Q)|R }
\end{mathpar}

\begin{mathpar}
  \inferrule* [lab=alpha-equivalence] {} { (x)P \equiv (y)P\{y/x\} \and y \not\in \freenames{P} }
\end{mathpar}

\begin{definition}
Then two processes, $P,Q$, are alpha-equivalent if $P = Q\{\vec{y}/\vec{x}\}$ for
some $\vec{x} \in \boundnames{Q},\vec{y} \in \boundnames{P}$, where $Q\{\vec{y}/\vec{x}\}$
denotes the capture-avoiding substitution of $\vec{y}$ for $\vec{x}$ in $Q$.
\end{definition}

\begin{definition}
  The {\em structural congruence} \cite{SangiorgiWalker} , $\equiv$,
  between processes is the least congruence containing
  alpha-equivalence, satisfying the abelian monoid laws
  (associativity, commutativity and $\pzero$ as identity) for parallel
  composition $|$ and for summation $+$.
\end{definition}

\subsection{Name equivalence}

We take name equivalence, written $\nameeq$, to be the smallest
equivalence relation generated by the following rules.

\begin{mathpar}
\inferrule*[lab=Quote-drop]
{ }
{ \quotep{@{x}} \nameeq x }

\inferrule*[lab=Struct-equiv]
{ P \scong Q }
{ \quotep{P} \nameeq \quotep{Q} }
\end{mathpar}

The astute reader will have noticed that the mutual recursion of names
and processes imposes a mutual recursion on alpha-equivalence and
structural equivalence via name-equivalence. Fortunately, all of this
works out pleasantly and we may calculate in the natural way, free of
concern. The reader interested in the details is referred to the
appendix \ref{appendix:rho_details}.

\subsection{Substitution}

We use $\Proc$ for the set of processes, $\QProc$ for the set of
names, and $\id{\{}\vec{y} / \vec{x} \id{\}}$ to denote partial maps,
$s : \QProc \rightarrow \QProc$. A map, $s$ lifts, uniquely, to a map
on process terms, $\widehat{s} : \Proc \rightarrow \Proc$ by the
following equations.

\begin{mathpar}
  (0) \psubstp{Q}{P} := 0 \\
  (R \juxtap S) \psubstp{Q}{P}
  :=    
  (R)\psubstp{Q}{P} \juxtap (S) \psubstp{Q}{P} \\
  (x?(y).R) \psubstp{Q}{P}    
  :=    
  (x)\substp{Q}{P} (z)\concat( (R \psubstn{z}{y}) \psubstp{Q}{P} ) \\
  (\lift{x}{R}) \psubstp{Q}{P}  
  :=
  \lift{(x)\substp{Q}{P}}{ R \psubstp{Q}{P} } \\
%   (\dropn{x})  \psubstp{Q}{P}       
%   := 
%   \left\{ 
%     \begin{array}{ccc} 
%       \dropn{\quotep{Q}} & & x \nameeq \quotep{P} \\
%       \dropn{x} & & otherwise \\
%     \end{array}
%   \right. 
  (\dropn{x})  \psubstp{Q}{P}       
  := 
  \left\{ 
    \begin{array}{ccc} 
      Q & & x \nameeq \quotep{P} \\
      \dropn{x} & & otherwise \\
    \end{array}
  \right.
\end{mathpar}
 

where

\begin{eqnarray}
  (x)\id{\{} \lpquote Q \rpquote / \lpquote P \rpquote \id{\}}            = 
  \left\{ 
    \begin{array}{ccc}
      \lpquote Q \rpquote & & x \nameeq \lpquote P \rpquote \\
      x & & otherwise \\
    \end{array}
  \right. \nonumber
\end{eqnarray}

and $z$ is chosen distinct from $\quotep{P}$, $\quotep{Q}$, the free
names in $Q$, and all the names in $R$. Our $\alpha$-equivalence will
be built in the standard way from this substitution.

\begin{remark}\label{rem:no_self_referential_names}
  One consequence of these definitions is that $\forall P. \quotep{P}
  \not\in \freenames{P}$.
\end{remark}

\subsection{ Dynamic quote: an example }

Anticipating something of what's to come, consider applying the
substitution, $\widehat{\id{\{}u / z \id{\}}}$, to the following pair
of processes, $\lift{w}{y!(z)}$ and $w[ \lpquote y!(z) \rpquote ]$.

\begin{eqnarray}
	\lift{w}{y!(z)}\widehat{\id{\{}u / z \id{\}}}
		& = &
		\lift{w}{y!(u)} \nonumber\\
	w[ \lpquote y!(z) \rpquote ] \widehat{ \id{\{}u / z \id{\}} }
		& = &
		w[ \lpquote y!(z) \rpquote ] \nonumber
\end{eqnarray}

Because the body of the process between quotes is impervious to
substitution, we get radically different answers. In fact, by
examining the first process in an input context,
e.g. $x?(z).\lift{w}{y!(z)}$, we see that the process under the lift
operator may be shaped by prefixed inputs binding a name inside it. In
this sense, the lift operator will be seen as a way to dynamically
construct processes before reifying them as names.

Finally equipped with these standard features we can present the
dynamics of the calculus.

\subsubsection{Operational semantics} 

Finally, we introduce the computational dynamics. What marks these
algebras as distinct from other more traditionally studied algebraic
structures, e.g. vector spaces or polynomial rings, is the manner in
which dynamics is captured. In traditional structures, dynamics is typically
expressed through morphisms between such structures, as in linear maps
between vector spaces or morphisms between rings. In algebras
associated with the semantics of computation, the dynamics is
expressed as part of the algebraic structure itself, through a
reduction reduction relation typically denoted by $\red$. Below, we
give a recursive presentation of this relation for the calculus used
in the encoding.

$\red \subseteq \pi \times \pi$
$\red : \pi \to \mathcal{P}(\pi)$

\begin{mathpar}
  \inferrule* [lab=Comm] { \textsf{match}( x_{src}, x_{trgt} ) } { x_{trgt}?(y)P \; | \; x_{src}!\langle {Q} \rangle \red P\{\quotep{Q}/y}\} }
  \and \\
  \inferrule* [lab=Par] {{P} \red {P}'} {{{P} | {Q}} \red {{P}' | {Q}}}
  \and
  \inferrule* [lab=Equiv]{{{P} \scong {P}'} \andalso {{P}' \red {Q}'} \andalso {{Q}' \scong {Q}}}{{P} \red {Q}}
\end{mathpar}

\begin{eqnarray*}
  match_{\equiv} (\quotep{P},\quotep{Q}) & := & P \equiv Q \\
  match_{\dagger}(\quotep{P},\quotep{Q}) & := & \forall R. P|Q \red^{*} R => R \red^{*} 0 \\
  match_{K}(\quotep{P},\quotep{Q}) & := & K \mbox{ for some context } K
\end{eqnarray*}

$u?(x)P | u!\langle Q \rangle \red P\{\quotep{Q}/x\}$

%We write $\wred$ for $\red^*$, and $P\red$ if $\exists Q $ such that $ P \red Q$.
We write $P\red$ if $\exists Q $ such that $ P \red Q$ and $P\not\red$, otherwise.

\section{Replication}

As mentioned before, it is known that replication (and hence
recursion) can be implemented in a higher-order process algebra
\cite{SangiorgiWalker}. As our first example of calculation with the
machinery thus far presented we give the construction explicitly in
the {\rhoc}.

\begin{eqnarray}
	D_{x} & := & \prefix{x}{y}{(\binpar{\outputp{x}{y}}{@{y}})} \nonumber\\
	\bangp_{x}{P} & := & \binpar{{x}!\langle{\binpar{D_{x}}{P}}\rangle}{D_{x}} \nonumber
\end{eqnarray}

\begin{eqnarray}
	\bangp_{x}{P} & & \nonumber\\
	=
	& {x}!\langle{(\prefix{x}{y}{(\outputp{x}{y} | @{y})) | P}}\rangle 
	      | \prefix{x}{y}{(\outputp{x}{y} | @{y})} & \nonumber\\
	\red
	& (\outputp{x}{y} | @{y})\substn{\quotep{(\prefix{x}{y}{(@{y} | \outputp{x}{y})) | P}}}{y} & \nonumber\\
	=
	& \outputp{x}{\quotep{(\prefix{x}{y}{(\outputp{x}{y} | @{y})) | P}}}
	  | {(\prefix{x}{y}{(\outputp{x}{y} | @{y})) | P}} & \nonumber\\
	\red
	& \ldots & \nonumber\\
	\red^*
	& P | P | \ldots & \nonumber
\end{eqnarray}

Of course, this encoding, as an implementation, runs away, unfolding
$\bangp{P}$ eagerly. A lazier and more implementable replication
operator, restricted to input-guarded processes, may be obtained as follows.

\begin{eqnarray}
\bangp{\prefix{u}{v}{P}} 
	:= 
	\binpar{\lift{x}{\prefix{u}{v}{(\binpar{D(x)}{P})}}}{D(x)} \nonumber
\end{eqnarray}

\begin{remark}
  Note that the lazier definition still does not deal with summation
  or mixed summation (i.e. sums over input and output). The reader is
  invited to construct definitions of replication that deal with these
  features. 

  Further, the definitions are parameterized in a name, $x$. Can you,
  gentle reader, make a definition that eliminates this parameter and
  guarantees no accidental interaction between the replication
  machinery and the process being replicated -- i.e. no accidental
  sharing of names used by the process to get its work done and the
  name(s) used by the replication to effect copying. This latter
  revision of the definition of replication is crucial to obtaining
  the expected identity $!!P \sim !P$.
\end{remark}

\begin{remark}\label{rem:paradoxical_combinator}
  The reader familiar with the lambda calculus will have noticed the
  similarity between $D$ and the paradoxical combinator.

  [Ed. note: the existence of this seems to suggest we have to be more
  restrictive on the set of processes and names we admit if we are to
  support no-cloning.]
\end{remark}

\subsubsection{Bisimulation}

The computational dynamics gives rise to another kind of equivalence,
the equivalence of computational behavior. As previously mentioned
this is typically captured \emph{via} some form of bisimulation.

% The notion we use in this paper is weak barbed bisimulation
% \cite{milner91polyadicpi}.

The notion we use in this paper is derived from weak barbed
bisimulation \cite{milner91polyadicpi}. 

\begin{definition}
An \emph{observation relation}, $\downarrow_{\mathcal N}$, over a set
of names, $\mathcal N$, is the smallest relation satisfying the rules
below.

\infrule[Out-barb]{y \in {\mathcal N}, \; x \nameeq y}
		  {\outputp{x}{v} \downarrow_{\mathcal N} x}
\infrule[Par-barb]{\mbox{$P\downarrow_{\mathcal N} x$ or $Q\downarrow_{\mathcal N} x$}}
		  {\binpar{P}{Q} \downarrow_{\mathcal N} x}

We write $P \Downarrow_{\mathcal N} x$ if there is $Q$ such that 
$P \wred Q$ and $Q \downarrow_{\mathcal N} x$.
\end{definition}

\begin{definition}
%\label{def.bbisim}
An  ${\mathcal N}$-\emph{barbed bisimulation} over a set of names, ${\mathcal N}$, is a symmetric binary relation 
${\mathcal S}_{\mathcal N}$ between agents such that $P\rel{S}_{\mathcal N}Q$ implies:
\begin{enumerate}
\item If $P \red P'$ then $Q \wred Q'$ and $P'\rel{S}_{\mathcal N} Q'$.
\item If $P\downarrow_{\mathcal N} x$, then $Q\Downarrow_{\mathcal N} x$.
\end{enumerate}
$P$ is ${\mathcal N}$-barbed bisimilar to $Q$, written
$P \wbbisim_{\mathcal N} Q$, if $P \rel{S}_{\mathcal N} Q$ for some ${\mathcal N}$-barbed bisimulation ${\mathcal S}_{\mathcal N}$.
\end{definition}

$\mathcal{R} \subseteq \pi \times \pi$

$P \mathcal{R} Q => \forall P'. P \red P' \Rightarrow \exists Q'. Q \red Q', P' \mathcal{R} Q'$

$P \vdash x \Rightarrow Q \vdash x$

\begin{mathpar}
  \inferrule*[lab=Out-barb]{x \nameeq y}{{y}!\langle{Q}\rangle \vdash x}
  \and
  \inferrule*[lab=Par-barb]{\mbox{$P\vdash x$ or $Q\vdash x$}}{\binpar{P}{Q} \vdash x}
\end{mathpar}

\subsubsection{Contexts}

One of the principle advantages of computational calculi like the
$\pi$-calculus is a well-defined notion of context,
contextual-equivalence and a correlation between
contextual-equivalence and notions of bisimulation. The notion of
context allows the decomposition of a process into (sub-)process and
its syntactic environment, its context. Thus, a context may be
thought of as a process with a ``hole'' (written $\Box$) in it. The
application of a context $M$ to a process $P$, written $M[P]$, is
tantamount to filling the hole in $M$ with $P$. In this paper we do
not need the full weight of this theory, but do make use of the notion
of context in the proof the main theorem. 

\begin{mathpar}
  \inferrule* [lab=summation] {} {{M_{M},M_{N}} \bc \Box \;|\; x.M_{A} \;|\; M_{M}+M_{N}}
  \and
  \inferrule* [lab=agent] {} {{M_{A}} \bc (\vec{x})M_{P} \;| \; \clift{P_0,\ldots,M_{P},\ldots,P_N}}
  \and \\
  \inferrule* [lab=process] {} {{M_{P}} \bc M_{N} \;| \;P|M_{P} }
\end{mathpar} 

\begin{mathpar}
  \inferrule* [lab=sychronization] {} {M_{N} \bc \Box \;|\; x?M_{F} \;|\; x!M_{C}}
  \and
  \inferrule* [lab=abstraction] {} {{M_{F}} \bc (x)M_{P} }
  \and
  \inferrule* [lab=concretion] {} {{M_{C}} \bc \langle M_{P} \rangle }
  \and \\
  \inferrule* [lab=process] {} {{M_{P}} \bc M_{N} \;| \;P|M_{P} }
\end{mathpar}

\begin{definition}[contextual application] Given a context $M$, and
  process $P$, we define the \emph{contextual application}, $M[P] :=
  M\{P/\Box\}$. That is, the contextual application of M to P is the
  substitution of $P$ for $\Box$ in $M$.
\end{definition}

$\meaningof{-} : L \to \mathcal{P}(\pi)$

\begin{mathpar}
  \inferrule* [lab=collection] {} {\meaningof{true} = \pi, \and \meaningof{~E} = \pi \setminus \meaningof{E}, \and \meaningof{E_{1} \& E_{2}} = \meaningof{E_{1}} \cap \meaningof{E_{2}}}
\end{mathpar}

\begin{mathpar}
  \inferrule* [lab=structure] {} {\meaningof{0} = \{ P \in \pi | P \equiv 0 \}, \and \\ \meaningof{E_1 | E_2} = \{ P \in \pi | P \equiv P_{1} | P_{2}, P_{1} \in \meaningof{E_{1}}, P_{2} \in \meaningof{E_2}\} }
\end{mathpar}

\begin{mathpar}
 \inferrule* [lab=behavior] {} {\meaningof{\langle a?b \rangle E} = \{ P \in \pi | P \equiv Q | u?(y)P', \\ \and \\\\ \and \\ \;\;\; u \in \meaningof{a}, \forall z.P'\{z/y\} \in \meaningof{E\{z/b\}}\}, \and \\ \meaningof{a!E} = \{ P \in \pi | P \equiv Q | x!\langle P' \rangle, x \in \meaningof{a} P' \in \meaningof{E}\} }
\end{mathpar}

\begin{mathpar}
 \inferrule* [lab=nominal] {} {\meaningof{\quotep{E}} = \{ \quotep{P} \in \quotep{\pi} | P \in \meaningof{E} \}, \and \meaningof{\quotep{P}} = \{ \quotep{Q} \in \quotep{\pi} | P \equiv Q \} \and \\ \meaningof{@\quotep{E}} = \{ P \in \pi | P \equiv @x, x \in \meaningof{E} \}}
\end{mathpar}

\begin{eqnarray*}
  \\
  \meaningof{-} : TS \to ST
\end{eqnarray*}

\begin{eqnarray*}
  \\
  L : TS \to ST
\end{eqnarray*}

\begin{eqnarray*}
  \\
  P \models E \iff P \in \meaningof{E}
\end{eqnarray*}

\begin{eqnarray*}
  P \approx_{L} Q \iff \forall E \in L. P \models E \iff Q \models E
\end{eqnarray*}

\begin{eqnarray*}
  P \approx_{K} Q
\end{eqnarray*}

\begin{eqnarray*}
  P \approx Q
\end{eqnarray*}

$\approx_{K} = \approx = \approx_{L}$

\subsubsection{Contextual duality}

Note that contexts extend the quotation operation to a family of
operations from processes to names. Given a context, $M$, we can
define a \emph{nominal context}, $\quotep{M}$ by $\quotep{M}[P] :=
\quotep{M[P]}$. To foreshadow what is to come we observe that these
operations enjoy a duality with processes very much like the duality
between vectors and maps from vectors to scalars.

Further, because the calculus is essentially higher-order, we have a
correspondence between contexts and processes. More specifically,
given a name $x$ and a context $M$ we can construct $M^{*}_{x}$ such
that 

\begin{mathpar}
  M^{*}_{x} | \lift{x}{P} \red M[P]
\end{mathpar}

namely,

\begin{mathpar}
  M^{*}_{x} := x?(u).M[\dropn{u}]
\end{mathpar}

The dependence of $M^{*}_{x}$ on a name makes it an abstraction, 

\begin{mathpar}
  M^{*} := (x)x?(u).M[\dropn{u}]
\end{mathpar}

\subsection{Additional notation}

It will sometimes be convenient to denote the process a name
quotes. We already have the notation $x = \quotep{P}$, but it will be
convenient to introduce an alternate notation, $\procn{x}$, when we
want to emphasize the connection to the use of the name. Note that, by
virtue of name equivalence, $\quotep{\procn{x}} \nameeq x$; so, the
notation is consistent with previous definitions.

Further, because names have structure it is possible to effect
substitutions on the basis of that structure. This means we need to
upgrade our notation for substitutions, which we accomplish by
adapting comprehension notation. Thus,

\begin{mathpar}
  P\{ y / x : x \in S \}
\end{mathpar}

is interpreted to mean the process derived from P by replacing (in a
capture-avoiding manner) each occurrence of $x$ in $S$ by $y$. For example,

\begin{mathpar}
  P\{ \quotep{\procn{x}|\procn{x}} / x : x \in \freenames{P} \}
\end{mathpar}

will replace each (occurrence) of a free name $x$ in $P$ by
$\quotep{\procn{x}|\procn{x}}$.

Also, we will avail ourselves of the notation $x^{L}$ and $x^{R}$ to
denote injections of a name into disjoint copies of the name
space. There are numerous ways to accomplish this. One example can be
found in \cite{MeredithR05}. This notation overloads to vectors of
names: $\vec{x}^{\pi} := (x_{i}^{\pi} \; : \; 0 \leq i < |\vec{x}| )$ where $\pi \in \{L,R\}$.

We also use $P^{\Box} := P|\Box$.

In \cite{MeredithR05} an interpretation of the new operator is
given. It turns out that there are several possible interpretations
all enjoying the requisite algebraic properties of the operator (see
\cite{milner91polyadicpi}). We will therefore make liberal use of
$(\nu\; \vec{x})P$.

% subsection the_syntax_and_semantics_of_the_notation_system (end)   

\input{qm2pi.qmops} 

\input{qm2pi.sterngerlach} 

\input{qm2pi.metric} 

% section concurrent_process_calculi (end)

%\input{qm2pi.proofsketch}

% section proof sketch (end)

%\input{qm2pi.slviaknots} 

% section spatial logic via knots (end)

\input{qm2pi.conclusion}

% section conclusion (end)

%\input{qm2pi.dtcodes} 

% section wiring algorithm (end)

\input{qm2pi.ack} 

% section acknowledgments (end)

\newpage


\bibliographystyle{plain}   
\bibliography{../../biblios/main.bib}

\input{qm2pi.rhodetails}

\end{document}

 

%\documentclass[12pt]{llncs}
%\documentclass{jktr}

\usepackage[pdftex]{hyperref}                   
\usepackage {listings}
\usepackage {mathpartir}
\usepackage{bcprules}
%\usepackage{listings}
                       
\usepackage{graphicx} 
%\usepackage[margins=2.5cm,nohead,nofoot]{geometry}
%\usepackage{geometry}
\usepackage{amsfonts}
\usepackage{amstext}
\usepackage{latexsym}
\usepackage{amssymb}
\usepackage{color}


%\include{myPreamble}
\include{qm2pi.local} 

%\ifpdf
%\usepackage[pdftex]{graphicx}
%\else
%\usepackage{graphicx}
%\fi

 % \ifpdf
%  \usepackage{pdfsync}
%  \if


%\title{Brief Article}
%\author{David F. Snyder}
%\author{L.G. Meredith}

%\address{Dept. of Math., Texas State University--San Marcos, San Marcos, TX 78666}
       
\pagestyle{empty}


\begin{document}

\lstset{language=[Objective]Caml,frame=shadowbox}

\input{qm2pi.front}

% section front matter (end)

\input{qm2pi.intro} 
 
% section introduction (end)

% \input{qm2pi.knotations} 

% section notation (end)

\input{qm2pi.process.calculi} 

% section concurrent_process_calculi_and_spatial_logics_ (end)
    
%\input{qm2pi.knots2pi} 

%\input{qm2pi.trefoil} 

%\input{qm2pi.mainthm} 

% subsection basic_interpretation (end)

%\input{qm2pi.rho.presentation} 
\subsection{The syntax and semantics of the notation system}\label{sub:the_syntax_and_semantics_of_the_notation_system} % (fold)

We now summarize a technical presentation of the calculus that
embodies our theory of dynamics. The typical presentation of such a
calculus follows the style of giving generators and relations on
them. The grammar, below, describing term constructors, freely
generates the set of processes, $\Proc$. This set is then quotiented
by a relation known as structural congruence and it is over this set
that the notion of dynamics is expressed. This presentation is
essentially that of \cite{MeredithR05} with the addition of
polyadicity and summation. For readability we have relegated some of
the technical subtleties to an appendix.

\subsubsection{Process grammar}\label{subsub:process_grammar}

\begin{mathpar}
  \inferrule* [lab=synchronization] {} {{M} \bc \pzero \;|\; x?F \;|\; x!C }
  \and
  \inferrule* [lab=abstraction] {} {{F} \bc (x)P}
  \and
  \inferrule* [lab=concretion] {} {{C} \bc \langle Q \rangle}
  \and
  \inferrule* [lab=process] {} {{P,Q} \bc M \;| \;P|Q \;|\; @{x}}
  \and
  \inferrule* [lab=name] {} {{x} \bc \quotep{P}}
\end{mathpar} 

Note that $\vec{x}$ (resp. $\vec{P}$) denotes a vector of names
(resp. processes) of length $|\vec{x}|$ (resp. $|\vec{P}|$). We adopt
the following useful abbreviations.

\begin{mathpar}
   x?(\vec{y}).P := x.(\vec{y})P \and  x\clift{\vec{P}} := x.\clift{\vec{P}}
   \and x!(y) := \lift{x}{\dropn{y}}
   \and \Pi_{i=0}^{n-1}P_i := P_0 | \ldots | P_{n-1}
\end{mathpar}

\subsubsection{Structural congruence}

\paragraph{Free and bound names and alpha-equivalence.} At the
core of structural equivalence is alpha-equivalence which identifies
process that are the same up to a change of variable. Formally, we
recognize the distinction between free and bound names. The free names
of a process, $\freenames{P}$, may be calculated recursively as
follows:

\begin{mathpar}
\freenames{\pzero} := \emptyset
  \and \\
  \freenames{x?(y).P} := \{ x \} \cup (\freenames{P} \setminus \{ y \})
  \and 
  \freenames{x!\langle P \rangle} := \{ x \} \cup \{ P \} 
  \and \\
  \freenames{P|Q} := \freenames{P} \cup \freenames{Q}
  \and \\
  \freenames{@{x}} := \{ x \}
\end{mathpar}

$\pi$
$\quotep{\pi}$

$\freenames{-} : \pi \to \mathcal{P}(\quotep{\pi})$

\begin{eqnarray*}
  \freenames{\pzero} & := & \emptyset \\
  \freenames{x?(y).P} & := & \{ x \} \cup (\freenames{P} \setminus \{ y \}) \\
  \freenames{x!\langle P \rangle} & := & \{ x \} \cup \{ P \} \\
  \freenames{P|Q} & := & \freenames{P} \cup \freenames{Q} \\
  \freenames{\dropn{x}} & := & \{ x \}
\end{eqnarray*}

The bound names of a process, $\boundnames{P}$, are those names occurring in $P$
that are not free. For example, in $x?(y).0$, the name $x$ is free, while $y$ is bound.

\begin{mathpar}
  \inferrule* [lab=monoidal-laws] {} { P|Q \equiv Q|P \and P|0 \equiv P \and P|(Q|R) \equiv (P|Q)|R }
\end{mathpar}

\begin{mathpar}
  \inferrule* [lab=alpha-equivalence] {} { (x)P \equiv (y)P\{y/x\} \and y \not\in \freenames{P} }
\end{mathpar}

\begin{definition}
Then two processes, $P,Q$, are alpha-equivalent if $P = Q\{\vec{y}/\vec{x}\}$ for
some $\vec{x} \in \boundnames{Q},\vec{y} \in \boundnames{P}$, where $Q\{\vec{y}/\vec{x}\}$
denotes the capture-avoiding substitution of $\vec{y}$ for $\vec{x}$ in $Q$.
\end{definition}

\begin{definition}
  The {\em structural congruence} \cite{SangiorgiWalker} , $\equiv$,
  between processes is the least congruence containing
  alpha-equivalence, satisfying the abelian monoid laws
  (associativity, commutativity and $\pzero$ as identity) for parallel
  composition $|$ and for summation $+$.
\end{definition}

\subsection{Name equivalence}

We take name equivalence, written $\nameeq$, to be the smallest
equivalence relation generated by the following rules.

\begin{mathpar}
\inferrule*[lab=Quote-drop]
{ }
{ \quotep{@{x}} \nameeq x }

\inferrule*[lab=Struct-equiv]
{ P \scong Q }
{ \quotep{P} \nameeq \quotep{Q} }
\end{mathpar}

The astute reader will have noticed that the mutual recursion of names
and processes imposes a mutual recursion on alpha-equivalence and
structural equivalence via name-equivalence. Fortunately, all of this
works out pleasantly and we may calculate in the natural way, free of
concern. The reader interested in the details is referred to the
appendix \ref{appendix:rho_details}.

\subsection{Substitution}

We use $\Proc$ for the set of processes, $\QProc$ for the set of
names, and $\id{\{}\vec{y} / \vec{x} \id{\}}$ to denote partial maps,
$s : \QProc \rightarrow \QProc$. A map, $s$ lifts, uniquely, to a map
on process terms, $\widehat{s} : \Proc \rightarrow \Proc$ by the
following equations.

\begin{mathpar}
  (0) \psubstp{Q}{P} := 0 \\
  (R \juxtap S) \psubstp{Q}{P}
  :=    
  (R)\psubstp{Q}{P} \juxtap (S) \psubstp{Q}{P} \\
  (x?(y).R) \psubstp{Q}{P}    
  :=    
  (x)\substp{Q}{P} (z)\concat( (R \psubstn{z}{y}) \psubstp{Q}{P} ) \\
  (\lift{x}{R}) \psubstp{Q}{P}  
  :=
  \lift{(x)\substp{Q}{P}}{ R \psubstp{Q}{P} } \\
%   (\dropn{x})  \psubstp{Q}{P}       
%   := 
%   \left\{ 
%     \begin{array}{ccc} 
%       \dropn{\quotep{Q}} & & x \nameeq \quotep{P} \\
%       \dropn{x} & & otherwise \\
%     \end{array}
%   \right. 
  (\dropn{x})  \psubstp{Q}{P}       
  := 
  \left\{ 
    \begin{array}{ccc} 
      Q & & x \nameeq \quotep{P} \\
      \dropn{x} & & otherwise \\
    \end{array}
  \right.
\end{mathpar}
 

where

\begin{eqnarray}
  (x)\id{\{} \lpquote Q \rpquote / \lpquote P \rpquote \id{\}}            = 
  \left\{ 
    \begin{array}{ccc}
      \lpquote Q \rpquote & & x \nameeq \lpquote P \rpquote \\
      x & & otherwise \\
    \end{array}
  \right. \nonumber
\end{eqnarray}

and $z$ is chosen distinct from $\quotep{P}$, $\quotep{Q}$, the free
names in $Q$, and all the names in $R$. Our $\alpha$-equivalence will
be built in the standard way from this substitution.

\begin{remark}\label{rem:no_self_referential_names}
  One consequence of these definitions is that $\forall P. \quotep{P}
  \not\in \freenames{P}$.
\end{remark}

\subsection{ Dynamic quote: an example }

Anticipating something of what's to come, consider applying the
substitution, $\widehat{\id{\{}u / z \id{\}}}$, to the following pair
of processes, $\lift{w}{y!(z)}$ and $w[ \lpquote y!(z) \rpquote ]$.

\begin{eqnarray}
	\lift{w}{y!(z)}\widehat{\id{\{}u / z \id{\}}}
		& = &
		\lift{w}{y!(u)} \nonumber\\
	w[ \lpquote y!(z) \rpquote ] \widehat{ \id{\{}u / z \id{\}} }
		& = &
		w[ \lpquote y!(z) \rpquote ] \nonumber
\end{eqnarray}

Because the body of the process between quotes is impervious to
substitution, we get radically different answers. In fact, by
examining the first process in an input context,
e.g. $x?(z).\lift{w}{y!(z)}$, we see that the process under the lift
operator may be shaped by prefixed inputs binding a name inside it. In
this sense, the lift operator will be seen as a way to dynamically
construct processes before reifying them as names.

Finally equipped with these standard features we can present the
dynamics of the calculus.

\subsubsection{Operational semantics} 

Finally, we introduce the computational dynamics. What marks these
algebras as distinct from other more traditionally studied algebraic
structures, e.g. vector spaces or polynomial rings, is the manner in
which dynamics is captured. In traditional structures, dynamics is typically
expressed through morphisms between such structures, as in linear maps
between vector spaces or morphisms between rings. In algebras
associated with the semantics of computation, the dynamics is
expressed as part of the algebraic structure itself, through a
reduction reduction relation typically denoted by $\red$. Below, we
give a recursive presentation of this relation for the calculus used
in the encoding.

$\red \subseteq \pi \times \pi$
$\red : \pi \to \mathcal{P}(\pi)$

\begin{mathpar}
  \inferrule* [lab=Comm] { \textsf{match}( x_{src}, x_{trgt} ) } { x_{trgt}?(y)P \; | \; x_{src}!\langle {Q} \rangle \red P\{\quotep{Q}/y}\} }
  \and \\
  \inferrule* [lab=Par] {{P} \red {P}'} {{{P} | {Q}} \red {{P}' | {Q}}}
  \and
  \inferrule* [lab=Equiv]{{{P} \scong {P}'} \andalso {{P}' \red {Q}'} \andalso {{Q}' \scong {Q}}}{{P} \red {Q}}
\end{mathpar}

\begin{eqnarray*}
  match_{\equiv} (\quotep{P},\quotep{Q}) & := & P \equiv Q \\
  match_{\dagger}(\quotep{P},\quotep{Q}) & := & \forall R. P|Q \red^{*} R => R \red^{*} 0 \\
  match_{K}(\quotep{P},\quotep{Q}) & := & K \mbox{ for some context } K
\end{eqnarray*}

$u?(x)P | u!\langle Q \rangle \red P\{\quotep{Q}/x\}$

%We write $\wred$ for $\red^*$, and $P\red$ if $\exists Q $ such that $ P \red Q$.
We write $P\red$ if $\exists Q $ such that $ P \red Q$ and $P\not\red$, otherwise.

\section{Replication}

As mentioned before, it is known that replication (and hence
recursion) can be implemented in a higher-order process algebra
\cite{SangiorgiWalker}. As our first example of calculation with the
machinery thus far presented we give the construction explicitly in
the {\rhoc}.

\begin{eqnarray}
	D_{x} & := & \prefix{x}{y}{(\binpar{\outputp{x}{y}}{@{y}})} \nonumber\\
	\bangp_{x}{P} & := & \binpar{{x}!\langle{\binpar{D_{x}}{P}}\rangle}{D_{x}} \nonumber
\end{eqnarray}

\begin{eqnarray}
	\bangp_{x}{P} & & \nonumber\\
	=
	& {x}!\langle{(\prefix{x}{y}{(\outputp{x}{y} | @{y})) | P}}\rangle 
	      | \prefix{x}{y}{(\outputp{x}{y} | @{y})} & \nonumber\\
	\red
	& (\outputp{x}{y} | @{y})\substn{\quotep{(\prefix{x}{y}{(@{y} | \outputp{x}{y})) | P}}}{y} & \nonumber\\
	=
	& \outputp{x}{\quotep{(\prefix{x}{y}{(\outputp{x}{y} | @{y})) | P}}}
	  | {(\prefix{x}{y}{(\outputp{x}{y} | @{y})) | P}} & \nonumber\\
	\red
	& \ldots & \nonumber\\
	\red^*
	& P | P | \ldots & \nonumber
\end{eqnarray}

Of course, this encoding, as an implementation, runs away, unfolding
$\bangp{P}$ eagerly. A lazier and more implementable replication
operator, restricted to input-guarded processes, may be obtained as follows.

\begin{eqnarray}
\bangp{\prefix{u}{v}{P}} 
	:= 
	\binpar{\lift{x}{\prefix{u}{v}{(\binpar{D(x)}{P})}}}{D(x)} \nonumber
\end{eqnarray}

\begin{remark}
  Note that the lazier definition still does not deal with summation
  or mixed summation (i.e. sums over input and output). The reader is
  invited to construct definitions of replication that deal with these
  features. 

  Further, the definitions are parameterized in a name, $x$. Can you,
  gentle reader, make a definition that eliminates this parameter and
  guarantees no accidental interaction between the replication
  machinery and the process being replicated -- i.e. no accidental
  sharing of names used by the process to get its work done and the
  name(s) used by the replication to effect copying. This latter
  revision of the definition of replication is crucial to obtaining
  the expected identity $!!P \sim !P$.
\end{remark}

\begin{remark}\label{rem:paradoxical_combinator}
  The reader familiar with the lambda calculus will have noticed the
  similarity between $D$ and the paradoxical combinator.

  [Ed. note: the existence of this seems to suggest we have to be more
  restrictive on the set of processes and names we admit if we are to
  support no-cloning.]
\end{remark}

\subsubsection{Bisimulation}

The computational dynamics gives rise to another kind of equivalence,
the equivalence of computational behavior. As previously mentioned
this is typically captured \emph{via} some form of bisimulation.

% The notion we use in this paper is weak barbed bisimulation
% \cite{milner91polyadicpi}.

The notion we use in this paper is derived from weak barbed
bisimulation \cite{milner91polyadicpi}. 

\begin{definition}
An \emph{observation relation}, $\downarrow_{\mathcal N}$, over a set
of names, $\mathcal N$, is the smallest relation satisfying the rules
below.

\infrule[Out-barb]{y \in {\mathcal N}, \; x \nameeq y}
		  {\outputp{x}{v} \downarrow_{\mathcal N} x}
\infrule[Par-barb]{\mbox{$P\downarrow_{\mathcal N} x$ or $Q\downarrow_{\mathcal N} x$}}
		  {\binpar{P}{Q} \downarrow_{\mathcal N} x}

We write $P \Downarrow_{\mathcal N} x$ if there is $Q$ such that 
$P \wred Q$ and $Q \downarrow_{\mathcal N} x$.
\end{definition}

\begin{definition}
%\label{def.bbisim}
An  ${\mathcal N}$-\emph{barbed bisimulation} over a set of names, ${\mathcal N}$, is a symmetric binary relation 
${\mathcal S}_{\mathcal N}$ between agents such that $P\rel{S}_{\mathcal N}Q$ implies:
\begin{enumerate}
\item If $P \red P'$ then $Q \wred Q'$ and $P'\rel{S}_{\mathcal N} Q'$.
\item If $P\downarrow_{\mathcal N} x$, then $Q\Downarrow_{\mathcal N} x$.
\end{enumerate}
$P$ is ${\mathcal N}$-barbed bisimilar to $Q$, written
$P \wbbisim_{\mathcal N} Q$, if $P \rel{S}_{\mathcal N} Q$ for some ${\mathcal N}$-barbed bisimulation ${\mathcal S}_{\mathcal N}$.
\end{definition}

$\mathcal{R} \subseteq \pi \times \pi$

$P \mathcal{R} Q => \forall P'. P \red P' \Rightarrow \exists Q'. Q \red Q', P' \mathcal{R} Q'$

$P \vdash x \Rightarrow Q \vdash x$

\begin{mathpar}
  \inferrule*[lab=Out-barb]{x \nameeq y}{{y}!\langle{Q}\rangle \vdash x}
  \and
  \inferrule*[lab=Par-barb]{\mbox{$P\vdash x$ or $Q\vdash x$}}{\binpar{P}{Q} \vdash x}
\end{mathpar}

\subsubsection{Contexts}

One of the principle advantages of computational calculi like the
$\pi$-calculus is a well-defined notion of context,
contextual-equivalence and a correlation between
contextual-equivalence and notions of bisimulation. The notion of
context allows the decomposition of a process into (sub-)process and
its syntactic environment, its context. Thus, a context may be
thought of as a process with a ``hole'' (written $\Box$) in it. The
application of a context $M$ to a process $P$, written $M[P]$, is
tantamount to filling the hole in $M$ with $P$. In this paper we do
not need the full weight of this theory, but do make use of the notion
of context in the proof the main theorem. 

\begin{mathpar}
  \inferrule* [lab=summation] {} {{M_{M},M_{N}} \bc \Box \;|\; x.M_{A} \;|\; M_{M}+M_{N}}
  \and
  \inferrule* [lab=agent] {} {{M_{A}} \bc (\vec{x})M_{P} \;| \; \clift{P_0,\ldots,M_{P},\ldots,P_N}}
  \and \\
  \inferrule* [lab=process] {} {{M_{P}} \bc M_{N} \;| \;P|M_{P} }
\end{mathpar} 

\begin{mathpar}
  \inferrule* [lab=sychronization] {} {M_{N} \bc \Box \;|\; x?M_{F} \;|\; x!M_{C}}
  \and
  \inferrule* [lab=abstraction] {} {{M_{F}} \bc (x)M_{P} }
  \and
  \inferrule* [lab=concretion] {} {{M_{C}} \bc \langle M_{P} \rangle }
  \and \\
  \inferrule* [lab=process] {} {{M_{P}} \bc M_{N} \;| \;P|M_{P} }
\end{mathpar}

\begin{definition}[contextual application] Given a context $M$, and
  process $P$, we define the \emph{contextual application}, $M[P] :=
  M\{P/\Box\}$. That is, the contextual application of M to P is the
  substitution of $P$ for $\Box$ in $M$.
\end{definition}

$\meaningof{-} : L \to \mathcal{P}(\pi)$

\begin{mathpar}
  \inferrule* [lab=collection] {} {\meaningof{true} = \pi, \and \meaningof{~E} = \pi \setminus \meaningof{E}, \and \meaningof{E_{1} \& E_{2}} = \meaningof{E_{1}} \cap \meaningof{E_{2}}}
\end{mathpar}

\begin{mathpar}
  \inferrule* [lab=structure] {} {\meaningof{0} = \{ P \in \pi | P \equiv 0 \}, \and \\ \meaningof{E_1 | E_2} = \{ P \in \pi | P \equiv P_{1} | P_{2}, P_{1} \in \meaningof{E_{1}}, P_{2} \in \meaningof{E_2}\} }
\end{mathpar}

\begin{mathpar}
 \inferrule* [lab=behavior] {} {\meaningof{\langle a?b \rangle E} = \{ P \in \pi | P \equiv Q | u?(y)P', \\ \and \\\\ \and \\ \;\;\; u \in \meaningof{a}, \forall z.P'\{z/y\} \in \meaningof{E\{z/b\}}\}, \and \\ \meaningof{a!E} = \{ P \in \pi | P \equiv Q | x!\langle P' \rangle, x \in \meaningof{a} P' \in \meaningof{E}\} }
\end{mathpar}

\begin{mathpar}
 \inferrule* [lab=nominal] {} {\meaningof{\quotep{E}} = \{ \quotep{P} \in \quotep{\pi} | P \in \meaningof{E} \}, \and \meaningof{\quotep{P}} = \{ \quotep{Q} \in \quotep{\pi} | P \equiv Q \} \and \\ \meaningof{@\quotep{E}} = \{ P \in \pi | P \equiv @x, x \in \meaningof{E} \}}
\end{mathpar}

\begin{eqnarray*}
  \\
  \meaningof{-} : TS \to ST
\end{eqnarray*}

\begin{eqnarray*}
  \\
  L : TS \to ST
\end{eqnarray*}

\begin{eqnarray*}
  \\
  P \models E \iff P \in \meaningof{E}
\end{eqnarray*}

\begin{eqnarray*}
  P \approx_{L} Q \iff \forall E \in L. P \models E \iff Q \models E
\end{eqnarray*}

\begin{eqnarray*}
  P \approx_{K} Q
\end{eqnarray*}

\begin{eqnarray*}
  P \approx Q
\end{eqnarray*}

$\approx_{K} = \approx = \approx_{L}$

\subsubsection{Contextual duality}

Note that contexts extend the quotation operation to a family of
operations from processes to names. Given a context, $M$, we can
define a \emph{nominal context}, $\quotep{M}$ by $\quotep{M}[P] :=
\quotep{M[P]}$. To foreshadow what is to come we observe that these
operations enjoy a duality with processes very much like the duality
between vectors and maps from vectors to scalars.

Further, because the calculus is essentially higher-order, we have a
correspondence between contexts and processes. More specifically,
given a name $x$ and a context $M$ we can construct $M^{*}_{x}$ such
that 

\begin{mathpar}
  M^{*}_{x} | \lift{x}{P} \red M[P]
\end{mathpar}

namely,

\begin{mathpar}
  M^{*}_{x} := x?(u).M[\dropn{u}]
\end{mathpar}

The dependence of $M^{*}_{x}$ on a name makes it an abstraction, 

\begin{mathpar}
  M^{*} := (x)x?(u).M[\dropn{u}]
\end{mathpar}

\subsection{Additional notation}

It will sometimes be convenient to denote the process a name
quotes. We already have the notation $x = \quotep{P}$, but it will be
convenient to introduce an alternate notation, $\procn{x}$, when we
want to emphasize the connection to the use of the name. Note that, by
virtue of name equivalence, $\quotep{\procn{x}} \nameeq x$; so, the
notation is consistent with previous definitions.

Further, because names have structure it is possible to effect
substitutions on the basis of that structure. This means we need to
upgrade our notation for substitutions, which we accomplish by
adapting comprehension notation. Thus,

\begin{mathpar}
  P\{ y / x : x \in S \}
\end{mathpar}

is interpreted to mean the process derived from P by replacing (in a
capture-avoiding manner) each occurrence of $x$ in $S$ by $y$. For example,

\begin{mathpar}
  P\{ \quotep{\procn{x}|\procn{x}} / x : x \in \freenames{P} \}
\end{mathpar}

will replace each (occurrence) of a free name $x$ in $P$ by
$\quotep{\procn{x}|\procn{x}}$.

Also, we will avail ourselves of the notation $x^{L}$ and $x^{R}$ to
denote injections of a name into disjoint copies of the name
space. There are numerous ways to accomplish this. One example can be
found in \cite{MeredithR05}. This notation overloads to vectors of
names: $\vec{x}^{\pi} := (x_{i}^{\pi} \; : \; 0 \leq i < |\vec{x}| )$ where $\pi \in \{L,R\}$.

We also use $P^{\Box} := P|\Box$.

In \cite{MeredithR05} an interpretation of the new operator is
given. It turns out that there are several possible interpretations
all enjoying the requisite algebraic properties of the operator (see
\cite{milner91polyadicpi}). We will therefore make liberal use of
$(\nu\; \vec{x})P$.

% subsection the_syntax_and_semantics_of_the_notation_system (end)   

\input{qm2pi.qmops} 

\input{qm2pi.sterngerlach} 

\input{qm2pi.metric} 

% section concurrent_process_calculi (end)

%\input{qm2pi.proofsketch}

% section proof sketch (end)

%\input{qm2pi.slviaknots} 

% section spatial logic via knots (end)

\input{qm2pi.conclusion}

% section conclusion (end)

%\input{qm2pi.dtcodes} 

% section wiring algorithm (end)

\input{qm2pi.ack} 

% section acknowledgments (end)

\newpage


\bibliographystyle{plain}   
\bibliography{../../biblios/main.bib}

\input{qm2pi.rhodetails}

\end{document}

 

%\documentclass[12pt]{llncs}
%\documentclass{jktr}

\usepackage[pdftex]{hyperref}                   
\usepackage {listings}
\usepackage {mathpartir}
\usepackage{bcprules}
%\usepackage{listings}
                       
\usepackage{graphicx} 
%\usepackage[margins=2.5cm,nohead,nofoot]{geometry}
%\usepackage{geometry}
\usepackage{amsfonts}
\usepackage{amstext}
\usepackage{latexsym}
\usepackage{amssymb}
\usepackage{color}


%\include{myPreamble}
\include{qm2pi.local} 

%\ifpdf
%\usepackage[pdftex]{graphicx}
%\else
%\usepackage{graphicx}
%\fi

 % \ifpdf
%  \usepackage{pdfsync}
%  \if


%\title{Brief Article}
%\author{David F. Snyder}
%\author{L.G. Meredith}

%\address{Dept. of Math., Texas State University--San Marcos, San Marcos, TX 78666}
       
\pagestyle{empty}


\begin{document}

\lstset{language=[Objective]Caml,frame=shadowbox}

\input{qm2pi.front}

% section front matter (end)

\input{qm2pi.intro} 
 
% section introduction (end)

% \input{qm2pi.knotations} 

% section notation (end)

\input{qm2pi.process.calculi} 

% section concurrent_process_calculi_and_spatial_logics_ (end)
    
%\input{qm2pi.knots2pi} 

%\input{qm2pi.trefoil} 

%\input{qm2pi.mainthm} 

% subsection basic_interpretation (end)

%\input{qm2pi.rho.presentation} 
\subsection{The syntax and semantics of the notation system}\label{sub:the_syntax_and_semantics_of_the_notation_system} % (fold)

We now summarize a technical presentation of the calculus that
embodies our theory of dynamics. The typical presentation of such a
calculus follows the style of giving generators and relations on
them. The grammar, below, describing term constructors, freely
generates the set of processes, $\Proc$. This set is then quotiented
by a relation known as structural congruence and it is over this set
that the notion of dynamics is expressed. This presentation is
essentially that of \cite{MeredithR05} with the addition of
polyadicity and summation. For readability we have relegated some of
the technical subtleties to an appendix.

\subsubsection{Process grammar}\label{subsub:process_grammar}

\begin{mathpar}
  \inferrule* [lab=synchronization] {} {{M} \bc \pzero \;|\; x?F \;|\; x!C }
  \and
  \inferrule* [lab=abstraction] {} {{F} \bc (x)P}
  \and
  \inferrule* [lab=concretion] {} {{C} \bc \langle Q \rangle}
  \and
  \inferrule* [lab=process] {} {{P,Q} \bc M \;| \;P|Q \;|\; @{x}}
  \and
  \inferrule* [lab=name] {} {{x} \bc \quotep{P}}
\end{mathpar} 

Note that $\vec{x}$ (resp. $\vec{P}$) denotes a vector of names
(resp. processes) of length $|\vec{x}|$ (resp. $|\vec{P}|$). We adopt
the following useful abbreviations.

\begin{mathpar}
   x?(\vec{y}).P := x.(\vec{y})P \and  x\clift{\vec{P}} := x.\clift{\vec{P}}
   \and x!(y) := \lift{x}{\dropn{y}}
   \and \Pi_{i=0}^{n-1}P_i := P_0 | \ldots | P_{n-1}
\end{mathpar}

\subsubsection{Structural congruence}

\paragraph{Free and bound names and alpha-equivalence.} At the
core of structural equivalence is alpha-equivalence which identifies
process that are the same up to a change of variable. Formally, we
recognize the distinction between free and bound names. The free names
of a process, $\freenames{P}$, may be calculated recursively as
follows:

\begin{mathpar}
\freenames{\pzero} := \emptyset
  \and \\
  \freenames{x?(y).P} := \{ x \} \cup (\freenames{P} \setminus \{ y \})
  \and 
  \freenames{x!\langle P \rangle} := \{ x \} \cup \{ P \} 
  \and \\
  \freenames{P|Q} := \freenames{P} \cup \freenames{Q}
  \and \\
  \freenames{@{x}} := \{ x \}
\end{mathpar}

$\pi$
$\quotep{\pi}$

$\freenames{-} : \pi \to \mathcal{P}(\quotep{\pi})$

\begin{eqnarray*}
  \freenames{\pzero} & := & \emptyset \\
  \freenames{x?(y).P} & := & \{ x \} \cup (\freenames{P} \setminus \{ y \}) \\
  \freenames{x!\langle P \rangle} & := & \{ x \} \cup \{ P \} \\
  \freenames{P|Q} & := & \freenames{P} \cup \freenames{Q} \\
  \freenames{\dropn{x}} & := & \{ x \}
\end{eqnarray*}

The bound names of a process, $\boundnames{P}$, are those names occurring in $P$
that are not free. For example, in $x?(y).0$, the name $x$ is free, while $y$ is bound.

\begin{mathpar}
  \inferrule* [lab=monoidal-laws] {} { P|Q \equiv Q|P \and P|0 \equiv P \and P|(Q|R) \equiv (P|Q)|R }
\end{mathpar}

\begin{mathpar}
  \inferrule* [lab=alpha-equivalence] {} { (x)P \equiv (y)P\{y/x\} \and y \not\in \freenames{P} }
\end{mathpar}

\begin{definition}
Then two processes, $P,Q$, are alpha-equivalent if $P = Q\{\vec{y}/\vec{x}\}$ for
some $\vec{x} \in \boundnames{Q},\vec{y} \in \boundnames{P}$, where $Q\{\vec{y}/\vec{x}\}$
denotes the capture-avoiding substitution of $\vec{y}$ for $\vec{x}$ in $Q$.
\end{definition}

\begin{definition}
  The {\em structural congruence} \cite{SangiorgiWalker} , $\equiv$,
  between processes is the least congruence containing
  alpha-equivalence, satisfying the abelian monoid laws
  (associativity, commutativity and $\pzero$ as identity) for parallel
  composition $|$ and for summation $+$.
\end{definition}

\subsection{Name equivalence}

We take name equivalence, written $\nameeq$, to be the smallest
equivalence relation generated by the following rules.

\begin{mathpar}
\inferrule*[lab=Quote-drop]
{ }
{ \quotep{@{x}} \nameeq x }

\inferrule*[lab=Struct-equiv]
{ P \scong Q }
{ \quotep{P} \nameeq \quotep{Q} }
\end{mathpar}

The astute reader will have noticed that the mutual recursion of names
and processes imposes a mutual recursion on alpha-equivalence and
structural equivalence via name-equivalence. Fortunately, all of this
works out pleasantly and we may calculate in the natural way, free of
concern. The reader interested in the details is referred to the
appendix \ref{appendix:rho_details}.

\subsection{Substitution}

We use $\Proc$ for the set of processes, $\QProc$ for the set of
names, and $\id{\{}\vec{y} / \vec{x} \id{\}}$ to denote partial maps,
$s : \QProc \rightarrow \QProc$. A map, $s$ lifts, uniquely, to a map
on process terms, $\widehat{s} : \Proc \rightarrow \Proc$ by the
following equations.

\begin{mathpar}
  (0) \psubstp{Q}{P} := 0 \\
  (R \juxtap S) \psubstp{Q}{P}
  :=    
  (R)\psubstp{Q}{P} \juxtap (S) \psubstp{Q}{P} \\
  (x?(y).R) \psubstp{Q}{P}    
  :=    
  (x)\substp{Q}{P} (z)\concat( (R \psubstn{z}{y}) \psubstp{Q}{P} ) \\
  (\lift{x}{R}) \psubstp{Q}{P}  
  :=
  \lift{(x)\substp{Q}{P}}{ R \psubstp{Q}{P} } \\
%   (\dropn{x})  \psubstp{Q}{P}       
%   := 
%   \left\{ 
%     \begin{array}{ccc} 
%       \dropn{\quotep{Q}} & & x \nameeq \quotep{P} \\
%       \dropn{x} & & otherwise \\
%     \end{array}
%   \right. 
  (\dropn{x})  \psubstp{Q}{P}       
  := 
  \left\{ 
    \begin{array}{ccc} 
      Q & & x \nameeq \quotep{P} \\
      \dropn{x} & & otherwise \\
    \end{array}
  \right.
\end{mathpar}
 

where

\begin{eqnarray}
  (x)\id{\{} \lpquote Q \rpquote / \lpquote P \rpquote \id{\}}            = 
  \left\{ 
    \begin{array}{ccc}
      \lpquote Q \rpquote & & x \nameeq \lpquote P \rpquote \\
      x & & otherwise \\
    \end{array}
  \right. \nonumber
\end{eqnarray}

and $z$ is chosen distinct from $\quotep{P}$, $\quotep{Q}$, the free
names in $Q$, and all the names in $R$. Our $\alpha$-equivalence will
be built in the standard way from this substitution.

\begin{remark}\label{rem:no_self_referential_names}
  One consequence of these definitions is that $\forall P. \quotep{P}
  \not\in \freenames{P}$.
\end{remark}

\subsection{ Dynamic quote: an example }

Anticipating something of what's to come, consider applying the
substitution, $\widehat{\id{\{}u / z \id{\}}}$, to the following pair
of processes, $\lift{w}{y!(z)}$ and $w[ \lpquote y!(z) \rpquote ]$.

\begin{eqnarray}
	\lift{w}{y!(z)}\widehat{\id{\{}u / z \id{\}}}
		& = &
		\lift{w}{y!(u)} \nonumber\\
	w[ \lpquote y!(z) \rpquote ] \widehat{ \id{\{}u / z \id{\}} }
		& = &
		w[ \lpquote y!(z) \rpquote ] \nonumber
\end{eqnarray}

Because the body of the process between quotes is impervious to
substitution, we get radically different answers. In fact, by
examining the first process in an input context,
e.g. $x?(z).\lift{w}{y!(z)}$, we see that the process under the lift
operator may be shaped by prefixed inputs binding a name inside it. In
this sense, the lift operator will be seen as a way to dynamically
construct processes before reifying them as names.

Finally equipped with these standard features we can present the
dynamics of the calculus.

\subsubsection{Operational semantics} 

Finally, we introduce the computational dynamics. What marks these
algebras as distinct from other more traditionally studied algebraic
structures, e.g. vector spaces or polynomial rings, is the manner in
which dynamics is captured. In traditional structures, dynamics is typically
expressed through morphisms between such structures, as in linear maps
between vector spaces or morphisms between rings. In algebras
associated with the semantics of computation, the dynamics is
expressed as part of the algebraic structure itself, through a
reduction reduction relation typically denoted by $\red$. Below, we
give a recursive presentation of this relation for the calculus used
in the encoding.

$\red \subseteq \pi \times \pi$
$\red : \pi \to \mathcal{P}(\pi)$

\begin{mathpar}
  \inferrule* [lab=Comm] { \textsf{match}( x_{src}, x_{trgt} ) } { x_{trgt}?(y)P \; | \; x_{src}!\langle {Q} \rangle \red P\{\quotep{Q}/y}\} }
  \and \\
  \inferrule* [lab=Par] {{P} \red {P}'} {{{P} | {Q}} \red {{P}' | {Q}}}
  \and
  \inferrule* [lab=Equiv]{{{P} \scong {P}'} \andalso {{P}' \red {Q}'} \andalso {{Q}' \scong {Q}}}{{P} \red {Q}}
\end{mathpar}

\begin{eqnarray*}
  match_{\equiv} (\quotep{P},\quotep{Q}) & := & P \equiv Q \\
  match_{\dagger}(\quotep{P},\quotep{Q}) & := & \forall R. P|Q \red^{*} R => R \red^{*} 0 \\
  match_{K}(\quotep{P},\quotep{Q}) & := & K \mbox{ for some context } K
\end{eqnarray*}

$u?(x)P | u!\langle Q \rangle \red P\{\quotep{Q}/x\}$

%We write $\wred$ for $\red^*$, and $P\red$ if $\exists Q $ such that $ P \red Q$.
We write $P\red$ if $\exists Q $ such that $ P \red Q$ and $P\not\red$, otherwise.

\section{Replication}

As mentioned before, it is known that replication (and hence
recursion) can be implemented in a higher-order process algebra
\cite{SangiorgiWalker}. As our first example of calculation with the
machinery thus far presented we give the construction explicitly in
the {\rhoc}.

\begin{eqnarray}
	D_{x} & := & \prefix{x}{y}{(\binpar{\outputp{x}{y}}{@{y}})} \nonumber\\
	\bangp_{x}{P} & := & \binpar{{x}!\langle{\binpar{D_{x}}{P}}\rangle}{D_{x}} \nonumber
\end{eqnarray}

\begin{eqnarray}
	\bangp_{x}{P} & & \nonumber\\
	=
	& {x}!\langle{(\prefix{x}{y}{(\outputp{x}{y} | @{y})) | P}}\rangle 
	      | \prefix{x}{y}{(\outputp{x}{y} | @{y})} & \nonumber\\
	\red
	& (\outputp{x}{y} | @{y})\substn{\quotep{(\prefix{x}{y}{(@{y} | \outputp{x}{y})) | P}}}{y} & \nonumber\\
	=
	& \outputp{x}{\quotep{(\prefix{x}{y}{(\outputp{x}{y} | @{y})) | P}}}
	  | {(\prefix{x}{y}{(\outputp{x}{y} | @{y})) | P}} & \nonumber\\
	\red
	& \ldots & \nonumber\\
	\red^*
	& P | P | \ldots & \nonumber
\end{eqnarray}

Of course, this encoding, as an implementation, runs away, unfolding
$\bangp{P}$ eagerly. A lazier and more implementable replication
operator, restricted to input-guarded processes, may be obtained as follows.

\begin{eqnarray}
\bangp{\prefix{u}{v}{P}} 
	:= 
	\binpar{\lift{x}{\prefix{u}{v}{(\binpar{D(x)}{P})}}}{D(x)} \nonumber
\end{eqnarray}

\begin{remark}
  Note that the lazier definition still does not deal with summation
  or mixed summation (i.e. sums over input and output). The reader is
  invited to construct definitions of replication that deal with these
  features. 

  Further, the definitions are parameterized in a name, $x$. Can you,
  gentle reader, make a definition that eliminates this parameter and
  guarantees no accidental interaction between the replication
  machinery and the process being replicated -- i.e. no accidental
  sharing of names used by the process to get its work done and the
  name(s) used by the replication to effect copying. This latter
  revision of the definition of replication is crucial to obtaining
  the expected identity $!!P \sim !P$.
\end{remark}

\begin{remark}\label{rem:paradoxical_combinator}
  The reader familiar with the lambda calculus will have noticed the
  similarity between $D$ and the paradoxical combinator.

  [Ed. note: the existence of this seems to suggest we have to be more
  restrictive on the set of processes and names we admit if we are to
  support no-cloning.]
\end{remark}

\subsubsection{Bisimulation}

The computational dynamics gives rise to another kind of equivalence,
the equivalence of computational behavior. As previously mentioned
this is typically captured \emph{via} some form of bisimulation.

% The notion we use in this paper is weak barbed bisimulation
% \cite{milner91polyadicpi}.

The notion we use in this paper is derived from weak barbed
bisimulation \cite{milner91polyadicpi}. 

\begin{definition}
An \emph{observation relation}, $\downarrow_{\mathcal N}$, over a set
of names, $\mathcal N$, is the smallest relation satisfying the rules
below.

\infrule[Out-barb]{y \in {\mathcal N}, \; x \nameeq y}
		  {\outputp{x}{v} \downarrow_{\mathcal N} x}
\infrule[Par-barb]{\mbox{$P\downarrow_{\mathcal N} x$ or $Q\downarrow_{\mathcal N} x$}}
		  {\binpar{P}{Q} \downarrow_{\mathcal N} x}

We write $P \Downarrow_{\mathcal N} x$ if there is $Q$ such that 
$P \wred Q$ and $Q \downarrow_{\mathcal N} x$.
\end{definition}

\begin{definition}
%\label{def.bbisim}
An  ${\mathcal N}$-\emph{barbed bisimulation} over a set of names, ${\mathcal N}$, is a symmetric binary relation 
${\mathcal S}_{\mathcal N}$ between agents such that $P\rel{S}_{\mathcal N}Q$ implies:
\begin{enumerate}
\item If $P \red P'$ then $Q \wred Q'$ and $P'\rel{S}_{\mathcal N} Q'$.
\item If $P\downarrow_{\mathcal N} x$, then $Q\Downarrow_{\mathcal N} x$.
\end{enumerate}
$P$ is ${\mathcal N}$-barbed bisimilar to $Q$, written
$P \wbbisim_{\mathcal N} Q$, if $P \rel{S}_{\mathcal N} Q$ for some ${\mathcal N}$-barbed bisimulation ${\mathcal S}_{\mathcal N}$.
\end{definition}

$\mathcal{R} \subseteq \pi \times \pi$

$P \mathcal{R} Q => \forall P'. P \red P' \Rightarrow \exists Q'. Q \red Q', P' \mathcal{R} Q'$

$P \vdash x \Rightarrow Q \vdash x$

\begin{mathpar}
  \inferrule*[lab=Out-barb]{x \nameeq y}{{y}!\langle{Q}\rangle \vdash x}
  \and
  \inferrule*[lab=Par-barb]{\mbox{$P\vdash x$ or $Q\vdash x$}}{\binpar{P}{Q} \vdash x}
\end{mathpar}

\subsubsection{Contexts}

One of the principle advantages of computational calculi like the
$\pi$-calculus is a well-defined notion of context,
contextual-equivalence and a correlation between
contextual-equivalence and notions of bisimulation. The notion of
context allows the decomposition of a process into (sub-)process and
its syntactic environment, its context. Thus, a context may be
thought of as a process with a ``hole'' (written $\Box$) in it. The
application of a context $M$ to a process $P$, written $M[P]$, is
tantamount to filling the hole in $M$ with $P$. In this paper we do
not need the full weight of this theory, but do make use of the notion
of context in the proof the main theorem. 

\begin{mathpar}
  \inferrule* [lab=summation] {} {{M_{M},M_{N}} \bc \Box \;|\; x.M_{A} \;|\; M_{M}+M_{N}}
  \and
  \inferrule* [lab=agent] {} {{M_{A}} \bc (\vec{x})M_{P} \;| \; \clift{P_0,\ldots,M_{P},\ldots,P_N}}
  \and \\
  \inferrule* [lab=process] {} {{M_{P}} \bc M_{N} \;| \;P|M_{P} }
\end{mathpar} 

\begin{mathpar}
  \inferrule* [lab=sychronization] {} {M_{N} \bc \Box \;|\; x?M_{F} \;|\; x!M_{C}}
  \and
  \inferrule* [lab=abstraction] {} {{M_{F}} \bc (x)M_{P} }
  \and
  \inferrule* [lab=concretion] {} {{M_{C}} \bc \langle M_{P} \rangle }
  \and \\
  \inferrule* [lab=process] {} {{M_{P}} \bc M_{N} \;| \;P|M_{P} }
\end{mathpar}

\begin{definition}[contextual application] Given a context $M$, and
  process $P$, we define the \emph{contextual application}, $M[P] :=
  M\{P/\Box\}$. That is, the contextual application of M to P is the
  substitution of $P$ for $\Box$ in $M$.
\end{definition}

$\meaningof{-} : L \to \mathcal{P}(\pi)$

\begin{mathpar}
  \inferrule* [lab=collection] {} {\meaningof{true} = \pi, \and \meaningof{~E} = \pi \setminus \meaningof{E}, \and \meaningof{E_{1} \& E_{2}} = \meaningof{E_{1}} \cap \meaningof{E_{2}}}
\end{mathpar}

\begin{mathpar}
  \inferrule* [lab=structure] {} {\meaningof{0} = \{ P \in \pi | P \equiv 0 \}, \and \\ \meaningof{E_1 | E_2} = \{ P \in \pi | P \equiv P_{1} | P_{2}, P_{1} \in \meaningof{E_{1}}, P_{2} \in \meaningof{E_2}\} }
\end{mathpar}

\begin{mathpar}
 \inferrule* [lab=behavior] {} {\meaningof{\langle a?b \rangle E} = \{ P \in \pi | P \equiv Q | u?(y)P', \\ \and \\\\ \and \\ \;\;\; u \in \meaningof{a}, \forall z.P'\{z/y\} \in \meaningof{E\{z/b\}}\}, \and \\ \meaningof{a!E} = \{ P \in \pi | P \equiv Q | x!\langle P' \rangle, x \in \meaningof{a} P' \in \meaningof{E}\} }
\end{mathpar}

\begin{mathpar}
 \inferrule* [lab=nominal] {} {\meaningof{\quotep{E}} = \{ \quotep{P} \in \quotep{\pi} | P \in \meaningof{E} \}, \and \meaningof{\quotep{P}} = \{ \quotep{Q} \in \quotep{\pi} | P \equiv Q \} \and \\ \meaningof{@\quotep{E}} = \{ P \in \pi | P \equiv @x, x \in \meaningof{E} \}}
\end{mathpar}

\begin{eqnarray*}
  \\
  \meaningof{-} : TS \to ST
\end{eqnarray*}

\begin{eqnarray*}
  \\
  L : TS \to ST
\end{eqnarray*}

\begin{eqnarray*}
  \\
  P \models E \iff P \in \meaningof{E}
\end{eqnarray*}

\begin{eqnarray*}
  P \approx_{L} Q \iff \forall E \in L. P \models E \iff Q \models E
\end{eqnarray*}

\begin{eqnarray*}
  P \approx_{K} Q
\end{eqnarray*}

\begin{eqnarray*}
  P \approx Q
\end{eqnarray*}

$\approx_{K} = \approx = \approx_{L}$

\subsubsection{Contextual duality}

Note that contexts extend the quotation operation to a family of
operations from processes to names. Given a context, $M$, we can
define a \emph{nominal context}, $\quotep{M}$ by $\quotep{M}[P] :=
\quotep{M[P]}$. To foreshadow what is to come we observe that these
operations enjoy a duality with processes very much like the duality
between vectors and maps from vectors to scalars.

Further, because the calculus is essentially higher-order, we have a
correspondence between contexts and processes. More specifically,
given a name $x$ and a context $M$ we can construct $M^{*}_{x}$ such
that 

\begin{mathpar}
  M^{*}_{x} | \lift{x}{P} \red M[P]
\end{mathpar}

namely,

\begin{mathpar}
  M^{*}_{x} := x?(u).M[\dropn{u}]
\end{mathpar}

The dependence of $M^{*}_{x}$ on a name makes it an abstraction, 

\begin{mathpar}
  M^{*} := (x)x?(u).M[\dropn{u}]
\end{mathpar}

\subsection{Additional notation}

It will sometimes be convenient to denote the process a name
quotes. We already have the notation $x = \quotep{P}$, but it will be
convenient to introduce an alternate notation, $\procn{x}$, when we
want to emphasize the connection to the use of the name. Note that, by
virtue of name equivalence, $\quotep{\procn{x}} \nameeq x$; so, the
notation is consistent with previous definitions.

Further, because names have structure it is possible to effect
substitutions on the basis of that structure. This means we need to
upgrade our notation for substitutions, which we accomplish by
adapting comprehension notation. Thus,

\begin{mathpar}
  P\{ y / x : x \in S \}
\end{mathpar}

is interpreted to mean the process derived from P by replacing (in a
capture-avoiding manner) each occurrence of $x$ in $S$ by $y$. For example,

\begin{mathpar}
  P\{ \quotep{\procn{x}|\procn{x}} / x : x \in \freenames{P} \}
\end{mathpar}

will replace each (occurrence) of a free name $x$ in $P$ by
$\quotep{\procn{x}|\procn{x}}$.

Also, we will avail ourselves of the notation $x^{L}$ and $x^{R}$ to
denote injections of a name into disjoint copies of the name
space. There are numerous ways to accomplish this. One example can be
found in \cite{MeredithR05}. This notation overloads to vectors of
names: $\vec{x}^{\pi} := (x_{i}^{\pi} \; : \; 0 \leq i < |\vec{x}| )$ where $\pi \in \{L,R\}$.

We also use $P^{\Box} := P|\Box$.

In \cite{MeredithR05} an interpretation of the new operator is
given. It turns out that there are several possible interpretations
all enjoying the requisite algebraic properties of the operator (see
\cite{milner91polyadicpi}). We will therefore make liberal use of
$(\nu\; \vec{x})P$.

% subsection the_syntax_and_semantics_of_the_notation_system (end)   

\input{qm2pi.qmops} 

\input{qm2pi.sterngerlach} 

\input{qm2pi.metric} 

% section concurrent_process_calculi (end)

%\input{qm2pi.proofsketch}

% section proof sketch (end)

%\input{qm2pi.slviaknots} 

% section spatial logic via knots (end)

\input{qm2pi.conclusion}

% section conclusion (end)

%\input{qm2pi.dtcodes} 

% section wiring algorithm (end)

\input{qm2pi.ack} 

% section acknowledgments (end)

\newpage


\bibliographystyle{plain}   
\bibliography{../../biblios/main.bib}

\input{qm2pi.rhodetails}

\end{document}

 

% subsection basic_interpretation (end)

%\input{qm2pi.rho.presentation} 
\subsection{The syntax and semantics of the notation system}\label{sub:the_syntax_and_semantics_of_the_notation_system} % (fold)

We now summarize a technical presentation of the calculus that
embodies our theory of dynamics. The typical presentation of such a
calculus follows the style of giving generators and relations on
them. The grammar, below, describing term constructors, freely
generates the set of processes, $\Proc$. This set is then quotiented
by a relation known as structural congruence and it is over this set
that the notion of dynamics is expressed. This presentation is
essentially that of \cite{MeredithR05} with the addition of
polyadicity and summation. For readability we have relegated some of
the technical subtleties to an appendix.

\subsubsection{Process grammar}\label{subsub:process_grammar}

\begin{mathpar}
  \inferrule* [lab=synchronization] {} {{M} \bc \pzero \;|\; x?F \;|\; x!C }
  \and
  \inferrule* [lab=abstraction] {} {{F} \bc (x)P}
  \and
  \inferrule* [lab=concretion] {} {{C} \bc \langle Q \rangle}
  \and
  \inferrule* [lab=process] {} {{P,Q} \bc M \;| \;P|Q \;|\; @{x}}
  \and
  \inferrule* [lab=name] {} {{x} \bc \quotep{P}}
\end{mathpar} 

Note that $\vec{x}$ (resp. $\vec{P}$) denotes a vector of names
(resp. processes) of length $|\vec{x}|$ (resp. $|\vec{P}|$). We adopt
the following useful abbreviations.

\begin{mathpar}
   x?(\vec{y}).P := x.(\vec{y})P \and  x\clift{\vec{P}} := x.\clift{\vec{P}}
   \and x!(y) := \lift{x}{\dropn{y}}
   \and \Pi_{i=0}^{n-1}P_i := P_0 | \ldots | P_{n-1}
\end{mathpar}

\subsubsection{Structural congruence}

\paragraph{Free and bound names and alpha-equivalence.} At the
core of structural equivalence is alpha-equivalence which identifies
process that are the same up to a change of variable. Formally, we
recognize the distinction between free and bound names. The free names
of a process, $\freenames{P}$, may be calculated recursively as
follows:

\begin{mathpar}
\freenames{\pzero} := \emptyset
  \and \\
  \freenames{x?(y).P} := \{ x \} \cup (\freenames{P} \setminus \{ y \})
  \and 
  \freenames{x!\langle P \rangle} := \{ x \} \cup \{ P \} 
  \and \\
  \freenames{P|Q} := \freenames{P} \cup \freenames{Q}
  \and \\
  \freenames{@{x}} := \{ x \}
\end{mathpar}

$\pi$
$\quotep{\pi}$

$\freenames{-} : \pi \to \mathcal{P}(\quotep{\pi})$

\begin{eqnarray*}
  \freenames{\pzero} & := & \emptyset \\
  \freenames{x?(y).P} & := & \{ x \} \cup (\freenames{P} \setminus \{ y \}) \\
  \freenames{x!\langle P \rangle} & := & \{ x \} \cup \{ P \} \\
  \freenames{P|Q} & := & \freenames{P} \cup \freenames{Q} \\
  \freenames{\dropn{x}} & := & \{ x \}
\end{eqnarray*}

The bound names of a process, $\boundnames{P}$, are those names occurring in $P$
that are not free. For example, in $x?(y).0$, the name $x$ is free, while $y$ is bound.

\begin{mathpar}
  \inferrule* [lab=monoidal-laws] {} { P|Q \equiv Q|P \and P|0 \equiv P \and P|(Q|R) \equiv (P|Q)|R }
\end{mathpar}

\begin{mathpar}
  \inferrule* [lab=alpha-equivalence] {} { (x)P \equiv (y)P\{y/x\} \and y \not\in \freenames{P} }
\end{mathpar}

\begin{definition}
Then two processes, $P,Q$, are alpha-equivalent if $P = Q\{\vec{y}/\vec{x}\}$ for
some $\vec{x} \in \boundnames{Q},\vec{y} \in \boundnames{P}$, where $Q\{\vec{y}/\vec{x}\}$
denotes the capture-avoiding substitution of $\vec{y}$ for $\vec{x}$ in $Q$.
\end{definition}

\begin{definition}
  The {\em structural congruence} \cite{SangiorgiWalker} , $\equiv$,
  between processes is the least congruence containing
  alpha-equivalence, satisfying the abelian monoid laws
  (associativity, commutativity and $\pzero$ as identity) for parallel
  composition $|$ and for summation $+$.
\end{definition}

\subsection{Name equivalence}

We take name equivalence, written $\nameeq$, to be the smallest
equivalence relation generated by the following rules.

\begin{mathpar}
\inferrule*[lab=Quote-drop]
{ }
{ \quotep{@{x}} \nameeq x }

\inferrule*[lab=Struct-equiv]
{ P \scong Q }
{ \quotep{P} \nameeq \quotep{Q} }
\end{mathpar}

The astute reader will have noticed that the mutual recursion of names
and processes imposes a mutual recursion on alpha-equivalence and
structural equivalence via name-equivalence. Fortunately, all of this
works out pleasantly and we may calculate in the natural way, free of
concern. The reader interested in the details is referred to the
appendix \ref{appendix:rho_details}.

\subsection{Substitution}

We use $\Proc$ for the set of processes, $\QProc$ for the set of
names, and $\id{\{}\vec{y} / \vec{x} \id{\}}$ to denote partial maps,
$s : \QProc \rightarrow \QProc$. A map, $s$ lifts, uniquely, to a map
on process terms, $\widehat{s} : \Proc \rightarrow \Proc$ by the
following equations.

\begin{mathpar}
  (0) \psubstp{Q}{P} := 0 \\
  (R \juxtap S) \psubstp{Q}{P}
  :=    
  (R)\psubstp{Q}{P} \juxtap (S) \psubstp{Q}{P} \\
  (x?(y).R) \psubstp{Q}{P}    
  :=    
  (x)\substp{Q}{P} (z)\concat( (R \psubstn{z}{y}) \psubstp{Q}{P} ) \\
  (\lift{x}{R}) \psubstp{Q}{P}  
  :=
  \lift{(x)\substp{Q}{P}}{ R \psubstp{Q}{P} } \\
%   (\dropn{x})  \psubstp{Q}{P}       
%   := 
%   \left\{ 
%     \begin{array}{ccc} 
%       \dropn{\quotep{Q}} & & x \nameeq \quotep{P} \\
%       \dropn{x} & & otherwise \\
%     \end{array}
%   \right. 
  (\dropn{x})  \psubstp{Q}{P}       
  := 
  \left\{ 
    \begin{array}{ccc} 
      Q & & x \nameeq \quotep{P} \\
      \dropn{x} & & otherwise \\
    \end{array}
  \right.
\end{mathpar}
 

where

\begin{eqnarray}
  (x)\id{\{} \lpquote Q \rpquote / \lpquote P \rpquote \id{\}}            = 
  \left\{ 
    \begin{array}{ccc}
      \lpquote Q \rpquote & & x \nameeq \lpquote P \rpquote \\
      x & & otherwise \\
    \end{array}
  \right. \nonumber
\end{eqnarray}

and $z$ is chosen distinct from $\quotep{P}$, $\quotep{Q}$, the free
names in $Q$, and all the names in $R$. Our $\alpha$-equivalence will
be built in the standard way from this substitution.

\begin{remark}\label{rem:no_self_referential_names}
  One consequence of these definitions is that $\forall P. \quotep{P}
  \not\in \freenames{P}$.
\end{remark}

\subsection{ Dynamic quote: an example }

Anticipating something of what's to come, consider applying the
substitution, $\widehat{\id{\{}u / z \id{\}}}$, to the following pair
of processes, $\lift{w}{y!(z)}$ and $w[ \lpquote y!(z) \rpquote ]$.

\begin{eqnarray}
	\lift{w}{y!(z)}\widehat{\id{\{}u / z \id{\}}}
		& = &
		\lift{w}{y!(u)} \nonumber\\
	w[ \lpquote y!(z) \rpquote ] \widehat{ \id{\{}u / z \id{\}} }
		& = &
		w[ \lpquote y!(z) \rpquote ] \nonumber
\end{eqnarray}

Because the body of the process between quotes is impervious to
substitution, we get radically different answers. In fact, by
examining the first process in an input context,
e.g. $x?(z).\lift{w}{y!(z)}$, we see that the process under the lift
operator may be shaped by prefixed inputs binding a name inside it. In
this sense, the lift operator will be seen as a way to dynamically
construct processes before reifying them as names.

Finally equipped with these standard features we can present the
dynamics of the calculus.

\subsubsection{Operational semantics} 

Finally, we introduce the computational dynamics. What marks these
algebras as distinct from other more traditionally studied algebraic
structures, e.g. vector spaces or polynomial rings, is the manner in
which dynamics is captured. In traditional structures, dynamics is typically
expressed through morphisms between such structures, as in linear maps
between vector spaces or morphisms between rings. In algebras
associated with the semantics of computation, the dynamics is
expressed as part of the algebraic structure itself, through a
reduction reduction relation typically denoted by $\red$. Below, we
give a recursive presentation of this relation for the calculus used
in the encoding.

$\red \subseteq \pi \times \pi$
$\red : \pi \to \mathcal{P}(\pi)$

\begin{mathpar}
  \inferrule* [lab=Comm] { \textsf{match}( x_{src}, x_{trgt} ) } { x_{trgt}?(y)P \; | \; x_{src}!\langle {Q} \rangle \red P\{\quotep{Q}/y}\} }
  \and \\
  \inferrule* [lab=Par] {{P} \red {P}'} {{{P} | {Q}} \red {{P}' | {Q}}}
  \and
  \inferrule* [lab=Equiv]{{{P} \scong {P}'} \andalso {{P}' \red {Q}'} \andalso {{Q}' \scong {Q}}}{{P} \red {Q}}
\end{mathpar}

\begin{eqnarray*}
  match_{\equiv} (\quotep{P},\quotep{Q}) & := & P \equiv Q \\
  match_{\dagger}(\quotep{P},\quotep{Q}) & := & \forall R. P|Q \red^{*} R => R \red^{*} 0 \\
  match_{K}(\quotep{P},\quotep{Q}) & := & K \mbox{ for some context } K
\end{eqnarray*}

$u?(x)P | u!\langle Q \rangle \red P\{\quotep{Q}/x\}$

%We write $\wred$ for $\red^*$, and $P\red$ if $\exists Q $ such that $ P \red Q$.
We write $P\red$ if $\exists Q $ such that $ P \red Q$ and $P\not\red$, otherwise.

\section{Replication}

As mentioned before, it is known that replication (and hence
recursion) can be implemented in a higher-order process algebra
\cite{SangiorgiWalker}. As our first example of calculation with the
machinery thus far presented we give the construction explicitly in
the {\rhoc}.

\begin{eqnarray}
	D_{x} & := & \prefix{x}{y}{(\binpar{\outputp{x}{y}}{@{y}})} \nonumber\\
	\bangp_{x}{P} & := & \binpar{{x}!\langle{\binpar{D_{x}}{P}}\rangle}{D_{x}} \nonumber
\end{eqnarray}

\begin{eqnarray}
	\bangp_{x}{P} & & \nonumber\\
	=
	& {x}!\langle{(\prefix{x}{y}{(\outputp{x}{y} | @{y})) | P}}\rangle 
	      | \prefix{x}{y}{(\outputp{x}{y} | @{y})} & \nonumber\\
	\red
	& (\outputp{x}{y} | @{y})\substn{\quotep{(\prefix{x}{y}{(@{y} | \outputp{x}{y})) | P}}}{y} & \nonumber\\
	=
	& \outputp{x}{\quotep{(\prefix{x}{y}{(\outputp{x}{y} | @{y})) | P}}}
	  | {(\prefix{x}{y}{(\outputp{x}{y} | @{y})) | P}} & \nonumber\\
	\red
	& \ldots & \nonumber\\
	\red^*
	& P | P | \ldots & \nonumber
\end{eqnarray}

Of course, this encoding, as an implementation, runs away, unfolding
$\bangp{P}$ eagerly. A lazier and more implementable replication
operator, restricted to input-guarded processes, may be obtained as follows.

\begin{eqnarray}
\bangp{\prefix{u}{v}{P}} 
	:= 
	\binpar{\lift{x}{\prefix{u}{v}{(\binpar{D(x)}{P})}}}{D(x)} \nonumber
\end{eqnarray}

\begin{remark}
  Note that the lazier definition still does not deal with summation
  or mixed summation (i.e. sums over input and output). The reader is
  invited to construct definitions of replication that deal with these
  features. 

  Further, the definitions are parameterized in a name, $x$. Can you,
  gentle reader, make a definition that eliminates this parameter and
  guarantees no accidental interaction between the replication
  machinery and the process being replicated -- i.e. no accidental
  sharing of names used by the process to get its work done and the
  name(s) used by the replication to effect copying. This latter
  revision of the definition of replication is crucial to obtaining
  the expected identity $!!P \sim !P$.
\end{remark}

\begin{remark}\label{rem:paradoxical_combinator}
  The reader familiar with the lambda calculus will have noticed the
  similarity between $D$ and the paradoxical combinator.

  [Ed. note: the existence of this seems to suggest we have to be more
  restrictive on the set of processes and names we admit if we are to
  support no-cloning.]
\end{remark}

\subsubsection{Bisimulation}

The computational dynamics gives rise to another kind of equivalence,
the equivalence of computational behavior. As previously mentioned
this is typically captured \emph{via} some form of bisimulation.

% The notion we use in this paper is weak barbed bisimulation
% \cite{milner91polyadicpi}.

The notion we use in this paper is derived from weak barbed
bisimulation \cite{milner91polyadicpi}. 

\begin{definition}
An \emph{observation relation}, $\downarrow_{\mathcal N}$, over a set
of names, $\mathcal N$, is the smallest relation satisfying the rules
below.

\infrule[Out-barb]{y \in {\mathcal N}, \; x \nameeq y}
		  {\outputp{x}{v} \downarrow_{\mathcal N} x}
\infrule[Par-barb]{\mbox{$P\downarrow_{\mathcal N} x$ or $Q\downarrow_{\mathcal N} x$}}
		  {\binpar{P}{Q} \downarrow_{\mathcal N} x}

We write $P \Downarrow_{\mathcal N} x$ if there is $Q$ such that 
$P \wred Q$ and $Q \downarrow_{\mathcal N} x$.
\end{definition}

\begin{definition}
%\label{def.bbisim}
An  ${\mathcal N}$-\emph{barbed bisimulation} over a set of names, ${\mathcal N}$, is a symmetric binary relation 
${\mathcal S}_{\mathcal N}$ between agents such that $P\rel{S}_{\mathcal N}Q$ implies:
\begin{enumerate}
\item If $P \red P'$ then $Q \wred Q'$ and $P'\rel{S}_{\mathcal N} Q'$.
\item If $P\downarrow_{\mathcal N} x$, then $Q\Downarrow_{\mathcal N} x$.
\end{enumerate}
$P$ is ${\mathcal N}$-barbed bisimilar to $Q$, written
$P \wbbisim_{\mathcal N} Q$, if $P \rel{S}_{\mathcal N} Q$ for some ${\mathcal N}$-barbed bisimulation ${\mathcal S}_{\mathcal N}$.
\end{definition}

$\mathcal{R} \subseteq \pi \times \pi$

$P \mathcal{R} Q => \forall P'. P \red P' \Rightarrow \exists Q'. Q \red Q', P' \mathcal{R} Q'$

$P \vdash x \Rightarrow Q \vdash x$

\begin{mathpar}
  \inferrule*[lab=Out-barb]{x \nameeq y}{{y}!\langle{Q}\rangle \vdash x}
  \and
  \inferrule*[lab=Par-barb]{\mbox{$P\vdash x$ or $Q\vdash x$}}{\binpar{P}{Q} \vdash x}
\end{mathpar}

\subsubsection{Contexts}

One of the principle advantages of computational calculi like the
$\pi$-calculus is a well-defined notion of context,
contextual-equivalence and a correlation between
contextual-equivalence and notions of bisimulation. The notion of
context allows the decomposition of a process into (sub-)process and
its syntactic environment, its context. Thus, a context may be
thought of as a process with a ``hole'' (written $\Box$) in it. The
application of a context $M$ to a process $P$, written $M[P]$, is
tantamount to filling the hole in $M$ with $P$. In this paper we do
not need the full weight of this theory, but do make use of the notion
of context in the proof the main theorem. 

\begin{mathpar}
  \inferrule* [lab=summation] {} {{M_{M},M_{N}} \bc \Box \;|\; x.M_{A} \;|\; M_{M}+M_{N}}
  \and
  \inferrule* [lab=agent] {} {{M_{A}} \bc (\vec{x})M_{P} \;| \; \clift{P_0,\ldots,M_{P},\ldots,P_N}}
  \and \\
  \inferrule* [lab=process] {} {{M_{P}} \bc M_{N} \;| \;P|M_{P} }
\end{mathpar} 

\begin{mathpar}
  \inferrule* [lab=sychronization] {} {M_{N} \bc \Box \;|\; x?M_{F} \;|\; x!M_{C}}
  \and
  \inferrule* [lab=abstraction] {} {{M_{F}} \bc (x)M_{P} }
  \and
  \inferrule* [lab=concretion] {} {{M_{C}} \bc \langle M_{P} \rangle }
  \and \\
  \inferrule* [lab=process] {} {{M_{P}} \bc M_{N} \;| \;P|M_{P} }
\end{mathpar}

\begin{definition}[contextual application] Given a context $M$, and
  process $P$, we define the \emph{contextual application}, $M[P] :=
  M\{P/\Box\}$. That is, the contextual application of M to P is the
  substitution of $P$ for $\Box$ in $M$.
\end{definition}

$\meaningof{-} : L \to \mathcal{P}(\pi)$

\begin{mathpar}
  \inferrule* [lab=collection] {} {\meaningof{true} = \pi, \and \meaningof{~E} = \pi \setminus \meaningof{E}, \and \meaningof{E_{1} \& E_{2}} = \meaningof{E_{1}} \cap \meaningof{E_{2}}}
\end{mathpar}

\begin{mathpar}
  \inferrule* [lab=structure] {} {\meaningof{0} = \{ P \in \pi | P \equiv 0 \}, \and \\ \meaningof{E_1 | E_2} = \{ P \in \pi | P \equiv P_{1} | P_{2}, P_{1} \in \meaningof{E_{1}}, P_{2} \in \meaningof{E_2}\} }
\end{mathpar}

\begin{mathpar}
 \inferrule* [lab=behavior] {} {\meaningof{\langle a?b \rangle E} = \{ P \in \pi | P \equiv Q | u?(y)P', \\ \and \\\\ \and \\ \;\;\; u \in \meaningof{a}, \forall z.P'\{z/y\} \in \meaningof{E\{z/b\}}\}, \and \\ \meaningof{a!E} = \{ P \in \pi | P \equiv Q | x!\langle P' \rangle, x \in \meaningof{a} P' \in \meaningof{E}\} }
\end{mathpar}

\begin{mathpar}
 \inferrule* [lab=nominal] {} {\meaningof{\quotep{E}} = \{ \quotep{P} \in \quotep{\pi} | P \in \meaningof{E} \}, \and \meaningof{\quotep{P}} = \{ \quotep{Q} \in \quotep{\pi} | P \equiv Q \} \and \\ \meaningof{@\quotep{E}} = \{ P \in \pi | P \equiv @x, x \in \meaningof{E} \}}
\end{mathpar}

\begin{eqnarray*}
  \\
  \meaningof{-} : TS \to ST
\end{eqnarray*}

\begin{eqnarray*}
  \\
  L : TS \to ST
\end{eqnarray*}

\begin{eqnarray*}
  \\
  P \models E \iff P \in \meaningof{E}
\end{eqnarray*}

\begin{eqnarray*}
  P \approx_{L} Q \iff \forall E \in L. P \models E \iff Q \models E
\end{eqnarray*}

\begin{eqnarray*}
  P \approx_{K} Q
\end{eqnarray*}

\begin{eqnarray*}
  P \approx Q
\end{eqnarray*}

$\approx_{K} = \approx = \approx_{L}$

\subsubsection{Contextual duality}

Note that contexts extend the quotation operation to a family of
operations from processes to names. Given a context, $M$, we can
define a \emph{nominal context}, $\quotep{M}$ by $\quotep{M}[P] :=
\quotep{M[P]}$. To foreshadow what is to come we observe that these
operations enjoy a duality with processes very much like the duality
between vectors and maps from vectors to scalars.

Further, because the calculus is essentially higher-order, we have a
correspondence between contexts and processes. More specifically,
given a name $x$ and a context $M$ we can construct $M^{*}_{x}$ such
that 

\begin{mathpar}
  M^{*}_{x} | \lift{x}{P} \red M[P]
\end{mathpar}

namely,

\begin{mathpar}
  M^{*}_{x} := x?(u).M[\dropn{u}]
\end{mathpar}

The dependence of $M^{*}_{x}$ on a name makes it an abstraction, 

\begin{mathpar}
  M^{*} := (x)x?(u).M[\dropn{u}]
\end{mathpar}

\subsection{Additional notation}

It will sometimes be convenient to denote the process a name
quotes. We already have the notation $x = \quotep{P}$, but it will be
convenient to introduce an alternate notation, $\procn{x}$, when we
want to emphasize the connection to the use of the name. Note that, by
virtue of name equivalence, $\quotep{\procn{x}} \nameeq x$; so, the
notation is consistent with previous definitions.

Further, because names have structure it is possible to effect
substitutions on the basis of that structure. This means we need to
upgrade our notation for substitutions, which we accomplish by
adapting comprehension notation. Thus,

\begin{mathpar}
  P\{ y / x : x \in S \}
\end{mathpar}

is interpreted to mean the process derived from P by replacing (in a
capture-avoiding manner) each occurrence of $x$ in $S$ by $y$. For example,

\begin{mathpar}
  P\{ \quotep{\procn{x}|\procn{x}} / x : x \in \freenames{P} \}
\end{mathpar}

will replace each (occurrence) of a free name $x$ in $P$ by
$\quotep{\procn{x}|\procn{x}}$.

Also, we will avail ourselves of the notation $x^{L}$ and $x^{R}$ to
denote injections of a name into disjoint copies of the name
space. There are numerous ways to accomplish this. One example can be
found in \cite{MeredithR05}. This notation overloads to vectors of
names: $\vec{x}^{\pi} := (x_{i}^{\pi} \; : \; 0 \leq i < |\vec{x}| )$ where $\pi \in \{L,R\}$.

We also use $P^{\Box} := P|\Box$.

In \cite{MeredithR05} an interpretation of the new operator is
given. It turns out that there are several possible interpretations
all enjoying the requisite algebraic properties of the operator (see
\cite{milner91polyadicpi}). We will therefore make liberal use of
$(\nu\; \vec{x})P$.

% subsection the_syntax_and_semantics_of_the_notation_system (end)   

\section{Interpretation of QM}
\subsection{Supporting definitions}
\subsubsection{Multiplication}
\begin{mathpar}
  \quotep{Q} \cdot \quotep{R} := \quotep{Q|R}
  \and \\
  \quotep{Q} \cdot P := P\{ \quotep{Q|R} / \quotep{R} : \quotep{R} \in \freenames{P} \}
\end{mathpar}

\paragraph{Discussion}
The first line needs little explanation. The second line says that
each free name of the process is replaced with the multiplication of
that name by the scalar. Multiplication of a scalar (name) by a state
(process) results in a process all the names of which have been `moved
over' by parallel composition with the process the scalar
quotes. There is a subtlety that the bound names have to be
manipulated so that multiplied names aren't accidentally
captured. There are many ways to achieve this.

\begin{remark}\label{rem:multiplication_identities}
  The reader is invited to verify that for all $x,y,z \in \QProc$ and $P \in \Proc$
  \begin{mathpar}
    x \cdot \quotep{0} \equiv x 
    \and
    x \cdot y \equiv y \cdot x
    \and
    x \cdot (y \cdot z) \equiv (x \cdot y) \cdot z
    \and \\
    \quotep{0} \cdot P \equiv P
    \and \\
    x \cdot (y \cdot P) \equiv (x \cdot y) \cdot P
    \and \\
    x \cdot (P|Q) \equiv (x \cdot P) | (x \cdot Q)
    \and \\    
  \end{mathpar}
\end{remark}

\subsubsection{Tensor product}

We define a tensor product on processes by structural induction.

\paragraph{Tensor of sums} First note that all summations, including
$\pzero$ and sequence, can be written $\Sigma_{i} x_{i}.A_{i} +
\Sigma_{j} x_{j}.C_{j}$, where we have grouped input-guarded processes
together and output-guarded processes together.

Thus, we can define the tensor product of two summations, $N_{1}\otimes N_{2}$, where

\begin{mathpar}
  N_{1} := \Sigma_{i} x_{i}.A_{i} + \Sigma_{j} x_{j}.C_{j}
  \and
  N_{2} := \Sigma_{i'} y_{i'}.B_{i'} + \Sigma_{j'} y_{j'}.D_{j'} 
\end{mathpar}

as follows.

\begin{mathpar}
  \Sigma_{i} x_{i}.A_{i} + \Sigma_{j} x_{j}.C_{j} \otimes \Sigma_{i'}
  y_{i'}.B_{i'} + \Sigma_{j'} y_{j'}.D_{j'} 
  \and \\
  := \; \Sigma_{i} \Sigma_{i'} \quotep{\stackrel{\vee}{x_{i}}| \stackrel{\vee}{y_{i'}}}.(A_{i}\otimes B_{i'}) \; | \; \Sigma_{i'} \Sigma_{i} \quotep{\stackrel{\vee}{y_{i'}}|\stackrel{\vee}{x_{i}}}.(B_{i'}\otimes A_{i})
  \and
  \;\; | \;\; \Sigma_{j} \Sigma_{j'} \quotep{\stackrel{\vee}{x_{j}}|\stackrel{\vee}{y_{j'}}}.(A_{j}\otimes B_{j'}) \; | \; \Sigma_{j'} \Sigma_{j} \quotep{\stackrel{\vee}{y_{j'}}|\stackrel{\vee}{x_{j}}}.(B_{j'}\otimes A_{j})
\end{mathpar}

\begin{remark}
  Do we need to $x^{L}$ and $y^{R}$ for this construction as well?
\end{remark}

\paragraph{Tensor of parallel compositions} Next, we distribute tensor
over par.

\begin{mathpar}
  P_{1}|P_{2} \otimes Q_{1}|Q_{2} := (P_{1} \otimes Q_{1}) | (P_{1}
  \otimes Q_{2}) | (P_{2} \otimes Q_{1}) | (P_{2} \otimes Q_{2})
\end{mathpar}

\paragraph{Tensor with dropped names} We treat tensor of a
process with a dropped name as parallel composition.

\begin{mathpar}
  P \otimes \dropn{x} := P | \dropn{x}
\end{mathpar}

\paragraph{Tensor of agents}

Finally, we need to define tensor on agents. Note that the definition
of tensor on normal products only tensors inputs with inputs and
outputs with outputs. Thus, we only have to define the operation on
``homogeneous'' pairings.

\begin{mathpar}
  (\vec{x})P \otimes (\vec{y})Q
  \and \\
  := (x_{0}^{L}|y_{0}^{R},\ldots,x_{0}^{L}|y_{n}^{R},\ldots,x_{m}^{L}|y_{0}^{R},\ldots,x_{m}^{L}|y_{n}^R)(P\{ \vec{x}^{L}/\vec{x}\} \otimes Q \{ \vec{y}^{R}/\vec{y}\})
  \and \\
  \clift{\vec{P}} \otimes \clift{\vec{Q}}
  \and \\
  := \clift{P_{0}\otimes Q_{0},\ldots,P_{0}\otimes Q_{n},\ldots,P_{m}\otimes Q_{0},\ldots,P_{m}\otimes Q_{n}}
\end{mathpar}

\begin{remark}
  Observe that arities of tensored abstractions matches arities of
  tensored concretions if the original arities matched. Note also that
  the length of the arities corresponds to the increase in dimension
  we see in ordinary vector space tensor product.
\end{remark}

\begin{remark}
  Operationally, this definition distributes the tensor down to
  components ``linked'' by summation. Tensor over summation is
  intriguing in that it mixes names. Moreover, as a consequence of the
  way it mixes names we have the identities for all $x \in \QProc$ and
  $P,Q \in \Proc$

  \begin{mathpar}
    (x \cdot P) \otimes Q \equiv x \cdot (P \otimes Q) \equiv P \otimes (x \cdot Q)
    \and
    P \otimes \pzero \equiv P
  \end{mathpar}

  that the reader is invited to verify.
\end{remark}

\subsubsection{Annihilation}
\begin{mathpar}
  P^{\perp} := \{ Q | \forall R. P|Q \red^{*} R \Rightarrow R \red^{*} \pzero \}
  \and \\
  P^{\underline{\perp}} := \Sigma_{Q \in P^{\perp}} \quotep{Q}?(y).(\dropn{y}|Q) | \Sigma_{Q \in P^{\perp}} \quotep{Q}\clift{\Box}
\end{mathpar}

\paragraph{Discussion} The reader will note that $P^{\perp}$ is a
\emph{set} of processes, while $P^{\underline{\perp}}$ is a
\emph{context}. We call the set $P^{\perp}$ the \emph{annihilators} of
$P$. The parallel composition of a process in the annihilators of $P$
with $P$ will result in a process, the state space of which has all
paths eventually leading to $\pzero$. Execution may endure loops; but
under reasonable conditions of fairness (naturally guaranteed under
most notions of bisimulation) such a composite process cannot get
stuck in such a loop and will, eventually pop out and terminate.

The context $P^{\underline{\perp}}$ is ready and willing to ``take the
$P$ out of'' the process to which it is applied. It will effectively
transmit the code of the process to which it is applied to one of the
annihilators and run the process against it.

\subsubsection{Evaluation}
We fix $M$ a domain of fully abstract interpretation with an equality
coincident with bisimulation. We take $\meaningof{\cdot} : \Proc \to
M$ to be the map interpreting processes and $\nmeaningof{\cdot} : \M
\to Proc$ to be the map running the other way. Then we define

\begin{mathpar}
  \int P := \nmeaningof{\meaningof{P}}
\end{mathpar}

\paragraph{Discussion}
There are many fully abstract interpretations of Milner's
$\pi$-calculus. Any of them can be used as a basis for interpreting
the reflective calculus here. Equipped with such a domain it is
largely a matter of grinding through to check that the Yoneda
construction for the normalization-by-evaluation program can be
extended to this setting.

\begin{remark}
  The reader is invited to verify that $\int (P^{\underline{\perp}}[P]) = 0$.
\end{remark}

\subsection{Quantum mechanics}

Table \ref{tbl:core_qm_op_defns} gives the core operational definitions

\begin{table}[htp]\label{tbl:core_qm_op_defns}
  \center{
    \fbox{
      \begin{tabular}{c|c}
        quantum mechanics & process calculus \\
        \hline
        scalar & $x := \quotep{P}$ \\
        state vector & $\state{P} := P$ \\
        dual & $\state{P}^{*} := \event{P^{\underline{\perp}}} := \quotep{P^{\underline{\perp}}}[-]$ \\
        matrix & $ \Sigma_{\alpha} \state{P_{\alpha}}x_{\alpha}\event{Q_{\alpha}}$ \\
        vector addition & $\state{P} + \state{Q} := \state{P | Q}$ \\
        tensor product & $\state{P} \otimes \state{Q} := \state{P \otimes Q}$ \\
        inner product & $\innerprod{P}{Q} := \quotep{\int P^{\underline{\perp}}[Q]}$ \\
      \end{tabular}
    }
  }
  \caption{QM - operational definitions}
\end{table}

where

\begin{mathpar}
  \prmatrix{P}{Q} := \fprmatrix{P}{\quotep{\pzero}}{Q}
  \and
  \fprmatrix{P}{x}{Q} := (\state{P},x,\event{Q})
  \and
  (\fprmatrix{P}{x}{Q})(\state{R}) := x \cdot \innerprod{Q}{R} \cdot \state{P}
  \and
  (\fprmatrix{P}{x}{Q})(\event{R}) := x \cdot \innerprod{R}{P} \cdot \event{Q}
\end{mathpar}

\paragraph{Discussion}
As promised: vectors (aka states) are represented as processes; duals
as contextual duals; inner product definition should be compared with
standard inner product definition for ....

\begin{remark}
  Assuming $\int (P^{\underline{\perp}}[P]) = 0$, the reader is
  invited to verify that $(\fprmatrix{P}{x}{P})(\state{P}) = x \cdot \state{P}$.
\end{remark}

\begin{remark}
  The reader is invited to verify that $\innerprod{P}{Q}$ could
  equally well have been written $\quotep{\int \stackrel{\vee}{x}}$
  where $x = \event{P^{\underline{\perp}}}(Q)$.

  One of the motivations for this remark is that there is another way
  to factor these operations. We could package up evaluation in the dual:

  \begin{mathpar}
    \state{P}^{*} := \event{\int P^{\underline{\perp}}} := \quotep{\int P^{\underline{\perp}}}[-]
  \end{mathpar}

  and then have inner product defined by
  
  \begin{mathpar}
    \innerprod{P}{Q} := \event{P}(Q)
  \end{mathpar}

  Hopefully, experience with the calculations will provide guidance on
  the best factoring.
\end{remark}

\begin{remark}
  Assuming $\int (P^{\underline{\perp}}[P]) = 0$, the reader is
  invited to verify that $\forall P,Q. (\prmatrix{0}{Q})(\state{0}) =
  \state{0}$ and dually $(\prmatrix{P}{0})(\event{0}) = \event{0}$.
\end{remark}

\begin{remark}
  i'm a little worried that i don't (yet) have proper support for
  complex conjugacy. But, the observation above may give us a
  clue. According to Abramsky, it must be the case that the scalars
  are iso to the homset of the identity for the tensor -- which the
  observation above characterizes. 

  For now, we will simply bookmark the notion with $\overline{x}$.
\end{remark}

\subsubsection{Adjointness}

We need to give a definition of $(\cdot)^{\dagger}$ for matrices. The
obvious candidate definition is
\begin{mathpar}
(\Sigma_{\alpha}\fprmatrix{P_{\alpha}}{x_{\alpha}}{Q_{\alpha}})^{\dagger}
= \Sigma_{\alpha}\fprmatrix{(Q_{\alpha}^{\underline{\perp}})^{*}}{\overline{x}_{\alpha}}{P_{\alpha}^{\underline{\perp}}} 
\end{mathpar}

But, $(Q_{\alpha}^{\underline{\perp}})^{*}$ requires a name along
which to communicate the process to achieve the context application.

\subsubsection{Basis for a basis}
If processes label states and ``addition'' of states (a.k.a. vector
addition) is interpreted as parallel composition, what corresponds to
notions of linear independence and basis? Here, we recall that Yoshida
has developed a set of \emph{combinators} for an asynchronous verison
of Milner's $\pi$-calculus. These are a finite set of processes such
any process can be expressed as parallel composition of these
combinators together with liberal uses of the new operator and
replication. We can simply give a translation of these into the
present calculus and have reasonable expectation that the property
carries over. That is, that the resultant set allows to express all
processes via parallel composition. Note, however, that there is no
new operator or replication in this calculus. As a result, we expect
that the corresponding set is actually infinite. That is, we expect
that the space is actually infinite dimensional.

\begin{remark}
  The attentive reader may be a bit concerned. Certainly, the
  collection $S$, $K$ and $I$ is a finite set of
  combinators. Shouldn't we expect to see a finite set of combinators
  for an effectively equivalent system? i am very sympathetic to this
  critique and feel it warrants full attention. On the other hand, i
  also have in mind the following analogy. The natural numbers, as a
  monoid under addition, has exactly $1$ generator, while the natural
  numbers, as a monoid under multiplication, has countably many
  generators (the primes). We observe that the application of the
  lambda calculus is much less resource sensitive than the parallel
  composition of the $\pi$-calculus. Could it be the case that we have
  an analogy of the form
  
  \begin{mathpar}
    m + n : MN :: m*n : M|N
  \end{mathpar}

  giving a similar blow up in the set of ``primes''?  This is such a
  wonderful thought that, even if it's not true, i think it's worth
  writing down.
\end{remark}
 

\documentclass[12pt]{llncs}
%\documentclass{jktr}

\usepackage[pdftex]{hyperref}                   
\usepackage {listings}
\usepackage {mathpartir}
\usepackage{bcprules}
%\usepackage{listings}
                       
\usepackage{graphicx} 
%\usepackage[margins=2.5cm,nohead,nofoot]{geometry}
%\usepackage{geometry}
\usepackage{amsfonts}
\usepackage{amstext}
\usepackage{latexsym}
\usepackage{amssymb}
\usepackage{color}


%\include{myPreamble}
\include{qm2pi.local} 

%\ifpdf
%\usepackage[pdftex]{graphicx}
%\else
%\usepackage{graphicx}
%\fi

 % \ifpdf
%  \usepackage{pdfsync}
%  \if


%\title{Brief Article}
%\author{David F. Snyder}
%\author{L.G. Meredith}

%\address{Dept. of Math., Texas State University--San Marcos, San Marcos, TX 78666}
       
\pagestyle{empty}


\begin{document}

\lstset{language=[Objective]Caml,frame=shadowbox}

\input{qm2pi.front}

% section front matter (end)

\input{qm2pi.intro} 
 
% section introduction (end)

% \input{qm2pi.knotations} 

% section notation (end)

\input{qm2pi.process.calculi} 

% section concurrent_process_calculi_and_spatial_logics_ (end)
    
%\input{qm2pi.knots2pi} 

%\input{qm2pi.trefoil} 

%\input{qm2pi.mainthm} 

% subsection basic_interpretation (end)

%\input{qm2pi.rho.presentation} 
\subsection{The syntax and semantics of the notation system}\label{sub:the_syntax_and_semantics_of_the_notation_system} % (fold)

We now summarize a technical presentation of the calculus that
embodies our theory of dynamics. The typical presentation of such a
calculus follows the style of giving generators and relations on
them. The grammar, below, describing term constructors, freely
generates the set of processes, $\Proc$. This set is then quotiented
by a relation known as structural congruence and it is over this set
that the notion of dynamics is expressed. This presentation is
essentially that of \cite{MeredithR05} with the addition of
polyadicity and summation. For readability we have relegated some of
the technical subtleties to an appendix.

\subsubsection{Process grammar}\label{subsub:process_grammar}

\begin{mathpar}
  \inferrule* [lab=synchronization] {} {{M} \bc \pzero \;|\; x?F \;|\; x!C }
  \and
  \inferrule* [lab=abstraction] {} {{F} \bc (x)P}
  \and
  \inferrule* [lab=concretion] {} {{C} \bc \langle Q \rangle}
  \and
  \inferrule* [lab=process] {} {{P,Q} \bc M \;| \;P|Q \;|\; @{x}}
  \and
  \inferrule* [lab=name] {} {{x} \bc \quotep{P}}
\end{mathpar} 

Note that $\vec{x}$ (resp. $\vec{P}$) denotes a vector of names
(resp. processes) of length $|\vec{x}|$ (resp. $|\vec{P}|$). We adopt
the following useful abbreviations.

\begin{mathpar}
   x?(\vec{y}).P := x.(\vec{y})P \and  x\clift{\vec{P}} := x.\clift{\vec{P}}
   \and x!(y) := \lift{x}{\dropn{y}}
   \and \Pi_{i=0}^{n-1}P_i := P_0 | \ldots | P_{n-1}
\end{mathpar}

\subsubsection{Structural congruence}

\paragraph{Free and bound names and alpha-equivalence.} At the
core of structural equivalence is alpha-equivalence which identifies
process that are the same up to a change of variable. Formally, we
recognize the distinction between free and bound names. The free names
of a process, $\freenames{P}$, may be calculated recursively as
follows:

\begin{mathpar}
\freenames{\pzero} := \emptyset
  \and \\
  \freenames{x?(y).P} := \{ x \} \cup (\freenames{P} \setminus \{ y \})
  \and 
  \freenames{x!\langle P \rangle} := \{ x \} \cup \{ P \} 
  \and \\
  \freenames{P|Q} := \freenames{P} \cup \freenames{Q}
  \and \\
  \freenames{@{x}} := \{ x \}
\end{mathpar}

$\pi$
$\quotep{\pi}$

$\freenames{-} : \pi \to \mathcal{P}(\quotep{\pi})$

\begin{eqnarray*}
  \freenames{\pzero} & := & \emptyset \\
  \freenames{x?(y).P} & := & \{ x \} \cup (\freenames{P} \setminus \{ y \}) \\
  \freenames{x!\langle P \rangle} & := & \{ x \} \cup \{ P \} \\
  \freenames{P|Q} & := & \freenames{P} \cup \freenames{Q} \\
  \freenames{\dropn{x}} & := & \{ x \}
\end{eqnarray*}

The bound names of a process, $\boundnames{P}$, are those names occurring in $P$
that are not free. For example, in $x?(y).0$, the name $x$ is free, while $y$ is bound.

\begin{mathpar}
  \inferrule* [lab=monoidal-laws] {} { P|Q \equiv Q|P \and P|0 \equiv P \and P|(Q|R) \equiv (P|Q)|R }
\end{mathpar}

\begin{mathpar}
  \inferrule* [lab=alpha-equivalence] {} { (x)P \equiv (y)P\{y/x\} \and y \not\in \freenames{P} }
\end{mathpar}

\begin{definition}
Then two processes, $P,Q$, are alpha-equivalent if $P = Q\{\vec{y}/\vec{x}\}$ for
some $\vec{x} \in \boundnames{Q},\vec{y} \in \boundnames{P}$, where $Q\{\vec{y}/\vec{x}\}$
denotes the capture-avoiding substitution of $\vec{y}$ for $\vec{x}$ in $Q$.
\end{definition}

\begin{definition}
  The {\em structural congruence} \cite{SangiorgiWalker} , $\equiv$,
  between processes is the least congruence containing
  alpha-equivalence, satisfying the abelian monoid laws
  (associativity, commutativity and $\pzero$ as identity) for parallel
  composition $|$ and for summation $+$.
\end{definition}

\subsection{Name equivalence}

We take name equivalence, written $\nameeq$, to be the smallest
equivalence relation generated by the following rules.

\begin{mathpar}
\inferrule*[lab=Quote-drop]
{ }
{ \quotep{@{x}} \nameeq x }

\inferrule*[lab=Struct-equiv]
{ P \scong Q }
{ \quotep{P} \nameeq \quotep{Q} }
\end{mathpar}

The astute reader will have noticed that the mutual recursion of names
and processes imposes a mutual recursion on alpha-equivalence and
structural equivalence via name-equivalence. Fortunately, all of this
works out pleasantly and we may calculate in the natural way, free of
concern. The reader interested in the details is referred to the
appendix \ref{appendix:rho_details}.

\subsection{Substitution}

We use $\Proc$ for the set of processes, $\QProc$ for the set of
names, and $\id{\{}\vec{y} / \vec{x} \id{\}}$ to denote partial maps,
$s : \QProc \rightarrow \QProc$. A map, $s$ lifts, uniquely, to a map
on process terms, $\widehat{s} : \Proc \rightarrow \Proc$ by the
following equations.

\begin{mathpar}
  (0) \psubstp{Q}{P} := 0 \\
  (R \juxtap S) \psubstp{Q}{P}
  :=    
  (R)\psubstp{Q}{P} \juxtap (S) \psubstp{Q}{P} \\
  (x?(y).R) \psubstp{Q}{P}    
  :=    
  (x)\substp{Q}{P} (z)\concat( (R \psubstn{z}{y}) \psubstp{Q}{P} ) \\
  (\lift{x}{R}) \psubstp{Q}{P}  
  :=
  \lift{(x)\substp{Q}{P}}{ R \psubstp{Q}{P} } \\
%   (\dropn{x})  \psubstp{Q}{P}       
%   := 
%   \left\{ 
%     \begin{array}{ccc} 
%       \dropn{\quotep{Q}} & & x \nameeq \quotep{P} \\
%       \dropn{x} & & otherwise \\
%     \end{array}
%   \right. 
  (\dropn{x})  \psubstp{Q}{P}       
  := 
  \left\{ 
    \begin{array}{ccc} 
      Q & & x \nameeq \quotep{P} \\
      \dropn{x} & & otherwise \\
    \end{array}
  \right.
\end{mathpar}
 

where

\begin{eqnarray}
  (x)\id{\{} \lpquote Q \rpquote / \lpquote P \rpquote \id{\}}            = 
  \left\{ 
    \begin{array}{ccc}
      \lpquote Q \rpquote & & x \nameeq \lpquote P \rpquote \\
      x & & otherwise \\
    \end{array}
  \right. \nonumber
\end{eqnarray}

and $z$ is chosen distinct from $\quotep{P}$, $\quotep{Q}$, the free
names in $Q$, and all the names in $R$. Our $\alpha$-equivalence will
be built in the standard way from this substitution.

\begin{remark}\label{rem:no_self_referential_names}
  One consequence of these definitions is that $\forall P. \quotep{P}
  \not\in \freenames{P}$.
\end{remark}

\subsection{ Dynamic quote: an example }

Anticipating something of what's to come, consider applying the
substitution, $\widehat{\id{\{}u / z \id{\}}}$, to the following pair
of processes, $\lift{w}{y!(z)}$ and $w[ \lpquote y!(z) \rpquote ]$.

\begin{eqnarray}
	\lift{w}{y!(z)}\widehat{\id{\{}u / z \id{\}}}
		& = &
		\lift{w}{y!(u)} \nonumber\\
	w[ \lpquote y!(z) \rpquote ] \widehat{ \id{\{}u / z \id{\}} }
		& = &
		w[ \lpquote y!(z) \rpquote ] \nonumber
\end{eqnarray}

Because the body of the process between quotes is impervious to
substitution, we get radically different answers. In fact, by
examining the first process in an input context,
e.g. $x?(z).\lift{w}{y!(z)}$, we see that the process under the lift
operator may be shaped by prefixed inputs binding a name inside it. In
this sense, the lift operator will be seen as a way to dynamically
construct processes before reifying them as names.

Finally equipped with these standard features we can present the
dynamics of the calculus.

\subsubsection{Operational semantics} 

Finally, we introduce the computational dynamics. What marks these
algebras as distinct from other more traditionally studied algebraic
structures, e.g. vector spaces or polynomial rings, is the manner in
which dynamics is captured. In traditional structures, dynamics is typically
expressed through morphisms between such structures, as in linear maps
between vector spaces or morphisms between rings. In algebras
associated with the semantics of computation, the dynamics is
expressed as part of the algebraic structure itself, through a
reduction reduction relation typically denoted by $\red$. Below, we
give a recursive presentation of this relation for the calculus used
in the encoding.

$\red \subseteq \pi \times \pi$
$\red : \pi \to \mathcal{P}(\pi)$

\begin{mathpar}
  \inferrule* [lab=Comm] { \textsf{match}( x_{src}, x_{trgt} ) } { x_{trgt}?(y)P \; | \; x_{src}!\langle {Q} \rangle \red P\{\quotep{Q}/y}\} }
  \and \\
  \inferrule* [lab=Par] {{P} \red {P}'} {{{P} | {Q}} \red {{P}' | {Q}}}
  \and
  \inferrule* [lab=Equiv]{{{P} \scong {P}'} \andalso {{P}' \red {Q}'} \andalso {{Q}' \scong {Q}}}{{P} \red {Q}}
\end{mathpar}

\begin{eqnarray*}
  match_{\equiv} (\quotep{P},\quotep{Q}) & := & P \equiv Q \\
  match_{\dagger}(\quotep{P},\quotep{Q}) & := & \forall R. P|Q \red^{*} R => R \red^{*} 0 \\
  match_{K}(\quotep{P},\quotep{Q}) & := & K \mbox{ for some context } K
\end{eqnarray*}

$u?(x)P | u!\langle Q \rangle \red P\{\quotep{Q}/x\}$

%We write $\wred$ for $\red^*$, and $P\red$ if $\exists Q $ such that $ P \red Q$.
We write $P\red$ if $\exists Q $ such that $ P \red Q$ and $P\not\red$, otherwise.

\section{Replication}

As mentioned before, it is known that replication (and hence
recursion) can be implemented in a higher-order process algebra
\cite{SangiorgiWalker}. As our first example of calculation with the
machinery thus far presented we give the construction explicitly in
the {\rhoc}.

\begin{eqnarray}
	D_{x} & := & \prefix{x}{y}{(\binpar{\outputp{x}{y}}{@{y}})} \nonumber\\
	\bangp_{x}{P} & := & \binpar{{x}!\langle{\binpar{D_{x}}{P}}\rangle}{D_{x}} \nonumber
\end{eqnarray}

\begin{eqnarray}
	\bangp_{x}{P} & & \nonumber\\
	=
	& {x}!\langle{(\prefix{x}{y}{(\outputp{x}{y} | @{y})) | P}}\rangle 
	      | \prefix{x}{y}{(\outputp{x}{y} | @{y})} & \nonumber\\
	\red
	& (\outputp{x}{y} | @{y})\substn{\quotep{(\prefix{x}{y}{(@{y} | \outputp{x}{y})) | P}}}{y} & \nonumber\\
	=
	& \outputp{x}{\quotep{(\prefix{x}{y}{(\outputp{x}{y} | @{y})) | P}}}
	  | {(\prefix{x}{y}{(\outputp{x}{y} | @{y})) | P}} & \nonumber\\
	\red
	& \ldots & \nonumber\\
	\red^*
	& P | P | \ldots & \nonumber
\end{eqnarray}

Of course, this encoding, as an implementation, runs away, unfolding
$\bangp{P}$ eagerly. A lazier and more implementable replication
operator, restricted to input-guarded processes, may be obtained as follows.

\begin{eqnarray}
\bangp{\prefix{u}{v}{P}} 
	:= 
	\binpar{\lift{x}{\prefix{u}{v}{(\binpar{D(x)}{P})}}}{D(x)} \nonumber
\end{eqnarray}

\begin{remark}
  Note that the lazier definition still does not deal with summation
  or mixed summation (i.e. sums over input and output). The reader is
  invited to construct definitions of replication that deal with these
  features. 

  Further, the definitions are parameterized in a name, $x$. Can you,
  gentle reader, make a definition that eliminates this parameter and
  guarantees no accidental interaction between the replication
  machinery and the process being replicated -- i.e. no accidental
  sharing of names used by the process to get its work done and the
  name(s) used by the replication to effect copying. This latter
  revision of the definition of replication is crucial to obtaining
  the expected identity $!!P \sim !P$.
\end{remark}

\begin{remark}\label{rem:paradoxical_combinator}
  The reader familiar with the lambda calculus will have noticed the
  similarity between $D$ and the paradoxical combinator.

  [Ed. note: the existence of this seems to suggest we have to be more
  restrictive on the set of processes and names we admit if we are to
  support no-cloning.]
\end{remark}

\subsubsection{Bisimulation}

The computational dynamics gives rise to another kind of equivalence,
the equivalence of computational behavior. As previously mentioned
this is typically captured \emph{via} some form of bisimulation.

% The notion we use in this paper is weak barbed bisimulation
% \cite{milner91polyadicpi}.

The notion we use in this paper is derived from weak barbed
bisimulation \cite{milner91polyadicpi}. 

\begin{definition}
An \emph{observation relation}, $\downarrow_{\mathcal N}$, over a set
of names, $\mathcal N$, is the smallest relation satisfying the rules
below.

\infrule[Out-barb]{y \in {\mathcal N}, \; x \nameeq y}
		  {\outputp{x}{v} \downarrow_{\mathcal N} x}
\infrule[Par-barb]{\mbox{$P\downarrow_{\mathcal N} x$ or $Q\downarrow_{\mathcal N} x$}}
		  {\binpar{P}{Q} \downarrow_{\mathcal N} x}

We write $P \Downarrow_{\mathcal N} x$ if there is $Q$ such that 
$P \wred Q$ and $Q \downarrow_{\mathcal N} x$.
\end{definition}

\begin{definition}
%\label{def.bbisim}
An  ${\mathcal N}$-\emph{barbed bisimulation} over a set of names, ${\mathcal N}$, is a symmetric binary relation 
${\mathcal S}_{\mathcal N}$ between agents such that $P\rel{S}_{\mathcal N}Q$ implies:
\begin{enumerate}
\item If $P \red P'$ then $Q \wred Q'$ and $P'\rel{S}_{\mathcal N} Q'$.
\item If $P\downarrow_{\mathcal N} x$, then $Q\Downarrow_{\mathcal N} x$.
\end{enumerate}
$P$ is ${\mathcal N}$-barbed bisimilar to $Q$, written
$P \wbbisim_{\mathcal N} Q$, if $P \rel{S}_{\mathcal N} Q$ for some ${\mathcal N}$-barbed bisimulation ${\mathcal S}_{\mathcal N}$.
\end{definition}

$\mathcal{R} \subseteq \pi \times \pi$

$P \mathcal{R} Q => \forall P'. P \red P' \Rightarrow \exists Q'. Q \red Q', P' \mathcal{R} Q'$

$P \vdash x \Rightarrow Q \vdash x$

\begin{mathpar}
  \inferrule*[lab=Out-barb]{x \nameeq y}{{y}!\langle{Q}\rangle \vdash x}
  \and
  \inferrule*[lab=Par-barb]{\mbox{$P\vdash x$ or $Q\vdash x$}}{\binpar{P}{Q} \vdash x}
\end{mathpar}

\subsubsection{Contexts}

One of the principle advantages of computational calculi like the
$\pi$-calculus is a well-defined notion of context,
contextual-equivalence and a correlation between
contextual-equivalence and notions of bisimulation. The notion of
context allows the decomposition of a process into (sub-)process and
its syntactic environment, its context. Thus, a context may be
thought of as a process with a ``hole'' (written $\Box$) in it. The
application of a context $M$ to a process $P$, written $M[P]$, is
tantamount to filling the hole in $M$ with $P$. In this paper we do
not need the full weight of this theory, but do make use of the notion
of context in the proof the main theorem. 

\begin{mathpar}
  \inferrule* [lab=summation] {} {{M_{M},M_{N}} \bc \Box \;|\; x.M_{A} \;|\; M_{M}+M_{N}}
  \and
  \inferrule* [lab=agent] {} {{M_{A}} \bc (\vec{x})M_{P} \;| \; \clift{P_0,\ldots,M_{P},\ldots,P_N}}
  \and \\
  \inferrule* [lab=process] {} {{M_{P}} \bc M_{N} \;| \;P|M_{P} }
\end{mathpar} 

\begin{mathpar}
  \inferrule* [lab=sychronization] {} {M_{N} \bc \Box \;|\; x?M_{F} \;|\; x!M_{C}}
  \and
  \inferrule* [lab=abstraction] {} {{M_{F}} \bc (x)M_{P} }
  \and
  \inferrule* [lab=concretion] {} {{M_{C}} \bc \langle M_{P} \rangle }
  \and \\
  \inferrule* [lab=process] {} {{M_{P}} \bc M_{N} \;| \;P|M_{P} }
\end{mathpar}

\begin{definition}[contextual application] Given a context $M$, and
  process $P$, we define the \emph{contextual application}, $M[P] :=
  M\{P/\Box\}$. That is, the contextual application of M to P is the
  substitution of $P$ for $\Box$ in $M$.
\end{definition}

$\meaningof{-} : L \to \mathcal{P}(\pi)$

\begin{mathpar}
  \inferrule* [lab=collection] {} {\meaningof{true} = \pi, \and \meaningof{~E} = \pi \setminus \meaningof{E}, \and \meaningof{E_{1} \& E_{2}} = \meaningof{E_{1}} \cap \meaningof{E_{2}}}
\end{mathpar}

\begin{mathpar}
  \inferrule* [lab=structure] {} {\meaningof{0} = \{ P \in \pi | P \equiv 0 \}, \and \\ \meaningof{E_1 | E_2} = \{ P \in \pi | P \equiv P_{1} | P_{2}, P_{1} \in \meaningof{E_{1}}, P_{2} \in \meaningof{E_2}\} }
\end{mathpar}

\begin{mathpar}
 \inferrule* [lab=behavior] {} {\meaningof{\langle a?b \rangle E} = \{ P \in \pi | P \equiv Q | u?(y)P', \\ \and \\\\ \and \\ \;\;\; u \in \meaningof{a}, \forall z.P'\{z/y\} \in \meaningof{E\{z/b\}}\}, \and \\ \meaningof{a!E} = \{ P \in \pi | P \equiv Q | x!\langle P' \rangle, x \in \meaningof{a} P' \in \meaningof{E}\} }
\end{mathpar}

\begin{mathpar}
 \inferrule* [lab=nominal] {} {\meaningof{\quotep{E}} = \{ \quotep{P} \in \quotep{\pi} | P \in \meaningof{E} \}, \and \meaningof{\quotep{P}} = \{ \quotep{Q} \in \quotep{\pi} | P \equiv Q \} \and \\ \meaningof{@\quotep{E}} = \{ P \in \pi | P \equiv @x, x \in \meaningof{E} \}}
\end{mathpar}

\begin{eqnarray*}
  \\
  \meaningof{-} : TS \to ST
\end{eqnarray*}

\begin{eqnarray*}
  \\
  L : TS \to ST
\end{eqnarray*}

\begin{eqnarray*}
  \\
  P \models E \iff P \in \meaningof{E}
\end{eqnarray*}

\begin{eqnarray*}
  P \approx_{L} Q \iff \forall E \in L. P \models E \iff Q \models E
\end{eqnarray*}

\begin{eqnarray*}
  P \approx_{K} Q
\end{eqnarray*}

\begin{eqnarray*}
  P \approx Q
\end{eqnarray*}

$\approx_{K} = \approx = \approx_{L}$

\subsubsection{Contextual duality}

Note that contexts extend the quotation operation to a family of
operations from processes to names. Given a context, $M$, we can
define a \emph{nominal context}, $\quotep{M}$ by $\quotep{M}[P] :=
\quotep{M[P]}$. To foreshadow what is to come we observe that these
operations enjoy a duality with processes very much like the duality
between vectors and maps from vectors to scalars.

Further, because the calculus is essentially higher-order, we have a
correspondence between contexts and processes. More specifically,
given a name $x$ and a context $M$ we can construct $M^{*}_{x}$ such
that 

\begin{mathpar}
  M^{*}_{x} | \lift{x}{P} \red M[P]
\end{mathpar}

namely,

\begin{mathpar}
  M^{*}_{x} := x?(u).M[\dropn{u}]
\end{mathpar}

The dependence of $M^{*}_{x}$ on a name makes it an abstraction, 

\begin{mathpar}
  M^{*} := (x)x?(u).M[\dropn{u}]
\end{mathpar}

\subsection{Additional notation}

It will sometimes be convenient to denote the process a name
quotes. We already have the notation $x = \quotep{P}$, but it will be
convenient to introduce an alternate notation, $\procn{x}$, when we
want to emphasize the connection to the use of the name. Note that, by
virtue of name equivalence, $\quotep{\procn{x}} \nameeq x$; so, the
notation is consistent with previous definitions.

Further, because names have structure it is possible to effect
substitutions on the basis of that structure. This means we need to
upgrade our notation for substitutions, which we accomplish by
adapting comprehension notation. Thus,

\begin{mathpar}
  P\{ y / x : x \in S \}
\end{mathpar}

is interpreted to mean the process derived from P by replacing (in a
capture-avoiding manner) each occurrence of $x$ in $S$ by $y$. For example,

\begin{mathpar}
  P\{ \quotep{\procn{x}|\procn{x}} / x : x \in \freenames{P} \}
\end{mathpar}

will replace each (occurrence) of a free name $x$ in $P$ by
$\quotep{\procn{x}|\procn{x}}$.

Also, we will avail ourselves of the notation $x^{L}$ and $x^{R}$ to
denote injections of a name into disjoint copies of the name
space. There are numerous ways to accomplish this. One example can be
found in \cite{MeredithR05}. This notation overloads to vectors of
names: $\vec{x}^{\pi} := (x_{i}^{\pi} \; : \; 0 \leq i < |\vec{x}| )$ where $\pi \in \{L,R\}$.

We also use $P^{\Box} := P|\Box$.

In \cite{MeredithR05} an interpretation of the new operator is
given. It turns out that there are several possible interpretations
all enjoying the requisite algebraic properties of the operator (see
\cite{milner91polyadicpi}). We will therefore make liberal use of
$(\nu\; \vec{x})P$.

% subsection the_syntax_and_semantics_of_the_notation_system (end)   

\input{qm2pi.qmops} 

\input{qm2pi.sterngerlach} 

\input{qm2pi.metric} 

% section concurrent_process_calculi (end)

%\input{qm2pi.proofsketch}

% section proof sketch (end)

%\input{qm2pi.slviaknots} 

% section spatial logic via knots (end)

\input{qm2pi.conclusion}

% section conclusion (end)

%\input{qm2pi.dtcodes} 

% section wiring algorithm (end)

\input{qm2pi.ack} 

% section acknowledgments (end)

\newpage


\bibliographystyle{plain}   
\bibliography{../../biblios/main.bib}

\input{qm2pi.rhodetails}

\end{document}

 

\documentclass[12pt]{llncs}
%\documentclass{jktr}

\usepackage[pdftex]{hyperref}                   
\usepackage {listings}
\usepackage {mathpartir}
\usepackage{bcprules}
%\usepackage{listings}
                       
\usepackage{graphicx} 
%\usepackage[margins=2.5cm,nohead,nofoot]{geometry}
%\usepackage{geometry}
\usepackage{amsfonts}
\usepackage{amstext}
\usepackage{latexsym}
\usepackage{amssymb}
\usepackage{color}


%\include{myPreamble}
\include{qm2pi.local} 

%\ifpdf
%\usepackage[pdftex]{graphicx}
%\else
%\usepackage{graphicx}
%\fi

 % \ifpdf
%  \usepackage{pdfsync}
%  \if


%\title{Brief Article}
%\author{David F. Snyder}
%\author{L.G. Meredith}

%\address{Dept. of Math., Texas State University--San Marcos, San Marcos, TX 78666}
       
\pagestyle{empty}


\begin{document}

\lstset{language=[Objective]Caml,frame=shadowbox}

\input{qm2pi.front}

% section front matter (end)

\input{qm2pi.intro} 
 
% section introduction (end)

% \input{qm2pi.knotations} 

% section notation (end)

\input{qm2pi.process.calculi} 

% section concurrent_process_calculi_and_spatial_logics_ (end)
    
%\input{qm2pi.knots2pi} 

%\input{qm2pi.trefoil} 

%\input{qm2pi.mainthm} 

% subsection basic_interpretation (end)

%\input{qm2pi.rho.presentation} 
\subsection{The syntax and semantics of the notation system}\label{sub:the_syntax_and_semantics_of_the_notation_system} % (fold)

We now summarize a technical presentation of the calculus that
embodies our theory of dynamics. The typical presentation of such a
calculus follows the style of giving generators and relations on
them. The grammar, below, describing term constructors, freely
generates the set of processes, $\Proc$. This set is then quotiented
by a relation known as structural congruence and it is over this set
that the notion of dynamics is expressed. This presentation is
essentially that of \cite{MeredithR05} with the addition of
polyadicity and summation. For readability we have relegated some of
the technical subtleties to an appendix.

\subsubsection{Process grammar}\label{subsub:process_grammar}

\begin{mathpar}
  \inferrule* [lab=synchronization] {} {{M} \bc \pzero \;|\; x?F \;|\; x!C }
  \and
  \inferrule* [lab=abstraction] {} {{F} \bc (x)P}
  \and
  \inferrule* [lab=concretion] {} {{C} \bc \langle Q \rangle}
  \and
  \inferrule* [lab=process] {} {{P,Q} \bc M \;| \;P|Q \;|\; @{x}}
  \and
  \inferrule* [lab=name] {} {{x} \bc \quotep{P}}
\end{mathpar} 

Note that $\vec{x}$ (resp. $\vec{P}$) denotes a vector of names
(resp. processes) of length $|\vec{x}|$ (resp. $|\vec{P}|$). We adopt
the following useful abbreviations.

\begin{mathpar}
   x?(\vec{y}).P := x.(\vec{y})P \and  x\clift{\vec{P}} := x.\clift{\vec{P}}
   \and x!(y) := \lift{x}{\dropn{y}}
   \and \Pi_{i=0}^{n-1}P_i := P_0 | \ldots | P_{n-1}
\end{mathpar}

\subsubsection{Structural congruence}

\paragraph{Free and bound names and alpha-equivalence.} At the
core of structural equivalence is alpha-equivalence which identifies
process that are the same up to a change of variable. Formally, we
recognize the distinction between free and bound names. The free names
of a process, $\freenames{P}$, may be calculated recursively as
follows:

\begin{mathpar}
\freenames{\pzero} := \emptyset
  \and \\
  \freenames{x?(y).P} := \{ x \} \cup (\freenames{P} \setminus \{ y \})
  \and 
  \freenames{x!\langle P \rangle} := \{ x \} \cup \{ P \} 
  \and \\
  \freenames{P|Q} := \freenames{P} \cup \freenames{Q}
  \and \\
  \freenames{@{x}} := \{ x \}
\end{mathpar}

$\pi$
$\quotep{\pi}$

$\freenames{-} : \pi \to \mathcal{P}(\quotep{\pi})$

\begin{eqnarray*}
  \freenames{\pzero} & := & \emptyset \\
  \freenames{x?(y).P} & := & \{ x \} \cup (\freenames{P} \setminus \{ y \}) \\
  \freenames{x!\langle P \rangle} & := & \{ x \} \cup \{ P \} \\
  \freenames{P|Q} & := & \freenames{P} \cup \freenames{Q} \\
  \freenames{\dropn{x}} & := & \{ x \}
\end{eqnarray*}

The bound names of a process, $\boundnames{P}$, are those names occurring in $P$
that are not free. For example, in $x?(y).0$, the name $x$ is free, while $y$ is bound.

\begin{mathpar}
  \inferrule* [lab=monoidal-laws] {} { P|Q \equiv Q|P \and P|0 \equiv P \and P|(Q|R) \equiv (P|Q)|R }
\end{mathpar}

\begin{mathpar}
  \inferrule* [lab=alpha-equivalence] {} { (x)P \equiv (y)P\{y/x\} \and y \not\in \freenames{P} }
\end{mathpar}

\begin{definition}
Then two processes, $P,Q$, are alpha-equivalent if $P = Q\{\vec{y}/\vec{x}\}$ for
some $\vec{x} \in \boundnames{Q},\vec{y} \in \boundnames{P}$, where $Q\{\vec{y}/\vec{x}\}$
denotes the capture-avoiding substitution of $\vec{y}$ for $\vec{x}$ in $Q$.
\end{definition}

\begin{definition}
  The {\em structural congruence} \cite{SangiorgiWalker} , $\equiv$,
  between processes is the least congruence containing
  alpha-equivalence, satisfying the abelian monoid laws
  (associativity, commutativity and $\pzero$ as identity) for parallel
  composition $|$ and for summation $+$.
\end{definition}

\subsection{Name equivalence}

We take name equivalence, written $\nameeq$, to be the smallest
equivalence relation generated by the following rules.

\begin{mathpar}
\inferrule*[lab=Quote-drop]
{ }
{ \quotep{@{x}} \nameeq x }

\inferrule*[lab=Struct-equiv]
{ P \scong Q }
{ \quotep{P} \nameeq \quotep{Q} }
\end{mathpar}

The astute reader will have noticed that the mutual recursion of names
and processes imposes a mutual recursion on alpha-equivalence and
structural equivalence via name-equivalence. Fortunately, all of this
works out pleasantly and we may calculate in the natural way, free of
concern. The reader interested in the details is referred to the
appendix \ref{appendix:rho_details}.

\subsection{Substitution}

We use $\Proc$ for the set of processes, $\QProc$ for the set of
names, and $\id{\{}\vec{y} / \vec{x} \id{\}}$ to denote partial maps,
$s : \QProc \rightarrow \QProc$. A map, $s$ lifts, uniquely, to a map
on process terms, $\widehat{s} : \Proc \rightarrow \Proc$ by the
following equations.

\begin{mathpar}
  (0) \psubstp{Q}{P} := 0 \\
  (R \juxtap S) \psubstp{Q}{P}
  :=    
  (R)\psubstp{Q}{P} \juxtap (S) \psubstp{Q}{P} \\
  (x?(y).R) \psubstp{Q}{P}    
  :=    
  (x)\substp{Q}{P} (z)\concat( (R \psubstn{z}{y}) \psubstp{Q}{P} ) \\
  (\lift{x}{R}) \psubstp{Q}{P}  
  :=
  \lift{(x)\substp{Q}{P}}{ R \psubstp{Q}{P} } \\
%   (\dropn{x})  \psubstp{Q}{P}       
%   := 
%   \left\{ 
%     \begin{array}{ccc} 
%       \dropn{\quotep{Q}} & & x \nameeq \quotep{P} \\
%       \dropn{x} & & otherwise \\
%     \end{array}
%   \right. 
  (\dropn{x})  \psubstp{Q}{P}       
  := 
  \left\{ 
    \begin{array}{ccc} 
      Q & & x \nameeq \quotep{P} \\
      \dropn{x} & & otherwise \\
    \end{array}
  \right.
\end{mathpar}
 

where

\begin{eqnarray}
  (x)\id{\{} \lpquote Q \rpquote / \lpquote P \rpquote \id{\}}            = 
  \left\{ 
    \begin{array}{ccc}
      \lpquote Q \rpquote & & x \nameeq \lpquote P \rpquote \\
      x & & otherwise \\
    \end{array}
  \right. \nonumber
\end{eqnarray}

and $z$ is chosen distinct from $\quotep{P}$, $\quotep{Q}$, the free
names in $Q$, and all the names in $R$. Our $\alpha$-equivalence will
be built in the standard way from this substitution.

\begin{remark}\label{rem:no_self_referential_names}
  One consequence of these definitions is that $\forall P. \quotep{P}
  \not\in \freenames{P}$.
\end{remark}

\subsection{ Dynamic quote: an example }

Anticipating something of what's to come, consider applying the
substitution, $\widehat{\id{\{}u / z \id{\}}}$, to the following pair
of processes, $\lift{w}{y!(z)}$ and $w[ \lpquote y!(z) \rpquote ]$.

\begin{eqnarray}
	\lift{w}{y!(z)}\widehat{\id{\{}u / z \id{\}}}
		& = &
		\lift{w}{y!(u)} \nonumber\\
	w[ \lpquote y!(z) \rpquote ] \widehat{ \id{\{}u / z \id{\}} }
		& = &
		w[ \lpquote y!(z) \rpquote ] \nonumber
\end{eqnarray}

Because the body of the process between quotes is impervious to
substitution, we get radically different answers. In fact, by
examining the first process in an input context,
e.g. $x?(z).\lift{w}{y!(z)}$, we see that the process under the lift
operator may be shaped by prefixed inputs binding a name inside it. In
this sense, the lift operator will be seen as a way to dynamically
construct processes before reifying them as names.

Finally equipped with these standard features we can present the
dynamics of the calculus.

\subsubsection{Operational semantics} 

Finally, we introduce the computational dynamics. What marks these
algebras as distinct from other more traditionally studied algebraic
structures, e.g. vector spaces or polynomial rings, is the manner in
which dynamics is captured. In traditional structures, dynamics is typically
expressed through morphisms between such structures, as in linear maps
between vector spaces or morphisms between rings. In algebras
associated with the semantics of computation, the dynamics is
expressed as part of the algebraic structure itself, through a
reduction reduction relation typically denoted by $\red$. Below, we
give a recursive presentation of this relation for the calculus used
in the encoding.

$\red \subseteq \pi \times \pi$
$\red : \pi \to \mathcal{P}(\pi)$

\begin{mathpar}
  \inferrule* [lab=Comm] { \textsf{match}( x_{src}, x_{trgt} ) } { x_{trgt}?(y)P \; | \; x_{src}!\langle {Q} \rangle \red P\{\quotep{Q}/y}\} }
  \and \\
  \inferrule* [lab=Par] {{P} \red {P}'} {{{P} | {Q}} \red {{P}' | {Q}}}
  \and
  \inferrule* [lab=Equiv]{{{P} \scong {P}'} \andalso {{P}' \red {Q}'} \andalso {{Q}' \scong {Q}}}{{P} \red {Q}}
\end{mathpar}

\begin{eqnarray*}
  match_{\equiv} (\quotep{P},\quotep{Q}) & := & P \equiv Q \\
  match_{\dagger}(\quotep{P},\quotep{Q}) & := & \forall R. P|Q \red^{*} R => R \red^{*} 0 \\
  match_{K}(\quotep{P},\quotep{Q}) & := & K \mbox{ for some context } K
\end{eqnarray*}

$u?(x)P | u!\langle Q \rangle \red P\{\quotep{Q}/x\}$

%We write $\wred$ for $\red^*$, and $P\red$ if $\exists Q $ such that $ P \red Q$.
We write $P\red$ if $\exists Q $ such that $ P \red Q$ and $P\not\red$, otherwise.

\section{Replication}

As mentioned before, it is known that replication (and hence
recursion) can be implemented in a higher-order process algebra
\cite{SangiorgiWalker}. As our first example of calculation with the
machinery thus far presented we give the construction explicitly in
the {\rhoc}.

\begin{eqnarray}
	D_{x} & := & \prefix{x}{y}{(\binpar{\outputp{x}{y}}{@{y}})} \nonumber\\
	\bangp_{x}{P} & := & \binpar{{x}!\langle{\binpar{D_{x}}{P}}\rangle}{D_{x}} \nonumber
\end{eqnarray}

\begin{eqnarray}
	\bangp_{x}{P} & & \nonumber\\
	=
	& {x}!\langle{(\prefix{x}{y}{(\outputp{x}{y} | @{y})) | P}}\rangle 
	      | \prefix{x}{y}{(\outputp{x}{y} | @{y})} & \nonumber\\
	\red
	& (\outputp{x}{y} | @{y})\substn{\quotep{(\prefix{x}{y}{(@{y} | \outputp{x}{y})) | P}}}{y} & \nonumber\\
	=
	& \outputp{x}{\quotep{(\prefix{x}{y}{(\outputp{x}{y} | @{y})) | P}}}
	  | {(\prefix{x}{y}{(\outputp{x}{y} | @{y})) | P}} & \nonumber\\
	\red
	& \ldots & \nonumber\\
	\red^*
	& P | P | \ldots & \nonumber
\end{eqnarray}

Of course, this encoding, as an implementation, runs away, unfolding
$\bangp{P}$ eagerly. A lazier and more implementable replication
operator, restricted to input-guarded processes, may be obtained as follows.

\begin{eqnarray}
\bangp{\prefix{u}{v}{P}} 
	:= 
	\binpar{\lift{x}{\prefix{u}{v}{(\binpar{D(x)}{P})}}}{D(x)} \nonumber
\end{eqnarray}

\begin{remark}
  Note that the lazier definition still does not deal with summation
  or mixed summation (i.e. sums over input and output). The reader is
  invited to construct definitions of replication that deal with these
  features. 

  Further, the definitions are parameterized in a name, $x$. Can you,
  gentle reader, make a definition that eliminates this parameter and
  guarantees no accidental interaction between the replication
  machinery and the process being replicated -- i.e. no accidental
  sharing of names used by the process to get its work done and the
  name(s) used by the replication to effect copying. This latter
  revision of the definition of replication is crucial to obtaining
  the expected identity $!!P \sim !P$.
\end{remark}

\begin{remark}\label{rem:paradoxical_combinator}
  The reader familiar with the lambda calculus will have noticed the
  similarity between $D$ and the paradoxical combinator.

  [Ed. note: the existence of this seems to suggest we have to be more
  restrictive on the set of processes and names we admit if we are to
  support no-cloning.]
\end{remark}

\subsubsection{Bisimulation}

The computational dynamics gives rise to another kind of equivalence,
the equivalence of computational behavior. As previously mentioned
this is typically captured \emph{via} some form of bisimulation.

% The notion we use in this paper is weak barbed bisimulation
% \cite{milner91polyadicpi}.

The notion we use in this paper is derived from weak barbed
bisimulation \cite{milner91polyadicpi}. 

\begin{definition}
An \emph{observation relation}, $\downarrow_{\mathcal N}$, over a set
of names, $\mathcal N$, is the smallest relation satisfying the rules
below.

\infrule[Out-barb]{y \in {\mathcal N}, \; x \nameeq y}
		  {\outputp{x}{v} \downarrow_{\mathcal N} x}
\infrule[Par-barb]{\mbox{$P\downarrow_{\mathcal N} x$ or $Q\downarrow_{\mathcal N} x$}}
		  {\binpar{P}{Q} \downarrow_{\mathcal N} x}

We write $P \Downarrow_{\mathcal N} x$ if there is $Q$ such that 
$P \wred Q$ and $Q \downarrow_{\mathcal N} x$.
\end{definition}

\begin{definition}
%\label{def.bbisim}
An  ${\mathcal N}$-\emph{barbed bisimulation} over a set of names, ${\mathcal N}$, is a symmetric binary relation 
${\mathcal S}_{\mathcal N}$ between agents such that $P\rel{S}_{\mathcal N}Q$ implies:
\begin{enumerate}
\item If $P \red P'$ then $Q \wred Q'$ and $P'\rel{S}_{\mathcal N} Q'$.
\item If $P\downarrow_{\mathcal N} x$, then $Q\Downarrow_{\mathcal N} x$.
\end{enumerate}
$P$ is ${\mathcal N}$-barbed bisimilar to $Q$, written
$P \wbbisim_{\mathcal N} Q$, if $P \rel{S}_{\mathcal N} Q$ for some ${\mathcal N}$-barbed bisimulation ${\mathcal S}_{\mathcal N}$.
\end{definition}

$\mathcal{R} \subseteq \pi \times \pi$

$P \mathcal{R} Q => \forall P'. P \red P' \Rightarrow \exists Q'. Q \red Q', P' \mathcal{R} Q'$

$P \vdash x \Rightarrow Q \vdash x$

\begin{mathpar}
  \inferrule*[lab=Out-barb]{x \nameeq y}{{y}!\langle{Q}\rangle \vdash x}
  \and
  \inferrule*[lab=Par-barb]{\mbox{$P\vdash x$ or $Q\vdash x$}}{\binpar{P}{Q} \vdash x}
\end{mathpar}

\subsubsection{Contexts}

One of the principle advantages of computational calculi like the
$\pi$-calculus is a well-defined notion of context,
contextual-equivalence and a correlation between
contextual-equivalence and notions of bisimulation. The notion of
context allows the decomposition of a process into (sub-)process and
its syntactic environment, its context. Thus, a context may be
thought of as a process with a ``hole'' (written $\Box$) in it. The
application of a context $M$ to a process $P$, written $M[P]$, is
tantamount to filling the hole in $M$ with $P$. In this paper we do
not need the full weight of this theory, but do make use of the notion
of context in the proof the main theorem. 

\begin{mathpar}
  \inferrule* [lab=summation] {} {{M_{M},M_{N}} \bc \Box \;|\; x.M_{A} \;|\; M_{M}+M_{N}}
  \and
  \inferrule* [lab=agent] {} {{M_{A}} \bc (\vec{x})M_{P} \;| \; \clift{P_0,\ldots,M_{P},\ldots,P_N}}
  \and \\
  \inferrule* [lab=process] {} {{M_{P}} \bc M_{N} \;| \;P|M_{P} }
\end{mathpar} 

\begin{mathpar}
  \inferrule* [lab=sychronization] {} {M_{N} \bc \Box \;|\; x?M_{F} \;|\; x!M_{C}}
  \and
  \inferrule* [lab=abstraction] {} {{M_{F}} \bc (x)M_{P} }
  \and
  \inferrule* [lab=concretion] {} {{M_{C}} \bc \langle M_{P} \rangle }
  \and \\
  \inferrule* [lab=process] {} {{M_{P}} \bc M_{N} \;| \;P|M_{P} }
\end{mathpar}

\begin{definition}[contextual application] Given a context $M$, and
  process $P$, we define the \emph{contextual application}, $M[P] :=
  M\{P/\Box\}$. That is, the contextual application of M to P is the
  substitution of $P$ for $\Box$ in $M$.
\end{definition}

$\meaningof{-} : L \to \mathcal{P}(\pi)$

\begin{mathpar}
  \inferrule* [lab=collection] {} {\meaningof{true} = \pi, \and \meaningof{~E} = \pi \setminus \meaningof{E}, \and \meaningof{E_{1} \& E_{2}} = \meaningof{E_{1}} \cap \meaningof{E_{2}}}
\end{mathpar}

\begin{mathpar}
  \inferrule* [lab=structure] {} {\meaningof{0} = \{ P \in \pi | P \equiv 0 \}, \and \\ \meaningof{E_1 | E_2} = \{ P \in \pi | P \equiv P_{1} | P_{2}, P_{1} \in \meaningof{E_{1}}, P_{2} \in \meaningof{E_2}\} }
\end{mathpar}

\begin{mathpar}
 \inferrule* [lab=behavior] {} {\meaningof{\langle a?b \rangle E} = \{ P \in \pi | P \equiv Q | u?(y)P', \\ \and \\\\ \and \\ \;\;\; u \in \meaningof{a}, \forall z.P'\{z/y\} \in \meaningof{E\{z/b\}}\}, \and \\ \meaningof{a!E} = \{ P \in \pi | P \equiv Q | x!\langle P' \rangle, x \in \meaningof{a} P' \in \meaningof{E}\} }
\end{mathpar}

\begin{mathpar}
 \inferrule* [lab=nominal] {} {\meaningof{\quotep{E}} = \{ \quotep{P} \in \quotep{\pi} | P \in \meaningof{E} \}, \and \meaningof{\quotep{P}} = \{ \quotep{Q} \in \quotep{\pi} | P \equiv Q \} \and \\ \meaningof{@\quotep{E}} = \{ P \in \pi | P \equiv @x, x \in \meaningof{E} \}}
\end{mathpar}

\begin{eqnarray*}
  \\
  \meaningof{-} : TS \to ST
\end{eqnarray*}

\begin{eqnarray*}
  \\
  L : TS \to ST
\end{eqnarray*}

\begin{eqnarray*}
  \\
  P \models E \iff P \in \meaningof{E}
\end{eqnarray*}

\begin{eqnarray*}
  P \approx_{L} Q \iff \forall E \in L. P \models E \iff Q \models E
\end{eqnarray*}

\begin{eqnarray*}
  P \approx_{K} Q
\end{eqnarray*}

\begin{eqnarray*}
  P \approx Q
\end{eqnarray*}

$\approx_{K} = \approx = \approx_{L}$

\subsubsection{Contextual duality}

Note that contexts extend the quotation operation to a family of
operations from processes to names. Given a context, $M$, we can
define a \emph{nominal context}, $\quotep{M}$ by $\quotep{M}[P] :=
\quotep{M[P]}$. To foreshadow what is to come we observe that these
operations enjoy a duality with processes very much like the duality
between vectors and maps from vectors to scalars.

Further, because the calculus is essentially higher-order, we have a
correspondence between contexts and processes. More specifically,
given a name $x$ and a context $M$ we can construct $M^{*}_{x}$ such
that 

\begin{mathpar}
  M^{*}_{x} | \lift{x}{P} \red M[P]
\end{mathpar}

namely,

\begin{mathpar}
  M^{*}_{x} := x?(u).M[\dropn{u}]
\end{mathpar}

The dependence of $M^{*}_{x}$ on a name makes it an abstraction, 

\begin{mathpar}
  M^{*} := (x)x?(u).M[\dropn{u}]
\end{mathpar}

\subsection{Additional notation}

It will sometimes be convenient to denote the process a name
quotes. We already have the notation $x = \quotep{P}$, but it will be
convenient to introduce an alternate notation, $\procn{x}$, when we
want to emphasize the connection to the use of the name. Note that, by
virtue of name equivalence, $\quotep{\procn{x}} \nameeq x$; so, the
notation is consistent with previous definitions.

Further, because names have structure it is possible to effect
substitutions on the basis of that structure. This means we need to
upgrade our notation for substitutions, which we accomplish by
adapting comprehension notation. Thus,

\begin{mathpar}
  P\{ y / x : x \in S \}
\end{mathpar}

is interpreted to mean the process derived from P by replacing (in a
capture-avoiding manner) each occurrence of $x$ in $S$ by $y$. For example,

\begin{mathpar}
  P\{ \quotep{\procn{x}|\procn{x}} / x : x \in \freenames{P} \}
\end{mathpar}

will replace each (occurrence) of a free name $x$ in $P$ by
$\quotep{\procn{x}|\procn{x}}$.

Also, we will avail ourselves of the notation $x^{L}$ and $x^{R}$ to
denote injections of a name into disjoint copies of the name
space. There are numerous ways to accomplish this. One example can be
found in \cite{MeredithR05}. This notation overloads to vectors of
names: $\vec{x}^{\pi} := (x_{i}^{\pi} \; : \; 0 \leq i < |\vec{x}| )$ where $\pi \in \{L,R\}$.

We also use $P^{\Box} := P|\Box$.

In \cite{MeredithR05} an interpretation of the new operator is
given. It turns out that there are several possible interpretations
all enjoying the requisite algebraic properties of the operator (see
\cite{milner91polyadicpi}). We will therefore make liberal use of
$(\nu\; \vec{x})P$.

% subsection the_syntax_and_semantics_of_the_notation_system (end)   

\input{qm2pi.qmops} 

\input{qm2pi.sterngerlach} 

\input{qm2pi.metric} 

% section concurrent_process_calculi (end)

%\input{qm2pi.proofsketch}

% section proof sketch (end)

%\input{qm2pi.slviaknots} 

% section spatial logic via knots (end)

\input{qm2pi.conclusion}

% section conclusion (end)

%\input{qm2pi.dtcodes} 

% section wiring algorithm (end)

\input{qm2pi.ack} 

% section acknowledgments (end)

\newpage


\bibliographystyle{plain}   
\bibliography{../../biblios/main.bib}

\input{qm2pi.rhodetails}

\end{document}

 

% section concurrent_process_calculi (end)

%\documentclass[12pt]{llncs}
%\documentclass{jktr}

\usepackage[pdftex]{hyperref}                   
\usepackage {listings}
\usepackage {mathpartir}
\usepackage{bcprules}
%\usepackage{listings}
                       
\usepackage{graphicx} 
%\usepackage[margins=2.5cm,nohead,nofoot]{geometry}
%\usepackage{geometry}
\usepackage{amsfonts}
\usepackage{amstext}
\usepackage{latexsym}
\usepackage{amssymb}
\usepackage{color}


%\include{myPreamble}
\include{qm2pi.local} 

%\ifpdf
%\usepackage[pdftex]{graphicx}
%\else
%\usepackage{graphicx}
%\fi

 % \ifpdf
%  \usepackage{pdfsync}
%  \if


%\title{Brief Article}
%\author{David F. Snyder}
%\author{L.G. Meredith}

%\address{Dept. of Math., Texas State University--San Marcos, San Marcos, TX 78666}
       
\pagestyle{empty}


\begin{document}

\lstset{language=[Objective]Caml,frame=shadowbox}

\input{qm2pi.front}

% section front matter (end)

\input{qm2pi.intro} 
 
% section introduction (end)

% \input{qm2pi.knotations} 

% section notation (end)

\input{qm2pi.process.calculi} 

% section concurrent_process_calculi_and_spatial_logics_ (end)
    
%\input{qm2pi.knots2pi} 

%\input{qm2pi.trefoil} 

%\input{qm2pi.mainthm} 

% subsection basic_interpretation (end)

%\input{qm2pi.rho.presentation} 
\subsection{The syntax and semantics of the notation system}\label{sub:the_syntax_and_semantics_of_the_notation_system} % (fold)

We now summarize a technical presentation of the calculus that
embodies our theory of dynamics. The typical presentation of such a
calculus follows the style of giving generators and relations on
them. The grammar, below, describing term constructors, freely
generates the set of processes, $\Proc$. This set is then quotiented
by a relation known as structural congruence and it is over this set
that the notion of dynamics is expressed. This presentation is
essentially that of \cite{MeredithR05} with the addition of
polyadicity and summation. For readability we have relegated some of
the technical subtleties to an appendix.

\subsubsection{Process grammar}\label{subsub:process_grammar}

\begin{mathpar}
  \inferrule* [lab=synchronization] {} {{M} \bc \pzero \;|\; x?F \;|\; x!C }
  \and
  \inferrule* [lab=abstraction] {} {{F} \bc (x)P}
  \and
  \inferrule* [lab=concretion] {} {{C} \bc \langle Q \rangle}
  \and
  \inferrule* [lab=process] {} {{P,Q} \bc M \;| \;P|Q \;|\; @{x}}
  \and
  \inferrule* [lab=name] {} {{x} \bc \quotep{P}}
\end{mathpar} 

Note that $\vec{x}$ (resp. $\vec{P}$) denotes a vector of names
(resp. processes) of length $|\vec{x}|$ (resp. $|\vec{P}|$). We adopt
the following useful abbreviations.

\begin{mathpar}
   x?(\vec{y}).P := x.(\vec{y})P \and  x\clift{\vec{P}} := x.\clift{\vec{P}}
   \and x!(y) := \lift{x}{\dropn{y}}
   \and \Pi_{i=0}^{n-1}P_i := P_0 | \ldots | P_{n-1}
\end{mathpar}

\subsubsection{Structural congruence}

\paragraph{Free and bound names and alpha-equivalence.} At the
core of structural equivalence is alpha-equivalence which identifies
process that are the same up to a change of variable. Formally, we
recognize the distinction between free and bound names. The free names
of a process, $\freenames{P}$, may be calculated recursively as
follows:

\begin{mathpar}
\freenames{\pzero} := \emptyset
  \and \\
  \freenames{x?(y).P} := \{ x \} \cup (\freenames{P} \setminus \{ y \})
  \and 
  \freenames{x!\langle P \rangle} := \{ x \} \cup \{ P \} 
  \and \\
  \freenames{P|Q} := \freenames{P} \cup \freenames{Q}
  \and \\
  \freenames{@{x}} := \{ x \}
\end{mathpar}

$\pi$
$\quotep{\pi}$

$\freenames{-} : \pi \to \mathcal{P}(\quotep{\pi})$

\begin{eqnarray*}
  \freenames{\pzero} & := & \emptyset \\
  \freenames{x?(y).P} & := & \{ x \} \cup (\freenames{P} \setminus \{ y \}) \\
  \freenames{x!\langle P \rangle} & := & \{ x \} \cup \{ P \} \\
  \freenames{P|Q} & := & \freenames{P} \cup \freenames{Q} \\
  \freenames{\dropn{x}} & := & \{ x \}
\end{eqnarray*}

The bound names of a process, $\boundnames{P}$, are those names occurring in $P$
that are not free. For example, in $x?(y).0$, the name $x$ is free, while $y$ is bound.

\begin{mathpar}
  \inferrule* [lab=monoidal-laws] {} { P|Q \equiv Q|P \and P|0 \equiv P \and P|(Q|R) \equiv (P|Q)|R }
\end{mathpar}

\begin{mathpar}
  \inferrule* [lab=alpha-equivalence] {} { (x)P \equiv (y)P\{y/x\} \and y \not\in \freenames{P} }
\end{mathpar}

\begin{definition}
Then two processes, $P,Q$, are alpha-equivalent if $P = Q\{\vec{y}/\vec{x}\}$ for
some $\vec{x} \in \boundnames{Q},\vec{y} \in \boundnames{P}$, where $Q\{\vec{y}/\vec{x}\}$
denotes the capture-avoiding substitution of $\vec{y}$ for $\vec{x}$ in $Q$.
\end{definition}

\begin{definition}
  The {\em structural congruence} \cite{SangiorgiWalker} , $\equiv$,
  between processes is the least congruence containing
  alpha-equivalence, satisfying the abelian monoid laws
  (associativity, commutativity and $\pzero$ as identity) for parallel
  composition $|$ and for summation $+$.
\end{definition}

\subsection{Name equivalence}

We take name equivalence, written $\nameeq$, to be the smallest
equivalence relation generated by the following rules.

\begin{mathpar}
\inferrule*[lab=Quote-drop]
{ }
{ \quotep{@{x}} \nameeq x }

\inferrule*[lab=Struct-equiv]
{ P \scong Q }
{ \quotep{P} \nameeq \quotep{Q} }
\end{mathpar}

The astute reader will have noticed that the mutual recursion of names
and processes imposes a mutual recursion on alpha-equivalence and
structural equivalence via name-equivalence. Fortunately, all of this
works out pleasantly and we may calculate in the natural way, free of
concern. The reader interested in the details is referred to the
appendix \ref{appendix:rho_details}.

\subsection{Substitution}

We use $\Proc$ for the set of processes, $\QProc$ for the set of
names, and $\id{\{}\vec{y} / \vec{x} \id{\}}$ to denote partial maps,
$s : \QProc \rightarrow \QProc$. A map, $s$ lifts, uniquely, to a map
on process terms, $\widehat{s} : \Proc \rightarrow \Proc$ by the
following equations.

\begin{mathpar}
  (0) \psubstp{Q}{P} := 0 \\
  (R \juxtap S) \psubstp{Q}{P}
  :=    
  (R)\psubstp{Q}{P} \juxtap (S) \psubstp{Q}{P} \\
  (x?(y).R) \psubstp{Q}{P}    
  :=    
  (x)\substp{Q}{P} (z)\concat( (R \psubstn{z}{y}) \psubstp{Q}{P} ) \\
  (\lift{x}{R}) \psubstp{Q}{P}  
  :=
  \lift{(x)\substp{Q}{P}}{ R \psubstp{Q}{P} } \\
%   (\dropn{x})  \psubstp{Q}{P}       
%   := 
%   \left\{ 
%     \begin{array}{ccc} 
%       \dropn{\quotep{Q}} & & x \nameeq \quotep{P} \\
%       \dropn{x} & & otherwise \\
%     \end{array}
%   \right. 
  (\dropn{x})  \psubstp{Q}{P}       
  := 
  \left\{ 
    \begin{array}{ccc} 
      Q & & x \nameeq \quotep{P} \\
      \dropn{x} & & otherwise \\
    \end{array}
  \right.
\end{mathpar}
 

where

\begin{eqnarray}
  (x)\id{\{} \lpquote Q \rpquote / \lpquote P \rpquote \id{\}}            = 
  \left\{ 
    \begin{array}{ccc}
      \lpquote Q \rpquote & & x \nameeq \lpquote P \rpquote \\
      x & & otherwise \\
    \end{array}
  \right. \nonumber
\end{eqnarray}

and $z$ is chosen distinct from $\quotep{P}$, $\quotep{Q}$, the free
names in $Q$, and all the names in $R$. Our $\alpha$-equivalence will
be built in the standard way from this substitution.

\begin{remark}\label{rem:no_self_referential_names}
  One consequence of these definitions is that $\forall P. \quotep{P}
  \not\in \freenames{P}$.
\end{remark}

\subsection{ Dynamic quote: an example }

Anticipating something of what's to come, consider applying the
substitution, $\widehat{\id{\{}u / z \id{\}}}$, to the following pair
of processes, $\lift{w}{y!(z)}$ and $w[ \lpquote y!(z) \rpquote ]$.

\begin{eqnarray}
	\lift{w}{y!(z)}\widehat{\id{\{}u / z \id{\}}}
		& = &
		\lift{w}{y!(u)} \nonumber\\
	w[ \lpquote y!(z) \rpquote ] \widehat{ \id{\{}u / z \id{\}} }
		& = &
		w[ \lpquote y!(z) \rpquote ] \nonumber
\end{eqnarray}

Because the body of the process between quotes is impervious to
substitution, we get radically different answers. In fact, by
examining the first process in an input context,
e.g. $x?(z).\lift{w}{y!(z)}$, we see that the process under the lift
operator may be shaped by prefixed inputs binding a name inside it. In
this sense, the lift operator will be seen as a way to dynamically
construct processes before reifying them as names.

Finally equipped with these standard features we can present the
dynamics of the calculus.

\subsubsection{Operational semantics} 

Finally, we introduce the computational dynamics. What marks these
algebras as distinct from other more traditionally studied algebraic
structures, e.g. vector spaces or polynomial rings, is the manner in
which dynamics is captured. In traditional structures, dynamics is typically
expressed through morphisms between such structures, as in linear maps
between vector spaces or morphisms between rings. In algebras
associated with the semantics of computation, the dynamics is
expressed as part of the algebraic structure itself, through a
reduction reduction relation typically denoted by $\red$. Below, we
give a recursive presentation of this relation for the calculus used
in the encoding.

$\red \subseteq \pi \times \pi$
$\red : \pi \to \mathcal{P}(\pi)$

\begin{mathpar}
  \inferrule* [lab=Comm] { \textsf{match}( x_{src}, x_{trgt} ) } { x_{trgt}?(y)P \; | \; x_{src}!\langle {Q} \rangle \red P\{\quotep{Q}/y}\} }
  \and \\
  \inferrule* [lab=Par] {{P} \red {P}'} {{{P} | {Q}} \red {{P}' | {Q}}}
  \and
  \inferrule* [lab=Equiv]{{{P} \scong {P}'} \andalso {{P}' \red {Q}'} \andalso {{Q}' \scong {Q}}}{{P} \red {Q}}
\end{mathpar}

\begin{eqnarray*}
  match_{\equiv} (\quotep{P},\quotep{Q}) & := & P \equiv Q \\
  match_{\dagger}(\quotep{P},\quotep{Q}) & := & \forall R. P|Q \red^{*} R => R \red^{*} 0 \\
  match_{K}(\quotep{P},\quotep{Q}) & := & K \mbox{ for some context } K
\end{eqnarray*}

$u?(x)P | u!\langle Q \rangle \red P\{\quotep{Q}/x\}$

%We write $\wred$ for $\red^*$, and $P\red$ if $\exists Q $ such that $ P \red Q$.
We write $P\red$ if $\exists Q $ such that $ P \red Q$ and $P\not\red$, otherwise.

\section{Replication}

As mentioned before, it is known that replication (and hence
recursion) can be implemented in a higher-order process algebra
\cite{SangiorgiWalker}. As our first example of calculation with the
machinery thus far presented we give the construction explicitly in
the {\rhoc}.

\begin{eqnarray}
	D_{x} & := & \prefix{x}{y}{(\binpar{\outputp{x}{y}}{@{y}})} \nonumber\\
	\bangp_{x}{P} & := & \binpar{{x}!\langle{\binpar{D_{x}}{P}}\rangle}{D_{x}} \nonumber
\end{eqnarray}

\begin{eqnarray}
	\bangp_{x}{P} & & \nonumber\\
	=
	& {x}!\langle{(\prefix{x}{y}{(\outputp{x}{y} | @{y})) | P}}\rangle 
	      | \prefix{x}{y}{(\outputp{x}{y} | @{y})} & \nonumber\\
	\red
	& (\outputp{x}{y} | @{y})\substn{\quotep{(\prefix{x}{y}{(@{y} | \outputp{x}{y})) | P}}}{y} & \nonumber\\
	=
	& \outputp{x}{\quotep{(\prefix{x}{y}{(\outputp{x}{y} | @{y})) | P}}}
	  | {(\prefix{x}{y}{(\outputp{x}{y} | @{y})) | P}} & \nonumber\\
	\red
	& \ldots & \nonumber\\
	\red^*
	& P | P | \ldots & \nonumber
\end{eqnarray}

Of course, this encoding, as an implementation, runs away, unfolding
$\bangp{P}$ eagerly. A lazier and more implementable replication
operator, restricted to input-guarded processes, may be obtained as follows.

\begin{eqnarray}
\bangp{\prefix{u}{v}{P}} 
	:= 
	\binpar{\lift{x}{\prefix{u}{v}{(\binpar{D(x)}{P})}}}{D(x)} \nonumber
\end{eqnarray}

\begin{remark}
  Note that the lazier definition still does not deal with summation
  or mixed summation (i.e. sums over input and output). The reader is
  invited to construct definitions of replication that deal with these
  features. 

  Further, the definitions are parameterized in a name, $x$. Can you,
  gentle reader, make a definition that eliminates this parameter and
  guarantees no accidental interaction between the replication
  machinery and the process being replicated -- i.e. no accidental
  sharing of names used by the process to get its work done and the
  name(s) used by the replication to effect copying. This latter
  revision of the definition of replication is crucial to obtaining
  the expected identity $!!P \sim !P$.
\end{remark}

\begin{remark}\label{rem:paradoxical_combinator}
  The reader familiar with the lambda calculus will have noticed the
  similarity between $D$ and the paradoxical combinator.

  [Ed. note: the existence of this seems to suggest we have to be more
  restrictive on the set of processes and names we admit if we are to
  support no-cloning.]
\end{remark}

\subsubsection{Bisimulation}

The computational dynamics gives rise to another kind of equivalence,
the equivalence of computational behavior. As previously mentioned
this is typically captured \emph{via} some form of bisimulation.

% The notion we use in this paper is weak barbed bisimulation
% \cite{milner91polyadicpi}.

The notion we use in this paper is derived from weak barbed
bisimulation \cite{milner91polyadicpi}. 

\begin{definition}
An \emph{observation relation}, $\downarrow_{\mathcal N}$, over a set
of names, $\mathcal N$, is the smallest relation satisfying the rules
below.

\infrule[Out-barb]{y \in {\mathcal N}, \; x \nameeq y}
		  {\outputp{x}{v} \downarrow_{\mathcal N} x}
\infrule[Par-barb]{\mbox{$P\downarrow_{\mathcal N} x$ or $Q\downarrow_{\mathcal N} x$}}
		  {\binpar{P}{Q} \downarrow_{\mathcal N} x}

We write $P \Downarrow_{\mathcal N} x$ if there is $Q$ such that 
$P \wred Q$ and $Q \downarrow_{\mathcal N} x$.
\end{definition}

\begin{definition}
%\label{def.bbisim}
An  ${\mathcal N}$-\emph{barbed bisimulation} over a set of names, ${\mathcal N}$, is a symmetric binary relation 
${\mathcal S}_{\mathcal N}$ between agents such that $P\rel{S}_{\mathcal N}Q$ implies:
\begin{enumerate}
\item If $P \red P'$ then $Q \wred Q'$ and $P'\rel{S}_{\mathcal N} Q'$.
\item If $P\downarrow_{\mathcal N} x$, then $Q\Downarrow_{\mathcal N} x$.
\end{enumerate}
$P$ is ${\mathcal N}$-barbed bisimilar to $Q$, written
$P \wbbisim_{\mathcal N} Q$, if $P \rel{S}_{\mathcal N} Q$ for some ${\mathcal N}$-barbed bisimulation ${\mathcal S}_{\mathcal N}$.
\end{definition}

$\mathcal{R} \subseteq \pi \times \pi$

$P \mathcal{R} Q => \forall P'. P \red P' \Rightarrow \exists Q'. Q \red Q', P' \mathcal{R} Q'$

$P \vdash x \Rightarrow Q \vdash x$

\begin{mathpar}
  \inferrule*[lab=Out-barb]{x \nameeq y}{{y}!\langle{Q}\rangle \vdash x}
  \and
  \inferrule*[lab=Par-barb]{\mbox{$P\vdash x$ or $Q\vdash x$}}{\binpar{P}{Q} \vdash x}
\end{mathpar}

\subsubsection{Contexts}

One of the principle advantages of computational calculi like the
$\pi$-calculus is a well-defined notion of context,
contextual-equivalence and a correlation between
contextual-equivalence and notions of bisimulation. The notion of
context allows the decomposition of a process into (sub-)process and
its syntactic environment, its context. Thus, a context may be
thought of as a process with a ``hole'' (written $\Box$) in it. The
application of a context $M$ to a process $P$, written $M[P]$, is
tantamount to filling the hole in $M$ with $P$. In this paper we do
not need the full weight of this theory, but do make use of the notion
of context in the proof the main theorem. 

\begin{mathpar}
  \inferrule* [lab=summation] {} {{M_{M},M_{N}} \bc \Box \;|\; x.M_{A} \;|\; M_{M}+M_{N}}
  \and
  \inferrule* [lab=agent] {} {{M_{A}} \bc (\vec{x})M_{P} \;| \; \clift{P_0,\ldots,M_{P},\ldots,P_N}}
  \and \\
  \inferrule* [lab=process] {} {{M_{P}} \bc M_{N} \;| \;P|M_{P} }
\end{mathpar} 

\begin{mathpar}
  \inferrule* [lab=sychronization] {} {M_{N} \bc \Box \;|\; x?M_{F} \;|\; x!M_{C}}
  \and
  \inferrule* [lab=abstraction] {} {{M_{F}} \bc (x)M_{P} }
  \and
  \inferrule* [lab=concretion] {} {{M_{C}} \bc \langle M_{P} \rangle }
  \and \\
  \inferrule* [lab=process] {} {{M_{P}} \bc M_{N} \;| \;P|M_{P} }
\end{mathpar}

\begin{definition}[contextual application] Given a context $M$, and
  process $P$, we define the \emph{contextual application}, $M[P] :=
  M\{P/\Box\}$. That is, the contextual application of M to P is the
  substitution of $P$ for $\Box$ in $M$.
\end{definition}

$\meaningof{-} : L \to \mathcal{P}(\pi)$

\begin{mathpar}
  \inferrule* [lab=collection] {} {\meaningof{true} = \pi, \and \meaningof{~E} = \pi \setminus \meaningof{E}, \and \meaningof{E_{1} \& E_{2}} = \meaningof{E_{1}} \cap \meaningof{E_{2}}}
\end{mathpar}

\begin{mathpar}
  \inferrule* [lab=structure] {} {\meaningof{0} = \{ P \in \pi | P \equiv 0 \}, \and \\ \meaningof{E_1 | E_2} = \{ P \in \pi | P \equiv P_{1} | P_{2}, P_{1} \in \meaningof{E_{1}}, P_{2} \in \meaningof{E_2}\} }
\end{mathpar}

\begin{mathpar}
 \inferrule* [lab=behavior] {} {\meaningof{\langle a?b \rangle E} = \{ P \in \pi | P \equiv Q | u?(y)P', \\ \and \\\\ \and \\ \;\;\; u \in \meaningof{a}, \forall z.P'\{z/y\} \in \meaningof{E\{z/b\}}\}, \and \\ \meaningof{a!E} = \{ P \in \pi | P \equiv Q | x!\langle P' \rangle, x \in \meaningof{a} P' \in \meaningof{E}\} }
\end{mathpar}

\begin{mathpar}
 \inferrule* [lab=nominal] {} {\meaningof{\quotep{E}} = \{ \quotep{P} \in \quotep{\pi} | P \in \meaningof{E} \}, \and \meaningof{\quotep{P}} = \{ \quotep{Q} \in \quotep{\pi} | P \equiv Q \} \and \\ \meaningof{@\quotep{E}} = \{ P \in \pi | P \equiv @x, x \in \meaningof{E} \}}
\end{mathpar}

\begin{eqnarray*}
  \\
  \meaningof{-} : TS \to ST
\end{eqnarray*}

\begin{eqnarray*}
  \\
  L : TS \to ST
\end{eqnarray*}

\begin{eqnarray*}
  \\
  P \models E \iff P \in \meaningof{E}
\end{eqnarray*}

\begin{eqnarray*}
  P \approx_{L} Q \iff \forall E \in L. P \models E \iff Q \models E
\end{eqnarray*}

\begin{eqnarray*}
  P \approx_{K} Q
\end{eqnarray*}

\begin{eqnarray*}
  P \approx Q
\end{eqnarray*}

$\approx_{K} = \approx = \approx_{L}$

\subsubsection{Contextual duality}

Note that contexts extend the quotation operation to a family of
operations from processes to names. Given a context, $M$, we can
define a \emph{nominal context}, $\quotep{M}$ by $\quotep{M}[P] :=
\quotep{M[P]}$. To foreshadow what is to come we observe that these
operations enjoy a duality with processes very much like the duality
between vectors and maps from vectors to scalars.

Further, because the calculus is essentially higher-order, we have a
correspondence between contexts and processes. More specifically,
given a name $x$ and a context $M$ we can construct $M^{*}_{x}$ such
that 

\begin{mathpar}
  M^{*}_{x} | \lift{x}{P} \red M[P]
\end{mathpar}

namely,

\begin{mathpar}
  M^{*}_{x} := x?(u).M[\dropn{u}]
\end{mathpar}

The dependence of $M^{*}_{x}$ on a name makes it an abstraction, 

\begin{mathpar}
  M^{*} := (x)x?(u).M[\dropn{u}]
\end{mathpar}

\subsection{Additional notation}

It will sometimes be convenient to denote the process a name
quotes. We already have the notation $x = \quotep{P}$, but it will be
convenient to introduce an alternate notation, $\procn{x}$, when we
want to emphasize the connection to the use of the name. Note that, by
virtue of name equivalence, $\quotep{\procn{x}} \nameeq x$; so, the
notation is consistent with previous definitions.

Further, because names have structure it is possible to effect
substitutions on the basis of that structure. This means we need to
upgrade our notation for substitutions, which we accomplish by
adapting comprehension notation. Thus,

\begin{mathpar}
  P\{ y / x : x \in S \}
\end{mathpar}

is interpreted to mean the process derived from P by replacing (in a
capture-avoiding manner) each occurrence of $x$ in $S$ by $y$. For example,

\begin{mathpar}
  P\{ \quotep{\procn{x}|\procn{x}} / x : x \in \freenames{P} \}
\end{mathpar}

will replace each (occurrence) of a free name $x$ in $P$ by
$\quotep{\procn{x}|\procn{x}}$.

Also, we will avail ourselves of the notation $x^{L}$ and $x^{R}$ to
denote injections of a name into disjoint copies of the name
space. There are numerous ways to accomplish this. One example can be
found in \cite{MeredithR05}. This notation overloads to vectors of
names: $\vec{x}^{\pi} := (x_{i}^{\pi} \; : \; 0 \leq i < |\vec{x}| )$ where $\pi \in \{L,R\}$.

We also use $P^{\Box} := P|\Box$.

In \cite{MeredithR05} an interpretation of the new operator is
given. It turns out that there are several possible interpretations
all enjoying the requisite algebraic properties of the operator (see
\cite{milner91polyadicpi}). We will therefore make liberal use of
$(\nu\; \vec{x})P$.

% subsection the_syntax_and_semantics_of_the_notation_system (end)   

\input{qm2pi.qmops} 

\input{qm2pi.sterngerlach} 

\input{qm2pi.metric} 

% section concurrent_process_calculi (end)

%\input{qm2pi.proofsketch}

% section proof sketch (end)

%\input{qm2pi.slviaknots} 

% section spatial logic via knots (end)

\input{qm2pi.conclusion}

% section conclusion (end)

%\input{qm2pi.dtcodes} 

% section wiring algorithm (end)

\input{qm2pi.ack} 

% section acknowledgments (end)

\newpage


\bibliographystyle{plain}   
\bibliography{../../biblios/main.bib}

\input{qm2pi.rhodetails}

\end{document}



% section proof sketch (end)

%\section{Unlikely characters: spatial logic for
  knots}\label{sub:characteristic_formulae} % (fold)

Associated to the mobile process calculi are a family of logics known
as the Hennessy-Milner logics. These logics typically enjoy a
semantics interpreting formulae as sets of processes that when
factored through the encoding outlined above allows an identification
of classes of knots with logical formulae. In the context of this
encoding the sub-family known as the spatial logics \cite{CairesC03}
\cite{CairesC04} \cite{Caires04} are of particular interest providing
several important features for expressing and reasoning about
properties (i.e. classes) of knots. We hint here at how this may be done.

%\begin{description}
%\item [structural connectives] 
\subsubsection{Structural connectives} The spatial logics enjoy
structural connectives corresponding, at the logical level, to the
parallel composition ($P | Q$) and new name ($(\nu \; x)P$)
connectives for processes. As illustrated in the examples below, these
connectives are extremely expressive given the shape of our encoding.
%\item [decideable satisfaction]

\subsubsection{Decideable satisfaction}
In \cite{Caires04} the satisfaction relation is shown to be decideable
for a rich class of processes. It further turns out that the image of
the our encoding is a proper subset of that class. This result
provides the basis for an algorithm by which to search for knots
enjoying a given property.
%\item [characteristic formulae]

\subsubsection{Characteristic formulae}
In the same paper \cite{Caires04} , Caires presents a means of calculating
characteristic formulae, selecting equivalence classes of processes
up to a pre--specified depth limit on the support set of names. Composed with our
encoding, this characteristic formula can be used to select
characteristic formulae for knots.
%\end{description}

\subsubsection{Spatial logic formulae}

The grammar below (segmented for comprehension) summarizes the syntax
of spatial logic formulae. We employ illustrative examples in the
sequel to provide an intuitive understanding of their meaning
referring the reader to \cite{Caires04} for a more detailed explication
of the semantics.

\begin{mathpar}
  \inferrule* [lab=boolean] {} {{A,B} \bc T \;|\; \neg A \;|\; A \wedge B \;|\; \eta = \eta'}
  \and
  \inferrule* [lab=spatial] {} {|\; \pzero \;|\; A | B \;|\; x \text{\textregistered} A \;|\; \forall x . A \;|\;  H x . A}
  \and
  \inferrule* [lab=behavioral] {} {|\; \alpha . A}
  \and 
  \inferrule* [lab=recursion] {} {|\; X(\vec{u}) \;|\; \mu X(\vec{u}) . A}
  \and
  \inferrule* [lab=action] {} {\alpha \bc \langle x?(\vec{y}) \rangle \;|\; \langle x!(\vec{y}) \rangle \;|\; \langle \tau \rangle}
  \and 
  \inferrule* [lab=name] {} {\eta \bc x \;|\; \tau}
\end{mathpar} 

% subsection characteristic_formulae (end)   	 

\subsection{Example formulae}\label{sub:example_formulae_} % (fold)

\subsubsection{Crossing as formula.}
% 
% \begin{align*}
%   \frac{d}{dx} \sin x &= \cos x 
%   & \frac{d}{dx} e^x &= e^x \\
%   \frac{d}{dx} \cos x &= - \sin x 
%   & \frac{d}{dx} \log x &= \frac{1}{x} \\
% \end{align*} 

\begin{align*}
 \mu C(x_{0},x_{1},y_{0},y_{1},u).&(\langle x_{0}?(z) \rangle(\langle u! \rangle\langle y_{1}!z \rangle C(x_{0},x_{1},y_{0},y_{1},u)) & \\
  & \wedge \langle y_{1}?(z) \rangle (\langle u! \rangle \langle x_{0}!z \rangle C(x_{0},x_{1},y_{0},y_{1},u)) & \\
  & \wedge \langle x_{1}?(z) \rangle (\langle u? \rangle \langle y_{0}!z \rangle C(x_{0},x_{1},y_{0},y_{1},u)) & \\
  & \wedge \langle y_{0}?(z) \rangle (\langle u? \rangle \langle x_{1}!z \rangle C(x_{0},x_{1},y_{0},y_{1},u))) &
\end{align*}

The lexicographical similarity between the shape of this formulae and
the shape of definition of the process representing a crossing reveals
the intuitive meaning of this formulae. It describes the capabilities
of a process that has the right to represent a crossing. For example
it picks out processes that may perform an input on the port $x_0$ in
its initial menu of capabilities. What differentiates the formula
from the process, however, is that the crossing process is the
smallest candidate to satisfy the formula. Infinitely many other
processes -- with internal behavior hidden behind this interface, so
to speak -- also satisfy this formula. Even this simple formula,
then, can be seen to open a new view onto knots, providing a
computational interpretation of \emph{virtual} knots.

Note that this formula is derived by hand. A similar formula can be
derived by employing Caires' calculation of characteristic formula
\cite{Caires04} to the process representing a crossing. In light of
this discussion, we let
$\meaningof{C}_{\phi}(x0,x1,y0,y1,u)$ denote a formula specifying the
dynamics we wish to capture of a crossing. To guarantee we preserve
the shape of the interface and minimal semantics we demand that
$\meaningof{C}_{\phi}(x0,x1,y0,y1,u) \Rightarrow
\textbf{C}(x0,x1,y0,y1,u)$ where $\textbf{C}(x0,x1,y0,y1,u)$ denotes
the formula above.
                            
\subsubsection{Crossing number constraints.}
The moral content of the context lemma (Lemma \ref{context}) is that the notion of
``locality'' in the Reidemeister moves is effectively captured by the
parallel composition operator of the process calculus. This intuition
extends through the logic. Given a formula,
$\meaningof{C}_{\phi}(x0,x1,y0,y1,u)$, we can use the structural
connectives to specify constraints on crossing numbers, such as at
least $n$ crossings, or exactly $n$ crossings.
\begin{mathpar}
  \inferrule* [lab=at-least-n] {} { K^{\geq n}_{\phi}(\vec{xs},\vec{ys}) := \Pi_{i=0}^{n-1} Hu . \meaningof{C}_{\phi}(xs_i,ys_i,u) | T }
  \and 
  \inferrule* [lab=exactly-n] {} { K^{= n}_{\phi}(\vec{xs},\vec{ys}) := \Pi_{i=0}^{n-1} Hu . \meaningof{C}_{\phi}(xs_i,ys_i,u) | \neg (\forall x_0,y_0,x_1,y_1,u . \meaningof{C}_{\phi}(x_0,y_0,x_1,y_1,u) | T) }
\end{mathpar}

To round out this section, recall that the encoding of an $n$-crossing
knot decomposes into a parallel composition of $n$ \emph{copies} of a
crossing process together with a wiring harness. To specify different
knot classes with the same crossing number amounts to specifying
logical constraints on the wiring harness. In the interest of space,
we defer examples to a forthcoming paper. Suffice it to say that both
the conditions ``alternating knot'' and ``contains the tangle
corresponding to 5/3'' are expressible. For example, it is possible to
calculate the characteristic formula of a process corresponding to the
tangle 5/3 and conjoin it into the classifying formula via the
composition connective of the logic.

Finally, we wish to observe that it is entirely within reason to
contemplate a more domain-specific version of spatial logic tailored
to the shape of processes in the image of the encoding. Such a
domain-specific logic would have a better claim to the title formal
language of knot properties.

% subsection example_formulae_ (end)

% section knots_as_processes (end) 

% section spatial logic via knots (end)

\section{Conclusions and future work}

\paragraph{Testing physical space}
You, gentle reader, may wonder why of all the theorems to be proved
given this set up we pick the one above. In some sense it's hardly
central to quantum mechanics. We see it as central in the sense that
it firmly establishes a notion of physical space arising from a notion
of the equivalence of behavior. Relating bisimulation to a metric is a
big step forward, but one is faced with interpreting the relationship
of that metric space to something more physical. Quantum mechanical
notions of ``physical'' space are still far from intuitive, but by
relating this idea of distance as testing to calculations that predict
physical circumstances we are making a not insignificant step forward
toward an understanding of the physical space we inhabit as
essentially dynamic.

\paragraph{Effectivity and simulation}
One of the observations we have yet to make is that the entire program
spelled out here is effective. We have built various interpreters for
the reflective calculus at work in this interpretation. In principle,
then, we can simulate quantum mechanics on a computer. The place where
the simulation may lose fidelity is the infinitely branching summation
for the annihilator.

In this connection i also want to point out that the evaluation style
calculation of the inner product puts the non-determinism of the
summation right at the heart of measurement. This suggests that
Milner's original reduction-based formulation of the dynamics of his
calculi in terms of sums was not just notationally suggestive of a
notion of measure-and-continue but captured some significant part of
the physics.

\paragraph{Quantum continuations}
In light of this last observation i want to point out that the
predominant account of quantum mechanics is missing a key aspect of a
truly compositional story of the physical situation. In a real lab,
when a measurement is made the observation can be made to feed into
another device that then makes another measurement conditioned on the
results of the first. This means that after the superposition was
collapsed the entire experimental set up remained in
superposition. While QM offers a means of writing this down it doesn't
quite line up well with the well-trodden formulation of computation
and continuation that we see so succinctly expressed in Milner's
calculi. This suggests that there might be advantages to this account
of dynamics waiting to be explored.

\paragraph{Quantum logic}
In this connection, we also note that by virtue of having the
Hennessy-Milner construction, we can pull the construction through the
interpretation of QM. This gives us a natural candidate for a quantum
logic that enjoys an extremely tight connection with it's domain of
interpretation, making the construction much less ad hoc (rather it is
the image of functor!).

\paragraph{Quantum probabiity}
i have questions about the basis of the interpretation of inner
product as probability amplitude. In particular, using which
axiomatization of probability theory does the notion of probability
amplitude earn the right to be so dubbed? In other words, where is the
proof that the operation for calculating a probability amplitude (and
then squaring) satisfies the axioms of what it means to calculate a
probability? Even if such a proof exists (i have yet to find it in the
literature), i wonder if it might not be possible to turn things on
their heads. Can we view the calculation of the probability amplitude
as an axiomatization of probability? If so, then the definition we
give for calculating probability amplitude may provide the basis for
an \emph{effective} theory of probability.

\paragraph{Quantum vs ``biological'' information}
Finally, i want to conclude with a more philosophical observation. At
a recent workshop in which QM was a predominant topic i noticed
something about quantum information. The speaker was giving a riveting
discussion of axiomatic QM and showing how properties of ``no
cloning'' and ``no deleting'' emerged as consequences of the
axiomatization. Theorems of this form are necessary to give us a sense
of confidence that our axioms characterize the physical theory. What
struck me, though, was that if quantum information is neither erasable
nor replicable it is markedly different from \emph{life}. Two of the
things we know about life is that

\begin{itemize}
  \item it ends;
  \item to gain some measure of persistence, to transcend it's
    finitude it is imminently copyable.
\end{itemize}

Both of these qualities are summarized succinctly in the aphorism: all
flesh is grass. For me these two kinds of ``information'' -- call them
quantum and biological -- are end points on a spectrum of strategies
for persistence. At one end, we have those curious entities that enjoy
uniqueness and permanence; at the other, we have those who in the face
of a certain end and an uncertain present make a go of passing
something on. To me one of the more remarkable aspects of the latter
strategy is that in the presence of noise (and certain features of
copying) we get a kind of dynamism, a chance for improvement against a
given persistent condition.

% subsection other_calculi_other_bisimulations_and_geometry_as_behavior (end)




% section conclusion (end)

%\documentclass[12pt]{llncs}
%\documentclass{jktr}

\usepackage[pdftex]{hyperref}                   
\usepackage {listings}
\usepackage {mathpartir}
\usepackage{bcprules}
%\usepackage{listings}
                       
\usepackage{graphicx} 
%\usepackage[margins=2.5cm,nohead,nofoot]{geometry}
%\usepackage{geometry}
\usepackage{amsfonts}
\usepackage{amstext}
\usepackage{latexsym}
\usepackage{amssymb}
\usepackage{color}


%\include{myPreamble}
\include{qm2pi.local} 

%\ifpdf
%\usepackage[pdftex]{graphicx}
%\else
%\usepackage{graphicx}
%\fi

 % \ifpdf
%  \usepackage{pdfsync}
%  \if


%\title{Brief Article}
%\author{David F. Snyder}
%\author{L.G. Meredith}

%\address{Dept. of Math., Texas State University--San Marcos, San Marcos, TX 78666}
       
\pagestyle{empty}


\begin{document}

\lstset{language=[Objective]Caml,frame=shadowbox}

\input{qm2pi.front}

% section front matter (end)

\input{qm2pi.intro} 
 
% section introduction (end)

% \input{qm2pi.knotations} 

% section notation (end)

\input{qm2pi.process.calculi} 

% section concurrent_process_calculi_and_spatial_logics_ (end)
    
%\input{qm2pi.knots2pi} 

%\input{qm2pi.trefoil} 

%\input{qm2pi.mainthm} 

% subsection basic_interpretation (end)

%\input{qm2pi.rho.presentation} 
\subsection{The syntax and semantics of the notation system}\label{sub:the_syntax_and_semantics_of_the_notation_system} % (fold)

We now summarize a technical presentation of the calculus that
embodies our theory of dynamics. The typical presentation of such a
calculus follows the style of giving generators and relations on
them. The grammar, below, describing term constructors, freely
generates the set of processes, $\Proc$. This set is then quotiented
by a relation known as structural congruence and it is over this set
that the notion of dynamics is expressed. This presentation is
essentially that of \cite{MeredithR05} with the addition of
polyadicity and summation. For readability we have relegated some of
the technical subtleties to an appendix.

\subsubsection{Process grammar}\label{subsub:process_grammar}

\begin{mathpar}
  \inferrule* [lab=synchronization] {} {{M} \bc \pzero \;|\; x?F \;|\; x!C }
  \and
  \inferrule* [lab=abstraction] {} {{F} \bc (x)P}
  \and
  \inferrule* [lab=concretion] {} {{C} \bc \langle Q \rangle}
  \and
  \inferrule* [lab=process] {} {{P,Q} \bc M \;| \;P|Q \;|\; @{x}}
  \and
  \inferrule* [lab=name] {} {{x} \bc \quotep{P}}
\end{mathpar} 

Note that $\vec{x}$ (resp. $\vec{P}$) denotes a vector of names
(resp. processes) of length $|\vec{x}|$ (resp. $|\vec{P}|$). We adopt
the following useful abbreviations.

\begin{mathpar}
   x?(\vec{y}).P := x.(\vec{y})P \and  x\clift{\vec{P}} := x.\clift{\vec{P}}
   \and x!(y) := \lift{x}{\dropn{y}}
   \and \Pi_{i=0}^{n-1}P_i := P_0 | \ldots | P_{n-1}
\end{mathpar}

\subsubsection{Structural congruence}

\paragraph{Free and bound names and alpha-equivalence.} At the
core of structural equivalence is alpha-equivalence which identifies
process that are the same up to a change of variable. Formally, we
recognize the distinction between free and bound names. The free names
of a process, $\freenames{P}$, may be calculated recursively as
follows:

\begin{mathpar}
\freenames{\pzero} := \emptyset
  \and \\
  \freenames{x?(y).P} := \{ x \} \cup (\freenames{P} \setminus \{ y \})
  \and 
  \freenames{x!\langle P \rangle} := \{ x \} \cup \{ P \} 
  \and \\
  \freenames{P|Q} := \freenames{P} \cup \freenames{Q}
  \and \\
  \freenames{@{x}} := \{ x \}
\end{mathpar}

$\pi$
$\quotep{\pi}$

$\freenames{-} : \pi \to \mathcal{P}(\quotep{\pi})$

\begin{eqnarray*}
  \freenames{\pzero} & := & \emptyset \\
  \freenames{x?(y).P} & := & \{ x \} \cup (\freenames{P} \setminus \{ y \}) \\
  \freenames{x!\langle P \rangle} & := & \{ x \} \cup \{ P \} \\
  \freenames{P|Q} & := & \freenames{P} \cup \freenames{Q} \\
  \freenames{\dropn{x}} & := & \{ x \}
\end{eqnarray*}

The bound names of a process, $\boundnames{P}$, are those names occurring in $P$
that are not free. For example, in $x?(y).0$, the name $x$ is free, while $y$ is bound.

\begin{mathpar}
  \inferrule* [lab=monoidal-laws] {} { P|Q \equiv Q|P \and P|0 \equiv P \and P|(Q|R) \equiv (P|Q)|R }
\end{mathpar}

\begin{mathpar}
  \inferrule* [lab=alpha-equivalence] {} { (x)P \equiv (y)P\{y/x\} \and y \not\in \freenames{P} }
\end{mathpar}

\begin{definition}
Then two processes, $P,Q$, are alpha-equivalent if $P = Q\{\vec{y}/\vec{x}\}$ for
some $\vec{x} \in \boundnames{Q},\vec{y} \in \boundnames{P}$, where $Q\{\vec{y}/\vec{x}\}$
denotes the capture-avoiding substitution of $\vec{y}$ for $\vec{x}$ in $Q$.
\end{definition}

\begin{definition}
  The {\em structural congruence} \cite{SangiorgiWalker} , $\equiv$,
  between processes is the least congruence containing
  alpha-equivalence, satisfying the abelian monoid laws
  (associativity, commutativity and $\pzero$ as identity) for parallel
  composition $|$ and for summation $+$.
\end{definition}

\subsection{Name equivalence}

We take name equivalence, written $\nameeq$, to be the smallest
equivalence relation generated by the following rules.

\begin{mathpar}
\inferrule*[lab=Quote-drop]
{ }
{ \quotep{@{x}} \nameeq x }

\inferrule*[lab=Struct-equiv]
{ P \scong Q }
{ \quotep{P} \nameeq \quotep{Q} }
\end{mathpar}

The astute reader will have noticed that the mutual recursion of names
and processes imposes a mutual recursion on alpha-equivalence and
structural equivalence via name-equivalence. Fortunately, all of this
works out pleasantly and we may calculate in the natural way, free of
concern. The reader interested in the details is referred to the
appendix \ref{appendix:rho_details}.

\subsection{Substitution}

We use $\Proc$ for the set of processes, $\QProc$ for the set of
names, and $\id{\{}\vec{y} / \vec{x} \id{\}}$ to denote partial maps,
$s : \QProc \rightarrow \QProc$. A map, $s$ lifts, uniquely, to a map
on process terms, $\widehat{s} : \Proc \rightarrow \Proc$ by the
following equations.

\begin{mathpar}
  (0) \psubstp{Q}{P} := 0 \\
  (R \juxtap S) \psubstp{Q}{P}
  :=    
  (R)\psubstp{Q}{P} \juxtap (S) \psubstp{Q}{P} \\
  (x?(y).R) \psubstp{Q}{P}    
  :=    
  (x)\substp{Q}{P} (z)\concat( (R \psubstn{z}{y}) \psubstp{Q}{P} ) \\
  (\lift{x}{R}) \psubstp{Q}{P}  
  :=
  \lift{(x)\substp{Q}{P}}{ R \psubstp{Q}{P} } \\
%   (\dropn{x})  \psubstp{Q}{P}       
%   := 
%   \left\{ 
%     \begin{array}{ccc} 
%       \dropn{\quotep{Q}} & & x \nameeq \quotep{P} \\
%       \dropn{x} & & otherwise \\
%     \end{array}
%   \right. 
  (\dropn{x})  \psubstp{Q}{P}       
  := 
  \left\{ 
    \begin{array}{ccc} 
      Q & & x \nameeq \quotep{P} \\
      \dropn{x} & & otherwise \\
    \end{array}
  \right.
\end{mathpar}
 

where

\begin{eqnarray}
  (x)\id{\{} \lpquote Q \rpquote / \lpquote P \rpquote \id{\}}            = 
  \left\{ 
    \begin{array}{ccc}
      \lpquote Q \rpquote & & x \nameeq \lpquote P \rpquote \\
      x & & otherwise \\
    \end{array}
  \right. \nonumber
\end{eqnarray}

and $z$ is chosen distinct from $\quotep{P}$, $\quotep{Q}$, the free
names in $Q$, and all the names in $R$. Our $\alpha$-equivalence will
be built in the standard way from this substitution.

\begin{remark}\label{rem:no_self_referential_names}
  One consequence of these definitions is that $\forall P. \quotep{P}
  \not\in \freenames{P}$.
\end{remark}

\subsection{ Dynamic quote: an example }

Anticipating something of what's to come, consider applying the
substitution, $\widehat{\id{\{}u / z \id{\}}}$, to the following pair
of processes, $\lift{w}{y!(z)}$ and $w[ \lpquote y!(z) \rpquote ]$.

\begin{eqnarray}
	\lift{w}{y!(z)}\widehat{\id{\{}u / z \id{\}}}
		& = &
		\lift{w}{y!(u)} \nonumber\\
	w[ \lpquote y!(z) \rpquote ] \widehat{ \id{\{}u / z \id{\}} }
		& = &
		w[ \lpquote y!(z) \rpquote ] \nonumber
\end{eqnarray}

Because the body of the process between quotes is impervious to
substitution, we get radically different answers. In fact, by
examining the first process in an input context,
e.g. $x?(z).\lift{w}{y!(z)}$, we see that the process under the lift
operator may be shaped by prefixed inputs binding a name inside it. In
this sense, the lift operator will be seen as a way to dynamically
construct processes before reifying them as names.

Finally equipped with these standard features we can present the
dynamics of the calculus.

\subsubsection{Operational semantics} 

Finally, we introduce the computational dynamics. What marks these
algebras as distinct from other more traditionally studied algebraic
structures, e.g. vector spaces or polynomial rings, is the manner in
which dynamics is captured. In traditional structures, dynamics is typically
expressed through morphisms between such structures, as in linear maps
between vector spaces or morphisms between rings. In algebras
associated with the semantics of computation, the dynamics is
expressed as part of the algebraic structure itself, through a
reduction reduction relation typically denoted by $\red$. Below, we
give a recursive presentation of this relation for the calculus used
in the encoding.

$\red \subseteq \pi \times \pi$
$\red : \pi \to \mathcal{P}(\pi)$

\begin{mathpar}
  \inferrule* [lab=Comm] { \textsf{match}( x_{src}, x_{trgt} ) } { x_{trgt}?(y)P \; | \; x_{src}!\langle {Q} \rangle \red P\{\quotep{Q}/y}\} }
  \and \\
  \inferrule* [lab=Par] {{P} \red {P}'} {{{P} | {Q}} \red {{P}' | {Q}}}
  \and
  \inferrule* [lab=Equiv]{{{P} \scong {P}'} \andalso {{P}' \red {Q}'} \andalso {{Q}' \scong {Q}}}{{P} \red {Q}}
\end{mathpar}

\begin{eqnarray*}
  match_{\equiv} (\quotep{P},\quotep{Q}) & := & P \equiv Q \\
  match_{\dagger}(\quotep{P},\quotep{Q}) & := & \forall R. P|Q \red^{*} R => R \red^{*} 0 \\
  match_{K}(\quotep{P},\quotep{Q}) & := & K \mbox{ for some context } K
\end{eqnarray*}

$u?(x)P | u!\langle Q \rangle \red P\{\quotep{Q}/x\}$

%We write $\wred$ for $\red^*$, and $P\red$ if $\exists Q $ such that $ P \red Q$.
We write $P\red$ if $\exists Q $ such that $ P \red Q$ and $P\not\red$, otherwise.

\section{Replication}

As mentioned before, it is known that replication (and hence
recursion) can be implemented in a higher-order process algebra
\cite{SangiorgiWalker}. As our first example of calculation with the
machinery thus far presented we give the construction explicitly in
the {\rhoc}.

\begin{eqnarray}
	D_{x} & := & \prefix{x}{y}{(\binpar{\outputp{x}{y}}{@{y}})} \nonumber\\
	\bangp_{x}{P} & := & \binpar{{x}!\langle{\binpar{D_{x}}{P}}\rangle}{D_{x}} \nonumber
\end{eqnarray}

\begin{eqnarray}
	\bangp_{x}{P} & & \nonumber\\
	=
	& {x}!\langle{(\prefix{x}{y}{(\outputp{x}{y} | @{y})) | P}}\rangle 
	      | \prefix{x}{y}{(\outputp{x}{y} | @{y})} & \nonumber\\
	\red
	& (\outputp{x}{y} | @{y})\substn{\quotep{(\prefix{x}{y}{(@{y} | \outputp{x}{y})) | P}}}{y} & \nonumber\\
	=
	& \outputp{x}{\quotep{(\prefix{x}{y}{(\outputp{x}{y} | @{y})) | P}}}
	  | {(\prefix{x}{y}{(\outputp{x}{y} | @{y})) | P}} & \nonumber\\
	\red
	& \ldots & \nonumber\\
	\red^*
	& P | P | \ldots & \nonumber
\end{eqnarray}

Of course, this encoding, as an implementation, runs away, unfolding
$\bangp{P}$ eagerly. A lazier and more implementable replication
operator, restricted to input-guarded processes, may be obtained as follows.

\begin{eqnarray}
\bangp{\prefix{u}{v}{P}} 
	:= 
	\binpar{\lift{x}{\prefix{u}{v}{(\binpar{D(x)}{P})}}}{D(x)} \nonumber
\end{eqnarray}

\begin{remark}
  Note that the lazier definition still does not deal with summation
  or mixed summation (i.e. sums over input and output). The reader is
  invited to construct definitions of replication that deal with these
  features. 

  Further, the definitions are parameterized in a name, $x$. Can you,
  gentle reader, make a definition that eliminates this parameter and
  guarantees no accidental interaction between the replication
  machinery and the process being replicated -- i.e. no accidental
  sharing of names used by the process to get its work done and the
  name(s) used by the replication to effect copying. This latter
  revision of the definition of replication is crucial to obtaining
  the expected identity $!!P \sim !P$.
\end{remark}

\begin{remark}\label{rem:paradoxical_combinator}
  The reader familiar with the lambda calculus will have noticed the
  similarity between $D$ and the paradoxical combinator.

  [Ed. note: the existence of this seems to suggest we have to be more
  restrictive on the set of processes and names we admit if we are to
  support no-cloning.]
\end{remark}

\subsubsection{Bisimulation}

The computational dynamics gives rise to another kind of equivalence,
the equivalence of computational behavior. As previously mentioned
this is typically captured \emph{via} some form of bisimulation.

% The notion we use in this paper is weak barbed bisimulation
% \cite{milner91polyadicpi}.

The notion we use in this paper is derived from weak barbed
bisimulation \cite{milner91polyadicpi}. 

\begin{definition}
An \emph{observation relation}, $\downarrow_{\mathcal N}$, over a set
of names, $\mathcal N$, is the smallest relation satisfying the rules
below.

\infrule[Out-barb]{y \in {\mathcal N}, \; x \nameeq y}
		  {\outputp{x}{v} \downarrow_{\mathcal N} x}
\infrule[Par-barb]{\mbox{$P\downarrow_{\mathcal N} x$ or $Q\downarrow_{\mathcal N} x$}}
		  {\binpar{P}{Q} \downarrow_{\mathcal N} x}

We write $P \Downarrow_{\mathcal N} x$ if there is $Q$ such that 
$P \wred Q$ and $Q \downarrow_{\mathcal N} x$.
\end{definition}

\begin{definition}
%\label{def.bbisim}
An  ${\mathcal N}$-\emph{barbed bisimulation} over a set of names, ${\mathcal N}$, is a symmetric binary relation 
${\mathcal S}_{\mathcal N}$ between agents such that $P\rel{S}_{\mathcal N}Q$ implies:
\begin{enumerate}
\item If $P \red P'$ then $Q \wred Q'$ and $P'\rel{S}_{\mathcal N} Q'$.
\item If $P\downarrow_{\mathcal N} x$, then $Q\Downarrow_{\mathcal N} x$.
\end{enumerate}
$P$ is ${\mathcal N}$-barbed bisimilar to $Q$, written
$P \wbbisim_{\mathcal N} Q$, if $P \rel{S}_{\mathcal N} Q$ for some ${\mathcal N}$-barbed bisimulation ${\mathcal S}_{\mathcal N}$.
\end{definition}

$\mathcal{R} \subseteq \pi \times \pi$

$P \mathcal{R} Q => \forall P'. P \red P' \Rightarrow \exists Q'. Q \red Q', P' \mathcal{R} Q'$

$P \vdash x \Rightarrow Q \vdash x$

\begin{mathpar}
  \inferrule*[lab=Out-barb]{x \nameeq y}{{y}!\langle{Q}\rangle \vdash x}
  \and
  \inferrule*[lab=Par-barb]{\mbox{$P\vdash x$ or $Q\vdash x$}}{\binpar{P}{Q} \vdash x}
\end{mathpar}

\subsubsection{Contexts}

One of the principle advantages of computational calculi like the
$\pi$-calculus is a well-defined notion of context,
contextual-equivalence and a correlation between
contextual-equivalence and notions of bisimulation. The notion of
context allows the decomposition of a process into (sub-)process and
its syntactic environment, its context. Thus, a context may be
thought of as a process with a ``hole'' (written $\Box$) in it. The
application of a context $M$ to a process $P$, written $M[P]$, is
tantamount to filling the hole in $M$ with $P$. In this paper we do
not need the full weight of this theory, but do make use of the notion
of context in the proof the main theorem. 

\begin{mathpar}
  \inferrule* [lab=summation] {} {{M_{M},M_{N}} \bc \Box \;|\; x.M_{A} \;|\; M_{M}+M_{N}}
  \and
  \inferrule* [lab=agent] {} {{M_{A}} \bc (\vec{x})M_{P} \;| \; \clift{P_0,\ldots,M_{P},\ldots,P_N}}
  \and \\
  \inferrule* [lab=process] {} {{M_{P}} \bc M_{N} \;| \;P|M_{P} }
\end{mathpar} 

\begin{mathpar}
  \inferrule* [lab=sychronization] {} {M_{N} \bc \Box \;|\; x?M_{F} \;|\; x!M_{C}}
  \and
  \inferrule* [lab=abstraction] {} {{M_{F}} \bc (x)M_{P} }
  \and
  \inferrule* [lab=concretion] {} {{M_{C}} \bc \langle M_{P} \rangle }
  \and \\
  \inferrule* [lab=process] {} {{M_{P}} \bc M_{N} \;| \;P|M_{P} }
\end{mathpar}

\begin{definition}[contextual application] Given a context $M$, and
  process $P$, we define the \emph{contextual application}, $M[P] :=
  M\{P/\Box\}$. That is, the contextual application of M to P is the
  substitution of $P$ for $\Box$ in $M$.
\end{definition}

$\meaningof{-} : L \to \mathcal{P}(\pi)$

\begin{mathpar}
  \inferrule* [lab=collection] {} {\meaningof{true} = \pi, \and \meaningof{~E} = \pi \setminus \meaningof{E}, \and \meaningof{E_{1} \& E_{2}} = \meaningof{E_{1}} \cap \meaningof{E_{2}}}
\end{mathpar}

\begin{mathpar}
  \inferrule* [lab=structure] {} {\meaningof{0} = \{ P \in \pi | P \equiv 0 \}, \and \\ \meaningof{E_1 | E_2} = \{ P \in \pi | P \equiv P_{1} | P_{2}, P_{1} \in \meaningof{E_{1}}, P_{2} \in \meaningof{E_2}\} }
\end{mathpar}

\begin{mathpar}
 \inferrule* [lab=behavior] {} {\meaningof{\langle a?b \rangle E} = \{ P \in \pi | P \equiv Q | u?(y)P', \\ \and \\\\ \and \\ \;\;\; u \in \meaningof{a}, \forall z.P'\{z/y\} \in \meaningof{E\{z/b\}}\}, \and \\ \meaningof{a!E} = \{ P \in \pi | P \equiv Q | x!\langle P' \rangle, x \in \meaningof{a} P' \in \meaningof{E}\} }
\end{mathpar}

\begin{mathpar}
 \inferrule* [lab=nominal] {} {\meaningof{\quotep{E}} = \{ \quotep{P} \in \quotep{\pi} | P \in \meaningof{E} \}, \and \meaningof{\quotep{P}} = \{ \quotep{Q} \in \quotep{\pi} | P \equiv Q \} \and \\ \meaningof{@\quotep{E}} = \{ P \in \pi | P \equiv @x, x \in \meaningof{E} \}}
\end{mathpar}

\begin{eqnarray*}
  \\
  \meaningof{-} : TS \to ST
\end{eqnarray*}

\begin{eqnarray*}
  \\
  L : TS \to ST
\end{eqnarray*}

\begin{eqnarray*}
  \\
  P \models E \iff P \in \meaningof{E}
\end{eqnarray*}

\begin{eqnarray*}
  P \approx_{L} Q \iff \forall E \in L. P \models E \iff Q \models E
\end{eqnarray*}

\begin{eqnarray*}
  P \approx_{K} Q
\end{eqnarray*}

\begin{eqnarray*}
  P \approx Q
\end{eqnarray*}

$\approx_{K} = \approx = \approx_{L}$

\subsubsection{Contextual duality}

Note that contexts extend the quotation operation to a family of
operations from processes to names. Given a context, $M$, we can
define a \emph{nominal context}, $\quotep{M}$ by $\quotep{M}[P] :=
\quotep{M[P]}$. To foreshadow what is to come we observe that these
operations enjoy a duality with processes very much like the duality
between vectors and maps from vectors to scalars.

Further, because the calculus is essentially higher-order, we have a
correspondence between contexts and processes. More specifically,
given a name $x$ and a context $M$ we can construct $M^{*}_{x}$ such
that 

\begin{mathpar}
  M^{*}_{x} | \lift{x}{P} \red M[P]
\end{mathpar}

namely,

\begin{mathpar}
  M^{*}_{x} := x?(u).M[\dropn{u}]
\end{mathpar}

The dependence of $M^{*}_{x}$ on a name makes it an abstraction, 

\begin{mathpar}
  M^{*} := (x)x?(u).M[\dropn{u}]
\end{mathpar}

\subsection{Additional notation}

It will sometimes be convenient to denote the process a name
quotes. We already have the notation $x = \quotep{P}$, but it will be
convenient to introduce an alternate notation, $\procn{x}$, when we
want to emphasize the connection to the use of the name. Note that, by
virtue of name equivalence, $\quotep{\procn{x}} \nameeq x$; so, the
notation is consistent with previous definitions.

Further, because names have structure it is possible to effect
substitutions on the basis of that structure. This means we need to
upgrade our notation for substitutions, which we accomplish by
adapting comprehension notation. Thus,

\begin{mathpar}
  P\{ y / x : x \in S \}
\end{mathpar}

is interpreted to mean the process derived from P by replacing (in a
capture-avoiding manner) each occurrence of $x$ in $S$ by $y$. For example,

\begin{mathpar}
  P\{ \quotep{\procn{x}|\procn{x}} / x : x \in \freenames{P} \}
\end{mathpar}

will replace each (occurrence) of a free name $x$ in $P$ by
$\quotep{\procn{x}|\procn{x}}$.

Also, we will avail ourselves of the notation $x^{L}$ and $x^{R}$ to
denote injections of a name into disjoint copies of the name
space. There are numerous ways to accomplish this. One example can be
found in \cite{MeredithR05}. This notation overloads to vectors of
names: $\vec{x}^{\pi} := (x_{i}^{\pi} \; : \; 0 \leq i < |\vec{x}| )$ where $\pi \in \{L,R\}$.

We also use $P^{\Box} := P|\Box$.

In \cite{MeredithR05} an interpretation of the new operator is
given. It turns out that there are several possible interpretations
all enjoying the requisite algebraic properties of the operator (see
\cite{milner91polyadicpi}). We will therefore make liberal use of
$(\nu\; \vec{x})P$.

% subsection the_syntax_and_semantics_of_the_notation_system (end)   

\input{qm2pi.qmops} 

\input{qm2pi.sterngerlach} 

\input{qm2pi.metric} 

% section concurrent_process_calculi (end)

%\input{qm2pi.proofsketch}

% section proof sketch (end)

%\input{qm2pi.slviaknots} 

% section spatial logic via knots (end)

\input{qm2pi.conclusion}

% section conclusion (end)

%\input{qm2pi.dtcodes} 

% section wiring algorithm (end)

\input{qm2pi.ack} 

% section acknowledgments (end)

\newpage


\bibliographystyle{plain}   
\bibliography{../../biblios/main.bib}

\input{qm2pi.rhodetails}

\end{document}

 

% section wiring algorithm (end)

\documentclass[12pt]{llncs}
%\documentclass{jktr}

\usepackage[pdftex]{hyperref}                   
\usepackage {listings}
\usepackage {mathpartir}
\usepackage{bcprules}
%\usepackage{listings}
                       
\usepackage{graphicx} 
%\usepackage[margins=2.5cm,nohead,nofoot]{geometry}
%\usepackage{geometry}
\usepackage{amsfonts}
\usepackage{amstext}
\usepackage{latexsym}
\usepackage{amssymb}
\usepackage{color}


%\include{myPreamble}
\include{qm2pi.local} 

%\ifpdf
%\usepackage[pdftex]{graphicx}
%\else
%\usepackage{graphicx}
%\fi

 % \ifpdf
%  \usepackage{pdfsync}
%  \if


%\title{Brief Article}
%\author{David F. Snyder}
%\author{L.G. Meredith}

%\address{Dept. of Math., Texas State University--San Marcos, San Marcos, TX 78666}
       
\pagestyle{empty}


\begin{document}

\lstset{language=[Objective]Caml,frame=shadowbox}

\input{qm2pi.front}

% section front matter (end)

\input{qm2pi.intro} 
 
% section introduction (end)

% \input{qm2pi.knotations} 

% section notation (end)

\input{qm2pi.process.calculi} 

% section concurrent_process_calculi_and_spatial_logics_ (end)
    
%\input{qm2pi.knots2pi} 

%\input{qm2pi.trefoil} 

%\input{qm2pi.mainthm} 

% subsection basic_interpretation (end)

%\input{qm2pi.rho.presentation} 
\subsection{The syntax and semantics of the notation system}\label{sub:the_syntax_and_semantics_of_the_notation_system} % (fold)

We now summarize a technical presentation of the calculus that
embodies our theory of dynamics. The typical presentation of such a
calculus follows the style of giving generators and relations on
them. The grammar, below, describing term constructors, freely
generates the set of processes, $\Proc$. This set is then quotiented
by a relation known as structural congruence and it is over this set
that the notion of dynamics is expressed. This presentation is
essentially that of \cite{MeredithR05} with the addition of
polyadicity and summation. For readability we have relegated some of
the technical subtleties to an appendix.

\subsubsection{Process grammar}\label{subsub:process_grammar}

\begin{mathpar}
  \inferrule* [lab=synchronization] {} {{M} \bc \pzero \;|\; x?F \;|\; x!C }
  \and
  \inferrule* [lab=abstraction] {} {{F} \bc (x)P}
  \and
  \inferrule* [lab=concretion] {} {{C} \bc \langle Q \rangle}
  \and
  \inferrule* [lab=process] {} {{P,Q} \bc M \;| \;P|Q \;|\; @{x}}
  \and
  \inferrule* [lab=name] {} {{x} \bc \quotep{P}}
\end{mathpar} 

Note that $\vec{x}$ (resp. $\vec{P}$) denotes a vector of names
(resp. processes) of length $|\vec{x}|$ (resp. $|\vec{P}|$). We adopt
the following useful abbreviations.

\begin{mathpar}
   x?(\vec{y}).P := x.(\vec{y})P \and  x\clift{\vec{P}} := x.\clift{\vec{P}}
   \and x!(y) := \lift{x}{\dropn{y}}
   \and \Pi_{i=0}^{n-1}P_i := P_0 | \ldots | P_{n-1}
\end{mathpar}

\subsubsection{Structural congruence}

\paragraph{Free and bound names and alpha-equivalence.} At the
core of structural equivalence is alpha-equivalence which identifies
process that are the same up to a change of variable. Formally, we
recognize the distinction between free and bound names. The free names
of a process, $\freenames{P}$, may be calculated recursively as
follows:

\begin{mathpar}
\freenames{\pzero} := \emptyset
  \and \\
  \freenames{x?(y).P} := \{ x \} \cup (\freenames{P} \setminus \{ y \})
  \and 
  \freenames{x!\langle P \rangle} := \{ x \} \cup \{ P \} 
  \and \\
  \freenames{P|Q} := \freenames{P} \cup \freenames{Q}
  \and \\
  \freenames{@{x}} := \{ x \}
\end{mathpar}

$\pi$
$\quotep{\pi}$

$\freenames{-} : \pi \to \mathcal{P}(\quotep{\pi})$

\begin{eqnarray*}
  \freenames{\pzero} & := & \emptyset \\
  \freenames{x?(y).P} & := & \{ x \} \cup (\freenames{P} \setminus \{ y \}) \\
  \freenames{x!\langle P \rangle} & := & \{ x \} \cup \{ P \} \\
  \freenames{P|Q} & := & \freenames{P} \cup \freenames{Q} \\
  \freenames{\dropn{x}} & := & \{ x \}
\end{eqnarray*}

The bound names of a process, $\boundnames{P}$, are those names occurring in $P$
that are not free. For example, in $x?(y).0$, the name $x$ is free, while $y$ is bound.

\begin{mathpar}
  \inferrule* [lab=monoidal-laws] {} { P|Q \equiv Q|P \and P|0 \equiv P \and P|(Q|R) \equiv (P|Q)|R }
\end{mathpar}

\begin{mathpar}
  \inferrule* [lab=alpha-equivalence] {} { (x)P \equiv (y)P\{y/x\} \and y \not\in \freenames{P} }
\end{mathpar}

\begin{definition}
Then two processes, $P,Q$, are alpha-equivalent if $P = Q\{\vec{y}/\vec{x}\}$ for
some $\vec{x} \in \boundnames{Q},\vec{y} \in \boundnames{P}$, where $Q\{\vec{y}/\vec{x}\}$
denotes the capture-avoiding substitution of $\vec{y}$ for $\vec{x}$ in $Q$.
\end{definition}

\begin{definition}
  The {\em structural congruence} \cite{SangiorgiWalker} , $\equiv$,
  between processes is the least congruence containing
  alpha-equivalence, satisfying the abelian monoid laws
  (associativity, commutativity and $\pzero$ as identity) for parallel
  composition $|$ and for summation $+$.
\end{definition}

\subsection{Name equivalence}

We take name equivalence, written $\nameeq$, to be the smallest
equivalence relation generated by the following rules.

\begin{mathpar}
\inferrule*[lab=Quote-drop]
{ }
{ \quotep{@{x}} \nameeq x }

\inferrule*[lab=Struct-equiv]
{ P \scong Q }
{ \quotep{P} \nameeq \quotep{Q} }
\end{mathpar}

The astute reader will have noticed that the mutual recursion of names
and processes imposes a mutual recursion on alpha-equivalence and
structural equivalence via name-equivalence. Fortunately, all of this
works out pleasantly and we may calculate in the natural way, free of
concern. The reader interested in the details is referred to the
appendix \ref{appendix:rho_details}.

\subsection{Substitution}

We use $\Proc$ for the set of processes, $\QProc$ for the set of
names, and $\id{\{}\vec{y} / \vec{x} \id{\}}$ to denote partial maps,
$s : \QProc \rightarrow \QProc$. A map, $s$ lifts, uniquely, to a map
on process terms, $\widehat{s} : \Proc \rightarrow \Proc$ by the
following equations.

\begin{mathpar}
  (0) \psubstp{Q}{P} := 0 \\
  (R \juxtap S) \psubstp{Q}{P}
  :=    
  (R)\psubstp{Q}{P} \juxtap (S) \psubstp{Q}{P} \\
  (x?(y).R) \psubstp{Q}{P}    
  :=    
  (x)\substp{Q}{P} (z)\concat( (R \psubstn{z}{y}) \psubstp{Q}{P} ) \\
  (\lift{x}{R}) \psubstp{Q}{P}  
  :=
  \lift{(x)\substp{Q}{P}}{ R \psubstp{Q}{P} } \\
%   (\dropn{x})  \psubstp{Q}{P}       
%   := 
%   \left\{ 
%     \begin{array}{ccc} 
%       \dropn{\quotep{Q}} & & x \nameeq \quotep{P} \\
%       \dropn{x} & & otherwise \\
%     \end{array}
%   \right. 
  (\dropn{x})  \psubstp{Q}{P}       
  := 
  \left\{ 
    \begin{array}{ccc} 
      Q & & x \nameeq \quotep{P} \\
      \dropn{x} & & otherwise \\
    \end{array}
  \right.
\end{mathpar}
 

where

\begin{eqnarray}
  (x)\id{\{} \lpquote Q \rpquote / \lpquote P \rpquote \id{\}}            = 
  \left\{ 
    \begin{array}{ccc}
      \lpquote Q \rpquote & & x \nameeq \lpquote P \rpquote \\
      x & & otherwise \\
    \end{array}
  \right. \nonumber
\end{eqnarray}

and $z$ is chosen distinct from $\quotep{P}$, $\quotep{Q}$, the free
names in $Q$, and all the names in $R$. Our $\alpha$-equivalence will
be built in the standard way from this substitution.

\begin{remark}\label{rem:no_self_referential_names}
  One consequence of these definitions is that $\forall P. \quotep{P}
  \not\in \freenames{P}$.
\end{remark}

\subsection{ Dynamic quote: an example }

Anticipating something of what's to come, consider applying the
substitution, $\widehat{\id{\{}u / z \id{\}}}$, to the following pair
of processes, $\lift{w}{y!(z)}$ and $w[ \lpquote y!(z) \rpquote ]$.

\begin{eqnarray}
	\lift{w}{y!(z)}\widehat{\id{\{}u / z \id{\}}}
		& = &
		\lift{w}{y!(u)} \nonumber\\
	w[ \lpquote y!(z) \rpquote ] \widehat{ \id{\{}u / z \id{\}} }
		& = &
		w[ \lpquote y!(z) \rpquote ] \nonumber
\end{eqnarray}

Because the body of the process between quotes is impervious to
substitution, we get radically different answers. In fact, by
examining the first process in an input context,
e.g. $x?(z).\lift{w}{y!(z)}$, we see that the process under the lift
operator may be shaped by prefixed inputs binding a name inside it. In
this sense, the lift operator will be seen as a way to dynamically
construct processes before reifying them as names.

Finally equipped with these standard features we can present the
dynamics of the calculus.

\subsubsection{Operational semantics} 

Finally, we introduce the computational dynamics. What marks these
algebras as distinct from other more traditionally studied algebraic
structures, e.g. vector spaces or polynomial rings, is the manner in
which dynamics is captured. In traditional structures, dynamics is typically
expressed through morphisms between such structures, as in linear maps
between vector spaces or morphisms between rings. In algebras
associated with the semantics of computation, the dynamics is
expressed as part of the algebraic structure itself, through a
reduction reduction relation typically denoted by $\red$. Below, we
give a recursive presentation of this relation for the calculus used
in the encoding.

$\red \subseteq \pi \times \pi$
$\red : \pi \to \mathcal{P}(\pi)$

\begin{mathpar}
  \inferrule* [lab=Comm] { \textsf{match}( x_{src}, x_{trgt} ) } { x_{trgt}?(y)P \; | \; x_{src}!\langle {Q} \rangle \red P\{\quotep{Q}/y}\} }
  \and \\
  \inferrule* [lab=Par] {{P} \red {P}'} {{{P} | {Q}} \red {{P}' | {Q}}}
  \and
  \inferrule* [lab=Equiv]{{{P} \scong {P}'} \andalso {{P}' \red {Q}'} \andalso {{Q}' \scong {Q}}}{{P} \red {Q}}
\end{mathpar}

\begin{eqnarray*}
  match_{\equiv} (\quotep{P},\quotep{Q}) & := & P \equiv Q \\
  match_{\dagger}(\quotep{P},\quotep{Q}) & := & \forall R. P|Q \red^{*} R => R \red^{*} 0 \\
  match_{K}(\quotep{P},\quotep{Q}) & := & K \mbox{ for some context } K
\end{eqnarray*}

$u?(x)P | u!\langle Q \rangle \red P\{\quotep{Q}/x\}$

%We write $\wred$ for $\red^*$, and $P\red$ if $\exists Q $ such that $ P \red Q$.
We write $P\red$ if $\exists Q $ such that $ P \red Q$ and $P\not\red$, otherwise.

\section{Replication}

As mentioned before, it is known that replication (and hence
recursion) can be implemented in a higher-order process algebra
\cite{SangiorgiWalker}. As our first example of calculation with the
machinery thus far presented we give the construction explicitly in
the {\rhoc}.

\begin{eqnarray}
	D_{x} & := & \prefix{x}{y}{(\binpar{\outputp{x}{y}}{@{y}})} \nonumber\\
	\bangp_{x}{P} & := & \binpar{{x}!\langle{\binpar{D_{x}}{P}}\rangle}{D_{x}} \nonumber
\end{eqnarray}

\begin{eqnarray}
	\bangp_{x}{P} & & \nonumber\\
	=
	& {x}!\langle{(\prefix{x}{y}{(\outputp{x}{y} | @{y})) | P}}\rangle 
	      | \prefix{x}{y}{(\outputp{x}{y} | @{y})} & \nonumber\\
	\red
	& (\outputp{x}{y} | @{y})\substn{\quotep{(\prefix{x}{y}{(@{y} | \outputp{x}{y})) | P}}}{y} & \nonumber\\
	=
	& \outputp{x}{\quotep{(\prefix{x}{y}{(\outputp{x}{y} | @{y})) | P}}}
	  | {(\prefix{x}{y}{(\outputp{x}{y} | @{y})) | P}} & \nonumber\\
	\red
	& \ldots & \nonumber\\
	\red^*
	& P | P | \ldots & \nonumber
\end{eqnarray}

Of course, this encoding, as an implementation, runs away, unfolding
$\bangp{P}$ eagerly. A lazier and more implementable replication
operator, restricted to input-guarded processes, may be obtained as follows.

\begin{eqnarray}
\bangp{\prefix{u}{v}{P}} 
	:= 
	\binpar{\lift{x}{\prefix{u}{v}{(\binpar{D(x)}{P})}}}{D(x)} \nonumber
\end{eqnarray}

\begin{remark}
  Note that the lazier definition still does not deal with summation
  or mixed summation (i.e. sums over input and output). The reader is
  invited to construct definitions of replication that deal with these
  features. 

  Further, the definitions are parameterized in a name, $x$. Can you,
  gentle reader, make a definition that eliminates this parameter and
  guarantees no accidental interaction between the replication
  machinery and the process being replicated -- i.e. no accidental
  sharing of names used by the process to get its work done and the
  name(s) used by the replication to effect copying. This latter
  revision of the definition of replication is crucial to obtaining
  the expected identity $!!P \sim !P$.
\end{remark}

\begin{remark}\label{rem:paradoxical_combinator}
  The reader familiar with the lambda calculus will have noticed the
  similarity between $D$ and the paradoxical combinator.

  [Ed. note: the existence of this seems to suggest we have to be more
  restrictive on the set of processes and names we admit if we are to
  support no-cloning.]
\end{remark}

\subsubsection{Bisimulation}

The computational dynamics gives rise to another kind of equivalence,
the equivalence of computational behavior. As previously mentioned
this is typically captured \emph{via} some form of bisimulation.

% The notion we use in this paper is weak barbed bisimulation
% \cite{milner91polyadicpi}.

The notion we use in this paper is derived from weak barbed
bisimulation \cite{milner91polyadicpi}. 

\begin{definition}
An \emph{observation relation}, $\downarrow_{\mathcal N}$, over a set
of names, $\mathcal N$, is the smallest relation satisfying the rules
below.

\infrule[Out-barb]{y \in {\mathcal N}, \; x \nameeq y}
		  {\outputp{x}{v} \downarrow_{\mathcal N} x}
\infrule[Par-barb]{\mbox{$P\downarrow_{\mathcal N} x$ or $Q\downarrow_{\mathcal N} x$}}
		  {\binpar{P}{Q} \downarrow_{\mathcal N} x}

We write $P \Downarrow_{\mathcal N} x$ if there is $Q$ such that 
$P \wred Q$ and $Q \downarrow_{\mathcal N} x$.
\end{definition}

\begin{definition}
%\label{def.bbisim}
An  ${\mathcal N}$-\emph{barbed bisimulation} over a set of names, ${\mathcal N}$, is a symmetric binary relation 
${\mathcal S}_{\mathcal N}$ between agents such that $P\rel{S}_{\mathcal N}Q$ implies:
\begin{enumerate}
\item If $P \red P'$ then $Q \wred Q'$ and $P'\rel{S}_{\mathcal N} Q'$.
\item If $P\downarrow_{\mathcal N} x$, then $Q\Downarrow_{\mathcal N} x$.
\end{enumerate}
$P$ is ${\mathcal N}$-barbed bisimilar to $Q$, written
$P \wbbisim_{\mathcal N} Q$, if $P \rel{S}_{\mathcal N} Q$ for some ${\mathcal N}$-barbed bisimulation ${\mathcal S}_{\mathcal N}$.
\end{definition}

$\mathcal{R} \subseteq \pi \times \pi$

$P \mathcal{R} Q => \forall P'. P \red P' \Rightarrow \exists Q'. Q \red Q', P' \mathcal{R} Q'$

$P \vdash x \Rightarrow Q \vdash x$

\begin{mathpar}
  \inferrule*[lab=Out-barb]{x \nameeq y}{{y}!\langle{Q}\rangle \vdash x}
  \and
  \inferrule*[lab=Par-barb]{\mbox{$P\vdash x$ or $Q\vdash x$}}{\binpar{P}{Q} \vdash x}
\end{mathpar}

\subsubsection{Contexts}

One of the principle advantages of computational calculi like the
$\pi$-calculus is a well-defined notion of context,
contextual-equivalence and a correlation between
contextual-equivalence and notions of bisimulation. The notion of
context allows the decomposition of a process into (sub-)process and
its syntactic environment, its context. Thus, a context may be
thought of as a process with a ``hole'' (written $\Box$) in it. The
application of a context $M$ to a process $P$, written $M[P]$, is
tantamount to filling the hole in $M$ with $P$. In this paper we do
not need the full weight of this theory, but do make use of the notion
of context in the proof the main theorem. 

\begin{mathpar}
  \inferrule* [lab=summation] {} {{M_{M},M_{N}} \bc \Box \;|\; x.M_{A} \;|\; M_{M}+M_{N}}
  \and
  \inferrule* [lab=agent] {} {{M_{A}} \bc (\vec{x})M_{P} \;| \; \clift{P_0,\ldots,M_{P},\ldots,P_N}}
  \and \\
  \inferrule* [lab=process] {} {{M_{P}} \bc M_{N} \;| \;P|M_{P} }
\end{mathpar} 

\begin{mathpar}
  \inferrule* [lab=sychronization] {} {M_{N} \bc \Box \;|\; x?M_{F} \;|\; x!M_{C}}
  \and
  \inferrule* [lab=abstraction] {} {{M_{F}} \bc (x)M_{P} }
  \and
  \inferrule* [lab=concretion] {} {{M_{C}} \bc \langle M_{P} \rangle }
  \and \\
  \inferrule* [lab=process] {} {{M_{P}} \bc M_{N} \;| \;P|M_{P} }
\end{mathpar}

\begin{definition}[contextual application] Given a context $M$, and
  process $P$, we define the \emph{contextual application}, $M[P] :=
  M\{P/\Box\}$. That is, the contextual application of M to P is the
  substitution of $P$ for $\Box$ in $M$.
\end{definition}

$\meaningof{-} : L \to \mathcal{P}(\pi)$

\begin{mathpar}
  \inferrule* [lab=collection] {} {\meaningof{true} = \pi, \and \meaningof{~E} = \pi \setminus \meaningof{E}, \and \meaningof{E_{1} \& E_{2}} = \meaningof{E_{1}} \cap \meaningof{E_{2}}}
\end{mathpar}

\begin{mathpar}
  \inferrule* [lab=structure] {} {\meaningof{0} = \{ P \in \pi | P \equiv 0 \}, \and \\ \meaningof{E_1 | E_2} = \{ P \in \pi | P \equiv P_{1} | P_{2}, P_{1} \in \meaningof{E_{1}}, P_{2} \in \meaningof{E_2}\} }
\end{mathpar}

\begin{mathpar}
 \inferrule* [lab=behavior] {} {\meaningof{\langle a?b \rangle E} = \{ P \in \pi | P \equiv Q | u?(y)P', \\ \and \\\\ \and \\ \;\;\; u \in \meaningof{a}, \forall z.P'\{z/y\} \in \meaningof{E\{z/b\}}\}, \and \\ \meaningof{a!E} = \{ P \in \pi | P \equiv Q | x!\langle P' \rangle, x \in \meaningof{a} P' \in \meaningof{E}\} }
\end{mathpar}

\begin{mathpar}
 \inferrule* [lab=nominal] {} {\meaningof{\quotep{E}} = \{ \quotep{P} \in \quotep{\pi} | P \in \meaningof{E} \}, \and \meaningof{\quotep{P}} = \{ \quotep{Q} \in \quotep{\pi} | P \equiv Q \} \and \\ \meaningof{@\quotep{E}} = \{ P \in \pi | P \equiv @x, x \in \meaningof{E} \}}
\end{mathpar}

\begin{eqnarray*}
  \\
  \meaningof{-} : TS \to ST
\end{eqnarray*}

\begin{eqnarray*}
  \\
  L : TS \to ST
\end{eqnarray*}

\begin{eqnarray*}
  \\
  P \models E \iff P \in \meaningof{E}
\end{eqnarray*}

\begin{eqnarray*}
  P \approx_{L} Q \iff \forall E \in L. P \models E \iff Q \models E
\end{eqnarray*}

\begin{eqnarray*}
  P \approx_{K} Q
\end{eqnarray*}

\begin{eqnarray*}
  P \approx Q
\end{eqnarray*}

$\approx_{K} = \approx = \approx_{L}$

\subsubsection{Contextual duality}

Note that contexts extend the quotation operation to a family of
operations from processes to names. Given a context, $M$, we can
define a \emph{nominal context}, $\quotep{M}$ by $\quotep{M}[P] :=
\quotep{M[P]}$. To foreshadow what is to come we observe that these
operations enjoy a duality with processes very much like the duality
between vectors and maps from vectors to scalars.

Further, because the calculus is essentially higher-order, we have a
correspondence between contexts and processes. More specifically,
given a name $x$ and a context $M$ we can construct $M^{*}_{x}$ such
that 

\begin{mathpar}
  M^{*}_{x} | \lift{x}{P} \red M[P]
\end{mathpar}

namely,

\begin{mathpar}
  M^{*}_{x} := x?(u).M[\dropn{u}]
\end{mathpar}

The dependence of $M^{*}_{x}$ on a name makes it an abstraction, 

\begin{mathpar}
  M^{*} := (x)x?(u).M[\dropn{u}]
\end{mathpar}

\subsection{Additional notation}

It will sometimes be convenient to denote the process a name
quotes. We already have the notation $x = \quotep{P}$, but it will be
convenient to introduce an alternate notation, $\procn{x}$, when we
want to emphasize the connection to the use of the name. Note that, by
virtue of name equivalence, $\quotep{\procn{x}} \nameeq x$; so, the
notation is consistent with previous definitions.

Further, because names have structure it is possible to effect
substitutions on the basis of that structure. This means we need to
upgrade our notation for substitutions, which we accomplish by
adapting comprehension notation. Thus,

\begin{mathpar}
  P\{ y / x : x \in S \}
\end{mathpar}

is interpreted to mean the process derived from P by replacing (in a
capture-avoiding manner) each occurrence of $x$ in $S$ by $y$. For example,

\begin{mathpar}
  P\{ \quotep{\procn{x}|\procn{x}} / x : x \in \freenames{P} \}
\end{mathpar}

will replace each (occurrence) of a free name $x$ in $P$ by
$\quotep{\procn{x}|\procn{x}}$.

Also, we will avail ourselves of the notation $x^{L}$ and $x^{R}$ to
denote injections of a name into disjoint copies of the name
space. There are numerous ways to accomplish this. One example can be
found in \cite{MeredithR05}. This notation overloads to vectors of
names: $\vec{x}^{\pi} := (x_{i}^{\pi} \; : \; 0 \leq i < |\vec{x}| )$ where $\pi \in \{L,R\}$.

We also use $P^{\Box} := P|\Box$.

In \cite{MeredithR05} an interpretation of the new operator is
given. It turns out that there are several possible interpretations
all enjoying the requisite algebraic properties of the operator (see
\cite{milner91polyadicpi}). We will therefore make liberal use of
$(\nu\; \vec{x})P$.

% subsection the_syntax_and_semantics_of_the_notation_system (end)   

\input{qm2pi.qmops} 

\input{qm2pi.sterngerlach} 

\input{qm2pi.metric} 

% section concurrent_process_calculi (end)

%\input{qm2pi.proofsketch}

% section proof sketch (end)

%\input{qm2pi.slviaknots} 

% section spatial logic via knots (end)

\input{qm2pi.conclusion}

% section conclusion (end)

%\input{qm2pi.dtcodes} 

% section wiring algorithm (end)

\input{qm2pi.ack} 

% section acknowledgments (end)

\newpage


\bibliographystyle{plain}   
\bibliography{../../biblios/main.bib}

\input{qm2pi.rhodetails}

\end{document}

 

% section acknowledgments (end)

\newpage


\bibliographystyle{plain}   
\bibliography{../../biblios/main.bib}

\documentclass[12pt]{llncs}
%\documentclass{jktr}

\usepackage[pdftex]{hyperref}                   
\usepackage {listings}
\usepackage {mathpartir}
\usepackage{bcprules}
%\usepackage{listings}
                       
\usepackage{graphicx} 
%\usepackage[margins=2.5cm,nohead,nofoot]{geometry}
%\usepackage{geometry}
\usepackage{amsfonts}
\usepackage{amstext}
\usepackage{latexsym}
\usepackage{amssymb}
\usepackage{color}


%\include{myPreamble}
\include{qm2pi.local} 

%\ifpdf
%\usepackage[pdftex]{graphicx}
%\else
%\usepackage{graphicx}
%\fi

 % \ifpdf
%  \usepackage{pdfsync}
%  \if


%\title{Brief Article}
%\author{David F. Snyder}
%\author{L.G. Meredith}

%\address{Dept. of Math., Texas State University--San Marcos, San Marcos, TX 78666}
       
\pagestyle{empty}


\begin{document}

\lstset{language=[Objective]Caml,frame=shadowbox}

\input{qm2pi.front}

% section front matter (end)

\input{qm2pi.intro} 
 
% section introduction (end)

% \input{qm2pi.knotations} 

% section notation (end)

\input{qm2pi.process.calculi} 

% section concurrent_process_calculi_and_spatial_logics_ (end)
    
%\input{qm2pi.knots2pi} 

%\input{qm2pi.trefoil} 

%\input{qm2pi.mainthm} 

% subsection basic_interpretation (end)

%\input{qm2pi.rho.presentation} 
\subsection{The syntax and semantics of the notation system}\label{sub:the_syntax_and_semantics_of_the_notation_system} % (fold)

We now summarize a technical presentation of the calculus that
embodies our theory of dynamics. The typical presentation of such a
calculus follows the style of giving generators and relations on
them. The grammar, below, describing term constructors, freely
generates the set of processes, $\Proc$. This set is then quotiented
by a relation known as structural congruence and it is over this set
that the notion of dynamics is expressed. This presentation is
essentially that of \cite{MeredithR05} with the addition of
polyadicity and summation. For readability we have relegated some of
the technical subtleties to an appendix.

\subsubsection{Process grammar}\label{subsub:process_grammar}

\begin{mathpar}
  \inferrule* [lab=synchronization] {} {{M} \bc \pzero \;|\; x?F \;|\; x!C }
  \and
  \inferrule* [lab=abstraction] {} {{F} \bc (x)P}
  \and
  \inferrule* [lab=concretion] {} {{C} \bc \langle Q \rangle}
  \and
  \inferrule* [lab=process] {} {{P,Q} \bc M \;| \;P|Q \;|\; @{x}}
  \and
  \inferrule* [lab=name] {} {{x} \bc \quotep{P}}
\end{mathpar} 

Note that $\vec{x}$ (resp. $\vec{P}$) denotes a vector of names
(resp. processes) of length $|\vec{x}|$ (resp. $|\vec{P}|$). We adopt
the following useful abbreviations.

\begin{mathpar}
   x?(\vec{y}).P := x.(\vec{y})P \and  x\clift{\vec{P}} := x.\clift{\vec{P}}
   \and x!(y) := \lift{x}{\dropn{y}}
   \and \Pi_{i=0}^{n-1}P_i := P_0 | \ldots | P_{n-1}
\end{mathpar}

\subsubsection{Structural congruence}

\paragraph{Free and bound names and alpha-equivalence.} At the
core of structural equivalence is alpha-equivalence which identifies
process that are the same up to a change of variable. Formally, we
recognize the distinction between free and bound names. The free names
of a process, $\freenames{P}$, may be calculated recursively as
follows:

\begin{mathpar}
\freenames{\pzero} := \emptyset
  \and \\
  \freenames{x?(y).P} := \{ x \} \cup (\freenames{P} \setminus \{ y \})
  \and 
  \freenames{x!\langle P \rangle} := \{ x \} \cup \{ P \} 
  \and \\
  \freenames{P|Q} := \freenames{P} \cup \freenames{Q}
  \and \\
  \freenames{@{x}} := \{ x \}
\end{mathpar}

$\pi$
$\quotep{\pi}$

$\freenames{-} : \pi \to \mathcal{P}(\quotep{\pi})$

\begin{eqnarray*}
  \freenames{\pzero} & := & \emptyset \\
  \freenames{x?(y).P} & := & \{ x \} \cup (\freenames{P} \setminus \{ y \}) \\
  \freenames{x!\langle P \rangle} & := & \{ x \} \cup \{ P \} \\
  \freenames{P|Q} & := & \freenames{P} \cup \freenames{Q} \\
  \freenames{\dropn{x}} & := & \{ x \}
\end{eqnarray*}

The bound names of a process, $\boundnames{P}$, are those names occurring in $P$
that are not free. For example, in $x?(y).0$, the name $x$ is free, while $y$ is bound.

\begin{mathpar}
  \inferrule* [lab=monoidal-laws] {} { P|Q \equiv Q|P \and P|0 \equiv P \and P|(Q|R) \equiv (P|Q)|R }
\end{mathpar}

\begin{mathpar}
  \inferrule* [lab=alpha-equivalence] {} { (x)P \equiv (y)P\{y/x\} \and y \not\in \freenames{P} }
\end{mathpar}

\begin{definition}
Then two processes, $P,Q$, are alpha-equivalent if $P = Q\{\vec{y}/\vec{x}\}$ for
some $\vec{x} \in \boundnames{Q},\vec{y} \in \boundnames{P}$, where $Q\{\vec{y}/\vec{x}\}$
denotes the capture-avoiding substitution of $\vec{y}$ for $\vec{x}$ in $Q$.
\end{definition}

\begin{definition}
  The {\em structural congruence} \cite{SangiorgiWalker} , $\equiv$,
  between processes is the least congruence containing
  alpha-equivalence, satisfying the abelian monoid laws
  (associativity, commutativity and $\pzero$ as identity) for parallel
  composition $|$ and for summation $+$.
\end{definition}

\subsection{Name equivalence}

We take name equivalence, written $\nameeq$, to be the smallest
equivalence relation generated by the following rules.

\begin{mathpar}
\inferrule*[lab=Quote-drop]
{ }
{ \quotep{@{x}} \nameeq x }

\inferrule*[lab=Struct-equiv]
{ P \scong Q }
{ \quotep{P} \nameeq \quotep{Q} }
\end{mathpar}

The astute reader will have noticed that the mutual recursion of names
and processes imposes a mutual recursion on alpha-equivalence and
structural equivalence via name-equivalence. Fortunately, all of this
works out pleasantly and we may calculate in the natural way, free of
concern. The reader interested in the details is referred to the
appendix \ref{appendix:rho_details}.

\subsection{Substitution}

We use $\Proc$ for the set of processes, $\QProc$ for the set of
names, and $\id{\{}\vec{y} / \vec{x} \id{\}}$ to denote partial maps,
$s : \QProc \rightarrow \QProc$. A map, $s$ lifts, uniquely, to a map
on process terms, $\widehat{s} : \Proc \rightarrow \Proc$ by the
following equations.

\begin{mathpar}
  (0) \psubstp{Q}{P} := 0 \\
  (R \juxtap S) \psubstp{Q}{P}
  :=    
  (R)\psubstp{Q}{P} \juxtap (S) \psubstp{Q}{P} \\
  (x?(y).R) \psubstp{Q}{P}    
  :=    
  (x)\substp{Q}{P} (z)\concat( (R \psubstn{z}{y}) \psubstp{Q}{P} ) \\
  (\lift{x}{R}) \psubstp{Q}{P}  
  :=
  \lift{(x)\substp{Q}{P}}{ R \psubstp{Q}{P} } \\
%   (\dropn{x})  \psubstp{Q}{P}       
%   := 
%   \left\{ 
%     \begin{array}{ccc} 
%       \dropn{\quotep{Q}} & & x \nameeq \quotep{P} \\
%       \dropn{x} & & otherwise \\
%     \end{array}
%   \right. 
  (\dropn{x})  \psubstp{Q}{P}       
  := 
  \left\{ 
    \begin{array}{ccc} 
      Q & & x \nameeq \quotep{P} \\
      \dropn{x} & & otherwise \\
    \end{array}
  \right.
\end{mathpar}
 

where

\begin{eqnarray}
  (x)\id{\{} \lpquote Q \rpquote / \lpquote P \rpquote \id{\}}            = 
  \left\{ 
    \begin{array}{ccc}
      \lpquote Q \rpquote & & x \nameeq \lpquote P \rpquote \\
      x & & otherwise \\
    \end{array}
  \right. \nonumber
\end{eqnarray}

and $z$ is chosen distinct from $\quotep{P}$, $\quotep{Q}$, the free
names in $Q$, and all the names in $R$. Our $\alpha$-equivalence will
be built in the standard way from this substitution.

\begin{remark}\label{rem:no_self_referential_names}
  One consequence of these definitions is that $\forall P. \quotep{P}
  \not\in \freenames{P}$.
\end{remark}

\subsection{ Dynamic quote: an example }

Anticipating something of what's to come, consider applying the
substitution, $\widehat{\id{\{}u / z \id{\}}}$, to the following pair
of processes, $\lift{w}{y!(z)}$ and $w[ \lpquote y!(z) \rpquote ]$.

\begin{eqnarray}
	\lift{w}{y!(z)}\widehat{\id{\{}u / z \id{\}}}
		& = &
		\lift{w}{y!(u)} \nonumber\\
	w[ \lpquote y!(z) \rpquote ] \widehat{ \id{\{}u / z \id{\}} }
		& = &
		w[ \lpquote y!(z) \rpquote ] \nonumber
\end{eqnarray}

Because the body of the process between quotes is impervious to
substitution, we get radically different answers. In fact, by
examining the first process in an input context,
e.g. $x?(z).\lift{w}{y!(z)}$, we see that the process under the lift
operator may be shaped by prefixed inputs binding a name inside it. In
this sense, the lift operator will be seen as a way to dynamically
construct processes before reifying them as names.

Finally equipped with these standard features we can present the
dynamics of the calculus.

\subsubsection{Operational semantics} 

Finally, we introduce the computational dynamics. What marks these
algebras as distinct from other more traditionally studied algebraic
structures, e.g. vector spaces or polynomial rings, is the manner in
which dynamics is captured. In traditional structures, dynamics is typically
expressed through morphisms between such structures, as in linear maps
between vector spaces or morphisms between rings. In algebras
associated with the semantics of computation, the dynamics is
expressed as part of the algebraic structure itself, through a
reduction reduction relation typically denoted by $\red$. Below, we
give a recursive presentation of this relation for the calculus used
in the encoding.

$\red \subseteq \pi \times \pi$
$\red : \pi \to \mathcal{P}(\pi)$

\begin{mathpar}
  \inferrule* [lab=Comm] { \textsf{match}( x_{src}, x_{trgt} ) } { x_{trgt}?(y)P \; | \; x_{src}!\langle {Q} \rangle \red P\{\quotep{Q}/y}\} }
  \and \\
  \inferrule* [lab=Par] {{P} \red {P}'} {{{P} | {Q}} \red {{P}' | {Q}}}
  \and
  \inferrule* [lab=Equiv]{{{P} \scong {P}'} \andalso {{P}' \red {Q}'} \andalso {{Q}' \scong {Q}}}{{P} \red {Q}}
\end{mathpar}

\begin{eqnarray*}
  match_{\equiv} (\quotep{P},\quotep{Q}) & := & P \equiv Q \\
  match_{\dagger}(\quotep{P},\quotep{Q}) & := & \forall R. P|Q \red^{*} R => R \red^{*} 0 \\
  match_{K}(\quotep{P},\quotep{Q}) & := & K \mbox{ for some context } K
\end{eqnarray*}

$u?(x)P | u!\langle Q \rangle \red P\{\quotep{Q}/x\}$

%We write $\wred$ for $\red^*$, and $P\red$ if $\exists Q $ such that $ P \red Q$.
We write $P\red$ if $\exists Q $ such that $ P \red Q$ and $P\not\red$, otherwise.

\section{Replication}

As mentioned before, it is known that replication (and hence
recursion) can be implemented in a higher-order process algebra
\cite{SangiorgiWalker}. As our first example of calculation with the
machinery thus far presented we give the construction explicitly in
the {\rhoc}.

\begin{eqnarray}
	D_{x} & := & \prefix{x}{y}{(\binpar{\outputp{x}{y}}{@{y}})} \nonumber\\
	\bangp_{x}{P} & := & \binpar{{x}!\langle{\binpar{D_{x}}{P}}\rangle}{D_{x}} \nonumber
\end{eqnarray}

\begin{eqnarray}
	\bangp_{x}{P} & & \nonumber\\
	=
	& {x}!\langle{(\prefix{x}{y}{(\outputp{x}{y} | @{y})) | P}}\rangle 
	      | \prefix{x}{y}{(\outputp{x}{y} | @{y})} & \nonumber\\
	\red
	& (\outputp{x}{y} | @{y})\substn{\quotep{(\prefix{x}{y}{(@{y} | \outputp{x}{y})) | P}}}{y} & \nonumber\\
	=
	& \outputp{x}{\quotep{(\prefix{x}{y}{(\outputp{x}{y} | @{y})) | P}}}
	  | {(\prefix{x}{y}{(\outputp{x}{y} | @{y})) | P}} & \nonumber\\
	\red
	& \ldots & \nonumber\\
	\red^*
	& P | P | \ldots & \nonumber
\end{eqnarray}

Of course, this encoding, as an implementation, runs away, unfolding
$\bangp{P}$ eagerly. A lazier and more implementable replication
operator, restricted to input-guarded processes, may be obtained as follows.

\begin{eqnarray}
\bangp{\prefix{u}{v}{P}} 
	:= 
	\binpar{\lift{x}{\prefix{u}{v}{(\binpar{D(x)}{P})}}}{D(x)} \nonumber
\end{eqnarray}

\begin{remark}
  Note that the lazier definition still does not deal with summation
  or mixed summation (i.e. sums over input and output). The reader is
  invited to construct definitions of replication that deal with these
  features. 

  Further, the definitions are parameterized in a name, $x$. Can you,
  gentle reader, make a definition that eliminates this parameter and
  guarantees no accidental interaction between the replication
  machinery and the process being replicated -- i.e. no accidental
  sharing of names used by the process to get its work done and the
  name(s) used by the replication to effect copying. This latter
  revision of the definition of replication is crucial to obtaining
  the expected identity $!!P \sim !P$.
\end{remark}

\begin{remark}\label{rem:paradoxical_combinator}
  The reader familiar with the lambda calculus will have noticed the
  similarity between $D$ and the paradoxical combinator.

  [Ed. note: the existence of this seems to suggest we have to be more
  restrictive on the set of processes and names we admit if we are to
  support no-cloning.]
\end{remark}

\subsubsection{Bisimulation}

The computational dynamics gives rise to another kind of equivalence,
the equivalence of computational behavior. As previously mentioned
this is typically captured \emph{via} some form of bisimulation.

% The notion we use in this paper is weak barbed bisimulation
% \cite{milner91polyadicpi}.

The notion we use in this paper is derived from weak barbed
bisimulation \cite{milner91polyadicpi}. 

\begin{definition}
An \emph{observation relation}, $\downarrow_{\mathcal N}$, over a set
of names, $\mathcal N$, is the smallest relation satisfying the rules
below.

\infrule[Out-barb]{y \in {\mathcal N}, \; x \nameeq y}
		  {\outputp{x}{v} \downarrow_{\mathcal N} x}
\infrule[Par-barb]{\mbox{$P\downarrow_{\mathcal N} x$ or $Q\downarrow_{\mathcal N} x$}}
		  {\binpar{P}{Q} \downarrow_{\mathcal N} x}

We write $P \Downarrow_{\mathcal N} x$ if there is $Q$ such that 
$P \wred Q$ and $Q \downarrow_{\mathcal N} x$.
\end{definition}

\begin{definition}
%\label{def.bbisim}
An  ${\mathcal N}$-\emph{barbed bisimulation} over a set of names, ${\mathcal N}$, is a symmetric binary relation 
${\mathcal S}_{\mathcal N}$ between agents such that $P\rel{S}_{\mathcal N}Q$ implies:
\begin{enumerate}
\item If $P \red P'$ then $Q \wred Q'$ and $P'\rel{S}_{\mathcal N} Q'$.
\item If $P\downarrow_{\mathcal N} x$, then $Q\Downarrow_{\mathcal N} x$.
\end{enumerate}
$P$ is ${\mathcal N}$-barbed bisimilar to $Q$, written
$P \wbbisim_{\mathcal N} Q$, if $P \rel{S}_{\mathcal N} Q$ for some ${\mathcal N}$-barbed bisimulation ${\mathcal S}_{\mathcal N}$.
\end{definition}

$\mathcal{R} \subseteq \pi \times \pi$

$P \mathcal{R} Q => \forall P'. P \red P' \Rightarrow \exists Q'. Q \red Q', P' \mathcal{R} Q'$

$P \vdash x \Rightarrow Q \vdash x$

\begin{mathpar}
  \inferrule*[lab=Out-barb]{x \nameeq y}{{y}!\langle{Q}\rangle \vdash x}
  \and
  \inferrule*[lab=Par-barb]{\mbox{$P\vdash x$ or $Q\vdash x$}}{\binpar{P}{Q} \vdash x}
\end{mathpar}

\subsubsection{Contexts}

One of the principle advantages of computational calculi like the
$\pi$-calculus is a well-defined notion of context,
contextual-equivalence and a correlation between
contextual-equivalence and notions of bisimulation. The notion of
context allows the decomposition of a process into (sub-)process and
its syntactic environment, its context. Thus, a context may be
thought of as a process with a ``hole'' (written $\Box$) in it. The
application of a context $M$ to a process $P$, written $M[P]$, is
tantamount to filling the hole in $M$ with $P$. In this paper we do
not need the full weight of this theory, but do make use of the notion
of context in the proof the main theorem. 

\begin{mathpar}
  \inferrule* [lab=summation] {} {{M_{M},M_{N}} \bc \Box \;|\; x.M_{A} \;|\; M_{M}+M_{N}}
  \and
  \inferrule* [lab=agent] {} {{M_{A}} \bc (\vec{x})M_{P} \;| \; \clift{P_0,\ldots,M_{P},\ldots,P_N}}
  \and \\
  \inferrule* [lab=process] {} {{M_{P}} \bc M_{N} \;| \;P|M_{P} }
\end{mathpar} 

\begin{mathpar}
  \inferrule* [lab=sychronization] {} {M_{N} \bc \Box \;|\; x?M_{F} \;|\; x!M_{C}}
  \and
  \inferrule* [lab=abstraction] {} {{M_{F}} \bc (x)M_{P} }
  \and
  \inferrule* [lab=concretion] {} {{M_{C}} \bc \langle M_{P} \rangle }
  \and \\
  \inferrule* [lab=process] {} {{M_{P}} \bc M_{N} \;| \;P|M_{P} }
\end{mathpar}

\begin{definition}[contextual application] Given a context $M$, and
  process $P$, we define the \emph{contextual application}, $M[P] :=
  M\{P/\Box\}$. That is, the contextual application of M to P is the
  substitution of $P$ for $\Box$ in $M$.
\end{definition}

$\meaningof{-} : L \to \mathcal{P}(\pi)$

\begin{mathpar}
  \inferrule* [lab=collection] {} {\meaningof{true} = \pi, \and \meaningof{~E} = \pi \setminus \meaningof{E}, \and \meaningof{E_{1} \& E_{2}} = \meaningof{E_{1}} \cap \meaningof{E_{2}}}
\end{mathpar}

\begin{mathpar}
  \inferrule* [lab=structure] {} {\meaningof{0} = \{ P \in \pi | P \equiv 0 \}, \and \\ \meaningof{E_1 | E_2} = \{ P \in \pi | P \equiv P_{1} | P_{2}, P_{1} \in \meaningof{E_{1}}, P_{2} \in \meaningof{E_2}\} }
\end{mathpar}

\begin{mathpar}
 \inferrule* [lab=behavior] {} {\meaningof{\langle a?b \rangle E} = \{ P \in \pi | P \equiv Q | u?(y)P', \\ \and \\\\ \and \\ \;\;\; u \in \meaningof{a}, \forall z.P'\{z/y\} \in \meaningof{E\{z/b\}}\}, \and \\ \meaningof{a!E} = \{ P \in \pi | P \equiv Q | x!\langle P' \rangle, x \in \meaningof{a} P' \in \meaningof{E}\} }
\end{mathpar}

\begin{mathpar}
 \inferrule* [lab=nominal] {} {\meaningof{\quotep{E}} = \{ \quotep{P} \in \quotep{\pi} | P \in \meaningof{E} \}, \and \meaningof{\quotep{P}} = \{ \quotep{Q} \in \quotep{\pi} | P \equiv Q \} \and \\ \meaningof{@\quotep{E}} = \{ P \in \pi | P \equiv @x, x \in \meaningof{E} \}}
\end{mathpar}

\begin{eqnarray*}
  \\
  \meaningof{-} : TS \to ST
\end{eqnarray*}

\begin{eqnarray*}
  \\
  L : TS \to ST
\end{eqnarray*}

\begin{eqnarray*}
  \\
  P \models E \iff P \in \meaningof{E}
\end{eqnarray*}

\begin{eqnarray*}
  P \approx_{L} Q \iff \forall E \in L. P \models E \iff Q \models E
\end{eqnarray*}

\begin{eqnarray*}
  P \approx_{K} Q
\end{eqnarray*}

\begin{eqnarray*}
  P \approx Q
\end{eqnarray*}

$\approx_{K} = \approx = \approx_{L}$

\subsubsection{Contextual duality}

Note that contexts extend the quotation operation to a family of
operations from processes to names. Given a context, $M$, we can
define a \emph{nominal context}, $\quotep{M}$ by $\quotep{M}[P] :=
\quotep{M[P]}$. To foreshadow what is to come we observe that these
operations enjoy a duality with processes very much like the duality
between vectors and maps from vectors to scalars.

Further, because the calculus is essentially higher-order, we have a
correspondence between contexts and processes. More specifically,
given a name $x$ and a context $M$ we can construct $M^{*}_{x}$ such
that 

\begin{mathpar}
  M^{*}_{x} | \lift{x}{P} \red M[P]
\end{mathpar}

namely,

\begin{mathpar}
  M^{*}_{x} := x?(u).M[\dropn{u}]
\end{mathpar}

The dependence of $M^{*}_{x}$ on a name makes it an abstraction, 

\begin{mathpar}
  M^{*} := (x)x?(u).M[\dropn{u}]
\end{mathpar}

\subsection{Additional notation}

It will sometimes be convenient to denote the process a name
quotes. We already have the notation $x = \quotep{P}$, but it will be
convenient to introduce an alternate notation, $\procn{x}$, when we
want to emphasize the connection to the use of the name. Note that, by
virtue of name equivalence, $\quotep{\procn{x}} \nameeq x$; so, the
notation is consistent with previous definitions.

Further, because names have structure it is possible to effect
substitutions on the basis of that structure. This means we need to
upgrade our notation for substitutions, which we accomplish by
adapting comprehension notation. Thus,

\begin{mathpar}
  P\{ y / x : x \in S \}
\end{mathpar}

is interpreted to mean the process derived from P by replacing (in a
capture-avoiding manner) each occurrence of $x$ in $S$ by $y$. For example,

\begin{mathpar}
  P\{ \quotep{\procn{x}|\procn{x}} / x : x \in \freenames{P} \}
\end{mathpar}

will replace each (occurrence) of a free name $x$ in $P$ by
$\quotep{\procn{x}|\procn{x}}$.

Also, we will avail ourselves of the notation $x^{L}$ and $x^{R}$ to
denote injections of a name into disjoint copies of the name
space. There are numerous ways to accomplish this. One example can be
found in \cite{MeredithR05}. This notation overloads to vectors of
names: $\vec{x}^{\pi} := (x_{i}^{\pi} \; : \; 0 \leq i < |\vec{x}| )$ where $\pi \in \{L,R\}$.

We also use $P^{\Box} := P|\Box$.

In \cite{MeredithR05} an interpretation of the new operator is
given. It turns out that there are several possible interpretations
all enjoying the requisite algebraic properties of the operator (see
\cite{milner91polyadicpi}). We will therefore make liberal use of
$(\nu\; \vec{x})P$.

% subsection the_syntax_and_semantics_of_the_notation_system (end)   

\input{qm2pi.qmops} 

\input{qm2pi.sterngerlach} 

\input{qm2pi.metric} 

% section concurrent_process_calculi (end)

%\input{qm2pi.proofsketch}

% section proof sketch (end)

%\input{qm2pi.slviaknots} 

% section spatial logic via knots (end)

\input{qm2pi.conclusion}

% section conclusion (end)

%\input{qm2pi.dtcodes} 

% section wiring algorithm (end)

\input{qm2pi.ack} 

% section acknowledgments (end)

\newpage


\bibliographystyle{plain}   
\bibliography{../../biblios/main.bib}

\input{qm2pi.rhodetails}

\end{document}



\end{document}



\end{document}

 

%\documentclass[12pt]{llncs}
%\documentclass{jktr}

\usepackage[pdftex]{hyperref}                   
\usepackage {listings}
\usepackage {mathpartir}
\usepackage{bcprules}
%\usepackage{listings}
                       
\usepackage{graphicx} 
%\usepackage[margins=2.5cm,nohead,nofoot]{geometry}
%\usepackage{geometry}
\usepackage{amsfonts}
\usepackage{amstext}
\usepackage{latexsym}
\usepackage{amssymb}
\usepackage{color}


%\include{myPreamble}
\documentclass[12pt]{llncs}
%\documentclass{jktr}

\usepackage[pdftex]{hyperref}                   
\usepackage {listings}
\usepackage {mathpartir}
\usepackage{bcprules}
%\usepackage{listings}
                       
\usepackage{graphicx} 
%\usepackage[margins=2.5cm,nohead,nofoot]{geometry}
%\usepackage{geometry}
\usepackage{amsfonts}
\usepackage{amstext}
\usepackage{latexsym}
\usepackage{amssymb}
\usepackage{color}


%\include{myPreamble}
\documentclass[12pt]{llncs}
%\documentclass{jktr}

\usepackage[pdftex]{hyperref}                   
\usepackage {listings}
\usepackage {mathpartir}
\usepackage{bcprules}
%\usepackage{listings}
                       
\usepackage{graphicx} 
%\usepackage[margins=2.5cm,nohead,nofoot]{geometry}
%\usepackage{geometry}
\usepackage{amsfonts}
\usepackage{amstext}
\usepackage{latexsym}
\usepackage{amssymb}
\usepackage{color}


%\include{myPreamble}
\include{qm2pi.local} 

%\ifpdf
%\usepackage[pdftex]{graphicx}
%\else
%\usepackage{graphicx}
%\fi

 % \ifpdf
%  \usepackage{pdfsync}
%  \if


%\title{Brief Article}
%\author{David F. Snyder}
%\author{L.G. Meredith}

%\address{Dept. of Math., Texas State University--San Marcos, San Marcos, TX 78666}
       
\pagestyle{empty}


\begin{document}

\lstset{language=[Objective]Caml,frame=shadowbox}

\input{qm2pi.front}

% section front matter (end)

\input{qm2pi.intro} 
 
% section introduction (end)

% \input{qm2pi.knotations} 

% section notation (end)

\input{qm2pi.process.calculi} 

% section concurrent_process_calculi_and_spatial_logics_ (end)
    
%\input{qm2pi.knots2pi} 

%\input{qm2pi.trefoil} 

%\input{qm2pi.mainthm} 

% subsection basic_interpretation (end)

%\input{qm2pi.rho.presentation} 
\subsection{The syntax and semantics of the notation system}\label{sub:the_syntax_and_semantics_of_the_notation_system} % (fold)

We now summarize a technical presentation of the calculus that
embodies our theory of dynamics. The typical presentation of such a
calculus follows the style of giving generators and relations on
them. The grammar, below, describing term constructors, freely
generates the set of processes, $\Proc$. This set is then quotiented
by a relation known as structural congruence and it is over this set
that the notion of dynamics is expressed. This presentation is
essentially that of \cite{MeredithR05} with the addition of
polyadicity and summation. For readability we have relegated some of
the technical subtleties to an appendix.

\subsubsection{Process grammar}\label{subsub:process_grammar}

\begin{mathpar}
  \inferrule* [lab=synchronization] {} {{M} \bc \pzero \;|\; x?F \;|\; x!C }
  \and
  \inferrule* [lab=abstraction] {} {{F} \bc (x)P}
  \and
  \inferrule* [lab=concretion] {} {{C} \bc \langle Q \rangle}
  \and
  \inferrule* [lab=process] {} {{P,Q} \bc M \;| \;P|Q \;|\; @{x}}
  \and
  \inferrule* [lab=name] {} {{x} \bc \quotep{P}}
\end{mathpar} 

Note that $\vec{x}$ (resp. $\vec{P}$) denotes a vector of names
(resp. processes) of length $|\vec{x}|$ (resp. $|\vec{P}|$). We adopt
the following useful abbreviations.

\begin{mathpar}
   x?(\vec{y}).P := x.(\vec{y})P \and  x\clift{\vec{P}} := x.\clift{\vec{P}}
   \and x!(y) := \lift{x}{\dropn{y}}
   \and \Pi_{i=0}^{n-1}P_i := P_0 | \ldots | P_{n-1}
\end{mathpar}

\subsubsection{Structural congruence}

\paragraph{Free and bound names and alpha-equivalence.} At the
core of structural equivalence is alpha-equivalence which identifies
process that are the same up to a change of variable. Formally, we
recognize the distinction between free and bound names. The free names
of a process, $\freenames{P}$, may be calculated recursively as
follows:

\begin{mathpar}
\freenames{\pzero} := \emptyset
  \and \\
  \freenames{x?(y).P} := \{ x \} \cup (\freenames{P} \setminus \{ y \})
  \and 
  \freenames{x!\langle P \rangle} := \{ x \} \cup \{ P \} 
  \and \\
  \freenames{P|Q} := \freenames{P} \cup \freenames{Q}
  \and \\
  \freenames{@{x}} := \{ x \}
\end{mathpar}

$\pi$
$\quotep{\pi}$

$\freenames{-} : \pi \to \mathcal{P}(\quotep{\pi})$

\begin{eqnarray*}
  \freenames{\pzero} & := & \emptyset \\
  \freenames{x?(y).P} & := & \{ x \} \cup (\freenames{P} \setminus \{ y \}) \\
  \freenames{x!\langle P \rangle} & := & \{ x \} \cup \{ P \} \\
  \freenames{P|Q} & := & \freenames{P} \cup \freenames{Q} \\
  \freenames{\dropn{x}} & := & \{ x \}
\end{eqnarray*}

The bound names of a process, $\boundnames{P}$, are those names occurring in $P$
that are not free. For example, in $x?(y).0$, the name $x$ is free, while $y$ is bound.

\begin{mathpar}
  \inferrule* [lab=monoidal-laws] {} { P|Q \equiv Q|P \and P|0 \equiv P \and P|(Q|R) \equiv (P|Q)|R }
\end{mathpar}

\begin{mathpar}
  \inferrule* [lab=alpha-equivalence] {} { (x)P \equiv (y)P\{y/x\} \and y \not\in \freenames{P} }
\end{mathpar}

\begin{definition}
Then two processes, $P,Q$, are alpha-equivalent if $P = Q\{\vec{y}/\vec{x}\}$ for
some $\vec{x} \in \boundnames{Q},\vec{y} \in \boundnames{P}$, where $Q\{\vec{y}/\vec{x}\}$
denotes the capture-avoiding substitution of $\vec{y}$ for $\vec{x}$ in $Q$.
\end{definition}

\begin{definition}
  The {\em structural congruence} \cite{SangiorgiWalker} , $\equiv$,
  between processes is the least congruence containing
  alpha-equivalence, satisfying the abelian monoid laws
  (associativity, commutativity and $\pzero$ as identity) for parallel
  composition $|$ and for summation $+$.
\end{definition}

\subsection{Name equivalence}

We take name equivalence, written $\nameeq$, to be the smallest
equivalence relation generated by the following rules.

\begin{mathpar}
\inferrule*[lab=Quote-drop]
{ }
{ \quotep{@{x}} \nameeq x }

\inferrule*[lab=Struct-equiv]
{ P \scong Q }
{ \quotep{P} \nameeq \quotep{Q} }
\end{mathpar}

The astute reader will have noticed that the mutual recursion of names
and processes imposes a mutual recursion on alpha-equivalence and
structural equivalence via name-equivalence. Fortunately, all of this
works out pleasantly and we may calculate in the natural way, free of
concern. The reader interested in the details is referred to the
appendix \ref{appendix:rho_details}.

\subsection{Substitution}

We use $\Proc$ for the set of processes, $\QProc$ for the set of
names, and $\id{\{}\vec{y} / \vec{x} \id{\}}$ to denote partial maps,
$s : \QProc \rightarrow \QProc$. A map, $s$ lifts, uniquely, to a map
on process terms, $\widehat{s} : \Proc \rightarrow \Proc$ by the
following equations.

\begin{mathpar}
  (0) \psubstp{Q}{P} := 0 \\
  (R \juxtap S) \psubstp{Q}{P}
  :=    
  (R)\psubstp{Q}{P} \juxtap (S) \psubstp{Q}{P} \\
  (x?(y).R) \psubstp{Q}{P}    
  :=    
  (x)\substp{Q}{P} (z)\concat( (R \psubstn{z}{y}) \psubstp{Q}{P} ) \\
  (\lift{x}{R}) \psubstp{Q}{P}  
  :=
  \lift{(x)\substp{Q}{P}}{ R \psubstp{Q}{P} } \\
%   (\dropn{x})  \psubstp{Q}{P}       
%   := 
%   \left\{ 
%     \begin{array}{ccc} 
%       \dropn{\quotep{Q}} & & x \nameeq \quotep{P} \\
%       \dropn{x} & & otherwise \\
%     \end{array}
%   \right. 
  (\dropn{x})  \psubstp{Q}{P}       
  := 
  \left\{ 
    \begin{array}{ccc} 
      Q & & x \nameeq \quotep{P} \\
      \dropn{x} & & otherwise \\
    \end{array}
  \right.
\end{mathpar}
 

where

\begin{eqnarray}
  (x)\id{\{} \lpquote Q \rpquote / \lpquote P \rpquote \id{\}}            = 
  \left\{ 
    \begin{array}{ccc}
      \lpquote Q \rpquote & & x \nameeq \lpquote P \rpquote \\
      x & & otherwise \\
    \end{array}
  \right. \nonumber
\end{eqnarray}

and $z$ is chosen distinct from $\quotep{P}$, $\quotep{Q}$, the free
names in $Q$, and all the names in $R$. Our $\alpha$-equivalence will
be built in the standard way from this substitution.

\begin{remark}\label{rem:no_self_referential_names}
  One consequence of these definitions is that $\forall P. \quotep{P}
  \not\in \freenames{P}$.
\end{remark}

\subsection{ Dynamic quote: an example }

Anticipating something of what's to come, consider applying the
substitution, $\widehat{\id{\{}u / z \id{\}}}$, to the following pair
of processes, $\lift{w}{y!(z)}$ and $w[ \lpquote y!(z) \rpquote ]$.

\begin{eqnarray}
	\lift{w}{y!(z)}\widehat{\id{\{}u / z \id{\}}}
		& = &
		\lift{w}{y!(u)} \nonumber\\
	w[ \lpquote y!(z) \rpquote ] \widehat{ \id{\{}u / z \id{\}} }
		& = &
		w[ \lpquote y!(z) \rpquote ] \nonumber
\end{eqnarray}

Because the body of the process between quotes is impervious to
substitution, we get radically different answers. In fact, by
examining the first process in an input context,
e.g. $x?(z).\lift{w}{y!(z)}$, we see that the process under the lift
operator may be shaped by prefixed inputs binding a name inside it. In
this sense, the lift operator will be seen as a way to dynamically
construct processes before reifying them as names.

Finally equipped with these standard features we can present the
dynamics of the calculus.

\subsubsection{Operational semantics} 

Finally, we introduce the computational dynamics. What marks these
algebras as distinct from other more traditionally studied algebraic
structures, e.g. vector spaces or polynomial rings, is the manner in
which dynamics is captured. In traditional structures, dynamics is typically
expressed through morphisms between such structures, as in linear maps
between vector spaces or morphisms between rings. In algebras
associated with the semantics of computation, the dynamics is
expressed as part of the algebraic structure itself, through a
reduction reduction relation typically denoted by $\red$. Below, we
give a recursive presentation of this relation for the calculus used
in the encoding.

$\red \subseteq \pi \times \pi$
$\red : \pi \to \mathcal{P}(\pi)$

\begin{mathpar}
  \inferrule* [lab=Comm] { \textsf{match}( x_{src}, x_{trgt} ) } { x_{trgt}?(y)P \; | \; x_{src}!\langle {Q} \rangle \red P\{\quotep{Q}/y}\} }
  \and \\
  \inferrule* [lab=Par] {{P} \red {P}'} {{{P} | {Q}} \red {{P}' | {Q}}}
  \and
  \inferrule* [lab=Equiv]{{{P} \scong {P}'} \andalso {{P}' \red {Q}'} \andalso {{Q}' \scong {Q}}}{{P} \red {Q}}
\end{mathpar}

\begin{eqnarray*}
  match_{\equiv} (\quotep{P},\quotep{Q}) & := & P \equiv Q \\
  match_{\dagger}(\quotep{P},\quotep{Q}) & := & \forall R. P|Q \red^{*} R => R \red^{*} 0 \\
  match_{K}(\quotep{P},\quotep{Q}) & := & K \mbox{ for some context } K
\end{eqnarray*}

$u?(x)P | u!\langle Q \rangle \red P\{\quotep{Q}/x\}$

%We write $\wred$ for $\red^*$, and $P\red$ if $\exists Q $ such that $ P \red Q$.
We write $P\red$ if $\exists Q $ such that $ P \red Q$ and $P\not\red$, otherwise.

\section{Replication}

As mentioned before, it is known that replication (and hence
recursion) can be implemented in a higher-order process algebra
\cite{SangiorgiWalker}. As our first example of calculation with the
machinery thus far presented we give the construction explicitly in
the {\rhoc}.

\begin{eqnarray}
	D_{x} & := & \prefix{x}{y}{(\binpar{\outputp{x}{y}}{@{y}})} \nonumber\\
	\bangp_{x}{P} & := & \binpar{{x}!\langle{\binpar{D_{x}}{P}}\rangle}{D_{x}} \nonumber
\end{eqnarray}

\begin{eqnarray}
	\bangp_{x}{P} & & \nonumber\\
	=
	& {x}!\langle{(\prefix{x}{y}{(\outputp{x}{y} | @{y})) | P}}\rangle 
	      | \prefix{x}{y}{(\outputp{x}{y} | @{y})} & \nonumber\\
	\red
	& (\outputp{x}{y} | @{y})\substn{\quotep{(\prefix{x}{y}{(@{y} | \outputp{x}{y})) | P}}}{y} & \nonumber\\
	=
	& \outputp{x}{\quotep{(\prefix{x}{y}{(\outputp{x}{y} | @{y})) | P}}}
	  | {(\prefix{x}{y}{(\outputp{x}{y} | @{y})) | P}} & \nonumber\\
	\red
	& \ldots & \nonumber\\
	\red^*
	& P | P | \ldots & \nonumber
\end{eqnarray}

Of course, this encoding, as an implementation, runs away, unfolding
$\bangp{P}$ eagerly. A lazier and more implementable replication
operator, restricted to input-guarded processes, may be obtained as follows.

\begin{eqnarray}
\bangp{\prefix{u}{v}{P}} 
	:= 
	\binpar{\lift{x}{\prefix{u}{v}{(\binpar{D(x)}{P})}}}{D(x)} \nonumber
\end{eqnarray}

\begin{remark}
  Note that the lazier definition still does not deal with summation
  or mixed summation (i.e. sums over input and output). The reader is
  invited to construct definitions of replication that deal with these
  features. 

  Further, the definitions are parameterized in a name, $x$. Can you,
  gentle reader, make a definition that eliminates this parameter and
  guarantees no accidental interaction between the replication
  machinery and the process being replicated -- i.e. no accidental
  sharing of names used by the process to get its work done and the
  name(s) used by the replication to effect copying. This latter
  revision of the definition of replication is crucial to obtaining
  the expected identity $!!P \sim !P$.
\end{remark}

\begin{remark}\label{rem:paradoxical_combinator}
  The reader familiar with the lambda calculus will have noticed the
  similarity between $D$ and the paradoxical combinator.

  [Ed. note: the existence of this seems to suggest we have to be more
  restrictive on the set of processes and names we admit if we are to
  support no-cloning.]
\end{remark}

\subsubsection{Bisimulation}

The computational dynamics gives rise to another kind of equivalence,
the equivalence of computational behavior. As previously mentioned
this is typically captured \emph{via} some form of bisimulation.

% The notion we use in this paper is weak barbed bisimulation
% \cite{milner91polyadicpi}.

The notion we use in this paper is derived from weak barbed
bisimulation \cite{milner91polyadicpi}. 

\begin{definition}
An \emph{observation relation}, $\downarrow_{\mathcal N}$, over a set
of names, $\mathcal N$, is the smallest relation satisfying the rules
below.

\infrule[Out-barb]{y \in {\mathcal N}, \; x \nameeq y}
		  {\outputp{x}{v} \downarrow_{\mathcal N} x}
\infrule[Par-barb]{\mbox{$P\downarrow_{\mathcal N} x$ or $Q\downarrow_{\mathcal N} x$}}
		  {\binpar{P}{Q} \downarrow_{\mathcal N} x}

We write $P \Downarrow_{\mathcal N} x$ if there is $Q$ such that 
$P \wred Q$ and $Q \downarrow_{\mathcal N} x$.
\end{definition}

\begin{definition}
%\label{def.bbisim}
An  ${\mathcal N}$-\emph{barbed bisimulation} over a set of names, ${\mathcal N}$, is a symmetric binary relation 
${\mathcal S}_{\mathcal N}$ between agents such that $P\rel{S}_{\mathcal N}Q$ implies:
\begin{enumerate}
\item If $P \red P'$ then $Q \wred Q'$ and $P'\rel{S}_{\mathcal N} Q'$.
\item If $P\downarrow_{\mathcal N} x$, then $Q\Downarrow_{\mathcal N} x$.
\end{enumerate}
$P$ is ${\mathcal N}$-barbed bisimilar to $Q$, written
$P \wbbisim_{\mathcal N} Q$, if $P \rel{S}_{\mathcal N} Q$ for some ${\mathcal N}$-barbed bisimulation ${\mathcal S}_{\mathcal N}$.
\end{definition}

$\mathcal{R} \subseteq \pi \times \pi$

$P \mathcal{R} Q => \forall P'. P \red P' \Rightarrow \exists Q'. Q \red Q', P' \mathcal{R} Q'$

$P \vdash x \Rightarrow Q \vdash x$

\begin{mathpar}
  \inferrule*[lab=Out-barb]{x \nameeq y}{{y}!\langle{Q}\rangle \vdash x}
  \and
  \inferrule*[lab=Par-barb]{\mbox{$P\vdash x$ or $Q\vdash x$}}{\binpar{P}{Q} \vdash x}
\end{mathpar}

\subsubsection{Contexts}

One of the principle advantages of computational calculi like the
$\pi$-calculus is a well-defined notion of context,
contextual-equivalence and a correlation between
contextual-equivalence and notions of bisimulation. The notion of
context allows the decomposition of a process into (sub-)process and
its syntactic environment, its context. Thus, a context may be
thought of as a process with a ``hole'' (written $\Box$) in it. The
application of a context $M$ to a process $P$, written $M[P]$, is
tantamount to filling the hole in $M$ with $P$. In this paper we do
not need the full weight of this theory, but do make use of the notion
of context in the proof the main theorem. 

\begin{mathpar}
  \inferrule* [lab=summation] {} {{M_{M},M_{N}} \bc \Box \;|\; x.M_{A} \;|\; M_{M}+M_{N}}
  \and
  \inferrule* [lab=agent] {} {{M_{A}} \bc (\vec{x})M_{P} \;| \; \clift{P_0,\ldots,M_{P},\ldots,P_N}}
  \and \\
  \inferrule* [lab=process] {} {{M_{P}} \bc M_{N} \;| \;P|M_{P} }
\end{mathpar} 

\begin{mathpar}
  \inferrule* [lab=sychronization] {} {M_{N} \bc \Box \;|\; x?M_{F} \;|\; x!M_{C}}
  \and
  \inferrule* [lab=abstraction] {} {{M_{F}} \bc (x)M_{P} }
  \and
  \inferrule* [lab=concretion] {} {{M_{C}} \bc \langle M_{P} \rangle }
  \and \\
  \inferrule* [lab=process] {} {{M_{P}} \bc M_{N} \;| \;P|M_{P} }
\end{mathpar}

\begin{definition}[contextual application] Given a context $M$, and
  process $P$, we define the \emph{contextual application}, $M[P] :=
  M\{P/\Box\}$. That is, the contextual application of M to P is the
  substitution of $P$ for $\Box$ in $M$.
\end{definition}

$\meaningof{-} : L \to \mathcal{P}(\pi)$

\begin{mathpar}
  \inferrule* [lab=collection] {} {\meaningof{true} = \pi, \and \meaningof{~E} = \pi \setminus \meaningof{E}, \and \meaningof{E_{1} \& E_{2}} = \meaningof{E_{1}} \cap \meaningof{E_{2}}}
\end{mathpar}

\begin{mathpar}
  \inferrule* [lab=structure] {} {\meaningof{0} = \{ P \in \pi | P \equiv 0 \}, \and \\ \meaningof{E_1 | E_2} = \{ P \in \pi | P \equiv P_{1} | P_{2}, P_{1} \in \meaningof{E_{1}}, P_{2} \in \meaningof{E_2}\} }
\end{mathpar}

\begin{mathpar}
 \inferrule* [lab=behavior] {} {\meaningof{\langle a?b \rangle E} = \{ P \in \pi | P \equiv Q | u?(y)P', \\ \and \\\\ \and \\ \;\;\; u \in \meaningof{a}, \forall z.P'\{z/y\} \in \meaningof{E\{z/b\}}\}, \and \\ \meaningof{a!E} = \{ P \in \pi | P \equiv Q | x!\langle P' \rangle, x \in \meaningof{a} P' \in \meaningof{E}\} }
\end{mathpar}

\begin{mathpar}
 \inferrule* [lab=nominal] {} {\meaningof{\quotep{E}} = \{ \quotep{P} \in \quotep{\pi} | P \in \meaningof{E} \}, \and \meaningof{\quotep{P}} = \{ \quotep{Q} \in \quotep{\pi} | P \equiv Q \} \and \\ \meaningof{@\quotep{E}} = \{ P \in \pi | P \equiv @x, x \in \meaningof{E} \}}
\end{mathpar}

\begin{eqnarray*}
  \\
  \meaningof{-} : TS \to ST
\end{eqnarray*}

\begin{eqnarray*}
  \\
  L : TS \to ST
\end{eqnarray*}

\begin{eqnarray*}
  \\
  P \models E \iff P \in \meaningof{E}
\end{eqnarray*}

\begin{eqnarray*}
  P \approx_{L} Q \iff \forall E \in L. P \models E \iff Q \models E
\end{eqnarray*}

\begin{eqnarray*}
  P \approx_{K} Q
\end{eqnarray*}

\begin{eqnarray*}
  P \approx Q
\end{eqnarray*}

$\approx_{K} = \approx = \approx_{L}$

\subsubsection{Contextual duality}

Note that contexts extend the quotation operation to a family of
operations from processes to names. Given a context, $M$, we can
define a \emph{nominal context}, $\quotep{M}$ by $\quotep{M}[P] :=
\quotep{M[P]}$. To foreshadow what is to come we observe that these
operations enjoy a duality with processes very much like the duality
between vectors and maps from vectors to scalars.

Further, because the calculus is essentially higher-order, we have a
correspondence between contexts and processes. More specifically,
given a name $x$ and a context $M$ we can construct $M^{*}_{x}$ such
that 

\begin{mathpar}
  M^{*}_{x} | \lift{x}{P} \red M[P]
\end{mathpar}

namely,

\begin{mathpar}
  M^{*}_{x} := x?(u).M[\dropn{u}]
\end{mathpar}

The dependence of $M^{*}_{x}$ on a name makes it an abstraction, 

\begin{mathpar}
  M^{*} := (x)x?(u).M[\dropn{u}]
\end{mathpar}

\subsection{Additional notation}

It will sometimes be convenient to denote the process a name
quotes. We already have the notation $x = \quotep{P}$, but it will be
convenient to introduce an alternate notation, $\procn{x}$, when we
want to emphasize the connection to the use of the name. Note that, by
virtue of name equivalence, $\quotep{\procn{x}} \nameeq x$; so, the
notation is consistent with previous definitions.

Further, because names have structure it is possible to effect
substitutions on the basis of that structure. This means we need to
upgrade our notation for substitutions, which we accomplish by
adapting comprehension notation. Thus,

\begin{mathpar}
  P\{ y / x : x \in S \}
\end{mathpar}

is interpreted to mean the process derived from P by replacing (in a
capture-avoiding manner) each occurrence of $x$ in $S$ by $y$. For example,

\begin{mathpar}
  P\{ \quotep{\procn{x}|\procn{x}} / x : x \in \freenames{P} \}
\end{mathpar}

will replace each (occurrence) of a free name $x$ in $P$ by
$\quotep{\procn{x}|\procn{x}}$.

Also, we will avail ourselves of the notation $x^{L}$ and $x^{R}$ to
denote injections of a name into disjoint copies of the name
space. There are numerous ways to accomplish this. One example can be
found in \cite{MeredithR05}. This notation overloads to vectors of
names: $\vec{x}^{\pi} := (x_{i}^{\pi} \; : \; 0 \leq i < |\vec{x}| )$ where $\pi \in \{L,R\}$.

We also use $P^{\Box} := P|\Box$.

In \cite{MeredithR05} an interpretation of the new operator is
given. It turns out that there are several possible interpretations
all enjoying the requisite algebraic properties of the operator (see
\cite{milner91polyadicpi}). We will therefore make liberal use of
$(\nu\; \vec{x})P$.

% subsection the_syntax_and_semantics_of_the_notation_system (end)   

\input{qm2pi.qmops} 

\input{qm2pi.sterngerlach} 

\input{qm2pi.metric} 

% section concurrent_process_calculi (end)

%\input{qm2pi.proofsketch}

% section proof sketch (end)

%\input{qm2pi.slviaknots} 

% section spatial logic via knots (end)

\input{qm2pi.conclusion}

% section conclusion (end)

%\input{qm2pi.dtcodes} 

% section wiring algorithm (end)

\input{qm2pi.ack} 

% section acknowledgments (end)

\newpage


\bibliographystyle{plain}   
\bibliography{../../biblios/main.bib}

\input{qm2pi.rhodetails}

\end{document}

 

%\ifpdf
%\usepackage[pdftex]{graphicx}
%\else
%\usepackage{graphicx}
%\fi

 % \ifpdf
%  \usepackage{pdfsync}
%  \if


%\title{Brief Article}
%\author{David F. Snyder}
%\author{L.G. Meredith}

%\address{Dept. of Math., Texas State University--San Marcos, San Marcos, TX 78666}
       
\pagestyle{empty}


\begin{document}

\lstset{language=[Objective]Caml,frame=shadowbox}

\documentclass[12pt]{llncs}
%\documentclass{jktr}

\usepackage[pdftex]{hyperref}                   
\usepackage {listings}
\usepackage {mathpartir}
\usepackage{bcprules}
%\usepackage{listings}
                       
\usepackage{graphicx} 
%\usepackage[margins=2.5cm,nohead,nofoot]{geometry}
%\usepackage{geometry}
\usepackage{amsfonts}
\usepackage{amstext}
\usepackage{latexsym}
\usepackage{amssymb}
\usepackage{color}


%\include{myPreamble}
\include{qm2pi.local} 

%\ifpdf
%\usepackage[pdftex]{graphicx}
%\else
%\usepackage{graphicx}
%\fi

 % \ifpdf
%  \usepackage{pdfsync}
%  \if


%\title{Brief Article}
%\author{David F. Snyder}
%\author{L.G. Meredith}

%\address{Dept. of Math., Texas State University--San Marcos, San Marcos, TX 78666}
       
\pagestyle{empty}


\begin{document}

\lstset{language=[Objective]Caml,frame=shadowbox}

\input{qm2pi.front}

% section front matter (end)

\input{qm2pi.intro} 
 
% section introduction (end)

% \input{qm2pi.knotations} 

% section notation (end)

\input{qm2pi.process.calculi} 

% section concurrent_process_calculi_and_spatial_logics_ (end)
    
%\input{qm2pi.knots2pi} 

%\input{qm2pi.trefoil} 

%\input{qm2pi.mainthm} 

% subsection basic_interpretation (end)

%\input{qm2pi.rho.presentation} 
\subsection{The syntax and semantics of the notation system}\label{sub:the_syntax_and_semantics_of_the_notation_system} % (fold)

We now summarize a technical presentation of the calculus that
embodies our theory of dynamics. The typical presentation of such a
calculus follows the style of giving generators and relations on
them. The grammar, below, describing term constructors, freely
generates the set of processes, $\Proc$. This set is then quotiented
by a relation known as structural congruence and it is over this set
that the notion of dynamics is expressed. This presentation is
essentially that of \cite{MeredithR05} with the addition of
polyadicity and summation. For readability we have relegated some of
the technical subtleties to an appendix.

\subsubsection{Process grammar}\label{subsub:process_grammar}

\begin{mathpar}
  \inferrule* [lab=synchronization] {} {{M} \bc \pzero \;|\; x?F \;|\; x!C }
  \and
  \inferrule* [lab=abstraction] {} {{F} \bc (x)P}
  \and
  \inferrule* [lab=concretion] {} {{C} \bc \langle Q \rangle}
  \and
  \inferrule* [lab=process] {} {{P,Q} \bc M \;| \;P|Q \;|\; @{x}}
  \and
  \inferrule* [lab=name] {} {{x} \bc \quotep{P}}
\end{mathpar} 

Note that $\vec{x}$ (resp. $\vec{P}$) denotes a vector of names
(resp. processes) of length $|\vec{x}|$ (resp. $|\vec{P}|$). We adopt
the following useful abbreviations.

\begin{mathpar}
   x?(\vec{y}).P := x.(\vec{y})P \and  x\clift{\vec{P}} := x.\clift{\vec{P}}
   \and x!(y) := \lift{x}{\dropn{y}}
   \and \Pi_{i=0}^{n-1}P_i := P_0 | \ldots | P_{n-1}
\end{mathpar}

\subsubsection{Structural congruence}

\paragraph{Free and bound names and alpha-equivalence.} At the
core of structural equivalence is alpha-equivalence which identifies
process that are the same up to a change of variable. Formally, we
recognize the distinction between free and bound names. The free names
of a process, $\freenames{P}$, may be calculated recursively as
follows:

\begin{mathpar}
\freenames{\pzero} := \emptyset
  \and \\
  \freenames{x?(y).P} := \{ x \} \cup (\freenames{P} \setminus \{ y \})
  \and 
  \freenames{x!\langle P \rangle} := \{ x \} \cup \{ P \} 
  \and \\
  \freenames{P|Q} := \freenames{P} \cup \freenames{Q}
  \and \\
  \freenames{@{x}} := \{ x \}
\end{mathpar}

$\pi$
$\quotep{\pi}$

$\freenames{-} : \pi \to \mathcal{P}(\quotep{\pi})$

\begin{eqnarray*}
  \freenames{\pzero} & := & \emptyset \\
  \freenames{x?(y).P} & := & \{ x \} \cup (\freenames{P} \setminus \{ y \}) \\
  \freenames{x!\langle P \rangle} & := & \{ x \} \cup \{ P \} \\
  \freenames{P|Q} & := & \freenames{P} \cup \freenames{Q} \\
  \freenames{\dropn{x}} & := & \{ x \}
\end{eqnarray*}

The bound names of a process, $\boundnames{P}$, are those names occurring in $P$
that are not free. For example, in $x?(y).0$, the name $x$ is free, while $y$ is bound.

\begin{mathpar}
  \inferrule* [lab=monoidal-laws] {} { P|Q \equiv Q|P \and P|0 \equiv P \and P|(Q|R) \equiv (P|Q)|R }
\end{mathpar}

\begin{mathpar}
  \inferrule* [lab=alpha-equivalence] {} { (x)P \equiv (y)P\{y/x\} \and y \not\in \freenames{P} }
\end{mathpar}

\begin{definition}
Then two processes, $P,Q$, are alpha-equivalent if $P = Q\{\vec{y}/\vec{x}\}$ for
some $\vec{x} \in \boundnames{Q},\vec{y} \in \boundnames{P}$, where $Q\{\vec{y}/\vec{x}\}$
denotes the capture-avoiding substitution of $\vec{y}$ for $\vec{x}$ in $Q$.
\end{definition}

\begin{definition}
  The {\em structural congruence} \cite{SangiorgiWalker} , $\equiv$,
  between processes is the least congruence containing
  alpha-equivalence, satisfying the abelian monoid laws
  (associativity, commutativity and $\pzero$ as identity) for parallel
  composition $|$ and for summation $+$.
\end{definition}

\subsection{Name equivalence}

We take name equivalence, written $\nameeq$, to be the smallest
equivalence relation generated by the following rules.

\begin{mathpar}
\inferrule*[lab=Quote-drop]
{ }
{ \quotep{@{x}} \nameeq x }

\inferrule*[lab=Struct-equiv]
{ P \scong Q }
{ \quotep{P} \nameeq \quotep{Q} }
\end{mathpar}

The astute reader will have noticed that the mutual recursion of names
and processes imposes a mutual recursion on alpha-equivalence and
structural equivalence via name-equivalence. Fortunately, all of this
works out pleasantly and we may calculate in the natural way, free of
concern. The reader interested in the details is referred to the
appendix \ref{appendix:rho_details}.

\subsection{Substitution}

We use $\Proc$ for the set of processes, $\QProc$ for the set of
names, and $\id{\{}\vec{y} / \vec{x} \id{\}}$ to denote partial maps,
$s : \QProc \rightarrow \QProc$. A map, $s$ lifts, uniquely, to a map
on process terms, $\widehat{s} : \Proc \rightarrow \Proc$ by the
following equations.

\begin{mathpar}
  (0) \psubstp{Q}{P} := 0 \\
  (R \juxtap S) \psubstp{Q}{P}
  :=    
  (R)\psubstp{Q}{P} \juxtap (S) \psubstp{Q}{P} \\
  (x?(y).R) \psubstp{Q}{P}    
  :=    
  (x)\substp{Q}{P} (z)\concat( (R \psubstn{z}{y}) \psubstp{Q}{P} ) \\
  (\lift{x}{R}) \psubstp{Q}{P}  
  :=
  \lift{(x)\substp{Q}{P}}{ R \psubstp{Q}{P} } \\
%   (\dropn{x})  \psubstp{Q}{P}       
%   := 
%   \left\{ 
%     \begin{array}{ccc} 
%       \dropn{\quotep{Q}} & & x \nameeq \quotep{P} \\
%       \dropn{x} & & otherwise \\
%     \end{array}
%   \right. 
  (\dropn{x})  \psubstp{Q}{P}       
  := 
  \left\{ 
    \begin{array}{ccc} 
      Q & & x \nameeq \quotep{P} \\
      \dropn{x} & & otherwise \\
    \end{array}
  \right.
\end{mathpar}
 

where

\begin{eqnarray}
  (x)\id{\{} \lpquote Q \rpquote / \lpquote P \rpquote \id{\}}            = 
  \left\{ 
    \begin{array}{ccc}
      \lpquote Q \rpquote & & x \nameeq \lpquote P \rpquote \\
      x & & otherwise \\
    \end{array}
  \right. \nonumber
\end{eqnarray}

and $z$ is chosen distinct from $\quotep{P}$, $\quotep{Q}$, the free
names in $Q$, and all the names in $R$. Our $\alpha$-equivalence will
be built in the standard way from this substitution.

\begin{remark}\label{rem:no_self_referential_names}
  One consequence of these definitions is that $\forall P. \quotep{P}
  \not\in \freenames{P}$.
\end{remark}

\subsection{ Dynamic quote: an example }

Anticipating something of what's to come, consider applying the
substitution, $\widehat{\id{\{}u / z \id{\}}}$, to the following pair
of processes, $\lift{w}{y!(z)}$ and $w[ \lpquote y!(z) \rpquote ]$.

\begin{eqnarray}
	\lift{w}{y!(z)}\widehat{\id{\{}u / z \id{\}}}
		& = &
		\lift{w}{y!(u)} \nonumber\\
	w[ \lpquote y!(z) \rpquote ] \widehat{ \id{\{}u / z \id{\}} }
		& = &
		w[ \lpquote y!(z) \rpquote ] \nonumber
\end{eqnarray}

Because the body of the process between quotes is impervious to
substitution, we get radically different answers. In fact, by
examining the first process in an input context,
e.g. $x?(z).\lift{w}{y!(z)}$, we see that the process under the lift
operator may be shaped by prefixed inputs binding a name inside it. In
this sense, the lift operator will be seen as a way to dynamically
construct processes before reifying them as names.

Finally equipped with these standard features we can present the
dynamics of the calculus.

\subsubsection{Operational semantics} 

Finally, we introduce the computational dynamics. What marks these
algebras as distinct from other more traditionally studied algebraic
structures, e.g. vector spaces or polynomial rings, is the manner in
which dynamics is captured. In traditional structures, dynamics is typically
expressed through morphisms between such structures, as in linear maps
between vector spaces or morphisms between rings. In algebras
associated with the semantics of computation, the dynamics is
expressed as part of the algebraic structure itself, through a
reduction reduction relation typically denoted by $\red$. Below, we
give a recursive presentation of this relation for the calculus used
in the encoding.

$\red \subseteq \pi \times \pi$
$\red : \pi \to \mathcal{P}(\pi)$

\begin{mathpar}
  \inferrule* [lab=Comm] { \textsf{match}( x_{src}, x_{trgt} ) } { x_{trgt}?(y)P \; | \; x_{src}!\langle {Q} \rangle \red P\{\quotep{Q}/y}\} }
  \and \\
  \inferrule* [lab=Par] {{P} \red {P}'} {{{P} | {Q}} \red {{P}' | {Q}}}
  \and
  \inferrule* [lab=Equiv]{{{P} \scong {P}'} \andalso {{P}' \red {Q}'} \andalso {{Q}' \scong {Q}}}{{P} \red {Q}}
\end{mathpar}

\begin{eqnarray*}
  match_{\equiv} (\quotep{P},\quotep{Q}) & := & P \equiv Q \\
  match_{\dagger}(\quotep{P},\quotep{Q}) & := & \forall R. P|Q \red^{*} R => R \red^{*} 0 \\
  match_{K}(\quotep{P},\quotep{Q}) & := & K \mbox{ for some context } K
\end{eqnarray*}

$u?(x)P | u!\langle Q \rangle \red P\{\quotep{Q}/x\}$

%We write $\wred$ for $\red^*$, and $P\red$ if $\exists Q $ such that $ P \red Q$.
We write $P\red$ if $\exists Q $ such that $ P \red Q$ and $P\not\red$, otherwise.

\section{Replication}

As mentioned before, it is known that replication (and hence
recursion) can be implemented in a higher-order process algebra
\cite{SangiorgiWalker}. As our first example of calculation with the
machinery thus far presented we give the construction explicitly in
the {\rhoc}.

\begin{eqnarray}
	D_{x} & := & \prefix{x}{y}{(\binpar{\outputp{x}{y}}{@{y}})} \nonumber\\
	\bangp_{x}{P} & := & \binpar{{x}!\langle{\binpar{D_{x}}{P}}\rangle}{D_{x}} \nonumber
\end{eqnarray}

\begin{eqnarray}
	\bangp_{x}{P} & & \nonumber\\
	=
	& {x}!\langle{(\prefix{x}{y}{(\outputp{x}{y} | @{y})) | P}}\rangle 
	      | \prefix{x}{y}{(\outputp{x}{y} | @{y})} & \nonumber\\
	\red
	& (\outputp{x}{y} | @{y})\substn{\quotep{(\prefix{x}{y}{(@{y} | \outputp{x}{y})) | P}}}{y} & \nonumber\\
	=
	& \outputp{x}{\quotep{(\prefix{x}{y}{(\outputp{x}{y} | @{y})) | P}}}
	  | {(\prefix{x}{y}{(\outputp{x}{y} | @{y})) | P}} & \nonumber\\
	\red
	& \ldots & \nonumber\\
	\red^*
	& P | P | \ldots & \nonumber
\end{eqnarray}

Of course, this encoding, as an implementation, runs away, unfolding
$\bangp{P}$ eagerly. A lazier and more implementable replication
operator, restricted to input-guarded processes, may be obtained as follows.

\begin{eqnarray}
\bangp{\prefix{u}{v}{P}} 
	:= 
	\binpar{\lift{x}{\prefix{u}{v}{(\binpar{D(x)}{P})}}}{D(x)} \nonumber
\end{eqnarray}

\begin{remark}
  Note that the lazier definition still does not deal with summation
  or mixed summation (i.e. sums over input and output). The reader is
  invited to construct definitions of replication that deal with these
  features. 

  Further, the definitions are parameterized in a name, $x$. Can you,
  gentle reader, make a definition that eliminates this parameter and
  guarantees no accidental interaction between the replication
  machinery and the process being replicated -- i.e. no accidental
  sharing of names used by the process to get its work done and the
  name(s) used by the replication to effect copying. This latter
  revision of the definition of replication is crucial to obtaining
  the expected identity $!!P \sim !P$.
\end{remark}

\begin{remark}\label{rem:paradoxical_combinator}
  The reader familiar with the lambda calculus will have noticed the
  similarity between $D$ and the paradoxical combinator.

  [Ed. note: the existence of this seems to suggest we have to be more
  restrictive on the set of processes and names we admit if we are to
  support no-cloning.]
\end{remark}

\subsubsection{Bisimulation}

The computational dynamics gives rise to another kind of equivalence,
the equivalence of computational behavior. As previously mentioned
this is typically captured \emph{via} some form of bisimulation.

% The notion we use in this paper is weak barbed bisimulation
% \cite{milner91polyadicpi}.

The notion we use in this paper is derived from weak barbed
bisimulation \cite{milner91polyadicpi}. 

\begin{definition}
An \emph{observation relation}, $\downarrow_{\mathcal N}$, over a set
of names, $\mathcal N$, is the smallest relation satisfying the rules
below.

\infrule[Out-barb]{y \in {\mathcal N}, \; x \nameeq y}
		  {\outputp{x}{v} \downarrow_{\mathcal N} x}
\infrule[Par-barb]{\mbox{$P\downarrow_{\mathcal N} x$ or $Q\downarrow_{\mathcal N} x$}}
		  {\binpar{P}{Q} \downarrow_{\mathcal N} x}

We write $P \Downarrow_{\mathcal N} x$ if there is $Q$ such that 
$P \wred Q$ and $Q \downarrow_{\mathcal N} x$.
\end{definition}

\begin{definition}
%\label{def.bbisim}
An  ${\mathcal N}$-\emph{barbed bisimulation} over a set of names, ${\mathcal N}$, is a symmetric binary relation 
${\mathcal S}_{\mathcal N}$ between agents such that $P\rel{S}_{\mathcal N}Q$ implies:
\begin{enumerate}
\item If $P \red P'$ then $Q \wred Q'$ and $P'\rel{S}_{\mathcal N} Q'$.
\item If $P\downarrow_{\mathcal N} x$, then $Q\Downarrow_{\mathcal N} x$.
\end{enumerate}
$P$ is ${\mathcal N}$-barbed bisimilar to $Q$, written
$P \wbbisim_{\mathcal N} Q$, if $P \rel{S}_{\mathcal N} Q$ for some ${\mathcal N}$-barbed bisimulation ${\mathcal S}_{\mathcal N}$.
\end{definition}

$\mathcal{R} \subseteq \pi \times \pi$

$P \mathcal{R} Q => \forall P'. P \red P' \Rightarrow \exists Q'. Q \red Q', P' \mathcal{R} Q'$

$P \vdash x \Rightarrow Q \vdash x$

\begin{mathpar}
  \inferrule*[lab=Out-barb]{x \nameeq y}{{y}!\langle{Q}\rangle \vdash x}
  \and
  \inferrule*[lab=Par-barb]{\mbox{$P\vdash x$ or $Q\vdash x$}}{\binpar{P}{Q} \vdash x}
\end{mathpar}

\subsubsection{Contexts}

One of the principle advantages of computational calculi like the
$\pi$-calculus is a well-defined notion of context,
contextual-equivalence and a correlation between
contextual-equivalence and notions of bisimulation. The notion of
context allows the decomposition of a process into (sub-)process and
its syntactic environment, its context. Thus, a context may be
thought of as a process with a ``hole'' (written $\Box$) in it. The
application of a context $M$ to a process $P$, written $M[P]$, is
tantamount to filling the hole in $M$ with $P$. In this paper we do
not need the full weight of this theory, but do make use of the notion
of context in the proof the main theorem. 

\begin{mathpar}
  \inferrule* [lab=summation] {} {{M_{M},M_{N}} \bc \Box \;|\; x.M_{A} \;|\; M_{M}+M_{N}}
  \and
  \inferrule* [lab=agent] {} {{M_{A}} \bc (\vec{x})M_{P} \;| \; \clift{P_0,\ldots,M_{P},\ldots,P_N}}
  \and \\
  \inferrule* [lab=process] {} {{M_{P}} \bc M_{N} \;| \;P|M_{P} }
\end{mathpar} 

\begin{mathpar}
  \inferrule* [lab=sychronization] {} {M_{N} \bc \Box \;|\; x?M_{F} \;|\; x!M_{C}}
  \and
  \inferrule* [lab=abstraction] {} {{M_{F}} \bc (x)M_{P} }
  \and
  \inferrule* [lab=concretion] {} {{M_{C}} \bc \langle M_{P} \rangle }
  \and \\
  \inferrule* [lab=process] {} {{M_{P}} \bc M_{N} \;| \;P|M_{P} }
\end{mathpar}

\begin{definition}[contextual application] Given a context $M$, and
  process $P$, we define the \emph{contextual application}, $M[P] :=
  M\{P/\Box\}$. That is, the contextual application of M to P is the
  substitution of $P$ for $\Box$ in $M$.
\end{definition}

$\meaningof{-} : L \to \mathcal{P}(\pi)$

\begin{mathpar}
  \inferrule* [lab=collection] {} {\meaningof{true} = \pi, \and \meaningof{~E} = \pi \setminus \meaningof{E}, \and \meaningof{E_{1} \& E_{2}} = \meaningof{E_{1}} \cap \meaningof{E_{2}}}
\end{mathpar}

\begin{mathpar}
  \inferrule* [lab=structure] {} {\meaningof{0} = \{ P \in \pi | P \equiv 0 \}, \and \\ \meaningof{E_1 | E_2} = \{ P \in \pi | P \equiv P_{1} | P_{2}, P_{1} \in \meaningof{E_{1}}, P_{2} \in \meaningof{E_2}\} }
\end{mathpar}

\begin{mathpar}
 \inferrule* [lab=behavior] {} {\meaningof{\langle a?b \rangle E} = \{ P \in \pi | P \equiv Q | u?(y)P', \\ \and \\\\ \and \\ \;\;\; u \in \meaningof{a}, \forall z.P'\{z/y\} \in \meaningof{E\{z/b\}}\}, \and \\ \meaningof{a!E} = \{ P \in \pi | P \equiv Q | x!\langle P' \rangle, x \in \meaningof{a} P' \in \meaningof{E}\} }
\end{mathpar}

\begin{mathpar}
 \inferrule* [lab=nominal] {} {\meaningof{\quotep{E}} = \{ \quotep{P} \in \quotep{\pi} | P \in \meaningof{E} \}, \and \meaningof{\quotep{P}} = \{ \quotep{Q} \in \quotep{\pi} | P \equiv Q \} \and \\ \meaningof{@\quotep{E}} = \{ P \in \pi | P \equiv @x, x \in \meaningof{E} \}}
\end{mathpar}

\begin{eqnarray*}
  \\
  \meaningof{-} : TS \to ST
\end{eqnarray*}

\begin{eqnarray*}
  \\
  L : TS \to ST
\end{eqnarray*}

\begin{eqnarray*}
  \\
  P \models E \iff P \in \meaningof{E}
\end{eqnarray*}

\begin{eqnarray*}
  P \approx_{L} Q \iff \forall E \in L. P \models E \iff Q \models E
\end{eqnarray*}

\begin{eqnarray*}
  P \approx_{K} Q
\end{eqnarray*}

\begin{eqnarray*}
  P \approx Q
\end{eqnarray*}

$\approx_{K} = \approx = \approx_{L}$

\subsubsection{Contextual duality}

Note that contexts extend the quotation operation to a family of
operations from processes to names. Given a context, $M$, we can
define a \emph{nominal context}, $\quotep{M}$ by $\quotep{M}[P] :=
\quotep{M[P]}$. To foreshadow what is to come we observe that these
operations enjoy a duality with processes very much like the duality
between vectors and maps from vectors to scalars.

Further, because the calculus is essentially higher-order, we have a
correspondence between contexts and processes. More specifically,
given a name $x$ and a context $M$ we can construct $M^{*}_{x}$ such
that 

\begin{mathpar}
  M^{*}_{x} | \lift{x}{P} \red M[P]
\end{mathpar}

namely,

\begin{mathpar}
  M^{*}_{x} := x?(u).M[\dropn{u}]
\end{mathpar}

The dependence of $M^{*}_{x}$ on a name makes it an abstraction, 

\begin{mathpar}
  M^{*} := (x)x?(u).M[\dropn{u}]
\end{mathpar}

\subsection{Additional notation}

It will sometimes be convenient to denote the process a name
quotes. We already have the notation $x = \quotep{P}$, but it will be
convenient to introduce an alternate notation, $\procn{x}$, when we
want to emphasize the connection to the use of the name. Note that, by
virtue of name equivalence, $\quotep{\procn{x}} \nameeq x$; so, the
notation is consistent with previous definitions.

Further, because names have structure it is possible to effect
substitutions on the basis of that structure. This means we need to
upgrade our notation for substitutions, which we accomplish by
adapting comprehension notation. Thus,

\begin{mathpar}
  P\{ y / x : x \in S \}
\end{mathpar}

is interpreted to mean the process derived from P by replacing (in a
capture-avoiding manner) each occurrence of $x$ in $S$ by $y$. For example,

\begin{mathpar}
  P\{ \quotep{\procn{x}|\procn{x}} / x : x \in \freenames{P} \}
\end{mathpar}

will replace each (occurrence) of a free name $x$ in $P$ by
$\quotep{\procn{x}|\procn{x}}$.

Also, we will avail ourselves of the notation $x^{L}$ and $x^{R}$ to
denote injections of a name into disjoint copies of the name
space. There are numerous ways to accomplish this. One example can be
found in \cite{MeredithR05}. This notation overloads to vectors of
names: $\vec{x}^{\pi} := (x_{i}^{\pi} \; : \; 0 \leq i < |\vec{x}| )$ where $\pi \in \{L,R\}$.

We also use $P^{\Box} := P|\Box$.

In \cite{MeredithR05} an interpretation of the new operator is
given. It turns out that there are several possible interpretations
all enjoying the requisite algebraic properties of the operator (see
\cite{milner91polyadicpi}). We will therefore make liberal use of
$(\nu\; \vec{x})P$.

% subsection the_syntax_and_semantics_of_the_notation_system (end)   

\input{qm2pi.qmops} 

\input{qm2pi.sterngerlach} 

\input{qm2pi.metric} 

% section concurrent_process_calculi (end)

%\input{qm2pi.proofsketch}

% section proof sketch (end)

%\input{qm2pi.slviaknots} 

% section spatial logic via knots (end)

\input{qm2pi.conclusion}

% section conclusion (end)

%\input{qm2pi.dtcodes} 

% section wiring algorithm (end)

\input{qm2pi.ack} 

% section acknowledgments (end)

\newpage


\bibliographystyle{plain}   
\bibliography{../../biblios/main.bib}

\input{qm2pi.rhodetails}

\end{document}



% section front matter (end)

\section{Introduction}\label{sec:introduction} % (fold)
In this draft of the material i am going to have to dispense with the
usual writing conventions adopted in papers on these topics. i'm going
to have adopt whatever tone i need at the time i'm writing up the
calculations. Sometimes this may be very conversational; others it may
be the barest mathematical grunts; others still it may be that i have
lifted text from one of my other papers because the exposition of some
point was better said there. i hope that my readers are not unduly put
out by this decision. i'm not doing this to flout convention or be
rebellious. i find these calculations very technically challenging. To
keep everything going technically, something has to give; i have to
let go of some cognitive burden. So, the academic writing style --
with all of its trade-offs in terms of facilitating technical
communication -- is what i'm letting go of. Perhaps subsequent drafts
can be tightened and polished, but for now, i'm going to speak as if
we were sitting together in a coffee shop with a laptop, wifi and a
pad of paper and a pencil.

So, here's what i have to say. We -- you and i, comfortably ensconced
in our coffee shop and well-equipped with our tools -- can realize and
carry out the calculations of quantum mechanics over a very different
formal theory of dynamics, a formal theory of dynamics that
corresponds to a theory of concurrent computation with
\emph{reflection}. It has the advantage that the underlying theory is
already `quantized', but supports analogues all of the continuuous
operations. Strikingly, this underlying theory has recently been
connected with a notion of metric that we can show, by calculating
together, coincides with the metric induced by the inner product.

There are a lot of reasons why you might be interested in seeing
calculations of this form. Here's why i'm interested. For the past
several centuries there has been no competitor to the ``Newtonian''
account of dynamics. As a result the predominant share of accounts of
dynamical systems and situations have had to be formulated in terms of
the Newtonian machinery. i view this as an intellectually dangerous
position to occupy. Everything, despite it's intrinsic shape, turns
into a nail to be hit with this hammer. Recently, however, the theory
of computation has matured to the point where we have candidates for
theories of dynamics that offer very different perspective on
reasoning about dynamical systems and situations. Testing these
candidates against very successful accounts of dynamical situations,
like quantum mechanics, is going to give us some sense of how mature
they are and some measure of the quality of these accounts of
dynamics.

\subsection{Summary of contributions and outline of paper}

So, we're going to develop an interpretation of the operations of
quantum mechanics normally interpreted by Hilbert spaces and
operators. We're going to do this over a theory of computation. Note
that this is very different than the usual quantum computation program
which develops notions of computation over quantum mechanics. Rather,
we are developing a story that aligns with Wheeler's slogan: It from
Bit. To do this we will first provide an account of the theory of
computation at play here. Then we will dive into a calculation-driven
interpretation of the operations of quantum mechanics.

The reason we take this approach is that -- until very recently --
there hasn't been an axiomatic account of quantum mechanics. As a
result there has been no sharp delineation of the mathematical theory
supporting interpretation of the physical theory and the physical
theory, itself. So, ambient features of the maths are free to be
exploited (or supressed) without a real accounting of their physical
relevance. There is no sharp statement ``here's the physical theory''
qua \emph{theory} and ``here's the mathematical interpretation''
enabling a judgment of how faithful the interpretation is -- apart
from experimental observation. When there is an axiomatic account we
can judge how well a given mathematical formalism supports an
interpretation of the axioms, independent of
experimentation. Likewise, we can judge how well we have captured our
physical evidence and experience with our axiomatics, independent of
any specific mathematical implementation, with accidental detail that
may or may not have physical significance. 

In lieu of a fully fleshed out and vetted axiomatic account of quantum
mechanics, interpreting the operational notions in service of modeling
physical systems will have to suffice. In other words, we are not in
the business of providing a model of Hilbert spaces and operators. We
are in the business of providing a model of quantum mechanics because
we are motivated by testing our notions of dynamics against physical
theory; and, the predictive calculations of the physical theory must
serve as the best formulation -- shy of a fully fleshed out axiomatic
account -- of the physical theory itself (as they have for scientific
theories since time immemorial). Put another way, despite a
whole-hearted commitment to an It-from-Bit ontology, we are firmly
aligned with the shut-up-and-calculate camp as the best way to obtain
results either from the physical perspective or as a quality assurance
measure of our fledgling theory of dynamics.

In detail, we present a reflective process calculus. Then we develop
intuitive correspondences between the notions available in this
calculus and the usual physical notions supporting quantum mechanical
calculations. Thus, 

\begin{table}[htp]
  \center{
    \fbox{
      \begin{tabular}{c|c}
        quantum mechanics & process calculus \\
        \hline
        scalar & name \\
        state vector & process \\
        dual & contextual duals \\
        matrix & formal sums of process-context-dual pairs \\
        orthogonality & process annihilation \\
        inner product & execution-formula + quoting
      \end{tabular}
    }
  }
  \caption{QM - process calculi correspondences}
\end{table}

Then we tighten up these intuitions to operational definitions. We
employ the Dirac notation as the best proxy we can find for an
abstract syntax of the quantum mechanical notions. The definitions we
develop put us in contact with equational constraints coming from the
theory that we demonstrate the definitions and calculations satisfy.

This puts us in a position to shut up and calculate for the
Stern-Gerlach experimental set up, showing how these predictive
calculations become calculations on processes in our theory of a
reflective process calculus.

Penultimately, we demonstrate that the notion of metric coming from
the inner product coincides with the notion of metric available from
the theory of bisimulation. This demonstration gives us the right to
think of space as arising from behavior. Finally, we consider where we
might go from the new vantage point we have obtained.

% section introduction (end) 
 
% section introduction (end)

% \documentclass[12pt]{llncs}
%\documentclass{jktr}

\usepackage[pdftex]{hyperref}                   
\usepackage {listings}
\usepackage {mathpartir}
\usepackage{bcprules}
%\usepackage{listings}
                       
\usepackage{graphicx} 
%\usepackage[margins=2.5cm,nohead,nofoot]{geometry}
%\usepackage{geometry}
\usepackage{amsfonts}
\usepackage{amstext}
\usepackage{latexsym}
\usepackage{amssymb}
\usepackage{color}


%\include{myPreamble}
\include{qm2pi.local} 

%\ifpdf
%\usepackage[pdftex]{graphicx}
%\else
%\usepackage{graphicx}
%\fi

 % \ifpdf
%  \usepackage{pdfsync}
%  \if


%\title{Brief Article}
%\author{David F. Snyder}
%\author{L.G. Meredith}

%\address{Dept. of Math., Texas State University--San Marcos, San Marcos, TX 78666}
       
\pagestyle{empty}


\begin{document}

\lstset{language=[Objective]Caml,frame=shadowbox}

\input{qm2pi.front}

% section front matter (end)

\input{qm2pi.intro} 
 
% section introduction (end)

% \input{qm2pi.knotations} 

% section notation (end)

\input{qm2pi.process.calculi} 

% section concurrent_process_calculi_and_spatial_logics_ (end)
    
%\input{qm2pi.knots2pi} 

%\input{qm2pi.trefoil} 

%\input{qm2pi.mainthm} 

% subsection basic_interpretation (end)

%\input{qm2pi.rho.presentation} 
\subsection{The syntax and semantics of the notation system}\label{sub:the_syntax_and_semantics_of_the_notation_system} % (fold)

We now summarize a technical presentation of the calculus that
embodies our theory of dynamics. The typical presentation of such a
calculus follows the style of giving generators and relations on
them. The grammar, below, describing term constructors, freely
generates the set of processes, $\Proc$. This set is then quotiented
by a relation known as structural congruence and it is over this set
that the notion of dynamics is expressed. This presentation is
essentially that of \cite{MeredithR05} with the addition of
polyadicity and summation. For readability we have relegated some of
the technical subtleties to an appendix.

\subsubsection{Process grammar}\label{subsub:process_grammar}

\begin{mathpar}
  \inferrule* [lab=synchronization] {} {{M} \bc \pzero \;|\; x?F \;|\; x!C }
  \and
  \inferrule* [lab=abstraction] {} {{F} \bc (x)P}
  \and
  \inferrule* [lab=concretion] {} {{C} \bc \langle Q \rangle}
  \and
  \inferrule* [lab=process] {} {{P,Q} \bc M \;| \;P|Q \;|\; @{x}}
  \and
  \inferrule* [lab=name] {} {{x} \bc \quotep{P}}
\end{mathpar} 

Note that $\vec{x}$ (resp. $\vec{P}$) denotes a vector of names
(resp. processes) of length $|\vec{x}|$ (resp. $|\vec{P}|$). We adopt
the following useful abbreviations.

\begin{mathpar}
   x?(\vec{y}).P := x.(\vec{y})P \and  x\clift{\vec{P}} := x.\clift{\vec{P}}
   \and x!(y) := \lift{x}{\dropn{y}}
   \and \Pi_{i=0}^{n-1}P_i := P_0 | \ldots | P_{n-1}
\end{mathpar}

\subsubsection{Structural congruence}

\paragraph{Free and bound names and alpha-equivalence.} At the
core of structural equivalence is alpha-equivalence which identifies
process that are the same up to a change of variable. Formally, we
recognize the distinction between free and bound names. The free names
of a process, $\freenames{P}$, may be calculated recursively as
follows:

\begin{mathpar}
\freenames{\pzero} := \emptyset
  \and \\
  \freenames{x?(y).P} := \{ x \} \cup (\freenames{P} \setminus \{ y \})
  \and 
  \freenames{x!\langle P \rangle} := \{ x \} \cup \{ P \} 
  \and \\
  \freenames{P|Q} := \freenames{P} \cup \freenames{Q}
  \and \\
  \freenames{@{x}} := \{ x \}
\end{mathpar}

$\pi$
$\quotep{\pi}$

$\freenames{-} : \pi \to \mathcal{P}(\quotep{\pi})$

\begin{eqnarray*}
  \freenames{\pzero} & := & \emptyset \\
  \freenames{x?(y).P} & := & \{ x \} \cup (\freenames{P} \setminus \{ y \}) \\
  \freenames{x!\langle P \rangle} & := & \{ x \} \cup \{ P \} \\
  \freenames{P|Q} & := & \freenames{P} \cup \freenames{Q} \\
  \freenames{\dropn{x}} & := & \{ x \}
\end{eqnarray*}

The bound names of a process, $\boundnames{P}$, are those names occurring in $P$
that are not free. For example, in $x?(y).0$, the name $x$ is free, while $y$ is bound.

\begin{mathpar}
  \inferrule* [lab=monoidal-laws] {} { P|Q \equiv Q|P \and P|0 \equiv P \and P|(Q|R) \equiv (P|Q)|R }
\end{mathpar}

\begin{mathpar}
  \inferrule* [lab=alpha-equivalence] {} { (x)P \equiv (y)P\{y/x\} \and y \not\in \freenames{P} }
\end{mathpar}

\begin{definition}
Then two processes, $P,Q$, are alpha-equivalent if $P = Q\{\vec{y}/\vec{x}\}$ for
some $\vec{x} \in \boundnames{Q},\vec{y} \in \boundnames{P}$, where $Q\{\vec{y}/\vec{x}\}$
denotes the capture-avoiding substitution of $\vec{y}$ for $\vec{x}$ in $Q$.
\end{definition}

\begin{definition}
  The {\em structural congruence} \cite{SangiorgiWalker} , $\equiv$,
  between processes is the least congruence containing
  alpha-equivalence, satisfying the abelian monoid laws
  (associativity, commutativity and $\pzero$ as identity) for parallel
  composition $|$ and for summation $+$.
\end{definition}

\subsection{Name equivalence}

We take name equivalence, written $\nameeq$, to be the smallest
equivalence relation generated by the following rules.

\begin{mathpar}
\inferrule*[lab=Quote-drop]
{ }
{ \quotep{@{x}} \nameeq x }

\inferrule*[lab=Struct-equiv]
{ P \scong Q }
{ \quotep{P} \nameeq \quotep{Q} }
\end{mathpar}

The astute reader will have noticed that the mutual recursion of names
and processes imposes a mutual recursion on alpha-equivalence and
structural equivalence via name-equivalence. Fortunately, all of this
works out pleasantly and we may calculate in the natural way, free of
concern. The reader interested in the details is referred to the
appendix \ref{appendix:rho_details}.

\subsection{Substitution}

We use $\Proc$ for the set of processes, $\QProc$ for the set of
names, and $\id{\{}\vec{y} / \vec{x} \id{\}}$ to denote partial maps,
$s : \QProc \rightarrow \QProc$. A map, $s$ lifts, uniquely, to a map
on process terms, $\widehat{s} : \Proc \rightarrow \Proc$ by the
following equations.

\begin{mathpar}
  (0) \psubstp{Q}{P} := 0 \\
  (R \juxtap S) \psubstp{Q}{P}
  :=    
  (R)\psubstp{Q}{P} \juxtap (S) \psubstp{Q}{P} \\
  (x?(y).R) \psubstp{Q}{P}    
  :=    
  (x)\substp{Q}{P} (z)\concat( (R \psubstn{z}{y}) \psubstp{Q}{P} ) \\
  (\lift{x}{R}) \psubstp{Q}{P}  
  :=
  \lift{(x)\substp{Q}{P}}{ R \psubstp{Q}{P} } \\
%   (\dropn{x})  \psubstp{Q}{P}       
%   := 
%   \left\{ 
%     \begin{array}{ccc} 
%       \dropn{\quotep{Q}} & & x \nameeq \quotep{P} \\
%       \dropn{x} & & otherwise \\
%     \end{array}
%   \right. 
  (\dropn{x})  \psubstp{Q}{P}       
  := 
  \left\{ 
    \begin{array}{ccc} 
      Q & & x \nameeq \quotep{P} \\
      \dropn{x} & & otherwise \\
    \end{array}
  \right.
\end{mathpar}
 

where

\begin{eqnarray}
  (x)\id{\{} \lpquote Q \rpquote / \lpquote P \rpquote \id{\}}            = 
  \left\{ 
    \begin{array}{ccc}
      \lpquote Q \rpquote & & x \nameeq \lpquote P \rpquote \\
      x & & otherwise \\
    \end{array}
  \right. \nonumber
\end{eqnarray}

and $z$ is chosen distinct from $\quotep{P}$, $\quotep{Q}$, the free
names in $Q$, and all the names in $R$. Our $\alpha$-equivalence will
be built in the standard way from this substitution.

\begin{remark}\label{rem:no_self_referential_names}
  One consequence of these definitions is that $\forall P. \quotep{P}
  \not\in \freenames{P}$.
\end{remark}

\subsection{ Dynamic quote: an example }

Anticipating something of what's to come, consider applying the
substitution, $\widehat{\id{\{}u / z \id{\}}}$, to the following pair
of processes, $\lift{w}{y!(z)}$ and $w[ \lpquote y!(z) \rpquote ]$.

\begin{eqnarray}
	\lift{w}{y!(z)}\widehat{\id{\{}u / z \id{\}}}
		& = &
		\lift{w}{y!(u)} \nonumber\\
	w[ \lpquote y!(z) \rpquote ] \widehat{ \id{\{}u / z \id{\}} }
		& = &
		w[ \lpquote y!(z) \rpquote ] \nonumber
\end{eqnarray}

Because the body of the process between quotes is impervious to
substitution, we get radically different answers. In fact, by
examining the first process in an input context,
e.g. $x?(z).\lift{w}{y!(z)}$, we see that the process under the lift
operator may be shaped by prefixed inputs binding a name inside it. In
this sense, the lift operator will be seen as a way to dynamically
construct processes before reifying them as names.

Finally equipped with these standard features we can present the
dynamics of the calculus.

\subsubsection{Operational semantics} 

Finally, we introduce the computational dynamics. What marks these
algebras as distinct from other more traditionally studied algebraic
structures, e.g. vector spaces or polynomial rings, is the manner in
which dynamics is captured. In traditional structures, dynamics is typically
expressed through morphisms between such structures, as in linear maps
between vector spaces or morphisms between rings. In algebras
associated with the semantics of computation, the dynamics is
expressed as part of the algebraic structure itself, through a
reduction reduction relation typically denoted by $\red$. Below, we
give a recursive presentation of this relation for the calculus used
in the encoding.

$\red \subseteq \pi \times \pi$
$\red : \pi \to \mathcal{P}(\pi)$

\begin{mathpar}
  \inferrule* [lab=Comm] { \textsf{match}( x_{src}, x_{trgt} ) } { x_{trgt}?(y)P \; | \; x_{src}!\langle {Q} \rangle \red P\{\quotep{Q}/y}\} }
  \and \\
  \inferrule* [lab=Par] {{P} \red {P}'} {{{P} | {Q}} \red {{P}' | {Q}}}
  \and
  \inferrule* [lab=Equiv]{{{P} \scong {P}'} \andalso {{P}' \red {Q}'} \andalso {{Q}' \scong {Q}}}{{P} \red {Q}}
\end{mathpar}

\begin{eqnarray*}
  match_{\equiv} (\quotep{P},\quotep{Q}) & := & P \equiv Q \\
  match_{\dagger}(\quotep{P},\quotep{Q}) & := & \forall R. P|Q \red^{*} R => R \red^{*} 0 \\
  match_{K}(\quotep{P},\quotep{Q}) & := & K \mbox{ for some context } K
\end{eqnarray*}

$u?(x)P | u!\langle Q \rangle \red P\{\quotep{Q}/x\}$

%We write $\wred$ for $\red^*$, and $P\red$ if $\exists Q $ such that $ P \red Q$.
We write $P\red$ if $\exists Q $ such that $ P \red Q$ and $P\not\red$, otherwise.

\section{Replication}

As mentioned before, it is known that replication (and hence
recursion) can be implemented in a higher-order process algebra
\cite{SangiorgiWalker}. As our first example of calculation with the
machinery thus far presented we give the construction explicitly in
the {\rhoc}.

\begin{eqnarray}
	D_{x} & := & \prefix{x}{y}{(\binpar{\outputp{x}{y}}{@{y}})} \nonumber\\
	\bangp_{x}{P} & := & \binpar{{x}!\langle{\binpar{D_{x}}{P}}\rangle}{D_{x}} \nonumber
\end{eqnarray}

\begin{eqnarray}
	\bangp_{x}{P} & & \nonumber\\
	=
	& {x}!\langle{(\prefix{x}{y}{(\outputp{x}{y} | @{y})) | P}}\rangle 
	      | \prefix{x}{y}{(\outputp{x}{y} | @{y})} & \nonumber\\
	\red
	& (\outputp{x}{y} | @{y})\substn{\quotep{(\prefix{x}{y}{(@{y} | \outputp{x}{y})) | P}}}{y} & \nonumber\\
	=
	& \outputp{x}{\quotep{(\prefix{x}{y}{(\outputp{x}{y} | @{y})) | P}}}
	  | {(\prefix{x}{y}{(\outputp{x}{y} | @{y})) | P}} & \nonumber\\
	\red
	& \ldots & \nonumber\\
	\red^*
	& P | P | \ldots & \nonumber
\end{eqnarray}

Of course, this encoding, as an implementation, runs away, unfolding
$\bangp{P}$ eagerly. A lazier and more implementable replication
operator, restricted to input-guarded processes, may be obtained as follows.

\begin{eqnarray}
\bangp{\prefix{u}{v}{P}} 
	:= 
	\binpar{\lift{x}{\prefix{u}{v}{(\binpar{D(x)}{P})}}}{D(x)} \nonumber
\end{eqnarray}

\begin{remark}
  Note that the lazier definition still does not deal with summation
  or mixed summation (i.e. sums over input and output). The reader is
  invited to construct definitions of replication that deal with these
  features. 

  Further, the definitions are parameterized in a name, $x$. Can you,
  gentle reader, make a definition that eliminates this parameter and
  guarantees no accidental interaction between the replication
  machinery and the process being replicated -- i.e. no accidental
  sharing of names used by the process to get its work done and the
  name(s) used by the replication to effect copying. This latter
  revision of the definition of replication is crucial to obtaining
  the expected identity $!!P \sim !P$.
\end{remark}

\begin{remark}\label{rem:paradoxical_combinator}
  The reader familiar with the lambda calculus will have noticed the
  similarity between $D$ and the paradoxical combinator.

  [Ed. note: the existence of this seems to suggest we have to be more
  restrictive on the set of processes and names we admit if we are to
  support no-cloning.]
\end{remark}

\subsubsection{Bisimulation}

The computational dynamics gives rise to another kind of equivalence,
the equivalence of computational behavior. As previously mentioned
this is typically captured \emph{via} some form of bisimulation.

% The notion we use in this paper is weak barbed bisimulation
% \cite{milner91polyadicpi}.

The notion we use in this paper is derived from weak barbed
bisimulation \cite{milner91polyadicpi}. 

\begin{definition}
An \emph{observation relation}, $\downarrow_{\mathcal N}$, over a set
of names, $\mathcal N$, is the smallest relation satisfying the rules
below.

\infrule[Out-barb]{y \in {\mathcal N}, \; x \nameeq y}
		  {\outputp{x}{v} \downarrow_{\mathcal N} x}
\infrule[Par-barb]{\mbox{$P\downarrow_{\mathcal N} x$ or $Q\downarrow_{\mathcal N} x$}}
		  {\binpar{P}{Q} \downarrow_{\mathcal N} x}

We write $P \Downarrow_{\mathcal N} x$ if there is $Q$ such that 
$P \wred Q$ and $Q \downarrow_{\mathcal N} x$.
\end{definition}

\begin{definition}
%\label{def.bbisim}
An  ${\mathcal N}$-\emph{barbed bisimulation} over a set of names, ${\mathcal N}$, is a symmetric binary relation 
${\mathcal S}_{\mathcal N}$ between agents such that $P\rel{S}_{\mathcal N}Q$ implies:
\begin{enumerate}
\item If $P \red P'$ then $Q \wred Q'$ and $P'\rel{S}_{\mathcal N} Q'$.
\item If $P\downarrow_{\mathcal N} x$, then $Q\Downarrow_{\mathcal N} x$.
\end{enumerate}
$P$ is ${\mathcal N}$-barbed bisimilar to $Q$, written
$P \wbbisim_{\mathcal N} Q$, if $P \rel{S}_{\mathcal N} Q$ for some ${\mathcal N}$-barbed bisimulation ${\mathcal S}_{\mathcal N}$.
\end{definition}

$\mathcal{R} \subseteq \pi \times \pi$

$P \mathcal{R} Q => \forall P'. P \red P' \Rightarrow \exists Q'. Q \red Q', P' \mathcal{R} Q'$

$P \vdash x \Rightarrow Q \vdash x$

\begin{mathpar}
  \inferrule*[lab=Out-barb]{x \nameeq y}{{y}!\langle{Q}\rangle \vdash x}
  \and
  \inferrule*[lab=Par-barb]{\mbox{$P\vdash x$ or $Q\vdash x$}}{\binpar{P}{Q} \vdash x}
\end{mathpar}

\subsubsection{Contexts}

One of the principle advantages of computational calculi like the
$\pi$-calculus is a well-defined notion of context,
contextual-equivalence and a correlation between
contextual-equivalence and notions of bisimulation. The notion of
context allows the decomposition of a process into (sub-)process and
its syntactic environment, its context. Thus, a context may be
thought of as a process with a ``hole'' (written $\Box$) in it. The
application of a context $M$ to a process $P$, written $M[P]$, is
tantamount to filling the hole in $M$ with $P$. In this paper we do
not need the full weight of this theory, but do make use of the notion
of context in the proof the main theorem. 

\begin{mathpar}
  \inferrule* [lab=summation] {} {{M_{M},M_{N}} \bc \Box \;|\; x.M_{A} \;|\; M_{M}+M_{N}}
  \and
  \inferrule* [lab=agent] {} {{M_{A}} \bc (\vec{x})M_{P} \;| \; \clift{P_0,\ldots,M_{P},\ldots,P_N}}
  \and \\
  \inferrule* [lab=process] {} {{M_{P}} \bc M_{N} \;| \;P|M_{P} }
\end{mathpar} 

\begin{mathpar}
  \inferrule* [lab=sychronization] {} {M_{N} \bc \Box \;|\; x?M_{F} \;|\; x!M_{C}}
  \and
  \inferrule* [lab=abstraction] {} {{M_{F}} \bc (x)M_{P} }
  \and
  \inferrule* [lab=concretion] {} {{M_{C}} \bc \langle M_{P} \rangle }
  \and \\
  \inferrule* [lab=process] {} {{M_{P}} \bc M_{N} \;| \;P|M_{P} }
\end{mathpar}

\begin{definition}[contextual application] Given a context $M$, and
  process $P$, we define the \emph{contextual application}, $M[P] :=
  M\{P/\Box\}$. That is, the contextual application of M to P is the
  substitution of $P$ for $\Box$ in $M$.
\end{definition}

$\meaningof{-} : L \to \mathcal{P}(\pi)$

\begin{mathpar}
  \inferrule* [lab=collection] {} {\meaningof{true} = \pi, \and \meaningof{~E} = \pi \setminus \meaningof{E}, \and \meaningof{E_{1} \& E_{2}} = \meaningof{E_{1}} \cap \meaningof{E_{2}}}
\end{mathpar}

\begin{mathpar}
  \inferrule* [lab=structure] {} {\meaningof{0} = \{ P \in \pi | P \equiv 0 \}, \and \\ \meaningof{E_1 | E_2} = \{ P \in \pi | P \equiv P_{1} | P_{2}, P_{1} \in \meaningof{E_{1}}, P_{2} \in \meaningof{E_2}\} }
\end{mathpar}

\begin{mathpar}
 \inferrule* [lab=behavior] {} {\meaningof{\langle a?b \rangle E} = \{ P \in \pi | P \equiv Q | u?(y)P', \\ \and \\\\ \and \\ \;\;\; u \in \meaningof{a}, \forall z.P'\{z/y\} \in \meaningof{E\{z/b\}}\}, \and \\ \meaningof{a!E} = \{ P \in \pi | P \equiv Q | x!\langle P' \rangle, x \in \meaningof{a} P' \in \meaningof{E}\} }
\end{mathpar}

\begin{mathpar}
 \inferrule* [lab=nominal] {} {\meaningof{\quotep{E}} = \{ \quotep{P} \in \quotep{\pi} | P \in \meaningof{E} \}, \and \meaningof{\quotep{P}} = \{ \quotep{Q} \in \quotep{\pi} | P \equiv Q \} \and \\ \meaningof{@\quotep{E}} = \{ P \in \pi | P \equiv @x, x \in \meaningof{E} \}}
\end{mathpar}

\begin{eqnarray*}
  \\
  \meaningof{-} : TS \to ST
\end{eqnarray*}

\begin{eqnarray*}
  \\
  L : TS \to ST
\end{eqnarray*}

\begin{eqnarray*}
  \\
  P \models E \iff P \in \meaningof{E}
\end{eqnarray*}

\begin{eqnarray*}
  P \approx_{L} Q \iff \forall E \in L. P \models E \iff Q \models E
\end{eqnarray*}

\begin{eqnarray*}
  P \approx_{K} Q
\end{eqnarray*}

\begin{eqnarray*}
  P \approx Q
\end{eqnarray*}

$\approx_{K} = \approx = \approx_{L}$

\subsubsection{Contextual duality}

Note that contexts extend the quotation operation to a family of
operations from processes to names. Given a context, $M$, we can
define a \emph{nominal context}, $\quotep{M}$ by $\quotep{M}[P] :=
\quotep{M[P]}$. To foreshadow what is to come we observe that these
operations enjoy a duality with processes very much like the duality
between vectors and maps from vectors to scalars.

Further, because the calculus is essentially higher-order, we have a
correspondence between contexts and processes. More specifically,
given a name $x$ and a context $M$ we can construct $M^{*}_{x}$ such
that 

\begin{mathpar}
  M^{*}_{x} | \lift{x}{P} \red M[P]
\end{mathpar}

namely,

\begin{mathpar}
  M^{*}_{x} := x?(u).M[\dropn{u}]
\end{mathpar}

The dependence of $M^{*}_{x}$ on a name makes it an abstraction, 

\begin{mathpar}
  M^{*} := (x)x?(u).M[\dropn{u}]
\end{mathpar}

\subsection{Additional notation}

It will sometimes be convenient to denote the process a name
quotes. We already have the notation $x = \quotep{P}$, but it will be
convenient to introduce an alternate notation, $\procn{x}$, when we
want to emphasize the connection to the use of the name. Note that, by
virtue of name equivalence, $\quotep{\procn{x}} \nameeq x$; so, the
notation is consistent with previous definitions.

Further, because names have structure it is possible to effect
substitutions on the basis of that structure. This means we need to
upgrade our notation for substitutions, which we accomplish by
adapting comprehension notation. Thus,

\begin{mathpar}
  P\{ y / x : x \in S \}
\end{mathpar}

is interpreted to mean the process derived from P by replacing (in a
capture-avoiding manner) each occurrence of $x$ in $S$ by $y$. For example,

\begin{mathpar}
  P\{ \quotep{\procn{x}|\procn{x}} / x : x \in \freenames{P} \}
\end{mathpar}

will replace each (occurrence) of a free name $x$ in $P$ by
$\quotep{\procn{x}|\procn{x}}$.

Also, we will avail ourselves of the notation $x^{L}$ and $x^{R}$ to
denote injections of a name into disjoint copies of the name
space. There are numerous ways to accomplish this. One example can be
found in \cite{MeredithR05}. This notation overloads to vectors of
names: $\vec{x}^{\pi} := (x_{i}^{\pi} \; : \; 0 \leq i < |\vec{x}| )$ where $\pi \in \{L,R\}$.

We also use $P^{\Box} := P|\Box$.

In \cite{MeredithR05} an interpretation of the new operator is
given. It turns out that there are several possible interpretations
all enjoying the requisite algebraic properties of the operator (see
\cite{milner91polyadicpi}). We will therefore make liberal use of
$(\nu\; \vec{x})P$.

% subsection the_syntax_and_semantics_of_the_notation_system (end)   

\input{qm2pi.qmops} 

\input{qm2pi.sterngerlach} 

\input{qm2pi.metric} 

% section concurrent_process_calculi (end)

%\input{qm2pi.proofsketch}

% section proof sketch (end)

%\input{qm2pi.slviaknots} 

% section spatial logic via knots (end)

\input{qm2pi.conclusion}

% section conclusion (end)

%\input{qm2pi.dtcodes} 

% section wiring algorithm (end)

\input{qm2pi.ack} 

% section acknowledgments (end)

\newpage


\bibliographystyle{plain}   
\bibliography{../../biblios/main.bib}

\input{qm2pi.rhodetails}

\end{document}

 

% section notation (end)

\input{qm2pi.process.calculi} 

% section concurrent_process_calculi_and_spatial_logics_ (end)
    
%\documentclass[12pt]{llncs}
%\documentclass{jktr}

\usepackage[pdftex]{hyperref}                   
\usepackage {listings}
\usepackage {mathpartir}
\usepackage{bcprules}
%\usepackage{listings}
                       
\usepackage{graphicx} 
%\usepackage[margins=2.5cm,nohead,nofoot]{geometry}
%\usepackage{geometry}
\usepackage{amsfonts}
\usepackage{amstext}
\usepackage{latexsym}
\usepackage{amssymb}
\usepackage{color}


%\include{myPreamble}
\include{qm2pi.local} 

%\ifpdf
%\usepackage[pdftex]{graphicx}
%\else
%\usepackage{graphicx}
%\fi

 % \ifpdf
%  \usepackage{pdfsync}
%  \if


%\title{Brief Article}
%\author{David F. Snyder}
%\author{L.G. Meredith}

%\address{Dept. of Math., Texas State University--San Marcos, San Marcos, TX 78666}
       
\pagestyle{empty}


\begin{document}

\lstset{language=[Objective]Caml,frame=shadowbox}

\input{qm2pi.front}

% section front matter (end)

\input{qm2pi.intro} 
 
% section introduction (end)

% \input{qm2pi.knotations} 

% section notation (end)

\input{qm2pi.process.calculi} 

% section concurrent_process_calculi_and_spatial_logics_ (end)
    
%\input{qm2pi.knots2pi} 

%\input{qm2pi.trefoil} 

%\input{qm2pi.mainthm} 

% subsection basic_interpretation (end)

%\input{qm2pi.rho.presentation} 
\subsection{The syntax and semantics of the notation system}\label{sub:the_syntax_and_semantics_of_the_notation_system} % (fold)

We now summarize a technical presentation of the calculus that
embodies our theory of dynamics. The typical presentation of such a
calculus follows the style of giving generators and relations on
them. The grammar, below, describing term constructors, freely
generates the set of processes, $\Proc$. This set is then quotiented
by a relation known as structural congruence and it is over this set
that the notion of dynamics is expressed. This presentation is
essentially that of \cite{MeredithR05} with the addition of
polyadicity and summation. For readability we have relegated some of
the technical subtleties to an appendix.

\subsubsection{Process grammar}\label{subsub:process_grammar}

\begin{mathpar}
  \inferrule* [lab=synchronization] {} {{M} \bc \pzero \;|\; x?F \;|\; x!C }
  \and
  \inferrule* [lab=abstraction] {} {{F} \bc (x)P}
  \and
  \inferrule* [lab=concretion] {} {{C} \bc \langle Q \rangle}
  \and
  \inferrule* [lab=process] {} {{P,Q} \bc M \;| \;P|Q \;|\; @{x}}
  \and
  \inferrule* [lab=name] {} {{x} \bc \quotep{P}}
\end{mathpar} 

Note that $\vec{x}$ (resp. $\vec{P}$) denotes a vector of names
(resp. processes) of length $|\vec{x}|$ (resp. $|\vec{P}|$). We adopt
the following useful abbreviations.

\begin{mathpar}
   x?(\vec{y}).P := x.(\vec{y})P \and  x\clift{\vec{P}} := x.\clift{\vec{P}}
   \and x!(y) := \lift{x}{\dropn{y}}
   \and \Pi_{i=0}^{n-1}P_i := P_0 | \ldots | P_{n-1}
\end{mathpar}

\subsubsection{Structural congruence}

\paragraph{Free and bound names and alpha-equivalence.} At the
core of structural equivalence is alpha-equivalence which identifies
process that are the same up to a change of variable. Formally, we
recognize the distinction between free and bound names. The free names
of a process, $\freenames{P}$, may be calculated recursively as
follows:

\begin{mathpar}
\freenames{\pzero} := \emptyset
  \and \\
  \freenames{x?(y).P} := \{ x \} \cup (\freenames{P} \setminus \{ y \})
  \and 
  \freenames{x!\langle P \rangle} := \{ x \} \cup \{ P \} 
  \and \\
  \freenames{P|Q} := \freenames{P} \cup \freenames{Q}
  \and \\
  \freenames{@{x}} := \{ x \}
\end{mathpar}

$\pi$
$\quotep{\pi}$

$\freenames{-} : \pi \to \mathcal{P}(\quotep{\pi})$

\begin{eqnarray*}
  \freenames{\pzero} & := & \emptyset \\
  \freenames{x?(y).P} & := & \{ x \} \cup (\freenames{P} \setminus \{ y \}) \\
  \freenames{x!\langle P \rangle} & := & \{ x \} \cup \{ P \} \\
  \freenames{P|Q} & := & \freenames{P} \cup \freenames{Q} \\
  \freenames{\dropn{x}} & := & \{ x \}
\end{eqnarray*}

The bound names of a process, $\boundnames{P}$, are those names occurring in $P$
that are not free. For example, in $x?(y).0$, the name $x$ is free, while $y$ is bound.

\begin{mathpar}
  \inferrule* [lab=monoidal-laws] {} { P|Q \equiv Q|P \and P|0 \equiv P \and P|(Q|R) \equiv (P|Q)|R }
\end{mathpar}

\begin{mathpar}
  \inferrule* [lab=alpha-equivalence] {} { (x)P \equiv (y)P\{y/x\} \and y \not\in \freenames{P} }
\end{mathpar}

\begin{definition}
Then two processes, $P,Q$, are alpha-equivalent if $P = Q\{\vec{y}/\vec{x}\}$ for
some $\vec{x} \in \boundnames{Q},\vec{y} \in \boundnames{P}$, where $Q\{\vec{y}/\vec{x}\}$
denotes the capture-avoiding substitution of $\vec{y}$ for $\vec{x}$ in $Q$.
\end{definition}

\begin{definition}
  The {\em structural congruence} \cite{SangiorgiWalker} , $\equiv$,
  between processes is the least congruence containing
  alpha-equivalence, satisfying the abelian monoid laws
  (associativity, commutativity and $\pzero$ as identity) for parallel
  composition $|$ and for summation $+$.
\end{definition}

\subsection{Name equivalence}

We take name equivalence, written $\nameeq$, to be the smallest
equivalence relation generated by the following rules.

\begin{mathpar}
\inferrule*[lab=Quote-drop]
{ }
{ \quotep{@{x}} \nameeq x }

\inferrule*[lab=Struct-equiv]
{ P \scong Q }
{ \quotep{P} \nameeq \quotep{Q} }
\end{mathpar}

The astute reader will have noticed that the mutual recursion of names
and processes imposes a mutual recursion on alpha-equivalence and
structural equivalence via name-equivalence. Fortunately, all of this
works out pleasantly and we may calculate in the natural way, free of
concern. The reader interested in the details is referred to the
appendix \ref{appendix:rho_details}.

\subsection{Substitution}

We use $\Proc$ for the set of processes, $\QProc$ for the set of
names, and $\id{\{}\vec{y} / \vec{x} \id{\}}$ to denote partial maps,
$s : \QProc \rightarrow \QProc$. A map, $s$ lifts, uniquely, to a map
on process terms, $\widehat{s} : \Proc \rightarrow \Proc$ by the
following equations.

\begin{mathpar}
  (0) \psubstp{Q}{P} := 0 \\
  (R \juxtap S) \psubstp{Q}{P}
  :=    
  (R)\psubstp{Q}{P} \juxtap (S) \psubstp{Q}{P} \\
  (x?(y).R) \psubstp{Q}{P}    
  :=    
  (x)\substp{Q}{P} (z)\concat( (R \psubstn{z}{y}) \psubstp{Q}{P} ) \\
  (\lift{x}{R}) \psubstp{Q}{P}  
  :=
  \lift{(x)\substp{Q}{P}}{ R \psubstp{Q}{P} } \\
%   (\dropn{x})  \psubstp{Q}{P}       
%   := 
%   \left\{ 
%     \begin{array}{ccc} 
%       \dropn{\quotep{Q}} & & x \nameeq \quotep{P} \\
%       \dropn{x} & & otherwise \\
%     \end{array}
%   \right. 
  (\dropn{x})  \psubstp{Q}{P}       
  := 
  \left\{ 
    \begin{array}{ccc} 
      Q & & x \nameeq \quotep{P} \\
      \dropn{x} & & otherwise \\
    \end{array}
  \right.
\end{mathpar}
 

where

\begin{eqnarray}
  (x)\id{\{} \lpquote Q \rpquote / \lpquote P \rpquote \id{\}}            = 
  \left\{ 
    \begin{array}{ccc}
      \lpquote Q \rpquote & & x \nameeq \lpquote P \rpquote \\
      x & & otherwise \\
    \end{array}
  \right. \nonumber
\end{eqnarray}

and $z$ is chosen distinct from $\quotep{P}$, $\quotep{Q}$, the free
names in $Q$, and all the names in $R$. Our $\alpha$-equivalence will
be built in the standard way from this substitution.

\begin{remark}\label{rem:no_self_referential_names}
  One consequence of these definitions is that $\forall P. \quotep{P}
  \not\in \freenames{P}$.
\end{remark}

\subsection{ Dynamic quote: an example }

Anticipating something of what's to come, consider applying the
substitution, $\widehat{\id{\{}u / z \id{\}}}$, to the following pair
of processes, $\lift{w}{y!(z)}$ and $w[ \lpquote y!(z) \rpquote ]$.

\begin{eqnarray}
	\lift{w}{y!(z)}\widehat{\id{\{}u / z \id{\}}}
		& = &
		\lift{w}{y!(u)} \nonumber\\
	w[ \lpquote y!(z) \rpquote ] \widehat{ \id{\{}u / z \id{\}} }
		& = &
		w[ \lpquote y!(z) \rpquote ] \nonumber
\end{eqnarray}

Because the body of the process between quotes is impervious to
substitution, we get radically different answers. In fact, by
examining the first process in an input context,
e.g. $x?(z).\lift{w}{y!(z)}$, we see that the process under the lift
operator may be shaped by prefixed inputs binding a name inside it. In
this sense, the lift operator will be seen as a way to dynamically
construct processes before reifying them as names.

Finally equipped with these standard features we can present the
dynamics of the calculus.

\subsubsection{Operational semantics} 

Finally, we introduce the computational dynamics. What marks these
algebras as distinct from other more traditionally studied algebraic
structures, e.g. vector spaces or polynomial rings, is the manner in
which dynamics is captured. In traditional structures, dynamics is typically
expressed through morphisms between such structures, as in linear maps
between vector spaces or morphisms between rings. In algebras
associated with the semantics of computation, the dynamics is
expressed as part of the algebraic structure itself, through a
reduction reduction relation typically denoted by $\red$. Below, we
give a recursive presentation of this relation for the calculus used
in the encoding.

$\red \subseteq \pi \times \pi$
$\red : \pi \to \mathcal{P}(\pi)$

\begin{mathpar}
  \inferrule* [lab=Comm] { \textsf{match}( x_{src}, x_{trgt} ) } { x_{trgt}?(y)P \; | \; x_{src}!\langle {Q} \rangle \red P\{\quotep{Q}/y}\} }
  \and \\
  \inferrule* [lab=Par] {{P} \red {P}'} {{{P} | {Q}} \red {{P}' | {Q}}}
  \and
  \inferrule* [lab=Equiv]{{{P} \scong {P}'} \andalso {{P}' \red {Q}'} \andalso {{Q}' \scong {Q}}}{{P} \red {Q}}
\end{mathpar}

\begin{eqnarray*}
  match_{\equiv} (\quotep{P},\quotep{Q}) & := & P \equiv Q \\
  match_{\dagger}(\quotep{P},\quotep{Q}) & := & \forall R. P|Q \red^{*} R => R \red^{*} 0 \\
  match_{K}(\quotep{P},\quotep{Q}) & := & K \mbox{ for some context } K
\end{eqnarray*}

$u?(x)P | u!\langle Q \rangle \red P\{\quotep{Q}/x\}$

%We write $\wred$ for $\red^*$, and $P\red$ if $\exists Q $ such that $ P \red Q$.
We write $P\red$ if $\exists Q $ such that $ P \red Q$ and $P\not\red$, otherwise.

\section{Replication}

As mentioned before, it is known that replication (and hence
recursion) can be implemented in a higher-order process algebra
\cite{SangiorgiWalker}. As our first example of calculation with the
machinery thus far presented we give the construction explicitly in
the {\rhoc}.

\begin{eqnarray}
	D_{x} & := & \prefix{x}{y}{(\binpar{\outputp{x}{y}}{@{y}})} \nonumber\\
	\bangp_{x}{P} & := & \binpar{{x}!\langle{\binpar{D_{x}}{P}}\rangle}{D_{x}} \nonumber
\end{eqnarray}

\begin{eqnarray}
	\bangp_{x}{P} & & \nonumber\\
	=
	& {x}!\langle{(\prefix{x}{y}{(\outputp{x}{y} | @{y})) | P}}\rangle 
	      | \prefix{x}{y}{(\outputp{x}{y} | @{y})} & \nonumber\\
	\red
	& (\outputp{x}{y} | @{y})\substn{\quotep{(\prefix{x}{y}{(@{y} | \outputp{x}{y})) | P}}}{y} & \nonumber\\
	=
	& \outputp{x}{\quotep{(\prefix{x}{y}{(\outputp{x}{y} | @{y})) | P}}}
	  | {(\prefix{x}{y}{(\outputp{x}{y} | @{y})) | P}} & \nonumber\\
	\red
	& \ldots & \nonumber\\
	\red^*
	& P | P | \ldots & \nonumber
\end{eqnarray}

Of course, this encoding, as an implementation, runs away, unfolding
$\bangp{P}$ eagerly. A lazier and more implementable replication
operator, restricted to input-guarded processes, may be obtained as follows.

\begin{eqnarray}
\bangp{\prefix{u}{v}{P}} 
	:= 
	\binpar{\lift{x}{\prefix{u}{v}{(\binpar{D(x)}{P})}}}{D(x)} \nonumber
\end{eqnarray}

\begin{remark}
  Note that the lazier definition still does not deal with summation
  or mixed summation (i.e. sums over input and output). The reader is
  invited to construct definitions of replication that deal with these
  features. 

  Further, the definitions are parameterized in a name, $x$. Can you,
  gentle reader, make a definition that eliminates this parameter and
  guarantees no accidental interaction between the replication
  machinery and the process being replicated -- i.e. no accidental
  sharing of names used by the process to get its work done and the
  name(s) used by the replication to effect copying. This latter
  revision of the definition of replication is crucial to obtaining
  the expected identity $!!P \sim !P$.
\end{remark}

\begin{remark}\label{rem:paradoxical_combinator}
  The reader familiar with the lambda calculus will have noticed the
  similarity between $D$ and the paradoxical combinator.

  [Ed. note: the existence of this seems to suggest we have to be more
  restrictive on the set of processes and names we admit if we are to
  support no-cloning.]
\end{remark}

\subsubsection{Bisimulation}

The computational dynamics gives rise to another kind of equivalence,
the equivalence of computational behavior. As previously mentioned
this is typically captured \emph{via} some form of bisimulation.

% The notion we use in this paper is weak barbed bisimulation
% \cite{milner91polyadicpi}.

The notion we use in this paper is derived from weak barbed
bisimulation \cite{milner91polyadicpi}. 

\begin{definition}
An \emph{observation relation}, $\downarrow_{\mathcal N}$, over a set
of names, $\mathcal N$, is the smallest relation satisfying the rules
below.

\infrule[Out-barb]{y \in {\mathcal N}, \; x \nameeq y}
		  {\outputp{x}{v} \downarrow_{\mathcal N} x}
\infrule[Par-barb]{\mbox{$P\downarrow_{\mathcal N} x$ or $Q\downarrow_{\mathcal N} x$}}
		  {\binpar{P}{Q} \downarrow_{\mathcal N} x}

We write $P \Downarrow_{\mathcal N} x$ if there is $Q$ such that 
$P \wred Q$ and $Q \downarrow_{\mathcal N} x$.
\end{definition}

\begin{definition}
%\label{def.bbisim}
An  ${\mathcal N}$-\emph{barbed bisimulation} over a set of names, ${\mathcal N}$, is a symmetric binary relation 
${\mathcal S}_{\mathcal N}$ between agents such that $P\rel{S}_{\mathcal N}Q$ implies:
\begin{enumerate}
\item If $P \red P'$ then $Q \wred Q'$ and $P'\rel{S}_{\mathcal N} Q'$.
\item If $P\downarrow_{\mathcal N} x$, then $Q\Downarrow_{\mathcal N} x$.
\end{enumerate}
$P$ is ${\mathcal N}$-barbed bisimilar to $Q$, written
$P \wbbisim_{\mathcal N} Q$, if $P \rel{S}_{\mathcal N} Q$ for some ${\mathcal N}$-barbed bisimulation ${\mathcal S}_{\mathcal N}$.
\end{definition}

$\mathcal{R} \subseteq \pi \times \pi$

$P \mathcal{R} Q => \forall P'. P \red P' \Rightarrow \exists Q'. Q \red Q', P' \mathcal{R} Q'$

$P \vdash x \Rightarrow Q \vdash x$

\begin{mathpar}
  \inferrule*[lab=Out-barb]{x \nameeq y}{{y}!\langle{Q}\rangle \vdash x}
  \and
  \inferrule*[lab=Par-barb]{\mbox{$P\vdash x$ or $Q\vdash x$}}{\binpar{P}{Q} \vdash x}
\end{mathpar}

\subsubsection{Contexts}

One of the principle advantages of computational calculi like the
$\pi$-calculus is a well-defined notion of context,
contextual-equivalence and a correlation between
contextual-equivalence and notions of bisimulation. The notion of
context allows the decomposition of a process into (sub-)process and
its syntactic environment, its context. Thus, a context may be
thought of as a process with a ``hole'' (written $\Box$) in it. The
application of a context $M$ to a process $P$, written $M[P]$, is
tantamount to filling the hole in $M$ with $P$. In this paper we do
not need the full weight of this theory, but do make use of the notion
of context in the proof the main theorem. 

\begin{mathpar}
  \inferrule* [lab=summation] {} {{M_{M},M_{N}} \bc \Box \;|\; x.M_{A} \;|\; M_{M}+M_{N}}
  \and
  \inferrule* [lab=agent] {} {{M_{A}} \bc (\vec{x})M_{P} \;| \; \clift{P_0,\ldots,M_{P},\ldots,P_N}}
  \and \\
  \inferrule* [lab=process] {} {{M_{P}} \bc M_{N} \;| \;P|M_{P} }
\end{mathpar} 

\begin{mathpar}
  \inferrule* [lab=sychronization] {} {M_{N} \bc \Box \;|\; x?M_{F} \;|\; x!M_{C}}
  \and
  \inferrule* [lab=abstraction] {} {{M_{F}} \bc (x)M_{P} }
  \and
  \inferrule* [lab=concretion] {} {{M_{C}} \bc \langle M_{P} \rangle }
  \and \\
  \inferrule* [lab=process] {} {{M_{P}} \bc M_{N} \;| \;P|M_{P} }
\end{mathpar}

\begin{definition}[contextual application] Given a context $M$, and
  process $P$, we define the \emph{contextual application}, $M[P] :=
  M\{P/\Box\}$. That is, the contextual application of M to P is the
  substitution of $P$ for $\Box$ in $M$.
\end{definition}

$\meaningof{-} : L \to \mathcal{P}(\pi)$

\begin{mathpar}
  \inferrule* [lab=collection] {} {\meaningof{true} = \pi, \and \meaningof{~E} = \pi \setminus \meaningof{E}, \and \meaningof{E_{1} \& E_{2}} = \meaningof{E_{1}} \cap \meaningof{E_{2}}}
\end{mathpar}

\begin{mathpar}
  \inferrule* [lab=structure] {} {\meaningof{0} = \{ P \in \pi | P \equiv 0 \}, \and \\ \meaningof{E_1 | E_2} = \{ P \in \pi | P \equiv P_{1} | P_{2}, P_{1} \in \meaningof{E_{1}}, P_{2} \in \meaningof{E_2}\} }
\end{mathpar}

\begin{mathpar}
 \inferrule* [lab=behavior] {} {\meaningof{\langle a?b \rangle E} = \{ P \in \pi | P \equiv Q | u?(y)P', \\ \and \\\\ \and \\ \;\;\; u \in \meaningof{a}, \forall z.P'\{z/y\} \in \meaningof{E\{z/b\}}\}, \and \\ \meaningof{a!E} = \{ P \in \pi | P \equiv Q | x!\langle P' \rangle, x \in \meaningof{a} P' \in \meaningof{E}\} }
\end{mathpar}

\begin{mathpar}
 \inferrule* [lab=nominal] {} {\meaningof{\quotep{E}} = \{ \quotep{P} \in \quotep{\pi} | P \in \meaningof{E} \}, \and \meaningof{\quotep{P}} = \{ \quotep{Q} \in \quotep{\pi} | P \equiv Q \} \and \\ \meaningof{@\quotep{E}} = \{ P \in \pi | P \equiv @x, x \in \meaningof{E} \}}
\end{mathpar}

\begin{eqnarray*}
  \\
  \meaningof{-} : TS \to ST
\end{eqnarray*}

\begin{eqnarray*}
  \\
  L : TS \to ST
\end{eqnarray*}

\begin{eqnarray*}
  \\
  P \models E \iff P \in \meaningof{E}
\end{eqnarray*}

\begin{eqnarray*}
  P \approx_{L} Q \iff \forall E \in L. P \models E \iff Q \models E
\end{eqnarray*}

\begin{eqnarray*}
  P \approx_{K} Q
\end{eqnarray*}

\begin{eqnarray*}
  P \approx Q
\end{eqnarray*}

$\approx_{K} = \approx = \approx_{L}$

\subsubsection{Contextual duality}

Note that contexts extend the quotation operation to a family of
operations from processes to names. Given a context, $M$, we can
define a \emph{nominal context}, $\quotep{M}$ by $\quotep{M}[P] :=
\quotep{M[P]}$. To foreshadow what is to come we observe that these
operations enjoy a duality with processes very much like the duality
between vectors and maps from vectors to scalars.

Further, because the calculus is essentially higher-order, we have a
correspondence between contexts and processes. More specifically,
given a name $x$ and a context $M$ we can construct $M^{*}_{x}$ such
that 

\begin{mathpar}
  M^{*}_{x} | \lift{x}{P} \red M[P]
\end{mathpar}

namely,

\begin{mathpar}
  M^{*}_{x} := x?(u).M[\dropn{u}]
\end{mathpar}

The dependence of $M^{*}_{x}$ on a name makes it an abstraction, 

\begin{mathpar}
  M^{*} := (x)x?(u).M[\dropn{u}]
\end{mathpar}

\subsection{Additional notation}

It will sometimes be convenient to denote the process a name
quotes. We already have the notation $x = \quotep{P}$, but it will be
convenient to introduce an alternate notation, $\procn{x}$, when we
want to emphasize the connection to the use of the name. Note that, by
virtue of name equivalence, $\quotep{\procn{x}} \nameeq x$; so, the
notation is consistent with previous definitions.

Further, because names have structure it is possible to effect
substitutions on the basis of that structure. This means we need to
upgrade our notation for substitutions, which we accomplish by
adapting comprehension notation. Thus,

\begin{mathpar}
  P\{ y / x : x \in S \}
\end{mathpar}

is interpreted to mean the process derived from P by replacing (in a
capture-avoiding manner) each occurrence of $x$ in $S$ by $y$. For example,

\begin{mathpar}
  P\{ \quotep{\procn{x}|\procn{x}} / x : x \in \freenames{P} \}
\end{mathpar}

will replace each (occurrence) of a free name $x$ in $P$ by
$\quotep{\procn{x}|\procn{x}}$.

Also, we will avail ourselves of the notation $x^{L}$ and $x^{R}$ to
denote injections of a name into disjoint copies of the name
space. There are numerous ways to accomplish this. One example can be
found in \cite{MeredithR05}. This notation overloads to vectors of
names: $\vec{x}^{\pi} := (x_{i}^{\pi} \; : \; 0 \leq i < |\vec{x}| )$ where $\pi \in \{L,R\}$.

We also use $P^{\Box} := P|\Box$.

In \cite{MeredithR05} an interpretation of the new operator is
given. It turns out that there are several possible interpretations
all enjoying the requisite algebraic properties of the operator (see
\cite{milner91polyadicpi}). We will therefore make liberal use of
$(\nu\; \vec{x})P$.

% subsection the_syntax_and_semantics_of_the_notation_system (end)   

\input{qm2pi.qmops} 

\input{qm2pi.sterngerlach} 

\input{qm2pi.metric} 

% section concurrent_process_calculi (end)

%\input{qm2pi.proofsketch}

% section proof sketch (end)

%\input{qm2pi.slviaknots} 

% section spatial logic via knots (end)

\input{qm2pi.conclusion}

% section conclusion (end)

%\input{qm2pi.dtcodes} 

% section wiring algorithm (end)

\input{qm2pi.ack} 

% section acknowledgments (end)

\newpage


\bibliographystyle{plain}   
\bibliography{../../biblios/main.bib}

\input{qm2pi.rhodetails}

\end{document}

 

%\documentclass[12pt]{llncs}
%\documentclass{jktr}

\usepackage[pdftex]{hyperref}                   
\usepackage {listings}
\usepackage {mathpartir}
\usepackage{bcprules}
%\usepackage{listings}
                       
\usepackage{graphicx} 
%\usepackage[margins=2.5cm,nohead,nofoot]{geometry}
%\usepackage{geometry}
\usepackage{amsfonts}
\usepackage{amstext}
\usepackage{latexsym}
\usepackage{amssymb}
\usepackage{color}


%\include{myPreamble}
\include{qm2pi.local} 

%\ifpdf
%\usepackage[pdftex]{graphicx}
%\else
%\usepackage{graphicx}
%\fi

 % \ifpdf
%  \usepackage{pdfsync}
%  \if


%\title{Brief Article}
%\author{David F. Snyder}
%\author{L.G. Meredith}

%\address{Dept. of Math., Texas State University--San Marcos, San Marcos, TX 78666}
       
\pagestyle{empty}


\begin{document}

\lstset{language=[Objective]Caml,frame=shadowbox}

\input{qm2pi.front}

% section front matter (end)

\input{qm2pi.intro} 
 
% section introduction (end)

% \input{qm2pi.knotations} 

% section notation (end)

\input{qm2pi.process.calculi} 

% section concurrent_process_calculi_and_spatial_logics_ (end)
    
%\input{qm2pi.knots2pi} 

%\input{qm2pi.trefoil} 

%\input{qm2pi.mainthm} 

% subsection basic_interpretation (end)

%\input{qm2pi.rho.presentation} 
\subsection{The syntax and semantics of the notation system}\label{sub:the_syntax_and_semantics_of_the_notation_system} % (fold)

We now summarize a technical presentation of the calculus that
embodies our theory of dynamics. The typical presentation of such a
calculus follows the style of giving generators and relations on
them. The grammar, below, describing term constructors, freely
generates the set of processes, $\Proc$. This set is then quotiented
by a relation known as structural congruence and it is over this set
that the notion of dynamics is expressed. This presentation is
essentially that of \cite{MeredithR05} with the addition of
polyadicity and summation. For readability we have relegated some of
the technical subtleties to an appendix.

\subsubsection{Process grammar}\label{subsub:process_grammar}

\begin{mathpar}
  \inferrule* [lab=synchronization] {} {{M} \bc \pzero \;|\; x?F \;|\; x!C }
  \and
  \inferrule* [lab=abstraction] {} {{F} \bc (x)P}
  \and
  \inferrule* [lab=concretion] {} {{C} \bc \langle Q \rangle}
  \and
  \inferrule* [lab=process] {} {{P,Q} \bc M \;| \;P|Q \;|\; @{x}}
  \and
  \inferrule* [lab=name] {} {{x} \bc \quotep{P}}
\end{mathpar} 

Note that $\vec{x}$ (resp. $\vec{P}$) denotes a vector of names
(resp. processes) of length $|\vec{x}|$ (resp. $|\vec{P}|$). We adopt
the following useful abbreviations.

\begin{mathpar}
   x?(\vec{y}).P := x.(\vec{y})P \and  x\clift{\vec{P}} := x.\clift{\vec{P}}
   \and x!(y) := \lift{x}{\dropn{y}}
   \and \Pi_{i=0}^{n-1}P_i := P_0 | \ldots | P_{n-1}
\end{mathpar}

\subsubsection{Structural congruence}

\paragraph{Free and bound names and alpha-equivalence.} At the
core of structural equivalence is alpha-equivalence which identifies
process that are the same up to a change of variable. Formally, we
recognize the distinction between free and bound names. The free names
of a process, $\freenames{P}$, may be calculated recursively as
follows:

\begin{mathpar}
\freenames{\pzero} := \emptyset
  \and \\
  \freenames{x?(y).P} := \{ x \} \cup (\freenames{P} \setminus \{ y \})
  \and 
  \freenames{x!\langle P \rangle} := \{ x \} \cup \{ P \} 
  \and \\
  \freenames{P|Q} := \freenames{P} \cup \freenames{Q}
  \and \\
  \freenames{@{x}} := \{ x \}
\end{mathpar}

$\pi$
$\quotep{\pi}$

$\freenames{-} : \pi \to \mathcal{P}(\quotep{\pi})$

\begin{eqnarray*}
  \freenames{\pzero} & := & \emptyset \\
  \freenames{x?(y).P} & := & \{ x \} \cup (\freenames{P} \setminus \{ y \}) \\
  \freenames{x!\langle P \rangle} & := & \{ x \} \cup \{ P \} \\
  \freenames{P|Q} & := & \freenames{P} \cup \freenames{Q} \\
  \freenames{\dropn{x}} & := & \{ x \}
\end{eqnarray*}

The bound names of a process, $\boundnames{P}$, are those names occurring in $P$
that are not free. For example, in $x?(y).0$, the name $x$ is free, while $y$ is bound.

\begin{mathpar}
  \inferrule* [lab=monoidal-laws] {} { P|Q \equiv Q|P \and P|0 \equiv P \and P|(Q|R) \equiv (P|Q)|R }
\end{mathpar}

\begin{mathpar}
  \inferrule* [lab=alpha-equivalence] {} { (x)P \equiv (y)P\{y/x\} \and y \not\in \freenames{P} }
\end{mathpar}

\begin{definition}
Then two processes, $P,Q$, are alpha-equivalent if $P = Q\{\vec{y}/\vec{x}\}$ for
some $\vec{x} \in \boundnames{Q},\vec{y} \in \boundnames{P}$, where $Q\{\vec{y}/\vec{x}\}$
denotes the capture-avoiding substitution of $\vec{y}$ for $\vec{x}$ in $Q$.
\end{definition}

\begin{definition}
  The {\em structural congruence} \cite{SangiorgiWalker} , $\equiv$,
  between processes is the least congruence containing
  alpha-equivalence, satisfying the abelian monoid laws
  (associativity, commutativity and $\pzero$ as identity) for parallel
  composition $|$ and for summation $+$.
\end{definition}

\subsection{Name equivalence}

We take name equivalence, written $\nameeq$, to be the smallest
equivalence relation generated by the following rules.

\begin{mathpar}
\inferrule*[lab=Quote-drop]
{ }
{ \quotep{@{x}} \nameeq x }

\inferrule*[lab=Struct-equiv]
{ P \scong Q }
{ \quotep{P} \nameeq \quotep{Q} }
\end{mathpar}

The astute reader will have noticed that the mutual recursion of names
and processes imposes a mutual recursion on alpha-equivalence and
structural equivalence via name-equivalence. Fortunately, all of this
works out pleasantly and we may calculate in the natural way, free of
concern. The reader interested in the details is referred to the
appendix \ref{appendix:rho_details}.

\subsection{Substitution}

We use $\Proc$ for the set of processes, $\QProc$ for the set of
names, and $\id{\{}\vec{y} / \vec{x} \id{\}}$ to denote partial maps,
$s : \QProc \rightarrow \QProc$. A map, $s$ lifts, uniquely, to a map
on process terms, $\widehat{s} : \Proc \rightarrow \Proc$ by the
following equations.

\begin{mathpar}
  (0) \psubstp{Q}{P} := 0 \\
  (R \juxtap S) \psubstp{Q}{P}
  :=    
  (R)\psubstp{Q}{P} \juxtap (S) \psubstp{Q}{P} \\
  (x?(y).R) \psubstp{Q}{P}    
  :=    
  (x)\substp{Q}{P} (z)\concat( (R \psubstn{z}{y}) \psubstp{Q}{P} ) \\
  (\lift{x}{R}) \psubstp{Q}{P}  
  :=
  \lift{(x)\substp{Q}{P}}{ R \psubstp{Q}{P} } \\
%   (\dropn{x})  \psubstp{Q}{P}       
%   := 
%   \left\{ 
%     \begin{array}{ccc} 
%       \dropn{\quotep{Q}} & & x \nameeq \quotep{P} \\
%       \dropn{x} & & otherwise \\
%     \end{array}
%   \right. 
  (\dropn{x})  \psubstp{Q}{P}       
  := 
  \left\{ 
    \begin{array}{ccc} 
      Q & & x \nameeq \quotep{P} \\
      \dropn{x} & & otherwise \\
    \end{array}
  \right.
\end{mathpar}
 

where

\begin{eqnarray}
  (x)\id{\{} \lpquote Q \rpquote / \lpquote P \rpquote \id{\}}            = 
  \left\{ 
    \begin{array}{ccc}
      \lpquote Q \rpquote & & x \nameeq \lpquote P \rpquote \\
      x & & otherwise \\
    \end{array}
  \right. \nonumber
\end{eqnarray}

and $z$ is chosen distinct from $\quotep{P}$, $\quotep{Q}$, the free
names in $Q$, and all the names in $R$. Our $\alpha$-equivalence will
be built in the standard way from this substitution.

\begin{remark}\label{rem:no_self_referential_names}
  One consequence of these definitions is that $\forall P. \quotep{P}
  \not\in \freenames{P}$.
\end{remark}

\subsection{ Dynamic quote: an example }

Anticipating something of what's to come, consider applying the
substitution, $\widehat{\id{\{}u / z \id{\}}}$, to the following pair
of processes, $\lift{w}{y!(z)}$ and $w[ \lpquote y!(z) \rpquote ]$.

\begin{eqnarray}
	\lift{w}{y!(z)}\widehat{\id{\{}u / z \id{\}}}
		& = &
		\lift{w}{y!(u)} \nonumber\\
	w[ \lpquote y!(z) \rpquote ] \widehat{ \id{\{}u / z \id{\}} }
		& = &
		w[ \lpquote y!(z) \rpquote ] \nonumber
\end{eqnarray}

Because the body of the process between quotes is impervious to
substitution, we get radically different answers. In fact, by
examining the first process in an input context,
e.g. $x?(z).\lift{w}{y!(z)}$, we see that the process under the lift
operator may be shaped by prefixed inputs binding a name inside it. In
this sense, the lift operator will be seen as a way to dynamically
construct processes before reifying them as names.

Finally equipped with these standard features we can present the
dynamics of the calculus.

\subsubsection{Operational semantics} 

Finally, we introduce the computational dynamics. What marks these
algebras as distinct from other more traditionally studied algebraic
structures, e.g. vector spaces or polynomial rings, is the manner in
which dynamics is captured. In traditional structures, dynamics is typically
expressed through morphisms between such structures, as in linear maps
between vector spaces or morphisms between rings. In algebras
associated with the semantics of computation, the dynamics is
expressed as part of the algebraic structure itself, through a
reduction reduction relation typically denoted by $\red$. Below, we
give a recursive presentation of this relation for the calculus used
in the encoding.

$\red \subseteq \pi \times \pi$
$\red : \pi \to \mathcal{P}(\pi)$

\begin{mathpar}
  \inferrule* [lab=Comm] { \textsf{match}( x_{src}, x_{trgt} ) } { x_{trgt}?(y)P \; | \; x_{src}!\langle {Q} \rangle \red P\{\quotep{Q}/y}\} }
  \and \\
  \inferrule* [lab=Par] {{P} \red {P}'} {{{P} | {Q}} \red {{P}' | {Q}}}
  \and
  \inferrule* [lab=Equiv]{{{P} \scong {P}'} \andalso {{P}' \red {Q}'} \andalso {{Q}' \scong {Q}}}{{P} \red {Q}}
\end{mathpar}

\begin{eqnarray*}
  match_{\equiv} (\quotep{P},\quotep{Q}) & := & P \equiv Q \\
  match_{\dagger}(\quotep{P},\quotep{Q}) & := & \forall R. P|Q \red^{*} R => R \red^{*} 0 \\
  match_{K}(\quotep{P},\quotep{Q}) & := & K \mbox{ for some context } K
\end{eqnarray*}

$u?(x)P | u!\langle Q \rangle \red P\{\quotep{Q}/x\}$

%We write $\wred$ for $\red^*$, and $P\red$ if $\exists Q $ such that $ P \red Q$.
We write $P\red$ if $\exists Q $ such that $ P \red Q$ and $P\not\red$, otherwise.

\section{Replication}

As mentioned before, it is known that replication (and hence
recursion) can be implemented in a higher-order process algebra
\cite{SangiorgiWalker}. As our first example of calculation with the
machinery thus far presented we give the construction explicitly in
the {\rhoc}.

\begin{eqnarray}
	D_{x} & := & \prefix{x}{y}{(\binpar{\outputp{x}{y}}{@{y}})} \nonumber\\
	\bangp_{x}{P} & := & \binpar{{x}!\langle{\binpar{D_{x}}{P}}\rangle}{D_{x}} \nonumber
\end{eqnarray}

\begin{eqnarray}
	\bangp_{x}{P} & & \nonumber\\
	=
	& {x}!\langle{(\prefix{x}{y}{(\outputp{x}{y} | @{y})) | P}}\rangle 
	      | \prefix{x}{y}{(\outputp{x}{y} | @{y})} & \nonumber\\
	\red
	& (\outputp{x}{y} | @{y})\substn{\quotep{(\prefix{x}{y}{(@{y} | \outputp{x}{y})) | P}}}{y} & \nonumber\\
	=
	& \outputp{x}{\quotep{(\prefix{x}{y}{(\outputp{x}{y} | @{y})) | P}}}
	  | {(\prefix{x}{y}{(\outputp{x}{y} | @{y})) | P}} & \nonumber\\
	\red
	& \ldots & \nonumber\\
	\red^*
	& P | P | \ldots & \nonumber
\end{eqnarray}

Of course, this encoding, as an implementation, runs away, unfolding
$\bangp{P}$ eagerly. A lazier and more implementable replication
operator, restricted to input-guarded processes, may be obtained as follows.

\begin{eqnarray}
\bangp{\prefix{u}{v}{P}} 
	:= 
	\binpar{\lift{x}{\prefix{u}{v}{(\binpar{D(x)}{P})}}}{D(x)} \nonumber
\end{eqnarray}

\begin{remark}
  Note that the lazier definition still does not deal with summation
  or mixed summation (i.e. sums over input and output). The reader is
  invited to construct definitions of replication that deal with these
  features. 

  Further, the definitions are parameterized in a name, $x$. Can you,
  gentle reader, make a definition that eliminates this parameter and
  guarantees no accidental interaction between the replication
  machinery and the process being replicated -- i.e. no accidental
  sharing of names used by the process to get its work done and the
  name(s) used by the replication to effect copying. This latter
  revision of the definition of replication is crucial to obtaining
  the expected identity $!!P \sim !P$.
\end{remark}

\begin{remark}\label{rem:paradoxical_combinator}
  The reader familiar with the lambda calculus will have noticed the
  similarity between $D$ and the paradoxical combinator.

  [Ed. note: the existence of this seems to suggest we have to be more
  restrictive on the set of processes and names we admit if we are to
  support no-cloning.]
\end{remark}

\subsubsection{Bisimulation}

The computational dynamics gives rise to another kind of equivalence,
the equivalence of computational behavior. As previously mentioned
this is typically captured \emph{via} some form of bisimulation.

% The notion we use in this paper is weak barbed bisimulation
% \cite{milner91polyadicpi}.

The notion we use in this paper is derived from weak barbed
bisimulation \cite{milner91polyadicpi}. 

\begin{definition}
An \emph{observation relation}, $\downarrow_{\mathcal N}$, over a set
of names, $\mathcal N$, is the smallest relation satisfying the rules
below.

\infrule[Out-barb]{y \in {\mathcal N}, \; x \nameeq y}
		  {\outputp{x}{v} \downarrow_{\mathcal N} x}
\infrule[Par-barb]{\mbox{$P\downarrow_{\mathcal N} x$ or $Q\downarrow_{\mathcal N} x$}}
		  {\binpar{P}{Q} \downarrow_{\mathcal N} x}

We write $P \Downarrow_{\mathcal N} x$ if there is $Q$ such that 
$P \wred Q$ and $Q \downarrow_{\mathcal N} x$.
\end{definition}

\begin{definition}
%\label{def.bbisim}
An  ${\mathcal N}$-\emph{barbed bisimulation} over a set of names, ${\mathcal N}$, is a symmetric binary relation 
${\mathcal S}_{\mathcal N}$ between agents such that $P\rel{S}_{\mathcal N}Q$ implies:
\begin{enumerate}
\item If $P \red P'$ then $Q \wred Q'$ and $P'\rel{S}_{\mathcal N} Q'$.
\item If $P\downarrow_{\mathcal N} x$, then $Q\Downarrow_{\mathcal N} x$.
\end{enumerate}
$P$ is ${\mathcal N}$-barbed bisimilar to $Q$, written
$P \wbbisim_{\mathcal N} Q$, if $P \rel{S}_{\mathcal N} Q$ for some ${\mathcal N}$-barbed bisimulation ${\mathcal S}_{\mathcal N}$.
\end{definition}

$\mathcal{R} \subseteq \pi \times \pi$

$P \mathcal{R} Q => \forall P'. P \red P' \Rightarrow \exists Q'. Q \red Q', P' \mathcal{R} Q'$

$P \vdash x \Rightarrow Q \vdash x$

\begin{mathpar}
  \inferrule*[lab=Out-barb]{x \nameeq y}{{y}!\langle{Q}\rangle \vdash x}
  \and
  \inferrule*[lab=Par-barb]{\mbox{$P\vdash x$ or $Q\vdash x$}}{\binpar{P}{Q} \vdash x}
\end{mathpar}

\subsubsection{Contexts}

One of the principle advantages of computational calculi like the
$\pi$-calculus is a well-defined notion of context,
contextual-equivalence and a correlation between
contextual-equivalence and notions of bisimulation. The notion of
context allows the decomposition of a process into (sub-)process and
its syntactic environment, its context. Thus, a context may be
thought of as a process with a ``hole'' (written $\Box$) in it. The
application of a context $M$ to a process $P$, written $M[P]$, is
tantamount to filling the hole in $M$ with $P$. In this paper we do
not need the full weight of this theory, but do make use of the notion
of context in the proof the main theorem. 

\begin{mathpar}
  \inferrule* [lab=summation] {} {{M_{M},M_{N}} \bc \Box \;|\; x.M_{A} \;|\; M_{M}+M_{N}}
  \and
  \inferrule* [lab=agent] {} {{M_{A}} \bc (\vec{x})M_{P} \;| \; \clift{P_0,\ldots,M_{P},\ldots,P_N}}
  \and \\
  \inferrule* [lab=process] {} {{M_{P}} \bc M_{N} \;| \;P|M_{P} }
\end{mathpar} 

\begin{mathpar}
  \inferrule* [lab=sychronization] {} {M_{N} \bc \Box \;|\; x?M_{F} \;|\; x!M_{C}}
  \and
  \inferrule* [lab=abstraction] {} {{M_{F}} \bc (x)M_{P} }
  \and
  \inferrule* [lab=concretion] {} {{M_{C}} \bc \langle M_{P} \rangle }
  \and \\
  \inferrule* [lab=process] {} {{M_{P}} \bc M_{N} \;| \;P|M_{P} }
\end{mathpar}

\begin{definition}[contextual application] Given a context $M$, and
  process $P$, we define the \emph{contextual application}, $M[P] :=
  M\{P/\Box\}$. That is, the contextual application of M to P is the
  substitution of $P$ for $\Box$ in $M$.
\end{definition}

$\meaningof{-} : L \to \mathcal{P}(\pi)$

\begin{mathpar}
  \inferrule* [lab=collection] {} {\meaningof{true} = \pi, \and \meaningof{~E} = \pi \setminus \meaningof{E}, \and \meaningof{E_{1} \& E_{2}} = \meaningof{E_{1}} \cap \meaningof{E_{2}}}
\end{mathpar}

\begin{mathpar}
  \inferrule* [lab=structure] {} {\meaningof{0} = \{ P \in \pi | P \equiv 0 \}, \and \\ \meaningof{E_1 | E_2} = \{ P \in \pi | P \equiv P_{1} | P_{2}, P_{1} \in \meaningof{E_{1}}, P_{2} \in \meaningof{E_2}\} }
\end{mathpar}

\begin{mathpar}
 \inferrule* [lab=behavior] {} {\meaningof{\langle a?b \rangle E} = \{ P \in \pi | P \equiv Q | u?(y)P', \\ \and \\\\ \and \\ \;\;\; u \in \meaningof{a}, \forall z.P'\{z/y\} \in \meaningof{E\{z/b\}}\}, \and \\ \meaningof{a!E} = \{ P \in \pi | P \equiv Q | x!\langle P' \rangle, x \in \meaningof{a} P' \in \meaningof{E}\} }
\end{mathpar}

\begin{mathpar}
 \inferrule* [lab=nominal] {} {\meaningof{\quotep{E}} = \{ \quotep{P} \in \quotep{\pi} | P \in \meaningof{E} \}, \and \meaningof{\quotep{P}} = \{ \quotep{Q} \in \quotep{\pi} | P \equiv Q \} \and \\ \meaningof{@\quotep{E}} = \{ P \in \pi | P \equiv @x, x \in \meaningof{E} \}}
\end{mathpar}

\begin{eqnarray*}
  \\
  \meaningof{-} : TS \to ST
\end{eqnarray*}

\begin{eqnarray*}
  \\
  L : TS \to ST
\end{eqnarray*}

\begin{eqnarray*}
  \\
  P \models E \iff P \in \meaningof{E}
\end{eqnarray*}

\begin{eqnarray*}
  P \approx_{L} Q \iff \forall E \in L. P \models E \iff Q \models E
\end{eqnarray*}

\begin{eqnarray*}
  P \approx_{K} Q
\end{eqnarray*}

\begin{eqnarray*}
  P \approx Q
\end{eqnarray*}

$\approx_{K} = \approx = \approx_{L}$

\subsubsection{Contextual duality}

Note that contexts extend the quotation operation to a family of
operations from processes to names. Given a context, $M$, we can
define a \emph{nominal context}, $\quotep{M}$ by $\quotep{M}[P] :=
\quotep{M[P]}$. To foreshadow what is to come we observe that these
operations enjoy a duality with processes very much like the duality
between vectors and maps from vectors to scalars.

Further, because the calculus is essentially higher-order, we have a
correspondence between contexts and processes. More specifically,
given a name $x$ and a context $M$ we can construct $M^{*}_{x}$ such
that 

\begin{mathpar}
  M^{*}_{x} | \lift{x}{P} \red M[P]
\end{mathpar}

namely,

\begin{mathpar}
  M^{*}_{x} := x?(u).M[\dropn{u}]
\end{mathpar}

The dependence of $M^{*}_{x}$ on a name makes it an abstraction, 

\begin{mathpar}
  M^{*} := (x)x?(u).M[\dropn{u}]
\end{mathpar}

\subsection{Additional notation}

It will sometimes be convenient to denote the process a name
quotes. We already have the notation $x = \quotep{P}$, but it will be
convenient to introduce an alternate notation, $\procn{x}$, when we
want to emphasize the connection to the use of the name. Note that, by
virtue of name equivalence, $\quotep{\procn{x}} \nameeq x$; so, the
notation is consistent with previous definitions.

Further, because names have structure it is possible to effect
substitutions on the basis of that structure. This means we need to
upgrade our notation for substitutions, which we accomplish by
adapting comprehension notation. Thus,

\begin{mathpar}
  P\{ y / x : x \in S \}
\end{mathpar}

is interpreted to mean the process derived from P by replacing (in a
capture-avoiding manner) each occurrence of $x$ in $S$ by $y$. For example,

\begin{mathpar}
  P\{ \quotep{\procn{x}|\procn{x}} / x : x \in \freenames{P} \}
\end{mathpar}

will replace each (occurrence) of a free name $x$ in $P$ by
$\quotep{\procn{x}|\procn{x}}$.

Also, we will avail ourselves of the notation $x^{L}$ and $x^{R}$ to
denote injections of a name into disjoint copies of the name
space. There are numerous ways to accomplish this. One example can be
found in \cite{MeredithR05}. This notation overloads to vectors of
names: $\vec{x}^{\pi} := (x_{i}^{\pi} \; : \; 0 \leq i < |\vec{x}| )$ where $\pi \in \{L,R\}$.

We also use $P^{\Box} := P|\Box$.

In \cite{MeredithR05} an interpretation of the new operator is
given. It turns out that there are several possible interpretations
all enjoying the requisite algebraic properties of the operator (see
\cite{milner91polyadicpi}). We will therefore make liberal use of
$(\nu\; \vec{x})P$.

% subsection the_syntax_and_semantics_of_the_notation_system (end)   

\input{qm2pi.qmops} 

\input{qm2pi.sterngerlach} 

\input{qm2pi.metric} 

% section concurrent_process_calculi (end)

%\input{qm2pi.proofsketch}

% section proof sketch (end)

%\input{qm2pi.slviaknots} 

% section spatial logic via knots (end)

\input{qm2pi.conclusion}

% section conclusion (end)

%\input{qm2pi.dtcodes} 

% section wiring algorithm (end)

\input{qm2pi.ack} 

% section acknowledgments (end)

\newpage


\bibliographystyle{plain}   
\bibliography{../../biblios/main.bib}

\input{qm2pi.rhodetails}

\end{document}

 

%\documentclass[12pt]{llncs}
%\documentclass{jktr}

\usepackage[pdftex]{hyperref}                   
\usepackage {listings}
\usepackage {mathpartir}
\usepackage{bcprules}
%\usepackage{listings}
                       
\usepackage{graphicx} 
%\usepackage[margins=2.5cm,nohead,nofoot]{geometry}
%\usepackage{geometry}
\usepackage{amsfonts}
\usepackage{amstext}
\usepackage{latexsym}
\usepackage{amssymb}
\usepackage{color}


%\include{myPreamble}
\include{qm2pi.local} 

%\ifpdf
%\usepackage[pdftex]{graphicx}
%\else
%\usepackage{graphicx}
%\fi

 % \ifpdf
%  \usepackage{pdfsync}
%  \if


%\title{Brief Article}
%\author{David F. Snyder}
%\author{L.G. Meredith}

%\address{Dept. of Math., Texas State University--San Marcos, San Marcos, TX 78666}
       
\pagestyle{empty}


\begin{document}

\lstset{language=[Objective]Caml,frame=shadowbox}

\input{qm2pi.front}

% section front matter (end)

\input{qm2pi.intro} 
 
% section introduction (end)

% \input{qm2pi.knotations} 

% section notation (end)

\input{qm2pi.process.calculi} 

% section concurrent_process_calculi_and_spatial_logics_ (end)
    
%\input{qm2pi.knots2pi} 

%\input{qm2pi.trefoil} 

%\input{qm2pi.mainthm} 

% subsection basic_interpretation (end)

%\input{qm2pi.rho.presentation} 
\subsection{The syntax and semantics of the notation system}\label{sub:the_syntax_and_semantics_of_the_notation_system} % (fold)

We now summarize a technical presentation of the calculus that
embodies our theory of dynamics. The typical presentation of such a
calculus follows the style of giving generators and relations on
them. The grammar, below, describing term constructors, freely
generates the set of processes, $\Proc$. This set is then quotiented
by a relation known as structural congruence and it is over this set
that the notion of dynamics is expressed. This presentation is
essentially that of \cite{MeredithR05} with the addition of
polyadicity and summation. For readability we have relegated some of
the technical subtleties to an appendix.

\subsubsection{Process grammar}\label{subsub:process_grammar}

\begin{mathpar}
  \inferrule* [lab=synchronization] {} {{M} \bc \pzero \;|\; x?F \;|\; x!C }
  \and
  \inferrule* [lab=abstraction] {} {{F} \bc (x)P}
  \and
  \inferrule* [lab=concretion] {} {{C} \bc \langle Q \rangle}
  \and
  \inferrule* [lab=process] {} {{P,Q} \bc M \;| \;P|Q \;|\; @{x}}
  \and
  \inferrule* [lab=name] {} {{x} \bc \quotep{P}}
\end{mathpar} 

Note that $\vec{x}$ (resp. $\vec{P}$) denotes a vector of names
(resp. processes) of length $|\vec{x}|$ (resp. $|\vec{P}|$). We adopt
the following useful abbreviations.

\begin{mathpar}
   x?(\vec{y}).P := x.(\vec{y})P \and  x\clift{\vec{P}} := x.\clift{\vec{P}}
   \and x!(y) := \lift{x}{\dropn{y}}
   \and \Pi_{i=0}^{n-1}P_i := P_0 | \ldots | P_{n-1}
\end{mathpar}

\subsubsection{Structural congruence}

\paragraph{Free and bound names and alpha-equivalence.} At the
core of structural equivalence is alpha-equivalence which identifies
process that are the same up to a change of variable. Formally, we
recognize the distinction between free and bound names. The free names
of a process, $\freenames{P}$, may be calculated recursively as
follows:

\begin{mathpar}
\freenames{\pzero} := \emptyset
  \and \\
  \freenames{x?(y).P} := \{ x \} \cup (\freenames{P} \setminus \{ y \})
  \and 
  \freenames{x!\langle P \rangle} := \{ x \} \cup \{ P \} 
  \and \\
  \freenames{P|Q} := \freenames{P} \cup \freenames{Q}
  \and \\
  \freenames{@{x}} := \{ x \}
\end{mathpar}

$\pi$
$\quotep{\pi}$

$\freenames{-} : \pi \to \mathcal{P}(\quotep{\pi})$

\begin{eqnarray*}
  \freenames{\pzero} & := & \emptyset \\
  \freenames{x?(y).P} & := & \{ x \} \cup (\freenames{P} \setminus \{ y \}) \\
  \freenames{x!\langle P \rangle} & := & \{ x \} \cup \{ P \} \\
  \freenames{P|Q} & := & \freenames{P} \cup \freenames{Q} \\
  \freenames{\dropn{x}} & := & \{ x \}
\end{eqnarray*}

The bound names of a process, $\boundnames{P}$, are those names occurring in $P$
that are not free. For example, in $x?(y).0$, the name $x$ is free, while $y$ is bound.

\begin{mathpar}
  \inferrule* [lab=monoidal-laws] {} { P|Q \equiv Q|P \and P|0 \equiv P \and P|(Q|R) \equiv (P|Q)|R }
\end{mathpar}

\begin{mathpar}
  \inferrule* [lab=alpha-equivalence] {} { (x)P \equiv (y)P\{y/x\} \and y \not\in \freenames{P} }
\end{mathpar}

\begin{definition}
Then two processes, $P,Q$, are alpha-equivalent if $P = Q\{\vec{y}/\vec{x}\}$ for
some $\vec{x} \in \boundnames{Q},\vec{y} \in \boundnames{P}$, where $Q\{\vec{y}/\vec{x}\}$
denotes the capture-avoiding substitution of $\vec{y}$ for $\vec{x}$ in $Q$.
\end{definition}

\begin{definition}
  The {\em structural congruence} \cite{SangiorgiWalker} , $\equiv$,
  between processes is the least congruence containing
  alpha-equivalence, satisfying the abelian monoid laws
  (associativity, commutativity and $\pzero$ as identity) for parallel
  composition $|$ and for summation $+$.
\end{definition}

\subsection{Name equivalence}

We take name equivalence, written $\nameeq$, to be the smallest
equivalence relation generated by the following rules.

\begin{mathpar}
\inferrule*[lab=Quote-drop]
{ }
{ \quotep{@{x}} \nameeq x }

\inferrule*[lab=Struct-equiv]
{ P \scong Q }
{ \quotep{P} \nameeq \quotep{Q} }
\end{mathpar}

The astute reader will have noticed that the mutual recursion of names
and processes imposes a mutual recursion on alpha-equivalence and
structural equivalence via name-equivalence. Fortunately, all of this
works out pleasantly and we may calculate in the natural way, free of
concern. The reader interested in the details is referred to the
appendix \ref{appendix:rho_details}.

\subsection{Substitution}

We use $\Proc$ for the set of processes, $\QProc$ for the set of
names, and $\id{\{}\vec{y} / \vec{x} \id{\}}$ to denote partial maps,
$s : \QProc \rightarrow \QProc$. A map, $s$ lifts, uniquely, to a map
on process terms, $\widehat{s} : \Proc \rightarrow \Proc$ by the
following equations.

\begin{mathpar}
  (0) \psubstp{Q}{P} := 0 \\
  (R \juxtap S) \psubstp{Q}{P}
  :=    
  (R)\psubstp{Q}{P} \juxtap (S) \psubstp{Q}{P} \\
  (x?(y).R) \psubstp{Q}{P}    
  :=    
  (x)\substp{Q}{P} (z)\concat( (R \psubstn{z}{y}) \psubstp{Q}{P} ) \\
  (\lift{x}{R}) \psubstp{Q}{P}  
  :=
  \lift{(x)\substp{Q}{P}}{ R \psubstp{Q}{P} } \\
%   (\dropn{x})  \psubstp{Q}{P}       
%   := 
%   \left\{ 
%     \begin{array}{ccc} 
%       \dropn{\quotep{Q}} & & x \nameeq \quotep{P} \\
%       \dropn{x} & & otherwise \\
%     \end{array}
%   \right. 
  (\dropn{x})  \psubstp{Q}{P}       
  := 
  \left\{ 
    \begin{array}{ccc} 
      Q & & x \nameeq \quotep{P} \\
      \dropn{x} & & otherwise \\
    \end{array}
  \right.
\end{mathpar}
 

where

\begin{eqnarray}
  (x)\id{\{} \lpquote Q \rpquote / \lpquote P \rpquote \id{\}}            = 
  \left\{ 
    \begin{array}{ccc}
      \lpquote Q \rpquote & & x \nameeq \lpquote P \rpquote \\
      x & & otherwise \\
    \end{array}
  \right. \nonumber
\end{eqnarray}

and $z$ is chosen distinct from $\quotep{P}$, $\quotep{Q}$, the free
names in $Q$, and all the names in $R$. Our $\alpha$-equivalence will
be built in the standard way from this substitution.

\begin{remark}\label{rem:no_self_referential_names}
  One consequence of these definitions is that $\forall P. \quotep{P}
  \not\in \freenames{P}$.
\end{remark}

\subsection{ Dynamic quote: an example }

Anticipating something of what's to come, consider applying the
substitution, $\widehat{\id{\{}u / z \id{\}}}$, to the following pair
of processes, $\lift{w}{y!(z)}$ and $w[ \lpquote y!(z) \rpquote ]$.

\begin{eqnarray}
	\lift{w}{y!(z)}\widehat{\id{\{}u / z \id{\}}}
		& = &
		\lift{w}{y!(u)} \nonumber\\
	w[ \lpquote y!(z) \rpquote ] \widehat{ \id{\{}u / z \id{\}} }
		& = &
		w[ \lpquote y!(z) \rpquote ] \nonumber
\end{eqnarray}

Because the body of the process between quotes is impervious to
substitution, we get radically different answers. In fact, by
examining the first process in an input context,
e.g. $x?(z).\lift{w}{y!(z)}$, we see that the process under the lift
operator may be shaped by prefixed inputs binding a name inside it. In
this sense, the lift operator will be seen as a way to dynamically
construct processes before reifying them as names.

Finally equipped with these standard features we can present the
dynamics of the calculus.

\subsubsection{Operational semantics} 

Finally, we introduce the computational dynamics. What marks these
algebras as distinct from other more traditionally studied algebraic
structures, e.g. vector spaces or polynomial rings, is the manner in
which dynamics is captured. In traditional structures, dynamics is typically
expressed through morphisms between such structures, as in linear maps
between vector spaces or morphisms between rings. In algebras
associated with the semantics of computation, the dynamics is
expressed as part of the algebraic structure itself, through a
reduction reduction relation typically denoted by $\red$. Below, we
give a recursive presentation of this relation for the calculus used
in the encoding.

$\red \subseteq \pi \times \pi$
$\red : \pi \to \mathcal{P}(\pi)$

\begin{mathpar}
  \inferrule* [lab=Comm] { \textsf{match}( x_{src}, x_{trgt} ) } { x_{trgt}?(y)P \; | \; x_{src}!\langle {Q} \rangle \red P\{\quotep{Q}/y}\} }
  \and \\
  \inferrule* [lab=Par] {{P} \red {P}'} {{{P} | {Q}} \red {{P}' | {Q}}}
  \and
  \inferrule* [lab=Equiv]{{{P} \scong {P}'} \andalso {{P}' \red {Q}'} \andalso {{Q}' \scong {Q}}}{{P} \red {Q}}
\end{mathpar}

\begin{eqnarray*}
  match_{\equiv} (\quotep{P},\quotep{Q}) & := & P \equiv Q \\
  match_{\dagger}(\quotep{P},\quotep{Q}) & := & \forall R. P|Q \red^{*} R => R \red^{*} 0 \\
  match_{K}(\quotep{P},\quotep{Q}) & := & K \mbox{ for some context } K
\end{eqnarray*}

$u?(x)P | u!\langle Q \rangle \red P\{\quotep{Q}/x\}$

%We write $\wred$ for $\red^*$, and $P\red$ if $\exists Q $ such that $ P \red Q$.
We write $P\red$ if $\exists Q $ such that $ P \red Q$ and $P\not\red$, otherwise.

\section{Replication}

As mentioned before, it is known that replication (and hence
recursion) can be implemented in a higher-order process algebra
\cite{SangiorgiWalker}. As our first example of calculation with the
machinery thus far presented we give the construction explicitly in
the {\rhoc}.

\begin{eqnarray}
	D_{x} & := & \prefix{x}{y}{(\binpar{\outputp{x}{y}}{@{y}})} \nonumber\\
	\bangp_{x}{P} & := & \binpar{{x}!\langle{\binpar{D_{x}}{P}}\rangle}{D_{x}} \nonumber
\end{eqnarray}

\begin{eqnarray}
	\bangp_{x}{P} & & \nonumber\\
	=
	& {x}!\langle{(\prefix{x}{y}{(\outputp{x}{y} | @{y})) | P}}\rangle 
	      | \prefix{x}{y}{(\outputp{x}{y} | @{y})} & \nonumber\\
	\red
	& (\outputp{x}{y} | @{y})\substn{\quotep{(\prefix{x}{y}{(@{y} | \outputp{x}{y})) | P}}}{y} & \nonumber\\
	=
	& \outputp{x}{\quotep{(\prefix{x}{y}{(\outputp{x}{y} | @{y})) | P}}}
	  | {(\prefix{x}{y}{(\outputp{x}{y} | @{y})) | P}} & \nonumber\\
	\red
	& \ldots & \nonumber\\
	\red^*
	& P | P | \ldots & \nonumber
\end{eqnarray}

Of course, this encoding, as an implementation, runs away, unfolding
$\bangp{P}$ eagerly. A lazier and more implementable replication
operator, restricted to input-guarded processes, may be obtained as follows.

\begin{eqnarray}
\bangp{\prefix{u}{v}{P}} 
	:= 
	\binpar{\lift{x}{\prefix{u}{v}{(\binpar{D(x)}{P})}}}{D(x)} \nonumber
\end{eqnarray}

\begin{remark}
  Note that the lazier definition still does not deal with summation
  or mixed summation (i.e. sums over input and output). The reader is
  invited to construct definitions of replication that deal with these
  features. 

  Further, the definitions are parameterized in a name, $x$. Can you,
  gentle reader, make a definition that eliminates this parameter and
  guarantees no accidental interaction between the replication
  machinery and the process being replicated -- i.e. no accidental
  sharing of names used by the process to get its work done and the
  name(s) used by the replication to effect copying. This latter
  revision of the definition of replication is crucial to obtaining
  the expected identity $!!P \sim !P$.
\end{remark}

\begin{remark}\label{rem:paradoxical_combinator}
  The reader familiar with the lambda calculus will have noticed the
  similarity between $D$ and the paradoxical combinator.

  [Ed. note: the existence of this seems to suggest we have to be more
  restrictive on the set of processes and names we admit if we are to
  support no-cloning.]
\end{remark}

\subsubsection{Bisimulation}

The computational dynamics gives rise to another kind of equivalence,
the equivalence of computational behavior. As previously mentioned
this is typically captured \emph{via} some form of bisimulation.

% The notion we use in this paper is weak barbed bisimulation
% \cite{milner91polyadicpi}.

The notion we use in this paper is derived from weak barbed
bisimulation \cite{milner91polyadicpi}. 

\begin{definition}
An \emph{observation relation}, $\downarrow_{\mathcal N}$, over a set
of names, $\mathcal N$, is the smallest relation satisfying the rules
below.

\infrule[Out-barb]{y \in {\mathcal N}, \; x \nameeq y}
		  {\outputp{x}{v} \downarrow_{\mathcal N} x}
\infrule[Par-barb]{\mbox{$P\downarrow_{\mathcal N} x$ or $Q\downarrow_{\mathcal N} x$}}
		  {\binpar{P}{Q} \downarrow_{\mathcal N} x}

We write $P \Downarrow_{\mathcal N} x$ if there is $Q$ such that 
$P \wred Q$ and $Q \downarrow_{\mathcal N} x$.
\end{definition}

\begin{definition}
%\label{def.bbisim}
An  ${\mathcal N}$-\emph{barbed bisimulation} over a set of names, ${\mathcal N}$, is a symmetric binary relation 
${\mathcal S}_{\mathcal N}$ between agents such that $P\rel{S}_{\mathcal N}Q$ implies:
\begin{enumerate}
\item If $P \red P'$ then $Q \wred Q'$ and $P'\rel{S}_{\mathcal N} Q'$.
\item If $P\downarrow_{\mathcal N} x$, then $Q\Downarrow_{\mathcal N} x$.
\end{enumerate}
$P$ is ${\mathcal N}$-barbed bisimilar to $Q$, written
$P \wbbisim_{\mathcal N} Q$, if $P \rel{S}_{\mathcal N} Q$ for some ${\mathcal N}$-barbed bisimulation ${\mathcal S}_{\mathcal N}$.
\end{definition}

$\mathcal{R} \subseteq \pi \times \pi$

$P \mathcal{R} Q => \forall P'. P \red P' \Rightarrow \exists Q'. Q \red Q', P' \mathcal{R} Q'$

$P \vdash x \Rightarrow Q \vdash x$

\begin{mathpar}
  \inferrule*[lab=Out-barb]{x \nameeq y}{{y}!\langle{Q}\rangle \vdash x}
  \and
  \inferrule*[lab=Par-barb]{\mbox{$P\vdash x$ or $Q\vdash x$}}{\binpar{P}{Q} \vdash x}
\end{mathpar}

\subsubsection{Contexts}

One of the principle advantages of computational calculi like the
$\pi$-calculus is a well-defined notion of context,
contextual-equivalence and a correlation between
contextual-equivalence and notions of bisimulation. The notion of
context allows the decomposition of a process into (sub-)process and
its syntactic environment, its context. Thus, a context may be
thought of as a process with a ``hole'' (written $\Box$) in it. The
application of a context $M$ to a process $P$, written $M[P]$, is
tantamount to filling the hole in $M$ with $P$. In this paper we do
not need the full weight of this theory, but do make use of the notion
of context in the proof the main theorem. 

\begin{mathpar}
  \inferrule* [lab=summation] {} {{M_{M},M_{N}} \bc \Box \;|\; x.M_{A} \;|\; M_{M}+M_{N}}
  \and
  \inferrule* [lab=agent] {} {{M_{A}} \bc (\vec{x})M_{P} \;| \; \clift{P_0,\ldots,M_{P},\ldots,P_N}}
  \and \\
  \inferrule* [lab=process] {} {{M_{P}} \bc M_{N} \;| \;P|M_{P} }
\end{mathpar} 

\begin{mathpar}
  \inferrule* [lab=sychronization] {} {M_{N} \bc \Box \;|\; x?M_{F} \;|\; x!M_{C}}
  \and
  \inferrule* [lab=abstraction] {} {{M_{F}} \bc (x)M_{P} }
  \and
  \inferrule* [lab=concretion] {} {{M_{C}} \bc \langle M_{P} \rangle }
  \and \\
  \inferrule* [lab=process] {} {{M_{P}} \bc M_{N} \;| \;P|M_{P} }
\end{mathpar}

\begin{definition}[contextual application] Given a context $M$, and
  process $P$, we define the \emph{contextual application}, $M[P] :=
  M\{P/\Box\}$. That is, the contextual application of M to P is the
  substitution of $P$ for $\Box$ in $M$.
\end{definition}

$\meaningof{-} : L \to \mathcal{P}(\pi)$

\begin{mathpar}
  \inferrule* [lab=collection] {} {\meaningof{true} = \pi, \and \meaningof{~E} = \pi \setminus \meaningof{E}, \and \meaningof{E_{1} \& E_{2}} = \meaningof{E_{1}} \cap \meaningof{E_{2}}}
\end{mathpar}

\begin{mathpar}
  \inferrule* [lab=structure] {} {\meaningof{0} = \{ P \in \pi | P \equiv 0 \}, \and \\ \meaningof{E_1 | E_2} = \{ P \in \pi | P \equiv P_{1} | P_{2}, P_{1} \in \meaningof{E_{1}}, P_{2} \in \meaningof{E_2}\} }
\end{mathpar}

\begin{mathpar}
 \inferrule* [lab=behavior] {} {\meaningof{\langle a?b \rangle E} = \{ P \in \pi | P \equiv Q | u?(y)P', \\ \and \\\\ \and \\ \;\;\; u \in \meaningof{a}, \forall z.P'\{z/y\} \in \meaningof{E\{z/b\}}\}, \and \\ \meaningof{a!E} = \{ P \in \pi | P \equiv Q | x!\langle P' \rangle, x \in \meaningof{a} P' \in \meaningof{E}\} }
\end{mathpar}

\begin{mathpar}
 \inferrule* [lab=nominal] {} {\meaningof{\quotep{E}} = \{ \quotep{P} \in \quotep{\pi} | P \in \meaningof{E} \}, \and \meaningof{\quotep{P}} = \{ \quotep{Q} \in \quotep{\pi} | P \equiv Q \} \and \\ \meaningof{@\quotep{E}} = \{ P \in \pi | P \equiv @x, x \in \meaningof{E} \}}
\end{mathpar}

\begin{eqnarray*}
  \\
  \meaningof{-} : TS \to ST
\end{eqnarray*}

\begin{eqnarray*}
  \\
  L : TS \to ST
\end{eqnarray*}

\begin{eqnarray*}
  \\
  P \models E \iff P \in \meaningof{E}
\end{eqnarray*}

\begin{eqnarray*}
  P \approx_{L} Q \iff \forall E \in L. P \models E \iff Q \models E
\end{eqnarray*}

\begin{eqnarray*}
  P \approx_{K} Q
\end{eqnarray*}

\begin{eqnarray*}
  P \approx Q
\end{eqnarray*}

$\approx_{K} = \approx = \approx_{L}$

\subsubsection{Contextual duality}

Note that contexts extend the quotation operation to a family of
operations from processes to names. Given a context, $M$, we can
define a \emph{nominal context}, $\quotep{M}$ by $\quotep{M}[P] :=
\quotep{M[P]}$. To foreshadow what is to come we observe that these
operations enjoy a duality with processes very much like the duality
between vectors and maps from vectors to scalars.

Further, because the calculus is essentially higher-order, we have a
correspondence between contexts and processes. More specifically,
given a name $x$ and a context $M$ we can construct $M^{*}_{x}$ such
that 

\begin{mathpar}
  M^{*}_{x} | \lift{x}{P} \red M[P]
\end{mathpar}

namely,

\begin{mathpar}
  M^{*}_{x} := x?(u).M[\dropn{u}]
\end{mathpar}

The dependence of $M^{*}_{x}$ on a name makes it an abstraction, 

\begin{mathpar}
  M^{*} := (x)x?(u).M[\dropn{u}]
\end{mathpar}

\subsection{Additional notation}

It will sometimes be convenient to denote the process a name
quotes. We already have the notation $x = \quotep{P}$, but it will be
convenient to introduce an alternate notation, $\procn{x}$, when we
want to emphasize the connection to the use of the name. Note that, by
virtue of name equivalence, $\quotep{\procn{x}} \nameeq x$; so, the
notation is consistent with previous definitions.

Further, because names have structure it is possible to effect
substitutions on the basis of that structure. This means we need to
upgrade our notation for substitutions, which we accomplish by
adapting comprehension notation. Thus,

\begin{mathpar}
  P\{ y / x : x \in S \}
\end{mathpar}

is interpreted to mean the process derived from P by replacing (in a
capture-avoiding manner) each occurrence of $x$ in $S$ by $y$. For example,

\begin{mathpar}
  P\{ \quotep{\procn{x}|\procn{x}} / x : x \in \freenames{P} \}
\end{mathpar}

will replace each (occurrence) of a free name $x$ in $P$ by
$\quotep{\procn{x}|\procn{x}}$.

Also, we will avail ourselves of the notation $x^{L}$ and $x^{R}$ to
denote injections of a name into disjoint copies of the name
space. There are numerous ways to accomplish this. One example can be
found in \cite{MeredithR05}. This notation overloads to vectors of
names: $\vec{x}^{\pi} := (x_{i}^{\pi} \; : \; 0 \leq i < |\vec{x}| )$ where $\pi \in \{L,R\}$.

We also use $P^{\Box} := P|\Box$.

In \cite{MeredithR05} an interpretation of the new operator is
given. It turns out that there are several possible interpretations
all enjoying the requisite algebraic properties of the operator (see
\cite{milner91polyadicpi}). We will therefore make liberal use of
$(\nu\; \vec{x})P$.

% subsection the_syntax_and_semantics_of_the_notation_system (end)   

\input{qm2pi.qmops} 

\input{qm2pi.sterngerlach} 

\input{qm2pi.metric} 

% section concurrent_process_calculi (end)

%\input{qm2pi.proofsketch}

% section proof sketch (end)

%\input{qm2pi.slviaknots} 

% section spatial logic via knots (end)

\input{qm2pi.conclusion}

% section conclusion (end)

%\input{qm2pi.dtcodes} 

% section wiring algorithm (end)

\input{qm2pi.ack} 

% section acknowledgments (end)

\newpage


\bibliographystyle{plain}   
\bibliography{../../biblios/main.bib}

\input{qm2pi.rhodetails}

\end{document}

 

% subsection basic_interpretation (end)

%\input{qm2pi.rho.presentation} 
\subsection{The syntax and semantics of the notation system}\label{sub:the_syntax_and_semantics_of_the_notation_system} % (fold)

We now summarize a technical presentation of the calculus that
embodies our theory of dynamics. The typical presentation of such a
calculus follows the style of giving generators and relations on
them. The grammar, below, describing term constructors, freely
generates the set of processes, $\Proc$. This set is then quotiented
by a relation known as structural congruence and it is over this set
that the notion of dynamics is expressed. This presentation is
essentially that of \cite{MeredithR05} with the addition of
polyadicity and summation. For readability we have relegated some of
the technical subtleties to an appendix.

\subsubsection{Process grammar}\label{subsub:process_grammar}

\begin{mathpar}
  \inferrule* [lab=synchronization] {} {{M} \bc \pzero \;|\; x?F \;|\; x!C }
  \and
  \inferrule* [lab=abstraction] {} {{F} \bc (x)P}
  \and
  \inferrule* [lab=concretion] {} {{C} \bc \langle Q \rangle}
  \and
  \inferrule* [lab=process] {} {{P,Q} \bc M \;| \;P|Q \;|\; @{x}}
  \and
  \inferrule* [lab=name] {} {{x} \bc \quotep{P}}
\end{mathpar} 

Note that $\vec{x}$ (resp. $\vec{P}$) denotes a vector of names
(resp. processes) of length $|\vec{x}|$ (resp. $|\vec{P}|$). We adopt
the following useful abbreviations.

\begin{mathpar}
   x?(\vec{y}).P := x.(\vec{y})P \and  x\clift{\vec{P}} := x.\clift{\vec{P}}
   \and x!(y) := \lift{x}{\dropn{y}}
   \and \Pi_{i=0}^{n-1}P_i := P_0 | \ldots | P_{n-1}
\end{mathpar}

\subsubsection{Structural congruence}

\paragraph{Free and bound names and alpha-equivalence.} At the
core of structural equivalence is alpha-equivalence which identifies
process that are the same up to a change of variable. Formally, we
recognize the distinction between free and bound names. The free names
of a process, $\freenames{P}$, may be calculated recursively as
follows:

\begin{mathpar}
\freenames{\pzero} := \emptyset
  \and \\
  \freenames{x?(y).P} := \{ x \} \cup (\freenames{P} \setminus \{ y \})
  \and 
  \freenames{x!\langle P \rangle} := \{ x \} \cup \{ P \} 
  \and \\
  \freenames{P|Q} := \freenames{P} \cup \freenames{Q}
  \and \\
  \freenames{@{x}} := \{ x \}
\end{mathpar}

$\pi$
$\quotep{\pi}$

$\freenames{-} : \pi \to \mathcal{P}(\quotep{\pi})$

\begin{eqnarray*}
  \freenames{\pzero} & := & \emptyset \\
  \freenames{x?(y).P} & := & \{ x \} \cup (\freenames{P} \setminus \{ y \}) \\
  \freenames{x!\langle P \rangle} & := & \{ x \} \cup \{ P \} \\
  \freenames{P|Q} & := & \freenames{P} \cup \freenames{Q} \\
  \freenames{\dropn{x}} & := & \{ x \}
\end{eqnarray*}

The bound names of a process, $\boundnames{P}$, are those names occurring in $P$
that are not free. For example, in $x?(y).0$, the name $x$ is free, while $y$ is bound.

\begin{mathpar}
  \inferrule* [lab=monoidal-laws] {} { P|Q \equiv Q|P \and P|0 \equiv P \and P|(Q|R) \equiv (P|Q)|R }
\end{mathpar}

\begin{mathpar}
  \inferrule* [lab=alpha-equivalence] {} { (x)P \equiv (y)P\{y/x\} \and y \not\in \freenames{P} }
\end{mathpar}

\begin{definition}
Then two processes, $P,Q$, are alpha-equivalent if $P = Q\{\vec{y}/\vec{x}\}$ for
some $\vec{x} \in \boundnames{Q},\vec{y} \in \boundnames{P}$, where $Q\{\vec{y}/\vec{x}\}$
denotes the capture-avoiding substitution of $\vec{y}$ for $\vec{x}$ in $Q$.
\end{definition}

\begin{definition}
  The {\em structural congruence} \cite{SangiorgiWalker} , $\equiv$,
  between processes is the least congruence containing
  alpha-equivalence, satisfying the abelian monoid laws
  (associativity, commutativity and $\pzero$ as identity) for parallel
  composition $|$ and for summation $+$.
\end{definition}

\subsection{Name equivalence}

We take name equivalence, written $\nameeq$, to be the smallest
equivalence relation generated by the following rules.

\begin{mathpar}
\inferrule*[lab=Quote-drop]
{ }
{ \quotep{@{x}} \nameeq x }

\inferrule*[lab=Struct-equiv]
{ P \scong Q }
{ \quotep{P} \nameeq \quotep{Q} }
\end{mathpar}

The astute reader will have noticed that the mutual recursion of names
and processes imposes a mutual recursion on alpha-equivalence and
structural equivalence via name-equivalence. Fortunately, all of this
works out pleasantly and we may calculate in the natural way, free of
concern. The reader interested in the details is referred to the
appendix \ref{appendix:rho_details}.

\subsection{Substitution}

We use $\Proc$ for the set of processes, $\QProc$ for the set of
names, and $\id{\{}\vec{y} / \vec{x} \id{\}}$ to denote partial maps,
$s : \QProc \rightarrow \QProc$. A map, $s$ lifts, uniquely, to a map
on process terms, $\widehat{s} : \Proc \rightarrow \Proc$ by the
following equations.

\begin{mathpar}
  (0) \psubstp{Q}{P} := 0 \\
  (R \juxtap S) \psubstp{Q}{P}
  :=    
  (R)\psubstp{Q}{P} \juxtap (S) \psubstp{Q}{P} \\
  (x?(y).R) \psubstp{Q}{P}    
  :=    
  (x)\substp{Q}{P} (z)\concat( (R \psubstn{z}{y}) \psubstp{Q}{P} ) \\
  (\lift{x}{R}) \psubstp{Q}{P}  
  :=
  \lift{(x)\substp{Q}{P}}{ R \psubstp{Q}{P} } \\
%   (\dropn{x})  \psubstp{Q}{P}       
%   := 
%   \left\{ 
%     \begin{array}{ccc} 
%       \dropn{\quotep{Q}} & & x \nameeq \quotep{P} \\
%       \dropn{x} & & otherwise \\
%     \end{array}
%   \right. 
  (\dropn{x})  \psubstp{Q}{P}       
  := 
  \left\{ 
    \begin{array}{ccc} 
      Q & & x \nameeq \quotep{P} \\
      \dropn{x} & & otherwise \\
    \end{array}
  \right.
\end{mathpar}
 

where

\begin{eqnarray}
  (x)\id{\{} \lpquote Q \rpquote / \lpquote P \rpquote \id{\}}            = 
  \left\{ 
    \begin{array}{ccc}
      \lpquote Q \rpquote & & x \nameeq \lpquote P \rpquote \\
      x & & otherwise \\
    \end{array}
  \right. \nonumber
\end{eqnarray}

and $z$ is chosen distinct from $\quotep{P}$, $\quotep{Q}$, the free
names in $Q$, and all the names in $R$. Our $\alpha$-equivalence will
be built in the standard way from this substitution.

\begin{remark}\label{rem:no_self_referential_names}
  One consequence of these definitions is that $\forall P. \quotep{P}
  \not\in \freenames{P}$.
\end{remark}

\subsection{ Dynamic quote: an example }

Anticipating something of what's to come, consider applying the
substitution, $\widehat{\id{\{}u / z \id{\}}}$, to the following pair
of processes, $\lift{w}{y!(z)}$ and $w[ \lpquote y!(z) \rpquote ]$.

\begin{eqnarray}
	\lift{w}{y!(z)}\widehat{\id{\{}u / z \id{\}}}
		& = &
		\lift{w}{y!(u)} \nonumber\\
	w[ \lpquote y!(z) \rpquote ] \widehat{ \id{\{}u / z \id{\}} }
		& = &
		w[ \lpquote y!(z) \rpquote ] \nonumber
\end{eqnarray}

Because the body of the process between quotes is impervious to
substitution, we get radically different answers. In fact, by
examining the first process in an input context,
e.g. $x?(z).\lift{w}{y!(z)}$, we see that the process under the lift
operator may be shaped by prefixed inputs binding a name inside it. In
this sense, the lift operator will be seen as a way to dynamically
construct processes before reifying them as names.

Finally equipped with these standard features we can present the
dynamics of the calculus.

\subsubsection{Operational semantics} 

Finally, we introduce the computational dynamics. What marks these
algebras as distinct from other more traditionally studied algebraic
structures, e.g. vector spaces or polynomial rings, is the manner in
which dynamics is captured. In traditional structures, dynamics is typically
expressed through morphisms between such structures, as in linear maps
between vector spaces or morphisms between rings. In algebras
associated with the semantics of computation, the dynamics is
expressed as part of the algebraic structure itself, through a
reduction reduction relation typically denoted by $\red$. Below, we
give a recursive presentation of this relation for the calculus used
in the encoding.

$\red \subseteq \pi \times \pi$
$\red : \pi \to \mathcal{P}(\pi)$

\begin{mathpar}
  \inferrule* [lab=Comm] { \textsf{match}( x_{src}, x_{trgt} ) } { x_{trgt}?(y)P \; | \; x_{src}!\langle {Q} \rangle \red P\{\quotep{Q}/y}\} }
  \and \\
  \inferrule* [lab=Par] {{P} \red {P}'} {{{P} | {Q}} \red {{P}' | {Q}}}
  \and
  \inferrule* [lab=Equiv]{{{P} \scong {P}'} \andalso {{P}' \red {Q}'} \andalso {{Q}' \scong {Q}}}{{P} \red {Q}}
\end{mathpar}

\begin{eqnarray*}
  match_{\equiv} (\quotep{P},\quotep{Q}) & := & P \equiv Q \\
  match_{\dagger}(\quotep{P},\quotep{Q}) & := & \forall R. P|Q \red^{*} R => R \red^{*} 0 \\
  match_{K}(\quotep{P},\quotep{Q}) & := & K \mbox{ for some context } K
\end{eqnarray*}

$u?(x)P | u!\langle Q \rangle \red P\{\quotep{Q}/x\}$

%We write $\wred$ for $\red^*$, and $P\red$ if $\exists Q $ such that $ P \red Q$.
We write $P\red$ if $\exists Q $ such that $ P \red Q$ and $P\not\red$, otherwise.

\section{Replication}

As mentioned before, it is known that replication (and hence
recursion) can be implemented in a higher-order process algebra
\cite{SangiorgiWalker}. As our first example of calculation with the
machinery thus far presented we give the construction explicitly in
the {\rhoc}.

\begin{eqnarray}
	D_{x} & := & \prefix{x}{y}{(\binpar{\outputp{x}{y}}{@{y}})} \nonumber\\
	\bangp_{x}{P} & := & \binpar{{x}!\langle{\binpar{D_{x}}{P}}\rangle}{D_{x}} \nonumber
\end{eqnarray}

\begin{eqnarray}
	\bangp_{x}{P} & & \nonumber\\
	=
	& {x}!\langle{(\prefix{x}{y}{(\outputp{x}{y} | @{y})) | P}}\rangle 
	      | \prefix{x}{y}{(\outputp{x}{y} | @{y})} & \nonumber\\
	\red
	& (\outputp{x}{y} | @{y})\substn{\quotep{(\prefix{x}{y}{(@{y} | \outputp{x}{y})) | P}}}{y} & \nonumber\\
	=
	& \outputp{x}{\quotep{(\prefix{x}{y}{(\outputp{x}{y} | @{y})) | P}}}
	  | {(\prefix{x}{y}{(\outputp{x}{y} | @{y})) | P}} & \nonumber\\
	\red
	& \ldots & \nonumber\\
	\red^*
	& P | P | \ldots & \nonumber
\end{eqnarray}

Of course, this encoding, as an implementation, runs away, unfolding
$\bangp{P}$ eagerly. A lazier and more implementable replication
operator, restricted to input-guarded processes, may be obtained as follows.

\begin{eqnarray}
\bangp{\prefix{u}{v}{P}} 
	:= 
	\binpar{\lift{x}{\prefix{u}{v}{(\binpar{D(x)}{P})}}}{D(x)} \nonumber
\end{eqnarray}

\begin{remark}
  Note that the lazier definition still does not deal with summation
  or mixed summation (i.e. sums over input and output). The reader is
  invited to construct definitions of replication that deal with these
  features. 

  Further, the definitions are parameterized in a name, $x$. Can you,
  gentle reader, make a definition that eliminates this parameter and
  guarantees no accidental interaction between the replication
  machinery and the process being replicated -- i.e. no accidental
  sharing of names used by the process to get its work done and the
  name(s) used by the replication to effect copying. This latter
  revision of the definition of replication is crucial to obtaining
  the expected identity $!!P \sim !P$.
\end{remark}

\begin{remark}\label{rem:paradoxical_combinator}
  The reader familiar with the lambda calculus will have noticed the
  similarity between $D$ and the paradoxical combinator.

  [Ed. note: the existence of this seems to suggest we have to be more
  restrictive on the set of processes and names we admit if we are to
  support no-cloning.]
\end{remark}

\subsubsection{Bisimulation}

The computational dynamics gives rise to another kind of equivalence,
the equivalence of computational behavior. As previously mentioned
this is typically captured \emph{via} some form of bisimulation.

% The notion we use in this paper is weak barbed bisimulation
% \cite{milner91polyadicpi}.

The notion we use in this paper is derived from weak barbed
bisimulation \cite{milner91polyadicpi}. 

\begin{definition}
An \emph{observation relation}, $\downarrow_{\mathcal N}$, over a set
of names, $\mathcal N$, is the smallest relation satisfying the rules
below.

\infrule[Out-barb]{y \in {\mathcal N}, \; x \nameeq y}
		  {\outputp{x}{v} \downarrow_{\mathcal N} x}
\infrule[Par-barb]{\mbox{$P\downarrow_{\mathcal N} x$ or $Q\downarrow_{\mathcal N} x$}}
		  {\binpar{P}{Q} \downarrow_{\mathcal N} x}

We write $P \Downarrow_{\mathcal N} x$ if there is $Q$ such that 
$P \wred Q$ and $Q \downarrow_{\mathcal N} x$.
\end{definition}

\begin{definition}
%\label{def.bbisim}
An  ${\mathcal N}$-\emph{barbed bisimulation} over a set of names, ${\mathcal N}$, is a symmetric binary relation 
${\mathcal S}_{\mathcal N}$ between agents such that $P\rel{S}_{\mathcal N}Q$ implies:
\begin{enumerate}
\item If $P \red P'$ then $Q \wred Q'$ and $P'\rel{S}_{\mathcal N} Q'$.
\item If $P\downarrow_{\mathcal N} x$, then $Q\Downarrow_{\mathcal N} x$.
\end{enumerate}
$P$ is ${\mathcal N}$-barbed bisimilar to $Q$, written
$P \wbbisim_{\mathcal N} Q$, if $P \rel{S}_{\mathcal N} Q$ for some ${\mathcal N}$-barbed bisimulation ${\mathcal S}_{\mathcal N}$.
\end{definition}

$\mathcal{R} \subseteq \pi \times \pi$

$P \mathcal{R} Q => \forall P'. P \red P' \Rightarrow \exists Q'. Q \red Q', P' \mathcal{R} Q'$

$P \vdash x \Rightarrow Q \vdash x$

\begin{mathpar}
  \inferrule*[lab=Out-barb]{x \nameeq y}{{y}!\langle{Q}\rangle \vdash x}
  \and
  \inferrule*[lab=Par-barb]{\mbox{$P\vdash x$ or $Q\vdash x$}}{\binpar{P}{Q} \vdash x}
\end{mathpar}

\subsubsection{Contexts}

One of the principle advantages of computational calculi like the
$\pi$-calculus is a well-defined notion of context,
contextual-equivalence and a correlation between
contextual-equivalence and notions of bisimulation. The notion of
context allows the decomposition of a process into (sub-)process and
its syntactic environment, its context. Thus, a context may be
thought of as a process with a ``hole'' (written $\Box$) in it. The
application of a context $M$ to a process $P$, written $M[P]$, is
tantamount to filling the hole in $M$ with $P$. In this paper we do
not need the full weight of this theory, but do make use of the notion
of context in the proof the main theorem. 

\begin{mathpar}
  \inferrule* [lab=summation] {} {{M_{M},M_{N}} \bc \Box \;|\; x.M_{A} \;|\; M_{M}+M_{N}}
  \and
  \inferrule* [lab=agent] {} {{M_{A}} \bc (\vec{x})M_{P} \;| \; \clift{P_0,\ldots,M_{P},\ldots,P_N}}
  \and \\
  \inferrule* [lab=process] {} {{M_{P}} \bc M_{N} \;| \;P|M_{P} }
\end{mathpar} 

\begin{mathpar}
  \inferrule* [lab=sychronization] {} {M_{N} \bc \Box \;|\; x?M_{F} \;|\; x!M_{C}}
  \and
  \inferrule* [lab=abstraction] {} {{M_{F}} \bc (x)M_{P} }
  \and
  \inferrule* [lab=concretion] {} {{M_{C}} \bc \langle M_{P} \rangle }
  \and \\
  \inferrule* [lab=process] {} {{M_{P}} \bc M_{N} \;| \;P|M_{P} }
\end{mathpar}

\begin{definition}[contextual application] Given a context $M$, and
  process $P$, we define the \emph{contextual application}, $M[P] :=
  M\{P/\Box\}$. That is, the contextual application of M to P is the
  substitution of $P$ for $\Box$ in $M$.
\end{definition}

$\meaningof{-} : L \to \mathcal{P}(\pi)$

\begin{mathpar}
  \inferrule* [lab=collection] {} {\meaningof{true} = \pi, \and \meaningof{~E} = \pi \setminus \meaningof{E}, \and \meaningof{E_{1} \& E_{2}} = \meaningof{E_{1}} \cap \meaningof{E_{2}}}
\end{mathpar}

\begin{mathpar}
  \inferrule* [lab=structure] {} {\meaningof{0} = \{ P \in \pi | P \equiv 0 \}, \and \\ \meaningof{E_1 | E_2} = \{ P \in \pi | P \equiv P_{1} | P_{2}, P_{1} \in \meaningof{E_{1}}, P_{2} \in \meaningof{E_2}\} }
\end{mathpar}

\begin{mathpar}
 \inferrule* [lab=behavior] {} {\meaningof{\langle a?b \rangle E} = \{ P \in \pi | P \equiv Q | u?(y)P', \\ \and \\\\ \and \\ \;\;\; u \in \meaningof{a}, \forall z.P'\{z/y\} \in \meaningof{E\{z/b\}}\}, \and \\ \meaningof{a!E} = \{ P \in \pi | P \equiv Q | x!\langle P' \rangle, x \in \meaningof{a} P' \in \meaningof{E}\} }
\end{mathpar}

\begin{mathpar}
 \inferrule* [lab=nominal] {} {\meaningof{\quotep{E}} = \{ \quotep{P} \in \quotep{\pi} | P \in \meaningof{E} \}, \and \meaningof{\quotep{P}} = \{ \quotep{Q} \in \quotep{\pi} | P \equiv Q \} \and \\ \meaningof{@\quotep{E}} = \{ P \in \pi | P \equiv @x, x \in \meaningof{E} \}}
\end{mathpar}

\begin{eqnarray*}
  \\
  \meaningof{-} : TS \to ST
\end{eqnarray*}

\begin{eqnarray*}
  \\
  L : TS \to ST
\end{eqnarray*}

\begin{eqnarray*}
  \\
  P \models E \iff P \in \meaningof{E}
\end{eqnarray*}

\begin{eqnarray*}
  P \approx_{L} Q \iff \forall E \in L. P \models E \iff Q \models E
\end{eqnarray*}

\begin{eqnarray*}
  P \approx_{K} Q
\end{eqnarray*}

\begin{eqnarray*}
  P \approx Q
\end{eqnarray*}

$\approx_{K} = \approx = \approx_{L}$

\subsubsection{Contextual duality}

Note that contexts extend the quotation operation to a family of
operations from processes to names. Given a context, $M$, we can
define a \emph{nominal context}, $\quotep{M}$ by $\quotep{M}[P] :=
\quotep{M[P]}$. To foreshadow what is to come we observe that these
operations enjoy a duality with processes very much like the duality
between vectors and maps from vectors to scalars.

Further, because the calculus is essentially higher-order, we have a
correspondence between contexts and processes. More specifically,
given a name $x$ and a context $M$ we can construct $M^{*}_{x}$ such
that 

\begin{mathpar}
  M^{*}_{x} | \lift{x}{P} \red M[P]
\end{mathpar}

namely,

\begin{mathpar}
  M^{*}_{x} := x?(u).M[\dropn{u}]
\end{mathpar}

The dependence of $M^{*}_{x}$ on a name makes it an abstraction, 

\begin{mathpar}
  M^{*} := (x)x?(u).M[\dropn{u}]
\end{mathpar}

\subsection{Additional notation}

It will sometimes be convenient to denote the process a name
quotes. We already have the notation $x = \quotep{P}$, but it will be
convenient to introduce an alternate notation, $\procn{x}$, when we
want to emphasize the connection to the use of the name. Note that, by
virtue of name equivalence, $\quotep{\procn{x}} \nameeq x$; so, the
notation is consistent with previous definitions.

Further, because names have structure it is possible to effect
substitutions on the basis of that structure. This means we need to
upgrade our notation for substitutions, which we accomplish by
adapting comprehension notation. Thus,

\begin{mathpar}
  P\{ y / x : x \in S \}
\end{mathpar}

is interpreted to mean the process derived from P by replacing (in a
capture-avoiding manner) each occurrence of $x$ in $S$ by $y$. For example,

\begin{mathpar}
  P\{ \quotep{\procn{x}|\procn{x}} / x : x \in \freenames{P} \}
\end{mathpar}

will replace each (occurrence) of a free name $x$ in $P$ by
$\quotep{\procn{x}|\procn{x}}$.

Also, we will avail ourselves of the notation $x^{L}$ and $x^{R}$ to
denote injections of a name into disjoint copies of the name
space. There are numerous ways to accomplish this. One example can be
found in \cite{MeredithR05}. This notation overloads to vectors of
names: $\vec{x}^{\pi} := (x_{i}^{\pi} \; : \; 0 \leq i < |\vec{x}| )$ where $\pi \in \{L,R\}$.

We also use $P^{\Box} := P|\Box$.

In \cite{MeredithR05} an interpretation of the new operator is
given. It turns out that there are several possible interpretations
all enjoying the requisite algebraic properties of the operator (see
\cite{milner91polyadicpi}). We will therefore make liberal use of
$(\nu\; \vec{x})P$.

% subsection the_syntax_and_semantics_of_the_notation_system (end)   

\section{Interpretation of QM}
\subsection{Supporting definitions}
\subsubsection{Multiplication}
\begin{mathpar}
  \quotep{Q} \cdot \quotep{R} := \quotep{Q|R}
  \and \\
  \quotep{Q} \cdot P := P\{ \quotep{Q|R} / \quotep{R} : \quotep{R} \in \freenames{P} \}
\end{mathpar}

\paragraph{Discussion}
The first line needs little explanation. The second line says that
each free name of the process is replaced with the multiplication of
that name by the scalar. Multiplication of a scalar (name) by a state
(process) results in a process all the names of which have been `moved
over' by parallel composition with the process the scalar
quotes. There is a subtlety that the bound names have to be
manipulated so that multiplied names aren't accidentally
captured. There are many ways to achieve this.

\begin{remark}\label{rem:multiplication_identities}
  The reader is invited to verify that for all $x,y,z \in \QProc$ and $P \in \Proc$
  \begin{mathpar}
    x \cdot \quotep{0} \equiv x 
    \and
    x \cdot y \equiv y \cdot x
    \and
    x \cdot (y \cdot z) \equiv (x \cdot y) \cdot z
    \and \\
    \quotep{0} \cdot P \equiv P
    \and \\
    x \cdot (y \cdot P) \equiv (x \cdot y) \cdot P
    \and \\
    x \cdot (P|Q) \equiv (x \cdot P) | (x \cdot Q)
    \and \\    
  \end{mathpar}
\end{remark}

\subsubsection{Tensor product}

We define a tensor product on processes by structural induction.

\paragraph{Tensor of sums} First note that all summations, including
$\pzero$ and sequence, can be written $\Sigma_{i} x_{i}.A_{i} +
\Sigma_{j} x_{j}.C_{j}$, where we have grouped input-guarded processes
together and output-guarded processes together.

Thus, we can define the tensor product of two summations, $N_{1}\otimes N_{2}$, where

\begin{mathpar}
  N_{1} := \Sigma_{i} x_{i}.A_{i} + \Sigma_{j} x_{j}.C_{j}
  \and
  N_{2} := \Sigma_{i'} y_{i'}.B_{i'} + \Sigma_{j'} y_{j'}.D_{j'} 
\end{mathpar}

as follows.

\begin{mathpar}
  \Sigma_{i} x_{i}.A_{i} + \Sigma_{j} x_{j}.C_{j} \otimes \Sigma_{i'}
  y_{i'}.B_{i'} + \Sigma_{j'} y_{j'}.D_{j'} 
  \and \\
  := \; \Sigma_{i} \Sigma_{i'} \quotep{\stackrel{\vee}{x_{i}}| \stackrel{\vee}{y_{i'}}}.(A_{i}\otimes B_{i'}) \; | \; \Sigma_{i'} \Sigma_{i} \quotep{\stackrel{\vee}{y_{i'}}|\stackrel{\vee}{x_{i}}}.(B_{i'}\otimes A_{i})
  \and
  \;\; | \;\; \Sigma_{j} \Sigma_{j'} \quotep{\stackrel{\vee}{x_{j}}|\stackrel{\vee}{y_{j'}}}.(A_{j}\otimes B_{j'}) \; | \; \Sigma_{j'} \Sigma_{j} \quotep{\stackrel{\vee}{y_{j'}}|\stackrel{\vee}{x_{j}}}.(B_{j'}\otimes A_{j})
\end{mathpar}

\begin{remark}
  Do we need to $x^{L}$ and $y^{R}$ for this construction as well?
\end{remark}

\paragraph{Tensor of parallel compositions} Next, we distribute tensor
over par.

\begin{mathpar}
  P_{1}|P_{2} \otimes Q_{1}|Q_{2} := (P_{1} \otimes Q_{1}) | (P_{1}
  \otimes Q_{2}) | (P_{2} \otimes Q_{1}) | (P_{2} \otimes Q_{2})
\end{mathpar}

\paragraph{Tensor with dropped names} We treat tensor of a
process with a dropped name as parallel composition.

\begin{mathpar}
  P \otimes \dropn{x} := P | \dropn{x}
\end{mathpar}

\paragraph{Tensor of agents}

Finally, we need to define tensor on agents. Note that the definition
of tensor on normal products only tensors inputs with inputs and
outputs with outputs. Thus, we only have to define the operation on
``homogeneous'' pairings.

\begin{mathpar}
  (\vec{x})P \otimes (\vec{y})Q
  \and \\
  := (x_{0}^{L}|y_{0}^{R},\ldots,x_{0}^{L}|y_{n}^{R},\ldots,x_{m}^{L}|y_{0}^{R},\ldots,x_{m}^{L}|y_{n}^R)(P\{ \vec{x}^{L}/\vec{x}\} \otimes Q \{ \vec{y}^{R}/\vec{y}\})
  \and \\
  \clift{\vec{P}} \otimes \clift{\vec{Q}}
  \and \\
  := \clift{P_{0}\otimes Q_{0},\ldots,P_{0}\otimes Q_{n},\ldots,P_{m}\otimes Q_{0},\ldots,P_{m}\otimes Q_{n}}
\end{mathpar}

\begin{remark}
  Observe that arities of tensored abstractions matches arities of
  tensored concretions if the original arities matched. Note also that
  the length of the arities corresponds to the increase in dimension
  we see in ordinary vector space tensor product.
\end{remark}

\begin{remark}
  Operationally, this definition distributes the tensor down to
  components ``linked'' by summation. Tensor over summation is
  intriguing in that it mixes names. Moreover, as a consequence of the
  way it mixes names we have the identities for all $x \in \QProc$ and
  $P,Q \in \Proc$

  \begin{mathpar}
    (x \cdot P) \otimes Q \equiv x \cdot (P \otimes Q) \equiv P \otimes (x \cdot Q)
    \and
    P \otimes \pzero \equiv P
  \end{mathpar}

  that the reader is invited to verify.
\end{remark}

\subsubsection{Annihilation}
\begin{mathpar}
  P^{\perp} := \{ Q | \forall R. P|Q \red^{*} R \Rightarrow R \red^{*} \pzero \}
  \and \\
  P^{\underline{\perp}} := \Sigma_{Q \in P^{\perp}} \quotep{Q}?(y).(\dropn{y}|Q) | \Sigma_{Q \in P^{\perp}} \quotep{Q}\clift{\Box}
\end{mathpar}

\paragraph{Discussion} The reader will note that $P^{\perp}$ is a
\emph{set} of processes, while $P^{\underline{\perp}}$ is a
\emph{context}. We call the set $P^{\perp}$ the \emph{annihilators} of
$P$. The parallel composition of a process in the annihilators of $P$
with $P$ will result in a process, the state space of which has all
paths eventually leading to $\pzero$. Execution may endure loops; but
under reasonable conditions of fairness (naturally guaranteed under
most notions of bisimulation) such a composite process cannot get
stuck in such a loop and will, eventually pop out and terminate.

The context $P^{\underline{\perp}}$ is ready and willing to ``take the
$P$ out of'' the process to which it is applied. It will effectively
transmit the code of the process to which it is applied to one of the
annihilators and run the process against it.

\subsubsection{Evaluation}
We fix $M$ a domain of fully abstract interpretation with an equality
coincident with bisimulation. We take $\meaningof{\cdot} : \Proc \to
M$ to be the map interpreting processes and $\nmeaningof{\cdot} : \M
\to Proc$ to be the map running the other way. Then we define

\begin{mathpar}
  \int P := \nmeaningof{\meaningof{P}}
\end{mathpar}

\paragraph{Discussion}
There are many fully abstract interpretations of Milner's
$\pi$-calculus. Any of them can be used as a basis for interpreting
the reflective calculus here. Equipped with such a domain it is
largely a matter of grinding through to check that the Yoneda
construction for the normalization-by-evaluation program can be
extended to this setting.

\begin{remark}
  The reader is invited to verify that $\int (P^{\underline{\perp}}[P]) = 0$.
\end{remark}

\subsection{Quantum mechanics}

Table \ref{tbl:core_qm_op_defns} gives the core operational definitions

\begin{table}[htp]\label{tbl:core_qm_op_defns}
  \center{
    \fbox{
      \begin{tabular}{c|c}
        quantum mechanics & process calculus \\
        \hline
        scalar & $x := \quotep{P}$ \\
        state vector & $\state{P} := P$ \\
        dual & $\state{P}^{*} := \event{P^{\underline{\perp}}} := \quotep{P^{\underline{\perp}}}[-]$ \\
        matrix & $ \Sigma_{\alpha} \state{P_{\alpha}}x_{\alpha}\event{Q_{\alpha}}$ \\
        vector addition & $\state{P} + \state{Q} := \state{P | Q}$ \\
        tensor product & $\state{P} \otimes \state{Q} := \state{P \otimes Q}$ \\
        inner product & $\innerprod{P}{Q} := \quotep{\int P^{\underline{\perp}}[Q]}$ \\
      \end{tabular}
    }
  }
  \caption{QM - operational definitions}
\end{table}

where

\begin{mathpar}
  \prmatrix{P}{Q} := \fprmatrix{P}{\quotep{\pzero}}{Q}
  \and
  \fprmatrix{P}{x}{Q} := (\state{P},x,\event{Q})
  \and
  (\fprmatrix{P}{x}{Q})(\state{R}) := x \cdot \innerprod{Q}{R} \cdot \state{P}
  \and
  (\fprmatrix{P}{x}{Q})(\event{R}) := x \cdot \innerprod{R}{P} \cdot \event{Q}
\end{mathpar}

\paragraph{Discussion}
As promised: vectors (aka states) are represented as processes; duals
as contextual duals; inner product definition should be compared with
standard inner product definition for ....

\begin{remark}
  Assuming $\int (P^{\underline{\perp}}[P]) = 0$, the reader is
  invited to verify that $(\fprmatrix{P}{x}{P})(\state{P}) = x \cdot \state{P}$.
\end{remark}

\begin{remark}
  The reader is invited to verify that $\innerprod{P}{Q}$ could
  equally well have been written $\quotep{\int \stackrel{\vee}{x}}$
  where $x = \event{P^{\underline{\perp}}}(Q)$.

  One of the motivations for this remark is that there is another way
  to factor these operations. We could package up evaluation in the dual:

  \begin{mathpar}
    \state{P}^{*} := \event{\int P^{\underline{\perp}}} := \quotep{\int P^{\underline{\perp}}}[-]
  \end{mathpar}

  and then have inner product defined by
  
  \begin{mathpar}
    \innerprod{P}{Q} := \event{P}(Q)
  \end{mathpar}

  Hopefully, experience with the calculations will provide guidance on
  the best factoring.
\end{remark}

\begin{remark}
  Assuming $\int (P^{\underline{\perp}}[P]) = 0$, the reader is
  invited to verify that $\forall P,Q. (\prmatrix{0}{Q})(\state{0}) =
  \state{0}$ and dually $(\prmatrix{P}{0})(\event{0}) = \event{0}$.
\end{remark}

\begin{remark}
  i'm a little worried that i don't (yet) have proper support for
  complex conjugacy. But, the observation above may give us a
  clue. According to Abramsky, it must be the case that the scalars
  are iso to the homset of the identity for the tensor -- which the
  observation above characterizes. 

  For now, we will simply bookmark the notion with $\overline{x}$.
\end{remark}

\subsubsection{Adjointness}

We need to give a definition of $(\cdot)^{\dagger}$ for matrices. The
obvious candidate definition is
\begin{mathpar}
(\Sigma_{\alpha}\fprmatrix{P_{\alpha}}{x_{\alpha}}{Q_{\alpha}})^{\dagger}
= \Sigma_{\alpha}\fprmatrix{(Q_{\alpha}^{\underline{\perp}})^{*}}{\overline{x}_{\alpha}}{P_{\alpha}^{\underline{\perp}}} 
\end{mathpar}

But, $(Q_{\alpha}^{\underline{\perp}})^{*}$ requires a name along
which to communicate the process to achieve the context application.

\subsubsection{Basis for a basis}
If processes label states and ``addition'' of states (a.k.a. vector
addition) is interpreted as parallel composition, what corresponds to
notions of linear independence and basis? Here, we recall that Yoshida
has developed a set of \emph{combinators} for an asynchronous verison
of Milner's $\pi$-calculus. These are a finite set of processes such
any process can be expressed as parallel composition of these
combinators together with liberal uses of the new operator and
replication. We can simply give a translation of these into the
present calculus and have reasonable expectation that the property
carries over. That is, that the resultant set allows to express all
processes via parallel composition. Note, however, that there is no
new operator or replication in this calculus. As a result, we expect
that the corresponding set is actually infinite. That is, we expect
that the space is actually infinite dimensional.

\begin{remark}
  The attentive reader may be a bit concerned. Certainly, the
  collection $S$, $K$ and $I$ is a finite set of
  combinators. Shouldn't we expect to see a finite set of combinators
  for an effectively equivalent system? i am very sympathetic to this
  critique and feel it warrants full attention. On the other hand, i
  also have in mind the following analogy. The natural numbers, as a
  monoid under addition, has exactly $1$ generator, while the natural
  numbers, as a monoid under multiplication, has countably many
  generators (the primes). We observe that the application of the
  lambda calculus is much less resource sensitive than the parallel
  composition of the $\pi$-calculus. Could it be the case that we have
  an analogy of the form
  
  \begin{mathpar}
    m + n : MN :: m*n : M|N
  \end{mathpar}

  giving a similar blow up in the set of ``primes''?  This is such a
  wonderful thought that, even if it's not true, i think it's worth
  writing down.
\end{remark}
 

\documentclass[12pt]{llncs}
%\documentclass{jktr}

\usepackage[pdftex]{hyperref}                   
\usepackage {listings}
\usepackage {mathpartir}
\usepackage{bcprules}
%\usepackage{listings}
                       
\usepackage{graphicx} 
%\usepackage[margins=2.5cm,nohead,nofoot]{geometry}
%\usepackage{geometry}
\usepackage{amsfonts}
\usepackage{amstext}
\usepackage{latexsym}
\usepackage{amssymb}
\usepackage{color}


%\include{myPreamble}
\include{qm2pi.local} 

%\ifpdf
%\usepackage[pdftex]{graphicx}
%\else
%\usepackage{graphicx}
%\fi

 % \ifpdf
%  \usepackage{pdfsync}
%  \if


%\title{Brief Article}
%\author{David F. Snyder}
%\author{L.G. Meredith}

%\address{Dept. of Math., Texas State University--San Marcos, San Marcos, TX 78666}
       
\pagestyle{empty}


\begin{document}

\lstset{language=[Objective]Caml,frame=shadowbox}

\input{qm2pi.front}

% section front matter (end)

\input{qm2pi.intro} 
 
% section introduction (end)

% \input{qm2pi.knotations} 

% section notation (end)

\input{qm2pi.process.calculi} 

% section concurrent_process_calculi_and_spatial_logics_ (end)
    
%\input{qm2pi.knots2pi} 

%\input{qm2pi.trefoil} 

%\input{qm2pi.mainthm} 

% subsection basic_interpretation (end)

%\input{qm2pi.rho.presentation} 
\subsection{The syntax and semantics of the notation system}\label{sub:the_syntax_and_semantics_of_the_notation_system} % (fold)

We now summarize a technical presentation of the calculus that
embodies our theory of dynamics. The typical presentation of such a
calculus follows the style of giving generators and relations on
them. The grammar, below, describing term constructors, freely
generates the set of processes, $\Proc$. This set is then quotiented
by a relation known as structural congruence and it is over this set
that the notion of dynamics is expressed. This presentation is
essentially that of \cite{MeredithR05} with the addition of
polyadicity and summation. For readability we have relegated some of
the technical subtleties to an appendix.

\subsubsection{Process grammar}\label{subsub:process_grammar}

\begin{mathpar}
  \inferrule* [lab=synchronization] {} {{M} \bc \pzero \;|\; x?F \;|\; x!C }
  \and
  \inferrule* [lab=abstraction] {} {{F} \bc (x)P}
  \and
  \inferrule* [lab=concretion] {} {{C} \bc \langle Q \rangle}
  \and
  \inferrule* [lab=process] {} {{P,Q} \bc M \;| \;P|Q \;|\; @{x}}
  \and
  \inferrule* [lab=name] {} {{x} \bc \quotep{P}}
\end{mathpar} 

Note that $\vec{x}$ (resp. $\vec{P}$) denotes a vector of names
(resp. processes) of length $|\vec{x}|$ (resp. $|\vec{P}|$). We adopt
the following useful abbreviations.

\begin{mathpar}
   x?(\vec{y}).P := x.(\vec{y})P \and  x\clift{\vec{P}} := x.\clift{\vec{P}}
   \and x!(y) := \lift{x}{\dropn{y}}
   \and \Pi_{i=0}^{n-1}P_i := P_0 | \ldots | P_{n-1}
\end{mathpar}

\subsubsection{Structural congruence}

\paragraph{Free and bound names and alpha-equivalence.} At the
core of structural equivalence is alpha-equivalence which identifies
process that are the same up to a change of variable. Formally, we
recognize the distinction between free and bound names. The free names
of a process, $\freenames{P}$, may be calculated recursively as
follows:

\begin{mathpar}
\freenames{\pzero} := \emptyset
  \and \\
  \freenames{x?(y).P} := \{ x \} \cup (\freenames{P} \setminus \{ y \})
  \and 
  \freenames{x!\langle P \rangle} := \{ x \} \cup \{ P \} 
  \and \\
  \freenames{P|Q} := \freenames{P} \cup \freenames{Q}
  \and \\
  \freenames{@{x}} := \{ x \}
\end{mathpar}

$\pi$
$\quotep{\pi}$

$\freenames{-} : \pi \to \mathcal{P}(\quotep{\pi})$

\begin{eqnarray*}
  \freenames{\pzero} & := & \emptyset \\
  \freenames{x?(y).P} & := & \{ x \} \cup (\freenames{P} \setminus \{ y \}) \\
  \freenames{x!\langle P \rangle} & := & \{ x \} \cup \{ P \} \\
  \freenames{P|Q} & := & \freenames{P} \cup \freenames{Q} \\
  \freenames{\dropn{x}} & := & \{ x \}
\end{eqnarray*}

The bound names of a process, $\boundnames{P}$, are those names occurring in $P$
that are not free. For example, in $x?(y).0$, the name $x$ is free, while $y$ is bound.

\begin{mathpar}
  \inferrule* [lab=monoidal-laws] {} { P|Q \equiv Q|P \and P|0 \equiv P \and P|(Q|R) \equiv (P|Q)|R }
\end{mathpar}

\begin{mathpar}
  \inferrule* [lab=alpha-equivalence] {} { (x)P \equiv (y)P\{y/x\} \and y \not\in \freenames{P} }
\end{mathpar}

\begin{definition}
Then two processes, $P,Q$, are alpha-equivalent if $P = Q\{\vec{y}/\vec{x}\}$ for
some $\vec{x} \in \boundnames{Q},\vec{y} \in \boundnames{P}$, where $Q\{\vec{y}/\vec{x}\}$
denotes the capture-avoiding substitution of $\vec{y}$ for $\vec{x}$ in $Q$.
\end{definition}

\begin{definition}
  The {\em structural congruence} \cite{SangiorgiWalker} , $\equiv$,
  between processes is the least congruence containing
  alpha-equivalence, satisfying the abelian monoid laws
  (associativity, commutativity and $\pzero$ as identity) for parallel
  composition $|$ and for summation $+$.
\end{definition}

\subsection{Name equivalence}

We take name equivalence, written $\nameeq$, to be the smallest
equivalence relation generated by the following rules.

\begin{mathpar}
\inferrule*[lab=Quote-drop]
{ }
{ \quotep{@{x}} \nameeq x }

\inferrule*[lab=Struct-equiv]
{ P \scong Q }
{ \quotep{P} \nameeq \quotep{Q} }
\end{mathpar}

The astute reader will have noticed that the mutual recursion of names
and processes imposes a mutual recursion on alpha-equivalence and
structural equivalence via name-equivalence. Fortunately, all of this
works out pleasantly and we may calculate in the natural way, free of
concern. The reader interested in the details is referred to the
appendix \ref{appendix:rho_details}.

\subsection{Substitution}

We use $\Proc$ for the set of processes, $\QProc$ for the set of
names, and $\id{\{}\vec{y} / \vec{x} \id{\}}$ to denote partial maps,
$s : \QProc \rightarrow \QProc$. A map, $s$ lifts, uniquely, to a map
on process terms, $\widehat{s} : \Proc \rightarrow \Proc$ by the
following equations.

\begin{mathpar}
  (0) \psubstp{Q}{P} := 0 \\
  (R \juxtap S) \psubstp{Q}{P}
  :=    
  (R)\psubstp{Q}{P} \juxtap (S) \psubstp{Q}{P} \\
  (x?(y).R) \psubstp{Q}{P}    
  :=    
  (x)\substp{Q}{P} (z)\concat( (R \psubstn{z}{y}) \psubstp{Q}{P} ) \\
  (\lift{x}{R}) \psubstp{Q}{P}  
  :=
  \lift{(x)\substp{Q}{P}}{ R \psubstp{Q}{P} } \\
%   (\dropn{x})  \psubstp{Q}{P}       
%   := 
%   \left\{ 
%     \begin{array}{ccc} 
%       \dropn{\quotep{Q}} & & x \nameeq \quotep{P} \\
%       \dropn{x} & & otherwise \\
%     \end{array}
%   \right. 
  (\dropn{x})  \psubstp{Q}{P}       
  := 
  \left\{ 
    \begin{array}{ccc} 
      Q & & x \nameeq \quotep{P} \\
      \dropn{x} & & otherwise \\
    \end{array}
  \right.
\end{mathpar}
 

where

\begin{eqnarray}
  (x)\id{\{} \lpquote Q \rpquote / \lpquote P \rpquote \id{\}}            = 
  \left\{ 
    \begin{array}{ccc}
      \lpquote Q \rpquote & & x \nameeq \lpquote P \rpquote \\
      x & & otherwise \\
    \end{array}
  \right. \nonumber
\end{eqnarray}

and $z$ is chosen distinct from $\quotep{P}$, $\quotep{Q}$, the free
names in $Q$, and all the names in $R$. Our $\alpha$-equivalence will
be built in the standard way from this substitution.

\begin{remark}\label{rem:no_self_referential_names}
  One consequence of these definitions is that $\forall P. \quotep{P}
  \not\in \freenames{P}$.
\end{remark}

\subsection{ Dynamic quote: an example }

Anticipating something of what's to come, consider applying the
substitution, $\widehat{\id{\{}u / z \id{\}}}$, to the following pair
of processes, $\lift{w}{y!(z)}$ and $w[ \lpquote y!(z) \rpquote ]$.

\begin{eqnarray}
	\lift{w}{y!(z)}\widehat{\id{\{}u / z \id{\}}}
		& = &
		\lift{w}{y!(u)} \nonumber\\
	w[ \lpquote y!(z) \rpquote ] \widehat{ \id{\{}u / z \id{\}} }
		& = &
		w[ \lpquote y!(z) \rpquote ] \nonumber
\end{eqnarray}

Because the body of the process between quotes is impervious to
substitution, we get radically different answers. In fact, by
examining the first process in an input context,
e.g. $x?(z).\lift{w}{y!(z)}$, we see that the process under the lift
operator may be shaped by prefixed inputs binding a name inside it. In
this sense, the lift operator will be seen as a way to dynamically
construct processes before reifying them as names.

Finally equipped with these standard features we can present the
dynamics of the calculus.

\subsubsection{Operational semantics} 

Finally, we introduce the computational dynamics. What marks these
algebras as distinct from other more traditionally studied algebraic
structures, e.g. vector spaces or polynomial rings, is the manner in
which dynamics is captured. In traditional structures, dynamics is typically
expressed through morphisms between such structures, as in linear maps
between vector spaces or morphisms between rings. In algebras
associated with the semantics of computation, the dynamics is
expressed as part of the algebraic structure itself, through a
reduction reduction relation typically denoted by $\red$. Below, we
give a recursive presentation of this relation for the calculus used
in the encoding.

$\red \subseteq \pi \times \pi$
$\red : \pi \to \mathcal{P}(\pi)$

\begin{mathpar}
  \inferrule* [lab=Comm] { \textsf{match}( x_{src}, x_{trgt} ) } { x_{trgt}?(y)P \; | \; x_{src}!\langle {Q} \rangle \red P\{\quotep{Q}/y}\} }
  \and \\
  \inferrule* [lab=Par] {{P} \red {P}'} {{{P} | {Q}} \red {{P}' | {Q}}}
  \and
  \inferrule* [lab=Equiv]{{{P} \scong {P}'} \andalso {{P}' \red {Q}'} \andalso {{Q}' \scong {Q}}}{{P} \red {Q}}
\end{mathpar}

\begin{eqnarray*}
  match_{\equiv} (\quotep{P},\quotep{Q}) & := & P \equiv Q \\
  match_{\dagger}(\quotep{P},\quotep{Q}) & := & \forall R. P|Q \red^{*} R => R \red^{*} 0 \\
  match_{K}(\quotep{P},\quotep{Q}) & := & K \mbox{ for some context } K
\end{eqnarray*}

$u?(x)P | u!\langle Q \rangle \red P\{\quotep{Q}/x\}$

%We write $\wred$ for $\red^*$, and $P\red$ if $\exists Q $ such that $ P \red Q$.
We write $P\red$ if $\exists Q $ such that $ P \red Q$ and $P\not\red$, otherwise.

\section{Replication}

As mentioned before, it is known that replication (and hence
recursion) can be implemented in a higher-order process algebra
\cite{SangiorgiWalker}. As our first example of calculation with the
machinery thus far presented we give the construction explicitly in
the {\rhoc}.

\begin{eqnarray}
	D_{x} & := & \prefix{x}{y}{(\binpar{\outputp{x}{y}}{@{y}})} \nonumber\\
	\bangp_{x}{P} & := & \binpar{{x}!\langle{\binpar{D_{x}}{P}}\rangle}{D_{x}} \nonumber
\end{eqnarray}

\begin{eqnarray}
	\bangp_{x}{P} & & \nonumber\\
	=
	& {x}!\langle{(\prefix{x}{y}{(\outputp{x}{y} | @{y})) | P}}\rangle 
	      | \prefix{x}{y}{(\outputp{x}{y} | @{y})} & \nonumber\\
	\red
	& (\outputp{x}{y} | @{y})\substn{\quotep{(\prefix{x}{y}{(@{y} | \outputp{x}{y})) | P}}}{y} & \nonumber\\
	=
	& \outputp{x}{\quotep{(\prefix{x}{y}{(\outputp{x}{y} | @{y})) | P}}}
	  | {(\prefix{x}{y}{(\outputp{x}{y} | @{y})) | P}} & \nonumber\\
	\red
	& \ldots & \nonumber\\
	\red^*
	& P | P | \ldots & \nonumber
\end{eqnarray}

Of course, this encoding, as an implementation, runs away, unfolding
$\bangp{P}$ eagerly. A lazier and more implementable replication
operator, restricted to input-guarded processes, may be obtained as follows.

\begin{eqnarray}
\bangp{\prefix{u}{v}{P}} 
	:= 
	\binpar{\lift{x}{\prefix{u}{v}{(\binpar{D(x)}{P})}}}{D(x)} \nonumber
\end{eqnarray}

\begin{remark}
  Note that the lazier definition still does not deal with summation
  or mixed summation (i.e. sums over input and output). The reader is
  invited to construct definitions of replication that deal with these
  features. 

  Further, the definitions are parameterized in a name, $x$. Can you,
  gentle reader, make a definition that eliminates this parameter and
  guarantees no accidental interaction between the replication
  machinery and the process being replicated -- i.e. no accidental
  sharing of names used by the process to get its work done and the
  name(s) used by the replication to effect copying. This latter
  revision of the definition of replication is crucial to obtaining
  the expected identity $!!P \sim !P$.
\end{remark}

\begin{remark}\label{rem:paradoxical_combinator}
  The reader familiar with the lambda calculus will have noticed the
  similarity between $D$ and the paradoxical combinator.

  [Ed. note: the existence of this seems to suggest we have to be more
  restrictive on the set of processes and names we admit if we are to
  support no-cloning.]
\end{remark}

\subsubsection{Bisimulation}

The computational dynamics gives rise to another kind of equivalence,
the equivalence of computational behavior. As previously mentioned
this is typically captured \emph{via} some form of bisimulation.

% The notion we use in this paper is weak barbed bisimulation
% \cite{milner91polyadicpi}.

The notion we use in this paper is derived from weak barbed
bisimulation \cite{milner91polyadicpi}. 

\begin{definition}
An \emph{observation relation}, $\downarrow_{\mathcal N}$, over a set
of names, $\mathcal N$, is the smallest relation satisfying the rules
below.

\infrule[Out-barb]{y \in {\mathcal N}, \; x \nameeq y}
		  {\outputp{x}{v} \downarrow_{\mathcal N} x}
\infrule[Par-barb]{\mbox{$P\downarrow_{\mathcal N} x$ or $Q\downarrow_{\mathcal N} x$}}
		  {\binpar{P}{Q} \downarrow_{\mathcal N} x}

We write $P \Downarrow_{\mathcal N} x$ if there is $Q$ such that 
$P \wred Q$ and $Q \downarrow_{\mathcal N} x$.
\end{definition}

\begin{definition}
%\label{def.bbisim}
An  ${\mathcal N}$-\emph{barbed bisimulation} over a set of names, ${\mathcal N}$, is a symmetric binary relation 
${\mathcal S}_{\mathcal N}$ between agents such that $P\rel{S}_{\mathcal N}Q$ implies:
\begin{enumerate}
\item If $P \red P'$ then $Q \wred Q'$ and $P'\rel{S}_{\mathcal N} Q'$.
\item If $P\downarrow_{\mathcal N} x$, then $Q\Downarrow_{\mathcal N} x$.
\end{enumerate}
$P$ is ${\mathcal N}$-barbed bisimilar to $Q$, written
$P \wbbisim_{\mathcal N} Q$, if $P \rel{S}_{\mathcal N} Q$ for some ${\mathcal N}$-barbed bisimulation ${\mathcal S}_{\mathcal N}$.
\end{definition}

$\mathcal{R} \subseteq \pi \times \pi$

$P \mathcal{R} Q => \forall P'. P \red P' \Rightarrow \exists Q'. Q \red Q', P' \mathcal{R} Q'$

$P \vdash x \Rightarrow Q \vdash x$

\begin{mathpar}
  \inferrule*[lab=Out-barb]{x \nameeq y}{{y}!\langle{Q}\rangle \vdash x}
  \and
  \inferrule*[lab=Par-barb]{\mbox{$P\vdash x$ or $Q\vdash x$}}{\binpar{P}{Q} \vdash x}
\end{mathpar}

\subsubsection{Contexts}

One of the principle advantages of computational calculi like the
$\pi$-calculus is a well-defined notion of context,
contextual-equivalence and a correlation between
contextual-equivalence and notions of bisimulation. The notion of
context allows the decomposition of a process into (sub-)process and
its syntactic environment, its context. Thus, a context may be
thought of as a process with a ``hole'' (written $\Box$) in it. The
application of a context $M$ to a process $P$, written $M[P]$, is
tantamount to filling the hole in $M$ with $P$. In this paper we do
not need the full weight of this theory, but do make use of the notion
of context in the proof the main theorem. 

\begin{mathpar}
  \inferrule* [lab=summation] {} {{M_{M},M_{N}} \bc \Box \;|\; x.M_{A} \;|\; M_{M}+M_{N}}
  \and
  \inferrule* [lab=agent] {} {{M_{A}} \bc (\vec{x})M_{P} \;| \; \clift{P_0,\ldots,M_{P},\ldots,P_N}}
  \and \\
  \inferrule* [lab=process] {} {{M_{P}} \bc M_{N} \;| \;P|M_{P} }
\end{mathpar} 

\begin{mathpar}
  \inferrule* [lab=sychronization] {} {M_{N} \bc \Box \;|\; x?M_{F} \;|\; x!M_{C}}
  \and
  \inferrule* [lab=abstraction] {} {{M_{F}} \bc (x)M_{P} }
  \and
  \inferrule* [lab=concretion] {} {{M_{C}} \bc \langle M_{P} \rangle }
  \and \\
  \inferrule* [lab=process] {} {{M_{P}} \bc M_{N} \;| \;P|M_{P} }
\end{mathpar}

\begin{definition}[contextual application] Given a context $M$, and
  process $P$, we define the \emph{contextual application}, $M[P] :=
  M\{P/\Box\}$. That is, the contextual application of M to P is the
  substitution of $P$ for $\Box$ in $M$.
\end{definition}

$\meaningof{-} : L \to \mathcal{P}(\pi)$

\begin{mathpar}
  \inferrule* [lab=collection] {} {\meaningof{true} = \pi, \and \meaningof{~E} = \pi \setminus \meaningof{E}, \and \meaningof{E_{1} \& E_{2}} = \meaningof{E_{1}} \cap \meaningof{E_{2}}}
\end{mathpar}

\begin{mathpar}
  \inferrule* [lab=structure] {} {\meaningof{0} = \{ P \in \pi | P \equiv 0 \}, \and \\ \meaningof{E_1 | E_2} = \{ P \in \pi | P \equiv P_{1} | P_{2}, P_{1} \in \meaningof{E_{1}}, P_{2} \in \meaningof{E_2}\} }
\end{mathpar}

\begin{mathpar}
 \inferrule* [lab=behavior] {} {\meaningof{\langle a?b \rangle E} = \{ P \in \pi | P \equiv Q | u?(y)P', \\ \and \\\\ \and \\ \;\;\; u \in \meaningof{a}, \forall z.P'\{z/y\} \in \meaningof{E\{z/b\}}\}, \and \\ \meaningof{a!E} = \{ P \in \pi | P \equiv Q | x!\langle P' \rangle, x \in \meaningof{a} P' \in \meaningof{E}\} }
\end{mathpar}

\begin{mathpar}
 \inferrule* [lab=nominal] {} {\meaningof{\quotep{E}} = \{ \quotep{P} \in \quotep{\pi} | P \in \meaningof{E} \}, \and \meaningof{\quotep{P}} = \{ \quotep{Q} \in \quotep{\pi} | P \equiv Q \} \and \\ \meaningof{@\quotep{E}} = \{ P \in \pi | P \equiv @x, x \in \meaningof{E} \}}
\end{mathpar}

\begin{eqnarray*}
  \\
  \meaningof{-} : TS \to ST
\end{eqnarray*}

\begin{eqnarray*}
  \\
  L : TS \to ST
\end{eqnarray*}

\begin{eqnarray*}
  \\
  P \models E \iff P \in \meaningof{E}
\end{eqnarray*}

\begin{eqnarray*}
  P \approx_{L} Q \iff \forall E \in L. P \models E \iff Q \models E
\end{eqnarray*}

\begin{eqnarray*}
  P \approx_{K} Q
\end{eqnarray*}

\begin{eqnarray*}
  P \approx Q
\end{eqnarray*}

$\approx_{K} = \approx = \approx_{L}$

\subsubsection{Contextual duality}

Note that contexts extend the quotation operation to a family of
operations from processes to names. Given a context, $M$, we can
define a \emph{nominal context}, $\quotep{M}$ by $\quotep{M}[P] :=
\quotep{M[P]}$. To foreshadow what is to come we observe that these
operations enjoy a duality with processes very much like the duality
between vectors and maps from vectors to scalars.

Further, because the calculus is essentially higher-order, we have a
correspondence between contexts and processes. More specifically,
given a name $x$ and a context $M$ we can construct $M^{*}_{x}$ such
that 

\begin{mathpar}
  M^{*}_{x} | \lift{x}{P} \red M[P]
\end{mathpar}

namely,

\begin{mathpar}
  M^{*}_{x} := x?(u).M[\dropn{u}]
\end{mathpar}

The dependence of $M^{*}_{x}$ on a name makes it an abstraction, 

\begin{mathpar}
  M^{*} := (x)x?(u).M[\dropn{u}]
\end{mathpar}

\subsection{Additional notation}

It will sometimes be convenient to denote the process a name
quotes. We already have the notation $x = \quotep{P}$, but it will be
convenient to introduce an alternate notation, $\procn{x}$, when we
want to emphasize the connection to the use of the name. Note that, by
virtue of name equivalence, $\quotep{\procn{x}} \nameeq x$; so, the
notation is consistent with previous definitions.

Further, because names have structure it is possible to effect
substitutions on the basis of that structure. This means we need to
upgrade our notation for substitutions, which we accomplish by
adapting comprehension notation. Thus,

\begin{mathpar}
  P\{ y / x : x \in S \}
\end{mathpar}

is interpreted to mean the process derived from P by replacing (in a
capture-avoiding manner) each occurrence of $x$ in $S$ by $y$. For example,

\begin{mathpar}
  P\{ \quotep{\procn{x}|\procn{x}} / x : x \in \freenames{P} \}
\end{mathpar}

will replace each (occurrence) of a free name $x$ in $P$ by
$\quotep{\procn{x}|\procn{x}}$.

Also, we will avail ourselves of the notation $x^{L}$ and $x^{R}$ to
denote injections of a name into disjoint copies of the name
space. There are numerous ways to accomplish this. One example can be
found in \cite{MeredithR05}. This notation overloads to vectors of
names: $\vec{x}^{\pi} := (x_{i}^{\pi} \; : \; 0 \leq i < |\vec{x}| )$ where $\pi \in \{L,R\}$.

We also use $P^{\Box} := P|\Box$.

In \cite{MeredithR05} an interpretation of the new operator is
given. It turns out that there are several possible interpretations
all enjoying the requisite algebraic properties of the operator (see
\cite{milner91polyadicpi}). We will therefore make liberal use of
$(\nu\; \vec{x})P$.

% subsection the_syntax_and_semantics_of_the_notation_system (end)   

\input{qm2pi.qmops} 

\input{qm2pi.sterngerlach} 

\input{qm2pi.metric} 

% section concurrent_process_calculi (end)

%\input{qm2pi.proofsketch}

% section proof sketch (end)

%\input{qm2pi.slviaknots} 

% section spatial logic via knots (end)

\input{qm2pi.conclusion}

% section conclusion (end)

%\input{qm2pi.dtcodes} 

% section wiring algorithm (end)

\input{qm2pi.ack} 

% section acknowledgments (end)

\newpage


\bibliographystyle{plain}   
\bibliography{../../biblios/main.bib}

\input{qm2pi.rhodetails}

\end{document}

 

\documentclass[12pt]{llncs}
%\documentclass{jktr}

\usepackage[pdftex]{hyperref}                   
\usepackage {listings}
\usepackage {mathpartir}
\usepackage{bcprules}
%\usepackage{listings}
                       
\usepackage{graphicx} 
%\usepackage[margins=2.5cm,nohead,nofoot]{geometry}
%\usepackage{geometry}
\usepackage{amsfonts}
\usepackage{amstext}
\usepackage{latexsym}
\usepackage{amssymb}
\usepackage{color}


%\include{myPreamble}
\include{qm2pi.local} 

%\ifpdf
%\usepackage[pdftex]{graphicx}
%\else
%\usepackage{graphicx}
%\fi

 % \ifpdf
%  \usepackage{pdfsync}
%  \if


%\title{Brief Article}
%\author{David F. Snyder}
%\author{L.G. Meredith}

%\address{Dept. of Math., Texas State University--San Marcos, San Marcos, TX 78666}
       
\pagestyle{empty}


\begin{document}

\lstset{language=[Objective]Caml,frame=shadowbox}

\input{qm2pi.front}

% section front matter (end)

\input{qm2pi.intro} 
 
% section introduction (end)

% \input{qm2pi.knotations} 

% section notation (end)

\input{qm2pi.process.calculi} 

% section concurrent_process_calculi_and_spatial_logics_ (end)
    
%\input{qm2pi.knots2pi} 

%\input{qm2pi.trefoil} 

%\input{qm2pi.mainthm} 

% subsection basic_interpretation (end)

%\input{qm2pi.rho.presentation} 
\subsection{The syntax and semantics of the notation system}\label{sub:the_syntax_and_semantics_of_the_notation_system} % (fold)

We now summarize a technical presentation of the calculus that
embodies our theory of dynamics. The typical presentation of such a
calculus follows the style of giving generators and relations on
them. The grammar, below, describing term constructors, freely
generates the set of processes, $\Proc$. This set is then quotiented
by a relation known as structural congruence and it is over this set
that the notion of dynamics is expressed. This presentation is
essentially that of \cite{MeredithR05} with the addition of
polyadicity and summation. For readability we have relegated some of
the technical subtleties to an appendix.

\subsubsection{Process grammar}\label{subsub:process_grammar}

\begin{mathpar}
  \inferrule* [lab=synchronization] {} {{M} \bc \pzero \;|\; x?F \;|\; x!C }
  \and
  \inferrule* [lab=abstraction] {} {{F} \bc (x)P}
  \and
  \inferrule* [lab=concretion] {} {{C} \bc \langle Q \rangle}
  \and
  \inferrule* [lab=process] {} {{P,Q} \bc M \;| \;P|Q \;|\; @{x}}
  \and
  \inferrule* [lab=name] {} {{x} \bc \quotep{P}}
\end{mathpar} 

Note that $\vec{x}$ (resp. $\vec{P}$) denotes a vector of names
(resp. processes) of length $|\vec{x}|$ (resp. $|\vec{P}|$). We adopt
the following useful abbreviations.

\begin{mathpar}
   x?(\vec{y}).P := x.(\vec{y})P \and  x\clift{\vec{P}} := x.\clift{\vec{P}}
   \and x!(y) := \lift{x}{\dropn{y}}
   \and \Pi_{i=0}^{n-1}P_i := P_0 | \ldots | P_{n-1}
\end{mathpar}

\subsubsection{Structural congruence}

\paragraph{Free and bound names and alpha-equivalence.} At the
core of structural equivalence is alpha-equivalence which identifies
process that are the same up to a change of variable. Formally, we
recognize the distinction between free and bound names. The free names
of a process, $\freenames{P}$, may be calculated recursively as
follows:

\begin{mathpar}
\freenames{\pzero} := \emptyset
  \and \\
  \freenames{x?(y).P} := \{ x \} \cup (\freenames{P} \setminus \{ y \})
  \and 
  \freenames{x!\langle P \rangle} := \{ x \} \cup \{ P \} 
  \and \\
  \freenames{P|Q} := \freenames{P} \cup \freenames{Q}
  \and \\
  \freenames{@{x}} := \{ x \}
\end{mathpar}

$\pi$
$\quotep{\pi}$

$\freenames{-} : \pi \to \mathcal{P}(\quotep{\pi})$

\begin{eqnarray*}
  \freenames{\pzero} & := & \emptyset \\
  \freenames{x?(y).P} & := & \{ x \} \cup (\freenames{P} \setminus \{ y \}) \\
  \freenames{x!\langle P \rangle} & := & \{ x \} \cup \{ P \} \\
  \freenames{P|Q} & := & \freenames{P} \cup \freenames{Q} \\
  \freenames{\dropn{x}} & := & \{ x \}
\end{eqnarray*}

The bound names of a process, $\boundnames{P}$, are those names occurring in $P$
that are not free. For example, in $x?(y).0$, the name $x$ is free, while $y$ is bound.

\begin{mathpar}
  \inferrule* [lab=monoidal-laws] {} { P|Q \equiv Q|P \and P|0 \equiv P \and P|(Q|R) \equiv (P|Q)|R }
\end{mathpar}

\begin{mathpar}
  \inferrule* [lab=alpha-equivalence] {} { (x)P \equiv (y)P\{y/x\} \and y \not\in \freenames{P} }
\end{mathpar}

\begin{definition}
Then two processes, $P,Q$, are alpha-equivalent if $P = Q\{\vec{y}/\vec{x}\}$ for
some $\vec{x} \in \boundnames{Q},\vec{y} \in \boundnames{P}$, where $Q\{\vec{y}/\vec{x}\}$
denotes the capture-avoiding substitution of $\vec{y}$ for $\vec{x}$ in $Q$.
\end{definition}

\begin{definition}
  The {\em structural congruence} \cite{SangiorgiWalker} , $\equiv$,
  between processes is the least congruence containing
  alpha-equivalence, satisfying the abelian monoid laws
  (associativity, commutativity and $\pzero$ as identity) for parallel
  composition $|$ and for summation $+$.
\end{definition}

\subsection{Name equivalence}

We take name equivalence, written $\nameeq$, to be the smallest
equivalence relation generated by the following rules.

\begin{mathpar}
\inferrule*[lab=Quote-drop]
{ }
{ \quotep{@{x}} \nameeq x }

\inferrule*[lab=Struct-equiv]
{ P \scong Q }
{ \quotep{P} \nameeq \quotep{Q} }
\end{mathpar}

The astute reader will have noticed that the mutual recursion of names
and processes imposes a mutual recursion on alpha-equivalence and
structural equivalence via name-equivalence. Fortunately, all of this
works out pleasantly and we may calculate in the natural way, free of
concern. The reader interested in the details is referred to the
appendix \ref{appendix:rho_details}.

\subsection{Substitution}

We use $\Proc$ for the set of processes, $\QProc$ for the set of
names, and $\id{\{}\vec{y} / \vec{x} \id{\}}$ to denote partial maps,
$s : \QProc \rightarrow \QProc$. A map, $s$ lifts, uniquely, to a map
on process terms, $\widehat{s} : \Proc \rightarrow \Proc$ by the
following equations.

\begin{mathpar}
  (0) \psubstp{Q}{P} := 0 \\
  (R \juxtap S) \psubstp{Q}{P}
  :=    
  (R)\psubstp{Q}{P} \juxtap (S) \psubstp{Q}{P} \\
  (x?(y).R) \psubstp{Q}{P}    
  :=    
  (x)\substp{Q}{P} (z)\concat( (R \psubstn{z}{y}) \psubstp{Q}{P} ) \\
  (\lift{x}{R}) \psubstp{Q}{P}  
  :=
  \lift{(x)\substp{Q}{P}}{ R \psubstp{Q}{P} } \\
%   (\dropn{x})  \psubstp{Q}{P}       
%   := 
%   \left\{ 
%     \begin{array}{ccc} 
%       \dropn{\quotep{Q}} & & x \nameeq \quotep{P} \\
%       \dropn{x} & & otherwise \\
%     \end{array}
%   \right. 
  (\dropn{x})  \psubstp{Q}{P}       
  := 
  \left\{ 
    \begin{array}{ccc} 
      Q & & x \nameeq \quotep{P} \\
      \dropn{x} & & otherwise \\
    \end{array}
  \right.
\end{mathpar}
 

where

\begin{eqnarray}
  (x)\id{\{} \lpquote Q \rpquote / \lpquote P \rpquote \id{\}}            = 
  \left\{ 
    \begin{array}{ccc}
      \lpquote Q \rpquote & & x \nameeq \lpquote P \rpquote \\
      x & & otherwise \\
    \end{array}
  \right. \nonumber
\end{eqnarray}

and $z$ is chosen distinct from $\quotep{P}$, $\quotep{Q}$, the free
names in $Q$, and all the names in $R$. Our $\alpha$-equivalence will
be built in the standard way from this substitution.

\begin{remark}\label{rem:no_self_referential_names}
  One consequence of these definitions is that $\forall P. \quotep{P}
  \not\in \freenames{P}$.
\end{remark}

\subsection{ Dynamic quote: an example }

Anticipating something of what's to come, consider applying the
substitution, $\widehat{\id{\{}u / z \id{\}}}$, to the following pair
of processes, $\lift{w}{y!(z)}$ and $w[ \lpquote y!(z) \rpquote ]$.

\begin{eqnarray}
	\lift{w}{y!(z)}\widehat{\id{\{}u / z \id{\}}}
		& = &
		\lift{w}{y!(u)} \nonumber\\
	w[ \lpquote y!(z) \rpquote ] \widehat{ \id{\{}u / z \id{\}} }
		& = &
		w[ \lpquote y!(z) \rpquote ] \nonumber
\end{eqnarray}

Because the body of the process between quotes is impervious to
substitution, we get radically different answers. In fact, by
examining the first process in an input context,
e.g. $x?(z).\lift{w}{y!(z)}$, we see that the process under the lift
operator may be shaped by prefixed inputs binding a name inside it. In
this sense, the lift operator will be seen as a way to dynamically
construct processes before reifying them as names.

Finally equipped with these standard features we can present the
dynamics of the calculus.

\subsubsection{Operational semantics} 

Finally, we introduce the computational dynamics. What marks these
algebras as distinct from other more traditionally studied algebraic
structures, e.g. vector spaces or polynomial rings, is the manner in
which dynamics is captured. In traditional structures, dynamics is typically
expressed through morphisms between such structures, as in linear maps
between vector spaces or morphisms between rings. In algebras
associated with the semantics of computation, the dynamics is
expressed as part of the algebraic structure itself, through a
reduction reduction relation typically denoted by $\red$. Below, we
give a recursive presentation of this relation for the calculus used
in the encoding.

$\red \subseteq \pi \times \pi$
$\red : \pi \to \mathcal{P}(\pi)$

\begin{mathpar}
  \inferrule* [lab=Comm] { \textsf{match}( x_{src}, x_{trgt} ) } { x_{trgt}?(y)P \; | \; x_{src}!\langle {Q} \rangle \red P\{\quotep{Q}/y}\} }
  \and \\
  \inferrule* [lab=Par] {{P} \red {P}'} {{{P} | {Q}} \red {{P}' | {Q}}}
  \and
  \inferrule* [lab=Equiv]{{{P} \scong {P}'} \andalso {{P}' \red {Q}'} \andalso {{Q}' \scong {Q}}}{{P} \red {Q}}
\end{mathpar}

\begin{eqnarray*}
  match_{\equiv} (\quotep{P},\quotep{Q}) & := & P \equiv Q \\
  match_{\dagger}(\quotep{P},\quotep{Q}) & := & \forall R. P|Q \red^{*} R => R \red^{*} 0 \\
  match_{K}(\quotep{P},\quotep{Q}) & := & K \mbox{ for some context } K
\end{eqnarray*}

$u?(x)P | u!\langle Q \rangle \red P\{\quotep{Q}/x\}$

%We write $\wred$ for $\red^*$, and $P\red$ if $\exists Q $ such that $ P \red Q$.
We write $P\red$ if $\exists Q $ such that $ P \red Q$ and $P\not\red$, otherwise.

\section{Replication}

As mentioned before, it is known that replication (and hence
recursion) can be implemented in a higher-order process algebra
\cite{SangiorgiWalker}. As our first example of calculation with the
machinery thus far presented we give the construction explicitly in
the {\rhoc}.

\begin{eqnarray}
	D_{x} & := & \prefix{x}{y}{(\binpar{\outputp{x}{y}}{@{y}})} \nonumber\\
	\bangp_{x}{P} & := & \binpar{{x}!\langle{\binpar{D_{x}}{P}}\rangle}{D_{x}} \nonumber
\end{eqnarray}

\begin{eqnarray}
	\bangp_{x}{P} & & \nonumber\\
	=
	& {x}!\langle{(\prefix{x}{y}{(\outputp{x}{y} | @{y})) | P}}\rangle 
	      | \prefix{x}{y}{(\outputp{x}{y} | @{y})} & \nonumber\\
	\red
	& (\outputp{x}{y} | @{y})\substn{\quotep{(\prefix{x}{y}{(@{y} | \outputp{x}{y})) | P}}}{y} & \nonumber\\
	=
	& \outputp{x}{\quotep{(\prefix{x}{y}{(\outputp{x}{y} | @{y})) | P}}}
	  | {(\prefix{x}{y}{(\outputp{x}{y} | @{y})) | P}} & \nonumber\\
	\red
	& \ldots & \nonumber\\
	\red^*
	& P | P | \ldots & \nonumber
\end{eqnarray}

Of course, this encoding, as an implementation, runs away, unfolding
$\bangp{P}$ eagerly. A lazier and more implementable replication
operator, restricted to input-guarded processes, may be obtained as follows.

\begin{eqnarray}
\bangp{\prefix{u}{v}{P}} 
	:= 
	\binpar{\lift{x}{\prefix{u}{v}{(\binpar{D(x)}{P})}}}{D(x)} \nonumber
\end{eqnarray}

\begin{remark}
  Note that the lazier definition still does not deal with summation
  or mixed summation (i.e. sums over input and output). The reader is
  invited to construct definitions of replication that deal with these
  features. 

  Further, the definitions are parameterized in a name, $x$. Can you,
  gentle reader, make a definition that eliminates this parameter and
  guarantees no accidental interaction between the replication
  machinery and the process being replicated -- i.e. no accidental
  sharing of names used by the process to get its work done and the
  name(s) used by the replication to effect copying. This latter
  revision of the definition of replication is crucial to obtaining
  the expected identity $!!P \sim !P$.
\end{remark}

\begin{remark}\label{rem:paradoxical_combinator}
  The reader familiar with the lambda calculus will have noticed the
  similarity between $D$ and the paradoxical combinator.

  [Ed. note: the existence of this seems to suggest we have to be more
  restrictive on the set of processes and names we admit if we are to
  support no-cloning.]
\end{remark}

\subsubsection{Bisimulation}

The computational dynamics gives rise to another kind of equivalence,
the equivalence of computational behavior. As previously mentioned
this is typically captured \emph{via} some form of bisimulation.

% The notion we use in this paper is weak barbed bisimulation
% \cite{milner91polyadicpi}.

The notion we use in this paper is derived from weak barbed
bisimulation \cite{milner91polyadicpi}. 

\begin{definition}
An \emph{observation relation}, $\downarrow_{\mathcal N}$, over a set
of names, $\mathcal N$, is the smallest relation satisfying the rules
below.

\infrule[Out-barb]{y \in {\mathcal N}, \; x \nameeq y}
		  {\outputp{x}{v} \downarrow_{\mathcal N} x}
\infrule[Par-barb]{\mbox{$P\downarrow_{\mathcal N} x$ or $Q\downarrow_{\mathcal N} x$}}
		  {\binpar{P}{Q} \downarrow_{\mathcal N} x}

We write $P \Downarrow_{\mathcal N} x$ if there is $Q$ such that 
$P \wred Q$ and $Q \downarrow_{\mathcal N} x$.
\end{definition}

\begin{definition}
%\label{def.bbisim}
An  ${\mathcal N}$-\emph{barbed bisimulation} over a set of names, ${\mathcal N}$, is a symmetric binary relation 
${\mathcal S}_{\mathcal N}$ between agents such that $P\rel{S}_{\mathcal N}Q$ implies:
\begin{enumerate}
\item If $P \red P'$ then $Q \wred Q'$ and $P'\rel{S}_{\mathcal N} Q'$.
\item If $P\downarrow_{\mathcal N} x$, then $Q\Downarrow_{\mathcal N} x$.
\end{enumerate}
$P$ is ${\mathcal N}$-barbed bisimilar to $Q$, written
$P \wbbisim_{\mathcal N} Q$, if $P \rel{S}_{\mathcal N} Q$ for some ${\mathcal N}$-barbed bisimulation ${\mathcal S}_{\mathcal N}$.
\end{definition}

$\mathcal{R} \subseteq \pi \times \pi$

$P \mathcal{R} Q => \forall P'. P \red P' \Rightarrow \exists Q'. Q \red Q', P' \mathcal{R} Q'$

$P \vdash x \Rightarrow Q \vdash x$

\begin{mathpar}
  \inferrule*[lab=Out-barb]{x \nameeq y}{{y}!\langle{Q}\rangle \vdash x}
  \and
  \inferrule*[lab=Par-barb]{\mbox{$P\vdash x$ or $Q\vdash x$}}{\binpar{P}{Q} \vdash x}
\end{mathpar}

\subsubsection{Contexts}

One of the principle advantages of computational calculi like the
$\pi$-calculus is a well-defined notion of context,
contextual-equivalence and a correlation between
contextual-equivalence and notions of bisimulation. The notion of
context allows the decomposition of a process into (sub-)process and
its syntactic environment, its context. Thus, a context may be
thought of as a process with a ``hole'' (written $\Box$) in it. The
application of a context $M$ to a process $P$, written $M[P]$, is
tantamount to filling the hole in $M$ with $P$. In this paper we do
not need the full weight of this theory, but do make use of the notion
of context in the proof the main theorem. 

\begin{mathpar}
  \inferrule* [lab=summation] {} {{M_{M},M_{N}} \bc \Box \;|\; x.M_{A} \;|\; M_{M}+M_{N}}
  \and
  \inferrule* [lab=agent] {} {{M_{A}} \bc (\vec{x})M_{P} \;| \; \clift{P_0,\ldots,M_{P},\ldots,P_N}}
  \and \\
  \inferrule* [lab=process] {} {{M_{P}} \bc M_{N} \;| \;P|M_{P} }
\end{mathpar} 

\begin{mathpar}
  \inferrule* [lab=sychronization] {} {M_{N} \bc \Box \;|\; x?M_{F} \;|\; x!M_{C}}
  \and
  \inferrule* [lab=abstraction] {} {{M_{F}} \bc (x)M_{P} }
  \and
  \inferrule* [lab=concretion] {} {{M_{C}} \bc \langle M_{P} \rangle }
  \and \\
  \inferrule* [lab=process] {} {{M_{P}} \bc M_{N} \;| \;P|M_{P} }
\end{mathpar}

\begin{definition}[contextual application] Given a context $M$, and
  process $P$, we define the \emph{contextual application}, $M[P] :=
  M\{P/\Box\}$. That is, the contextual application of M to P is the
  substitution of $P$ for $\Box$ in $M$.
\end{definition}

$\meaningof{-} : L \to \mathcal{P}(\pi)$

\begin{mathpar}
  \inferrule* [lab=collection] {} {\meaningof{true} = \pi, \and \meaningof{~E} = \pi \setminus \meaningof{E}, \and \meaningof{E_{1} \& E_{2}} = \meaningof{E_{1}} \cap \meaningof{E_{2}}}
\end{mathpar}

\begin{mathpar}
  \inferrule* [lab=structure] {} {\meaningof{0} = \{ P \in \pi | P \equiv 0 \}, \and \\ \meaningof{E_1 | E_2} = \{ P \in \pi | P \equiv P_{1} | P_{2}, P_{1} \in \meaningof{E_{1}}, P_{2} \in \meaningof{E_2}\} }
\end{mathpar}

\begin{mathpar}
 \inferrule* [lab=behavior] {} {\meaningof{\langle a?b \rangle E} = \{ P \in \pi | P \equiv Q | u?(y)P', \\ \and \\\\ \and \\ \;\;\; u \in \meaningof{a}, \forall z.P'\{z/y\} \in \meaningof{E\{z/b\}}\}, \and \\ \meaningof{a!E} = \{ P \in \pi | P \equiv Q | x!\langle P' \rangle, x \in \meaningof{a} P' \in \meaningof{E}\} }
\end{mathpar}

\begin{mathpar}
 \inferrule* [lab=nominal] {} {\meaningof{\quotep{E}} = \{ \quotep{P} \in \quotep{\pi} | P \in \meaningof{E} \}, \and \meaningof{\quotep{P}} = \{ \quotep{Q} \in \quotep{\pi} | P \equiv Q \} \and \\ \meaningof{@\quotep{E}} = \{ P \in \pi | P \equiv @x, x \in \meaningof{E} \}}
\end{mathpar}

\begin{eqnarray*}
  \\
  \meaningof{-} : TS \to ST
\end{eqnarray*}

\begin{eqnarray*}
  \\
  L : TS \to ST
\end{eqnarray*}

\begin{eqnarray*}
  \\
  P \models E \iff P \in \meaningof{E}
\end{eqnarray*}

\begin{eqnarray*}
  P \approx_{L} Q \iff \forall E \in L. P \models E \iff Q \models E
\end{eqnarray*}

\begin{eqnarray*}
  P \approx_{K} Q
\end{eqnarray*}

\begin{eqnarray*}
  P \approx Q
\end{eqnarray*}

$\approx_{K} = \approx = \approx_{L}$

\subsubsection{Contextual duality}

Note that contexts extend the quotation operation to a family of
operations from processes to names. Given a context, $M$, we can
define a \emph{nominal context}, $\quotep{M}$ by $\quotep{M}[P] :=
\quotep{M[P]}$. To foreshadow what is to come we observe that these
operations enjoy a duality with processes very much like the duality
between vectors and maps from vectors to scalars.

Further, because the calculus is essentially higher-order, we have a
correspondence between contexts and processes. More specifically,
given a name $x$ and a context $M$ we can construct $M^{*}_{x}$ such
that 

\begin{mathpar}
  M^{*}_{x} | \lift{x}{P} \red M[P]
\end{mathpar}

namely,

\begin{mathpar}
  M^{*}_{x} := x?(u).M[\dropn{u}]
\end{mathpar}

The dependence of $M^{*}_{x}$ on a name makes it an abstraction, 

\begin{mathpar}
  M^{*} := (x)x?(u).M[\dropn{u}]
\end{mathpar}

\subsection{Additional notation}

It will sometimes be convenient to denote the process a name
quotes. We already have the notation $x = \quotep{P}$, but it will be
convenient to introduce an alternate notation, $\procn{x}$, when we
want to emphasize the connection to the use of the name. Note that, by
virtue of name equivalence, $\quotep{\procn{x}} \nameeq x$; so, the
notation is consistent with previous definitions.

Further, because names have structure it is possible to effect
substitutions on the basis of that structure. This means we need to
upgrade our notation for substitutions, which we accomplish by
adapting comprehension notation. Thus,

\begin{mathpar}
  P\{ y / x : x \in S \}
\end{mathpar}

is interpreted to mean the process derived from P by replacing (in a
capture-avoiding manner) each occurrence of $x$ in $S$ by $y$. For example,

\begin{mathpar}
  P\{ \quotep{\procn{x}|\procn{x}} / x : x \in \freenames{P} \}
\end{mathpar}

will replace each (occurrence) of a free name $x$ in $P$ by
$\quotep{\procn{x}|\procn{x}}$.

Also, we will avail ourselves of the notation $x^{L}$ and $x^{R}$ to
denote injections of a name into disjoint copies of the name
space. There are numerous ways to accomplish this. One example can be
found in \cite{MeredithR05}. This notation overloads to vectors of
names: $\vec{x}^{\pi} := (x_{i}^{\pi} \; : \; 0 \leq i < |\vec{x}| )$ where $\pi \in \{L,R\}$.

We also use $P^{\Box} := P|\Box$.

In \cite{MeredithR05} an interpretation of the new operator is
given. It turns out that there are several possible interpretations
all enjoying the requisite algebraic properties of the operator (see
\cite{milner91polyadicpi}). We will therefore make liberal use of
$(\nu\; \vec{x})P$.

% subsection the_syntax_and_semantics_of_the_notation_system (end)   

\input{qm2pi.qmops} 

\input{qm2pi.sterngerlach} 

\input{qm2pi.metric} 

% section concurrent_process_calculi (end)

%\input{qm2pi.proofsketch}

% section proof sketch (end)

%\input{qm2pi.slviaknots} 

% section spatial logic via knots (end)

\input{qm2pi.conclusion}

% section conclusion (end)

%\input{qm2pi.dtcodes} 

% section wiring algorithm (end)

\input{qm2pi.ack} 

% section acknowledgments (end)

\newpage


\bibliographystyle{plain}   
\bibliography{../../biblios/main.bib}

\input{qm2pi.rhodetails}

\end{document}

 

% section concurrent_process_calculi (end)

%\documentclass[12pt]{llncs}
%\documentclass{jktr}

\usepackage[pdftex]{hyperref}                   
\usepackage {listings}
\usepackage {mathpartir}
\usepackage{bcprules}
%\usepackage{listings}
                       
\usepackage{graphicx} 
%\usepackage[margins=2.5cm,nohead,nofoot]{geometry}
%\usepackage{geometry}
\usepackage{amsfonts}
\usepackage{amstext}
\usepackage{latexsym}
\usepackage{amssymb}
\usepackage{color}


%\include{myPreamble}
\include{qm2pi.local} 

%\ifpdf
%\usepackage[pdftex]{graphicx}
%\else
%\usepackage{graphicx}
%\fi

 % \ifpdf
%  \usepackage{pdfsync}
%  \if


%\title{Brief Article}
%\author{David F. Snyder}
%\author{L.G. Meredith}

%\address{Dept. of Math., Texas State University--San Marcos, San Marcos, TX 78666}
       
\pagestyle{empty}


\begin{document}

\lstset{language=[Objective]Caml,frame=shadowbox}

\input{qm2pi.front}

% section front matter (end)

\input{qm2pi.intro} 
 
% section introduction (end)

% \input{qm2pi.knotations} 

% section notation (end)

\input{qm2pi.process.calculi} 

% section concurrent_process_calculi_and_spatial_logics_ (end)
    
%\input{qm2pi.knots2pi} 

%\input{qm2pi.trefoil} 

%\input{qm2pi.mainthm} 

% subsection basic_interpretation (end)

%\input{qm2pi.rho.presentation} 
\subsection{The syntax and semantics of the notation system}\label{sub:the_syntax_and_semantics_of_the_notation_system} % (fold)

We now summarize a technical presentation of the calculus that
embodies our theory of dynamics. The typical presentation of such a
calculus follows the style of giving generators and relations on
them. The grammar, below, describing term constructors, freely
generates the set of processes, $\Proc$. This set is then quotiented
by a relation known as structural congruence and it is over this set
that the notion of dynamics is expressed. This presentation is
essentially that of \cite{MeredithR05} with the addition of
polyadicity and summation. For readability we have relegated some of
the technical subtleties to an appendix.

\subsubsection{Process grammar}\label{subsub:process_grammar}

\begin{mathpar}
  \inferrule* [lab=synchronization] {} {{M} \bc \pzero \;|\; x?F \;|\; x!C }
  \and
  \inferrule* [lab=abstraction] {} {{F} \bc (x)P}
  \and
  \inferrule* [lab=concretion] {} {{C} \bc \langle Q \rangle}
  \and
  \inferrule* [lab=process] {} {{P,Q} \bc M \;| \;P|Q \;|\; @{x}}
  \and
  \inferrule* [lab=name] {} {{x} \bc \quotep{P}}
\end{mathpar} 

Note that $\vec{x}$ (resp. $\vec{P}$) denotes a vector of names
(resp. processes) of length $|\vec{x}|$ (resp. $|\vec{P}|$). We adopt
the following useful abbreviations.

\begin{mathpar}
   x?(\vec{y}).P := x.(\vec{y})P \and  x\clift{\vec{P}} := x.\clift{\vec{P}}
   \and x!(y) := \lift{x}{\dropn{y}}
   \and \Pi_{i=0}^{n-1}P_i := P_0 | \ldots | P_{n-1}
\end{mathpar}

\subsubsection{Structural congruence}

\paragraph{Free and bound names and alpha-equivalence.} At the
core of structural equivalence is alpha-equivalence which identifies
process that are the same up to a change of variable. Formally, we
recognize the distinction between free and bound names. The free names
of a process, $\freenames{P}$, may be calculated recursively as
follows:

\begin{mathpar}
\freenames{\pzero} := \emptyset
  \and \\
  \freenames{x?(y).P} := \{ x \} \cup (\freenames{P} \setminus \{ y \})
  \and 
  \freenames{x!\langle P \rangle} := \{ x \} \cup \{ P \} 
  \and \\
  \freenames{P|Q} := \freenames{P} \cup \freenames{Q}
  \and \\
  \freenames{@{x}} := \{ x \}
\end{mathpar}

$\pi$
$\quotep{\pi}$

$\freenames{-} : \pi \to \mathcal{P}(\quotep{\pi})$

\begin{eqnarray*}
  \freenames{\pzero} & := & \emptyset \\
  \freenames{x?(y).P} & := & \{ x \} \cup (\freenames{P} \setminus \{ y \}) \\
  \freenames{x!\langle P \rangle} & := & \{ x \} \cup \{ P \} \\
  \freenames{P|Q} & := & \freenames{P} \cup \freenames{Q} \\
  \freenames{\dropn{x}} & := & \{ x \}
\end{eqnarray*}

The bound names of a process, $\boundnames{P}$, are those names occurring in $P$
that are not free. For example, in $x?(y).0$, the name $x$ is free, while $y$ is bound.

\begin{mathpar}
  \inferrule* [lab=monoidal-laws] {} { P|Q \equiv Q|P \and P|0 \equiv P \and P|(Q|R) \equiv (P|Q)|R }
\end{mathpar}

\begin{mathpar}
  \inferrule* [lab=alpha-equivalence] {} { (x)P \equiv (y)P\{y/x\} \and y \not\in \freenames{P} }
\end{mathpar}

\begin{definition}
Then two processes, $P,Q$, are alpha-equivalent if $P = Q\{\vec{y}/\vec{x}\}$ for
some $\vec{x} \in \boundnames{Q},\vec{y} \in \boundnames{P}$, where $Q\{\vec{y}/\vec{x}\}$
denotes the capture-avoiding substitution of $\vec{y}$ for $\vec{x}$ in $Q$.
\end{definition}

\begin{definition}
  The {\em structural congruence} \cite{SangiorgiWalker} , $\equiv$,
  between processes is the least congruence containing
  alpha-equivalence, satisfying the abelian monoid laws
  (associativity, commutativity and $\pzero$ as identity) for parallel
  composition $|$ and for summation $+$.
\end{definition}

\subsection{Name equivalence}

We take name equivalence, written $\nameeq$, to be the smallest
equivalence relation generated by the following rules.

\begin{mathpar}
\inferrule*[lab=Quote-drop]
{ }
{ \quotep{@{x}} \nameeq x }

\inferrule*[lab=Struct-equiv]
{ P \scong Q }
{ \quotep{P} \nameeq \quotep{Q} }
\end{mathpar}

The astute reader will have noticed that the mutual recursion of names
and processes imposes a mutual recursion on alpha-equivalence and
structural equivalence via name-equivalence. Fortunately, all of this
works out pleasantly and we may calculate in the natural way, free of
concern. The reader interested in the details is referred to the
appendix \ref{appendix:rho_details}.

\subsection{Substitution}

We use $\Proc$ for the set of processes, $\QProc$ for the set of
names, and $\id{\{}\vec{y} / \vec{x} \id{\}}$ to denote partial maps,
$s : \QProc \rightarrow \QProc$. A map, $s$ lifts, uniquely, to a map
on process terms, $\widehat{s} : \Proc \rightarrow \Proc$ by the
following equations.

\begin{mathpar}
  (0) \psubstp{Q}{P} := 0 \\
  (R \juxtap S) \psubstp{Q}{P}
  :=    
  (R)\psubstp{Q}{P} \juxtap (S) \psubstp{Q}{P} \\
  (x?(y).R) \psubstp{Q}{P}    
  :=    
  (x)\substp{Q}{P} (z)\concat( (R \psubstn{z}{y}) \psubstp{Q}{P} ) \\
  (\lift{x}{R}) \psubstp{Q}{P}  
  :=
  \lift{(x)\substp{Q}{P}}{ R \psubstp{Q}{P} } \\
%   (\dropn{x})  \psubstp{Q}{P}       
%   := 
%   \left\{ 
%     \begin{array}{ccc} 
%       \dropn{\quotep{Q}} & & x \nameeq \quotep{P} \\
%       \dropn{x} & & otherwise \\
%     \end{array}
%   \right. 
  (\dropn{x})  \psubstp{Q}{P}       
  := 
  \left\{ 
    \begin{array}{ccc} 
      Q & & x \nameeq \quotep{P} \\
      \dropn{x} & & otherwise \\
    \end{array}
  \right.
\end{mathpar}
 

where

\begin{eqnarray}
  (x)\id{\{} \lpquote Q \rpquote / \lpquote P \rpquote \id{\}}            = 
  \left\{ 
    \begin{array}{ccc}
      \lpquote Q \rpquote & & x \nameeq \lpquote P \rpquote \\
      x & & otherwise \\
    \end{array}
  \right. \nonumber
\end{eqnarray}

and $z$ is chosen distinct from $\quotep{P}$, $\quotep{Q}$, the free
names in $Q$, and all the names in $R$. Our $\alpha$-equivalence will
be built in the standard way from this substitution.

\begin{remark}\label{rem:no_self_referential_names}
  One consequence of these definitions is that $\forall P. \quotep{P}
  \not\in \freenames{P}$.
\end{remark}

\subsection{ Dynamic quote: an example }

Anticipating something of what's to come, consider applying the
substitution, $\widehat{\id{\{}u / z \id{\}}}$, to the following pair
of processes, $\lift{w}{y!(z)}$ and $w[ \lpquote y!(z) \rpquote ]$.

\begin{eqnarray}
	\lift{w}{y!(z)}\widehat{\id{\{}u / z \id{\}}}
		& = &
		\lift{w}{y!(u)} \nonumber\\
	w[ \lpquote y!(z) \rpquote ] \widehat{ \id{\{}u / z \id{\}} }
		& = &
		w[ \lpquote y!(z) \rpquote ] \nonumber
\end{eqnarray}

Because the body of the process between quotes is impervious to
substitution, we get radically different answers. In fact, by
examining the first process in an input context,
e.g. $x?(z).\lift{w}{y!(z)}$, we see that the process under the lift
operator may be shaped by prefixed inputs binding a name inside it. In
this sense, the lift operator will be seen as a way to dynamically
construct processes before reifying them as names.

Finally equipped with these standard features we can present the
dynamics of the calculus.

\subsubsection{Operational semantics} 

Finally, we introduce the computational dynamics. What marks these
algebras as distinct from other more traditionally studied algebraic
structures, e.g. vector spaces or polynomial rings, is the manner in
which dynamics is captured. In traditional structures, dynamics is typically
expressed through morphisms between such structures, as in linear maps
between vector spaces or morphisms between rings. In algebras
associated with the semantics of computation, the dynamics is
expressed as part of the algebraic structure itself, through a
reduction reduction relation typically denoted by $\red$. Below, we
give a recursive presentation of this relation for the calculus used
in the encoding.

$\red \subseteq \pi \times \pi$
$\red : \pi \to \mathcal{P}(\pi)$

\begin{mathpar}
  \inferrule* [lab=Comm] { \textsf{match}( x_{src}, x_{trgt} ) } { x_{trgt}?(y)P \; | \; x_{src}!\langle {Q} \rangle \red P\{\quotep{Q}/y}\} }
  \and \\
  \inferrule* [lab=Par] {{P} \red {P}'} {{{P} | {Q}} \red {{P}' | {Q}}}
  \and
  \inferrule* [lab=Equiv]{{{P} \scong {P}'} \andalso {{P}' \red {Q}'} \andalso {{Q}' \scong {Q}}}{{P} \red {Q}}
\end{mathpar}

\begin{eqnarray*}
  match_{\equiv} (\quotep{P},\quotep{Q}) & := & P \equiv Q \\
  match_{\dagger}(\quotep{P},\quotep{Q}) & := & \forall R. P|Q \red^{*} R => R \red^{*} 0 \\
  match_{K}(\quotep{P},\quotep{Q}) & := & K \mbox{ for some context } K
\end{eqnarray*}

$u?(x)P | u!\langle Q \rangle \red P\{\quotep{Q}/x\}$

%We write $\wred$ for $\red^*$, and $P\red$ if $\exists Q $ such that $ P \red Q$.
We write $P\red$ if $\exists Q $ such that $ P \red Q$ and $P\not\red$, otherwise.

\section{Replication}

As mentioned before, it is known that replication (and hence
recursion) can be implemented in a higher-order process algebra
\cite{SangiorgiWalker}. As our first example of calculation with the
machinery thus far presented we give the construction explicitly in
the {\rhoc}.

\begin{eqnarray}
	D_{x} & := & \prefix{x}{y}{(\binpar{\outputp{x}{y}}{@{y}})} \nonumber\\
	\bangp_{x}{P} & := & \binpar{{x}!\langle{\binpar{D_{x}}{P}}\rangle}{D_{x}} \nonumber
\end{eqnarray}

\begin{eqnarray}
	\bangp_{x}{P} & & \nonumber\\
	=
	& {x}!\langle{(\prefix{x}{y}{(\outputp{x}{y} | @{y})) | P}}\rangle 
	      | \prefix{x}{y}{(\outputp{x}{y} | @{y})} & \nonumber\\
	\red
	& (\outputp{x}{y} | @{y})\substn{\quotep{(\prefix{x}{y}{(@{y} | \outputp{x}{y})) | P}}}{y} & \nonumber\\
	=
	& \outputp{x}{\quotep{(\prefix{x}{y}{(\outputp{x}{y} | @{y})) | P}}}
	  | {(\prefix{x}{y}{(\outputp{x}{y} | @{y})) | P}} & \nonumber\\
	\red
	& \ldots & \nonumber\\
	\red^*
	& P | P | \ldots & \nonumber
\end{eqnarray}

Of course, this encoding, as an implementation, runs away, unfolding
$\bangp{P}$ eagerly. A lazier and more implementable replication
operator, restricted to input-guarded processes, may be obtained as follows.

\begin{eqnarray}
\bangp{\prefix{u}{v}{P}} 
	:= 
	\binpar{\lift{x}{\prefix{u}{v}{(\binpar{D(x)}{P})}}}{D(x)} \nonumber
\end{eqnarray}

\begin{remark}
  Note that the lazier definition still does not deal with summation
  or mixed summation (i.e. sums over input and output). The reader is
  invited to construct definitions of replication that deal with these
  features. 

  Further, the definitions are parameterized in a name, $x$. Can you,
  gentle reader, make a definition that eliminates this parameter and
  guarantees no accidental interaction between the replication
  machinery and the process being replicated -- i.e. no accidental
  sharing of names used by the process to get its work done and the
  name(s) used by the replication to effect copying. This latter
  revision of the definition of replication is crucial to obtaining
  the expected identity $!!P \sim !P$.
\end{remark}

\begin{remark}\label{rem:paradoxical_combinator}
  The reader familiar with the lambda calculus will have noticed the
  similarity between $D$ and the paradoxical combinator.

  [Ed. note: the existence of this seems to suggest we have to be more
  restrictive on the set of processes and names we admit if we are to
  support no-cloning.]
\end{remark}

\subsubsection{Bisimulation}

The computational dynamics gives rise to another kind of equivalence,
the equivalence of computational behavior. As previously mentioned
this is typically captured \emph{via} some form of bisimulation.

% The notion we use in this paper is weak barbed bisimulation
% \cite{milner91polyadicpi}.

The notion we use in this paper is derived from weak barbed
bisimulation \cite{milner91polyadicpi}. 

\begin{definition}
An \emph{observation relation}, $\downarrow_{\mathcal N}$, over a set
of names, $\mathcal N$, is the smallest relation satisfying the rules
below.

\infrule[Out-barb]{y \in {\mathcal N}, \; x \nameeq y}
		  {\outputp{x}{v} \downarrow_{\mathcal N} x}
\infrule[Par-barb]{\mbox{$P\downarrow_{\mathcal N} x$ or $Q\downarrow_{\mathcal N} x$}}
		  {\binpar{P}{Q} \downarrow_{\mathcal N} x}

We write $P \Downarrow_{\mathcal N} x$ if there is $Q$ such that 
$P \wred Q$ and $Q \downarrow_{\mathcal N} x$.
\end{definition}

\begin{definition}
%\label{def.bbisim}
An  ${\mathcal N}$-\emph{barbed bisimulation} over a set of names, ${\mathcal N}$, is a symmetric binary relation 
${\mathcal S}_{\mathcal N}$ between agents such that $P\rel{S}_{\mathcal N}Q$ implies:
\begin{enumerate}
\item If $P \red P'$ then $Q \wred Q'$ and $P'\rel{S}_{\mathcal N} Q'$.
\item If $P\downarrow_{\mathcal N} x$, then $Q\Downarrow_{\mathcal N} x$.
\end{enumerate}
$P$ is ${\mathcal N}$-barbed bisimilar to $Q$, written
$P \wbbisim_{\mathcal N} Q$, if $P \rel{S}_{\mathcal N} Q$ for some ${\mathcal N}$-barbed bisimulation ${\mathcal S}_{\mathcal N}$.
\end{definition}

$\mathcal{R} \subseteq \pi \times \pi$

$P \mathcal{R} Q => \forall P'. P \red P' \Rightarrow \exists Q'. Q \red Q', P' \mathcal{R} Q'$

$P \vdash x \Rightarrow Q \vdash x$

\begin{mathpar}
  \inferrule*[lab=Out-barb]{x \nameeq y}{{y}!\langle{Q}\rangle \vdash x}
  \and
  \inferrule*[lab=Par-barb]{\mbox{$P\vdash x$ or $Q\vdash x$}}{\binpar{P}{Q} \vdash x}
\end{mathpar}

\subsubsection{Contexts}

One of the principle advantages of computational calculi like the
$\pi$-calculus is a well-defined notion of context,
contextual-equivalence and a correlation between
contextual-equivalence and notions of bisimulation. The notion of
context allows the decomposition of a process into (sub-)process and
its syntactic environment, its context. Thus, a context may be
thought of as a process with a ``hole'' (written $\Box$) in it. The
application of a context $M$ to a process $P$, written $M[P]$, is
tantamount to filling the hole in $M$ with $P$. In this paper we do
not need the full weight of this theory, but do make use of the notion
of context in the proof the main theorem. 

\begin{mathpar}
  \inferrule* [lab=summation] {} {{M_{M},M_{N}} \bc \Box \;|\; x.M_{A} \;|\; M_{M}+M_{N}}
  \and
  \inferrule* [lab=agent] {} {{M_{A}} \bc (\vec{x})M_{P} \;| \; \clift{P_0,\ldots,M_{P},\ldots,P_N}}
  \and \\
  \inferrule* [lab=process] {} {{M_{P}} \bc M_{N} \;| \;P|M_{P} }
\end{mathpar} 

\begin{mathpar}
  \inferrule* [lab=sychronization] {} {M_{N} \bc \Box \;|\; x?M_{F} \;|\; x!M_{C}}
  \and
  \inferrule* [lab=abstraction] {} {{M_{F}} \bc (x)M_{P} }
  \and
  \inferrule* [lab=concretion] {} {{M_{C}} \bc \langle M_{P} \rangle }
  \and \\
  \inferrule* [lab=process] {} {{M_{P}} \bc M_{N} \;| \;P|M_{P} }
\end{mathpar}

\begin{definition}[contextual application] Given a context $M$, and
  process $P$, we define the \emph{contextual application}, $M[P] :=
  M\{P/\Box\}$. That is, the contextual application of M to P is the
  substitution of $P$ for $\Box$ in $M$.
\end{definition}

$\meaningof{-} : L \to \mathcal{P}(\pi)$

\begin{mathpar}
  \inferrule* [lab=collection] {} {\meaningof{true} = \pi, \and \meaningof{~E} = \pi \setminus \meaningof{E}, \and \meaningof{E_{1} \& E_{2}} = \meaningof{E_{1}} \cap \meaningof{E_{2}}}
\end{mathpar}

\begin{mathpar}
  \inferrule* [lab=structure] {} {\meaningof{0} = \{ P \in \pi | P \equiv 0 \}, \and \\ \meaningof{E_1 | E_2} = \{ P \in \pi | P \equiv P_{1} | P_{2}, P_{1} \in \meaningof{E_{1}}, P_{2} \in \meaningof{E_2}\} }
\end{mathpar}

\begin{mathpar}
 \inferrule* [lab=behavior] {} {\meaningof{\langle a?b \rangle E} = \{ P \in \pi | P \equiv Q | u?(y)P', \\ \and \\\\ \and \\ \;\;\; u \in \meaningof{a}, \forall z.P'\{z/y\} \in \meaningof{E\{z/b\}}\}, \and \\ \meaningof{a!E} = \{ P \in \pi | P \equiv Q | x!\langle P' \rangle, x \in \meaningof{a} P' \in \meaningof{E}\} }
\end{mathpar}

\begin{mathpar}
 \inferrule* [lab=nominal] {} {\meaningof{\quotep{E}} = \{ \quotep{P} \in \quotep{\pi} | P \in \meaningof{E} \}, \and \meaningof{\quotep{P}} = \{ \quotep{Q} \in \quotep{\pi} | P \equiv Q \} \and \\ \meaningof{@\quotep{E}} = \{ P \in \pi | P \equiv @x, x \in \meaningof{E} \}}
\end{mathpar}

\begin{eqnarray*}
  \\
  \meaningof{-} : TS \to ST
\end{eqnarray*}

\begin{eqnarray*}
  \\
  L : TS \to ST
\end{eqnarray*}

\begin{eqnarray*}
  \\
  P \models E \iff P \in \meaningof{E}
\end{eqnarray*}

\begin{eqnarray*}
  P \approx_{L} Q \iff \forall E \in L. P \models E \iff Q \models E
\end{eqnarray*}

\begin{eqnarray*}
  P \approx_{K} Q
\end{eqnarray*}

\begin{eqnarray*}
  P \approx Q
\end{eqnarray*}

$\approx_{K} = \approx = \approx_{L}$

\subsubsection{Contextual duality}

Note that contexts extend the quotation operation to a family of
operations from processes to names. Given a context, $M$, we can
define a \emph{nominal context}, $\quotep{M}$ by $\quotep{M}[P] :=
\quotep{M[P]}$. To foreshadow what is to come we observe that these
operations enjoy a duality with processes very much like the duality
between vectors and maps from vectors to scalars.

Further, because the calculus is essentially higher-order, we have a
correspondence between contexts and processes. More specifically,
given a name $x$ and a context $M$ we can construct $M^{*}_{x}$ such
that 

\begin{mathpar}
  M^{*}_{x} | \lift{x}{P} \red M[P]
\end{mathpar}

namely,

\begin{mathpar}
  M^{*}_{x} := x?(u).M[\dropn{u}]
\end{mathpar}

The dependence of $M^{*}_{x}$ on a name makes it an abstraction, 

\begin{mathpar}
  M^{*} := (x)x?(u).M[\dropn{u}]
\end{mathpar}

\subsection{Additional notation}

It will sometimes be convenient to denote the process a name
quotes. We already have the notation $x = \quotep{P}$, but it will be
convenient to introduce an alternate notation, $\procn{x}$, when we
want to emphasize the connection to the use of the name. Note that, by
virtue of name equivalence, $\quotep{\procn{x}} \nameeq x$; so, the
notation is consistent with previous definitions.

Further, because names have structure it is possible to effect
substitutions on the basis of that structure. This means we need to
upgrade our notation for substitutions, which we accomplish by
adapting comprehension notation. Thus,

\begin{mathpar}
  P\{ y / x : x \in S \}
\end{mathpar}

is interpreted to mean the process derived from P by replacing (in a
capture-avoiding manner) each occurrence of $x$ in $S$ by $y$. For example,

\begin{mathpar}
  P\{ \quotep{\procn{x}|\procn{x}} / x : x \in \freenames{P} \}
\end{mathpar}

will replace each (occurrence) of a free name $x$ in $P$ by
$\quotep{\procn{x}|\procn{x}}$.

Also, we will avail ourselves of the notation $x^{L}$ and $x^{R}$ to
denote injections of a name into disjoint copies of the name
space. There are numerous ways to accomplish this. One example can be
found in \cite{MeredithR05}. This notation overloads to vectors of
names: $\vec{x}^{\pi} := (x_{i}^{\pi} \; : \; 0 \leq i < |\vec{x}| )$ where $\pi \in \{L,R\}$.

We also use $P^{\Box} := P|\Box$.

In \cite{MeredithR05} an interpretation of the new operator is
given. It turns out that there are several possible interpretations
all enjoying the requisite algebraic properties of the operator (see
\cite{milner91polyadicpi}). We will therefore make liberal use of
$(\nu\; \vec{x})P$.

% subsection the_syntax_and_semantics_of_the_notation_system (end)   

\input{qm2pi.qmops} 

\input{qm2pi.sterngerlach} 

\input{qm2pi.metric} 

% section concurrent_process_calculi (end)

%\input{qm2pi.proofsketch}

% section proof sketch (end)

%\input{qm2pi.slviaknots} 

% section spatial logic via knots (end)

\input{qm2pi.conclusion}

% section conclusion (end)

%\input{qm2pi.dtcodes} 

% section wiring algorithm (end)

\input{qm2pi.ack} 

% section acknowledgments (end)

\newpage


\bibliographystyle{plain}   
\bibliography{../../biblios/main.bib}

\input{qm2pi.rhodetails}

\end{document}



% section proof sketch (end)

%\section{Unlikely characters: spatial logic for
  knots}\label{sub:characteristic_formulae} % (fold)

Associated to the mobile process calculi are a family of logics known
as the Hennessy-Milner logics. These logics typically enjoy a
semantics interpreting formulae as sets of processes that when
factored through the encoding outlined above allows an identification
of classes of knots with logical formulae. In the context of this
encoding the sub-family known as the spatial logics \cite{CairesC03}
\cite{CairesC04} \cite{Caires04} are of particular interest providing
several important features for expressing and reasoning about
properties (i.e. classes) of knots. We hint here at how this may be done.

%\begin{description}
%\item [structural connectives] 
\subsubsection{Structural connectives} The spatial logics enjoy
structural connectives corresponding, at the logical level, to the
parallel composition ($P | Q$) and new name ($(\nu \; x)P$)
connectives for processes. As illustrated in the examples below, these
connectives are extremely expressive given the shape of our encoding.
%\item [decideable satisfaction]

\subsubsection{Decideable satisfaction}
In \cite{Caires04} the satisfaction relation is shown to be decideable
for a rich class of processes. It further turns out that the image of
the our encoding is a proper subset of that class. This result
provides the basis for an algorithm by which to search for knots
enjoying a given property.
%\item [characteristic formulae]

\subsubsection{Characteristic formulae}
In the same paper \cite{Caires04} , Caires presents a means of calculating
characteristic formulae, selecting equivalence classes of processes
up to a pre--specified depth limit on the support set of names. Composed with our
encoding, this characteristic formula can be used to select
characteristic formulae for knots.
%\end{description}

\subsubsection{Spatial logic formulae}

The grammar below (segmented for comprehension) summarizes the syntax
of spatial logic formulae. We employ illustrative examples in the
sequel to provide an intuitive understanding of their meaning
referring the reader to \cite{Caires04} for a more detailed explication
of the semantics.

\begin{mathpar}
  \inferrule* [lab=boolean] {} {{A,B} \bc T \;|\; \neg A \;|\; A \wedge B \;|\; \eta = \eta'}
  \and
  \inferrule* [lab=spatial] {} {|\; \pzero \;|\; A | B \;|\; x \text{\textregistered} A \;|\; \forall x . A \;|\;  H x . A}
  \and
  \inferrule* [lab=behavioral] {} {|\; \alpha . A}
  \and 
  \inferrule* [lab=recursion] {} {|\; X(\vec{u}) \;|\; \mu X(\vec{u}) . A}
  \and
  \inferrule* [lab=action] {} {\alpha \bc \langle x?(\vec{y}) \rangle \;|\; \langle x!(\vec{y}) \rangle \;|\; \langle \tau \rangle}
  \and 
  \inferrule* [lab=name] {} {\eta \bc x \;|\; \tau}
\end{mathpar} 

% subsection characteristic_formulae (end)   	 

\subsection{Example formulae}\label{sub:example_formulae_} % (fold)

\subsubsection{Crossing as formula.}
% 
% \begin{align*}
%   \frac{d}{dx} \sin x &= \cos x 
%   & \frac{d}{dx} e^x &= e^x \\
%   \frac{d}{dx} \cos x &= - \sin x 
%   & \frac{d}{dx} \log x &= \frac{1}{x} \\
% \end{align*} 

\begin{align*}
 \mu C(x_{0},x_{1},y_{0},y_{1},u).&(\langle x_{0}?(z) \rangle(\langle u! \rangle\langle y_{1}!z \rangle C(x_{0},x_{1},y_{0},y_{1},u)) & \\
  & \wedge \langle y_{1}?(z) \rangle (\langle u! \rangle \langle x_{0}!z \rangle C(x_{0},x_{1},y_{0},y_{1},u)) & \\
  & \wedge \langle x_{1}?(z) \rangle (\langle u? \rangle \langle y_{0}!z \rangle C(x_{0},x_{1},y_{0},y_{1},u)) & \\
  & \wedge \langle y_{0}?(z) \rangle (\langle u? \rangle \langle x_{1}!z \rangle C(x_{0},x_{1},y_{0},y_{1},u))) &
\end{align*}

The lexicographical similarity between the shape of this formulae and
the shape of definition of the process representing a crossing reveals
the intuitive meaning of this formulae. It describes the capabilities
of a process that has the right to represent a crossing. For example
it picks out processes that may perform an input on the port $x_0$ in
its initial menu of capabilities. What differentiates the formula
from the process, however, is that the crossing process is the
smallest candidate to satisfy the formula. Infinitely many other
processes -- with internal behavior hidden behind this interface, so
to speak -- also satisfy this formula. Even this simple formula,
then, can be seen to open a new view onto knots, providing a
computational interpretation of \emph{virtual} knots.

Note that this formula is derived by hand. A similar formula can be
derived by employing Caires' calculation of characteristic formula
\cite{Caires04} to the process representing a crossing. In light of
this discussion, we let
$\meaningof{C}_{\phi}(x0,x1,y0,y1,u)$ denote a formula specifying the
dynamics we wish to capture of a crossing. To guarantee we preserve
the shape of the interface and minimal semantics we demand that
$\meaningof{C}_{\phi}(x0,x1,y0,y1,u) \Rightarrow
\textbf{C}(x0,x1,y0,y1,u)$ where $\textbf{C}(x0,x1,y0,y1,u)$ denotes
the formula above.
                            
\subsubsection{Crossing number constraints.}
The moral content of the context lemma (Lemma \ref{context}) is that the notion of
``locality'' in the Reidemeister moves is effectively captured by the
parallel composition operator of the process calculus. This intuition
extends through the logic. Given a formula,
$\meaningof{C}_{\phi}(x0,x1,y0,y1,u)$, we can use the structural
connectives to specify constraints on crossing numbers, such as at
least $n$ crossings, or exactly $n$ crossings.
\begin{mathpar}
  \inferrule* [lab=at-least-n] {} { K^{\geq n}_{\phi}(\vec{xs},\vec{ys}) := \Pi_{i=0}^{n-1} Hu . \meaningof{C}_{\phi}(xs_i,ys_i,u) | T }
  \and 
  \inferrule* [lab=exactly-n] {} { K^{= n}_{\phi}(\vec{xs},\vec{ys}) := \Pi_{i=0}^{n-1} Hu . \meaningof{C}_{\phi}(xs_i,ys_i,u) | \neg (\forall x_0,y_0,x_1,y_1,u . \meaningof{C}_{\phi}(x_0,y_0,x_1,y_1,u) | T) }
\end{mathpar}

To round out this section, recall that the encoding of an $n$-crossing
knot decomposes into a parallel composition of $n$ \emph{copies} of a
crossing process together with a wiring harness. To specify different
knot classes with the same crossing number amounts to specifying
logical constraints on the wiring harness. In the interest of space,
we defer examples to a forthcoming paper. Suffice it to say that both
the conditions ``alternating knot'' and ``contains the tangle
corresponding to 5/3'' are expressible. For example, it is possible to
calculate the characteristic formula of a process corresponding to the
tangle 5/3 and conjoin it into the classifying formula via the
composition connective of the logic.

Finally, we wish to observe that it is entirely within reason to
contemplate a more domain-specific version of spatial logic tailored
to the shape of processes in the image of the encoding. Such a
domain-specific logic would have a better claim to the title formal
language of knot properties.

% subsection example_formulae_ (end)

% section knots_as_processes (end) 

% section spatial logic via knots (end)

\section{Conclusions and future work}

\paragraph{Testing physical space}
You, gentle reader, may wonder why of all the theorems to be proved
given this set up we pick the one above. In some sense it's hardly
central to quantum mechanics. We see it as central in the sense that
it firmly establishes a notion of physical space arising from a notion
of the equivalence of behavior. Relating bisimulation to a metric is a
big step forward, but one is faced with interpreting the relationship
of that metric space to something more physical. Quantum mechanical
notions of ``physical'' space are still far from intuitive, but by
relating this idea of distance as testing to calculations that predict
physical circumstances we are making a not insignificant step forward
toward an understanding of the physical space we inhabit as
essentially dynamic.

\paragraph{Effectivity and simulation}
One of the observations we have yet to make is that the entire program
spelled out here is effective. We have built various interpreters for
the reflective calculus at work in this interpretation. In principle,
then, we can simulate quantum mechanics on a computer. The place where
the simulation may lose fidelity is the infinitely branching summation
for the annihilator.

In this connection i also want to point out that the evaluation style
calculation of the inner product puts the non-determinism of the
summation right at the heart of measurement. This suggests that
Milner's original reduction-based formulation of the dynamics of his
calculi in terms of sums was not just notationally suggestive of a
notion of measure-and-continue but captured some significant part of
the physics.

\paragraph{Quantum continuations}
In light of this last observation i want to point out that the
predominant account of quantum mechanics is missing a key aspect of a
truly compositional story of the physical situation. In a real lab,
when a measurement is made the observation can be made to feed into
another device that then makes another measurement conditioned on the
results of the first. This means that after the superposition was
collapsed the entire experimental set up remained in
superposition. While QM offers a means of writing this down it doesn't
quite line up well with the well-trodden formulation of computation
and continuation that we see so succinctly expressed in Milner's
calculi. This suggests that there might be advantages to this account
of dynamics waiting to be explored.

\paragraph{Quantum logic}
In this connection, we also note that by virtue of having the
Hennessy-Milner construction, we can pull the construction through the
interpretation of QM. This gives us a natural candidate for a quantum
logic that enjoys an extremely tight connection with it's domain of
interpretation, making the construction much less ad hoc (rather it is
the image of functor!).

\paragraph{Quantum probabiity}
i have questions about the basis of the interpretation of inner
product as probability amplitude. In particular, using which
axiomatization of probability theory does the notion of probability
amplitude earn the right to be so dubbed? In other words, where is the
proof that the operation for calculating a probability amplitude (and
then squaring) satisfies the axioms of what it means to calculate a
probability? Even if such a proof exists (i have yet to find it in the
literature), i wonder if it might not be possible to turn things on
their heads. Can we view the calculation of the probability amplitude
as an axiomatization of probability? If so, then the definition we
give for calculating probability amplitude may provide the basis for
an \emph{effective} theory of probability.

\paragraph{Quantum vs ``biological'' information}
Finally, i want to conclude with a more philosophical observation. At
a recent workshop in which QM was a predominant topic i noticed
something about quantum information. The speaker was giving a riveting
discussion of axiomatic QM and showing how properties of ``no
cloning'' and ``no deleting'' emerged as consequences of the
axiomatization. Theorems of this form are necessary to give us a sense
of confidence that our axioms characterize the physical theory. What
struck me, though, was that if quantum information is neither erasable
nor replicable it is markedly different from \emph{life}. Two of the
things we know about life is that

\begin{itemize}
  \item it ends;
  \item to gain some measure of persistence, to transcend it's
    finitude it is imminently copyable.
\end{itemize}

Both of these qualities are summarized succinctly in the aphorism: all
flesh is grass. For me these two kinds of ``information'' -- call them
quantum and biological -- are end points on a spectrum of strategies
for persistence. At one end, we have those curious entities that enjoy
uniqueness and permanence; at the other, we have those who in the face
of a certain end and an uncertain present make a go of passing
something on. To me one of the more remarkable aspects of the latter
strategy is that in the presence of noise (and certain features of
copying) we get a kind of dynamism, a chance for improvement against a
given persistent condition.

% subsection other_calculi_other_bisimulations_and_geometry_as_behavior (end)




% section conclusion (end)

%\documentclass[12pt]{llncs}
%\documentclass{jktr}

\usepackage[pdftex]{hyperref}                   
\usepackage {listings}
\usepackage {mathpartir}
\usepackage{bcprules}
%\usepackage{listings}
                       
\usepackage{graphicx} 
%\usepackage[margins=2.5cm,nohead,nofoot]{geometry}
%\usepackage{geometry}
\usepackage{amsfonts}
\usepackage{amstext}
\usepackage{latexsym}
\usepackage{amssymb}
\usepackage{color}


%\include{myPreamble}
\include{qm2pi.local} 

%\ifpdf
%\usepackage[pdftex]{graphicx}
%\else
%\usepackage{graphicx}
%\fi

 % \ifpdf
%  \usepackage{pdfsync}
%  \if


%\title{Brief Article}
%\author{David F. Snyder}
%\author{L.G. Meredith}

%\address{Dept. of Math., Texas State University--San Marcos, San Marcos, TX 78666}
       
\pagestyle{empty}


\begin{document}

\lstset{language=[Objective]Caml,frame=shadowbox}

\input{qm2pi.front}

% section front matter (end)

\input{qm2pi.intro} 
 
% section introduction (end)

% \input{qm2pi.knotations} 

% section notation (end)

\input{qm2pi.process.calculi} 

% section concurrent_process_calculi_and_spatial_logics_ (end)
    
%\input{qm2pi.knots2pi} 

%\input{qm2pi.trefoil} 

%\input{qm2pi.mainthm} 

% subsection basic_interpretation (end)

%\input{qm2pi.rho.presentation} 
\subsection{The syntax and semantics of the notation system}\label{sub:the_syntax_and_semantics_of_the_notation_system} % (fold)

We now summarize a technical presentation of the calculus that
embodies our theory of dynamics. The typical presentation of such a
calculus follows the style of giving generators and relations on
them. The grammar, below, describing term constructors, freely
generates the set of processes, $\Proc$. This set is then quotiented
by a relation known as structural congruence and it is over this set
that the notion of dynamics is expressed. This presentation is
essentially that of \cite{MeredithR05} with the addition of
polyadicity and summation. For readability we have relegated some of
the technical subtleties to an appendix.

\subsubsection{Process grammar}\label{subsub:process_grammar}

\begin{mathpar}
  \inferrule* [lab=synchronization] {} {{M} \bc \pzero \;|\; x?F \;|\; x!C }
  \and
  \inferrule* [lab=abstraction] {} {{F} \bc (x)P}
  \and
  \inferrule* [lab=concretion] {} {{C} \bc \langle Q \rangle}
  \and
  \inferrule* [lab=process] {} {{P,Q} \bc M \;| \;P|Q \;|\; @{x}}
  \and
  \inferrule* [lab=name] {} {{x} \bc \quotep{P}}
\end{mathpar} 

Note that $\vec{x}$ (resp. $\vec{P}$) denotes a vector of names
(resp. processes) of length $|\vec{x}|$ (resp. $|\vec{P}|$). We adopt
the following useful abbreviations.

\begin{mathpar}
   x?(\vec{y}).P := x.(\vec{y})P \and  x\clift{\vec{P}} := x.\clift{\vec{P}}
   \and x!(y) := \lift{x}{\dropn{y}}
   \and \Pi_{i=0}^{n-1}P_i := P_0 | \ldots | P_{n-1}
\end{mathpar}

\subsubsection{Structural congruence}

\paragraph{Free and bound names and alpha-equivalence.} At the
core of structural equivalence is alpha-equivalence which identifies
process that are the same up to a change of variable. Formally, we
recognize the distinction between free and bound names. The free names
of a process, $\freenames{P}$, may be calculated recursively as
follows:

\begin{mathpar}
\freenames{\pzero} := \emptyset
  \and \\
  \freenames{x?(y).P} := \{ x \} \cup (\freenames{P} \setminus \{ y \})
  \and 
  \freenames{x!\langle P \rangle} := \{ x \} \cup \{ P \} 
  \and \\
  \freenames{P|Q} := \freenames{P} \cup \freenames{Q}
  \and \\
  \freenames{@{x}} := \{ x \}
\end{mathpar}

$\pi$
$\quotep{\pi}$

$\freenames{-} : \pi \to \mathcal{P}(\quotep{\pi})$

\begin{eqnarray*}
  \freenames{\pzero} & := & \emptyset \\
  \freenames{x?(y).P} & := & \{ x \} \cup (\freenames{P} \setminus \{ y \}) \\
  \freenames{x!\langle P \rangle} & := & \{ x \} \cup \{ P \} \\
  \freenames{P|Q} & := & \freenames{P} \cup \freenames{Q} \\
  \freenames{\dropn{x}} & := & \{ x \}
\end{eqnarray*}

The bound names of a process, $\boundnames{P}$, are those names occurring in $P$
that are not free. For example, in $x?(y).0$, the name $x$ is free, while $y$ is bound.

\begin{mathpar}
  \inferrule* [lab=monoidal-laws] {} { P|Q \equiv Q|P \and P|0 \equiv P \and P|(Q|R) \equiv (P|Q)|R }
\end{mathpar}

\begin{mathpar}
  \inferrule* [lab=alpha-equivalence] {} { (x)P \equiv (y)P\{y/x\} \and y \not\in \freenames{P} }
\end{mathpar}

\begin{definition}
Then two processes, $P,Q$, are alpha-equivalent if $P = Q\{\vec{y}/\vec{x}\}$ for
some $\vec{x} \in \boundnames{Q},\vec{y} \in \boundnames{P}$, where $Q\{\vec{y}/\vec{x}\}$
denotes the capture-avoiding substitution of $\vec{y}$ for $\vec{x}$ in $Q$.
\end{definition}

\begin{definition}
  The {\em structural congruence} \cite{SangiorgiWalker} , $\equiv$,
  between processes is the least congruence containing
  alpha-equivalence, satisfying the abelian monoid laws
  (associativity, commutativity and $\pzero$ as identity) for parallel
  composition $|$ and for summation $+$.
\end{definition}

\subsection{Name equivalence}

We take name equivalence, written $\nameeq$, to be the smallest
equivalence relation generated by the following rules.

\begin{mathpar}
\inferrule*[lab=Quote-drop]
{ }
{ \quotep{@{x}} \nameeq x }

\inferrule*[lab=Struct-equiv]
{ P \scong Q }
{ \quotep{P} \nameeq \quotep{Q} }
\end{mathpar}

The astute reader will have noticed that the mutual recursion of names
and processes imposes a mutual recursion on alpha-equivalence and
structural equivalence via name-equivalence. Fortunately, all of this
works out pleasantly and we may calculate in the natural way, free of
concern. The reader interested in the details is referred to the
appendix \ref{appendix:rho_details}.

\subsection{Substitution}

We use $\Proc$ for the set of processes, $\QProc$ for the set of
names, and $\id{\{}\vec{y} / \vec{x} \id{\}}$ to denote partial maps,
$s : \QProc \rightarrow \QProc$. A map, $s$ lifts, uniquely, to a map
on process terms, $\widehat{s} : \Proc \rightarrow \Proc$ by the
following equations.

\begin{mathpar}
  (0) \psubstp{Q}{P} := 0 \\
  (R \juxtap S) \psubstp{Q}{P}
  :=    
  (R)\psubstp{Q}{P} \juxtap (S) \psubstp{Q}{P} \\
  (x?(y).R) \psubstp{Q}{P}    
  :=    
  (x)\substp{Q}{P} (z)\concat( (R \psubstn{z}{y}) \psubstp{Q}{P} ) \\
  (\lift{x}{R}) \psubstp{Q}{P}  
  :=
  \lift{(x)\substp{Q}{P}}{ R \psubstp{Q}{P} } \\
%   (\dropn{x})  \psubstp{Q}{P}       
%   := 
%   \left\{ 
%     \begin{array}{ccc} 
%       \dropn{\quotep{Q}} & & x \nameeq \quotep{P} \\
%       \dropn{x} & & otherwise \\
%     \end{array}
%   \right. 
  (\dropn{x})  \psubstp{Q}{P}       
  := 
  \left\{ 
    \begin{array}{ccc} 
      Q & & x \nameeq \quotep{P} \\
      \dropn{x} & & otherwise \\
    \end{array}
  \right.
\end{mathpar}
 

where

\begin{eqnarray}
  (x)\id{\{} \lpquote Q \rpquote / \lpquote P \rpquote \id{\}}            = 
  \left\{ 
    \begin{array}{ccc}
      \lpquote Q \rpquote & & x \nameeq \lpquote P \rpquote \\
      x & & otherwise \\
    \end{array}
  \right. \nonumber
\end{eqnarray}

and $z$ is chosen distinct from $\quotep{P}$, $\quotep{Q}$, the free
names in $Q$, and all the names in $R$. Our $\alpha$-equivalence will
be built in the standard way from this substitution.

\begin{remark}\label{rem:no_self_referential_names}
  One consequence of these definitions is that $\forall P. \quotep{P}
  \not\in \freenames{P}$.
\end{remark}

\subsection{ Dynamic quote: an example }

Anticipating something of what's to come, consider applying the
substitution, $\widehat{\id{\{}u / z \id{\}}}$, to the following pair
of processes, $\lift{w}{y!(z)}$ and $w[ \lpquote y!(z) \rpquote ]$.

\begin{eqnarray}
	\lift{w}{y!(z)}\widehat{\id{\{}u / z \id{\}}}
		& = &
		\lift{w}{y!(u)} \nonumber\\
	w[ \lpquote y!(z) \rpquote ] \widehat{ \id{\{}u / z \id{\}} }
		& = &
		w[ \lpquote y!(z) \rpquote ] \nonumber
\end{eqnarray}

Because the body of the process between quotes is impervious to
substitution, we get radically different answers. In fact, by
examining the first process in an input context,
e.g. $x?(z).\lift{w}{y!(z)}$, we see that the process under the lift
operator may be shaped by prefixed inputs binding a name inside it. In
this sense, the lift operator will be seen as a way to dynamically
construct processes before reifying them as names.

Finally equipped with these standard features we can present the
dynamics of the calculus.

\subsubsection{Operational semantics} 

Finally, we introduce the computational dynamics. What marks these
algebras as distinct from other more traditionally studied algebraic
structures, e.g. vector spaces or polynomial rings, is the manner in
which dynamics is captured. In traditional structures, dynamics is typically
expressed through morphisms between such structures, as in linear maps
between vector spaces or morphisms between rings. In algebras
associated with the semantics of computation, the dynamics is
expressed as part of the algebraic structure itself, through a
reduction reduction relation typically denoted by $\red$. Below, we
give a recursive presentation of this relation for the calculus used
in the encoding.

$\red \subseteq \pi \times \pi$
$\red : \pi \to \mathcal{P}(\pi)$

\begin{mathpar}
  \inferrule* [lab=Comm] { \textsf{match}( x_{src}, x_{trgt} ) } { x_{trgt}?(y)P \; | \; x_{src}!\langle {Q} \rangle \red P\{\quotep{Q}/y}\} }
  \and \\
  \inferrule* [lab=Par] {{P} \red {P}'} {{{P} | {Q}} \red {{P}' | {Q}}}
  \and
  \inferrule* [lab=Equiv]{{{P} \scong {P}'} \andalso {{P}' \red {Q}'} \andalso {{Q}' \scong {Q}}}{{P} \red {Q}}
\end{mathpar}

\begin{eqnarray*}
  match_{\equiv} (\quotep{P},\quotep{Q}) & := & P \equiv Q \\
  match_{\dagger}(\quotep{P},\quotep{Q}) & := & \forall R. P|Q \red^{*} R => R \red^{*} 0 \\
  match_{K}(\quotep{P},\quotep{Q}) & := & K \mbox{ for some context } K
\end{eqnarray*}

$u?(x)P | u!\langle Q \rangle \red P\{\quotep{Q}/x\}$

%We write $\wred$ for $\red^*$, and $P\red$ if $\exists Q $ such that $ P \red Q$.
We write $P\red$ if $\exists Q $ such that $ P \red Q$ and $P\not\red$, otherwise.

\section{Replication}

As mentioned before, it is known that replication (and hence
recursion) can be implemented in a higher-order process algebra
\cite{SangiorgiWalker}. As our first example of calculation with the
machinery thus far presented we give the construction explicitly in
the {\rhoc}.

\begin{eqnarray}
	D_{x} & := & \prefix{x}{y}{(\binpar{\outputp{x}{y}}{@{y}})} \nonumber\\
	\bangp_{x}{P} & := & \binpar{{x}!\langle{\binpar{D_{x}}{P}}\rangle}{D_{x}} \nonumber
\end{eqnarray}

\begin{eqnarray}
	\bangp_{x}{P} & & \nonumber\\
	=
	& {x}!\langle{(\prefix{x}{y}{(\outputp{x}{y} | @{y})) | P}}\rangle 
	      | \prefix{x}{y}{(\outputp{x}{y} | @{y})} & \nonumber\\
	\red
	& (\outputp{x}{y} | @{y})\substn{\quotep{(\prefix{x}{y}{(@{y} | \outputp{x}{y})) | P}}}{y} & \nonumber\\
	=
	& \outputp{x}{\quotep{(\prefix{x}{y}{(\outputp{x}{y} | @{y})) | P}}}
	  | {(\prefix{x}{y}{(\outputp{x}{y} | @{y})) | P}} & \nonumber\\
	\red
	& \ldots & \nonumber\\
	\red^*
	& P | P | \ldots & \nonumber
\end{eqnarray}

Of course, this encoding, as an implementation, runs away, unfolding
$\bangp{P}$ eagerly. A lazier and more implementable replication
operator, restricted to input-guarded processes, may be obtained as follows.

\begin{eqnarray}
\bangp{\prefix{u}{v}{P}} 
	:= 
	\binpar{\lift{x}{\prefix{u}{v}{(\binpar{D(x)}{P})}}}{D(x)} \nonumber
\end{eqnarray}

\begin{remark}
  Note that the lazier definition still does not deal with summation
  or mixed summation (i.e. sums over input and output). The reader is
  invited to construct definitions of replication that deal with these
  features. 

  Further, the definitions are parameterized in a name, $x$. Can you,
  gentle reader, make a definition that eliminates this parameter and
  guarantees no accidental interaction between the replication
  machinery and the process being replicated -- i.e. no accidental
  sharing of names used by the process to get its work done and the
  name(s) used by the replication to effect copying. This latter
  revision of the definition of replication is crucial to obtaining
  the expected identity $!!P \sim !P$.
\end{remark}

\begin{remark}\label{rem:paradoxical_combinator}
  The reader familiar with the lambda calculus will have noticed the
  similarity between $D$ and the paradoxical combinator.

  [Ed. note: the existence of this seems to suggest we have to be more
  restrictive on the set of processes and names we admit if we are to
  support no-cloning.]
\end{remark}

\subsubsection{Bisimulation}

The computational dynamics gives rise to another kind of equivalence,
the equivalence of computational behavior. As previously mentioned
this is typically captured \emph{via} some form of bisimulation.

% The notion we use in this paper is weak barbed bisimulation
% \cite{milner91polyadicpi}.

The notion we use in this paper is derived from weak barbed
bisimulation \cite{milner91polyadicpi}. 

\begin{definition}
An \emph{observation relation}, $\downarrow_{\mathcal N}$, over a set
of names, $\mathcal N$, is the smallest relation satisfying the rules
below.

\infrule[Out-barb]{y \in {\mathcal N}, \; x \nameeq y}
		  {\outputp{x}{v} \downarrow_{\mathcal N} x}
\infrule[Par-barb]{\mbox{$P\downarrow_{\mathcal N} x$ or $Q\downarrow_{\mathcal N} x$}}
		  {\binpar{P}{Q} \downarrow_{\mathcal N} x}

We write $P \Downarrow_{\mathcal N} x$ if there is $Q$ such that 
$P \wred Q$ and $Q \downarrow_{\mathcal N} x$.
\end{definition}

\begin{definition}
%\label{def.bbisim}
An  ${\mathcal N}$-\emph{barbed bisimulation} over a set of names, ${\mathcal N}$, is a symmetric binary relation 
${\mathcal S}_{\mathcal N}$ between agents such that $P\rel{S}_{\mathcal N}Q$ implies:
\begin{enumerate}
\item If $P \red P'$ then $Q \wred Q'$ and $P'\rel{S}_{\mathcal N} Q'$.
\item If $P\downarrow_{\mathcal N} x$, then $Q\Downarrow_{\mathcal N} x$.
\end{enumerate}
$P$ is ${\mathcal N}$-barbed bisimilar to $Q$, written
$P \wbbisim_{\mathcal N} Q$, if $P \rel{S}_{\mathcal N} Q$ for some ${\mathcal N}$-barbed bisimulation ${\mathcal S}_{\mathcal N}$.
\end{definition}

$\mathcal{R} \subseteq \pi \times \pi$

$P \mathcal{R} Q => \forall P'. P \red P' \Rightarrow \exists Q'. Q \red Q', P' \mathcal{R} Q'$

$P \vdash x \Rightarrow Q \vdash x$

\begin{mathpar}
  \inferrule*[lab=Out-barb]{x \nameeq y}{{y}!\langle{Q}\rangle \vdash x}
  \and
  \inferrule*[lab=Par-barb]{\mbox{$P\vdash x$ or $Q\vdash x$}}{\binpar{P}{Q} \vdash x}
\end{mathpar}

\subsubsection{Contexts}

One of the principle advantages of computational calculi like the
$\pi$-calculus is a well-defined notion of context,
contextual-equivalence and a correlation between
contextual-equivalence and notions of bisimulation. The notion of
context allows the decomposition of a process into (sub-)process and
its syntactic environment, its context. Thus, a context may be
thought of as a process with a ``hole'' (written $\Box$) in it. The
application of a context $M$ to a process $P$, written $M[P]$, is
tantamount to filling the hole in $M$ with $P$. In this paper we do
not need the full weight of this theory, but do make use of the notion
of context in the proof the main theorem. 

\begin{mathpar}
  \inferrule* [lab=summation] {} {{M_{M},M_{N}} \bc \Box \;|\; x.M_{A} \;|\; M_{M}+M_{N}}
  \and
  \inferrule* [lab=agent] {} {{M_{A}} \bc (\vec{x})M_{P} \;| \; \clift{P_0,\ldots,M_{P},\ldots,P_N}}
  \and \\
  \inferrule* [lab=process] {} {{M_{P}} \bc M_{N} \;| \;P|M_{P} }
\end{mathpar} 

\begin{mathpar}
  \inferrule* [lab=sychronization] {} {M_{N} \bc \Box \;|\; x?M_{F} \;|\; x!M_{C}}
  \and
  \inferrule* [lab=abstraction] {} {{M_{F}} \bc (x)M_{P} }
  \and
  \inferrule* [lab=concretion] {} {{M_{C}} \bc \langle M_{P} \rangle }
  \and \\
  \inferrule* [lab=process] {} {{M_{P}} \bc M_{N} \;| \;P|M_{P} }
\end{mathpar}

\begin{definition}[contextual application] Given a context $M$, and
  process $P$, we define the \emph{contextual application}, $M[P] :=
  M\{P/\Box\}$. That is, the contextual application of M to P is the
  substitution of $P$ for $\Box$ in $M$.
\end{definition}

$\meaningof{-} : L \to \mathcal{P}(\pi)$

\begin{mathpar}
  \inferrule* [lab=collection] {} {\meaningof{true} = \pi, \and \meaningof{~E} = \pi \setminus \meaningof{E}, \and \meaningof{E_{1} \& E_{2}} = \meaningof{E_{1}} \cap \meaningof{E_{2}}}
\end{mathpar}

\begin{mathpar}
  \inferrule* [lab=structure] {} {\meaningof{0} = \{ P \in \pi | P \equiv 0 \}, \and \\ \meaningof{E_1 | E_2} = \{ P \in \pi | P \equiv P_{1} | P_{2}, P_{1} \in \meaningof{E_{1}}, P_{2} \in \meaningof{E_2}\} }
\end{mathpar}

\begin{mathpar}
 \inferrule* [lab=behavior] {} {\meaningof{\langle a?b \rangle E} = \{ P \in \pi | P \equiv Q | u?(y)P', \\ \and \\\\ \and \\ \;\;\; u \in \meaningof{a}, \forall z.P'\{z/y\} \in \meaningof{E\{z/b\}}\}, \and \\ \meaningof{a!E} = \{ P \in \pi | P \equiv Q | x!\langle P' \rangle, x \in \meaningof{a} P' \in \meaningof{E}\} }
\end{mathpar}

\begin{mathpar}
 \inferrule* [lab=nominal] {} {\meaningof{\quotep{E}} = \{ \quotep{P} \in \quotep{\pi} | P \in \meaningof{E} \}, \and \meaningof{\quotep{P}} = \{ \quotep{Q} \in \quotep{\pi} | P \equiv Q \} \and \\ \meaningof{@\quotep{E}} = \{ P \in \pi | P \equiv @x, x \in \meaningof{E} \}}
\end{mathpar}

\begin{eqnarray*}
  \\
  \meaningof{-} : TS \to ST
\end{eqnarray*}

\begin{eqnarray*}
  \\
  L : TS \to ST
\end{eqnarray*}

\begin{eqnarray*}
  \\
  P \models E \iff P \in \meaningof{E}
\end{eqnarray*}

\begin{eqnarray*}
  P \approx_{L} Q \iff \forall E \in L. P \models E \iff Q \models E
\end{eqnarray*}

\begin{eqnarray*}
  P \approx_{K} Q
\end{eqnarray*}

\begin{eqnarray*}
  P \approx Q
\end{eqnarray*}

$\approx_{K} = \approx = \approx_{L}$

\subsubsection{Contextual duality}

Note that contexts extend the quotation operation to a family of
operations from processes to names. Given a context, $M$, we can
define a \emph{nominal context}, $\quotep{M}$ by $\quotep{M}[P] :=
\quotep{M[P]}$. To foreshadow what is to come we observe that these
operations enjoy a duality with processes very much like the duality
between vectors and maps from vectors to scalars.

Further, because the calculus is essentially higher-order, we have a
correspondence between contexts and processes. More specifically,
given a name $x$ and a context $M$ we can construct $M^{*}_{x}$ such
that 

\begin{mathpar}
  M^{*}_{x} | \lift{x}{P} \red M[P]
\end{mathpar}

namely,

\begin{mathpar}
  M^{*}_{x} := x?(u).M[\dropn{u}]
\end{mathpar}

The dependence of $M^{*}_{x}$ on a name makes it an abstraction, 

\begin{mathpar}
  M^{*} := (x)x?(u).M[\dropn{u}]
\end{mathpar}

\subsection{Additional notation}

It will sometimes be convenient to denote the process a name
quotes. We already have the notation $x = \quotep{P}$, but it will be
convenient to introduce an alternate notation, $\procn{x}$, when we
want to emphasize the connection to the use of the name. Note that, by
virtue of name equivalence, $\quotep{\procn{x}} \nameeq x$; so, the
notation is consistent with previous definitions.

Further, because names have structure it is possible to effect
substitutions on the basis of that structure. This means we need to
upgrade our notation for substitutions, which we accomplish by
adapting comprehension notation. Thus,

\begin{mathpar}
  P\{ y / x : x \in S \}
\end{mathpar}

is interpreted to mean the process derived from P by replacing (in a
capture-avoiding manner) each occurrence of $x$ in $S$ by $y$. For example,

\begin{mathpar}
  P\{ \quotep{\procn{x}|\procn{x}} / x : x \in \freenames{P} \}
\end{mathpar}

will replace each (occurrence) of a free name $x$ in $P$ by
$\quotep{\procn{x}|\procn{x}}$.

Also, we will avail ourselves of the notation $x^{L}$ and $x^{R}$ to
denote injections of a name into disjoint copies of the name
space. There are numerous ways to accomplish this. One example can be
found in \cite{MeredithR05}. This notation overloads to vectors of
names: $\vec{x}^{\pi} := (x_{i}^{\pi} \; : \; 0 \leq i < |\vec{x}| )$ where $\pi \in \{L,R\}$.

We also use $P^{\Box} := P|\Box$.

In \cite{MeredithR05} an interpretation of the new operator is
given. It turns out that there are several possible interpretations
all enjoying the requisite algebraic properties of the operator (see
\cite{milner91polyadicpi}). We will therefore make liberal use of
$(\nu\; \vec{x})P$.

% subsection the_syntax_and_semantics_of_the_notation_system (end)   

\input{qm2pi.qmops} 

\input{qm2pi.sterngerlach} 

\input{qm2pi.metric} 

% section concurrent_process_calculi (end)

%\input{qm2pi.proofsketch}

% section proof sketch (end)

%\input{qm2pi.slviaknots} 

% section spatial logic via knots (end)

\input{qm2pi.conclusion}

% section conclusion (end)

%\input{qm2pi.dtcodes} 

% section wiring algorithm (end)

\input{qm2pi.ack} 

% section acknowledgments (end)

\newpage


\bibliographystyle{plain}   
\bibliography{../../biblios/main.bib}

\input{qm2pi.rhodetails}

\end{document}

 

% section wiring algorithm (end)

\documentclass[12pt]{llncs}
%\documentclass{jktr}

\usepackage[pdftex]{hyperref}                   
\usepackage {listings}
\usepackage {mathpartir}
\usepackage{bcprules}
%\usepackage{listings}
                       
\usepackage{graphicx} 
%\usepackage[margins=2.5cm,nohead,nofoot]{geometry}
%\usepackage{geometry}
\usepackage{amsfonts}
\usepackage{amstext}
\usepackage{latexsym}
\usepackage{amssymb}
\usepackage{color}


%\include{myPreamble}
\include{qm2pi.local} 

%\ifpdf
%\usepackage[pdftex]{graphicx}
%\else
%\usepackage{graphicx}
%\fi

 % \ifpdf
%  \usepackage{pdfsync}
%  \if


%\title{Brief Article}
%\author{David F. Snyder}
%\author{L.G. Meredith}

%\address{Dept. of Math., Texas State University--San Marcos, San Marcos, TX 78666}
       
\pagestyle{empty}


\begin{document}

\lstset{language=[Objective]Caml,frame=shadowbox}

\input{qm2pi.front}

% section front matter (end)

\input{qm2pi.intro} 
 
% section introduction (end)

% \input{qm2pi.knotations} 

% section notation (end)

\input{qm2pi.process.calculi} 

% section concurrent_process_calculi_and_spatial_logics_ (end)
    
%\input{qm2pi.knots2pi} 

%\input{qm2pi.trefoil} 

%\input{qm2pi.mainthm} 

% subsection basic_interpretation (end)

%\input{qm2pi.rho.presentation} 
\subsection{The syntax and semantics of the notation system}\label{sub:the_syntax_and_semantics_of_the_notation_system} % (fold)

We now summarize a technical presentation of the calculus that
embodies our theory of dynamics. The typical presentation of such a
calculus follows the style of giving generators and relations on
them. The grammar, below, describing term constructors, freely
generates the set of processes, $\Proc$. This set is then quotiented
by a relation known as structural congruence and it is over this set
that the notion of dynamics is expressed. This presentation is
essentially that of \cite{MeredithR05} with the addition of
polyadicity and summation. For readability we have relegated some of
the technical subtleties to an appendix.

\subsubsection{Process grammar}\label{subsub:process_grammar}

\begin{mathpar}
  \inferrule* [lab=synchronization] {} {{M} \bc \pzero \;|\; x?F \;|\; x!C }
  \and
  \inferrule* [lab=abstraction] {} {{F} \bc (x)P}
  \and
  \inferrule* [lab=concretion] {} {{C} \bc \langle Q \rangle}
  \and
  \inferrule* [lab=process] {} {{P,Q} \bc M \;| \;P|Q \;|\; @{x}}
  \and
  \inferrule* [lab=name] {} {{x} \bc \quotep{P}}
\end{mathpar} 

Note that $\vec{x}$ (resp. $\vec{P}$) denotes a vector of names
(resp. processes) of length $|\vec{x}|$ (resp. $|\vec{P}|$). We adopt
the following useful abbreviations.

\begin{mathpar}
   x?(\vec{y}).P := x.(\vec{y})P \and  x\clift{\vec{P}} := x.\clift{\vec{P}}
   \and x!(y) := \lift{x}{\dropn{y}}
   \and \Pi_{i=0}^{n-1}P_i := P_0 | \ldots | P_{n-1}
\end{mathpar}

\subsubsection{Structural congruence}

\paragraph{Free and bound names and alpha-equivalence.} At the
core of structural equivalence is alpha-equivalence which identifies
process that are the same up to a change of variable. Formally, we
recognize the distinction between free and bound names. The free names
of a process, $\freenames{P}$, may be calculated recursively as
follows:

\begin{mathpar}
\freenames{\pzero} := \emptyset
  \and \\
  \freenames{x?(y).P} := \{ x \} \cup (\freenames{P} \setminus \{ y \})
  \and 
  \freenames{x!\langle P \rangle} := \{ x \} \cup \{ P \} 
  \and \\
  \freenames{P|Q} := \freenames{P} \cup \freenames{Q}
  \and \\
  \freenames{@{x}} := \{ x \}
\end{mathpar}

$\pi$
$\quotep{\pi}$

$\freenames{-} : \pi \to \mathcal{P}(\quotep{\pi})$

\begin{eqnarray*}
  \freenames{\pzero} & := & \emptyset \\
  \freenames{x?(y).P} & := & \{ x \} \cup (\freenames{P} \setminus \{ y \}) \\
  \freenames{x!\langle P \rangle} & := & \{ x \} \cup \{ P \} \\
  \freenames{P|Q} & := & \freenames{P} \cup \freenames{Q} \\
  \freenames{\dropn{x}} & := & \{ x \}
\end{eqnarray*}

The bound names of a process, $\boundnames{P}$, are those names occurring in $P$
that are not free. For example, in $x?(y).0$, the name $x$ is free, while $y$ is bound.

\begin{mathpar}
  \inferrule* [lab=monoidal-laws] {} { P|Q \equiv Q|P \and P|0 \equiv P \and P|(Q|R) \equiv (P|Q)|R }
\end{mathpar}

\begin{mathpar}
  \inferrule* [lab=alpha-equivalence] {} { (x)P \equiv (y)P\{y/x\} \and y \not\in \freenames{P} }
\end{mathpar}

\begin{definition}
Then two processes, $P,Q$, are alpha-equivalent if $P = Q\{\vec{y}/\vec{x}\}$ for
some $\vec{x} \in \boundnames{Q},\vec{y} \in \boundnames{P}$, where $Q\{\vec{y}/\vec{x}\}$
denotes the capture-avoiding substitution of $\vec{y}$ for $\vec{x}$ in $Q$.
\end{definition}

\begin{definition}
  The {\em structural congruence} \cite{SangiorgiWalker} , $\equiv$,
  between processes is the least congruence containing
  alpha-equivalence, satisfying the abelian monoid laws
  (associativity, commutativity and $\pzero$ as identity) for parallel
  composition $|$ and for summation $+$.
\end{definition}

\subsection{Name equivalence}

We take name equivalence, written $\nameeq$, to be the smallest
equivalence relation generated by the following rules.

\begin{mathpar}
\inferrule*[lab=Quote-drop]
{ }
{ \quotep{@{x}} \nameeq x }

\inferrule*[lab=Struct-equiv]
{ P \scong Q }
{ \quotep{P} \nameeq \quotep{Q} }
\end{mathpar}

The astute reader will have noticed that the mutual recursion of names
and processes imposes a mutual recursion on alpha-equivalence and
structural equivalence via name-equivalence. Fortunately, all of this
works out pleasantly and we may calculate in the natural way, free of
concern. The reader interested in the details is referred to the
appendix \ref{appendix:rho_details}.

\subsection{Substitution}

We use $\Proc$ for the set of processes, $\QProc$ for the set of
names, and $\id{\{}\vec{y} / \vec{x} \id{\}}$ to denote partial maps,
$s : \QProc \rightarrow \QProc$. A map, $s$ lifts, uniquely, to a map
on process terms, $\widehat{s} : \Proc \rightarrow \Proc$ by the
following equations.

\begin{mathpar}
  (0) \psubstp{Q}{P} := 0 \\
  (R \juxtap S) \psubstp{Q}{P}
  :=    
  (R)\psubstp{Q}{P} \juxtap (S) \psubstp{Q}{P} \\
  (x?(y).R) \psubstp{Q}{P}    
  :=    
  (x)\substp{Q}{P} (z)\concat( (R \psubstn{z}{y}) \psubstp{Q}{P} ) \\
  (\lift{x}{R}) \psubstp{Q}{P}  
  :=
  \lift{(x)\substp{Q}{P}}{ R \psubstp{Q}{P} } \\
%   (\dropn{x})  \psubstp{Q}{P}       
%   := 
%   \left\{ 
%     \begin{array}{ccc} 
%       \dropn{\quotep{Q}} & & x \nameeq \quotep{P} \\
%       \dropn{x} & & otherwise \\
%     \end{array}
%   \right. 
  (\dropn{x})  \psubstp{Q}{P}       
  := 
  \left\{ 
    \begin{array}{ccc} 
      Q & & x \nameeq \quotep{P} \\
      \dropn{x} & & otherwise \\
    \end{array}
  \right.
\end{mathpar}
 

where

\begin{eqnarray}
  (x)\id{\{} \lpquote Q \rpquote / \lpquote P \rpquote \id{\}}            = 
  \left\{ 
    \begin{array}{ccc}
      \lpquote Q \rpquote & & x \nameeq \lpquote P \rpquote \\
      x & & otherwise \\
    \end{array}
  \right. \nonumber
\end{eqnarray}

and $z$ is chosen distinct from $\quotep{P}$, $\quotep{Q}$, the free
names in $Q$, and all the names in $R$. Our $\alpha$-equivalence will
be built in the standard way from this substitution.

\begin{remark}\label{rem:no_self_referential_names}
  One consequence of these definitions is that $\forall P. \quotep{P}
  \not\in \freenames{P}$.
\end{remark}

\subsection{ Dynamic quote: an example }

Anticipating something of what's to come, consider applying the
substitution, $\widehat{\id{\{}u / z \id{\}}}$, to the following pair
of processes, $\lift{w}{y!(z)}$ and $w[ \lpquote y!(z) \rpquote ]$.

\begin{eqnarray}
	\lift{w}{y!(z)}\widehat{\id{\{}u / z \id{\}}}
		& = &
		\lift{w}{y!(u)} \nonumber\\
	w[ \lpquote y!(z) \rpquote ] \widehat{ \id{\{}u / z \id{\}} }
		& = &
		w[ \lpquote y!(z) \rpquote ] \nonumber
\end{eqnarray}

Because the body of the process between quotes is impervious to
substitution, we get radically different answers. In fact, by
examining the first process in an input context,
e.g. $x?(z).\lift{w}{y!(z)}$, we see that the process under the lift
operator may be shaped by prefixed inputs binding a name inside it. In
this sense, the lift operator will be seen as a way to dynamically
construct processes before reifying them as names.

Finally equipped with these standard features we can present the
dynamics of the calculus.

\subsubsection{Operational semantics} 

Finally, we introduce the computational dynamics. What marks these
algebras as distinct from other more traditionally studied algebraic
structures, e.g. vector spaces or polynomial rings, is the manner in
which dynamics is captured. In traditional structures, dynamics is typically
expressed through morphisms between such structures, as in linear maps
between vector spaces or morphisms between rings. In algebras
associated with the semantics of computation, the dynamics is
expressed as part of the algebraic structure itself, through a
reduction reduction relation typically denoted by $\red$. Below, we
give a recursive presentation of this relation for the calculus used
in the encoding.

$\red \subseteq \pi \times \pi$
$\red : \pi \to \mathcal{P}(\pi)$

\begin{mathpar}
  \inferrule* [lab=Comm] { \textsf{match}( x_{src}, x_{trgt} ) } { x_{trgt}?(y)P \; | \; x_{src}!\langle {Q} \rangle \red P\{\quotep{Q}/y}\} }
  \and \\
  \inferrule* [lab=Par] {{P} \red {P}'} {{{P} | {Q}} \red {{P}' | {Q}}}
  \and
  \inferrule* [lab=Equiv]{{{P} \scong {P}'} \andalso {{P}' \red {Q}'} \andalso {{Q}' \scong {Q}}}{{P} \red {Q}}
\end{mathpar}

\begin{eqnarray*}
  match_{\equiv} (\quotep{P},\quotep{Q}) & := & P \equiv Q \\
  match_{\dagger}(\quotep{P},\quotep{Q}) & := & \forall R. P|Q \red^{*} R => R \red^{*} 0 \\
  match_{K}(\quotep{P},\quotep{Q}) & := & K \mbox{ for some context } K
\end{eqnarray*}

$u?(x)P | u!\langle Q \rangle \red P\{\quotep{Q}/x\}$

%We write $\wred$ for $\red^*$, and $P\red$ if $\exists Q $ such that $ P \red Q$.
We write $P\red$ if $\exists Q $ such that $ P \red Q$ and $P\not\red$, otherwise.

\section{Replication}

As mentioned before, it is known that replication (and hence
recursion) can be implemented in a higher-order process algebra
\cite{SangiorgiWalker}. As our first example of calculation with the
machinery thus far presented we give the construction explicitly in
the {\rhoc}.

\begin{eqnarray}
	D_{x} & := & \prefix{x}{y}{(\binpar{\outputp{x}{y}}{@{y}})} \nonumber\\
	\bangp_{x}{P} & := & \binpar{{x}!\langle{\binpar{D_{x}}{P}}\rangle}{D_{x}} \nonumber
\end{eqnarray}

\begin{eqnarray}
	\bangp_{x}{P} & & \nonumber\\
	=
	& {x}!\langle{(\prefix{x}{y}{(\outputp{x}{y} | @{y})) | P}}\rangle 
	      | \prefix{x}{y}{(\outputp{x}{y} | @{y})} & \nonumber\\
	\red
	& (\outputp{x}{y} | @{y})\substn{\quotep{(\prefix{x}{y}{(@{y} | \outputp{x}{y})) | P}}}{y} & \nonumber\\
	=
	& \outputp{x}{\quotep{(\prefix{x}{y}{(\outputp{x}{y} | @{y})) | P}}}
	  | {(\prefix{x}{y}{(\outputp{x}{y} | @{y})) | P}} & \nonumber\\
	\red
	& \ldots & \nonumber\\
	\red^*
	& P | P | \ldots & \nonumber
\end{eqnarray}

Of course, this encoding, as an implementation, runs away, unfolding
$\bangp{P}$ eagerly. A lazier and more implementable replication
operator, restricted to input-guarded processes, may be obtained as follows.

\begin{eqnarray}
\bangp{\prefix{u}{v}{P}} 
	:= 
	\binpar{\lift{x}{\prefix{u}{v}{(\binpar{D(x)}{P})}}}{D(x)} \nonumber
\end{eqnarray}

\begin{remark}
  Note that the lazier definition still does not deal with summation
  or mixed summation (i.e. sums over input and output). The reader is
  invited to construct definitions of replication that deal with these
  features. 

  Further, the definitions are parameterized in a name, $x$. Can you,
  gentle reader, make a definition that eliminates this parameter and
  guarantees no accidental interaction between the replication
  machinery and the process being replicated -- i.e. no accidental
  sharing of names used by the process to get its work done and the
  name(s) used by the replication to effect copying. This latter
  revision of the definition of replication is crucial to obtaining
  the expected identity $!!P \sim !P$.
\end{remark}

\begin{remark}\label{rem:paradoxical_combinator}
  The reader familiar with the lambda calculus will have noticed the
  similarity between $D$ and the paradoxical combinator.

  [Ed. note: the existence of this seems to suggest we have to be more
  restrictive on the set of processes and names we admit if we are to
  support no-cloning.]
\end{remark}

\subsubsection{Bisimulation}

The computational dynamics gives rise to another kind of equivalence,
the equivalence of computational behavior. As previously mentioned
this is typically captured \emph{via} some form of bisimulation.

% The notion we use in this paper is weak barbed bisimulation
% \cite{milner91polyadicpi}.

The notion we use in this paper is derived from weak barbed
bisimulation \cite{milner91polyadicpi}. 

\begin{definition}
An \emph{observation relation}, $\downarrow_{\mathcal N}$, over a set
of names, $\mathcal N$, is the smallest relation satisfying the rules
below.

\infrule[Out-barb]{y \in {\mathcal N}, \; x \nameeq y}
		  {\outputp{x}{v} \downarrow_{\mathcal N} x}
\infrule[Par-barb]{\mbox{$P\downarrow_{\mathcal N} x$ or $Q\downarrow_{\mathcal N} x$}}
		  {\binpar{P}{Q} \downarrow_{\mathcal N} x}

We write $P \Downarrow_{\mathcal N} x$ if there is $Q$ such that 
$P \wred Q$ and $Q \downarrow_{\mathcal N} x$.
\end{definition}

\begin{definition}
%\label{def.bbisim}
An  ${\mathcal N}$-\emph{barbed bisimulation} over a set of names, ${\mathcal N}$, is a symmetric binary relation 
${\mathcal S}_{\mathcal N}$ between agents such that $P\rel{S}_{\mathcal N}Q$ implies:
\begin{enumerate}
\item If $P \red P'$ then $Q \wred Q'$ and $P'\rel{S}_{\mathcal N} Q'$.
\item If $P\downarrow_{\mathcal N} x$, then $Q\Downarrow_{\mathcal N} x$.
\end{enumerate}
$P$ is ${\mathcal N}$-barbed bisimilar to $Q$, written
$P \wbbisim_{\mathcal N} Q$, if $P \rel{S}_{\mathcal N} Q$ for some ${\mathcal N}$-barbed bisimulation ${\mathcal S}_{\mathcal N}$.
\end{definition}

$\mathcal{R} \subseteq \pi \times \pi$

$P \mathcal{R} Q => \forall P'. P \red P' \Rightarrow \exists Q'. Q \red Q', P' \mathcal{R} Q'$

$P \vdash x \Rightarrow Q \vdash x$

\begin{mathpar}
  \inferrule*[lab=Out-barb]{x \nameeq y}{{y}!\langle{Q}\rangle \vdash x}
  \and
  \inferrule*[lab=Par-barb]{\mbox{$P\vdash x$ or $Q\vdash x$}}{\binpar{P}{Q} \vdash x}
\end{mathpar}

\subsubsection{Contexts}

One of the principle advantages of computational calculi like the
$\pi$-calculus is a well-defined notion of context,
contextual-equivalence and a correlation between
contextual-equivalence and notions of bisimulation. The notion of
context allows the decomposition of a process into (sub-)process and
its syntactic environment, its context. Thus, a context may be
thought of as a process with a ``hole'' (written $\Box$) in it. The
application of a context $M$ to a process $P$, written $M[P]$, is
tantamount to filling the hole in $M$ with $P$. In this paper we do
not need the full weight of this theory, but do make use of the notion
of context in the proof the main theorem. 

\begin{mathpar}
  \inferrule* [lab=summation] {} {{M_{M},M_{N}} \bc \Box \;|\; x.M_{A} \;|\; M_{M}+M_{N}}
  \and
  \inferrule* [lab=agent] {} {{M_{A}} \bc (\vec{x})M_{P} \;| \; \clift{P_0,\ldots,M_{P},\ldots,P_N}}
  \and \\
  \inferrule* [lab=process] {} {{M_{P}} \bc M_{N} \;| \;P|M_{P} }
\end{mathpar} 

\begin{mathpar}
  \inferrule* [lab=sychronization] {} {M_{N} \bc \Box \;|\; x?M_{F} \;|\; x!M_{C}}
  \and
  \inferrule* [lab=abstraction] {} {{M_{F}} \bc (x)M_{P} }
  \and
  \inferrule* [lab=concretion] {} {{M_{C}} \bc \langle M_{P} \rangle }
  \and \\
  \inferrule* [lab=process] {} {{M_{P}} \bc M_{N} \;| \;P|M_{P} }
\end{mathpar}

\begin{definition}[contextual application] Given a context $M$, and
  process $P$, we define the \emph{contextual application}, $M[P] :=
  M\{P/\Box\}$. That is, the contextual application of M to P is the
  substitution of $P$ for $\Box$ in $M$.
\end{definition}

$\meaningof{-} : L \to \mathcal{P}(\pi)$

\begin{mathpar}
  \inferrule* [lab=collection] {} {\meaningof{true} = \pi, \and \meaningof{~E} = \pi \setminus \meaningof{E}, \and \meaningof{E_{1} \& E_{2}} = \meaningof{E_{1}} \cap \meaningof{E_{2}}}
\end{mathpar}

\begin{mathpar}
  \inferrule* [lab=structure] {} {\meaningof{0} = \{ P \in \pi | P \equiv 0 \}, \and \\ \meaningof{E_1 | E_2} = \{ P \in \pi | P \equiv P_{1} | P_{2}, P_{1} \in \meaningof{E_{1}}, P_{2} \in \meaningof{E_2}\} }
\end{mathpar}

\begin{mathpar}
 \inferrule* [lab=behavior] {} {\meaningof{\langle a?b \rangle E} = \{ P \in \pi | P \equiv Q | u?(y)P', \\ \and \\\\ \and \\ \;\;\; u \in \meaningof{a}, \forall z.P'\{z/y\} \in \meaningof{E\{z/b\}}\}, \and \\ \meaningof{a!E} = \{ P \in \pi | P \equiv Q | x!\langle P' \rangle, x \in \meaningof{a} P' \in \meaningof{E}\} }
\end{mathpar}

\begin{mathpar}
 \inferrule* [lab=nominal] {} {\meaningof{\quotep{E}} = \{ \quotep{P} \in \quotep{\pi} | P \in \meaningof{E} \}, \and \meaningof{\quotep{P}} = \{ \quotep{Q} \in \quotep{\pi} | P \equiv Q \} \and \\ \meaningof{@\quotep{E}} = \{ P \in \pi | P \equiv @x, x \in \meaningof{E} \}}
\end{mathpar}

\begin{eqnarray*}
  \\
  \meaningof{-} : TS \to ST
\end{eqnarray*}

\begin{eqnarray*}
  \\
  L : TS \to ST
\end{eqnarray*}

\begin{eqnarray*}
  \\
  P \models E \iff P \in \meaningof{E}
\end{eqnarray*}

\begin{eqnarray*}
  P \approx_{L} Q \iff \forall E \in L. P \models E \iff Q \models E
\end{eqnarray*}

\begin{eqnarray*}
  P \approx_{K} Q
\end{eqnarray*}

\begin{eqnarray*}
  P \approx Q
\end{eqnarray*}

$\approx_{K} = \approx = \approx_{L}$

\subsubsection{Contextual duality}

Note that contexts extend the quotation operation to a family of
operations from processes to names. Given a context, $M$, we can
define a \emph{nominal context}, $\quotep{M}$ by $\quotep{M}[P] :=
\quotep{M[P]}$. To foreshadow what is to come we observe that these
operations enjoy a duality with processes very much like the duality
between vectors and maps from vectors to scalars.

Further, because the calculus is essentially higher-order, we have a
correspondence between contexts and processes. More specifically,
given a name $x$ and a context $M$ we can construct $M^{*}_{x}$ such
that 

\begin{mathpar}
  M^{*}_{x} | \lift{x}{P} \red M[P]
\end{mathpar}

namely,

\begin{mathpar}
  M^{*}_{x} := x?(u).M[\dropn{u}]
\end{mathpar}

The dependence of $M^{*}_{x}$ on a name makes it an abstraction, 

\begin{mathpar}
  M^{*} := (x)x?(u).M[\dropn{u}]
\end{mathpar}

\subsection{Additional notation}

It will sometimes be convenient to denote the process a name
quotes. We already have the notation $x = \quotep{P}$, but it will be
convenient to introduce an alternate notation, $\procn{x}$, when we
want to emphasize the connection to the use of the name. Note that, by
virtue of name equivalence, $\quotep{\procn{x}} \nameeq x$; so, the
notation is consistent with previous definitions.

Further, because names have structure it is possible to effect
substitutions on the basis of that structure. This means we need to
upgrade our notation for substitutions, which we accomplish by
adapting comprehension notation. Thus,

\begin{mathpar}
  P\{ y / x : x \in S \}
\end{mathpar}

is interpreted to mean the process derived from P by replacing (in a
capture-avoiding manner) each occurrence of $x$ in $S$ by $y$. For example,

\begin{mathpar}
  P\{ \quotep{\procn{x}|\procn{x}} / x : x \in \freenames{P} \}
\end{mathpar}

will replace each (occurrence) of a free name $x$ in $P$ by
$\quotep{\procn{x}|\procn{x}}$.

Also, we will avail ourselves of the notation $x^{L}$ and $x^{R}$ to
denote injections of a name into disjoint copies of the name
space. There are numerous ways to accomplish this. One example can be
found in \cite{MeredithR05}. This notation overloads to vectors of
names: $\vec{x}^{\pi} := (x_{i}^{\pi} \; : \; 0 \leq i < |\vec{x}| )$ where $\pi \in \{L,R\}$.

We also use $P^{\Box} := P|\Box$.

In \cite{MeredithR05} an interpretation of the new operator is
given. It turns out that there are several possible interpretations
all enjoying the requisite algebraic properties of the operator (see
\cite{milner91polyadicpi}). We will therefore make liberal use of
$(\nu\; \vec{x})P$.

% subsection the_syntax_and_semantics_of_the_notation_system (end)   

\input{qm2pi.qmops} 

\input{qm2pi.sterngerlach} 

\input{qm2pi.metric} 

% section concurrent_process_calculi (end)

%\input{qm2pi.proofsketch}

% section proof sketch (end)

%\input{qm2pi.slviaknots} 

% section spatial logic via knots (end)

\input{qm2pi.conclusion}

% section conclusion (end)

%\input{qm2pi.dtcodes} 

% section wiring algorithm (end)

\input{qm2pi.ack} 

% section acknowledgments (end)

\newpage


\bibliographystyle{plain}   
\bibliography{../../biblios/main.bib}

\input{qm2pi.rhodetails}

\end{document}

 

% section acknowledgments (end)

\newpage


\bibliographystyle{plain}   
\bibliography{../../biblios/main.bib}

\documentclass[12pt]{llncs}
%\documentclass{jktr}

\usepackage[pdftex]{hyperref}                   
\usepackage {listings}
\usepackage {mathpartir}
\usepackage{bcprules}
%\usepackage{listings}
                       
\usepackage{graphicx} 
%\usepackage[margins=2.5cm,nohead,nofoot]{geometry}
%\usepackage{geometry}
\usepackage{amsfonts}
\usepackage{amstext}
\usepackage{latexsym}
\usepackage{amssymb}
\usepackage{color}


%\include{myPreamble}
\include{qm2pi.local} 

%\ifpdf
%\usepackage[pdftex]{graphicx}
%\else
%\usepackage{graphicx}
%\fi

 % \ifpdf
%  \usepackage{pdfsync}
%  \if


%\title{Brief Article}
%\author{David F. Snyder}
%\author{L.G. Meredith}

%\address{Dept. of Math., Texas State University--San Marcos, San Marcos, TX 78666}
       
\pagestyle{empty}


\begin{document}

\lstset{language=[Objective]Caml,frame=shadowbox}

\input{qm2pi.front}

% section front matter (end)

\input{qm2pi.intro} 
 
% section introduction (end)

% \input{qm2pi.knotations} 

% section notation (end)

\input{qm2pi.process.calculi} 

% section concurrent_process_calculi_and_spatial_logics_ (end)
    
%\input{qm2pi.knots2pi} 

%\input{qm2pi.trefoil} 

%\input{qm2pi.mainthm} 

% subsection basic_interpretation (end)

%\input{qm2pi.rho.presentation} 
\subsection{The syntax and semantics of the notation system}\label{sub:the_syntax_and_semantics_of_the_notation_system} % (fold)

We now summarize a technical presentation of the calculus that
embodies our theory of dynamics. The typical presentation of such a
calculus follows the style of giving generators and relations on
them. The grammar, below, describing term constructors, freely
generates the set of processes, $\Proc$. This set is then quotiented
by a relation known as structural congruence and it is over this set
that the notion of dynamics is expressed. This presentation is
essentially that of \cite{MeredithR05} with the addition of
polyadicity and summation. For readability we have relegated some of
the technical subtleties to an appendix.

\subsubsection{Process grammar}\label{subsub:process_grammar}

\begin{mathpar}
  \inferrule* [lab=synchronization] {} {{M} \bc \pzero \;|\; x?F \;|\; x!C }
  \and
  \inferrule* [lab=abstraction] {} {{F} \bc (x)P}
  \and
  \inferrule* [lab=concretion] {} {{C} \bc \langle Q \rangle}
  \and
  \inferrule* [lab=process] {} {{P,Q} \bc M \;| \;P|Q \;|\; @{x}}
  \and
  \inferrule* [lab=name] {} {{x} \bc \quotep{P}}
\end{mathpar} 

Note that $\vec{x}$ (resp. $\vec{P}$) denotes a vector of names
(resp. processes) of length $|\vec{x}|$ (resp. $|\vec{P}|$). We adopt
the following useful abbreviations.

\begin{mathpar}
   x?(\vec{y}).P := x.(\vec{y})P \and  x\clift{\vec{P}} := x.\clift{\vec{P}}
   \and x!(y) := \lift{x}{\dropn{y}}
   \and \Pi_{i=0}^{n-1}P_i := P_0 | \ldots | P_{n-1}
\end{mathpar}

\subsubsection{Structural congruence}

\paragraph{Free and bound names and alpha-equivalence.} At the
core of structural equivalence is alpha-equivalence which identifies
process that are the same up to a change of variable. Formally, we
recognize the distinction between free and bound names. The free names
of a process, $\freenames{P}$, may be calculated recursively as
follows:

\begin{mathpar}
\freenames{\pzero} := \emptyset
  \and \\
  \freenames{x?(y).P} := \{ x \} \cup (\freenames{P} \setminus \{ y \})
  \and 
  \freenames{x!\langle P \rangle} := \{ x \} \cup \{ P \} 
  \and \\
  \freenames{P|Q} := \freenames{P} \cup \freenames{Q}
  \and \\
  \freenames{@{x}} := \{ x \}
\end{mathpar}

$\pi$
$\quotep{\pi}$

$\freenames{-} : \pi \to \mathcal{P}(\quotep{\pi})$

\begin{eqnarray*}
  \freenames{\pzero} & := & \emptyset \\
  \freenames{x?(y).P} & := & \{ x \} \cup (\freenames{P} \setminus \{ y \}) \\
  \freenames{x!\langle P \rangle} & := & \{ x \} \cup \{ P \} \\
  \freenames{P|Q} & := & \freenames{P} \cup \freenames{Q} \\
  \freenames{\dropn{x}} & := & \{ x \}
\end{eqnarray*}

The bound names of a process, $\boundnames{P}$, are those names occurring in $P$
that are not free. For example, in $x?(y).0$, the name $x$ is free, while $y$ is bound.

\begin{mathpar}
  \inferrule* [lab=monoidal-laws] {} { P|Q \equiv Q|P \and P|0 \equiv P \and P|(Q|R) \equiv (P|Q)|R }
\end{mathpar}

\begin{mathpar}
  \inferrule* [lab=alpha-equivalence] {} { (x)P \equiv (y)P\{y/x\} \and y \not\in \freenames{P} }
\end{mathpar}

\begin{definition}
Then two processes, $P,Q$, are alpha-equivalent if $P = Q\{\vec{y}/\vec{x}\}$ for
some $\vec{x} \in \boundnames{Q},\vec{y} \in \boundnames{P}$, where $Q\{\vec{y}/\vec{x}\}$
denotes the capture-avoiding substitution of $\vec{y}$ for $\vec{x}$ in $Q$.
\end{definition}

\begin{definition}
  The {\em structural congruence} \cite{SangiorgiWalker} , $\equiv$,
  between processes is the least congruence containing
  alpha-equivalence, satisfying the abelian monoid laws
  (associativity, commutativity and $\pzero$ as identity) for parallel
  composition $|$ and for summation $+$.
\end{definition}

\subsection{Name equivalence}

We take name equivalence, written $\nameeq$, to be the smallest
equivalence relation generated by the following rules.

\begin{mathpar}
\inferrule*[lab=Quote-drop]
{ }
{ \quotep{@{x}} \nameeq x }

\inferrule*[lab=Struct-equiv]
{ P \scong Q }
{ \quotep{P} \nameeq \quotep{Q} }
\end{mathpar}

The astute reader will have noticed that the mutual recursion of names
and processes imposes a mutual recursion on alpha-equivalence and
structural equivalence via name-equivalence. Fortunately, all of this
works out pleasantly and we may calculate in the natural way, free of
concern. The reader interested in the details is referred to the
appendix \ref{appendix:rho_details}.

\subsection{Substitution}

We use $\Proc$ for the set of processes, $\QProc$ for the set of
names, and $\id{\{}\vec{y} / \vec{x} \id{\}}$ to denote partial maps,
$s : \QProc \rightarrow \QProc$. A map, $s$ lifts, uniquely, to a map
on process terms, $\widehat{s} : \Proc \rightarrow \Proc$ by the
following equations.

\begin{mathpar}
  (0) \psubstp{Q}{P} := 0 \\
  (R \juxtap S) \psubstp{Q}{P}
  :=    
  (R)\psubstp{Q}{P} \juxtap (S) \psubstp{Q}{P} \\
  (x?(y).R) \psubstp{Q}{P}    
  :=    
  (x)\substp{Q}{P} (z)\concat( (R \psubstn{z}{y}) \psubstp{Q}{P} ) \\
  (\lift{x}{R}) \psubstp{Q}{P}  
  :=
  \lift{(x)\substp{Q}{P}}{ R \psubstp{Q}{P} } \\
%   (\dropn{x})  \psubstp{Q}{P}       
%   := 
%   \left\{ 
%     \begin{array}{ccc} 
%       \dropn{\quotep{Q}} & & x \nameeq \quotep{P} \\
%       \dropn{x} & & otherwise \\
%     \end{array}
%   \right. 
  (\dropn{x})  \psubstp{Q}{P}       
  := 
  \left\{ 
    \begin{array}{ccc} 
      Q & & x \nameeq \quotep{P} \\
      \dropn{x} & & otherwise \\
    \end{array}
  \right.
\end{mathpar}
 

where

\begin{eqnarray}
  (x)\id{\{} \lpquote Q \rpquote / \lpquote P \rpquote \id{\}}            = 
  \left\{ 
    \begin{array}{ccc}
      \lpquote Q \rpquote & & x \nameeq \lpquote P \rpquote \\
      x & & otherwise \\
    \end{array}
  \right. \nonumber
\end{eqnarray}

and $z$ is chosen distinct from $\quotep{P}$, $\quotep{Q}$, the free
names in $Q$, and all the names in $R$. Our $\alpha$-equivalence will
be built in the standard way from this substitution.

\begin{remark}\label{rem:no_self_referential_names}
  One consequence of these definitions is that $\forall P. \quotep{P}
  \not\in \freenames{P}$.
\end{remark}

\subsection{ Dynamic quote: an example }

Anticipating something of what's to come, consider applying the
substitution, $\widehat{\id{\{}u / z \id{\}}}$, to the following pair
of processes, $\lift{w}{y!(z)}$ and $w[ \lpquote y!(z) \rpquote ]$.

\begin{eqnarray}
	\lift{w}{y!(z)}\widehat{\id{\{}u / z \id{\}}}
		& = &
		\lift{w}{y!(u)} \nonumber\\
	w[ \lpquote y!(z) \rpquote ] \widehat{ \id{\{}u / z \id{\}} }
		& = &
		w[ \lpquote y!(z) \rpquote ] \nonumber
\end{eqnarray}

Because the body of the process between quotes is impervious to
substitution, we get radically different answers. In fact, by
examining the first process in an input context,
e.g. $x?(z).\lift{w}{y!(z)}$, we see that the process under the lift
operator may be shaped by prefixed inputs binding a name inside it. In
this sense, the lift operator will be seen as a way to dynamically
construct processes before reifying them as names.

Finally equipped with these standard features we can present the
dynamics of the calculus.

\subsubsection{Operational semantics} 

Finally, we introduce the computational dynamics. What marks these
algebras as distinct from other more traditionally studied algebraic
structures, e.g. vector spaces or polynomial rings, is the manner in
which dynamics is captured. In traditional structures, dynamics is typically
expressed through morphisms between such structures, as in linear maps
between vector spaces or morphisms between rings. In algebras
associated with the semantics of computation, the dynamics is
expressed as part of the algebraic structure itself, through a
reduction reduction relation typically denoted by $\red$. Below, we
give a recursive presentation of this relation for the calculus used
in the encoding.

$\red \subseteq \pi \times \pi$
$\red : \pi \to \mathcal{P}(\pi)$

\begin{mathpar}
  \inferrule* [lab=Comm] { \textsf{match}( x_{src}, x_{trgt} ) } { x_{trgt}?(y)P \; | \; x_{src}!\langle {Q} \rangle \red P\{\quotep{Q}/y}\} }
  \and \\
  \inferrule* [lab=Par] {{P} \red {P}'} {{{P} | {Q}} \red {{P}' | {Q}}}
  \and
  \inferrule* [lab=Equiv]{{{P} \scong {P}'} \andalso {{P}' \red {Q}'} \andalso {{Q}' \scong {Q}}}{{P} \red {Q}}
\end{mathpar}

\begin{eqnarray*}
  match_{\equiv} (\quotep{P},\quotep{Q}) & := & P \equiv Q \\
  match_{\dagger}(\quotep{P},\quotep{Q}) & := & \forall R. P|Q \red^{*} R => R \red^{*} 0 \\
  match_{K}(\quotep{P},\quotep{Q}) & := & K \mbox{ for some context } K
\end{eqnarray*}

$u?(x)P | u!\langle Q \rangle \red P\{\quotep{Q}/x\}$

%We write $\wred$ for $\red^*$, and $P\red$ if $\exists Q $ such that $ P \red Q$.
We write $P\red$ if $\exists Q $ such that $ P \red Q$ and $P\not\red$, otherwise.

\section{Replication}

As mentioned before, it is known that replication (and hence
recursion) can be implemented in a higher-order process algebra
\cite{SangiorgiWalker}. As our first example of calculation with the
machinery thus far presented we give the construction explicitly in
the {\rhoc}.

\begin{eqnarray}
	D_{x} & := & \prefix{x}{y}{(\binpar{\outputp{x}{y}}{@{y}})} \nonumber\\
	\bangp_{x}{P} & := & \binpar{{x}!\langle{\binpar{D_{x}}{P}}\rangle}{D_{x}} \nonumber
\end{eqnarray}

\begin{eqnarray}
	\bangp_{x}{P} & & \nonumber\\
	=
	& {x}!\langle{(\prefix{x}{y}{(\outputp{x}{y} | @{y})) | P}}\rangle 
	      | \prefix{x}{y}{(\outputp{x}{y} | @{y})} & \nonumber\\
	\red
	& (\outputp{x}{y} | @{y})\substn{\quotep{(\prefix{x}{y}{(@{y} | \outputp{x}{y})) | P}}}{y} & \nonumber\\
	=
	& \outputp{x}{\quotep{(\prefix{x}{y}{(\outputp{x}{y} | @{y})) | P}}}
	  | {(\prefix{x}{y}{(\outputp{x}{y} | @{y})) | P}} & \nonumber\\
	\red
	& \ldots & \nonumber\\
	\red^*
	& P | P | \ldots & \nonumber
\end{eqnarray}

Of course, this encoding, as an implementation, runs away, unfolding
$\bangp{P}$ eagerly. A lazier and more implementable replication
operator, restricted to input-guarded processes, may be obtained as follows.

\begin{eqnarray}
\bangp{\prefix{u}{v}{P}} 
	:= 
	\binpar{\lift{x}{\prefix{u}{v}{(\binpar{D(x)}{P})}}}{D(x)} \nonumber
\end{eqnarray}

\begin{remark}
  Note that the lazier definition still does not deal with summation
  or mixed summation (i.e. sums over input and output). The reader is
  invited to construct definitions of replication that deal with these
  features. 

  Further, the definitions are parameterized in a name, $x$. Can you,
  gentle reader, make a definition that eliminates this parameter and
  guarantees no accidental interaction between the replication
  machinery and the process being replicated -- i.e. no accidental
  sharing of names used by the process to get its work done and the
  name(s) used by the replication to effect copying. This latter
  revision of the definition of replication is crucial to obtaining
  the expected identity $!!P \sim !P$.
\end{remark}

\begin{remark}\label{rem:paradoxical_combinator}
  The reader familiar with the lambda calculus will have noticed the
  similarity between $D$ and the paradoxical combinator.

  [Ed. note: the existence of this seems to suggest we have to be more
  restrictive on the set of processes and names we admit if we are to
  support no-cloning.]
\end{remark}

\subsubsection{Bisimulation}

The computational dynamics gives rise to another kind of equivalence,
the equivalence of computational behavior. As previously mentioned
this is typically captured \emph{via} some form of bisimulation.

% The notion we use in this paper is weak barbed bisimulation
% \cite{milner91polyadicpi}.

The notion we use in this paper is derived from weak barbed
bisimulation \cite{milner91polyadicpi}. 

\begin{definition}
An \emph{observation relation}, $\downarrow_{\mathcal N}$, over a set
of names, $\mathcal N$, is the smallest relation satisfying the rules
below.

\infrule[Out-barb]{y \in {\mathcal N}, \; x \nameeq y}
		  {\outputp{x}{v} \downarrow_{\mathcal N} x}
\infrule[Par-barb]{\mbox{$P\downarrow_{\mathcal N} x$ or $Q\downarrow_{\mathcal N} x$}}
		  {\binpar{P}{Q} \downarrow_{\mathcal N} x}

We write $P \Downarrow_{\mathcal N} x$ if there is $Q$ such that 
$P \wred Q$ and $Q \downarrow_{\mathcal N} x$.
\end{definition}

\begin{definition}
%\label{def.bbisim}
An  ${\mathcal N}$-\emph{barbed bisimulation} over a set of names, ${\mathcal N}$, is a symmetric binary relation 
${\mathcal S}_{\mathcal N}$ between agents such that $P\rel{S}_{\mathcal N}Q$ implies:
\begin{enumerate}
\item If $P \red P'$ then $Q \wred Q'$ and $P'\rel{S}_{\mathcal N} Q'$.
\item If $P\downarrow_{\mathcal N} x$, then $Q\Downarrow_{\mathcal N} x$.
\end{enumerate}
$P$ is ${\mathcal N}$-barbed bisimilar to $Q$, written
$P \wbbisim_{\mathcal N} Q$, if $P \rel{S}_{\mathcal N} Q$ for some ${\mathcal N}$-barbed bisimulation ${\mathcal S}_{\mathcal N}$.
\end{definition}

$\mathcal{R} \subseteq \pi \times \pi$

$P \mathcal{R} Q => \forall P'. P \red P' \Rightarrow \exists Q'. Q \red Q', P' \mathcal{R} Q'$

$P \vdash x \Rightarrow Q \vdash x$

\begin{mathpar}
  \inferrule*[lab=Out-barb]{x \nameeq y}{{y}!\langle{Q}\rangle \vdash x}
  \and
  \inferrule*[lab=Par-barb]{\mbox{$P\vdash x$ or $Q\vdash x$}}{\binpar{P}{Q} \vdash x}
\end{mathpar}

\subsubsection{Contexts}

One of the principle advantages of computational calculi like the
$\pi$-calculus is a well-defined notion of context,
contextual-equivalence and a correlation between
contextual-equivalence and notions of bisimulation. The notion of
context allows the decomposition of a process into (sub-)process and
its syntactic environment, its context. Thus, a context may be
thought of as a process with a ``hole'' (written $\Box$) in it. The
application of a context $M$ to a process $P$, written $M[P]$, is
tantamount to filling the hole in $M$ with $P$. In this paper we do
not need the full weight of this theory, but do make use of the notion
of context in the proof the main theorem. 

\begin{mathpar}
  \inferrule* [lab=summation] {} {{M_{M},M_{N}} \bc \Box \;|\; x.M_{A} \;|\; M_{M}+M_{N}}
  \and
  \inferrule* [lab=agent] {} {{M_{A}} \bc (\vec{x})M_{P} \;| \; \clift{P_0,\ldots,M_{P},\ldots,P_N}}
  \and \\
  \inferrule* [lab=process] {} {{M_{P}} \bc M_{N} \;| \;P|M_{P} }
\end{mathpar} 

\begin{mathpar}
  \inferrule* [lab=sychronization] {} {M_{N} \bc \Box \;|\; x?M_{F} \;|\; x!M_{C}}
  \and
  \inferrule* [lab=abstraction] {} {{M_{F}} \bc (x)M_{P} }
  \and
  \inferrule* [lab=concretion] {} {{M_{C}} \bc \langle M_{P} \rangle }
  \and \\
  \inferrule* [lab=process] {} {{M_{P}} \bc M_{N} \;| \;P|M_{P} }
\end{mathpar}

\begin{definition}[contextual application] Given a context $M$, and
  process $P$, we define the \emph{contextual application}, $M[P] :=
  M\{P/\Box\}$. That is, the contextual application of M to P is the
  substitution of $P$ for $\Box$ in $M$.
\end{definition}

$\meaningof{-} : L \to \mathcal{P}(\pi)$

\begin{mathpar}
  \inferrule* [lab=collection] {} {\meaningof{true} = \pi, \and \meaningof{~E} = \pi \setminus \meaningof{E}, \and \meaningof{E_{1} \& E_{2}} = \meaningof{E_{1}} \cap \meaningof{E_{2}}}
\end{mathpar}

\begin{mathpar}
  \inferrule* [lab=structure] {} {\meaningof{0} = \{ P \in \pi | P \equiv 0 \}, \and \\ \meaningof{E_1 | E_2} = \{ P \in \pi | P \equiv P_{1} | P_{2}, P_{1} \in \meaningof{E_{1}}, P_{2} \in \meaningof{E_2}\} }
\end{mathpar}

\begin{mathpar}
 \inferrule* [lab=behavior] {} {\meaningof{\langle a?b \rangle E} = \{ P \in \pi | P \equiv Q | u?(y)P', \\ \and \\\\ \and \\ \;\;\; u \in \meaningof{a}, \forall z.P'\{z/y\} \in \meaningof{E\{z/b\}}\}, \and \\ \meaningof{a!E} = \{ P \in \pi | P \equiv Q | x!\langle P' \rangle, x \in \meaningof{a} P' \in \meaningof{E}\} }
\end{mathpar}

\begin{mathpar}
 \inferrule* [lab=nominal] {} {\meaningof{\quotep{E}} = \{ \quotep{P} \in \quotep{\pi} | P \in \meaningof{E} \}, \and \meaningof{\quotep{P}} = \{ \quotep{Q} \in \quotep{\pi} | P \equiv Q \} \and \\ \meaningof{@\quotep{E}} = \{ P \in \pi | P \equiv @x, x \in \meaningof{E} \}}
\end{mathpar}

\begin{eqnarray*}
  \\
  \meaningof{-} : TS \to ST
\end{eqnarray*}

\begin{eqnarray*}
  \\
  L : TS \to ST
\end{eqnarray*}

\begin{eqnarray*}
  \\
  P \models E \iff P \in \meaningof{E}
\end{eqnarray*}

\begin{eqnarray*}
  P \approx_{L} Q \iff \forall E \in L. P \models E \iff Q \models E
\end{eqnarray*}

\begin{eqnarray*}
  P \approx_{K} Q
\end{eqnarray*}

\begin{eqnarray*}
  P \approx Q
\end{eqnarray*}

$\approx_{K} = \approx = \approx_{L}$

\subsubsection{Contextual duality}

Note that contexts extend the quotation operation to a family of
operations from processes to names. Given a context, $M$, we can
define a \emph{nominal context}, $\quotep{M}$ by $\quotep{M}[P] :=
\quotep{M[P]}$. To foreshadow what is to come we observe that these
operations enjoy a duality with processes very much like the duality
between vectors and maps from vectors to scalars.

Further, because the calculus is essentially higher-order, we have a
correspondence between contexts and processes. More specifically,
given a name $x$ and a context $M$ we can construct $M^{*}_{x}$ such
that 

\begin{mathpar}
  M^{*}_{x} | \lift{x}{P} \red M[P]
\end{mathpar}

namely,

\begin{mathpar}
  M^{*}_{x} := x?(u).M[\dropn{u}]
\end{mathpar}

The dependence of $M^{*}_{x}$ on a name makes it an abstraction, 

\begin{mathpar}
  M^{*} := (x)x?(u).M[\dropn{u}]
\end{mathpar}

\subsection{Additional notation}

It will sometimes be convenient to denote the process a name
quotes. We already have the notation $x = \quotep{P}$, but it will be
convenient to introduce an alternate notation, $\procn{x}$, when we
want to emphasize the connection to the use of the name. Note that, by
virtue of name equivalence, $\quotep{\procn{x}} \nameeq x$; so, the
notation is consistent with previous definitions.

Further, because names have structure it is possible to effect
substitutions on the basis of that structure. This means we need to
upgrade our notation for substitutions, which we accomplish by
adapting comprehension notation. Thus,

\begin{mathpar}
  P\{ y / x : x \in S \}
\end{mathpar}

is interpreted to mean the process derived from P by replacing (in a
capture-avoiding manner) each occurrence of $x$ in $S$ by $y$. For example,

\begin{mathpar}
  P\{ \quotep{\procn{x}|\procn{x}} / x : x \in \freenames{P} \}
\end{mathpar}

will replace each (occurrence) of a free name $x$ in $P$ by
$\quotep{\procn{x}|\procn{x}}$.

Also, we will avail ourselves of the notation $x^{L}$ and $x^{R}$ to
denote injections of a name into disjoint copies of the name
space. There are numerous ways to accomplish this. One example can be
found in \cite{MeredithR05}. This notation overloads to vectors of
names: $\vec{x}^{\pi} := (x_{i}^{\pi} \; : \; 0 \leq i < |\vec{x}| )$ where $\pi \in \{L,R\}$.

We also use $P^{\Box} := P|\Box$.

In \cite{MeredithR05} an interpretation of the new operator is
given. It turns out that there are several possible interpretations
all enjoying the requisite algebraic properties of the operator (see
\cite{milner91polyadicpi}). We will therefore make liberal use of
$(\nu\; \vec{x})P$.

% subsection the_syntax_and_semantics_of_the_notation_system (end)   

\input{qm2pi.qmops} 

\input{qm2pi.sterngerlach} 

\input{qm2pi.metric} 

% section concurrent_process_calculi (end)

%\input{qm2pi.proofsketch}

% section proof sketch (end)

%\input{qm2pi.slviaknots} 

% section spatial logic via knots (end)

\input{qm2pi.conclusion}

% section conclusion (end)

%\input{qm2pi.dtcodes} 

% section wiring algorithm (end)

\input{qm2pi.ack} 

% section acknowledgments (end)

\newpage


\bibliographystyle{plain}   
\bibliography{../../biblios/main.bib}

\input{qm2pi.rhodetails}

\end{document}



\end{document}

 

%\ifpdf
%\usepackage[pdftex]{graphicx}
%\else
%\usepackage{graphicx}
%\fi

 % \ifpdf
%  \usepackage{pdfsync}
%  \if


%\title{Brief Article}
%\author{David F. Snyder}
%\author{L.G. Meredith}

%\address{Dept. of Math., Texas State University--San Marcos, San Marcos, TX 78666}
       
\pagestyle{empty}


\begin{document}

\lstset{language=[Objective]Caml,frame=shadowbox}

\documentclass[12pt]{llncs}
%\documentclass{jktr}

\usepackage[pdftex]{hyperref}                   
\usepackage {listings}
\usepackage {mathpartir}
\usepackage{bcprules}
%\usepackage{listings}
                       
\usepackage{graphicx} 
%\usepackage[margins=2.5cm,nohead,nofoot]{geometry}
%\usepackage{geometry}
\usepackage{amsfonts}
\usepackage{amstext}
\usepackage{latexsym}
\usepackage{amssymb}
\usepackage{color}


%\include{myPreamble}
\documentclass[12pt]{llncs}
%\documentclass{jktr}

\usepackage[pdftex]{hyperref}                   
\usepackage {listings}
\usepackage {mathpartir}
\usepackage{bcprules}
%\usepackage{listings}
                       
\usepackage{graphicx} 
%\usepackage[margins=2.5cm,nohead,nofoot]{geometry}
%\usepackage{geometry}
\usepackage{amsfonts}
\usepackage{amstext}
\usepackage{latexsym}
\usepackage{amssymb}
\usepackage{color}


%\include{myPreamble}
\include{qm2pi.local} 

%\ifpdf
%\usepackage[pdftex]{graphicx}
%\else
%\usepackage{graphicx}
%\fi

 % \ifpdf
%  \usepackage{pdfsync}
%  \if


%\title{Brief Article}
%\author{David F. Snyder}
%\author{L.G. Meredith}

%\address{Dept. of Math., Texas State University--San Marcos, San Marcos, TX 78666}
       
\pagestyle{empty}


\begin{document}

\lstset{language=[Objective]Caml,frame=shadowbox}

\input{qm2pi.front}

% section front matter (end)

\input{qm2pi.intro} 
 
% section introduction (end)

% \input{qm2pi.knotations} 

% section notation (end)

\input{qm2pi.process.calculi} 

% section concurrent_process_calculi_and_spatial_logics_ (end)
    
%\input{qm2pi.knots2pi} 

%\input{qm2pi.trefoil} 

%\input{qm2pi.mainthm} 

% subsection basic_interpretation (end)

%\input{qm2pi.rho.presentation} 
\subsection{The syntax and semantics of the notation system}\label{sub:the_syntax_and_semantics_of_the_notation_system} % (fold)

We now summarize a technical presentation of the calculus that
embodies our theory of dynamics. The typical presentation of such a
calculus follows the style of giving generators and relations on
them. The grammar, below, describing term constructors, freely
generates the set of processes, $\Proc$. This set is then quotiented
by a relation known as structural congruence and it is over this set
that the notion of dynamics is expressed. This presentation is
essentially that of \cite{MeredithR05} with the addition of
polyadicity and summation. For readability we have relegated some of
the technical subtleties to an appendix.

\subsubsection{Process grammar}\label{subsub:process_grammar}

\begin{mathpar}
  \inferrule* [lab=synchronization] {} {{M} \bc \pzero \;|\; x?F \;|\; x!C }
  \and
  \inferrule* [lab=abstraction] {} {{F} \bc (x)P}
  \and
  \inferrule* [lab=concretion] {} {{C} \bc \langle Q \rangle}
  \and
  \inferrule* [lab=process] {} {{P,Q} \bc M \;| \;P|Q \;|\; @{x}}
  \and
  \inferrule* [lab=name] {} {{x} \bc \quotep{P}}
\end{mathpar} 

Note that $\vec{x}$ (resp. $\vec{P}$) denotes a vector of names
(resp. processes) of length $|\vec{x}|$ (resp. $|\vec{P}|$). We adopt
the following useful abbreviations.

\begin{mathpar}
   x?(\vec{y}).P := x.(\vec{y})P \and  x\clift{\vec{P}} := x.\clift{\vec{P}}
   \and x!(y) := \lift{x}{\dropn{y}}
   \and \Pi_{i=0}^{n-1}P_i := P_0 | \ldots | P_{n-1}
\end{mathpar}

\subsubsection{Structural congruence}

\paragraph{Free and bound names and alpha-equivalence.} At the
core of structural equivalence is alpha-equivalence which identifies
process that are the same up to a change of variable. Formally, we
recognize the distinction between free and bound names. The free names
of a process, $\freenames{P}$, may be calculated recursively as
follows:

\begin{mathpar}
\freenames{\pzero} := \emptyset
  \and \\
  \freenames{x?(y).P} := \{ x \} \cup (\freenames{P} \setminus \{ y \})
  \and 
  \freenames{x!\langle P \rangle} := \{ x \} \cup \{ P \} 
  \and \\
  \freenames{P|Q} := \freenames{P} \cup \freenames{Q}
  \and \\
  \freenames{@{x}} := \{ x \}
\end{mathpar}

$\pi$
$\quotep{\pi}$

$\freenames{-} : \pi \to \mathcal{P}(\quotep{\pi})$

\begin{eqnarray*}
  \freenames{\pzero} & := & \emptyset \\
  \freenames{x?(y).P} & := & \{ x \} \cup (\freenames{P} \setminus \{ y \}) \\
  \freenames{x!\langle P \rangle} & := & \{ x \} \cup \{ P \} \\
  \freenames{P|Q} & := & \freenames{P} \cup \freenames{Q} \\
  \freenames{\dropn{x}} & := & \{ x \}
\end{eqnarray*}

The bound names of a process, $\boundnames{P}$, are those names occurring in $P$
that are not free. For example, in $x?(y).0$, the name $x$ is free, while $y$ is bound.

\begin{mathpar}
  \inferrule* [lab=monoidal-laws] {} { P|Q \equiv Q|P \and P|0 \equiv P \and P|(Q|R) \equiv (P|Q)|R }
\end{mathpar}

\begin{mathpar}
  \inferrule* [lab=alpha-equivalence] {} { (x)P \equiv (y)P\{y/x\} \and y \not\in \freenames{P} }
\end{mathpar}

\begin{definition}
Then two processes, $P,Q$, are alpha-equivalent if $P = Q\{\vec{y}/\vec{x}\}$ for
some $\vec{x} \in \boundnames{Q},\vec{y} \in \boundnames{P}$, where $Q\{\vec{y}/\vec{x}\}$
denotes the capture-avoiding substitution of $\vec{y}$ for $\vec{x}$ in $Q$.
\end{definition}

\begin{definition}
  The {\em structural congruence} \cite{SangiorgiWalker} , $\equiv$,
  between processes is the least congruence containing
  alpha-equivalence, satisfying the abelian monoid laws
  (associativity, commutativity and $\pzero$ as identity) for parallel
  composition $|$ and for summation $+$.
\end{definition}

\subsection{Name equivalence}

We take name equivalence, written $\nameeq$, to be the smallest
equivalence relation generated by the following rules.

\begin{mathpar}
\inferrule*[lab=Quote-drop]
{ }
{ \quotep{@{x}} \nameeq x }

\inferrule*[lab=Struct-equiv]
{ P \scong Q }
{ \quotep{P} \nameeq \quotep{Q} }
\end{mathpar}

The astute reader will have noticed that the mutual recursion of names
and processes imposes a mutual recursion on alpha-equivalence and
structural equivalence via name-equivalence. Fortunately, all of this
works out pleasantly and we may calculate in the natural way, free of
concern. The reader interested in the details is referred to the
appendix \ref{appendix:rho_details}.

\subsection{Substitution}

We use $\Proc$ for the set of processes, $\QProc$ for the set of
names, and $\id{\{}\vec{y} / \vec{x} \id{\}}$ to denote partial maps,
$s : \QProc \rightarrow \QProc$. A map, $s$ lifts, uniquely, to a map
on process terms, $\widehat{s} : \Proc \rightarrow \Proc$ by the
following equations.

\begin{mathpar}
  (0) \psubstp{Q}{P} := 0 \\
  (R \juxtap S) \psubstp{Q}{P}
  :=    
  (R)\psubstp{Q}{P} \juxtap (S) \psubstp{Q}{P} \\
  (x?(y).R) \psubstp{Q}{P}    
  :=    
  (x)\substp{Q}{P} (z)\concat( (R \psubstn{z}{y}) \psubstp{Q}{P} ) \\
  (\lift{x}{R}) \psubstp{Q}{P}  
  :=
  \lift{(x)\substp{Q}{P}}{ R \psubstp{Q}{P} } \\
%   (\dropn{x})  \psubstp{Q}{P}       
%   := 
%   \left\{ 
%     \begin{array}{ccc} 
%       \dropn{\quotep{Q}} & & x \nameeq \quotep{P} \\
%       \dropn{x} & & otherwise \\
%     \end{array}
%   \right. 
  (\dropn{x})  \psubstp{Q}{P}       
  := 
  \left\{ 
    \begin{array}{ccc} 
      Q & & x \nameeq \quotep{P} \\
      \dropn{x} & & otherwise \\
    \end{array}
  \right.
\end{mathpar}
 

where

\begin{eqnarray}
  (x)\id{\{} \lpquote Q \rpquote / \lpquote P \rpquote \id{\}}            = 
  \left\{ 
    \begin{array}{ccc}
      \lpquote Q \rpquote & & x \nameeq \lpquote P \rpquote \\
      x & & otherwise \\
    \end{array}
  \right. \nonumber
\end{eqnarray}

and $z$ is chosen distinct from $\quotep{P}$, $\quotep{Q}$, the free
names in $Q$, and all the names in $R$. Our $\alpha$-equivalence will
be built in the standard way from this substitution.

\begin{remark}\label{rem:no_self_referential_names}
  One consequence of these definitions is that $\forall P. \quotep{P}
  \not\in \freenames{P}$.
\end{remark}

\subsection{ Dynamic quote: an example }

Anticipating something of what's to come, consider applying the
substitution, $\widehat{\id{\{}u / z \id{\}}}$, to the following pair
of processes, $\lift{w}{y!(z)}$ and $w[ \lpquote y!(z) \rpquote ]$.

\begin{eqnarray}
	\lift{w}{y!(z)}\widehat{\id{\{}u / z \id{\}}}
		& = &
		\lift{w}{y!(u)} \nonumber\\
	w[ \lpquote y!(z) \rpquote ] \widehat{ \id{\{}u / z \id{\}} }
		& = &
		w[ \lpquote y!(z) \rpquote ] \nonumber
\end{eqnarray}

Because the body of the process between quotes is impervious to
substitution, we get radically different answers. In fact, by
examining the first process in an input context,
e.g. $x?(z).\lift{w}{y!(z)}$, we see that the process under the lift
operator may be shaped by prefixed inputs binding a name inside it. In
this sense, the lift operator will be seen as a way to dynamically
construct processes before reifying them as names.

Finally equipped with these standard features we can present the
dynamics of the calculus.

\subsubsection{Operational semantics} 

Finally, we introduce the computational dynamics. What marks these
algebras as distinct from other more traditionally studied algebraic
structures, e.g. vector spaces or polynomial rings, is the manner in
which dynamics is captured. In traditional structures, dynamics is typically
expressed through morphisms between such structures, as in linear maps
between vector spaces or morphisms between rings. In algebras
associated with the semantics of computation, the dynamics is
expressed as part of the algebraic structure itself, through a
reduction reduction relation typically denoted by $\red$. Below, we
give a recursive presentation of this relation for the calculus used
in the encoding.

$\red \subseteq \pi \times \pi$
$\red : \pi \to \mathcal{P}(\pi)$

\begin{mathpar}
  \inferrule* [lab=Comm] { \textsf{match}( x_{src}, x_{trgt} ) } { x_{trgt}?(y)P \; | \; x_{src}!\langle {Q} \rangle \red P\{\quotep{Q}/y}\} }
  \and \\
  \inferrule* [lab=Par] {{P} \red {P}'} {{{P} | {Q}} \red {{P}' | {Q}}}
  \and
  \inferrule* [lab=Equiv]{{{P} \scong {P}'} \andalso {{P}' \red {Q}'} \andalso {{Q}' \scong {Q}}}{{P} \red {Q}}
\end{mathpar}

\begin{eqnarray*}
  match_{\equiv} (\quotep{P},\quotep{Q}) & := & P \equiv Q \\
  match_{\dagger}(\quotep{P},\quotep{Q}) & := & \forall R. P|Q \red^{*} R => R \red^{*} 0 \\
  match_{K}(\quotep{P},\quotep{Q}) & := & K \mbox{ for some context } K
\end{eqnarray*}

$u?(x)P | u!\langle Q \rangle \red P\{\quotep{Q}/x\}$

%We write $\wred$ for $\red^*$, and $P\red$ if $\exists Q $ such that $ P \red Q$.
We write $P\red$ if $\exists Q $ such that $ P \red Q$ and $P\not\red$, otherwise.

\section{Replication}

As mentioned before, it is known that replication (and hence
recursion) can be implemented in a higher-order process algebra
\cite{SangiorgiWalker}. As our first example of calculation with the
machinery thus far presented we give the construction explicitly in
the {\rhoc}.

\begin{eqnarray}
	D_{x} & := & \prefix{x}{y}{(\binpar{\outputp{x}{y}}{@{y}})} \nonumber\\
	\bangp_{x}{P} & := & \binpar{{x}!\langle{\binpar{D_{x}}{P}}\rangle}{D_{x}} \nonumber
\end{eqnarray}

\begin{eqnarray}
	\bangp_{x}{P} & & \nonumber\\
	=
	& {x}!\langle{(\prefix{x}{y}{(\outputp{x}{y} | @{y})) | P}}\rangle 
	      | \prefix{x}{y}{(\outputp{x}{y} | @{y})} & \nonumber\\
	\red
	& (\outputp{x}{y} | @{y})\substn{\quotep{(\prefix{x}{y}{(@{y} | \outputp{x}{y})) | P}}}{y} & \nonumber\\
	=
	& \outputp{x}{\quotep{(\prefix{x}{y}{(\outputp{x}{y} | @{y})) | P}}}
	  | {(\prefix{x}{y}{(\outputp{x}{y} | @{y})) | P}} & \nonumber\\
	\red
	& \ldots & \nonumber\\
	\red^*
	& P | P | \ldots & \nonumber
\end{eqnarray}

Of course, this encoding, as an implementation, runs away, unfolding
$\bangp{P}$ eagerly. A lazier and more implementable replication
operator, restricted to input-guarded processes, may be obtained as follows.

\begin{eqnarray}
\bangp{\prefix{u}{v}{P}} 
	:= 
	\binpar{\lift{x}{\prefix{u}{v}{(\binpar{D(x)}{P})}}}{D(x)} \nonumber
\end{eqnarray}

\begin{remark}
  Note that the lazier definition still does not deal with summation
  or mixed summation (i.e. sums over input and output). The reader is
  invited to construct definitions of replication that deal with these
  features. 

  Further, the definitions are parameterized in a name, $x$. Can you,
  gentle reader, make a definition that eliminates this parameter and
  guarantees no accidental interaction between the replication
  machinery and the process being replicated -- i.e. no accidental
  sharing of names used by the process to get its work done and the
  name(s) used by the replication to effect copying. This latter
  revision of the definition of replication is crucial to obtaining
  the expected identity $!!P \sim !P$.
\end{remark}

\begin{remark}\label{rem:paradoxical_combinator}
  The reader familiar with the lambda calculus will have noticed the
  similarity between $D$ and the paradoxical combinator.

  [Ed. note: the existence of this seems to suggest we have to be more
  restrictive on the set of processes and names we admit if we are to
  support no-cloning.]
\end{remark}

\subsubsection{Bisimulation}

The computational dynamics gives rise to another kind of equivalence,
the equivalence of computational behavior. As previously mentioned
this is typically captured \emph{via} some form of bisimulation.

% The notion we use in this paper is weak barbed bisimulation
% \cite{milner91polyadicpi}.

The notion we use in this paper is derived from weak barbed
bisimulation \cite{milner91polyadicpi}. 

\begin{definition}
An \emph{observation relation}, $\downarrow_{\mathcal N}$, over a set
of names, $\mathcal N$, is the smallest relation satisfying the rules
below.

\infrule[Out-barb]{y \in {\mathcal N}, \; x \nameeq y}
		  {\outputp{x}{v} \downarrow_{\mathcal N} x}
\infrule[Par-barb]{\mbox{$P\downarrow_{\mathcal N} x$ or $Q\downarrow_{\mathcal N} x$}}
		  {\binpar{P}{Q} \downarrow_{\mathcal N} x}

We write $P \Downarrow_{\mathcal N} x$ if there is $Q$ such that 
$P \wred Q$ and $Q \downarrow_{\mathcal N} x$.
\end{definition}

\begin{definition}
%\label{def.bbisim}
An  ${\mathcal N}$-\emph{barbed bisimulation} over a set of names, ${\mathcal N}$, is a symmetric binary relation 
${\mathcal S}_{\mathcal N}$ between agents such that $P\rel{S}_{\mathcal N}Q$ implies:
\begin{enumerate}
\item If $P \red P'$ then $Q \wred Q'$ and $P'\rel{S}_{\mathcal N} Q'$.
\item If $P\downarrow_{\mathcal N} x$, then $Q\Downarrow_{\mathcal N} x$.
\end{enumerate}
$P$ is ${\mathcal N}$-barbed bisimilar to $Q$, written
$P \wbbisim_{\mathcal N} Q$, if $P \rel{S}_{\mathcal N} Q$ for some ${\mathcal N}$-barbed bisimulation ${\mathcal S}_{\mathcal N}$.
\end{definition}

$\mathcal{R} \subseteq \pi \times \pi$

$P \mathcal{R} Q => \forall P'. P \red P' \Rightarrow \exists Q'. Q \red Q', P' \mathcal{R} Q'$

$P \vdash x \Rightarrow Q \vdash x$

\begin{mathpar}
  \inferrule*[lab=Out-barb]{x \nameeq y}{{y}!\langle{Q}\rangle \vdash x}
  \and
  \inferrule*[lab=Par-barb]{\mbox{$P\vdash x$ or $Q\vdash x$}}{\binpar{P}{Q} \vdash x}
\end{mathpar}

\subsubsection{Contexts}

One of the principle advantages of computational calculi like the
$\pi$-calculus is a well-defined notion of context,
contextual-equivalence and a correlation between
contextual-equivalence and notions of bisimulation. The notion of
context allows the decomposition of a process into (sub-)process and
its syntactic environment, its context. Thus, a context may be
thought of as a process with a ``hole'' (written $\Box$) in it. The
application of a context $M$ to a process $P$, written $M[P]$, is
tantamount to filling the hole in $M$ with $P$. In this paper we do
not need the full weight of this theory, but do make use of the notion
of context in the proof the main theorem. 

\begin{mathpar}
  \inferrule* [lab=summation] {} {{M_{M},M_{N}} \bc \Box \;|\; x.M_{A} \;|\; M_{M}+M_{N}}
  \and
  \inferrule* [lab=agent] {} {{M_{A}} \bc (\vec{x})M_{P} \;| \; \clift{P_0,\ldots,M_{P},\ldots,P_N}}
  \and \\
  \inferrule* [lab=process] {} {{M_{P}} \bc M_{N} \;| \;P|M_{P} }
\end{mathpar} 

\begin{mathpar}
  \inferrule* [lab=sychronization] {} {M_{N} \bc \Box \;|\; x?M_{F} \;|\; x!M_{C}}
  \and
  \inferrule* [lab=abstraction] {} {{M_{F}} \bc (x)M_{P} }
  \and
  \inferrule* [lab=concretion] {} {{M_{C}} \bc \langle M_{P} \rangle }
  \and \\
  \inferrule* [lab=process] {} {{M_{P}} \bc M_{N} \;| \;P|M_{P} }
\end{mathpar}

\begin{definition}[contextual application] Given a context $M$, and
  process $P$, we define the \emph{contextual application}, $M[P] :=
  M\{P/\Box\}$. That is, the contextual application of M to P is the
  substitution of $P$ for $\Box$ in $M$.
\end{definition}

$\meaningof{-} : L \to \mathcal{P}(\pi)$

\begin{mathpar}
  \inferrule* [lab=collection] {} {\meaningof{true} = \pi, \and \meaningof{~E} = \pi \setminus \meaningof{E}, \and \meaningof{E_{1} \& E_{2}} = \meaningof{E_{1}} \cap \meaningof{E_{2}}}
\end{mathpar}

\begin{mathpar}
  \inferrule* [lab=structure] {} {\meaningof{0} = \{ P \in \pi | P \equiv 0 \}, \and \\ \meaningof{E_1 | E_2} = \{ P \in \pi | P \equiv P_{1} | P_{2}, P_{1} \in \meaningof{E_{1}}, P_{2} \in \meaningof{E_2}\} }
\end{mathpar}

\begin{mathpar}
 \inferrule* [lab=behavior] {} {\meaningof{\langle a?b \rangle E} = \{ P \in \pi | P \equiv Q | u?(y)P', \\ \and \\\\ \and \\ \;\;\; u \in \meaningof{a}, \forall z.P'\{z/y\} \in \meaningof{E\{z/b\}}\}, \and \\ \meaningof{a!E} = \{ P \in \pi | P \equiv Q | x!\langle P' \rangle, x \in \meaningof{a} P' \in \meaningof{E}\} }
\end{mathpar}

\begin{mathpar}
 \inferrule* [lab=nominal] {} {\meaningof{\quotep{E}} = \{ \quotep{P} \in \quotep{\pi} | P \in \meaningof{E} \}, \and \meaningof{\quotep{P}} = \{ \quotep{Q} \in \quotep{\pi} | P \equiv Q \} \and \\ \meaningof{@\quotep{E}} = \{ P \in \pi | P \equiv @x, x \in \meaningof{E} \}}
\end{mathpar}

\begin{eqnarray*}
  \\
  \meaningof{-} : TS \to ST
\end{eqnarray*}

\begin{eqnarray*}
  \\
  L : TS \to ST
\end{eqnarray*}

\begin{eqnarray*}
  \\
  P \models E \iff P \in \meaningof{E}
\end{eqnarray*}

\begin{eqnarray*}
  P \approx_{L} Q \iff \forall E \in L. P \models E \iff Q \models E
\end{eqnarray*}

\begin{eqnarray*}
  P \approx_{K} Q
\end{eqnarray*}

\begin{eqnarray*}
  P \approx Q
\end{eqnarray*}

$\approx_{K} = \approx = \approx_{L}$

\subsubsection{Contextual duality}

Note that contexts extend the quotation operation to a family of
operations from processes to names. Given a context, $M$, we can
define a \emph{nominal context}, $\quotep{M}$ by $\quotep{M}[P] :=
\quotep{M[P]}$. To foreshadow what is to come we observe that these
operations enjoy a duality with processes very much like the duality
between vectors and maps from vectors to scalars.

Further, because the calculus is essentially higher-order, we have a
correspondence between contexts and processes. More specifically,
given a name $x$ and a context $M$ we can construct $M^{*}_{x}$ such
that 

\begin{mathpar}
  M^{*}_{x} | \lift{x}{P} \red M[P]
\end{mathpar}

namely,

\begin{mathpar}
  M^{*}_{x} := x?(u).M[\dropn{u}]
\end{mathpar}

The dependence of $M^{*}_{x}$ on a name makes it an abstraction, 

\begin{mathpar}
  M^{*} := (x)x?(u).M[\dropn{u}]
\end{mathpar}

\subsection{Additional notation}

It will sometimes be convenient to denote the process a name
quotes. We already have the notation $x = \quotep{P}$, but it will be
convenient to introduce an alternate notation, $\procn{x}$, when we
want to emphasize the connection to the use of the name. Note that, by
virtue of name equivalence, $\quotep{\procn{x}} \nameeq x$; so, the
notation is consistent with previous definitions.

Further, because names have structure it is possible to effect
substitutions on the basis of that structure. This means we need to
upgrade our notation for substitutions, which we accomplish by
adapting comprehension notation. Thus,

\begin{mathpar}
  P\{ y / x : x \in S \}
\end{mathpar}

is interpreted to mean the process derived from P by replacing (in a
capture-avoiding manner) each occurrence of $x$ in $S$ by $y$. For example,

\begin{mathpar}
  P\{ \quotep{\procn{x}|\procn{x}} / x : x \in \freenames{P} \}
\end{mathpar}

will replace each (occurrence) of a free name $x$ in $P$ by
$\quotep{\procn{x}|\procn{x}}$.

Also, we will avail ourselves of the notation $x^{L}$ and $x^{R}$ to
denote injections of a name into disjoint copies of the name
space. There are numerous ways to accomplish this. One example can be
found in \cite{MeredithR05}. This notation overloads to vectors of
names: $\vec{x}^{\pi} := (x_{i}^{\pi} \; : \; 0 \leq i < |\vec{x}| )$ where $\pi \in \{L,R\}$.

We also use $P^{\Box} := P|\Box$.

In \cite{MeredithR05} an interpretation of the new operator is
given. It turns out that there are several possible interpretations
all enjoying the requisite algebraic properties of the operator (see
\cite{milner91polyadicpi}). We will therefore make liberal use of
$(\nu\; \vec{x})P$.

% subsection the_syntax_and_semantics_of_the_notation_system (end)   

\input{qm2pi.qmops} 

\input{qm2pi.sterngerlach} 

\input{qm2pi.metric} 

% section concurrent_process_calculi (end)

%\input{qm2pi.proofsketch}

% section proof sketch (end)

%\input{qm2pi.slviaknots} 

% section spatial logic via knots (end)

\input{qm2pi.conclusion}

% section conclusion (end)

%\input{qm2pi.dtcodes} 

% section wiring algorithm (end)

\input{qm2pi.ack} 

% section acknowledgments (end)

\newpage


\bibliographystyle{plain}   
\bibliography{../../biblios/main.bib}

\input{qm2pi.rhodetails}

\end{document}

 

%\ifpdf
%\usepackage[pdftex]{graphicx}
%\else
%\usepackage{graphicx}
%\fi

 % \ifpdf
%  \usepackage{pdfsync}
%  \if


%\title{Brief Article}
%\author{David F. Snyder}
%\author{L.G. Meredith}

%\address{Dept. of Math., Texas State University--San Marcos, San Marcos, TX 78666}
       
\pagestyle{empty}


\begin{document}

\lstset{language=[Objective]Caml,frame=shadowbox}

\documentclass[12pt]{llncs}
%\documentclass{jktr}

\usepackage[pdftex]{hyperref}                   
\usepackage {listings}
\usepackage {mathpartir}
\usepackage{bcprules}
%\usepackage{listings}
                       
\usepackage{graphicx} 
%\usepackage[margins=2.5cm,nohead,nofoot]{geometry}
%\usepackage{geometry}
\usepackage{amsfonts}
\usepackage{amstext}
\usepackage{latexsym}
\usepackage{amssymb}
\usepackage{color}


%\include{myPreamble}
\include{qm2pi.local} 

%\ifpdf
%\usepackage[pdftex]{graphicx}
%\else
%\usepackage{graphicx}
%\fi

 % \ifpdf
%  \usepackage{pdfsync}
%  \if


%\title{Brief Article}
%\author{David F. Snyder}
%\author{L.G. Meredith}

%\address{Dept. of Math., Texas State University--San Marcos, San Marcos, TX 78666}
       
\pagestyle{empty}


\begin{document}

\lstset{language=[Objective]Caml,frame=shadowbox}

\input{qm2pi.front}

% section front matter (end)

\input{qm2pi.intro} 
 
% section introduction (end)

% \input{qm2pi.knotations} 

% section notation (end)

\input{qm2pi.process.calculi} 

% section concurrent_process_calculi_and_spatial_logics_ (end)
    
%\input{qm2pi.knots2pi} 

%\input{qm2pi.trefoil} 

%\input{qm2pi.mainthm} 

% subsection basic_interpretation (end)

%\input{qm2pi.rho.presentation} 
\subsection{The syntax and semantics of the notation system}\label{sub:the_syntax_and_semantics_of_the_notation_system} % (fold)

We now summarize a technical presentation of the calculus that
embodies our theory of dynamics. The typical presentation of such a
calculus follows the style of giving generators and relations on
them. The grammar, below, describing term constructors, freely
generates the set of processes, $\Proc$. This set is then quotiented
by a relation known as structural congruence and it is over this set
that the notion of dynamics is expressed. This presentation is
essentially that of \cite{MeredithR05} with the addition of
polyadicity and summation. For readability we have relegated some of
the technical subtleties to an appendix.

\subsubsection{Process grammar}\label{subsub:process_grammar}

\begin{mathpar}
  \inferrule* [lab=synchronization] {} {{M} \bc \pzero \;|\; x?F \;|\; x!C }
  \and
  \inferrule* [lab=abstraction] {} {{F} \bc (x)P}
  \and
  \inferrule* [lab=concretion] {} {{C} \bc \langle Q \rangle}
  \and
  \inferrule* [lab=process] {} {{P,Q} \bc M \;| \;P|Q \;|\; @{x}}
  \and
  \inferrule* [lab=name] {} {{x} \bc \quotep{P}}
\end{mathpar} 

Note that $\vec{x}$ (resp. $\vec{P}$) denotes a vector of names
(resp. processes) of length $|\vec{x}|$ (resp. $|\vec{P}|$). We adopt
the following useful abbreviations.

\begin{mathpar}
   x?(\vec{y}).P := x.(\vec{y})P \and  x\clift{\vec{P}} := x.\clift{\vec{P}}
   \and x!(y) := \lift{x}{\dropn{y}}
   \and \Pi_{i=0}^{n-1}P_i := P_0 | \ldots | P_{n-1}
\end{mathpar}

\subsubsection{Structural congruence}

\paragraph{Free and bound names and alpha-equivalence.} At the
core of structural equivalence is alpha-equivalence which identifies
process that are the same up to a change of variable. Formally, we
recognize the distinction between free and bound names. The free names
of a process, $\freenames{P}$, may be calculated recursively as
follows:

\begin{mathpar}
\freenames{\pzero} := \emptyset
  \and \\
  \freenames{x?(y).P} := \{ x \} \cup (\freenames{P} \setminus \{ y \})
  \and 
  \freenames{x!\langle P \rangle} := \{ x \} \cup \{ P \} 
  \and \\
  \freenames{P|Q} := \freenames{P} \cup \freenames{Q}
  \and \\
  \freenames{@{x}} := \{ x \}
\end{mathpar}

$\pi$
$\quotep{\pi}$

$\freenames{-} : \pi \to \mathcal{P}(\quotep{\pi})$

\begin{eqnarray*}
  \freenames{\pzero} & := & \emptyset \\
  \freenames{x?(y).P} & := & \{ x \} \cup (\freenames{P} \setminus \{ y \}) \\
  \freenames{x!\langle P \rangle} & := & \{ x \} \cup \{ P \} \\
  \freenames{P|Q} & := & \freenames{P} \cup \freenames{Q} \\
  \freenames{\dropn{x}} & := & \{ x \}
\end{eqnarray*}

The bound names of a process, $\boundnames{P}$, are those names occurring in $P$
that are not free. For example, in $x?(y).0$, the name $x$ is free, while $y$ is bound.

\begin{mathpar}
  \inferrule* [lab=monoidal-laws] {} { P|Q \equiv Q|P \and P|0 \equiv P \and P|(Q|R) \equiv (P|Q)|R }
\end{mathpar}

\begin{mathpar}
  \inferrule* [lab=alpha-equivalence] {} { (x)P \equiv (y)P\{y/x\} \and y \not\in \freenames{P} }
\end{mathpar}

\begin{definition}
Then two processes, $P,Q$, are alpha-equivalent if $P = Q\{\vec{y}/\vec{x}\}$ for
some $\vec{x} \in \boundnames{Q},\vec{y} \in \boundnames{P}$, where $Q\{\vec{y}/\vec{x}\}$
denotes the capture-avoiding substitution of $\vec{y}$ for $\vec{x}$ in $Q$.
\end{definition}

\begin{definition}
  The {\em structural congruence} \cite{SangiorgiWalker} , $\equiv$,
  between processes is the least congruence containing
  alpha-equivalence, satisfying the abelian monoid laws
  (associativity, commutativity and $\pzero$ as identity) for parallel
  composition $|$ and for summation $+$.
\end{definition}

\subsection{Name equivalence}

We take name equivalence, written $\nameeq$, to be the smallest
equivalence relation generated by the following rules.

\begin{mathpar}
\inferrule*[lab=Quote-drop]
{ }
{ \quotep{@{x}} \nameeq x }

\inferrule*[lab=Struct-equiv]
{ P \scong Q }
{ \quotep{P} \nameeq \quotep{Q} }
\end{mathpar}

The astute reader will have noticed that the mutual recursion of names
and processes imposes a mutual recursion on alpha-equivalence and
structural equivalence via name-equivalence. Fortunately, all of this
works out pleasantly and we may calculate in the natural way, free of
concern. The reader interested in the details is referred to the
appendix \ref{appendix:rho_details}.

\subsection{Substitution}

We use $\Proc$ for the set of processes, $\QProc$ for the set of
names, and $\id{\{}\vec{y} / \vec{x} \id{\}}$ to denote partial maps,
$s : \QProc \rightarrow \QProc$. A map, $s$ lifts, uniquely, to a map
on process terms, $\widehat{s} : \Proc \rightarrow \Proc$ by the
following equations.

\begin{mathpar}
  (0) \psubstp{Q}{P} := 0 \\
  (R \juxtap S) \psubstp{Q}{P}
  :=    
  (R)\psubstp{Q}{P} \juxtap (S) \psubstp{Q}{P} \\
  (x?(y).R) \psubstp{Q}{P}    
  :=    
  (x)\substp{Q}{P} (z)\concat( (R \psubstn{z}{y}) \psubstp{Q}{P} ) \\
  (\lift{x}{R}) \psubstp{Q}{P}  
  :=
  \lift{(x)\substp{Q}{P}}{ R \psubstp{Q}{P} } \\
%   (\dropn{x})  \psubstp{Q}{P}       
%   := 
%   \left\{ 
%     \begin{array}{ccc} 
%       \dropn{\quotep{Q}} & & x \nameeq \quotep{P} \\
%       \dropn{x} & & otherwise \\
%     \end{array}
%   \right. 
  (\dropn{x})  \psubstp{Q}{P}       
  := 
  \left\{ 
    \begin{array}{ccc} 
      Q & & x \nameeq \quotep{P} \\
      \dropn{x} & & otherwise \\
    \end{array}
  \right.
\end{mathpar}
 

where

\begin{eqnarray}
  (x)\id{\{} \lpquote Q \rpquote / \lpquote P \rpquote \id{\}}            = 
  \left\{ 
    \begin{array}{ccc}
      \lpquote Q \rpquote & & x \nameeq \lpquote P \rpquote \\
      x & & otherwise \\
    \end{array}
  \right. \nonumber
\end{eqnarray}

and $z$ is chosen distinct from $\quotep{P}$, $\quotep{Q}$, the free
names in $Q$, and all the names in $R$. Our $\alpha$-equivalence will
be built in the standard way from this substitution.

\begin{remark}\label{rem:no_self_referential_names}
  One consequence of these definitions is that $\forall P. \quotep{P}
  \not\in \freenames{P}$.
\end{remark}

\subsection{ Dynamic quote: an example }

Anticipating something of what's to come, consider applying the
substitution, $\widehat{\id{\{}u / z \id{\}}}$, to the following pair
of processes, $\lift{w}{y!(z)}$ and $w[ \lpquote y!(z) \rpquote ]$.

\begin{eqnarray}
	\lift{w}{y!(z)}\widehat{\id{\{}u / z \id{\}}}
		& = &
		\lift{w}{y!(u)} \nonumber\\
	w[ \lpquote y!(z) \rpquote ] \widehat{ \id{\{}u / z \id{\}} }
		& = &
		w[ \lpquote y!(z) \rpquote ] \nonumber
\end{eqnarray}

Because the body of the process between quotes is impervious to
substitution, we get radically different answers. In fact, by
examining the first process in an input context,
e.g. $x?(z).\lift{w}{y!(z)}$, we see that the process under the lift
operator may be shaped by prefixed inputs binding a name inside it. In
this sense, the lift operator will be seen as a way to dynamically
construct processes before reifying them as names.

Finally equipped with these standard features we can present the
dynamics of the calculus.

\subsubsection{Operational semantics} 

Finally, we introduce the computational dynamics. What marks these
algebras as distinct from other more traditionally studied algebraic
structures, e.g. vector spaces or polynomial rings, is the manner in
which dynamics is captured. In traditional structures, dynamics is typically
expressed through morphisms between such structures, as in linear maps
between vector spaces or morphisms between rings. In algebras
associated with the semantics of computation, the dynamics is
expressed as part of the algebraic structure itself, through a
reduction reduction relation typically denoted by $\red$. Below, we
give a recursive presentation of this relation for the calculus used
in the encoding.

$\red \subseteq \pi \times \pi$
$\red : \pi \to \mathcal{P}(\pi)$

\begin{mathpar}
  \inferrule* [lab=Comm] { \textsf{match}( x_{src}, x_{trgt} ) } { x_{trgt}?(y)P \; | \; x_{src}!\langle {Q} \rangle \red P\{\quotep{Q}/y}\} }
  \and \\
  \inferrule* [lab=Par] {{P} \red {P}'} {{{P} | {Q}} \red {{P}' | {Q}}}
  \and
  \inferrule* [lab=Equiv]{{{P} \scong {P}'} \andalso {{P}' \red {Q}'} \andalso {{Q}' \scong {Q}}}{{P} \red {Q}}
\end{mathpar}

\begin{eqnarray*}
  match_{\equiv} (\quotep{P},\quotep{Q}) & := & P \equiv Q \\
  match_{\dagger}(\quotep{P},\quotep{Q}) & := & \forall R. P|Q \red^{*} R => R \red^{*} 0 \\
  match_{K}(\quotep{P},\quotep{Q}) & := & K \mbox{ for some context } K
\end{eqnarray*}

$u?(x)P | u!\langle Q \rangle \red P\{\quotep{Q}/x\}$

%We write $\wred$ for $\red^*$, and $P\red$ if $\exists Q $ such that $ P \red Q$.
We write $P\red$ if $\exists Q $ such that $ P \red Q$ and $P\not\red$, otherwise.

\section{Replication}

As mentioned before, it is known that replication (and hence
recursion) can be implemented in a higher-order process algebra
\cite{SangiorgiWalker}. As our first example of calculation with the
machinery thus far presented we give the construction explicitly in
the {\rhoc}.

\begin{eqnarray}
	D_{x} & := & \prefix{x}{y}{(\binpar{\outputp{x}{y}}{@{y}})} \nonumber\\
	\bangp_{x}{P} & := & \binpar{{x}!\langle{\binpar{D_{x}}{P}}\rangle}{D_{x}} \nonumber
\end{eqnarray}

\begin{eqnarray}
	\bangp_{x}{P} & & \nonumber\\
	=
	& {x}!\langle{(\prefix{x}{y}{(\outputp{x}{y} | @{y})) | P}}\rangle 
	      | \prefix{x}{y}{(\outputp{x}{y} | @{y})} & \nonumber\\
	\red
	& (\outputp{x}{y} | @{y})\substn{\quotep{(\prefix{x}{y}{(@{y} | \outputp{x}{y})) | P}}}{y} & \nonumber\\
	=
	& \outputp{x}{\quotep{(\prefix{x}{y}{(\outputp{x}{y} | @{y})) | P}}}
	  | {(\prefix{x}{y}{(\outputp{x}{y} | @{y})) | P}} & \nonumber\\
	\red
	& \ldots & \nonumber\\
	\red^*
	& P | P | \ldots & \nonumber
\end{eqnarray}

Of course, this encoding, as an implementation, runs away, unfolding
$\bangp{P}$ eagerly. A lazier and more implementable replication
operator, restricted to input-guarded processes, may be obtained as follows.

\begin{eqnarray}
\bangp{\prefix{u}{v}{P}} 
	:= 
	\binpar{\lift{x}{\prefix{u}{v}{(\binpar{D(x)}{P})}}}{D(x)} \nonumber
\end{eqnarray}

\begin{remark}
  Note that the lazier definition still does not deal with summation
  or mixed summation (i.e. sums over input and output). The reader is
  invited to construct definitions of replication that deal with these
  features. 

  Further, the definitions are parameterized in a name, $x$. Can you,
  gentle reader, make a definition that eliminates this parameter and
  guarantees no accidental interaction between the replication
  machinery and the process being replicated -- i.e. no accidental
  sharing of names used by the process to get its work done and the
  name(s) used by the replication to effect copying. This latter
  revision of the definition of replication is crucial to obtaining
  the expected identity $!!P \sim !P$.
\end{remark}

\begin{remark}\label{rem:paradoxical_combinator}
  The reader familiar with the lambda calculus will have noticed the
  similarity between $D$ and the paradoxical combinator.

  [Ed. note: the existence of this seems to suggest we have to be more
  restrictive on the set of processes and names we admit if we are to
  support no-cloning.]
\end{remark}

\subsubsection{Bisimulation}

The computational dynamics gives rise to another kind of equivalence,
the equivalence of computational behavior. As previously mentioned
this is typically captured \emph{via} some form of bisimulation.

% The notion we use in this paper is weak barbed bisimulation
% \cite{milner91polyadicpi}.

The notion we use in this paper is derived from weak barbed
bisimulation \cite{milner91polyadicpi}. 

\begin{definition}
An \emph{observation relation}, $\downarrow_{\mathcal N}$, over a set
of names, $\mathcal N$, is the smallest relation satisfying the rules
below.

\infrule[Out-barb]{y \in {\mathcal N}, \; x \nameeq y}
		  {\outputp{x}{v} \downarrow_{\mathcal N} x}
\infrule[Par-barb]{\mbox{$P\downarrow_{\mathcal N} x$ or $Q\downarrow_{\mathcal N} x$}}
		  {\binpar{P}{Q} \downarrow_{\mathcal N} x}

We write $P \Downarrow_{\mathcal N} x$ if there is $Q$ such that 
$P \wred Q$ and $Q \downarrow_{\mathcal N} x$.
\end{definition}

\begin{definition}
%\label{def.bbisim}
An  ${\mathcal N}$-\emph{barbed bisimulation} over a set of names, ${\mathcal N}$, is a symmetric binary relation 
${\mathcal S}_{\mathcal N}$ between agents such that $P\rel{S}_{\mathcal N}Q$ implies:
\begin{enumerate}
\item If $P \red P'$ then $Q \wred Q'$ and $P'\rel{S}_{\mathcal N} Q'$.
\item If $P\downarrow_{\mathcal N} x$, then $Q\Downarrow_{\mathcal N} x$.
\end{enumerate}
$P$ is ${\mathcal N}$-barbed bisimilar to $Q$, written
$P \wbbisim_{\mathcal N} Q$, if $P \rel{S}_{\mathcal N} Q$ for some ${\mathcal N}$-barbed bisimulation ${\mathcal S}_{\mathcal N}$.
\end{definition}

$\mathcal{R} \subseteq \pi \times \pi$

$P \mathcal{R} Q => \forall P'. P \red P' \Rightarrow \exists Q'. Q \red Q', P' \mathcal{R} Q'$

$P \vdash x \Rightarrow Q \vdash x$

\begin{mathpar}
  \inferrule*[lab=Out-barb]{x \nameeq y}{{y}!\langle{Q}\rangle \vdash x}
  \and
  \inferrule*[lab=Par-barb]{\mbox{$P\vdash x$ or $Q\vdash x$}}{\binpar{P}{Q} \vdash x}
\end{mathpar}

\subsubsection{Contexts}

One of the principle advantages of computational calculi like the
$\pi$-calculus is a well-defined notion of context,
contextual-equivalence and a correlation between
contextual-equivalence and notions of bisimulation. The notion of
context allows the decomposition of a process into (sub-)process and
its syntactic environment, its context. Thus, a context may be
thought of as a process with a ``hole'' (written $\Box$) in it. The
application of a context $M$ to a process $P$, written $M[P]$, is
tantamount to filling the hole in $M$ with $P$. In this paper we do
not need the full weight of this theory, but do make use of the notion
of context in the proof the main theorem. 

\begin{mathpar}
  \inferrule* [lab=summation] {} {{M_{M},M_{N}} \bc \Box \;|\; x.M_{A} \;|\; M_{M}+M_{N}}
  \and
  \inferrule* [lab=agent] {} {{M_{A}} \bc (\vec{x})M_{P} \;| \; \clift{P_0,\ldots,M_{P},\ldots,P_N}}
  \and \\
  \inferrule* [lab=process] {} {{M_{P}} \bc M_{N} \;| \;P|M_{P} }
\end{mathpar} 

\begin{mathpar}
  \inferrule* [lab=sychronization] {} {M_{N} \bc \Box \;|\; x?M_{F} \;|\; x!M_{C}}
  \and
  \inferrule* [lab=abstraction] {} {{M_{F}} \bc (x)M_{P} }
  \and
  \inferrule* [lab=concretion] {} {{M_{C}} \bc \langle M_{P} \rangle }
  \and \\
  \inferrule* [lab=process] {} {{M_{P}} \bc M_{N} \;| \;P|M_{P} }
\end{mathpar}

\begin{definition}[contextual application] Given a context $M$, and
  process $P$, we define the \emph{contextual application}, $M[P] :=
  M\{P/\Box\}$. That is, the contextual application of M to P is the
  substitution of $P$ for $\Box$ in $M$.
\end{definition}

$\meaningof{-} : L \to \mathcal{P}(\pi)$

\begin{mathpar}
  \inferrule* [lab=collection] {} {\meaningof{true} = \pi, \and \meaningof{~E} = \pi \setminus \meaningof{E}, \and \meaningof{E_{1} \& E_{2}} = \meaningof{E_{1}} \cap \meaningof{E_{2}}}
\end{mathpar}

\begin{mathpar}
  \inferrule* [lab=structure] {} {\meaningof{0} = \{ P \in \pi | P \equiv 0 \}, \and \\ \meaningof{E_1 | E_2} = \{ P \in \pi | P \equiv P_{1} | P_{2}, P_{1} \in \meaningof{E_{1}}, P_{2} \in \meaningof{E_2}\} }
\end{mathpar}

\begin{mathpar}
 \inferrule* [lab=behavior] {} {\meaningof{\langle a?b \rangle E} = \{ P \in \pi | P \equiv Q | u?(y)P', \\ \and \\\\ \and \\ \;\;\; u \in \meaningof{a}, \forall z.P'\{z/y\} \in \meaningof{E\{z/b\}}\}, \and \\ \meaningof{a!E} = \{ P \in \pi | P \equiv Q | x!\langle P' \rangle, x \in \meaningof{a} P' \in \meaningof{E}\} }
\end{mathpar}

\begin{mathpar}
 \inferrule* [lab=nominal] {} {\meaningof{\quotep{E}} = \{ \quotep{P} \in \quotep{\pi} | P \in \meaningof{E} \}, \and \meaningof{\quotep{P}} = \{ \quotep{Q} \in \quotep{\pi} | P \equiv Q \} \and \\ \meaningof{@\quotep{E}} = \{ P \in \pi | P \equiv @x, x \in \meaningof{E} \}}
\end{mathpar}

\begin{eqnarray*}
  \\
  \meaningof{-} : TS \to ST
\end{eqnarray*}

\begin{eqnarray*}
  \\
  L : TS \to ST
\end{eqnarray*}

\begin{eqnarray*}
  \\
  P \models E \iff P \in \meaningof{E}
\end{eqnarray*}

\begin{eqnarray*}
  P \approx_{L} Q \iff \forall E \in L. P \models E \iff Q \models E
\end{eqnarray*}

\begin{eqnarray*}
  P \approx_{K} Q
\end{eqnarray*}

\begin{eqnarray*}
  P \approx Q
\end{eqnarray*}

$\approx_{K} = \approx = \approx_{L}$

\subsubsection{Contextual duality}

Note that contexts extend the quotation operation to a family of
operations from processes to names. Given a context, $M$, we can
define a \emph{nominal context}, $\quotep{M}$ by $\quotep{M}[P] :=
\quotep{M[P]}$. To foreshadow what is to come we observe that these
operations enjoy a duality with processes very much like the duality
between vectors and maps from vectors to scalars.

Further, because the calculus is essentially higher-order, we have a
correspondence between contexts and processes. More specifically,
given a name $x$ and a context $M$ we can construct $M^{*}_{x}$ such
that 

\begin{mathpar}
  M^{*}_{x} | \lift{x}{P} \red M[P]
\end{mathpar}

namely,

\begin{mathpar}
  M^{*}_{x} := x?(u).M[\dropn{u}]
\end{mathpar}

The dependence of $M^{*}_{x}$ on a name makes it an abstraction, 

\begin{mathpar}
  M^{*} := (x)x?(u).M[\dropn{u}]
\end{mathpar}

\subsection{Additional notation}

It will sometimes be convenient to denote the process a name
quotes. We already have the notation $x = \quotep{P}$, but it will be
convenient to introduce an alternate notation, $\procn{x}$, when we
want to emphasize the connection to the use of the name. Note that, by
virtue of name equivalence, $\quotep{\procn{x}} \nameeq x$; so, the
notation is consistent with previous definitions.

Further, because names have structure it is possible to effect
substitutions on the basis of that structure. This means we need to
upgrade our notation for substitutions, which we accomplish by
adapting comprehension notation. Thus,

\begin{mathpar}
  P\{ y / x : x \in S \}
\end{mathpar}

is interpreted to mean the process derived from P by replacing (in a
capture-avoiding manner) each occurrence of $x$ in $S$ by $y$. For example,

\begin{mathpar}
  P\{ \quotep{\procn{x}|\procn{x}} / x : x \in \freenames{P} \}
\end{mathpar}

will replace each (occurrence) of a free name $x$ in $P$ by
$\quotep{\procn{x}|\procn{x}}$.

Also, we will avail ourselves of the notation $x^{L}$ and $x^{R}$ to
denote injections of a name into disjoint copies of the name
space. There are numerous ways to accomplish this. One example can be
found in \cite{MeredithR05}. This notation overloads to vectors of
names: $\vec{x}^{\pi} := (x_{i}^{\pi} \; : \; 0 \leq i < |\vec{x}| )$ where $\pi \in \{L,R\}$.

We also use $P^{\Box} := P|\Box$.

In \cite{MeredithR05} an interpretation of the new operator is
given. It turns out that there are several possible interpretations
all enjoying the requisite algebraic properties of the operator (see
\cite{milner91polyadicpi}). We will therefore make liberal use of
$(\nu\; \vec{x})P$.

% subsection the_syntax_and_semantics_of_the_notation_system (end)   

\input{qm2pi.qmops} 

\input{qm2pi.sterngerlach} 

\input{qm2pi.metric} 

% section concurrent_process_calculi (end)

%\input{qm2pi.proofsketch}

% section proof sketch (end)

%\input{qm2pi.slviaknots} 

% section spatial logic via knots (end)

\input{qm2pi.conclusion}

% section conclusion (end)

%\input{qm2pi.dtcodes} 

% section wiring algorithm (end)

\input{qm2pi.ack} 

% section acknowledgments (end)

\newpage


\bibliographystyle{plain}   
\bibliography{../../biblios/main.bib}

\input{qm2pi.rhodetails}

\end{document}



% section front matter (end)

\section{Introduction}\label{sec:introduction} % (fold)
In this draft of the material i am going to have to dispense with the
usual writing conventions adopted in papers on these topics. i'm going
to have adopt whatever tone i need at the time i'm writing up the
calculations. Sometimes this may be very conversational; others it may
be the barest mathematical grunts; others still it may be that i have
lifted text from one of my other papers because the exposition of some
point was better said there. i hope that my readers are not unduly put
out by this decision. i'm not doing this to flout convention or be
rebellious. i find these calculations very technically challenging. To
keep everything going technically, something has to give; i have to
let go of some cognitive burden. So, the academic writing style --
with all of its trade-offs in terms of facilitating technical
communication -- is what i'm letting go of. Perhaps subsequent drafts
can be tightened and polished, but for now, i'm going to speak as if
we were sitting together in a coffee shop with a laptop, wifi and a
pad of paper and a pencil.

So, here's what i have to say. We -- you and i, comfortably ensconced
in our coffee shop and well-equipped with our tools -- can realize and
carry out the calculations of quantum mechanics over a very different
formal theory of dynamics, a formal theory of dynamics that
corresponds to a theory of concurrent computation with
\emph{reflection}. It has the advantage that the underlying theory is
already `quantized', but supports analogues all of the continuuous
operations. Strikingly, this underlying theory has recently been
connected with a notion of metric that we can show, by calculating
together, coincides with the metric induced by the inner product.

There are a lot of reasons why you might be interested in seeing
calculations of this form. Here's why i'm interested. For the past
several centuries there has been no competitor to the ``Newtonian''
account of dynamics. As a result the predominant share of accounts of
dynamical systems and situations have had to be formulated in terms of
the Newtonian machinery. i view this as an intellectually dangerous
position to occupy. Everything, despite it's intrinsic shape, turns
into a nail to be hit with this hammer. Recently, however, the theory
of computation has matured to the point where we have candidates for
theories of dynamics that offer very different perspective on
reasoning about dynamical systems and situations. Testing these
candidates against very successful accounts of dynamical situations,
like quantum mechanics, is going to give us some sense of how mature
they are and some measure of the quality of these accounts of
dynamics.

\subsection{Summary of contributions and outline of paper}

So, we're going to develop an interpretation of the operations of
quantum mechanics normally interpreted by Hilbert spaces and
operators. We're going to do this over a theory of computation. Note
that this is very different than the usual quantum computation program
which develops notions of computation over quantum mechanics. Rather,
we are developing a story that aligns with Wheeler's slogan: It from
Bit. To do this we will first provide an account of the theory of
computation at play here. Then we will dive into a calculation-driven
interpretation of the operations of quantum mechanics.

The reason we take this approach is that -- until very recently --
there hasn't been an axiomatic account of quantum mechanics. As a
result there has been no sharp delineation of the mathematical theory
supporting interpretation of the physical theory and the physical
theory, itself. So, ambient features of the maths are free to be
exploited (or supressed) without a real accounting of their physical
relevance. There is no sharp statement ``here's the physical theory''
qua \emph{theory} and ``here's the mathematical interpretation''
enabling a judgment of how faithful the interpretation is -- apart
from experimental observation. When there is an axiomatic account we
can judge how well a given mathematical formalism supports an
interpretation of the axioms, independent of
experimentation. Likewise, we can judge how well we have captured our
physical evidence and experience with our axiomatics, independent of
any specific mathematical implementation, with accidental detail that
may or may not have physical significance. 

In lieu of a fully fleshed out and vetted axiomatic account of quantum
mechanics, interpreting the operational notions in service of modeling
physical systems will have to suffice. In other words, we are not in
the business of providing a model of Hilbert spaces and operators. We
are in the business of providing a model of quantum mechanics because
we are motivated by testing our notions of dynamics against physical
theory; and, the predictive calculations of the physical theory must
serve as the best formulation -- shy of a fully fleshed out axiomatic
account -- of the physical theory itself (as they have for scientific
theories since time immemorial). Put another way, despite a
whole-hearted commitment to an It-from-Bit ontology, we are firmly
aligned with the shut-up-and-calculate camp as the best way to obtain
results either from the physical perspective or as a quality assurance
measure of our fledgling theory of dynamics.

In detail, we present a reflective process calculus. Then we develop
intuitive correspondences between the notions available in this
calculus and the usual physical notions supporting quantum mechanical
calculations. Thus, 

\begin{table}[htp]
  \center{
    \fbox{
      \begin{tabular}{c|c}
        quantum mechanics & process calculus \\
        \hline
        scalar & name \\
        state vector & process \\
        dual & contextual duals \\
        matrix & formal sums of process-context-dual pairs \\
        orthogonality & process annihilation \\
        inner product & execution-formula + quoting
      \end{tabular}
    }
  }
  \caption{QM - process calculi correspondences}
\end{table}

Then we tighten up these intuitions to operational definitions. We
employ the Dirac notation as the best proxy we can find for an
abstract syntax of the quantum mechanical notions. The definitions we
develop put us in contact with equational constraints coming from the
theory that we demonstrate the definitions and calculations satisfy.

This puts us in a position to shut up and calculate for the
Stern-Gerlach experimental set up, showing how these predictive
calculations become calculations on processes in our theory of a
reflective process calculus.

Penultimately, we demonstrate that the notion of metric coming from
the inner product coincides with the notion of metric available from
the theory of bisimulation. This demonstration gives us the right to
think of space as arising from behavior. Finally, we consider where we
might go from the new vantage point we have obtained.

% section introduction (end) 
 
% section introduction (end)

% \documentclass[12pt]{llncs}
%\documentclass{jktr}

\usepackage[pdftex]{hyperref}                   
\usepackage {listings}
\usepackage {mathpartir}
\usepackage{bcprules}
%\usepackage{listings}
                       
\usepackage{graphicx} 
%\usepackage[margins=2.5cm,nohead,nofoot]{geometry}
%\usepackage{geometry}
\usepackage{amsfonts}
\usepackage{amstext}
\usepackage{latexsym}
\usepackage{amssymb}
\usepackage{color}


%\include{myPreamble}
\include{qm2pi.local} 

%\ifpdf
%\usepackage[pdftex]{graphicx}
%\else
%\usepackage{graphicx}
%\fi

 % \ifpdf
%  \usepackage{pdfsync}
%  \if


%\title{Brief Article}
%\author{David F. Snyder}
%\author{L.G. Meredith}

%\address{Dept. of Math., Texas State University--San Marcos, San Marcos, TX 78666}
       
\pagestyle{empty}


\begin{document}

\lstset{language=[Objective]Caml,frame=shadowbox}

\input{qm2pi.front}

% section front matter (end)

\input{qm2pi.intro} 
 
% section introduction (end)

% \input{qm2pi.knotations} 

% section notation (end)

\input{qm2pi.process.calculi} 

% section concurrent_process_calculi_and_spatial_logics_ (end)
    
%\input{qm2pi.knots2pi} 

%\input{qm2pi.trefoil} 

%\input{qm2pi.mainthm} 

% subsection basic_interpretation (end)

%\input{qm2pi.rho.presentation} 
\subsection{The syntax and semantics of the notation system}\label{sub:the_syntax_and_semantics_of_the_notation_system} % (fold)

We now summarize a technical presentation of the calculus that
embodies our theory of dynamics. The typical presentation of such a
calculus follows the style of giving generators and relations on
them. The grammar, below, describing term constructors, freely
generates the set of processes, $\Proc$. This set is then quotiented
by a relation known as structural congruence and it is over this set
that the notion of dynamics is expressed. This presentation is
essentially that of \cite{MeredithR05} with the addition of
polyadicity and summation. For readability we have relegated some of
the technical subtleties to an appendix.

\subsubsection{Process grammar}\label{subsub:process_grammar}

\begin{mathpar}
  \inferrule* [lab=synchronization] {} {{M} \bc \pzero \;|\; x?F \;|\; x!C }
  \and
  \inferrule* [lab=abstraction] {} {{F} \bc (x)P}
  \and
  \inferrule* [lab=concretion] {} {{C} \bc \langle Q \rangle}
  \and
  \inferrule* [lab=process] {} {{P,Q} \bc M \;| \;P|Q \;|\; @{x}}
  \and
  \inferrule* [lab=name] {} {{x} \bc \quotep{P}}
\end{mathpar} 

Note that $\vec{x}$ (resp. $\vec{P}$) denotes a vector of names
(resp. processes) of length $|\vec{x}|$ (resp. $|\vec{P}|$). We adopt
the following useful abbreviations.

\begin{mathpar}
   x?(\vec{y}).P := x.(\vec{y})P \and  x\clift{\vec{P}} := x.\clift{\vec{P}}
   \and x!(y) := \lift{x}{\dropn{y}}
   \and \Pi_{i=0}^{n-1}P_i := P_0 | \ldots | P_{n-1}
\end{mathpar}

\subsubsection{Structural congruence}

\paragraph{Free and bound names and alpha-equivalence.} At the
core of structural equivalence is alpha-equivalence which identifies
process that are the same up to a change of variable. Formally, we
recognize the distinction between free and bound names. The free names
of a process, $\freenames{P}$, may be calculated recursively as
follows:

\begin{mathpar}
\freenames{\pzero} := \emptyset
  \and \\
  \freenames{x?(y).P} := \{ x \} \cup (\freenames{P} \setminus \{ y \})
  \and 
  \freenames{x!\langle P \rangle} := \{ x \} \cup \{ P \} 
  \and \\
  \freenames{P|Q} := \freenames{P} \cup \freenames{Q}
  \and \\
  \freenames{@{x}} := \{ x \}
\end{mathpar}

$\pi$
$\quotep{\pi}$

$\freenames{-} : \pi \to \mathcal{P}(\quotep{\pi})$

\begin{eqnarray*}
  \freenames{\pzero} & := & \emptyset \\
  \freenames{x?(y).P} & := & \{ x \} \cup (\freenames{P} \setminus \{ y \}) \\
  \freenames{x!\langle P \rangle} & := & \{ x \} \cup \{ P \} \\
  \freenames{P|Q} & := & \freenames{P} \cup \freenames{Q} \\
  \freenames{\dropn{x}} & := & \{ x \}
\end{eqnarray*}

The bound names of a process, $\boundnames{P}$, are those names occurring in $P$
that are not free. For example, in $x?(y).0$, the name $x$ is free, while $y$ is bound.

\begin{mathpar}
  \inferrule* [lab=monoidal-laws] {} { P|Q \equiv Q|P \and P|0 \equiv P \and P|(Q|R) \equiv (P|Q)|R }
\end{mathpar}

\begin{mathpar}
  \inferrule* [lab=alpha-equivalence] {} { (x)P \equiv (y)P\{y/x\} \and y \not\in \freenames{P} }
\end{mathpar}

\begin{definition}
Then two processes, $P,Q$, are alpha-equivalent if $P = Q\{\vec{y}/\vec{x}\}$ for
some $\vec{x} \in \boundnames{Q},\vec{y} \in \boundnames{P}$, where $Q\{\vec{y}/\vec{x}\}$
denotes the capture-avoiding substitution of $\vec{y}$ for $\vec{x}$ in $Q$.
\end{definition}

\begin{definition}
  The {\em structural congruence} \cite{SangiorgiWalker} , $\equiv$,
  between processes is the least congruence containing
  alpha-equivalence, satisfying the abelian monoid laws
  (associativity, commutativity and $\pzero$ as identity) for parallel
  composition $|$ and for summation $+$.
\end{definition}

\subsection{Name equivalence}

We take name equivalence, written $\nameeq$, to be the smallest
equivalence relation generated by the following rules.

\begin{mathpar}
\inferrule*[lab=Quote-drop]
{ }
{ \quotep{@{x}} \nameeq x }

\inferrule*[lab=Struct-equiv]
{ P \scong Q }
{ \quotep{P} \nameeq \quotep{Q} }
\end{mathpar}

The astute reader will have noticed that the mutual recursion of names
and processes imposes a mutual recursion on alpha-equivalence and
structural equivalence via name-equivalence. Fortunately, all of this
works out pleasantly and we may calculate in the natural way, free of
concern. The reader interested in the details is referred to the
appendix \ref{appendix:rho_details}.

\subsection{Substitution}

We use $\Proc$ for the set of processes, $\QProc$ for the set of
names, and $\id{\{}\vec{y} / \vec{x} \id{\}}$ to denote partial maps,
$s : \QProc \rightarrow \QProc$. A map, $s$ lifts, uniquely, to a map
on process terms, $\widehat{s} : \Proc \rightarrow \Proc$ by the
following equations.

\begin{mathpar}
  (0) \psubstp{Q}{P} := 0 \\
  (R \juxtap S) \psubstp{Q}{P}
  :=    
  (R)\psubstp{Q}{P} \juxtap (S) \psubstp{Q}{P} \\
  (x?(y).R) \psubstp{Q}{P}    
  :=    
  (x)\substp{Q}{P} (z)\concat( (R \psubstn{z}{y}) \psubstp{Q}{P} ) \\
  (\lift{x}{R}) \psubstp{Q}{P}  
  :=
  \lift{(x)\substp{Q}{P}}{ R \psubstp{Q}{P} } \\
%   (\dropn{x})  \psubstp{Q}{P}       
%   := 
%   \left\{ 
%     \begin{array}{ccc} 
%       \dropn{\quotep{Q}} & & x \nameeq \quotep{P} \\
%       \dropn{x} & & otherwise \\
%     \end{array}
%   \right. 
  (\dropn{x})  \psubstp{Q}{P}       
  := 
  \left\{ 
    \begin{array}{ccc} 
      Q & & x \nameeq \quotep{P} \\
      \dropn{x} & & otherwise \\
    \end{array}
  \right.
\end{mathpar}
 

where

\begin{eqnarray}
  (x)\id{\{} \lpquote Q \rpquote / \lpquote P \rpquote \id{\}}            = 
  \left\{ 
    \begin{array}{ccc}
      \lpquote Q \rpquote & & x \nameeq \lpquote P \rpquote \\
      x & & otherwise \\
    \end{array}
  \right. \nonumber
\end{eqnarray}

and $z$ is chosen distinct from $\quotep{P}$, $\quotep{Q}$, the free
names in $Q$, and all the names in $R$. Our $\alpha$-equivalence will
be built in the standard way from this substitution.

\begin{remark}\label{rem:no_self_referential_names}
  One consequence of these definitions is that $\forall P. \quotep{P}
  \not\in \freenames{P}$.
\end{remark}

\subsection{ Dynamic quote: an example }

Anticipating something of what's to come, consider applying the
substitution, $\widehat{\id{\{}u / z \id{\}}}$, to the following pair
of processes, $\lift{w}{y!(z)}$ and $w[ \lpquote y!(z) \rpquote ]$.

\begin{eqnarray}
	\lift{w}{y!(z)}\widehat{\id{\{}u / z \id{\}}}
		& = &
		\lift{w}{y!(u)} \nonumber\\
	w[ \lpquote y!(z) \rpquote ] \widehat{ \id{\{}u / z \id{\}} }
		& = &
		w[ \lpquote y!(z) \rpquote ] \nonumber
\end{eqnarray}

Because the body of the process between quotes is impervious to
substitution, we get radically different answers. In fact, by
examining the first process in an input context,
e.g. $x?(z).\lift{w}{y!(z)}$, we see that the process under the lift
operator may be shaped by prefixed inputs binding a name inside it. In
this sense, the lift operator will be seen as a way to dynamically
construct processes before reifying them as names.

Finally equipped with these standard features we can present the
dynamics of the calculus.

\subsubsection{Operational semantics} 

Finally, we introduce the computational dynamics. What marks these
algebras as distinct from other more traditionally studied algebraic
structures, e.g. vector spaces or polynomial rings, is the manner in
which dynamics is captured. In traditional structures, dynamics is typically
expressed through morphisms between such structures, as in linear maps
between vector spaces or morphisms between rings. In algebras
associated with the semantics of computation, the dynamics is
expressed as part of the algebraic structure itself, through a
reduction reduction relation typically denoted by $\red$. Below, we
give a recursive presentation of this relation for the calculus used
in the encoding.

$\red \subseteq \pi \times \pi$
$\red : \pi \to \mathcal{P}(\pi)$

\begin{mathpar}
  \inferrule* [lab=Comm] { \textsf{match}( x_{src}, x_{trgt} ) } { x_{trgt}?(y)P \; | \; x_{src}!\langle {Q} \rangle \red P\{\quotep{Q}/y}\} }
  \and \\
  \inferrule* [lab=Par] {{P} \red {P}'} {{{P} | {Q}} \red {{P}' | {Q}}}
  \and
  \inferrule* [lab=Equiv]{{{P} \scong {P}'} \andalso {{P}' \red {Q}'} \andalso {{Q}' \scong {Q}}}{{P} \red {Q}}
\end{mathpar}

\begin{eqnarray*}
  match_{\equiv} (\quotep{P},\quotep{Q}) & := & P \equiv Q \\
  match_{\dagger}(\quotep{P},\quotep{Q}) & := & \forall R. P|Q \red^{*} R => R \red^{*} 0 \\
  match_{K}(\quotep{P},\quotep{Q}) & := & K \mbox{ for some context } K
\end{eqnarray*}

$u?(x)P | u!\langle Q \rangle \red P\{\quotep{Q}/x\}$

%We write $\wred$ for $\red^*$, and $P\red$ if $\exists Q $ such that $ P \red Q$.
We write $P\red$ if $\exists Q $ such that $ P \red Q$ and $P\not\red$, otherwise.

\section{Replication}

As mentioned before, it is known that replication (and hence
recursion) can be implemented in a higher-order process algebra
\cite{SangiorgiWalker}. As our first example of calculation with the
machinery thus far presented we give the construction explicitly in
the {\rhoc}.

\begin{eqnarray}
	D_{x} & := & \prefix{x}{y}{(\binpar{\outputp{x}{y}}{@{y}})} \nonumber\\
	\bangp_{x}{P} & := & \binpar{{x}!\langle{\binpar{D_{x}}{P}}\rangle}{D_{x}} \nonumber
\end{eqnarray}

\begin{eqnarray}
	\bangp_{x}{P} & & \nonumber\\
	=
	& {x}!\langle{(\prefix{x}{y}{(\outputp{x}{y} | @{y})) | P}}\rangle 
	      | \prefix{x}{y}{(\outputp{x}{y} | @{y})} & \nonumber\\
	\red
	& (\outputp{x}{y} | @{y})\substn{\quotep{(\prefix{x}{y}{(@{y} | \outputp{x}{y})) | P}}}{y} & \nonumber\\
	=
	& \outputp{x}{\quotep{(\prefix{x}{y}{(\outputp{x}{y} | @{y})) | P}}}
	  | {(\prefix{x}{y}{(\outputp{x}{y} | @{y})) | P}} & \nonumber\\
	\red
	& \ldots & \nonumber\\
	\red^*
	& P | P | \ldots & \nonumber
\end{eqnarray}

Of course, this encoding, as an implementation, runs away, unfolding
$\bangp{P}$ eagerly. A lazier and more implementable replication
operator, restricted to input-guarded processes, may be obtained as follows.

\begin{eqnarray}
\bangp{\prefix{u}{v}{P}} 
	:= 
	\binpar{\lift{x}{\prefix{u}{v}{(\binpar{D(x)}{P})}}}{D(x)} \nonumber
\end{eqnarray}

\begin{remark}
  Note that the lazier definition still does not deal with summation
  or mixed summation (i.e. sums over input and output). The reader is
  invited to construct definitions of replication that deal with these
  features. 

  Further, the definitions are parameterized in a name, $x$. Can you,
  gentle reader, make a definition that eliminates this parameter and
  guarantees no accidental interaction between the replication
  machinery and the process being replicated -- i.e. no accidental
  sharing of names used by the process to get its work done and the
  name(s) used by the replication to effect copying. This latter
  revision of the definition of replication is crucial to obtaining
  the expected identity $!!P \sim !P$.
\end{remark}

\begin{remark}\label{rem:paradoxical_combinator}
  The reader familiar with the lambda calculus will have noticed the
  similarity between $D$ and the paradoxical combinator.

  [Ed. note: the existence of this seems to suggest we have to be more
  restrictive on the set of processes and names we admit if we are to
  support no-cloning.]
\end{remark}

\subsubsection{Bisimulation}

The computational dynamics gives rise to another kind of equivalence,
the equivalence of computational behavior. As previously mentioned
this is typically captured \emph{via} some form of bisimulation.

% The notion we use in this paper is weak barbed bisimulation
% \cite{milner91polyadicpi}.

The notion we use in this paper is derived from weak barbed
bisimulation \cite{milner91polyadicpi}. 

\begin{definition}
An \emph{observation relation}, $\downarrow_{\mathcal N}$, over a set
of names, $\mathcal N$, is the smallest relation satisfying the rules
below.

\infrule[Out-barb]{y \in {\mathcal N}, \; x \nameeq y}
		  {\outputp{x}{v} \downarrow_{\mathcal N} x}
\infrule[Par-barb]{\mbox{$P\downarrow_{\mathcal N} x$ or $Q\downarrow_{\mathcal N} x$}}
		  {\binpar{P}{Q} \downarrow_{\mathcal N} x}

We write $P \Downarrow_{\mathcal N} x$ if there is $Q$ such that 
$P \wred Q$ and $Q \downarrow_{\mathcal N} x$.
\end{definition}

\begin{definition}
%\label{def.bbisim}
An  ${\mathcal N}$-\emph{barbed bisimulation} over a set of names, ${\mathcal N}$, is a symmetric binary relation 
${\mathcal S}_{\mathcal N}$ between agents such that $P\rel{S}_{\mathcal N}Q$ implies:
\begin{enumerate}
\item If $P \red P'$ then $Q \wred Q'$ and $P'\rel{S}_{\mathcal N} Q'$.
\item If $P\downarrow_{\mathcal N} x$, then $Q\Downarrow_{\mathcal N} x$.
\end{enumerate}
$P$ is ${\mathcal N}$-barbed bisimilar to $Q$, written
$P \wbbisim_{\mathcal N} Q$, if $P \rel{S}_{\mathcal N} Q$ for some ${\mathcal N}$-barbed bisimulation ${\mathcal S}_{\mathcal N}$.
\end{definition}

$\mathcal{R} \subseteq \pi \times \pi$

$P \mathcal{R} Q => \forall P'. P \red P' \Rightarrow \exists Q'. Q \red Q', P' \mathcal{R} Q'$

$P \vdash x \Rightarrow Q \vdash x$

\begin{mathpar}
  \inferrule*[lab=Out-barb]{x \nameeq y}{{y}!\langle{Q}\rangle \vdash x}
  \and
  \inferrule*[lab=Par-barb]{\mbox{$P\vdash x$ or $Q\vdash x$}}{\binpar{P}{Q} \vdash x}
\end{mathpar}

\subsubsection{Contexts}

One of the principle advantages of computational calculi like the
$\pi$-calculus is a well-defined notion of context,
contextual-equivalence and a correlation between
contextual-equivalence and notions of bisimulation. The notion of
context allows the decomposition of a process into (sub-)process and
its syntactic environment, its context. Thus, a context may be
thought of as a process with a ``hole'' (written $\Box$) in it. The
application of a context $M$ to a process $P$, written $M[P]$, is
tantamount to filling the hole in $M$ with $P$. In this paper we do
not need the full weight of this theory, but do make use of the notion
of context in the proof the main theorem. 

\begin{mathpar}
  \inferrule* [lab=summation] {} {{M_{M},M_{N}} \bc \Box \;|\; x.M_{A} \;|\; M_{M}+M_{N}}
  \and
  \inferrule* [lab=agent] {} {{M_{A}} \bc (\vec{x})M_{P} \;| \; \clift{P_0,\ldots,M_{P},\ldots,P_N}}
  \and \\
  \inferrule* [lab=process] {} {{M_{P}} \bc M_{N} \;| \;P|M_{P} }
\end{mathpar} 

\begin{mathpar}
  \inferrule* [lab=sychronization] {} {M_{N} \bc \Box \;|\; x?M_{F} \;|\; x!M_{C}}
  \and
  \inferrule* [lab=abstraction] {} {{M_{F}} \bc (x)M_{P} }
  \and
  \inferrule* [lab=concretion] {} {{M_{C}} \bc \langle M_{P} \rangle }
  \and \\
  \inferrule* [lab=process] {} {{M_{P}} \bc M_{N} \;| \;P|M_{P} }
\end{mathpar}

\begin{definition}[contextual application] Given a context $M$, and
  process $P$, we define the \emph{contextual application}, $M[P] :=
  M\{P/\Box\}$. That is, the contextual application of M to P is the
  substitution of $P$ for $\Box$ in $M$.
\end{definition}

$\meaningof{-} : L \to \mathcal{P}(\pi)$

\begin{mathpar}
  \inferrule* [lab=collection] {} {\meaningof{true} = \pi, \and \meaningof{~E} = \pi \setminus \meaningof{E}, \and \meaningof{E_{1} \& E_{2}} = \meaningof{E_{1}} \cap \meaningof{E_{2}}}
\end{mathpar}

\begin{mathpar}
  \inferrule* [lab=structure] {} {\meaningof{0} = \{ P \in \pi | P \equiv 0 \}, \and \\ \meaningof{E_1 | E_2} = \{ P \in \pi | P \equiv P_{1} | P_{2}, P_{1} \in \meaningof{E_{1}}, P_{2} \in \meaningof{E_2}\} }
\end{mathpar}

\begin{mathpar}
 \inferrule* [lab=behavior] {} {\meaningof{\langle a?b \rangle E} = \{ P \in \pi | P \equiv Q | u?(y)P', \\ \and \\\\ \and \\ \;\;\; u \in \meaningof{a}, \forall z.P'\{z/y\} \in \meaningof{E\{z/b\}}\}, \and \\ \meaningof{a!E} = \{ P \in \pi | P \equiv Q | x!\langle P' \rangle, x \in \meaningof{a} P' \in \meaningof{E}\} }
\end{mathpar}

\begin{mathpar}
 \inferrule* [lab=nominal] {} {\meaningof{\quotep{E}} = \{ \quotep{P} \in \quotep{\pi} | P \in \meaningof{E} \}, \and \meaningof{\quotep{P}} = \{ \quotep{Q} \in \quotep{\pi} | P \equiv Q \} \and \\ \meaningof{@\quotep{E}} = \{ P \in \pi | P \equiv @x, x \in \meaningof{E} \}}
\end{mathpar}

\begin{eqnarray*}
  \\
  \meaningof{-} : TS \to ST
\end{eqnarray*}

\begin{eqnarray*}
  \\
  L : TS \to ST
\end{eqnarray*}

\begin{eqnarray*}
  \\
  P \models E \iff P \in \meaningof{E}
\end{eqnarray*}

\begin{eqnarray*}
  P \approx_{L} Q \iff \forall E \in L. P \models E \iff Q \models E
\end{eqnarray*}

\begin{eqnarray*}
  P \approx_{K} Q
\end{eqnarray*}

\begin{eqnarray*}
  P \approx Q
\end{eqnarray*}

$\approx_{K} = \approx = \approx_{L}$

\subsubsection{Contextual duality}

Note that contexts extend the quotation operation to a family of
operations from processes to names. Given a context, $M$, we can
define a \emph{nominal context}, $\quotep{M}$ by $\quotep{M}[P] :=
\quotep{M[P]}$. To foreshadow what is to come we observe that these
operations enjoy a duality with processes very much like the duality
between vectors and maps from vectors to scalars.

Further, because the calculus is essentially higher-order, we have a
correspondence between contexts and processes. More specifically,
given a name $x$ and a context $M$ we can construct $M^{*}_{x}$ such
that 

\begin{mathpar}
  M^{*}_{x} | \lift{x}{P} \red M[P]
\end{mathpar}

namely,

\begin{mathpar}
  M^{*}_{x} := x?(u).M[\dropn{u}]
\end{mathpar}

The dependence of $M^{*}_{x}$ on a name makes it an abstraction, 

\begin{mathpar}
  M^{*} := (x)x?(u).M[\dropn{u}]
\end{mathpar}

\subsection{Additional notation}

It will sometimes be convenient to denote the process a name
quotes. We already have the notation $x = \quotep{P}$, but it will be
convenient to introduce an alternate notation, $\procn{x}$, when we
want to emphasize the connection to the use of the name. Note that, by
virtue of name equivalence, $\quotep{\procn{x}} \nameeq x$; so, the
notation is consistent with previous definitions.

Further, because names have structure it is possible to effect
substitutions on the basis of that structure. This means we need to
upgrade our notation for substitutions, which we accomplish by
adapting comprehension notation. Thus,

\begin{mathpar}
  P\{ y / x : x \in S \}
\end{mathpar}

is interpreted to mean the process derived from P by replacing (in a
capture-avoiding manner) each occurrence of $x$ in $S$ by $y$. For example,

\begin{mathpar}
  P\{ \quotep{\procn{x}|\procn{x}} / x : x \in \freenames{P} \}
\end{mathpar}

will replace each (occurrence) of a free name $x$ in $P$ by
$\quotep{\procn{x}|\procn{x}}$.

Also, we will avail ourselves of the notation $x^{L}$ and $x^{R}$ to
denote injections of a name into disjoint copies of the name
space. There are numerous ways to accomplish this. One example can be
found in \cite{MeredithR05}. This notation overloads to vectors of
names: $\vec{x}^{\pi} := (x_{i}^{\pi} \; : \; 0 \leq i < |\vec{x}| )$ where $\pi \in \{L,R\}$.

We also use $P^{\Box} := P|\Box$.

In \cite{MeredithR05} an interpretation of the new operator is
given. It turns out that there are several possible interpretations
all enjoying the requisite algebraic properties of the operator (see
\cite{milner91polyadicpi}). We will therefore make liberal use of
$(\nu\; \vec{x})P$.

% subsection the_syntax_and_semantics_of_the_notation_system (end)   

\input{qm2pi.qmops} 

\input{qm2pi.sterngerlach} 

\input{qm2pi.metric} 

% section concurrent_process_calculi (end)

%\input{qm2pi.proofsketch}

% section proof sketch (end)

%\input{qm2pi.slviaknots} 

% section spatial logic via knots (end)

\input{qm2pi.conclusion}

% section conclusion (end)

%\input{qm2pi.dtcodes} 

% section wiring algorithm (end)

\input{qm2pi.ack} 

% section acknowledgments (end)

\newpage


\bibliographystyle{plain}   
\bibliography{../../biblios/main.bib}

\input{qm2pi.rhodetails}

\end{document}

 

% section notation (end)

\input{qm2pi.process.calculi} 

% section concurrent_process_calculi_and_spatial_logics_ (end)
    
%\documentclass[12pt]{llncs}
%\documentclass{jktr}

\usepackage[pdftex]{hyperref}                   
\usepackage {listings}
\usepackage {mathpartir}
\usepackage{bcprules}
%\usepackage{listings}
                       
\usepackage{graphicx} 
%\usepackage[margins=2.5cm,nohead,nofoot]{geometry}
%\usepackage{geometry}
\usepackage{amsfonts}
\usepackage{amstext}
\usepackage{latexsym}
\usepackage{amssymb}
\usepackage{color}


%\include{myPreamble}
\include{qm2pi.local} 

%\ifpdf
%\usepackage[pdftex]{graphicx}
%\else
%\usepackage{graphicx}
%\fi

 % \ifpdf
%  \usepackage{pdfsync}
%  \if


%\title{Brief Article}
%\author{David F. Snyder}
%\author{L.G. Meredith}

%\address{Dept. of Math., Texas State University--San Marcos, San Marcos, TX 78666}
       
\pagestyle{empty}


\begin{document}

\lstset{language=[Objective]Caml,frame=shadowbox}

\input{qm2pi.front}

% section front matter (end)

\input{qm2pi.intro} 
 
% section introduction (end)

% \input{qm2pi.knotations} 

% section notation (end)

\input{qm2pi.process.calculi} 

% section concurrent_process_calculi_and_spatial_logics_ (end)
    
%\input{qm2pi.knots2pi} 

%\input{qm2pi.trefoil} 

%\input{qm2pi.mainthm} 

% subsection basic_interpretation (end)

%\input{qm2pi.rho.presentation} 
\subsection{The syntax and semantics of the notation system}\label{sub:the_syntax_and_semantics_of_the_notation_system} % (fold)

We now summarize a technical presentation of the calculus that
embodies our theory of dynamics. The typical presentation of such a
calculus follows the style of giving generators and relations on
them. The grammar, below, describing term constructors, freely
generates the set of processes, $\Proc$. This set is then quotiented
by a relation known as structural congruence and it is over this set
that the notion of dynamics is expressed. This presentation is
essentially that of \cite{MeredithR05} with the addition of
polyadicity and summation. For readability we have relegated some of
the technical subtleties to an appendix.

\subsubsection{Process grammar}\label{subsub:process_grammar}

\begin{mathpar}
  \inferrule* [lab=synchronization] {} {{M} \bc \pzero \;|\; x?F \;|\; x!C }
  \and
  \inferrule* [lab=abstraction] {} {{F} \bc (x)P}
  \and
  \inferrule* [lab=concretion] {} {{C} \bc \langle Q \rangle}
  \and
  \inferrule* [lab=process] {} {{P,Q} \bc M \;| \;P|Q \;|\; @{x}}
  \and
  \inferrule* [lab=name] {} {{x} \bc \quotep{P}}
\end{mathpar} 

Note that $\vec{x}$ (resp. $\vec{P}$) denotes a vector of names
(resp. processes) of length $|\vec{x}|$ (resp. $|\vec{P}|$). We adopt
the following useful abbreviations.

\begin{mathpar}
   x?(\vec{y}).P := x.(\vec{y})P \and  x\clift{\vec{P}} := x.\clift{\vec{P}}
   \and x!(y) := \lift{x}{\dropn{y}}
   \and \Pi_{i=0}^{n-1}P_i := P_0 | \ldots | P_{n-1}
\end{mathpar}

\subsubsection{Structural congruence}

\paragraph{Free and bound names and alpha-equivalence.} At the
core of structural equivalence is alpha-equivalence which identifies
process that are the same up to a change of variable. Formally, we
recognize the distinction between free and bound names. The free names
of a process, $\freenames{P}$, may be calculated recursively as
follows:

\begin{mathpar}
\freenames{\pzero} := \emptyset
  \and \\
  \freenames{x?(y).P} := \{ x \} \cup (\freenames{P} \setminus \{ y \})
  \and 
  \freenames{x!\langle P \rangle} := \{ x \} \cup \{ P \} 
  \and \\
  \freenames{P|Q} := \freenames{P} \cup \freenames{Q}
  \and \\
  \freenames{@{x}} := \{ x \}
\end{mathpar}

$\pi$
$\quotep{\pi}$

$\freenames{-} : \pi \to \mathcal{P}(\quotep{\pi})$

\begin{eqnarray*}
  \freenames{\pzero} & := & \emptyset \\
  \freenames{x?(y).P} & := & \{ x \} \cup (\freenames{P} \setminus \{ y \}) \\
  \freenames{x!\langle P \rangle} & := & \{ x \} \cup \{ P \} \\
  \freenames{P|Q} & := & \freenames{P} \cup \freenames{Q} \\
  \freenames{\dropn{x}} & := & \{ x \}
\end{eqnarray*}

The bound names of a process, $\boundnames{P}$, are those names occurring in $P$
that are not free. For example, in $x?(y).0$, the name $x$ is free, while $y$ is bound.

\begin{mathpar}
  \inferrule* [lab=monoidal-laws] {} { P|Q \equiv Q|P \and P|0 \equiv P \and P|(Q|R) \equiv (P|Q)|R }
\end{mathpar}

\begin{mathpar}
  \inferrule* [lab=alpha-equivalence] {} { (x)P \equiv (y)P\{y/x\} \and y \not\in \freenames{P} }
\end{mathpar}

\begin{definition}
Then two processes, $P,Q$, are alpha-equivalent if $P = Q\{\vec{y}/\vec{x}\}$ for
some $\vec{x} \in \boundnames{Q},\vec{y} \in \boundnames{P}$, where $Q\{\vec{y}/\vec{x}\}$
denotes the capture-avoiding substitution of $\vec{y}$ for $\vec{x}$ in $Q$.
\end{definition}

\begin{definition}
  The {\em structural congruence} \cite{SangiorgiWalker} , $\equiv$,
  between processes is the least congruence containing
  alpha-equivalence, satisfying the abelian monoid laws
  (associativity, commutativity and $\pzero$ as identity) for parallel
  composition $|$ and for summation $+$.
\end{definition}

\subsection{Name equivalence}

We take name equivalence, written $\nameeq$, to be the smallest
equivalence relation generated by the following rules.

\begin{mathpar}
\inferrule*[lab=Quote-drop]
{ }
{ \quotep{@{x}} \nameeq x }

\inferrule*[lab=Struct-equiv]
{ P \scong Q }
{ \quotep{P} \nameeq \quotep{Q} }
\end{mathpar}

The astute reader will have noticed that the mutual recursion of names
and processes imposes a mutual recursion on alpha-equivalence and
structural equivalence via name-equivalence. Fortunately, all of this
works out pleasantly and we may calculate in the natural way, free of
concern. The reader interested in the details is referred to the
appendix \ref{appendix:rho_details}.

\subsection{Substitution}

We use $\Proc$ for the set of processes, $\QProc$ for the set of
names, and $\id{\{}\vec{y} / \vec{x} \id{\}}$ to denote partial maps,
$s : \QProc \rightarrow \QProc$. A map, $s$ lifts, uniquely, to a map
on process terms, $\widehat{s} : \Proc \rightarrow \Proc$ by the
following equations.

\begin{mathpar}
  (0) \psubstp{Q}{P} := 0 \\
  (R \juxtap S) \psubstp{Q}{P}
  :=    
  (R)\psubstp{Q}{P} \juxtap (S) \psubstp{Q}{P} \\
  (x?(y).R) \psubstp{Q}{P}    
  :=    
  (x)\substp{Q}{P} (z)\concat( (R \psubstn{z}{y}) \psubstp{Q}{P} ) \\
  (\lift{x}{R}) \psubstp{Q}{P}  
  :=
  \lift{(x)\substp{Q}{P}}{ R \psubstp{Q}{P} } \\
%   (\dropn{x})  \psubstp{Q}{P}       
%   := 
%   \left\{ 
%     \begin{array}{ccc} 
%       \dropn{\quotep{Q}} & & x \nameeq \quotep{P} \\
%       \dropn{x} & & otherwise \\
%     \end{array}
%   \right. 
  (\dropn{x})  \psubstp{Q}{P}       
  := 
  \left\{ 
    \begin{array}{ccc} 
      Q & & x \nameeq \quotep{P} \\
      \dropn{x} & & otherwise \\
    \end{array}
  \right.
\end{mathpar}
 

where

\begin{eqnarray}
  (x)\id{\{} \lpquote Q \rpquote / \lpquote P \rpquote \id{\}}            = 
  \left\{ 
    \begin{array}{ccc}
      \lpquote Q \rpquote & & x \nameeq \lpquote P \rpquote \\
      x & & otherwise \\
    \end{array}
  \right. \nonumber
\end{eqnarray}

and $z$ is chosen distinct from $\quotep{P}$, $\quotep{Q}$, the free
names in $Q$, and all the names in $R$. Our $\alpha$-equivalence will
be built in the standard way from this substitution.

\begin{remark}\label{rem:no_self_referential_names}
  One consequence of these definitions is that $\forall P. \quotep{P}
  \not\in \freenames{P}$.
\end{remark}

\subsection{ Dynamic quote: an example }

Anticipating something of what's to come, consider applying the
substitution, $\widehat{\id{\{}u / z \id{\}}}$, to the following pair
of processes, $\lift{w}{y!(z)}$ and $w[ \lpquote y!(z) \rpquote ]$.

\begin{eqnarray}
	\lift{w}{y!(z)}\widehat{\id{\{}u / z \id{\}}}
		& = &
		\lift{w}{y!(u)} \nonumber\\
	w[ \lpquote y!(z) \rpquote ] \widehat{ \id{\{}u / z \id{\}} }
		& = &
		w[ \lpquote y!(z) \rpquote ] \nonumber
\end{eqnarray}

Because the body of the process between quotes is impervious to
substitution, we get radically different answers. In fact, by
examining the first process in an input context,
e.g. $x?(z).\lift{w}{y!(z)}$, we see that the process under the lift
operator may be shaped by prefixed inputs binding a name inside it. In
this sense, the lift operator will be seen as a way to dynamically
construct processes before reifying them as names.

Finally equipped with these standard features we can present the
dynamics of the calculus.

\subsubsection{Operational semantics} 

Finally, we introduce the computational dynamics. What marks these
algebras as distinct from other more traditionally studied algebraic
structures, e.g. vector spaces or polynomial rings, is the manner in
which dynamics is captured. In traditional structures, dynamics is typically
expressed through morphisms between such structures, as in linear maps
between vector spaces or morphisms between rings. In algebras
associated with the semantics of computation, the dynamics is
expressed as part of the algebraic structure itself, through a
reduction reduction relation typically denoted by $\red$. Below, we
give a recursive presentation of this relation for the calculus used
in the encoding.

$\red \subseteq \pi \times \pi$
$\red : \pi \to \mathcal{P}(\pi)$

\begin{mathpar}
  \inferrule* [lab=Comm] { \textsf{match}( x_{src}, x_{trgt} ) } { x_{trgt}?(y)P \; | \; x_{src}!\langle {Q} \rangle \red P\{\quotep{Q}/y}\} }
  \and \\
  \inferrule* [lab=Par] {{P} \red {P}'} {{{P} | {Q}} \red {{P}' | {Q}}}
  \and
  \inferrule* [lab=Equiv]{{{P} \scong {P}'} \andalso {{P}' \red {Q}'} \andalso {{Q}' \scong {Q}}}{{P} \red {Q}}
\end{mathpar}

\begin{eqnarray*}
  match_{\equiv} (\quotep{P},\quotep{Q}) & := & P \equiv Q \\
  match_{\dagger}(\quotep{P},\quotep{Q}) & := & \forall R. P|Q \red^{*} R => R \red^{*} 0 \\
  match_{K}(\quotep{P},\quotep{Q}) & := & K \mbox{ for some context } K
\end{eqnarray*}

$u?(x)P | u!\langle Q \rangle \red P\{\quotep{Q}/x\}$

%We write $\wred$ for $\red^*$, and $P\red$ if $\exists Q $ such that $ P \red Q$.
We write $P\red$ if $\exists Q $ such that $ P \red Q$ and $P\not\red$, otherwise.

\section{Replication}

As mentioned before, it is known that replication (and hence
recursion) can be implemented in a higher-order process algebra
\cite{SangiorgiWalker}. As our first example of calculation with the
machinery thus far presented we give the construction explicitly in
the {\rhoc}.

\begin{eqnarray}
	D_{x} & := & \prefix{x}{y}{(\binpar{\outputp{x}{y}}{@{y}})} \nonumber\\
	\bangp_{x}{P} & := & \binpar{{x}!\langle{\binpar{D_{x}}{P}}\rangle}{D_{x}} \nonumber
\end{eqnarray}

\begin{eqnarray}
	\bangp_{x}{P} & & \nonumber\\
	=
	& {x}!\langle{(\prefix{x}{y}{(\outputp{x}{y} | @{y})) | P}}\rangle 
	      | \prefix{x}{y}{(\outputp{x}{y} | @{y})} & \nonumber\\
	\red
	& (\outputp{x}{y} | @{y})\substn{\quotep{(\prefix{x}{y}{(@{y} | \outputp{x}{y})) | P}}}{y} & \nonumber\\
	=
	& \outputp{x}{\quotep{(\prefix{x}{y}{(\outputp{x}{y} | @{y})) | P}}}
	  | {(\prefix{x}{y}{(\outputp{x}{y} | @{y})) | P}} & \nonumber\\
	\red
	& \ldots & \nonumber\\
	\red^*
	& P | P | \ldots & \nonumber
\end{eqnarray}

Of course, this encoding, as an implementation, runs away, unfolding
$\bangp{P}$ eagerly. A lazier and more implementable replication
operator, restricted to input-guarded processes, may be obtained as follows.

\begin{eqnarray}
\bangp{\prefix{u}{v}{P}} 
	:= 
	\binpar{\lift{x}{\prefix{u}{v}{(\binpar{D(x)}{P})}}}{D(x)} \nonumber
\end{eqnarray}

\begin{remark}
  Note that the lazier definition still does not deal with summation
  or mixed summation (i.e. sums over input and output). The reader is
  invited to construct definitions of replication that deal with these
  features. 

  Further, the definitions are parameterized in a name, $x$. Can you,
  gentle reader, make a definition that eliminates this parameter and
  guarantees no accidental interaction between the replication
  machinery and the process being replicated -- i.e. no accidental
  sharing of names used by the process to get its work done and the
  name(s) used by the replication to effect copying. This latter
  revision of the definition of replication is crucial to obtaining
  the expected identity $!!P \sim !P$.
\end{remark}

\begin{remark}\label{rem:paradoxical_combinator}
  The reader familiar with the lambda calculus will have noticed the
  similarity between $D$ and the paradoxical combinator.

  [Ed. note: the existence of this seems to suggest we have to be more
  restrictive on the set of processes and names we admit if we are to
  support no-cloning.]
\end{remark}

\subsubsection{Bisimulation}

The computational dynamics gives rise to another kind of equivalence,
the equivalence of computational behavior. As previously mentioned
this is typically captured \emph{via} some form of bisimulation.

% The notion we use in this paper is weak barbed bisimulation
% \cite{milner91polyadicpi}.

The notion we use in this paper is derived from weak barbed
bisimulation \cite{milner91polyadicpi}. 

\begin{definition}
An \emph{observation relation}, $\downarrow_{\mathcal N}$, over a set
of names, $\mathcal N$, is the smallest relation satisfying the rules
below.

\infrule[Out-barb]{y \in {\mathcal N}, \; x \nameeq y}
		  {\outputp{x}{v} \downarrow_{\mathcal N} x}
\infrule[Par-barb]{\mbox{$P\downarrow_{\mathcal N} x$ or $Q\downarrow_{\mathcal N} x$}}
		  {\binpar{P}{Q} \downarrow_{\mathcal N} x}

We write $P \Downarrow_{\mathcal N} x$ if there is $Q$ such that 
$P \wred Q$ and $Q \downarrow_{\mathcal N} x$.
\end{definition}

\begin{definition}
%\label{def.bbisim}
An  ${\mathcal N}$-\emph{barbed bisimulation} over a set of names, ${\mathcal N}$, is a symmetric binary relation 
${\mathcal S}_{\mathcal N}$ between agents such that $P\rel{S}_{\mathcal N}Q$ implies:
\begin{enumerate}
\item If $P \red P'$ then $Q \wred Q'$ and $P'\rel{S}_{\mathcal N} Q'$.
\item If $P\downarrow_{\mathcal N} x$, then $Q\Downarrow_{\mathcal N} x$.
\end{enumerate}
$P$ is ${\mathcal N}$-barbed bisimilar to $Q$, written
$P \wbbisim_{\mathcal N} Q$, if $P \rel{S}_{\mathcal N} Q$ for some ${\mathcal N}$-barbed bisimulation ${\mathcal S}_{\mathcal N}$.
\end{definition}

$\mathcal{R} \subseteq \pi \times \pi$

$P \mathcal{R} Q => \forall P'. P \red P' \Rightarrow \exists Q'. Q \red Q', P' \mathcal{R} Q'$

$P \vdash x \Rightarrow Q \vdash x$

\begin{mathpar}
  \inferrule*[lab=Out-barb]{x \nameeq y}{{y}!\langle{Q}\rangle \vdash x}
  \and
  \inferrule*[lab=Par-barb]{\mbox{$P\vdash x$ or $Q\vdash x$}}{\binpar{P}{Q} \vdash x}
\end{mathpar}

\subsubsection{Contexts}

One of the principle advantages of computational calculi like the
$\pi$-calculus is a well-defined notion of context,
contextual-equivalence and a correlation between
contextual-equivalence and notions of bisimulation. The notion of
context allows the decomposition of a process into (sub-)process and
its syntactic environment, its context. Thus, a context may be
thought of as a process with a ``hole'' (written $\Box$) in it. The
application of a context $M$ to a process $P$, written $M[P]$, is
tantamount to filling the hole in $M$ with $P$. In this paper we do
not need the full weight of this theory, but do make use of the notion
of context in the proof the main theorem. 

\begin{mathpar}
  \inferrule* [lab=summation] {} {{M_{M},M_{N}} \bc \Box \;|\; x.M_{A} \;|\; M_{M}+M_{N}}
  \and
  \inferrule* [lab=agent] {} {{M_{A}} \bc (\vec{x})M_{P} \;| \; \clift{P_0,\ldots,M_{P},\ldots,P_N}}
  \and \\
  \inferrule* [lab=process] {} {{M_{P}} \bc M_{N} \;| \;P|M_{P} }
\end{mathpar} 

\begin{mathpar}
  \inferrule* [lab=sychronization] {} {M_{N} \bc \Box \;|\; x?M_{F} \;|\; x!M_{C}}
  \and
  \inferrule* [lab=abstraction] {} {{M_{F}} \bc (x)M_{P} }
  \and
  \inferrule* [lab=concretion] {} {{M_{C}} \bc \langle M_{P} \rangle }
  \and \\
  \inferrule* [lab=process] {} {{M_{P}} \bc M_{N} \;| \;P|M_{P} }
\end{mathpar}

\begin{definition}[contextual application] Given a context $M$, and
  process $P$, we define the \emph{contextual application}, $M[P] :=
  M\{P/\Box\}$. That is, the contextual application of M to P is the
  substitution of $P$ for $\Box$ in $M$.
\end{definition}

$\meaningof{-} : L \to \mathcal{P}(\pi)$

\begin{mathpar}
  \inferrule* [lab=collection] {} {\meaningof{true} = \pi, \and \meaningof{~E} = \pi \setminus \meaningof{E}, \and \meaningof{E_{1} \& E_{2}} = \meaningof{E_{1}} \cap \meaningof{E_{2}}}
\end{mathpar}

\begin{mathpar}
  \inferrule* [lab=structure] {} {\meaningof{0} = \{ P \in \pi | P \equiv 0 \}, \and \\ \meaningof{E_1 | E_2} = \{ P \in \pi | P \equiv P_{1} | P_{2}, P_{1} \in \meaningof{E_{1}}, P_{2} \in \meaningof{E_2}\} }
\end{mathpar}

\begin{mathpar}
 \inferrule* [lab=behavior] {} {\meaningof{\langle a?b \rangle E} = \{ P \in \pi | P \equiv Q | u?(y)P', \\ \and \\\\ \and \\ \;\;\; u \in \meaningof{a}, \forall z.P'\{z/y\} \in \meaningof{E\{z/b\}}\}, \and \\ \meaningof{a!E} = \{ P \in \pi | P \equiv Q | x!\langle P' \rangle, x \in \meaningof{a} P' \in \meaningof{E}\} }
\end{mathpar}

\begin{mathpar}
 \inferrule* [lab=nominal] {} {\meaningof{\quotep{E}} = \{ \quotep{P} \in \quotep{\pi} | P \in \meaningof{E} \}, \and \meaningof{\quotep{P}} = \{ \quotep{Q} \in \quotep{\pi} | P \equiv Q \} \and \\ \meaningof{@\quotep{E}} = \{ P \in \pi | P \equiv @x, x \in \meaningof{E} \}}
\end{mathpar}

\begin{eqnarray*}
  \\
  \meaningof{-} : TS \to ST
\end{eqnarray*}

\begin{eqnarray*}
  \\
  L : TS \to ST
\end{eqnarray*}

\begin{eqnarray*}
  \\
  P \models E \iff P \in \meaningof{E}
\end{eqnarray*}

\begin{eqnarray*}
  P \approx_{L} Q \iff \forall E \in L. P \models E \iff Q \models E
\end{eqnarray*}

\begin{eqnarray*}
  P \approx_{K} Q
\end{eqnarray*}

\begin{eqnarray*}
  P \approx Q
\end{eqnarray*}

$\approx_{K} = \approx = \approx_{L}$

\subsubsection{Contextual duality}

Note that contexts extend the quotation operation to a family of
operations from processes to names. Given a context, $M$, we can
define a \emph{nominal context}, $\quotep{M}$ by $\quotep{M}[P] :=
\quotep{M[P]}$. To foreshadow what is to come we observe that these
operations enjoy a duality with processes very much like the duality
between vectors and maps from vectors to scalars.

Further, because the calculus is essentially higher-order, we have a
correspondence between contexts and processes. More specifically,
given a name $x$ and a context $M$ we can construct $M^{*}_{x}$ such
that 

\begin{mathpar}
  M^{*}_{x} | \lift{x}{P} \red M[P]
\end{mathpar}

namely,

\begin{mathpar}
  M^{*}_{x} := x?(u).M[\dropn{u}]
\end{mathpar}

The dependence of $M^{*}_{x}$ on a name makes it an abstraction, 

\begin{mathpar}
  M^{*} := (x)x?(u).M[\dropn{u}]
\end{mathpar}

\subsection{Additional notation}

It will sometimes be convenient to denote the process a name
quotes. We already have the notation $x = \quotep{P}$, but it will be
convenient to introduce an alternate notation, $\procn{x}$, when we
want to emphasize the connection to the use of the name. Note that, by
virtue of name equivalence, $\quotep{\procn{x}} \nameeq x$; so, the
notation is consistent with previous definitions.

Further, because names have structure it is possible to effect
substitutions on the basis of that structure. This means we need to
upgrade our notation for substitutions, which we accomplish by
adapting comprehension notation. Thus,

\begin{mathpar}
  P\{ y / x : x \in S \}
\end{mathpar}

is interpreted to mean the process derived from P by replacing (in a
capture-avoiding manner) each occurrence of $x$ in $S$ by $y$. For example,

\begin{mathpar}
  P\{ \quotep{\procn{x}|\procn{x}} / x : x \in \freenames{P} \}
\end{mathpar}

will replace each (occurrence) of a free name $x$ in $P$ by
$\quotep{\procn{x}|\procn{x}}$.

Also, we will avail ourselves of the notation $x^{L}$ and $x^{R}$ to
denote injections of a name into disjoint copies of the name
space. There are numerous ways to accomplish this. One example can be
found in \cite{MeredithR05}. This notation overloads to vectors of
names: $\vec{x}^{\pi} := (x_{i}^{\pi} \; : \; 0 \leq i < |\vec{x}| )$ where $\pi \in \{L,R\}$.

We also use $P^{\Box} := P|\Box$.

In \cite{MeredithR05} an interpretation of the new operator is
given. It turns out that there are several possible interpretations
all enjoying the requisite algebraic properties of the operator (see
\cite{milner91polyadicpi}). We will therefore make liberal use of
$(\nu\; \vec{x})P$.

% subsection the_syntax_and_semantics_of_the_notation_system (end)   

\input{qm2pi.qmops} 

\input{qm2pi.sterngerlach} 

\input{qm2pi.metric} 

% section concurrent_process_calculi (end)

%\input{qm2pi.proofsketch}

% section proof sketch (end)

%\input{qm2pi.slviaknots} 

% section spatial logic via knots (end)

\input{qm2pi.conclusion}

% section conclusion (end)

%\input{qm2pi.dtcodes} 

% section wiring algorithm (end)

\input{qm2pi.ack} 

% section acknowledgments (end)

\newpage


\bibliographystyle{plain}   
\bibliography{../../biblios/main.bib}

\input{qm2pi.rhodetails}

\end{document}

 

%\documentclass[12pt]{llncs}
%\documentclass{jktr}

\usepackage[pdftex]{hyperref}                   
\usepackage {listings}
\usepackage {mathpartir}
\usepackage{bcprules}
%\usepackage{listings}
                       
\usepackage{graphicx} 
%\usepackage[margins=2.5cm,nohead,nofoot]{geometry}
%\usepackage{geometry}
\usepackage{amsfonts}
\usepackage{amstext}
\usepackage{latexsym}
\usepackage{amssymb}
\usepackage{color}


%\include{myPreamble}
\include{qm2pi.local} 

%\ifpdf
%\usepackage[pdftex]{graphicx}
%\else
%\usepackage{graphicx}
%\fi

 % \ifpdf
%  \usepackage{pdfsync}
%  \if


%\title{Brief Article}
%\author{David F. Snyder}
%\author{L.G. Meredith}

%\address{Dept. of Math., Texas State University--San Marcos, San Marcos, TX 78666}
       
\pagestyle{empty}


\begin{document}

\lstset{language=[Objective]Caml,frame=shadowbox}

\input{qm2pi.front}

% section front matter (end)

\input{qm2pi.intro} 
 
% section introduction (end)

% \input{qm2pi.knotations} 

% section notation (end)

\input{qm2pi.process.calculi} 

% section concurrent_process_calculi_and_spatial_logics_ (end)
    
%\input{qm2pi.knots2pi} 

%\input{qm2pi.trefoil} 

%\input{qm2pi.mainthm} 

% subsection basic_interpretation (end)

%\input{qm2pi.rho.presentation} 
\subsection{The syntax and semantics of the notation system}\label{sub:the_syntax_and_semantics_of_the_notation_system} % (fold)

We now summarize a technical presentation of the calculus that
embodies our theory of dynamics. The typical presentation of such a
calculus follows the style of giving generators and relations on
them. The grammar, below, describing term constructors, freely
generates the set of processes, $\Proc$. This set is then quotiented
by a relation known as structural congruence and it is over this set
that the notion of dynamics is expressed. This presentation is
essentially that of \cite{MeredithR05} with the addition of
polyadicity and summation. For readability we have relegated some of
the technical subtleties to an appendix.

\subsubsection{Process grammar}\label{subsub:process_grammar}

\begin{mathpar}
  \inferrule* [lab=synchronization] {} {{M} \bc \pzero \;|\; x?F \;|\; x!C }
  \and
  \inferrule* [lab=abstraction] {} {{F} \bc (x)P}
  \and
  \inferrule* [lab=concretion] {} {{C} \bc \langle Q \rangle}
  \and
  \inferrule* [lab=process] {} {{P,Q} \bc M \;| \;P|Q \;|\; @{x}}
  \and
  \inferrule* [lab=name] {} {{x} \bc \quotep{P}}
\end{mathpar} 

Note that $\vec{x}$ (resp. $\vec{P}$) denotes a vector of names
(resp. processes) of length $|\vec{x}|$ (resp. $|\vec{P}|$). We adopt
the following useful abbreviations.

\begin{mathpar}
   x?(\vec{y}).P := x.(\vec{y})P \and  x\clift{\vec{P}} := x.\clift{\vec{P}}
   \and x!(y) := \lift{x}{\dropn{y}}
   \and \Pi_{i=0}^{n-1}P_i := P_0 | \ldots | P_{n-1}
\end{mathpar}

\subsubsection{Structural congruence}

\paragraph{Free and bound names and alpha-equivalence.} At the
core of structural equivalence is alpha-equivalence which identifies
process that are the same up to a change of variable. Formally, we
recognize the distinction between free and bound names. The free names
of a process, $\freenames{P}$, may be calculated recursively as
follows:

\begin{mathpar}
\freenames{\pzero} := \emptyset
  \and \\
  \freenames{x?(y).P} := \{ x \} \cup (\freenames{P} \setminus \{ y \})
  \and 
  \freenames{x!\langle P \rangle} := \{ x \} \cup \{ P \} 
  \and \\
  \freenames{P|Q} := \freenames{P} \cup \freenames{Q}
  \and \\
  \freenames{@{x}} := \{ x \}
\end{mathpar}

$\pi$
$\quotep{\pi}$

$\freenames{-} : \pi \to \mathcal{P}(\quotep{\pi})$

\begin{eqnarray*}
  \freenames{\pzero} & := & \emptyset \\
  \freenames{x?(y).P} & := & \{ x \} \cup (\freenames{P} \setminus \{ y \}) \\
  \freenames{x!\langle P \rangle} & := & \{ x \} \cup \{ P \} \\
  \freenames{P|Q} & := & \freenames{P} \cup \freenames{Q} \\
  \freenames{\dropn{x}} & := & \{ x \}
\end{eqnarray*}

The bound names of a process, $\boundnames{P}$, are those names occurring in $P$
that are not free. For example, in $x?(y).0$, the name $x$ is free, while $y$ is bound.

\begin{mathpar}
  \inferrule* [lab=monoidal-laws] {} { P|Q \equiv Q|P \and P|0 \equiv P \and P|(Q|R) \equiv (P|Q)|R }
\end{mathpar}

\begin{mathpar}
  \inferrule* [lab=alpha-equivalence] {} { (x)P \equiv (y)P\{y/x\} \and y \not\in \freenames{P} }
\end{mathpar}

\begin{definition}
Then two processes, $P,Q$, are alpha-equivalent if $P = Q\{\vec{y}/\vec{x}\}$ for
some $\vec{x} \in \boundnames{Q},\vec{y} \in \boundnames{P}$, where $Q\{\vec{y}/\vec{x}\}$
denotes the capture-avoiding substitution of $\vec{y}$ for $\vec{x}$ in $Q$.
\end{definition}

\begin{definition}
  The {\em structural congruence} \cite{SangiorgiWalker} , $\equiv$,
  between processes is the least congruence containing
  alpha-equivalence, satisfying the abelian monoid laws
  (associativity, commutativity and $\pzero$ as identity) for parallel
  composition $|$ and for summation $+$.
\end{definition}

\subsection{Name equivalence}

We take name equivalence, written $\nameeq$, to be the smallest
equivalence relation generated by the following rules.

\begin{mathpar}
\inferrule*[lab=Quote-drop]
{ }
{ \quotep{@{x}} \nameeq x }

\inferrule*[lab=Struct-equiv]
{ P \scong Q }
{ \quotep{P} \nameeq \quotep{Q} }
\end{mathpar}

The astute reader will have noticed that the mutual recursion of names
and processes imposes a mutual recursion on alpha-equivalence and
structural equivalence via name-equivalence. Fortunately, all of this
works out pleasantly and we may calculate in the natural way, free of
concern. The reader interested in the details is referred to the
appendix \ref{appendix:rho_details}.

\subsection{Substitution}

We use $\Proc$ for the set of processes, $\QProc$ for the set of
names, and $\id{\{}\vec{y} / \vec{x} \id{\}}$ to denote partial maps,
$s : \QProc \rightarrow \QProc$. A map, $s$ lifts, uniquely, to a map
on process terms, $\widehat{s} : \Proc \rightarrow \Proc$ by the
following equations.

\begin{mathpar}
  (0) \psubstp{Q}{P} := 0 \\
  (R \juxtap S) \psubstp{Q}{P}
  :=    
  (R)\psubstp{Q}{P} \juxtap (S) \psubstp{Q}{P} \\
  (x?(y).R) \psubstp{Q}{P}    
  :=    
  (x)\substp{Q}{P} (z)\concat( (R \psubstn{z}{y}) \psubstp{Q}{P} ) \\
  (\lift{x}{R}) \psubstp{Q}{P}  
  :=
  \lift{(x)\substp{Q}{P}}{ R \psubstp{Q}{P} } \\
%   (\dropn{x})  \psubstp{Q}{P}       
%   := 
%   \left\{ 
%     \begin{array}{ccc} 
%       \dropn{\quotep{Q}} & & x \nameeq \quotep{P} \\
%       \dropn{x} & & otherwise \\
%     \end{array}
%   \right. 
  (\dropn{x})  \psubstp{Q}{P}       
  := 
  \left\{ 
    \begin{array}{ccc} 
      Q & & x \nameeq \quotep{P} \\
      \dropn{x} & & otherwise \\
    \end{array}
  \right.
\end{mathpar}
 

where

\begin{eqnarray}
  (x)\id{\{} \lpquote Q \rpquote / \lpquote P \rpquote \id{\}}            = 
  \left\{ 
    \begin{array}{ccc}
      \lpquote Q \rpquote & & x \nameeq \lpquote P \rpquote \\
      x & & otherwise \\
    \end{array}
  \right. \nonumber
\end{eqnarray}

and $z$ is chosen distinct from $\quotep{P}$, $\quotep{Q}$, the free
names in $Q$, and all the names in $R$. Our $\alpha$-equivalence will
be built in the standard way from this substitution.

\begin{remark}\label{rem:no_self_referential_names}
  One consequence of these definitions is that $\forall P. \quotep{P}
  \not\in \freenames{P}$.
\end{remark}

\subsection{ Dynamic quote: an example }

Anticipating something of what's to come, consider applying the
substitution, $\widehat{\id{\{}u / z \id{\}}}$, to the following pair
of processes, $\lift{w}{y!(z)}$ and $w[ \lpquote y!(z) \rpquote ]$.

\begin{eqnarray}
	\lift{w}{y!(z)}\widehat{\id{\{}u / z \id{\}}}
		& = &
		\lift{w}{y!(u)} \nonumber\\
	w[ \lpquote y!(z) \rpquote ] \widehat{ \id{\{}u / z \id{\}} }
		& = &
		w[ \lpquote y!(z) \rpquote ] \nonumber
\end{eqnarray}

Because the body of the process between quotes is impervious to
substitution, we get radically different answers. In fact, by
examining the first process in an input context,
e.g. $x?(z).\lift{w}{y!(z)}$, we see that the process under the lift
operator may be shaped by prefixed inputs binding a name inside it. In
this sense, the lift operator will be seen as a way to dynamically
construct processes before reifying them as names.

Finally equipped with these standard features we can present the
dynamics of the calculus.

\subsubsection{Operational semantics} 

Finally, we introduce the computational dynamics. What marks these
algebras as distinct from other more traditionally studied algebraic
structures, e.g. vector spaces or polynomial rings, is the manner in
which dynamics is captured. In traditional structures, dynamics is typically
expressed through morphisms between such structures, as in linear maps
between vector spaces or morphisms between rings. In algebras
associated with the semantics of computation, the dynamics is
expressed as part of the algebraic structure itself, through a
reduction reduction relation typically denoted by $\red$. Below, we
give a recursive presentation of this relation for the calculus used
in the encoding.

$\red \subseteq \pi \times \pi$
$\red : \pi \to \mathcal{P}(\pi)$

\begin{mathpar}
  \inferrule* [lab=Comm] { \textsf{match}( x_{src}, x_{trgt} ) } { x_{trgt}?(y)P \; | \; x_{src}!\langle {Q} \rangle \red P\{\quotep{Q}/y}\} }
  \and \\
  \inferrule* [lab=Par] {{P} \red {P}'} {{{P} | {Q}} \red {{P}' | {Q}}}
  \and
  \inferrule* [lab=Equiv]{{{P} \scong {P}'} \andalso {{P}' \red {Q}'} \andalso {{Q}' \scong {Q}}}{{P} \red {Q}}
\end{mathpar}

\begin{eqnarray*}
  match_{\equiv} (\quotep{P},\quotep{Q}) & := & P \equiv Q \\
  match_{\dagger}(\quotep{P},\quotep{Q}) & := & \forall R. P|Q \red^{*} R => R \red^{*} 0 \\
  match_{K}(\quotep{P},\quotep{Q}) & := & K \mbox{ for some context } K
\end{eqnarray*}

$u?(x)P | u!\langle Q \rangle \red P\{\quotep{Q}/x\}$

%We write $\wred$ for $\red^*$, and $P\red$ if $\exists Q $ such that $ P \red Q$.
We write $P\red$ if $\exists Q $ such that $ P \red Q$ and $P\not\red$, otherwise.

\section{Replication}

As mentioned before, it is known that replication (and hence
recursion) can be implemented in a higher-order process algebra
\cite{SangiorgiWalker}. As our first example of calculation with the
machinery thus far presented we give the construction explicitly in
the {\rhoc}.

\begin{eqnarray}
	D_{x} & := & \prefix{x}{y}{(\binpar{\outputp{x}{y}}{@{y}})} \nonumber\\
	\bangp_{x}{P} & := & \binpar{{x}!\langle{\binpar{D_{x}}{P}}\rangle}{D_{x}} \nonumber
\end{eqnarray}

\begin{eqnarray}
	\bangp_{x}{P} & & \nonumber\\
	=
	& {x}!\langle{(\prefix{x}{y}{(\outputp{x}{y} | @{y})) | P}}\rangle 
	      | \prefix{x}{y}{(\outputp{x}{y} | @{y})} & \nonumber\\
	\red
	& (\outputp{x}{y} | @{y})\substn{\quotep{(\prefix{x}{y}{(@{y} | \outputp{x}{y})) | P}}}{y} & \nonumber\\
	=
	& \outputp{x}{\quotep{(\prefix{x}{y}{(\outputp{x}{y} | @{y})) | P}}}
	  | {(\prefix{x}{y}{(\outputp{x}{y} | @{y})) | P}} & \nonumber\\
	\red
	& \ldots & \nonumber\\
	\red^*
	& P | P | \ldots & \nonumber
\end{eqnarray}

Of course, this encoding, as an implementation, runs away, unfolding
$\bangp{P}$ eagerly. A lazier and more implementable replication
operator, restricted to input-guarded processes, may be obtained as follows.

\begin{eqnarray}
\bangp{\prefix{u}{v}{P}} 
	:= 
	\binpar{\lift{x}{\prefix{u}{v}{(\binpar{D(x)}{P})}}}{D(x)} \nonumber
\end{eqnarray}

\begin{remark}
  Note that the lazier definition still does not deal with summation
  or mixed summation (i.e. sums over input and output). The reader is
  invited to construct definitions of replication that deal with these
  features. 

  Further, the definitions are parameterized in a name, $x$. Can you,
  gentle reader, make a definition that eliminates this parameter and
  guarantees no accidental interaction between the replication
  machinery and the process being replicated -- i.e. no accidental
  sharing of names used by the process to get its work done and the
  name(s) used by the replication to effect copying. This latter
  revision of the definition of replication is crucial to obtaining
  the expected identity $!!P \sim !P$.
\end{remark}

\begin{remark}\label{rem:paradoxical_combinator}
  The reader familiar with the lambda calculus will have noticed the
  similarity between $D$ and the paradoxical combinator.

  [Ed. note: the existence of this seems to suggest we have to be more
  restrictive on the set of processes and names we admit if we are to
  support no-cloning.]
\end{remark}

\subsubsection{Bisimulation}

The computational dynamics gives rise to another kind of equivalence,
the equivalence of computational behavior. As previously mentioned
this is typically captured \emph{via} some form of bisimulation.

% The notion we use in this paper is weak barbed bisimulation
% \cite{milner91polyadicpi}.

The notion we use in this paper is derived from weak barbed
bisimulation \cite{milner91polyadicpi}. 

\begin{definition}
An \emph{observation relation}, $\downarrow_{\mathcal N}$, over a set
of names, $\mathcal N$, is the smallest relation satisfying the rules
below.

\infrule[Out-barb]{y \in {\mathcal N}, \; x \nameeq y}
		  {\outputp{x}{v} \downarrow_{\mathcal N} x}
\infrule[Par-barb]{\mbox{$P\downarrow_{\mathcal N} x$ or $Q\downarrow_{\mathcal N} x$}}
		  {\binpar{P}{Q} \downarrow_{\mathcal N} x}

We write $P \Downarrow_{\mathcal N} x$ if there is $Q$ such that 
$P \wred Q$ and $Q \downarrow_{\mathcal N} x$.
\end{definition}

\begin{definition}
%\label{def.bbisim}
An  ${\mathcal N}$-\emph{barbed bisimulation} over a set of names, ${\mathcal N}$, is a symmetric binary relation 
${\mathcal S}_{\mathcal N}$ between agents such that $P\rel{S}_{\mathcal N}Q$ implies:
\begin{enumerate}
\item If $P \red P'$ then $Q \wred Q'$ and $P'\rel{S}_{\mathcal N} Q'$.
\item If $P\downarrow_{\mathcal N} x$, then $Q\Downarrow_{\mathcal N} x$.
\end{enumerate}
$P$ is ${\mathcal N}$-barbed bisimilar to $Q$, written
$P \wbbisim_{\mathcal N} Q$, if $P \rel{S}_{\mathcal N} Q$ for some ${\mathcal N}$-barbed bisimulation ${\mathcal S}_{\mathcal N}$.
\end{definition}

$\mathcal{R} \subseteq \pi \times \pi$

$P \mathcal{R} Q => \forall P'. P \red P' \Rightarrow \exists Q'. Q \red Q', P' \mathcal{R} Q'$

$P \vdash x \Rightarrow Q \vdash x$

\begin{mathpar}
  \inferrule*[lab=Out-barb]{x \nameeq y}{{y}!\langle{Q}\rangle \vdash x}
  \and
  \inferrule*[lab=Par-barb]{\mbox{$P\vdash x$ or $Q\vdash x$}}{\binpar{P}{Q} \vdash x}
\end{mathpar}

\subsubsection{Contexts}

One of the principle advantages of computational calculi like the
$\pi$-calculus is a well-defined notion of context,
contextual-equivalence and a correlation between
contextual-equivalence and notions of bisimulation. The notion of
context allows the decomposition of a process into (sub-)process and
its syntactic environment, its context. Thus, a context may be
thought of as a process with a ``hole'' (written $\Box$) in it. The
application of a context $M$ to a process $P$, written $M[P]$, is
tantamount to filling the hole in $M$ with $P$. In this paper we do
not need the full weight of this theory, but do make use of the notion
of context in the proof the main theorem. 

\begin{mathpar}
  \inferrule* [lab=summation] {} {{M_{M},M_{N}} \bc \Box \;|\; x.M_{A} \;|\; M_{M}+M_{N}}
  \and
  \inferrule* [lab=agent] {} {{M_{A}} \bc (\vec{x})M_{P} \;| \; \clift{P_0,\ldots,M_{P},\ldots,P_N}}
  \and \\
  \inferrule* [lab=process] {} {{M_{P}} \bc M_{N} \;| \;P|M_{P} }
\end{mathpar} 

\begin{mathpar}
  \inferrule* [lab=sychronization] {} {M_{N} \bc \Box \;|\; x?M_{F} \;|\; x!M_{C}}
  \and
  \inferrule* [lab=abstraction] {} {{M_{F}} \bc (x)M_{P} }
  \and
  \inferrule* [lab=concretion] {} {{M_{C}} \bc \langle M_{P} \rangle }
  \and \\
  \inferrule* [lab=process] {} {{M_{P}} \bc M_{N} \;| \;P|M_{P} }
\end{mathpar}

\begin{definition}[contextual application] Given a context $M$, and
  process $P$, we define the \emph{contextual application}, $M[P] :=
  M\{P/\Box\}$. That is, the contextual application of M to P is the
  substitution of $P$ for $\Box$ in $M$.
\end{definition}

$\meaningof{-} : L \to \mathcal{P}(\pi)$

\begin{mathpar}
  \inferrule* [lab=collection] {} {\meaningof{true} = \pi, \and \meaningof{~E} = \pi \setminus \meaningof{E}, \and \meaningof{E_{1} \& E_{2}} = \meaningof{E_{1}} \cap \meaningof{E_{2}}}
\end{mathpar}

\begin{mathpar}
  \inferrule* [lab=structure] {} {\meaningof{0} = \{ P \in \pi | P \equiv 0 \}, \and \\ \meaningof{E_1 | E_2} = \{ P \in \pi | P \equiv P_{1} | P_{2}, P_{1} \in \meaningof{E_{1}}, P_{2} \in \meaningof{E_2}\} }
\end{mathpar}

\begin{mathpar}
 \inferrule* [lab=behavior] {} {\meaningof{\langle a?b \rangle E} = \{ P \in \pi | P \equiv Q | u?(y)P', \\ \and \\\\ \and \\ \;\;\; u \in \meaningof{a}, \forall z.P'\{z/y\} \in \meaningof{E\{z/b\}}\}, \and \\ \meaningof{a!E} = \{ P \in \pi | P \equiv Q | x!\langle P' \rangle, x \in \meaningof{a} P' \in \meaningof{E}\} }
\end{mathpar}

\begin{mathpar}
 \inferrule* [lab=nominal] {} {\meaningof{\quotep{E}} = \{ \quotep{P} \in \quotep{\pi} | P \in \meaningof{E} \}, \and \meaningof{\quotep{P}} = \{ \quotep{Q} \in \quotep{\pi} | P \equiv Q \} \and \\ \meaningof{@\quotep{E}} = \{ P \in \pi | P \equiv @x, x \in \meaningof{E} \}}
\end{mathpar}

\begin{eqnarray*}
  \\
  \meaningof{-} : TS \to ST
\end{eqnarray*}

\begin{eqnarray*}
  \\
  L : TS \to ST
\end{eqnarray*}

\begin{eqnarray*}
  \\
  P \models E \iff P \in \meaningof{E}
\end{eqnarray*}

\begin{eqnarray*}
  P \approx_{L} Q \iff \forall E \in L. P \models E \iff Q \models E
\end{eqnarray*}

\begin{eqnarray*}
  P \approx_{K} Q
\end{eqnarray*}

\begin{eqnarray*}
  P \approx Q
\end{eqnarray*}

$\approx_{K} = \approx = \approx_{L}$

\subsubsection{Contextual duality}

Note that contexts extend the quotation operation to a family of
operations from processes to names. Given a context, $M$, we can
define a \emph{nominal context}, $\quotep{M}$ by $\quotep{M}[P] :=
\quotep{M[P]}$. To foreshadow what is to come we observe that these
operations enjoy a duality with processes very much like the duality
between vectors and maps from vectors to scalars.

Further, because the calculus is essentially higher-order, we have a
correspondence between contexts and processes. More specifically,
given a name $x$ and a context $M$ we can construct $M^{*}_{x}$ such
that 

\begin{mathpar}
  M^{*}_{x} | \lift{x}{P} \red M[P]
\end{mathpar}

namely,

\begin{mathpar}
  M^{*}_{x} := x?(u).M[\dropn{u}]
\end{mathpar}

The dependence of $M^{*}_{x}$ on a name makes it an abstraction, 

\begin{mathpar}
  M^{*} := (x)x?(u).M[\dropn{u}]
\end{mathpar}

\subsection{Additional notation}

It will sometimes be convenient to denote the process a name
quotes. We already have the notation $x = \quotep{P}$, but it will be
convenient to introduce an alternate notation, $\procn{x}$, when we
want to emphasize the connection to the use of the name. Note that, by
virtue of name equivalence, $\quotep{\procn{x}} \nameeq x$; so, the
notation is consistent with previous definitions.

Further, because names have structure it is possible to effect
substitutions on the basis of that structure. This means we need to
upgrade our notation for substitutions, which we accomplish by
adapting comprehension notation. Thus,

\begin{mathpar}
  P\{ y / x : x \in S \}
\end{mathpar}

is interpreted to mean the process derived from P by replacing (in a
capture-avoiding manner) each occurrence of $x$ in $S$ by $y$. For example,

\begin{mathpar}
  P\{ \quotep{\procn{x}|\procn{x}} / x : x \in \freenames{P} \}
\end{mathpar}

will replace each (occurrence) of a free name $x$ in $P$ by
$\quotep{\procn{x}|\procn{x}}$.

Also, we will avail ourselves of the notation $x^{L}$ and $x^{R}$ to
denote injections of a name into disjoint copies of the name
space. There are numerous ways to accomplish this. One example can be
found in \cite{MeredithR05}. This notation overloads to vectors of
names: $\vec{x}^{\pi} := (x_{i}^{\pi} \; : \; 0 \leq i < |\vec{x}| )$ where $\pi \in \{L,R\}$.

We also use $P^{\Box} := P|\Box$.

In \cite{MeredithR05} an interpretation of the new operator is
given. It turns out that there are several possible interpretations
all enjoying the requisite algebraic properties of the operator (see
\cite{milner91polyadicpi}). We will therefore make liberal use of
$(\nu\; \vec{x})P$.

% subsection the_syntax_and_semantics_of_the_notation_system (end)   

\input{qm2pi.qmops} 

\input{qm2pi.sterngerlach} 

\input{qm2pi.metric} 

% section concurrent_process_calculi (end)

%\input{qm2pi.proofsketch}

% section proof sketch (end)

%\input{qm2pi.slviaknots} 

% section spatial logic via knots (end)

\input{qm2pi.conclusion}

% section conclusion (end)

%\input{qm2pi.dtcodes} 

% section wiring algorithm (end)

\input{qm2pi.ack} 

% section acknowledgments (end)

\newpage


\bibliographystyle{plain}   
\bibliography{../../biblios/main.bib}

\input{qm2pi.rhodetails}

\end{document}

 

%\documentclass[12pt]{llncs}
%\documentclass{jktr}

\usepackage[pdftex]{hyperref}                   
\usepackage {listings}
\usepackage {mathpartir}
\usepackage{bcprules}
%\usepackage{listings}
                       
\usepackage{graphicx} 
%\usepackage[margins=2.5cm,nohead,nofoot]{geometry}
%\usepackage{geometry}
\usepackage{amsfonts}
\usepackage{amstext}
\usepackage{latexsym}
\usepackage{amssymb}
\usepackage{color}


%\include{myPreamble}
\include{qm2pi.local} 

%\ifpdf
%\usepackage[pdftex]{graphicx}
%\else
%\usepackage{graphicx}
%\fi

 % \ifpdf
%  \usepackage{pdfsync}
%  \if


%\title{Brief Article}
%\author{David F. Snyder}
%\author{L.G. Meredith}

%\address{Dept. of Math., Texas State University--San Marcos, San Marcos, TX 78666}
       
\pagestyle{empty}


\begin{document}

\lstset{language=[Objective]Caml,frame=shadowbox}

\input{qm2pi.front}

% section front matter (end)

\input{qm2pi.intro} 
 
% section introduction (end)

% \input{qm2pi.knotations} 

% section notation (end)

\input{qm2pi.process.calculi} 

% section concurrent_process_calculi_and_spatial_logics_ (end)
    
%\input{qm2pi.knots2pi} 

%\input{qm2pi.trefoil} 

%\input{qm2pi.mainthm} 

% subsection basic_interpretation (end)

%\input{qm2pi.rho.presentation} 
\subsection{The syntax and semantics of the notation system}\label{sub:the_syntax_and_semantics_of_the_notation_system} % (fold)

We now summarize a technical presentation of the calculus that
embodies our theory of dynamics. The typical presentation of such a
calculus follows the style of giving generators and relations on
them. The grammar, below, describing term constructors, freely
generates the set of processes, $\Proc$. This set is then quotiented
by a relation known as structural congruence and it is over this set
that the notion of dynamics is expressed. This presentation is
essentially that of \cite{MeredithR05} with the addition of
polyadicity and summation. For readability we have relegated some of
the technical subtleties to an appendix.

\subsubsection{Process grammar}\label{subsub:process_grammar}

\begin{mathpar}
  \inferrule* [lab=synchronization] {} {{M} \bc \pzero \;|\; x?F \;|\; x!C }
  \and
  \inferrule* [lab=abstraction] {} {{F} \bc (x)P}
  \and
  \inferrule* [lab=concretion] {} {{C} \bc \langle Q \rangle}
  \and
  \inferrule* [lab=process] {} {{P,Q} \bc M \;| \;P|Q \;|\; @{x}}
  \and
  \inferrule* [lab=name] {} {{x} \bc \quotep{P}}
\end{mathpar} 

Note that $\vec{x}$ (resp. $\vec{P}$) denotes a vector of names
(resp. processes) of length $|\vec{x}|$ (resp. $|\vec{P}|$). We adopt
the following useful abbreviations.

\begin{mathpar}
   x?(\vec{y}).P := x.(\vec{y})P \and  x\clift{\vec{P}} := x.\clift{\vec{P}}
   \and x!(y) := \lift{x}{\dropn{y}}
   \and \Pi_{i=0}^{n-1}P_i := P_0 | \ldots | P_{n-1}
\end{mathpar}

\subsubsection{Structural congruence}

\paragraph{Free and bound names and alpha-equivalence.} At the
core of structural equivalence is alpha-equivalence which identifies
process that are the same up to a change of variable. Formally, we
recognize the distinction between free and bound names. The free names
of a process, $\freenames{P}$, may be calculated recursively as
follows:

\begin{mathpar}
\freenames{\pzero} := \emptyset
  \and \\
  \freenames{x?(y).P} := \{ x \} \cup (\freenames{P} \setminus \{ y \})
  \and 
  \freenames{x!\langle P \rangle} := \{ x \} \cup \{ P \} 
  \and \\
  \freenames{P|Q} := \freenames{P} \cup \freenames{Q}
  \and \\
  \freenames{@{x}} := \{ x \}
\end{mathpar}

$\pi$
$\quotep{\pi}$

$\freenames{-} : \pi \to \mathcal{P}(\quotep{\pi})$

\begin{eqnarray*}
  \freenames{\pzero} & := & \emptyset \\
  \freenames{x?(y).P} & := & \{ x \} \cup (\freenames{P} \setminus \{ y \}) \\
  \freenames{x!\langle P \rangle} & := & \{ x \} \cup \{ P \} \\
  \freenames{P|Q} & := & \freenames{P} \cup \freenames{Q} \\
  \freenames{\dropn{x}} & := & \{ x \}
\end{eqnarray*}

The bound names of a process, $\boundnames{P}$, are those names occurring in $P$
that are not free. For example, in $x?(y).0$, the name $x$ is free, while $y$ is bound.

\begin{mathpar}
  \inferrule* [lab=monoidal-laws] {} { P|Q \equiv Q|P \and P|0 \equiv P \and P|(Q|R) \equiv (P|Q)|R }
\end{mathpar}

\begin{mathpar}
  \inferrule* [lab=alpha-equivalence] {} { (x)P \equiv (y)P\{y/x\} \and y \not\in \freenames{P} }
\end{mathpar}

\begin{definition}
Then two processes, $P,Q$, are alpha-equivalent if $P = Q\{\vec{y}/\vec{x}\}$ for
some $\vec{x} \in \boundnames{Q},\vec{y} \in \boundnames{P}$, where $Q\{\vec{y}/\vec{x}\}$
denotes the capture-avoiding substitution of $\vec{y}$ for $\vec{x}$ in $Q$.
\end{definition}

\begin{definition}
  The {\em structural congruence} \cite{SangiorgiWalker} , $\equiv$,
  between processes is the least congruence containing
  alpha-equivalence, satisfying the abelian monoid laws
  (associativity, commutativity and $\pzero$ as identity) for parallel
  composition $|$ and for summation $+$.
\end{definition}

\subsection{Name equivalence}

We take name equivalence, written $\nameeq$, to be the smallest
equivalence relation generated by the following rules.

\begin{mathpar}
\inferrule*[lab=Quote-drop]
{ }
{ \quotep{@{x}} \nameeq x }

\inferrule*[lab=Struct-equiv]
{ P \scong Q }
{ \quotep{P} \nameeq \quotep{Q} }
\end{mathpar}

The astute reader will have noticed that the mutual recursion of names
and processes imposes a mutual recursion on alpha-equivalence and
structural equivalence via name-equivalence. Fortunately, all of this
works out pleasantly and we may calculate in the natural way, free of
concern. The reader interested in the details is referred to the
appendix \ref{appendix:rho_details}.

\subsection{Substitution}

We use $\Proc$ for the set of processes, $\QProc$ for the set of
names, and $\id{\{}\vec{y} / \vec{x} \id{\}}$ to denote partial maps,
$s : \QProc \rightarrow \QProc$. A map, $s$ lifts, uniquely, to a map
on process terms, $\widehat{s} : \Proc \rightarrow \Proc$ by the
following equations.

\begin{mathpar}
  (0) \psubstp{Q}{P} := 0 \\
  (R \juxtap S) \psubstp{Q}{P}
  :=    
  (R)\psubstp{Q}{P} \juxtap (S) \psubstp{Q}{P} \\
  (x?(y).R) \psubstp{Q}{P}    
  :=    
  (x)\substp{Q}{P} (z)\concat( (R \psubstn{z}{y}) \psubstp{Q}{P} ) \\
  (\lift{x}{R}) \psubstp{Q}{P}  
  :=
  \lift{(x)\substp{Q}{P}}{ R \psubstp{Q}{P} } \\
%   (\dropn{x})  \psubstp{Q}{P}       
%   := 
%   \left\{ 
%     \begin{array}{ccc} 
%       \dropn{\quotep{Q}} & & x \nameeq \quotep{P} \\
%       \dropn{x} & & otherwise \\
%     \end{array}
%   \right. 
  (\dropn{x})  \psubstp{Q}{P}       
  := 
  \left\{ 
    \begin{array}{ccc} 
      Q & & x \nameeq \quotep{P} \\
      \dropn{x} & & otherwise \\
    \end{array}
  \right.
\end{mathpar}
 

where

\begin{eqnarray}
  (x)\id{\{} \lpquote Q \rpquote / \lpquote P \rpquote \id{\}}            = 
  \left\{ 
    \begin{array}{ccc}
      \lpquote Q \rpquote & & x \nameeq \lpquote P \rpquote \\
      x & & otherwise \\
    \end{array}
  \right. \nonumber
\end{eqnarray}

and $z$ is chosen distinct from $\quotep{P}$, $\quotep{Q}$, the free
names in $Q$, and all the names in $R$. Our $\alpha$-equivalence will
be built in the standard way from this substitution.

\begin{remark}\label{rem:no_self_referential_names}
  One consequence of these definitions is that $\forall P. \quotep{P}
  \not\in \freenames{P}$.
\end{remark}

\subsection{ Dynamic quote: an example }

Anticipating something of what's to come, consider applying the
substitution, $\widehat{\id{\{}u / z \id{\}}}$, to the following pair
of processes, $\lift{w}{y!(z)}$ and $w[ \lpquote y!(z) \rpquote ]$.

\begin{eqnarray}
	\lift{w}{y!(z)}\widehat{\id{\{}u / z \id{\}}}
		& = &
		\lift{w}{y!(u)} \nonumber\\
	w[ \lpquote y!(z) \rpquote ] \widehat{ \id{\{}u / z \id{\}} }
		& = &
		w[ \lpquote y!(z) \rpquote ] \nonumber
\end{eqnarray}

Because the body of the process between quotes is impervious to
substitution, we get radically different answers. In fact, by
examining the first process in an input context,
e.g. $x?(z).\lift{w}{y!(z)}$, we see that the process under the lift
operator may be shaped by prefixed inputs binding a name inside it. In
this sense, the lift operator will be seen as a way to dynamically
construct processes before reifying them as names.

Finally equipped with these standard features we can present the
dynamics of the calculus.

\subsubsection{Operational semantics} 

Finally, we introduce the computational dynamics. What marks these
algebras as distinct from other more traditionally studied algebraic
structures, e.g. vector spaces or polynomial rings, is the manner in
which dynamics is captured. In traditional structures, dynamics is typically
expressed through morphisms between such structures, as in linear maps
between vector spaces or morphisms between rings. In algebras
associated with the semantics of computation, the dynamics is
expressed as part of the algebraic structure itself, through a
reduction reduction relation typically denoted by $\red$. Below, we
give a recursive presentation of this relation for the calculus used
in the encoding.

$\red \subseteq \pi \times \pi$
$\red : \pi \to \mathcal{P}(\pi)$

\begin{mathpar}
  \inferrule* [lab=Comm] { \textsf{match}( x_{src}, x_{trgt} ) } { x_{trgt}?(y)P \; | \; x_{src}!\langle {Q} \rangle \red P\{\quotep{Q}/y}\} }
  \and \\
  \inferrule* [lab=Par] {{P} \red {P}'} {{{P} | {Q}} \red {{P}' | {Q}}}
  \and
  \inferrule* [lab=Equiv]{{{P} \scong {P}'} \andalso {{P}' \red {Q}'} \andalso {{Q}' \scong {Q}}}{{P} \red {Q}}
\end{mathpar}

\begin{eqnarray*}
  match_{\equiv} (\quotep{P},\quotep{Q}) & := & P \equiv Q \\
  match_{\dagger}(\quotep{P},\quotep{Q}) & := & \forall R. P|Q \red^{*} R => R \red^{*} 0 \\
  match_{K}(\quotep{P},\quotep{Q}) & := & K \mbox{ for some context } K
\end{eqnarray*}

$u?(x)P | u!\langle Q \rangle \red P\{\quotep{Q}/x\}$

%We write $\wred$ for $\red^*$, and $P\red$ if $\exists Q $ such that $ P \red Q$.
We write $P\red$ if $\exists Q $ such that $ P \red Q$ and $P\not\red$, otherwise.

\section{Replication}

As mentioned before, it is known that replication (and hence
recursion) can be implemented in a higher-order process algebra
\cite{SangiorgiWalker}. As our first example of calculation with the
machinery thus far presented we give the construction explicitly in
the {\rhoc}.

\begin{eqnarray}
	D_{x} & := & \prefix{x}{y}{(\binpar{\outputp{x}{y}}{@{y}})} \nonumber\\
	\bangp_{x}{P} & := & \binpar{{x}!\langle{\binpar{D_{x}}{P}}\rangle}{D_{x}} \nonumber
\end{eqnarray}

\begin{eqnarray}
	\bangp_{x}{P} & & \nonumber\\
	=
	& {x}!\langle{(\prefix{x}{y}{(\outputp{x}{y} | @{y})) | P}}\rangle 
	      | \prefix{x}{y}{(\outputp{x}{y} | @{y})} & \nonumber\\
	\red
	& (\outputp{x}{y} | @{y})\substn{\quotep{(\prefix{x}{y}{(@{y} | \outputp{x}{y})) | P}}}{y} & \nonumber\\
	=
	& \outputp{x}{\quotep{(\prefix{x}{y}{(\outputp{x}{y} | @{y})) | P}}}
	  | {(\prefix{x}{y}{(\outputp{x}{y} | @{y})) | P}} & \nonumber\\
	\red
	& \ldots & \nonumber\\
	\red^*
	& P | P | \ldots & \nonumber
\end{eqnarray}

Of course, this encoding, as an implementation, runs away, unfolding
$\bangp{P}$ eagerly. A lazier and more implementable replication
operator, restricted to input-guarded processes, may be obtained as follows.

\begin{eqnarray}
\bangp{\prefix{u}{v}{P}} 
	:= 
	\binpar{\lift{x}{\prefix{u}{v}{(\binpar{D(x)}{P})}}}{D(x)} \nonumber
\end{eqnarray}

\begin{remark}
  Note that the lazier definition still does not deal with summation
  or mixed summation (i.e. sums over input and output). The reader is
  invited to construct definitions of replication that deal with these
  features. 

  Further, the definitions are parameterized in a name, $x$. Can you,
  gentle reader, make a definition that eliminates this parameter and
  guarantees no accidental interaction between the replication
  machinery and the process being replicated -- i.e. no accidental
  sharing of names used by the process to get its work done and the
  name(s) used by the replication to effect copying. This latter
  revision of the definition of replication is crucial to obtaining
  the expected identity $!!P \sim !P$.
\end{remark}

\begin{remark}\label{rem:paradoxical_combinator}
  The reader familiar with the lambda calculus will have noticed the
  similarity between $D$ and the paradoxical combinator.

  [Ed. note: the existence of this seems to suggest we have to be more
  restrictive on the set of processes and names we admit if we are to
  support no-cloning.]
\end{remark}

\subsubsection{Bisimulation}

The computational dynamics gives rise to another kind of equivalence,
the equivalence of computational behavior. As previously mentioned
this is typically captured \emph{via} some form of bisimulation.

% The notion we use in this paper is weak barbed bisimulation
% \cite{milner91polyadicpi}.

The notion we use in this paper is derived from weak barbed
bisimulation \cite{milner91polyadicpi}. 

\begin{definition}
An \emph{observation relation}, $\downarrow_{\mathcal N}$, over a set
of names, $\mathcal N$, is the smallest relation satisfying the rules
below.

\infrule[Out-barb]{y \in {\mathcal N}, \; x \nameeq y}
		  {\outputp{x}{v} \downarrow_{\mathcal N} x}
\infrule[Par-barb]{\mbox{$P\downarrow_{\mathcal N} x$ or $Q\downarrow_{\mathcal N} x$}}
		  {\binpar{P}{Q} \downarrow_{\mathcal N} x}

We write $P \Downarrow_{\mathcal N} x$ if there is $Q$ such that 
$P \wred Q$ and $Q \downarrow_{\mathcal N} x$.
\end{definition}

\begin{definition}
%\label{def.bbisim}
An  ${\mathcal N}$-\emph{barbed bisimulation} over a set of names, ${\mathcal N}$, is a symmetric binary relation 
${\mathcal S}_{\mathcal N}$ between agents such that $P\rel{S}_{\mathcal N}Q$ implies:
\begin{enumerate}
\item If $P \red P'$ then $Q \wred Q'$ and $P'\rel{S}_{\mathcal N} Q'$.
\item If $P\downarrow_{\mathcal N} x$, then $Q\Downarrow_{\mathcal N} x$.
\end{enumerate}
$P$ is ${\mathcal N}$-barbed bisimilar to $Q$, written
$P \wbbisim_{\mathcal N} Q$, if $P \rel{S}_{\mathcal N} Q$ for some ${\mathcal N}$-barbed bisimulation ${\mathcal S}_{\mathcal N}$.
\end{definition}

$\mathcal{R} \subseteq \pi \times \pi$

$P \mathcal{R} Q => \forall P'. P \red P' \Rightarrow \exists Q'. Q \red Q', P' \mathcal{R} Q'$

$P \vdash x \Rightarrow Q \vdash x$

\begin{mathpar}
  \inferrule*[lab=Out-barb]{x \nameeq y}{{y}!\langle{Q}\rangle \vdash x}
  \and
  \inferrule*[lab=Par-barb]{\mbox{$P\vdash x$ or $Q\vdash x$}}{\binpar{P}{Q} \vdash x}
\end{mathpar}

\subsubsection{Contexts}

One of the principle advantages of computational calculi like the
$\pi$-calculus is a well-defined notion of context,
contextual-equivalence and a correlation between
contextual-equivalence and notions of bisimulation. The notion of
context allows the decomposition of a process into (sub-)process and
its syntactic environment, its context. Thus, a context may be
thought of as a process with a ``hole'' (written $\Box$) in it. The
application of a context $M$ to a process $P$, written $M[P]$, is
tantamount to filling the hole in $M$ with $P$. In this paper we do
not need the full weight of this theory, but do make use of the notion
of context in the proof the main theorem. 

\begin{mathpar}
  \inferrule* [lab=summation] {} {{M_{M},M_{N}} \bc \Box \;|\; x.M_{A} \;|\; M_{M}+M_{N}}
  \and
  \inferrule* [lab=agent] {} {{M_{A}} \bc (\vec{x})M_{P} \;| \; \clift{P_0,\ldots,M_{P},\ldots,P_N}}
  \and \\
  \inferrule* [lab=process] {} {{M_{P}} \bc M_{N} \;| \;P|M_{P} }
\end{mathpar} 

\begin{mathpar}
  \inferrule* [lab=sychronization] {} {M_{N} \bc \Box \;|\; x?M_{F} \;|\; x!M_{C}}
  \and
  \inferrule* [lab=abstraction] {} {{M_{F}} \bc (x)M_{P} }
  \and
  \inferrule* [lab=concretion] {} {{M_{C}} \bc \langle M_{P} \rangle }
  \and \\
  \inferrule* [lab=process] {} {{M_{P}} \bc M_{N} \;| \;P|M_{P} }
\end{mathpar}

\begin{definition}[contextual application] Given a context $M$, and
  process $P$, we define the \emph{contextual application}, $M[P] :=
  M\{P/\Box\}$. That is, the contextual application of M to P is the
  substitution of $P$ for $\Box$ in $M$.
\end{definition}

$\meaningof{-} : L \to \mathcal{P}(\pi)$

\begin{mathpar}
  \inferrule* [lab=collection] {} {\meaningof{true} = \pi, \and \meaningof{~E} = \pi \setminus \meaningof{E}, \and \meaningof{E_{1} \& E_{2}} = \meaningof{E_{1}} \cap \meaningof{E_{2}}}
\end{mathpar}

\begin{mathpar}
  \inferrule* [lab=structure] {} {\meaningof{0} = \{ P \in \pi | P \equiv 0 \}, \and \\ \meaningof{E_1 | E_2} = \{ P \in \pi | P \equiv P_{1} | P_{2}, P_{1} \in \meaningof{E_{1}}, P_{2} \in \meaningof{E_2}\} }
\end{mathpar}

\begin{mathpar}
 \inferrule* [lab=behavior] {} {\meaningof{\langle a?b \rangle E} = \{ P \in \pi | P \equiv Q | u?(y)P', \\ \and \\\\ \and \\ \;\;\; u \in \meaningof{a}, \forall z.P'\{z/y\} \in \meaningof{E\{z/b\}}\}, \and \\ \meaningof{a!E} = \{ P \in \pi | P \equiv Q | x!\langle P' \rangle, x \in \meaningof{a} P' \in \meaningof{E}\} }
\end{mathpar}

\begin{mathpar}
 \inferrule* [lab=nominal] {} {\meaningof{\quotep{E}} = \{ \quotep{P} \in \quotep{\pi} | P \in \meaningof{E} \}, \and \meaningof{\quotep{P}} = \{ \quotep{Q} \in \quotep{\pi} | P \equiv Q \} \and \\ \meaningof{@\quotep{E}} = \{ P \in \pi | P \equiv @x, x \in \meaningof{E} \}}
\end{mathpar}

\begin{eqnarray*}
  \\
  \meaningof{-} : TS \to ST
\end{eqnarray*}

\begin{eqnarray*}
  \\
  L : TS \to ST
\end{eqnarray*}

\begin{eqnarray*}
  \\
  P \models E \iff P \in \meaningof{E}
\end{eqnarray*}

\begin{eqnarray*}
  P \approx_{L} Q \iff \forall E \in L. P \models E \iff Q \models E
\end{eqnarray*}

\begin{eqnarray*}
  P \approx_{K} Q
\end{eqnarray*}

\begin{eqnarray*}
  P \approx Q
\end{eqnarray*}

$\approx_{K} = \approx = \approx_{L}$

\subsubsection{Contextual duality}

Note that contexts extend the quotation operation to a family of
operations from processes to names. Given a context, $M$, we can
define a \emph{nominal context}, $\quotep{M}$ by $\quotep{M}[P] :=
\quotep{M[P]}$. To foreshadow what is to come we observe that these
operations enjoy a duality with processes very much like the duality
between vectors and maps from vectors to scalars.

Further, because the calculus is essentially higher-order, we have a
correspondence between contexts and processes. More specifically,
given a name $x$ and a context $M$ we can construct $M^{*}_{x}$ such
that 

\begin{mathpar}
  M^{*}_{x} | \lift{x}{P} \red M[P]
\end{mathpar}

namely,

\begin{mathpar}
  M^{*}_{x} := x?(u).M[\dropn{u}]
\end{mathpar}

The dependence of $M^{*}_{x}$ on a name makes it an abstraction, 

\begin{mathpar}
  M^{*} := (x)x?(u).M[\dropn{u}]
\end{mathpar}

\subsection{Additional notation}

It will sometimes be convenient to denote the process a name
quotes. We already have the notation $x = \quotep{P}$, but it will be
convenient to introduce an alternate notation, $\procn{x}$, when we
want to emphasize the connection to the use of the name. Note that, by
virtue of name equivalence, $\quotep{\procn{x}} \nameeq x$; so, the
notation is consistent with previous definitions.

Further, because names have structure it is possible to effect
substitutions on the basis of that structure. This means we need to
upgrade our notation for substitutions, which we accomplish by
adapting comprehension notation. Thus,

\begin{mathpar}
  P\{ y / x : x \in S \}
\end{mathpar}

is interpreted to mean the process derived from P by replacing (in a
capture-avoiding manner) each occurrence of $x$ in $S$ by $y$. For example,

\begin{mathpar}
  P\{ \quotep{\procn{x}|\procn{x}} / x : x \in \freenames{P} \}
\end{mathpar}

will replace each (occurrence) of a free name $x$ in $P$ by
$\quotep{\procn{x}|\procn{x}}$.

Also, we will avail ourselves of the notation $x^{L}$ and $x^{R}$ to
denote injections of a name into disjoint copies of the name
space. There are numerous ways to accomplish this. One example can be
found in \cite{MeredithR05}. This notation overloads to vectors of
names: $\vec{x}^{\pi} := (x_{i}^{\pi} \; : \; 0 \leq i < |\vec{x}| )$ where $\pi \in \{L,R\}$.

We also use $P^{\Box} := P|\Box$.

In \cite{MeredithR05} an interpretation of the new operator is
given. It turns out that there are several possible interpretations
all enjoying the requisite algebraic properties of the operator (see
\cite{milner91polyadicpi}). We will therefore make liberal use of
$(\nu\; \vec{x})P$.

% subsection the_syntax_and_semantics_of_the_notation_system (end)   

\input{qm2pi.qmops} 

\input{qm2pi.sterngerlach} 

\input{qm2pi.metric} 

% section concurrent_process_calculi (end)

%\input{qm2pi.proofsketch}

% section proof sketch (end)

%\input{qm2pi.slviaknots} 

% section spatial logic via knots (end)

\input{qm2pi.conclusion}

% section conclusion (end)

%\input{qm2pi.dtcodes} 

% section wiring algorithm (end)

\input{qm2pi.ack} 

% section acknowledgments (end)

\newpage


\bibliographystyle{plain}   
\bibliography{../../biblios/main.bib}

\input{qm2pi.rhodetails}

\end{document}

 

% subsection basic_interpretation (end)

%\input{qm2pi.rho.presentation} 
\subsection{The syntax and semantics of the notation system}\label{sub:the_syntax_and_semantics_of_the_notation_system} % (fold)

We now summarize a technical presentation of the calculus that
embodies our theory of dynamics. The typical presentation of such a
calculus follows the style of giving generators and relations on
them. The grammar, below, describing term constructors, freely
generates the set of processes, $\Proc$. This set is then quotiented
by a relation known as structural congruence and it is over this set
that the notion of dynamics is expressed. This presentation is
essentially that of \cite{MeredithR05} with the addition of
polyadicity and summation. For readability we have relegated some of
the technical subtleties to an appendix.

\subsubsection{Process grammar}\label{subsub:process_grammar}

\begin{mathpar}
  \inferrule* [lab=synchronization] {} {{M} \bc \pzero \;|\; x?F \;|\; x!C }
  \and
  \inferrule* [lab=abstraction] {} {{F} \bc (x)P}
  \and
  \inferrule* [lab=concretion] {} {{C} \bc \langle Q \rangle}
  \and
  \inferrule* [lab=process] {} {{P,Q} \bc M \;| \;P|Q \;|\; @{x}}
  \and
  \inferrule* [lab=name] {} {{x} \bc \quotep{P}}
\end{mathpar} 

Note that $\vec{x}$ (resp. $\vec{P}$) denotes a vector of names
(resp. processes) of length $|\vec{x}|$ (resp. $|\vec{P}|$). We adopt
the following useful abbreviations.

\begin{mathpar}
   x?(\vec{y}).P := x.(\vec{y})P \and  x\clift{\vec{P}} := x.\clift{\vec{P}}
   \and x!(y) := \lift{x}{\dropn{y}}
   \and \Pi_{i=0}^{n-1}P_i := P_0 | \ldots | P_{n-1}
\end{mathpar}

\subsubsection{Structural congruence}

\paragraph{Free and bound names and alpha-equivalence.} At the
core of structural equivalence is alpha-equivalence which identifies
process that are the same up to a change of variable. Formally, we
recognize the distinction between free and bound names. The free names
of a process, $\freenames{P}$, may be calculated recursively as
follows:

\begin{mathpar}
\freenames{\pzero} := \emptyset
  \and \\
  \freenames{x?(y).P} := \{ x \} \cup (\freenames{P} \setminus \{ y \})
  \and 
  \freenames{x!\langle P \rangle} := \{ x \} \cup \{ P \} 
  \and \\
  \freenames{P|Q} := \freenames{P} \cup \freenames{Q}
  \and \\
  \freenames{@{x}} := \{ x \}
\end{mathpar}

$\pi$
$\quotep{\pi}$

$\freenames{-} : \pi \to \mathcal{P}(\quotep{\pi})$

\begin{eqnarray*}
  \freenames{\pzero} & := & \emptyset \\
  \freenames{x?(y).P} & := & \{ x \} \cup (\freenames{P} \setminus \{ y \}) \\
  \freenames{x!\langle P \rangle} & := & \{ x \} \cup \{ P \} \\
  \freenames{P|Q} & := & \freenames{P} \cup \freenames{Q} \\
  \freenames{\dropn{x}} & := & \{ x \}
\end{eqnarray*}

The bound names of a process, $\boundnames{P}$, are those names occurring in $P$
that are not free. For example, in $x?(y).0$, the name $x$ is free, while $y$ is bound.

\begin{mathpar}
  \inferrule* [lab=monoidal-laws] {} { P|Q \equiv Q|P \and P|0 \equiv P \and P|(Q|R) \equiv (P|Q)|R }
\end{mathpar}

\begin{mathpar}
  \inferrule* [lab=alpha-equivalence] {} { (x)P \equiv (y)P\{y/x\} \and y \not\in \freenames{P} }
\end{mathpar}

\begin{definition}
Then two processes, $P,Q$, are alpha-equivalent if $P = Q\{\vec{y}/\vec{x}\}$ for
some $\vec{x} \in \boundnames{Q},\vec{y} \in \boundnames{P}$, where $Q\{\vec{y}/\vec{x}\}$
denotes the capture-avoiding substitution of $\vec{y}$ for $\vec{x}$ in $Q$.
\end{definition}

\begin{definition}
  The {\em structural congruence} \cite{SangiorgiWalker} , $\equiv$,
  between processes is the least congruence containing
  alpha-equivalence, satisfying the abelian monoid laws
  (associativity, commutativity and $\pzero$ as identity) for parallel
  composition $|$ and for summation $+$.
\end{definition}

\subsection{Name equivalence}

We take name equivalence, written $\nameeq$, to be the smallest
equivalence relation generated by the following rules.

\begin{mathpar}
\inferrule*[lab=Quote-drop]
{ }
{ \quotep{@{x}} \nameeq x }

\inferrule*[lab=Struct-equiv]
{ P \scong Q }
{ \quotep{P} \nameeq \quotep{Q} }
\end{mathpar}

The astute reader will have noticed that the mutual recursion of names
and processes imposes a mutual recursion on alpha-equivalence and
structural equivalence via name-equivalence. Fortunately, all of this
works out pleasantly and we may calculate in the natural way, free of
concern. The reader interested in the details is referred to the
appendix \ref{appendix:rho_details}.

\subsection{Substitution}

We use $\Proc$ for the set of processes, $\QProc$ for the set of
names, and $\id{\{}\vec{y} / \vec{x} \id{\}}$ to denote partial maps,
$s : \QProc \rightarrow \QProc$. A map, $s$ lifts, uniquely, to a map
on process terms, $\widehat{s} : \Proc \rightarrow \Proc$ by the
following equations.

\begin{mathpar}
  (0) \psubstp{Q}{P} := 0 \\
  (R \juxtap S) \psubstp{Q}{P}
  :=    
  (R)\psubstp{Q}{P} \juxtap (S) \psubstp{Q}{P} \\
  (x?(y).R) \psubstp{Q}{P}    
  :=    
  (x)\substp{Q}{P} (z)\concat( (R \psubstn{z}{y}) \psubstp{Q}{P} ) \\
  (\lift{x}{R}) \psubstp{Q}{P}  
  :=
  \lift{(x)\substp{Q}{P}}{ R \psubstp{Q}{P} } \\
%   (\dropn{x})  \psubstp{Q}{P}       
%   := 
%   \left\{ 
%     \begin{array}{ccc} 
%       \dropn{\quotep{Q}} & & x \nameeq \quotep{P} \\
%       \dropn{x} & & otherwise \\
%     \end{array}
%   \right. 
  (\dropn{x})  \psubstp{Q}{P}       
  := 
  \left\{ 
    \begin{array}{ccc} 
      Q & & x \nameeq \quotep{P} \\
      \dropn{x} & & otherwise \\
    \end{array}
  \right.
\end{mathpar}
 

where

\begin{eqnarray}
  (x)\id{\{} \lpquote Q \rpquote / \lpquote P \rpquote \id{\}}            = 
  \left\{ 
    \begin{array}{ccc}
      \lpquote Q \rpquote & & x \nameeq \lpquote P \rpquote \\
      x & & otherwise \\
    \end{array}
  \right. \nonumber
\end{eqnarray}

and $z$ is chosen distinct from $\quotep{P}$, $\quotep{Q}$, the free
names in $Q$, and all the names in $R$. Our $\alpha$-equivalence will
be built in the standard way from this substitution.

\begin{remark}\label{rem:no_self_referential_names}
  One consequence of these definitions is that $\forall P. \quotep{P}
  \not\in \freenames{P}$.
\end{remark}

\subsection{ Dynamic quote: an example }

Anticipating something of what's to come, consider applying the
substitution, $\widehat{\id{\{}u / z \id{\}}}$, to the following pair
of processes, $\lift{w}{y!(z)}$ and $w[ \lpquote y!(z) \rpquote ]$.

\begin{eqnarray}
	\lift{w}{y!(z)}\widehat{\id{\{}u / z \id{\}}}
		& = &
		\lift{w}{y!(u)} \nonumber\\
	w[ \lpquote y!(z) \rpquote ] \widehat{ \id{\{}u / z \id{\}} }
		& = &
		w[ \lpquote y!(z) \rpquote ] \nonumber
\end{eqnarray}

Because the body of the process between quotes is impervious to
substitution, we get radically different answers. In fact, by
examining the first process in an input context,
e.g. $x?(z).\lift{w}{y!(z)}$, we see that the process under the lift
operator may be shaped by prefixed inputs binding a name inside it. In
this sense, the lift operator will be seen as a way to dynamically
construct processes before reifying them as names.

Finally equipped with these standard features we can present the
dynamics of the calculus.

\subsubsection{Operational semantics} 

Finally, we introduce the computational dynamics. What marks these
algebras as distinct from other more traditionally studied algebraic
structures, e.g. vector spaces or polynomial rings, is the manner in
which dynamics is captured. In traditional structures, dynamics is typically
expressed through morphisms between such structures, as in linear maps
between vector spaces or morphisms between rings. In algebras
associated with the semantics of computation, the dynamics is
expressed as part of the algebraic structure itself, through a
reduction reduction relation typically denoted by $\red$. Below, we
give a recursive presentation of this relation for the calculus used
in the encoding.

$\red \subseteq \pi \times \pi$
$\red : \pi \to \mathcal{P}(\pi)$

\begin{mathpar}
  \inferrule* [lab=Comm] { \textsf{match}( x_{src}, x_{trgt} ) } { x_{trgt}?(y)P \; | \; x_{src}!\langle {Q} \rangle \red P\{\quotep{Q}/y}\} }
  \and \\
  \inferrule* [lab=Par] {{P} \red {P}'} {{{P} | {Q}} \red {{P}' | {Q}}}
  \and
  \inferrule* [lab=Equiv]{{{P} \scong {P}'} \andalso {{P}' \red {Q}'} \andalso {{Q}' \scong {Q}}}{{P} \red {Q}}
\end{mathpar}

\begin{eqnarray*}
  match_{\equiv} (\quotep{P},\quotep{Q}) & := & P \equiv Q \\
  match_{\dagger}(\quotep{P},\quotep{Q}) & := & \forall R. P|Q \red^{*} R => R \red^{*} 0 \\
  match_{K}(\quotep{P},\quotep{Q}) & := & K \mbox{ for some context } K
\end{eqnarray*}

$u?(x)P | u!\langle Q \rangle \red P\{\quotep{Q}/x\}$

%We write $\wred$ for $\red^*$, and $P\red$ if $\exists Q $ such that $ P \red Q$.
We write $P\red$ if $\exists Q $ such that $ P \red Q$ and $P\not\red$, otherwise.

\section{Replication}

As mentioned before, it is known that replication (and hence
recursion) can be implemented in a higher-order process algebra
\cite{SangiorgiWalker}. As our first example of calculation with the
machinery thus far presented we give the construction explicitly in
the {\rhoc}.

\begin{eqnarray}
	D_{x} & := & \prefix{x}{y}{(\binpar{\outputp{x}{y}}{@{y}})} \nonumber\\
	\bangp_{x}{P} & := & \binpar{{x}!\langle{\binpar{D_{x}}{P}}\rangle}{D_{x}} \nonumber
\end{eqnarray}

\begin{eqnarray}
	\bangp_{x}{P} & & \nonumber\\
	=
	& {x}!\langle{(\prefix{x}{y}{(\outputp{x}{y} | @{y})) | P}}\rangle 
	      | \prefix{x}{y}{(\outputp{x}{y} | @{y})} & \nonumber\\
	\red
	& (\outputp{x}{y} | @{y})\substn{\quotep{(\prefix{x}{y}{(@{y} | \outputp{x}{y})) | P}}}{y} & \nonumber\\
	=
	& \outputp{x}{\quotep{(\prefix{x}{y}{(\outputp{x}{y} | @{y})) | P}}}
	  | {(\prefix{x}{y}{(\outputp{x}{y} | @{y})) | P}} & \nonumber\\
	\red
	& \ldots & \nonumber\\
	\red^*
	& P | P | \ldots & \nonumber
\end{eqnarray}

Of course, this encoding, as an implementation, runs away, unfolding
$\bangp{P}$ eagerly. A lazier and more implementable replication
operator, restricted to input-guarded processes, may be obtained as follows.

\begin{eqnarray}
\bangp{\prefix{u}{v}{P}} 
	:= 
	\binpar{\lift{x}{\prefix{u}{v}{(\binpar{D(x)}{P})}}}{D(x)} \nonumber
\end{eqnarray}

\begin{remark}
  Note that the lazier definition still does not deal with summation
  or mixed summation (i.e. sums over input and output). The reader is
  invited to construct definitions of replication that deal with these
  features. 

  Further, the definitions are parameterized in a name, $x$. Can you,
  gentle reader, make a definition that eliminates this parameter and
  guarantees no accidental interaction between the replication
  machinery and the process being replicated -- i.e. no accidental
  sharing of names used by the process to get its work done and the
  name(s) used by the replication to effect copying. This latter
  revision of the definition of replication is crucial to obtaining
  the expected identity $!!P \sim !P$.
\end{remark}

\begin{remark}\label{rem:paradoxical_combinator}
  The reader familiar with the lambda calculus will have noticed the
  similarity between $D$ and the paradoxical combinator.

  [Ed. note: the existence of this seems to suggest we have to be more
  restrictive on the set of processes and names we admit if we are to
  support no-cloning.]
\end{remark}

\subsubsection{Bisimulation}

The computational dynamics gives rise to another kind of equivalence,
the equivalence of computational behavior. As previously mentioned
this is typically captured \emph{via} some form of bisimulation.

% The notion we use in this paper is weak barbed bisimulation
% \cite{milner91polyadicpi}.

The notion we use in this paper is derived from weak barbed
bisimulation \cite{milner91polyadicpi}. 

\begin{definition}
An \emph{observation relation}, $\downarrow_{\mathcal N}$, over a set
of names, $\mathcal N$, is the smallest relation satisfying the rules
below.

\infrule[Out-barb]{y \in {\mathcal N}, \; x \nameeq y}
		  {\outputp{x}{v} \downarrow_{\mathcal N} x}
\infrule[Par-barb]{\mbox{$P\downarrow_{\mathcal N} x$ or $Q\downarrow_{\mathcal N} x$}}
		  {\binpar{P}{Q} \downarrow_{\mathcal N} x}

We write $P \Downarrow_{\mathcal N} x$ if there is $Q$ such that 
$P \wred Q$ and $Q \downarrow_{\mathcal N} x$.
\end{definition}

\begin{definition}
%\label{def.bbisim}
An  ${\mathcal N}$-\emph{barbed bisimulation} over a set of names, ${\mathcal N}$, is a symmetric binary relation 
${\mathcal S}_{\mathcal N}$ between agents such that $P\rel{S}_{\mathcal N}Q$ implies:
\begin{enumerate}
\item If $P \red P'$ then $Q \wred Q'$ and $P'\rel{S}_{\mathcal N} Q'$.
\item If $P\downarrow_{\mathcal N} x$, then $Q\Downarrow_{\mathcal N} x$.
\end{enumerate}
$P$ is ${\mathcal N}$-barbed bisimilar to $Q$, written
$P \wbbisim_{\mathcal N} Q$, if $P \rel{S}_{\mathcal N} Q$ for some ${\mathcal N}$-barbed bisimulation ${\mathcal S}_{\mathcal N}$.
\end{definition}

$\mathcal{R} \subseteq \pi \times \pi$

$P \mathcal{R} Q => \forall P'. P \red P' \Rightarrow \exists Q'. Q \red Q', P' \mathcal{R} Q'$

$P \vdash x \Rightarrow Q \vdash x$

\begin{mathpar}
  \inferrule*[lab=Out-barb]{x \nameeq y}{{y}!\langle{Q}\rangle \vdash x}
  \and
  \inferrule*[lab=Par-barb]{\mbox{$P\vdash x$ or $Q\vdash x$}}{\binpar{P}{Q} \vdash x}
\end{mathpar}

\subsubsection{Contexts}

One of the principle advantages of computational calculi like the
$\pi$-calculus is a well-defined notion of context,
contextual-equivalence and a correlation between
contextual-equivalence and notions of bisimulation. The notion of
context allows the decomposition of a process into (sub-)process and
its syntactic environment, its context. Thus, a context may be
thought of as a process with a ``hole'' (written $\Box$) in it. The
application of a context $M$ to a process $P$, written $M[P]$, is
tantamount to filling the hole in $M$ with $P$. In this paper we do
not need the full weight of this theory, but do make use of the notion
of context in the proof the main theorem. 

\begin{mathpar}
  \inferrule* [lab=summation] {} {{M_{M},M_{N}} \bc \Box \;|\; x.M_{A} \;|\; M_{M}+M_{N}}
  \and
  \inferrule* [lab=agent] {} {{M_{A}} \bc (\vec{x})M_{P} \;| \; \clift{P_0,\ldots,M_{P},\ldots,P_N}}
  \and \\
  \inferrule* [lab=process] {} {{M_{P}} \bc M_{N} \;| \;P|M_{P} }
\end{mathpar} 

\begin{mathpar}
  \inferrule* [lab=sychronization] {} {M_{N} \bc \Box \;|\; x?M_{F} \;|\; x!M_{C}}
  \and
  \inferrule* [lab=abstraction] {} {{M_{F}} \bc (x)M_{P} }
  \and
  \inferrule* [lab=concretion] {} {{M_{C}} \bc \langle M_{P} \rangle }
  \and \\
  \inferrule* [lab=process] {} {{M_{P}} \bc M_{N} \;| \;P|M_{P} }
\end{mathpar}

\begin{definition}[contextual application] Given a context $M$, and
  process $P$, we define the \emph{contextual application}, $M[P] :=
  M\{P/\Box\}$. That is, the contextual application of M to P is the
  substitution of $P$ for $\Box$ in $M$.
\end{definition}

$\meaningof{-} : L \to \mathcal{P}(\pi)$

\begin{mathpar}
  \inferrule* [lab=collection] {} {\meaningof{true} = \pi, \and \meaningof{~E} = \pi \setminus \meaningof{E}, \and \meaningof{E_{1} \& E_{2}} = \meaningof{E_{1}} \cap \meaningof{E_{2}}}
\end{mathpar}

\begin{mathpar}
  \inferrule* [lab=structure] {} {\meaningof{0} = \{ P \in \pi | P \equiv 0 \}, \and \\ \meaningof{E_1 | E_2} = \{ P \in \pi | P \equiv P_{1} | P_{2}, P_{1} \in \meaningof{E_{1}}, P_{2} \in \meaningof{E_2}\} }
\end{mathpar}

\begin{mathpar}
 \inferrule* [lab=behavior] {} {\meaningof{\langle a?b \rangle E} = \{ P \in \pi | P \equiv Q | u?(y)P', \\ \and \\\\ \and \\ \;\;\; u \in \meaningof{a}, \forall z.P'\{z/y\} \in \meaningof{E\{z/b\}}\}, \and \\ \meaningof{a!E} = \{ P \in \pi | P \equiv Q | x!\langle P' \rangle, x \in \meaningof{a} P' \in \meaningof{E}\} }
\end{mathpar}

\begin{mathpar}
 \inferrule* [lab=nominal] {} {\meaningof{\quotep{E}} = \{ \quotep{P} \in \quotep{\pi} | P \in \meaningof{E} \}, \and \meaningof{\quotep{P}} = \{ \quotep{Q} \in \quotep{\pi} | P \equiv Q \} \and \\ \meaningof{@\quotep{E}} = \{ P \in \pi | P \equiv @x, x \in \meaningof{E} \}}
\end{mathpar}

\begin{eqnarray*}
  \\
  \meaningof{-} : TS \to ST
\end{eqnarray*}

\begin{eqnarray*}
  \\
  L : TS \to ST
\end{eqnarray*}

\begin{eqnarray*}
  \\
  P \models E \iff P \in \meaningof{E}
\end{eqnarray*}

\begin{eqnarray*}
  P \approx_{L} Q \iff \forall E \in L. P \models E \iff Q \models E
\end{eqnarray*}

\begin{eqnarray*}
  P \approx_{K} Q
\end{eqnarray*}

\begin{eqnarray*}
  P \approx Q
\end{eqnarray*}

$\approx_{K} = \approx = \approx_{L}$

\subsubsection{Contextual duality}

Note that contexts extend the quotation operation to a family of
operations from processes to names. Given a context, $M$, we can
define a \emph{nominal context}, $\quotep{M}$ by $\quotep{M}[P] :=
\quotep{M[P]}$. To foreshadow what is to come we observe that these
operations enjoy a duality with processes very much like the duality
between vectors and maps from vectors to scalars.

Further, because the calculus is essentially higher-order, we have a
correspondence between contexts and processes. More specifically,
given a name $x$ and a context $M$ we can construct $M^{*}_{x}$ such
that 

\begin{mathpar}
  M^{*}_{x} | \lift{x}{P} \red M[P]
\end{mathpar}

namely,

\begin{mathpar}
  M^{*}_{x} := x?(u).M[\dropn{u}]
\end{mathpar}

The dependence of $M^{*}_{x}$ on a name makes it an abstraction, 

\begin{mathpar}
  M^{*} := (x)x?(u).M[\dropn{u}]
\end{mathpar}

\subsection{Additional notation}

It will sometimes be convenient to denote the process a name
quotes. We already have the notation $x = \quotep{P}$, but it will be
convenient to introduce an alternate notation, $\procn{x}$, when we
want to emphasize the connection to the use of the name. Note that, by
virtue of name equivalence, $\quotep{\procn{x}} \nameeq x$; so, the
notation is consistent with previous definitions.

Further, because names have structure it is possible to effect
substitutions on the basis of that structure. This means we need to
upgrade our notation for substitutions, which we accomplish by
adapting comprehension notation. Thus,

\begin{mathpar}
  P\{ y / x : x \in S \}
\end{mathpar}

is interpreted to mean the process derived from P by replacing (in a
capture-avoiding manner) each occurrence of $x$ in $S$ by $y$. For example,

\begin{mathpar}
  P\{ \quotep{\procn{x}|\procn{x}} / x : x \in \freenames{P} \}
\end{mathpar}

will replace each (occurrence) of a free name $x$ in $P$ by
$\quotep{\procn{x}|\procn{x}}$.

Also, we will avail ourselves of the notation $x^{L}$ and $x^{R}$ to
denote injections of a name into disjoint copies of the name
space. There are numerous ways to accomplish this. One example can be
found in \cite{MeredithR05}. This notation overloads to vectors of
names: $\vec{x}^{\pi} := (x_{i}^{\pi} \; : \; 0 \leq i < |\vec{x}| )$ where $\pi \in \{L,R\}$.

We also use $P^{\Box} := P|\Box$.

In \cite{MeredithR05} an interpretation of the new operator is
given. It turns out that there are several possible interpretations
all enjoying the requisite algebraic properties of the operator (see
\cite{milner91polyadicpi}). We will therefore make liberal use of
$(\nu\; \vec{x})P$.

% subsection the_syntax_and_semantics_of_the_notation_system (end)   

\section{Interpretation of QM}
\subsection{Supporting definitions}
\subsubsection{Multiplication}
\begin{mathpar}
  \quotep{Q} \cdot \quotep{R} := \quotep{Q|R}
  \and \\
  \quotep{Q} \cdot P := P\{ \quotep{Q|R} / \quotep{R} : \quotep{R} \in \freenames{P} \}
\end{mathpar}

\paragraph{Discussion}
The first line needs little explanation. The second line says that
each free name of the process is replaced with the multiplication of
that name by the scalar. Multiplication of a scalar (name) by a state
(process) results in a process all the names of which have been `moved
over' by parallel composition with the process the scalar
quotes. There is a subtlety that the bound names have to be
manipulated so that multiplied names aren't accidentally
captured. There are many ways to achieve this.

\begin{remark}\label{rem:multiplication_identities}
  The reader is invited to verify that for all $x,y,z \in \QProc$ and $P \in \Proc$
  \begin{mathpar}
    x \cdot \quotep{0} \equiv x 
    \and
    x \cdot y \equiv y \cdot x
    \and
    x \cdot (y \cdot z) \equiv (x \cdot y) \cdot z
    \and \\
    \quotep{0} \cdot P \equiv P
    \and \\
    x \cdot (y \cdot P) \equiv (x \cdot y) \cdot P
    \and \\
    x \cdot (P|Q) \equiv (x \cdot P) | (x \cdot Q)
    \and \\    
  \end{mathpar}
\end{remark}

\subsubsection{Tensor product}

We define a tensor product on processes by structural induction.

\paragraph{Tensor of sums} First note that all summations, including
$\pzero$ and sequence, can be written $\Sigma_{i} x_{i}.A_{i} +
\Sigma_{j} x_{j}.C_{j}$, where we have grouped input-guarded processes
together and output-guarded processes together.

Thus, we can define the tensor product of two summations, $N_{1}\otimes N_{2}$, where

\begin{mathpar}
  N_{1} := \Sigma_{i} x_{i}.A_{i} + \Sigma_{j} x_{j}.C_{j}
  \and
  N_{2} := \Sigma_{i'} y_{i'}.B_{i'} + \Sigma_{j'} y_{j'}.D_{j'} 
\end{mathpar}

as follows.

\begin{mathpar}
  \Sigma_{i} x_{i}.A_{i} + \Sigma_{j} x_{j}.C_{j} \otimes \Sigma_{i'}
  y_{i'}.B_{i'} + \Sigma_{j'} y_{j'}.D_{j'} 
  \and \\
  := \; \Sigma_{i} \Sigma_{i'} \quotep{\stackrel{\vee}{x_{i}}| \stackrel{\vee}{y_{i'}}}.(A_{i}\otimes B_{i'}) \; | \; \Sigma_{i'} \Sigma_{i} \quotep{\stackrel{\vee}{y_{i'}}|\stackrel{\vee}{x_{i}}}.(B_{i'}\otimes A_{i})
  \and
  \;\; | \;\; \Sigma_{j} \Sigma_{j'} \quotep{\stackrel{\vee}{x_{j}}|\stackrel{\vee}{y_{j'}}}.(A_{j}\otimes B_{j'}) \; | \; \Sigma_{j'} \Sigma_{j} \quotep{\stackrel{\vee}{y_{j'}}|\stackrel{\vee}{x_{j}}}.(B_{j'}\otimes A_{j})
\end{mathpar}

\begin{remark}
  Do we need to $x^{L}$ and $y^{R}$ for this construction as well?
\end{remark}

\paragraph{Tensor of parallel compositions} Next, we distribute tensor
over par.

\begin{mathpar}
  P_{1}|P_{2} \otimes Q_{1}|Q_{2} := (P_{1} \otimes Q_{1}) | (P_{1}
  \otimes Q_{2}) | (P_{2} \otimes Q_{1}) | (P_{2} \otimes Q_{2})
\end{mathpar}

\paragraph{Tensor with dropped names} We treat tensor of a
process with a dropped name as parallel composition.

\begin{mathpar}
  P \otimes \dropn{x} := P | \dropn{x}
\end{mathpar}

\paragraph{Tensor of agents}

Finally, we need to define tensor on agents. Note that the definition
of tensor on normal products only tensors inputs with inputs and
outputs with outputs. Thus, we only have to define the operation on
``homogeneous'' pairings.

\begin{mathpar}
  (\vec{x})P \otimes (\vec{y})Q
  \and \\
  := (x_{0}^{L}|y_{0}^{R},\ldots,x_{0}^{L}|y_{n}^{R},\ldots,x_{m}^{L}|y_{0}^{R},\ldots,x_{m}^{L}|y_{n}^R)(P\{ \vec{x}^{L}/\vec{x}\} \otimes Q \{ \vec{y}^{R}/\vec{y}\})
  \and \\
  \clift{\vec{P}} \otimes \clift{\vec{Q}}
  \and \\
  := \clift{P_{0}\otimes Q_{0},\ldots,P_{0}\otimes Q_{n},\ldots,P_{m}\otimes Q_{0},\ldots,P_{m}\otimes Q_{n}}
\end{mathpar}

\begin{remark}
  Observe that arities of tensored abstractions matches arities of
  tensored concretions if the original arities matched. Note also that
  the length of the arities corresponds to the increase in dimension
  we see in ordinary vector space tensor product.
\end{remark}

\begin{remark}
  Operationally, this definition distributes the tensor down to
  components ``linked'' by summation. Tensor over summation is
  intriguing in that it mixes names. Moreover, as a consequence of the
  way it mixes names we have the identities for all $x \in \QProc$ and
  $P,Q \in \Proc$

  \begin{mathpar}
    (x \cdot P) \otimes Q \equiv x \cdot (P \otimes Q) \equiv P \otimes (x \cdot Q)
    \and
    P \otimes \pzero \equiv P
  \end{mathpar}

  that the reader is invited to verify.
\end{remark}

\subsubsection{Annihilation}
\begin{mathpar}
  P^{\perp} := \{ Q | \forall R. P|Q \red^{*} R \Rightarrow R \red^{*} \pzero \}
  \and \\
  P^{\underline{\perp}} := \Sigma_{Q \in P^{\perp}} \quotep{Q}?(y).(\dropn{y}|Q) | \Sigma_{Q \in P^{\perp}} \quotep{Q}\clift{\Box}
\end{mathpar}

\paragraph{Discussion} The reader will note that $P^{\perp}$ is a
\emph{set} of processes, while $P^{\underline{\perp}}$ is a
\emph{context}. We call the set $P^{\perp}$ the \emph{annihilators} of
$P$. The parallel composition of a process in the annihilators of $P$
with $P$ will result in a process, the state space of which has all
paths eventually leading to $\pzero$. Execution may endure loops; but
under reasonable conditions of fairness (naturally guaranteed under
most notions of bisimulation) such a composite process cannot get
stuck in such a loop and will, eventually pop out and terminate.

The context $P^{\underline{\perp}}$ is ready and willing to ``take the
$P$ out of'' the process to which it is applied. It will effectively
transmit the code of the process to which it is applied to one of the
annihilators and run the process against it.

\subsubsection{Evaluation}
We fix $M$ a domain of fully abstract interpretation with an equality
coincident with bisimulation. We take $\meaningof{\cdot} : \Proc \to
M$ to be the map interpreting processes and $\nmeaningof{\cdot} : \M
\to Proc$ to be the map running the other way. Then we define

\begin{mathpar}
  \int P := \nmeaningof{\meaningof{P}}
\end{mathpar}

\paragraph{Discussion}
There are many fully abstract interpretations of Milner's
$\pi$-calculus. Any of them can be used as a basis for interpreting
the reflective calculus here. Equipped with such a domain it is
largely a matter of grinding through to check that the Yoneda
construction for the normalization-by-evaluation program can be
extended to this setting.

\begin{remark}
  The reader is invited to verify that $\int (P^{\underline{\perp}}[P]) = 0$.
\end{remark}

\subsection{Quantum mechanics}

Table \ref{tbl:core_qm_op_defns} gives the core operational definitions

\begin{table}[htp]\label{tbl:core_qm_op_defns}
  \center{
    \fbox{
      \begin{tabular}{c|c}
        quantum mechanics & process calculus \\
        \hline
        scalar & $x := \quotep{P}$ \\
        state vector & $\state{P} := P$ \\
        dual & $\state{P}^{*} := \event{P^{\underline{\perp}}} := \quotep{P^{\underline{\perp}}}[-]$ \\
        matrix & $ \Sigma_{\alpha} \state{P_{\alpha}}x_{\alpha}\event{Q_{\alpha}}$ \\
        vector addition & $\state{P} + \state{Q} := \state{P | Q}$ \\
        tensor product & $\state{P} \otimes \state{Q} := \state{P \otimes Q}$ \\
        inner product & $\innerprod{P}{Q} := \quotep{\int P^{\underline{\perp}}[Q]}$ \\
      \end{tabular}
    }
  }
  \caption{QM - operational definitions}
\end{table}

where

\begin{mathpar}
  \prmatrix{P}{Q} := \fprmatrix{P}{\quotep{\pzero}}{Q}
  \and
  \fprmatrix{P}{x}{Q} := (\state{P},x,\event{Q})
  \and
  (\fprmatrix{P}{x}{Q})(\state{R}) := x \cdot \innerprod{Q}{R} \cdot \state{P}
  \and
  (\fprmatrix{P}{x}{Q})(\event{R}) := x \cdot \innerprod{R}{P} \cdot \event{Q}
\end{mathpar}

\paragraph{Discussion}
As promised: vectors (aka states) are represented as processes; duals
as contextual duals; inner product definition should be compared with
standard inner product definition for ....

\begin{remark}
  Assuming $\int (P^{\underline{\perp}}[P]) = 0$, the reader is
  invited to verify that $(\fprmatrix{P}{x}{P})(\state{P}) = x \cdot \state{P}$.
\end{remark}

\begin{remark}
  The reader is invited to verify that $\innerprod{P}{Q}$ could
  equally well have been written $\quotep{\int \stackrel{\vee}{x}}$
  where $x = \event{P^{\underline{\perp}}}(Q)$.

  One of the motivations for this remark is that there is another way
  to factor these operations. We could package up evaluation in the dual:

  \begin{mathpar}
    \state{P}^{*} := \event{\int P^{\underline{\perp}}} := \quotep{\int P^{\underline{\perp}}}[-]
  \end{mathpar}

  and then have inner product defined by
  
  \begin{mathpar}
    \innerprod{P}{Q} := \event{P}(Q)
  \end{mathpar}

  Hopefully, experience with the calculations will provide guidance on
  the best factoring.
\end{remark}

\begin{remark}
  Assuming $\int (P^{\underline{\perp}}[P]) = 0$, the reader is
  invited to verify that $\forall P,Q. (\prmatrix{0}{Q})(\state{0}) =
  \state{0}$ and dually $(\prmatrix{P}{0})(\event{0}) = \event{0}$.
\end{remark}

\begin{remark}
  i'm a little worried that i don't (yet) have proper support for
  complex conjugacy. But, the observation above may give us a
  clue. According to Abramsky, it must be the case that the scalars
  are iso to the homset of the identity for the tensor -- which the
  observation above characterizes. 

  For now, we will simply bookmark the notion with $\overline{x}$.
\end{remark}

\subsubsection{Adjointness}

We need to give a definition of $(\cdot)^{\dagger}$ for matrices. The
obvious candidate definition is
\begin{mathpar}
(\Sigma_{\alpha}\fprmatrix{P_{\alpha}}{x_{\alpha}}{Q_{\alpha}})^{\dagger}
= \Sigma_{\alpha}\fprmatrix{(Q_{\alpha}^{\underline{\perp}})^{*}}{\overline{x}_{\alpha}}{P_{\alpha}^{\underline{\perp}}} 
\end{mathpar}

But, $(Q_{\alpha}^{\underline{\perp}})^{*}$ requires a name along
which to communicate the process to achieve the context application.

\subsubsection{Basis for a basis}
If processes label states and ``addition'' of states (a.k.a. vector
addition) is interpreted as parallel composition, what corresponds to
notions of linear independence and basis? Here, we recall that Yoshida
has developed a set of \emph{combinators} for an asynchronous verison
of Milner's $\pi$-calculus. These are a finite set of processes such
any process can be expressed as parallel composition of these
combinators together with liberal uses of the new operator and
replication. We can simply give a translation of these into the
present calculus and have reasonable expectation that the property
carries over. That is, that the resultant set allows to express all
processes via parallel composition. Note, however, that there is no
new operator or replication in this calculus. As a result, we expect
that the corresponding set is actually infinite. That is, we expect
that the space is actually infinite dimensional.

\begin{remark}
  The attentive reader may be a bit concerned. Certainly, the
  collection $S$, $K$ and $I$ is a finite set of
  combinators. Shouldn't we expect to see a finite set of combinators
  for an effectively equivalent system? i am very sympathetic to this
  critique and feel it warrants full attention. On the other hand, i
  also have in mind the following analogy. The natural numbers, as a
  monoid under addition, has exactly $1$ generator, while the natural
  numbers, as a monoid under multiplication, has countably many
  generators (the primes). We observe that the application of the
  lambda calculus is much less resource sensitive than the parallel
  composition of the $\pi$-calculus. Could it be the case that we have
  an analogy of the form
  
  \begin{mathpar}
    m + n : MN :: m*n : M|N
  \end{mathpar}

  giving a similar blow up in the set of ``primes''?  This is such a
  wonderful thought that, even if it's not true, i think it's worth
  writing down.
\end{remark}
 

\documentclass[12pt]{llncs}
%\documentclass{jktr}

\usepackage[pdftex]{hyperref}                   
\usepackage {listings}
\usepackage {mathpartir}
\usepackage{bcprules}
%\usepackage{listings}
                       
\usepackage{graphicx} 
%\usepackage[margins=2.5cm,nohead,nofoot]{geometry}
%\usepackage{geometry}
\usepackage{amsfonts}
\usepackage{amstext}
\usepackage{latexsym}
\usepackage{amssymb}
\usepackage{color}


%\include{myPreamble}
\include{qm2pi.local} 

%\ifpdf
%\usepackage[pdftex]{graphicx}
%\else
%\usepackage{graphicx}
%\fi

 % \ifpdf
%  \usepackage{pdfsync}
%  \if


%\title{Brief Article}
%\author{David F. Snyder}
%\author{L.G. Meredith}

%\address{Dept. of Math., Texas State University--San Marcos, San Marcos, TX 78666}
       
\pagestyle{empty}


\begin{document}

\lstset{language=[Objective]Caml,frame=shadowbox}

\input{qm2pi.front}

% section front matter (end)

\input{qm2pi.intro} 
 
% section introduction (end)

% \input{qm2pi.knotations} 

% section notation (end)

\input{qm2pi.process.calculi} 

% section concurrent_process_calculi_and_spatial_logics_ (end)
    
%\input{qm2pi.knots2pi} 

%\input{qm2pi.trefoil} 

%\input{qm2pi.mainthm} 

% subsection basic_interpretation (end)

%\input{qm2pi.rho.presentation} 
\subsection{The syntax and semantics of the notation system}\label{sub:the_syntax_and_semantics_of_the_notation_system} % (fold)

We now summarize a technical presentation of the calculus that
embodies our theory of dynamics. The typical presentation of such a
calculus follows the style of giving generators and relations on
them. The grammar, below, describing term constructors, freely
generates the set of processes, $\Proc$. This set is then quotiented
by a relation known as structural congruence and it is over this set
that the notion of dynamics is expressed. This presentation is
essentially that of \cite{MeredithR05} with the addition of
polyadicity and summation. For readability we have relegated some of
the technical subtleties to an appendix.

\subsubsection{Process grammar}\label{subsub:process_grammar}

\begin{mathpar}
  \inferrule* [lab=synchronization] {} {{M} \bc \pzero \;|\; x?F \;|\; x!C }
  \and
  \inferrule* [lab=abstraction] {} {{F} \bc (x)P}
  \and
  \inferrule* [lab=concretion] {} {{C} \bc \langle Q \rangle}
  \and
  \inferrule* [lab=process] {} {{P,Q} \bc M \;| \;P|Q \;|\; @{x}}
  \and
  \inferrule* [lab=name] {} {{x} \bc \quotep{P}}
\end{mathpar} 

Note that $\vec{x}$ (resp. $\vec{P}$) denotes a vector of names
(resp. processes) of length $|\vec{x}|$ (resp. $|\vec{P}|$). We adopt
the following useful abbreviations.

\begin{mathpar}
   x?(\vec{y}).P := x.(\vec{y})P \and  x\clift{\vec{P}} := x.\clift{\vec{P}}
   \and x!(y) := \lift{x}{\dropn{y}}
   \and \Pi_{i=0}^{n-1}P_i := P_0 | \ldots | P_{n-1}
\end{mathpar}

\subsubsection{Structural congruence}

\paragraph{Free and bound names and alpha-equivalence.} At the
core of structural equivalence is alpha-equivalence which identifies
process that are the same up to a change of variable. Formally, we
recognize the distinction between free and bound names. The free names
of a process, $\freenames{P}$, may be calculated recursively as
follows:

\begin{mathpar}
\freenames{\pzero} := \emptyset
  \and \\
  \freenames{x?(y).P} := \{ x \} \cup (\freenames{P} \setminus \{ y \})
  \and 
  \freenames{x!\langle P \rangle} := \{ x \} \cup \{ P \} 
  \and \\
  \freenames{P|Q} := \freenames{P} \cup \freenames{Q}
  \and \\
  \freenames{@{x}} := \{ x \}
\end{mathpar}

$\pi$
$\quotep{\pi}$

$\freenames{-} : \pi \to \mathcal{P}(\quotep{\pi})$

\begin{eqnarray*}
  \freenames{\pzero} & := & \emptyset \\
  \freenames{x?(y).P} & := & \{ x \} \cup (\freenames{P} \setminus \{ y \}) \\
  \freenames{x!\langle P \rangle} & := & \{ x \} \cup \{ P \} \\
  \freenames{P|Q} & := & \freenames{P} \cup \freenames{Q} \\
  \freenames{\dropn{x}} & := & \{ x \}
\end{eqnarray*}

The bound names of a process, $\boundnames{P}$, are those names occurring in $P$
that are not free. For example, in $x?(y).0$, the name $x$ is free, while $y$ is bound.

\begin{mathpar}
  \inferrule* [lab=monoidal-laws] {} { P|Q \equiv Q|P \and P|0 \equiv P \and P|(Q|R) \equiv (P|Q)|R }
\end{mathpar}

\begin{mathpar}
  \inferrule* [lab=alpha-equivalence] {} { (x)P \equiv (y)P\{y/x\} \and y \not\in \freenames{P} }
\end{mathpar}

\begin{definition}
Then two processes, $P,Q$, are alpha-equivalent if $P = Q\{\vec{y}/\vec{x}\}$ for
some $\vec{x} \in \boundnames{Q},\vec{y} \in \boundnames{P}$, where $Q\{\vec{y}/\vec{x}\}$
denotes the capture-avoiding substitution of $\vec{y}$ for $\vec{x}$ in $Q$.
\end{definition}

\begin{definition}
  The {\em structural congruence} \cite{SangiorgiWalker} , $\equiv$,
  between processes is the least congruence containing
  alpha-equivalence, satisfying the abelian monoid laws
  (associativity, commutativity and $\pzero$ as identity) for parallel
  composition $|$ and for summation $+$.
\end{definition}

\subsection{Name equivalence}

We take name equivalence, written $\nameeq$, to be the smallest
equivalence relation generated by the following rules.

\begin{mathpar}
\inferrule*[lab=Quote-drop]
{ }
{ \quotep{@{x}} \nameeq x }

\inferrule*[lab=Struct-equiv]
{ P \scong Q }
{ \quotep{P} \nameeq \quotep{Q} }
\end{mathpar}

The astute reader will have noticed that the mutual recursion of names
and processes imposes a mutual recursion on alpha-equivalence and
structural equivalence via name-equivalence. Fortunately, all of this
works out pleasantly and we may calculate in the natural way, free of
concern. The reader interested in the details is referred to the
appendix \ref{appendix:rho_details}.

\subsection{Substitution}

We use $\Proc$ for the set of processes, $\QProc$ for the set of
names, and $\id{\{}\vec{y} / \vec{x} \id{\}}$ to denote partial maps,
$s : \QProc \rightarrow \QProc$. A map, $s$ lifts, uniquely, to a map
on process terms, $\widehat{s} : \Proc \rightarrow \Proc$ by the
following equations.

\begin{mathpar}
  (0) \psubstp{Q}{P} := 0 \\
  (R \juxtap S) \psubstp{Q}{P}
  :=    
  (R)\psubstp{Q}{P} \juxtap (S) \psubstp{Q}{P} \\
  (x?(y).R) \psubstp{Q}{P}    
  :=    
  (x)\substp{Q}{P} (z)\concat( (R \psubstn{z}{y}) \psubstp{Q}{P} ) \\
  (\lift{x}{R}) \psubstp{Q}{P}  
  :=
  \lift{(x)\substp{Q}{P}}{ R \psubstp{Q}{P} } \\
%   (\dropn{x})  \psubstp{Q}{P}       
%   := 
%   \left\{ 
%     \begin{array}{ccc} 
%       \dropn{\quotep{Q}} & & x \nameeq \quotep{P} \\
%       \dropn{x} & & otherwise \\
%     \end{array}
%   \right. 
  (\dropn{x})  \psubstp{Q}{P}       
  := 
  \left\{ 
    \begin{array}{ccc} 
      Q & & x \nameeq \quotep{P} \\
      \dropn{x} & & otherwise \\
    \end{array}
  \right.
\end{mathpar}
 

where

\begin{eqnarray}
  (x)\id{\{} \lpquote Q \rpquote / \lpquote P \rpquote \id{\}}            = 
  \left\{ 
    \begin{array}{ccc}
      \lpquote Q \rpquote & & x \nameeq \lpquote P \rpquote \\
      x & & otherwise \\
    \end{array}
  \right. \nonumber
\end{eqnarray}

and $z$ is chosen distinct from $\quotep{P}$, $\quotep{Q}$, the free
names in $Q$, and all the names in $R$. Our $\alpha$-equivalence will
be built in the standard way from this substitution.

\begin{remark}\label{rem:no_self_referential_names}
  One consequence of these definitions is that $\forall P. \quotep{P}
  \not\in \freenames{P}$.
\end{remark}

\subsection{ Dynamic quote: an example }

Anticipating something of what's to come, consider applying the
substitution, $\widehat{\id{\{}u / z \id{\}}}$, to the following pair
of processes, $\lift{w}{y!(z)}$ and $w[ \lpquote y!(z) \rpquote ]$.

\begin{eqnarray}
	\lift{w}{y!(z)}\widehat{\id{\{}u / z \id{\}}}
		& = &
		\lift{w}{y!(u)} \nonumber\\
	w[ \lpquote y!(z) \rpquote ] \widehat{ \id{\{}u / z \id{\}} }
		& = &
		w[ \lpquote y!(z) \rpquote ] \nonumber
\end{eqnarray}

Because the body of the process between quotes is impervious to
substitution, we get radically different answers. In fact, by
examining the first process in an input context,
e.g. $x?(z).\lift{w}{y!(z)}$, we see that the process under the lift
operator may be shaped by prefixed inputs binding a name inside it. In
this sense, the lift operator will be seen as a way to dynamically
construct processes before reifying them as names.

Finally equipped with these standard features we can present the
dynamics of the calculus.

\subsubsection{Operational semantics} 

Finally, we introduce the computational dynamics. What marks these
algebras as distinct from other more traditionally studied algebraic
structures, e.g. vector spaces or polynomial rings, is the manner in
which dynamics is captured. In traditional structures, dynamics is typically
expressed through morphisms between such structures, as in linear maps
between vector spaces or morphisms between rings. In algebras
associated with the semantics of computation, the dynamics is
expressed as part of the algebraic structure itself, through a
reduction reduction relation typically denoted by $\red$. Below, we
give a recursive presentation of this relation for the calculus used
in the encoding.

$\red \subseteq \pi \times \pi$
$\red : \pi \to \mathcal{P}(\pi)$

\begin{mathpar}
  \inferrule* [lab=Comm] { \textsf{match}( x_{src}, x_{trgt} ) } { x_{trgt}?(y)P \; | \; x_{src}!\langle {Q} \rangle \red P\{\quotep{Q}/y}\} }
  \and \\
  \inferrule* [lab=Par] {{P} \red {P}'} {{{P} | {Q}} \red {{P}' | {Q}}}
  \and
  \inferrule* [lab=Equiv]{{{P} \scong {P}'} \andalso {{P}' \red {Q}'} \andalso {{Q}' \scong {Q}}}{{P} \red {Q}}
\end{mathpar}

\begin{eqnarray*}
  match_{\equiv} (\quotep{P},\quotep{Q}) & := & P \equiv Q \\
  match_{\dagger}(\quotep{P},\quotep{Q}) & := & \forall R. P|Q \red^{*} R => R \red^{*} 0 \\
  match_{K}(\quotep{P},\quotep{Q}) & := & K \mbox{ for some context } K
\end{eqnarray*}

$u?(x)P | u!\langle Q \rangle \red P\{\quotep{Q}/x\}$

%We write $\wred$ for $\red^*$, and $P\red$ if $\exists Q $ such that $ P \red Q$.
We write $P\red$ if $\exists Q $ such that $ P \red Q$ and $P\not\red$, otherwise.

\section{Replication}

As mentioned before, it is known that replication (and hence
recursion) can be implemented in a higher-order process algebra
\cite{SangiorgiWalker}. As our first example of calculation with the
machinery thus far presented we give the construction explicitly in
the {\rhoc}.

\begin{eqnarray}
	D_{x} & := & \prefix{x}{y}{(\binpar{\outputp{x}{y}}{@{y}})} \nonumber\\
	\bangp_{x}{P} & := & \binpar{{x}!\langle{\binpar{D_{x}}{P}}\rangle}{D_{x}} \nonumber
\end{eqnarray}

\begin{eqnarray}
	\bangp_{x}{P} & & \nonumber\\
	=
	& {x}!\langle{(\prefix{x}{y}{(\outputp{x}{y} | @{y})) | P}}\rangle 
	      | \prefix{x}{y}{(\outputp{x}{y} | @{y})} & \nonumber\\
	\red
	& (\outputp{x}{y} | @{y})\substn{\quotep{(\prefix{x}{y}{(@{y} | \outputp{x}{y})) | P}}}{y} & \nonumber\\
	=
	& \outputp{x}{\quotep{(\prefix{x}{y}{(\outputp{x}{y} | @{y})) | P}}}
	  | {(\prefix{x}{y}{(\outputp{x}{y} | @{y})) | P}} & \nonumber\\
	\red
	& \ldots & \nonumber\\
	\red^*
	& P | P | \ldots & \nonumber
\end{eqnarray}

Of course, this encoding, as an implementation, runs away, unfolding
$\bangp{P}$ eagerly. A lazier and more implementable replication
operator, restricted to input-guarded processes, may be obtained as follows.

\begin{eqnarray}
\bangp{\prefix{u}{v}{P}} 
	:= 
	\binpar{\lift{x}{\prefix{u}{v}{(\binpar{D(x)}{P})}}}{D(x)} \nonumber
\end{eqnarray}

\begin{remark}
  Note that the lazier definition still does not deal with summation
  or mixed summation (i.e. sums over input and output). The reader is
  invited to construct definitions of replication that deal with these
  features. 

  Further, the definitions are parameterized in a name, $x$. Can you,
  gentle reader, make a definition that eliminates this parameter and
  guarantees no accidental interaction between the replication
  machinery and the process being replicated -- i.e. no accidental
  sharing of names used by the process to get its work done and the
  name(s) used by the replication to effect copying. This latter
  revision of the definition of replication is crucial to obtaining
  the expected identity $!!P \sim !P$.
\end{remark}

\begin{remark}\label{rem:paradoxical_combinator}
  The reader familiar with the lambda calculus will have noticed the
  similarity between $D$ and the paradoxical combinator.

  [Ed. note: the existence of this seems to suggest we have to be more
  restrictive on the set of processes and names we admit if we are to
  support no-cloning.]
\end{remark}

\subsubsection{Bisimulation}

The computational dynamics gives rise to another kind of equivalence,
the equivalence of computational behavior. As previously mentioned
this is typically captured \emph{via} some form of bisimulation.

% The notion we use in this paper is weak barbed bisimulation
% \cite{milner91polyadicpi}.

The notion we use in this paper is derived from weak barbed
bisimulation \cite{milner91polyadicpi}. 

\begin{definition}
An \emph{observation relation}, $\downarrow_{\mathcal N}$, over a set
of names, $\mathcal N$, is the smallest relation satisfying the rules
below.

\infrule[Out-barb]{y \in {\mathcal N}, \; x \nameeq y}
		  {\outputp{x}{v} \downarrow_{\mathcal N} x}
\infrule[Par-barb]{\mbox{$P\downarrow_{\mathcal N} x$ or $Q\downarrow_{\mathcal N} x$}}
		  {\binpar{P}{Q} \downarrow_{\mathcal N} x}

We write $P \Downarrow_{\mathcal N} x$ if there is $Q$ such that 
$P \wred Q$ and $Q \downarrow_{\mathcal N} x$.
\end{definition}

\begin{definition}
%\label{def.bbisim}
An  ${\mathcal N}$-\emph{barbed bisimulation} over a set of names, ${\mathcal N}$, is a symmetric binary relation 
${\mathcal S}_{\mathcal N}$ between agents such that $P\rel{S}_{\mathcal N}Q$ implies:
\begin{enumerate}
\item If $P \red P'$ then $Q \wred Q'$ and $P'\rel{S}_{\mathcal N} Q'$.
\item If $P\downarrow_{\mathcal N} x$, then $Q\Downarrow_{\mathcal N} x$.
\end{enumerate}
$P$ is ${\mathcal N}$-barbed bisimilar to $Q$, written
$P \wbbisim_{\mathcal N} Q$, if $P \rel{S}_{\mathcal N} Q$ for some ${\mathcal N}$-barbed bisimulation ${\mathcal S}_{\mathcal N}$.
\end{definition}

$\mathcal{R} \subseteq \pi \times \pi$

$P \mathcal{R} Q => \forall P'. P \red P' \Rightarrow \exists Q'. Q \red Q', P' \mathcal{R} Q'$

$P \vdash x \Rightarrow Q \vdash x$

\begin{mathpar}
  \inferrule*[lab=Out-barb]{x \nameeq y}{{y}!\langle{Q}\rangle \vdash x}
  \and
  \inferrule*[lab=Par-barb]{\mbox{$P\vdash x$ or $Q\vdash x$}}{\binpar{P}{Q} \vdash x}
\end{mathpar}

\subsubsection{Contexts}

One of the principle advantages of computational calculi like the
$\pi$-calculus is a well-defined notion of context,
contextual-equivalence and a correlation between
contextual-equivalence and notions of bisimulation. The notion of
context allows the decomposition of a process into (sub-)process and
its syntactic environment, its context. Thus, a context may be
thought of as a process with a ``hole'' (written $\Box$) in it. The
application of a context $M$ to a process $P$, written $M[P]$, is
tantamount to filling the hole in $M$ with $P$. In this paper we do
not need the full weight of this theory, but do make use of the notion
of context in the proof the main theorem. 

\begin{mathpar}
  \inferrule* [lab=summation] {} {{M_{M},M_{N}} \bc \Box \;|\; x.M_{A} \;|\; M_{M}+M_{N}}
  \and
  \inferrule* [lab=agent] {} {{M_{A}} \bc (\vec{x})M_{P} \;| \; \clift{P_0,\ldots,M_{P},\ldots,P_N}}
  \and \\
  \inferrule* [lab=process] {} {{M_{P}} \bc M_{N} \;| \;P|M_{P} }
\end{mathpar} 

\begin{mathpar}
  \inferrule* [lab=sychronization] {} {M_{N} \bc \Box \;|\; x?M_{F} \;|\; x!M_{C}}
  \and
  \inferrule* [lab=abstraction] {} {{M_{F}} \bc (x)M_{P} }
  \and
  \inferrule* [lab=concretion] {} {{M_{C}} \bc \langle M_{P} \rangle }
  \and \\
  \inferrule* [lab=process] {} {{M_{P}} \bc M_{N} \;| \;P|M_{P} }
\end{mathpar}

\begin{definition}[contextual application] Given a context $M$, and
  process $P$, we define the \emph{contextual application}, $M[P] :=
  M\{P/\Box\}$. That is, the contextual application of M to P is the
  substitution of $P$ for $\Box$ in $M$.
\end{definition}

$\meaningof{-} : L \to \mathcal{P}(\pi)$

\begin{mathpar}
  \inferrule* [lab=collection] {} {\meaningof{true} = \pi, \and \meaningof{~E} = \pi \setminus \meaningof{E}, \and \meaningof{E_{1} \& E_{2}} = \meaningof{E_{1}} \cap \meaningof{E_{2}}}
\end{mathpar}

\begin{mathpar}
  \inferrule* [lab=structure] {} {\meaningof{0} = \{ P \in \pi | P \equiv 0 \}, \and \\ \meaningof{E_1 | E_2} = \{ P \in \pi | P \equiv P_{1} | P_{2}, P_{1} \in \meaningof{E_{1}}, P_{2} \in \meaningof{E_2}\} }
\end{mathpar}

\begin{mathpar}
 \inferrule* [lab=behavior] {} {\meaningof{\langle a?b \rangle E} = \{ P \in \pi | P \equiv Q | u?(y)P', \\ \and \\\\ \and \\ \;\;\; u \in \meaningof{a}, \forall z.P'\{z/y\} \in \meaningof{E\{z/b\}}\}, \and \\ \meaningof{a!E} = \{ P \in \pi | P \equiv Q | x!\langle P' \rangle, x \in \meaningof{a} P' \in \meaningof{E}\} }
\end{mathpar}

\begin{mathpar}
 \inferrule* [lab=nominal] {} {\meaningof{\quotep{E}} = \{ \quotep{P} \in \quotep{\pi} | P \in \meaningof{E} \}, \and \meaningof{\quotep{P}} = \{ \quotep{Q} \in \quotep{\pi} | P \equiv Q \} \and \\ \meaningof{@\quotep{E}} = \{ P \in \pi | P \equiv @x, x \in \meaningof{E} \}}
\end{mathpar}

\begin{eqnarray*}
  \\
  \meaningof{-} : TS \to ST
\end{eqnarray*}

\begin{eqnarray*}
  \\
  L : TS \to ST
\end{eqnarray*}

\begin{eqnarray*}
  \\
  P \models E \iff P \in \meaningof{E}
\end{eqnarray*}

\begin{eqnarray*}
  P \approx_{L} Q \iff \forall E \in L. P \models E \iff Q \models E
\end{eqnarray*}

\begin{eqnarray*}
  P \approx_{K} Q
\end{eqnarray*}

\begin{eqnarray*}
  P \approx Q
\end{eqnarray*}

$\approx_{K} = \approx = \approx_{L}$

\subsubsection{Contextual duality}

Note that contexts extend the quotation operation to a family of
operations from processes to names. Given a context, $M$, we can
define a \emph{nominal context}, $\quotep{M}$ by $\quotep{M}[P] :=
\quotep{M[P]}$. To foreshadow what is to come we observe that these
operations enjoy a duality with processes very much like the duality
between vectors and maps from vectors to scalars.

Further, because the calculus is essentially higher-order, we have a
correspondence between contexts and processes. More specifically,
given a name $x$ and a context $M$ we can construct $M^{*}_{x}$ such
that 

\begin{mathpar}
  M^{*}_{x} | \lift{x}{P} \red M[P]
\end{mathpar}

namely,

\begin{mathpar}
  M^{*}_{x} := x?(u).M[\dropn{u}]
\end{mathpar}

The dependence of $M^{*}_{x}$ on a name makes it an abstraction, 

\begin{mathpar}
  M^{*} := (x)x?(u).M[\dropn{u}]
\end{mathpar}

\subsection{Additional notation}

It will sometimes be convenient to denote the process a name
quotes. We already have the notation $x = \quotep{P}$, but it will be
convenient to introduce an alternate notation, $\procn{x}$, when we
want to emphasize the connection to the use of the name. Note that, by
virtue of name equivalence, $\quotep{\procn{x}} \nameeq x$; so, the
notation is consistent with previous definitions.

Further, because names have structure it is possible to effect
substitutions on the basis of that structure. This means we need to
upgrade our notation for substitutions, which we accomplish by
adapting comprehension notation. Thus,

\begin{mathpar}
  P\{ y / x : x \in S \}
\end{mathpar}

is interpreted to mean the process derived from P by replacing (in a
capture-avoiding manner) each occurrence of $x$ in $S$ by $y$. For example,

\begin{mathpar}
  P\{ \quotep{\procn{x}|\procn{x}} / x : x \in \freenames{P} \}
\end{mathpar}

will replace each (occurrence) of a free name $x$ in $P$ by
$\quotep{\procn{x}|\procn{x}}$.

Also, we will avail ourselves of the notation $x^{L}$ and $x^{R}$ to
denote injections of a name into disjoint copies of the name
space. There are numerous ways to accomplish this. One example can be
found in \cite{MeredithR05}. This notation overloads to vectors of
names: $\vec{x}^{\pi} := (x_{i}^{\pi} \; : \; 0 \leq i < |\vec{x}| )$ where $\pi \in \{L,R\}$.

We also use $P^{\Box} := P|\Box$.

In \cite{MeredithR05} an interpretation of the new operator is
given. It turns out that there are several possible interpretations
all enjoying the requisite algebraic properties of the operator (see
\cite{milner91polyadicpi}). We will therefore make liberal use of
$(\nu\; \vec{x})P$.

% subsection the_syntax_and_semantics_of_the_notation_system (end)   

\input{qm2pi.qmops} 

\input{qm2pi.sterngerlach} 

\input{qm2pi.metric} 

% section concurrent_process_calculi (end)

%\input{qm2pi.proofsketch}

% section proof sketch (end)

%\input{qm2pi.slviaknots} 

% section spatial logic via knots (end)

\input{qm2pi.conclusion}

% section conclusion (end)

%\input{qm2pi.dtcodes} 

% section wiring algorithm (end)

\input{qm2pi.ack} 

% section acknowledgments (end)

\newpage


\bibliographystyle{plain}   
\bibliography{../../biblios/main.bib}

\input{qm2pi.rhodetails}

\end{document}

 

\documentclass[12pt]{llncs}
%\documentclass{jktr}

\usepackage[pdftex]{hyperref}                   
\usepackage {listings}
\usepackage {mathpartir}
\usepackage{bcprules}
%\usepackage{listings}
                       
\usepackage{graphicx} 
%\usepackage[margins=2.5cm,nohead,nofoot]{geometry}
%\usepackage{geometry}
\usepackage{amsfonts}
\usepackage{amstext}
\usepackage{latexsym}
\usepackage{amssymb}
\usepackage{color}


%\include{myPreamble}
\include{qm2pi.local} 

%\ifpdf
%\usepackage[pdftex]{graphicx}
%\else
%\usepackage{graphicx}
%\fi

 % \ifpdf
%  \usepackage{pdfsync}
%  \if


%\title{Brief Article}
%\author{David F. Snyder}
%\author{L.G. Meredith}

%\address{Dept. of Math., Texas State University--San Marcos, San Marcos, TX 78666}
       
\pagestyle{empty}


\begin{document}

\lstset{language=[Objective]Caml,frame=shadowbox}

\input{qm2pi.front}

% section front matter (end)

\input{qm2pi.intro} 
 
% section introduction (end)

% \input{qm2pi.knotations} 

% section notation (end)

\input{qm2pi.process.calculi} 

% section concurrent_process_calculi_and_spatial_logics_ (end)
    
%\input{qm2pi.knots2pi} 

%\input{qm2pi.trefoil} 

%\input{qm2pi.mainthm} 

% subsection basic_interpretation (end)

%\input{qm2pi.rho.presentation} 
\subsection{The syntax and semantics of the notation system}\label{sub:the_syntax_and_semantics_of_the_notation_system} % (fold)

We now summarize a technical presentation of the calculus that
embodies our theory of dynamics. The typical presentation of such a
calculus follows the style of giving generators and relations on
them. The grammar, below, describing term constructors, freely
generates the set of processes, $\Proc$. This set is then quotiented
by a relation known as structural congruence and it is over this set
that the notion of dynamics is expressed. This presentation is
essentially that of \cite{MeredithR05} with the addition of
polyadicity and summation. For readability we have relegated some of
the technical subtleties to an appendix.

\subsubsection{Process grammar}\label{subsub:process_grammar}

\begin{mathpar}
  \inferrule* [lab=synchronization] {} {{M} \bc \pzero \;|\; x?F \;|\; x!C }
  \and
  \inferrule* [lab=abstraction] {} {{F} \bc (x)P}
  \and
  \inferrule* [lab=concretion] {} {{C} \bc \langle Q \rangle}
  \and
  \inferrule* [lab=process] {} {{P,Q} \bc M \;| \;P|Q \;|\; @{x}}
  \and
  \inferrule* [lab=name] {} {{x} \bc \quotep{P}}
\end{mathpar} 

Note that $\vec{x}$ (resp. $\vec{P}$) denotes a vector of names
(resp. processes) of length $|\vec{x}|$ (resp. $|\vec{P}|$). We adopt
the following useful abbreviations.

\begin{mathpar}
   x?(\vec{y}).P := x.(\vec{y})P \and  x\clift{\vec{P}} := x.\clift{\vec{P}}
   \and x!(y) := \lift{x}{\dropn{y}}
   \and \Pi_{i=0}^{n-1}P_i := P_0 | \ldots | P_{n-1}
\end{mathpar}

\subsubsection{Structural congruence}

\paragraph{Free and bound names and alpha-equivalence.} At the
core of structural equivalence is alpha-equivalence which identifies
process that are the same up to a change of variable. Formally, we
recognize the distinction between free and bound names. The free names
of a process, $\freenames{P}$, may be calculated recursively as
follows:

\begin{mathpar}
\freenames{\pzero} := \emptyset
  \and \\
  \freenames{x?(y).P} := \{ x \} \cup (\freenames{P} \setminus \{ y \})
  \and 
  \freenames{x!\langle P \rangle} := \{ x \} \cup \{ P \} 
  \and \\
  \freenames{P|Q} := \freenames{P} \cup \freenames{Q}
  \and \\
  \freenames{@{x}} := \{ x \}
\end{mathpar}

$\pi$
$\quotep{\pi}$

$\freenames{-} : \pi \to \mathcal{P}(\quotep{\pi})$

\begin{eqnarray*}
  \freenames{\pzero} & := & \emptyset \\
  \freenames{x?(y).P} & := & \{ x \} \cup (\freenames{P} \setminus \{ y \}) \\
  \freenames{x!\langle P \rangle} & := & \{ x \} \cup \{ P \} \\
  \freenames{P|Q} & := & \freenames{P} \cup \freenames{Q} \\
  \freenames{\dropn{x}} & := & \{ x \}
\end{eqnarray*}

The bound names of a process, $\boundnames{P}$, are those names occurring in $P$
that are not free. For example, in $x?(y).0$, the name $x$ is free, while $y$ is bound.

\begin{mathpar}
  \inferrule* [lab=monoidal-laws] {} { P|Q \equiv Q|P \and P|0 \equiv P \and P|(Q|R) \equiv (P|Q)|R }
\end{mathpar}

\begin{mathpar}
  \inferrule* [lab=alpha-equivalence] {} { (x)P \equiv (y)P\{y/x\} \and y \not\in \freenames{P} }
\end{mathpar}

\begin{definition}
Then two processes, $P,Q$, are alpha-equivalent if $P = Q\{\vec{y}/\vec{x}\}$ for
some $\vec{x} \in \boundnames{Q},\vec{y} \in \boundnames{P}$, where $Q\{\vec{y}/\vec{x}\}$
denotes the capture-avoiding substitution of $\vec{y}$ for $\vec{x}$ in $Q$.
\end{definition}

\begin{definition}
  The {\em structural congruence} \cite{SangiorgiWalker} , $\equiv$,
  between processes is the least congruence containing
  alpha-equivalence, satisfying the abelian monoid laws
  (associativity, commutativity and $\pzero$ as identity) for parallel
  composition $|$ and for summation $+$.
\end{definition}

\subsection{Name equivalence}

We take name equivalence, written $\nameeq$, to be the smallest
equivalence relation generated by the following rules.

\begin{mathpar}
\inferrule*[lab=Quote-drop]
{ }
{ \quotep{@{x}} \nameeq x }

\inferrule*[lab=Struct-equiv]
{ P \scong Q }
{ \quotep{P} \nameeq \quotep{Q} }
\end{mathpar}

The astute reader will have noticed that the mutual recursion of names
and processes imposes a mutual recursion on alpha-equivalence and
structural equivalence via name-equivalence. Fortunately, all of this
works out pleasantly and we may calculate in the natural way, free of
concern. The reader interested in the details is referred to the
appendix \ref{appendix:rho_details}.

\subsection{Substitution}

We use $\Proc$ for the set of processes, $\QProc$ for the set of
names, and $\id{\{}\vec{y} / \vec{x} \id{\}}$ to denote partial maps,
$s : \QProc \rightarrow \QProc$. A map, $s$ lifts, uniquely, to a map
on process terms, $\widehat{s} : \Proc \rightarrow \Proc$ by the
following equations.

\begin{mathpar}
  (0) \psubstp{Q}{P} := 0 \\
  (R \juxtap S) \psubstp{Q}{P}
  :=    
  (R)\psubstp{Q}{P} \juxtap (S) \psubstp{Q}{P} \\
  (x?(y).R) \psubstp{Q}{P}    
  :=    
  (x)\substp{Q}{P} (z)\concat( (R \psubstn{z}{y}) \psubstp{Q}{P} ) \\
  (\lift{x}{R}) \psubstp{Q}{P}  
  :=
  \lift{(x)\substp{Q}{P}}{ R \psubstp{Q}{P} } \\
%   (\dropn{x})  \psubstp{Q}{P}       
%   := 
%   \left\{ 
%     \begin{array}{ccc} 
%       \dropn{\quotep{Q}} & & x \nameeq \quotep{P} \\
%       \dropn{x} & & otherwise \\
%     \end{array}
%   \right. 
  (\dropn{x})  \psubstp{Q}{P}       
  := 
  \left\{ 
    \begin{array}{ccc} 
      Q & & x \nameeq \quotep{P} \\
      \dropn{x} & & otherwise \\
    \end{array}
  \right.
\end{mathpar}
 

where

\begin{eqnarray}
  (x)\id{\{} \lpquote Q \rpquote / \lpquote P \rpquote \id{\}}            = 
  \left\{ 
    \begin{array}{ccc}
      \lpquote Q \rpquote & & x \nameeq \lpquote P \rpquote \\
      x & & otherwise \\
    \end{array}
  \right. \nonumber
\end{eqnarray}

and $z$ is chosen distinct from $\quotep{P}$, $\quotep{Q}$, the free
names in $Q$, and all the names in $R$. Our $\alpha$-equivalence will
be built in the standard way from this substitution.

\begin{remark}\label{rem:no_self_referential_names}
  One consequence of these definitions is that $\forall P. \quotep{P}
  \not\in \freenames{P}$.
\end{remark}

\subsection{ Dynamic quote: an example }

Anticipating something of what's to come, consider applying the
substitution, $\widehat{\id{\{}u / z \id{\}}}$, to the following pair
of processes, $\lift{w}{y!(z)}$ and $w[ \lpquote y!(z) \rpquote ]$.

\begin{eqnarray}
	\lift{w}{y!(z)}\widehat{\id{\{}u / z \id{\}}}
		& = &
		\lift{w}{y!(u)} \nonumber\\
	w[ \lpquote y!(z) \rpquote ] \widehat{ \id{\{}u / z \id{\}} }
		& = &
		w[ \lpquote y!(z) \rpquote ] \nonumber
\end{eqnarray}

Because the body of the process between quotes is impervious to
substitution, we get radically different answers. In fact, by
examining the first process in an input context,
e.g. $x?(z).\lift{w}{y!(z)}$, we see that the process under the lift
operator may be shaped by prefixed inputs binding a name inside it. In
this sense, the lift operator will be seen as a way to dynamically
construct processes before reifying them as names.

Finally equipped with these standard features we can present the
dynamics of the calculus.

\subsubsection{Operational semantics} 

Finally, we introduce the computational dynamics. What marks these
algebras as distinct from other more traditionally studied algebraic
structures, e.g. vector spaces or polynomial rings, is the manner in
which dynamics is captured. In traditional structures, dynamics is typically
expressed through morphisms between such structures, as in linear maps
between vector spaces or morphisms between rings. In algebras
associated with the semantics of computation, the dynamics is
expressed as part of the algebraic structure itself, through a
reduction reduction relation typically denoted by $\red$. Below, we
give a recursive presentation of this relation for the calculus used
in the encoding.

$\red \subseteq \pi \times \pi$
$\red : \pi \to \mathcal{P}(\pi)$

\begin{mathpar}
  \inferrule* [lab=Comm] { \textsf{match}( x_{src}, x_{trgt} ) } { x_{trgt}?(y)P \; | \; x_{src}!\langle {Q} \rangle \red P\{\quotep{Q}/y}\} }
  \and \\
  \inferrule* [lab=Par] {{P} \red {P}'} {{{P} | {Q}} \red {{P}' | {Q}}}
  \and
  \inferrule* [lab=Equiv]{{{P} \scong {P}'} \andalso {{P}' \red {Q}'} \andalso {{Q}' \scong {Q}}}{{P} \red {Q}}
\end{mathpar}

\begin{eqnarray*}
  match_{\equiv} (\quotep{P},\quotep{Q}) & := & P \equiv Q \\
  match_{\dagger}(\quotep{P},\quotep{Q}) & := & \forall R. P|Q \red^{*} R => R \red^{*} 0 \\
  match_{K}(\quotep{P},\quotep{Q}) & := & K \mbox{ for some context } K
\end{eqnarray*}

$u?(x)P | u!\langle Q \rangle \red P\{\quotep{Q}/x\}$

%We write $\wred$ for $\red^*$, and $P\red$ if $\exists Q $ such that $ P \red Q$.
We write $P\red$ if $\exists Q $ such that $ P \red Q$ and $P\not\red$, otherwise.

\section{Replication}

As mentioned before, it is known that replication (and hence
recursion) can be implemented in a higher-order process algebra
\cite{SangiorgiWalker}. As our first example of calculation with the
machinery thus far presented we give the construction explicitly in
the {\rhoc}.

\begin{eqnarray}
	D_{x} & := & \prefix{x}{y}{(\binpar{\outputp{x}{y}}{@{y}})} \nonumber\\
	\bangp_{x}{P} & := & \binpar{{x}!\langle{\binpar{D_{x}}{P}}\rangle}{D_{x}} \nonumber
\end{eqnarray}

\begin{eqnarray}
	\bangp_{x}{P} & & \nonumber\\
	=
	& {x}!\langle{(\prefix{x}{y}{(\outputp{x}{y} | @{y})) | P}}\rangle 
	      | \prefix{x}{y}{(\outputp{x}{y} | @{y})} & \nonumber\\
	\red
	& (\outputp{x}{y} | @{y})\substn{\quotep{(\prefix{x}{y}{(@{y} | \outputp{x}{y})) | P}}}{y} & \nonumber\\
	=
	& \outputp{x}{\quotep{(\prefix{x}{y}{(\outputp{x}{y} | @{y})) | P}}}
	  | {(\prefix{x}{y}{(\outputp{x}{y} | @{y})) | P}} & \nonumber\\
	\red
	& \ldots & \nonumber\\
	\red^*
	& P | P | \ldots & \nonumber
\end{eqnarray}

Of course, this encoding, as an implementation, runs away, unfolding
$\bangp{P}$ eagerly. A lazier and more implementable replication
operator, restricted to input-guarded processes, may be obtained as follows.

\begin{eqnarray}
\bangp{\prefix{u}{v}{P}} 
	:= 
	\binpar{\lift{x}{\prefix{u}{v}{(\binpar{D(x)}{P})}}}{D(x)} \nonumber
\end{eqnarray}

\begin{remark}
  Note that the lazier definition still does not deal with summation
  or mixed summation (i.e. sums over input and output). The reader is
  invited to construct definitions of replication that deal with these
  features. 

  Further, the definitions are parameterized in a name, $x$. Can you,
  gentle reader, make a definition that eliminates this parameter and
  guarantees no accidental interaction between the replication
  machinery and the process being replicated -- i.e. no accidental
  sharing of names used by the process to get its work done and the
  name(s) used by the replication to effect copying. This latter
  revision of the definition of replication is crucial to obtaining
  the expected identity $!!P \sim !P$.
\end{remark}

\begin{remark}\label{rem:paradoxical_combinator}
  The reader familiar with the lambda calculus will have noticed the
  similarity between $D$ and the paradoxical combinator.

  [Ed. note: the existence of this seems to suggest we have to be more
  restrictive on the set of processes and names we admit if we are to
  support no-cloning.]
\end{remark}

\subsubsection{Bisimulation}

The computational dynamics gives rise to another kind of equivalence,
the equivalence of computational behavior. As previously mentioned
this is typically captured \emph{via} some form of bisimulation.

% The notion we use in this paper is weak barbed bisimulation
% \cite{milner91polyadicpi}.

The notion we use in this paper is derived from weak barbed
bisimulation \cite{milner91polyadicpi}. 

\begin{definition}
An \emph{observation relation}, $\downarrow_{\mathcal N}$, over a set
of names, $\mathcal N$, is the smallest relation satisfying the rules
below.

\infrule[Out-barb]{y \in {\mathcal N}, \; x \nameeq y}
		  {\outputp{x}{v} \downarrow_{\mathcal N} x}
\infrule[Par-barb]{\mbox{$P\downarrow_{\mathcal N} x$ or $Q\downarrow_{\mathcal N} x$}}
		  {\binpar{P}{Q} \downarrow_{\mathcal N} x}

We write $P \Downarrow_{\mathcal N} x$ if there is $Q$ such that 
$P \wred Q$ and $Q \downarrow_{\mathcal N} x$.
\end{definition}

\begin{definition}
%\label{def.bbisim}
An  ${\mathcal N}$-\emph{barbed bisimulation} over a set of names, ${\mathcal N}$, is a symmetric binary relation 
${\mathcal S}_{\mathcal N}$ between agents such that $P\rel{S}_{\mathcal N}Q$ implies:
\begin{enumerate}
\item If $P \red P'$ then $Q \wred Q'$ and $P'\rel{S}_{\mathcal N} Q'$.
\item If $P\downarrow_{\mathcal N} x$, then $Q\Downarrow_{\mathcal N} x$.
\end{enumerate}
$P$ is ${\mathcal N}$-barbed bisimilar to $Q$, written
$P \wbbisim_{\mathcal N} Q$, if $P \rel{S}_{\mathcal N} Q$ for some ${\mathcal N}$-barbed bisimulation ${\mathcal S}_{\mathcal N}$.
\end{definition}

$\mathcal{R} \subseteq \pi \times \pi$

$P \mathcal{R} Q => \forall P'. P \red P' \Rightarrow \exists Q'. Q \red Q', P' \mathcal{R} Q'$

$P \vdash x \Rightarrow Q \vdash x$

\begin{mathpar}
  \inferrule*[lab=Out-barb]{x \nameeq y}{{y}!\langle{Q}\rangle \vdash x}
  \and
  \inferrule*[lab=Par-barb]{\mbox{$P\vdash x$ or $Q\vdash x$}}{\binpar{P}{Q} \vdash x}
\end{mathpar}

\subsubsection{Contexts}

One of the principle advantages of computational calculi like the
$\pi$-calculus is a well-defined notion of context,
contextual-equivalence and a correlation between
contextual-equivalence and notions of bisimulation. The notion of
context allows the decomposition of a process into (sub-)process and
its syntactic environment, its context. Thus, a context may be
thought of as a process with a ``hole'' (written $\Box$) in it. The
application of a context $M$ to a process $P$, written $M[P]$, is
tantamount to filling the hole in $M$ with $P$. In this paper we do
not need the full weight of this theory, but do make use of the notion
of context in the proof the main theorem. 

\begin{mathpar}
  \inferrule* [lab=summation] {} {{M_{M},M_{N}} \bc \Box \;|\; x.M_{A} \;|\; M_{M}+M_{N}}
  \and
  \inferrule* [lab=agent] {} {{M_{A}} \bc (\vec{x})M_{P} \;| \; \clift{P_0,\ldots,M_{P},\ldots,P_N}}
  \and \\
  \inferrule* [lab=process] {} {{M_{P}} \bc M_{N} \;| \;P|M_{P} }
\end{mathpar} 

\begin{mathpar}
  \inferrule* [lab=sychronization] {} {M_{N} \bc \Box \;|\; x?M_{F} \;|\; x!M_{C}}
  \and
  \inferrule* [lab=abstraction] {} {{M_{F}} \bc (x)M_{P} }
  \and
  \inferrule* [lab=concretion] {} {{M_{C}} \bc \langle M_{P} \rangle }
  \and \\
  \inferrule* [lab=process] {} {{M_{P}} \bc M_{N} \;| \;P|M_{P} }
\end{mathpar}

\begin{definition}[contextual application] Given a context $M$, and
  process $P$, we define the \emph{contextual application}, $M[P] :=
  M\{P/\Box\}$. That is, the contextual application of M to P is the
  substitution of $P$ for $\Box$ in $M$.
\end{definition}

$\meaningof{-} : L \to \mathcal{P}(\pi)$

\begin{mathpar}
  \inferrule* [lab=collection] {} {\meaningof{true} = \pi, \and \meaningof{~E} = \pi \setminus \meaningof{E}, \and \meaningof{E_{1} \& E_{2}} = \meaningof{E_{1}} \cap \meaningof{E_{2}}}
\end{mathpar}

\begin{mathpar}
  \inferrule* [lab=structure] {} {\meaningof{0} = \{ P \in \pi | P \equiv 0 \}, \and \\ \meaningof{E_1 | E_2} = \{ P \in \pi | P \equiv P_{1} | P_{2}, P_{1} \in \meaningof{E_{1}}, P_{2} \in \meaningof{E_2}\} }
\end{mathpar}

\begin{mathpar}
 \inferrule* [lab=behavior] {} {\meaningof{\langle a?b \rangle E} = \{ P \in \pi | P \equiv Q | u?(y)P', \\ \and \\\\ \and \\ \;\;\; u \in \meaningof{a}, \forall z.P'\{z/y\} \in \meaningof{E\{z/b\}}\}, \and \\ \meaningof{a!E} = \{ P \in \pi | P \equiv Q | x!\langle P' \rangle, x \in \meaningof{a} P' \in \meaningof{E}\} }
\end{mathpar}

\begin{mathpar}
 \inferrule* [lab=nominal] {} {\meaningof{\quotep{E}} = \{ \quotep{P} \in \quotep{\pi} | P \in \meaningof{E} \}, \and \meaningof{\quotep{P}} = \{ \quotep{Q} \in \quotep{\pi} | P \equiv Q \} \and \\ \meaningof{@\quotep{E}} = \{ P \in \pi | P \equiv @x, x \in \meaningof{E} \}}
\end{mathpar}

\begin{eqnarray*}
  \\
  \meaningof{-} : TS \to ST
\end{eqnarray*}

\begin{eqnarray*}
  \\
  L : TS \to ST
\end{eqnarray*}

\begin{eqnarray*}
  \\
  P \models E \iff P \in \meaningof{E}
\end{eqnarray*}

\begin{eqnarray*}
  P \approx_{L} Q \iff \forall E \in L. P \models E \iff Q \models E
\end{eqnarray*}

\begin{eqnarray*}
  P \approx_{K} Q
\end{eqnarray*}

\begin{eqnarray*}
  P \approx Q
\end{eqnarray*}

$\approx_{K} = \approx = \approx_{L}$

\subsubsection{Contextual duality}

Note that contexts extend the quotation operation to a family of
operations from processes to names. Given a context, $M$, we can
define a \emph{nominal context}, $\quotep{M}$ by $\quotep{M}[P] :=
\quotep{M[P]}$. To foreshadow what is to come we observe that these
operations enjoy a duality with processes very much like the duality
between vectors and maps from vectors to scalars.

Further, because the calculus is essentially higher-order, we have a
correspondence between contexts and processes. More specifically,
given a name $x$ and a context $M$ we can construct $M^{*}_{x}$ such
that 

\begin{mathpar}
  M^{*}_{x} | \lift{x}{P} \red M[P]
\end{mathpar}

namely,

\begin{mathpar}
  M^{*}_{x} := x?(u).M[\dropn{u}]
\end{mathpar}

The dependence of $M^{*}_{x}$ on a name makes it an abstraction, 

\begin{mathpar}
  M^{*} := (x)x?(u).M[\dropn{u}]
\end{mathpar}

\subsection{Additional notation}

It will sometimes be convenient to denote the process a name
quotes. We already have the notation $x = \quotep{P}$, but it will be
convenient to introduce an alternate notation, $\procn{x}$, when we
want to emphasize the connection to the use of the name. Note that, by
virtue of name equivalence, $\quotep{\procn{x}} \nameeq x$; so, the
notation is consistent with previous definitions.

Further, because names have structure it is possible to effect
substitutions on the basis of that structure. This means we need to
upgrade our notation for substitutions, which we accomplish by
adapting comprehension notation. Thus,

\begin{mathpar}
  P\{ y / x : x \in S \}
\end{mathpar}

is interpreted to mean the process derived from P by replacing (in a
capture-avoiding manner) each occurrence of $x$ in $S$ by $y$. For example,

\begin{mathpar}
  P\{ \quotep{\procn{x}|\procn{x}} / x : x \in \freenames{P} \}
\end{mathpar}

will replace each (occurrence) of a free name $x$ in $P$ by
$\quotep{\procn{x}|\procn{x}}$.

Also, we will avail ourselves of the notation $x^{L}$ and $x^{R}$ to
denote injections of a name into disjoint copies of the name
space. There are numerous ways to accomplish this. One example can be
found in \cite{MeredithR05}. This notation overloads to vectors of
names: $\vec{x}^{\pi} := (x_{i}^{\pi} \; : \; 0 \leq i < |\vec{x}| )$ where $\pi \in \{L,R\}$.

We also use $P^{\Box} := P|\Box$.

In \cite{MeredithR05} an interpretation of the new operator is
given. It turns out that there are several possible interpretations
all enjoying the requisite algebraic properties of the operator (see
\cite{milner91polyadicpi}). We will therefore make liberal use of
$(\nu\; \vec{x})P$.

% subsection the_syntax_and_semantics_of_the_notation_system (end)   

\input{qm2pi.qmops} 

\input{qm2pi.sterngerlach} 

\input{qm2pi.metric} 

% section concurrent_process_calculi (end)

%\input{qm2pi.proofsketch}

% section proof sketch (end)

%\input{qm2pi.slviaknots} 

% section spatial logic via knots (end)

\input{qm2pi.conclusion}

% section conclusion (end)

%\input{qm2pi.dtcodes} 

% section wiring algorithm (end)

\input{qm2pi.ack} 

% section acknowledgments (end)

\newpage


\bibliographystyle{plain}   
\bibliography{../../biblios/main.bib}

\input{qm2pi.rhodetails}

\end{document}

 

% section concurrent_process_calculi (end)

%\documentclass[12pt]{llncs}
%\documentclass{jktr}

\usepackage[pdftex]{hyperref}                   
\usepackage {listings}
\usepackage {mathpartir}
\usepackage{bcprules}
%\usepackage{listings}
                       
\usepackage{graphicx} 
%\usepackage[margins=2.5cm,nohead,nofoot]{geometry}
%\usepackage{geometry}
\usepackage{amsfonts}
\usepackage{amstext}
\usepackage{latexsym}
\usepackage{amssymb}
\usepackage{color}


%\include{myPreamble}
\include{qm2pi.local} 

%\ifpdf
%\usepackage[pdftex]{graphicx}
%\else
%\usepackage{graphicx}
%\fi

 % \ifpdf
%  \usepackage{pdfsync}
%  \if


%\title{Brief Article}
%\author{David F. Snyder}
%\author{L.G. Meredith}

%\address{Dept. of Math., Texas State University--San Marcos, San Marcos, TX 78666}
       
\pagestyle{empty}


\begin{document}

\lstset{language=[Objective]Caml,frame=shadowbox}

\input{qm2pi.front}

% section front matter (end)

\input{qm2pi.intro} 
 
% section introduction (end)

% \input{qm2pi.knotations} 

% section notation (end)

\input{qm2pi.process.calculi} 

% section concurrent_process_calculi_and_spatial_logics_ (end)
    
%\input{qm2pi.knots2pi} 

%\input{qm2pi.trefoil} 

%\input{qm2pi.mainthm} 

% subsection basic_interpretation (end)

%\input{qm2pi.rho.presentation} 
\subsection{The syntax and semantics of the notation system}\label{sub:the_syntax_and_semantics_of_the_notation_system} % (fold)

We now summarize a technical presentation of the calculus that
embodies our theory of dynamics. The typical presentation of such a
calculus follows the style of giving generators and relations on
them. The grammar, below, describing term constructors, freely
generates the set of processes, $\Proc$. This set is then quotiented
by a relation known as structural congruence and it is over this set
that the notion of dynamics is expressed. This presentation is
essentially that of \cite{MeredithR05} with the addition of
polyadicity and summation. For readability we have relegated some of
the technical subtleties to an appendix.

\subsubsection{Process grammar}\label{subsub:process_grammar}

\begin{mathpar}
  \inferrule* [lab=synchronization] {} {{M} \bc \pzero \;|\; x?F \;|\; x!C }
  \and
  \inferrule* [lab=abstraction] {} {{F} \bc (x)P}
  \and
  \inferrule* [lab=concretion] {} {{C} \bc \langle Q \rangle}
  \and
  \inferrule* [lab=process] {} {{P,Q} \bc M \;| \;P|Q \;|\; @{x}}
  \and
  \inferrule* [lab=name] {} {{x} \bc \quotep{P}}
\end{mathpar} 

Note that $\vec{x}$ (resp. $\vec{P}$) denotes a vector of names
(resp. processes) of length $|\vec{x}|$ (resp. $|\vec{P}|$). We adopt
the following useful abbreviations.

\begin{mathpar}
   x?(\vec{y}).P := x.(\vec{y})P \and  x\clift{\vec{P}} := x.\clift{\vec{P}}
   \and x!(y) := \lift{x}{\dropn{y}}
   \and \Pi_{i=0}^{n-1}P_i := P_0 | \ldots | P_{n-1}
\end{mathpar}

\subsubsection{Structural congruence}

\paragraph{Free and bound names and alpha-equivalence.} At the
core of structural equivalence is alpha-equivalence which identifies
process that are the same up to a change of variable. Formally, we
recognize the distinction between free and bound names. The free names
of a process, $\freenames{P}$, may be calculated recursively as
follows:

\begin{mathpar}
\freenames{\pzero} := \emptyset
  \and \\
  \freenames{x?(y).P} := \{ x \} \cup (\freenames{P} \setminus \{ y \})
  \and 
  \freenames{x!\langle P \rangle} := \{ x \} \cup \{ P \} 
  \and \\
  \freenames{P|Q} := \freenames{P} \cup \freenames{Q}
  \and \\
  \freenames{@{x}} := \{ x \}
\end{mathpar}

$\pi$
$\quotep{\pi}$

$\freenames{-} : \pi \to \mathcal{P}(\quotep{\pi})$

\begin{eqnarray*}
  \freenames{\pzero} & := & \emptyset \\
  \freenames{x?(y).P} & := & \{ x \} \cup (\freenames{P} \setminus \{ y \}) \\
  \freenames{x!\langle P \rangle} & := & \{ x \} \cup \{ P \} \\
  \freenames{P|Q} & := & \freenames{P} \cup \freenames{Q} \\
  \freenames{\dropn{x}} & := & \{ x \}
\end{eqnarray*}

The bound names of a process, $\boundnames{P}$, are those names occurring in $P$
that are not free. For example, in $x?(y).0$, the name $x$ is free, while $y$ is bound.

\begin{mathpar}
  \inferrule* [lab=monoidal-laws] {} { P|Q \equiv Q|P \and P|0 \equiv P \and P|(Q|R) \equiv (P|Q)|R }
\end{mathpar}

\begin{mathpar}
  \inferrule* [lab=alpha-equivalence] {} { (x)P \equiv (y)P\{y/x\} \and y \not\in \freenames{P} }
\end{mathpar}

\begin{definition}
Then two processes, $P,Q$, are alpha-equivalent if $P = Q\{\vec{y}/\vec{x}\}$ for
some $\vec{x} \in \boundnames{Q},\vec{y} \in \boundnames{P}$, where $Q\{\vec{y}/\vec{x}\}$
denotes the capture-avoiding substitution of $\vec{y}$ for $\vec{x}$ in $Q$.
\end{definition}

\begin{definition}
  The {\em structural congruence} \cite{SangiorgiWalker} , $\equiv$,
  between processes is the least congruence containing
  alpha-equivalence, satisfying the abelian monoid laws
  (associativity, commutativity and $\pzero$ as identity) for parallel
  composition $|$ and for summation $+$.
\end{definition}

\subsection{Name equivalence}

We take name equivalence, written $\nameeq$, to be the smallest
equivalence relation generated by the following rules.

\begin{mathpar}
\inferrule*[lab=Quote-drop]
{ }
{ \quotep{@{x}} \nameeq x }

\inferrule*[lab=Struct-equiv]
{ P \scong Q }
{ \quotep{P} \nameeq \quotep{Q} }
\end{mathpar}

The astute reader will have noticed that the mutual recursion of names
and processes imposes a mutual recursion on alpha-equivalence and
structural equivalence via name-equivalence. Fortunately, all of this
works out pleasantly and we may calculate in the natural way, free of
concern. The reader interested in the details is referred to the
appendix \ref{appendix:rho_details}.

\subsection{Substitution}

We use $\Proc$ for the set of processes, $\QProc$ for the set of
names, and $\id{\{}\vec{y} / \vec{x} \id{\}}$ to denote partial maps,
$s : \QProc \rightarrow \QProc$. A map, $s$ lifts, uniquely, to a map
on process terms, $\widehat{s} : \Proc \rightarrow \Proc$ by the
following equations.

\begin{mathpar}
  (0) \psubstp{Q}{P} := 0 \\
  (R \juxtap S) \psubstp{Q}{P}
  :=    
  (R)\psubstp{Q}{P} \juxtap (S) \psubstp{Q}{P} \\
  (x?(y).R) \psubstp{Q}{P}    
  :=    
  (x)\substp{Q}{P} (z)\concat( (R \psubstn{z}{y}) \psubstp{Q}{P} ) \\
  (\lift{x}{R}) \psubstp{Q}{P}  
  :=
  \lift{(x)\substp{Q}{P}}{ R \psubstp{Q}{P} } \\
%   (\dropn{x})  \psubstp{Q}{P}       
%   := 
%   \left\{ 
%     \begin{array}{ccc} 
%       \dropn{\quotep{Q}} & & x \nameeq \quotep{P} \\
%       \dropn{x} & & otherwise \\
%     \end{array}
%   \right. 
  (\dropn{x})  \psubstp{Q}{P}       
  := 
  \left\{ 
    \begin{array}{ccc} 
      Q & & x \nameeq \quotep{P} \\
      \dropn{x} & & otherwise \\
    \end{array}
  \right.
\end{mathpar}
 

where

\begin{eqnarray}
  (x)\id{\{} \lpquote Q \rpquote / \lpquote P \rpquote \id{\}}            = 
  \left\{ 
    \begin{array}{ccc}
      \lpquote Q \rpquote & & x \nameeq \lpquote P \rpquote \\
      x & & otherwise \\
    \end{array}
  \right. \nonumber
\end{eqnarray}

and $z$ is chosen distinct from $\quotep{P}$, $\quotep{Q}$, the free
names in $Q$, and all the names in $R$. Our $\alpha$-equivalence will
be built in the standard way from this substitution.

\begin{remark}\label{rem:no_self_referential_names}
  One consequence of these definitions is that $\forall P. \quotep{P}
  \not\in \freenames{P}$.
\end{remark}

\subsection{ Dynamic quote: an example }

Anticipating something of what's to come, consider applying the
substitution, $\widehat{\id{\{}u / z \id{\}}}$, to the following pair
of processes, $\lift{w}{y!(z)}$ and $w[ \lpquote y!(z) \rpquote ]$.

\begin{eqnarray}
	\lift{w}{y!(z)}\widehat{\id{\{}u / z \id{\}}}
		& = &
		\lift{w}{y!(u)} \nonumber\\
	w[ \lpquote y!(z) \rpquote ] \widehat{ \id{\{}u / z \id{\}} }
		& = &
		w[ \lpquote y!(z) \rpquote ] \nonumber
\end{eqnarray}

Because the body of the process between quotes is impervious to
substitution, we get radically different answers. In fact, by
examining the first process in an input context,
e.g. $x?(z).\lift{w}{y!(z)}$, we see that the process under the lift
operator may be shaped by prefixed inputs binding a name inside it. In
this sense, the lift operator will be seen as a way to dynamically
construct processes before reifying them as names.

Finally equipped with these standard features we can present the
dynamics of the calculus.

\subsubsection{Operational semantics} 

Finally, we introduce the computational dynamics. What marks these
algebras as distinct from other more traditionally studied algebraic
structures, e.g. vector spaces or polynomial rings, is the manner in
which dynamics is captured. In traditional structures, dynamics is typically
expressed through morphisms between such structures, as in linear maps
between vector spaces or morphisms between rings. In algebras
associated with the semantics of computation, the dynamics is
expressed as part of the algebraic structure itself, through a
reduction reduction relation typically denoted by $\red$. Below, we
give a recursive presentation of this relation for the calculus used
in the encoding.

$\red \subseteq \pi \times \pi$
$\red : \pi \to \mathcal{P}(\pi)$

\begin{mathpar}
  \inferrule* [lab=Comm] { \textsf{match}( x_{src}, x_{trgt} ) } { x_{trgt}?(y)P \; | \; x_{src}!\langle {Q} \rangle \red P\{\quotep{Q}/y}\} }
  \and \\
  \inferrule* [lab=Par] {{P} \red {P}'} {{{P} | {Q}} \red {{P}' | {Q}}}
  \and
  \inferrule* [lab=Equiv]{{{P} \scong {P}'} \andalso {{P}' \red {Q}'} \andalso {{Q}' \scong {Q}}}{{P} \red {Q}}
\end{mathpar}

\begin{eqnarray*}
  match_{\equiv} (\quotep{P},\quotep{Q}) & := & P \equiv Q \\
  match_{\dagger}(\quotep{P},\quotep{Q}) & := & \forall R. P|Q \red^{*} R => R \red^{*} 0 \\
  match_{K}(\quotep{P},\quotep{Q}) & := & K \mbox{ for some context } K
\end{eqnarray*}

$u?(x)P | u!\langle Q \rangle \red P\{\quotep{Q}/x\}$

%We write $\wred$ for $\red^*$, and $P\red$ if $\exists Q $ such that $ P \red Q$.
We write $P\red$ if $\exists Q $ such that $ P \red Q$ and $P\not\red$, otherwise.

\section{Replication}

As mentioned before, it is known that replication (and hence
recursion) can be implemented in a higher-order process algebra
\cite{SangiorgiWalker}. As our first example of calculation with the
machinery thus far presented we give the construction explicitly in
the {\rhoc}.

\begin{eqnarray}
	D_{x} & := & \prefix{x}{y}{(\binpar{\outputp{x}{y}}{@{y}})} \nonumber\\
	\bangp_{x}{P} & := & \binpar{{x}!\langle{\binpar{D_{x}}{P}}\rangle}{D_{x}} \nonumber
\end{eqnarray}

\begin{eqnarray}
	\bangp_{x}{P} & & \nonumber\\
	=
	& {x}!\langle{(\prefix{x}{y}{(\outputp{x}{y} | @{y})) | P}}\rangle 
	      | \prefix{x}{y}{(\outputp{x}{y} | @{y})} & \nonumber\\
	\red
	& (\outputp{x}{y} | @{y})\substn{\quotep{(\prefix{x}{y}{(@{y} | \outputp{x}{y})) | P}}}{y} & \nonumber\\
	=
	& \outputp{x}{\quotep{(\prefix{x}{y}{(\outputp{x}{y} | @{y})) | P}}}
	  | {(\prefix{x}{y}{(\outputp{x}{y} | @{y})) | P}} & \nonumber\\
	\red
	& \ldots & \nonumber\\
	\red^*
	& P | P | \ldots & \nonumber
\end{eqnarray}

Of course, this encoding, as an implementation, runs away, unfolding
$\bangp{P}$ eagerly. A lazier and more implementable replication
operator, restricted to input-guarded processes, may be obtained as follows.

\begin{eqnarray}
\bangp{\prefix{u}{v}{P}} 
	:= 
	\binpar{\lift{x}{\prefix{u}{v}{(\binpar{D(x)}{P})}}}{D(x)} \nonumber
\end{eqnarray}

\begin{remark}
  Note that the lazier definition still does not deal with summation
  or mixed summation (i.e. sums over input and output). The reader is
  invited to construct definitions of replication that deal with these
  features. 

  Further, the definitions are parameterized in a name, $x$. Can you,
  gentle reader, make a definition that eliminates this parameter and
  guarantees no accidental interaction between the replication
  machinery and the process being replicated -- i.e. no accidental
  sharing of names used by the process to get its work done and the
  name(s) used by the replication to effect copying. This latter
  revision of the definition of replication is crucial to obtaining
  the expected identity $!!P \sim !P$.
\end{remark}

\begin{remark}\label{rem:paradoxical_combinator}
  The reader familiar with the lambda calculus will have noticed the
  similarity between $D$ and the paradoxical combinator.

  [Ed. note: the existence of this seems to suggest we have to be more
  restrictive on the set of processes and names we admit if we are to
  support no-cloning.]
\end{remark}

\subsubsection{Bisimulation}

The computational dynamics gives rise to another kind of equivalence,
the equivalence of computational behavior. As previously mentioned
this is typically captured \emph{via} some form of bisimulation.

% The notion we use in this paper is weak barbed bisimulation
% \cite{milner91polyadicpi}.

The notion we use in this paper is derived from weak barbed
bisimulation \cite{milner91polyadicpi}. 

\begin{definition}
An \emph{observation relation}, $\downarrow_{\mathcal N}$, over a set
of names, $\mathcal N$, is the smallest relation satisfying the rules
below.

\infrule[Out-barb]{y \in {\mathcal N}, \; x \nameeq y}
		  {\outputp{x}{v} \downarrow_{\mathcal N} x}
\infrule[Par-barb]{\mbox{$P\downarrow_{\mathcal N} x$ or $Q\downarrow_{\mathcal N} x$}}
		  {\binpar{P}{Q} \downarrow_{\mathcal N} x}

We write $P \Downarrow_{\mathcal N} x$ if there is $Q$ such that 
$P \wred Q$ and $Q \downarrow_{\mathcal N} x$.
\end{definition}

\begin{definition}
%\label{def.bbisim}
An  ${\mathcal N}$-\emph{barbed bisimulation} over a set of names, ${\mathcal N}$, is a symmetric binary relation 
${\mathcal S}_{\mathcal N}$ between agents such that $P\rel{S}_{\mathcal N}Q$ implies:
\begin{enumerate}
\item If $P \red P'$ then $Q \wred Q'$ and $P'\rel{S}_{\mathcal N} Q'$.
\item If $P\downarrow_{\mathcal N} x$, then $Q\Downarrow_{\mathcal N} x$.
\end{enumerate}
$P$ is ${\mathcal N}$-barbed bisimilar to $Q$, written
$P \wbbisim_{\mathcal N} Q$, if $P \rel{S}_{\mathcal N} Q$ for some ${\mathcal N}$-barbed bisimulation ${\mathcal S}_{\mathcal N}$.
\end{definition}

$\mathcal{R} \subseteq \pi \times \pi$

$P \mathcal{R} Q => \forall P'. P \red P' \Rightarrow \exists Q'. Q \red Q', P' \mathcal{R} Q'$

$P \vdash x \Rightarrow Q \vdash x$

\begin{mathpar}
  \inferrule*[lab=Out-barb]{x \nameeq y}{{y}!\langle{Q}\rangle \vdash x}
  \and
  \inferrule*[lab=Par-barb]{\mbox{$P\vdash x$ or $Q\vdash x$}}{\binpar{P}{Q} \vdash x}
\end{mathpar}

\subsubsection{Contexts}

One of the principle advantages of computational calculi like the
$\pi$-calculus is a well-defined notion of context,
contextual-equivalence and a correlation between
contextual-equivalence and notions of bisimulation. The notion of
context allows the decomposition of a process into (sub-)process and
its syntactic environment, its context. Thus, a context may be
thought of as a process with a ``hole'' (written $\Box$) in it. The
application of a context $M$ to a process $P$, written $M[P]$, is
tantamount to filling the hole in $M$ with $P$. In this paper we do
not need the full weight of this theory, but do make use of the notion
of context in the proof the main theorem. 

\begin{mathpar}
  \inferrule* [lab=summation] {} {{M_{M},M_{N}} \bc \Box \;|\; x.M_{A} \;|\; M_{M}+M_{N}}
  \and
  \inferrule* [lab=agent] {} {{M_{A}} \bc (\vec{x})M_{P} \;| \; \clift{P_0,\ldots,M_{P},\ldots,P_N}}
  \and \\
  \inferrule* [lab=process] {} {{M_{P}} \bc M_{N} \;| \;P|M_{P} }
\end{mathpar} 

\begin{mathpar}
  \inferrule* [lab=sychronization] {} {M_{N} \bc \Box \;|\; x?M_{F} \;|\; x!M_{C}}
  \and
  \inferrule* [lab=abstraction] {} {{M_{F}} \bc (x)M_{P} }
  \and
  \inferrule* [lab=concretion] {} {{M_{C}} \bc \langle M_{P} \rangle }
  \and \\
  \inferrule* [lab=process] {} {{M_{P}} \bc M_{N} \;| \;P|M_{P} }
\end{mathpar}

\begin{definition}[contextual application] Given a context $M$, and
  process $P$, we define the \emph{contextual application}, $M[P] :=
  M\{P/\Box\}$. That is, the contextual application of M to P is the
  substitution of $P$ for $\Box$ in $M$.
\end{definition}

$\meaningof{-} : L \to \mathcal{P}(\pi)$

\begin{mathpar}
  \inferrule* [lab=collection] {} {\meaningof{true} = \pi, \and \meaningof{~E} = \pi \setminus \meaningof{E}, \and \meaningof{E_{1} \& E_{2}} = \meaningof{E_{1}} \cap \meaningof{E_{2}}}
\end{mathpar}

\begin{mathpar}
  \inferrule* [lab=structure] {} {\meaningof{0} = \{ P \in \pi | P \equiv 0 \}, \and \\ \meaningof{E_1 | E_2} = \{ P \in \pi | P \equiv P_{1} | P_{2}, P_{1} \in \meaningof{E_{1}}, P_{2} \in \meaningof{E_2}\} }
\end{mathpar}

\begin{mathpar}
 \inferrule* [lab=behavior] {} {\meaningof{\langle a?b \rangle E} = \{ P \in \pi | P \equiv Q | u?(y)P', \\ \and \\\\ \and \\ \;\;\; u \in \meaningof{a}, \forall z.P'\{z/y\} \in \meaningof{E\{z/b\}}\}, \and \\ \meaningof{a!E} = \{ P \in \pi | P \equiv Q | x!\langle P' \rangle, x \in \meaningof{a} P' \in \meaningof{E}\} }
\end{mathpar}

\begin{mathpar}
 \inferrule* [lab=nominal] {} {\meaningof{\quotep{E}} = \{ \quotep{P} \in \quotep{\pi} | P \in \meaningof{E} \}, \and \meaningof{\quotep{P}} = \{ \quotep{Q} \in \quotep{\pi} | P \equiv Q \} \and \\ \meaningof{@\quotep{E}} = \{ P \in \pi | P \equiv @x, x \in \meaningof{E} \}}
\end{mathpar}

\begin{eqnarray*}
  \\
  \meaningof{-} : TS \to ST
\end{eqnarray*}

\begin{eqnarray*}
  \\
  L : TS \to ST
\end{eqnarray*}

\begin{eqnarray*}
  \\
  P \models E \iff P \in \meaningof{E}
\end{eqnarray*}

\begin{eqnarray*}
  P \approx_{L} Q \iff \forall E \in L. P \models E \iff Q \models E
\end{eqnarray*}

\begin{eqnarray*}
  P \approx_{K} Q
\end{eqnarray*}

\begin{eqnarray*}
  P \approx Q
\end{eqnarray*}

$\approx_{K} = \approx = \approx_{L}$

\subsubsection{Contextual duality}

Note that contexts extend the quotation operation to a family of
operations from processes to names. Given a context, $M$, we can
define a \emph{nominal context}, $\quotep{M}$ by $\quotep{M}[P] :=
\quotep{M[P]}$. To foreshadow what is to come we observe that these
operations enjoy a duality with processes very much like the duality
between vectors and maps from vectors to scalars.

Further, because the calculus is essentially higher-order, we have a
correspondence between contexts and processes. More specifically,
given a name $x$ and a context $M$ we can construct $M^{*}_{x}$ such
that 

\begin{mathpar}
  M^{*}_{x} | \lift{x}{P} \red M[P]
\end{mathpar}

namely,

\begin{mathpar}
  M^{*}_{x} := x?(u).M[\dropn{u}]
\end{mathpar}

The dependence of $M^{*}_{x}$ on a name makes it an abstraction, 

\begin{mathpar}
  M^{*} := (x)x?(u).M[\dropn{u}]
\end{mathpar}

\subsection{Additional notation}

It will sometimes be convenient to denote the process a name
quotes. We already have the notation $x = \quotep{P}$, but it will be
convenient to introduce an alternate notation, $\procn{x}$, when we
want to emphasize the connection to the use of the name. Note that, by
virtue of name equivalence, $\quotep{\procn{x}} \nameeq x$; so, the
notation is consistent with previous definitions.

Further, because names have structure it is possible to effect
substitutions on the basis of that structure. This means we need to
upgrade our notation for substitutions, which we accomplish by
adapting comprehension notation. Thus,

\begin{mathpar}
  P\{ y / x : x \in S \}
\end{mathpar}

is interpreted to mean the process derived from P by replacing (in a
capture-avoiding manner) each occurrence of $x$ in $S$ by $y$. For example,

\begin{mathpar}
  P\{ \quotep{\procn{x}|\procn{x}} / x : x \in \freenames{P} \}
\end{mathpar}

will replace each (occurrence) of a free name $x$ in $P$ by
$\quotep{\procn{x}|\procn{x}}$.

Also, we will avail ourselves of the notation $x^{L}$ and $x^{R}$ to
denote injections of a name into disjoint copies of the name
space. There are numerous ways to accomplish this. One example can be
found in \cite{MeredithR05}. This notation overloads to vectors of
names: $\vec{x}^{\pi} := (x_{i}^{\pi} \; : \; 0 \leq i < |\vec{x}| )$ where $\pi \in \{L,R\}$.

We also use $P^{\Box} := P|\Box$.

In \cite{MeredithR05} an interpretation of the new operator is
given. It turns out that there are several possible interpretations
all enjoying the requisite algebraic properties of the operator (see
\cite{milner91polyadicpi}). We will therefore make liberal use of
$(\nu\; \vec{x})P$.

% subsection the_syntax_and_semantics_of_the_notation_system (end)   

\input{qm2pi.qmops} 

\input{qm2pi.sterngerlach} 

\input{qm2pi.metric} 

% section concurrent_process_calculi (end)

%\input{qm2pi.proofsketch}

% section proof sketch (end)

%\input{qm2pi.slviaknots} 

% section spatial logic via knots (end)

\input{qm2pi.conclusion}

% section conclusion (end)

%\input{qm2pi.dtcodes} 

% section wiring algorithm (end)

\input{qm2pi.ack} 

% section acknowledgments (end)

\newpage


\bibliographystyle{plain}   
\bibliography{../../biblios/main.bib}

\input{qm2pi.rhodetails}

\end{document}



% section proof sketch (end)

%\section{Unlikely characters: spatial logic for
  knots}\label{sub:characteristic_formulae} % (fold)

Associated to the mobile process calculi are a family of logics known
as the Hennessy-Milner logics. These logics typically enjoy a
semantics interpreting formulae as sets of processes that when
factored through the encoding outlined above allows an identification
of classes of knots with logical formulae. In the context of this
encoding the sub-family known as the spatial logics \cite{CairesC03}
\cite{CairesC04} \cite{Caires04} are of particular interest providing
several important features for expressing and reasoning about
properties (i.e. classes) of knots. We hint here at how this may be done.

%\begin{description}
%\item [structural connectives] 
\subsubsection{Structural connectives} The spatial logics enjoy
structural connectives corresponding, at the logical level, to the
parallel composition ($P | Q$) and new name ($(\nu \; x)P$)
connectives for processes. As illustrated in the examples below, these
connectives are extremely expressive given the shape of our encoding.
%\item [decideable satisfaction]

\subsubsection{Decideable satisfaction}
In \cite{Caires04} the satisfaction relation is shown to be decideable
for a rich class of processes. It further turns out that the image of
the our encoding is a proper subset of that class. This result
provides the basis for an algorithm by which to search for knots
enjoying a given property.
%\item [characteristic formulae]

\subsubsection{Characteristic formulae}
In the same paper \cite{Caires04} , Caires presents a means of calculating
characteristic formulae, selecting equivalence classes of processes
up to a pre--specified depth limit on the support set of names. Composed with our
encoding, this characteristic formula can be used to select
characteristic formulae for knots.
%\end{description}

\subsubsection{Spatial logic formulae}

The grammar below (segmented for comprehension) summarizes the syntax
of spatial logic formulae. We employ illustrative examples in the
sequel to provide an intuitive understanding of their meaning
referring the reader to \cite{Caires04} for a more detailed explication
of the semantics.

\begin{mathpar}
  \inferrule* [lab=boolean] {} {{A,B} \bc T \;|\; \neg A \;|\; A \wedge B \;|\; \eta = \eta'}
  \and
  \inferrule* [lab=spatial] {} {|\; \pzero \;|\; A | B \;|\; x \text{\textregistered} A \;|\; \forall x . A \;|\;  H x . A}
  \and
  \inferrule* [lab=behavioral] {} {|\; \alpha . A}
  \and 
  \inferrule* [lab=recursion] {} {|\; X(\vec{u}) \;|\; \mu X(\vec{u}) . A}
  \and
  \inferrule* [lab=action] {} {\alpha \bc \langle x?(\vec{y}) \rangle \;|\; \langle x!(\vec{y}) \rangle \;|\; \langle \tau \rangle}
  \and 
  \inferrule* [lab=name] {} {\eta \bc x \;|\; \tau}
\end{mathpar} 

% subsection characteristic_formulae (end)   	 

\subsection{Example formulae}\label{sub:example_formulae_} % (fold)

\subsubsection{Crossing as formula.}
% 
% \begin{align*}
%   \frac{d}{dx} \sin x &= \cos x 
%   & \frac{d}{dx} e^x &= e^x \\
%   \frac{d}{dx} \cos x &= - \sin x 
%   & \frac{d}{dx} \log x &= \frac{1}{x} \\
% \end{align*} 

\begin{align*}
 \mu C(x_{0},x_{1},y_{0},y_{1},u).&(\langle x_{0}?(z) \rangle(\langle u! \rangle\langle y_{1}!z \rangle C(x_{0},x_{1},y_{0},y_{1},u)) & \\
  & \wedge \langle y_{1}?(z) \rangle (\langle u! \rangle \langle x_{0}!z \rangle C(x_{0},x_{1},y_{0},y_{1},u)) & \\
  & \wedge \langle x_{1}?(z) \rangle (\langle u? \rangle \langle y_{0}!z \rangle C(x_{0},x_{1},y_{0},y_{1},u)) & \\
  & \wedge \langle y_{0}?(z) \rangle (\langle u? \rangle \langle x_{1}!z \rangle C(x_{0},x_{1},y_{0},y_{1},u))) &
\end{align*}

The lexicographical similarity between the shape of this formulae and
the shape of definition of the process representing a crossing reveals
the intuitive meaning of this formulae. It describes the capabilities
of a process that has the right to represent a crossing. For example
it picks out processes that may perform an input on the port $x_0$ in
its initial menu of capabilities. What differentiates the formula
from the process, however, is that the crossing process is the
smallest candidate to satisfy the formula. Infinitely many other
processes -- with internal behavior hidden behind this interface, so
to speak -- also satisfy this formula. Even this simple formula,
then, can be seen to open a new view onto knots, providing a
computational interpretation of \emph{virtual} knots.

Note that this formula is derived by hand. A similar formula can be
derived by employing Caires' calculation of characteristic formula
\cite{Caires04} to the process representing a crossing. In light of
this discussion, we let
$\meaningof{C}_{\phi}(x0,x1,y0,y1,u)$ denote a formula specifying the
dynamics we wish to capture of a crossing. To guarantee we preserve
the shape of the interface and minimal semantics we demand that
$\meaningof{C}_{\phi}(x0,x1,y0,y1,u) \Rightarrow
\textbf{C}(x0,x1,y0,y1,u)$ where $\textbf{C}(x0,x1,y0,y1,u)$ denotes
the formula above.
                            
\subsubsection{Crossing number constraints.}
The moral content of the context lemma (Lemma \ref{context}) is that the notion of
``locality'' in the Reidemeister moves is effectively captured by the
parallel composition operator of the process calculus. This intuition
extends through the logic. Given a formula,
$\meaningof{C}_{\phi}(x0,x1,y0,y1,u)$, we can use the structural
connectives to specify constraints on crossing numbers, such as at
least $n$ crossings, or exactly $n$ crossings.
\begin{mathpar}
  \inferrule* [lab=at-least-n] {} { K^{\geq n}_{\phi}(\vec{xs},\vec{ys}) := \Pi_{i=0}^{n-1} Hu . \meaningof{C}_{\phi}(xs_i,ys_i,u) | T }
  \and 
  \inferrule* [lab=exactly-n] {} { K^{= n}_{\phi}(\vec{xs},\vec{ys}) := \Pi_{i=0}^{n-1} Hu . \meaningof{C}_{\phi}(xs_i,ys_i,u) | \neg (\forall x_0,y_0,x_1,y_1,u . \meaningof{C}_{\phi}(x_0,y_0,x_1,y_1,u) | T) }
\end{mathpar}

To round out this section, recall that the encoding of an $n$-crossing
knot decomposes into a parallel composition of $n$ \emph{copies} of a
crossing process together with a wiring harness. To specify different
knot classes with the same crossing number amounts to specifying
logical constraints on the wiring harness. In the interest of space,
we defer examples to a forthcoming paper. Suffice it to say that both
the conditions ``alternating knot'' and ``contains the tangle
corresponding to 5/3'' are expressible. For example, it is possible to
calculate the characteristic formula of a process corresponding to the
tangle 5/3 and conjoin it into the classifying formula via the
composition connective of the logic.

Finally, we wish to observe that it is entirely within reason to
contemplate a more domain-specific version of spatial logic tailored
to the shape of processes in the image of the encoding. Such a
domain-specific logic would have a better claim to the title formal
language of knot properties.

% subsection example_formulae_ (end)

% section knots_as_processes (end) 

% section spatial logic via knots (end)

\section{Conclusions and future work}

\paragraph{Testing physical space}
You, gentle reader, may wonder why of all the theorems to be proved
given this set up we pick the one above. In some sense it's hardly
central to quantum mechanics. We see it as central in the sense that
it firmly establishes a notion of physical space arising from a notion
of the equivalence of behavior. Relating bisimulation to a metric is a
big step forward, but one is faced with interpreting the relationship
of that metric space to something more physical. Quantum mechanical
notions of ``physical'' space are still far from intuitive, but by
relating this idea of distance as testing to calculations that predict
physical circumstances we are making a not insignificant step forward
toward an understanding of the physical space we inhabit as
essentially dynamic.

\paragraph{Effectivity and simulation}
One of the observations we have yet to make is that the entire program
spelled out here is effective. We have built various interpreters for
the reflective calculus at work in this interpretation. In principle,
then, we can simulate quantum mechanics on a computer. The place where
the simulation may lose fidelity is the infinitely branching summation
for the annihilator.

In this connection i also want to point out that the evaluation style
calculation of the inner product puts the non-determinism of the
summation right at the heart of measurement. This suggests that
Milner's original reduction-based formulation of the dynamics of his
calculi in terms of sums was not just notationally suggestive of a
notion of measure-and-continue but captured some significant part of
the physics.

\paragraph{Quantum continuations}
In light of this last observation i want to point out that the
predominant account of quantum mechanics is missing a key aspect of a
truly compositional story of the physical situation. In a real lab,
when a measurement is made the observation can be made to feed into
another device that then makes another measurement conditioned on the
results of the first. This means that after the superposition was
collapsed the entire experimental set up remained in
superposition. While QM offers a means of writing this down it doesn't
quite line up well with the well-trodden formulation of computation
and continuation that we see so succinctly expressed in Milner's
calculi. This suggests that there might be advantages to this account
of dynamics waiting to be explored.

\paragraph{Quantum logic}
In this connection, we also note that by virtue of having the
Hennessy-Milner construction, we can pull the construction through the
interpretation of QM. This gives us a natural candidate for a quantum
logic that enjoys an extremely tight connection with it's domain of
interpretation, making the construction much less ad hoc (rather it is
the image of functor!).

\paragraph{Quantum probabiity}
i have questions about the basis of the interpretation of inner
product as probability amplitude. In particular, using which
axiomatization of probability theory does the notion of probability
amplitude earn the right to be so dubbed? In other words, where is the
proof that the operation for calculating a probability amplitude (and
then squaring) satisfies the axioms of what it means to calculate a
probability? Even if such a proof exists (i have yet to find it in the
literature), i wonder if it might not be possible to turn things on
their heads. Can we view the calculation of the probability amplitude
as an axiomatization of probability? If so, then the definition we
give for calculating probability amplitude may provide the basis for
an \emph{effective} theory of probability.

\paragraph{Quantum vs ``biological'' information}
Finally, i want to conclude with a more philosophical observation. At
a recent workshop in which QM was a predominant topic i noticed
something about quantum information. The speaker was giving a riveting
discussion of axiomatic QM and showing how properties of ``no
cloning'' and ``no deleting'' emerged as consequences of the
axiomatization. Theorems of this form are necessary to give us a sense
of confidence that our axioms characterize the physical theory. What
struck me, though, was that if quantum information is neither erasable
nor replicable it is markedly different from \emph{life}. Two of the
things we know about life is that

\begin{itemize}
  \item it ends;
  \item to gain some measure of persistence, to transcend it's
    finitude it is imminently copyable.
\end{itemize}

Both of these qualities are summarized succinctly in the aphorism: all
flesh is grass. For me these two kinds of ``information'' -- call them
quantum and biological -- are end points on a spectrum of strategies
for persistence. At one end, we have those curious entities that enjoy
uniqueness and permanence; at the other, we have those who in the face
of a certain end and an uncertain present make a go of passing
something on. To me one of the more remarkable aspects of the latter
strategy is that in the presence of noise (and certain features of
copying) we get a kind of dynamism, a chance for improvement against a
given persistent condition.

% subsection other_calculi_other_bisimulations_and_geometry_as_behavior (end)




% section conclusion (end)

%\documentclass[12pt]{llncs}
%\documentclass{jktr}

\usepackage[pdftex]{hyperref}                   
\usepackage {listings}
\usepackage {mathpartir}
\usepackage{bcprules}
%\usepackage{listings}
                       
\usepackage{graphicx} 
%\usepackage[margins=2.5cm,nohead,nofoot]{geometry}
%\usepackage{geometry}
\usepackage{amsfonts}
\usepackage{amstext}
\usepackage{latexsym}
\usepackage{amssymb}
\usepackage{color}


%\include{myPreamble}
\include{qm2pi.local} 

%\ifpdf
%\usepackage[pdftex]{graphicx}
%\else
%\usepackage{graphicx}
%\fi

 % \ifpdf
%  \usepackage{pdfsync}
%  \if


%\title{Brief Article}
%\author{David F. Snyder}
%\author{L.G. Meredith}

%\address{Dept. of Math., Texas State University--San Marcos, San Marcos, TX 78666}
       
\pagestyle{empty}


\begin{document}

\lstset{language=[Objective]Caml,frame=shadowbox}

\input{qm2pi.front}

% section front matter (end)

\input{qm2pi.intro} 
 
% section introduction (end)

% \input{qm2pi.knotations} 

% section notation (end)

\input{qm2pi.process.calculi} 

% section concurrent_process_calculi_and_spatial_logics_ (end)
    
%\input{qm2pi.knots2pi} 

%\input{qm2pi.trefoil} 

%\input{qm2pi.mainthm} 

% subsection basic_interpretation (end)

%\input{qm2pi.rho.presentation} 
\subsection{The syntax and semantics of the notation system}\label{sub:the_syntax_and_semantics_of_the_notation_system} % (fold)

We now summarize a technical presentation of the calculus that
embodies our theory of dynamics. The typical presentation of such a
calculus follows the style of giving generators and relations on
them. The grammar, below, describing term constructors, freely
generates the set of processes, $\Proc$. This set is then quotiented
by a relation known as structural congruence and it is over this set
that the notion of dynamics is expressed. This presentation is
essentially that of \cite{MeredithR05} with the addition of
polyadicity and summation. For readability we have relegated some of
the technical subtleties to an appendix.

\subsubsection{Process grammar}\label{subsub:process_grammar}

\begin{mathpar}
  \inferrule* [lab=synchronization] {} {{M} \bc \pzero \;|\; x?F \;|\; x!C }
  \and
  \inferrule* [lab=abstraction] {} {{F} \bc (x)P}
  \and
  \inferrule* [lab=concretion] {} {{C} \bc \langle Q \rangle}
  \and
  \inferrule* [lab=process] {} {{P,Q} \bc M \;| \;P|Q \;|\; @{x}}
  \and
  \inferrule* [lab=name] {} {{x} \bc \quotep{P}}
\end{mathpar} 

Note that $\vec{x}$ (resp. $\vec{P}$) denotes a vector of names
(resp. processes) of length $|\vec{x}|$ (resp. $|\vec{P}|$). We adopt
the following useful abbreviations.

\begin{mathpar}
   x?(\vec{y}).P := x.(\vec{y})P \and  x\clift{\vec{P}} := x.\clift{\vec{P}}
   \and x!(y) := \lift{x}{\dropn{y}}
   \and \Pi_{i=0}^{n-1}P_i := P_0 | \ldots | P_{n-1}
\end{mathpar}

\subsubsection{Structural congruence}

\paragraph{Free and bound names and alpha-equivalence.} At the
core of structural equivalence is alpha-equivalence which identifies
process that are the same up to a change of variable. Formally, we
recognize the distinction between free and bound names. The free names
of a process, $\freenames{P}$, may be calculated recursively as
follows:

\begin{mathpar}
\freenames{\pzero} := \emptyset
  \and \\
  \freenames{x?(y).P} := \{ x \} \cup (\freenames{P} \setminus \{ y \})
  \and 
  \freenames{x!\langle P \rangle} := \{ x \} \cup \{ P \} 
  \and \\
  \freenames{P|Q} := \freenames{P} \cup \freenames{Q}
  \and \\
  \freenames{@{x}} := \{ x \}
\end{mathpar}

$\pi$
$\quotep{\pi}$

$\freenames{-} : \pi \to \mathcal{P}(\quotep{\pi})$

\begin{eqnarray*}
  \freenames{\pzero} & := & \emptyset \\
  \freenames{x?(y).P} & := & \{ x \} \cup (\freenames{P} \setminus \{ y \}) \\
  \freenames{x!\langle P \rangle} & := & \{ x \} \cup \{ P \} \\
  \freenames{P|Q} & := & \freenames{P} \cup \freenames{Q} \\
  \freenames{\dropn{x}} & := & \{ x \}
\end{eqnarray*}

The bound names of a process, $\boundnames{P}$, are those names occurring in $P$
that are not free. For example, in $x?(y).0$, the name $x$ is free, while $y$ is bound.

\begin{mathpar}
  \inferrule* [lab=monoidal-laws] {} { P|Q \equiv Q|P \and P|0 \equiv P \and P|(Q|R) \equiv (P|Q)|R }
\end{mathpar}

\begin{mathpar}
  \inferrule* [lab=alpha-equivalence] {} { (x)P \equiv (y)P\{y/x\} \and y \not\in \freenames{P} }
\end{mathpar}

\begin{definition}
Then two processes, $P,Q$, are alpha-equivalent if $P = Q\{\vec{y}/\vec{x}\}$ for
some $\vec{x} \in \boundnames{Q},\vec{y} \in \boundnames{P}$, where $Q\{\vec{y}/\vec{x}\}$
denotes the capture-avoiding substitution of $\vec{y}$ for $\vec{x}$ in $Q$.
\end{definition}

\begin{definition}
  The {\em structural congruence} \cite{SangiorgiWalker} , $\equiv$,
  between processes is the least congruence containing
  alpha-equivalence, satisfying the abelian monoid laws
  (associativity, commutativity and $\pzero$ as identity) for parallel
  composition $|$ and for summation $+$.
\end{definition}

\subsection{Name equivalence}

We take name equivalence, written $\nameeq$, to be the smallest
equivalence relation generated by the following rules.

\begin{mathpar}
\inferrule*[lab=Quote-drop]
{ }
{ \quotep{@{x}} \nameeq x }

\inferrule*[lab=Struct-equiv]
{ P \scong Q }
{ \quotep{P} \nameeq \quotep{Q} }
\end{mathpar}

The astute reader will have noticed that the mutual recursion of names
and processes imposes a mutual recursion on alpha-equivalence and
structural equivalence via name-equivalence. Fortunately, all of this
works out pleasantly and we may calculate in the natural way, free of
concern. The reader interested in the details is referred to the
appendix \ref{appendix:rho_details}.

\subsection{Substitution}

We use $\Proc$ for the set of processes, $\QProc$ for the set of
names, and $\id{\{}\vec{y} / \vec{x} \id{\}}$ to denote partial maps,
$s : \QProc \rightarrow \QProc$. A map, $s$ lifts, uniquely, to a map
on process terms, $\widehat{s} : \Proc \rightarrow \Proc$ by the
following equations.

\begin{mathpar}
  (0) \psubstp{Q}{P} := 0 \\
  (R \juxtap S) \psubstp{Q}{P}
  :=    
  (R)\psubstp{Q}{P} \juxtap (S) \psubstp{Q}{P} \\
  (x?(y).R) \psubstp{Q}{P}    
  :=    
  (x)\substp{Q}{P} (z)\concat( (R \psubstn{z}{y}) \psubstp{Q}{P} ) \\
  (\lift{x}{R}) \psubstp{Q}{P}  
  :=
  \lift{(x)\substp{Q}{P}}{ R \psubstp{Q}{P} } \\
%   (\dropn{x})  \psubstp{Q}{P}       
%   := 
%   \left\{ 
%     \begin{array}{ccc} 
%       \dropn{\quotep{Q}} & & x \nameeq \quotep{P} \\
%       \dropn{x} & & otherwise \\
%     \end{array}
%   \right. 
  (\dropn{x})  \psubstp{Q}{P}       
  := 
  \left\{ 
    \begin{array}{ccc} 
      Q & & x \nameeq \quotep{P} \\
      \dropn{x} & & otherwise \\
    \end{array}
  \right.
\end{mathpar}
 

where

\begin{eqnarray}
  (x)\id{\{} \lpquote Q \rpquote / \lpquote P \rpquote \id{\}}            = 
  \left\{ 
    \begin{array}{ccc}
      \lpquote Q \rpquote & & x \nameeq \lpquote P \rpquote \\
      x & & otherwise \\
    \end{array}
  \right. \nonumber
\end{eqnarray}

and $z$ is chosen distinct from $\quotep{P}$, $\quotep{Q}$, the free
names in $Q$, and all the names in $R$. Our $\alpha$-equivalence will
be built in the standard way from this substitution.

\begin{remark}\label{rem:no_self_referential_names}
  One consequence of these definitions is that $\forall P. \quotep{P}
  \not\in \freenames{P}$.
\end{remark}

\subsection{ Dynamic quote: an example }

Anticipating something of what's to come, consider applying the
substitution, $\widehat{\id{\{}u / z \id{\}}}$, to the following pair
of processes, $\lift{w}{y!(z)}$ and $w[ \lpquote y!(z) \rpquote ]$.

\begin{eqnarray}
	\lift{w}{y!(z)}\widehat{\id{\{}u / z \id{\}}}
		& = &
		\lift{w}{y!(u)} \nonumber\\
	w[ \lpquote y!(z) \rpquote ] \widehat{ \id{\{}u / z \id{\}} }
		& = &
		w[ \lpquote y!(z) \rpquote ] \nonumber
\end{eqnarray}

Because the body of the process between quotes is impervious to
substitution, we get radically different answers. In fact, by
examining the first process in an input context,
e.g. $x?(z).\lift{w}{y!(z)}$, we see that the process under the lift
operator may be shaped by prefixed inputs binding a name inside it. In
this sense, the lift operator will be seen as a way to dynamically
construct processes before reifying them as names.

Finally equipped with these standard features we can present the
dynamics of the calculus.

\subsubsection{Operational semantics} 

Finally, we introduce the computational dynamics. What marks these
algebras as distinct from other more traditionally studied algebraic
structures, e.g. vector spaces or polynomial rings, is the manner in
which dynamics is captured. In traditional structures, dynamics is typically
expressed through morphisms between such structures, as in linear maps
between vector spaces or morphisms between rings. In algebras
associated with the semantics of computation, the dynamics is
expressed as part of the algebraic structure itself, through a
reduction reduction relation typically denoted by $\red$. Below, we
give a recursive presentation of this relation for the calculus used
in the encoding.

$\red \subseteq \pi \times \pi$
$\red : \pi \to \mathcal{P}(\pi)$

\begin{mathpar}
  \inferrule* [lab=Comm] { \textsf{match}( x_{src}, x_{trgt} ) } { x_{trgt}?(y)P \; | \; x_{src}!\langle {Q} \rangle \red P\{\quotep{Q}/y}\} }
  \and \\
  \inferrule* [lab=Par] {{P} \red {P}'} {{{P} | {Q}} \red {{P}' | {Q}}}
  \and
  \inferrule* [lab=Equiv]{{{P} \scong {P}'} \andalso {{P}' \red {Q}'} \andalso {{Q}' \scong {Q}}}{{P} \red {Q}}
\end{mathpar}

\begin{eqnarray*}
  match_{\equiv} (\quotep{P},\quotep{Q}) & := & P \equiv Q \\
  match_{\dagger}(\quotep{P},\quotep{Q}) & := & \forall R. P|Q \red^{*} R => R \red^{*} 0 \\
  match_{K}(\quotep{P},\quotep{Q}) & := & K \mbox{ for some context } K
\end{eqnarray*}

$u?(x)P | u!\langle Q \rangle \red P\{\quotep{Q}/x\}$

%We write $\wred$ for $\red^*$, and $P\red$ if $\exists Q $ such that $ P \red Q$.
We write $P\red$ if $\exists Q $ such that $ P \red Q$ and $P\not\red$, otherwise.

\section{Replication}

As mentioned before, it is known that replication (and hence
recursion) can be implemented in a higher-order process algebra
\cite{SangiorgiWalker}. As our first example of calculation with the
machinery thus far presented we give the construction explicitly in
the {\rhoc}.

\begin{eqnarray}
	D_{x} & := & \prefix{x}{y}{(\binpar{\outputp{x}{y}}{@{y}})} \nonumber\\
	\bangp_{x}{P} & := & \binpar{{x}!\langle{\binpar{D_{x}}{P}}\rangle}{D_{x}} \nonumber
\end{eqnarray}

\begin{eqnarray}
	\bangp_{x}{P} & & \nonumber\\
	=
	& {x}!\langle{(\prefix{x}{y}{(\outputp{x}{y} | @{y})) | P}}\rangle 
	      | \prefix{x}{y}{(\outputp{x}{y} | @{y})} & \nonumber\\
	\red
	& (\outputp{x}{y} | @{y})\substn{\quotep{(\prefix{x}{y}{(@{y} | \outputp{x}{y})) | P}}}{y} & \nonumber\\
	=
	& \outputp{x}{\quotep{(\prefix{x}{y}{(\outputp{x}{y} | @{y})) | P}}}
	  | {(\prefix{x}{y}{(\outputp{x}{y} | @{y})) | P}} & \nonumber\\
	\red
	& \ldots & \nonumber\\
	\red^*
	& P | P | \ldots & \nonumber
\end{eqnarray}

Of course, this encoding, as an implementation, runs away, unfolding
$\bangp{P}$ eagerly. A lazier and more implementable replication
operator, restricted to input-guarded processes, may be obtained as follows.

\begin{eqnarray}
\bangp{\prefix{u}{v}{P}} 
	:= 
	\binpar{\lift{x}{\prefix{u}{v}{(\binpar{D(x)}{P})}}}{D(x)} \nonumber
\end{eqnarray}

\begin{remark}
  Note that the lazier definition still does not deal with summation
  or mixed summation (i.e. sums over input and output). The reader is
  invited to construct definitions of replication that deal with these
  features. 

  Further, the definitions are parameterized in a name, $x$. Can you,
  gentle reader, make a definition that eliminates this parameter and
  guarantees no accidental interaction between the replication
  machinery and the process being replicated -- i.e. no accidental
  sharing of names used by the process to get its work done and the
  name(s) used by the replication to effect copying. This latter
  revision of the definition of replication is crucial to obtaining
  the expected identity $!!P \sim !P$.
\end{remark}

\begin{remark}\label{rem:paradoxical_combinator}
  The reader familiar with the lambda calculus will have noticed the
  similarity between $D$ and the paradoxical combinator.

  [Ed. note: the existence of this seems to suggest we have to be more
  restrictive on the set of processes and names we admit if we are to
  support no-cloning.]
\end{remark}

\subsubsection{Bisimulation}

The computational dynamics gives rise to another kind of equivalence,
the equivalence of computational behavior. As previously mentioned
this is typically captured \emph{via} some form of bisimulation.

% The notion we use in this paper is weak barbed bisimulation
% \cite{milner91polyadicpi}.

The notion we use in this paper is derived from weak barbed
bisimulation \cite{milner91polyadicpi}. 

\begin{definition}
An \emph{observation relation}, $\downarrow_{\mathcal N}$, over a set
of names, $\mathcal N$, is the smallest relation satisfying the rules
below.

\infrule[Out-barb]{y \in {\mathcal N}, \; x \nameeq y}
		  {\outputp{x}{v} \downarrow_{\mathcal N} x}
\infrule[Par-barb]{\mbox{$P\downarrow_{\mathcal N} x$ or $Q\downarrow_{\mathcal N} x$}}
		  {\binpar{P}{Q} \downarrow_{\mathcal N} x}

We write $P \Downarrow_{\mathcal N} x$ if there is $Q$ such that 
$P \wred Q$ and $Q \downarrow_{\mathcal N} x$.
\end{definition}

\begin{definition}
%\label{def.bbisim}
An  ${\mathcal N}$-\emph{barbed bisimulation} over a set of names, ${\mathcal N}$, is a symmetric binary relation 
${\mathcal S}_{\mathcal N}$ between agents such that $P\rel{S}_{\mathcal N}Q$ implies:
\begin{enumerate}
\item If $P \red P'$ then $Q \wred Q'$ and $P'\rel{S}_{\mathcal N} Q'$.
\item If $P\downarrow_{\mathcal N} x$, then $Q\Downarrow_{\mathcal N} x$.
\end{enumerate}
$P$ is ${\mathcal N}$-barbed bisimilar to $Q$, written
$P \wbbisim_{\mathcal N} Q$, if $P \rel{S}_{\mathcal N} Q$ for some ${\mathcal N}$-barbed bisimulation ${\mathcal S}_{\mathcal N}$.
\end{definition}

$\mathcal{R} \subseteq \pi \times \pi$

$P \mathcal{R} Q => \forall P'. P \red P' \Rightarrow \exists Q'. Q \red Q', P' \mathcal{R} Q'$

$P \vdash x \Rightarrow Q \vdash x$

\begin{mathpar}
  \inferrule*[lab=Out-barb]{x \nameeq y}{{y}!\langle{Q}\rangle \vdash x}
  \and
  \inferrule*[lab=Par-barb]{\mbox{$P\vdash x$ or $Q\vdash x$}}{\binpar{P}{Q} \vdash x}
\end{mathpar}

\subsubsection{Contexts}

One of the principle advantages of computational calculi like the
$\pi$-calculus is a well-defined notion of context,
contextual-equivalence and a correlation between
contextual-equivalence and notions of bisimulation. The notion of
context allows the decomposition of a process into (sub-)process and
its syntactic environment, its context. Thus, a context may be
thought of as a process with a ``hole'' (written $\Box$) in it. The
application of a context $M$ to a process $P$, written $M[P]$, is
tantamount to filling the hole in $M$ with $P$. In this paper we do
not need the full weight of this theory, but do make use of the notion
of context in the proof the main theorem. 

\begin{mathpar}
  \inferrule* [lab=summation] {} {{M_{M},M_{N}} \bc \Box \;|\; x.M_{A} \;|\; M_{M}+M_{N}}
  \and
  \inferrule* [lab=agent] {} {{M_{A}} \bc (\vec{x})M_{P} \;| \; \clift{P_0,\ldots,M_{P},\ldots,P_N}}
  \and \\
  \inferrule* [lab=process] {} {{M_{P}} \bc M_{N} \;| \;P|M_{P} }
\end{mathpar} 

\begin{mathpar}
  \inferrule* [lab=sychronization] {} {M_{N} \bc \Box \;|\; x?M_{F} \;|\; x!M_{C}}
  \and
  \inferrule* [lab=abstraction] {} {{M_{F}} \bc (x)M_{P} }
  \and
  \inferrule* [lab=concretion] {} {{M_{C}} \bc \langle M_{P} \rangle }
  \and \\
  \inferrule* [lab=process] {} {{M_{P}} \bc M_{N} \;| \;P|M_{P} }
\end{mathpar}

\begin{definition}[contextual application] Given a context $M$, and
  process $P$, we define the \emph{contextual application}, $M[P] :=
  M\{P/\Box\}$. That is, the contextual application of M to P is the
  substitution of $P$ for $\Box$ in $M$.
\end{definition}

$\meaningof{-} : L \to \mathcal{P}(\pi)$

\begin{mathpar}
  \inferrule* [lab=collection] {} {\meaningof{true} = \pi, \and \meaningof{~E} = \pi \setminus \meaningof{E}, \and \meaningof{E_{1} \& E_{2}} = \meaningof{E_{1}} \cap \meaningof{E_{2}}}
\end{mathpar}

\begin{mathpar}
  \inferrule* [lab=structure] {} {\meaningof{0} = \{ P \in \pi | P \equiv 0 \}, \and \\ \meaningof{E_1 | E_2} = \{ P \in \pi | P \equiv P_{1} | P_{2}, P_{1} \in \meaningof{E_{1}}, P_{2} \in \meaningof{E_2}\} }
\end{mathpar}

\begin{mathpar}
 \inferrule* [lab=behavior] {} {\meaningof{\langle a?b \rangle E} = \{ P \in \pi | P \equiv Q | u?(y)P', \\ \and \\\\ \and \\ \;\;\; u \in \meaningof{a}, \forall z.P'\{z/y\} \in \meaningof{E\{z/b\}}\}, \and \\ \meaningof{a!E} = \{ P \in \pi | P \equiv Q | x!\langle P' \rangle, x \in \meaningof{a} P' \in \meaningof{E}\} }
\end{mathpar}

\begin{mathpar}
 \inferrule* [lab=nominal] {} {\meaningof{\quotep{E}} = \{ \quotep{P} \in \quotep{\pi} | P \in \meaningof{E} \}, \and \meaningof{\quotep{P}} = \{ \quotep{Q} \in \quotep{\pi} | P \equiv Q \} \and \\ \meaningof{@\quotep{E}} = \{ P \in \pi | P \equiv @x, x \in \meaningof{E} \}}
\end{mathpar}

\begin{eqnarray*}
  \\
  \meaningof{-} : TS \to ST
\end{eqnarray*}

\begin{eqnarray*}
  \\
  L : TS \to ST
\end{eqnarray*}

\begin{eqnarray*}
  \\
  P \models E \iff P \in \meaningof{E}
\end{eqnarray*}

\begin{eqnarray*}
  P \approx_{L} Q \iff \forall E \in L. P \models E \iff Q \models E
\end{eqnarray*}

\begin{eqnarray*}
  P \approx_{K} Q
\end{eqnarray*}

\begin{eqnarray*}
  P \approx Q
\end{eqnarray*}

$\approx_{K} = \approx = \approx_{L}$

\subsubsection{Contextual duality}

Note that contexts extend the quotation operation to a family of
operations from processes to names. Given a context, $M$, we can
define a \emph{nominal context}, $\quotep{M}$ by $\quotep{M}[P] :=
\quotep{M[P]}$. To foreshadow what is to come we observe that these
operations enjoy a duality with processes very much like the duality
between vectors and maps from vectors to scalars.

Further, because the calculus is essentially higher-order, we have a
correspondence between contexts and processes. More specifically,
given a name $x$ and a context $M$ we can construct $M^{*}_{x}$ such
that 

\begin{mathpar}
  M^{*}_{x} | \lift{x}{P} \red M[P]
\end{mathpar}

namely,

\begin{mathpar}
  M^{*}_{x} := x?(u).M[\dropn{u}]
\end{mathpar}

The dependence of $M^{*}_{x}$ on a name makes it an abstraction, 

\begin{mathpar}
  M^{*} := (x)x?(u).M[\dropn{u}]
\end{mathpar}

\subsection{Additional notation}

It will sometimes be convenient to denote the process a name
quotes. We already have the notation $x = \quotep{P}$, but it will be
convenient to introduce an alternate notation, $\procn{x}$, when we
want to emphasize the connection to the use of the name. Note that, by
virtue of name equivalence, $\quotep{\procn{x}} \nameeq x$; so, the
notation is consistent with previous definitions.

Further, because names have structure it is possible to effect
substitutions on the basis of that structure. This means we need to
upgrade our notation for substitutions, which we accomplish by
adapting comprehension notation. Thus,

\begin{mathpar}
  P\{ y / x : x \in S \}
\end{mathpar}

is interpreted to mean the process derived from P by replacing (in a
capture-avoiding manner) each occurrence of $x$ in $S$ by $y$. For example,

\begin{mathpar}
  P\{ \quotep{\procn{x}|\procn{x}} / x : x \in \freenames{P} \}
\end{mathpar}

will replace each (occurrence) of a free name $x$ in $P$ by
$\quotep{\procn{x}|\procn{x}}$.

Also, we will avail ourselves of the notation $x^{L}$ and $x^{R}$ to
denote injections of a name into disjoint copies of the name
space. There are numerous ways to accomplish this. One example can be
found in \cite{MeredithR05}. This notation overloads to vectors of
names: $\vec{x}^{\pi} := (x_{i}^{\pi} \; : \; 0 \leq i < |\vec{x}| )$ where $\pi \in \{L,R\}$.

We also use $P^{\Box} := P|\Box$.

In \cite{MeredithR05} an interpretation of the new operator is
given. It turns out that there are several possible interpretations
all enjoying the requisite algebraic properties of the operator (see
\cite{milner91polyadicpi}). We will therefore make liberal use of
$(\nu\; \vec{x})P$.

% subsection the_syntax_and_semantics_of_the_notation_system (end)   

\input{qm2pi.qmops} 

\input{qm2pi.sterngerlach} 

\input{qm2pi.metric} 

% section concurrent_process_calculi (end)

%\input{qm2pi.proofsketch}

% section proof sketch (end)

%\input{qm2pi.slviaknots} 

% section spatial logic via knots (end)

\input{qm2pi.conclusion}

% section conclusion (end)

%\input{qm2pi.dtcodes} 

% section wiring algorithm (end)

\input{qm2pi.ack} 

% section acknowledgments (end)

\newpage


\bibliographystyle{plain}   
\bibliography{../../biblios/main.bib}

\input{qm2pi.rhodetails}

\end{document}

 

% section wiring algorithm (end)

\documentclass[12pt]{llncs}
%\documentclass{jktr}

\usepackage[pdftex]{hyperref}                   
\usepackage {listings}
\usepackage {mathpartir}
\usepackage{bcprules}
%\usepackage{listings}
                       
\usepackage{graphicx} 
%\usepackage[margins=2.5cm,nohead,nofoot]{geometry}
%\usepackage{geometry}
\usepackage{amsfonts}
\usepackage{amstext}
\usepackage{latexsym}
\usepackage{amssymb}
\usepackage{color}


%\include{myPreamble}
\include{qm2pi.local} 

%\ifpdf
%\usepackage[pdftex]{graphicx}
%\else
%\usepackage{graphicx}
%\fi

 % \ifpdf
%  \usepackage{pdfsync}
%  \if


%\title{Brief Article}
%\author{David F. Snyder}
%\author{L.G. Meredith}

%\address{Dept. of Math., Texas State University--San Marcos, San Marcos, TX 78666}
       
\pagestyle{empty}


\begin{document}

\lstset{language=[Objective]Caml,frame=shadowbox}

\input{qm2pi.front}

% section front matter (end)

\input{qm2pi.intro} 
 
% section introduction (end)

% \input{qm2pi.knotations} 

% section notation (end)

\input{qm2pi.process.calculi} 

% section concurrent_process_calculi_and_spatial_logics_ (end)
    
%\input{qm2pi.knots2pi} 

%\input{qm2pi.trefoil} 

%\input{qm2pi.mainthm} 

% subsection basic_interpretation (end)

%\input{qm2pi.rho.presentation} 
\subsection{The syntax and semantics of the notation system}\label{sub:the_syntax_and_semantics_of_the_notation_system} % (fold)

We now summarize a technical presentation of the calculus that
embodies our theory of dynamics. The typical presentation of such a
calculus follows the style of giving generators and relations on
them. The grammar, below, describing term constructors, freely
generates the set of processes, $\Proc$. This set is then quotiented
by a relation known as structural congruence and it is over this set
that the notion of dynamics is expressed. This presentation is
essentially that of \cite{MeredithR05} with the addition of
polyadicity and summation. For readability we have relegated some of
the technical subtleties to an appendix.

\subsubsection{Process grammar}\label{subsub:process_grammar}

\begin{mathpar}
  \inferrule* [lab=synchronization] {} {{M} \bc \pzero \;|\; x?F \;|\; x!C }
  \and
  \inferrule* [lab=abstraction] {} {{F} \bc (x)P}
  \and
  \inferrule* [lab=concretion] {} {{C} \bc \langle Q \rangle}
  \and
  \inferrule* [lab=process] {} {{P,Q} \bc M \;| \;P|Q \;|\; @{x}}
  \and
  \inferrule* [lab=name] {} {{x} \bc \quotep{P}}
\end{mathpar} 

Note that $\vec{x}$ (resp. $\vec{P}$) denotes a vector of names
(resp. processes) of length $|\vec{x}|$ (resp. $|\vec{P}|$). We adopt
the following useful abbreviations.

\begin{mathpar}
   x?(\vec{y}).P := x.(\vec{y})P \and  x\clift{\vec{P}} := x.\clift{\vec{P}}
   \and x!(y) := \lift{x}{\dropn{y}}
   \and \Pi_{i=0}^{n-1}P_i := P_0 | \ldots | P_{n-1}
\end{mathpar}

\subsubsection{Structural congruence}

\paragraph{Free and bound names and alpha-equivalence.} At the
core of structural equivalence is alpha-equivalence which identifies
process that are the same up to a change of variable. Formally, we
recognize the distinction between free and bound names. The free names
of a process, $\freenames{P}$, may be calculated recursively as
follows:

\begin{mathpar}
\freenames{\pzero} := \emptyset
  \and \\
  \freenames{x?(y).P} := \{ x \} \cup (\freenames{P} \setminus \{ y \})
  \and 
  \freenames{x!\langle P \rangle} := \{ x \} \cup \{ P \} 
  \and \\
  \freenames{P|Q} := \freenames{P} \cup \freenames{Q}
  \and \\
  \freenames{@{x}} := \{ x \}
\end{mathpar}

$\pi$
$\quotep{\pi}$

$\freenames{-} : \pi \to \mathcal{P}(\quotep{\pi})$

\begin{eqnarray*}
  \freenames{\pzero} & := & \emptyset \\
  \freenames{x?(y).P} & := & \{ x \} \cup (\freenames{P} \setminus \{ y \}) \\
  \freenames{x!\langle P \rangle} & := & \{ x \} \cup \{ P \} \\
  \freenames{P|Q} & := & \freenames{P} \cup \freenames{Q} \\
  \freenames{\dropn{x}} & := & \{ x \}
\end{eqnarray*}

The bound names of a process, $\boundnames{P}$, are those names occurring in $P$
that are not free. For example, in $x?(y).0$, the name $x$ is free, while $y$ is bound.

\begin{mathpar}
  \inferrule* [lab=monoidal-laws] {} { P|Q \equiv Q|P \and P|0 \equiv P \and P|(Q|R) \equiv (P|Q)|R }
\end{mathpar}

\begin{mathpar}
  \inferrule* [lab=alpha-equivalence] {} { (x)P \equiv (y)P\{y/x\} \and y \not\in \freenames{P} }
\end{mathpar}

\begin{definition}
Then two processes, $P,Q$, are alpha-equivalent if $P = Q\{\vec{y}/\vec{x}\}$ for
some $\vec{x} \in \boundnames{Q},\vec{y} \in \boundnames{P}$, where $Q\{\vec{y}/\vec{x}\}$
denotes the capture-avoiding substitution of $\vec{y}$ for $\vec{x}$ in $Q$.
\end{definition}

\begin{definition}
  The {\em structural congruence} \cite{SangiorgiWalker} , $\equiv$,
  between processes is the least congruence containing
  alpha-equivalence, satisfying the abelian monoid laws
  (associativity, commutativity and $\pzero$ as identity) for parallel
  composition $|$ and for summation $+$.
\end{definition}

\subsection{Name equivalence}

We take name equivalence, written $\nameeq$, to be the smallest
equivalence relation generated by the following rules.

\begin{mathpar}
\inferrule*[lab=Quote-drop]
{ }
{ \quotep{@{x}} \nameeq x }

\inferrule*[lab=Struct-equiv]
{ P \scong Q }
{ \quotep{P} \nameeq \quotep{Q} }
\end{mathpar}

The astute reader will have noticed that the mutual recursion of names
and processes imposes a mutual recursion on alpha-equivalence and
structural equivalence via name-equivalence. Fortunately, all of this
works out pleasantly and we may calculate in the natural way, free of
concern. The reader interested in the details is referred to the
appendix \ref{appendix:rho_details}.

\subsection{Substitution}

We use $\Proc$ for the set of processes, $\QProc$ for the set of
names, and $\id{\{}\vec{y} / \vec{x} \id{\}}$ to denote partial maps,
$s : \QProc \rightarrow \QProc$. A map, $s$ lifts, uniquely, to a map
on process terms, $\widehat{s} : \Proc \rightarrow \Proc$ by the
following equations.

\begin{mathpar}
  (0) \psubstp{Q}{P} := 0 \\
  (R \juxtap S) \psubstp{Q}{P}
  :=    
  (R)\psubstp{Q}{P} \juxtap (S) \psubstp{Q}{P} \\
  (x?(y).R) \psubstp{Q}{P}    
  :=    
  (x)\substp{Q}{P} (z)\concat( (R \psubstn{z}{y}) \psubstp{Q}{P} ) \\
  (\lift{x}{R}) \psubstp{Q}{P}  
  :=
  \lift{(x)\substp{Q}{P}}{ R \psubstp{Q}{P} } \\
%   (\dropn{x})  \psubstp{Q}{P}       
%   := 
%   \left\{ 
%     \begin{array}{ccc} 
%       \dropn{\quotep{Q}} & & x \nameeq \quotep{P} \\
%       \dropn{x} & & otherwise \\
%     \end{array}
%   \right. 
  (\dropn{x})  \psubstp{Q}{P}       
  := 
  \left\{ 
    \begin{array}{ccc} 
      Q & & x \nameeq \quotep{P} \\
      \dropn{x} & & otherwise \\
    \end{array}
  \right.
\end{mathpar}
 

where

\begin{eqnarray}
  (x)\id{\{} \lpquote Q \rpquote / \lpquote P \rpquote \id{\}}            = 
  \left\{ 
    \begin{array}{ccc}
      \lpquote Q \rpquote & & x \nameeq \lpquote P \rpquote \\
      x & & otherwise \\
    \end{array}
  \right. \nonumber
\end{eqnarray}

and $z$ is chosen distinct from $\quotep{P}$, $\quotep{Q}$, the free
names in $Q$, and all the names in $R$. Our $\alpha$-equivalence will
be built in the standard way from this substitution.

\begin{remark}\label{rem:no_self_referential_names}
  One consequence of these definitions is that $\forall P. \quotep{P}
  \not\in \freenames{P}$.
\end{remark}

\subsection{ Dynamic quote: an example }

Anticipating something of what's to come, consider applying the
substitution, $\widehat{\id{\{}u / z \id{\}}}$, to the following pair
of processes, $\lift{w}{y!(z)}$ and $w[ \lpquote y!(z) \rpquote ]$.

\begin{eqnarray}
	\lift{w}{y!(z)}\widehat{\id{\{}u / z \id{\}}}
		& = &
		\lift{w}{y!(u)} \nonumber\\
	w[ \lpquote y!(z) \rpquote ] \widehat{ \id{\{}u / z \id{\}} }
		& = &
		w[ \lpquote y!(z) \rpquote ] \nonumber
\end{eqnarray}

Because the body of the process between quotes is impervious to
substitution, we get radically different answers. In fact, by
examining the first process in an input context,
e.g. $x?(z).\lift{w}{y!(z)}$, we see that the process under the lift
operator may be shaped by prefixed inputs binding a name inside it. In
this sense, the lift operator will be seen as a way to dynamically
construct processes before reifying them as names.

Finally equipped with these standard features we can present the
dynamics of the calculus.

\subsubsection{Operational semantics} 

Finally, we introduce the computational dynamics. What marks these
algebras as distinct from other more traditionally studied algebraic
structures, e.g. vector spaces or polynomial rings, is the manner in
which dynamics is captured. In traditional structures, dynamics is typically
expressed through morphisms between such structures, as in linear maps
between vector spaces or morphisms between rings. In algebras
associated with the semantics of computation, the dynamics is
expressed as part of the algebraic structure itself, through a
reduction reduction relation typically denoted by $\red$. Below, we
give a recursive presentation of this relation for the calculus used
in the encoding.

$\red \subseteq \pi \times \pi$
$\red : \pi \to \mathcal{P}(\pi)$

\begin{mathpar}
  \inferrule* [lab=Comm] { \textsf{match}( x_{src}, x_{trgt} ) } { x_{trgt}?(y)P \; | \; x_{src}!\langle {Q} \rangle \red P\{\quotep{Q}/y}\} }
  \and \\
  \inferrule* [lab=Par] {{P} \red {P}'} {{{P} | {Q}} \red {{P}' | {Q}}}
  \and
  \inferrule* [lab=Equiv]{{{P} \scong {P}'} \andalso {{P}' \red {Q}'} \andalso {{Q}' \scong {Q}}}{{P} \red {Q}}
\end{mathpar}

\begin{eqnarray*}
  match_{\equiv} (\quotep{P},\quotep{Q}) & := & P \equiv Q \\
  match_{\dagger}(\quotep{P},\quotep{Q}) & := & \forall R. P|Q \red^{*} R => R \red^{*} 0 \\
  match_{K}(\quotep{P},\quotep{Q}) & := & K \mbox{ for some context } K
\end{eqnarray*}

$u?(x)P | u!\langle Q \rangle \red P\{\quotep{Q}/x\}$

%We write $\wred$ for $\red^*$, and $P\red$ if $\exists Q $ such that $ P \red Q$.
We write $P\red$ if $\exists Q $ such that $ P \red Q$ and $P\not\red$, otherwise.

\section{Replication}

As mentioned before, it is known that replication (and hence
recursion) can be implemented in a higher-order process algebra
\cite{SangiorgiWalker}. As our first example of calculation with the
machinery thus far presented we give the construction explicitly in
the {\rhoc}.

\begin{eqnarray}
	D_{x} & := & \prefix{x}{y}{(\binpar{\outputp{x}{y}}{@{y}})} \nonumber\\
	\bangp_{x}{P} & := & \binpar{{x}!\langle{\binpar{D_{x}}{P}}\rangle}{D_{x}} \nonumber
\end{eqnarray}

\begin{eqnarray}
	\bangp_{x}{P} & & \nonumber\\
	=
	& {x}!\langle{(\prefix{x}{y}{(\outputp{x}{y} | @{y})) | P}}\rangle 
	      | \prefix{x}{y}{(\outputp{x}{y} | @{y})} & \nonumber\\
	\red
	& (\outputp{x}{y} | @{y})\substn{\quotep{(\prefix{x}{y}{(@{y} | \outputp{x}{y})) | P}}}{y} & \nonumber\\
	=
	& \outputp{x}{\quotep{(\prefix{x}{y}{(\outputp{x}{y} | @{y})) | P}}}
	  | {(\prefix{x}{y}{(\outputp{x}{y} | @{y})) | P}} & \nonumber\\
	\red
	& \ldots & \nonumber\\
	\red^*
	& P | P | \ldots & \nonumber
\end{eqnarray}

Of course, this encoding, as an implementation, runs away, unfolding
$\bangp{P}$ eagerly. A lazier and more implementable replication
operator, restricted to input-guarded processes, may be obtained as follows.

\begin{eqnarray}
\bangp{\prefix{u}{v}{P}} 
	:= 
	\binpar{\lift{x}{\prefix{u}{v}{(\binpar{D(x)}{P})}}}{D(x)} \nonumber
\end{eqnarray}

\begin{remark}
  Note that the lazier definition still does not deal with summation
  or mixed summation (i.e. sums over input and output). The reader is
  invited to construct definitions of replication that deal with these
  features. 

  Further, the definitions are parameterized in a name, $x$. Can you,
  gentle reader, make a definition that eliminates this parameter and
  guarantees no accidental interaction between the replication
  machinery and the process being replicated -- i.e. no accidental
  sharing of names used by the process to get its work done and the
  name(s) used by the replication to effect copying. This latter
  revision of the definition of replication is crucial to obtaining
  the expected identity $!!P \sim !P$.
\end{remark}

\begin{remark}\label{rem:paradoxical_combinator}
  The reader familiar with the lambda calculus will have noticed the
  similarity between $D$ and the paradoxical combinator.

  [Ed. note: the existence of this seems to suggest we have to be more
  restrictive on the set of processes and names we admit if we are to
  support no-cloning.]
\end{remark}

\subsubsection{Bisimulation}

The computational dynamics gives rise to another kind of equivalence,
the equivalence of computational behavior. As previously mentioned
this is typically captured \emph{via} some form of bisimulation.

% The notion we use in this paper is weak barbed bisimulation
% \cite{milner91polyadicpi}.

The notion we use in this paper is derived from weak barbed
bisimulation \cite{milner91polyadicpi}. 

\begin{definition}
An \emph{observation relation}, $\downarrow_{\mathcal N}$, over a set
of names, $\mathcal N$, is the smallest relation satisfying the rules
below.

\infrule[Out-barb]{y \in {\mathcal N}, \; x \nameeq y}
		  {\outputp{x}{v} \downarrow_{\mathcal N} x}
\infrule[Par-barb]{\mbox{$P\downarrow_{\mathcal N} x$ or $Q\downarrow_{\mathcal N} x$}}
		  {\binpar{P}{Q} \downarrow_{\mathcal N} x}

We write $P \Downarrow_{\mathcal N} x$ if there is $Q$ such that 
$P \wred Q$ and $Q \downarrow_{\mathcal N} x$.
\end{definition}

\begin{definition}
%\label{def.bbisim}
An  ${\mathcal N}$-\emph{barbed bisimulation} over a set of names, ${\mathcal N}$, is a symmetric binary relation 
${\mathcal S}_{\mathcal N}$ between agents such that $P\rel{S}_{\mathcal N}Q$ implies:
\begin{enumerate}
\item If $P \red P'$ then $Q \wred Q'$ and $P'\rel{S}_{\mathcal N} Q'$.
\item If $P\downarrow_{\mathcal N} x$, then $Q\Downarrow_{\mathcal N} x$.
\end{enumerate}
$P$ is ${\mathcal N}$-barbed bisimilar to $Q$, written
$P \wbbisim_{\mathcal N} Q$, if $P \rel{S}_{\mathcal N} Q$ for some ${\mathcal N}$-barbed bisimulation ${\mathcal S}_{\mathcal N}$.
\end{definition}

$\mathcal{R} \subseteq \pi \times \pi$

$P \mathcal{R} Q => \forall P'. P \red P' \Rightarrow \exists Q'. Q \red Q', P' \mathcal{R} Q'$

$P \vdash x \Rightarrow Q \vdash x$

\begin{mathpar}
  \inferrule*[lab=Out-barb]{x \nameeq y}{{y}!\langle{Q}\rangle \vdash x}
  \and
  \inferrule*[lab=Par-barb]{\mbox{$P\vdash x$ or $Q\vdash x$}}{\binpar{P}{Q} \vdash x}
\end{mathpar}

\subsubsection{Contexts}

One of the principle advantages of computational calculi like the
$\pi$-calculus is a well-defined notion of context,
contextual-equivalence and a correlation between
contextual-equivalence and notions of bisimulation. The notion of
context allows the decomposition of a process into (sub-)process and
its syntactic environment, its context. Thus, a context may be
thought of as a process with a ``hole'' (written $\Box$) in it. The
application of a context $M$ to a process $P$, written $M[P]$, is
tantamount to filling the hole in $M$ with $P$. In this paper we do
not need the full weight of this theory, but do make use of the notion
of context in the proof the main theorem. 

\begin{mathpar}
  \inferrule* [lab=summation] {} {{M_{M},M_{N}} \bc \Box \;|\; x.M_{A} \;|\; M_{M}+M_{N}}
  \and
  \inferrule* [lab=agent] {} {{M_{A}} \bc (\vec{x})M_{P} \;| \; \clift{P_0,\ldots,M_{P},\ldots,P_N}}
  \and \\
  \inferrule* [lab=process] {} {{M_{P}} \bc M_{N} \;| \;P|M_{P} }
\end{mathpar} 

\begin{mathpar}
  \inferrule* [lab=sychronization] {} {M_{N} \bc \Box \;|\; x?M_{F} \;|\; x!M_{C}}
  \and
  \inferrule* [lab=abstraction] {} {{M_{F}} \bc (x)M_{P} }
  \and
  \inferrule* [lab=concretion] {} {{M_{C}} \bc \langle M_{P} \rangle }
  \and \\
  \inferrule* [lab=process] {} {{M_{P}} \bc M_{N} \;| \;P|M_{P} }
\end{mathpar}

\begin{definition}[contextual application] Given a context $M$, and
  process $P$, we define the \emph{contextual application}, $M[P] :=
  M\{P/\Box\}$. That is, the contextual application of M to P is the
  substitution of $P$ for $\Box$ in $M$.
\end{definition}

$\meaningof{-} : L \to \mathcal{P}(\pi)$

\begin{mathpar}
  \inferrule* [lab=collection] {} {\meaningof{true} = \pi, \and \meaningof{~E} = \pi \setminus \meaningof{E}, \and \meaningof{E_{1} \& E_{2}} = \meaningof{E_{1}} \cap \meaningof{E_{2}}}
\end{mathpar}

\begin{mathpar}
  \inferrule* [lab=structure] {} {\meaningof{0} = \{ P \in \pi | P \equiv 0 \}, \and \\ \meaningof{E_1 | E_2} = \{ P \in \pi | P \equiv P_{1} | P_{2}, P_{1} \in \meaningof{E_{1}}, P_{2} \in \meaningof{E_2}\} }
\end{mathpar}

\begin{mathpar}
 \inferrule* [lab=behavior] {} {\meaningof{\langle a?b \rangle E} = \{ P \in \pi | P \equiv Q | u?(y)P', \\ \and \\\\ \and \\ \;\;\; u \in \meaningof{a}, \forall z.P'\{z/y\} \in \meaningof{E\{z/b\}}\}, \and \\ \meaningof{a!E} = \{ P \in \pi | P \equiv Q | x!\langle P' \rangle, x \in \meaningof{a} P' \in \meaningof{E}\} }
\end{mathpar}

\begin{mathpar}
 \inferrule* [lab=nominal] {} {\meaningof{\quotep{E}} = \{ \quotep{P} \in \quotep{\pi} | P \in \meaningof{E} \}, \and \meaningof{\quotep{P}} = \{ \quotep{Q} \in \quotep{\pi} | P \equiv Q \} \and \\ \meaningof{@\quotep{E}} = \{ P \in \pi | P \equiv @x, x \in \meaningof{E} \}}
\end{mathpar}

\begin{eqnarray*}
  \\
  \meaningof{-} : TS \to ST
\end{eqnarray*}

\begin{eqnarray*}
  \\
  L : TS \to ST
\end{eqnarray*}

\begin{eqnarray*}
  \\
  P \models E \iff P \in \meaningof{E}
\end{eqnarray*}

\begin{eqnarray*}
  P \approx_{L} Q \iff \forall E \in L. P \models E \iff Q \models E
\end{eqnarray*}

\begin{eqnarray*}
  P \approx_{K} Q
\end{eqnarray*}

\begin{eqnarray*}
  P \approx Q
\end{eqnarray*}

$\approx_{K} = \approx = \approx_{L}$

\subsubsection{Contextual duality}

Note that contexts extend the quotation operation to a family of
operations from processes to names. Given a context, $M$, we can
define a \emph{nominal context}, $\quotep{M}$ by $\quotep{M}[P] :=
\quotep{M[P]}$. To foreshadow what is to come we observe that these
operations enjoy a duality with processes very much like the duality
between vectors and maps from vectors to scalars.

Further, because the calculus is essentially higher-order, we have a
correspondence between contexts and processes. More specifically,
given a name $x$ and a context $M$ we can construct $M^{*}_{x}$ such
that 

\begin{mathpar}
  M^{*}_{x} | \lift{x}{P} \red M[P]
\end{mathpar}

namely,

\begin{mathpar}
  M^{*}_{x} := x?(u).M[\dropn{u}]
\end{mathpar}

The dependence of $M^{*}_{x}$ on a name makes it an abstraction, 

\begin{mathpar}
  M^{*} := (x)x?(u).M[\dropn{u}]
\end{mathpar}

\subsection{Additional notation}

It will sometimes be convenient to denote the process a name
quotes. We already have the notation $x = \quotep{P}$, but it will be
convenient to introduce an alternate notation, $\procn{x}$, when we
want to emphasize the connection to the use of the name. Note that, by
virtue of name equivalence, $\quotep{\procn{x}} \nameeq x$; so, the
notation is consistent with previous definitions.

Further, because names have structure it is possible to effect
substitutions on the basis of that structure. This means we need to
upgrade our notation for substitutions, which we accomplish by
adapting comprehension notation. Thus,

\begin{mathpar}
  P\{ y / x : x \in S \}
\end{mathpar}

is interpreted to mean the process derived from P by replacing (in a
capture-avoiding manner) each occurrence of $x$ in $S$ by $y$. For example,

\begin{mathpar}
  P\{ \quotep{\procn{x}|\procn{x}} / x : x \in \freenames{P} \}
\end{mathpar}

will replace each (occurrence) of a free name $x$ in $P$ by
$\quotep{\procn{x}|\procn{x}}$.

Also, we will avail ourselves of the notation $x^{L}$ and $x^{R}$ to
denote injections of a name into disjoint copies of the name
space. There are numerous ways to accomplish this. One example can be
found in \cite{MeredithR05}. This notation overloads to vectors of
names: $\vec{x}^{\pi} := (x_{i}^{\pi} \; : \; 0 \leq i < |\vec{x}| )$ where $\pi \in \{L,R\}$.

We also use $P^{\Box} := P|\Box$.

In \cite{MeredithR05} an interpretation of the new operator is
given. It turns out that there are several possible interpretations
all enjoying the requisite algebraic properties of the operator (see
\cite{milner91polyadicpi}). We will therefore make liberal use of
$(\nu\; \vec{x})P$.

% subsection the_syntax_and_semantics_of_the_notation_system (end)   

\input{qm2pi.qmops} 

\input{qm2pi.sterngerlach} 

\input{qm2pi.metric} 

% section concurrent_process_calculi (end)

%\input{qm2pi.proofsketch}

% section proof sketch (end)

%\input{qm2pi.slviaknots} 

% section spatial logic via knots (end)

\input{qm2pi.conclusion}

% section conclusion (end)

%\input{qm2pi.dtcodes} 

% section wiring algorithm (end)

\input{qm2pi.ack} 

% section acknowledgments (end)

\newpage


\bibliographystyle{plain}   
\bibliography{../../biblios/main.bib}

\input{qm2pi.rhodetails}

\end{document}

 

% section acknowledgments (end)

\newpage


\bibliographystyle{plain}   
\bibliography{../../biblios/main.bib}

\documentclass[12pt]{llncs}
%\documentclass{jktr}

\usepackage[pdftex]{hyperref}                   
\usepackage {listings}
\usepackage {mathpartir}
\usepackage{bcprules}
%\usepackage{listings}
                       
\usepackage{graphicx} 
%\usepackage[margins=2.5cm,nohead,nofoot]{geometry}
%\usepackage{geometry}
\usepackage{amsfonts}
\usepackage{amstext}
\usepackage{latexsym}
\usepackage{amssymb}
\usepackage{color}


%\include{myPreamble}
\include{qm2pi.local} 

%\ifpdf
%\usepackage[pdftex]{graphicx}
%\else
%\usepackage{graphicx}
%\fi

 % \ifpdf
%  \usepackage{pdfsync}
%  \if


%\title{Brief Article}
%\author{David F. Snyder}
%\author{L.G. Meredith}

%\address{Dept. of Math., Texas State University--San Marcos, San Marcos, TX 78666}
       
\pagestyle{empty}


\begin{document}

\lstset{language=[Objective]Caml,frame=shadowbox}

\input{qm2pi.front}

% section front matter (end)

\input{qm2pi.intro} 
 
% section introduction (end)

% \input{qm2pi.knotations} 

% section notation (end)

\input{qm2pi.process.calculi} 

% section concurrent_process_calculi_and_spatial_logics_ (end)
    
%\input{qm2pi.knots2pi} 

%\input{qm2pi.trefoil} 

%\input{qm2pi.mainthm} 

% subsection basic_interpretation (end)

%\input{qm2pi.rho.presentation} 
\subsection{The syntax and semantics of the notation system}\label{sub:the_syntax_and_semantics_of_the_notation_system} % (fold)

We now summarize a technical presentation of the calculus that
embodies our theory of dynamics. The typical presentation of such a
calculus follows the style of giving generators and relations on
them. The grammar, below, describing term constructors, freely
generates the set of processes, $\Proc$. This set is then quotiented
by a relation known as structural congruence and it is over this set
that the notion of dynamics is expressed. This presentation is
essentially that of \cite{MeredithR05} with the addition of
polyadicity and summation. For readability we have relegated some of
the technical subtleties to an appendix.

\subsubsection{Process grammar}\label{subsub:process_grammar}

\begin{mathpar}
  \inferrule* [lab=synchronization] {} {{M} \bc \pzero \;|\; x?F \;|\; x!C }
  \and
  \inferrule* [lab=abstraction] {} {{F} \bc (x)P}
  \and
  \inferrule* [lab=concretion] {} {{C} \bc \langle Q \rangle}
  \and
  \inferrule* [lab=process] {} {{P,Q} \bc M \;| \;P|Q \;|\; @{x}}
  \and
  \inferrule* [lab=name] {} {{x} \bc \quotep{P}}
\end{mathpar} 

Note that $\vec{x}$ (resp. $\vec{P}$) denotes a vector of names
(resp. processes) of length $|\vec{x}|$ (resp. $|\vec{P}|$). We adopt
the following useful abbreviations.

\begin{mathpar}
   x?(\vec{y}).P := x.(\vec{y})P \and  x\clift{\vec{P}} := x.\clift{\vec{P}}
   \and x!(y) := \lift{x}{\dropn{y}}
   \and \Pi_{i=0}^{n-1}P_i := P_0 | \ldots | P_{n-1}
\end{mathpar}

\subsubsection{Structural congruence}

\paragraph{Free and bound names and alpha-equivalence.} At the
core of structural equivalence is alpha-equivalence which identifies
process that are the same up to a change of variable. Formally, we
recognize the distinction between free and bound names. The free names
of a process, $\freenames{P}$, may be calculated recursively as
follows:

\begin{mathpar}
\freenames{\pzero} := \emptyset
  \and \\
  \freenames{x?(y).P} := \{ x \} \cup (\freenames{P} \setminus \{ y \})
  \and 
  \freenames{x!\langle P \rangle} := \{ x \} \cup \{ P \} 
  \and \\
  \freenames{P|Q} := \freenames{P} \cup \freenames{Q}
  \and \\
  \freenames{@{x}} := \{ x \}
\end{mathpar}

$\pi$
$\quotep{\pi}$

$\freenames{-} : \pi \to \mathcal{P}(\quotep{\pi})$

\begin{eqnarray*}
  \freenames{\pzero} & := & \emptyset \\
  \freenames{x?(y).P} & := & \{ x \} \cup (\freenames{P} \setminus \{ y \}) \\
  \freenames{x!\langle P \rangle} & := & \{ x \} \cup \{ P \} \\
  \freenames{P|Q} & := & \freenames{P} \cup \freenames{Q} \\
  \freenames{\dropn{x}} & := & \{ x \}
\end{eqnarray*}

The bound names of a process, $\boundnames{P}$, are those names occurring in $P$
that are not free. For example, in $x?(y).0$, the name $x$ is free, while $y$ is bound.

\begin{mathpar}
  \inferrule* [lab=monoidal-laws] {} { P|Q \equiv Q|P \and P|0 \equiv P \and P|(Q|R) \equiv (P|Q)|R }
\end{mathpar}

\begin{mathpar}
  \inferrule* [lab=alpha-equivalence] {} { (x)P \equiv (y)P\{y/x\} \and y \not\in \freenames{P} }
\end{mathpar}

\begin{definition}
Then two processes, $P,Q$, are alpha-equivalent if $P = Q\{\vec{y}/\vec{x}\}$ for
some $\vec{x} \in \boundnames{Q},\vec{y} \in \boundnames{P}$, where $Q\{\vec{y}/\vec{x}\}$
denotes the capture-avoiding substitution of $\vec{y}$ for $\vec{x}$ in $Q$.
\end{definition}

\begin{definition}
  The {\em structural congruence} \cite{SangiorgiWalker} , $\equiv$,
  between processes is the least congruence containing
  alpha-equivalence, satisfying the abelian monoid laws
  (associativity, commutativity and $\pzero$ as identity) for parallel
  composition $|$ and for summation $+$.
\end{definition}

\subsection{Name equivalence}

We take name equivalence, written $\nameeq$, to be the smallest
equivalence relation generated by the following rules.

\begin{mathpar}
\inferrule*[lab=Quote-drop]
{ }
{ \quotep{@{x}} \nameeq x }

\inferrule*[lab=Struct-equiv]
{ P \scong Q }
{ \quotep{P} \nameeq \quotep{Q} }
\end{mathpar}

The astute reader will have noticed that the mutual recursion of names
and processes imposes a mutual recursion on alpha-equivalence and
structural equivalence via name-equivalence. Fortunately, all of this
works out pleasantly and we may calculate in the natural way, free of
concern. The reader interested in the details is referred to the
appendix \ref{appendix:rho_details}.

\subsection{Substitution}

We use $\Proc$ for the set of processes, $\QProc$ for the set of
names, and $\id{\{}\vec{y} / \vec{x} \id{\}}$ to denote partial maps,
$s : \QProc \rightarrow \QProc$. A map, $s$ lifts, uniquely, to a map
on process terms, $\widehat{s} : \Proc \rightarrow \Proc$ by the
following equations.

\begin{mathpar}
  (0) \psubstp{Q}{P} := 0 \\
  (R \juxtap S) \psubstp{Q}{P}
  :=    
  (R)\psubstp{Q}{P} \juxtap (S) \psubstp{Q}{P} \\
  (x?(y).R) \psubstp{Q}{P}    
  :=    
  (x)\substp{Q}{P} (z)\concat( (R \psubstn{z}{y}) \psubstp{Q}{P} ) \\
  (\lift{x}{R}) \psubstp{Q}{P}  
  :=
  \lift{(x)\substp{Q}{P}}{ R \psubstp{Q}{P} } \\
%   (\dropn{x})  \psubstp{Q}{P}       
%   := 
%   \left\{ 
%     \begin{array}{ccc} 
%       \dropn{\quotep{Q}} & & x \nameeq \quotep{P} \\
%       \dropn{x} & & otherwise \\
%     \end{array}
%   \right. 
  (\dropn{x})  \psubstp{Q}{P}       
  := 
  \left\{ 
    \begin{array}{ccc} 
      Q & & x \nameeq \quotep{P} \\
      \dropn{x} & & otherwise \\
    \end{array}
  \right.
\end{mathpar}
 

where

\begin{eqnarray}
  (x)\id{\{} \lpquote Q \rpquote / \lpquote P \rpquote \id{\}}            = 
  \left\{ 
    \begin{array}{ccc}
      \lpquote Q \rpquote & & x \nameeq \lpquote P \rpquote \\
      x & & otherwise \\
    \end{array}
  \right. \nonumber
\end{eqnarray}

and $z$ is chosen distinct from $\quotep{P}$, $\quotep{Q}$, the free
names in $Q$, and all the names in $R$. Our $\alpha$-equivalence will
be built in the standard way from this substitution.

\begin{remark}\label{rem:no_self_referential_names}
  One consequence of these definitions is that $\forall P. \quotep{P}
  \not\in \freenames{P}$.
\end{remark}

\subsection{ Dynamic quote: an example }

Anticipating something of what's to come, consider applying the
substitution, $\widehat{\id{\{}u / z \id{\}}}$, to the following pair
of processes, $\lift{w}{y!(z)}$ and $w[ \lpquote y!(z) \rpquote ]$.

\begin{eqnarray}
	\lift{w}{y!(z)}\widehat{\id{\{}u / z \id{\}}}
		& = &
		\lift{w}{y!(u)} \nonumber\\
	w[ \lpquote y!(z) \rpquote ] \widehat{ \id{\{}u / z \id{\}} }
		& = &
		w[ \lpquote y!(z) \rpquote ] \nonumber
\end{eqnarray}

Because the body of the process between quotes is impervious to
substitution, we get radically different answers. In fact, by
examining the first process in an input context,
e.g. $x?(z).\lift{w}{y!(z)}$, we see that the process under the lift
operator may be shaped by prefixed inputs binding a name inside it. In
this sense, the lift operator will be seen as a way to dynamically
construct processes before reifying them as names.

Finally equipped with these standard features we can present the
dynamics of the calculus.

\subsubsection{Operational semantics} 

Finally, we introduce the computational dynamics. What marks these
algebras as distinct from other more traditionally studied algebraic
structures, e.g. vector spaces or polynomial rings, is the manner in
which dynamics is captured. In traditional structures, dynamics is typically
expressed through morphisms between such structures, as in linear maps
between vector spaces or morphisms between rings. In algebras
associated with the semantics of computation, the dynamics is
expressed as part of the algebraic structure itself, through a
reduction reduction relation typically denoted by $\red$. Below, we
give a recursive presentation of this relation for the calculus used
in the encoding.

$\red \subseteq \pi \times \pi$
$\red : \pi \to \mathcal{P}(\pi)$

\begin{mathpar}
  \inferrule* [lab=Comm] { \textsf{match}( x_{src}, x_{trgt} ) } { x_{trgt}?(y)P \; | \; x_{src}!\langle {Q} \rangle \red P\{\quotep{Q}/y}\} }
  \and \\
  \inferrule* [lab=Par] {{P} \red {P}'} {{{P} | {Q}} \red {{P}' | {Q}}}
  \and
  \inferrule* [lab=Equiv]{{{P} \scong {P}'} \andalso {{P}' \red {Q}'} \andalso {{Q}' \scong {Q}}}{{P} \red {Q}}
\end{mathpar}

\begin{eqnarray*}
  match_{\equiv} (\quotep{P},\quotep{Q}) & := & P \equiv Q \\
  match_{\dagger}(\quotep{P},\quotep{Q}) & := & \forall R. P|Q \red^{*} R => R \red^{*} 0 \\
  match_{K}(\quotep{P},\quotep{Q}) & := & K \mbox{ for some context } K
\end{eqnarray*}

$u?(x)P | u!\langle Q \rangle \red P\{\quotep{Q}/x\}$

%We write $\wred$ for $\red^*$, and $P\red$ if $\exists Q $ such that $ P \red Q$.
We write $P\red$ if $\exists Q $ such that $ P \red Q$ and $P\not\red$, otherwise.

\section{Replication}

As mentioned before, it is known that replication (and hence
recursion) can be implemented in a higher-order process algebra
\cite{SangiorgiWalker}. As our first example of calculation with the
machinery thus far presented we give the construction explicitly in
the {\rhoc}.

\begin{eqnarray}
	D_{x} & := & \prefix{x}{y}{(\binpar{\outputp{x}{y}}{@{y}})} \nonumber\\
	\bangp_{x}{P} & := & \binpar{{x}!\langle{\binpar{D_{x}}{P}}\rangle}{D_{x}} \nonumber
\end{eqnarray}

\begin{eqnarray}
	\bangp_{x}{P} & & \nonumber\\
	=
	& {x}!\langle{(\prefix{x}{y}{(\outputp{x}{y} | @{y})) | P}}\rangle 
	      | \prefix{x}{y}{(\outputp{x}{y} | @{y})} & \nonumber\\
	\red
	& (\outputp{x}{y} | @{y})\substn{\quotep{(\prefix{x}{y}{(@{y} | \outputp{x}{y})) | P}}}{y} & \nonumber\\
	=
	& \outputp{x}{\quotep{(\prefix{x}{y}{(\outputp{x}{y} | @{y})) | P}}}
	  | {(\prefix{x}{y}{(\outputp{x}{y} | @{y})) | P}} & \nonumber\\
	\red
	& \ldots & \nonumber\\
	\red^*
	& P | P | \ldots & \nonumber
\end{eqnarray}

Of course, this encoding, as an implementation, runs away, unfolding
$\bangp{P}$ eagerly. A lazier and more implementable replication
operator, restricted to input-guarded processes, may be obtained as follows.

\begin{eqnarray}
\bangp{\prefix{u}{v}{P}} 
	:= 
	\binpar{\lift{x}{\prefix{u}{v}{(\binpar{D(x)}{P})}}}{D(x)} \nonumber
\end{eqnarray}

\begin{remark}
  Note that the lazier definition still does not deal with summation
  or mixed summation (i.e. sums over input and output). The reader is
  invited to construct definitions of replication that deal with these
  features. 

  Further, the definitions are parameterized in a name, $x$. Can you,
  gentle reader, make a definition that eliminates this parameter and
  guarantees no accidental interaction between the replication
  machinery and the process being replicated -- i.e. no accidental
  sharing of names used by the process to get its work done and the
  name(s) used by the replication to effect copying. This latter
  revision of the definition of replication is crucial to obtaining
  the expected identity $!!P \sim !P$.
\end{remark}

\begin{remark}\label{rem:paradoxical_combinator}
  The reader familiar with the lambda calculus will have noticed the
  similarity between $D$ and the paradoxical combinator.

  [Ed. note: the existence of this seems to suggest we have to be more
  restrictive on the set of processes and names we admit if we are to
  support no-cloning.]
\end{remark}

\subsubsection{Bisimulation}

The computational dynamics gives rise to another kind of equivalence,
the equivalence of computational behavior. As previously mentioned
this is typically captured \emph{via} some form of bisimulation.

% The notion we use in this paper is weak barbed bisimulation
% \cite{milner91polyadicpi}.

The notion we use in this paper is derived from weak barbed
bisimulation \cite{milner91polyadicpi}. 

\begin{definition}
An \emph{observation relation}, $\downarrow_{\mathcal N}$, over a set
of names, $\mathcal N$, is the smallest relation satisfying the rules
below.

\infrule[Out-barb]{y \in {\mathcal N}, \; x \nameeq y}
		  {\outputp{x}{v} \downarrow_{\mathcal N} x}
\infrule[Par-barb]{\mbox{$P\downarrow_{\mathcal N} x$ or $Q\downarrow_{\mathcal N} x$}}
		  {\binpar{P}{Q} \downarrow_{\mathcal N} x}

We write $P \Downarrow_{\mathcal N} x$ if there is $Q$ such that 
$P \wred Q$ and $Q \downarrow_{\mathcal N} x$.
\end{definition}

\begin{definition}
%\label{def.bbisim}
An  ${\mathcal N}$-\emph{barbed bisimulation} over a set of names, ${\mathcal N}$, is a symmetric binary relation 
${\mathcal S}_{\mathcal N}$ between agents such that $P\rel{S}_{\mathcal N}Q$ implies:
\begin{enumerate}
\item If $P \red P'$ then $Q \wred Q'$ and $P'\rel{S}_{\mathcal N} Q'$.
\item If $P\downarrow_{\mathcal N} x$, then $Q\Downarrow_{\mathcal N} x$.
\end{enumerate}
$P$ is ${\mathcal N}$-barbed bisimilar to $Q$, written
$P \wbbisim_{\mathcal N} Q$, if $P \rel{S}_{\mathcal N} Q$ for some ${\mathcal N}$-barbed bisimulation ${\mathcal S}_{\mathcal N}$.
\end{definition}

$\mathcal{R} \subseteq \pi \times \pi$

$P \mathcal{R} Q => \forall P'. P \red P' \Rightarrow \exists Q'. Q \red Q', P' \mathcal{R} Q'$

$P \vdash x \Rightarrow Q \vdash x$

\begin{mathpar}
  \inferrule*[lab=Out-barb]{x \nameeq y}{{y}!\langle{Q}\rangle \vdash x}
  \and
  \inferrule*[lab=Par-barb]{\mbox{$P\vdash x$ or $Q\vdash x$}}{\binpar{P}{Q} \vdash x}
\end{mathpar}

\subsubsection{Contexts}

One of the principle advantages of computational calculi like the
$\pi$-calculus is a well-defined notion of context,
contextual-equivalence and a correlation between
contextual-equivalence and notions of bisimulation. The notion of
context allows the decomposition of a process into (sub-)process and
its syntactic environment, its context. Thus, a context may be
thought of as a process with a ``hole'' (written $\Box$) in it. The
application of a context $M$ to a process $P$, written $M[P]$, is
tantamount to filling the hole in $M$ with $P$. In this paper we do
not need the full weight of this theory, but do make use of the notion
of context in the proof the main theorem. 

\begin{mathpar}
  \inferrule* [lab=summation] {} {{M_{M},M_{N}} \bc \Box \;|\; x.M_{A} \;|\; M_{M}+M_{N}}
  \and
  \inferrule* [lab=agent] {} {{M_{A}} \bc (\vec{x})M_{P} \;| \; \clift{P_0,\ldots,M_{P},\ldots,P_N}}
  \and \\
  \inferrule* [lab=process] {} {{M_{P}} \bc M_{N} \;| \;P|M_{P} }
\end{mathpar} 

\begin{mathpar}
  \inferrule* [lab=sychronization] {} {M_{N} \bc \Box \;|\; x?M_{F} \;|\; x!M_{C}}
  \and
  \inferrule* [lab=abstraction] {} {{M_{F}} \bc (x)M_{P} }
  \and
  \inferrule* [lab=concretion] {} {{M_{C}} \bc \langle M_{P} \rangle }
  \and \\
  \inferrule* [lab=process] {} {{M_{P}} \bc M_{N} \;| \;P|M_{P} }
\end{mathpar}

\begin{definition}[contextual application] Given a context $M$, and
  process $P$, we define the \emph{contextual application}, $M[P] :=
  M\{P/\Box\}$. That is, the contextual application of M to P is the
  substitution of $P$ for $\Box$ in $M$.
\end{definition}

$\meaningof{-} : L \to \mathcal{P}(\pi)$

\begin{mathpar}
  \inferrule* [lab=collection] {} {\meaningof{true} = \pi, \and \meaningof{~E} = \pi \setminus \meaningof{E}, \and \meaningof{E_{1} \& E_{2}} = \meaningof{E_{1}} \cap \meaningof{E_{2}}}
\end{mathpar}

\begin{mathpar}
  \inferrule* [lab=structure] {} {\meaningof{0} = \{ P \in \pi | P \equiv 0 \}, \and \\ \meaningof{E_1 | E_2} = \{ P \in \pi | P \equiv P_{1} | P_{2}, P_{1} \in \meaningof{E_{1}}, P_{2} \in \meaningof{E_2}\} }
\end{mathpar}

\begin{mathpar}
 \inferrule* [lab=behavior] {} {\meaningof{\langle a?b \rangle E} = \{ P \in \pi | P \equiv Q | u?(y)P', \\ \and \\\\ \and \\ \;\;\; u \in \meaningof{a}, \forall z.P'\{z/y\} \in \meaningof{E\{z/b\}}\}, \and \\ \meaningof{a!E} = \{ P \in \pi | P \equiv Q | x!\langle P' \rangle, x \in \meaningof{a} P' \in \meaningof{E}\} }
\end{mathpar}

\begin{mathpar}
 \inferrule* [lab=nominal] {} {\meaningof{\quotep{E}} = \{ \quotep{P} \in \quotep{\pi} | P \in \meaningof{E} \}, \and \meaningof{\quotep{P}} = \{ \quotep{Q} \in \quotep{\pi} | P \equiv Q \} \and \\ \meaningof{@\quotep{E}} = \{ P \in \pi | P \equiv @x, x \in \meaningof{E} \}}
\end{mathpar}

\begin{eqnarray*}
  \\
  \meaningof{-} : TS \to ST
\end{eqnarray*}

\begin{eqnarray*}
  \\
  L : TS \to ST
\end{eqnarray*}

\begin{eqnarray*}
  \\
  P \models E \iff P \in \meaningof{E}
\end{eqnarray*}

\begin{eqnarray*}
  P \approx_{L} Q \iff \forall E \in L. P \models E \iff Q \models E
\end{eqnarray*}

\begin{eqnarray*}
  P \approx_{K} Q
\end{eqnarray*}

\begin{eqnarray*}
  P \approx Q
\end{eqnarray*}

$\approx_{K} = \approx = \approx_{L}$

\subsubsection{Contextual duality}

Note that contexts extend the quotation operation to a family of
operations from processes to names. Given a context, $M$, we can
define a \emph{nominal context}, $\quotep{M}$ by $\quotep{M}[P] :=
\quotep{M[P]}$. To foreshadow what is to come we observe that these
operations enjoy a duality with processes very much like the duality
between vectors and maps from vectors to scalars.

Further, because the calculus is essentially higher-order, we have a
correspondence between contexts and processes. More specifically,
given a name $x$ and a context $M$ we can construct $M^{*}_{x}$ such
that 

\begin{mathpar}
  M^{*}_{x} | \lift{x}{P} \red M[P]
\end{mathpar}

namely,

\begin{mathpar}
  M^{*}_{x} := x?(u).M[\dropn{u}]
\end{mathpar}

The dependence of $M^{*}_{x}$ on a name makes it an abstraction, 

\begin{mathpar}
  M^{*} := (x)x?(u).M[\dropn{u}]
\end{mathpar}

\subsection{Additional notation}

It will sometimes be convenient to denote the process a name
quotes. We already have the notation $x = \quotep{P}$, but it will be
convenient to introduce an alternate notation, $\procn{x}$, when we
want to emphasize the connection to the use of the name. Note that, by
virtue of name equivalence, $\quotep{\procn{x}} \nameeq x$; so, the
notation is consistent with previous definitions.

Further, because names have structure it is possible to effect
substitutions on the basis of that structure. This means we need to
upgrade our notation for substitutions, which we accomplish by
adapting comprehension notation. Thus,

\begin{mathpar}
  P\{ y / x : x \in S \}
\end{mathpar}

is interpreted to mean the process derived from P by replacing (in a
capture-avoiding manner) each occurrence of $x$ in $S$ by $y$. For example,

\begin{mathpar}
  P\{ \quotep{\procn{x}|\procn{x}} / x : x \in \freenames{P} \}
\end{mathpar}

will replace each (occurrence) of a free name $x$ in $P$ by
$\quotep{\procn{x}|\procn{x}}$.

Also, we will avail ourselves of the notation $x^{L}$ and $x^{R}$ to
denote injections of a name into disjoint copies of the name
space. There are numerous ways to accomplish this. One example can be
found in \cite{MeredithR05}. This notation overloads to vectors of
names: $\vec{x}^{\pi} := (x_{i}^{\pi} \; : \; 0 \leq i < |\vec{x}| )$ where $\pi \in \{L,R\}$.

We also use $P^{\Box} := P|\Box$.

In \cite{MeredithR05} an interpretation of the new operator is
given. It turns out that there are several possible interpretations
all enjoying the requisite algebraic properties of the operator (see
\cite{milner91polyadicpi}). We will therefore make liberal use of
$(\nu\; \vec{x})P$.

% subsection the_syntax_and_semantics_of_the_notation_system (end)   

\input{qm2pi.qmops} 

\input{qm2pi.sterngerlach} 

\input{qm2pi.metric} 

% section concurrent_process_calculi (end)

%\input{qm2pi.proofsketch}

% section proof sketch (end)

%\input{qm2pi.slviaknots} 

% section spatial logic via knots (end)

\input{qm2pi.conclusion}

% section conclusion (end)

%\input{qm2pi.dtcodes} 

% section wiring algorithm (end)

\input{qm2pi.ack} 

% section acknowledgments (end)

\newpage


\bibliographystyle{plain}   
\bibliography{../../biblios/main.bib}

\input{qm2pi.rhodetails}

\end{document}



\end{document}



% section front matter (end)

\section{Introduction}\label{sec:introduction} % (fold)
In this draft of the material i am going to have to dispense with the
usual writing conventions adopted in papers on these topics. i'm going
to have adopt whatever tone i need at the time i'm writing up the
calculations. Sometimes this may be very conversational; others it may
be the barest mathematical grunts; others still it may be that i have
lifted text from one of my other papers because the exposition of some
point was better said there. i hope that my readers are not unduly put
out by this decision. i'm not doing this to flout convention or be
rebellious. i find these calculations very technically challenging. To
keep everything going technically, something has to give; i have to
let go of some cognitive burden. So, the academic writing style --
with all of its trade-offs in terms of facilitating technical
communication -- is what i'm letting go of. Perhaps subsequent drafts
can be tightened and polished, but for now, i'm going to speak as if
we were sitting together in a coffee shop with a laptop, wifi and a
pad of paper and a pencil.

So, here's what i have to say. We -- you and i, comfortably ensconced
in our coffee shop and well-equipped with our tools -- can realize and
carry out the calculations of quantum mechanics over a very different
formal theory of dynamics, a formal theory of dynamics that
corresponds to a theory of concurrent computation with
\emph{reflection}. It has the advantage that the underlying theory is
already `quantized', but supports analogues all of the continuuous
operations. Strikingly, this underlying theory has recently been
connected with a notion of metric that we can show, by calculating
together, coincides with the metric induced by the inner product.

There are a lot of reasons why you might be interested in seeing
calculations of this form. Here's why i'm interested. For the past
several centuries there has been no competitor to the ``Newtonian''
account of dynamics. As a result the predominant share of accounts of
dynamical systems and situations have had to be formulated in terms of
the Newtonian machinery. i view this as an intellectually dangerous
position to occupy. Everything, despite it's intrinsic shape, turns
into a nail to be hit with this hammer. Recently, however, the theory
of computation has matured to the point where we have candidates for
theories of dynamics that offer very different perspective on
reasoning about dynamical systems and situations. Testing these
candidates against very successful accounts of dynamical situations,
like quantum mechanics, is going to give us some sense of how mature
they are and some measure of the quality of these accounts of
dynamics.

\subsection{Summary of contributions and outline of paper}

So, we're going to develop an interpretation of the operations of
quantum mechanics normally interpreted by Hilbert spaces and
operators. We're going to do this over a theory of computation. Note
that this is very different than the usual quantum computation program
which develops notions of computation over quantum mechanics. Rather,
we are developing a story that aligns with Wheeler's slogan: It from
Bit. To do this we will first provide an account of the theory of
computation at play here. Then we will dive into a calculation-driven
interpretation of the operations of quantum mechanics.

The reason we take this approach is that -- until very recently --
there hasn't been an axiomatic account of quantum mechanics. As a
result there has been no sharp delineation of the mathematical theory
supporting interpretation of the physical theory and the physical
theory, itself. So, ambient features of the maths are free to be
exploited (or supressed) without a real accounting of their physical
relevance. There is no sharp statement ``here's the physical theory''
qua \emph{theory} and ``here's the mathematical interpretation''
enabling a judgment of how faithful the interpretation is -- apart
from experimental observation. When there is an axiomatic account we
can judge how well a given mathematical formalism supports an
interpretation of the axioms, independent of
experimentation. Likewise, we can judge how well we have captured our
physical evidence and experience with our axiomatics, independent of
any specific mathematical implementation, with accidental detail that
may or may not have physical significance. 

In lieu of a fully fleshed out and vetted axiomatic account of quantum
mechanics, interpreting the operational notions in service of modeling
physical systems will have to suffice. In other words, we are not in
the business of providing a model of Hilbert spaces and operators. We
are in the business of providing a model of quantum mechanics because
we are motivated by testing our notions of dynamics against physical
theory; and, the predictive calculations of the physical theory must
serve as the best formulation -- shy of a fully fleshed out axiomatic
account -- of the physical theory itself (as they have for scientific
theories since time immemorial). Put another way, despite a
whole-hearted commitment to an It-from-Bit ontology, we are firmly
aligned with the shut-up-and-calculate camp as the best way to obtain
results either from the physical perspective or as a quality assurance
measure of our fledgling theory of dynamics.

In detail, we present a reflective process calculus. Then we develop
intuitive correspondences between the notions available in this
calculus and the usual physical notions supporting quantum mechanical
calculations. Thus, 

\begin{table}[htp]
  \center{
    \fbox{
      \begin{tabular}{c|c}
        quantum mechanics & process calculus \\
        \hline
        scalar & name \\
        state vector & process \\
        dual & contextual duals \\
        matrix & formal sums of process-context-dual pairs \\
        orthogonality & process annihilation \\
        inner product & execution-formula + quoting
      \end{tabular}
    }
  }
  \caption{QM - process calculi correspondences}
\end{table}

Then we tighten up these intuitions to operational definitions. We
employ the Dirac notation as the best proxy we can find for an
abstract syntax of the quantum mechanical notions. The definitions we
develop put us in contact with equational constraints coming from the
theory that we demonstrate the definitions and calculations satisfy.

This puts us in a position to shut up and calculate for the
Stern-Gerlach experimental set up, showing how these predictive
calculations become calculations on processes in our theory of a
reflective process calculus.

Penultimately, we demonstrate that the notion of metric coming from
the inner product coincides with the notion of metric available from
the theory of bisimulation. This demonstration gives us the right to
think of space as arising from behavior. Finally, we consider where we
might go from the new vantage point we have obtained.

% section introduction (end) 
 
% section introduction (end)

% \documentclass[12pt]{llncs}
%\documentclass{jktr}

\usepackage[pdftex]{hyperref}                   
\usepackage {listings}
\usepackage {mathpartir}
\usepackage{bcprules}
%\usepackage{listings}
                       
\usepackage{graphicx} 
%\usepackage[margins=2.5cm,nohead,nofoot]{geometry}
%\usepackage{geometry}
\usepackage{amsfonts}
\usepackage{amstext}
\usepackage{latexsym}
\usepackage{amssymb}
\usepackage{color}


%\include{myPreamble}
\documentclass[12pt]{llncs}
%\documentclass{jktr}

\usepackage[pdftex]{hyperref}                   
\usepackage {listings}
\usepackage {mathpartir}
\usepackage{bcprules}
%\usepackage{listings}
                       
\usepackage{graphicx} 
%\usepackage[margins=2.5cm,nohead,nofoot]{geometry}
%\usepackage{geometry}
\usepackage{amsfonts}
\usepackage{amstext}
\usepackage{latexsym}
\usepackage{amssymb}
\usepackage{color}


%\include{myPreamble}
\include{qm2pi.local} 

%\ifpdf
%\usepackage[pdftex]{graphicx}
%\else
%\usepackage{graphicx}
%\fi

 % \ifpdf
%  \usepackage{pdfsync}
%  \if


%\title{Brief Article}
%\author{David F. Snyder}
%\author{L.G. Meredith}

%\address{Dept. of Math., Texas State University--San Marcos, San Marcos, TX 78666}
       
\pagestyle{empty}


\begin{document}

\lstset{language=[Objective]Caml,frame=shadowbox}

\input{qm2pi.front}

% section front matter (end)

\input{qm2pi.intro} 
 
% section introduction (end)

% \input{qm2pi.knotations} 

% section notation (end)

\input{qm2pi.process.calculi} 

% section concurrent_process_calculi_and_spatial_logics_ (end)
    
%\input{qm2pi.knots2pi} 

%\input{qm2pi.trefoil} 

%\input{qm2pi.mainthm} 

% subsection basic_interpretation (end)

%\input{qm2pi.rho.presentation} 
\subsection{The syntax and semantics of the notation system}\label{sub:the_syntax_and_semantics_of_the_notation_system} % (fold)

We now summarize a technical presentation of the calculus that
embodies our theory of dynamics. The typical presentation of such a
calculus follows the style of giving generators and relations on
them. The grammar, below, describing term constructors, freely
generates the set of processes, $\Proc$. This set is then quotiented
by a relation known as structural congruence and it is over this set
that the notion of dynamics is expressed. This presentation is
essentially that of \cite{MeredithR05} with the addition of
polyadicity and summation. For readability we have relegated some of
the technical subtleties to an appendix.

\subsubsection{Process grammar}\label{subsub:process_grammar}

\begin{mathpar}
  \inferrule* [lab=synchronization] {} {{M} \bc \pzero \;|\; x?F \;|\; x!C }
  \and
  \inferrule* [lab=abstraction] {} {{F} \bc (x)P}
  \and
  \inferrule* [lab=concretion] {} {{C} \bc \langle Q \rangle}
  \and
  \inferrule* [lab=process] {} {{P,Q} \bc M \;| \;P|Q \;|\; @{x}}
  \and
  \inferrule* [lab=name] {} {{x} \bc \quotep{P}}
\end{mathpar} 

Note that $\vec{x}$ (resp. $\vec{P}$) denotes a vector of names
(resp. processes) of length $|\vec{x}|$ (resp. $|\vec{P}|$). We adopt
the following useful abbreviations.

\begin{mathpar}
   x?(\vec{y}).P := x.(\vec{y})P \and  x\clift{\vec{P}} := x.\clift{\vec{P}}
   \and x!(y) := \lift{x}{\dropn{y}}
   \and \Pi_{i=0}^{n-1}P_i := P_0 | \ldots | P_{n-1}
\end{mathpar}

\subsubsection{Structural congruence}

\paragraph{Free and bound names and alpha-equivalence.} At the
core of structural equivalence is alpha-equivalence which identifies
process that are the same up to a change of variable. Formally, we
recognize the distinction between free and bound names. The free names
of a process, $\freenames{P}$, may be calculated recursively as
follows:

\begin{mathpar}
\freenames{\pzero} := \emptyset
  \and \\
  \freenames{x?(y).P} := \{ x \} \cup (\freenames{P} \setminus \{ y \})
  \and 
  \freenames{x!\langle P \rangle} := \{ x \} \cup \{ P \} 
  \and \\
  \freenames{P|Q} := \freenames{P} \cup \freenames{Q}
  \and \\
  \freenames{@{x}} := \{ x \}
\end{mathpar}

$\pi$
$\quotep{\pi}$

$\freenames{-} : \pi \to \mathcal{P}(\quotep{\pi})$

\begin{eqnarray*}
  \freenames{\pzero} & := & \emptyset \\
  \freenames{x?(y).P} & := & \{ x \} \cup (\freenames{P} \setminus \{ y \}) \\
  \freenames{x!\langle P \rangle} & := & \{ x \} \cup \{ P \} \\
  \freenames{P|Q} & := & \freenames{P} \cup \freenames{Q} \\
  \freenames{\dropn{x}} & := & \{ x \}
\end{eqnarray*}

The bound names of a process, $\boundnames{P}$, are those names occurring in $P$
that are not free. For example, in $x?(y).0$, the name $x$ is free, while $y$ is bound.

\begin{mathpar}
  \inferrule* [lab=monoidal-laws] {} { P|Q \equiv Q|P \and P|0 \equiv P \and P|(Q|R) \equiv (P|Q)|R }
\end{mathpar}

\begin{mathpar}
  \inferrule* [lab=alpha-equivalence] {} { (x)P \equiv (y)P\{y/x\} \and y \not\in \freenames{P} }
\end{mathpar}

\begin{definition}
Then two processes, $P,Q$, are alpha-equivalent if $P = Q\{\vec{y}/\vec{x}\}$ for
some $\vec{x} \in \boundnames{Q},\vec{y} \in \boundnames{P}$, where $Q\{\vec{y}/\vec{x}\}$
denotes the capture-avoiding substitution of $\vec{y}$ for $\vec{x}$ in $Q$.
\end{definition}

\begin{definition}
  The {\em structural congruence} \cite{SangiorgiWalker} , $\equiv$,
  between processes is the least congruence containing
  alpha-equivalence, satisfying the abelian monoid laws
  (associativity, commutativity and $\pzero$ as identity) for parallel
  composition $|$ and for summation $+$.
\end{definition}

\subsection{Name equivalence}

We take name equivalence, written $\nameeq$, to be the smallest
equivalence relation generated by the following rules.

\begin{mathpar}
\inferrule*[lab=Quote-drop]
{ }
{ \quotep{@{x}} \nameeq x }

\inferrule*[lab=Struct-equiv]
{ P \scong Q }
{ \quotep{P} \nameeq \quotep{Q} }
\end{mathpar}

The astute reader will have noticed that the mutual recursion of names
and processes imposes a mutual recursion on alpha-equivalence and
structural equivalence via name-equivalence. Fortunately, all of this
works out pleasantly and we may calculate in the natural way, free of
concern. The reader interested in the details is referred to the
appendix \ref{appendix:rho_details}.

\subsection{Substitution}

We use $\Proc$ for the set of processes, $\QProc$ for the set of
names, and $\id{\{}\vec{y} / \vec{x} \id{\}}$ to denote partial maps,
$s : \QProc \rightarrow \QProc$. A map, $s$ lifts, uniquely, to a map
on process terms, $\widehat{s} : \Proc \rightarrow \Proc$ by the
following equations.

\begin{mathpar}
  (0) \psubstp{Q}{P} := 0 \\
  (R \juxtap S) \psubstp{Q}{P}
  :=    
  (R)\psubstp{Q}{P} \juxtap (S) \psubstp{Q}{P} \\
  (x?(y).R) \psubstp{Q}{P}    
  :=    
  (x)\substp{Q}{P} (z)\concat( (R \psubstn{z}{y}) \psubstp{Q}{P} ) \\
  (\lift{x}{R}) \psubstp{Q}{P}  
  :=
  \lift{(x)\substp{Q}{P}}{ R \psubstp{Q}{P} } \\
%   (\dropn{x})  \psubstp{Q}{P}       
%   := 
%   \left\{ 
%     \begin{array}{ccc} 
%       \dropn{\quotep{Q}} & & x \nameeq \quotep{P} \\
%       \dropn{x} & & otherwise \\
%     \end{array}
%   \right. 
  (\dropn{x})  \psubstp{Q}{P}       
  := 
  \left\{ 
    \begin{array}{ccc} 
      Q & & x \nameeq \quotep{P} \\
      \dropn{x} & & otherwise \\
    \end{array}
  \right.
\end{mathpar}
 

where

\begin{eqnarray}
  (x)\id{\{} \lpquote Q \rpquote / \lpquote P \rpquote \id{\}}            = 
  \left\{ 
    \begin{array}{ccc}
      \lpquote Q \rpquote & & x \nameeq \lpquote P \rpquote \\
      x & & otherwise \\
    \end{array}
  \right. \nonumber
\end{eqnarray}

and $z$ is chosen distinct from $\quotep{P}$, $\quotep{Q}$, the free
names in $Q$, and all the names in $R$. Our $\alpha$-equivalence will
be built in the standard way from this substitution.

\begin{remark}\label{rem:no_self_referential_names}
  One consequence of these definitions is that $\forall P. \quotep{P}
  \not\in \freenames{P}$.
\end{remark}

\subsection{ Dynamic quote: an example }

Anticipating something of what's to come, consider applying the
substitution, $\widehat{\id{\{}u / z \id{\}}}$, to the following pair
of processes, $\lift{w}{y!(z)}$ and $w[ \lpquote y!(z) \rpquote ]$.

\begin{eqnarray}
	\lift{w}{y!(z)}\widehat{\id{\{}u / z \id{\}}}
		& = &
		\lift{w}{y!(u)} \nonumber\\
	w[ \lpquote y!(z) \rpquote ] \widehat{ \id{\{}u / z \id{\}} }
		& = &
		w[ \lpquote y!(z) \rpquote ] \nonumber
\end{eqnarray}

Because the body of the process between quotes is impervious to
substitution, we get radically different answers. In fact, by
examining the first process in an input context,
e.g. $x?(z).\lift{w}{y!(z)}$, we see that the process under the lift
operator may be shaped by prefixed inputs binding a name inside it. In
this sense, the lift operator will be seen as a way to dynamically
construct processes before reifying them as names.

Finally equipped with these standard features we can present the
dynamics of the calculus.

\subsubsection{Operational semantics} 

Finally, we introduce the computational dynamics. What marks these
algebras as distinct from other more traditionally studied algebraic
structures, e.g. vector spaces or polynomial rings, is the manner in
which dynamics is captured. In traditional structures, dynamics is typically
expressed through morphisms between such structures, as in linear maps
between vector spaces or morphisms between rings. In algebras
associated with the semantics of computation, the dynamics is
expressed as part of the algebraic structure itself, through a
reduction reduction relation typically denoted by $\red$. Below, we
give a recursive presentation of this relation for the calculus used
in the encoding.

$\red \subseteq \pi \times \pi$
$\red : \pi \to \mathcal{P}(\pi)$

\begin{mathpar}
  \inferrule* [lab=Comm] { \textsf{match}( x_{src}, x_{trgt} ) } { x_{trgt}?(y)P \; | \; x_{src}!\langle {Q} \rangle \red P\{\quotep{Q}/y}\} }
  \and \\
  \inferrule* [lab=Par] {{P} \red {P}'} {{{P} | {Q}} \red {{P}' | {Q}}}
  \and
  \inferrule* [lab=Equiv]{{{P} \scong {P}'} \andalso {{P}' \red {Q}'} \andalso {{Q}' \scong {Q}}}{{P} \red {Q}}
\end{mathpar}

\begin{eqnarray*}
  match_{\equiv} (\quotep{P},\quotep{Q}) & := & P \equiv Q \\
  match_{\dagger}(\quotep{P},\quotep{Q}) & := & \forall R. P|Q \red^{*} R => R \red^{*} 0 \\
  match_{K}(\quotep{P},\quotep{Q}) & := & K \mbox{ for some context } K
\end{eqnarray*}

$u?(x)P | u!\langle Q \rangle \red P\{\quotep{Q}/x\}$

%We write $\wred$ for $\red^*$, and $P\red$ if $\exists Q $ such that $ P \red Q$.
We write $P\red$ if $\exists Q $ such that $ P \red Q$ and $P\not\red$, otherwise.

\section{Replication}

As mentioned before, it is known that replication (and hence
recursion) can be implemented in a higher-order process algebra
\cite{SangiorgiWalker}. As our first example of calculation with the
machinery thus far presented we give the construction explicitly in
the {\rhoc}.

\begin{eqnarray}
	D_{x} & := & \prefix{x}{y}{(\binpar{\outputp{x}{y}}{@{y}})} \nonumber\\
	\bangp_{x}{P} & := & \binpar{{x}!\langle{\binpar{D_{x}}{P}}\rangle}{D_{x}} \nonumber
\end{eqnarray}

\begin{eqnarray}
	\bangp_{x}{P} & & \nonumber\\
	=
	& {x}!\langle{(\prefix{x}{y}{(\outputp{x}{y} | @{y})) | P}}\rangle 
	      | \prefix{x}{y}{(\outputp{x}{y} | @{y})} & \nonumber\\
	\red
	& (\outputp{x}{y} | @{y})\substn{\quotep{(\prefix{x}{y}{(@{y} | \outputp{x}{y})) | P}}}{y} & \nonumber\\
	=
	& \outputp{x}{\quotep{(\prefix{x}{y}{(\outputp{x}{y} | @{y})) | P}}}
	  | {(\prefix{x}{y}{(\outputp{x}{y} | @{y})) | P}} & \nonumber\\
	\red
	& \ldots & \nonumber\\
	\red^*
	& P | P | \ldots & \nonumber
\end{eqnarray}

Of course, this encoding, as an implementation, runs away, unfolding
$\bangp{P}$ eagerly. A lazier and more implementable replication
operator, restricted to input-guarded processes, may be obtained as follows.

\begin{eqnarray}
\bangp{\prefix{u}{v}{P}} 
	:= 
	\binpar{\lift{x}{\prefix{u}{v}{(\binpar{D(x)}{P})}}}{D(x)} \nonumber
\end{eqnarray}

\begin{remark}
  Note that the lazier definition still does not deal with summation
  or mixed summation (i.e. sums over input and output). The reader is
  invited to construct definitions of replication that deal with these
  features. 

  Further, the definitions are parameterized in a name, $x$. Can you,
  gentle reader, make a definition that eliminates this parameter and
  guarantees no accidental interaction between the replication
  machinery and the process being replicated -- i.e. no accidental
  sharing of names used by the process to get its work done and the
  name(s) used by the replication to effect copying. This latter
  revision of the definition of replication is crucial to obtaining
  the expected identity $!!P \sim !P$.
\end{remark}

\begin{remark}\label{rem:paradoxical_combinator}
  The reader familiar with the lambda calculus will have noticed the
  similarity between $D$ and the paradoxical combinator.

  [Ed. note: the existence of this seems to suggest we have to be more
  restrictive on the set of processes and names we admit if we are to
  support no-cloning.]
\end{remark}

\subsubsection{Bisimulation}

The computational dynamics gives rise to another kind of equivalence,
the equivalence of computational behavior. As previously mentioned
this is typically captured \emph{via} some form of bisimulation.

% The notion we use in this paper is weak barbed bisimulation
% \cite{milner91polyadicpi}.

The notion we use in this paper is derived from weak barbed
bisimulation \cite{milner91polyadicpi}. 

\begin{definition}
An \emph{observation relation}, $\downarrow_{\mathcal N}$, over a set
of names, $\mathcal N$, is the smallest relation satisfying the rules
below.

\infrule[Out-barb]{y \in {\mathcal N}, \; x \nameeq y}
		  {\outputp{x}{v} \downarrow_{\mathcal N} x}
\infrule[Par-barb]{\mbox{$P\downarrow_{\mathcal N} x$ or $Q\downarrow_{\mathcal N} x$}}
		  {\binpar{P}{Q} \downarrow_{\mathcal N} x}

We write $P \Downarrow_{\mathcal N} x$ if there is $Q$ such that 
$P \wred Q$ and $Q \downarrow_{\mathcal N} x$.
\end{definition}

\begin{definition}
%\label{def.bbisim}
An  ${\mathcal N}$-\emph{barbed bisimulation} over a set of names, ${\mathcal N}$, is a symmetric binary relation 
${\mathcal S}_{\mathcal N}$ between agents such that $P\rel{S}_{\mathcal N}Q$ implies:
\begin{enumerate}
\item If $P \red P'$ then $Q \wred Q'$ and $P'\rel{S}_{\mathcal N} Q'$.
\item If $P\downarrow_{\mathcal N} x$, then $Q\Downarrow_{\mathcal N} x$.
\end{enumerate}
$P$ is ${\mathcal N}$-barbed bisimilar to $Q$, written
$P \wbbisim_{\mathcal N} Q$, if $P \rel{S}_{\mathcal N} Q$ for some ${\mathcal N}$-barbed bisimulation ${\mathcal S}_{\mathcal N}$.
\end{definition}

$\mathcal{R} \subseteq \pi \times \pi$

$P \mathcal{R} Q => \forall P'. P \red P' \Rightarrow \exists Q'. Q \red Q', P' \mathcal{R} Q'$

$P \vdash x \Rightarrow Q \vdash x$

\begin{mathpar}
  \inferrule*[lab=Out-barb]{x \nameeq y}{{y}!\langle{Q}\rangle \vdash x}
  \and
  \inferrule*[lab=Par-barb]{\mbox{$P\vdash x$ or $Q\vdash x$}}{\binpar{P}{Q} \vdash x}
\end{mathpar}

\subsubsection{Contexts}

One of the principle advantages of computational calculi like the
$\pi$-calculus is a well-defined notion of context,
contextual-equivalence and a correlation between
contextual-equivalence and notions of bisimulation. The notion of
context allows the decomposition of a process into (sub-)process and
its syntactic environment, its context. Thus, a context may be
thought of as a process with a ``hole'' (written $\Box$) in it. The
application of a context $M$ to a process $P$, written $M[P]$, is
tantamount to filling the hole in $M$ with $P$. In this paper we do
not need the full weight of this theory, but do make use of the notion
of context in the proof the main theorem. 

\begin{mathpar}
  \inferrule* [lab=summation] {} {{M_{M},M_{N}} \bc \Box \;|\; x.M_{A} \;|\; M_{M}+M_{N}}
  \and
  \inferrule* [lab=agent] {} {{M_{A}} \bc (\vec{x})M_{P} \;| \; \clift{P_0,\ldots,M_{P},\ldots,P_N}}
  \and \\
  \inferrule* [lab=process] {} {{M_{P}} \bc M_{N} \;| \;P|M_{P} }
\end{mathpar} 

\begin{mathpar}
  \inferrule* [lab=sychronization] {} {M_{N} \bc \Box \;|\; x?M_{F} \;|\; x!M_{C}}
  \and
  \inferrule* [lab=abstraction] {} {{M_{F}} \bc (x)M_{P} }
  \and
  \inferrule* [lab=concretion] {} {{M_{C}} \bc \langle M_{P} \rangle }
  \and \\
  \inferrule* [lab=process] {} {{M_{P}} \bc M_{N} \;| \;P|M_{P} }
\end{mathpar}

\begin{definition}[contextual application] Given a context $M$, and
  process $P$, we define the \emph{contextual application}, $M[P] :=
  M\{P/\Box\}$. That is, the contextual application of M to P is the
  substitution of $P$ for $\Box$ in $M$.
\end{definition}

$\meaningof{-} : L \to \mathcal{P}(\pi)$

\begin{mathpar}
  \inferrule* [lab=collection] {} {\meaningof{true} = \pi, \and \meaningof{~E} = \pi \setminus \meaningof{E}, \and \meaningof{E_{1} \& E_{2}} = \meaningof{E_{1}} \cap \meaningof{E_{2}}}
\end{mathpar}

\begin{mathpar}
  \inferrule* [lab=structure] {} {\meaningof{0} = \{ P \in \pi | P \equiv 0 \}, \and \\ \meaningof{E_1 | E_2} = \{ P \in \pi | P \equiv P_{1} | P_{2}, P_{1} \in \meaningof{E_{1}}, P_{2} \in \meaningof{E_2}\} }
\end{mathpar}

\begin{mathpar}
 \inferrule* [lab=behavior] {} {\meaningof{\langle a?b \rangle E} = \{ P \in \pi | P \equiv Q | u?(y)P', \\ \and \\\\ \and \\ \;\;\; u \in \meaningof{a}, \forall z.P'\{z/y\} \in \meaningof{E\{z/b\}}\}, \and \\ \meaningof{a!E} = \{ P \in \pi | P \equiv Q | x!\langle P' \rangle, x \in \meaningof{a} P' \in \meaningof{E}\} }
\end{mathpar}

\begin{mathpar}
 \inferrule* [lab=nominal] {} {\meaningof{\quotep{E}} = \{ \quotep{P} \in \quotep{\pi} | P \in \meaningof{E} \}, \and \meaningof{\quotep{P}} = \{ \quotep{Q} \in \quotep{\pi} | P \equiv Q \} \and \\ \meaningof{@\quotep{E}} = \{ P \in \pi | P \equiv @x, x \in \meaningof{E} \}}
\end{mathpar}

\begin{eqnarray*}
  \\
  \meaningof{-} : TS \to ST
\end{eqnarray*}

\begin{eqnarray*}
  \\
  L : TS \to ST
\end{eqnarray*}

\begin{eqnarray*}
  \\
  P \models E \iff P \in \meaningof{E}
\end{eqnarray*}

\begin{eqnarray*}
  P \approx_{L} Q \iff \forall E \in L. P \models E \iff Q \models E
\end{eqnarray*}

\begin{eqnarray*}
  P \approx_{K} Q
\end{eqnarray*}

\begin{eqnarray*}
  P \approx Q
\end{eqnarray*}

$\approx_{K} = \approx = \approx_{L}$

\subsubsection{Contextual duality}

Note that contexts extend the quotation operation to a family of
operations from processes to names. Given a context, $M$, we can
define a \emph{nominal context}, $\quotep{M}$ by $\quotep{M}[P] :=
\quotep{M[P]}$. To foreshadow what is to come we observe that these
operations enjoy a duality with processes very much like the duality
between vectors and maps from vectors to scalars.

Further, because the calculus is essentially higher-order, we have a
correspondence between contexts and processes. More specifically,
given a name $x$ and a context $M$ we can construct $M^{*}_{x}$ such
that 

\begin{mathpar}
  M^{*}_{x} | \lift{x}{P} \red M[P]
\end{mathpar}

namely,

\begin{mathpar}
  M^{*}_{x} := x?(u).M[\dropn{u}]
\end{mathpar}

The dependence of $M^{*}_{x}$ on a name makes it an abstraction, 

\begin{mathpar}
  M^{*} := (x)x?(u).M[\dropn{u}]
\end{mathpar}

\subsection{Additional notation}

It will sometimes be convenient to denote the process a name
quotes. We already have the notation $x = \quotep{P}$, but it will be
convenient to introduce an alternate notation, $\procn{x}$, when we
want to emphasize the connection to the use of the name. Note that, by
virtue of name equivalence, $\quotep{\procn{x}} \nameeq x$; so, the
notation is consistent with previous definitions.

Further, because names have structure it is possible to effect
substitutions on the basis of that structure. This means we need to
upgrade our notation for substitutions, which we accomplish by
adapting comprehension notation. Thus,

\begin{mathpar}
  P\{ y / x : x \in S \}
\end{mathpar}

is interpreted to mean the process derived from P by replacing (in a
capture-avoiding manner) each occurrence of $x$ in $S$ by $y$. For example,

\begin{mathpar}
  P\{ \quotep{\procn{x}|\procn{x}} / x : x \in \freenames{P} \}
\end{mathpar}

will replace each (occurrence) of a free name $x$ in $P$ by
$\quotep{\procn{x}|\procn{x}}$.

Also, we will avail ourselves of the notation $x^{L}$ and $x^{R}$ to
denote injections of a name into disjoint copies of the name
space. There are numerous ways to accomplish this. One example can be
found in \cite{MeredithR05}. This notation overloads to vectors of
names: $\vec{x}^{\pi} := (x_{i}^{\pi} \; : \; 0 \leq i < |\vec{x}| )$ where $\pi \in \{L,R\}$.

We also use $P^{\Box} := P|\Box$.

In \cite{MeredithR05} an interpretation of the new operator is
given. It turns out that there are several possible interpretations
all enjoying the requisite algebraic properties of the operator (see
\cite{milner91polyadicpi}). We will therefore make liberal use of
$(\nu\; \vec{x})P$.

% subsection the_syntax_and_semantics_of_the_notation_system (end)   

\input{qm2pi.qmops} 

\input{qm2pi.sterngerlach} 

\input{qm2pi.metric} 

% section concurrent_process_calculi (end)

%\input{qm2pi.proofsketch}

% section proof sketch (end)

%\input{qm2pi.slviaknots} 

% section spatial logic via knots (end)

\input{qm2pi.conclusion}

% section conclusion (end)

%\input{qm2pi.dtcodes} 

% section wiring algorithm (end)

\input{qm2pi.ack} 

% section acknowledgments (end)

\newpage


\bibliographystyle{plain}   
\bibliography{../../biblios/main.bib}

\input{qm2pi.rhodetails}

\end{document}

 

%\ifpdf
%\usepackage[pdftex]{graphicx}
%\else
%\usepackage{graphicx}
%\fi

 % \ifpdf
%  \usepackage{pdfsync}
%  \if


%\title{Brief Article}
%\author{David F. Snyder}
%\author{L.G. Meredith}

%\address{Dept. of Math., Texas State University--San Marcos, San Marcos, TX 78666}
       
\pagestyle{empty}


\begin{document}

\lstset{language=[Objective]Caml,frame=shadowbox}

\documentclass[12pt]{llncs}
%\documentclass{jktr}

\usepackage[pdftex]{hyperref}                   
\usepackage {listings}
\usepackage {mathpartir}
\usepackage{bcprules}
%\usepackage{listings}
                       
\usepackage{graphicx} 
%\usepackage[margins=2.5cm,nohead,nofoot]{geometry}
%\usepackage{geometry}
\usepackage{amsfonts}
\usepackage{amstext}
\usepackage{latexsym}
\usepackage{amssymb}
\usepackage{color}


%\include{myPreamble}
\include{qm2pi.local} 

%\ifpdf
%\usepackage[pdftex]{graphicx}
%\else
%\usepackage{graphicx}
%\fi

 % \ifpdf
%  \usepackage{pdfsync}
%  \if


%\title{Brief Article}
%\author{David F. Snyder}
%\author{L.G. Meredith}

%\address{Dept. of Math., Texas State University--San Marcos, San Marcos, TX 78666}
       
\pagestyle{empty}


\begin{document}

\lstset{language=[Objective]Caml,frame=shadowbox}

\input{qm2pi.front}

% section front matter (end)

\input{qm2pi.intro} 
 
% section introduction (end)

% \input{qm2pi.knotations} 

% section notation (end)

\input{qm2pi.process.calculi} 

% section concurrent_process_calculi_and_spatial_logics_ (end)
    
%\input{qm2pi.knots2pi} 

%\input{qm2pi.trefoil} 

%\input{qm2pi.mainthm} 

% subsection basic_interpretation (end)

%\input{qm2pi.rho.presentation} 
\subsection{The syntax and semantics of the notation system}\label{sub:the_syntax_and_semantics_of_the_notation_system} % (fold)

We now summarize a technical presentation of the calculus that
embodies our theory of dynamics. The typical presentation of such a
calculus follows the style of giving generators and relations on
them. The grammar, below, describing term constructors, freely
generates the set of processes, $\Proc$. This set is then quotiented
by a relation known as structural congruence and it is over this set
that the notion of dynamics is expressed. This presentation is
essentially that of \cite{MeredithR05} with the addition of
polyadicity and summation. For readability we have relegated some of
the technical subtleties to an appendix.

\subsubsection{Process grammar}\label{subsub:process_grammar}

\begin{mathpar}
  \inferrule* [lab=synchronization] {} {{M} \bc \pzero \;|\; x?F \;|\; x!C }
  \and
  \inferrule* [lab=abstraction] {} {{F} \bc (x)P}
  \and
  \inferrule* [lab=concretion] {} {{C} \bc \langle Q \rangle}
  \and
  \inferrule* [lab=process] {} {{P,Q} \bc M \;| \;P|Q \;|\; @{x}}
  \and
  \inferrule* [lab=name] {} {{x} \bc \quotep{P}}
\end{mathpar} 

Note that $\vec{x}$ (resp. $\vec{P}$) denotes a vector of names
(resp. processes) of length $|\vec{x}|$ (resp. $|\vec{P}|$). We adopt
the following useful abbreviations.

\begin{mathpar}
   x?(\vec{y}).P := x.(\vec{y})P \and  x\clift{\vec{P}} := x.\clift{\vec{P}}
   \and x!(y) := \lift{x}{\dropn{y}}
   \and \Pi_{i=0}^{n-1}P_i := P_0 | \ldots | P_{n-1}
\end{mathpar}

\subsubsection{Structural congruence}

\paragraph{Free and bound names and alpha-equivalence.} At the
core of structural equivalence is alpha-equivalence which identifies
process that are the same up to a change of variable. Formally, we
recognize the distinction between free and bound names. The free names
of a process, $\freenames{P}$, may be calculated recursively as
follows:

\begin{mathpar}
\freenames{\pzero} := \emptyset
  \and \\
  \freenames{x?(y).P} := \{ x \} \cup (\freenames{P} \setminus \{ y \})
  \and 
  \freenames{x!\langle P \rangle} := \{ x \} \cup \{ P \} 
  \and \\
  \freenames{P|Q} := \freenames{P} \cup \freenames{Q}
  \and \\
  \freenames{@{x}} := \{ x \}
\end{mathpar}

$\pi$
$\quotep{\pi}$

$\freenames{-} : \pi \to \mathcal{P}(\quotep{\pi})$

\begin{eqnarray*}
  \freenames{\pzero} & := & \emptyset \\
  \freenames{x?(y).P} & := & \{ x \} \cup (\freenames{P} \setminus \{ y \}) \\
  \freenames{x!\langle P \rangle} & := & \{ x \} \cup \{ P \} \\
  \freenames{P|Q} & := & \freenames{P} \cup \freenames{Q} \\
  \freenames{\dropn{x}} & := & \{ x \}
\end{eqnarray*}

The bound names of a process, $\boundnames{P}$, are those names occurring in $P$
that are not free. For example, in $x?(y).0$, the name $x$ is free, while $y$ is bound.

\begin{mathpar}
  \inferrule* [lab=monoidal-laws] {} { P|Q \equiv Q|P \and P|0 \equiv P \and P|(Q|R) \equiv (P|Q)|R }
\end{mathpar}

\begin{mathpar}
  \inferrule* [lab=alpha-equivalence] {} { (x)P \equiv (y)P\{y/x\} \and y \not\in \freenames{P} }
\end{mathpar}

\begin{definition}
Then two processes, $P,Q$, are alpha-equivalent if $P = Q\{\vec{y}/\vec{x}\}$ for
some $\vec{x} \in \boundnames{Q},\vec{y} \in \boundnames{P}$, where $Q\{\vec{y}/\vec{x}\}$
denotes the capture-avoiding substitution of $\vec{y}$ for $\vec{x}$ in $Q$.
\end{definition}

\begin{definition}
  The {\em structural congruence} \cite{SangiorgiWalker} , $\equiv$,
  between processes is the least congruence containing
  alpha-equivalence, satisfying the abelian monoid laws
  (associativity, commutativity and $\pzero$ as identity) for parallel
  composition $|$ and for summation $+$.
\end{definition}

\subsection{Name equivalence}

We take name equivalence, written $\nameeq$, to be the smallest
equivalence relation generated by the following rules.

\begin{mathpar}
\inferrule*[lab=Quote-drop]
{ }
{ \quotep{@{x}} \nameeq x }

\inferrule*[lab=Struct-equiv]
{ P \scong Q }
{ \quotep{P} \nameeq \quotep{Q} }
\end{mathpar}

The astute reader will have noticed that the mutual recursion of names
and processes imposes a mutual recursion on alpha-equivalence and
structural equivalence via name-equivalence. Fortunately, all of this
works out pleasantly and we may calculate in the natural way, free of
concern. The reader interested in the details is referred to the
appendix \ref{appendix:rho_details}.

\subsection{Substitution}

We use $\Proc$ for the set of processes, $\QProc$ for the set of
names, and $\id{\{}\vec{y} / \vec{x} \id{\}}$ to denote partial maps,
$s : \QProc \rightarrow \QProc$. A map, $s$ lifts, uniquely, to a map
on process terms, $\widehat{s} : \Proc \rightarrow \Proc$ by the
following equations.

\begin{mathpar}
  (0) \psubstp{Q}{P} := 0 \\
  (R \juxtap S) \psubstp{Q}{P}
  :=    
  (R)\psubstp{Q}{P} \juxtap (S) \psubstp{Q}{P} \\
  (x?(y).R) \psubstp{Q}{P}    
  :=    
  (x)\substp{Q}{P} (z)\concat( (R \psubstn{z}{y}) \psubstp{Q}{P} ) \\
  (\lift{x}{R}) \psubstp{Q}{P}  
  :=
  \lift{(x)\substp{Q}{P}}{ R \psubstp{Q}{P} } \\
%   (\dropn{x})  \psubstp{Q}{P}       
%   := 
%   \left\{ 
%     \begin{array}{ccc} 
%       \dropn{\quotep{Q}} & & x \nameeq \quotep{P} \\
%       \dropn{x} & & otherwise \\
%     \end{array}
%   \right. 
  (\dropn{x})  \psubstp{Q}{P}       
  := 
  \left\{ 
    \begin{array}{ccc} 
      Q & & x \nameeq \quotep{P} \\
      \dropn{x} & & otherwise \\
    \end{array}
  \right.
\end{mathpar}
 

where

\begin{eqnarray}
  (x)\id{\{} \lpquote Q \rpquote / \lpquote P \rpquote \id{\}}            = 
  \left\{ 
    \begin{array}{ccc}
      \lpquote Q \rpquote & & x \nameeq \lpquote P \rpquote \\
      x & & otherwise \\
    \end{array}
  \right. \nonumber
\end{eqnarray}

and $z$ is chosen distinct from $\quotep{P}$, $\quotep{Q}$, the free
names in $Q$, and all the names in $R$. Our $\alpha$-equivalence will
be built in the standard way from this substitution.

\begin{remark}\label{rem:no_self_referential_names}
  One consequence of these definitions is that $\forall P. \quotep{P}
  \not\in \freenames{P}$.
\end{remark}

\subsection{ Dynamic quote: an example }

Anticipating something of what's to come, consider applying the
substitution, $\widehat{\id{\{}u / z \id{\}}}$, to the following pair
of processes, $\lift{w}{y!(z)}$ and $w[ \lpquote y!(z) \rpquote ]$.

\begin{eqnarray}
	\lift{w}{y!(z)}\widehat{\id{\{}u / z \id{\}}}
		& = &
		\lift{w}{y!(u)} \nonumber\\
	w[ \lpquote y!(z) \rpquote ] \widehat{ \id{\{}u / z \id{\}} }
		& = &
		w[ \lpquote y!(z) \rpquote ] \nonumber
\end{eqnarray}

Because the body of the process between quotes is impervious to
substitution, we get radically different answers. In fact, by
examining the first process in an input context,
e.g. $x?(z).\lift{w}{y!(z)}$, we see that the process under the lift
operator may be shaped by prefixed inputs binding a name inside it. In
this sense, the lift operator will be seen as a way to dynamically
construct processes before reifying them as names.

Finally equipped with these standard features we can present the
dynamics of the calculus.

\subsubsection{Operational semantics} 

Finally, we introduce the computational dynamics. What marks these
algebras as distinct from other more traditionally studied algebraic
structures, e.g. vector spaces or polynomial rings, is the manner in
which dynamics is captured. In traditional structures, dynamics is typically
expressed through morphisms between such structures, as in linear maps
between vector spaces or morphisms between rings. In algebras
associated with the semantics of computation, the dynamics is
expressed as part of the algebraic structure itself, through a
reduction reduction relation typically denoted by $\red$. Below, we
give a recursive presentation of this relation for the calculus used
in the encoding.

$\red \subseteq \pi \times \pi$
$\red : \pi \to \mathcal{P}(\pi)$

\begin{mathpar}
  \inferrule* [lab=Comm] { \textsf{match}( x_{src}, x_{trgt} ) } { x_{trgt}?(y)P \; | \; x_{src}!\langle {Q} \rangle \red P\{\quotep{Q}/y}\} }
  \and \\
  \inferrule* [lab=Par] {{P} \red {P}'} {{{P} | {Q}} \red {{P}' | {Q}}}
  \and
  \inferrule* [lab=Equiv]{{{P} \scong {P}'} \andalso {{P}' \red {Q}'} \andalso {{Q}' \scong {Q}}}{{P} \red {Q}}
\end{mathpar}

\begin{eqnarray*}
  match_{\equiv} (\quotep{P},\quotep{Q}) & := & P \equiv Q \\
  match_{\dagger}(\quotep{P},\quotep{Q}) & := & \forall R. P|Q \red^{*} R => R \red^{*} 0 \\
  match_{K}(\quotep{P},\quotep{Q}) & := & K \mbox{ for some context } K
\end{eqnarray*}

$u?(x)P | u!\langle Q \rangle \red P\{\quotep{Q}/x\}$

%We write $\wred$ for $\red^*$, and $P\red$ if $\exists Q $ such that $ P \red Q$.
We write $P\red$ if $\exists Q $ such that $ P \red Q$ and $P\not\red$, otherwise.

\section{Replication}

As mentioned before, it is known that replication (and hence
recursion) can be implemented in a higher-order process algebra
\cite{SangiorgiWalker}. As our first example of calculation with the
machinery thus far presented we give the construction explicitly in
the {\rhoc}.

\begin{eqnarray}
	D_{x} & := & \prefix{x}{y}{(\binpar{\outputp{x}{y}}{@{y}})} \nonumber\\
	\bangp_{x}{P} & := & \binpar{{x}!\langle{\binpar{D_{x}}{P}}\rangle}{D_{x}} \nonumber
\end{eqnarray}

\begin{eqnarray}
	\bangp_{x}{P} & & \nonumber\\
	=
	& {x}!\langle{(\prefix{x}{y}{(\outputp{x}{y} | @{y})) | P}}\rangle 
	      | \prefix{x}{y}{(\outputp{x}{y} | @{y})} & \nonumber\\
	\red
	& (\outputp{x}{y} | @{y})\substn{\quotep{(\prefix{x}{y}{(@{y} | \outputp{x}{y})) | P}}}{y} & \nonumber\\
	=
	& \outputp{x}{\quotep{(\prefix{x}{y}{(\outputp{x}{y} | @{y})) | P}}}
	  | {(\prefix{x}{y}{(\outputp{x}{y} | @{y})) | P}} & \nonumber\\
	\red
	& \ldots & \nonumber\\
	\red^*
	& P | P | \ldots & \nonumber
\end{eqnarray}

Of course, this encoding, as an implementation, runs away, unfolding
$\bangp{P}$ eagerly. A lazier and more implementable replication
operator, restricted to input-guarded processes, may be obtained as follows.

\begin{eqnarray}
\bangp{\prefix{u}{v}{P}} 
	:= 
	\binpar{\lift{x}{\prefix{u}{v}{(\binpar{D(x)}{P})}}}{D(x)} \nonumber
\end{eqnarray}

\begin{remark}
  Note that the lazier definition still does not deal with summation
  or mixed summation (i.e. sums over input and output). The reader is
  invited to construct definitions of replication that deal with these
  features. 

  Further, the definitions are parameterized in a name, $x$. Can you,
  gentle reader, make a definition that eliminates this parameter and
  guarantees no accidental interaction between the replication
  machinery and the process being replicated -- i.e. no accidental
  sharing of names used by the process to get its work done and the
  name(s) used by the replication to effect copying. This latter
  revision of the definition of replication is crucial to obtaining
  the expected identity $!!P \sim !P$.
\end{remark}

\begin{remark}\label{rem:paradoxical_combinator}
  The reader familiar with the lambda calculus will have noticed the
  similarity between $D$ and the paradoxical combinator.

  [Ed. note: the existence of this seems to suggest we have to be more
  restrictive on the set of processes and names we admit if we are to
  support no-cloning.]
\end{remark}

\subsubsection{Bisimulation}

The computational dynamics gives rise to another kind of equivalence,
the equivalence of computational behavior. As previously mentioned
this is typically captured \emph{via} some form of bisimulation.

% The notion we use in this paper is weak barbed bisimulation
% \cite{milner91polyadicpi}.

The notion we use in this paper is derived from weak barbed
bisimulation \cite{milner91polyadicpi}. 

\begin{definition}
An \emph{observation relation}, $\downarrow_{\mathcal N}$, over a set
of names, $\mathcal N$, is the smallest relation satisfying the rules
below.

\infrule[Out-barb]{y \in {\mathcal N}, \; x \nameeq y}
		  {\outputp{x}{v} \downarrow_{\mathcal N} x}
\infrule[Par-barb]{\mbox{$P\downarrow_{\mathcal N} x$ or $Q\downarrow_{\mathcal N} x$}}
		  {\binpar{P}{Q} \downarrow_{\mathcal N} x}

We write $P \Downarrow_{\mathcal N} x$ if there is $Q$ such that 
$P \wred Q$ and $Q \downarrow_{\mathcal N} x$.
\end{definition}

\begin{definition}
%\label{def.bbisim}
An  ${\mathcal N}$-\emph{barbed bisimulation} over a set of names, ${\mathcal N}$, is a symmetric binary relation 
${\mathcal S}_{\mathcal N}$ between agents such that $P\rel{S}_{\mathcal N}Q$ implies:
\begin{enumerate}
\item If $P \red P'$ then $Q \wred Q'$ and $P'\rel{S}_{\mathcal N} Q'$.
\item If $P\downarrow_{\mathcal N} x$, then $Q\Downarrow_{\mathcal N} x$.
\end{enumerate}
$P$ is ${\mathcal N}$-barbed bisimilar to $Q$, written
$P \wbbisim_{\mathcal N} Q$, if $P \rel{S}_{\mathcal N} Q$ for some ${\mathcal N}$-barbed bisimulation ${\mathcal S}_{\mathcal N}$.
\end{definition}

$\mathcal{R} \subseteq \pi \times \pi$

$P \mathcal{R} Q => \forall P'. P \red P' \Rightarrow \exists Q'. Q \red Q', P' \mathcal{R} Q'$

$P \vdash x \Rightarrow Q \vdash x$

\begin{mathpar}
  \inferrule*[lab=Out-barb]{x \nameeq y}{{y}!\langle{Q}\rangle \vdash x}
  \and
  \inferrule*[lab=Par-barb]{\mbox{$P\vdash x$ or $Q\vdash x$}}{\binpar{P}{Q} \vdash x}
\end{mathpar}

\subsubsection{Contexts}

One of the principle advantages of computational calculi like the
$\pi$-calculus is a well-defined notion of context,
contextual-equivalence and a correlation between
contextual-equivalence and notions of bisimulation. The notion of
context allows the decomposition of a process into (sub-)process and
its syntactic environment, its context. Thus, a context may be
thought of as a process with a ``hole'' (written $\Box$) in it. The
application of a context $M$ to a process $P$, written $M[P]$, is
tantamount to filling the hole in $M$ with $P$. In this paper we do
not need the full weight of this theory, but do make use of the notion
of context in the proof the main theorem. 

\begin{mathpar}
  \inferrule* [lab=summation] {} {{M_{M},M_{N}} \bc \Box \;|\; x.M_{A} \;|\; M_{M}+M_{N}}
  \and
  \inferrule* [lab=agent] {} {{M_{A}} \bc (\vec{x})M_{P} \;| \; \clift{P_0,\ldots,M_{P},\ldots,P_N}}
  \and \\
  \inferrule* [lab=process] {} {{M_{P}} \bc M_{N} \;| \;P|M_{P} }
\end{mathpar} 

\begin{mathpar}
  \inferrule* [lab=sychronization] {} {M_{N} \bc \Box \;|\; x?M_{F} \;|\; x!M_{C}}
  \and
  \inferrule* [lab=abstraction] {} {{M_{F}} \bc (x)M_{P} }
  \and
  \inferrule* [lab=concretion] {} {{M_{C}} \bc \langle M_{P} \rangle }
  \and \\
  \inferrule* [lab=process] {} {{M_{P}} \bc M_{N} \;| \;P|M_{P} }
\end{mathpar}

\begin{definition}[contextual application] Given a context $M$, and
  process $P$, we define the \emph{contextual application}, $M[P] :=
  M\{P/\Box\}$. That is, the contextual application of M to P is the
  substitution of $P$ for $\Box$ in $M$.
\end{definition}

$\meaningof{-} : L \to \mathcal{P}(\pi)$

\begin{mathpar}
  \inferrule* [lab=collection] {} {\meaningof{true} = \pi, \and \meaningof{~E} = \pi \setminus \meaningof{E}, \and \meaningof{E_{1} \& E_{2}} = \meaningof{E_{1}} \cap \meaningof{E_{2}}}
\end{mathpar}

\begin{mathpar}
  \inferrule* [lab=structure] {} {\meaningof{0} = \{ P \in \pi | P \equiv 0 \}, \and \\ \meaningof{E_1 | E_2} = \{ P \in \pi | P \equiv P_{1} | P_{2}, P_{1} \in \meaningof{E_{1}}, P_{2} \in \meaningof{E_2}\} }
\end{mathpar}

\begin{mathpar}
 \inferrule* [lab=behavior] {} {\meaningof{\langle a?b \rangle E} = \{ P \in \pi | P \equiv Q | u?(y)P', \\ \and \\\\ \and \\ \;\;\; u \in \meaningof{a}, \forall z.P'\{z/y\} \in \meaningof{E\{z/b\}}\}, \and \\ \meaningof{a!E} = \{ P \in \pi | P \equiv Q | x!\langle P' \rangle, x \in \meaningof{a} P' \in \meaningof{E}\} }
\end{mathpar}

\begin{mathpar}
 \inferrule* [lab=nominal] {} {\meaningof{\quotep{E}} = \{ \quotep{P} \in \quotep{\pi} | P \in \meaningof{E} \}, \and \meaningof{\quotep{P}} = \{ \quotep{Q} \in \quotep{\pi} | P \equiv Q \} \and \\ \meaningof{@\quotep{E}} = \{ P \in \pi | P \equiv @x, x \in \meaningof{E} \}}
\end{mathpar}

\begin{eqnarray*}
  \\
  \meaningof{-} : TS \to ST
\end{eqnarray*}

\begin{eqnarray*}
  \\
  L : TS \to ST
\end{eqnarray*}

\begin{eqnarray*}
  \\
  P \models E \iff P \in \meaningof{E}
\end{eqnarray*}

\begin{eqnarray*}
  P \approx_{L} Q \iff \forall E \in L. P \models E \iff Q \models E
\end{eqnarray*}

\begin{eqnarray*}
  P \approx_{K} Q
\end{eqnarray*}

\begin{eqnarray*}
  P \approx Q
\end{eqnarray*}

$\approx_{K} = \approx = \approx_{L}$

\subsubsection{Contextual duality}

Note that contexts extend the quotation operation to a family of
operations from processes to names. Given a context, $M$, we can
define a \emph{nominal context}, $\quotep{M}$ by $\quotep{M}[P] :=
\quotep{M[P]}$. To foreshadow what is to come we observe that these
operations enjoy a duality with processes very much like the duality
between vectors and maps from vectors to scalars.

Further, because the calculus is essentially higher-order, we have a
correspondence between contexts and processes. More specifically,
given a name $x$ and a context $M$ we can construct $M^{*}_{x}$ such
that 

\begin{mathpar}
  M^{*}_{x} | \lift{x}{P} \red M[P]
\end{mathpar}

namely,

\begin{mathpar}
  M^{*}_{x} := x?(u).M[\dropn{u}]
\end{mathpar}

The dependence of $M^{*}_{x}$ on a name makes it an abstraction, 

\begin{mathpar}
  M^{*} := (x)x?(u).M[\dropn{u}]
\end{mathpar}

\subsection{Additional notation}

It will sometimes be convenient to denote the process a name
quotes. We already have the notation $x = \quotep{P}$, but it will be
convenient to introduce an alternate notation, $\procn{x}$, when we
want to emphasize the connection to the use of the name. Note that, by
virtue of name equivalence, $\quotep{\procn{x}} \nameeq x$; so, the
notation is consistent with previous definitions.

Further, because names have structure it is possible to effect
substitutions on the basis of that structure. This means we need to
upgrade our notation for substitutions, which we accomplish by
adapting comprehension notation. Thus,

\begin{mathpar}
  P\{ y / x : x \in S \}
\end{mathpar}

is interpreted to mean the process derived from P by replacing (in a
capture-avoiding manner) each occurrence of $x$ in $S$ by $y$. For example,

\begin{mathpar}
  P\{ \quotep{\procn{x}|\procn{x}} / x : x \in \freenames{P} \}
\end{mathpar}

will replace each (occurrence) of a free name $x$ in $P$ by
$\quotep{\procn{x}|\procn{x}}$.

Also, we will avail ourselves of the notation $x^{L}$ and $x^{R}$ to
denote injections of a name into disjoint copies of the name
space. There are numerous ways to accomplish this. One example can be
found in \cite{MeredithR05}. This notation overloads to vectors of
names: $\vec{x}^{\pi} := (x_{i}^{\pi} \; : \; 0 \leq i < |\vec{x}| )$ where $\pi \in \{L,R\}$.

We also use $P^{\Box} := P|\Box$.

In \cite{MeredithR05} an interpretation of the new operator is
given. It turns out that there are several possible interpretations
all enjoying the requisite algebraic properties of the operator (see
\cite{milner91polyadicpi}). We will therefore make liberal use of
$(\nu\; \vec{x})P$.

% subsection the_syntax_and_semantics_of_the_notation_system (end)   

\input{qm2pi.qmops} 

\input{qm2pi.sterngerlach} 

\input{qm2pi.metric} 

% section concurrent_process_calculi (end)

%\input{qm2pi.proofsketch}

% section proof sketch (end)

%\input{qm2pi.slviaknots} 

% section spatial logic via knots (end)

\input{qm2pi.conclusion}

% section conclusion (end)

%\input{qm2pi.dtcodes} 

% section wiring algorithm (end)

\input{qm2pi.ack} 

% section acknowledgments (end)

\newpage


\bibliographystyle{plain}   
\bibliography{../../biblios/main.bib}

\input{qm2pi.rhodetails}

\end{document}



% section front matter (end)

\section{Introduction}\label{sec:introduction} % (fold)
In this draft of the material i am going to have to dispense with the
usual writing conventions adopted in papers on these topics. i'm going
to have adopt whatever tone i need at the time i'm writing up the
calculations. Sometimes this may be very conversational; others it may
be the barest mathematical grunts; others still it may be that i have
lifted text from one of my other papers because the exposition of some
point was better said there. i hope that my readers are not unduly put
out by this decision. i'm not doing this to flout convention or be
rebellious. i find these calculations very technically challenging. To
keep everything going technically, something has to give; i have to
let go of some cognitive burden. So, the academic writing style --
with all of its trade-offs in terms of facilitating technical
communication -- is what i'm letting go of. Perhaps subsequent drafts
can be tightened and polished, but for now, i'm going to speak as if
we were sitting together in a coffee shop with a laptop, wifi and a
pad of paper and a pencil.

So, here's what i have to say. We -- you and i, comfortably ensconced
in our coffee shop and well-equipped with our tools -- can realize and
carry out the calculations of quantum mechanics over a very different
formal theory of dynamics, a formal theory of dynamics that
corresponds to a theory of concurrent computation with
\emph{reflection}. It has the advantage that the underlying theory is
already `quantized', but supports analogues all of the continuuous
operations. Strikingly, this underlying theory has recently been
connected with a notion of metric that we can show, by calculating
together, coincides with the metric induced by the inner product.

There are a lot of reasons why you might be interested in seeing
calculations of this form. Here's why i'm interested. For the past
several centuries there has been no competitor to the ``Newtonian''
account of dynamics. As a result the predominant share of accounts of
dynamical systems and situations have had to be formulated in terms of
the Newtonian machinery. i view this as an intellectually dangerous
position to occupy. Everything, despite it's intrinsic shape, turns
into a nail to be hit with this hammer. Recently, however, the theory
of computation has matured to the point where we have candidates for
theories of dynamics that offer very different perspective on
reasoning about dynamical systems and situations. Testing these
candidates against very successful accounts of dynamical situations,
like quantum mechanics, is going to give us some sense of how mature
they are and some measure of the quality of these accounts of
dynamics.

\subsection{Summary of contributions and outline of paper}

So, we're going to develop an interpretation of the operations of
quantum mechanics normally interpreted by Hilbert spaces and
operators. We're going to do this over a theory of computation. Note
that this is very different than the usual quantum computation program
which develops notions of computation over quantum mechanics. Rather,
we are developing a story that aligns with Wheeler's slogan: It from
Bit. To do this we will first provide an account of the theory of
computation at play here. Then we will dive into a calculation-driven
interpretation of the operations of quantum mechanics.

The reason we take this approach is that -- until very recently --
there hasn't been an axiomatic account of quantum mechanics. As a
result there has been no sharp delineation of the mathematical theory
supporting interpretation of the physical theory and the physical
theory, itself. So, ambient features of the maths are free to be
exploited (or supressed) without a real accounting of their physical
relevance. There is no sharp statement ``here's the physical theory''
qua \emph{theory} and ``here's the mathematical interpretation''
enabling a judgment of how faithful the interpretation is -- apart
from experimental observation. When there is an axiomatic account we
can judge how well a given mathematical formalism supports an
interpretation of the axioms, independent of
experimentation. Likewise, we can judge how well we have captured our
physical evidence and experience with our axiomatics, independent of
any specific mathematical implementation, with accidental detail that
may or may not have physical significance. 

In lieu of a fully fleshed out and vetted axiomatic account of quantum
mechanics, interpreting the operational notions in service of modeling
physical systems will have to suffice. In other words, we are not in
the business of providing a model of Hilbert spaces and operators. We
are in the business of providing a model of quantum mechanics because
we are motivated by testing our notions of dynamics against physical
theory; and, the predictive calculations of the physical theory must
serve as the best formulation -- shy of a fully fleshed out axiomatic
account -- of the physical theory itself (as they have for scientific
theories since time immemorial). Put another way, despite a
whole-hearted commitment to an It-from-Bit ontology, we are firmly
aligned with the shut-up-and-calculate camp as the best way to obtain
results either from the physical perspective or as a quality assurance
measure of our fledgling theory of dynamics.

In detail, we present a reflective process calculus. Then we develop
intuitive correspondences between the notions available in this
calculus and the usual physical notions supporting quantum mechanical
calculations. Thus, 

\begin{table}[htp]
  \center{
    \fbox{
      \begin{tabular}{c|c}
        quantum mechanics & process calculus \\
        \hline
        scalar & name \\
        state vector & process \\
        dual & contextual duals \\
        matrix & formal sums of process-context-dual pairs \\
        orthogonality & process annihilation \\
        inner product & execution-formula + quoting
      \end{tabular}
    }
  }
  \caption{QM - process calculi correspondences}
\end{table}

Then we tighten up these intuitions to operational definitions. We
employ the Dirac notation as the best proxy we can find for an
abstract syntax of the quantum mechanical notions. The definitions we
develop put us in contact with equational constraints coming from the
theory that we demonstrate the definitions and calculations satisfy.

This puts us in a position to shut up and calculate for the
Stern-Gerlach experimental set up, showing how these predictive
calculations become calculations on processes in our theory of a
reflective process calculus.

Penultimately, we demonstrate that the notion of metric coming from
the inner product coincides with the notion of metric available from
the theory of bisimulation. This demonstration gives us the right to
think of space as arising from behavior. Finally, we consider where we
might go from the new vantage point we have obtained.

% section introduction (end) 
 
% section introduction (end)

% \documentclass[12pt]{llncs}
%\documentclass{jktr}

\usepackage[pdftex]{hyperref}                   
\usepackage {listings}
\usepackage {mathpartir}
\usepackage{bcprules}
%\usepackage{listings}
                       
\usepackage{graphicx} 
%\usepackage[margins=2.5cm,nohead,nofoot]{geometry}
%\usepackage{geometry}
\usepackage{amsfonts}
\usepackage{amstext}
\usepackage{latexsym}
\usepackage{amssymb}
\usepackage{color}


%\include{myPreamble}
\include{qm2pi.local} 

%\ifpdf
%\usepackage[pdftex]{graphicx}
%\else
%\usepackage{graphicx}
%\fi

 % \ifpdf
%  \usepackage{pdfsync}
%  \if


%\title{Brief Article}
%\author{David F. Snyder}
%\author{L.G. Meredith}

%\address{Dept. of Math., Texas State University--San Marcos, San Marcos, TX 78666}
       
\pagestyle{empty}


\begin{document}

\lstset{language=[Objective]Caml,frame=shadowbox}

\input{qm2pi.front}

% section front matter (end)

\input{qm2pi.intro} 
 
% section introduction (end)

% \input{qm2pi.knotations} 

% section notation (end)

\input{qm2pi.process.calculi} 

% section concurrent_process_calculi_and_spatial_logics_ (end)
    
%\input{qm2pi.knots2pi} 

%\input{qm2pi.trefoil} 

%\input{qm2pi.mainthm} 

% subsection basic_interpretation (end)

%\input{qm2pi.rho.presentation} 
\subsection{The syntax and semantics of the notation system}\label{sub:the_syntax_and_semantics_of_the_notation_system} % (fold)

We now summarize a technical presentation of the calculus that
embodies our theory of dynamics. The typical presentation of such a
calculus follows the style of giving generators and relations on
them. The grammar, below, describing term constructors, freely
generates the set of processes, $\Proc$. This set is then quotiented
by a relation known as structural congruence and it is over this set
that the notion of dynamics is expressed. This presentation is
essentially that of \cite{MeredithR05} with the addition of
polyadicity and summation. For readability we have relegated some of
the technical subtleties to an appendix.

\subsubsection{Process grammar}\label{subsub:process_grammar}

\begin{mathpar}
  \inferrule* [lab=synchronization] {} {{M} \bc \pzero \;|\; x?F \;|\; x!C }
  \and
  \inferrule* [lab=abstraction] {} {{F} \bc (x)P}
  \and
  \inferrule* [lab=concretion] {} {{C} \bc \langle Q \rangle}
  \and
  \inferrule* [lab=process] {} {{P,Q} \bc M \;| \;P|Q \;|\; @{x}}
  \and
  \inferrule* [lab=name] {} {{x} \bc \quotep{P}}
\end{mathpar} 

Note that $\vec{x}$ (resp. $\vec{P}$) denotes a vector of names
(resp. processes) of length $|\vec{x}|$ (resp. $|\vec{P}|$). We adopt
the following useful abbreviations.

\begin{mathpar}
   x?(\vec{y}).P := x.(\vec{y})P \and  x\clift{\vec{P}} := x.\clift{\vec{P}}
   \and x!(y) := \lift{x}{\dropn{y}}
   \and \Pi_{i=0}^{n-1}P_i := P_0 | \ldots | P_{n-1}
\end{mathpar}

\subsubsection{Structural congruence}

\paragraph{Free and bound names and alpha-equivalence.} At the
core of structural equivalence is alpha-equivalence which identifies
process that are the same up to a change of variable. Formally, we
recognize the distinction between free and bound names. The free names
of a process, $\freenames{P}$, may be calculated recursively as
follows:

\begin{mathpar}
\freenames{\pzero} := \emptyset
  \and \\
  \freenames{x?(y).P} := \{ x \} \cup (\freenames{P} \setminus \{ y \})
  \and 
  \freenames{x!\langle P \rangle} := \{ x \} \cup \{ P \} 
  \and \\
  \freenames{P|Q} := \freenames{P} \cup \freenames{Q}
  \and \\
  \freenames{@{x}} := \{ x \}
\end{mathpar}

$\pi$
$\quotep{\pi}$

$\freenames{-} : \pi \to \mathcal{P}(\quotep{\pi})$

\begin{eqnarray*}
  \freenames{\pzero} & := & \emptyset \\
  \freenames{x?(y).P} & := & \{ x \} \cup (\freenames{P} \setminus \{ y \}) \\
  \freenames{x!\langle P \rangle} & := & \{ x \} \cup \{ P \} \\
  \freenames{P|Q} & := & \freenames{P} \cup \freenames{Q} \\
  \freenames{\dropn{x}} & := & \{ x \}
\end{eqnarray*}

The bound names of a process, $\boundnames{P}$, are those names occurring in $P$
that are not free. For example, in $x?(y).0$, the name $x$ is free, while $y$ is bound.

\begin{mathpar}
  \inferrule* [lab=monoidal-laws] {} { P|Q \equiv Q|P \and P|0 \equiv P \and P|(Q|R) \equiv (P|Q)|R }
\end{mathpar}

\begin{mathpar}
  \inferrule* [lab=alpha-equivalence] {} { (x)P \equiv (y)P\{y/x\} \and y \not\in \freenames{P} }
\end{mathpar}

\begin{definition}
Then two processes, $P,Q$, are alpha-equivalent if $P = Q\{\vec{y}/\vec{x}\}$ for
some $\vec{x} \in \boundnames{Q},\vec{y} \in \boundnames{P}$, where $Q\{\vec{y}/\vec{x}\}$
denotes the capture-avoiding substitution of $\vec{y}$ for $\vec{x}$ in $Q$.
\end{definition}

\begin{definition}
  The {\em structural congruence} \cite{SangiorgiWalker} , $\equiv$,
  between processes is the least congruence containing
  alpha-equivalence, satisfying the abelian monoid laws
  (associativity, commutativity and $\pzero$ as identity) for parallel
  composition $|$ and for summation $+$.
\end{definition}

\subsection{Name equivalence}

We take name equivalence, written $\nameeq$, to be the smallest
equivalence relation generated by the following rules.

\begin{mathpar}
\inferrule*[lab=Quote-drop]
{ }
{ \quotep{@{x}} \nameeq x }

\inferrule*[lab=Struct-equiv]
{ P \scong Q }
{ \quotep{P} \nameeq \quotep{Q} }
\end{mathpar}

The astute reader will have noticed that the mutual recursion of names
and processes imposes a mutual recursion on alpha-equivalence and
structural equivalence via name-equivalence. Fortunately, all of this
works out pleasantly and we may calculate in the natural way, free of
concern. The reader interested in the details is referred to the
appendix \ref{appendix:rho_details}.

\subsection{Substitution}

We use $\Proc$ for the set of processes, $\QProc$ for the set of
names, and $\id{\{}\vec{y} / \vec{x} \id{\}}$ to denote partial maps,
$s : \QProc \rightarrow \QProc$. A map, $s$ lifts, uniquely, to a map
on process terms, $\widehat{s} : \Proc \rightarrow \Proc$ by the
following equations.

\begin{mathpar}
  (0) \psubstp{Q}{P} := 0 \\
  (R \juxtap S) \psubstp{Q}{P}
  :=    
  (R)\psubstp{Q}{P} \juxtap (S) \psubstp{Q}{P} \\
  (x?(y).R) \psubstp{Q}{P}    
  :=    
  (x)\substp{Q}{P} (z)\concat( (R \psubstn{z}{y}) \psubstp{Q}{P} ) \\
  (\lift{x}{R}) \psubstp{Q}{P}  
  :=
  \lift{(x)\substp{Q}{P}}{ R \psubstp{Q}{P} } \\
%   (\dropn{x})  \psubstp{Q}{P}       
%   := 
%   \left\{ 
%     \begin{array}{ccc} 
%       \dropn{\quotep{Q}} & & x \nameeq \quotep{P} \\
%       \dropn{x} & & otherwise \\
%     \end{array}
%   \right. 
  (\dropn{x})  \psubstp{Q}{P}       
  := 
  \left\{ 
    \begin{array}{ccc} 
      Q & & x \nameeq \quotep{P} \\
      \dropn{x} & & otherwise \\
    \end{array}
  \right.
\end{mathpar}
 

where

\begin{eqnarray}
  (x)\id{\{} \lpquote Q \rpquote / \lpquote P \rpquote \id{\}}            = 
  \left\{ 
    \begin{array}{ccc}
      \lpquote Q \rpquote & & x \nameeq \lpquote P \rpquote \\
      x & & otherwise \\
    \end{array}
  \right. \nonumber
\end{eqnarray}

and $z$ is chosen distinct from $\quotep{P}$, $\quotep{Q}$, the free
names in $Q$, and all the names in $R$. Our $\alpha$-equivalence will
be built in the standard way from this substitution.

\begin{remark}\label{rem:no_self_referential_names}
  One consequence of these definitions is that $\forall P. \quotep{P}
  \not\in \freenames{P}$.
\end{remark}

\subsection{ Dynamic quote: an example }

Anticipating something of what's to come, consider applying the
substitution, $\widehat{\id{\{}u / z \id{\}}}$, to the following pair
of processes, $\lift{w}{y!(z)}$ and $w[ \lpquote y!(z) \rpquote ]$.

\begin{eqnarray}
	\lift{w}{y!(z)}\widehat{\id{\{}u / z \id{\}}}
		& = &
		\lift{w}{y!(u)} \nonumber\\
	w[ \lpquote y!(z) \rpquote ] \widehat{ \id{\{}u / z \id{\}} }
		& = &
		w[ \lpquote y!(z) \rpquote ] \nonumber
\end{eqnarray}

Because the body of the process between quotes is impervious to
substitution, we get radically different answers. In fact, by
examining the first process in an input context,
e.g. $x?(z).\lift{w}{y!(z)}$, we see that the process under the lift
operator may be shaped by prefixed inputs binding a name inside it. In
this sense, the lift operator will be seen as a way to dynamically
construct processes before reifying them as names.

Finally equipped with these standard features we can present the
dynamics of the calculus.

\subsubsection{Operational semantics} 

Finally, we introduce the computational dynamics. What marks these
algebras as distinct from other more traditionally studied algebraic
structures, e.g. vector spaces or polynomial rings, is the manner in
which dynamics is captured. In traditional structures, dynamics is typically
expressed through morphisms between such structures, as in linear maps
between vector spaces or morphisms between rings. In algebras
associated with the semantics of computation, the dynamics is
expressed as part of the algebraic structure itself, through a
reduction reduction relation typically denoted by $\red$. Below, we
give a recursive presentation of this relation for the calculus used
in the encoding.

$\red \subseteq \pi \times \pi$
$\red : \pi \to \mathcal{P}(\pi)$

\begin{mathpar}
  \inferrule* [lab=Comm] { \textsf{match}( x_{src}, x_{trgt} ) } { x_{trgt}?(y)P \; | \; x_{src}!\langle {Q} \rangle \red P\{\quotep{Q}/y}\} }
  \and \\
  \inferrule* [lab=Par] {{P} \red {P}'} {{{P} | {Q}} \red {{P}' | {Q}}}
  \and
  \inferrule* [lab=Equiv]{{{P} \scong {P}'} \andalso {{P}' \red {Q}'} \andalso {{Q}' \scong {Q}}}{{P} \red {Q}}
\end{mathpar}

\begin{eqnarray*}
  match_{\equiv} (\quotep{P},\quotep{Q}) & := & P \equiv Q \\
  match_{\dagger}(\quotep{P},\quotep{Q}) & := & \forall R. P|Q \red^{*} R => R \red^{*} 0 \\
  match_{K}(\quotep{P},\quotep{Q}) & := & K \mbox{ for some context } K
\end{eqnarray*}

$u?(x)P | u!\langle Q \rangle \red P\{\quotep{Q}/x\}$

%We write $\wred$ for $\red^*$, and $P\red$ if $\exists Q $ such that $ P \red Q$.
We write $P\red$ if $\exists Q $ such that $ P \red Q$ and $P\not\red$, otherwise.

\section{Replication}

As mentioned before, it is known that replication (and hence
recursion) can be implemented in a higher-order process algebra
\cite{SangiorgiWalker}. As our first example of calculation with the
machinery thus far presented we give the construction explicitly in
the {\rhoc}.

\begin{eqnarray}
	D_{x} & := & \prefix{x}{y}{(\binpar{\outputp{x}{y}}{@{y}})} \nonumber\\
	\bangp_{x}{P} & := & \binpar{{x}!\langle{\binpar{D_{x}}{P}}\rangle}{D_{x}} \nonumber
\end{eqnarray}

\begin{eqnarray}
	\bangp_{x}{P} & & \nonumber\\
	=
	& {x}!\langle{(\prefix{x}{y}{(\outputp{x}{y} | @{y})) | P}}\rangle 
	      | \prefix{x}{y}{(\outputp{x}{y} | @{y})} & \nonumber\\
	\red
	& (\outputp{x}{y} | @{y})\substn{\quotep{(\prefix{x}{y}{(@{y} | \outputp{x}{y})) | P}}}{y} & \nonumber\\
	=
	& \outputp{x}{\quotep{(\prefix{x}{y}{(\outputp{x}{y} | @{y})) | P}}}
	  | {(\prefix{x}{y}{(\outputp{x}{y} | @{y})) | P}} & \nonumber\\
	\red
	& \ldots & \nonumber\\
	\red^*
	& P | P | \ldots & \nonumber
\end{eqnarray}

Of course, this encoding, as an implementation, runs away, unfolding
$\bangp{P}$ eagerly. A lazier and more implementable replication
operator, restricted to input-guarded processes, may be obtained as follows.

\begin{eqnarray}
\bangp{\prefix{u}{v}{P}} 
	:= 
	\binpar{\lift{x}{\prefix{u}{v}{(\binpar{D(x)}{P})}}}{D(x)} \nonumber
\end{eqnarray}

\begin{remark}
  Note that the lazier definition still does not deal with summation
  or mixed summation (i.e. sums over input and output). The reader is
  invited to construct definitions of replication that deal with these
  features. 

  Further, the definitions are parameterized in a name, $x$. Can you,
  gentle reader, make a definition that eliminates this parameter and
  guarantees no accidental interaction between the replication
  machinery and the process being replicated -- i.e. no accidental
  sharing of names used by the process to get its work done and the
  name(s) used by the replication to effect copying. This latter
  revision of the definition of replication is crucial to obtaining
  the expected identity $!!P \sim !P$.
\end{remark}

\begin{remark}\label{rem:paradoxical_combinator}
  The reader familiar with the lambda calculus will have noticed the
  similarity between $D$ and the paradoxical combinator.

  [Ed. note: the existence of this seems to suggest we have to be more
  restrictive on the set of processes and names we admit if we are to
  support no-cloning.]
\end{remark}

\subsubsection{Bisimulation}

The computational dynamics gives rise to another kind of equivalence,
the equivalence of computational behavior. As previously mentioned
this is typically captured \emph{via} some form of bisimulation.

% The notion we use in this paper is weak barbed bisimulation
% \cite{milner91polyadicpi}.

The notion we use in this paper is derived from weak barbed
bisimulation \cite{milner91polyadicpi}. 

\begin{definition}
An \emph{observation relation}, $\downarrow_{\mathcal N}$, over a set
of names, $\mathcal N$, is the smallest relation satisfying the rules
below.

\infrule[Out-barb]{y \in {\mathcal N}, \; x \nameeq y}
		  {\outputp{x}{v} \downarrow_{\mathcal N} x}
\infrule[Par-barb]{\mbox{$P\downarrow_{\mathcal N} x$ or $Q\downarrow_{\mathcal N} x$}}
		  {\binpar{P}{Q} \downarrow_{\mathcal N} x}

We write $P \Downarrow_{\mathcal N} x$ if there is $Q$ such that 
$P \wred Q$ and $Q \downarrow_{\mathcal N} x$.
\end{definition}

\begin{definition}
%\label{def.bbisim}
An  ${\mathcal N}$-\emph{barbed bisimulation} over a set of names, ${\mathcal N}$, is a symmetric binary relation 
${\mathcal S}_{\mathcal N}$ between agents such that $P\rel{S}_{\mathcal N}Q$ implies:
\begin{enumerate}
\item If $P \red P'$ then $Q \wred Q'$ and $P'\rel{S}_{\mathcal N} Q'$.
\item If $P\downarrow_{\mathcal N} x$, then $Q\Downarrow_{\mathcal N} x$.
\end{enumerate}
$P$ is ${\mathcal N}$-barbed bisimilar to $Q$, written
$P \wbbisim_{\mathcal N} Q$, if $P \rel{S}_{\mathcal N} Q$ for some ${\mathcal N}$-barbed bisimulation ${\mathcal S}_{\mathcal N}$.
\end{definition}

$\mathcal{R} \subseteq \pi \times \pi$

$P \mathcal{R} Q => \forall P'. P \red P' \Rightarrow \exists Q'. Q \red Q', P' \mathcal{R} Q'$

$P \vdash x \Rightarrow Q \vdash x$

\begin{mathpar}
  \inferrule*[lab=Out-barb]{x \nameeq y}{{y}!\langle{Q}\rangle \vdash x}
  \and
  \inferrule*[lab=Par-barb]{\mbox{$P\vdash x$ or $Q\vdash x$}}{\binpar{P}{Q} \vdash x}
\end{mathpar}

\subsubsection{Contexts}

One of the principle advantages of computational calculi like the
$\pi$-calculus is a well-defined notion of context,
contextual-equivalence and a correlation between
contextual-equivalence and notions of bisimulation. The notion of
context allows the decomposition of a process into (sub-)process and
its syntactic environment, its context. Thus, a context may be
thought of as a process with a ``hole'' (written $\Box$) in it. The
application of a context $M$ to a process $P$, written $M[P]$, is
tantamount to filling the hole in $M$ with $P$. In this paper we do
not need the full weight of this theory, but do make use of the notion
of context in the proof the main theorem. 

\begin{mathpar}
  \inferrule* [lab=summation] {} {{M_{M},M_{N}} \bc \Box \;|\; x.M_{A} \;|\; M_{M}+M_{N}}
  \and
  \inferrule* [lab=agent] {} {{M_{A}} \bc (\vec{x})M_{P} \;| \; \clift{P_0,\ldots,M_{P},\ldots,P_N}}
  \and \\
  \inferrule* [lab=process] {} {{M_{P}} \bc M_{N} \;| \;P|M_{P} }
\end{mathpar} 

\begin{mathpar}
  \inferrule* [lab=sychronization] {} {M_{N} \bc \Box \;|\; x?M_{F} \;|\; x!M_{C}}
  \and
  \inferrule* [lab=abstraction] {} {{M_{F}} \bc (x)M_{P} }
  \and
  \inferrule* [lab=concretion] {} {{M_{C}} \bc \langle M_{P} \rangle }
  \and \\
  \inferrule* [lab=process] {} {{M_{P}} \bc M_{N} \;| \;P|M_{P} }
\end{mathpar}

\begin{definition}[contextual application] Given a context $M$, and
  process $P$, we define the \emph{contextual application}, $M[P] :=
  M\{P/\Box\}$. That is, the contextual application of M to P is the
  substitution of $P$ for $\Box$ in $M$.
\end{definition}

$\meaningof{-} : L \to \mathcal{P}(\pi)$

\begin{mathpar}
  \inferrule* [lab=collection] {} {\meaningof{true} = \pi, \and \meaningof{~E} = \pi \setminus \meaningof{E}, \and \meaningof{E_{1} \& E_{2}} = \meaningof{E_{1}} \cap \meaningof{E_{2}}}
\end{mathpar}

\begin{mathpar}
  \inferrule* [lab=structure] {} {\meaningof{0} = \{ P \in \pi | P \equiv 0 \}, \and \\ \meaningof{E_1 | E_2} = \{ P \in \pi | P \equiv P_{1} | P_{2}, P_{1} \in \meaningof{E_{1}}, P_{2} \in \meaningof{E_2}\} }
\end{mathpar}

\begin{mathpar}
 \inferrule* [lab=behavior] {} {\meaningof{\langle a?b \rangle E} = \{ P \in \pi | P \equiv Q | u?(y)P', \\ \and \\\\ \and \\ \;\;\; u \in \meaningof{a}, \forall z.P'\{z/y\} \in \meaningof{E\{z/b\}}\}, \and \\ \meaningof{a!E} = \{ P \in \pi | P \equiv Q | x!\langle P' \rangle, x \in \meaningof{a} P' \in \meaningof{E}\} }
\end{mathpar}

\begin{mathpar}
 \inferrule* [lab=nominal] {} {\meaningof{\quotep{E}} = \{ \quotep{P} \in \quotep{\pi} | P \in \meaningof{E} \}, \and \meaningof{\quotep{P}} = \{ \quotep{Q} \in \quotep{\pi} | P \equiv Q \} \and \\ \meaningof{@\quotep{E}} = \{ P \in \pi | P \equiv @x, x \in \meaningof{E} \}}
\end{mathpar}

\begin{eqnarray*}
  \\
  \meaningof{-} : TS \to ST
\end{eqnarray*}

\begin{eqnarray*}
  \\
  L : TS \to ST
\end{eqnarray*}

\begin{eqnarray*}
  \\
  P \models E \iff P \in \meaningof{E}
\end{eqnarray*}

\begin{eqnarray*}
  P \approx_{L} Q \iff \forall E \in L. P \models E \iff Q \models E
\end{eqnarray*}

\begin{eqnarray*}
  P \approx_{K} Q
\end{eqnarray*}

\begin{eqnarray*}
  P \approx Q
\end{eqnarray*}

$\approx_{K} = \approx = \approx_{L}$

\subsubsection{Contextual duality}

Note that contexts extend the quotation operation to a family of
operations from processes to names. Given a context, $M$, we can
define a \emph{nominal context}, $\quotep{M}$ by $\quotep{M}[P] :=
\quotep{M[P]}$. To foreshadow what is to come we observe that these
operations enjoy a duality with processes very much like the duality
between vectors and maps from vectors to scalars.

Further, because the calculus is essentially higher-order, we have a
correspondence between contexts and processes. More specifically,
given a name $x$ and a context $M$ we can construct $M^{*}_{x}$ such
that 

\begin{mathpar}
  M^{*}_{x} | \lift{x}{P} \red M[P]
\end{mathpar}

namely,

\begin{mathpar}
  M^{*}_{x} := x?(u).M[\dropn{u}]
\end{mathpar}

The dependence of $M^{*}_{x}$ on a name makes it an abstraction, 

\begin{mathpar}
  M^{*} := (x)x?(u).M[\dropn{u}]
\end{mathpar}

\subsection{Additional notation}

It will sometimes be convenient to denote the process a name
quotes. We already have the notation $x = \quotep{P}$, but it will be
convenient to introduce an alternate notation, $\procn{x}$, when we
want to emphasize the connection to the use of the name. Note that, by
virtue of name equivalence, $\quotep{\procn{x}} \nameeq x$; so, the
notation is consistent with previous definitions.

Further, because names have structure it is possible to effect
substitutions on the basis of that structure. This means we need to
upgrade our notation for substitutions, which we accomplish by
adapting comprehension notation. Thus,

\begin{mathpar}
  P\{ y / x : x \in S \}
\end{mathpar}

is interpreted to mean the process derived from P by replacing (in a
capture-avoiding manner) each occurrence of $x$ in $S$ by $y$. For example,

\begin{mathpar}
  P\{ \quotep{\procn{x}|\procn{x}} / x : x \in \freenames{P} \}
\end{mathpar}

will replace each (occurrence) of a free name $x$ in $P$ by
$\quotep{\procn{x}|\procn{x}}$.

Also, we will avail ourselves of the notation $x^{L}$ and $x^{R}$ to
denote injections of a name into disjoint copies of the name
space. There are numerous ways to accomplish this. One example can be
found in \cite{MeredithR05}. This notation overloads to vectors of
names: $\vec{x}^{\pi} := (x_{i}^{\pi} \; : \; 0 \leq i < |\vec{x}| )$ where $\pi \in \{L,R\}$.

We also use $P^{\Box} := P|\Box$.

In \cite{MeredithR05} an interpretation of the new operator is
given. It turns out that there are several possible interpretations
all enjoying the requisite algebraic properties of the operator (see
\cite{milner91polyadicpi}). We will therefore make liberal use of
$(\nu\; \vec{x})P$.

% subsection the_syntax_and_semantics_of_the_notation_system (end)   

\input{qm2pi.qmops} 

\input{qm2pi.sterngerlach} 

\input{qm2pi.metric} 

% section concurrent_process_calculi (end)

%\input{qm2pi.proofsketch}

% section proof sketch (end)

%\input{qm2pi.slviaknots} 

% section spatial logic via knots (end)

\input{qm2pi.conclusion}

% section conclusion (end)

%\input{qm2pi.dtcodes} 

% section wiring algorithm (end)

\input{qm2pi.ack} 

% section acknowledgments (end)

\newpage


\bibliographystyle{plain}   
\bibliography{../../biblios/main.bib}

\input{qm2pi.rhodetails}

\end{document}

 

% section notation (end)

\input{qm2pi.process.calculi} 

% section concurrent_process_calculi_and_spatial_logics_ (end)
    
%\documentclass[12pt]{llncs}
%\documentclass{jktr}

\usepackage[pdftex]{hyperref}                   
\usepackage {listings}
\usepackage {mathpartir}
\usepackage{bcprules}
%\usepackage{listings}
                       
\usepackage{graphicx} 
%\usepackage[margins=2.5cm,nohead,nofoot]{geometry}
%\usepackage{geometry}
\usepackage{amsfonts}
\usepackage{amstext}
\usepackage{latexsym}
\usepackage{amssymb}
\usepackage{color}


%\include{myPreamble}
\include{qm2pi.local} 

%\ifpdf
%\usepackage[pdftex]{graphicx}
%\else
%\usepackage{graphicx}
%\fi

 % \ifpdf
%  \usepackage{pdfsync}
%  \if


%\title{Brief Article}
%\author{David F. Snyder}
%\author{L.G. Meredith}

%\address{Dept. of Math., Texas State University--San Marcos, San Marcos, TX 78666}
       
\pagestyle{empty}


\begin{document}

\lstset{language=[Objective]Caml,frame=shadowbox}

\input{qm2pi.front}

% section front matter (end)

\input{qm2pi.intro} 
 
% section introduction (end)

% \input{qm2pi.knotations} 

% section notation (end)

\input{qm2pi.process.calculi} 

% section concurrent_process_calculi_and_spatial_logics_ (end)
    
%\input{qm2pi.knots2pi} 

%\input{qm2pi.trefoil} 

%\input{qm2pi.mainthm} 

% subsection basic_interpretation (end)

%\input{qm2pi.rho.presentation} 
\subsection{The syntax and semantics of the notation system}\label{sub:the_syntax_and_semantics_of_the_notation_system} % (fold)

We now summarize a technical presentation of the calculus that
embodies our theory of dynamics. The typical presentation of such a
calculus follows the style of giving generators and relations on
them. The grammar, below, describing term constructors, freely
generates the set of processes, $\Proc$. This set is then quotiented
by a relation known as structural congruence and it is over this set
that the notion of dynamics is expressed. This presentation is
essentially that of \cite{MeredithR05} with the addition of
polyadicity and summation. For readability we have relegated some of
the technical subtleties to an appendix.

\subsubsection{Process grammar}\label{subsub:process_grammar}

\begin{mathpar}
  \inferrule* [lab=synchronization] {} {{M} \bc \pzero \;|\; x?F \;|\; x!C }
  \and
  \inferrule* [lab=abstraction] {} {{F} \bc (x)P}
  \and
  \inferrule* [lab=concretion] {} {{C} \bc \langle Q \rangle}
  \and
  \inferrule* [lab=process] {} {{P,Q} \bc M \;| \;P|Q \;|\; @{x}}
  \and
  \inferrule* [lab=name] {} {{x} \bc \quotep{P}}
\end{mathpar} 

Note that $\vec{x}$ (resp. $\vec{P}$) denotes a vector of names
(resp. processes) of length $|\vec{x}|$ (resp. $|\vec{P}|$). We adopt
the following useful abbreviations.

\begin{mathpar}
   x?(\vec{y}).P := x.(\vec{y})P \and  x\clift{\vec{P}} := x.\clift{\vec{P}}
   \and x!(y) := \lift{x}{\dropn{y}}
   \and \Pi_{i=0}^{n-1}P_i := P_0 | \ldots | P_{n-1}
\end{mathpar}

\subsubsection{Structural congruence}

\paragraph{Free and bound names and alpha-equivalence.} At the
core of structural equivalence is alpha-equivalence which identifies
process that are the same up to a change of variable. Formally, we
recognize the distinction between free and bound names. The free names
of a process, $\freenames{P}$, may be calculated recursively as
follows:

\begin{mathpar}
\freenames{\pzero} := \emptyset
  \and \\
  \freenames{x?(y).P} := \{ x \} \cup (\freenames{P} \setminus \{ y \})
  \and 
  \freenames{x!\langle P \rangle} := \{ x \} \cup \{ P \} 
  \and \\
  \freenames{P|Q} := \freenames{P} \cup \freenames{Q}
  \and \\
  \freenames{@{x}} := \{ x \}
\end{mathpar}

$\pi$
$\quotep{\pi}$

$\freenames{-} : \pi \to \mathcal{P}(\quotep{\pi})$

\begin{eqnarray*}
  \freenames{\pzero} & := & \emptyset \\
  \freenames{x?(y).P} & := & \{ x \} \cup (\freenames{P} \setminus \{ y \}) \\
  \freenames{x!\langle P \rangle} & := & \{ x \} \cup \{ P \} \\
  \freenames{P|Q} & := & \freenames{P} \cup \freenames{Q} \\
  \freenames{\dropn{x}} & := & \{ x \}
\end{eqnarray*}

The bound names of a process, $\boundnames{P}$, are those names occurring in $P$
that are not free. For example, in $x?(y).0$, the name $x$ is free, while $y$ is bound.

\begin{mathpar}
  \inferrule* [lab=monoidal-laws] {} { P|Q \equiv Q|P \and P|0 \equiv P \and P|(Q|R) \equiv (P|Q)|R }
\end{mathpar}

\begin{mathpar}
  \inferrule* [lab=alpha-equivalence] {} { (x)P \equiv (y)P\{y/x\} \and y \not\in \freenames{P} }
\end{mathpar}

\begin{definition}
Then two processes, $P,Q$, are alpha-equivalent if $P = Q\{\vec{y}/\vec{x}\}$ for
some $\vec{x} \in \boundnames{Q},\vec{y} \in \boundnames{P}$, where $Q\{\vec{y}/\vec{x}\}$
denotes the capture-avoiding substitution of $\vec{y}$ for $\vec{x}$ in $Q$.
\end{definition}

\begin{definition}
  The {\em structural congruence} \cite{SangiorgiWalker} , $\equiv$,
  between processes is the least congruence containing
  alpha-equivalence, satisfying the abelian monoid laws
  (associativity, commutativity and $\pzero$ as identity) for parallel
  composition $|$ and for summation $+$.
\end{definition}

\subsection{Name equivalence}

We take name equivalence, written $\nameeq$, to be the smallest
equivalence relation generated by the following rules.

\begin{mathpar}
\inferrule*[lab=Quote-drop]
{ }
{ \quotep{@{x}} \nameeq x }

\inferrule*[lab=Struct-equiv]
{ P \scong Q }
{ \quotep{P} \nameeq \quotep{Q} }
\end{mathpar}

The astute reader will have noticed that the mutual recursion of names
and processes imposes a mutual recursion on alpha-equivalence and
structural equivalence via name-equivalence. Fortunately, all of this
works out pleasantly and we may calculate in the natural way, free of
concern. The reader interested in the details is referred to the
appendix \ref{appendix:rho_details}.

\subsection{Substitution}

We use $\Proc$ for the set of processes, $\QProc$ for the set of
names, and $\id{\{}\vec{y} / \vec{x} \id{\}}$ to denote partial maps,
$s : \QProc \rightarrow \QProc$. A map, $s$ lifts, uniquely, to a map
on process terms, $\widehat{s} : \Proc \rightarrow \Proc$ by the
following equations.

\begin{mathpar}
  (0) \psubstp{Q}{P} := 0 \\
  (R \juxtap S) \psubstp{Q}{P}
  :=    
  (R)\psubstp{Q}{P} \juxtap (S) \psubstp{Q}{P} \\
  (x?(y).R) \psubstp{Q}{P}    
  :=    
  (x)\substp{Q}{P} (z)\concat( (R \psubstn{z}{y}) \psubstp{Q}{P} ) \\
  (\lift{x}{R}) \psubstp{Q}{P}  
  :=
  \lift{(x)\substp{Q}{P}}{ R \psubstp{Q}{P} } \\
%   (\dropn{x})  \psubstp{Q}{P}       
%   := 
%   \left\{ 
%     \begin{array}{ccc} 
%       \dropn{\quotep{Q}} & & x \nameeq \quotep{P} \\
%       \dropn{x} & & otherwise \\
%     \end{array}
%   \right. 
  (\dropn{x})  \psubstp{Q}{P}       
  := 
  \left\{ 
    \begin{array}{ccc} 
      Q & & x \nameeq \quotep{P} \\
      \dropn{x} & & otherwise \\
    \end{array}
  \right.
\end{mathpar}
 

where

\begin{eqnarray}
  (x)\id{\{} \lpquote Q \rpquote / \lpquote P \rpquote \id{\}}            = 
  \left\{ 
    \begin{array}{ccc}
      \lpquote Q \rpquote & & x \nameeq \lpquote P \rpquote \\
      x & & otherwise \\
    \end{array}
  \right. \nonumber
\end{eqnarray}

and $z$ is chosen distinct from $\quotep{P}$, $\quotep{Q}$, the free
names in $Q$, and all the names in $R$. Our $\alpha$-equivalence will
be built in the standard way from this substitution.

\begin{remark}\label{rem:no_self_referential_names}
  One consequence of these definitions is that $\forall P. \quotep{P}
  \not\in \freenames{P}$.
\end{remark}

\subsection{ Dynamic quote: an example }

Anticipating something of what's to come, consider applying the
substitution, $\widehat{\id{\{}u / z \id{\}}}$, to the following pair
of processes, $\lift{w}{y!(z)}$ and $w[ \lpquote y!(z) \rpquote ]$.

\begin{eqnarray}
	\lift{w}{y!(z)}\widehat{\id{\{}u / z \id{\}}}
		& = &
		\lift{w}{y!(u)} \nonumber\\
	w[ \lpquote y!(z) \rpquote ] \widehat{ \id{\{}u / z \id{\}} }
		& = &
		w[ \lpquote y!(z) \rpquote ] \nonumber
\end{eqnarray}

Because the body of the process between quotes is impervious to
substitution, we get radically different answers. In fact, by
examining the first process in an input context,
e.g. $x?(z).\lift{w}{y!(z)}$, we see that the process under the lift
operator may be shaped by prefixed inputs binding a name inside it. In
this sense, the lift operator will be seen as a way to dynamically
construct processes before reifying them as names.

Finally equipped with these standard features we can present the
dynamics of the calculus.

\subsubsection{Operational semantics} 

Finally, we introduce the computational dynamics. What marks these
algebras as distinct from other more traditionally studied algebraic
structures, e.g. vector spaces or polynomial rings, is the manner in
which dynamics is captured. In traditional structures, dynamics is typically
expressed through morphisms between such structures, as in linear maps
between vector spaces or morphisms between rings. In algebras
associated with the semantics of computation, the dynamics is
expressed as part of the algebraic structure itself, through a
reduction reduction relation typically denoted by $\red$. Below, we
give a recursive presentation of this relation for the calculus used
in the encoding.

$\red \subseteq \pi \times \pi$
$\red : \pi \to \mathcal{P}(\pi)$

\begin{mathpar}
  \inferrule* [lab=Comm] { \textsf{match}( x_{src}, x_{trgt} ) } { x_{trgt}?(y)P \; | \; x_{src}!\langle {Q} \rangle \red P\{\quotep{Q}/y}\} }
  \and \\
  \inferrule* [lab=Par] {{P} \red {P}'} {{{P} | {Q}} \red {{P}' | {Q}}}
  \and
  \inferrule* [lab=Equiv]{{{P} \scong {P}'} \andalso {{P}' \red {Q}'} \andalso {{Q}' \scong {Q}}}{{P} \red {Q}}
\end{mathpar}

\begin{eqnarray*}
  match_{\equiv} (\quotep{P},\quotep{Q}) & := & P \equiv Q \\
  match_{\dagger}(\quotep{P},\quotep{Q}) & := & \forall R. P|Q \red^{*} R => R \red^{*} 0 \\
  match_{K}(\quotep{P},\quotep{Q}) & := & K \mbox{ for some context } K
\end{eqnarray*}

$u?(x)P | u!\langle Q \rangle \red P\{\quotep{Q}/x\}$

%We write $\wred$ for $\red^*$, and $P\red$ if $\exists Q $ such that $ P \red Q$.
We write $P\red$ if $\exists Q $ such that $ P \red Q$ and $P\not\red$, otherwise.

\section{Replication}

As mentioned before, it is known that replication (and hence
recursion) can be implemented in a higher-order process algebra
\cite{SangiorgiWalker}. As our first example of calculation with the
machinery thus far presented we give the construction explicitly in
the {\rhoc}.

\begin{eqnarray}
	D_{x} & := & \prefix{x}{y}{(\binpar{\outputp{x}{y}}{@{y}})} \nonumber\\
	\bangp_{x}{P} & := & \binpar{{x}!\langle{\binpar{D_{x}}{P}}\rangle}{D_{x}} \nonumber
\end{eqnarray}

\begin{eqnarray}
	\bangp_{x}{P} & & \nonumber\\
	=
	& {x}!\langle{(\prefix{x}{y}{(\outputp{x}{y} | @{y})) | P}}\rangle 
	      | \prefix{x}{y}{(\outputp{x}{y} | @{y})} & \nonumber\\
	\red
	& (\outputp{x}{y} | @{y})\substn{\quotep{(\prefix{x}{y}{(@{y} | \outputp{x}{y})) | P}}}{y} & \nonumber\\
	=
	& \outputp{x}{\quotep{(\prefix{x}{y}{(\outputp{x}{y} | @{y})) | P}}}
	  | {(\prefix{x}{y}{(\outputp{x}{y} | @{y})) | P}} & \nonumber\\
	\red
	& \ldots & \nonumber\\
	\red^*
	& P | P | \ldots & \nonumber
\end{eqnarray}

Of course, this encoding, as an implementation, runs away, unfolding
$\bangp{P}$ eagerly. A lazier and more implementable replication
operator, restricted to input-guarded processes, may be obtained as follows.

\begin{eqnarray}
\bangp{\prefix{u}{v}{P}} 
	:= 
	\binpar{\lift{x}{\prefix{u}{v}{(\binpar{D(x)}{P})}}}{D(x)} \nonumber
\end{eqnarray}

\begin{remark}
  Note that the lazier definition still does not deal with summation
  or mixed summation (i.e. sums over input and output). The reader is
  invited to construct definitions of replication that deal with these
  features. 

  Further, the definitions are parameterized in a name, $x$. Can you,
  gentle reader, make a definition that eliminates this parameter and
  guarantees no accidental interaction between the replication
  machinery and the process being replicated -- i.e. no accidental
  sharing of names used by the process to get its work done and the
  name(s) used by the replication to effect copying. This latter
  revision of the definition of replication is crucial to obtaining
  the expected identity $!!P \sim !P$.
\end{remark}

\begin{remark}\label{rem:paradoxical_combinator}
  The reader familiar with the lambda calculus will have noticed the
  similarity between $D$ and the paradoxical combinator.

  [Ed. note: the existence of this seems to suggest we have to be more
  restrictive on the set of processes and names we admit if we are to
  support no-cloning.]
\end{remark}

\subsubsection{Bisimulation}

The computational dynamics gives rise to another kind of equivalence,
the equivalence of computational behavior. As previously mentioned
this is typically captured \emph{via} some form of bisimulation.

% The notion we use in this paper is weak barbed bisimulation
% \cite{milner91polyadicpi}.

The notion we use in this paper is derived from weak barbed
bisimulation \cite{milner91polyadicpi}. 

\begin{definition}
An \emph{observation relation}, $\downarrow_{\mathcal N}$, over a set
of names, $\mathcal N$, is the smallest relation satisfying the rules
below.

\infrule[Out-barb]{y \in {\mathcal N}, \; x \nameeq y}
		  {\outputp{x}{v} \downarrow_{\mathcal N} x}
\infrule[Par-barb]{\mbox{$P\downarrow_{\mathcal N} x$ or $Q\downarrow_{\mathcal N} x$}}
		  {\binpar{P}{Q} \downarrow_{\mathcal N} x}

We write $P \Downarrow_{\mathcal N} x$ if there is $Q$ such that 
$P \wred Q$ and $Q \downarrow_{\mathcal N} x$.
\end{definition}

\begin{definition}
%\label{def.bbisim}
An  ${\mathcal N}$-\emph{barbed bisimulation} over a set of names, ${\mathcal N}$, is a symmetric binary relation 
${\mathcal S}_{\mathcal N}$ between agents such that $P\rel{S}_{\mathcal N}Q$ implies:
\begin{enumerate}
\item If $P \red P'$ then $Q \wred Q'$ and $P'\rel{S}_{\mathcal N} Q'$.
\item If $P\downarrow_{\mathcal N} x$, then $Q\Downarrow_{\mathcal N} x$.
\end{enumerate}
$P$ is ${\mathcal N}$-barbed bisimilar to $Q$, written
$P \wbbisim_{\mathcal N} Q$, if $P \rel{S}_{\mathcal N} Q$ for some ${\mathcal N}$-barbed bisimulation ${\mathcal S}_{\mathcal N}$.
\end{definition}

$\mathcal{R} \subseteq \pi \times \pi$

$P \mathcal{R} Q => \forall P'. P \red P' \Rightarrow \exists Q'. Q \red Q', P' \mathcal{R} Q'$

$P \vdash x \Rightarrow Q \vdash x$

\begin{mathpar}
  \inferrule*[lab=Out-barb]{x \nameeq y}{{y}!\langle{Q}\rangle \vdash x}
  \and
  \inferrule*[lab=Par-barb]{\mbox{$P\vdash x$ or $Q\vdash x$}}{\binpar{P}{Q} \vdash x}
\end{mathpar}

\subsubsection{Contexts}

One of the principle advantages of computational calculi like the
$\pi$-calculus is a well-defined notion of context,
contextual-equivalence and a correlation between
contextual-equivalence and notions of bisimulation. The notion of
context allows the decomposition of a process into (sub-)process and
its syntactic environment, its context. Thus, a context may be
thought of as a process with a ``hole'' (written $\Box$) in it. The
application of a context $M$ to a process $P$, written $M[P]$, is
tantamount to filling the hole in $M$ with $P$. In this paper we do
not need the full weight of this theory, but do make use of the notion
of context in the proof the main theorem. 

\begin{mathpar}
  \inferrule* [lab=summation] {} {{M_{M},M_{N}} \bc \Box \;|\; x.M_{A} \;|\; M_{M}+M_{N}}
  \and
  \inferrule* [lab=agent] {} {{M_{A}} \bc (\vec{x})M_{P} \;| \; \clift{P_0,\ldots,M_{P},\ldots,P_N}}
  \and \\
  \inferrule* [lab=process] {} {{M_{P}} \bc M_{N} \;| \;P|M_{P} }
\end{mathpar} 

\begin{mathpar}
  \inferrule* [lab=sychronization] {} {M_{N} \bc \Box \;|\; x?M_{F} \;|\; x!M_{C}}
  \and
  \inferrule* [lab=abstraction] {} {{M_{F}} \bc (x)M_{P} }
  \and
  \inferrule* [lab=concretion] {} {{M_{C}} \bc \langle M_{P} \rangle }
  \and \\
  \inferrule* [lab=process] {} {{M_{P}} \bc M_{N} \;| \;P|M_{P} }
\end{mathpar}

\begin{definition}[contextual application] Given a context $M$, and
  process $P$, we define the \emph{contextual application}, $M[P] :=
  M\{P/\Box\}$. That is, the contextual application of M to P is the
  substitution of $P$ for $\Box$ in $M$.
\end{definition}

$\meaningof{-} : L \to \mathcal{P}(\pi)$

\begin{mathpar}
  \inferrule* [lab=collection] {} {\meaningof{true} = \pi, \and \meaningof{~E} = \pi \setminus \meaningof{E}, \and \meaningof{E_{1} \& E_{2}} = \meaningof{E_{1}} \cap \meaningof{E_{2}}}
\end{mathpar}

\begin{mathpar}
  \inferrule* [lab=structure] {} {\meaningof{0} = \{ P \in \pi | P \equiv 0 \}, \and \\ \meaningof{E_1 | E_2} = \{ P \in \pi | P \equiv P_{1} | P_{2}, P_{1} \in \meaningof{E_{1}}, P_{2} \in \meaningof{E_2}\} }
\end{mathpar}

\begin{mathpar}
 \inferrule* [lab=behavior] {} {\meaningof{\langle a?b \rangle E} = \{ P \in \pi | P \equiv Q | u?(y)P', \\ \and \\\\ \and \\ \;\;\; u \in \meaningof{a}, \forall z.P'\{z/y\} \in \meaningof{E\{z/b\}}\}, \and \\ \meaningof{a!E} = \{ P \in \pi | P \equiv Q | x!\langle P' \rangle, x \in \meaningof{a} P' \in \meaningof{E}\} }
\end{mathpar}

\begin{mathpar}
 \inferrule* [lab=nominal] {} {\meaningof{\quotep{E}} = \{ \quotep{P} \in \quotep{\pi} | P \in \meaningof{E} \}, \and \meaningof{\quotep{P}} = \{ \quotep{Q} \in \quotep{\pi} | P \equiv Q \} \and \\ \meaningof{@\quotep{E}} = \{ P \in \pi | P \equiv @x, x \in \meaningof{E} \}}
\end{mathpar}

\begin{eqnarray*}
  \\
  \meaningof{-} : TS \to ST
\end{eqnarray*}

\begin{eqnarray*}
  \\
  L : TS \to ST
\end{eqnarray*}

\begin{eqnarray*}
  \\
  P \models E \iff P \in \meaningof{E}
\end{eqnarray*}

\begin{eqnarray*}
  P \approx_{L} Q \iff \forall E \in L. P \models E \iff Q \models E
\end{eqnarray*}

\begin{eqnarray*}
  P \approx_{K} Q
\end{eqnarray*}

\begin{eqnarray*}
  P \approx Q
\end{eqnarray*}

$\approx_{K} = \approx = \approx_{L}$

\subsubsection{Contextual duality}

Note that contexts extend the quotation operation to a family of
operations from processes to names. Given a context, $M$, we can
define a \emph{nominal context}, $\quotep{M}$ by $\quotep{M}[P] :=
\quotep{M[P]}$. To foreshadow what is to come we observe that these
operations enjoy a duality with processes very much like the duality
between vectors and maps from vectors to scalars.

Further, because the calculus is essentially higher-order, we have a
correspondence between contexts and processes. More specifically,
given a name $x$ and a context $M$ we can construct $M^{*}_{x}$ such
that 

\begin{mathpar}
  M^{*}_{x} | \lift{x}{P} \red M[P]
\end{mathpar}

namely,

\begin{mathpar}
  M^{*}_{x} := x?(u).M[\dropn{u}]
\end{mathpar}

The dependence of $M^{*}_{x}$ on a name makes it an abstraction, 

\begin{mathpar}
  M^{*} := (x)x?(u).M[\dropn{u}]
\end{mathpar}

\subsection{Additional notation}

It will sometimes be convenient to denote the process a name
quotes. We already have the notation $x = \quotep{P}$, but it will be
convenient to introduce an alternate notation, $\procn{x}$, when we
want to emphasize the connection to the use of the name. Note that, by
virtue of name equivalence, $\quotep{\procn{x}} \nameeq x$; so, the
notation is consistent with previous definitions.

Further, because names have structure it is possible to effect
substitutions on the basis of that structure. This means we need to
upgrade our notation for substitutions, which we accomplish by
adapting comprehension notation. Thus,

\begin{mathpar}
  P\{ y / x : x \in S \}
\end{mathpar}

is interpreted to mean the process derived from P by replacing (in a
capture-avoiding manner) each occurrence of $x$ in $S$ by $y$. For example,

\begin{mathpar}
  P\{ \quotep{\procn{x}|\procn{x}} / x : x \in \freenames{P} \}
\end{mathpar}

will replace each (occurrence) of a free name $x$ in $P$ by
$\quotep{\procn{x}|\procn{x}}$.

Also, we will avail ourselves of the notation $x^{L}$ and $x^{R}$ to
denote injections of a name into disjoint copies of the name
space. There are numerous ways to accomplish this. One example can be
found in \cite{MeredithR05}. This notation overloads to vectors of
names: $\vec{x}^{\pi} := (x_{i}^{\pi} \; : \; 0 \leq i < |\vec{x}| )$ where $\pi \in \{L,R\}$.

We also use $P^{\Box} := P|\Box$.

In \cite{MeredithR05} an interpretation of the new operator is
given. It turns out that there are several possible interpretations
all enjoying the requisite algebraic properties of the operator (see
\cite{milner91polyadicpi}). We will therefore make liberal use of
$(\nu\; \vec{x})P$.

% subsection the_syntax_and_semantics_of_the_notation_system (end)   

\input{qm2pi.qmops} 

\input{qm2pi.sterngerlach} 

\input{qm2pi.metric} 

% section concurrent_process_calculi (end)

%\input{qm2pi.proofsketch}

% section proof sketch (end)

%\input{qm2pi.slviaknots} 

% section spatial logic via knots (end)

\input{qm2pi.conclusion}

% section conclusion (end)

%\input{qm2pi.dtcodes} 

% section wiring algorithm (end)

\input{qm2pi.ack} 

% section acknowledgments (end)

\newpage


\bibliographystyle{plain}   
\bibliography{../../biblios/main.bib}

\input{qm2pi.rhodetails}

\end{document}

 

%\documentclass[12pt]{llncs}
%\documentclass{jktr}

\usepackage[pdftex]{hyperref}                   
\usepackage {listings}
\usepackage {mathpartir}
\usepackage{bcprules}
%\usepackage{listings}
                       
\usepackage{graphicx} 
%\usepackage[margins=2.5cm,nohead,nofoot]{geometry}
%\usepackage{geometry}
\usepackage{amsfonts}
\usepackage{amstext}
\usepackage{latexsym}
\usepackage{amssymb}
\usepackage{color}


%\include{myPreamble}
\include{qm2pi.local} 

%\ifpdf
%\usepackage[pdftex]{graphicx}
%\else
%\usepackage{graphicx}
%\fi

 % \ifpdf
%  \usepackage{pdfsync}
%  \if


%\title{Brief Article}
%\author{David F. Snyder}
%\author{L.G. Meredith}

%\address{Dept. of Math., Texas State University--San Marcos, San Marcos, TX 78666}
       
\pagestyle{empty}


\begin{document}

\lstset{language=[Objective]Caml,frame=shadowbox}

\input{qm2pi.front}

% section front matter (end)

\input{qm2pi.intro} 
 
% section introduction (end)

% \input{qm2pi.knotations} 

% section notation (end)

\input{qm2pi.process.calculi} 

% section concurrent_process_calculi_and_spatial_logics_ (end)
    
%\input{qm2pi.knots2pi} 

%\input{qm2pi.trefoil} 

%\input{qm2pi.mainthm} 

% subsection basic_interpretation (end)

%\input{qm2pi.rho.presentation} 
\subsection{The syntax and semantics of the notation system}\label{sub:the_syntax_and_semantics_of_the_notation_system} % (fold)

We now summarize a technical presentation of the calculus that
embodies our theory of dynamics. The typical presentation of such a
calculus follows the style of giving generators and relations on
them. The grammar, below, describing term constructors, freely
generates the set of processes, $\Proc$. This set is then quotiented
by a relation known as structural congruence and it is over this set
that the notion of dynamics is expressed. This presentation is
essentially that of \cite{MeredithR05} with the addition of
polyadicity and summation. For readability we have relegated some of
the technical subtleties to an appendix.

\subsubsection{Process grammar}\label{subsub:process_grammar}

\begin{mathpar}
  \inferrule* [lab=synchronization] {} {{M} \bc \pzero \;|\; x?F \;|\; x!C }
  \and
  \inferrule* [lab=abstraction] {} {{F} \bc (x)P}
  \and
  \inferrule* [lab=concretion] {} {{C} \bc \langle Q \rangle}
  \and
  \inferrule* [lab=process] {} {{P,Q} \bc M \;| \;P|Q \;|\; @{x}}
  \and
  \inferrule* [lab=name] {} {{x} \bc \quotep{P}}
\end{mathpar} 

Note that $\vec{x}$ (resp. $\vec{P}$) denotes a vector of names
(resp. processes) of length $|\vec{x}|$ (resp. $|\vec{P}|$). We adopt
the following useful abbreviations.

\begin{mathpar}
   x?(\vec{y}).P := x.(\vec{y})P \and  x\clift{\vec{P}} := x.\clift{\vec{P}}
   \and x!(y) := \lift{x}{\dropn{y}}
   \and \Pi_{i=0}^{n-1}P_i := P_0 | \ldots | P_{n-1}
\end{mathpar}

\subsubsection{Structural congruence}

\paragraph{Free and bound names and alpha-equivalence.} At the
core of structural equivalence is alpha-equivalence which identifies
process that are the same up to a change of variable. Formally, we
recognize the distinction between free and bound names. The free names
of a process, $\freenames{P}$, may be calculated recursively as
follows:

\begin{mathpar}
\freenames{\pzero} := \emptyset
  \and \\
  \freenames{x?(y).P} := \{ x \} \cup (\freenames{P} \setminus \{ y \})
  \and 
  \freenames{x!\langle P \rangle} := \{ x \} \cup \{ P \} 
  \and \\
  \freenames{P|Q} := \freenames{P} \cup \freenames{Q}
  \and \\
  \freenames{@{x}} := \{ x \}
\end{mathpar}

$\pi$
$\quotep{\pi}$

$\freenames{-} : \pi \to \mathcal{P}(\quotep{\pi})$

\begin{eqnarray*}
  \freenames{\pzero} & := & \emptyset \\
  \freenames{x?(y).P} & := & \{ x \} \cup (\freenames{P} \setminus \{ y \}) \\
  \freenames{x!\langle P \rangle} & := & \{ x \} \cup \{ P \} \\
  \freenames{P|Q} & := & \freenames{P} \cup \freenames{Q} \\
  \freenames{\dropn{x}} & := & \{ x \}
\end{eqnarray*}

The bound names of a process, $\boundnames{P}$, are those names occurring in $P$
that are not free. For example, in $x?(y).0$, the name $x$ is free, while $y$ is bound.

\begin{mathpar}
  \inferrule* [lab=monoidal-laws] {} { P|Q \equiv Q|P \and P|0 \equiv P \and P|(Q|R) \equiv (P|Q)|R }
\end{mathpar}

\begin{mathpar}
  \inferrule* [lab=alpha-equivalence] {} { (x)P \equiv (y)P\{y/x\} \and y \not\in \freenames{P} }
\end{mathpar}

\begin{definition}
Then two processes, $P,Q$, are alpha-equivalent if $P = Q\{\vec{y}/\vec{x}\}$ for
some $\vec{x} \in \boundnames{Q},\vec{y} \in \boundnames{P}$, where $Q\{\vec{y}/\vec{x}\}$
denotes the capture-avoiding substitution of $\vec{y}$ for $\vec{x}$ in $Q$.
\end{definition}

\begin{definition}
  The {\em structural congruence} \cite{SangiorgiWalker} , $\equiv$,
  between processes is the least congruence containing
  alpha-equivalence, satisfying the abelian monoid laws
  (associativity, commutativity and $\pzero$ as identity) for parallel
  composition $|$ and for summation $+$.
\end{definition}

\subsection{Name equivalence}

We take name equivalence, written $\nameeq$, to be the smallest
equivalence relation generated by the following rules.

\begin{mathpar}
\inferrule*[lab=Quote-drop]
{ }
{ \quotep{@{x}} \nameeq x }

\inferrule*[lab=Struct-equiv]
{ P \scong Q }
{ \quotep{P} \nameeq \quotep{Q} }
\end{mathpar}

The astute reader will have noticed that the mutual recursion of names
and processes imposes a mutual recursion on alpha-equivalence and
structural equivalence via name-equivalence. Fortunately, all of this
works out pleasantly and we may calculate in the natural way, free of
concern. The reader interested in the details is referred to the
appendix \ref{appendix:rho_details}.

\subsection{Substitution}

We use $\Proc$ for the set of processes, $\QProc$ for the set of
names, and $\id{\{}\vec{y} / \vec{x} \id{\}}$ to denote partial maps,
$s : \QProc \rightarrow \QProc$. A map, $s$ lifts, uniquely, to a map
on process terms, $\widehat{s} : \Proc \rightarrow \Proc$ by the
following equations.

\begin{mathpar}
  (0) \psubstp{Q}{P} := 0 \\
  (R \juxtap S) \psubstp{Q}{P}
  :=    
  (R)\psubstp{Q}{P} \juxtap (S) \psubstp{Q}{P} \\
  (x?(y).R) \psubstp{Q}{P}    
  :=    
  (x)\substp{Q}{P} (z)\concat( (R \psubstn{z}{y}) \psubstp{Q}{P} ) \\
  (\lift{x}{R}) \psubstp{Q}{P}  
  :=
  \lift{(x)\substp{Q}{P}}{ R \psubstp{Q}{P} } \\
%   (\dropn{x})  \psubstp{Q}{P}       
%   := 
%   \left\{ 
%     \begin{array}{ccc} 
%       \dropn{\quotep{Q}} & & x \nameeq \quotep{P} \\
%       \dropn{x} & & otherwise \\
%     \end{array}
%   \right. 
  (\dropn{x})  \psubstp{Q}{P}       
  := 
  \left\{ 
    \begin{array}{ccc} 
      Q & & x \nameeq \quotep{P} \\
      \dropn{x} & & otherwise \\
    \end{array}
  \right.
\end{mathpar}
 

where

\begin{eqnarray}
  (x)\id{\{} \lpquote Q \rpquote / \lpquote P \rpquote \id{\}}            = 
  \left\{ 
    \begin{array}{ccc}
      \lpquote Q \rpquote & & x \nameeq \lpquote P \rpquote \\
      x & & otherwise \\
    \end{array}
  \right. \nonumber
\end{eqnarray}

and $z$ is chosen distinct from $\quotep{P}$, $\quotep{Q}$, the free
names in $Q$, and all the names in $R$. Our $\alpha$-equivalence will
be built in the standard way from this substitution.

\begin{remark}\label{rem:no_self_referential_names}
  One consequence of these definitions is that $\forall P. \quotep{P}
  \not\in \freenames{P}$.
\end{remark}

\subsection{ Dynamic quote: an example }

Anticipating something of what's to come, consider applying the
substitution, $\widehat{\id{\{}u / z \id{\}}}$, to the following pair
of processes, $\lift{w}{y!(z)}$ and $w[ \lpquote y!(z) \rpquote ]$.

\begin{eqnarray}
	\lift{w}{y!(z)}\widehat{\id{\{}u / z \id{\}}}
		& = &
		\lift{w}{y!(u)} \nonumber\\
	w[ \lpquote y!(z) \rpquote ] \widehat{ \id{\{}u / z \id{\}} }
		& = &
		w[ \lpquote y!(z) \rpquote ] \nonumber
\end{eqnarray}

Because the body of the process between quotes is impervious to
substitution, we get radically different answers. In fact, by
examining the first process in an input context,
e.g. $x?(z).\lift{w}{y!(z)}$, we see that the process under the lift
operator may be shaped by prefixed inputs binding a name inside it. In
this sense, the lift operator will be seen as a way to dynamically
construct processes before reifying them as names.

Finally equipped with these standard features we can present the
dynamics of the calculus.

\subsubsection{Operational semantics} 

Finally, we introduce the computational dynamics. What marks these
algebras as distinct from other more traditionally studied algebraic
structures, e.g. vector spaces or polynomial rings, is the manner in
which dynamics is captured. In traditional structures, dynamics is typically
expressed through morphisms between such structures, as in linear maps
between vector spaces or morphisms between rings. In algebras
associated with the semantics of computation, the dynamics is
expressed as part of the algebraic structure itself, through a
reduction reduction relation typically denoted by $\red$. Below, we
give a recursive presentation of this relation for the calculus used
in the encoding.

$\red \subseteq \pi \times \pi$
$\red : \pi \to \mathcal{P}(\pi)$

\begin{mathpar}
  \inferrule* [lab=Comm] { \textsf{match}( x_{src}, x_{trgt} ) } { x_{trgt}?(y)P \; | \; x_{src}!\langle {Q} \rangle \red P\{\quotep{Q}/y}\} }
  \and \\
  \inferrule* [lab=Par] {{P} \red {P}'} {{{P} | {Q}} \red {{P}' | {Q}}}
  \and
  \inferrule* [lab=Equiv]{{{P} \scong {P}'} \andalso {{P}' \red {Q}'} \andalso {{Q}' \scong {Q}}}{{P} \red {Q}}
\end{mathpar}

\begin{eqnarray*}
  match_{\equiv} (\quotep{P},\quotep{Q}) & := & P \equiv Q \\
  match_{\dagger}(\quotep{P},\quotep{Q}) & := & \forall R. P|Q \red^{*} R => R \red^{*} 0 \\
  match_{K}(\quotep{P},\quotep{Q}) & := & K \mbox{ for some context } K
\end{eqnarray*}

$u?(x)P | u!\langle Q \rangle \red P\{\quotep{Q}/x\}$

%We write $\wred$ for $\red^*$, and $P\red$ if $\exists Q $ such that $ P \red Q$.
We write $P\red$ if $\exists Q $ such that $ P \red Q$ and $P\not\red$, otherwise.

\section{Replication}

As mentioned before, it is known that replication (and hence
recursion) can be implemented in a higher-order process algebra
\cite{SangiorgiWalker}. As our first example of calculation with the
machinery thus far presented we give the construction explicitly in
the {\rhoc}.

\begin{eqnarray}
	D_{x} & := & \prefix{x}{y}{(\binpar{\outputp{x}{y}}{@{y}})} \nonumber\\
	\bangp_{x}{P} & := & \binpar{{x}!\langle{\binpar{D_{x}}{P}}\rangle}{D_{x}} \nonumber
\end{eqnarray}

\begin{eqnarray}
	\bangp_{x}{P} & & \nonumber\\
	=
	& {x}!\langle{(\prefix{x}{y}{(\outputp{x}{y} | @{y})) | P}}\rangle 
	      | \prefix{x}{y}{(\outputp{x}{y} | @{y})} & \nonumber\\
	\red
	& (\outputp{x}{y} | @{y})\substn{\quotep{(\prefix{x}{y}{(@{y} | \outputp{x}{y})) | P}}}{y} & \nonumber\\
	=
	& \outputp{x}{\quotep{(\prefix{x}{y}{(\outputp{x}{y} | @{y})) | P}}}
	  | {(\prefix{x}{y}{(\outputp{x}{y} | @{y})) | P}} & \nonumber\\
	\red
	& \ldots & \nonumber\\
	\red^*
	& P | P | \ldots & \nonumber
\end{eqnarray}

Of course, this encoding, as an implementation, runs away, unfolding
$\bangp{P}$ eagerly. A lazier and more implementable replication
operator, restricted to input-guarded processes, may be obtained as follows.

\begin{eqnarray}
\bangp{\prefix{u}{v}{P}} 
	:= 
	\binpar{\lift{x}{\prefix{u}{v}{(\binpar{D(x)}{P})}}}{D(x)} \nonumber
\end{eqnarray}

\begin{remark}
  Note that the lazier definition still does not deal with summation
  or mixed summation (i.e. sums over input and output). The reader is
  invited to construct definitions of replication that deal with these
  features. 

  Further, the definitions are parameterized in a name, $x$. Can you,
  gentle reader, make a definition that eliminates this parameter and
  guarantees no accidental interaction between the replication
  machinery and the process being replicated -- i.e. no accidental
  sharing of names used by the process to get its work done and the
  name(s) used by the replication to effect copying. This latter
  revision of the definition of replication is crucial to obtaining
  the expected identity $!!P \sim !P$.
\end{remark}

\begin{remark}\label{rem:paradoxical_combinator}
  The reader familiar with the lambda calculus will have noticed the
  similarity between $D$ and the paradoxical combinator.

  [Ed. note: the existence of this seems to suggest we have to be more
  restrictive on the set of processes and names we admit if we are to
  support no-cloning.]
\end{remark}

\subsubsection{Bisimulation}

The computational dynamics gives rise to another kind of equivalence,
the equivalence of computational behavior. As previously mentioned
this is typically captured \emph{via} some form of bisimulation.

% The notion we use in this paper is weak barbed bisimulation
% \cite{milner91polyadicpi}.

The notion we use in this paper is derived from weak barbed
bisimulation \cite{milner91polyadicpi}. 

\begin{definition}
An \emph{observation relation}, $\downarrow_{\mathcal N}$, over a set
of names, $\mathcal N$, is the smallest relation satisfying the rules
below.

\infrule[Out-barb]{y \in {\mathcal N}, \; x \nameeq y}
		  {\outputp{x}{v} \downarrow_{\mathcal N} x}
\infrule[Par-barb]{\mbox{$P\downarrow_{\mathcal N} x$ or $Q\downarrow_{\mathcal N} x$}}
		  {\binpar{P}{Q} \downarrow_{\mathcal N} x}

We write $P \Downarrow_{\mathcal N} x$ if there is $Q$ such that 
$P \wred Q$ and $Q \downarrow_{\mathcal N} x$.
\end{definition}

\begin{definition}
%\label{def.bbisim}
An  ${\mathcal N}$-\emph{barbed bisimulation} over a set of names, ${\mathcal N}$, is a symmetric binary relation 
${\mathcal S}_{\mathcal N}$ between agents such that $P\rel{S}_{\mathcal N}Q$ implies:
\begin{enumerate}
\item If $P \red P'$ then $Q \wred Q'$ and $P'\rel{S}_{\mathcal N} Q'$.
\item If $P\downarrow_{\mathcal N} x$, then $Q\Downarrow_{\mathcal N} x$.
\end{enumerate}
$P$ is ${\mathcal N}$-barbed bisimilar to $Q$, written
$P \wbbisim_{\mathcal N} Q$, if $P \rel{S}_{\mathcal N} Q$ for some ${\mathcal N}$-barbed bisimulation ${\mathcal S}_{\mathcal N}$.
\end{definition}

$\mathcal{R} \subseteq \pi \times \pi$

$P \mathcal{R} Q => \forall P'. P \red P' \Rightarrow \exists Q'. Q \red Q', P' \mathcal{R} Q'$

$P \vdash x \Rightarrow Q \vdash x$

\begin{mathpar}
  \inferrule*[lab=Out-barb]{x \nameeq y}{{y}!\langle{Q}\rangle \vdash x}
  \and
  \inferrule*[lab=Par-barb]{\mbox{$P\vdash x$ or $Q\vdash x$}}{\binpar{P}{Q} \vdash x}
\end{mathpar}

\subsubsection{Contexts}

One of the principle advantages of computational calculi like the
$\pi$-calculus is a well-defined notion of context,
contextual-equivalence and a correlation between
contextual-equivalence and notions of bisimulation. The notion of
context allows the decomposition of a process into (sub-)process and
its syntactic environment, its context. Thus, a context may be
thought of as a process with a ``hole'' (written $\Box$) in it. The
application of a context $M$ to a process $P$, written $M[P]$, is
tantamount to filling the hole in $M$ with $P$. In this paper we do
not need the full weight of this theory, but do make use of the notion
of context in the proof the main theorem. 

\begin{mathpar}
  \inferrule* [lab=summation] {} {{M_{M},M_{N}} \bc \Box \;|\; x.M_{A} \;|\; M_{M}+M_{N}}
  \and
  \inferrule* [lab=agent] {} {{M_{A}} \bc (\vec{x})M_{P} \;| \; \clift{P_0,\ldots,M_{P},\ldots,P_N}}
  \and \\
  \inferrule* [lab=process] {} {{M_{P}} \bc M_{N} \;| \;P|M_{P} }
\end{mathpar} 

\begin{mathpar}
  \inferrule* [lab=sychronization] {} {M_{N} \bc \Box \;|\; x?M_{F} \;|\; x!M_{C}}
  \and
  \inferrule* [lab=abstraction] {} {{M_{F}} \bc (x)M_{P} }
  \and
  \inferrule* [lab=concretion] {} {{M_{C}} \bc \langle M_{P} \rangle }
  \and \\
  \inferrule* [lab=process] {} {{M_{P}} \bc M_{N} \;| \;P|M_{P} }
\end{mathpar}

\begin{definition}[contextual application] Given a context $M$, and
  process $P$, we define the \emph{contextual application}, $M[P] :=
  M\{P/\Box\}$. That is, the contextual application of M to P is the
  substitution of $P$ for $\Box$ in $M$.
\end{definition}

$\meaningof{-} : L \to \mathcal{P}(\pi)$

\begin{mathpar}
  \inferrule* [lab=collection] {} {\meaningof{true} = \pi, \and \meaningof{~E} = \pi \setminus \meaningof{E}, \and \meaningof{E_{1} \& E_{2}} = \meaningof{E_{1}} \cap \meaningof{E_{2}}}
\end{mathpar}

\begin{mathpar}
  \inferrule* [lab=structure] {} {\meaningof{0} = \{ P \in \pi | P \equiv 0 \}, \and \\ \meaningof{E_1 | E_2} = \{ P \in \pi | P \equiv P_{1} | P_{2}, P_{1} \in \meaningof{E_{1}}, P_{2} \in \meaningof{E_2}\} }
\end{mathpar}

\begin{mathpar}
 \inferrule* [lab=behavior] {} {\meaningof{\langle a?b \rangle E} = \{ P \in \pi | P \equiv Q | u?(y)P', \\ \and \\\\ \and \\ \;\;\; u \in \meaningof{a}, \forall z.P'\{z/y\} \in \meaningof{E\{z/b\}}\}, \and \\ \meaningof{a!E} = \{ P \in \pi | P \equiv Q | x!\langle P' \rangle, x \in \meaningof{a} P' \in \meaningof{E}\} }
\end{mathpar}

\begin{mathpar}
 \inferrule* [lab=nominal] {} {\meaningof{\quotep{E}} = \{ \quotep{P} \in \quotep{\pi} | P \in \meaningof{E} \}, \and \meaningof{\quotep{P}} = \{ \quotep{Q} \in \quotep{\pi} | P \equiv Q \} \and \\ \meaningof{@\quotep{E}} = \{ P \in \pi | P \equiv @x, x \in \meaningof{E} \}}
\end{mathpar}

\begin{eqnarray*}
  \\
  \meaningof{-} : TS \to ST
\end{eqnarray*}

\begin{eqnarray*}
  \\
  L : TS \to ST
\end{eqnarray*}

\begin{eqnarray*}
  \\
  P \models E \iff P \in \meaningof{E}
\end{eqnarray*}

\begin{eqnarray*}
  P \approx_{L} Q \iff \forall E \in L. P \models E \iff Q \models E
\end{eqnarray*}

\begin{eqnarray*}
  P \approx_{K} Q
\end{eqnarray*}

\begin{eqnarray*}
  P \approx Q
\end{eqnarray*}

$\approx_{K} = \approx = \approx_{L}$

\subsubsection{Contextual duality}

Note that contexts extend the quotation operation to a family of
operations from processes to names. Given a context, $M$, we can
define a \emph{nominal context}, $\quotep{M}$ by $\quotep{M}[P] :=
\quotep{M[P]}$. To foreshadow what is to come we observe that these
operations enjoy a duality with processes very much like the duality
between vectors and maps from vectors to scalars.

Further, because the calculus is essentially higher-order, we have a
correspondence between contexts and processes. More specifically,
given a name $x$ and a context $M$ we can construct $M^{*}_{x}$ such
that 

\begin{mathpar}
  M^{*}_{x} | \lift{x}{P} \red M[P]
\end{mathpar}

namely,

\begin{mathpar}
  M^{*}_{x} := x?(u).M[\dropn{u}]
\end{mathpar}

The dependence of $M^{*}_{x}$ on a name makes it an abstraction, 

\begin{mathpar}
  M^{*} := (x)x?(u).M[\dropn{u}]
\end{mathpar}

\subsection{Additional notation}

It will sometimes be convenient to denote the process a name
quotes. We already have the notation $x = \quotep{P}$, but it will be
convenient to introduce an alternate notation, $\procn{x}$, when we
want to emphasize the connection to the use of the name. Note that, by
virtue of name equivalence, $\quotep{\procn{x}} \nameeq x$; so, the
notation is consistent with previous definitions.

Further, because names have structure it is possible to effect
substitutions on the basis of that structure. This means we need to
upgrade our notation for substitutions, which we accomplish by
adapting comprehension notation. Thus,

\begin{mathpar}
  P\{ y / x : x \in S \}
\end{mathpar}

is interpreted to mean the process derived from P by replacing (in a
capture-avoiding manner) each occurrence of $x$ in $S$ by $y$. For example,

\begin{mathpar}
  P\{ \quotep{\procn{x}|\procn{x}} / x : x \in \freenames{P} \}
\end{mathpar}

will replace each (occurrence) of a free name $x$ in $P$ by
$\quotep{\procn{x}|\procn{x}}$.

Also, we will avail ourselves of the notation $x^{L}$ and $x^{R}$ to
denote injections of a name into disjoint copies of the name
space. There are numerous ways to accomplish this. One example can be
found in \cite{MeredithR05}. This notation overloads to vectors of
names: $\vec{x}^{\pi} := (x_{i}^{\pi} \; : \; 0 \leq i < |\vec{x}| )$ where $\pi \in \{L,R\}$.

We also use $P^{\Box} := P|\Box$.

In \cite{MeredithR05} an interpretation of the new operator is
given. It turns out that there are several possible interpretations
all enjoying the requisite algebraic properties of the operator (see
\cite{milner91polyadicpi}). We will therefore make liberal use of
$(\nu\; \vec{x})P$.

% subsection the_syntax_and_semantics_of_the_notation_system (end)   

\input{qm2pi.qmops} 

\input{qm2pi.sterngerlach} 

\input{qm2pi.metric} 

% section concurrent_process_calculi (end)

%\input{qm2pi.proofsketch}

% section proof sketch (end)

%\input{qm2pi.slviaknots} 

% section spatial logic via knots (end)

\input{qm2pi.conclusion}

% section conclusion (end)

%\input{qm2pi.dtcodes} 

% section wiring algorithm (end)

\input{qm2pi.ack} 

% section acknowledgments (end)

\newpage


\bibliographystyle{plain}   
\bibliography{../../biblios/main.bib}

\input{qm2pi.rhodetails}

\end{document}

 

%\documentclass[12pt]{llncs}
%\documentclass{jktr}

\usepackage[pdftex]{hyperref}                   
\usepackage {listings}
\usepackage {mathpartir}
\usepackage{bcprules}
%\usepackage{listings}
                       
\usepackage{graphicx} 
%\usepackage[margins=2.5cm,nohead,nofoot]{geometry}
%\usepackage{geometry}
\usepackage{amsfonts}
\usepackage{amstext}
\usepackage{latexsym}
\usepackage{amssymb}
\usepackage{color}


%\include{myPreamble}
\include{qm2pi.local} 

%\ifpdf
%\usepackage[pdftex]{graphicx}
%\else
%\usepackage{graphicx}
%\fi

 % \ifpdf
%  \usepackage{pdfsync}
%  \if


%\title{Brief Article}
%\author{David F. Snyder}
%\author{L.G. Meredith}

%\address{Dept. of Math., Texas State University--San Marcos, San Marcos, TX 78666}
       
\pagestyle{empty}


\begin{document}

\lstset{language=[Objective]Caml,frame=shadowbox}

\input{qm2pi.front}

% section front matter (end)

\input{qm2pi.intro} 
 
% section introduction (end)

% \input{qm2pi.knotations} 

% section notation (end)

\input{qm2pi.process.calculi} 

% section concurrent_process_calculi_and_spatial_logics_ (end)
    
%\input{qm2pi.knots2pi} 

%\input{qm2pi.trefoil} 

%\input{qm2pi.mainthm} 

% subsection basic_interpretation (end)

%\input{qm2pi.rho.presentation} 
\subsection{The syntax and semantics of the notation system}\label{sub:the_syntax_and_semantics_of_the_notation_system} % (fold)

We now summarize a technical presentation of the calculus that
embodies our theory of dynamics. The typical presentation of such a
calculus follows the style of giving generators and relations on
them. The grammar, below, describing term constructors, freely
generates the set of processes, $\Proc$. This set is then quotiented
by a relation known as structural congruence and it is over this set
that the notion of dynamics is expressed. This presentation is
essentially that of \cite{MeredithR05} with the addition of
polyadicity and summation. For readability we have relegated some of
the technical subtleties to an appendix.

\subsubsection{Process grammar}\label{subsub:process_grammar}

\begin{mathpar}
  \inferrule* [lab=synchronization] {} {{M} \bc \pzero \;|\; x?F \;|\; x!C }
  \and
  \inferrule* [lab=abstraction] {} {{F} \bc (x)P}
  \and
  \inferrule* [lab=concretion] {} {{C} \bc \langle Q \rangle}
  \and
  \inferrule* [lab=process] {} {{P,Q} \bc M \;| \;P|Q \;|\; @{x}}
  \and
  \inferrule* [lab=name] {} {{x} \bc \quotep{P}}
\end{mathpar} 

Note that $\vec{x}$ (resp. $\vec{P}$) denotes a vector of names
(resp. processes) of length $|\vec{x}|$ (resp. $|\vec{P}|$). We adopt
the following useful abbreviations.

\begin{mathpar}
   x?(\vec{y}).P := x.(\vec{y})P \and  x\clift{\vec{P}} := x.\clift{\vec{P}}
   \and x!(y) := \lift{x}{\dropn{y}}
   \and \Pi_{i=0}^{n-1}P_i := P_0 | \ldots | P_{n-1}
\end{mathpar}

\subsubsection{Structural congruence}

\paragraph{Free and bound names and alpha-equivalence.} At the
core of structural equivalence is alpha-equivalence which identifies
process that are the same up to a change of variable. Formally, we
recognize the distinction between free and bound names. The free names
of a process, $\freenames{P}$, may be calculated recursively as
follows:

\begin{mathpar}
\freenames{\pzero} := \emptyset
  \and \\
  \freenames{x?(y).P} := \{ x \} \cup (\freenames{P} \setminus \{ y \})
  \and 
  \freenames{x!\langle P \rangle} := \{ x \} \cup \{ P \} 
  \and \\
  \freenames{P|Q} := \freenames{P} \cup \freenames{Q}
  \and \\
  \freenames{@{x}} := \{ x \}
\end{mathpar}

$\pi$
$\quotep{\pi}$

$\freenames{-} : \pi \to \mathcal{P}(\quotep{\pi})$

\begin{eqnarray*}
  \freenames{\pzero} & := & \emptyset \\
  \freenames{x?(y).P} & := & \{ x \} \cup (\freenames{P} \setminus \{ y \}) \\
  \freenames{x!\langle P \rangle} & := & \{ x \} \cup \{ P \} \\
  \freenames{P|Q} & := & \freenames{P} \cup \freenames{Q} \\
  \freenames{\dropn{x}} & := & \{ x \}
\end{eqnarray*}

The bound names of a process, $\boundnames{P}$, are those names occurring in $P$
that are not free. For example, in $x?(y).0$, the name $x$ is free, while $y$ is bound.

\begin{mathpar}
  \inferrule* [lab=monoidal-laws] {} { P|Q \equiv Q|P \and P|0 \equiv P \and P|(Q|R) \equiv (P|Q)|R }
\end{mathpar}

\begin{mathpar}
  \inferrule* [lab=alpha-equivalence] {} { (x)P \equiv (y)P\{y/x\} \and y \not\in \freenames{P} }
\end{mathpar}

\begin{definition}
Then two processes, $P,Q$, are alpha-equivalent if $P = Q\{\vec{y}/\vec{x}\}$ for
some $\vec{x} \in \boundnames{Q},\vec{y} \in \boundnames{P}$, where $Q\{\vec{y}/\vec{x}\}$
denotes the capture-avoiding substitution of $\vec{y}$ for $\vec{x}$ in $Q$.
\end{definition}

\begin{definition}
  The {\em structural congruence} \cite{SangiorgiWalker} , $\equiv$,
  between processes is the least congruence containing
  alpha-equivalence, satisfying the abelian monoid laws
  (associativity, commutativity and $\pzero$ as identity) for parallel
  composition $|$ and for summation $+$.
\end{definition}

\subsection{Name equivalence}

We take name equivalence, written $\nameeq$, to be the smallest
equivalence relation generated by the following rules.

\begin{mathpar}
\inferrule*[lab=Quote-drop]
{ }
{ \quotep{@{x}} \nameeq x }

\inferrule*[lab=Struct-equiv]
{ P \scong Q }
{ \quotep{P} \nameeq \quotep{Q} }
\end{mathpar}

The astute reader will have noticed that the mutual recursion of names
and processes imposes a mutual recursion on alpha-equivalence and
structural equivalence via name-equivalence. Fortunately, all of this
works out pleasantly and we may calculate in the natural way, free of
concern. The reader interested in the details is referred to the
appendix \ref{appendix:rho_details}.

\subsection{Substitution}

We use $\Proc$ for the set of processes, $\QProc$ for the set of
names, and $\id{\{}\vec{y} / \vec{x} \id{\}}$ to denote partial maps,
$s : \QProc \rightarrow \QProc$. A map, $s$ lifts, uniquely, to a map
on process terms, $\widehat{s} : \Proc \rightarrow \Proc$ by the
following equations.

\begin{mathpar}
  (0) \psubstp{Q}{P} := 0 \\
  (R \juxtap S) \psubstp{Q}{P}
  :=    
  (R)\psubstp{Q}{P} \juxtap (S) \psubstp{Q}{P} \\
  (x?(y).R) \psubstp{Q}{P}    
  :=    
  (x)\substp{Q}{P} (z)\concat( (R \psubstn{z}{y}) \psubstp{Q}{P} ) \\
  (\lift{x}{R}) \psubstp{Q}{P}  
  :=
  \lift{(x)\substp{Q}{P}}{ R \psubstp{Q}{P} } \\
%   (\dropn{x})  \psubstp{Q}{P}       
%   := 
%   \left\{ 
%     \begin{array}{ccc} 
%       \dropn{\quotep{Q}} & & x \nameeq \quotep{P} \\
%       \dropn{x} & & otherwise \\
%     \end{array}
%   \right. 
  (\dropn{x})  \psubstp{Q}{P}       
  := 
  \left\{ 
    \begin{array}{ccc} 
      Q & & x \nameeq \quotep{P} \\
      \dropn{x} & & otherwise \\
    \end{array}
  \right.
\end{mathpar}
 

where

\begin{eqnarray}
  (x)\id{\{} \lpquote Q \rpquote / \lpquote P \rpquote \id{\}}            = 
  \left\{ 
    \begin{array}{ccc}
      \lpquote Q \rpquote & & x \nameeq \lpquote P \rpquote \\
      x & & otherwise \\
    \end{array}
  \right. \nonumber
\end{eqnarray}

and $z$ is chosen distinct from $\quotep{P}$, $\quotep{Q}$, the free
names in $Q$, and all the names in $R$. Our $\alpha$-equivalence will
be built in the standard way from this substitution.

\begin{remark}\label{rem:no_self_referential_names}
  One consequence of these definitions is that $\forall P. \quotep{P}
  \not\in \freenames{P}$.
\end{remark}

\subsection{ Dynamic quote: an example }

Anticipating something of what's to come, consider applying the
substitution, $\widehat{\id{\{}u / z \id{\}}}$, to the following pair
of processes, $\lift{w}{y!(z)}$ and $w[ \lpquote y!(z) \rpquote ]$.

\begin{eqnarray}
	\lift{w}{y!(z)}\widehat{\id{\{}u / z \id{\}}}
		& = &
		\lift{w}{y!(u)} \nonumber\\
	w[ \lpquote y!(z) \rpquote ] \widehat{ \id{\{}u / z \id{\}} }
		& = &
		w[ \lpquote y!(z) \rpquote ] \nonumber
\end{eqnarray}

Because the body of the process between quotes is impervious to
substitution, we get radically different answers. In fact, by
examining the first process in an input context,
e.g. $x?(z).\lift{w}{y!(z)}$, we see that the process under the lift
operator may be shaped by prefixed inputs binding a name inside it. In
this sense, the lift operator will be seen as a way to dynamically
construct processes before reifying them as names.

Finally equipped with these standard features we can present the
dynamics of the calculus.

\subsubsection{Operational semantics} 

Finally, we introduce the computational dynamics. What marks these
algebras as distinct from other more traditionally studied algebraic
structures, e.g. vector spaces or polynomial rings, is the manner in
which dynamics is captured. In traditional structures, dynamics is typically
expressed through morphisms between such structures, as in linear maps
between vector spaces or morphisms between rings. In algebras
associated with the semantics of computation, the dynamics is
expressed as part of the algebraic structure itself, through a
reduction reduction relation typically denoted by $\red$. Below, we
give a recursive presentation of this relation for the calculus used
in the encoding.

$\red \subseteq \pi \times \pi$
$\red : \pi \to \mathcal{P}(\pi)$

\begin{mathpar}
  \inferrule* [lab=Comm] { \textsf{match}( x_{src}, x_{trgt} ) } { x_{trgt}?(y)P \; | \; x_{src}!\langle {Q} \rangle \red P\{\quotep{Q}/y}\} }
  \and \\
  \inferrule* [lab=Par] {{P} \red {P}'} {{{P} | {Q}} \red {{P}' | {Q}}}
  \and
  \inferrule* [lab=Equiv]{{{P} \scong {P}'} \andalso {{P}' \red {Q}'} \andalso {{Q}' \scong {Q}}}{{P} \red {Q}}
\end{mathpar}

\begin{eqnarray*}
  match_{\equiv} (\quotep{P},\quotep{Q}) & := & P \equiv Q \\
  match_{\dagger}(\quotep{P},\quotep{Q}) & := & \forall R. P|Q \red^{*} R => R \red^{*} 0 \\
  match_{K}(\quotep{P},\quotep{Q}) & := & K \mbox{ for some context } K
\end{eqnarray*}

$u?(x)P | u!\langle Q \rangle \red P\{\quotep{Q}/x\}$

%We write $\wred$ for $\red^*$, and $P\red$ if $\exists Q $ such that $ P \red Q$.
We write $P\red$ if $\exists Q $ such that $ P \red Q$ and $P\not\red$, otherwise.

\section{Replication}

As mentioned before, it is known that replication (and hence
recursion) can be implemented in a higher-order process algebra
\cite{SangiorgiWalker}. As our first example of calculation with the
machinery thus far presented we give the construction explicitly in
the {\rhoc}.

\begin{eqnarray}
	D_{x} & := & \prefix{x}{y}{(\binpar{\outputp{x}{y}}{@{y}})} \nonumber\\
	\bangp_{x}{P} & := & \binpar{{x}!\langle{\binpar{D_{x}}{P}}\rangle}{D_{x}} \nonumber
\end{eqnarray}

\begin{eqnarray}
	\bangp_{x}{P} & & \nonumber\\
	=
	& {x}!\langle{(\prefix{x}{y}{(\outputp{x}{y} | @{y})) | P}}\rangle 
	      | \prefix{x}{y}{(\outputp{x}{y} | @{y})} & \nonumber\\
	\red
	& (\outputp{x}{y} | @{y})\substn{\quotep{(\prefix{x}{y}{(@{y} | \outputp{x}{y})) | P}}}{y} & \nonumber\\
	=
	& \outputp{x}{\quotep{(\prefix{x}{y}{(\outputp{x}{y} | @{y})) | P}}}
	  | {(\prefix{x}{y}{(\outputp{x}{y} | @{y})) | P}} & \nonumber\\
	\red
	& \ldots & \nonumber\\
	\red^*
	& P | P | \ldots & \nonumber
\end{eqnarray}

Of course, this encoding, as an implementation, runs away, unfolding
$\bangp{P}$ eagerly. A lazier and more implementable replication
operator, restricted to input-guarded processes, may be obtained as follows.

\begin{eqnarray}
\bangp{\prefix{u}{v}{P}} 
	:= 
	\binpar{\lift{x}{\prefix{u}{v}{(\binpar{D(x)}{P})}}}{D(x)} \nonumber
\end{eqnarray}

\begin{remark}
  Note that the lazier definition still does not deal with summation
  or mixed summation (i.e. sums over input and output). The reader is
  invited to construct definitions of replication that deal with these
  features. 

  Further, the definitions are parameterized in a name, $x$. Can you,
  gentle reader, make a definition that eliminates this parameter and
  guarantees no accidental interaction between the replication
  machinery and the process being replicated -- i.e. no accidental
  sharing of names used by the process to get its work done and the
  name(s) used by the replication to effect copying. This latter
  revision of the definition of replication is crucial to obtaining
  the expected identity $!!P \sim !P$.
\end{remark}

\begin{remark}\label{rem:paradoxical_combinator}
  The reader familiar with the lambda calculus will have noticed the
  similarity between $D$ and the paradoxical combinator.

  [Ed. note: the existence of this seems to suggest we have to be more
  restrictive on the set of processes and names we admit if we are to
  support no-cloning.]
\end{remark}

\subsubsection{Bisimulation}

The computational dynamics gives rise to another kind of equivalence,
the equivalence of computational behavior. As previously mentioned
this is typically captured \emph{via} some form of bisimulation.

% The notion we use in this paper is weak barbed bisimulation
% \cite{milner91polyadicpi}.

The notion we use in this paper is derived from weak barbed
bisimulation \cite{milner91polyadicpi}. 

\begin{definition}
An \emph{observation relation}, $\downarrow_{\mathcal N}$, over a set
of names, $\mathcal N$, is the smallest relation satisfying the rules
below.

\infrule[Out-barb]{y \in {\mathcal N}, \; x \nameeq y}
		  {\outputp{x}{v} \downarrow_{\mathcal N} x}
\infrule[Par-barb]{\mbox{$P\downarrow_{\mathcal N} x$ or $Q\downarrow_{\mathcal N} x$}}
		  {\binpar{P}{Q} \downarrow_{\mathcal N} x}

We write $P \Downarrow_{\mathcal N} x$ if there is $Q$ such that 
$P \wred Q$ and $Q \downarrow_{\mathcal N} x$.
\end{definition}

\begin{definition}
%\label{def.bbisim}
An  ${\mathcal N}$-\emph{barbed bisimulation} over a set of names, ${\mathcal N}$, is a symmetric binary relation 
${\mathcal S}_{\mathcal N}$ between agents such that $P\rel{S}_{\mathcal N}Q$ implies:
\begin{enumerate}
\item If $P \red P'$ then $Q \wred Q'$ and $P'\rel{S}_{\mathcal N} Q'$.
\item If $P\downarrow_{\mathcal N} x$, then $Q\Downarrow_{\mathcal N} x$.
\end{enumerate}
$P$ is ${\mathcal N}$-barbed bisimilar to $Q$, written
$P \wbbisim_{\mathcal N} Q$, if $P \rel{S}_{\mathcal N} Q$ for some ${\mathcal N}$-barbed bisimulation ${\mathcal S}_{\mathcal N}$.
\end{definition}

$\mathcal{R} \subseteq \pi \times \pi$

$P \mathcal{R} Q => \forall P'. P \red P' \Rightarrow \exists Q'. Q \red Q', P' \mathcal{R} Q'$

$P \vdash x \Rightarrow Q \vdash x$

\begin{mathpar}
  \inferrule*[lab=Out-barb]{x \nameeq y}{{y}!\langle{Q}\rangle \vdash x}
  \and
  \inferrule*[lab=Par-barb]{\mbox{$P\vdash x$ or $Q\vdash x$}}{\binpar{P}{Q} \vdash x}
\end{mathpar}

\subsubsection{Contexts}

One of the principle advantages of computational calculi like the
$\pi$-calculus is a well-defined notion of context,
contextual-equivalence and a correlation between
contextual-equivalence and notions of bisimulation. The notion of
context allows the decomposition of a process into (sub-)process and
its syntactic environment, its context. Thus, a context may be
thought of as a process with a ``hole'' (written $\Box$) in it. The
application of a context $M$ to a process $P$, written $M[P]$, is
tantamount to filling the hole in $M$ with $P$. In this paper we do
not need the full weight of this theory, but do make use of the notion
of context in the proof the main theorem. 

\begin{mathpar}
  \inferrule* [lab=summation] {} {{M_{M},M_{N}} \bc \Box \;|\; x.M_{A} \;|\; M_{M}+M_{N}}
  \and
  \inferrule* [lab=agent] {} {{M_{A}} \bc (\vec{x})M_{P} \;| \; \clift{P_0,\ldots,M_{P},\ldots,P_N}}
  \and \\
  \inferrule* [lab=process] {} {{M_{P}} \bc M_{N} \;| \;P|M_{P} }
\end{mathpar} 

\begin{mathpar}
  \inferrule* [lab=sychronization] {} {M_{N} \bc \Box \;|\; x?M_{F} \;|\; x!M_{C}}
  \and
  \inferrule* [lab=abstraction] {} {{M_{F}} \bc (x)M_{P} }
  \and
  \inferrule* [lab=concretion] {} {{M_{C}} \bc \langle M_{P} \rangle }
  \and \\
  \inferrule* [lab=process] {} {{M_{P}} \bc M_{N} \;| \;P|M_{P} }
\end{mathpar}

\begin{definition}[contextual application] Given a context $M$, and
  process $P$, we define the \emph{contextual application}, $M[P] :=
  M\{P/\Box\}$. That is, the contextual application of M to P is the
  substitution of $P$ for $\Box$ in $M$.
\end{definition}

$\meaningof{-} : L \to \mathcal{P}(\pi)$

\begin{mathpar}
  \inferrule* [lab=collection] {} {\meaningof{true} = \pi, \and \meaningof{~E} = \pi \setminus \meaningof{E}, \and \meaningof{E_{1} \& E_{2}} = \meaningof{E_{1}} \cap \meaningof{E_{2}}}
\end{mathpar}

\begin{mathpar}
  \inferrule* [lab=structure] {} {\meaningof{0} = \{ P \in \pi | P \equiv 0 \}, \and \\ \meaningof{E_1 | E_2} = \{ P \in \pi | P \equiv P_{1} | P_{2}, P_{1} \in \meaningof{E_{1}}, P_{2} \in \meaningof{E_2}\} }
\end{mathpar}

\begin{mathpar}
 \inferrule* [lab=behavior] {} {\meaningof{\langle a?b \rangle E} = \{ P \in \pi | P \equiv Q | u?(y)P', \\ \and \\\\ \and \\ \;\;\; u \in \meaningof{a}, \forall z.P'\{z/y\} \in \meaningof{E\{z/b\}}\}, \and \\ \meaningof{a!E} = \{ P \in \pi | P \equiv Q | x!\langle P' \rangle, x \in \meaningof{a} P' \in \meaningof{E}\} }
\end{mathpar}

\begin{mathpar}
 \inferrule* [lab=nominal] {} {\meaningof{\quotep{E}} = \{ \quotep{P} \in \quotep{\pi} | P \in \meaningof{E} \}, \and \meaningof{\quotep{P}} = \{ \quotep{Q} \in \quotep{\pi} | P \equiv Q \} \and \\ \meaningof{@\quotep{E}} = \{ P \in \pi | P \equiv @x, x \in \meaningof{E} \}}
\end{mathpar}

\begin{eqnarray*}
  \\
  \meaningof{-} : TS \to ST
\end{eqnarray*}

\begin{eqnarray*}
  \\
  L : TS \to ST
\end{eqnarray*}

\begin{eqnarray*}
  \\
  P \models E \iff P \in \meaningof{E}
\end{eqnarray*}

\begin{eqnarray*}
  P \approx_{L} Q \iff \forall E \in L. P \models E \iff Q \models E
\end{eqnarray*}

\begin{eqnarray*}
  P \approx_{K} Q
\end{eqnarray*}

\begin{eqnarray*}
  P \approx Q
\end{eqnarray*}

$\approx_{K} = \approx = \approx_{L}$

\subsubsection{Contextual duality}

Note that contexts extend the quotation operation to a family of
operations from processes to names. Given a context, $M$, we can
define a \emph{nominal context}, $\quotep{M}$ by $\quotep{M}[P] :=
\quotep{M[P]}$. To foreshadow what is to come we observe that these
operations enjoy a duality with processes very much like the duality
between vectors and maps from vectors to scalars.

Further, because the calculus is essentially higher-order, we have a
correspondence between contexts and processes. More specifically,
given a name $x$ and a context $M$ we can construct $M^{*}_{x}$ such
that 

\begin{mathpar}
  M^{*}_{x} | \lift{x}{P} \red M[P]
\end{mathpar}

namely,

\begin{mathpar}
  M^{*}_{x} := x?(u).M[\dropn{u}]
\end{mathpar}

The dependence of $M^{*}_{x}$ on a name makes it an abstraction, 

\begin{mathpar}
  M^{*} := (x)x?(u).M[\dropn{u}]
\end{mathpar}

\subsection{Additional notation}

It will sometimes be convenient to denote the process a name
quotes. We already have the notation $x = \quotep{P}$, but it will be
convenient to introduce an alternate notation, $\procn{x}$, when we
want to emphasize the connection to the use of the name. Note that, by
virtue of name equivalence, $\quotep{\procn{x}} \nameeq x$; so, the
notation is consistent with previous definitions.

Further, because names have structure it is possible to effect
substitutions on the basis of that structure. This means we need to
upgrade our notation for substitutions, which we accomplish by
adapting comprehension notation. Thus,

\begin{mathpar}
  P\{ y / x : x \in S \}
\end{mathpar}

is interpreted to mean the process derived from P by replacing (in a
capture-avoiding manner) each occurrence of $x$ in $S$ by $y$. For example,

\begin{mathpar}
  P\{ \quotep{\procn{x}|\procn{x}} / x : x \in \freenames{P} \}
\end{mathpar}

will replace each (occurrence) of a free name $x$ in $P$ by
$\quotep{\procn{x}|\procn{x}}$.

Also, we will avail ourselves of the notation $x^{L}$ and $x^{R}$ to
denote injections of a name into disjoint copies of the name
space. There are numerous ways to accomplish this. One example can be
found in \cite{MeredithR05}. This notation overloads to vectors of
names: $\vec{x}^{\pi} := (x_{i}^{\pi} \; : \; 0 \leq i < |\vec{x}| )$ where $\pi \in \{L,R\}$.

We also use $P^{\Box} := P|\Box$.

In \cite{MeredithR05} an interpretation of the new operator is
given. It turns out that there are several possible interpretations
all enjoying the requisite algebraic properties of the operator (see
\cite{milner91polyadicpi}). We will therefore make liberal use of
$(\nu\; \vec{x})P$.

% subsection the_syntax_and_semantics_of_the_notation_system (end)   

\input{qm2pi.qmops} 

\input{qm2pi.sterngerlach} 

\input{qm2pi.metric} 

% section concurrent_process_calculi (end)

%\input{qm2pi.proofsketch}

% section proof sketch (end)

%\input{qm2pi.slviaknots} 

% section spatial logic via knots (end)

\input{qm2pi.conclusion}

% section conclusion (end)

%\input{qm2pi.dtcodes} 

% section wiring algorithm (end)

\input{qm2pi.ack} 

% section acknowledgments (end)

\newpage


\bibliographystyle{plain}   
\bibliography{../../biblios/main.bib}

\input{qm2pi.rhodetails}

\end{document}

 

% subsection basic_interpretation (end)

%\input{qm2pi.rho.presentation} 
\subsection{The syntax and semantics of the notation system}\label{sub:the_syntax_and_semantics_of_the_notation_system} % (fold)

We now summarize a technical presentation of the calculus that
embodies our theory of dynamics. The typical presentation of such a
calculus follows the style of giving generators and relations on
them. The grammar, below, describing term constructors, freely
generates the set of processes, $\Proc$. This set is then quotiented
by a relation known as structural congruence and it is over this set
that the notion of dynamics is expressed. This presentation is
essentially that of \cite{MeredithR05} with the addition of
polyadicity and summation. For readability we have relegated some of
the technical subtleties to an appendix.

\subsubsection{Process grammar}\label{subsub:process_grammar}

\begin{mathpar}
  \inferrule* [lab=synchronization] {} {{M} \bc \pzero \;|\; x?F \;|\; x!C }
  \and
  \inferrule* [lab=abstraction] {} {{F} \bc (x)P}
  \and
  \inferrule* [lab=concretion] {} {{C} \bc \langle Q \rangle}
  \and
  \inferrule* [lab=process] {} {{P,Q} \bc M \;| \;P|Q \;|\; @{x}}
  \and
  \inferrule* [lab=name] {} {{x} \bc \quotep{P}}
\end{mathpar} 

Note that $\vec{x}$ (resp. $\vec{P}$) denotes a vector of names
(resp. processes) of length $|\vec{x}|$ (resp. $|\vec{P}|$). We adopt
the following useful abbreviations.

\begin{mathpar}
   x?(\vec{y}).P := x.(\vec{y})P \and  x\clift{\vec{P}} := x.\clift{\vec{P}}
   \and x!(y) := \lift{x}{\dropn{y}}
   \and \Pi_{i=0}^{n-1}P_i := P_0 | \ldots | P_{n-1}
\end{mathpar}

\subsubsection{Structural congruence}

\paragraph{Free and bound names and alpha-equivalence.} At the
core of structural equivalence is alpha-equivalence which identifies
process that are the same up to a change of variable. Formally, we
recognize the distinction between free and bound names. The free names
of a process, $\freenames{P}$, may be calculated recursively as
follows:

\begin{mathpar}
\freenames{\pzero} := \emptyset
  \and \\
  \freenames{x?(y).P} := \{ x \} \cup (\freenames{P} \setminus \{ y \})
  \and 
  \freenames{x!\langle P \rangle} := \{ x \} \cup \{ P \} 
  \and \\
  \freenames{P|Q} := \freenames{P} \cup \freenames{Q}
  \and \\
  \freenames{@{x}} := \{ x \}
\end{mathpar}

$\pi$
$\quotep{\pi}$

$\freenames{-} : \pi \to \mathcal{P}(\quotep{\pi})$

\begin{eqnarray*}
  \freenames{\pzero} & := & \emptyset \\
  \freenames{x?(y).P} & := & \{ x \} \cup (\freenames{P} \setminus \{ y \}) \\
  \freenames{x!\langle P \rangle} & := & \{ x \} \cup \{ P \} \\
  \freenames{P|Q} & := & \freenames{P} \cup \freenames{Q} \\
  \freenames{\dropn{x}} & := & \{ x \}
\end{eqnarray*}

The bound names of a process, $\boundnames{P}$, are those names occurring in $P$
that are not free. For example, in $x?(y).0$, the name $x$ is free, while $y$ is bound.

\begin{mathpar}
  \inferrule* [lab=monoidal-laws] {} { P|Q \equiv Q|P \and P|0 \equiv P \and P|(Q|R) \equiv (P|Q)|R }
\end{mathpar}

\begin{mathpar}
  \inferrule* [lab=alpha-equivalence] {} { (x)P \equiv (y)P\{y/x\} \and y \not\in \freenames{P} }
\end{mathpar}

\begin{definition}
Then two processes, $P,Q$, are alpha-equivalent if $P = Q\{\vec{y}/\vec{x}\}$ for
some $\vec{x} \in \boundnames{Q},\vec{y} \in \boundnames{P}$, where $Q\{\vec{y}/\vec{x}\}$
denotes the capture-avoiding substitution of $\vec{y}$ for $\vec{x}$ in $Q$.
\end{definition}

\begin{definition}
  The {\em structural congruence} \cite{SangiorgiWalker} , $\equiv$,
  between processes is the least congruence containing
  alpha-equivalence, satisfying the abelian monoid laws
  (associativity, commutativity and $\pzero$ as identity) for parallel
  composition $|$ and for summation $+$.
\end{definition}

\subsection{Name equivalence}

We take name equivalence, written $\nameeq$, to be the smallest
equivalence relation generated by the following rules.

\begin{mathpar}
\inferrule*[lab=Quote-drop]
{ }
{ \quotep{@{x}} \nameeq x }

\inferrule*[lab=Struct-equiv]
{ P \scong Q }
{ \quotep{P} \nameeq \quotep{Q} }
\end{mathpar}

The astute reader will have noticed that the mutual recursion of names
and processes imposes a mutual recursion on alpha-equivalence and
structural equivalence via name-equivalence. Fortunately, all of this
works out pleasantly and we may calculate in the natural way, free of
concern. The reader interested in the details is referred to the
appendix \ref{appendix:rho_details}.

\subsection{Substitution}

We use $\Proc$ for the set of processes, $\QProc$ for the set of
names, and $\id{\{}\vec{y} / \vec{x} \id{\}}$ to denote partial maps,
$s : \QProc \rightarrow \QProc$. A map, $s$ lifts, uniquely, to a map
on process terms, $\widehat{s} : \Proc \rightarrow \Proc$ by the
following equations.

\begin{mathpar}
  (0) \psubstp{Q}{P} := 0 \\
  (R \juxtap S) \psubstp{Q}{P}
  :=    
  (R)\psubstp{Q}{P} \juxtap (S) \psubstp{Q}{P} \\
  (x?(y).R) \psubstp{Q}{P}    
  :=    
  (x)\substp{Q}{P} (z)\concat( (R \psubstn{z}{y}) \psubstp{Q}{P} ) \\
  (\lift{x}{R}) \psubstp{Q}{P}  
  :=
  \lift{(x)\substp{Q}{P}}{ R \psubstp{Q}{P} } \\
%   (\dropn{x})  \psubstp{Q}{P}       
%   := 
%   \left\{ 
%     \begin{array}{ccc} 
%       \dropn{\quotep{Q}} & & x \nameeq \quotep{P} \\
%       \dropn{x} & & otherwise \\
%     \end{array}
%   \right. 
  (\dropn{x})  \psubstp{Q}{P}       
  := 
  \left\{ 
    \begin{array}{ccc} 
      Q & & x \nameeq \quotep{P} \\
      \dropn{x} & & otherwise \\
    \end{array}
  \right.
\end{mathpar}
 

where

\begin{eqnarray}
  (x)\id{\{} \lpquote Q \rpquote / \lpquote P \rpquote \id{\}}            = 
  \left\{ 
    \begin{array}{ccc}
      \lpquote Q \rpquote & & x \nameeq \lpquote P \rpquote \\
      x & & otherwise \\
    \end{array}
  \right. \nonumber
\end{eqnarray}

and $z$ is chosen distinct from $\quotep{P}$, $\quotep{Q}$, the free
names in $Q$, and all the names in $R$. Our $\alpha$-equivalence will
be built in the standard way from this substitution.

\begin{remark}\label{rem:no_self_referential_names}
  One consequence of these definitions is that $\forall P. \quotep{P}
  \not\in \freenames{P}$.
\end{remark}

\subsection{ Dynamic quote: an example }

Anticipating something of what's to come, consider applying the
substitution, $\widehat{\id{\{}u / z \id{\}}}$, to the following pair
of processes, $\lift{w}{y!(z)}$ and $w[ \lpquote y!(z) \rpquote ]$.

\begin{eqnarray}
	\lift{w}{y!(z)}\widehat{\id{\{}u / z \id{\}}}
		& = &
		\lift{w}{y!(u)} \nonumber\\
	w[ \lpquote y!(z) \rpquote ] \widehat{ \id{\{}u / z \id{\}} }
		& = &
		w[ \lpquote y!(z) \rpquote ] \nonumber
\end{eqnarray}

Because the body of the process between quotes is impervious to
substitution, we get radically different answers. In fact, by
examining the first process in an input context,
e.g. $x?(z).\lift{w}{y!(z)}$, we see that the process under the lift
operator may be shaped by prefixed inputs binding a name inside it. In
this sense, the lift operator will be seen as a way to dynamically
construct processes before reifying them as names.

Finally equipped with these standard features we can present the
dynamics of the calculus.

\subsubsection{Operational semantics} 

Finally, we introduce the computational dynamics. What marks these
algebras as distinct from other more traditionally studied algebraic
structures, e.g. vector spaces or polynomial rings, is the manner in
which dynamics is captured. In traditional structures, dynamics is typically
expressed through morphisms between such structures, as in linear maps
between vector spaces or morphisms between rings. In algebras
associated with the semantics of computation, the dynamics is
expressed as part of the algebraic structure itself, through a
reduction reduction relation typically denoted by $\red$. Below, we
give a recursive presentation of this relation for the calculus used
in the encoding.

$\red \subseteq \pi \times \pi$
$\red : \pi \to \mathcal{P}(\pi)$

\begin{mathpar}
  \inferrule* [lab=Comm] { \textsf{match}( x_{src}, x_{trgt} ) } { x_{trgt}?(y)P \; | \; x_{src}!\langle {Q} \rangle \red P\{\quotep{Q}/y}\} }
  \and \\
  \inferrule* [lab=Par] {{P} \red {P}'} {{{P} | {Q}} \red {{P}' | {Q}}}
  \and
  \inferrule* [lab=Equiv]{{{P} \scong {P}'} \andalso {{P}' \red {Q}'} \andalso {{Q}' \scong {Q}}}{{P} \red {Q}}
\end{mathpar}

\begin{eqnarray*}
  match_{\equiv} (\quotep{P},\quotep{Q}) & := & P \equiv Q \\
  match_{\dagger}(\quotep{P},\quotep{Q}) & := & \forall R. P|Q \red^{*} R => R \red^{*} 0 \\
  match_{K}(\quotep{P},\quotep{Q}) & := & K \mbox{ for some context } K
\end{eqnarray*}

$u?(x)P | u!\langle Q \rangle \red P\{\quotep{Q}/x\}$

%We write $\wred$ for $\red^*$, and $P\red$ if $\exists Q $ such that $ P \red Q$.
We write $P\red$ if $\exists Q $ such that $ P \red Q$ and $P\not\red$, otherwise.

\section{Replication}

As mentioned before, it is known that replication (and hence
recursion) can be implemented in a higher-order process algebra
\cite{SangiorgiWalker}. As our first example of calculation with the
machinery thus far presented we give the construction explicitly in
the {\rhoc}.

\begin{eqnarray}
	D_{x} & := & \prefix{x}{y}{(\binpar{\outputp{x}{y}}{@{y}})} \nonumber\\
	\bangp_{x}{P} & := & \binpar{{x}!\langle{\binpar{D_{x}}{P}}\rangle}{D_{x}} \nonumber
\end{eqnarray}

\begin{eqnarray}
	\bangp_{x}{P} & & \nonumber\\
	=
	& {x}!\langle{(\prefix{x}{y}{(\outputp{x}{y} | @{y})) | P}}\rangle 
	      | \prefix{x}{y}{(\outputp{x}{y} | @{y})} & \nonumber\\
	\red
	& (\outputp{x}{y} | @{y})\substn{\quotep{(\prefix{x}{y}{(@{y} | \outputp{x}{y})) | P}}}{y} & \nonumber\\
	=
	& \outputp{x}{\quotep{(\prefix{x}{y}{(\outputp{x}{y} | @{y})) | P}}}
	  | {(\prefix{x}{y}{(\outputp{x}{y} | @{y})) | P}} & \nonumber\\
	\red
	& \ldots & \nonumber\\
	\red^*
	& P | P | \ldots & \nonumber
\end{eqnarray}

Of course, this encoding, as an implementation, runs away, unfolding
$\bangp{P}$ eagerly. A lazier and more implementable replication
operator, restricted to input-guarded processes, may be obtained as follows.

\begin{eqnarray}
\bangp{\prefix{u}{v}{P}} 
	:= 
	\binpar{\lift{x}{\prefix{u}{v}{(\binpar{D(x)}{P})}}}{D(x)} \nonumber
\end{eqnarray}

\begin{remark}
  Note that the lazier definition still does not deal with summation
  or mixed summation (i.e. sums over input and output). The reader is
  invited to construct definitions of replication that deal with these
  features. 

  Further, the definitions are parameterized in a name, $x$. Can you,
  gentle reader, make a definition that eliminates this parameter and
  guarantees no accidental interaction between the replication
  machinery and the process being replicated -- i.e. no accidental
  sharing of names used by the process to get its work done and the
  name(s) used by the replication to effect copying. This latter
  revision of the definition of replication is crucial to obtaining
  the expected identity $!!P \sim !P$.
\end{remark}

\begin{remark}\label{rem:paradoxical_combinator}
  The reader familiar with the lambda calculus will have noticed the
  similarity between $D$ and the paradoxical combinator.

  [Ed. note: the existence of this seems to suggest we have to be more
  restrictive on the set of processes and names we admit if we are to
  support no-cloning.]
\end{remark}

\subsubsection{Bisimulation}

The computational dynamics gives rise to another kind of equivalence,
the equivalence of computational behavior. As previously mentioned
this is typically captured \emph{via} some form of bisimulation.

% The notion we use in this paper is weak barbed bisimulation
% \cite{milner91polyadicpi}.

The notion we use in this paper is derived from weak barbed
bisimulation \cite{milner91polyadicpi}. 

\begin{definition}
An \emph{observation relation}, $\downarrow_{\mathcal N}$, over a set
of names, $\mathcal N$, is the smallest relation satisfying the rules
below.

\infrule[Out-barb]{y \in {\mathcal N}, \; x \nameeq y}
		  {\outputp{x}{v} \downarrow_{\mathcal N} x}
\infrule[Par-barb]{\mbox{$P\downarrow_{\mathcal N} x$ or $Q\downarrow_{\mathcal N} x$}}
		  {\binpar{P}{Q} \downarrow_{\mathcal N} x}

We write $P \Downarrow_{\mathcal N} x$ if there is $Q$ such that 
$P \wred Q$ and $Q \downarrow_{\mathcal N} x$.
\end{definition}

\begin{definition}
%\label{def.bbisim}
An  ${\mathcal N}$-\emph{barbed bisimulation} over a set of names, ${\mathcal N}$, is a symmetric binary relation 
${\mathcal S}_{\mathcal N}$ between agents such that $P\rel{S}_{\mathcal N}Q$ implies:
\begin{enumerate}
\item If $P \red P'$ then $Q \wred Q'$ and $P'\rel{S}_{\mathcal N} Q'$.
\item If $P\downarrow_{\mathcal N} x$, then $Q\Downarrow_{\mathcal N} x$.
\end{enumerate}
$P$ is ${\mathcal N}$-barbed bisimilar to $Q$, written
$P \wbbisim_{\mathcal N} Q$, if $P \rel{S}_{\mathcal N} Q$ for some ${\mathcal N}$-barbed bisimulation ${\mathcal S}_{\mathcal N}$.
\end{definition}

$\mathcal{R} \subseteq \pi \times \pi$

$P \mathcal{R} Q => \forall P'. P \red P' \Rightarrow \exists Q'. Q \red Q', P' \mathcal{R} Q'$

$P \vdash x \Rightarrow Q \vdash x$

\begin{mathpar}
  \inferrule*[lab=Out-barb]{x \nameeq y}{{y}!\langle{Q}\rangle \vdash x}
  \and
  \inferrule*[lab=Par-barb]{\mbox{$P\vdash x$ or $Q\vdash x$}}{\binpar{P}{Q} \vdash x}
\end{mathpar}

\subsubsection{Contexts}

One of the principle advantages of computational calculi like the
$\pi$-calculus is a well-defined notion of context,
contextual-equivalence and a correlation between
contextual-equivalence and notions of bisimulation. The notion of
context allows the decomposition of a process into (sub-)process and
its syntactic environment, its context. Thus, a context may be
thought of as a process with a ``hole'' (written $\Box$) in it. The
application of a context $M$ to a process $P$, written $M[P]$, is
tantamount to filling the hole in $M$ with $P$. In this paper we do
not need the full weight of this theory, but do make use of the notion
of context in the proof the main theorem. 

\begin{mathpar}
  \inferrule* [lab=summation] {} {{M_{M},M_{N}} \bc \Box \;|\; x.M_{A} \;|\; M_{M}+M_{N}}
  \and
  \inferrule* [lab=agent] {} {{M_{A}} \bc (\vec{x})M_{P} \;| \; \clift{P_0,\ldots,M_{P},\ldots,P_N}}
  \and \\
  \inferrule* [lab=process] {} {{M_{P}} \bc M_{N} \;| \;P|M_{P} }
\end{mathpar} 

\begin{mathpar}
  \inferrule* [lab=sychronization] {} {M_{N} \bc \Box \;|\; x?M_{F} \;|\; x!M_{C}}
  \and
  \inferrule* [lab=abstraction] {} {{M_{F}} \bc (x)M_{P} }
  \and
  \inferrule* [lab=concretion] {} {{M_{C}} \bc \langle M_{P} \rangle }
  \and \\
  \inferrule* [lab=process] {} {{M_{P}} \bc M_{N} \;| \;P|M_{P} }
\end{mathpar}

\begin{definition}[contextual application] Given a context $M$, and
  process $P$, we define the \emph{contextual application}, $M[P] :=
  M\{P/\Box\}$. That is, the contextual application of M to P is the
  substitution of $P$ for $\Box$ in $M$.
\end{definition}

$\meaningof{-} : L \to \mathcal{P}(\pi)$

\begin{mathpar}
  \inferrule* [lab=collection] {} {\meaningof{true} = \pi, \and \meaningof{~E} = \pi \setminus \meaningof{E}, \and \meaningof{E_{1} \& E_{2}} = \meaningof{E_{1}} \cap \meaningof{E_{2}}}
\end{mathpar}

\begin{mathpar}
  \inferrule* [lab=structure] {} {\meaningof{0} = \{ P \in \pi | P \equiv 0 \}, \and \\ \meaningof{E_1 | E_2} = \{ P \in \pi | P \equiv P_{1} | P_{2}, P_{1} \in \meaningof{E_{1}}, P_{2} \in \meaningof{E_2}\} }
\end{mathpar}

\begin{mathpar}
 \inferrule* [lab=behavior] {} {\meaningof{\langle a?b \rangle E} = \{ P \in \pi | P \equiv Q | u?(y)P', \\ \and \\\\ \and \\ \;\;\; u \in \meaningof{a}, \forall z.P'\{z/y\} \in \meaningof{E\{z/b\}}\}, \and \\ \meaningof{a!E} = \{ P \in \pi | P \equiv Q | x!\langle P' \rangle, x \in \meaningof{a} P' \in \meaningof{E}\} }
\end{mathpar}

\begin{mathpar}
 \inferrule* [lab=nominal] {} {\meaningof{\quotep{E}} = \{ \quotep{P} \in \quotep{\pi} | P \in \meaningof{E} \}, \and \meaningof{\quotep{P}} = \{ \quotep{Q} \in \quotep{\pi} | P \equiv Q \} \and \\ \meaningof{@\quotep{E}} = \{ P \in \pi | P \equiv @x, x \in \meaningof{E} \}}
\end{mathpar}

\begin{eqnarray*}
  \\
  \meaningof{-} : TS \to ST
\end{eqnarray*}

\begin{eqnarray*}
  \\
  L : TS \to ST
\end{eqnarray*}

\begin{eqnarray*}
  \\
  P \models E \iff P \in \meaningof{E}
\end{eqnarray*}

\begin{eqnarray*}
  P \approx_{L} Q \iff \forall E \in L. P \models E \iff Q \models E
\end{eqnarray*}

\begin{eqnarray*}
  P \approx_{K} Q
\end{eqnarray*}

\begin{eqnarray*}
  P \approx Q
\end{eqnarray*}

$\approx_{K} = \approx = \approx_{L}$

\subsubsection{Contextual duality}

Note that contexts extend the quotation operation to a family of
operations from processes to names. Given a context, $M$, we can
define a \emph{nominal context}, $\quotep{M}$ by $\quotep{M}[P] :=
\quotep{M[P]}$. To foreshadow what is to come we observe that these
operations enjoy a duality with processes very much like the duality
between vectors and maps from vectors to scalars.

Further, because the calculus is essentially higher-order, we have a
correspondence between contexts and processes. More specifically,
given a name $x$ and a context $M$ we can construct $M^{*}_{x}$ such
that 

\begin{mathpar}
  M^{*}_{x} | \lift{x}{P} \red M[P]
\end{mathpar}

namely,

\begin{mathpar}
  M^{*}_{x} := x?(u).M[\dropn{u}]
\end{mathpar}

The dependence of $M^{*}_{x}$ on a name makes it an abstraction, 

\begin{mathpar}
  M^{*} := (x)x?(u).M[\dropn{u}]
\end{mathpar}

\subsection{Additional notation}

It will sometimes be convenient to denote the process a name
quotes. We already have the notation $x = \quotep{P}$, but it will be
convenient to introduce an alternate notation, $\procn{x}$, when we
want to emphasize the connection to the use of the name. Note that, by
virtue of name equivalence, $\quotep{\procn{x}} \nameeq x$; so, the
notation is consistent with previous definitions.

Further, because names have structure it is possible to effect
substitutions on the basis of that structure. This means we need to
upgrade our notation for substitutions, which we accomplish by
adapting comprehension notation. Thus,

\begin{mathpar}
  P\{ y / x : x \in S \}
\end{mathpar}

is interpreted to mean the process derived from P by replacing (in a
capture-avoiding manner) each occurrence of $x$ in $S$ by $y$. For example,

\begin{mathpar}
  P\{ \quotep{\procn{x}|\procn{x}} / x : x \in \freenames{P} \}
\end{mathpar}

will replace each (occurrence) of a free name $x$ in $P$ by
$\quotep{\procn{x}|\procn{x}}$.

Also, we will avail ourselves of the notation $x^{L}$ and $x^{R}$ to
denote injections of a name into disjoint copies of the name
space. There are numerous ways to accomplish this. One example can be
found in \cite{MeredithR05}. This notation overloads to vectors of
names: $\vec{x}^{\pi} := (x_{i}^{\pi} \; : \; 0 \leq i < |\vec{x}| )$ where $\pi \in \{L,R\}$.

We also use $P^{\Box} := P|\Box$.

In \cite{MeredithR05} an interpretation of the new operator is
given. It turns out that there are several possible interpretations
all enjoying the requisite algebraic properties of the operator (see
\cite{milner91polyadicpi}). We will therefore make liberal use of
$(\nu\; \vec{x})P$.

% subsection the_syntax_and_semantics_of_the_notation_system (end)   

\section{Interpretation of QM}
\subsection{Supporting definitions}
\subsubsection{Multiplication}
\begin{mathpar}
  \quotep{Q} \cdot \quotep{R} := \quotep{Q|R}
  \and \\
  \quotep{Q} \cdot P := P\{ \quotep{Q|R} / \quotep{R} : \quotep{R} \in \freenames{P} \}
\end{mathpar}

\paragraph{Discussion}
The first line needs little explanation. The second line says that
each free name of the process is replaced with the multiplication of
that name by the scalar. Multiplication of a scalar (name) by a state
(process) results in a process all the names of which have been `moved
over' by parallel composition with the process the scalar
quotes. There is a subtlety that the bound names have to be
manipulated so that multiplied names aren't accidentally
captured. There are many ways to achieve this.

\begin{remark}\label{rem:multiplication_identities}
  The reader is invited to verify that for all $x,y,z \in \QProc$ and $P \in \Proc$
  \begin{mathpar}
    x \cdot \quotep{0} \equiv x 
    \and
    x \cdot y \equiv y \cdot x
    \and
    x \cdot (y \cdot z) \equiv (x \cdot y) \cdot z
    \and \\
    \quotep{0} \cdot P \equiv P
    \and \\
    x \cdot (y \cdot P) \equiv (x \cdot y) \cdot P
    \and \\
    x \cdot (P|Q) \equiv (x \cdot P) | (x \cdot Q)
    \and \\    
  \end{mathpar}
\end{remark}

\subsubsection{Tensor product}

We define a tensor product on processes by structural induction.

\paragraph{Tensor of sums} First note that all summations, including
$\pzero$ and sequence, can be written $\Sigma_{i} x_{i}.A_{i} +
\Sigma_{j} x_{j}.C_{j}$, where we have grouped input-guarded processes
together and output-guarded processes together.

Thus, we can define the tensor product of two summations, $N_{1}\otimes N_{2}$, where

\begin{mathpar}
  N_{1} := \Sigma_{i} x_{i}.A_{i} + \Sigma_{j} x_{j}.C_{j}
  \and
  N_{2} := \Sigma_{i'} y_{i'}.B_{i'} + \Sigma_{j'} y_{j'}.D_{j'} 
\end{mathpar}

as follows.

\begin{mathpar}
  \Sigma_{i} x_{i}.A_{i} + \Sigma_{j} x_{j}.C_{j} \otimes \Sigma_{i'}
  y_{i'}.B_{i'} + \Sigma_{j'} y_{j'}.D_{j'} 
  \and \\
  := \; \Sigma_{i} \Sigma_{i'} \quotep{\stackrel{\vee}{x_{i}}| \stackrel{\vee}{y_{i'}}}.(A_{i}\otimes B_{i'}) \; | \; \Sigma_{i'} \Sigma_{i} \quotep{\stackrel{\vee}{y_{i'}}|\stackrel{\vee}{x_{i}}}.(B_{i'}\otimes A_{i})
  \and
  \;\; | \;\; \Sigma_{j} \Sigma_{j'} \quotep{\stackrel{\vee}{x_{j}}|\stackrel{\vee}{y_{j'}}}.(A_{j}\otimes B_{j'}) \; | \; \Sigma_{j'} \Sigma_{j} \quotep{\stackrel{\vee}{y_{j'}}|\stackrel{\vee}{x_{j}}}.(B_{j'}\otimes A_{j})
\end{mathpar}

\begin{remark}
  Do we need to $x^{L}$ and $y^{R}$ for this construction as well?
\end{remark}

\paragraph{Tensor of parallel compositions} Next, we distribute tensor
over par.

\begin{mathpar}
  P_{1}|P_{2} \otimes Q_{1}|Q_{2} := (P_{1} \otimes Q_{1}) | (P_{1}
  \otimes Q_{2}) | (P_{2} \otimes Q_{1}) | (P_{2} \otimes Q_{2})
\end{mathpar}

\paragraph{Tensor with dropped names} We treat tensor of a
process with a dropped name as parallel composition.

\begin{mathpar}
  P \otimes \dropn{x} := P | \dropn{x}
\end{mathpar}

\paragraph{Tensor of agents}

Finally, we need to define tensor on agents. Note that the definition
of tensor on normal products only tensors inputs with inputs and
outputs with outputs. Thus, we only have to define the operation on
``homogeneous'' pairings.

\begin{mathpar}
  (\vec{x})P \otimes (\vec{y})Q
  \and \\
  := (x_{0}^{L}|y_{0}^{R},\ldots,x_{0}^{L}|y_{n}^{R},\ldots,x_{m}^{L}|y_{0}^{R},\ldots,x_{m}^{L}|y_{n}^R)(P\{ \vec{x}^{L}/\vec{x}\} \otimes Q \{ \vec{y}^{R}/\vec{y}\})
  \and \\
  \clift{\vec{P}} \otimes \clift{\vec{Q}}
  \and \\
  := \clift{P_{0}\otimes Q_{0},\ldots,P_{0}\otimes Q_{n},\ldots,P_{m}\otimes Q_{0},\ldots,P_{m}\otimes Q_{n}}
\end{mathpar}

\begin{remark}
  Observe that arities of tensored abstractions matches arities of
  tensored concretions if the original arities matched. Note also that
  the length of the arities corresponds to the increase in dimension
  we see in ordinary vector space tensor product.
\end{remark}

\begin{remark}
  Operationally, this definition distributes the tensor down to
  components ``linked'' by summation. Tensor over summation is
  intriguing in that it mixes names. Moreover, as a consequence of the
  way it mixes names we have the identities for all $x \in \QProc$ and
  $P,Q \in \Proc$

  \begin{mathpar}
    (x \cdot P) \otimes Q \equiv x \cdot (P \otimes Q) \equiv P \otimes (x \cdot Q)
    \and
    P \otimes \pzero \equiv P
  \end{mathpar}

  that the reader is invited to verify.
\end{remark}

\subsubsection{Annihilation}
\begin{mathpar}
  P^{\perp} := \{ Q | \forall R. P|Q \red^{*} R \Rightarrow R \red^{*} \pzero \}
  \and \\
  P^{\underline{\perp}} := \Sigma_{Q \in P^{\perp}} \quotep{Q}?(y).(\dropn{y}|Q) | \Sigma_{Q \in P^{\perp}} \quotep{Q}\clift{\Box}
\end{mathpar}

\paragraph{Discussion} The reader will note that $P^{\perp}$ is a
\emph{set} of processes, while $P^{\underline{\perp}}$ is a
\emph{context}. We call the set $P^{\perp}$ the \emph{annihilators} of
$P$. The parallel composition of a process in the annihilators of $P$
with $P$ will result in a process, the state space of which has all
paths eventually leading to $\pzero$. Execution may endure loops; but
under reasonable conditions of fairness (naturally guaranteed under
most notions of bisimulation) such a composite process cannot get
stuck in such a loop and will, eventually pop out and terminate.

The context $P^{\underline{\perp}}$ is ready and willing to ``take the
$P$ out of'' the process to which it is applied. It will effectively
transmit the code of the process to which it is applied to one of the
annihilators and run the process against it.

\subsubsection{Evaluation}
We fix $M$ a domain of fully abstract interpretation with an equality
coincident with bisimulation. We take $\meaningof{\cdot} : \Proc \to
M$ to be the map interpreting processes and $\nmeaningof{\cdot} : \M
\to Proc$ to be the map running the other way. Then we define

\begin{mathpar}
  \int P := \nmeaningof{\meaningof{P}}
\end{mathpar}

\paragraph{Discussion}
There are many fully abstract interpretations of Milner's
$\pi$-calculus. Any of them can be used as a basis for interpreting
the reflective calculus here. Equipped with such a domain it is
largely a matter of grinding through to check that the Yoneda
construction for the normalization-by-evaluation program can be
extended to this setting.

\begin{remark}
  The reader is invited to verify that $\int (P^{\underline{\perp}}[P]) = 0$.
\end{remark}

\subsection{Quantum mechanics}

Table \ref{tbl:core_qm_op_defns} gives the core operational definitions

\begin{table}[htp]\label{tbl:core_qm_op_defns}
  \center{
    \fbox{
      \begin{tabular}{c|c}
        quantum mechanics & process calculus \\
        \hline
        scalar & $x := \quotep{P}$ \\
        state vector & $\state{P} := P$ \\
        dual & $\state{P}^{*} := \event{P^{\underline{\perp}}} := \quotep{P^{\underline{\perp}}}[-]$ \\
        matrix & $ \Sigma_{\alpha} \state{P_{\alpha}}x_{\alpha}\event{Q_{\alpha}}$ \\
        vector addition & $\state{P} + \state{Q} := \state{P | Q}$ \\
        tensor product & $\state{P} \otimes \state{Q} := \state{P \otimes Q}$ \\
        inner product & $\innerprod{P}{Q} := \quotep{\int P^{\underline{\perp}}[Q]}$ \\
      \end{tabular}
    }
  }
  \caption{QM - operational definitions}
\end{table}

where

\begin{mathpar}
  \prmatrix{P}{Q} := \fprmatrix{P}{\quotep{\pzero}}{Q}
  \and
  \fprmatrix{P}{x}{Q} := (\state{P},x,\event{Q})
  \and
  (\fprmatrix{P}{x}{Q})(\state{R}) := x \cdot \innerprod{Q}{R} \cdot \state{P}
  \and
  (\fprmatrix{P}{x}{Q})(\event{R}) := x \cdot \innerprod{R}{P} \cdot \event{Q}
\end{mathpar}

\paragraph{Discussion}
As promised: vectors (aka states) are represented as processes; duals
as contextual duals; inner product definition should be compared with
standard inner product definition for ....

\begin{remark}
  Assuming $\int (P^{\underline{\perp}}[P]) = 0$, the reader is
  invited to verify that $(\fprmatrix{P}{x}{P})(\state{P}) = x \cdot \state{P}$.
\end{remark}

\begin{remark}
  The reader is invited to verify that $\innerprod{P}{Q}$ could
  equally well have been written $\quotep{\int \stackrel{\vee}{x}}$
  where $x = \event{P^{\underline{\perp}}}(Q)$.

  One of the motivations for this remark is that there is another way
  to factor these operations. We could package up evaluation in the dual:

  \begin{mathpar}
    \state{P}^{*} := \event{\int P^{\underline{\perp}}} := \quotep{\int P^{\underline{\perp}}}[-]
  \end{mathpar}

  and then have inner product defined by
  
  \begin{mathpar}
    \innerprod{P}{Q} := \event{P}(Q)
  \end{mathpar}

  Hopefully, experience with the calculations will provide guidance on
  the best factoring.
\end{remark}

\begin{remark}
  Assuming $\int (P^{\underline{\perp}}[P]) = 0$, the reader is
  invited to verify that $\forall P,Q. (\prmatrix{0}{Q})(\state{0}) =
  \state{0}$ and dually $(\prmatrix{P}{0})(\event{0}) = \event{0}$.
\end{remark}

\begin{remark}
  i'm a little worried that i don't (yet) have proper support for
  complex conjugacy. But, the observation above may give us a
  clue. According to Abramsky, it must be the case that the scalars
  are iso to the homset of the identity for the tensor -- which the
  observation above characterizes. 

  For now, we will simply bookmark the notion with $\overline{x}$.
\end{remark}

\subsubsection{Adjointness}

We need to give a definition of $(\cdot)^{\dagger}$ for matrices. The
obvious candidate definition is
\begin{mathpar}
(\Sigma_{\alpha}\fprmatrix{P_{\alpha}}{x_{\alpha}}{Q_{\alpha}})^{\dagger}
= \Sigma_{\alpha}\fprmatrix{(Q_{\alpha}^{\underline{\perp}})^{*}}{\overline{x}_{\alpha}}{P_{\alpha}^{\underline{\perp}}} 
\end{mathpar}

But, $(Q_{\alpha}^{\underline{\perp}})^{*}$ requires a name along
which to communicate the process to achieve the context application.

\subsubsection{Basis for a basis}
If processes label states and ``addition'' of states (a.k.a. vector
addition) is interpreted as parallel composition, what corresponds to
notions of linear independence and basis? Here, we recall that Yoshida
has developed a set of \emph{combinators} for an asynchronous verison
of Milner's $\pi$-calculus. These are a finite set of processes such
any process can be expressed as parallel composition of these
combinators together with liberal uses of the new operator and
replication. We can simply give a translation of these into the
present calculus and have reasonable expectation that the property
carries over. That is, that the resultant set allows to express all
processes via parallel composition. Note, however, that there is no
new operator or replication in this calculus. As a result, we expect
that the corresponding set is actually infinite. That is, we expect
that the space is actually infinite dimensional.

\begin{remark}
  The attentive reader may be a bit concerned. Certainly, the
  collection $S$, $K$ and $I$ is a finite set of
  combinators. Shouldn't we expect to see a finite set of combinators
  for an effectively equivalent system? i am very sympathetic to this
  critique and feel it warrants full attention. On the other hand, i
  also have in mind the following analogy. The natural numbers, as a
  monoid under addition, has exactly $1$ generator, while the natural
  numbers, as a monoid under multiplication, has countably many
  generators (the primes). We observe that the application of the
  lambda calculus is much less resource sensitive than the parallel
  composition of the $\pi$-calculus. Could it be the case that we have
  an analogy of the form
  
  \begin{mathpar}
    m + n : MN :: m*n : M|N
  \end{mathpar}

  giving a similar blow up in the set of ``primes''?  This is such a
  wonderful thought that, even if it's not true, i think it's worth
  writing down.
\end{remark}
 

\documentclass[12pt]{llncs}
%\documentclass{jktr}

\usepackage[pdftex]{hyperref}                   
\usepackage {listings}
\usepackage {mathpartir}
\usepackage{bcprules}
%\usepackage{listings}
                       
\usepackage{graphicx} 
%\usepackage[margins=2.5cm,nohead,nofoot]{geometry}
%\usepackage{geometry}
\usepackage{amsfonts}
\usepackage{amstext}
\usepackage{latexsym}
\usepackage{amssymb}
\usepackage{color}


%\include{myPreamble}
\include{qm2pi.local} 

%\ifpdf
%\usepackage[pdftex]{graphicx}
%\else
%\usepackage{graphicx}
%\fi

 % \ifpdf
%  \usepackage{pdfsync}
%  \if


%\title{Brief Article}
%\author{David F. Snyder}
%\author{L.G. Meredith}

%\address{Dept. of Math., Texas State University--San Marcos, San Marcos, TX 78666}
       
\pagestyle{empty}


\begin{document}

\lstset{language=[Objective]Caml,frame=shadowbox}

\input{qm2pi.front}

% section front matter (end)

\input{qm2pi.intro} 
 
% section introduction (end)

% \input{qm2pi.knotations} 

% section notation (end)

\input{qm2pi.process.calculi} 

% section concurrent_process_calculi_and_spatial_logics_ (end)
    
%\input{qm2pi.knots2pi} 

%\input{qm2pi.trefoil} 

%\input{qm2pi.mainthm} 

% subsection basic_interpretation (end)

%\input{qm2pi.rho.presentation} 
\subsection{The syntax and semantics of the notation system}\label{sub:the_syntax_and_semantics_of_the_notation_system} % (fold)

We now summarize a technical presentation of the calculus that
embodies our theory of dynamics. The typical presentation of such a
calculus follows the style of giving generators and relations on
them. The grammar, below, describing term constructors, freely
generates the set of processes, $\Proc$. This set is then quotiented
by a relation known as structural congruence and it is over this set
that the notion of dynamics is expressed. This presentation is
essentially that of \cite{MeredithR05} with the addition of
polyadicity and summation. For readability we have relegated some of
the technical subtleties to an appendix.

\subsubsection{Process grammar}\label{subsub:process_grammar}

\begin{mathpar}
  \inferrule* [lab=synchronization] {} {{M} \bc \pzero \;|\; x?F \;|\; x!C }
  \and
  \inferrule* [lab=abstraction] {} {{F} \bc (x)P}
  \and
  \inferrule* [lab=concretion] {} {{C} \bc \langle Q \rangle}
  \and
  \inferrule* [lab=process] {} {{P,Q} \bc M \;| \;P|Q \;|\; @{x}}
  \and
  \inferrule* [lab=name] {} {{x} \bc \quotep{P}}
\end{mathpar} 

Note that $\vec{x}$ (resp. $\vec{P}$) denotes a vector of names
(resp. processes) of length $|\vec{x}|$ (resp. $|\vec{P}|$). We adopt
the following useful abbreviations.

\begin{mathpar}
   x?(\vec{y}).P := x.(\vec{y})P \and  x\clift{\vec{P}} := x.\clift{\vec{P}}
   \and x!(y) := \lift{x}{\dropn{y}}
   \and \Pi_{i=0}^{n-1}P_i := P_0 | \ldots | P_{n-1}
\end{mathpar}

\subsubsection{Structural congruence}

\paragraph{Free and bound names and alpha-equivalence.} At the
core of structural equivalence is alpha-equivalence which identifies
process that are the same up to a change of variable. Formally, we
recognize the distinction between free and bound names. The free names
of a process, $\freenames{P}$, may be calculated recursively as
follows:

\begin{mathpar}
\freenames{\pzero} := \emptyset
  \and \\
  \freenames{x?(y).P} := \{ x \} \cup (\freenames{P} \setminus \{ y \})
  \and 
  \freenames{x!\langle P \rangle} := \{ x \} \cup \{ P \} 
  \and \\
  \freenames{P|Q} := \freenames{P} \cup \freenames{Q}
  \and \\
  \freenames{@{x}} := \{ x \}
\end{mathpar}

$\pi$
$\quotep{\pi}$

$\freenames{-} : \pi \to \mathcal{P}(\quotep{\pi})$

\begin{eqnarray*}
  \freenames{\pzero} & := & \emptyset \\
  \freenames{x?(y).P} & := & \{ x \} \cup (\freenames{P} \setminus \{ y \}) \\
  \freenames{x!\langle P \rangle} & := & \{ x \} \cup \{ P \} \\
  \freenames{P|Q} & := & \freenames{P} \cup \freenames{Q} \\
  \freenames{\dropn{x}} & := & \{ x \}
\end{eqnarray*}

The bound names of a process, $\boundnames{P}$, are those names occurring in $P$
that are not free. For example, in $x?(y).0$, the name $x$ is free, while $y$ is bound.

\begin{mathpar}
  \inferrule* [lab=monoidal-laws] {} { P|Q \equiv Q|P \and P|0 \equiv P \and P|(Q|R) \equiv (P|Q)|R }
\end{mathpar}

\begin{mathpar}
  \inferrule* [lab=alpha-equivalence] {} { (x)P \equiv (y)P\{y/x\} \and y \not\in \freenames{P} }
\end{mathpar}

\begin{definition}
Then two processes, $P,Q$, are alpha-equivalent if $P = Q\{\vec{y}/\vec{x}\}$ for
some $\vec{x} \in \boundnames{Q},\vec{y} \in \boundnames{P}$, where $Q\{\vec{y}/\vec{x}\}$
denotes the capture-avoiding substitution of $\vec{y}$ for $\vec{x}$ in $Q$.
\end{definition}

\begin{definition}
  The {\em structural congruence} \cite{SangiorgiWalker} , $\equiv$,
  between processes is the least congruence containing
  alpha-equivalence, satisfying the abelian monoid laws
  (associativity, commutativity and $\pzero$ as identity) for parallel
  composition $|$ and for summation $+$.
\end{definition}

\subsection{Name equivalence}

We take name equivalence, written $\nameeq$, to be the smallest
equivalence relation generated by the following rules.

\begin{mathpar}
\inferrule*[lab=Quote-drop]
{ }
{ \quotep{@{x}} \nameeq x }

\inferrule*[lab=Struct-equiv]
{ P \scong Q }
{ \quotep{P} \nameeq \quotep{Q} }
\end{mathpar}

The astute reader will have noticed that the mutual recursion of names
and processes imposes a mutual recursion on alpha-equivalence and
structural equivalence via name-equivalence. Fortunately, all of this
works out pleasantly and we may calculate in the natural way, free of
concern. The reader interested in the details is referred to the
appendix \ref{appendix:rho_details}.

\subsection{Substitution}

We use $\Proc$ for the set of processes, $\QProc$ for the set of
names, and $\id{\{}\vec{y} / \vec{x} \id{\}}$ to denote partial maps,
$s : \QProc \rightarrow \QProc$. A map, $s$ lifts, uniquely, to a map
on process terms, $\widehat{s} : \Proc \rightarrow \Proc$ by the
following equations.

\begin{mathpar}
  (0) \psubstp{Q}{P} := 0 \\
  (R \juxtap S) \psubstp{Q}{P}
  :=    
  (R)\psubstp{Q}{P} \juxtap (S) \psubstp{Q}{P} \\
  (x?(y).R) \psubstp{Q}{P}    
  :=    
  (x)\substp{Q}{P} (z)\concat( (R \psubstn{z}{y}) \psubstp{Q}{P} ) \\
  (\lift{x}{R}) \psubstp{Q}{P}  
  :=
  \lift{(x)\substp{Q}{P}}{ R \psubstp{Q}{P} } \\
%   (\dropn{x})  \psubstp{Q}{P}       
%   := 
%   \left\{ 
%     \begin{array}{ccc} 
%       \dropn{\quotep{Q}} & & x \nameeq \quotep{P} \\
%       \dropn{x} & & otherwise \\
%     \end{array}
%   \right. 
  (\dropn{x})  \psubstp{Q}{P}       
  := 
  \left\{ 
    \begin{array}{ccc} 
      Q & & x \nameeq \quotep{P} \\
      \dropn{x} & & otherwise \\
    \end{array}
  \right.
\end{mathpar}
 

where

\begin{eqnarray}
  (x)\id{\{} \lpquote Q \rpquote / \lpquote P \rpquote \id{\}}            = 
  \left\{ 
    \begin{array}{ccc}
      \lpquote Q \rpquote & & x \nameeq \lpquote P \rpquote \\
      x & & otherwise \\
    \end{array}
  \right. \nonumber
\end{eqnarray}

and $z$ is chosen distinct from $\quotep{P}$, $\quotep{Q}$, the free
names in $Q$, and all the names in $R$. Our $\alpha$-equivalence will
be built in the standard way from this substitution.

\begin{remark}\label{rem:no_self_referential_names}
  One consequence of these definitions is that $\forall P. \quotep{P}
  \not\in \freenames{P}$.
\end{remark}

\subsection{ Dynamic quote: an example }

Anticipating something of what's to come, consider applying the
substitution, $\widehat{\id{\{}u / z \id{\}}}$, to the following pair
of processes, $\lift{w}{y!(z)}$ and $w[ \lpquote y!(z) \rpquote ]$.

\begin{eqnarray}
	\lift{w}{y!(z)}\widehat{\id{\{}u / z \id{\}}}
		& = &
		\lift{w}{y!(u)} \nonumber\\
	w[ \lpquote y!(z) \rpquote ] \widehat{ \id{\{}u / z \id{\}} }
		& = &
		w[ \lpquote y!(z) \rpquote ] \nonumber
\end{eqnarray}

Because the body of the process between quotes is impervious to
substitution, we get radically different answers. In fact, by
examining the first process in an input context,
e.g. $x?(z).\lift{w}{y!(z)}$, we see that the process under the lift
operator may be shaped by prefixed inputs binding a name inside it. In
this sense, the lift operator will be seen as a way to dynamically
construct processes before reifying them as names.

Finally equipped with these standard features we can present the
dynamics of the calculus.

\subsubsection{Operational semantics} 

Finally, we introduce the computational dynamics. What marks these
algebras as distinct from other more traditionally studied algebraic
structures, e.g. vector spaces or polynomial rings, is the manner in
which dynamics is captured. In traditional structures, dynamics is typically
expressed through morphisms between such structures, as in linear maps
between vector spaces or morphisms between rings. In algebras
associated with the semantics of computation, the dynamics is
expressed as part of the algebraic structure itself, through a
reduction reduction relation typically denoted by $\red$. Below, we
give a recursive presentation of this relation for the calculus used
in the encoding.

$\red \subseteq \pi \times \pi$
$\red : \pi \to \mathcal{P}(\pi)$

\begin{mathpar}
  \inferrule* [lab=Comm] { \textsf{match}( x_{src}, x_{trgt} ) } { x_{trgt}?(y)P \; | \; x_{src}!\langle {Q} \rangle \red P\{\quotep{Q}/y}\} }
  \and \\
  \inferrule* [lab=Par] {{P} \red {P}'} {{{P} | {Q}} \red {{P}' | {Q}}}
  \and
  \inferrule* [lab=Equiv]{{{P} \scong {P}'} \andalso {{P}' \red {Q}'} \andalso {{Q}' \scong {Q}}}{{P} \red {Q}}
\end{mathpar}

\begin{eqnarray*}
  match_{\equiv} (\quotep{P},\quotep{Q}) & := & P \equiv Q \\
  match_{\dagger}(\quotep{P},\quotep{Q}) & := & \forall R. P|Q \red^{*} R => R \red^{*} 0 \\
  match_{K}(\quotep{P},\quotep{Q}) & := & K \mbox{ for some context } K
\end{eqnarray*}

$u?(x)P | u!\langle Q \rangle \red P\{\quotep{Q}/x\}$

%We write $\wred$ for $\red^*$, and $P\red$ if $\exists Q $ such that $ P \red Q$.
We write $P\red$ if $\exists Q $ such that $ P \red Q$ and $P\not\red$, otherwise.

\section{Replication}

As mentioned before, it is known that replication (and hence
recursion) can be implemented in a higher-order process algebra
\cite{SangiorgiWalker}. As our first example of calculation with the
machinery thus far presented we give the construction explicitly in
the {\rhoc}.

\begin{eqnarray}
	D_{x} & := & \prefix{x}{y}{(\binpar{\outputp{x}{y}}{@{y}})} \nonumber\\
	\bangp_{x}{P} & := & \binpar{{x}!\langle{\binpar{D_{x}}{P}}\rangle}{D_{x}} \nonumber
\end{eqnarray}

\begin{eqnarray}
	\bangp_{x}{P} & & \nonumber\\
	=
	& {x}!\langle{(\prefix{x}{y}{(\outputp{x}{y} | @{y})) | P}}\rangle 
	      | \prefix{x}{y}{(\outputp{x}{y} | @{y})} & \nonumber\\
	\red
	& (\outputp{x}{y} | @{y})\substn{\quotep{(\prefix{x}{y}{(@{y} | \outputp{x}{y})) | P}}}{y} & \nonumber\\
	=
	& \outputp{x}{\quotep{(\prefix{x}{y}{(\outputp{x}{y} | @{y})) | P}}}
	  | {(\prefix{x}{y}{(\outputp{x}{y} | @{y})) | P}} & \nonumber\\
	\red
	& \ldots & \nonumber\\
	\red^*
	& P | P | \ldots & \nonumber
\end{eqnarray}

Of course, this encoding, as an implementation, runs away, unfolding
$\bangp{P}$ eagerly. A lazier and more implementable replication
operator, restricted to input-guarded processes, may be obtained as follows.

\begin{eqnarray}
\bangp{\prefix{u}{v}{P}} 
	:= 
	\binpar{\lift{x}{\prefix{u}{v}{(\binpar{D(x)}{P})}}}{D(x)} \nonumber
\end{eqnarray}

\begin{remark}
  Note that the lazier definition still does not deal with summation
  or mixed summation (i.e. sums over input and output). The reader is
  invited to construct definitions of replication that deal with these
  features. 

  Further, the definitions are parameterized in a name, $x$. Can you,
  gentle reader, make a definition that eliminates this parameter and
  guarantees no accidental interaction between the replication
  machinery and the process being replicated -- i.e. no accidental
  sharing of names used by the process to get its work done and the
  name(s) used by the replication to effect copying. This latter
  revision of the definition of replication is crucial to obtaining
  the expected identity $!!P \sim !P$.
\end{remark}

\begin{remark}\label{rem:paradoxical_combinator}
  The reader familiar with the lambda calculus will have noticed the
  similarity between $D$ and the paradoxical combinator.

  [Ed. note: the existence of this seems to suggest we have to be more
  restrictive on the set of processes and names we admit if we are to
  support no-cloning.]
\end{remark}

\subsubsection{Bisimulation}

The computational dynamics gives rise to another kind of equivalence,
the equivalence of computational behavior. As previously mentioned
this is typically captured \emph{via} some form of bisimulation.

% The notion we use in this paper is weak barbed bisimulation
% \cite{milner91polyadicpi}.

The notion we use in this paper is derived from weak barbed
bisimulation \cite{milner91polyadicpi}. 

\begin{definition}
An \emph{observation relation}, $\downarrow_{\mathcal N}$, over a set
of names, $\mathcal N$, is the smallest relation satisfying the rules
below.

\infrule[Out-barb]{y \in {\mathcal N}, \; x \nameeq y}
		  {\outputp{x}{v} \downarrow_{\mathcal N} x}
\infrule[Par-barb]{\mbox{$P\downarrow_{\mathcal N} x$ or $Q\downarrow_{\mathcal N} x$}}
		  {\binpar{P}{Q} \downarrow_{\mathcal N} x}

We write $P \Downarrow_{\mathcal N} x$ if there is $Q$ such that 
$P \wred Q$ and $Q \downarrow_{\mathcal N} x$.
\end{definition}

\begin{definition}
%\label{def.bbisim}
An  ${\mathcal N}$-\emph{barbed bisimulation} over a set of names, ${\mathcal N}$, is a symmetric binary relation 
${\mathcal S}_{\mathcal N}$ between agents such that $P\rel{S}_{\mathcal N}Q$ implies:
\begin{enumerate}
\item If $P \red P'$ then $Q \wred Q'$ and $P'\rel{S}_{\mathcal N} Q'$.
\item If $P\downarrow_{\mathcal N} x$, then $Q\Downarrow_{\mathcal N} x$.
\end{enumerate}
$P$ is ${\mathcal N}$-barbed bisimilar to $Q$, written
$P \wbbisim_{\mathcal N} Q$, if $P \rel{S}_{\mathcal N} Q$ for some ${\mathcal N}$-barbed bisimulation ${\mathcal S}_{\mathcal N}$.
\end{definition}

$\mathcal{R} \subseteq \pi \times \pi$

$P \mathcal{R} Q => \forall P'. P \red P' \Rightarrow \exists Q'. Q \red Q', P' \mathcal{R} Q'$

$P \vdash x \Rightarrow Q \vdash x$

\begin{mathpar}
  \inferrule*[lab=Out-barb]{x \nameeq y}{{y}!\langle{Q}\rangle \vdash x}
  \and
  \inferrule*[lab=Par-barb]{\mbox{$P\vdash x$ or $Q\vdash x$}}{\binpar{P}{Q} \vdash x}
\end{mathpar}

\subsubsection{Contexts}

One of the principle advantages of computational calculi like the
$\pi$-calculus is a well-defined notion of context,
contextual-equivalence and a correlation between
contextual-equivalence and notions of bisimulation. The notion of
context allows the decomposition of a process into (sub-)process and
its syntactic environment, its context. Thus, a context may be
thought of as a process with a ``hole'' (written $\Box$) in it. The
application of a context $M$ to a process $P$, written $M[P]$, is
tantamount to filling the hole in $M$ with $P$. In this paper we do
not need the full weight of this theory, but do make use of the notion
of context in the proof the main theorem. 

\begin{mathpar}
  \inferrule* [lab=summation] {} {{M_{M},M_{N}} \bc \Box \;|\; x.M_{A} \;|\; M_{M}+M_{N}}
  \and
  \inferrule* [lab=agent] {} {{M_{A}} \bc (\vec{x})M_{P} \;| \; \clift{P_0,\ldots,M_{P},\ldots,P_N}}
  \and \\
  \inferrule* [lab=process] {} {{M_{P}} \bc M_{N} \;| \;P|M_{P} }
\end{mathpar} 

\begin{mathpar}
  \inferrule* [lab=sychronization] {} {M_{N} \bc \Box \;|\; x?M_{F} \;|\; x!M_{C}}
  \and
  \inferrule* [lab=abstraction] {} {{M_{F}} \bc (x)M_{P} }
  \and
  \inferrule* [lab=concretion] {} {{M_{C}} \bc \langle M_{P} \rangle }
  \and \\
  \inferrule* [lab=process] {} {{M_{P}} \bc M_{N} \;| \;P|M_{P} }
\end{mathpar}

\begin{definition}[contextual application] Given a context $M$, and
  process $P$, we define the \emph{contextual application}, $M[P] :=
  M\{P/\Box\}$. That is, the contextual application of M to P is the
  substitution of $P$ for $\Box$ in $M$.
\end{definition}

$\meaningof{-} : L \to \mathcal{P}(\pi)$

\begin{mathpar}
  \inferrule* [lab=collection] {} {\meaningof{true} = \pi, \and \meaningof{~E} = \pi \setminus \meaningof{E}, \and \meaningof{E_{1} \& E_{2}} = \meaningof{E_{1}} \cap \meaningof{E_{2}}}
\end{mathpar}

\begin{mathpar}
  \inferrule* [lab=structure] {} {\meaningof{0} = \{ P \in \pi | P \equiv 0 \}, \and \\ \meaningof{E_1 | E_2} = \{ P \in \pi | P \equiv P_{1} | P_{2}, P_{1} \in \meaningof{E_{1}}, P_{2} \in \meaningof{E_2}\} }
\end{mathpar}

\begin{mathpar}
 \inferrule* [lab=behavior] {} {\meaningof{\langle a?b \rangle E} = \{ P \in \pi | P \equiv Q | u?(y)P', \\ \and \\\\ \and \\ \;\;\; u \in \meaningof{a}, \forall z.P'\{z/y\} \in \meaningof{E\{z/b\}}\}, \and \\ \meaningof{a!E} = \{ P \in \pi | P \equiv Q | x!\langle P' \rangle, x \in \meaningof{a} P' \in \meaningof{E}\} }
\end{mathpar}

\begin{mathpar}
 \inferrule* [lab=nominal] {} {\meaningof{\quotep{E}} = \{ \quotep{P} \in \quotep{\pi} | P \in \meaningof{E} \}, \and \meaningof{\quotep{P}} = \{ \quotep{Q} \in \quotep{\pi} | P \equiv Q \} \and \\ \meaningof{@\quotep{E}} = \{ P \in \pi | P \equiv @x, x \in \meaningof{E} \}}
\end{mathpar}

\begin{eqnarray*}
  \\
  \meaningof{-} : TS \to ST
\end{eqnarray*}

\begin{eqnarray*}
  \\
  L : TS \to ST
\end{eqnarray*}

\begin{eqnarray*}
  \\
  P \models E \iff P \in \meaningof{E}
\end{eqnarray*}

\begin{eqnarray*}
  P \approx_{L} Q \iff \forall E \in L. P \models E \iff Q \models E
\end{eqnarray*}

\begin{eqnarray*}
  P \approx_{K} Q
\end{eqnarray*}

\begin{eqnarray*}
  P \approx Q
\end{eqnarray*}

$\approx_{K} = \approx = \approx_{L}$

\subsubsection{Contextual duality}

Note that contexts extend the quotation operation to a family of
operations from processes to names. Given a context, $M$, we can
define a \emph{nominal context}, $\quotep{M}$ by $\quotep{M}[P] :=
\quotep{M[P]}$. To foreshadow what is to come we observe that these
operations enjoy a duality with processes very much like the duality
between vectors and maps from vectors to scalars.

Further, because the calculus is essentially higher-order, we have a
correspondence between contexts and processes. More specifically,
given a name $x$ and a context $M$ we can construct $M^{*}_{x}$ such
that 

\begin{mathpar}
  M^{*}_{x} | \lift{x}{P} \red M[P]
\end{mathpar}

namely,

\begin{mathpar}
  M^{*}_{x} := x?(u).M[\dropn{u}]
\end{mathpar}

The dependence of $M^{*}_{x}$ on a name makes it an abstraction, 

\begin{mathpar}
  M^{*} := (x)x?(u).M[\dropn{u}]
\end{mathpar}

\subsection{Additional notation}

It will sometimes be convenient to denote the process a name
quotes. We already have the notation $x = \quotep{P}$, but it will be
convenient to introduce an alternate notation, $\procn{x}$, when we
want to emphasize the connection to the use of the name. Note that, by
virtue of name equivalence, $\quotep{\procn{x}} \nameeq x$; so, the
notation is consistent with previous definitions.

Further, because names have structure it is possible to effect
substitutions on the basis of that structure. This means we need to
upgrade our notation for substitutions, which we accomplish by
adapting comprehension notation. Thus,

\begin{mathpar}
  P\{ y / x : x \in S \}
\end{mathpar}

is interpreted to mean the process derived from P by replacing (in a
capture-avoiding manner) each occurrence of $x$ in $S$ by $y$. For example,

\begin{mathpar}
  P\{ \quotep{\procn{x}|\procn{x}} / x : x \in \freenames{P} \}
\end{mathpar}

will replace each (occurrence) of a free name $x$ in $P$ by
$\quotep{\procn{x}|\procn{x}}$.

Also, we will avail ourselves of the notation $x^{L}$ and $x^{R}$ to
denote injections of a name into disjoint copies of the name
space. There are numerous ways to accomplish this. One example can be
found in \cite{MeredithR05}. This notation overloads to vectors of
names: $\vec{x}^{\pi} := (x_{i}^{\pi} \; : \; 0 \leq i < |\vec{x}| )$ where $\pi \in \{L,R\}$.

We also use $P^{\Box} := P|\Box$.

In \cite{MeredithR05} an interpretation of the new operator is
given. It turns out that there are several possible interpretations
all enjoying the requisite algebraic properties of the operator (see
\cite{milner91polyadicpi}). We will therefore make liberal use of
$(\nu\; \vec{x})P$.

% subsection the_syntax_and_semantics_of_the_notation_system (end)   

\input{qm2pi.qmops} 

\input{qm2pi.sterngerlach} 

\input{qm2pi.metric} 

% section concurrent_process_calculi (end)

%\input{qm2pi.proofsketch}

% section proof sketch (end)

%\input{qm2pi.slviaknots} 

% section spatial logic via knots (end)

\input{qm2pi.conclusion}

% section conclusion (end)

%\input{qm2pi.dtcodes} 

% section wiring algorithm (end)

\input{qm2pi.ack} 

% section acknowledgments (end)

\newpage


\bibliographystyle{plain}   
\bibliography{../../biblios/main.bib}

\input{qm2pi.rhodetails}

\end{document}

 

\documentclass[12pt]{llncs}
%\documentclass{jktr}

\usepackage[pdftex]{hyperref}                   
\usepackage {listings}
\usepackage {mathpartir}
\usepackage{bcprules}
%\usepackage{listings}
                       
\usepackage{graphicx} 
%\usepackage[margins=2.5cm,nohead,nofoot]{geometry}
%\usepackage{geometry}
\usepackage{amsfonts}
\usepackage{amstext}
\usepackage{latexsym}
\usepackage{amssymb}
\usepackage{color}


%\include{myPreamble}
\include{qm2pi.local} 

%\ifpdf
%\usepackage[pdftex]{graphicx}
%\else
%\usepackage{graphicx}
%\fi

 % \ifpdf
%  \usepackage{pdfsync}
%  \if


%\title{Brief Article}
%\author{David F. Snyder}
%\author{L.G. Meredith}

%\address{Dept. of Math., Texas State University--San Marcos, San Marcos, TX 78666}
       
\pagestyle{empty}


\begin{document}

\lstset{language=[Objective]Caml,frame=shadowbox}

\input{qm2pi.front}

% section front matter (end)

\input{qm2pi.intro} 
 
% section introduction (end)

% \input{qm2pi.knotations} 

% section notation (end)

\input{qm2pi.process.calculi} 

% section concurrent_process_calculi_and_spatial_logics_ (end)
    
%\input{qm2pi.knots2pi} 

%\input{qm2pi.trefoil} 

%\input{qm2pi.mainthm} 

% subsection basic_interpretation (end)

%\input{qm2pi.rho.presentation} 
\subsection{The syntax and semantics of the notation system}\label{sub:the_syntax_and_semantics_of_the_notation_system} % (fold)

We now summarize a technical presentation of the calculus that
embodies our theory of dynamics. The typical presentation of such a
calculus follows the style of giving generators and relations on
them. The grammar, below, describing term constructors, freely
generates the set of processes, $\Proc$. This set is then quotiented
by a relation known as structural congruence and it is over this set
that the notion of dynamics is expressed. This presentation is
essentially that of \cite{MeredithR05} with the addition of
polyadicity and summation. For readability we have relegated some of
the technical subtleties to an appendix.

\subsubsection{Process grammar}\label{subsub:process_grammar}

\begin{mathpar}
  \inferrule* [lab=synchronization] {} {{M} \bc \pzero \;|\; x?F \;|\; x!C }
  \and
  \inferrule* [lab=abstraction] {} {{F} \bc (x)P}
  \and
  \inferrule* [lab=concretion] {} {{C} \bc \langle Q \rangle}
  \and
  \inferrule* [lab=process] {} {{P,Q} \bc M \;| \;P|Q \;|\; @{x}}
  \and
  \inferrule* [lab=name] {} {{x} \bc \quotep{P}}
\end{mathpar} 

Note that $\vec{x}$ (resp. $\vec{P}$) denotes a vector of names
(resp. processes) of length $|\vec{x}|$ (resp. $|\vec{P}|$). We adopt
the following useful abbreviations.

\begin{mathpar}
   x?(\vec{y}).P := x.(\vec{y})P \and  x\clift{\vec{P}} := x.\clift{\vec{P}}
   \and x!(y) := \lift{x}{\dropn{y}}
   \and \Pi_{i=0}^{n-1}P_i := P_0 | \ldots | P_{n-1}
\end{mathpar}

\subsubsection{Structural congruence}

\paragraph{Free and bound names and alpha-equivalence.} At the
core of structural equivalence is alpha-equivalence which identifies
process that are the same up to a change of variable. Formally, we
recognize the distinction between free and bound names. The free names
of a process, $\freenames{P}$, may be calculated recursively as
follows:

\begin{mathpar}
\freenames{\pzero} := \emptyset
  \and \\
  \freenames{x?(y).P} := \{ x \} \cup (\freenames{P} \setminus \{ y \})
  \and 
  \freenames{x!\langle P \rangle} := \{ x \} \cup \{ P \} 
  \and \\
  \freenames{P|Q} := \freenames{P} \cup \freenames{Q}
  \and \\
  \freenames{@{x}} := \{ x \}
\end{mathpar}

$\pi$
$\quotep{\pi}$

$\freenames{-} : \pi \to \mathcal{P}(\quotep{\pi})$

\begin{eqnarray*}
  \freenames{\pzero} & := & \emptyset \\
  \freenames{x?(y).P} & := & \{ x \} \cup (\freenames{P} \setminus \{ y \}) \\
  \freenames{x!\langle P \rangle} & := & \{ x \} \cup \{ P \} \\
  \freenames{P|Q} & := & \freenames{P} \cup \freenames{Q} \\
  \freenames{\dropn{x}} & := & \{ x \}
\end{eqnarray*}

The bound names of a process, $\boundnames{P}$, are those names occurring in $P$
that are not free. For example, in $x?(y).0$, the name $x$ is free, while $y$ is bound.

\begin{mathpar}
  \inferrule* [lab=monoidal-laws] {} { P|Q \equiv Q|P \and P|0 \equiv P \and P|(Q|R) \equiv (P|Q)|R }
\end{mathpar}

\begin{mathpar}
  \inferrule* [lab=alpha-equivalence] {} { (x)P \equiv (y)P\{y/x\} \and y \not\in \freenames{P} }
\end{mathpar}

\begin{definition}
Then two processes, $P,Q$, are alpha-equivalent if $P = Q\{\vec{y}/\vec{x}\}$ for
some $\vec{x} \in \boundnames{Q},\vec{y} \in \boundnames{P}$, where $Q\{\vec{y}/\vec{x}\}$
denotes the capture-avoiding substitution of $\vec{y}$ for $\vec{x}$ in $Q$.
\end{definition}

\begin{definition}
  The {\em structural congruence} \cite{SangiorgiWalker} , $\equiv$,
  between processes is the least congruence containing
  alpha-equivalence, satisfying the abelian monoid laws
  (associativity, commutativity and $\pzero$ as identity) for parallel
  composition $|$ and for summation $+$.
\end{definition}

\subsection{Name equivalence}

We take name equivalence, written $\nameeq$, to be the smallest
equivalence relation generated by the following rules.

\begin{mathpar}
\inferrule*[lab=Quote-drop]
{ }
{ \quotep{@{x}} \nameeq x }

\inferrule*[lab=Struct-equiv]
{ P \scong Q }
{ \quotep{P} \nameeq \quotep{Q} }
\end{mathpar}

The astute reader will have noticed that the mutual recursion of names
and processes imposes a mutual recursion on alpha-equivalence and
structural equivalence via name-equivalence. Fortunately, all of this
works out pleasantly and we may calculate in the natural way, free of
concern. The reader interested in the details is referred to the
appendix \ref{appendix:rho_details}.

\subsection{Substitution}

We use $\Proc$ for the set of processes, $\QProc$ for the set of
names, and $\id{\{}\vec{y} / \vec{x} \id{\}}$ to denote partial maps,
$s : \QProc \rightarrow \QProc$. A map, $s$ lifts, uniquely, to a map
on process terms, $\widehat{s} : \Proc \rightarrow \Proc$ by the
following equations.

\begin{mathpar}
  (0) \psubstp{Q}{P} := 0 \\
  (R \juxtap S) \psubstp{Q}{P}
  :=    
  (R)\psubstp{Q}{P} \juxtap (S) \psubstp{Q}{P} \\
  (x?(y).R) \psubstp{Q}{P}    
  :=    
  (x)\substp{Q}{P} (z)\concat( (R \psubstn{z}{y}) \psubstp{Q}{P} ) \\
  (\lift{x}{R}) \psubstp{Q}{P}  
  :=
  \lift{(x)\substp{Q}{P}}{ R \psubstp{Q}{P} } \\
%   (\dropn{x})  \psubstp{Q}{P}       
%   := 
%   \left\{ 
%     \begin{array}{ccc} 
%       \dropn{\quotep{Q}} & & x \nameeq \quotep{P} \\
%       \dropn{x} & & otherwise \\
%     \end{array}
%   \right. 
  (\dropn{x})  \psubstp{Q}{P}       
  := 
  \left\{ 
    \begin{array}{ccc} 
      Q & & x \nameeq \quotep{P} \\
      \dropn{x} & & otherwise \\
    \end{array}
  \right.
\end{mathpar}
 

where

\begin{eqnarray}
  (x)\id{\{} \lpquote Q \rpquote / \lpquote P \rpquote \id{\}}            = 
  \left\{ 
    \begin{array}{ccc}
      \lpquote Q \rpquote & & x \nameeq \lpquote P \rpquote \\
      x & & otherwise \\
    \end{array}
  \right. \nonumber
\end{eqnarray}

and $z$ is chosen distinct from $\quotep{P}$, $\quotep{Q}$, the free
names in $Q$, and all the names in $R$. Our $\alpha$-equivalence will
be built in the standard way from this substitution.

\begin{remark}\label{rem:no_self_referential_names}
  One consequence of these definitions is that $\forall P. \quotep{P}
  \not\in \freenames{P}$.
\end{remark}

\subsection{ Dynamic quote: an example }

Anticipating something of what's to come, consider applying the
substitution, $\widehat{\id{\{}u / z \id{\}}}$, to the following pair
of processes, $\lift{w}{y!(z)}$ and $w[ \lpquote y!(z) \rpquote ]$.

\begin{eqnarray}
	\lift{w}{y!(z)}\widehat{\id{\{}u / z \id{\}}}
		& = &
		\lift{w}{y!(u)} \nonumber\\
	w[ \lpquote y!(z) \rpquote ] \widehat{ \id{\{}u / z \id{\}} }
		& = &
		w[ \lpquote y!(z) \rpquote ] \nonumber
\end{eqnarray}

Because the body of the process between quotes is impervious to
substitution, we get radically different answers. In fact, by
examining the first process in an input context,
e.g. $x?(z).\lift{w}{y!(z)}$, we see that the process under the lift
operator may be shaped by prefixed inputs binding a name inside it. In
this sense, the lift operator will be seen as a way to dynamically
construct processes before reifying them as names.

Finally equipped with these standard features we can present the
dynamics of the calculus.

\subsubsection{Operational semantics} 

Finally, we introduce the computational dynamics. What marks these
algebras as distinct from other more traditionally studied algebraic
structures, e.g. vector spaces or polynomial rings, is the manner in
which dynamics is captured. In traditional structures, dynamics is typically
expressed through morphisms between such structures, as in linear maps
between vector spaces or morphisms between rings. In algebras
associated with the semantics of computation, the dynamics is
expressed as part of the algebraic structure itself, through a
reduction reduction relation typically denoted by $\red$. Below, we
give a recursive presentation of this relation for the calculus used
in the encoding.

$\red \subseteq \pi \times \pi$
$\red : \pi \to \mathcal{P}(\pi)$

\begin{mathpar}
  \inferrule* [lab=Comm] { \textsf{match}( x_{src}, x_{trgt} ) } { x_{trgt}?(y)P \; | \; x_{src}!\langle {Q} \rangle \red P\{\quotep{Q}/y}\} }
  \and \\
  \inferrule* [lab=Par] {{P} \red {P}'} {{{P} | {Q}} \red {{P}' | {Q}}}
  \and
  \inferrule* [lab=Equiv]{{{P} \scong {P}'} \andalso {{P}' \red {Q}'} \andalso {{Q}' \scong {Q}}}{{P} \red {Q}}
\end{mathpar}

\begin{eqnarray*}
  match_{\equiv} (\quotep{P},\quotep{Q}) & := & P \equiv Q \\
  match_{\dagger}(\quotep{P},\quotep{Q}) & := & \forall R. P|Q \red^{*} R => R \red^{*} 0 \\
  match_{K}(\quotep{P},\quotep{Q}) & := & K \mbox{ for some context } K
\end{eqnarray*}

$u?(x)P | u!\langle Q \rangle \red P\{\quotep{Q}/x\}$

%We write $\wred$ for $\red^*$, and $P\red$ if $\exists Q $ such that $ P \red Q$.
We write $P\red$ if $\exists Q $ such that $ P \red Q$ and $P\not\red$, otherwise.

\section{Replication}

As mentioned before, it is known that replication (and hence
recursion) can be implemented in a higher-order process algebra
\cite{SangiorgiWalker}. As our first example of calculation with the
machinery thus far presented we give the construction explicitly in
the {\rhoc}.

\begin{eqnarray}
	D_{x} & := & \prefix{x}{y}{(\binpar{\outputp{x}{y}}{@{y}})} \nonumber\\
	\bangp_{x}{P} & := & \binpar{{x}!\langle{\binpar{D_{x}}{P}}\rangle}{D_{x}} \nonumber
\end{eqnarray}

\begin{eqnarray}
	\bangp_{x}{P} & & \nonumber\\
	=
	& {x}!\langle{(\prefix{x}{y}{(\outputp{x}{y} | @{y})) | P}}\rangle 
	      | \prefix{x}{y}{(\outputp{x}{y} | @{y})} & \nonumber\\
	\red
	& (\outputp{x}{y} | @{y})\substn{\quotep{(\prefix{x}{y}{(@{y} | \outputp{x}{y})) | P}}}{y} & \nonumber\\
	=
	& \outputp{x}{\quotep{(\prefix{x}{y}{(\outputp{x}{y} | @{y})) | P}}}
	  | {(\prefix{x}{y}{(\outputp{x}{y} | @{y})) | P}} & \nonumber\\
	\red
	& \ldots & \nonumber\\
	\red^*
	& P | P | \ldots & \nonumber
\end{eqnarray}

Of course, this encoding, as an implementation, runs away, unfolding
$\bangp{P}$ eagerly. A lazier and more implementable replication
operator, restricted to input-guarded processes, may be obtained as follows.

\begin{eqnarray}
\bangp{\prefix{u}{v}{P}} 
	:= 
	\binpar{\lift{x}{\prefix{u}{v}{(\binpar{D(x)}{P})}}}{D(x)} \nonumber
\end{eqnarray}

\begin{remark}
  Note that the lazier definition still does not deal with summation
  or mixed summation (i.e. sums over input and output). The reader is
  invited to construct definitions of replication that deal with these
  features. 

  Further, the definitions are parameterized in a name, $x$. Can you,
  gentle reader, make a definition that eliminates this parameter and
  guarantees no accidental interaction between the replication
  machinery and the process being replicated -- i.e. no accidental
  sharing of names used by the process to get its work done and the
  name(s) used by the replication to effect copying. This latter
  revision of the definition of replication is crucial to obtaining
  the expected identity $!!P \sim !P$.
\end{remark}

\begin{remark}\label{rem:paradoxical_combinator}
  The reader familiar with the lambda calculus will have noticed the
  similarity between $D$ and the paradoxical combinator.

  [Ed. note: the existence of this seems to suggest we have to be more
  restrictive on the set of processes and names we admit if we are to
  support no-cloning.]
\end{remark}

\subsubsection{Bisimulation}

The computational dynamics gives rise to another kind of equivalence,
the equivalence of computational behavior. As previously mentioned
this is typically captured \emph{via} some form of bisimulation.

% The notion we use in this paper is weak barbed bisimulation
% \cite{milner91polyadicpi}.

The notion we use in this paper is derived from weak barbed
bisimulation \cite{milner91polyadicpi}. 

\begin{definition}
An \emph{observation relation}, $\downarrow_{\mathcal N}$, over a set
of names, $\mathcal N$, is the smallest relation satisfying the rules
below.

\infrule[Out-barb]{y \in {\mathcal N}, \; x \nameeq y}
		  {\outputp{x}{v} \downarrow_{\mathcal N} x}
\infrule[Par-barb]{\mbox{$P\downarrow_{\mathcal N} x$ or $Q\downarrow_{\mathcal N} x$}}
		  {\binpar{P}{Q} \downarrow_{\mathcal N} x}

We write $P \Downarrow_{\mathcal N} x$ if there is $Q$ such that 
$P \wred Q$ and $Q \downarrow_{\mathcal N} x$.
\end{definition}

\begin{definition}
%\label{def.bbisim}
An  ${\mathcal N}$-\emph{barbed bisimulation} over a set of names, ${\mathcal N}$, is a symmetric binary relation 
${\mathcal S}_{\mathcal N}$ between agents such that $P\rel{S}_{\mathcal N}Q$ implies:
\begin{enumerate}
\item If $P \red P'$ then $Q \wred Q'$ and $P'\rel{S}_{\mathcal N} Q'$.
\item If $P\downarrow_{\mathcal N} x$, then $Q\Downarrow_{\mathcal N} x$.
\end{enumerate}
$P$ is ${\mathcal N}$-barbed bisimilar to $Q$, written
$P \wbbisim_{\mathcal N} Q$, if $P \rel{S}_{\mathcal N} Q$ for some ${\mathcal N}$-barbed bisimulation ${\mathcal S}_{\mathcal N}$.
\end{definition}

$\mathcal{R} \subseteq \pi \times \pi$

$P \mathcal{R} Q => \forall P'. P \red P' \Rightarrow \exists Q'. Q \red Q', P' \mathcal{R} Q'$

$P \vdash x \Rightarrow Q \vdash x$

\begin{mathpar}
  \inferrule*[lab=Out-barb]{x \nameeq y}{{y}!\langle{Q}\rangle \vdash x}
  \and
  \inferrule*[lab=Par-barb]{\mbox{$P\vdash x$ or $Q\vdash x$}}{\binpar{P}{Q} \vdash x}
\end{mathpar}

\subsubsection{Contexts}

One of the principle advantages of computational calculi like the
$\pi$-calculus is a well-defined notion of context,
contextual-equivalence and a correlation between
contextual-equivalence and notions of bisimulation. The notion of
context allows the decomposition of a process into (sub-)process and
its syntactic environment, its context. Thus, a context may be
thought of as a process with a ``hole'' (written $\Box$) in it. The
application of a context $M$ to a process $P$, written $M[P]$, is
tantamount to filling the hole in $M$ with $P$. In this paper we do
not need the full weight of this theory, but do make use of the notion
of context in the proof the main theorem. 

\begin{mathpar}
  \inferrule* [lab=summation] {} {{M_{M},M_{N}} \bc \Box \;|\; x.M_{A} \;|\; M_{M}+M_{N}}
  \and
  \inferrule* [lab=agent] {} {{M_{A}} \bc (\vec{x})M_{P} \;| \; \clift{P_0,\ldots,M_{P},\ldots,P_N}}
  \and \\
  \inferrule* [lab=process] {} {{M_{P}} \bc M_{N} \;| \;P|M_{P} }
\end{mathpar} 

\begin{mathpar}
  \inferrule* [lab=sychronization] {} {M_{N} \bc \Box \;|\; x?M_{F} \;|\; x!M_{C}}
  \and
  \inferrule* [lab=abstraction] {} {{M_{F}} \bc (x)M_{P} }
  \and
  \inferrule* [lab=concretion] {} {{M_{C}} \bc \langle M_{P} \rangle }
  \and \\
  \inferrule* [lab=process] {} {{M_{P}} \bc M_{N} \;| \;P|M_{P} }
\end{mathpar}

\begin{definition}[contextual application] Given a context $M$, and
  process $P$, we define the \emph{contextual application}, $M[P] :=
  M\{P/\Box\}$. That is, the contextual application of M to P is the
  substitution of $P$ for $\Box$ in $M$.
\end{definition}

$\meaningof{-} : L \to \mathcal{P}(\pi)$

\begin{mathpar}
  \inferrule* [lab=collection] {} {\meaningof{true} = \pi, \and \meaningof{~E} = \pi \setminus \meaningof{E}, \and \meaningof{E_{1} \& E_{2}} = \meaningof{E_{1}} \cap \meaningof{E_{2}}}
\end{mathpar}

\begin{mathpar}
  \inferrule* [lab=structure] {} {\meaningof{0} = \{ P \in \pi | P \equiv 0 \}, \and \\ \meaningof{E_1 | E_2} = \{ P \in \pi | P \equiv P_{1} | P_{2}, P_{1} \in \meaningof{E_{1}}, P_{2} \in \meaningof{E_2}\} }
\end{mathpar}

\begin{mathpar}
 \inferrule* [lab=behavior] {} {\meaningof{\langle a?b \rangle E} = \{ P \in \pi | P \equiv Q | u?(y)P', \\ \and \\\\ \and \\ \;\;\; u \in \meaningof{a}, \forall z.P'\{z/y\} \in \meaningof{E\{z/b\}}\}, \and \\ \meaningof{a!E} = \{ P \in \pi | P \equiv Q | x!\langle P' \rangle, x \in \meaningof{a} P' \in \meaningof{E}\} }
\end{mathpar}

\begin{mathpar}
 \inferrule* [lab=nominal] {} {\meaningof{\quotep{E}} = \{ \quotep{P} \in \quotep{\pi} | P \in \meaningof{E} \}, \and \meaningof{\quotep{P}} = \{ \quotep{Q} \in \quotep{\pi} | P \equiv Q \} \and \\ \meaningof{@\quotep{E}} = \{ P \in \pi | P \equiv @x, x \in \meaningof{E} \}}
\end{mathpar}

\begin{eqnarray*}
  \\
  \meaningof{-} : TS \to ST
\end{eqnarray*}

\begin{eqnarray*}
  \\
  L : TS \to ST
\end{eqnarray*}

\begin{eqnarray*}
  \\
  P \models E \iff P \in \meaningof{E}
\end{eqnarray*}

\begin{eqnarray*}
  P \approx_{L} Q \iff \forall E \in L. P \models E \iff Q \models E
\end{eqnarray*}

\begin{eqnarray*}
  P \approx_{K} Q
\end{eqnarray*}

\begin{eqnarray*}
  P \approx Q
\end{eqnarray*}

$\approx_{K} = \approx = \approx_{L}$

\subsubsection{Contextual duality}

Note that contexts extend the quotation operation to a family of
operations from processes to names. Given a context, $M$, we can
define a \emph{nominal context}, $\quotep{M}$ by $\quotep{M}[P] :=
\quotep{M[P]}$. To foreshadow what is to come we observe that these
operations enjoy a duality with processes very much like the duality
between vectors and maps from vectors to scalars.

Further, because the calculus is essentially higher-order, we have a
correspondence between contexts and processes. More specifically,
given a name $x$ and a context $M$ we can construct $M^{*}_{x}$ such
that 

\begin{mathpar}
  M^{*}_{x} | \lift{x}{P} \red M[P]
\end{mathpar}

namely,

\begin{mathpar}
  M^{*}_{x} := x?(u).M[\dropn{u}]
\end{mathpar}

The dependence of $M^{*}_{x}$ on a name makes it an abstraction, 

\begin{mathpar}
  M^{*} := (x)x?(u).M[\dropn{u}]
\end{mathpar}

\subsection{Additional notation}

It will sometimes be convenient to denote the process a name
quotes. We already have the notation $x = \quotep{P}$, but it will be
convenient to introduce an alternate notation, $\procn{x}$, when we
want to emphasize the connection to the use of the name. Note that, by
virtue of name equivalence, $\quotep{\procn{x}} \nameeq x$; so, the
notation is consistent with previous definitions.

Further, because names have structure it is possible to effect
substitutions on the basis of that structure. This means we need to
upgrade our notation for substitutions, which we accomplish by
adapting comprehension notation. Thus,

\begin{mathpar}
  P\{ y / x : x \in S \}
\end{mathpar}

is interpreted to mean the process derived from P by replacing (in a
capture-avoiding manner) each occurrence of $x$ in $S$ by $y$. For example,

\begin{mathpar}
  P\{ \quotep{\procn{x}|\procn{x}} / x : x \in \freenames{P} \}
\end{mathpar}

will replace each (occurrence) of a free name $x$ in $P$ by
$\quotep{\procn{x}|\procn{x}}$.

Also, we will avail ourselves of the notation $x^{L}$ and $x^{R}$ to
denote injections of a name into disjoint copies of the name
space. There are numerous ways to accomplish this. One example can be
found in \cite{MeredithR05}. This notation overloads to vectors of
names: $\vec{x}^{\pi} := (x_{i}^{\pi} \; : \; 0 \leq i < |\vec{x}| )$ where $\pi \in \{L,R\}$.

We also use $P^{\Box} := P|\Box$.

In \cite{MeredithR05} an interpretation of the new operator is
given. It turns out that there are several possible interpretations
all enjoying the requisite algebraic properties of the operator (see
\cite{milner91polyadicpi}). We will therefore make liberal use of
$(\nu\; \vec{x})P$.

% subsection the_syntax_and_semantics_of_the_notation_system (end)   

\input{qm2pi.qmops} 

\input{qm2pi.sterngerlach} 

\input{qm2pi.metric} 

% section concurrent_process_calculi (end)

%\input{qm2pi.proofsketch}

% section proof sketch (end)

%\input{qm2pi.slviaknots} 

% section spatial logic via knots (end)

\input{qm2pi.conclusion}

% section conclusion (end)

%\input{qm2pi.dtcodes} 

% section wiring algorithm (end)

\input{qm2pi.ack} 

% section acknowledgments (end)

\newpage


\bibliographystyle{plain}   
\bibliography{../../biblios/main.bib}

\input{qm2pi.rhodetails}

\end{document}

 

% section concurrent_process_calculi (end)

%\documentclass[12pt]{llncs}
%\documentclass{jktr}

\usepackage[pdftex]{hyperref}                   
\usepackage {listings}
\usepackage {mathpartir}
\usepackage{bcprules}
%\usepackage{listings}
                       
\usepackage{graphicx} 
%\usepackage[margins=2.5cm,nohead,nofoot]{geometry}
%\usepackage{geometry}
\usepackage{amsfonts}
\usepackage{amstext}
\usepackage{latexsym}
\usepackage{amssymb}
\usepackage{color}


%\include{myPreamble}
\include{qm2pi.local} 

%\ifpdf
%\usepackage[pdftex]{graphicx}
%\else
%\usepackage{graphicx}
%\fi

 % \ifpdf
%  \usepackage{pdfsync}
%  \if


%\title{Brief Article}
%\author{David F. Snyder}
%\author{L.G. Meredith}

%\address{Dept. of Math., Texas State University--San Marcos, San Marcos, TX 78666}
       
\pagestyle{empty}


\begin{document}

\lstset{language=[Objective]Caml,frame=shadowbox}

\input{qm2pi.front}

% section front matter (end)

\input{qm2pi.intro} 
 
% section introduction (end)

% \input{qm2pi.knotations} 

% section notation (end)

\input{qm2pi.process.calculi} 

% section concurrent_process_calculi_and_spatial_logics_ (end)
    
%\input{qm2pi.knots2pi} 

%\input{qm2pi.trefoil} 

%\input{qm2pi.mainthm} 

% subsection basic_interpretation (end)

%\input{qm2pi.rho.presentation} 
\subsection{The syntax and semantics of the notation system}\label{sub:the_syntax_and_semantics_of_the_notation_system} % (fold)

We now summarize a technical presentation of the calculus that
embodies our theory of dynamics. The typical presentation of such a
calculus follows the style of giving generators and relations on
them. The grammar, below, describing term constructors, freely
generates the set of processes, $\Proc$. This set is then quotiented
by a relation known as structural congruence and it is over this set
that the notion of dynamics is expressed. This presentation is
essentially that of \cite{MeredithR05} with the addition of
polyadicity and summation. For readability we have relegated some of
the technical subtleties to an appendix.

\subsubsection{Process grammar}\label{subsub:process_grammar}

\begin{mathpar}
  \inferrule* [lab=synchronization] {} {{M} \bc \pzero \;|\; x?F \;|\; x!C }
  \and
  \inferrule* [lab=abstraction] {} {{F} \bc (x)P}
  \and
  \inferrule* [lab=concretion] {} {{C} \bc \langle Q \rangle}
  \and
  \inferrule* [lab=process] {} {{P,Q} \bc M \;| \;P|Q \;|\; @{x}}
  \and
  \inferrule* [lab=name] {} {{x} \bc \quotep{P}}
\end{mathpar} 

Note that $\vec{x}$ (resp. $\vec{P}$) denotes a vector of names
(resp. processes) of length $|\vec{x}|$ (resp. $|\vec{P}|$). We adopt
the following useful abbreviations.

\begin{mathpar}
   x?(\vec{y}).P := x.(\vec{y})P \and  x\clift{\vec{P}} := x.\clift{\vec{P}}
   \and x!(y) := \lift{x}{\dropn{y}}
   \and \Pi_{i=0}^{n-1}P_i := P_0 | \ldots | P_{n-1}
\end{mathpar}

\subsubsection{Structural congruence}

\paragraph{Free and bound names and alpha-equivalence.} At the
core of structural equivalence is alpha-equivalence which identifies
process that are the same up to a change of variable. Formally, we
recognize the distinction between free and bound names. The free names
of a process, $\freenames{P}$, may be calculated recursively as
follows:

\begin{mathpar}
\freenames{\pzero} := \emptyset
  \and \\
  \freenames{x?(y).P} := \{ x \} \cup (\freenames{P} \setminus \{ y \})
  \and 
  \freenames{x!\langle P \rangle} := \{ x \} \cup \{ P \} 
  \and \\
  \freenames{P|Q} := \freenames{P} \cup \freenames{Q}
  \and \\
  \freenames{@{x}} := \{ x \}
\end{mathpar}

$\pi$
$\quotep{\pi}$

$\freenames{-} : \pi \to \mathcal{P}(\quotep{\pi})$

\begin{eqnarray*}
  \freenames{\pzero} & := & \emptyset \\
  \freenames{x?(y).P} & := & \{ x \} \cup (\freenames{P} \setminus \{ y \}) \\
  \freenames{x!\langle P \rangle} & := & \{ x \} \cup \{ P \} \\
  \freenames{P|Q} & := & \freenames{P} \cup \freenames{Q} \\
  \freenames{\dropn{x}} & := & \{ x \}
\end{eqnarray*}

The bound names of a process, $\boundnames{P}$, are those names occurring in $P$
that are not free. For example, in $x?(y).0$, the name $x$ is free, while $y$ is bound.

\begin{mathpar}
  \inferrule* [lab=monoidal-laws] {} { P|Q \equiv Q|P \and P|0 \equiv P \and P|(Q|R) \equiv (P|Q)|R }
\end{mathpar}

\begin{mathpar}
  \inferrule* [lab=alpha-equivalence] {} { (x)P \equiv (y)P\{y/x\} \and y \not\in \freenames{P} }
\end{mathpar}

\begin{definition}
Then two processes, $P,Q$, are alpha-equivalent if $P = Q\{\vec{y}/\vec{x}\}$ for
some $\vec{x} \in \boundnames{Q},\vec{y} \in \boundnames{P}$, where $Q\{\vec{y}/\vec{x}\}$
denotes the capture-avoiding substitution of $\vec{y}$ for $\vec{x}$ in $Q$.
\end{definition}

\begin{definition}
  The {\em structural congruence} \cite{SangiorgiWalker} , $\equiv$,
  between processes is the least congruence containing
  alpha-equivalence, satisfying the abelian monoid laws
  (associativity, commutativity and $\pzero$ as identity) for parallel
  composition $|$ and for summation $+$.
\end{definition}

\subsection{Name equivalence}

We take name equivalence, written $\nameeq$, to be the smallest
equivalence relation generated by the following rules.

\begin{mathpar}
\inferrule*[lab=Quote-drop]
{ }
{ \quotep{@{x}} \nameeq x }

\inferrule*[lab=Struct-equiv]
{ P \scong Q }
{ \quotep{P} \nameeq \quotep{Q} }
\end{mathpar}

The astute reader will have noticed that the mutual recursion of names
and processes imposes a mutual recursion on alpha-equivalence and
structural equivalence via name-equivalence. Fortunately, all of this
works out pleasantly and we may calculate in the natural way, free of
concern. The reader interested in the details is referred to the
appendix \ref{appendix:rho_details}.

\subsection{Substitution}

We use $\Proc$ for the set of processes, $\QProc$ for the set of
names, and $\id{\{}\vec{y} / \vec{x} \id{\}}$ to denote partial maps,
$s : \QProc \rightarrow \QProc$. A map, $s$ lifts, uniquely, to a map
on process terms, $\widehat{s} : \Proc \rightarrow \Proc$ by the
following equations.

\begin{mathpar}
  (0) \psubstp{Q}{P} := 0 \\
  (R \juxtap S) \psubstp{Q}{P}
  :=    
  (R)\psubstp{Q}{P} \juxtap (S) \psubstp{Q}{P} \\
  (x?(y).R) \psubstp{Q}{P}    
  :=    
  (x)\substp{Q}{P} (z)\concat( (R \psubstn{z}{y}) \psubstp{Q}{P} ) \\
  (\lift{x}{R}) \psubstp{Q}{P}  
  :=
  \lift{(x)\substp{Q}{P}}{ R \psubstp{Q}{P} } \\
%   (\dropn{x})  \psubstp{Q}{P}       
%   := 
%   \left\{ 
%     \begin{array}{ccc} 
%       \dropn{\quotep{Q}} & & x \nameeq \quotep{P} \\
%       \dropn{x} & & otherwise \\
%     \end{array}
%   \right. 
  (\dropn{x})  \psubstp{Q}{P}       
  := 
  \left\{ 
    \begin{array}{ccc} 
      Q & & x \nameeq \quotep{P} \\
      \dropn{x} & & otherwise \\
    \end{array}
  \right.
\end{mathpar}
 

where

\begin{eqnarray}
  (x)\id{\{} \lpquote Q \rpquote / \lpquote P \rpquote \id{\}}            = 
  \left\{ 
    \begin{array}{ccc}
      \lpquote Q \rpquote & & x \nameeq \lpquote P \rpquote \\
      x & & otherwise \\
    \end{array}
  \right. \nonumber
\end{eqnarray}

and $z$ is chosen distinct from $\quotep{P}$, $\quotep{Q}$, the free
names in $Q$, and all the names in $R$. Our $\alpha$-equivalence will
be built in the standard way from this substitution.

\begin{remark}\label{rem:no_self_referential_names}
  One consequence of these definitions is that $\forall P. \quotep{P}
  \not\in \freenames{P}$.
\end{remark}

\subsection{ Dynamic quote: an example }

Anticipating something of what's to come, consider applying the
substitution, $\widehat{\id{\{}u / z \id{\}}}$, to the following pair
of processes, $\lift{w}{y!(z)}$ and $w[ \lpquote y!(z) \rpquote ]$.

\begin{eqnarray}
	\lift{w}{y!(z)}\widehat{\id{\{}u / z \id{\}}}
		& = &
		\lift{w}{y!(u)} \nonumber\\
	w[ \lpquote y!(z) \rpquote ] \widehat{ \id{\{}u / z \id{\}} }
		& = &
		w[ \lpquote y!(z) \rpquote ] \nonumber
\end{eqnarray}

Because the body of the process between quotes is impervious to
substitution, we get radically different answers. In fact, by
examining the first process in an input context,
e.g. $x?(z).\lift{w}{y!(z)}$, we see that the process under the lift
operator may be shaped by prefixed inputs binding a name inside it. In
this sense, the lift operator will be seen as a way to dynamically
construct processes before reifying them as names.

Finally equipped with these standard features we can present the
dynamics of the calculus.

\subsubsection{Operational semantics} 

Finally, we introduce the computational dynamics. What marks these
algebras as distinct from other more traditionally studied algebraic
structures, e.g. vector spaces or polynomial rings, is the manner in
which dynamics is captured. In traditional structures, dynamics is typically
expressed through morphisms between such structures, as in linear maps
between vector spaces or morphisms between rings. In algebras
associated with the semantics of computation, the dynamics is
expressed as part of the algebraic structure itself, through a
reduction reduction relation typically denoted by $\red$. Below, we
give a recursive presentation of this relation for the calculus used
in the encoding.

$\red \subseteq \pi \times \pi$
$\red : \pi \to \mathcal{P}(\pi)$

\begin{mathpar}
  \inferrule* [lab=Comm] { \textsf{match}( x_{src}, x_{trgt} ) } { x_{trgt}?(y)P \; | \; x_{src}!\langle {Q} \rangle \red P\{\quotep{Q}/y}\} }
  \and \\
  \inferrule* [lab=Par] {{P} \red {P}'} {{{P} | {Q}} \red {{P}' | {Q}}}
  \and
  \inferrule* [lab=Equiv]{{{P} \scong {P}'} \andalso {{P}' \red {Q}'} \andalso {{Q}' \scong {Q}}}{{P} \red {Q}}
\end{mathpar}

\begin{eqnarray*}
  match_{\equiv} (\quotep{P},\quotep{Q}) & := & P \equiv Q \\
  match_{\dagger}(\quotep{P},\quotep{Q}) & := & \forall R. P|Q \red^{*} R => R \red^{*} 0 \\
  match_{K}(\quotep{P},\quotep{Q}) & := & K \mbox{ for some context } K
\end{eqnarray*}

$u?(x)P | u!\langle Q \rangle \red P\{\quotep{Q}/x\}$

%We write $\wred$ for $\red^*$, and $P\red$ if $\exists Q $ such that $ P \red Q$.
We write $P\red$ if $\exists Q $ such that $ P \red Q$ and $P\not\red$, otherwise.

\section{Replication}

As mentioned before, it is known that replication (and hence
recursion) can be implemented in a higher-order process algebra
\cite{SangiorgiWalker}. As our first example of calculation with the
machinery thus far presented we give the construction explicitly in
the {\rhoc}.

\begin{eqnarray}
	D_{x} & := & \prefix{x}{y}{(\binpar{\outputp{x}{y}}{@{y}})} \nonumber\\
	\bangp_{x}{P} & := & \binpar{{x}!\langle{\binpar{D_{x}}{P}}\rangle}{D_{x}} \nonumber
\end{eqnarray}

\begin{eqnarray}
	\bangp_{x}{P} & & \nonumber\\
	=
	& {x}!\langle{(\prefix{x}{y}{(\outputp{x}{y} | @{y})) | P}}\rangle 
	      | \prefix{x}{y}{(\outputp{x}{y} | @{y})} & \nonumber\\
	\red
	& (\outputp{x}{y} | @{y})\substn{\quotep{(\prefix{x}{y}{(@{y} | \outputp{x}{y})) | P}}}{y} & \nonumber\\
	=
	& \outputp{x}{\quotep{(\prefix{x}{y}{(\outputp{x}{y} | @{y})) | P}}}
	  | {(\prefix{x}{y}{(\outputp{x}{y} | @{y})) | P}} & \nonumber\\
	\red
	& \ldots & \nonumber\\
	\red^*
	& P | P | \ldots & \nonumber
\end{eqnarray}

Of course, this encoding, as an implementation, runs away, unfolding
$\bangp{P}$ eagerly. A lazier and more implementable replication
operator, restricted to input-guarded processes, may be obtained as follows.

\begin{eqnarray}
\bangp{\prefix{u}{v}{P}} 
	:= 
	\binpar{\lift{x}{\prefix{u}{v}{(\binpar{D(x)}{P})}}}{D(x)} \nonumber
\end{eqnarray}

\begin{remark}
  Note that the lazier definition still does not deal with summation
  or mixed summation (i.e. sums over input and output). The reader is
  invited to construct definitions of replication that deal with these
  features. 

  Further, the definitions are parameterized in a name, $x$. Can you,
  gentle reader, make a definition that eliminates this parameter and
  guarantees no accidental interaction between the replication
  machinery and the process being replicated -- i.e. no accidental
  sharing of names used by the process to get its work done and the
  name(s) used by the replication to effect copying. This latter
  revision of the definition of replication is crucial to obtaining
  the expected identity $!!P \sim !P$.
\end{remark}

\begin{remark}\label{rem:paradoxical_combinator}
  The reader familiar with the lambda calculus will have noticed the
  similarity between $D$ and the paradoxical combinator.

  [Ed. note: the existence of this seems to suggest we have to be more
  restrictive on the set of processes and names we admit if we are to
  support no-cloning.]
\end{remark}

\subsubsection{Bisimulation}

The computational dynamics gives rise to another kind of equivalence,
the equivalence of computational behavior. As previously mentioned
this is typically captured \emph{via} some form of bisimulation.

% The notion we use in this paper is weak barbed bisimulation
% \cite{milner91polyadicpi}.

The notion we use in this paper is derived from weak barbed
bisimulation \cite{milner91polyadicpi}. 

\begin{definition}
An \emph{observation relation}, $\downarrow_{\mathcal N}$, over a set
of names, $\mathcal N$, is the smallest relation satisfying the rules
below.

\infrule[Out-barb]{y \in {\mathcal N}, \; x \nameeq y}
		  {\outputp{x}{v} \downarrow_{\mathcal N} x}
\infrule[Par-barb]{\mbox{$P\downarrow_{\mathcal N} x$ or $Q\downarrow_{\mathcal N} x$}}
		  {\binpar{P}{Q} \downarrow_{\mathcal N} x}

We write $P \Downarrow_{\mathcal N} x$ if there is $Q$ such that 
$P \wred Q$ and $Q \downarrow_{\mathcal N} x$.
\end{definition}

\begin{definition}
%\label{def.bbisim}
An  ${\mathcal N}$-\emph{barbed bisimulation} over a set of names, ${\mathcal N}$, is a symmetric binary relation 
${\mathcal S}_{\mathcal N}$ between agents such that $P\rel{S}_{\mathcal N}Q$ implies:
\begin{enumerate}
\item If $P \red P'$ then $Q \wred Q'$ and $P'\rel{S}_{\mathcal N} Q'$.
\item If $P\downarrow_{\mathcal N} x$, then $Q\Downarrow_{\mathcal N} x$.
\end{enumerate}
$P$ is ${\mathcal N}$-barbed bisimilar to $Q$, written
$P \wbbisim_{\mathcal N} Q$, if $P \rel{S}_{\mathcal N} Q$ for some ${\mathcal N}$-barbed bisimulation ${\mathcal S}_{\mathcal N}$.
\end{definition}

$\mathcal{R} \subseteq \pi \times \pi$

$P \mathcal{R} Q => \forall P'. P \red P' \Rightarrow \exists Q'. Q \red Q', P' \mathcal{R} Q'$

$P \vdash x \Rightarrow Q \vdash x$

\begin{mathpar}
  \inferrule*[lab=Out-barb]{x \nameeq y}{{y}!\langle{Q}\rangle \vdash x}
  \and
  \inferrule*[lab=Par-barb]{\mbox{$P\vdash x$ or $Q\vdash x$}}{\binpar{P}{Q} \vdash x}
\end{mathpar}

\subsubsection{Contexts}

One of the principle advantages of computational calculi like the
$\pi$-calculus is a well-defined notion of context,
contextual-equivalence and a correlation between
contextual-equivalence and notions of bisimulation. The notion of
context allows the decomposition of a process into (sub-)process and
its syntactic environment, its context. Thus, a context may be
thought of as a process with a ``hole'' (written $\Box$) in it. The
application of a context $M$ to a process $P$, written $M[P]$, is
tantamount to filling the hole in $M$ with $P$. In this paper we do
not need the full weight of this theory, but do make use of the notion
of context in the proof the main theorem. 

\begin{mathpar}
  \inferrule* [lab=summation] {} {{M_{M},M_{N}} \bc \Box \;|\; x.M_{A} \;|\; M_{M}+M_{N}}
  \and
  \inferrule* [lab=agent] {} {{M_{A}} \bc (\vec{x})M_{P} \;| \; \clift{P_0,\ldots,M_{P},\ldots,P_N}}
  \and \\
  \inferrule* [lab=process] {} {{M_{P}} \bc M_{N} \;| \;P|M_{P} }
\end{mathpar} 

\begin{mathpar}
  \inferrule* [lab=sychronization] {} {M_{N} \bc \Box \;|\; x?M_{F} \;|\; x!M_{C}}
  \and
  \inferrule* [lab=abstraction] {} {{M_{F}} \bc (x)M_{P} }
  \and
  \inferrule* [lab=concretion] {} {{M_{C}} \bc \langle M_{P} \rangle }
  \and \\
  \inferrule* [lab=process] {} {{M_{P}} \bc M_{N} \;| \;P|M_{P} }
\end{mathpar}

\begin{definition}[contextual application] Given a context $M$, and
  process $P$, we define the \emph{contextual application}, $M[P] :=
  M\{P/\Box\}$. That is, the contextual application of M to P is the
  substitution of $P$ for $\Box$ in $M$.
\end{definition}

$\meaningof{-} : L \to \mathcal{P}(\pi)$

\begin{mathpar}
  \inferrule* [lab=collection] {} {\meaningof{true} = \pi, \and \meaningof{~E} = \pi \setminus \meaningof{E}, \and \meaningof{E_{1} \& E_{2}} = \meaningof{E_{1}} \cap \meaningof{E_{2}}}
\end{mathpar}

\begin{mathpar}
  \inferrule* [lab=structure] {} {\meaningof{0} = \{ P \in \pi | P \equiv 0 \}, \and \\ \meaningof{E_1 | E_2} = \{ P \in \pi | P \equiv P_{1} | P_{2}, P_{1} \in \meaningof{E_{1}}, P_{2} \in \meaningof{E_2}\} }
\end{mathpar}

\begin{mathpar}
 \inferrule* [lab=behavior] {} {\meaningof{\langle a?b \rangle E} = \{ P \in \pi | P \equiv Q | u?(y)P', \\ \and \\\\ \and \\ \;\;\; u \in \meaningof{a}, \forall z.P'\{z/y\} \in \meaningof{E\{z/b\}}\}, \and \\ \meaningof{a!E} = \{ P \in \pi | P \equiv Q | x!\langle P' \rangle, x \in \meaningof{a} P' \in \meaningof{E}\} }
\end{mathpar}

\begin{mathpar}
 \inferrule* [lab=nominal] {} {\meaningof{\quotep{E}} = \{ \quotep{P} \in \quotep{\pi} | P \in \meaningof{E} \}, \and \meaningof{\quotep{P}} = \{ \quotep{Q} \in \quotep{\pi} | P \equiv Q \} \and \\ \meaningof{@\quotep{E}} = \{ P \in \pi | P \equiv @x, x \in \meaningof{E} \}}
\end{mathpar}

\begin{eqnarray*}
  \\
  \meaningof{-} : TS \to ST
\end{eqnarray*}

\begin{eqnarray*}
  \\
  L : TS \to ST
\end{eqnarray*}

\begin{eqnarray*}
  \\
  P \models E \iff P \in \meaningof{E}
\end{eqnarray*}

\begin{eqnarray*}
  P \approx_{L} Q \iff \forall E \in L. P \models E \iff Q \models E
\end{eqnarray*}

\begin{eqnarray*}
  P \approx_{K} Q
\end{eqnarray*}

\begin{eqnarray*}
  P \approx Q
\end{eqnarray*}

$\approx_{K} = \approx = \approx_{L}$

\subsubsection{Contextual duality}

Note that contexts extend the quotation operation to a family of
operations from processes to names. Given a context, $M$, we can
define a \emph{nominal context}, $\quotep{M}$ by $\quotep{M}[P] :=
\quotep{M[P]}$. To foreshadow what is to come we observe that these
operations enjoy a duality with processes very much like the duality
between vectors and maps from vectors to scalars.

Further, because the calculus is essentially higher-order, we have a
correspondence between contexts and processes. More specifically,
given a name $x$ and a context $M$ we can construct $M^{*}_{x}$ such
that 

\begin{mathpar}
  M^{*}_{x} | \lift{x}{P} \red M[P]
\end{mathpar}

namely,

\begin{mathpar}
  M^{*}_{x} := x?(u).M[\dropn{u}]
\end{mathpar}

The dependence of $M^{*}_{x}$ on a name makes it an abstraction, 

\begin{mathpar}
  M^{*} := (x)x?(u).M[\dropn{u}]
\end{mathpar}

\subsection{Additional notation}

It will sometimes be convenient to denote the process a name
quotes. We already have the notation $x = \quotep{P}$, but it will be
convenient to introduce an alternate notation, $\procn{x}$, when we
want to emphasize the connection to the use of the name. Note that, by
virtue of name equivalence, $\quotep{\procn{x}} \nameeq x$; so, the
notation is consistent with previous definitions.

Further, because names have structure it is possible to effect
substitutions on the basis of that structure. This means we need to
upgrade our notation for substitutions, which we accomplish by
adapting comprehension notation. Thus,

\begin{mathpar}
  P\{ y / x : x \in S \}
\end{mathpar}

is interpreted to mean the process derived from P by replacing (in a
capture-avoiding manner) each occurrence of $x$ in $S$ by $y$. For example,

\begin{mathpar}
  P\{ \quotep{\procn{x}|\procn{x}} / x : x \in \freenames{P} \}
\end{mathpar}

will replace each (occurrence) of a free name $x$ in $P$ by
$\quotep{\procn{x}|\procn{x}}$.

Also, we will avail ourselves of the notation $x^{L}$ and $x^{R}$ to
denote injections of a name into disjoint copies of the name
space. There are numerous ways to accomplish this. One example can be
found in \cite{MeredithR05}. This notation overloads to vectors of
names: $\vec{x}^{\pi} := (x_{i}^{\pi} \; : \; 0 \leq i < |\vec{x}| )$ where $\pi \in \{L,R\}$.

We also use $P^{\Box} := P|\Box$.

In \cite{MeredithR05} an interpretation of the new operator is
given. It turns out that there are several possible interpretations
all enjoying the requisite algebraic properties of the operator (see
\cite{milner91polyadicpi}). We will therefore make liberal use of
$(\nu\; \vec{x})P$.

% subsection the_syntax_and_semantics_of_the_notation_system (end)   

\input{qm2pi.qmops} 

\input{qm2pi.sterngerlach} 

\input{qm2pi.metric} 

% section concurrent_process_calculi (end)

%\input{qm2pi.proofsketch}

% section proof sketch (end)

%\input{qm2pi.slviaknots} 

% section spatial logic via knots (end)

\input{qm2pi.conclusion}

% section conclusion (end)

%\input{qm2pi.dtcodes} 

% section wiring algorithm (end)

\input{qm2pi.ack} 

% section acknowledgments (end)

\newpage


\bibliographystyle{plain}   
\bibliography{../../biblios/main.bib}

\input{qm2pi.rhodetails}

\end{document}



% section proof sketch (end)

%\section{Unlikely characters: spatial logic for
  knots}\label{sub:characteristic_formulae} % (fold)

Associated to the mobile process calculi are a family of logics known
as the Hennessy-Milner logics. These logics typically enjoy a
semantics interpreting formulae as sets of processes that when
factored through the encoding outlined above allows an identification
of classes of knots with logical formulae. In the context of this
encoding the sub-family known as the spatial logics \cite{CairesC03}
\cite{CairesC04} \cite{Caires04} are of particular interest providing
several important features for expressing and reasoning about
properties (i.e. classes) of knots. We hint here at how this may be done.

%\begin{description}
%\item [structural connectives] 
\subsubsection{Structural connectives} The spatial logics enjoy
structural connectives corresponding, at the logical level, to the
parallel composition ($P | Q$) and new name ($(\nu \; x)P$)
connectives for processes. As illustrated in the examples below, these
connectives are extremely expressive given the shape of our encoding.
%\item [decideable satisfaction]

\subsubsection{Decideable satisfaction}
In \cite{Caires04} the satisfaction relation is shown to be decideable
for a rich class of processes. It further turns out that the image of
the our encoding is a proper subset of that class. This result
provides the basis for an algorithm by which to search for knots
enjoying a given property.
%\item [characteristic formulae]

\subsubsection{Characteristic formulae}
In the same paper \cite{Caires04} , Caires presents a means of calculating
characteristic formulae, selecting equivalence classes of processes
up to a pre--specified depth limit on the support set of names. Composed with our
encoding, this characteristic formula can be used to select
characteristic formulae for knots.
%\end{description}

\subsubsection{Spatial logic formulae}

The grammar below (segmented for comprehension) summarizes the syntax
of spatial logic formulae. We employ illustrative examples in the
sequel to provide an intuitive understanding of their meaning
referring the reader to \cite{Caires04} for a more detailed explication
of the semantics.

\begin{mathpar}
  \inferrule* [lab=boolean] {} {{A,B} \bc T \;|\; \neg A \;|\; A \wedge B \;|\; \eta = \eta'}
  \and
  \inferrule* [lab=spatial] {} {|\; \pzero \;|\; A | B \;|\; x \text{\textregistered} A \;|\; \forall x . A \;|\;  H x . A}
  \and
  \inferrule* [lab=behavioral] {} {|\; \alpha . A}
  \and 
  \inferrule* [lab=recursion] {} {|\; X(\vec{u}) \;|\; \mu X(\vec{u}) . A}
  \and
  \inferrule* [lab=action] {} {\alpha \bc \langle x?(\vec{y}) \rangle \;|\; \langle x!(\vec{y}) \rangle \;|\; \langle \tau \rangle}
  \and 
  \inferrule* [lab=name] {} {\eta \bc x \;|\; \tau}
\end{mathpar} 

% subsection characteristic_formulae (end)   	 

\subsection{Example formulae}\label{sub:example_formulae_} % (fold)

\subsubsection{Crossing as formula.}
% 
% \begin{align*}
%   \frac{d}{dx} \sin x &= \cos x 
%   & \frac{d}{dx} e^x &= e^x \\
%   \frac{d}{dx} \cos x &= - \sin x 
%   & \frac{d}{dx} \log x &= \frac{1}{x} \\
% \end{align*} 

\begin{align*}
 \mu C(x_{0},x_{1},y_{0},y_{1},u).&(\langle x_{0}?(z) \rangle(\langle u! \rangle\langle y_{1}!z \rangle C(x_{0},x_{1},y_{0},y_{1},u)) & \\
  & \wedge \langle y_{1}?(z) \rangle (\langle u! \rangle \langle x_{0}!z \rangle C(x_{0},x_{1},y_{0},y_{1},u)) & \\
  & \wedge \langle x_{1}?(z) \rangle (\langle u? \rangle \langle y_{0}!z \rangle C(x_{0},x_{1},y_{0},y_{1},u)) & \\
  & \wedge \langle y_{0}?(z) \rangle (\langle u? \rangle \langle x_{1}!z \rangle C(x_{0},x_{1},y_{0},y_{1},u))) &
\end{align*}

The lexicographical similarity between the shape of this formulae and
the shape of definition of the process representing a crossing reveals
the intuitive meaning of this formulae. It describes the capabilities
of a process that has the right to represent a crossing. For example
it picks out processes that may perform an input on the port $x_0$ in
its initial menu of capabilities. What differentiates the formula
from the process, however, is that the crossing process is the
smallest candidate to satisfy the formula. Infinitely many other
processes -- with internal behavior hidden behind this interface, so
to speak -- also satisfy this formula. Even this simple formula,
then, can be seen to open a new view onto knots, providing a
computational interpretation of \emph{virtual} knots.

Note that this formula is derived by hand. A similar formula can be
derived by employing Caires' calculation of characteristic formula
\cite{Caires04} to the process representing a crossing. In light of
this discussion, we let
$\meaningof{C}_{\phi}(x0,x1,y0,y1,u)$ denote a formula specifying the
dynamics we wish to capture of a crossing. To guarantee we preserve
the shape of the interface and minimal semantics we demand that
$\meaningof{C}_{\phi}(x0,x1,y0,y1,u) \Rightarrow
\textbf{C}(x0,x1,y0,y1,u)$ where $\textbf{C}(x0,x1,y0,y1,u)$ denotes
the formula above.
                            
\subsubsection{Crossing number constraints.}
The moral content of the context lemma (Lemma \ref{context}) is that the notion of
``locality'' in the Reidemeister moves is effectively captured by the
parallel composition operator of the process calculus. This intuition
extends through the logic. Given a formula,
$\meaningof{C}_{\phi}(x0,x1,y0,y1,u)$, we can use the structural
connectives to specify constraints on crossing numbers, such as at
least $n$ crossings, or exactly $n$ crossings.
\begin{mathpar}
  \inferrule* [lab=at-least-n] {} { K^{\geq n}_{\phi}(\vec{xs},\vec{ys}) := \Pi_{i=0}^{n-1} Hu . \meaningof{C}_{\phi}(xs_i,ys_i,u) | T }
  \and 
  \inferrule* [lab=exactly-n] {} { K^{= n}_{\phi}(\vec{xs},\vec{ys}) := \Pi_{i=0}^{n-1} Hu . \meaningof{C}_{\phi}(xs_i,ys_i,u) | \neg (\forall x_0,y_0,x_1,y_1,u . \meaningof{C}_{\phi}(x_0,y_0,x_1,y_1,u) | T) }
\end{mathpar}

To round out this section, recall that the encoding of an $n$-crossing
knot decomposes into a parallel composition of $n$ \emph{copies} of a
crossing process together with a wiring harness. To specify different
knot classes with the same crossing number amounts to specifying
logical constraints on the wiring harness. In the interest of space,
we defer examples to a forthcoming paper. Suffice it to say that both
the conditions ``alternating knot'' and ``contains the tangle
corresponding to 5/3'' are expressible. For example, it is possible to
calculate the characteristic formula of a process corresponding to the
tangle 5/3 and conjoin it into the classifying formula via the
composition connective of the logic.

Finally, we wish to observe that it is entirely within reason to
contemplate a more domain-specific version of spatial logic tailored
to the shape of processes in the image of the encoding. Such a
domain-specific logic would have a better claim to the title formal
language of knot properties.

% subsection example_formulae_ (end)

% section knots_as_processes (end) 

% section spatial logic via knots (end)

\section{Conclusions and future work}

\paragraph{Testing physical space}
You, gentle reader, may wonder why of all the theorems to be proved
given this set up we pick the one above. In some sense it's hardly
central to quantum mechanics. We see it as central in the sense that
it firmly establishes a notion of physical space arising from a notion
of the equivalence of behavior. Relating bisimulation to a metric is a
big step forward, but one is faced with interpreting the relationship
of that metric space to something more physical. Quantum mechanical
notions of ``physical'' space are still far from intuitive, but by
relating this idea of distance as testing to calculations that predict
physical circumstances we are making a not insignificant step forward
toward an understanding of the physical space we inhabit as
essentially dynamic.

\paragraph{Effectivity and simulation}
One of the observations we have yet to make is that the entire program
spelled out here is effective. We have built various interpreters for
the reflective calculus at work in this interpretation. In principle,
then, we can simulate quantum mechanics on a computer. The place where
the simulation may lose fidelity is the infinitely branching summation
for the annihilator.

In this connection i also want to point out that the evaluation style
calculation of the inner product puts the non-determinism of the
summation right at the heart of measurement. This suggests that
Milner's original reduction-based formulation of the dynamics of his
calculi in terms of sums was not just notationally suggestive of a
notion of measure-and-continue but captured some significant part of
the physics.

\paragraph{Quantum continuations}
In light of this last observation i want to point out that the
predominant account of quantum mechanics is missing a key aspect of a
truly compositional story of the physical situation. In a real lab,
when a measurement is made the observation can be made to feed into
another device that then makes another measurement conditioned on the
results of the first. This means that after the superposition was
collapsed the entire experimental set up remained in
superposition. While QM offers a means of writing this down it doesn't
quite line up well with the well-trodden formulation of computation
and continuation that we see so succinctly expressed in Milner's
calculi. This suggests that there might be advantages to this account
of dynamics waiting to be explored.

\paragraph{Quantum logic}
In this connection, we also note that by virtue of having the
Hennessy-Milner construction, we can pull the construction through the
interpretation of QM. This gives us a natural candidate for a quantum
logic that enjoys an extremely tight connection with it's domain of
interpretation, making the construction much less ad hoc (rather it is
the image of functor!).

\paragraph{Quantum probabiity}
i have questions about the basis of the interpretation of inner
product as probability amplitude. In particular, using which
axiomatization of probability theory does the notion of probability
amplitude earn the right to be so dubbed? In other words, where is the
proof that the operation for calculating a probability amplitude (and
then squaring) satisfies the axioms of what it means to calculate a
probability? Even if such a proof exists (i have yet to find it in the
literature), i wonder if it might not be possible to turn things on
their heads. Can we view the calculation of the probability amplitude
as an axiomatization of probability? If so, then the definition we
give for calculating probability amplitude may provide the basis for
an \emph{effective} theory of probability.

\paragraph{Quantum vs ``biological'' information}
Finally, i want to conclude with a more philosophical observation. At
a recent workshop in which QM was a predominant topic i noticed
something about quantum information. The speaker was giving a riveting
discussion of axiomatic QM and showing how properties of ``no
cloning'' and ``no deleting'' emerged as consequences of the
axiomatization. Theorems of this form are necessary to give us a sense
of confidence that our axioms characterize the physical theory. What
struck me, though, was that if quantum information is neither erasable
nor replicable it is markedly different from \emph{life}. Two of the
things we know about life is that

\begin{itemize}
  \item it ends;
  \item to gain some measure of persistence, to transcend it's
    finitude it is imminently copyable.
\end{itemize}

Both of these qualities are summarized succinctly in the aphorism: all
flesh is grass. For me these two kinds of ``information'' -- call them
quantum and biological -- are end points on a spectrum of strategies
for persistence. At one end, we have those curious entities that enjoy
uniqueness and permanence; at the other, we have those who in the face
of a certain end and an uncertain present make a go of passing
something on. To me one of the more remarkable aspects of the latter
strategy is that in the presence of noise (and certain features of
copying) we get a kind of dynamism, a chance for improvement against a
given persistent condition.

% subsection other_calculi_other_bisimulations_and_geometry_as_behavior (end)




% section conclusion (end)

%\documentclass[12pt]{llncs}
%\documentclass{jktr}

\usepackage[pdftex]{hyperref}                   
\usepackage {listings}
\usepackage {mathpartir}
\usepackage{bcprules}
%\usepackage{listings}
                       
\usepackage{graphicx} 
%\usepackage[margins=2.5cm,nohead,nofoot]{geometry}
%\usepackage{geometry}
\usepackage{amsfonts}
\usepackage{amstext}
\usepackage{latexsym}
\usepackage{amssymb}
\usepackage{color}


%\include{myPreamble}
\include{qm2pi.local} 

%\ifpdf
%\usepackage[pdftex]{graphicx}
%\else
%\usepackage{graphicx}
%\fi

 % \ifpdf
%  \usepackage{pdfsync}
%  \if


%\title{Brief Article}
%\author{David F. Snyder}
%\author{L.G. Meredith}

%\address{Dept. of Math., Texas State University--San Marcos, San Marcos, TX 78666}
       
\pagestyle{empty}


\begin{document}

\lstset{language=[Objective]Caml,frame=shadowbox}

\input{qm2pi.front}

% section front matter (end)

\input{qm2pi.intro} 
 
% section introduction (end)

% \input{qm2pi.knotations} 

% section notation (end)

\input{qm2pi.process.calculi} 

% section concurrent_process_calculi_and_spatial_logics_ (end)
    
%\input{qm2pi.knots2pi} 

%\input{qm2pi.trefoil} 

%\input{qm2pi.mainthm} 

% subsection basic_interpretation (end)

%\input{qm2pi.rho.presentation} 
\subsection{The syntax and semantics of the notation system}\label{sub:the_syntax_and_semantics_of_the_notation_system} % (fold)

We now summarize a technical presentation of the calculus that
embodies our theory of dynamics. The typical presentation of such a
calculus follows the style of giving generators and relations on
them. The grammar, below, describing term constructors, freely
generates the set of processes, $\Proc$. This set is then quotiented
by a relation known as structural congruence and it is over this set
that the notion of dynamics is expressed. This presentation is
essentially that of \cite{MeredithR05} with the addition of
polyadicity and summation. For readability we have relegated some of
the technical subtleties to an appendix.

\subsubsection{Process grammar}\label{subsub:process_grammar}

\begin{mathpar}
  \inferrule* [lab=synchronization] {} {{M} \bc \pzero \;|\; x?F \;|\; x!C }
  \and
  \inferrule* [lab=abstraction] {} {{F} \bc (x)P}
  \and
  \inferrule* [lab=concretion] {} {{C} \bc \langle Q \rangle}
  \and
  \inferrule* [lab=process] {} {{P,Q} \bc M \;| \;P|Q \;|\; @{x}}
  \and
  \inferrule* [lab=name] {} {{x} \bc \quotep{P}}
\end{mathpar} 

Note that $\vec{x}$ (resp. $\vec{P}$) denotes a vector of names
(resp. processes) of length $|\vec{x}|$ (resp. $|\vec{P}|$). We adopt
the following useful abbreviations.

\begin{mathpar}
   x?(\vec{y}).P := x.(\vec{y})P \and  x\clift{\vec{P}} := x.\clift{\vec{P}}
   \and x!(y) := \lift{x}{\dropn{y}}
   \and \Pi_{i=0}^{n-1}P_i := P_0 | \ldots | P_{n-1}
\end{mathpar}

\subsubsection{Structural congruence}

\paragraph{Free and bound names and alpha-equivalence.} At the
core of structural equivalence is alpha-equivalence which identifies
process that are the same up to a change of variable. Formally, we
recognize the distinction between free and bound names. The free names
of a process, $\freenames{P}$, may be calculated recursively as
follows:

\begin{mathpar}
\freenames{\pzero} := \emptyset
  \and \\
  \freenames{x?(y).P} := \{ x \} \cup (\freenames{P} \setminus \{ y \})
  \and 
  \freenames{x!\langle P \rangle} := \{ x \} \cup \{ P \} 
  \and \\
  \freenames{P|Q} := \freenames{P} \cup \freenames{Q}
  \and \\
  \freenames{@{x}} := \{ x \}
\end{mathpar}

$\pi$
$\quotep{\pi}$

$\freenames{-} : \pi \to \mathcal{P}(\quotep{\pi})$

\begin{eqnarray*}
  \freenames{\pzero} & := & \emptyset \\
  \freenames{x?(y).P} & := & \{ x \} \cup (\freenames{P} \setminus \{ y \}) \\
  \freenames{x!\langle P \rangle} & := & \{ x \} \cup \{ P \} \\
  \freenames{P|Q} & := & \freenames{P} \cup \freenames{Q} \\
  \freenames{\dropn{x}} & := & \{ x \}
\end{eqnarray*}

The bound names of a process, $\boundnames{P}$, are those names occurring in $P$
that are not free. For example, in $x?(y).0$, the name $x$ is free, while $y$ is bound.

\begin{mathpar}
  \inferrule* [lab=monoidal-laws] {} { P|Q \equiv Q|P \and P|0 \equiv P \and P|(Q|R) \equiv (P|Q)|R }
\end{mathpar}

\begin{mathpar}
  \inferrule* [lab=alpha-equivalence] {} { (x)P \equiv (y)P\{y/x\} \and y \not\in \freenames{P} }
\end{mathpar}

\begin{definition}
Then two processes, $P,Q$, are alpha-equivalent if $P = Q\{\vec{y}/\vec{x}\}$ for
some $\vec{x} \in \boundnames{Q},\vec{y} \in \boundnames{P}$, where $Q\{\vec{y}/\vec{x}\}$
denotes the capture-avoiding substitution of $\vec{y}$ for $\vec{x}$ in $Q$.
\end{definition}

\begin{definition}
  The {\em structural congruence} \cite{SangiorgiWalker} , $\equiv$,
  between processes is the least congruence containing
  alpha-equivalence, satisfying the abelian monoid laws
  (associativity, commutativity and $\pzero$ as identity) for parallel
  composition $|$ and for summation $+$.
\end{definition}

\subsection{Name equivalence}

We take name equivalence, written $\nameeq$, to be the smallest
equivalence relation generated by the following rules.

\begin{mathpar}
\inferrule*[lab=Quote-drop]
{ }
{ \quotep{@{x}} \nameeq x }

\inferrule*[lab=Struct-equiv]
{ P \scong Q }
{ \quotep{P} \nameeq \quotep{Q} }
\end{mathpar}

The astute reader will have noticed that the mutual recursion of names
and processes imposes a mutual recursion on alpha-equivalence and
structural equivalence via name-equivalence. Fortunately, all of this
works out pleasantly and we may calculate in the natural way, free of
concern. The reader interested in the details is referred to the
appendix \ref{appendix:rho_details}.

\subsection{Substitution}

We use $\Proc$ for the set of processes, $\QProc$ for the set of
names, and $\id{\{}\vec{y} / \vec{x} \id{\}}$ to denote partial maps,
$s : \QProc \rightarrow \QProc$. A map, $s$ lifts, uniquely, to a map
on process terms, $\widehat{s} : \Proc \rightarrow \Proc$ by the
following equations.

\begin{mathpar}
  (0) \psubstp{Q}{P} := 0 \\
  (R \juxtap S) \psubstp{Q}{P}
  :=    
  (R)\psubstp{Q}{P} \juxtap (S) \psubstp{Q}{P} \\
  (x?(y).R) \psubstp{Q}{P}    
  :=    
  (x)\substp{Q}{P} (z)\concat( (R \psubstn{z}{y}) \psubstp{Q}{P} ) \\
  (\lift{x}{R}) \psubstp{Q}{P}  
  :=
  \lift{(x)\substp{Q}{P}}{ R \psubstp{Q}{P} } \\
%   (\dropn{x})  \psubstp{Q}{P}       
%   := 
%   \left\{ 
%     \begin{array}{ccc} 
%       \dropn{\quotep{Q}} & & x \nameeq \quotep{P} \\
%       \dropn{x} & & otherwise \\
%     \end{array}
%   \right. 
  (\dropn{x})  \psubstp{Q}{P}       
  := 
  \left\{ 
    \begin{array}{ccc} 
      Q & & x \nameeq \quotep{P} \\
      \dropn{x} & & otherwise \\
    \end{array}
  \right.
\end{mathpar}
 

where

\begin{eqnarray}
  (x)\id{\{} \lpquote Q \rpquote / \lpquote P \rpquote \id{\}}            = 
  \left\{ 
    \begin{array}{ccc}
      \lpquote Q \rpquote & & x \nameeq \lpquote P \rpquote \\
      x & & otherwise \\
    \end{array}
  \right. \nonumber
\end{eqnarray}

and $z$ is chosen distinct from $\quotep{P}$, $\quotep{Q}$, the free
names in $Q$, and all the names in $R$. Our $\alpha$-equivalence will
be built in the standard way from this substitution.

\begin{remark}\label{rem:no_self_referential_names}
  One consequence of these definitions is that $\forall P. \quotep{P}
  \not\in \freenames{P}$.
\end{remark}

\subsection{ Dynamic quote: an example }

Anticipating something of what's to come, consider applying the
substitution, $\widehat{\id{\{}u / z \id{\}}}$, to the following pair
of processes, $\lift{w}{y!(z)}$ and $w[ \lpquote y!(z) \rpquote ]$.

\begin{eqnarray}
	\lift{w}{y!(z)}\widehat{\id{\{}u / z \id{\}}}
		& = &
		\lift{w}{y!(u)} \nonumber\\
	w[ \lpquote y!(z) \rpquote ] \widehat{ \id{\{}u / z \id{\}} }
		& = &
		w[ \lpquote y!(z) \rpquote ] \nonumber
\end{eqnarray}

Because the body of the process between quotes is impervious to
substitution, we get radically different answers. In fact, by
examining the first process in an input context,
e.g. $x?(z).\lift{w}{y!(z)}$, we see that the process under the lift
operator may be shaped by prefixed inputs binding a name inside it. In
this sense, the lift operator will be seen as a way to dynamically
construct processes before reifying them as names.

Finally equipped with these standard features we can present the
dynamics of the calculus.

\subsubsection{Operational semantics} 

Finally, we introduce the computational dynamics. What marks these
algebras as distinct from other more traditionally studied algebraic
structures, e.g. vector spaces or polynomial rings, is the manner in
which dynamics is captured. In traditional structures, dynamics is typically
expressed through morphisms between such structures, as in linear maps
between vector spaces or morphisms between rings. In algebras
associated with the semantics of computation, the dynamics is
expressed as part of the algebraic structure itself, through a
reduction reduction relation typically denoted by $\red$. Below, we
give a recursive presentation of this relation for the calculus used
in the encoding.

$\red \subseteq \pi \times \pi$
$\red : \pi \to \mathcal{P}(\pi)$

\begin{mathpar}
  \inferrule* [lab=Comm] { \textsf{match}( x_{src}, x_{trgt} ) } { x_{trgt}?(y)P \; | \; x_{src}!\langle {Q} \rangle \red P\{\quotep{Q}/y}\} }
  \and \\
  \inferrule* [lab=Par] {{P} \red {P}'} {{{P} | {Q}} \red {{P}' | {Q}}}
  \and
  \inferrule* [lab=Equiv]{{{P} \scong {P}'} \andalso {{P}' \red {Q}'} \andalso {{Q}' \scong {Q}}}{{P} \red {Q}}
\end{mathpar}

\begin{eqnarray*}
  match_{\equiv} (\quotep{P},\quotep{Q}) & := & P \equiv Q \\
  match_{\dagger}(\quotep{P},\quotep{Q}) & := & \forall R. P|Q \red^{*} R => R \red^{*} 0 \\
  match_{K}(\quotep{P},\quotep{Q}) & := & K \mbox{ for some context } K
\end{eqnarray*}

$u?(x)P | u!\langle Q \rangle \red P\{\quotep{Q}/x\}$

%We write $\wred$ for $\red^*$, and $P\red$ if $\exists Q $ such that $ P \red Q$.
We write $P\red$ if $\exists Q $ such that $ P \red Q$ and $P\not\red$, otherwise.

\section{Replication}

As mentioned before, it is known that replication (and hence
recursion) can be implemented in a higher-order process algebra
\cite{SangiorgiWalker}. As our first example of calculation with the
machinery thus far presented we give the construction explicitly in
the {\rhoc}.

\begin{eqnarray}
	D_{x} & := & \prefix{x}{y}{(\binpar{\outputp{x}{y}}{@{y}})} \nonumber\\
	\bangp_{x}{P} & := & \binpar{{x}!\langle{\binpar{D_{x}}{P}}\rangle}{D_{x}} \nonumber
\end{eqnarray}

\begin{eqnarray}
	\bangp_{x}{P} & & \nonumber\\
	=
	& {x}!\langle{(\prefix{x}{y}{(\outputp{x}{y} | @{y})) | P}}\rangle 
	      | \prefix{x}{y}{(\outputp{x}{y} | @{y})} & \nonumber\\
	\red
	& (\outputp{x}{y} | @{y})\substn{\quotep{(\prefix{x}{y}{(@{y} | \outputp{x}{y})) | P}}}{y} & \nonumber\\
	=
	& \outputp{x}{\quotep{(\prefix{x}{y}{(\outputp{x}{y} | @{y})) | P}}}
	  | {(\prefix{x}{y}{(\outputp{x}{y} | @{y})) | P}} & \nonumber\\
	\red
	& \ldots & \nonumber\\
	\red^*
	& P | P | \ldots & \nonumber
\end{eqnarray}

Of course, this encoding, as an implementation, runs away, unfolding
$\bangp{P}$ eagerly. A lazier and more implementable replication
operator, restricted to input-guarded processes, may be obtained as follows.

\begin{eqnarray}
\bangp{\prefix{u}{v}{P}} 
	:= 
	\binpar{\lift{x}{\prefix{u}{v}{(\binpar{D(x)}{P})}}}{D(x)} \nonumber
\end{eqnarray}

\begin{remark}
  Note that the lazier definition still does not deal with summation
  or mixed summation (i.e. sums over input and output). The reader is
  invited to construct definitions of replication that deal with these
  features. 

  Further, the definitions are parameterized in a name, $x$. Can you,
  gentle reader, make a definition that eliminates this parameter and
  guarantees no accidental interaction between the replication
  machinery and the process being replicated -- i.e. no accidental
  sharing of names used by the process to get its work done and the
  name(s) used by the replication to effect copying. This latter
  revision of the definition of replication is crucial to obtaining
  the expected identity $!!P \sim !P$.
\end{remark}

\begin{remark}\label{rem:paradoxical_combinator}
  The reader familiar with the lambda calculus will have noticed the
  similarity between $D$ and the paradoxical combinator.

  [Ed. note: the existence of this seems to suggest we have to be more
  restrictive on the set of processes and names we admit if we are to
  support no-cloning.]
\end{remark}

\subsubsection{Bisimulation}

The computational dynamics gives rise to another kind of equivalence,
the equivalence of computational behavior. As previously mentioned
this is typically captured \emph{via} some form of bisimulation.

% The notion we use in this paper is weak barbed bisimulation
% \cite{milner91polyadicpi}.

The notion we use in this paper is derived from weak barbed
bisimulation \cite{milner91polyadicpi}. 

\begin{definition}
An \emph{observation relation}, $\downarrow_{\mathcal N}$, over a set
of names, $\mathcal N$, is the smallest relation satisfying the rules
below.

\infrule[Out-barb]{y \in {\mathcal N}, \; x \nameeq y}
		  {\outputp{x}{v} \downarrow_{\mathcal N} x}
\infrule[Par-barb]{\mbox{$P\downarrow_{\mathcal N} x$ or $Q\downarrow_{\mathcal N} x$}}
		  {\binpar{P}{Q} \downarrow_{\mathcal N} x}

We write $P \Downarrow_{\mathcal N} x$ if there is $Q$ such that 
$P \wred Q$ and $Q \downarrow_{\mathcal N} x$.
\end{definition}

\begin{definition}
%\label{def.bbisim}
An  ${\mathcal N}$-\emph{barbed bisimulation} over a set of names, ${\mathcal N}$, is a symmetric binary relation 
${\mathcal S}_{\mathcal N}$ between agents such that $P\rel{S}_{\mathcal N}Q$ implies:
\begin{enumerate}
\item If $P \red P'$ then $Q \wred Q'$ and $P'\rel{S}_{\mathcal N} Q'$.
\item If $P\downarrow_{\mathcal N} x$, then $Q\Downarrow_{\mathcal N} x$.
\end{enumerate}
$P$ is ${\mathcal N}$-barbed bisimilar to $Q$, written
$P \wbbisim_{\mathcal N} Q$, if $P \rel{S}_{\mathcal N} Q$ for some ${\mathcal N}$-barbed bisimulation ${\mathcal S}_{\mathcal N}$.
\end{definition}

$\mathcal{R} \subseteq \pi \times \pi$

$P \mathcal{R} Q => \forall P'. P \red P' \Rightarrow \exists Q'. Q \red Q', P' \mathcal{R} Q'$

$P \vdash x \Rightarrow Q \vdash x$

\begin{mathpar}
  \inferrule*[lab=Out-barb]{x \nameeq y}{{y}!\langle{Q}\rangle \vdash x}
  \and
  \inferrule*[lab=Par-barb]{\mbox{$P\vdash x$ or $Q\vdash x$}}{\binpar{P}{Q} \vdash x}
\end{mathpar}

\subsubsection{Contexts}

One of the principle advantages of computational calculi like the
$\pi$-calculus is a well-defined notion of context,
contextual-equivalence and a correlation between
contextual-equivalence and notions of bisimulation. The notion of
context allows the decomposition of a process into (sub-)process and
its syntactic environment, its context. Thus, a context may be
thought of as a process with a ``hole'' (written $\Box$) in it. The
application of a context $M$ to a process $P$, written $M[P]$, is
tantamount to filling the hole in $M$ with $P$. In this paper we do
not need the full weight of this theory, but do make use of the notion
of context in the proof the main theorem. 

\begin{mathpar}
  \inferrule* [lab=summation] {} {{M_{M},M_{N}} \bc \Box \;|\; x.M_{A} \;|\; M_{M}+M_{N}}
  \and
  \inferrule* [lab=agent] {} {{M_{A}} \bc (\vec{x})M_{P} \;| \; \clift{P_0,\ldots,M_{P},\ldots,P_N}}
  \and \\
  \inferrule* [lab=process] {} {{M_{P}} \bc M_{N} \;| \;P|M_{P} }
\end{mathpar} 

\begin{mathpar}
  \inferrule* [lab=sychronization] {} {M_{N} \bc \Box \;|\; x?M_{F} \;|\; x!M_{C}}
  \and
  \inferrule* [lab=abstraction] {} {{M_{F}} \bc (x)M_{P} }
  \and
  \inferrule* [lab=concretion] {} {{M_{C}} \bc \langle M_{P} \rangle }
  \and \\
  \inferrule* [lab=process] {} {{M_{P}} \bc M_{N} \;| \;P|M_{P} }
\end{mathpar}

\begin{definition}[contextual application] Given a context $M$, and
  process $P$, we define the \emph{contextual application}, $M[P] :=
  M\{P/\Box\}$. That is, the contextual application of M to P is the
  substitution of $P$ for $\Box$ in $M$.
\end{definition}

$\meaningof{-} : L \to \mathcal{P}(\pi)$

\begin{mathpar}
  \inferrule* [lab=collection] {} {\meaningof{true} = \pi, \and \meaningof{~E} = \pi \setminus \meaningof{E}, \and \meaningof{E_{1} \& E_{2}} = \meaningof{E_{1}} \cap \meaningof{E_{2}}}
\end{mathpar}

\begin{mathpar}
  \inferrule* [lab=structure] {} {\meaningof{0} = \{ P \in \pi | P \equiv 0 \}, \and \\ \meaningof{E_1 | E_2} = \{ P \in \pi | P \equiv P_{1} | P_{2}, P_{1} \in \meaningof{E_{1}}, P_{2} \in \meaningof{E_2}\} }
\end{mathpar}

\begin{mathpar}
 \inferrule* [lab=behavior] {} {\meaningof{\langle a?b \rangle E} = \{ P \in \pi | P \equiv Q | u?(y)P', \\ \and \\\\ \and \\ \;\;\; u \in \meaningof{a}, \forall z.P'\{z/y\} \in \meaningof{E\{z/b\}}\}, \and \\ \meaningof{a!E} = \{ P \in \pi | P \equiv Q | x!\langle P' \rangle, x \in \meaningof{a} P' \in \meaningof{E}\} }
\end{mathpar}

\begin{mathpar}
 \inferrule* [lab=nominal] {} {\meaningof{\quotep{E}} = \{ \quotep{P} \in \quotep{\pi} | P \in \meaningof{E} \}, \and \meaningof{\quotep{P}} = \{ \quotep{Q} \in \quotep{\pi} | P \equiv Q \} \and \\ \meaningof{@\quotep{E}} = \{ P \in \pi | P \equiv @x, x \in \meaningof{E} \}}
\end{mathpar}

\begin{eqnarray*}
  \\
  \meaningof{-} : TS \to ST
\end{eqnarray*}

\begin{eqnarray*}
  \\
  L : TS \to ST
\end{eqnarray*}

\begin{eqnarray*}
  \\
  P \models E \iff P \in \meaningof{E}
\end{eqnarray*}

\begin{eqnarray*}
  P \approx_{L} Q \iff \forall E \in L. P \models E \iff Q \models E
\end{eqnarray*}

\begin{eqnarray*}
  P \approx_{K} Q
\end{eqnarray*}

\begin{eqnarray*}
  P \approx Q
\end{eqnarray*}

$\approx_{K} = \approx = \approx_{L}$

\subsubsection{Contextual duality}

Note that contexts extend the quotation operation to a family of
operations from processes to names. Given a context, $M$, we can
define a \emph{nominal context}, $\quotep{M}$ by $\quotep{M}[P] :=
\quotep{M[P]}$. To foreshadow what is to come we observe that these
operations enjoy a duality with processes very much like the duality
between vectors and maps from vectors to scalars.

Further, because the calculus is essentially higher-order, we have a
correspondence between contexts and processes. More specifically,
given a name $x$ and a context $M$ we can construct $M^{*}_{x}$ such
that 

\begin{mathpar}
  M^{*}_{x} | \lift{x}{P} \red M[P]
\end{mathpar}

namely,

\begin{mathpar}
  M^{*}_{x} := x?(u).M[\dropn{u}]
\end{mathpar}

The dependence of $M^{*}_{x}$ on a name makes it an abstraction, 

\begin{mathpar}
  M^{*} := (x)x?(u).M[\dropn{u}]
\end{mathpar}

\subsection{Additional notation}

It will sometimes be convenient to denote the process a name
quotes. We already have the notation $x = \quotep{P}$, but it will be
convenient to introduce an alternate notation, $\procn{x}$, when we
want to emphasize the connection to the use of the name. Note that, by
virtue of name equivalence, $\quotep{\procn{x}} \nameeq x$; so, the
notation is consistent with previous definitions.

Further, because names have structure it is possible to effect
substitutions on the basis of that structure. This means we need to
upgrade our notation for substitutions, which we accomplish by
adapting comprehension notation. Thus,

\begin{mathpar}
  P\{ y / x : x \in S \}
\end{mathpar}

is interpreted to mean the process derived from P by replacing (in a
capture-avoiding manner) each occurrence of $x$ in $S$ by $y$. For example,

\begin{mathpar}
  P\{ \quotep{\procn{x}|\procn{x}} / x : x \in \freenames{P} \}
\end{mathpar}

will replace each (occurrence) of a free name $x$ in $P$ by
$\quotep{\procn{x}|\procn{x}}$.

Also, we will avail ourselves of the notation $x^{L}$ and $x^{R}$ to
denote injections of a name into disjoint copies of the name
space. There are numerous ways to accomplish this. One example can be
found in \cite{MeredithR05}. This notation overloads to vectors of
names: $\vec{x}^{\pi} := (x_{i}^{\pi} \; : \; 0 \leq i < |\vec{x}| )$ where $\pi \in \{L,R\}$.

We also use $P^{\Box} := P|\Box$.

In \cite{MeredithR05} an interpretation of the new operator is
given. It turns out that there are several possible interpretations
all enjoying the requisite algebraic properties of the operator (see
\cite{milner91polyadicpi}). We will therefore make liberal use of
$(\nu\; \vec{x})P$.

% subsection the_syntax_and_semantics_of_the_notation_system (end)   

\input{qm2pi.qmops} 

\input{qm2pi.sterngerlach} 

\input{qm2pi.metric} 

% section concurrent_process_calculi (end)

%\input{qm2pi.proofsketch}

% section proof sketch (end)

%\input{qm2pi.slviaknots} 

% section spatial logic via knots (end)

\input{qm2pi.conclusion}

% section conclusion (end)

%\input{qm2pi.dtcodes} 

% section wiring algorithm (end)

\input{qm2pi.ack} 

% section acknowledgments (end)

\newpage


\bibliographystyle{plain}   
\bibliography{../../biblios/main.bib}

\input{qm2pi.rhodetails}

\end{document}

 

% section wiring algorithm (end)

\documentclass[12pt]{llncs}
%\documentclass{jktr}

\usepackage[pdftex]{hyperref}                   
\usepackage {listings}
\usepackage {mathpartir}
\usepackage{bcprules}
%\usepackage{listings}
                       
\usepackage{graphicx} 
%\usepackage[margins=2.5cm,nohead,nofoot]{geometry}
%\usepackage{geometry}
\usepackage{amsfonts}
\usepackage{amstext}
\usepackage{latexsym}
\usepackage{amssymb}
\usepackage{color}


%\include{myPreamble}
\include{qm2pi.local} 

%\ifpdf
%\usepackage[pdftex]{graphicx}
%\else
%\usepackage{graphicx}
%\fi

 % \ifpdf
%  \usepackage{pdfsync}
%  \if


%\title{Brief Article}
%\author{David F. Snyder}
%\author{L.G. Meredith}

%\address{Dept. of Math., Texas State University--San Marcos, San Marcos, TX 78666}
       
\pagestyle{empty}


\begin{document}

\lstset{language=[Objective]Caml,frame=shadowbox}

\input{qm2pi.front}

% section front matter (end)

\input{qm2pi.intro} 
 
% section introduction (end)

% \input{qm2pi.knotations} 

% section notation (end)

\input{qm2pi.process.calculi} 

% section concurrent_process_calculi_and_spatial_logics_ (end)
    
%\input{qm2pi.knots2pi} 

%\input{qm2pi.trefoil} 

%\input{qm2pi.mainthm} 

% subsection basic_interpretation (end)

%\input{qm2pi.rho.presentation} 
\subsection{The syntax and semantics of the notation system}\label{sub:the_syntax_and_semantics_of_the_notation_system} % (fold)

We now summarize a technical presentation of the calculus that
embodies our theory of dynamics. The typical presentation of such a
calculus follows the style of giving generators and relations on
them. The grammar, below, describing term constructors, freely
generates the set of processes, $\Proc$. This set is then quotiented
by a relation known as structural congruence and it is over this set
that the notion of dynamics is expressed. This presentation is
essentially that of \cite{MeredithR05} with the addition of
polyadicity and summation. For readability we have relegated some of
the technical subtleties to an appendix.

\subsubsection{Process grammar}\label{subsub:process_grammar}

\begin{mathpar}
  \inferrule* [lab=synchronization] {} {{M} \bc \pzero \;|\; x?F \;|\; x!C }
  \and
  \inferrule* [lab=abstraction] {} {{F} \bc (x)P}
  \and
  \inferrule* [lab=concretion] {} {{C} \bc \langle Q \rangle}
  \and
  \inferrule* [lab=process] {} {{P,Q} \bc M \;| \;P|Q \;|\; @{x}}
  \and
  \inferrule* [lab=name] {} {{x} \bc \quotep{P}}
\end{mathpar} 

Note that $\vec{x}$ (resp. $\vec{P}$) denotes a vector of names
(resp. processes) of length $|\vec{x}|$ (resp. $|\vec{P}|$). We adopt
the following useful abbreviations.

\begin{mathpar}
   x?(\vec{y}).P := x.(\vec{y})P \and  x\clift{\vec{P}} := x.\clift{\vec{P}}
   \and x!(y) := \lift{x}{\dropn{y}}
   \and \Pi_{i=0}^{n-1}P_i := P_0 | \ldots | P_{n-1}
\end{mathpar}

\subsubsection{Structural congruence}

\paragraph{Free and bound names and alpha-equivalence.} At the
core of structural equivalence is alpha-equivalence which identifies
process that are the same up to a change of variable. Formally, we
recognize the distinction between free and bound names. The free names
of a process, $\freenames{P}$, may be calculated recursively as
follows:

\begin{mathpar}
\freenames{\pzero} := \emptyset
  \and \\
  \freenames{x?(y).P} := \{ x \} \cup (\freenames{P} \setminus \{ y \})
  \and 
  \freenames{x!\langle P \rangle} := \{ x \} \cup \{ P \} 
  \and \\
  \freenames{P|Q} := \freenames{P} \cup \freenames{Q}
  \and \\
  \freenames{@{x}} := \{ x \}
\end{mathpar}

$\pi$
$\quotep{\pi}$

$\freenames{-} : \pi \to \mathcal{P}(\quotep{\pi})$

\begin{eqnarray*}
  \freenames{\pzero} & := & \emptyset \\
  \freenames{x?(y).P} & := & \{ x \} \cup (\freenames{P} \setminus \{ y \}) \\
  \freenames{x!\langle P \rangle} & := & \{ x \} \cup \{ P \} \\
  \freenames{P|Q} & := & \freenames{P} \cup \freenames{Q} \\
  \freenames{\dropn{x}} & := & \{ x \}
\end{eqnarray*}

The bound names of a process, $\boundnames{P}$, are those names occurring in $P$
that are not free. For example, in $x?(y).0$, the name $x$ is free, while $y$ is bound.

\begin{mathpar}
  \inferrule* [lab=monoidal-laws] {} { P|Q \equiv Q|P \and P|0 \equiv P \and P|(Q|R) \equiv (P|Q)|R }
\end{mathpar}

\begin{mathpar}
  \inferrule* [lab=alpha-equivalence] {} { (x)P \equiv (y)P\{y/x\} \and y \not\in \freenames{P} }
\end{mathpar}

\begin{definition}
Then two processes, $P,Q$, are alpha-equivalent if $P = Q\{\vec{y}/\vec{x}\}$ for
some $\vec{x} \in \boundnames{Q},\vec{y} \in \boundnames{P}$, where $Q\{\vec{y}/\vec{x}\}$
denotes the capture-avoiding substitution of $\vec{y}$ for $\vec{x}$ in $Q$.
\end{definition}

\begin{definition}
  The {\em structural congruence} \cite{SangiorgiWalker} , $\equiv$,
  between processes is the least congruence containing
  alpha-equivalence, satisfying the abelian monoid laws
  (associativity, commutativity and $\pzero$ as identity) for parallel
  composition $|$ and for summation $+$.
\end{definition}

\subsection{Name equivalence}

We take name equivalence, written $\nameeq$, to be the smallest
equivalence relation generated by the following rules.

\begin{mathpar}
\inferrule*[lab=Quote-drop]
{ }
{ \quotep{@{x}} \nameeq x }

\inferrule*[lab=Struct-equiv]
{ P \scong Q }
{ \quotep{P} \nameeq \quotep{Q} }
\end{mathpar}

The astute reader will have noticed that the mutual recursion of names
and processes imposes a mutual recursion on alpha-equivalence and
structural equivalence via name-equivalence. Fortunately, all of this
works out pleasantly and we may calculate in the natural way, free of
concern. The reader interested in the details is referred to the
appendix \ref{appendix:rho_details}.

\subsection{Substitution}

We use $\Proc$ for the set of processes, $\QProc$ for the set of
names, and $\id{\{}\vec{y} / \vec{x} \id{\}}$ to denote partial maps,
$s : \QProc \rightarrow \QProc$. A map, $s$ lifts, uniquely, to a map
on process terms, $\widehat{s} : \Proc \rightarrow \Proc$ by the
following equations.

\begin{mathpar}
  (0) \psubstp{Q}{P} := 0 \\
  (R \juxtap S) \psubstp{Q}{P}
  :=    
  (R)\psubstp{Q}{P} \juxtap (S) \psubstp{Q}{P} \\
  (x?(y).R) \psubstp{Q}{P}    
  :=    
  (x)\substp{Q}{P} (z)\concat( (R \psubstn{z}{y}) \psubstp{Q}{P} ) \\
  (\lift{x}{R}) \psubstp{Q}{P}  
  :=
  \lift{(x)\substp{Q}{P}}{ R \psubstp{Q}{P} } \\
%   (\dropn{x})  \psubstp{Q}{P}       
%   := 
%   \left\{ 
%     \begin{array}{ccc} 
%       \dropn{\quotep{Q}} & & x \nameeq \quotep{P} \\
%       \dropn{x} & & otherwise \\
%     \end{array}
%   \right. 
  (\dropn{x})  \psubstp{Q}{P}       
  := 
  \left\{ 
    \begin{array}{ccc} 
      Q & & x \nameeq \quotep{P} \\
      \dropn{x} & & otherwise \\
    \end{array}
  \right.
\end{mathpar}
 

where

\begin{eqnarray}
  (x)\id{\{} \lpquote Q \rpquote / \lpquote P \rpquote \id{\}}            = 
  \left\{ 
    \begin{array}{ccc}
      \lpquote Q \rpquote & & x \nameeq \lpquote P \rpquote \\
      x & & otherwise \\
    \end{array}
  \right. \nonumber
\end{eqnarray}

and $z$ is chosen distinct from $\quotep{P}$, $\quotep{Q}$, the free
names in $Q$, and all the names in $R$. Our $\alpha$-equivalence will
be built in the standard way from this substitution.

\begin{remark}\label{rem:no_self_referential_names}
  One consequence of these definitions is that $\forall P. \quotep{P}
  \not\in \freenames{P}$.
\end{remark}

\subsection{ Dynamic quote: an example }

Anticipating something of what's to come, consider applying the
substitution, $\widehat{\id{\{}u / z \id{\}}}$, to the following pair
of processes, $\lift{w}{y!(z)}$ and $w[ \lpquote y!(z) \rpquote ]$.

\begin{eqnarray}
	\lift{w}{y!(z)}\widehat{\id{\{}u / z \id{\}}}
		& = &
		\lift{w}{y!(u)} \nonumber\\
	w[ \lpquote y!(z) \rpquote ] \widehat{ \id{\{}u / z \id{\}} }
		& = &
		w[ \lpquote y!(z) \rpquote ] \nonumber
\end{eqnarray}

Because the body of the process between quotes is impervious to
substitution, we get radically different answers. In fact, by
examining the first process in an input context,
e.g. $x?(z).\lift{w}{y!(z)}$, we see that the process under the lift
operator may be shaped by prefixed inputs binding a name inside it. In
this sense, the lift operator will be seen as a way to dynamically
construct processes before reifying them as names.

Finally equipped with these standard features we can present the
dynamics of the calculus.

\subsubsection{Operational semantics} 

Finally, we introduce the computational dynamics. What marks these
algebras as distinct from other more traditionally studied algebraic
structures, e.g. vector spaces or polynomial rings, is the manner in
which dynamics is captured. In traditional structures, dynamics is typically
expressed through morphisms between such structures, as in linear maps
between vector spaces or morphisms between rings. In algebras
associated with the semantics of computation, the dynamics is
expressed as part of the algebraic structure itself, through a
reduction reduction relation typically denoted by $\red$. Below, we
give a recursive presentation of this relation for the calculus used
in the encoding.

$\red \subseteq \pi \times \pi$
$\red : \pi \to \mathcal{P}(\pi)$

\begin{mathpar}
  \inferrule* [lab=Comm] { \textsf{match}( x_{src}, x_{trgt} ) } { x_{trgt}?(y)P \; | \; x_{src}!\langle {Q} \rangle \red P\{\quotep{Q}/y}\} }
  \and \\
  \inferrule* [lab=Par] {{P} \red {P}'} {{{P} | {Q}} \red {{P}' | {Q}}}
  \and
  \inferrule* [lab=Equiv]{{{P} \scong {P}'} \andalso {{P}' \red {Q}'} \andalso {{Q}' \scong {Q}}}{{P} \red {Q}}
\end{mathpar}

\begin{eqnarray*}
  match_{\equiv} (\quotep{P},\quotep{Q}) & := & P \equiv Q \\
  match_{\dagger}(\quotep{P},\quotep{Q}) & := & \forall R. P|Q \red^{*} R => R \red^{*} 0 \\
  match_{K}(\quotep{P},\quotep{Q}) & := & K \mbox{ for some context } K
\end{eqnarray*}

$u?(x)P | u!\langle Q \rangle \red P\{\quotep{Q}/x\}$

%We write $\wred$ for $\red^*$, and $P\red$ if $\exists Q $ such that $ P \red Q$.
We write $P\red$ if $\exists Q $ such that $ P \red Q$ and $P\not\red$, otherwise.

\section{Replication}

As mentioned before, it is known that replication (and hence
recursion) can be implemented in a higher-order process algebra
\cite{SangiorgiWalker}. As our first example of calculation with the
machinery thus far presented we give the construction explicitly in
the {\rhoc}.

\begin{eqnarray}
	D_{x} & := & \prefix{x}{y}{(\binpar{\outputp{x}{y}}{@{y}})} \nonumber\\
	\bangp_{x}{P} & := & \binpar{{x}!\langle{\binpar{D_{x}}{P}}\rangle}{D_{x}} \nonumber
\end{eqnarray}

\begin{eqnarray}
	\bangp_{x}{P} & & \nonumber\\
	=
	& {x}!\langle{(\prefix{x}{y}{(\outputp{x}{y} | @{y})) | P}}\rangle 
	      | \prefix{x}{y}{(\outputp{x}{y} | @{y})} & \nonumber\\
	\red
	& (\outputp{x}{y} | @{y})\substn{\quotep{(\prefix{x}{y}{(@{y} | \outputp{x}{y})) | P}}}{y} & \nonumber\\
	=
	& \outputp{x}{\quotep{(\prefix{x}{y}{(\outputp{x}{y} | @{y})) | P}}}
	  | {(\prefix{x}{y}{(\outputp{x}{y} | @{y})) | P}} & \nonumber\\
	\red
	& \ldots & \nonumber\\
	\red^*
	& P | P | \ldots & \nonumber
\end{eqnarray}

Of course, this encoding, as an implementation, runs away, unfolding
$\bangp{P}$ eagerly. A lazier and more implementable replication
operator, restricted to input-guarded processes, may be obtained as follows.

\begin{eqnarray}
\bangp{\prefix{u}{v}{P}} 
	:= 
	\binpar{\lift{x}{\prefix{u}{v}{(\binpar{D(x)}{P})}}}{D(x)} \nonumber
\end{eqnarray}

\begin{remark}
  Note that the lazier definition still does not deal with summation
  or mixed summation (i.e. sums over input and output). The reader is
  invited to construct definitions of replication that deal with these
  features. 

  Further, the definitions are parameterized in a name, $x$. Can you,
  gentle reader, make a definition that eliminates this parameter and
  guarantees no accidental interaction between the replication
  machinery and the process being replicated -- i.e. no accidental
  sharing of names used by the process to get its work done and the
  name(s) used by the replication to effect copying. This latter
  revision of the definition of replication is crucial to obtaining
  the expected identity $!!P \sim !P$.
\end{remark}

\begin{remark}\label{rem:paradoxical_combinator}
  The reader familiar with the lambda calculus will have noticed the
  similarity between $D$ and the paradoxical combinator.

  [Ed. note: the existence of this seems to suggest we have to be more
  restrictive on the set of processes and names we admit if we are to
  support no-cloning.]
\end{remark}

\subsubsection{Bisimulation}

The computational dynamics gives rise to another kind of equivalence,
the equivalence of computational behavior. As previously mentioned
this is typically captured \emph{via} some form of bisimulation.

% The notion we use in this paper is weak barbed bisimulation
% \cite{milner91polyadicpi}.

The notion we use in this paper is derived from weak barbed
bisimulation \cite{milner91polyadicpi}. 

\begin{definition}
An \emph{observation relation}, $\downarrow_{\mathcal N}$, over a set
of names, $\mathcal N$, is the smallest relation satisfying the rules
below.

\infrule[Out-barb]{y \in {\mathcal N}, \; x \nameeq y}
		  {\outputp{x}{v} \downarrow_{\mathcal N} x}
\infrule[Par-barb]{\mbox{$P\downarrow_{\mathcal N} x$ or $Q\downarrow_{\mathcal N} x$}}
		  {\binpar{P}{Q} \downarrow_{\mathcal N} x}

We write $P \Downarrow_{\mathcal N} x$ if there is $Q$ such that 
$P \wred Q$ and $Q \downarrow_{\mathcal N} x$.
\end{definition}

\begin{definition}
%\label{def.bbisim}
An  ${\mathcal N}$-\emph{barbed bisimulation} over a set of names, ${\mathcal N}$, is a symmetric binary relation 
${\mathcal S}_{\mathcal N}$ between agents such that $P\rel{S}_{\mathcal N}Q$ implies:
\begin{enumerate}
\item If $P \red P'$ then $Q \wred Q'$ and $P'\rel{S}_{\mathcal N} Q'$.
\item If $P\downarrow_{\mathcal N} x$, then $Q\Downarrow_{\mathcal N} x$.
\end{enumerate}
$P$ is ${\mathcal N}$-barbed bisimilar to $Q$, written
$P \wbbisim_{\mathcal N} Q$, if $P \rel{S}_{\mathcal N} Q$ for some ${\mathcal N}$-barbed bisimulation ${\mathcal S}_{\mathcal N}$.
\end{definition}

$\mathcal{R} \subseteq \pi \times \pi$

$P \mathcal{R} Q => \forall P'. P \red P' \Rightarrow \exists Q'. Q \red Q', P' \mathcal{R} Q'$

$P \vdash x \Rightarrow Q \vdash x$

\begin{mathpar}
  \inferrule*[lab=Out-barb]{x \nameeq y}{{y}!\langle{Q}\rangle \vdash x}
  \and
  \inferrule*[lab=Par-barb]{\mbox{$P\vdash x$ or $Q\vdash x$}}{\binpar{P}{Q} \vdash x}
\end{mathpar}

\subsubsection{Contexts}

One of the principle advantages of computational calculi like the
$\pi$-calculus is a well-defined notion of context,
contextual-equivalence and a correlation between
contextual-equivalence and notions of bisimulation. The notion of
context allows the decomposition of a process into (sub-)process and
its syntactic environment, its context. Thus, a context may be
thought of as a process with a ``hole'' (written $\Box$) in it. The
application of a context $M$ to a process $P$, written $M[P]$, is
tantamount to filling the hole in $M$ with $P$. In this paper we do
not need the full weight of this theory, but do make use of the notion
of context in the proof the main theorem. 

\begin{mathpar}
  \inferrule* [lab=summation] {} {{M_{M},M_{N}} \bc \Box \;|\; x.M_{A} \;|\; M_{M}+M_{N}}
  \and
  \inferrule* [lab=agent] {} {{M_{A}} \bc (\vec{x})M_{P} \;| \; \clift{P_0,\ldots,M_{P},\ldots,P_N}}
  \and \\
  \inferrule* [lab=process] {} {{M_{P}} \bc M_{N} \;| \;P|M_{P} }
\end{mathpar} 

\begin{mathpar}
  \inferrule* [lab=sychronization] {} {M_{N} \bc \Box \;|\; x?M_{F} \;|\; x!M_{C}}
  \and
  \inferrule* [lab=abstraction] {} {{M_{F}} \bc (x)M_{P} }
  \and
  \inferrule* [lab=concretion] {} {{M_{C}} \bc \langle M_{P} \rangle }
  \and \\
  \inferrule* [lab=process] {} {{M_{P}} \bc M_{N} \;| \;P|M_{P} }
\end{mathpar}

\begin{definition}[contextual application] Given a context $M$, and
  process $P$, we define the \emph{contextual application}, $M[P] :=
  M\{P/\Box\}$. That is, the contextual application of M to P is the
  substitution of $P$ for $\Box$ in $M$.
\end{definition}

$\meaningof{-} : L \to \mathcal{P}(\pi)$

\begin{mathpar}
  \inferrule* [lab=collection] {} {\meaningof{true} = \pi, \and \meaningof{~E} = \pi \setminus \meaningof{E}, \and \meaningof{E_{1} \& E_{2}} = \meaningof{E_{1}} \cap \meaningof{E_{2}}}
\end{mathpar}

\begin{mathpar}
  \inferrule* [lab=structure] {} {\meaningof{0} = \{ P \in \pi | P \equiv 0 \}, \and \\ \meaningof{E_1 | E_2} = \{ P \in \pi | P \equiv P_{1} | P_{2}, P_{1} \in \meaningof{E_{1}}, P_{2} \in \meaningof{E_2}\} }
\end{mathpar}

\begin{mathpar}
 \inferrule* [lab=behavior] {} {\meaningof{\langle a?b \rangle E} = \{ P \in \pi | P \equiv Q | u?(y)P', \\ \and \\\\ \and \\ \;\;\; u \in \meaningof{a}, \forall z.P'\{z/y\} \in \meaningof{E\{z/b\}}\}, \and \\ \meaningof{a!E} = \{ P \in \pi | P \equiv Q | x!\langle P' \rangle, x \in \meaningof{a} P' \in \meaningof{E}\} }
\end{mathpar}

\begin{mathpar}
 \inferrule* [lab=nominal] {} {\meaningof{\quotep{E}} = \{ \quotep{P} \in \quotep{\pi} | P \in \meaningof{E} \}, \and \meaningof{\quotep{P}} = \{ \quotep{Q} \in \quotep{\pi} | P \equiv Q \} \and \\ \meaningof{@\quotep{E}} = \{ P \in \pi | P \equiv @x, x \in \meaningof{E} \}}
\end{mathpar}

\begin{eqnarray*}
  \\
  \meaningof{-} : TS \to ST
\end{eqnarray*}

\begin{eqnarray*}
  \\
  L : TS \to ST
\end{eqnarray*}

\begin{eqnarray*}
  \\
  P \models E \iff P \in \meaningof{E}
\end{eqnarray*}

\begin{eqnarray*}
  P \approx_{L} Q \iff \forall E \in L. P \models E \iff Q \models E
\end{eqnarray*}

\begin{eqnarray*}
  P \approx_{K} Q
\end{eqnarray*}

\begin{eqnarray*}
  P \approx Q
\end{eqnarray*}

$\approx_{K} = \approx = \approx_{L}$

\subsubsection{Contextual duality}

Note that contexts extend the quotation operation to a family of
operations from processes to names. Given a context, $M$, we can
define a \emph{nominal context}, $\quotep{M}$ by $\quotep{M}[P] :=
\quotep{M[P]}$. To foreshadow what is to come we observe that these
operations enjoy a duality with processes very much like the duality
between vectors and maps from vectors to scalars.

Further, because the calculus is essentially higher-order, we have a
correspondence between contexts and processes. More specifically,
given a name $x$ and a context $M$ we can construct $M^{*}_{x}$ such
that 

\begin{mathpar}
  M^{*}_{x} | \lift{x}{P} \red M[P]
\end{mathpar}

namely,

\begin{mathpar}
  M^{*}_{x} := x?(u).M[\dropn{u}]
\end{mathpar}

The dependence of $M^{*}_{x}$ on a name makes it an abstraction, 

\begin{mathpar}
  M^{*} := (x)x?(u).M[\dropn{u}]
\end{mathpar}

\subsection{Additional notation}

It will sometimes be convenient to denote the process a name
quotes. We already have the notation $x = \quotep{P}$, but it will be
convenient to introduce an alternate notation, $\procn{x}$, when we
want to emphasize the connection to the use of the name. Note that, by
virtue of name equivalence, $\quotep{\procn{x}} \nameeq x$; so, the
notation is consistent with previous definitions.

Further, because names have structure it is possible to effect
substitutions on the basis of that structure. This means we need to
upgrade our notation for substitutions, which we accomplish by
adapting comprehension notation. Thus,

\begin{mathpar}
  P\{ y / x : x \in S \}
\end{mathpar}

is interpreted to mean the process derived from P by replacing (in a
capture-avoiding manner) each occurrence of $x$ in $S$ by $y$. For example,

\begin{mathpar}
  P\{ \quotep{\procn{x}|\procn{x}} / x : x \in \freenames{P} \}
\end{mathpar}

will replace each (occurrence) of a free name $x$ in $P$ by
$\quotep{\procn{x}|\procn{x}}$.

Also, we will avail ourselves of the notation $x^{L}$ and $x^{R}$ to
denote injections of a name into disjoint copies of the name
space. There are numerous ways to accomplish this. One example can be
found in \cite{MeredithR05}. This notation overloads to vectors of
names: $\vec{x}^{\pi} := (x_{i}^{\pi} \; : \; 0 \leq i < |\vec{x}| )$ where $\pi \in \{L,R\}$.

We also use $P^{\Box} := P|\Box$.

In \cite{MeredithR05} an interpretation of the new operator is
given. It turns out that there are several possible interpretations
all enjoying the requisite algebraic properties of the operator (see
\cite{milner91polyadicpi}). We will therefore make liberal use of
$(\nu\; \vec{x})P$.

% subsection the_syntax_and_semantics_of_the_notation_system (end)   

\input{qm2pi.qmops} 

\input{qm2pi.sterngerlach} 

\input{qm2pi.metric} 

% section concurrent_process_calculi (end)

%\input{qm2pi.proofsketch}

% section proof sketch (end)

%\input{qm2pi.slviaknots} 

% section spatial logic via knots (end)

\input{qm2pi.conclusion}

% section conclusion (end)

%\input{qm2pi.dtcodes} 

% section wiring algorithm (end)

\input{qm2pi.ack} 

% section acknowledgments (end)

\newpage


\bibliographystyle{plain}   
\bibliography{../../biblios/main.bib}

\input{qm2pi.rhodetails}

\end{document}

 

% section acknowledgments (end)

\newpage


\bibliographystyle{plain}   
\bibliography{../../biblios/main.bib}

\documentclass[12pt]{llncs}
%\documentclass{jktr}

\usepackage[pdftex]{hyperref}                   
\usepackage {listings}
\usepackage {mathpartir}
\usepackage{bcprules}
%\usepackage{listings}
                       
\usepackage{graphicx} 
%\usepackage[margins=2.5cm,nohead,nofoot]{geometry}
%\usepackage{geometry}
\usepackage{amsfonts}
\usepackage{amstext}
\usepackage{latexsym}
\usepackage{amssymb}
\usepackage{color}


%\include{myPreamble}
\include{qm2pi.local} 

%\ifpdf
%\usepackage[pdftex]{graphicx}
%\else
%\usepackage{graphicx}
%\fi

 % \ifpdf
%  \usepackage{pdfsync}
%  \if


%\title{Brief Article}
%\author{David F. Snyder}
%\author{L.G. Meredith}

%\address{Dept. of Math., Texas State University--San Marcos, San Marcos, TX 78666}
       
\pagestyle{empty}


\begin{document}

\lstset{language=[Objective]Caml,frame=shadowbox}

\input{qm2pi.front}

% section front matter (end)

\input{qm2pi.intro} 
 
% section introduction (end)

% \input{qm2pi.knotations} 

% section notation (end)

\input{qm2pi.process.calculi} 

% section concurrent_process_calculi_and_spatial_logics_ (end)
    
%\input{qm2pi.knots2pi} 

%\input{qm2pi.trefoil} 

%\input{qm2pi.mainthm} 

% subsection basic_interpretation (end)

%\input{qm2pi.rho.presentation} 
\subsection{The syntax and semantics of the notation system}\label{sub:the_syntax_and_semantics_of_the_notation_system} % (fold)

We now summarize a technical presentation of the calculus that
embodies our theory of dynamics. The typical presentation of such a
calculus follows the style of giving generators and relations on
them. The grammar, below, describing term constructors, freely
generates the set of processes, $\Proc$. This set is then quotiented
by a relation known as structural congruence and it is over this set
that the notion of dynamics is expressed. This presentation is
essentially that of \cite{MeredithR05} with the addition of
polyadicity and summation. For readability we have relegated some of
the technical subtleties to an appendix.

\subsubsection{Process grammar}\label{subsub:process_grammar}

\begin{mathpar}
  \inferrule* [lab=synchronization] {} {{M} \bc \pzero \;|\; x?F \;|\; x!C }
  \and
  \inferrule* [lab=abstraction] {} {{F} \bc (x)P}
  \and
  \inferrule* [lab=concretion] {} {{C} \bc \langle Q \rangle}
  \and
  \inferrule* [lab=process] {} {{P,Q} \bc M \;| \;P|Q \;|\; @{x}}
  \and
  \inferrule* [lab=name] {} {{x} \bc \quotep{P}}
\end{mathpar} 

Note that $\vec{x}$ (resp. $\vec{P}$) denotes a vector of names
(resp. processes) of length $|\vec{x}|$ (resp. $|\vec{P}|$). We adopt
the following useful abbreviations.

\begin{mathpar}
   x?(\vec{y}).P := x.(\vec{y})P \and  x\clift{\vec{P}} := x.\clift{\vec{P}}
   \and x!(y) := \lift{x}{\dropn{y}}
   \and \Pi_{i=0}^{n-1}P_i := P_0 | \ldots | P_{n-1}
\end{mathpar}

\subsubsection{Structural congruence}

\paragraph{Free and bound names and alpha-equivalence.} At the
core of structural equivalence is alpha-equivalence which identifies
process that are the same up to a change of variable. Formally, we
recognize the distinction between free and bound names. The free names
of a process, $\freenames{P}$, may be calculated recursively as
follows:

\begin{mathpar}
\freenames{\pzero} := \emptyset
  \and \\
  \freenames{x?(y).P} := \{ x \} \cup (\freenames{P} \setminus \{ y \})
  \and 
  \freenames{x!\langle P \rangle} := \{ x \} \cup \{ P \} 
  \and \\
  \freenames{P|Q} := \freenames{P} \cup \freenames{Q}
  \and \\
  \freenames{@{x}} := \{ x \}
\end{mathpar}

$\pi$
$\quotep{\pi}$

$\freenames{-} : \pi \to \mathcal{P}(\quotep{\pi})$

\begin{eqnarray*}
  \freenames{\pzero} & := & \emptyset \\
  \freenames{x?(y).P} & := & \{ x \} \cup (\freenames{P} \setminus \{ y \}) \\
  \freenames{x!\langle P \rangle} & := & \{ x \} \cup \{ P \} \\
  \freenames{P|Q} & := & \freenames{P} \cup \freenames{Q} \\
  \freenames{\dropn{x}} & := & \{ x \}
\end{eqnarray*}

The bound names of a process, $\boundnames{P}$, are those names occurring in $P$
that are not free. For example, in $x?(y).0$, the name $x$ is free, while $y$ is bound.

\begin{mathpar}
  \inferrule* [lab=monoidal-laws] {} { P|Q \equiv Q|P \and P|0 \equiv P \and P|(Q|R) \equiv (P|Q)|R }
\end{mathpar}

\begin{mathpar}
  \inferrule* [lab=alpha-equivalence] {} { (x)P \equiv (y)P\{y/x\} \and y \not\in \freenames{P} }
\end{mathpar}

\begin{definition}
Then two processes, $P,Q$, are alpha-equivalent if $P = Q\{\vec{y}/\vec{x}\}$ for
some $\vec{x} \in \boundnames{Q},\vec{y} \in \boundnames{P}$, where $Q\{\vec{y}/\vec{x}\}$
denotes the capture-avoiding substitution of $\vec{y}$ for $\vec{x}$ in $Q$.
\end{definition}

\begin{definition}
  The {\em structural congruence} \cite{SangiorgiWalker} , $\equiv$,
  between processes is the least congruence containing
  alpha-equivalence, satisfying the abelian monoid laws
  (associativity, commutativity and $\pzero$ as identity) for parallel
  composition $|$ and for summation $+$.
\end{definition}

\subsection{Name equivalence}

We take name equivalence, written $\nameeq$, to be the smallest
equivalence relation generated by the following rules.

\begin{mathpar}
\inferrule*[lab=Quote-drop]
{ }
{ \quotep{@{x}} \nameeq x }

\inferrule*[lab=Struct-equiv]
{ P \scong Q }
{ \quotep{P} \nameeq \quotep{Q} }
\end{mathpar}

The astute reader will have noticed that the mutual recursion of names
and processes imposes a mutual recursion on alpha-equivalence and
structural equivalence via name-equivalence. Fortunately, all of this
works out pleasantly and we may calculate in the natural way, free of
concern. The reader interested in the details is referred to the
appendix \ref{appendix:rho_details}.

\subsection{Substitution}

We use $\Proc$ for the set of processes, $\QProc$ for the set of
names, and $\id{\{}\vec{y} / \vec{x} \id{\}}$ to denote partial maps,
$s : \QProc \rightarrow \QProc$. A map, $s$ lifts, uniquely, to a map
on process terms, $\widehat{s} : \Proc \rightarrow \Proc$ by the
following equations.

\begin{mathpar}
  (0) \psubstp{Q}{P} := 0 \\
  (R \juxtap S) \psubstp{Q}{P}
  :=    
  (R)\psubstp{Q}{P} \juxtap (S) \psubstp{Q}{P} \\
  (x?(y).R) \psubstp{Q}{P}    
  :=    
  (x)\substp{Q}{P} (z)\concat( (R \psubstn{z}{y}) \psubstp{Q}{P} ) \\
  (\lift{x}{R}) \psubstp{Q}{P}  
  :=
  \lift{(x)\substp{Q}{P}}{ R \psubstp{Q}{P} } \\
%   (\dropn{x})  \psubstp{Q}{P}       
%   := 
%   \left\{ 
%     \begin{array}{ccc} 
%       \dropn{\quotep{Q}} & & x \nameeq \quotep{P} \\
%       \dropn{x} & & otherwise \\
%     \end{array}
%   \right. 
  (\dropn{x})  \psubstp{Q}{P}       
  := 
  \left\{ 
    \begin{array}{ccc} 
      Q & & x \nameeq \quotep{P} \\
      \dropn{x} & & otherwise \\
    \end{array}
  \right.
\end{mathpar}
 

where

\begin{eqnarray}
  (x)\id{\{} \lpquote Q \rpquote / \lpquote P \rpquote \id{\}}            = 
  \left\{ 
    \begin{array}{ccc}
      \lpquote Q \rpquote & & x \nameeq \lpquote P \rpquote \\
      x & & otherwise \\
    \end{array}
  \right. \nonumber
\end{eqnarray}

and $z$ is chosen distinct from $\quotep{P}$, $\quotep{Q}$, the free
names in $Q$, and all the names in $R$. Our $\alpha$-equivalence will
be built in the standard way from this substitution.

\begin{remark}\label{rem:no_self_referential_names}
  One consequence of these definitions is that $\forall P. \quotep{P}
  \not\in \freenames{P}$.
\end{remark}

\subsection{ Dynamic quote: an example }

Anticipating something of what's to come, consider applying the
substitution, $\widehat{\id{\{}u / z \id{\}}}$, to the following pair
of processes, $\lift{w}{y!(z)}$ and $w[ \lpquote y!(z) \rpquote ]$.

\begin{eqnarray}
	\lift{w}{y!(z)}\widehat{\id{\{}u / z \id{\}}}
		& = &
		\lift{w}{y!(u)} \nonumber\\
	w[ \lpquote y!(z) \rpquote ] \widehat{ \id{\{}u / z \id{\}} }
		& = &
		w[ \lpquote y!(z) \rpquote ] \nonumber
\end{eqnarray}

Because the body of the process between quotes is impervious to
substitution, we get radically different answers. In fact, by
examining the first process in an input context,
e.g. $x?(z).\lift{w}{y!(z)}$, we see that the process under the lift
operator may be shaped by prefixed inputs binding a name inside it. In
this sense, the lift operator will be seen as a way to dynamically
construct processes before reifying them as names.

Finally equipped with these standard features we can present the
dynamics of the calculus.

\subsubsection{Operational semantics} 

Finally, we introduce the computational dynamics. What marks these
algebras as distinct from other more traditionally studied algebraic
structures, e.g. vector spaces or polynomial rings, is the manner in
which dynamics is captured. In traditional structures, dynamics is typically
expressed through morphisms between such structures, as in linear maps
between vector spaces or morphisms between rings. In algebras
associated with the semantics of computation, the dynamics is
expressed as part of the algebraic structure itself, through a
reduction reduction relation typically denoted by $\red$. Below, we
give a recursive presentation of this relation for the calculus used
in the encoding.

$\red \subseteq \pi \times \pi$
$\red : \pi \to \mathcal{P}(\pi)$

\begin{mathpar}
  \inferrule* [lab=Comm] { \textsf{match}( x_{src}, x_{trgt} ) } { x_{trgt}?(y)P \; | \; x_{src}!\langle {Q} \rangle \red P\{\quotep{Q}/y}\} }
  \and \\
  \inferrule* [lab=Par] {{P} \red {P}'} {{{P} | {Q}} \red {{P}' | {Q}}}
  \and
  \inferrule* [lab=Equiv]{{{P} \scong {P}'} \andalso {{P}' \red {Q}'} \andalso {{Q}' \scong {Q}}}{{P} \red {Q}}
\end{mathpar}

\begin{eqnarray*}
  match_{\equiv} (\quotep{P},\quotep{Q}) & := & P \equiv Q \\
  match_{\dagger}(\quotep{P},\quotep{Q}) & := & \forall R. P|Q \red^{*} R => R \red^{*} 0 \\
  match_{K}(\quotep{P},\quotep{Q}) & := & K \mbox{ for some context } K
\end{eqnarray*}

$u?(x)P | u!\langle Q \rangle \red P\{\quotep{Q}/x\}$

%We write $\wred$ for $\red^*$, and $P\red$ if $\exists Q $ such that $ P \red Q$.
We write $P\red$ if $\exists Q $ such that $ P \red Q$ and $P\not\red$, otherwise.

\section{Replication}

As mentioned before, it is known that replication (and hence
recursion) can be implemented in a higher-order process algebra
\cite{SangiorgiWalker}. As our first example of calculation with the
machinery thus far presented we give the construction explicitly in
the {\rhoc}.

\begin{eqnarray}
	D_{x} & := & \prefix{x}{y}{(\binpar{\outputp{x}{y}}{@{y}})} \nonumber\\
	\bangp_{x}{P} & := & \binpar{{x}!\langle{\binpar{D_{x}}{P}}\rangle}{D_{x}} \nonumber
\end{eqnarray}

\begin{eqnarray}
	\bangp_{x}{P} & & \nonumber\\
	=
	& {x}!\langle{(\prefix{x}{y}{(\outputp{x}{y} | @{y})) | P}}\rangle 
	      | \prefix{x}{y}{(\outputp{x}{y} | @{y})} & \nonumber\\
	\red
	& (\outputp{x}{y} | @{y})\substn{\quotep{(\prefix{x}{y}{(@{y} | \outputp{x}{y})) | P}}}{y} & \nonumber\\
	=
	& \outputp{x}{\quotep{(\prefix{x}{y}{(\outputp{x}{y} | @{y})) | P}}}
	  | {(\prefix{x}{y}{(\outputp{x}{y} | @{y})) | P}} & \nonumber\\
	\red
	& \ldots & \nonumber\\
	\red^*
	& P | P | \ldots & \nonumber
\end{eqnarray}

Of course, this encoding, as an implementation, runs away, unfolding
$\bangp{P}$ eagerly. A lazier and more implementable replication
operator, restricted to input-guarded processes, may be obtained as follows.

\begin{eqnarray}
\bangp{\prefix{u}{v}{P}} 
	:= 
	\binpar{\lift{x}{\prefix{u}{v}{(\binpar{D(x)}{P})}}}{D(x)} \nonumber
\end{eqnarray}

\begin{remark}
  Note that the lazier definition still does not deal with summation
  or mixed summation (i.e. sums over input and output). The reader is
  invited to construct definitions of replication that deal with these
  features. 

  Further, the definitions are parameterized in a name, $x$. Can you,
  gentle reader, make a definition that eliminates this parameter and
  guarantees no accidental interaction between the replication
  machinery and the process being replicated -- i.e. no accidental
  sharing of names used by the process to get its work done and the
  name(s) used by the replication to effect copying. This latter
  revision of the definition of replication is crucial to obtaining
  the expected identity $!!P \sim !P$.
\end{remark}

\begin{remark}\label{rem:paradoxical_combinator}
  The reader familiar with the lambda calculus will have noticed the
  similarity between $D$ and the paradoxical combinator.

  [Ed. note: the existence of this seems to suggest we have to be more
  restrictive on the set of processes and names we admit if we are to
  support no-cloning.]
\end{remark}

\subsubsection{Bisimulation}

The computational dynamics gives rise to another kind of equivalence,
the equivalence of computational behavior. As previously mentioned
this is typically captured \emph{via} some form of bisimulation.

% The notion we use in this paper is weak barbed bisimulation
% \cite{milner91polyadicpi}.

The notion we use in this paper is derived from weak barbed
bisimulation \cite{milner91polyadicpi}. 

\begin{definition}
An \emph{observation relation}, $\downarrow_{\mathcal N}$, over a set
of names, $\mathcal N$, is the smallest relation satisfying the rules
below.

\infrule[Out-barb]{y \in {\mathcal N}, \; x \nameeq y}
		  {\outputp{x}{v} \downarrow_{\mathcal N} x}
\infrule[Par-barb]{\mbox{$P\downarrow_{\mathcal N} x$ or $Q\downarrow_{\mathcal N} x$}}
		  {\binpar{P}{Q} \downarrow_{\mathcal N} x}

We write $P \Downarrow_{\mathcal N} x$ if there is $Q$ such that 
$P \wred Q$ and $Q \downarrow_{\mathcal N} x$.
\end{definition}

\begin{definition}
%\label{def.bbisim}
An  ${\mathcal N}$-\emph{barbed bisimulation} over a set of names, ${\mathcal N}$, is a symmetric binary relation 
${\mathcal S}_{\mathcal N}$ between agents such that $P\rel{S}_{\mathcal N}Q$ implies:
\begin{enumerate}
\item If $P \red P'$ then $Q \wred Q'$ and $P'\rel{S}_{\mathcal N} Q'$.
\item If $P\downarrow_{\mathcal N} x$, then $Q\Downarrow_{\mathcal N} x$.
\end{enumerate}
$P$ is ${\mathcal N}$-barbed bisimilar to $Q$, written
$P \wbbisim_{\mathcal N} Q$, if $P \rel{S}_{\mathcal N} Q$ for some ${\mathcal N}$-barbed bisimulation ${\mathcal S}_{\mathcal N}$.
\end{definition}

$\mathcal{R} \subseteq \pi \times \pi$

$P \mathcal{R} Q => \forall P'. P \red P' \Rightarrow \exists Q'. Q \red Q', P' \mathcal{R} Q'$

$P \vdash x \Rightarrow Q \vdash x$

\begin{mathpar}
  \inferrule*[lab=Out-barb]{x \nameeq y}{{y}!\langle{Q}\rangle \vdash x}
  \and
  \inferrule*[lab=Par-barb]{\mbox{$P\vdash x$ or $Q\vdash x$}}{\binpar{P}{Q} \vdash x}
\end{mathpar}

\subsubsection{Contexts}

One of the principle advantages of computational calculi like the
$\pi$-calculus is a well-defined notion of context,
contextual-equivalence and a correlation between
contextual-equivalence and notions of bisimulation. The notion of
context allows the decomposition of a process into (sub-)process and
its syntactic environment, its context. Thus, a context may be
thought of as a process with a ``hole'' (written $\Box$) in it. The
application of a context $M$ to a process $P$, written $M[P]$, is
tantamount to filling the hole in $M$ with $P$. In this paper we do
not need the full weight of this theory, but do make use of the notion
of context in the proof the main theorem. 

\begin{mathpar}
  \inferrule* [lab=summation] {} {{M_{M},M_{N}} \bc \Box \;|\; x.M_{A} \;|\; M_{M}+M_{N}}
  \and
  \inferrule* [lab=agent] {} {{M_{A}} \bc (\vec{x})M_{P} \;| \; \clift{P_0,\ldots,M_{P},\ldots,P_N}}
  \and \\
  \inferrule* [lab=process] {} {{M_{P}} \bc M_{N} \;| \;P|M_{P} }
\end{mathpar} 

\begin{mathpar}
  \inferrule* [lab=sychronization] {} {M_{N} \bc \Box \;|\; x?M_{F} \;|\; x!M_{C}}
  \and
  \inferrule* [lab=abstraction] {} {{M_{F}} \bc (x)M_{P} }
  \and
  \inferrule* [lab=concretion] {} {{M_{C}} \bc \langle M_{P} \rangle }
  \and \\
  \inferrule* [lab=process] {} {{M_{P}} \bc M_{N} \;| \;P|M_{P} }
\end{mathpar}

\begin{definition}[contextual application] Given a context $M$, and
  process $P$, we define the \emph{contextual application}, $M[P] :=
  M\{P/\Box\}$. That is, the contextual application of M to P is the
  substitution of $P$ for $\Box$ in $M$.
\end{definition}

$\meaningof{-} : L \to \mathcal{P}(\pi)$

\begin{mathpar}
  \inferrule* [lab=collection] {} {\meaningof{true} = \pi, \and \meaningof{~E} = \pi \setminus \meaningof{E}, \and \meaningof{E_{1} \& E_{2}} = \meaningof{E_{1}} \cap \meaningof{E_{2}}}
\end{mathpar}

\begin{mathpar}
  \inferrule* [lab=structure] {} {\meaningof{0} = \{ P \in \pi | P \equiv 0 \}, \and \\ \meaningof{E_1 | E_2} = \{ P \in \pi | P \equiv P_{1} | P_{2}, P_{1} \in \meaningof{E_{1}}, P_{2} \in \meaningof{E_2}\} }
\end{mathpar}

\begin{mathpar}
 \inferrule* [lab=behavior] {} {\meaningof{\langle a?b \rangle E} = \{ P \in \pi | P \equiv Q | u?(y)P', \\ \and \\\\ \and \\ \;\;\; u \in \meaningof{a}, \forall z.P'\{z/y\} \in \meaningof{E\{z/b\}}\}, \and \\ \meaningof{a!E} = \{ P \in \pi | P \equiv Q | x!\langle P' \rangle, x \in \meaningof{a} P' \in \meaningof{E}\} }
\end{mathpar}

\begin{mathpar}
 \inferrule* [lab=nominal] {} {\meaningof{\quotep{E}} = \{ \quotep{P} \in \quotep{\pi} | P \in \meaningof{E} \}, \and \meaningof{\quotep{P}} = \{ \quotep{Q} \in \quotep{\pi} | P \equiv Q \} \and \\ \meaningof{@\quotep{E}} = \{ P \in \pi | P \equiv @x, x \in \meaningof{E} \}}
\end{mathpar}

\begin{eqnarray*}
  \\
  \meaningof{-} : TS \to ST
\end{eqnarray*}

\begin{eqnarray*}
  \\
  L : TS \to ST
\end{eqnarray*}

\begin{eqnarray*}
  \\
  P \models E \iff P \in \meaningof{E}
\end{eqnarray*}

\begin{eqnarray*}
  P \approx_{L} Q \iff \forall E \in L. P \models E \iff Q \models E
\end{eqnarray*}

\begin{eqnarray*}
  P \approx_{K} Q
\end{eqnarray*}

\begin{eqnarray*}
  P \approx Q
\end{eqnarray*}

$\approx_{K} = \approx = \approx_{L}$

\subsubsection{Contextual duality}

Note that contexts extend the quotation operation to a family of
operations from processes to names. Given a context, $M$, we can
define a \emph{nominal context}, $\quotep{M}$ by $\quotep{M}[P] :=
\quotep{M[P]}$. To foreshadow what is to come we observe that these
operations enjoy a duality with processes very much like the duality
between vectors and maps from vectors to scalars.

Further, because the calculus is essentially higher-order, we have a
correspondence between contexts and processes. More specifically,
given a name $x$ and a context $M$ we can construct $M^{*}_{x}$ such
that 

\begin{mathpar}
  M^{*}_{x} | \lift{x}{P} \red M[P]
\end{mathpar}

namely,

\begin{mathpar}
  M^{*}_{x} := x?(u).M[\dropn{u}]
\end{mathpar}

The dependence of $M^{*}_{x}$ on a name makes it an abstraction, 

\begin{mathpar}
  M^{*} := (x)x?(u).M[\dropn{u}]
\end{mathpar}

\subsection{Additional notation}

It will sometimes be convenient to denote the process a name
quotes. We already have the notation $x = \quotep{P}$, but it will be
convenient to introduce an alternate notation, $\procn{x}$, when we
want to emphasize the connection to the use of the name. Note that, by
virtue of name equivalence, $\quotep{\procn{x}} \nameeq x$; so, the
notation is consistent with previous definitions.

Further, because names have structure it is possible to effect
substitutions on the basis of that structure. This means we need to
upgrade our notation for substitutions, which we accomplish by
adapting comprehension notation. Thus,

\begin{mathpar}
  P\{ y / x : x \in S \}
\end{mathpar}

is interpreted to mean the process derived from P by replacing (in a
capture-avoiding manner) each occurrence of $x$ in $S$ by $y$. For example,

\begin{mathpar}
  P\{ \quotep{\procn{x}|\procn{x}} / x : x \in \freenames{P} \}
\end{mathpar}

will replace each (occurrence) of a free name $x$ in $P$ by
$\quotep{\procn{x}|\procn{x}}$.

Also, we will avail ourselves of the notation $x^{L}$ and $x^{R}$ to
denote injections of a name into disjoint copies of the name
space. There are numerous ways to accomplish this. One example can be
found in \cite{MeredithR05}. This notation overloads to vectors of
names: $\vec{x}^{\pi} := (x_{i}^{\pi} \; : \; 0 \leq i < |\vec{x}| )$ where $\pi \in \{L,R\}$.

We also use $P^{\Box} := P|\Box$.

In \cite{MeredithR05} an interpretation of the new operator is
given. It turns out that there are several possible interpretations
all enjoying the requisite algebraic properties of the operator (see
\cite{milner91polyadicpi}). We will therefore make liberal use of
$(\nu\; \vec{x})P$.

% subsection the_syntax_and_semantics_of_the_notation_system (end)   

\input{qm2pi.qmops} 

\input{qm2pi.sterngerlach} 

\input{qm2pi.metric} 

% section concurrent_process_calculi (end)

%\input{qm2pi.proofsketch}

% section proof sketch (end)

%\input{qm2pi.slviaknots} 

% section spatial logic via knots (end)

\input{qm2pi.conclusion}

% section conclusion (end)

%\input{qm2pi.dtcodes} 

% section wiring algorithm (end)

\input{qm2pi.ack} 

% section acknowledgments (end)

\newpage


\bibliographystyle{plain}   
\bibliography{../../biblios/main.bib}

\input{qm2pi.rhodetails}

\end{document}



\end{document}

 

% section notation (end)

\input{qm2pi.process.calculi} 

% section concurrent_process_calculi_and_spatial_logics_ (end)
    
%\documentclass[12pt]{llncs}
%\documentclass{jktr}

\usepackage[pdftex]{hyperref}                   
\usepackage {listings}
\usepackage {mathpartir}
\usepackage{bcprules}
%\usepackage{listings}
                       
\usepackage{graphicx} 
%\usepackage[margins=2.5cm,nohead,nofoot]{geometry}
%\usepackage{geometry}
\usepackage{amsfonts}
\usepackage{amstext}
\usepackage{latexsym}
\usepackage{amssymb}
\usepackage{color}


%\include{myPreamble}
\documentclass[12pt]{llncs}
%\documentclass{jktr}

\usepackage[pdftex]{hyperref}                   
\usepackage {listings}
\usepackage {mathpartir}
\usepackage{bcprules}
%\usepackage{listings}
                       
\usepackage{graphicx} 
%\usepackage[margins=2.5cm,nohead,nofoot]{geometry}
%\usepackage{geometry}
\usepackage{amsfonts}
\usepackage{amstext}
\usepackage{latexsym}
\usepackage{amssymb}
\usepackage{color}


%\include{myPreamble}
\include{qm2pi.local} 

%\ifpdf
%\usepackage[pdftex]{graphicx}
%\else
%\usepackage{graphicx}
%\fi

 % \ifpdf
%  \usepackage{pdfsync}
%  \if


%\title{Brief Article}
%\author{David F. Snyder}
%\author{L.G. Meredith}

%\address{Dept. of Math., Texas State University--San Marcos, San Marcos, TX 78666}
       
\pagestyle{empty}


\begin{document}

\lstset{language=[Objective]Caml,frame=shadowbox}

\input{qm2pi.front}

% section front matter (end)

\input{qm2pi.intro} 
 
% section introduction (end)

% \input{qm2pi.knotations} 

% section notation (end)

\input{qm2pi.process.calculi} 

% section concurrent_process_calculi_and_spatial_logics_ (end)
    
%\input{qm2pi.knots2pi} 

%\input{qm2pi.trefoil} 

%\input{qm2pi.mainthm} 

% subsection basic_interpretation (end)

%\input{qm2pi.rho.presentation} 
\subsection{The syntax and semantics of the notation system}\label{sub:the_syntax_and_semantics_of_the_notation_system} % (fold)

We now summarize a technical presentation of the calculus that
embodies our theory of dynamics. The typical presentation of such a
calculus follows the style of giving generators and relations on
them. The grammar, below, describing term constructors, freely
generates the set of processes, $\Proc$. This set is then quotiented
by a relation known as structural congruence and it is over this set
that the notion of dynamics is expressed. This presentation is
essentially that of \cite{MeredithR05} with the addition of
polyadicity and summation. For readability we have relegated some of
the technical subtleties to an appendix.

\subsubsection{Process grammar}\label{subsub:process_grammar}

\begin{mathpar}
  \inferrule* [lab=synchronization] {} {{M} \bc \pzero \;|\; x?F \;|\; x!C }
  \and
  \inferrule* [lab=abstraction] {} {{F} \bc (x)P}
  \and
  \inferrule* [lab=concretion] {} {{C} \bc \langle Q \rangle}
  \and
  \inferrule* [lab=process] {} {{P,Q} \bc M \;| \;P|Q \;|\; @{x}}
  \and
  \inferrule* [lab=name] {} {{x} \bc \quotep{P}}
\end{mathpar} 

Note that $\vec{x}$ (resp. $\vec{P}$) denotes a vector of names
(resp. processes) of length $|\vec{x}|$ (resp. $|\vec{P}|$). We adopt
the following useful abbreviations.

\begin{mathpar}
   x?(\vec{y}).P := x.(\vec{y})P \and  x\clift{\vec{P}} := x.\clift{\vec{P}}
   \and x!(y) := \lift{x}{\dropn{y}}
   \and \Pi_{i=0}^{n-1}P_i := P_0 | \ldots | P_{n-1}
\end{mathpar}

\subsubsection{Structural congruence}

\paragraph{Free and bound names and alpha-equivalence.} At the
core of structural equivalence is alpha-equivalence which identifies
process that are the same up to a change of variable. Formally, we
recognize the distinction between free and bound names. The free names
of a process, $\freenames{P}$, may be calculated recursively as
follows:

\begin{mathpar}
\freenames{\pzero} := \emptyset
  \and \\
  \freenames{x?(y).P} := \{ x \} \cup (\freenames{P} \setminus \{ y \})
  \and 
  \freenames{x!\langle P \rangle} := \{ x \} \cup \{ P \} 
  \and \\
  \freenames{P|Q} := \freenames{P} \cup \freenames{Q}
  \and \\
  \freenames{@{x}} := \{ x \}
\end{mathpar}

$\pi$
$\quotep{\pi}$

$\freenames{-} : \pi \to \mathcal{P}(\quotep{\pi})$

\begin{eqnarray*}
  \freenames{\pzero} & := & \emptyset \\
  \freenames{x?(y).P} & := & \{ x \} \cup (\freenames{P} \setminus \{ y \}) \\
  \freenames{x!\langle P \rangle} & := & \{ x \} \cup \{ P \} \\
  \freenames{P|Q} & := & \freenames{P} \cup \freenames{Q} \\
  \freenames{\dropn{x}} & := & \{ x \}
\end{eqnarray*}

The bound names of a process, $\boundnames{P}$, are those names occurring in $P$
that are not free. For example, in $x?(y).0$, the name $x$ is free, while $y$ is bound.

\begin{mathpar}
  \inferrule* [lab=monoidal-laws] {} { P|Q \equiv Q|P \and P|0 \equiv P \and P|(Q|R) \equiv (P|Q)|R }
\end{mathpar}

\begin{mathpar}
  \inferrule* [lab=alpha-equivalence] {} { (x)P \equiv (y)P\{y/x\} \and y \not\in \freenames{P} }
\end{mathpar}

\begin{definition}
Then two processes, $P,Q$, are alpha-equivalent if $P = Q\{\vec{y}/\vec{x}\}$ for
some $\vec{x} \in \boundnames{Q},\vec{y} \in \boundnames{P}$, where $Q\{\vec{y}/\vec{x}\}$
denotes the capture-avoiding substitution of $\vec{y}$ for $\vec{x}$ in $Q$.
\end{definition}

\begin{definition}
  The {\em structural congruence} \cite{SangiorgiWalker} , $\equiv$,
  between processes is the least congruence containing
  alpha-equivalence, satisfying the abelian monoid laws
  (associativity, commutativity and $\pzero$ as identity) for parallel
  composition $|$ and for summation $+$.
\end{definition}

\subsection{Name equivalence}

We take name equivalence, written $\nameeq$, to be the smallest
equivalence relation generated by the following rules.

\begin{mathpar}
\inferrule*[lab=Quote-drop]
{ }
{ \quotep{@{x}} \nameeq x }

\inferrule*[lab=Struct-equiv]
{ P \scong Q }
{ \quotep{P} \nameeq \quotep{Q} }
\end{mathpar}

The astute reader will have noticed that the mutual recursion of names
and processes imposes a mutual recursion on alpha-equivalence and
structural equivalence via name-equivalence. Fortunately, all of this
works out pleasantly and we may calculate in the natural way, free of
concern. The reader interested in the details is referred to the
appendix \ref{appendix:rho_details}.

\subsection{Substitution}

We use $\Proc$ for the set of processes, $\QProc$ for the set of
names, and $\id{\{}\vec{y} / \vec{x} \id{\}}$ to denote partial maps,
$s : \QProc \rightarrow \QProc$. A map, $s$ lifts, uniquely, to a map
on process terms, $\widehat{s} : \Proc \rightarrow \Proc$ by the
following equations.

\begin{mathpar}
  (0) \psubstp{Q}{P} := 0 \\
  (R \juxtap S) \psubstp{Q}{P}
  :=    
  (R)\psubstp{Q}{P} \juxtap (S) \psubstp{Q}{P} \\
  (x?(y).R) \psubstp{Q}{P}    
  :=    
  (x)\substp{Q}{P} (z)\concat( (R \psubstn{z}{y}) \psubstp{Q}{P} ) \\
  (\lift{x}{R}) \psubstp{Q}{P}  
  :=
  \lift{(x)\substp{Q}{P}}{ R \psubstp{Q}{P} } \\
%   (\dropn{x})  \psubstp{Q}{P}       
%   := 
%   \left\{ 
%     \begin{array}{ccc} 
%       \dropn{\quotep{Q}} & & x \nameeq \quotep{P} \\
%       \dropn{x} & & otherwise \\
%     \end{array}
%   \right. 
  (\dropn{x})  \psubstp{Q}{P}       
  := 
  \left\{ 
    \begin{array}{ccc} 
      Q & & x \nameeq \quotep{P} \\
      \dropn{x} & & otherwise \\
    \end{array}
  \right.
\end{mathpar}
 

where

\begin{eqnarray}
  (x)\id{\{} \lpquote Q \rpquote / \lpquote P \rpquote \id{\}}            = 
  \left\{ 
    \begin{array}{ccc}
      \lpquote Q \rpquote & & x \nameeq \lpquote P \rpquote \\
      x & & otherwise \\
    \end{array}
  \right. \nonumber
\end{eqnarray}

and $z$ is chosen distinct from $\quotep{P}$, $\quotep{Q}$, the free
names in $Q$, and all the names in $R$. Our $\alpha$-equivalence will
be built in the standard way from this substitution.

\begin{remark}\label{rem:no_self_referential_names}
  One consequence of these definitions is that $\forall P. \quotep{P}
  \not\in \freenames{P}$.
\end{remark}

\subsection{ Dynamic quote: an example }

Anticipating something of what's to come, consider applying the
substitution, $\widehat{\id{\{}u / z \id{\}}}$, to the following pair
of processes, $\lift{w}{y!(z)}$ and $w[ \lpquote y!(z) \rpquote ]$.

\begin{eqnarray}
	\lift{w}{y!(z)}\widehat{\id{\{}u / z \id{\}}}
		& = &
		\lift{w}{y!(u)} \nonumber\\
	w[ \lpquote y!(z) \rpquote ] \widehat{ \id{\{}u / z \id{\}} }
		& = &
		w[ \lpquote y!(z) \rpquote ] \nonumber
\end{eqnarray}

Because the body of the process between quotes is impervious to
substitution, we get radically different answers. In fact, by
examining the first process in an input context,
e.g. $x?(z).\lift{w}{y!(z)}$, we see that the process under the lift
operator may be shaped by prefixed inputs binding a name inside it. In
this sense, the lift operator will be seen as a way to dynamically
construct processes before reifying them as names.

Finally equipped with these standard features we can present the
dynamics of the calculus.

\subsubsection{Operational semantics} 

Finally, we introduce the computational dynamics. What marks these
algebras as distinct from other more traditionally studied algebraic
structures, e.g. vector spaces or polynomial rings, is the manner in
which dynamics is captured. In traditional structures, dynamics is typically
expressed through morphisms between such structures, as in linear maps
between vector spaces or morphisms between rings. In algebras
associated with the semantics of computation, the dynamics is
expressed as part of the algebraic structure itself, through a
reduction reduction relation typically denoted by $\red$. Below, we
give a recursive presentation of this relation for the calculus used
in the encoding.

$\red \subseteq \pi \times \pi$
$\red : \pi \to \mathcal{P}(\pi)$

\begin{mathpar}
  \inferrule* [lab=Comm] { \textsf{match}( x_{src}, x_{trgt} ) } { x_{trgt}?(y)P \; | \; x_{src}!\langle {Q} \rangle \red P\{\quotep{Q}/y}\} }
  \and \\
  \inferrule* [lab=Par] {{P} \red {P}'} {{{P} | {Q}} \red {{P}' | {Q}}}
  \and
  \inferrule* [lab=Equiv]{{{P} \scong {P}'} \andalso {{P}' \red {Q}'} \andalso {{Q}' \scong {Q}}}{{P} \red {Q}}
\end{mathpar}

\begin{eqnarray*}
  match_{\equiv} (\quotep{P},\quotep{Q}) & := & P \equiv Q \\
  match_{\dagger}(\quotep{P},\quotep{Q}) & := & \forall R. P|Q \red^{*} R => R \red^{*} 0 \\
  match_{K}(\quotep{P},\quotep{Q}) & := & K \mbox{ for some context } K
\end{eqnarray*}

$u?(x)P | u!\langle Q \rangle \red P\{\quotep{Q}/x\}$

%We write $\wred$ for $\red^*$, and $P\red$ if $\exists Q $ such that $ P \red Q$.
We write $P\red$ if $\exists Q $ such that $ P \red Q$ and $P\not\red$, otherwise.

\section{Replication}

As mentioned before, it is known that replication (and hence
recursion) can be implemented in a higher-order process algebra
\cite{SangiorgiWalker}. As our first example of calculation with the
machinery thus far presented we give the construction explicitly in
the {\rhoc}.

\begin{eqnarray}
	D_{x} & := & \prefix{x}{y}{(\binpar{\outputp{x}{y}}{@{y}})} \nonumber\\
	\bangp_{x}{P} & := & \binpar{{x}!\langle{\binpar{D_{x}}{P}}\rangle}{D_{x}} \nonumber
\end{eqnarray}

\begin{eqnarray}
	\bangp_{x}{P} & & \nonumber\\
	=
	& {x}!\langle{(\prefix{x}{y}{(\outputp{x}{y} | @{y})) | P}}\rangle 
	      | \prefix{x}{y}{(\outputp{x}{y} | @{y})} & \nonumber\\
	\red
	& (\outputp{x}{y} | @{y})\substn{\quotep{(\prefix{x}{y}{(@{y} | \outputp{x}{y})) | P}}}{y} & \nonumber\\
	=
	& \outputp{x}{\quotep{(\prefix{x}{y}{(\outputp{x}{y} | @{y})) | P}}}
	  | {(\prefix{x}{y}{(\outputp{x}{y} | @{y})) | P}} & \nonumber\\
	\red
	& \ldots & \nonumber\\
	\red^*
	& P | P | \ldots & \nonumber
\end{eqnarray}

Of course, this encoding, as an implementation, runs away, unfolding
$\bangp{P}$ eagerly. A lazier and more implementable replication
operator, restricted to input-guarded processes, may be obtained as follows.

\begin{eqnarray}
\bangp{\prefix{u}{v}{P}} 
	:= 
	\binpar{\lift{x}{\prefix{u}{v}{(\binpar{D(x)}{P})}}}{D(x)} \nonumber
\end{eqnarray}

\begin{remark}
  Note that the lazier definition still does not deal with summation
  or mixed summation (i.e. sums over input and output). The reader is
  invited to construct definitions of replication that deal with these
  features. 

  Further, the definitions are parameterized in a name, $x$. Can you,
  gentle reader, make a definition that eliminates this parameter and
  guarantees no accidental interaction between the replication
  machinery and the process being replicated -- i.e. no accidental
  sharing of names used by the process to get its work done and the
  name(s) used by the replication to effect copying. This latter
  revision of the definition of replication is crucial to obtaining
  the expected identity $!!P \sim !P$.
\end{remark}

\begin{remark}\label{rem:paradoxical_combinator}
  The reader familiar with the lambda calculus will have noticed the
  similarity between $D$ and the paradoxical combinator.

  [Ed. note: the existence of this seems to suggest we have to be more
  restrictive on the set of processes and names we admit if we are to
  support no-cloning.]
\end{remark}

\subsubsection{Bisimulation}

The computational dynamics gives rise to another kind of equivalence,
the equivalence of computational behavior. As previously mentioned
this is typically captured \emph{via} some form of bisimulation.

% The notion we use in this paper is weak barbed bisimulation
% \cite{milner91polyadicpi}.

The notion we use in this paper is derived from weak barbed
bisimulation \cite{milner91polyadicpi}. 

\begin{definition}
An \emph{observation relation}, $\downarrow_{\mathcal N}$, over a set
of names, $\mathcal N$, is the smallest relation satisfying the rules
below.

\infrule[Out-barb]{y \in {\mathcal N}, \; x \nameeq y}
		  {\outputp{x}{v} \downarrow_{\mathcal N} x}
\infrule[Par-barb]{\mbox{$P\downarrow_{\mathcal N} x$ or $Q\downarrow_{\mathcal N} x$}}
		  {\binpar{P}{Q} \downarrow_{\mathcal N} x}

We write $P \Downarrow_{\mathcal N} x$ if there is $Q$ such that 
$P \wred Q$ and $Q \downarrow_{\mathcal N} x$.
\end{definition}

\begin{definition}
%\label{def.bbisim}
An  ${\mathcal N}$-\emph{barbed bisimulation} over a set of names, ${\mathcal N}$, is a symmetric binary relation 
${\mathcal S}_{\mathcal N}$ between agents such that $P\rel{S}_{\mathcal N}Q$ implies:
\begin{enumerate}
\item If $P \red P'$ then $Q \wred Q'$ and $P'\rel{S}_{\mathcal N} Q'$.
\item If $P\downarrow_{\mathcal N} x$, then $Q\Downarrow_{\mathcal N} x$.
\end{enumerate}
$P$ is ${\mathcal N}$-barbed bisimilar to $Q$, written
$P \wbbisim_{\mathcal N} Q$, if $P \rel{S}_{\mathcal N} Q$ for some ${\mathcal N}$-barbed bisimulation ${\mathcal S}_{\mathcal N}$.
\end{definition}

$\mathcal{R} \subseteq \pi \times \pi$

$P \mathcal{R} Q => \forall P'. P \red P' \Rightarrow \exists Q'. Q \red Q', P' \mathcal{R} Q'$

$P \vdash x \Rightarrow Q \vdash x$

\begin{mathpar}
  \inferrule*[lab=Out-barb]{x \nameeq y}{{y}!\langle{Q}\rangle \vdash x}
  \and
  \inferrule*[lab=Par-barb]{\mbox{$P\vdash x$ or $Q\vdash x$}}{\binpar{P}{Q} \vdash x}
\end{mathpar}

\subsubsection{Contexts}

One of the principle advantages of computational calculi like the
$\pi$-calculus is a well-defined notion of context,
contextual-equivalence and a correlation between
contextual-equivalence and notions of bisimulation. The notion of
context allows the decomposition of a process into (sub-)process and
its syntactic environment, its context. Thus, a context may be
thought of as a process with a ``hole'' (written $\Box$) in it. The
application of a context $M$ to a process $P$, written $M[P]$, is
tantamount to filling the hole in $M$ with $P$. In this paper we do
not need the full weight of this theory, but do make use of the notion
of context in the proof the main theorem. 

\begin{mathpar}
  \inferrule* [lab=summation] {} {{M_{M},M_{N}} \bc \Box \;|\; x.M_{A} \;|\; M_{M}+M_{N}}
  \and
  \inferrule* [lab=agent] {} {{M_{A}} \bc (\vec{x})M_{P} \;| \; \clift{P_0,\ldots,M_{P},\ldots,P_N}}
  \and \\
  \inferrule* [lab=process] {} {{M_{P}} \bc M_{N} \;| \;P|M_{P} }
\end{mathpar} 

\begin{mathpar}
  \inferrule* [lab=sychronization] {} {M_{N} \bc \Box \;|\; x?M_{F} \;|\; x!M_{C}}
  \and
  \inferrule* [lab=abstraction] {} {{M_{F}} \bc (x)M_{P} }
  \and
  \inferrule* [lab=concretion] {} {{M_{C}} \bc \langle M_{P} \rangle }
  \and \\
  \inferrule* [lab=process] {} {{M_{P}} \bc M_{N} \;| \;P|M_{P} }
\end{mathpar}

\begin{definition}[contextual application] Given a context $M$, and
  process $P$, we define the \emph{contextual application}, $M[P] :=
  M\{P/\Box\}$. That is, the contextual application of M to P is the
  substitution of $P$ for $\Box$ in $M$.
\end{definition}

$\meaningof{-} : L \to \mathcal{P}(\pi)$

\begin{mathpar}
  \inferrule* [lab=collection] {} {\meaningof{true} = \pi, \and \meaningof{~E} = \pi \setminus \meaningof{E}, \and \meaningof{E_{1} \& E_{2}} = \meaningof{E_{1}} \cap \meaningof{E_{2}}}
\end{mathpar}

\begin{mathpar}
  \inferrule* [lab=structure] {} {\meaningof{0} = \{ P \in \pi | P \equiv 0 \}, \and \\ \meaningof{E_1 | E_2} = \{ P \in \pi | P \equiv P_{1} | P_{2}, P_{1} \in \meaningof{E_{1}}, P_{2} \in \meaningof{E_2}\} }
\end{mathpar}

\begin{mathpar}
 \inferrule* [lab=behavior] {} {\meaningof{\langle a?b \rangle E} = \{ P \in \pi | P \equiv Q | u?(y)P', \\ \and \\\\ \and \\ \;\;\; u \in \meaningof{a}, \forall z.P'\{z/y\} \in \meaningof{E\{z/b\}}\}, \and \\ \meaningof{a!E} = \{ P \in \pi | P \equiv Q | x!\langle P' \rangle, x \in \meaningof{a} P' \in \meaningof{E}\} }
\end{mathpar}

\begin{mathpar}
 \inferrule* [lab=nominal] {} {\meaningof{\quotep{E}} = \{ \quotep{P} \in \quotep{\pi} | P \in \meaningof{E} \}, \and \meaningof{\quotep{P}} = \{ \quotep{Q} \in \quotep{\pi} | P \equiv Q \} \and \\ \meaningof{@\quotep{E}} = \{ P \in \pi | P \equiv @x, x \in \meaningof{E} \}}
\end{mathpar}

\begin{eqnarray*}
  \\
  \meaningof{-} : TS \to ST
\end{eqnarray*}

\begin{eqnarray*}
  \\
  L : TS \to ST
\end{eqnarray*}

\begin{eqnarray*}
  \\
  P \models E \iff P \in \meaningof{E}
\end{eqnarray*}

\begin{eqnarray*}
  P \approx_{L} Q \iff \forall E \in L. P \models E \iff Q \models E
\end{eqnarray*}

\begin{eqnarray*}
  P \approx_{K} Q
\end{eqnarray*}

\begin{eqnarray*}
  P \approx Q
\end{eqnarray*}

$\approx_{K} = \approx = \approx_{L}$

\subsubsection{Contextual duality}

Note that contexts extend the quotation operation to a family of
operations from processes to names. Given a context, $M$, we can
define a \emph{nominal context}, $\quotep{M}$ by $\quotep{M}[P] :=
\quotep{M[P]}$. To foreshadow what is to come we observe that these
operations enjoy a duality with processes very much like the duality
between vectors and maps from vectors to scalars.

Further, because the calculus is essentially higher-order, we have a
correspondence between contexts and processes. More specifically,
given a name $x$ and a context $M$ we can construct $M^{*}_{x}$ such
that 

\begin{mathpar}
  M^{*}_{x} | \lift{x}{P} \red M[P]
\end{mathpar}

namely,

\begin{mathpar}
  M^{*}_{x} := x?(u).M[\dropn{u}]
\end{mathpar}

The dependence of $M^{*}_{x}$ on a name makes it an abstraction, 

\begin{mathpar}
  M^{*} := (x)x?(u).M[\dropn{u}]
\end{mathpar}

\subsection{Additional notation}

It will sometimes be convenient to denote the process a name
quotes. We already have the notation $x = \quotep{P}$, but it will be
convenient to introduce an alternate notation, $\procn{x}$, when we
want to emphasize the connection to the use of the name. Note that, by
virtue of name equivalence, $\quotep{\procn{x}} \nameeq x$; so, the
notation is consistent with previous definitions.

Further, because names have structure it is possible to effect
substitutions on the basis of that structure. This means we need to
upgrade our notation for substitutions, which we accomplish by
adapting comprehension notation. Thus,

\begin{mathpar}
  P\{ y / x : x \in S \}
\end{mathpar}

is interpreted to mean the process derived from P by replacing (in a
capture-avoiding manner) each occurrence of $x$ in $S$ by $y$. For example,

\begin{mathpar}
  P\{ \quotep{\procn{x}|\procn{x}} / x : x \in \freenames{P} \}
\end{mathpar}

will replace each (occurrence) of a free name $x$ in $P$ by
$\quotep{\procn{x}|\procn{x}}$.

Also, we will avail ourselves of the notation $x^{L}$ and $x^{R}$ to
denote injections of a name into disjoint copies of the name
space. There are numerous ways to accomplish this. One example can be
found in \cite{MeredithR05}. This notation overloads to vectors of
names: $\vec{x}^{\pi} := (x_{i}^{\pi} \; : \; 0 \leq i < |\vec{x}| )$ where $\pi \in \{L,R\}$.

We also use $P^{\Box} := P|\Box$.

In \cite{MeredithR05} an interpretation of the new operator is
given. It turns out that there are several possible interpretations
all enjoying the requisite algebraic properties of the operator (see
\cite{milner91polyadicpi}). We will therefore make liberal use of
$(\nu\; \vec{x})P$.

% subsection the_syntax_and_semantics_of_the_notation_system (end)   

\input{qm2pi.qmops} 

\input{qm2pi.sterngerlach} 

\input{qm2pi.metric} 

% section concurrent_process_calculi (end)

%\input{qm2pi.proofsketch}

% section proof sketch (end)

%\input{qm2pi.slviaknots} 

% section spatial logic via knots (end)

\input{qm2pi.conclusion}

% section conclusion (end)

%\input{qm2pi.dtcodes} 

% section wiring algorithm (end)

\input{qm2pi.ack} 

% section acknowledgments (end)

\newpage


\bibliographystyle{plain}   
\bibliography{../../biblios/main.bib}

\input{qm2pi.rhodetails}

\end{document}

 

%\ifpdf
%\usepackage[pdftex]{graphicx}
%\else
%\usepackage{graphicx}
%\fi

 % \ifpdf
%  \usepackage{pdfsync}
%  \if


%\title{Brief Article}
%\author{David F. Snyder}
%\author{L.G. Meredith}

%\address{Dept. of Math., Texas State University--San Marcos, San Marcos, TX 78666}
       
\pagestyle{empty}


\begin{document}

\lstset{language=[Objective]Caml,frame=shadowbox}

\documentclass[12pt]{llncs}
%\documentclass{jktr}

\usepackage[pdftex]{hyperref}                   
\usepackage {listings}
\usepackage {mathpartir}
\usepackage{bcprules}
%\usepackage{listings}
                       
\usepackage{graphicx} 
%\usepackage[margins=2.5cm,nohead,nofoot]{geometry}
%\usepackage{geometry}
\usepackage{amsfonts}
\usepackage{amstext}
\usepackage{latexsym}
\usepackage{amssymb}
\usepackage{color}


%\include{myPreamble}
\include{qm2pi.local} 

%\ifpdf
%\usepackage[pdftex]{graphicx}
%\else
%\usepackage{graphicx}
%\fi

 % \ifpdf
%  \usepackage{pdfsync}
%  \if


%\title{Brief Article}
%\author{David F. Snyder}
%\author{L.G. Meredith}

%\address{Dept. of Math., Texas State University--San Marcos, San Marcos, TX 78666}
       
\pagestyle{empty}


\begin{document}

\lstset{language=[Objective]Caml,frame=shadowbox}

\input{qm2pi.front}

% section front matter (end)

\input{qm2pi.intro} 
 
% section introduction (end)

% \input{qm2pi.knotations} 

% section notation (end)

\input{qm2pi.process.calculi} 

% section concurrent_process_calculi_and_spatial_logics_ (end)
    
%\input{qm2pi.knots2pi} 

%\input{qm2pi.trefoil} 

%\input{qm2pi.mainthm} 

% subsection basic_interpretation (end)

%\input{qm2pi.rho.presentation} 
\subsection{The syntax and semantics of the notation system}\label{sub:the_syntax_and_semantics_of_the_notation_system} % (fold)

We now summarize a technical presentation of the calculus that
embodies our theory of dynamics. The typical presentation of such a
calculus follows the style of giving generators and relations on
them. The grammar, below, describing term constructors, freely
generates the set of processes, $\Proc$. This set is then quotiented
by a relation known as structural congruence and it is over this set
that the notion of dynamics is expressed. This presentation is
essentially that of \cite{MeredithR05} with the addition of
polyadicity and summation. For readability we have relegated some of
the technical subtleties to an appendix.

\subsubsection{Process grammar}\label{subsub:process_grammar}

\begin{mathpar}
  \inferrule* [lab=synchronization] {} {{M} \bc \pzero \;|\; x?F \;|\; x!C }
  \and
  \inferrule* [lab=abstraction] {} {{F} \bc (x)P}
  \and
  \inferrule* [lab=concretion] {} {{C} \bc \langle Q \rangle}
  \and
  \inferrule* [lab=process] {} {{P,Q} \bc M \;| \;P|Q \;|\; @{x}}
  \and
  \inferrule* [lab=name] {} {{x} \bc \quotep{P}}
\end{mathpar} 

Note that $\vec{x}$ (resp. $\vec{P}$) denotes a vector of names
(resp. processes) of length $|\vec{x}|$ (resp. $|\vec{P}|$). We adopt
the following useful abbreviations.

\begin{mathpar}
   x?(\vec{y}).P := x.(\vec{y})P \and  x\clift{\vec{P}} := x.\clift{\vec{P}}
   \and x!(y) := \lift{x}{\dropn{y}}
   \and \Pi_{i=0}^{n-1}P_i := P_0 | \ldots | P_{n-1}
\end{mathpar}

\subsubsection{Structural congruence}

\paragraph{Free and bound names and alpha-equivalence.} At the
core of structural equivalence is alpha-equivalence which identifies
process that are the same up to a change of variable. Formally, we
recognize the distinction between free and bound names. The free names
of a process, $\freenames{P}$, may be calculated recursively as
follows:

\begin{mathpar}
\freenames{\pzero} := \emptyset
  \and \\
  \freenames{x?(y).P} := \{ x \} \cup (\freenames{P} \setminus \{ y \})
  \and 
  \freenames{x!\langle P \rangle} := \{ x \} \cup \{ P \} 
  \and \\
  \freenames{P|Q} := \freenames{P} \cup \freenames{Q}
  \and \\
  \freenames{@{x}} := \{ x \}
\end{mathpar}

$\pi$
$\quotep{\pi}$

$\freenames{-} : \pi \to \mathcal{P}(\quotep{\pi})$

\begin{eqnarray*}
  \freenames{\pzero} & := & \emptyset \\
  \freenames{x?(y).P} & := & \{ x \} \cup (\freenames{P} \setminus \{ y \}) \\
  \freenames{x!\langle P \rangle} & := & \{ x \} \cup \{ P \} \\
  \freenames{P|Q} & := & \freenames{P} \cup \freenames{Q} \\
  \freenames{\dropn{x}} & := & \{ x \}
\end{eqnarray*}

The bound names of a process, $\boundnames{P}$, are those names occurring in $P$
that are not free. For example, in $x?(y).0$, the name $x$ is free, while $y$ is bound.

\begin{mathpar}
  \inferrule* [lab=monoidal-laws] {} { P|Q \equiv Q|P \and P|0 \equiv P \and P|(Q|R) \equiv (P|Q)|R }
\end{mathpar}

\begin{mathpar}
  \inferrule* [lab=alpha-equivalence] {} { (x)P \equiv (y)P\{y/x\} \and y \not\in \freenames{P} }
\end{mathpar}

\begin{definition}
Then two processes, $P,Q$, are alpha-equivalent if $P = Q\{\vec{y}/\vec{x}\}$ for
some $\vec{x} \in \boundnames{Q},\vec{y} \in \boundnames{P}$, where $Q\{\vec{y}/\vec{x}\}$
denotes the capture-avoiding substitution of $\vec{y}$ for $\vec{x}$ in $Q$.
\end{definition}

\begin{definition}
  The {\em structural congruence} \cite{SangiorgiWalker} , $\equiv$,
  between processes is the least congruence containing
  alpha-equivalence, satisfying the abelian monoid laws
  (associativity, commutativity and $\pzero$ as identity) for parallel
  composition $|$ and for summation $+$.
\end{definition}

\subsection{Name equivalence}

We take name equivalence, written $\nameeq$, to be the smallest
equivalence relation generated by the following rules.

\begin{mathpar}
\inferrule*[lab=Quote-drop]
{ }
{ \quotep{@{x}} \nameeq x }

\inferrule*[lab=Struct-equiv]
{ P \scong Q }
{ \quotep{P} \nameeq \quotep{Q} }
\end{mathpar}

The astute reader will have noticed that the mutual recursion of names
and processes imposes a mutual recursion on alpha-equivalence and
structural equivalence via name-equivalence. Fortunately, all of this
works out pleasantly and we may calculate in the natural way, free of
concern. The reader interested in the details is referred to the
appendix \ref{appendix:rho_details}.

\subsection{Substitution}

We use $\Proc$ for the set of processes, $\QProc$ for the set of
names, and $\id{\{}\vec{y} / \vec{x} \id{\}}$ to denote partial maps,
$s : \QProc \rightarrow \QProc$. A map, $s$ lifts, uniquely, to a map
on process terms, $\widehat{s} : \Proc \rightarrow \Proc$ by the
following equations.

\begin{mathpar}
  (0) \psubstp{Q}{P} := 0 \\
  (R \juxtap S) \psubstp{Q}{P}
  :=    
  (R)\psubstp{Q}{P} \juxtap (S) \psubstp{Q}{P} \\
  (x?(y).R) \psubstp{Q}{P}    
  :=    
  (x)\substp{Q}{P} (z)\concat( (R \psubstn{z}{y}) \psubstp{Q}{P} ) \\
  (\lift{x}{R}) \psubstp{Q}{P}  
  :=
  \lift{(x)\substp{Q}{P}}{ R \psubstp{Q}{P} } \\
%   (\dropn{x})  \psubstp{Q}{P}       
%   := 
%   \left\{ 
%     \begin{array}{ccc} 
%       \dropn{\quotep{Q}} & & x \nameeq \quotep{P} \\
%       \dropn{x} & & otherwise \\
%     \end{array}
%   \right. 
  (\dropn{x})  \psubstp{Q}{P}       
  := 
  \left\{ 
    \begin{array}{ccc} 
      Q & & x \nameeq \quotep{P} \\
      \dropn{x} & & otherwise \\
    \end{array}
  \right.
\end{mathpar}
 

where

\begin{eqnarray}
  (x)\id{\{} \lpquote Q \rpquote / \lpquote P \rpquote \id{\}}            = 
  \left\{ 
    \begin{array}{ccc}
      \lpquote Q \rpquote & & x \nameeq \lpquote P \rpquote \\
      x & & otherwise \\
    \end{array}
  \right. \nonumber
\end{eqnarray}

and $z$ is chosen distinct from $\quotep{P}$, $\quotep{Q}$, the free
names in $Q$, and all the names in $R$. Our $\alpha$-equivalence will
be built in the standard way from this substitution.

\begin{remark}\label{rem:no_self_referential_names}
  One consequence of these definitions is that $\forall P. \quotep{P}
  \not\in \freenames{P}$.
\end{remark}

\subsection{ Dynamic quote: an example }

Anticipating something of what's to come, consider applying the
substitution, $\widehat{\id{\{}u / z \id{\}}}$, to the following pair
of processes, $\lift{w}{y!(z)}$ and $w[ \lpquote y!(z) \rpquote ]$.

\begin{eqnarray}
	\lift{w}{y!(z)}\widehat{\id{\{}u / z \id{\}}}
		& = &
		\lift{w}{y!(u)} \nonumber\\
	w[ \lpquote y!(z) \rpquote ] \widehat{ \id{\{}u / z \id{\}} }
		& = &
		w[ \lpquote y!(z) \rpquote ] \nonumber
\end{eqnarray}

Because the body of the process between quotes is impervious to
substitution, we get radically different answers. In fact, by
examining the first process in an input context,
e.g. $x?(z).\lift{w}{y!(z)}$, we see that the process under the lift
operator may be shaped by prefixed inputs binding a name inside it. In
this sense, the lift operator will be seen as a way to dynamically
construct processes before reifying them as names.

Finally equipped with these standard features we can present the
dynamics of the calculus.

\subsubsection{Operational semantics} 

Finally, we introduce the computational dynamics. What marks these
algebras as distinct from other more traditionally studied algebraic
structures, e.g. vector spaces or polynomial rings, is the manner in
which dynamics is captured. In traditional structures, dynamics is typically
expressed through morphisms between such structures, as in linear maps
between vector spaces or morphisms between rings. In algebras
associated with the semantics of computation, the dynamics is
expressed as part of the algebraic structure itself, through a
reduction reduction relation typically denoted by $\red$. Below, we
give a recursive presentation of this relation for the calculus used
in the encoding.

$\red \subseteq \pi \times \pi$
$\red : \pi \to \mathcal{P}(\pi)$

\begin{mathpar}
  \inferrule* [lab=Comm] { \textsf{match}( x_{src}, x_{trgt} ) } { x_{trgt}?(y)P \; | \; x_{src}!\langle {Q} \rangle \red P\{\quotep{Q}/y}\} }
  \and \\
  \inferrule* [lab=Par] {{P} \red {P}'} {{{P} | {Q}} \red {{P}' | {Q}}}
  \and
  \inferrule* [lab=Equiv]{{{P} \scong {P}'} \andalso {{P}' \red {Q}'} \andalso {{Q}' \scong {Q}}}{{P} \red {Q}}
\end{mathpar}

\begin{eqnarray*}
  match_{\equiv} (\quotep{P},\quotep{Q}) & := & P \equiv Q \\
  match_{\dagger}(\quotep{P},\quotep{Q}) & := & \forall R. P|Q \red^{*} R => R \red^{*} 0 \\
  match_{K}(\quotep{P},\quotep{Q}) & := & K \mbox{ for some context } K
\end{eqnarray*}

$u?(x)P | u!\langle Q \rangle \red P\{\quotep{Q}/x\}$

%We write $\wred$ for $\red^*$, and $P\red$ if $\exists Q $ such that $ P \red Q$.
We write $P\red$ if $\exists Q $ such that $ P \red Q$ and $P\not\red$, otherwise.

\section{Replication}

As mentioned before, it is known that replication (and hence
recursion) can be implemented in a higher-order process algebra
\cite{SangiorgiWalker}. As our first example of calculation with the
machinery thus far presented we give the construction explicitly in
the {\rhoc}.

\begin{eqnarray}
	D_{x} & := & \prefix{x}{y}{(\binpar{\outputp{x}{y}}{@{y}})} \nonumber\\
	\bangp_{x}{P} & := & \binpar{{x}!\langle{\binpar{D_{x}}{P}}\rangle}{D_{x}} \nonumber
\end{eqnarray}

\begin{eqnarray}
	\bangp_{x}{P} & & \nonumber\\
	=
	& {x}!\langle{(\prefix{x}{y}{(\outputp{x}{y} | @{y})) | P}}\rangle 
	      | \prefix{x}{y}{(\outputp{x}{y} | @{y})} & \nonumber\\
	\red
	& (\outputp{x}{y} | @{y})\substn{\quotep{(\prefix{x}{y}{(@{y} | \outputp{x}{y})) | P}}}{y} & \nonumber\\
	=
	& \outputp{x}{\quotep{(\prefix{x}{y}{(\outputp{x}{y} | @{y})) | P}}}
	  | {(\prefix{x}{y}{(\outputp{x}{y} | @{y})) | P}} & \nonumber\\
	\red
	& \ldots & \nonumber\\
	\red^*
	& P | P | \ldots & \nonumber
\end{eqnarray}

Of course, this encoding, as an implementation, runs away, unfolding
$\bangp{P}$ eagerly. A lazier and more implementable replication
operator, restricted to input-guarded processes, may be obtained as follows.

\begin{eqnarray}
\bangp{\prefix{u}{v}{P}} 
	:= 
	\binpar{\lift{x}{\prefix{u}{v}{(\binpar{D(x)}{P})}}}{D(x)} \nonumber
\end{eqnarray}

\begin{remark}
  Note that the lazier definition still does not deal with summation
  or mixed summation (i.e. sums over input and output). The reader is
  invited to construct definitions of replication that deal with these
  features. 

  Further, the definitions are parameterized in a name, $x$. Can you,
  gentle reader, make a definition that eliminates this parameter and
  guarantees no accidental interaction between the replication
  machinery and the process being replicated -- i.e. no accidental
  sharing of names used by the process to get its work done and the
  name(s) used by the replication to effect copying. This latter
  revision of the definition of replication is crucial to obtaining
  the expected identity $!!P \sim !P$.
\end{remark}

\begin{remark}\label{rem:paradoxical_combinator}
  The reader familiar with the lambda calculus will have noticed the
  similarity between $D$ and the paradoxical combinator.

  [Ed. note: the existence of this seems to suggest we have to be more
  restrictive on the set of processes and names we admit if we are to
  support no-cloning.]
\end{remark}

\subsubsection{Bisimulation}

The computational dynamics gives rise to another kind of equivalence,
the equivalence of computational behavior. As previously mentioned
this is typically captured \emph{via} some form of bisimulation.

% The notion we use in this paper is weak barbed bisimulation
% \cite{milner91polyadicpi}.

The notion we use in this paper is derived from weak barbed
bisimulation \cite{milner91polyadicpi}. 

\begin{definition}
An \emph{observation relation}, $\downarrow_{\mathcal N}$, over a set
of names, $\mathcal N$, is the smallest relation satisfying the rules
below.

\infrule[Out-barb]{y \in {\mathcal N}, \; x \nameeq y}
		  {\outputp{x}{v} \downarrow_{\mathcal N} x}
\infrule[Par-barb]{\mbox{$P\downarrow_{\mathcal N} x$ or $Q\downarrow_{\mathcal N} x$}}
		  {\binpar{P}{Q} \downarrow_{\mathcal N} x}

We write $P \Downarrow_{\mathcal N} x$ if there is $Q$ such that 
$P \wred Q$ and $Q \downarrow_{\mathcal N} x$.
\end{definition}

\begin{definition}
%\label{def.bbisim}
An  ${\mathcal N}$-\emph{barbed bisimulation} over a set of names, ${\mathcal N}$, is a symmetric binary relation 
${\mathcal S}_{\mathcal N}$ between agents such that $P\rel{S}_{\mathcal N}Q$ implies:
\begin{enumerate}
\item If $P \red P'$ then $Q \wred Q'$ and $P'\rel{S}_{\mathcal N} Q'$.
\item If $P\downarrow_{\mathcal N} x$, then $Q\Downarrow_{\mathcal N} x$.
\end{enumerate}
$P$ is ${\mathcal N}$-barbed bisimilar to $Q$, written
$P \wbbisim_{\mathcal N} Q$, if $P \rel{S}_{\mathcal N} Q$ for some ${\mathcal N}$-barbed bisimulation ${\mathcal S}_{\mathcal N}$.
\end{definition}

$\mathcal{R} \subseteq \pi \times \pi$

$P \mathcal{R} Q => \forall P'. P \red P' \Rightarrow \exists Q'. Q \red Q', P' \mathcal{R} Q'$

$P \vdash x \Rightarrow Q \vdash x$

\begin{mathpar}
  \inferrule*[lab=Out-barb]{x \nameeq y}{{y}!\langle{Q}\rangle \vdash x}
  \and
  \inferrule*[lab=Par-barb]{\mbox{$P\vdash x$ or $Q\vdash x$}}{\binpar{P}{Q} \vdash x}
\end{mathpar}

\subsubsection{Contexts}

One of the principle advantages of computational calculi like the
$\pi$-calculus is a well-defined notion of context,
contextual-equivalence and a correlation between
contextual-equivalence and notions of bisimulation. The notion of
context allows the decomposition of a process into (sub-)process and
its syntactic environment, its context. Thus, a context may be
thought of as a process with a ``hole'' (written $\Box$) in it. The
application of a context $M$ to a process $P$, written $M[P]$, is
tantamount to filling the hole in $M$ with $P$. In this paper we do
not need the full weight of this theory, but do make use of the notion
of context in the proof the main theorem. 

\begin{mathpar}
  \inferrule* [lab=summation] {} {{M_{M},M_{N}} \bc \Box \;|\; x.M_{A} \;|\; M_{M}+M_{N}}
  \and
  \inferrule* [lab=agent] {} {{M_{A}} \bc (\vec{x})M_{P} \;| \; \clift{P_0,\ldots,M_{P},\ldots,P_N}}
  \and \\
  \inferrule* [lab=process] {} {{M_{P}} \bc M_{N} \;| \;P|M_{P} }
\end{mathpar} 

\begin{mathpar}
  \inferrule* [lab=sychronization] {} {M_{N} \bc \Box \;|\; x?M_{F} \;|\; x!M_{C}}
  \and
  \inferrule* [lab=abstraction] {} {{M_{F}} \bc (x)M_{P} }
  \and
  \inferrule* [lab=concretion] {} {{M_{C}} \bc \langle M_{P} \rangle }
  \and \\
  \inferrule* [lab=process] {} {{M_{P}} \bc M_{N} \;| \;P|M_{P} }
\end{mathpar}

\begin{definition}[contextual application] Given a context $M$, and
  process $P$, we define the \emph{contextual application}, $M[P] :=
  M\{P/\Box\}$. That is, the contextual application of M to P is the
  substitution of $P$ for $\Box$ in $M$.
\end{definition}

$\meaningof{-} : L \to \mathcal{P}(\pi)$

\begin{mathpar}
  \inferrule* [lab=collection] {} {\meaningof{true} = \pi, \and \meaningof{~E} = \pi \setminus \meaningof{E}, \and \meaningof{E_{1} \& E_{2}} = \meaningof{E_{1}} \cap \meaningof{E_{2}}}
\end{mathpar}

\begin{mathpar}
  \inferrule* [lab=structure] {} {\meaningof{0} = \{ P \in \pi | P \equiv 0 \}, \and \\ \meaningof{E_1 | E_2} = \{ P \in \pi | P \equiv P_{1} | P_{2}, P_{1} \in \meaningof{E_{1}}, P_{2} \in \meaningof{E_2}\} }
\end{mathpar}

\begin{mathpar}
 \inferrule* [lab=behavior] {} {\meaningof{\langle a?b \rangle E} = \{ P \in \pi | P \equiv Q | u?(y)P', \\ \and \\\\ \and \\ \;\;\; u \in \meaningof{a}, \forall z.P'\{z/y\} \in \meaningof{E\{z/b\}}\}, \and \\ \meaningof{a!E} = \{ P \in \pi | P \equiv Q | x!\langle P' \rangle, x \in \meaningof{a} P' \in \meaningof{E}\} }
\end{mathpar}

\begin{mathpar}
 \inferrule* [lab=nominal] {} {\meaningof{\quotep{E}} = \{ \quotep{P} \in \quotep{\pi} | P \in \meaningof{E} \}, \and \meaningof{\quotep{P}} = \{ \quotep{Q} \in \quotep{\pi} | P \equiv Q \} \and \\ \meaningof{@\quotep{E}} = \{ P \in \pi | P \equiv @x, x \in \meaningof{E} \}}
\end{mathpar}

\begin{eqnarray*}
  \\
  \meaningof{-} : TS \to ST
\end{eqnarray*}

\begin{eqnarray*}
  \\
  L : TS \to ST
\end{eqnarray*}

\begin{eqnarray*}
  \\
  P \models E \iff P \in \meaningof{E}
\end{eqnarray*}

\begin{eqnarray*}
  P \approx_{L} Q \iff \forall E \in L. P \models E \iff Q \models E
\end{eqnarray*}

\begin{eqnarray*}
  P \approx_{K} Q
\end{eqnarray*}

\begin{eqnarray*}
  P \approx Q
\end{eqnarray*}

$\approx_{K} = \approx = \approx_{L}$

\subsubsection{Contextual duality}

Note that contexts extend the quotation operation to a family of
operations from processes to names. Given a context, $M$, we can
define a \emph{nominal context}, $\quotep{M}$ by $\quotep{M}[P] :=
\quotep{M[P]}$. To foreshadow what is to come we observe that these
operations enjoy a duality with processes very much like the duality
between vectors and maps from vectors to scalars.

Further, because the calculus is essentially higher-order, we have a
correspondence between contexts and processes. More specifically,
given a name $x$ and a context $M$ we can construct $M^{*}_{x}$ such
that 

\begin{mathpar}
  M^{*}_{x} | \lift{x}{P} \red M[P]
\end{mathpar}

namely,

\begin{mathpar}
  M^{*}_{x} := x?(u).M[\dropn{u}]
\end{mathpar}

The dependence of $M^{*}_{x}$ on a name makes it an abstraction, 

\begin{mathpar}
  M^{*} := (x)x?(u).M[\dropn{u}]
\end{mathpar}

\subsection{Additional notation}

It will sometimes be convenient to denote the process a name
quotes. We already have the notation $x = \quotep{P}$, but it will be
convenient to introduce an alternate notation, $\procn{x}$, when we
want to emphasize the connection to the use of the name. Note that, by
virtue of name equivalence, $\quotep{\procn{x}} \nameeq x$; so, the
notation is consistent with previous definitions.

Further, because names have structure it is possible to effect
substitutions on the basis of that structure. This means we need to
upgrade our notation for substitutions, which we accomplish by
adapting comprehension notation. Thus,

\begin{mathpar}
  P\{ y / x : x \in S \}
\end{mathpar}

is interpreted to mean the process derived from P by replacing (in a
capture-avoiding manner) each occurrence of $x$ in $S$ by $y$. For example,

\begin{mathpar}
  P\{ \quotep{\procn{x}|\procn{x}} / x : x \in \freenames{P} \}
\end{mathpar}

will replace each (occurrence) of a free name $x$ in $P$ by
$\quotep{\procn{x}|\procn{x}}$.

Also, we will avail ourselves of the notation $x^{L}$ and $x^{R}$ to
denote injections of a name into disjoint copies of the name
space. There are numerous ways to accomplish this. One example can be
found in \cite{MeredithR05}. This notation overloads to vectors of
names: $\vec{x}^{\pi} := (x_{i}^{\pi} \; : \; 0 \leq i < |\vec{x}| )$ where $\pi \in \{L,R\}$.

We also use $P^{\Box} := P|\Box$.

In \cite{MeredithR05} an interpretation of the new operator is
given. It turns out that there are several possible interpretations
all enjoying the requisite algebraic properties of the operator (see
\cite{milner91polyadicpi}). We will therefore make liberal use of
$(\nu\; \vec{x})P$.

% subsection the_syntax_and_semantics_of_the_notation_system (end)   

\input{qm2pi.qmops} 

\input{qm2pi.sterngerlach} 

\input{qm2pi.metric} 

% section concurrent_process_calculi (end)

%\input{qm2pi.proofsketch}

% section proof sketch (end)

%\input{qm2pi.slviaknots} 

% section spatial logic via knots (end)

\input{qm2pi.conclusion}

% section conclusion (end)

%\input{qm2pi.dtcodes} 

% section wiring algorithm (end)

\input{qm2pi.ack} 

% section acknowledgments (end)

\newpage


\bibliographystyle{plain}   
\bibliography{../../biblios/main.bib}

\input{qm2pi.rhodetails}

\end{document}



% section front matter (end)

\section{Introduction}\label{sec:introduction} % (fold)
In this draft of the material i am going to have to dispense with the
usual writing conventions adopted in papers on these topics. i'm going
to have adopt whatever tone i need at the time i'm writing up the
calculations. Sometimes this may be very conversational; others it may
be the barest mathematical grunts; others still it may be that i have
lifted text from one of my other papers because the exposition of some
point was better said there. i hope that my readers are not unduly put
out by this decision. i'm not doing this to flout convention or be
rebellious. i find these calculations very technically challenging. To
keep everything going technically, something has to give; i have to
let go of some cognitive burden. So, the academic writing style --
with all of its trade-offs in terms of facilitating technical
communication -- is what i'm letting go of. Perhaps subsequent drafts
can be tightened and polished, but for now, i'm going to speak as if
we were sitting together in a coffee shop with a laptop, wifi and a
pad of paper and a pencil.

So, here's what i have to say. We -- you and i, comfortably ensconced
in our coffee shop and well-equipped with our tools -- can realize and
carry out the calculations of quantum mechanics over a very different
formal theory of dynamics, a formal theory of dynamics that
corresponds to a theory of concurrent computation with
\emph{reflection}. It has the advantage that the underlying theory is
already `quantized', but supports analogues all of the continuuous
operations. Strikingly, this underlying theory has recently been
connected with a notion of metric that we can show, by calculating
together, coincides with the metric induced by the inner product.

There are a lot of reasons why you might be interested in seeing
calculations of this form. Here's why i'm interested. For the past
several centuries there has been no competitor to the ``Newtonian''
account of dynamics. As a result the predominant share of accounts of
dynamical systems and situations have had to be formulated in terms of
the Newtonian machinery. i view this as an intellectually dangerous
position to occupy. Everything, despite it's intrinsic shape, turns
into a nail to be hit with this hammer. Recently, however, the theory
of computation has matured to the point where we have candidates for
theories of dynamics that offer very different perspective on
reasoning about dynamical systems and situations. Testing these
candidates against very successful accounts of dynamical situations,
like quantum mechanics, is going to give us some sense of how mature
they are and some measure of the quality of these accounts of
dynamics.

\subsection{Summary of contributions and outline of paper}

So, we're going to develop an interpretation of the operations of
quantum mechanics normally interpreted by Hilbert spaces and
operators. We're going to do this over a theory of computation. Note
that this is very different than the usual quantum computation program
which develops notions of computation over quantum mechanics. Rather,
we are developing a story that aligns with Wheeler's slogan: It from
Bit. To do this we will first provide an account of the theory of
computation at play here. Then we will dive into a calculation-driven
interpretation of the operations of quantum mechanics.

The reason we take this approach is that -- until very recently --
there hasn't been an axiomatic account of quantum mechanics. As a
result there has been no sharp delineation of the mathematical theory
supporting interpretation of the physical theory and the physical
theory, itself. So, ambient features of the maths are free to be
exploited (or supressed) without a real accounting of their physical
relevance. There is no sharp statement ``here's the physical theory''
qua \emph{theory} and ``here's the mathematical interpretation''
enabling a judgment of how faithful the interpretation is -- apart
from experimental observation. When there is an axiomatic account we
can judge how well a given mathematical formalism supports an
interpretation of the axioms, independent of
experimentation. Likewise, we can judge how well we have captured our
physical evidence and experience with our axiomatics, independent of
any specific mathematical implementation, with accidental detail that
may or may not have physical significance. 

In lieu of a fully fleshed out and vetted axiomatic account of quantum
mechanics, interpreting the operational notions in service of modeling
physical systems will have to suffice. In other words, we are not in
the business of providing a model of Hilbert spaces and operators. We
are in the business of providing a model of quantum mechanics because
we are motivated by testing our notions of dynamics against physical
theory; and, the predictive calculations of the physical theory must
serve as the best formulation -- shy of a fully fleshed out axiomatic
account -- of the physical theory itself (as they have for scientific
theories since time immemorial). Put another way, despite a
whole-hearted commitment to an It-from-Bit ontology, we are firmly
aligned with the shut-up-and-calculate camp as the best way to obtain
results either from the physical perspective or as a quality assurance
measure of our fledgling theory of dynamics.

In detail, we present a reflective process calculus. Then we develop
intuitive correspondences between the notions available in this
calculus and the usual physical notions supporting quantum mechanical
calculations. Thus, 

\begin{table}[htp]
  \center{
    \fbox{
      \begin{tabular}{c|c}
        quantum mechanics & process calculus \\
        \hline
        scalar & name \\
        state vector & process \\
        dual & contextual duals \\
        matrix & formal sums of process-context-dual pairs \\
        orthogonality & process annihilation \\
        inner product & execution-formula + quoting
      \end{tabular}
    }
  }
  \caption{QM - process calculi correspondences}
\end{table}

Then we tighten up these intuitions to operational definitions. We
employ the Dirac notation as the best proxy we can find for an
abstract syntax of the quantum mechanical notions. The definitions we
develop put us in contact with equational constraints coming from the
theory that we demonstrate the definitions and calculations satisfy.

This puts us in a position to shut up and calculate for the
Stern-Gerlach experimental set up, showing how these predictive
calculations become calculations on processes in our theory of a
reflective process calculus.

Penultimately, we demonstrate that the notion of metric coming from
the inner product coincides with the notion of metric available from
the theory of bisimulation. This demonstration gives us the right to
think of space as arising from behavior. Finally, we consider where we
might go from the new vantage point we have obtained.

% section introduction (end) 
 
% section introduction (end)

% \documentclass[12pt]{llncs}
%\documentclass{jktr}

\usepackage[pdftex]{hyperref}                   
\usepackage {listings}
\usepackage {mathpartir}
\usepackage{bcprules}
%\usepackage{listings}
                       
\usepackage{graphicx} 
%\usepackage[margins=2.5cm,nohead,nofoot]{geometry}
%\usepackage{geometry}
\usepackage{amsfonts}
\usepackage{amstext}
\usepackage{latexsym}
\usepackage{amssymb}
\usepackage{color}


%\include{myPreamble}
\include{qm2pi.local} 

%\ifpdf
%\usepackage[pdftex]{graphicx}
%\else
%\usepackage{graphicx}
%\fi

 % \ifpdf
%  \usepackage{pdfsync}
%  \if


%\title{Brief Article}
%\author{David F. Snyder}
%\author{L.G. Meredith}

%\address{Dept. of Math., Texas State University--San Marcos, San Marcos, TX 78666}
       
\pagestyle{empty}


\begin{document}

\lstset{language=[Objective]Caml,frame=shadowbox}

\input{qm2pi.front}

% section front matter (end)

\input{qm2pi.intro} 
 
% section introduction (end)

% \input{qm2pi.knotations} 

% section notation (end)

\input{qm2pi.process.calculi} 

% section concurrent_process_calculi_and_spatial_logics_ (end)
    
%\input{qm2pi.knots2pi} 

%\input{qm2pi.trefoil} 

%\input{qm2pi.mainthm} 

% subsection basic_interpretation (end)

%\input{qm2pi.rho.presentation} 
\subsection{The syntax and semantics of the notation system}\label{sub:the_syntax_and_semantics_of_the_notation_system} % (fold)

We now summarize a technical presentation of the calculus that
embodies our theory of dynamics. The typical presentation of such a
calculus follows the style of giving generators and relations on
them. The grammar, below, describing term constructors, freely
generates the set of processes, $\Proc$. This set is then quotiented
by a relation known as structural congruence and it is over this set
that the notion of dynamics is expressed. This presentation is
essentially that of \cite{MeredithR05} with the addition of
polyadicity and summation. For readability we have relegated some of
the technical subtleties to an appendix.

\subsubsection{Process grammar}\label{subsub:process_grammar}

\begin{mathpar}
  \inferrule* [lab=synchronization] {} {{M} \bc \pzero \;|\; x?F \;|\; x!C }
  \and
  \inferrule* [lab=abstraction] {} {{F} \bc (x)P}
  \and
  \inferrule* [lab=concretion] {} {{C} \bc \langle Q \rangle}
  \and
  \inferrule* [lab=process] {} {{P,Q} \bc M \;| \;P|Q \;|\; @{x}}
  \and
  \inferrule* [lab=name] {} {{x} \bc \quotep{P}}
\end{mathpar} 

Note that $\vec{x}$ (resp. $\vec{P}$) denotes a vector of names
(resp. processes) of length $|\vec{x}|$ (resp. $|\vec{P}|$). We adopt
the following useful abbreviations.

\begin{mathpar}
   x?(\vec{y}).P := x.(\vec{y})P \and  x\clift{\vec{P}} := x.\clift{\vec{P}}
   \and x!(y) := \lift{x}{\dropn{y}}
   \and \Pi_{i=0}^{n-1}P_i := P_0 | \ldots | P_{n-1}
\end{mathpar}

\subsubsection{Structural congruence}

\paragraph{Free and bound names and alpha-equivalence.} At the
core of structural equivalence is alpha-equivalence which identifies
process that are the same up to a change of variable. Formally, we
recognize the distinction between free and bound names. The free names
of a process, $\freenames{P}$, may be calculated recursively as
follows:

\begin{mathpar}
\freenames{\pzero} := \emptyset
  \and \\
  \freenames{x?(y).P} := \{ x \} \cup (\freenames{P} \setminus \{ y \})
  \and 
  \freenames{x!\langle P \rangle} := \{ x \} \cup \{ P \} 
  \and \\
  \freenames{P|Q} := \freenames{P} \cup \freenames{Q}
  \and \\
  \freenames{@{x}} := \{ x \}
\end{mathpar}

$\pi$
$\quotep{\pi}$

$\freenames{-} : \pi \to \mathcal{P}(\quotep{\pi})$

\begin{eqnarray*}
  \freenames{\pzero} & := & \emptyset \\
  \freenames{x?(y).P} & := & \{ x \} \cup (\freenames{P} \setminus \{ y \}) \\
  \freenames{x!\langle P \rangle} & := & \{ x \} \cup \{ P \} \\
  \freenames{P|Q} & := & \freenames{P} \cup \freenames{Q} \\
  \freenames{\dropn{x}} & := & \{ x \}
\end{eqnarray*}

The bound names of a process, $\boundnames{P}$, are those names occurring in $P$
that are not free. For example, in $x?(y).0$, the name $x$ is free, while $y$ is bound.

\begin{mathpar}
  \inferrule* [lab=monoidal-laws] {} { P|Q \equiv Q|P \and P|0 \equiv P \and P|(Q|R) \equiv (P|Q)|R }
\end{mathpar}

\begin{mathpar}
  \inferrule* [lab=alpha-equivalence] {} { (x)P \equiv (y)P\{y/x\} \and y \not\in \freenames{P} }
\end{mathpar}

\begin{definition}
Then two processes, $P,Q$, are alpha-equivalent if $P = Q\{\vec{y}/\vec{x}\}$ for
some $\vec{x} \in \boundnames{Q},\vec{y} \in \boundnames{P}$, where $Q\{\vec{y}/\vec{x}\}$
denotes the capture-avoiding substitution of $\vec{y}$ for $\vec{x}$ in $Q$.
\end{definition}

\begin{definition}
  The {\em structural congruence} \cite{SangiorgiWalker} , $\equiv$,
  between processes is the least congruence containing
  alpha-equivalence, satisfying the abelian monoid laws
  (associativity, commutativity and $\pzero$ as identity) for parallel
  composition $|$ and for summation $+$.
\end{definition}

\subsection{Name equivalence}

We take name equivalence, written $\nameeq$, to be the smallest
equivalence relation generated by the following rules.

\begin{mathpar}
\inferrule*[lab=Quote-drop]
{ }
{ \quotep{@{x}} \nameeq x }

\inferrule*[lab=Struct-equiv]
{ P \scong Q }
{ \quotep{P} \nameeq \quotep{Q} }
\end{mathpar}

The astute reader will have noticed that the mutual recursion of names
and processes imposes a mutual recursion on alpha-equivalence and
structural equivalence via name-equivalence. Fortunately, all of this
works out pleasantly and we may calculate in the natural way, free of
concern. The reader interested in the details is referred to the
appendix \ref{appendix:rho_details}.

\subsection{Substitution}

We use $\Proc$ for the set of processes, $\QProc$ for the set of
names, and $\id{\{}\vec{y} / \vec{x} \id{\}}$ to denote partial maps,
$s : \QProc \rightarrow \QProc$. A map, $s$ lifts, uniquely, to a map
on process terms, $\widehat{s} : \Proc \rightarrow \Proc$ by the
following equations.

\begin{mathpar}
  (0) \psubstp{Q}{P} := 0 \\
  (R \juxtap S) \psubstp{Q}{P}
  :=    
  (R)\psubstp{Q}{P} \juxtap (S) \psubstp{Q}{P} \\
  (x?(y).R) \psubstp{Q}{P}    
  :=    
  (x)\substp{Q}{P} (z)\concat( (R \psubstn{z}{y}) \psubstp{Q}{P} ) \\
  (\lift{x}{R}) \psubstp{Q}{P}  
  :=
  \lift{(x)\substp{Q}{P}}{ R \psubstp{Q}{P} } \\
%   (\dropn{x})  \psubstp{Q}{P}       
%   := 
%   \left\{ 
%     \begin{array}{ccc} 
%       \dropn{\quotep{Q}} & & x \nameeq \quotep{P} \\
%       \dropn{x} & & otherwise \\
%     \end{array}
%   \right. 
  (\dropn{x})  \psubstp{Q}{P}       
  := 
  \left\{ 
    \begin{array}{ccc} 
      Q & & x \nameeq \quotep{P} \\
      \dropn{x} & & otherwise \\
    \end{array}
  \right.
\end{mathpar}
 

where

\begin{eqnarray}
  (x)\id{\{} \lpquote Q \rpquote / \lpquote P \rpquote \id{\}}            = 
  \left\{ 
    \begin{array}{ccc}
      \lpquote Q \rpquote & & x \nameeq \lpquote P \rpquote \\
      x & & otherwise \\
    \end{array}
  \right. \nonumber
\end{eqnarray}

and $z$ is chosen distinct from $\quotep{P}$, $\quotep{Q}$, the free
names in $Q$, and all the names in $R$. Our $\alpha$-equivalence will
be built in the standard way from this substitution.

\begin{remark}\label{rem:no_self_referential_names}
  One consequence of these definitions is that $\forall P. \quotep{P}
  \not\in \freenames{P}$.
\end{remark}

\subsection{ Dynamic quote: an example }

Anticipating something of what's to come, consider applying the
substitution, $\widehat{\id{\{}u / z \id{\}}}$, to the following pair
of processes, $\lift{w}{y!(z)}$ and $w[ \lpquote y!(z) \rpquote ]$.

\begin{eqnarray}
	\lift{w}{y!(z)}\widehat{\id{\{}u / z \id{\}}}
		& = &
		\lift{w}{y!(u)} \nonumber\\
	w[ \lpquote y!(z) \rpquote ] \widehat{ \id{\{}u / z \id{\}} }
		& = &
		w[ \lpquote y!(z) \rpquote ] \nonumber
\end{eqnarray}

Because the body of the process between quotes is impervious to
substitution, we get radically different answers. In fact, by
examining the first process in an input context,
e.g. $x?(z).\lift{w}{y!(z)}$, we see that the process under the lift
operator may be shaped by prefixed inputs binding a name inside it. In
this sense, the lift operator will be seen as a way to dynamically
construct processes before reifying them as names.

Finally equipped with these standard features we can present the
dynamics of the calculus.

\subsubsection{Operational semantics} 

Finally, we introduce the computational dynamics. What marks these
algebras as distinct from other more traditionally studied algebraic
structures, e.g. vector spaces or polynomial rings, is the manner in
which dynamics is captured. In traditional structures, dynamics is typically
expressed through morphisms between such structures, as in linear maps
between vector spaces or morphisms between rings. In algebras
associated with the semantics of computation, the dynamics is
expressed as part of the algebraic structure itself, through a
reduction reduction relation typically denoted by $\red$. Below, we
give a recursive presentation of this relation for the calculus used
in the encoding.

$\red \subseteq \pi \times \pi$
$\red : \pi \to \mathcal{P}(\pi)$

\begin{mathpar}
  \inferrule* [lab=Comm] { \textsf{match}( x_{src}, x_{trgt} ) } { x_{trgt}?(y)P \; | \; x_{src}!\langle {Q} \rangle \red P\{\quotep{Q}/y}\} }
  \and \\
  \inferrule* [lab=Par] {{P} \red {P}'} {{{P} | {Q}} \red {{P}' | {Q}}}
  \and
  \inferrule* [lab=Equiv]{{{P} \scong {P}'} \andalso {{P}' \red {Q}'} \andalso {{Q}' \scong {Q}}}{{P} \red {Q}}
\end{mathpar}

\begin{eqnarray*}
  match_{\equiv} (\quotep{P},\quotep{Q}) & := & P \equiv Q \\
  match_{\dagger}(\quotep{P},\quotep{Q}) & := & \forall R. P|Q \red^{*} R => R \red^{*} 0 \\
  match_{K}(\quotep{P},\quotep{Q}) & := & K \mbox{ for some context } K
\end{eqnarray*}

$u?(x)P | u!\langle Q \rangle \red P\{\quotep{Q}/x\}$

%We write $\wred$ for $\red^*$, and $P\red$ if $\exists Q $ such that $ P \red Q$.
We write $P\red$ if $\exists Q $ such that $ P \red Q$ and $P\not\red$, otherwise.

\section{Replication}

As mentioned before, it is known that replication (and hence
recursion) can be implemented in a higher-order process algebra
\cite{SangiorgiWalker}. As our first example of calculation with the
machinery thus far presented we give the construction explicitly in
the {\rhoc}.

\begin{eqnarray}
	D_{x} & := & \prefix{x}{y}{(\binpar{\outputp{x}{y}}{@{y}})} \nonumber\\
	\bangp_{x}{P} & := & \binpar{{x}!\langle{\binpar{D_{x}}{P}}\rangle}{D_{x}} \nonumber
\end{eqnarray}

\begin{eqnarray}
	\bangp_{x}{P} & & \nonumber\\
	=
	& {x}!\langle{(\prefix{x}{y}{(\outputp{x}{y} | @{y})) | P}}\rangle 
	      | \prefix{x}{y}{(\outputp{x}{y} | @{y})} & \nonumber\\
	\red
	& (\outputp{x}{y} | @{y})\substn{\quotep{(\prefix{x}{y}{(@{y} | \outputp{x}{y})) | P}}}{y} & \nonumber\\
	=
	& \outputp{x}{\quotep{(\prefix{x}{y}{(\outputp{x}{y} | @{y})) | P}}}
	  | {(\prefix{x}{y}{(\outputp{x}{y} | @{y})) | P}} & \nonumber\\
	\red
	& \ldots & \nonumber\\
	\red^*
	& P | P | \ldots & \nonumber
\end{eqnarray}

Of course, this encoding, as an implementation, runs away, unfolding
$\bangp{P}$ eagerly. A lazier and more implementable replication
operator, restricted to input-guarded processes, may be obtained as follows.

\begin{eqnarray}
\bangp{\prefix{u}{v}{P}} 
	:= 
	\binpar{\lift{x}{\prefix{u}{v}{(\binpar{D(x)}{P})}}}{D(x)} \nonumber
\end{eqnarray}

\begin{remark}
  Note that the lazier definition still does not deal with summation
  or mixed summation (i.e. sums over input and output). The reader is
  invited to construct definitions of replication that deal with these
  features. 

  Further, the definitions are parameterized in a name, $x$. Can you,
  gentle reader, make a definition that eliminates this parameter and
  guarantees no accidental interaction between the replication
  machinery and the process being replicated -- i.e. no accidental
  sharing of names used by the process to get its work done and the
  name(s) used by the replication to effect copying. This latter
  revision of the definition of replication is crucial to obtaining
  the expected identity $!!P \sim !P$.
\end{remark}

\begin{remark}\label{rem:paradoxical_combinator}
  The reader familiar with the lambda calculus will have noticed the
  similarity between $D$ and the paradoxical combinator.

  [Ed. note: the existence of this seems to suggest we have to be more
  restrictive on the set of processes and names we admit if we are to
  support no-cloning.]
\end{remark}

\subsubsection{Bisimulation}

The computational dynamics gives rise to another kind of equivalence,
the equivalence of computational behavior. As previously mentioned
this is typically captured \emph{via} some form of bisimulation.

% The notion we use in this paper is weak barbed bisimulation
% \cite{milner91polyadicpi}.

The notion we use in this paper is derived from weak barbed
bisimulation \cite{milner91polyadicpi}. 

\begin{definition}
An \emph{observation relation}, $\downarrow_{\mathcal N}$, over a set
of names, $\mathcal N$, is the smallest relation satisfying the rules
below.

\infrule[Out-barb]{y \in {\mathcal N}, \; x \nameeq y}
		  {\outputp{x}{v} \downarrow_{\mathcal N} x}
\infrule[Par-barb]{\mbox{$P\downarrow_{\mathcal N} x$ or $Q\downarrow_{\mathcal N} x$}}
		  {\binpar{P}{Q} \downarrow_{\mathcal N} x}

We write $P \Downarrow_{\mathcal N} x$ if there is $Q$ such that 
$P \wred Q$ and $Q \downarrow_{\mathcal N} x$.
\end{definition}

\begin{definition}
%\label{def.bbisim}
An  ${\mathcal N}$-\emph{barbed bisimulation} over a set of names, ${\mathcal N}$, is a symmetric binary relation 
${\mathcal S}_{\mathcal N}$ between agents such that $P\rel{S}_{\mathcal N}Q$ implies:
\begin{enumerate}
\item If $P \red P'$ then $Q \wred Q'$ and $P'\rel{S}_{\mathcal N} Q'$.
\item If $P\downarrow_{\mathcal N} x$, then $Q\Downarrow_{\mathcal N} x$.
\end{enumerate}
$P$ is ${\mathcal N}$-barbed bisimilar to $Q$, written
$P \wbbisim_{\mathcal N} Q$, if $P \rel{S}_{\mathcal N} Q$ for some ${\mathcal N}$-barbed bisimulation ${\mathcal S}_{\mathcal N}$.
\end{definition}

$\mathcal{R} \subseteq \pi \times \pi$

$P \mathcal{R} Q => \forall P'. P \red P' \Rightarrow \exists Q'. Q \red Q', P' \mathcal{R} Q'$

$P \vdash x \Rightarrow Q \vdash x$

\begin{mathpar}
  \inferrule*[lab=Out-barb]{x \nameeq y}{{y}!\langle{Q}\rangle \vdash x}
  \and
  \inferrule*[lab=Par-barb]{\mbox{$P\vdash x$ or $Q\vdash x$}}{\binpar{P}{Q} \vdash x}
\end{mathpar}

\subsubsection{Contexts}

One of the principle advantages of computational calculi like the
$\pi$-calculus is a well-defined notion of context,
contextual-equivalence and a correlation between
contextual-equivalence and notions of bisimulation. The notion of
context allows the decomposition of a process into (sub-)process and
its syntactic environment, its context. Thus, a context may be
thought of as a process with a ``hole'' (written $\Box$) in it. The
application of a context $M$ to a process $P$, written $M[P]$, is
tantamount to filling the hole in $M$ with $P$. In this paper we do
not need the full weight of this theory, but do make use of the notion
of context in the proof the main theorem. 

\begin{mathpar}
  \inferrule* [lab=summation] {} {{M_{M},M_{N}} \bc \Box \;|\; x.M_{A} \;|\; M_{M}+M_{N}}
  \and
  \inferrule* [lab=agent] {} {{M_{A}} \bc (\vec{x})M_{P} \;| \; \clift{P_0,\ldots,M_{P},\ldots,P_N}}
  \and \\
  \inferrule* [lab=process] {} {{M_{P}} \bc M_{N} \;| \;P|M_{P} }
\end{mathpar} 

\begin{mathpar}
  \inferrule* [lab=sychronization] {} {M_{N} \bc \Box \;|\; x?M_{F} \;|\; x!M_{C}}
  \and
  \inferrule* [lab=abstraction] {} {{M_{F}} \bc (x)M_{P} }
  \and
  \inferrule* [lab=concretion] {} {{M_{C}} \bc \langle M_{P} \rangle }
  \and \\
  \inferrule* [lab=process] {} {{M_{P}} \bc M_{N} \;| \;P|M_{P} }
\end{mathpar}

\begin{definition}[contextual application] Given a context $M$, and
  process $P$, we define the \emph{contextual application}, $M[P] :=
  M\{P/\Box\}$. That is, the contextual application of M to P is the
  substitution of $P$ for $\Box$ in $M$.
\end{definition}

$\meaningof{-} : L \to \mathcal{P}(\pi)$

\begin{mathpar}
  \inferrule* [lab=collection] {} {\meaningof{true} = \pi, \and \meaningof{~E} = \pi \setminus \meaningof{E}, \and \meaningof{E_{1} \& E_{2}} = \meaningof{E_{1}} \cap \meaningof{E_{2}}}
\end{mathpar}

\begin{mathpar}
  \inferrule* [lab=structure] {} {\meaningof{0} = \{ P \in \pi | P \equiv 0 \}, \and \\ \meaningof{E_1 | E_2} = \{ P \in \pi | P \equiv P_{1} | P_{2}, P_{1} \in \meaningof{E_{1}}, P_{2} \in \meaningof{E_2}\} }
\end{mathpar}

\begin{mathpar}
 \inferrule* [lab=behavior] {} {\meaningof{\langle a?b \rangle E} = \{ P \in \pi | P \equiv Q | u?(y)P', \\ \and \\\\ \and \\ \;\;\; u \in \meaningof{a}, \forall z.P'\{z/y\} \in \meaningof{E\{z/b\}}\}, \and \\ \meaningof{a!E} = \{ P \in \pi | P \equiv Q | x!\langle P' \rangle, x \in \meaningof{a} P' \in \meaningof{E}\} }
\end{mathpar}

\begin{mathpar}
 \inferrule* [lab=nominal] {} {\meaningof{\quotep{E}} = \{ \quotep{P} \in \quotep{\pi} | P \in \meaningof{E} \}, \and \meaningof{\quotep{P}} = \{ \quotep{Q} \in \quotep{\pi} | P \equiv Q \} \and \\ \meaningof{@\quotep{E}} = \{ P \in \pi | P \equiv @x, x \in \meaningof{E} \}}
\end{mathpar}

\begin{eqnarray*}
  \\
  \meaningof{-} : TS \to ST
\end{eqnarray*}

\begin{eqnarray*}
  \\
  L : TS \to ST
\end{eqnarray*}

\begin{eqnarray*}
  \\
  P \models E \iff P \in \meaningof{E}
\end{eqnarray*}

\begin{eqnarray*}
  P \approx_{L} Q \iff \forall E \in L. P \models E \iff Q \models E
\end{eqnarray*}

\begin{eqnarray*}
  P \approx_{K} Q
\end{eqnarray*}

\begin{eqnarray*}
  P \approx Q
\end{eqnarray*}

$\approx_{K} = \approx = \approx_{L}$

\subsubsection{Contextual duality}

Note that contexts extend the quotation operation to a family of
operations from processes to names. Given a context, $M$, we can
define a \emph{nominal context}, $\quotep{M}$ by $\quotep{M}[P] :=
\quotep{M[P]}$. To foreshadow what is to come we observe that these
operations enjoy a duality with processes very much like the duality
between vectors and maps from vectors to scalars.

Further, because the calculus is essentially higher-order, we have a
correspondence between contexts and processes. More specifically,
given a name $x$ and a context $M$ we can construct $M^{*}_{x}$ such
that 

\begin{mathpar}
  M^{*}_{x} | \lift{x}{P} \red M[P]
\end{mathpar}

namely,

\begin{mathpar}
  M^{*}_{x} := x?(u).M[\dropn{u}]
\end{mathpar}

The dependence of $M^{*}_{x}$ on a name makes it an abstraction, 

\begin{mathpar}
  M^{*} := (x)x?(u).M[\dropn{u}]
\end{mathpar}

\subsection{Additional notation}

It will sometimes be convenient to denote the process a name
quotes. We already have the notation $x = \quotep{P}$, but it will be
convenient to introduce an alternate notation, $\procn{x}$, when we
want to emphasize the connection to the use of the name. Note that, by
virtue of name equivalence, $\quotep{\procn{x}} \nameeq x$; so, the
notation is consistent with previous definitions.

Further, because names have structure it is possible to effect
substitutions on the basis of that structure. This means we need to
upgrade our notation for substitutions, which we accomplish by
adapting comprehension notation. Thus,

\begin{mathpar}
  P\{ y / x : x \in S \}
\end{mathpar}

is interpreted to mean the process derived from P by replacing (in a
capture-avoiding manner) each occurrence of $x$ in $S$ by $y$. For example,

\begin{mathpar}
  P\{ \quotep{\procn{x}|\procn{x}} / x : x \in \freenames{P} \}
\end{mathpar}

will replace each (occurrence) of a free name $x$ in $P$ by
$\quotep{\procn{x}|\procn{x}}$.

Also, we will avail ourselves of the notation $x^{L}$ and $x^{R}$ to
denote injections of a name into disjoint copies of the name
space. There are numerous ways to accomplish this. One example can be
found in \cite{MeredithR05}. This notation overloads to vectors of
names: $\vec{x}^{\pi} := (x_{i}^{\pi} \; : \; 0 \leq i < |\vec{x}| )$ where $\pi \in \{L,R\}$.

We also use $P^{\Box} := P|\Box$.

In \cite{MeredithR05} an interpretation of the new operator is
given. It turns out that there are several possible interpretations
all enjoying the requisite algebraic properties of the operator (see
\cite{milner91polyadicpi}). We will therefore make liberal use of
$(\nu\; \vec{x})P$.

% subsection the_syntax_and_semantics_of_the_notation_system (end)   

\input{qm2pi.qmops} 

\input{qm2pi.sterngerlach} 

\input{qm2pi.metric} 

% section concurrent_process_calculi (end)

%\input{qm2pi.proofsketch}

% section proof sketch (end)

%\input{qm2pi.slviaknots} 

% section spatial logic via knots (end)

\input{qm2pi.conclusion}

% section conclusion (end)

%\input{qm2pi.dtcodes} 

% section wiring algorithm (end)

\input{qm2pi.ack} 

% section acknowledgments (end)

\newpage


\bibliographystyle{plain}   
\bibliography{../../biblios/main.bib}

\input{qm2pi.rhodetails}

\end{document}

 

% section notation (end)

\input{qm2pi.process.calculi} 

% section concurrent_process_calculi_and_spatial_logics_ (end)
    
%\documentclass[12pt]{llncs}
%\documentclass{jktr}

\usepackage[pdftex]{hyperref}                   
\usepackage {listings}
\usepackage {mathpartir}
\usepackage{bcprules}
%\usepackage{listings}
                       
\usepackage{graphicx} 
%\usepackage[margins=2.5cm,nohead,nofoot]{geometry}
%\usepackage{geometry}
\usepackage{amsfonts}
\usepackage{amstext}
\usepackage{latexsym}
\usepackage{amssymb}
\usepackage{color}


%\include{myPreamble}
\include{qm2pi.local} 

%\ifpdf
%\usepackage[pdftex]{graphicx}
%\else
%\usepackage{graphicx}
%\fi

 % \ifpdf
%  \usepackage{pdfsync}
%  \if


%\title{Brief Article}
%\author{David F. Snyder}
%\author{L.G. Meredith}

%\address{Dept. of Math., Texas State University--San Marcos, San Marcos, TX 78666}
       
\pagestyle{empty}


\begin{document}

\lstset{language=[Objective]Caml,frame=shadowbox}

\input{qm2pi.front}

% section front matter (end)

\input{qm2pi.intro} 
 
% section introduction (end)

% \input{qm2pi.knotations} 

% section notation (end)

\input{qm2pi.process.calculi} 

% section concurrent_process_calculi_and_spatial_logics_ (end)
    
%\input{qm2pi.knots2pi} 

%\input{qm2pi.trefoil} 

%\input{qm2pi.mainthm} 

% subsection basic_interpretation (end)

%\input{qm2pi.rho.presentation} 
\subsection{The syntax and semantics of the notation system}\label{sub:the_syntax_and_semantics_of_the_notation_system} % (fold)

We now summarize a technical presentation of the calculus that
embodies our theory of dynamics. The typical presentation of such a
calculus follows the style of giving generators and relations on
them. The grammar, below, describing term constructors, freely
generates the set of processes, $\Proc$. This set is then quotiented
by a relation known as structural congruence and it is over this set
that the notion of dynamics is expressed. This presentation is
essentially that of \cite{MeredithR05} with the addition of
polyadicity and summation. For readability we have relegated some of
the technical subtleties to an appendix.

\subsubsection{Process grammar}\label{subsub:process_grammar}

\begin{mathpar}
  \inferrule* [lab=synchronization] {} {{M} \bc \pzero \;|\; x?F \;|\; x!C }
  \and
  \inferrule* [lab=abstraction] {} {{F} \bc (x)P}
  \and
  \inferrule* [lab=concretion] {} {{C} \bc \langle Q \rangle}
  \and
  \inferrule* [lab=process] {} {{P,Q} \bc M \;| \;P|Q \;|\; @{x}}
  \and
  \inferrule* [lab=name] {} {{x} \bc \quotep{P}}
\end{mathpar} 

Note that $\vec{x}$ (resp. $\vec{P}$) denotes a vector of names
(resp. processes) of length $|\vec{x}|$ (resp. $|\vec{P}|$). We adopt
the following useful abbreviations.

\begin{mathpar}
   x?(\vec{y}).P := x.(\vec{y})P \and  x\clift{\vec{P}} := x.\clift{\vec{P}}
   \and x!(y) := \lift{x}{\dropn{y}}
   \and \Pi_{i=0}^{n-1}P_i := P_0 | \ldots | P_{n-1}
\end{mathpar}

\subsubsection{Structural congruence}

\paragraph{Free and bound names and alpha-equivalence.} At the
core of structural equivalence is alpha-equivalence which identifies
process that are the same up to a change of variable. Formally, we
recognize the distinction between free and bound names. The free names
of a process, $\freenames{P}$, may be calculated recursively as
follows:

\begin{mathpar}
\freenames{\pzero} := \emptyset
  \and \\
  \freenames{x?(y).P} := \{ x \} \cup (\freenames{P} \setminus \{ y \})
  \and 
  \freenames{x!\langle P \rangle} := \{ x \} \cup \{ P \} 
  \and \\
  \freenames{P|Q} := \freenames{P} \cup \freenames{Q}
  \and \\
  \freenames{@{x}} := \{ x \}
\end{mathpar}

$\pi$
$\quotep{\pi}$

$\freenames{-} : \pi \to \mathcal{P}(\quotep{\pi})$

\begin{eqnarray*}
  \freenames{\pzero} & := & \emptyset \\
  \freenames{x?(y).P} & := & \{ x \} \cup (\freenames{P} \setminus \{ y \}) \\
  \freenames{x!\langle P \rangle} & := & \{ x \} \cup \{ P \} \\
  \freenames{P|Q} & := & \freenames{P} \cup \freenames{Q} \\
  \freenames{\dropn{x}} & := & \{ x \}
\end{eqnarray*}

The bound names of a process, $\boundnames{P}$, are those names occurring in $P$
that are not free. For example, in $x?(y).0$, the name $x$ is free, while $y$ is bound.

\begin{mathpar}
  \inferrule* [lab=monoidal-laws] {} { P|Q \equiv Q|P \and P|0 \equiv P \and P|(Q|R) \equiv (P|Q)|R }
\end{mathpar}

\begin{mathpar}
  \inferrule* [lab=alpha-equivalence] {} { (x)P \equiv (y)P\{y/x\} \and y \not\in \freenames{P} }
\end{mathpar}

\begin{definition}
Then two processes, $P,Q$, are alpha-equivalent if $P = Q\{\vec{y}/\vec{x}\}$ for
some $\vec{x} \in \boundnames{Q},\vec{y} \in \boundnames{P}$, where $Q\{\vec{y}/\vec{x}\}$
denotes the capture-avoiding substitution of $\vec{y}$ for $\vec{x}$ in $Q$.
\end{definition}

\begin{definition}
  The {\em structural congruence} \cite{SangiorgiWalker} , $\equiv$,
  between processes is the least congruence containing
  alpha-equivalence, satisfying the abelian monoid laws
  (associativity, commutativity and $\pzero$ as identity) for parallel
  composition $|$ and for summation $+$.
\end{definition}

\subsection{Name equivalence}

We take name equivalence, written $\nameeq$, to be the smallest
equivalence relation generated by the following rules.

\begin{mathpar}
\inferrule*[lab=Quote-drop]
{ }
{ \quotep{@{x}} \nameeq x }

\inferrule*[lab=Struct-equiv]
{ P \scong Q }
{ \quotep{P} \nameeq \quotep{Q} }
\end{mathpar}

The astute reader will have noticed that the mutual recursion of names
and processes imposes a mutual recursion on alpha-equivalence and
structural equivalence via name-equivalence. Fortunately, all of this
works out pleasantly and we may calculate in the natural way, free of
concern. The reader interested in the details is referred to the
appendix \ref{appendix:rho_details}.

\subsection{Substitution}

We use $\Proc$ for the set of processes, $\QProc$ for the set of
names, and $\id{\{}\vec{y} / \vec{x} \id{\}}$ to denote partial maps,
$s : \QProc \rightarrow \QProc$. A map, $s$ lifts, uniquely, to a map
on process terms, $\widehat{s} : \Proc \rightarrow \Proc$ by the
following equations.

\begin{mathpar}
  (0) \psubstp{Q}{P} := 0 \\
  (R \juxtap S) \psubstp{Q}{P}
  :=    
  (R)\psubstp{Q}{P} \juxtap (S) \psubstp{Q}{P} \\
  (x?(y).R) \psubstp{Q}{P}    
  :=    
  (x)\substp{Q}{P} (z)\concat( (R \psubstn{z}{y}) \psubstp{Q}{P} ) \\
  (\lift{x}{R}) \psubstp{Q}{P}  
  :=
  \lift{(x)\substp{Q}{P}}{ R \psubstp{Q}{P} } \\
%   (\dropn{x})  \psubstp{Q}{P}       
%   := 
%   \left\{ 
%     \begin{array}{ccc} 
%       \dropn{\quotep{Q}} & & x \nameeq \quotep{P} \\
%       \dropn{x} & & otherwise \\
%     \end{array}
%   \right. 
  (\dropn{x})  \psubstp{Q}{P}       
  := 
  \left\{ 
    \begin{array}{ccc} 
      Q & & x \nameeq \quotep{P} \\
      \dropn{x} & & otherwise \\
    \end{array}
  \right.
\end{mathpar}
 

where

\begin{eqnarray}
  (x)\id{\{} \lpquote Q \rpquote / \lpquote P \rpquote \id{\}}            = 
  \left\{ 
    \begin{array}{ccc}
      \lpquote Q \rpquote & & x \nameeq \lpquote P \rpquote \\
      x & & otherwise \\
    \end{array}
  \right. \nonumber
\end{eqnarray}

and $z$ is chosen distinct from $\quotep{P}$, $\quotep{Q}$, the free
names in $Q$, and all the names in $R$. Our $\alpha$-equivalence will
be built in the standard way from this substitution.

\begin{remark}\label{rem:no_self_referential_names}
  One consequence of these definitions is that $\forall P. \quotep{P}
  \not\in \freenames{P}$.
\end{remark}

\subsection{ Dynamic quote: an example }

Anticipating something of what's to come, consider applying the
substitution, $\widehat{\id{\{}u / z \id{\}}}$, to the following pair
of processes, $\lift{w}{y!(z)}$ and $w[ \lpquote y!(z) \rpquote ]$.

\begin{eqnarray}
	\lift{w}{y!(z)}\widehat{\id{\{}u / z \id{\}}}
		& = &
		\lift{w}{y!(u)} \nonumber\\
	w[ \lpquote y!(z) \rpquote ] \widehat{ \id{\{}u / z \id{\}} }
		& = &
		w[ \lpquote y!(z) \rpquote ] \nonumber
\end{eqnarray}

Because the body of the process between quotes is impervious to
substitution, we get radically different answers. In fact, by
examining the first process in an input context,
e.g. $x?(z).\lift{w}{y!(z)}$, we see that the process under the lift
operator may be shaped by prefixed inputs binding a name inside it. In
this sense, the lift operator will be seen as a way to dynamically
construct processes before reifying them as names.

Finally equipped with these standard features we can present the
dynamics of the calculus.

\subsubsection{Operational semantics} 

Finally, we introduce the computational dynamics. What marks these
algebras as distinct from other more traditionally studied algebraic
structures, e.g. vector spaces or polynomial rings, is the manner in
which dynamics is captured. In traditional structures, dynamics is typically
expressed through morphisms between such structures, as in linear maps
between vector spaces or morphisms between rings. In algebras
associated with the semantics of computation, the dynamics is
expressed as part of the algebraic structure itself, through a
reduction reduction relation typically denoted by $\red$. Below, we
give a recursive presentation of this relation for the calculus used
in the encoding.

$\red \subseteq \pi \times \pi$
$\red : \pi \to \mathcal{P}(\pi)$

\begin{mathpar}
  \inferrule* [lab=Comm] { \textsf{match}( x_{src}, x_{trgt} ) } { x_{trgt}?(y)P \; | \; x_{src}!\langle {Q} \rangle \red P\{\quotep{Q}/y}\} }
  \and \\
  \inferrule* [lab=Par] {{P} \red {P}'} {{{P} | {Q}} \red {{P}' | {Q}}}
  \and
  \inferrule* [lab=Equiv]{{{P} \scong {P}'} \andalso {{P}' \red {Q}'} \andalso {{Q}' \scong {Q}}}{{P} \red {Q}}
\end{mathpar}

\begin{eqnarray*}
  match_{\equiv} (\quotep{P},\quotep{Q}) & := & P \equiv Q \\
  match_{\dagger}(\quotep{P},\quotep{Q}) & := & \forall R. P|Q \red^{*} R => R \red^{*} 0 \\
  match_{K}(\quotep{P},\quotep{Q}) & := & K \mbox{ for some context } K
\end{eqnarray*}

$u?(x)P | u!\langle Q \rangle \red P\{\quotep{Q}/x\}$

%We write $\wred$ for $\red^*$, and $P\red$ if $\exists Q $ such that $ P \red Q$.
We write $P\red$ if $\exists Q $ such that $ P \red Q$ and $P\not\red$, otherwise.

\section{Replication}

As mentioned before, it is known that replication (and hence
recursion) can be implemented in a higher-order process algebra
\cite{SangiorgiWalker}. As our first example of calculation with the
machinery thus far presented we give the construction explicitly in
the {\rhoc}.

\begin{eqnarray}
	D_{x} & := & \prefix{x}{y}{(\binpar{\outputp{x}{y}}{@{y}})} \nonumber\\
	\bangp_{x}{P} & := & \binpar{{x}!\langle{\binpar{D_{x}}{P}}\rangle}{D_{x}} \nonumber
\end{eqnarray}

\begin{eqnarray}
	\bangp_{x}{P} & & \nonumber\\
	=
	& {x}!\langle{(\prefix{x}{y}{(\outputp{x}{y} | @{y})) | P}}\rangle 
	      | \prefix{x}{y}{(\outputp{x}{y} | @{y})} & \nonumber\\
	\red
	& (\outputp{x}{y} | @{y})\substn{\quotep{(\prefix{x}{y}{(@{y} | \outputp{x}{y})) | P}}}{y} & \nonumber\\
	=
	& \outputp{x}{\quotep{(\prefix{x}{y}{(\outputp{x}{y} | @{y})) | P}}}
	  | {(\prefix{x}{y}{(\outputp{x}{y} | @{y})) | P}} & \nonumber\\
	\red
	& \ldots & \nonumber\\
	\red^*
	& P | P | \ldots & \nonumber
\end{eqnarray}

Of course, this encoding, as an implementation, runs away, unfolding
$\bangp{P}$ eagerly. A lazier and more implementable replication
operator, restricted to input-guarded processes, may be obtained as follows.

\begin{eqnarray}
\bangp{\prefix{u}{v}{P}} 
	:= 
	\binpar{\lift{x}{\prefix{u}{v}{(\binpar{D(x)}{P})}}}{D(x)} \nonumber
\end{eqnarray}

\begin{remark}
  Note that the lazier definition still does not deal with summation
  or mixed summation (i.e. sums over input and output). The reader is
  invited to construct definitions of replication that deal with these
  features. 

  Further, the definitions are parameterized in a name, $x$. Can you,
  gentle reader, make a definition that eliminates this parameter and
  guarantees no accidental interaction between the replication
  machinery and the process being replicated -- i.e. no accidental
  sharing of names used by the process to get its work done and the
  name(s) used by the replication to effect copying. This latter
  revision of the definition of replication is crucial to obtaining
  the expected identity $!!P \sim !P$.
\end{remark}

\begin{remark}\label{rem:paradoxical_combinator}
  The reader familiar with the lambda calculus will have noticed the
  similarity between $D$ and the paradoxical combinator.

  [Ed. note: the existence of this seems to suggest we have to be more
  restrictive on the set of processes and names we admit if we are to
  support no-cloning.]
\end{remark}

\subsubsection{Bisimulation}

The computational dynamics gives rise to another kind of equivalence,
the equivalence of computational behavior. As previously mentioned
this is typically captured \emph{via} some form of bisimulation.

% The notion we use in this paper is weak barbed bisimulation
% \cite{milner91polyadicpi}.

The notion we use in this paper is derived from weak barbed
bisimulation \cite{milner91polyadicpi}. 

\begin{definition}
An \emph{observation relation}, $\downarrow_{\mathcal N}$, over a set
of names, $\mathcal N$, is the smallest relation satisfying the rules
below.

\infrule[Out-barb]{y \in {\mathcal N}, \; x \nameeq y}
		  {\outputp{x}{v} \downarrow_{\mathcal N} x}
\infrule[Par-barb]{\mbox{$P\downarrow_{\mathcal N} x$ or $Q\downarrow_{\mathcal N} x$}}
		  {\binpar{P}{Q} \downarrow_{\mathcal N} x}

We write $P \Downarrow_{\mathcal N} x$ if there is $Q$ such that 
$P \wred Q$ and $Q \downarrow_{\mathcal N} x$.
\end{definition}

\begin{definition}
%\label{def.bbisim}
An  ${\mathcal N}$-\emph{barbed bisimulation} over a set of names, ${\mathcal N}$, is a symmetric binary relation 
${\mathcal S}_{\mathcal N}$ between agents such that $P\rel{S}_{\mathcal N}Q$ implies:
\begin{enumerate}
\item If $P \red P'$ then $Q \wred Q'$ and $P'\rel{S}_{\mathcal N} Q'$.
\item If $P\downarrow_{\mathcal N} x$, then $Q\Downarrow_{\mathcal N} x$.
\end{enumerate}
$P$ is ${\mathcal N}$-barbed bisimilar to $Q$, written
$P \wbbisim_{\mathcal N} Q$, if $P \rel{S}_{\mathcal N} Q$ for some ${\mathcal N}$-barbed bisimulation ${\mathcal S}_{\mathcal N}$.
\end{definition}

$\mathcal{R} \subseteq \pi \times \pi$

$P \mathcal{R} Q => \forall P'. P \red P' \Rightarrow \exists Q'. Q \red Q', P' \mathcal{R} Q'$

$P \vdash x \Rightarrow Q \vdash x$

\begin{mathpar}
  \inferrule*[lab=Out-barb]{x \nameeq y}{{y}!\langle{Q}\rangle \vdash x}
  \and
  \inferrule*[lab=Par-barb]{\mbox{$P\vdash x$ or $Q\vdash x$}}{\binpar{P}{Q} \vdash x}
\end{mathpar}

\subsubsection{Contexts}

One of the principle advantages of computational calculi like the
$\pi$-calculus is a well-defined notion of context,
contextual-equivalence and a correlation between
contextual-equivalence and notions of bisimulation. The notion of
context allows the decomposition of a process into (sub-)process and
its syntactic environment, its context. Thus, a context may be
thought of as a process with a ``hole'' (written $\Box$) in it. The
application of a context $M$ to a process $P$, written $M[P]$, is
tantamount to filling the hole in $M$ with $P$. In this paper we do
not need the full weight of this theory, but do make use of the notion
of context in the proof the main theorem. 

\begin{mathpar}
  \inferrule* [lab=summation] {} {{M_{M},M_{N}} \bc \Box \;|\; x.M_{A} \;|\; M_{M}+M_{N}}
  \and
  \inferrule* [lab=agent] {} {{M_{A}} \bc (\vec{x})M_{P} \;| \; \clift{P_0,\ldots,M_{P},\ldots,P_N}}
  \and \\
  \inferrule* [lab=process] {} {{M_{P}} \bc M_{N} \;| \;P|M_{P} }
\end{mathpar} 

\begin{mathpar}
  \inferrule* [lab=sychronization] {} {M_{N} \bc \Box \;|\; x?M_{F} \;|\; x!M_{C}}
  \and
  \inferrule* [lab=abstraction] {} {{M_{F}} \bc (x)M_{P} }
  \and
  \inferrule* [lab=concretion] {} {{M_{C}} \bc \langle M_{P} \rangle }
  \and \\
  \inferrule* [lab=process] {} {{M_{P}} \bc M_{N} \;| \;P|M_{P} }
\end{mathpar}

\begin{definition}[contextual application] Given a context $M$, and
  process $P$, we define the \emph{contextual application}, $M[P] :=
  M\{P/\Box\}$. That is, the contextual application of M to P is the
  substitution of $P$ for $\Box$ in $M$.
\end{definition}

$\meaningof{-} : L \to \mathcal{P}(\pi)$

\begin{mathpar}
  \inferrule* [lab=collection] {} {\meaningof{true} = \pi, \and \meaningof{~E} = \pi \setminus \meaningof{E}, \and \meaningof{E_{1} \& E_{2}} = \meaningof{E_{1}} \cap \meaningof{E_{2}}}
\end{mathpar}

\begin{mathpar}
  \inferrule* [lab=structure] {} {\meaningof{0} = \{ P \in \pi | P \equiv 0 \}, \and \\ \meaningof{E_1 | E_2} = \{ P \in \pi | P \equiv P_{1} | P_{2}, P_{1} \in \meaningof{E_{1}}, P_{2} \in \meaningof{E_2}\} }
\end{mathpar}

\begin{mathpar}
 \inferrule* [lab=behavior] {} {\meaningof{\langle a?b \rangle E} = \{ P \in \pi | P \equiv Q | u?(y)P', \\ \and \\\\ \and \\ \;\;\; u \in \meaningof{a}, \forall z.P'\{z/y\} \in \meaningof{E\{z/b\}}\}, \and \\ \meaningof{a!E} = \{ P \in \pi | P \equiv Q | x!\langle P' \rangle, x \in \meaningof{a} P' \in \meaningof{E}\} }
\end{mathpar}

\begin{mathpar}
 \inferrule* [lab=nominal] {} {\meaningof{\quotep{E}} = \{ \quotep{P} \in \quotep{\pi} | P \in \meaningof{E} \}, \and \meaningof{\quotep{P}} = \{ \quotep{Q} \in \quotep{\pi} | P \equiv Q \} \and \\ \meaningof{@\quotep{E}} = \{ P \in \pi | P \equiv @x, x \in \meaningof{E} \}}
\end{mathpar}

\begin{eqnarray*}
  \\
  \meaningof{-} : TS \to ST
\end{eqnarray*}

\begin{eqnarray*}
  \\
  L : TS \to ST
\end{eqnarray*}

\begin{eqnarray*}
  \\
  P \models E \iff P \in \meaningof{E}
\end{eqnarray*}

\begin{eqnarray*}
  P \approx_{L} Q \iff \forall E \in L. P \models E \iff Q \models E
\end{eqnarray*}

\begin{eqnarray*}
  P \approx_{K} Q
\end{eqnarray*}

\begin{eqnarray*}
  P \approx Q
\end{eqnarray*}

$\approx_{K} = \approx = \approx_{L}$

\subsubsection{Contextual duality}

Note that contexts extend the quotation operation to a family of
operations from processes to names. Given a context, $M$, we can
define a \emph{nominal context}, $\quotep{M}$ by $\quotep{M}[P] :=
\quotep{M[P]}$. To foreshadow what is to come we observe that these
operations enjoy a duality with processes very much like the duality
between vectors and maps from vectors to scalars.

Further, because the calculus is essentially higher-order, we have a
correspondence between contexts and processes. More specifically,
given a name $x$ and a context $M$ we can construct $M^{*}_{x}$ such
that 

\begin{mathpar}
  M^{*}_{x} | \lift{x}{P} \red M[P]
\end{mathpar}

namely,

\begin{mathpar}
  M^{*}_{x} := x?(u).M[\dropn{u}]
\end{mathpar}

The dependence of $M^{*}_{x}$ on a name makes it an abstraction, 

\begin{mathpar}
  M^{*} := (x)x?(u).M[\dropn{u}]
\end{mathpar}

\subsection{Additional notation}

It will sometimes be convenient to denote the process a name
quotes. We already have the notation $x = \quotep{P}$, but it will be
convenient to introduce an alternate notation, $\procn{x}$, when we
want to emphasize the connection to the use of the name. Note that, by
virtue of name equivalence, $\quotep{\procn{x}} \nameeq x$; so, the
notation is consistent with previous definitions.

Further, because names have structure it is possible to effect
substitutions on the basis of that structure. This means we need to
upgrade our notation for substitutions, which we accomplish by
adapting comprehension notation. Thus,

\begin{mathpar}
  P\{ y / x : x \in S \}
\end{mathpar}

is interpreted to mean the process derived from P by replacing (in a
capture-avoiding manner) each occurrence of $x$ in $S$ by $y$. For example,

\begin{mathpar}
  P\{ \quotep{\procn{x}|\procn{x}} / x : x \in \freenames{P} \}
\end{mathpar}

will replace each (occurrence) of a free name $x$ in $P$ by
$\quotep{\procn{x}|\procn{x}}$.

Also, we will avail ourselves of the notation $x^{L}$ and $x^{R}$ to
denote injections of a name into disjoint copies of the name
space. There are numerous ways to accomplish this. One example can be
found in \cite{MeredithR05}. This notation overloads to vectors of
names: $\vec{x}^{\pi} := (x_{i}^{\pi} \; : \; 0 \leq i < |\vec{x}| )$ where $\pi \in \{L,R\}$.

We also use $P^{\Box} := P|\Box$.

In \cite{MeredithR05} an interpretation of the new operator is
given. It turns out that there are several possible interpretations
all enjoying the requisite algebraic properties of the operator (see
\cite{milner91polyadicpi}). We will therefore make liberal use of
$(\nu\; \vec{x})P$.

% subsection the_syntax_and_semantics_of_the_notation_system (end)   

\input{qm2pi.qmops} 

\input{qm2pi.sterngerlach} 

\input{qm2pi.metric} 

% section concurrent_process_calculi (end)

%\input{qm2pi.proofsketch}

% section proof sketch (end)

%\input{qm2pi.slviaknots} 

% section spatial logic via knots (end)

\input{qm2pi.conclusion}

% section conclusion (end)

%\input{qm2pi.dtcodes} 

% section wiring algorithm (end)

\input{qm2pi.ack} 

% section acknowledgments (end)

\newpage


\bibliographystyle{plain}   
\bibliography{../../biblios/main.bib}

\input{qm2pi.rhodetails}

\end{document}

 

%\documentclass[12pt]{llncs}
%\documentclass{jktr}

\usepackage[pdftex]{hyperref}                   
\usepackage {listings}
\usepackage {mathpartir}
\usepackage{bcprules}
%\usepackage{listings}
                       
\usepackage{graphicx} 
%\usepackage[margins=2.5cm,nohead,nofoot]{geometry}
%\usepackage{geometry}
\usepackage{amsfonts}
\usepackage{amstext}
\usepackage{latexsym}
\usepackage{amssymb}
\usepackage{color}


%\include{myPreamble}
\include{qm2pi.local} 

%\ifpdf
%\usepackage[pdftex]{graphicx}
%\else
%\usepackage{graphicx}
%\fi

 % \ifpdf
%  \usepackage{pdfsync}
%  \if


%\title{Brief Article}
%\author{David F. Snyder}
%\author{L.G. Meredith}

%\address{Dept. of Math., Texas State University--San Marcos, San Marcos, TX 78666}
       
\pagestyle{empty}


\begin{document}

\lstset{language=[Objective]Caml,frame=shadowbox}

\input{qm2pi.front}

% section front matter (end)

\input{qm2pi.intro} 
 
% section introduction (end)

% \input{qm2pi.knotations} 

% section notation (end)

\input{qm2pi.process.calculi} 

% section concurrent_process_calculi_and_spatial_logics_ (end)
    
%\input{qm2pi.knots2pi} 

%\input{qm2pi.trefoil} 

%\input{qm2pi.mainthm} 

% subsection basic_interpretation (end)

%\input{qm2pi.rho.presentation} 
\subsection{The syntax and semantics of the notation system}\label{sub:the_syntax_and_semantics_of_the_notation_system} % (fold)

We now summarize a technical presentation of the calculus that
embodies our theory of dynamics. The typical presentation of such a
calculus follows the style of giving generators and relations on
them. The grammar, below, describing term constructors, freely
generates the set of processes, $\Proc$. This set is then quotiented
by a relation known as structural congruence and it is over this set
that the notion of dynamics is expressed. This presentation is
essentially that of \cite{MeredithR05} with the addition of
polyadicity and summation. For readability we have relegated some of
the technical subtleties to an appendix.

\subsubsection{Process grammar}\label{subsub:process_grammar}

\begin{mathpar}
  \inferrule* [lab=synchronization] {} {{M} \bc \pzero \;|\; x?F \;|\; x!C }
  \and
  \inferrule* [lab=abstraction] {} {{F} \bc (x)P}
  \and
  \inferrule* [lab=concretion] {} {{C} \bc \langle Q \rangle}
  \and
  \inferrule* [lab=process] {} {{P,Q} \bc M \;| \;P|Q \;|\; @{x}}
  \and
  \inferrule* [lab=name] {} {{x} \bc \quotep{P}}
\end{mathpar} 

Note that $\vec{x}$ (resp. $\vec{P}$) denotes a vector of names
(resp. processes) of length $|\vec{x}|$ (resp. $|\vec{P}|$). We adopt
the following useful abbreviations.

\begin{mathpar}
   x?(\vec{y}).P := x.(\vec{y})P \and  x\clift{\vec{P}} := x.\clift{\vec{P}}
   \and x!(y) := \lift{x}{\dropn{y}}
   \and \Pi_{i=0}^{n-1}P_i := P_0 | \ldots | P_{n-1}
\end{mathpar}

\subsubsection{Structural congruence}

\paragraph{Free and bound names and alpha-equivalence.} At the
core of structural equivalence is alpha-equivalence which identifies
process that are the same up to a change of variable. Formally, we
recognize the distinction between free and bound names. The free names
of a process, $\freenames{P}$, may be calculated recursively as
follows:

\begin{mathpar}
\freenames{\pzero} := \emptyset
  \and \\
  \freenames{x?(y).P} := \{ x \} \cup (\freenames{P} \setminus \{ y \})
  \and 
  \freenames{x!\langle P \rangle} := \{ x \} \cup \{ P \} 
  \and \\
  \freenames{P|Q} := \freenames{P} \cup \freenames{Q}
  \and \\
  \freenames{@{x}} := \{ x \}
\end{mathpar}

$\pi$
$\quotep{\pi}$

$\freenames{-} : \pi \to \mathcal{P}(\quotep{\pi})$

\begin{eqnarray*}
  \freenames{\pzero} & := & \emptyset \\
  \freenames{x?(y).P} & := & \{ x \} \cup (\freenames{P} \setminus \{ y \}) \\
  \freenames{x!\langle P \rangle} & := & \{ x \} \cup \{ P \} \\
  \freenames{P|Q} & := & \freenames{P} \cup \freenames{Q} \\
  \freenames{\dropn{x}} & := & \{ x \}
\end{eqnarray*}

The bound names of a process, $\boundnames{P}$, are those names occurring in $P$
that are not free. For example, in $x?(y).0$, the name $x$ is free, while $y$ is bound.

\begin{mathpar}
  \inferrule* [lab=monoidal-laws] {} { P|Q \equiv Q|P \and P|0 \equiv P \and P|(Q|R) \equiv (P|Q)|R }
\end{mathpar}

\begin{mathpar}
  \inferrule* [lab=alpha-equivalence] {} { (x)P \equiv (y)P\{y/x\} \and y \not\in \freenames{P} }
\end{mathpar}

\begin{definition}
Then two processes, $P,Q$, are alpha-equivalent if $P = Q\{\vec{y}/\vec{x}\}$ for
some $\vec{x} \in \boundnames{Q},\vec{y} \in \boundnames{P}$, where $Q\{\vec{y}/\vec{x}\}$
denotes the capture-avoiding substitution of $\vec{y}$ for $\vec{x}$ in $Q$.
\end{definition}

\begin{definition}
  The {\em structural congruence} \cite{SangiorgiWalker} , $\equiv$,
  between processes is the least congruence containing
  alpha-equivalence, satisfying the abelian monoid laws
  (associativity, commutativity and $\pzero$ as identity) for parallel
  composition $|$ and for summation $+$.
\end{definition}

\subsection{Name equivalence}

We take name equivalence, written $\nameeq$, to be the smallest
equivalence relation generated by the following rules.

\begin{mathpar}
\inferrule*[lab=Quote-drop]
{ }
{ \quotep{@{x}} \nameeq x }

\inferrule*[lab=Struct-equiv]
{ P \scong Q }
{ \quotep{P} \nameeq \quotep{Q} }
\end{mathpar}

The astute reader will have noticed that the mutual recursion of names
and processes imposes a mutual recursion on alpha-equivalence and
structural equivalence via name-equivalence. Fortunately, all of this
works out pleasantly and we may calculate in the natural way, free of
concern. The reader interested in the details is referred to the
appendix \ref{appendix:rho_details}.

\subsection{Substitution}

We use $\Proc$ for the set of processes, $\QProc$ for the set of
names, and $\id{\{}\vec{y} / \vec{x} \id{\}}$ to denote partial maps,
$s : \QProc \rightarrow \QProc$. A map, $s$ lifts, uniquely, to a map
on process terms, $\widehat{s} : \Proc \rightarrow \Proc$ by the
following equations.

\begin{mathpar}
  (0) \psubstp{Q}{P} := 0 \\
  (R \juxtap S) \psubstp{Q}{P}
  :=    
  (R)\psubstp{Q}{P} \juxtap (S) \psubstp{Q}{P} \\
  (x?(y).R) \psubstp{Q}{P}    
  :=    
  (x)\substp{Q}{P} (z)\concat( (R \psubstn{z}{y}) \psubstp{Q}{P} ) \\
  (\lift{x}{R}) \psubstp{Q}{P}  
  :=
  \lift{(x)\substp{Q}{P}}{ R \psubstp{Q}{P} } \\
%   (\dropn{x})  \psubstp{Q}{P}       
%   := 
%   \left\{ 
%     \begin{array}{ccc} 
%       \dropn{\quotep{Q}} & & x \nameeq \quotep{P} \\
%       \dropn{x} & & otherwise \\
%     \end{array}
%   \right. 
  (\dropn{x})  \psubstp{Q}{P}       
  := 
  \left\{ 
    \begin{array}{ccc} 
      Q & & x \nameeq \quotep{P} \\
      \dropn{x} & & otherwise \\
    \end{array}
  \right.
\end{mathpar}
 

where

\begin{eqnarray}
  (x)\id{\{} \lpquote Q \rpquote / \lpquote P \rpquote \id{\}}            = 
  \left\{ 
    \begin{array}{ccc}
      \lpquote Q \rpquote & & x \nameeq \lpquote P \rpquote \\
      x & & otherwise \\
    \end{array}
  \right. \nonumber
\end{eqnarray}

and $z$ is chosen distinct from $\quotep{P}$, $\quotep{Q}$, the free
names in $Q$, and all the names in $R$. Our $\alpha$-equivalence will
be built in the standard way from this substitution.

\begin{remark}\label{rem:no_self_referential_names}
  One consequence of these definitions is that $\forall P. \quotep{P}
  \not\in \freenames{P}$.
\end{remark}

\subsection{ Dynamic quote: an example }

Anticipating something of what's to come, consider applying the
substitution, $\widehat{\id{\{}u / z \id{\}}}$, to the following pair
of processes, $\lift{w}{y!(z)}$ and $w[ \lpquote y!(z) \rpquote ]$.

\begin{eqnarray}
	\lift{w}{y!(z)}\widehat{\id{\{}u / z \id{\}}}
		& = &
		\lift{w}{y!(u)} \nonumber\\
	w[ \lpquote y!(z) \rpquote ] \widehat{ \id{\{}u / z \id{\}} }
		& = &
		w[ \lpquote y!(z) \rpquote ] \nonumber
\end{eqnarray}

Because the body of the process between quotes is impervious to
substitution, we get radically different answers. In fact, by
examining the first process in an input context,
e.g. $x?(z).\lift{w}{y!(z)}$, we see that the process under the lift
operator may be shaped by prefixed inputs binding a name inside it. In
this sense, the lift operator will be seen as a way to dynamically
construct processes before reifying them as names.

Finally equipped with these standard features we can present the
dynamics of the calculus.

\subsubsection{Operational semantics} 

Finally, we introduce the computational dynamics. What marks these
algebras as distinct from other more traditionally studied algebraic
structures, e.g. vector spaces or polynomial rings, is the manner in
which dynamics is captured. In traditional structures, dynamics is typically
expressed through morphisms between such structures, as in linear maps
between vector spaces or morphisms between rings. In algebras
associated with the semantics of computation, the dynamics is
expressed as part of the algebraic structure itself, through a
reduction reduction relation typically denoted by $\red$. Below, we
give a recursive presentation of this relation for the calculus used
in the encoding.

$\red \subseteq \pi \times \pi$
$\red : \pi \to \mathcal{P}(\pi)$

\begin{mathpar}
  \inferrule* [lab=Comm] { \textsf{match}( x_{src}, x_{trgt} ) } { x_{trgt}?(y)P \; | \; x_{src}!\langle {Q} \rangle \red P\{\quotep{Q}/y}\} }
  \and \\
  \inferrule* [lab=Par] {{P} \red {P}'} {{{P} | {Q}} \red {{P}' | {Q}}}
  \and
  \inferrule* [lab=Equiv]{{{P} \scong {P}'} \andalso {{P}' \red {Q}'} \andalso {{Q}' \scong {Q}}}{{P} \red {Q}}
\end{mathpar}

\begin{eqnarray*}
  match_{\equiv} (\quotep{P},\quotep{Q}) & := & P \equiv Q \\
  match_{\dagger}(\quotep{P},\quotep{Q}) & := & \forall R. P|Q \red^{*} R => R \red^{*} 0 \\
  match_{K}(\quotep{P},\quotep{Q}) & := & K \mbox{ for some context } K
\end{eqnarray*}

$u?(x)P | u!\langle Q \rangle \red P\{\quotep{Q}/x\}$

%We write $\wred$ for $\red^*$, and $P\red$ if $\exists Q $ such that $ P \red Q$.
We write $P\red$ if $\exists Q $ such that $ P \red Q$ and $P\not\red$, otherwise.

\section{Replication}

As mentioned before, it is known that replication (and hence
recursion) can be implemented in a higher-order process algebra
\cite{SangiorgiWalker}. As our first example of calculation with the
machinery thus far presented we give the construction explicitly in
the {\rhoc}.

\begin{eqnarray}
	D_{x} & := & \prefix{x}{y}{(\binpar{\outputp{x}{y}}{@{y}})} \nonumber\\
	\bangp_{x}{P} & := & \binpar{{x}!\langle{\binpar{D_{x}}{P}}\rangle}{D_{x}} \nonumber
\end{eqnarray}

\begin{eqnarray}
	\bangp_{x}{P} & & \nonumber\\
	=
	& {x}!\langle{(\prefix{x}{y}{(\outputp{x}{y} | @{y})) | P}}\rangle 
	      | \prefix{x}{y}{(\outputp{x}{y} | @{y})} & \nonumber\\
	\red
	& (\outputp{x}{y} | @{y})\substn{\quotep{(\prefix{x}{y}{(@{y} | \outputp{x}{y})) | P}}}{y} & \nonumber\\
	=
	& \outputp{x}{\quotep{(\prefix{x}{y}{(\outputp{x}{y} | @{y})) | P}}}
	  | {(\prefix{x}{y}{(\outputp{x}{y} | @{y})) | P}} & \nonumber\\
	\red
	& \ldots & \nonumber\\
	\red^*
	& P | P | \ldots & \nonumber
\end{eqnarray}

Of course, this encoding, as an implementation, runs away, unfolding
$\bangp{P}$ eagerly. A lazier and more implementable replication
operator, restricted to input-guarded processes, may be obtained as follows.

\begin{eqnarray}
\bangp{\prefix{u}{v}{P}} 
	:= 
	\binpar{\lift{x}{\prefix{u}{v}{(\binpar{D(x)}{P})}}}{D(x)} \nonumber
\end{eqnarray}

\begin{remark}
  Note that the lazier definition still does not deal with summation
  or mixed summation (i.e. sums over input and output). The reader is
  invited to construct definitions of replication that deal with these
  features. 

  Further, the definitions are parameterized in a name, $x$. Can you,
  gentle reader, make a definition that eliminates this parameter and
  guarantees no accidental interaction between the replication
  machinery and the process being replicated -- i.e. no accidental
  sharing of names used by the process to get its work done and the
  name(s) used by the replication to effect copying. This latter
  revision of the definition of replication is crucial to obtaining
  the expected identity $!!P \sim !P$.
\end{remark}

\begin{remark}\label{rem:paradoxical_combinator}
  The reader familiar with the lambda calculus will have noticed the
  similarity between $D$ and the paradoxical combinator.

  [Ed. note: the existence of this seems to suggest we have to be more
  restrictive on the set of processes and names we admit if we are to
  support no-cloning.]
\end{remark}

\subsubsection{Bisimulation}

The computational dynamics gives rise to another kind of equivalence,
the equivalence of computational behavior. As previously mentioned
this is typically captured \emph{via} some form of bisimulation.

% The notion we use in this paper is weak barbed bisimulation
% \cite{milner91polyadicpi}.

The notion we use in this paper is derived from weak barbed
bisimulation \cite{milner91polyadicpi}. 

\begin{definition}
An \emph{observation relation}, $\downarrow_{\mathcal N}$, over a set
of names, $\mathcal N$, is the smallest relation satisfying the rules
below.

\infrule[Out-barb]{y \in {\mathcal N}, \; x \nameeq y}
		  {\outputp{x}{v} \downarrow_{\mathcal N} x}
\infrule[Par-barb]{\mbox{$P\downarrow_{\mathcal N} x$ or $Q\downarrow_{\mathcal N} x$}}
		  {\binpar{P}{Q} \downarrow_{\mathcal N} x}

We write $P \Downarrow_{\mathcal N} x$ if there is $Q$ such that 
$P \wred Q$ and $Q \downarrow_{\mathcal N} x$.
\end{definition}

\begin{definition}
%\label{def.bbisim}
An  ${\mathcal N}$-\emph{barbed bisimulation} over a set of names, ${\mathcal N}$, is a symmetric binary relation 
${\mathcal S}_{\mathcal N}$ between agents such that $P\rel{S}_{\mathcal N}Q$ implies:
\begin{enumerate}
\item If $P \red P'$ then $Q \wred Q'$ and $P'\rel{S}_{\mathcal N} Q'$.
\item If $P\downarrow_{\mathcal N} x$, then $Q\Downarrow_{\mathcal N} x$.
\end{enumerate}
$P$ is ${\mathcal N}$-barbed bisimilar to $Q$, written
$P \wbbisim_{\mathcal N} Q$, if $P \rel{S}_{\mathcal N} Q$ for some ${\mathcal N}$-barbed bisimulation ${\mathcal S}_{\mathcal N}$.
\end{definition}

$\mathcal{R} \subseteq \pi \times \pi$

$P \mathcal{R} Q => \forall P'. P \red P' \Rightarrow \exists Q'. Q \red Q', P' \mathcal{R} Q'$

$P \vdash x \Rightarrow Q \vdash x$

\begin{mathpar}
  \inferrule*[lab=Out-barb]{x \nameeq y}{{y}!\langle{Q}\rangle \vdash x}
  \and
  \inferrule*[lab=Par-barb]{\mbox{$P\vdash x$ or $Q\vdash x$}}{\binpar{P}{Q} \vdash x}
\end{mathpar}

\subsubsection{Contexts}

One of the principle advantages of computational calculi like the
$\pi$-calculus is a well-defined notion of context,
contextual-equivalence and a correlation between
contextual-equivalence and notions of bisimulation. The notion of
context allows the decomposition of a process into (sub-)process and
its syntactic environment, its context. Thus, a context may be
thought of as a process with a ``hole'' (written $\Box$) in it. The
application of a context $M$ to a process $P$, written $M[P]$, is
tantamount to filling the hole in $M$ with $P$. In this paper we do
not need the full weight of this theory, but do make use of the notion
of context in the proof the main theorem. 

\begin{mathpar}
  \inferrule* [lab=summation] {} {{M_{M},M_{N}} \bc \Box \;|\; x.M_{A} \;|\; M_{M}+M_{N}}
  \and
  \inferrule* [lab=agent] {} {{M_{A}} \bc (\vec{x})M_{P} \;| \; \clift{P_0,\ldots,M_{P},\ldots,P_N}}
  \and \\
  \inferrule* [lab=process] {} {{M_{P}} \bc M_{N} \;| \;P|M_{P} }
\end{mathpar} 

\begin{mathpar}
  \inferrule* [lab=sychronization] {} {M_{N} \bc \Box \;|\; x?M_{F} \;|\; x!M_{C}}
  \and
  \inferrule* [lab=abstraction] {} {{M_{F}} \bc (x)M_{P} }
  \and
  \inferrule* [lab=concretion] {} {{M_{C}} \bc \langle M_{P} \rangle }
  \and \\
  \inferrule* [lab=process] {} {{M_{P}} \bc M_{N} \;| \;P|M_{P} }
\end{mathpar}

\begin{definition}[contextual application] Given a context $M$, and
  process $P$, we define the \emph{contextual application}, $M[P] :=
  M\{P/\Box\}$. That is, the contextual application of M to P is the
  substitution of $P$ for $\Box$ in $M$.
\end{definition}

$\meaningof{-} : L \to \mathcal{P}(\pi)$

\begin{mathpar}
  \inferrule* [lab=collection] {} {\meaningof{true} = \pi, \and \meaningof{~E} = \pi \setminus \meaningof{E}, \and \meaningof{E_{1} \& E_{2}} = \meaningof{E_{1}} \cap \meaningof{E_{2}}}
\end{mathpar}

\begin{mathpar}
  \inferrule* [lab=structure] {} {\meaningof{0} = \{ P \in \pi | P \equiv 0 \}, \and \\ \meaningof{E_1 | E_2} = \{ P \in \pi | P \equiv P_{1} | P_{2}, P_{1} \in \meaningof{E_{1}}, P_{2} \in \meaningof{E_2}\} }
\end{mathpar}

\begin{mathpar}
 \inferrule* [lab=behavior] {} {\meaningof{\langle a?b \rangle E} = \{ P \in \pi | P \equiv Q | u?(y)P', \\ \and \\\\ \and \\ \;\;\; u \in \meaningof{a}, \forall z.P'\{z/y\} \in \meaningof{E\{z/b\}}\}, \and \\ \meaningof{a!E} = \{ P \in \pi | P \equiv Q | x!\langle P' \rangle, x \in \meaningof{a} P' \in \meaningof{E}\} }
\end{mathpar}

\begin{mathpar}
 \inferrule* [lab=nominal] {} {\meaningof{\quotep{E}} = \{ \quotep{P} \in \quotep{\pi} | P \in \meaningof{E} \}, \and \meaningof{\quotep{P}} = \{ \quotep{Q} \in \quotep{\pi} | P \equiv Q \} \and \\ \meaningof{@\quotep{E}} = \{ P \in \pi | P \equiv @x, x \in \meaningof{E} \}}
\end{mathpar}

\begin{eqnarray*}
  \\
  \meaningof{-} : TS \to ST
\end{eqnarray*}

\begin{eqnarray*}
  \\
  L : TS \to ST
\end{eqnarray*}

\begin{eqnarray*}
  \\
  P \models E \iff P \in \meaningof{E}
\end{eqnarray*}

\begin{eqnarray*}
  P \approx_{L} Q \iff \forall E \in L. P \models E \iff Q \models E
\end{eqnarray*}

\begin{eqnarray*}
  P \approx_{K} Q
\end{eqnarray*}

\begin{eqnarray*}
  P \approx Q
\end{eqnarray*}

$\approx_{K} = \approx = \approx_{L}$

\subsubsection{Contextual duality}

Note that contexts extend the quotation operation to a family of
operations from processes to names. Given a context, $M$, we can
define a \emph{nominal context}, $\quotep{M}$ by $\quotep{M}[P] :=
\quotep{M[P]}$. To foreshadow what is to come we observe that these
operations enjoy a duality with processes very much like the duality
between vectors and maps from vectors to scalars.

Further, because the calculus is essentially higher-order, we have a
correspondence between contexts and processes. More specifically,
given a name $x$ and a context $M$ we can construct $M^{*}_{x}$ such
that 

\begin{mathpar}
  M^{*}_{x} | \lift{x}{P} \red M[P]
\end{mathpar}

namely,

\begin{mathpar}
  M^{*}_{x} := x?(u).M[\dropn{u}]
\end{mathpar}

The dependence of $M^{*}_{x}$ on a name makes it an abstraction, 

\begin{mathpar}
  M^{*} := (x)x?(u).M[\dropn{u}]
\end{mathpar}

\subsection{Additional notation}

It will sometimes be convenient to denote the process a name
quotes. We already have the notation $x = \quotep{P}$, but it will be
convenient to introduce an alternate notation, $\procn{x}$, when we
want to emphasize the connection to the use of the name. Note that, by
virtue of name equivalence, $\quotep{\procn{x}} \nameeq x$; so, the
notation is consistent with previous definitions.

Further, because names have structure it is possible to effect
substitutions on the basis of that structure. This means we need to
upgrade our notation for substitutions, which we accomplish by
adapting comprehension notation. Thus,

\begin{mathpar}
  P\{ y / x : x \in S \}
\end{mathpar}

is interpreted to mean the process derived from P by replacing (in a
capture-avoiding manner) each occurrence of $x$ in $S$ by $y$. For example,

\begin{mathpar}
  P\{ \quotep{\procn{x}|\procn{x}} / x : x \in \freenames{P} \}
\end{mathpar}

will replace each (occurrence) of a free name $x$ in $P$ by
$\quotep{\procn{x}|\procn{x}}$.

Also, we will avail ourselves of the notation $x^{L}$ and $x^{R}$ to
denote injections of a name into disjoint copies of the name
space. There are numerous ways to accomplish this. One example can be
found in \cite{MeredithR05}. This notation overloads to vectors of
names: $\vec{x}^{\pi} := (x_{i}^{\pi} \; : \; 0 \leq i < |\vec{x}| )$ where $\pi \in \{L,R\}$.

We also use $P^{\Box} := P|\Box$.

In \cite{MeredithR05} an interpretation of the new operator is
given. It turns out that there are several possible interpretations
all enjoying the requisite algebraic properties of the operator (see
\cite{milner91polyadicpi}). We will therefore make liberal use of
$(\nu\; \vec{x})P$.

% subsection the_syntax_and_semantics_of_the_notation_system (end)   

\input{qm2pi.qmops} 

\input{qm2pi.sterngerlach} 

\input{qm2pi.metric} 

% section concurrent_process_calculi (end)

%\input{qm2pi.proofsketch}

% section proof sketch (end)

%\input{qm2pi.slviaknots} 

% section spatial logic via knots (end)

\input{qm2pi.conclusion}

% section conclusion (end)

%\input{qm2pi.dtcodes} 

% section wiring algorithm (end)

\input{qm2pi.ack} 

% section acknowledgments (end)

\newpage


\bibliographystyle{plain}   
\bibliography{../../biblios/main.bib}

\input{qm2pi.rhodetails}

\end{document}

 

%\documentclass[12pt]{llncs}
%\documentclass{jktr}

\usepackage[pdftex]{hyperref}                   
\usepackage {listings}
\usepackage {mathpartir}
\usepackage{bcprules}
%\usepackage{listings}
                       
\usepackage{graphicx} 
%\usepackage[margins=2.5cm,nohead,nofoot]{geometry}
%\usepackage{geometry}
\usepackage{amsfonts}
\usepackage{amstext}
\usepackage{latexsym}
\usepackage{amssymb}
\usepackage{color}


%\include{myPreamble}
\include{qm2pi.local} 

%\ifpdf
%\usepackage[pdftex]{graphicx}
%\else
%\usepackage{graphicx}
%\fi

 % \ifpdf
%  \usepackage{pdfsync}
%  \if


%\title{Brief Article}
%\author{David F. Snyder}
%\author{L.G. Meredith}

%\address{Dept. of Math., Texas State University--San Marcos, San Marcos, TX 78666}
       
\pagestyle{empty}


\begin{document}

\lstset{language=[Objective]Caml,frame=shadowbox}

\input{qm2pi.front}

% section front matter (end)

\input{qm2pi.intro} 
 
% section introduction (end)

% \input{qm2pi.knotations} 

% section notation (end)

\input{qm2pi.process.calculi} 

% section concurrent_process_calculi_and_spatial_logics_ (end)
    
%\input{qm2pi.knots2pi} 

%\input{qm2pi.trefoil} 

%\input{qm2pi.mainthm} 

% subsection basic_interpretation (end)

%\input{qm2pi.rho.presentation} 
\subsection{The syntax and semantics of the notation system}\label{sub:the_syntax_and_semantics_of_the_notation_system} % (fold)

We now summarize a technical presentation of the calculus that
embodies our theory of dynamics. The typical presentation of such a
calculus follows the style of giving generators and relations on
them. The grammar, below, describing term constructors, freely
generates the set of processes, $\Proc$. This set is then quotiented
by a relation known as structural congruence and it is over this set
that the notion of dynamics is expressed. This presentation is
essentially that of \cite{MeredithR05} with the addition of
polyadicity and summation. For readability we have relegated some of
the technical subtleties to an appendix.

\subsubsection{Process grammar}\label{subsub:process_grammar}

\begin{mathpar}
  \inferrule* [lab=synchronization] {} {{M} \bc \pzero \;|\; x?F \;|\; x!C }
  \and
  \inferrule* [lab=abstraction] {} {{F} \bc (x)P}
  \and
  \inferrule* [lab=concretion] {} {{C} \bc \langle Q \rangle}
  \and
  \inferrule* [lab=process] {} {{P,Q} \bc M \;| \;P|Q \;|\; @{x}}
  \and
  \inferrule* [lab=name] {} {{x} \bc \quotep{P}}
\end{mathpar} 

Note that $\vec{x}$ (resp. $\vec{P}$) denotes a vector of names
(resp. processes) of length $|\vec{x}|$ (resp. $|\vec{P}|$). We adopt
the following useful abbreviations.

\begin{mathpar}
   x?(\vec{y}).P := x.(\vec{y})P \and  x\clift{\vec{P}} := x.\clift{\vec{P}}
   \and x!(y) := \lift{x}{\dropn{y}}
   \and \Pi_{i=0}^{n-1}P_i := P_0 | \ldots | P_{n-1}
\end{mathpar}

\subsubsection{Structural congruence}

\paragraph{Free and bound names and alpha-equivalence.} At the
core of structural equivalence is alpha-equivalence which identifies
process that are the same up to a change of variable. Formally, we
recognize the distinction between free and bound names. The free names
of a process, $\freenames{P}$, may be calculated recursively as
follows:

\begin{mathpar}
\freenames{\pzero} := \emptyset
  \and \\
  \freenames{x?(y).P} := \{ x \} \cup (\freenames{P} \setminus \{ y \})
  \and 
  \freenames{x!\langle P \rangle} := \{ x \} \cup \{ P \} 
  \and \\
  \freenames{P|Q} := \freenames{P} \cup \freenames{Q}
  \and \\
  \freenames{@{x}} := \{ x \}
\end{mathpar}

$\pi$
$\quotep{\pi}$

$\freenames{-} : \pi \to \mathcal{P}(\quotep{\pi})$

\begin{eqnarray*}
  \freenames{\pzero} & := & \emptyset \\
  \freenames{x?(y).P} & := & \{ x \} \cup (\freenames{P} \setminus \{ y \}) \\
  \freenames{x!\langle P \rangle} & := & \{ x \} \cup \{ P \} \\
  \freenames{P|Q} & := & \freenames{P} \cup \freenames{Q} \\
  \freenames{\dropn{x}} & := & \{ x \}
\end{eqnarray*}

The bound names of a process, $\boundnames{P}$, are those names occurring in $P$
that are not free. For example, in $x?(y).0$, the name $x$ is free, while $y$ is bound.

\begin{mathpar}
  \inferrule* [lab=monoidal-laws] {} { P|Q \equiv Q|P \and P|0 \equiv P \and P|(Q|R) \equiv (P|Q)|R }
\end{mathpar}

\begin{mathpar}
  \inferrule* [lab=alpha-equivalence] {} { (x)P \equiv (y)P\{y/x\} \and y \not\in \freenames{P} }
\end{mathpar}

\begin{definition}
Then two processes, $P,Q$, are alpha-equivalent if $P = Q\{\vec{y}/\vec{x}\}$ for
some $\vec{x} \in \boundnames{Q},\vec{y} \in \boundnames{P}$, where $Q\{\vec{y}/\vec{x}\}$
denotes the capture-avoiding substitution of $\vec{y}$ for $\vec{x}$ in $Q$.
\end{definition}

\begin{definition}
  The {\em structural congruence} \cite{SangiorgiWalker} , $\equiv$,
  between processes is the least congruence containing
  alpha-equivalence, satisfying the abelian monoid laws
  (associativity, commutativity and $\pzero$ as identity) for parallel
  composition $|$ and for summation $+$.
\end{definition}

\subsection{Name equivalence}

We take name equivalence, written $\nameeq$, to be the smallest
equivalence relation generated by the following rules.

\begin{mathpar}
\inferrule*[lab=Quote-drop]
{ }
{ \quotep{@{x}} \nameeq x }

\inferrule*[lab=Struct-equiv]
{ P \scong Q }
{ \quotep{P} \nameeq \quotep{Q} }
\end{mathpar}

The astute reader will have noticed that the mutual recursion of names
and processes imposes a mutual recursion on alpha-equivalence and
structural equivalence via name-equivalence. Fortunately, all of this
works out pleasantly and we may calculate in the natural way, free of
concern. The reader interested in the details is referred to the
appendix \ref{appendix:rho_details}.

\subsection{Substitution}

We use $\Proc$ for the set of processes, $\QProc$ for the set of
names, and $\id{\{}\vec{y} / \vec{x} \id{\}}$ to denote partial maps,
$s : \QProc \rightarrow \QProc$. A map, $s$ lifts, uniquely, to a map
on process terms, $\widehat{s} : \Proc \rightarrow \Proc$ by the
following equations.

\begin{mathpar}
  (0) \psubstp{Q}{P} := 0 \\
  (R \juxtap S) \psubstp{Q}{P}
  :=    
  (R)\psubstp{Q}{P} \juxtap (S) \psubstp{Q}{P} \\
  (x?(y).R) \psubstp{Q}{P}    
  :=    
  (x)\substp{Q}{P} (z)\concat( (R \psubstn{z}{y}) \psubstp{Q}{P} ) \\
  (\lift{x}{R}) \psubstp{Q}{P}  
  :=
  \lift{(x)\substp{Q}{P}}{ R \psubstp{Q}{P} } \\
%   (\dropn{x})  \psubstp{Q}{P}       
%   := 
%   \left\{ 
%     \begin{array}{ccc} 
%       \dropn{\quotep{Q}} & & x \nameeq \quotep{P} \\
%       \dropn{x} & & otherwise \\
%     \end{array}
%   \right. 
  (\dropn{x})  \psubstp{Q}{P}       
  := 
  \left\{ 
    \begin{array}{ccc} 
      Q & & x \nameeq \quotep{P} \\
      \dropn{x} & & otherwise \\
    \end{array}
  \right.
\end{mathpar}
 

where

\begin{eqnarray}
  (x)\id{\{} \lpquote Q \rpquote / \lpquote P \rpquote \id{\}}            = 
  \left\{ 
    \begin{array}{ccc}
      \lpquote Q \rpquote & & x \nameeq \lpquote P \rpquote \\
      x & & otherwise \\
    \end{array}
  \right. \nonumber
\end{eqnarray}

and $z$ is chosen distinct from $\quotep{P}$, $\quotep{Q}$, the free
names in $Q$, and all the names in $R$. Our $\alpha$-equivalence will
be built in the standard way from this substitution.

\begin{remark}\label{rem:no_self_referential_names}
  One consequence of these definitions is that $\forall P. \quotep{P}
  \not\in \freenames{P}$.
\end{remark}

\subsection{ Dynamic quote: an example }

Anticipating something of what's to come, consider applying the
substitution, $\widehat{\id{\{}u / z \id{\}}}$, to the following pair
of processes, $\lift{w}{y!(z)}$ and $w[ \lpquote y!(z) \rpquote ]$.

\begin{eqnarray}
	\lift{w}{y!(z)}\widehat{\id{\{}u / z \id{\}}}
		& = &
		\lift{w}{y!(u)} \nonumber\\
	w[ \lpquote y!(z) \rpquote ] \widehat{ \id{\{}u / z \id{\}} }
		& = &
		w[ \lpquote y!(z) \rpquote ] \nonumber
\end{eqnarray}

Because the body of the process between quotes is impervious to
substitution, we get radically different answers. In fact, by
examining the first process in an input context,
e.g. $x?(z).\lift{w}{y!(z)}$, we see that the process under the lift
operator may be shaped by prefixed inputs binding a name inside it. In
this sense, the lift operator will be seen as a way to dynamically
construct processes before reifying them as names.

Finally equipped with these standard features we can present the
dynamics of the calculus.

\subsubsection{Operational semantics} 

Finally, we introduce the computational dynamics. What marks these
algebras as distinct from other more traditionally studied algebraic
structures, e.g. vector spaces or polynomial rings, is the manner in
which dynamics is captured. In traditional structures, dynamics is typically
expressed through morphisms between such structures, as in linear maps
between vector spaces or morphisms between rings. In algebras
associated with the semantics of computation, the dynamics is
expressed as part of the algebraic structure itself, through a
reduction reduction relation typically denoted by $\red$. Below, we
give a recursive presentation of this relation for the calculus used
in the encoding.

$\red \subseteq \pi \times \pi$
$\red : \pi \to \mathcal{P}(\pi)$

\begin{mathpar}
  \inferrule* [lab=Comm] { \textsf{match}( x_{src}, x_{trgt} ) } { x_{trgt}?(y)P \; | \; x_{src}!\langle {Q} \rangle \red P\{\quotep{Q}/y}\} }
  \and \\
  \inferrule* [lab=Par] {{P} \red {P}'} {{{P} | {Q}} \red {{P}' | {Q}}}
  \and
  \inferrule* [lab=Equiv]{{{P} \scong {P}'} \andalso {{P}' \red {Q}'} \andalso {{Q}' \scong {Q}}}{{P} \red {Q}}
\end{mathpar}

\begin{eqnarray*}
  match_{\equiv} (\quotep{P},\quotep{Q}) & := & P \equiv Q \\
  match_{\dagger}(\quotep{P},\quotep{Q}) & := & \forall R. P|Q \red^{*} R => R \red^{*} 0 \\
  match_{K}(\quotep{P},\quotep{Q}) & := & K \mbox{ for some context } K
\end{eqnarray*}

$u?(x)P | u!\langle Q \rangle \red P\{\quotep{Q}/x\}$

%We write $\wred$ for $\red^*$, and $P\red$ if $\exists Q $ such that $ P \red Q$.
We write $P\red$ if $\exists Q $ such that $ P \red Q$ and $P\not\red$, otherwise.

\section{Replication}

As mentioned before, it is known that replication (and hence
recursion) can be implemented in a higher-order process algebra
\cite{SangiorgiWalker}. As our first example of calculation with the
machinery thus far presented we give the construction explicitly in
the {\rhoc}.

\begin{eqnarray}
	D_{x} & := & \prefix{x}{y}{(\binpar{\outputp{x}{y}}{@{y}})} \nonumber\\
	\bangp_{x}{P} & := & \binpar{{x}!\langle{\binpar{D_{x}}{P}}\rangle}{D_{x}} \nonumber
\end{eqnarray}

\begin{eqnarray}
	\bangp_{x}{P} & & \nonumber\\
	=
	& {x}!\langle{(\prefix{x}{y}{(\outputp{x}{y} | @{y})) | P}}\rangle 
	      | \prefix{x}{y}{(\outputp{x}{y} | @{y})} & \nonumber\\
	\red
	& (\outputp{x}{y} | @{y})\substn{\quotep{(\prefix{x}{y}{(@{y} | \outputp{x}{y})) | P}}}{y} & \nonumber\\
	=
	& \outputp{x}{\quotep{(\prefix{x}{y}{(\outputp{x}{y} | @{y})) | P}}}
	  | {(\prefix{x}{y}{(\outputp{x}{y} | @{y})) | P}} & \nonumber\\
	\red
	& \ldots & \nonumber\\
	\red^*
	& P | P | \ldots & \nonumber
\end{eqnarray}

Of course, this encoding, as an implementation, runs away, unfolding
$\bangp{P}$ eagerly. A lazier and more implementable replication
operator, restricted to input-guarded processes, may be obtained as follows.

\begin{eqnarray}
\bangp{\prefix{u}{v}{P}} 
	:= 
	\binpar{\lift{x}{\prefix{u}{v}{(\binpar{D(x)}{P})}}}{D(x)} \nonumber
\end{eqnarray}

\begin{remark}
  Note that the lazier definition still does not deal with summation
  or mixed summation (i.e. sums over input and output). The reader is
  invited to construct definitions of replication that deal with these
  features. 

  Further, the definitions are parameterized in a name, $x$. Can you,
  gentle reader, make a definition that eliminates this parameter and
  guarantees no accidental interaction between the replication
  machinery and the process being replicated -- i.e. no accidental
  sharing of names used by the process to get its work done and the
  name(s) used by the replication to effect copying. This latter
  revision of the definition of replication is crucial to obtaining
  the expected identity $!!P \sim !P$.
\end{remark}

\begin{remark}\label{rem:paradoxical_combinator}
  The reader familiar with the lambda calculus will have noticed the
  similarity between $D$ and the paradoxical combinator.

  [Ed. note: the existence of this seems to suggest we have to be more
  restrictive on the set of processes and names we admit if we are to
  support no-cloning.]
\end{remark}

\subsubsection{Bisimulation}

The computational dynamics gives rise to another kind of equivalence,
the equivalence of computational behavior. As previously mentioned
this is typically captured \emph{via} some form of bisimulation.

% The notion we use in this paper is weak barbed bisimulation
% \cite{milner91polyadicpi}.

The notion we use in this paper is derived from weak barbed
bisimulation \cite{milner91polyadicpi}. 

\begin{definition}
An \emph{observation relation}, $\downarrow_{\mathcal N}$, over a set
of names, $\mathcal N$, is the smallest relation satisfying the rules
below.

\infrule[Out-barb]{y \in {\mathcal N}, \; x \nameeq y}
		  {\outputp{x}{v} \downarrow_{\mathcal N} x}
\infrule[Par-barb]{\mbox{$P\downarrow_{\mathcal N} x$ or $Q\downarrow_{\mathcal N} x$}}
		  {\binpar{P}{Q} \downarrow_{\mathcal N} x}

We write $P \Downarrow_{\mathcal N} x$ if there is $Q$ such that 
$P \wred Q$ and $Q \downarrow_{\mathcal N} x$.
\end{definition}

\begin{definition}
%\label{def.bbisim}
An  ${\mathcal N}$-\emph{barbed bisimulation} over a set of names, ${\mathcal N}$, is a symmetric binary relation 
${\mathcal S}_{\mathcal N}$ between agents such that $P\rel{S}_{\mathcal N}Q$ implies:
\begin{enumerate}
\item If $P \red P'$ then $Q \wred Q'$ and $P'\rel{S}_{\mathcal N} Q'$.
\item If $P\downarrow_{\mathcal N} x$, then $Q\Downarrow_{\mathcal N} x$.
\end{enumerate}
$P$ is ${\mathcal N}$-barbed bisimilar to $Q$, written
$P \wbbisim_{\mathcal N} Q$, if $P \rel{S}_{\mathcal N} Q$ for some ${\mathcal N}$-barbed bisimulation ${\mathcal S}_{\mathcal N}$.
\end{definition}

$\mathcal{R} \subseteq \pi \times \pi$

$P \mathcal{R} Q => \forall P'. P \red P' \Rightarrow \exists Q'. Q \red Q', P' \mathcal{R} Q'$

$P \vdash x \Rightarrow Q \vdash x$

\begin{mathpar}
  \inferrule*[lab=Out-barb]{x \nameeq y}{{y}!\langle{Q}\rangle \vdash x}
  \and
  \inferrule*[lab=Par-barb]{\mbox{$P\vdash x$ or $Q\vdash x$}}{\binpar{P}{Q} \vdash x}
\end{mathpar}

\subsubsection{Contexts}

One of the principle advantages of computational calculi like the
$\pi$-calculus is a well-defined notion of context,
contextual-equivalence and a correlation between
contextual-equivalence and notions of bisimulation. The notion of
context allows the decomposition of a process into (sub-)process and
its syntactic environment, its context. Thus, a context may be
thought of as a process with a ``hole'' (written $\Box$) in it. The
application of a context $M$ to a process $P$, written $M[P]$, is
tantamount to filling the hole in $M$ with $P$. In this paper we do
not need the full weight of this theory, but do make use of the notion
of context in the proof the main theorem. 

\begin{mathpar}
  \inferrule* [lab=summation] {} {{M_{M},M_{N}} \bc \Box \;|\; x.M_{A} \;|\; M_{M}+M_{N}}
  \and
  \inferrule* [lab=agent] {} {{M_{A}} \bc (\vec{x})M_{P} \;| \; \clift{P_0,\ldots,M_{P},\ldots,P_N}}
  \and \\
  \inferrule* [lab=process] {} {{M_{P}} \bc M_{N} \;| \;P|M_{P} }
\end{mathpar} 

\begin{mathpar}
  \inferrule* [lab=sychronization] {} {M_{N} \bc \Box \;|\; x?M_{F} \;|\; x!M_{C}}
  \and
  \inferrule* [lab=abstraction] {} {{M_{F}} \bc (x)M_{P} }
  \and
  \inferrule* [lab=concretion] {} {{M_{C}} \bc \langle M_{P} \rangle }
  \and \\
  \inferrule* [lab=process] {} {{M_{P}} \bc M_{N} \;| \;P|M_{P} }
\end{mathpar}

\begin{definition}[contextual application] Given a context $M$, and
  process $P$, we define the \emph{contextual application}, $M[P] :=
  M\{P/\Box\}$. That is, the contextual application of M to P is the
  substitution of $P$ for $\Box$ in $M$.
\end{definition}

$\meaningof{-} : L \to \mathcal{P}(\pi)$

\begin{mathpar}
  \inferrule* [lab=collection] {} {\meaningof{true} = \pi, \and \meaningof{~E} = \pi \setminus \meaningof{E}, \and \meaningof{E_{1} \& E_{2}} = \meaningof{E_{1}} \cap \meaningof{E_{2}}}
\end{mathpar}

\begin{mathpar}
  \inferrule* [lab=structure] {} {\meaningof{0} = \{ P \in \pi | P \equiv 0 \}, \and \\ \meaningof{E_1 | E_2} = \{ P \in \pi | P \equiv P_{1} | P_{2}, P_{1} \in \meaningof{E_{1}}, P_{2} \in \meaningof{E_2}\} }
\end{mathpar}

\begin{mathpar}
 \inferrule* [lab=behavior] {} {\meaningof{\langle a?b \rangle E} = \{ P \in \pi | P \equiv Q | u?(y)P', \\ \and \\\\ \and \\ \;\;\; u \in \meaningof{a}, \forall z.P'\{z/y\} \in \meaningof{E\{z/b\}}\}, \and \\ \meaningof{a!E} = \{ P \in \pi | P \equiv Q | x!\langle P' \rangle, x \in \meaningof{a} P' \in \meaningof{E}\} }
\end{mathpar}

\begin{mathpar}
 \inferrule* [lab=nominal] {} {\meaningof{\quotep{E}} = \{ \quotep{P} \in \quotep{\pi} | P \in \meaningof{E} \}, \and \meaningof{\quotep{P}} = \{ \quotep{Q} \in \quotep{\pi} | P \equiv Q \} \and \\ \meaningof{@\quotep{E}} = \{ P \in \pi | P \equiv @x, x \in \meaningof{E} \}}
\end{mathpar}

\begin{eqnarray*}
  \\
  \meaningof{-} : TS \to ST
\end{eqnarray*}

\begin{eqnarray*}
  \\
  L : TS \to ST
\end{eqnarray*}

\begin{eqnarray*}
  \\
  P \models E \iff P \in \meaningof{E}
\end{eqnarray*}

\begin{eqnarray*}
  P \approx_{L} Q \iff \forall E \in L. P \models E \iff Q \models E
\end{eqnarray*}

\begin{eqnarray*}
  P \approx_{K} Q
\end{eqnarray*}

\begin{eqnarray*}
  P \approx Q
\end{eqnarray*}

$\approx_{K} = \approx = \approx_{L}$

\subsubsection{Contextual duality}

Note that contexts extend the quotation operation to a family of
operations from processes to names. Given a context, $M$, we can
define a \emph{nominal context}, $\quotep{M}$ by $\quotep{M}[P] :=
\quotep{M[P]}$. To foreshadow what is to come we observe that these
operations enjoy a duality with processes very much like the duality
between vectors and maps from vectors to scalars.

Further, because the calculus is essentially higher-order, we have a
correspondence between contexts and processes. More specifically,
given a name $x$ and a context $M$ we can construct $M^{*}_{x}$ such
that 

\begin{mathpar}
  M^{*}_{x} | \lift{x}{P} \red M[P]
\end{mathpar}

namely,

\begin{mathpar}
  M^{*}_{x} := x?(u).M[\dropn{u}]
\end{mathpar}

The dependence of $M^{*}_{x}$ on a name makes it an abstraction, 

\begin{mathpar}
  M^{*} := (x)x?(u).M[\dropn{u}]
\end{mathpar}

\subsection{Additional notation}

It will sometimes be convenient to denote the process a name
quotes. We already have the notation $x = \quotep{P}$, but it will be
convenient to introduce an alternate notation, $\procn{x}$, when we
want to emphasize the connection to the use of the name. Note that, by
virtue of name equivalence, $\quotep{\procn{x}} \nameeq x$; so, the
notation is consistent with previous definitions.

Further, because names have structure it is possible to effect
substitutions on the basis of that structure. This means we need to
upgrade our notation for substitutions, which we accomplish by
adapting comprehension notation. Thus,

\begin{mathpar}
  P\{ y / x : x \in S \}
\end{mathpar}

is interpreted to mean the process derived from P by replacing (in a
capture-avoiding manner) each occurrence of $x$ in $S$ by $y$. For example,

\begin{mathpar}
  P\{ \quotep{\procn{x}|\procn{x}} / x : x \in \freenames{P} \}
\end{mathpar}

will replace each (occurrence) of a free name $x$ in $P$ by
$\quotep{\procn{x}|\procn{x}}$.

Also, we will avail ourselves of the notation $x^{L}$ and $x^{R}$ to
denote injections of a name into disjoint copies of the name
space. There are numerous ways to accomplish this. One example can be
found in \cite{MeredithR05}. This notation overloads to vectors of
names: $\vec{x}^{\pi} := (x_{i}^{\pi} \; : \; 0 \leq i < |\vec{x}| )$ where $\pi \in \{L,R\}$.

We also use $P^{\Box} := P|\Box$.

In \cite{MeredithR05} an interpretation of the new operator is
given. It turns out that there are several possible interpretations
all enjoying the requisite algebraic properties of the operator (see
\cite{milner91polyadicpi}). We will therefore make liberal use of
$(\nu\; \vec{x})P$.

% subsection the_syntax_and_semantics_of_the_notation_system (end)   

\input{qm2pi.qmops} 

\input{qm2pi.sterngerlach} 

\input{qm2pi.metric} 

% section concurrent_process_calculi (end)

%\input{qm2pi.proofsketch}

% section proof sketch (end)

%\input{qm2pi.slviaknots} 

% section spatial logic via knots (end)

\input{qm2pi.conclusion}

% section conclusion (end)

%\input{qm2pi.dtcodes} 

% section wiring algorithm (end)

\input{qm2pi.ack} 

% section acknowledgments (end)

\newpage


\bibliographystyle{plain}   
\bibliography{../../biblios/main.bib}

\input{qm2pi.rhodetails}

\end{document}

 

% subsection basic_interpretation (end)

%\input{qm2pi.rho.presentation} 
\subsection{The syntax and semantics of the notation system}\label{sub:the_syntax_and_semantics_of_the_notation_system} % (fold)

We now summarize a technical presentation of the calculus that
embodies our theory of dynamics. The typical presentation of such a
calculus follows the style of giving generators and relations on
them. The grammar, below, describing term constructors, freely
generates the set of processes, $\Proc$. This set is then quotiented
by a relation known as structural congruence and it is over this set
that the notion of dynamics is expressed. This presentation is
essentially that of \cite{MeredithR05} with the addition of
polyadicity and summation. For readability we have relegated some of
the technical subtleties to an appendix.

\subsubsection{Process grammar}\label{subsub:process_grammar}

\begin{mathpar}
  \inferrule* [lab=synchronization] {} {{M} \bc \pzero \;|\; x?F \;|\; x!C }
  \and
  \inferrule* [lab=abstraction] {} {{F} \bc (x)P}
  \and
  \inferrule* [lab=concretion] {} {{C} \bc \langle Q \rangle}
  \and
  \inferrule* [lab=process] {} {{P,Q} \bc M \;| \;P|Q \;|\; @{x}}
  \and
  \inferrule* [lab=name] {} {{x} \bc \quotep{P}}
\end{mathpar} 

Note that $\vec{x}$ (resp. $\vec{P}$) denotes a vector of names
(resp. processes) of length $|\vec{x}|$ (resp. $|\vec{P}|$). We adopt
the following useful abbreviations.

\begin{mathpar}
   x?(\vec{y}).P := x.(\vec{y})P \and  x\clift{\vec{P}} := x.\clift{\vec{P}}
   \and x!(y) := \lift{x}{\dropn{y}}
   \and \Pi_{i=0}^{n-1}P_i := P_0 | \ldots | P_{n-1}
\end{mathpar}

\subsubsection{Structural congruence}

\paragraph{Free and bound names and alpha-equivalence.} At the
core of structural equivalence is alpha-equivalence which identifies
process that are the same up to a change of variable. Formally, we
recognize the distinction between free and bound names. The free names
of a process, $\freenames{P}$, may be calculated recursively as
follows:

\begin{mathpar}
\freenames{\pzero} := \emptyset
  \and \\
  \freenames{x?(y).P} := \{ x \} \cup (\freenames{P} \setminus \{ y \})
  \and 
  \freenames{x!\langle P \rangle} := \{ x \} \cup \{ P \} 
  \and \\
  \freenames{P|Q} := \freenames{P} \cup \freenames{Q}
  \and \\
  \freenames{@{x}} := \{ x \}
\end{mathpar}

$\pi$
$\quotep{\pi}$

$\freenames{-} : \pi \to \mathcal{P}(\quotep{\pi})$

\begin{eqnarray*}
  \freenames{\pzero} & := & \emptyset \\
  \freenames{x?(y).P} & := & \{ x \} \cup (\freenames{P} \setminus \{ y \}) \\
  \freenames{x!\langle P \rangle} & := & \{ x \} \cup \{ P \} \\
  \freenames{P|Q} & := & \freenames{P} \cup \freenames{Q} \\
  \freenames{\dropn{x}} & := & \{ x \}
\end{eqnarray*}

The bound names of a process, $\boundnames{P}$, are those names occurring in $P$
that are not free. For example, in $x?(y).0$, the name $x$ is free, while $y$ is bound.

\begin{mathpar}
  \inferrule* [lab=monoidal-laws] {} { P|Q \equiv Q|P \and P|0 \equiv P \and P|(Q|R) \equiv (P|Q)|R }
\end{mathpar}

\begin{mathpar}
  \inferrule* [lab=alpha-equivalence] {} { (x)P \equiv (y)P\{y/x\} \and y \not\in \freenames{P} }
\end{mathpar}

\begin{definition}
Then two processes, $P,Q$, are alpha-equivalent if $P = Q\{\vec{y}/\vec{x}\}$ for
some $\vec{x} \in \boundnames{Q},\vec{y} \in \boundnames{P}$, where $Q\{\vec{y}/\vec{x}\}$
denotes the capture-avoiding substitution of $\vec{y}$ for $\vec{x}$ in $Q$.
\end{definition}

\begin{definition}
  The {\em structural congruence} \cite{SangiorgiWalker} , $\equiv$,
  between processes is the least congruence containing
  alpha-equivalence, satisfying the abelian monoid laws
  (associativity, commutativity and $\pzero$ as identity) for parallel
  composition $|$ and for summation $+$.
\end{definition}

\subsection{Name equivalence}

We take name equivalence, written $\nameeq$, to be the smallest
equivalence relation generated by the following rules.

\begin{mathpar}
\inferrule*[lab=Quote-drop]
{ }
{ \quotep{@{x}} \nameeq x }

\inferrule*[lab=Struct-equiv]
{ P \scong Q }
{ \quotep{P} \nameeq \quotep{Q} }
\end{mathpar}

The astute reader will have noticed that the mutual recursion of names
and processes imposes a mutual recursion on alpha-equivalence and
structural equivalence via name-equivalence. Fortunately, all of this
works out pleasantly and we may calculate in the natural way, free of
concern. The reader interested in the details is referred to the
appendix \ref{appendix:rho_details}.

\subsection{Substitution}

We use $\Proc$ for the set of processes, $\QProc$ for the set of
names, and $\id{\{}\vec{y} / \vec{x} \id{\}}$ to denote partial maps,
$s : \QProc \rightarrow \QProc$. A map, $s$ lifts, uniquely, to a map
on process terms, $\widehat{s} : \Proc \rightarrow \Proc$ by the
following equations.

\begin{mathpar}
  (0) \psubstp{Q}{P} := 0 \\
  (R \juxtap S) \psubstp{Q}{P}
  :=    
  (R)\psubstp{Q}{P} \juxtap (S) \psubstp{Q}{P} \\
  (x?(y).R) \psubstp{Q}{P}    
  :=    
  (x)\substp{Q}{P} (z)\concat( (R \psubstn{z}{y}) \psubstp{Q}{P} ) \\
  (\lift{x}{R}) \psubstp{Q}{P}  
  :=
  \lift{(x)\substp{Q}{P}}{ R \psubstp{Q}{P} } \\
%   (\dropn{x})  \psubstp{Q}{P}       
%   := 
%   \left\{ 
%     \begin{array}{ccc} 
%       \dropn{\quotep{Q}} & & x \nameeq \quotep{P} \\
%       \dropn{x} & & otherwise \\
%     \end{array}
%   \right. 
  (\dropn{x})  \psubstp{Q}{P}       
  := 
  \left\{ 
    \begin{array}{ccc} 
      Q & & x \nameeq \quotep{P} \\
      \dropn{x} & & otherwise \\
    \end{array}
  \right.
\end{mathpar}
 

where

\begin{eqnarray}
  (x)\id{\{} \lpquote Q \rpquote / \lpquote P \rpquote \id{\}}            = 
  \left\{ 
    \begin{array}{ccc}
      \lpquote Q \rpquote & & x \nameeq \lpquote P \rpquote \\
      x & & otherwise \\
    \end{array}
  \right. \nonumber
\end{eqnarray}

and $z$ is chosen distinct from $\quotep{P}$, $\quotep{Q}$, the free
names in $Q$, and all the names in $R$. Our $\alpha$-equivalence will
be built in the standard way from this substitution.

\begin{remark}\label{rem:no_self_referential_names}
  One consequence of these definitions is that $\forall P. \quotep{P}
  \not\in \freenames{P}$.
\end{remark}

\subsection{ Dynamic quote: an example }

Anticipating something of what's to come, consider applying the
substitution, $\widehat{\id{\{}u / z \id{\}}}$, to the following pair
of processes, $\lift{w}{y!(z)}$ and $w[ \lpquote y!(z) \rpquote ]$.

\begin{eqnarray}
	\lift{w}{y!(z)}\widehat{\id{\{}u / z \id{\}}}
		& = &
		\lift{w}{y!(u)} \nonumber\\
	w[ \lpquote y!(z) \rpquote ] \widehat{ \id{\{}u / z \id{\}} }
		& = &
		w[ \lpquote y!(z) \rpquote ] \nonumber
\end{eqnarray}

Because the body of the process between quotes is impervious to
substitution, we get radically different answers. In fact, by
examining the first process in an input context,
e.g. $x?(z).\lift{w}{y!(z)}$, we see that the process under the lift
operator may be shaped by prefixed inputs binding a name inside it. In
this sense, the lift operator will be seen as a way to dynamically
construct processes before reifying them as names.

Finally equipped with these standard features we can present the
dynamics of the calculus.

\subsubsection{Operational semantics} 

Finally, we introduce the computational dynamics. What marks these
algebras as distinct from other more traditionally studied algebraic
structures, e.g. vector spaces or polynomial rings, is the manner in
which dynamics is captured. In traditional structures, dynamics is typically
expressed through morphisms between such structures, as in linear maps
between vector spaces or morphisms between rings. In algebras
associated with the semantics of computation, the dynamics is
expressed as part of the algebraic structure itself, through a
reduction reduction relation typically denoted by $\red$. Below, we
give a recursive presentation of this relation for the calculus used
in the encoding.

$\red \subseteq \pi \times \pi$
$\red : \pi \to \mathcal{P}(\pi)$

\begin{mathpar}
  \inferrule* [lab=Comm] { \textsf{match}( x_{src}, x_{trgt} ) } { x_{trgt}?(y)P \; | \; x_{src}!\langle {Q} \rangle \red P\{\quotep{Q}/y}\} }
  \and \\
  \inferrule* [lab=Par] {{P} \red {P}'} {{{P} | {Q}} \red {{P}' | {Q}}}
  \and
  \inferrule* [lab=Equiv]{{{P} \scong {P}'} \andalso {{P}' \red {Q}'} \andalso {{Q}' \scong {Q}}}{{P} \red {Q}}
\end{mathpar}

\begin{eqnarray*}
  match_{\equiv} (\quotep{P},\quotep{Q}) & := & P \equiv Q \\
  match_{\dagger}(\quotep{P},\quotep{Q}) & := & \forall R. P|Q \red^{*} R => R \red^{*} 0 \\
  match_{K}(\quotep{P},\quotep{Q}) & := & K \mbox{ for some context } K
\end{eqnarray*}

$u?(x)P | u!\langle Q \rangle \red P\{\quotep{Q}/x\}$

%We write $\wred$ for $\red^*$, and $P\red$ if $\exists Q $ such that $ P \red Q$.
We write $P\red$ if $\exists Q $ such that $ P \red Q$ and $P\not\red$, otherwise.

\section{Replication}

As mentioned before, it is known that replication (and hence
recursion) can be implemented in a higher-order process algebra
\cite{SangiorgiWalker}. As our first example of calculation with the
machinery thus far presented we give the construction explicitly in
the {\rhoc}.

\begin{eqnarray}
	D_{x} & := & \prefix{x}{y}{(\binpar{\outputp{x}{y}}{@{y}})} \nonumber\\
	\bangp_{x}{P} & := & \binpar{{x}!\langle{\binpar{D_{x}}{P}}\rangle}{D_{x}} \nonumber
\end{eqnarray}

\begin{eqnarray}
	\bangp_{x}{P} & & \nonumber\\
	=
	& {x}!\langle{(\prefix{x}{y}{(\outputp{x}{y} | @{y})) | P}}\rangle 
	      | \prefix{x}{y}{(\outputp{x}{y} | @{y})} & \nonumber\\
	\red
	& (\outputp{x}{y} | @{y})\substn{\quotep{(\prefix{x}{y}{(@{y} | \outputp{x}{y})) | P}}}{y} & \nonumber\\
	=
	& \outputp{x}{\quotep{(\prefix{x}{y}{(\outputp{x}{y} | @{y})) | P}}}
	  | {(\prefix{x}{y}{(\outputp{x}{y} | @{y})) | P}} & \nonumber\\
	\red
	& \ldots & \nonumber\\
	\red^*
	& P | P | \ldots & \nonumber
\end{eqnarray}

Of course, this encoding, as an implementation, runs away, unfolding
$\bangp{P}$ eagerly. A lazier and more implementable replication
operator, restricted to input-guarded processes, may be obtained as follows.

\begin{eqnarray}
\bangp{\prefix{u}{v}{P}} 
	:= 
	\binpar{\lift{x}{\prefix{u}{v}{(\binpar{D(x)}{P})}}}{D(x)} \nonumber
\end{eqnarray}

\begin{remark}
  Note that the lazier definition still does not deal with summation
  or mixed summation (i.e. sums over input and output). The reader is
  invited to construct definitions of replication that deal with these
  features. 

  Further, the definitions are parameterized in a name, $x$. Can you,
  gentle reader, make a definition that eliminates this parameter and
  guarantees no accidental interaction between the replication
  machinery and the process being replicated -- i.e. no accidental
  sharing of names used by the process to get its work done and the
  name(s) used by the replication to effect copying. This latter
  revision of the definition of replication is crucial to obtaining
  the expected identity $!!P \sim !P$.
\end{remark}

\begin{remark}\label{rem:paradoxical_combinator}
  The reader familiar with the lambda calculus will have noticed the
  similarity between $D$ and the paradoxical combinator.

  [Ed. note: the existence of this seems to suggest we have to be more
  restrictive on the set of processes and names we admit if we are to
  support no-cloning.]
\end{remark}

\subsubsection{Bisimulation}

The computational dynamics gives rise to another kind of equivalence,
the equivalence of computational behavior. As previously mentioned
this is typically captured \emph{via} some form of bisimulation.

% The notion we use in this paper is weak barbed bisimulation
% \cite{milner91polyadicpi}.

The notion we use in this paper is derived from weak barbed
bisimulation \cite{milner91polyadicpi}. 

\begin{definition}
An \emph{observation relation}, $\downarrow_{\mathcal N}$, over a set
of names, $\mathcal N$, is the smallest relation satisfying the rules
below.

\infrule[Out-barb]{y \in {\mathcal N}, \; x \nameeq y}
		  {\outputp{x}{v} \downarrow_{\mathcal N} x}
\infrule[Par-barb]{\mbox{$P\downarrow_{\mathcal N} x$ or $Q\downarrow_{\mathcal N} x$}}
		  {\binpar{P}{Q} \downarrow_{\mathcal N} x}

We write $P \Downarrow_{\mathcal N} x$ if there is $Q$ such that 
$P \wred Q$ and $Q \downarrow_{\mathcal N} x$.
\end{definition}

\begin{definition}
%\label{def.bbisim}
An  ${\mathcal N}$-\emph{barbed bisimulation} over a set of names, ${\mathcal N}$, is a symmetric binary relation 
${\mathcal S}_{\mathcal N}$ between agents such that $P\rel{S}_{\mathcal N}Q$ implies:
\begin{enumerate}
\item If $P \red P'$ then $Q \wred Q'$ and $P'\rel{S}_{\mathcal N} Q'$.
\item If $P\downarrow_{\mathcal N} x$, then $Q\Downarrow_{\mathcal N} x$.
\end{enumerate}
$P$ is ${\mathcal N}$-barbed bisimilar to $Q$, written
$P \wbbisim_{\mathcal N} Q$, if $P \rel{S}_{\mathcal N} Q$ for some ${\mathcal N}$-barbed bisimulation ${\mathcal S}_{\mathcal N}$.
\end{definition}

$\mathcal{R} \subseteq \pi \times \pi$

$P \mathcal{R} Q => \forall P'. P \red P' \Rightarrow \exists Q'. Q \red Q', P' \mathcal{R} Q'$

$P \vdash x \Rightarrow Q \vdash x$

\begin{mathpar}
  \inferrule*[lab=Out-barb]{x \nameeq y}{{y}!\langle{Q}\rangle \vdash x}
  \and
  \inferrule*[lab=Par-barb]{\mbox{$P\vdash x$ or $Q\vdash x$}}{\binpar{P}{Q} \vdash x}
\end{mathpar}

\subsubsection{Contexts}

One of the principle advantages of computational calculi like the
$\pi$-calculus is a well-defined notion of context,
contextual-equivalence and a correlation between
contextual-equivalence and notions of bisimulation. The notion of
context allows the decomposition of a process into (sub-)process and
its syntactic environment, its context. Thus, a context may be
thought of as a process with a ``hole'' (written $\Box$) in it. The
application of a context $M$ to a process $P$, written $M[P]$, is
tantamount to filling the hole in $M$ with $P$. In this paper we do
not need the full weight of this theory, but do make use of the notion
of context in the proof the main theorem. 

\begin{mathpar}
  \inferrule* [lab=summation] {} {{M_{M},M_{N}} \bc \Box \;|\; x.M_{A} \;|\; M_{M}+M_{N}}
  \and
  \inferrule* [lab=agent] {} {{M_{A}} \bc (\vec{x})M_{P} \;| \; \clift{P_0,\ldots,M_{P},\ldots,P_N}}
  \and \\
  \inferrule* [lab=process] {} {{M_{P}} \bc M_{N} \;| \;P|M_{P} }
\end{mathpar} 

\begin{mathpar}
  \inferrule* [lab=sychronization] {} {M_{N} \bc \Box \;|\; x?M_{F} \;|\; x!M_{C}}
  \and
  \inferrule* [lab=abstraction] {} {{M_{F}} \bc (x)M_{P} }
  \and
  \inferrule* [lab=concretion] {} {{M_{C}} \bc \langle M_{P} \rangle }
  \and \\
  \inferrule* [lab=process] {} {{M_{P}} \bc M_{N} \;| \;P|M_{P} }
\end{mathpar}

\begin{definition}[contextual application] Given a context $M$, and
  process $P$, we define the \emph{contextual application}, $M[P] :=
  M\{P/\Box\}$. That is, the contextual application of M to P is the
  substitution of $P$ for $\Box$ in $M$.
\end{definition}

$\meaningof{-} : L \to \mathcal{P}(\pi)$

\begin{mathpar}
  \inferrule* [lab=collection] {} {\meaningof{true} = \pi, \and \meaningof{~E} = \pi \setminus \meaningof{E}, \and \meaningof{E_{1} \& E_{2}} = \meaningof{E_{1}} \cap \meaningof{E_{2}}}
\end{mathpar}

\begin{mathpar}
  \inferrule* [lab=structure] {} {\meaningof{0} = \{ P \in \pi | P \equiv 0 \}, \and \\ \meaningof{E_1 | E_2} = \{ P \in \pi | P \equiv P_{1} | P_{2}, P_{1} \in \meaningof{E_{1}}, P_{2} \in \meaningof{E_2}\} }
\end{mathpar}

\begin{mathpar}
 \inferrule* [lab=behavior] {} {\meaningof{\langle a?b \rangle E} = \{ P \in \pi | P \equiv Q | u?(y)P', \\ \and \\\\ \and \\ \;\;\; u \in \meaningof{a}, \forall z.P'\{z/y\} \in \meaningof{E\{z/b\}}\}, \and \\ \meaningof{a!E} = \{ P \in \pi | P \equiv Q | x!\langle P' \rangle, x \in \meaningof{a} P' \in \meaningof{E}\} }
\end{mathpar}

\begin{mathpar}
 \inferrule* [lab=nominal] {} {\meaningof{\quotep{E}} = \{ \quotep{P} \in \quotep{\pi} | P \in \meaningof{E} \}, \and \meaningof{\quotep{P}} = \{ \quotep{Q} \in \quotep{\pi} | P \equiv Q \} \and \\ \meaningof{@\quotep{E}} = \{ P \in \pi | P \equiv @x, x \in \meaningof{E} \}}
\end{mathpar}

\begin{eqnarray*}
  \\
  \meaningof{-} : TS \to ST
\end{eqnarray*}

\begin{eqnarray*}
  \\
  L : TS \to ST
\end{eqnarray*}

\begin{eqnarray*}
  \\
  P \models E \iff P \in \meaningof{E}
\end{eqnarray*}

\begin{eqnarray*}
  P \approx_{L} Q \iff \forall E \in L. P \models E \iff Q \models E
\end{eqnarray*}

\begin{eqnarray*}
  P \approx_{K} Q
\end{eqnarray*}

\begin{eqnarray*}
  P \approx Q
\end{eqnarray*}

$\approx_{K} = \approx = \approx_{L}$

\subsubsection{Contextual duality}

Note that contexts extend the quotation operation to a family of
operations from processes to names. Given a context, $M$, we can
define a \emph{nominal context}, $\quotep{M}$ by $\quotep{M}[P] :=
\quotep{M[P]}$. To foreshadow what is to come we observe that these
operations enjoy a duality with processes very much like the duality
between vectors and maps from vectors to scalars.

Further, because the calculus is essentially higher-order, we have a
correspondence between contexts and processes. More specifically,
given a name $x$ and a context $M$ we can construct $M^{*}_{x}$ such
that 

\begin{mathpar}
  M^{*}_{x} | \lift{x}{P} \red M[P]
\end{mathpar}

namely,

\begin{mathpar}
  M^{*}_{x} := x?(u).M[\dropn{u}]
\end{mathpar}

The dependence of $M^{*}_{x}$ on a name makes it an abstraction, 

\begin{mathpar}
  M^{*} := (x)x?(u).M[\dropn{u}]
\end{mathpar}

\subsection{Additional notation}

It will sometimes be convenient to denote the process a name
quotes. We already have the notation $x = \quotep{P}$, but it will be
convenient to introduce an alternate notation, $\procn{x}$, when we
want to emphasize the connection to the use of the name. Note that, by
virtue of name equivalence, $\quotep{\procn{x}} \nameeq x$; so, the
notation is consistent with previous definitions.

Further, because names have structure it is possible to effect
substitutions on the basis of that structure. This means we need to
upgrade our notation for substitutions, which we accomplish by
adapting comprehension notation. Thus,

\begin{mathpar}
  P\{ y / x : x \in S \}
\end{mathpar}

is interpreted to mean the process derived from P by replacing (in a
capture-avoiding manner) each occurrence of $x$ in $S$ by $y$. For example,

\begin{mathpar}
  P\{ \quotep{\procn{x}|\procn{x}} / x : x \in \freenames{P} \}
\end{mathpar}

will replace each (occurrence) of a free name $x$ in $P$ by
$\quotep{\procn{x}|\procn{x}}$.

Also, we will avail ourselves of the notation $x^{L}$ and $x^{R}$ to
denote injections of a name into disjoint copies of the name
space. There are numerous ways to accomplish this. One example can be
found in \cite{MeredithR05}. This notation overloads to vectors of
names: $\vec{x}^{\pi} := (x_{i}^{\pi} \; : \; 0 \leq i < |\vec{x}| )$ where $\pi \in \{L,R\}$.

We also use $P^{\Box} := P|\Box$.

In \cite{MeredithR05} an interpretation of the new operator is
given. It turns out that there are several possible interpretations
all enjoying the requisite algebraic properties of the operator (see
\cite{milner91polyadicpi}). We will therefore make liberal use of
$(\nu\; \vec{x})P$.

% subsection the_syntax_and_semantics_of_the_notation_system (end)   

\section{Interpretation of QM}
\subsection{Supporting definitions}
\subsubsection{Multiplication}
\begin{mathpar}
  \quotep{Q} \cdot \quotep{R} := \quotep{Q|R}
  \and \\
  \quotep{Q} \cdot P := P\{ \quotep{Q|R} / \quotep{R} : \quotep{R} \in \freenames{P} \}
\end{mathpar}

\paragraph{Discussion}
The first line needs little explanation. The second line says that
each free name of the process is replaced with the multiplication of
that name by the scalar. Multiplication of a scalar (name) by a state
(process) results in a process all the names of which have been `moved
over' by parallel composition with the process the scalar
quotes. There is a subtlety that the bound names have to be
manipulated so that multiplied names aren't accidentally
captured. There are many ways to achieve this.

\begin{remark}\label{rem:multiplication_identities}
  The reader is invited to verify that for all $x,y,z \in \QProc$ and $P \in \Proc$
  \begin{mathpar}
    x \cdot \quotep{0} \equiv x 
    \and
    x \cdot y \equiv y \cdot x
    \and
    x \cdot (y \cdot z) \equiv (x \cdot y) \cdot z
    \and \\
    \quotep{0} \cdot P \equiv P
    \and \\
    x \cdot (y \cdot P) \equiv (x \cdot y) \cdot P
    \and \\
    x \cdot (P|Q) \equiv (x \cdot P) | (x \cdot Q)
    \and \\    
  \end{mathpar}
\end{remark}

\subsubsection{Tensor product}

We define a tensor product on processes by structural induction.

\paragraph{Tensor of sums} First note that all summations, including
$\pzero$ and sequence, can be written $\Sigma_{i} x_{i}.A_{i} +
\Sigma_{j} x_{j}.C_{j}$, where we have grouped input-guarded processes
together and output-guarded processes together.

Thus, we can define the tensor product of two summations, $N_{1}\otimes N_{2}$, where

\begin{mathpar}
  N_{1} := \Sigma_{i} x_{i}.A_{i} + \Sigma_{j} x_{j}.C_{j}
  \and
  N_{2} := \Sigma_{i'} y_{i'}.B_{i'} + \Sigma_{j'} y_{j'}.D_{j'} 
\end{mathpar}

as follows.

\begin{mathpar}
  \Sigma_{i} x_{i}.A_{i} + \Sigma_{j} x_{j}.C_{j} \otimes \Sigma_{i'}
  y_{i'}.B_{i'} + \Sigma_{j'} y_{j'}.D_{j'} 
  \and \\
  := \; \Sigma_{i} \Sigma_{i'} \quotep{\stackrel{\vee}{x_{i}}| \stackrel{\vee}{y_{i'}}}.(A_{i}\otimes B_{i'}) \; | \; \Sigma_{i'} \Sigma_{i} \quotep{\stackrel{\vee}{y_{i'}}|\stackrel{\vee}{x_{i}}}.(B_{i'}\otimes A_{i})
  \and
  \;\; | \;\; \Sigma_{j} \Sigma_{j'} \quotep{\stackrel{\vee}{x_{j}}|\stackrel{\vee}{y_{j'}}}.(A_{j}\otimes B_{j'}) \; | \; \Sigma_{j'} \Sigma_{j} \quotep{\stackrel{\vee}{y_{j'}}|\stackrel{\vee}{x_{j}}}.(B_{j'}\otimes A_{j})
\end{mathpar}

\begin{remark}
  Do we need to $x^{L}$ and $y^{R}$ for this construction as well?
\end{remark}

\paragraph{Tensor of parallel compositions} Next, we distribute tensor
over par.

\begin{mathpar}
  P_{1}|P_{2} \otimes Q_{1}|Q_{2} := (P_{1} \otimes Q_{1}) | (P_{1}
  \otimes Q_{2}) | (P_{2} \otimes Q_{1}) | (P_{2} \otimes Q_{2})
\end{mathpar}

\paragraph{Tensor with dropped names} We treat tensor of a
process with a dropped name as parallel composition.

\begin{mathpar}
  P \otimes \dropn{x} := P | \dropn{x}
\end{mathpar}

\paragraph{Tensor of agents}

Finally, we need to define tensor on agents. Note that the definition
of tensor on normal products only tensors inputs with inputs and
outputs with outputs. Thus, we only have to define the operation on
``homogeneous'' pairings.

\begin{mathpar}
  (\vec{x})P \otimes (\vec{y})Q
  \and \\
  := (x_{0}^{L}|y_{0}^{R},\ldots,x_{0}^{L}|y_{n}^{R},\ldots,x_{m}^{L}|y_{0}^{R},\ldots,x_{m}^{L}|y_{n}^R)(P\{ \vec{x}^{L}/\vec{x}\} \otimes Q \{ \vec{y}^{R}/\vec{y}\})
  \and \\
  \clift{\vec{P}} \otimes \clift{\vec{Q}}
  \and \\
  := \clift{P_{0}\otimes Q_{0},\ldots,P_{0}\otimes Q_{n},\ldots,P_{m}\otimes Q_{0},\ldots,P_{m}\otimes Q_{n}}
\end{mathpar}

\begin{remark}
  Observe that arities of tensored abstractions matches arities of
  tensored concretions if the original arities matched. Note also that
  the length of the arities corresponds to the increase in dimension
  we see in ordinary vector space tensor product.
\end{remark}

\begin{remark}
  Operationally, this definition distributes the tensor down to
  components ``linked'' by summation. Tensor over summation is
  intriguing in that it mixes names. Moreover, as a consequence of the
  way it mixes names we have the identities for all $x \in \QProc$ and
  $P,Q \in \Proc$

  \begin{mathpar}
    (x \cdot P) \otimes Q \equiv x \cdot (P \otimes Q) \equiv P \otimes (x \cdot Q)
    \and
    P \otimes \pzero \equiv P
  \end{mathpar}

  that the reader is invited to verify.
\end{remark}

\subsubsection{Annihilation}
\begin{mathpar}
  P^{\perp} := \{ Q | \forall R. P|Q \red^{*} R \Rightarrow R \red^{*} \pzero \}
  \and \\
  P^{\underline{\perp}} := \Sigma_{Q \in P^{\perp}} \quotep{Q}?(y).(\dropn{y}|Q) | \Sigma_{Q \in P^{\perp}} \quotep{Q}\clift{\Box}
\end{mathpar}

\paragraph{Discussion} The reader will note that $P^{\perp}$ is a
\emph{set} of processes, while $P^{\underline{\perp}}$ is a
\emph{context}. We call the set $P^{\perp}$ the \emph{annihilators} of
$P$. The parallel composition of a process in the annihilators of $P$
with $P$ will result in a process, the state space of which has all
paths eventually leading to $\pzero$. Execution may endure loops; but
under reasonable conditions of fairness (naturally guaranteed under
most notions of bisimulation) such a composite process cannot get
stuck in such a loop and will, eventually pop out and terminate.

The context $P^{\underline{\perp}}$ is ready and willing to ``take the
$P$ out of'' the process to which it is applied. It will effectively
transmit the code of the process to which it is applied to one of the
annihilators and run the process against it.

\subsubsection{Evaluation}
We fix $M$ a domain of fully abstract interpretation with an equality
coincident with bisimulation. We take $\meaningof{\cdot} : \Proc \to
M$ to be the map interpreting processes and $\nmeaningof{\cdot} : \M
\to Proc$ to be the map running the other way. Then we define

\begin{mathpar}
  \int P := \nmeaningof{\meaningof{P}}
\end{mathpar}

\paragraph{Discussion}
There are many fully abstract interpretations of Milner's
$\pi$-calculus. Any of them can be used as a basis for interpreting
the reflective calculus here. Equipped with such a domain it is
largely a matter of grinding through to check that the Yoneda
construction for the normalization-by-evaluation program can be
extended to this setting.

\begin{remark}
  The reader is invited to verify that $\int (P^{\underline{\perp}}[P]) = 0$.
\end{remark}

\subsection{Quantum mechanics}

Table \ref{tbl:core_qm_op_defns} gives the core operational definitions

\begin{table}[htp]\label{tbl:core_qm_op_defns}
  \center{
    \fbox{
      \begin{tabular}{c|c}
        quantum mechanics & process calculus \\
        \hline
        scalar & $x := \quotep{P}$ \\
        state vector & $\state{P} := P$ \\
        dual & $\state{P}^{*} := \event{P^{\underline{\perp}}} := \quotep{P^{\underline{\perp}}}[-]$ \\
        matrix & $ \Sigma_{\alpha} \state{P_{\alpha}}x_{\alpha}\event{Q_{\alpha}}$ \\
        vector addition & $\state{P} + \state{Q} := \state{P | Q}$ \\
        tensor product & $\state{P} \otimes \state{Q} := \state{P \otimes Q}$ \\
        inner product & $\innerprod{P}{Q} := \quotep{\int P^{\underline{\perp}}[Q]}$ \\
      \end{tabular}
    }
  }
  \caption{QM - operational definitions}
\end{table}

where

\begin{mathpar}
  \prmatrix{P}{Q} := \fprmatrix{P}{\quotep{\pzero}}{Q}
  \and
  \fprmatrix{P}{x}{Q} := (\state{P},x,\event{Q})
  \and
  (\fprmatrix{P}{x}{Q})(\state{R}) := x \cdot \innerprod{Q}{R} \cdot \state{P}
  \and
  (\fprmatrix{P}{x}{Q})(\event{R}) := x \cdot \innerprod{R}{P} \cdot \event{Q}
\end{mathpar}

\paragraph{Discussion}
As promised: vectors (aka states) are represented as processes; duals
as contextual duals; inner product definition should be compared with
standard inner product definition for ....

\begin{remark}
  Assuming $\int (P^{\underline{\perp}}[P]) = 0$, the reader is
  invited to verify that $(\fprmatrix{P}{x}{P})(\state{P}) = x \cdot \state{P}$.
\end{remark}

\begin{remark}
  The reader is invited to verify that $\innerprod{P}{Q}$ could
  equally well have been written $\quotep{\int \stackrel{\vee}{x}}$
  where $x = \event{P^{\underline{\perp}}}(Q)$.

  One of the motivations for this remark is that there is another way
  to factor these operations. We could package up evaluation in the dual:

  \begin{mathpar}
    \state{P}^{*} := \event{\int P^{\underline{\perp}}} := \quotep{\int P^{\underline{\perp}}}[-]
  \end{mathpar}

  and then have inner product defined by
  
  \begin{mathpar}
    \innerprod{P}{Q} := \event{P}(Q)
  \end{mathpar}

  Hopefully, experience with the calculations will provide guidance on
  the best factoring.
\end{remark}

\begin{remark}
  Assuming $\int (P^{\underline{\perp}}[P]) = 0$, the reader is
  invited to verify that $\forall P,Q. (\prmatrix{0}{Q})(\state{0}) =
  \state{0}$ and dually $(\prmatrix{P}{0})(\event{0}) = \event{0}$.
\end{remark}

\begin{remark}
  i'm a little worried that i don't (yet) have proper support for
  complex conjugacy. But, the observation above may give us a
  clue. According to Abramsky, it must be the case that the scalars
  are iso to the homset of the identity for the tensor -- which the
  observation above characterizes. 

  For now, we will simply bookmark the notion with $\overline{x}$.
\end{remark}

\subsubsection{Adjointness}

We need to give a definition of $(\cdot)^{\dagger}$ for matrices. The
obvious candidate definition is
\begin{mathpar}
(\Sigma_{\alpha}\fprmatrix{P_{\alpha}}{x_{\alpha}}{Q_{\alpha}})^{\dagger}
= \Sigma_{\alpha}\fprmatrix{(Q_{\alpha}^{\underline{\perp}})^{*}}{\overline{x}_{\alpha}}{P_{\alpha}^{\underline{\perp}}} 
\end{mathpar}

But, $(Q_{\alpha}^{\underline{\perp}})^{*}$ requires a name along
which to communicate the process to achieve the context application.

\subsubsection{Basis for a basis}
If processes label states and ``addition'' of states (a.k.a. vector
addition) is interpreted as parallel composition, what corresponds to
notions of linear independence and basis? Here, we recall that Yoshida
has developed a set of \emph{combinators} for an asynchronous verison
of Milner's $\pi$-calculus. These are a finite set of processes such
any process can be expressed as parallel composition of these
combinators together with liberal uses of the new operator and
replication. We can simply give a translation of these into the
present calculus and have reasonable expectation that the property
carries over. That is, that the resultant set allows to express all
processes via parallel composition. Note, however, that there is no
new operator or replication in this calculus. As a result, we expect
that the corresponding set is actually infinite. That is, we expect
that the space is actually infinite dimensional.

\begin{remark}
  The attentive reader may be a bit concerned. Certainly, the
  collection $S$, $K$ and $I$ is a finite set of
  combinators. Shouldn't we expect to see a finite set of combinators
  for an effectively equivalent system? i am very sympathetic to this
  critique and feel it warrants full attention. On the other hand, i
  also have in mind the following analogy. The natural numbers, as a
  monoid under addition, has exactly $1$ generator, while the natural
  numbers, as a monoid under multiplication, has countably many
  generators (the primes). We observe that the application of the
  lambda calculus is much less resource sensitive than the parallel
  composition of the $\pi$-calculus. Could it be the case that we have
  an analogy of the form
  
  \begin{mathpar}
    m + n : MN :: m*n : M|N
  \end{mathpar}

  giving a similar blow up in the set of ``primes''?  This is such a
  wonderful thought that, even if it's not true, i think it's worth
  writing down.
\end{remark}
 

\documentclass[12pt]{llncs}
%\documentclass{jktr}

\usepackage[pdftex]{hyperref}                   
\usepackage {listings}
\usepackage {mathpartir}
\usepackage{bcprules}
%\usepackage{listings}
                       
\usepackage{graphicx} 
%\usepackage[margins=2.5cm,nohead,nofoot]{geometry}
%\usepackage{geometry}
\usepackage{amsfonts}
\usepackage{amstext}
\usepackage{latexsym}
\usepackage{amssymb}
\usepackage{color}


%\include{myPreamble}
\include{qm2pi.local} 

%\ifpdf
%\usepackage[pdftex]{graphicx}
%\else
%\usepackage{graphicx}
%\fi

 % \ifpdf
%  \usepackage{pdfsync}
%  \if


%\title{Brief Article}
%\author{David F. Snyder}
%\author{L.G. Meredith}

%\address{Dept. of Math., Texas State University--San Marcos, San Marcos, TX 78666}
       
\pagestyle{empty}


\begin{document}

\lstset{language=[Objective]Caml,frame=shadowbox}

\input{qm2pi.front}

% section front matter (end)

\input{qm2pi.intro} 
 
% section introduction (end)

% \input{qm2pi.knotations} 

% section notation (end)

\input{qm2pi.process.calculi} 

% section concurrent_process_calculi_and_spatial_logics_ (end)
    
%\input{qm2pi.knots2pi} 

%\input{qm2pi.trefoil} 

%\input{qm2pi.mainthm} 

% subsection basic_interpretation (end)

%\input{qm2pi.rho.presentation} 
\subsection{The syntax and semantics of the notation system}\label{sub:the_syntax_and_semantics_of_the_notation_system} % (fold)

We now summarize a technical presentation of the calculus that
embodies our theory of dynamics. The typical presentation of such a
calculus follows the style of giving generators and relations on
them. The grammar, below, describing term constructors, freely
generates the set of processes, $\Proc$. This set is then quotiented
by a relation known as structural congruence and it is over this set
that the notion of dynamics is expressed. This presentation is
essentially that of \cite{MeredithR05} with the addition of
polyadicity and summation. For readability we have relegated some of
the technical subtleties to an appendix.

\subsubsection{Process grammar}\label{subsub:process_grammar}

\begin{mathpar}
  \inferrule* [lab=synchronization] {} {{M} \bc \pzero \;|\; x?F \;|\; x!C }
  \and
  \inferrule* [lab=abstraction] {} {{F} \bc (x)P}
  \and
  \inferrule* [lab=concretion] {} {{C} \bc \langle Q \rangle}
  \and
  \inferrule* [lab=process] {} {{P,Q} \bc M \;| \;P|Q \;|\; @{x}}
  \and
  \inferrule* [lab=name] {} {{x} \bc \quotep{P}}
\end{mathpar} 

Note that $\vec{x}$ (resp. $\vec{P}$) denotes a vector of names
(resp. processes) of length $|\vec{x}|$ (resp. $|\vec{P}|$). We adopt
the following useful abbreviations.

\begin{mathpar}
   x?(\vec{y}).P := x.(\vec{y})P \and  x\clift{\vec{P}} := x.\clift{\vec{P}}
   \and x!(y) := \lift{x}{\dropn{y}}
   \and \Pi_{i=0}^{n-1}P_i := P_0 | \ldots | P_{n-1}
\end{mathpar}

\subsubsection{Structural congruence}

\paragraph{Free and bound names and alpha-equivalence.} At the
core of structural equivalence is alpha-equivalence which identifies
process that are the same up to a change of variable. Formally, we
recognize the distinction between free and bound names. The free names
of a process, $\freenames{P}$, may be calculated recursively as
follows:

\begin{mathpar}
\freenames{\pzero} := \emptyset
  \and \\
  \freenames{x?(y).P} := \{ x \} \cup (\freenames{P} \setminus \{ y \})
  \and 
  \freenames{x!\langle P \rangle} := \{ x \} \cup \{ P \} 
  \and \\
  \freenames{P|Q} := \freenames{P} \cup \freenames{Q}
  \and \\
  \freenames{@{x}} := \{ x \}
\end{mathpar}

$\pi$
$\quotep{\pi}$

$\freenames{-} : \pi \to \mathcal{P}(\quotep{\pi})$

\begin{eqnarray*}
  \freenames{\pzero} & := & \emptyset \\
  \freenames{x?(y).P} & := & \{ x \} \cup (\freenames{P} \setminus \{ y \}) \\
  \freenames{x!\langle P \rangle} & := & \{ x \} \cup \{ P \} \\
  \freenames{P|Q} & := & \freenames{P} \cup \freenames{Q} \\
  \freenames{\dropn{x}} & := & \{ x \}
\end{eqnarray*}

The bound names of a process, $\boundnames{P}$, are those names occurring in $P$
that are not free. For example, in $x?(y).0$, the name $x$ is free, while $y$ is bound.

\begin{mathpar}
  \inferrule* [lab=monoidal-laws] {} { P|Q \equiv Q|P \and P|0 \equiv P \and P|(Q|R) \equiv (P|Q)|R }
\end{mathpar}

\begin{mathpar}
  \inferrule* [lab=alpha-equivalence] {} { (x)P \equiv (y)P\{y/x\} \and y \not\in \freenames{P} }
\end{mathpar}

\begin{definition}
Then two processes, $P,Q$, are alpha-equivalent if $P = Q\{\vec{y}/\vec{x}\}$ for
some $\vec{x} \in \boundnames{Q},\vec{y} \in \boundnames{P}$, where $Q\{\vec{y}/\vec{x}\}$
denotes the capture-avoiding substitution of $\vec{y}$ for $\vec{x}$ in $Q$.
\end{definition}

\begin{definition}
  The {\em structural congruence} \cite{SangiorgiWalker} , $\equiv$,
  between processes is the least congruence containing
  alpha-equivalence, satisfying the abelian monoid laws
  (associativity, commutativity and $\pzero$ as identity) for parallel
  composition $|$ and for summation $+$.
\end{definition}

\subsection{Name equivalence}

We take name equivalence, written $\nameeq$, to be the smallest
equivalence relation generated by the following rules.

\begin{mathpar}
\inferrule*[lab=Quote-drop]
{ }
{ \quotep{@{x}} \nameeq x }

\inferrule*[lab=Struct-equiv]
{ P \scong Q }
{ \quotep{P} \nameeq \quotep{Q} }
\end{mathpar}

The astute reader will have noticed that the mutual recursion of names
and processes imposes a mutual recursion on alpha-equivalence and
structural equivalence via name-equivalence. Fortunately, all of this
works out pleasantly and we may calculate in the natural way, free of
concern. The reader interested in the details is referred to the
appendix \ref{appendix:rho_details}.

\subsection{Substitution}

We use $\Proc$ for the set of processes, $\QProc$ for the set of
names, and $\id{\{}\vec{y} / \vec{x} \id{\}}$ to denote partial maps,
$s : \QProc \rightarrow \QProc$. A map, $s$ lifts, uniquely, to a map
on process terms, $\widehat{s} : \Proc \rightarrow \Proc$ by the
following equations.

\begin{mathpar}
  (0) \psubstp{Q}{P} := 0 \\
  (R \juxtap S) \psubstp{Q}{P}
  :=    
  (R)\psubstp{Q}{P} \juxtap (S) \psubstp{Q}{P} \\
  (x?(y).R) \psubstp{Q}{P}    
  :=    
  (x)\substp{Q}{P} (z)\concat( (R \psubstn{z}{y}) \psubstp{Q}{P} ) \\
  (\lift{x}{R}) \psubstp{Q}{P}  
  :=
  \lift{(x)\substp{Q}{P}}{ R \psubstp{Q}{P} } \\
%   (\dropn{x})  \psubstp{Q}{P}       
%   := 
%   \left\{ 
%     \begin{array}{ccc} 
%       \dropn{\quotep{Q}} & & x \nameeq \quotep{P} \\
%       \dropn{x} & & otherwise \\
%     \end{array}
%   \right. 
  (\dropn{x})  \psubstp{Q}{P}       
  := 
  \left\{ 
    \begin{array}{ccc} 
      Q & & x \nameeq \quotep{P} \\
      \dropn{x} & & otherwise \\
    \end{array}
  \right.
\end{mathpar}
 

where

\begin{eqnarray}
  (x)\id{\{} \lpquote Q \rpquote / \lpquote P \rpquote \id{\}}            = 
  \left\{ 
    \begin{array}{ccc}
      \lpquote Q \rpquote & & x \nameeq \lpquote P \rpquote \\
      x & & otherwise \\
    \end{array}
  \right. \nonumber
\end{eqnarray}

and $z$ is chosen distinct from $\quotep{P}$, $\quotep{Q}$, the free
names in $Q$, and all the names in $R$. Our $\alpha$-equivalence will
be built in the standard way from this substitution.

\begin{remark}\label{rem:no_self_referential_names}
  One consequence of these definitions is that $\forall P. \quotep{P}
  \not\in \freenames{P}$.
\end{remark}

\subsection{ Dynamic quote: an example }

Anticipating something of what's to come, consider applying the
substitution, $\widehat{\id{\{}u / z \id{\}}}$, to the following pair
of processes, $\lift{w}{y!(z)}$ and $w[ \lpquote y!(z) \rpquote ]$.

\begin{eqnarray}
	\lift{w}{y!(z)}\widehat{\id{\{}u / z \id{\}}}
		& = &
		\lift{w}{y!(u)} \nonumber\\
	w[ \lpquote y!(z) \rpquote ] \widehat{ \id{\{}u / z \id{\}} }
		& = &
		w[ \lpquote y!(z) \rpquote ] \nonumber
\end{eqnarray}

Because the body of the process between quotes is impervious to
substitution, we get radically different answers. In fact, by
examining the first process in an input context,
e.g. $x?(z).\lift{w}{y!(z)}$, we see that the process under the lift
operator may be shaped by prefixed inputs binding a name inside it. In
this sense, the lift operator will be seen as a way to dynamically
construct processes before reifying them as names.

Finally equipped with these standard features we can present the
dynamics of the calculus.

\subsubsection{Operational semantics} 

Finally, we introduce the computational dynamics. What marks these
algebras as distinct from other more traditionally studied algebraic
structures, e.g. vector spaces or polynomial rings, is the manner in
which dynamics is captured. In traditional structures, dynamics is typically
expressed through morphisms between such structures, as in linear maps
between vector spaces or morphisms between rings. In algebras
associated with the semantics of computation, the dynamics is
expressed as part of the algebraic structure itself, through a
reduction reduction relation typically denoted by $\red$. Below, we
give a recursive presentation of this relation for the calculus used
in the encoding.

$\red \subseteq \pi \times \pi$
$\red : \pi \to \mathcal{P}(\pi)$

\begin{mathpar}
  \inferrule* [lab=Comm] { \textsf{match}( x_{src}, x_{trgt} ) } { x_{trgt}?(y)P \; | \; x_{src}!\langle {Q} \rangle \red P\{\quotep{Q}/y}\} }
  \and \\
  \inferrule* [lab=Par] {{P} \red {P}'} {{{P} | {Q}} \red {{P}' | {Q}}}
  \and
  \inferrule* [lab=Equiv]{{{P} \scong {P}'} \andalso {{P}' \red {Q}'} \andalso {{Q}' \scong {Q}}}{{P} \red {Q}}
\end{mathpar}

\begin{eqnarray*}
  match_{\equiv} (\quotep{P},\quotep{Q}) & := & P \equiv Q \\
  match_{\dagger}(\quotep{P},\quotep{Q}) & := & \forall R. P|Q \red^{*} R => R \red^{*} 0 \\
  match_{K}(\quotep{P},\quotep{Q}) & := & K \mbox{ for some context } K
\end{eqnarray*}

$u?(x)P | u!\langle Q \rangle \red P\{\quotep{Q}/x\}$

%We write $\wred$ for $\red^*$, and $P\red$ if $\exists Q $ such that $ P \red Q$.
We write $P\red$ if $\exists Q $ such that $ P \red Q$ and $P\not\red$, otherwise.

\section{Replication}

As mentioned before, it is known that replication (and hence
recursion) can be implemented in a higher-order process algebra
\cite{SangiorgiWalker}. As our first example of calculation with the
machinery thus far presented we give the construction explicitly in
the {\rhoc}.

\begin{eqnarray}
	D_{x} & := & \prefix{x}{y}{(\binpar{\outputp{x}{y}}{@{y}})} \nonumber\\
	\bangp_{x}{P} & := & \binpar{{x}!\langle{\binpar{D_{x}}{P}}\rangle}{D_{x}} \nonumber
\end{eqnarray}

\begin{eqnarray}
	\bangp_{x}{P} & & \nonumber\\
	=
	& {x}!\langle{(\prefix{x}{y}{(\outputp{x}{y} | @{y})) | P}}\rangle 
	      | \prefix{x}{y}{(\outputp{x}{y} | @{y})} & \nonumber\\
	\red
	& (\outputp{x}{y} | @{y})\substn{\quotep{(\prefix{x}{y}{(@{y} | \outputp{x}{y})) | P}}}{y} & \nonumber\\
	=
	& \outputp{x}{\quotep{(\prefix{x}{y}{(\outputp{x}{y} | @{y})) | P}}}
	  | {(\prefix{x}{y}{(\outputp{x}{y} | @{y})) | P}} & \nonumber\\
	\red
	& \ldots & \nonumber\\
	\red^*
	& P | P | \ldots & \nonumber
\end{eqnarray}

Of course, this encoding, as an implementation, runs away, unfolding
$\bangp{P}$ eagerly. A lazier and more implementable replication
operator, restricted to input-guarded processes, may be obtained as follows.

\begin{eqnarray}
\bangp{\prefix{u}{v}{P}} 
	:= 
	\binpar{\lift{x}{\prefix{u}{v}{(\binpar{D(x)}{P})}}}{D(x)} \nonumber
\end{eqnarray}

\begin{remark}
  Note that the lazier definition still does not deal with summation
  or mixed summation (i.e. sums over input and output). The reader is
  invited to construct definitions of replication that deal with these
  features. 

  Further, the definitions are parameterized in a name, $x$. Can you,
  gentle reader, make a definition that eliminates this parameter and
  guarantees no accidental interaction between the replication
  machinery and the process being replicated -- i.e. no accidental
  sharing of names used by the process to get its work done and the
  name(s) used by the replication to effect copying. This latter
  revision of the definition of replication is crucial to obtaining
  the expected identity $!!P \sim !P$.
\end{remark}

\begin{remark}\label{rem:paradoxical_combinator}
  The reader familiar with the lambda calculus will have noticed the
  similarity between $D$ and the paradoxical combinator.

  [Ed. note: the existence of this seems to suggest we have to be more
  restrictive on the set of processes and names we admit if we are to
  support no-cloning.]
\end{remark}

\subsubsection{Bisimulation}

The computational dynamics gives rise to another kind of equivalence,
the equivalence of computational behavior. As previously mentioned
this is typically captured \emph{via} some form of bisimulation.

% The notion we use in this paper is weak barbed bisimulation
% \cite{milner91polyadicpi}.

The notion we use in this paper is derived from weak barbed
bisimulation \cite{milner91polyadicpi}. 

\begin{definition}
An \emph{observation relation}, $\downarrow_{\mathcal N}$, over a set
of names, $\mathcal N$, is the smallest relation satisfying the rules
below.

\infrule[Out-barb]{y \in {\mathcal N}, \; x \nameeq y}
		  {\outputp{x}{v} \downarrow_{\mathcal N} x}
\infrule[Par-barb]{\mbox{$P\downarrow_{\mathcal N} x$ or $Q\downarrow_{\mathcal N} x$}}
		  {\binpar{P}{Q} \downarrow_{\mathcal N} x}

We write $P \Downarrow_{\mathcal N} x$ if there is $Q$ such that 
$P \wred Q$ and $Q \downarrow_{\mathcal N} x$.
\end{definition}

\begin{definition}
%\label{def.bbisim}
An  ${\mathcal N}$-\emph{barbed bisimulation} over a set of names, ${\mathcal N}$, is a symmetric binary relation 
${\mathcal S}_{\mathcal N}$ between agents such that $P\rel{S}_{\mathcal N}Q$ implies:
\begin{enumerate}
\item If $P \red P'$ then $Q \wred Q'$ and $P'\rel{S}_{\mathcal N} Q'$.
\item If $P\downarrow_{\mathcal N} x$, then $Q\Downarrow_{\mathcal N} x$.
\end{enumerate}
$P$ is ${\mathcal N}$-barbed bisimilar to $Q$, written
$P \wbbisim_{\mathcal N} Q$, if $P \rel{S}_{\mathcal N} Q$ for some ${\mathcal N}$-barbed bisimulation ${\mathcal S}_{\mathcal N}$.
\end{definition}

$\mathcal{R} \subseteq \pi \times \pi$

$P \mathcal{R} Q => \forall P'. P \red P' \Rightarrow \exists Q'. Q \red Q', P' \mathcal{R} Q'$

$P \vdash x \Rightarrow Q \vdash x$

\begin{mathpar}
  \inferrule*[lab=Out-barb]{x \nameeq y}{{y}!\langle{Q}\rangle \vdash x}
  \and
  \inferrule*[lab=Par-barb]{\mbox{$P\vdash x$ or $Q\vdash x$}}{\binpar{P}{Q} \vdash x}
\end{mathpar}

\subsubsection{Contexts}

One of the principle advantages of computational calculi like the
$\pi$-calculus is a well-defined notion of context,
contextual-equivalence and a correlation between
contextual-equivalence and notions of bisimulation. The notion of
context allows the decomposition of a process into (sub-)process and
its syntactic environment, its context. Thus, a context may be
thought of as a process with a ``hole'' (written $\Box$) in it. The
application of a context $M$ to a process $P$, written $M[P]$, is
tantamount to filling the hole in $M$ with $P$. In this paper we do
not need the full weight of this theory, but do make use of the notion
of context in the proof the main theorem. 

\begin{mathpar}
  \inferrule* [lab=summation] {} {{M_{M},M_{N}} \bc \Box \;|\; x.M_{A} \;|\; M_{M}+M_{N}}
  \and
  \inferrule* [lab=agent] {} {{M_{A}} \bc (\vec{x})M_{P} \;| \; \clift{P_0,\ldots,M_{P},\ldots,P_N}}
  \and \\
  \inferrule* [lab=process] {} {{M_{P}} \bc M_{N} \;| \;P|M_{P} }
\end{mathpar} 

\begin{mathpar}
  \inferrule* [lab=sychronization] {} {M_{N} \bc \Box \;|\; x?M_{F} \;|\; x!M_{C}}
  \and
  \inferrule* [lab=abstraction] {} {{M_{F}} \bc (x)M_{P} }
  \and
  \inferrule* [lab=concretion] {} {{M_{C}} \bc \langle M_{P} \rangle }
  \and \\
  \inferrule* [lab=process] {} {{M_{P}} \bc M_{N} \;| \;P|M_{P} }
\end{mathpar}

\begin{definition}[contextual application] Given a context $M$, and
  process $P$, we define the \emph{contextual application}, $M[P] :=
  M\{P/\Box\}$. That is, the contextual application of M to P is the
  substitution of $P$ for $\Box$ in $M$.
\end{definition}

$\meaningof{-} : L \to \mathcal{P}(\pi)$

\begin{mathpar}
  \inferrule* [lab=collection] {} {\meaningof{true} = \pi, \and \meaningof{~E} = \pi \setminus \meaningof{E}, \and \meaningof{E_{1} \& E_{2}} = \meaningof{E_{1}} \cap \meaningof{E_{2}}}
\end{mathpar}

\begin{mathpar}
  \inferrule* [lab=structure] {} {\meaningof{0} = \{ P \in \pi | P \equiv 0 \}, \and \\ \meaningof{E_1 | E_2} = \{ P \in \pi | P \equiv P_{1} | P_{2}, P_{1} \in \meaningof{E_{1}}, P_{2} \in \meaningof{E_2}\} }
\end{mathpar}

\begin{mathpar}
 \inferrule* [lab=behavior] {} {\meaningof{\langle a?b \rangle E} = \{ P \in \pi | P \equiv Q | u?(y)P', \\ \and \\\\ \and \\ \;\;\; u \in \meaningof{a}, \forall z.P'\{z/y\} \in \meaningof{E\{z/b\}}\}, \and \\ \meaningof{a!E} = \{ P \in \pi | P \equiv Q | x!\langle P' \rangle, x \in \meaningof{a} P' \in \meaningof{E}\} }
\end{mathpar}

\begin{mathpar}
 \inferrule* [lab=nominal] {} {\meaningof{\quotep{E}} = \{ \quotep{P} \in \quotep{\pi} | P \in \meaningof{E} \}, \and \meaningof{\quotep{P}} = \{ \quotep{Q} \in \quotep{\pi} | P \equiv Q \} \and \\ \meaningof{@\quotep{E}} = \{ P \in \pi | P \equiv @x, x \in \meaningof{E} \}}
\end{mathpar}

\begin{eqnarray*}
  \\
  \meaningof{-} : TS \to ST
\end{eqnarray*}

\begin{eqnarray*}
  \\
  L : TS \to ST
\end{eqnarray*}

\begin{eqnarray*}
  \\
  P \models E \iff P \in \meaningof{E}
\end{eqnarray*}

\begin{eqnarray*}
  P \approx_{L} Q \iff \forall E \in L. P \models E \iff Q \models E
\end{eqnarray*}

\begin{eqnarray*}
  P \approx_{K} Q
\end{eqnarray*}

\begin{eqnarray*}
  P \approx Q
\end{eqnarray*}

$\approx_{K} = \approx = \approx_{L}$

\subsubsection{Contextual duality}

Note that contexts extend the quotation operation to a family of
operations from processes to names. Given a context, $M$, we can
define a \emph{nominal context}, $\quotep{M}$ by $\quotep{M}[P] :=
\quotep{M[P]}$. To foreshadow what is to come we observe that these
operations enjoy a duality with processes very much like the duality
between vectors and maps from vectors to scalars.

Further, because the calculus is essentially higher-order, we have a
correspondence between contexts and processes. More specifically,
given a name $x$ and a context $M$ we can construct $M^{*}_{x}$ such
that 

\begin{mathpar}
  M^{*}_{x} | \lift{x}{P} \red M[P]
\end{mathpar}

namely,

\begin{mathpar}
  M^{*}_{x} := x?(u).M[\dropn{u}]
\end{mathpar}

The dependence of $M^{*}_{x}$ on a name makes it an abstraction, 

\begin{mathpar}
  M^{*} := (x)x?(u).M[\dropn{u}]
\end{mathpar}

\subsection{Additional notation}

It will sometimes be convenient to denote the process a name
quotes. We already have the notation $x = \quotep{P}$, but it will be
convenient to introduce an alternate notation, $\procn{x}$, when we
want to emphasize the connection to the use of the name. Note that, by
virtue of name equivalence, $\quotep{\procn{x}} \nameeq x$; so, the
notation is consistent with previous definitions.

Further, because names have structure it is possible to effect
substitutions on the basis of that structure. This means we need to
upgrade our notation for substitutions, which we accomplish by
adapting comprehension notation. Thus,

\begin{mathpar}
  P\{ y / x : x \in S \}
\end{mathpar}

is interpreted to mean the process derived from P by replacing (in a
capture-avoiding manner) each occurrence of $x$ in $S$ by $y$. For example,

\begin{mathpar}
  P\{ \quotep{\procn{x}|\procn{x}} / x : x \in \freenames{P} \}
\end{mathpar}

will replace each (occurrence) of a free name $x$ in $P$ by
$\quotep{\procn{x}|\procn{x}}$.

Also, we will avail ourselves of the notation $x^{L}$ and $x^{R}$ to
denote injections of a name into disjoint copies of the name
space. There are numerous ways to accomplish this. One example can be
found in \cite{MeredithR05}. This notation overloads to vectors of
names: $\vec{x}^{\pi} := (x_{i}^{\pi} \; : \; 0 \leq i < |\vec{x}| )$ where $\pi \in \{L,R\}$.

We also use $P^{\Box} := P|\Box$.

In \cite{MeredithR05} an interpretation of the new operator is
given. It turns out that there are several possible interpretations
all enjoying the requisite algebraic properties of the operator (see
\cite{milner91polyadicpi}). We will therefore make liberal use of
$(\nu\; \vec{x})P$.

% subsection the_syntax_and_semantics_of_the_notation_system (end)   

\input{qm2pi.qmops} 

\input{qm2pi.sterngerlach} 

\input{qm2pi.metric} 

% section concurrent_process_calculi (end)

%\input{qm2pi.proofsketch}

% section proof sketch (end)

%\input{qm2pi.slviaknots} 

% section spatial logic via knots (end)

\input{qm2pi.conclusion}

% section conclusion (end)

%\input{qm2pi.dtcodes} 

% section wiring algorithm (end)

\input{qm2pi.ack} 

% section acknowledgments (end)

\newpage


\bibliographystyle{plain}   
\bibliography{../../biblios/main.bib}

\input{qm2pi.rhodetails}

\end{document}

 

\documentclass[12pt]{llncs}
%\documentclass{jktr}

\usepackage[pdftex]{hyperref}                   
\usepackage {listings}
\usepackage {mathpartir}
\usepackage{bcprules}
%\usepackage{listings}
                       
\usepackage{graphicx} 
%\usepackage[margins=2.5cm,nohead,nofoot]{geometry}
%\usepackage{geometry}
\usepackage{amsfonts}
\usepackage{amstext}
\usepackage{latexsym}
\usepackage{amssymb}
\usepackage{color}


%\include{myPreamble}
\include{qm2pi.local} 

%\ifpdf
%\usepackage[pdftex]{graphicx}
%\else
%\usepackage{graphicx}
%\fi

 % \ifpdf
%  \usepackage{pdfsync}
%  \if


%\title{Brief Article}
%\author{David F. Snyder}
%\author{L.G. Meredith}

%\address{Dept. of Math., Texas State University--San Marcos, San Marcos, TX 78666}
       
\pagestyle{empty}


\begin{document}

\lstset{language=[Objective]Caml,frame=shadowbox}

\input{qm2pi.front}

% section front matter (end)

\input{qm2pi.intro} 
 
% section introduction (end)

% \input{qm2pi.knotations} 

% section notation (end)

\input{qm2pi.process.calculi} 

% section concurrent_process_calculi_and_spatial_logics_ (end)
    
%\input{qm2pi.knots2pi} 

%\input{qm2pi.trefoil} 

%\input{qm2pi.mainthm} 

% subsection basic_interpretation (end)

%\input{qm2pi.rho.presentation} 
\subsection{The syntax and semantics of the notation system}\label{sub:the_syntax_and_semantics_of_the_notation_system} % (fold)

We now summarize a technical presentation of the calculus that
embodies our theory of dynamics. The typical presentation of such a
calculus follows the style of giving generators and relations on
them. The grammar, below, describing term constructors, freely
generates the set of processes, $\Proc$. This set is then quotiented
by a relation known as structural congruence and it is over this set
that the notion of dynamics is expressed. This presentation is
essentially that of \cite{MeredithR05} with the addition of
polyadicity and summation. For readability we have relegated some of
the technical subtleties to an appendix.

\subsubsection{Process grammar}\label{subsub:process_grammar}

\begin{mathpar}
  \inferrule* [lab=synchronization] {} {{M} \bc \pzero \;|\; x?F \;|\; x!C }
  \and
  \inferrule* [lab=abstraction] {} {{F} \bc (x)P}
  \and
  \inferrule* [lab=concretion] {} {{C} \bc \langle Q \rangle}
  \and
  \inferrule* [lab=process] {} {{P,Q} \bc M \;| \;P|Q \;|\; @{x}}
  \and
  \inferrule* [lab=name] {} {{x} \bc \quotep{P}}
\end{mathpar} 

Note that $\vec{x}$ (resp. $\vec{P}$) denotes a vector of names
(resp. processes) of length $|\vec{x}|$ (resp. $|\vec{P}|$). We adopt
the following useful abbreviations.

\begin{mathpar}
   x?(\vec{y}).P := x.(\vec{y})P \and  x\clift{\vec{P}} := x.\clift{\vec{P}}
   \and x!(y) := \lift{x}{\dropn{y}}
   \and \Pi_{i=0}^{n-1}P_i := P_0 | \ldots | P_{n-1}
\end{mathpar}

\subsubsection{Structural congruence}

\paragraph{Free and bound names and alpha-equivalence.} At the
core of structural equivalence is alpha-equivalence which identifies
process that are the same up to a change of variable. Formally, we
recognize the distinction between free and bound names. The free names
of a process, $\freenames{P}$, may be calculated recursively as
follows:

\begin{mathpar}
\freenames{\pzero} := \emptyset
  \and \\
  \freenames{x?(y).P} := \{ x \} \cup (\freenames{P} \setminus \{ y \})
  \and 
  \freenames{x!\langle P \rangle} := \{ x \} \cup \{ P \} 
  \and \\
  \freenames{P|Q} := \freenames{P} \cup \freenames{Q}
  \and \\
  \freenames{@{x}} := \{ x \}
\end{mathpar}

$\pi$
$\quotep{\pi}$

$\freenames{-} : \pi \to \mathcal{P}(\quotep{\pi})$

\begin{eqnarray*}
  \freenames{\pzero} & := & \emptyset \\
  \freenames{x?(y).P} & := & \{ x \} \cup (\freenames{P} \setminus \{ y \}) \\
  \freenames{x!\langle P \rangle} & := & \{ x \} \cup \{ P \} \\
  \freenames{P|Q} & := & \freenames{P} \cup \freenames{Q} \\
  \freenames{\dropn{x}} & := & \{ x \}
\end{eqnarray*}

The bound names of a process, $\boundnames{P}$, are those names occurring in $P$
that are not free. For example, in $x?(y).0$, the name $x$ is free, while $y$ is bound.

\begin{mathpar}
  \inferrule* [lab=monoidal-laws] {} { P|Q \equiv Q|P \and P|0 \equiv P \and P|(Q|R) \equiv (P|Q)|R }
\end{mathpar}

\begin{mathpar}
  \inferrule* [lab=alpha-equivalence] {} { (x)P \equiv (y)P\{y/x\} \and y \not\in \freenames{P} }
\end{mathpar}

\begin{definition}
Then two processes, $P,Q$, are alpha-equivalent if $P = Q\{\vec{y}/\vec{x}\}$ for
some $\vec{x} \in \boundnames{Q},\vec{y} \in \boundnames{P}$, where $Q\{\vec{y}/\vec{x}\}$
denotes the capture-avoiding substitution of $\vec{y}$ for $\vec{x}$ in $Q$.
\end{definition}

\begin{definition}
  The {\em structural congruence} \cite{SangiorgiWalker} , $\equiv$,
  between processes is the least congruence containing
  alpha-equivalence, satisfying the abelian monoid laws
  (associativity, commutativity and $\pzero$ as identity) for parallel
  composition $|$ and for summation $+$.
\end{definition}

\subsection{Name equivalence}

We take name equivalence, written $\nameeq$, to be the smallest
equivalence relation generated by the following rules.

\begin{mathpar}
\inferrule*[lab=Quote-drop]
{ }
{ \quotep{@{x}} \nameeq x }

\inferrule*[lab=Struct-equiv]
{ P \scong Q }
{ \quotep{P} \nameeq \quotep{Q} }
\end{mathpar}

The astute reader will have noticed that the mutual recursion of names
and processes imposes a mutual recursion on alpha-equivalence and
structural equivalence via name-equivalence. Fortunately, all of this
works out pleasantly and we may calculate in the natural way, free of
concern. The reader interested in the details is referred to the
appendix \ref{appendix:rho_details}.

\subsection{Substitution}

We use $\Proc$ for the set of processes, $\QProc$ for the set of
names, and $\id{\{}\vec{y} / \vec{x} \id{\}}$ to denote partial maps,
$s : \QProc \rightarrow \QProc$. A map, $s$ lifts, uniquely, to a map
on process terms, $\widehat{s} : \Proc \rightarrow \Proc$ by the
following equations.

\begin{mathpar}
  (0) \psubstp{Q}{P} := 0 \\
  (R \juxtap S) \psubstp{Q}{P}
  :=    
  (R)\psubstp{Q}{P} \juxtap (S) \psubstp{Q}{P} \\
  (x?(y).R) \psubstp{Q}{P}    
  :=    
  (x)\substp{Q}{P} (z)\concat( (R \psubstn{z}{y}) \psubstp{Q}{P} ) \\
  (\lift{x}{R}) \psubstp{Q}{P}  
  :=
  \lift{(x)\substp{Q}{P}}{ R \psubstp{Q}{P} } \\
%   (\dropn{x})  \psubstp{Q}{P}       
%   := 
%   \left\{ 
%     \begin{array}{ccc} 
%       \dropn{\quotep{Q}} & & x \nameeq \quotep{P} \\
%       \dropn{x} & & otherwise \\
%     \end{array}
%   \right. 
  (\dropn{x})  \psubstp{Q}{P}       
  := 
  \left\{ 
    \begin{array}{ccc} 
      Q & & x \nameeq \quotep{P} \\
      \dropn{x} & & otherwise \\
    \end{array}
  \right.
\end{mathpar}
 

where

\begin{eqnarray}
  (x)\id{\{} \lpquote Q \rpquote / \lpquote P \rpquote \id{\}}            = 
  \left\{ 
    \begin{array}{ccc}
      \lpquote Q \rpquote & & x \nameeq \lpquote P \rpquote \\
      x & & otherwise \\
    \end{array}
  \right. \nonumber
\end{eqnarray}

and $z$ is chosen distinct from $\quotep{P}$, $\quotep{Q}$, the free
names in $Q$, and all the names in $R$. Our $\alpha$-equivalence will
be built in the standard way from this substitution.

\begin{remark}\label{rem:no_self_referential_names}
  One consequence of these definitions is that $\forall P. \quotep{P}
  \not\in \freenames{P}$.
\end{remark}

\subsection{ Dynamic quote: an example }

Anticipating something of what's to come, consider applying the
substitution, $\widehat{\id{\{}u / z \id{\}}}$, to the following pair
of processes, $\lift{w}{y!(z)}$ and $w[ \lpquote y!(z) \rpquote ]$.

\begin{eqnarray}
	\lift{w}{y!(z)}\widehat{\id{\{}u / z \id{\}}}
		& = &
		\lift{w}{y!(u)} \nonumber\\
	w[ \lpquote y!(z) \rpquote ] \widehat{ \id{\{}u / z \id{\}} }
		& = &
		w[ \lpquote y!(z) \rpquote ] \nonumber
\end{eqnarray}

Because the body of the process between quotes is impervious to
substitution, we get radically different answers. In fact, by
examining the first process in an input context,
e.g. $x?(z).\lift{w}{y!(z)}$, we see that the process under the lift
operator may be shaped by prefixed inputs binding a name inside it. In
this sense, the lift operator will be seen as a way to dynamically
construct processes before reifying them as names.

Finally equipped with these standard features we can present the
dynamics of the calculus.

\subsubsection{Operational semantics} 

Finally, we introduce the computational dynamics. What marks these
algebras as distinct from other more traditionally studied algebraic
structures, e.g. vector spaces or polynomial rings, is the manner in
which dynamics is captured. In traditional structures, dynamics is typically
expressed through morphisms between such structures, as in linear maps
between vector spaces or morphisms between rings. In algebras
associated with the semantics of computation, the dynamics is
expressed as part of the algebraic structure itself, through a
reduction reduction relation typically denoted by $\red$. Below, we
give a recursive presentation of this relation for the calculus used
in the encoding.

$\red \subseteq \pi \times \pi$
$\red : \pi \to \mathcal{P}(\pi)$

\begin{mathpar}
  \inferrule* [lab=Comm] { \textsf{match}( x_{src}, x_{trgt} ) } { x_{trgt}?(y)P \; | \; x_{src}!\langle {Q} \rangle \red P\{\quotep{Q}/y}\} }
  \and \\
  \inferrule* [lab=Par] {{P} \red {P}'} {{{P} | {Q}} \red {{P}' | {Q}}}
  \and
  \inferrule* [lab=Equiv]{{{P} \scong {P}'} \andalso {{P}' \red {Q}'} \andalso {{Q}' \scong {Q}}}{{P} \red {Q}}
\end{mathpar}

\begin{eqnarray*}
  match_{\equiv} (\quotep{P},\quotep{Q}) & := & P \equiv Q \\
  match_{\dagger}(\quotep{P},\quotep{Q}) & := & \forall R. P|Q \red^{*} R => R \red^{*} 0 \\
  match_{K}(\quotep{P},\quotep{Q}) & := & K \mbox{ for some context } K
\end{eqnarray*}

$u?(x)P | u!\langle Q \rangle \red P\{\quotep{Q}/x\}$

%We write $\wred$ for $\red^*$, and $P\red$ if $\exists Q $ such that $ P \red Q$.
We write $P\red$ if $\exists Q $ such that $ P \red Q$ and $P\not\red$, otherwise.

\section{Replication}

As mentioned before, it is known that replication (and hence
recursion) can be implemented in a higher-order process algebra
\cite{SangiorgiWalker}. As our first example of calculation with the
machinery thus far presented we give the construction explicitly in
the {\rhoc}.

\begin{eqnarray}
	D_{x} & := & \prefix{x}{y}{(\binpar{\outputp{x}{y}}{@{y}})} \nonumber\\
	\bangp_{x}{P} & := & \binpar{{x}!\langle{\binpar{D_{x}}{P}}\rangle}{D_{x}} \nonumber
\end{eqnarray}

\begin{eqnarray}
	\bangp_{x}{P} & & \nonumber\\
	=
	& {x}!\langle{(\prefix{x}{y}{(\outputp{x}{y} | @{y})) | P}}\rangle 
	      | \prefix{x}{y}{(\outputp{x}{y} | @{y})} & \nonumber\\
	\red
	& (\outputp{x}{y} | @{y})\substn{\quotep{(\prefix{x}{y}{(@{y} | \outputp{x}{y})) | P}}}{y} & \nonumber\\
	=
	& \outputp{x}{\quotep{(\prefix{x}{y}{(\outputp{x}{y} | @{y})) | P}}}
	  | {(\prefix{x}{y}{(\outputp{x}{y} | @{y})) | P}} & \nonumber\\
	\red
	& \ldots & \nonumber\\
	\red^*
	& P | P | \ldots & \nonumber
\end{eqnarray}

Of course, this encoding, as an implementation, runs away, unfolding
$\bangp{P}$ eagerly. A lazier and more implementable replication
operator, restricted to input-guarded processes, may be obtained as follows.

\begin{eqnarray}
\bangp{\prefix{u}{v}{P}} 
	:= 
	\binpar{\lift{x}{\prefix{u}{v}{(\binpar{D(x)}{P})}}}{D(x)} \nonumber
\end{eqnarray}

\begin{remark}
  Note that the lazier definition still does not deal with summation
  or mixed summation (i.e. sums over input and output). The reader is
  invited to construct definitions of replication that deal with these
  features. 

  Further, the definitions are parameterized in a name, $x$. Can you,
  gentle reader, make a definition that eliminates this parameter and
  guarantees no accidental interaction between the replication
  machinery and the process being replicated -- i.e. no accidental
  sharing of names used by the process to get its work done and the
  name(s) used by the replication to effect copying. This latter
  revision of the definition of replication is crucial to obtaining
  the expected identity $!!P \sim !P$.
\end{remark}

\begin{remark}\label{rem:paradoxical_combinator}
  The reader familiar with the lambda calculus will have noticed the
  similarity between $D$ and the paradoxical combinator.

  [Ed. note: the existence of this seems to suggest we have to be more
  restrictive on the set of processes and names we admit if we are to
  support no-cloning.]
\end{remark}

\subsubsection{Bisimulation}

The computational dynamics gives rise to another kind of equivalence,
the equivalence of computational behavior. As previously mentioned
this is typically captured \emph{via} some form of bisimulation.

% The notion we use in this paper is weak barbed bisimulation
% \cite{milner91polyadicpi}.

The notion we use in this paper is derived from weak barbed
bisimulation \cite{milner91polyadicpi}. 

\begin{definition}
An \emph{observation relation}, $\downarrow_{\mathcal N}$, over a set
of names, $\mathcal N$, is the smallest relation satisfying the rules
below.

\infrule[Out-barb]{y \in {\mathcal N}, \; x \nameeq y}
		  {\outputp{x}{v} \downarrow_{\mathcal N} x}
\infrule[Par-barb]{\mbox{$P\downarrow_{\mathcal N} x$ or $Q\downarrow_{\mathcal N} x$}}
		  {\binpar{P}{Q} \downarrow_{\mathcal N} x}

We write $P \Downarrow_{\mathcal N} x$ if there is $Q$ such that 
$P \wred Q$ and $Q \downarrow_{\mathcal N} x$.
\end{definition}

\begin{definition}
%\label{def.bbisim}
An  ${\mathcal N}$-\emph{barbed bisimulation} over a set of names, ${\mathcal N}$, is a symmetric binary relation 
${\mathcal S}_{\mathcal N}$ between agents such that $P\rel{S}_{\mathcal N}Q$ implies:
\begin{enumerate}
\item If $P \red P'$ then $Q \wred Q'$ and $P'\rel{S}_{\mathcal N} Q'$.
\item If $P\downarrow_{\mathcal N} x$, then $Q\Downarrow_{\mathcal N} x$.
\end{enumerate}
$P$ is ${\mathcal N}$-barbed bisimilar to $Q$, written
$P \wbbisim_{\mathcal N} Q$, if $P \rel{S}_{\mathcal N} Q$ for some ${\mathcal N}$-barbed bisimulation ${\mathcal S}_{\mathcal N}$.
\end{definition}

$\mathcal{R} \subseteq \pi \times \pi$

$P \mathcal{R} Q => \forall P'. P \red P' \Rightarrow \exists Q'. Q \red Q', P' \mathcal{R} Q'$

$P \vdash x \Rightarrow Q \vdash x$

\begin{mathpar}
  \inferrule*[lab=Out-barb]{x \nameeq y}{{y}!\langle{Q}\rangle \vdash x}
  \and
  \inferrule*[lab=Par-barb]{\mbox{$P\vdash x$ or $Q\vdash x$}}{\binpar{P}{Q} \vdash x}
\end{mathpar}

\subsubsection{Contexts}

One of the principle advantages of computational calculi like the
$\pi$-calculus is a well-defined notion of context,
contextual-equivalence and a correlation between
contextual-equivalence and notions of bisimulation. The notion of
context allows the decomposition of a process into (sub-)process and
its syntactic environment, its context. Thus, a context may be
thought of as a process with a ``hole'' (written $\Box$) in it. The
application of a context $M$ to a process $P$, written $M[P]$, is
tantamount to filling the hole in $M$ with $P$. In this paper we do
not need the full weight of this theory, but do make use of the notion
of context in the proof the main theorem. 

\begin{mathpar}
  \inferrule* [lab=summation] {} {{M_{M},M_{N}} \bc \Box \;|\; x.M_{A} \;|\; M_{M}+M_{N}}
  \and
  \inferrule* [lab=agent] {} {{M_{A}} \bc (\vec{x})M_{P} \;| \; \clift{P_0,\ldots,M_{P},\ldots,P_N}}
  \and \\
  \inferrule* [lab=process] {} {{M_{P}} \bc M_{N} \;| \;P|M_{P} }
\end{mathpar} 

\begin{mathpar}
  \inferrule* [lab=sychronization] {} {M_{N} \bc \Box \;|\; x?M_{F} \;|\; x!M_{C}}
  \and
  \inferrule* [lab=abstraction] {} {{M_{F}} \bc (x)M_{P} }
  \and
  \inferrule* [lab=concretion] {} {{M_{C}} \bc \langle M_{P} \rangle }
  \and \\
  \inferrule* [lab=process] {} {{M_{P}} \bc M_{N} \;| \;P|M_{P} }
\end{mathpar}

\begin{definition}[contextual application] Given a context $M$, and
  process $P$, we define the \emph{contextual application}, $M[P] :=
  M\{P/\Box\}$. That is, the contextual application of M to P is the
  substitution of $P$ for $\Box$ in $M$.
\end{definition}

$\meaningof{-} : L \to \mathcal{P}(\pi)$

\begin{mathpar}
  \inferrule* [lab=collection] {} {\meaningof{true} = \pi, \and \meaningof{~E} = \pi \setminus \meaningof{E}, \and \meaningof{E_{1} \& E_{2}} = \meaningof{E_{1}} \cap \meaningof{E_{2}}}
\end{mathpar}

\begin{mathpar}
  \inferrule* [lab=structure] {} {\meaningof{0} = \{ P \in \pi | P \equiv 0 \}, \and \\ \meaningof{E_1 | E_2} = \{ P \in \pi | P \equiv P_{1} | P_{2}, P_{1} \in \meaningof{E_{1}}, P_{2} \in \meaningof{E_2}\} }
\end{mathpar}

\begin{mathpar}
 \inferrule* [lab=behavior] {} {\meaningof{\langle a?b \rangle E} = \{ P \in \pi | P \equiv Q | u?(y)P', \\ \and \\\\ \and \\ \;\;\; u \in \meaningof{a}, \forall z.P'\{z/y\} \in \meaningof{E\{z/b\}}\}, \and \\ \meaningof{a!E} = \{ P \in \pi | P \equiv Q | x!\langle P' \rangle, x \in \meaningof{a} P' \in \meaningof{E}\} }
\end{mathpar}

\begin{mathpar}
 \inferrule* [lab=nominal] {} {\meaningof{\quotep{E}} = \{ \quotep{P} \in \quotep{\pi} | P \in \meaningof{E} \}, \and \meaningof{\quotep{P}} = \{ \quotep{Q} \in \quotep{\pi} | P \equiv Q \} \and \\ \meaningof{@\quotep{E}} = \{ P \in \pi | P \equiv @x, x \in \meaningof{E} \}}
\end{mathpar}

\begin{eqnarray*}
  \\
  \meaningof{-} : TS \to ST
\end{eqnarray*}

\begin{eqnarray*}
  \\
  L : TS \to ST
\end{eqnarray*}

\begin{eqnarray*}
  \\
  P \models E \iff P \in \meaningof{E}
\end{eqnarray*}

\begin{eqnarray*}
  P \approx_{L} Q \iff \forall E \in L. P \models E \iff Q \models E
\end{eqnarray*}

\begin{eqnarray*}
  P \approx_{K} Q
\end{eqnarray*}

\begin{eqnarray*}
  P \approx Q
\end{eqnarray*}

$\approx_{K} = \approx = \approx_{L}$

\subsubsection{Contextual duality}

Note that contexts extend the quotation operation to a family of
operations from processes to names. Given a context, $M$, we can
define a \emph{nominal context}, $\quotep{M}$ by $\quotep{M}[P] :=
\quotep{M[P]}$. To foreshadow what is to come we observe that these
operations enjoy a duality with processes very much like the duality
between vectors and maps from vectors to scalars.

Further, because the calculus is essentially higher-order, we have a
correspondence between contexts and processes. More specifically,
given a name $x$ and a context $M$ we can construct $M^{*}_{x}$ such
that 

\begin{mathpar}
  M^{*}_{x} | \lift{x}{P} \red M[P]
\end{mathpar}

namely,

\begin{mathpar}
  M^{*}_{x} := x?(u).M[\dropn{u}]
\end{mathpar}

The dependence of $M^{*}_{x}$ on a name makes it an abstraction, 

\begin{mathpar}
  M^{*} := (x)x?(u).M[\dropn{u}]
\end{mathpar}

\subsection{Additional notation}

It will sometimes be convenient to denote the process a name
quotes. We already have the notation $x = \quotep{P}$, but it will be
convenient to introduce an alternate notation, $\procn{x}$, when we
want to emphasize the connection to the use of the name. Note that, by
virtue of name equivalence, $\quotep{\procn{x}} \nameeq x$; so, the
notation is consistent with previous definitions.

Further, because names have structure it is possible to effect
substitutions on the basis of that structure. This means we need to
upgrade our notation for substitutions, which we accomplish by
adapting comprehension notation. Thus,

\begin{mathpar}
  P\{ y / x : x \in S \}
\end{mathpar}

is interpreted to mean the process derived from P by replacing (in a
capture-avoiding manner) each occurrence of $x$ in $S$ by $y$. For example,

\begin{mathpar}
  P\{ \quotep{\procn{x}|\procn{x}} / x : x \in \freenames{P} \}
\end{mathpar}

will replace each (occurrence) of a free name $x$ in $P$ by
$\quotep{\procn{x}|\procn{x}}$.

Also, we will avail ourselves of the notation $x^{L}$ and $x^{R}$ to
denote injections of a name into disjoint copies of the name
space. There are numerous ways to accomplish this. One example can be
found in \cite{MeredithR05}. This notation overloads to vectors of
names: $\vec{x}^{\pi} := (x_{i}^{\pi} \; : \; 0 \leq i < |\vec{x}| )$ where $\pi \in \{L,R\}$.

We also use $P^{\Box} := P|\Box$.

In \cite{MeredithR05} an interpretation of the new operator is
given. It turns out that there are several possible interpretations
all enjoying the requisite algebraic properties of the operator (see
\cite{milner91polyadicpi}). We will therefore make liberal use of
$(\nu\; \vec{x})P$.

% subsection the_syntax_and_semantics_of_the_notation_system (end)   

\input{qm2pi.qmops} 

\input{qm2pi.sterngerlach} 

\input{qm2pi.metric} 

% section concurrent_process_calculi (end)

%\input{qm2pi.proofsketch}

% section proof sketch (end)

%\input{qm2pi.slviaknots} 

% section spatial logic via knots (end)

\input{qm2pi.conclusion}

% section conclusion (end)

%\input{qm2pi.dtcodes} 

% section wiring algorithm (end)

\input{qm2pi.ack} 

% section acknowledgments (end)

\newpage


\bibliographystyle{plain}   
\bibliography{../../biblios/main.bib}

\input{qm2pi.rhodetails}

\end{document}

 

% section concurrent_process_calculi (end)

%\documentclass[12pt]{llncs}
%\documentclass{jktr}

\usepackage[pdftex]{hyperref}                   
\usepackage {listings}
\usepackage {mathpartir}
\usepackage{bcprules}
%\usepackage{listings}
                       
\usepackage{graphicx} 
%\usepackage[margins=2.5cm,nohead,nofoot]{geometry}
%\usepackage{geometry}
\usepackage{amsfonts}
\usepackage{amstext}
\usepackage{latexsym}
\usepackage{amssymb}
\usepackage{color}


%\include{myPreamble}
\include{qm2pi.local} 

%\ifpdf
%\usepackage[pdftex]{graphicx}
%\else
%\usepackage{graphicx}
%\fi

 % \ifpdf
%  \usepackage{pdfsync}
%  \if


%\title{Brief Article}
%\author{David F. Snyder}
%\author{L.G. Meredith}

%\address{Dept. of Math., Texas State University--San Marcos, San Marcos, TX 78666}
       
\pagestyle{empty}


\begin{document}

\lstset{language=[Objective]Caml,frame=shadowbox}

\input{qm2pi.front}

% section front matter (end)

\input{qm2pi.intro} 
 
% section introduction (end)

% \input{qm2pi.knotations} 

% section notation (end)

\input{qm2pi.process.calculi} 

% section concurrent_process_calculi_and_spatial_logics_ (end)
    
%\input{qm2pi.knots2pi} 

%\input{qm2pi.trefoil} 

%\input{qm2pi.mainthm} 

% subsection basic_interpretation (end)

%\input{qm2pi.rho.presentation} 
\subsection{The syntax and semantics of the notation system}\label{sub:the_syntax_and_semantics_of_the_notation_system} % (fold)

We now summarize a technical presentation of the calculus that
embodies our theory of dynamics. The typical presentation of such a
calculus follows the style of giving generators and relations on
them. The grammar, below, describing term constructors, freely
generates the set of processes, $\Proc$. This set is then quotiented
by a relation known as structural congruence and it is over this set
that the notion of dynamics is expressed. This presentation is
essentially that of \cite{MeredithR05} with the addition of
polyadicity and summation. For readability we have relegated some of
the technical subtleties to an appendix.

\subsubsection{Process grammar}\label{subsub:process_grammar}

\begin{mathpar}
  \inferrule* [lab=synchronization] {} {{M} \bc \pzero \;|\; x?F \;|\; x!C }
  \and
  \inferrule* [lab=abstraction] {} {{F} \bc (x)P}
  \and
  \inferrule* [lab=concretion] {} {{C} \bc \langle Q \rangle}
  \and
  \inferrule* [lab=process] {} {{P,Q} \bc M \;| \;P|Q \;|\; @{x}}
  \and
  \inferrule* [lab=name] {} {{x} \bc \quotep{P}}
\end{mathpar} 

Note that $\vec{x}$ (resp. $\vec{P}$) denotes a vector of names
(resp. processes) of length $|\vec{x}|$ (resp. $|\vec{P}|$). We adopt
the following useful abbreviations.

\begin{mathpar}
   x?(\vec{y}).P := x.(\vec{y})P \and  x\clift{\vec{P}} := x.\clift{\vec{P}}
   \and x!(y) := \lift{x}{\dropn{y}}
   \and \Pi_{i=0}^{n-1}P_i := P_0 | \ldots | P_{n-1}
\end{mathpar}

\subsubsection{Structural congruence}

\paragraph{Free and bound names and alpha-equivalence.} At the
core of structural equivalence is alpha-equivalence which identifies
process that are the same up to a change of variable. Formally, we
recognize the distinction between free and bound names. The free names
of a process, $\freenames{P}$, may be calculated recursively as
follows:

\begin{mathpar}
\freenames{\pzero} := \emptyset
  \and \\
  \freenames{x?(y).P} := \{ x \} \cup (\freenames{P} \setminus \{ y \})
  \and 
  \freenames{x!\langle P \rangle} := \{ x \} \cup \{ P \} 
  \and \\
  \freenames{P|Q} := \freenames{P} \cup \freenames{Q}
  \and \\
  \freenames{@{x}} := \{ x \}
\end{mathpar}

$\pi$
$\quotep{\pi}$

$\freenames{-} : \pi \to \mathcal{P}(\quotep{\pi})$

\begin{eqnarray*}
  \freenames{\pzero} & := & \emptyset \\
  \freenames{x?(y).P} & := & \{ x \} \cup (\freenames{P} \setminus \{ y \}) \\
  \freenames{x!\langle P \rangle} & := & \{ x \} \cup \{ P \} \\
  \freenames{P|Q} & := & \freenames{P} \cup \freenames{Q} \\
  \freenames{\dropn{x}} & := & \{ x \}
\end{eqnarray*}

The bound names of a process, $\boundnames{P}$, are those names occurring in $P$
that are not free. For example, in $x?(y).0$, the name $x$ is free, while $y$ is bound.

\begin{mathpar}
  \inferrule* [lab=monoidal-laws] {} { P|Q \equiv Q|P \and P|0 \equiv P \and P|(Q|R) \equiv (P|Q)|R }
\end{mathpar}

\begin{mathpar}
  \inferrule* [lab=alpha-equivalence] {} { (x)P \equiv (y)P\{y/x\} \and y \not\in \freenames{P} }
\end{mathpar}

\begin{definition}
Then two processes, $P,Q$, are alpha-equivalent if $P = Q\{\vec{y}/\vec{x}\}$ for
some $\vec{x} \in \boundnames{Q},\vec{y} \in \boundnames{P}$, where $Q\{\vec{y}/\vec{x}\}$
denotes the capture-avoiding substitution of $\vec{y}$ for $\vec{x}$ in $Q$.
\end{definition}

\begin{definition}
  The {\em structural congruence} \cite{SangiorgiWalker} , $\equiv$,
  between processes is the least congruence containing
  alpha-equivalence, satisfying the abelian monoid laws
  (associativity, commutativity and $\pzero$ as identity) for parallel
  composition $|$ and for summation $+$.
\end{definition}

\subsection{Name equivalence}

We take name equivalence, written $\nameeq$, to be the smallest
equivalence relation generated by the following rules.

\begin{mathpar}
\inferrule*[lab=Quote-drop]
{ }
{ \quotep{@{x}} \nameeq x }

\inferrule*[lab=Struct-equiv]
{ P \scong Q }
{ \quotep{P} \nameeq \quotep{Q} }
\end{mathpar}

The astute reader will have noticed that the mutual recursion of names
and processes imposes a mutual recursion on alpha-equivalence and
structural equivalence via name-equivalence. Fortunately, all of this
works out pleasantly and we may calculate in the natural way, free of
concern. The reader interested in the details is referred to the
appendix \ref{appendix:rho_details}.

\subsection{Substitution}

We use $\Proc$ for the set of processes, $\QProc$ for the set of
names, and $\id{\{}\vec{y} / \vec{x} \id{\}}$ to denote partial maps,
$s : \QProc \rightarrow \QProc$. A map, $s$ lifts, uniquely, to a map
on process terms, $\widehat{s} : \Proc \rightarrow \Proc$ by the
following equations.

\begin{mathpar}
  (0) \psubstp{Q}{P} := 0 \\
  (R \juxtap S) \psubstp{Q}{P}
  :=    
  (R)\psubstp{Q}{P} \juxtap (S) \psubstp{Q}{P} \\
  (x?(y).R) \psubstp{Q}{P}    
  :=    
  (x)\substp{Q}{P} (z)\concat( (R \psubstn{z}{y}) \psubstp{Q}{P} ) \\
  (\lift{x}{R}) \psubstp{Q}{P}  
  :=
  \lift{(x)\substp{Q}{P}}{ R \psubstp{Q}{P} } \\
%   (\dropn{x})  \psubstp{Q}{P}       
%   := 
%   \left\{ 
%     \begin{array}{ccc} 
%       \dropn{\quotep{Q}} & & x \nameeq \quotep{P} \\
%       \dropn{x} & & otherwise \\
%     \end{array}
%   \right. 
  (\dropn{x})  \psubstp{Q}{P}       
  := 
  \left\{ 
    \begin{array}{ccc} 
      Q & & x \nameeq \quotep{P} \\
      \dropn{x} & & otherwise \\
    \end{array}
  \right.
\end{mathpar}
 

where

\begin{eqnarray}
  (x)\id{\{} \lpquote Q \rpquote / \lpquote P \rpquote \id{\}}            = 
  \left\{ 
    \begin{array}{ccc}
      \lpquote Q \rpquote & & x \nameeq \lpquote P \rpquote \\
      x & & otherwise \\
    \end{array}
  \right. \nonumber
\end{eqnarray}

and $z$ is chosen distinct from $\quotep{P}$, $\quotep{Q}$, the free
names in $Q$, and all the names in $R$. Our $\alpha$-equivalence will
be built in the standard way from this substitution.

\begin{remark}\label{rem:no_self_referential_names}
  One consequence of these definitions is that $\forall P. \quotep{P}
  \not\in \freenames{P}$.
\end{remark}

\subsection{ Dynamic quote: an example }

Anticipating something of what's to come, consider applying the
substitution, $\widehat{\id{\{}u / z \id{\}}}$, to the following pair
of processes, $\lift{w}{y!(z)}$ and $w[ \lpquote y!(z) \rpquote ]$.

\begin{eqnarray}
	\lift{w}{y!(z)}\widehat{\id{\{}u / z \id{\}}}
		& = &
		\lift{w}{y!(u)} \nonumber\\
	w[ \lpquote y!(z) \rpquote ] \widehat{ \id{\{}u / z \id{\}} }
		& = &
		w[ \lpquote y!(z) \rpquote ] \nonumber
\end{eqnarray}

Because the body of the process between quotes is impervious to
substitution, we get radically different answers. In fact, by
examining the first process in an input context,
e.g. $x?(z).\lift{w}{y!(z)}$, we see that the process under the lift
operator may be shaped by prefixed inputs binding a name inside it. In
this sense, the lift operator will be seen as a way to dynamically
construct processes before reifying them as names.

Finally equipped with these standard features we can present the
dynamics of the calculus.

\subsubsection{Operational semantics} 

Finally, we introduce the computational dynamics. What marks these
algebras as distinct from other more traditionally studied algebraic
structures, e.g. vector spaces or polynomial rings, is the manner in
which dynamics is captured. In traditional structures, dynamics is typically
expressed through morphisms between such structures, as in linear maps
between vector spaces or morphisms between rings. In algebras
associated with the semantics of computation, the dynamics is
expressed as part of the algebraic structure itself, through a
reduction reduction relation typically denoted by $\red$. Below, we
give a recursive presentation of this relation for the calculus used
in the encoding.

$\red \subseteq \pi \times \pi$
$\red : \pi \to \mathcal{P}(\pi)$

\begin{mathpar}
  \inferrule* [lab=Comm] { \textsf{match}( x_{src}, x_{trgt} ) } { x_{trgt}?(y)P \; | \; x_{src}!\langle {Q} \rangle \red P\{\quotep{Q}/y}\} }
  \and \\
  \inferrule* [lab=Par] {{P} \red {P}'} {{{P} | {Q}} \red {{P}' | {Q}}}
  \and
  \inferrule* [lab=Equiv]{{{P} \scong {P}'} \andalso {{P}' \red {Q}'} \andalso {{Q}' \scong {Q}}}{{P} \red {Q}}
\end{mathpar}

\begin{eqnarray*}
  match_{\equiv} (\quotep{P},\quotep{Q}) & := & P \equiv Q \\
  match_{\dagger}(\quotep{P},\quotep{Q}) & := & \forall R. P|Q \red^{*} R => R \red^{*} 0 \\
  match_{K}(\quotep{P},\quotep{Q}) & := & K \mbox{ for some context } K
\end{eqnarray*}

$u?(x)P | u!\langle Q \rangle \red P\{\quotep{Q}/x\}$

%We write $\wred$ for $\red^*$, and $P\red$ if $\exists Q $ such that $ P \red Q$.
We write $P\red$ if $\exists Q $ such that $ P \red Q$ and $P\not\red$, otherwise.

\section{Replication}

As mentioned before, it is known that replication (and hence
recursion) can be implemented in a higher-order process algebra
\cite{SangiorgiWalker}. As our first example of calculation with the
machinery thus far presented we give the construction explicitly in
the {\rhoc}.

\begin{eqnarray}
	D_{x} & := & \prefix{x}{y}{(\binpar{\outputp{x}{y}}{@{y}})} \nonumber\\
	\bangp_{x}{P} & := & \binpar{{x}!\langle{\binpar{D_{x}}{P}}\rangle}{D_{x}} \nonumber
\end{eqnarray}

\begin{eqnarray}
	\bangp_{x}{P} & & \nonumber\\
	=
	& {x}!\langle{(\prefix{x}{y}{(\outputp{x}{y} | @{y})) | P}}\rangle 
	      | \prefix{x}{y}{(\outputp{x}{y} | @{y})} & \nonumber\\
	\red
	& (\outputp{x}{y} | @{y})\substn{\quotep{(\prefix{x}{y}{(@{y} | \outputp{x}{y})) | P}}}{y} & \nonumber\\
	=
	& \outputp{x}{\quotep{(\prefix{x}{y}{(\outputp{x}{y} | @{y})) | P}}}
	  | {(\prefix{x}{y}{(\outputp{x}{y} | @{y})) | P}} & \nonumber\\
	\red
	& \ldots & \nonumber\\
	\red^*
	& P | P | \ldots & \nonumber
\end{eqnarray}

Of course, this encoding, as an implementation, runs away, unfolding
$\bangp{P}$ eagerly. A lazier and more implementable replication
operator, restricted to input-guarded processes, may be obtained as follows.

\begin{eqnarray}
\bangp{\prefix{u}{v}{P}} 
	:= 
	\binpar{\lift{x}{\prefix{u}{v}{(\binpar{D(x)}{P})}}}{D(x)} \nonumber
\end{eqnarray}

\begin{remark}
  Note that the lazier definition still does not deal with summation
  or mixed summation (i.e. sums over input and output). The reader is
  invited to construct definitions of replication that deal with these
  features. 

  Further, the definitions are parameterized in a name, $x$. Can you,
  gentle reader, make a definition that eliminates this parameter and
  guarantees no accidental interaction between the replication
  machinery and the process being replicated -- i.e. no accidental
  sharing of names used by the process to get its work done and the
  name(s) used by the replication to effect copying. This latter
  revision of the definition of replication is crucial to obtaining
  the expected identity $!!P \sim !P$.
\end{remark}

\begin{remark}\label{rem:paradoxical_combinator}
  The reader familiar with the lambda calculus will have noticed the
  similarity between $D$ and the paradoxical combinator.

  [Ed. note: the existence of this seems to suggest we have to be more
  restrictive on the set of processes and names we admit if we are to
  support no-cloning.]
\end{remark}

\subsubsection{Bisimulation}

The computational dynamics gives rise to another kind of equivalence,
the equivalence of computational behavior. As previously mentioned
this is typically captured \emph{via} some form of bisimulation.

% The notion we use in this paper is weak barbed bisimulation
% \cite{milner91polyadicpi}.

The notion we use in this paper is derived from weak barbed
bisimulation \cite{milner91polyadicpi}. 

\begin{definition}
An \emph{observation relation}, $\downarrow_{\mathcal N}$, over a set
of names, $\mathcal N$, is the smallest relation satisfying the rules
below.

\infrule[Out-barb]{y \in {\mathcal N}, \; x \nameeq y}
		  {\outputp{x}{v} \downarrow_{\mathcal N} x}
\infrule[Par-barb]{\mbox{$P\downarrow_{\mathcal N} x$ or $Q\downarrow_{\mathcal N} x$}}
		  {\binpar{P}{Q} \downarrow_{\mathcal N} x}

We write $P \Downarrow_{\mathcal N} x$ if there is $Q$ such that 
$P \wred Q$ and $Q \downarrow_{\mathcal N} x$.
\end{definition}

\begin{definition}
%\label{def.bbisim}
An  ${\mathcal N}$-\emph{barbed bisimulation} over a set of names, ${\mathcal N}$, is a symmetric binary relation 
${\mathcal S}_{\mathcal N}$ between agents such that $P\rel{S}_{\mathcal N}Q$ implies:
\begin{enumerate}
\item If $P \red P'$ then $Q \wred Q'$ and $P'\rel{S}_{\mathcal N} Q'$.
\item If $P\downarrow_{\mathcal N} x$, then $Q\Downarrow_{\mathcal N} x$.
\end{enumerate}
$P$ is ${\mathcal N}$-barbed bisimilar to $Q$, written
$P \wbbisim_{\mathcal N} Q$, if $P \rel{S}_{\mathcal N} Q$ for some ${\mathcal N}$-barbed bisimulation ${\mathcal S}_{\mathcal N}$.
\end{definition}

$\mathcal{R} \subseteq \pi \times \pi$

$P \mathcal{R} Q => \forall P'. P \red P' \Rightarrow \exists Q'. Q \red Q', P' \mathcal{R} Q'$

$P \vdash x \Rightarrow Q \vdash x$

\begin{mathpar}
  \inferrule*[lab=Out-barb]{x \nameeq y}{{y}!\langle{Q}\rangle \vdash x}
  \and
  \inferrule*[lab=Par-barb]{\mbox{$P\vdash x$ or $Q\vdash x$}}{\binpar{P}{Q} \vdash x}
\end{mathpar}

\subsubsection{Contexts}

One of the principle advantages of computational calculi like the
$\pi$-calculus is a well-defined notion of context,
contextual-equivalence and a correlation between
contextual-equivalence and notions of bisimulation. The notion of
context allows the decomposition of a process into (sub-)process and
its syntactic environment, its context. Thus, a context may be
thought of as a process with a ``hole'' (written $\Box$) in it. The
application of a context $M$ to a process $P$, written $M[P]$, is
tantamount to filling the hole in $M$ with $P$. In this paper we do
not need the full weight of this theory, but do make use of the notion
of context in the proof the main theorem. 

\begin{mathpar}
  \inferrule* [lab=summation] {} {{M_{M},M_{N}} \bc \Box \;|\; x.M_{A} \;|\; M_{M}+M_{N}}
  \and
  \inferrule* [lab=agent] {} {{M_{A}} \bc (\vec{x})M_{P} \;| \; \clift{P_0,\ldots,M_{P},\ldots,P_N}}
  \and \\
  \inferrule* [lab=process] {} {{M_{P}} \bc M_{N} \;| \;P|M_{P} }
\end{mathpar} 

\begin{mathpar}
  \inferrule* [lab=sychronization] {} {M_{N} \bc \Box \;|\; x?M_{F} \;|\; x!M_{C}}
  \and
  \inferrule* [lab=abstraction] {} {{M_{F}} \bc (x)M_{P} }
  \and
  \inferrule* [lab=concretion] {} {{M_{C}} \bc \langle M_{P} \rangle }
  \and \\
  \inferrule* [lab=process] {} {{M_{P}} \bc M_{N} \;| \;P|M_{P} }
\end{mathpar}

\begin{definition}[contextual application] Given a context $M$, and
  process $P$, we define the \emph{contextual application}, $M[P] :=
  M\{P/\Box\}$. That is, the contextual application of M to P is the
  substitution of $P$ for $\Box$ in $M$.
\end{definition}

$\meaningof{-} : L \to \mathcal{P}(\pi)$

\begin{mathpar}
  \inferrule* [lab=collection] {} {\meaningof{true} = \pi, \and \meaningof{~E} = \pi \setminus \meaningof{E}, \and \meaningof{E_{1} \& E_{2}} = \meaningof{E_{1}} \cap \meaningof{E_{2}}}
\end{mathpar}

\begin{mathpar}
  \inferrule* [lab=structure] {} {\meaningof{0} = \{ P \in \pi | P \equiv 0 \}, \and \\ \meaningof{E_1 | E_2} = \{ P \in \pi | P \equiv P_{1} | P_{2}, P_{1} \in \meaningof{E_{1}}, P_{2} \in \meaningof{E_2}\} }
\end{mathpar}

\begin{mathpar}
 \inferrule* [lab=behavior] {} {\meaningof{\langle a?b \rangle E} = \{ P \in \pi | P \equiv Q | u?(y)P', \\ \and \\\\ \and \\ \;\;\; u \in \meaningof{a}, \forall z.P'\{z/y\} \in \meaningof{E\{z/b\}}\}, \and \\ \meaningof{a!E} = \{ P \in \pi | P \equiv Q | x!\langle P' \rangle, x \in \meaningof{a} P' \in \meaningof{E}\} }
\end{mathpar}

\begin{mathpar}
 \inferrule* [lab=nominal] {} {\meaningof{\quotep{E}} = \{ \quotep{P} \in \quotep{\pi} | P \in \meaningof{E} \}, \and \meaningof{\quotep{P}} = \{ \quotep{Q} \in \quotep{\pi} | P \equiv Q \} \and \\ \meaningof{@\quotep{E}} = \{ P \in \pi | P \equiv @x, x \in \meaningof{E} \}}
\end{mathpar}

\begin{eqnarray*}
  \\
  \meaningof{-} : TS \to ST
\end{eqnarray*}

\begin{eqnarray*}
  \\
  L : TS \to ST
\end{eqnarray*}

\begin{eqnarray*}
  \\
  P \models E \iff P \in \meaningof{E}
\end{eqnarray*}

\begin{eqnarray*}
  P \approx_{L} Q \iff \forall E \in L. P \models E \iff Q \models E
\end{eqnarray*}

\begin{eqnarray*}
  P \approx_{K} Q
\end{eqnarray*}

\begin{eqnarray*}
  P \approx Q
\end{eqnarray*}

$\approx_{K} = \approx = \approx_{L}$

\subsubsection{Contextual duality}

Note that contexts extend the quotation operation to a family of
operations from processes to names. Given a context, $M$, we can
define a \emph{nominal context}, $\quotep{M}$ by $\quotep{M}[P] :=
\quotep{M[P]}$. To foreshadow what is to come we observe that these
operations enjoy a duality with processes very much like the duality
between vectors and maps from vectors to scalars.

Further, because the calculus is essentially higher-order, we have a
correspondence between contexts and processes. More specifically,
given a name $x$ and a context $M$ we can construct $M^{*}_{x}$ such
that 

\begin{mathpar}
  M^{*}_{x} | \lift{x}{P} \red M[P]
\end{mathpar}

namely,

\begin{mathpar}
  M^{*}_{x} := x?(u).M[\dropn{u}]
\end{mathpar}

The dependence of $M^{*}_{x}$ on a name makes it an abstraction, 

\begin{mathpar}
  M^{*} := (x)x?(u).M[\dropn{u}]
\end{mathpar}

\subsection{Additional notation}

It will sometimes be convenient to denote the process a name
quotes. We already have the notation $x = \quotep{P}$, but it will be
convenient to introduce an alternate notation, $\procn{x}$, when we
want to emphasize the connection to the use of the name. Note that, by
virtue of name equivalence, $\quotep{\procn{x}} \nameeq x$; so, the
notation is consistent with previous definitions.

Further, because names have structure it is possible to effect
substitutions on the basis of that structure. This means we need to
upgrade our notation for substitutions, which we accomplish by
adapting comprehension notation. Thus,

\begin{mathpar}
  P\{ y / x : x \in S \}
\end{mathpar}

is interpreted to mean the process derived from P by replacing (in a
capture-avoiding manner) each occurrence of $x$ in $S$ by $y$. For example,

\begin{mathpar}
  P\{ \quotep{\procn{x}|\procn{x}} / x : x \in \freenames{P} \}
\end{mathpar}

will replace each (occurrence) of a free name $x$ in $P$ by
$\quotep{\procn{x}|\procn{x}}$.

Also, we will avail ourselves of the notation $x^{L}$ and $x^{R}$ to
denote injections of a name into disjoint copies of the name
space. There are numerous ways to accomplish this. One example can be
found in \cite{MeredithR05}. This notation overloads to vectors of
names: $\vec{x}^{\pi} := (x_{i}^{\pi} \; : \; 0 \leq i < |\vec{x}| )$ where $\pi \in \{L,R\}$.

We also use $P^{\Box} := P|\Box$.

In \cite{MeredithR05} an interpretation of the new operator is
given. It turns out that there are several possible interpretations
all enjoying the requisite algebraic properties of the operator (see
\cite{milner91polyadicpi}). We will therefore make liberal use of
$(\nu\; \vec{x})P$.

% subsection the_syntax_and_semantics_of_the_notation_system (end)   

\input{qm2pi.qmops} 

\input{qm2pi.sterngerlach} 

\input{qm2pi.metric} 

% section concurrent_process_calculi (end)

%\input{qm2pi.proofsketch}

% section proof sketch (end)

%\input{qm2pi.slviaknots} 

% section spatial logic via knots (end)

\input{qm2pi.conclusion}

% section conclusion (end)

%\input{qm2pi.dtcodes} 

% section wiring algorithm (end)

\input{qm2pi.ack} 

% section acknowledgments (end)

\newpage


\bibliographystyle{plain}   
\bibliography{../../biblios/main.bib}

\input{qm2pi.rhodetails}

\end{document}



% section proof sketch (end)

%\section{Unlikely characters: spatial logic for
  knots}\label{sub:characteristic_formulae} % (fold)

Associated to the mobile process calculi are a family of logics known
as the Hennessy-Milner logics. These logics typically enjoy a
semantics interpreting formulae as sets of processes that when
factored through the encoding outlined above allows an identification
of classes of knots with logical formulae. In the context of this
encoding the sub-family known as the spatial logics \cite{CairesC03}
\cite{CairesC04} \cite{Caires04} are of particular interest providing
several important features for expressing and reasoning about
properties (i.e. classes) of knots. We hint here at how this may be done.

%\begin{description}
%\item [structural connectives] 
\subsubsection{Structural connectives} The spatial logics enjoy
structural connectives corresponding, at the logical level, to the
parallel composition ($P | Q$) and new name ($(\nu \; x)P$)
connectives for processes. As illustrated in the examples below, these
connectives are extremely expressive given the shape of our encoding.
%\item [decideable satisfaction]

\subsubsection{Decideable satisfaction}
In \cite{Caires04} the satisfaction relation is shown to be decideable
for a rich class of processes. It further turns out that the image of
the our encoding is a proper subset of that class. This result
provides the basis for an algorithm by which to search for knots
enjoying a given property.
%\item [characteristic formulae]

\subsubsection{Characteristic formulae}
In the same paper \cite{Caires04} , Caires presents a means of calculating
characteristic formulae, selecting equivalence classes of processes
up to a pre--specified depth limit on the support set of names. Composed with our
encoding, this characteristic formula can be used to select
characteristic formulae for knots.
%\end{description}

\subsubsection{Spatial logic formulae}

The grammar below (segmented for comprehension) summarizes the syntax
of spatial logic formulae. We employ illustrative examples in the
sequel to provide an intuitive understanding of their meaning
referring the reader to \cite{Caires04} for a more detailed explication
of the semantics.

\begin{mathpar}
  \inferrule* [lab=boolean] {} {{A,B} \bc T \;|\; \neg A \;|\; A \wedge B \;|\; \eta = \eta'}
  \and
  \inferrule* [lab=spatial] {} {|\; \pzero \;|\; A | B \;|\; x \text{\textregistered} A \;|\; \forall x . A \;|\;  H x . A}
  \and
  \inferrule* [lab=behavioral] {} {|\; \alpha . A}
  \and 
  \inferrule* [lab=recursion] {} {|\; X(\vec{u}) \;|\; \mu X(\vec{u}) . A}
  \and
  \inferrule* [lab=action] {} {\alpha \bc \langle x?(\vec{y}) \rangle \;|\; \langle x!(\vec{y}) \rangle \;|\; \langle \tau \rangle}
  \and 
  \inferrule* [lab=name] {} {\eta \bc x \;|\; \tau}
\end{mathpar} 

% subsection characteristic_formulae (end)   	 

\subsection{Example formulae}\label{sub:example_formulae_} % (fold)

\subsubsection{Crossing as formula.}
% 
% \begin{align*}
%   \frac{d}{dx} \sin x &= \cos x 
%   & \frac{d}{dx} e^x &= e^x \\
%   \frac{d}{dx} \cos x &= - \sin x 
%   & \frac{d}{dx} \log x &= \frac{1}{x} \\
% \end{align*} 

\begin{align*}
 \mu C(x_{0},x_{1},y_{0},y_{1},u).&(\langle x_{0}?(z) \rangle(\langle u! \rangle\langle y_{1}!z \rangle C(x_{0},x_{1},y_{0},y_{1},u)) & \\
  & \wedge \langle y_{1}?(z) \rangle (\langle u! \rangle \langle x_{0}!z \rangle C(x_{0},x_{1},y_{0},y_{1},u)) & \\
  & \wedge \langle x_{1}?(z) \rangle (\langle u? \rangle \langle y_{0}!z \rangle C(x_{0},x_{1},y_{0},y_{1},u)) & \\
  & \wedge \langle y_{0}?(z) \rangle (\langle u? \rangle \langle x_{1}!z \rangle C(x_{0},x_{1},y_{0},y_{1},u))) &
\end{align*}

The lexicographical similarity between the shape of this formulae and
the shape of definition of the process representing a crossing reveals
the intuitive meaning of this formulae. It describes the capabilities
of a process that has the right to represent a crossing. For example
it picks out processes that may perform an input on the port $x_0$ in
its initial menu of capabilities. What differentiates the formula
from the process, however, is that the crossing process is the
smallest candidate to satisfy the formula. Infinitely many other
processes -- with internal behavior hidden behind this interface, so
to speak -- also satisfy this formula. Even this simple formula,
then, can be seen to open a new view onto knots, providing a
computational interpretation of \emph{virtual} knots.

Note that this formula is derived by hand. A similar formula can be
derived by employing Caires' calculation of characteristic formula
\cite{Caires04} to the process representing a crossing. In light of
this discussion, we let
$\meaningof{C}_{\phi}(x0,x1,y0,y1,u)$ denote a formula specifying the
dynamics we wish to capture of a crossing. To guarantee we preserve
the shape of the interface and minimal semantics we demand that
$\meaningof{C}_{\phi}(x0,x1,y0,y1,u) \Rightarrow
\textbf{C}(x0,x1,y0,y1,u)$ where $\textbf{C}(x0,x1,y0,y1,u)$ denotes
the formula above.
                            
\subsubsection{Crossing number constraints.}
The moral content of the context lemma (Lemma \ref{context}) is that the notion of
``locality'' in the Reidemeister moves is effectively captured by the
parallel composition operator of the process calculus. This intuition
extends through the logic. Given a formula,
$\meaningof{C}_{\phi}(x0,x1,y0,y1,u)$, we can use the structural
connectives to specify constraints on crossing numbers, such as at
least $n$ crossings, or exactly $n$ crossings.
\begin{mathpar}
  \inferrule* [lab=at-least-n] {} { K^{\geq n}_{\phi}(\vec{xs},\vec{ys}) := \Pi_{i=0}^{n-1} Hu . \meaningof{C}_{\phi}(xs_i,ys_i,u) | T }
  \and 
  \inferrule* [lab=exactly-n] {} { K^{= n}_{\phi}(\vec{xs},\vec{ys}) := \Pi_{i=0}^{n-1} Hu . \meaningof{C}_{\phi}(xs_i,ys_i,u) | \neg (\forall x_0,y_0,x_1,y_1,u . \meaningof{C}_{\phi}(x_0,y_0,x_1,y_1,u) | T) }
\end{mathpar}

To round out this section, recall that the encoding of an $n$-crossing
knot decomposes into a parallel composition of $n$ \emph{copies} of a
crossing process together with a wiring harness. To specify different
knot classes with the same crossing number amounts to specifying
logical constraints on the wiring harness. In the interest of space,
we defer examples to a forthcoming paper. Suffice it to say that both
the conditions ``alternating knot'' and ``contains the tangle
corresponding to 5/3'' are expressible. For example, it is possible to
calculate the characteristic formula of a process corresponding to the
tangle 5/3 and conjoin it into the classifying formula via the
composition connective of the logic.

Finally, we wish to observe that it is entirely within reason to
contemplate a more domain-specific version of spatial logic tailored
to the shape of processes in the image of the encoding. Such a
domain-specific logic would have a better claim to the title formal
language of knot properties.

% subsection example_formulae_ (end)

% section knots_as_processes (end) 

% section spatial logic via knots (end)

\section{Conclusions and future work}

\paragraph{Testing physical space}
You, gentle reader, may wonder why of all the theorems to be proved
given this set up we pick the one above. In some sense it's hardly
central to quantum mechanics. We see it as central in the sense that
it firmly establishes a notion of physical space arising from a notion
of the equivalence of behavior. Relating bisimulation to a metric is a
big step forward, but one is faced with interpreting the relationship
of that metric space to something more physical. Quantum mechanical
notions of ``physical'' space are still far from intuitive, but by
relating this idea of distance as testing to calculations that predict
physical circumstances we are making a not insignificant step forward
toward an understanding of the physical space we inhabit as
essentially dynamic.

\paragraph{Effectivity and simulation}
One of the observations we have yet to make is that the entire program
spelled out here is effective. We have built various interpreters for
the reflective calculus at work in this interpretation. In principle,
then, we can simulate quantum mechanics on a computer. The place where
the simulation may lose fidelity is the infinitely branching summation
for the annihilator.

In this connection i also want to point out that the evaluation style
calculation of the inner product puts the non-determinism of the
summation right at the heart of measurement. This suggests that
Milner's original reduction-based formulation of the dynamics of his
calculi in terms of sums was not just notationally suggestive of a
notion of measure-and-continue but captured some significant part of
the physics.

\paragraph{Quantum continuations}
In light of this last observation i want to point out that the
predominant account of quantum mechanics is missing a key aspect of a
truly compositional story of the physical situation. In a real lab,
when a measurement is made the observation can be made to feed into
another device that then makes another measurement conditioned on the
results of the first. This means that after the superposition was
collapsed the entire experimental set up remained in
superposition. While QM offers a means of writing this down it doesn't
quite line up well with the well-trodden formulation of computation
and continuation that we see so succinctly expressed in Milner's
calculi. This suggests that there might be advantages to this account
of dynamics waiting to be explored.

\paragraph{Quantum logic}
In this connection, we also note that by virtue of having the
Hennessy-Milner construction, we can pull the construction through the
interpretation of QM. This gives us a natural candidate for a quantum
logic that enjoys an extremely tight connection with it's domain of
interpretation, making the construction much less ad hoc (rather it is
the image of functor!).

\paragraph{Quantum probabiity}
i have questions about the basis of the interpretation of inner
product as probability amplitude. In particular, using which
axiomatization of probability theory does the notion of probability
amplitude earn the right to be so dubbed? In other words, where is the
proof that the operation for calculating a probability amplitude (and
then squaring) satisfies the axioms of what it means to calculate a
probability? Even if such a proof exists (i have yet to find it in the
literature), i wonder if it might not be possible to turn things on
their heads. Can we view the calculation of the probability amplitude
as an axiomatization of probability? If so, then the definition we
give for calculating probability amplitude may provide the basis for
an \emph{effective} theory of probability.

\paragraph{Quantum vs ``biological'' information}
Finally, i want to conclude with a more philosophical observation. At
a recent workshop in which QM was a predominant topic i noticed
something about quantum information. The speaker was giving a riveting
discussion of axiomatic QM and showing how properties of ``no
cloning'' and ``no deleting'' emerged as consequences of the
axiomatization. Theorems of this form are necessary to give us a sense
of confidence that our axioms characterize the physical theory. What
struck me, though, was that if quantum information is neither erasable
nor replicable it is markedly different from \emph{life}. Two of the
things we know about life is that

\begin{itemize}
  \item it ends;
  \item to gain some measure of persistence, to transcend it's
    finitude it is imminently copyable.
\end{itemize}

Both of these qualities are summarized succinctly in the aphorism: all
flesh is grass. For me these two kinds of ``information'' -- call them
quantum and biological -- are end points on a spectrum of strategies
for persistence. At one end, we have those curious entities that enjoy
uniqueness and permanence; at the other, we have those who in the face
of a certain end and an uncertain present make a go of passing
something on. To me one of the more remarkable aspects of the latter
strategy is that in the presence of noise (and certain features of
copying) we get a kind of dynamism, a chance for improvement against a
given persistent condition.

% subsection other_calculi_other_bisimulations_and_geometry_as_behavior (end)




% section conclusion (end)

%\documentclass[12pt]{llncs}
%\documentclass{jktr}

\usepackage[pdftex]{hyperref}                   
\usepackage {listings}
\usepackage {mathpartir}
\usepackage{bcprules}
%\usepackage{listings}
                       
\usepackage{graphicx} 
%\usepackage[margins=2.5cm,nohead,nofoot]{geometry}
%\usepackage{geometry}
\usepackage{amsfonts}
\usepackage{amstext}
\usepackage{latexsym}
\usepackage{amssymb}
\usepackage{color}


%\include{myPreamble}
\include{qm2pi.local} 

%\ifpdf
%\usepackage[pdftex]{graphicx}
%\else
%\usepackage{graphicx}
%\fi

 % \ifpdf
%  \usepackage{pdfsync}
%  \if


%\title{Brief Article}
%\author{David F. Snyder}
%\author{L.G. Meredith}

%\address{Dept. of Math., Texas State University--San Marcos, San Marcos, TX 78666}
       
\pagestyle{empty}


\begin{document}

\lstset{language=[Objective]Caml,frame=shadowbox}

\input{qm2pi.front}

% section front matter (end)

\input{qm2pi.intro} 
 
% section introduction (end)

% \input{qm2pi.knotations} 

% section notation (end)

\input{qm2pi.process.calculi} 

% section concurrent_process_calculi_and_spatial_logics_ (end)
    
%\input{qm2pi.knots2pi} 

%\input{qm2pi.trefoil} 

%\input{qm2pi.mainthm} 

% subsection basic_interpretation (end)

%\input{qm2pi.rho.presentation} 
\subsection{The syntax and semantics of the notation system}\label{sub:the_syntax_and_semantics_of_the_notation_system} % (fold)

We now summarize a technical presentation of the calculus that
embodies our theory of dynamics. The typical presentation of such a
calculus follows the style of giving generators and relations on
them. The grammar, below, describing term constructors, freely
generates the set of processes, $\Proc$. This set is then quotiented
by a relation known as structural congruence and it is over this set
that the notion of dynamics is expressed. This presentation is
essentially that of \cite{MeredithR05} with the addition of
polyadicity and summation. For readability we have relegated some of
the technical subtleties to an appendix.

\subsubsection{Process grammar}\label{subsub:process_grammar}

\begin{mathpar}
  \inferrule* [lab=synchronization] {} {{M} \bc \pzero \;|\; x?F \;|\; x!C }
  \and
  \inferrule* [lab=abstraction] {} {{F} \bc (x)P}
  \and
  \inferrule* [lab=concretion] {} {{C} \bc \langle Q \rangle}
  \and
  \inferrule* [lab=process] {} {{P,Q} \bc M \;| \;P|Q \;|\; @{x}}
  \and
  \inferrule* [lab=name] {} {{x} \bc \quotep{P}}
\end{mathpar} 

Note that $\vec{x}$ (resp. $\vec{P}$) denotes a vector of names
(resp. processes) of length $|\vec{x}|$ (resp. $|\vec{P}|$). We adopt
the following useful abbreviations.

\begin{mathpar}
   x?(\vec{y}).P := x.(\vec{y})P \and  x\clift{\vec{P}} := x.\clift{\vec{P}}
   \and x!(y) := \lift{x}{\dropn{y}}
   \and \Pi_{i=0}^{n-1}P_i := P_0 | \ldots | P_{n-1}
\end{mathpar}

\subsubsection{Structural congruence}

\paragraph{Free and bound names and alpha-equivalence.} At the
core of structural equivalence is alpha-equivalence which identifies
process that are the same up to a change of variable. Formally, we
recognize the distinction between free and bound names. The free names
of a process, $\freenames{P}$, may be calculated recursively as
follows:

\begin{mathpar}
\freenames{\pzero} := \emptyset
  \and \\
  \freenames{x?(y).P} := \{ x \} \cup (\freenames{P} \setminus \{ y \})
  \and 
  \freenames{x!\langle P \rangle} := \{ x \} \cup \{ P \} 
  \and \\
  \freenames{P|Q} := \freenames{P} \cup \freenames{Q}
  \and \\
  \freenames{@{x}} := \{ x \}
\end{mathpar}

$\pi$
$\quotep{\pi}$

$\freenames{-} : \pi \to \mathcal{P}(\quotep{\pi})$

\begin{eqnarray*}
  \freenames{\pzero} & := & \emptyset \\
  \freenames{x?(y).P} & := & \{ x \} \cup (\freenames{P} \setminus \{ y \}) \\
  \freenames{x!\langle P \rangle} & := & \{ x \} \cup \{ P \} \\
  \freenames{P|Q} & := & \freenames{P} \cup \freenames{Q} \\
  \freenames{\dropn{x}} & := & \{ x \}
\end{eqnarray*}

The bound names of a process, $\boundnames{P}$, are those names occurring in $P$
that are not free. For example, in $x?(y).0$, the name $x$ is free, while $y$ is bound.

\begin{mathpar}
  \inferrule* [lab=monoidal-laws] {} { P|Q \equiv Q|P \and P|0 \equiv P \and P|(Q|R) \equiv (P|Q)|R }
\end{mathpar}

\begin{mathpar}
  \inferrule* [lab=alpha-equivalence] {} { (x)P \equiv (y)P\{y/x\} \and y \not\in \freenames{P} }
\end{mathpar}

\begin{definition}
Then two processes, $P,Q$, are alpha-equivalent if $P = Q\{\vec{y}/\vec{x}\}$ for
some $\vec{x} \in \boundnames{Q},\vec{y} \in \boundnames{P}$, where $Q\{\vec{y}/\vec{x}\}$
denotes the capture-avoiding substitution of $\vec{y}$ for $\vec{x}$ in $Q$.
\end{definition}

\begin{definition}
  The {\em structural congruence} \cite{SangiorgiWalker} , $\equiv$,
  between processes is the least congruence containing
  alpha-equivalence, satisfying the abelian monoid laws
  (associativity, commutativity and $\pzero$ as identity) for parallel
  composition $|$ and for summation $+$.
\end{definition}

\subsection{Name equivalence}

We take name equivalence, written $\nameeq$, to be the smallest
equivalence relation generated by the following rules.

\begin{mathpar}
\inferrule*[lab=Quote-drop]
{ }
{ \quotep{@{x}} \nameeq x }

\inferrule*[lab=Struct-equiv]
{ P \scong Q }
{ \quotep{P} \nameeq \quotep{Q} }
\end{mathpar}

The astute reader will have noticed that the mutual recursion of names
and processes imposes a mutual recursion on alpha-equivalence and
structural equivalence via name-equivalence. Fortunately, all of this
works out pleasantly and we may calculate in the natural way, free of
concern. The reader interested in the details is referred to the
appendix \ref{appendix:rho_details}.

\subsection{Substitution}

We use $\Proc$ for the set of processes, $\QProc$ for the set of
names, and $\id{\{}\vec{y} / \vec{x} \id{\}}$ to denote partial maps,
$s : \QProc \rightarrow \QProc$. A map, $s$ lifts, uniquely, to a map
on process terms, $\widehat{s} : \Proc \rightarrow \Proc$ by the
following equations.

\begin{mathpar}
  (0) \psubstp{Q}{P} := 0 \\
  (R \juxtap S) \psubstp{Q}{P}
  :=    
  (R)\psubstp{Q}{P} \juxtap (S) \psubstp{Q}{P} \\
  (x?(y).R) \psubstp{Q}{P}    
  :=    
  (x)\substp{Q}{P} (z)\concat( (R \psubstn{z}{y}) \psubstp{Q}{P} ) \\
  (\lift{x}{R}) \psubstp{Q}{P}  
  :=
  \lift{(x)\substp{Q}{P}}{ R \psubstp{Q}{P} } \\
%   (\dropn{x})  \psubstp{Q}{P}       
%   := 
%   \left\{ 
%     \begin{array}{ccc} 
%       \dropn{\quotep{Q}} & & x \nameeq \quotep{P} \\
%       \dropn{x} & & otherwise \\
%     \end{array}
%   \right. 
  (\dropn{x})  \psubstp{Q}{P}       
  := 
  \left\{ 
    \begin{array}{ccc} 
      Q & & x \nameeq \quotep{P} \\
      \dropn{x} & & otherwise \\
    \end{array}
  \right.
\end{mathpar}
 

where

\begin{eqnarray}
  (x)\id{\{} \lpquote Q \rpquote / \lpquote P \rpquote \id{\}}            = 
  \left\{ 
    \begin{array}{ccc}
      \lpquote Q \rpquote & & x \nameeq \lpquote P \rpquote \\
      x & & otherwise \\
    \end{array}
  \right. \nonumber
\end{eqnarray}

and $z$ is chosen distinct from $\quotep{P}$, $\quotep{Q}$, the free
names in $Q$, and all the names in $R$. Our $\alpha$-equivalence will
be built in the standard way from this substitution.

\begin{remark}\label{rem:no_self_referential_names}
  One consequence of these definitions is that $\forall P. \quotep{P}
  \not\in \freenames{P}$.
\end{remark}

\subsection{ Dynamic quote: an example }

Anticipating something of what's to come, consider applying the
substitution, $\widehat{\id{\{}u / z \id{\}}}$, to the following pair
of processes, $\lift{w}{y!(z)}$ and $w[ \lpquote y!(z) \rpquote ]$.

\begin{eqnarray}
	\lift{w}{y!(z)}\widehat{\id{\{}u / z \id{\}}}
		& = &
		\lift{w}{y!(u)} \nonumber\\
	w[ \lpquote y!(z) \rpquote ] \widehat{ \id{\{}u / z \id{\}} }
		& = &
		w[ \lpquote y!(z) \rpquote ] \nonumber
\end{eqnarray}

Because the body of the process between quotes is impervious to
substitution, we get radically different answers. In fact, by
examining the first process in an input context,
e.g. $x?(z).\lift{w}{y!(z)}$, we see that the process under the lift
operator may be shaped by prefixed inputs binding a name inside it. In
this sense, the lift operator will be seen as a way to dynamically
construct processes before reifying them as names.

Finally equipped with these standard features we can present the
dynamics of the calculus.

\subsubsection{Operational semantics} 

Finally, we introduce the computational dynamics. What marks these
algebras as distinct from other more traditionally studied algebraic
structures, e.g. vector spaces or polynomial rings, is the manner in
which dynamics is captured. In traditional structures, dynamics is typically
expressed through morphisms between such structures, as in linear maps
between vector spaces or morphisms between rings. In algebras
associated with the semantics of computation, the dynamics is
expressed as part of the algebraic structure itself, through a
reduction reduction relation typically denoted by $\red$. Below, we
give a recursive presentation of this relation for the calculus used
in the encoding.

$\red \subseteq \pi \times \pi$
$\red : \pi \to \mathcal{P}(\pi)$

\begin{mathpar}
  \inferrule* [lab=Comm] { \textsf{match}( x_{src}, x_{trgt} ) } { x_{trgt}?(y)P \; | \; x_{src}!\langle {Q} \rangle \red P\{\quotep{Q}/y}\} }
  \and \\
  \inferrule* [lab=Par] {{P} \red {P}'} {{{P} | {Q}} \red {{P}' | {Q}}}
  \and
  \inferrule* [lab=Equiv]{{{P} \scong {P}'} \andalso {{P}' \red {Q}'} \andalso {{Q}' \scong {Q}}}{{P} \red {Q}}
\end{mathpar}

\begin{eqnarray*}
  match_{\equiv} (\quotep{P},\quotep{Q}) & := & P \equiv Q \\
  match_{\dagger}(\quotep{P},\quotep{Q}) & := & \forall R. P|Q \red^{*} R => R \red^{*} 0 \\
  match_{K}(\quotep{P},\quotep{Q}) & := & K \mbox{ for some context } K
\end{eqnarray*}

$u?(x)P | u!\langle Q \rangle \red P\{\quotep{Q}/x\}$

%We write $\wred$ for $\red^*$, and $P\red$ if $\exists Q $ such that $ P \red Q$.
We write $P\red$ if $\exists Q $ such that $ P \red Q$ and $P\not\red$, otherwise.

\section{Replication}

As mentioned before, it is known that replication (and hence
recursion) can be implemented in a higher-order process algebra
\cite{SangiorgiWalker}. As our first example of calculation with the
machinery thus far presented we give the construction explicitly in
the {\rhoc}.

\begin{eqnarray}
	D_{x} & := & \prefix{x}{y}{(\binpar{\outputp{x}{y}}{@{y}})} \nonumber\\
	\bangp_{x}{P} & := & \binpar{{x}!\langle{\binpar{D_{x}}{P}}\rangle}{D_{x}} \nonumber
\end{eqnarray}

\begin{eqnarray}
	\bangp_{x}{P} & & \nonumber\\
	=
	& {x}!\langle{(\prefix{x}{y}{(\outputp{x}{y} | @{y})) | P}}\rangle 
	      | \prefix{x}{y}{(\outputp{x}{y} | @{y})} & \nonumber\\
	\red
	& (\outputp{x}{y} | @{y})\substn{\quotep{(\prefix{x}{y}{(@{y} | \outputp{x}{y})) | P}}}{y} & \nonumber\\
	=
	& \outputp{x}{\quotep{(\prefix{x}{y}{(\outputp{x}{y} | @{y})) | P}}}
	  | {(\prefix{x}{y}{(\outputp{x}{y} | @{y})) | P}} & \nonumber\\
	\red
	& \ldots & \nonumber\\
	\red^*
	& P | P | \ldots & \nonumber
\end{eqnarray}

Of course, this encoding, as an implementation, runs away, unfolding
$\bangp{P}$ eagerly. A lazier and more implementable replication
operator, restricted to input-guarded processes, may be obtained as follows.

\begin{eqnarray}
\bangp{\prefix{u}{v}{P}} 
	:= 
	\binpar{\lift{x}{\prefix{u}{v}{(\binpar{D(x)}{P})}}}{D(x)} \nonumber
\end{eqnarray}

\begin{remark}
  Note that the lazier definition still does not deal with summation
  or mixed summation (i.e. sums over input and output). The reader is
  invited to construct definitions of replication that deal with these
  features. 

  Further, the definitions are parameterized in a name, $x$. Can you,
  gentle reader, make a definition that eliminates this parameter and
  guarantees no accidental interaction between the replication
  machinery and the process being replicated -- i.e. no accidental
  sharing of names used by the process to get its work done and the
  name(s) used by the replication to effect copying. This latter
  revision of the definition of replication is crucial to obtaining
  the expected identity $!!P \sim !P$.
\end{remark}

\begin{remark}\label{rem:paradoxical_combinator}
  The reader familiar with the lambda calculus will have noticed the
  similarity between $D$ and the paradoxical combinator.

  [Ed. note: the existence of this seems to suggest we have to be more
  restrictive on the set of processes and names we admit if we are to
  support no-cloning.]
\end{remark}

\subsubsection{Bisimulation}

The computational dynamics gives rise to another kind of equivalence,
the equivalence of computational behavior. As previously mentioned
this is typically captured \emph{via} some form of bisimulation.

% The notion we use in this paper is weak barbed bisimulation
% \cite{milner91polyadicpi}.

The notion we use in this paper is derived from weak barbed
bisimulation \cite{milner91polyadicpi}. 

\begin{definition}
An \emph{observation relation}, $\downarrow_{\mathcal N}$, over a set
of names, $\mathcal N$, is the smallest relation satisfying the rules
below.

\infrule[Out-barb]{y \in {\mathcal N}, \; x \nameeq y}
		  {\outputp{x}{v} \downarrow_{\mathcal N} x}
\infrule[Par-barb]{\mbox{$P\downarrow_{\mathcal N} x$ or $Q\downarrow_{\mathcal N} x$}}
		  {\binpar{P}{Q} \downarrow_{\mathcal N} x}

We write $P \Downarrow_{\mathcal N} x$ if there is $Q$ such that 
$P \wred Q$ and $Q \downarrow_{\mathcal N} x$.
\end{definition}

\begin{definition}
%\label{def.bbisim}
An  ${\mathcal N}$-\emph{barbed bisimulation} over a set of names, ${\mathcal N}$, is a symmetric binary relation 
${\mathcal S}_{\mathcal N}$ between agents such that $P\rel{S}_{\mathcal N}Q$ implies:
\begin{enumerate}
\item If $P \red P'$ then $Q \wred Q'$ and $P'\rel{S}_{\mathcal N} Q'$.
\item If $P\downarrow_{\mathcal N} x$, then $Q\Downarrow_{\mathcal N} x$.
\end{enumerate}
$P$ is ${\mathcal N}$-barbed bisimilar to $Q$, written
$P \wbbisim_{\mathcal N} Q$, if $P \rel{S}_{\mathcal N} Q$ for some ${\mathcal N}$-barbed bisimulation ${\mathcal S}_{\mathcal N}$.
\end{definition}

$\mathcal{R} \subseteq \pi \times \pi$

$P \mathcal{R} Q => \forall P'. P \red P' \Rightarrow \exists Q'. Q \red Q', P' \mathcal{R} Q'$

$P \vdash x \Rightarrow Q \vdash x$

\begin{mathpar}
  \inferrule*[lab=Out-barb]{x \nameeq y}{{y}!\langle{Q}\rangle \vdash x}
  \and
  \inferrule*[lab=Par-barb]{\mbox{$P\vdash x$ or $Q\vdash x$}}{\binpar{P}{Q} \vdash x}
\end{mathpar}

\subsubsection{Contexts}

One of the principle advantages of computational calculi like the
$\pi$-calculus is a well-defined notion of context,
contextual-equivalence and a correlation between
contextual-equivalence and notions of bisimulation. The notion of
context allows the decomposition of a process into (sub-)process and
its syntactic environment, its context. Thus, a context may be
thought of as a process with a ``hole'' (written $\Box$) in it. The
application of a context $M$ to a process $P$, written $M[P]$, is
tantamount to filling the hole in $M$ with $P$. In this paper we do
not need the full weight of this theory, but do make use of the notion
of context in the proof the main theorem. 

\begin{mathpar}
  \inferrule* [lab=summation] {} {{M_{M},M_{N}} \bc \Box \;|\; x.M_{A} \;|\; M_{M}+M_{N}}
  \and
  \inferrule* [lab=agent] {} {{M_{A}} \bc (\vec{x})M_{P} \;| \; \clift{P_0,\ldots,M_{P},\ldots,P_N}}
  \and \\
  \inferrule* [lab=process] {} {{M_{P}} \bc M_{N} \;| \;P|M_{P} }
\end{mathpar} 

\begin{mathpar}
  \inferrule* [lab=sychronization] {} {M_{N} \bc \Box \;|\; x?M_{F} \;|\; x!M_{C}}
  \and
  \inferrule* [lab=abstraction] {} {{M_{F}} \bc (x)M_{P} }
  \and
  \inferrule* [lab=concretion] {} {{M_{C}} \bc \langle M_{P} \rangle }
  \and \\
  \inferrule* [lab=process] {} {{M_{P}} \bc M_{N} \;| \;P|M_{P} }
\end{mathpar}

\begin{definition}[contextual application] Given a context $M$, and
  process $P$, we define the \emph{contextual application}, $M[P] :=
  M\{P/\Box\}$. That is, the contextual application of M to P is the
  substitution of $P$ for $\Box$ in $M$.
\end{definition}

$\meaningof{-} : L \to \mathcal{P}(\pi)$

\begin{mathpar}
  \inferrule* [lab=collection] {} {\meaningof{true} = \pi, \and \meaningof{~E} = \pi \setminus \meaningof{E}, \and \meaningof{E_{1} \& E_{2}} = \meaningof{E_{1}} \cap \meaningof{E_{2}}}
\end{mathpar}

\begin{mathpar}
  \inferrule* [lab=structure] {} {\meaningof{0} = \{ P \in \pi | P \equiv 0 \}, \and \\ \meaningof{E_1 | E_2} = \{ P \in \pi | P \equiv P_{1} | P_{2}, P_{1} \in \meaningof{E_{1}}, P_{2} \in \meaningof{E_2}\} }
\end{mathpar}

\begin{mathpar}
 \inferrule* [lab=behavior] {} {\meaningof{\langle a?b \rangle E} = \{ P \in \pi | P \equiv Q | u?(y)P', \\ \and \\\\ \and \\ \;\;\; u \in \meaningof{a}, \forall z.P'\{z/y\} \in \meaningof{E\{z/b\}}\}, \and \\ \meaningof{a!E} = \{ P \in \pi | P \equiv Q | x!\langle P' \rangle, x \in \meaningof{a} P' \in \meaningof{E}\} }
\end{mathpar}

\begin{mathpar}
 \inferrule* [lab=nominal] {} {\meaningof{\quotep{E}} = \{ \quotep{P} \in \quotep{\pi} | P \in \meaningof{E} \}, \and \meaningof{\quotep{P}} = \{ \quotep{Q} \in \quotep{\pi} | P \equiv Q \} \and \\ \meaningof{@\quotep{E}} = \{ P \in \pi | P \equiv @x, x \in \meaningof{E} \}}
\end{mathpar}

\begin{eqnarray*}
  \\
  \meaningof{-} : TS \to ST
\end{eqnarray*}

\begin{eqnarray*}
  \\
  L : TS \to ST
\end{eqnarray*}

\begin{eqnarray*}
  \\
  P \models E \iff P \in \meaningof{E}
\end{eqnarray*}

\begin{eqnarray*}
  P \approx_{L} Q \iff \forall E \in L. P \models E \iff Q \models E
\end{eqnarray*}

\begin{eqnarray*}
  P \approx_{K} Q
\end{eqnarray*}

\begin{eqnarray*}
  P \approx Q
\end{eqnarray*}

$\approx_{K} = \approx = \approx_{L}$

\subsubsection{Contextual duality}

Note that contexts extend the quotation operation to a family of
operations from processes to names. Given a context, $M$, we can
define a \emph{nominal context}, $\quotep{M}$ by $\quotep{M}[P] :=
\quotep{M[P]}$. To foreshadow what is to come we observe that these
operations enjoy a duality with processes very much like the duality
between vectors and maps from vectors to scalars.

Further, because the calculus is essentially higher-order, we have a
correspondence between contexts and processes. More specifically,
given a name $x$ and a context $M$ we can construct $M^{*}_{x}$ such
that 

\begin{mathpar}
  M^{*}_{x} | \lift{x}{P} \red M[P]
\end{mathpar}

namely,

\begin{mathpar}
  M^{*}_{x} := x?(u).M[\dropn{u}]
\end{mathpar}

The dependence of $M^{*}_{x}$ on a name makes it an abstraction, 

\begin{mathpar}
  M^{*} := (x)x?(u).M[\dropn{u}]
\end{mathpar}

\subsection{Additional notation}

It will sometimes be convenient to denote the process a name
quotes. We already have the notation $x = \quotep{P}$, but it will be
convenient to introduce an alternate notation, $\procn{x}$, when we
want to emphasize the connection to the use of the name. Note that, by
virtue of name equivalence, $\quotep{\procn{x}} \nameeq x$; so, the
notation is consistent with previous definitions.

Further, because names have structure it is possible to effect
substitutions on the basis of that structure. This means we need to
upgrade our notation for substitutions, which we accomplish by
adapting comprehension notation. Thus,

\begin{mathpar}
  P\{ y / x : x \in S \}
\end{mathpar}

is interpreted to mean the process derived from P by replacing (in a
capture-avoiding manner) each occurrence of $x$ in $S$ by $y$. For example,

\begin{mathpar}
  P\{ \quotep{\procn{x}|\procn{x}} / x : x \in \freenames{P} \}
\end{mathpar}

will replace each (occurrence) of a free name $x$ in $P$ by
$\quotep{\procn{x}|\procn{x}}$.

Also, we will avail ourselves of the notation $x^{L}$ and $x^{R}$ to
denote injections of a name into disjoint copies of the name
space. There are numerous ways to accomplish this. One example can be
found in \cite{MeredithR05}. This notation overloads to vectors of
names: $\vec{x}^{\pi} := (x_{i}^{\pi} \; : \; 0 \leq i < |\vec{x}| )$ where $\pi \in \{L,R\}$.

We also use $P^{\Box} := P|\Box$.

In \cite{MeredithR05} an interpretation of the new operator is
given. It turns out that there are several possible interpretations
all enjoying the requisite algebraic properties of the operator (see
\cite{milner91polyadicpi}). We will therefore make liberal use of
$(\nu\; \vec{x})P$.

% subsection the_syntax_and_semantics_of_the_notation_system (end)   

\input{qm2pi.qmops} 

\input{qm2pi.sterngerlach} 

\input{qm2pi.metric} 

% section concurrent_process_calculi (end)

%\input{qm2pi.proofsketch}

% section proof sketch (end)

%\input{qm2pi.slviaknots} 

% section spatial logic via knots (end)

\input{qm2pi.conclusion}

% section conclusion (end)

%\input{qm2pi.dtcodes} 

% section wiring algorithm (end)

\input{qm2pi.ack} 

% section acknowledgments (end)

\newpage


\bibliographystyle{plain}   
\bibliography{../../biblios/main.bib}

\input{qm2pi.rhodetails}

\end{document}

 

% section wiring algorithm (end)

\documentclass[12pt]{llncs}
%\documentclass{jktr}

\usepackage[pdftex]{hyperref}                   
\usepackage {listings}
\usepackage {mathpartir}
\usepackage{bcprules}
%\usepackage{listings}
                       
\usepackage{graphicx} 
%\usepackage[margins=2.5cm,nohead,nofoot]{geometry}
%\usepackage{geometry}
\usepackage{amsfonts}
\usepackage{amstext}
\usepackage{latexsym}
\usepackage{amssymb}
\usepackage{color}


%\include{myPreamble}
\include{qm2pi.local} 

%\ifpdf
%\usepackage[pdftex]{graphicx}
%\else
%\usepackage{graphicx}
%\fi

 % \ifpdf
%  \usepackage{pdfsync}
%  \if


%\title{Brief Article}
%\author{David F. Snyder}
%\author{L.G. Meredith}

%\address{Dept. of Math., Texas State University--San Marcos, San Marcos, TX 78666}
       
\pagestyle{empty}


\begin{document}

\lstset{language=[Objective]Caml,frame=shadowbox}

\input{qm2pi.front}

% section front matter (end)

\input{qm2pi.intro} 
 
% section introduction (end)

% \input{qm2pi.knotations} 

% section notation (end)

\input{qm2pi.process.calculi} 

% section concurrent_process_calculi_and_spatial_logics_ (end)
    
%\input{qm2pi.knots2pi} 

%\input{qm2pi.trefoil} 

%\input{qm2pi.mainthm} 

% subsection basic_interpretation (end)

%\input{qm2pi.rho.presentation} 
\subsection{The syntax and semantics of the notation system}\label{sub:the_syntax_and_semantics_of_the_notation_system} % (fold)

We now summarize a technical presentation of the calculus that
embodies our theory of dynamics. The typical presentation of such a
calculus follows the style of giving generators and relations on
them. The grammar, below, describing term constructors, freely
generates the set of processes, $\Proc$. This set is then quotiented
by a relation known as structural congruence and it is over this set
that the notion of dynamics is expressed. This presentation is
essentially that of \cite{MeredithR05} with the addition of
polyadicity and summation. For readability we have relegated some of
the technical subtleties to an appendix.

\subsubsection{Process grammar}\label{subsub:process_grammar}

\begin{mathpar}
  \inferrule* [lab=synchronization] {} {{M} \bc \pzero \;|\; x?F \;|\; x!C }
  \and
  \inferrule* [lab=abstraction] {} {{F} \bc (x)P}
  \and
  \inferrule* [lab=concretion] {} {{C} \bc \langle Q \rangle}
  \and
  \inferrule* [lab=process] {} {{P,Q} \bc M \;| \;P|Q \;|\; @{x}}
  \and
  \inferrule* [lab=name] {} {{x} \bc \quotep{P}}
\end{mathpar} 

Note that $\vec{x}$ (resp. $\vec{P}$) denotes a vector of names
(resp. processes) of length $|\vec{x}|$ (resp. $|\vec{P}|$). We adopt
the following useful abbreviations.

\begin{mathpar}
   x?(\vec{y}).P := x.(\vec{y})P \and  x\clift{\vec{P}} := x.\clift{\vec{P}}
   \and x!(y) := \lift{x}{\dropn{y}}
   \and \Pi_{i=0}^{n-1}P_i := P_0 | \ldots | P_{n-1}
\end{mathpar}

\subsubsection{Structural congruence}

\paragraph{Free and bound names and alpha-equivalence.} At the
core of structural equivalence is alpha-equivalence which identifies
process that are the same up to a change of variable. Formally, we
recognize the distinction between free and bound names. The free names
of a process, $\freenames{P}$, may be calculated recursively as
follows:

\begin{mathpar}
\freenames{\pzero} := \emptyset
  \and \\
  \freenames{x?(y).P} := \{ x \} \cup (\freenames{P} \setminus \{ y \})
  \and 
  \freenames{x!\langle P \rangle} := \{ x \} \cup \{ P \} 
  \and \\
  \freenames{P|Q} := \freenames{P} \cup \freenames{Q}
  \and \\
  \freenames{@{x}} := \{ x \}
\end{mathpar}

$\pi$
$\quotep{\pi}$

$\freenames{-} : \pi \to \mathcal{P}(\quotep{\pi})$

\begin{eqnarray*}
  \freenames{\pzero} & := & \emptyset \\
  \freenames{x?(y).P} & := & \{ x \} \cup (\freenames{P} \setminus \{ y \}) \\
  \freenames{x!\langle P \rangle} & := & \{ x \} \cup \{ P \} \\
  \freenames{P|Q} & := & \freenames{P} \cup \freenames{Q} \\
  \freenames{\dropn{x}} & := & \{ x \}
\end{eqnarray*}

The bound names of a process, $\boundnames{P}$, are those names occurring in $P$
that are not free. For example, in $x?(y).0$, the name $x$ is free, while $y$ is bound.

\begin{mathpar}
  \inferrule* [lab=monoidal-laws] {} { P|Q \equiv Q|P \and P|0 \equiv P \and P|(Q|R) \equiv (P|Q)|R }
\end{mathpar}

\begin{mathpar}
  \inferrule* [lab=alpha-equivalence] {} { (x)P \equiv (y)P\{y/x\} \and y \not\in \freenames{P} }
\end{mathpar}

\begin{definition}
Then two processes, $P,Q$, are alpha-equivalent if $P = Q\{\vec{y}/\vec{x}\}$ for
some $\vec{x} \in \boundnames{Q},\vec{y} \in \boundnames{P}$, where $Q\{\vec{y}/\vec{x}\}$
denotes the capture-avoiding substitution of $\vec{y}$ for $\vec{x}$ in $Q$.
\end{definition}

\begin{definition}
  The {\em structural congruence} \cite{SangiorgiWalker} , $\equiv$,
  between processes is the least congruence containing
  alpha-equivalence, satisfying the abelian monoid laws
  (associativity, commutativity and $\pzero$ as identity) for parallel
  composition $|$ and for summation $+$.
\end{definition}

\subsection{Name equivalence}

We take name equivalence, written $\nameeq$, to be the smallest
equivalence relation generated by the following rules.

\begin{mathpar}
\inferrule*[lab=Quote-drop]
{ }
{ \quotep{@{x}} \nameeq x }

\inferrule*[lab=Struct-equiv]
{ P \scong Q }
{ \quotep{P} \nameeq \quotep{Q} }
\end{mathpar}

The astute reader will have noticed that the mutual recursion of names
and processes imposes a mutual recursion on alpha-equivalence and
structural equivalence via name-equivalence. Fortunately, all of this
works out pleasantly and we may calculate in the natural way, free of
concern. The reader interested in the details is referred to the
appendix \ref{appendix:rho_details}.

\subsection{Substitution}

We use $\Proc$ for the set of processes, $\QProc$ for the set of
names, and $\id{\{}\vec{y} / \vec{x} \id{\}}$ to denote partial maps,
$s : \QProc \rightarrow \QProc$. A map, $s$ lifts, uniquely, to a map
on process terms, $\widehat{s} : \Proc \rightarrow \Proc$ by the
following equations.

\begin{mathpar}
  (0) \psubstp{Q}{P} := 0 \\
  (R \juxtap S) \psubstp{Q}{P}
  :=    
  (R)\psubstp{Q}{P} \juxtap (S) \psubstp{Q}{P} \\
  (x?(y).R) \psubstp{Q}{P}    
  :=    
  (x)\substp{Q}{P} (z)\concat( (R \psubstn{z}{y}) \psubstp{Q}{P} ) \\
  (\lift{x}{R}) \psubstp{Q}{P}  
  :=
  \lift{(x)\substp{Q}{P}}{ R \psubstp{Q}{P} } \\
%   (\dropn{x})  \psubstp{Q}{P}       
%   := 
%   \left\{ 
%     \begin{array}{ccc} 
%       \dropn{\quotep{Q}} & & x \nameeq \quotep{P} \\
%       \dropn{x} & & otherwise \\
%     \end{array}
%   \right. 
  (\dropn{x})  \psubstp{Q}{P}       
  := 
  \left\{ 
    \begin{array}{ccc} 
      Q & & x \nameeq \quotep{P} \\
      \dropn{x} & & otherwise \\
    \end{array}
  \right.
\end{mathpar}
 

where

\begin{eqnarray}
  (x)\id{\{} \lpquote Q \rpquote / \lpquote P \rpquote \id{\}}            = 
  \left\{ 
    \begin{array}{ccc}
      \lpquote Q \rpquote & & x \nameeq \lpquote P \rpquote \\
      x & & otherwise \\
    \end{array}
  \right. \nonumber
\end{eqnarray}

and $z$ is chosen distinct from $\quotep{P}$, $\quotep{Q}$, the free
names in $Q$, and all the names in $R$. Our $\alpha$-equivalence will
be built in the standard way from this substitution.

\begin{remark}\label{rem:no_self_referential_names}
  One consequence of these definitions is that $\forall P. \quotep{P}
  \not\in \freenames{P}$.
\end{remark}

\subsection{ Dynamic quote: an example }

Anticipating something of what's to come, consider applying the
substitution, $\widehat{\id{\{}u / z \id{\}}}$, to the following pair
of processes, $\lift{w}{y!(z)}$ and $w[ \lpquote y!(z) \rpquote ]$.

\begin{eqnarray}
	\lift{w}{y!(z)}\widehat{\id{\{}u / z \id{\}}}
		& = &
		\lift{w}{y!(u)} \nonumber\\
	w[ \lpquote y!(z) \rpquote ] \widehat{ \id{\{}u / z \id{\}} }
		& = &
		w[ \lpquote y!(z) \rpquote ] \nonumber
\end{eqnarray}

Because the body of the process between quotes is impervious to
substitution, we get radically different answers. In fact, by
examining the first process in an input context,
e.g. $x?(z).\lift{w}{y!(z)}$, we see that the process under the lift
operator may be shaped by prefixed inputs binding a name inside it. In
this sense, the lift operator will be seen as a way to dynamically
construct processes before reifying them as names.

Finally equipped with these standard features we can present the
dynamics of the calculus.

\subsubsection{Operational semantics} 

Finally, we introduce the computational dynamics. What marks these
algebras as distinct from other more traditionally studied algebraic
structures, e.g. vector spaces or polynomial rings, is the manner in
which dynamics is captured. In traditional structures, dynamics is typically
expressed through morphisms between such structures, as in linear maps
between vector spaces or morphisms between rings. In algebras
associated with the semantics of computation, the dynamics is
expressed as part of the algebraic structure itself, through a
reduction reduction relation typically denoted by $\red$. Below, we
give a recursive presentation of this relation for the calculus used
in the encoding.

$\red \subseteq \pi \times \pi$
$\red : \pi \to \mathcal{P}(\pi)$

\begin{mathpar}
  \inferrule* [lab=Comm] { \textsf{match}( x_{src}, x_{trgt} ) } { x_{trgt}?(y)P \; | \; x_{src}!\langle {Q} \rangle \red P\{\quotep{Q}/y}\} }
  \and \\
  \inferrule* [lab=Par] {{P} \red {P}'} {{{P} | {Q}} \red {{P}' | {Q}}}
  \and
  \inferrule* [lab=Equiv]{{{P} \scong {P}'} \andalso {{P}' \red {Q}'} \andalso {{Q}' \scong {Q}}}{{P} \red {Q}}
\end{mathpar}

\begin{eqnarray*}
  match_{\equiv} (\quotep{P},\quotep{Q}) & := & P \equiv Q \\
  match_{\dagger}(\quotep{P},\quotep{Q}) & := & \forall R. P|Q \red^{*} R => R \red^{*} 0 \\
  match_{K}(\quotep{P},\quotep{Q}) & := & K \mbox{ for some context } K
\end{eqnarray*}

$u?(x)P | u!\langle Q \rangle \red P\{\quotep{Q}/x\}$

%We write $\wred$ for $\red^*$, and $P\red$ if $\exists Q $ such that $ P \red Q$.
We write $P\red$ if $\exists Q $ such that $ P \red Q$ and $P\not\red$, otherwise.

\section{Replication}

As mentioned before, it is known that replication (and hence
recursion) can be implemented in a higher-order process algebra
\cite{SangiorgiWalker}. As our first example of calculation with the
machinery thus far presented we give the construction explicitly in
the {\rhoc}.

\begin{eqnarray}
	D_{x} & := & \prefix{x}{y}{(\binpar{\outputp{x}{y}}{@{y}})} \nonumber\\
	\bangp_{x}{P} & := & \binpar{{x}!\langle{\binpar{D_{x}}{P}}\rangle}{D_{x}} \nonumber
\end{eqnarray}

\begin{eqnarray}
	\bangp_{x}{P} & & \nonumber\\
	=
	& {x}!\langle{(\prefix{x}{y}{(\outputp{x}{y} | @{y})) | P}}\rangle 
	      | \prefix{x}{y}{(\outputp{x}{y} | @{y})} & \nonumber\\
	\red
	& (\outputp{x}{y} | @{y})\substn{\quotep{(\prefix{x}{y}{(@{y} | \outputp{x}{y})) | P}}}{y} & \nonumber\\
	=
	& \outputp{x}{\quotep{(\prefix{x}{y}{(\outputp{x}{y} | @{y})) | P}}}
	  | {(\prefix{x}{y}{(\outputp{x}{y} | @{y})) | P}} & \nonumber\\
	\red
	& \ldots & \nonumber\\
	\red^*
	& P | P | \ldots & \nonumber
\end{eqnarray}

Of course, this encoding, as an implementation, runs away, unfolding
$\bangp{P}$ eagerly. A lazier and more implementable replication
operator, restricted to input-guarded processes, may be obtained as follows.

\begin{eqnarray}
\bangp{\prefix{u}{v}{P}} 
	:= 
	\binpar{\lift{x}{\prefix{u}{v}{(\binpar{D(x)}{P})}}}{D(x)} \nonumber
\end{eqnarray}

\begin{remark}
  Note that the lazier definition still does not deal with summation
  or mixed summation (i.e. sums over input and output). The reader is
  invited to construct definitions of replication that deal with these
  features. 

  Further, the definitions are parameterized in a name, $x$. Can you,
  gentle reader, make a definition that eliminates this parameter and
  guarantees no accidental interaction between the replication
  machinery and the process being replicated -- i.e. no accidental
  sharing of names used by the process to get its work done and the
  name(s) used by the replication to effect copying. This latter
  revision of the definition of replication is crucial to obtaining
  the expected identity $!!P \sim !P$.
\end{remark}

\begin{remark}\label{rem:paradoxical_combinator}
  The reader familiar with the lambda calculus will have noticed the
  similarity between $D$ and the paradoxical combinator.

  [Ed. note: the existence of this seems to suggest we have to be more
  restrictive on the set of processes and names we admit if we are to
  support no-cloning.]
\end{remark}

\subsubsection{Bisimulation}

The computational dynamics gives rise to another kind of equivalence,
the equivalence of computational behavior. As previously mentioned
this is typically captured \emph{via} some form of bisimulation.

% The notion we use in this paper is weak barbed bisimulation
% \cite{milner91polyadicpi}.

The notion we use in this paper is derived from weak barbed
bisimulation \cite{milner91polyadicpi}. 

\begin{definition}
An \emph{observation relation}, $\downarrow_{\mathcal N}$, over a set
of names, $\mathcal N$, is the smallest relation satisfying the rules
below.

\infrule[Out-barb]{y \in {\mathcal N}, \; x \nameeq y}
		  {\outputp{x}{v} \downarrow_{\mathcal N} x}
\infrule[Par-barb]{\mbox{$P\downarrow_{\mathcal N} x$ or $Q\downarrow_{\mathcal N} x$}}
		  {\binpar{P}{Q} \downarrow_{\mathcal N} x}

We write $P \Downarrow_{\mathcal N} x$ if there is $Q$ such that 
$P \wred Q$ and $Q \downarrow_{\mathcal N} x$.
\end{definition}

\begin{definition}
%\label{def.bbisim}
An  ${\mathcal N}$-\emph{barbed bisimulation} over a set of names, ${\mathcal N}$, is a symmetric binary relation 
${\mathcal S}_{\mathcal N}$ between agents such that $P\rel{S}_{\mathcal N}Q$ implies:
\begin{enumerate}
\item If $P \red P'$ then $Q \wred Q'$ and $P'\rel{S}_{\mathcal N} Q'$.
\item If $P\downarrow_{\mathcal N} x$, then $Q\Downarrow_{\mathcal N} x$.
\end{enumerate}
$P$ is ${\mathcal N}$-barbed bisimilar to $Q$, written
$P \wbbisim_{\mathcal N} Q$, if $P \rel{S}_{\mathcal N} Q$ for some ${\mathcal N}$-barbed bisimulation ${\mathcal S}_{\mathcal N}$.
\end{definition}

$\mathcal{R} \subseteq \pi \times \pi$

$P \mathcal{R} Q => \forall P'. P \red P' \Rightarrow \exists Q'. Q \red Q', P' \mathcal{R} Q'$

$P \vdash x \Rightarrow Q \vdash x$

\begin{mathpar}
  \inferrule*[lab=Out-barb]{x \nameeq y}{{y}!\langle{Q}\rangle \vdash x}
  \and
  \inferrule*[lab=Par-barb]{\mbox{$P\vdash x$ or $Q\vdash x$}}{\binpar{P}{Q} \vdash x}
\end{mathpar}

\subsubsection{Contexts}

One of the principle advantages of computational calculi like the
$\pi$-calculus is a well-defined notion of context,
contextual-equivalence and a correlation between
contextual-equivalence and notions of bisimulation. The notion of
context allows the decomposition of a process into (sub-)process and
its syntactic environment, its context. Thus, a context may be
thought of as a process with a ``hole'' (written $\Box$) in it. The
application of a context $M$ to a process $P$, written $M[P]$, is
tantamount to filling the hole in $M$ with $P$. In this paper we do
not need the full weight of this theory, but do make use of the notion
of context in the proof the main theorem. 

\begin{mathpar}
  \inferrule* [lab=summation] {} {{M_{M},M_{N}} \bc \Box \;|\; x.M_{A} \;|\; M_{M}+M_{N}}
  \and
  \inferrule* [lab=agent] {} {{M_{A}} \bc (\vec{x})M_{P} \;| \; \clift{P_0,\ldots,M_{P},\ldots,P_N}}
  \and \\
  \inferrule* [lab=process] {} {{M_{P}} \bc M_{N} \;| \;P|M_{P} }
\end{mathpar} 

\begin{mathpar}
  \inferrule* [lab=sychronization] {} {M_{N} \bc \Box \;|\; x?M_{F} \;|\; x!M_{C}}
  \and
  \inferrule* [lab=abstraction] {} {{M_{F}} \bc (x)M_{P} }
  \and
  \inferrule* [lab=concretion] {} {{M_{C}} \bc \langle M_{P} \rangle }
  \and \\
  \inferrule* [lab=process] {} {{M_{P}} \bc M_{N} \;| \;P|M_{P} }
\end{mathpar}

\begin{definition}[contextual application] Given a context $M$, and
  process $P$, we define the \emph{contextual application}, $M[P] :=
  M\{P/\Box\}$. That is, the contextual application of M to P is the
  substitution of $P$ for $\Box$ in $M$.
\end{definition}

$\meaningof{-} : L \to \mathcal{P}(\pi)$

\begin{mathpar}
  \inferrule* [lab=collection] {} {\meaningof{true} = \pi, \and \meaningof{~E} = \pi \setminus \meaningof{E}, \and \meaningof{E_{1} \& E_{2}} = \meaningof{E_{1}} \cap \meaningof{E_{2}}}
\end{mathpar}

\begin{mathpar}
  \inferrule* [lab=structure] {} {\meaningof{0} = \{ P \in \pi | P \equiv 0 \}, \and \\ \meaningof{E_1 | E_2} = \{ P \in \pi | P \equiv P_{1} | P_{2}, P_{1} \in \meaningof{E_{1}}, P_{2} \in \meaningof{E_2}\} }
\end{mathpar}

\begin{mathpar}
 \inferrule* [lab=behavior] {} {\meaningof{\langle a?b \rangle E} = \{ P \in \pi | P \equiv Q | u?(y)P', \\ \and \\\\ \and \\ \;\;\; u \in \meaningof{a}, \forall z.P'\{z/y\} \in \meaningof{E\{z/b\}}\}, \and \\ \meaningof{a!E} = \{ P \in \pi | P \equiv Q | x!\langle P' \rangle, x \in \meaningof{a} P' \in \meaningof{E}\} }
\end{mathpar}

\begin{mathpar}
 \inferrule* [lab=nominal] {} {\meaningof{\quotep{E}} = \{ \quotep{P} \in \quotep{\pi} | P \in \meaningof{E} \}, \and \meaningof{\quotep{P}} = \{ \quotep{Q} \in \quotep{\pi} | P \equiv Q \} \and \\ \meaningof{@\quotep{E}} = \{ P \in \pi | P \equiv @x, x \in \meaningof{E} \}}
\end{mathpar}

\begin{eqnarray*}
  \\
  \meaningof{-} : TS \to ST
\end{eqnarray*}

\begin{eqnarray*}
  \\
  L : TS \to ST
\end{eqnarray*}

\begin{eqnarray*}
  \\
  P \models E \iff P \in \meaningof{E}
\end{eqnarray*}

\begin{eqnarray*}
  P \approx_{L} Q \iff \forall E \in L. P \models E \iff Q \models E
\end{eqnarray*}

\begin{eqnarray*}
  P \approx_{K} Q
\end{eqnarray*}

\begin{eqnarray*}
  P \approx Q
\end{eqnarray*}

$\approx_{K} = \approx = \approx_{L}$

\subsubsection{Contextual duality}

Note that contexts extend the quotation operation to a family of
operations from processes to names. Given a context, $M$, we can
define a \emph{nominal context}, $\quotep{M}$ by $\quotep{M}[P] :=
\quotep{M[P]}$. To foreshadow what is to come we observe that these
operations enjoy a duality with processes very much like the duality
between vectors and maps from vectors to scalars.

Further, because the calculus is essentially higher-order, we have a
correspondence between contexts and processes. More specifically,
given a name $x$ and a context $M$ we can construct $M^{*}_{x}$ such
that 

\begin{mathpar}
  M^{*}_{x} | \lift{x}{P} \red M[P]
\end{mathpar}

namely,

\begin{mathpar}
  M^{*}_{x} := x?(u).M[\dropn{u}]
\end{mathpar}

The dependence of $M^{*}_{x}$ on a name makes it an abstraction, 

\begin{mathpar}
  M^{*} := (x)x?(u).M[\dropn{u}]
\end{mathpar}

\subsection{Additional notation}

It will sometimes be convenient to denote the process a name
quotes. We already have the notation $x = \quotep{P}$, but it will be
convenient to introduce an alternate notation, $\procn{x}$, when we
want to emphasize the connection to the use of the name. Note that, by
virtue of name equivalence, $\quotep{\procn{x}} \nameeq x$; so, the
notation is consistent with previous definitions.

Further, because names have structure it is possible to effect
substitutions on the basis of that structure. This means we need to
upgrade our notation for substitutions, which we accomplish by
adapting comprehension notation. Thus,

\begin{mathpar}
  P\{ y / x : x \in S \}
\end{mathpar}

is interpreted to mean the process derived from P by replacing (in a
capture-avoiding manner) each occurrence of $x$ in $S$ by $y$. For example,

\begin{mathpar}
  P\{ \quotep{\procn{x}|\procn{x}} / x : x \in \freenames{P} \}
\end{mathpar}

will replace each (occurrence) of a free name $x$ in $P$ by
$\quotep{\procn{x}|\procn{x}}$.

Also, we will avail ourselves of the notation $x^{L}$ and $x^{R}$ to
denote injections of a name into disjoint copies of the name
space. There are numerous ways to accomplish this. One example can be
found in \cite{MeredithR05}. This notation overloads to vectors of
names: $\vec{x}^{\pi} := (x_{i}^{\pi} \; : \; 0 \leq i < |\vec{x}| )$ where $\pi \in \{L,R\}$.

We also use $P^{\Box} := P|\Box$.

In \cite{MeredithR05} an interpretation of the new operator is
given. It turns out that there are several possible interpretations
all enjoying the requisite algebraic properties of the operator (see
\cite{milner91polyadicpi}). We will therefore make liberal use of
$(\nu\; \vec{x})P$.

% subsection the_syntax_and_semantics_of_the_notation_system (end)   

\input{qm2pi.qmops} 

\input{qm2pi.sterngerlach} 

\input{qm2pi.metric} 

% section concurrent_process_calculi (end)

%\input{qm2pi.proofsketch}

% section proof sketch (end)

%\input{qm2pi.slviaknots} 

% section spatial logic via knots (end)

\input{qm2pi.conclusion}

% section conclusion (end)

%\input{qm2pi.dtcodes} 

% section wiring algorithm (end)

\input{qm2pi.ack} 

% section acknowledgments (end)

\newpage


\bibliographystyle{plain}   
\bibliography{../../biblios/main.bib}

\input{qm2pi.rhodetails}

\end{document}

 

% section acknowledgments (end)

\newpage


\bibliographystyle{plain}   
\bibliography{../../biblios/main.bib}

\documentclass[12pt]{llncs}
%\documentclass{jktr}

\usepackage[pdftex]{hyperref}                   
\usepackage {listings}
\usepackage {mathpartir}
\usepackage{bcprules}
%\usepackage{listings}
                       
\usepackage{graphicx} 
%\usepackage[margins=2.5cm,nohead,nofoot]{geometry}
%\usepackage{geometry}
\usepackage{amsfonts}
\usepackage{amstext}
\usepackage{latexsym}
\usepackage{amssymb}
\usepackage{color}


%\include{myPreamble}
\include{qm2pi.local} 

%\ifpdf
%\usepackage[pdftex]{graphicx}
%\else
%\usepackage{graphicx}
%\fi

 % \ifpdf
%  \usepackage{pdfsync}
%  \if


%\title{Brief Article}
%\author{David F. Snyder}
%\author{L.G. Meredith}

%\address{Dept. of Math., Texas State University--San Marcos, San Marcos, TX 78666}
       
\pagestyle{empty}


\begin{document}

\lstset{language=[Objective]Caml,frame=shadowbox}

\input{qm2pi.front}

% section front matter (end)

\input{qm2pi.intro} 
 
% section introduction (end)

% \input{qm2pi.knotations} 

% section notation (end)

\input{qm2pi.process.calculi} 

% section concurrent_process_calculi_and_spatial_logics_ (end)
    
%\input{qm2pi.knots2pi} 

%\input{qm2pi.trefoil} 

%\input{qm2pi.mainthm} 

% subsection basic_interpretation (end)

%\input{qm2pi.rho.presentation} 
\subsection{The syntax and semantics of the notation system}\label{sub:the_syntax_and_semantics_of_the_notation_system} % (fold)

We now summarize a technical presentation of the calculus that
embodies our theory of dynamics. The typical presentation of such a
calculus follows the style of giving generators and relations on
them. The grammar, below, describing term constructors, freely
generates the set of processes, $\Proc$. This set is then quotiented
by a relation known as structural congruence and it is over this set
that the notion of dynamics is expressed. This presentation is
essentially that of \cite{MeredithR05} with the addition of
polyadicity and summation. For readability we have relegated some of
the technical subtleties to an appendix.

\subsubsection{Process grammar}\label{subsub:process_grammar}

\begin{mathpar}
  \inferrule* [lab=synchronization] {} {{M} \bc \pzero \;|\; x?F \;|\; x!C }
  \and
  \inferrule* [lab=abstraction] {} {{F} \bc (x)P}
  \and
  \inferrule* [lab=concretion] {} {{C} \bc \langle Q \rangle}
  \and
  \inferrule* [lab=process] {} {{P,Q} \bc M \;| \;P|Q \;|\; @{x}}
  \and
  \inferrule* [lab=name] {} {{x} \bc \quotep{P}}
\end{mathpar} 

Note that $\vec{x}$ (resp. $\vec{P}$) denotes a vector of names
(resp. processes) of length $|\vec{x}|$ (resp. $|\vec{P}|$). We adopt
the following useful abbreviations.

\begin{mathpar}
   x?(\vec{y}).P := x.(\vec{y})P \and  x\clift{\vec{P}} := x.\clift{\vec{P}}
   \and x!(y) := \lift{x}{\dropn{y}}
   \and \Pi_{i=0}^{n-1}P_i := P_0 | \ldots | P_{n-1}
\end{mathpar}

\subsubsection{Structural congruence}

\paragraph{Free and bound names and alpha-equivalence.} At the
core of structural equivalence is alpha-equivalence which identifies
process that are the same up to a change of variable. Formally, we
recognize the distinction between free and bound names. The free names
of a process, $\freenames{P}$, may be calculated recursively as
follows:

\begin{mathpar}
\freenames{\pzero} := \emptyset
  \and \\
  \freenames{x?(y).P} := \{ x \} \cup (\freenames{P} \setminus \{ y \})
  \and 
  \freenames{x!\langle P \rangle} := \{ x \} \cup \{ P \} 
  \and \\
  \freenames{P|Q} := \freenames{P} \cup \freenames{Q}
  \and \\
  \freenames{@{x}} := \{ x \}
\end{mathpar}

$\pi$
$\quotep{\pi}$

$\freenames{-} : \pi \to \mathcal{P}(\quotep{\pi})$

\begin{eqnarray*}
  \freenames{\pzero} & := & \emptyset \\
  \freenames{x?(y).P} & := & \{ x \} \cup (\freenames{P} \setminus \{ y \}) \\
  \freenames{x!\langle P \rangle} & := & \{ x \} \cup \{ P \} \\
  \freenames{P|Q} & := & \freenames{P} \cup \freenames{Q} \\
  \freenames{\dropn{x}} & := & \{ x \}
\end{eqnarray*}

The bound names of a process, $\boundnames{P}$, are those names occurring in $P$
that are not free. For example, in $x?(y).0$, the name $x$ is free, while $y$ is bound.

\begin{mathpar}
  \inferrule* [lab=monoidal-laws] {} { P|Q \equiv Q|P \and P|0 \equiv P \and P|(Q|R) \equiv (P|Q)|R }
\end{mathpar}

\begin{mathpar}
  \inferrule* [lab=alpha-equivalence] {} { (x)P \equiv (y)P\{y/x\} \and y \not\in \freenames{P} }
\end{mathpar}

\begin{definition}
Then two processes, $P,Q$, are alpha-equivalent if $P = Q\{\vec{y}/\vec{x}\}$ for
some $\vec{x} \in \boundnames{Q},\vec{y} \in \boundnames{P}$, where $Q\{\vec{y}/\vec{x}\}$
denotes the capture-avoiding substitution of $\vec{y}$ for $\vec{x}$ in $Q$.
\end{definition}

\begin{definition}
  The {\em structural congruence} \cite{SangiorgiWalker} , $\equiv$,
  between processes is the least congruence containing
  alpha-equivalence, satisfying the abelian monoid laws
  (associativity, commutativity and $\pzero$ as identity) for parallel
  composition $|$ and for summation $+$.
\end{definition}

\subsection{Name equivalence}

We take name equivalence, written $\nameeq$, to be the smallest
equivalence relation generated by the following rules.

\begin{mathpar}
\inferrule*[lab=Quote-drop]
{ }
{ \quotep{@{x}} \nameeq x }

\inferrule*[lab=Struct-equiv]
{ P \scong Q }
{ \quotep{P} \nameeq \quotep{Q} }
\end{mathpar}

The astute reader will have noticed that the mutual recursion of names
and processes imposes a mutual recursion on alpha-equivalence and
structural equivalence via name-equivalence. Fortunately, all of this
works out pleasantly and we may calculate in the natural way, free of
concern. The reader interested in the details is referred to the
appendix \ref{appendix:rho_details}.

\subsection{Substitution}

We use $\Proc$ for the set of processes, $\QProc$ for the set of
names, and $\id{\{}\vec{y} / \vec{x} \id{\}}$ to denote partial maps,
$s : \QProc \rightarrow \QProc$. A map, $s$ lifts, uniquely, to a map
on process terms, $\widehat{s} : \Proc \rightarrow \Proc$ by the
following equations.

\begin{mathpar}
  (0) \psubstp{Q}{P} := 0 \\
  (R \juxtap S) \psubstp{Q}{P}
  :=    
  (R)\psubstp{Q}{P} \juxtap (S) \psubstp{Q}{P} \\
  (x?(y).R) \psubstp{Q}{P}    
  :=    
  (x)\substp{Q}{P} (z)\concat( (R \psubstn{z}{y}) \psubstp{Q}{P} ) \\
  (\lift{x}{R}) \psubstp{Q}{P}  
  :=
  \lift{(x)\substp{Q}{P}}{ R \psubstp{Q}{P} } \\
%   (\dropn{x})  \psubstp{Q}{P}       
%   := 
%   \left\{ 
%     \begin{array}{ccc} 
%       \dropn{\quotep{Q}} & & x \nameeq \quotep{P} \\
%       \dropn{x} & & otherwise \\
%     \end{array}
%   \right. 
  (\dropn{x})  \psubstp{Q}{P}       
  := 
  \left\{ 
    \begin{array}{ccc} 
      Q & & x \nameeq \quotep{P} \\
      \dropn{x} & & otherwise \\
    \end{array}
  \right.
\end{mathpar}
 

where

\begin{eqnarray}
  (x)\id{\{} \lpquote Q \rpquote / \lpquote P \rpquote \id{\}}            = 
  \left\{ 
    \begin{array}{ccc}
      \lpquote Q \rpquote & & x \nameeq \lpquote P \rpquote \\
      x & & otherwise \\
    \end{array}
  \right. \nonumber
\end{eqnarray}

and $z$ is chosen distinct from $\quotep{P}$, $\quotep{Q}$, the free
names in $Q$, and all the names in $R$. Our $\alpha$-equivalence will
be built in the standard way from this substitution.

\begin{remark}\label{rem:no_self_referential_names}
  One consequence of these definitions is that $\forall P. \quotep{P}
  \not\in \freenames{P}$.
\end{remark}

\subsection{ Dynamic quote: an example }

Anticipating something of what's to come, consider applying the
substitution, $\widehat{\id{\{}u / z \id{\}}}$, to the following pair
of processes, $\lift{w}{y!(z)}$ and $w[ \lpquote y!(z) \rpquote ]$.

\begin{eqnarray}
	\lift{w}{y!(z)}\widehat{\id{\{}u / z \id{\}}}
		& = &
		\lift{w}{y!(u)} \nonumber\\
	w[ \lpquote y!(z) \rpquote ] \widehat{ \id{\{}u / z \id{\}} }
		& = &
		w[ \lpquote y!(z) \rpquote ] \nonumber
\end{eqnarray}

Because the body of the process between quotes is impervious to
substitution, we get radically different answers. In fact, by
examining the first process in an input context,
e.g. $x?(z).\lift{w}{y!(z)}$, we see that the process under the lift
operator may be shaped by prefixed inputs binding a name inside it. In
this sense, the lift operator will be seen as a way to dynamically
construct processes before reifying them as names.

Finally equipped with these standard features we can present the
dynamics of the calculus.

\subsubsection{Operational semantics} 

Finally, we introduce the computational dynamics. What marks these
algebras as distinct from other more traditionally studied algebraic
structures, e.g. vector spaces or polynomial rings, is the manner in
which dynamics is captured. In traditional structures, dynamics is typically
expressed through morphisms between such structures, as in linear maps
between vector spaces or morphisms between rings. In algebras
associated with the semantics of computation, the dynamics is
expressed as part of the algebraic structure itself, through a
reduction reduction relation typically denoted by $\red$. Below, we
give a recursive presentation of this relation for the calculus used
in the encoding.

$\red \subseteq \pi \times \pi$
$\red : \pi \to \mathcal{P}(\pi)$

\begin{mathpar}
  \inferrule* [lab=Comm] { \textsf{match}( x_{src}, x_{trgt} ) } { x_{trgt}?(y)P \; | \; x_{src}!\langle {Q} \rangle \red P\{\quotep{Q}/y}\} }
  \and \\
  \inferrule* [lab=Par] {{P} \red {P}'} {{{P} | {Q}} \red {{P}' | {Q}}}
  \and
  \inferrule* [lab=Equiv]{{{P} \scong {P}'} \andalso {{P}' \red {Q}'} \andalso {{Q}' \scong {Q}}}{{P} \red {Q}}
\end{mathpar}

\begin{eqnarray*}
  match_{\equiv} (\quotep{P},\quotep{Q}) & := & P \equiv Q \\
  match_{\dagger}(\quotep{P},\quotep{Q}) & := & \forall R. P|Q \red^{*} R => R \red^{*} 0 \\
  match_{K}(\quotep{P},\quotep{Q}) & := & K \mbox{ for some context } K
\end{eqnarray*}

$u?(x)P | u!\langle Q \rangle \red P\{\quotep{Q}/x\}$

%We write $\wred$ for $\red^*$, and $P\red$ if $\exists Q $ such that $ P \red Q$.
We write $P\red$ if $\exists Q $ such that $ P \red Q$ and $P\not\red$, otherwise.

\section{Replication}

As mentioned before, it is known that replication (and hence
recursion) can be implemented in a higher-order process algebra
\cite{SangiorgiWalker}. As our first example of calculation with the
machinery thus far presented we give the construction explicitly in
the {\rhoc}.

\begin{eqnarray}
	D_{x} & := & \prefix{x}{y}{(\binpar{\outputp{x}{y}}{@{y}})} \nonumber\\
	\bangp_{x}{P} & := & \binpar{{x}!\langle{\binpar{D_{x}}{P}}\rangle}{D_{x}} \nonumber
\end{eqnarray}

\begin{eqnarray}
	\bangp_{x}{P} & & \nonumber\\
	=
	& {x}!\langle{(\prefix{x}{y}{(\outputp{x}{y} | @{y})) | P}}\rangle 
	      | \prefix{x}{y}{(\outputp{x}{y} | @{y})} & \nonumber\\
	\red
	& (\outputp{x}{y} | @{y})\substn{\quotep{(\prefix{x}{y}{(@{y} | \outputp{x}{y})) | P}}}{y} & \nonumber\\
	=
	& \outputp{x}{\quotep{(\prefix{x}{y}{(\outputp{x}{y} | @{y})) | P}}}
	  | {(\prefix{x}{y}{(\outputp{x}{y} | @{y})) | P}} & \nonumber\\
	\red
	& \ldots & \nonumber\\
	\red^*
	& P | P | \ldots & \nonumber
\end{eqnarray}

Of course, this encoding, as an implementation, runs away, unfolding
$\bangp{P}$ eagerly. A lazier and more implementable replication
operator, restricted to input-guarded processes, may be obtained as follows.

\begin{eqnarray}
\bangp{\prefix{u}{v}{P}} 
	:= 
	\binpar{\lift{x}{\prefix{u}{v}{(\binpar{D(x)}{P})}}}{D(x)} \nonumber
\end{eqnarray}

\begin{remark}
  Note that the lazier definition still does not deal with summation
  or mixed summation (i.e. sums over input and output). The reader is
  invited to construct definitions of replication that deal with these
  features. 

  Further, the definitions are parameterized in a name, $x$. Can you,
  gentle reader, make a definition that eliminates this parameter and
  guarantees no accidental interaction between the replication
  machinery and the process being replicated -- i.e. no accidental
  sharing of names used by the process to get its work done and the
  name(s) used by the replication to effect copying. This latter
  revision of the definition of replication is crucial to obtaining
  the expected identity $!!P \sim !P$.
\end{remark}

\begin{remark}\label{rem:paradoxical_combinator}
  The reader familiar with the lambda calculus will have noticed the
  similarity between $D$ and the paradoxical combinator.

  [Ed. note: the existence of this seems to suggest we have to be more
  restrictive on the set of processes and names we admit if we are to
  support no-cloning.]
\end{remark}

\subsubsection{Bisimulation}

The computational dynamics gives rise to another kind of equivalence,
the equivalence of computational behavior. As previously mentioned
this is typically captured \emph{via} some form of bisimulation.

% The notion we use in this paper is weak barbed bisimulation
% \cite{milner91polyadicpi}.

The notion we use in this paper is derived from weak barbed
bisimulation \cite{milner91polyadicpi}. 

\begin{definition}
An \emph{observation relation}, $\downarrow_{\mathcal N}$, over a set
of names, $\mathcal N$, is the smallest relation satisfying the rules
below.

\infrule[Out-barb]{y \in {\mathcal N}, \; x \nameeq y}
		  {\outputp{x}{v} \downarrow_{\mathcal N} x}
\infrule[Par-barb]{\mbox{$P\downarrow_{\mathcal N} x$ or $Q\downarrow_{\mathcal N} x$}}
		  {\binpar{P}{Q} \downarrow_{\mathcal N} x}

We write $P \Downarrow_{\mathcal N} x$ if there is $Q$ such that 
$P \wred Q$ and $Q \downarrow_{\mathcal N} x$.
\end{definition}

\begin{definition}
%\label{def.bbisim}
An  ${\mathcal N}$-\emph{barbed bisimulation} over a set of names, ${\mathcal N}$, is a symmetric binary relation 
${\mathcal S}_{\mathcal N}$ between agents such that $P\rel{S}_{\mathcal N}Q$ implies:
\begin{enumerate}
\item If $P \red P'$ then $Q \wred Q'$ and $P'\rel{S}_{\mathcal N} Q'$.
\item If $P\downarrow_{\mathcal N} x$, then $Q\Downarrow_{\mathcal N} x$.
\end{enumerate}
$P$ is ${\mathcal N}$-barbed bisimilar to $Q$, written
$P \wbbisim_{\mathcal N} Q$, if $P \rel{S}_{\mathcal N} Q$ for some ${\mathcal N}$-barbed bisimulation ${\mathcal S}_{\mathcal N}$.
\end{definition}

$\mathcal{R} \subseteq \pi \times \pi$

$P \mathcal{R} Q => \forall P'. P \red P' \Rightarrow \exists Q'. Q \red Q', P' \mathcal{R} Q'$

$P \vdash x \Rightarrow Q \vdash x$

\begin{mathpar}
  \inferrule*[lab=Out-barb]{x \nameeq y}{{y}!\langle{Q}\rangle \vdash x}
  \and
  \inferrule*[lab=Par-barb]{\mbox{$P\vdash x$ or $Q\vdash x$}}{\binpar{P}{Q} \vdash x}
\end{mathpar}

\subsubsection{Contexts}

One of the principle advantages of computational calculi like the
$\pi$-calculus is a well-defined notion of context,
contextual-equivalence and a correlation between
contextual-equivalence and notions of bisimulation. The notion of
context allows the decomposition of a process into (sub-)process and
its syntactic environment, its context. Thus, a context may be
thought of as a process with a ``hole'' (written $\Box$) in it. The
application of a context $M$ to a process $P$, written $M[P]$, is
tantamount to filling the hole in $M$ with $P$. In this paper we do
not need the full weight of this theory, but do make use of the notion
of context in the proof the main theorem. 

\begin{mathpar}
  \inferrule* [lab=summation] {} {{M_{M},M_{N}} \bc \Box \;|\; x.M_{A} \;|\; M_{M}+M_{N}}
  \and
  \inferrule* [lab=agent] {} {{M_{A}} \bc (\vec{x})M_{P} \;| \; \clift{P_0,\ldots,M_{P},\ldots,P_N}}
  \and \\
  \inferrule* [lab=process] {} {{M_{P}} \bc M_{N} \;| \;P|M_{P} }
\end{mathpar} 

\begin{mathpar}
  \inferrule* [lab=sychronization] {} {M_{N} \bc \Box \;|\; x?M_{F} \;|\; x!M_{C}}
  \and
  \inferrule* [lab=abstraction] {} {{M_{F}} \bc (x)M_{P} }
  \and
  \inferrule* [lab=concretion] {} {{M_{C}} \bc \langle M_{P} \rangle }
  \and \\
  \inferrule* [lab=process] {} {{M_{P}} \bc M_{N} \;| \;P|M_{P} }
\end{mathpar}

\begin{definition}[contextual application] Given a context $M$, and
  process $P$, we define the \emph{contextual application}, $M[P] :=
  M\{P/\Box\}$. That is, the contextual application of M to P is the
  substitution of $P$ for $\Box$ in $M$.
\end{definition}

$\meaningof{-} : L \to \mathcal{P}(\pi)$

\begin{mathpar}
  \inferrule* [lab=collection] {} {\meaningof{true} = \pi, \and \meaningof{~E} = \pi \setminus \meaningof{E}, \and \meaningof{E_{1} \& E_{2}} = \meaningof{E_{1}} \cap \meaningof{E_{2}}}
\end{mathpar}

\begin{mathpar}
  \inferrule* [lab=structure] {} {\meaningof{0} = \{ P \in \pi | P \equiv 0 \}, \and \\ \meaningof{E_1 | E_2} = \{ P \in \pi | P \equiv P_{1} | P_{2}, P_{1} \in \meaningof{E_{1}}, P_{2} \in \meaningof{E_2}\} }
\end{mathpar}

\begin{mathpar}
 \inferrule* [lab=behavior] {} {\meaningof{\langle a?b \rangle E} = \{ P \in \pi | P \equiv Q | u?(y)P', \\ \and \\\\ \and \\ \;\;\; u \in \meaningof{a}, \forall z.P'\{z/y\} \in \meaningof{E\{z/b\}}\}, \and \\ \meaningof{a!E} = \{ P \in \pi | P \equiv Q | x!\langle P' \rangle, x \in \meaningof{a} P' \in \meaningof{E}\} }
\end{mathpar}

\begin{mathpar}
 \inferrule* [lab=nominal] {} {\meaningof{\quotep{E}} = \{ \quotep{P} \in \quotep{\pi} | P \in \meaningof{E} \}, \and \meaningof{\quotep{P}} = \{ \quotep{Q} \in \quotep{\pi} | P \equiv Q \} \and \\ \meaningof{@\quotep{E}} = \{ P \in \pi | P \equiv @x, x \in \meaningof{E} \}}
\end{mathpar}

\begin{eqnarray*}
  \\
  \meaningof{-} : TS \to ST
\end{eqnarray*}

\begin{eqnarray*}
  \\
  L : TS \to ST
\end{eqnarray*}

\begin{eqnarray*}
  \\
  P \models E \iff P \in \meaningof{E}
\end{eqnarray*}

\begin{eqnarray*}
  P \approx_{L} Q \iff \forall E \in L. P \models E \iff Q \models E
\end{eqnarray*}

\begin{eqnarray*}
  P \approx_{K} Q
\end{eqnarray*}

\begin{eqnarray*}
  P \approx Q
\end{eqnarray*}

$\approx_{K} = \approx = \approx_{L}$

\subsubsection{Contextual duality}

Note that contexts extend the quotation operation to a family of
operations from processes to names. Given a context, $M$, we can
define a \emph{nominal context}, $\quotep{M}$ by $\quotep{M}[P] :=
\quotep{M[P]}$. To foreshadow what is to come we observe that these
operations enjoy a duality with processes very much like the duality
between vectors and maps from vectors to scalars.

Further, because the calculus is essentially higher-order, we have a
correspondence between contexts and processes. More specifically,
given a name $x$ and a context $M$ we can construct $M^{*}_{x}$ such
that 

\begin{mathpar}
  M^{*}_{x} | \lift{x}{P} \red M[P]
\end{mathpar}

namely,

\begin{mathpar}
  M^{*}_{x} := x?(u).M[\dropn{u}]
\end{mathpar}

The dependence of $M^{*}_{x}$ on a name makes it an abstraction, 

\begin{mathpar}
  M^{*} := (x)x?(u).M[\dropn{u}]
\end{mathpar}

\subsection{Additional notation}

It will sometimes be convenient to denote the process a name
quotes. We already have the notation $x = \quotep{P}$, but it will be
convenient to introduce an alternate notation, $\procn{x}$, when we
want to emphasize the connection to the use of the name. Note that, by
virtue of name equivalence, $\quotep{\procn{x}} \nameeq x$; so, the
notation is consistent with previous definitions.

Further, because names have structure it is possible to effect
substitutions on the basis of that structure. This means we need to
upgrade our notation for substitutions, which we accomplish by
adapting comprehension notation. Thus,

\begin{mathpar}
  P\{ y / x : x \in S \}
\end{mathpar}

is interpreted to mean the process derived from P by replacing (in a
capture-avoiding manner) each occurrence of $x$ in $S$ by $y$. For example,

\begin{mathpar}
  P\{ \quotep{\procn{x}|\procn{x}} / x : x \in \freenames{P} \}
\end{mathpar}

will replace each (occurrence) of a free name $x$ in $P$ by
$\quotep{\procn{x}|\procn{x}}$.

Also, we will avail ourselves of the notation $x^{L}$ and $x^{R}$ to
denote injections of a name into disjoint copies of the name
space. There are numerous ways to accomplish this. One example can be
found in \cite{MeredithR05}. This notation overloads to vectors of
names: $\vec{x}^{\pi} := (x_{i}^{\pi} \; : \; 0 \leq i < |\vec{x}| )$ where $\pi \in \{L,R\}$.

We also use $P^{\Box} := P|\Box$.

In \cite{MeredithR05} an interpretation of the new operator is
given. It turns out that there are several possible interpretations
all enjoying the requisite algebraic properties of the operator (see
\cite{milner91polyadicpi}). We will therefore make liberal use of
$(\nu\; \vec{x})P$.

% subsection the_syntax_and_semantics_of_the_notation_system (end)   

\input{qm2pi.qmops} 

\input{qm2pi.sterngerlach} 

\input{qm2pi.metric} 

% section concurrent_process_calculi (end)

%\input{qm2pi.proofsketch}

% section proof sketch (end)

%\input{qm2pi.slviaknots} 

% section spatial logic via knots (end)

\input{qm2pi.conclusion}

% section conclusion (end)

%\input{qm2pi.dtcodes} 

% section wiring algorithm (end)

\input{qm2pi.ack} 

% section acknowledgments (end)

\newpage


\bibliographystyle{plain}   
\bibliography{../../biblios/main.bib}

\input{qm2pi.rhodetails}

\end{document}



\end{document}

 

%\documentclass[12pt]{llncs}
%\documentclass{jktr}

\usepackage[pdftex]{hyperref}                   
\usepackage {listings}
\usepackage {mathpartir}
\usepackage{bcprules}
%\usepackage{listings}
                       
\usepackage{graphicx} 
%\usepackage[margins=2.5cm,nohead,nofoot]{geometry}
%\usepackage{geometry}
\usepackage{amsfonts}
\usepackage{amstext}
\usepackage{latexsym}
\usepackage{amssymb}
\usepackage{color}


%\include{myPreamble}
\documentclass[12pt]{llncs}
%\documentclass{jktr}

\usepackage[pdftex]{hyperref}                   
\usepackage {listings}
\usepackage {mathpartir}
\usepackage{bcprules}
%\usepackage{listings}
                       
\usepackage{graphicx} 
%\usepackage[margins=2.5cm,nohead,nofoot]{geometry}
%\usepackage{geometry}
\usepackage{amsfonts}
\usepackage{amstext}
\usepackage{latexsym}
\usepackage{amssymb}
\usepackage{color}


%\include{myPreamble}
\include{qm2pi.local} 

%\ifpdf
%\usepackage[pdftex]{graphicx}
%\else
%\usepackage{graphicx}
%\fi

 % \ifpdf
%  \usepackage{pdfsync}
%  \if


%\title{Brief Article}
%\author{David F. Snyder}
%\author{L.G. Meredith}

%\address{Dept. of Math., Texas State University--San Marcos, San Marcos, TX 78666}
       
\pagestyle{empty}


\begin{document}

\lstset{language=[Objective]Caml,frame=shadowbox}

\input{qm2pi.front}

% section front matter (end)

\input{qm2pi.intro} 
 
% section introduction (end)

% \input{qm2pi.knotations} 

% section notation (end)

\input{qm2pi.process.calculi} 

% section concurrent_process_calculi_and_spatial_logics_ (end)
    
%\input{qm2pi.knots2pi} 

%\input{qm2pi.trefoil} 

%\input{qm2pi.mainthm} 

% subsection basic_interpretation (end)

%\input{qm2pi.rho.presentation} 
\subsection{The syntax and semantics of the notation system}\label{sub:the_syntax_and_semantics_of_the_notation_system} % (fold)

We now summarize a technical presentation of the calculus that
embodies our theory of dynamics. The typical presentation of such a
calculus follows the style of giving generators and relations on
them. The grammar, below, describing term constructors, freely
generates the set of processes, $\Proc$. This set is then quotiented
by a relation known as structural congruence and it is over this set
that the notion of dynamics is expressed. This presentation is
essentially that of \cite{MeredithR05} with the addition of
polyadicity and summation. For readability we have relegated some of
the technical subtleties to an appendix.

\subsubsection{Process grammar}\label{subsub:process_grammar}

\begin{mathpar}
  \inferrule* [lab=synchronization] {} {{M} \bc \pzero \;|\; x?F \;|\; x!C }
  \and
  \inferrule* [lab=abstraction] {} {{F} \bc (x)P}
  \and
  \inferrule* [lab=concretion] {} {{C} \bc \langle Q \rangle}
  \and
  \inferrule* [lab=process] {} {{P,Q} \bc M \;| \;P|Q \;|\; @{x}}
  \and
  \inferrule* [lab=name] {} {{x} \bc \quotep{P}}
\end{mathpar} 

Note that $\vec{x}$ (resp. $\vec{P}$) denotes a vector of names
(resp. processes) of length $|\vec{x}|$ (resp. $|\vec{P}|$). We adopt
the following useful abbreviations.

\begin{mathpar}
   x?(\vec{y}).P := x.(\vec{y})P \and  x\clift{\vec{P}} := x.\clift{\vec{P}}
   \and x!(y) := \lift{x}{\dropn{y}}
   \and \Pi_{i=0}^{n-1}P_i := P_0 | \ldots | P_{n-1}
\end{mathpar}

\subsubsection{Structural congruence}

\paragraph{Free and bound names and alpha-equivalence.} At the
core of structural equivalence is alpha-equivalence which identifies
process that are the same up to a change of variable. Formally, we
recognize the distinction between free and bound names. The free names
of a process, $\freenames{P}$, may be calculated recursively as
follows:

\begin{mathpar}
\freenames{\pzero} := \emptyset
  \and \\
  \freenames{x?(y).P} := \{ x \} \cup (\freenames{P} \setminus \{ y \})
  \and 
  \freenames{x!\langle P \rangle} := \{ x \} \cup \{ P \} 
  \and \\
  \freenames{P|Q} := \freenames{P} \cup \freenames{Q}
  \and \\
  \freenames{@{x}} := \{ x \}
\end{mathpar}

$\pi$
$\quotep{\pi}$

$\freenames{-} : \pi \to \mathcal{P}(\quotep{\pi})$

\begin{eqnarray*}
  \freenames{\pzero} & := & \emptyset \\
  \freenames{x?(y).P} & := & \{ x \} \cup (\freenames{P} \setminus \{ y \}) \\
  \freenames{x!\langle P \rangle} & := & \{ x \} \cup \{ P \} \\
  \freenames{P|Q} & := & \freenames{P} \cup \freenames{Q} \\
  \freenames{\dropn{x}} & := & \{ x \}
\end{eqnarray*}

The bound names of a process, $\boundnames{P}$, are those names occurring in $P$
that are not free. For example, in $x?(y).0$, the name $x$ is free, while $y$ is bound.

\begin{mathpar}
  \inferrule* [lab=monoidal-laws] {} { P|Q \equiv Q|P \and P|0 \equiv P \and P|(Q|R) \equiv (P|Q)|R }
\end{mathpar}

\begin{mathpar}
  \inferrule* [lab=alpha-equivalence] {} { (x)P \equiv (y)P\{y/x\} \and y \not\in \freenames{P} }
\end{mathpar}

\begin{definition}
Then two processes, $P,Q$, are alpha-equivalent if $P = Q\{\vec{y}/\vec{x}\}$ for
some $\vec{x} \in \boundnames{Q},\vec{y} \in \boundnames{P}$, where $Q\{\vec{y}/\vec{x}\}$
denotes the capture-avoiding substitution of $\vec{y}$ for $\vec{x}$ in $Q$.
\end{definition}

\begin{definition}
  The {\em structural congruence} \cite{SangiorgiWalker} , $\equiv$,
  between processes is the least congruence containing
  alpha-equivalence, satisfying the abelian monoid laws
  (associativity, commutativity and $\pzero$ as identity) for parallel
  composition $|$ and for summation $+$.
\end{definition}

\subsection{Name equivalence}

We take name equivalence, written $\nameeq$, to be the smallest
equivalence relation generated by the following rules.

\begin{mathpar}
\inferrule*[lab=Quote-drop]
{ }
{ \quotep{@{x}} \nameeq x }

\inferrule*[lab=Struct-equiv]
{ P \scong Q }
{ \quotep{P} \nameeq \quotep{Q} }
\end{mathpar}

The astute reader will have noticed that the mutual recursion of names
and processes imposes a mutual recursion on alpha-equivalence and
structural equivalence via name-equivalence. Fortunately, all of this
works out pleasantly and we may calculate in the natural way, free of
concern. The reader interested in the details is referred to the
appendix \ref{appendix:rho_details}.

\subsection{Substitution}

We use $\Proc$ for the set of processes, $\QProc$ for the set of
names, and $\id{\{}\vec{y} / \vec{x} \id{\}}$ to denote partial maps,
$s : \QProc \rightarrow \QProc$. A map, $s$ lifts, uniquely, to a map
on process terms, $\widehat{s} : \Proc \rightarrow \Proc$ by the
following equations.

\begin{mathpar}
  (0) \psubstp{Q}{P} := 0 \\
  (R \juxtap S) \psubstp{Q}{P}
  :=    
  (R)\psubstp{Q}{P} \juxtap (S) \psubstp{Q}{P} \\
  (x?(y).R) \psubstp{Q}{P}    
  :=    
  (x)\substp{Q}{P} (z)\concat( (R \psubstn{z}{y}) \psubstp{Q}{P} ) \\
  (\lift{x}{R}) \psubstp{Q}{P}  
  :=
  \lift{(x)\substp{Q}{P}}{ R \psubstp{Q}{P} } \\
%   (\dropn{x})  \psubstp{Q}{P}       
%   := 
%   \left\{ 
%     \begin{array}{ccc} 
%       \dropn{\quotep{Q}} & & x \nameeq \quotep{P} \\
%       \dropn{x} & & otherwise \\
%     \end{array}
%   \right. 
  (\dropn{x})  \psubstp{Q}{P}       
  := 
  \left\{ 
    \begin{array}{ccc} 
      Q & & x \nameeq \quotep{P} \\
      \dropn{x} & & otherwise \\
    \end{array}
  \right.
\end{mathpar}
 

where

\begin{eqnarray}
  (x)\id{\{} \lpquote Q \rpquote / \lpquote P \rpquote \id{\}}            = 
  \left\{ 
    \begin{array}{ccc}
      \lpquote Q \rpquote & & x \nameeq \lpquote P \rpquote \\
      x & & otherwise \\
    \end{array}
  \right. \nonumber
\end{eqnarray}

and $z$ is chosen distinct from $\quotep{P}$, $\quotep{Q}$, the free
names in $Q$, and all the names in $R$. Our $\alpha$-equivalence will
be built in the standard way from this substitution.

\begin{remark}\label{rem:no_self_referential_names}
  One consequence of these definitions is that $\forall P. \quotep{P}
  \not\in \freenames{P}$.
\end{remark}

\subsection{ Dynamic quote: an example }

Anticipating something of what's to come, consider applying the
substitution, $\widehat{\id{\{}u / z \id{\}}}$, to the following pair
of processes, $\lift{w}{y!(z)}$ and $w[ \lpquote y!(z) \rpquote ]$.

\begin{eqnarray}
	\lift{w}{y!(z)}\widehat{\id{\{}u / z \id{\}}}
		& = &
		\lift{w}{y!(u)} \nonumber\\
	w[ \lpquote y!(z) \rpquote ] \widehat{ \id{\{}u / z \id{\}} }
		& = &
		w[ \lpquote y!(z) \rpquote ] \nonumber
\end{eqnarray}

Because the body of the process between quotes is impervious to
substitution, we get radically different answers. In fact, by
examining the first process in an input context,
e.g. $x?(z).\lift{w}{y!(z)}$, we see that the process under the lift
operator may be shaped by prefixed inputs binding a name inside it. In
this sense, the lift operator will be seen as a way to dynamically
construct processes before reifying them as names.

Finally equipped with these standard features we can present the
dynamics of the calculus.

\subsubsection{Operational semantics} 

Finally, we introduce the computational dynamics. What marks these
algebras as distinct from other more traditionally studied algebraic
structures, e.g. vector spaces or polynomial rings, is the manner in
which dynamics is captured. In traditional structures, dynamics is typically
expressed through morphisms between such structures, as in linear maps
between vector spaces or morphisms between rings. In algebras
associated with the semantics of computation, the dynamics is
expressed as part of the algebraic structure itself, through a
reduction reduction relation typically denoted by $\red$. Below, we
give a recursive presentation of this relation for the calculus used
in the encoding.

$\red \subseteq \pi \times \pi$
$\red : \pi \to \mathcal{P}(\pi)$

\begin{mathpar}
  \inferrule* [lab=Comm] { \textsf{match}( x_{src}, x_{trgt} ) } { x_{trgt}?(y)P \; | \; x_{src}!\langle {Q} \rangle \red P\{\quotep{Q}/y}\} }
  \and \\
  \inferrule* [lab=Par] {{P} \red {P}'} {{{P} | {Q}} \red {{P}' | {Q}}}
  \and
  \inferrule* [lab=Equiv]{{{P} \scong {P}'} \andalso {{P}' \red {Q}'} \andalso {{Q}' \scong {Q}}}{{P} \red {Q}}
\end{mathpar}

\begin{eqnarray*}
  match_{\equiv} (\quotep{P},\quotep{Q}) & := & P \equiv Q \\
  match_{\dagger}(\quotep{P},\quotep{Q}) & := & \forall R. P|Q \red^{*} R => R \red^{*} 0 \\
  match_{K}(\quotep{P},\quotep{Q}) & := & K \mbox{ for some context } K
\end{eqnarray*}

$u?(x)P | u!\langle Q \rangle \red P\{\quotep{Q}/x\}$

%We write $\wred$ for $\red^*$, and $P\red$ if $\exists Q $ such that $ P \red Q$.
We write $P\red$ if $\exists Q $ such that $ P \red Q$ and $P\not\red$, otherwise.

\section{Replication}

As mentioned before, it is known that replication (and hence
recursion) can be implemented in a higher-order process algebra
\cite{SangiorgiWalker}. As our first example of calculation with the
machinery thus far presented we give the construction explicitly in
the {\rhoc}.

\begin{eqnarray}
	D_{x} & := & \prefix{x}{y}{(\binpar{\outputp{x}{y}}{@{y}})} \nonumber\\
	\bangp_{x}{P} & := & \binpar{{x}!\langle{\binpar{D_{x}}{P}}\rangle}{D_{x}} \nonumber
\end{eqnarray}

\begin{eqnarray}
	\bangp_{x}{P} & & \nonumber\\
	=
	& {x}!\langle{(\prefix{x}{y}{(\outputp{x}{y} | @{y})) | P}}\rangle 
	      | \prefix{x}{y}{(\outputp{x}{y} | @{y})} & \nonumber\\
	\red
	& (\outputp{x}{y} | @{y})\substn{\quotep{(\prefix{x}{y}{(@{y} | \outputp{x}{y})) | P}}}{y} & \nonumber\\
	=
	& \outputp{x}{\quotep{(\prefix{x}{y}{(\outputp{x}{y} | @{y})) | P}}}
	  | {(\prefix{x}{y}{(\outputp{x}{y} | @{y})) | P}} & \nonumber\\
	\red
	& \ldots & \nonumber\\
	\red^*
	& P | P | \ldots & \nonumber
\end{eqnarray}

Of course, this encoding, as an implementation, runs away, unfolding
$\bangp{P}$ eagerly. A lazier and more implementable replication
operator, restricted to input-guarded processes, may be obtained as follows.

\begin{eqnarray}
\bangp{\prefix{u}{v}{P}} 
	:= 
	\binpar{\lift{x}{\prefix{u}{v}{(\binpar{D(x)}{P})}}}{D(x)} \nonumber
\end{eqnarray}

\begin{remark}
  Note that the lazier definition still does not deal with summation
  or mixed summation (i.e. sums over input and output). The reader is
  invited to construct definitions of replication that deal with these
  features. 

  Further, the definitions are parameterized in a name, $x$. Can you,
  gentle reader, make a definition that eliminates this parameter and
  guarantees no accidental interaction between the replication
  machinery and the process being replicated -- i.e. no accidental
  sharing of names used by the process to get its work done and the
  name(s) used by the replication to effect copying. This latter
  revision of the definition of replication is crucial to obtaining
  the expected identity $!!P \sim !P$.
\end{remark}

\begin{remark}\label{rem:paradoxical_combinator}
  The reader familiar with the lambda calculus will have noticed the
  similarity between $D$ and the paradoxical combinator.

  [Ed. note: the existence of this seems to suggest we have to be more
  restrictive on the set of processes and names we admit if we are to
  support no-cloning.]
\end{remark}

\subsubsection{Bisimulation}

The computational dynamics gives rise to another kind of equivalence,
the equivalence of computational behavior. As previously mentioned
this is typically captured \emph{via} some form of bisimulation.

% The notion we use in this paper is weak barbed bisimulation
% \cite{milner91polyadicpi}.

The notion we use in this paper is derived from weak barbed
bisimulation \cite{milner91polyadicpi}. 

\begin{definition}
An \emph{observation relation}, $\downarrow_{\mathcal N}$, over a set
of names, $\mathcal N$, is the smallest relation satisfying the rules
below.

\infrule[Out-barb]{y \in {\mathcal N}, \; x \nameeq y}
		  {\outputp{x}{v} \downarrow_{\mathcal N} x}
\infrule[Par-barb]{\mbox{$P\downarrow_{\mathcal N} x$ or $Q\downarrow_{\mathcal N} x$}}
		  {\binpar{P}{Q} \downarrow_{\mathcal N} x}

We write $P \Downarrow_{\mathcal N} x$ if there is $Q$ such that 
$P \wred Q$ and $Q \downarrow_{\mathcal N} x$.
\end{definition}

\begin{definition}
%\label{def.bbisim}
An  ${\mathcal N}$-\emph{barbed bisimulation} over a set of names, ${\mathcal N}$, is a symmetric binary relation 
${\mathcal S}_{\mathcal N}$ between agents such that $P\rel{S}_{\mathcal N}Q$ implies:
\begin{enumerate}
\item If $P \red P'$ then $Q \wred Q'$ and $P'\rel{S}_{\mathcal N} Q'$.
\item If $P\downarrow_{\mathcal N} x$, then $Q\Downarrow_{\mathcal N} x$.
\end{enumerate}
$P$ is ${\mathcal N}$-barbed bisimilar to $Q$, written
$P \wbbisim_{\mathcal N} Q$, if $P \rel{S}_{\mathcal N} Q$ for some ${\mathcal N}$-barbed bisimulation ${\mathcal S}_{\mathcal N}$.
\end{definition}

$\mathcal{R} \subseteq \pi \times \pi$

$P \mathcal{R} Q => \forall P'. P \red P' \Rightarrow \exists Q'. Q \red Q', P' \mathcal{R} Q'$

$P \vdash x \Rightarrow Q \vdash x$

\begin{mathpar}
  \inferrule*[lab=Out-barb]{x \nameeq y}{{y}!\langle{Q}\rangle \vdash x}
  \and
  \inferrule*[lab=Par-barb]{\mbox{$P\vdash x$ or $Q\vdash x$}}{\binpar{P}{Q} \vdash x}
\end{mathpar}

\subsubsection{Contexts}

One of the principle advantages of computational calculi like the
$\pi$-calculus is a well-defined notion of context,
contextual-equivalence and a correlation between
contextual-equivalence and notions of bisimulation. The notion of
context allows the decomposition of a process into (sub-)process and
its syntactic environment, its context. Thus, a context may be
thought of as a process with a ``hole'' (written $\Box$) in it. The
application of a context $M$ to a process $P$, written $M[P]$, is
tantamount to filling the hole in $M$ with $P$. In this paper we do
not need the full weight of this theory, but do make use of the notion
of context in the proof the main theorem. 

\begin{mathpar}
  \inferrule* [lab=summation] {} {{M_{M},M_{N}} \bc \Box \;|\; x.M_{A} \;|\; M_{M}+M_{N}}
  \and
  \inferrule* [lab=agent] {} {{M_{A}} \bc (\vec{x})M_{P} \;| \; \clift{P_0,\ldots,M_{P},\ldots,P_N}}
  \and \\
  \inferrule* [lab=process] {} {{M_{P}} \bc M_{N} \;| \;P|M_{P} }
\end{mathpar} 

\begin{mathpar}
  \inferrule* [lab=sychronization] {} {M_{N} \bc \Box \;|\; x?M_{F} \;|\; x!M_{C}}
  \and
  \inferrule* [lab=abstraction] {} {{M_{F}} \bc (x)M_{P} }
  \and
  \inferrule* [lab=concretion] {} {{M_{C}} \bc \langle M_{P} \rangle }
  \and \\
  \inferrule* [lab=process] {} {{M_{P}} \bc M_{N} \;| \;P|M_{P} }
\end{mathpar}

\begin{definition}[contextual application] Given a context $M$, and
  process $P$, we define the \emph{contextual application}, $M[P] :=
  M\{P/\Box\}$. That is, the contextual application of M to P is the
  substitution of $P$ for $\Box$ in $M$.
\end{definition}

$\meaningof{-} : L \to \mathcal{P}(\pi)$

\begin{mathpar}
  \inferrule* [lab=collection] {} {\meaningof{true} = \pi, \and \meaningof{~E} = \pi \setminus \meaningof{E}, \and \meaningof{E_{1} \& E_{2}} = \meaningof{E_{1}} \cap \meaningof{E_{2}}}
\end{mathpar}

\begin{mathpar}
  \inferrule* [lab=structure] {} {\meaningof{0} = \{ P \in \pi | P \equiv 0 \}, \and \\ \meaningof{E_1 | E_2} = \{ P \in \pi | P \equiv P_{1} | P_{2}, P_{1} \in \meaningof{E_{1}}, P_{2} \in \meaningof{E_2}\} }
\end{mathpar}

\begin{mathpar}
 \inferrule* [lab=behavior] {} {\meaningof{\langle a?b \rangle E} = \{ P \in \pi | P \equiv Q | u?(y)P', \\ \and \\\\ \and \\ \;\;\; u \in \meaningof{a}, \forall z.P'\{z/y\} \in \meaningof{E\{z/b\}}\}, \and \\ \meaningof{a!E} = \{ P \in \pi | P \equiv Q | x!\langle P' \rangle, x \in \meaningof{a} P' \in \meaningof{E}\} }
\end{mathpar}

\begin{mathpar}
 \inferrule* [lab=nominal] {} {\meaningof{\quotep{E}} = \{ \quotep{P} \in \quotep{\pi} | P \in \meaningof{E} \}, \and \meaningof{\quotep{P}} = \{ \quotep{Q} \in \quotep{\pi} | P \equiv Q \} \and \\ \meaningof{@\quotep{E}} = \{ P \in \pi | P \equiv @x, x \in \meaningof{E} \}}
\end{mathpar}

\begin{eqnarray*}
  \\
  \meaningof{-} : TS \to ST
\end{eqnarray*}

\begin{eqnarray*}
  \\
  L : TS \to ST
\end{eqnarray*}

\begin{eqnarray*}
  \\
  P \models E \iff P \in \meaningof{E}
\end{eqnarray*}

\begin{eqnarray*}
  P \approx_{L} Q \iff \forall E \in L. P \models E \iff Q \models E
\end{eqnarray*}

\begin{eqnarray*}
  P \approx_{K} Q
\end{eqnarray*}

\begin{eqnarray*}
  P \approx Q
\end{eqnarray*}

$\approx_{K} = \approx = \approx_{L}$

\subsubsection{Contextual duality}

Note that contexts extend the quotation operation to a family of
operations from processes to names. Given a context, $M$, we can
define a \emph{nominal context}, $\quotep{M}$ by $\quotep{M}[P] :=
\quotep{M[P]}$. To foreshadow what is to come we observe that these
operations enjoy a duality with processes very much like the duality
between vectors and maps from vectors to scalars.

Further, because the calculus is essentially higher-order, we have a
correspondence between contexts and processes. More specifically,
given a name $x$ and a context $M$ we can construct $M^{*}_{x}$ such
that 

\begin{mathpar}
  M^{*}_{x} | \lift{x}{P} \red M[P]
\end{mathpar}

namely,

\begin{mathpar}
  M^{*}_{x} := x?(u).M[\dropn{u}]
\end{mathpar}

The dependence of $M^{*}_{x}$ on a name makes it an abstraction, 

\begin{mathpar}
  M^{*} := (x)x?(u).M[\dropn{u}]
\end{mathpar}

\subsection{Additional notation}

It will sometimes be convenient to denote the process a name
quotes. We already have the notation $x = \quotep{P}$, but it will be
convenient to introduce an alternate notation, $\procn{x}$, when we
want to emphasize the connection to the use of the name. Note that, by
virtue of name equivalence, $\quotep{\procn{x}} \nameeq x$; so, the
notation is consistent with previous definitions.

Further, because names have structure it is possible to effect
substitutions on the basis of that structure. This means we need to
upgrade our notation for substitutions, which we accomplish by
adapting comprehension notation. Thus,

\begin{mathpar}
  P\{ y / x : x \in S \}
\end{mathpar}

is interpreted to mean the process derived from P by replacing (in a
capture-avoiding manner) each occurrence of $x$ in $S$ by $y$. For example,

\begin{mathpar}
  P\{ \quotep{\procn{x}|\procn{x}} / x : x \in \freenames{P} \}
\end{mathpar}

will replace each (occurrence) of a free name $x$ in $P$ by
$\quotep{\procn{x}|\procn{x}}$.

Also, we will avail ourselves of the notation $x^{L}$ and $x^{R}$ to
denote injections of a name into disjoint copies of the name
space. There are numerous ways to accomplish this. One example can be
found in \cite{MeredithR05}. This notation overloads to vectors of
names: $\vec{x}^{\pi} := (x_{i}^{\pi} \; : \; 0 \leq i < |\vec{x}| )$ where $\pi \in \{L,R\}$.

We also use $P^{\Box} := P|\Box$.

In \cite{MeredithR05} an interpretation of the new operator is
given. It turns out that there are several possible interpretations
all enjoying the requisite algebraic properties of the operator (see
\cite{milner91polyadicpi}). We will therefore make liberal use of
$(\nu\; \vec{x})P$.

% subsection the_syntax_and_semantics_of_the_notation_system (end)   

\input{qm2pi.qmops} 

\input{qm2pi.sterngerlach} 

\input{qm2pi.metric} 

% section concurrent_process_calculi (end)

%\input{qm2pi.proofsketch}

% section proof sketch (end)

%\input{qm2pi.slviaknots} 

% section spatial logic via knots (end)

\input{qm2pi.conclusion}

% section conclusion (end)

%\input{qm2pi.dtcodes} 

% section wiring algorithm (end)

\input{qm2pi.ack} 

% section acknowledgments (end)

\newpage


\bibliographystyle{plain}   
\bibliography{../../biblios/main.bib}

\input{qm2pi.rhodetails}

\end{document}

 

%\ifpdf
%\usepackage[pdftex]{graphicx}
%\else
%\usepackage{graphicx}
%\fi

 % \ifpdf
%  \usepackage{pdfsync}
%  \if


%\title{Brief Article}
%\author{David F. Snyder}
%\author{L.G. Meredith}

%\address{Dept. of Math., Texas State University--San Marcos, San Marcos, TX 78666}
       
\pagestyle{empty}


\begin{document}

\lstset{language=[Objective]Caml,frame=shadowbox}

\documentclass[12pt]{llncs}
%\documentclass{jktr}

\usepackage[pdftex]{hyperref}                   
\usepackage {listings}
\usepackage {mathpartir}
\usepackage{bcprules}
%\usepackage{listings}
                       
\usepackage{graphicx} 
%\usepackage[margins=2.5cm,nohead,nofoot]{geometry}
%\usepackage{geometry}
\usepackage{amsfonts}
\usepackage{amstext}
\usepackage{latexsym}
\usepackage{amssymb}
\usepackage{color}


%\include{myPreamble}
\include{qm2pi.local} 

%\ifpdf
%\usepackage[pdftex]{graphicx}
%\else
%\usepackage{graphicx}
%\fi

 % \ifpdf
%  \usepackage{pdfsync}
%  \if


%\title{Brief Article}
%\author{David F. Snyder}
%\author{L.G. Meredith}

%\address{Dept. of Math., Texas State University--San Marcos, San Marcos, TX 78666}
       
\pagestyle{empty}


\begin{document}

\lstset{language=[Objective]Caml,frame=shadowbox}

\input{qm2pi.front}

% section front matter (end)

\input{qm2pi.intro} 
 
% section introduction (end)

% \input{qm2pi.knotations} 

% section notation (end)

\input{qm2pi.process.calculi} 

% section concurrent_process_calculi_and_spatial_logics_ (end)
    
%\input{qm2pi.knots2pi} 

%\input{qm2pi.trefoil} 

%\input{qm2pi.mainthm} 

% subsection basic_interpretation (end)

%\input{qm2pi.rho.presentation} 
\subsection{The syntax and semantics of the notation system}\label{sub:the_syntax_and_semantics_of_the_notation_system} % (fold)

We now summarize a technical presentation of the calculus that
embodies our theory of dynamics. The typical presentation of such a
calculus follows the style of giving generators and relations on
them. The grammar, below, describing term constructors, freely
generates the set of processes, $\Proc$. This set is then quotiented
by a relation known as structural congruence and it is over this set
that the notion of dynamics is expressed. This presentation is
essentially that of \cite{MeredithR05} with the addition of
polyadicity and summation. For readability we have relegated some of
the technical subtleties to an appendix.

\subsubsection{Process grammar}\label{subsub:process_grammar}

\begin{mathpar}
  \inferrule* [lab=synchronization] {} {{M} \bc \pzero \;|\; x?F \;|\; x!C }
  \and
  \inferrule* [lab=abstraction] {} {{F} \bc (x)P}
  \and
  \inferrule* [lab=concretion] {} {{C} \bc \langle Q \rangle}
  \and
  \inferrule* [lab=process] {} {{P,Q} \bc M \;| \;P|Q \;|\; @{x}}
  \and
  \inferrule* [lab=name] {} {{x} \bc \quotep{P}}
\end{mathpar} 

Note that $\vec{x}$ (resp. $\vec{P}$) denotes a vector of names
(resp. processes) of length $|\vec{x}|$ (resp. $|\vec{P}|$). We adopt
the following useful abbreviations.

\begin{mathpar}
   x?(\vec{y}).P := x.(\vec{y})P \and  x\clift{\vec{P}} := x.\clift{\vec{P}}
   \and x!(y) := \lift{x}{\dropn{y}}
   \and \Pi_{i=0}^{n-1}P_i := P_0 | \ldots | P_{n-1}
\end{mathpar}

\subsubsection{Structural congruence}

\paragraph{Free and bound names and alpha-equivalence.} At the
core of structural equivalence is alpha-equivalence which identifies
process that are the same up to a change of variable. Formally, we
recognize the distinction between free and bound names. The free names
of a process, $\freenames{P}$, may be calculated recursively as
follows:

\begin{mathpar}
\freenames{\pzero} := \emptyset
  \and \\
  \freenames{x?(y).P} := \{ x \} \cup (\freenames{P} \setminus \{ y \})
  \and 
  \freenames{x!\langle P \rangle} := \{ x \} \cup \{ P \} 
  \and \\
  \freenames{P|Q} := \freenames{P} \cup \freenames{Q}
  \and \\
  \freenames{@{x}} := \{ x \}
\end{mathpar}

$\pi$
$\quotep{\pi}$

$\freenames{-} : \pi \to \mathcal{P}(\quotep{\pi})$

\begin{eqnarray*}
  \freenames{\pzero} & := & \emptyset \\
  \freenames{x?(y).P} & := & \{ x \} \cup (\freenames{P} \setminus \{ y \}) \\
  \freenames{x!\langle P \rangle} & := & \{ x \} \cup \{ P \} \\
  \freenames{P|Q} & := & \freenames{P} \cup \freenames{Q} \\
  \freenames{\dropn{x}} & := & \{ x \}
\end{eqnarray*}

The bound names of a process, $\boundnames{P}$, are those names occurring in $P$
that are not free. For example, in $x?(y).0$, the name $x$ is free, while $y$ is bound.

\begin{mathpar}
  \inferrule* [lab=monoidal-laws] {} { P|Q \equiv Q|P \and P|0 \equiv P \and P|(Q|R) \equiv (P|Q)|R }
\end{mathpar}

\begin{mathpar}
  \inferrule* [lab=alpha-equivalence] {} { (x)P \equiv (y)P\{y/x\} \and y \not\in \freenames{P} }
\end{mathpar}

\begin{definition}
Then two processes, $P,Q$, are alpha-equivalent if $P = Q\{\vec{y}/\vec{x}\}$ for
some $\vec{x} \in \boundnames{Q},\vec{y} \in \boundnames{P}$, where $Q\{\vec{y}/\vec{x}\}$
denotes the capture-avoiding substitution of $\vec{y}$ for $\vec{x}$ in $Q$.
\end{definition}

\begin{definition}
  The {\em structural congruence} \cite{SangiorgiWalker} , $\equiv$,
  between processes is the least congruence containing
  alpha-equivalence, satisfying the abelian monoid laws
  (associativity, commutativity and $\pzero$ as identity) for parallel
  composition $|$ and for summation $+$.
\end{definition}

\subsection{Name equivalence}

We take name equivalence, written $\nameeq$, to be the smallest
equivalence relation generated by the following rules.

\begin{mathpar}
\inferrule*[lab=Quote-drop]
{ }
{ \quotep{@{x}} \nameeq x }

\inferrule*[lab=Struct-equiv]
{ P \scong Q }
{ \quotep{P} \nameeq \quotep{Q} }
\end{mathpar}

The astute reader will have noticed that the mutual recursion of names
and processes imposes a mutual recursion on alpha-equivalence and
structural equivalence via name-equivalence. Fortunately, all of this
works out pleasantly and we may calculate in the natural way, free of
concern. The reader interested in the details is referred to the
appendix \ref{appendix:rho_details}.

\subsection{Substitution}

We use $\Proc$ for the set of processes, $\QProc$ for the set of
names, and $\id{\{}\vec{y} / \vec{x} \id{\}}$ to denote partial maps,
$s : \QProc \rightarrow \QProc$. A map, $s$ lifts, uniquely, to a map
on process terms, $\widehat{s} : \Proc \rightarrow \Proc$ by the
following equations.

\begin{mathpar}
  (0) \psubstp{Q}{P} := 0 \\
  (R \juxtap S) \psubstp{Q}{P}
  :=    
  (R)\psubstp{Q}{P} \juxtap (S) \psubstp{Q}{P} \\
  (x?(y).R) \psubstp{Q}{P}    
  :=    
  (x)\substp{Q}{P} (z)\concat( (R \psubstn{z}{y}) \psubstp{Q}{P} ) \\
  (\lift{x}{R}) \psubstp{Q}{P}  
  :=
  \lift{(x)\substp{Q}{P}}{ R \psubstp{Q}{P} } \\
%   (\dropn{x})  \psubstp{Q}{P}       
%   := 
%   \left\{ 
%     \begin{array}{ccc} 
%       \dropn{\quotep{Q}} & & x \nameeq \quotep{P} \\
%       \dropn{x} & & otherwise \\
%     \end{array}
%   \right. 
  (\dropn{x})  \psubstp{Q}{P}       
  := 
  \left\{ 
    \begin{array}{ccc} 
      Q & & x \nameeq \quotep{P} \\
      \dropn{x} & & otherwise \\
    \end{array}
  \right.
\end{mathpar}
 

where

\begin{eqnarray}
  (x)\id{\{} \lpquote Q \rpquote / \lpquote P \rpquote \id{\}}            = 
  \left\{ 
    \begin{array}{ccc}
      \lpquote Q \rpquote & & x \nameeq \lpquote P \rpquote \\
      x & & otherwise \\
    \end{array}
  \right. \nonumber
\end{eqnarray}

and $z$ is chosen distinct from $\quotep{P}$, $\quotep{Q}$, the free
names in $Q$, and all the names in $R$. Our $\alpha$-equivalence will
be built in the standard way from this substitution.

\begin{remark}\label{rem:no_self_referential_names}
  One consequence of these definitions is that $\forall P. \quotep{P}
  \not\in \freenames{P}$.
\end{remark}

\subsection{ Dynamic quote: an example }

Anticipating something of what's to come, consider applying the
substitution, $\widehat{\id{\{}u / z \id{\}}}$, to the following pair
of processes, $\lift{w}{y!(z)}$ and $w[ \lpquote y!(z) \rpquote ]$.

\begin{eqnarray}
	\lift{w}{y!(z)}\widehat{\id{\{}u / z \id{\}}}
		& = &
		\lift{w}{y!(u)} \nonumber\\
	w[ \lpquote y!(z) \rpquote ] \widehat{ \id{\{}u / z \id{\}} }
		& = &
		w[ \lpquote y!(z) \rpquote ] \nonumber
\end{eqnarray}

Because the body of the process between quotes is impervious to
substitution, we get radically different answers. In fact, by
examining the first process in an input context,
e.g. $x?(z).\lift{w}{y!(z)}$, we see that the process under the lift
operator may be shaped by prefixed inputs binding a name inside it. In
this sense, the lift operator will be seen as a way to dynamically
construct processes before reifying them as names.

Finally equipped with these standard features we can present the
dynamics of the calculus.

\subsubsection{Operational semantics} 

Finally, we introduce the computational dynamics. What marks these
algebras as distinct from other more traditionally studied algebraic
structures, e.g. vector spaces or polynomial rings, is the manner in
which dynamics is captured. In traditional structures, dynamics is typically
expressed through morphisms between such structures, as in linear maps
between vector spaces or morphisms between rings. In algebras
associated with the semantics of computation, the dynamics is
expressed as part of the algebraic structure itself, through a
reduction reduction relation typically denoted by $\red$. Below, we
give a recursive presentation of this relation for the calculus used
in the encoding.

$\red \subseteq \pi \times \pi$
$\red : \pi \to \mathcal{P}(\pi)$

\begin{mathpar}
  \inferrule* [lab=Comm] { \textsf{match}( x_{src}, x_{trgt} ) } { x_{trgt}?(y)P \; | \; x_{src}!\langle {Q} \rangle \red P\{\quotep{Q}/y}\} }
  \and \\
  \inferrule* [lab=Par] {{P} \red {P}'} {{{P} | {Q}} \red {{P}' | {Q}}}
  \and
  \inferrule* [lab=Equiv]{{{P} \scong {P}'} \andalso {{P}' \red {Q}'} \andalso {{Q}' \scong {Q}}}{{P} \red {Q}}
\end{mathpar}

\begin{eqnarray*}
  match_{\equiv} (\quotep{P},\quotep{Q}) & := & P \equiv Q \\
  match_{\dagger}(\quotep{P},\quotep{Q}) & := & \forall R. P|Q \red^{*} R => R \red^{*} 0 \\
  match_{K}(\quotep{P},\quotep{Q}) & := & K \mbox{ for some context } K
\end{eqnarray*}

$u?(x)P | u!\langle Q \rangle \red P\{\quotep{Q}/x\}$

%We write $\wred$ for $\red^*$, and $P\red$ if $\exists Q $ such that $ P \red Q$.
We write $P\red$ if $\exists Q $ such that $ P \red Q$ and $P\not\red$, otherwise.

\section{Replication}

As mentioned before, it is known that replication (and hence
recursion) can be implemented in a higher-order process algebra
\cite{SangiorgiWalker}. As our first example of calculation with the
machinery thus far presented we give the construction explicitly in
the {\rhoc}.

\begin{eqnarray}
	D_{x} & := & \prefix{x}{y}{(\binpar{\outputp{x}{y}}{@{y}})} \nonumber\\
	\bangp_{x}{P} & := & \binpar{{x}!\langle{\binpar{D_{x}}{P}}\rangle}{D_{x}} \nonumber
\end{eqnarray}

\begin{eqnarray}
	\bangp_{x}{P} & & \nonumber\\
	=
	& {x}!\langle{(\prefix{x}{y}{(\outputp{x}{y} | @{y})) | P}}\rangle 
	      | \prefix{x}{y}{(\outputp{x}{y} | @{y})} & \nonumber\\
	\red
	& (\outputp{x}{y} | @{y})\substn{\quotep{(\prefix{x}{y}{(@{y} | \outputp{x}{y})) | P}}}{y} & \nonumber\\
	=
	& \outputp{x}{\quotep{(\prefix{x}{y}{(\outputp{x}{y} | @{y})) | P}}}
	  | {(\prefix{x}{y}{(\outputp{x}{y} | @{y})) | P}} & \nonumber\\
	\red
	& \ldots & \nonumber\\
	\red^*
	& P | P | \ldots & \nonumber
\end{eqnarray}

Of course, this encoding, as an implementation, runs away, unfolding
$\bangp{P}$ eagerly. A lazier and more implementable replication
operator, restricted to input-guarded processes, may be obtained as follows.

\begin{eqnarray}
\bangp{\prefix{u}{v}{P}} 
	:= 
	\binpar{\lift{x}{\prefix{u}{v}{(\binpar{D(x)}{P})}}}{D(x)} \nonumber
\end{eqnarray}

\begin{remark}
  Note that the lazier definition still does not deal with summation
  or mixed summation (i.e. sums over input and output). The reader is
  invited to construct definitions of replication that deal with these
  features. 

  Further, the definitions are parameterized in a name, $x$. Can you,
  gentle reader, make a definition that eliminates this parameter and
  guarantees no accidental interaction between the replication
  machinery and the process being replicated -- i.e. no accidental
  sharing of names used by the process to get its work done and the
  name(s) used by the replication to effect copying. This latter
  revision of the definition of replication is crucial to obtaining
  the expected identity $!!P \sim !P$.
\end{remark}

\begin{remark}\label{rem:paradoxical_combinator}
  The reader familiar with the lambda calculus will have noticed the
  similarity between $D$ and the paradoxical combinator.

  [Ed. note: the existence of this seems to suggest we have to be more
  restrictive on the set of processes and names we admit if we are to
  support no-cloning.]
\end{remark}

\subsubsection{Bisimulation}

The computational dynamics gives rise to another kind of equivalence,
the equivalence of computational behavior. As previously mentioned
this is typically captured \emph{via} some form of bisimulation.

% The notion we use in this paper is weak barbed bisimulation
% \cite{milner91polyadicpi}.

The notion we use in this paper is derived from weak barbed
bisimulation \cite{milner91polyadicpi}. 

\begin{definition}
An \emph{observation relation}, $\downarrow_{\mathcal N}$, over a set
of names, $\mathcal N$, is the smallest relation satisfying the rules
below.

\infrule[Out-barb]{y \in {\mathcal N}, \; x \nameeq y}
		  {\outputp{x}{v} \downarrow_{\mathcal N} x}
\infrule[Par-barb]{\mbox{$P\downarrow_{\mathcal N} x$ or $Q\downarrow_{\mathcal N} x$}}
		  {\binpar{P}{Q} \downarrow_{\mathcal N} x}

We write $P \Downarrow_{\mathcal N} x$ if there is $Q$ such that 
$P \wred Q$ and $Q \downarrow_{\mathcal N} x$.
\end{definition}

\begin{definition}
%\label{def.bbisim}
An  ${\mathcal N}$-\emph{barbed bisimulation} over a set of names, ${\mathcal N}$, is a symmetric binary relation 
${\mathcal S}_{\mathcal N}$ between agents such that $P\rel{S}_{\mathcal N}Q$ implies:
\begin{enumerate}
\item If $P \red P'$ then $Q \wred Q'$ and $P'\rel{S}_{\mathcal N} Q'$.
\item If $P\downarrow_{\mathcal N} x$, then $Q\Downarrow_{\mathcal N} x$.
\end{enumerate}
$P$ is ${\mathcal N}$-barbed bisimilar to $Q$, written
$P \wbbisim_{\mathcal N} Q$, if $P \rel{S}_{\mathcal N} Q$ for some ${\mathcal N}$-barbed bisimulation ${\mathcal S}_{\mathcal N}$.
\end{definition}

$\mathcal{R} \subseteq \pi \times \pi$

$P \mathcal{R} Q => \forall P'. P \red P' \Rightarrow \exists Q'. Q \red Q', P' \mathcal{R} Q'$

$P \vdash x \Rightarrow Q \vdash x$

\begin{mathpar}
  \inferrule*[lab=Out-barb]{x \nameeq y}{{y}!\langle{Q}\rangle \vdash x}
  \and
  \inferrule*[lab=Par-barb]{\mbox{$P\vdash x$ or $Q\vdash x$}}{\binpar{P}{Q} \vdash x}
\end{mathpar}

\subsubsection{Contexts}

One of the principle advantages of computational calculi like the
$\pi$-calculus is a well-defined notion of context,
contextual-equivalence and a correlation between
contextual-equivalence and notions of bisimulation. The notion of
context allows the decomposition of a process into (sub-)process and
its syntactic environment, its context. Thus, a context may be
thought of as a process with a ``hole'' (written $\Box$) in it. The
application of a context $M$ to a process $P$, written $M[P]$, is
tantamount to filling the hole in $M$ with $P$. In this paper we do
not need the full weight of this theory, but do make use of the notion
of context in the proof the main theorem. 

\begin{mathpar}
  \inferrule* [lab=summation] {} {{M_{M},M_{N}} \bc \Box \;|\; x.M_{A} \;|\; M_{M}+M_{N}}
  \and
  \inferrule* [lab=agent] {} {{M_{A}} \bc (\vec{x})M_{P} \;| \; \clift{P_0,\ldots,M_{P},\ldots,P_N}}
  \and \\
  \inferrule* [lab=process] {} {{M_{P}} \bc M_{N} \;| \;P|M_{P} }
\end{mathpar} 

\begin{mathpar}
  \inferrule* [lab=sychronization] {} {M_{N} \bc \Box \;|\; x?M_{F} \;|\; x!M_{C}}
  \and
  \inferrule* [lab=abstraction] {} {{M_{F}} \bc (x)M_{P} }
  \and
  \inferrule* [lab=concretion] {} {{M_{C}} \bc \langle M_{P} \rangle }
  \and \\
  \inferrule* [lab=process] {} {{M_{P}} \bc M_{N} \;| \;P|M_{P} }
\end{mathpar}

\begin{definition}[contextual application] Given a context $M$, and
  process $P$, we define the \emph{contextual application}, $M[P] :=
  M\{P/\Box\}$. That is, the contextual application of M to P is the
  substitution of $P$ for $\Box$ in $M$.
\end{definition}

$\meaningof{-} : L \to \mathcal{P}(\pi)$

\begin{mathpar}
  \inferrule* [lab=collection] {} {\meaningof{true} = \pi, \and \meaningof{~E} = \pi \setminus \meaningof{E}, \and \meaningof{E_{1} \& E_{2}} = \meaningof{E_{1}} \cap \meaningof{E_{2}}}
\end{mathpar}

\begin{mathpar}
  \inferrule* [lab=structure] {} {\meaningof{0} = \{ P \in \pi | P \equiv 0 \}, \and \\ \meaningof{E_1 | E_2} = \{ P \in \pi | P \equiv P_{1} | P_{2}, P_{1} \in \meaningof{E_{1}}, P_{2} \in \meaningof{E_2}\} }
\end{mathpar}

\begin{mathpar}
 \inferrule* [lab=behavior] {} {\meaningof{\langle a?b \rangle E} = \{ P \in \pi | P \equiv Q | u?(y)P', \\ \and \\\\ \and \\ \;\;\; u \in \meaningof{a}, \forall z.P'\{z/y\} \in \meaningof{E\{z/b\}}\}, \and \\ \meaningof{a!E} = \{ P \in \pi | P \equiv Q | x!\langle P' \rangle, x \in \meaningof{a} P' \in \meaningof{E}\} }
\end{mathpar}

\begin{mathpar}
 \inferrule* [lab=nominal] {} {\meaningof{\quotep{E}} = \{ \quotep{P} \in \quotep{\pi} | P \in \meaningof{E} \}, \and \meaningof{\quotep{P}} = \{ \quotep{Q} \in \quotep{\pi} | P \equiv Q \} \and \\ \meaningof{@\quotep{E}} = \{ P \in \pi | P \equiv @x, x \in \meaningof{E} \}}
\end{mathpar}

\begin{eqnarray*}
  \\
  \meaningof{-} : TS \to ST
\end{eqnarray*}

\begin{eqnarray*}
  \\
  L : TS \to ST
\end{eqnarray*}

\begin{eqnarray*}
  \\
  P \models E \iff P \in \meaningof{E}
\end{eqnarray*}

\begin{eqnarray*}
  P \approx_{L} Q \iff \forall E \in L. P \models E \iff Q \models E
\end{eqnarray*}

\begin{eqnarray*}
  P \approx_{K} Q
\end{eqnarray*}

\begin{eqnarray*}
  P \approx Q
\end{eqnarray*}

$\approx_{K} = \approx = \approx_{L}$

\subsubsection{Contextual duality}

Note that contexts extend the quotation operation to a family of
operations from processes to names. Given a context, $M$, we can
define a \emph{nominal context}, $\quotep{M}$ by $\quotep{M}[P] :=
\quotep{M[P]}$. To foreshadow what is to come we observe that these
operations enjoy a duality with processes very much like the duality
between vectors and maps from vectors to scalars.

Further, because the calculus is essentially higher-order, we have a
correspondence between contexts and processes. More specifically,
given a name $x$ and a context $M$ we can construct $M^{*}_{x}$ such
that 

\begin{mathpar}
  M^{*}_{x} | \lift{x}{P} \red M[P]
\end{mathpar}

namely,

\begin{mathpar}
  M^{*}_{x} := x?(u).M[\dropn{u}]
\end{mathpar}

The dependence of $M^{*}_{x}$ on a name makes it an abstraction, 

\begin{mathpar}
  M^{*} := (x)x?(u).M[\dropn{u}]
\end{mathpar}

\subsection{Additional notation}

It will sometimes be convenient to denote the process a name
quotes. We already have the notation $x = \quotep{P}$, but it will be
convenient to introduce an alternate notation, $\procn{x}$, when we
want to emphasize the connection to the use of the name. Note that, by
virtue of name equivalence, $\quotep{\procn{x}} \nameeq x$; so, the
notation is consistent with previous definitions.

Further, because names have structure it is possible to effect
substitutions on the basis of that structure. This means we need to
upgrade our notation for substitutions, which we accomplish by
adapting comprehension notation. Thus,

\begin{mathpar}
  P\{ y / x : x \in S \}
\end{mathpar}

is interpreted to mean the process derived from P by replacing (in a
capture-avoiding manner) each occurrence of $x$ in $S$ by $y$. For example,

\begin{mathpar}
  P\{ \quotep{\procn{x}|\procn{x}} / x : x \in \freenames{P} \}
\end{mathpar}

will replace each (occurrence) of a free name $x$ in $P$ by
$\quotep{\procn{x}|\procn{x}}$.

Also, we will avail ourselves of the notation $x^{L}$ and $x^{R}$ to
denote injections of a name into disjoint copies of the name
space. There are numerous ways to accomplish this. One example can be
found in \cite{MeredithR05}. This notation overloads to vectors of
names: $\vec{x}^{\pi} := (x_{i}^{\pi} \; : \; 0 \leq i < |\vec{x}| )$ where $\pi \in \{L,R\}$.

We also use $P^{\Box} := P|\Box$.

In \cite{MeredithR05} an interpretation of the new operator is
given. It turns out that there are several possible interpretations
all enjoying the requisite algebraic properties of the operator (see
\cite{milner91polyadicpi}). We will therefore make liberal use of
$(\nu\; \vec{x})P$.

% subsection the_syntax_and_semantics_of_the_notation_system (end)   

\input{qm2pi.qmops} 

\input{qm2pi.sterngerlach} 

\input{qm2pi.metric} 

% section concurrent_process_calculi (end)

%\input{qm2pi.proofsketch}

% section proof sketch (end)

%\input{qm2pi.slviaknots} 

% section spatial logic via knots (end)

\input{qm2pi.conclusion}

% section conclusion (end)

%\input{qm2pi.dtcodes} 

% section wiring algorithm (end)

\input{qm2pi.ack} 

% section acknowledgments (end)

\newpage


\bibliographystyle{plain}   
\bibliography{../../biblios/main.bib}

\input{qm2pi.rhodetails}

\end{document}



% section front matter (end)

\section{Introduction}\label{sec:introduction} % (fold)
In this draft of the material i am going to have to dispense with the
usual writing conventions adopted in papers on these topics. i'm going
to have adopt whatever tone i need at the time i'm writing up the
calculations. Sometimes this may be very conversational; others it may
be the barest mathematical grunts; others still it may be that i have
lifted text from one of my other papers because the exposition of some
point was better said there. i hope that my readers are not unduly put
out by this decision. i'm not doing this to flout convention or be
rebellious. i find these calculations very technically challenging. To
keep everything going technically, something has to give; i have to
let go of some cognitive burden. So, the academic writing style --
with all of its trade-offs in terms of facilitating technical
communication -- is what i'm letting go of. Perhaps subsequent drafts
can be tightened and polished, but for now, i'm going to speak as if
we were sitting together in a coffee shop with a laptop, wifi and a
pad of paper and a pencil.

So, here's what i have to say. We -- you and i, comfortably ensconced
in our coffee shop and well-equipped with our tools -- can realize and
carry out the calculations of quantum mechanics over a very different
formal theory of dynamics, a formal theory of dynamics that
corresponds to a theory of concurrent computation with
\emph{reflection}. It has the advantage that the underlying theory is
already `quantized', but supports analogues all of the continuuous
operations. Strikingly, this underlying theory has recently been
connected with a notion of metric that we can show, by calculating
together, coincides with the metric induced by the inner product.

There are a lot of reasons why you might be interested in seeing
calculations of this form. Here's why i'm interested. For the past
several centuries there has been no competitor to the ``Newtonian''
account of dynamics. As a result the predominant share of accounts of
dynamical systems and situations have had to be formulated in terms of
the Newtonian machinery. i view this as an intellectually dangerous
position to occupy. Everything, despite it's intrinsic shape, turns
into a nail to be hit with this hammer. Recently, however, the theory
of computation has matured to the point where we have candidates for
theories of dynamics that offer very different perspective on
reasoning about dynamical systems and situations. Testing these
candidates against very successful accounts of dynamical situations,
like quantum mechanics, is going to give us some sense of how mature
they are and some measure of the quality of these accounts of
dynamics.

\subsection{Summary of contributions and outline of paper}

So, we're going to develop an interpretation of the operations of
quantum mechanics normally interpreted by Hilbert spaces and
operators. We're going to do this over a theory of computation. Note
that this is very different than the usual quantum computation program
which develops notions of computation over quantum mechanics. Rather,
we are developing a story that aligns with Wheeler's slogan: It from
Bit. To do this we will first provide an account of the theory of
computation at play here. Then we will dive into a calculation-driven
interpretation of the operations of quantum mechanics.

The reason we take this approach is that -- until very recently --
there hasn't been an axiomatic account of quantum mechanics. As a
result there has been no sharp delineation of the mathematical theory
supporting interpretation of the physical theory and the physical
theory, itself. So, ambient features of the maths are free to be
exploited (or supressed) without a real accounting of their physical
relevance. There is no sharp statement ``here's the physical theory''
qua \emph{theory} and ``here's the mathematical interpretation''
enabling a judgment of how faithful the interpretation is -- apart
from experimental observation. When there is an axiomatic account we
can judge how well a given mathematical formalism supports an
interpretation of the axioms, independent of
experimentation. Likewise, we can judge how well we have captured our
physical evidence and experience with our axiomatics, independent of
any specific mathematical implementation, with accidental detail that
may or may not have physical significance. 

In lieu of a fully fleshed out and vetted axiomatic account of quantum
mechanics, interpreting the operational notions in service of modeling
physical systems will have to suffice. In other words, we are not in
the business of providing a model of Hilbert spaces and operators. We
are in the business of providing a model of quantum mechanics because
we are motivated by testing our notions of dynamics against physical
theory; and, the predictive calculations of the physical theory must
serve as the best formulation -- shy of a fully fleshed out axiomatic
account -- of the physical theory itself (as they have for scientific
theories since time immemorial). Put another way, despite a
whole-hearted commitment to an It-from-Bit ontology, we are firmly
aligned with the shut-up-and-calculate camp as the best way to obtain
results either from the physical perspective or as a quality assurance
measure of our fledgling theory of dynamics.

In detail, we present a reflective process calculus. Then we develop
intuitive correspondences between the notions available in this
calculus and the usual physical notions supporting quantum mechanical
calculations. Thus, 

\begin{table}[htp]
  \center{
    \fbox{
      \begin{tabular}{c|c}
        quantum mechanics & process calculus \\
        \hline
        scalar & name \\
        state vector & process \\
        dual & contextual duals \\
        matrix & formal sums of process-context-dual pairs \\
        orthogonality & process annihilation \\
        inner product & execution-formula + quoting
      \end{tabular}
    }
  }
  \caption{QM - process calculi correspondences}
\end{table}

Then we tighten up these intuitions to operational definitions. We
employ the Dirac notation as the best proxy we can find for an
abstract syntax of the quantum mechanical notions. The definitions we
develop put us in contact with equational constraints coming from the
theory that we demonstrate the definitions and calculations satisfy.

This puts us in a position to shut up and calculate for the
Stern-Gerlach experimental set up, showing how these predictive
calculations become calculations on processes in our theory of a
reflective process calculus.

Penultimately, we demonstrate that the notion of metric coming from
the inner product coincides with the notion of metric available from
the theory of bisimulation. This demonstration gives us the right to
think of space as arising from behavior. Finally, we consider where we
might go from the new vantage point we have obtained.

% section introduction (end) 
 
% section introduction (end)

% \documentclass[12pt]{llncs}
%\documentclass{jktr}

\usepackage[pdftex]{hyperref}                   
\usepackage {listings}
\usepackage {mathpartir}
\usepackage{bcprules}
%\usepackage{listings}
                       
\usepackage{graphicx} 
%\usepackage[margins=2.5cm,nohead,nofoot]{geometry}
%\usepackage{geometry}
\usepackage{amsfonts}
\usepackage{amstext}
\usepackage{latexsym}
\usepackage{amssymb}
\usepackage{color}


%\include{myPreamble}
\include{qm2pi.local} 

%\ifpdf
%\usepackage[pdftex]{graphicx}
%\else
%\usepackage{graphicx}
%\fi

 % \ifpdf
%  \usepackage{pdfsync}
%  \if


%\title{Brief Article}
%\author{David F. Snyder}
%\author{L.G. Meredith}

%\address{Dept. of Math., Texas State University--San Marcos, San Marcos, TX 78666}
       
\pagestyle{empty}


\begin{document}

\lstset{language=[Objective]Caml,frame=shadowbox}

\input{qm2pi.front}

% section front matter (end)

\input{qm2pi.intro} 
 
% section introduction (end)

% \input{qm2pi.knotations} 

% section notation (end)

\input{qm2pi.process.calculi} 

% section concurrent_process_calculi_and_spatial_logics_ (end)
    
%\input{qm2pi.knots2pi} 

%\input{qm2pi.trefoil} 

%\input{qm2pi.mainthm} 

% subsection basic_interpretation (end)

%\input{qm2pi.rho.presentation} 
\subsection{The syntax and semantics of the notation system}\label{sub:the_syntax_and_semantics_of_the_notation_system} % (fold)

We now summarize a technical presentation of the calculus that
embodies our theory of dynamics. The typical presentation of such a
calculus follows the style of giving generators and relations on
them. The grammar, below, describing term constructors, freely
generates the set of processes, $\Proc$. This set is then quotiented
by a relation known as structural congruence and it is over this set
that the notion of dynamics is expressed. This presentation is
essentially that of \cite{MeredithR05} with the addition of
polyadicity and summation. For readability we have relegated some of
the technical subtleties to an appendix.

\subsubsection{Process grammar}\label{subsub:process_grammar}

\begin{mathpar}
  \inferrule* [lab=synchronization] {} {{M} \bc \pzero \;|\; x?F \;|\; x!C }
  \and
  \inferrule* [lab=abstraction] {} {{F} \bc (x)P}
  \and
  \inferrule* [lab=concretion] {} {{C} \bc \langle Q \rangle}
  \and
  \inferrule* [lab=process] {} {{P,Q} \bc M \;| \;P|Q \;|\; @{x}}
  \and
  \inferrule* [lab=name] {} {{x} \bc \quotep{P}}
\end{mathpar} 

Note that $\vec{x}$ (resp. $\vec{P}$) denotes a vector of names
(resp. processes) of length $|\vec{x}|$ (resp. $|\vec{P}|$). We adopt
the following useful abbreviations.

\begin{mathpar}
   x?(\vec{y}).P := x.(\vec{y})P \and  x\clift{\vec{P}} := x.\clift{\vec{P}}
   \and x!(y) := \lift{x}{\dropn{y}}
   \and \Pi_{i=0}^{n-1}P_i := P_0 | \ldots | P_{n-1}
\end{mathpar}

\subsubsection{Structural congruence}

\paragraph{Free and bound names and alpha-equivalence.} At the
core of structural equivalence is alpha-equivalence which identifies
process that are the same up to a change of variable. Formally, we
recognize the distinction between free and bound names. The free names
of a process, $\freenames{P}$, may be calculated recursively as
follows:

\begin{mathpar}
\freenames{\pzero} := \emptyset
  \and \\
  \freenames{x?(y).P} := \{ x \} \cup (\freenames{P} \setminus \{ y \})
  \and 
  \freenames{x!\langle P \rangle} := \{ x \} \cup \{ P \} 
  \and \\
  \freenames{P|Q} := \freenames{P} \cup \freenames{Q}
  \and \\
  \freenames{@{x}} := \{ x \}
\end{mathpar}

$\pi$
$\quotep{\pi}$

$\freenames{-} : \pi \to \mathcal{P}(\quotep{\pi})$

\begin{eqnarray*}
  \freenames{\pzero} & := & \emptyset \\
  \freenames{x?(y).P} & := & \{ x \} \cup (\freenames{P} \setminus \{ y \}) \\
  \freenames{x!\langle P \rangle} & := & \{ x \} \cup \{ P \} \\
  \freenames{P|Q} & := & \freenames{P} \cup \freenames{Q} \\
  \freenames{\dropn{x}} & := & \{ x \}
\end{eqnarray*}

The bound names of a process, $\boundnames{P}$, are those names occurring in $P$
that are not free. For example, in $x?(y).0$, the name $x$ is free, while $y$ is bound.

\begin{mathpar}
  \inferrule* [lab=monoidal-laws] {} { P|Q \equiv Q|P \and P|0 \equiv P \and P|(Q|R) \equiv (P|Q)|R }
\end{mathpar}

\begin{mathpar}
  \inferrule* [lab=alpha-equivalence] {} { (x)P \equiv (y)P\{y/x\} \and y \not\in \freenames{P} }
\end{mathpar}

\begin{definition}
Then two processes, $P,Q$, are alpha-equivalent if $P = Q\{\vec{y}/\vec{x}\}$ for
some $\vec{x} \in \boundnames{Q},\vec{y} \in \boundnames{P}$, where $Q\{\vec{y}/\vec{x}\}$
denotes the capture-avoiding substitution of $\vec{y}$ for $\vec{x}$ in $Q$.
\end{definition}

\begin{definition}
  The {\em structural congruence} \cite{SangiorgiWalker} , $\equiv$,
  between processes is the least congruence containing
  alpha-equivalence, satisfying the abelian monoid laws
  (associativity, commutativity and $\pzero$ as identity) for parallel
  composition $|$ and for summation $+$.
\end{definition}

\subsection{Name equivalence}

We take name equivalence, written $\nameeq$, to be the smallest
equivalence relation generated by the following rules.

\begin{mathpar}
\inferrule*[lab=Quote-drop]
{ }
{ \quotep{@{x}} \nameeq x }

\inferrule*[lab=Struct-equiv]
{ P \scong Q }
{ \quotep{P} \nameeq \quotep{Q} }
\end{mathpar}

The astute reader will have noticed that the mutual recursion of names
and processes imposes a mutual recursion on alpha-equivalence and
structural equivalence via name-equivalence. Fortunately, all of this
works out pleasantly and we may calculate in the natural way, free of
concern. The reader interested in the details is referred to the
appendix \ref{appendix:rho_details}.

\subsection{Substitution}

We use $\Proc$ for the set of processes, $\QProc$ for the set of
names, and $\id{\{}\vec{y} / \vec{x} \id{\}}$ to denote partial maps,
$s : \QProc \rightarrow \QProc$. A map, $s$ lifts, uniquely, to a map
on process terms, $\widehat{s} : \Proc \rightarrow \Proc$ by the
following equations.

\begin{mathpar}
  (0) \psubstp{Q}{P} := 0 \\
  (R \juxtap S) \psubstp{Q}{P}
  :=    
  (R)\psubstp{Q}{P} \juxtap (S) \psubstp{Q}{P} \\
  (x?(y).R) \psubstp{Q}{P}    
  :=    
  (x)\substp{Q}{P} (z)\concat( (R \psubstn{z}{y}) \psubstp{Q}{P} ) \\
  (\lift{x}{R}) \psubstp{Q}{P}  
  :=
  \lift{(x)\substp{Q}{P}}{ R \psubstp{Q}{P} } \\
%   (\dropn{x})  \psubstp{Q}{P}       
%   := 
%   \left\{ 
%     \begin{array}{ccc} 
%       \dropn{\quotep{Q}} & & x \nameeq \quotep{P} \\
%       \dropn{x} & & otherwise \\
%     \end{array}
%   \right. 
  (\dropn{x})  \psubstp{Q}{P}       
  := 
  \left\{ 
    \begin{array}{ccc} 
      Q & & x \nameeq \quotep{P} \\
      \dropn{x} & & otherwise \\
    \end{array}
  \right.
\end{mathpar}
 

where

\begin{eqnarray}
  (x)\id{\{} \lpquote Q \rpquote / \lpquote P \rpquote \id{\}}            = 
  \left\{ 
    \begin{array}{ccc}
      \lpquote Q \rpquote & & x \nameeq \lpquote P \rpquote \\
      x & & otherwise \\
    \end{array}
  \right. \nonumber
\end{eqnarray}

and $z$ is chosen distinct from $\quotep{P}$, $\quotep{Q}$, the free
names in $Q$, and all the names in $R$. Our $\alpha$-equivalence will
be built in the standard way from this substitution.

\begin{remark}\label{rem:no_self_referential_names}
  One consequence of these definitions is that $\forall P. \quotep{P}
  \not\in \freenames{P}$.
\end{remark}

\subsection{ Dynamic quote: an example }

Anticipating something of what's to come, consider applying the
substitution, $\widehat{\id{\{}u / z \id{\}}}$, to the following pair
of processes, $\lift{w}{y!(z)}$ and $w[ \lpquote y!(z) \rpquote ]$.

\begin{eqnarray}
	\lift{w}{y!(z)}\widehat{\id{\{}u / z \id{\}}}
		& = &
		\lift{w}{y!(u)} \nonumber\\
	w[ \lpquote y!(z) \rpquote ] \widehat{ \id{\{}u / z \id{\}} }
		& = &
		w[ \lpquote y!(z) \rpquote ] \nonumber
\end{eqnarray}

Because the body of the process between quotes is impervious to
substitution, we get radically different answers. In fact, by
examining the first process in an input context,
e.g. $x?(z).\lift{w}{y!(z)}$, we see that the process under the lift
operator may be shaped by prefixed inputs binding a name inside it. In
this sense, the lift operator will be seen as a way to dynamically
construct processes before reifying them as names.

Finally equipped with these standard features we can present the
dynamics of the calculus.

\subsubsection{Operational semantics} 

Finally, we introduce the computational dynamics. What marks these
algebras as distinct from other more traditionally studied algebraic
structures, e.g. vector spaces or polynomial rings, is the manner in
which dynamics is captured. In traditional structures, dynamics is typically
expressed through morphisms between such structures, as in linear maps
between vector spaces or morphisms between rings. In algebras
associated with the semantics of computation, the dynamics is
expressed as part of the algebraic structure itself, through a
reduction reduction relation typically denoted by $\red$. Below, we
give a recursive presentation of this relation for the calculus used
in the encoding.

$\red \subseteq \pi \times \pi$
$\red : \pi \to \mathcal{P}(\pi)$

\begin{mathpar}
  \inferrule* [lab=Comm] { \textsf{match}( x_{src}, x_{trgt} ) } { x_{trgt}?(y)P \; | \; x_{src}!\langle {Q} \rangle \red P\{\quotep{Q}/y}\} }
  \and \\
  \inferrule* [lab=Par] {{P} \red {P}'} {{{P} | {Q}} \red {{P}' | {Q}}}
  \and
  \inferrule* [lab=Equiv]{{{P} \scong {P}'} \andalso {{P}' \red {Q}'} \andalso {{Q}' \scong {Q}}}{{P} \red {Q}}
\end{mathpar}

\begin{eqnarray*}
  match_{\equiv} (\quotep{P},\quotep{Q}) & := & P \equiv Q \\
  match_{\dagger}(\quotep{P},\quotep{Q}) & := & \forall R. P|Q \red^{*} R => R \red^{*} 0 \\
  match_{K}(\quotep{P},\quotep{Q}) & := & K \mbox{ for some context } K
\end{eqnarray*}

$u?(x)P | u!\langle Q \rangle \red P\{\quotep{Q}/x\}$

%We write $\wred$ for $\red^*$, and $P\red$ if $\exists Q $ such that $ P \red Q$.
We write $P\red$ if $\exists Q $ such that $ P \red Q$ and $P\not\red$, otherwise.

\section{Replication}

As mentioned before, it is known that replication (and hence
recursion) can be implemented in a higher-order process algebra
\cite{SangiorgiWalker}. As our first example of calculation with the
machinery thus far presented we give the construction explicitly in
the {\rhoc}.

\begin{eqnarray}
	D_{x} & := & \prefix{x}{y}{(\binpar{\outputp{x}{y}}{@{y}})} \nonumber\\
	\bangp_{x}{P} & := & \binpar{{x}!\langle{\binpar{D_{x}}{P}}\rangle}{D_{x}} \nonumber
\end{eqnarray}

\begin{eqnarray}
	\bangp_{x}{P} & & \nonumber\\
	=
	& {x}!\langle{(\prefix{x}{y}{(\outputp{x}{y} | @{y})) | P}}\rangle 
	      | \prefix{x}{y}{(\outputp{x}{y} | @{y})} & \nonumber\\
	\red
	& (\outputp{x}{y} | @{y})\substn{\quotep{(\prefix{x}{y}{(@{y} | \outputp{x}{y})) | P}}}{y} & \nonumber\\
	=
	& \outputp{x}{\quotep{(\prefix{x}{y}{(\outputp{x}{y} | @{y})) | P}}}
	  | {(\prefix{x}{y}{(\outputp{x}{y} | @{y})) | P}} & \nonumber\\
	\red
	& \ldots & \nonumber\\
	\red^*
	& P | P | \ldots & \nonumber
\end{eqnarray}

Of course, this encoding, as an implementation, runs away, unfolding
$\bangp{P}$ eagerly. A lazier and more implementable replication
operator, restricted to input-guarded processes, may be obtained as follows.

\begin{eqnarray}
\bangp{\prefix{u}{v}{P}} 
	:= 
	\binpar{\lift{x}{\prefix{u}{v}{(\binpar{D(x)}{P})}}}{D(x)} \nonumber
\end{eqnarray}

\begin{remark}
  Note that the lazier definition still does not deal with summation
  or mixed summation (i.e. sums over input and output). The reader is
  invited to construct definitions of replication that deal with these
  features. 

  Further, the definitions are parameterized in a name, $x$. Can you,
  gentle reader, make a definition that eliminates this parameter and
  guarantees no accidental interaction between the replication
  machinery and the process being replicated -- i.e. no accidental
  sharing of names used by the process to get its work done and the
  name(s) used by the replication to effect copying. This latter
  revision of the definition of replication is crucial to obtaining
  the expected identity $!!P \sim !P$.
\end{remark}

\begin{remark}\label{rem:paradoxical_combinator}
  The reader familiar with the lambda calculus will have noticed the
  similarity between $D$ and the paradoxical combinator.

  [Ed. note: the existence of this seems to suggest we have to be more
  restrictive on the set of processes and names we admit if we are to
  support no-cloning.]
\end{remark}

\subsubsection{Bisimulation}

The computational dynamics gives rise to another kind of equivalence,
the equivalence of computational behavior. As previously mentioned
this is typically captured \emph{via} some form of bisimulation.

% The notion we use in this paper is weak barbed bisimulation
% \cite{milner91polyadicpi}.

The notion we use in this paper is derived from weak barbed
bisimulation \cite{milner91polyadicpi}. 

\begin{definition}
An \emph{observation relation}, $\downarrow_{\mathcal N}$, over a set
of names, $\mathcal N$, is the smallest relation satisfying the rules
below.

\infrule[Out-barb]{y \in {\mathcal N}, \; x \nameeq y}
		  {\outputp{x}{v} \downarrow_{\mathcal N} x}
\infrule[Par-barb]{\mbox{$P\downarrow_{\mathcal N} x$ or $Q\downarrow_{\mathcal N} x$}}
		  {\binpar{P}{Q} \downarrow_{\mathcal N} x}

We write $P \Downarrow_{\mathcal N} x$ if there is $Q$ such that 
$P \wred Q$ and $Q \downarrow_{\mathcal N} x$.
\end{definition}

\begin{definition}
%\label{def.bbisim}
An  ${\mathcal N}$-\emph{barbed bisimulation} over a set of names, ${\mathcal N}$, is a symmetric binary relation 
${\mathcal S}_{\mathcal N}$ between agents such that $P\rel{S}_{\mathcal N}Q$ implies:
\begin{enumerate}
\item If $P \red P'$ then $Q \wred Q'$ and $P'\rel{S}_{\mathcal N} Q'$.
\item If $P\downarrow_{\mathcal N} x$, then $Q\Downarrow_{\mathcal N} x$.
\end{enumerate}
$P$ is ${\mathcal N}$-barbed bisimilar to $Q$, written
$P \wbbisim_{\mathcal N} Q$, if $P \rel{S}_{\mathcal N} Q$ for some ${\mathcal N}$-barbed bisimulation ${\mathcal S}_{\mathcal N}$.
\end{definition}

$\mathcal{R} \subseteq \pi \times \pi$

$P \mathcal{R} Q => \forall P'. P \red P' \Rightarrow \exists Q'. Q \red Q', P' \mathcal{R} Q'$

$P \vdash x \Rightarrow Q \vdash x$

\begin{mathpar}
  \inferrule*[lab=Out-barb]{x \nameeq y}{{y}!\langle{Q}\rangle \vdash x}
  \and
  \inferrule*[lab=Par-barb]{\mbox{$P\vdash x$ or $Q\vdash x$}}{\binpar{P}{Q} \vdash x}
\end{mathpar}

\subsubsection{Contexts}

One of the principle advantages of computational calculi like the
$\pi$-calculus is a well-defined notion of context,
contextual-equivalence and a correlation between
contextual-equivalence and notions of bisimulation. The notion of
context allows the decomposition of a process into (sub-)process and
its syntactic environment, its context. Thus, a context may be
thought of as a process with a ``hole'' (written $\Box$) in it. The
application of a context $M$ to a process $P$, written $M[P]$, is
tantamount to filling the hole in $M$ with $P$. In this paper we do
not need the full weight of this theory, but do make use of the notion
of context in the proof the main theorem. 

\begin{mathpar}
  \inferrule* [lab=summation] {} {{M_{M},M_{N}} \bc \Box \;|\; x.M_{A} \;|\; M_{M}+M_{N}}
  \and
  \inferrule* [lab=agent] {} {{M_{A}} \bc (\vec{x})M_{P} \;| \; \clift{P_0,\ldots,M_{P},\ldots,P_N}}
  \and \\
  \inferrule* [lab=process] {} {{M_{P}} \bc M_{N} \;| \;P|M_{P} }
\end{mathpar} 

\begin{mathpar}
  \inferrule* [lab=sychronization] {} {M_{N} \bc \Box \;|\; x?M_{F} \;|\; x!M_{C}}
  \and
  \inferrule* [lab=abstraction] {} {{M_{F}} \bc (x)M_{P} }
  \and
  \inferrule* [lab=concretion] {} {{M_{C}} \bc \langle M_{P} \rangle }
  \and \\
  \inferrule* [lab=process] {} {{M_{P}} \bc M_{N} \;| \;P|M_{P} }
\end{mathpar}

\begin{definition}[contextual application] Given a context $M$, and
  process $P$, we define the \emph{contextual application}, $M[P] :=
  M\{P/\Box\}$. That is, the contextual application of M to P is the
  substitution of $P$ for $\Box$ in $M$.
\end{definition}

$\meaningof{-} : L \to \mathcal{P}(\pi)$

\begin{mathpar}
  \inferrule* [lab=collection] {} {\meaningof{true} = \pi, \and \meaningof{~E} = \pi \setminus \meaningof{E}, \and \meaningof{E_{1} \& E_{2}} = \meaningof{E_{1}} \cap \meaningof{E_{2}}}
\end{mathpar}

\begin{mathpar}
  \inferrule* [lab=structure] {} {\meaningof{0} = \{ P \in \pi | P \equiv 0 \}, \and \\ \meaningof{E_1 | E_2} = \{ P \in \pi | P \equiv P_{1} | P_{2}, P_{1} \in \meaningof{E_{1}}, P_{2} \in \meaningof{E_2}\} }
\end{mathpar}

\begin{mathpar}
 \inferrule* [lab=behavior] {} {\meaningof{\langle a?b \rangle E} = \{ P \in \pi | P \equiv Q | u?(y)P', \\ \and \\\\ \and \\ \;\;\; u \in \meaningof{a}, \forall z.P'\{z/y\} \in \meaningof{E\{z/b\}}\}, \and \\ \meaningof{a!E} = \{ P \in \pi | P \equiv Q | x!\langle P' \rangle, x \in \meaningof{a} P' \in \meaningof{E}\} }
\end{mathpar}

\begin{mathpar}
 \inferrule* [lab=nominal] {} {\meaningof{\quotep{E}} = \{ \quotep{P} \in \quotep{\pi} | P \in \meaningof{E} \}, \and \meaningof{\quotep{P}} = \{ \quotep{Q} \in \quotep{\pi} | P \equiv Q \} \and \\ \meaningof{@\quotep{E}} = \{ P \in \pi | P \equiv @x, x \in \meaningof{E} \}}
\end{mathpar}

\begin{eqnarray*}
  \\
  \meaningof{-} : TS \to ST
\end{eqnarray*}

\begin{eqnarray*}
  \\
  L : TS \to ST
\end{eqnarray*}

\begin{eqnarray*}
  \\
  P \models E \iff P \in \meaningof{E}
\end{eqnarray*}

\begin{eqnarray*}
  P \approx_{L} Q \iff \forall E \in L. P \models E \iff Q \models E
\end{eqnarray*}

\begin{eqnarray*}
  P \approx_{K} Q
\end{eqnarray*}

\begin{eqnarray*}
  P \approx Q
\end{eqnarray*}

$\approx_{K} = \approx = \approx_{L}$

\subsubsection{Contextual duality}

Note that contexts extend the quotation operation to a family of
operations from processes to names. Given a context, $M$, we can
define a \emph{nominal context}, $\quotep{M}$ by $\quotep{M}[P] :=
\quotep{M[P]}$. To foreshadow what is to come we observe that these
operations enjoy a duality with processes very much like the duality
between vectors and maps from vectors to scalars.

Further, because the calculus is essentially higher-order, we have a
correspondence between contexts and processes. More specifically,
given a name $x$ and a context $M$ we can construct $M^{*}_{x}$ such
that 

\begin{mathpar}
  M^{*}_{x} | \lift{x}{P} \red M[P]
\end{mathpar}

namely,

\begin{mathpar}
  M^{*}_{x} := x?(u).M[\dropn{u}]
\end{mathpar}

The dependence of $M^{*}_{x}$ on a name makes it an abstraction, 

\begin{mathpar}
  M^{*} := (x)x?(u).M[\dropn{u}]
\end{mathpar}

\subsection{Additional notation}

It will sometimes be convenient to denote the process a name
quotes. We already have the notation $x = \quotep{P}$, but it will be
convenient to introduce an alternate notation, $\procn{x}$, when we
want to emphasize the connection to the use of the name. Note that, by
virtue of name equivalence, $\quotep{\procn{x}} \nameeq x$; so, the
notation is consistent with previous definitions.

Further, because names have structure it is possible to effect
substitutions on the basis of that structure. This means we need to
upgrade our notation for substitutions, which we accomplish by
adapting comprehension notation. Thus,

\begin{mathpar}
  P\{ y / x : x \in S \}
\end{mathpar}

is interpreted to mean the process derived from P by replacing (in a
capture-avoiding manner) each occurrence of $x$ in $S$ by $y$. For example,

\begin{mathpar}
  P\{ \quotep{\procn{x}|\procn{x}} / x : x \in \freenames{P} \}
\end{mathpar}

will replace each (occurrence) of a free name $x$ in $P$ by
$\quotep{\procn{x}|\procn{x}}$.

Also, we will avail ourselves of the notation $x^{L}$ and $x^{R}$ to
denote injections of a name into disjoint copies of the name
space. There are numerous ways to accomplish this. One example can be
found in \cite{MeredithR05}. This notation overloads to vectors of
names: $\vec{x}^{\pi} := (x_{i}^{\pi} \; : \; 0 \leq i < |\vec{x}| )$ where $\pi \in \{L,R\}$.

We also use $P^{\Box} := P|\Box$.

In \cite{MeredithR05} an interpretation of the new operator is
given. It turns out that there are several possible interpretations
all enjoying the requisite algebraic properties of the operator (see
\cite{milner91polyadicpi}). We will therefore make liberal use of
$(\nu\; \vec{x})P$.

% subsection the_syntax_and_semantics_of_the_notation_system (end)   

\input{qm2pi.qmops} 

\input{qm2pi.sterngerlach} 

\input{qm2pi.metric} 

% section concurrent_process_calculi (end)

%\input{qm2pi.proofsketch}

% section proof sketch (end)

%\input{qm2pi.slviaknots} 

% section spatial logic via knots (end)

\input{qm2pi.conclusion}

% section conclusion (end)

%\input{qm2pi.dtcodes} 

% section wiring algorithm (end)

\input{qm2pi.ack} 

% section acknowledgments (end)

\newpage


\bibliographystyle{plain}   
\bibliography{../../biblios/main.bib}

\input{qm2pi.rhodetails}

\end{document}

 

% section notation (end)

\input{qm2pi.process.calculi} 

% section concurrent_process_calculi_and_spatial_logics_ (end)
    
%\documentclass[12pt]{llncs}
%\documentclass{jktr}

\usepackage[pdftex]{hyperref}                   
\usepackage {listings}
\usepackage {mathpartir}
\usepackage{bcprules}
%\usepackage{listings}
                       
\usepackage{graphicx} 
%\usepackage[margins=2.5cm,nohead,nofoot]{geometry}
%\usepackage{geometry}
\usepackage{amsfonts}
\usepackage{amstext}
\usepackage{latexsym}
\usepackage{amssymb}
\usepackage{color}


%\include{myPreamble}
\include{qm2pi.local} 

%\ifpdf
%\usepackage[pdftex]{graphicx}
%\else
%\usepackage{graphicx}
%\fi

 % \ifpdf
%  \usepackage{pdfsync}
%  \if


%\title{Brief Article}
%\author{David F. Snyder}
%\author{L.G. Meredith}

%\address{Dept. of Math., Texas State University--San Marcos, San Marcos, TX 78666}
       
\pagestyle{empty}


\begin{document}

\lstset{language=[Objective]Caml,frame=shadowbox}

\input{qm2pi.front}

% section front matter (end)

\input{qm2pi.intro} 
 
% section introduction (end)

% \input{qm2pi.knotations} 

% section notation (end)

\input{qm2pi.process.calculi} 

% section concurrent_process_calculi_and_spatial_logics_ (end)
    
%\input{qm2pi.knots2pi} 

%\input{qm2pi.trefoil} 

%\input{qm2pi.mainthm} 

% subsection basic_interpretation (end)

%\input{qm2pi.rho.presentation} 
\subsection{The syntax and semantics of the notation system}\label{sub:the_syntax_and_semantics_of_the_notation_system} % (fold)

We now summarize a technical presentation of the calculus that
embodies our theory of dynamics. The typical presentation of such a
calculus follows the style of giving generators and relations on
them. The grammar, below, describing term constructors, freely
generates the set of processes, $\Proc$. This set is then quotiented
by a relation known as structural congruence and it is over this set
that the notion of dynamics is expressed. This presentation is
essentially that of \cite{MeredithR05} with the addition of
polyadicity and summation. For readability we have relegated some of
the technical subtleties to an appendix.

\subsubsection{Process grammar}\label{subsub:process_grammar}

\begin{mathpar}
  \inferrule* [lab=synchronization] {} {{M} \bc \pzero \;|\; x?F \;|\; x!C }
  \and
  \inferrule* [lab=abstraction] {} {{F} \bc (x)P}
  \and
  \inferrule* [lab=concretion] {} {{C} \bc \langle Q \rangle}
  \and
  \inferrule* [lab=process] {} {{P,Q} \bc M \;| \;P|Q \;|\; @{x}}
  \and
  \inferrule* [lab=name] {} {{x} \bc \quotep{P}}
\end{mathpar} 

Note that $\vec{x}$ (resp. $\vec{P}$) denotes a vector of names
(resp. processes) of length $|\vec{x}|$ (resp. $|\vec{P}|$). We adopt
the following useful abbreviations.

\begin{mathpar}
   x?(\vec{y}).P := x.(\vec{y})P \and  x\clift{\vec{P}} := x.\clift{\vec{P}}
   \and x!(y) := \lift{x}{\dropn{y}}
   \and \Pi_{i=0}^{n-1}P_i := P_0 | \ldots | P_{n-1}
\end{mathpar}

\subsubsection{Structural congruence}

\paragraph{Free and bound names and alpha-equivalence.} At the
core of structural equivalence is alpha-equivalence which identifies
process that are the same up to a change of variable. Formally, we
recognize the distinction between free and bound names. The free names
of a process, $\freenames{P}$, may be calculated recursively as
follows:

\begin{mathpar}
\freenames{\pzero} := \emptyset
  \and \\
  \freenames{x?(y).P} := \{ x \} \cup (\freenames{P} \setminus \{ y \})
  \and 
  \freenames{x!\langle P \rangle} := \{ x \} \cup \{ P \} 
  \and \\
  \freenames{P|Q} := \freenames{P} \cup \freenames{Q}
  \and \\
  \freenames{@{x}} := \{ x \}
\end{mathpar}

$\pi$
$\quotep{\pi}$

$\freenames{-} : \pi \to \mathcal{P}(\quotep{\pi})$

\begin{eqnarray*}
  \freenames{\pzero} & := & \emptyset \\
  \freenames{x?(y).P} & := & \{ x \} \cup (\freenames{P} \setminus \{ y \}) \\
  \freenames{x!\langle P \rangle} & := & \{ x \} \cup \{ P \} \\
  \freenames{P|Q} & := & \freenames{P} \cup \freenames{Q} \\
  \freenames{\dropn{x}} & := & \{ x \}
\end{eqnarray*}

The bound names of a process, $\boundnames{P}$, are those names occurring in $P$
that are not free. For example, in $x?(y).0$, the name $x$ is free, while $y$ is bound.

\begin{mathpar}
  \inferrule* [lab=monoidal-laws] {} { P|Q \equiv Q|P \and P|0 \equiv P \and P|(Q|R) \equiv (P|Q)|R }
\end{mathpar}

\begin{mathpar}
  \inferrule* [lab=alpha-equivalence] {} { (x)P \equiv (y)P\{y/x\} \and y \not\in \freenames{P} }
\end{mathpar}

\begin{definition}
Then two processes, $P,Q$, are alpha-equivalent if $P = Q\{\vec{y}/\vec{x}\}$ for
some $\vec{x} \in \boundnames{Q},\vec{y} \in \boundnames{P}$, where $Q\{\vec{y}/\vec{x}\}$
denotes the capture-avoiding substitution of $\vec{y}$ for $\vec{x}$ in $Q$.
\end{definition}

\begin{definition}
  The {\em structural congruence} \cite{SangiorgiWalker} , $\equiv$,
  between processes is the least congruence containing
  alpha-equivalence, satisfying the abelian monoid laws
  (associativity, commutativity and $\pzero$ as identity) for parallel
  composition $|$ and for summation $+$.
\end{definition}

\subsection{Name equivalence}

We take name equivalence, written $\nameeq$, to be the smallest
equivalence relation generated by the following rules.

\begin{mathpar}
\inferrule*[lab=Quote-drop]
{ }
{ \quotep{@{x}} \nameeq x }

\inferrule*[lab=Struct-equiv]
{ P \scong Q }
{ \quotep{P} \nameeq \quotep{Q} }
\end{mathpar}

The astute reader will have noticed that the mutual recursion of names
and processes imposes a mutual recursion on alpha-equivalence and
structural equivalence via name-equivalence. Fortunately, all of this
works out pleasantly and we may calculate in the natural way, free of
concern. The reader interested in the details is referred to the
appendix \ref{appendix:rho_details}.

\subsection{Substitution}

We use $\Proc$ for the set of processes, $\QProc$ for the set of
names, and $\id{\{}\vec{y} / \vec{x} \id{\}}$ to denote partial maps,
$s : \QProc \rightarrow \QProc$. A map, $s$ lifts, uniquely, to a map
on process terms, $\widehat{s} : \Proc \rightarrow \Proc$ by the
following equations.

\begin{mathpar}
  (0) \psubstp{Q}{P} := 0 \\
  (R \juxtap S) \psubstp{Q}{P}
  :=    
  (R)\psubstp{Q}{P} \juxtap (S) \psubstp{Q}{P} \\
  (x?(y).R) \psubstp{Q}{P}    
  :=    
  (x)\substp{Q}{P} (z)\concat( (R \psubstn{z}{y}) \psubstp{Q}{P} ) \\
  (\lift{x}{R}) \psubstp{Q}{P}  
  :=
  \lift{(x)\substp{Q}{P}}{ R \psubstp{Q}{P} } \\
%   (\dropn{x})  \psubstp{Q}{P}       
%   := 
%   \left\{ 
%     \begin{array}{ccc} 
%       \dropn{\quotep{Q}} & & x \nameeq \quotep{P} \\
%       \dropn{x} & & otherwise \\
%     \end{array}
%   \right. 
  (\dropn{x})  \psubstp{Q}{P}       
  := 
  \left\{ 
    \begin{array}{ccc} 
      Q & & x \nameeq \quotep{P} \\
      \dropn{x} & & otherwise \\
    \end{array}
  \right.
\end{mathpar}
 

where

\begin{eqnarray}
  (x)\id{\{} \lpquote Q \rpquote / \lpquote P \rpquote \id{\}}            = 
  \left\{ 
    \begin{array}{ccc}
      \lpquote Q \rpquote & & x \nameeq \lpquote P \rpquote \\
      x & & otherwise \\
    \end{array}
  \right. \nonumber
\end{eqnarray}

and $z$ is chosen distinct from $\quotep{P}$, $\quotep{Q}$, the free
names in $Q$, and all the names in $R$. Our $\alpha$-equivalence will
be built in the standard way from this substitution.

\begin{remark}\label{rem:no_self_referential_names}
  One consequence of these definitions is that $\forall P. \quotep{P}
  \not\in \freenames{P}$.
\end{remark}

\subsection{ Dynamic quote: an example }

Anticipating something of what's to come, consider applying the
substitution, $\widehat{\id{\{}u / z \id{\}}}$, to the following pair
of processes, $\lift{w}{y!(z)}$ and $w[ \lpquote y!(z) \rpquote ]$.

\begin{eqnarray}
	\lift{w}{y!(z)}\widehat{\id{\{}u / z \id{\}}}
		& = &
		\lift{w}{y!(u)} \nonumber\\
	w[ \lpquote y!(z) \rpquote ] \widehat{ \id{\{}u / z \id{\}} }
		& = &
		w[ \lpquote y!(z) \rpquote ] \nonumber
\end{eqnarray}

Because the body of the process between quotes is impervious to
substitution, we get radically different answers. In fact, by
examining the first process in an input context,
e.g. $x?(z).\lift{w}{y!(z)}$, we see that the process under the lift
operator may be shaped by prefixed inputs binding a name inside it. In
this sense, the lift operator will be seen as a way to dynamically
construct processes before reifying them as names.

Finally equipped with these standard features we can present the
dynamics of the calculus.

\subsubsection{Operational semantics} 

Finally, we introduce the computational dynamics. What marks these
algebras as distinct from other more traditionally studied algebraic
structures, e.g. vector spaces or polynomial rings, is the manner in
which dynamics is captured. In traditional structures, dynamics is typically
expressed through morphisms between such structures, as in linear maps
between vector spaces or morphisms between rings. In algebras
associated with the semantics of computation, the dynamics is
expressed as part of the algebraic structure itself, through a
reduction reduction relation typically denoted by $\red$. Below, we
give a recursive presentation of this relation for the calculus used
in the encoding.

$\red \subseteq \pi \times \pi$
$\red : \pi \to \mathcal{P}(\pi)$

\begin{mathpar}
  \inferrule* [lab=Comm] { \textsf{match}( x_{src}, x_{trgt} ) } { x_{trgt}?(y)P \; | \; x_{src}!\langle {Q} \rangle \red P\{\quotep{Q}/y}\} }
  \and \\
  \inferrule* [lab=Par] {{P} \red {P}'} {{{P} | {Q}} \red {{P}' | {Q}}}
  \and
  \inferrule* [lab=Equiv]{{{P} \scong {P}'} \andalso {{P}' \red {Q}'} \andalso {{Q}' \scong {Q}}}{{P} \red {Q}}
\end{mathpar}

\begin{eqnarray*}
  match_{\equiv} (\quotep{P},\quotep{Q}) & := & P \equiv Q \\
  match_{\dagger}(\quotep{P},\quotep{Q}) & := & \forall R. P|Q \red^{*} R => R \red^{*} 0 \\
  match_{K}(\quotep{P},\quotep{Q}) & := & K \mbox{ for some context } K
\end{eqnarray*}

$u?(x)P | u!\langle Q \rangle \red P\{\quotep{Q}/x\}$

%We write $\wred$ for $\red^*$, and $P\red$ if $\exists Q $ such that $ P \red Q$.
We write $P\red$ if $\exists Q $ such that $ P \red Q$ and $P\not\red$, otherwise.

\section{Replication}

As mentioned before, it is known that replication (and hence
recursion) can be implemented in a higher-order process algebra
\cite{SangiorgiWalker}. As our first example of calculation with the
machinery thus far presented we give the construction explicitly in
the {\rhoc}.

\begin{eqnarray}
	D_{x} & := & \prefix{x}{y}{(\binpar{\outputp{x}{y}}{@{y}})} \nonumber\\
	\bangp_{x}{P} & := & \binpar{{x}!\langle{\binpar{D_{x}}{P}}\rangle}{D_{x}} \nonumber
\end{eqnarray}

\begin{eqnarray}
	\bangp_{x}{P} & & \nonumber\\
	=
	& {x}!\langle{(\prefix{x}{y}{(\outputp{x}{y} | @{y})) | P}}\rangle 
	      | \prefix{x}{y}{(\outputp{x}{y} | @{y})} & \nonumber\\
	\red
	& (\outputp{x}{y} | @{y})\substn{\quotep{(\prefix{x}{y}{(@{y} | \outputp{x}{y})) | P}}}{y} & \nonumber\\
	=
	& \outputp{x}{\quotep{(\prefix{x}{y}{(\outputp{x}{y} | @{y})) | P}}}
	  | {(\prefix{x}{y}{(\outputp{x}{y} | @{y})) | P}} & \nonumber\\
	\red
	& \ldots & \nonumber\\
	\red^*
	& P | P | \ldots & \nonumber
\end{eqnarray}

Of course, this encoding, as an implementation, runs away, unfolding
$\bangp{P}$ eagerly. A lazier and more implementable replication
operator, restricted to input-guarded processes, may be obtained as follows.

\begin{eqnarray}
\bangp{\prefix{u}{v}{P}} 
	:= 
	\binpar{\lift{x}{\prefix{u}{v}{(\binpar{D(x)}{P})}}}{D(x)} \nonumber
\end{eqnarray}

\begin{remark}
  Note that the lazier definition still does not deal with summation
  or mixed summation (i.e. sums over input and output). The reader is
  invited to construct definitions of replication that deal with these
  features. 

  Further, the definitions are parameterized in a name, $x$. Can you,
  gentle reader, make a definition that eliminates this parameter and
  guarantees no accidental interaction between the replication
  machinery and the process being replicated -- i.e. no accidental
  sharing of names used by the process to get its work done and the
  name(s) used by the replication to effect copying. This latter
  revision of the definition of replication is crucial to obtaining
  the expected identity $!!P \sim !P$.
\end{remark}

\begin{remark}\label{rem:paradoxical_combinator}
  The reader familiar with the lambda calculus will have noticed the
  similarity between $D$ and the paradoxical combinator.

  [Ed. note: the existence of this seems to suggest we have to be more
  restrictive on the set of processes and names we admit if we are to
  support no-cloning.]
\end{remark}

\subsubsection{Bisimulation}

The computational dynamics gives rise to another kind of equivalence,
the equivalence of computational behavior. As previously mentioned
this is typically captured \emph{via} some form of bisimulation.

% The notion we use in this paper is weak barbed bisimulation
% \cite{milner91polyadicpi}.

The notion we use in this paper is derived from weak barbed
bisimulation \cite{milner91polyadicpi}. 

\begin{definition}
An \emph{observation relation}, $\downarrow_{\mathcal N}$, over a set
of names, $\mathcal N$, is the smallest relation satisfying the rules
below.

\infrule[Out-barb]{y \in {\mathcal N}, \; x \nameeq y}
		  {\outputp{x}{v} \downarrow_{\mathcal N} x}
\infrule[Par-barb]{\mbox{$P\downarrow_{\mathcal N} x$ or $Q\downarrow_{\mathcal N} x$}}
		  {\binpar{P}{Q} \downarrow_{\mathcal N} x}

We write $P \Downarrow_{\mathcal N} x$ if there is $Q$ such that 
$P \wred Q$ and $Q \downarrow_{\mathcal N} x$.
\end{definition}

\begin{definition}
%\label{def.bbisim}
An  ${\mathcal N}$-\emph{barbed bisimulation} over a set of names, ${\mathcal N}$, is a symmetric binary relation 
${\mathcal S}_{\mathcal N}$ between agents such that $P\rel{S}_{\mathcal N}Q$ implies:
\begin{enumerate}
\item If $P \red P'$ then $Q \wred Q'$ and $P'\rel{S}_{\mathcal N} Q'$.
\item If $P\downarrow_{\mathcal N} x$, then $Q\Downarrow_{\mathcal N} x$.
\end{enumerate}
$P$ is ${\mathcal N}$-barbed bisimilar to $Q$, written
$P \wbbisim_{\mathcal N} Q$, if $P \rel{S}_{\mathcal N} Q$ for some ${\mathcal N}$-barbed bisimulation ${\mathcal S}_{\mathcal N}$.
\end{definition}

$\mathcal{R} \subseteq \pi \times \pi$

$P \mathcal{R} Q => \forall P'. P \red P' \Rightarrow \exists Q'. Q \red Q', P' \mathcal{R} Q'$

$P \vdash x \Rightarrow Q \vdash x$

\begin{mathpar}
  \inferrule*[lab=Out-barb]{x \nameeq y}{{y}!\langle{Q}\rangle \vdash x}
  \and
  \inferrule*[lab=Par-barb]{\mbox{$P\vdash x$ or $Q\vdash x$}}{\binpar{P}{Q} \vdash x}
\end{mathpar}

\subsubsection{Contexts}

One of the principle advantages of computational calculi like the
$\pi$-calculus is a well-defined notion of context,
contextual-equivalence and a correlation between
contextual-equivalence and notions of bisimulation. The notion of
context allows the decomposition of a process into (sub-)process and
its syntactic environment, its context. Thus, a context may be
thought of as a process with a ``hole'' (written $\Box$) in it. The
application of a context $M$ to a process $P$, written $M[P]$, is
tantamount to filling the hole in $M$ with $P$. In this paper we do
not need the full weight of this theory, but do make use of the notion
of context in the proof the main theorem. 

\begin{mathpar}
  \inferrule* [lab=summation] {} {{M_{M},M_{N}} \bc \Box \;|\; x.M_{A} \;|\; M_{M}+M_{N}}
  \and
  \inferrule* [lab=agent] {} {{M_{A}} \bc (\vec{x})M_{P} \;| \; \clift{P_0,\ldots,M_{P},\ldots,P_N}}
  \and \\
  \inferrule* [lab=process] {} {{M_{P}} \bc M_{N} \;| \;P|M_{P} }
\end{mathpar} 

\begin{mathpar}
  \inferrule* [lab=sychronization] {} {M_{N} \bc \Box \;|\; x?M_{F} \;|\; x!M_{C}}
  \and
  \inferrule* [lab=abstraction] {} {{M_{F}} \bc (x)M_{P} }
  \and
  \inferrule* [lab=concretion] {} {{M_{C}} \bc \langle M_{P} \rangle }
  \and \\
  \inferrule* [lab=process] {} {{M_{P}} \bc M_{N} \;| \;P|M_{P} }
\end{mathpar}

\begin{definition}[contextual application] Given a context $M$, and
  process $P$, we define the \emph{contextual application}, $M[P] :=
  M\{P/\Box\}$. That is, the contextual application of M to P is the
  substitution of $P$ for $\Box$ in $M$.
\end{definition}

$\meaningof{-} : L \to \mathcal{P}(\pi)$

\begin{mathpar}
  \inferrule* [lab=collection] {} {\meaningof{true} = \pi, \and \meaningof{~E} = \pi \setminus \meaningof{E}, \and \meaningof{E_{1} \& E_{2}} = \meaningof{E_{1}} \cap \meaningof{E_{2}}}
\end{mathpar}

\begin{mathpar}
  \inferrule* [lab=structure] {} {\meaningof{0} = \{ P \in \pi | P \equiv 0 \}, \and \\ \meaningof{E_1 | E_2} = \{ P \in \pi | P \equiv P_{1} | P_{2}, P_{1} \in \meaningof{E_{1}}, P_{2} \in \meaningof{E_2}\} }
\end{mathpar}

\begin{mathpar}
 \inferrule* [lab=behavior] {} {\meaningof{\langle a?b \rangle E} = \{ P \in \pi | P \equiv Q | u?(y)P', \\ \and \\\\ \and \\ \;\;\; u \in \meaningof{a}, \forall z.P'\{z/y\} \in \meaningof{E\{z/b\}}\}, \and \\ \meaningof{a!E} = \{ P \in \pi | P \equiv Q | x!\langle P' \rangle, x \in \meaningof{a} P' \in \meaningof{E}\} }
\end{mathpar}

\begin{mathpar}
 \inferrule* [lab=nominal] {} {\meaningof{\quotep{E}} = \{ \quotep{P} \in \quotep{\pi} | P \in \meaningof{E} \}, \and \meaningof{\quotep{P}} = \{ \quotep{Q} \in \quotep{\pi} | P \equiv Q \} \and \\ \meaningof{@\quotep{E}} = \{ P \in \pi | P \equiv @x, x \in \meaningof{E} \}}
\end{mathpar}

\begin{eqnarray*}
  \\
  \meaningof{-} : TS \to ST
\end{eqnarray*}

\begin{eqnarray*}
  \\
  L : TS \to ST
\end{eqnarray*}

\begin{eqnarray*}
  \\
  P \models E \iff P \in \meaningof{E}
\end{eqnarray*}

\begin{eqnarray*}
  P \approx_{L} Q \iff \forall E \in L. P \models E \iff Q \models E
\end{eqnarray*}

\begin{eqnarray*}
  P \approx_{K} Q
\end{eqnarray*}

\begin{eqnarray*}
  P \approx Q
\end{eqnarray*}

$\approx_{K} = \approx = \approx_{L}$

\subsubsection{Contextual duality}

Note that contexts extend the quotation operation to a family of
operations from processes to names. Given a context, $M$, we can
define a \emph{nominal context}, $\quotep{M}$ by $\quotep{M}[P] :=
\quotep{M[P]}$. To foreshadow what is to come we observe that these
operations enjoy a duality with processes very much like the duality
between vectors and maps from vectors to scalars.

Further, because the calculus is essentially higher-order, we have a
correspondence between contexts and processes. More specifically,
given a name $x$ and a context $M$ we can construct $M^{*}_{x}$ such
that 

\begin{mathpar}
  M^{*}_{x} | \lift{x}{P} \red M[P]
\end{mathpar}

namely,

\begin{mathpar}
  M^{*}_{x} := x?(u).M[\dropn{u}]
\end{mathpar}

The dependence of $M^{*}_{x}$ on a name makes it an abstraction, 

\begin{mathpar}
  M^{*} := (x)x?(u).M[\dropn{u}]
\end{mathpar}

\subsection{Additional notation}

It will sometimes be convenient to denote the process a name
quotes. We already have the notation $x = \quotep{P}$, but it will be
convenient to introduce an alternate notation, $\procn{x}$, when we
want to emphasize the connection to the use of the name. Note that, by
virtue of name equivalence, $\quotep{\procn{x}} \nameeq x$; so, the
notation is consistent with previous definitions.

Further, because names have structure it is possible to effect
substitutions on the basis of that structure. This means we need to
upgrade our notation for substitutions, which we accomplish by
adapting comprehension notation. Thus,

\begin{mathpar}
  P\{ y / x : x \in S \}
\end{mathpar}

is interpreted to mean the process derived from P by replacing (in a
capture-avoiding manner) each occurrence of $x$ in $S$ by $y$. For example,

\begin{mathpar}
  P\{ \quotep{\procn{x}|\procn{x}} / x : x \in \freenames{P} \}
\end{mathpar}

will replace each (occurrence) of a free name $x$ in $P$ by
$\quotep{\procn{x}|\procn{x}}$.

Also, we will avail ourselves of the notation $x^{L}$ and $x^{R}$ to
denote injections of a name into disjoint copies of the name
space. There are numerous ways to accomplish this. One example can be
found in \cite{MeredithR05}. This notation overloads to vectors of
names: $\vec{x}^{\pi} := (x_{i}^{\pi} \; : \; 0 \leq i < |\vec{x}| )$ where $\pi \in \{L,R\}$.

We also use $P^{\Box} := P|\Box$.

In \cite{MeredithR05} an interpretation of the new operator is
given. It turns out that there are several possible interpretations
all enjoying the requisite algebraic properties of the operator (see
\cite{milner91polyadicpi}). We will therefore make liberal use of
$(\nu\; \vec{x})P$.

% subsection the_syntax_and_semantics_of_the_notation_system (end)   

\input{qm2pi.qmops} 

\input{qm2pi.sterngerlach} 

\input{qm2pi.metric} 

% section concurrent_process_calculi (end)

%\input{qm2pi.proofsketch}

% section proof sketch (end)

%\input{qm2pi.slviaknots} 

% section spatial logic via knots (end)

\input{qm2pi.conclusion}

% section conclusion (end)

%\input{qm2pi.dtcodes} 

% section wiring algorithm (end)

\input{qm2pi.ack} 

% section acknowledgments (end)

\newpage


\bibliographystyle{plain}   
\bibliography{../../biblios/main.bib}

\input{qm2pi.rhodetails}

\end{document}

 

%\documentclass[12pt]{llncs}
%\documentclass{jktr}

\usepackage[pdftex]{hyperref}                   
\usepackage {listings}
\usepackage {mathpartir}
\usepackage{bcprules}
%\usepackage{listings}
                       
\usepackage{graphicx} 
%\usepackage[margins=2.5cm,nohead,nofoot]{geometry}
%\usepackage{geometry}
\usepackage{amsfonts}
\usepackage{amstext}
\usepackage{latexsym}
\usepackage{amssymb}
\usepackage{color}


%\include{myPreamble}
\include{qm2pi.local} 

%\ifpdf
%\usepackage[pdftex]{graphicx}
%\else
%\usepackage{graphicx}
%\fi

 % \ifpdf
%  \usepackage{pdfsync}
%  \if


%\title{Brief Article}
%\author{David F. Snyder}
%\author{L.G. Meredith}

%\address{Dept. of Math., Texas State University--San Marcos, San Marcos, TX 78666}
       
\pagestyle{empty}


\begin{document}

\lstset{language=[Objective]Caml,frame=shadowbox}

\input{qm2pi.front}

% section front matter (end)

\input{qm2pi.intro} 
 
% section introduction (end)

% \input{qm2pi.knotations} 

% section notation (end)

\input{qm2pi.process.calculi} 

% section concurrent_process_calculi_and_spatial_logics_ (end)
    
%\input{qm2pi.knots2pi} 

%\input{qm2pi.trefoil} 

%\input{qm2pi.mainthm} 

% subsection basic_interpretation (end)

%\input{qm2pi.rho.presentation} 
\subsection{The syntax and semantics of the notation system}\label{sub:the_syntax_and_semantics_of_the_notation_system} % (fold)

We now summarize a technical presentation of the calculus that
embodies our theory of dynamics. The typical presentation of such a
calculus follows the style of giving generators and relations on
them. The grammar, below, describing term constructors, freely
generates the set of processes, $\Proc$. This set is then quotiented
by a relation known as structural congruence and it is over this set
that the notion of dynamics is expressed. This presentation is
essentially that of \cite{MeredithR05} with the addition of
polyadicity and summation. For readability we have relegated some of
the technical subtleties to an appendix.

\subsubsection{Process grammar}\label{subsub:process_grammar}

\begin{mathpar}
  \inferrule* [lab=synchronization] {} {{M} \bc \pzero \;|\; x?F \;|\; x!C }
  \and
  \inferrule* [lab=abstraction] {} {{F} \bc (x)P}
  \and
  \inferrule* [lab=concretion] {} {{C} \bc \langle Q \rangle}
  \and
  \inferrule* [lab=process] {} {{P,Q} \bc M \;| \;P|Q \;|\; @{x}}
  \and
  \inferrule* [lab=name] {} {{x} \bc \quotep{P}}
\end{mathpar} 

Note that $\vec{x}$ (resp. $\vec{P}$) denotes a vector of names
(resp. processes) of length $|\vec{x}|$ (resp. $|\vec{P}|$). We adopt
the following useful abbreviations.

\begin{mathpar}
   x?(\vec{y}).P := x.(\vec{y})P \and  x\clift{\vec{P}} := x.\clift{\vec{P}}
   \and x!(y) := \lift{x}{\dropn{y}}
   \and \Pi_{i=0}^{n-1}P_i := P_0 | \ldots | P_{n-1}
\end{mathpar}

\subsubsection{Structural congruence}

\paragraph{Free and bound names and alpha-equivalence.} At the
core of structural equivalence is alpha-equivalence which identifies
process that are the same up to a change of variable. Formally, we
recognize the distinction between free and bound names. The free names
of a process, $\freenames{P}$, may be calculated recursively as
follows:

\begin{mathpar}
\freenames{\pzero} := \emptyset
  \and \\
  \freenames{x?(y).P} := \{ x \} \cup (\freenames{P} \setminus \{ y \})
  \and 
  \freenames{x!\langle P \rangle} := \{ x \} \cup \{ P \} 
  \and \\
  \freenames{P|Q} := \freenames{P} \cup \freenames{Q}
  \and \\
  \freenames{@{x}} := \{ x \}
\end{mathpar}

$\pi$
$\quotep{\pi}$

$\freenames{-} : \pi \to \mathcal{P}(\quotep{\pi})$

\begin{eqnarray*}
  \freenames{\pzero} & := & \emptyset \\
  \freenames{x?(y).P} & := & \{ x \} \cup (\freenames{P} \setminus \{ y \}) \\
  \freenames{x!\langle P \rangle} & := & \{ x \} \cup \{ P \} \\
  \freenames{P|Q} & := & \freenames{P} \cup \freenames{Q} \\
  \freenames{\dropn{x}} & := & \{ x \}
\end{eqnarray*}

The bound names of a process, $\boundnames{P}$, are those names occurring in $P$
that are not free. For example, in $x?(y).0$, the name $x$ is free, while $y$ is bound.

\begin{mathpar}
  \inferrule* [lab=monoidal-laws] {} { P|Q \equiv Q|P \and P|0 \equiv P \and P|(Q|R) \equiv (P|Q)|R }
\end{mathpar}

\begin{mathpar}
  \inferrule* [lab=alpha-equivalence] {} { (x)P \equiv (y)P\{y/x\} \and y \not\in \freenames{P} }
\end{mathpar}

\begin{definition}
Then two processes, $P,Q$, are alpha-equivalent if $P = Q\{\vec{y}/\vec{x}\}$ for
some $\vec{x} \in \boundnames{Q},\vec{y} \in \boundnames{P}$, where $Q\{\vec{y}/\vec{x}\}$
denotes the capture-avoiding substitution of $\vec{y}$ for $\vec{x}$ in $Q$.
\end{definition}

\begin{definition}
  The {\em structural congruence} \cite{SangiorgiWalker} , $\equiv$,
  between processes is the least congruence containing
  alpha-equivalence, satisfying the abelian monoid laws
  (associativity, commutativity and $\pzero$ as identity) for parallel
  composition $|$ and for summation $+$.
\end{definition}

\subsection{Name equivalence}

We take name equivalence, written $\nameeq$, to be the smallest
equivalence relation generated by the following rules.

\begin{mathpar}
\inferrule*[lab=Quote-drop]
{ }
{ \quotep{@{x}} \nameeq x }

\inferrule*[lab=Struct-equiv]
{ P \scong Q }
{ \quotep{P} \nameeq \quotep{Q} }
\end{mathpar}

The astute reader will have noticed that the mutual recursion of names
and processes imposes a mutual recursion on alpha-equivalence and
structural equivalence via name-equivalence. Fortunately, all of this
works out pleasantly and we may calculate in the natural way, free of
concern. The reader interested in the details is referred to the
appendix \ref{appendix:rho_details}.

\subsection{Substitution}

We use $\Proc$ for the set of processes, $\QProc$ for the set of
names, and $\id{\{}\vec{y} / \vec{x} \id{\}}$ to denote partial maps,
$s : \QProc \rightarrow \QProc$. A map, $s$ lifts, uniquely, to a map
on process terms, $\widehat{s} : \Proc \rightarrow \Proc$ by the
following equations.

\begin{mathpar}
  (0) \psubstp{Q}{P} := 0 \\
  (R \juxtap S) \psubstp{Q}{P}
  :=    
  (R)\psubstp{Q}{P} \juxtap (S) \psubstp{Q}{P} \\
  (x?(y).R) \psubstp{Q}{P}    
  :=    
  (x)\substp{Q}{P} (z)\concat( (R \psubstn{z}{y}) \psubstp{Q}{P} ) \\
  (\lift{x}{R}) \psubstp{Q}{P}  
  :=
  \lift{(x)\substp{Q}{P}}{ R \psubstp{Q}{P} } \\
%   (\dropn{x})  \psubstp{Q}{P}       
%   := 
%   \left\{ 
%     \begin{array}{ccc} 
%       \dropn{\quotep{Q}} & & x \nameeq \quotep{P} \\
%       \dropn{x} & & otherwise \\
%     \end{array}
%   \right. 
  (\dropn{x})  \psubstp{Q}{P}       
  := 
  \left\{ 
    \begin{array}{ccc} 
      Q & & x \nameeq \quotep{P} \\
      \dropn{x} & & otherwise \\
    \end{array}
  \right.
\end{mathpar}
 

where

\begin{eqnarray}
  (x)\id{\{} \lpquote Q \rpquote / \lpquote P \rpquote \id{\}}            = 
  \left\{ 
    \begin{array}{ccc}
      \lpquote Q \rpquote & & x \nameeq \lpquote P \rpquote \\
      x & & otherwise \\
    \end{array}
  \right. \nonumber
\end{eqnarray}

and $z$ is chosen distinct from $\quotep{P}$, $\quotep{Q}$, the free
names in $Q$, and all the names in $R$. Our $\alpha$-equivalence will
be built in the standard way from this substitution.

\begin{remark}\label{rem:no_self_referential_names}
  One consequence of these definitions is that $\forall P. \quotep{P}
  \not\in \freenames{P}$.
\end{remark}

\subsection{ Dynamic quote: an example }

Anticipating something of what's to come, consider applying the
substitution, $\widehat{\id{\{}u / z \id{\}}}$, to the following pair
of processes, $\lift{w}{y!(z)}$ and $w[ \lpquote y!(z) \rpquote ]$.

\begin{eqnarray}
	\lift{w}{y!(z)}\widehat{\id{\{}u / z \id{\}}}
		& = &
		\lift{w}{y!(u)} \nonumber\\
	w[ \lpquote y!(z) \rpquote ] \widehat{ \id{\{}u / z \id{\}} }
		& = &
		w[ \lpquote y!(z) \rpquote ] \nonumber
\end{eqnarray}

Because the body of the process between quotes is impervious to
substitution, we get radically different answers. In fact, by
examining the first process in an input context,
e.g. $x?(z).\lift{w}{y!(z)}$, we see that the process under the lift
operator may be shaped by prefixed inputs binding a name inside it. In
this sense, the lift operator will be seen as a way to dynamically
construct processes before reifying them as names.

Finally equipped with these standard features we can present the
dynamics of the calculus.

\subsubsection{Operational semantics} 

Finally, we introduce the computational dynamics. What marks these
algebras as distinct from other more traditionally studied algebraic
structures, e.g. vector spaces or polynomial rings, is the manner in
which dynamics is captured. In traditional structures, dynamics is typically
expressed through morphisms between such structures, as in linear maps
between vector spaces or morphisms between rings. In algebras
associated with the semantics of computation, the dynamics is
expressed as part of the algebraic structure itself, through a
reduction reduction relation typically denoted by $\red$. Below, we
give a recursive presentation of this relation for the calculus used
in the encoding.

$\red \subseteq \pi \times \pi$
$\red : \pi \to \mathcal{P}(\pi)$

\begin{mathpar}
  \inferrule* [lab=Comm] { \textsf{match}( x_{src}, x_{trgt} ) } { x_{trgt}?(y)P \; | \; x_{src}!\langle {Q} \rangle \red P\{\quotep{Q}/y}\} }
  \and \\
  \inferrule* [lab=Par] {{P} \red {P}'} {{{P} | {Q}} \red {{P}' | {Q}}}
  \and
  \inferrule* [lab=Equiv]{{{P} \scong {P}'} \andalso {{P}' \red {Q}'} \andalso {{Q}' \scong {Q}}}{{P} \red {Q}}
\end{mathpar}

\begin{eqnarray*}
  match_{\equiv} (\quotep{P},\quotep{Q}) & := & P \equiv Q \\
  match_{\dagger}(\quotep{P},\quotep{Q}) & := & \forall R. P|Q \red^{*} R => R \red^{*} 0 \\
  match_{K}(\quotep{P},\quotep{Q}) & := & K \mbox{ for some context } K
\end{eqnarray*}

$u?(x)P | u!\langle Q \rangle \red P\{\quotep{Q}/x\}$

%We write $\wred$ for $\red^*$, and $P\red$ if $\exists Q $ such that $ P \red Q$.
We write $P\red$ if $\exists Q $ such that $ P \red Q$ and $P\not\red$, otherwise.

\section{Replication}

As mentioned before, it is known that replication (and hence
recursion) can be implemented in a higher-order process algebra
\cite{SangiorgiWalker}. As our first example of calculation with the
machinery thus far presented we give the construction explicitly in
the {\rhoc}.

\begin{eqnarray}
	D_{x} & := & \prefix{x}{y}{(\binpar{\outputp{x}{y}}{@{y}})} \nonumber\\
	\bangp_{x}{P} & := & \binpar{{x}!\langle{\binpar{D_{x}}{P}}\rangle}{D_{x}} \nonumber
\end{eqnarray}

\begin{eqnarray}
	\bangp_{x}{P} & & \nonumber\\
	=
	& {x}!\langle{(\prefix{x}{y}{(\outputp{x}{y} | @{y})) | P}}\rangle 
	      | \prefix{x}{y}{(\outputp{x}{y} | @{y})} & \nonumber\\
	\red
	& (\outputp{x}{y} | @{y})\substn{\quotep{(\prefix{x}{y}{(@{y} | \outputp{x}{y})) | P}}}{y} & \nonumber\\
	=
	& \outputp{x}{\quotep{(\prefix{x}{y}{(\outputp{x}{y} | @{y})) | P}}}
	  | {(\prefix{x}{y}{(\outputp{x}{y} | @{y})) | P}} & \nonumber\\
	\red
	& \ldots & \nonumber\\
	\red^*
	& P | P | \ldots & \nonumber
\end{eqnarray}

Of course, this encoding, as an implementation, runs away, unfolding
$\bangp{P}$ eagerly. A lazier and more implementable replication
operator, restricted to input-guarded processes, may be obtained as follows.

\begin{eqnarray}
\bangp{\prefix{u}{v}{P}} 
	:= 
	\binpar{\lift{x}{\prefix{u}{v}{(\binpar{D(x)}{P})}}}{D(x)} \nonumber
\end{eqnarray}

\begin{remark}
  Note that the lazier definition still does not deal with summation
  or mixed summation (i.e. sums over input and output). The reader is
  invited to construct definitions of replication that deal with these
  features. 

  Further, the definitions are parameterized in a name, $x$. Can you,
  gentle reader, make a definition that eliminates this parameter and
  guarantees no accidental interaction between the replication
  machinery and the process being replicated -- i.e. no accidental
  sharing of names used by the process to get its work done and the
  name(s) used by the replication to effect copying. This latter
  revision of the definition of replication is crucial to obtaining
  the expected identity $!!P \sim !P$.
\end{remark}

\begin{remark}\label{rem:paradoxical_combinator}
  The reader familiar with the lambda calculus will have noticed the
  similarity between $D$ and the paradoxical combinator.

  [Ed. note: the existence of this seems to suggest we have to be more
  restrictive on the set of processes and names we admit if we are to
  support no-cloning.]
\end{remark}

\subsubsection{Bisimulation}

The computational dynamics gives rise to another kind of equivalence,
the equivalence of computational behavior. As previously mentioned
this is typically captured \emph{via} some form of bisimulation.

% The notion we use in this paper is weak barbed bisimulation
% \cite{milner91polyadicpi}.

The notion we use in this paper is derived from weak barbed
bisimulation \cite{milner91polyadicpi}. 

\begin{definition}
An \emph{observation relation}, $\downarrow_{\mathcal N}$, over a set
of names, $\mathcal N$, is the smallest relation satisfying the rules
below.

\infrule[Out-barb]{y \in {\mathcal N}, \; x \nameeq y}
		  {\outputp{x}{v} \downarrow_{\mathcal N} x}
\infrule[Par-barb]{\mbox{$P\downarrow_{\mathcal N} x$ or $Q\downarrow_{\mathcal N} x$}}
		  {\binpar{P}{Q} \downarrow_{\mathcal N} x}

We write $P \Downarrow_{\mathcal N} x$ if there is $Q$ such that 
$P \wred Q$ and $Q \downarrow_{\mathcal N} x$.
\end{definition}

\begin{definition}
%\label{def.bbisim}
An  ${\mathcal N}$-\emph{barbed bisimulation} over a set of names, ${\mathcal N}$, is a symmetric binary relation 
${\mathcal S}_{\mathcal N}$ between agents such that $P\rel{S}_{\mathcal N}Q$ implies:
\begin{enumerate}
\item If $P \red P'$ then $Q \wred Q'$ and $P'\rel{S}_{\mathcal N} Q'$.
\item If $P\downarrow_{\mathcal N} x$, then $Q\Downarrow_{\mathcal N} x$.
\end{enumerate}
$P$ is ${\mathcal N}$-barbed bisimilar to $Q$, written
$P \wbbisim_{\mathcal N} Q$, if $P \rel{S}_{\mathcal N} Q$ for some ${\mathcal N}$-barbed bisimulation ${\mathcal S}_{\mathcal N}$.
\end{definition}

$\mathcal{R} \subseteq \pi \times \pi$

$P \mathcal{R} Q => \forall P'. P \red P' \Rightarrow \exists Q'. Q \red Q', P' \mathcal{R} Q'$

$P \vdash x \Rightarrow Q \vdash x$

\begin{mathpar}
  \inferrule*[lab=Out-barb]{x \nameeq y}{{y}!\langle{Q}\rangle \vdash x}
  \and
  \inferrule*[lab=Par-barb]{\mbox{$P\vdash x$ or $Q\vdash x$}}{\binpar{P}{Q} \vdash x}
\end{mathpar}

\subsubsection{Contexts}

One of the principle advantages of computational calculi like the
$\pi$-calculus is a well-defined notion of context,
contextual-equivalence and a correlation between
contextual-equivalence and notions of bisimulation. The notion of
context allows the decomposition of a process into (sub-)process and
its syntactic environment, its context. Thus, a context may be
thought of as a process with a ``hole'' (written $\Box$) in it. The
application of a context $M$ to a process $P$, written $M[P]$, is
tantamount to filling the hole in $M$ with $P$. In this paper we do
not need the full weight of this theory, but do make use of the notion
of context in the proof the main theorem. 

\begin{mathpar}
  \inferrule* [lab=summation] {} {{M_{M},M_{N}} \bc \Box \;|\; x.M_{A} \;|\; M_{M}+M_{N}}
  \and
  \inferrule* [lab=agent] {} {{M_{A}} \bc (\vec{x})M_{P} \;| \; \clift{P_0,\ldots,M_{P},\ldots,P_N}}
  \and \\
  \inferrule* [lab=process] {} {{M_{P}} \bc M_{N} \;| \;P|M_{P} }
\end{mathpar} 

\begin{mathpar}
  \inferrule* [lab=sychronization] {} {M_{N} \bc \Box \;|\; x?M_{F} \;|\; x!M_{C}}
  \and
  \inferrule* [lab=abstraction] {} {{M_{F}} \bc (x)M_{P} }
  \and
  \inferrule* [lab=concretion] {} {{M_{C}} \bc \langle M_{P} \rangle }
  \and \\
  \inferrule* [lab=process] {} {{M_{P}} \bc M_{N} \;| \;P|M_{P} }
\end{mathpar}

\begin{definition}[contextual application] Given a context $M$, and
  process $P$, we define the \emph{contextual application}, $M[P] :=
  M\{P/\Box\}$. That is, the contextual application of M to P is the
  substitution of $P$ for $\Box$ in $M$.
\end{definition}

$\meaningof{-} : L \to \mathcal{P}(\pi)$

\begin{mathpar}
  \inferrule* [lab=collection] {} {\meaningof{true} = \pi, \and \meaningof{~E} = \pi \setminus \meaningof{E}, \and \meaningof{E_{1} \& E_{2}} = \meaningof{E_{1}} \cap \meaningof{E_{2}}}
\end{mathpar}

\begin{mathpar}
  \inferrule* [lab=structure] {} {\meaningof{0} = \{ P \in \pi | P \equiv 0 \}, \and \\ \meaningof{E_1 | E_2} = \{ P \in \pi | P \equiv P_{1} | P_{2}, P_{1} \in \meaningof{E_{1}}, P_{2} \in \meaningof{E_2}\} }
\end{mathpar}

\begin{mathpar}
 \inferrule* [lab=behavior] {} {\meaningof{\langle a?b \rangle E} = \{ P \in \pi | P \equiv Q | u?(y)P', \\ \and \\\\ \and \\ \;\;\; u \in \meaningof{a}, \forall z.P'\{z/y\} \in \meaningof{E\{z/b\}}\}, \and \\ \meaningof{a!E} = \{ P \in \pi | P \equiv Q | x!\langle P' \rangle, x \in \meaningof{a} P' \in \meaningof{E}\} }
\end{mathpar}

\begin{mathpar}
 \inferrule* [lab=nominal] {} {\meaningof{\quotep{E}} = \{ \quotep{P} \in \quotep{\pi} | P \in \meaningof{E} \}, \and \meaningof{\quotep{P}} = \{ \quotep{Q} \in \quotep{\pi} | P \equiv Q \} \and \\ \meaningof{@\quotep{E}} = \{ P \in \pi | P \equiv @x, x \in \meaningof{E} \}}
\end{mathpar}

\begin{eqnarray*}
  \\
  \meaningof{-} : TS \to ST
\end{eqnarray*}

\begin{eqnarray*}
  \\
  L : TS \to ST
\end{eqnarray*}

\begin{eqnarray*}
  \\
  P \models E \iff P \in \meaningof{E}
\end{eqnarray*}

\begin{eqnarray*}
  P \approx_{L} Q \iff \forall E \in L. P \models E \iff Q \models E
\end{eqnarray*}

\begin{eqnarray*}
  P \approx_{K} Q
\end{eqnarray*}

\begin{eqnarray*}
  P \approx Q
\end{eqnarray*}

$\approx_{K} = \approx = \approx_{L}$

\subsubsection{Contextual duality}

Note that contexts extend the quotation operation to a family of
operations from processes to names. Given a context, $M$, we can
define a \emph{nominal context}, $\quotep{M}$ by $\quotep{M}[P] :=
\quotep{M[P]}$. To foreshadow what is to come we observe that these
operations enjoy a duality with processes very much like the duality
between vectors and maps from vectors to scalars.

Further, because the calculus is essentially higher-order, we have a
correspondence between contexts and processes. More specifically,
given a name $x$ and a context $M$ we can construct $M^{*}_{x}$ such
that 

\begin{mathpar}
  M^{*}_{x} | \lift{x}{P} \red M[P]
\end{mathpar}

namely,

\begin{mathpar}
  M^{*}_{x} := x?(u).M[\dropn{u}]
\end{mathpar}

The dependence of $M^{*}_{x}$ on a name makes it an abstraction, 

\begin{mathpar}
  M^{*} := (x)x?(u).M[\dropn{u}]
\end{mathpar}

\subsection{Additional notation}

It will sometimes be convenient to denote the process a name
quotes. We already have the notation $x = \quotep{P}$, but it will be
convenient to introduce an alternate notation, $\procn{x}$, when we
want to emphasize the connection to the use of the name. Note that, by
virtue of name equivalence, $\quotep{\procn{x}} \nameeq x$; so, the
notation is consistent with previous definitions.

Further, because names have structure it is possible to effect
substitutions on the basis of that structure. This means we need to
upgrade our notation for substitutions, which we accomplish by
adapting comprehension notation. Thus,

\begin{mathpar}
  P\{ y / x : x \in S \}
\end{mathpar}

is interpreted to mean the process derived from P by replacing (in a
capture-avoiding manner) each occurrence of $x$ in $S$ by $y$. For example,

\begin{mathpar}
  P\{ \quotep{\procn{x}|\procn{x}} / x : x \in \freenames{P} \}
\end{mathpar}

will replace each (occurrence) of a free name $x$ in $P$ by
$\quotep{\procn{x}|\procn{x}}$.

Also, we will avail ourselves of the notation $x^{L}$ and $x^{R}$ to
denote injections of a name into disjoint copies of the name
space. There are numerous ways to accomplish this. One example can be
found in \cite{MeredithR05}. This notation overloads to vectors of
names: $\vec{x}^{\pi} := (x_{i}^{\pi} \; : \; 0 \leq i < |\vec{x}| )$ where $\pi \in \{L,R\}$.

We also use $P^{\Box} := P|\Box$.

In \cite{MeredithR05} an interpretation of the new operator is
given. It turns out that there are several possible interpretations
all enjoying the requisite algebraic properties of the operator (see
\cite{milner91polyadicpi}). We will therefore make liberal use of
$(\nu\; \vec{x})P$.

% subsection the_syntax_and_semantics_of_the_notation_system (end)   

\input{qm2pi.qmops} 

\input{qm2pi.sterngerlach} 

\input{qm2pi.metric} 

% section concurrent_process_calculi (end)

%\input{qm2pi.proofsketch}

% section proof sketch (end)

%\input{qm2pi.slviaknots} 

% section spatial logic via knots (end)

\input{qm2pi.conclusion}

% section conclusion (end)

%\input{qm2pi.dtcodes} 

% section wiring algorithm (end)

\input{qm2pi.ack} 

% section acknowledgments (end)

\newpage


\bibliographystyle{plain}   
\bibliography{../../biblios/main.bib}

\input{qm2pi.rhodetails}

\end{document}

 

%\documentclass[12pt]{llncs}
%\documentclass{jktr}

\usepackage[pdftex]{hyperref}                   
\usepackage {listings}
\usepackage {mathpartir}
\usepackage{bcprules}
%\usepackage{listings}
                       
\usepackage{graphicx} 
%\usepackage[margins=2.5cm,nohead,nofoot]{geometry}
%\usepackage{geometry}
\usepackage{amsfonts}
\usepackage{amstext}
\usepackage{latexsym}
\usepackage{amssymb}
\usepackage{color}


%\include{myPreamble}
\include{qm2pi.local} 

%\ifpdf
%\usepackage[pdftex]{graphicx}
%\else
%\usepackage{graphicx}
%\fi

 % \ifpdf
%  \usepackage{pdfsync}
%  \if


%\title{Brief Article}
%\author{David F. Snyder}
%\author{L.G. Meredith}

%\address{Dept. of Math., Texas State University--San Marcos, San Marcos, TX 78666}
       
\pagestyle{empty}


\begin{document}

\lstset{language=[Objective]Caml,frame=shadowbox}

\input{qm2pi.front}

% section front matter (end)

\input{qm2pi.intro} 
 
% section introduction (end)

% \input{qm2pi.knotations} 

% section notation (end)

\input{qm2pi.process.calculi} 

% section concurrent_process_calculi_and_spatial_logics_ (end)
    
%\input{qm2pi.knots2pi} 

%\input{qm2pi.trefoil} 

%\input{qm2pi.mainthm} 

% subsection basic_interpretation (end)

%\input{qm2pi.rho.presentation} 
\subsection{The syntax and semantics of the notation system}\label{sub:the_syntax_and_semantics_of_the_notation_system} % (fold)

We now summarize a technical presentation of the calculus that
embodies our theory of dynamics. The typical presentation of such a
calculus follows the style of giving generators and relations on
them. The grammar, below, describing term constructors, freely
generates the set of processes, $\Proc$. This set is then quotiented
by a relation known as structural congruence and it is over this set
that the notion of dynamics is expressed. This presentation is
essentially that of \cite{MeredithR05} with the addition of
polyadicity and summation. For readability we have relegated some of
the technical subtleties to an appendix.

\subsubsection{Process grammar}\label{subsub:process_grammar}

\begin{mathpar}
  \inferrule* [lab=synchronization] {} {{M} \bc \pzero \;|\; x?F \;|\; x!C }
  \and
  \inferrule* [lab=abstraction] {} {{F} \bc (x)P}
  \and
  \inferrule* [lab=concretion] {} {{C} \bc \langle Q \rangle}
  \and
  \inferrule* [lab=process] {} {{P,Q} \bc M \;| \;P|Q \;|\; @{x}}
  \and
  \inferrule* [lab=name] {} {{x} \bc \quotep{P}}
\end{mathpar} 

Note that $\vec{x}$ (resp. $\vec{P}$) denotes a vector of names
(resp. processes) of length $|\vec{x}|$ (resp. $|\vec{P}|$). We adopt
the following useful abbreviations.

\begin{mathpar}
   x?(\vec{y}).P := x.(\vec{y})P \and  x\clift{\vec{P}} := x.\clift{\vec{P}}
   \and x!(y) := \lift{x}{\dropn{y}}
   \and \Pi_{i=0}^{n-1}P_i := P_0 | \ldots | P_{n-1}
\end{mathpar}

\subsubsection{Structural congruence}

\paragraph{Free and bound names and alpha-equivalence.} At the
core of structural equivalence is alpha-equivalence which identifies
process that are the same up to a change of variable. Formally, we
recognize the distinction between free and bound names. The free names
of a process, $\freenames{P}$, may be calculated recursively as
follows:

\begin{mathpar}
\freenames{\pzero} := \emptyset
  \and \\
  \freenames{x?(y).P} := \{ x \} \cup (\freenames{P} \setminus \{ y \})
  \and 
  \freenames{x!\langle P \rangle} := \{ x \} \cup \{ P \} 
  \and \\
  \freenames{P|Q} := \freenames{P} \cup \freenames{Q}
  \and \\
  \freenames{@{x}} := \{ x \}
\end{mathpar}

$\pi$
$\quotep{\pi}$

$\freenames{-} : \pi \to \mathcal{P}(\quotep{\pi})$

\begin{eqnarray*}
  \freenames{\pzero} & := & \emptyset \\
  \freenames{x?(y).P} & := & \{ x \} \cup (\freenames{P} \setminus \{ y \}) \\
  \freenames{x!\langle P \rangle} & := & \{ x \} \cup \{ P \} \\
  \freenames{P|Q} & := & \freenames{P} \cup \freenames{Q} \\
  \freenames{\dropn{x}} & := & \{ x \}
\end{eqnarray*}

The bound names of a process, $\boundnames{P}$, are those names occurring in $P$
that are not free. For example, in $x?(y).0$, the name $x$ is free, while $y$ is bound.

\begin{mathpar}
  \inferrule* [lab=monoidal-laws] {} { P|Q \equiv Q|P \and P|0 \equiv P \and P|(Q|R) \equiv (P|Q)|R }
\end{mathpar}

\begin{mathpar}
  \inferrule* [lab=alpha-equivalence] {} { (x)P \equiv (y)P\{y/x\} \and y \not\in \freenames{P} }
\end{mathpar}

\begin{definition}
Then two processes, $P,Q$, are alpha-equivalent if $P = Q\{\vec{y}/\vec{x}\}$ for
some $\vec{x} \in \boundnames{Q},\vec{y} \in \boundnames{P}$, where $Q\{\vec{y}/\vec{x}\}$
denotes the capture-avoiding substitution of $\vec{y}$ for $\vec{x}$ in $Q$.
\end{definition}

\begin{definition}
  The {\em structural congruence} \cite{SangiorgiWalker} , $\equiv$,
  between processes is the least congruence containing
  alpha-equivalence, satisfying the abelian monoid laws
  (associativity, commutativity and $\pzero$ as identity) for parallel
  composition $|$ and for summation $+$.
\end{definition}

\subsection{Name equivalence}

We take name equivalence, written $\nameeq$, to be the smallest
equivalence relation generated by the following rules.

\begin{mathpar}
\inferrule*[lab=Quote-drop]
{ }
{ \quotep{@{x}} \nameeq x }

\inferrule*[lab=Struct-equiv]
{ P \scong Q }
{ \quotep{P} \nameeq \quotep{Q} }
\end{mathpar}

The astute reader will have noticed that the mutual recursion of names
and processes imposes a mutual recursion on alpha-equivalence and
structural equivalence via name-equivalence. Fortunately, all of this
works out pleasantly and we may calculate in the natural way, free of
concern. The reader interested in the details is referred to the
appendix \ref{appendix:rho_details}.

\subsection{Substitution}

We use $\Proc$ for the set of processes, $\QProc$ for the set of
names, and $\id{\{}\vec{y} / \vec{x} \id{\}}$ to denote partial maps,
$s : \QProc \rightarrow \QProc$. A map, $s$ lifts, uniquely, to a map
on process terms, $\widehat{s} : \Proc \rightarrow \Proc$ by the
following equations.

\begin{mathpar}
  (0) \psubstp{Q}{P} := 0 \\
  (R \juxtap S) \psubstp{Q}{P}
  :=    
  (R)\psubstp{Q}{P} \juxtap (S) \psubstp{Q}{P} \\
  (x?(y).R) \psubstp{Q}{P}    
  :=    
  (x)\substp{Q}{P} (z)\concat( (R \psubstn{z}{y}) \psubstp{Q}{P} ) \\
  (\lift{x}{R}) \psubstp{Q}{P}  
  :=
  \lift{(x)\substp{Q}{P}}{ R \psubstp{Q}{P} } \\
%   (\dropn{x})  \psubstp{Q}{P}       
%   := 
%   \left\{ 
%     \begin{array}{ccc} 
%       \dropn{\quotep{Q}} & & x \nameeq \quotep{P} \\
%       \dropn{x} & & otherwise \\
%     \end{array}
%   \right. 
  (\dropn{x})  \psubstp{Q}{P}       
  := 
  \left\{ 
    \begin{array}{ccc} 
      Q & & x \nameeq \quotep{P} \\
      \dropn{x} & & otherwise \\
    \end{array}
  \right.
\end{mathpar}
 

where

\begin{eqnarray}
  (x)\id{\{} \lpquote Q \rpquote / \lpquote P \rpquote \id{\}}            = 
  \left\{ 
    \begin{array}{ccc}
      \lpquote Q \rpquote & & x \nameeq \lpquote P \rpquote \\
      x & & otherwise \\
    \end{array}
  \right. \nonumber
\end{eqnarray}

and $z$ is chosen distinct from $\quotep{P}$, $\quotep{Q}$, the free
names in $Q$, and all the names in $R$. Our $\alpha$-equivalence will
be built in the standard way from this substitution.

\begin{remark}\label{rem:no_self_referential_names}
  One consequence of these definitions is that $\forall P. \quotep{P}
  \not\in \freenames{P}$.
\end{remark}

\subsection{ Dynamic quote: an example }

Anticipating something of what's to come, consider applying the
substitution, $\widehat{\id{\{}u / z \id{\}}}$, to the following pair
of processes, $\lift{w}{y!(z)}$ and $w[ \lpquote y!(z) \rpquote ]$.

\begin{eqnarray}
	\lift{w}{y!(z)}\widehat{\id{\{}u / z \id{\}}}
		& = &
		\lift{w}{y!(u)} \nonumber\\
	w[ \lpquote y!(z) \rpquote ] \widehat{ \id{\{}u / z \id{\}} }
		& = &
		w[ \lpquote y!(z) \rpquote ] \nonumber
\end{eqnarray}

Because the body of the process between quotes is impervious to
substitution, we get radically different answers. In fact, by
examining the first process in an input context,
e.g. $x?(z).\lift{w}{y!(z)}$, we see that the process under the lift
operator may be shaped by prefixed inputs binding a name inside it. In
this sense, the lift operator will be seen as a way to dynamically
construct processes before reifying them as names.

Finally equipped with these standard features we can present the
dynamics of the calculus.

\subsubsection{Operational semantics} 

Finally, we introduce the computational dynamics. What marks these
algebras as distinct from other more traditionally studied algebraic
structures, e.g. vector spaces or polynomial rings, is the manner in
which dynamics is captured. In traditional structures, dynamics is typically
expressed through morphisms between such structures, as in linear maps
between vector spaces or morphisms between rings. In algebras
associated with the semantics of computation, the dynamics is
expressed as part of the algebraic structure itself, through a
reduction reduction relation typically denoted by $\red$. Below, we
give a recursive presentation of this relation for the calculus used
in the encoding.

$\red \subseteq \pi \times \pi$
$\red : \pi \to \mathcal{P}(\pi)$

\begin{mathpar}
  \inferrule* [lab=Comm] { \textsf{match}( x_{src}, x_{trgt} ) } { x_{trgt}?(y)P \; | \; x_{src}!\langle {Q} \rangle \red P\{\quotep{Q}/y}\} }
  \and \\
  \inferrule* [lab=Par] {{P} \red {P}'} {{{P} | {Q}} \red {{P}' | {Q}}}
  \and
  \inferrule* [lab=Equiv]{{{P} \scong {P}'} \andalso {{P}' \red {Q}'} \andalso {{Q}' \scong {Q}}}{{P} \red {Q}}
\end{mathpar}

\begin{eqnarray*}
  match_{\equiv} (\quotep{P},\quotep{Q}) & := & P \equiv Q \\
  match_{\dagger}(\quotep{P},\quotep{Q}) & := & \forall R. P|Q \red^{*} R => R \red^{*} 0 \\
  match_{K}(\quotep{P},\quotep{Q}) & := & K \mbox{ for some context } K
\end{eqnarray*}

$u?(x)P | u!\langle Q \rangle \red P\{\quotep{Q}/x\}$

%We write $\wred$ for $\red^*$, and $P\red$ if $\exists Q $ such that $ P \red Q$.
We write $P\red$ if $\exists Q $ such that $ P \red Q$ and $P\not\red$, otherwise.

\section{Replication}

As mentioned before, it is known that replication (and hence
recursion) can be implemented in a higher-order process algebra
\cite{SangiorgiWalker}. As our first example of calculation with the
machinery thus far presented we give the construction explicitly in
the {\rhoc}.

\begin{eqnarray}
	D_{x} & := & \prefix{x}{y}{(\binpar{\outputp{x}{y}}{@{y}})} \nonumber\\
	\bangp_{x}{P} & := & \binpar{{x}!\langle{\binpar{D_{x}}{P}}\rangle}{D_{x}} \nonumber
\end{eqnarray}

\begin{eqnarray}
	\bangp_{x}{P} & & \nonumber\\
	=
	& {x}!\langle{(\prefix{x}{y}{(\outputp{x}{y} | @{y})) | P}}\rangle 
	      | \prefix{x}{y}{(\outputp{x}{y} | @{y})} & \nonumber\\
	\red
	& (\outputp{x}{y} | @{y})\substn{\quotep{(\prefix{x}{y}{(@{y} | \outputp{x}{y})) | P}}}{y} & \nonumber\\
	=
	& \outputp{x}{\quotep{(\prefix{x}{y}{(\outputp{x}{y} | @{y})) | P}}}
	  | {(\prefix{x}{y}{(\outputp{x}{y} | @{y})) | P}} & \nonumber\\
	\red
	& \ldots & \nonumber\\
	\red^*
	& P | P | \ldots & \nonumber
\end{eqnarray}

Of course, this encoding, as an implementation, runs away, unfolding
$\bangp{P}$ eagerly. A lazier and more implementable replication
operator, restricted to input-guarded processes, may be obtained as follows.

\begin{eqnarray}
\bangp{\prefix{u}{v}{P}} 
	:= 
	\binpar{\lift{x}{\prefix{u}{v}{(\binpar{D(x)}{P})}}}{D(x)} \nonumber
\end{eqnarray}

\begin{remark}
  Note that the lazier definition still does not deal with summation
  or mixed summation (i.e. sums over input and output). The reader is
  invited to construct definitions of replication that deal with these
  features. 

  Further, the definitions are parameterized in a name, $x$. Can you,
  gentle reader, make a definition that eliminates this parameter and
  guarantees no accidental interaction between the replication
  machinery and the process being replicated -- i.e. no accidental
  sharing of names used by the process to get its work done and the
  name(s) used by the replication to effect copying. This latter
  revision of the definition of replication is crucial to obtaining
  the expected identity $!!P \sim !P$.
\end{remark}

\begin{remark}\label{rem:paradoxical_combinator}
  The reader familiar with the lambda calculus will have noticed the
  similarity between $D$ and the paradoxical combinator.

  [Ed. note: the existence of this seems to suggest we have to be more
  restrictive on the set of processes and names we admit if we are to
  support no-cloning.]
\end{remark}

\subsubsection{Bisimulation}

The computational dynamics gives rise to another kind of equivalence,
the equivalence of computational behavior. As previously mentioned
this is typically captured \emph{via} some form of bisimulation.

% The notion we use in this paper is weak barbed bisimulation
% \cite{milner91polyadicpi}.

The notion we use in this paper is derived from weak barbed
bisimulation \cite{milner91polyadicpi}. 

\begin{definition}
An \emph{observation relation}, $\downarrow_{\mathcal N}$, over a set
of names, $\mathcal N$, is the smallest relation satisfying the rules
below.

\infrule[Out-barb]{y \in {\mathcal N}, \; x \nameeq y}
		  {\outputp{x}{v} \downarrow_{\mathcal N} x}
\infrule[Par-barb]{\mbox{$P\downarrow_{\mathcal N} x$ or $Q\downarrow_{\mathcal N} x$}}
		  {\binpar{P}{Q} \downarrow_{\mathcal N} x}

We write $P \Downarrow_{\mathcal N} x$ if there is $Q$ such that 
$P \wred Q$ and $Q \downarrow_{\mathcal N} x$.
\end{definition}

\begin{definition}
%\label{def.bbisim}
An  ${\mathcal N}$-\emph{barbed bisimulation} over a set of names, ${\mathcal N}$, is a symmetric binary relation 
${\mathcal S}_{\mathcal N}$ between agents such that $P\rel{S}_{\mathcal N}Q$ implies:
\begin{enumerate}
\item If $P \red P'$ then $Q \wred Q'$ and $P'\rel{S}_{\mathcal N} Q'$.
\item If $P\downarrow_{\mathcal N} x$, then $Q\Downarrow_{\mathcal N} x$.
\end{enumerate}
$P$ is ${\mathcal N}$-barbed bisimilar to $Q$, written
$P \wbbisim_{\mathcal N} Q$, if $P \rel{S}_{\mathcal N} Q$ for some ${\mathcal N}$-barbed bisimulation ${\mathcal S}_{\mathcal N}$.
\end{definition}

$\mathcal{R} \subseteq \pi \times \pi$

$P \mathcal{R} Q => \forall P'. P \red P' \Rightarrow \exists Q'. Q \red Q', P' \mathcal{R} Q'$

$P \vdash x \Rightarrow Q \vdash x$

\begin{mathpar}
  \inferrule*[lab=Out-barb]{x \nameeq y}{{y}!\langle{Q}\rangle \vdash x}
  \and
  \inferrule*[lab=Par-barb]{\mbox{$P\vdash x$ or $Q\vdash x$}}{\binpar{P}{Q} \vdash x}
\end{mathpar}

\subsubsection{Contexts}

One of the principle advantages of computational calculi like the
$\pi$-calculus is a well-defined notion of context,
contextual-equivalence and a correlation between
contextual-equivalence and notions of bisimulation. The notion of
context allows the decomposition of a process into (sub-)process and
its syntactic environment, its context. Thus, a context may be
thought of as a process with a ``hole'' (written $\Box$) in it. The
application of a context $M$ to a process $P$, written $M[P]$, is
tantamount to filling the hole in $M$ with $P$. In this paper we do
not need the full weight of this theory, but do make use of the notion
of context in the proof the main theorem. 

\begin{mathpar}
  \inferrule* [lab=summation] {} {{M_{M},M_{N}} \bc \Box \;|\; x.M_{A} \;|\; M_{M}+M_{N}}
  \and
  \inferrule* [lab=agent] {} {{M_{A}} \bc (\vec{x})M_{P} \;| \; \clift{P_0,\ldots,M_{P},\ldots,P_N}}
  \and \\
  \inferrule* [lab=process] {} {{M_{P}} \bc M_{N} \;| \;P|M_{P} }
\end{mathpar} 

\begin{mathpar}
  \inferrule* [lab=sychronization] {} {M_{N} \bc \Box \;|\; x?M_{F} \;|\; x!M_{C}}
  \and
  \inferrule* [lab=abstraction] {} {{M_{F}} \bc (x)M_{P} }
  \and
  \inferrule* [lab=concretion] {} {{M_{C}} \bc \langle M_{P} \rangle }
  \and \\
  \inferrule* [lab=process] {} {{M_{P}} \bc M_{N} \;| \;P|M_{P} }
\end{mathpar}

\begin{definition}[contextual application] Given a context $M$, and
  process $P$, we define the \emph{contextual application}, $M[P] :=
  M\{P/\Box\}$. That is, the contextual application of M to P is the
  substitution of $P$ for $\Box$ in $M$.
\end{definition}

$\meaningof{-} : L \to \mathcal{P}(\pi)$

\begin{mathpar}
  \inferrule* [lab=collection] {} {\meaningof{true} = \pi, \and \meaningof{~E} = \pi \setminus \meaningof{E}, \and \meaningof{E_{1} \& E_{2}} = \meaningof{E_{1}} \cap \meaningof{E_{2}}}
\end{mathpar}

\begin{mathpar}
  \inferrule* [lab=structure] {} {\meaningof{0} = \{ P \in \pi | P \equiv 0 \}, \and \\ \meaningof{E_1 | E_2} = \{ P \in \pi | P \equiv P_{1} | P_{2}, P_{1} \in \meaningof{E_{1}}, P_{2} \in \meaningof{E_2}\} }
\end{mathpar}

\begin{mathpar}
 \inferrule* [lab=behavior] {} {\meaningof{\langle a?b \rangle E} = \{ P \in \pi | P \equiv Q | u?(y)P', \\ \and \\\\ \and \\ \;\;\; u \in \meaningof{a}, \forall z.P'\{z/y\} \in \meaningof{E\{z/b\}}\}, \and \\ \meaningof{a!E} = \{ P \in \pi | P \equiv Q | x!\langle P' \rangle, x \in \meaningof{a} P' \in \meaningof{E}\} }
\end{mathpar}

\begin{mathpar}
 \inferrule* [lab=nominal] {} {\meaningof{\quotep{E}} = \{ \quotep{P} \in \quotep{\pi} | P \in \meaningof{E} \}, \and \meaningof{\quotep{P}} = \{ \quotep{Q} \in \quotep{\pi} | P \equiv Q \} \and \\ \meaningof{@\quotep{E}} = \{ P \in \pi | P \equiv @x, x \in \meaningof{E} \}}
\end{mathpar}

\begin{eqnarray*}
  \\
  \meaningof{-} : TS \to ST
\end{eqnarray*}

\begin{eqnarray*}
  \\
  L : TS \to ST
\end{eqnarray*}

\begin{eqnarray*}
  \\
  P \models E \iff P \in \meaningof{E}
\end{eqnarray*}

\begin{eqnarray*}
  P \approx_{L} Q \iff \forall E \in L. P \models E \iff Q \models E
\end{eqnarray*}

\begin{eqnarray*}
  P \approx_{K} Q
\end{eqnarray*}

\begin{eqnarray*}
  P \approx Q
\end{eqnarray*}

$\approx_{K} = \approx = \approx_{L}$

\subsubsection{Contextual duality}

Note that contexts extend the quotation operation to a family of
operations from processes to names. Given a context, $M$, we can
define a \emph{nominal context}, $\quotep{M}$ by $\quotep{M}[P] :=
\quotep{M[P]}$. To foreshadow what is to come we observe that these
operations enjoy a duality with processes very much like the duality
between vectors and maps from vectors to scalars.

Further, because the calculus is essentially higher-order, we have a
correspondence between contexts and processes. More specifically,
given a name $x$ and a context $M$ we can construct $M^{*}_{x}$ such
that 

\begin{mathpar}
  M^{*}_{x} | \lift{x}{P} \red M[P]
\end{mathpar}

namely,

\begin{mathpar}
  M^{*}_{x} := x?(u).M[\dropn{u}]
\end{mathpar}

The dependence of $M^{*}_{x}$ on a name makes it an abstraction, 

\begin{mathpar}
  M^{*} := (x)x?(u).M[\dropn{u}]
\end{mathpar}

\subsection{Additional notation}

It will sometimes be convenient to denote the process a name
quotes. We already have the notation $x = \quotep{P}$, but it will be
convenient to introduce an alternate notation, $\procn{x}$, when we
want to emphasize the connection to the use of the name. Note that, by
virtue of name equivalence, $\quotep{\procn{x}} \nameeq x$; so, the
notation is consistent with previous definitions.

Further, because names have structure it is possible to effect
substitutions on the basis of that structure. This means we need to
upgrade our notation for substitutions, which we accomplish by
adapting comprehension notation. Thus,

\begin{mathpar}
  P\{ y / x : x \in S \}
\end{mathpar}

is interpreted to mean the process derived from P by replacing (in a
capture-avoiding manner) each occurrence of $x$ in $S$ by $y$. For example,

\begin{mathpar}
  P\{ \quotep{\procn{x}|\procn{x}} / x : x \in \freenames{P} \}
\end{mathpar}

will replace each (occurrence) of a free name $x$ in $P$ by
$\quotep{\procn{x}|\procn{x}}$.

Also, we will avail ourselves of the notation $x^{L}$ and $x^{R}$ to
denote injections of a name into disjoint copies of the name
space. There are numerous ways to accomplish this. One example can be
found in \cite{MeredithR05}. This notation overloads to vectors of
names: $\vec{x}^{\pi} := (x_{i}^{\pi} \; : \; 0 \leq i < |\vec{x}| )$ where $\pi \in \{L,R\}$.

We also use $P^{\Box} := P|\Box$.

In \cite{MeredithR05} an interpretation of the new operator is
given. It turns out that there are several possible interpretations
all enjoying the requisite algebraic properties of the operator (see
\cite{milner91polyadicpi}). We will therefore make liberal use of
$(\nu\; \vec{x})P$.

% subsection the_syntax_and_semantics_of_the_notation_system (end)   

\input{qm2pi.qmops} 

\input{qm2pi.sterngerlach} 

\input{qm2pi.metric} 

% section concurrent_process_calculi (end)

%\input{qm2pi.proofsketch}

% section proof sketch (end)

%\input{qm2pi.slviaknots} 

% section spatial logic via knots (end)

\input{qm2pi.conclusion}

% section conclusion (end)

%\input{qm2pi.dtcodes} 

% section wiring algorithm (end)

\input{qm2pi.ack} 

% section acknowledgments (end)

\newpage


\bibliographystyle{plain}   
\bibliography{../../biblios/main.bib}

\input{qm2pi.rhodetails}

\end{document}

 

% subsection basic_interpretation (end)

%\input{qm2pi.rho.presentation} 
\subsection{The syntax and semantics of the notation system}\label{sub:the_syntax_and_semantics_of_the_notation_system} % (fold)

We now summarize a technical presentation of the calculus that
embodies our theory of dynamics. The typical presentation of such a
calculus follows the style of giving generators and relations on
them. The grammar, below, describing term constructors, freely
generates the set of processes, $\Proc$. This set is then quotiented
by a relation known as structural congruence and it is over this set
that the notion of dynamics is expressed. This presentation is
essentially that of \cite{MeredithR05} with the addition of
polyadicity and summation. For readability we have relegated some of
the technical subtleties to an appendix.

\subsubsection{Process grammar}\label{subsub:process_grammar}

\begin{mathpar}
  \inferrule* [lab=synchronization] {} {{M} \bc \pzero \;|\; x?F \;|\; x!C }
  \and
  \inferrule* [lab=abstraction] {} {{F} \bc (x)P}
  \and
  \inferrule* [lab=concretion] {} {{C} \bc \langle Q \rangle}
  \and
  \inferrule* [lab=process] {} {{P,Q} \bc M \;| \;P|Q \;|\; @{x}}
  \and
  \inferrule* [lab=name] {} {{x} \bc \quotep{P}}
\end{mathpar} 

Note that $\vec{x}$ (resp. $\vec{P}$) denotes a vector of names
(resp. processes) of length $|\vec{x}|$ (resp. $|\vec{P}|$). We adopt
the following useful abbreviations.

\begin{mathpar}
   x?(\vec{y}).P := x.(\vec{y})P \and  x\clift{\vec{P}} := x.\clift{\vec{P}}
   \and x!(y) := \lift{x}{\dropn{y}}
   \and \Pi_{i=0}^{n-1}P_i := P_0 | \ldots | P_{n-1}
\end{mathpar}

\subsubsection{Structural congruence}

\paragraph{Free and bound names and alpha-equivalence.} At the
core of structural equivalence is alpha-equivalence which identifies
process that are the same up to a change of variable. Formally, we
recognize the distinction between free and bound names. The free names
of a process, $\freenames{P}$, may be calculated recursively as
follows:

\begin{mathpar}
\freenames{\pzero} := \emptyset
  \and \\
  \freenames{x?(y).P} := \{ x \} \cup (\freenames{P} \setminus \{ y \})
  \and 
  \freenames{x!\langle P \rangle} := \{ x \} \cup \{ P \} 
  \and \\
  \freenames{P|Q} := \freenames{P} \cup \freenames{Q}
  \and \\
  \freenames{@{x}} := \{ x \}
\end{mathpar}

$\pi$
$\quotep{\pi}$

$\freenames{-} : \pi \to \mathcal{P}(\quotep{\pi})$

\begin{eqnarray*}
  \freenames{\pzero} & := & \emptyset \\
  \freenames{x?(y).P} & := & \{ x \} \cup (\freenames{P} \setminus \{ y \}) \\
  \freenames{x!\langle P \rangle} & := & \{ x \} \cup \{ P \} \\
  \freenames{P|Q} & := & \freenames{P} \cup \freenames{Q} \\
  \freenames{\dropn{x}} & := & \{ x \}
\end{eqnarray*}

The bound names of a process, $\boundnames{P}$, are those names occurring in $P$
that are not free. For example, in $x?(y).0$, the name $x$ is free, while $y$ is bound.

\begin{mathpar}
  \inferrule* [lab=monoidal-laws] {} { P|Q \equiv Q|P \and P|0 \equiv P \and P|(Q|R) \equiv (P|Q)|R }
\end{mathpar}

\begin{mathpar}
  \inferrule* [lab=alpha-equivalence] {} { (x)P \equiv (y)P\{y/x\} \and y \not\in \freenames{P} }
\end{mathpar}

\begin{definition}
Then two processes, $P,Q$, are alpha-equivalent if $P = Q\{\vec{y}/\vec{x}\}$ for
some $\vec{x} \in \boundnames{Q},\vec{y} \in \boundnames{P}$, where $Q\{\vec{y}/\vec{x}\}$
denotes the capture-avoiding substitution of $\vec{y}$ for $\vec{x}$ in $Q$.
\end{definition}

\begin{definition}
  The {\em structural congruence} \cite{SangiorgiWalker} , $\equiv$,
  between processes is the least congruence containing
  alpha-equivalence, satisfying the abelian monoid laws
  (associativity, commutativity and $\pzero$ as identity) for parallel
  composition $|$ and for summation $+$.
\end{definition}

\subsection{Name equivalence}

We take name equivalence, written $\nameeq$, to be the smallest
equivalence relation generated by the following rules.

\begin{mathpar}
\inferrule*[lab=Quote-drop]
{ }
{ \quotep{@{x}} \nameeq x }

\inferrule*[lab=Struct-equiv]
{ P \scong Q }
{ \quotep{P} \nameeq \quotep{Q} }
\end{mathpar}

The astute reader will have noticed that the mutual recursion of names
and processes imposes a mutual recursion on alpha-equivalence and
structural equivalence via name-equivalence. Fortunately, all of this
works out pleasantly and we may calculate in the natural way, free of
concern. The reader interested in the details is referred to the
appendix \ref{appendix:rho_details}.

\subsection{Substitution}

We use $\Proc$ for the set of processes, $\QProc$ for the set of
names, and $\id{\{}\vec{y} / \vec{x} \id{\}}$ to denote partial maps,
$s : \QProc \rightarrow \QProc$. A map, $s$ lifts, uniquely, to a map
on process terms, $\widehat{s} : \Proc \rightarrow \Proc$ by the
following equations.

\begin{mathpar}
  (0) \psubstp{Q}{P} := 0 \\
  (R \juxtap S) \psubstp{Q}{P}
  :=    
  (R)\psubstp{Q}{P} \juxtap (S) \psubstp{Q}{P} \\
  (x?(y).R) \psubstp{Q}{P}    
  :=    
  (x)\substp{Q}{P} (z)\concat( (R \psubstn{z}{y}) \psubstp{Q}{P} ) \\
  (\lift{x}{R}) \psubstp{Q}{P}  
  :=
  \lift{(x)\substp{Q}{P}}{ R \psubstp{Q}{P} } \\
%   (\dropn{x})  \psubstp{Q}{P}       
%   := 
%   \left\{ 
%     \begin{array}{ccc} 
%       \dropn{\quotep{Q}} & & x \nameeq \quotep{P} \\
%       \dropn{x} & & otherwise \\
%     \end{array}
%   \right. 
  (\dropn{x})  \psubstp{Q}{P}       
  := 
  \left\{ 
    \begin{array}{ccc} 
      Q & & x \nameeq \quotep{P} \\
      \dropn{x} & & otherwise \\
    \end{array}
  \right.
\end{mathpar}
 

where

\begin{eqnarray}
  (x)\id{\{} \lpquote Q \rpquote / \lpquote P \rpquote \id{\}}            = 
  \left\{ 
    \begin{array}{ccc}
      \lpquote Q \rpquote & & x \nameeq \lpquote P \rpquote \\
      x & & otherwise \\
    \end{array}
  \right. \nonumber
\end{eqnarray}

and $z$ is chosen distinct from $\quotep{P}$, $\quotep{Q}$, the free
names in $Q$, and all the names in $R$. Our $\alpha$-equivalence will
be built in the standard way from this substitution.

\begin{remark}\label{rem:no_self_referential_names}
  One consequence of these definitions is that $\forall P. \quotep{P}
  \not\in \freenames{P}$.
\end{remark}

\subsection{ Dynamic quote: an example }

Anticipating something of what's to come, consider applying the
substitution, $\widehat{\id{\{}u / z \id{\}}}$, to the following pair
of processes, $\lift{w}{y!(z)}$ and $w[ \lpquote y!(z) \rpquote ]$.

\begin{eqnarray}
	\lift{w}{y!(z)}\widehat{\id{\{}u / z \id{\}}}
		& = &
		\lift{w}{y!(u)} \nonumber\\
	w[ \lpquote y!(z) \rpquote ] \widehat{ \id{\{}u / z \id{\}} }
		& = &
		w[ \lpquote y!(z) \rpquote ] \nonumber
\end{eqnarray}

Because the body of the process between quotes is impervious to
substitution, we get radically different answers. In fact, by
examining the first process in an input context,
e.g. $x?(z).\lift{w}{y!(z)}$, we see that the process under the lift
operator may be shaped by prefixed inputs binding a name inside it. In
this sense, the lift operator will be seen as a way to dynamically
construct processes before reifying them as names.

Finally equipped with these standard features we can present the
dynamics of the calculus.

\subsubsection{Operational semantics} 

Finally, we introduce the computational dynamics. What marks these
algebras as distinct from other more traditionally studied algebraic
structures, e.g. vector spaces or polynomial rings, is the manner in
which dynamics is captured. In traditional structures, dynamics is typically
expressed through morphisms between such structures, as in linear maps
between vector spaces or morphisms between rings. In algebras
associated with the semantics of computation, the dynamics is
expressed as part of the algebraic structure itself, through a
reduction reduction relation typically denoted by $\red$. Below, we
give a recursive presentation of this relation for the calculus used
in the encoding.

$\red \subseteq \pi \times \pi$
$\red : \pi \to \mathcal{P}(\pi)$

\begin{mathpar}
  \inferrule* [lab=Comm] { \textsf{match}( x_{src}, x_{trgt} ) } { x_{trgt}?(y)P \; | \; x_{src}!\langle {Q} \rangle \red P\{\quotep{Q}/y}\} }
  \and \\
  \inferrule* [lab=Par] {{P} \red {P}'} {{{P} | {Q}} \red {{P}' | {Q}}}
  \and
  \inferrule* [lab=Equiv]{{{P} \scong {P}'} \andalso {{P}' \red {Q}'} \andalso {{Q}' \scong {Q}}}{{P} \red {Q}}
\end{mathpar}

\begin{eqnarray*}
  match_{\equiv} (\quotep{P},\quotep{Q}) & := & P \equiv Q \\
  match_{\dagger}(\quotep{P},\quotep{Q}) & := & \forall R. P|Q \red^{*} R => R \red^{*} 0 \\
  match_{K}(\quotep{P},\quotep{Q}) & := & K \mbox{ for some context } K
\end{eqnarray*}

$u?(x)P | u!\langle Q \rangle \red P\{\quotep{Q}/x\}$

%We write $\wred$ for $\red^*$, and $P\red$ if $\exists Q $ such that $ P \red Q$.
We write $P\red$ if $\exists Q $ such that $ P \red Q$ and $P\not\red$, otherwise.

\section{Replication}

As mentioned before, it is known that replication (and hence
recursion) can be implemented in a higher-order process algebra
\cite{SangiorgiWalker}. As our first example of calculation with the
machinery thus far presented we give the construction explicitly in
the {\rhoc}.

\begin{eqnarray}
	D_{x} & := & \prefix{x}{y}{(\binpar{\outputp{x}{y}}{@{y}})} \nonumber\\
	\bangp_{x}{P} & := & \binpar{{x}!\langle{\binpar{D_{x}}{P}}\rangle}{D_{x}} \nonumber
\end{eqnarray}

\begin{eqnarray}
	\bangp_{x}{P} & & \nonumber\\
	=
	& {x}!\langle{(\prefix{x}{y}{(\outputp{x}{y} | @{y})) | P}}\rangle 
	      | \prefix{x}{y}{(\outputp{x}{y} | @{y})} & \nonumber\\
	\red
	& (\outputp{x}{y} | @{y})\substn{\quotep{(\prefix{x}{y}{(@{y} | \outputp{x}{y})) | P}}}{y} & \nonumber\\
	=
	& \outputp{x}{\quotep{(\prefix{x}{y}{(\outputp{x}{y} | @{y})) | P}}}
	  | {(\prefix{x}{y}{(\outputp{x}{y} | @{y})) | P}} & \nonumber\\
	\red
	& \ldots & \nonumber\\
	\red^*
	& P | P | \ldots & \nonumber
\end{eqnarray}

Of course, this encoding, as an implementation, runs away, unfolding
$\bangp{P}$ eagerly. A lazier and more implementable replication
operator, restricted to input-guarded processes, may be obtained as follows.

\begin{eqnarray}
\bangp{\prefix{u}{v}{P}} 
	:= 
	\binpar{\lift{x}{\prefix{u}{v}{(\binpar{D(x)}{P})}}}{D(x)} \nonumber
\end{eqnarray}

\begin{remark}
  Note that the lazier definition still does not deal with summation
  or mixed summation (i.e. sums over input and output). The reader is
  invited to construct definitions of replication that deal with these
  features. 

  Further, the definitions are parameterized in a name, $x$. Can you,
  gentle reader, make a definition that eliminates this parameter and
  guarantees no accidental interaction between the replication
  machinery and the process being replicated -- i.e. no accidental
  sharing of names used by the process to get its work done and the
  name(s) used by the replication to effect copying. This latter
  revision of the definition of replication is crucial to obtaining
  the expected identity $!!P \sim !P$.
\end{remark}

\begin{remark}\label{rem:paradoxical_combinator}
  The reader familiar with the lambda calculus will have noticed the
  similarity between $D$ and the paradoxical combinator.

  [Ed. note: the existence of this seems to suggest we have to be more
  restrictive on the set of processes and names we admit if we are to
  support no-cloning.]
\end{remark}

\subsubsection{Bisimulation}

The computational dynamics gives rise to another kind of equivalence,
the equivalence of computational behavior. As previously mentioned
this is typically captured \emph{via} some form of bisimulation.

% The notion we use in this paper is weak barbed bisimulation
% \cite{milner91polyadicpi}.

The notion we use in this paper is derived from weak barbed
bisimulation \cite{milner91polyadicpi}. 

\begin{definition}
An \emph{observation relation}, $\downarrow_{\mathcal N}$, over a set
of names, $\mathcal N$, is the smallest relation satisfying the rules
below.

\infrule[Out-barb]{y \in {\mathcal N}, \; x \nameeq y}
		  {\outputp{x}{v} \downarrow_{\mathcal N} x}
\infrule[Par-barb]{\mbox{$P\downarrow_{\mathcal N} x$ or $Q\downarrow_{\mathcal N} x$}}
		  {\binpar{P}{Q} \downarrow_{\mathcal N} x}

We write $P \Downarrow_{\mathcal N} x$ if there is $Q$ such that 
$P \wred Q$ and $Q \downarrow_{\mathcal N} x$.
\end{definition}

\begin{definition}
%\label{def.bbisim}
An  ${\mathcal N}$-\emph{barbed bisimulation} over a set of names, ${\mathcal N}$, is a symmetric binary relation 
${\mathcal S}_{\mathcal N}$ between agents such that $P\rel{S}_{\mathcal N}Q$ implies:
\begin{enumerate}
\item If $P \red P'$ then $Q \wred Q'$ and $P'\rel{S}_{\mathcal N} Q'$.
\item If $P\downarrow_{\mathcal N} x$, then $Q\Downarrow_{\mathcal N} x$.
\end{enumerate}
$P$ is ${\mathcal N}$-barbed bisimilar to $Q$, written
$P \wbbisim_{\mathcal N} Q$, if $P \rel{S}_{\mathcal N} Q$ for some ${\mathcal N}$-barbed bisimulation ${\mathcal S}_{\mathcal N}$.
\end{definition}

$\mathcal{R} \subseteq \pi \times \pi$

$P \mathcal{R} Q => \forall P'. P \red P' \Rightarrow \exists Q'. Q \red Q', P' \mathcal{R} Q'$

$P \vdash x \Rightarrow Q \vdash x$

\begin{mathpar}
  \inferrule*[lab=Out-barb]{x \nameeq y}{{y}!\langle{Q}\rangle \vdash x}
  \and
  \inferrule*[lab=Par-barb]{\mbox{$P\vdash x$ or $Q\vdash x$}}{\binpar{P}{Q} \vdash x}
\end{mathpar}

\subsubsection{Contexts}

One of the principle advantages of computational calculi like the
$\pi$-calculus is a well-defined notion of context,
contextual-equivalence and a correlation between
contextual-equivalence and notions of bisimulation. The notion of
context allows the decomposition of a process into (sub-)process and
its syntactic environment, its context. Thus, a context may be
thought of as a process with a ``hole'' (written $\Box$) in it. The
application of a context $M$ to a process $P$, written $M[P]$, is
tantamount to filling the hole in $M$ with $P$. In this paper we do
not need the full weight of this theory, but do make use of the notion
of context in the proof the main theorem. 

\begin{mathpar}
  \inferrule* [lab=summation] {} {{M_{M},M_{N}} \bc \Box \;|\; x.M_{A} \;|\; M_{M}+M_{N}}
  \and
  \inferrule* [lab=agent] {} {{M_{A}} \bc (\vec{x})M_{P} \;| \; \clift{P_0,\ldots,M_{P},\ldots,P_N}}
  \and \\
  \inferrule* [lab=process] {} {{M_{P}} \bc M_{N} \;| \;P|M_{P} }
\end{mathpar} 

\begin{mathpar}
  \inferrule* [lab=sychronization] {} {M_{N} \bc \Box \;|\; x?M_{F} \;|\; x!M_{C}}
  \and
  \inferrule* [lab=abstraction] {} {{M_{F}} \bc (x)M_{P} }
  \and
  \inferrule* [lab=concretion] {} {{M_{C}} \bc \langle M_{P} \rangle }
  \and \\
  \inferrule* [lab=process] {} {{M_{P}} \bc M_{N} \;| \;P|M_{P} }
\end{mathpar}

\begin{definition}[contextual application] Given a context $M$, and
  process $P$, we define the \emph{contextual application}, $M[P] :=
  M\{P/\Box\}$. That is, the contextual application of M to P is the
  substitution of $P$ for $\Box$ in $M$.
\end{definition}

$\meaningof{-} : L \to \mathcal{P}(\pi)$

\begin{mathpar}
  \inferrule* [lab=collection] {} {\meaningof{true} = \pi, \and \meaningof{~E} = \pi \setminus \meaningof{E}, \and \meaningof{E_{1} \& E_{2}} = \meaningof{E_{1}} \cap \meaningof{E_{2}}}
\end{mathpar}

\begin{mathpar}
  \inferrule* [lab=structure] {} {\meaningof{0} = \{ P \in \pi | P \equiv 0 \}, \and \\ \meaningof{E_1 | E_2} = \{ P \in \pi | P \equiv P_{1} | P_{2}, P_{1} \in \meaningof{E_{1}}, P_{2} \in \meaningof{E_2}\} }
\end{mathpar}

\begin{mathpar}
 \inferrule* [lab=behavior] {} {\meaningof{\langle a?b \rangle E} = \{ P \in \pi | P \equiv Q | u?(y)P', \\ \and \\\\ \and \\ \;\;\; u \in \meaningof{a}, \forall z.P'\{z/y\} \in \meaningof{E\{z/b\}}\}, \and \\ \meaningof{a!E} = \{ P \in \pi | P \equiv Q | x!\langle P' \rangle, x \in \meaningof{a} P' \in \meaningof{E}\} }
\end{mathpar}

\begin{mathpar}
 \inferrule* [lab=nominal] {} {\meaningof{\quotep{E}} = \{ \quotep{P} \in \quotep{\pi} | P \in \meaningof{E} \}, \and \meaningof{\quotep{P}} = \{ \quotep{Q} \in \quotep{\pi} | P \equiv Q \} \and \\ \meaningof{@\quotep{E}} = \{ P \in \pi | P \equiv @x, x \in \meaningof{E} \}}
\end{mathpar}

\begin{eqnarray*}
  \\
  \meaningof{-} : TS \to ST
\end{eqnarray*}

\begin{eqnarray*}
  \\
  L : TS \to ST
\end{eqnarray*}

\begin{eqnarray*}
  \\
  P \models E \iff P \in \meaningof{E}
\end{eqnarray*}

\begin{eqnarray*}
  P \approx_{L} Q \iff \forall E \in L. P \models E \iff Q \models E
\end{eqnarray*}

\begin{eqnarray*}
  P \approx_{K} Q
\end{eqnarray*}

\begin{eqnarray*}
  P \approx Q
\end{eqnarray*}

$\approx_{K} = \approx = \approx_{L}$

\subsubsection{Contextual duality}

Note that contexts extend the quotation operation to a family of
operations from processes to names. Given a context, $M$, we can
define a \emph{nominal context}, $\quotep{M}$ by $\quotep{M}[P] :=
\quotep{M[P]}$. To foreshadow what is to come we observe that these
operations enjoy a duality with processes very much like the duality
between vectors and maps from vectors to scalars.

Further, because the calculus is essentially higher-order, we have a
correspondence between contexts and processes. More specifically,
given a name $x$ and a context $M$ we can construct $M^{*}_{x}$ such
that 

\begin{mathpar}
  M^{*}_{x} | \lift{x}{P} \red M[P]
\end{mathpar}

namely,

\begin{mathpar}
  M^{*}_{x} := x?(u).M[\dropn{u}]
\end{mathpar}

The dependence of $M^{*}_{x}$ on a name makes it an abstraction, 

\begin{mathpar}
  M^{*} := (x)x?(u).M[\dropn{u}]
\end{mathpar}

\subsection{Additional notation}

It will sometimes be convenient to denote the process a name
quotes. We already have the notation $x = \quotep{P}$, but it will be
convenient to introduce an alternate notation, $\procn{x}$, when we
want to emphasize the connection to the use of the name. Note that, by
virtue of name equivalence, $\quotep{\procn{x}} \nameeq x$; so, the
notation is consistent with previous definitions.

Further, because names have structure it is possible to effect
substitutions on the basis of that structure. This means we need to
upgrade our notation for substitutions, which we accomplish by
adapting comprehension notation. Thus,

\begin{mathpar}
  P\{ y / x : x \in S \}
\end{mathpar}

is interpreted to mean the process derived from P by replacing (in a
capture-avoiding manner) each occurrence of $x$ in $S$ by $y$. For example,

\begin{mathpar}
  P\{ \quotep{\procn{x}|\procn{x}} / x : x \in \freenames{P} \}
\end{mathpar}

will replace each (occurrence) of a free name $x$ in $P$ by
$\quotep{\procn{x}|\procn{x}}$.

Also, we will avail ourselves of the notation $x^{L}$ and $x^{R}$ to
denote injections of a name into disjoint copies of the name
space. There are numerous ways to accomplish this. One example can be
found in \cite{MeredithR05}. This notation overloads to vectors of
names: $\vec{x}^{\pi} := (x_{i}^{\pi} \; : \; 0 \leq i < |\vec{x}| )$ where $\pi \in \{L,R\}$.

We also use $P^{\Box} := P|\Box$.

In \cite{MeredithR05} an interpretation of the new operator is
given. It turns out that there are several possible interpretations
all enjoying the requisite algebraic properties of the operator (see
\cite{milner91polyadicpi}). We will therefore make liberal use of
$(\nu\; \vec{x})P$.

% subsection the_syntax_and_semantics_of_the_notation_system (end)   

\section{Interpretation of QM}
\subsection{Supporting definitions}
\subsubsection{Multiplication}
\begin{mathpar}
  \quotep{Q} \cdot \quotep{R} := \quotep{Q|R}
  \and \\
  \quotep{Q} \cdot P := P\{ \quotep{Q|R} / \quotep{R} : \quotep{R} \in \freenames{P} \}
\end{mathpar}

\paragraph{Discussion}
The first line needs little explanation. The second line says that
each free name of the process is replaced with the multiplication of
that name by the scalar. Multiplication of a scalar (name) by a state
(process) results in a process all the names of which have been `moved
over' by parallel composition with the process the scalar
quotes. There is a subtlety that the bound names have to be
manipulated so that multiplied names aren't accidentally
captured. There are many ways to achieve this.

\begin{remark}\label{rem:multiplication_identities}
  The reader is invited to verify that for all $x,y,z \in \QProc$ and $P \in \Proc$
  \begin{mathpar}
    x \cdot \quotep{0} \equiv x 
    \and
    x \cdot y \equiv y \cdot x
    \and
    x \cdot (y \cdot z) \equiv (x \cdot y) \cdot z
    \and \\
    \quotep{0} \cdot P \equiv P
    \and \\
    x \cdot (y \cdot P) \equiv (x \cdot y) \cdot P
    \and \\
    x \cdot (P|Q) \equiv (x \cdot P) | (x \cdot Q)
    \and \\    
  \end{mathpar}
\end{remark}

\subsubsection{Tensor product}

We define a tensor product on processes by structural induction.

\paragraph{Tensor of sums} First note that all summations, including
$\pzero$ and sequence, can be written $\Sigma_{i} x_{i}.A_{i} +
\Sigma_{j} x_{j}.C_{j}$, where we have grouped input-guarded processes
together and output-guarded processes together.

Thus, we can define the tensor product of two summations, $N_{1}\otimes N_{2}$, where

\begin{mathpar}
  N_{1} := \Sigma_{i} x_{i}.A_{i} + \Sigma_{j} x_{j}.C_{j}
  \and
  N_{2} := \Sigma_{i'} y_{i'}.B_{i'} + \Sigma_{j'} y_{j'}.D_{j'} 
\end{mathpar}

as follows.

\begin{mathpar}
  \Sigma_{i} x_{i}.A_{i} + \Sigma_{j} x_{j}.C_{j} \otimes \Sigma_{i'}
  y_{i'}.B_{i'} + \Sigma_{j'} y_{j'}.D_{j'} 
  \and \\
  := \; \Sigma_{i} \Sigma_{i'} \quotep{\stackrel{\vee}{x_{i}}| \stackrel{\vee}{y_{i'}}}.(A_{i}\otimes B_{i'}) \; | \; \Sigma_{i'} \Sigma_{i} \quotep{\stackrel{\vee}{y_{i'}}|\stackrel{\vee}{x_{i}}}.(B_{i'}\otimes A_{i})
  \and
  \;\; | \;\; \Sigma_{j} \Sigma_{j'} \quotep{\stackrel{\vee}{x_{j}}|\stackrel{\vee}{y_{j'}}}.(A_{j}\otimes B_{j'}) \; | \; \Sigma_{j'} \Sigma_{j} \quotep{\stackrel{\vee}{y_{j'}}|\stackrel{\vee}{x_{j}}}.(B_{j'}\otimes A_{j})
\end{mathpar}

\begin{remark}
  Do we need to $x^{L}$ and $y^{R}$ for this construction as well?
\end{remark}

\paragraph{Tensor of parallel compositions} Next, we distribute tensor
over par.

\begin{mathpar}
  P_{1}|P_{2} \otimes Q_{1}|Q_{2} := (P_{1} \otimes Q_{1}) | (P_{1}
  \otimes Q_{2}) | (P_{2} \otimes Q_{1}) | (P_{2} \otimes Q_{2})
\end{mathpar}

\paragraph{Tensor with dropped names} We treat tensor of a
process with a dropped name as parallel composition.

\begin{mathpar}
  P \otimes \dropn{x} := P | \dropn{x}
\end{mathpar}

\paragraph{Tensor of agents}

Finally, we need to define tensor on agents. Note that the definition
of tensor on normal products only tensors inputs with inputs and
outputs with outputs. Thus, we only have to define the operation on
``homogeneous'' pairings.

\begin{mathpar}
  (\vec{x})P \otimes (\vec{y})Q
  \and \\
  := (x_{0}^{L}|y_{0}^{R},\ldots,x_{0}^{L}|y_{n}^{R},\ldots,x_{m}^{L}|y_{0}^{R},\ldots,x_{m}^{L}|y_{n}^R)(P\{ \vec{x}^{L}/\vec{x}\} \otimes Q \{ \vec{y}^{R}/\vec{y}\})
  \and \\
  \clift{\vec{P}} \otimes \clift{\vec{Q}}
  \and \\
  := \clift{P_{0}\otimes Q_{0},\ldots,P_{0}\otimes Q_{n},\ldots,P_{m}\otimes Q_{0},\ldots,P_{m}\otimes Q_{n}}
\end{mathpar}

\begin{remark}
  Observe that arities of tensored abstractions matches arities of
  tensored concretions if the original arities matched. Note also that
  the length of the arities corresponds to the increase in dimension
  we see in ordinary vector space tensor product.
\end{remark}

\begin{remark}
  Operationally, this definition distributes the tensor down to
  components ``linked'' by summation. Tensor over summation is
  intriguing in that it mixes names. Moreover, as a consequence of the
  way it mixes names we have the identities for all $x \in \QProc$ and
  $P,Q \in \Proc$

  \begin{mathpar}
    (x \cdot P) \otimes Q \equiv x \cdot (P \otimes Q) \equiv P \otimes (x \cdot Q)
    \and
    P \otimes \pzero \equiv P
  \end{mathpar}

  that the reader is invited to verify.
\end{remark}

\subsubsection{Annihilation}
\begin{mathpar}
  P^{\perp} := \{ Q | \forall R. P|Q \red^{*} R \Rightarrow R \red^{*} \pzero \}
  \and \\
  P^{\underline{\perp}} := \Sigma_{Q \in P^{\perp}} \quotep{Q}?(y).(\dropn{y}|Q) | \Sigma_{Q \in P^{\perp}} \quotep{Q}\clift{\Box}
\end{mathpar}

\paragraph{Discussion} The reader will note that $P^{\perp}$ is a
\emph{set} of processes, while $P^{\underline{\perp}}$ is a
\emph{context}. We call the set $P^{\perp}$ the \emph{annihilators} of
$P$. The parallel composition of a process in the annihilators of $P$
with $P$ will result in a process, the state space of which has all
paths eventually leading to $\pzero$. Execution may endure loops; but
under reasonable conditions of fairness (naturally guaranteed under
most notions of bisimulation) such a composite process cannot get
stuck in such a loop and will, eventually pop out and terminate.

The context $P^{\underline{\perp}}$ is ready and willing to ``take the
$P$ out of'' the process to which it is applied. It will effectively
transmit the code of the process to which it is applied to one of the
annihilators and run the process against it.

\subsubsection{Evaluation}
We fix $M$ a domain of fully abstract interpretation with an equality
coincident with bisimulation. We take $\meaningof{\cdot} : \Proc \to
M$ to be the map interpreting processes and $\nmeaningof{\cdot} : \M
\to Proc$ to be the map running the other way. Then we define

\begin{mathpar}
  \int P := \nmeaningof{\meaningof{P}}
\end{mathpar}

\paragraph{Discussion}
There are many fully abstract interpretations of Milner's
$\pi$-calculus. Any of them can be used as a basis for interpreting
the reflective calculus here. Equipped with such a domain it is
largely a matter of grinding through to check that the Yoneda
construction for the normalization-by-evaluation program can be
extended to this setting.

\begin{remark}
  The reader is invited to verify that $\int (P^{\underline{\perp}}[P]) = 0$.
\end{remark}

\subsection{Quantum mechanics}

Table \ref{tbl:core_qm_op_defns} gives the core operational definitions

\begin{table}[htp]\label{tbl:core_qm_op_defns}
  \center{
    \fbox{
      \begin{tabular}{c|c}
        quantum mechanics & process calculus \\
        \hline
        scalar & $x := \quotep{P}$ \\
        state vector & $\state{P} := P$ \\
        dual & $\state{P}^{*} := \event{P^{\underline{\perp}}} := \quotep{P^{\underline{\perp}}}[-]$ \\
        matrix & $ \Sigma_{\alpha} \state{P_{\alpha}}x_{\alpha}\event{Q_{\alpha}}$ \\
        vector addition & $\state{P} + \state{Q} := \state{P | Q}$ \\
        tensor product & $\state{P} \otimes \state{Q} := \state{P \otimes Q}$ \\
        inner product & $\innerprod{P}{Q} := \quotep{\int P^{\underline{\perp}}[Q]}$ \\
      \end{tabular}
    }
  }
  \caption{QM - operational definitions}
\end{table}

where

\begin{mathpar}
  \prmatrix{P}{Q} := \fprmatrix{P}{\quotep{\pzero}}{Q}
  \and
  \fprmatrix{P}{x}{Q} := (\state{P},x,\event{Q})
  \and
  (\fprmatrix{P}{x}{Q})(\state{R}) := x \cdot \innerprod{Q}{R} \cdot \state{P}
  \and
  (\fprmatrix{P}{x}{Q})(\event{R}) := x \cdot \innerprod{R}{P} \cdot \event{Q}
\end{mathpar}

\paragraph{Discussion}
As promised: vectors (aka states) are represented as processes; duals
as contextual duals; inner product definition should be compared with
standard inner product definition for ....

\begin{remark}
  Assuming $\int (P^{\underline{\perp}}[P]) = 0$, the reader is
  invited to verify that $(\fprmatrix{P}{x}{P})(\state{P}) = x \cdot \state{P}$.
\end{remark}

\begin{remark}
  The reader is invited to verify that $\innerprod{P}{Q}$ could
  equally well have been written $\quotep{\int \stackrel{\vee}{x}}$
  where $x = \event{P^{\underline{\perp}}}(Q)$.

  One of the motivations for this remark is that there is another way
  to factor these operations. We could package up evaluation in the dual:

  \begin{mathpar}
    \state{P}^{*} := \event{\int P^{\underline{\perp}}} := \quotep{\int P^{\underline{\perp}}}[-]
  \end{mathpar}

  and then have inner product defined by
  
  \begin{mathpar}
    \innerprod{P}{Q} := \event{P}(Q)
  \end{mathpar}

  Hopefully, experience with the calculations will provide guidance on
  the best factoring.
\end{remark}

\begin{remark}
  Assuming $\int (P^{\underline{\perp}}[P]) = 0$, the reader is
  invited to verify that $\forall P,Q. (\prmatrix{0}{Q})(\state{0}) =
  \state{0}$ and dually $(\prmatrix{P}{0})(\event{0}) = \event{0}$.
\end{remark}

\begin{remark}
  i'm a little worried that i don't (yet) have proper support for
  complex conjugacy. But, the observation above may give us a
  clue. According to Abramsky, it must be the case that the scalars
  are iso to the homset of the identity for the tensor -- which the
  observation above characterizes. 

  For now, we will simply bookmark the notion with $\overline{x}$.
\end{remark}

\subsubsection{Adjointness}

We need to give a definition of $(\cdot)^{\dagger}$ for matrices. The
obvious candidate definition is
\begin{mathpar}
(\Sigma_{\alpha}\fprmatrix{P_{\alpha}}{x_{\alpha}}{Q_{\alpha}})^{\dagger}
= \Sigma_{\alpha}\fprmatrix{(Q_{\alpha}^{\underline{\perp}})^{*}}{\overline{x}_{\alpha}}{P_{\alpha}^{\underline{\perp}}} 
\end{mathpar}

But, $(Q_{\alpha}^{\underline{\perp}})^{*}$ requires a name along
which to communicate the process to achieve the context application.

\subsubsection{Basis for a basis}
If processes label states and ``addition'' of states (a.k.a. vector
addition) is interpreted as parallel composition, what corresponds to
notions of linear independence and basis? Here, we recall that Yoshida
has developed a set of \emph{combinators} for an asynchronous verison
of Milner's $\pi$-calculus. These are a finite set of processes such
any process can be expressed as parallel composition of these
combinators together with liberal uses of the new operator and
replication. We can simply give a translation of these into the
present calculus and have reasonable expectation that the property
carries over. That is, that the resultant set allows to express all
processes via parallel composition. Note, however, that there is no
new operator or replication in this calculus. As a result, we expect
that the corresponding set is actually infinite. That is, we expect
that the space is actually infinite dimensional.

\begin{remark}
  The attentive reader may be a bit concerned. Certainly, the
  collection $S$, $K$ and $I$ is a finite set of
  combinators. Shouldn't we expect to see a finite set of combinators
  for an effectively equivalent system? i am very sympathetic to this
  critique and feel it warrants full attention. On the other hand, i
  also have in mind the following analogy. The natural numbers, as a
  monoid under addition, has exactly $1$ generator, while the natural
  numbers, as a monoid under multiplication, has countably many
  generators (the primes). We observe that the application of the
  lambda calculus is much less resource sensitive than the parallel
  composition of the $\pi$-calculus. Could it be the case that we have
  an analogy of the form
  
  \begin{mathpar}
    m + n : MN :: m*n : M|N
  \end{mathpar}

  giving a similar blow up in the set of ``primes''?  This is such a
  wonderful thought that, even if it's not true, i think it's worth
  writing down.
\end{remark}
 

\documentclass[12pt]{llncs}
%\documentclass{jktr}

\usepackage[pdftex]{hyperref}                   
\usepackage {listings}
\usepackage {mathpartir}
\usepackage{bcprules}
%\usepackage{listings}
                       
\usepackage{graphicx} 
%\usepackage[margins=2.5cm,nohead,nofoot]{geometry}
%\usepackage{geometry}
\usepackage{amsfonts}
\usepackage{amstext}
\usepackage{latexsym}
\usepackage{amssymb}
\usepackage{color}


%\include{myPreamble}
\include{qm2pi.local} 

%\ifpdf
%\usepackage[pdftex]{graphicx}
%\else
%\usepackage{graphicx}
%\fi

 % \ifpdf
%  \usepackage{pdfsync}
%  \if


%\title{Brief Article}
%\author{David F. Snyder}
%\author{L.G. Meredith}

%\address{Dept. of Math., Texas State University--San Marcos, San Marcos, TX 78666}
       
\pagestyle{empty}


\begin{document}

\lstset{language=[Objective]Caml,frame=shadowbox}

\input{qm2pi.front}

% section front matter (end)

\input{qm2pi.intro} 
 
% section introduction (end)

% \input{qm2pi.knotations} 

% section notation (end)

\input{qm2pi.process.calculi} 

% section concurrent_process_calculi_and_spatial_logics_ (end)
    
%\input{qm2pi.knots2pi} 

%\input{qm2pi.trefoil} 

%\input{qm2pi.mainthm} 

% subsection basic_interpretation (end)

%\input{qm2pi.rho.presentation} 
\subsection{The syntax and semantics of the notation system}\label{sub:the_syntax_and_semantics_of_the_notation_system} % (fold)

We now summarize a technical presentation of the calculus that
embodies our theory of dynamics. The typical presentation of such a
calculus follows the style of giving generators and relations on
them. The grammar, below, describing term constructors, freely
generates the set of processes, $\Proc$. This set is then quotiented
by a relation known as structural congruence and it is over this set
that the notion of dynamics is expressed. This presentation is
essentially that of \cite{MeredithR05} with the addition of
polyadicity and summation. For readability we have relegated some of
the technical subtleties to an appendix.

\subsubsection{Process grammar}\label{subsub:process_grammar}

\begin{mathpar}
  \inferrule* [lab=synchronization] {} {{M} \bc \pzero \;|\; x?F \;|\; x!C }
  \and
  \inferrule* [lab=abstraction] {} {{F} \bc (x)P}
  \and
  \inferrule* [lab=concretion] {} {{C} \bc \langle Q \rangle}
  \and
  \inferrule* [lab=process] {} {{P,Q} \bc M \;| \;P|Q \;|\; @{x}}
  \and
  \inferrule* [lab=name] {} {{x} \bc \quotep{P}}
\end{mathpar} 

Note that $\vec{x}$ (resp. $\vec{P}$) denotes a vector of names
(resp. processes) of length $|\vec{x}|$ (resp. $|\vec{P}|$). We adopt
the following useful abbreviations.

\begin{mathpar}
   x?(\vec{y}).P := x.(\vec{y})P \and  x\clift{\vec{P}} := x.\clift{\vec{P}}
   \and x!(y) := \lift{x}{\dropn{y}}
   \and \Pi_{i=0}^{n-1}P_i := P_0 | \ldots | P_{n-1}
\end{mathpar}

\subsubsection{Structural congruence}

\paragraph{Free and bound names and alpha-equivalence.} At the
core of structural equivalence is alpha-equivalence which identifies
process that are the same up to a change of variable. Formally, we
recognize the distinction between free and bound names. The free names
of a process, $\freenames{P}$, may be calculated recursively as
follows:

\begin{mathpar}
\freenames{\pzero} := \emptyset
  \and \\
  \freenames{x?(y).P} := \{ x \} \cup (\freenames{P} \setminus \{ y \})
  \and 
  \freenames{x!\langle P \rangle} := \{ x \} \cup \{ P \} 
  \and \\
  \freenames{P|Q} := \freenames{P} \cup \freenames{Q}
  \and \\
  \freenames{@{x}} := \{ x \}
\end{mathpar}

$\pi$
$\quotep{\pi}$

$\freenames{-} : \pi \to \mathcal{P}(\quotep{\pi})$

\begin{eqnarray*}
  \freenames{\pzero} & := & \emptyset \\
  \freenames{x?(y).P} & := & \{ x \} \cup (\freenames{P} \setminus \{ y \}) \\
  \freenames{x!\langle P \rangle} & := & \{ x \} \cup \{ P \} \\
  \freenames{P|Q} & := & \freenames{P} \cup \freenames{Q} \\
  \freenames{\dropn{x}} & := & \{ x \}
\end{eqnarray*}

The bound names of a process, $\boundnames{P}$, are those names occurring in $P$
that are not free. For example, in $x?(y).0$, the name $x$ is free, while $y$ is bound.

\begin{mathpar}
  \inferrule* [lab=monoidal-laws] {} { P|Q \equiv Q|P \and P|0 \equiv P \and P|(Q|R) \equiv (P|Q)|R }
\end{mathpar}

\begin{mathpar}
  \inferrule* [lab=alpha-equivalence] {} { (x)P \equiv (y)P\{y/x\} \and y \not\in \freenames{P} }
\end{mathpar}

\begin{definition}
Then two processes, $P,Q$, are alpha-equivalent if $P = Q\{\vec{y}/\vec{x}\}$ for
some $\vec{x} \in \boundnames{Q},\vec{y} \in \boundnames{P}$, where $Q\{\vec{y}/\vec{x}\}$
denotes the capture-avoiding substitution of $\vec{y}$ for $\vec{x}$ in $Q$.
\end{definition}

\begin{definition}
  The {\em structural congruence} \cite{SangiorgiWalker} , $\equiv$,
  between processes is the least congruence containing
  alpha-equivalence, satisfying the abelian monoid laws
  (associativity, commutativity and $\pzero$ as identity) for parallel
  composition $|$ and for summation $+$.
\end{definition}

\subsection{Name equivalence}

We take name equivalence, written $\nameeq$, to be the smallest
equivalence relation generated by the following rules.

\begin{mathpar}
\inferrule*[lab=Quote-drop]
{ }
{ \quotep{@{x}} \nameeq x }

\inferrule*[lab=Struct-equiv]
{ P \scong Q }
{ \quotep{P} \nameeq \quotep{Q} }
\end{mathpar}

The astute reader will have noticed that the mutual recursion of names
and processes imposes a mutual recursion on alpha-equivalence and
structural equivalence via name-equivalence. Fortunately, all of this
works out pleasantly and we may calculate in the natural way, free of
concern. The reader interested in the details is referred to the
appendix \ref{appendix:rho_details}.

\subsection{Substitution}

We use $\Proc$ for the set of processes, $\QProc$ for the set of
names, and $\id{\{}\vec{y} / \vec{x} \id{\}}$ to denote partial maps,
$s : \QProc \rightarrow \QProc$. A map, $s$ lifts, uniquely, to a map
on process terms, $\widehat{s} : \Proc \rightarrow \Proc$ by the
following equations.

\begin{mathpar}
  (0) \psubstp{Q}{P} := 0 \\
  (R \juxtap S) \psubstp{Q}{P}
  :=    
  (R)\psubstp{Q}{P} \juxtap (S) \psubstp{Q}{P} \\
  (x?(y).R) \psubstp{Q}{P}    
  :=    
  (x)\substp{Q}{P} (z)\concat( (R \psubstn{z}{y}) \psubstp{Q}{P} ) \\
  (\lift{x}{R}) \psubstp{Q}{P}  
  :=
  \lift{(x)\substp{Q}{P}}{ R \psubstp{Q}{P} } \\
%   (\dropn{x})  \psubstp{Q}{P}       
%   := 
%   \left\{ 
%     \begin{array}{ccc} 
%       \dropn{\quotep{Q}} & & x \nameeq \quotep{P} \\
%       \dropn{x} & & otherwise \\
%     \end{array}
%   \right. 
  (\dropn{x})  \psubstp{Q}{P}       
  := 
  \left\{ 
    \begin{array}{ccc} 
      Q & & x \nameeq \quotep{P} \\
      \dropn{x} & & otherwise \\
    \end{array}
  \right.
\end{mathpar}
 

where

\begin{eqnarray}
  (x)\id{\{} \lpquote Q \rpquote / \lpquote P \rpquote \id{\}}            = 
  \left\{ 
    \begin{array}{ccc}
      \lpquote Q \rpquote & & x \nameeq \lpquote P \rpquote \\
      x & & otherwise \\
    \end{array}
  \right. \nonumber
\end{eqnarray}

and $z$ is chosen distinct from $\quotep{P}$, $\quotep{Q}$, the free
names in $Q$, and all the names in $R$. Our $\alpha$-equivalence will
be built in the standard way from this substitution.

\begin{remark}\label{rem:no_self_referential_names}
  One consequence of these definitions is that $\forall P. \quotep{P}
  \not\in \freenames{P}$.
\end{remark}

\subsection{ Dynamic quote: an example }

Anticipating something of what's to come, consider applying the
substitution, $\widehat{\id{\{}u / z \id{\}}}$, to the following pair
of processes, $\lift{w}{y!(z)}$ and $w[ \lpquote y!(z) \rpquote ]$.

\begin{eqnarray}
	\lift{w}{y!(z)}\widehat{\id{\{}u / z \id{\}}}
		& = &
		\lift{w}{y!(u)} \nonumber\\
	w[ \lpquote y!(z) \rpquote ] \widehat{ \id{\{}u / z \id{\}} }
		& = &
		w[ \lpquote y!(z) \rpquote ] \nonumber
\end{eqnarray}

Because the body of the process between quotes is impervious to
substitution, we get radically different answers. In fact, by
examining the first process in an input context,
e.g. $x?(z).\lift{w}{y!(z)}$, we see that the process under the lift
operator may be shaped by prefixed inputs binding a name inside it. In
this sense, the lift operator will be seen as a way to dynamically
construct processes before reifying them as names.

Finally equipped with these standard features we can present the
dynamics of the calculus.

\subsubsection{Operational semantics} 

Finally, we introduce the computational dynamics. What marks these
algebras as distinct from other more traditionally studied algebraic
structures, e.g. vector spaces or polynomial rings, is the manner in
which dynamics is captured. In traditional structures, dynamics is typically
expressed through morphisms between such structures, as in linear maps
between vector spaces or morphisms between rings. In algebras
associated with the semantics of computation, the dynamics is
expressed as part of the algebraic structure itself, through a
reduction reduction relation typically denoted by $\red$. Below, we
give a recursive presentation of this relation for the calculus used
in the encoding.

$\red \subseteq \pi \times \pi$
$\red : \pi \to \mathcal{P}(\pi)$

\begin{mathpar}
  \inferrule* [lab=Comm] { \textsf{match}( x_{src}, x_{trgt} ) } { x_{trgt}?(y)P \; | \; x_{src}!\langle {Q} \rangle \red P\{\quotep{Q}/y}\} }
  \and \\
  \inferrule* [lab=Par] {{P} \red {P}'} {{{P} | {Q}} \red {{P}' | {Q}}}
  \and
  \inferrule* [lab=Equiv]{{{P} \scong {P}'} \andalso {{P}' \red {Q}'} \andalso {{Q}' \scong {Q}}}{{P} \red {Q}}
\end{mathpar}

\begin{eqnarray*}
  match_{\equiv} (\quotep{P},\quotep{Q}) & := & P \equiv Q \\
  match_{\dagger}(\quotep{P},\quotep{Q}) & := & \forall R. P|Q \red^{*} R => R \red^{*} 0 \\
  match_{K}(\quotep{P},\quotep{Q}) & := & K \mbox{ for some context } K
\end{eqnarray*}

$u?(x)P | u!\langle Q \rangle \red P\{\quotep{Q}/x\}$

%We write $\wred$ for $\red^*$, and $P\red$ if $\exists Q $ such that $ P \red Q$.
We write $P\red$ if $\exists Q $ such that $ P \red Q$ and $P\not\red$, otherwise.

\section{Replication}

As mentioned before, it is known that replication (and hence
recursion) can be implemented in a higher-order process algebra
\cite{SangiorgiWalker}. As our first example of calculation with the
machinery thus far presented we give the construction explicitly in
the {\rhoc}.

\begin{eqnarray}
	D_{x} & := & \prefix{x}{y}{(\binpar{\outputp{x}{y}}{@{y}})} \nonumber\\
	\bangp_{x}{P} & := & \binpar{{x}!\langle{\binpar{D_{x}}{P}}\rangle}{D_{x}} \nonumber
\end{eqnarray}

\begin{eqnarray}
	\bangp_{x}{P} & & \nonumber\\
	=
	& {x}!\langle{(\prefix{x}{y}{(\outputp{x}{y} | @{y})) | P}}\rangle 
	      | \prefix{x}{y}{(\outputp{x}{y} | @{y})} & \nonumber\\
	\red
	& (\outputp{x}{y} | @{y})\substn{\quotep{(\prefix{x}{y}{(@{y} | \outputp{x}{y})) | P}}}{y} & \nonumber\\
	=
	& \outputp{x}{\quotep{(\prefix{x}{y}{(\outputp{x}{y} | @{y})) | P}}}
	  | {(\prefix{x}{y}{(\outputp{x}{y} | @{y})) | P}} & \nonumber\\
	\red
	& \ldots & \nonumber\\
	\red^*
	& P | P | \ldots & \nonumber
\end{eqnarray}

Of course, this encoding, as an implementation, runs away, unfolding
$\bangp{P}$ eagerly. A lazier and more implementable replication
operator, restricted to input-guarded processes, may be obtained as follows.

\begin{eqnarray}
\bangp{\prefix{u}{v}{P}} 
	:= 
	\binpar{\lift{x}{\prefix{u}{v}{(\binpar{D(x)}{P})}}}{D(x)} \nonumber
\end{eqnarray}

\begin{remark}
  Note that the lazier definition still does not deal with summation
  or mixed summation (i.e. sums over input and output). The reader is
  invited to construct definitions of replication that deal with these
  features. 

  Further, the definitions are parameterized in a name, $x$. Can you,
  gentle reader, make a definition that eliminates this parameter and
  guarantees no accidental interaction between the replication
  machinery and the process being replicated -- i.e. no accidental
  sharing of names used by the process to get its work done and the
  name(s) used by the replication to effect copying. This latter
  revision of the definition of replication is crucial to obtaining
  the expected identity $!!P \sim !P$.
\end{remark}

\begin{remark}\label{rem:paradoxical_combinator}
  The reader familiar with the lambda calculus will have noticed the
  similarity between $D$ and the paradoxical combinator.

  [Ed. note: the existence of this seems to suggest we have to be more
  restrictive on the set of processes and names we admit if we are to
  support no-cloning.]
\end{remark}

\subsubsection{Bisimulation}

The computational dynamics gives rise to another kind of equivalence,
the equivalence of computational behavior. As previously mentioned
this is typically captured \emph{via} some form of bisimulation.

% The notion we use in this paper is weak barbed bisimulation
% \cite{milner91polyadicpi}.

The notion we use in this paper is derived from weak barbed
bisimulation \cite{milner91polyadicpi}. 

\begin{definition}
An \emph{observation relation}, $\downarrow_{\mathcal N}$, over a set
of names, $\mathcal N$, is the smallest relation satisfying the rules
below.

\infrule[Out-barb]{y \in {\mathcal N}, \; x \nameeq y}
		  {\outputp{x}{v} \downarrow_{\mathcal N} x}
\infrule[Par-barb]{\mbox{$P\downarrow_{\mathcal N} x$ or $Q\downarrow_{\mathcal N} x$}}
		  {\binpar{P}{Q} \downarrow_{\mathcal N} x}

We write $P \Downarrow_{\mathcal N} x$ if there is $Q$ such that 
$P \wred Q$ and $Q \downarrow_{\mathcal N} x$.
\end{definition}

\begin{definition}
%\label{def.bbisim}
An  ${\mathcal N}$-\emph{barbed bisimulation} over a set of names, ${\mathcal N}$, is a symmetric binary relation 
${\mathcal S}_{\mathcal N}$ between agents such that $P\rel{S}_{\mathcal N}Q$ implies:
\begin{enumerate}
\item If $P \red P'$ then $Q \wred Q'$ and $P'\rel{S}_{\mathcal N} Q'$.
\item If $P\downarrow_{\mathcal N} x$, then $Q\Downarrow_{\mathcal N} x$.
\end{enumerate}
$P$ is ${\mathcal N}$-barbed bisimilar to $Q$, written
$P \wbbisim_{\mathcal N} Q$, if $P \rel{S}_{\mathcal N} Q$ for some ${\mathcal N}$-barbed bisimulation ${\mathcal S}_{\mathcal N}$.
\end{definition}

$\mathcal{R} \subseteq \pi \times \pi$

$P \mathcal{R} Q => \forall P'. P \red P' \Rightarrow \exists Q'. Q \red Q', P' \mathcal{R} Q'$

$P \vdash x \Rightarrow Q \vdash x$

\begin{mathpar}
  \inferrule*[lab=Out-barb]{x \nameeq y}{{y}!\langle{Q}\rangle \vdash x}
  \and
  \inferrule*[lab=Par-barb]{\mbox{$P\vdash x$ or $Q\vdash x$}}{\binpar{P}{Q} \vdash x}
\end{mathpar}

\subsubsection{Contexts}

One of the principle advantages of computational calculi like the
$\pi$-calculus is a well-defined notion of context,
contextual-equivalence and a correlation between
contextual-equivalence and notions of bisimulation. The notion of
context allows the decomposition of a process into (sub-)process and
its syntactic environment, its context. Thus, a context may be
thought of as a process with a ``hole'' (written $\Box$) in it. The
application of a context $M$ to a process $P$, written $M[P]$, is
tantamount to filling the hole in $M$ with $P$. In this paper we do
not need the full weight of this theory, but do make use of the notion
of context in the proof the main theorem. 

\begin{mathpar}
  \inferrule* [lab=summation] {} {{M_{M},M_{N}} \bc \Box \;|\; x.M_{A} \;|\; M_{M}+M_{N}}
  \and
  \inferrule* [lab=agent] {} {{M_{A}} \bc (\vec{x})M_{P} \;| \; \clift{P_0,\ldots,M_{P},\ldots,P_N}}
  \and \\
  \inferrule* [lab=process] {} {{M_{P}} \bc M_{N} \;| \;P|M_{P} }
\end{mathpar} 

\begin{mathpar}
  \inferrule* [lab=sychronization] {} {M_{N} \bc \Box \;|\; x?M_{F} \;|\; x!M_{C}}
  \and
  \inferrule* [lab=abstraction] {} {{M_{F}} \bc (x)M_{P} }
  \and
  \inferrule* [lab=concretion] {} {{M_{C}} \bc \langle M_{P} \rangle }
  \and \\
  \inferrule* [lab=process] {} {{M_{P}} \bc M_{N} \;| \;P|M_{P} }
\end{mathpar}

\begin{definition}[contextual application] Given a context $M$, and
  process $P$, we define the \emph{contextual application}, $M[P] :=
  M\{P/\Box\}$. That is, the contextual application of M to P is the
  substitution of $P$ for $\Box$ in $M$.
\end{definition}

$\meaningof{-} : L \to \mathcal{P}(\pi)$

\begin{mathpar}
  \inferrule* [lab=collection] {} {\meaningof{true} = \pi, \and \meaningof{~E} = \pi \setminus \meaningof{E}, \and \meaningof{E_{1} \& E_{2}} = \meaningof{E_{1}} \cap \meaningof{E_{2}}}
\end{mathpar}

\begin{mathpar}
  \inferrule* [lab=structure] {} {\meaningof{0} = \{ P \in \pi | P \equiv 0 \}, \and \\ \meaningof{E_1 | E_2} = \{ P \in \pi | P \equiv P_{1} | P_{2}, P_{1} \in \meaningof{E_{1}}, P_{2} \in \meaningof{E_2}\} }
\end{mathpar}

\begin{mathpar}
 \inferrule* [lab=behavior] {} {\meaningof{\langle a?b \rangle E} = \{ P \in \pi | P \equiv Q | u?(y)P', \\ \and \\\\ \and \\ \;\;\; u \in \meaningof{a}, \forall z.P'\{z/y\} \in \meaningof{E\{z/b\}}\}, \and \\ \meaningof{a!E} = \{ P \in \pi | P \equiv Q | x!\langle P' \rangle, x \in \meaningof{a} P' \in \meaningof{E}\} }
\end{mathpar}

\begin{mathpar}
 \inferrule* [lab=nominal] {} {\meaningof{\quotep{E}} = \{ \quotep{P} \in \quotep{\pi} | P \in \meaningof{E} \}, \and \meaningof{\quotep{P}} = \{ \quotep{Q} \in \quotep{\pi} | P \equiv Q \} \and \\ \meaningof{@\quotep{E}} = \{ P \in \pi | P \equiv @x, x \in \meaningof{E} \}}
\end{mathpar}

\begin{eqnarray*}
  \\
  \meaningof{-} : TS \to ST
\end{eqnarray*}

\begin{eqnarray*}
  \\
  L : TS \to ST
\end{eqnarray*}

\begin{eqnarray*}
  \\
  P \models E \iff P \in \meaningof{E}
\end{eqnarray*}

\begin{eqnarray*}
  P \approx_{L} Q \iff \forall E \in L. P \models E \iff Q \models E
\end{eqnarray*}

\begin{eqnarray*}
  P \approx_{K} Q
\end{eqnarray*}

\begin{eqnarray*}
  P \approx Q
\end{eqnarray*}

$\approx_{K} = \approx = \approx_{L}$

\subsubsection{Contextual duality}

Note that contexts extend the quotation operation to a family of
operations from processes to names. Given a context, $M$, we can
define a \emph{nominal context}, $\quotep{M}$ by $\quotep{M}[P] :=
\quotep{M[P]}$. To foreshadow what is to come we observe that these
operations enjoy a duality with processes very much like the duality
between vectors and maps from vectors to scalars.

Further, because the calculus is essentially higher-order, we have a
correspondence between contexts and processes. More specifically,
given a name $x$ and a context $M$ we can construct $M^{*}_{x}$ such
that 

\begin{mathpar}
  M^{*}_{x} | \lift{x}{P} \red M[P]
\end{mathpar}

namely,

\begin{mathpar}
  M^{*}_{x} := x?(u).M[\dropn{u}]
\end{mathpar}

The dependence of $M^{*}_{x}$ on a name makes it an abstraction, 

\begin{mathpar}
  M^{*} := (x)x?(u).M[\dropn{u}]
\end{mathpar}

\subsection{Additional notation}

It will sometimes be convenient to denote the process a name
quotes. We already have the notation $x = \quotep{P}$, but it will be
convenient to introduce an alternate notation, $\procn{x}$, when we
want to emphasize the connection to the use of the name. Note that, by
virtue of name equivalence, $\quotep{\procn{x}} \nameeq x$; so, the
notation is consistent with previous definitions.

Further, because names have structure it is possible to effect
substitutions on the basis of that structure. This means we need to
upgrade our notation for substitutions, which we accomplish by
adapting comprehension notation. Thus,

\begin{mathpar}
  P\{ y / x : x \in S \}
\end{mathpar}

is interpreted to mean the process derived from P by replacing (in a
capture-avoiding manner) each occurrence of $x$ in $S$ by $y$. For example,

\begin{mathpar}
  P\{ \quotep{\procn{x}|\procn{x}} / x : x \in \freenames{P} \}
\end{mathpar}

will replace each (occurrence) of a free name $x$ in $P$ by
$\quotep{\procn{x}|\procn{x}}$.

Also, we will avail ourselves of the notation $x^{L}$ and $x^{R}$ to
denote injections of a name into disjoint copies of the name
space. There are numerous ways to accomplish this. One example can be
found in \cite{MeredithR05}. This notation overloads to vectors of
names: $\vec{x}^{\pi} := (x_{i}^{\pi} \; : \; 0 \leq i < |\vec{x}| )$ where $\pi \in \{L,R\}$.

We also use $P^{\Box} := P|\Box$.

In \cite{MeredithR05} an interpretation of the new operator is
given. It turns out that there are several possible interpretations
all enjoying the requisite algebraic properties of the operator (see
\cite{milner91polyadicpi}). We will therefore make liberal use of
$(\nu\; \vec{x})P$.

% subsection the_syntax_and_semantics_of_the_notation_system (end)   

\input{qm2pi.qmops} 

\input{qm2pi.sterngerlach} 

\input{qm2pi.metric} 

% section concurrent_process_calculi (end)

%\input{qm2pi.proofsketch}

% section proof sketch (end)

%\input{qm2pi.slviaknots} 

% section spatial logic via knots (end)

\input{qm2pi.conclusion}

% section conclusion (end)

%\input{qm2pi.dtcodes} 

% section wiring algorithm (end)

\input{qm2pi.ack} 

% section acknowledgments (end)

\newpage


\bibliographystyle{plain}   
\bibliography{../../biblios/main.bib}

\input{qm2pi.rhodetails}

\end{document}

 

\documentclass[12pt]{llncs}
%\documentclass{jktr}

\usepackage[pdftex]{hyperref}                   
\usepackage {listings}
\usepackage {mathpartir}
\usepackage{bcprules}
%\usepackage{listings}
                       
\usepackage{graphicx} 
%\usepackage[margins=2.5cm,nohead,nofoot]{geometry}
%\usepackage{geometry}
\usepackage{amsfonts}
\usepackage{amstext}
\usepackage{latexsym}
\usepackage{amssymb}
\usepackage{color}


%\include{myPreamble}
\include{qm2pi.local} 

%\ifpdf
%\usepackage[pdftex]{graphicx}
%\else
%\usepackage{graphicx}
%\fi

 % \ifpdf
%  \usepackage{pdfsync}
%  \if


%\title{Brief Article}
%\author{David F. Snyder}
%\author{L.G. Meredith}

%\address{Dept. of Math., Texas State University--San Marcos, San Marcos, TX 78666}
       
\pagestyle{empty}


\begin{document}

\lstset{language=[Objective]Caml,frame=shadowbox}

\input{qm2pi.front}

% section front matter (end)

\input{qm2pi.intro} 
 
% section introduction (end)

% \input{qm2pi.knotations} 

% section notation (end)

\input{qm2pi.process.calculi} 

% section concurrent_process_calculi_and_spatial_logics_ (end)
    
%\input{qm2pi.knots2pi} 

%\input{qm2pi.trefoil} 

%\input{qm2pi.mainthm} 

% subsection basic_interpretation (end)

%\input{qm2pi.rho.presentation} 
\subsection{The syntax and semantics of the notation system}\label{sub:the_syntax_and_semantics_of_the_notation_system} % (fold)

We now summarize a technical presentation of the calculus that
embodies our theory of dynamics. The typical presentation of such a
calculus follows the style of giving generators and relations on
them. The grammar, below, describing term constructors, freely
generates the set of processes, $\Proc$. This set is then quotiented
by a relation known as structural congruence and it is over this set
that the notion of dynamics is expressed. This presentation is
essentially that of \cite{MeredithR05} with the addition of
polyadicity and summation. For readability we have relegated some of
the technical subtleties to an appendix.

\subsubsection{Process grammar}\label{subsub:process_grammar}

\begin{mathpar}
  \inferrule* [lab=synchronization] {} {{M} \bc \pzero \;|\; x?F \;|\; x!C }
  \and
  \inferrule* [lab=abstraction] {} {{F} \bc (x)P}
  \and
  \inferrule* [lab=concretion] {} {{C} \bc \langle Q \rangle}
  \and
  \inferrule* [lab=process] {} {{P,Q} \bc M \;| \;P|Q \;|\; @{x}}
  \and
  \inferrule* [lab=name] {} {{x} \bc \quotep{P}}
\end{mathpar} 

Note that $\vec{x}$ (resp. $\vec{P}$) denotes a vector of names
(resp. processes) of length $|\vec{x}|$ (resp. $|\vec{P}|$). We adopt
the following useful abbreviations.

\begin{mathpar}
   x?(\vec{y}).P := x.(\vec{y})P \and  x\clift{\vec{P}} := x.\clift{\vec{P}}
   \and x!(y) := \lift{x}{\dropn{y}}
   \and \Pi_{i=0}^{n-1}P_i := P_0 | \ldots | P_{n-1}
\end{mathpar}

\subsubsection{Structural congruence}

\paragraph{Free and bound names and alpha-equivalence.} At the
core of structural equivalence is alpha-equivalence which identifies
process that are the same up to a change of variable. Formally, we
recognize the distinction between free and bound names. The free names
of a process, $\freenames{P}$, may be calculated recursively as
follows:

\begin{mathpar}
\freenames{\pzero} := \emptyset
  \and \\
  \freenames{x?(y).P} := \{ x \} \cup (\freenames{P} \setminus \{ y \})
  \and 
  \freenames{x!\langle P \rangle} := \{ x \} \cup \{ P \} 
  \and \\
  \freenames{P|Q} := \freenames{P} \cup \freenames{Q}
  \and \\
  \freenames{@{x}} := \{ x \}
\end{mathpar}

$\pi$
$\quotep{\pi}$

$\freenames{-} : \pi \to \mathcal{P}(\quotep{\pi})$

\begin{eqnarray*}
  \freenames{\pzero} & := & \emptyset \\
  \freenames{x?(y).P} & := & \{ x \} \cup (\freenames{P} \setminus \{ y \}) \\
  \freenames{x!\langle P \rangle} & := & \{ x \} \cup \{ P \} \\
  \freenames{P|Q} & := & \freenames{P} \cup \freenames{Q} \\
  \freenames{\dropn{x}} & := & \{ x \}
\end{eqnarray*}

The bound names of a process, $\boundnames{P}$, are those names occurring in $P$
that are not free. For example, in $x?(y).0$, the name $x$ is free, while $y$ is bound.

\begin{mathpar}
  \inferrule* [lab=monoidal-laws] {} { P|Q \equiv Q|P \and P|0 \equiv P \and P|(Q|R) \equiv (P|Q)|R }
\end{mathpar}

\begin{mathpar}
  \inferrule* [lab=alpha-equivalence] {} { (x)P \equiv (y)P\{y/x\} \and y \not\in \freenames{P} }
\end{mathpar}

\begin{definition}
Then two processes, $P,Q$, are alpha-equivalent if $P = Q\{\vec{y}/\vec{x}\}$ for
some $\vec{x} \in \boundnames{Q},\vec{y} \in \boundnames{P}$, where $Q\{\vec{y}/\vec{x}\}$
denotes the capture-avoiding substitution of $\vec{y}$ for $\vec{x}$ in $Q$.
\end{definition}

\begin{definition}
  The {\em structural congruence} \cite{SangiorgiWalker} , $\equiv$,
  between processes is the least congruence containing
  alpha-equivalence, satisfying the abelian monoid laws
  (associativity, commutativity and $\pzero$ as identity) for parallel
  composition $|$ and for summation $+$.
\end{definition}

\subsection{Name equivalence}

We take name equivalence, written $\nameeq$, to be the smallest
equivalence relation generated by the following rules.

\begin{mathpar}
\inferrule*[lab=Quote-drop]
{ }
{ \quotep{@{x}} \nameeq x }

\inferrule*[lab=Struct-equiv]
{ P \scong Q }
{ \quotep{P} \nameeq \quotep{Q} }
\end{mathpar}

The astute reader will have noticed that the mutual recursion of names
and processes imposes a mutual recursion on alpha-equivalence and
structural equivalence via name-equivalence. Fortunately, all of this
works out pleasantly and we may calculate in the natural way, free of
concern. The reader interested in the details is referred to the
appendix \ref{appendix:rho_details}.

\subsection{Substitution}

We use $\Proc$ for the set of processes, $\QProc$ for the set of
names, and $\id{\{}\vec{y} / \vec{x} \id{\}}$ to denote partial maps,
$s : \QProc \rightarrow \QProc$. A map, $s$ lifts, uniquely, to a map
on process terms, $\widehat{s} : \Proc \rightarrow \Proc$ by the
following equations.

\begin{mathpar}
  (0) \psubstp{Q}{P} := 0 \\
  (R \juxtap S) \psubstp{Q}{P}
  :=    
  (R)\psubstp{Q}{P} \juxtap (S) \psubstp{Q}{P} \\
  (x?(y).R) \psubstp{Q}{P}    
  :=    
  (x)\substp{Q}{P} (z)\concat( (R \psubstn{z}{y}) \psubstp{Q}{P} ) \\
  (\lift{x}{R}) \psubstp{Q}{P}  
  :=
  \lift{(x)\substp{Q}{P}}{ R \psubstp{Q}{P} } \\
%   (\dropn{x})  \psubstp{Q}{P}       
%   := 
%   \left\{ 
%     \begin{array}{ccc} 
%       \dropn{\quotep{Q}} & & x \nameeq \quotep{P} \\
%       \dropn{x} & & otherwise \\
%     \end{array}
%   \right. 
  (\dropn{x})  \psubstp{Q}{P}       
  := 
  \left\{ 
    \begin{array}{ccc} 
      Q & & x \nameeq \quotep{P} \\
      \dropn{x} & & otherwise \\
    \end{array}
  \right.
\end{mathpar}
 

where

\begin{eqnarray}
  (x)\id{\{} \lpquote Q \rpquote / \lpquote P \rpquote \id{\}}            = 
  \left\{ 
    \begin{array}{ccc}
      \lpquote Q \rpquote & & x \nameeq \lpquote P \rpquote \\
      x & & otherwise \\
    \end{array}
  \right. \nonumber
\end{eqnarray}

and $z$ is chosen distinct from $\quotep{P}$, $\quotep{Q}$, the free
names in $Q$, and all the names in $R$. Our $\alpha$-equivalence will
be built in the standard way from this substitution.

\begin{remark}\label{rem:no_self_referential_names}
  One consequence of these definitions is that $\forall P. \quotep{P}
  \not\in \freenames{P}$.
\end{remark}

\subsection{ Dynamic quote: an example }

Anticipating something of what's to come, consider applying the
substitution, $\widehat{\id{\{}u / z \id{\}}}$, to the following pair
of processes, $\lift{w}{y!(z)}$ and $w[ \lpquote y!(z) \rpquote ]$.

\begin{eqnarray}
	\lift{w}{y!(z)}\widehat{\id{\{}u / z \id{\}}}
		& = &
		\lift{w}{y!(u)} \nonumber\\
	w[ \lpquote y!(z) \rpquote ] \widehat{ \id{\{}u / z \id{\}} }
		& = &
		w[ \lpquote y!(z) \rpquote ] \nonumber
\end{eqnarray}

Because the body of the process between quotes is impervious to
substitution, we get radically different answers. In fact, by
examining the first process in an input context,
e.g. $x?(z).\lift{w}{y!(z)}$, we see that the process under the lift
operator may be shaped by prefixed inputs binding a name inside it. In
this sense, the lift operator will be seen as a way to dynamically
construct processes before reifying them as names.

Finally equipped with these standard features we can present the
dynamics of the calculus.

\subsubsection{Operational semantics} 

Finally, we introduce the computational dynamics. What marks these
algebras as distinct from other more traditionally studied algebraic
structures, e.g. vector spaces or polynomial rings, is the manner in
which dynamics is captured. In traditional structures, dynamics is typically
expressed through morphisms between such structures, as in linear maps
between vector spaces or morphisms between rings. In algebras
associated with the semantics of computation, the dynamics is
expressed as part of the algebraic structure itself, through a
reduction reduction relation typically denoted by $\red$. Below, we
give a recursive presentation of this relation for the calculus used
in the encoding.

$\red \subseteq \pi \times \pi$
$\red : \pi \to \mathcal{P}(\pi)$

\begin{mathpar}
  \inferrule* [lab=Comm] { \textsf{match}( x_{src}, x_{trgt} ) } { x_{trgt}?(y)P \; | \; x_{src}!\langle {Q} \rangle \red P\{\quotep{Q}/y}\} }
  \and \\
  \inferrule* [lab=Par] {{P} \red {P}'} {{{P} | {Q}} \red {{P}' | {Q}}}
  \and
  \inferrule* [lab=Equiv]{{{P} \scong {P}'} \andalso {{P}' \red {Q}'} \andalso {{Q}' \scong {Q}}}{{P} \red {Q}}
\end{mathpar}

\begin{eqnarray*}
  match_{\equiv} (\quotep{P},\quotep{Q}) & := & P \equiv Q \\
  match_{\dagger}(\quotep{P},\quotep{Q}) & := & \forall R. P|Q \red^{*} R => R \red^{*} 0 \\
  match_{K}(\quotep{P},\quotep{Q}) & := & K \mbox{ for some context } K
\end{eqnarray*}

$u?(x)P | u!\langle Q \rangle \red P\{\quotep{Q}/x\}$

%We write $\wred$ for $\red^*$, and $P\red$ if $\exists Q $ such that $ P \red Q$.
We write $P\red$ if $\exists Q $ such that $ P \red Q$ and $P\not\red$, otherwise.

\section{Replication}

As mentioned before, it is known that replication (and hence
recursion) can be implemented in a higher-order process algebra
\cite{SangiorgiWalker}. As our first example of calculation with the
machinery thus far presented we give the construction explicitly in
the {\rhoc}.

\begin{eqnarray}
	D_{x} & := & \prefix{x}{y}{(\binpar{\outputp{x}{y}}{@{y}})} \nonumber\\
	\bangp_{x}{P} & := & \binpar{{x}!\langle{\binpar{D_{x}}{P}}\rangle}{D_{x}} \nonumber
\end{eqnarray}

\begin{eqnarray}
	\bangp_{x}{P} & & \nonumber\\
	=
	& {x}!\langle{(\prefix{x}{y}{(\outputp{x}{y} | @{y})) | P}}\rangle 
	      | \prefix{x}{y}{(\outputp{x}{y} | @{y})} & \nonumber\\
	\red
	& (\outputp{x}{y} | @{y})\substn{\quotep{(\prefix{x}{y}{(@{y} | \outputp{x}{y})) | P}}}{y} & \nonumber\\
	=
	& \outputp{x}{\quotep{(\prefix{x}{y}{(\outputp{x}{y} | @{y})) | P}}}
	  | {(\prefix{x}{y}{(\outputp{x}{y} | @{y})) | P}} & \nonumber\\
	\red
	& \ldots & \nonumber\\
	\red^*
	& P | P | \ldots & \nonumber
\end{eqnarray}

Of course, this encoding, as an implementation, runs away, unfolding
$\bangp{P}$ eagerly. A lazier and more implementable replication
operator, restricted to input-guarded processes, may be obtained as follows.

\begin{eqnarray}
\bangp{\prefix{u}{v}{P}} 
	:= 
	\binpar{\lift{x}{\prefix{u}{v}{(\binpar{D(x)}{P})}}}{D(x)} \nonumber
\end{eqnarray}

\begin{remark}
  Note that the lazier definition still does not deal with summation
  or mixed summation (i.e. sums over input and output). The reader is
  invited to construct definitions of replication that deal with these
  features. 

  Further, the definitions are parameterized in a name, $x$. Can you,
  gentle reader, make a definition that eliminates this parameter and
  guarantees no accidental interaction between the replication
  machinery and the process being replicated -- i.e. no accidental
  sharing of names used by the process to get its work done and the
  name(s) used by the replication to effect copying. This latter
  revision of the definition of replication is crucial to obtaining
  the expected identity $!!P \sim !P$.
\end{remark}

\begin{remark}\label{rem:paradoxical_combinator}
  The reader familiar with the lambda calculus will have noticed the
  similarity between $D$ and the paradoxical combinator.

  [Ed. note: the existence of this seems to suggest we have to be more
  restrictive on the set of processes and names we admit if we are to
  support no-cloning.]
\end{remark}

\subsubsection{Bisimulation}

The computational dynamics gives rise to another kind of equivalence,
the equivalence of computational behavior. As previously mentioned
this is typically captured \emph{via} some form of bisimulation.

% The notion we use in this paper is weak barbed bisimulation
% \cite{milner91polyadicpi}.

The notion we use in this paper is derived from weak barbed
bisimulation \cite{milner91polyadicpi}. 

\begin{definition}
An \emph{observation relation}, $\downarrow_{\mathcal N}$, over a set
of names, $\mathcal N$, is the smallest relation satisfying the rules
below.

\infrule[Out-barb]{y \in {\mathcal N}, \; x \nameeq y}
		  {\outputp{x}{v} \downarrow_{\mathcal N} x}
\infrule[Par-barb]{\mbox{$P\downarrow_{\mathcal N} x$ or $Q\downarrow_{\mathcal N} x$}}
		  {\binpar{P}{Q} \downarrow_{\mathcal N} x}

We write $P \Downarrow_{\mathcal N} x$ if there is $Q$ such that 
$P \wred Q$ and $Q \downarrow_{\mathcal N} x$.
\end{definition}

\begin{definition}
%\label{def.bbisim}
An  ${\mathcal N}$-\emph{barbed bisimulation} over a set of names, ${\mathcal N}$, is a symmetric binary relation 
${\mathcal S}_{\mathcal N}$ between agents such that $P\rel{S}_{\mathcal N}Q$ implies:
\begin{enumerate}
\item If $P \red P'$ then $Q \wred Q'$ and $P'\rel{S}_{\mathcal N} Q'$.
\item If $P\downarrow_{\mathcal N} x$, then $Q\Downarrow_{\mathcal N} x$.
\end{enumerate}
$P$ is ${\mathcal N}$-barbed bisimilar to $Q$, written
$P \wbbisim_{\mathcal N} Q$, if $P \rel{S}_{\mathcal N} Q$ for some ${\mathcal N}$-barbed bisimulation ${\mathcal S}_{\mathcal N}$.
\end{definition}

$\mathcal{R} \subseteq \pi \times \pi$

$P \mathcal{R} Q => \forall P'. P \red P' \Rightarrow \exists Q'. Q \red Q', P' \mathcal{R} Q'$

$P \vdash x \Rightarrow Q \vdash x$

\begin{mathpar}
  \inferrule*[lab=Out-barb]{x \nameeq y}{{y}!\langle{Q}\rangle \vdash x}
  \and
  \inferrule*[lab=Par-barb]{\mbox{$P\vdash x$ or $Q\vdash x$}}{\binpar{P}{Q} \vdash x}
\end{mathpar}

\subsubsection{Contexts}

One of the principle advantages of computational calculi like the
$\pi$-calculus is a well-defined notion of context,
contextual-equivalence and a correlation between
contextual-equivalence and notions of bisimulation. The notion of
context allows the decomposition of a process into (sub-)process and
its syntactic environment, its context. Thus, a context may be
thought of as a process with a ``hole'' (written $\Box$) in it. The
application of a context $M$ to a process $P$, written $M[P]$, is
tantamount to filling the hole in $M$ with $P$. In this paper we do
not need the full weight of this theory, but do make use of the notion
of context in the proof the main theorem. 

\begin{mathpar}
  \inferrule* [lab=summation] {} {{M_{M},M_{N}} \bc \Box \;|\; x.M_{A} \;|\; M_{M}+M_{N}}
  \and
  \inferrule* [lab=agent] {} {{M_{A}} \bc (\vec{x})M_{P} \;| \; \clift{P_0,\ldots,M_{P},\ldots,P_N}}
  \and \\
  \inferrule* [lab=process] {} {{M_{P}} \bc M_{N} \;| \;P|M_{P} }
\end{mathpar} 

\begin{mathpar}
  \inferrule* [lab=sychronization] {} {M_{N} \bc \Box \;|\; x?M_{F} \;|\; x!M_{C}}
  \and
  \inferrule* [lab=abstraction] {} {{M_{F}} \bc (x)M_{P} }
  \and
  \inferrule* [lab=concretion] {} {{M_{C}} \bc \langle M_{P} \rangle }
  \and \\
  \inferrule* [lab=process] {} {{M_{P}} \bc M_{N} \;| \;P|M_{P} }
\end{mathpar}

\begin{definition}[contextual application] Given a context $M$, and
  process $P$, we define the \emph{contextual application}, $M[P] :=
  M\{P/\Box\}$. That is, the contextual application of M to P is the
  substitution of $P$ for $\Box$ in $M$.
\end{definition}

$\meaningof{-} : L \to \mathcal{P}(\pi)$

\begin{mathpar}
  \inferrule* [lab=collection] {} {\meaningof{true} = \pi, \and \meaningof{~E} = \pi \setminus \meaningof{E}, \and \meaningof{E_{1} \& E_{2}} = \meaningof{E_{1}} \cap \meaningof{E_{2}}}
\end{mathpar}

\begin{mathpar}
  \inferrule* [lab=structure] {} {\meaningof{0} = \{ P \in \pi | P \equiv 0 \}, \and \\ \meaningof{E_1 | E_2} = \{ P \in \pi | P \equiv P_{1} | P_{2}, P_{1} \in \meaningof{E_{1}}, P_{2} \in \meaningof{E_2}\} }
\end{mathpar}

\begin{mathpar}
 \inferrule* [lab=behavior] {} {\meaningof{\langle a?b \rangle E} = \{ P \in \pi | P \equiv Q | u?(y)P', \\ \and \\\\ \and \\ \;\;\; u \in \meaningof{a}, \forall z.P'\{z/y\} \in \meaningof{E\{z/b\}}\}, \and \\ \meaningof{a!E} = \{ P \in \pi | P \equiv Q | x!\langle P' \rangle, x \in \meaningof{a} P' \in \meaningof{E}\} }
\end{mathpar}

\begin{mathpar}
 \inferrule* [lab=nominal] {} {\meaningof{\quotep{E}} = \{ \quotep{P} \in \quotep{\pi} | P \in \meaningof{E} \}, \and \meaningof{\quotep{P}} = \{ \quotep{Q} \in \quotep{\pi} | P \equiv Q \} \and \\ \meaningof{@\quotep{E}} = \{ P \in \pi | P \equiv @x, x \in \meaningof{E} \}}
\end{mathpar}

\begin{eqnarray*}
  \\
  \meaningof{-} : TS \to ST
\end{eqnarray*}

\begin{eqnarray*}
  \\
  L : TS \to ST
\end{eqnarray*}

\begin{eqnarray*}
  \\
  P \models E \iff P \in \meaningof{E}
\end{eqnarray*}

\begin{eqnarray*}
  P \approx_{L} Q \iff \forall E \in L. P \models E \iff Q \models E
\end{eqnarray*}

\begin{eqnarray*}
  P \approx_{K} Q
\end{eqnarray*}

\begin{eqnarray*}
  P \approx Q
\end{eqnarray*}

$\approx_{K} = \approx = \approx_{L}$

\subsubsection{Contextual duality}

Note that contexts extend the quotation operation to a family of
operations from processes to names. Given a context, $M$, we can
define a \emph{nominal context}, $\quotep{M}$ by $\quotep{M}[P] :=
\quotep{M[P]}$. To foreshadow what is to come we observe that these
operations enjoy a duality with processes very much like the duality
between vectors and maps from vectors to scalars.

Further, because the calculus is essentially higher-order, we have a
correspondence between contexts and processes. More specifically,
given a name $x$ and a context $M$ we can construct $M^{*}_{x}$ such
that 

\begin{mathpar}
  M^{*}_{x} | \lift{x}{P} \red M[P]
\end{mathpar}

namely,

\begin{mathpar}
  M^{*}_{x} := x?(u).M[\dropn{u}]
\end{mathpar}

The dependence of $M^{*}_{x}$ on a name makes it an abstraction, 

\begin{mathpar}
  M^{*} := (x)x?(u).M[\dropn{u}]
\end{mathpar}

\subsection{Additional notation}

It will sometimes be convenient to denote the process a name
quotes. We already have the notation $x = \quotep{P}$, but it will be
convenient to introduce an alternate notation, $\procn{x}$, when we
want to emphasize the connection to the use of the name. Note that, by
virtue of name equivalence, $\quotep{\procn{x}} \nameeq x$; so, the
notation is consistent with previous definitions.

Further, because names have structure it is possible to effect
substitutions on the basis of that structure. This means we need to
upgrade our notation for substitutions, which we accomplish by
adapting comprehension notation. Thus,

\begin{mathpar}
  P\{ y / x : x \in S \}
\end{mathpar}

is interpreted to mean the process derived from P by replacing (in a
capture-avoiding manner) each occurrence of $x$ in $S$ by $y$. For example,

\begin{mathpar}
  P\{ \quotep{\procn{x}|\procn{x}} / x : x \in \freenames{P} \}
\end{mathpar}

will replace each (occurrence) of a free name $x$ in $P$ by
$\quotep{\procn{x}|\procn{x}}$.

Also, we will avail ourselves of the notation $x^{L}$ and $x^{R}$ to
denote injections of a name into disjoint copies of the name
space. There are numerous ways to accomplish this. One example can be
found in \cite{MeredithR05}. This notation overloads to vectors of
names: $\vec{x}^{\pi} := (x_{i}^{\pi} \; : \; 0 \leq i < |\vec{x}| )$ where $\pi \in \{L,R\}$.

We also use $P^{\Box} := P|\Box$.

In \cite{MeredithR05} an interpretation of the new operator is
given. It turns out that there are several possible interpretations
all enjoying the requisite algebraic properties of the operator (see
\cite{milner91polyadicpi}). We will therefore make liberal use of
$(\nu\; \vec{x})P$.

% subsection the_syntax_and_semantics_of_the_notation_system (end)   

\input{qm2pi.qmops} 

\input{qm2pi.sterngerlach} 

\input{qm2pi.metric} 

% section concurrent_process_calculi (end)

%\input{qm2pi.proofsketch}

% section proof sketch (end)

%\input{qm2pi.slviaknots} 

% section spatial logic via knots (end)

\input{qm2pi.conclusion}

% section conclusion (end)

%\input{qm2pi.dtcodes} 

% section wiring algorithm (end)

\input{qm2pi.ack} 

% section acknowledgments (end)

\newpage


\bibliographystyle{plain}   
\bibliography{../../biblios/main.bib}

\input{qm2pi.rhodetails}

\end{document}

 

% section concurrent_process_calculi (end)

%\documentclass[12pt]{llncs}
%\documentclass{jktr}

\usepackage[pdftex]{hyperref}                   
\usepackage {listings}
\usepackage {mathpartir}
\usepackage{bcprules}
%\usepackage{listings}
                       
\usepackage{graphicx} 
%\usepackage[margins=2.5cm,nohead,nofoot]{geometry}
%\usepackage{geometry}
\usepackage{amsfonts}
\usepackage{amstext}
\usepackage{latexsym}
\usepackage{amssymb}
\usepackage{color}


%\include{myPreamble}
\include{qm2pi.local} 

%\ifpdf
%\usepackage[pdftex]{graphicx}
%\else
%\usepackage{graphicx}
%\fi

 % \ifpdf
%  \usepackage{pdfsync}
%  \if


%\title{Brief Article}
%\author{David F. Snyder}
%\author{L.G. Meredith}

%\address{Dept. of Math., Texas State University--San Marcos, San Marcos, TX 78666}
       
\pagestyle{empty}


\begin{document}

\lstset{language=[Objective]Caml,frame=shadowbox}

\input{qm2pi.front}

% section front matter (end)

\input{qm2pi.intro} 
 
% section introduction (end)

% \input{qm2pi.knotations} 

% section notation (end)

\input{qm2pi.process.calculi} 

% section concurrent_process_calculi_and_spatial_logics_ (end)
    
%\input{qm2pi.knots2pi} 

%\input{qm2pi.trefoil} 

%\input{qm2pi.mainthm} 

% subsection basic_interpretation (end)

%\input{qm2pi.rho.presentation} 
\subsection{The syntax and semantics of the notation system}\label{sub:the_syntax_and_semantics_of_the_notation_system} % (fold)

We now summarize a technical presentation of the calculus that
embodies our theory of dynamics. The typical presentation of such a
calculus follows the style of giving generators and relations on
them. The grammar, below, describing term constructors, freely
generates the set of processes, $\Proc$. This set is then quotiented
by a relation known as structural congruence and it is over this set
that the notion of dynamics is expressed. This presentation is
essentially that of \cite{MeredithR05} with the addition of
polyadicity and summation. For readability we have relegated some of
the technical subtleties to an appendix.

\subsubsection{Process grammar}\label{subsub:process_grammar}

\begin{mathpar}
  \inferrule* [lab=synchronization] {} {{M} \bc \pzero \;|\; x?F \;|\; x!C }
  \and
  \inferrule* [lab=abstraction] {} {{F} \bc (x)P}
  \and
  \inferrule* [lab=concretion] {} {{C} \bc \langle Q \rangle}
  \and
  \inferrule* [lab=process] {} {{P,Q} \bc M \;| \;P|Q \;|\; @{x}}
  \and
  \inferrule* [lab=name] {} {{x} \bc \quotep{P}}
\end{mathpar} 

Note that $\vec{x}$ (resp. $\vec{P}$) denotes a vector of names
(resp. processes) of length $|\vec{x}|$ (resp. $|\vec{P}|$). We adopt
the following useful abbreviations.

\begin{mathpar}
   x?(\vec{y}).P := x.(\vec{y})P \and  x\clift{\vec{P}} := x.\clift{\vec{P}}
   \and x!(y) := \lift{x}{\dropn{y}}
   \and \Pi_{i=0}^{n-1}P_i := P_0 | \ldots | P_{n-1}
\end{mathpar}

\subsubsection{Structural congruence}

\paragraph{Free and bound names and alpha-equivalence.} At the
core of structural equivalence is alpha-equivalence which identifies
process that are the same up to a change of variable. Formally, we
recognize the distinction between free and bound names. The free names
of a process, $\freenames{P}$, may be calculated recursively as
follows:

\begin{mathpar}
\freenames{\pzero} := \emptyset
  \and \\
  \freenames{x?(y).P} := \{ x \} \cup (\freenames{P} \setminus \{ y \})
  \and 
  \freenames{x!\langle P \rangle} := \{ x \} \cup \{ P \} 
  \and \\
  \freenames{P|Q} := \freenames{P} \cup \freenames{Q}
  \and \\
  \freenames{@{x}} := \{ x \}
\end{mathpar}

$\pi$
$\quotep{\pi}$

$\freenames{-} : \pi \to \mathcal{P}(\quotep{\pi})$

\begin{eqnarray*}
  \freenames{\pzero} & := & \emptyset \\
  \freenames{x?(y).P} & := & \{ x \} \cup (\freenames{P} \setminus \{ y \}) \\
  \freenames{x!\langle P \rangle} & := & \{ x \} \cup \{ P \} \\
  \freenames{P|Q} & := & \freenames{P} \cup \freenames{Q} \\
  \freenames{\dropn{x}} & := & \{ x \}
\end{eqnarray*}

The bound names of a process, $\boundnames{P}$, are those names occurring in $P$
that are not free. For example, in $x?(y).0$, the name $x$ is free, while $y$ is bound.

\begin{mathpar}
  \inferrule* [lab=monoidal-laws] {} { P|Q \equiv Q|P \and P|0 \equiv P \and P|(Q|R) \equiv (P|Q)|R }
\end{mathpar}

\begin{mathpar}
  \inferrule* [lab=alpha-equivalence] {} { (x)P \equiv (y)P\{y/x\} \and y \not\in \freenames{P} }
\end{mathpar}

\begin{definition}
Then two processes, $P,Q$, are alpha-equivalent if $P = Q\{\vec{y}/\vec{x}\}$ for
some $\vec{x} \in \boundnames{Q},\vec{y} \in \boundnames{P}$, where $Q\{\vec{y}/\vec{x}\}$
denotes the capture-avoiding substitution of $\vec{y}$ for $\vec{x}$ in $Q$.
\end{definition}

\begin{definition}
  The {\em structural congruence} \cite{SangiorgiWalker} , $\equiv$,
  between processes is the least congruence containing
  alpha-equivalence, satisfying the abelian monoid laws
  (associativity, commutativity and $\pzero$ as identity) for parallel
  composition $|$ and for summation $+$.
\end{definition}

\subsection{Name equivalence}

We take name equivalence, written $\nameeq$, to be the smallest
equivalence relation generated by the following rules.

\begin{mathpar}
\inferrule*[lab=Quote-drop]
{ }
{ \quotep{@{x}} \nameeq x }

\inferrule*[lab=Struct-equiv]
{ P \scong Q }
{ \quotep{P} \nameeq \quotep{Q} }
\end{mathpar}

The astute reader will have noticed that the mutual recursion of names
and processes imposes a mutual recursion on alpha-equivalence and
structural equivalence via name-equivalence. Fortunately, all of this
works out pleasantly and we may calculate in the natural way, free of
concern. The reader interested in the details is referred to the
appendix \ref{appendix:rho_details}.

\subsection{Substitution}

We use $\Proc$ for the set of processes, $\QProc$ for the set of
names, and $\id{\{}\vec{y} / \vec{x} \id{\}}$ to denote partial maps,
$s : \QProc \rightarrow \QProc$. A map, $s$ lifts, uniquely, to a map
on process terms, $\widehat{s} : \Proc \rightarrow \Proc$ by the
following equations.

\begin{mathpar}
  (0) \psubstp{Q}{P} := 0 \\
  (R \juxtap S) \psubstp{Q}{P}
  :=    
  (R)\psubstp{Q}{P} \juxtap (S) \psubstp{Q}{P} \\
  (x?(y).R) \psubstp{Q}{P}    
  :=    
  (x)\substp{Q}{P} (z)\concat( (R \psubstn{z}{y}) \psubstp{Q}{P} ) \\
  (\lift{x}{R}) \psubstp{Q}{P}  
  :=
  \lift{(x)\substp{Q}{P}}{ R \psubstp{Q}{P} } \\
%   (\dropn{x})  \psubstp{Q}{P}       
%   := 
%   \left\{ 
%     \begin{array}{ccc} 
%       \dropn{\quotep{Q}} & & x \nameeq \quotep{P} \\
%       \dropn{x} & & otherwise \\
%     \end{array}
%   \right. 
  (\dropn{x})  \psubstp{Q}{P}       
  := 
  \left\{ 
    \begin{array}{ccc} 
      Q & & x \nameeq \quotep{P} \\
      \dropn{x} & & otherwise \\
    \end{array}
  \right.
\end{mathpar}
 

where

\begin{eqnarray}
  (x)\id{\{} \lpquote Q \rpquote / \lpquote P \rpquote \id{\}}            = 
  \left\{ 
    \begin{array}{ccc}
      \lpquote Q \rpquote & & x \nameeq \lpquote P \rpquote \\
      x & & otherwise \\
    \end{array}
  \right. \nonumber
\end{eqnarray}

and $z$ is chosen distinct from $\quotep{P}$, $\quotep{Q}$, the free
names in $Q$, and all the names in $R$. Our $\alpha$-equivalence will
be built in the standard way from this substitution.

\begin{remark}\label{rem:no_self_referential_names}
  One consequence of these definitions is that $\forall P. \quotep{P}
  \not\in \freenames{P}$.
\end{remark}

\subsection{ Dynamic quote: an example }

Anticipating something of what's to come, consider applying the
substitution, $\widehat{\id{\{}u / z \id{\}}}$, to the following pair
of processes, $\lift{w}{y!(z)}$ and $w[ \lpquote y!(z) \rpquote ]$.

\begin{eqnarray}
	\lift{w}{y!(z)}\widehat{\id{\{}u / z \id{\}}}
		& = &
		\lift{w}{y!(u)} \nonumber\\
	w[ \lpquote y!(z) \rpquote ] \widehat{ \id{\{}u / z \id{\}} }
		& = &
		w[ \lpquote y!(z) \rpquote ] \nonumber
\end{eqnarray}

Because the body of the process between quotes is impervious to
substitution, we get radically different answers. In fact, by
examining the first process in an input context,
e.g. $x?(z).\lift{w}{y!(z)}$, we see that the process under the lift
operator may be shaped by prefixed inputs binding a name inside it. In
this sense, the lift operator will be seen as a way to dynamically
construct processes before reifying them as names.

Finally equipped with these standard features we can present the
dynamics of the calculus.

\subsubsection{Operational semantics} 

Finally, we introduce the computational dynamics. What marks these
algebras as distinct from other more traditionally studied algebraic
structures, e.g. vector spaces or polynomial rings, is the manner in
which dynamics is captured. In traditional structures, dynamics is typically
expressed through morphisms between such structures, as in linear maps
between vector spaces or morphisms between rings. In algebras
associated with the semantics of computation, the dynamics is
expressed as part of the algebraic structure itself, through a
reduction reduction relation typically denoted by $\red$. Below, we
give a recursive presentation of this relation for the calculus used
in the encoding.

$\red \subseteq \pi \times \pi$
$\red : \pi \to \mathcal{P}(\pi)$

\begin{mathpar}
  \inferrule* [lab=Comm] { \textsf{match}( x_{src}, x_{trgt} ) } { x_{trgt}?(y)P \; | \; x_{src}!\langle {Q} \rangle \red P\{\quotep{Q}/y}\} }
  \and \\
  \inferrule* [lab=Par] {{P} \red {P}'} {{{P} | {Q}} \red {{P}' | {Q}}}
  \and
  \inferrule* [lab=Equiv]{{{P} \scong {P}'} \andalso {{P}' \red {Q}'} \andalso {{Q}' \scong {Q}}}{{P} \red {Q}}
\end{mathpar}

\begin{eqnarray*}
  match_{\equiv} (\quotep{P},\quotep{Q}) & := & P \equiv Q \\
  match_{\dagger}(\quotep{P},\quotep{Q}) & := & \forall R. P|Q \red^{*} R => R \red^{*} 0 \\
  match_{K}(\quotep{P},\quotep{Q}) & := & K \mbox{ for some context } K
\end{eqnarray*}

$u?(x)P | u!\langle Q \rangle \red P\{\quotep{Q}/x\}$

%We write $\wred$ for $\red^*$, and $P\red$ if $\exists Q $ such that $ P \red Q$.
We write $P\red$ if $\exists Q $ such that $ P \red Q$ and $P\not\red$, otherwise.

\section{Replication}

As mentioned before, it is known that replication (and hence
recursion) can be implemented in a higher-order process algebra
\cite{SangiorgiWalker}. As our first example of calculation with the
machinery thus far presented we give the construction explicitly in
the {\rhoc}.

\begin{eqnarray}
	D_{x} & := & \prefix{x}{y}{(\binpar{\outputp{x}{y}}{@{y}})} \nonumber\\
	\bangp_{x}{P} & := & \binpar{{x}!\langle{\binpar{D_{x}}{P}}\rangle}{D_{x}} \nonumber
\end{eqnarray}

\begin{eqnarray}
	\bangp_{x}{P} & & \nonumber\\
	=
	& {x}!\langle{(\prefix{x}{y}{(\outputp{x}{y} | @{y})) | P}}\rangle 
	      | \prefix{x}{y}{(\outputp{x}{y} | @{y})} & \nonumber\\
	\red
	& (\outputp{x}{y} | @{y})\substn{\quotep{(\prefix{x}{y}{(@{y} | \outputp{x}{y})) | P}}}{y} & \nonumber\\
	=
	& \outputp{x}{\quotep{(\prefix{x}{y}{(\outputp{x}{y} | @{y})) | P}}}
	  | {(\prefix{x}{y}{(\outputp{x}{y} | @{y})) | P}} & \nonumber\\
	\red
	& \ldots & \nonumber\\
	\red^*
	& P | P | \ldots & \nonumber
\end{eqnarray}

Of course, this encoding, as an implementation, runs away, unfolding
$\bangp{P}$ eagerly. A lazier and more implementable replication
operator, restricted to input-guarded processes, may be obtained as follows.

\begin{eqnarray}
\bangp{\prefix{u}{v}{P}} 
	:= 
	\binpar{\lift{x}{\prefix{u}{v}{(\binpar{D(x)}{P})}}}{D(x)} \nonumber
\end{eqnarray}

\begin{remark}
  Note that the lazier definition still does not deal with summation
  or mixed summation (i.e. sums over input and output). The reader is
  invited to construct definitions of replication that deal with these
  features. 

  Further, the definitions are parameterized in a name, $x$. Can you,
  gentle reader, make a definition that eliminates this parameter and
  guarantees no accidental interaction between the replication
  machinery and the process being replicated -- i.e. no accidental
  sharing of names used by the process to get its work done and the
  name(s) used by the replication to effect copying. This latter
  revision of the definition of replication is crucial to obtaining
  the expected identity $!!P \sim !P$.
\end{remark}

\begin{remark}\label{rem:paradoxical_combinator}
  The reader familiar with the lambda calculus will have noticed the
  similarity between $D$ and the paradoxical combinator.

  [Ed. note: the existence of this seems to suggest we have to be more
  restrictive on the set of processes and names we admit if we are to
  support no-cloning.]
\end{remark}

\subsubsection{Bisimulation}

The computational dynamics gives rise to another kind of equivalence,
the equivalence of computational behavior. As previously mentioned
this is typically captured \emph{via} some form of bisimulation.

% The notion we use in this paper is weak barbed bisimulation
% \cite{milner91polyadicpi}.

The notion we use in this paper is derived from weak barbed
bisimulation \cite{milner91polyadicpi}. 

\begin{definition}
An \emph{observation relation}, $\downarrow_{\mathcal N}$, over a set
of names, $\mathcal N$, is the smallest relation satisfying the rules
below.

\infrule[Out-barb]{y \in {\mathcal N}, \; x \nameeq y}
		  {\outputp{x}{v} \downarrow_{\mathcal N} x}
\infrule[Par-barb]{\mbox{$P\downarrow_{\mathcal N} x$ or $Q\downarrow_{\mathcal N} x$}}
		  {\binpar{P}{Q} \downarrow_{\mathcal N} x}

We write $P \Downarrow_{\mathcal N} x$ if there is $Q$ such that 
$P \wred Q$ and $Q \downarrow_{\mathcal N} x$.
\end{definition}

\begin{definition}
%\label{def.bbisim}
An  ${\mathcal N}$-\emph{barbed bisimulation} over a set of names, ${\mathcal N}$, is a symmetric binary relation 
${\mathcal S}_{\mathcal N}$ between agents such that $P\rel{S}_{\mathcal N}Q$ implies:
\begin{enumerate}
\item If $P \red P'$ then $Q \wred Q'$ and $P'\rel{S}_{\mathcal N} Q'$.
\item If $P\downarrow_{\mathcal N} x$, then $Q\Downarrow_{\mathcal N} x$.
\end{enumerate}
$P$ is ${\mathcal N}$-barbed bisimilar to $Q$, written
$P \wbbisim_{\mathcal N} Q$, if $P \rel{S}_{\mathcal N} Q$ for some ${\mathcal N}$-barbed bisimulation ${\mathcal S}_{\mathcal N}$.
\end{definition}

$\mathcal{R} \subseteq \pi \times \pi$

$P \mathcal{R} Q => \forall P'. P \red P' \Rightarrow \exists Q'. Q \red Q', P' \mathcal{R} Q'$

$P \vdash x \Rightarrow Q \vdash x$

\begin{mathpar}
  \inferrule*[lab=Out-barb]{x \nameeq y}{{y}!\langle{Q}\rangle \vdash x}
  \and
  \inferrule*[lab=Par-barb]{\mbox{$P\vdash x$ or $Q\vdash x$}}{\binpar{P}{Q} \vdash x}
\end{mathpar}

\subsubsection{Contexts}

One of the principle advantages of computational calculi like the
$\pi$-calculus is a well-defined notion of context,
contextual-equivalence and a correlation between
contextual-equivalence and notions of bisimulation. The notion of
context allows the decomposition of a process into (sub-)process and
its syntactic environment, its context. Thus, a context may be
thought of as a process with a ``hole'' (written $\Box$) in it. The
application of a context $M$ to a process $P$, written $M[P]$, is
tantamount to filling the hole in $M$ with $P$. In this paper we do
not need the full weight of this theory, but do make use of the notion
of context in the proof the main theorem. 

\begin{mathpar}
  \inferrule* [lab=summation] {} {{M_{M},M_{N}} \bc \Box \;|\; x.M_{A} \;|\; M_{M}+M_{N}}
  \and
  \inferrule* [lab=agent] {} {{M_{A}} \bc (\vec{x})M_{P} \;| \; \clift{P_0,\ldots,M_{P},\ldots,P_N}}
  \and \\
  \inferrule* [lab=process] {} {{M_{P}} \bc M_{N} \;| \;P|M_{P} }
\end{mathpar} 

\begin{mathpar}
  \inferrule* [lab=sychronization] {} {M_{N} \bc \Box \;|\; x?M_{F} \;|\; x!M_{C}}
  \and
  \inferrule* [lab=abstraction] {} {{M_{F}} \bc (x)M_{P} }
  \and
  \inferrule* [lab=concretion] {} {{M_{C}} \bc \langle M_{P} \rangle }
  \and \\
  \inferrule* [lab=process] {} {{M_{P}} \bc M_{N} \;| \;P|M_{P} }
\end{mathpar}

\begin{definition}[contextual application] Given a context $M$, and
  process $P$, we define the \emph{contextual application}, $M[P] :=
  M\{P/\Box\}$. That is, the contextual application of M to P is the
  substitution of $P$ for $\Box$ in $M$.
\end{definition}

$\meaningof{-} : L \to \mathcal{P}(\pi)$

\begin{mathpar}
  \inferrule* [lab=collection] {} {\meaningof{true} = \pi, \and \meaningof{~E} = \pi \setminus \meaningof{E}, \and \meaningof{E_{1} \& E_{2}} = \meaningof{E_{1}} \cap \meaningof{E_{2}}}
\end{mathpar}

\begin{mathpar}
  \inferrule* [lab=structure] {} {\meaningof{0} = \{ P \in \pi | P \equiv 0 \}, \and \\ \meaningof{E_1 | E_2} = \{ P \in \pi | P \equiv P_{1} | P_{2}, P_{1} \in \meaningof{E_{1}}, P_{2} \in \meaningof{E_2}\} }
\end{mathpar}

\begin{mathpar}
 \inferrule* [lab=behavior] {} {\meaningof{\langle a?b \rangle E} = \{ P \in \pi | P \equiv Q | u?(y)P', \\ \and \\\\ \and \\ \;\;\; u \in \meaningof{a}, \forall z.P'\{z/y\} \in \meaningof{E\{z/b\}}\}, \and \\ \meaningof{a!E} = \{ P \in \pi | P \equiv Q | x!\langle P' \rangle, x \in \meaningof{a} P' \in \meaningof{E}\} }
\end{mathpar}

\begin{mathpar}
 \inferrule* [lab=nominal] {} {\meaningof{\quotep{E}} = \{ \quotep{P} \in \quotep{\pi} | P \in \meaningof{E} \}, \and \meaningof{\quotep{P}} = \{ \quotep{Q} \in \quotep{\pi} | P \equiv Q \} \and \\ \meaningof{@\quotep{E}} = \{ P \in \pi | P \equiv @x, x \in \meaningof{E} \}}
\end{mathpar}

\begin{eqnarray*}
  \\
  \meaningof{-} : TS \to ST
\end{eqnarray*}

\begin{eqnarray*}
  \\
  L : TS \to ST
\end{eqnarray*}

\begin{eqnarray*}
  \\
  P \models E \iff P \in \meaningof{E}
\end{eqnarray*}

\begin{eqnarray*}
  P \approx_{L} Q \iff \forall E \in L. P \models E \iff Q \models E
\end{eqnarray*}

\begin{eqnarray*}
  P \approx_{K} Q
\end{eqnarray*}

\begin{eqnarray*}
  P \approx Q
\end{eqnarray*}

$\approx_{K} = \approx = \approx_{L}$

\subsubsection{Contextual duality}

Note that contexts extend the quotation operation to a family of
operations from processes to names. Given a context, $M$, we can
define a \emph{nominal context}, $\quotep{M}$ by $\quotep{M}[P] :=
\quotep{M[P]}$. To foreshadow what is to come we observe that these
operations enjoy a duality with processes very much like the duality
between vectors and maps from vectors to scalars.

Further, because the calculus is essentially higher-order, we have a
correspondence between contexts and processes. More specifically,
given a name $x$ and a context $M$ we can construct $M^{*}_{x}$ such
that 

\begin{mathpar}
  M^{*}_{x} | \lift{x}{P} \red M[P]
\end{mathpar}

namely,

\begin{mathpar}
  M^{*}_{x} := x?(u).M[\dropn{u}]
\end{mathpar}

The dependence of $M^{*}_{x}$ on a name makes it an abstraction, 

\begin{mathpar}
  M^{*} := (x)x?(u).M[\dropn{u}]
\end{mathpar}

\subsection{Additional notation}

It will sometimes be convenient to denote the process a name
quotes. We already have the notation $x = \quotep{P}$, but it will be
convenient to introduce an alternate notation, $\procn{x}$, when we
want to emphasize the connection to the use of the name. Note that, by
virtue of name equivalence, $\quotep{\procn{x}} \nameeq x$; so, the
notation is consistent with previous definitions.

Further, because names have structure it is possible to effect
substitutions on the basis of that structure. This means we need to
upgrade our notation for substitutions, which we accomplish by
adapting comprehension notation. Thus,

\begin{mathpar}
  P\{ y / x : x \in S \}
\end{mathpar}

is interpreted to mean the process derived from P by replacing (in a
capture-avoiding manner) each occurrence of $x$ in $S$ by $y$. For example,

\begin{mathpar}
  P\{ \quotep{\procn{x}|\procn{x}} / x : x \in \freenames{P} \}
\end{mathpar}

will replace each (occurrence) of a free name $x$ in $P$ by
$\quotep{\procn{x}|\procn{x}}$.

Also, we will avail ourselves of the notation $x^{L}$ and $x^{R}$ to
denote injections of a name into disjoint copies of the name
space. There are numerous ways to accomplish this. One example can be
found in \cite{MeredithR05}. This notation overloads to vectors of
names: $\vec{x}^{\pi} := (x_{i}^{\pi} \; : \; 0 \leq i < |\vec{x}| )$ where $\pi \in \{L,R\}$.

We also use $P^{\Box} := P|\Box$.

In \cite{MeredithR05} an interpretation of the new operator is
given. It turns out that there are several possible interpretations
all enjoying the requisite algebraic properties of the operator (see
\cite{milner91polyadicpi}). We will therefore make liberal use of
$(\nu\; \vec{x})P$.

% subsection the_syntax_and_semantics_of_the_notation_system (end)   

\input{qm2pi.qmops} 

\input{qm2pi.sterngerlach} 

\input{qm2pi.metric} 

% section concurrent_process_calculi (end)

%\input{qm2pi.proofsketch}

% section proof sketch (end)

%\input{qm2pi.slviaknots} 

% section spatial logic via knots (end)

\input{qm2pi.conclusion}

% section conclusion (end)

%\input{qm2pi.dtcodes} 

% section wiring algorithm (end)

\input{qm2pi.ack} 

% section acknowledgments (end)

\newpage


\bibliographystyle{plain}   
\bibliography{../../biblios/main.bib}

\input{qm2pi.rhodetails}

\end{document}



% section proof sketch (end)

%\section{Unlikely characters: spatial logic for
  knots}\label{sub:characteristic_formulae} % (fold)

Associated to the mobile process calculi are a family of logics known
as the Hennessy-Milner logics. These logics typically enjoy a
semantics interpreting formulae as sets of processes that when
factored through the encoding outlined above allows an identification
of classes of knots with logical formulae. In the context of this
encoding the sub-family known as the spatial logics \cite{CairesC03}
\cite{CairesC04} \cite{Caires04} are of particular interest providing
several important features for expressing and reasoning about
properties (i.e. classes) of knots. We hint here at how this may be done.

%\begin{description}
%\item [structural connectives] 
\subsubsection{Structural connectives} The spatial logics enjoy
structural connectives corresponding, at the logical level, to the
parallel composition ($P | Q$) and new name ($(\nu \; x)P$)
connectives for processes. As illustrated in the examples below, these
connectives are extremely expressive given the shape of our encoding.
%\item [decideable satisfaction]

\subsubsection{Decideable satisfaction}
In \cite{Caires04} the satisfaction relation is shown to be decideable
for a rich class of processes. It further turns out that the image of
the our encoding is a proper subset of that class. This result
provides the basis for an algorithm by which to search for knots
enjoying a given property.
%\item [characteristic formulae]

\subsubsection{Characteristic formulae}
In the same paper \cite{Caires04} , Caires presents a means of calculating
characteristic formulae, selecting equivalence classes of processes
up to a pre--specified depth limit on the support set of names. Composed with our
encoding, this characteristic formula can be used to select
characteristic formulae for knots.
%\end{description}

\subsubsection{Spatial logic formulae}

The grammar below (segmented for comprehension) summarizes the syntax
of spatial logic formulae. We employ illustrative examples in the
sequel to provide an intuitive understanding of their meaning
referring the reader to \cite{Caires04} for a more detailed explication
of the semantics.

\begin{mathpar}
  \inferrule* [lab=boolean] {} {{A,B} \bc T \;|\; \neg A \;|\; A \wedge B \;|\; \eta = \eta'}
  \and
  \inferrule* [lab=spatial] {} {|\; \pzero \;|\; A | B \;|\; x \text{\textregistered} A \;|\; \forall x . A \;|\;  H x . A}
  \and
  \inferrule* [lab=behavioral] {} {|\; \alpha . A}
  \and 
  \inferrule* [lab=recursion] {} {|\; X(\vec{u}) \;|\; \mu X(\vec{u}) . A}
  \and
  \inferrule* [lab=action] {} {\alpha \bc \langle x?(\vec{y}) \rangle \;|\; \langle x!(\vec{y}) \rangle \;|\; \langle \tau \rangle}
  \and 
  \inferrule* [lab=name] {} {\eta \bc x \;|\; \tau}
\end{mathpar} 

% subsection characteristic_formulae (end)   	 

\subsection{Example formulae}\label{sub:example_formulae_} % (fold)

\subsubsection{Crossing as formula.}
% 
% \begin{align*}
%   \frac{d}{dx} \sin x &= \cos x 
%   & \frac{d}{dx} e^x &= e^x \\
%   \frac{d}{dx} \cos x &= - \sin x 
%   & \frac{d}{dx} \log x &= \frac{1}{x} \\
% \end{align*} 

\begin{align*}
 \mu C(x_{0},x_{1},y_{0},y_{1},u).&(\langle x_{0}?(z) \rangle(\langle u! \rangle\langle y_{1}!z \rangle C(x_{0},x_{1},y_{0},y_{1},u)) & \\
  & \wedge \langle y_{1}?(z) \rangle (\langle u! \rangle \langle x_{0}!z \rangle C(x_{0},x_{1},y_{0},y_{1},u)) & \\
  & \wedge \langle x_{1}?(z) \rangle (\langle u? \rangle \langle y_{0}!z \rangle C(x_{0},x_{1},y_{0},y_{1},u)) & \\
  & \wedge \langle y_{0}?(z) \rangle (\langle u? \rangle \langle x_{1}!z \rangle C(x_{0},x_{1},y_{0},y_{1},u))) &
\end{align*}

The lexicographical similarity between the shape of this formulae and
the shape of definition of the process representing a crossing reveals
the intuitive meaning of this formulae. It describes the capabilities
of a process that has the right to represent a crossing. For example
it picks out processes that may perform an input on the port $x_0$ in
its initial menu of capabilities. What differentiates the formula
from the process, however, is that the crossing process is the
smallest candidate to satisfy the formula. Infinitely many other
processes -- with internal behavior hidden behind this interface, so
to speak -- also satisfy this formula. Even this simple formula,
then, can be seen to open a new view onto knots, providing a
computational interpretation of \emph{virtual} knots.

Note that this formula is derived by hand. A similar formula can be
derived by employing Caires' calculation of characteristic formula
\cite{Caires04} to the process representing a crossing. In light of
this discussion, we let
$\meaningof{C}_{\phi}(x0,x1,y0,y1,u)$ denote a formula specifying the
dynamics we wish to capture of a crossing. To guarantee we preserve
the shape of the interface and minimal semantics we demand that
$\meaningof{C}_{\phi}(x0,x1,y0,y1,u) \Rightarrow
\textbf{C}(x0,x1,y0,y1,u)$ where $\textbf{C}(x0,x1,y0,y1,u)$ denotes
the formula above.
                            
\subsubsection{Crossing number constraints.}
The moral content of the context lemma (Lemma \ref{context}) is that the notion of
``locality'' in the Reidemeister moves is effectively captured by the
parallel composition operator of the process calculus. This intuition
extends through the logic. Given a formula,
$\meaningof{C}_{\phi}(x0,x1,y0,y1,u)$, we can use the structural
connectives to specify constraints on crossing numbers, such as at
least $n$ crossings, or exactly $n$ crossings.
\begin{mathpar}
  \inferrule* [lab=at-least-n] {} { K^{\geq n}_{\phi}(\vec{xs},\vec{ys}) := \Pi_{i=0}^{n-1} Hu . \meaningof{C}_{\phi}(xs_i,ys_i,u) | T }
  \and 
  \inferrule* [lab=exactly-n] {} { K^{= n}_{\phi}(\vec{xs},\vec{ys}) := \Pi_{i=0}^{n-1} Hu . \meaningof{C}_{\phi}(xs_i,ys_i,u) | \neg (\forall x_0,y_0,x_1,y_1,u . \meaningof{C}_{\phi}(x_0,y_0,x_1,y_1,u) | T) }
\end{mathpar}

To round out this section, recall that the encoding of an $n$-crossing
knot decomposes into a parallel composition of $n$ \emph{copies} of a
crossing process together with a wiring harness. To specify different
knot classes with the same crossing number amounts to specifying
logical constraints on the wiring harness. In the interest of space,
we defer examples to a forthcoming paper. Suffice it to say that both
the conditions ``alternating knot'' and ``contains the tangle
corresponding to 5/3'' are expressible. For example, it is possible to
calculate the characteristic formula of a process corresponding to the
tangle 5/3 and conjoin it into the classifying formula via the
composition connective of the logic.

Finally, we wish to observe that it is entirely within reason to
contemplate a more domain-specific version of spatial logic tailored
to the shape of processes in the image of the encoding. Such a
domain-specific logic would have a better claim to the title formal
language of knot properties.

% subsection example_formulae_ (end)

% section knots_as_processes (end) 

% section spatial logic via knots (end)

\section{Conclusions and future work}

\paragraph{Testing physical space}
You, gentle reader, may wonder why of all the theorems to be proved
given this set up we pick the one above. In some sense it's hardly
central to quantum mechanics. We see it as central in the sense that
it firmly establishes a notion of physical space arising from a notion
of the equivalence of behavior. Relating bisimulation to a metric is a
big step forward, but one is faced with interpreting the relationship
of that metric space to something more physical. Quantum mechanical
notions of ``physical'' space are still far from intuitive, but by
relating this idea of distance as testing to calculations that predict
physical circumstances we are making a not insignificant step forward
toward an understanding of the physical space we inhabit as
essentially dynamic.

\paragraph{Effectivity and simulation}
One of the observations we have yet to make is that the entire program
spelled out here is effective. We have built various interpreters for
the reflective calculus at work in this interpretation. In principle,
then, we can simulate quantum mechanics on a computer. The place where
the simulation may lose fidelity is the infinitely branching summation
for the annihilator.

In this connection i also want to point out that the evaluation style
calculation of the inner product puts the non-determinism of the
summation right at the heart of measurement. This suggests that
Milner's original reduction-based formulation of the dynamics of his
calculi in terms of sums was not just notationally suggestive of a
notion of measure-and-continue but captured some significant part of
the physics.

\paragraph{Quantum continuations}
In light of this last observation i want to point out that the
predominant account of quantum mechanics is missing a key aspect of a
truly compositional story of the physical situation. In a real lab,
when a measurement is made the observation can be made to feed into
another device that then makes another measurement conditioned on the
results of the first. This means that after the superposition was
collapsed the entire experimental set up remained in
superposition. While QM offers a means of writing this down it doesn't
quite line up well with the well-trodden formulation of computation
and continuation that we see so succinctly expressed in Milner's
calculi. This suggests that there might be advantages to this account
of dynamics waiting to be explored.

\paragraph{Quantum logic}
In this connection, we also note that by virtue of having the
Hennessy-Milner construction, we can pull the construction through the
interpretation of QM. This gives us a natural candidate for a quantum
logic that enjoys an extremely tight connection with it's domain of
interpretation, making the construction much less ad hoc (rather it is
the image of functor!).

\paragraph{Quantum probabiity}
i have questions about the basis of the interpretation of inner
product as probability amplitude. In particular, using which
axiomatization of probability theory does the notion of probability
amplitude earn the right to be so dubbed? In other words, where is the
proof that the operation for calculating a probability amplitude (and
then squaring) satisfies the axioms of what it means to calculate a
probability? Even if such a proof exists (i have yet to find it in the
literature), i wonder if it might not be possible to turn things on
their heads. Can we view the calculation of the probability amplitude
as an axiomatization of probability? If so, then the definition we
give for calculating probability amplitude may provide the basis for
an \emph{effective} theory of probability.

\paragraph{Quantum vs ``biological'' information}
Finally, i want to conclude with a more philosophical observation. At
a recent workshop in which QM was a predominant topic i noticed
something about quantum information. The speaker was giving a riveting
discussion of axiomatic QM and showing how properties of ``no
cloning'' and ``no deleting'' emerged as consequences of the
axiomatization. Theorems of this form are necessary to give us a sense
of confidence that our axioms characterize the physical theory. What
struck me, though, was that if quantum information is neither erasable
nor replicable it is markedly different from \emph{life}. Two of the
things we know about life is that

\begin{itemize}
  \item it ends;
  \item to gain some measure of persistence, to transcend it's
    finitude it is imminently copyable.
\end{itemize}

Both of these qualities are summarized succinctly in the aphorism: all
flesh is grass. For me these two kinds of ``information'' -- call them
quantum and biological -- are end points on a spectrum of strategies
for persistence. At one end, we have those curious entities that enjoy
uniqueness and permanence; at the other, we have those who in the face
of a certain end and an uncertain present make a go of passing
something on. To me one of the more remarkable aspects of the latter
strategy is that in the presence of noise (and certain features of
copying) we get a kind of dynamism, a chance for improvement against a
given persistent condition.

% subsection other_calculi_other_bisimulations_and_geometry_as_behavior (end)




% section conclusion (end)

%\documentclass[12pt]{llncs}
%\documentclass{jktr}

\usepackage[pdftex]{hyperref}                   
\usepackage {listings}
\usepackage {mathpartir}
\usepackage{bcprules}
%\usepackage{listings}
                       
\usepackage{graphicx} 
%\usepackage[margins=2.5cm,nohead,nofoot]{geometry}
%\usepackage{geometry}
\usepackage{amsfonts}
\usepackage{amstext}
\usepackage{latexsym}
\usepackage{amssymb}
\usepackage{color}


%\include{myPreamble}
\include{qm2pi.local} 

%\ifpdf
%\usepackage[pdftex]{graphicx}
%\else
%\usepackage{graphicx}
%\fi

 % \ifpdf
%  \usepackage{pdfsync}
%  \if


%\title{Brief Article}
%\author{David F. Snyder}
%\author{L.G. Meredith}

%\address{Dept. of Math., Texas State University--San Marcos, San Marcos, TX 78666}
       
\pagestyle{empty}


\begin{document}

\lstset{language=[Objective]Caml,frame=shadowbox}

\input{qm2pi.front}

% section front matter (end)

\input{qm2pi.intro} 
 
% section introduction (end)

% \input{qm2pi.knotations} 

% section notation (end)

\input{qm2pi.process.calculi} 

% section concurrent_process_calculi_and_spatial_logics_ (end)
    
%\input{qm2pi.knots2pi} 

%\input{qm2pi.trefoil} 

%\input{qm2pi.mainthm} 

% subsection basic_interpretation (end)

%\input{qm2pi.rho.presentation} 
\subsection{The syntax and semantics of the notation system}\label{sub:the_syntax_and_semantics_of_the_notation_system} % (fold)

We now summarize a technical presentation of the calculus that
embodies our theory of dynamics. The typical presentation of such a
calculus follows the style of giving generators and relations on
them. The grammar, below, describing term constructors, freely
generates the set of processes, $\Proc$. This set is then quotiented
by a relation known as structural congruence and it is over this set
that the notion of dynamics is expressed. This presentation is
essentially that of \cite{MeredithR05} with the addition of
polyadicity and summation. For readability we have relegated some of
the technical subtleties to an appendix.

\subsubsection{Process grammar}\label{subsub:process_grammar}

\begin{mathpar}
  \inferrule* [lab=synchronization] {} {{M} \bc \pzero \;|\; x?F \;|\; x!C }
  \and
  \inferrule* [lab=abstraction] {} {{F} \bc (x)P}
  \and
  \inferrule* [lab=concretion] {} {{C} \bc \langle Q \rangle}
  \and
  \inferrule* [lab=process] {} {{P,Q} \bc M \;| \;P|Q \;|\; @{x}}
  \and
  \inferrule* [lab=name] {} {{x} \bc \quotep{P}}
\end{mathpar} 

Note that $\vec{x}$ (resp. $\vec{P}$) denotes a vector of names
(resp. processes) of length $|\vec{x}|$ (resp. $|\vec{P}|$). We adopt
the following useful abbreviations.

\begin{mathpar}
   x?(\vec{y}).P := x.(\vec{y})P \and  x\clift{\vec{P}} := x.\clift{\vec{P}}
   \and x!(y) := \lift{x}{\dropn{y}}
   \and \Pi_{i=0}^{n-1}P_i := P_0 | \ldots | P_{n-1}
\end{mathpar}

\subsubsection{Structural congruence}

\paragraph{Free and bound names and alpha-equivalence.} At the
core of structural equivalence is alpha-equivalence which identifies
process that are the same up to a change of variable. Formally, we
recognize the distinction between free and bound names. The free names
of a process, $\freenames{P}$, may be calculated recursively as
follows:

\begin{mathpar}
\freenames{\pzero} := \emptyset
  \and \\
  \freenames{x?(y).P} := \{ x \} \cup (\freenames{P} \setminus \{ y \})
  \and 
  \freenames{x!\langle P \rangle} := \{ x \} \cup \{ P \} 
  \and \\
  \freenames{P|Q} := \freenames{P} \cup \freenames{Q}
  \and \\
  \freenames{@{x}} := \{ x \}
\end{mathpar}

$\pi$
$\quotep{\pi}$

$\freenames{-} : \pi \to \mathcal{P}(\quotep{\pi})$

\begin{eqnarray*}
  \freenames{\pzero} & := & \emptyset \\
  \freenames{x?(y).P} & := & \{ x \} \cup (\freenames{P} \setminus \{ y \}) \\
  \freenames{x!\langle P \rangle} & := & \{ x \} \cup \{ P \} \\
  \freenames{P|Q} & := & \freenames{P} \cup \freenames{Q} \\
  \freenames{\dropn{x}} & := & \{ x \}
\end{eqnarray*}

The bound names of a process, $\boundnames{P}$, are those names occurring in $P$
that are not free. For example, in $x?(y).0$, the name $x$ is free, while $y$ is bound.

\begin{mathpar}
  \inferrule* [lab=monoidal-laws] {} { P|Q \equiv Q|P \and P|0 \equiv P \and P|(Q|R) \equiv (P|Q)|R }
\end{mathpar}

\begin{mathpar}
  \inferrule* [lab=alpha-equivalence] {} { (x)P \equiv (y)P\{y/x\} \and y \not\in \freenames{P} }
\end{mathpar}

\begin{definition}
Then two processes, $P,Q$, are alpha-equivalent if $P = Q\{\vec{y}/\vec{x}\}$ for
some $\vec{x} \in \boundnames{Q},\vec{y} \in \boundnames{P}$, where $Q\{\vec{y}/\vec{x}\}$
denotes the capture-avoiding substitution of $\vec{y}$ for $\vec{x}$ in $Q$.
\end{definition}

\begin{definition}
  The {\em structural congruence} \cite{SangiorgiWalker} , $\equiv$,
  between processes is the least congruence containing
  alpha-equivalence, satisfying the abelian monoid laws
  (associativity, commutativity and $\pzero$ as identity) for parallel
  composition $|$ and for summation $+$.
\end{definition}

\subsection{Name equivalence}

We take name equivalence, written $\nameeq$, to be the smallest
equivalence relation generated by the following rules.

\begin{mathpar}
\inferrule*[lab=Quote-drop]
{ }
{ \quotep{@{x}} \nameeq x }

\inferrule*[lab=Struct-equiv]
{ P \scong Q }
{ \quotep{P} \nameeq \quotep{Q} }
\end{mathpar}

The astute reader will have noticed that the mutual recursion of names
and processes imposes a mutual recursion on alpha-equivalence and
structural equivalence via name-equivalence. Fortunately, all of this
works out pleasantly and we may calculate in the natural way, free of
concern. The reader interested in the details is referred to the
appendix \ref{appendix:rho_details}.

\subsection{Substitution}

We use $\Proc$ for the set of processes, $\QProc$ for the set of
names, and $\id{\{}\vec{y} / \vec{x} \id{\}}$ to denote partial maps,
$s : \QProc \rightarrow \QProc$. A map, $s$ lifts, uniquely, to a map
on process terms, $\widehat{s} : \Proc \rightarrow \Proc$ by the
following equations.

\begin{mathpar}
  (0) \psubstp{Q}{P} := 0 \\
  (R \juxtap S) \psubstp{Q}{P}
  :=    
  (R)\psubstp{Q}{P} \juxtap (S) \psubstp{Q}{P} \\
  (x?(y).R) \psubstp{Q}{P}    
  :=    
  (x)\substp{Q}{P} (z)\concat( (R \psubstn{z}{y}) \psubstp{Q}{P} ) \\
  (\lift{x}{R}) \psubstp{Q}{P}  
  :=
  \lift{(x)\substp{Q}{P}}{ R \psubstp{Q}{P} } \\
%   (\dropn{x})  \psubstp{Q}{P}       
%   := 
%   \left\{ 
%     \begin{array}{ccc} 
%       \dropn{\quotep{Q}} & & x \nameeq \quotep{P} \\
%       \dropn{x} & & otherwise \\
%     \end{array}
%   \right. 
  (\dropn{x})  \psubstp{Q}{P}       
  := 
  \left\{ 
    \begin{array}{ccc} 
      Q & & x \nameeq \quotep{P} \\
      \dropn{x} & & otherwise \\
    \end{array}
  \right.
\end{mathpar}
 

where

\begin{eqnarray}
  (x)\id{\{} \lpquote Q \rpquote / \lpquote P \rpquote \id{\}}            = 
  \left\{ 
    \begin{array}{ccc}
      \lpquote Q \rpquote & & x \nameeq \lpquote P \rpquote \\
      x & & otherwise \\
    \end{array}
  \right. \nonumber
\end{eqnarray}

and $z$ is chosen distinct from $\quotep{P}$, $\quotep{Q}$, the free
names in $Q$, and all the names in $R$. Our $\alpha$-equivalence will
be built in the standard way from this substitution.

\begin{remark}\label{rem:no_self_referential_names}
  One consequence of these definitions is that $\forall P. \quotep{P}
  \not\in \freenames{P}$.
\end{remark}

\subsection{ Dynamic quote: an example }

Anticipating something of what's to come, consider applying the
substitution, $\widehat{\id{\{}u / z \id{\}}}$, to the following pair
of processes, $\lift{w}{y!(z)}$ and $w[ \lpquote y!(z) \rpquote ]$.

\begin{eqnarray}
	\lift{w}{y!(z)}\widehat{\id{\{}u / z \id{\}}}
		& = &
		\lift{w}{y!(u)} \nonumber\\
	w[ \lpquote y!(z) \rpquote ] \widehat{ \id{\{}u / z \id{\}} }
		& = &
		w[ \lpquote y!(z) \rpquote ] \nonumber
\end{eqnarray}

Because the body of the process between quotes is impervious to
substitution, we get radically different answers. In fact, by
examining the first process in an input context,
e.g. $x?(z).\lift{w}{y!(z)}$, we see that the process under the lift
operator may be shaped by prefixed inputs binding a name inside it. In
this sense, the lift operator will be seen as a way to dynamically
construct processes before reifying them as names.

Finally equipped with these standard features we can present the
dynamics of the calculus.

\subsubsection{Operational semantics} 

Finally, we introduce the computational dynamics. What marks these
algebras as distinct from other more traditionally studied algebraic
structures, e.g. vector spaces or polynomial rings, is the manner in
which dynamics is captured. In traditional structures, dynamics is typically
expressed through morphisms between such structures, as in linear maps
between vector spaces or morphisms between rings. In algebras
associated with the semantics of computation, the dynamics is
expressed as part of the algebraic structure itself, through a
reduction reduction relation typically denoted by $\red$. Below, we
give a recursive presentation of this relation for the calculus used
in the encoding.

$\red \subseteq \pi \times \pi$
$\red : \pi \to \mathcal{P}(\pi)$

\begin{mathpar}
  \inferrule* [lab=Comm] { \textsf{match}( x_{src}, x_{trgt} ) } { x_{trgt}?(y)P \; | \; x_{src}!\langle {Q} \rangle \red P\{\quotep{Q}/y}\} }
  \and \\
  \inferrule* [lab=Par] {{P} \red {P}'} {{{P} | {Q}} \red {{P}' | {Q}}}
  \and
  \inferrule* [lab=Equiv]{{{P} \scong {P}'} \andalso {{P}' \red {Q}'} \andalso {{Q}' \scong {Q}}}{{P} \red {Q}}
\end{mathpar}

\begin{eqnarray*}
  match_{\equiv} (\quotep{P},\quotep{Q}) & := & P \equiv Q \\
  match_{\dagger}(\quotep{P},\quotep{Q}) & := & \forall R. P|Q \red^{*} R => R \red^{*} 0 \\
  match_{K}(\quotep{P},\quotep{Q}) & := & K \mbox{ for some context } K
\end{eqnarray*}

$u?(x)P | u!\langle Q \rangle \red P\{\quotep{Q}/x\}$

%We write $\wred$ for $\red^*$, and $P\red$ if $\exists Q $ such that $ P \red Q$.
We write $P\red$ if $\exists Q $ such that $ P \red Q$ and $P\not\red$, otherwise.

\section{Replication}

As mentioned before, it is known that replication (and hence
recursion) can be implemented in a higher-order process algebra
\cite{SangiorgiWalker}. As our first example of calculation with the
machinery thus far presented we give the construction explicitly in
the {\rhoc}.

\begin{eqnarray}
	D_{x} & := & \prefix{x}{y}{(\binpar{\outputp{x}{y}}{@{y}})} \nonumber\\
	\bangp_{x}{P} & := & \binpar{{x}!\langle{\binpar{D_{x}}{P}}\rangle}{D_{x}} \nonumber
\end{eqnarray}

\begin{eqnarray}
	\bangp_{x}{P} & & \nonumber\\
	=
	& {x}!\langle{(\prefix{x}{y}{(\outputp{x}{y} | @{y})) | P}}\rangle 
	      | \prefix{x}{y}{(\outputp{x}{y} | @{y})} & \nonumber\\
	\red
	& (\outputp{x}{y} | @{y})\substn{\quotep{(\prefix{x}{y}{(@{y} | \outputp{x}{y})) | P}}}{y} & \nonumber\\
	=
	& \outputp{x}{\quotep{(\prefix{x}{y}{(\outputp{x}{y} | @{y})) | P}}}
	  | {(\prefix{x}{y}{(\outputp{x}{y} | @{y})) | P}} & \nonumber\\
	\red
	& \ldots & \nonumber\\
	\red^*
	& P | P | \ldots & \nonumber
\end{eqnarray}

Of course, this encoding, as an implementation, runs away, unfolding
$\bangp{P}$ eagerly. A lazier and more implementable replication
operator, restricted to input-guarded processes, may be obtained as follows.

\begin{eqnarray}
\bangp{\prefix{u}{v}{P}} 
	:= 
	\binpar{\lift{x}{\prefix{u}{v}{(\binpar{D(x)}{P})}}}{D(x)} \nonumber
\end{eqnarray}

\begin{remark}
  Note that the lazier definition still does not deal with summation
  or mixed summation (i.e. sums over input and output). The reader is
  invited to construct definitions of replication that deal with these
  features. 

  Further, the definitions are parameterized in a name, $x$. Can you,
  gentle reader, make a definition that eliminates this parameter and
  guarantees no accidental interaction between the replication
  machinery and the process being replicated -- i.e. no accidental
  sharing of names used by the process to get its work done and the
  name(s) used by the replication to effect copying. This latter
  revision of the definition of replication is crucial to obtaining
  the expected identity $!!P \sim !P$.
\end{remark}

\begin{remark}\label{rem:paradoxical_combinator}
  The reader familiar with the lambda calculus will have noticed the
  similarity between $D$ and the paradoxical combinator.

  [Ed. note: the existence of this seems to suggest we have to be more
  restrictive on the set of processes and names we admit if we are to
  support no-cloning.]
\end{remark}

\subsubsection{Bisimulation}

The computational dynamics gives rise to another kind of equivalence,
the equivalence of computational behavior. As previously mentioned
this is typically captured \emph{via} some form of bisimulation.

% The notion we use in this paper is weak barbed bisimulation
% \cite{milner91polyadicpi}.

The notion we use in this paper is derived from weak barbed
bisimulation \cite{milner91polyadicpi}. 

\begin{definition}
An \emph{observation relation}, $\downarrow_{\mathcal N}$, over a set
of names, $\mathcal N$, is the smallest relation satisfying the rules
below.

\infrule[Out-barb]{y \in {\mathcal N}, \; x \nameeq y}
		  {\outputp{x}{v} \downarrow_{\mathcal N} x}
\infrule[Par-barb]{\mbox{$P\downarrow_{\mathcal N} x$ or $Q\downarrow_{\mathcal N} x$}}
		  {\binpar{P}{Q} \downarrow_{\mathcal N} x}

We write $P \Downarrow_{\mathcal N} x$ if there is $Q$ such that 
$P \wred Q$ and $Q \downarrow_{\mathcal N} x$.
\end{definition}

\begin{definition}
%\label{def.bbisim}
An  ${\mathcal N}$-\emph{barbed bisimulation} over a set of names, ${\mathcal N}$, is a symmetric binary relation 
${\mathcal S}_{\mathcal N}$ between agents such that $P\rel{S}_{\mathcal N}Q$ implies:
\begin{enumerate}
\item If $P \red P'$ then $Q \wred Q'$ and $P'\rel{S}_{\mathcal N} Q'$.
\item If $P\downarrow_{\mathcal N} x$, then $Q\Downarrow_{\mathcal N} x$.
\end{enumerate}
$P$ is ${\mathcal N}$-barbed bisimilar to $Q$, written
$P \wbbisim_{\mathcal N} Q$, if $P \rel{S}_{\mathcal N} Q$ for some ${\mathcal N}$-barbed bisimulation ${\mathcal S}_{\mathcal N}$.
\end{definition}

$\mathcal{R} \subseteq \pi \times \pi$

$P \mathcal{R} Q => \forall P'. P \red P' \Rightarrow \exists Q'. Q \red Q', P' \mathcal{R} Q'$

$P \vdash x \Rightarrow Q \vdash x$

\begin{mathpar}
  \inferrule*[lab=Out-barb]{x \nameeq y}{{y}!\langle{Q}\rangle \vdash x}
  \and
  \inferrule*[lab=Par-barb]{\mbox{$P\vdash x$ or $Q\vdash x$}}{\binpar{P}{Q} \vdash x}
\end{mathpar}

\subsubsection{Contexts}

One of the principle advantages of computational calculi like the
$\pi$-calculus is a well-defined notion of context,
contextual-equivalence and a correlation between
contextual-equivalence and notions of bisimulation. The notion of
context allows the decomposition of a process into (sub-)process and
its syntactic environment, its context. Thus, a context may be
thought of as a process with a ``hole'' (written $\Box$) in it. The
application of a context $M$ to a process $P$, written $M[P]$, is
tantamount to filling the hole in $M$ with $P$. In this paper we do
not need the full weight of this theory, but do make use of the notion
of context in the proof the main theorem. 

\begin{mathpar}
  \inferrule* [lab=summation] {} {{M_{M},M_{N}} \bc \Box \;|\; x.M_{A} \;|\; M_{M}+M_{N}}
  \and
  \inferrule* [lab=agent] {} {{M_{A}} \bc (\vec{x})M_{P} \;| \; \clift{P_0,\ldots,M_{P},\ldots,P_N}}
  \and \\
  \inferrule* [lab=process] {} {{M_{P}} \bc M_{N} \;| \;P|M_{P} }
\end{mathpar} 

\begin{mathpar}
  \inferrule* [lab=sychronization] {} {M_{N} \bc \Box \;|\; x?M_{F} \;|\; x!M_{C}}
  \and
  \inferrule* [lab=abstraction] {} {{M_{F}} \bc (x)M_{P} }
  \and
  \inferrule* [lab=concretion] {} {{M_{C}} \bc \langle M_{P} \rangle }
  \and \\
  \inferrule* [lab=process] {} {{M_{P}} \bc M_{N} \;| \;P|M_{P} }
\end{mathpar}

\begin{definition}[contextual application] Given a context $M$, and
  process $P$, we define the \emph{contextual application}, $M[P] :=
  M\{P/\Box\}$. That is, the contextual application of M to P is the
  substitution of $P$ for $\Box$ in $M$.
\end{definition}

$\meaningof{-} : L \to \mathcal{P}(\pi)$

\begin{mathpar}
  \inferrule* [lab=collection] {} {\meaningof{true} = \pi, \and \meaningof{~E} = \pi \setminus \meaningof{E}, \and \meaningof{E_{1} \& E_{2}} = \meaningof{E_{1}} \cap \meaningof{E_{2}}}
\end{mathpar}

\begin{mathpar}
  \inferrule* [lab=structure] {} {\meaningof{0} = \{ P \in \pi | P \equiv 0 \}, \and \\ \meaningof{E_1 | E_2} = \{ P \in \pi | P \equiv P_{1} | P_{2}, P_{1} \in \meaningof{E_{1}}, P_{2} \in \meaningof{E_2}\} }
\end{mathpar}

\begin{mathpar}
 \inferrule* [lab=behavior] {} {\meaningof{\langle a?b \rangle E} = \{ P \in \pi | P \equiv Q | u?(y)P', \\ \and \\\\ \and \\ \;\;\; u \in \meaningof{a}, \forall z.P'\{z/y\} \in \meaningof{E\{z/b\}}\}, \and \\ \meaningof{a!E} = \{ P \in \pi | P \equiv Q | x!\langle P' \rangle, x \in \meaningof{a} P' \in \meaningof{E}\} }
\end{mathpar}

\begin{mathpar}
 \inferrule* [lab=nominal] {} {\meaningof{\quotep{E}} = \{ \quotep{P} \in \quotep{\pi} | P \in \meaningof{E} \}, \and \meaningof{\quotep{P}} = \{ \quotep{Q} \in \quotep{\pi} | P \equiv Q \} \and \\ \meaningof{@\quotep{E}} = \{ P \in \pi | P \equiv @x, x \in \meaningof{E} \}}
\end{mathpar}

\begin{eqnarray*}
  \\
  \meaningof{-} : TS \to ST
\end{eqnarray*}

\begin{eqnarray*}
  \\
  L : TS \to ST
\end{eqnarray*}

\begin{eqnarray*}
  \\
  P \models E \iff P \in \meaningof{E}
\end{eqnarray*}

\begin{eqnarray*}
  P \approx_{L} Q \iff \forall E \in L. P \models E \iff Q \models E
\end{eqnarray*}

\begin{eqnarray*}
  P \approx_{K} Q
\end{eqnarray*}

\begin{eqnarray*}
  P \approx Q
\end{eqnarray*}

$\approx_{K} = \approx = \approx_{L}$

\subsubsection{Contextual duality}

Note that contexts extend the quotation operation to a family of
operations from processes to names. Given a context, $M$, we can
define a \emph{nominal context}, $\quotep{M}$ by $\quotep{M}[P] :=
\quotep{M[P]}$. To foreshadow what is to come we observe that these
operations enjoy a duality with processes very much like the duality
between vectors and maps from vectors to scalars.

Further, because the calculus is essentially higher-order, we have a
correspondence between contexts and processes. More specifically,
given a name $x$ and a context $M$ we can construct $M^{*}_{x}$ such
that 

\begin{mathpar}
  M^{*}_{x} | \lift{x}{P} \red M[P]
\end{mathpar}

namely,

\begin{mathpar}
  M^{*}_{x} := x?(u).M[\dropn{u}]
\end{mathpar}

The dependence of $M^{*}_{x}$ on a name makes it an abstraction, 

\begin{mathpar}
  M^{*} := (x)x?(u).M[\dropn{u}]
\end{mathpar}

\subsection{Additional notation}

It will sometimes be convenient to denote the process a name
quotes. We already have the notation $x = \quotep{P}$, but it will be
convenient to introduce an alternate notation, $\procn{x}$, when we
want to emphasize the connection to the use of the name. Note that, by
virtue of name equivalence, $\quotep{\procn{x}} \nameeq x$; so, the
notation is consistent with previous definitions.

Further, because names have structure it is possible to effect
substitutions on the basis of that structure. This means we need to
upgrade our notation for substitutions, which we accomplish by
adapting comprehension notation. Thus,

\begin{mathpar}
  P\{ y / x : x \in S \}
\end{mathpar}

is interpreted to mean the process derived from P by replacing (in a
capture-avoiding manner) each occurrence of $x$ in $S$ by $y$. For example,

\begin{mathpar}
  P\{ \quotep{\procn{x}|\procn{x}} / x : x \in \freenames{P} \}
\end{mathpar}

will replace each (occurrence) of a free name $x$ in $P$ by
$\quotep{\procn{x}|\procn{x}}$.

Also, we will avail ourselves of the notation $x^{L}$ and $x^{R}$ to
denote injections of a name into disjoint copies of the name
space. There are numerous ways to accomplish this. One example can be
found in \cite{MeredithR05}. This notation overloads to vectors of
names: $\vec{x}^{\pi} := (x_{i}^{\pi} \; : \; 0 \leq i < |\vec{x}| )$ where $\pi \in \{L,R\}$.

We also use $P^{\Box} := P|\Box$.

In \cite{MeredithR05} an interpretation of the new operator is
given. It turns out that there are several possible interpretations
all enjoying the requisite algebraic properties of the operator (see
\cite{milner91polyadicpi}). We will therefore make liberal use of
$(\nu\; \vec{x})P$.

% subsection the_syntax_and_semantics_of_the_notation_system (end)   

\input{qm2pi.qmops} 

\input{qm2pi.sterngerlach} 

\input{qm2pi.metric} 

% section concurrent_process_calculi (end)

%\input{qm2pi.proofsketch}

% section proof sketch (end)

%\input{qm2pi.slviaknots} 

% section spatial logic via knots (end)

\input{qm2pi.conclusion}

% section conclusion (end)

%\input{qm2pi.dtcodes} 

% section wiring algorithm (end)

\input{qm2pi.ack} 

% section acknowledgments (end)

\newpage


\bibliographystyle{plain}   
\bibliography{../../biblios/main.bib}

\input{qm2pi.rhodetails}

\end{document}

 

% section wiring algorithm (end)

\documentclass[12pt]{llncs}
%\documentclass{jktr}

\usepackage[pdftex]{hyperref}                   
\usepackage {listings}
\usepackage {mathpartir}
\usepackage{bcprules}
%\usepackage{listings}
                       
\usepackage{graphicx} 
%\usepackage[margins=2.5cm,nohead,nofoot]{geometry}
%\usepackage{geometry}
\usepackage{amsfonts}
\usepackage{amstext}
\usepackage{latexsym}
\usepackage{amssymb}
\usepackage{color}


%\include{myPreamble}
\include{qm2pi.local} 

%\ifpdf
%\usepackage[pdftex]{graphicx}
%\else
%\usepackage{graphicx}
%\fi

 % \ifpdf
%  \usepackage{pdfsync}
%  \if


%\title{Brief Article}
%\author{David F. Snyder}
%\author{L.G. Meredith}

%\address{Dept. of Math., Texas State University--San Marcos, San Marcos, TX 78666}
       
\pagestyle{empty}


\begin{document}

\lstset{language=[Objective]Caml,frame=shadowbox}

\input{qm2pi.front}

% section front matter (end)

\input{qm2pi.intro} 
 
% section introduction (end)

% \input{qm2pi.knotations} 

% section notation (end)

\input{qm2pi.process.calculi} 

% section concurrent_process_calculi_and_spatial_logics_ (end)
    
%\input{qm2pi.knots2pi} 

%\input{qm2pi.trefoil} 

%\input{qm2pi.mainthm} 

% subsection basic_interpretation (end)

%\input{qm2pi.rho.presentation} 
\subsection{The syntax and semantics of the notation system}\label{sub:the_syntax_and_semantics_of_the_notation_system} % (fold)

We now summarize a technical presentation of the calculus that
embodies our theory of dynamics. The typical presentation of such a
calculus follows the style of giving generators and relations on
them. The grammar, below, describing term constructors, freely
generates the set of processes, $\Proc$. This set is then quotiented
by a relation known as structural congruence and it is over this set
that the notion of dynamics is expressed. This presentation is
essentially that of \cite{MeredithR05} with the addition of
polyadicity and summation. For readability we have relegated some of
the technical subtleties to an appendix.

\subsubsection{Process grammar}\label{subsub:process_grammar}

\begin{mathpar}
  \inferrule* [lab=synchronization] {} {{M} \bc \pzero \;|\; x?F \;|\; x!C }
  \and
  \inferrule* [lab=abstraction] {} {{F} \bc (x)P}
  \and
  \inferrule* [lab=concretion] {} {{C} \bc \langle Q \rangle}
  \and
  \inferrule* [lab=process] {} {{P,Q} \bc M \;| \;P|Q \;|\; @{x}}
  \and
  \inferrule* [lab=name] {} {{x} \bc \quotep{P}}
\end{mathpar} 

Note that $\vec{x}$ (resp. $\vec{P}$) denotes a vector of names
(resp. processes) of length $|\vec{x}|$ (resp. $|\vec{P}|$). We adopt
the following useful abbreviations.

\begin{mathpar}
   x?(\vec{y}).P := x.(\vec{y})P \and  x\clift{\vec{P}} := x.\clift{\vec{P}}
   \and x!(y) := \lift{x}{\dropn{y}}
   \and \Pi_{i=0}^{n-1}P_i := P_0 | \ldots | P_{n-1}
\end{mathpar}

\subsubsection{Structural congruence}

\paragraph{Free and bound names and alpha-equivalence.} At the
core of structural equivalence is alpha-equivalence which identifies
process that are the same up to a change of variable. Formally, we
recognize the distinction between free and bound names. The free names
of a process, $\freenames{P}$, may be calculated recursively as
follows:

\begin{mathpar}
\freenames{\pzero} := \emptyset
  \and \\
  \freenames{x?(y).P} := \{ x \} \cup (\freenames{P} \setminus \{ y \})
  \and 
  \freenames{x!\langle P \rangle} := \{ x \} \cup \{ P \} 
  \and \\
  \freenames{P|Q} := \freenames{P} \cup \freenames{Q}
  \and \\
  \freenames{@{x}} := \{ x \}
\end{mathpar}

$\pi$
$\quotep{\pi}$

$\freenames{-} : \pi \to \mathcal{P}(\quotep{\pi})$

\begin{eqnarray*}
  \freenames{\pzero} & := & \emptyset \\
  \freenames{x?(y).P} & := & \{ x \} \cup (\freenames{P} \setminus \{ y \}) \\
  \freenames{x!\langle P \rangle} & := & \{ x \} \cup \{ P \} \\
  \freenames{P|Q} & := & \freenames{P} \cup \freenames{Q} \\
  \freenames{\dropn{x}} & := & \{ x \}
\end{eqnarray*}

The bound names of a process, $\boundnames{P}$, are those names occurring in $P$
that are not free. For example, in $x?(y).0$, the name $x$ is free, while $y$ is bound.

\begin{mathpar}
  \inferrule* [lab=monoidal-laws] {} { P|Q \equiv Q|P \and P|0 \equiv P \and P|(Q|R) \equiv (P|Q)|R }
\end{mathpar}

\begin{mathpar}
  \inferrule* [lab=alpha-equivalence] {} { (x)P \equiv (y)P\{y/x\} \and y \not\in \freenames{P} }
\end{mathpar}

\begin{definition}
Then two processes, $P,Q$, are alpha-equivalent if $P = Q\{\vec{y}/\vec{x}\}$ for
some $\vec{x} \in \boundnames{Q},\vec{y} \in \boundnames{P}$, where $Q\{\vec{y}/\vec{x}\}$
denotes the capture-avoiding substitution of $\vec{y}$ for $\vec{x}$ in $Q$.
\end{definition}

\begin{definition}
  The {\em structural congruence} \cite{SangiorgiWalker} , $\equiv$,
  between processes is the least congruence containing
  alpha-equivalence, satisfying the abelian monoid laws
  (associativity, commutativity and $\pzero$ as identity) for parallel
  composition $|$ and for summation $+$.
\end{definition}

\subsection{Name equivalence}

We take name equivalence, written $\nameeq$, to be the smallest
equivalence relation generated by the following rules.

\begin{mathpar}
\inferrule*[lab=Quote-drop]
{ }
{ \quotep{@{x}} \nameeq x }

\inferrule*[lab=Struct-equiv]
{ P \scong Q }
{ \quotep{P} \nameeq \quotep{Q} }
\end{mathpar}

The astute reader will have noticed that the mutual recursion of names
and processes imposes a mutual recursion on alpha-equivalence and
structural equivalence via name-equivalence. Fortunately, all of this
works out pleasantly and we may calculate in the natural way, free of
concern. The reader interested in the details is referred to the
appendix \ref{appendix:rho_details}.

\subsection{Substitution}

We use $\Proc$ for the set of processes, $\QProc$ for the set of
names, and $\id{\{}\vec{y} / \vec{x} \id{\}}$ to denote partial maps,
$s : \QProc \rightarrow \QProc$. A map, $s$ lifts, uniquely, to a map
on process terms, $\widehat{s} : \Proc \rightarrow \Proc$ by the
following equations.

\begin{mathpar}
  (0) \psubstp{Q}{P} := 0 \\
  (R \juxtap S) \psubstp{Q}{P}
  :=    
  (R)\psubstp{Q}{P} \juxtap (S) \psubstp{Q}{P} \\
  (x?(y).R) \psubstp{Q}{P}    
  :=    
  (x)\substp{Q}{P} (z)\concat( (R \psubstn{z}{y}) \psubstp{Q}{P} ) \\
  (\lift{x}{R}) \psubstp{Q}{P}  
  :=
  \lift{(x)\substp{Q}{P}}{ R \psubstp{Q}{P} } \\
%   (\dropn{x})  \psubstp{Q}{P}       
%   := 
%   \left\{ 
%     \begin{array}{ccc} 
%       \dropn{\quotep{Q}} & & x \nameeq \quotep{P} \\
%       \dropn{x} & & otherwise \\
%     \end{array}
%   \right. 
  (\dropn{x})  \psubstp{Q}{P}       
  := 
  \left\{ 
    \begin{array}{ccc} 
      Q & & x \nameeq \quotep{P} \\
      \dropn{x} & & otherwise \\
    \end{array}
  \right.
\end{mathpar}
 

where

\begin{eqnarray}
  (x)\id{\{} \lpquote Q \rpquote / \lpquote P \rpquote \id{\}}            = 
  \left\{ 
    \begin{array}{ccc}
      \lpquote Q \rpquote & & x \nameeq \lpquote P \rpquote \\
      x & & otherwise \\
    \end{array}
  \right. \nonumber
\end{eqnarray}

and $z$ is chosen distinct from $\quotep{P}$, $\quotep{Q}$, the free
names in $Q$, and all the names in $R$. Our $\alpha$-equivalence will
be built in the standard way from this substitution.

\begin{remark}\label{rem:no_self_referential_names}
  One consequence of these definitions is that $\forall P. \quotep{P}
  \not\in \freenames{P}$.
\end{remark}

\subsection{ Dynamic quote: an example }

Anticipating something of what's to come, consider applying the
substitution, $\widehat{\id{\{}u / z \id{\}}}$, to the following pair
of processes, $\lift{w}{y!(z)}$ and $w[ \lpquote y!(z) \rpquote ]$.

\begin{eqnarray}
	\lift{w}{y!(z)}\widehat{\id{\{}u / z \id{\}}}
		& = &
		\lift{w}{y!(u)} \nonumber\\
	w[ \lpquote y!(z) \rpquote ] \widehat{ \id{\{}u / z \id{\}} }
		& = &
		w[ \lpquote y!(z) \rpquote ] \nonumber
\end{eqnarray}

Because the body of the process between quotes is impervious to
substitution, we get radically different answers. In fact, by
examining the first process in an input context,
e.g. $x?(z).\lift{w}{y!(z)}$, we see that the process under the lift
operator may be shaped by prefixed inputs binding a name inside it. In
this sense, the lift operator will be seen as a way to dynamically
construct processes before reifying them as names.

Finally equipped with these standard features we can present the
dynamics of the calculus.

\subsubsection{Operational semantics} 

Finally, we introduce the computational dynamics. What marks these
algebras as distinct from other more traditionally studied algebraic
structures, e.g. vector spaces or polynomial rings, is the manner in
which dynamics is captured. In traditional structures, dynamics is typically
expressed through morphisms between such structures, as in linear maps
between vector spaces or morphisms between rings. In algebras
associated with the semantics of computation, the dynamics is
expressed as part of the algebraic structure itself, through a
reduction reduction relation typically denoted by $\red$. Below, we
give a recursive presentation of this relation for the calculus used
in the encoding.

$\red \subseteq \pi \times \pi$
$\red : \pi \to \mathcal{P}(\pi)$

\begin{mathpar}
  \inferrule* [lab=Comm] { \textsf{match}( x_{src}, x_{trgt} ) } { x_{trgt}?(y)P \; | \; x_{src}!\langle {Q} \rangle \red P\{\quotep{Q}/y}\} }
  \and \\
  \inferrule* [lab=Par] {{P} \red {P}'} {{{P} | {Q}} \red {{P}' | {Q}}}
  \and
  \inferrule* [lab=Equiv]{{{P} \scong {P}'} \andalso {{P}' \red {Q}'} \andalso {{Q}' \scong {Q}}}{{P} \red {Q}}
\end{mathpar}

\begin{eqnarray*}
  match_{\equiv} (\quotep{P},\quotep{Q}) & := & P \equiv Q \\
  match_{\dagger}(\quotep{P},\quotep{Q}) & := & \forall R. P|Q \red^{*} R => R \red^{*} 0 \\
  match_{K}(\quotep{P},\quotep{Q}) & := & K \mbox{ for some context } K
\end{eqnarray*}

$u?(x)P | u!\langle Q \rangle \red P\{\quotep{Q}/x\}$

%We write $\wred$ for $\red^*$, and $P\red$ if $\exists Q $ such that $ P \red Q$.
We write $P\red$ if $\exists Q $ such that $ P \red Q$ and $P\not\red$, otherwise.

\section{Replication}

As mentioned before, it is known that replication (and hence
recursion) can be implemented in a higher-order process algebra
\cite{SangiorgiWalker}. As our first example of calculation with the
machinery thus far presented we give the construction explicitly in
the {\rhoc}.

\begin{eqnarray}
	D_{x} & := & \prefix{x}{y}{(\binpar{\outputp{x}{y}}{@{y}})} \nonumber\\
	\bangp_{x}{P} & := & \binpar{{x}!\langle{\binpar{D_{x}}{P}}\rangle}{D_{x}} \nonumber
\end{eqnarray}

\begin{eqnarray}
	\bangp_{x}{P} & & \nonumber\\
	=
	& {x}!\langle{(\prefix{x}{y}{(\outputp{x}{y} | @{y})) | P}}\rangle 
	      | \prefix{x}{y}{(\outputp{x}{y} | @{y})} & \nonumber\\
	\red
	& (\outputp{x}{y} | @{y})\substn{\quotep{(\prefix{x}{y}{(@{y} | \outputp{x}{y})) | P}}}{y} & \nonumber\\
	=
	& \outputp{x}{\quotep{(\prefix{x}{y}{(\outputp{x}{y} | @{y})) | P}}}
	  | {(\prefix{x}{y}{(\outputp{x}{y} | @{y})) | P}} & \nonumber\\
	\red
	& \ldots & \nonumber\\
	\red^*
	& P | P | \ldots & \nonumber
\end{eqnarray}

Of course, this encoding, as an implementation, runs away, unfolding
$\bangp{P}$ eagerly. A lazier and more implementable replication
operator, restricted to input-guarded processes, may be obtained as follows.

\begin{eqnarray}
\bangp{\prefix{u}{v}{P}} 
	:= 
	\binpar{\lift{x}{\prefix{u}{v}{(\binpar{D(x)}{P})}}}{D(x)} \nonumber
\end{eqnarray}

\begin{remark}
  Note that the lazier definition still does not deal with summation
  or mixed summation (i.e. sums over input and output). The reader is
  invited to construct definitions of replication that deal with these
  features. 

  Further, the definitions are parameterized in a name, $x$. Can you,
  gentle reader, make a definition that eliminates this parameter and
  guarantees no accidental interaction between the replication
  machinery and the process being replicated -- i.e. no accidental
  sharing of names used by the process to get its work done and the
  name(s) used by the replication to effect copying. This latter
  revision of the definition of replication is crucial to obtaining
  the expected identity $!!P \sim !P$.
\end{remark}

\begin{remark}\label{rem:paradoxical_combinator}
  The reader familiar with the lambda calculus will have noticed the
  similarity between $D$ and the paradoxical combinator.

  [Ed. note: the existence of this seems to suggest we have to be more
  restrictive on the set of processes and names we admit if we are to
  support no-cloning.]
\end{remark}

\subsubsection{Bisimulation}

The computational dynamics gives rise to another kind of equivalence,
the equivalence of computational behavior. As previously mentioned
this is typically captured \emph{via} some form of bisimulation.

% The notion we use in this paper is weak barbed bisimulation
% \cite{milner91polyadicpi}.

The notion we use in this paper is derived from weak barbed
bisimulation \cite{milner91polyadicpi}. 

\begin{definition}
An \emph{observation relation}, $\downarrow_{\mathcal N}$, over a set
of names, $\mathcal N$, is the smallest relation satisfying the rules
below.

\infrule[Out-barb]{y \in {\mathcal N}, \; x \nameeq y}
		  {\outputp{x}{v} \downarrow_{\mathcal N} x}
\infrule[Par-barb]{\mbox{$P\downarrow_{\mathcal N} x$ or $Q\downarrow_{\mathcal N} x$}}
		  {\binpar{P}{Q} \downarrow_{\mathcal N} x}

We write $P \Downarrow_{\mathcal N} x$ if there is $Q$ such that 
$P \wred Q$ and $Q \downarrow_{\mathcal N} x$.
\end{definition}

\begin{definition}
%\label{def.bbisim}
An  ${\mathcal N}$-\emph{barbed bisimulation} over a set of names, ${\mathcal N}$, is a symmetric binary relation 
${\mathcal S}_{\mathcal N}$ between agents such that $P\rel{S}_{\mathcal N}Q$ implies:
\begin{enumerate}
\item If $P \red P'$ then $Q \wred Q'$ and $P'\rel{S}_{\mathcal N} Q'$.
\item If $P\downarrow_{\mathcal N} x$, then $Q\Downarrow_{\mathcal N} x$.
\end{enumerate}
$P$ is ${\mathcal N}$-barbed bisimilar to $Q$, written
$P \wbbisim_{\mathcal N} Q$, if $P \rel{S}_{\mathcal N} Q$ for some ${\mathcal N}$-barbed bisimulation ${\mathcal S}_{\mathcal N}$.
\end{definition}

$\mathcal{R} \subseteq \pi \times \pi$

$P \mathcal{R} Q => \forall P'. P \red P' \Rightarrow \exists Q'. Q \red Q', P' \mathcal{R} Q'$

$P \vdash x \Rightarrow Q \vdash x$

\begin{mathpar}
  \inferrule*[lab=Out-barb]{x \nameeq y}{{y}!\langle{Q}\rangle \vdash x}
  \and
  \inferrule*[lab=Par-barb]{\mbox{$P\vdash x$ or $Q\vdash x$}}{\binpar{P}{Q} \vdash x}
\end{mathpar}

\subsubsection{Contexts}

One of the principle advantages of computational calculi like the
$\pi$-calculus is a well-defined notion of context,
contextual-equivalence and a correlation between
contextual-equivalence and notions of bisimulation. The notion of
context allows the decomposition of a process into (sub-)process and
its syntactic environment, its context. Thus, a context may be
thought of as a process with a ``hole'' (written $\Box$) in it. The
application of a context $M$ to a process $P$, written $M[P]$, is
tantamount to filling the hole in $M$ with $P$. In this paper we do
not need the full weight of this theory, but do make use of the notion
of context in the proof the main theorem. 

\begin{mathpar}
  \inferrule* [lab=summation] {} {{M_{M},M_{N}} \bc \Box \;|\; x.M_{A} \;|\; M_{M}+M_{N}}
  \and
  \inferrule* [lab=agent] {} {{M_{A}} \bc (\vec{x})M_{P} \;| \; \clift{P_0,\ldots,M_{P},\ldots,P_N}}
  \and \\
  \inferrule* [lab=process] {} {{M_{P}} \bc M_{N} \;| \;P|M_{P} }
\end{mathpar} 

\begin{mathpar}
  \inferrule* [lab=sychronization] {} {M_{N} \bc \Box \;|\; x?M_{F} \;|\; x!M_{C}}
  \and
  \inferrule* [lab=abstraction] {} {{M_{F}} \bc (x)M_{P} }
  \and
  \inferrule* [lab=concretion] {} {{M_{C}} \bc \langle M_{P} \rangle }
  \and \\
  \inferrule* [lab=process] {} {{M_{P}} \bc M_{N} \;| \;P|M_{P} }
\end{mathpar}

\begin{definition}[contextual application] Given a context $M$, and
  process $P$, we define the \emph{contextual application}, $M[P] :=
  M\{P/\Box\}$. That is, the contextual application of M to P is the
  substitution of $P$ for $\Box$ in $M$.
\end{definition}

$\meaningof{-} : L \to \mathcal{P}(\pi)$

\begin{mathpar}
  \inferrule* [lab=collection] {} {\meaningof{true} = \pi, \and \meaningof{~E} = \pi \setminus \meaningof{E}, \and \meaningof{E_{1} \& E_{2}} = \meaningof{E_{1}} \cap \meaningof{E_{2}}}
\end{mathpar}

\begin{mathpar}
  \inferrule* [lab=structure] {} {\meaningof{0} = \{ P \in \pi | P \equiv 0 \}, \and \\ \meaningof{E_1 | E_2} = \{ P \in \pi | P \equiv P_{1} | P_{2}, P_{1} \in \meaningof{E_{1}}, P_{2} \in \meaningof{E_2}\} }
\end{mathpar}

\begin{mathpar}
 \inferrule* [lab=behavior] {} {\meaningof{\langle a?b \rangle E} = \{ P \in \pi | P \equiv Q | u?(y)P', \\ \and \\\\ \and \\ \;\;\; u \in \meaningof{a}, \forall z.P'\{z/y\} \in \meaningof{E\{z/b\}}\}, \and \\ \meaningof{a!E} = \{ P \in \pi | P \equiv Q | x!\langle P' \rangle, x \in \meaningof{a} P' \in \meaningof{E}\} }
\end{mathpar}

\begin{mathpar}
 \inferrule* [lab=nominal] {} {\meaningof{\quotep{E}} = \{ \quotep{P} \in \quotep{\pi} | P \in \meaningof{E} \}, \and \meaningof{\quotep{P}} = \{ \quotep{Q} \in \quotep{\pi} | P \equiv Q \} \and \\ \meaningof{@\quotep{E}} = \{ P \in \pi | P \equiv @x, x \in \meaningof{E} \}}
\end{mathpar}

\begin{eqnarray*}
  \\
  \meaningof{-} : TS \to ST
\end{eqnarray*}

\begin{eqnarray*}
  \\
  L : TS \to ST
\end{eqnarray*}

\begin{eqnarray*}
  \\
  P \models E \iff P \in \meaningof{E}
\end{eqnarray*}

\begin{eqnarray*}
  P \approx_{L} Q \iff \forall E \in L. P \models E \iff Q \models E
\end{eqnarray*}

\begin{eqnarray*}
  P \approx_{K} Q
\end{eqnarray*}

\begin{eqnarray*}
  P \approx Q
\end{eqnarray*}

$\approx_{K} = \approx = \approx_{L}$

\subsubsection{Contextual duality}

Note that contexts extend the quotation operation to a family of
operations from processes to names. Given a context, $M$, we can
define a \emph{nominal context}, $\quotep{M}$ by $\quotep{M}[P] :=
\quotep{M[P]}$. To foreshadow what is to come we observe that these
operations enjoy a duality with processes very much like the duality
between vectors and maps from vectors to scalars.

Further, because the calculus is essentially higher-order, we have a
correspondence between contexts and processes. More specifically,
given a name $x$ and a context $M$ we can construct $M^{*}_{x}$ such
that 

\begin{mathpar}
  M^{*}_{x} | \lift{x}{P} \red M[P]
\end{mathpar}

namely,

\begin{mathpar}
  M^{*}_{x} := x?(u).M[\dropn{u}]
\end{mathpar}

The dependence of $M^{*}_{x}$ on a name makes it an abstraction, 

\begin{mathpar}
  M^{*} := (x)x?(u).M[\dropn{u}]
\end{mathpar}

\subsection{Additional notation}

It will sometimes be convenient to denote the process a name
quotes. We already have the notation $x = \quotep{P}$, but it will be
convenient to introduce an alternate notation, $\procn{x}$, when we
want to emphasize the connection to the use of the name. Note that, by
virtue of name equivalence, $\quotep{\procn{x}} \nameeq x$; so, the
notation is consistent with previous definitions.

Further, because names have structure it is possible to effect
substitutions on the basis of that structure. This means we need to
upgrade our notation for substitutions, which we accomplish by
adapting comprehension notation. Thus,

\begin{mathpar}
  P\{ y / x : x \in S \}
\end{mathpar}

is interpreted to mean the process derived from P by replacing (in a
capture-avoiding manner) each occurrence of $x$ in $S$ by $y$. For example,

\begin{mathpar}
  P\{ \quotep{\procn{x}|\procn{x}} / x : x \in \freenames{P} \}
\end{mathpar}

will replace each (occurrence) of a free name $x$ in $P$ by
$\quotep{\procn{x}|\procn{x}}$.

Also, we will avail ourselves of the notation $x^{L}$ and $x^{R}$ to
denote injections of a name into disjoint copies of the name
space. There are numerous ways to accomplish this. One example can be
found in \cite{MeredithR05}. This notation overloads to vectors of
names: $\vec{x}^{\pi} := (x_{i}^{\pi} \; : \; 0 \leq i < |\vec{x}| )$ where $\pi \in \{L,R\}$.

We also use $P^{\Box} := P|\Box$.

In \cite{MeredithR05} an interpretation of the new operator is
given. It turns out that there are several possible interpretations
all enjoying the requisite algebraic properties of the operator (see
\cite{milner91polyadicpi}). We will therefore make liberal use of
$(\nu\; \vec{x})P$.

% subsection the_syntax_and_semantics_of_the_notation_system (end)   

\input{qm2pi.qmops} 

\input{qm2pi.sterngerlach} 

\input{qm2pi.metric} 

% section concurrent_process_calculi (end)

%\input{qm2pi.proofsketch}

% section proof sketch (end)

%\input{qm2pi.slviaknots} 

% section spatial logic via knots (end)

\input{qm2pi.conclusion}

% section conclusion (end)

%\input{qm2pi.dtcodes} 

% section wiring algorithm (end)

\input{qm2pi.ack} 

% section acknowledgments (end)

\newpage


\bibliographystyle{plain}   
\bibliography{../../biblios/main.bib}

\input{qm2pi.rhodetails}

\end{document}

 

% section acknowledgments (end)

\newpage


\bibliographystyle{plain}   
\bibliography{../../biblios/main.bib}

\documentclass[12pt]{llncs}
%\documentclass{jktr}

\usepackage[pdftex]{hyperref}                   
\usepackage {listings}
\usepackage {mathpartir}
\usepackage{bcprules}
%\usepackage{listings}
                       
\usepackage{graphicx} 
%\usepackage[margins=2.5cm,nohead,nofoot]{geometry}
%\usepackage{geometry}
\usepackage{amsfonts}
\usepackage{amstext}
\usepackage{latexsym}
\usepackage{amssymb}
\usepackage{color}


%\include{myPreamble}
\include{qm2pi.local} 

%\ifpdf
%\usepackage[pdftex]{graphicx}
%\else
%\usepackage{graphicx}
%\fi

 % \ifpdf
%  \usepackage{pdfsync}
%  \if


%\title{Brief Article}
%\author{David F. Snyder}
%\author{L.G. Meredith}

%\address{Dept. of Math., Texas State University--San Marcos, San Marcos, TX 78666}
       
\pagestyle{empty}


\begin{document}

\lstset{language=[Objective]Caml,frame=shadowbox}

\input{qm2pi.front}

% section front matter (end)

\input{qm2pi.intro} 
 
% section introduction (end)

% \input{qm2pi.knotations} 

% section notation (end)

\input{qm2pi.process.calculi} 

% section concurrent_process_calculi_and_spatial_logics_ (end)
    
%\input{qm2pi.knots2pi} 

%\input{qm2pi.trefoil} 

%\input{qm2pi.mainthm} 

% subsection basic_interpretation (end)

%\input{qm2pi.rho.presentation} 
\subsection{The syntax and semantics of the notation system}\label{sub:the_syntax_and_semantics_of_the_notation_system} % (fold)

We now summarize a technical presentation of the calculus that
embodies our theory of dynamics. The typical presentation of such a
calculus follows the style of giving generators and relations on
them. The grammar, below, describing term constructors, freely
generates the set of processes, $\Proc$. This set is then quotiented
by a relation known as structural congruence and it is over this set
that the notion of dynamics is expressed. This presentation is
essentially that of \cite{MeredithR05} with the addition of
polyadicity and summation. For readability we have relegated some of
the technical subtleties to an appendix.

\subsubsection{Process grammar}\label{subsub:process_grammar}

\begin{mathpar}
  \inferrule* [lab=synchronization] {} {{M} \bc \pzero \;|\; x?F \;|\; x!C }
  \and
  \inferrule* [lab=abstraction] {} {{F} \bc (x)P}
  \and
  \inferrule* [lab=concretion] {} {{C} \bc \langle Q \rangle}
  \and
  \inferrule* [lab=process] {} {{P,Q} \bc M \;| \;P|Q \;|\; @{x}}
  \and
  \inferrule* [lab=name] {} {{x} \bc \quotep{P}}
\end{mathpar} 

Note that $\vec{x}$ (resp. $\vec{P}$) denotes a vector of names
(resp. processes) of length $|\vec{x}|$ (resp. $|\vec{P}|$). We adopt
the following useful abbreviations.

\begin{mathpar}
   x?(\vec{y}).P := x.(\vec{y})P \and  x\clift{\vec{P}} := x.\clift{\vec{P}}
   \and x!(y) := \lift{x}{\dropn{y}}
   \and \Pi_{i=0}^{n-1}P_i := P_0 | \ldots | P_{n-1}
\end{mathpar}

\subsubsection{Structural congruence}

\paragraph{Free and bound names and alpha-equivalence.} At the
core of structural equivalence is alpha-equivalence which identifies
process that are the same up to a change of variable. Formally, we
recognize the distinction between free and bound names. The free names
of a process, $\freenames{P}$, may be calculated recursively as
follows:

\begin{mathpar}
\freenames{\pzero} := \emptyset
  \and \\
  \freenames{x?(y).P} := \{ x \} \cup (\freenames{P} \setminus \{ y \})
  \and 
  \freenames{x!\langle P \rangle} := \{ x \} \cup \{ P \} 
  \and \\
  \freenames{P|Q} := \freenames{P} \cup \freenames{Q}
  \and \\
  \freenames{@{x}} := \{ x \}
\end{mathpar}

$\pi$
$\quotep{\pi}$

$\freenames{-} : \pi \to \mathcal{P}(\quotep{\pi})$

\begin{eqnarray*}
  \freenames{\pzero} & := & \emptyset \\
  \freenames{x?(y).P} & := & \{ x \} \cup (\freenames{P} \setminus \{ y \}) \\
  \freenames{x!\langle P \rangle} & := & \{ x \} \cup \{ P \} \\
  \freenames{P|Q} & := & \freenames{P} \cup \freenames{Q} \\
  \freenames{\dropn{x}} & := & \{ x \}
\end{eqnarray*}

The bound names of a process, $\boundnames{P}$, are those names occurring in $P$
that are not free. For example, in $x?(y).0$, the name $x$ is free, while $y$ is bound.

\begin{mathpar}
  \inferrule* [lab=monoidal-laws] {} { P|Q \equiv Q|P \and P|0 \equiv P \and P|(Q|R) \equiv (P|Q)|R }
\end{mathpar}

\begin{mathpar}
  \inferrule* [lab=alpha-equivalence] {} { (x)P \equiv (y)P\{y/x\} \and y \not\in \freenames{P} }
\end{mathpar}

\begin{definition}
Then two processes, $P,Q$, are alpha-equivalent if $P = Q\{\vec{y}/\vec{x}\}$ for
some $\vec{x} \in \boundnames{Q},\vec{y} \in \boundnames{P}$, where $Q\{\vec{y}/\vec{x}\}$
denotes the capture-avoiding substitution of $\vec{y}$ for $\vec{x}$ in $Q$.
\end{definition}

\begin{definition}
  The {\em structural congruence} \cite{SangiorgiWalker} , $\equiv$,
  between processes is the least congruence containing
  alpha-equivalence, satisfying the abelian monoid laws
  (associativity, commutativity and $\pzero$ as identity) for parallel
  composition $|$ and for summation $+$.
\end{definition}

\subsection{Name equivalence}

We take name equivalence, written $\nameeq$, to be the smallest
equivalence relation generated by the following rules.

\begin{mathpar}
\inferrule*[lab=Quote-drop]
{ }
{ \quotep{@{x}} \nameeq x }

\inferrule*[lab=Struct-equiv]
{ P \scong Q }
{ \quotep{P} \nameeq \quotep{Q} }
\end{mathpar}

The astute reader will have noticed that the mutual recursion of names
and processes imposes a mutual recursion on alpha-equivalence and
structural equivalence via name-equivalence. Fortunately, all of this
works out pleasantly and we may calculate in the natural way, free of
concern. The reader interested in the details is referred to the
appendix \ref{appendix:rho_details}.

\subsection{Substitution}

We use $\Proc$ for the set of processes, $\QProc$ for the set of
names, and $\id{\{}\vec{y} / \vec{x} \id{\}}$ to denote partial maps,
$s : \QProc \rightarrow \QProc$. A map, $s$ lifts, uniquely, to a map
on process terms, $\widehat{s} : \Proc \rightarrow \Proc$ by the
following equations.

\begin{mathpar}
  (0) \psubstp{Q}{P} := 0 \\
  (R \juxtap S) \psubstp{Q}{P}
  :=    
  (R)\psubstp{Q}{P} \juxtap (S) \psubstp{Q}{P} \\
  (x?(y).R) \psubstp{Q}{P}    
  :=    
  (x)\substp{Q}{P} (z)\concat( (R \psubstn{z}{y}) \psubstp{Q}{P} ) \\
  (\lift{x}{R}) \psubstp{Q}{P}  
  :=
  \lift{(x)\substp{Q}{P}}{ R \psubstp{Q}{P} } \\
%   (\dropn{x})  \psubstp{Q}{P}       
%   := 
%   \left\{ 
%     \begin{array}{ccc} 
%       \dropn{\quotep{Q}} & & x \nameeq \quotep{P} \\
%       \dropn{x} & & otherwise \\
%     \end{array}
%   \right. 
  (\dropn{x})  \psubstp{Q}{P}       
  := 
  \left\{ 
    \begin{array}{ccc} 
      Q & & x \nameeq \quotep{P} \\
      \dropn{x} & & otherwise \\
    \end{array}
  \right.
\end{mathpar}
 

where

\begin{eqnarray}
  (x)\id{\{} \lpquote Q \rpquote / \lpquote P \rpquote \id{\}}            = 
  \left\{ 
    \begin{array}{ccc}
      \lpquote Q \rpquote & & x \nameeq \lpquote P \rpquote \\
      x & & otherwise \\
    \end{array}
  \right. \nonumber
\end{eqnarray}

and $z$ is chosen distinct from $\quotep{P}$, $\quotep{Q}$, the free
names in $Q$, and all the names in $R$. Our $\alpha$-equivalence will
be built in the standard way from this substitution.

\begin{remark}\label{rem:no_self_referential_names}
  One consequence of these definitions is that $\forall P. \quotep{P}
  \not\in \freenames{P}$.
\end{remark}

\subsection{ Dynamic quote: an example }

Anticipating something of what's to come, consider applying the
substitution, $\widehat{\id{\{}u / z \id{\}}}$, to the following pair
of processes, $\lift{w}{y!(z)}$ and $w[ \lpquote y!(z) \rpquote ]$.

\begin{eqnarray}
	\lift{w}{y!(z)}\widehat{\id{\{}u / z \id{\}}}
		& = &
		\lift{w}{y!(u)} \nonumber\\
	w[ \lpquote y!(z) \rpquote ] \widehat{ \id{\{}u / z \id{\}} }
		& = &
		w[ \lpquote y!(z) \rpquote ] \nonumber
\end{eqnarray}

Because the body of the process between quotes is impervious to
substitution, we get radically different answers. In fact, by
examining the first process in an input context,
e.g. $x?(z).\lift{w}{y!(z)}$, we see that the process under the lift
operator may be shaped by prefixed inputs binding a name inside it. In
this sense, the lift operator will be seen as a way to dynamically
construct processes before reifying them as names.

Finally equipped with these standard features we can present the
dynamics of the calculus.

\subsubsection{Operational semantics} 

Finally, we introduce the computational dynamics. What marks these
algebras as distinct from other more traditionally studied algebraic
structures, e.g. vector spaces or polynomial rings, is the manner in
which dynamics is captured. In traditional structures, dynamics is typically
expressed through morphisms between such structures, as in linear maps
between vector spaces or morphisms between rings. In algebras
associated with the semantics of computation, the dynamics is
expressed as part of the algebraic structure itself, through a
reduction reduction relation typically denoted by $\red$. Below, we
give a recursive presentation of this relation for the calculus used
in the encoding.

$\red \subseteq \pi \times \pi$
$\red : \pi \to \mathcal{P}(\pi)$

\begin{mathpar}
  \inferrule* [lab=Comm] { \textsf{match}( x_{src}, x_{trgt} ) } { x_{trgt}?(y)P \; | \; x_{src}!\langle {Q} \rangle \red P\{\quotep{Q}/y}\} }
  \and \\
  \inferrule* [lab=Par] {{P} \red {P}'} {{{P} | {Q}} \red {{P}' | {Q}}}
  \and
  \inferrule* [lab=Equiv]{{{P} \scong {P}'} \andalso {{P}' \red {Q}'} \andalso {{Q}' \scong {Q}}}{{P} \red {Q}}
\end{mathpar}

\begin{eqnarray*}
  match_{\equiv} (\quotep{P},\quotep{Q}) & := & P \equiv Q \\
  match_{\dagger}(\quotep{P},\quotep{Q}) & := & \forall R. P|Q \red^{*} R => R \red^{*} 0 \\
  match_{K}(\quotep{P},\quotep{Q}) & := & K \mbox{ for some context } K
\end{eqnarray*}

$u?(x)P | u!\langle Q \rangle \red P\{\quotep{Q}/x\}$

%We write $\wred$ for $\red^*$, and $P\red$ if $\exists Q $ such that $ P \red Q$.
We write $P\red$ if $\exists Q $ such that $ P \red Q$ and $P\not\red$, otherwise.

\section{Replication}

As mentioned before, it is known that replication (and hence
recursion) can be implemented in a higher-order process algebra
\cite{SangiorgiWalker}. As our first example of calculation with the
machinery thus far presented we give the construction explicitly in
the {\rhoc}.

\begin{eqnarray}
	D_{x} & := & \prefix{x}{y}{(\binpar{\outputp{x}{y}}{@{y}})} \nonumber\\
	\bangp_{x}{P} & := & \binpar{{x}!\langle{\binpar{D_{x}}{P}}\rangle}{D_{x}} \nonumber
\end{eqnarray}

\begin{eqnarray}
	\bangp_{x}{P} & & \nonumber\\
	=
	& {x}!\langle{(\prefix{x}{y}{(\outputp{x}{y} | @{y})) | P}}\rangle 
	      | \prefix{x}{y}{(\outputp{x}{y} | @{y})} & \nonumber\\
	\red
	& (\outputp{x}{y} | @{y})\substn{\quotep{(\prefix{x}{y}{(@{y} | \outputp{x}{y})) | P}}}{y} & \nonumber\\
	=
	& \outputp{x}{\quotep{(\prefix{x}{y}{(\outputp{x}{y} | @{y})) | P}}}
	  | {(\prefix{x}{y}{(\outputp{x}{y} | @{y})) | P}} & \nonumber\\
	\red
	& \ldots & \nonumber\\
	\red^*
	& P | P | \ldots & \nonumber
\end{eqnarray}

Of course, this encoding, as an implementation, runs away, unfolding
$\bangp{P}$ eagerly. A lazier and more implementable replication
operator, restricted to input-guarded processes, may be obtained as follows.

\begin{eqnarray}
\bangp{\prefix{u}{v}{P}} 
	:= 
	\binpar{\lift{x}{\prefix{u}{v}{(\binpar{D(x)}{P})}}}{D(x)} \nonumber
\end{eqnarray}

\begin{remark}
  Note that the lazier definition still does not deal with summation
  or mixed summation (i.e. sums over input and output). The reader is
  invited to construct definitions of replication that deal with these
  features. 

  Further, the definitions are parameterized in a name, $x$. Can you,
  gentle reader, make a definition that eliminates this parameter and
  guarantees no accidental interaction between the replication
  machinery and the process being replicated -- i.e. no accidental
  sharing of names used by the process to get its work done and the
  name(s) used by the replication to effect copying. This latter
  revision of the definition of replication is crucial to obtaining
  the expected identity $!!P \sim !P$.
\end{remark}

\begin{remark}\label{rem:paradoxical_combinator}
  The reader familiar with the lambda calculus will have noticed the
  similarity between $D$ and the paradoxical combinator.

  [Ed. note: the existence of this seems to suggest we have to be more
  restrictive on the set of processes and names we admit if we are to
  support no-cloning.]
\end{remark}

\subsubsection{Bisimulation}

The computational dynamics gives rise to another kind of equivalence,
the equivalence of computational behavior. As previously mentioned
this is typically captured \emph{via} some form of bisimulation.

% The notion we use in this paper is weak barbed bisimulation
% \cite{milner91polyadicpi}.

The notion we use in this paper is derived from weak barbed
bisimulation \cite{milner91polyadicpi}. 

\begin{definition}
An \emph{observation relation}, $\downarrow_{\mathcal N}$, over a set
of names, $\mathcal N$, is the smallest relation satisfying the rules
below.

\infrule[Out-barb]{y \in {\mathcal N}, \; x \nameeq y}
		  {\outputp{x}{v} \downarrow_{\mathcal N} x}
\infrule[Par-barb]{\mbox{$P\downarrow_{\mathcal N} x$ or $Q\downarrow_{\mathcal N} x$}}
		  {\binpar{P}{Q} \downarrow_{\mathcal N} x}

We write $P \Downarrow_{\mathcal N} x$ if there is $Q$ such that 
$P \wred Q$ and $Q \downarrow_{\mathcal N} x$.
\end{definition}

\begin{definition}
%\label{def.bbisim}
An  ${\mathcal N}$-\emph{barbed bisimulation} over a set of names, ${\mathcal N}$, is a symmetric binary relation 
${\mathcal S}_{\mathcal N}$ between agents such that $P\rel{S}_{\mathcal N}Q$ implies:
\begin{enumerate}
\item If $P \red P'$ then $Q \wred Q'$ and $P'\rel{S}_{\mathcal N} Q'$.
\item If $P\downarrow_{\mathcal N} x$, then $Q\Downarrow_{\mathcal N} x$.
\end{enumerate}
$P$ is ${\mathcal N}$-barbed bisimilar to $Q$, written
$P \wbbisim_{\mathcal N} Q$, if $P \rel{S}_{\mathcal N} Q$ for some ${\mathcal N}$-barbed bisimulation ${\mathcal S}_{\mathcal N}$.
\end{definition}

$\mathcal{R} \subseteq \pi \times \pi$

$P \mathcal{R} Q => \forall P'. P \red P' \Rightarrow \exists Q'. Q \red Q', P' \mathcal{R} Q'$

$P \vdash x \Rightarrow Q \vdash x$

\begin{mathpar}
  \inferrule*[lab=Out-barb]{x \nameeq y}{{y}!\langle{Q}\rangle \vdash x}
  \and
  \inferrule*[lab=Par-barb]{\mbox{$P\vdash x$ or $Q\vdash x$}}{\binpar{P}{Q} \vdash x}
\end{mathpar}

\subsubsection{Contexts}

One of the principle advantages of computational calculi like the
$\pi$-calculus is a well-defined notion of context,
contextual-equivalence and a correlation between
contextual-equivalence and notions of bisimulation. The notion of
context allows the decomposition of a process into (sub-)process and
its syntactic environment, its context. Thus, a context may be
thought of as a process with a ``hole'' (written $\Box$) in it. The
application of a context $M$ to a process $P$, written $M[P]$, is
tantamount to filling the hole in $M$ with $P$. In this paper we do
not need the full weight of this theory, but do make use of the notion
of context in the proof the main theorem. 

\begin{mathpar}
  \inferrule* [lab=summation] {} {{M_{M},M_{N}} \bc \Box \;|\; x.M_{A} \;|\; M_{M}+M_{N}}
  \and
  \inferrule* [lab=agent] {} {{M_{A}} \bc (\vec{x})M_{P} \;| \; \clift{P_0,\ldots,M_{P},\ldots,P_N}}
  \and \\
  \inferrule* [lab=process] {} {{M_{P}} \bc M_{N} \;| \;P|M_{P} }
\end{mathpar} 

\begin{mathpar}
  \inferrule* [lab=sychronization] {} {M_{N} \bc \Box \;|\; x?M_{F} \;|\; x!M_{C}}
  \and
  \inferrule* [lab=abstraction] {} {{M_{F}} \bc (x)M_{P} }
  \and
  \inferrule* [lab=concretion] {} {{M_{C}} \bc \langle M_{P} \rangle }
  \and \\
  \inferrule* [lab=process] {} {{M_{P}} \bc M_{N} \;| \;P|M_{P} }
\end{mathpar}

\begin{definition}[contextual application] Given a context $M$, and
  process $P$, we define the \emph{contextual application}, $M[P] :=
  M\{P/\Box\}$. That is, the contextual application of M to P is the
  substitution of $P$ for $\Box$ in $M$.
\end{definition}

$\meaningof{-} : L \to \mathcal{P}(\pi)$

\begin{mathpar}
  \inferrule* [lab=collection] {} {\meaningof{true} = \pi, \and \meaningof{~E} = \pi \setminus \meaningof{E}, \and \meaningof{E_{1} \& E_{2}} = \meaningof{E_{1}} \cap \meaningof{E_{2}}}
\end{mathpar}

\begin{mathpar}
  \inferrule* [lab=structure] {} {\meaningof{0} = \{ P \in \pi | P \equiv 0 \}, \and \\ \meaningof{E_1 | E_2} = \{ P \in \pi | P \equiv P_{1} | P_{2}, P_{1} \in \meaningof{E_{1}}, P_{2} \in \meaningof{E_2}\} }
\end{mathpar}

\begin{mathpar}
 \inferrule* [lab=behavior] {} {\meaningof{\langle a?b \rangle E} = \{ P \in \pi | P \equiv Q | u?(y)P', \\ \and \\\\ \and \\ \;\;\; u \in \meaningof{a}, \forall z.P'\{z/y\} \in \meaningof{E\{z/b\}}\}, \and \\ \meaningof{a!E} = \{ P \in \pi | P \equiv Q | x!\langle P' \rangle, x \in \meaningof{a} P' \in \meaningof{E}\} }
\end{mathpar}

\begin{mathpar}
 \inferrule* [lab=nominal] {} {\meaningof{\quotep{E}} = \{ \quotep{P} \in \quotep{\pi} | P \in \meaningof{E} \}, \and \meaningof{\quotep{P}} = \{ \quotep{Q} \in \quotep{\pi} | P \equiv Q \} \and \\ \meaningof{@\quotep{E}} = \{ P \in \pi | P \equiv @x, x \in \meaningof{E} \}}
\end{mathpar}

\begin{eqnarray*}
  \\
  \meaningof{-} : TS \to ST
\end{eqnarray*}

\begin{eqnarray*}
  \\
  L : TS \to ST
\end{eqnarray*}

\begin{eqnarray*}
  \\
  P \models E \iff P \in \meaningof{E}
\end{eqnarray*}

\begin{eqnarray*}
  P \approx_{L} Q \iff \forall E \in L. P \models E \iff Q \models E
\end{eqnarray*}

\begin{eqnarray*}
  P \approx_{K} Q
\end{eqnarray*}

\begin{eqnarray*}
  P \approx Q
\end{eqnarray*}

$\approx_{K} = \approx = \approx_{L}$

\subsubsection{Contextual duality}

Note that contexts extend the quotation operation to a family of
operations from processes to names. Given a context, $M$, we can
define a \emph{nominal context}, $\quotep{M}$ by $\quotep{M}[P] :=
\quotep{M[P]}$. To foreshadow what is to come we observe that these
operations enjoy a duality with processes very much like the duality
between vectors and maps from vectors to scalars.

Further, because the calculus is essentially higher-order, we have a
correspondence between contexts and processes. More specifically,
given a name $x$ and a context $M$ we can construct $M^{*}_{x}$ such
that 

\begin{mathpar}
  M^{*}_{x} | \lift{x}{P} \red M[P]
\end{mathpar}

namely,

\begin{mathpar}
  M^{*}_{x} := x?(u).M[\dropn{u}]
\end{mathpar}

The dependence of $M^{*}_{x}$ on a name makes it an abstraction, 

\begin{mathpar}
  M^{*} := (x)x?(u).M[\dropn{u}]
\end{mathpar}

\subsection{Additional notation}

It will sometimes be convenient to denote the process a name
quotes. We already have the notation $x = \quotep{P}$, but it will be
convenient to introduce an alternate notation, $\procn{x}$, when we
want to emphasize the connection to the use of the name. Note that, by
virtue of name equivalence, $\quotep{\procn{x}} \nameeq x$; so, the
notation is consistent with previous definitions.

Further, because names have structure it is possible to effect
substitutions on the basis of that structure. This means we need to
upgrade our notation for substitutions, which we accomplish by
adapting comprehension notation. Thus,

\begin{mathpar}
  P\{ y / x : x \in S \}
\end{mathpar}

is interpreted to mean the process derived from P by replacing (in a
capture-avoiding manner) each occurrence of $x$ in $S$ by $y$. For example,

\begin{mathpar}
  P\{ \quotep{\procn{x}|\procn{x}} / x : x \in \freenames{P} \}
\end{mathpar}

will replace each (occurrence) of a free name $x$ in $P$ by
$\quotep{\procn{x}|\procn{x}}$.

Also, we will avail ourselves of the notation $x^{L}$ and $x^{R}$ to
denote injections of a name into disjoint copies of the name
space. There are numerous ways to accomplish this. One example can be
found in \cite{MeredithR05}. This notation overloads to vectors of
names: $\vec{x}^{\pi} := (x_{i}^{\pi} \; : \; 0 \leq i < |\vec{x}| )$ where $\pi \in \{L,R\}$.

We also use $P^{\Box} := P|\Box$.

In \cite{MeredithR05} an interpretation of the new operator is
given. It turns out that there are several possible interpretations
all enjoying the requisite algebraic properties of the operator (see
\cite{milner91polyadicpi}). We will therefore make liberal use of
$(\nu\; \vec{x})P$.

% subsection the_syntax_and_semantics_of_the_notation_system (end)   

\input{qm2pi.qmops} 

\input{qm2pi.sterngerlach} 

\input{qm2pi.metric} 

% section concurrent_process_calculi (end)

%\input{qm2pi.proofsketch}

% section proof sketch (end)

%\input{qm2pi.slviaknots} 

% section spatial logic via knots (end)

\input{qm2pi.conclusion}

% section conclusion (end)

%\input{qm2pi.dtcodes} 

% section wiring algorithm (end)

\input{qm2pi.ack} 

% section acknowledgments (end)

\newpage


\bibliographystyle{plain}   
\bibliography{../../biblios/main.bib}

\input{qm2pi.rhodetails}

\end{document}



\end{document}

 

%\documentclass[12pt]{llncs}
%\documentclass{jktr}

\usepackage[pdftex]{hyperref}                   
\usepackage {listings}
\usepackage {mathpartir}
\usepackage{bcprules}
%\usepackage{listings}
                       
\usepackage{graphicx} 
%\usepackage[margins=2.5cm,nohead,nofoot]{geometry}
%\usepackage{geometry}
\usepackage{amsfonts}
\usepackage{amstext}
\usepackage{latexsym}
\usepackage{amssymb}
\usepackage{color}


%\include{myPreamble}
\documentclass[12pt]{llncs}
%\documentclass{jktr}

\usepackage[pdftex]{hyperref}                   
\usepackage {listings}
\usepackage {mathpartir}
\usepackage{bcprules}
%\usepackage{listings}
                       
\usepackage{graphicx} 
%\usepackage[margins=2.5cm,nohead,nofoot]{geometry}
%\usepackage{geometry}
\usepackage{amsfonts}
\usepackage{amstext}
\usepackage{latexsym}
\usepackage{amssymb}
\usepackage{color}


%\include{myPreamble}
\include{qm2pi.local} 

%\ifpdf
%\usepackage[pdftex]{graphicx}
%\else
%\usepackage{graphicx}
%\fi

 % \ifpdf
%  \usepackage{pdfsync}
%  \if


%\title{Brief Article}
%\author{David F. Snyder}
%\author{L.G. Meredith}

%\address{Dept. of Math., Texas State University--San Marcos, San Marcos, TX 78666}
       
\pagestyle{empty}


\begin{document}

\lstset{language=[Objective]Caml,frame=shadowbox}

\input{qm2pi.front}

% section front matter (end)

\input{qm2pi.intro} 
 
% section introduction (end)

% \input{qm2pi.knotations} 

% section notation (end)

\input{qm2pi.process.calculi} 

% section concurrent_process_calculi_and_spatial_logics_ (end)
    
%\input{qm2pi.knots2pi} 

%\input{qm2pi.trefoil} 

%\input{qm2pi.mainthm} 

% subsection basic_interpretation (end)

%\input{qm2pi.rho.presentation} 
\subsection{The syntax and semantics of the notation system}\label{sub:the_syntax_and_semantics_of_the_notation_system} % (fold)

We now summarize a technical presentation of the calculus that
embodies our theory of dynamics. The typical presentation of such a
calculus follows the style of giving generators and relations on
them. The grammar, below, describing term constructors, freely
generates the set of processes, $\Proc$. This set is then quotiented
by a relation known as structural congruence and it is over this set
that the notion of dynamics is expressed. This presentation is
essentially that of \cite{MeredithR05} with the addition of
polyadicity and summation. For readability we have relegated some of
the technical subtleties to an appendix.

\subsubsection{Process grammar}\label{subsub:process_grammar}

\begin{mathpar}
  \inferrule* [lab=synchronization] {} {{M} \bc \pzero \;|\; x?F \;|\; x!C }
  \and
  \inferrule* [lab=abstraction] {} {{F} \bc (x)P}
  \and
  \inferrule* [lab=concretion] {} {{C} \bc \langle Q \rangle}
  \and
  \inferrule* [lab=process] {} {{P,Q} \bc M \;| \;P|Q \;|\; @{x}}
  \and
  \inferrule* [lab=name] {} {{x} \bc \quotep{P}}
\end{mathpar} 

Note that $\vec{x}$ (resp. $\vec{P}$) denotes a vector of names
(resp. processes) of length $|\vec{x}|$ (resp. $|\vec{P}|$). We adopt
the following useful abbreviations.

\begin{mathpar}
   x?(\vec{y}).P := x.(\vec{y})P \and  x\clift{\vec{P}} := x.\clift{\vec{P}}
   \and x!(y) := \lift{x}{\dropn{y}}
   \and \Pi_{i=0}^{n-1}P_i := P_0 | \ldots | P_{n-1}
\end{mathpar}

\subsubsection{Structural congruence}

\paragraph{Free and bound names and alpha-equivalence.} At the
core of structural equivalence is alpha-equivalence which identifies
process that are the same up to a change of variable. Formally, we
recognize the distinction between free and bound names. The free names
of a process, $\freenames{P}$, may be calculated recursively as
follows:

\begin{mathpar}
\freenames{\pzero} := \emptyset
  \and \\
  \freenames{x?(y).P} := \{ x \} \cup (\freenames{P} \setminus \{ y \})
  \and 
  \freenames{x!\langle P \rangle} := \{ x \} \cup \{ P \} 
  \and \\
  \freenames{P|Q} := \freenames{P} \cup \freenames{Q}
  \and \\
  \freenames{@{x}} := \{ x \}
\end{mathpar}

$\pi$
$\quotep{\pi}$

$\freenames{-} : \pi \to \mathcal{P}(\quotep{\pi})$

\begin{eqnarray*}
  \freenames{\pzero} & := & \emptyset \\
  \freenames{x?(y).P} & := & \{ x \} \cup (\freenames{P} \setminus \{ y \}) \\
  \freenames{x!\langle P \rangle} & := & \{ x \} \cup \{ P \} \\
  \freenames{P|Q} & := & \freenames{P} \cup \freenames{Q} \\
  \freenames{\dropn{x}} & := & \{ x \}
\end{eqnarray*}

The bound names of a process, $\boundnames{P}$, are those names occurring in $P$
that are not free. For example, in $x?(y).0$, the name $x$ is free, while $y$ is bound.

\begin{mathpar}
  \inferrule* [lab=monoidal-laws] {} { P|Q \equiv Q|P \and P|0 \equiv P \and P|(Q|R) \equiv (P|Q)|R }
\end{mathpar}

\begin{mathpar}
  \inferrule* [lab=alpha-equivalence] {} { (x)P \equiv (y)P\{y/x\} \and y \not\in \freenames{P} }
\end{mathpar}

\begin{definition}
Then two processes, $P,Q$, are alpha-equivalent if $P = Q\{\vec{y}/\vec{x}\}$ for
some $\vec{x} \in \boundnames{Q},\vec{y} \in \boundnames{P}$, where $Q\{\vec{y}/\vec{x}\}$
denotes the capture-avoiding substitution of $\vec{y}$ for $\vec{x}$ in $Q$.
\end{definition}

\begin{definition}
  The {\em structural congruence} \cite{SangiorgiWalker} , $\equiv$,
  between processes is the least congruence containing
  alpha-equivalence, satisfying the abelian monoid laws
  (associativity, commutativity and $\pzero$ as identity) for parallel
  composition $|$ and for summation $+$.
\end{definition}

\subsection{Name equivalence}

We take name equivalence, written $\nameeq$, to be the smallest
equivalence relation generated by the following rules.

\begin{mathpar}
\inferrule*[lab=Quote-drop]
{ }
{ \quotep{@{x}} \nameeq x }

\inferrule*[lab=Struct-equiv]
{ P \scong Q }
{ \quotep{P} \nameeq \quotep{Q} }
\end{mathpar}

The astute reader will have noticed that the mutual recursion of names
and processes imposes a mutual recursion on alpha-equivalence and
structural equivalence via name-equivalence. Fortunately, all of this
works out pleasantly and we may calculate in the natural way, free of
concern. The reader interested in the details is referred to the
appendix \ref{appendix:rho_details}.

\subsection{Substitution}

We use $\Proc$ for the set of processes, $\QProc$ for the set of
names, and $\id{\{}\vec{y} / \vec{x} \id{\}}$ to denote partial maps,
$s : \QProc \rightarrow \QProc$. A map, $s$ lifts, uniquely, to a map
on process terms, $\widehat{s} : \Proc \rightarrow \Proc$ by the
following equations.

\begin{mathpar}
  (0) \psubstp{Q}{P} := 0 \\
  (R \juxtap S) \psubstp{Q}{P}
  :=    
  (R)\psubstp{Q}{P} \juxtap (S) \psubstp{Q}{P} \\
  (x?(y).R) \psubstp{Q}{P}    
  :=    
  (x)\substp{Q}{P} (z)\concat( (R \psubstn{z}{y}) \psubstp{Q}{P} ) \\
  (\lift{x}{R}) \psubstp{Q}{P}  
  :=
  \lift{(x)\substp{Q}{P}}{ R \psubstp{Q}{P} } \\
%   (\dropn{x})  \psubstp{Q}{P}       
%   := 
%   \left\{ 
%     \begin{array}{ccc} 
%       \dropn{\quotep{Q}} & & x \nameeq \quotep{P} \\
%       \dropn{x} & & otherwise \\
%     \end{array}
%   \right. 
  (\dropn{x})  \psubstp{Q}{P}       
  := 
  \left\{ 
    \begin{array}{ccc} 
      Q & & x \nameeq \quotep{P} \\
      \dropn{x} & & otherwise \\
    \end{array}
  \right.
\end{mathpar}
 

where

\begin{eqnarray}
  (x)\id{\{} \lpquote Q \rpquote / \lpquote P \rpquote \id{\}}            = 
  \left\{ 
    \begin{array}{ccc}
      \lpquote Q \rpquote & & x \nameeq \lpquote P \rpquote \\
      x & & otherwise \\
    \end{array}
  \right. \nonumber
\end{eqnarray}

and $z$ is chosen distinct from $\quotep{P}$, $\quotep{Q}$, the free
names in $Q$, and all the names in $R$. Our $\alpha$-equivalence will
be built in the standard way from this substitution.

\begin{remark}\label{rem:no_self_referential_names}
  One consequence of these definitions is that $\forall P. \quotep{P}
  \not\in \freenames{P}$.
\end{remark}

\subsection{ Dynamic quote: an example }

Anticipating something of what's to come, consider applying the
substitution, $\widehat{\id{\{}u / z \id{\}}}$, to the following pair
of processes, $\lift{w}{y!(z)}$ and $w[ \lpquote y!(z) \rpquote ]$.

\begin{eqnarray}
	\lift{w}{y!(z)}\widehat{\id{\{}u / z \id{\}}}
		& = &
		\lift{w}{y!(u)} \nonumber\\
	w[ \lpquote y!(z) \rpquote ] \widehat{ \id{\{}u / z \id{\}} }
		& = &
		w[ \lpquote y!(z) \rpquote ] \nonumber
\end{eqnarray}

Because the body of the process between quotes is impervious to
substitution, we get radically different answers. In fact, by
examining the first process in an input context,
e.g. $x?(z).\lift{w}{y!(z)}$, we see that the process under the lift
operator may be shaped by prefixed inputs binding a name inside it. In
this sense, the lift operator will be seen as a way to dynamically
construct processes before reifying them as names.

Finally equipped with these standard features we can present the
dynamics of the calculus.

\subsubsection{Operational semantics} 

Finally, we introduce the computational dynamics. What marks these
algebras as distinct from other more traditionally studied algebraic
structures, e.g. vector spaces or polynomial rings, is the manner in
which dynamics is captured. In traditional structures, dynamics is typically
expressed through morphisms between such structures, as in linear maps
between vector spaces or morphisms between rings. In algebras
associated with the semantics of computation, the dynamics is
expressed as part of the algebraic structure itself, through a
reduction reduction relation typically denoted by $\red$. Below, we
give a recursive presentation of this relation for the calculus used
in the encoding.

$\red \subseteq \pi \times \pi$
$\red : \pi \to \mathcal{P}(\pi)$

\begin{mathpar}
  \inferrule* [lab=Comm] { \textsf{match}( x_{src}, x_{trgt} ) } { x_{trgt}?(y)P \; | \; x_{src}!\langle {Q} \rangle \red P\{\quotep{Q}/y}\} }
  \and \\
  \inferrule* [lab=Par] {{P} \red {P}'} {{{P} | {Q}} \red {{P}' | {Q}}}
  \and
  \inferrule* [lab=Equiv]{{{P} \scong {P}'} \andalso {{P}' \red {Q}'} \andalso {{Q}' \scong {Q}}}{{P} \red {Q}}
\end{mathpar}

\begin{eqnarray*}
  match_{\equiv} (\quotep{P},\quotep{Q}) & := & P \equiv Q \\
  match_{\dagger}(\quotep{P},\quotep{Q}) & := & \forall R. P|Q \red^{*} R => R \red^{*} 0 \\
  match_{K}(\quotep{P},\quotep{Q}) & := & K \mbox{ for some context } K
\end{eqnarray*}

$u?(x)P | u!\langle Q \rangle \red P\{\quotep{Q}/x\}$

%We write $\wred$ for $\red^*$, and $P\red$ if $\exists Q $ such that $ P \red Q$.
We write $P\red$ if $\exists Q $ such that $ P \red Q$ and $P\not\red$, otherwise.

\section{Replication}

As mentioned before, it is known that replication (and hence
recursion) can be implemented in a higher-order process algebra
\cite{SangiorgiWalker}. As our first example of calculation with the
machinery thus far presented we give the construction explicitly in
the {\rhoc}.

\begin{eqnarray}
	D_{x} & := & \prefix{x}{y}{(\binpar{\outputp{x}{y}}{@{y}})} \nonumber\\
	\bangp_{x}{P} & := & \binpar{{x}!\langle{\binpar{D_{x}}{P}}\rangle}{D_{x}} \nonumber
\end{eqnarray}

\begin{eqnarray}
	\bangp_{x}{P} & & \nonumber\\
	=
	& {x}!\langle{(\prefix{x}{y}{(\outputp{x}{y} | @{y})) | P}}\rangle 
	      | \prefix{x}{y}{(\outputp{x}{y} | @{y})} & \nonumber\\
	\red
	& (\outputp{x}{y} | @{y})\substn{\quotep{(\prefix{x}{y}{(@{y} | \outputp{x}{y})) | P}}}{y} & \nonumber\\
	=
	& \outputp{x}{\quotep{(\prefix{x}{y}{(\outputp{x}{y} | @{y})) | P}}}
	  | {(\prefix{x}{y}{(\outputp{x}{y} | @{y})) | P}} & \nonumber\\
	\red
	& \ldots & \nonumber\\
	\red^*
	& P | P | \ldots & \nonumber
\end{eqnarray}

Of course, this encoding, as an implementation, runs away, unfolding
$\bangp{P}$ eagerly. A lazier and more implementable replication
operator, restricted to input-guarded processes, may be obtained as follows.

\begin{eqnarray}
\bangp{\prefix{u}{v}{P}} 
	:= 
	\binpar{\lift{x}{\prefix{u}{v}{(\binpar{D(x)}{P})}}}{D(x)} \nonumber
\end{eqnarray}

\begin{remark}
  Note that the lazier definition still does not deal with summation
  or mixed summation (i.e. sums over input and output). The reader is
  invited to construct definitions of replication that deal with these
  features. 

  Further, the definitions are parameterized in a name, $x$. Can you,
  gentle reader, make a definition that eliminates this parameter and
  guarantees no accidental interaction between the replication
  machinery and the process being replicated -- i.e. no accidental
  sharing of names used by the process to get its work done and the
  name(s) used by the replication to effect copying. This latter
  revision of the definition of replication is crucial to obtaining
  the expected identity $!!P \sim !P$.
\end{remark}

\begin{remark}\label{rem:paradoxical_combinator}
  The reader familiar with the lambda calculus will have noticed the
  similarity between $D$ and the paradoxical combinator.

  [Ed. note: the existence of this seems to suggest we have to be more
  restrictive on the set of processes and names we admit if we are to
  support no-cloning.]
\end{remark}

\subsubsection{Bisimulation}

The computational dynamics gives rise to another kind of equivalence,
the equivalence of computational behavior. As previously mentioned
this is typically captured \emph{via} some form of bisimulation.

% The notion we use in this paper is weak barbed bisimulation
% \cite{milner91polyadicpi}.

The notion we use in this paper is derived from weak barbed
bisimulation \cite{milner91polyadicpi}. 

\begin{definition}
An \emph{observation relation}, $\downarrow_{\mathcal N}$, over a set
of names, $\mathcal N$, is the smallest relation satisfying the rules
below.

\infrule[Out-barb]{y \in {\mathcal N}, \; x \nameeq y}
		  {\outputp{x}{v} \downarrow_{\mathcal N} x}
\infrule[Par-barb]{\mbox{$P\downarrow_{\mathcal N} x$ or $Q\downarrow_{\mathcal N} x$}}
		  {\binpar{P}{Q} \downarrow_{\mathcal N} x}

We write $P \Downarrow_{\mathcal N} x$ if there is $Q$ such that 
$P \wred Q$ and $Q \downarrow_{\mathcal N} x$.
\end{definition}

\begin{definition}
%\label{def.bbisim}
An  ${\mathcal N}$-\emph{barbed bisimulation} over a set of names, ${\mathcal N}$, is a symmetric binary relation 
${\mathcal S}_{\mathcal N}$ between agents such that $P\rel{S}_{\mathcal N}Q$ implies:
\begin{enumerate}
\item If $P \red P'$ then $Q \wred Q'$ and $P'\rel{S}_{\mathcal N} Q'$.
\item If $P\downarrow_{\mathcal N} x$, then $Q\Downarrow_{\mathcal N} x$.
\end{enumerate}
$P$ is ${\mathcal N}$-barbed bisimilar to $Q$, written
$P \wbbisim_{\mathcal N} Q$, if $P \rel{S}_{\mathcal N} Q$ for some ${\mathcal N}$-barbed bisimulation ${\mathcal S}_{\mathcal N}$.
\end{definition}

$\mathcal{R} \subseteq \pi \times \pi$

$P \mathcal{R} Q => \forall P'. P \red P' \Rightarrow \exists Q'. Q \red Q', P' \mathcal{R} Q'$

$P \vdash x \Rightarrow Q \vdash x$

\begin{mathpar}
  \inferrule*[lab=Out-barb]{x \nameeq y}{{y}!\langle{Q}\rangle \vdash x}
  \and
  \inferrule*[lab=Par-barb]{\mbox{$P\vdash x$ or $Q\vdash x$}}{\binpar{P}{Q} \vdash x}
\end{mathpar}

\subsubsection{Contexts}

One of the principle advantages of computational calculi like the
$\pi$-calculus is a well-defined notion of context,
contextual-equivalence and a correlation between
contextual-equivalence and notions of bisimulation. The notion of
context allows the decomposition of a process into (sub-)process and
its syntactic environment, its context. Thus, a context may be
thought of as a process with a ``hole'' (written $\Box$) in it. The
application of a context $M$ to a process $P$, written $M[P]$, is
tantamount to filling the hole in $M$ with $P$. In this paper we do
not need the full weight of this theory, but do make use of the notion
of context in the proof the main theorem. 

\begin{mathpar}
  \inferrule* [lab=summation] {} {{M_{M},M_{N}} \bc \Box \;|\; x.M_{A} \;|\; M_{M}+M_{N}}
  \and
  \inferrule* [lab=agent] {} {{M_{A}} \bc (\vec{x})M_{P} \;| \; \clift{P_0,\ldots,M_{P},\ldots,P_N}}
  \and \\
  \inferrule* [lab=process] {} {{M_{P}} \bc M_{N} \;| \;P|M_{P} }
\end{mathpar} 

\begin{mathpar}
  \inferrule* [lab=sychronization] {} {M_{N} \bc \Box \;|\; x?M_{F} \;|\; x!M_{C}}
  \and
  \inferrule* [lab=abstraction] {} {{M_{F}} \bc (x)M_{P} }
  \and
  \inferrule* [lab=concretion] {} {{M_{C}} \bc \langle M_{P} \rangle }
  \and \\
  \inferrule* [lab=process] {} {{M_{P}} \bc M_{N} \;| \;P|M_{P} }
\end{mathpar}

\begin{definition}[contextual application] Given a context $M$, and
  process $P$, we define the \emph{contextual application}, $M[P] :=
  M\{P/\Box\}$. That is, the contextual application of M to P is the
  substitution of $P$ for $\Box$ in $M$.
\end{definition}

$\meaningof{-} : L \to \mathcal{P}(\pi)$

\begin{mathpar}
  \inferrule* [lab=collection] {} {\meaningof{true} = \pi, \and \meaningof{~E} = \pi \setminus \meaningof{E}, \and \meaningof{E_{1} \& E_{2}} = \meaningof{E_{1}} \cap \meaningof{E_{2}}}
\end{mathpar}

\begin{mathpar}
  \inferrule* [lab=structure] {} {\meaningof{0} = \{ P \in \pi | P \equiv 0 \}, \and \\ \meaningof{E_1 | E_2} = \{ P \in \pi | P \equiv P_{1} | P_{2}, P_{1} \in \meaningof{E_{1}}, P_{2} \in \meaningof{E_2}\} }
\end{mathpar}

\begin{mathpar}
 \inferrule* [lab=behavior] {} {\meaningof{\langle a?b \rangle E} = \{ P \in \pi | P \equiv Q | u?(y)P', \\ \and \\\\ \and \\ \;\;\; u \in \meaningof{a}, \forall z.P'\{z/y\} \in \meaningof{E\{z/b\}}\}, \and \\ \meaningof{a!E} = \{ P \in \pi | P \equiv Q | x!\langle P' \rangle, x \in \meaningof{a} P' \in \meaningof{E}\} }
\end{mathpar}

\begin{mathpar}
 \inferrule* [lab=nominal] {} {\meaningof{\quotep{E}} = \{ \quotep{P} \in \quotep{\pi} | P \in \meaningof{E} \}, \and \meaningof{\quotep{P}} = \{ \quotep{Q} \in \quotep{\pi} | P \equiv Q \} \and \\ \meaningof{@\quotep{E}} = \{ P \in \pi | P \equiv @x, x \in \meaningof{E} \}}
\end{mathpar}

\begin{eqnarray*}
  \\
  \meaningof{-} : TS \to ST
\end{eqnarray*}

\begin{eqnarray*}
  \\
  L : TS \to ST
\end{eqnarray*}

\begin{eqnarray*}
  \\
  P \models E \iff P \in \meaningof{E}
\end{eqnarray*}

\begin{eqnarray*}
  P \approx_{L} Q \iff \forall E \in L. P \models E \iff Q \models E
\end{eqnarray*}

\begin{eqnarray*}
  P \approx_{K} Q
\end{eqnarray*}

\begin{eqnarray*}
  P \approx Q
\end{eqnarray*}

$\approx_{K} = \approx = \approx_{L}$

\subsubsection{Contextual duality}

Note that contexts extend the quotation operation to a family of
operations from processes to names. Given a context, $M$, we can
define a \emph{nominal context}, $\quotep{M}$ by $\quotep{M}[P] :=
\quotep{M[P]}$. To foreshadow what is to come we observe that these
operations enjoy a duality with processes very much like the duality
between vectors and maps from vectors to scalars.

Further, because the calculus is essentially higher-order, we have a
correspondence between contexts and processes. More specifically,
given a name $x$ and a context $M$ we can construct $M^{*}_{x}$ such
that 

\begin{mathpar}
  M^{*}_{x} | \lift{x}{P} \red M[P]
\end{mathpar}

namely,

\begin{mathpar}
  M^{*}_{x} := x?(u).M[\dropn{u}]
\end{mathpar}

The dependence of $M^{*}_{x}$ on a name makes it an abstraction, 

\begin{mathpar}
  M^{*} := (x)x?(u).M[\dropn{u}]
\end{mathpar}

\subsection{Additional notation}

It will sometimes be convenient to denote the process a name
quotes. We already have the notation $x = \quotep{P}$, but it will be
convenient to introduce an alternate notation, $\procn{x}$, when we
want to emphasize the connection to the use of the name. Note that, by
virtue of name equivalence, $\quotep{\procn{x}} \nameeq x$; so, the
notation is consistent with previous definitions.

Further, because names have structure it is possible to effect
substitutions on the basis of that structure. This means we need to
upgrade our notation for substitutions, which we accomplish by
adapting comprehension notation. Thus,

\begin{mathpar}
  P\{ y / x : x \in S \}
\end{mathpar}

is interpreted to mean the process derived from P by replacing (in a
capture-avoiding manner) each occurrence of $x$ in $S$ by $y$. For example,

\begin{mathpar}
  P\{ \quotep{\procn{x}|\procn{x}} / x : x \in \freenames{P} \}
\end{mathpar}

will replace each (occurrence) of a free name $x$ in $P$ by
$\quotep{\procn{x}|\procn{x}}$.

Also, we will avail ourselves of the notation $x^{L}$ and $x^{R}$ to
denote injections of a name into disjoint copies of the name
space. There are numerous ways to accomplish this. One example can be
found in \cite{MeredithR05}. This notation overloads to vectors of
names: $\vec{x}^{\pi} := (x_{i}^{\pi} \; : \; 0 \leq i < |\vec{x}| )$ where $\pi \in \{L,R\}$.

We also use $P^{\Box} := P|\Box$.

In \cite{MeredithR05} an interpretation of the new operator is
given. It turns out that there are several possible interpretations
all enjoying the requisite algebraic properties of the operator (see
\cite{milner91polyadicpi}). We will therefore make liberal use of
$(\nu\; \vec{x})P$.

% subsection the_syntax_and_semantics_of_the_notation_system (end)   

\input{qm2pi.qmops} 

\input{qm2pi.sterngerlach} 

\input{qm2pi.metric} 

% section concurrent_process_calculi (end)

%\input{qm2pi.proofsketch}

% section proof sketch (end)

%\input{qm2pi.slviaknots} 

% section spatial logic via knots (end)

\input{qm2pi.conclusion}

% section conclusion (end)

%\input{qm2pi.dtcodes} 

% section wiring algorithm (end)

\input{qm2pi.ack} 

% section acknowledgments (end)

\newpage


\bibliographystyle{plain}   
\bibliography{../../biblios/main.bib}

\input{qm2pi.rhodetails}

\end{document}

 

%\ifpdf
%\usepackage[pdftex]{graphicx}
%\else
%\usepackage{graphicx}
%\fi

 % \ifpdf
%  \usepackage{pdfsync}
%  \if


%\title{Brief Article}
%\author{David F. Snyder}
%\author{L.G. Meredith}

%\address{Dept. of Math., Texas State University--San Marcos, San Marcos, TX 78666}
       
\pagestyle{empty}


\begin{document}

\lstset{language=[Objective]Caml,frame=shadowbox}

\documentclass[12pt]{llncs}
%\documentclass{jktr}

\usepackage[pdftex]{hyperref}                   
\usepackage {listings}
\usepackage {mathpartir}
\usepackage{bcprules}
%\usepackage{listings}
                       
\usepackage{graphicx} 
%\usepackage[margins=2.5cm,nohead,nofoot]{geometry}
%\usepackage{geometry}
\usepackage{amsfonts}
\usepackage{amstext}
\usepackage{latexsym}
\usepackage{amssymb}
\usepackage{color}


%\include{myPreamble}
\include{qm2pi.local} 

%\ifpdf
%\usepackage[pdftex]{graphicx}
%\else
%\usepackage{graphicx}
%\fi

 % \ifpdf
%  \usepackage{pdfsync}
%  \if


%\title{Brief Article}
%\author{David F. Snyder}
%\author{L.G. Meredith}

%\address{Dept. of Math., Texas State University--San Marcos, San Marcos, TX 78666}
       
\pagestyle{empty}


\begin{document}

\lstset{language=[Objective]Caml,frame=shadowbox}

\input{qm2pi.front}

% section front matter (end)

\input{qm2pi.intro} 
 
% section introduction (end)

% \input{qm2pi.knotations} 

% section notation (end)

\input{qm2pi.process.calculi} 

% section concurrent_process_calculi_and_spatial_logics_ (end)
    
%\input{qm2pi.knots2pi} 

%\input{qm2pi.trefoil} 

%\input{qm2pi.mainthm} 

% subsection basic_interpretation (end)

%\input{qm2pi.rho.presentation} 
\subsection{The syntax and semantics of the notation system}\label{sub:the_syntax_and_semantics_of_the_notation_system} % (fold)

We now summarize a technical presentation of the calculus that
embodies our theory of dynamics. The typical presentation of such a
calculus follows the style of giving generators and relations on
them. The grammar, below, describing term constructors, freely
generates the set of processes, $\Proc$. This set is then quotiented
by a relation known as structural congruence and it is over this set
that the notion of dynamics is expressed. This presentation is
essentially that of \cite{MeredithR05} with the addition of
polyadicity and summation. For readability we have relegated some of
the technical subtleties to an appendix.

\subsubsection{Process grammar}\label{subsub:process_grammar}

\begin{mathpar}
  \inferrule* [lab=synchronization] {} {{M} \bc \pzero \;|\; x?F \;|\; x!C }
  \and
  \inferrule* [lab=abstraction] {} {{F} \bc (x)P}
  \and
  \inferrule* [lab=concretion] {} {{C} \bc \langle Q \rangle}
  \and
  \inferrule* [lab=process] {} {{P,Q} \bc M \;| \;P|Q \;|\; @{x}}
  \and
  \inferrule* [lab=name] {} {{x} \bc \quotep{P}}
\end{mathpar} 

Note that $\vec{x}$ (resp. $\vec{P}$) denotes a vector of names
(resp. processes) of length $|\vec{x}|$ (resp. $|\vec{P}|$). We adopt
the following useful abbreviations.

\begin{mathpar}
   x?(\vec{y}).P := x.(\vec{y})P \and  x\clift{\vec{P}} := x.\clift{\vec{P}}
   \and x!(y) := \lift{x}{\dropn{y}}
   \and \Pi_{i=0}^{n-1}P_i := P_0 | \ldots | P_{n-1}
\end{mathpar}

\subsubsection{Structural congruence}

\paragraph{Free and bound names and alpha-equivalence.} At the
core of structural equivalence is alpha-equivalence which identifies
process that are the same up to a change of variable. Formally, we
recognize the distinction between free and bound names. The free names
of a process, $\freenames{P}$, may be calculated recursively as
follows:

\begin{mathpar}
\freenames{\pzero} := \emptyset
  \and \\
  \freenames{x?(y).P} := \{ x \} \cup (\freenames{P} \setminus \{ y \})
  \and 
  \freenames{x!\langle P \rangle} := \{ x \} \cup \{ P \} 
  \and \\
  \freenames{P|Q} := \freenames{P} \cup \freenames{Q}
  \and \\
  \freenames{@{x}} := \{ x \}
\end{mathpar}

$\pi$
$\quotep{\pi}$

$\freenames{-} : \pi \to \mathcal{P}(\quotep{\pi})$

\begin{eqnarray*}
  \freenames{\pzero} & := & \emptyset \\
  \freenames{x?(y).P} & := & \{ x \} \cup (\freenames{P} \setminus \{ y \}) \\
  \freenames{x!\langle P \rangle} & := & \{ x \} \cup \{ P \} \\
  \freenames{P|Q} & := & \freenames{P} \cup \freenames{Q} \\
  \freenames{\dropn{x}} & := & \{ x \}
\end{eqnarray*}

The bound names of a process, $\boundnames{P}$, are those names occurring in $P$
that are not free. For example, in $x?(y).0$, the name $x$ is free, while $y$ is bound.

\begin{mathpar}
  \inferrule* [lab=monoidal-laws] {} { P|Q \equiv Q|P \and P|0 \equiv P \and P|(Q|R) \equiv (P|Q)|R }
\end{mathpar}

\begin{mathpar}
  \inferrule* [lab=alpha-equivalence] {} { (x)P \equiv (y)P\{y/x\} \and y \not\in \freenames{P} }
\end{mathpar}

\begin{definition}
Then two processes, $P,Q$, are alpha-equivalent if $P = Q\{\vec{y}/\vec{x}\}$ for
some $\vec{x} \in \boundnames{Q},\vec{y} \in \boundnames{P}$, where $Q\{\vec{y}/\vec{x}\}$
denotes the capture-avoiding substitution of $\vec{y}$ for $\vec{x}$ in $Q$.
\end{definition}

\begin{definition}
  The {\em structural congruence} \cite{SangiorgiWalker} , $\equiv$,
  between processes is the least congruence containing
  alpha-equivalence, satisfying the abelian monoid laws
  (associativity, commutativity and $\pzero$ as identity) for parallel
  composition $|$ and for summation $+$.
\end{definition}

\subsection{Name equivalence}

We take name equivalence, written $\nameeq$, to be the smallest
equivalence relation generated by the following rules.

\begin{mathpar}
\inferrule*[lab=Quote-drop]
{ }
{ \quotep{@{x}} \nameeq x }

\inferrule*[lab=Struct-equiv]
{ P \scong Q }
{ \quotep{P} \nameeq \quotep{Q} }
\end{mathpar}

The astute reader will have noticed that the mutual recursion of names
and processes imposes a mutual recursion on alpha-equivalence and
structural equivalence via name-equivalence. Fortunately, all of this
works out pleasantly and we may calculate in the natural way, free of
concern. The reader interested in the details is referred to the
appendix \ref{appendix:rho_details}.

\subsection{Substitution}

We use $\Proc$ for the set of processes, $\QProc$ for the set of
names, and $\id{\{}\vec{y} / \vec{x} \id{\}}$ to denote partial maps,
$s : \QProc \rightarrow \QProc$. A map, $s$ lifts, uniquely, to a map
on process terms, $\widehat{s} : \Proc \rightarrow \Proc$ by the
following equations.

\begin{mathpar}
  (0) \psubstp{Q}{P} := 0 \\
  (R \juxtap S) \psubstp{Q}{P}
  :=    
  (R)\psubstp{Q}{P} \juxtap (S) \psubstp{Q}{P} \\
  (x?(y).R) \psubstp{Q}{P}    
  :=    
  (x)\substp{Q}{P} (z)\concat( (R \psubstn{z}{y}) \psubstp{Q}{P} ) \\
  (\lift{x}{R}) \psubstp{Q}{P}  
  :=
  \lift{(x)\substp{Q}{P}}{ R \psubstp{Q}{P} } \\
%   (\dropn{x})  \psubstp{Q}{P}       
%   := 
%   \left\{ 
%     \begin{array}{ccc} 
%       \dropn{\quotep{Q}} & & x \nameeq \quotep{P} \\
%       \dropn{x} & & otherwise \\
%     \end{array}
%   \right. 
  (\dropn{x})  \psubstp{Q}{P}       
  := 
  \left\{ 
    \begin{array}{ccc} 
      Q & & x \nameeq \quotep{P} \\
      \dropn{x} & & otherwise \\
    \end{array}
  \right.
\end{mathpar}
 

where

\begin{eqnarray}
  (x)\id{\{} \lpquote Q \rpquote / \lpquote P \rpquote \id{\}}            = 
  \left\{ 
    \begin{array}{ccc}
      \lpquote Q \rpquote & & x \nameeq \lpquote P \rpquote \\
      x & & otherwise \\
    \end{array}
  \right. \nonumber
\end{eqnarray}

and $z$ is chosen distinct from $\quotep{P}$, $\quotep{Q}$, the free
names in $Q$, and all the names in $R$. Our $\alpha$-equivalence will
be built in the standard way from this substitution.

\begin{remark}\label{rem:no_self_referential_names}
  One consequence of these definitions is that $\forall P. \quotep{P}
  \not\in \freenames{P}$.
\end{remark}

\subsection{ Dynamic quote: an example }

Anticipating something of what's to come, consider applying the
substitution, $\widehat{\id{\{}u / z \id{\}}}$, to the following pair
of processes, $\lift{w}{y!(z)}$ and $w[ \lpquote y!(z) \rpquote ]$.

\begin{eqnarray}
	\lift{w}{y!(z)}\widehat{\id{\{}u / z \id{\}}}
		& = &
		\lift{w}{y!(u)} \nonumber\\
	w[ \lpquote y!(z) \rpquote ] \widehat{ \id{\{}u / z \id{\}} }
		& = &
		w[ \lpquote y!(z) \rpquote ] \nonumber
\end{eqnarray}

Because the body of the process between quotes is impervious to
substitution, we get radically different answers. In fact, by
examining the first process in an input context,
e.g. $x?(z).\lift{w}{y!(z)}$, we see that the process under the lift
operator may be shaped by prefixed inputs binding a name inside it. In
this sense, the lift operator will be seen as a way to dynamically
construct processes before reifying them as names.

Finally equipped with these standard features we can present the
dynamics of the calculus.

\subsubsection{Operational semantics} 

Finally, we introduce the computational dynamics. What marks these
algebras as distinct from other more traditionally studied algebraic
structures, e.g. vector spaces or polynomial rings, is the manner in
which dynamics is captured. In traditional structures, dynamics is typically
expressed through morphisms between such structures, as in linear maps
between vector spaces or morphisms between rings. In algebras
associated with the semantics of computation, the dynamics is
expressed as part of the algebraic structure itself, through a
reduction reduction relation typically denoted by $\red$. Below, we
give a recursive presentation of this relation for the calculus used
in the encoding.

$\red \subseteq \pi \times \pi$
$\red : \pi \to \mathcal{P}(\pi)$

\begin{mathpar}
  \inferrule* [lab=Comm] { \textsf{match}( x_{src}, x_{trgt} ) } { x_{trgt}?(y)P \; | \; x_{src}!\langle {Q} \rangle \red P\{\quotep{Q}/y}\} }
  \and \\
  \inferrule* [lab=Par] {{P} \red {P}'} {{{P} | {Q}} \red {{P}' | {Q}}}
  \and
  \inferrule* [lab=Equiv]{{{P} \scong {P}'} \andalso {{P}' \red {Q}'} \andalso {{Q}' \scong {Q}}}{{P} \red {Q}}
\end{mathpar}

\begin{eqnarray*}
  match_{\equiv} (\quotep{P},\quotep{Q}) & := & P \equiv Q \\
  match_{\dagger}(\quotep{P},\quotep{Q}) & := & \forall R. P|Q \red^{*} R => R \red^{*} 0 \\
  match_{K}(\quotep{P},\quotep{Q}) & := & K \mbox{ for some context } K
\end{eqnarray*}

$u?(x)P | u!\langle Q \rangle \red P\{\quotep{Q}/x\}$

%We write $\wred$ for $\red^*$, and $P\red$ if $\exists Q $ such that $ P \red Q$.
We write $P\red$ if $\exists Q $ such that $ P \red Q$ and $P\not\red$, otherwise.

\section{Replication}

As mentioned before, it is known that replication (and hence
recursion) can be implemented in a higher-order process algebra
\cite{SangiorgiWalker}. As our first example of calculation with the
machinery thus far presented we give the construction explicitly in
the {\rhoc}.

\begin{eqnarray}
	D_{x} & := & \prefix{x}{y}{(\binpar{\outputp{x}{y}}{@{y}})} \nonumber\\
	\bangp_{x}{P} & := & \binpar{{x}!\langle{\binpar{D_{x}}{P}}\rangle}{D_{x}} \nonumber
\end{eqnarray}

\begin{eqnarray}
	\bangp_{x}{P} & & \nonumber\\
	=
	& {x}!\langle{(\prefix{x}{y}{(\outputp{x}{y} | @{y})) | P}}\rangle 
	      | \prefix{x}{y}{(\outputp{x}{y} | @{y})} & \nonumber\\
	\red
	& (\outputp{x}{y} | @{y})\substn{\quotep{(\prefix{x}{y}{(@{y} | \outputp{x}{y})) | P}}}{y} & \nonumber\\
	=
	& \outputp{x}{\quotep{(\prefix{x}{y}{(\outputp{x}{y} | @{y})) | P}}}
	  | {(\prefix{x}{y}{(\outputp{x}{y} | @{y})) | P}} & \nonumber\\
	\red
	& \ldots & \nonumber\\
	\red^*
	& P | P | \ldots & \nonumber
\end{eqnarray}

Of course, this encoding, as an implementation, runs away, unfolding
$\bangp{P}$ eagerly. A lazier and more implementable replication
operator, restricted to input-guarded processes, may be obtained as follows.

\begin{eqnarray}
\bangp{\prefix{u}{v}{P}} 
	:= 
	\binpar{\lift{x}{\prefix{u}{v}{(\binpar{D(x)}{P})}}}{D(x)} \nonumber
\end{eqnarray}

\begin{remark}
  Note that the lazier definition still does not deal with summation
  or mixed summation (i.e. sums over input and output). The reader is
  invited to construct definitions of replication that deal with these
  features. 

  Further, the definitions are parameterized in a name, $x$. Can you,
  gentle reader, make a definition that eliminates this parameter and
  guarantees no accidental interaction between the replication
  machinery and the process being replicated -- i.e. no accidental
  sharing of names used by the process to get its work done and the
  name(s) used by the replication to effect copying. This latter
  revision of the definition of replication is crucial to obtaining
  the expected identity $!!P \sim !P$.
\end{remark}

\begin{remark}\label{rem:paradoxical_combinator}
  The reader familiar with the lambda calculus will have noticed the
  similarity between $D$ and the paradoxical combinator.

  [Ed. note: the existence of this seems to suggest we have to be more
  restrictive on the set of processes and names we admit if we are to
  support no-cloning.]
\end{remark}

\subsubsection{Bisimulation}

The computational dynamics gives rise to another kind of equivalence,
the equivalence of computational behavior. As previously mentioned
this is typically captured \emph{via} some form of bisimulation.

% The notion we use in this paper is weak barbed bisimulation
% \cite{milner91polyadicpi}.

The notion we use in this paper is derived from weak barbed
bisimulation \cite{milner91polyadicpi}. 

\begin{definition}
An \emph{observation relation}, $\downarrow_{\mathcal N}$, over a set
of names, $\mathcal N$, is the smallest relation satisfying the rules
below.

\infrule[Out-barb]{y \in {\mathcal N}, \; x \nameeq y}
		  {\outputp{x}{v} \downarrow_{\mathcal N} x}
\infrule[Par-barb]{\mbox{$P\downarrow_{\mathcal N} x$ or $Q\downarrow_{\mathcal N} x$}}
		  {\binpar{P}{Q} \downarrow_{\mathcal N} x}

We write $P \Downarrow_{\mathcal N} x$ if there is $Q$ such that 
$P \wred Q$ and $Q \downarrow_{\mathcal N} x$.
\end{definition}

\begin{definition}
%\label{def.bbisim}
An  ${\mathcal N}$-\emph{barbed bisimulation} over a set of names, ${\mathcal N}$, is a symmetric binary relation 
${\mathcal S}_{\mathcal N}$ between agents such that $P\rel{S}_{\mathcal N}Q$ implies:
\begin{enumerate}
\item If $P \red P'$ then $Q \wred Q'$ and $P'\rel{S}_{\mathcal N} Q'$.
\item If $P\downarrow_{\mathcal N} x$, then $Q\Downarrow_{\mathcal N} x$.
\end{enumerate}
$P$ is ${\mathcal N}$-barbed bisimilar to $Q$, written
$P \wbbisim_{\mathcal N} Q$, if $P \rel{S}_{\mathcal N} Q$ for some ${\mathcal N}$-barbed bisimulation ${\mathcal S}_{\mathcal N}$.
\end{definition}

$\mathcal{R} \subseteq \pi \times \pi$

$P \mathcal{R} Q => \forall P'. P \red P' \Rightarrow \exists Q'. Q \red Q', P' \mathcal{R} Q'$

$P \vdash x \Rightarrow Q \vdash x$

\begin{mathpar}
  \inferrule*[lab=Out-barb]{x \nameeq y}{{y}!\langle{Q}\rangle \vdash x}
  \and
  \inferrule*[lab=Par-barb]{\mbox{$P\vdash x$ or $Q\vdash x$}}{\binpar{P}{Q} \vdash x}
\end{mathpar}

\subsubsection{Contexts}

One of the principle advantages of computational calculi like the
$\pi$-calculus is a well-defined notion of context,
contextual-equivalence and a correlation between
contextual-equivalence and notions of bisimulation. The notion of
context allows the decomposition of a process into (sub-)process and
its syntactic environment, its context. Thus, a context may be
thought of as a process with a ``hole'' (written $\Box$) in it. The
application of a context $M$ to a process $P$, written $M[P]$, is
tantamount to filling the hole in $M$ with $P$. In this paper we do
not need the full weight of this theory, but do make use of the notion
of context in the proof the main theorem. 

\begin{mathpar}
  \inferrule* [lab=summation] {} {{M_{M},M_{N}} \bc \Box \;|\; x.M_{A} \;|\; M_{M}+M_{N}}
  \and
  \inferrule* [lab=agent] {} {{M_{A}} \bc (\vec{x})M_{P} \;| \; \clift{P_0,\ldots,M_{P},\ldots,P_N}}
  \and \\
  \inferrule* [lab=process] {} {{M_{P}} \bc M_{N} \;| \;P|M_{P} }
\end{mathpar} 

\begin{mathpar}
  \inferrule* [lab=sychronization] {} {M_{N} \bc \Box \;|\; x?M_{F} \;|\; x!M_{C}}
  \and
  \inferrule* [lab=abstraction] {} {{M_{F}} \bc (x)M_{P} }
  \and
  \inferrule* [lab=concretion] {} {{M_{C}} \bc \langle M_{P} \rangle }
  \and \\
  \inferrule* [lab=process] {} {{M_{P}} \bc M_{N} \;| \;P|M_{P} }
\end{mathpar}

\begin{definition}[contextual application] Given a context $M$, and
  process $P$, we define the \emph{contextual application}, $M[P] :=
  M\{P/\Box\}$. That is, the contextual application of M to P is the
  substitution of $P$ for $\Box$ in $M$.
\end{definition}

$\meaningof{-} : L \to \mathcal{P}(\pi)$

\begin{mathpar}
  \inferrule* [lab=collection] {} {\meaningof{true} = \pi, \and \meaningof{~E} = \pi \setminus \meaningof{E}, \and \meaningof{E_{1} \& E_{2}} = \meaningof{E_{1}} \cap \meaningof{E_{2}}}
\end{mathpar}

\begin{mathpar}
  \inferrule* [lab=structure] {} {\meaningof{0} = \{ P \in \pi | P \equiv 0 \}, \and \\ \meaningof{E_1 | E_2} = \{ P \in \pi | P \equiv P_{1} | P_{2}, P_{1} \in \meaningof{E_{1}}, P_{2} \in \meaningof{E_2}\} }
\end{mathpar}

\begin{mathpar}
 \inferrule* [lab=behavior] {} {\meaningof{\langle a?b \rangle E} = \{ P \in \pi | P \equiv Q | u?(y)P', \\ \and \\\\ \and \\ \;\;\; u \in \meaningof{a}, \forall z.P'\{z/y\} \in \meaningof{E\{z/b\}}\}, \and \\ \meaningof{a!E} = \{ P \in \pi | P \equiv Q | x!\langle P' \rangle, x \in \meaningof{a} P' \in \meaningof{E}\} }
\end{mathpar}

\begin{mathpar}
 \inferrule* [lab=nominal] {} {\meaningof{\quotep{E}} = \{ \quotep{P} \in \quotep{\pi} | P \in \meaningof{E} \}, \and \meaningof{\quotep{P}} = \{ \quotep{Q} \in \quotep{\pi} | P \equiv Q \} \and \\ \meaningof{@\quotep{E}} = \{ P \in \pi | P \equiv @x, x \in \meaningof{E} \}}
\end{mathpar}

\begin{eqnarray*}
  \\
  \meaningof{-} : TS \to ST
\end{eqnarray*}

\begin{eqnarray*}
  \\
  L : TS \to ST
\end{eqnarray*}

\begin{eqnarray*}
  \\
  P \models E \iff P \in \meaningof{E}
\end{eqnarray*}

\begin{eqnarray*}
  P \approx_{L} Q \iff \forall E \in L. P \models E \iff Q \models E
\end{eqnarray*}

\begin{eqnarray*}
  P \approx_{K} Q
\end{eqnarray*}

\begin{eqnarray*}
  P \approx Q
\end{eqnarray*}

$\approx_{K} = \approx = \approx_{L}$

\subsubsection{Contextual duality}

Note that contexts extend the quotation operation to a family of
operations from processes to names. Given a context, $M$, we can
define a \emph{nominal context}, $\quotep{M}$ by $\quotep{M}[P] :=
\quotep{M[P]}$. To foreshadow what is to come we observe that these
operations enjoy a duality with processes very much like the duality
between vectors and maps from vectors to scalars.

Further, because the calculus is essentially higher-order, we have a
correspondence between contexts and processes. More specifically,
given a name $x$ and a context $M$ we can construct $M^{*}_{x}$ such
that 

\begin{mathpar}
  M^{*}_{x} | \lift{x}{P} \red M[P]
\end{mathpar}

namely,

\begin{mathpar}
  M^{*}_{x} := x?(u).M[\dropn{u}]
\end{mathpar}

The dependence of $M^{*}_{x}$ on a name makes it an abstraction, 

\begin{mathpar}
  M^{*} := (x)x?(u).M[\dropn{u}]
\end{mathpar}

\subsection{Additional notation}

It will sometimes be convenient to denote the process a name
quotes. We already have the notation $x = \quotep{P}$, but it will be
convenient to introduce an alternate notation, $\procn{x}$, when we
want to emphasize the connection to the use of the name. Note that, by
virtue of name equivalence, $\quotep{\procn{x}} \nameeq x$; so, the
notation is consistent with previous definitions.

Further, because names have structure it is possible to effect
substitutions on the basis of that structure. This means we need to
upgrade our notation for substitutions, which we accomplish by
adapting comprehension notation. Thus,

\begin{mathpar}
  P\{ y / x : x \in S \}
\end{mathpar}

is interpreted to mean the process derived from P by replacing (in a
capture-avoiding manner) each occurrence of $x$ in $S$ by $y$. For example,

\begin{mathpar}
  P\{ \quotep{\procn{x}|\procn{x}} / x : x \in \freenames{P} \}
\end{mathpar}

will replace each (occurrence) of a free name $x$ in $P$ by
$\quotep{\procn{x}|\procn{x}}$.

Also, we will avail ourselves of the notation $x^{L}$ and $x^{R}$ to
denote injections of a name into disjoint copies of the name
space. There are numerous ways to accomplish this. One example can be
found in \cite{MeredithR05}. This notation overloads to vectors of
names: $\vec{x}^{\pi} := (x_{i}^{\pi} \; : \; 0 \leq i < |\vec{x}| )$ where $\pi \in \{L,R\}$.

We also use $P^{\Box} := P|\Box$.

In \cite{MeredithR05} an interpretation of the new operator is
given. It turns out that there are several possible interpretations
all enjoying the requisite algebraic properties of the operator (see
\cite{milner91polyadicpi}). We will therefore make liberal use of
$(\nu\; \vec{x})P$.

% subsection the_syntax_and_semantics_of_the_notation_system (end)   

\input{qm2pi.qmops} 

\input{qm2pi.sterngerlach} 

\input{qm2pi.metric} 

% section concurrent_process_calculi (end)

%\input{qm2pi.proofsketch}

% section proof sketch (end)

%\input{qm2pi.slviaknots} 

% section spatial logic via knots (end)

\input{qm2pi.conclusion}

% section conclusion (end)

%\input{qm2pi.dtcodes} 

% section wiring algorithm (end)

\input{qm2pi.ack} 

% section acknowledgments (end)

\newpage


\bibliographystyle{plain}   
\bibliography{../../biblios/main.bib}

\input{qm2pi.rhodetails}

\end{document}



% section front matter (end)

\section{Introduction}\label{sec:introduction} % (fold)
In this draft of the material i am going to have to dispense with the
usual writing conventions adopted in papers on these topics. i'm going
to have adopt whatever tone i need at the time i'm writing up the
calculations. Sometimes this may be very conversational; others it may
be the barest mathematical grunts; others still it may be that i have
lifted text from one of my other papers because the exposition of some
point was better said there. i hope that my readers are not unduly put
out by this decision. i'm not doing this to flout convention or be
rebellious. i find these calculations very technically challenging. To
keep everything going technically, something has to give; i have to
let go of some cognitive burden. So, the academic writing style --
with all of its trade-offs in terms of facilitating technical
communication -- is what i'm letting go of. Perhaps subsequent drafts
can be tightened and polished, but for now, i'm going to speak as if
we were sitting together in a coffee shop with a laptop, wifi and a
pad of paper and a pencil.

So, here's what i have to say. We -- you and i, comfortably ensconced
in our coffee shop and well-equipped with our tools -- can realize and
carry out the calculations of quantum mechanics over a very different
formal theory of dynamics, a formal theory of dynamics that
corresponds to a theory of concurrent computation with
\emph{reflection}. It has the advantage that the underlying theory is
already `quantized', but supports analogues all of the continuuous
operations. Strikingly, this underlying theory has recently been
connected with a notion of metric that we can show, by calculating
together, coincides with the metric induced by the inner product.

There are a lot of reasons why you might be interested in seeing
calculations of this form. Here's why i'm interested. For the past
several centuries there has been no competitor to the ``Newtonian''
account of dynamics. As a result the predominant share of accounts of
dynamical systems and situations have had to be formulated in terms of
the Newtonian machinery. i view this as an intellectually dangerous
position to occupy. Everything, despite it's intrinsic shape, turns
into a nail to be hit with this hammer. Recently, however, the theory
of computation has matured to the point where we have candidates for
theories of dynamics that offer very different perspective on
reasoning about dynamical systems and situations. Testing these
candidates against very successful accounts of dynamical situations,
like quantum mechanics, is going to give us some sense of how mature
they are and some measure of the quality of these accounts of
dynamics.

\subsection{Summary of contributions and outline of paper}

So, we're going to develop an interpretation of the operations of
quantum mechanics normally interpreted by Hilbert spaces and
operators. We're going to do this over a theory of computation. Note
that this is very different than the usual quantum computation program
which develops notions of computation over quantum mechanics. Rather,
we are developing a story that aligns with Wheeler's slogan: It from
Bit. To do this we will first provide an account of the theory of
computation at play here. Then we will dive into a calculation-driven
interpretation of the operations of quantum mechanics.

The reason we take this approach is that -- until very recently --
there hasn't been an axiomatic account of quantum mechanics. As a
result there has been no sharp delineation of the mathematical theory
supporting interpretation of the physical theory and the physical
theory, itself. So, ambient features of the maths are free to be
exploited (or supressed) without a real accounting of their physical
relevance. There is no sharp statement ``here's the physical theory''
qua \emph{theory} and ``here's the mathematical interpretation''
enabling a judgment of how faithful the interpretation is -- apart
from experimental observation. When there is an axiomatic account we
can judge how well a given mathematical formalism supports an
interpretation of the axioms, independent of
experimentation. Likewise, we can judge how well we have captured our
physical evidence and experience with our axiomatics, independent of
any specific mathematical implementation, with accidental detail that
may or may not have physical significance. 

In lieu of a fully fleshed out and vetted axiomatic account of quantum
mechanics, interpreting the operational notions in service of modeling
physical systems will have to suffice. In other words, we are not in
the business of providing a model of Hilbert spaces and operators. We
are in the business of providing a model of quantum mechanics because
we are motivated by testing our notions of dynamics against physical
theory; and, the predictive calculations of the physical theory must
serve as the best formulation -- shy of a fully fleshed out axiomatic
account -- of the physical theory itself (as they have for scientific
theories since time immemorial). Put another way, despite a
whole-hearted commitment to an It-from-Bit ontology, we are firmly
aligned with the shut-up-and-calculate camp as the best way to obtain
results either from the physical perspective or as a quality assurance
measure of our fledgling theory of dynamics.

In detail, we present a reflective process calculus. Then we develop
intuitive correspondences between the notions available in this
calculus and the usual physical notions supporting quantum mechanical
calculations. Thus, 

\begin{table}[htp]
  \center{
    \fbox{
      \begin{tabular}{c|c}
        quantum mechanics & process calculus \\
        \hline
        scalar & name \\
        state vector & process \\
        dual & contextual duals \\
        matrix & formal sums of process-context-dual pairs \\
        orthogonality & process annihilation \\
        inner product & execution-formula + quoting
      \end{tabular}
    }
  }
  \caption{QM - process calculi correspondences}
\end{table}

Then we tighten up these intuitions to operational definitions. We
employ the Dirac notation as the best proxy we can find for an
abstract syntax of the quantum mechanical notions. The definitions we
develop put us in contact with equational constraints coming from the
theory that we demonstrate the definitions and calculations satisfy.

This puts us in a position to shut up and calculate for the
Stern-Gerlach experimental set up, showing how these predictive
calculations become calculations on processes in our theory of a
reflective process calculus.

Penultimately, we demonstrate that the notion of metric coming from
the inner product coincides with the notion of metric available from
the theory of bisimulation. This demonstration gives us the right to
think of space as arising from behavior. Finally, we consider where we
might go from the new vantage point we have obtained.

% section introduction (end) 
 
% section introduction (end)

% \documentclass[12pt]{llncs}
%\documentclass{jktr}

\usepackage[pdftex]{hyperref}                   
\usepackage {listings}
\usepackage {mathpartir}
\usepackage{bcprules}
%\usepackage{listings}
                       
\usepackage{graphicx} 
%\usepackage[margins=2.5cm,nohead,nofoot]{geometry}
%\usepackage{geometry}
\usepackage{amsfonts}
\usepackage{amstext}
\usepackage{latexsym}
\usepackage{amssymb}
\usepackage{color}


%\include{myPreamble}
\include{qm2pi.local} 

%\ifpdf
%\usepackage[pdftex]{graphicx}
%\else
%\usepackage{graphicx}
%\fi

 % \ifpdf
%  \usepackage{pdfsync}
%  \if


%\title{Brief Article}
%\author{David F. Snyder}
%\author{L.G. Meredith}

%\address{Dept. of Math., Texas State University--San Marcos, San Marcos, TX 78666}
       
\pagestyle{empty}


\begin{document}

\lstset{language=[Objective]Caml,frame=shadowbox}

\input{qm2pi.front}

% section front matter (end)

\input{qm2pi.intro} 
 
% section introduction (end)

% \input{qm2pi.knotations} 

% section notation (end)

\input{qm2pi.process.calculi} 

% section concurrent_process_calculi_and_spatial_logics_ (end)
    
%\input{qm2pi.knots2pi} 

%\input{qm2pi.trefoil} 

%\input{qm2pi.mainthm} 

% subsection basic_interpretation (end)

%\input{qm2pi.rho.presentation} 
\subsection{The syntax and semantics of the notation system}\label{sub:the_syntax_and_semantics_of_the_notation_system} % (fold)

We now summarize a technical presentation of the calculus that
embodies our theory of dynamics. The typical presentation of such a
calculus follows the style of giving generators and relations on
them. The grammar, below, describing term constructors, freely
generates the set of processes, $\Proc$. This set is then quotiented
by a relation known as structural congruence and it is over this set
that the notion of dynamics is expressed. This presentation is
essentially that of \cite{MeredithR05} with the addition of
polyadicity and summation. For readability we have relegated some of
the technical subtleties to an appendix.

\subsubsection{Process grammar}\label{subsub:process_grammar}

\begin{mathpar}
  \inferrule* [lab=synchronization] {} {{M} \bc \pzero \;|\; x?F \;|\; x!C }
  \and
  \inferrule* [lab=abstraction] {} {{F} \bc (x)P}
  \and
  \inferrule* [lab=concretion] {} {{C} \bc \langle Q \rangle}
  \and
  \inferrule* [lab=process] {} {{P,Q} \bc M \;| \;P|Q \;|\; @{x}}
  \and
  \inferrule* [lab=name] {} {{x} \bc \quotep{P}}
\end{mathpar} 

Note that $\vec{x}$ (resp. $\vec{P}$) denotes a vector of names
(resp. processes) of length $|\vec{x}|$ (resp. $|\vec{P}|$). We adopt
the following useful abbreviations.

\begin{mathpar}
   x?(\vec{y}).P := x.(\vec{y})P \and  x\clift{\vec{P}} := x.\clift{\vec{P}}
   \and x!(y) := \lift{x}{\dropn{y}}
   \and \Pi_{i=0}^{n-1}P_i := P_0 | \ldots | P_{n-1}
\end{mathpar}

\subsubsection{Structural congruence}

\paragraph{Free and bound names and alpha-equivalence.} At the
core of structural equivalence is alpha-equivalence which identifies
process that are the same up to a change of variable. Formally, we
recognize the distinction between free and bound names. The free names
of a process, $\freenames{P}$, may be calculated recursively as
follows:

\begin{mathpar}
\freenames{\pzero} := \emptyset
  \and \\
  \freenames{x?(y).P} := \{ x \} \cup (\freenames{P} \setminus \{ y \})
  \and 
  \freenames{x!\langle P \rangle} := \{ x \} \cup \{ P \} 
  \and \\
  \freenames{P|Q} := \freenames{P} \cup \freenames{Q}
  \and \\
  \freenames{@{x}} := \{ x \}
\end{mathpar}

$\pi$
$\quotep{\pi}$

$\freenames{-} : \pi \to \mathcal{P}(\quotep{\pi})$

\begin{eqnarray*}
  \freenames{\pzero} & := & \emptyset \\
  \freenames{x?(y).P} & := & \{ x \} \cup (\freenames{P} \setminus \{ y \}) \\
  \freenames{x!\langle P \rangle} & := & \{ x \} \cup \{ P \} \\
  \freenames{P|Q} & := & \freenames{P} \cup \freenames{Q} \\
  \freenames{\dropn{x}} & := & \{ x \}
\end{eqnarray*}

The bound names of a process, $\boundnames{P}$, are those names occurring in $P$
that are not free. For example, in $x?(y).0$, the name $x$ is free, while $y$ is bound.

\begin{mathpar}
  \inferrule* [lab=monoidal-laws] {} { P|Q \equiv Q|P \and P|0 \equiv P \and P|(Q|R) \equiv (P|Q)|R }
\end{mathpar}

\begin{mathpar}
  \inferrule* [lab=alpha-equivalence] {} { (x)P \equiv (y)P\{y/x\} \and y \not\in \freenames{P} }
\end{mathpar}

\begin{definition}
Then two processes, $P,Q$, are alpha-equivalent if $P = Q\{\vec{y}/\vec{x}\}$ for
some $\vec{x} \in \boundnames{Q},\vec{y} \in \boundnames{P}$, where $Q\{\vec{y}/\vec{x}\}$
denotes the capture-avoiding substitution of $\vec{y}$ for $\vec{x}$ in $Q$.
\end{definition}

\begin{definition}
  The {\em structural congruence} \cite{SangiorgiWalker} , $\equiv$,
  between processes is the least congruence containing
  alpha-equivalence, satisfying the abelian monoid laws
  (associativity, commutativity and $\pzero$ as identity) for parallel
  composition $|$ and for summation $+$.
\end{definition}

\subsection{Name equivalence}

We take name equivalence, written $\nameeq$, to be the smallest
equivalence relation generated by the following rules.

\begin{mathpar}
\inferrule*[lab=Quote-drop]
{ }
{ \quotep{@{x}} \nameeq x }

\inferrule*[lab=Struct-equiv]
{ P \scong Q }
{ \quotep{P} \nameeq \quotep{Q} }
\end{mathpar}

The astute reader will have noticed that the mutual recursion of names
and processes imposes a mutual recursion on alpha-equivalence and
structural equivalence via name-equivalence. Fortunately, all of this
works out pleasantly and we may calculate in the natural way, free of
concern. The reader interested in the details is referred to the
appendix \ref{appendix:rho_details}.

\subsection{Substitution}

We use $\Proc$ for the set of processes, $\QProc$ for the set of
names, and $\id{\{}\vec{y} / \vec{x} \id{\}}$ to denote partial maps,
$s : \QProc \rightarrow \QProc$. A map, $s$ lifts, uniquely, to a map
on process terms, $\widehat{s} : \Proc \rightarrow \Proc$ by the
following equations.

\begin{mathpar}
  (0) \psubstp{Q}{P} := 0 \\
  (R \juxtap S) \psubstp{Q}{P}
  :=    
  (R)\psubstp{Q}{P} \juxtap (S) \psubstp{Q}{P} \\
  (x?(y).R) \psubstp{Q}{P}    
  :=    
  (x)\substp{Q}{P} (z)\concat( (R \psubstn{z}{y}) \psubstp{Q}{P} ) \\
  (\lift{x}{R}) \psubstp{Q}{P}  
  :=
  \lift{(x)\substp{Q}{P}}{ R \psubstp{Q}{P} } \\
%   (\dropn{x})  \psubstp{Q}{P}       
%   := 
%   \left\{ 
%     \begin{array}{ccc} 
%       \dropn{\quotep{Q}} & & x \nameeq \quotep{P} \\
%       \dropn{x} & & otherwise \\
%     \end{array}
%   \right. 
  (\dropn{x})  \psubstp{Q}{P}       
  := 
  \left\{ 
    \begin{array}{ccc} 
      Q & & x \nameeq \quotep{P} \\
      \dropn{x} & & otherwise \\
    \end{array}
  \right.
\end{mathpar}
 

where

\begin{eqnarray}
  (x)\id{\{} \lpquote Q \rpquote / \lpquote P \rpquote \id{\}}            = 
  \left\{ 
    \begin{array}{ccc}
      \lpquote Q \rpquote & & x \nameeq \lpquote P \rpquote \\
      x & & otherwise \\
    \end{array}
  \right. \nonumber
\end{eqnarray}

and $z$ is chosen distinct from $\quotep{P}$, $\quotep{Q}$, the free
names in $Q$, and all the names in $R$. Our $\alpha$-equivalence will
be built in the standard way from this substitution.

\begin{remark}\label{rem:no_self_referential_names}
  One consequence of these definitions is that $\forall P. \quotep{P}
  \not\in \freenames{P}$.
\end{remark}

\subsection{ Dynamic quote: an example }

Anticipating something of what's to come, consider applying the
substitution, $\widehat{\id{\{}u / z \id{\}}}$, to the following pair
of processes, $\lift{w}{y!(z)}$ and $w[ \lpquote y!(z) \rpquote ]$.

\begin{eqnarray}
	\lift{w}{y!(z)}\widehat{\id{\{}u / z \id{\}}}
		& = &
		\lift{w}{y!(u)} \nonumber\\
	w[ \lpquote y!(z) \rpquote ] \widehat{ \id{\{}u / z \id{\}} }
		& = &
		w[ \lpquote y!(z) \rpquote ] \nonumber
\end{eqnarray}

Because the body of the process between quotes is impervious to
substitution, we get radically different answers. In fact, by
examining the first process in an input context,
e.g. $x?(z).\lift{w}{y!(z)}$, we see that the process under the lift
operator may be shaped by prefixed inputs binding a name inside it. In
this sense, the lift operator will be seen as a way to dynamically
construct processes before reifying them as names.

Finally equipped with these standard features we can present the
dynamics of the calculus.

\subsubsection{Operational semantics} 

Finally, we introduce the computational dynamics. What marks these
algebras as distinct from other more traditionally studied algebraic
structures, e.g. vector spaces or polynomial rings, is the manner in
which dynamics is captured. In traditional structures, dynamics is typically
expressed through morphisms between such structures, as in linear maps
between vector spaces or morphisms between rings. In algebras
associated with the semantics of computation, the dynamics is
expressed as part of the algebraic structure itself, through a
reduction reduction relation typically denoted by $\red$. Below, we
give a recursive presentation of this relation for the calculus used
in the encoding.

$\red \subseteq \pi \times \pi$
$\red : \pi \to \mathcal{P}(\pi)$

\begin{mathpar}
  \inferrule* [lab=Comm] { \textsf{match}( x_{src}, x_{trgt} ) } { x_{trgt}?(y)P \; | \; x_{src}!\langle {Q} \rangle \red P\{\quotep{Q}/y}\} }
  \and \\
  \inferrule* [lab=Par] {{P} \red {P}'} {{{P} | {Q}} \red {{P}' | {Q}}}
  \and
  \inferrule* [lab=Equiv]{{{P} \scong {P}'} \andalso {{P}' \red {Q}'} \andalso {{Q}' \scong {Q}}}{{P} \red {Q}}
\end{mathpar}

\begin{eqnarray*}
  match_{\equiv} (\quotep{P},\quotep{Q}) & := & P \equiv Q \\
  match_{\dagger}(\quotep{P},\quotep{Q}) & := & \forall R. P|Q \red^{*} R => R \red^{*} 0 \\
  match_{K}(\quotep{P},\quotep{Q}) & := & K \mbox{ for some context } K
\end{eqnarray*}

$u?(x)P | u!\langle Q \rangle \red P\{\quotep{Q}/x\}$

%We write $\wred$ for $\red^*$, and $P\red$ if $\exists Q $ such that $ P \red Q$.
We write $P\red$ if $\exists Q $ such that $ P \red Q$ and $P\not\red$, otherwise.

\section{Replication}

As mentioned before, it is known that replication (and hence
recursion) can be implemented in a higher-order process algebra
\cite{SangiorgiWalker}. As our first example of calculation with the
machinery thus far presented we give the construction explicitly in
the {\rhoc}.

\begin{eqnarray}
	D_{x} & := & \prefix{x}{y}{(\binpar{\outputp{x}{y}}{@{y}})} \nonumber\\
	\bangp_{x}{P} & := & \binpar{{x}!\langle{\binpar{D_{x}}{P}}\rangle}{D_{x}} \nonumber
\end{eqnarray}

\begin{eqnarray}
	\bangp_{x}{P} & & \nonumber\\
	=
	& {x}!\langle{(\prefix{x}{y}{(\outputp{x}{y} | @{y})) | P}}\rangle 
	      | \prefix{x}{y}{(\outputp{x}{y} | @{y})} & \nonumber\\
	\red
	& (\outputp{x}{y} | @{y})\substn{\quotep{(\prefix{x}{y}{(@{y} | \outputp{x}{y})) | P}}}{y} & \nonumber\\
	=
	& \outputp{x}{\quotep{(\prefix{x}{y}{(\outputp{x}{y} | @{y})) | P}}}
	  | {(\prefix{x}{y}{(\outputp{x}{y} | @{y})) | P}} & \nonumber\\
	\red
	& \ldots & \nonumber\\
	\red^*
	& P | P | \ldots & \nonumber
\end{eqnarray}

Of course, this encoding, as an implementation, runs away, unfolding
$\bangp{P}$ eagerly. A lazier and more implementable replication
operator, restricted to input-guarded processes, may be obtained as follows.

\begin{eqnarray}
\bangp{\prefix{u}{v}{P}} 
	:= 
	\binpar{\lift{x}{\prefix{u}{v}{(\binpar{D(x)}{P})}}}{D(x)} \nonumber
\end{eqnarray}

\begin{remark}
  Note that the lazier definition still does not deal with summation
  or mixed summation (i.e. sums over input and output). The reader is
  invited to construct definitions of replication that deal with these
  features. 

  Further, the definitions are parameterized in a name, $x$. Can you,
  gentle reader, make a definition that eliminates this parameter and
  guarantees no accidental interaction between the replication
  machinery and the process being replicated -- i.e. no accidental
  sharing of names used by the process to get its work done and the
  name(s) used by the replication to effect copying. This latter
  revision of the definition of replication is crucial to obtaining
  the expected identity $!!P \sim !P$.
\end{remark}

\begin{remark}\label{rem:paradoxical_combinator}
  The reader familiar with the lambda calculus will have noticed the
  similarity between $D$ and the paradoxical combinator.

  [Ed. note: the existence of this seems to suggest we have to be more
  restrictive on the set of processes and names we admit if we are to
  support no-cloning.]
\end{remark}

\subsubsection{Bisimulation}

The computational dynamics gives rise to another kind of equivalence,
the equivalence of computational behavior. As previously mentioned
this is typically captured \emph{via} some form of bisimulation.

% The notion we use in this paper is weak barbed bisimulation
% \cite{milner91polyadicpi}.

The notion we use in this paper is derived from weak barbed
bisimulation \cite{milner91polyadicpi}. 

\begin{definition}
An \emph{observation relation}, $\downarrow_{\mathcal N}$, over a set
of names, $\mathcal N$, is the smallest relation satisfying the rules
below.

\infrule[Out-barb]{y \in {\mathcal N}, \; x \nameeq y}
		  {\outputp{x}{v} \downarrow_{\mathcal N} x}
\infrule[Par-barb]{\mbox{$P\downarrow_{\mathcal N} x$ or $Q\downarrow_{\mathcal N} x$}}
		  {\binpar{P}{Q} \downarrow_{\mathcal N} x}

We write $P \Downarrow_{\mathcal N} x$ if there is $Q$ such that 
$P \wred Q$ and $Q \downarrow_{\mathcal N} x$.
\end{definition}

\begin{definition}
%\label{def.bbisim}
An  ${\mathcal N}$-\emph{barbed bisimulation} over a set of names, ${\mathcal N}$, is a symmetric binary relation 
${\mathcal S}_{\mathcal N}$ between agents such that $P\rel{S}_{\mathcal N}Q$ implies:
\begin{enumerate}
\item If $P \red P'$ then $Q \wred Q'$ and $P'\rel{S}_{\mathcal N} Q'$.
\item If $P\downarrow_{\mathcal N} x$, then $Q\Downarrow_{\mathcal N} x$.
\end{enumerate}
$P$ is ${\mathcal N}$-barbed bisimilar to $Q$, written
$P \wbbisim_{\mathcal N} Q$, if $P \rel{S}_{\mathcal N} Q$ for some ${\mathcal N}$-barbed bisimulation ${\mathcal S}_{\mathcal N}$.
\end{definition}

$\mathcal{R} \subseteq \pi \times \pi$

$P \mathcal{R} Q => \forall P'. P \red P' \Rightarrow \exists Q'. Q \red Q', P' \mathcal{R} Q'$

$P \vdash x \Rightarrow Q \vdash x$

\begin{mathpar}
  \inferrule*[lab=Out-barb]{x \nameeq y}{{y}!\langle{Q}\rangle \vdash x}
  \and
  \inferrule*[lab=Par-barb]{\mbox{$P\vdash x$ or $Q\vdash x$}}{\binpar{P}{Q} \vdash x}
\end{mathpar}

\subsubsection{Contexts}

One of the principle advantages of computational calculi like the
$\pi$-calculus is a well-defined notion of context,
contextual-equivalence and a correlation between
contextual-equivalence and notions of bisimulation. The notion of
context allows the decomposition of a process into (sub-)process and
its syntactic environment, its context. Thus, a context may be
thought of as a process with a ``hole'' (written $\Box$) in it. The
application of a context $M$ to a process $P$, written $M[P]$, is
tantamount to filling the hole in $M$ with $P$. In this paper we do
not need the full weight of this theory, but do make use of the notion
of context in the proof the main theorem. 

\begin{mathpar}
  \inferrule* [lab=summation] {} {{M_{M},M_{N}} \bc \Box \;|\; x.M_{A} \;|\; M_{M}+M_{N}}
  \and
  \inferrule* [lab=agent] {} {{M_{A}} \bc (\vec{x})M_{P} \;| \; \clift{P_0,\ldots,M_{P},\ldots,P_N}}
  \and \\
  \inferrule* [lab=process] {} {{M_{P}} \bc M_{N} \;| \;P|M_{P} }
\end{mathpar} 

\begin{mathpar}
  \inferrule* [lab=sychronization] {} {M_{N} \bc \Box \;|\; x?M_{F} \;|\; x!M_{C}}
  \and
  \inferrule* [lab=abstraction] {} {{M_{F}} \bc (x)M_{P} }
  \and
  \inferrule* [lab=concretion] {} {{M_{C}} \bc \langle M_{P} \rangle }
  \and \\
  \inferrule* [lab=process] {} {{M_{P}} \bc M_{N} \;| \;P|M_{P} }
\end{mathpar}

\begin{definition}[contextual application] Given a context $M$, and
  process $P$, we define the \emph{contextual application}, $M[P] :=
  M\{P/\Box\}$. That is, the contextual application of M to P is the
  substitution of $P$ for $\Box$ in $M$.
\end{definition}

$\meaningof{-} : L \to \mathcal{P}(\pi)$

\begin{mathpar}
  \inferrule* [lab=collection] {} {\meaningof{true} = \pi, \and \meaningof{~E} = \pi \setminus \meaningof{E}, \and \meaningof{E_{1} \& E_{2}} = \meaningof{E_{1}} \cap \meaningof{E_{2}}}
\end{mathpar}

\begin{mathpar}
  \inferrule* [lab=structure] {} {\meaningof{0} = \{ P \in \pi | P \equiv 0 \}, \and \\ \meaningof{E_1 | E_2} = \{ P \in \pi | P \equiv P_{1} | P_{2}, P_{1} \in \meaningof{E_{1}}, P_{2} \in \meaningof{E_2}\} }
\end{mathpar}

\begin{mathpar}
 \inferrule* [lab=behavior] {} {\meaningof{\langle a?b \rangle E} = \{ P \in \pi | P \equiv Q | u?(y)P', \\ \and \\\\ \and \\ \;\;\; u \in \meaningof{a}, \forall z.P'\{z/y\} \in \meaningof{E\{z/b\}}\}, \and \\ \meaningof{a!E} = \{ P \in \pi | P \equiv Q | x!\langle P' \rangle, x \in \meaningof{a} P' \in \meaningof{E}\} }
\end{mathpar}

\begin{mathpar}
 \inferrule* [lab=nominal] {} {\meaningof{\quotep{E}} = \{ \quotep{P} \in \quotep{\pi} | P \in \meaningof{E} \}, \and \meaningof{\quotep{P}} = \{ \quotep{Q} \in \quotep{\pi} | P \equiv Q \} \and \\ \meaningof{@\quotep{E}} = \{ P \in \pi | P \equiv @x, x \in \meaningof{E} \}}
\end{mathpar}

\begin{eqnarray*}
  \\
  \meaningof{-} : TS \to ST
\end{eqnarray*}

\begin{eqnarray*}
  \\
  L : TS \to ST
\end{eqnarray*}

\begin{eqnarray*}
  \\
  P \models E \iff P \in \meaningof{E}
\end{eqnarray*}

\begin{eqnarray*}
  P \approx_{L} Q \iff \forall E \in L. P \models E \iff Q \models E
\end{eqnarray*}

\begin{eqnarray*}
  P \approx_{K} Q
\end{eqnarray*}

\begin{eqnarray*}
  P \approx Q
\end{eqnarray*}

$\approx_{K} = \approx = \approx_{L}$

\subsubsection{Contextual duality}

Note that contexts extend the quotation operation to a family of
operations from processes to names. Given a context, $M$, we can
define a \emph{nominal context}, $\quotep{M}$ by $\quotep{M}[P] :=
\quotep{M[P]}$. To foreshadow what is to come we observe that these
operations enjoy a duality with processes very much like the duality
between vectors and maps from vectors to scalars.

Further, because the calculus is essentially higher-order, we have a
correspondence between contexts and processes. More specifically,
given a name $x$ and a context $M$ we can construct $M^{*}_{x}$ such
that 

\begin{mathpar}
  M^{*}_{x} | \lift{x}{P} \red M[P]
\end{mathpar}

namely,

\begin{mathpar}
  M^{*}_{x} := x?(u).M[\dropn{u}]
\end{mathpar}

The dependence of $M^{*}_{x}$ on a name makes it an abstraction, 

\begin{mathpar}
  M^{*} := (x)x?(u).M[\dropn{u}]
\end{mathpar}

\subsection{Additional notation}

It will sometimes be convenient to denote the process a name
quotes. We already have the notation $x = \quotep{P}$, but it will be
convenient to introduce an alternate notation, $\procn{x}$, when we
want to emphasize the connection to the use of the name. Note that, by
virtue of name equivalence, $\quotep{\procn{x}} \nameeq x$; so, the
notation is consistent with previous definitions.

Further, because names have structure it is possible to effect
substitutions on the basis of that structure. This means we need to
upgrade our notation for substitutions, which we accomplish by
adapting comprehension notation. Thus,

\begin{mathpar}
  P\{ y / x : x \in S \}
\end{mathpar}

is interpreted to mean the process derived from P by replacing (in a
capture-avoiding manner) each occurrence of $x$ in $S$ by $y$. For example,

\begin{mathpar}
  P\{ \quotep{\procn{x}|\procn{x}} / x : x \in \freenames{P} \}
\end{mathpar}

will replace each (occurrence) of a free name $x$ in $P$ by
$\quotep{\procn{x}|\procn{x}}$.

Also, we will avail ourselves of the notation $x^{L}$ and $x^{R}$ to
denote injections of a name into disjoint copies of the name
space. There are numerous ways to accomplish this. One example can be
found in \cite{MeredithR05}. This notation overloads to vectors of
names: $\vec{x}^{\pi} := (x_{i}^{\pi} \; : \; 0 \leq i < |\vec{x}| )$ where $\pi \in \{L,R\}$.

We also use $P^{\Box} := P|\Box$.

In \cite{MeredithR05} an interpretation of the new operator is
given. It turns out that there are several possible interpretations
all enjoying the requisite algebraic properties of the operator (see
\cite{milner91polyadicpi}). We will therefore make liberal use of
$(\nu\; \vec{x})P$.

% subsection the_syntax_and_semantics_of_the_notation_system (end)   

\input{qm2pi.qmops} 

\input{qm2pi.sterngerlach} 

\input{qm2pi.metric} 

% section concurrent_process_calculi (end)

%\input{qm2pi.proofsketch}

% section proof sketch (end)

%\input{qm2pi.slviaknots} 

% section spatial logic via knots (end)

\input{qm2pi.conclusion}

% section conclusion (end)

%\input{qm2pi.dtcodes} 

% section wiring algorithm (end)

\input{qm2pi.ack} 

% section acknowledgments (end)

\newpage


\bibliographystyle{plain}   
\bibliography{../../biblios/main.bib}

\input{qm2pi.rhodetails}

\end{document}

 

% section notation (end)

\input{qm2pi.process.calculi} 

% section concurrent_process_calculi_and_spatial_logics_ (end)
    
%\documentclass[12pt]{llncs}
%\documentclass{jktr}

\usepackage[pdftex]{hyperref}                   
\usepackage {listings}
\usepackage {mathpartir}
\usepackage{bcprules}
%\usepackage{listings}
                       
\usepackage{graphicx} 
%\usepackage[margins=2.5cm,nohead,nofoot]{geometry}
%\usepackage{geometry}
\usepackage{amsfonts}
\usepackage{amstext}
\usepackage{latexsym}
\usepackage{amssymb}
\usepackage{color}


%\include{myPreamble}
\include{qm2pi.local} 

%\ifpdf
%\usepackage[pdftex]{graphicx}
%\else
%\usepackage{graphicx}
%\fi

 % \ifpdf
%  \usepackage{pdfsync}
%  \if


%\title{Brief Article}
%\author{David F. Snyder}
%\author{L.G. Meredith}

%\address{Dept. of Math., Texas State University--San Marcos, San Marcos, TX 78666}
       
\pagestyle{empty}


\begin{document}

\lstset{language=[Objective]Caml,frame=shadowbox}

\input{qm2pi.front}

% section front matter (end)

\input{qm2pi.intro} 
 
% section introduction (end)

% \input{qm2pi.knotations} 

% section notation (end)

\input{qm2pi.process.calculi} 

% section concurrent_process_calculi_and_spatial_logics_ (end)
    
%\input{qm2pi.knots2pi} 

%\input{qm2pi.trefoil} 

%\input{qm2pi.mainthm} 

% subsection basic_interpretation (end)

%\input{qm2pi.rho.presentation} 
\subsection{The syntax and semantics of the notation system}\label{sub:the_syntax_and_semantics_of_the_notation_system} % (fold)

We now summarize a technical presentation of the calculus that
embodies our theory of dynamics. The typical presentation of such a
calculus follows the style of giving generators and relations on
them. The grammar, below, describing term constructors, freely
generates the set of processes, $\Proc$. This set is then quotiented
by a relation known as structural congruence and it is over this set
that the notion of dynamics is expressed. This presentation is
essentially that of \cite{MeredithR05} with the addition of
polyadicity and summation. For readability we have relegated some of
the technical subtleties to an appendix.

\subsubsection{Process grammar}\label{subsub:process_grammar}

\begin{mathpar}
  \inferrule* [lab=synchronization] {} {{M} \bc \pzero \;|\; x?F \;|\; x!C }
  \and
  \inferrule* [lab=abstraction] {} {{F} \bc (x)P}
  \and
  \inferrule* [lab=concretion] {} {{C} \bc \langle Q \rangle}
  \and
  \inferrule* [lab=process] {} {{P,Q} \bc M \;| \;P|Q \;|\; @{x}}
  \and
  \inferrule* [lab=name] {} {{x} \bc \quotep{P}}
\end{mathpar} 

Note that $\vec{x}$ (resp. $\vec{P}$) denotes a vector of names
(resp. processes) of length $|\vec{x}|$ (resp. $|\vec{P}|$). We adopt
the following useful abbreviations.

\begin{mathpar}
   x?(\vec{y}).P := x.(\vec{y})P \and  x\clift{\vec{P}} := x.\clift{\vec{P}}
   \and x!(y) := \lift{x}{\dropn{y}}
   \and \Pi_{i=0}^{n-1}P_i := P_0 | \ldots | P_{n-1}
\end{mathpar}

\subsubsection{Structural congruence}

\paragraph{Free and bound names and alpha-equivalence.} At the
core of structural equivalence is alpha-equivalence which identifies
process that are the same up to a change of variable. Formally, we
recognize the distinction between free and bound names. The free names
of a process, $\freenames{P}$, may be calculated recursively as
follows:

\begin{mathpar}
\freenames{\pzero} := \emptyset
  \and \\
  \freenames{x?(y).P} := \{ x \} \cup (\freenames{P} \setminus \{ y \})
  \and 
  \freenames{x!\langle P \rangle} := \{ x \} \cup \{ P \} 
  \and \\
  \freenames{P|Q} := \freenames{P} \cup \freenames{Q}
  \and \\
  \freenames{@{x}} := \{ x \}
\end{mathpar}

$\pi$
$\quotep{\pi}$

$\freenames{-} : \pi \to \mathcal{P}(\quotep{\pi})$

\begin{eqnarray*}
  \freenames{\pzero} & := & \emptyset \\
  \freenames{x?(y).P} & := & \{ x \} \cup (\freenames{P} \setminus \{ y \}) \\
  \freenames{x!\langle P \rangle} & := & \{ x \} \cup \{ P \} \\
  \freenames{P|Q} & := & \freenames{P} \cup \freenames{Q} \\
  \freenames{\dropn{x}} & := & \{ x \}
\end{eqnarray*}

The bound names of a process, $\boundnames{P}$, are those names occurring in $P$
that are not free. For example, in $x?(y).0$, the name $x$ is free, while $y$ is bound.

\begin{mathpar}
  \inferrule* [lab=monoidal-laws] {} { P|Q \equiv Q|P \and P|0 \equiv P \and P|(Q|R) \equiv (P|Q)|R }
\end{mathpar}

\begin{mathpar}
  \inferrule* [lab=alpha-equivalence] {} { (x)P \equiv (y)P\{y/x\} \and y \not\in \freenames{P} }
\end{mathpar}

\begin{definition}
Then two processes, $P,Q$, are alpha-equivalent if $P = Q\{\vec{y}/\vec{x}\}$ for
some $\vec{x} \in \boundnames{Q},\vec{y} \in \boundnames{P}$, where $Q\{\vec{y}/\vec{x}\}$
denotes the capture-avoiding substitution of $\vec{y}$ for $\vec{x}$ in $Q$.
\end{definition}

\begin{definition}
  The {\em structural congruence} \cite{SangiorgiWalker} , $\equiv$,
  between processes is the least congruence containing
  alpha-equivalence, satisfying the abelian monoid laws
  (associativity, commutativity and $\pzero$ as identity) for parallel
  composition $|$ and for summation $+$.
\end{definition}

\subsection{Name equivalence}

We take name equivalence, written $\nameeq$, to be the smallest
equivalence relation generated by the following rules.

\begin{mathpar}
\inferrule*[lab=Quote-drop]
{ }
{ \quotep{@{x}} \nameeq x }

\inferrule*[lab=Struct-equiv]
{ P \scong Q }
{ \quotep{P} \nameeq \quotep{Q} }
\end{mathpar}

The astute reader will have noticed that the mutual recursion of names
and processes imposes a mutual recursion on alpha-equivalence and
structural equivalence via name-equivalence. Fortunately, all of this
works out pleasantly and we may calculate in the natural way, free of
concern. The reader interested in the details is referred to the
appendix \ref{appendix:rho_details}.

\subsection{Substitution}

We use $\Proc$ for the set of processes, $\QProc$ for the set of
names, and $\id{\{}\vec{y} / \vec{x} \id{\}}$ to denote partial maps,
$s : \QProc \rightarrow \QProc$. A map, $s$ lifts, uniquely, to a map
on process terms, $\widehat{s} : \Proc \rightarrow \Proc$ by the
following equations.

\begin{mathpar}
  (0) \psubstp{Q}{P} := 0 \\
  (R \juxtap S) \psubstp{Q}{P}
  :=    
  (R)\psubstp{Q}{P} \juxtap (S) \psubstp{Q}{P} \\
  (x?(y).R) \psubstp{Q}{P}    
  :=    
  (x)\substp{Q}{P} (z)\concat( (R \psubstn{z}{y}) \psubstp{Q}{P} ) \\
  (\lift{x}{R}) \psubstp{Q}{P}  
  :=
  \lift{(x)\substp{Q}{P}}{ R \psubstp{Q}{P} } \\
%   (\dropn{x})  \psubstp{Q}{P}       
%   := 
%   \left\{ 
%     \begin{array}{ccc} 
%       \dropn{\quotep{Q}} & & x \nameeq \quotep{P} \\
%       \dropn{x} & & otherwise \\
%     \end{array}
%   \right. 
  (\dropn{x})  \psubstp{Q}{P}       
  := 
  \left\{ 
    \begin{array}{ccc} 
      Q & & x \nameeq \quotep{P} \\
      \dropn{x} & & otherwise \\
    \end{array}
  \right.
\end{mathpar}
 

where

\begin{eqnarray}
  (x)\id{\{} \lpquote Q \rpquote / \lpquote P \rpquote \id{\}}            = 
  \left\{ 
    \begin{array}{ccc}
      \lpquote Q \rpquote & & x \nameeq \lpquote P \rpquote \\
      x & & otherwise \\
    \end{array}
  \right. \nonumber
\end{eqnarray}

and $z$ is chosen distinct from $\quotep{P}$, $\quotep{Q}$, the free
names in $Q$, and all the names in $R$. Our $\alpha$-equivalence will
be built in the standard way from this substitution.

\begin{remark}\label{rem:no_self_referential_names}
  One consequence of these definitions is that $\forall P. \quotep{P}
  \not\in \freenames{P}$.
\end{remark}

\subsection{ Dynamic quote: an example }

Anticipating something of what's to come, consider applying the
substitution, $\widehat{\id{\{}u / z \id{\}}}$, to the following pair
of processes, $\lift{w}{y!(z)}$ and $w[ \lpquote y!(z) \rpquote ]$.

\begin{eqnarray}
	\lift{w}{y!(z)}\widehat{\id{\{}u / z \id{\}}}
		& = &
		\lift{w}{y!(u)} \nonumber\\
	w[ \lpquote y!(z) \rpquote ] \widehat{ \id{\{}u / z \id{\}} }
		& = &
		w[ \lpquote y!(z) \rpquote ] \nonumber
\end{eqnarray}

Because the body of the process between quotes is impervious to
substitution, we get radically different answers. In fact, by
examining the first process in an input context,
e.g. $x?(z).\lift{w}{y!(z)}$, we see that the process under the lift
operator may be shaped by prefixed inputs binding a name inside it. In
this sense, the lift operator will be seen as a way to dynamically
construct processes before reifying them as names.

Finally equipped with these standard features we can present the
dynamics of the calculus.

\subsubsection{Operational semantics} 

Finally, we introduce the computational dynamics. What marks these
algebras as distinct from other more traditionally studied algebraic
structures, e.g. vector spaces or polynomial rings, is the manner in
which dynamics is captured. In traditional structures, dynamics is typically
expressed through morphisms between such structures, as in linear maps
between vector spaces or morphisms between rings. In algebras
associated with the semantics of computation, the dynamics is
expressed as part of the algebraic structure itself, through a
reduction reduction relation typically denoted by $\red$. Below, we
give a recursive presentation of this relation for the calculus used
in the encoding.

$\red \subseteq \pi \times \pi$
$\red : \pi \to \mathcal{P}(\pi)$

\begin{mathpar}
  \inferrule* [lab=Comm] { \textsf{match}( x_{src}, x_{trgt} ) } { x_{trgt}?(y)P \; | \; x_{src}!\langle {Q} \rangle \red P\{\quotep{Q}/y}\} }
  \and \\
  \inferrule* [lab=Par] {{P} \red {P}'} {{{P} | {Q}} \red {{P}' | {Q}}}
  \and
  \inferrule* [lab=Equiv]{{{P} \scong {P}'} \andalso {{P}' \red {Q}'} \andalso {{Q}' \scong {Q}}}{{P} \red {Q}}
\end{mathpar}

\begin{eqnarray*}
  match_{\equiv} (\quotep{P},\quotep{Q}) & := & P \equiv Q \\
  match_{\dagger}(\quotep{P},\quotep{Q}) & := & \forall R. P|Q \red^{*} R => R \red^{*} 0 \\
  match_{K}(\quotep{P},\quotep{Q}) & := & K \mbox{ for some context } K
\end{eqnarray*}

$u?(x)P | u!\langle Q \rangle \red P\{\quotep{Q}/x\}$

%We write $\wred$ for $\red^*$, and $P\red$ if $\exists Q $ such that $ P \red Q$.
We write $P\red$ if $\exists Q $ such that $ P \red Q$ and $P\not\red$, otherwise.

\section{Replication}

As mentioned before, it is known that replication (and hence
recursion) can be implemented in a higher-order process algebra
\cite{SangiorgiWalker}. As our first example of calculation with the
machinery thus far presented we give the construction explicitly in
the {\rhoc}.

\begin{eqnarray}
	D_{x} & := & \prefix{x}{y}{(\binpar{\outputp{x}{y}}{@{y}})} \nonumber\\
	\bangp_{x}{P} & := & \binpar{{x}!\langle{\binpar{D_{x}}{P}}\rangle}{D_{x}} \nonumber
\end{eqnarray}

\begin{eqnarray}
	\bangp_{x}{P} & & \nonumber\\
	=
	& {x}!\langle{(\prefix{x}{y}{(\outputp{x}{y} | @{y})) | P}}\rangle 
	      | \prefix{x}{y}{(\outputp{x}{y} | @{y})} & \nonumber\\
	\red
	& (\outputp{x}{y} | @{y})\substn{\quotep{(\prefix{x}{y}{(@{y} | \outputp{x}{y})) | P}}}{y} & \nonumber\\
	=
	& \outputp{x}{\quotep{(\prefix{x}{y}{(\outputp{x}{y} | @{y})) | P}}}
	  | {(\prefix{x}{y}{(\outputp{x}{y} | @{y})) | P}} & \nonumber\\
	\red
	& \ldots & \nonumber\\
	\red^*
	& P | P | \ldots & \nonumber
\end{eqnarray}

Of course, this encoding, as an implementation, runs away, unfolding
$\bangp{P}$ eagerly. A lazier and more implementable replication
operator, restricted to input-guarded processes, may be obtained as follows.

\begin{eqnarray}
\bangp{\prefix{u}{v}{P}} 
	:= 
	\binpar{\lift{x}{\prefix{u}{v}{(\binpar{D(x)}{P})}}}{D(x)} \nonumber
\end{eqnarray}

\begin{remark}
  Note that the lazier definition still does not deal with summation
  or mixed summation (i.e. sums over input and output). The reader is
  invited to construct definitions of replication that deal with these
  features. 

  Further, the definitions are parameterized in a name, $x$. Can you,
  gentle reader, make a definition that eliminates this parameter and
  guarantees no accidental interaction between the replication
  machinery and the process being replicated -- i.e. no accidental
  sharing of names used by the process to get its work done and the
  name(s) used by the replication to effect copying. This latter
  revision of the definition of replication is crucial to obtaining
  the expected identity $!!P \sim !P$.
\end{remark}

\begin{remark}\label{rem:paradoxical_combinator}
  The reader familiar with the lambda calculus will have noticed the
  similarity between $D$ and the paradoxical combinator.

  [Ed. note: the existence of this seems to suggest we have to be more
  restrictive on the set of processes and names we admit if we are to
  support no-cloning.]
\end{remark}

\subsubsection{Bisimulation}

The computational dynamics gives rise to another kind of equivalence,
the equivalence of computational behavior. As previously mentioned
this is typically captured \emph{via} some form of bisimulation.

% The notion we use in this paper is weak barbed bisimulation
% \cite{milner91polyadicpi}.

The notion we use in this paper is derived from weak barbed
bisimulation \cite{milner91polyadicpi}. 

\begin{definition}
An \emph{observation relation}, $\downarrow_{\mathcal N}$, over a set
of names, $\mathcal N$, is the smallest relation satisfying the rules
below.

\infrule[Out-barb]{y \in {\mathcal N}, \; x \nameeq y}
		  {\outputp{x}{v} \downarrow_{\mathcal N} x}
\infrule[Par-barb]{\mbox{$P\downarrow_{\mathcal N} x$ or $Q\downarrow_{\mathcal N} x$}}
		  {\binpar{P}{Q} \downarrow_{\mathcal N} x}

We write $P \Downarrow_{\mathcal N} x$ if there is $Q$ such that 
$P \wred Q$ and $Q \downarrow_{\mathcal N} x$.
\end{definition}

\begin{definition}
%\label{def.bbisim}
An  ${\mathcal N}$-\emph{barbed bisimulation} over a set of names, ${\mathcal N}$, is a symmetric binary relation 
${\mathcal S}_{\mathcal N}$ between agents such that $P\rel{S}_{\mathcal N}Q$ implies:
\begin{enumerate}
\item If $P \red P'$ then $Q \wred Q'$ and $P'\rel{S}_{\mathcal N} Q'$.
\item If $P\downarrow_{\mathcal N} x$, then $Q\Downarrow_{\mathcal N} x$.
\end{enumerate}
$P$ is ${\mathcal N}$-barbed bisimilar to $Q$, written
$P \wbbisim_{\mathcal N} Q$, if $P \rel{S}_{\mathcal N} Q$ for some ${\mathcal N}$-barbed bisimulation ${\mathcal S}_{\mathcal N}$.
\end{definition}

$\mathcal{R} \subseteq \pi \times \pi$

$P \mathcal{R} Q => \forall P'. P \red P' \Rightarrow \exists Q'. Q \red Q', P' \mathcal{R} Q'$

$P \vdash x \Rightarrow Q \vdash x$

\begin{mathpar}
  \inferrule*[lab=Out-barb]{x \nameeq y}{{y}!\langle{Q}\rangle \vdash x}
  \and
  \inferrule*[lab=Par-barb]{\mbox{$P\vdash x$ or $Q\vdash x$}}{\binpar{P}{Q} \vdash x}
\end{mathpar}

\subsubsection{Contexts}

One of the principle advantages of computational calculi like the
$\pi$-calculus is a well-defined notion of context,
contextual-equivalence and a correlation between
contextual-equivalence and notions of bisimulation. The notion of
context allows the decomposition of a process into (sub-)process and
its syntactic environment, its context. Thus, a context may be
thought of as a process with a ``hole'' (written $\Box$) in it. The
application of a context $M$ to a process $P$, written $M[P]$, is
tantamount to filling the hole in $M$ with $P$. In this paper we do
not need the full weight of this theory, but do make use of the notion
of context in the proof the main theorem. 

\begin{mathpar}
  \inferrule* [lab=summation] {} {{M_{M},M_{N}} \bc \Box \;|\; x.M_{A} \;|\; M_{M}+M_{N}}
  \and
  \inferrule* [lab=agent] {} {{M_{A}} \bc (\vec{x})M_{P} \;| \; \clift{P_0,\ldots,M_{P},\ldots,P_N}}
  \and \\
  \inferrule* [lab=process] {} {{M_{P}} \bc M_{N} \;| \;P|M_{P} }
\end{mathpar} 

\begin{mathpar}
  \inferrule* [lab=sychronization] {} {M_{N} \bc \Box \;|\; x?M_{F} \;|\; x!M_{C}}
  \and
  \inferrule* [lab=abstraction] {} {{M_{F}} \bc (x)M_{P} }
  \and
  \inferrule* [lab=concretion] {} {{M_{C}} \bc \langle M_{P} \rangle }
  \and \\
  \inferrule* [lab=process] {} {{M_{P}} \bc M_{N} \;| \;P|M_{P} }
\end{mathpar}

\begin{definition}[contextual application] Given a context $M$, and
  process $P$, we define the \emph{contextual application}, $M[P] :=
  M\{P/\Box\}$. That is, the contextual application of M to P is the
  substitution of $P$ for $\Box$ in $M$.
\end{definition}

$\meaningof{-} : L \to \mathcal{P}(\pi)$

\begin{mathpar}
  \inferrule* [lab=collection] {} {\meaningof{true} = \pi, \and \meaningof{~E} = \pi \setminus \meaningof{E}, \and \meaningof{E_{1} \& E_{2}} = \meaningof{E_{1}} \cap \meaningof{E_{2}}}
\end{mathpar}

\begin{mathpar}
  \inferrule* [lab=structure] {} {\meaningof{0} = \{ P \in \pi | P \equiv 0 \}, \and \\ \meaningof{E_1 | E_2} = \{ P \in \pi | P \equiv P_{1} | P_{2}, P_{1} \in \meaningof{E_{1}}, P_{2} \in \meaningof{E_2}\} }
\end{mathpar}

\begin{mathpar}
 \inferrule* [lab=behavior] {} {\meaningof{\langle a?b \rangle E} = \{ P \in \pi | P \equiv Q | u?(y)P', \\ \and \\\\ \and \\ \;\;\; u \in \meaningof{a}, \forall z.P'\{z/y\} \in \meaningof{E\{z/b\}}\}, \and \\ \meaningof{a!E} = \{ P \in \pi | P \equiv Q | x!\langle P' \rangle, x \in \meaningof{a} P' \in \meaningof{E}\} }
\end{mathpar}

\begin{mathpar}
 \inferrule* [lab=nominal] {} {\meaningof{\quotep{E}} = \{ \quotep{P} \in \quotep{\pi} | P \in \meaningof{E} \}, \and \meaningof{\quotep{P}} = \{ \quotep{Q} \in \quotep{\pi} | P \equiv Q \} \and \\ \meaningof{@\quotep{E}} = \{ P \in \pi | P \equiv @x, x \in \meaningof{E} \}}
\end{mathpar}

\begin{eqnarray*}
  \\
  \meaningof{-} : TS \to ST
\end{eqnarray*}

\begin{eqnarray*}
  \\
  L : TS \to ST
\end{eqnarray*}

\begin{eqnarray*}
  \\
  P \models E \iff P \in \meaningof{E}
\end{eqnarray*}

\begin{eqnarray*}
  P \approx_{L} Q \iff \forall E \in L. P \models E \iff Q \models E
\end{eqnarray*}

\begin{eqnarray*}
  P \approx_{K} Q
\end{eqnarray*}

\begin{eqnarray*}
  P \approx Q
\end{eqnarray*}

$\approx_{K} = \approx = \approx_{L}$

\subsubsection{Contextual duality}

Note that contexts extend the quotation operation to a family of
operations from processes to names. Given a context, $M$, we can
define a \emph{nominal context}, $\quotep{M}$ by $\quotep{M}[P] :=
\quotep{M[P]}$. To foreshadow what is to come we observe that these
operations enjoy a duality with processes very much like the duality
between vectors and maps from vectors to scalars.

Further, because the calculus is essentially higher-order, we have a
correspondence between contexts and processes. More specifically,
given a name $x$ and a context $M$ we can construct $M^{*}_{x}$ such
that 

\begin{mathpar}
  M^{*}_{x} | \lift{x}{P} \red M[P]
\end{mathpar}

namely,

\begin{mathpar}
  M^{*}_{x} := x?(u).M[\dropn{u}]
\end{mathpar}

The dependence of $M^{*}_{x}$ on a name makes it an abstraction, 

\begin{mathpar}
  M^{*} := (x)x?(u).M[\dropn{u}]
\end{mathpar}

\subsection{Additional notation}

It will sometimes be convenient to denote the process a name
quotes. We already have the notation $x = \quotep{P}$, but it will be
convenient to introduce an alternate notation, $\procn{x}$, when we
want to emphasize the connection to the use of the name. Note that, by
virtue of name equivalence, $\quotep{\procn{x}} \nameeq x$; so, the
notation is consistent with previous definitions.

Further, because names have structure it is possible to effect
substitutions on the basis of that structure. This means we need to
upgrade our notation for substitutions, which we accomplish by
adapting comprehension notation. Thus,

\begin{mathpar}
  P\{ y / x : x \in S \}
\end{mathpar}

is interpreted to mean the process derived from P by replacing (in a
capture-avoiding manner) each occurrence of $x$ in $S$ by $y$. For example,

\begin{mathpar}
  P\{ \quotep{\procn{x}|\procn{x}} / x : x \in \freenames{P} \}
\end{mathpar}

will replace each (occurrence) of a free name $x$ in $P$ by
$\quotep{\procn{x}|\procn{x}}$.

Also, we will avail ourselves of the notation $x^{L}$ and $x^{R}$ to
denote injections of a name into disjoint copies of the name
space. There are numerous ways to accomplish this. One example can be
found in \cite{MeredithR05}. This notation overloads to vectors of
names: $\vec{x}^{\pi} := (x_{i}^{\pi} \; : \; 0 \leq i < |\vec{x}| )$ where $\pi \in \{L,R\}$.

We also use $P^{\Box} := P|\Box$.

In \cite{MeredithR05} an interpretation of the new operator is
given. It turns out that there are several possible interpretations
all enjoying the requisite algebraic properties of the operator (see
\cite{milner91polyadicpi}). We will therefore make liberal use of
$(\nu\; \vec{x})P$.

% subsection the_syntax_and_semantics_of_the_notation_system (end)   

\input{qm2pi.qmops} 

\input{qm2pi.sterngerlach} 

\input{qm2pi.metric} 

% section concurrent_process_calculi (end)

%\input{qm2pi.proofsketch}

% section proof sketch (end)

%\input{qm2pi.slviaknots} 

% section spatial logic via knots (end)

\input{qm2pi.conclusion}

% section conclusion (end)

%\input{qm2pi.dtcodes} 

% section wiring algorithm (end)

\input{qm2pi.ack} 

% section acknowledgments (end)

\newpage


\bibliographystyle{plain}   
\bibliography{../../biblios/main.bib}

\input{qm2pi.rhodetails}

\end{document}

 

%\documentclass[12pt]{llncs}
%\documentclass{jktr}

\usepackage[pdftex]{hyperref}                   
\usepackage {listings}
\usepackage {mathpartir}
\usepackage{bcprules}
%\usepackage{listings}
                       
\usepackage{graphicx} 
%\usepackage[margins=2.5cm,nohead,nofoot]{geometry}
%\usepackage{geometry}
\usepackage{amsfonts}
\usepackage{amstext}
\usepackage{latexsym}
\usepackage{amssymb}
\usepackage{color}


%\include{myPreamble}
\include{qm2pi.local} 

%\ifpdf
%\usepackage[pdftex]{graphicx}
%\else
%\usepackage{graphicx}
%\fi

 % \ifpdf
%  \usepackage{pdfsync}
%  \if


%\title{Brief Article}
%\author{David F. Snyder}
%\author{L.G. Meredith}

%\address{Dept. of Math., Texas State University--San Marcos, San Marcos, TX 78666}
       
\pagestyle{empty}


\begin{document}

\lstset{language=[Objective]Caml,frame=shadowbox}

\input{qm2pi.front}

% section front matter (end)

\input{qm2pi.intro} 
 
% section introduction (end)

% \input{qm2pi.knotations} 

% section notation (end)

\input{qm2pi.process.calculi} 

% section concurrent_process_calculi_and_spatial_logics_ (end)
    
%\input{qm2pi.knots2pi} 

%\input{qm2pi.trefoil} 

%\input{qm2pi.mainthm} 

% subsection basic_interpretation (end)

%\input{qm2pi.rho.presentation} 
\subsection{The syntax and semantics of the notation system}\label{sub:the_syntax_and_semantics_of_the_notation_system} % (fold)

We now summarize a technical presentation of the calculus that
embodies our theory of dynamics. The typical presentation of such a
calculus follows the style of giving generators and relations on
them. The grammar, below, describing term constructors, freely
generates the set of processes, $\Proc$. This set is then quotiented
by a relation known as structural congruence and it is over this set
that the notion of dynamics is expressed. This presentation is
essentially that of \cite{MeredithR05} with the addition of
polyadicity and summation. For readability we have relegated some of
the technical subtleties to an appendix.

\subsubsection{Process grammar}\label{subsub:process_grammar}

\begin{mathpar}
  \inferrule* [lab=synchronization] {} {{M} \bc \pzero \;|\; x?F \;|\; x!C }
  \and
  \inferrule* [lab=abstraction] {} {{F} \bc (x)P}
  \and
  \inferrule* [lab=concretion] {} {{C} \bc \langle Q \rangle}
  \and
  \inferrule* [lab=process] {} {{P,Q} \bc M \;| \;P|Q \;|\; @{x}}
  \and
  \inferrule* [lab=name] {} {{x} \bc \quotep{P}}
\end{mathpar} 

Note that $\vec{x}$ (resp. $\vec{P}$) denotes a vector of names
(resp. processes) of length $|\vec{x}|$ (resp. $|\vec{P}|$). We adopt
the following useful abbreviations.

\begin{mathpar}
   x?(\vec{y}).P := x.(\vec{y})P \and  x\clift{\vec{P}} := x.\clift{\vec{P}}
   \and x!(y) := \lift{x}{\dropn{y}}
   \and \Pi_{i=0}^{n-1}P_i := P_0 | \ldots | P_{n-1}
\end{mathpar}

\subsubsection{Structural congruence}

\paragraph{Free and bound names and alpha-equivalence.} At the
core of structural equivalence is alpha-equivalence which identifies
process that are the same up to a change of variable. Formally, we
recognize the distinction between free and bound names. The free names
of a process, $\freenames{P}$, may be calculated recursively as
follows:

\begin{mathpar}
\freenames{\pzero} := \emptyset
  \and \\
  \freenames{x?(y).P} := \{ x \} \cup (\freenames{P} \setminus \{ y \})
  \and 
  \freenames{x!\langle P \rangle} := \{ x \} \cup \{ P \} 
  \and \\
  \freenames{P|Q} := \freenames{P} \cup \freenames{Q}
  \and \\
  \freenames{@{x}} := \{ x \}
\end{mathpar}

$\pi$
$\quotep{\pi}$

$\freenames{-} : \pi \to \mathcal{P}(\quotep{\pi})$

\begin{eqnarray*}
  \freenames{\pzero} & := & \emptyset \\
  \freenames{x?(y).P} & := & \{ x \} \cup (\freenames{P} \setminus \{ y \}) \\
  \freenames{x!\langle P \rangle} & := & \{ x \} \cup \{ P \} \\
  \freenames{P|Q} & := & \freenames{P} \cup \freenames{Q} \\
  \freenames{\dropn{x}} & := & \{ x \}
\end{eqnarray*}

The bound names of a process, $\boundnames{P}$, are those names occurring in $P$
that are not free. For example, in $x?(y).0$, the name $x$ is free, while $y$ is bound.

\begin{mathpar}
  \inferrule* [lab=monoidal-laws] {} { P|Q \equiv Q|P \and P|0 \equiv P \and P|(Q|R) \equiv (P|Q)|R }
\end{mathpar}

\begin{mathpar}
  \inferrule* [lab=alpha-equivalence] {} { (x)P \equiv (y)P\{y/x\} \and y \not\in \freenames{P} }
\end{mathpar}

\begin{definition}
Then two processes, $P,Q$, are alpha-equivalent if $P = Q\{\vec{y}/\vec{x}\}$ for
some $\vec{x} \in \boundnames{Q},\vec{y} \in \boundnames{P}$, where $Q\{\vec{y}/\vec{x}\}$
denotes the capture-avoiding substitution of $\vec{y}$ for $\vec{x}$ in $Q$.
\end{definition}

\begin{definition}
  The {\em structural congruence} \cite{SangiorgiWalker} , $\equiv$,
  between processes is the least congruence containing
  alpha-equivalence, satisfying the abelian monoid laws
  (associativity, commutativity and $\pzero$ as identity) for parallel
  composition $|$ and for summation $+$.
\end{definition}

\subsection{Name equivalence}

We take name equivalence, written $\nameeq$, to be the smallest
equivalence relation generated by the following rules.

\begin{mathpar}
\inferrule*[lab=Quote-drop]
{ }
{ \quotep{@{x}} \nameeq x }

\inferrule*[lab=Struct-equiv]
{ P \scong Q }
{ \quotep{P} \nameeq \quotep{Q} }
\end{mathpar}

The astute reader will have noticed that the mutual recursion of names
and processes imposes a mutual recursion on alpha-equivalence and
structural equivalence via name-equivalence. Fortunately, all of this
works out pleasantly and we may calculate in the natural way, free of
concern. The reader interested in the details is referred to the
appendix \ref{appendix:rho_details}.

\subsection{Substitution}

We use $\Proc$ for the set of processes, $\QProc$ for the set of
names, and $\id{\{}\vec{y} / \vec{x} \id{\}}$ to denote partial maps,
$s : \QProc \rightarrow \QProc$. A map, $s$ lifts, uniquely, to a map
on process terms, $\widehat{s} : \Proc \rightarrow \Proc$ by the
following equations.

\begin{mathpar}
  (0) \psubstp{Q}{P} := 0 \\
  (R \juxtap S) \psubstp{Q}{P}
  :=    
  (R)\psubstp{Q}{P} \juxtap (S) \psubstp{Q}{P} \\
  (x?(y).R) \psubstp{Q}{P}    
  :=    
  (x)\substp{Q}{P} (z)\concat( (R \psubstn{z}{y}) \psubstp{Q}{P} ) \\
  (\lift{x}{R}) \psubstp{Q}{P}  
  :=
  \lift{(x)\substp{Q}{P}}{ R \psubstp{Q}{P} } \\
%   (\dropn{x})  \psubstp{Q}{P}       
%   := 
%   \left\{ 
%     \begin{array}{ccc} 
%       \dropn{\quotep{Q}} & & x \nameeq \quotep{P} \\
%       \dropn{x} & & otherwise \\
%     \end{array}
%   \right. 
  (\dropn{x})  \psubstp{Q}{P}       
  := 
  \left\{ 
    \begin{array}{ccc} 
      Q & & x \nameeq \quotep{P} \\
      \dropn{x} & & otherwise \\
    \end{array}
  \right.
\end{mathpar}
 

where

\begin{eqnarray}
  (x)\id{\{} \lpquote Q \rpquote / \lpquote P \rpquote \id{\}}            = 
  \left\{ 
    \begin{array}{ccc}
      \lpquote Q \rpquote & & x \nameeq \lpquote P \rpquote \\
      x & & otherwise \\
    \end{array}
  \right. \nonumber
\end{eqnarray}

and $z$ is chosen distinct from $\quotep{P}$, $\quotep{Q}$, the free
names in $Q$, and all the names in $R$. Our $\alpha$-equivalence will
be built in the standard way from this substitution.

\begin{remark}\label{rem:no_self_referential_names}
  One consequence of these definitions is that $\forall P. \quotep{P}
  \not\in \freenames{P}$.
\end{remark}

\subsection{ Dynamic quote: an example }

Anticipating something of what's to come, consider applying the
substitution, $\widehat{\id{\{}u / z \id{\}}}$, to the following pair
of processes, $\lift{w}{y!(z)}$ and $w[ \lpquote y!(z) \rpquote ]$.

\begin{eqnarray}
	\lift{w}{y!(z)}\widehat{\id{\{}u / z \id{\}}}
		& = &
		\lift{w}{y!(u)} \nonumber\\
	w[ \lpquote y!(z) \rpquote ] \widehat{ \id{\{}u / z \id{\}} }
		& = &
		w[ \lpquote y!(z) \rpquote ] \nonumber
\end{eqnarray}

Because the body of the process between quotes is impervious to
substitution, we get radically different answers. In fact, by
examining the first process in an input context,
e.g. $x?(z).\lift{w}{y!(z)}$, we see that the process under the lift
operator may be shaped by prefixed inputs binding a name inside it. In
this sense, the lift operator will be seen as a way to dynamically
construct processes before reifying them as names.

Finally equipped with these standard features we can present the
dynamics of the calculus.

\subsubsection{Operational semantics} 

Finally, we introduce the computational dynamics. What marks these
algebras as distinct from other more traditionally studied algebraic
structures, e.g. vector spaces or polynomial rings, is the manner in
which dynamics is captured. In traditional structures, dynamics is typically
expressed through morphisms between such structures, as in linear maps
between vector spaces or morphisms between rings. In algebras
associated with the semantics of computation, the dynamics is
expressed as part of the algebraic structure itself, through a
reduction reduction relation typically denoted by $\red$. Below, we
give a recursive presentation of this relation for the calculus used
in the encoding.

$\red \subseteq \pi \times \pi$
$\red : \pi \to \mathcal{P}(\pi)$

\begin{mathpar}
  \inferrule* [lab=Comm] { \textsf{match}( x_{src}, x_{trgt} ) } { x_{trgt}?(y)P \; | \; x_{src}!\langle {Q} \rangle \red P\{\quotep{Q}/y}\} }
  \and \\
  \inferrule* [lab=Par] {{P} \red {P}'} {{{P} | {Q}} \red {{P}' | {Q}}}
  \and
  \inferrule* [lab=Equiv]{{{P} \scong {P}'} \andalso {{P}' \red {Q}'} \andalso {{Q}' \scong {Q}}}{{P} \red {Q}}
\end{mathpar}

\begin{eqnarray*}
  match_{\equiv} (\quotep{P},\quotep{Q}) & := & P \equiv Q \\
  match_{\dagger}(\quotep{P},\quotep{Q}) & := & \forall R. P|Q \red^{*} R => R \red^{*} 0 \\
  match_{K}(\quotep{P},\quotep{Q}) & := & K \mbox{ for some context } K
\end{eqnarray*}

$u?(x)P | u!\langle Q \rangle \red P\{\quotep{Q}/x\}$

%We write $\wred$ for $\red^*$, and $P\red$ if $\exists Q $ such that $ P \red Q$.
We write $P\red$ if $\exists Q $ such that $ P \red Q$ and $P\not\red$, otherwise.

\section{Replication}

As mentioned before, it is known that replication (and hence
recursion) can be implemented in a higher-order process algebra
\cite{SangiorgiWalker}. As our first example of calculation with the
machinery thus far presented we give the construction explicitly in
the {\rhoc}.

\begin{eqnarray}
	D_{x} & := & \prefix{x}{y}{(\binpar{\outputp{x}{y}}{@{y}})} \nonumber\\
	\bangp_{x}{P} & := & \binpar{{x}!\langle{\binpar{D_{x}}{P}}\rangle}{D_{x}} \nonumber
\end{eqnarray}

\begin{eqnarray}
	\bangp_{x}{P} & & \nonumber\\
	=
	& {x}!\langle{(\prefix{x}{y}{(\outputp{x}{y} | @{y})) | P}}\rangle 
	      | \prefix{x}{y}{(\outputp{x}{y} | @{y})} & \nonumber\\
	\red
	& (\outputp{x}{y} | @{y})\substn{\quotep{(\prefix{x}{y}{(@{y} | \outputp{x}{y})) | P}}}{y} & \nonumber\\
	=
	& \outputp{x}{\quotep{(\prefix{x}{y}{(\outputp{x}{y} | @{y})) | P}}}
	  | {(\prefix{x}{y}{(\outputp{x}{y} | @{y})) | P}} & \nonumber\\
	\red
	& \ldots & \nonumber\\
	\red^*
	& P | P | \ldots & \nonumber
\end{eqnarray}

Of course, this encoding, as an implementation, runs away, unfolding
$\bangp{P}$ eagerly. A lazier and more implementable replication
operator, restricted to input-guarded processes, may be obtained as follows.

\begin{eqnarray}
\bangp{\prefix{u}{v}{P}} 
	:= 
	\binpar{\lift{x}{\prefix{u}{v}{(\binpar{D(x)}{P})}}}{D(x)} \nonumber
\end{eqnarray}

\begin{remark}
  Note that the lazier definition still does not deal with summation
  or mixed summation (i.e. sums over input and output). The reader is
  invited to construct definitions of replication that deal with these
  features. 

  Further, the definitions are parameterized in a name, $x$. Can you,
  gentle reader, make a definition that eliminates this parameter and
  guarantees no accidental interaction between the replication
  machinery and the process being replicated -- i.e. no accidental
  sharing of names used by the process to get its work done and the
  name(s) used by the replication to effect copying. This latter
  revision of the definition of replication is crucial to obtaining
  the expected identity $!!P \sim !P$.
\end{remark}

\begin{remark}\label{rem:paradoxical_combinator}
  The reader familiar with the lambda calculus will have noticed the
  similarity between $D$ and the paradoxical combinator.

  [Ed. note: the existence of this seems to suggest we have to be more
  restrictive on the set of processes and names we admit if we are to
  support no-cloning.]
\end{remark}

\subsubsection{Bisimulation}

The computational dynamics gives rise to another kind of equivalence,
the equivalence of computational behavior. As previously mentioned
this is typically captured \emph{via} some form of bisimulation.

% The notion we use in this paper is weak barbed bisimulation
% \cite{milner91polyadicpi}.

The notion we use in this paper is derived from weak barbed
bisimulation \cite{milner91polyadicpi}. 

\begin{definition}
An \emph{observation relation}, $\downarrow_{\mathcal N}$, over a set
of names, $\mathcal N$, is the smallest relation satisfying the rules
below.

\infrule[Out-barb]{y \in {\mathcal N}, \; x \nameeq y}
		  {\outputp{x}{v} \downarrow_{\mathcal N} x}
\infrule[Par-barb]{\mbox{$P\downarrow_{\mathcal N} x$ or $Q\downarrow_{\mathcal N} x$}}
		  {\binpar{P}{Q} \downarrow_{\mathcal N} x}

We write $P \Downarrow_{\mathcal N} x$ if there is $Q$ such that 
$P \wred Q$ and $Q \downarrow_{\mathcal N} x$.
\end{definition}

\begin{definition}
%\label{def.bbisim}
An  ${\mathcal N}$-\emph{barbed bisimulation} over a set of names, ${\mathcal N}$, is a symmetric binary relation 
${\mathcal S}_{\mathcal N}$ between agents such that $P\rel{S}_{\mathcal N}Q$ implies:
\begin{enumerate}
\item If $P \red P'$ then $Q \wred Q'$ and $P'\rel{S}_{\mathcal N} Q'$.
\item If $P\downarrow_{\mathcal N} x$, then $Q\Downarrow_{\mathcal N} x$.
\end{enumerate}
$P$ is ${\mathcal N}$-barbed bisimilar to $Q$, written
$P \wbbisim_{\mathcal N} Q$, if $P \rel{S}_{\mathcal N} Q$ for some ${\mathcal N}$-barbed bisimulation ${\mathcal S}_{\mathcal N}$.
\end{definition}

$\mathcal{R} \subseteq \pi \times \pi$

$P \mathcal{R} Q => \forall P'. P \red P' \Rightarrow \exists Q'. Q \red Q', P' \mathcal{R} Q'$

$P \vdash x \Rightarrow Q \vdash x$

\begin{mathpar}
  \inferrule*[lab=Out-barb]{x \nameeq y}{{y}!\langle{Q}\rangle \vdash x}
  \and
  \inferrule*[lab=Par-barb]{\mbox{$P\vdash x$ or $Q\vdash x$}}{\binpar{P}{Q} \vdash x}
\end{mathpar}

\subsubsection{Contexts}

One of the principle advantages of computational calculi like the
$\pi$-calculus is a well-defined notion of context,
contextual-equivalence and a correlation between
contextual-equivalence and notions of bisimulation. The notion of
context allows the decomposition of a process into (sub-)process and
its syntactic environment, its context. Thus, a context may be
thought of as a process with a ``hole'' (written $\Box$) in it. The
application of a context $M$ to a process $P$, written $M[P]$, is
tantamount to filling the hole in $M$ with $P$. In this paper we do
not need the full weight of this theory, but do make use of the notion
of context in the proof the main theorem. 

\begin{mathpar}
  \inferrule* [lab=summation] {} {{M_{M},M_{N}} \bc \Box \;|\; x.M_{A} \;|\; M_{M}+M_{N}}
  \and
  \inferrule* [lab=agent] {} {{M_{A}} \bc (\vec{x})M_{P} \;| \; \clift{P_0,\ldots,M_{P},\ldots,P_N}}
  \and \\
  \inferrule* [lab=process] {} {{M_{P}} \bc M_{N} \;| \;P|M_{P} }
\end{mathpar} 

\begin{mathpar}
  \inferrule* [lab=sychronization] {} {M_{N} \bc \Box \;|\; x?M_{F} \;|\; x!M_{C}}
  \and
  \inferrule* [lab=abstraction] {} {{M_{F}} \bc (x)M_{P} }
  \and
  \inferrule* [lab=concretion] {} {{M_{C}} \bc \langle M_{P} \rangle }
  \and \\
  \inferrule* [lab=process] {} {{M_{P}} \bc M_{N} \;| \;P|M_{P} }
\end{mathpar}

\begin{definition}[contextual application] Given a context $M$, and
  process $P$, we define the \emph{contextual application}, $M[P] :=
  M\{P/\Box\}$. That is, the contextual application of M to P is the
  substitution of $P$ for $\Box$ in $M$.
\end{definition}

$\meaningof{-} : L \to \mathcal{P}(\pi)$

\begin{mathpar}
  \inferrule* [lab=collection] {} {\meaningof{true} = \pi, \and \meaningof{~E} = \pi \setminus \meaningof{E}, \and \meaningof{E_{1} \& E_{2}} = \meaningof{E_{1}} \cap \meaningof{E_{2}}}
\end{mathpar}

\begin{mathpar}
  \inferrule* [lab=structure] {} {\meaningof{0} = \{ P \in \pi | P \equiv 0 \}, \and \\ \meaningof{E_1 | E_2} = \{ P \in \pi | P \equiv P_{1} | P_{2}, P_{1} \in \meaningof{E_{1}}, P_{2} \in \meaningof{E_2}\} }
\end{mathpar}

\begin{mathpar}
 \inferrule* [lab=behavior] {} {\meaningof{\langle a?b \rangle E} = \{ P \in \pi | P \equiv Q | u?(y)P', \\ \and \\\\ \and \\ \;\;\; u \in \meaningof{a}, \forall z.P'\{z/y\} \in \meaningof{E\{z/b\}}\}, \and \\ \meaningof{a!E} = \{ P \in \pi | P \equiv Q | x!\langle P' \rangle, x \in \meaningof{a} P' \in \meaningof{E}\} }
\end{mathpar}

\begin{mathpar}
 \inferrule* [lab=nominal] {} {\meaningof{\quotep{E}} = \{ \quotep{P} \in \quotep{\pi} | P \in \meaningof{E} \}, \and \meaningof{\quotep{P}} = \{ \quotep{Q} \in \quotep{\pi} | P \equiv Q \} \and \\ \meaningof{@\quotep{E}} = \{ P \in \pi | P \equiv @x, x \in \meaningof{E} \}}
\end{mathpar}

\begin{eqnarray*}
  \\
  \meaningof{-} : TS \to ST
\end{eqnarray*}

\begin{eqnarray*}
  \\
  L : TS \to ST
\end{eqnarray*}

\begin{eqnarray*}
  \\
  P \models E \iff P \in \meaningof{E}
\end{eqnarray*}

\begin{eqnarray*}
  P \approx_{L} Q \iff \forall E \in L. P \models E \iff Q \models E
\end{eqnarray*}

\begin{eqnarray*}
  P \approx_{K} Q
\end{eqnarray*}

\begin{eqnarray*}
  P \approx Q
\end{eqnarray*}

$\approx_{K} = \approx = \approx_{L}$

\subsubsection{Contextual duality}

Note that contexts extend the quotation operation to a family of
operations from processes to names. Given a context, $M$, we can
define a \emph{nominal context}, $\quotep{M}$ by $\quotep{M}[P] :=
\quotep{M[P]}$. To foreshadow what is to come we observe that these
operations enjoy a duality with processes very much like the duality
between vectors and maps from vectors to scalars.

Further, because the calculus is essentially higher-order, we have a
correspondence between contexts and processes. More specifically,
given a name $x$ and a context $M$ we can construct $M^{*}_{x}$ such
that 

\begin{mathpar}
  M^{*}_{x} | \lift{x}{P} \red M[P]
\end{mathpar}

namely,

\begin{mathpar}
  M^{*}_{x} := x?(u).M[\dropn{u}]
\end{mathpar}

The dependence of $M^{*}_{x}$ on a name makes it an abstraction, 

\begin{mathpar}
  M^{*} := (x)x?(u).M[\dropn{u}]
\end{mathpar}

\subsection{Additional notation}

It will sometimes be convenient to denote the process a name
quotes. We already have the notation $x = \quotep{P}$, but it will be
convenient to introduce an alternate notation, $\procn{x}$, when we
want to emphasize the connection to the use of the name. Note that, by
virtue of name equivalence, $\quotep{\procn{x}} \nameeq x$; so, the
notation is consistent with previous definitions.

Further, because names have structure it is possible to effect
substitutions on the basis of that structure. This means we need to
upgrade our notation for substitutions, which we accomplish by
adapting comprehension notation. Thus,

\begin{mathpar}
  P\{ y / x : x \in S \}
\end{mathpar}

is interpreted to mean the process derived from P by replacing (in a
capture-avoiding manner) each occurrence of $x$ in $S$ by $y$. For example,

\begin{mathpar}
  P\{ \quotep{\procn{x}|\procn{x}} / x : x \in \freenames{P} \}
\end{mathpar}

will replace each (occurrence) of a free name $x$ in $P$ by
$\quotep{\procn{x}|\procn{x}}$.

Also, we will avail ourselves of the notation $x^{L}$ and $x^{R}$ to
denote injections of a name into disjoint copies of the name
space. There are numerous ways to accomplish this. One example can be
found in \cite{MeredithR05}. This notation overloads to vectors of
names: $\vec{x}^{\pi} := (x_{i}^{\pi} \; : \; 0 \leq i < |\vec{x}| )$ where $\pi \in \{L,R\}$.

We also use $P^{\Box} := P|\Box$.

In \cite{MeredithR05} an interpretation of the new operator is
given. It turns out that there are several possible interpretations
all enjoying the requisite algebraic properties of the operator (see
\cite{milner91polyadicpi}). We will therefore make liberal use of
$(\nu\; \vec{x})P$.

% subsection the_syntax_and_semantics_of_the_notation_system (end)   

\input{qm2pi.qmops} 

\input{qm2pi.sterngerlach} 

\input{qm2pi.metric} 

% section concurrent_process_calculi (end)

%\input{qm2pi.proofsketch}

% section proof sketch (end)

%\input{qm2pi.slviaknots} 

% section spatial logic via knots (end)

\input{qm2pi.conclusion}

% section conclusion (end)

%\input{qm2pi.dtcodes} 

% section wiring algorithm (end)

\input{qm2pi.ack} 

% section acknowledgments (end)

\newpage


\bibliographystyle{plain}   
\bibliography{../../biblios/main.bib}

\input{qm2pi.rhodetails}

\end{document}

 

%\documentclass[12pt]{llncs}
%\documentclass{jktr}

\usepackage[pdftex]{hyperref}                   
\usepackage {listings}
\usepackage {mathpartir}
\usepackage{bcprules}
%\usepackage{listings}
                       
\usepackage{graphicx} 
%\usepackage[margins=2.5cm,nohead,nofoot]{geometry}
%\usepackage{geometry}
\usepackage{amsfonts}
\usepackage{amstext}
\usepackage{latexsym}
\usepackage{amssymb}
\usepackage{color}


%\include{myPreamble}
\include{qm2pi.local} 

%\ifpdf
%\usepackage[pdftex]{graphicx}
%\else
%\usepackage{graphicx}
%\fi

 % \ifpdf
%  \usepackage{pdfsync}
%  \if


%\title{Brief Article}
%\author{David F. Snyder}
%\author{L.G. Meredith}

%\address{Dept. of Math., Texas State University--San Marcos, San Marcos, TX 78666}
       
\pagestyle{empty}


\begin{document}

\lstset{language=[Objective]Caml,frame=shadowbox}

\input{qm2pi.front}

% section front matter (end)

\input{qm2pi.intro} 
 
% section introduction (end)

% \input{qm2pi.knotations} 

% section notation (end)

\input{qm2pi.process.calculi} 

% section concurrent_process_calculi_and_spatial_logics_ (end)
    
%\input{qm2pi.knots2pi} 

%\input{qm2pi.trefoil} 

%\input{qm2pi.mainthm} 

% subsection basic_interpretation (end)

%\input{qm2pi.rho.presentation} 
\subsection{The syntax and semantics of the notation system}\label{sub:the_syntax_and_semantics_of_the_notation_system} % (fold)

We now summarize a technical presentation of the calculus that
embodies our theory of dynamics. The typical presentation of such a
calculus follows the style of giving generators and relations on
them. The grammar, below, describing term constructors, freely
generates the set of processes, $\Proc$. This set is then quotiented
by a relation known as structural congruence and it is over this set
that the notion of dynamics is expressed. This presentation is
essentially that of \cite{MeredithR05} with the addition of
polyadicity and summation. For readability we have relegated some of
the technical subtleties to an appendix.

\subsubsection{Process grammar}\label{subsub:process_grammar}

\begin{mathpar}
  \inferrule* [lab=synchronization] {} {{M} \bc \pzero \;|\; x?F \;|\; x!C }
  \and
  \inferrule* [lab=abstraction] {} {{F} \bc (x)P}
  \and
  \inferrule* [lab=concretion] {} {{C} \bc \langle Q \rangle}
  \and
  \inferrule* [lab=process] {} {{P,Q} \bc M \;| \;P|Q \;|\; @{x}}
  \and
  \inferrule* [lab=name] {} {{x} \bc \quotep{P}}
\end{mathpar} 

Note that $\vec{x}$ (resp. $\vec{P}$) denotes a vector of names
(resp. processes) of length $|\vec{x}|$ (resp. $|\vec{P}|$). We adopt
the following useful abbreviations.

\begin{mathpar}
   x?(\vec{y}).P := x.(\vec{y})P \and  x\clift{\vec{P}} := x.\clift{\vec{P}}
   \and x!(y) := \lift{x}{\dropn{y}}
   \and \Pi_{i=0}^{n-1}P_i := P_0 | \ldots | P_{n-1}
\end{mathpar}

\subsubsection{Structural congruence}

\paragraph{Free and bound names and alpha-equivalence.} At the
core of structural equivalence is alpha-equivalence which identifies
process that are the same up to a change of variable. Formally, we
recognize the distinction between free and bound names. The free names
of a process, $\freenames{P}$, may be calculated recursively as
follows:

\begin{mathpar}
\freenames{\pzero} := \emptyset
  \and \\
  \freenames{x?(y).P} := \{ x \} \cup (\freenames{P} \setminus \{ y \})
  \and 
  \freenames{x!\langle P \rangle} := \{ x \} \cup \{ P \} 
  \and \\
  \freenames{P|Q} := \freenames{P} \cup \freenames{Q}
  \and \\
  \freenames{@{x}} := \{ x \}
\end{mathpar}

$\pi$
$\quotep{\pi}$

$\freenames{-} : \pi \to \mathcal{P}(\quotep{\pi})$

\begin{eqnarray*}
  \freenames{\pzero} & := & \emptyset \\
  \freenames{x?(y).P} & := & \{ x \} \cup (\freenames{P} \setminus \{ y \}) \\
  \freenames{x!\langle P \rangle} & := & \{ x \} \cup \{ P \} \\
  \freenames{P|Q} & := & \freenames{P} \cup \freenames{Q} \\
  \freenames{\dropn{x}} & := & \{ x \}
\end{eqnarray*}

The bound names of a process, $\boundnames{P}$, are those names occurring in $P$
that are not free. For example, in $x?(y).0$, the name $x$ is free, while $y$ is bound.

\begin{mathpar}
  \inferrule* [lab=monoidal-laws] {} { P|Q \equiv Q|P \and P|0 \equiv P \and P|(Q|R) \equiv (P|Q)|R }
\end{mathpar}

\begin{mathpar}
  \inferrule* [lab=alpha-equivalence] {} { (x)P \equiv (y)P\{y/x\} \and y \not\in \freenames{P} }
\end{mathpar}

\begin{definition}
Then two processes, $P,Q$, are alpha-equivalent if $P = Q\{\vec{y}/\vec{x}\}$ for
some $\vec{x} \in \boundnames{Q},\vec{y} \in \boundnames{P}$, where $Q\{\vec{y}/\vec{x}\}$
denotes the capture-avoiding substitution of $\vec{y}$ for $\vec{x}$ in $Q$.
\end{definition}

\begin{definition}
  The {\em structural congruence} \cite{SangiorgiWalker} , $\equiv$,
  between processes is the least congruence containing
  alpha-equivalence, satisfying the abelian monoid laws
  (associativity, commutativity and $\pzero$ as identity) for parallel
  composition $|$ and for summation $+$.
\end{definition}

\subsection{Name equivalence}

We take name equivalence, written $\nameeq$, to be the smallest
equivalence relation generated by the following rules.

\begin{mathpar}
\inferrule*[lab=Quote-drop]
{ }
{ \quotep{@{x}} \nameeq x }

\inferrule*[lab=Struct-equiv]
{ P \scong Q }
{ \quotep{P} \nameeq \quotep{Q} }
\end{mathpar}

The astute reader will have noticed that the mutual recursion of names
and processes imposes a mutual recursion on alpha-equivalence and
structural equivalence via name-equivalence. Fortunately, all of this
works out pleasantly and we may calculate in the natural way, free of
concern. The reader interested in the details is referred to the
appendix \ref{appendix:rho_details}.

\subsection{Substitution}

We use $\Proc$ for the set of processes, $\QProc$ for the set of
names, and $\id{\{}\vec{y} / \vec{x} \id{\}}$ to denote partial maps,
$s : \QProc \rightarrow \QProc$. A map, $s$ lifts, uniquely, to a map
on process terms, $\widehat{s} : \Proc \rightarrow \Proc$ by the
following equations.

\begin{mathpar}
  (0) \psubstp{Q}{P} := 0 \\
  (R \juxtap S) \psubstp{Q}{P}
  :=    
  (R)\psubstp{Q}{P} \juxtap (S) \psubstp{Q}{P} \\
  (x?(y).R) \psubstp{Q}{P}    
  :=    
  (x)\substp{Q}{P} (z)\concat( (R \psubstn{z}{y}) \psubstp{Q}{P} ) \\
  (\lift{x}{R}) \psubstp{Q}{P}  
  :=
  \lift{(x)\substp{Q}{P}}{ R \psubstp{Q}{P} } \\
%   (\dropn{x})  \psubstp{Q}{P}       
%   := 
%   \left\{ 
%     \begin{array}{ccc} 
%       \dropn{\quotep{Q}} & & x \nameeq \quotep{P} \\
%       \dropn{x} & & otherwise \\
%     \end{array}
%   \right. 
  (\dropn{x})  \psubstp{Q}{P}       
  := 
  \left\{ 
    \begin{array}{ccc} 
      Q & & x \nameeq \quotep{P} \\
      \dropn{x} & & otherwise \\
    \end{array}
  \right.
\end{mathpar}
 

where

\begin{eqnarray}
  (x)\id{\{} \lpquote Q \rpquote / \lpquote P \rpquote \id{\}}            = 
  \left\{ 
    \begin{array}{ccc}
      \lpquote Q \rpquote & & x \nameeq \lpquote P \rpquote \\
      x & & otherwise \\
    \end{array}
  \right. \nonumber
\end{eqnarray}

and $z$ is chosen distinct from $\quotep{P}$, $\quotep{Q}$, the free
names in $Q$, and all the names in $R$. Our $\alpha$-equivalence will
be built in the standard way from this substitution.

\begin{remark}\label{rem:no_self_referential_names}
  One consequence of these definitions is that $\forall P. \quotep{P}
  \not\in \freenames{P}$.
\end{remark}

\subsection{ Dynamic quote: an example }

Anticipating something of what's to come, consider applying the
substitution, $\widehat{\id{\{}u / z \id{\}}}$, to the following pair
of processes, $\lift{w}{y!(z)}$ and $w[ \lpquote y!(z) \rpquote ]$.

\begin{eqnarray}
	\lift{w}{y!(z)}\widehat{\id{\{}u / z \id{\}}}
		& = &
		\lift{w}{y!(u)} \nonumber\\
	w[ \lpquote y!(z) \rpquote ] \widehat{ \id{\{}u / z \id{\}} }
		& = &
		w[ \lpquote y!(z) \rpquote ] \nonumber
\end{eqnarray}

Because the body of the process between quotes is impervious to
substitution, we get radically different answers. In fact, by
examining the first process in an input context,
e.g. $x?(z).\lift{w}{y!(z)}$, we see that the process under the lift
operator may be shaped by prefixed inputs binding a name inside it. In
this sense, the lift operator will be seen as a way to dynamically
construct processes before reifying them as names.

Finally equipped with these standard features we can present the
dynamics of the calculus.

\subsubsection{Operational semantics} 

Finally, we introduce the computational dynamics. What marks these
algebras as distinct from other more traditionally studied algebraic
structures, e.g. vector spaces or polynomial rings, is the manner in
which dynamics is captured. In traditional structures, dynamics is typically
expressed through morphisms between such structures, as in linear maps
between vector spaces or morphisms between rings. In algebras
associated with the semantics of computation, the dynamics is
expressed as part of the algebraic structure itself, through a
reduction reduction relation typically denoted by $\red$. Below, we
give a recursive presentation of this relation for the calculus used
in the encoding.

$\red \subseteq \pi \times \pi$
$\red : \pi \to \mathcal{P}(\pi)$

\begin{mathpar}
  \inferrule* [lab=Comm] { \textsf{match}( x_{src}, x_{trgt} ) } { x_{trgt}?(y)P \; | \; x_{src}!\langle {Q} \rangle \red P\{\quotep{Q}/y}\} }
  \and \\
  \inferrule* [lab=Par] {{P} \red {P}'} {{{P} | {Q}} \red {{P}' | {Q}}}
  \and
  \inferrule* [lab=Equiv]{{{P} \scong {P}'} \andalso {{P}' \red {Q}'} \andalso {{Q}' \scong {Q}}}{{P} \red {Q}}
\end{mathpar}

\begin{eqnarray*}
  match_{\equiv} (\quotep{P},\quotep{Q}) & := & P \equiv Q \\
  match_{\dagger}(\quotep{P},\quotep{Q}) & := & \forall R. P|Q \red^{*} R => R \red^{*} 0 \\
  match_{K}(\quotep{P},\quotep{Q}) & := & K \mbox{ for some context } K
\end{eqnarray*}

$u?(x)P | u!\langle Q \rangle \red P\{\quotep{Q}/x\}$

%We write $\wred$ for $\red^*$, and $P\red$ if $\exists Q $ such that $ P \red Q$.
We write $P\red$ if $\exists Q $ such that $ P \red Q$ and $P\not\red$, otherwise.

\section{Replication}

As mentioned before, it is known that replication (and hence
recursion) can be implemented in a higher-order process algebra
\cite{SangiorgiWalker}. As our first example of calculation with the
machinery thus far presented we give the construction explicitly in
the {\rhoc}.

\begin{eqnarray}
	D_{x} & := & \prefix{x}{y}{(\binpar{\outputp{x}{y}}{@{y}})} \nonumber\\
	\bangp_{x}{P} & := & \binpar{{x}!\langle{\binpar{D_{x}}{P}}\rangle}{D_{x}} \nonumber
\end{eqnarray}

\begin{eqnarray}
	\bangp_{x}{P} & & \nonumber\\
	=
	& {x}!\langle{(\prefix{x}{y}{(\outputp{x}{y} | @{y})) | P}}\rangle 
	      | \prefix{x}{y}{(\outputp{x}{y} | @{y})} & \nonumber\\
	\red
	& (\outputp{x}{y} | @{y})\substn{\quotep{(\prefix{x}{y}{(@{y} | \outputp{x}{y})) | P}}}{y} & \nonumber\\
	=
	& \outputp{x}{\quotep{(\prefix{x}{y}{(\outputp{x}{y} | @{y})) | P}}}
	  | {(\prefix{x}{y}{(\outputp{x}{y} | @{y})) | P}} & \nonumber\\
	\red
	& \ldots & \nonumber\\
	\red^*
	& P | P | \ldots & \nonumber
\end{eqnarray}

Of course, this encoding, as an implementation, runs away, unfolding
$\bangp{P}$ eagerly. A lazier and more implementable replication
operator, restricted to input-guarded processes, may be obtained as follows.

\begin{eqnarray}
\bangp{\prefix{u}{v}{P}} 
	:= 
	\binpar{\lift{x}{\prefix{u}{v}{(\binpar{D(x)}{P})}}}{D(x)} \nonumber
\end{eqnarray}

\begin{remark}
  Note that the lazier definition still does not deal with summation
  or mixed summation (i.e. sums over input and output). The reader is
  invited to construct definitions of replication that deal with these
  features. 

  Further, the definitions are parameterized in a name, $x$. Can you,
  gentle reader, make a definition that eliminates this parameter and
  guarantees no accidental interaction between the replication
  machinery and the process being replicated -- i.e. no accidental
  sharing of names used by the process to get its work done and the
  name(s) used by the replication to effect copying. This latter
  revision of the definition of replication is crucial to obtaining
  the expected identity $!!P \sim !P$.
\end{remark}

\begin{remark}\label{rem:paradoxical_combinator}
  The reader familiar with the lambda calculus will have noticed the
  similarity between $D$ and the paradoxical combinator.

  [Ed. note: the existence of this seems to suggest we have to be more
  restrictive on the set of processes and names we admit if we are to
  support no-cloning.]
\end{remark}

\subsubsection{Bisimulation}

The computational dynamics gives rise to another kind of equivalence,
the equivalence of computational behavior. As previously mentioned
this is typically captured \emph{via} some form of bisimulation.

% The notion we use in this paper is weak barbed bisimulation
% \cite{milner91polyadicpi}.

The notion we use in this paper is derived from weak barbed
bisimulation \cite{milner91polyadicpi}. 

\begin{definition}
An \emph{observation relation}, $\downarrow_{\mathcal N}$, over a set
of names, $\mathcal N$, is the smallest relation satisfying the rules
below.

\infrule[Out-barb]{y \in {\mathcal N}, \; x \nameeq y}
		  {\outputp{x}{v} \downarrow_{\mathcal N} x}
\infrule[Par-barb]{\mbox{$P\downarrow_{\mathcal N} x$ or $Q\downarrow_{\mathcal N} x$}}
		  {\binpar{P}{Q} \downarrow_{\mathcal N} x}

We write $P \Downarrow_{\mathcal N} x$ if there is $Q$ such that 
$P \wred Q$ and $Q \downarrow_{\mathcal N} x$.
\end{definition}

\begin{definition}
%\label{def.bbisim}
An  ${\mathcal N}$-\emph{barbed bisimulation} over a set of names, ${\mathcal N}$, is a symmetric binary relation 
${\mathcal S}_{\mathcal N}$ between agents such that $P\rel{S}_{\mathcal N}Q$ implies:
\begin{enumerate}
\item If $P \red P'$ then $Q \wred Q'$ and $P'\rel{S}_{\mathcal N} Q'$.
\item If $P\downarrow_{\mathcal N} x$, then $Q\Downarrow_{\mathcal N} x$.
\end{enumerate}
$P$ is ${\mathcal N}$-barbed bisimilar to $Q$, written
$P \wbbisim_{\mathcal N} Q$, if $P \rel{S}_{\mathcal N} Q$ for some ${\mathcal N}$-barbed bisimulation ${\mathcal S}_{\mathcal N}$.
\end{definition}

$\mathcal{R} \subseteq \pi \times \pi$

$P \mathcal{R} Q => \forall P'. P \red P' \Rightarrow \exists Q'. Q \red Q', P' \mathcal{R} Q'$

$P \vdash x \Rightarrow Q \vdash x$

\begin{mathpar}
  \inferrule*[lab=Out-barb]{x \nameeq y}{{y}!\langle{Q}\rangle \vdash x}
  \and
  \inferrule*[lab=Par-barb]{\mbox{$P\vdash x$ or $Q\vdash x$}}{\binpar{P}{Q} \vdash x}
\end{mathpar}

\subsubsection{Contexts}

One of the principle advantages of computational calculi like the
$\pi$-calculus is a well-defined notion of context,
contextual-equivalence and a correlation between
contextual-equivalence and notions of bisimulation. The notion of
context allows the decomposition of a process into (sub-)process and
its syntactic environment, its context. Thus, a context may be
thought of as a process with a ``hole'' (written $\Box$) in it. The
application of a context $M$ to a process $P$, written $M[P]$, is
tantamount to filling the hole in $M$ with $P$. In this paper we do
not need the full weight of this theory, but do make use of the notion
of context in the proof the main theorem. 

\begin{mathpar}
  \inferrule* [lab=summation] {} {{M_{M},M_{N}} \bc \Box \;|\; x.M_{A} \;|\; M_{M}+M_{N}}
  \and
  \inferrule* [lab=agent] {} {{M_{A}} \bc (\vec{x})M_{P} \;| \; \clift{P_0,\ldots,M_{P},\ldots,P_N}}
  \and \\
  \inferrule* [lab=process] {} {{M_{P}} \bc M_{N} \;| \;P|M_{P} }
\end{mathpar} 

\begin{mathpar}
  \inferrule* [lab=sychronization] {} {M_{N} \bc \Box \;|\; x?M_{F} \;|\; x!M_{C}}
  \and
  \inferrule* [lab=abstraction] {} {{M_{F}} \bc (x)M_{P} }
  \and
  \inferrule* [lab=concretion] {} {{M_{C}} \bc \langle M_{P} \rangle }
  \and \\
  \inferrule* [lab=process] {} {{M_{P}} \bc M_{N} \;| \;P|M_{P} }
\end{mathpar}

\begin{definition}[contextual application] Given a context $M$, and
  process $P$, we define the \emph{contextual application}, $M[P] :=
  M\{P/\Box\}$. That is, the contextual application of M to P is the
  substitution of $P$ for $\Box$ in $M$.
\end{definition}

$\meaningof{-} : L \to \mathcal{P}(\pi)$

\begin{mathpar}
  \inferrule* [lab=collection] {} {\meaningof{true} = \pi, \and \meaningof{~E} = \pi \setminus \meaningof{E}, \and \meaningof{E_{1} \& E_{2}} = \meaningof{E_{1}} \cap \meaningof{E_{2}}}
\end{mathpar}

\begin{mathpar}
  \inferrule* [lab=structure] {} {\meaningof{0} = \{ P \in \pi | P \equiv 0 \}, \and \\ \meaningof{E_1 | E_2} = \{ P \in \pi | P \equiv P_{1} | P_{2}, P_{1} \in \meaningof{E_{1}}, P_{2} \in \meaningof{E_2}\} }
\end{mathpar}

\begin{mathpar}
 \inferrule* [lab=behavior] {} {\meaningof{\langle a?b \rangle E} = \{ P \in \pi | P \equiv Q | u?(y)P', \\ \and \\\\ \and \\ \;\;\; u \in \meaningof{a}, \forall z.P'\{z/y\} \in \meaningof{E\{z/b\}}\}, \and \\ \meaningof{a!E} = \{ P \in \pi | P \equiv Q | x!\langle P' \rangle, x \in \meaningof{a} P' \in \meaningof{E}\} }
\end{mathpar}

\begin{mathpar}
 \inferrule* [lab=nominal] {} {\meaningof{\quotep{E}} = \{ \quotep{P} \in \quotep{\pi} | P \in \meaningof{E} \}, \and \meaningof{\quotep{P}} = \{ \quotep{Q} \in \quotep{\pi} | P \equiv Q \} \and \\ \meaningof{@\quotep{E}} = \{ P \in \pi | P \equiv @x, x \in \meaningof{E} \}}
\end{mathpar}

\begin{eqnarray*}
  \\
  \meaningof{-} : TS \to ST
\end{eqnarray*}

\begin{eqnarray*}
  \\
  L : TS \to ST
\end{eqnarray*}

\begin{eqnarray*}
  \\
  P \models E \iff P \in \meaningof{E}
\end{eqnarray*}

\begin{eqnarray*}
  P \approx_{L} Q \iff \forall E \in L. P \models E \iff Q \models E
\end{eqnarray*}

\begin{eqnarray*}
  P \approx_{K} Q
\end{eqnarray*}

\begin{eqnarray*}
  P \approx Q
\end{eqnarray*}

$\approx_{K} = \approx = \approx_{L}$

\subsubsection{Contextual duality}

Note that contexts extend the quotation operation to a family of
operations from processes to names. Given a context, $M$, we can
define a \emph{nominal context}, $\quotep{M}$ by $\quotep{M}[P] :=
\quotep{M[P]}$. To foreshadow what is to come we observe that these
operations enjoy a duality with processes very much like the duality
between vectors and maps from vectors to scalars.

Further, because the calculus is essentially higher-order, we have a
correspondence between contexts and processes. More specifically,
given a name $x$ and a context $M$ we can construct $M^{*}_{x}$ such
that 

\begin{mathpar}
  M^{*}_{x} | \lift{x}{P} \red M[P]
\end{mathpar}

namely,

\begin{mathpar}
  M^{*}_{x} := x?(u).M[\dropn{u}]
\end{mathpar}

The dependence of $M^{*}_{x}$ on a name makes it an abstraction, 

\begin{mathpar}
  M^{*} := (x)x?(u).M[\dropn{u}]
\end{mathpar}

\subsection{Additional notation}

It will sometimes be convenient to denote the process a name
quotes. We already have the notation $x = \quotep{P}$, but it will be
convenient to introduce an alternate notation, $\procn{x}$, when we
want to emphasize the connection to the use of the name. Note that, by
virtue of name equivalence, $\quotep{\procn{x}} \nameeq x$; so, the
notation is consistent with previous definitions.

Further, because names have structure it is possible to effect
substitutions on the basis of that structure. This means we need to
upgrade our notation for substitutions, which we accomplish by
adapting comprehension notation. Thus,

\begin{mathpar}
  P\{ y / x : x \in S \}
\end{mathpar}

is interpreted to mean the process derived from P by replacing (in a
capture-avoiding manner) each occurrence of $x$ in $S$ by $y$. For example,

\begin{mathpar}
  P\{ \quotep{\procn{x}|\procn{x}} / x : x \in \freenames{P} \}
\end{mathpar}

will replace each (occurrence) of a free name $x$ in $P$ by
$\quotep{\procn{x}|\procn{x}}$.

Also, we will avail ourselves of the notation $x^{L}$ and $x^{R}$ to
denote injections of a name into disjoint copies of the name
space. There are numerous ways to accomplish this. One example can be
found in \cite{MeredithR05}. This notation overloads to vectors of
names: $\vec{x}^{\pi} := (x_{i}^{\pi} \; : \; 0 \leq i < |\vec{x}| )$ where $\pi \in \{L,R\}$.

We also use $P^{\Box} := P|\Box$.

In \cite{MeredithR05} an interpretation of the new operator is
given. It turns out that there are several possible interpretations
all enjoying the requisite algebraic properties of the operator (see
\cite{milner91polyadicpi}). We will therefore make liberal use of
$(\nu\; \vec{x})P$.

% subsection the_syntax_and_semantics_of_the_notation_system (end)   

\input{qm2pi.qmops} 

\input{qm2pi.sterngerlach} 

\input{qm2pi.metric} 

% section concurrent_process_calculi (end)

%\input{qm2pi.proofsketch}

% section proof sketch (end)

%\input{qm2pi.slviaknots} 

% section spatial logic via knots (end)

\input{qm2pi.conclusion}

% section conclusion (end)

%\input{qm2pi.dtcodes} 

% section wiring algorithm (end)

\input{qm2pi.ack} 

% section acknowledgments (end)

\newpage


\bibliographystyle{plain}   
\bibliography{../../biblios/main.bib}

\input{qm2pi.rhodetails}

\end{document}

 

% subsection basic_interpretation (end)

%\input{qm2pi.rho.presentation} 
\subsection{The syntax and semantics of the notation system}\label{sub:the_syntax_and_semantics_of_the_notation_system} % (fold)

We now summarize a technical presentation of the calculus that
embodies our theory of dynamics. The typical presentation of such a
calculus follows the style of giving generators and relations on
them. The grammar, below, describing term constructors, freely
generates the set of processes, $\Proc$. This set is then quotiented
by a relation known as structural congruence and it is over this set
that the notion of dynamics is expressed. This presentation is
essentially that of \cite{MeredithR05} with the addition of
polyadicity and summation. For readability we have relegated some of
the technical subtleties to an appendix.

\subsubsection{Process grammar}\label{subsub:process_grammar}

\begin{mathpar}
  \inferrule* [lab=synchronization] {} {{M} \bc \pzero \;|\; x?F \;|\; x!C }
  \and
  \inferrule* [lab=abstraction] {} {{F} \bc (x)P}
  \and
  \inferrule* [lab=concretion] {} {{C} \bc \langle Q \rangle}
  \and
  \inferrule* [lab=process] {} {{P,Q} \bc M \;| \;P|Q \;|\; @{x}}
  \and
  \inferrule* [lab=name] {} {{x} \bc \quotep{P}}
\end{mathpar} 

Note that $\vec{x}$ (resp. $\vec{P}$) denotes a vector of names
(resp. processes) of length $|\vec{x}|$ (resp. $|\vec{P}|$). We adopt
the following useful abbreviations.

\begin{mathpar}
   x?(\vec{y}).P := x.(\vec{y})P \and  x\clift{\vec{P}} := x.\clift{\vec{P}}
   \and x!(y) := \lift{x}{\dropn{y}}
   \and \Pi_{i=0}^{n-1}P_i := P_0 | \ldots | P_{n-1}
\end{mathpar}

\subsubsection{Structural congruence}

\paragraph{Free and bound names and alpha-equivalence.} At the
core of structural equivalence is alpha-equivalence which identifies
process that are the same up to a change of variable. Formally, we
recognize the distinction between free and bound names. The free names
of a process, $\freenames{P}$, may be calculated recursively as
follows:

\begin{mathpar}
\freenames{\pzero} := \emptyset
  \and \\
  \freenames{x?(y).P} := \{ x \} \cup (\freenames{P} \setminus \{ y \})
  \and 
  \freenames{x!\langle P \rangle} := \{ x \} \cup \{ P \} 
  \and \\
  \freenames{P|Q} := \freenames{P} \cup \freenames{Q}
  \and \\
  \freenames{@{x}} := \{ x \}
\end{mathpar}

$\pi$
$\quotep{\pi}$

$\freenames{-} : \pi \to \mathcal{P}(\quotep{\pi})$

\begin{eqnarray*}
  \freenames{\pzero} & := & \emptyset \\
  \freenames{x?(y).P} & := & \{ x \} \cup (\freenames{P} \setminus \{ y \}) \\
  \freenames{x!\langle P \rangle} & := & \{ x \} \cup \{ P \} \\
  \freenames{P|Q} & := & \freenames{P} \cup \freenames{Q} \\
  \freenames{\dropn{x}} & := & \{ x \}
\end{eqnarray*}

The bound names of a process, $\boundnames{P}$, are those names occurring in $P$
that are not free. For example, in $x?(y).0$, the name $x$ is free, while $y$ is bound.

\begin{mathpar}
  \inferrule* [lab=monoidal-laws] {} { P|Q \equiv Q|P \and P|0 \equiv P \and P|(Q|R) \equiv (P|Q)|R }
\end{mathpar}

\begin{mathpar}
  \inferrule* [lab=alpha-equivalence] {} { (x)P \equiv (y)P\{y/x\} \and y \not\in \freenames{P} }
\end{mathpar}

\begin{definition}
Then two processes, $P,Q$, are alpha-equivalent if $P = Q\{\vec{y}/\vec{x}\}$ for
some $\vec{x} \in \boundnames{Q},\vec{y} \in \boundnames{P}$, where $Q\{\vec{y}/\vec{x}\}$
denotes the capture-avoiding substitution of $\vec{y}$ for $\vec{x}$ in $Q$.
\end{definition}

\begin{definition}
  The {\em structural congruence} \cite{SangiorgiWalker} , $\equiv$,
  between processes is the least congruence containing
  alpha-equivalence, satisfying the abelian monoid laws
  (associativity, commutativity and $\pzero$ as identity) for parallel
  composition $|$ and for summation $+$.
\end{definition}

\subsection{Name equivalence}

We take name equivalence, written $\nameeq$, to be the smallest
equivalence relation generated by the following rules.

\begin{mathpar}
\inferrule*[lab=Quote-drop]
{ }
{ \quotep{@{x}} \nameeq x }

\inferrule*[lab=Struct-equiv]
{ P \scong Q }
{ \quotep{P} \nameeq \quotep{Q} }
\end{mathpar}

The astute reader will have noticed that the mutual recursion of names
and processes imposes a mutual recursion on alpha-equivalence and
structural equivalence via name-equivalence. Fortunately, all of this
works out pleasantly and we may calculate in the natural way, free of
concern. The reader interested in the details is referred to the
appendix \ref{appendix:rho_details}.

\subsection{Substitution}

We use $\Proc$ for the set of processes, $\QProc$ for the set of
names, and $\id{\{}\vec{y} / \vec{x} \id{\}}$ to denote partial maps,
$s : \QProc \rightarrow \QProc$. A map, $s$ lifts, uniquely, to a map
on process terms, $\widehat{s} : \Proc \rightarrow \Proc$ by the
following equations.

\begin{mathpar}
  (0) \psubstp{Q}{P} := 0 \\
  (R \juxtap S) \psubstp{Q}{P}
  :=    
  (R)\psubstp{Q}{P} \juxtap (S) \psubstp{Q}{P} \\
  (x?(y).R) \psubstp{Q}{P}    
  :=    
  (x)\substp{Q}{P} (z)\concat( (R \psubstn{z}{y}) \psubstp{Q}{P} ) \\
  (\lift{x}{R}) \psubstp{Q}{P}  
  :=
  \lift{(x)\substp{Q}{P}}{ R \psubstp{Q}{P} } \\
%   (\dropn{x})  \psubstp{Q}{P}       
%   := 
%   \left\{ 
%     \begin{array}{ccc} 
%       \dropn{\quotep{Q}} & & x \nameeq \quotep{P} \\
%       \dropn{x} & & otherwise \\
%     \end{array}
%   \right. 
  (\dropn{x})  \psubstp{Q}{P}       
  := 
  \left\{ 
    \begin{array}{ccc} 
      Q & & x \nameeq \quotep{P} \\
      \dropn{x} & & otherwise \\
    \end{array}
  \right.
\end{mathpar}
 

where

\begin{eqnarray}
  (x)\id{\{} \lpquote Q \rpquote / \lpquote P \rpquote \id{\}}            = 
  \left\{ 
    \begin{array}{ccc}
      \lpquote Q \rpquote & & x \nameeq \lpquote P \rpquote \\
      x & & otherwise \\
    \end{array}
  \right. \nonumber
\end{eqnarray}

and $z$ is chosen distinct from $\quotep{P}$, $\quotep{Q}$, the free
names in $Q$, and all the names in $R$. Our $\alpha$-equivalence will
be built in the standard way from this substitution.

\begin{remark}\label{rem:no_self_referential_names}
  One consequence of these definitions is that $\forall P. \quotep{P}
  \not\in \freenames{P}$.
\end{remark}

\subsection{ Dynamic quote: an example }

Anticipating something of what's to come, consider applying the
substitution, $\widehat{\id{\{}u / z \id{\}}}$, to the following pair
of processes, $\lift{w}{y!(z)}$ and $w[ \lpquote y!(z) \rpquote ]$.

\begin{eqnarray}
	\lift{w}{y!(z)}\widehat{\id{\{}u / z \id{\}}}
		& = &
		\lift{w}{y!(u)} \nonumber\\
	w[ \lpquote y!(z) \rpquote ] \widehat{ \id{\{}u / z \id{\}} }
		& = &
		w[ \lpquote y!(z) \rpquote ] \nonumber
\end{eqnarray}

Because the body of the process between quotes is impervious to
substitution, we get radically different answers. In fact, by
examining the first process in an input context,
e.g. $x?(z).\lift{w}{y!(z)}$, we see that the process under the lift
operator may be shaped by prefixed inputs binding a name inside it. In
this sense, the lift operator will be seen as a way to dynamically
construct processes before reifying them as names.

Finally equipped with these standard features we can present the
dynamics of the calculus.

\subsubsection{Operational semantics} 

Finally, we introduce the computational dynamics. What marks these
algebras as distinct from other more traditionally studied algebraic
structures, e.g. vector spaces or polynomial rings, is the manner in
which dynamics is captured. In traditional structures, dynamics is typically
expressed through morphisms between such structures, as in linear maps
between vector spaces or morphisms between rings. In algebras
associated with the semantics of computation, the dynamics is
expressed as part of the algebraic structure itself, through a
reduction reduction relation typically denoted by $\red$. Below, we
give a recursive presentation of this relation for the calculus used
in the encoding.

$\red \subseteq \pi \times \pi$
$\red : \pi \to \mathcal{P}(\pi)$

\begin{mathpar}
  \inferrule* [lab=Comm] { \textsf{match}( x_{src}, x_{trgt} ) } { x_{trgt}?(y)P \; | \; x_{src}!\langle {Q} \rangle \red P\{\quotep{Q}/y}\} }
  \and \\
  \inferrule* [lab=Par] {{P} \red {P}'} {{{P} | {Q}} \red {{P}' | {Q}}}
  \and
  \inferrule* [lab=Equiv]{{{P} \scong {P}'} \andalso {{P}' \red {Q}'} \andalso {{Q}' \scong {Q}}}{{P} \red {Q}}
\end{mathpar}

\begin{eqnarray*}
  match_{\equiv} (\quotep{P},\quotep{Q}) & := & P \equiv Q \\
  match_{\dagger}(\quotep{P},\quotep{Q}) & := & \forall R. P|Q \red^{*} R => R \red^{*} 0 \\
  match_{K}(\quotep{P},\quotep{Q}) & := & K \mbox{ for some context } K
\end{eqnarray*}

$u?(x)P | u!\langle Q \rangle \red P\{\quotep{Q}/x\}$

%We write $\wred$ for $\red^*$, and $P\red$ if $\exists Q $ such that $ P \red Q$.
We write $P\red$ if $\exists Q $ such that $ P \red Q$ and $P\not\red$, otherwise.

\section{Replication}

As mentioned before, it is known that replication (and hence
recursion) can be implemented in a higher-order process algebra
\cite{SangiorgiWalker}. As our first example of calculation with the
machinery thus far presented we give the construction explicitly in
the {\rhoc}.

\begin{eqnarray}
	D_{x} & := & \prefix{x}{y}{(\binpar{\outputp{x}{y}}{@{y}})} \nonumber\\
	\bangp_{x}{P} & := & \binpar{{x}!\langle{\binpar{D_{x}}{P}}\rangle}{D_{x}} \nonumber
\end{eqnarray}

\begin{eqnarray}
	\bangp_{x}{P} & & \nonumber\\
	=
	& {x}!\langle{(\prefix{x}{y}{(\outputp{x}{y} | @{y})) | P}}\rangle 
	      | \prefix{x}{y}{(\outputp{x}{y} | @{y})} & \nonumber\\
	\red
	& (\outputp{x}{y} | @{y})\substn{\quotep{(\prefix{x}{y}{(@{y} | \outputp{x}{y})) | P}}}{y} & \nonumber\\
	=
	& \outputp{x}{\quotep{(\prefix{x}{y}{(\outputp{x}{y} | @{y})) | P}}}
	  | {(\prefix{x}{y}{(\outputp{x}{y} | @{y})) | P}} & \nonumber\\
	\red
	& \ldots & \nonumber\\
	\red^*
	& P | P | \ldots & \nonumber
\end{eqnarray}

Of course, this encoding, as an implementation, runs away, unfolding
$\bangp{P}$ eagerly. A lazier and more implementable replication
operator, restricted to input-guarded processes, may be obtained as follows.

\begin{eqnarray}
\bangp{\prefix{u}{v}{P}} 
	:= 
	\binpar{\lift{x}{\prefix{u}{v}{(\binpar{D(x)}{P})}}}{D(x)} \nonumber
\end{eqnarray}

\begin{remark}
  Note that the lazier definition still does not deal with summation
  or mixed summation (i.e. sums over input and output). The reader is
  invited to construct definitions of replication that deal with these
  features. 

  Further, the definitions are parameterized in a name, $x$. Can you,
  gentle reader, make a definition that eliminates this parameter and
  guarantees no accidental interaction between the replication
  machinery and the process being replicated -- i.e. no accidental
  sharing of names used by the process to get its work done and the
  name(s) used by the replication to effect copying. This latter
  revision of the definition of replication is crucial to obtaining
  the expected identity $!!P \sim !P$.
\end{remark}

\begin{remark}\label{rem:paradoxical_combinator}
  The reader familiar with the lambda calculus will have noticed the
  similarity between $D$ and the paradoxical combinator.

  [Ed. note: the existence of this seems to suggest we have to be more
  restrictive on the set of processes and names we admit if we are to
  support no-cloning.]
\end{remark}

\subsubsection{Bisimulation}

The computational dynamics gives rise to another kind of equivalence,
the equivalence of computational behavior. As previously mentioned
this is typically captured \emph{via} some form of bisimulation.

% The notion we use in this paper is weak barbed bisimulation
% \cite{milner91polyadicpi}.

The notion we use in this paper is derived from weak barbed
bisimulation \cite{milner91polyadicpi}. 

\begin{definition}
An \emph{observation relation}, $\downarrow_{\mathcal N}$, over a set
of names, $\mathcal N$, is the smallest relation satisfying the rules
below.

\infrule[Out-barb]{y \in {\mathcal N}, \; x \nameeq y}
		  {\outputp{x}{v} \downarrow_{\mathcal N} x}
\infrule[Par-barb]{\mbox{$P\downarrow_{\mathcal N} x$ or $Q\downarrow_{\mathcal N} x$}}
		  {\binpar{P}{Q} \downarrow_{\mathcal N} x}

We write $P \Downarrow_{\mathcal N} x$ if there is $Q$ such that 
$P \wred Q$ and $Q \downarrow_{\mathcal N} x$.
\end{definition}

\begin{definition}
%\label{def.bbisim}
An  ${\mathcal N}$-\emph{barbed bisimulation} over a set of names, ${\mathcal N}$, is a symmetric binary relation 
${\mathcal S}_{\mathcal N}$ between agents such that $P\rel{S}_{\mathcal N}Q$ implies:
\begin{enumerate}
\item If $P \red P'$ then $Q \wred Q'$ and $P'\rel{S}_{\mathcal N} Q'$.
\item If $P\downarrow_{\mathcal N} x$, then $Q\Downarrow_{\mathcal N} x$.
\end{enumerate}
$P$ is ${\mathcal N}$-barbed bisimilar to $Q$, written
$P \wbbisim_{\mathcal N} Q$, if $P \rel{S}_{\mathcal N} Q$ for some ${\mathcal N}$-barbed bisimulation ${\mathcal S}_{\mathcal N}$.
\end{definition}

$\mathcal{R} \subseteq \pi \times \pi$

$P \mathcal{R} Q => \forall P'. P \red P' \Rightarrow \exists Q'. Q \red Q', P' \mathcal{R} Q'$

$P \vdash x \Rightarrow Q \vdash x$

\begin{mathpar}
  \inferrule*[lab=Out-barb]{x \nameeq y}{{y}!\langle{Q}\rangle \vdash x}
  \and
  \inferrule*[lab=Par-barb]{\mbox{$P\vdash x$ or $Q\vdash x$}}{\binpar{P}{Q} \vdash x}
\end{mathpar}

\subsubsection{Contexts}

One of the principle advantages of computational calculi like the
$\pi$-calculus is a well-defined notion of context,
contextual-equivalence and a correlation between
contextual-equivalence and notions of bisimulation. The notion of
context allows the decomposition of a process into (sub-)process and
its syntactic environment, its context. Thus, a context may be
thought of as a process with a ``hole'' (written $\Box$) in it. The
application of a context $M$ to a process $P$, written $M[P]$, is
tantamount to filling the hole in $M$ with $P$. In this paper we do
not need the full weight of this theory, but do make use of the notion
of context in the proof the main theorem. 

\begin{mathpar}
  \inferrule* [lab=summation] {} {{M_{M},M_{N}} \bc \Box \;|\; x.M_{A} \;|\; M_{M}+M_{N}}
  \and
  \inferrule* [lab=agent] {} {{M_{A}} \bc (\vec{x})M_{P} \;| \; \clift{P_0,\ldots,M_{P},\ldots,P_N}}
  \and \\
  \inferrule* [lab=process] {} {{M_{P}} \bc M_{N} \;| \;P|M_{P} }
\end{mathpar} 

\begin{mathpar}
  \inferrule* [lab=sychronization] {} {M_{N} \bc \Box \;|\; x?M_{F} \;|\; x!M_{C}}
  \and
  \inferrule* [lab=abstraction] {} {{M_{F}} \bc (x)M_{P} }
  \and
  \inferrule* [lab=concretion] {} {{M_{C}} \bc \langle M_{P} \rangle }
  \and \\
  \inferrule* [lab=process] {} {{M_{P}} \bc M_{N} \;| \;P|M_{P} }
\end{mathpar}

\begin{definition}[contextual application] Given a context $M$, and
  process $P$, we define the \emph{contextual application}, $M[P] :=
  M\{P/\Box\}$. That is, the contextual application of M to P is the
  substitution of $P$ for $\Box$ in $M$.
\end{definition}

$\meaningof{-} : L \to \mathcal{P}(\pi)$

\begin{mathpar}
  \inferrule* [lab=collection] {} {\meaningof{true} = \pi, \and \meaningof{~E} = \pi \setminus \meaningof{E}, \and \meaningof{E_{1} \& E_{2}} = \meaningof{E_{1}} \cap \meaningof{E_{2}}}
\end{mathpar}

\begin{mathpar}
  \inferrule* [lab=structure] {} {\meaningof{0} = \{ P \in \pi | P \equiv 0 \}, \and \\ \meaningof{E_1 | E_2} = \{ P \in \pi | P \equiv P_{1} | P_{2}, P_{1} \in \meaningof{E_{1}}, P_{2} \in \meaningof{E_2}\} }
\end{mathpar}

\begin{mathpar}
 \inferrule* [lab=behavior] {} {\meaningof{\langle a?b \rangle E} = \{ P \in \pi | P \equiv Q | u?(y)P', \\ \and \\\\ \and \\ \;\;\; u \in \meaningof{a}, \forall z.P'\{z/y\} \in \meaningof{E\{z/b\}}\}, \and \\ \meaningof{a!E} = \{ P \in \pi | P \equiv Q | x!\langle P' \rangle, x \in \meaningof{a} P' \in \meaningof{E}\} }
\end{mathpar}

\begin{mathpar}
 \inferrule* [lab=nominal] {} {\meaningof{\quotep{E}} = \{ \quotep{P} \in \quotep{\pi} | P \in \meaningof{E} \}, \and \meaningof{\quotep{P}} = \{ \quotep{Q} \in \quotep{\pi} | P \equiv Q \} \and \\ \meaningof{@\quotep{E}} = \{ P \in \pi | P \equiv @x, x \in \meaningof{E} \}}
\end{mathpar}

\begin{eqnarray*}
  \\
  \meaningof{-} : TS \to ST
\end{eqnarray*}

\begin{eqnarray*}
  \\
  L : TS \to ST
\end{eqnarray*}

\begin{eqnarray*}
  \\
  P \models E \iff P \in \meaningof{E}
\end{eqnarray*}

\begin{eqnarray*}
  P \approx_{L} Q \iff \forall E \in L. P \models E \iff Q \models E
\end{eqnarray*}

\begin{eqnarray*}
  P \approx_{K} Q
\end{eqnarray*}

\begin{eqnarray*}
  P \approx Q
\end{eqnarray*}

$\approx_{K} = \approx = \approx_{L}$

\subsubsection{Contextual duality}

Note that contexts extend the quotation operation to a family of
operations from processes to names. Given a context, $M$, we can
define a \emph{nominal context}, $\quotep{M}$ by $\quotep{M}[P] :=
\quotep{M[P]}$. To foreshadow what is to come we observe that these
operations enjoy a duality with processes very much like the duality
between vectors and maps from vectors to scalars.

Further, because the calculus is essentially higher-order, we have a
correspondence between contexts and processes. More specifically,
given a name $x$ and a context $M$ we can construct $M^{*}_{x}$ such
that 

\begin{mathpar}
  M^{*}_{x} | \lift{x}{P} \red M[P]
\end{mathpar}

namely,

\begin{mathpar}
  M^{*}_{x} := x?(u).M[\dropn{u}]
\end{mathpar}

The dependence of $M^{*}_{x}$ on a name makes it an abstraction, 

\begin{mathpar}
  M^{*} := (x)x?(u).M[\dropn{u}]
\end{mathpar}

\subsection{Additional notation}

It will sometimes be convenient to denote the process a name
quotes. We already have the notation $x = \quotep{P}$, but it will be
convenient to introduce an alternate notation, $\procn{x}$, when we
want to emphasize the connection to the use of the name. Note that, by
virtue of name equivalence, $\quotep{\procn{x}} \nameeq x$; so, the
notation is consistent with previous definitions.

Further, because names have structure it is possible to effect
substitutions on the basis of that structure. This means we need to
upgrade our notation for substitutions, which we accomplish by
adapting comprehension notation. Thus,

\begin{mathpar}
  P\{ y / x : x \in S \}
\end{mathpar}

is interpreted to mean the process derived from P by replacing (in a
capture-avoiding manner) each occurrence of $x$ in $S$ by $y$. For example,

\begin{mathpar}
  P\{ \quotep{\procn{x}|\procn{x}} / x : x \in \freenames{P} \}
\end{mathpar}

will replace each (occurrence) of a free name $x$ in $P$ by
$\quotep{\procn{x}|\procn{x}}$.

Also, we will avail ourselves of the notation $x^{L}$ and $x^{R}$ to
denote injections of a name into disjoint copies of the name
space. There are numerous ways to accomplish this. One example can be
found in \cite{MeredithR05}. This notation overloads to vectors of
names: $\vec{x}^{\pi} := (x_{i}^{\pi} \; : \; 0 \leq i < |\vec{x}| )$ where $\pi \in \{L,R\}$.

We also use $P^{\Box} := P|\Box$.

In \cite{MeredithR05} an interpretation of the new operator is
given. It turns out that there are several possible interpretations
all enjoying the requisite algebraic properties of the operator (see
\cite{milner91polyadicpi}). We will therefore make liberal use of
$(\nu\; \vec{x})P$.

% subsection the_syntax_and_semantics_of_the_notation_system (end)   

\section{Interpretation of QM}
\subsection{Supporting definitions}
\subsubsection{Multiplication}
\begin{mathpar}
  \quotep{Q} \cdot \quotep{R} := \quotep{Q|R}
  \and \\
  \quotep{Q} \cdot P := P\{ \quotep{Q|R} / \quotep{R} : \quotep{R} \in \freenames{P} \}
\end{mathpar}

\paragraph{Discussion}
The first line needs little explanation. The second line says that
each free name of the process is replaced with the multiplication of
that name by the scalar. Multiplication of a scalar (name) by a state
(process) results in a process all the names of which have been `moved
over' by parallel composition with the process the scalar
quotes. There is a subtlety that the bound names have to be
manipulated so that multiplied names aren't accidentally
captured. There are many ways to achieve this.

\begin{remark}\label{rem:multiplication_identities}
  The reader is invited to verify that for all $x,y,z \in \QProc$ and $P \in \Proc$
  \begin{mathpar}
    x \cdot \quotep{0} \equiv x 
    \and
    x \cdot y \equiv y \cdot x
    \and
    x \cdot (y \cdot z) \equiv (x \cdot y) \cdot z
    \and \\
    \quotep{0} \cdot P \equiv P
    \and \\
    x \cdot (y \cdot P) \equiv (x \cdot y) \cdot P
    \and \\
    x \cdot (P|Q) \equiv (x \cdot P) | (x \cdot Q)
    \and \\    
  \end{mathpar}
\end{remark}

\subsubsection{Tensor product}

We define a tensor product on processes by structural induction.

\paragraph{Tensor of sums} First note that all summations, including
$\pzero$ and sequence, can be written $\Sigma_{i} x_{i}.A_{i} +
\Sigma_{j} x_{j}.C_{j}$, where we have grouped input-guarded processes
together and output-guarded processes together.

Thus, we can define the tensor product of two summations, $N_{1}\otimes N_{2}$, where

\begin{mathpar}
  N_{1} := \Sigma_{i} x_{i}.A_{i} + \Sigma_{j} x_{j}.C_{j}
  \and
  N_{2} := \Sigma_{i'} y_{i'}.B_{i'} + \Sigma_{j'} y_{j'}.D_{j'} 
\end{mathpar}

as follows.

\begin{mathpar}
  \Sigma_{i} x_{i}.A_{i} + \Sigma_{j} x_{j}.C_{j} \otimes \Sigma_{i'}
  y_{i'}.B_{i'} + \Sigma_{j'} y_{j'}.D_{j'} 
  \and \\
  := \; \Sigma_{i} \Sigma_{i'} \quotep{\stackrel{\vee}{x_{i}}| \stackrel{\vee}{y_{i'}}}.(A_{i}\otimes B_{i'}) \; | \; \Sigma_{i'} \Sigma_{i} \quotep{\stackrel{\vee}{y_{i'}}|\stackrel{\vee}{x_{i}}}.(B_{i'}\otimes A_{i})
  \and
  \;\; | \;\; \Sigma_{j} \Sigma_{j'} \quotep{\stackrel{\vee}{x_{j}}|\stackrel{\vee}{y_{j'}}}.(A_{j}\otimes B_{j'}) \; | \; \Sigma_{j'} \Sigma_{j} \quotep{\stackrel{\vee}{y_{j'}}|\stackrel{\vee}{x_{j}}}.(B_{j'}\otimes A_{j})
\end{mathpar}

\begin{remark}
  Do we need to $x^{L}$ and $y^{R}$ for this construction as well?
\end{remark}

\paragraph{Tensor of parallel compositions} Next, we distribute tensor
over par.

\begin{mathpar}
  P_{1}|P_{2} \otimes Q_{1}|Q_{2} := (P_{1} \otimes Q_{1}) | (P_{1}
  \otimes Q_{2}) | (P_{2} \otimes Q_{1}) | (P_{2} \otimes Q_{2})
\end{mathpar}

\paragraph{Tensor with dropped names} We treat tensor of a
process with a dropped name as parallel composition.

\begin{mathpar}
  P \otimes \dropn{x} := P | \dropn{x}
\end{mathpar}

\paragraph{Tensor of agents}

Finally, we need to define tensor on agents. Note that the definition
of tensor on normal products only tensors inputs with inputs and
outputs with outputs. Thus, we only have to define the operation on
``homogeneous'' pairings.

\begin{mathpar}
  (\vec{x})P \otimes (\vec{y})Q
  \and \\
  := (x_{0}^{L}|y_{0}^{R},\ldots,x_{0}^{L}|y_{n}^{R},\ldots,x_{m}^{L}|y_{0}^{R},\ldots,x_{m}^{L}|y_{n}^R)(P\{ \vec{x}^{L}/\vec{x}\} \otimes Q \{ \vec{y}^{R}/\vec{y}\})
  \and \\
  \clift{\vec{P}} \otimes \clift{\vec{Q}}
  \and \\
  := \clift{P_{0}\otimes Q_{0},\ldots,P_{0}\otimes Q_{n},\ldots,P_{m}\otimes Q_{0},\ldots,P_{m}\otimes Q_{n}}
\end{mathpar}

\begin{remark}
  Observe that arities of tensored abstractions matches arities of
  tensored concretions if the original arities matched. Note also that
  the length of the arities corresponds to the increase in dimension
  we see in ordinary vector space tensor product.
\end{remark}

\begin{remark}
  Operationally, this definition distributes the tensor down to
  components ``linked'' by summation. Tensor over summation is
  intriguing in that it mixes names. Moreover, as a consequence of the
  way it mixes names we have the identities for all $x \in \QProc$ and
  $P,Q \in \Proc$

  \begin{mathpar}
    (x \cdot P) \otimes Q \equiv x \cdot (P \otimes Q) \equiv P \otimes (x \cdot Q)
    \and
    P \otimes \pzero \equiv P
  \end{mathpar}

  that the reader is invited to verify.
\end{remark}

\subsubsection{Annihilation}
\begin{mathpar}
  P^{\perp} := \{ Q | \forall R. P|Q \red^{*} R \Rightarrow R \red^{*} \pzero \}
  \and \\
  P^{\underline{\perp}} := \Sigma_{Q \in P^{\perp}} \quotep{Q}?(y).(\dropn{y}|Q) | \Sigma_{Q \in P^{\perp}} \quotep{Q}\clift{\Box}
\end{mathpar}

\paragraph{Discussion} The reader will note that $P^{\perp}$ is a
\emph{set} of processes, while $P^{\underline{\perp}}$ is a
\emph{context}. We call the set $P^{\perp}$ the \emph{annihilators} of
$P$. The parallel composition of a process in the annihilators of $P$
with $P$ will result in a process, the state space of which has all
paths eventually leading to $\pzero$. Execution may endure loops; but
under reasonable conditions of fairness (naturally guaranteed under
most notions of bisimulation) such a composite process cannot get
stuck in such a loop and will, eventually pop out and terminate.

The context $P^{\underline{\perp}}$ is ready and willing to ``take the
$P$ out of'' the process to which it is applied. It will effectively
transmit the code of the process to which it is applied to one of the
annihilators and run the process against it.

\subsubsection{Evaluation}
We fix $M$ a domain of fully abstract interpretation with an equality
coincident with bisimulation. We take $\meaningof{\cdot} : \Proc \to
M$ to be the map interpreting processes and $\nmeaningof{\cdot} : \M
\to Proc$ to be the map running the other way. Then we define

\begin{mathpar}
  \int P := \nmeaningof{\meaningof{P}}
\end{mathpar}

\paragraph{Discussion}
There are many fully abstract interpretations of Milner's
$\pi$-calculus. Any of them can be used as a basis for interpreting
the reflective calculus here. Equipped with such a domain it is
largely a matter of grinding through to check that the Yoneda
construction for the normalization-by-evaluation program can be
extended to this setting.

\begin{remark}
  The reader is invited to verify that $\int (P^{\underline{\perp}}[P]) = 0$.
\end{remark}

\subsection{Quantum mechanics}

Table \ref{tbl:core_qm_op_defns} gives the core operational definitions

\begin{table}[htp]\label{tbl:core_qm_op_defns}
  \center{
    \fbox{
      \begin{tabular}{c|c}
        quantum mechanics & process calculus \\
        \hline
        scalar & $x := \quotep{P}$ \\
        state vector & $\state{P} := P$ \\
        dual & $\state{P}^{*} := \event{P^{\underline{\perp}}} := \quotep{P^{\underline{\perp}}}[-]$ \\
        matrix & $ \Sigma_{\alpha} \state{P_{\alpha}}x_{\alpha}\event{Q_{\alpha}}$ \\
        vector addition & $\state{P} + \state{Q} := \state{P | Q}$ \\
        tensor product & $\state{P} \otimes \state{Q} := \state{P \otimes Q}$ \\
        inner product & $\innerprod{P}{Q} := \quotep{\int P^{\underline{\perp}}[Q]}$ \\
      \end{tabular}
    }
  }
  \caption{QM - operational definitions}
\end{table}

where

\begin{mathpar}
  \prmatrix{P}{Q} := \fprmatrix{P}{\quotep{\pzero}}{Q}
  \and
  \fprmatrix{P}{x}{Q} := (\state{P},x,\event{Q})
  \and
  (\fprmatrix{P}{x}{Q})(\state{R}) := x \cdot \innerprod{Q}{R} \cdot \state{P}
  \and
  (\fprmatrix{P}{x}{Q})(\event{R}) := x \cdot \innerprod{R}{P} \cdot \event{Q}
\end{mathpar}

\paragraph{Discussion}
As promised: vectors (aka states) are represented as processes; duals
as contextual duals; inner product definition should be compared with
standard inner product definition for ....

\begin{remark}
  Assuming $\int (P^{\underline{\perp}}[P]) = 0$, the reader is
  invited to verify that $(\fprmatrix{P}{x}{P})(\state{P}) = x \cdot \state{P}$.
\end{remark}

\begin{remark}
  The reader is invited to verify that $\innerprod{P}{Q}$ could
  equally well have been written $\quotep{\int \stackrel{\vee}{x}}$
  where $x = \event{P^{\underline{\perp}}}(Q)$.

  One of the motivations for this remark is that there is another way
  to factor these operations. We could package up evaluation in the dual:

  \begin{mathpar}
    \state{P}^{*} := \event{\int P^{\underline{\perp}}} := \quotep{\int P^{\underline{\perp}}}[-]
  \end{mathpar}

  and then have inner product defined by
  
  \begin{mathpar}
    \innerprod{P}{Q} := \event{P}(Q)
  \end{mathpar}

  Hopefully, experience with the calculations will provide guidance on
  the best factoring.
\end{remark}

\begin{remark}
  Assuming $\int (P^{\underline{\perp}}[P]) = 0$, the reader is
  invited to verify that $\forall P,Q. (\prmatrix{0}{Q})(\state{0}) =
  \state{0}$ and dually $(\prmatrix{P}{0})(\event{0}) = \event{0}$.
\end{remark}

\begin{remark}
  i'm a little worried that i don't (yet) have proper support for
  complex conjugacy. But, the observation above may give us a
  clue. According to Abramsky, it must be the case that the scalars
  are iso to the homset of the identity for the tensor -- which the
  observation above characterizes. 

  For now, we will simply bookmark the notion with $\overline{x}$.
\end{remark}

\subsubsection{Adjointness}

We need to give a definition of $(\cdot)^{\dagger}$ for matrices. The
obvious candidate definition is
\begin{mathpar}
(\Sigma_{\alpha}\fprmatrix{P_{\alpha}}{x_{\alpha}}{Q_{\alpha}})^{\dagger}
= \Sigma_{\alpha}\fprmatrix{(Q_{\alpha}^{\underline{\perp}})^{*}}{\overline{x}_{\alpha}}{P_{\alpha}^{\underline{\perp}}} 
\end{mathpar}

But, $(Q_{\alpha}^{\underline{\perp}})^{*}$ requires a name along
which to communicate the process to achieve the context application.

\subsubsection{Basis for a basis}
If processes label states and ``addition'' of states (a.k.a. vector
addition) is interpreted as parallel composition, what corresponds to
notions of linear independence and basis? Here, we recall that Yoshida
has developed a set of \emph{combinators} for an asynchronous verison
of Milner's $\pi$-calculus. These are a finite set of processes such
any process can be expressed as parallel composition of these
combinators together with liberal uses of the new operator and
replication. We can simply give a translation of these into the
present calculus and have reasonable expectation that the property
carries over. That is, that the resultant set allows to express all
processes via parallel composition. Note, however, that there is no
new operator or replication in this calculus. As a result, we expect
that the corresponding set is actually infinite. That is, we expect
that the space is actually infinite dimensional.

\begin{remark}
  The attentive reader may be a bit concerned. Certainly, the
  collection $S$, $K$ and $I$ is a finite set of
  combinators. Shouldn't we expect to see a finite set of combinators
  for an effectively equivalent system? i am very sympathetic to this
  critique and feel it warrants full attention. On the other hand, i
  also have in mind the following analogy. The natural numbers, as a
  monoid under addition, has exactly $1$ generator, while the natural
  numbers, as a monoid under multiplication, has countably many
  generators (the primes). We observe that the application of the
  lambda calculus is much less resource sensitive than the parallel
  composition of the $\pi$-calculus. Could it be the case that we have
  an analogy of the form
  
  \begin{mathpar}
    m + n : MN :: m*n : M|N
  \end{mathpar}

  giving a similar blow up in the set of ``primes''?  This is such a
  wonderful thought that, even if it's not true, i think it's worth
  writing down.
\end{remark}
 

\documentclass[12pt]{llncs}
%\documentclass{jktr}

\usepackage[pdftex]{hyperref}                   
\usepackage {listings}
\usepackage {mathpartir}
\usepackage{bcprules}
%\usepackage{listings}
                       
\usepackage{graphicx} 
%\usepackage[margins=2.5cm,nohead,nofoot]{geometry}
%\usepackage{geometry}
\usepackage{amsfonts}
\usepackage{amstext}
\usepackage{latexsym}
\usepackage{amssymb}
\usepackage{color}


%\include{myPreamble}
\include{qm2pi.local} 

%\ifpdf
%\usepackage[pdftex]{graphicx}
%\else
%\usepackage{graphicx}
%\fi

 % \ifpdf
%  \usepackage{pdfsync}
%  \if


%\title{Brief Article}
%\author{David F. Snyder}
%\author{L.G. Meredith}

%\address{Dept. of Math., Texas State University--San Marcos, San Marcos, TX 78666}
       
\pagestyle{empty}


\begin{document}

\lstset{language=[Objective]Caml,frame=shadowbox}

\input{qm2pi.front}

% section front matter (end)

\input{qm2pi.intro} 
 
% section introduction (end)

% \input{qm2pi.knotations} 

% section notation (end)

\input{qm2pi.process.calculi} 

% section concurrent_process_calculi_and_spatial_logics_ (end)
    
%\input{qm2pi.knots2pi} 

%\input{qm2pi.trefoil} 

%\input{qm2pi.mainthm} 

% subsection basic_interpretation (end)

%\input{qm2pi.rho.presentation} 
\subsection{The syntax and semantics of the notation system}\label{sub:the_syntax_and_semantics_of_the_notation_system} % (fold)

We now summarize a technical presentation of the calculus that
embodies our theory of dynamics. The typical presentation of such a
calculus follows the style of giving generators and relations on
them. The grammar, below, describing term constructors, freely
generates the set of processes, $\Proc$. This set is then quotiented
by a relation known as structural congruence and it is over this set
that the notion of dynamics is expressed. This presentation is
essentially that of \cite{MeredithR05} with the addition of
polyadicity and summation. For readability we have relegated some of
the technical subtleties to an appendix.

\subsubsection{Process grammar}\label{subsub:process_grammar}

\begin{mathpar}
  \inferrule* [lab=synchronization] {} {{M} \bc \pzero \;|\; x?F \;|\; x!C }
  \and
  \inferrule* [lab=abstraction] {} {{F} \bc (x)P}
  \and
  \inferrule* [lab=concretion] {} {{C} \bc \langle Q \rangle}
  \and
  \inferrule* [lab=process] {} {{P,Q} \bc M \;| \;P|Q \;|\; @{x}}
  \and
  \inferrule* [lab=name] {} {{x} \bc \quotep{P}}
\end{mathpar} 

Note that $\vec{x}$ (resp. $\vec{P}$) denotes a vector of names
(resp. processes) of length $|\vec{x}|$ (resp. $|\vec{P}|$). We adopt
the following useful abbreviations.

\begin{mathpar}
   x?(\vec{y}).P := x.(\vec{y})P \and  x\clift{\vec{P}} := x.\clift{\vec{P}}
   \and x!(y) := \lift{x}{\dropn{y}}
   \and \Pi_{i=0}^{n-1}P_i := P_0 | \ldots | P_{n-1}
\end{mathpar}

\subsubsection{Structural congruence}

\paragraph{Free and bound names and alpha-equivalence.} At the
core of structural equivalence is alpha-equivalence which identifies
process that are the same up to a change of variable. Formally, we
recognize the distinction between free and bound names. The free names
of a process, $\freenames{P}$, may be calculated recursively as
follows:

\begin{mathpar}
\freenames{\pzero} := \emptyset
  \and \\
  \freenames{x?(y).P} := \{ x \} \cup (\freenames{P} \setminus \{ y \})
  \and 
  \freenames{x!\langle P \rangle} := \{ x \} \cup \{ P \} 
  \and \\
  \freenames{P|Q} := \freenames{P} \cup \freenames{Q}
  \and \\
  \freenames{@{x}} := \{ x \}
\end{mathpar}

$\pi$
$\quotep{\pi}$

$\freenames{-} : \pi \to \mathcal{P}(\quotep{\pi})$

\begin{eqnarray*}
  \freenames{\pzero} & := & \emptyset \\
  \freenames{x?(y).P} & := & \{ x \} \cup (\freenames{P} \setminus \{ y \}) \\
  \freenames{x!\langle P \rangle} & := & \{ x \} \cup \{ P \} \\
  \freenames{P|Q} & := & \freenames{P} \cup \freenames{Q} \\
  \freenames{\dropn{x}} & := & \{ x \}
\end{eqnarray*}

The bound names of a process, $\boundnames{P}$, are those names occurring in $P$
that are not free. For example, in $x?(y).0$, the name $x$ is free, while $y$ is bound.

\begin{mathpar}
  \inferrule* [lab=monoidal-laws] {} { P|Q \equiv Q|P \and P|0 \equiv P \and P|(Q|R) \equiv (P|Q)|R }
\end{mathpar}

\begin{mathpar}
  \inferrule* [lab=alpha-equivalence] {} { (x)P \equiv (y)P\{y/x\} \and y \not\in \freenames{P} }
\end{mathpar}

\begin{definition}
Then two processes, $P,Q$, are alpha-equivalent if $P = Q\{\vec{y}/\vec{x}\}$ for
some $\vec{x} \in \boundnames{Q},\vec{y} \in \boundnames{P}$, where $Q\{\vec{y}/\vec{x}\}$
denotes the capture-avoiding substitution of $\vec{y}$ for $\vec{x}$ in $Q$.
\end{definition}

\begin{definition}
  The {\em structural congruence} \cite{SangiorgiWalker} , $\equiv$,
  between processes is the least congruence containing
  alpha-equivalence, satisfying the abelian monoid laws
  (associativity, commutativity and $\pzero$ as identity) for parallel
  composition $|$ and for summation $+$.
\end{definition}

\subsection{Name equivalence}

We take name equivalence, written $\nameeq$, to be the smallest
equivalence relation generated by the following rules.

\begin{mathpar}
\inferrule*[lab=Quote-drop]
{ }
{ \quotep{@{x}} \nameeq x }

\inferrule*[lab=Struct-equiv]
{ P \scong Q }
{ \quotep{P} \nameeq \quotep{Q} }
\end{mathpar}

The astute reader will have noticed that the mutual recursion of names
and processes imposes a mutual recursion on alpha-equivalence and
structural equivalence via name-equivalence. Fortunately, all of this
works out pleasantly and we may calculate in the natural way, free of
concern. The reader interested in the details is referred to the
appendix \ref{appendix:rho_details}.

\subsection{Substitution}

We use $\Proc$ for the set of processes, $\QProc$ for the set of
names, and $\id{\{}\vec{y} / \vec{x} \id{\}}$ to denote partial maps,
$s : \QProc \rightarrow \QProc$. A map, $s$ lifts, uniquely, to a map
on process terms, $\widehat{s} : \Proc \rightarrow \Proc$ by the
following equations.

\begin{mathpar}
  (0) \psubstp{Q}{P} := 0 \\
  (R \juxtap S) \psubstp{Q}{P}
  :=    
  (R)\psubstp{Q}{P} \juxtap (S) \psubstp{Q}{P} \\
  (x?(y).R) \psubstp{Q}{P}    
  :=    
  (x)\substp{Q}{P} (z)\concat( (R \psubstn{z}{y}) \psubstp{Q}{P} ) \\
  (\lift{x}{R}) \psubstp{Q}{P}  
  :=
  \lift{(x)\substp{Q}{P}}{ R \psubstp{Q}{P} } \\
%   (\dropn{x})  \psubstp{Q}{P}       
%   := 
%   \left\{ 
%     \begin{array}{ccc} 
%       \dropn{\quotep{Q}} & & x \nameeq \quotep{P} \\
%       \dropn{x} & & otherwise \\
%     \end{array}
%   \right. 
  (\dropn{x})  \psubstp{Q}{P}       
  := 
  \left\{ 
    \begin{array}{ccc} 
      Q & & x \nameeq \quotep{P} \\
      \dropn{x} & & otherwise \\
    \end{array}
  \right.
\end{mathpar}
 

where

\begin{eqnarray}
  (x)\id{\{} \lpquote Q \rpquote / \lpquote P \rpquote \id{\}}            = 
  \left\{ 
    \begin{array}{ccc}
      \lpquote Q \rpquote & & x \nameeq \lpquote P \rpquote \\
      x & & otherwise \\
    \end{array}
  \right. \nonumber
\end{eqnarray}

and $z$ is chosen distinct from $\quotep{P}$, $\quotep{Q}$, the free
names in $Q$, and all the names in $R$. Our $\alpha$-equivalence will
be built in the standard way from this substitution.

\begin{remark}\label{rem:no_self_referential_names}
  One consequence of these definitions is that $\forall P. \quotep{P}
  \not\in \freenames{P}$.
\end{remark}

\subsection{ Dynamic quote: an example }

Anticipating something of what's to come, consider applying the
substitution, $\widehat{\id{\{}u / z \id{\}}}$, to the following pair
of processes, $\lift{w}{y!(z)}$ and $w[ \lpquote y!(z) \rpquote ]$.

\begin{eqnarray}
	\lift{w}{y!(z)}\widehat{\id{\{}u / z \id{\}}}
		& = &
		\lift{w}{y!(u)} \nonumber\\
	w[ \lpquote y!(z) \rpquote ] \widehat{ \id{\{}u / z \id{\}} }
		& = &
		w[ \lpquote y!(z) \rpquote ] \nonumber
\end{eqnarray}

Because the body of the process between quotes is impervious to
substitution, we get radically different answers. In fact, by
examining the first process in an input context,
e.g. $x?(z).\lift{w}{y!(z)}$, we see that the process under the lift
operator may be shaped by prefixed inputs binding a name inside it. In
this sense, the lift operator will be seen as a way to dynamically
construct processes before reifying them as names.

Finally equipped with these standard features we can present the
dynamics of the calculus.

\subsubsection{Operational semantics} 

Finally, we introduce the computational dynamics. What marks these
algebras as distinct from other more traditionally studied algebraic
structures, e.g. vector spaces or polynomial rings, is the manner in
which dynamics is captured. In traditional structures, dynamics is typically
expressed through morphisms between such structures, as in linear maps
between vector spaces or morphisms between rings. In algebras
associated with the semantics of computation, the dynamics is
expressed as part of the algebraic structure itself, through a
reduction reduction relation typically denoted by $\red$. Below, we
give a recursive presentation of this relation for the calculus used
in the encoding.

$\red \subseteq \pi \times \pi$
$\red : \pi \to \mathcal{P}(\pi)$

\begin{mathpar}
  \inferrule* [lab=Comm] { \textsf{match}( x_{src}, x_{trgt} ) } { x_{trgt}?(y)P \; | \; x_{src}!\langle {Q} \rangle \red P\{\quotep{Q}/y}\} }
  \and \\
  \inferrule* [lab=Par] {{P} \red {P}'} {{{P} | {Q}} \red {{P}' | {Q}}}
  \and
  \inferrule* [lab=Equiv]{{{P} \scong {P}'} \andalso {{P}' \red {Q}'} \andalso {{Q}' \scong {Q}}}{{P} \red {Q}}
\end{mathpar}

\begin{eqnarray*}
  match_{\equiv} (\quotep{P},\quotep{Q}) & := & P \equiv Q \\
  match_{\dagger}(\quotep{P},\quotep{Q}) & := & \forall R. P|Q \red^{*} R => R \red^{*} 0 \\
  match_{K}(\quotep{P},\quotep{Q}) & := & K \mbox{ for some context } K
\end{eqnarray*}

$u?(x)P | u!\langle Q \rangle \red P\{\quotep{Q}/x\}$

%We write $\wred$ for $\red^*$, and $P\red$ if $\exists Q $ such that $ P \red Q$.
We write $P\red$ if $\exists Q $ such that $ P \red Q$ and $P\not\red$, otherwise.

\section{Replication}

As mentioned before, it is known that replication (and hence
recursion) can be implemented in a higher-order process algebra
\cite{SangiorgiWalker}. As our first example of calculation with the
machinery thus far presented we give the construction explicitly in
the {\rhoc}.

\begin{eqnarray}
	D_{x} & := & \prefix{x}{y}{(\binpar{\outputp{x}{y}}{@{y}})} \nonumber\\
	\bangp_{x}{P} & := & \binpar{{x}!\langle{\binpar{D_{x}}{P}}\rangle}{D_{x}} \nonumber
\end{eqnarray}

\begin{eqnarray}
	\bangp_{x}{P} & & \nonumber\\
	=
	& {x}!\langle{(\prefix{x}{y}{(\outputp{x}{y} | @{y})) | P}}\rangle 
	      | \prefix{x}{y}{(\outputp{x}{y} | @{y})} & \nonumber\\
	\red
	& (\outputp{x}{y} | @{y})\substn{\quotep{(\prefix{x}{y}{(@{y} | \outputp{x}{y})) | P}}}{y} & \nonumber\\
	=
	& \outputp{x}{\quotep{(\prefix{x}{y}{(\outputp{x}{y} | @{y})) | P}}}
	  | {(\prefix{x}{y}{(\outputp{x}{y} | @{y})) | P}} & \nonumber\\
	\red
	& \ldots & \nonumber\\
	\red^*
	& P | P | \ldots & \nonumber
\end{eqnarray}

Of course, this encoding, as an implementation, runs away, unfolding
$\bangp{P}$ eagerly. A lazier and more implementable replication
operator, restricted to input-guarded processes, may be obtained as follows.

\begin{eqnarray}
\bangp{\prefix{u}{v}{P}} 
	:= 
	\binpar{\lift{x}{\prefix{u}{v}{(\binpar{D(x)}{P})}}}{D(x)} \nonumber
\end{eqnarray}

\begin{remark}
  Note that the lazier definition still does not deal with summation
  or mixed summation (i.e. sums over input and output). The reader is
  invited to construct definitions of replication that deal with these
  features. 

  Further, the definitions are parameterized in a name, $x$. Can you,
  gentle reader, make a definition that eliminates this parameter and
  guarantees no accidental interaction between the replication
  machinery and the process being replicated -- i.e. no accidental
  sharing of names used by the process to get its work done and the
  name(s) used by the replication to effect copying. This latter
  revision of the definition of replication is crucial to obtaining
  the expected identity $!!P \sim !P$.
\end{remark}

\begin{remark}\label{rem:paradoxical_combinator}
  The reader familiar with the lambda calculus will have noticed the
  similarity between $D$ and the paradoxical combinator.

  [Ed. note: the existence of this seems to suggest we have to be more
  restrictive on the set of processes and names we admit if we are to
  support no-cloning.]
\end{remark}

\subsubsection{Bisimulation}

The computational dynamics gives rise to another kind of equivalence,
the equivalence of computational behavior. As previously mentioned
this is typically captured \emph{via} some form of bisimulation.

% The notion we use in this paper is weak barbed bisimulation
% \cite{milner91polyadicpi}.

The notion we use in this paper is derived from weak barbed
bisimulation \cite{milner91polyadicpi}. 

\begin{definition}
An \emph{observation relation}, $\downarrow_{\mathcal N}$, over a set
of names, $\mathcal N$, is the smallest relation satisfying the rules
below.

\infrule[Out-barb]{y \in {\mathcal N}, \; x \nameeq y}
		  {\outputp{x}{v} \downarrow_{\mathcal N} x}
\infrule[Par-barb]{\mbox{$P\downarrow_{\mathcal N} x$ or $Q\downarrow_{\mathcal N} x$}}
		  {\binpar{P}{Q} \downarrow_{\mathcal N} x}

We write $P \Downarrow_{\mathcal N} x$ if there is $Q$ such that 
$P \wred Q$ and $Q \downarrow_{\mathcal N} x$.
\end{definition}

\begin{definition}
%\label{def.bbisim}
An  ${\mathcal N}$-\emph{barbed bisimulation} over a set of names, ${\mathcal N}$, is a symmetric binary relation 
${\mathcal S}_{\mathcal N}$ between agents such that $P\rel{S}_{\mathcal N}Q$ implies:
\begin{enumerate}
\item If $P \red P'$ then $Q \wred Q'$ and $P'\rel{S}_{\mathcal N} Q'$.
\item If $P\downarrow_{\mathcal N} x$, then $Q\Downarrow_{\mathcal N} x$.
\end{enumerate}
$P$ is ${\mathcal N}$-barbed bisimilar to $Q$, written
$P \wbbisim_{\mathcal N} Q$, if $P \rel{S}_{\mathcal N} Q$ for some ${\mathcal N}$-barbed bisimulation ${\mathcal S}_{\mathcal N}$.
\end{definition}

$\mathcal{R} \subseteq \pi \times \pi$

$P \mathcal{R} Q => \forall P'. P \red P' \Rightarrow \exists Q'. Q \red Q', P' \mathcal{R} Q'$

$P \vdash x \Rightarrow Q \vdash x$

\begin{mathpar}
  \inferrule*[lab=Out-barb]{x \nameeq y}{{y}!\langle{Q}\rangle \vdash x}
  \and
  \inferrule*[lab=Par-barb]{\mbox{$P\vdash x$ or $Q\vdash x$}}{\binpar{P}{Q} \vdash x}
\end{mathpar}

\subsubsection{Contexts}

One of the principle advantages of computational calculi like the
$\pi$-calculus is a well-defined notion of context,
contextual-equivalence and a correlation between
contextual-equivalence and notions of bisimulation. The notion of
context allows the decomposition of a process into (sub-)process and
its syntactic environment, its context. Thus, a context may be
thought of as a process with a ``hole'' (written $\Box$) in it. The
application of a context $M$ to a process $P$, written $M[P]$, is
tantamount to filling the hole in $M$ with $P$. In this paper we do
not need the full weight of this theory, but do make use of the notion
of context in the proof the main theorem. 

\begin{mathpar}
  \inferrule* [lab=summation] {} {{M_{M},M_{N}} \bc \Box \;|\; x.M_{A} \;|\; M_{M}+M_{N}}
  \and
  \inferrule* [lab=agent] {} {{M_{A}} \bc (\vec{x})M_{P} \;| \; \clift{P_0,\ldots,M_{P},\ldots,P_N}}
  \and \\
  \inferrule* [lab=process] {} {{M_{P}} \bc M_{N} \;| \;P|M_{P} }
\end{mathpar} 

\begin{mathpar}
  \inferrule* [lab=sychronization] {} {M_{N} \bc \Box \;|\; x?M_{F} \;|\; x!M_{C}}
  \and
  \inferrule* [lab=abstraction] {} {{M_{F}} \bc (x)M_{P} }
  \and
  \inferrule* [lab=concretion] {} {{M_{C}} \bc \langle M_{P} \rangle }
  \and \\
  \inferrule* [lab=process] {} {{M_{P}} \bc M_{N} \;| \;P|M_{P} }
\end{mathpar}

\begin{definition}[contextual application] Given a context $M$, and
  process $P$, we define the \emph{contextual application}, $M[P] :=
  M\{P/\Box\}$. That is, the contextual application of M to P is the
  substitution of $P$ for $\Box$ in $M$.
\end{definition}

$\meaningof{-} : L \to \mathcal{P}(\pi)$

\begin{mathpar}
  \inferrule* [lab=collection] {} {\meaningof{true} = \pi, \and \meaningof{~E} = \pi \setminus \meaningof{E}, \and \meaningof{E_{1} \& E_{2}} = \meaningof{E_{1}} \cap \meaningof{E_{2}}}
\end{mathpar}

\begin{mathpar}
  \inferrule* [lab=structure] {} {\meaningof{0} = \{ P \in \pi | P \equiv 0 \}, \and \\ \meaningof{E_1 | E_2} = \{ P \in \pi | P \equiv P_{1} | P_{2}, P_{1} \in \meaningof{E_{1}}, P_{2} \in \meaningof{E_2}\} }
\end{mathpar}

\begin{mathpar}
 \inferrule* [lab=behavior] {} {\meaningof{\langle a?b \rangle E} = \{ P \in \pi | P \equiv Q | u?(y)P', \\ \and \\\\ \and \\ \;\;\; u \in \meaningof{a}, \forall z.P'\{z/y\} \in \meaningof{E\{z/b\}}\}, \and \\ \meaningof{a!E} = \{ P \in \pi | P \equiv Q | x!\langle P' \rangle, x \in \meaningof{a} P' \in \meaningof{E}\} }
\end{mathpar}

\begin{mathpar}
 \inferrule* [lab=nominal] {} {\meaningof{\quotep{E}} = \{ \quotep{P} \in \quotep{\pi} | P \in \meaningof{E} \}, \and \meaningof{\quotep{P}} = \{ \quotep{Q} \in \quotep{\pi} | P \equiv Q \} \and \\ \meaningof{@\quotep{E}} = \{ P \in \pi | P \equiv @x, x \in \meaningof{E} \}}
\end{mathpar}

\begin{eqnarray*}
  \\
  \meaningof{-} : TS \to ST
\end{eqnarray*}

\begin{eqnarray*}
  \\
  L : TS \to ST
\end{eqnarray*}

\begin{eqnarray*}
  \\
  P \models E \iff P \in \meaningof{E}
\end{eqnarray*}

\begin{eqnarray*}
  P \approx_{L} Q \iff \forall E \in L. P \models E \iff Q \models E
\end{eqnarray*}

\begin{eqnarray*}
  P \approx_{K} Q
\end{eqnarray*}

\begin{eqnarray*}
  P \approx Q
\end{eqnarray*}

$\approx_{K} = \approx = \approx_{L}$

\subsubsection{Contextual duality}

Note that contexts extend the quotation operation to a family of
operations from processes to names. Given a context, $M$, we can
define a \emph{nominal context}, $\quotep{M}$ by $\quotep{M}[P] :=
\quotep{M[P]}$. To foreshadow what is to come we observe that these
operations enjoy a duality with processes very much like the duality
between vectors and maps from vectors to scalars.

Further, because the calculus is essentially higher-order, we have a
correspondence between contexts and processes. More specifically,
given a name $x$ and a context $M$ we can construct $M^{*}_{x}$ such
that 

\begin{mathpar}
  M^{*}_{x} | \lift{x}{P} \red M[P]
\end{mathpar}

namely,

\begin{mathpar}
  M^{*}_{x} := x?(u).M[\dropn{u}]
\end{mathpar}

The dependence of $M^{*}_{x}$ on a name makes it an abstraction, 

\begin{mathpar}
  M^{*} := (x)x?(u).M[\dropn{u}]
\end{mathpar}

\subsection{Additional notation}

It will sometimes be convenient to denote the process a name
quotes. We already have the notation $x = \quotep{P}$, but it will be
convenient to introduce an alternate notation, $\procn{x}$, when we
want to emphasize the connection to the use of the name. Note that, by
virtue of name equivalence, $\quotep{\procn{x}} \nameeq x$; so, the
notation is consistent with previous definitions.

Further, because names have structure it is possible to effect
substitutions on the basis of that structure. This means we need to
upgrade our notation for substitutions, which we accomplish by
adapting comprehension notation. Thus,

\begin{mathpar}
  P\{ y / x : x \in S \}
\end{mathpar}

is interpreted to mean the process derived from P by replacing (in a
capture-avoiding manner) each occurrence of $x$ in $S$ by $y$. For example,

\begin{mathpar}
  P\{ \quotep{\procn{x}|\procn{x}} / x : x \in \freenames{P} \}
\end{mathpar}

will replace each (occurrence) of a free name $x$ in $P$ by
$\quotep{\procn{x}|\procn{x}}$.

Also, we will avail ourselves of the notation $x^{L}$ and $x^{R}$ to
denote injections of a name into disjoint copies of the name
space. There are numerous ways to accomplish this. One example can be
found in \cite{MeredithR05}. This notation overloads to vectors of
names: $\vec{x}^{\pi} := (x_{i}^{\pi} \; : \; 0 \leq i < |\vec{x}| )$ where $\pi \in \{L,R\}$.

We also use $P^{\Box} := P|\Box$.

In \cite{MeredithR05} an interpretation of the new operator is
given. It turns out that there are several possible interpretations
all enjoying the requisite algebraic properties of the operator (see
\cite{milner91polyadicpi}). We will therefore make liberal use of
$(\nu\; \vec{x})P$.

% subsection the_syntax_and_semantics_of_the_notation_system (end)   

\input{qm2pi.qmops} 

\input{qm2pi.sterngerlach} 

\input{qm2pi.metric} 

% section concurrent_process_calculi (end)

%\input{qm2pi.proofsketch}

% section proof sketch (end)

%\input{qm2pi.slviaknots} 

% section spatial logic via knots (end)

\input{qm2pi.conclusion}

% section conclusion (end)

%\input{qm2pi.dtcodes} 

% section wiring algorithm (end)

\input{qm2pi.ack} 

% section acknowledgments (end)

\newpage


\bibliographystyle{plain}   
\bibliography{../../biblios/main.bib}

\input{qm2pi.rhodetails}

\end{document}

 

\documentclass[12pt]{llncs}
%\documentclass{jktr}

\usepackage[pdftex]{hyperref}                   
\usepackage {listings}
\usepackage {mathpartir}
\usepackage{bcprules}
%\usepackage{listings}
                       
\usepackage{graphicx} 
%\usepackage[margins=2.5cm,nohead,nofoot]{geometry}
%\usepackage{geometry}
\usepackage{amsfonts}
\usepackage{amstext}
\usepackage{latexsym}
\usepackage{amssymb}
\usepackage{color}


%\include{myPreamble}
\include{qm2pi.local} 

%\ifpdf
%\usepackage[pdftex]{graphicx}
%\else
%\usepackage{graphicx}
%\fi

 % \ifpdf
%  \usepackage{pdfsync}
%  \if


%\title{Brief Article}
%\author{David F. Snyder}
%\author{L.G. Meredith}

%\address{Dept. of Math., Texas State University--San Marcos, San Marcos, TX 78666}
       
\pagestyle{empty}


\begin{document}

\lstset{language=[Objective]Caml,frame=shadowbox}

\input{qm2pi.front}

% section front matter (end)

\input{qm2pi.intro} 
 
% section introduction (end)

% \input{qm2pi.knotations} 

% section notation (end)

\input{qm2pi.process.calculi} 

% section concurrent_process_calculi_and_spatial_logics_ (end)
    
%\input{qm2pi.knots2pi} 

%\input{qm2pi.trefoil} 

%\input{qm2pi.mainthm} 

% subsection basic_interpretation (end)

%\input{qm2pi.rho.presentation} 
\subsection{The syntax and semantics of the notation system}\label{sub:the_syntax_and_semantics_of_the_notation_system} % (fold)

We now summarize a technical presentation of the calculus that
embodies our theory of dynamics. The typical presentation of such a
calculus follows the style of giving generators and relations on
them. The grammar, below, describing term constructors, freely
generates the set of processes, $\Proc$. This set is then quotiented
by a relation known as structural congruence and it is over this set
that the notion of dynamics is expressed. This presentation is
essentially that of \cite{MeredithR05} with the addition of
polyadicity and summation. For readability we have relegated some of
the technical subtleties to an appendix.

\subsubsection{Process grammar}\label{subsub:process_grammar}

\begin{mathpar}
  \inferrule* [lab=synchronization] {} {{M} \bc \pzero \;|\; x?F \;|\; x!C }
  \and
  \inferrule* [lab=abstraction] {} {{F} \bc (x)P}
  \and
  \inferrule* [lab=concretion] {} {{C} \bc \langle Q \rangle}
  \and
  \inferrule* [lab=process] {} {{P,Q} \bc M \;| \;P|Q \;|\; @{x}}
  \and
  \inferrule* [lab=name] {} {{x} \bc \quotep{P}}
\end{mathpar} 

Note that $\vec{x}$ (resp. $\vec{P}$) denotes a vector of names
(resp. processes) of length $|\vec{x}|$ (resp. $|\vec{P}|$). We adopt
the following useful abbreviations.

\begin{mathpar}
   x?(\vec{y}).P := x.(\vec{y})P \and  x\clift{\vec{P}} := x.\clift{\vec{P}}
   \and x!(y) := \lift{x}{\dropn{y}}
   \and \Pi_{i=0}^{n-1}P_i := P_0 | \ldots | P_{n-1}
\end{mathpar}

\subsubsection{Structural congruence}

\paragraph{Free and bound names and alpha-equivalence.} At the
core of structural equivalence is alpha-equivalence which identifies
process that are the same up to a change of variable. Formally, we
recognize the distinction between free and bound names. The free names
of a process, $\freenames{P}$, may be calculated recursively as
follows:

\begin{mathpar}
\freenames{\pzero} := \emptyset
  \and \\
  \freenames{x?(y).P} := \{ x \} \cup (\freenames{P} \setminus \{ y \})
  \and 
  \freenames{x!\langle P \rangle} := \{ x \} \cup \{ P \} 
  \and \\
  \freenames{P|Q} := \freenames{P} \cup \freenames{Q}
  \and \\
  \freenames{@{x}} := \{ x \}
\end{mathpar}

$\pi$
$\quotep{\pi}$

$\freenames{-} : \pi \to \mathcal{P}(\quotep{\pi})$

\begin{eqnarray*}
  \freenames{\pzero} & := & \emptyset \\
  \freenames{x?(y).P} & := & \{ x \} \cup (\freenames{P} \setminus \{ y \}) \\
  \freenames{x!\langle P \rangle} & := & \{ x \} \cup \{ P \} \\
  \freenames{P|Q} & := & \freenames{P} \cup \freenames{Q} \\
  \freenames{\dropn{x}} & := & \{ x \}
\end{eqnarray*}

The bound names of a process, $\boundnames{P}$, are those names occurring in $P$
that are not free. For example, in $x?(y).0$, the name $x$ is free, while $y$ is bound.

\begin{mathpar}
  \inferrule* [lab=monoidal-laws] {} { P|Q \equiv Q|P \and P|0 \equiv P \and P|(Q|R) \equiv (P|Q)|R }
\end{mathpar}

\begin{mathpar}
  \inferrule* [lab=alpha-equivalence] {} { (x)P \equiv (y)P\{y/x\} \and y \not\in \freenames{P} }
\end{mathpar}

\begin{definition}
Then two processes, $P,Q$, are alpha-equivalent if $P = Q\{\vec{y}/\vec{x}\}$ for
some $\vec{x} \in \boundnames{Q},\vec{y} \in \boundnames{P}$, where $Q\{\vec{y}/\vec{x}\}$
denotes the capture-avoiding substitution of $\vec{y}$ for $\vec{x}$ in $Q$.
\end{definition}

\begin{definition}
  The {\em structural congruence} \cite{SangiorgiWalker} , $\equiv$,
  between processes is the least congruence containing
  alpha-equivalence, satisfying the abelian monoid laws
  (associativity, commutativity and $\pzero$ as identity) for parallel
  composition $|$ and for summation $+$.
\end{definition}

\subsection{Name equivalence}

We take name equivalence, written $\nameeq$, to be the smallest
equivalence relation generated by the following rules.

\begin{mathpar}
\inferrule*[lab=Quote-drop]
{ }
{ \quotep{@{x}} \nameeq x }

\inferrule*[lab=Struct-equiv]
{ P \scong Q }
{ \quotep{P} \nameeq \quotep{Q} }
\end{mathpar}

The astute reader will have noticed that the mutual recursion of names
and processes imposes a mutual recursion on alpha-equivalence and
structural equivalence via name-equivalence. Fortunately, all of this
works out pleasantly and we may calculate in the natural way, free of
concern. The reader interested in the details is referred to the
appendix \ref{appendix:rho_details}.

\subsection{Substitution}

We use $\Proc$ for the set of processes, $\QProc$ for the set of
names, and $\id{\{}\vec{y} / \vec{x} \id{\}}$ to denote partial maps,
$s : \QProc \rightarrow \QProc$. A map, $s$ lifts, uniquely, to a map
on process terms, $\widehat{s} : \Proc \rightarrow \Proc$ by the
following equations.

\begin{mathpar}
  (0) \psubstp{Q}{P} := 0 \\
  (R \juxtap S) \psubstp{Q}{P}
  :=    
  (R)\psubstp{Q}{P} \juxtap (S) \psubstp{Q}{P} \\
  (x?(y).R) \psubstp{Q}{P}    
  :=    
  (x)\substp{Q}{P} (z)\concat( (R \psubstn{z}{y}) \psubstp{Q}{P} ) \\
  (\lift{x}{R}) \psubstp{Q}{P}  
  :=
  \lift{(x)\substp{Q}{P}}{ R \psubstp{Q}{P} } \\
%   (\dropn{x})  \psubstp{Q}{P}       
%   := 
%   \left\{ 
%     \begin{array}{ccc} 
%       \dropn{\quotep{Q}} & & x \nameeq \quotep{P} \\
%       \dropn{x} & & otherwise \\
%     \end{array}
%   \right. 
  (\dropn{x})  \psubstp{Q}{P}       
  := 
  \left\{ 
    \begin{array}{ccc} 
      Q & & x \nameeq \quotep{P} \\
      \dropn{x} & & otherwise \\
    \end{array}
  \right.
\end{mathpar}
 

where

\begin{eqnarray}
  (x)\id{\{} \lpquote Q \rpquote / \lpquote P \rpquote \id{\}}            = 
  \left\{ 
    \begin{array}{ccc}
      \lpquote Q \rpquote & & x \nameeq \lpquote P \rpquote \\
      x & & otherwise \\
    \end{array}
  \right. \nonumber
\end{eqnarray}

and $z$ is chosen distinct from $\quotep{P}$, $\quotep{Q}$, the free
names in $Q$, and all the names in $R$. Our $\alpha$-equivalence will
be built in the standard way from this substitution.

\begin{remark}\label{rem:no_self_referential_names}
  One consequence of these definitions is that $\forall P. \quotep{P}
  \not\in \freenames{P}$.
\end{remark}

\subsection{ Dynamic quote: an example }

Anticipating something of what's to come, consider applying the
substitution, $\widehat{\id{\{}u / z \id{\}}}$, to the following pair
of processes, $\lift{w}{y!(z)}$ and $w[ \lpquote y!(z) \rpquote ]$.

\begin{eqnarray}
	\lift{w}{y!(z)}\widehat{\id{\{}u / z \id{\}}}
		& = &
		\lift{w}{y!(u)} \nonumber\\
	w[ \lpquote y!(z) \rpquote ] \widehat{ \id{\{}u / z \id{\}} }
		& = &
		w[ \lpquote y!(z) \rpquote ] \nonumber
\end{eqnarray}

Because the body of the process between quotes is impervious to
substitution, we get radically different answers. In fact, by
examining the first process in an input context,
e.g. $x?(z).\lift{w}{y!(z)}$, we see that the process under the lift
operator may be shaped by prefixed inputs binding a name inside it. In
this sense, the lift operator will be seen as a way to dynamically
construct processes before reifying them as names.

Finally equipped with these standard features we can present the
dynamics of the calculus.

\subsubsection{Operational semantics} 

Finally, we introduce the computational dynamics. What marks these
algebras as distinct from other more traditionally studied algebraic
structures, e.g. vector spaces or polynomial rings, is the manner in
which dynamics is captured. In traditional structures, dynamics is typically
expressed through morphisms between such structures, as in linear maps
between vector spaces or morphisms between rings. In algebras
associated with the semantics of computation, the dynamics is
expressed as part of the algebraic structure itself, through a
reduction reduction relation typically denoted by $\red$. Below, we
give a recursive presentation of this relation for the calculus used
in the encoding.

$\red \subseteq \pi \times \pi$
$\red : \pi \to \mathcal{P}(\pi)$

\begin{mathpar}
  \inferrule* [lab=Comm] { \textsf{match}( x_{src}, x_{trgt} ) } { x_{trgt}?(y)P \; | \; x_{src}!\langle {Q} \rangle \red P\{\quotep{Q}/y}\} }
  \and \\
  \inferrule* [lab=Par] {{P} \red {P}'} {{{P} | {Q}} \red {{P}' | {Q}}}
  \and
  \inferrule* [lab=Equiv]{{{P} \scong {P}'} \andalso {{P}' \red {Q}'} \andalso {{Q}' \scong {Q}}}{{P} \red {Q}}
\end{mathpar}

\begin{eqnarray*}
  match_{\equiv} (\quotep{P},\quotep{Q}) & := & P \equiv Q \\
  match_{\dagger}(\quotep{P},\quotep{Q}) & := & \forall R. P|Q \red^{*} R => R \red^{*} 0 \\
  match_{K}(\quotep{P},\quotep{Q}) & := & K \mbox{ for some context } K
\end{eqnarray*}

$u?(x)P | u!\langle Q \rangle \red P\{\quotep{Q}/x\}$

%We write $\wred$ for $\red^*$, and $P\red$ if $\exists Q $ such that $ P \red Q$.
We write $P\red$ if $\exists Q $ such that $ P \red Q$ and $P\not\red$, otherwise.

\section{Replication}

As mentioned before, it is known that replication (and hence
recursion) can be implemented in a higher-order process algebra
\cite{SangiorgiWalker}. As our first example of calculation with the
machinery thus far presented we give the construction explicitly in
the {\rhoc}.

\begin{eqnarray}
	D_{x} & := & \prefix{x}{y}{(\binpar{\outputp{x}{y}}{@{y}})} \nonumber\\
	\bangp_{x}{P} & := & \binpar{{x}!\langle{\binpar{D_{x}}{P}}\rangle}{D_{x}} \nonumber
\end{eqnarray}

\begin{eqnarray}
	\bangp_{x}{P} & & \nonumber\\
	=
	& {x}!\langle{(\prefix{x}{y}{(\outputp{x}{y} | @{y})) | P}}\rangle 
	      | \prefix{x}{y}{(\outputp{x}{y} | @{y})} & \nonumber\\
	\red
	& (\outputp{x}{y} | @{y})\substn{\quotep{(\prefix{x}{y}{(@{y} | \outputp{x}{y})) | P}}}{y} & \nonumber\\
	=
	& \outputp{x}{\quotep{(\prefix{x}{y}{(\outputp{x}{y} | @{y})) | P}}}
	  | {(\prefix{x}{y}{(\outputp{x}{y} | @{y})) | P}} & \nonumber\\
	\red
	& \ldots & \nonumber\\
	\red^*
	& P | P | \ldots & \nonumber
\end{eqnarray}

Of course, this encoding, as an implementation, runs away, unfolding
$\bangp{P}$ eagerly. A lazier and more implementable replication
operator, restricted to input-guarded processes, may be obtained as follows.

\begin{eqnarray}
\bangp{\prefix{u}{v}{P}} 
	:= 
	\binpar{\lift{x}{\prefix{u}{v}{(\binpar{D(x)}{P})}}}{D(x)} \nonumber
\end{eqnarray}

\begin{remark}
  Note that the lazier definition still does not deal with summation
  or mixed summation (i.e. sums over input and output). The reader is
  invited to construct definitions of replication that deal with these
  features. 

  Further, the definitions are parameterized in a name, $x$. Can you,
  gentle reader, make a definition that eliminates this parameter and
  guarantees no accidental interaction between the replication
  machinery and the process being replicated -- i.e. no accidental
  sharing of names used by the process to get its work done and the
  name(s) used by the replication to effect copying. This latter
  revision of the definition of replication is crucial to obtaining
  the expected identity $!!P \sim !P$.
\end{remark}

\begin{remark}\label{rem:paradoxical_combinator}
  The reader familiar with the lambda calculus will have noticed the
  similarity between $D$ and the paradoxical combinator.

  [Ed. note: the existence of this seems to suggest we have to be more
  restrictive on the set of processes and names we admit if we are to
  support no-cloning.]
\end{remark}

\subsubsection{Bisimulation}

The computational dynamics gives rise to another kind of equivalence,
the equivalence of computational behavior. As previously mentioned
this is typically captured \emph{via} some form of bisimulation.

% The notion we use in this paper is weak barbed bisimulation
% \cite{milner91polyadicpi}.

The notion we use in this paper is derived from weak barbed
bisimulation \cite{milner91polyadicpi}. 

\begin{definition}
An \emph{observation relation}, $\downarrow_{\mathcal N}$, over a set
of names, $\mathcal N$, is the smallest relation satisfying the rules
below.

\infrule[Out-barb]{y \in {\mathcal N}, \; x \nameeq y}
		  {\outputp{x}{v} \downarrow_{\mathcal N} x}
\infrule[Par-barb]{\mbox{$P\downarrow_{\mathcal N} x$ or $Q\downarrow_{\mathcal N} x$}}
		  {\binpar{P}{Q} \downarrow_{\mathcal N} x}

We write $P \Downarrow_{\mathcal N} x$ if there is $Q$ such that 
$P \wred Q$ and $Q \downarrow_{\mathcal N} x$.
\end{definition}

\begin{definition}
%\label{def.bbisim}
An  ${\mathcal N}$-\emph{barbed bisimulation} over a set of names, ${\mathcal N}$, is a symmetric binary relation 
${\mathcal S}_{\mathcal N}$ between agents such that $P\rel{S}_{\mathcal N}Q$ implies:
\begin{enumerate}
\item If $P \red P'$ then $Q \wred Q'$ and $P'\rel{S}_{\mathcal N} Q'$.
\item If $P\downarrow_{\mathcal N} x$, then $Q\Downarrow_{\mathcal N} x$.
\end{enumerate}
$P$ is ${\mathcal N}$-barbed bisimilar to $Q$, written
$P \wbbisim_{\mathcal N} Q$, if $P \rel{S}_{\mathcal N} Q$ for some ${\mathcal N}$-barbed bisimulation ${\mathcal S}_{\mathcal N}$.
\end{definition}

$\mathcal{R} \subseteq \pi \times \pi$

$P \mathcal{R} Q => \forall P'. P \red P' \Rightarrow \exists Q'. Q \red Q', P' \mathcal{R} Q'$

$P \vdash x \Rightarrow Q \vdash x$

\begin{mathpar}
  \inferrule*[lab=Out-barb]{x \nameeq y}{{y}!\langle{Q}\rangle \vdash x}
  \and
  \inferrule*[lab=Par-barb]{\mbox{$P\vdash x$ or $Q\vdash x$}}{\binpar{P}{Q} \vdash x}
\end{mathpar}

\subsubsection{Contexts}

One of the principle advantages of computational calculi like the
$\pi$-calculus is a well-defined notion of context,
contextual-equivalence and a correlation between
contextual-equivalence and notions of bisimulation. The notion of
context allows the decomposition of a process into (sub-)process and
its syntactic environment, its context. Thus, a context may be
thought of as a process with a ``hole'' (written $\Box$) in it. The
application of a context $M$ to a process $P$, written $M[P]$, is
tantamount to filling the hole in $M$ with $P$. In this paper we do
not need the full weight of this theory, but do make use of the notion
of context in the proof the main theorem. 

\begin{mathpar}
  \inferrule* [lab=summation] {} {{M_{M},M_{N}} \bc \Box \;|\; x.M_{A} \;|\; M_{M}+M_{N}}
  \and
  \inferrule* [lab=agent] {} {{M_{A}} \bc (\vec{x})M_{P} \;| \; \clift{P_0,\ldots,M_{P},\ldots,P_N}}
  \and \\
  \inferrule* [lab=process] {} {{M_{P}} \bc M_{N} \;| \;P|M_{P} }
\end{mathpar} 

\begin{mathpar}
  \inferrule* [lab=sychronization] {} {M_{N} \bc \Box \;|\; x?M_{F} \;|\; x!M_{C}}
  \and
  \inferrule* [lab=abstraction] {} {{M_{F}} \bc (x)M_{P} }
  \and
  \inferrule* [lab=concretion] {} {{M_{C}} \bc \langle M_{P} \rangle }
  \and \\
  \inferrule* [lab=process] {} {{M_{P}} \bc M_{N} \;| \;P|M_{P} }
\end{mathpar}

\begin{definition}[contextual application] Given a context $M$, and
  process $P$, we define the \emph{contextual application}, $M[P] :=
  M\{P/\Box\}$. That is, the contextual application of M to P is the
  substitution of $P$ for $\Box$ in $M$.
\end{definition}

$\meaningof{-} : L \to \mathcal{P}(\pi)$

\begin{mathpar}
  \inferrule* [lab=collection] {} {\meaningof{true} = \pi, \and \meaningof{~E} = \pi \setminus \meaningof{E}, \and \meaningof{E_{1} \& E_{2}} = \meaningof{E_{1}} \cap \meaningof{E_{2}}}
\end{mathpar}

\begin{mathpar}
  \inferrule* [lab=structure] {} {\meaningof{0} = \{ P \in \pi | P \equiv 0 \}, \and \\ \meaningof{E_1 | E_2} = \{ P \in \pi | P \equiv P_{1} | P_{2}, P_{1} \in \meaningof{E_{1}}, P_{2} \in \meaningof{E_2}\} }
\end{mathpar}

\begin{mathpar}
 \inferrule* [lab=behavior] {} {\meaningof{\langle a?b \rangle E} = \{ P \in \pi | P \equiv Q | u?(y)P', \\ \and \\\\ \and \\ \;\;\; u \in \meaningof{a}, \forall z.P'\{z/y\} \in \meaningof{E\{z/b\}}\}, \and \\ \meaningof{a!E} = \{ P \in \pi | P \equiv Q | x!\langle P' \rangle, x \in \meaningof{a} P' \in \meaningof{E}\} }
\end{mathpar}

\begin{mathpar}
 \inferrule* [lab=nominal] {} {\meaningof{\quotep{E}} = \{ \quotep{P} \in \quotep{\pi} | P \in \meaningof{E} \}, \and \meaningof{\quotep{P}} = \{ \quotep{Q} \in \quotep{\pi} | P \equiv Q \} \and \\ \meaningof{@\quotep{E}} = \{ P \in \pi | P \equiv @x, x \in \meaningof{E} \}}
\end{mathpar}

\begin{eqnarray*}
  \\
  \meaningof{-} : TS \to ST
\end{eqnarray*}

\begin{eqnarray*}
  \\
  L : TS \to ST
\end{eqnarray*}

\begin{eqnarray*}
  \\
  P \models E \iff P \in \meaningof{E}
\end{eqnarray*}

\begin{eqnarray*}
  P \approx_{L} Q \iff \forall E \in L. P \models E \iff Q \models E
\end{eqnarray*}

\begin{eqnarray*}
  P \approx_{K} Q
\end{eqnarray*}

\begin{eqnarray*}
  P \approx Q
\end{eqnarray*}

$\approx_{K} = \approx = \approx_{L}$

\subsubsection{Contextual duality}

Note that contexts extend the quotation operation to a family of
operations from processes to names. Given a context, $M$, we can
define a \emph{nominal context}, $\quotep{M}$ by $\quotep{M}[P] :=
\quotep{M[P]}$. To foreshadow what is to come we observe that these
operations enjoy a duality with processes very much like the duality
between vectors and maps from vectors to scalars.

Further, because the calculus is essentially higher-order, we have a
correspondence between contexts and processes. More specifically,
given a name $x$ and a context $M$ we can construct $M^{*}_{x}$ such
that 

\begin{mathpar}
  M^{*}_{x} | \lift{x}{P} \red M[P]
\end{mathpar}

namely,

\begin{mathpar}
  M^{*}_{x} := x?(u).M[\dropn{u}]
\end{mathpar}

The dependence of $M^{*}_{x}$ on a name makes it an abstraction, 

\begin{mathpar}
  M^{*} := (x)x?(u).M[\dropn{u}]
\end{mathpar}

\subsection{Additional notation}

It will sometimes be convenient to denote the process a name
quotes. We already have the notation $x = \quotep{P}$, but it will be
convenient to introduce an alternate notation, $\procn{x}$, when we
want to emphasize the connection to the use of the name. Note that, by
virtue of name equivalence, $\quotep{\procn{x}} \nameeq x$; so, the
notation is consistent with previous definitions.

Further, because names have structure it is possible to effect
substitutions on the basis of that structure. This means we need to
upgrade our notation for substitutions, which we accomplish by
adapting comprehension notation. Thus,

\begin{mathpar}
  P\{ y / x : x \in S \}
\end{mathpar}

is interpreted to mean the process derived from P by replacing (in a
capture-avoiding manner) each occurrence of $x$ in $S$ by $y$. For example,

\begin{mathpar}
  P\{ \quotep{\procn{x}|\procn{x}} / x : x \in \freenames{P} \}
\end{mathpar}

will replace each (occurrence) of a free name $x$ in $P$ by
$\quotep{\procn{x}|\procn{x}}$.

Also, we will avail ourselves of the notation $x^{L}$ and $x^{R}$ to
denote injections of a name into disjoint copies of the name
space. There are numerous ways to accomplish this. One example can be
found in \cite{MeredithR05}. This notation overloads to vectors of
names: $\vec{x}^{\pi} := (x_{i}^{\pi} \; : \; 0 \leq i < |\vec{x}| )$ where $\pi \in \{L,R\}$.

We also use $P^{\Box} := P|\Box$.

In \cite{MeredithR05} an interpretation of the new operator is
given. It turns out that there are several possible interpretations
all enjoying the requisite algebraic properties of the operator (see
\cite{milner91polyadicpi}). We will therefore make liberal use of
$(\nu\; \vec{x})P$.

% subsection the_syntax_and_semantics_of_the_notation_system (end)   

\input{qm2pi.qmops} 

\input{qm2pi.sterngerlach} 

\input{qm2pi.metric} 

% section concurrent_process_calculi (end)

%\input{qm2pi.proofsketch}

% section proof sketch (end)

%\input{qm2pi.slviaknots} 

% section spatial logic via knots (end)

\input{qm2pi.conclusion}

% section conclusion (end)

%\input{qm2pi.dtcodes} 

% section wiring algorithm (end)

\input{qm2pi.ack} 

% section acknowledgments (end)

\newpage


\bibliographystyle{plain}   
\bibliography{../../biblios/main.bib}

\input{qm2pi.rhodetails}

\end{document}

 

% section concurrent_process_calculi (end)

%\documentclass[12pt]{llncs}
%\documentclass{jktr}

\usepackage[pdftex]{hyperref}                   
\usepackage {listings}
\usepackage {mathpartir}
\usepackage{bcprules}
%\usepackage{listings}
                       
\usepackage{graphicx} 
%\usepackage[margins=2.5cm,nohead,nofoot]{geometry}
%\usepackage{geometry}
\usepackage{amsfonts}
\usepackage{amstext}
\usepackage{latexsym}
\usepackage{amssymb}
\usepackage{color}


%\include{myPreamble}
\include{qm2pi.local} 

%\ifpdf
%\usepackage[pdftex]{graphicx}
%\else
%\usepackage{graphicx}
%\fi

 % \ifpdf
%  \usepackage{pdfsync}
%  \if


%\title{Brief Article}
%\author{David F. Snyder}
%\author{L.G. Meredith}

%\address{Dept. of Math., Texas State University--San Marcos, San Marcos, TX 78666}
       
\pagestyle{empty}


\begin{document}

\lstset{language=[Objective]Caml,frame=shadowbox}

\input{qm2pi.front}

% section front matter (end)

\input{qm2pi.intro} 
 
% section introduction (end)

% \input{qm2pi.knotations} 

% section notation (end)

\input{qm2pi.process.calculi} 

% section concurrent_process_calculi_and_spatial_logics_ (end)
    
%\input{qm2pi.knots2pi} 

%\input{qm2pi.trefoil} 

%\input{qm2pi.mainthm} 

% subsection basic_interpretation (end)

%\input{qm2pi.rho.presentation} 
\subsection{The syntax and semantics of the notation system}\label{sub:the_syntax_and_semantics_of_the_notation_system} % (fold)

We now summarize a technical presentation of the calculus that
embodies our theory of dynamics. The typical presentation of such a
calculus follows the style of giving generators and relations on
them. The grammar, below, describing term constructors, freely
generates the set of processes, $\Proc$. This set is then quotiented
by a relation known as structural congruence and it is over this set
that the notion of dynamics is expressed. This presentation is
essentially that of \cite{MeredithR05} with the addition of
polyadicity and summation. For readability we have relegated some of
the technical subtleties to an appendix.

\subsubsection{Process grammar}\label{subsub:process_grammar}

\begin{mathpar}
  \inferrule* [lab=synchronization] {} {{M} \bc \pzero \;|\; x?F \;|\; x!C }
  \and
  \inferrule* [lab=abstraction] {} {{F} \bc (x)P}
  \and
  \inferrule* [lab=concretion] {} {{C} \bc \langle Q \rangle}
  \and
  \inferrule* [lab=process] {} {{P,Q} \bc M \;| \;P|Q \;|\; @{x}}
  \and
  \inferrule* [lab=name] {} {{x} \bc \quotep{P}}
\end{mathpar} 

Note that $\vec{x}$ (resp. $\vec{P}$) denotes a vector of names
(resp. processes) of length $|\vec{x}|$ (resp. $|\vec{P}|$). We adopt
the following useful abbreviations.

\begin{mathpar}
   x?(\vec{y}).P := x.(\vec{y})P \and  x\clift{\vec{P}} := x.\clift{\vec{P}}
   \and x!(y) := \lift{x}{\dropn{y}}
   \and \Pi_{i=0}^{n-1}P_i := P_0 | \ldots | P_{n-1}
\end{mathpar}

\subsubsection{Structural congruence}

\paragraph{Free and bound names and alpha-equivalence.} At the
core of structural equivalence is alpha-equivalence which identifies
process that are the same up to a change of variable. Formally, we
recognize the distinction between free and bound names. The free names
of a process, $\freenames{P}$, may be calculated recursively as
follows:

\begin{mathpar}
\freenames{\pzero} := \emptyset
  \and \\
  \freenames{x?(y).P} := \{ x \} \cup (\freenames{P} \setminus \{ y \})
  \and 
  \freenames{x!\langle P \rangle} := \{ x \} \cup \{ P \} 
  \and \\
  \freenames{P|Q} := \freenames{P} \cup \freenames{Q}
  \and \\
  \freenames{@{x}} := \{ x \}
\end{mathpar}

$\pi$
$\quotep{\pi}$

$\freenames{-} : \pi \to \mathcal{P}(\quotep{\pi})$

\begin{eqnarray*}
  \freenames{\pzero} & := & \emptyset \\
  \freenames{x?(y).P} & := & \{ x \} \cup (\freenames{P} \setminus \{ y \}) \\
  \freenames{x!\langle P \rangle} & := & \{ x \} \cup \{ P \} \\
  \freenames{P|Q} & := & \freenames{P} \cup \freenames{Q} \\
  \freenames{\dropn{x}} & := & \{ x \}
\end{eqnarray*}

The bound names of a process, $\boundnames{P}$, are those names occurring in $P$
that are not free. For example, in $x?(y).0$, the name $x$ is free, while $y$ is bound.

\begin{mathpar}
  \inferrule* [lab=monoidal-laws] {} { P|Q \equiv Q|P \and P|0 \equiv P \and P|(Q|R) \equiv (P|Q)|R }
\end{mathpar}

\begin{mathpar}
  \inferrule* [lab=alpha-equivalence] {} { (x)P \equiv (y)P\{y/x\} \and y \not\in \freenames{P} }
\end{mathpar}

\begin{definition}
Then two processes, $P,Q$, are alpha-equivalent if $P = Q\{\vec{y}/\vec{x}\}$ for
some $\vec{x} \in \boundnames{Q},\vec{y} \in \boundnames{P}$, where $Q\{\vec{y}/\vec{x}\}$
denotes the capture-avoiding substitution of $\vec{y}$ for $\vec{x}$ in $Q$.
\end{definition}

\begin{definition}
  The {\em structural congruence} \cite{SangiorgiWalker} , $\equiv$,
  between processes is the least congruence containing
  alpha-equivalence, satisfying the abelian monoid laws
  (associativity, commutativity and $\pzero$ as identity) for parallel
  composition $|$ and for summation $+$.
\end{definition}

\subsection{Name equivalence}

We take name equivalence, written $\nameeq$, to be the smallest
equivalence relation generated by the following rules.

\begin{mathpar}
\inferrule*[lab=Quote-drop]
{ }
{ \quotep{@{x}} \nameeq x }

\inferrule*[lab=Struct-equiv]
{ P \scong Q }
{ \quotep{P} \nameeq \quotep{Q} }
\end{mathpar}

The astute reader will have noticed that the mutual recursion of names
and processes imposes a mutual recursion on alpha-equivalence and
structural equivalence via name-equivalence. Fortunately, all of this
works out pleasantly and we may calculate in the natural way, free of
concern. The reader interested in the details is referred to the
appendix \ref{appendix:rho_details}.

\subsection{Substitution}

We use $\Proc$ for the set of processes, $\QProc$ for the set of
names, and $\id{\{}\vec{y} / \vec{x} \id{\}}$ to denote partial maps,
$s : \QProc \rightarrow \QProc$. A map, $s$ lifts, uniquely, to a map
on process terms, $\widehat{s} : \Proc \rightarrow \Proc$ by the
following equations.

\begin{mathpar}
  (0) \psubstp{Q}{P} := 0 \\
  (R \juxtap S) \psubstp{Q}{P}
  :=    
  (R)\psubstp{Q}{P} \juxtap (S) \psubstp{Q}{P} \\
  (x?(y).R) \psubstp{Q}{P}    
  :=    
  (x)\substp{Q}{P} (z)\concat( (R \psubstn{z}{y}) \psubstp{Q}{P} ) \\
  (\lift{x}{R}) \psubstp{Q}{P}  
  :=
  \lift{(x)\substp{Q}{P}}{ R \psubstp{Q}{P} } \\
%   (\dropn{x})  \psubstp{Q}{P}       
%   := 
%   \left\{ 
%     \begin{array}{ccc} 
%       \dropn{\quotep{Q}} & & x \nameeq \quotep{P} \\
%       \dropn{x} & & otherwise \\
%     \end{array}
%   \right. 
  (\dropn{x})  \psubstp{Q}{P}       
  := 
  \left\{ 
    \begin{array}{ccc} 
      Q & & x \nameeq \quotep{P} \\
      \dropn{x} & & otherwise \\
    \end{array}
  \right.
\end{mathpar}
 

where

\begin{eqnarray}
  (x)\id{\{} \lpquote Q \rpquote / \lpquote P \rpquote \id{\}}            = 
  \left\{ 
    \begin{array}{ccc}
      \lpquote Q \rpquote & & x \nameeq \lpquote P \rpquote \\
      x & & otherwise \\
    \end{array}
  \right. \nonumber
\end{eqnarray}

and $z$ is chosen distinct from $\quotep{P}$, $\quotep{Q}$, the free
names in $Q$, and all the names in $R$. Our $\alpha$-equivalence will
be built in the standard way from this substitution.

\begin{remark}\label{rem:no_self_referential_names}
  One consequence of these definitions is that $\forall P. \quotep{P}
  \not\in \freenames{P}$.
\end{remark}

\subsection{ Dynamic quote: an example }

Anticipating something of what's to come, consider applying the
substitution, $\widehat{\id{\{}u / z \id{\}}}$, to the following pair
of processes, $\lift{w}{y!(z)}$ and $w[ \lpquote y!(z) \rpquote ]$.

\begin{eqnarray}
	\lift{w}{y!(z)}\widehat{\id{\{}u / z \id{\}}}
		& = &
		\lift{w}{y!(u)} \nonumber\\
	w[ \lpquote y!(z) \rpquote ] \widehat{ \id{\{}u / z \id{\}} }
		& = &
		w[ \lpquote y!(z) \rpquote ] \nonumber
\end{eqnarray}

Because the body of the process between quotes is impervious to
substitution, we get radically different answers. In fact, by
examining the first process in an input context,
e.g. $x?(z).\lift{w}{y!(z)}$, we see that the process under the lift
operator may be shaped by prefixed inputs binding a name inside it. In
this sense, the lift operator will be seen as a way to dynamically
construct processes before reifying them as names.

Finally equipped with these standard features we can present the
dynamics of the calculus.

\subsubsection{Operational semantics} 

Finally, we introduce the computational dynamics. What marks these
algebras as distinct from other more traditionally studied algebraic
structures, e.g. vector spaces or polynomial rings, is the manner in
which dynamics is captured. In traditional structures, dynamics is typically
expressed through morphisms between such structures, as in linear maps
between vector spaces or morphisms between rings. In algebras
associated with the semantics of computation, the dynamics is
expressed as part of the algebraic structure itself, through a
reduction reduction relation typically denoted by $\red$. Below, we
give a recursive presentation of this relation for the calculus used
in the encoding.

$\red \subseteq \pi \times \pi$
$\red : \pi \to \mathcal{P}(\pi)$

\begin{mathpar}
  \inferrule* [lab=Comm] { \textsf{match}( x_{src}, x_{trgt} ) } { x_{trgt}?(y)P \; | \; x_{src}!\langle {Q} \rangle \red P\{\quotep{Q}/y}\} }
  \and \\
  \inferrule* [lab=Par] {{P} \red {P}'} {{{P} | {Q}} \red {{P}' | {Q}}}
  \and
  \inferrule* [lab=Equiv]{{{P} \scong {P}'} \andalso {{P}' \red {Q}'} \andalso {{Q}' \scong {Q}}}{{P} \red {Q}}
\end{mathpar}

\begin{eqnarray*}
  match_{\equiv} (\quotep{P},\quotep{Q}) & := & P \equiv Q \\
  match_{\dagger}(\quotep{P},\quotep{Q}) & := & \forall R. P|Q \red^{*} R => R \red^{*} 0 \\
  match_{K}(\quotep{P},\quotep{Q}) & := & K \mbox{ for some context } K
\end{eqnarray*}

$u?(x)P | u!\langle Q \rangle \red P\{\quotep{Q}/x\}$

%We write $\wred$ for $\red^*$, and $P\red$ if $\exists Q $ such that $ P \red Q$.
We write $P\red$ if $\exists Q $ such that $ P \red Q$ and $P\not\red$, otherwise.

\section{Replication}

As mentioned before, it is known that replication (and hence
recursion) can be implemented in a higher-order process algebra
\cite{SangiorgiWalker}. As our first example of calculation with the
machinery thus far presented we give the construction explicitly in
the {\rhoc}.

\begin{eqnarray}
	D_{x} & := & \prefix{x}{y}{(\binpar{\outputp{x}{y}}{@{y}})} \nonumber\\
	\bangp_{x}{P} & := & \binpar{{x}!\langle{\binpar{D_{x}}{P}}\rangle}{D_{x}} \nonumber
\end{eqnarray}

\begin{eqnarray}
	\bangp_{x}{P} & & \nonumber\\
	=
	& {x}!\langle{(\prefix{x}{y}{(\outputp{x}{y} | @{y})) | P}}\rangle 
	      | \prefix{x}{y}{(\outputp{x}{y} | @{y})} & \nonumber\\
	\red
	& (\outputp{x}{y} | @{y})\substn{\quotep{(\prefix{x}{y}{(@{y} | \outputp{x}{y})) | P}}}{y} & \nonumber\\
	=
	& \outputp{x}{\quotep{(\prefix{x}{y}{(\outputp{x}{y} | @{y})) | P}}}
	  | {(\prefix{x}{y}{(\outputp{x}{y} | @{y})) | P}} & \nonumber\\
	\red
	& \ldots & \nonumber\\
	\red^*
	& P | P | \ldots & \nonumber
\end{eqnarray}

Of course, this encoding, as an implementation, runs away, unfolding
$\bangp{P}$ eagerly. A lazier and more implementable replication
operator, restricted to input-guarded processes, may be obtained as follows.

\begin{eqnarray}
\bangp{\prefix{u}{v}{P}} 
	:= 
	\binpar{\lift{x}{\prefix{u}{v}{(\binpar{D(x)}{P})}}}{D(x)} \nonumber
\end{eqnarray}

\begin{remark}
  Note that the lazier definition still does not deal with summation
  or mixed summation (i.e. sums over input and output). The reader is
  invited to construct definitions of replication that deal with these
  features. 

  Further, the definitions are parameterized in a name, $x$. Can you,
  gentle reader, make a definition that eliminates this parameter and
  guarantees no accidental interaction between the replication
  machinery and the process being replicated -- i.e. no accidental
  sharing of names used by the process to get its work done and the
  name(s) used by the replication to effect copying. This latter
  revision of the definition of replication is crucial to obtaining
  the expected identity $!!P \sim !P$.
\end{remark}

\begin{remark}\label{rem:paradoxical_combinator}
  The reader familiar with the lambda calculus will have noticed the
  similarity between $D$ and the paradoxical combinator.

  [Ed. note: the existence of this seems to suggest we have to be more
  restrictive on the set of processes and names we admit if we are to
  support no-cloning.]
\end{remark}

\subsubsection{Bisimulation}

The computational dynamics gives rise to another kind of equivalence,
the equivalence of computational behavior. As previously mentioned
this is typically captured \emph{via} some form of bisimulation.

% The notion we use in this paper is weak barbed bisimulation
% \cite{milner91polyadicpi}.

The notion we use in this paper is derived from weak barbed
bisimulation \cite{milner91polyadicpi}. 

\begin{definition}
An \emph{observation relation}, $\downarrow_{\mathcal N}$, over a set
of names, $\mathcal N$, is the smallest relation satisfying the rules
below.

\infrule[Out-barb]{y \in {\mathcal N}, \; x \nameeq y}
		  {\outputp{x}{v} \downarrow_{\mathcal N} x}
\infrule[Par-barb]{\mbox{$P\downarrow_{\mathcal N} x$ or $Q\downarrow_{\mathcal N} x$}}
		  {\binpar{P}{Q} \downarrow_{\mathcal N} x}

We write $P \Downarrow_{\mathcal N} x$ if there is $Q$ such that 
$P \wred Q$ and $Q \downarrow_{\mathcal N} x$.
\end{definition}

\begin{definition}
%\label{def.bbisim}
An  ${\mathcal N}$-\emph{barbed bisimulation} over a set of names, ${\mathcal N}$, is a symmetric binary relation 
${\mathcal S}_{\mathcal N}$ between agents such that $P\rel{S}_{\mathcal N}Q$ implies:
\begin{enumerate}
\item If $P \red P'$ then $Q \wred Q'$ and $P'\rel{S}_{\mathcal N} Q'$.
\item If $P\downarrow_{\mathcal N} x$, then $Q\Downarrow_{\mathcal N} x$.
\end{enumerate}
$P$ is ${\mathcal N}$-barbed bisimilar to $Q$, written
$P \wbbisim_{\mathcal N} Q$, if $P \rel{S}_{\mathcal N} Q$ for some ${\mathcal N}$-barbed bisimulation ${\mathcal S}_{\mathcal N}$.
\end{definition}

$\mathcal{R} \subseteq \pi \times \pi$

$P \mathcal{R} Q => \forall P'. P \red P' \Rightarrow \exists Q'. Q \red Q', P' \mathcal{R} Q'$

$P \vdash x \Rightarrow Q \vdash x$

\begin{mathpar}
  \inferrule*[lab=Out-barb]{x \nameeq y}{{y}!\langle{Q}\rangle \vdash x}
  \and
  \inferrule*[lab=Par-barb]{\mbox{$P\vdash x$ or $Q\vdash x$}}{\binpar{P}{Q} \vdash x}
\end{mathpar}

\subsubsection{Contexts}

One of the principle advantages of computational calculi like the
$\pi$-calculus is a well-defined notion of context,
contextual-equivalence and a correlation between
contextual-equivalence and notions of bisimulation. The notion of
context allows the decomposition of a process into (sub-)process and
its syntactic environment, its context. Thus, a context may be
thought of as a process with a ``hole'' (written $\Box$) in it. The
application of a context $M$ to a process $P$, written $M[P]$, is
tantamount to filling the hole in $M$ with $P$. In this paper we do
not need the full weight of this theory, but do make use of the notion
of context in the proof the main theorem. 

\begin{mathpar}
  \inferrule* [lab=summation] {} {{M_{M},M_{N}} \bc \Box \;|\; x.M_{A} \;|\; M_{M}+M_{N}}
  \and
  \inferrule* [lab=agent] {} {{M_{A}} \bc (\vec{x})M_{P} \;| \; \clift{P_0,\ldots,M_{P},\ldots,P_N}}
  \and \\
  \inferrule* [lab=process] {} {{M_{P}} \bc M_{N} \;| \;P|M_{P} }
\end{mathpar} 

\begin{mathpar}
  \inferrule* [lab=sychronization] {} {M_{N} \bc \Box \;|\; x?M_{F} \;|\; x!M_{C}}
  \and
  \inferrule* [lab=abstraction] {} {{M_{F}} \bc (x)M_{P} }
  \and
  \inferrule* [lab=concretion] {} {{M_{C}} \bc \langle M_{P} \rangle }
  \and \\
  \inferrule* [lab=process] {} {{M_{P}} \bc M_{N} \;| \;P|M_{P} }
\end{mathpar}

\begin{definition}[contextual application] Given a context $M$, and
  process $P$, we define the \emph{contextual application}, $M[P] :=
  M\{P/\Box\}$. That is, the contextual application of M to P is the
  substitution of $P$ for $\Box$ in $M$.
\end{definition}

$\meaningof{-} : L \to \mathcal{P}(\pi)$

\begin{mathpar}
  \inferrule* [lab=collection] {} {\meaningof{true} = \pi, \and \meaningof{~E} = \pi \setminus \meaningof{E}, \and \meaningof{E_{1} \& E_{2}} = \meaningof{E_{1}} \cap \meaningof{E_{2}}}
\end{mathpar}

\begin{mathpar}
  \inferrule* [lab=structure] {} {\meaningof{0} = \{ P \in \pi | P \equiv 0 \}, \and \\ \meaningof{E_1 | E_2} = \{ P \in \pi | P \equiv P_{1} | P_{2}, P_{1} \in \meaningof{E_{1}}, P_{2} \in \meaningof{E_2}\} }
\end{mathpar}

\begin{mathpar}
 \inferrule* [lab=behavior] {} {\meaningof{\langle a?b \rangle E} = \{ P \in \pi | P \equiv Q | u?(y)P', \\ \and \\\\ \and \\ \;\;\; u \in \meaningof{a}, \forall z.P'\{z/y\} \in \meaningof{E\{z/b\}}\}, \and \\ \meaningof{a!E} = \{ P \in \pi | P \equiv Q | x!\langle P' \rangle, x \in \meaningof{a} P' \in \meaningof{E}\} }
\end{mathpar}

\begin{mathpar}
 \inferrule* [lab=nominal] {} {\meaningof{\quotep{E}} = \{ \quotep{P} \in \quotep{\pi} | P \in \meaningof{E} \}, \and \meaningof{\quotep{P}} = \{ \quotep{Q} \in \quotep{\pi} | P \equiv Q \} \and \\ \meaningof{@\quotep{E}} = \{ P \in \pi | P \equiv @x, x \in \meaningof{E} \}}
\end{mathpar}

\begin{eqnarray*}
  \\
  \meaningof{-} : TS \to ST
\end{eqnarray*}

\begin{eqnarray*}
  \\
  L : TS \to ST
\end{eqnarray*}

\begin{eqnarray*}
  \\
  P \models E \iff P \in \meaningof{E}
\end{eqnarray*}

\begin{eqnarray*}
  P \approx_{L} Q \iff \forall E \in L. P \models E \iff Q \models E
\end{eqnarray*}

\begin{eqnarray*}
  P \approx_{K} Q
\end{eqnarray*}

\begin{eqnarray*}
  P \approx Q
\end{eqnarray*}

$\approx_{K} = \approx = \approx_{L}$

\subsubsection{Contextual duality}

Note that contexts extend the quotation operation to a family of
operations from processes to names. Given a context, $M$, we can
define a \emph{nominal context}, $\quotep{M}$ by $\quotep{M}[P] :=
\quotep{M[P]}$. To foreshadow what is to come we observe that these
operations enjoy a duality with processes very much like the duality
between vectors and maps from vectors to scalars.

Further, because the calculus is essentially higher-order, we have a
correspondence between contexts and processes. More specifically,
given a name $x$ and a context $M$ we can construct $M^{*}_{x}$ such
that 

\begin{mathpar}
  M^{*}_{x} | \lift{x}{P} \red M[P]
\end{mathpar}

namely,

\begin{mathpar}
  M^{*}_{x} := x?(u).M[\dropn{u}]
\end{mathpar}

The dependence of $M^{*}_{x}$ on a name makes it an abstraction, 

\begin{mathpar}
  M^{*} := (x)x?(u).M[\dropn{u}]
\end{mathpar}

\subsection{Additional notation}

It will sometimes be convenient to denote the process a name
quotes. We already have the notation $x = \quotep{P}$, but it will be
convenient to introduce an alternate notation, $\procn{x}$, when we
want to emphasize the connection to the use of the name. Note that, by
virtue of name equivalence, $\quotep{\procn{x}} \nameeq x$; so, the
notation is consistent with previous definitions.

Further, because names have structure it is possible to effect
substitutions on the basis of that structure. This means we need to
upgrade our notation for substitutions, which we accomplish by
adapting comprehension notation. Thus,

\begin{mathpar}
  P\{ y / x : x \in S \}
\end{mathpar}

is interpreted to mean the process derived from P by replacing (in a
capture-avoiding manner) each occurrence of $x$ in $S$ by $y$. For example,

\begin{mathpar}
  P\{ \quotep{\procn{x}|\procn{x}} / x : x \in \freenames{P} \}
\end{mathpar}

will replace each (occurrence) of a free name $x$ in $P$ by
$\quotep{\procn{x}|\procn{x}}$.

Also, we will avail ourselves of the notation $x^{L}$ and $x^{R}$ to
denote injections of a name into disjoint copies of the name
space. There are numerous ways to accomplish this. One example can be
found in \cite{MeredithR05}. This notation overloads to vectors of
names: $\vec{x}^{\pi} := (x_{i}^{\pi} \; : \; 0 \leq i < |\vec{x}| )$ where $\pi \in \{L,R\}$.

We also use $P^{\Box} := P|\Box$.

In \cite{MeredithR05} an interpretation of the new operator is
given. It turns out that there are several possible interpretations
all enjoying the requisite algebraic properties of the operator (see
\cite{milner91polyadicpi}). We will therefore make liberal use of
$(\nu\; \vec{x})P$.

% subsection the_syntax_and_semantics_of_the_notation_system (end)   

\input{qm2pi.qmops} 

\input{qm2pi.sterngerlach} 

\input{qm2pi.metric} 

% section concurrent_process_calculi (end)

%\input{qm2pi.proofsketch}

% section proof sketch (end)

%\input{qm2pi.slviaknots} 

% section spatial logic via knots (end)

\input{qm2pi.conclusion}

% section conclusion (end)

%\input{qm2pi.dtcodes} 

% section wiring algorithm (end)

\input{qm2pi.ack} 

% section acknowledgments (end)

\newpage


\bibliographystyle{plain}   
\bibliography{../../biblios/main.bib}

\input{qm2pi.rhodetails}

\end{document}



% section proof sketch (end)

%\section{Unlikely characters: spatial logic for
  knots}\label{sub:characteristic_formulae} % (fold)

Associated to the mobile process calculi are a family of logics known
as the Hennessy-Milner logics. These logics typically enjoy a
semantics interpreting formulae as sets of processes that when
factored through the encoding outlined above allows an identification
of classes of knots with logical formulae. In the context of this
encoding the sub-family known as the spatial logics \cite{CairesC03}
\cite{CairesC04} \cite{Caires04} are of particular interest providing
several important features for expressing and reasoning about
properties (i.e. classes) of knots. We hint here at how this may be done.

%\begin{description}
%\item [structural connectives] 
\subsubsection{Structural connectives} The spatial logics enjoy
structural connectives corresponding, at the logical level, to the
parallel composition ($P | Q$) and new name ($(\nu \; x)P$)
connectives for processes. As illustrated in the examples below, these
connectives are extremely expressive given the shape of our encoding.
%\item [decideable satisfaction]

\subsubsection{Decideable satisfaction}
In \cite{Caires04} the satisfaction relation is shown to be decideable
for a rich class of processes. It further turns out that the image of
the our encoding is a proper subset of that class. This result
provides the basis for an algorithm by which to search for knots
enjoying a given property.
%\item [characteristic formulae]

\subsubsection{Characteristic formulae}
In the same paper \cite{Caires04} , Caires presents a means of calculating
characteristic formulae, selecting equivalence classes of processes
up to a pre--specified depth limit on the support set of names. Composed with our
encoding, this characteristic formula can be used to select
characteristic formulae for knots.
%\end{description}

\subsubsection{Spatial logic formulae}

The grammar below (segmented for comprehension) summarizes the syntax
of spatial logic formulae. We employ illustrative examples in the
sequel to provide an intuitive understanding of their meaning
referring the reader to \cite{Caires04} for a more detailed explication
of the semantics.

\begin{mathpar}
  \inferrule* [lab=boolean] {} {{A,B} \bc T \;|\; \neg A \;|\; A \wedge B \;|\; \eta = \eta'}
  \and
  \inferrule* [lab=spatial] {} {|\; \pzero \;|\; A | B \;|\; x \text{\textregistered} A \;|\; \forall x . A \;|\;  H x . A}
  \and
  \inferrule* [lab=behavioral] {} {|\; \alpha . A}
  \and 
  \inferrule* [lab=recursion] {} {|\; X(\vec{u}) \;|\; \mu X(\vec{u}) . A}
  \and
  \inferrule* [lab=action] {} {\alpha \bc \langle x?(\vec{y}) \rangle \;|\; \langle x!(\vec{y}) \rangle \;|\; \langle \tau \rangle}
  \and 
  \inferrule* [lab=name] {} {\eta \bc x \;|\; \tau}
\end{mathpar} 

% subsection characteristic_formulae (end)   	 

\subsection{Example formulae}\label{sub:example_formulae_} % (fold)

\subsubsection{Crossing as formula.}
% 
% \begin{align*}
%   \frac{d}{dx} \sin x &= \cos x 
%   & \frac{d}{dx} e^x &= e^x \\
%   \frac{d}{dx} \cos x &= - \sin x 
%   & \frac{d}{dx} \log x &= \frac{1}{x} \\
% \end{align*} 

\begin{align*}
 \mu C(x_{0},x_{1},y_{0},y_{1},u).&(\langle x_{0}?(z) \rangle(\langle u! \rangle\langle y_{1}!z \rangle C(x_{0},x_{1},y_{0},y_{1},u)) & \\
  & \wedge \langle y_{1}?(z) \rangle (\langle u! \rangle \langle x_{0}!z \rangle C(x_{0},x_{1},y_{0},y_{1},u)) & \\
  & \wedge \langle x_{1}?(z) \rangle (\langle u? \rangle \langle y_{0}!z \rangle C(x_{0},x_{1},y_{0},y_{1},u)) & \\
  & \wedge \langle y_{0}?(z) \rangle (\langle u? \rangle \langle x_{1}!z \rangle C(x_{0},x_{1},y_{0},y_{1},u))) &
\end{align*}

The lexicographical similarity between the shape of this formulae and
the shape of definition of the process representing a crossing reveals
the intuitive meaning of this formulae. It describes the capabilities
of a process that has the right to represent a crossing. For example
it picks out processes that may perform an input on the port $x_0$ in
its initial menu of capabilities. What differentiates the formula
from the process, however, is that the crossing process is the
smallest candidate to satisfy the formula. Infinitely many other
processes -- with internal behavior hidden behind this interface, so
to speak -- also satisfy this formula. Even this simple formula,
then, can be seen to open a new view onto knots, providing a
computational interpretation of \emph{virtual} knots.

Note that this formula is derived by hand. A similar formula can be
derived by employing Caires' calculation of characteristic formula
\cite{Caires04} to the process representing a crossing. In light of
this discussion, we let
$\meaningof{C}_{\phi}(x0,x1,y0,y1,u)$ denote a formula specifying the
dynamics we wish to capture of a crossing. To guarantee we preserve
the shape of the interface and minimal semantics we demand that
$\meaningof{C}_{\phi}(x0,x1,y0,y1,u) \Rightarrow
\textbf{C}(x0,x1,y0,y1,u)$ where $\textbf{C}(x0,x1,y0,y1,u)$ denotes
the formula above.
                            
\subsubsection{Crossing number constraints.}
The moral content of the context lemma (Lemma \ref{context}) is that the notion of
``locality'' in the Reidemeister moves is effectively captured by the
parallel composition operator of the process calculus. This intuition
extends through the logic. Given a formula,
$\meaningof{C}_{\phi}(x0,x1,y0,y1,u)$, we can use the structural
connectives to specify constraints on crossing numbers, such as at
least $n$ crossings, or exactly $n$ crossings.
\begin{mathpar}
  \inferrule* [lab=at-least-n] {} { K^{\geq n}_{\phi}(\vec{xs},\vec{ys}) := \Pi_{i=0}^{n-1} Hu . \meaningof{C}_{\phi}(xs_i,ys_i,u) | T }
  \and 
  \inferrule* [lab=exactly-n] {} { K^{= n}_{\phi}(\vec{xs},\vec{ys}) := \Pi_{i=0}^{n-1} Hu . \meaningof{C}_{\phi}(xs_i,ys_i,u) | \neg (\forall x_0,y_0,x_1,y_1,u . \meaningof{C}_{\phi}(x_0,y_0,x_1,y_1,u) | T) }
\end{mathpar}

To round out this section, recall that the encoding of an $n$-crossing
knot decomposes into a parallel composition of $n$ \emph{copies} of a
crossing process together with a wiring harness. To specify different
knot classes with the same crossing number amounts to specifying
logical constraints on the wiring harness. In the interest of space,
we defer examples to a forthcoming paper. Suffice it to say that both
the conditions ``alternating knot'' and ``contains the tangle
corresponding to 5/3'' are expressible. For example, it is possible to
calculate the characteristic formula of a process corresponding to the
tangle 5/3 and conjoin it into the classifying formula via the
composition connective of the logic.

Finally, we wish to observe that it is entirely within reason to
contemplate a more domain-specific version of spatial logic tailored
to the shape of processes in the image of the encoding. Such a
domain-specific logic would have a better claim to the title formal
language of knot properties.

% subsection example_formulae_ (end)

% section knots_as_processes (end) 

% section spatial logic via knots (end)

\section{Conclusions and future work}

\paragraph{Testing physical space}
You, gentle reader, may wonder why of all the theorems to be proved
given this set up we pick the one above. In some sense it's hardly
central to quantum mechanics. We see it as central in the sense that
it firmly establishes a notion of physical space arising from a notion
of the equivalence of behavior. Relating bisimulation to a metric is a
big step forward, but one is faced with interpreting the relationship
of that metric space to something more physical. Quantum mechanical
notions of ``physical'' space are still far from intuitive, but by
relating this idea of distance as testing to calculations that predict
physical circumstances we are making a not insignificant step forward
toward an understanding of the physical space we inhabit as
essentially dynamic.

\paragraph{Effectivity and simulation}
One of the observations we have yet to make is that the entire program
spelled out here is effective. We have built various interpreters for
the reflective calculus at work in this interpretation. In principle,
then, we can simulate quantum mechanics on a computer. The place where
the simulation may lose fidelity is the infinitely branching summation
for the annihilator.

In this connection i also want to point out that the evaluation style
calculation of the inner product puts the non-determinism of the
summation right at the heart of measurement. This suggests that
Milner's original reduction-based formulation of the dynamics of his
calculi in terms of sums was not just notationally suggestive of a
notion of measure-and-continue but captured some significant part of
the physics.

\paragraph{Quantum continuations}
In light of this last observation i want to point out that the
predominant account of quantum mechanics is missing a key aspect of a
truly compositional story of the physical situation. In a real lab,
when a measurement is made the observation can be made to feed into
another device that then makes another measurement conditioned on the
results of the first. This means that after the superposition was
collapsed the entire experimental set up remained in
superposition. While QM offers a means of writing this down it doesn't
quite line up well with the well-trodden formulation of computation
and continuation that we see so succinctly expressed in Milner's
calculi. This suggests that there might be advantages to this account
of dynamics waiting to be explored.

\paragraph{Quantum logic}
In this connection, we also note that by virtue of having the
Hennessy-Milner construction, we can pull the construction through the
interpretation of QM. This gives us a natural candidate for a quantum
logic that enjoys an extremely tight connection with it's domain of
interpretation, making the construction much less ad hoc (rather it is
the image of functor!).

\paragraph{Quantum probabiity}
i have questions about the basis of the interpretation of inner
product as probability amplitude. In particular, using which
axiomatization of probability theory does the notion of probability
amplitude earn the right to be so dubbed? In other words, where is the
proof that the operation for calculating a probability amplitude (and
then squaring) satisfies the axioms of what it means to calculate a
probability? Even if such a proof exists (i have yet to find it in the
literature), i wonder if it might not be possible to turn things on
their heads. Can we view the calculation of the probability amplitude
as an axiomatization of probability? If so, then the definition we
give for calculating probability amplitude may provide the basis for
an \emph{effective} theory of probability.

\paragraph{Quantum vs ``biological'' information}
Finally, i want to conclude with a more philosophical observation. At
a recent workshop in which QM was a predominant topic i noticed
something about quantum information. The speaker was giving a riveting
discussion of axiomatic QM and showing how properties of ``no
cloning'' and ``no deleting'' emerged as consequences of the
axiomatization. Theorems of this form are necessary to give us a sense
of confidence that our axioms characterize the physical theory. What
struck me, though, was that if quantum information is neither erasable
nor replicable it is markedly different from \emph{life}. Two of the
things we know about life is that

\begin{itemize}
  \item it ends;
  \item to gain some measure of persistence, to transcend it's
    finitude it is imminently copyable.
\end{itemize}

Both of these qualities are summarized succinctly in the aphorism: all
flesh is grass. For me these two kinds of ``information'' -- call them
quantum and biological -- are end points on a spectrum of strategies
for persistence. At one end, we have those curious entities that enjoy
uniqueness and permanence; at the other, we have those who in the face
of a certain end and an uncertain present make a go of passing
something on. To me one of the more remarkable aspects of the latter
strategy is that in the presence of noise (and certain features of
copying) we get a kind of dynamism, a chance for improvement against a
given persistent condition.

% subsection other_calculi_other_bisimulations_and_geometry_as_behavior (end)




% section conclusion (end)

%\documentclass[12pt]{llncs}
%\documentclass{jktr}

\usepackage[pdftex]{hyperref}                   
\usepackage {listings}
\usepackage {mathpartir}
\usepackage{bcprules}
%\usepackage{listings}
                       
\usepackage{graphicx} 
%\usepackage[margins=2.5cm,nohead,nofoot]{geometry}
%\usepackage{geometry}
\usepackage{amsfonts}
\usepackage{amstext}
\usepackage{latexsym}
\usepackage{amssymb}
\usepackage{color}


%\include{myPreamble}
\include{qm2pi.local} 

%\ifpdf
%\usepackage[pdftex]{graphicx}
%\else
%\usepackage{graphicx}
%\fi

 % \ifpdf
%  \usepackage{pdfsync}
%  \if


%\title{Brief Article}
%\author{David F. Snyder}
%\author{L.G. Meredith}

%\address{Dept. of Math., Texas State University--San Marcos, San Marcos, TX 78666}
       
\pagestyle{empty}


\begin{document}

\lstset{language=[Objective]Caml,frame=shadowbox}

\input{qm2pi.front}

% section front matter (end)

\input{qm2pi.intro} 
 
% section introduction (end)

% \input{qm2pi.knotations} 

% section notation (end)

\input{qm2pi.process.calculi} 

% section concurrent_process_calculi_and_spatial_logics_ (end)
    
%\input{qm2pi.knots2pi} 

%\input{qm2pi.trefoil} 

%\input{qm2pi.mainthm} 

% subsection basic_interpretation (end)

%\input{qm2pi.rho.presentation} 
\subsection{The syntax and semantics of the notation system}\label{sub:the_syntax_and_semantics_of_the_notation_system} % (fold)

We now summarize a technical presentation of the calculus that
embodies our theory of dynamics. The typical presentation of such a
calculus follows the style of giving generators and relations on
them. The grammar, below, describing term constructors, freely
generates the set of processes, $\Proc$. This set is then quotiented
by a relation known as structural congruence and it is over this set
that the notion of dynamics is expressed. This presentation is
essentially that of \cite{MeredithR05} with the addition of
polyadicity and summation. For readability we have relegated some of
the technical subtleties to an appendix.

\subsubsection{Process grammar}\label{subsub:process_grammar}

\begin{mathpar}
  \inferrule* [lab=synchronization] {} {{M} \bc \pzero \;|\; x?F \;|\; x!C }
  \and
  \inferrule* [lab=abstraction] {} {{F} \bc (x)P}
  \and
  \inferrule* [lab=concretion] {} {{C} \bc \langle Q \rangle}
  \and
  \inferrule* [lab=process] {} {{P,Q} \bc M \;| \;P|Q \;|\; @{x}}
  \and
  \inferrule* [lab=name] {} {{x} \bc \quotep{P}}
\end{mathpar} 

Note that $\vec{x}$ (resp. $\vec{P}$) denotes a vector of names
(resp. processes) of length $|\vec{x}|$ (resp. $|\vec{P}|$). We adopt
the following useful abbreviations.

\begin{mathpar}
   x?(\vec{y}).P := x.(\vec{y})P \and  x\clift{\vec{P}} := x.\clift{\vec{P}}
   \and x!(y) := \lift{x}{\dropn{y}}
   \and \Pi_{i=0}^{n-1}P_i := P_0 | \ldots | P_{n-1}
\end{mathpar}

\subsubsection{Structural congruence}

\paragraph{Free and bound names and alpha-equivalence.} At the
core of structural equivalence is alpha-equivalence which identifies
process that are the same up to a change of variable. Formally, we
recognize the distinction between free and bound names. The free names
of a process, $\freenames{P}$, may be calculated recursively as
follows:

\begin{mathpar}
\freenames{\pzero} := \emptyset
  \and \\
  \freenames{x?(y).P} := \{ x \} \cup (\freenames{P} \setminus \{ y \})
  \and 
  \freenames{x!\langle P \rangle} := \{ x \} \cup \{ P \} 
  \and \\
  \freenames{P|Q} := \freenames{P} \cup \freenames{Q}
  \and \\
  \freenames{@{x}} := \{ x \}
\end{mathpar}

$\pi$
$\quotep{\pi}$

$\freenames{-} : \pi \to \mathcal{P}(\quotep{\pi})$

\begin{eqnarray*}
  \freenames{\pzero} & := & \emptyset \\
  \freenames{x?(y).P} & := & \{ x \} \cup (\freenames{P} \setminus \{ y \}) \\
  \freenames{x!\langle P \rangle} & := & \{ x \} \cup \{ P \} \\
  \freenames{P|Q} & := & \freenames{P} \cup \freenames{Q} \\
  \freenames{\dropn{x}} & := & \{ x \}
\end{eqnarray*}

The bound names of a process, $\boundnames{P}$, are those names occurring in $P$
that are not free. For example, in $x?(y).0$, the name $x$ is free, while $y$ is bound.

\begin{mathpar}
  \inferrule* [lab=monoidal-laws] {} { P|Q \equiv Q|P \and P|0 \equiv P \and P|(Q|R) \equiv (P|Q)|R }
\end{mathpar}

\begin{mathpar}
  \inferrule* [lab=alpha-equivalence] {} { (x)P \equiv (y)P\{y/x\} \and y \not\in \freenames{P} }
\end{mathpar}

\begin{definition}
Then two processes, $P,Q$, are alpha-equivalent if $P = Q\{\vec{y}/\vec{x}\}$ for
some $\vec{x} \in \boundnames{Q},\vec{y} \in \boundnames{P}$, where $Q\{\vec{y}/\vec{x}\}$
denotes the capture-avoiding substitution of $\vec{y}$ for $\vec{x}$ in $Q$.
\end{definition}

\begin{definition}
  The {\em structural congruence} \cite{SangiorgiWalker} , $\equiv$,
  between processes is the least congruence containing
  alpha-equivalence, satisfying the abelian monoid laws
  (associativity, commutativity and $\pzero$ as identity) for parallel
  composition $|$ and for summation $+$.
\end{definition}

\subsection{Name equivalence}

We take name equivalence, written $\nameeq$, to be the smallest
equivalence relation generated by the following rules.

\begin{mathpar}
\inferrule*[lab=Quote-drop]
{ }
{ \quotep{@{x}} \nameeq x }

\inferrule*[lab=Struct-equiv]
{ P \scong Q }
{ \quotep{P} \nameeq \quotep{Q} }
\end{mathpar}

The astute reader will have noticed that the mutual recursion of names
and processes imposes a mutual recursion on alpha-equivalence and
structural equivalence via name-equivalence. Fortunately, all of this
works out pleasantly and we may calculate in the natural way, free of
concern. The reader interested in the details is referred to the
appendix \ref{appendix:rho_details}.

\subsection{Substitution}

We use $\Proc$ for the set of processes, $\QProc$ for the set of
names, and $\id{\{}\vec{y} / \vec{x} \id{\}}$ to denote partial maps,
$s : \QProc \rightarrow \QProc$. A map, $s$ lifts, uniquely, to a map
on process terms, $\widehat{s} : \Proc \rightarrow \Proc$ by the
following equations.

\begin{mathpar}
  (0) \psubstp{Q}{P} := 0 \\
  (R \juxtap S) \psubstp{Q}{P}
  :=    
  (R)\psubstp{Q}{P} \juxtap (S) \psubstp{Q}{P} \\
  (x?(y).R) \psubstp{Q}{P}    
  :=    
  (x)\substp{Q}{P} (z)\concat( (R \psubstn{z}{y}) \psubstp{Q}{P} ) \\
  (\lift{x}{R}) \psubstp{Q}{P}  
  :=
  \lift{(x)\substp{Q}{P}}{ R \psubstp{Q}{P} } \\
%   (\dropn{x})  \psubstp{Q}{P}       
%   := 
%   \left\{ 
%     \begin{array}{ccc} 
%       \dropn{\quotep{Q}} & & x \nameeq \quotep{P} \\
%       \dropn{x} & & otherwise \\
%     \end{array}
%   \right. 
  (\dropn{x})  \psubstp{Q}{P}       
  := 
  \left\{ 
    \begin{array}{ccc} 
      Q & & x \nameeq \quotep{P} \\
      \dropn{x} & & otherwise \\
    \end{array}
  \right.
\end{mathpar}
 

where

\begin{eqnarray}
  (x)\id{\{} \lpquote Q \rpquote / \lpquote P \rpquote \id{\}}            = 
  \left\{ 
    \begin{array}{ccc}
      \lpquote Q \rpquote & & x \nameeq \lpquote P \rpquote \\
      x & & otherwise \\
    \end{array}
  \right. \nonumber
\end{eqnarray}

and $z$ is chosen distinct from $\quotep{P}$, $\quotep{Q}$, the free
names in $Q$, and all the names in $R$. Our $\alpha$-equivalence will
be built in the standard way from this substitution.

\begin{remark}\label{rem:no_self_referential_names}
  One consequence of these definitions is that $\forall P. \quotep{P}
  \not\in \freenames{P}$.
\end{remark}

\subsection{ Dynamic quote: an example }

Anticipating something of what's to come, consider applying the
substitution, $\widehat{\id{\{}u / z \id{\}}}$, to the following pair
of processes, $\lift{w}{y!(z)}$ and $w[ \lpquote y!(z) \rpquote ]$.

\begin{eqnarray}
	\lift{w}{y!(z)}\widehat{\id{\{}u / z \id{\}}}
		& = &
		\lift{w}{y!(u)} \nonumber\\
	w[ \lpquote y!(z) \rpquote ] \widehat{ \id{\{}u / z \id{\}} }
		& = &
		w[ \lpquote y!(z) \rpquote ] \nonumber
\end{eqnarray}

Because the body of the process between quotes is impervious to
substitution, we get radically different answers. In fact, by
examining the first process in an input context,
e.g. $x?(z).\lift{w}{y!(z)}$, we see that the process under the lift
operator may be shaped by prefixed inputs binding a name inside it. In
this sense, the lift operator will be seen as a way to dynamically
construct processes before reifying them as names.

Finally equipped with these standard features we can present the
dynamics of the calculus.

\subsubsection{Operational semantics} 

Finally, we introduce the computational dynamics. What marks these
algebras as distinct from other more traditionally studied algebraic
structures, e.g. vector spaces or polynomial rings, is the manner in
which dynamics is captured. In traditional structures, dynamics is typically
expressed through morphisms between such structures, as in linear maps
between vector spaces or morphisms between rings. In algebras
associated with the semantics of computation, the dynamics is
expressed as part of the algebraic structure itself, through a
reduction reduction relation typically denoted by $\red$. Below, we
give a recursive presentation of this relation for the calculus used
in the encoding.

$\red \subseteq \pi \times \pi$
$\red : \pi \to \mathcal{P}(\pi)$

\begin{mathpar}
  \inferrule* [lab=Comm] { \textsf{match}( x_{src}, x_{trgt} ) } { x_{trgt}?(y)P \; | \; x_{src}!\langle {Q} \rangle \red P\{\quotep{Q}/y}\} }
  \and \\
  \inferrule* [lab=Par] {{P} \red {P}'} {{{P} | {Q}} \red {{P}' | {Q}}}
  \and
  \inferrule* [lab=Equiv]{{{P} \scong {P}'} \andalso {{P}' \red {Q}'} \andalso {{Q}' \scong {Q}}}{{P} \red {Q}}
\end{mathpar}

\begin{eqnarray*}
  match_{\equiv} (\quotep{P},\quotep{Q}) & := & P \equiv Q \\
  match_{\dagger}(\quotep{P},\quotep{Q}) & := & \forall R. P|Q \red^{*} R => R \red^{*} 0 \\
  match_{K}(\quotep{P},\quotep{Q}) & := & K \mbox{ for some context } K
\end{eqnarray*}

$u?(x)P | u!\langle Q \rangle \red P\{\quotep{Q}/x\}$

%We write $\wred$ for $\red^*$, and $P\red$ if $\exists Q $ such that $ P \red Q$.
We write $P\red$ if $\exists Q $ such that $ P \red Q$ and $P\not\red$, otherwise.

\section{Replication}

As mentioned before, it is known that replication (and hence
recursion) can be implemented in a higher-order process algebra
\cite{SangiorgiWalker}. As our first example of calculation with the
machinery thus far presented we give the construction explicitly in
the {\rhoc}.

\begin{eqnarray}
	D_{x} & := & \prefix{x}{y}{(\binpar{\outputp{x}{y}}{@{y}})} \nonumber\\
	\bangp_{x}{P} & := & \binpar{{x}!\langle{\binpar{D_{x}}{P}}\rangle}{D_{x}} \nonumber
\end{eqnarray}

\begin{eqnarray}
	\bangp_{x}{P} & & \nonumber\\
	=
	& {x}!\langle{(\prefix{x}{y}{(\outputp{x}{y} | @{y})) | P}}\rangle 
	      | \prefix{x}{y}{(\outputp{x}{y} | @{y})} & \nonumber\\
	\red
	& (\outputp{x}{y} | @{y})\substn{\quotep{(\prefix{x}{y}{(@{y} | \outputp{x}{y})) | P}}}{y} & \nonumber\\
	=
	& \outputp{x}{\quotep{(\prefix{x}{y}{(\outputp{x}{y} | @{y})) | P}}}
	  | {(\prefix{x}{y}{(\outputp{x}{y} | @{y})) | P}} & \nonumber\\
	\red
	& \ldots & \nonumber\\
	\red^*
	& P | P | \ldots & \nonumber
\end{eqnarray}

Of course, this encoding, as an implementation, runs away, unfolding
$\bangp{P}$ eagerly. A lazier and more implementable replication
operator, restricted to input-guarded processes, may be obtained as follows.

\begin{eqnarray}
\bangp{\prefix{u}{v}{P}} 
	:= 
	\binpar{\lift{x}{\prefix{u}{v}{(\binpar{D(x)}{P})}}}{D(x)} \nonumber
\end{eqnarray}

\begin{remark}
  Note that the lazier definition still does not deal with summation
  or mixed summation (i.e. sums over input and output). The reader is
  invited to construct definitions of replication that deal with these
  features. 

  Further, the definitions are parameterized in a name, $x$. Can you,
  gentle reader, make a definition that eliminates this parameter and
  guarantees no accidental interaction between the replication
  machinery and the process being replicated -- i.e. no accidental
  sharing of names used by the process to get its work done and the
  name(s) used by the replication to effect copying. This latter
  revision of the definition of replication is crucial to obtaining
  the expected identity $!!P \sim !P$.
\end{remark}

\begin{remark}\label{rem:paradoxical_combinator}
  The reader familiar with the lambda calculus will have noticed the
  similarity between $D$ and the paradoxical combinator.

  [Ed. note: the existence of this seems to suggest we have to be more
  restrictive on the set of processes and names we admit if we are to
  support no-cloning.]
\end{remark}

\subsubsection{Bisimulation}

The computational dynamics gives rise to another kind of equivalence,
the equivalence of computational behavior. As previously mentioned
this is typically captured \emph{via} some form of bisimulation.

% The notion we use in this paper is weak barbed bisimulation
% \cite{milner91polyadicpi}.

The notion we use in this paper is derived from weak barbed
bisimulation \cite{milner91polyadicpi}. 

\begin{definition}
An \emph{observation relation}, $\downarrow_{\mathcal N}$, over a set
of names, $\mathcal N$, is the smallest relation satisfying the rules
below.

\infrule[Out-barb]{y \in {\mathcal N}, \; x \nameeq y}
		  {\outputp{x}{v} \downarrow_{\mathcal N} x}
\infrule[Par-barb]{\mbox{$P\downarrow_{\mathcal N} x$ or $Q\downarrow_{\mathcal N} x$}}
		  {\binpar{P}{Q} \downarrow_{\mathcal N} x}

We write $P \Downarrow_{\mathcal N} x$ if there is $Q$ such that 
$P \wred Q$ and $Q \downarrow_{\mathcal N} x$.
\end{definition}

\begin{definition}
%\label{def.bbisim}
An  ${\mathcal N}$-\emph{barbed bisimulation} over a set of names, ${\mathcal N}$, is a symmetric binary relation 
${\mathcal S}_{\mathcal N}$ between agents such that $P\rel{S}_{\mathcal N}Q$ implies:
\begin{enumerate}
\item If $P \red P'$ then $Q \wred Q'$ and $P'\rel{S}_{\mathcal N} Q'$.
\item If $P\downarrow_{\mathcal N} x$, then $Q\Downarrow_{\mathcal N} x$.
\end{enumerate}
$P$ is ${\mathcal N}$-barbed bisimilar to $Q$, written
$P \wbbisim_{\mathcal N} Q$, if $P \rel{S}_{\mathcal N} Q$ for some ${\mathcal N}$-barbed bisimulation ${\mathcal S}_{\mathcal N}$.
\end{definition}

$\mathcal{R} \subseteq \pi \times \pi$

$P \mathcal{R} Q => \forall P'. P \red P' \Rightarrow \exists Q'. Q \red Q', P' \mathcal{R} Q'$

$P \vdash x \Rightarrow Q \vdash x$

\begin{mathpar}
  \inferrule*[lab=Out-barb]{x \nameeq y}{{y}!\langle{Q}\rangle \vdash x}
  \and
  \inferrule*[lab=Par-barb]{\mbox{$P\vdash x$ or $Q\vdash x$}}{\binpar{P}{Q} \vdash x}
\end{mathpar}

\subsubsection{Contexts}

One of the principle advantages of computational calculi like the
$\pi$-calculus is a well-defined notion of context,
contextual-equivalence and a correlation between
contextual-equivalence and notions of bisimulation. The notion of
context allows the decomposition of a process into (sub-)process and
its syntactic environment, its context. Thus, a context may be
thought of as a process with a ``hole'' (written $\Box$) in it. The
application of a context $M$ to a process $P$, written $M[P]$, is
tantamount to filling the hole in $M$ with $P$. In this paper we do
not need the full weight of this theory, but do make use of the notion
of context in the proof the main theorem. 

\begin{mathpar}
  \inferrule* [lab=summation] {} {{M_{M},M_{N}} \bc \Box \;|\; x.M_{A} \;|\; M_{M}+M_{N}}
  \and
  \inferrule* [lab=agent] {} {{M_{A}} \bc (\vec{x})M_{P} \;| \; \clift{P_0,\ldots,M_{P},\ldots,P_N}}
  \and \\
  \inferrule* [lab=process] {} {{M_{P}} \bc M_{N} \;| \;P|M_{P} }
\end{mathpar} 

\begin{mathpar}
  \inferrule* [lab=sychronization] {} {M_{N} \bc \Box \;|\; x?M_{F} \;|\; x!M_{C}}
  \and
  \inferrule* [lab=abstraction] {} {{M_{F}} \bc (x)M_{P} }
  \and
  \inferrule* [lab=concretion] {} {{M_{C}} \bc \langle M_{P} \rangle }
  \and \\
  \inferrule* [lab=process] {} {{M_{P}} \bc M_{N} \;| \;P|M_{P} }
\end{mathpar}

\begin{definition}[contextual application] Given a context $M$, and
  process $P$, we define the \emph{contextual application}, $M[P] :=
  M\{P/\Box\}$. That is, the contextual application of M to P is the
  substitution of $P$ for $\Box$ in $M$.
\end{definition}

$\meaningof{-} : L \to \mathcal{P}(\pi)$

\begin{mathpar}
  \inferrule* [lab=collection] {} {\meaningof{true} = \pi, \and \meaningof{~E} = \pi \setminus \meaningof{E}, \and \meaningof{E_{1} \& E_{2}} = \meaningof{E_{1}} \cap \meaningof{E_{2}}}
\end{mathpar}

\begin{mathpar}
  \inferrule* [lab=structure] {} {\meaningof{0} = \{ P \in \pi | P \equiv 0 \}, \and \\ \meaningof{E_1 | E_2} = \{ P \in \pi | P \equiv P_{1} | P_{2}, P_{1} \in \meaningof{E_{1}}, P_{2} \in \meaningof{E_2}\} }
\end{mathpar}

\begin{mathpar}
 \inferrule* [lab=behavior] {} {\meaningof{\langle a?b \rangle E} = \{ P \in \pi | P \equiv Q | u?(y)P', \\ \and \\\\ \and \\ \;\;\; u \in \meaningof{a}, \forall z.P'\{z/y\} \in \meaningof{E\{z/b\}}\}, \and \\ \meaningof{a!E} = \{ P \in \pi | P \equiv Q | x!\langle P' \rangle, x \in \meaningof{a} P' \in \meaningof{E}\} }
\end{mathpar}

\begin{mathpar}
 \inferrule* [lab=nominal] {} {\meaningof{\quotep{E}} = \{ \quotep{P} \in \quotep{\pi} | P \in \meaningof{E} \}, \and \meaningof{\quotep{P}} = \{ \quotep{Q} \in \quotep{\pi} | P \equiv Q \} \and \\ \meaningof{@\quotep{E}} = \{ P \in \pi | P \equiv @x, x \in \meaningof{E} \}}
\end{mathpar}

\begin{eqnarray*}
  \\
  \meaningof{-} : TS \to ST
\end{eqnarray*}

\begin{eqnarray*}
  \\
  L : TS \to ST
\end{eqnarray*}

\begin{eqnarray*}
  \\
  P \models E \iff P \in \meaningof{E}
\end{eqnarray*}

\begin{eqnarray*}
  P \approx_{L} Q \iff \forall E \in L. P \models E \iff Q \models E
\end{eqnarray*}

\begin{eqnarray*}
  P \approx_{K} Q
\end{eqnarray*}

\begin{eqnarray*}
  P \approx Q
\end{eqnarray*}

$\approx_{K} = \approx = \approx_{L}$

\subsubsection{Contextual duality}

Note that contexts extend the quotation operation to a family of
operations from processes to names. Given a context, $M$, we can
define a \emph{nominal context}, $\quotep{M}$ by $\quotep{M}[P] :=
\quotep{M[P]}$. To foreshadow what is to come we observe that these
operations enjoy a duality with processes very much like the duality
between vectors and maps from vectors to scalars.

Further, because the calculus is essentially higher-order, we have a
correspondence between contexts and processes. More specifically,
given a name $x$ and a context $M$ we can construct $M^{*}_{x}$ such
that 

\begin{mathpar}
  M^{*}_{x} | \lift{x}{P} \red M[P]
\end{mathpar}

namely,

\begin{mathpar}
  M^{*}_{x} := x?(u).M[\dropn{u}]
\end{mathpar}

The dependence of $M^{*}_{x}$ on a name makes it an abstraction, 

\begin{mathpar}
  M^{*} := (x)x?(u).M[\dropn{u}]
\end{mathpar}

\subsection{Additional notation}

It will sometimes be convenient to denote the process a name
quotes. We already have the notation $x = \quotep{P}$, but it will be
convenient to introduce an alternate notation, $\procn{x}$, when we
want to emphasize the connection to the use of the name. Note that, by
virtue of name equivalence, $\quotep{\procn{x}} \nameeq x$; so, the
notation is consistent with previous definitions.

Further, because names have structure it is possible to effect
substitutions on the basis of that structure. This means we need to
upgrade our notation for substitutions, which we accomplish by
adapting comprehension notation. Thus,

\begin{mathpar}
  P\{ y / x : x \in S \}
\end{mathpar}

is interpreted to mean the process derived from P by replacing (in a
capture-avoiding manner) each occurrence of $x$ in $S$ by $y$. For example,

\begin{mathpar}
  P\{ \quotep{\procn{x}|\procn{x}} / x : x \in \freenames{P} \}
\end{mathpar}

will replace each (occurrence) of a free name $x$ in $P$ by
$\quotep{\procn{x}|\procn{x}}$.

Also, we will avail ourselves of the notation $x^{L}$ and $x^{R}$ to
denote injections of a name into disjoint copies of the name
space. There are numerous ways to accomplish this. One example can be
found in \cite{MeredithR05}. This notation overloads to vectors of
names: $\vec{x}^{\pi} := (x_{i}^{\pi} \; : \; 0 \leq i < |\vec{x}| )$ where $\pi \in \{L,R\}$.

We also use $P^{\Box} := P|\Box$.

In \cite{MeredithR05} an interpretation of the new operator is
given. It turns out that there are several possible interpretations
all enjoying the requisite algebraic properties of the operator (see
\cite{milner91polyadicpi}). We will therefore make liberal use of
$(\nu\; \vec{x})P$.

% subsection the_syntax_and_semantics_of_the_notation_system (end)   

\input{qm2pi.qmops} 

\input{qm2pi.sterngerlach} 

\input{qm2pi.metric} 

% section concurrent_process_calculi (end)

%\input{qm2pi.proofsketch}

% section proof sketch (end)

%\input{qm2pi.slviaknots} 

% section spatial logic via knots (end)

\input{qm2pi.conclusion}

% section conclusion (end)

%\input{qm2pi.dtcodes} 

% section wiring algorithm (end)

\input{qm2pi.ack} 

% section acknowledgments (end)

\newpage


\bibliographystyle{plain}   
\bibliography{../../biblios/main.bib}

\input{qm2pi.rhodetails}

\end{document}

 

% section wiring algorithm (end)

\documentclass[12pt]{llncs}
%\documentclass{jktr}

\usepackage[pdftex]{hyperref}                   
\usepackage {listings}
\usepackage {mathpartir}
\usepackage{bcprules}
%\usepackage{listings}
                       
\usepackage{graphicx} 
%\usepackage[margins=2.5cm,nohead,nofoot]{geometry}
%\usepackage{geometry}
\usepackage{amsfonts}
\usepackage{amstext}
\usepackage{latexsym}
\usepackage{amssymb}
\usepackage{color}


%\include{myPreamble}
\include{qm2pi.local} 

%\ifpdf
%\usepackage[pdftex]{graphicx}
%\else
%\usepackage{graphicx}
%\fi

 % \ifpdf
%  \usepackage{pdfsync}
%  \if


%\title{Brief Article}
%\author{David F. Snyder}
%\author{L.G. Meredith}

%\address{Dept. of Math., Texas State University--San Marcos, San Marcos, TX 78666}
       
\pagestyle{empty}


\begin{document}

\lstset{language=[Objective]Caml,frame=shadowbox}

\input{qm2pi.front}

% section front matter (end)

\input{qm2pi.intro} 
 
% section introduction (end)

% \input{qm2pi.knotations} 

% section notation (end)

\input{qm2pi.process.calculi} 

% section concurrent_process_calculi_and_spatial_logics_ (end)
    
%\input{qm2pi.knots2pi} 

%\input{qm2pi.trefoil} 

%\input{qm2pi.mainthm} 

% subsection basic_interpretation (end)

%\input{qm2pi.rho.presentation} 
\subsection{The syntax and semantics of the notation system}\label{sub:the_syntax_and_semantics_of_the_notation_system} % (fold)

We now summarize a technical presentation of the calculus that
embodies our theory of dynamics. The typical presentation of such a
calculus follows the style of giving generators and relations on
them. The grammar, below, describing term constructors, freely
generates the set of processes, $\Proc$. This set is then quotiented
by a relation known as structural congruence and it is over this set
that the notion of dynamics is expressed. This presentation is
essentially that of \cite{MeredithR05} with the addition of
polyadicity and summation. For readability we have relegated some of
the technical subtleties to an appendix.

\subsubsection{Process grammar}\label{subsub:process_grammar}

\begin{mathpar}
  \inferrule* [lab=synchronization] {} {{M} \bc \pzero \;|\; x?F \;|\; x!C }
  \and
  \inferrule* [lab=abstraction] {} {{F} \bc (x)P}
  \and
  \inferrule* [lab=concretion] {} {{C} \bc \langle Q \rangle}
  \and
  \inferrule* [lab=process] {} {{P,Q} \bc M \;| \;P|Q \;|\; @{x}}
  \and
  \inferrule* [lab=name] {} {{x} \bc \quotep{P}}
\end{mathpar} 

Note that $\vec{x}$ (resp. $\vec{P}$) denotes a vector of names
(resp. processes) of length $|\vec{x}|$ (resp. $|\vec{P}|$). We adopt
the following useful abbreviations.

\begin{mathpar}
   x?(\vec{y}).P := x.(\vec{y})P \and  x\clift{\vec{P}} := x.\clift{\vec{P}}
   \and x!(y) := \lift{x}{\dropn{y}}
   \and \Pi_{i=0}^{n-1}P_i := P_0 | \ldots | P_{n-1}
\end{mathpar}

\subsubsection{Structural congruence}

\paragraph{Free and bound names and alpha-equivalence.} At the
core of structural equivalence is alpha-equivalence which identifies
process that are the same up to a change of variable. Formally, we
recognize the distinction between free and bound names. The free names
of a process, $\freenames{P}$, may be calculated recursively as
follows:

\begin{mathpar}
\freenames{\pzero} := \emptyset
  \and \\
  \freenames{x?(y).P} := \{ x \} \cup (\freenames{P} \setminus \{ y \})
  \and 
  \freenames{x!\langle P \rangle} := \{ x \} \cup \{ P \} 
  \and \\
  \freenames{P|Q} := \freenames{P} \cup \freenames{Q}
  \and \\
  \freenames{@{x}} := \{ x \}
\end{mathpar}

$\pi$
$\quotep{\pi}$

$\freenames{-} : \pi \to \mathcal{P}(\quotep{\pi})$

\begin{eqnarray*}
  \freenames{\pzero} & := & \emptyset \\
  \freenames{x?(y).P} & := & \{ x \} \cup (\freenames{P} \setminus \{ y \}) \\
  \freenames{x!\langle P \rangle} & := & \{ x \} \cup \{ P \} \\
  \freenames{P|Q} & := & \freenames{P} \cup \freenames{Q} \\
  \freenames{\dropn{x}} & := & \{ x \}
\end{eqnarray*}

The bound names of a process, $\boundnames{P}$, are those names occurring in $P$
that are not free. For example, in $x?(y).0$, the name $x$ is free, while $y$ is bound.

\begin{mathpar}
  \inferrule* [lab=monoidal-laws] {} { P|Q \equiv Q|P \and P|0 \equiv P \and P|(Q|R) \equiv (P|Q)|R }
\end{mathpar}

\begin{mathpar}
  \inferrule* [lab=alpha-equivalence] {} { (x)P \equiv (y)P\{y/x\} \and y \not\in \freenames{P} }
\end{mathpar}

\begin{definition}
Then two processes, $P,Q$, are alpha-equivalent if $P = Q\{\vec{y}/\vec{x}\}$ for
some $\vec{x} \in \boundnames{Q},\vec{y} \in \boundnames{P}$, where $Q\{\vec{y}/\vec{x}\}$
denotes the capture-avoiding substitution of $\vec{y}$ for $\vec{x}$ in $Q$.
\end{definition}

\begin{definition}
  The {\em structural congruence} \cite{SangiorgiWalker} , $\equiv$,
  between processes is the least congruence containing
  alpha-equivalence, satisfying the abelian monoid laws
  (associativity, commutativity and $\pzero$ as identity) for parallel
  composition $|$ and for summation $+$.
\end{definition}

\subsection{Name equivalence}

We take name equivalence, written $\nameeq$, to be the smallest
equivalence relation generated by the following rules.

\begin{mathpar}
\inferrule*[lab=Quote-drop]
{ }
{ \quotep{@{x}} \nameeq x }

\inferrule*[lab=Struct-equiv]
{ P \scong Q }
{ \quotep{P} \nameeq \quotep{Q} }
\end{mathpar}

The astute reader will have noticed that the mutual recursion of names
and processes imposes a mutual recursion on alpha-equivalence and
structural equivalence via name-equivalence. Fortunately, all of this
works out pleasantly and we may calculate in the natural way, free of
concern. The reader interested in the details is referred to the
appendix \ref{appendix:rho_details}.

\subsection{Substitution}

We use $\Proc$ for the set of processes, $\QProc$ for the set of
names, and $\id{\{}\vec{y} / \vec{x} \id{\}}$ to denote partial maps,
$s : \QProc \rightarrow \QProc$. A map, $s$ lifts, uniquely, to a map
on process terms, $\widehat{s} : \Proc \rightarrow \Proc$ by the
following equations.

\begin{mathpar}
  (0) \psubstp{Q}{P} := 0 \\
  (R \juxtap S) \psubstp{Q}{P}
  :=    
  (R)\psubstp{Q}{P} \juxtap (S) \psubstp{Q}{P} \\
  (x?(y).R) \psubstp{Q}{P}    
  :=    
  (x)\substp{Q}{P} (z)\concat( (R \psubstn{z}{y}) \psubstp{Q}{P} ) \\
  (\lift{x}{R}) \psubstp{Q}{P}  
  :=
  \lift{(x)\substp{Q}{P}}{ R \psubstp{Q}{P} } \\
%   (\dropn{x})  \psubstp{Q}{P}       
%   := 
%   \left\{ 
%     \begin{array}{ccc} 
%       \dropn{\quotep{Q}} & & x \nameeq \quotep{P} \\
%       \dropn{x} & & otherwise \\
%     \end{array}
%   \right. 
  (\dropn{x})  \psubstp{Q}{P}       
  := 
  \left\{ 
    \begin{array}{ccc} 
      Q & & x \nameeq \quotep{P} \\
      \dropn{x} & & otherwise \\
    \end{array}
  \right.
\end{mathpar}
 

where

\begin{eqnarray}
  (x)\id{\{} \lpquote Q \rpquote / \lpquote P \rpquote \id{\}}            = 
  \left\{ 
    \begin{array}{ccc}
      \lpquote Q \rpquote & & x \nameeq \lpquote P \rpquote \\
      x & & otherwise \\
    \end{array}
  \right. \nonumber
\end{eqnarray}

and $z$ is chosen distinct from $\quotep{P}$, $\quotep{Q}$, the free
names in $Q$, and all the names in $R$. Our $\alpha$-equivalence will
be built in the standard way from this substitution.

\begin{remark}\label{rem:no_self_referential_names}
  One consequence of these definitions is that $\forall P. \quotep{P}
  \not\in \freenames{P}$.
\end{remark}

\subsection{ Dynamic quote: an example }

Anticipating something of what's to come, consider applying the
substitution, $\widehat{\id{\{}u / z \id{\}}}$, to the following pair
of processes, $\lift{w}{y!(z)}$ and $w[ \lpquote y!(z) \rpquote ]$.

\begin{eqnarray}
	\lift{w}{y!(z)}\widehat{\id{\{}u / z \id{\}}}
		& = &
		\lift{w}{y!(u)} \nonumber\\
	w[ \lpquote y!(z) \rpquote ] \widehat{ \id{\{}u / z \id{\}} }
		& = &
		w[ \lpquote y!(z) \rpquote ] \nonumber
\end{eqnarray}

Because the body of the process between quotes is impervious to
substitution, we get radically different answers. In fact, by
examining the first process in an input context,
e.g. $x?(z).\lift{w}{y!(z)}$, we see that the process under the lift
operator may be shaped by prefixed inputs binding a name inside it. In
this sense, the lift operator will be seen as a way to dynamically
construct processes before reifying them as names.

Finally equipped with these standard features we can present the
dynamics of the calculus.

\subsubsection{Operational semantics} 

Finally, we introduce the computational dynamics. What marks these
algebras as distinct from other more traditionally studied algebraic
structures, e.g. vector spaces or polynomial rings, is the manner in
which dynamics is captured. In traditional structures, dynamics is typically
expressed through morphisms between such structures, as in linear maps
between vector spaces or morphisms between rings. In algebras
associated with the semantics of computation, the dynamics is
expressed as part of the algebraic structure itself, through a
reduction reduction relation typically denoted by $\red$. Below, we
give a recursive presentation of this relation for the calculus used
in the encoding.

$\red \subseteq \pi \times \pi$
$\red : \pi \to \mathcal{P}(\pi)$

\begin{mathpar}
  \inferrule* [lab=Comm] { \textsf{match}( x_{src}, x_{trgt} ) } { x_{trgt}?(y)P \; | \; x_{src}!\langle {Q} \rangle \red P\{\quotep{Q}/y}\} }
  \and \\
  \inferrule* [lab=Par] {{P} \red {P}'} {{{P} | {Q}} \red {{P}' | {Q}}}
  \and
  \inferrule* [lab=Equiv]{{{P} \scong {P}'} \andalso {{P}' \red {Q}'} \andalso {{Q}' \scong {Q}}}{{P} \red {Q}}
\end{mathpar}

\begin{eqnarray*}
  match_{\equiv} (\quotep{P},\quotep{Q}) & := & P \equiv Q \\
  match_{\dagger}(\quotep{P},\quotep{Q}) & := & \forall R. P|Q \red^{*} R => R \red^{*} 0 \\
  match_{K}(\quotep{P},\quotep{Q}) & := & K \mbox{ for some context } K
\end{eqnarray*}

$u?(x)P | u!\langle Q \rangle \red P\{\quotep{Q}/x\}$

%We write $\wred$ for $\red^*$, and $P\red$ if $\exists Q $ such that $ P \red Q$.
We write $P\red$ if $\exists Q $ such that $ P \red Q$ and $P\not\red$, otherwise.

\section{Replication}

As mentioned before, it is known that replication (and hence
recursion) can be implemented in a higher-order process algebra
\cite{SangiorgiWalker}. As our first example of calculation with the
machinery thus far presented we give the construction explicitly in
the {\rhoc}.

\begin{eqnarray}
	D_{x} & := & \prefix{x}{y}{(\binpar{\outputp{x}{y}}{@{y}})} \nonumber\\
	\bangp_{x}{P} & := & \binpar{{x}!\langle{\binpar{D_{x}}{P}}\rangle}{D_{x}} \nonumber
\end{eqnarray}

\begin{eqnarray}
	\bangp_{x}{P} & & \nonumber\\
	=
	& {x}!\langle{(\prefix{x}{y}{(\outputp{x}{y} | @{y})) | P}}\rangle 
	      | \prefix{x}{y}{(\outputp{x}{y} | @{y})} & \nonumber\\
	\red
	& (\outputp{x}{y} | @{y})\substn{\quotep{(\prefix{x}{y}{(@{y} | \outputp{x}{y})) | P}}}{y} & \nonumber\\
	=
	& \outputp{x}{\quotep{(\prefix{x}{y}{(\outputp{x}{y} | @{y})) | P}}}
	  | {(\prefix{x}{y}{(\outputp{x}{y} | @{y})) | P}} & \nonumber\\
	\red
	& \ldots & \nonumber\\
	\red^*
	& P | P | \ldots & \nonumber
\end{eqnarray}

Of course, this encoding, as an implementation, runs away, unfolding
$\bangp{P}$ eagerly. A lazier and more implementable replication
operator, restricted to input-guarded processes, may be obtained as follows.

\begin{eqnarray}
\bangp{\prefix{u}{v}{P}} 
	:= 
	\binpar{\lift{x}{\prefix{u}{v}{(\binpar{D(x)}{P})}}}{D(x)} \nonumber
\end{eqnarray}

\begin{remark}
  Note that the lazier definition still does not deal with summation
  or mixed summation (i.e. sums over input and output). The reader is
  invited to construct definitions of replication that deal with these
  features. 

  Further, the definitions are parameterized in a name, $x$. Can you,
  gentle reader, make a definition that eliminates this parameter and
  guarantees no accidental interaction between the replication
  machinery and the process being replicated -- i.e. no accidental
  sharing of names used by the process to get its work done and the
  name(s) used by the replication to effect copying. This latter
  revision of the definition of replication is crucial to obtaining
  the expected identity $!!P \sim !P$.
\end{remark}

\begin{remark}\label{rem:paradoxical_combinator}
  The reader familiar with the lambda calculus will have noticed the
  similarity between $D$ and the paradoxical combinator.

  [Ed. note: the existence of this seems to suggest we have to be more
  restrictive on the set of processes and names we admit if we are to
  support no-cloning.]
\end{remark}

\subsubsection{Bisimulation}

The computational dynamics gives rise to another kind of equivalence,
the equivalence of computational behavior. As previously mentioned
this is typically captured \emph{via} some form of bisimulation.

% The notion we use in this paper is weak barbed bisimulation
% \cite{milner91polyadicpi}.

The notion we use in this paper is derived from weak barbed
bisimulation \cite{milner91polyadicpi}. 

\begin{definition}
An \emph{observation relation}, $\downarrow_{\mathcal N}$, over a set
of names, $\mathcal N$, is the smallest relation satisfying the rules
below.

\infrule[Out-barb]{y \in {\mathcal N}, \; x \nameeq y}
		  {\outputp{x}{v} \downarrow_{\mathcal N} x}
\infrule[Par-barb]{\mbox{$P\downarrow_{\mathcal N} x$ or $Q\downarrow_{\mathcal N} x$}}
		  {\binpar{P}{Q} \downarrow_{\mathcal N} x}

We write $P \Downarrow_{\mathcal N} x$ if there is $Q$ such that 
$P \wred Q$ and $Q \downarrow_{\mathcal N} x$.
\end{definition}

\begin{definition}
%\label{def.bbisim}
An  ${\mathcal N}$-\emph{barbed bisimulation} over a set of names, ${\mathcal N}$, is a symmetric binary relation 
${\mathcal S}_{\mathcal N}$ between agents such that $P\rel{S}_{\mathcal N}Q$ implies:
\begin{enumerate}
\item If $P \red P'$ then $Q \wred Q'$ and $P'\rel{S}_{\mathcal N} Q'$.
\item If $P\downarrow_{\mathcal N} x$, then $Q\Downarrow_{\mathcal N} x$.
\end{enumerate}
$P$ is ${\mathcal N}$-barbed bisimilar to $Q$, written
$P \wbbisim_{\mathcal N} Q$, if $P \rel{S}_{\mathcal N} Q$ for some ${\mathcal N}$-barbed bisimulation ${\mathcal S}_{\mathcal N}$.
\end{definition}

$\mathcal{R} \subseteq \pi \times \pi$

$P \mathcal{R} Q => \forall P'. P \red P' \Rightarrow \exists Q'. Q \red Q', P' \mathcal{R} Q'$

$P \vdash x \Rightarrow Q \vdash x$

\begin{mathpar}
  \inferrule*[lab=Out-barb]{x \nameeq y}{{y}!\langle{Q}\rangle \vdash x}
  \and
  \inferrule*[lab=Par-barb]{\mbox{$P\vdash x$ or $Q\vdash x$}}{\binpar{P}{Q} \vdash x}
\end{mathpar}

\subsubsection{Contexts}

One of the principle advantages of computational calculi like the
$\pi$-calculus is a well-defined notion of context,
contextual-equivalence and a correlation between
contextual-equivalence and notions of bisimulation. The notion of
context allows the decomposition of a process into (sub-)process and
its syntactic environment, its context. Thus, a context may be
thought of as a process with a ``hole'' (written $\Box$) in it. The
application of a context $M$ to a process $P$, written $M[P]$, is
tantamount to filling the hole in $M$ with $P$. In this paper we do
not need the full weight of this theory, but do make use of the notion
of context in the proof the main theorem. 

\begin{mathpar}
  \inferrule* [lab=summation] {} {{M_{M},M_{N}} \bc \Box \;|\; x.M_{A} \;|\; M_{M}+M_{N}}
  \and
  \inferrule* [lab=agent] {} {{M_{A}} \bc (\vec{x})M_{P} \;| \; \clift{P_0,\ldots,M_{P},\ldots,P_N}}
  \and \\
  \inferrule* [lab=process] {} {{M_{P}} \bc M_{N} \;| \;P|M_{P} }
\end{mathpar} 

\begin{mathpar}
  \inferrule* [lab=sychronization] {} {M_{N} \bc \Box \;|\; x?M_{F} \;|\; x!M_{C}}
  \and
  \inferrule* [lab=abstraction] {} {{M_{F}} \bc (x)M_{P} }
  \and
  \inferrule* [lab=concretion] {} {{M_{C}} \bc \langle M_{P} \rangle }
  \and \\
  \inferrule* [lab=process] {} {{M_{P}} \bc M_{N} \;| \;P|M_{P} }
\end{mathpar}

\begin{definition}[contextual application] Given a context $M$, and
  process $P$, we define the \emph{contextual application}, $M[P] :=
  M\{P/\Box\}$. That is, the contextual application of M to P is the
  substitution of $P$ for $\Box$ in $M$.
\end{definition}

$\meaningof{-} : L \to \mathcal{P}(\pi)$

\begin{mathpar}
  \inferrule* [lab=collection] {} {\meaningof{true} = \pi, \and \meaningof{~E} = \pi \setminus \meaningof{E}, \and \meaningof{E_{1} \& E_{2}} = \meaningof{E_{1}} \cap \meaningof{E_{2}}}
\end{mathpar}

\begin{mathpar}
  \inferrule* [lab=structure] {} {\meaningof{0} = \{ P \in \pi | P \equiv 0 \}, \and \\ \meaningof{E_1 | E_2} = \{ P \in \pi | P \equiv P_{1} | P_{2}, P_{1} \in \meaningof{E_{1}}, P_{2} \in \meaningof{E_2}\} }
\end{mathpar}

\begin{mathpar}
 \inferrule* [lab=behavior] {} {\meaningof{\langle a?b \rangle E} = \{ P \in \pi | P \equiv Q | u?(y)P', \\ \and \\\\ \and \\ \;\;\; u \in \meaningof{a}, \forall z.P'\{z/y\} \in \meaningof{E\{z/b\}}\}, \and \\ \meaningof{a!E} = \{ P \in \pi | P \equiv Q | x!\langle P' \rangle, x \in \meaningof{a} P' \in \meaningof{E}\} }
\end{mathpar}

\begin{mathpar}
 \inferrule* [lab=nominal] {} {\meaningof{\quotep{E}} = \{ \quotep{P} \in \quotep{\pi} | P \in \meaningof{E} \}, \and \meaningof{\quotep{P}} = \{ \quotep{Q} \in \quotep{\pi} | P \equiv Q \} \and \\ \meaningof{@\quotep{E}} = \{ P \in \pi | P \equiv @x, x \in \meaningof{E} \}}
\end{mathpar}

\begin{eqnarray*}
  \\
  \meaningof{-} : TS \to ST
\end{eqnarray*}

\begin{eqnarray*}
  \\
  L : TS \to ST
\end{eqnarray*}

\begin{eqnarray*}
  \\
  P \models E \iff P \in \meaningof{E}
\end{eqnarray*}

\begin{eqnarray*}
  P \approx_{L} Q \iff \forall E \in L. P \models E \iff Q \models E
\end{eqnarray*}

\begin{eqnarray*}
  P \approx_{K} Q
\end{eqnarray*}

\begin{eqnarray*}
  P \approx Q
\end{eqnarray*}

$\approx_{K} = \approx = \approx_{L}$

\subsubsection{Contextual duality}

Note that contexts extend the quotation operation to a family of
operations from processes to names. Given a context, $M$, we can
define a \emph{nominal context}, $\quotep{M}$ by $\quotep{M}[P] :=
\quotep{M[P]}$. To foreshadow what is to come we observe that these
operations enjoy a duality with processes very much like the duality
between vectors and maps from vectors to scalars.

Further, because the calculus is essentially higher-order, we have a
correspondence between contexts and processes. More specifically,
given a name $x$ and a context $M$ we can construct $M^{*}_{x}$ such
that 

\begin{mathpar}
  M^{*}_{x} | \lift{x}{P} \red M[P]
\end{mathpar}

namely,

\begin{mathpar}
  M^{*}_{x} := x?(u).M[\dropn{u}]
\end{mathpar}

The dependence of $M^{*}_{x}$ on a name makes it an abstraction, 

\begin{mathpar}
  M^{*} := (x)x?(u).M[\dropn{u}]
\end{mathpar}

\subsection{Additional notation}

It will sometimes be convenient to denote the process a name
quotes. We already have the notation $x = \quotep{P}$, but it will be
convenient to introduce an alternate notation, $\procn{x}$, when we
want to emphasize the connection to the use of the name. Note that, by
virtue of name equivalence, $\quotep{\procn{x}} \nameeq x$; so, the
notation is consistent with previous definitions.

Further, because names have structure it is possible to effect
substitutions on the basis of that structure. This means we need to
upgrade our notation for substitutions, which we accomplish by
adapting comprehension notation. Thus,

\begin{mathpar}
  P\{ y / x : x \in S \}
\end{mathpar}

is interpreted to mean the process derived from P by replacing (in a
capture-avoiding manner) each occurrence of $x$ in $S$ by $y$. For example,

\begin{mathpar}
  P\{ \quotep{\procn{x}|\procn{x}} / x : x \in \freenames{P} \}
\end{mathpar}

will replace each (occurrence) of a free name $x$ in $P$ by
$\quotep{\procn{x}|\procn{x}}$.

Also, we will avail ourselves of the notation $x^{L}$ and $x^{R}$ to
denote injections of a name into disjoint copies of the name
space. There are numerous ways to accomplish this. One example can be
found in \cite{MeredithR05}. This notation overloads to vectors of
names: $\vec{x}^{\pi} := (x_{i}^{\pi} \; : \; 0 \leq i < |\vec{x}| )$ where $\pi \in \{L,R\}$.

We also use $P^{\Box} := P|\Box$.

In \cite{MeredithR05} an interpretation of the new operator is
given. It turns out that there are several possible interpretations
all enjoying the requisite algebraic properties of the operator (see
\cite{milner91polyadicpi}). We will therefore make liberal use of
$(\nu\; \vec{x})P$.

% subsection the_syntax_and_semantics_of_the_notation_system (end)   

\input{qm2pi.qmops} 

\input{qm2pi.sterngerlach} 

\input{qm2pi.metric} 

% section concurrent_process_calculi (end)

%\input{qm2pi.proofsketch}

% section proof sketch (end)

%\input{qm2pi.slviaknots} 

% section spatial logic via knots (end)

\input{qm2pi.conclusion}

% section conclusion (end)

%\input{qm2pi.dtcodes} 

% section wiring algorithm (end)

\input{qm2pi.ack} 

% section acknowledgments (end)

\newpage


\bibliographystyle{plain}   
\bibliography{../../biblios/main.bib}

\input{qm2pi.rhodetails}

\end{document}

 

% section acknowledgments (end)

\newpage


\bibliographystyle{plain}   
\bibliography{../../biblios/main.bib}

\documentclass[12pt]{llncs}
%\documentclass{jktr}

\usepackage[pdftex]{hyperref}                   
\usepackage {listings}
\usepackage {mathpartir}
\usepackage{bcprules}
%\usepackage{listings}
                       
\usepackage{graphicx} 
%\usepackage[margins=2.5cm,nohead,nofoot]{geometry}
%\usepackage{geometry}
\usepackage{amsfonts}
\usepackage{amstext}
\usepackage{latexsym}
\usepackage{amssymb}
\usepackage{color}


%\include{myPreamble}
\include{qm2pi.local} 

%\ifpdf
%\usepackage[pdftex]{graphicx}
%\else
%\usepackage{graphicx}
%\fi

 % \ifpdf
%  \usepackage{pdfsync}
%  \if


%\title{Brief Article}
%\author{David F. Snyder}
%\author{L.G. Meredith}

%\address{Dept. of Math., Texas State University--San Marcos, San Marcos, TX 78666}
       
\pagestyle{empty}


\begin{document}

\lstset{language=[Objective]Caml,frame=shadowbox}

\input{qm2pi.front}

% section front matter (end)

\input{qm2pi.intro} 
 
% section introduction (end)

% \input{qm2pi.knotations} 

% section notation (end)

\input{qm2pi.process.calculi} 

% section concurrent_process_calculi_and_spatial_logics_ (end)
    
%\input{qm2pi.knots2pi} 

%\input{qm2pi.trefoil} 

%\input{qm2pi.mainthm} 

% subsection basic_interpretation (end)

%\input{qm2pi.rho.presentation} 
\subsection{The syntax and semantics of the notation system}\label{sub:the_syntax_and_semantics_of_the_notation_system} % (fold)

We now summarize a technical presentation of the calculus that
embodies our theory of dynamics. The typical presentation of such a
calculus follows the style of giving generators and relations on
them. The grammar, below, describing term constructors, freely
generates the set of processes, $\Proc$. This set is then quotiented
by a relation known as structural congruence and it is over this set
that the notion of dynamics is expressed. This presentation is
essentially that of \cite{MeredithR05} with the addition of
polyadicity and summation. For readability we have relegated some of
the technical subtleties to an appendix.

\subsubsection{Process grammar}\label{subsub:process_grammar}

\begin{mathpar}
  \inferrule* [lab=synchronization] {} {{M} \bc \pzero \;|\; x?F \;|\; x!C }
  \and
  \inferrule* [lab=abstraction] {} {{F} \bc (x)P}
  \and
  \inferrule* [lab=concretion] {} {{C} \bc \langle Q \rangle}
  \and
  \inferrule* [lab=process] {} {{P,Q} \bc M \;| \;P|Q \;|\; @{x}}
  \and
  \inferrule* [lab=name] {} {{x} \bc \quotep{P}}
\end{mathpar} 

Note that $\vec{x}$ (resp. $\vec{P}$) denotes a vector of names
(resp. processes) of length $|\vec{x}|$ (resp. $|\vec{P}|$). We adopt
the following useful abbreviations.

\begin{mathpar}
   x?(\vec{y}).P := x.(\vec{y})P \and  x\clift{\vec{P}} := x.\clift{\vec{P}}
   \and x!(y) := \lift{x}{\dropn{y}}
   \and \Pi_{i=0}^{n-1}P_i := P_0 | \ldots | P_{n-1}
\end{mathpar}

\subsubsection{Structural congruence}

\paragraph{Free and bound names and alpha-equivalence.} At the
core of structural equivalence is alpha-equivalence which identifies
process that are the same up to a change of variable. Formally, we
recognize the distinction between free and bound names. The free names
of a process, $\freenames{P}$, may be calculated recursively as
follows:

\begin{mathpar}
\freenames{\pzero} := \emptyset
  \and \\
  \freenames{x?(y).P} := \{ x \} \cup (\freenames{P} \setminus \{ y \})
  \and 
  \freenames{x!\langle P \rangle} := \{ x \} \cup \{ P \} 
  \and \\
  \freenames{P|Q} := \freenames{P} \cup \freenames{Q}
  \and \\
  \freenames{@{x}} := \{ x \}
\end{mathpar}

$\pi$
$\quotep{\pi}$

$\freenames{-} : \pi \to \mathcal{P}(\quotep{\pi})$

\begin{eqnarray*}
  \freenames{\pzero} & := & \emptyset \\
  \freenames{x?(y).P} & := & \{ x \} \cup (\freenames{P} \setminus \{ y \}) \\
  \freenames{x!\langle P \rangle} & := & \{ x \} \cup \{ P \} \\
  \freenames{P|Q} & := & \freenames{P} \cup \freenames{Q} \\
  \freenames{\dropn{x}} & := & \{ x \}
\end{eqnarray*}

The bound names of a process, $\boundnames{P}$, are those names occurring in $P$
that are not free. For example, in $x?(y).0$, the name $x$ is free, while $y$ is bound.

\begin{mathpar}
  \inferrule* [lab=monoidal-laws] {} { P|Q \equiv Q|P \and P|0 \equiv P \and P|(Q|R) \equiv (P|Q)|R }
\end{mathpar}

\begin{mathpar}
  \inferrule* [lab=alpha-equivalence] {} { (x)P \equiv (y)P\{y/x\} \and y \not\in \freenames{P} }
\end{mathpar}

\begin{definition}
Then two processes, $P,Q$, are alpha-equivalent if $P = Q\{\vec{y}/\vec{x}\}$ for
some $\vec{x} \in \boundnames{Q},\vec{y} \in \boundnames{P}$, where $Q\{\vec{y}/\vec{x}\}$
denotes the capture-avoiding substitution of $\vec{y}$ for $\vec{x}$ in $Q$.
\end{definition}

\begin{definition}
  The {\em structural congruence} \cite{SangiorgiWalker} , $\equiv$,
  between processes is the least congruence containing
  alpha-equivalence, satisfying the abelian monoid laws
  (associativity, commutativity and $\pzero$ as identity) for parallel
  composition $|$ and for summation $+$.
\end{definition}

\subsection{Name equivalence}

We take name equivalence, written $\nameeq$, to be the smallest
equivalence relation generated by the following rules.

\begin{mathpar}
\inferrule*[lab=Quote-drop]
{ }
{ \quotep{@{x}} \nameeq x }

\inferrule*[lab=Struct-equiv]
{ P \scong Q }
{ \quotep{P} \nameeq \quotep{Q} }
\end{mathpar}

The astute reader will have noticed that the mutual recursion of names
and processes imposes a mutual recursion on alpha-equivalence and
structural equivalence via name-equivalence. Fortunately, all of this
works out pleasantly and we may calculate in the natural way, free of
concern. The reader interested in the details is referred to the
appendix \ref{appendix:rho_details}.

\subsection{Substitution}

We use $\Proc$ for the set of processes, $\QProc$ for the set of
names, and $\id{\{}\vec{y} / \vec{x} \id{\}}$ to denote partial maps,
$s : \QProc \rightarrow \QProc$. A map, $s$ lifts, uniquely, to a map
on process terms, $\widehat{s} : \Proc \rightarrow \Proc$ by the
following equations.

\begin{mathpar}
  (0) \psubstp{Q}{P} := 0 \\
  (R \juxtap S) \psubstp{Q}{P}
  :=    
  (R)\psubstp{Q}{P} \juxtap (S) \psubstp{Q}{P} \\
  (x?(y).R) \psubstp{Q}{P}    
  :=    
  (x)\substp{Q}{P} (z)\concat( (R \psubstn{z}{y}) \psubstp{Q}{P} ) \\
  (\lift{x}{R}) \psubstp{Q}{P}  
  :=
  \lift{(x)\substp{Q}{P}}{ R \psubstp{Q}{P} } \\
%   (\dropn{x})  \psubstp{Q}{P}       
%   := 
%   \left\{ 
%     \begin{array}{ccc} 
%       \dropn{\quotep{Q}} & & x \nameeq \quotep{P} \\
%       \dropn{x} & & otherwise \\
%     \end{array}
%   \right. 
  (\dropn{x})  \psubstp{Q}{P}       
  := 
  \left\{ 
    \begin{array}{ccc} 
      Q & & x \nameeq \quotep{P} \\
      \dropn{x} & & otherwise \\
    \end{array}
  \right.
\end{mathpar}
 

where

\begin{eqnarray}
  (x)\id{\{} \lpquote Q \rpquote / \lpquote P \rpquote \id{\}}            = 
  \left\{ 
    \begin{array}{ccc}
      \lpquote Q \rpquote & & x \nameeq \lpquote P \rpquote \\
      x & & otherwise \\
    \end{array}
  \right. \nonumber
\end{eqnarray}

and $z$ is chosen distinct from $\quotep{P}$, $\quotep{Q}$, the free
names in $Q$, and all the names in $R$. Our $\alpha$-equivalence will
be built in the standard way from this substitution.

\begin{remark}\label{rem:no_self_referential_names}
  One consequence of these definitions is that $\forall P. \quotep{P}
  \not\in \freenames{P}$.
\end{remark}

\subsection{ Dynamic quote: an example }

Anticipating something of what's to come, consider applying the
substitution, $\widehat{\id{\{}u / z \id{\}}}$, to the following pair
of processes, $\lift{w}{y!(z)}$ and $w[ \lpquote y!(z) \rpquote ]$.

\begin{eqnarray}
	\lift{w}{y!(z)}\widehat{\id{\{}u / z \id{\}}}
		& = &
		\lift{w}{y!(u)} \nonumber\\
	w[ \lpquote y!(z) \rpquote ] \widehat{ \id{\{}u / z \id{\}} }
		& = &
		w[ \lpquote y!(z) \rpquote ] \nonumber
\end{eqnarray}

Because the body of the process between quotes is impervious to
substitution, we get radically different answers. In fact, by
examining the first process in an input context,
e.g. $x?(z).\lift{w}{y!(z)}$, we see that the process under the lift
operator may be shaped by prefixed inputs binding a name inside it. In
this sense, the lift operator will be seen as a way to dynamically
construct processes before reifying them as names.

Finally equipped with these standard features we can present the
dynamics of the calculus.

\subsubsection{Operational semantics} 

Finally, we introduce the computational dynamics. What marks these
algebras as distinct from other more traditionally studied algebraic
structures, e.g. vector spaces or polynomial rings, is the manner in
which dynamics is captured. In traditional structures, dynamics is typically
expressed through morphisms between such structures, as in linear maps
between vector spaces or morphisms between rings. In algebras
associated with the semantics of computation, the dynamics is
expressed as part of the algebraic structure itself, through a
reduction reduction relation typically denoted by $\red$. Below, we
give a recursive presentation of this relation for the calculus used
in the encoding.

$\red \subseteq \pi \times \pi$
$\red : \pi \to \mathcal{P}(\pi)$

\begin{mathpar}
  \inferrule* [lab=Comm] { \textsf{match}( x_{src}, x_{trgt} ) } { x_{trgt}?(y)P \; | \; x_{src}!\langle {Q} \rangle \red P\{\quotep{Q}/y}\} }
  \and \\
  \inferrule* [lab=Par] {{P} \red {P}'} {{{P} | {Q}} \red {{P}' | {Q}}}
  \and
  \inferrule* [lab=Equiv]{{{P} \scong {P}'} \andalso {{P}' \red {Q}'} \andalso {{Q}' \scong {Q}}}{{P} \red {Q}}
\end{mathpar}

\begin{eqnarray*}
  match_{\equiv} (\quotep{P},\quotep{Q}) & := & P \equiv Q \\
  match_{\dagger}(\quotep{P},\quotep{Q}) & := & \forall R. P|Q \red^{*} R => R \red^{*} 0 \\
  match_{K}(\quotep{P},\quotep{Q}) & := & K \mbox{ for some context } K
\end{eqnarray*}

$u?(x)P | u!\langle Q \rangle \red P\{\quotep{Q}/x\}$

%We write $\wred$ for $\red^*$, and $P\red$ if $\exists Q $ such that $ P \red Q$.
We write $P\red$ if $\exists Q $ such that $ P \red Q$ and $P\not\red$, otherwise.

\section{Replication}

As mentioned before, it is known that replication (and hence
recursion) can be implemented in a higher-order process algebra
\cite{SangiorgiWalker}. As our first example of calculation with the
machinery thus far presented we give the construction explicitly in
the {\rhoc}.

\begin{eqnarray}
	D_{x} & := & \prefix{x}{y}{(\binpar{\outputp{x}{y}}{@{y}})} \nonumber\\
	\bangp_{x}{P} & := & \binpar{{x}!\langle{\binpar{D_{x}}{P}}\rangle}{D_{x}} \nonumber
\end{eqnarray}

\begin{eqnarray}
	\bangp_{x}{P} & & \nonumber\\
	=
	& {x}!\langle{(\prefix{x}{y}{(\outputp{x}{y} | @{y})) | P}}\rangle 
	      | \prefix{x}{y}{(\outputp{x}{y} | @{y})} & \nonumber\\
	\red
	& (\outputp{x}{y} | @{y})\substn{\quotep{(\prefix{x}{y}{(@{y} | \outputp{x}{y})) | P}}}{y} & \nonumber\\
	=
	& \outputp{x}{\quotep{(\prefix{x}{y}{(\outputp{x}{y} | @{y})) | P}}}
	  | {(\prefix{x}{y}{(\outputp{x}{y} | @{y})) | P}} & \nonumber\\
	\red
	& \ldots & \nonumber\\
	\red^*
	& P | P | \ldots & \nonumber
\end{eqnarray}

Of course, this encoding, as an implementation, runs away, unfolding
$\bangp{P}$ eagerly. A lazier and more implementable replication
operator, restricted to input-guarded processes, may be obtained as follows.

\begin{eqnarray}
\bangp{\prefix{u}{v}{P}} 
	:= 
	\binpar{\lift{x}{\prefix{u}{v}{(\binpar{D(x)}{P})}}}{D(x)} \nonumber
\end{eqnarray}

\begin{remark}
  Note that the lazier definition still does not deal with summation
  or mixed summation (i.e. sums over input and output). The reader is
  invited to construct definitions of replication that deal with these
  features. 

  Further, the definitions are parameterized in a name, $x$. Can you,
  gentle reader, make a definition that eliminates this parameter and
  guarantees no accidental interaction between the replication
  machinery and the process being replicated -- i.e. no accidental
  sharing of names used by the process to get its work done and the
  name(s) used by the replication to effect copying. This latter
  revision of the definition of replication is crucial to obtaining
  the expected identity $!!P \sim !P$.
\end{remark}

\begin{remark}\label{rem:paradoxical_combinator}
  The reader familiar with the lambda calculus will have noticed the
  similarity between $D$ and the paradoxical combinator.

  [Ed. note: the existence of this seems to suggest we have to be more
  restrictive on the set of processes and names we admit if we are to
  support no-cloning.]
\end{remark}

\subsubsection{Bisimulation}

The computational dynamics gives rise to another kind of equivalence,
the equivalence of computational behavior. As previously mentioned
this is typically captured \emph{via} some form of bisimulation.

% The notion we use in this paper is weak barbed bisimulation
% \cite{milner91polyadicpi}.

The notion we use in this paper is derived from weak barbed
bisimulation \cite{milner91polyadicpi}. 

\begin{definition}
An \emph{observation relation}, $\downarrow_{\mathcal N}$, over a set
of names, $\mathcal N$, is the smallest relation satisfying the rules
below.

\infrule[Out-barb]{y \in {\mathcal N}, \; x \nameeq y}
		  {\outputp{x}{v} \downarrow_{\mathcal N} x}
\infrule[Par-barb]{\mbox{$P\downarrow_{\mathcal N} x$ or $Q\downarrow_{\mathcal N} x$}}
		  {\binpar{P}{Q} \downarrow_{\mathcal N} x}

We write $P \Downarrow_{\mathcal N} x$ if there is $Q$ such that 
$P \wred Q$ and $Q \downarrow_{\mathcal N} x$.
\end{definition}

\begin{definition}
%\label{def.bbisim}
An  ${\mathcal N}$-\emph{barbed bisimulation} over a set of names, ${\mathcal N}$, is a symmetric binary relation 
${\mathcal S}_{\mathcal N}$ between agents such that $P\rel{S}_{\mathcal N}Q$ implies:
\begin{enumerate}
\item If $P \red P'$ then $Q \wred Q'$ and $P'\rel{S}_{\mathcal N} Q'$.
\item If $P\downarrow_{\mathcal N} x$, then $Q\Downarrow_{\mathcal N} x$.
\end{enumerate}
$P$ is ${\mathcal N}$-barbed bisimilar to $Q$, written
$P \wbbisim_{\mathcal N} Q$, if $P \rel{S}_{\mathcal N} Q$ for some ${\mathcal N}$-barbed bisimulation ${\mathcal S}_{\mathcal N}$.
\end{definition}

$\mathcal{R} \subseteq \pi \times \pi$

$P \mathcal{R} Q => \forall P'. P \red P' \Rightarrow \exists Q'. Q \red Q', P' \mathcal{R} Q'$

$P \vdash x \Rightarrow Q \vdash x$

\begin{mathpar}
  \inferrule*[lab=Out-barb]{x \nameeq y}{{y}!\langle{Q}\rangle \vdash x}
  \and
  \inferrule*[lab=Par-barb]{\mbox{$P\vdash x$ or $Q\vdash x$}}{\binpar{P}{Q} \vdash x}
\end{mathpar}

\subsubsection{Contexts}

One of the principle advantages of computational calculi like the
$\pi$-calculus is a well-defined notion of context,
contextual-equivalence and a correlation between
contextual-equivalence and notions of bisimulation. The notion of
context allows the decomposition of a process into (sub-)process and
its syntactic environment, its context. Thus, a context may be
thought of as a process with a ``hole'' (written $\Box$) in it. The
application of a context $M$ to a process $P$, written $M[P]$, is
tantamount to filling the hole in $M$ with $P$. In this paper we do
not need the full weight of this theory, but do make use of the notion
of context in the proof the main theorem. 

\begin{mathpar}
  \inferrule* [lab=summation] {} {{M_{M},M_{N}} \bc \Box \;|\; x.M_{A} \;|\; M_{M}+M_{N}}
  \and
  \inferrule* [lab=agent] {} {{M_{A}} \bc (\vec{x})M_{P} \;| \; \clift{P_0,\ldots,M_{P},\ldots,P_N}}
  \and \\
  \inferrule* [lab=process] {} {{M_{P}} \bc M_{N} \;| \;P|M_{P} }
\end{mathpar} 

\begin{mathpar}
  \inferrule* [lab=sychronization] {} {M_{N} \bc \Box \;|\; x?M_{F} \;|\; x!M_{C}}
  \and
  \inferrule* [lab=abstraction] {} {{M_{F}} \bc (x)M_{P} }
  \and
  \inferrule* [lab=concretion] {} {{M_{C}} \bc \langle M_{P} \rangle }
  \and \\
  \inferrule* [lab=process] {} {{M_{P}} \bc M_{N} \;| \;P|M_{P} }
\end{mathpar}

\begin{definition}[contextual application] Given a context $M$, and
  process $P$, we define the \emph{contextual application}, $M[P] :=
  M\{P/\Box\}$. That is, the contextual application of M to P is the
  substitution of $P$ for $\Box$ in $M$.
\end{definition}

$\meaningof{-} : L \to \mathcal{P}(\pi)$

\begin{mathpar}
  \inferrule* [lab=collection] {} {\meaningof{true} = \pi, \and \meaningof{~E} = \pi \setminus \meaningof{E}, \and \meaningof{E_{1} \& E_{2}} = \meaningof{E_{1}} \cap \meaningof{E_{2}}}
\end{mathpar}

\begin{mathpar}
  \inferrule* [lab=structure] {} {\meaningof{0} = \{ P \in \pi | P \equiv 0 \}, \and \\ \meaningof{E_1 | E_2} = \{ P \in \pi | P \equiv P_{1} | P_{2}, P_{1} \in \meaningof{E_{1}}, P_{2} \in \meaningof{E_2}\} }
\end{mathpar}

\begin{mathpar}
 \inferrule* [lab=behavior] {} {\meaningof{\langle a?b \rangle E} = \{ P \in \pi | P \equiv Q | u?(y)P', \\ \and \\\\ \and \\ \;\;\; u \in \meaningof{a}, \forall z.P'\{z/y\} \in \meaningof{E\{z/b\}}\}, \and \\ \meaningof{a!E} = \{ P \in \pi | P \equiv Q | x!\langle P' \rangle, x \in \meaningof{a} P' \in \meaningof{E}\} }
\end{mathpar}

\begin{mathpar}
 \inferrule* [lab=nominal] {} {\meaningof{\quotep{E}} = \{ \quotep{P} \in \quotep{\pi} | P \in \meaningof{E} \}, \and \meaningof{\quotep{P}} = \{ \quotep{Q} \in \quotep{\pi} | P \equiv Q \} \and \\ \meaningof{@\quotep{E}} = \{ P \in \pi | P \equiv @x, x \in \meaningof{E} \}}
\end{mathpar}

\begin{eqnarray*}
  \\
  \meaningof{-} : TS \to ST
\end{eqnarray*}

\begin{eqnarray*}
  \\
  L : TS \to ST
\end{eqnarray*}

\begin{eqnarray*}
  \\
  P \models E \iff P \in \meaningof{E}
\end{eqnarray*}

\begin{eqnarray*}
  P \approx_{L} Q \iff \forall E \in L. P \models E \iff Q \models E
\end{eqnarray*}

\begin{eqnarray*}
  P \approx_{K} Q
\end{eqnarray*}

\begin{eqnarray*}
  P \approx Q
\end{eqnarray*}

$\approx_{K} = \approx = \approx_{L}$

\subsubsection{Contextual duality}

Note that contexts extend the quotation operation to a family of
operations from processes to names. Given a context, $M$, we can
define a \emph{nominal context}, $\quotep{M}$ by $\quotep{M}[P] :=
\quotep{M[P]}$. To foreshadow what is to come we observe that these
operations enjoy a duality with processes very much like the duality
between vectors and maps from vectors to scalars.

Further, because the calculus is essentially higher-order, we have a
correspondence between contexts and processes. More specifically,
given a name $x$ and a context $M$ we can construct $M^{*}_{x}$ such
that 

\begin{mathpar}
  M^{*}_{x} | \lift{x}{P} \red M[P]
\end{mathpar}

namely,

\begin{mathpar}
  M^{*}_{x} := x?(u).M[\dropn{u}]
\end{mathpar}

The dependence of $M^{*}_{x}$ on a name makes it an abstraction, 

\begin{mathpar}
  M^{*} := (x)x?(u).M[\dropn{u}]
\end{mathpar}

\subsection{Additional notation}

It will sometimes be convenient to denote the process a name
quotes. We already have the notation $x = \quotep{P}$, but it will be
convenient to introduce an alternate notation, $\procn{x}$, when we
want to emphasize the connection to the use of the name. Note that, by
virtue of name equivalence, $\quotep{\procn{x}} \nameeq x$; so, the
notation is consistent with previous definitions.

Further, because names have structure it is possible to effect
substitutions on the basis of that structure. This means we need to
upgrade our notation for substitutions, which we accomplish by
adapting comprehension notation. Thus,

\begin{mathpar}
  P\{ y / x : x \in S \}
\end{mathpar}

is interpreted to mean the process derived from P by replacing (in a
capture-avoiding manner) each occurrence of $x$ in $S$ by $y$. For example,

\begin{mathpar}
  P\{ \quotep{\procn{x}|\procn{x}} / x : x \in \freenames{P} \}
\end{mathpar}

will replace each (occurrence) of a free name $x$ in $P$ by
$\quotep{\procn{x}|\procn{x}}$.

Also, we will avail ourselves of the notation $x^{L}$ and $x^{R}$ to
denote injections of a name into disjoint copies of the name
space. There are numerous ways to accomplish this. One example can be
found in \cite{MeredithR05}. This notation overloads to vectors of
names: $\vec{x}^{\pi} := (x_{i}^{\pi} \; : \; 0 \leq i < |\vec{x}| )$ where $\pi \in \{L,R\}$.

We also use $P^{\Box} := P|\Box$.

In \cite{MeredithR05} an interpretation of the new operator is
given. It turns out that there are several possible interpretations
all enjoying the requisite algebraic properties of the operator (see
\cite{milner91polyadicpi}). We will therefore make liberal use of
$(\nu\; \vec{x})P$.

% subsection the_syntax_and_semantics_of_the_notation_system (end)   

\input{qm2pi.qmops} 

\input{qm2pi.sterngerlach} 

\input{qm2pi.metric} 

% section concurrent_process_calculi (end)

%\input{qm2pi.proofsketch}

% section proof sketch (end)

%\input{qm2pi.slviaknots} 

% section spatial logic via knots (end)

\input{qm2pi.conclusion}

% section conclusion (end)

%\input{qm2pi.dtcodes} 

% section wiring algorithm (end)

\input{qm2pi.ack} 

% section acknowledgments (end)

\newpage


\bibliographystyle{plain}   
\bibliography{../../biblios/main.bib}

\input{qm2pi.rhodetails}

\end{document}



\end{document}

 

% subsection basic_interpretation (end)

%\input{qm2pi.rho.presentation} 
\subsection{The syntax and semantics of the notation system}\label{sub:the_syntax_and_semantics_of_the_notation_system} % (fold)

We now summarize a technical presentation of the calculus that
embodies our theory of dynamics. The typical presentation of such a
calculus follows the style of giving generators and relations on
them. The grammar, below, describing term constructors, freely
generates the set of processes, $\Proc$. This set is then quotiented
by a relation known as structural congruence and it is over this set
that the notion of dynamics is expressed. This presentation is
essentially that of \cite{MeredithR05} with the addition of
polyadicity and summation. For readability we have relegated some of
the technical subtleties to an appendix.

\subsubsection{Process grammar}\label{subsub:process_grammar}

\begin{mathpar}
  \inferrule* [lab=synchronization] {} {{M} \bc \pzero \;|\; x?F \;|\; x!C }
  \and
  \inferrule* [lab=abstraction] {} {{F} \bc (x)P}
  \and
  \inferrule* [lab=concretion] {} {{C} \bc \langle Q \rangle}
  \and
  \inferrule* [lab=process] {} {{P,Q} \bc M \;| \;P|Q \;|\; @{x}}
  \and
  \inferrule* [lab=name] {} {{x} \bc \quotep{P}}
\end{mathpar} 

Note that $\vec{x}$ (resp. $\vec{P}$) denotes a vector of names
(resp. processes) of length $|\vec{x}|$ (resp. $|\vec{P}|$). We adopt
the following useful abbreviations.

\begin{mathpar}
   x?(\vec{y}).P := x.(\vec{y})P \and  x\clift{\vec{P}} := x.\clift{\vec{P}}
   \and x!(y) := \lift{x}{\dropn{y}}
   \and \Pi_{i=0}^{n-1}P_i := P_0 | \ldots | P_{n-1}
\end{mathpar}

\subsubsection{Structural congruence}

\paragraph{Free and bound names and alpha-equivalence.} At the
core of structural equivalence is alpha-equivalence which identifies
process that are the same up to a change of variable. Formally, we
recognize the distinction between free and bound names. The free names
of a process, $\freenames{P}$, may be calculated recursively as
follows:

\begin{mathpar}
\freenames{\pzero} := \emptyset
  \and \\
  \freenames{x?(y).P} := \{ x \} \cup (\freenames{P} \setminus \{ y \})
  \and 
  \freenames{x!\langle P \rangle} := \{ x \} \cup \{ P \} 
  \and \\
  \freenames{P|Q} := \freenames{P} \cup \freenames{Q}
  \and \\
  \freenames{@{x}} := \{ x \}
\end{mathpar}

$\pi$
$\quotep{\pi}$

$\freenames{-} : \pi \to \mathcal{P}(\quotep{\pi})$

\begin{eqnarray*}
  \freenames{\pzero} & := & \emptyset \\
  \freenames{x?(y).P} & := & \{ x \} \cup (\freenames{P} \setminus \{ y \}) \\
  \freenames{x!\langle P \rangle} & := & \{ x \} \cup \{ P \} \\
  \freenames{P|Q} & := & \freenames{P} \cup \freenames{Q} \\
  \freenames{\dropn{x}} & := & \{ x \}
\end{eqnarray*}

The bound names of a process, $\boundnames{P}$, are those names occurring in $P$
that are not free. For example, in $x?(y).0$, the name $x$ is free, while $y$ is bound.

\begin{mathpar}
  \inferrule* [lab=monoidal-laws] {} { P|Q \equiv Q|P \and P|0 \equiv P \and P|(Q|R) \equiv (P|Q)|R }
\end{mathpar}

\begin{mathpar}
  \inferrule* [lab=alpha-equivalence] {} { (x)P \equiv (y)P\{y/x\} \and y \not\in \freenames{P} }
\end{mathpar}

\begin{definition}
Then two processes, $P,Q$, are alpha-equivalent if $P = Q\{\vec{y}/\vec{x}\}$ for
some $\vec{x} \in \boundnames{Q},\vec{y} \in \boundnames{P}$, where $Q\{\vec{y}/\vec{x}\}$
denotes the capture-avoiding substitution of $\vec{y}$ for $\vec{x}$ in $Q$.
\end{definition}

\begin{definition}
  The {\em structural congruence} \cite{SangiorgiWalker} , $\equiv$,
  between processes is the least congruence containing
  alpha-equivalence, satisfying the abelian monoid laws
  (associativity, commutativity and $\pzero$ as identity) for parallel
  composition $|$ and for summation $+$.
\end{definition}

\subsection{Name equivalence}

We take name equivalence, written $\nameeq$, to be the smallest
equivalence relation generated by the following rules.

\begin{mathpar}
\inferrule*[lab=Quote-drop]
{ }
{ \quotep{@{x}} \nameeq x }

\inferrule*[lab=Struct-equiv]
{ P \scong Q }
{ \quotep{P} \nameeq \quotep{Q} }
\end{mathpar}

The astute reader will have noticed that the mutual recursion of names
and processes imposes a mutual recursion on alpha-equivalence and
structural equivalence via name-equivalence. Fortunately, all of this
works out pleasantly and we may calculate in the natural way, free of
concern. The reader interested in the details is referred to the
appendix \ref{appendix:rho_details}.

\subsection{Substitution}

We use $\Proc$ for the set of processes, $\QProc$ for the set of
names, and $\id{\{}\vec{y} / \vec{x} \id{\}}$ to denote partial maps,
$s : \QProc \rightarrow \QProc$. A map, $s$ lifts, uniquely, to a map
on process terms, $\widehat{s} : \Proc \rightarrow \Proc$ by the
following equations.

\begin{mathpar}
  (0) \psubstp{Q}{P} := 0 \\
  (R \juxtap S) \psubstp{Q}{P}
  :=    
  (R)\psubstp{Q}{P} \juxtap (S) \psubstp{Q}{P} \\
  (x?(y).R) \psubstp{Q}{P}    
  :=    
  (x)\substp{Q}{P} (z)\concat( (R \psubstn{z}{y}) \psubstp{Q}{P} ) \\
  (\lift{x}{R}) \psubstp{Q}{P}  
  :=
  \lift{(x)\substp{Q}{P}}{ R \psubstp{Q}{P} } \\
%   (\dropn{x})  \psubstp{Q}{P}       
%   := 
%   \left\{ 
%     \begin{array}{ccc} 
%       \dropn{\quotep{Q}} & & x \nameeq \quotep{P} \\
%       \dropn{x} & & otherwise \\
%     \end{array}
%   \right. 
  (\dropn{x})  \psubstp{Q}{P}       
  := 
  \left\{ 
    \begin{array}{ccc} 
      Q & & x \nameeq \quotep{P} \\
      \dropn{x} & & otherwise \\
    \end{array}
  \right.
\end{mathpar}
 

where

\begin{eqnarray}
  (x)\id{\{} \lpquote Q \rpquote / \lpquote P \rpquote \id{\}}            = 
  \left\{ 
    \begin{array}{ccc}
      \lpquote Q \rpquote & & x \nameeq \lpquote P \rpquote \\
      x & & otherwise \\
    \end{array}
  \right. \nonumber
\end{eqnarray}

and $z$ is chosen distinct from $\quotep{P}$, $\quotep{Q}$, the free
names in $Q$, and all the names in $R$. Our $\alpha$-equivalence will
be built in the standard way from this substitution.

\begin{remark}\label{rem:no_self_referential_names}
  One consequence of these definitions is that $\forall P. \quotep{P}
  \not\in \freenames{P}$.
\end{remark}

\subsection{ Dynamic quote: an example }

Anticipating something of what's to come, consider applying the
substitution, $\widehat{\id{\{}u / z \id{\}}}$, to the following pair
of processes, $\lift{w}{y!(z)}$ and $w[ \lpquote y!(z) \rpquote ]$.

\begin{eqnarray}
	\lift{w}{y!(z)}\widehat{\id{\{}u / z \id{\}}}
		& = &
		\lift{w}{y!(u)} \nonumber\\
	w[ \lpquote y!(z) \rpquote ] \widehat{ \id{\{}u / z \id{\}} }
		& = &
		w[ \lpquote y!(z) \rpquote ] \nonumber
\end{eqnarray}

Because the body of the process between quotes is impervious to
substitution, we get radically different answers. In fact, by
examining the first process in an input context,
e.g. $x?(z).\lift{w}{y!(z)}$, we see that the process under the lift
operator may be shaped by prefixed inputs binding a name inside it. In
this sense, the lift operator will be seen as a way to dynamically
construct processes before reifying them as names.

Finally equipped with these standard features we can present the
dynamics of the calculus.

\subsubsection{Operational semantics} 

Finally, we introduce the computational dynamics. What marks these
algebras as distinct from other more traditionally studied algebraic
structures, e.g. vector spaces or polynomial rings, is the manner in
which dynamics is captured. In traditional structures, dynamics is typically
expressed through morphisms between such structures, as in linear maps
between vector spaces or morphisms between rings. In algebras
associated with the semantics of computation, the dynamics is
expressed as part of the algebraic structure itself, through a
reduction reduction relation typically denoted by $\red$. Below, we
give a recursive presentation of this relation for the calculus used
in the encoding.

$\red \subseteq \pi \times \pi$
$\red : \pi \to \mathcal{P}(\pi)$

\begin{mathpar}
  \inferrule* [lab=Comm] { \textsf{match}( x_{src}, x_{trgt} ) } { x_{trgt}?(y)P \; | \; x_{src}!\langle {Q} \rangle \red P\{\quotep{Q}/y}\} }
  \and \\
  \inferrule* [lab=Par] {{P} \red {P}'} {{{P} | {Q}} \red {{P}' | {Q}}}
  \and
  \inferrule* [lab=Equiv]{{{P} \scong {P}'} \andalso {{P}' \red {Q}'} \andalso {{Q}' \scong {Q}}}{{P} \red {Q}}
\end{mathpar}

\begin{eqnarray*}
  match_{\equiv} (\quotep{P},\quotep{Q}) & := & P \equiv Q \\
  match_{\dagger}(\quotep{P},\quotep{Q}) & := & \forall R. P|Q \red^{*} R => R \red^{*} 0 \\
  match_{K}(\quotep{P},\quotep{Q}) & := & K \mbox{ for some context } K
\end{eqnarray*}

$u?(x)P | u!\langle Q \rangle \red P\{\quotep{Q}/x\}$

%We write $\wred$ for $\red^*$, and $P\red$ if $\exists Q $ such that $ P \red Q$.
We write $P\red$ if $\exists Q $ such that $ P \red Q$ and $P\not\red$, otherwise.

\section{Replication}

As mentioned before, it is known that replication (and hence
recursion) can be implemented in a higher-order process algebra
\cite{SangiorgiWalker}. As our first example of calculation with the
machinery thus far presented we give the construction explicitly in
the {\rhoc}.

\begin{eqnarray}
	D_{x} & := & \prefix{x}{y}{(\binpar{\outputp{x}{y}}{@{y}})} \nonumber\\
	\bangp_{x}{P} & := & \binpar{{x}!\langle{\binpar{D_{x}}{P}}\rangle}{D_{x}} \nonumber
\end{eqnarray}

\begin{eqnarray}
	\bangp_{x}{P} & & \nonumber\\
	=
	& {x}!\langle{(\prefix{x}{y}{(\outputp{x}{y} | @{y})) | P}}\rangle 
	      | \prefix{x}{y}{(\outputp{x}{y} | @{y})} & \nonumber\\
	\red
	& (\outputp{x}{y} | @{y})\substn{\quotep{(\prefix{x}{y}{(@{y} | \outputp{x}{y})) | P}}}{y} & \nonumber\\
	=
	& \outputp{x}{\quotep{(\prefix{x}{y}{(\outputp{x}{y} | @{y})) | P}}}
	  | {(\prefix{x}{y}{(\outputp{x}{y} | @{y})) | P}} & \nonumber\\
	\red
	& \ldots & \nonumber\\
	\red^*
	& P | P | \ldots & \nonumber
\end{eqnarray}

Of course, this encoding, as an implementation, runs away, unfolding
$\bangp{P}$ eagerly. A lazier and more implementable replication
operator, restricted to input-guarded processes, may be obtained as follows.

\begin{eqnarray}
\bangp{\prefix{u}{v}{P}} 
	:= 
	\binpar{\lift{x}{\prefix{u}{v}{(\binpar{D(x)}{P})}}}{D(x)} \nonumber
\end{eqnarray}

\begin{remark}
  Note that the lazier definition still does not deal with summation
  or mixed summation (i.e. sums over input and output). The reader is
  invited to construct definitions of replication that deal with these
  features. 

  Further, the definitions are parameterized in a name, $x$. Can you,
  gentle reader, make a definition that eliminates this parameter and
  guarantees no accidental interaction between the replication
  machinery and the process being replicated -- i.e. no accidental
  sharing of names used by the process to get its work done and the
  name(s) used by the replication to effect copying. This latter
  revision of the definition of replication is crucial to obtaining
  the expected identity $!!P \sim !P$.
\end{remark}

\begin{remark}\label{rem:paradoxical_combinator}
  The reader familiar with the lambda calculus will have noticed the
  similarity between $D$ and the paradoxical combinator.

  [Ed. note: the existence of this seems to suggest we have to be more
  restrictive on the set of processes and names we admit if we are to
  support no-cloning.]
\end{remark}

\subsubsection{Bisimulation}

The computational dynamics gives rise to another kind of equivalence,
the equivalence of computational behavior. As previously mentioned
this is typically captured \emph{via} some form of bisimulation.

% The notion we use in this paper is weak barbed bisimulation
% \cite{milner91polyadicpi}.

The notion we use in this paper is derived from weak barbed
bisimulation \cite{milner91polyadicpi}. 

\begin{definition}
An \emph{observation relation}, $\downarrow_{\mathcal N}$, over a set
of names, $\mathcal N$, is the smallest relation satisfying the rules
below.

\infrule[Out-barb]{y \in {\mathcal N}, \; x \nameeq y}
		  {\outputp{x}{v} \downarrow_{\mathcal N} x}
\infrule[Par-barb]{\mbox{$P\downarrow_{\mathcal N} x$ or $Q\downarrow_{\mathcal N} x$}}
		  {\binpar{P}{Q} \downarrow_{\mathcal N} x}

We write $P \Downarrow_{\mathcal N} x$ if there is $Q$ such that 
$P \wred Q$ and $Q \downarrow_{\mathcal N} x$.
\end{definition}

\begin{definition}
%\label{def.bbisim}
An  ${\mathcal N}$-\emph{barbed bisimulation} over a set of names, ${\mathcal N}$, is a symmetric binary relation 
${\mathcal S}_{\mathcal N}$ between agents such that $P\rel{S}_{\mathcal N}Q$ implies:
\begin{enumerate}
\item If $P \red P'$ then $Q \wred Q'$ and $P'\rel{S}_{\mathcal N} Q'$.
\item If $P\downarrow_{\mathcal N} x$, then $Q\Downarrow_{\mathcal N} x$.
\end{enumerate}
$P$ is ${\mathcal N}$-barbed bisimilar to $Q$, written
$P \wbbisim_{\mathcal N} Q$, if $P \rel{S}_{\mathcal N} Q$ for some ${\mathcal N}$-barbed bisimulation ${\mathcal S}_{\mathcal N}$.
\end{definition}

$\mathcal{R} \subseteq \pi \times \pi$

$P \mathcal{R} Q => \forall P'. P \red P' \Rightarrow \exists Q'. Q \red Q', P' \mathcal{R} Q'$

$P \vdash x \Rightarrow Q \vdash x$

\begin{mathpar}
  \inferrule*[lab=Out-barb]{x \nameeq y}{{y}!\langle{Q}\rangle \vdash x}
  \and
  \inferrule*[lab=Par-barb]{\mbox{$P\vdash x$ or $Q\vdash x$}}{\binpar{P}{Q} \vdash x}
\end{mathpar}

\subsubsection{Contexts}

One of the principle advantages of computational calculi like the
$\pi$-calculus is a well-defined notion of context,
contextual-equivalence and a correlation between
contextual-equivalence and notions of bisimulation. The notion of
context allows the decomposition of a process into (sub-)process and
its syntactic environment, its context. Thus, a context may be
thought of as a process with a ``hole'' (written $\Box$) in it. The
application of a context $M$ to a process $P$, written $M[P]$, is
tantamount to filling the hole in $M$ with $P$. In this paper we do
not need the full weight of this theory, but do make use of the notion
of context in the proof the main theorem. 

\begin{mathpar}
  \inferrule* [lab=summation] {} {{M_{M},M_{N}} \bc \Box \;|\; x.M_{A} \;|\; M_{M}+M_{N}}
  \and
  \inferrule* [lab=agent] {} {{M_{A}} \bc (\vec{x})M_{P} \;| \; \clift{P_0,\ldots,M_{P},\ldots,P_N}}
  \and \\
  \inferrule* [lab=process] {} {{M_{P}} \bc M_{N} \;| \;P|M_{P} }
\end{mathpar} 

\begin{mathpar}
  \inferrule* [lab=sychronization] {} {M_{N} \bc \Box \;|\; x?M_{F} \;|\; x!M_{C}}
  \and
  \inferrule* [lab=abstraction] {} {{M_{F}} \bc (x)M_{P} }
  \and
  \inferrule* [lab=concretion] {} {{M_{C}} \bc \langle M_{P} \rangle }
  \and \\
  \inferrule* [lab=process] {} {{M_{P}} \bc M_{N} \;| \;P|M_{P} }
\end{mathpar}

\begin{definition}[contextual application] Given a context $M$, and
  process $P$, we define the \emph{contextual application}, $M[P] :=
  M\{P/\Box\}$. That is, the contextual application of M to P is the
  substitution of $P$ for $\Box$ in $M$.
\end{definition}

$\meaningof{-} : L \to \mathcal{P}(\pi)$

\begin{mathpar}
  \inferrule* [lab=collection] {} {\meaningof{true} = \pi, \and \meaningof{~E} = \pi \setminus \meaningof{E}, \and \meaningof{E_{1} \& E_{2}} = \meaningof{E_{1}} \cap \meaningof{E_{2}}}
\end{mathpar}

\begin{mathpar}
  \inferrule* [lab=structure] {} {\meaningof{0} = \{ P \in \pi | P \equiv 0 \}, \and \\ \meaningof{E_1 | E_2} = \{ P \in \pi | P \equiv P_{1} | P_{2}, P_{1} \in \meaningof{E_{1}}, P_{2} \in \meaningof{E_2}\} }
\end{mathpar}

\begin{mathpar}
 \inferrule* [lab=behavior] {} {\meaningof{\langle a?b \rangle E} = \{ P \in \pi | P \equiv Q | u?(y)P', \\ \and \\\\ \and \\ \;\;\; u \in \meaningof{a}, \forall z.P'\{z/y\} \in \meaningof{E\{z/b\}}\}, \and \\ \meaningof{a!E} = \{ P \in \pi | P \equiv Q | x!\langle P' \rangle, x \in \meaningof{a} P' \in \meaningof{E}\} }
\end{mathpar}

\begin{mathpar}
 \inferrule* [lab=nominal] {} {\meaningof{\quotep{E}} = \{ \quotep{P} \in \quotep{\pi} | P \in \meaningof{E} \}, \and \meaningof{\quotep{P}} = \{ \quotep{Q} \in \quotep{\pi} | P \equiv Q \} \and \\ \meaningof{@\quotep{E}} = \{ P \in \pi | P \equiv @x, x \in \meaningof{E} \}}
\end{mathpar}

\begin{eqnarray*}
  \\
  \meaningof{-} : TS \to ST
\end{eqnarray*}

\begin{eqnarray*}
  \\
  L : TS \to ST
\end{eqnarray*}

\begin{eqnarray*}
  \\
  P \models E \iff P \in \meaningof{E}
\end{eqnarray*}

\begin{eqnarray*}
  P \approx_{L} Q \iff \forall E \in L. P \models E \iff Q \models E
\end{eqnarray*}

\begin{eqnarray*}
  P \approx_{K} Q
\end{eqnarray*}

\begin{eqnarray*}
  P \approx Q
\end{eqnarray*}

$\approx_{K} = \approx = \approx_{L}$

\subsubsection{Contextual duality}

Note that contexts extend the quotation operation to a family of
operations from processes to names. Given a context, $M$, we can
define a \emph{nominal context}, $\quotep{M}$ by $\quotep{M}[P] :=
\quotep{M[P]}$. To foreshadow what is to come we observe that these
operations enjoy a duality with processes very much like the duality
between vectors and maps from vectors to scalars.

Further, because the calculus is essentially higher-order, we have a
correspondence between contexts and processes. More specifically,
given a name $x$ and a context $M$ we can construct $M^{*}_{x}$ such
that 

\begin{mathpar}
  M^{*}_{x} | \lift{x}{P} \red M[P]
\end{mathpar}

namely,

\begin{mathpar}
  M^{*}_{x} := x?(u).M[\dropn{u}]
\end{mathpar}

The dependence of $M^{*}_{x}$ on a name makes it an abstraction, 

\begin{mathpar}
  M^{*} := (x)x?(u).M[\dropn{u}]
\end{mathpar}

\subsection{Additional notation}

It will sometimes be convenient to denote the process a name
quotes. We already have the notation $x = \quotep{P}$, but it will be
convenient to introduce an alternate notation, $\procn{x}$, when we
want to emphasize the connection to the use of the name. Note that, by
virtue of name equivalence, $\quotep{\procn{x}} \nameeq x$; so, the
notation is consistent with previous definitions.

Further, because names have structure it is possible to effect
substitutions on the basis of that structure. This means we need to
upgrade our notation for substitutions, which we accomplish by
adapting comprehension notation. Thus,

\begin{mathpar}
  P\{ y / x : x \in S \}
\end{mathpar}

is interpreted to mean the process derived from P by replacing (in a
capture-avoiding manner) each occurrence of $x$ in $S$ by $y$. For example,

\begin{mathpar}
  P\{ \quotep{\procn{x}|\procn{x}} / x : x \in \freenames{P} \}
\end{mathpar}

will replace each (occurrence) of a free name $x$ in $P$ by
$\quotep{\procn{x}|\procn{x}}$.

Also, we will avail ourselves of the notation $x^{L}$ and $x^{R}$ to
denote injections of a name into disjoint copies of the name
space. There are numerous ways to accomplish this. One example can be
found in \cite{MeredithR05}. This notation overloads to vectors of
names: $\vec{x}^{\pi} := (x_{i}^{\pi} \; : \; 0 \leq i < |\vec{x}| )$ where $\pi \in \{L,R\}$.

We also use $P^{\Box} := P|\Box$.

In \cite{MeredithR05} an interpretation of the new operator is
given. It turns out that there are several possible interpretations
all enjoying the requisite algebraic properties of the operator (see
\cite{milner91polyadicpi}). We will therefore make liberal use of
$(\nu\; \vec{x})P$.

% subsection the_syntax_and_semantics_of_the_notation_system (end)   

\section{Interpretation of QM}
\subsection{Supporting definitions}
\subsubsection{Multiplication}
\begin{mathpar}
  \quotep{Q} \cdot \quotep{R} := \quotep{Q|R}
  \and \\
  \quotep{Q} \cdot P := P\{ \quotep{Q|R} / \quotep{R} : \quotep{R} \in \freenames{P} \}
\end{mathpar}

\paragraph{Discussion}
The first line needs little explanation. The second line says that
each free name of the process is replaced with the multiplication of
that name by the scalar. Multiplication of a scalar (name) by a state
(process) results in a process all the names of which have been `moved
over' by parallel composition with the process the scalar
quotes. There is a subtlety that the bound names have to be
manipulated so that multiplied names aren't accidentally
captured. There are many ways to achieve this.

\begin{remark}\label{rem:multiplication_identities}
  The reader is invited to verify that for all $x,y,z \in \QProc$ and $P \in \Proc$
  \begin{mathpar}
    x \cdot \quotep{0} \equiv x 
    \and
    x \cdot y \equiv y \cdot x
    \and
    x \cdot (y \cdot z) \equiv (x \cdot y) \cdot z
    \and \\
    \quotep{0} \cdot P \equiv P
    \and \\
    x \cdot (y \cdot P) \equiv (x \cdot y) \cdot P
    \and \\
    x \cdot (P|Q) \equiv (x \cdot P) | (x \cdot Q)
    \and \\    
  \end{mathpar}
\end{remark}

\subsubsection{Tensor product}

We define a tensor product on processes by structural induction.

\paragraph{Tensor of sums} First note that all summations, including
$\pzero$ and sequence, can be written $\Sigma_{i} x_{i}.A_{i} +
\Sigma_{j} x_{j}.C_{j}$, where we have grouped input-guarded processes
together and output-guarded processes together.

Thus, we can define the tensor product of two summations, $N_{1}\otimes N_{2}$, where

\begin{mathpar}
  N_{1} := \Sigma_{i} x_{i}.A_{i} + \Sigma_{j} x_{j}.C_{j}
  \and
  N_{2} := \Sigma_{i'} y_{i'}.B_{i'} + \Sigma_{j'} y_{j'}.D_{j'} 
\end{mathpar}

as follows.

\begin{mathpar}
  \Sigma_{i} x_{i}.A_{i} + \Sigma_{j} x_{j}.C_{j} \otimes \Sigma_{i'}
  y_{i'}.B_{i'} + \Sigma_{j'} y_{j'}.D_{j'} 
  \and \\
  := \; \Sigma_{i} \Sigma_{i'} \quotep{\stackrel{\vee}{x_{i}}| \stackrel{\vee}{y_{i'}}}.(A_{i}\otimes B_{i'}) \; | \; \Sigma_{i'} \Sigma_{i} \quotep{\stackrel{\vee}{y_{i'}}|\stackrel{\vee}{x_{i}}}.(B_{i'}\otimes A_{i})
  \and
  \;\; | \;\; \Sigma_{j} \Sigma_{j'} \quotep{\stackrel{\vee}{x_{j}}|\stackrel{\vee}{y_{j'}}}.(A_{j}\otimes B_{j'}) \; | \; \Sigma_{j'} \Sigma_{j} \quotep{\stackrel{\vee}{y_{j'}}|\stackrel{\vee}{x_{j}}}.(B_{j'}\otimes A_{j})
\end{mathpar}

\begin{remark}
  Do we need to $x^{L}$ and $y^{R}$ for this construction as well?
\end{remark}

\paragraph{Tensor of parallel compositions} Next, we distribute tensor
over par.

\begin{mathpar}
  P_{1}|P_{2} \otimes Q_{1}|Q_{2} := (P_{1} \otimes Q_{1}) | (P_{1}
  \otimes Q_{2}) | (P_{2} \otimes Q_{1}) | (P_{2} \otimes Q_{2})
\end{mathpar}

\paragraph{Tensor with dropped names} We treat tensor of a
process with a dropped name as parallel composition.

\begin{mathpar}
  P \otimes \dropn{x} := P | \dropn{x}
\end{mathpar}

\paragraph{Tensor of agents}

Finally, we need to define tensor on agents. Note that the definition
of tensor on normal products only tensors inputs with inputs and
outputs with outputs. Thus, we only have to define the operation on
``homogeneous'' pairings.

\begin{mathpar}
  (\vec{x})P \otimes (\vec{y})Q
  \and \\
  := (x_{0}^{L}|y_{0}^{R},\ldots,x_{0}^{L}|y_{n}^{R},\ldots,x_{m}^{L}|y_{0}^{R},\ldots,x_{m}^{L}|y_{n}^R)(P\{ \vec{x}^{L}/\vec{x}\} \otimes Q \{ \vec{y}^{R}/\vec{y}\})
  \and \\
  \clift{\vec{P}} \otimes \clift{\vec{Q}}
  \and \\
  := \clift{P_{0}\otimes Q_{0},\ldots,P_{0}\otimes Q_{n},\ldots,P_{m}\otimes Q_{0},\ldots,P_{m}\otimes Q_{n}}
\end{mathpar}

\begin{remark}
  Observe that arities of tensored abstractions matches arities of
  tensored concretions if the original arities matched. Note also that
  the length of the arities corresponds to the increase in dimension
  we see in ordinary vector space tensor product.
\end{remark}

\begin{remark}
  Operationally, this definition distributes the tensor down to
  components ``linked'' by summation. Tensor over summation is
  intriguing in that it mixes names. Moreover, as a consequence of the
  way it mixes names we have the identities for all $x \in \QProc$ and
  $P,Q \in \Proc$

  \begin{mathpar}
    (x \cdot P) \otimes Q \equiv x \cdot (P \otimes Q) \equiv P \otimes (x \cdot Q)
    \and
    P \otimes \pzero \equiv P
  \end{mathpar}

  that the reader is invited to verify.
\end{remark}

\subsubsection{Annihilation}
\begin{mathpar}
  P^{\perp} := \{ Q | \forall R. P|Q \red^{*} R \Rightarrow R \red^{*} \pzero \}
  \and \\
  P^{\underline{\perp}} := \Sigma_{Q \in P^{\perp}} \quotep{Q}?(y).(\dropn{y}|Q) | \Sigma_{Q \in P^{\perp}} \quotep{Q}\clift{\Box}
\end{mathpar}

\paragraph{Discussion} The reader will note that $P^{\perp}$ is a
\emph{set} of processes, while $P^{\underline{\perp}}$ is a
\emph{context}. We call the set $P^{\perp}$ the \emph{annihilators} of
$P$. The parallel composition of a process in the annihilators of $P$
with $P$ will result in a process, the state space of which has all
paths eventually leading to $\pzero$. Execution may endure loops; but
under reasonable conditions of fairness (naturally guaranteed under
most notions of bisimulation) such a composite process cannot get
stuck in such a loop and will, eventually pop out and terminate.

The context $P^{\underline{\perp}}$ is ready and willing to ``take the
$P$ out of'' the process to which it is applied. It will effectively
transmit the code of the process to which it is applied to one of the
annihilators and run the process against it.

\subsubsection{Evaluation}
We fix $M$ a domain of fully abstract interpretation with an equality
coincident with bisimulation. We take $\meaningof{\cdot} : \Proc \to
M$ to be the map interpreting processes and $\nmeaningof{\cdot} : \M
\to Proc$ to be the map running the other way. Then we define

\begin{mathpar}
  \int P := \nmeaningof{\meaningof{P}}
\end{mathpar}

\paragraph{Discussion}
There are many fully abstract interpretations of Milner's
$\pi$-calculus. Any of them can be used as a basis for interpreting
the reflective calculus here. Equipped with such a domain it is
largely a matter of grinding through to check that the Yoneda
construction for the normalization-by-evaluation program can be
extended to this setting.

\begin{remark}
  The reader is invited to verify that $\int (P^{\underline{\perp}}[P]) = 0$.
\end{remark}

\subsection{Quantum mechanics}

Table \ref{tbl:core_qm_op_defns} gives the core operational definitions

\begin{table}[htp]\label{tbl:core_qm_op_defns}
  \center{
    \fbox{
      \begin{tabular}{c|c}
        quantum mechanics & process calculus \\
        \hline
        scalar & $x := \quotep{P}$ \\
        state vector & $\state{P} := P$ \\
        dual & $\state{P}^{*} := \event{P^{\underline{\perp}}} := \quotep{P^{\underline{\perp}}}[-]$ \\
        matrix & $ \Sigma_{\alpha} \state{P_{\alpha}}x_{\alpha}\event{Q_{\alpha}}$ \\
        vector addition & $\state{P} + \state{Q} := \state{P | Q}$ \\
        tensor product & $\state{P} \otimes \state{Q} := \state{P \otimes Q}$ \\
        inner product & $\innerprod{P}{Q} := \quotep{\int P^{\underline{\perp}}[Q]}$ \\
      \end{tabular}
    }
  }
  \caption{QM - operational definitions}
\end{table}

where

\begin{mathpar}
  \prmatrix{P}{Q} := \fprmatrix{P}{\quotep{\pzero}}{Q}
  \and
  \fprmatrix{P}{x}{Q} := (\state{P},x,\event{Q})
  \and
  (\fprmatrix{P}{x}{Q})(\state{R}) := x \cdot \innerprod{Q}{R} \cdot \state{P}
  \and
  (\fprmatrix{P}{x}{Q})(\event{R}) := x \cdot \innerprod{R}{P} \cdot \event{Q}
\end{mathpar}

\paragraph{Discussion}
As promised: vectors (aka states) are represented as processes; duals
as contextual duals; inner product definition should be compared with
standard inner product definition for ....

\begin{remark}
  Assuming $\int (P^{\underline{\perp}}[P]) = 0$, the reader is
  invited to verify that $(\fprmatrix{P}{x}{P})(\state{P}) = x \cdot \state{P}$.
\end{remark}

\begin{remark}
  The reader is invited to verify that $\innerprod{P}{Q}$ could
  equally well have been written $\quotep{\int \stackrel{\vee}{x}}$
  where $x = \event{P^{\underline{\perp}}}(Q)$.

  One of the motivations for this remark is that there is another way
  to factor these operations. We could package up evaluation in the dual:

  \begin{mathpar}
    \state{P}^{*} := \event{\int P^{\underline{\perp}}} := \quotep{\int P^{\underline{\perp}}}[-]
  \end{mathpar}

  and then have inner product defined by
  
  \begin{mathpar}
    \innerprod{P}{Q} := \event{P}(Q)
  \end{mathpar}

  Hopefully, experience with the calculations will provide guidance on
  the best factoring.
\end{remark}

\begin{remark}
  Assuming $\int (P^{\underline{\perp}}[P]) = 0$, the reader is
  invited to verify that $\forall P,Q. (\prmatrix{0}{Q})(\state{0}) =
  \state{0}$ and dually $(\prmatrix{P}{0})(\event{0}) = \event{0}$.
\end{remark}

\begin{remark}
  i'm a little worried that i don't (yet) have proper support for
  complex conjugacy. But, the observation above may give us a
  clue. According to Abramsky, it must be the case that the scalars
  are iso to the homset of the identity for the tensor -- which the
  observation above characterizes. 

  For now, we will simply bookmark the notion with $\overline{x}$.
\end{remark}

\subsubsection{Adjointness}

We need to give a definition of $(\cdot)^{\dagger}$ for matrices. The
obvious candidate definition is
\begin{mathpar}
(\Sigma_{\alpha}\fprmatrix{P_{\alpha}}{x_{\alpha}}{Q_{\alpha}})^{\dagger}
= \Sigma_{\alpha}\fprmatrix{(Q_{\alpha}^{\underline{\perp}})^{*}}{\overline{x}_{\alpha}}{P_{\alpha}^{\underline{\perp}}} 
\end{mathpar}

But, $(Q_{\alpha}^{\underline{\perp}})^{*}$ requires a name along
which to communicate the process to achieve the context application.

\subsubsection{Basis for a basis}
If processes label states and ``addition'' of states (a.k.a. vector
addition) is interpreted as parallel composition, what corresponds to
notions of linear independence and basis? Here, we recall that Yoshida
has developed a set of \emph{combinators} for an asynchronous verison
of Milner's $\pi$-calculus. These are a finite set of processes such
any process can be expressed as parallel composition of these
combinators together with liberal uses of the new operator and
replication. We can simply give a translation of these into the
present calculus and have reasonable expectation that the property
carries over. That is, that the resultant set allows to express all
processes via parallel composition. Note, however, that there is no
new operator or replication in this calculus. As a result, we expect
that the corresponding set is actually infinite. That is, we expect
that the space is actually infinite dimensional.

\begin{remark}
  The attentive reader may be a bit concerned. Certainly, the
  collection $S$, $K$ and $I$ is a finite set of
  combinators. Shouldn't we expect to see a finite set of combinators
  for an effectively equivalent system? i am very sympathetic to this
  critique and feel it warrants full attention. On the other hand, i
  also have in mind the following analogy. The natural numbers, as a
  monoid under addition, has exactly $1$ generator, while the natural
  numbers, as a monoid under multiplication, has countably many
  generators (the primes). We observe that the application of the
  lambda calculus is much less resource sensitive than the parallel
  composition of the $\pi$-calculus. Could it be the case that we have
  an analogy of the form
  
  \begin{mathpar}
    m + n : MN :: m*n : M|N
  \end{mathpar}

  giving a similar blow up in the set of ``primes''?  This is such a
  wonderful thought that, even if it's not true, i think it's worth
  writing down.
\end{remark}
 

\documentclass[12pt]{llncs}
%\documentclass{jktr}

\usepackage[pdftex]{hyperref}                   
\usepackage {listings}
\usepackage {mathpartir}
\usepackage{bcprules}
%\usepackage{listings}
                       
\usepackage{graphicx} 
%\usepackage[margins=2.5cm,nohead,nofoot]{geometry}
%\usepackage{geometry}
\usepackage{amsfonts}
\usepackage{amstext}
\usepackage{latexsym}
\usepackage{amssymb}
\usepackage{color}


%\include{myPreamble}
\documentclass[12pt]{llncs}
%\documentclass{jktr}

\usepackage[pdftex]{hyperref}                   
\usepackage {listings}
\usepackage {mathpartir}
\usepackage{bcprules}
%\usepackage{listings}
                       
\usepackage{graphicx} 
%\usepackage[margins=2.5cm,nohead,nofoot]{geometry}
%\usepackage{geometry}
\usepackage{amsfonts}
\usepackage{amstext}
\usepackage{latexsym}
\usepackage{amssymb}
\usepackage{color}


%\include{myPreamble}
\include{qm2pi.local} 

%\ifpdf
%\usepackage[pdftex]{graphicx}
%\else
%\usepackage{graphicx}
%\fi

 % \ifpdf
%  \usepackage{pdfsync}
%  \if


%\title{Brief Article}
%\author{David F. Snyder}
%\author{L.G. Meredith}

%\address{Dept. of Math., Texas State University--San Marcos, San Marcos, TX 78666}
       
\pagestyle{empty}


\begin{document}

\lstset{language=[Objective]Caml,frame=shadowbox}

\input{qm2pi.front}

% section front matter (end)

\input{qm2pi.intro} 
 
% section introduction (end)

% \input{qm2pi.knotations} 

% section notation (end)

\input{qm2pi.process.calculi} 

% section concurrent_process_calculi_and_spatial_logics_ (end)
    
%\input{qm2pi.knots2pi} 

%\input{qm2pi.trefoil} 

%\input{qm2pi.mainthm} 

% subsection basic_interpretation (end)

%\input{qm2pi.rho.presentation} 
\subsection{The syntax and semantics of the notation system}\label{sub:the_syntax_and_semantics_of_the_notation_system} % (fold)

We now summarize a technical presentation of the calculus that
embodies our theory of dynamics. The typical presentation of such a
calculus follows the style of giving generators and relations on
them. The grammar, below, describing term constructors, freely
generates the set of processes, $\Proc$. This set is then quotiented
by a relation known as structural congruence and it is over this set
that the notion of dynamics is expressed. This presentation is
essentially that of \cite{MeredithR05} with the addition of
polyadicity and summation. For readability we have relegated some of
the technical subtleties to an appendix.

\subsubsection{Process grammar}\label{subsub:process_grammar}

\begin{mathpar}
  \inferrule* [lab=synchronization] {} {{M} \bc \pzero \;|\; x?F \;|\; x!C }
  \and
  \inferrule* [lab=abstraction] {} {{F} \bc (x)P}
  \and
  \inferrule* [lab=concretion] {} {{C} \bc \langle Q \rangle}
  \and
  \inferrule* [lab=process] {} {{P,Q} \bc M \;| \;P|Q \;|\; @{x}}
  \and
  \inferrule* [lab=name] {} {{x} \bc \quotep{P}}
\end{mathpar} 

Note that $\vec{x}$ (resp. $\vec{P}$) denotes a vector of names
(resp. processes) of length $|\vec{x}|$ (resp. $|\vec{P}|$). We adopt
the following useful abbreviations.

\begin{mathpar}
   x?(\vec{y}).P := x.(\vec{y})P \and  x\clift{\vec{P}} := x.\clift{\vec{P}}
   \and x!(y) := \lift{x}{\dropn{y}}
   \and \Pi_{i=0}^{n-1}P_i := P_0 | \ldots | P_{n-1}
\end{mathpar}

\subsubsection{Structural congruence}

\paragraph{Free and bound names and alpha-equivalence.} At the
core of structural equivalence is alpha-equivalence which identifies
process that are the same up to a change of variable. Formally, we
recognize the distinction between free and bound names. The free names
of a process, $\freenames{P}$, may be calculated recursively as
follows:

\begin{mathpar}
\freenames{\pzero} := \emptyset
  \and \\
  \freenames{x?(y).P} := \{ x \} \cup (\freenames{P} \setminus \{ y \})
  \and 
  \freenames{x!\langle P \rangle} := \{ x \} \cup \{ P \} 
  \and \\
  \freenames{P|Q} := \freenames{P} \cup \freenames{Q}
  \and \\
  \freenames{@{x}} := \{ x \}
\end{mathpar}

$\pi$
$\quotep{\pi}$

$\freenames{-} : \pi \to \mathcal{P}(\quotep{\pi})$

\begin{eqnarray*}
  \freenames{\pzero} & := & \emptyset \\
  \freenames{x?(y).P} & := & \{ x \} \cup (\freenames{P} \setminus \{ y \}) \\
  \freenames{x!\langle P \rangle} & := & \{ x \} \cup \{ P \} \\
  \freenames{P|Q} & := & \freenames{P} \cup \freenames{Q} \\
  \freenames{\dropn{x}} & := & \{ x \}
\end{eqnarray*}

The bound names of a process, $\boundnames{P}$, are those names occurring in $P$
that are not free. For example, in $x?(y).0$, the name $x$ is free, while $y$ is bound.

\begin{mathpar}
  \inferrule* [lab=monoidal-laws] {} { P|Q \equiv Q|P \and P|0 \equiv P \and P|(Q|R) \equiv (P|Q)|R }
\end{mathpar}

\begin{mathpar}
  \inferrule* [lab=alpha-equivalence] {} { (x)P \equiv (y)P\{y/x\} \and y \not\in \freenames{P} }
\end{mathpar}

\begin{definition}
Then two processes, $P,Q$, are alpha-equivalent if $P = Q\{\vec{y}/\vec{x}\}$ for
some $\vec{x} \in \boundnames{Q},\vec{y} \in \boundnames{P}$, where $Q\{\vec{y}/\vec{x}\}$
denotes the capture-avoiding substitution of $\vec{y}$ for $\vec{x}$ in $Q$.
\end{definition}

\begin{definition}
  The {\em structural congruence} \cite{SangiorgiWalker} , $\equiv$,
  between processes is the least congruence containing
  alpha-equivalence, satisfying the abelian monoid laws
  (associativity, commutativity and $\pzero$ as identity) for parallel
  composition $|$ and for summation $+$.
\end{definition}

\subsection{Name equivalence}

We take name equivalence, written $\nameeq$, to be the smallest
equivalence relation generated by the following rules.

\begin{mathpar}
\inferrule*[lab=Quote-drop]
{ }
{ \quotep{@{x}} \nameeq x }

\inferrule*[lab=Struct-equiv]
{ P \scong Q }
{ \quotep{P} \nameeq \quotep{Q} }
\end{mathpar}

The astute reader will have noticed that the mutual recursion of names
and processes imposes a mutual recursion on alpha-equivalence and
structural equivalence via name-equivalence. Fortunately, all of this
works out pleasantly and we may calculate in the natural way, free of
concern. The reader interested in the details is referred to the
appendix \ref{appendix:rho_details}.

\subsection{Substitution}

We use $\Proc$ for the set of processes, $\QProc$ for the set of
names, and $\id{\{}\vec{y} / \vec{x} \id{\}}$ to denote partial maps,
$s : \QProc \rightarrow \QProc$. A map, $s$ lifts, uniquely, to a map
on process terms, $\widehat{s} : \Proc \rightarrow \Proc$ by the
following equations.

\begin{mathpar}
  (0) \psubstp{Q}{P} := 0 \\
  (R \juxtap S) \psubstp{Q}{P}
  :=    
  (R)\psubstp{Q}{P} \juxtap (S) \psubstp{Q}{P} \\
  (x?(y).R) \psubstp{Q}{P}    
  :=    
  (x)\substp{Q}{P} (z)\concat( (R \psubstn{z}{y}) \psubstp{Q}{P} ) \\
  (\lift{x}{R}) \psubstp{Q}{P}  
  :=
  \lift{(x)\substp{Q}{P}}{ R \psubstp{Q}{P} } \\
%   (\dropn{x})  \psubstp{Q}{P}       
%   := 
%   \left\{ 
%     \begin{array}{ccc} 
%       \dropn{\quotep{Q}} & & x \nameeq \quotep{P} \\
%       \dropn{x} & & otherwise \\
%     \end{array}
%   \right. 
  (\dropn{x})  \psubstp{Q}{P}       
  := 
  \left\{ 
    \begin{array}{ccc} 
      Q & & x \nameeq \quotep{P} \\
      \dropn{x} & & otherwise \\
    \end{array}
  \right.
\end{mathpar}
 

where

\begin{eqnarray}
  (x)\id{\{} \lpquote Q \rpquote / \lpquote P \rpquote \id{\}}            = 
  \left\{ 
    \begin{array}{ccc}
      \lpquote Q \rpquote & & x \nameeq \lpquote P \rpquote \\
      x & & otherwise \\
    \end{array}
  \right. \nonumber
\end{eqnarray}

and $z$ is chosen distinct from $\quotep{P}$, $\quotep{Q}$, the free
names in $Q$, and all the names in $R$. Our $\alpha$-equivalence will
be built in the standard way from this substitution.

\begin{remark}\label{rem:no_self_referential_names}
  One consequence of these definitions is that $\forall P. \quotep{P}
  \not\in \freenames{P}$.
\end{remark}

\subsection{ Dynamic quote: an example }

Anticipating something of what's to come, consider applying the
substitution, $\widehat{\id{\{}u / z \id{\}}}$, to the following pair
of processes, $\lift{w}{y!(z)}$ and $w[ \lpquote y!(z) \rpquote ]$.

\begin{eqnarray}
	\lift{w}{y!(z)}\widehat{\id{\{}u / z \id{\}}}
		& = &
		\lift{w}{y!(u)} \nonumber\\
	w[ \lpquote y!(z) \rpquote ] \widehat{ \id{\{}u / z \id{\}} }
		& = &
		w[ \lpquote y!(z) \rpquote ] \nonumber
\end{eqnarray}

Because the body of the process between quotes is impervious to
substitution, we get radically different answers. In fact, by
examining the first process in an input context,
e.g. $x?(z).\lift{w}{y!(z)}$, we see that the process under the lift
operator may be shaped by prefixed inputs binding a name inside it. In
this sense, the lift operator will be seen as a way to dynamically
construct processes before reifying them as names.

Finally equipped with these standard features we can present the
dynamics of the calculus.

\subsubsection{Operational semantics} 

Finally, we introduce the computational dynamics. What marks these
algebras as distinct from other more traditionally studied algebraic
structures, e.g. vector spaces or polynomial rings, is the manner in
which dynamics is captured. In traditional structures, dynamics is typically
expressed through morphisms between such structures, as in linear maps
between vector spaces or morphisms between rings. In algebras
associated with the semantics of computation, the dynamics is
expressed as part of the algebraic structure itself, through a
reduction reduction relation typically denoted by $\red$. Below, we
give a recursive presentation of this relation for the calculus used
in the encoding.

$\red \subseteq \pi \times \pi$
$\red : \pi \to \mathcal{P}(\pi)$

\begin{mathpar}
  \inferrule* [lab=Comm] { \textsf{match}( x_{src}, x_{trgt} ) } { x_{trgt}?(y)P \; | \; x_{src}!\langle {Q} \rangle \red P\{\quotep{Q}/y}\} }
  \and \\
  \inferrule* [lab=Par] {{P} \red {P}'} {{{P} | {Q}} \red {{P}' | {Q}}}
  \and
  \inferrule* [lab=Equiv]{{{P} \scong {P}'} \andalso {{P}' \red {Q}'} \andalso {{Q}' \scong {Q}}}{{P} \red {Q}}
\end{mathpar}

\begin{eqnarray*}
  match_{\equiv} (\quotep{P},\quotep{Q}) & := & P \equiv Q \\
  match_{\dagger}(\quotep{P},\quotep{Q}) & := & \forall R. P|Q \red^{*} R => R \red^{*} 0 \\
  match_{K}(\quotep{P},\quotep{Q}) & := & K \mbox{ for some context } K
\end{eqnarray*}

$u?(x)P | u!\langle Q \rangle \red P\{\quotep{Q}/x\}$

%We write $\wred$ for $\red^*$, and $P\red$ if $\exists Q $ such that $ P \red Q$.
We write $P\red$ if $\exists Q $ such that $ P \red Q$ and $P\not\red$, otherwise.

\section{Replication}

As mentioned before, it is known that replication (and hence
recursion) can be implemented in a higher-order process algebra
\cite{SangiorgiWalker}. As our first example of calculation with the
machinery thus far presented we give the construction explicitly in
the {\rhoc}.

\begin{eqnarray}
	D_{x} & := & \prefix{x}{y}{(\binpar{\outputp{x}{y}}{@{y}})} \nonumber\\
	\bangp_{x}{P} & := & \binpar{{x}!\langle{\binpar{D_{x}}{P}}\rangle}{D_{x}} \nonumber
\end{eqnarray}

\begin{eqnarray}
	\bangp_{x}{P} & & \nonumber\\
	=
	& {x}!\langle{(\prefix{x}{y}{(\outputp{x}{y} | @{y})) | P}}\rangle 
	      | \prefix{x}{y}{(\outputp{x}{y} | @{y})} & \nonumber\\
	\red
	& (\outputp{x}{y} | @{y})\substn{\quotep{(\prefix{x}{y}{(@{y} | \outputp{x}{y})) | P}}}{y} & \nonumber\\
	=
	& \outputp{x}{\quotep{(\prefix{x}{y}{(\outputp{x}{y} | @{y})) | P}}}
	  | {(\prefix{x}{y}{(\outputp{x}{y} | @{y})) | P}} & \nonumber\\
	\red
	& \ldots & \nonumber\\
	\red^*
	& P | P | \ldots & \nonumber
\end{eqnarray}

Of course, this encoding, as an implementation, runs away, unfolding
$\bangp{P}$ eagerly. A lazier and more implementable replication
operator, restricted to input-guarded processes, may be obtained as follows.

\begin{eqnarray}
\bangp{\prefix{u}{v}{P}} 
	:= 
	\binpar{\lift{x}{\prefix{u}{v}{(\binpar{D(x)}{P})}}}{D(x)} \nonumber
\end{eqnarray}

\begin{remark}
  Note that the lazier definition still does not deal with summation
  or mixed summation (i.e. sums over input and output). The reader is
  invited to construct definitions of replication that deal with these
  features. 

  Further, the definitions are parameterized in a name, $x$. Can you,
  gentle reader, make a definition that eliminates this parameter and
  guarantees no accidental interaction between the replication
  machinery and the process being replicated -- i.e. no accidental
  sharing of names used by the process to get its work done and the
  name(s) used by the replication to effect copying. This latter
  revision of the definition of replication is crucial to obtaining
  the expected identity $!!P \sim !P$.
\end{remark}

\begin{remark}\label{rem:paradoxical_combinator}
  The reader familiar with the lambda calculus will have noticed the
  similarity between $D$ and the paradoxical combinator.

  [Ed. note: the existence of this seems to suggest we have to be more
  restrictive on the set of processes and names we admit if we are to
  support no-cloning.]
\end{remark}

\subsubsection{Bisimulation}

The computational dynamics gives rise to another kind of equivalence,
the equivalence of computational behavior. As previously mentioned
this is typically captured \emph{via} some form of bisimulation.

% The notion we use in this paper is weak barbed bisimulation
% \cite{milner91polyadicpi}.

The notion we use in this paper is derived from weak barbed
bisimulation \cite{milner91polyadicpi}. 

\begin{definition}
An \emph{observation relation}, $\downarrow_{\mathcal N}$, over a set
of names, $\mathcal N$, is the smallest relation satisfying the rules
below.

\infrule[Out-barb]{y \in {\mathcal N}, \; x \nameeq y}
		  {\outputp{x}{v} \downarrow_{\mathcal N} x}
\infrule[Par-barb]{\mbox{$P\downarrow_{\mathcal N} x$ or $Q\downarrow_{\mathcal N} x$}}
		  {\binpar{P}{Q} \downarrow_{\mathcal N} x}

We write $P \Downarrow_{\mathcal N} x$ if there is $Q$ such that 
$P \wred Q$ and $Q \downarrow_{\mathcal N} x$.
\end{definition}

\begin{definition}
%\label{def.bbisim}
An  ${\mathcal N}$-\emph{barbed bisimulation} over a set of names, ${\mathcal N}$, is a symmetric binary relation 
${\mathcal S}_{\mathcal N}$ between agents such that $P\rel{S}_{\mathcal N}Q$ implies:
\begin{enumerate}
\item If $P \red P'$ then $Q \wred Q'$ and $P'\rel{S}_{\mathcal N} Q'$.
\item If $P\downarrow_{\mathcal N} x$, then $Q\Downarrow_{\mathcal N} x$.
\end{enumerate}
$P$ is ${\mathcal N}$-barbed bisimilar to $Q$, written
$P \wbbisim_{\mathcal N} Q$, if $P \rel{S}_{\mathcal N} Q$ for some ${\mathcal N}$-barbed bisimulation ${\mathcal S}_{\mathcal N}$.
\end{definition}

$\mathcal{R} \subseteq \pi \times \pi$

$P \mathcal{R} Q => \forall P'. P \red P' \Rightarrow \exists Q'. Q \red Q', P' \mathcal{R} Q'$

$P \vdash x \Rightarrow Q \vdash x$

\begin{mathpar}
  \inferrule*[lab=Out-barb]{x \nameeq y}{{y}!\langle{Q}\rangle \vdash x}
  \and
  \inferrule*[lab=Par-barb]{\mbox{$P\vdash x$ or $Q\vdash x$}}{\binpar{P}{Q} \vdash x}
\end{mathpar}

\subsubsection{Contexts}

One of the principle advantages of computational calculi like the
$\pi$-calculus is a well-defined notion of context,
contextual-equivalence and a correlation between
contextual-equivalence and notions of bisimulation. The notion of
context allows the decomposition of a process into (sub-)process and
its syntactic environment, its context. Thus, a context may be
thought of as a process with a ``hole'' (written $\Box$) in it. The
application of a context $M$ to a process $P$, written $M[P]$, is
tantamount to filling the hole in $M$ with $P$. In this paper we do
not need the full weight of this theory, but do make use of the notion
of context in the proof the main theorem. 

\begin{mathpar}
  \inferrule* [lab=summation] {} {{M_{M},M_{N}} \bc \Box \;|\; x.M_{A} \;|\; M_{M}+M_{N}}
  \and
  \inferrule* [lab=agent] {} {{M_{A}} \bc (\vec{x})M_{P} \;| \; \clift{P_0,\ldots,M_{P},\ldots,P_N}}
  \and \\
  \inferrule* [lab=process] {} {{M_{P}} \bc M_{N} \;| \;P|M_{P} }
\end{mathpar} 

\begin{mathpar}
  \inferrule* [lab=sychronization] {} {M_{N} \bc \Box \;|\; x?M_{F} \;|\; x!M_{C}}
  \and
  \inferrule* [lab=abstraction] {} {{M_{F}} \bc (x)M_{P} }
  \and
  \inferrule* [lab=concretion] {} {{M_{C}} \bc \langle M_{P} \rangle }
  \and \\
  \inferrule* [lab=process] {} {{M_{P}} \bc M_{N} \;| \;P|M_{P} }
\end{mathpar}

\begin{definition}[contextual application] Given a context $M$, and
  process $P$, we define the \emph{contextual application}, $M[P] :=
  M\{P/\Box\}$. That is, the contextual application of M to P is the
  substitution of $P$ for $\Box$ in $M$.
\end{definition}

$\meaningof{-} : L \to \mathcal{P}(\pi)$

\begin{mathpar}
  \inferrule* [lab=collection] {} {\meaningof{true} = \pi, \and \meaningof{~E} = \pi \setminus \meaningof{E}, \and \meaningof{E_{1} \& E_{2}} = \meaningof{E_{1}} \cap \meaningof{E_{2}}}
\end{mathpar}

\begin{mathpar}
  \inferrule* [lab=structure] {} {\meaningof{0} = \{ P \in \pi | P \equiv 0 \}, \and \\ \meaningof{E_1 | E_2} = \{ P \in \pi | P \equiv P_{1} | P_{2}, P_{1} \in \meaningof{E_{1}}, P_{2} \in \meaningof{E_2}\} }
\end{mathpar}

\begin{mathpar}
 \inferrule* [lab=behavior] {} {\meaningof{\langle a?b \rangle E} = \{ P \in \pi | P \equiv Q | u?(y)P', \\ \and \\\\ \and \\ \;\;\; u \in \meaningof{a}, \forall z.P'\{z/y\} \in \meaningof{E\{z/b\}}\}, \and \\ \meaningof{a!E} = \{ P \in \pi | P \equiv Q | x!\langle P' \rangle, x \in \meaningof{a} P' \in \meaningof{E}\} }
\end{mathpar}

\begin{mathpar}
 \inferrule* [lab=nominal] {} {\meaningof{\quotep{E}} = \{ \quotep{P} \in \quotep{\pi} | P \in \meaningof{E} \}, \and \meaningof{\quotep{P}} = \{ \quotep{Q} \in \quotep{\pi} | P \equiv Q \} \and \\ \meaningof{@\quotep{E}} = \{ P \in \pi | P \equiv @x, x \in \meaningof{E} \}}
\end{mathpar}

\begin{eqnarray*}
  \\
  \meaningof{-} : TS \to ST
\end{eqnarray*}

\begin{eqnarray*}
  \\
  L : TS \to ST
\end{eqnarray*}

\begin{eqnarray*}
  \\
  P \models E \iff P \in \meaningof{E}
\end{eqnarray*}

\begin{eqnarray*}
  P \approx_{L} Q \iff \forall E \in L. P \models E \iff Q \models E
\end{eqnarray*}

\begin{eqnarray*}
  P \approx_{K} Q
\end{eqnarray*}

\begin{eqnarray*}
  P \approx Q
\end{eqnarray*}

$\approx_{K} = \approx = \approx_{L}$

\subsubsection{Contextual duality}

Note that contexts extend the quotation operation to a family of
operations from processes to names. Given a context, $M$, we can
define a \emph{nominal context}, $\quotep{M}$ by $\quotep{M}[P] :=
\quotep{M[P]}$. To foreshadow what is to come we observe that these
operations enjoy a duality with processes very much like the duality
between vectors and maps from vectors to scalars.

Further, because the calculus is essentially higher-order, we have a
correspondence between contexts and processes. More specifically,
given a name $x$ and a context $M$ we can construct $M^{*}_{x}$ such
that 

\begin{mathpar}
  M^{*}_{x} | \lift{x}{P} \red M[P]
\end{mathpar}

namely,

\begin{mathpar}
  M^{*}_{x} := x?(u).M[\dropn{u}]
\end{mathpar}

The dependence of $M^{*}_{x}$ on a name makes it an abstraction, 

\begin{mathpar}
  M^{*} := (x)x?(u).M[\dropn{u}]
\end{mathpar}

\subsection{Additional notation}

It will sometimes be convenient to denote the process a name
quotes. We already have the notation $x = \quotep{P}$, but it will be
convenient to introduce an alternate notation, $\procn{x}$, when we
want to emphasize the connection to the use of the name. Note that, by
virtue of name equivalence, $\quotep{\procn{x}} \nameeq x$; so, the
notation is consistent with previous definitions.

Further, because names have structure it is possible to effect
substitutions on the basis of that structure. This means we need to
upgrade our notation for substitutions, which we accomplish by
adapting comprehension notation. Thus,

\begin{mathpar}
  P\{ y / x : x \in S \}
\end{mathpar}

is interpreted to mean the process derived from P by replacing (in a
capture-avoiding manner) each occurrence of $x$ in $S$ by $y$. For example,

\begin{mathpar}
  P\{ \quotep{\procn{x}|\procn{x}} / x : x \in \freenames{P} \}
\end{mathpar}

will replace each (occurrence) of a free name $x$ in $P$ by
$\quotep{\procn{x}|\procn{x}}$.

Also, we will avail ourselves of the notation $x^{L}$ and $x^{R}$ to
denote injections of a name into disjoint copies of the name
space. There are numerous ways to accomplish this. One example can be
found in \cite{MeredithR05}. This notation overloads to vectors of
names: $\vec{x}^{\pi} := (x_{i}^{\pi} \; : \; 0 \leq i < |\vec{x}| )$ where $\pi \in \{L,R\}$.

We also use $P^{\Box} := P|\Box$.

In \cite{MeredithR05} an interpretation of the new operator is
given. It turns out that there are several possible interpretations
all enjoying the requisite algebraic properties of the operator (see
\cite{milner91polyadicpi}). We will therefore make liberal use of
$(\nu\; \vec{x})P$.

% subsection the_syntax_and_semantics_of_the_notation_system (end)   

\input{qm2pi.qmops} 

\input{qm2pi.sterngerlach} 

\input{qm2pi.metric} 

% section concurrent_process_calculi (end)

%\input{qm2pi.proofsketch}

% section proof sketch (end)

%\input{qm2pi.slviaknots} 

% section spatial logic via knots (end)

\input{qm2pi.conclusion}

% section conclusion (end)

%\input{qm2pi.dtcodes} 

% section wiring algorithm (end)

\input{qm2pi.ack} 

% section acknowledgments (end)

\newpage


\bibliographystyle{plain}   
\bibliography{../../biblios/main.bib}

\input{qm2pi.rhodetails}

\end{document}

 

%\ifpdf
%\usepackage[pdftex]{graphicx}
%\else
%\usepackage{graphicx}
%\fi

 % \ifpdf
%  \usepackage{pdfsync}
%  \if


%\title{Brief Article}
%\author{David F. Snyder}
%\author{L.G. Meredith}

%\address{Dept. of Math., Texas State University--San Marcos, San Marcos, TX 78666}
       
\pagestyle{empty}


\begin{document}

\lstset{language=[Objective]Caml,frame=shadowbox}

\documentclass[12pt]{llncs}
%\documentclass{jktr}

\usepackage[pdftex]{hyperref}                   
\usepackage {listings}
\usepackage {mathpartir}
\usepackage{bcprules}
%\usepackage{listings}
                       
\usepackage{graphicx} 
%\usepackage[margins=2.5cm,nohead,nofoot]{geometry}
%\usepackage{geometry}
\usepackage{amsfonts}
\usepackage{amstext}
\usepackage{latexsym}
\usepackage{amssymb}
\usepackage{color}


%\include{myPreamble}
\include{qm2pi.local} 

%\ifpdf
%\usepackage[pdftex]{graphicx}
%\else
%\usepackage{graphicx}
%\fi

 % \ifpdf
%  \usepackage{pdfsync}
%  \if


%\title{Brief Article}
%\author{David F. Snyder}
%\author{L.G. Meredith}

%\address{Dept. of Math., Texas State University--San Marcos, San Marcos, TX 78666}
       
\pagestyle{empty}


\begin{document}

\lstset{language=[Objective]Caml,frame=shadowbox}

\input{qm2pi.front}

% section front matter (end)

\input{qm2pi.intro} 
 
% section introduction (end)

% \input{qm2pi.knotations} 

% section notation (end)

\input{qm2pi.process.calculi} 

% section concurrent_process_calculi_and_spatial_logics_ (end)
    
%\input{qm2pi.knots2pi} 

%\input{qm2pi.trefoil} 

%\input{qm2pi.mainthm} 

% subsection basic_interpretation (end)

%\input{qm2pi.rho.presentation} 
\subsection{The syntax and semantics of the notation system}\label{sub:the_syntax_and_semantics_of_the_notation_system} % (fold)

We now summarize a technical presentation of the calculus that
embodies our theory of dynamics. The typical presentation of such a
calculus follows the style of giving generators and relations on
them. The grammar, below, describing term constructors, freely
generates the set of processes, $\Proc$. This set is then quotiented
by a relation known as structural congruence and it is over this set
that the notion of dynamics is expressed. This presentation is
essentially that of \cite{MeredithR05} with the addition of
polyadicity and summation. For readability we have relegated some of
the technical subtleties to an appendix.

\subsubsection{Process grammar}\label{subsub:process_grammar}

\begin{mathpar}
  \inferrule* [lab=synchronization] {} {{M} \bc \pzero \;|\; x?F \;|\; x!C }
  \and
  \inferrule* [lab=abstraction] {} {{F} \bc (x)P}
  \and
  \inferrule* [lab=concretion] {} {{C} \bc \langle Q \rangle}
  \and
  \inferrule* [lab=process] {} {{P,Q} \bc M \;| \;P|Q \;|\; @{x}}
  \and
  \inferrule* [lab=name] {} {{x} \bc \quotep{P}}
\end{mathpar} 

Note that $\vec{x}$ (resp. $\vec{P}$) denotes a vector of names
(resp. processes) of length $|\vec{x}|$ (resp. $|\vec{P}|$). We adopt
the following useful abbreviations.

\begin{mathpar}
   x?(\vec{y}).P := x.(\vec{y})P \and  x\clift{\vec{P}} := x.\clift{\vec{P}}
   \and x!(y) := \lift{x}{\dropn{y}}
   \and \Pi_{i=0}^{n-1}P_i := P_0 | \ldots | P_{n-1}
\end{mathpar}

\subsubsection{Structural congruence}

\paragraph{Free and bound names and alpha-equivalence.} At the
core of structural equivalence is alpha-equivalence which identifies
process that are the same up to a change of variable. Formally, we
recognize the distinction between free and bound names. The free names
of a process, $\freenames{P}$, may be calculated recursively as
follows:

\begin{mathpar}
\freenames{\pzero} := \emptyset
  \and \\
  \freenames{x?(y).P} := \{ x \} \cup (\freenames{P} \setminus \{ y \})
  \and 
  \freenames{x!\langle P \rangle} := \{ x \} \cup \{ P \} 
  \and \\
  \freenames{P|Q} := \freenames{P} \cup \freenames{Q}
  \and \\
  \freenames{@{x}} := \{ x \}
\end{mathpar}

$\pi$
$\quotep{\pi}$

$\freenames{-} : \pi \to \mathcal{P}(\quotep{\pi})$

\begin{eqnarray*}
  \freenames{\pzero} & := & \emptyset \\
  \freenames{x?(y).P} & := & \{ x \} \cup (\freenames{P} \setminus \{ y \}) \\
  \freenames{x!\langle P \rangle} & := & \{ x \} \cup \{ P \} \\
  \freenames{P|Q} & := & \freenames{P} \cup \freenames{Q} \\
  \freenames{\dropn{x}} & := & \{ x \}
\end{eqnarray*}

The bound names of a process, $\boundnames{P}$, are those names occurring in $P$
that are not free. For example, in $x?(y).0$, the name $x$ is free, while $y$ is bound.

\begin{mathpar}
  \inferrule* [lab=monoidal-laws] {} { P|Q \equiv Q|P \and P|0 \equiv P \and P|(Q|R) \equiv (P|Q)|R }
\end{mathpar}

\begin{mathpar}
  \inferrule* [lab=alpha-equivalence] {} { (x)P \equiv (y)P\{y/x\} \and y \not\in \freenames{P} }
\end{mathpar}

\begin{definition}
Then two processes, $P,Q$, are alpha-equivalent if $P = Q\{\vec{y}/\vec{x}\}$ for
some $\vec{x} \in \boundnames{Q},\vec{y} \in \boundnames{P}$, where $Q\{\vec{y}/\vec{x}\}$
denotes the capture-avoiding substitution of $\vec{y}$ for $\vec{x}$ in $Q$.
\end{definition}

\begin{definition}
  The {\em structural congruence} \cite{SangiorgiWalker} , $\equiv$,
  between processes is the least congruence containing
  alpha-equivalence, satisfying the abelian monoid laws
  (associativity, commutativity and $\pzero$ as identity) for parallel
  composition $|$ and for summation $+$.
\end{definition}

\subsection{Name equivalence}

We take name equivalence, written $\nameeq$, to be the smallest
equivalence relation generated by the following rules.

\begin{mathpar}
\inferrule*[lab=Quote-drop]
{ }
{ \quotep{@{x}} \nameeq x }

\inferrule*[lab=Struct-equiv]
{ P \scong Q }
{ \quotep{P} \nameeq \quotep{Q} }
\end{mathpar}

The astute reader will have noticed that the mutual recursion of names
and processes imposes a mutual recursion on alpha-equivalence and
structural equivalence via name-equivalence. Fortunately, all of this
works out pleasantly and we may calculate in the natural way, free of
concern. The reader interested in the details is referred to the
appendix \ref{appendix:rho_details}.

\subsection{Substitution}

We use $\Proc$ for the set of processes, $\QProc$ for the set of
names, and $\id{\{}\vec{y} / \vec{x} \id{\}}$ to denote partial maps,
$s : \QProc \rightarrow \QProc$. A map, $s$ lifts, uniquely, to a map
on process terms, $\widehat{s} : \Proc \rightarrow \Proc$ by the
following equations.

\begin{mathpar}
  (0) \psubstp{Q}{P} := 0 \\
  (R \juxtap S) \psubstp{Q}{P}
  :=    
  (R)\psubstp{Q}{P} \juxtap (S) \psubstp{Q}{P} \\
  (x?(y).R) \psubstp{Q}{P}    
  :=    
  (x)\substp{Q}{P} (z)\concat( (R \psubstn{z}{y}) \psubstp{Q}{P} ) \\
  (\lift{x}{R}) \psubstp{Q}{P}  
  :=
  \lift{(x)\substp{Q}{P}}{ R \psubstp{Q}{P} } \\
%   (\dropn{x})  \psubstp{Q}{P}       
%   := 
%   \left\{ 
%     \begin{array}{ccc} 
%       \dropn{\quotep{Q}} & & x \nameeq \quotep{P} \\
%       \dropn{x} & & otherwise \\
%     \end{array}
%   \right. 
  (\dropn{x})  \psubstp{Q}{P}       
  := 
  \left\{ 
    \begin{array}{ccc} 
      Q & & x \nameeq \quotep{P} \\
      \dropn{x} & & otherwise \\
    \end{array}
  \right.
\end{mathpar}
 

where

\begin{eqnarray}
  (x)\id{\{} \lpquote Q \rpquote / \lpquote P \rpquote \id{\}}            = 
  \left\{ 
    \begin{array}{ccc}
      \lpquote Q \rpquote & & x \nameeq \lpquote P \rpquote \\
      x & & otherwise \\
    \end{array}
  \right. \nonumber
\end{eqnarray}

and $z$ is chosen distinct from $\quotep{P}$, $\quotep{Q}$, the free
names in $Q$, and all the names in $R$. Our $\alpha$-equivalence will
be built in the standard way from this substitution.

\begin{remark}\label{rem:no_self_referential_names}
  One consequence of these definitions is that $\forall P. \quotep{P}
  \not\in \freenames{P}$.
\end{remark}

\subsection{ Dynamic quote: an example }

Anticipating something of what's to come, consider applying the
substitution, $\widehat{\id{\{}u / z \id{\}}}$, to the following pair
of processes, $\lift{w}{y!(z)}$ and $w[ \lpquote y!(z) \rpquote ]$.

\begin{eqnarray}
	\lift{w}{y!(z)}\widehat{\id{\{}u / z \id{\}}}
		& = &
		\lift{w}{y!(u)} \nonumber\\
	w[ \lpquote y!(z) \rpquote ] \widehat{ \id{\{}u / z \id{\}} }
		& = &
		w[ \lpquote y!(z) \rpquote ] \nonumber
\end{eqnarray}

Because the body of the process between quotes is impervious to
substitution, we get radically different answers. In fact, by
examining the first process in an input context,
e.g. $x?(z).\lift{w}{y!(z)}$, we see that the process under the lift
operator may be shaped by prefixed inputs binding a name inside it. In
this sense, the lift operator will be seen as a way to dynamically
construct processes before reifying them as names.

Finally equipped with these standard features we can present the
dynamics of the calculus.

\subsubsection{Operational semantics} 

Finally, we introduce the computational dynamics. What marks these
algebras as distinct from other more traditionally studied algebraic
structures, e.g. vector spaces or polynomial rings, is the manner in
which dynamics is captured. In traditional structures, dynamics is typically
expressed through morphisms between such structures, as in linear maps
between vector spaces or morphisms between rings. In algebras
associated with the semantics of computation, the dynamics is
expressed as part of the algebraic structure itself, through a
reduction reduction relation typically denoted by $\red$. Below, we
give a recursive presentation of this relation for the calculus used
in the encoding.

$\red \subseteq \pi \times \pi$
$\red : \pi \to \mathcal{P}(\pi)$

\begin{mathpar}
  \inferrule* [lab=Comm] { \textsf{match}( x_{src}, x_{trgt} ) } { x_{trgt}?(y)P \; | \; x_{src}!\langle {Q} \rangle \red P\{\quotep{Q}/y}\} }
  \and \\
  \inferrule* [lab=Par] {{P} \red {P}'} {{{P} | {Q}} \red {{P}' | {Q}}}
  \and
  \inferrule* [lab=Equiv]{{{P} \scong {P}'} \andalso {{P}' \red {Q}'} \andalso {{Q}' \scong {Q}}}{{P} \red {Q}}
\end{mathpar}

\begin{eqnarray*}
  match_{\equiv} (\quotep{P},\quotep{Q}) & := & P \equiv Q \\
  match_{\dagger}(\quotep{P},\quotep{Q}) & := & \forall R. P|Q \red^{*} R => R \red^{*} 0 \\
  match_{K}(\quotep{P},\quotep{Q}) & := & K \mbox{ for some context } K
\end{eqnarray*}

$u?(x)P | u!\langle Q \rangle \red P\{\quotep{Q}/x\}$

%We write $\wred$ for $\red^*$, and $P\red$ if $\exists Q $ such that $ P \red Q$.
We write $P\red$ if $\exists Q $ such that $ P \red Q$ and $P\not\red$, otherwise.

\section{Replication}

As mentioned before, it is known that replication (and hence
recursion) can be implemented in a higher-order process algebra
\cite{SangiorgiWalker}. As our first example of calculation with the
machinery thus far presented we give the construction explicitly in
the {\rhoc}.

\begin{eqnarray}
	D_{x} & := & \prefix{x}{y}{(\binpar{\outputp{x}{y}}{@{y}})} \nonumber\\
	\bangp_{x}{P} & := & \binpar{{x}!\langle{\binpar{D_{x}}{P}}\rangle}{D_{x}} \nonumber
\end{eqnarray}

\begin{eqnarray}
	\bangp_{x}{P} & & \nonumber\\
	=
	& {x}!\langle{(\prefix{x}{y}{(\outputp{x}{y} | @{y})) | P}}\rangle 
	      | \prefix{x}{y}{(\outputp{x}{y} | @{y})} & \nonumber\\
	\red
	& (\outputp{x}{y} | @{y})\substn{\quotep{(\prefix{x}{y}{(@{y} | \outputp{x}{y})) | P}}}{y} & \nonumber\\
	=
	& \outputp{x}{\quotep{(\prefix{x}{y}{(\outputp{x}{y} | @{y})) | P}}}
	  | {(\prefix{x}{y}{(\outputp{x}{y} | @{y})) | P}} & \nonumber\\
	\red
	& \ldots & \nonumber\\
	\red^*
	& P | P | \ldots & \nonumber
\end{eqnarray}

Of course, this encoding, as an implementation, runs away, unfolding
$\bangp{P}$ eagerly. A lazier and more implementable replication
operator, restricted to input-guarded processes, may be obtained as follows.

\begin{eqnarray}
\bangp{\prefix{u}{v}{P}} 
	:= 
	\binpar{\lift{x}{\prefix{u}{v}{(\binpar{D(x)}{P})}}}{D(x)} \nonumber
\end{eqnarray}

\begin{remark}
  Note that the lazier definition still does not deal with summation
  or mixed summation (i.e. sums over input and output). The reader is
  invited to construct definitions of replication that deal with these
  features. 

  Further, the definitions are parameterized in a name, $x$. Can you,
  gentle reader, make a definition that eliminates this parameter and
  guarantees no accidental interaction between the replication
  machinery and the process being replicated -- i.e. no accidental
  sharing of names used by the process to get its work done and the
  name(s) used by the replication to effect copying. This latter
  revision of the definition of replication is crucial to obtaining
  the expected identity $!!P \sim !P$.
\end{remark}

\begin{remark}\label{rem:paradoxical_combinator}
  The reader familiar with the lambda calculus will have noticed the
  similarity between $D$ and the paradoxical combinator.

  [Ed. note: the existence of this seems to suggest we have to be more
  restrictive on the set of processes and names we admit if we are to
  support no-cloning.]
\end{remark}

\subsubsection{Bisimulation}

The computational dynamics gives rise to another kind of equivalence,
the equivalence of computational behavior. As previously mentioned
this is typically captured \emph{via} some form of bisimulation.

% The notion we use in this paper is weak barbed bisimulation
% \cite{milner91polyadicpi}.

The notion we use in this paper is derived from weak barbed
bisimulation \cite{milner91polyadicpi}. 

\begin{definition}
An \emph{observation relation}, $\downarrow_{\mathcal N}$, over a set
of names, $\mathcal N$, is the smallest relation satisfying the rules
below.

\infrule[Out-barb]{y \in {\mathcal N}, \; x \nameeq y}
		  {\outputp{x}{v} \downarrow_{\mathcal N} x}
\infrule[Par-barb]{\mbox{$P\downarrow_{\mathcal N} x$ or $Q\downarrow_{\mathcal N} x$}}
		  {\binpar{P}{Q} \downarrow_{\mathcal N} x}

We write $P \Downarrow_{\mathcal N} x$ if there is $Q$ such that 
$P \wred Q$ and $Q \downarrow_{\mathcal N} x$.
\end{definition}

\begin{definition}
%\label{def.bbisim}
An  ${\mathcal N}$-\emph{barbed bisimulation} over a set of names, ${\mathcal N}$, is a symmetric binary relation 
${\mathcal S}_{\mathcal N}$ between agents such that $P\rel{S}_{\mathcal N}Q$ implies:
\begin{enumerate}
\item If $P \red P'$ then $Q \wred Q'$ and $P'\rel{S}_{\mathcal N} Q'$.
\item If $P\downarrow_{\mathcal N} x$, then $Q\Downarrow_{\mathcal N} x$.
\end{enumerate}
$P$ is ${\mathcal N}$-barbed bisimilar to $Q$, written
$P \wbbisim_{\mathcal N} Q$, if $P \rel{S}_{\mathcal N} Q$ for some ${\mathcal N}$-barbed bisimulation ${\mathcal S}_{\mathcal N}$.
\end{definition}

$\mathcal{R} \subseteq \pi \times \pi$

$P \mathcal{R} Q => \forall P'. P \red P' \Rightarrow \exists Q'. Q \red Q', P' \mathcal{R} Q'$

$P \vdash x \Rightarrow Q \vdash x$

\begin{mathpar}
  \inferrule*[lab=Out-barb]{x \nameeq y}{{y}!\langle{Q}\rangle \vdash x}
  \and
  \inferrule*[lab=Par-barb]{\mbox{$P\vdash x$ or $Q\vdash x$}}{\binpar{P}{Q} \vdash x}
\end{mathpar}

\subsubsection{Contexts}

One of the principle advantages of computational calculi like the
$\pi$-calculus is a well-defined notion of context,
contextual-equivalence and a correlation between
contextual-equivalence and notions of bisimulation. The notion of
context allows the decomposition of a process into (sub-)process and
its syntactic environment, its context. Thus, a context may be
thought of as a process with a ``hole'' (written $\Box$) in it. The
application of a context $M$ to a process $P$, written $M[P]$, is
tantamount to filling the hole in $M$ with $P$. In this paper we do
not need the full weight of this theory, but do make use of the notion
of context in the proof the main theorem. 

\begin{mathpar}
  \inferrule* [lab=summation] {} {{M_{M},M_{N}} \bc \Box \;|\; x.M_{A} \;|\; M_{M}+M_{N}}
  \and
  \inferrule* [lab=agent] {} {{M_{A}} \bc (\vec{x})M_{P} \;| \; \clift{P_0,\ldots,M_{P},\ldots,P_N}}
  \and \\
  \inferrule* [lab=process] {} {{M_{P}} \bc M_{N} \;| \;P|M_{P} }
\end{mathpar} 

\begin{mathpar}
  \inferrule* [lab=sychronization] {} {M_{N} \bc \Box \;|\; x?M_{F} \;|\; x!M_{C}}
  \and
  \inferrule* [lab=abstraction] {} {{M_{F}} \bc (x)M_{P} }
  \and
  \inferrule* [lab=concretion] {} {{M_{C}} \bc \langle M_{P} \rangle }
  \and \\
  \inferrule* [lab=process] {} {{M_{P}} \bc M_{N} \;| \;P|M_{P} }
\end{mathpar}

\begin{definition}[contextual application] Given a context $M$, and
  process $P$, we define the \emph{contextual application}, $M[P] :=
  M\{P/\Box\}$. That is, the contextual application of M to P is the
  substitution of $P$ for $\Box$ in $M$.
\end{definition}

$\meaningof{-} : L \to \mathcal{P}(\pi)$

\begin{mathpar}
  \inferrule* [lab=collection] {} {\meaningof{true} = \pi, \and \meaningof{~E} = \pi \setminus \meaningof{E}, \and \meaningof{E_{1} \& E_{2}} = \meaningof{E_{1}} \cap \meaningof{E_{2}}}
\end{mathpar}

\begin{mathpar}
  \inferrule* [lab=structure] {} {\meaningof{0} = \{ P \in \pi | P \equiv 0 \}, \and \\ \meaningof{E_1 | E_2} = \{ P \in \pi | P \equiv P_{1} | P_{2}, P_{1} \in \meaningof{E_{1}}, P_{2} \in \meaningof{E_2}\} }
\end{mathpar}

\begin{mathpar}
 \inferrule* [lab=behavior] {} {\meaningof{\langle a?b \rangle E} = \{ P \in \pi | P \equiv Q | u?(y)P', \\ \and \\\\ \and \\ \;\;\; u \in \meaningof{a}, \forall z.P'\{z/y\} \in \meaningof{E\{z/b\}}\}, \and \\ \meaningof{a!E} = \{ P \in \pi | P \equiv Q | x!\langle P' \rangle, x \in \meaningof{a} P' \in \meaningof{E}\} }
\end{mathpar}

\begin{mathpar}
 \inferrule* [lab=nominal] {} {\meaningof{\quotep{E}} = \{ \quotep{P} \in \quotep{\pi} | P \in \meaningof{E} \}, \and \meaningof{\quotep{P}} = \{ \quotep{Q} \in \quotep{\pi} | P \equiv Q \} \and \\ \meaningof{@\quotep{E}} = \{ P \in \pi | P \equiv @x, x \in \meaningof{E} \}}
\end{mathpar}

\begin{eqnarray*}
  \\
  \meaningof{-} : TS \to ST
\end{eqnarray*}

\begin{eqnarray*}
  \\
  L : TS \to ST
\end{eqnarray*}

\begin{eqnarray*}
  \\
  P \models E \iff P \in \meaningof{E}
\end{eqnarray*}

\begin{eqnarray*}
  P \approx_{L} Q \iff \forall E \in L. P \models E \iff Q \models E
\end{eqnarray*}

\begin{eqnarray*}
  P \approx_{K} Q
\end{eqnarray*}

\begin{eqnarray*}
  P \approx Q
\end{eqnarray*}

$\approx_{K} = \approx = \approx_{L}$

\subsubsection{Contextual duality}

Note that contexts extend the quotation operation to a family of
operations from processes to names. Given a context, $M$, we can
define a \emph{nominal context}, $\quotep{M}$ by $\quotep{M}[P] :=
\quotep{M[P]}$. To foreshadow what is to come we observe that these
operations enjoy a duality with processes very much like the duality
between vectors and maps from vectors to scalars.

Further, because the calculus is essentially higher-order, we have a
correspondence between contexts and processes. More specifically,
given a name $x$ and a context $M$ we can construct $M^{*}_{x}$ such
that 

\begin{mathpar}
  M^{*}_{x} | \lift{x}{P} \red M[P]
\end{mathpar}

namely,

\begin{mathpar}
  M^{*}_{x} := x?(u).M[\dropn{u}]
\end{mathpar}

The dependence of $M^{*}_{x}$ on a name makes it an abstraction, 

\begin{mathpar}
  M^{*} := (x)x?(u).M[\dropn{u}]
\end{mathpar}

\subsection{Additional notation}

It will sometimes be convenient to denote the process a name
quotes. We already have the notation $x = \quotep{P}$, but it will be
convenient to introduce an alternate notation, $\procn{x}$, when we
want to emphasize the connection to the use of the name. Note that, by
virtue of name equivalence, $\quotep{\procn{x}} \nameeq x$; so, the
notation is consistent with previous definitions.

Further, because names have structure it is possible to effect
substitutions on the basis of that structure. This means we need to
upgrade our notation for substitutions, which we accomplish by
adapting comprehension notation. Thus,

\begin{mathpar}
  P\{ y / x : x \in S \}
\end{mathpar}

is interpreted to mean the process derived from P by replacing (in a
capture-avoiding manner) each occurrence of $x$ in $S$ by $y$. For example,

\begin{mathpar}
  P\{ \quotep{\procn{x}|\procn{x}} / x : x \in \freenames{P} \}
\end{mathpar}

will replace each (occurrence) of a free name $x$ in $P$ by
$\quotep{\procn{x}|\procn{x}}$.

Also, we will avail ourselves of the notation $x^{L}$ and $x^{R}$ to
denote injections of a name into disjoint copies of the name
space. There are numerous ways to accomplish this. One example can be
found in \cite{MeredithR05}. This notation overloads to vectors of
names: $\vec{x}^{\pi} := (x_{i}^{\pi} \; : \; 0 \leq i < |\vec{x}| )$ where $\pi \in \{L,R\}$.

We also use $P^{\Box} := P|\Box$.

In \cite{MeredithR05} an interpretation of the new operator is
given. It turns out that there are several possible interpretations
all enjoying the requisite algebraic properties of the operator (see
\cite{milner91polyadicpi}). We will therefore make liberal use of
$(\nu\; \vec{x})P$.

% subsection the_syntax_and_semantics_of_the_notation_system (end)   

\input{qm2pi.qmops} 

\input{qm2pi.sterngerlach} 

\input{qm2pi.metric} 

% section concurrent_process_calculi (end)

%\input{qm2pi.proofsketch}

% section proof sketch (end)

%\input{qm2pi.slviaknots} 

% section spatial logic via knots (end)

\input{qm2pi.conclusion}

% section conclusion (end)

%\input{qm2pi.dtcodes} 

% section wiring algorithm (end)

\input{qm2pi.ack} 

% section acknowledgments (end)

\newpage


\bibliographystyle{plain}   
\bibliography{../../biblios/main.bib}

\input{qm2pi.rhodetails}

\end{document}



% section front matter (end)

\section{Introduction}\label{sec:introduction} % (fold)
In this draft of the material i am going to have to dispense with the
usual writing conventions adopted in papers on these topics. i'm going
to have adopt whatever tone i need at the time i'm writing up the
calculations. Sometimes this may be very conversational; others it may
be the barest mathematical grunts; others still it may be that i have
lifted text from one of my other papers because the exposition of some
point was better said there. i hope that my readers are not unduly put
out by this decision. i'm not doing this to flout convention or be
rebellious. i find these calculations very technically challenging. To
keep everything going technically, something has to give; i have to
let go of some cognitive burden. So, the academic writing style --
with all of its trade-offs in terms of facilitating technical
communication -- is what i'm letting go of. Perhaps subsequent drafts
can be tightened and polished, but for now, i'm going to speak as if
we were sitting together in a coffee shop with a laptop, wifi and a
pad of paper and a pencil.

So, here's what i have to say. We -- you and i, comfortably ensconced
in our coffee shop and well-equipped with our tools -- can realize and
carry out the calculations of quantum mechanics over a very different
formal theory of dynamics, a formal theory of dynamics that
corresponds to a theory of concurrent computation with
\emph{reflection}. It has the advantage that the underlying theory is
already `quantized', but supports analogues all of the continuuous
operations. Strikingly, this underlying theory has recently been
connected with a notion of metric that we can show, by calculating
together, coincides with the metric induced by the inner product.

There are a lot of reasons why you might be interested in seeing
calculations of this form. Here's why i'm interested. For the past
several centuries there has been no competitor to the ``Newtonian''
account of dynamics. As a result the predominant share of accounts of
dynamical systems and situations have had to be formulated in terms of
the Newtonian machinery. i view this as an intellectually dangerous
position to occupy. Everything, despite it's intrinsic shape, turns
into a nail to be hit with this hammer. Recently, however, the theory
of computation has matured to the point where we have candidates for
theories of dynamics that offer very different perspective on
reasoning about dynamical systems and situations. Testing these
candidates against very successful accounts of dynamical situations,
like quantum mechanics, is going to give us some sense of how mature
they are and some measure of the quality of these accounts of
dynamics.

\subsection{Summary of contributions and outline of paper}

So, we're going to develop an interpretation of the operations of
quantum mechanics normally interpreted by Hilbert spaces and
operators. We're going to do this over a theory of computation. Note
that this is very different than the usual quantum computation program
which develops notions of computation over quantum mechanics. Rather,
we are developing a story that aligns with Wheeler's slogan: It from
Bit. To do this we will first provide an account of the theory of
computation at play here. Then we will dive into a calculation-driven
interpretation of the operations of quantum mechanics.

The reason we take this approach is that -- until very recently --
there hasn't been an axiomatic account of quantum mechanics. As a
result there has been no sharp delineation of the mathematical theory
supporting interpretation of the physical theory and the physical
theory, itself. So, ambient features of the maths are free to be
exploited (or supressed) without a real accounting of their physical
relevance. There is no sharp statement ``here's the physical theory''
qua \emph{theory} and ``here's the mathematical interpretation''
enabling a judgment of how faithful the interpretation is -- apart
from experimental observation. When there is an axiomatic account we
can judge how well a given mathematical formalism supports an
interpretation of the axioms, independent of
experimentation. Likewise, we can judge how well we have captured our
physical evidence and experience with our axiomatics, independent of
any specific mathematical implementation, with accidental detail that
may or may not have physical significance. 

In lieu of a fully fleshed out and vetted axiomatic account of quantum
mechanics, interpreting the operational notions in service of modeling
physical systems will have to suffice. In other words, we are not in
the business of providing a model of Hilbert spaces and operators. We
are in the business of providing a model of quantum mechanics because
we are motivated by testing our notions of dynamics against physical
theory; and, the predictive calculations of the physical theory must
serve as the best formulation -- shy of a fully fleshed out axiomatic
account -- of the physical theory itself (as they have for scientific
theories since time immemorial). Put another way, despite a
whole-hearted commitment to an It-from-Bit ontology, we are firmly
aligned with the shut-up-and-calculate camp as the best way to obtain
results either from the physical perspective or as a quality assurance
measure of our fledgling theory of dynamics.

In detail, we present a reflective process calculus. Then we develop
intuitive correspondences between the notions available in this
calculus and the usual physical notions supporting quantum mechanical
calculations. Thus, 

\begin{table}[htp]
  \center{
    \fbox{
      \begin{tabular}{c|c}
        quantum mechanics & process calculus \\
        \hline
        scalar & name \\
        state vector & process \\
        dual & contextual duals \\
        matrix & formal sums of process-context-dual pairs \\
        orthogonality & process annihilation \\
        inner product & execution-formula + quoting
      \end{tabular}
    }
  }
  \caption{QM - process calculi correspondences}
\end{table}

Then we tighten up these intuitions to operational definitions. We
employ the Dirac notation as the best proxy we can find for an
abstract syntax of the quantum mechanical notions. The definitions we
develop put us in contact with equational constraints coming from the
theory that we demonstrate the definitions and calculations satisfy.

This puts us in a position to shut up and calculate for the
Stern-Gerlach experimental set up, showing how these predictive
calculations become calculations on processes in our theory of a
reflective process calculus.

Penultimately, we demonstrate that the notion of metric coming from
the inner product coincides with the notion of metric available from
the theory of bisimulation. This demonstration gives us the right to
think of space as arising from behavior. Finally, we consider where we
might go from the new vantage point we have obtained.

% section introduction (end) 
 
% section introduction (end)

% \documentclass[12pt]{llncs}
%\documentclass{jktr}

\usepackage[pdftex]{hyperref}                   
\usepackage {listings}
\usepackage {mathpartir}
\usepackage{bcprules}
%\usepackage{listings}
                       
\usepackage{graphicx} 
%\usepackage[margins=2.5cm,nohead,nofoot]{geometry}
%\usepackage{geometry}
\usepackage{amsfonts}
\usepackage{amstext}
\usepackage{latexsym}
\usepackage{amssymb}
\usepackage{color}


%\include{myPreamble}
\include{qm2pi.local} 

%\ifpdf
%\usepackage[pdftex]{graphicx}
%\else
%\usepackage{graphicx}
%\fi

 % \ifpdf
%  \usepackage{pdfsync}
%  \if


%\title{Brief Article}
%\author{David F. Snyder}
%\author{L.G. Meredith}

%\address{Dept. of Math., Texas State University--San Marcos, San Marcos, TX 78666}
       
\pagestyle{empty}


\begin{document}

\lstset{language=[Objective]Caml,frame=shadowbox}

\input{qm2pi.front}

% section front matter (end)

\input{qm2pi.intro} 
 
% section introduction (end)

% \input{qm2pi.knotations} 

% section notation (end)

\input{qm2pi.process.calculi} 

% section concurrent_process_calculi_and_spatial_logics_ (end)
    
%\input{qm2pi.knots2pi} 

%\input{qm2pi.trefoil} 

%\input{qm2pi.mainthm} 

% subsection basic_interpretation (end)

%\input{qm2pi.rho.presentation} 
\subsection{The syntax and semantics of the notation system}\label{sub:the_syntax_and_semantics_of_the_notation_system} % (fold)

We now summarize a technical presentation of the calculus that
embodies our theory of dynamics. The typical presentation of such a
calculus follows the style of giving generators and relations on
them. The grammar, below, describing term constructors, freely
generates the set of processes, $\Proc$. This set is then quotiented
by a relation known as structural congruence and it is over this set
that the notion of dynamics is expressed. This presentation is
essentially that of \cite{MeredithR05} with the addition of
polyadicity and summation. For readability we have relegated some of
the technical subtleties to an appendix.

\subsubsection{Process grammar}\label{subsub:process_grammar}

\begin{mathpar}
  \inferrule* [lab=synchronization] {} {{M} \bc \pzero \;|\; x?F \;|\; x!C }
  \and
  \inferrule* [lab=abstraction] {} {{F} \bc (x)P}
  \and
  \inferrule* [lab=concretion] {} {{C} \bc \langle Q \rangle}
  \and
  \inferrule* [lab=process] {} {{P,Q} \bc M \;| \;P|Q \;|\; @{x}}
  \and
  \inferrule* [lab=name] {} {{x} \bc \quotep{P}}
\end{mathpar} 

Note that $\vec{x}$ (resp. $\vec{P}$) denotes a vector of names
(resp. processes) of length $|\vec{x}|$ (resp. $|\vec{P}|$). We adopt
the following useful abbreviations.

\begin{mathpar}
   x?(\vec{y}).P := x.(\vec{y})P \and  x\clift{\vec{P}} := x.\clift{\vec{P}}
   \and x!(y) := \lift{x}{\dropn{y}}
   \and \Pi_{i=0}^{n-1}P_i := P_0 | \ldots | P_{n-1}
\end{mathpar}

\subsubsection{Structural congruence}

\paragraph{Free and bound names and alpha-equivalence.} At the
core of structural equivalence is alpha-equivalence which identifies
process that are the same up to a change of variable. Formally, we
recognize the distinction between free and bound names. The free names
of a process, $\freenames{P}$, may be calculated recursively as
follows:

\begin{mathpar}
\freenames{\pzero} := \emptyset
  \and \\
  \freenames{x?(y).P} := \{ x \} \cup (\freenames{P} \setminus \{ y \})
  \and 
  \freenames{x!\langle P \rangle} := \{ x \} \cup \{ P \} 
  \and \\
  \freenames{P|Q} := \freenames{P} \cup \freenames{Q}
  \and \\
  \freenames{@{x}} := \{ x \}
\end{mathpar}

$\pi$
$\quotep{\pi}$

$\freenames{-} : \pi \to \mathcal{P}(\quotep{\pi})$

\begin{eqnarray*}
  \freenames{\pzero} & := & \emptyset \\
  \freenames{x?(y).P} & := & \{ x \} \cup (\freenames{P} \setminus \{ y \}) \\
  \freenames{x!\langle P \rangle} & := & \{ x \} \cup \{ P \} \\
  \freenames{P|Q} & := & \freenames{P} \cup \freenames{Q} \\
  \freenames{\dropn{x}} & := & \{ x \}
\end{eqnarray*}

The bound names of a process, $\boundnames{P}$, are those names occurring in $P$
that are not free. For example, in $x?(y).0$, the name $x$ is free, while $y$ is bound.

\begin{mathpar}
  \inferrule* [lab=monoidal-laws] {} { P|Q \equiv Q|P \and P|0 \equiv P \and P|(Q|R) \equiv (P|Q)|R }
\end{mathpar}

\begin{mathpar}
  \inferrule* [lab=alpha-equivalence] {} { (x)P \equiv (y)P\{y/x\} \and y \not\in \freenames{P} }
\end{mathpar}

\begin{definition}
Then two processes, $P,Q$, are alpha-equivalent if $P = Q\{\vec{y}/\vec{x}\}$ for
some $\vec{x} \in \boundnames{Q},\vec{y} \in \boundnames{P}$, where $Q\{\vec{y}/\vec{x}\}$
denotes the capture-avoiding substitution of $\vec{y}$ for $\vec{x}$ in $Q$.
\end{definition}

\begin{definition}
  The {\em structural congruence} \cite{SangiorgiWalker} , $\equiv$,
  between processes is the least congruence containing
  alpha-equivalence, satisfying the abelian monoid laws
  (associativity, commutativity and $\pzero$ as identity) for parallel
  composition $|$ and for summation $+$.
\end{definition}

\subsection{Name equivalence}

We take name equivalence, written $\nameeq$, to be the smallest
equivalence relation generated by the following rules.

\begin{mathpar}
\inferrule*[lab=Quote-drop]
{ }
{ \quotep{@{x}} \nameeq x }

\inferrule*[lab=Struct-equiv]
{ P \scong Q }
{ \quotep{P} \nameeq \quotep{Q} }
\end{mathpar}

The astute reader will have noticed that the mutual recursion of names
and processes imposes a mutual recursion on alpha-equivalence and
structural equivalence via name-equivalence. Fortunately, all of this
works out pleasantly and we may calculate in the natural way, free of
concern. The reader interested in the details is referred to the
appendix \ref{appendix:rho_details}.

\subsection{Substitution}

We use $\Proc$ for the set of processes, $\QProc$ for the set of
names, and $\id{\{}\vec{y} / \vec{x} \id{\}}$ to denote partial maps,
$s : \QProc \rightarrow \QProc$. A map, $s$ lifts, uniquely, to a map
on process terms, $\widehat{s} : \Proc \rightarrow \Proc$ by the
following equations.

\begin{mathpar}
  (0) \psubstp{Q}{P} := 0 \\
  (R \juxtap S) \psubstp{Q}{P}
  :=    
  (R)\psubstp{Q}{P} \juxtap (S) \psubstp{Q}{P} \\
  (x?(y).R) \psubstp{Q}{P}    
  :=    
  (x)\substp{Q}{P} (z)\concat( (R \psubstn{z}{y}) \psubstp{Q}{P} ) \\
  (\lift{x}{R}) \psubstp{Q}{P}  
  :=
  \lift{(x)\substp{Q}{P}}{ R \psubstp{Q}{P} } \\
%   (\dropn{x})  \psubstp{Q}{P}       
%   := 
%   \left\{ 
%     \begin{array}{ccc} 
%       \dropn{\quotep{Q}} & & x \nameeq \quotep{P} \\
%       \dropn{x} & & otherwise \\
%     \end{array}
%   \right. 
  (\dropn{x})  \psubstp{Q}{P}       
  := 
  \left\{ 
    \begin{array}{ccc} 
      Q & & x \nameeq \quotep{P} \\
      \dropn{x} & & otherwise \\
    \end{array}
  \right.
\end{mathpar}
 

where

\begin{eqnarray}
  (x)\id{\{} \lpquote Q \rpquote / \lpquote P \rpquote \id{\}}            = 
  \left\{ 
    \begin{array}{ccc}
      \lpquote Q \rpquote & & x \nameeq \lpquote P \rpquote \\
      x & & otherwise \\
    \end{array}
  \right. \nonumber
\end{eqnarray}

and $z$ is chosen distinct from $\quotep{P}$, $\quotep{Q}$, the free
names in $Q$, and all the names in $R$. Our $\alpha$-equivalence will
be built in the standard way from this substitution.

\begin{remark}\label{rem:no_self_referential_names}
  One consequence of these definitions is that $\forall P. \quotep{P}
  \not\in \freenames{P}$.
\end{remark}

\subsection{ Dynamic quote: an example }

Anticipating something of what's to come, consider applying the
substitution, $\widehat{\id{\{}u / z \id{\}}}$, to the following pair
of processes, $\lift{w}{y!(z)}$ and $w[ \lpquote y!(z) \rpquote ]$.

\begin{eqnarray}
	\lift{w}{y!(z)}\widehat{\id{\{}u / z \id{\}}}
		& = &
		\lift{w}{y!(u)} \nonumber\\
	w[ \lpquote y!(z) \rpquote ] \widehat{ \id{\{}u / z \id{\}} }
		& = &
		w[ \lpquote y!(z) \rpquote ] \nonumber
\end{eqnarray}

Because the body of the process between quotes is impervious to
substitution, we get radically different answers. In fact, by
examining the first process in an input context,
e.g. $x?(z).\lift{w}{y!(z)}$, we see that the process under the lift
operator may be shaped by prefixed inputs binding a name inside it. In
this sense, the lift operator will be seen as a way to dynamically
construct processes before reifying them as names.

Finally equipped with these standard features we can present the
dynamics of the calculus.

\subsubsection{Operational semantics} 

Finally, we introduce the computational dynamics. What marks these
algebras as distinct from other more traditionally studied algebraic
structures, e.g. vector spaces or polynomial rings, is the manner in
which dynamics is captured. In traditional structures, dynamics is typically
expressed through morphisms between such structures, as in linear maps
between vector spaces or morphisms between rings. In algebras
associated with the semantics of computation, the dynamics is
expressed as part of the algebraic structure itself, through a
reduction reduction relation typically denoted by $\red$. Below, we
give a recursive presentation of this relation for the calculus used
in the encoding.

$\red \subseteq \pi \times \pi$
$\red : \pi \to \mathcal{P}(\pi)$

\begin{mathpar}
  \inferrule* [lab=Comm] { \textsf{match}( x_{src}, x_{trgt} ) } { x_{trgt}?(y)P \; | \; x_{src}!\langle {Q} \rangle \red P\{\quotep{Q}/y}\} }
  \and \\
  \inferrule* [lab=Par] {{P} \red {P}'} {{{P} | {Q}} \red {{P}' | {Q}}}
  \and
  \inferrule* [lab=Equiv]{{{P} \scong {P}'} \andalso {{P}' \red {Q}'} \andalso {{Q}' \scong {Q}}}{{P} \red {Q}}
\end{mathpar}

\begin{eqnarray*}
  match_{\equiv} (\quotep{P},\quotep{Q}) & := & P \equiv Q \\
  match_{\dagger}(\quotep{P},\quotep{Q}) & := & \forall R. P|Q \red^{*} R => R \red^{*} 0 \\
  match_{K}(\quotep{P},\quotep{Q}) & := & K \mbox{ for some context } K
\end{eqnarray*}

$u?(x)P | u!\langle Q \rangle \red P\{\quotep{Q}/x\}$

%We write $\wred$ for $\red^*$, and $P\red$ if $\exists Q $ such that $ P \red Q$.
We write $P\red$ if $\exists Q $ such that $ P \red Q$ and $P\not\red$, otherwise.

\section{Replication}

As mentioned before, it is known that replication (and hence
recursion) can be implemented in a higher-order process algebra
\cite{SangiorgiWalker}. As our first example of calculation with the
machinery thus far presented we give the construction explicitly in
the {\rhoc}.

\begin{eqnarray}
	D_{x} & := & \prefix{x}{y}{(\binpar{\outputp{x}{y}}{@{y}})} \nonumber\\
	\bangp_{x}{P} & := & \binpar{{x}!\langle{\binpar{D_{x}}{P}}\rangle}{D_{x}} \nonumber
\end{eqnarray}

\begin{eqnarray}
	\bangp_{x}{P} & & \nonumber\\
	=
	& {x}!\langle{(\prefix{x}{y}{(\outputp{x}{y} | @{y})) | P}}\rangle 
	      | \prefix{x}{y}{(\outputp{x}{y} | @{y})} & \nonumber\\
	\red
	& (\outputp{x}{y} | @{y})\substn{\quotep{(\prefix{x}{y}{(@{y} | \outputp{x}{y})) | P}}}{y} & \nonumber\\
	=
	& \outputp{x}{\quotep{(\prefix{x}{y}{(\outputp{x}{y} | @{y})) | P}}}
	  | {(\prefix{x}{y}{(\outputp{x}{y} | @{y})) | P}} & \nonumber\\
	\red
	& \ldots & \nonumber\\
	\red^*
	& P | P | \ldots & \nonumber
\end{eqnarray}

Of course, this encoding, as an implementation, runs away, unfolding
$\bangp{P}$ eagerly. A lazier and more implementable replication
operator, restricted to input-guarded processes, may be obtained as follows.

\begin{eqnarray}
\bangp{\prefix{u}{v}{P}} 
	:= 
	\binpar{\lift{x}{\prefix{u}{v}{(\binpar{D(x)}{P})}}}{D(x)} \nonumber
\end{eqnarray}

\begin{remark}
  Note that the lazier definition still does not deal with summation
  or mixed summation (i.e. sums over input and output). The reader is
  invited to construct definitions of replication that deal with these
  features. 

  Further, the definitions are parameterized in a name, $x$. Can you,
  gentle reader, make a definition that eliminates this parameter and
  guarantees no accidental interaction between the replication
  machinery and the process being replicated -- i.e. no accidental
  sharing of names used by the process to get its work done and the
  name(s) used by the replication to effect copying. This latter
  revision of the definition of replication is crucial to obtaining
  the expected identity $!!P \sim !P$.
\end{remark}

\begin{remark}\label{rem:paradoxical_combinator}
  The reader familiar with the lambda calculus will have noticed the
  similarity between $D$ and the paradoxical combinator.

  [Ed. note: the existence of this seems to suggest we have to be more
  restrictive on the set of processes and names we admit if we are to
  support no-cloning.]
\end{remark}

\subsubsection{Bisimulation}

The computational dynamics gives rise to another kind of equivalence,
the equivalence of computational behavior. As previously mentioned
this is typically captured \emph{via} some form of bisimulation.

% The notion we use in this paper is weak barbed bisimulation
% \cite{milner91polyadicpi}.

The notion we use in this paper is derived from weak barbed
bisimulation \cite{milner91polyadicpi}. 

\begin{definition}
An \emph{observation relation}, $\downarrow_{\mathcal N}$, over a set
of names, $\mathcal N$, is the smallest relation satisfying the rules
below.

\infrule[Out-barb]{y \in {\mathcal N}, \; x \nameeq y}
		  {\outputp{x}{v} \downarrow_{\mathcal N} x}
\infrule[Par-barb]{\mbox{$P\downarrow_{\mathcal N} x$ or $Q\downarrow_{\mathcal N} x$}}
		  {\binpar{P}{Q} \downarrow_{\mathcal N} x}

We write $P \Downarrow_{\mathcal N} x$ if there is $Q$ such that 
$P \wred Q$ and $Q \downarrow_{\mathcal N} x$.
\end{definition}

\begin{definition}
%\label{def.bbisim}
An  ${\mathcal N}$-\emph{barbed bisimulation} over a set of names, ${\mathcal N}$, is a symmetric binary relation 
${\mathcal S}_{\mathcal N}$ between agents such that $P\rel{S}_{\mathcal N}Q$ implies:
\begin{enumerate}
\item If $P \red P'$ then $Q \wred Q'$ and $P'\rel{S}_{\mathcal N} Q'$.
\item If $P\downarrow_{\mathcal N} x$, then $Q\Downarrow_{\mathcal N} x$.
\end{enumerate}
$P$ is ${\mathcal N}$-barbed bisimilar to $Q$, written
$P \wbbisim_{\mathcal N} Q$, if $P \rel{S}_{\mathcal N} Q$ for some ${\mathcal N}$-barbed bisimulation ${\mathcal S}_{\mathcal N}$.
\end{definition}

$\mathcal{R} \subseteq \pi \times \pi$

$P \mathcal{R} Q => \forall P'. P \red P' \Rightarrow \exists Q'. Q \red Q', P' \mathcal{R} Q'$

$P \vdash x \Rightarrow Q \vdash x$

\begin{mathpar}
  \inferrule*[lab=Out-barb]{x \nameeq y}{{y}!\langle{Q}\rangle \vdash x}
  \and
  \inferrule*[lab=Par-barb]{\mbox{$P\vdash x$ or $Q\vdash x$}}{\binpar{P}{Q} \vdash x}
\end{mathpar}

\subsubsection{Contexts}

One of the principle advantages of computational calculi like the
$\pi$-calculus is a well-defined notion of context,
contextual-equivalence and a correlation between
contextual-equivalence and notions of bisimulation. The notion of
context allows the decomposition of a process into (sub-)process and
its syntactic environment, its context. Thus, a context may be
thought of as a process with a ``hole'' (written $\Box$) in it. The
application of a context $M$ to a process $P$, written $M[P]$, is
tantamount to filling the hole in $M$ with $P$. In this paper we do
not need the full weight of this theory, but do make use of the notion
of context in the proof the main theorem. 

\begin{mathpar}
  \inferrule* [lab=summation] {} {{M_{M},M_{N}} \bc \Box \;|\; x.M_{A} \;|\; M_{M}+M_{N}}
  \and
  \inferrule* [lab=agent] {} {{M_{A}} \bc (\vec{x})M_{P} \;| \; \clift{P_0,\ldots,M_{P},\ldots,P_N}}
  \and \\
  \inferrule* [lab=process] {} {{M_{P}} \bc M_{N} \;| \;P|M_{P} }
\end{mathpar} 

\begin{mathpar}
  \inferrule* [lab=sychronization] {} {M_{N} \bc \Box \;|\; x?M_{F} \;|\; x!M_{C}}
  \and
  \inferrule* [lab=abstraction] {} {{M_{F}} \bc (x)M_{P} }
  \and
  \inferrule* [lab=concretion] {} {{M_{C}} \bc \langle M_{P} \rangle }
  \and \\
  \inferrule* [lab=process] {} {{M_{P}} \bc M_{N} \;| \;P|M_{P} }
\end{mathpar}

\begin{definition}[contextual application] Given a context $M$, and
  process $P$, we define the \emph{contextual application}, $M[P] :=
  M\{P/\Box\}$. That is, the contextual application of M to P is the
  substitution of $P$ for $\Box$ in $M$.
\end{definition}

$\meaningof{-} : L \to \mathcal{P}(\pi)$

\begin{mathpar}
  \inferrule* [lab=collection] {} {\meaningof{true} = \pi, \and \meaningof{~E} = \pi \setminus \meaningof{E}, \and \meaningof{E_{1} \& E_{2}} = \meaningof{E_{1}} \cap \meaningof{E_{2}}}
\end{mathpar}

\begin{mathpar}
  \inferrule* [lab=structure] {} {\meaningof{0} = \{ P \in \pi | P \equiv 0 \}, \and \\ \meaningof{E_1 | E_2} = \{ P \in \pi | P \equiv P_{1} | P_{2}, P_{1} \in \meaningof{E_{1}}, P_{2} \in \meaningof{E_2}\} }
\end{mathpar}

\begin{mathpar}
 \inferrule* [lab=behavior] {} {\meaningof{\langle a?b \rangle E} = \{ P \in \pi | P \equiv Q | u?(y)P', \\ \and \\\\ \and \\ \;\;\; u \in \meaningof{a}, \forall z.P'\{z/y\} \in \meaningof{E\{z/b\}}\}, \and \\ \meaningof{a!E} = \{ P \in \pi | P \equiv Q | x!\langle P' \rangle, x \in \meaningof{a} P' \in \meaningof{E}\} }
\end{mathpar}

\begin{mathpar}
 \inferrule* [lab=nominal] {} {\meaningof{\quotep{E}} = \{ \quotep{P} \in \quotep{\pi} | P \in \meaningof{E} \}, \and \meaningof{\quotep{P}} = \{ \quotep{Q} \in \quotep{\pi} | P \equiv Q \} \and \\ \meaningof{@\quotep{E}} = \{ P \in \pi | P \equiv @x, x \in \meaningof{E} \}}
\end{mathpar}

\begin{eqnarray*}
  \\
  \meaningof{-} : TS \to ST
\end{eqnarray*}

\begin{eqnarray*}
  \\
  L : TS \to ST
\end{eqnarray*}

\begin{eqnarray*}
  \\
  P \models E \iff P \in \meaningof{E}
\end{eqnarray*}

\begin{eqnarray*}
  P \approx_{L} Q \iff \forall E \in L. P \models E \iff Q \models E
\end{eqnarray*}

\begin{eqnarray*}
  P \approx_{K} Q
\end{eqnarray*}

\begin{eqnarray*}
  P \approx Q
\end{eqnarray*}

$\approx_{K} = \approx = \approx_{L}$

\subsubsection{Contextual duality}

Note that contexts extend the quotation operation to a family of
operations from processes to names. Given a context, $M$, we can
define a \emph{nominal context}, $\quotep{M}$ by $\quotep{M}[P] :=
\quotep{M[P]}$. To foreshadow what is to come we observe that these
operations enjoy a duality with processes very much like the duality
between vectors and maps from vectors to scalars.

Further, because the calculus is essentially higher-order, we have a
correspondence between contexts and processes. More specifically,
given a name $x$ and a context $M$ we can construct $M^{*}_{x}$ such
that 

\begin{mathpar}
  M^{*}_{x} | \lift{x}{P} \red M[P]
\end{mathpar}

namely,

\begin{mathpar}
  M^{*}_{x} := x?(u).M[\dropn{u}]
\end{mathpar}

The dependence of $M^{*}_{x}$ on a name makes it an abstraction, 

\begin{mathpar}
  M^{*} := (x)x?(u).M[\dropn{u}]
\end{mathpar}

\subsection{Additional notation}

It will sometimes be convenient to denote the process a name
quotes. We already have the notation $x = \quotep{P}$, but it will be
convenient to introduce an alternate notation, $\procn{x}$, when we
want to emphasize the connection to the use of the name. Note that, by
virtue of name equivalence, $\quotep{\procn{x}} \nameeq x$; so, the
notation is consistent with previous definitions.

Further, because names have structure it is possible to effect
substitutions on the basis of that structure. This means we need to
upgrade our notation for substitutions, which we accomplish by
adapting comprehension notation. Thus,

\begin{mathpar}
  P\{ y / x : x \in S \}
\end{mathpar}

is interpreted to mean the process derived from P by replacing (in a
capture-avoiding manner) each occurrence of $x$ in $S$ by $y$. For example,

\begin{mathpar}
  P\{ \quotep{\procn{x}|\procn{x}} / x : x \in \freenames{P} \}
\end{mathpar}

will replace each (occurrence) of a free name $x$ in $P$ by
$\quotep{\procn{x}|\procn{x}}$.

Also, we will avail ourselves of the notation $x^{L}$ and $x^{R}$ to
denote injections of a name into disjoint copies of the name
space. There are numerous ways to accomplish this. One example can be
found in \cite{MeredithR05}. This notation overloads to vectors of
names: $\vec{x}^{\pi} := (x_{i}^{\pi} \; : \; 0 \leq i < |\vec{x}| )$ where $\pi \in \{L,R\}$.

We also use $P^{\Box} := P|\Box$.

In \cite{MeredithR05} an interpretation of the new operator is
given. It turns out that there are several possible interpretations
all enjoying the requisite algebraic properties of the operator (see
\cite{milner91polyadicpi}). We will therefore make liberal use of
$(\nu\; \vec{x})P$.

% subsection the_syntax_and_semantics_of_the_notation_system (end)   

\input{qm2pi.qmops} 

\input{qm2pi.sterngerlach} 

\input{qm2pi.metric} 

% section concurrent_process_calculi (end)

%\input{qm2pi.proofsketch}

% section proof sketch (end)

%\input{qm2pi.slviaknots} 

% section spatial logic via knots (end)

\input{qm2pi.conclusion}

% section conclusion (end)

%\input{qm2pi.dtcodes} 

% section wiring algorithm (end)

\input{qm2pi.ack} 

% section acknowledgments (end)

\newpage


\bibliographystyle{plain}   
\bibliography{../../biblios/main.bib}

\input{qm2pi.rhodetails}

\end{document}

 

% section notation (end)

\input{qm2pi.process.calculi} 

% section concurrent_process_calculi_and_spatial_logics_ (end)
    
%\documentclass[12pt]{llncs}
%\documentclass{jktr}

\usepackage[pdftex]{hyperref}                   
\usepackage {listings}
\usepackage {mathpartir}
\usepackage{bcprules}
%\usepackage{listings}
                       
\usepackage{graphicx} 
%\usepackage[margins=2.5cm,nohead,nofoot]{geometry}
%\usepackage{geometry}
\usepackage{amsfonts}
\usepackage{amstext}
\usepackage{latexsym}
\usepackage{amssymb}
\usepackage{color}


%\include{myPreamble}
\include{qm2pi.local} 

%\ifpdf
%\usepackage[pdftex]{graphicx}
%\else
%\usepackage{graphicx}
%\fi

 % \ifpdf
%  \usepackage{pdfsync}
%  \if


%\title{Brief Article}
%\author{David F. Snyder}
%\author{L.G. Meredith}

%\address{Dept. of Math., Texas State University--San Marcos, San Marcos, TX 78666}
       
\pagestyle{empty}


\begin{document}

\lstset{language=[Objective]Caml,frame=shadowbox}

\input{qm2pi.front}

% section front matter (end)

\input{qm2pi.intro} 
 
% section introduction (end)

% \input{qm2pi.knotations} 

% section notation (end)

\input{qm2pi.process.calculi} 

% section concurrent_process_calculi_and_spatial_logics_ (end)
    
%\input{qm2pi.knots2pi} 

%\input{qm2pi.trefoil} 

%\input{qm2pi.mainthm} 

% subsection basic_interpretation (end)

%\input{qm2pi.rho.presentation} 
\subsection{The syntax and semantics of the notation system}\label{sub:the_syntax_and_semantics_of_the_notation_system} % (fold)

We now summarize a technical presentation of the calculus that
embodies our theory of dynamics. The typical presentation of such a
calculus follows the style of giving generators and relations on
them. The grammar, below, describing term constructors, freely
generates the set of processes, $\Proc$. This set is then quotiented
by a relation known as structural congruence and it is over this set
that the notion of dynamics is expressed. This presentation is
essentially that of \cite{MeredithR05} with the addition of
polyadicity and summation. For readability we have relegated some of
the technical subtleties to an appendix.

\subsubsection{Process grammar}\label{subsub:process_grammar}

\begin{mathpar}
  \inferrule* [lab=synchronization] {} {{M} \bc \pzero \;|\; x?F \;|\; x!C }
  \and
  \inferrule* [lab=abstraction] {} {{F} \bc (x)P}
  \and
  \inferrule* [lab=concretion] {} {{C} \bc \langle Q \rangle}
  \and
  \inferrule* [lab=process] {} {{P,Q} \bc M \;| \;P|Q \;|\; @{x}}
  \and
  \inferrule* [lab=name] {} {{x} \bc \quotep{P}}
\end{mathpar} 

Note that $\vec{x}$ (resp. $\vec{P}$) denotes a vector of names
(resp. processes) of length $|\vec{x}|$ (resp. $|\vec{P}|$). We adopt
the following useful abbreviations.

\begin{mathpar}
   x?(\vec{y}).P := x.(\vec{y})P \and  x\clift{\vec{P}} := x.\clift{\vec{P}}
   \and x!(y) := \lift{x}{\dropn{y}}
   \and \Pi_{i=0}^{n-1}P_i := P_0 | \ldots | P_{n-1}
\end{mathpar}

\subsubsection{Structural congruence}

\paragraph{Free and bound names and alpha-equivalence.} At the
core of structural equivalence is alpha-equivalence which identifies
process that are the same up to a change of variable. Formally, we
recognize the distinction between free and bound names. The free names
of a process, $\freenames{P}$, may be calculated recursively as
follows:

\begin{mathpar}
\freenames{\pzero} := \emptyset
  \and \\
  \freenames{x?(y).P} := \{ x \} \cup (\freenames{P} \setminus \{ y \})
  \and 
  \freenames{x!\langle P \rangle} := \{ x \} \cup \{ P \} 
  \and \\
  \freenames{P|Q} := \freenames{P} \cup \freenames{Q}
  \and \\
  \freenames{@{x}} := \{ x \}
\end{mathpar}

$\pi$
$\quotep{\pi}$

$\freenames{-} : \pi \to \mathcal{P}(\quotep{\pi})$

\begin{eqnarray*}
  \freenames{\pzero} & := & \emptyset \\
  \freenames{x?(y).P} & := & \{ x \} \cup (\freenames{P} \setminus \{ y \}) \\
  \freenames{x!\langle P \rangle} & := & \{ x \} \cup \{ P \} \\
  \freenames{P|Q} & := & \freenames{P} \cup \freenames{Q} \\
  \freenames{\dropn{x}} & := & \{ x \}
\end{eqnarray*}

The bound names of a process, $\boundnames{P}$, are those names occurring in $P$
that are not free. For example, in $x?(y).0$, the name $x$ is free, while $y$ is bound.

\begin{mathpar}
  \inferrule* [lab=monoidal-laws] {} { P|Q \equiv Q|P \and P|0 \equiv P \and P|(Q|R) \equiv (P|Q)|R }
\end{mathpar}

\begin{mathpar}
  \inferrule* [lab=alpha-equivalence] {} { (x)P \equiv (y)P\{y/x\} \and y \not\in \freenames{P} }
\end{mathpar}

\begin{definition}
Then two processes, $P,Q$, are alpha-equivalent if $P = Q\{\vec{y}/\vec{x}\}$ for
some $\vec{x} \in \boundnames{Q},\vec{y} \in \boundnames{P}$, where $Q\{\vec{y}/\vec{x}\}$
denotes the capture-avoiding substitution of $\vec{y}$ for $\vec{x}$ in $Q$.
\end{definition}

\begin{definition}
  The {\em structural congruence} \cite{SangiorgiWalker} , $\equiv$,
  between processes is the least congruence containing
  alpha-equivalence, satisfying the abelian monoid laws
  (associativity, commutativity and $\pzero$ as identity) for parallel
  composition $|$ and for summation $+$.
\end{definition}

\subsection{Name equivalence}

We take name equivalence, written $\nameeq$, to be the smallest
equivalence relation generated by the following rules.

\begin{mathpar}
\inferrule*[lab=Quote-drop]
{ }
{ \quotep{@{x}} \nameeq x }

\inferrule*[lab=Struct-equiv]
{ P \scong Q }
{ \quotep{P} \nameeq \quotep{Q} }
\end{mathpar}

The astute reader will have noticed that the mutual recursion of names
and processes imposes a mutual recursion on alpha-equivalence and
structural equivalence via name-equivalence. Fortunately, all of this
works out pleasantly and we may calculate in the natural way, free of
concern. The reader interested in the details is referred to the
appendix \ref{appendix:rho_details}.

\subsection{Substitution}

We use $\Proc$ for the set of processes, $\QProc$ for the set of
names, and $\id{\{}\vec{y} / \vec{x} \id{\}}$ to denote partial maps,
$s : \QProc \rightarrow \QProc$. A map, $s$ lifts, uniquely, to a map
on process terms, $\widehat{s} : \Proc \rightarrow \Proc$ by the
following equations.

\begin{mathpar}
  (0) \psubstp{Q}{P} := 0 \\
  (R \juxtap S) \psubstp{Q}{P}
  :=    
  (R)\psubstp{Q}{P} \juxtap (S) \psubstp{Q}{P} \\
  (x?(y).R) \psubstp{Q}{P}    
  :=    
  (x)\substp{Q}{P} (z)\concat( (R \psubstn{z}{y}) \psubstp{Q}{P} ) \\
  (\lift{x}{R}) \psubstp{Q}{P}  
  :=
  \lift{(x)\substp{Q}{P}}{ R \psubstp{Q}{P} } \\
%   (\dropn{x})  \psubstp{Q}{P}       
%   := 
%   \left\{ 
%     \begin{array}{ccc} 
%       \dropn{\quotep{Q}} & & x \nameeq \quotep{P} \\
%       \dropn{x} & & otherwise \\
%     \end{array}
%   \right. 
  (\dropn{x})  \psubstp{Q}{P}       
  := 
  \left\{ 
    \begin{array}{ccc} 
      Q & & x \nameeq \quotep{P} \\
      \dropn{x} & & otherwise \\
    \end{array}
  \right.
\end{mathpar}
 

where

\begin{eqnarray}
  (x)\id{\{} \lpquote Q \rpquote / \lpquote P \rpquote \id{\}}            = 
  \left\{ 
    \begin{array}{ccc}
      \lpquote Q \rpquote & & x \nameeq \lpquote P \rpquote \\
      x & & otherwise \\
    \end{array}
  \right. \nonumber
\end{eqnarray}

and $z$ is chosen distinct from $\quotep{P}$, $\quotep{Q}$, the free
names in $Q$, and all the names in $R$. Our $\alpha$-equivalence will
be built in the standard way from this substitution.

\begin{remark}\label{rem:no_self_referential_names}
  One consequence of these definitions is that $\forall P. \quotep{P}
  \not\in \freenames{P}$.
\end{remark}

\subsection{ Dynamic quote: an example }

Anticipating something of what's to come, consider applying the
substitution, $\widehat{\id{\{}u / z \id{\}}}$, to the following pair
of processes, $\lift{w}{y!(z)}$ and $w[ \lpquote y!(z) \rpquote ]$.

\begin{eqnarray}
	\lift{w}{y!(z)}\widehat{\id{\{}u / z \id{\}}}
		& = &
		\lift{w}{y!(u)} \nonumber\\
	w[ \lpquote y!(z) \rpquote ] \widehat{ \id{\{}u / z \id{\}} }
		& = &
		w[ \lpquote y!(z) \rpquote ] \nonumber
\end{eqnarray}

Because the body of the process between quotes is impervious to
substitution, we get radically different answers. In fact, by
examining the first process in an input context,
e.g. $x?(z).\lift{w}{y!(z)}$, we see that the process under the lift
operator may be shaped by prefixed inputs binding a name inside it. In
this sense, the lift operator will be seen as a way to dynamically
construct processes before reifying them as names.

Finally equipped with these standard features we can present the
dynamics of the calculus.

\subsubsection{Operational semantics} 

Finally, we introduce the computational dynamics. What marks these
algebras as distinct from other more traditionally studied algebraic
structures, e.g. vector spaces or polynomial rings, is the manner in
which dynamics is captured. In traditional structures, dynamics is typically
expressed through morphisms between such structures, as in linear maps
between vector spaces or morphisms between rings. In algebras
associated with the semantics of computation, the dynamics is
expressed as part of the algebraic structure itself, through a
reduction reduction relation typically denoted by $\red$. Below, we
give a recursive presentation of this relation for the calculus used
in the encoding.

$\red \subseteq \pi \times \pi$
$\red : \pi \to \mathcal{P}(\pi)$

\begin{mathpar}
  \inferrule* [lab=Comm] { \textsf{match}( x_{src}, x_{trgt} ) } { x_{trgt}?(y)P \; | \; x_{src}!\langle {Q} \rangle \red P\{\quotep{Q}/y}\} }
  \and \\
  \inferrule* [lab=Par] {{P} \red {P}'} {{{P} | {Q}} \red {{P}' | {Q}}}
  \and
  \inferrule* [lab=Equiv]{{{P} \scong {P}'} \andalso {{P}' \red {Q}'} \andalso {{Q}' \scong {Q}}}{{P} \red {Q}}
\end{mathpar}

\begin{eqnarray*}
  match_{\equiv} (\quotep{P},\quotep{Q}) & := & P \equiv Q \\
  match_{\dagger}(\quotep{P},\quotep{Q}) & := & \forall R. P|Q \red^{*} R => R \red^{*} 0 \\
  match_{K}(\quotep{P},\quotep{Q}) & := & K \mbox{ for some context } K
\end{eqnarray*}

$u?(x)P | u!\langle Q \rangle \red P\{\quotep{Q}/x\}$

%We write $\wred$ for $\red^*$, and $P\red$ if $\exists Q $ such that $ P \red Q$.
We write $P\red$ if $\exists Q $ such that $ P \red Q$ and $P\not\red$, otherwise.

\section{Replication}

As mentioned before, it is known that replication (and hence
recursion) can be implemented in a higher-order process algebra
\cite{SangiorgiWalker}. As our first example of calculation with the
machinery thus far presented we give the construction explicitly in
the {\rhoc}.

\begin{eqnarray}
	D_{x} & := & \prefix{x}{y}{(\binpar{\outputp{x}{y}}{@{y}})} \nonumber\\
	\bangp_{x}{P} & := & \binpar{{x}!\langle{\binpar{D_{x}}{P}}\rangle}{D_{x}} \nonumber
\end{eqnarray}

\begin{eqnarray}
	\bangp_{x}{P} & & \nonumber\\
	=
	& {x}!\langle{(\prefix{x}{y}{(\outputp{x}{y} | @{y})) | P}}\rangle 
	      | \prefix{x}{y}{(\outputp{x}{y} | @{y})} & \nonumber\\
	\red
	& (\outputp{x}{y} | @{y})\substn{\quotep{(\prefix{x}{y}{(@{y} | \outputp{x}{y})) | P}}}{y} & \nonumber\\
	=
	& \outputp{x}{\quotep{(\prefix{x}{y}{(\outputp{x}{y} | @{y})) | P}}}
	  | {(\prefix{x}{y}{(\outputp{x}{y} | @{y})) | P}} & \nonumber\\
	\red
	& \ldots & \nonumber\\
	\red^*
	& P | P | \ldots & \nonumber
\end{eqnarray}

Of course, this encoding, as an implementation, runs away, unfolding
$\bangp{P}$ eagerly. A lazier and more implementable replication
operator, restricted to input-guarded processes, may be obtained as follows.

\begin{eqnarray}
\bangp{\prefix{u}{v}{P}} 
	:= 
	\binpar{\lift{x}{\prefix{u}{v}{(\binpar{D(x)}{P})}}}{D(x)} \nonumber
\end{eqnarray}

\begin{remark}
  Note that the lazier definition still does not deal with summation
  or mixed summation (i.e. sums over input and output). The reader is
  invited to construct definitions of replication that deal with these
  features. 

  Further, the definitions are parameterized in a name, $x$. Can you,
  gentle reader, make a definition that eliminates this parameter and
  guarantees no accidental interaction between the replication
  machinery and the process being replicated -- i.e. no accidental
  sharing of names used by the process to get its work done and the
  name(s) used by the replication to effect copying. This latter
  revision of the definition of replication is crucial to obtaining
  the expected identity $!!P \sim !P$.
\end{remark}

\begin{remark}\label{rem:paradoxical_combinator}
  The reader familiar with the lambda calculus will have noticed the
  similarity between $D$ and the paradoxical combinator.

  [Ed. note: the existence of this seems to suggest we have to be more
  restrictive on the set of processes and names we admit if we are to
  support no-cloning.]
\end{remark}

\subsubsection{Bisimulation}

The computational dynamics gives rise to another kind of equivalence,
the equivalence of computational behavior. As previously mentioned
this is typically captured \emph{via} some form of bisimulation.

% The notion we use in this paper is weak barbed bisimulation
% \cite{milner91polyadicpi}.

The notion we use in this paper is derived from weak barbed
bisimulation \cite{milner91polyadicpi}. 

\begin{definition}
An \emph{observation relation}, $\downarrow_{\mathcal N}$, over a set
of names, $\mathcal N$, is the smallest relation satisfying the rules
below.

\infrule[Out-barb]{y \in {\mathcal N}, \; x \nameeq y}
		  {\outputp{x}{v} \downarrow_{\mathcal N} x}
\infrule[Par-barb]{\mbox{$P\downarrow_{\mathcal N} x$ or $Q\downarrow_{\mathcal N} x$}}
		  {\binpar{P}{Q} \downarrow_{\mathcal N} x}

We write $P \Downarrow_{\mathcal N} x$ if there is $Q$ such that 
$P \wred Q$ and $Q \downarrow_{\mathcal N} x$.
\end{definition}

\begin{definition}
%\label{def.bbisim}
An  ${\mathcal N}$-\emph{barbed bisimulation} over a set of names, ${\mathcal N}$, is a symmetric binary relation 
${\mathcal S}_{\mathcal N}$ between agents such that $P\rel{S}_{\mathcal N}Q$ implies:
\begin{enumerate}
\item If $P \red P'$ then $Q \wred Q'$ and $P'\rel{S}_{\mathcal N} Q'$.
\item If $P\downarrow_{\mathcal N} x$, then $Q\Downarrow_{\mathcal N} x$.
\end{enumerate}
$P$ is ${\mathcal N}$-barbed bisimilar to $Q$, written
$P \wbbisim_{\mathcal N} Q$, if $P \rel{S}_{\mathcal N} Q$ for some ${\mathcal N}$-barbed bisimulation ${\mathcal S}_{\mathcal N}$.
\end{definition}

$\mathcal{R} \subseteq \pi \times \pi$

$P \mathcal{R} Q => \forall P'. P \red P' \Rightarrow \exists Q'. Q \red Q', P' \mathcal{R} Q'$

$P \vdash x \Rightarrow Q \vdash x$

\begin{mathpar}
  \inferrule*[lab=Out-barb]{x \nameeq y}{{y}!\langle{Q}\rangle \vdash x}
  \and
  \inferrule*[lab=Par-barb]{\mbox{$P\vdash x$ or $Q\vdash x$}}{\binpar{P}{Q} \vdash x}
\end{mathpar}

\subsubsection{Contexts}

One of the principle advantages of computational calculi like the
$\pi$-calculus is a well-defined notion of context,
contextual-equivalence and a correlation between
contextual-equivalence and notions of bisimulation. The notion of
context allows the decomposition of a process into (sub-)process and
its syntactic environment, its context. Thus, a context may be
thought of as a process with a ``hole'' (written $\Box$) in it. The
application of a context $M$ to a process $P$, written $M[P]$, is
tantamount to filling the hole in $M$ with $P$. In this paper we do
not need the full weight of this theory, but do make use of the notion
of context in the proof the main theorem. 

\begin{mathpar}
  \inferrule* [lab=summation] {} {{M_{M},M_{N}} \bc \Box \;|\; x.M_{A} \;|\; M_{M}+M_{N}}
  \and
  \inferrule* [lab=agent] {} {{M_{A}} \bc (\vec{x})M_{P} \;| \; \clift{P_0,\ldots,M_{P},\ldots,P_N}}
  \and \\
  \inferrule* [lab=process] {} {{M_{P}} \bc M_{N} \;| \;P|M_{P} }
\end{mathpar} 

\begin{mathpar}
  \inferrule* [lab=sychronization] {} {M_{N} \bc \Box \;|\; x?M_{F} \;|\; x!M_{C}}
  \and
  \inferrule* [lab=abstraction] {} {{M_{F}} \bc (x)M_{P} }
  \and
  \inferrule* [lab=concretion] {} {{M_{C}} \bc \langle M_{P} \rangle }
  \and \\
  \inferrule* [lab=process] {} {{M_{P}} \bc M_{N} \;| \;P|M_{P} }
\end{mathpar}

\begin{definition}[contextual application] Given a context $M$, and
  process $P$, we define the \emph{contextual application}, $M[P] :=
  M\{P/\Box\}$. That is, the contextual application of M to P is the
  substitution of $P$ for $\Box$ in $M$.
\end{definition}

$\meaningof{-} : L \to \mathcal{P}(\pi)$

\begin{mathpar}
  \inferrule* [lab=collection] {} {\meaningof{true} = \pi, \and \meaningof{~E} = \pi \setminus \meaningof{E}, \and \meaningof{E_{1} \& E_{2}} = \meaningof{E_{1}} \cap \meaningof{E_{2}}}
\end{mathpar}

\begin{mathpar}
  \inferrule* [lab=structure] {} {\meaningof{0} = \{ P \in \pi | P \equiv 0 \}, \and \\ \meaningof{E_1 | E_2} = \{ P \in \pi | P \equiv P_{1} | P_{2}, P_{1} \in \meaningof{E_{1}}, P_{2} \in \meaningof{E_2}\} }
\end{mathpar}

\begin{mathpar}
 \inferrule* [lab=behavior] {} {\meaningof{\langle a?b \rangle E} = \{ P \in \pi | P \equiv Q | u?(y)P', \\ \and \\\\ \and \\ \;\;\; u \in \meaningof{a}, \forall z.P'\{z/y\} \in \meaningof{E\{z/b\}}\}, \and \\ \meaningof{a!E} = \{ P \in \pi | P \equiv Q | x!\langle P' \rangle, x \in \meaningof{a} P' \in \meaningof{E}\} }
\end{mathpar}

\begin{mathpar}
 \inferrule* [lab=nominal] {} {\meaningof{\quotep{E}} = \{ \quotep{P} \in \quotep{\pi} | P \in \meaningof{E} \}, \and \meaningof{\quotep{P}} = \{ \quotep{Q} \in \quotep{\pi} | P \equiv Q \} \and \\ \meaningof{@\quotep{E}} = \{ P \in \pi | P \equiv @x, x \in \meaningof{E} \}}
\end{mathpar}

\begin{eqnarray*}
  \\
  \meaningof{-} : TS \to ST
\end{eqnarray*}

\begin{eqnarray*}
  \\
  L : TS \to ST
\end{eqnarray*}

\begin{eqnarray*}
  \\
  P \models E \iff P \in \meaningof{E}
\end{eqnarray*}

\begin{eqnarray*}
  P \approx_{L} Q \iff \forall E \in L. P \models E \iff Q \models E
\end{eqnarray*}

\begin{eqnarray*}
  P \approx_{K} Q
\end{eqnarray*}

\begin{eqnarray*}
  P \approx Q
\end{eqnarray*}

$\approx_{K} = \approx = \approx_{L}$

\subsubsection{Contextual duality}

Note that contexts extend the quotation operation to a family of
operations from processes to names. Given a context, $M$, we can
define a \emph{nominal context}, $\quotep{M}$ by $\quotep{M}[P] :=
\quotep{M[P]}$. To foreshadow what is to come we observe that these
operations enjoy a duality with processes very much like the duality
between vectors and maps from vectors to scalars.

Further, because the calculus is essentially higher-order, we have a
correspondence between contexts and processes. More specifically,
given a name $x$ and a context $M$ we can construct $M^{*}_{x}$ such
that 

\begin{mathpar}
  M^{*}_{x} | \lift{x}{P} \red M[P]
\end{mathpar}

namely,

\begin{mathpar}
  M^{*}_{x} := x?(u).M[\dropn{u}]
\end{mathpar}

The dependence of $M^{*}_{x}$ on a name makes it an abstraction, 

\begin{mathpar}
  M^{*} := (x)x?(u).M[\dropn{u}]
\end{mathpar}

\subsection{Additional notation}

It will sometimes be convenient to denote the process a name
quotes. We already have the notation $x = \quotep{P}$, but it will be
convenient to introduce an alternate notation, $\procn{x}$, when we
want to emphasize the connection to the use of the name. Note that, by
virtue of name equivalence, $\quotep{\procn{x}} \nameeq x$; so, the
notation is consistent with previous definitions.

Further, because names have structure it is possible to effect
substitutions on the basis of that structure. This means we need to
upgrade our notation for substitutions, which we accomplish by
adapting comprehension notation. Thus,

\begin{mathpar}
  P\{ y / x : x \in S \}
\end{mathpar}

is interpreted to mean the process derived from P by replacing (in a
capture-avoiding manner) each occurrence of $x$ in $S$ by $y$. For example,

\begin{mathpar}
  P\{ \quotep{\procn{x}|\procn{x}} / x : x \in \freenames{P} \}
\end{mathpar}

will replace each (occurrence) of a free name $x$ in $P$ by
$\quotep{\procn{x}|\procn{x}}$.

Also, we will avail ourselves of the notation $x^{L}$ and $x^{R}$ to
denote injections of a name into disjoint copies of the name
space. There are numerous ways to accomplish this. One example can be
found in \cite{MeredithR05}. This notation overloads to vectors of
names: $\vec{x}^{\pi} := (x_{i}^{\pi} \; : \; 0 \leq i < |\vec{x}| )$ where $\pi \in \{L,R\}$.

We also use $P^{\Box} := P|\Box$.

In \cite{MeredithR05} an interpretation of the new operator is
given. It turns out that there are several possible interpretations
all enjoying the requisite algebraic properties of the operator (see
\cite{milner91polyadicpi}). We will therefore make liberal use of
$(\nu\; \vec{x})P$.

% subsection the_syntax_and_semantics_of_the_notation_system (end)   

\input{qm2pi.qmops} 

\input{qm2pi.sterngerlach} 

\input{qm2pi.metric} 

% section concurrent_process_calculi (end)

%\input{qm2pi.proofsketch}

% section proof sketch (end)

%\input{qm2pi.slviaknots} 

% section spatial logic via knots (end)

\input{qm2pi.conclusion}

% section conclusion (end)

%\input{qm2pi.dtcodes} 

% section wiring algorithm (end)

\input{qm2pi.ack} 

% section acknowledgments (end)

\newpage


\bibliographystyle{plain}   
\bibliography{../../biblios/main.bib}

\input{qm2pi.rhodetails}

\end{document}

 

%\documentclass[12pt]{llncs}
%\documentclass{jktr}

\usepackage[pdftex]{hyperref}                   
\usepackage {listings}
\usepackage {mathpartir}
\usepackage{bcprules}
%\usepackage{listings}
                       
\usepackage{graphicx} 
%\usepackage[margins=2.5cm,nohead,nofoot]{geometry}
%\usepackage{geometry}
\usepackage{amsfonts}
\usepackage{amstext}
\usepackage{latexsym}
\usepackage{amssymb}
\usepackage{color}


%\include{myPreamble}
\include{qm2pi.local} 

%\ifpdf
%\usepackage[pdftex]{graphicx}
%\else
%\usepackage{graphicx}
%\fi

 % \ifpdf
%  \usepackage{pdfsync}
%  \if


%\title{Brief Article}
%\author{David F. Snyder}
%\author{L.G. Meredith}

%\address{Dept. of Math., Texas State University--San Marcos, San Marcos, TX 78666}
       
\pagestyle{empty}


\begin{document}

\lstset{language=[Objective]Caml,frame=shadowbox}

\input{qm2pi.front}

% section front matter (end)

\input{qm2pi.intro} 
 
% section introduction (end)

% \input{qm2pi.knotations} 

% section notation (end)

\input{qm2pi.process.calculi} 

% section concurrent_process_calculi_and_spatial_logics_ (end)
    
%\input{qm2pi.knots2pi} 

%\input{qm2pi.trefoil} 

%\input{qm2pi.mainthm} 

% subsection basic_interpretation (end)

%\input{qm2pi.rho.presentation} 
\subsection{The syntax and semantics of the notation system}\label{sub:the_syntax_and_semantics_of_the_notation_system} % (fold)

We now summarize a technical presentation of the calculus that
embodies our theory of dynamics. The typical presentation of such a
calculus follows the style of giving generators and relations on
them. The grammar, below, describing term constructors, freely
generates the set of processes, $\Proc$. This set is then quotiented
by a relation known as structural congruence and it is over this set
that the notion of dynamics is expressed. This presentation is
essentially that of \cite{MeredithR05} with the addition of
polyadicity and summation. For readability we have relegated some of
the technical subtleties to an appendix.

\subsubsection{Process grammar}\label{subsub:process_grammar}

\begin{mathpar}
  \inferrule* [lab=synchronization] {} {{M} \bc \pzero \;|\; x?F \;|\; x!C }
  \and
  \inferrule* [lab=abstraction] {} {{F} \bc (x)P}
  \and
  \inferrule* [lab=concretion] {} {{C} \bc \langle Q \rangle}
  \and
  \inferrule* [lab=process] {} {{P,Q} \bc M \;| \;P|Q \;|\; @{x}}
  \and
  \inferrule* [lab=name] {} {{x} \bc \quotep{P}}
\end{mathpar} 

Note that $\vec{x}$ (resp. $\vec{P}$) denotes a vector of names
(resp. processes) of length $|\vec{x}|$ (resp. $|\vec{P}|$). We adopt
the following useful abbreviations.

\begin{mathpar}
   x?(\vec{y}).P := x.(\vec{y})P \and  x\clift{\vec{P}} := x.\clift{\vec{P}}
   \and x!(y) := \lift{x}{\dropn{y}}
   \and \Pi_{i=0}^{n-1}P_i := P_0 | \ldots | P_{n-1}
\end{mathpar}

\subsubsection{Structural congruence}

\paragraph{Free and bound names and alpha-equivalence.} At the
core of structural equivalence is alpha-equivalence which identifies
process that are the same up to a change of variable. Formally, we
recognize the distinction between free and bound names. The free names
of a process, $\freenames{P}$, may be calculated recursively as
follows:

\begin{mathpar}
\freenames{\pzero} := \emptyset
  \and \\
  \freenames{x?(y).P} := \{ x \} \cup (\freenames{P} \setminus \{ y \})
  \and 
  \freenames{x!\langle P \rangle} := \{ x \} \cup \{ P \} 
  \and \\
  \freenames{P|Q} := \freenames{P} \cup \freenames{Q}
  \and \\
  \freenames{@{x}} := \{ x \}
\end{mathpar}

$\pi$
$\quotep{\pi}$

$\freenames{-} : \pi \to \mathcal{P}(\quotep{\pi})$

\begin{eqnarray*}
  \freenames{\pzero} & := & \emptyset \\
  \freenames{x?(y).P} & := & \{ x \} \cup (\freenames{P} \setminus \{ y \}) \\
  \freenames{x!\langle P \rangle} & := & \{ x \} \cup \{ P \} \\
  \freenames{P|Q} & := & \freenames{P} \cup \freenames{Q} \\
  \freenames{\dropn{x}} & := & \{ x \}
\end{eqnarray*}

The bound names of a process, $\boundnames{P}$, are those names occurring in $P$
that are not free. For example, in $x?(y).0$, the name $x$ is free, while $y$ is bound.

\begin{mathpar}
  \inferrule* [lab=monoidal-laws] {} { P|Q \equiv Q|P \and P|0 \equiv P \and P|(Q|R) \equiv (P|Q)|R }
\end{mathpar}

\begin{mathpar}
  \inferrule* [lab=alpha-equivalence] {} { (x)P \equiv (y)P\{y/x\} \and y \not\in \freenames{P} }
\end{mathpar}

\begin{definition}
Then two processes, $P,Q$, are alpha-equivalent if $P = Q\{\vec{y}/\vec{x}\}$ for
some $\vec{x} \in \boundnames{Q},\vec{y} \in \boundnames{P}$, where $Q\{\vec{y}/\vec{x}\}$
denotes the capture-avoiding substitution of $\vec{y}$ for $\vec{x}$ in $Q$.
\end{definition}

\begin{definition}
  The {\em structural congruence} \cite{SangiorgiWalker} , $\equiv$,
  between processes is the least congruence containing
  alpha-equivalence, satisfying the abelian monoid laws
  (associativity, commutativity and $\pzero$ as identity) for parallel
  composition $|$ and for summation $+$.
\end{definition}

\subsection{Name equivalence}

We take name equivalence, written $\nameeq$, to be the smallest
equivalence relation generated by the following rules.

\begin{mathpar}
\inferrule*[lab=Quote-drop]
{ }
{ \quotep{@{x}} \nameeq x }

\inferrule*[lab=Struct-equiv]
{ P \scong Q }
{ \quotep{P} \nameeq \quotep{Q} }
\end{mathpar}

The astute reader will have noticed that the mutual recursion of names
and processes imposes a mutual recursion on alpha-equivalence and
structural equivalence via name-equivalence. Fortunately, all of this
works out pleasantly and we may calculate in the natural way, free of
concern. The reader interested in the details is referred to the
appendix \ref{appendix:rho_details}.

\subsection{Substitution}

We use $\Proc$ for the set of processes, $\QProc$ for the set of
names, and $\id{\{}\vec{y} / \vec{x} \id{\}}$ to denote partial maps,
$s : \QProc \rightarrow \QProc$. A map, $s$ lifts, uniquely, to a map
on process terms, $\widehat{s} : \Proc \rightarrow \Proc$ by the
following equations.

\begin{mathpar}
  (0) \psubstp{Q}{P} := 0 \\
  (R \juxtap S) \psubstp{Q}{P}
  :=    
  (R)\psubstp{Q}{P} \juxtap (S) \psubstp{Q}{P} \\
  (x?(y).R) \psubstp{Q}{P}    
  :=    
  (x)\substp{Q}{P} (z)\concat( (R \psubstn{z}{y}) \psubstp{Q}{P} ) \\
  (\lift{x}{R}) \psubstp{Q}{P}  
  :=
  \lift{(x)\substp{Q}{P}}{ R \psubstp{Q}{P} } \\
%   (\dropn{x})  \psubstp{Q}{P}       
%   := 
%   \left\{ 
%     \begin{array}{ccc} 
%       \dropn{\quotep{Q}} & & x \nameeq \quotep{P} \\
%       \dropn{x} & & otherwise \\
%     \end{array}
%   \right. 
  (\dropn{x})  \psubstp{Q}{P}       
  := 
  \left\{ 
    \begin{array}{ccc} 
      Q & & x \nameeq \quotep{P} \\
      \dropn{x} & & otherwise \\
    \end{array}
  \right.
\end{mathpar}
 

where

\begin{eqnarray}
  (x)\id{\{} \lpquote Q \rpquote / \lpquote P \rpquote \id{\}}            = 
  \left\{ 
    \begin{array}{ccc}
      \lpquote Q \rpquote & & x \nameeq \lpquote P \rpquote \\
      x & & otherwise \\
    \end{array}
  \right. \nonumber
\end{eqnarray}

and $z$ is chosen distinct from $\quotep{P}$, $\quotep{Q}$, the free
names in $Q$, and all the names in $R$. Our $\alpha$-equivalence will
be built in the standard way from this substitution.

\begin{remark}\label{rem:no_self_referential_names}
  One consequence of these definitions is that $\forall P. \quotep{P}
  \not\in \freenames{P}$.
\end{remark}

\subsection{ Dynamic quote: an example }

Anticipating something of what's to come, consider applying the
substitution, $\widehat{\id{\{}u / z \id{\}}}$, to the following pair
of processes, $\lift{w}{y!(z)}$ and $w[ \lpquote y!(z) \rpquote ]$.

\begin{eqnarray}
	\lift{w}{y!(z)}\widehat{\id{\{}u / z \id{\}}}
		& = &
		\lift{w}{y!(u)} \nonumber\\
	w[ \lpquote y!(z) \rpquote ] \widehat{ \id{\{}u / z \id{\}} }
		& = &
		w[ \lpquote y!(z) \rpquote ] \nonumber
\end{eqnarray}

Because the body of the process between quotes is impervious to
substitution, we get radically different answers. In fact, by
examining the first process in an input context,
e.g. $x?(z).\lift{w}{y!(z)}$, we see that the process under the lift
operator may be shaped by prefixed inputs binding a name inside it. In
this sense, the lift operator will be seen as a way to dynamically
construct processes before reifying them as names.

Finally equipped with these standard features we can present the
dynamics of the calculus.

\subsubsection{Operational semantics} 

Finally, we introduce the computational dynamics. What marks these
algebras as distinct from other more traditionally studied algebraic
structures, e.g. vector spaces or polynomial rings, is the manner in
which dynamics is captured. In traditional structures, dynamics is typically
expressed through morphisms between such structures, as in linear maps
between vector spaces or morphisms between rings. In algebras
associated with the semantics of computation, the dynamics is
expressed as part of the algebraic structure itself, through a
reduction reduction relation typically denoted by $\red$. Below, we
give a recursive presentation of this relation for the calculus used
in the encoding.

$\red \subseteq \pi \times \pi$
$\red : \pi \to \mathcal{P}(\pi)$

\begin{mathpar}
  \inferrule* [lab=Comm] { \textsf{match}( x_{src}, x_{trgt} ) } { x_{trgt}?(y)P \; | \; x_{src}!\langle {Q} \rangle \red P\{\quotep{Q}/y}\} }
  \and \\
  \inferrule* [lab=Par] {{P} \red {P}'} {{{P} | {Q}} \red {{P}' | {Q}}}
  \and
  \inferrule* [lab=Equiv]{{{P} \scong {P}'} \andalso {{P}' \red {Q}'} \andalso {{Q}' \scong {Q}}}{{P} \red {Q}}
\end{mathpar}

\begin{eqnarray*}
  match_{\equiv} (\quotep{P},\quotep{Q}) & := & P \equiv Q \\
  match_{\dagger}(\quotep{P},\quotep{Q}) & := & \forall R. P|Q \red^{*} R => R \red^{*} 0 \\
  match_{K}(\quotep{P},\quotep{Q}) & := & K \mbox{ for some context } K
\end{eqnarray*}

$u?(x)P | u!\langle Q \rangle \red P\{\quotep{Q}/x\}$

%We write $\wred$ for $\red^*$, and $P\red$ if $\exists Q $ such that $ P \red Q$.
We write $P\red$ if $\exists Q $ such that $ P \red Q$ and $P\not\red$, otherwise.

\section{Replication}

As mentioned before, it is known that replication (and hence
recursion) can be implemented in a higher-order process algebra
\cite{SangiorgiWalker}. As our first example of calculation with the
machinery thus far presented we give the construction explicitly in
the {\rhoc}.

\begin{eqnarray}
	D_{x} & := & \prefix{x}{y}{(\binpar{\outputp{x}{y}}{@{y}})} \nonumber\\
	\bangp_{x}{P} & := & \binpar{{x}!\langle{\binpar{D_{x}}{P}}\rangle}{D_{x}} \nonumber
\end{eqnarray}

\begin{eqnarray}
	\bangp_{x}{P} & & \nonumber\\
	=
	& {x}!\langle{(\prefix{x}{y}{(\outputp{x}{y} | @{y})) | P}}\rangle 
	      | \prefix{x}{y}{(\outputp{x}{y} | @{y})} & \nonumber\\
	\red
	& (\outputp{x}{y} | @{y})\substn{\quotep{(\prefix{x}{y}{(@{y} | \outputp{x}{y})) | P}}}{y} & \nonumber\\
	=
	& \outputp{x}{\quotep{(\prefix{x}{y}{(\outputp{x}{y} | @{y})) | P}}}
	  | {(\prefix{x}{y}{(\outputp{x}{y} | @{y})) | P}} & \nonumber\\
	\red
	& \ldots & \nonumber\\
	\red^*
	& P | P | \ldots & \nonumber
\end{eqnarray}

Of course, this encoding, as an implementation, runs away, unfolding
$\bangp{P}$ eagerly. A lazier and more implementable replication
operator, restricted to input-guarded processes, may be obtained as follows.

\begin{eqnarray}
\bangp{\prefix{u}{v}{P}} 
	:= 
	\binpar{\lift{x}{\prefix{u}{v}{(\binpar{D(x)}{P})}}}{D(x)} \nonumber
\end{eqnarray}

\begin{remark}
  Note that the lazier definition still does not deal with summation
  or mixed summation (i.e. sums over input and output). The reader is
  invited to construct definitions of replication that deal with these
  features. 

  Further, the definitions are parameterized in a name, $x$. Can you,
  gentle reader, make a definition that eliminates this parameter and
  guarantees no accidental interaction between the replication
  machinery and the process being replicated -- i.e. no accidental
  sharing of names used by the process to get its work done and the
  name(s) used by the replication to effect copying. This latter
  revision of the definition of replication is crucial to obtaining
  the expected identity $!!P \sim !P$.
\end{remark}

\begin{remark}\label{rem:paradoxical_combinator}
  The reader familiar with the lambda calculus will have noticed the
  similarity between $D$ and the paradoxical combinator.

  [Ed. note: the existence of this seems to suggest we have to be more
  restrictive on the set of processes and names we admit if we are to
  support no-cloning.]
\end{remark}

\subsubsection{Bisimulation}

The computational dynamics gives rise to another kind of equivalence,
the equivalence of computational behavior. As previously mentioned
this is typically captured \emph{via} some form of bisimulation.

% The notion we use in this paper is weak barbed bisimulation
% \cite{milner91polyadicpi}.

The notion we use in this paper is derived from weak barbed
bisimulation \cite{milner91polyadicpi}. 

\begin{definition}
An \emph{observation relation}, $\downarrow_{\mathcal N}$, over a set
of names, $\mathcal N$, is the smallest relation satisfying the rules
below.

\infrule[Out-barb]{y \in {\mathcal N}, \; x \nameeq y}
		  {\outputp{x}{v} \downarrow_{\mathcal N} x}
\infrule[Par-barb]{\mbox{$P\downarrow_{\mathcal N} x$ or $Q\downarrow_{\mathcal N} x$}}
		  {\binpar{P}{Q} \downarrow_{\mathcal N} x}

We write $P \Downarrow_{\mathcal N} x$ if there is $Q$ such that 
$P \wred Q$ and $Q \downarrow_{\mathcal N} x$.
\end{definition}

\begin{definition}
%\label{def.bbisim}
An  ${\mathcal N}$-\emph{barbed bisimulation} over a set of names, ${\mathcal N}$, is a symmetric binary relation 
${\mathcal S}_{\mathcal N}$ between agents such that $P\rel{S}_{\mathcal N}Q$ implies:
\begin{enumerate}
\item If $P \red P'$ then $Q \wred Q'$ and $P'\rel{S}_{\mathcal N} Q'$.
\item If $P\downarrow_{\mathcal N} x$, then $Q\Downarrow_{\mathcal N} x$.
\end{enumerate}
$P$ is ${\mathcal N}$-barbed bisimilar to $Q$, written
$P \wbbisim_{\mathcal N} Q$, if $P \rel{S}_{\mathcal N} Q$ for some ${\mathcal N}$-barbed bisimulation ${\mathcal S}_{\mathcal N}$.
\end{definition}

$\mathcal{R} \subseteq \pi \times \pi$

$P \mathcal{R} Q => \forall P'. P \red P' \Rightarrow \exists Q'. Q \red Q', P' \mathcal{R} Q'$

$P \vdash x \Rightarrow Q \vdash x$

\begin{mathpar}
  \inferrule*[lab=Out-barb]{x \nameeq y}{{y}!\langle{Q}\rangle \vdash x}
  \and
  \inferrule*[lab=Par-barb]{\mbox{$P\vdash x$ or $Q\vdash x$}}{\binpar{P}{Q} \vdash x}
\end{mathpar}

\subsubsection{Contexts}

One of the principle advantages of computational calculi like the
$\pi$-calculus is a well-defined notion of context,
contextual-equivalence and a correlation between
contextual-equivalence and notions of bisimulation. The notion of
context allows the decomposition of a process into (sub-)process and
its syntactic environment, its context. Thus, a context may be
thought of as a process with a ``hole'' (written $\Box$) in it. The
application of a context $M$ to a process $P$, written $M[P]$, is
tantamount to filling the hole in $M$ with $P$. In this paper we do
not need the full weight of this theory, but do make use of the notion
of context in the proof the main theorem. 

\begin{mathpar}
  \inferrule* [lab=summation] {} {{M_{M},M_{N}} \bc \Box \;|\; x.M_{A} \;|\; M_{M}+M_{N}}
  \and
  \inferrule* [lab=agent] {} {{M_{A}} \bc (\vec{x})M_{P} \;| \; \clift{P_0,\ldots,M_{P},\ldots,P_N}}
  \and \\
  \inferrule* [lab=process] {} {{M_{P}} \bc M_{N} \;| \;P|M_{P} }
\end{mathpar} 

\begin{mathpar}
  \inferrule* [lab=sychronization] {} {M_{N} \bc \Box \;|\; x?M_{F} \;|\; x!M_{C}}
  \and
  \inferrule* [lab=abstraction] {} {{M_{F}} \bc (x)M_{P} }
  \and
  \inferrule* [lab=concretion] {} {{M_{C}} \bc \langle M_{P} \rangle }
  \and \\
  \inferrule* [lab=process] {} {{M_{P}} \bc M_{N} \;| \;P|M_{P} }
\end{mathpar}

\begin{definition}[contextual application] Given a context $M$, and
  process $P$, we define the \emph{contextual application}, $M[P] :=
  M\{P/\Box\}$. That is, the contextual application of M to P is the
  substitution of $P$ for $\Box$ in $M$.
\end{definition}

$\meaningof{-} : L \to \mathcal{P}(\pi)$

\begin{mathpar}
  \inferrule* [lab=collection] {} {\meaningof{true} = \pi, \and \meaningof{~E} = \pi \setminus \meaningof{E}, \and \meaningof{E_{1} \& E_{2}} = \meaningof{E_{1}} \cap \meaningof{E_{2}}}
\end{mathpar}

\begin{mathpar}
  \inferrule* [lab=structure] {} {\meaningof{0} = \{ P \in \pi | P \equiv 0 \}, \and \\ \meaningof{E_1 | E_2} = \{ P \in \pi | P \equiv P_{1} | P_{2}, P_{1} \in \meaningof{E_{1}}, P_{2} \in \meaningof{E_2}\} }
\end{mathpar}

\begin{mathpar}
 \inferrule* [lab=behavior] {} {\meaningof{\langle a?b \rangle E} = \{ P \in \pi | P \equiv Q | u?(y)P', \\ \and \\\\ \and \\ \;\;\; u \in \meaningof{a}, \forall z.P'\{z/y\} \in \meaningof{E\{z/b\}}\}, \and \\ \meaningof{a!E} = \{ P \in \pi | P \equiv Q | x!\langle P' \rangle, x \in \meaningof{a} P' \in \meaningof{E}\} }
\end{mathpar}

\begin{mathpar}
 \inferrule* [lab=nominal] {} {\meaningof{\quotep{E}} = \{ \quotep{P} \in \quotep{\pi} | P \in \meaningof{E} \}, \and \meaningof{\quotep{P}} = \{ \quotep{Q} \in \quotep{\pi} | P \equiv Q \} \and \\ \meaningof{@\quotep{E}} = \{ P \in \pi | P \equiv @x, x \in \meaningof{E} \}}
\end{mathpar}

\begin{eqnarray*}
  \\
  \meaningof{-} : TS \to ST
\end{eqnarray*}

\begin{eqnarray*}
  \\
  L : TS \to ST
\end{eqnarray*}

\begin{eqnarray*}
  \\
  P \models E \iff P \in \meaningof{E}
\end{eqnarray*}

\begin{eqnarray*}
  P \approx_{L} Q \iff \forall E \in L. P \models E \iff Q \models E
\end{eqnarray*}

\begin{eqnarray*}
  P \approx_{K} Q
\end{eqnarray*}

\begin{eqnarray*}
  P \approx Q
\end{eqnarray*}

$\approx_{K} = \approx = \approx_{L}$

\subsubsection{Contextual duality}

Note that contexts extend the quotation operation to a family of
operations from processes to names. Given a context, $M$, we can
define a \emph{nominal context}, $\quotep{M}$ by $\quotep{M}[P] :=
\quotep{M[P]}$. To foreshadow what is to come we observe that these
operations enjoy a duality with processes very much like the duality
between vectors and maps from vectors to scalars.

Further, because the calculus is essentially higher-order, we have a
correspondence between contexts and processes. More specifically,
given a name $x$ and a context $M$ we can construct $M^{*}_{x}$ such
that 

\begin{mathpar}
  M^{*}_{x} | \lift{x}{P} \red M[P]
\end{mathpar}

namely,

\begin{mathpar}
  M^{*}_{x} := x?(u).M[\dropn{u}]
\end{mathpar}

The dependence of $M^{*}_{x}$ on a name makes it an abstraction, 

\begin{mathpar}
  M^{*} := (x)x?(u).M[\dropn{u}]
\end{mathpar}

\subsection{Additional notation}

It will sometimes be convenient to denote the process a name
quotes. We already have the notation $x = \quotep{P}$, but it will be
convenient to introduce an alternate notation, $\procn{x}$, when we
want to emphasize the connection to the use of the name. Note that, by
virtue of name equivalence, $\quotep{\procn{x}} \nameeq x$; so, the
notation is consistent with previous definitions.

Further, because names have structure it is possible to effect
substitutions on the basis of that structure. This means we need to
upgrade our notation for substitutions, which we accomplish by
adapting comprehension notation. Thus,

\begin{mathpar}
  P\{ y / x : x \in S \}
\end{mathpar}

is interpreted to mean the process derived from P by replacing (in a
capture-avoiding manner) each occurrence of $x$ in $S$ by $y$. For example,

\begin{mathpar}
  P\{ \quotep{\procn{x}|\procn{x}} / x : x \in \freenames{P} \}
\end{mathpar}

will replace each (occurrence) of a free name $x$ in $P$ by
$\quotep{\procn{x}|\procn{x}}$.

Also, we will avail ourselves of the notation $x^{L}$ and $x^{R}$ to
denote injections of a name into disjoint copies of the name
space. There are numerous ways to accomplish this. One example can be
found in \cite{MeredithR05}. This notation overloads to vectors of
names: $\vec{x}^{\pi} := (x_{i}^{\pi} \; : \; 0 \leq i < |\vec{x}| )$ where $\pi \in \{L,R\}$.

We also use $P^{\Box} := P|\Box$.

In \cite{MeredithR05} an interpretation of the new operator is
given. It turns out that there are several possible interpretations
all enjoying the requisite algebraic properties of the operator (see
\cite{milner91polyadicpi}). We will therefore make liberal use of
$(\nu\; \vec{x})P$.

% subsection the_syntax_and_semantics_of_the_notation_system (end)   

\input{qm2pi.qmops} 

\input{qm2pi.sterngerlach} 

\input{qm2pi.metric} 

% section concurrent_process_calculi (end)

%\input{qm2pi.proofsketch}

% section proof sketch (end)

%\input{qm2pi.slviaknots} 

% section spatial logic via knots (end)

\input{qm2pi.conclusion}

% section conclusion (end)

%\input{qm2pi.dtcodes} 

% section wiring algorithm (end)

\input{qm2pi.ack} 

% section acknowledgments (end)

\newpage


\bibliographystyle{plain}   
\bibliography{../../biblios/main.bib}

\input{qm2pi.rhodetails}

\end{document}

 

%\documentclass[12pt]{llncs}
%\documentclass{jktr}

\usepackage[pdftex]{hyperref}                   
\usepackage {listings}
\usepackage {mathpartir}
\usepackage{bcprules}
%\usepackage{listings}
                       
\usepackage{graphicx} 
%\usepackage[margins=2.5cm,nohead,nofoot]{geometry}
%\usepackage{geometry}
\usepackage{amsfonts}
\usepackage{amstext}
\usepackage{latexsym}
\usepackage{amssymb}
\usepackage{color}


%\include{myPreamble}
\include{qm2pi.local} 

%\ifpdf
%\usepackage[pdftex]{graphicx}
%\else
%\usepackage{graphicx}
%\fi

 % \ifpdf
%  \usepackage{pdfsync}
%  \if


%\title{Brief Article}
%\author{David F. Snyder}
%\author{L.G. Meredith}

%\address{Dept. of Math., Texas State University--San Marcos, San Marcos, TX 78666}
       
\pagestyle{empty}


\begin{document}

\lstset{language=[Objective]Caml,frame=shadowbox}

\input{qm2pi.front}

% section front matter (end)

\input{qm2pi.intro} 
 
% section introduction (end)

% \input{qm2pi.knotations} 

% section notation (end)

\input{qm2pi.process.calculi} 

% section concurrent_process_calculi_and_spatial_logics_ (end)
    
%\input{qm2pi.knots2pi} 

%\input{qm2pi.trefoil} 

%\input{qm2pi.mainthm} 

% subsection basic_interpretation (end)

%\input{qm2pi.rho.presentation} 
\subsection{The syntax and semantics of the notation system}\label{sub:the_syntax_and_semantics_of_the_notation_system} % (fold)

We now summarize a technical presentation of the calculus that
embodies our theory of dynamics. The typical presentation of such a
calculus follows the style of giving generators and relations on
them. The grammar, below, describing term constructors, freely
generates the set of processes, $\Proc$. This set is then quotiented
by a relation known as structural congruence and it is over this set
that the notion of dynamics is expressed. This presentation is
essentially that of \cite{MeredithR05} with the addition of
polyadicity and summation. For readability we have relegated some of
the technical subtleties to an appendix.

\subsubsection{Process grammar}\label{subsub:process_grammar}

\begin{mathpar}
  \inferrule* [lab=synchronization] {} {{M} \bc \pzero \;|\; x?F \;|\; x!C }
  \and
  \inferrule* [lab=abstraction] {} {{F} \bc (x)P}
  \and
  \inferrule* [lab=concretion] {} {{C} \bc \langle Q \rangle}
  \and
  \inferrule* [lab=process] {} {{P,Q} \bc M \;| \;P|Q \;|\; @{x}}
  \and
  \inferrule* [lab=name] {} {{x} \bc \quotep{P}}
\end{mathpar} 

Note that $\vec{x}$ (resp. $\vec{P}$) denotes a vector of names
(resp. processes) of length $|\vec{x}|$ (resp. $|\vec{P}|$). We adopt
the following useful abbreviations.

\begin{mathpar}
   x?(\vec{y}).P := x.(\vec{y})P \and  x\clift{\vec{P}} := x.\clift{\vec{P}}
   \and x!(y) := \lift{x}{\dropn{y}}
   \and \Pi_{i=0}^{n-1}P_i := P_0 | \ldots | P_{n-1}
\end{mathpar}

\subsubsection{Structural congruence}

\paragraph{Free and bound names and alpha-equivalence.} At the
core of structural equivalence is alpha-equivalence which identifies
process that are the same up to a change of variable. Formally, we
recognize the distinction between free and bound names. The free names
of a process, $\freenames{P}$, may be calculated recursively as
follows:

\begin{mathpar}
\freenames{\pzero} := \emptyset
  \and \\
  \freenames{x?(y).P} := \{ x \} \cup (\freenames{P} \setminus \{ y \})
  \and 
  \freenames{x!\langle P \rangle} := \{ x \} \cup \{ P \} 
  \and \\
  \freenames{P|Q} := \freenames{P} \cup \freenames{Q}
  \and \\
  \freenames{@{x}} := \{ x \}
\end{mathpar}

$\pi$
$\quotep{\pi}$

$\freenames{-} : \pi \to \mathcal{P}(\quotep{\pi})$

\begin{eqnarray*}
  \freenames{\pzero} & := & \emptyset \\
  \freenames{x?(y).P} & := & \{ x \} \cup (\freenames{P} \setminus \{ y \}) \\
  \freenames{x!\langle P \rangle} & := & \{ x \} \cup \{ P \} \\
  \freenames{P|Q} & := & \freenames{P} \cup \freenames{Q} \\
  \freenames{\dropn{x}} & := & \{ x \}
\end{eqnarray*}

The bound names of a process, $\boundnames{P}$, are those names occurring in $P$
that are not free. For example, in $x?(y).0$, the name $x$ is free, while $y$ is bound.

\begin{mathpar}
  \inferrule* [lab=monoidal-laws] {} { P|Q \equiv Q|P \and P|0 \equiv P \and P|(Q|R) \equiv (P|Q)|R }
\end{mathpar}

\begin{mathpar}
  \inferrule* [lab=alpha-equivalence] {} { (x)P \equiv (y)P\{y/x\} \and y \not\in \freenames{P} }
\end{mathpar}

\begin{definition}
Then two processes, $P,Q$, are alpha-equivalent if $P = Q\{\vec{y}/\vec{x}\}$ for
some $\vec{x} \in \boundnames{Q},\vec{y} \in \boundnames{P}$, where $Q\{\vec{y}/\vec{x}\}$
denotes the capture-avoiding substitution of $\vec{y}$ for $\vec{x}$ in $Q$.
\end{definition}

\begin{definition}
  The {\em structural congruence} \cite{SangiorgiWalker} , $\equiv$,
  between processes is the least congruence containing
  alpha-equivalence, satisfying the abelian monoid laws
  (associativity, commutativity and $\pzero$ as identity) for parallel
  composition $|$ and for summation $+$.
\end{definition}

\subsection{Name equivalence}

We take name equivalence, written $\nameeq$, to be the smallest
equivalence relation generated by the following rules.

\begin{mathpar}
\inferrule*[lab=Quote-drop]
{ }
{ \quotep{@{x}} \nameeq x }

\inferrule*[lab=Struct-equiv]
{ P \scong Q }
{ \quotep{P} \nameeq \quotep{Q} }
\end{mathpar}

The astute reader will have noticed that the mutual recursion of names
and processes imposes a mutual recursion on alpha-equivalence and
structural equivalence via name-equivalence. Fortunately, all of this
works out pleasantly and we may calculate in the natural way, free of
concern. The reader interested in the details is referred to the
appendix \ref{appendix:rho_details}.

\subsection{Substitution}

We use $\Proc$ for the set of processes, $\QProc$ for the set of
names, and $\id{\{}\vec{y} / \vec{x} \id{\}}$ to denote partial maps,
$s : \QProc \rightarrow \QProc$. A map, $s$ lifts, uniquely, to a map
on process terms, $\widehat{s} : \Proc \rightarrow \Proc$ by the
following equations.

\begin{mathpar}
  (0) \psubstp{Q}{P} := 0 \\
  (R \juxtap S) \psubstp{Q}{P}
  :=    
  (R)\psubstp{Q}{P} \juxtap (S) \psubstp{Q}{P} \\
  (x?(y).R) \psubstp{Q}{P}    
  :=    
  (x)\substp{Q}{P} (z)\concat( (R \psubstn{z}{y}) \psubstp{Q}{P} ) \\
  (\lift{x}{R}) \psubstp{Q}{P}  
  :=
  \lift{(x)\substp{Q}{P}}{ R \psubstp{Q}{P} } \\
%   (\dropn{x})  \psubstp{Q}{P}       
%   := 
%   \left\{ 
%     \begin{array}{ccc} 
%       \dropn{\quotep{Q}} & & x \nameeq \quotep{P} \\
%       \dropn{x} & & otherwise \\
%     \end{array}
%   \right. 
  (\dropn{x})  \psubstp{Q}{P}       
  := 
  \left\{ 
    \begin{array}{ccc} 
      Q & & x \nameeq \quotep{P} \\
      \dropn{x} & & otherwise \\
    \end{array}
  \right.
\end{mathpar}
 

where

\begin{eqnarray}
  (x)\id{\{} \lpquote Q \rpquote / \lpquote P \rpquote \id{\}}            = 
  \left\{ 
    \begin{array}{ccc}
      \lpquote Q \rpquote & & x \nameeq \lpquote P \rpquote \\
      x & & otherwise \\
    \end{array}
  \right. \nonumber
\end{eqnarray}

and $z$ is chosen distinct from $\quotep{P}$, $\quotep{Q}$, the free
names in $Q$, and all the names in $R$. Our $\alpha$-equivalence will
be built in the standard way from this substitution.

\begin{remark}\label{rem:no_self_referential_names}
  One consequence of these definitions is that $\forall P. \quotep{P}
  \not\in \freenames{P}$.
\end{remark}

\subsection{ Dynamic quote: an example }

Anticipating something of what's to come, consider applying the
substitution, $\widehat{\id{\{}u / z \id{\}}}$, to the following pair
of processes, $\lift{w}{y!(z)}$ and $w[ \lpquote y!(z) \rpquote ]$.

\begin{eqnarray}
	\lift{w}{y!(z)}\widehat{\id{\{}u / z \id{\}}}
		& = &
		\lift{w}{y!(u)} \nonumber\\
	w[ \lpquote y!(z) \rpquote ] \widehat{ \id{\{}u / z \id{\}} }
		& = &
		w[ \lpquote y!(z) \rpquote ] \nonumber
\end{eqnarray}

Because the body of the process between quotes is impervious to
substitution, we get radically different answers. In fact, by
examining the first process in an input context,
e.g. $x?(z).\lift{w}{y!(z)}$, we see that the process under the lift
operator may be shaped by prefixed inputs binding a name inside it. In
this sense, the lift operator will be seen as a way to dynamically
construct processes before reifying them as names.

Finally equipped with these standard features we can present the
dynamics of the calculus.

\subsubsection{Operational semantics} 

Finally, we introduce the computational dynamics. What marks these
algebras as distinct from other more traditionally studied algebraic
structures, e.g. vector spaces or polynomial rings, is the manner in
which dynamics is captured. In traditional structures, dynamics is typically
expressed through morphisms between such structures, as in linear maps
between vector spaces or morphisms between rings. In algebras
associated with the semantics of computation, the dynamics is
expressed as part of the algebraic structure itself, through a
reduction reduction relation typically denoted by $\red$. Below, we
give a recursive presentation of this relation for the calculus used
in the encoding.

$\red \subseteq \pi \times \pi$
$\red : \pi \to \mathcal{P}(\pi)$

\begin{mathpar}
  \inferrule* [lab=Comm] { \textsf{match}( x_{src}, x_{trgt} ) } { x_{trgt}?(y)P \; | \; x_{src}!\langle {Q} \rangle \red P\{\quotep{Q}/y}\} }
  \and \\
  \inferrule* [lab=Par] {{P} \red {P}'} {{{P} | {Q}} \red {{P}' | {Q}}}
  \and
  \inferrule* [lab=Equiv]{{{P} \scong {P}'} \andalso {{P}' \red {Q}'} \andalso {{Q}' \scong {Q}}}{{P} \red {Q}}
\end{mathpar}

\begin{eqnarray*}
  match_{\equiv} (\quotep{P},\quotep{Q}) & := & P \equiv Q \\
  match_{\dagger}(\quotep{P},\quotep{Q}) & := & \forall R. P|Q \red^{*} R => R \red^{*} 0 \\
  match_{K}(\quotep{P},\quotep{Q}) & := & K \mbox{ for some context } K
\end{eqnarray*}

$u?(x)P | u!\langle Q \rangle \red P\{\quotep{Q}/x\}$

%We write $\wred$ for $\red^*$, and $P\red$ if $\exists Q $ such that $ P \red Q$.
We write $P\red$ if $\exists Q $ such that $ P \red Q$ and $P\not\red$, otherwise.

\section{Replication}

As mentioned before, it is known that replication (and hence
recursion) can be implemented in a higher-order process algebra
\cite{SangiorgiWalker}. As our first example of calculation with the
machinery thus far presented we give the construction explicitly in
the {\rhoc}.

\begin{eqnarray}
	D_{x} & := & \prefix{x}{y}{(\binpar{\outputp{x}{y}}{@{y}})} \nonumber\\
	\bangp_{x}{P} & := & \binpar{{x}!\langle{\binpar{D_{x}}{P}}\rangle}{D_{x}} \nonumber
\end{eqnarray}

\begin{eqnarray}
	\bangp_{x}{P} & & \nonumber\\
	=
	& {x}!\langle{(\prefix{x}{y}{(\outputp{x}{y} | @{y})) | P}}\rangle 
	      | \prefix{x}{y}{(\outputp{x}{y} | @{y})} & \nonumber\\
	\red
	& (\outputp{x}{y} | @{y})\substn{\quotep{(\prefix{x}{y}{(@{y} | \outputp{x}{y})) | P}}}{y} & \nonumber\\
	=
	& \outputp{x}{\quotep{(\prefix{x}{y}{(\outputp{x}{y} | @{y})) | P}}}
	  | {(\prefix{x}{y}{(\outputp{x}{y} | @{y})) | P}} & \nonumber\\
	\red
	& \ldots & \nonumber\\
	\red^*
	& P | P | \ldots & \nonumber
\end{eqnarray}

Of course, this encoding, as an implementation, runs away, unfolding
$\bangp{P}$ eagerly. A lazier and more implementable replication
operator, restricted to input-guarded processes, may be obtained as follows.

\begin{eqnarray}
\bangp{\prefix{u}{v}{P}} 
	:= 
	\binpar{\lift{x}{\prefix{u}{v}{(\binpar{D(x)}{P})}}}{D(x)} \nonumber
\end{eqnarray}

\begin{remark}
  Note that the lazier definition still does not deal with summation
  or mixed summation (i.e. sums over input and output). The reader is
  invited to construct definitions of replication that deal with these
  features. 

  Further, the definitions are parameterized in a name, $x$. Can you,
  gentle reader, make a definition that eliminates this parameter and
  guarantees no accidental interaction between the replication
  machinery and the process being replicated -- i.e. no accidental
  sharing of names used by the process to get its work done and the
  name(s) used by the replication to effect copying. This latter
  revision of the definition of replication is crucial to obtaining
  the expected identity $!!P \sim !P$.
\end{remark}

\begin{remark}\label{rem:paradoxical_combinator}
  The reader familiar with the lambda calculus will have noticed the
  similarity between $D$ and the paradoxical combinator.

  [Ed. note: the existence of this seems to suggest we have to be more
  restrictive on the set of processes and names we admit if we are to
  support no-cloning.]
\end{remark}

\subsubsection{Bisimulation}

The computational dynamics gives rise to another kind of equivalence,
the equivalence of computational behavior. As previously mentioned
this is typically captured \emph{via} some form of bisimulation.

% The notion we use in this paper is weak barbed bisimulation
% \cite{milner91polyadicpi}.

The notion we use in this paper is derived from weak barbed
bisimulation \cite{milner91polyadicpi}. 

\begin{definition}
An \emph{observation relation}, $\downarrow_{\mathcal N}$, over a set
of names, $\mathcal N$, is the smallest relation satisfying the rules
below.

\infrule[Out-barb]{y \in {\mathcal N}, \; x \nameeq y}
		  {\outputp{x}{v} \downarrow_{\mathcal N} x}
\infrule[Par-barb]{\mbox{$P\downarrow_{\mathcal N} x$ or $Q\downarrow_{\mathcal N} x$}}
		  {\binpar{P}{Q} \downarrow_{\mathcal N} x}

We write $P \Downarrow_{\mathcal N} x$ if there is $Q$ such that 
$P \wred Q$ and $Q \downarrow_{\mathcal N} x$.
\end{definition}

\begin{definition}
%\label{def.bbisim}
An  ${\mathcal N}$-\emph{barbed bisimulation} over a set of names, ${\mathcal N}$, is a symmetric binary relation 
${\mathcal S}_{\mathcal N}$ between agents such that $P\rel{S}_{\mathcal N}Q$ implies:
\begin{enumerate}
\item If $P \red P'$ then $Q \wred Q'$ and $P'\rel{S}_{\mathcal N} Q'$.
\item If $P\downarrow_{\mathcal N} x$, then $Q\Downarrow_{\mathcal N} x$.
\end{enumerate}
$P$ is ${\mathcal N}$-barbed bisimilar to $Q$, written
$P \wbbisim_{\mathcal N} Q$, if $P \rel{S}_{\mathcal N} Q$ for some ${\mathcal N}$-barbed bisimulation ${\mathcal S}_{\mathcal N}$.
\end{definition}

$\mathcal{R} \subseteq \pi \times \pi$

$P \mathcal{R} Q => \forall P'. P \red P' \Rightarrow \exists Q'. Q \red Q', P' \mathcal{R} Q'$

$P \vdash x \Rightarrow Q \vdash x$

\begin{mathpar}
  \inferrule*[lab=Out-barb]{x \nameeq y}{{y}!\langle{Q}\rangle \vdash x}
  \and
  \inferrule*[lab=Par-barb]{\mbox{$P\vdash x$ or $Q\vdash x$}}{\binpar{P}{Q} \vdash x}
\end{mathpar}

\subsubsection{Contexts}

One of the principle advantages of computational calculi like the
$\pi$-calculus is a well-defined notion of context,
contextual-equivalence and a correlation between
contextual-equivalence and notions of bisimulation. The notion of
context allows the decomposition of a process into (sub-)process and
its syntactic environment, its context. Thus, a context may be
thought of as a process with a ``hole'' (written $\Box$) in it. The
application of a context $M$ to a process $P$, written $M[P]$, is
tantamount to filling the hole in $M$ with $P$. In this paper we do
not need the full weight of this theory, but do make use of the notion
of context in the proof the main theorem. 

\begin{mathpar}
  \inferrule* [lab=summation] {} {{M_{M},M_{N}} \bc \Box \;|\; x.M_{A} \;|\; M_{M}+M_{N}}
  \and
  \inferrule* [lab=agent] {} {{M_{A}} \bc (\vec{x})M_{P} \;| \; \clift{P_0,\ldots,M_{P},\ldots,P_N}}
  \and \\
  \inferrule* [lab=process] {} {{M_{P}} \bc M_{N} \;| \;P|M_{P} }
\end{mathpar} 

\begin{mathpar}
  \inferrule* [lab=sychronization] {} {M_{N} \bc \Box \;|\; x?M_{F} \;|\; x!M_{C}}
  \and
  \inferrule* [lab=abstraction] {} {{M_{F}} \bc (x)M_{P} }
  \and
  \inferrule* [lab=concretion] {} {{M_{C}} \bc \langle M_{P} \rangle }
  \and \\
  \inferrule* [lab=process] {} {{M_{P}} \bc M_{N} \;| \;P|M_{P} }
\end{mathpar}

\begin{definition}[contextual application] Given a context $M$, and
  process $P$, we define the \emph{contextual application}, $M[P] :=
  M\{P/\Box\}$. That is, the contextual application of M to P is the
  substitution of $P$ for $\Box$ in $M$.
\end{definition}

$\meaningof{-} : L \to \mathcal{P}(\pi)$

\begin{mathpar}
  \inferrule* [lab=collection] {} {\meaningof{true} = \pi, \and \meaningof{~E} = \pi \setminus \meaningof{E}, \and \meaningof{E_{1} \& E_{2}} = \meaningof{E_{1}} \cap \meaningof{E_{2}}}
\end{mathpar}

\begin{mathpar}
  \inferrule* [lab=structure] {} {\meaningof{0} = \{ P \in \pi | P \equiv 0 \}, \and \\ \meaningof{E_1 | E_2} = \{ P \in \pi | P \equiv P_{1} | P_{2}, P_{1} \in \meaningof{E_{1}}, P_{2} \in \meaningof{E_2}\} }
\end{mathpar}

\begin{mathpar}
 \inferrule* [lab=behavior] {} {\meaningof{\langle a?b \rangle E} = \{ P \in \pi | P \equiv Q | u?(y)P', \\ \and \\\\ \and \\ \;\;\; u \in \meaningof{a}, \forall z.P'\{z/y\} \in \meaningof{E\{z/b\}}\}, \and \\ \meaningof{a!E} = \{ P \in \pi | P \equiv Q | x!\langle P' \rangle, x \in \meaningof{a} P' \in \meaningof{E}\} }
\end{mathpar}

\begin{mathpar}
 \inferrule* [lab=nominal] {} {\meaningof{\quotep{E}} = \{ \quotep{P} \in \quotep{\pi} | P \in \meaningof{E} \}, \and \meaningof{\quotep{P}} = \{ \quotep{Q} \in \quotep{\pi} | P \equiv Q \} \and \\ \meaningof{@\quotep{E}} = \{ P \in \pi | P \equiv @x, x \in \meaningof{E} \}}
\end{mathpar}

\begin{eqnarray*}
  \\
  \meaningof{-} : TS \to ST
\end{eqnarray*}

\begin{eqnarray*}
  \\
  L : TS \to ST
\end{eqnarray*}

\begin{eqnarray*}
  \\
  P \models E \iff P \in \meaningof{E}
\end{eqnarray*}

\begin{eqnarray*}
  P \approx_{L} Q \iff \forall E \in L. P \models E \iff Q \models E
\end{eqnarray*}

\begin{eqnarray*}
  P \approx_{K} Q
\end{eqnarray*}

\begin{eqnarray*}
  P \approx Q
\end{eqnarray*}

$\approx_{K} = \approx = \approx_{L}$

\subsubsection{Contextual duality}

Note that contexts extend the quotation operation to a family of
operations from processes to names. Given a context, $M$, we can
define a \emph{nominal context}, $\quotep{M}$ by $\quotep{M}[P] :=
\quotep{M[P]}$. To foreshadow what is to come we observe that these
operations enjoy a duality with processes very much like the duality
between vectors and maps from vectors to scalars.

Further, because the calculus is essentially higher-order, we have a
correspondence between contexts and processes. More specifically,
given a name $x$ and a context $M$ we can construct $M^{*}_{x}$ such
that 

\begin{mathpar}
  M^{*}_{x} | \lift{x}{P} \red M[P]
\end{mathpar}

namely,

\begin{mathpar}
  M^{*}_{x} := x?(u).M[\dropn{u}]
\end{mathpar}

The dependence of $M^{*}_{x}$ on a name makes it an abstraction, 

\begin{mathpar}
  M^{*} := (x)x?(u).M[\dropn{u}]
\end{mathpar}

\subsection{Additional notation}

It will sometimes be convenient to denote the process a name
quotes. We already have the notation $x = \quotep{P}$, but it will be
convenient to introduce an alternate notation, $\procn{x}$, when we
want to emphasize the connection to the use of the name. Note that, by
virtue of name equivalence, $\quotep{\procn{x}} \nameeq x$; so, the
notation is consistent with previous definitions.

Further, because names have structure it is possible to effect
substitutions on the basis of that structure. This means we need to
upgrade our notation for substitutions, which we accomplish by
adapting comprehension notation. Thus,

\begin{mathpar}
  P\{ y / x : x \in S \}
\end{mathpar}

is interpreted to mean the process derived from P by replacing (in a
capture-avoiding manner) each occurrence of $x$ in $S$ by $y$. For example,

\begin{mathpar}
  P\{ \quotep{\procn{x}|\procn{x}} / x : x \in \freenames{P} \}
\end{mathpar}

will replace each (occurrence) of a free name $x$ in $P$ by
$\quotep{\procn{x}|\procn{x}}$.

Also, we will avail ourselves of the notation $x^{L}$ and $x^{R}$ to
denote injections of a name into disjoint copies of the name
space. There are numerous ways to accomplish this. One example can be
found in \cite{MeredithR05}. This notation overloads to vectors of
names: $\vec{x}^{\pi} := (x_{i}^{\pi} \; : \; 0 \leq i < |\vec{x}| )$ where $\pi \in \{L,R\}$.

We also use $P^{\Box} := P|\Box$.

In \cite{MeredithR05} an interpretation of the new operator is
given. It turns out that there are several possible interpretations
all enjoying the requisite algebraic properties of the operator (see
\cite{milner91polyadicpi}). We will therefore make liberal use of
$(\nu\; \vec{x})P$.

% subsection the_syntax_and_semantics_of_the_notation_system (end)   

\input{qm2pi.qmops} 

\input{qm2pi.sterngerlach} 

\input{qm2pi.metric} 

% section concurrent_process_calculi (end)

%\input{qm2pi.proofsketch}

% section proof sketch (end)

%\input{qm2pi.slviaknots} 

% section spatial logic via knots (end)

\input{qm2pi.conclusion}

% section conclusion (end)

%\input{qm2pi.dtcodes} 

% section wiring algorithm (end)

\input{qm2pi.ack} 

% section acknowledgments (end)

\newpage


\bibliographystyle{plain}   
\bibliography{../../biblios/main.bib}

\input{qm2pi.rhodetails}

\end{document}

 

% subsection basic_interpretation (end)

%\input{qm2pi.rho.presentation} 
\subsection{The syntax and semantics of the notation system}\label{sub:the_syntax_and_semantics_of_the_notation_system} % (fold)

We now summarize a technical presentation of the calculus that
embodies our theory of dynamics. The typical presentation of such a
calculus follows the style of giving generators and relations on
them. The grammar, below, describing term constructors, freely
generates the set of processes, $\Proc$. This set is then quotiented
by a relation known as structural congruence and it is over this set
that the notion of dynamics is expressed. This presentation is
essentially that of \cite{MeredithR05} with the addition of
polyadicity and summation. For readability we have relegated some of
the technical subtleties to an appendix.

\subsubsection{Process grammar}\label{subsub:process_grammar}

\begin{mathpar}
  \inferrule* [lab=synchronization] {} {{M} \bc \pzero \;|\; x?F \;|\; x!C }
  \and
  \inferrule* [lab=abstraction] {} {{F} \bc (x)P}
  \and
  \inferrule* [lab=concretion] {} {{C} \bc \langle Q \rangle}
  \and
  \inferrule* [lab=process] {} {{P,Q} \bc M \;| \;P|Q \;|\; @{x}}
  \and
  \inferrule* [lab=name] {} {{x} \bc \quotep{P}}
\end{mathpar} 

Note that $\vec{x}$ (resp. $\vec{P}$) denotes a vector of names
(resp. processes) of length $|\vec{x}|$ (resp. $|\vec{P}|$). We adopt
the following useful abbreviations.

\begin{mathpar}
   x?(\vec{y}).P := x.(\vec{y})P \and  x\clift{\vec{P}} := x.\clift{\vec{P}}
   \and x!(y) := \lift{x}{\dropn{y}}
   \and \Pi_{i=0}^{n-1}P_i := P_0 | \ldots | P_{n-1}
\end{mathpar}

\subsubsection{Structural congruence}

\paragraph{Free and bound names and alpha-equivalence.} At the
core of structural equivalence is alpha-equivalence which identifies
process that are the same up to a change of variable. Formally, we
recognize the distinction between free and bound names. The free names
of a process, $\freenames{P}$, may be calculated recursively as
follows:

\begin{mathpar}
\freenames{\pzero} := \emptyset
  \and \\
  \freenames{x?(y).P} := \{ x \} \cup (\freenames{P} \setminus \{ y \})
  \and 
  \freenames{x!\langle P \rangle} := \{ x \} \cup \{ P \} 
  \and \\
  \freenames{P|Q} := \freenames{P} \cup \freenames{Q}
  \and \\
  \freenames{@{x}} := \{ x \}
\end{mathpar}

$\pi$
$\quotep{\pi}$

$\freenames{-} : \pi \to \mathcal{P}(\quotep{\pi})$

\begin{eqnarray*}
  \freenames{\pzero} & := & \emptyset \\
  \freenames{x?(y).P} & := & \{ x \} \cup (\freenames{P} \setminus \{ y \}) \\
  \freenames{x!\langle P \rangle} & := & \{ x \} \cup \{ P \} \\
  \freenames{P|Q} & := & \freenames{P} \cup \freenames{Q} \\
  \freenames{\dropn{x}} & := & \{ x \}
\end{eqnarray*}

The bound names of a process, $\boundnames{P}$, are those names occurring in $P$
that are not free. For example, in $x?(y).0$, the name $x$ is free, while $y$ is bound.

\begin{mathpar}
  \inferrule* [lab=monoidal-laws] {} { P|Q \equiv Q|P \and P|0 \equiv P \and P|(Q|R) \equiv (P|Q)|R }
\end{mathpar}

\begin{mathpar}
  \inferrule* [lab=alpha-equivalence] {} { (x)P \equiv (y)P\{y/x\} \and y \not\in \freenames{P} }
\end{mathpar}

\begin{definition}
Then two processes, $P,Q$, are alpha-equivalent if $P = Q\{\vec{y}/\vec{x}\}$ for
some $\vec{x} \in \boundnames{Q},\vec{y} \in \boundnames{P}$, where $Q\{\vec{y}/\vec{x}\}$
denotes the capture-avoiding substitution of $\vec{y}$ for $\vec{x}$ in $Q$.
\end{definition}

\begin{definition}
  The {\em structural congruence} \cite{SangiorgiWalker} , $\equiv$,
  between processes is the least congruence containing
  alpha-equivalence, satisfying the abelian monoid laws
  (associativity, commutativity and $\pzero$ as identity) for parallel
  composition $|$ and for summation $+$.
\end{definition}

\subsection{Name equivalence}

We take name equivalence, written $\nameeq$, to be the smallest
equivalence relation generated by the following rules.

\begin{mathpar}
\inferrule*[lab=Quote-drop]
{ }
{ \quotep{@{x}} \nameeq x }

\inferrule*[lab=Struct-equiv]
{ P \scong Q }
{ \quotep{P} \nameeq \quotep{Q} }
\end{mathpar}

The astute reader will have noticed that the mutual recursion of names
and processes imposes a mutual recursion on alpha-equivalence and
structural equivalence via name-equivalence. Fortunately, all of this
works out pleasantly and we may calculate in the natural way, free of
concern. The reader interested in the details is referred to the
appendix \ref{appendix:rho_details}.

\subsection{Substitution}

We use $\Proc$ for the set of processes, $\QProc$ for the set of
names, and $\id{\{}\vec{y} / \vec{x} \id{\}}$ to denote partial maps,
$s : \QProc \rightarrow \QProc$. A map, $s$ lifts, uniquely, to a map
on process terms, $\widehat{s} : \Proc \rightarrow \Proc$ by the
following equations.

\begin{mathpar}
  (0) \psubstp{Q}{P} := 0 \\
  (R \juxtap S) \psubstp{Q}{P}
  :=    
  (R)\psubstp{Q}{P} \juxtap (S) \psubstp{Q}{P} \\
  (x?(y).R) \psubstp{Q}{P}    
  :=    
  (x)\substp{Q}{P} (z)\concat( (R \psubstn{z}{y}) \psubstp{Q}{P} ) \\
  (\lift{x}{R}) \psubstp{Q}{P}  
  :=
  \lift{(x)\substp{Q}{P}}{ R \psubstp{Q}{P} } \\
%   (\dropn{x})  \psubstp{Q}{P}       
%   := 
%   \left\{ 
%     \begin{array}{ccc} 
%       \dropn{\quotep{Q}} & & x \nameeq \quotep{P} \\
%       \dropn{x} & & otherwise \\
%     \end{array}
%   \right. 
  (\dropn{x})  \psubstp{Q}{P}       
  := 
  \left\{ 
    \begin{array}{ccc} 
      Q & & x \nameeq \quotep{P} \\
      \dropn{x} & & otherwise \\
    \end{array}
  \right.
\end{mathpar}
 

where

\begin{eqnarray}
  (x)\id{\{} \lpquote Q \rpquote / \lpquote P \rpquote \id{\}}            = 
  \left\{ 
    \begin{array}{ccc}
      \lpquote Q \rpquote & & x \nameeq \lpquote P \rpquote \\
      x & & otherwise \\
    \end{array}
  \right. \nonumber
\end{eqnarray}

and $z$ is chosen distinct from $\quotep{P}$, $\quotep{Q}$, the free
names in $Q$, and all the names in $R$. Our $\alpha$-equivalence will
be built in the standard way from this substitution.

\begin{remark}\label{rem:no_self_referential_names}
  One consequence of these definitions is that $\forall P. \quotep{P}
  \not\in \freenames{P}$.
\end{remark}

\subsection{ Dynamic quote: an example }

Anticipating something of what's to come, consider applying the
substitution, $\widehat{\id{\{}u / z \id{\}}}$, to the following pair
of processes, $\lift{w}{y!(z)}$ and $w[ \lpquote y!(z) \rpquote ]$.

\begin{eqnarray}
	\lift{w}{y!(z)}\widehat{\id{\{}u / z \id{\}}}
		& = &
		\lift{w}{y!(u)} \nonumber\\
	w[ \lpquote y!(z) \rpquote ] \widehat{ \id{\{}u / z \id{\}} }
		& = &
		w[ \lpquote y!(z) \rpquote ] \nonumber
\end{eqnarray}

Because the body of the process between quotes is impervious to
substitution, we get radically different answers. In fact, by
examining the first process in an input context,
e.g. $x?(z).\lift{w}{y!(z)}$, we see that the process under the lift
operator may be shaped by prefixed inputs binding a name inside it. In
this sense, the lift operator will be seen as a way to dynamically
construct processes before reifying them as names.

Finally equipped with these standard features we can present the
dynamics of the calculus.

\subsubsection{Operational semantics} 

Finally, we introduce the computational dynamics. What marks these
algebras as distinct from other more traditionally studied algebraic
structures, e.g. vector spaces or polynomial rings, is the manner in
which dynamics is captured. In traditional structures, dynamics is typically
expressed through morphisms between such structures, as in linear maps
between vector spaces or morphisms between rings. In algebras
associated with the semantics of computation, the dynamics is
expressed as part of the algebraic structure itself, through a
reduction reduction relation typically denoted by $\red$. Below, we
give a recursive presentation of this relation for the calculus used
in the encoding.

$\red \subseteq \pi \times \pi$
$\red : \pi \to \mathcal{P}(\pi)$

\begin{mathpar}
  \inferrule* [lab=Comm] { \textsf{match}( x_{src}, x_{trgt} ) } { x_{trgt}?(y)P \; | \; x_{src}!\langle {Q} \rangle \red P\{\quotep{Q}/y}\} }
  \and \\
  \inferrule* [lab=Par] {{P} \red {P}'} {{{P} | {Q}} \red {{P}' | {Q}}}
  \and
  \inferrule* [lab=Equiv]{{{P} \scong {P}'} \andalso {{P}' \red {Q}'} \andalso {{Q}' \scong {Q}}}{{P} \red {Q}}
\end{mathpar}

\begin{eqnarray*}
  match_{\equiv} (\quotep{P},\quotep{Q}) & := & P \equiv Q \\
  match_{\dagger}(\quotep{P},\quotep{Q}) & := & \forall R. P|Q \red^{*} R => R \red^{*} 0 \\
  match_{K}(\quotep{P},\quotep{Q}) & := & K \mbox{ for some context } K
\end{eqnarray*}

$u?(x)P | u!\langle Q \rangle \red P\{\quotep{Q}/x\}$

%We write $\wred$ for $\red^*$, and $P\red$ if $\exists Q $ such that $ P \red Q$.
We write $P\red$ if $\exists Q $ such that $ P \red Q$ and $P\not\red$, otherwise.

\section{Replication}

As mentioned before, it is known that replication (and hence
recursion) can be implemented in a higher-order process algebra
\cite{SangiorgiWalker}. As our first example of calculation with the
machinery thus far presented we give the construction explicitly in
the {\rhoc}.

\begin{eqnarray}
	D_{x} & := & \prefix{x}{y}{(\binpar{\outputp{x}{y}}{@{y}})} \nonumber\\
	\bangp_{x}{P} & := & \binpar{{x}!\langle{\binpar{D_{x}}{P}}\rangle}{D_{x}} \nonumber
\end{eqnarray}

\begin{eqnarray}
	\bangp_{x}{P} & & \nonumber\\
	=
	& {x}!\langle{(\prefix{x}{y}{(\outputp{x}{y} | @{y})) | P}}\rangle 
	      | \prefix{x}{y}{(\outputp{x}{y} | @{y})} & \nonumber\\
	\red
	& (\outputp{x}{y} | @{y})\substn{\quotep{(\prefix{x}{y}{(@{y} | \outputp{x}{y})) | P}}}{y} & \nonumber\\
	=
	& \outputp{x}{\quotep{(\prefix{x}{y}{(\outputp{x}{y} | @{y})) | P}}}
	  | {(\prefix{x}{y}{(\outputp{x}{y} | @{y})) | P}} & \nonumber\\
	\red
	& \ldots & \nonumber\\
	\red^*
	& P | P | \ldots & \nonumber
\end{eqnarray}

Of course, this encoding, as an implementation, runs away, unfolding
$\bangp{P}$ eagerly. A lazier and more implementable replication
operator, restricted to input-guarded processes, may be obtained as follows.

\begin{eqnarray}
\bangp{\prefix{u}{v}{P}} 
	:= 
	\binpar{\lift{x}{\prefix{u}{v}{(\binpar{D(x)}{P})}}}{D(x)} \nonumber
\end{eqnarray}

\begin{remark}
  Note that the lazier definition still does not deal with summation
  or mixed summation (i.e. sums over input and output). The reader is
  invited to construct definitions of replication that deal with these
  features. 

  Further, the definitions are parameterized in a name, $x$. Can you,
  gentle reader, make a definition that eliminates this parameter and
  guarantees no accidental interaction between the replication
  machinery and the process being replicated -- i.e. no accidental
  sharing of names used by the process to get its work done and the
  name(s) used by the replication to effect copying. This latter
  revision of the definition of replication is crucial to obtaining
  the expected identity $!!P \sim !P$.
\end{remark}

\begin{remark}\label{rem:paradoxical_combinator}
  The reader familiar with the lambda calculus will have noticed the
  similarity between $D$ and the paradoxical combinator.

  [Ed. note: the existence of this seems to suggest we have to be more
  restrictive on the set of processes and names we admit if we are to
  support no-cloning.]
\end{remark}

\subsubsection{Bisimulation}

The computational dynamics gives rise to another kind of equivalence,
the equivalence of computational behavior. As previously mentioned
this is typically captured \emph{via} some form of bisimulation.

% The notion we use in this paper is weak barbed bisimulation
% \cite{milner91polyadicpi}.

The notion we use in this paper is derived from weak barbed
bisimulation \cite{milner91polyadicpi}. 

\begin{definition}
An \emph{observation relation}, $\downarrow_{\mathcal N}$, over a set
of names, $\mathcal N$, is the smallest relation satisfying the rules
below.

\infrule[Out-barb]{y \in {\mathcal N}, \; x \nameeq y}
		  {\outputp{x}{v} \downarrow_{\mathcal N} x}
\infrule[Par-barb]{\mbox{$P\downarrow_{\mathcal N} x$ or $Q\downarrow_{\mathcal N} x$}}
		  {\binpar{P}{Q} \downarrow_{\mathcal N} x}

We write $P \Downarrow_{\mathcal N} x$ if there is $Q$ such that 
$P \wred Q$ and $Q \downarrow_{\mathcal N} x$.
\end{definition}

\begin{definition}
%\label{def.bbisim}
An  ${\mathcal N}$-\emph{barbed bisimulation} over a set of names, ${\mathcal N}$, is a symmetric binary relation 
${\mathcal S}_{\mathcal N}$ between agents such that $P\rel{S}_{\mathcal N}Q$ implies:
\begin{enumerate}
\item If $P \red P'$ then $Q \wred Q'$ and $P'\rel{S}_{\mathcal N} Q'$.
\item If $P\downarrow_{\mathcal N} x$, then $Q\Downarrow_{\mathcal N} x$.
\end{enumerate}
$P$ is ${\mathcal N}$-barbed bisimilar to $Q$, written
$P \wbbisim_{\mathcal N} Q$, if $P \rel{S}_{\mathcal N} Q$ for some ${\mathcal N}$-barbed bisimulation ${\mathcal S}_{\mathcal N}$.
\end{definition}

$\mathcal{R} \subseteq \pi \times \pi$

$P \mathcal{R} Q => \forall P'. P \red P' \Rightarrow \exists Q'. Q \red Q', P' \mathcal{R} Q'$

$P \vdash x \Rightarrow Q \vdash x$

\begin{mathpar}
  \inferrule*[lab=Out-barb]{x \nameeq y}{{y}!\langle{Q}\rangle \vdash x}
  \and
  \inferrule*[lab=Par-barb]{\mbox{$P\vdash x$ or $Q\vdash x$}}{\binpar{P}{Q} \vdash x}
\end{mathpar}

\subsubsection{Contexts}

One of the principle advantages of computational calculi like the
$\pi$-calculus is a well-defined notion of context,
contextual-equivalence and a correlation between
contextual-equivalence and notions of bisimulation. The notion of
context allows the decomposition of a process into (sub-)process and
its syntactic environment, its context. Thus, a context may be
thought of as a process with a ``hole'' (written $\Box$) in it. The
application of a context $M$ to a process $P$, written $M[P]$, is
tantamount to filling the hole in $M$ with $P$. In this paper we do
not need the full weight of this theory, but do make use of the notion
of context in the proof the main theorem. 

\begin{mathpar}
  \inferrule* [lab=summation] {} {{M_{M},M_{N}} \bc \Box \;|\; x.M_{A} \;|\; M_{M}+M_{N}}
  \and
  \inferrule* [lab=agent] {} {{M_{A}} \bc (\vec{x})M_{P} \;| \; \clift{P_0,\ldots,M_{P},\ldots,P_N}}
  \and \\
  \inferrule* [lab=process] {} {{M_{P}} \bc M_{N} \;| \;P|M_{P} }
\end{mathpar} 

\begin{mathpar}
  \inferrule* [lab=sychronization] {} {M_{N} \bc \Box \;|\; x?M_{F} \;|\; x!M_{C}}
  \and
  \inferrule* [lab=abstraction] {} {{M_{F}} \bc (x)M_{P} }
  \and
  \inferrule* [lab=concretion] {} {{M_{C}} \bc \langle M_{P} \rangle }
  \and \\
  \inferrule* [lab=process] {} {{M_{P}} \bc M_{N} \;| \;P|M_{P} }
\end{mathpar}

\begin{definition}[contextual application] Given a context $M$, and
  process $P$, we define the \emph{contextual application}, $M[P] :=
  M\{P/\Box\}$. That is, the contextual application of M to P is the
  substitution of $P$ for $\Box$ in $M$.
\end{definition}

$\meaningof{-} : L \to \mathcal{P}(\pi)$

\begin{mathpar}
  \inferrule* [lab=collection] {} {\meaningof{true} = \pi, \and \meaningof{~E} = \pi \setminus \meaningof{E}, \and \meaningof{E_{1} \& E_{2}} = \meaningof{E_{1}} \cap \meaningof{E_{2}}}
\end{mathpar}

\begin{mathpar}
  \inferrule* [lab=structure] {} {\meaningof{0} = \{ P \in \pi | P \equiv 0 \}, \and \\ \meaningof{E_1 | E_2} = \{ P \in \pi | P \equiv P_{1} | P_{2}, P_{1} \in \meaningof{E_{1}}, P_{2} \in \meaningof{E_2}\} }
\end{mathpar}

\begin{mathpar}
 \inferrule* [lab=behavior] {} {\meaningof{\langle a?b \rangle E} = \{ P \in \pi | P \equiv Q | u?(y)P', \\ \and \\\\ \and \\ \;\;\; u \in \meaningof{a}, \forall z.P'\{z/y\} \in \meaningof{E\{z/b\}}\}, \and \\ \meaningof{a!E} = \{ P \in \pi | P \equiv Q | x!\langle P' \rangle, x \in \meaningof{a} P' \in \meaningof{E}\} }
\end{mathpar}

\begin{mathpar}
 \inferrule* [lab=nominal] {} {\meaningof{\quotep{E}} = \{ \quotep{P} \in \quotep{\pi} | P \in \meaningof{E} \}, \and \meaningof{\quotep{P}} = \{ \quotep{Q} \in \quotep{\pi} | P \equiv Q \} \and \\ \meaningof{@\quotep{E}} = \{ P \in \pi | P \equiv @x, x \in \meaningof{E} \}}
\end{mathpar}

\begin{eqnarray*}
  \\
  \meaningof{-} : TS \to ST
\end{eqnarray*}

\begin{eqnarray*}
  \\
  L : TS \to ST
\end{eqnarray*}

\begin{eqnarray*}
  \\
  P \models E \iff P \in \meaningof{E}
\end{eqnarray*}

\begin{eqnarray*}
  P \approx_{L} Q \iff \forall E \in L. P \models E \iff Q \models E
\end{eqnarray*}

\begin{eqnarray*}
  P \approx_{K} Q
\end{eqnarray*}

\begin{eqnarray*}
  P \approx Q
\end{eqnarray*}

$\approx_{K} = \approx = \approx_{L}$

\subsubsection{Contextual duality}

Note that contexts extend the quotation operation to a family of
operations from processes to names. Given a context, $M$, we can
define a \emph{nominal context}, $\quotep{M}$ by $\quotep{M}[P] :=
\quotep{M[P]}$. To foreshadow what is to come we observe that these
operations enjoy a duality with processes very much like the duality
between vectors and maps from vectors to scalars.

Further, because the calculus is essentially higher-order, we have a
correspondence between contexts and processes. More specifically,
given a name $x$ and a context $M$ we can construct $M^{*}_{x}$ such
that 

\begin{mathpar}
  M^{*}_{x} | \lift{x}{P} \red M[P]
\end{mathpar}

namely,

\begin{mathpar}
  M^{*}_{x} := x?(u).M[\dropn{u}]
\end{mathpar}

The dependence of $M^{*}_{x}$ on a name makes it an abstraction, 

\begin{mathpar}
  M^{*} := (x)x?(u).M[\dropn{u}]
\end{mathpar}

\subsection{Additional notation}

It will sometimes be convenient to denote the process a name
quotes. We already have the notation $x = \quotep{P}$, but it will be
convenient to introduce an alternate notation, $\procn{x}$, when we
want to emphasize the connection to the use of the name. Note that, by
virtue of name equivalence, $\quotep{\procn{x}} \nameeq x$; so, the
notation is consistent with previous definitions.

Further, because names have structure it is possible to effect
substitutions on the basis of that structure. This means we need to
upgrade our notation for substitutions, which we accomplish by
adapting comprehension notation. Thus,

\begin{mathpar}
  P\{ y / x : x \in S \}
\end{mathpar}

is interpreted to mean the process derived from P by replacing (in a
capture-avoiding manner) each occurrence of $x$ in $S$ by $y$. For example,

\begin{mathpar}
  P\{ \quotep{\procn{x}|\procn{x}} / x : x \in \freenames{P} \}
\end{mathpar}

will replace each (occurrence) of a free name $x$ in $P$ by
$\quotep{\procn{x}|\procn{x}}$.

Also, we will avail ourselves of the notation $x^{L}$ and $x^{R}$ to
denote injections of a name into disjoint copies of the name
space. There are numerous ways to accomplish this. One example can be
found in \cite{MeredithR05}. This notation overloads to vectors of
names: $\vec{x}^{\pi} := (x_{i}^{\pi} \; : \; 0 \leq i < |\vec{x}| )$ where $\pi \in \{L,R\}$.

We also use $P^{\Box} := P|\Box$.

In \cite{MeredithR05} an interpretation of the new operator is
given. It turns out that there are several possible interpretations
all enjoying the requisite algebraic properties of the operator (see
\cite{milner91polyadicpi}). We will therefore make liberal use of
$(\nu\; \vec{x})P$.

% subsection the_syntax_and_semantics_of_the_notation_system (end)   

\section{Interpretation of QM}
\subsection{Supporting definitions}
\subsubsection{Multiplication}
\begin{mathpar}
  \quotep{Q} \cdot \quotep{R} := \quotep{Q|R}
  \and \\
  \quotep{Q} \cdot P := P\{ \quotep{Q|R} / \quotep{R} : \quotep{R} \in \freenames{P} \}
\end{mathpar}

\paragraph{Discussion}
The first line needs little explanation. The second line says that
each free name of the process is replaced with the multiplication of
that name by the scalar. Multiplication of a scalar (name) by a state
(process) results in a process all the names of which have been `moved
over' by parallel composition with the process the scalar
quotes. There is a subtlety that the bound names have to be
manipulated so that multiplied names aren't accidentally
captured. There are many ways to achieve this.

\begin{remark}\label{rem:multiplication_identities}
  The reader is invited to verify that for all $x,y,z \in \QProc$ and $P \in \Proc$
  \begin{mathpar}
    x \cdot \quotep{0} \equiv x 
    \and
    x \cdot y \equiv y \cdot x
    \and
    x \cdot (y \cdot z) \equiv (x \cdot y) \cdot z
    \and \\
    \quotep{0} \cdot P \equiv P
    \and \\
    x \cdot (y \cdot P) \equiv (x \cdot y) \cdot P
    \and \\
    x \cdot (P|Q) \equiv (x \cdot P) | (x \cdot Q)
    \and \\    
  \end{mathpar}
\end{remark}

\subsubsection{Tensor product}

We define a tensor product on processes by structural induction.

\paragraph{Tensor of sums} First note that all summations, including
$\pzero$ and sequence, can be written $\Sigma_{i} x_{i}.A_{i} +
\Sigma_{j} x_{j}.C_{j}$, where we have grouped input-guarded processes
together and output-guarded processes together.

Thus, we can define the tensor product of two summations, $N_{1}\otimes N_{2}$, where

\begin{mathpar}
  N_{1} := \Sigma_{i} x_{i}.A_{i} + \Sigma_{j} x_{j}.C_{j}
  \and
  N_{2} := \Sigma_{i'} y_{i'}.B_{i'} + \Sigma_{j'} y_{j'}.D_{j'} 
\end{mathpar}

as follows.

\begin{mathpar}
  \Sigma_{i} x_{i}.A_{i} + \Sigma_{j} x_{j}.C_{j} \otimes \Sigma_{i'}
  y_{i'}.B_{i'} + \Sigma_{j'} y_{j'}.D_{j'} 
  \and \\
  := \; \Sigma_{i} \Sigma_{i'} \quotep{\stackrel{\vee}{x_{i}}| \stackrel{\vee}{y_{i'}}}.(A_{i}\otimes B_{i'}) \; | \; \Sigma_{i'} \Sigma_{i} \quotep{\stackrel{\vee}{y_{i'}}|\stackrel{\vee}{x_{i}}}.(B_{i'}\otimes A_{i})
  \and
  \;\; | \;\; \Sigma_{j} \Sigma_{j'} \quotep{\stackrel{\vee}{x_{j}}|\stackrel{\vee}{y_{j'}}}.(A_{j}\otimes B_{j'}) \; | \; \Sigma_{j'} \Sigma_{j} \quotep{\stackrel{\vee}{y_{j'}}|\stackrel{\vee}{x_{j}}}.(B_{j'}\otimes A_{j})
\end{mathpar}

\begin{remark}
  Do we need to $x^{L}$ and $y^{R}$ for this construction as well?
\end{remark}

\paragraph{Tensor of parallel compositions} Next, we distribute tensor
over par.

\begin{mathpar}
  P_{1}|P_{2} \otimes Q_{1}|Q_{2} := (P_{1} \otimes Q_{1}) | (P_{1}
  \otimes Q_{2}) | (P_{2} \otimes Q_{1}) | (P_{2} \otimes Q_{2})
\end{mathpar}

\paragraph{Tensor with dropped names} We treat tensor of a
process with a dropped name as parallel composition.

\begin{mathpar}
  P \otimes \dropn{x} := P | \dropn{x}
\end{mathpar}

\paragraph{Tensor of agents}

Finally, we need to define tensor on agents. Note that the definition
of tensor on normal products only tensors inputs with inputs and
outputs with outputs. Thus, we only have to define the operation on
``homogeneous'' pairings.

\begin{mathpar}
  (\vec{x})P \otimes (\vec{y})Q
  \and \\
  := (x_{0}^{L}|y_{0}^{R},\ldots,x_{0}^{L}|y_{n}^{R},\ldots,x_{m}^{L}|y_{0}^{R},\ldots,x_{m}^{L}|y_{n}^R)(P\{ \vec{x}^{L}/\vec{x}\} \otimes Q \{ \vec{y}^{R}/\vec{y}\})
  \and \\
  \clift{\vec{P}} \otimes \clift{\vec{Q}}
  \and \\
  := \clift{P_{0}\otimes Q_{0},\ldots,P_{0}\otimes Q_{n},\ldots,P_{m}\otimes Q_{0},\ldots,P_{m}\otimes Q_{n}}
\end{mathpar}

\begin{remark}
  Observe that arities of tensored abstractions matches arities of
  tensored concretions if the original arities matched. Note also that
  the length of the arities corresponds to the increase in dimension
  we see in ordinary vector space tensor product.
\end{remark}

\begin{remark}
  Operationally, this definition distributes the tensor down to
  components ``linked'' by summation. Tensor over summation is
  intriguing in that it mixes names. Moreover, as a consequence of the
  way it mixes names we have the identities for all $x \in \QProc$ and
  $P,Q \in \Proc$

  \begin{mathpar}
    (x \cdot P) \otimes Q \equiv x \cdot (P \otimes Q) \equiv P \otimes (x \cdot Q)
    \and
    P \otimes \pzero \equiv P
  \end{mathpar}

  that the reader is invited to verify.
\end{remark}

\subsubsection{Annihilation}
\begin{mathpar}
  P^{\perp} := \{ Q | \forall R. P|Q \red^{*} R \Rightarrow R \red^{*} \pzero \}
  \and \\
  P^{\underline{\perp}} := \Sigma_{Q \in P^{\perp}} \quotep{Q}?(y).(\dropn{y}|Q) | \Sigma_{Q \in P^{\perp}} \quotep{Q}\clift{\Box}
\end{mathpar}

\paragraph{Discussion} The reader will note that $P^{\perp}$ is a
\emph{set} of processes, while $P^{\underline{\perp}}$ is a
\emph{context}. We call the set $P^{\perp}$ the \emph{annihilators} of
$P$. The parallel composition of a process in the annihilators of $P$
with $P$ will result in a process, the state space of which has all
paths eventually leading to $\pzero$. Execution may endure loops; but
under reasonable conditions of fairness (naturally guaranteed under
most notions of bisimulation) such a composite process cannot get
stuck in such a loop and will, eventually pop out and terminate.

The context $P^{\underline{\perp}}$ is ready and willing to ``take the
$P$ out of'' the process to which it is applied. It will effectively
transmit the code of the process to which it is applied to one of the
annihilators and run the process against it.

\subsubsection{Evaluation}
We fix $M$ a domain of fully abstract interpretation with an equality
coincident with bisimulation. We take $\meaningof{\cdot} : \Proc \to
M$ to be the map interpreting processes and $\nmeaningof{\cdot} : \M
\to Proc$ to be the map running the other way. Then we define

\begin{mathpar}
  \int P := \nmeaningof{\meaningof{P}}
\end{mathpar}

\paragraph{Discussion}
There are many fully abstract interpretations of Milner's
$\pi$-calculus. Any of them can be used as a basis for interpreting
the reflective calculus here. Equipped with such a domain it is
largely a matter of grinding through to check that the Yoneda
construction for the normalization-by-evaluation program can be
extended to this setting.

\begin{remark}
  The reader is invited to verify that $\int (P^{\underline{\perp}}[P]) = 0$.
\end{remark}

\subsection{Quantum mechanics}

Table \ref{tbl:core_qm_op_defns} gives the core operational definitions

\begin{table}[htp]\label{tbl:core_qm_op_defns}
  \center{
    \fbox{
      \begin{tabular}{c|c}
        quantum mechanics & process calculus \\
        \hline
        scalar & $x := \quotep{P}$ \\
        state vector & $\state{P} := P$ \\
        dual & $\state{P}^{*} := \event{P^{\underline{\perp}}} := \quotep{P^{\underline{\perp}}}[-]$ \\
        matrix & $ \Sigma_{\alpha} \state{P_{\alpha}}x_{\alpha}\event{Q_{\alpha}}$ \\
        vector addition & $\state{P} + \state{Q} := \state{P | Q}$ \\
        tensor product & $\state{P} \otimes \state{Q} := \state{P \otimes Q}$ \\
        inner product & $\innerprod{P}{Q} := \quotep{\int P^{\underline{\perp}}[Q]}$ \\
      \end{tabular}
    }
  }
  \caption{QM - operational definitions}
\end{table}

where

\begin{mathpar}
  \prmatrix{P}{Q} := \fprmatrix{P}{\quotep{\pzero}}{Q}
  \and
  \fprmatrix{P}{x}{Q} := (\state{P},x,\event{Q})
  \and
  (\fprmatrix{P}{x}{Q})(\state{R}) := x \cdot \innerprod{Q}{R} \cdot \state{P}
  \and
  (\fprmatrix{P}{x}{Q})(\event{R}) := x \cdot \innerprod{R}{P} \cdot \event{Q}
\end{mathpar}

\paragraph{Discussion}
As promised: vectors (aka states) are represented as processes; duals
as contextual duals; inner product definition should be compared with
standard inner product definition for ....

\begin{remark}
  Assuming $\int (P^{\underline{\perp}}[P]) = 0$, the reader is
  invited to verify that $(\fprmatrix{P}{x}{P})(\state{P}) = x \cdot \state{P}$.
\end{remark}

\begin{remark}
  The reader is invited to verify that $\innerprod{P}{Q}$ could
  equally well have been written $\quotep{\int \stackrel{\vee}{x}}$
  where $x = \event{P^{\underline{\perp}}}(Q)$.

  One of the motivations for this remark is that there is another way
  to factor these operations. We could package up evaluation in the dual:

  \begin{mathpar}
    \state{P}^{*} := \event{\int P^{\underline{\perp}}} := \quotep{\int P^{\underline{\perp}}}[-]
  \end{mathpar}

  and then have inner product defined by
  
  \begin{mathpar}
    \innerprod{P}{Q} := \event{P}(Q)
  \end{mathpar}

  Hopefully, experience with the calculations will provide guidance on
  the best factoring.
\end{remark}

\begin{remark}
  Assuming $\int (P^{\underline{\perp}}[P]) = 0$, the reader is
  invited to verify that $\forall P,Q. (\prmatrix{0}{Q})(\state{0}) =
  \state{0}$ and dually $(\prmatrix{P}{0})(\event{0}) = \event{0}$.
\end{remark}

\begin{remark}
  i'm a little worried that i don't (yet) have proper support for
  complex conjugacy. But, the observation above may give us a
  clue. According to Abramsky, it must be the case that the scalars
  are iso to the homset of the identity for the tensor -- which the
  observation above characterizes. 

  For now, we will simply bookmark the notion with $\overline{x}$.
\end{remark}

\subsubsection{Adjointness}

We need to give a definition of $(\cdot)^{\dagger}$ for matrices. The
obvious candidate definition is
\begin{mathpar}
(\Sigma_{\alpha}\fprmatrix{P_{\alpha}}{x_{\alpha}}{Q_{\alpha}})^{\dagger}
= \Sigma_{\alpha}\fprmatrix{(Q_{\alpha}^{\underline{\perp}})^{*}}{\overline{x}_{\alpha}}{P_{\alpha}^{\underline{\perp}}} 
\end{mathpar}

But, $(Q_{\alpha}^{\underline{\perp}})^{*}$ requires a name along
which to communicate the process to achieve the context application.

\subsubsection{Basis for a basis}
If processes label states and ``addition'' of states (a.k.a. vector
addition) is interpreted as parallel composition, what corresponds to
notions of linear independence and basis? Here, we recall that Yoshida
has developed a set of \emph{combinators} for an asynchronous verison
of Milner's $\pi$-calculus. These are a finite set of processes such
any process can be expressed as parallel composition of these
combinators together with liberal uses of the new operator and
replication. We can simply give a translation of these into the
present calculus and have reasonable expectation that the property
carries over. That is, that the resultant set allows to express all
processes via parallel composition. Note, however, that there is no
new operator or replication in this calculus. As a result, we expect
that the corresponding set is actually infinite. That is, we expect
that the space is actually infinite dimensional.

\begin{remark}
  The attentive reader may be a bit concerned. Certainly, the
  collection $S$, $K$ and $I$ is a finite set of
  combinators. Shouldn't we expect to see a finite set of combinators
  for an effectively equivalent system? i am very sympathetic to this
  critique and feel it warrants full attention. On the other hand, i
  also have in mind the following analogy. The natural numbers, as a
  monoid under addition, has exactly $1$ generator, while the natural
  numbers, as a monoid under multiplication, has countably many
  generators (the primes). We observe that the application of the
  lambda calculus is much less resource sensitive than the parallel
  composition of the $\pi$-calculus. Could it be the case that we have
  an analogy of the form
  
  \begin{mathpar}
    m + n : MN :: m*n : M|N
  \end{mathpar}

  giving a similar blow up in the set of ``primes''?  This is such a
  wonderful thought that, even if it's not true, i think it's worth
  writing down.
\end{remark}
 

\documentclass[12pt]{llncs}
%\documentclass{jktr}

\usepackage[pdftex]{hyperref}                   
\usepackage {listings}
\usepackage {mathpartir}
\usepackage{bcprules}
%\usepackage{listings}
                       
\usepackage{graphicx} 
%\usepackage[margins=2.5cm,nohead,nofoot]{geometry}
%\usepackage{geometry}
\usepackage{amsfonts}
\usepackage{amstext}
\usepackage{latexsym}
\usepackage{amssymb}
\usepackage{color}


%\include{myPreamble}
\include{qm2pi.local} 

%\ifpdf
%\usepackage[pdftex]{graphicx}
%\else
%\usepackage{graphicx}
%\fi

 % \ifpdf
%  \usepackage{pdfsync}
%  \if


%\title{Brief Article}
%\author{David F. Snyder}
%\author{L.G. Meredith}

%\address{Dept. of Math., Texas State University--San Marcos, San Marcos, TX 78666}
       
\pagestyle{empty}


\begin{document}

\lstset{language=[Objective]Caml,frame=shadowbox}

\input{qm2pi.front}

% section front matter (end)

\input{qm2pi.intro} 
 
% section introduction (end)

% \input{qm2pi.knotations} 

% section notation (end)

\input{qm2pi.process.calculi} 

% section concurrent_process_calculi_and_spatial_logics_ (end)
    
%\input{qm2pi.knots2pi} 

%\input{qm2pi.trefoil} 

%\input{qm2pi.mainthm} 

% subsection basic_interpretation (end)

%\input{qm2pi.rho.presentation} 
\subsection{The syntax and semantics of the notation system}\label{sub:the_syntax_and_semantics_of_the_notation_system} % (fold)

We now summarize a technical presentation of the calculus that
embodies our theory of dynamics. The typical presentation of such a
calculus follows the style of giving generators and relations on
them. The grammar, below, describing term constructors, freely
generates the set of processes, $\Proc$. This set is then quotiented
by a relation known as structural congruence and it is over this set
that the notion of dynamics is expressed. This presentation is
essentially that of \cite{MeredithR05} with the addition of
polyadicity and summation. For readability we have relegated some of
the technical subtleties to an appendix.

\subsubsection{Process grammar}\label{subsub:process_grammar}

\begin{mathpar}
  \inferrule* [lab=synchronization] {} {{M} \bc \pzero \;|\; x?F \;|\; x!C }
  \and
  \inferrule* [lab=abstraction] {} {{F} \bc (x)P}
  \and
  \inferrule* [lab=concretion] {} {{C} \bc \langle Q \rangle}
  \and
  \inferrule* [lab=process] {} {{P,Q} \bc M \;| \;P|Q \;|\; @{x}}
  \and
  \inferrule* [lab=name] {} {{x} \bc \quotep{P}}
\end{mathpar} 

Note that $\vec{x}$ (resp. $\vec{P}$) denotes a vector of names
(resp. processes) of length $|\vec{x}|$ (resp. $|\vec{P}|$). We adopt
the following useful abbreviations.

\begin{mathpar}
   x?(\vec{y}).P := x.(\vec{y})P \and  x\clift{\vec{P}} := x.\clift{\vec{P}}
   \and x!(y) := \lift{x}{\dropn{y}}
   \and \Pi_{i=0}^{n-1}P_i := P_0 | \ldots | P_{n-1}
\end{mathpar}

\subsubsection{Structural congruence}

\paragraph{Free and bound names and alpha-equivalence.} At the
core of structural equivalence is alpha-equivalence which identifies
process that are the same up to a change of variable. Formally, we
recognize the distinction between free and bound names. The free names
of a process, $\freenames{P}$, may be calculated recursively as
follows:

\begin{mathpar}
\freenames{\pzero} := \emptyset
  \and \\
  \freenames{x?(y).P} := \{ x \} \cup (\freenames{P} \setminus \{ y \})
  \and 
  \freenames{x!\langle P \rangle} := \{ x \} \cup \{ P \} 
  \and \\
  \freenames{P|Q} := \freenames{P} \cup \freenames{Q}
  \and \\
  \freenames{@{x}} := \{ x \}
\end{mathpar}

$\pi$
$\quotep{\pi}$

$\freenames{-} : \pi \to \mathcal{P}(\quotep{\pi})$

\begin{eqnarray*}
  \freenames{\pzero} & := & \emptyset \\
  \freenames{x?(y).P} & := & \{ x \} \cup (\freenames{P} \setminus \{ y \}) \\
  \freenames{x!\langle P \rangle} & := & \{ x \} \cup \{ P \} \\
  \freenames{P|Q} & := & \freenames{P} \cup \freenames{Q} \\
  \freenames{\dropn{x}} & := & \{ x \}
\end{eqnarray*}

The bound names of a process, $\boundnames{P}$, are those names occurring in $P$
that are not free. For example, in $x?(y).0$, the name $x$ is free, while $y$ is bound.

\begin{mathpar}
  \inferrule* [lab=monoidal-laws] {} { P|Q \equiv Q|P \and P|0 \equiv P \and P|(Q|R) \equiv (P|Q)|R }
\end{mathpar}

\begin{mathpar}
  \inferrule* [lab=alpha-equivalence] {} { (x)P \equiv (y)P\{y/x\} \and y \not\in \freenames{P} }
\end{mathpar}

\begin{definition}
Then two processes, $P,Q$, are alpha-equivalent if $P = Q\{\vec{y}/\vec{x}\}$ for
some $\vec{x} \in \boundnames{Q},\vec{y} \in \boundnames{P}$, where $Q\{\vec{y}/\vec{x}\}$
denotes the capture-avoiding substitution of $\vec{y}$ for $\vec{x}$ in $Q$.
\end{definition}

\begin{definition}
  The {\em structural congruence} \cite{SangiorgiWalker} , $\equiv$,
  between processes is the least congruence containing
  alpha-equivalence, satisfying the abelian monoid laws
  (associativity, commutativity and $\pzero$ as identity) for parallel
  composition $|$ and for summation $+$.
\end{definition}

\subsection{Name equivalence}

We take name equivalence, written $\nameeq$, to be the smallest
equivalence relation generated by the following rules.

\begin{mathpar}
\inferrule*[lab=Quote-drop]
{ }
{ \quotep{@{x}} \nameeq x }

\inferrule*[lab=Struct-equiv]
{ P \scong Q }
{ \quotep{P} \nameeq \quotep{Q} }
\end{mathpar}

The astute reader will have noticed that the mutual recursion of names
and processes imposes a mutual recursion on alpha-equivalence and
structural equivalence via name-equivalence. Fortunately, all of this
works out pleasantly and we may calculate in the natural way, free of
concern. The reader interested in the details is referred to the
appendix \ref{appendix:rho_details}.

\subsection{Substitution}

We use $\Proc$ for the set of processes, $\QProc$ for the set of
names, and $\id{\{}\vec{y} / \vec{x} \id{\}}$ to denote partial maps,
$s : \QProc \rightarrow \QProc$. A map, $s$ lifts, uniquely, to a map
on process terms, $\widehat{s} : \Proc \rightarrow \Proc$ by the
following equations.

\begin{mathpar}
  (0) \psubstp{Q}{P} := 0 \\
  (R \juxtap S) \psubstp{Q}{P}
  :=    
  (R)\psubstp{Q}{P} \juxtap (S) \psubstp{Q}{P} \\
  (x?(y).R) \psubstp{Q}{P}    
  :=    
  (x)\substp{Q}{P} (z)\concat( (R \psubstn{z}{y}) \psubstp{Q}{P} ) \\
  (\lift{x}{R}) \psubstp{Q}{P}  
  :=
  \lift{(x)\substp{Q}{P}}{ R \psubstp{Q}{P} } \\
%   (\dropn{x})  \psubstp{Q}{P}       
%   := 
%   \left\{ 
%     \begin{array}{ccc} 
%       \dropn{\quotep{Q}} & & x \nameeq \quotep{P} \\
%       \dropn{x} & & otherwise \\
%     \end{array}
%   \right. 
  (\dropn{x})  \psubstp{Q}{P}       
  := 
  \left\{ 
    \begin{array}{ccc} 
      Q & & x \nameeq \quotep{P} \\
      \dropn{x} & & otherwise \\
    \end{array}
  \right.
\end{mathpar}
 

where

\begin{eqnarray}
  (x)\id{\{} \lpquote Q \rpquote / \lpquote P \rpquote \id{\}}            = 
  \left\{ 
    \begin{array}{ccc}
      \lpquote Q \rpquote & & x \nameeq \lpquote P \rpquote \\
      x & & otherwise \\
    \end{array}
  \right. \nonumber
\end{eqnarray}

and $z$ is chosen distinct from $\quotep{P}$, $\quotep{Q}$, the free
names in $Q$, and all the names in $R$. Our $\alpha$-equivalence will
be built in the standard way from this substitution.

\begin{remark}\label{rem:no_self_referential_names}
  One consequence of these definitions is that $\forall P. \quotep{P}
  \not\in \freenames{P}$.
\end{remark}

\subsection{ Dynamic quote: an example }

Anticipating something of what's to come, consider applying the
substitution, $\widehat{\id{\{}u / z \id{\}}}$, to the following pair
of processes, $\lift{w}{y!(z)}$ and $w[ \lpquote y!(z) \rpquote ]$.

\begin{eqnarray}
	\lift{w}{y!(z)}\widehat{\id{\{}u / z \id{\}}}
		& = &
		\lift{w}{y!(u)} \nonumber\\
	w[ \lpquote y!(z) \rpquote ] \widehat{ \id{\{}u / z \id{\}} }
		& = &
		w[ \lpquote y!(z) \rpquote ] \nonumber
\end{eqnarray}

Because the body of the process between quotes is impervious to
substitution, we get radically different answers. In fact, by
examining the first process in an input context,
e.g. $x?(z).\lift{w}{y!(z)}$, we see that the process under the lift
operator may be shaped by prefixed inputs binding a name inside it. In
this sense, the lift operator will be seen as a way to dynamically
construct processes before reifying them as names.

Finally equipped with these standard features we can present the
dynamics of the calculus.

\subsubsection{Operational semantics} 

Finally, we introduce the computational dynamics. What marks these
algebras as distinct from other more traditionally studied algebraic
structures, e.g. vector spaces or polynomial rings, is the manner in
which dynamics is captured. In traditional structures, dynamics is typically
expressed through morphisms between such structures, as in linear maps
between vector spaces or morphisms between rings. In algebras
associated with the semantics of computation, the dynamics is
expressed as part of the algebraic structure itself, through a
reduction reduction relation typically denoted by $\red$. Below, we
give a recursive presentation of this relation for the calculus used
in the encoding.

$\red \subseteq \pi \times \pi$
$\red : \pi \to \mathcal{P}(\pi)$

\begin{mathpar}
  \inferrule* [lab=Comm] { \textsf{match}( x_{src}, x_{trgt} ) } { x_{trgt}?(y)P \; | \; x_{src}!\langle {Q} \rangle \red P\{\quotep{Q}/y}\} }
  \and \\
  \inferrule* [lab=Par] {{P} \red {P}'} {{{P} | {Q}} \red {{P}' | {Q}}}
  \and
  \inferrule* [lab=Equiv]{{{P} \scong {P}'} \andalso {{P}' \red {Q}'} \andalso {{Q}' \scong {Q}}}{{P} \red {Q}}
\end{mathpar}

\begin{eqnarray*}
  match_{\equiv} (\quotep{P},\quotep{Q}) & := & P \equiv Q \\
  match_{\dagger}(\quotep{P},\quotep{Q}) & := & \forall R. P|Q \red^{*} R => R \red^{*} 0 \\
  match_{K}(\quotep{P},\quotep{Q}) & := & K \mbox{ for some context } K
\end{eqnarray*}

$u?(x)P | u!\langle Q \rangle \red P\{\quotep{Q}/x\}$

%We write $\wred$ for $\red^*$, and $P\red$ if $\exists Q $ such that $ P \red Q$.
We write $P\red$ if $\exists Q $ such that $ P \red Q$ and $P\not\red$, otherwise.

\section{Replication}

As mentioned before, it is known that replication (and hence
recursion) can be implemented in a higher-order process algebra
\cite{SangiorgiWalker}. As our first example of calculation with the
machinery thus far presented we give the construction explicitly in
the {\rhoc}.

\begin{eqnarray}
	D_{x} & := & \prefix{x}{y}{(\binpar{\outputp{x}{y}}{@{y}})} \nonumber\\
	\bangp_{x}{P} & := & \binpar{{x}!\langle{\binpar{D_{x}}{P}}\rangle}{D_{x}} \nonumber
\end{eqnarray}

\begin{eqnarray}
	\bangp_{x}{P} & & \nonumber\\
	=
	& {x}!\langle{(\prefix{x}{y}{(\outputp{x}{y} | @{y})) | P}}\rangle 
	      | \prefix{x}{y}{(\outputp{x}{y} | @{y})} & \nonumber\\
	\red
	& (\outputp{x}{y} | @{y})\substn{\quotep{(\prefix{x}{y}{(@{y} | \outputp{x}{y})) | P}}}{y} & \nonumber\\
	=
	& \outputp{x}{\quotep{(\prefix{x}{y}{(\outputp{x}{y} | @{y})) | P}}}
	  | {(\prefix{x}{y}{(\outputp{x}{y} | @{y})) | P}} & \nonumber\\
	\red
	& \ldots & \nonumber\\
	\red^*
	& P | P | \ldots & \nonumber
\end{eqnarray}

Of course, this encoding, as an implementation, runs away, unfolding
$\bangp{P}$ eagerly. A lazier and more implementable replication
operator, restricted to input-guarded processes, may be obtained as follows.

\begin{eqnarray}
\bangp{\prefix{u}{v}{P}} 
	:= 
	\binpar{\lift{x}{\prefix{u}{v}{(\binpar{D(x)}{P})}}}{D(x)} \nonumber
\end{eqnarray}

\begin{remark}
  Note that the lazier definition still does not deal with summation
  or mixed summation (i.e. sums over input and output). The reader is
  invited to construct definitions of replication that deal with these
  features. 

  Further, the definitions are parameterized in a name, $x$. Can you,
  gentle reader, make a definition that eliminates this parameter and
  guarantees no accidental interaction between the replication
  machinery and the process being replicated -- i.e. no accidental
  sharing of names used by the process to get its work done and the
  name(s) used by the replication to effect copying. This latter
  revision of the definition of replication is crucial to obtaining
  the expected identity $!!P \sim !P$.
\end{remark}

\begin{remark}\label{rem:paradoxical_combinator}
  The reader familiar with the lambda calculus will have noticed the
  similarity between $D$ and the paradoxical combinator.

  [Ed. note: the existence of this seems to suggest we have to be more
  restrictive on the set of processes and names we admit if we are to
  support no-cloning.]
\end{remark}

\subsubsection{Bisimulation}

The computational dynamics gives rise to another kind of equivalence,
the equivalence of computational behavior. As previously mentioned
this is typically captured \emph{via} some form of bisimulation.

% The notion we use in this paper is weak barbed bisimulation
% \cite{milner91polyadicpi}.

The notion we use in this paper is derived from weak barbed
bisimulation \cite{milner91polyadicpi}. 

\begin{definition}
An \emph{observation relation}, $\downarrow_{\mathcal N}$, over a set
of names, $\mathcal N$, is the smallest relation satisfying the rules
below.

\infrule[Out-barb]{y \in {\mathcal N}, \; x \nameeq y}
		  {\outputp{x}{v} \downarrow_{\mathcal N} x}
\infrule[Par-barb]{\mbox{$P\downarrow_{\mathcal N} x$ or $Q\downarrow_{\mathcal N} x$}}
		  {\binpar{P}{Q} \downarrow_{\mathcal N} x}

We write $P \Downarrow_{\mathcal N} x$ if there is $Q$ such that 
$P \wred Q$ and $Q \downarrow_{\mathcal N} x$.
\end{definition}

\begin{definition}
%\label{def.bbisim}
An  ${\mathcal N}$-\emph{barbed bisimulation} over a set of names, ${\mathcal N}$, is a symmetric binary relation 
${\mathcal S}_{\mathcal N}$ between agents such that $P\rel{S}_{\mathcal N}Q$ implies:
\begin{enumerate}
\item If $P \red P'$ then $Q \wred Q'$ and $P'\rel{S}_{\mathcal N} Q'$.
\item If $P\downarrow_{\mathcal N} x$, then $Q\Downarrow_{\mathcal N} x$.
\end{enumerate}
$P$ is ${\mathcal N}$-barbed bisimilar to $Q$, written
$P \wbbisim_{\mathcal N} Q$, if $P \rel{S}_{\mathcal N} Q$ for some ${\mathcal N}$-barbed bisimulation ${\mathcal S}_{\mathcal N}$.
\end{definition}

$\mathcal{R} \subseteq \pi \times \pi$

$P \mathcal{R} Q => \forall P'. P \red P' \Rightarrow \exists Q'. Q \red Q', P' \mathcal{R} Q'$

$P \vdash x \Rightarrow Q \vdash x$

\begin{mathpar}
  \inferrule*[lab=Out-barb]{x \nameeq y}{{y}!\langle{Q}\rangle \vdash x}
  \and
  \inferrule*[lab=Par-barb]{\mbox{$P\vdash x$ or $Q\vdash x$}}{\binpar{P}{Q} \vdash x}
\end{mathpar}

\subsubsection{Contexts}

One of the principle advantages of computational calculi like the
$\pi$-calculus is a well-defined notion of context,
contextual-equivalence and a correlation between
contextual-equivalence and notions of bisimulation. The notion of
context allows the decomposition of a process into (sub-)process and
its syntactic environment, its context. Thus, a context may be
thought of as a process with a ``hole'' (written $\Box$) in it. The
application of a context $M$ to a process $P$, written $M[P]$, is
tantamount to filling the hole in $M$ with $P$. In this paper we do
not need the full weight of this theory, but do make use of the notion
of context in the proof the main theorem. 

\begin{mathpar}
  \inferrule* [lab=summation] {} {{M_{M},M_{N}} \bc \Box \;|\; x.M_{A} \;|\; M_{M}+M_{N}}
  \and
  \inferrule* [lab=agent] {} {{M_{A}} \bc (\vec{x})M_{P} \;| \; \clift{P_0,\ldots,M_{P},\ldots,P_N}}
  \and \\
  \inferrule* [lab=process] {} {{M_{P}} \bc M_{N} \;| \;P|M_{P} }
\end{mathpar} 

\begin{mathpar}
  \inferrule* [lab=sychronization] {} {M_{N} \bc \Box \;|\; x?M_{F} \;|\; x!M_{C}}
  \and
  \inferrule* [lab=abstraction] {} {{M_{F}} \bc (x)M_{P} }
  \and
  \inferrule* [lab=concretion] {} {{M_{C}} \bc \langle M_{P} \rangle }
  \and \\
  \inferrule* [lab=process] {} {{M_{P}} \bc M_{N} \;| \;P|M_{P} }
\end{mathpar}

\begin{definition}[contextual application] Given a context $M$, and
  process $P$, we define the \emph{contextual application}, $M[P] :=
  M\{P/\Box\}$. That is, the contextual application of M to P is the
  substitution of $P$ for $\Box$ in $M$.
\end{definition}

$\meaningof{-} : L \to \mathcal{P}(\pi)$

\begin{mathpar}
  \inferrule* [lab=collection] {} {\meaningof{true} = \pi, \and \meaningof{~E} = \pi \setminus \meaningof{E}, \and \meaningof{E_{1} \& E_{2}} = \meaningof{E_{1}} \cap \meaningof{E_{2}}}
\end{mathpar}

\begin{mathpar}
  \inferrule* [lab=structure] {} {\meaningof{0} = \{ P \in \pi | P \equiv 0 \}, \and \\ \meaningof{E_1 | E_2} = \{ P \in \pi | P \equiv P_{1} | P_{2}, P_{1} \in \meaningof{E_{1}}, P_{2} \in \meaningof{E_2}\} }
\end{mathpar}

\begin{mathpar}
 \inferrule* [lab=behavior] {} {\meaningof{\langle a?b \rangle E} = \{ P \in \pi | P \equiv Q | u?(y)P', \\ \and \\\\ \and \\ \;\;\; u \in \meaningof{a}, \forall z.P'\{z/y\} \in \meaningof{E\{z/b\}}\}, \and \\ \meaningof{a!E} = \{ P \in \pi | P \equiv Q | x!\langle P' \rangle, x \in \meaningof{a} P' \in \meaningof{E}\} }
\end{mathpar}

\begin{mathpar}
 \inferrule* [lab=nominal] {} {\meaningof{\quotep{E}} = \{ \quotep{P} \in \quotep{\pi} | P \in \meaningof{E} \}, \and \meaningof{\quotep{P}} = \{ \quotep{Q} \in \quotep{\pi} | P \equiv Q \} \and \\ \meaningof{@\quotep{E}} = \{ P \in \pi | P \equiv @x, x \in \meaningof{E} \}}
\end{mathpar}

\begin{eqnarray*}
  \\
  \meaningof{-} : TS \to ST
\end{eqnarray*}

\begin{eqnarray*}
  \\
  L : TS \to ST
\end{eqnarray*}

\begin{eqnarray*}
  \\
  P \models E \iff P \in \meaningof{E}
\end{eqnarray*}

\begin{eqnarray*}
  P \approx_{L} Q \iff \forall E \in L. P \models E \iff Q \models E
\end{eqnarray*}

\begin{eqnarray*}
  P \approx_{K} Q
\end{eqnarray*}

\begin{eqnarray*}
  P \approx Q
\end{eqnarray*}

$\approx_{K} = \approx = \approx_{L}$

\subsubsection{Contextual duality}

Note that contexts extend the quotation operation to a family of
operations from processes to names. Given a context, $M$, we can
define a \emph{nominal context}, $\quotep{M}$ by $\quotep{M}[P] :=
\quotep{M[P]}$. To foreshadow what is to come we observe that these
operations enjoy a duality with processes very much like the duality
between vectors and maps from vectors to scalars.

Further, because the calculus is essentially higher-order, we have a
correspondence between contexts and processes. More specifically,
given a name $x$ and a context $M$ we can construct $M^{*}_{x}$ such
that 

\begin{mathpar}
  M^{*}_{x} | \lift{x}{P} \red M[P]
\end{mathpar}

namely,

\begin{mathpar}
  M^{*}_{x} := x?(u).M[\dropn{u}]
\end{mathpar}

The dependence of $M^{*}_{x}$ on a name makes it an abstraction, 

\begin{mathpar}
  M^{*} := (x)x?(u).M[\dropn{u}]
\end{mathpar}

\subsection{Additional notation}

It will sometimes be convenient to denote the process a name
quotes. We already have the notation $x = \quotep{P}$, but it will be
convenient to introduce an alternate notation, $\procn{x}$, when we
want to emphasize the connection to the use of the name. Note that, by
virtue of name equivalence, $\quotep{\procn{x}} \nameeq x$; so, the
notation is consistent with previous definitions.

Further, because names have structure it is possible to effect
substitutions on the basis of that structure. This means we need to
upgrade our notation for substitutions, which we accomplish by
adapting comprehension notation. Thus,

\begin{mathpar}
  P\{ y / x : x \in S \}
\end{mathpar}

is interpreted to mean the process derived from P by replacing (in a
capture-avoiding manner) each occurrence of $x$ in $S$ by $y$. For example,

\begin{mathpar}
  P\{ \quotep{\procn{x}|\procn{x}} / x : x \in \freenames{P} \}
\end{mathpar}

will replace each (occurrence) of a free name $x$ in $P$ by
$\quotep{\procn{x}|\procn{x}}$.

Also, we will avail ourselves of the notation $x^{L}$ and $x^{R}$ to
denote injections of a name into disjoint copies of the name
space. There are numerous ways to accomplish this. One example can be
found in \cite{MeredithR05}. This notation overloads to vectors of
names: $\vec{x}^{\pi} := (x_{i}^{\pi} \; : \; 0 \leq i < |\vec{x}| )$ where $\pi \in \{L,R\}$.

We also use $P^{\Box} := P|\Box$.

In \cite{MeredithR05} an interpretation of the new operator is
given. It turns out that there are several possible interpretations
all enjoying the requisite algebraic properties of the operator (see
\cite{milner91polyadicpi}). We will therefore make liberal use of
$(\nu\; \vec{x})P$.

% subsection the_syntax_and_semantics_of_the_notation_system (end)   

\input{qm2pi.qmops} 

\input{qm2pi.sterngerlach} 

\input{qm2pi.metric} 

% section concurrent_process_calculi (end)

%\input{qm2pi.proofsketch}

% section proof sketch (end)

%\input{qm2pi.slviaknots} 

% section spatial logic via knots (end)

\input{qm2pi.conclusion}

% section conclusion (end)

%\input{qm2pi.dtcodes} 

% section wiring algorithm (end)

\input{qm2pi.ack} 

% section acknowledgments (end)

\newpage


\bibliographystyle{plain}   
\bibliography{../../biblios/main.bib}

\input{qm2pi.rhodetails}

\end{document}

 

\documentclass[12pt]{llncs}
%\documentclass{jktr}

\usepackage[pdftex]{hyperref}                   
\usepackage {listings}
\usepackage {mathpartir}
\usepackage{bcprules}
%\usepackage{listings}
                       
\usepackage{graphicx} 
%\usepackage[margins=2.5cm,nohead,nofoot]{geometry}
%\usepackage{geometry}
\usepackage{amsfonts}
\usepackage{amstext}
\usepackage{latexsym}
\usepackage{amssymb}
\usepackage{color}


%\include{myPreamble}
\include{qm2pi.local} 

%\ifpdf
%\usepackage[pdftex]{graphicx}
%\else
%\usepackage{graphicx}
%\fi

 % \ifpdf
%  \usepackage{pdfsync}
%  \if


%\title{Brief Article}
%\author{David F. Snyder}
%\author{L.G. Meredith}

%\address{Dept. of Math., Texas State University--San Marcos, San Marcos, TX 78666}
       
\pagestyle{empty}


\begin{document}

\lstset{language=[Objective]Caml,frame=shadowbox}

\input{qm2pi.front}

% section front matter (end)

\input{qm2pi.intro} 
 
% section introduction (end)

% \input{qm2pi.knotations} 

% section notation (end)

\input{qm2pi.process.calculi} 

% section concurrent_process_calculi_and_spatial_logics_ (end)
    
%\input{qm2pi.knots2pi} 

%\input{qm2pi.trefoil} 

%\input{qm2pi.mainthm} 

% subsection basic_interpretation (end)

%\input{qm2pi.rho.presentation} 
\subsection{The syntax and semantics of the notation system}\label{sub:the_syntax_and_semantics_of_the_notation_system} % (fold)

We now summarize a technical presentation of the calculus that
embodies our theory of dynamics. The typical presentation of such a
calculus follows the style of giving generators and relations on
them. The grammar, below, describing term constructors, freely
generates the set of processes, $\Proc$. This set is then quotiented
by a relation known as structural congruence and it is over this set
that the notion of dynamics is expressed. This presentation is
essentially that of \cite{MeredithR05} with the addition of
polyadicity and summation. For readability we have relegated some of
the technical subtleties to an appendix.

\subsubsection{Process grammar}\label{subsub:process_grammar}

\begin{mathpar}
  \inferrule* [lab=synchronization] {} {{M} \bc \pzero \;|\; x?F \;|\; x!C }
  \and
  \inferrule* [lab=abstraction] {} {{F} \bc (x)P}
  \and
  \inferrule* [lab=concretion] {} {{C} \bc \langle Q \rangle}
  \and
  \inferrule* [lab=process] {} {{P,Q} \bc M \;| \;P|Q \;|\; @{x}}
  \and
  \inferrule* [lab=name] {} {{x} \bc \quotep{P}}
\end{mathpar} 

Note that $\vec{x}$ (resp. $\vec{P}$) denotes a vector of names
(resp. processes) of length $|\vec{x}|$ (resp. $|\vec{P}|$). We adopt
the following useful abbreviations.

\begin{mathpar}
   x?(\vec{y}).P := x.(\vec{y})P \and  x\clift{\vec{P}} := x.\clift{\vec{P}}
   \and x!(y) := \lift{x}{\dropn{y}}
   \and \Pi_{i=0}^{n-1}P_i := P_0 | \ldots | P_{n-1}
\end{mathpar}

\subsubsection{Structural congruence}

\paragraph{Free and bound names and alpha-equivalence.} At the
core of structural equivalence is alpha-equivalence which identifies
process that are the same up to a change of variable. Formally, we
recognize the distinction between free and bound names. The free names
of a process, $\freenames{P}$, may be calculated recursively as
follows:

\begin{mathpar}
\freenames{\pzero} := \emptyset
  \and \\
  \freenames{x?(y).P} := \{ x \} \cup (\freenames{P} \setminus \{ y \})
  \and 
  \freenames{x!\langle P \rangle} := \{ x \} \cup \{ P \} 
  \and \\
  \freenames{P|Q} := \freenames{P} \cup \freenames{Q}
  \and \\
  \freenames{@{x}} := \{ x \}
\end{mathpar}

$\pi$
$\quotep{\pi}$

$\freenames{-} : \pi \to \mathcal{P}(\quotep{\pi})$

\begin{eqnarray*}
  \freenames{\pzero} & := & \emptyset \\
  \freenames{x?(y).P} & := & \{ x \} \cup (\freenames{P} \setminus \{ y \}) \\
  \freenames{x!\langle P \rangle} & := & \{ x \} \cup \{ P \} \\
  \freenames{P|Q} & := & \freenames{P} \cup \freenames{Q} \\
  \freenames{\dropn{x}} & := & \{ x \}
\end{eqnarray*}

The bound names of a process, $\boundnames{P}$, are those names occurring in $P$
that are not free. For example, in $x?(y).0$, the name $x$ is free, while $y$ is bound.

\begin{mathpar}
  \inferrule* [lab=monoidal-laws] {} { P|Q \equiv Q|P \and P|0 \equiv P \and P|(Q|R) \equiv (P|Q)|R }
\end{mathpar}

\begin{mathpar}
  \inferrule* [lab=alpha-equivalence] {} { (x)P \equiv (y)P\{y/x\} \and y \not\in \freenames{P} }
\end{mathpar}

\begin{definition}
Then two processes, $P,Q$, are alpha-equivalent if $P = Q\{\vec{y}/\vec{x}\}$ for
some $\vec{x} \in \boundnames{Q},\vec{y} \in \boundnames{P}$, where $Q\{\vec{y}/\vec{x}\}$
denotes the capture-avoiding substitution of $\vec{y}$ for $\vec{x}$ in $Q$.
\end{definition}

\begin{definition}
  The {\em structural congruence} \cite{SangiorgiWalker} , $\equiv$,
  between processes is the least congruence containing
  alpha-equivalence, satisfying the abelian monoid laws
  (associativity, commutativity and $\pzero$ as identity) for parallel
  composition $|$ and for summation $+$.
\end{definition}

\subsection{Name equivalence}

We take name equivalence, written $\nameeq$, to be the smallest
equivalence relation generated by the following rules.

\begin{mathpar}
\inferrule*[lab=Quote-drop]
{ }
{ \quotep{@{x}} \nameeq x }

\inferrule*[lab=Struct-equiv]
{ P \scong Q }
{ \quotep{P} \nameeq \quotep{Q} }
\end{mathpar}

The astute reader will have noticed that the mutual recursion of names
and processes imposes a mutual recursion on alpha-equivalence and
structural equivalence via name-equivalence. Fortunately, all of this
works out pleasantly and we may calculate in the natural way, free of
concern. The reader interested in the details is referred to the
appendix \ref{appendix:rho_details}.

\subsection{Substitution}

We use $\Proc$ for the set of processes, $\QProc$ for the set of
names, and $\id{\{}\vec{y} / \vec{x} \id{\}}$ to denote partial maps,
$s : \QProc \rightarrow \QProc$. A map, $s$ lifts, uniquely, to a map
on process terms, $\widehat{s} : \Proc \rightarrow \Proc$ by the
following equations.

\begin{mathpar}
  (0) \psubstp{Q}{P} := 0 \\
  (R \juxtap S) \psubstp{Q}{P}
  :=    
  (R)\psubstp{Q}{P} \juxtap (S) \psubstp{Q}{P} \\
  (x?(y).R) \psubstp{Q}{P}    
  :=    
  (x)\substp{Q}{P} (z)\concat( (R \psubstn{z}{y}) \psubstp{Q}{P} ) \\
  (\lift{x}{R}) \psubstp{Q}{P}  
  :=
  \lift{(x)\substp{Q}{P}}{ R \psubstp{Q}{P} } \\
%   (\dropn{x})  \psubstp{Q}{P}       
%   := 
%   \left\{ 
%     \begin{array}{ccc} 
%       \dropn{\quotep{Q}} & & x \nameeq \quotep{P} \\
%       \dropn{x} & & otherwise \\
%     \end{array}
%   \right. 
  (\dropn{x})  \psubstp{Q}{P}       
  := 
  \left\{ 
    \begin{array}{ccc} 
      Q & & x \nameeq \quotep{P} \\
      \dropn{x} & & otherwise \\
    \end{array}
  \right.
\end{mathpar}
 

where

\begin{eqnarray}
  (x)\id{\{} \lpquote Q \rpquote / \lpquote P \rpquote \id{\}}            = 
  \left\{ 
    \begin{array}{ccc}
      \lpquote Q \rpquote & & x \nameeq \lpquote P \rpquote \\
      x & & otherwise \\
    \end{array}
  \right. \nonumber
\end{eqnarray}

and $z$ is chosen distinct from $\quotep{P}$, $\quotep{Q}$, the free
names in $Q$, and all the names in $R$. Our $\alpha$-equivalence will
be built in the standard way from this substitution.

\begin{remark}\label{rem:no_self_referential_names}
  One consequence of these definitions is that $\forall P. \quotep{P}
  \not\in \freenames{P}$.
\end{remark}

\subsection{ Dynamic quote: an example }

Anticipating something of what's to come, consider applying the
substitution, $\widehat{\id{\{}u / z \id{\}}}$, to the following pair
of processes, $\lift{w}{y!(z)}$ and $w[ \lpquote y!(z) \rpquote ]$.

\begin{eqnarray}
	\lift{w}{y!(z)}\widehat{\id{\{}u / z \id{\}}}
		& = &
		\lift{w}{y!(u)} \nonumber\\
	w[ \lpquote y!(z) \rpquote ] \widehat{ \id{\{}u / z \id{\}} }
		& = &
		w[ \lpquote y!(z) \rpquote ] \nonumber
\end{eqnarray}

Because the body of the process between quotes is impervious to
substitution, we get radically different answers. In fact, by
examining the first process in an input context,
e.g. $x?(z).\lift{w}{y!(z)}$, we see that the process under the lift
operator may be shaped by prefixed inputs binding a name inside it. In
this sense, the lift operator will be seen as a way to dynamically
construct processes before reifying them as names.

Finally equipped with these standard features we can present the
dynamics of the calculus.

\subsubsection{Operational semantics} 

Finally, we introduce the computational dynamics. What marks these
algebras as distinct from other more traditionally studied algebraic
structures, e.g. vector spaces or polynomial rings, is the manner in
which dynamics is captured. In traditional structures, dynamics is typically
expressed through morphisms between such structures, as in linear maps
between vector spaces or morphisms between rings. In algebras
associated with the semantics of computation, the dynamics is
expressed as part of the algebraic structure itself, through a
reduction reduction relation typically denoted by $\red$. Below, we
give a recursive presentation of this relation for the calculus used
in the encoding.

$\red \subseteq \pi \times \pi$
$\red : \pi \to \mathcal{P}(\pi)$

\begin{mathpar}
  \inferrule* [lab=Comm] { \textsf{match}( x_{src}, x_{trgt} ) } { x_{trgt}?(y)P \; | \; x_{src}!\langle {Q} \rangle \red P\{\quotep{Q}/y}\} }
  \and \\
  \inferrule* [lab=Par] {{P} \red {P}'} {{{P} | {Q}} \red {{P}' | {Q}}}
  \and
  \inferrule* [lab=Equiv]{{{P} \scong {P}'} \andalso {{P}' \red {Q}'} \andalso {{Q}' \scong {Q}}}{{P} \red {Q}}
\end{mathpar}

\begin{eqnarray*}
  match_{\equiv} (\quotep{P},\quotep{Q}) & := & P \equiv Q \\
  match_{\dagger}(\quotep{P},\quotep{Q}) & := & \forall R. P|Q \red^{*} R => R \red^{*} 0 \\
  match_{K}(\quotep{P},\quotep{Q}) & := & K \mbox{ for some context } K
\end{eqnarray*}

$u?(x)P | u!\langle Q \rangle \red P\{\quotep{Q}/x\}$

%We write $\wred$ for $\red^*$, and $P\red$ if $\exists Q $ such that $ P \red Q$.
We write $P\red$ if $\exists Q $ such that $ P \red Q$ and $P\not\red$, otherwise.

\section{Replication}

As mentioned before, it is known that replication (and hence
recursion) can be implemented in a higher-order process algebra
\cite{SangiorgiWalker}. As our first example of calculation with the
machinery thus far presented we give the construction explicitly in
the {\rhoc}.

\begin{eqnarray}
	D_{x} & := & \prefix{x}{y}{(\binpar{\outputp{x}{y}}{@{y}})} \nonumber\\
	\bangp_{x}{P} & := & \binpar{{x}!\langle{\binpar{D_{x}}{P}}\rangle}{D_{x}} \nonumber
\end{eqnarray}

\begin{eqnarray}
	\bangp_{x}{P} & & \nonumber\\
	=
	& {x}!\langle{(\prefix{x}{y}{(\outputp{x}{y} | @{y})) | P}}\rangle 
	      | \prefix{x}{y}{(\outputp{x}{y} | @{y})} & \nonumber\\
	\red
	& (\outputp{x}{y} | @{y})\substn{\quotep{(\prefix{x}{y}{(@{y} | \outputp{x}{y})) | P}}}{y} & \nonumber\\
	=
	& \outputp{x}{\quotep{(\prefix{x}{y}{(\outputp{x}{y} | @{y})) | P}}}
	  | {(\prefix{x}{y}{(\outputp{x}{y} | @{y})) | P}} & \nonumber\\
	\red
	& \ldots & \nonumber\\
	\red^*
	& P | P | \ldots & \nonumber
\end{eqnarray}

Of course, this encoding, as an implementation, runs away, unfolding
$\bangp{P}$ eagerly. A lazier and more implementable replication
operator, restricted to input-guarded processes, may be obtained as follows.

\begin{eqnarray}
\bangp{\prefix{u}{v}{P}} 
	:= 
	\binpar{\lift{x}{\prefix{u}{v}{(\binpar{D(x)}{P})}}}{D(x)} \nonumber
\end{eqnarray}

\begin{remark}
  Note that the lazier definition still does not deal with summation
  or mixed summation (i.e. sums over input and output). The reader is
  invited to construct definitions of replication that deal with these
  features. 

  Further, the definitions are parameterized in a name, $x$. Can you,
  gentle reader, make a definition that eliminates this parameter and
  guarantees no accidental interaction between the replication
  machinery and the process being replicated -- i.e. no accidental
  sharing of names used by the process to get its work done and the
  name(s) used by the replication to effect copying. This latter
  revision of the definition of replication is crucial to obtaining
  the expected identity $!!P \sim !P$.
\end{remark}

\begin{remark}\label{rem:paradoxical_combinator}
  The reader familiar with the lambda calculus will have noticed the
  similarity between $D$ and the paradoxical combinator.

  [Ed. note: the existence of this seems to suggest we have to be more
  restrictive on the set of processes and names we admit if we are to
  support no-cloning.]
\end{remark}

\subsubsection{Bisimulation}

The computational dynamics gives rise to another kind of equivalence,
the equivalence of computational behavior. As previously mentioned
this is typically captured \emph{via} some form of bisimulation.

% The notion we use in this paper is weak barbed bisimulation
% \cite{milner91polyadicpi}.

The notion we use in this paper is derived from weak barbed
bisimulation \cite{milner91polyadicpi}. 

\begin{definition}
An \emph{observation relation}, $\downarrow_{\mathcal N}$, over a set
of names, $\mathcal N$, is the smallest relation satisfying the rules
below.

\infrule[Out-barb]{y \in {\mathcal N}, \; x \nameeq y}
		  {\outputp{x}{v} \downarrow_{\mathcal N} x}
\infrule[Par-barb]{\mbox{$P\downarrow_{\mathcal N} x$ or $Q\downarrow_{\mathcal N} x$}}
		  {\binpar{P}{Q} \downarrow_{\mathcal N} x}

We write $P \Downarrow_{\mathcal N} x$ if there is $Q$ such that 
$P \wred Q$ and $Q \downarrow_{\mathcal N} x$.
\end{definition}

\begin{definition}
%\label{def.bbisim}
An  ${\mathcal N}$-\emph{barbed bisimulation} over a set of names, ${\mathcal N}$, is a symmetric binary relation 
${\mathcal S}_{\mathcal N}$ between agents such that $P\rel{S}_{\mathcal N}Q$ implies:
\begin{enumerate}
\item If $P \red P'$ then $Q \wred Q'$ and $P'\rel{S}_{\mathcal N} Q'$.
\item If $P\downarrow_{\mathcal N} x$, then $Q\Downarrow_{\mathcal N} x$.
\end{enumerate}
$P$ is ${\mathcal N}$-barbed bisimilar to $Q$, written
$P \wbbisim_{\mathcal N} Q$, if $P \rel{S}_{\mathcal N} Q$ for some ${\mathcal N}$-barbed bisimulation ${\mathcal S}_{\mathcal N}$.
\end{definition}

$\mathcal{R} \subseteq \pi \times \pi$

$P \mathcal{R} Q => \forall P'. P \red P' \Rightarrow \exists Q'. Q \red Q', P' \mathcal{R} Q'$

$P \vdash x \Rightarrow Q \vdash x$

\begin{mathpar}
  \inferrule*[lab=Out-barb]{x \nameeq y}{{y}!\langle{Q}\rangle \vdash x}
  \and
  \inferrule*[lab=Par-barb]{\mbox{$P\vdash x$ or $Q\vdash x$}}{\binpar{P}{Q} \vdash x}
\end{mathpar}

\subsubsection{Contexts}

One of the principle advantages of computational calculi like the
$\pi$-calculus is a well-defined notion of context,
contextual-equivalence and a correlation between
contextual-equivalence and notions of bisimulation. The notion of
context allows the decomposition of a process into (sub-)process and
its syntactic environment, its context. Thus, a context may be
thought of as a process with a ``hole'' (written $\Box$) in it. The
application of a context $M$ to a process $P$, written $M[P]$, is
tantamount to filling the hole in $M$ with $P$. In this paper we do
not need the full weight of this theory, but do make use of the notion
of context in the proof the main theorem. 

\begin{mathpar}
  \inferrule* [lab=summation] {} {{M_{M},M_{N}} \bc \Box \;|\; x.M_{A} \;|\; M_{M}+M_{N}}
  \and
  \inferrule* [lab=agent] {} {{M_{A}} \bc (\vec{x})M_{P} \;| \; \clift{P_0,\ldots,M_{P},\ldots,P_N}}
  \and \\
  \inferrule* [lab=process] {} {{M_{P}} \bc M_{N} \;| \;P|M_{P} }
\end{mathpar} 

\begin{mathpar}
  \inferrule* [lab=sychronization] {} {M_{N} \bc \Box \;|\; x?M_{F} \;|\; x!M_{C}}
  \and
  \inferrule* [lab=abstraction] {} {{M_{F}} \bc (x)M_{P} }
  \and
  \inferrule* [lab=concretion] {} {{M_{C}} \bc \langle M_{P} \rangle }
  \and \\
  \inferrule* [lab=process] {} {{M_{P}} \bc M_{N} \;| \;P|M_{P} }
\end{mathpar}

\begin{definition}[contextual application] Given a context $M$, and
  process $P$, we define the \emph{contextual application}, $M[P] :=
  M\{P/\Box\}$. That is, the contextual application of M to P is the
  substitution of $P$ for $\Box$ in $M$.
\end{definition}

$\meaningof{-} : L \to \mathcal{P}(\pi)$

\begin{mathpar}
  \inferrule* [lab=collection] {} {\meaningof{true} = \pi, \and \meaningof{~E} = \pi \setminus \meaningof{E}, \and \meaningof{E_{1} \& E_{2}} = \meaningof{E_{1}} \cap \meaningof{E_{2}}}
\end{mathpar}

\begin{mathpar}
  \inferrule* [lab=structure] {} {\meaningof{0} = \{ P \in \pi | P \equiv 0 \}, \and \\ \meaningof{E_1 | E_2} = \{ P \in \pi | P \equiv P_{1} | P_{2}, P_{1} \in \meaningof{E_{1}}, P_{2} \in \meaningof{E_2}\} }
\end{mathpar}

\begin{mathpar}
 \inferrule* [lab=behavior] {} {\meaningof{\langle a?b \rangle E} = \{ P \in \pi | P \equiv Q | u?(y)P', \\ \and \\\\ \and \\ \;\;\; u \in \meaningof{a}, \forall z.P'\{z/y\} \in \meaningof{E\{z/b\}}\}, \and \\ \meaningof{a!E} = \{ P \in \pi | P \equiv Q | x!\langle P' \rangle, x \in \meaningof{a} P' \in \meaningof{E}\} }
\end{mathpar}

\begin{mathpar}
 \inferrule* [lab=nominal] {} {\meaningof{\quotep{E}} = \{ \quotep{P} \in \quotep{\pi} | P \in \meaningof{E} \}, \and \meaningof{\quotep{P}} = \{ \quotep{Q} \in \quotep{\pi} | P \equiv Q \} \and \\ \meaningof{@\quotep{E}} = \{ P \in \pi | P \equiv @x, x \in \meaningof{E} \}}
\end{mathpar}

\begin{eqnarray*}
  \\
  \meaningof{-} : TS \to ST
\end{eqnarray*}

\begin{eqnarray*}
  \\
  L : TS \to ST
\end{eqnarray*}

\begin{eqnarray*}
  \\
  P \models E \iff P \in \meaningof{E}
\end{eqnarray*}

\begin{eqnarray*}
  P \approx_{L} Q \iff \forall E \in L. P \models E \iff Q \models E
\end{eqnarray*}

\begin{eqnarray*}
  P \approx_{K} Q
\end{eqnarray*}

\begin{eqnarray*}
  P \approx Q
\end{eqnarray*}

$\approx_{K} = \approx = \approx_{L}$

\subsubsection{Contextual duality}

Note that contexts extend the quotation operation to a family of
operations from processes to names. Given a context, $M$, we can
define a \emph{nominal context}, $\quotep{M}$ by $\quotep{M}[P] :=
\quotep{M[P]}$. To foreshadow what is to come we observe that these
operations enjoy a duality with processes very much like the duality
between vectors and maps from vectors to scalars.

Further, because the calculus is essentially higher-order, we have a
correspondence between contexts and processes. More specifically,
given a name $x$ and a context $M$ we can construct $M^{*}_{x}$ such
that 

\begin{mathpar}
  M^{*}_{x} | \lift{x}{P} \red M[P]
\end{mathpar}

namely,

\begin{mathpar}
  M^{*}_{x} := x?(u).M[\dropn{u}]
\end{mathpar}

The dependence of $M^{*}_{x}$ on a name makes it an abstraction, 

\begin{mathpar}
  M^{*} := (x)x?(u).M[\dropn{u}]
\end{mathpar}

\subsection{Additional notation}

It will sometimes be convenient to denote the process a name
quotes. We already have the notation $x = \quotep{P}$, but it will be
convenient to introduce an alternate notation, $\procn{x}$, when we
want to emphasize the connection to the use of the name. Note that, by
virtue of name equivalence, $\quotep{\procn{x}} \nameeq x$; so, the
notation is consistent with previous definitions.

Further, because names have structure it is possible to effect
substitutions on the basis of that structure. This means we need to
upgrade our notation for substitutions, which we accomplish by
adapting comprehension notation. Thus,

\begin{mathpar}
  P\{ y / x : x \in S \}
\end{mathpar}

is interpreted to mean the process derived from P by replacing (in a
capture-avoiding manner) each occurrence of $x$ in $S$ by $y$. For example,

\begin{mathpar}
  P\{ \quotep{\procn{x}|\procn{x}} / x : x \in \freenames{P} \}
\end{mathpar}

will replace each (occurrence) of a free name $x$ in $P$ by
$\quotep{\procn{x}|\procn{x}}$.

Also, we will avail ourselves of the notation $x^{L}$ and $x^{R}$ to
denote injections of a name into disjoint copies of the name
space. There are numerous ways to accomplish this. One example can be
found in \cite{MeredithR05}. This notation overloads to vectors of
names: $\vec{x}^{\pi} := (x_{i}^{\pi} \; : \; 0 \leq i < |\vec{x}| )$ where $\pi \in \{L,R\}$.

We also use $P^{\Box} := P|\Box$.

In \cite{MeredithR05} an interpretation of the new operator is
given. It turns out that there are several possible interpretations
all enjoying the requisite algebraic properties of the operator (see
\cite{milner91polyadicpi}). We will therefore make liberal use of
$(\nu\; \vec{x})P$.

% subsection the_syntax_and_semantics_of_the_notation_system (end)   

\input{qm2pi.qmops} 

\input{qm2pi.sterngerlach} 

\input{qm2pi.metric} 

% section concurrent_process_calculi (end)

%\input{qm2pi.proofsketch}

% section proof sketch (end)

%\input{qm2pi.slviaknots} 

% section spatial logic via knots (end)

\input{qm2pi.conclusion}

% section conclusion (end)

%\input{qm2pi.dtcodes} 

% section wiring algorithm (end)

\input{qm2pi.ack} 

% section acknowledgments (end)

\newpage


\bibliographystyle{plain}   
\bibliography{../../biblios/main.bib}

\input{qm2pi.rhodetails}

\end{document}

 

% section concurrent_process_calculi (end)

%\documentclass[12pt]{llncs}
%\documentclass{jktr}

\usepackage[pdftex]{hyperref}                   
\usepackage {listings}
\usepackage {mathpartir}
\usepackage{bcprules}
%\usepackage{listings}
                       
\usepackage{graphicx} 
%\usepackage[margins=2.5cm,nohead,nofoot]{geometry}
%\usepackage{geometry}
\usepackage{amsfonts}
\usepackage{amstext}
\usepackage{latexsym}
\usepackage{amssymb}
\usepackage{color}


%\include{myPreamble}
\include{qm2pi.local} 

%\ifpdf
%\usepackage[pdftex]{graphicx}
%\else
%\usepackage{graphicx}
%\fi

 % \ifpdf
%  \usepackage{pdfsync}
%  \if


%\title{Brief Article}
%\author{David F. Snyder}
%\author{L.G. Meredith}

%\address{Dept. of Math., Texas State University--San Marcos, San Marcos, TX 78666}
       
\pagestyle{empty}


\begin{document}

\lstset{language=[Objective]Caml,frame=shadowbox}

\input{qm2pi.front}

% section front matter (end)

\input{qm2pi.intro} 
 
% section introduction (end)

% \input{qm2pi.knotations} 

% section notation (end)

\input{qm2pi.process.calculi} 

% section concurrent_process_calculi_and_spatial_logics_ (end)
    
%\input{qm2pi.knots2pi} 

%\input{qm2pi.trefoil} 

%\input{qm2pi.mainthm} 

% subsection basic_interpretation (end)

%\input{qm2pi.rho.presentation} 
\subsection{The syntax and semantics of the notation system}\label{sub:the_syntax_and_semantics_of_the_notation_system} % (fold)

We now summarize a technical presentation of the calculus that
embodies our theory of dynamics. The typical presentation of such a
calculus follows the style of giving generators and relations on
them. The grammar, below, describing term constructors, freely
generates the set of processes, $\Proc$. This set is then quotiented
by a relation known as structural congruence and it is over this set
that the notion of dynamics is expressed. This presentation is
essentially that of \cite{MeredithR05} with the addition of
polyadicity and summation. For readability we have relegated some of
the technical subtleties to an appendix.

\subsubsection{Process grammar}\label{subsub:process_grammar}

\begin{mathpar}
  \inferrule* [lab=synchronization] {} {{M} \bc \pzero \;|\; x?F \;|\; x!C }
  \and
  \inferrule* [lab=abstraction] {} {{F} \bc (x)P}
  \and
  \inferrule* [lab=concretion] {} {{C} \bc \langle Q \rangle}
  \and
  \inferrule* [lab=process] {} {{P,Q} \bc M \;| \;P|Q \;|\; @{x}}
  \and
  \inferrule* [lab=name] {} {{x} \bc \quotep{P}}
\end{mathpar} 

Note that $\vec{x}$ (resp. $\vec{P}$) denotes a vector of names
(resp. processes) of length $|\vec{x}|$ (resp. $|\vec{P}|$). We adopt
the following useful abbreviations.

\begin{mathpar}
   x?(\vec{y}).P := x.(\vec{y})P \and  x\clift{\vec{P}} := x.\clift{\vec{P}}
   \and x!(y) := \lift{x}{\dropn{y}}
   \and \Pi_{i=0}^{n-1}P_i := P_0 | \ldots | P_{n-1}
\end{mathpar}

\subsubsection{Structural congruence}

\paragraph{Free and bound names and alpha-equivalence.} At the
core of structural equivalence is alpha-equivalence which identifies
process that are the same up to a change of variable. Formally, we
recognize the distinction between free and bound names. The free names
of a process, $\freenames{P}$, may be calculated recursively as
follows:

\begin{mathpar}
\freenames{\pzero} := \emptyset
  \and \\
  \freenames{x?(y).P} := \{ x \} \cup (\freenames{P} \setminus \{ y \})
  \and 
  \freenames{x!\langle P \rangle} := \{ x \} \cup \{ P \} 
  \and \\
  \freenames{P|Q} := \freenames{P} \cup \freenames{Q}
  \and \\
  \freenames{@{x}} := \{ x \}
\end{mathpar}

$\pi$
$\quotep{\pi}$

$\freenames{-} : \pi \to \mathcal{P}(\quotep{\pi})$

\begin{eqnarray*}
  \freenames{\pzero} & := & \emptyset \\
  \freenames{x?(y).P} & := & \{ x \} \cup (\freenames{P} \setminus \{ y \}) \\
  \freenames{x!\langle P \rangle} & := & \{ x \} \cup \{ P \} \\
  \freenames{P|Q} & := & \freenames{P} \cup \freenames{Q} \\
  \freenames{\dropn{x}} & := & \{ x \}
\end{eqnarray*}

The bound names of a process, $\boundnames{P}$, are those names occurring in $P$
that are not free. For example, in $x?(y).0$, the name $x$ is free, while $y$ is bound.

\begin{mathpar}
  \inferrule* [lab=monoidal-laws] {} { P|Q \equiv Q|P \and P|0 \equiv P \and P|(Q|R) \equiv (P|Q)|R }
\end{mathpar}

\begin{mathpar}
  \inferrule* [lab=alpha-equivalence] {} { (x)P \equiv (y)P\{y/x\} \and y \not\in \freenames{P} }
\end{mathpar}

\begin{definition}
Then two processes, $P,Q$, are alpha-equivalent if $P = Q\{\vec{y}/\vec{x}\}$ for
some $\vec{x} \in \boundnames{Q},\vec{y} \in \boundnames{P}$, where $Q\{\vec{y}/\vec{x}\}$
denotes the capture-avoiding substitution of $\vec{y}$ for $\vec{x}$ in $Q$.
\end{definition}

\begin{definition}
  The {\em structural congruence} \cite{SangiorgiWalker} , $\equiv$,
  between processes is the least congruence containing
  alpha-equivalence, satisfying the abelian monoid laws
  (associativity, commutativity and $\pzero$ as identity) for parallel
  composition $|$ and for summation $+$.
\end{definition}

\subsection{Name equivalence}

We take name equivalence, written $\nameeq$, to be the smallest
equivalence relation generated by the following rules.

\begin{mathpar}
\inferrule*[lab=Quote-drop]
{ }
{ \quotep{@{x}} \nameeq x }

\inferrule*[lab=Struct-equiv]
{ P \scong Q }
{ \quotep{P} \nameeq \quotep{Q} }
\end{mathpar}

The astute reader will have noticed that the mutual recursion of names
and processes imposes a mutual recursion on alpha-equivalence and
structural equivalence via name-equivalence. Fortunately, all of this
works out pleasantly and we may calculate in the natural way, free of
concern. The reader interested in the details is referred to the
appendix \ref{appendix:rho_details}.

\subsection{Substitution}

We use $\Proc$ for the set of processes, $\QProc$ for the set of
names, and $\id{\{}\vec{y} / \vec{x} \id{\}}$ to denote partial maps,
$s : \QProc \rightarrow \QProc$. A map, $s$ lifts, uniquely, to a map
on process terms, $\widehat{s} : \Proc \rightarrow \Proc$ by the
following equations.

\begin{mathpar}
  (0) \psubstp{Q}{P} := 0 \\
  (R \juxtap S) \psubstp{Q}{P}
  :=    
  (R)\psubstp{Q}{P} \juxtap (S) \psubstp{Q}{P} \\
  (x?(y).R) \psubstp{Q}{P}    
  :=    
  (x)\substp{Q}{P} (z)\concat( (R \psubstn{z}{y}) \psubstp{Q}{P} ) \\
  (\lift{x}{R}) \psubstp{Q}{P}  
  :=
  \lift{(x)\substp{Q}{P}}{ R \psubstp{Q}{P} } \\
%   (\dropn{x})  \psubstp{Q}{P}       
%   := 
%   \left\{ 
%     \begin{array}{ccc} 
%       \dropn{\quotep{Q}} & & x \nameeq \quotep{P} \\
%       \dropn{x} & & otherwise \\
%     \end{array}
%   \right. 
  (\dropn{x})  \psubstp{Q}{P}       
  := 
  \left\{ 
    \begin{array}{ccc} 
      Q & & x \nameeq \quotep{P} \\
      \dropn{x} & & otherwise \\
    \end{array}
  \right.
\end{mathpar}
 

where

\begin{eqnarray}
  (x)\id{\{} \lpquote Q \rpquote / \lpquote P \rpquote \id{\}}            = 
  \left\{ 
    \begin{array}{ccc}
      \lpquote Q \rpquote & & x \nameeq \lpquote P \rpquote \\
      x & & otherwise \\
    \end{array}
  \right. \nonumber
\end{eqnarray}

and $z$ is chosen distinct from $\quotep{P}$, $\quotep{Q}$, the free
names in $Q$, and all the names in $R$. Our $\alpha$-equivalence will
be built in the standard way from this substitution.

\begin{remark}\label{rem:no_self_referential_names}
  One consequence of these definitions is that $\forall P. \quotep{P}
  \not\in \freenames{P}$.
\end{remark}

\subsection{ Dynamic quote: an example }

Anticipating something of what's to come, consider applying the
substitution, $\widehat{\id{\{}u / z \id{\}}}$, to the following pair
of processes, $\lift{w}{y!(z)}$ and $w[ \lpquote y!(z) \rpquote ]$.

\begin{eqnarray}
	\lift{w}{y!(z)}\widehat{\id{\{}u / z \id{\}}}
		& = &
		\lift{w}{y!(u)} \nonumber\\
	w[ \lpquote y!(z) \rpquote ] \widehat{ \id{\{}u / z \id{\}} }
		& = &
		w[ \lpquote y!(z) \rpquote ] \nonumber
\end{eqnarray}

Because the body of the process between quotes is impervious to
substitution, we get radically different answers. In fact, by
examining the first process in an input context,
e.g. $x?(z).\lift{w}{y!(z)}$, we see that the process under the lift
operator may be shaped by prefixed inputs binding a name inside it. In
this sense, the lift operator will be seen as a way to dynamically
construct processes before reifying them as names.

Finally equipped with these standard features we can present the
dynamics of the calculus.

\subsubsection{Operational semantics} 

Finally, we introduce the computational dynamics. What marks these
algebras as distinct from other more traditionally studied algebraic
structures, e.g. vector spaces or polynomial rings, is the manner in
which dynamics is captured. In traditional structures, dynamics is typically
expressed through morphisms between such structures, as in linear maps
between vector spaces or morphisms between rings. In algebras
associated with the semantics of computation, the dynamics is
expressed as part of the algebraic structure itself, through a
reduction reduction relation typically denoted by $\red$. Below, we
give a recursive presentation of this relation for the calculus used
in the encoding.

$\red \subseteq \pi \times \pi$
$\red : \pi \to \mathcal{P}(\pi)$

\begin{mathpar}
  \inferrule* [lab=Comm] { \textsf{match}( x_{src}, x_{trgt} ) } { x_{trgt}?(y)P \; | \; x_{src}!\langle {Q} \rangle \red P\{\quotep{Q}/y}\} }
  \and \\
  \inferrule* [lab=Par] {{P} \red {P}'} {{{P} | {Q}} \red {{P}' | {Q}}}
  \and
  \inferrule* [lab=Equiv]{{{P} \scong {P}'} \andalso {{P}' \red {Q}'} \andalso {{Q}' \scong {Q}}}{{P} \red {Q}}
\end{mathpar}

\begin{eqnarray*}
  match_{\equiv} (\quotep{P},\quotep{Q}) & := & P \equiv Q \\
  match_{\dagger}(\quotep{P},\quotep{Q}) & := & \forall R. P|Q \red^{*} R => R \red^{*} 0 \\
  match_{K}(\quotep{P},\quotep{Q}) & := & K \mbox{ for some context } K
\end{eqnarray*}

$u?(x)P | u!\langle Q \rangle \red P\{\quotep{Q}/x\}$

%We write $\wred$ for $\red^*$, and $P\red$ if $\exists Q $ such that $ P \red Q$.
We write $P\red$ if $\exists Q $ such that $ P \red Q$ and $P\not\red$, otherwise.

\section{Replication}

As mentioned before, it is known that replication (and hence
recursion) can be implemented in a higher-order process algebra
\cite{SangiorgiWalker}. As our first example of calculation with the
machinery thus far presented we give the construction explicitly in
the {\rhoc}.

\begin{eqnarray}
	D_{x} & := & \prefix{x}{y}{(\binpar{\outputp{x}{y}}{@{y}})} \nonumber\\
	\bangp_{x}{P} & := & \binpar{{x}!\langle{\binpar{D_{x}}{P}}\rangle}{D_{x}} \nonumber
\end{eqnarray}

\begin{eqnarray}
	\bangp_{x}{P} & & \nonumber\\
	=
	& {x}!\langle{(\prefix{x}{y}{(\outputp{x}{y} | @{y})) | P}}\rangle 
	      | \prefix{x}{y}{(\outputp{x}{y} | @{y})} & \nonumber\\
	\red
	& (\outputp{x}{y} | @{y})\substn{\quotep{(\prefix{x}{y}{(@{y} | \outputp{x}{y})) | P}}}{y} & \nonumber\\
	=
	& \outputp{x}{\quotep{(\prefix{x}{y}{(\outputp{x}{y} | @{y})) | P}}}
	  | {(\prefix{x}{y}{(\outputp{x}{y} | @{y})) | P}} & \nonumber\\
	\red
	& \ldots & \nonumber\\
	\red^*
	& P | P | \ldots & \nonumber
\end{eqnarray}

Of course, this encoding, as an implementation, runs away, unfolding
$\bangp{P}$ eagerly. A lazier and more implementable replication
operator, restricted to input-guarded processes, may be obtained as follows.

\begin{eqnarray}
\bangp{\prefix{u}{v}{P}} 
	:= 
	\binpar{\lift{x}{\prefix{u}{v}{(\binpar{D(x)}{P})}}}{D(x)} \nonumber
\end{eqnarray}

\begin{remark}
  Note that the lazier definition still does not deal with summation
  or mixed summation (i.e. sums over input and output). The reader is
  invited to construct definitions of replication that deal with these
  features. 

  Further, the definitions are parameterized in a name, $x$. Can you,
  gentle reader, make a definition that eliminates this parameter and
  guarantees no accidental interaction between the replication
  machinery and the process being replicated -- i.e. no accidental
  sharing of names used by the process to get its work done and the
  name(s) used by the replication to effect copying. This latter
  revision of the definition of replication is crucial to obtaining
  the expected identity $!!P \sim !P$.
\end{remark}

\begin{remark}\label{rem:paradoxical_combinator}
  The reader familiar with the lambda calculus will have noticed the
  similarity between $D$ and the paradoxical combinator.

  [Ed. note: the existence of this seems to suggest we have to be more
  restrictive on the set of processes and names we admit if we are to
  support no-cloning.]
\end{remark}

\subsubsection{Bisimulation}

The computational dynamics gives rise to another kind of equivalence,
the equivalence of computational behavior. As previously mentioned
this is typically captured \emph{via} some form of bisimulation.

% The notion we use in this paper is weak barbed bisimulation
% \cite{milner91polyadicpi}.

The notion we use in this paper is derived from weak barbed
bisimulation \cite{milner91polyadicpi}. 

\begin{definition}
An \emph{observation relation}, $\downarrow_{\mathcal N}$, over a set
of names, $\mathcal N$, is the smallest relation satisfying the rules
below.

\infrule[Out-barb]{y \in {\mathcal N}, \; x \nameeq y}
		  {\outputp{x}{v} \downarrow_{\mathcal N} x}
\infrule[Par-barb]{\mbox{$P\downarrow_{\mathcal N} x$ or $Q\downarrow_{\mathcal N} x$}}
		  {\binpar{P}{Q} \downarrow_{\mathcal N} x}

We write $P \Downarrow_{\mathcal N} x$ if there is $Q$ such that 
$P \wred Q$ and $Q \downarrow_{\mathcal N} x$.
\end{definition}

\begin{definition}
%\label{def.bbisim}
An  ${\mathcal N}$-\emph{barbed bisimulation} over a set of names, ${\mathcal N}$, is a symmetric binary relation 
${\mathcal S}_{\mathcal N}$ between agents such that $P\rel{S}_{\mathcal N}Q$ implies:
\begin{enumerate}
\item If $P \red P'$ then $Q \wred Q'$ and $P'\rel{S}_{\mathcal N} Q'$.
\item If $P\downarrow_{\mathcal N} x$, then $Q\Downarrow_{\mathcal N} x$.
\end{enumerate}
$P$ is ${\mathcal N}$-barbed bisimilar to $Q$, written
$P \wbbisim_{\mathcal N} Q$, if $P \rel{S}_{\mathcal N} Q$ for some ${\mathcal N}$-barbed bisimulation ${\mathcal S}_{\mathcal N}$.
\end{definition}

$\mathcal{R} \subseteq \pi \times \pi$

$P \mathcal{R} Q => \forall P'. P \red P' \Rightarrow \exists Q'. Q \red Q', P' \mathcal{R} Q'$

$P \vdash x \Rightarrow Q \vdash x$

\begin{mathpar}
  \inferrule*[lab=Out-barb]{x \nameeq y}{{y}!\langle{Q}\rangle \vdash x}
  \and
  \inferrule*[lab=Par-barb]{\mbox{$P\vdash x$ or $Q\vdash x$}}{\binpar{P}{Q} \vdash x}
\end{mathpar}

\subsubsection{Contexts}

One of the principle advantages of computational calculi like the
$\pi$-calculus is a well-defined notion of context,
contextual-equivalence and a correlation between
contextual-equivalence and notions of bisimulation. The notion of
context allows the decomposition of a process into (sub-)process and
its syntactic environment, its context. Thus, a context may be
thought of as a process with a ``hole'' (written $\Box$) in it. The
application of a context $M$ to a process $P$, written $M[P]$, is
tantamount to filling the hole in $M$ with $P$. In this paper we do
not need the full weight of this theory, but do make use of the notion
of context in the proof the main theorem. 

\begin{mathpar}
  \inferrule* [lab=summation] {} {{M_{M},M_{N}} \bc \Box \;|\; x.M_{A} \;|\; M_{M}+M_{N}}
  \and
  \inferrule* [lab=agent] {} {{M_{A}} \bc (\vec{x})M_{P} \;| \; \clift{P_0,\ldots,M_{P},\ldots,P_N}}
  \and \\
  \inferrule* [lab=process] {} {{M_{P}} \bc M_{N} \;| \;P|M_{P} }
\end{mathpar} 

\begin{mathpar}
  \inferrule* [lab=sychronization] {} {M_{N} \bc \Box \;|\; x?M_{F} \;|\; x!M_{C}}
  \and
  \inferrule* [lab=abstraction] {} {{M_{F}} \bc (x)M_{P} }
  \and
  \inferrule* [lab=concretion] {} {{M_{C}} \bc \langle M_{P} \rangle }
  \and \\
  \inferrule* [lab=process] {} {{M_{P}} \bc M_{N} \;| \;P|M_{P} }
\end{mathpar}

\begin{definition}[contextual application] Given a context $M$, and
  process $P$, we define the \emph{contextual application}, $M[P] :=
  M\{P/\Box\}$. That is, the contextual application of M to P is the
  substitution of $P$ for $\Box$ in $M$.
\end{definition}

$\meaningof{-} : L \to \mathcal{P}(\pi)$

\begin{mathpar}
  \inferrule* [lab=collection] {} {\meaningof{true} = \pi, \and \meaningof{~E} = \pi \setminus \meaningof{E}, \and \meaningof{E_{1} \& E_{2}} = \meaningof{E_{1}} \cap \meaningof{E_{2}}}
\end{mathpar}

\begin{mathpar}
  \inferrule* [lab=structure] {} {\meaningof{0} = \{ P \in \pi | P \equiv 0 \}, \and \\ \meaningof{E_1 | E_2} = \{ P \in \pi | P \equiv P_{1} | P_{2}, P_{1} \in \meaningof{E_{1}}, P_{2} \in \meaningof{E_2}\} }
\end{mathpar}

\begin{mathpar}
 \inferrule* [lab=behavior] {} {\meaningof{\langle a?b \rangle E} = \{ P \in \pi | P \equiv Q | u?(y)P', \\ \and \\\\ \and \\ \;\;\; u \in \meaningof{a}, \forall z.P'\{z/y\} \in \meaningof{E\{z/b\}}\}, \and \\ \meaningof{a!E} = \{ P \in \pi | P \equiv Q | x!\langle P' \rangle, x \in \meaningof{a} P' \in \meaningof{E}\} }
\end{mathpar}

\begin{mathpar}
 \inferrule* [lab=nominal] {} {\meaningof{\quotep{E}} = \{ \quotep{P} \in \quotep{\pi} | P \in \meaningof{E} \}, \and \meaningof{\quotep{P}} = \{ \quotep{Q} \in \quotep{\pi} | P \equiv Q \} \and \\ \meaningof{@\quotep{E}} = \{ P \in \pi | P \equiv @x, x \in \meaningof{E} \}}
\end{mathpar}

\begin{eqnarray*}
  \\
  \meaningof{-} : TS \to ST
\end{eqnarray*}

\begin{eqnarray*}
  \\
  L : TS \to ST
\end{eqnarray*}

\begin{eqnarray*}
  \\
  P \models E \iff P \in \meaningof{E}
\end{eqnarray*}

\begin{eqnarray*}
  P \approx_{L} Q \iff \forall E \in L. P \models E \iff Q \models E
\end{eqnarray*}

\begin{eqnarray*}
  P \approx_{K} Q
\end{eqnarray*}

\begin{eqnarray*}
  P \approx Q
\end{eqnarray*}

$\approx_{K} = \approx = \approx_{L}$

\subsubsection{Contextual duality}

Note that contexts extend the quotation operation to a family of
operations from processes to names. Given a context, $M$, we can
define a \emph{nominal context}, $\quotep{M}$ by $\quotep{M}[P] :=
\quotep{M[P]}$. To foreshadow what is to come we observe that these
operations enjoy a duality with processes very much like the duality
between vectors and maps from vectors to scalars.

Further, because the calculus is essentially higher-order, we have a
correspondence between contexts and processes. More specifically,
given a name $x$ and a context $M$ we can construct $M^{*}_{x}$ such
that 

\begin{mathpar}
  M^{*}_{x} | \lift{x}{P} \red M[P]
\end{mathpar}

namely,

\begin{mathpar}
  M^{*}_{x} := x?(u).M[\dropn{u}]
\end{mathpar}

The dependence of $M^{*}_{x}$ on a name makes it an abstraction, 

\begin{mathpar}
  M^{*} := (x)x?(u).M[\dropn{u}]
\end{mathpar}

\subsection{Additional notation}

It will sometimes be convenient to denote the process a name
quotes. We already have the notation $x = \quotep{P}$, but it will be
convenient to introduce an alternate notation, $\procn{x}$, when we
want to emphasize the connection to the use of the name. Note that, by
virtue of name equivalence, $\quotep{\procn{x}} \nameeq x$; so, the
notation is consistent with previous definitions.

Further, because names have structure it is possible to effect
substitutions on the basis of that structure. This means we need to
upgrade our notation for substitutions, which we accomplish by
adapting comprehension notation. Thus,

\begin{mathpar}
  P\{ y / x : x \in S \}
\end{mathpar}

is interpreted to mean the process derived from P by replacing (in a
capture-avoiding manner) each occurrence of $x$ in $S$ by $y$. For example,

\begin{mathpar}
  P\{ \quotep{\procn{x}|\procn{x}} / x : x \in \freenames{P} \}
\end{mathpar}

will replace each (occurrence) of a free name $x$ in $P$ by
$\quotep{\procn{x}|\procn{x}}$.

Also, we will avail ourselves of the notation $x^{L}$ and $x^{R}$ to
denote injections of a name into disjoint copies of the name
space. There are numerous ways to accomplish this. One example can be
found in \cite{MeredithR05}. This notation overloads to vectors of
names: $\vec{x}^{\pi} := (x_{i}^{\pi} \; : \; 0 \leq i < |\vec{x}| )$ where $\pi \in \{L,R\}$.

We also use $P^{\Box} := P|\Box$.

In \cite{MeredithR05} an interpretation of the new operator is
given. It turns out that there are several possible interpretations
all enjoying the requisite algebraic properties of the operator (see
\cite{milner91polyadicpi}). We will therefore make liberal use of
$(\nu\; \vec{x})P$.

% subsection the_syntax_and_semantics_of_the_notation_system (end)   

\input{qm2pi.qmops} 

\input{qm2pi.sterngerlach} 

\input{qm2pi.metric} 

% section concurrent_process_calculi (end)

%\input{qm2pi.proofsketch}

% section proof sketch (end)

%\input{qm2pi.slviaknots} 

% section spatial logic via knots (end)

\input{qm2pi.conclusion}

% section conclusion (end)

%\input{qm2pi.dtcodes} 

% section wiring algorithm (end)

\input{qm2pi.ack} 

% section acknowledgments (end)

\newpage


\bibliographystyle{plain}   
\bibliography{../../biblios/main.bib}

\input{qm2pi.rhodetails}

\end{document}



% section proof sketch (end)

%\section{Unlikely characters: spatial logic for
  knots}\label{sub:characteristic_formulae} % (fold)

Associated to the mobile process calculi are a family of logics known
as the Hennessy-Milner logics. These logics typically enjoy a
semantics interpreting formulae as sets of processes that when
factored through the encoding outlined above allows an identification
of classes of knots with logical formulae. In the context of this
encoding the sub-family known as the spatial logics \cite{CairesC03}
\cite{CairesC04} \cite{Caires04} are of particular interest providing
several important features for expressing and reasoning about
properties (i.e. classes) of knots. We hint here at how this may be done.

%\begin{description}
%\item [structural connectives] 
\subsubsection{Structural connectives} The spatial logics enjoy
structural connectives corresponding, at the logical level, to the
parallel composition ($P | Q$) and new name ($(\nu \; x)P$)
connectives for processes. As illustrated in the examples below, these
connectives are extremely expressive given the shape of our encoding.
%\item [decideable satisfaction]

\subsubsection{Decideable satisfaction}
In \cite{Caires04} the satisfaction relation is shown to be decideable
for a rich class of processes. It further turns out that the image of
the our encoding is a proper subset of that class. This result
provides the basis for an algorithm by which to search for knots
enjoying a given property.
%\item [characteristic formulae]

\subsubsection{Characteristic formulae}
In the same paper \cite{Caires04} , Caires presents a means of calculating
characteristic formulae, selecting equivalence classes of processes
up to a pre--specified depth limit on the support set of names. Composed with our
encoding, this characteristic formula can be used to select
characteristic formulae for knots.
%\end{description}

\subsubsection{Spatial logic formulae}

The grammar below (segmented for comprehension) summarizes the syntax
of spatial logic formulae. We employ illustrative examples in the
sequel to provide an intuitive understanding of their meaning
referring the reader to \cite{Caires04} for a more detailed explication
of the semantics.

\begin{mathpar}
  \inferrule* [lab=boolean] {} {{A,B} \bc T \;|\; \neg A \;|\; A \wedge B \;|\; \eta = \eta'}
  \and
  \inferrule* [lab=spatial] {} {|\; \pzero \;|\; A | B \;|\; x \text{\textregistered} A \;|\; \forall x . A \;|\;  H x . A}
  \and
  \inferrule* [lab=behavioral] {} {|\; \alpha . A}
  \and 
  \inferrule* [lab=recursion] {} {|\; X(\vec{u}) \;|\; \mu X(\vec{u}) . A}
  \and
  \inferrule* [lab=action] {} {\alpha \bc \langle x?(\vec{y}) \rangle \;|\; \langle x!(\vec{y}) \rangle \;|\; \langle \tau \rangle}
  \and 
  \inferrule* [lab=name] {} {\eta \bc x \;|\; \tau}
\end{mathpar} 

% subsection characteristic_formulae (end)   	 

\subsection{Example formulae}\label{sub:example_formulae_} % (fold)

\subsubsection{Crossing as formula.}
% 
% \begin{align*}
%   \frac{d}{dx} \sin x &= \cos x 
%   & \frac{d}{dx} e^x &= e^x \\
%   \frac{d}{dx} \cos x &= - \sin x 
%   & \frac{d}{dx} \log x &= \frac{1}{x} \\
% \end{align*} 

\begin{align*}
 \mu C(x_{0},x_{1},y_{0},y_{1},u).&(\langle x_{0}?(z) \rangle(\langle u! \rangle\langle y_{1}!z \rangle C(x_{0},x_{1},y_{0},y_{1},u)) & \\
  & \wedge \langle y_{1}?(z) \rangle (\langle u! \rangle \langle x_{0}!z \rangle C(x_{0},x_{1},y_{0},y_{1},u)) & \\
  & \wedge \langle x_{1}?(z) \rangle (\langle u? \rangle \langle y_{0}!z \rangle C(x_{0},x_{1},y_{0},y_{1},u)) & \\
  & \wedge \langle y_{0}?(z) \rangle (\langle u? \rangle \langle x_{1}!z \rangle C(x_{0},x_{1},y_{0},y_{1},u))) &
\end{align*}

The lexicographical similarity between the shape of this formulae and
the shape of definition of the process representing a crossing reveals
the intuitive meaning of this formulae. It describes the capabilities
of a process that has the right to represent a crossing. For example
it picks out processes that may perform an input on the port $x_0$ in
its initial menu of capabilities. What differentiates the formula
from the process, however, is that the crossing process is the
smallest candidate to satisfy the formula. Infinitely many other
processes -- with internal behavior hidden behind this interface, so
to speak -- also satisfy this formula. Even this simple formula,
then, can be seen to open a new view onto knots, providing a
computational interpretation of \emph{virtual} knots.

Note that this formula is derived by hand. A similar formula can be
derived by employing Caires' calculation of characteristic formula
\cite{Caires04} to the process representing a crossing. In light of
this discussion, we let
$\meaningof{C}_{\phi}(x0,x1,y0,y1,u)$ denote a formula specifying the
dynamics we wish to capture of a crossing. To guarantee we preserve
the shape of the interface and minimal semantics we demand that
$\meaningof{C}_{\phi}(x0,x1,y0,y1,u) \Rightarrow
\textbf{C}(x0,x1,y0,y1,u)$ where $\textbf{C}(x0,x1,y0,y1,u)$ denotes
the formula above.
                            
\subsubsection{Crossing number constraints.}
The moral content of the context lemma (Lemma \ref{context}) is that the notion of
``locality'' in the Reidemeister moves is effectively captured by the
parallel composition operator of the process calculus. This intuition
extends through the logic. Given a formula,
$\meaningof{C}_{\phi}(x0,x1,y0,y1,u)$, we can use the structural
connectives to specify constraints on crossing numbers, such as at
least $n$ crossings, or exactly $n$ crossings.
\begin{mathpar}
  \inferrule* [lab=at-least-n] {} { K^{\geq n}_{\phi}(\vec{xs},\vec{ys}) := \Pi_{i=0}^{n-1} Hu . \meaningof{C}_{\phi}(xs_i,ys_i,u) | T }
  \and 
  \inferrule* [lab=exactly-n] {} { K^{= n}_{\phi}(\vec{xs},\vec{ys}) := \Pi_{i=0}^{n-1} Hu . \meaningof{C}_{\phi}(xs_i,ys_i,u) | \neg (\forall x_0,y_0,x_1,y_1,u . \meaningof{C}_{\phi}(x_0,y_0,x_1,y_1,u) | T) }
\end{mathpar}

To round out this section, recall that the encoding of an $n$-crossing
knot decomposes into a parallel composition of $n$ \emph{copies} of a
crossing process together with a wiring harness. To specify different
knot classes with the same crossing number amounts to specifying
logical constraints on the wiring harness. In the interest of space,
we defer examples to a forthcoming paper. Suffice it to say that both
the conditions ``alternating knot'' and ``contains the tangle
corresponding to 5/3'' are expressible. For example, it is possible to
calculate the characteristic formula of a process corresponding to the
tangle 5/3 and conjoin it into the classifying formula via the
composition connective of the logic.

Finally, we wish to observe that it is entirely within reason to
contemplate a more domain-specific version of spatial logic tailored
to the shape of processes in the image of the encoding. Such a
domain-specific logic would have a better claim to the title formal
language of knot properties.

% subsection example_formulae_ (end)

% section knots_as_processes (end) 

% section spatial logic via knots (end)

\section{Conclusions and future work}

\paragraph{Testing physical space}
You, gentle reader, may wonder why of all the theorems to be proved
given this set up we pick the one above. In some sense it's hardly
central to quantum mechanics. We see it as central in the sense that
it firmly establishes a notion of physical space arising from a notion
of the equivalence of behavior. Relating bisimulation to a metric is a
big step forward, but one is faced with interpreting the relationship
of that metric space to something more physical. Quantum mechanical
notions of ``physical'' space are still far from intuitive, but by
relating this idea of distance as testing to calculations that predict
physical circumstances we are making a not insignificant step forward
toward an understanding of the physical space we inhabit as
essentially dynamic.

\paragraph{Effectivity and simulation}
One of the observations we have yet to make is that the entire program
spelled out here is effective. We have built various interpreters for
the reflective calculus at work in this interpretation. In principle,
then, we can simulate quantum mechanics on a computer. The place where
the simulation may lose fidelity is the infinitely branching summation
for the annihilator.

In this connection i also want to point out that the evaluation style
calculation of the inner product puts the non-determinism of the
summation right at the heart of measurement. This suggests that
Milner's original reduction-based formulation of the dynamics of his
calculi in terms of sums was not just notationally suggestive of a
notion of measure-and-continue but captured some significant part of
the physics.

\paragraph{Quantum continuations}
In light of this last observation i want to point out that the
predominant account of quantum mechanics is missing a key aspect of a
truly compositional story of the physical situation. In a real lab,
when a measurement is made the observation can be made to feed into
another device that then makes another measurement conditioned on the
results of the first. This means that after the superposition was
collapsed the entire experimental set up remained in
superposition. While QM offers a means of writing this down it doesn't
quite line up well with the well-trodden formulation of computation
and continuation that we see so succinctly expressed in Milner's
calculi. This suggests that there might be advantages to this account
of dynamics waiting to be explored.

\paragraph{Quantum logic}
In this connection, we also note that by virtue of having the
Hennessy-Milner construction, we can pull the construction through the
interpretation of QM. This gives us a natural candidate for a quantum
logic that enjoys an extremely tight connection with it's domain of
interpretation, making the construction much less ad hoc (rather it is
the image of functor!).

\paragraph{Quantum probabiity}
i have questions about the basis of the interpretation of inner
product as probability amplitude. In particular, using which
axiomatization of probability theory does the notion of probability
amplitude earn the right to be so dubbed? In other words, where is the
proof that the operation for calculating a probability amplitude (and
then squaring) satisfies the axioms of what it means to calculate a
probability? Even if such a proof exists (i have yet to find it in the
literature), i wonder if it might not be possible to turn things on
their heads. Can we view the calculation of the probability amplitude
as an axiomatization of probability? If so, then the definition we
give for calculating probability amplitude may provide the basis for
an \emph{effective} theory of probability.

\paragraph{Quantum vs ``biological'' information}
Finally, i want to conclude with a more philosophical observation. At
a recent workshop in which QM was a predominant topic i noticed
something about quantum information. The speaker was giving a riveting
discussion of axiomatic QM and showing how properties of ``no
cloning'' and ``no deleting'' emerged as consequences of the
axiomatization. Theorems of this form are necessary to give us a sense
of confidence that our axioms characterize the physical theory. What
struck me, though, was that if quantum information is neither erasable
nor replicable it is markedly different from \emph{life}. Two of the
things we know about life is that

\begin{itemize}
  \item it ends;
  \item to gain some measure of persistence, to transcend it's
    finitude it is imminently copyable.
\end{itemize}

Both of these qualities are summarized succinctly in the aphorism: all
flesh is grass. For me these two kinds of ``information'' -- call them
quantum and biological -- are end points on a spectrum of strategies
for persistence. At one end, we have those curious entities that enjoy
uniqueness and permanence; at the other, we have those who in the face
of a certain end and an uncertain present make a go of passing
something on. To me one of the more remarkable aspects of the latter
strategy is that in the presence of noise (and certain features of
copying) we get a kind of dynamism, a chance for improvement against a
given persistent condition.

% subsection other_calculi_other_bisimulations_and_geometry_as_behavior (end)




% section conclusion (end)

%\documentclass[12pt]{llncs}
%\documentclass{jktr}

\usepackage[pdftex]{hyperref}                   
\usepackage {listings}
\usepackage {mathpartir}
\usepackage{bcprules}
%\usepackage{listings}
                       
\usepackage{graphicx} 
%\usepackage[margins=2.5cm,nohead,nofoot]{geometry}
%\usepackage{geometry}
\usepackage{amsfonts}
\usepackage{amstext}
\usepackage{latexsym}
\usepackage{amssymb}
\usepackage{color}


%\include{myPreamble}
\include{qm2pi.local} 

%\ifpdf
%\usepackage[pdftex]{graphicx}
%\else
%\usepackage{graphicx}
%\fi

 % \ifpdf
%  \usepackage{pdfsync}
%  \if


%\title{Brief Article}
%\author{David F. Snyder}
%\author{L.G. Meredith}

%\address{Dept. of Math., Texas State University--San Marcos, San Marcos, TX 78666}
       
\pagestyle{empty}


\begin{document}

\lstset{language=[Objective]Caml,frame=shadowbox}

\input{qm2pi.front}

% section front matter (end)

\input{qm2pi.intro} 
 
% section introduction (end)

% \input{qm2pi.knotations} 

% section notation (end)

\input{qm2pi.process.calculi} 

% section concurrent_process_calculi_and_spatial_logics_ (end)
    
%\input{qm2pi.knots2pi} 

%\input{qm2pi.trefoil} 

%\input{qm2pi.mainthm} 

% subsection basic_interpretation (end)

%\input{qm2pi.rho.presentation} 
\subsection{The syntax and semantics of the notation system}\label{sub:the_syntax_and_semantics_of_the_notation_system} % (fold)

We now summarize a technical presentation of the calculus that
embodies our theory of dynamics. The typical presentation of such a
calculus follows the style of giving generators and relations on
them. The grammar, below, describing term constructors, freely
generates the set of processes, $\Proc$. This set is then quotiented
by a relation known as structural congruence and it is over this set
that the notion of dynamics is expressed. This presentation is
essentially that of \cite{MeredithR05} with the addition of
polyadicity and summation. For readability we have relegated some of
the technical subtleties to an appendix.

\subsubsection{Process grammar}\label{subsub:process_grammar}

\begin{mathpar}
  \inferrule* [lab=synchronization] {} {{M} \bc \pzero \;|\; x?F \;|\; x!C }
  \and
  \inferrule* [lab=abstraction] {} {{F} \bc (x)P}
  \and
  \inferrule* [lab=concretion] {} {{C} \bc \langle Q \rangle}
  \and
  \inferrule* [lab=process] {} {{P,Q} \bc M \;| \;P|Q \;|\; @{x}}
  \and
  \inferrule* [lab=name] {} {{x} \bc \quotep{P}}
\end{mathpar} 

Note that $\vec{x}$ (resp. $\vec{P}$) denotes a vector of names
(resp. processes) of length $|\vec{x}|$ (resp. $|\vec{P}|$). We adopt
the following useful abbreviations.

\begin{mathpar}
   x?(\vec{y}).P := x.(\vec{y})P \and  x\clift{\vec{P}} := x.\clift{\vec{P}}
   \and x!(y) := \lift{x}{\dropn{y}}
   \and \Pi_{i=0}^{n-1}P_i := P_0 | \ldots | P_{n-1}
\end{mathpar}

\subsubsection{Structural congruence}

\paragraph{Free and bound names and alpha-equivalence.} At the
core of structural equivalence is alpha-equivalence which identifies
process that are the same up to a change of variable. Formally, we
recognize the distinction between free and bound names. The free names
of a process, $\freenames{P}$, may be calculated recursively as
follows:

\begin{mathpar}
\freenames{\pzero} := \emptyset
  \and \\
  \freenames{x?(y).P} := \{ x \} \cup (\freenames{P} \setminus \{ y \})
  \and 
  \freenames{x!\langle P \rangle} := \{ x \} \cup \{ P \} 
  \and \\
  \freenames{P|Q} := \freenames{P} \cup \freenames{Q}
  \and \\
  \freenames{@{x}} := \{ x \}
\end{mathpar}

$\pi$
$\quotep{\pi}$

$\freenames{-} : \pi \to \mathcal{P}(\quotep{\pi})$

\begin{eqnarray*}
  \freenames{\pzero} & := & \emptyset \\
  \freenames{x?(y).P} & := & \{ x \} \cup (\freenames{P} \setminus \{ y \}) \\
  \freenames{x!\langle P \rangle} & := & \{ x \} \cup \{ P \} \\
  \freenames{P|Q} & := & \freenames{P} \cup \freenames{Q} \\
  \freenames{\dropn{x}} & := & \{ x \}
\end{eqnarray*}

The bound names of a process, $\boundnames{P}$, are those names occurring in $P$
that are not free. For example, in $x?(y).0$, the name $x$ is free, while $y$ is bound.

\begin{mathpar}
  \inferrule* [lab=monoidal-laws] {} { P|Q \equiv Q|P \and P|0 \equiv P \and P|(Q|R) \equiv (P|Q)|R }
\end{mathpar}

\begin{mathpar}
  \inferrule* [lab=alpha-equivalence] {} { (x)P \equiv (y)P\{y/x\} \and y \not\in \freenames{P} }
\end{mathpar}

\begin{definition}
Then two processes, $P,Q$, are alpha-equivalent if $P = Q\{\vec{y}/\vec{x}\}$ for
some $\vec{x} \in \boundnames{Q},\vec{y} \in \boundnames{P}$, where $Q\{\vec{y}/\vec{x}\}$
denotes the capture-avoiding substitution of $\vec{y}$ for $\vec{x}$ in $Q$.
\end{definition}

\begin{definition}
  The {\em structural congruence} \cite{SangiorgiWalker} , $\equiv$,
  between processes is the least congruence containing
  alpha-equivalence, satisfying the abelian monoid laws
  (associativity, commutativity and $\pzero$ as identity) for parallel
  composition $|$ and for summation $+$.
\end{definition}

\subsection{Name equivalence}

We take name equivalence, written $\nameeq$, to be the smallest
equivalence relation generated by the following rules.

\begin{mathpar}
\inferrule*[lab=Quote-drop]
{ }
{ \quotep{@{x}} \nameeq x }

\inferrule*[lab=Struct-equiv]
{ P \scong Q }
{ \quotep{P} \nameeq \quotep{Q} }
\end{mathpar}

The astute reader will have noticed that the mutual recursion of names
and processes imposes a mutual recursion on alpha-equivalence and
structural equivalence via name-equivalence. Fortunately, all of this
works out pleasantly and we may calculate in the natural way, free of
concern. The reader interested in the details is referred to the
appendix \ref{appendix:rho_details}.

\subsection{Substitution}

We use $\Proc$ for the set of processes, $\QProc$ for the set of
names, and $\id{\{}\vec{y} / \vec{x} \id{\}}$ to denote partial maps,
$s : \QProc \rightarrow \QProc$. A map, $s$ lifts, uniquely, to a map
on process terms, $\widehat{s} : \Proc \rightarrow \Proc$ by the
following equations.

\begin{mathpar}
  (0) \psubstp{Q}{P} := 0 \\
  (R \juxtap S) \psubstp{Q}{P}
  :=    
  (R)\psubstp{Q}{P} \juxtap (S) \psubstp{Q}{P} \\
  (x?(y).R) \psubstp{Q}{P}    
  :=    
  (x)\substp{Q}{P} (z)\concat( (R \psubstn{z}{y}) \psubstp{Q}{P} ) \\
  (\lift{x}{R}) \psubstp{Q}{P}  
  :=
  \lift{(x)\substp{Q}{P}}{ R \psubstp{Q}{P} } \\
%   (\dropn{x})  \psubstp{Q}{P}       
%   := 
%   \left\{ 
%     \begin{array}{ccc} 
%       \dropn{\quotep{Q}} & & x \nameeq \quotep{P} \\
%       \dropn{x} & & otherwise \\
%     \end{array}
%   \right. 
  (\dropn{x})  \psubstp{Q}{P}       
  := 
  \left\{ 
    \begin{array}{ccc} 
      Q & & x \nameeq \quotep{P} \\
      \dropn{x} & & otherwise \\
    \end{array}
  \right.
\end{mathpar}
 

where

\begin{eqnarray}
  (x)\id{\{} \lpquote Q \rpquote / \lpquote P \rpquote \id{\}}            = 
  \left\{ 
    \begin{array}{ccc}
      \lpquote Q \rpquote & & x \nameeq \lpquote P \rpquote \\
      x & & otherwise \\
    \end{array}
  \right. \nonumber
\end{eqnarray}

and $z$ is chosen distinct from $\quotep{P}$, $\quotep{Q}$, the free
names in $Q$, and all the names in $R$. Our $\alpha$-equivalence will
be built in the standard way from this substitution.

\begin{remark}\label{rem:no_self_referential_names}
  One consequence of these definitions is that $\forall P. \quotep{P}
  \not\in \freenames{P}$.
\end{remark}

\subsection{ Dynamic quote: an example }

Anticipating something of what's to come, consider applying the
substitution, $\widehat{\id{\{}u / z \id{\}}}$, to the following pair
of processes, $\lift{w}{y!(z)}$ and $w[ \lpquote y!(z) \rpquote ]$.

\begin{eqnarray}
	\lift{w}{y!(z)}\widehat{\id{\{}u / z \id{\}}}
		& = &
		\lift{w}{y!(u)} \nonumber\\
	w[ \lpquote y!(z) \rpquote ] \widehat{ \id{\{}u / z \id{\}} }
		& = &
		w[ \lpquote y!(z) \rpquote ] \nonumber
\end{eqnarray}

Because the body of the process between quotes is impervious to
substitution, we get radically different answers. In fact, by
examining the first process in an input context,
e.g. $x?(z).\lift{w}{y!(z)}$, we see that the process under the lift
operator may be shaped by prefixed inputs binding a name inside it. In
this sense, the lift operator will be seen as a way to dynamically
construct processes before reifying them as names.

Finally equipped with these standard features we can present the
dynamics of the calculus.

\subsubsection{Operational semantics} 

Finally, we introduce the computational dynamics. What marks these
algebras as distinct from other more traditionally studied algebraic
structures, e.g. vector spaces or polynomial rings, is the manner in
which dynamics is captured. In traditional structures, dynamics is typically
expressed through morphisms between such structures, as in linear maps
between vector spaces or morphisms between rings. In algebras
associated with the semantics of computation, the dynamics is
expressed as part of the algebraic structure itself, through a
reduction reduction relation typically denoted by $\red$. Below, we
give a recursive presentation of this relation for the calculus used
in the encoding.

$\red \subseteq \pi \times \pi$
$\red : \pi \to \mathcal{P}(\pi)$

\begin{mathpar}
  \inferrule* [lab=Comm] { \textsf{match}( x_{src}, x_{trgt} ) } { x_{trgt}?(y)P \; | \; x_{src}!\langle {Q} \rangle \red P\{\quotep{Q}/y}\} }
  \and \\
  \inferrule* [lab=Par] {{P} \red {P}'} {{{P} | {Q}} \red {{P}' | {Q}}}
  \and
  \inferrule* [lab=Equiv]{{{P} \scong {P}'} \andalso {{P}' \red {Q}'} \andalso {{Q}' \scong {Q}}}{{P} \red {Q}}
\end{mathpar}

\begin{eqnarray*}
  match_{\equiv} (\quotep{P},\quotep{Q}) & := & P \equiv Q \\
  match_{\dagger}(\quotep{P},\quotep{Q}) & := & \forall R. P|Q \red^{*} R => R \red^{*} 0 \\
  match_{K}(\quotep{P},\quotep{Q}) & := & K \mbox{ for some context } K
\end{eqnarray*}

$u?(x)P | u!\langle Q \rangle \red P\{\quotep{Q}/x\}$

%We write $\wred$ for $\red^*$, and $P\red$ if $\exists Q $ such that $ P \red Q$.
We write $P\red$ if $\exists Q $ such that $ P \red Q$ and $P\not\red$, otherwise.

\section{Replication}

As mentioned before, it is known that replication (and hence
recursion) can be implemented in a higher-order process algebra
\cite{SangiorgiWalker}. As our first example of calculation with the
machinery thus far presented we give the construction explicitly in
the {\rhoc}.

\begin{eqnarray}
	D_{x} & := & \prefix{x}{y}{(\binpar{\outputp{x}{y}}{@{y}})} \nonumber\\
	\bangp_{x}{P} & := & \binpar{{x}!\langle{\binpar{D_{x}}{P}}\rangle}{D_{x}} \nonumber
\end{eqnarray}

\begin{eqnarray}
	\bangp_{x}{P} & & \nonumber\\
	=
	& {x}!\langle{(\prefix{x}{y}{(\outputp{x}{y} | @{y})) | P}}\rangle 
	      | \prefix{x}{y}{(\outputp{x}{y} | @{y})} & \nonumber\\
	\red
	& (\outputp{x}{y} | @{y})\substn{\quotep{(\prefix{x}{y}{(@{y} | \outputp{x}{y})) | P}}}{y} & \nonumber\\
	=
	& \outputp{x}{\quotep{(\prefix{x}{y}{(\outputp{x}{y} | @{y})) | P}}}
	  | {(\prefix{x}{y}{(\outputp{x}{y} | @{y})) | P}} & \nonumber\\
	\red
	& \ldots & \nonumber\\
	\red^*
	& P | P | \ldots & \nonumber
\end{eqnarray}

Of course, this encoding, as an implementation, runs away, unfolding
$\bangp{P}$ eagerly. A lazier and more implementable replication
operator, restricted to input-guarded processes, may be obtained as follows.

\begin{eqnarray}
\bangp{\prefix{u}{v}{P}} 
	:= 
	\binpar{\lift{x}{\prefix{u}{v}{(\binpar{D(x)}{P})}}}{D(x)} \nonumber
\end{eqnarray}

\begin{remark}
  Note that the lazier definition still does not deal with summation
  or mixed summation (i.e. sums over input and output). The reader is
  invited to construct definitions of replication that deal with these
  features. 

  Further, the definitions are parameterized in a name, $x$. Can you,
  gentle reader, make a definition that eliminates this parameter and
  guarantees no accidental interaction between the replication
  machinery and the process being replicated -- i.e. no accidental
  sharing of names used by the process to get its work done and the
  name(s) used by the replication to effect copying. This latter
  revision of the definition of replication is crucial to obtaining
  the expected identity $!!P \sim !P$.
\end{remark}

\begin{remark}\label{rem:paradoxical_combinator}
  The reader familiar with the lambda calculus will have noticed the
  similarity between $D$ and the paradoxical combinator.

  [Ed. note: the existence of this seems to suggest we have to be more
  restrictive on the set of processes and names we admit if we are to
  support no-cloning.]
\end{remark}

\subsubsection{Bisimulation}

The computational dynamics gives rise to another kind of equivalence,
the equivalence of computational behavior. As previously mentioned
this is typically captured \emph{via} some form of bisimulation.

% The notion we use in this paper is weak barbed bisimulation
% \cite{milner91polyadicpi}.

The notion we use in this paper is derived from weak barbed
bisimulation \cite{milner91polyadicpi}. 

\begin{definition}
An \emph{observation relation}, $\downarrow_{\mathcal N}$, over a set
of names, $\mathcal N$, is the smallest relation satisfying the rules
below.

\infrule[Out-barb]{y \in {\mathcal N}, \; x \nameeq y}
		  {\outputp{x}{v} \downarrow_{\mathcal N} x}
\infrule[Par-barb]{\mbox{$P\downarrow_{\mathcal N} x$ or $Q\downarrow_{\mathcal N} x$}}
		  {\binpar{P}{Q} \downarrow_{\mathcal N} x}

We write $P \Downarrow_{\mathcal N} x$ if there is $Q$ such that 
$P \wred Q$ and $Q \downarrow_{\mathcal N} x$.
\end{definition}

\begin{definition}
%\label{def.bbisim}
An  ${\mathcal N}$-\emph{barbed bisimulation} over a set of names, ${\mathcal N}$, is a symmetric binary relation 
${\mathcal S}_{\mathcal N}$ between agents such that $P\rel{S}_{\mathcal N}Q$ implies:
\begin{enumerate}
\item If $P \red P'$ then $Q \wred Q'$ and $P'\rel{S}_{\mathcal N} Q'$.
\item If $P\downarrow_{\mathcal N} x$, then $Q\Downarrow_{\mathcal N} x$.
\end{enumerate}
$P$ is ${\mathcal N}$-barbed bisimilar to $Q$, written
$P \wbbisim_{\mathcal N} Q$, if $P \rel{S}_{\mathcal N} Q$ for some ${\mathcal N}$-barbed bisimulation ${\mathcal S}_{\mathcal N}$.
\end{definition}

$\mathcal{R} \subseteq \pi \times \pi$

$P \mathcal{R} Q => \forall P'. P \red P' \Rightarrow \exists Q'. Q \red Q', P' \mathcal{R} Q'$

$P \vdash x \Rightarrow Q \vdash x$

\begin{mathpar}
  \inferrule*[lab=Out-barb]{x \nameeq y}{{y}!\langle{Q}\rangle \vdash x}
  \and
  \inferrule*[lab=Par-barb]{\mbox{$P\vdash x$ or $Q\vdash x$}}{\binpar{P}{Q} \vdash x}
\end{mathpar}

\subsubsection{Contexts}

One of the principle advantages of computational calculi like the
$\pi$-calculus is a well-defined notion of context,
contextual-equivalence and a correlation between
contextual-equivalence and notions of bisimulation. The notion of
context allows the decomposition of a process into (sub-)process and
its syntactic environment, its context. Thus, a context may be
thought of as a process with a ``hole'' (written $\Box$) in it. The
application of a context $M$ to a process $P$, written $M[P]$, is
tantamount to filling the hole in $M$ with $P$. In this paper we do
not need the full weight of this theory, but do make use of the notion
of context in the proof the main theorem. 

\begin{mathpar}
  \inferrule* [lab=summation] {} {{M_{M},M_{N}} \bc \Box \;|\; x.M_{A} \;|\; M_{M}+M_{N}}
  \and
  \inferrule* [lab=agent] {} {{M_{A}} \bc (\vec{x})M_{P} \;| \; \clift{P_0,\ldots,M_{P},\ldots,P_N}}
  \and \\
  \inferrule* [lab=process] {} {{M_{P}} \bc M_{N} \;| \;P|M_{P} }
\end{mathpar} 

\begin{mathpar}
  \inferrule* [lab=sychronization] {} {M_{N} \bc \Box \;|\; x?M_{F} \;|\; x!M_{C}}
  \and
  \inferrule* [lab=abstraction] {} {{M_{F}} \bc (x)M_{P} }
  \and
  \inferrule* [lab=concretion] {} {{M_{C}} \bc \langle M_{P} \rangle }
  \and \\
  \inferrule* [lab=process] {} {{M_{P}} \bc M_{N} \;| \;P|M_{P} }
\end{mathpar}

\begin{definition}[contextual application] Given a context $M$, and
  process $P$, we define the \emph{contextual application}, $M[P] :=
  M\{P/\Box\}$. That is, the contextual application of M to P is the
  substitution of $P$ for $\Box$ in $M$.
\end{definition}

$\meaningof{-} : L \to \mathcal{P}(\pi)$

\begin{mathpar}
  \inferrule* [lab=collection] {} {\meaningof{true} = \pi, \and \meaningof{~E} = \pi \setminus \meaningof{E}, \and \meaningof{E_{1} \& E_{2}} = \meaningof{E_{1}} \cap \meaningof{E_{2}}}
\end{mathpar}

\begin{mathpar}
  \inferrule* [lab=structure] {} {\meaningof{0} = \{ P \in \pi | P \equiv 0 \}, \and \\ \meaningof{E_1 | E_2} = \{ P \in \pi | P \equiv P_{1} | P_{2}, P_{1} \in \meaningof{E_{1}}, P_{2} \in \meaningof{E_2}\} }
\end{mathpar}

\begin{mathpar}
 \inferrule* [lab=behavior] {} {\meaningof{\langle a?b \rangle E} = \{ P \in \pi | P \equiv Q | u?(y)P', \\ \and \\\\ \and \\ \;\;\; u \in \meaningof{a}, \forall z.P'\{z/y\} \in \meaningof{E\{z/b\}}\}, \and \\ \meaningof{a!E} = \{ P \in \pi | P \equiv Q | x!\langle P' \rangle, x \in \meaningof{a} P' \in \meaningof{E}\} }
\end{mathpar}

\begin{mathpar}
 \inferrule* [lab=nominal] {} {\meaningof{\quotep{E}} = \{ \quotep{P} \in \quotep{\pi} | P \in \meaningof{E} \}, \and \meaningof{\quotep{P}} = \{ \quotep{Q} \in \quotep{\pi} | P \equiv Q \} \and \\ \meaningof{@\quotep{E}} = \{ P \in \pi | P \equiv @x, x \in \meaningof{E} \}}
\end{mathpar}

\begin{eqnarray*}
  \\
  \meaningof{-} : TS \to ST
\end{eqnarray*}

\begin{eqnarray*}
  \\
  L : TS \to ST
\end{eqnarray*}

\begin{eqnarray*}
  \\
  P \models E \iff P \in \meaningof{E}
\end{eqnarray*}

\begin{eqnarray*}
  P \approx_{L} Q \iff \forall E \in L. P \models E \iff Q \models E
\end{eqnarray*}

\begin{eqnarray*}
  P \approx_{K} Q
\end{eqnarray*}

\begin{eqnarray*}
  P \approx Q
\end{eqnarray*}

$\approx_{K} = \approx = \approx_{L}$

\subsubsection{Contextual duality}

Note that contexts extend the quotation operation to a family of
operations from processes to names. Given a context, $M$, we can
define a \emph{nominal context}, $\quotep{M}$ by $\quotep{M}[P] :=
\quotep{M[P]}$. To foreshadow what is to come we observe that these
operations enjoy a duality with processes very much like the duality
between vectors and maps from vectors to scalars.

Further, because the calculus is essentially higher-order, we have a
correspondence between contexts and processes. More specifically,
given a name $x$ and a context $M$ we can construct $M^{*}_{x}$ such
that 

\begin{mathpar}
  M^{*}_{x} | \lift{x}{P} \red M[P]
\end{mathpar}

namely,

\begin{mathpar}
  M^{*}_{x} := x?(u).M[\dropn{u}]
\end{mathpar}

The dependence of $M^{*}_{x}$ on a name makes it an abstraction, 

\begin{mathpar}
  M^{*} := (x)x?(u).M[\dropn{u}]
\end{mathpar}

\subsection{Additional notation}

It will sometimes be convenient to denote the process a name
quotes. We already have the notation $x = \quotep{P}$, but it will be
convenient to introduce an alternate notation, $\procn{x}$, when we
want to emphasize the connection to the use of the name. Note that, by
virtue of name equivalence, $\quotep{\procn{x}} \nameeq x$; so, the
notation is consistent with previous definitions.

Further, because names have structure it is possible to effect
substitutions on the basis of that structure. This means we need to
upgrade our notation for substitutions, which we accomplish by
adapting comprehension notation. Thus,

\begin{mathpar}
  P\{ y / x : x \in S \}
\end{mathpar}

is interpreted to mean the process derived from P by replacing (in a
capture-avoiding manner) each occurrence of $x$ in $S$ by $y$. For example,

\begin{mathpar}
  P\{ \quotep{\procn{x}|\procn{x}} / x : x \in \freenames{P} \}
\end{mathpar}

will replace each (occurrence) of a free name $x$ in $P$ by
$\quotep{\procn{x}|\procn{x}}$.

Also, we will avail ourselves of the notation $x^{L}$ and $x^{R}$ to
denote injections of a name into disjoint copies of the name
space. There are numerous ways to accomplish this. One example can be
found in \cite{MeredithR05}. This notation overloads to vectors of
names: $\vec{x}^{\pi} := (x_{i}^{\pi} \; : \; 0 \leq i < |\vec{x}| )$ where $\pi \in \{L,R\}$.

We also use $P^{\Box} := P|\Box$.

In \cite{MeredithR05} an interpretation of the new operator is
given. It turns out that there are several possible interpretations
all enjoying the requisite algebraic properties of the operator (see
\cite{milner91polyadicpi}). We will therefore make liberal use of
$(\nu\; \vec{x})P$.

% subsection the_syntax_and_semantics_of_the_notation_system (end)   

\input{qm2pi.qmops} 

\input{qm2pi.sterngerlach} 

\input{qm2pi.metric} 

% section concurrent_process_calculi (end)

%\input{qm2pi.proofsketch}

% section proof sketch (end)

%\input{qm2pi.slviaknots} 

% section spatial logic via knots (end)

\input{qm2pi.conclusion}

% section conclusion (end)

%\input{qm2pi.dtcodes} 

% section wiring algorithm (end)

\input{qm2pi.ack} 

% section acknowledgments (end)

\newpage


\bibliographystyle{plain}   
\bibliography{../../biblios/main.bib}

\input{qm2pi.rhodetails}

\end{document}

 

% section wiring algorithm (end)

\documentclass[12pt]{llncs}
%\documentclass{jktr}

\usepackage[pdftex]{hyperref}                   
\usepackage {listings}
\usepackage {mathpartir}
\usepackage{bcprules}
%\usepackage{listings}
                       
\usepackage{graphicx} 
%\usepackage[margins=2.5cm,nohead,nofoot]{geometry}
%\usepackage{geometry}
\usepackage{amsfonts}
\usepackage{amstext}
\usepackage{latexsym}
\usepackage{amssymb}
\usepackage{color}


%\include{myPreamble}
\include{qm2pi.local} 

%\ifpdf
%\usepackage[pdftex]{graphicx}
%\else
%\usepackage{graphicx}
%\fi

 % \ifpdf
%  \usepackage{pdfsync}
%  \if


%\title{Brief Article}
%\author{David F. Snyder}
%\author{L.G. Meredith}

%\address{Dept. of Math., Texas State University--San Marcos, San Marcos, TX 78666}
       
\pagestyle{empty}


\begin{document}

\lstset{language=[Objective]Caml,frame=shadowbox}

\input{qm2pi.front}

% section front matter (end)

\input{qm2pi.intro} 
 
% section introduction (end)

% \input{qm2pi.knotations} 

% section notation (end)

\input{qm2pi.process.calculi} 

% section concurrent_process_calculi_and_spatial_logics_ (end)
    
%\input{qm2pi.knots2pi} 

%\input{qm2pi.trefoil} 

%\input{qm2pi.mainthm} 

% subsection basic_interpretation (end)

%\input{qm2pi.rho.presentation} 
\subsection{The syntax and semantics of the notation system}\label{sub:the_syntax_and_semantics_of_the_notation_system} % (fold)

We now summarize a technical presentation of the calculus that
embodies our theory of dynamics. The typical presentation of such a
calculus follows the style of giving generators and relations on
them. The grammar, below, describing term constructors, freely
generates the set of processes, $\Proc$. This set is then quotiented
by a relation known as structural congruence and it is over this set
that the notion of dynamics is expressed. This presentation is
essentially that of \cite{MeredithR05} with the addition of
polyadicity and summation. For readability we have relegated some of
the technical subtleties to an appendix.

\subsubsection{Process grammar}\label{subsub:process_grammar}

\begin{mathpar}
  \inferrule* [lab=synchronization] {} {{M} \bc \pzero \;|\; x?F \;|\; x!C }
  \and
  \inferrule* [lab=abstraction] {} {{F} \bc (x)P}
  \and
  \inferrule* [lab=concretion] {} {{C} \bc \langle Q \rangle}
  \and
  \inferrule* [lab=process] {} {{P,Q} \bc M \;| \;P|Q \;|\; @{x}}
  \and
  \inferrule* [lab=name] {} {{x} \bc \quotep{P}}
\end{mathpar} 

Note that $\vec{x}$ (resp. $\vec{P}$) denotes a vector of names
(resp. processes) of length $|\vec{x}|$ (resp. $|\vec{P}|$). We adopt
the following useful abbreviations.

\begin{mathpar}
   x?(\vec{y}).P := x.(\vec{y})P \and  x\clift{\vec{P}} := x.\clift{\vec{P}}
   \and x!(y) := \lift{x}{\dropn{y}}
   \and \Pi_{i=0}^{n-1}P_i := P_0 | \ldots | P_{n-1}
\end{mathpar}

\subsubsection{Structural congruence}

\paragraph{Free and bound names and alpha-equivalence.} At the
core of structural equivalence is alpha-equivalence which identifies
process that are the same up to a change of variable. Formally, we
recognize the distinction between free and bound names. The free names
of a process, $\freenames{P}$, may be calculated recursively as
follows:

\begin{mathpar}
\freenames{\pzero} := \emptyset
  \and \\
  \freenames{x?(y).P} := \{ x \} \cup (\freenames{P} \setminus \{ y \})
  \and 
  \freenames{x!\langle P \rangle} := \{ x \} \cup \{ P \} 
  \and \\
  \freenames{P|Q} := \freenames{P} \cup \freenames{Q}
  \and \\
  \freenames{@{x}} := \{ x \}
\end{mathpar}

$\pi$
$\quotep{\pi}$

$\freenames{-} : \pi \to \mathcal{P}(\quotep{\pi})$

\begin{eqnarray*}
  \freenames{\pzero} & := & \emptyset \\
  \freenames{x?(y).P} & := & \{ x \} \cup (\freenames{P} \setminus \{ y \}) \\
  \freenames{x!\langle P \rangle} & := & \{ x \} \cup \{ P \} \\
  \freenames{P|Q} & := & \freenames{P} \cup \freenames{Q} \\
  \freenames{\dropn{x}} & := & \{ x \}
\end{eqnarray*}

The bound names of a process, $\boundnames{P}$, are those names occurring in $P$
that are not free. For example, in $x?(y).0$, the name $x$ is free, while $y$ is bound.

\begin{mathpar}
  \inferrule* [lab=monoidal-laws] {} { P|Q \equiv Q|P \and P|0 \equiv P \and P|(Q|R) \equiv (P|Q)|R }
\end{mathpar}

\begin{mathpar}
  \inferrule* [lab=alpha-equivalence] {} { (x)P \equiv (y)P\{y/x\} \and y \not\in \freenames{P} }
\end{mathpar}

\begin{definition}
Then two processes, $P,Q$, are alpha-equivalent if $P = Q\{\vec{y}/\vec{x}\}$ for
some $\vec{x} \in \boundnames{Q},\vec{y} \in \boundnames{P}$, where $Q\{\vec{y}/\vec{x}\}$
denotes the capture-avoiding substitution of $\vec{y}$ for $\vec{x}$ in $Q$.
\end{definition}

\begin{definition}
  The {\em structural congruence} \cite{SangiorgiWalker} , $\equiv$,
  between processes is the least congruence containing
  alpha-equivalence, satisfying the abelian monoid laws
  (associativity, commutativity and $\pzero$ as identity) for parallel
  composition $|$ and for summation $+$.
\end{definition}

\subsection{Name equivalence}

We take name equivalence, written $\nameeq$, to be the smallest
equivalence relation generated by the following rules.

\begin{mathpar}
\inferrule*[lab=Quote-drop]
{ }
{ \quotep{@{x}} \nameeq x }

\inferrule*[lab=Struct-equiv]
{ P \scong Q }
{ \quotep{P} \nameeq \quotep{Q} }
\end{mathpar}

The astute reader will have noticed that the mutual recursion of names
and processes imposes a mutual recursion on alpha-equivalence and
structural equivalence via name-equivalence. Fortunately, all of this
works out pleasantly and we may calculate in the natural way, free of
concern. The reader interested in the details is referred to the
appendix \ref{appendix:rho_details}.

\subsection{Substitution}

We use $\Proc$ for the set of processes, $\QProc$ for the set of
names, and $\id{\{}\vec{y} / \vec{x} \id{\}}$ to denote partial maps,
$s : \QProc \rightarrow \QProc$. A map, $s$ lifts, uniquely, to a map
on process terms, $\widehat{s} : \Proc \rightarrow \Proc$ by the
following equations.

\begin{mathpar}
  (0) \psubstp{Q}{P} := 0 \\
  (R \juxtap S) \psubstp{Q}{P}
  :=    
  (R)\psubstp{Q}{P} \juxtap (S) \psubstp{Q}{P} \\
  (x?(y).R) \psubstp{Q}{P}    
  :=    
  (x)\substp{Q}{P} (z)\concat( (R \psubstn{z}{y}) \psubstp{Q}{P} ) \\
  (\lift{x}{R}) \psubstp{Q}{P}  
  :=
  \lift{(x)\substp{Q}{P}}{ R \psubstp{Q}{P} } \\
%   (\dropn{x})  \psubstp{Q}{P}       
%   := 
%   \left\{ 
%     \begin{array}{ccc} 
%       \dropn{\quotep{Q}} & & x \nameeq \quotep{P} \\
%       \dropn{x} & & otherwise \\
%     \end{array}
%   \right. 
  (\dropn{x})  \psubstp{Q}{P}       
  := 
  \left\{ 
    \begin{array}{ccc} 
      Q & & x \nameeq \quotep{P} \\
      \dropn{x} & & otherwise \\
    \end{array}
  \right.
\end{mathpar}
 

where

\begin{eqnarray}
  (x)\id{\{} \lpquote Q \rpquote / \lpquote P \rpquote \id{\}}            = 
  \left\{ 
    \begin{array}{ccc}
      \lpquote Q \rpquote & & x \nameeq \lpquote P \rpquote \\
      x & & otherwise \\
    \end{array}
  \right. \nonumber
\end{eqnarray}

and $z$ is chosen distinct from $\quotep{P}$, $\quotep{Q}$, the free
names in $Q$, and all the names in $R$. Our $\alpha$-equivalence will
be built in the standard way from this substitution.

\begin{remark}\label{rem:no_self_referential_names}
  One consequence of these definitions is that $\forall P. \quotep{P}
  \not\in \freenames{P}$.
\end{remark}

\subsection{ Dynamic quote: an example }

Anticipating something of what's to come, consider applying the
substitution, $\widehat{\id{\{}u / z \id{\}}}$, to the following pair
of processes, $\lift{w}{y!(z)}$ and $w[ \lpquote y!(z) \rpquote ]$.

\begin{eqnarray}
	\lift{w}{y!(z)}\widehat{\id{\{}u / z \id{\}}}
		& = &
		\lift{w}{y!(u)} \nonumber\\
	w[ \lpquote y!(z) \rpquote ] \widehat{ \id{\{}u / z \id{\}} }
		& = &
		w[ \lpquote y!(z) \rpquote ] \nonumber
\end{eqnarray}

Because the body of the process between quotes is impervious to
substitution, we get radically different answers. In fact, by
examining the first process in an input context,
e.g. $x?(z).\lift{w}{y!(z)}$, we see that the process under the lift
operator may be shaped by prefixed inputs binding a name inside it. In
this sense, the lift operator will be seen as a way to dynamically
construct processes before reifying them as names.

Finally equipped with these standard features we can present the
dynamics of the calculus.

\subsubsection{Operational semantics} 

Finally, we introduce the computational dynamics. What marks these
algebras as distinct from other more traditionally studied algebraic
structures, e.g. vector spaces or polynomial rings, is the manner in
which dynamics is captured. In traditional structures, dynamics is typically
expressed through morphisms between such structures, as in linear maps
between vector spaces or morphisms between rings. In algebras
associated with the semantics of computation, the dynamics is
expressed as part of the algebraic structure itself, through a
reduction reduction relation typically denoted by $\red$. Below, we
give a recursive presentation of this relation for the calculus used
in the encoding.

$\red \subseteq \pi \times \pi$
$\red : \pi \to \mathcal{P}(\pi)$

\begin{mathpar}
  \inferrule* [lab=Comm] { \textsf{match}( x_{src}, x_{trgt} ) } { x_{trgt}?(y)P \; | \; x_{src}!\langle {Q} \rangle \red P\{\quotep{Q}/y}\} }
  \and \\
  \inferrule* [lab=Par] {{P} \red {P}'} {{{P} | {Q}} \red {{P}' | {Q}}}
  \and
  \inferrule* [lab=Equiv]{{{P} \scong {P}'} \andalso {{P}' \red {Q}'} \andalso {{Q}' \scong {Q}}}{{P} \red {Q}}
\end{mathpar}

\begin{eqnarray*}
  match_{\equiv} (\quotep{P},\quotep{Q}) & := & P \equiv Q \\
  match_{\dagger}(\quotep{P},\quotep{Q}) & := & \forall R. P|Q \red^{*} R => R \red^{*} 0 \\
  match_{K}(\quotep{P},\quotep{Q}) & := & K \mbox{ for some context } K
\end{eqnarray*}

$u?(x)P | u!\langle Q \rangle \red P\{\quotep{Q}/x\}$

%We write $\wred$ for $\red^*$, and $P\red$ if $\exists Q $ such that $ P \red Q$.
We write $P\red$ if $\exists Q $ such that $ P \red Q$ and $P\not\red$, otherwise.

\section{Replication}

As mentioned before, it is known that replication (and hence
recursion) can be implemented in a higher-order process algebra
\cite{SangiorgiWalker}. As our first example of calculation with the
machinery thus far presented we give the construction explicitly in
the {\rhoc}.

\begin{eqnarray}
	D_{x} & := & \prefix{x}{y}{(\binpar{\outputp{x}{y}}{@{y}})} \nonumber\\
	\bangp_{x}{P} & := & \binpar{{x}!\langle{\binpar{D_{x}}{P}}\rangle}{D_{x}} \nonumber
\end{eqnarray}

\begin{eqnarray}
	\bangp_{x}{P} & & \nonumber\\
	=
	& {x}!\langle{(\prefix{x}{y}{(\outputp{x}{y} | @{y})) | P}}\rangle 
	      | \prefix{x}{y}{(\outputp{x}{y} | @{y})} & \nonumber\\
	\red
	& (\outputp{x}{y} | @{y})\substn{\quotep{(\prefix{x}{y}{(@{y} | \outputp{x}{y})) | P}}}{y} & \nonumber\\
	=
	& \outputp{x}{\quotep{(\prefix{x}{y}{(\outputp{x}{y} | @{y})) | P}}}
	  | {(\prefix{x}{y}{(\outputp{x}{y} | @{y})) | P}} & \nonumber\\
	\red
	& \ldots & \nonumber\\
	\red^*
	& P | P | \ldots & \nonumber
\end{eqnarray}

Of course, this encoding, as an implementation, runs away, unfolding
$\bangp{P}$ eagerly. A lazier and more implementable replication
operator, restricted to input-guarded processes, may be obtained as follows.

\begin{eqnarray}
\bangp{\prefix{u}{v}{P}} 
	:= 
	\binpar{\lift{x}{\prefix{u}{v}{(\binpar{D(x)}{P})}}}{D(x)} \nonumber
\end{eqnarray}

\begin{remark}
  Note that the lazier definition still does not deal with summation
  or mixed summation (i.e. sums over input and output). The reader is
  invited to construct definitions of replication that deal with these
  features. 

  Further, the definitions are parameterized in a name, $x$. Can you,
  gentle reader, make a definition that eliminates this parameter and
  guarantees no accidental interaction between the replication
  machinery and the process being replicated -- i.e. no accidental
  sharing of names used by the process to get its work done and the
  name(s) used by the replication to effect copying. This latter
  revision of the definition of replication is crucial to obtaining
  the expected identity $!!P \sim !P$.
\end{remark}

\begin{remark}\label{rem:paradoxical_combinator}
  The reader familiar with the lambda calculus will have noticed the
  similarity between $D$ and the paradoxical combinator.

  [Ed. note: the existence of this seems to suggest we have to be more
  restrictive on the set of processes and names we admit if we are to
  support no-cloning.]
\end{remark}

\subsubsection{Bisimulation}

The computational dynamics gives rise to another kind of equivalence,
the equivalence of computational behavior. As previously mentioned
this is typically captured \emph{via} some form of bisimulation.

% The notion we use in this paper is weak barbed bisimulation
% \cite{milner91polyadicpi}.

The notion we use in this paper is derived from weak barbed
bisimulation \cite{milner91polyadicpi}. 

\begin{definition}
An \emph{observation relation}, $\downarrow_{\mathcal N}$, over a set
of names, $\mathcal N$, is the smallest relation satisfying the rules
below.

\infrule[Out-barb]{y \in {\mathcal N}, \; x \nameeq y}
		  {\outputp{x}{v} \downarrow_{\mathcal N} x}
\infrule[Par-barb]{\mbox{$P\downarrow_{\mathcal N} x$ or $Q\downarrow_{\mathcal N} x$}}
		  {\binpar{P}{Q} \downarrow_{\mathcal N} x}

We write $P \Downarrow_{\mathcal N} x$ if there is $Q$ such that 
$P \wred Q$ and $Q \downarrow_{\mathcal N} x$.
\end{definition}

\begin{definition}
%\label{def.bbisim}
An  ${\mathcal N}$-\emph{barbed bisimulation} over a set of names, ${\mathcal N}$, is a symmetric binary relation 
${\mathcal S}_{\mathcal N}$ between agents such that $P\rel{S}_{\mathcal N}Q$ implies:
\begin{enumerate}
\item If $P \red P'$ then $Q \wred Q'$ and $P'\rel{S}_{\mathcal N} Q'$.
\item If $P\downarrow_{\mathcal N} x$, then $Q\Downarrow_{\mathcal N} x$.
\end{enumerate}
$P$ is ${\mathcal N}$-barbed bisimilar to $Q$, written
$P \wbbisim_{\mathcal N} Q$, if $P \rel{S}_{\mathcal N} Q$ for some ${\mathcal N}$-barbed bisimulation ${\mathcal S}_{\mathcal N}$.
\end{definition}

$\mathcal{R} \subseteq \pi \times \pi$

$P \mathcal{R} Q => \forall P'. P \red P' \Rightarrow \exists Q'. Q \red Q', P' \mathcal{R} Q'$

$P \vdash x \Rightarrow Q \vdash x$

\begin{mathpar}
  \inferrule*[lab=Out-barb]{x \nameeq y}{{y}!\langle{Q}\rangle \vdash x}
  \and
  \inferrule*[lab=Par-barb]{\mbox{$P\vdash x$ or $Q\vdash x$}}{\binpar{P}{Q} \vdash x}
\end{mathpar}

\subsubsection{Contexts}

One of the principle advantages of computational calculi like the
$\pi$-calculus is a well-defined notion of context,
contextual-equivalence and a correlation between
contextual-equivalence and notions of bisimulation. The notion of
context allows the decomposition of a process into (sub-)process and
its syntactic environment, its context. Thus, a context may be
thought of as a process with a ``hole'' (written $\Box$) in it. The
application of a context $M$ to a process $P$, written $M[P]$, is
tantamount to filling the hole in $M$ with $P$. In this paper we do
not need the full weight of this theory, but do make use of the notion
of context in the proof the main theorem. 

\begin{mathpar}
  \inferrule* [lab=summation] {} {{M_{M},M_{N}} \bc \Box \;|\; x.M_{A} \;|\; M_{M}+M_{N}}
  \and
  \inferrule* [lab=agent] {} {{M_{A}} \bc (\vec{x})M_{P} \;| \; \clift{P_0,\ldots,M_{P},\ldots,P_N}}
  \and \\
  \inferrule* [lab=process] {} {{M_{P}} \bc M_{N} \;| \;P|M_{P} }
\end{mathpar} 

\begin{mathpar}
  \inferrule* [lab=sychronization] {} {M_{N} \bc \Box \;|\; x?M_{F} \;|\; x!M_{C}}
  \and
  \inferrule* [lab=abstraction] {} {{M_{F}} \bc (x)M_{P} }
  \and
  \inferrule* [lab=concretion] {} {{M_{C}} \bc \langle M_{P} \rangle }
  \and \\
  \inferrule* [lab=process] {} {{M_{P}} \bc M_{N} \;| \;P|M_{P} }
\end{mathpar}

\begin{definition}[contextual application] Given a context $M$, and
  process $P$, we define the \emph{contextual application}, $M[P] :=
  M\{P/\Box\}$. That is, the contextual application of M to P is the
  substitution of $P$ for $\Box$ in $M$.
\end{definition}

$\meaningof{-} : L \to \mathcal{P}(\pi)$

\begin{mathpar}
  \inferrule* [lab=collection] {} {\meaningof{true} = \pi, \and \meaningof{~E} = \pi \setminus \meaningof{E}, \and \meaningof{E_{1} \& E_{2}} = \meaningof{E_{1}} \cap \meaningof{E_{2}}}
\end{mathpar}

\begin{mathpar}
  \inferrule* [lab=structure] {} {\meaningof{0} = \{ P \in \pi | P \equiv 0 \}, \and \\ \meaningof{E_1 | E_2} = \{ P \in \pi | P \equiv P_{1} | P_{2}, P_{1} \in \meaningof{E_{1}}, P_{2} \in \meaningof{E_2}\} }
\end{mathpar}

\begin{mathpar}
 \inferrule* [lab=behavior] {} {\meaningof{\langle a?b \rangle E} = \{ P \in \pi | P \equiv Q | u?(y)P', \\ \and \\\\ \and \\ \;\;\; u \in \meaningof{a}, \forall z.P'\{z/y\} \in \meaningof{E\{z/b\}}\}, \and \\ \meaningof{a!E} = \{ P \in \pi | P \equiv Q | x!\langle P' \rangle, x \in \meaningof{a} P' \in \meaningof{E}\} }
\end{mathpar}

\begin{mathpar}
 \inferrule* [lab=nominal] {} {\meaningof{\quotep{E}} = \{ \quotep{P} \in \quotep{\pi} | P \in \meaningof{E} \}, \and \meaningof{\quotep{P}} = \{ \quotep{Q} \in \quotep{\pi} | P \equiv Q \} \and \\ \meaningof{@\quotep{E}} = \{ P \in \pi | P \equiv @x, x \in \meaningof{E} \}}
\end{mathpar}

\begin{eqnarray*}
  \\
  \meaningof{-} : TS \to ST
\end{eqnarray*}

\begin{eqnarray*}
  \\
  L : TS \to ST
\end{eqnarray*}

\begin{eqnarray*}
  \\
  P \models E \iff P \in \meaningof{E}
\end{eqnarray*}

\begin{eqnarray*}
  P \approx_{L} Q \iff \forall E \in L. P \models E \iff Q \models E
\end{eqnarray*}

\begin{eqnarray*}
  P \approx_{K} Q
\end{eqnarray*}

\begin{eqnarray*}
  P \approx Q
\end{eqnarray*}

$\approx_{K} = \approx = \approx_{L}$

\subsubsection{Contextual duality}

Note that contexts extend the quotation operation to a family of
operations from processes to names. Given a context, $M$, we can
define a \emph{nominal context}, $\quotep{M}$ by $\quotep{M}[P] :=
\quotep{M[P]}$. To foreshadow what is to come we observe that these
operations enjoy a duality with processes very much like the duality
between vectors and maps from vectors to scalars.

Further, because the calculus is essentially higher-order, we have a
correspondence between contexts and processes. More specifically,
given a name $x$ and a context $M$ we can construct $M^{*}_{x}$ such
that 

\begin{mathpar}
  M^{*}_{x} | \lift{x}{P} \red M[P]
\end{mathpar}

namely,

\begin{mathpar}
  M^{*}_{x} := x?(u).M[\dropn{u}]
\end{mathpar}

The dependence of $M^{*}_{x}$ on a name makes it an abstraction, 

\begin{mathpar}
  M^{*} := (x)x?(u).M[\dropn{u}]
\end{mathpar}

\subsection{Additional notation}

It will sometimes be convenient to denote the process a name
quotes. We already have the notation $x = \quotep{P}$, but it will be
convenient to introduce an alternate notation, $\procn{x}$, when we
want to emphasize the connection to the use of the name. Note that, by
virtue of name equivalence, $\quotep{\procn{x}} \nameeq x$; so, the
notation is consistent with previous definitions.

Further, because names have structure it is possible to effect
substitutions on the basis of that structure. This means we need to
upgrade our notation for substitutions, which we accomplish by
adapting comprehension notation. Thus,

\begin{mathpar}
  P\{ y / x : x \in S \}
\end{mathpar}

is interpreted to mean the process derived from P by replacing (in a
capture-avoiding manner) each occurrence of $x$ in $S$ by $y$. For example,

\begin{mathpar}
  P\{ \quotep{\procn{x}|\procn{x}} / x : x \in \freenames{P} \}
\end{mathpar}

will replace each (occurrence) of a free name $x$ in $P$ by
$\quotep{\procn{x}|\procn{x}}$.

Also, we will avail ourselves of the notation $x^{L}$ and $x^{R}$ to
denote injections of a name into disjoint copies of the name
space. There are numerous ways to accomplish this. One example can be
found in \cite{MeredithR05}. This notation overloads to vectors of
names: $\vec{x}^{\pi} := (x_{i}^{\pi} \; : \; 0 \leq i < |\vec{x}| )$ where $\pi \in \{L,R\}$.

We also use $P^{\Box} := P|\Box$.

In \cite{MeredithR05} an interpretation of the new operator is
given. It turns out that there are several possible interpretations
all enjoying the requisite algebraic properties of the operator (see
\cite{milner91polyadicpi}). We will therefore make liberal use of
$(\nu\; \vec{x})P$.

% subsection the_syntax_and_semantics_of_the_notation_system (end)   

\input{qm2pi.qmops} 

\input{qm2pi.sterngerlach} 

\input{qm2pi.metric} 

% section concurrent_process_calculi (end)

%\input{qm2pi.proofsketch}

% section proof sketch (end)

%\input{qm2pi.slviaknots} 

% section spatial logic via knots (end)

\input{qm2pi.conclusion}

% section conclusion (end)

%\input{qm2pi.dtcodes} 

% section wiring algorithm (end)

\input{qm2pi.ack} 

% section acknowledgments (end)

\newpage


\bibliographystyle{plain}   
\bibliography{../../biblios/main.bib}

\input{qm2pi.rhodetails}

\end{document}

 

% section acknowledgments (end)

\newpage


\bibliographystyle{plain}   
\bibliography{../../biblios/main.bib}

\documentclass[12pt]{llncs}
%\documentclass{jktr}

\usepackage[pdftex]{hyperref}                   
\usepackage {listings}
\usepackage {mathpartir}
\usepackage{bcprules}
%\usepackage{listings}
                       
\usepackage{graphicx} 
%\usepackage[margins=2.5cm,nohead,nofoot]{geometry}
%\usepackage{geometry}
\usepackage{amsfonts}
\usepackage{amstext}
\usepackage{latexsym}
\usepackage{amssymb}
\usepackage{color}


%\include{myPreamble}
\include{qm2pi.local} 

%\ifpdf
%\usepackage[pdftex]{graphicx}
%\else
%\usepackage{graphicx}
%\fi

 % \ifpdf
%  \usepackage{pdfsync}
%  \if


%\title{Brief Article}
%\author{David F. Snyder}
%\author{L.G. Meredith}

%\address{Dept. of Math., Texas State University--San Marcos, San Marcos, TX 78666}
       
\pagestyle{empty}


\begin{document}

\lstset{language=[Objective]Caml,frame=shadowbox}

\input{qm2pi.front}

% section front matter (end)

\input{qm2pi.intro} 
 
% section introduction (end)

% \input{qm2pi.knotations} 

% section notation (end)

\input{qm2pi.process.calculi} 

% section concurrent_process_calculi_and_spatial_logics_ (end)
    
%\input{qm2pi.knots2pi} 

%\input{qm2pi.trefoil} 

%\input{qm2pi.mainthm} 

% subsection basic_interpretation (end)

%\input{qm2pi.rho.presentation} 
\subsection{The syntax and semantics of the notation system}\label{sub:the_syntax_and_semantics_of_the_notation_system} % (fold)

We now summarize a technical presentation of the calculus that
embodies our theory of dynamics. The typical presentation of such a
calculus follows the style of giving generators and relations on
them. The grammar, below, describing term constructors, freely
generates the set of processes, $\Proc$. This set is then quotiented
by a relation known as structural congruence and it is over this set
that the notion of dynamics is expressed. This presentation is
essentially that of \cite{MeredithR05} with the addition of
polyadicity and summation. For readability we have relegated some of
the technical subtleties to an appendix.

\subsubsection{Process grammar}\label{subsub:process_grammar}

\begin{mathpar}
  \inferrule* [lab=synchronization] {} {{M} \bc \pzero \;|\; x?F \;|\; x!C }
  \and
  \inferrule* [lab=abstraction] {} {{F} \bc (x)P}
  \and
  \inferrule* [lab=concretion] {} {{C} \bc \langle Q \rangle}
  \and
  \inferrule* [lab=process] {} {{P,Q} \bc M \;| \;P|Q \;|\; @{x}}
  \and
  \inferrule* [lab=name] {} {{x} \bc \quotep{P}}
\end{mathpar} 

Note that $\vec{x}$ (resp. $\vec{P}$) denotes a vector of names
(resp. processes) of length $|\vec{x}|$ (resp. $|\vec{P}|$). We adopt
the following useful abbreviations.

\begin{mathpar}
   x?(\vec{y}).P := x.(\vec{y})P \and  x\clift{\vec{P}} := x.\clift{\vec{P}}
   \and x!(y) := \lift{x}{\dropn{y}}
   \and \Pi_{i=0}^{n-1}P_i := P_0 | \ldots | P_{n-1}
\end{mathpar}

\subsubsection{Structural congruence}

\paragraph{Free and bound names and alpha-equivalence.} At the
core of structural equivalence is alpha-equivalence which identifies
process that are the same up to a change of variable. Formally, we
recognize the distinction between free and bound names. The free names
of a process, $\freenames{P}$, may be calculated recursively as
follows:

\begin{mathpar}
\freenames{\pzero} := \emptyset
  \and \\
  \freenames{x?(y).P} := \{ x \} \cup (\freenames{P} \setminus \{ y \})
  \and 
  \freenames{x!\langle P \rangle} := \{ x \} \cup \{ P \} 
  \and \\
  \freenames{P|Q} := \freenames{P} \cup \freenames{Q}
  \and \\
  \freenames{@{x}} := \{ x \}
\end{mathpar}

$\pi$
$\quotep{\pi}$

$\freenames{-} : \pi \to \mathcal{P}(\quotep{\pi})$

\begin{eqnarray*}
  \freenames{\pzero} & := & \emptyset \\
  \freenames{x?(y).P} & := & \{ x \} \cup (\freenames{P} \setminus \{ y \}) \\
  \freenames{x!\langle P \rangle} & := & \{ x \} \cup \{ P \} \\
  \freenames{P|Q} & := & \freenames{P} \cup \freenames{Q} \\
  \freenames{\dropn{x}} & := & \{ x \}
\end{eqnarray*}

The bound names of a process, $\boundnames{P}$, are those names occurring in $P$
that are not free. For example, in $x?(y).0$, the name $x$ is free, while $y$ is bound.

\begin{mathpar}
  \inferrule* [lab=monoidal-laws] {} { P|Q \equiv Q|P \and P|0 \equiv P \and P|(Q|R) \equiv (P|Q)|R }
\end{mathpar}

\begin{mathpar}
  \inferrule* [lab=alpha-equivalence] {} { (x)P \equiv (y)P\{y/x\} \and y \not\in \freenames{P} }
\end{mathpar}

\begin{definition}
Then two processes, $P,Q$, are alpha-equivalent if $P = Q\{\vec{y}/\vec{x}\}$ for
some $\vec{x} \in \boundnames{Q},\vec{y} \in \boundnames{P}$, where $Q\{\vec{y}/\vec{x}\}$
denotes the capture-avoiding substitution of $\vec{y}$ for $\vec{x}$ in $Q$.
\end{definition}

\begin{definition}
  The {\em structural congruence} \cite{SangiorgiWalker} , $\equiv$,
  between processes is the least congruence containing
  alpha-equivalence, satisfying the abelian monoid laws
  (associativity, commutativity and $\pzero$ as identity) for parallel
  composition $|$ and for summation $+$.
\end{definition}

\subsection{Name equivalence}

We take name equivalence, written $\nameeq$, to be the smallest
equivalence relation generated by the following rules.

\begin{mathpar}
\inferrule*[lab=Quote-drop]
{ }
{ \quotep{@{x}} \nameeq x }

\inferrule*[lab=Struct-equiv]
{ P \scong Q }
{ \quotep{P} \nameeq \quotep{Q} }
\end{mathpar}

The astute reader will have noticed that the mutual recursion of names
and processes imposes a mutual recursion on alpha-equivalence and
structural equivalence via name-equivalence. Fortunately, all of this
works out pleasantly and we may calculate in the natural way, free of
concern. The reader interested in the details is referred to the
appendix \ref{appendix:rho_details}.

\subsection{Substitution}

We use $\Proc$ for the set of processes, $\QProc$ for the set of
names, and $\id{\{}\vec{y} / \vec{x} \id{\}}$ to denote partial maps,
$s : \QProc \rightarrow \QProc$. A map, $s$ lifts, uniquely, to a map
on process terms, $\widehat{s} : \Proc \rightarrow \Proc$ by the
following equations.

\begin{mathpar}
  (0) \psubstp{Q}{P} := 0 \\
  (R \juxtap S) \psubstp{Q}{P}
  :=    
  (R)\psubstp{Q}{P} \juxtap (S) \psubstp{Q}{P} \\
  (x?(y).R) \psubstp{Q}{P}    
  :=    
  (x)\substp{Q}{P} (z)\concat( (R \psubstn{z}{y}) \psubstp{Q}{P} ) \\
  (\lift{x}{R}) \psubstp{Q}{P}  
  :=
  \lift{(x)\substp{Q}{P}}{ R \psubstp{Q}{P} } \\
%   (\dropn{x})  \psubstp{Q}{P}       
%   := 
%   \left\{ 
%     \begin{array}{ccc} 
%       \dropn{\quotep{Q}} & & x \nameeq \quotep{P} \\
%       \dropn{x} & & otherwise \\
%     \end{array}
%   \right. 
  (\dropn{x})  \psubstp{Q}{P}       
  := 
  \left\{ 
    \begin{array}{ccc} 
      Q & & x \nameeq \quotep{P} \\
      \dropn{x} & & otherwise \\
    \end{array}
  \right.
\end{mathpar}
 

where

\begin{eqnarray}
  (x)\id{\{} \lpquote Q \rpquote / \lpquote P \rpquote \id{\}}            = 
  \left\{ 
    \begin{array}{ccc}
      \lpquote Q \rpquote & & x \nameeq \lpquote P \rpquote \\
      x & & otherwise \\
    \end{array}
  \right. \nonumber
\end{eqnarray}

and $z$ is chosen distinct from $\quotep{P}$, $\quotep{Q}$, the free
names in $Q$, and all the names in $R$. Our $\alpha$-equivalence will
be built in the standard way from this substitution.

\begin{remark}\label{rem:no_self_referential_names}
  One consequence of these definitions is that $\forall P. \quotep{P}
  \not\in \freenames{P}$.
\end{remark}

\subsection{ Dynamic quote: an example }

Anticipating something of what's to come, consider applying the
substitution, $\widehat{\id{\{}u / z \id{\}}}$, to the following pair
of processes, $\lift{w}{y!(z)}$ and $w[ \lpquote y!(z) \rpquote ]$.

\begin{eqnarray}
	\lift{w}{y!(z)}\widehat{\id{\{}u / z \id{\}}}
		& = &
		\lift{w}{y!(u)} \nonumber\\
	w[ \lpquote y!(z) \rpquote ] \widehat{ \id{\{}u / z \id{\}} }
		& = &
		w[ \lpquote y!(z) \rpquote ] \nonumber
\end{eqnarray}

Because the body of the process between quotes is impervious to
substitution, we get radically different answers. In fact, by
examining the first process in an input context,
e.g. $x?(z).\lift{w}{y!(z)}$, we see that the process under the lift
operator may be shaped by prefixed inputs binding a name inside it. In
this sense, the lift operator will be seen as a way to dynamically
construct processes before reifying them as names.

Finally equipped with these standard features we can present the
dynamics of the calculus.

\subsubsection{Operational semantics} 

Finally, we introduce the computational dynamics. What marks these
algebras as distinct from other more traditionally studied algebraic
structures, e.g. vector spaces or polynomial rings, is the manner in
which dynamics is captured. In traditional structures, dynamics is typically
expressed through morphisms between such structures, as in linear maps
between vector spaces or morphisms between rings. In algebras
associated with the semantics of computation, the dynamics is
expressed as part of the algebraic structure itself, through a
reduction reduction relation typically denoted by $\red$. Below, we
give a recursive presentation of this relation for the calculus used
in the encoding.

$\red \subseteq \pi \times \pi$
$\red : \pi \to \mathcal{P}(\pi)$

\begin{mathpar}
  \inferrule* [lab=Comm] { \textsf{match}( x_{src}, x_{trgt} ) } { x_{trgt}?(y)P \; | \; x_{src}!\langle {Q} \rangle \red P\{\quotep{Q}/y}\} }
  \and \\
  \inferrule* [lab=Par] {{P} \red {P}'} {{{P} | {Q}} \red {{P}' | {Q}}}
  \and
  \inferrule* [lab=Equiv]{{{P} \scong {P}'} \andalso {{P}' \red {Q}'} \andalso {{Q}' \scong {Q}}}{{P} \red {Q}}
\end{mathpar}

\begin{eqnarray*}
  match_{\equiv} (\quotep{P},\quotep{Q}) & := & P \equiv Q \\
  match_{\dagger}(\quotep{P},\quotep{Q}) & := & \forall R. P|Q \red^{*} R => R \red^{*} 0 \\
  match_{K}(\quotep{P},\quotep{Q}) & := & K \mbox{ for some context } K
\end{eqnarray*}

$u?(x)P | u!\langle Q \rangle \red P\{\quotep{Q}/x\}$

%We write $\wred$ for $\red^*$, and $P\red$ if $\exists Q $ such that $ P \red Q$.
We write $P\red$ if $\exists Q $ such that $ P \red Q$ and $P\not\red$, otherwise.

\section{Replication}

As mentioned before, it is known that replication (and hence
recursion) can be implemented in a higher-order process algebra
\cite{SangiorgiWalker}. As our first example of calculation with the
machinery thus far presented we give the construction explicitly in
the {\rhoc}.

\begin{eqnarray}
	D_{x} & := & \prefix{x}{y}{(\binpar{\outputp{x}{y}}{@{y}})} \nonumber\\
	\bangp_{x}{P} & := & \binpar{{x}!\langle{\binpar{D_{x}}{P}}\rangle}{D_{x}} \nonumber
\end{eqnarray}

\begin{eqnarray}
	\bangp_{x}{P} & & \nonumber\\
	=
	& {x}!\langle{(\prefix{x}{y}{(\outputp{x}{y} | @{y})) | P}}\rangle 
	      | \prefix{x}{y}{(\outputp{x}{y} | @{y})} & \nonumber\\
	\red
	& (\outputp{x}{y} | @{y})\substn{\quotep{(\prefix{x}{y}{(@{y} | \outputp{x}{y})) | P}}}{y} & \nonumber\\
	=
	& \outputp{x}{\quotep{(\prefix{x}{y}{(\outputp{x}{y} | @{y})) | P}}}
	  | {(\prefix{x}{y}{(\outputp{x}{y} | @{y})) | P}} & \nonumber\\
	\red
	& \ldots & \nonumber\\
	\red^*
	& P | P | \ldots & \nonumber
\end{eqnarray}

Of course, this encoding, as an implementation, runs away, unfolding
$\bangp{P}$ eagerly. A lazier and more implementable replication
operator, restricted to input-guarded processes, may be obtained as follows.

\begin{eqnarray}
\bangp{\prefix{u}{v}{P}} 
	:= 
	\binpar{\lift{x}{\prefix{u}{v}{(\binpar{D(x)}{P})}}}{D(x)} \nonumber
\end{eqnarray}

\begin{remark}
  Note that the lazier definition still does not deal with summation
  or mixed summation (i.e. sums over input and output). The reader is
  invited to construct definitions of replication that deal with these
  features. 

  Further, the definitions are parameterized in a name, $x$. Can you,
  gentle reader, make a definition that eliminates this parameter and
  guarantees no accidental interaction between the replication
  machinery and the process being replicated -- i.e. no accidental
  sharing of names used by the process to get its work done and the
  name(s) used by the replication to effect copying. This latter
  revision of the definition of replication is crucial to obtaining
  the expected identity $!!P \sim !P$.
\end{remark}

\begin{remark}\label{rem:paradoxical_combinator}
  The reader familiar with the lambda calculus will have noticed the
  similarity between $D$ and the paradoxical combinator.

  [Ed. note: the existence of this seems to suggest we have to be more
  restrictive on the set of processes and names we admit if we are to
  support no-cloning.]
\end{remark}

\subsubsection{Bisimulation}

The computational dynamics gives rise to another kind of equivalence,
the equivalence of computational behavior. As previously mentioned
this is typically captured \emph{via} some form of bisimulation.

% The notion we use in this paper is weak barbed bisimulation
% \cite{milner91polyadicpi}.

The notion we use in this paper is derived from weak barbed
bisimulation \cite{milner91polyadicpi}. 

\begin{definition}
An \emph{observation relation}, $\downarrow_{\mathcal N}$, over a set
of names, $\mathcal N$, is the smallest relation satisfying the rules
below.

\infrule[Out-barb]{y \in {\mathcal N}, \; x \nameeq y}
		  {\outputp{x}{v} \downarrow_{\mathcal N} x}
\infrule[Par-barb]{\mbox{$P\downarrow_{\mathcal N} x$ or $Q\downarrow_{\mathcal N} x$}}
		  {\binpar{P}{Q} \downarrow_{\mathcal N} x}

We write $P \Downarrow_{\mathcal N} x$ if there is $Q$ such that 
$P \wred Q$ and $Q \downarrow_{\mathcal N} x$.
\end{definition}

\begin{definition}
%\label{def.bbisim}
An  ${\mathcal N}$-\emph{barbed bisimulation} over a set of names, ${\mathcal N}$, is a symmetric binary relation 
${\mathcal S}_{\mathcal N}$ between agents such that $P\rel{S}_{\mathcal N}Q$ implies:
\begin{enumerate}
\item If $P \red P'$ then $Q \wred Q'$ and $P'\rel{S}_{\mathcal N} Q'$.
\item If $P\downarrow_{\mathcal N} x$, then $Q\Downarrow_{\mathcal N} x$.
\end{enumerate}
$P$ is ${\mathcal N}$-barbed bisimilar to $Q$, written
$P \wbbisim_{\mathcal N} Q$, if $P \rel{S}_{\mathcal N} Q$ for some ${\mathcal N}$-barbed bisimulation ${\mathcal S}_{\mathcal N}$.
\end{definition}

$\mathcal{R} \subseteq \pi \times \pi$

$P \mathcal{R} Q => \forall P'. P \red P' \Rightarrow \exists Q'. Q \red Q', P' \mathcal{R} Q'$

$P \vdash x \Rightarrow Q \vdash x$

\begin{mathpar}
  \inferrule*[lab=Out-barb]{x \nameeq y}{{y}!\langle{Q}\rangle \vdash x}
  \and
  \inferrule*[lab=Par-barb]{\mbox{$P\vdash x$ or $Q\vdash x$}}{\binpar{P}{Q} \vdash x}
\end{mathpar}

\subsubsection{Contexts}

One of the principle advantages of computational calculi like the
$\pi$-calculus is a well-defined notion of context,
contextual-equivalence and a correlation between
contextual-equivalence and notions of bisimulation. The notion of
context allows the decomposition of a process into (sub-)process and
its syntactic environment, its context. Thus, a context may be
thought of as a process with a ``hole'' (written $\Box$) in it. The
application of a context $M$ to a process $P$, written $M[P]$, is
tantamount to filling the hole in $M$ with $P$. In this paper we do
not need the full weight of this theory, but do make use of the notion
of context in the proof the main theorem. 

\begin{mathpar}
  \inferrule* [lab=summation] {} {{M_{M},M_{N}} \bc \Box \;|\; x.M_{A} \;|\; M_{M}+M_{N}}
  \and
  \inferrule* [lab=agent] {} {{M_{A}} \bc (\vec{x})M_{P} \;| \; \clift{P_0,\ldots,M_{P},\ldots,P_N}}
  \and \\
  \inferrule* [lab=process] {} {{M_{P}} \bc M_{N} \;| \;P|M_{P} }
\end{mathpar} 

\begin{mathpar}
  \inferrule* [lab=sychronization] {} {M_{N} \bc \Box \;|\; x?M_{F} \;|\; x!M_{C}}
  \and
  \inferrule* [lab=abstraction] {} {{M_{F}} \bc (x)M_{P} }
  \and
  \inferrule* [lab=concretion] {} {{M_{C}} \bc \langle M_{P} \rangle }
  \and \\
  \inferrule* [lab=process] {} {{M_{P}} \bc M_{N} \;| \;P|M_{P} }
\end{mathpar}

\begin{definition}[contextual application] Given a context $M$, and
  process $P$, we define the \emph{contextual application}, $M[P] :=
  M\{P/\Box\}$. That is, the contextual application of M to P is the
  substitution of $P$ for $\Box$ in $M$.
\end{definition}

$\meaningof{-} : L \to \mathcal{P}(\pi)$

\begin{mathpar}
  \inferrule* [lab=collection] {} {\meaningof{true} = \pi, \and \meaningof{~E} = \pi \setminus \meaningof{E}, \and \meaningof{E_{1} \& E_{2}} = \meaningof{E_{1}} \cap \meaningof{E_{2}}}
\end{mathpar}

\begin{mathpar}
  \inferrule* [lab=structure] {} {\meaningof{0} = \{ P \in \pi | P \equiv 0 \}, \and \\ \meaningof{E_1 | E_2} = \{ P \in \pi | P \equiv P_{1} | P_{2}, P_{1} \in \meaningof{E_{1}}, P_{2} \in \meaningof{E_2}\} }
\end{mathpar}

\begin{mathpar}
 \inferrule* [lab=behavior] {} {\meaningof{\langle a?b \rangle E} = \{ P \in \pi | P \equiv Q | u?(y)P', \\ \and \\\\ \and \\ \;\;\; u \in \meaningof{a}, \forall z.P'\{z/y\} \in \meaningof{E\{z/b\}}\}, \and \\ \meaningof{a!E} = \{ P \in \pi | P \equiv Q | x!\langle P' \rangle, x \in \meaningof{a} P' \in \meaningof{E}\} }
\end{mathpar}

\begin{mathpar}
 \inferrule* [lab=nominal] {} {\meaningof{\quotep{E}} = \{ \quotep{P} \in \quotep{\pi} | P \in \meaningof{E} \}, \and \meaningof{\quotep{P}} = \{ \quotep{Q} \in \quotep{\pi} | P \equiv Q \} \and \\ \meaningof{@\quotep{E}} = \{ P \in \pi | P \equiv @x, x \in \meaningof{E} \}}
\end{mathpar}

\begin{eqnarray*}
  \\
  \meaningof{-} : TS \to ST
\end{eqnarray*}

\begin{eqnarray*}
  \\
  L : TS \to ST
\end{eqnarray*}

\begin{eqnarray*}
  \\
  P \models E \iff P \in \meaningof{E}
\end{eqnarray*}

\begin{eqnarray*}
  P \approx_{L} Q \iff \forall E \in L. P \models E \iff Q \models E
\end{eqnarray*}

\begin{eqnarray*}
  P \approx_{K} Q
\end{eqnarray*}

\begin{eqnarray*}
  P \approx Q
\end{eqnarray*}

$\approx_{K} = \approx = \approx_{L}$

\subsubsection{Contextual duality}

Note that contexts extend the quotation operation to a family of
operations from processes to names. Given a context, $M$, we can
define a \emph{nominal context}, $\quotep{M}$ by $\quotep{M}[P] :=
\quotep{M[P]}$. To foreshadow what is to come we observe that these
operations enjoy a duality with processes very much like the duality
between vectors and maps from vectors to scalars.

Further, because the calculus is essentially higher-order, we have a
correspondence between contexts and processes. More specifically,
given a name $x$ and a context $M$ we can construct $M^{*}_{x}$ such
that 

\begin{mathpar}
  M^{*}_{x} | \lift{x}{P} \red M[P]
\end{mathpar}

namely,

\begin{mathpar}
  M^{*}_{x} := x?(u).M[\dropn{u}]
\end{mathpar}

The dependence of $M^{*}_{x}$ on a name makes it an abstraction, 

\begin{mathpar}
  M^{*} := (x)x?(u).M[\dropn{u}]
\end{mathpar}

\subsection{Additional notation}

It will sometimes be convenient to denote the process a name
quotes. We already have the notation $x = \quotep{P}$, but it will be
convenient to introduce an alternate notation, $\procn{x}$, when we
want to emphasize the connection to the use of the name. Note that, by
virtue of name equivalence, $\quotep{\procn{x}} \nameeq x$; so, the
notation is consistent with previous definitions.

Further, because names have structure it is possible to effect
substitutions on the basis of that structure. This means we need to
upgrade our notation for substitutions, which we accomplish by
adapting comprehension notation. Thus,

\begin{mathpar}
  P\{ y / x : x \in S \}
\end{mathpar}

is interpreted to mean the process derived from P by replacing (in a
capture-avoiding manner) each occurrence of $x$ in $S$ by $y$. For example,

\begin{mathpar}
  P\{ \quotep{\procn{x}|\procn{x}} / x : x \in \freenames{P} \}
\end{mathpar}

will replace each (occurrence) of a free name $x$ in $P$ by
$\quotep{\procn{x}|\procn{x}}$.

Also, we will avail ourselves of the notation $x^{L}$ and $x^{R}$ to
denote injections of a name into disjoint copies of the name
space. There are numerous ways to accomplish this. One example can be
found in \cite{MeredithR05}. This notation overloads to vectors of
names: $\vec{x}^{\pi} := (x_{i}^{\pi} \; : \; 0 \leq i < |\vec{x}| )$ where $\pi \in \{L,R\}$.

We also use $P^{\Box} := P|\Box$.

In \cite{MeredithR05} an interpretation of the new operator is
given. It turns out that there are several possible interpretations
all enjoying the requisite algebraic properties of the operator (see
\cite{milner91polyadicpi}). We will therefore make liberal use of
$(\nu\; \vec{x})P$.

% subsection the_syntax_and_semantics_of_the_notation_system (end)   

\input{qm2pi.qmops} 

\input{qm2pi.sterngerlach} 

\input{qm2pi.metric} 

% section concurrent_process_calculi (end)

%\input{qm2pi.proofsketch}

% section proof sketch (end)

%\input{qm2pi.slviaknots} 

% section spatial logic via knots (end)

\input{qm2pi.conclusion}

% section conclusion (end)

%\input{qm2pi.dtcodes} 

% section wiring algorithm (end)

\input{qm2pi.ack} 

% section acknowledgments (end)

\newpage


\bibliographystyle{plain}   
\bibliography{../../biblios/main.bib}

\input{qm2pi.rhodetails}

\end{document}



\end{document}

 

\documentclass[12pt]{llncs}
%\documentclass{jktr}

\usepackage[pdftex]{hyperref}                   
\usepackage {listings}
\usepackage {mathpartir}
\usepackage{bcprules}
%\usepackage{listings}
                       
\usepackage{graphicx} 
%\usepackage[margins=2.5cm,nohead,nofoot]{geometry}
%\usepackage{geometry}
\usepackage{amsfonts}
\usepackage{amstext}
\usepackage{latexsym}
\usepackage{amssymb}
\usepackage{color}


%\include{myPreamble}
\documentclass[12pt]{llncs}
%\documentclass{jktr}

\usepackage[pdftex]{hyperref}                   
\usepackage {listings}
\usepackage {mathpartir}
\usepackage{bcprules}
%\usepackage{listings}
                       
\usepackage{graphicx} 
%\usepackage[margins=2.5cm,nohead,nofoot]{geometry}
%\usepackage{geometry}
\usepackage{amsfonts}
\usepackage{amstext}
\usepackage{latexsym}
\usepackage{amssymb}
\usepackage{color}


%\include{myPreamble}
\include{qm2pi.local} 

%\ifpdf
%\usepackage[pdftex]{graphicx}
%\else
%\usepackage{graphicx}
%\fi

 % \ifpdf
%  \usepackage{pdfsync}
%  \if


%\title{Brief Article}
%\author{David F. Snyder}
%\author{L.G. Meredith}

%\address{Dept. of Math., Texas State University--San Marcos, San Marcos, TX 78666}
       
\pagestyle{empty}


\begin{document}

\lstset{language=[Objective]Caml,frame=shadowbox}

\input{qm2pi.front}

% section front matter (end)

\input{qm2pi.intro} 
 
% section introduction (end)

% \input{qm2pi.knotations} 

% section notation (end)

\input{qm2pi.process.calculi} 

% section concurrent_process_calculi_and_spatial_logics_ (end)
    
%\input{qm2pi.knots2pi} 

%\input{qm2pi.trefoil} 

%\input{qm2pi.mainthm} 

% subsection basic_interpretation (end)

%\input{qm2pi.rho.presentation} 
\subsection{The syntax and semantics of the notation system}\label{sub:the_syntax_and_semantics_of_the_notation_system} % (fold)

We now summarize a technical presentation of the calculus that
embodies our theory of dynamics. The typical presentation of such a
calculus follows the style of giving generators and relations on
them. The grammar, below, describing term constructors, freely
generates the set of processes, $\Proc$. This set is then quotiented
by a relation known as structural congruence and it is over this set
that the notion of dynamics is expressed. This presentation is
essentially that of \cite{MeredithR05} with the addition of
polyadicity and summation. For readability we have relegated some of
the technical subtleties to an appendix.

\subsubsection{Process grammar}\label{subsub:process_grammar}

\begin{mathpar}
  \inferrule* [lab=synchronization] {} {{M} \bc \pzero \;|\; x?F \;|\; x!C }
  \and
  \inferrule* [lab=abstraction] {} {{F} \bc (x)P}
  \and
  \inferrule* [lab=concretion] {} {{C} \bc \langle Q \rangle}
  \and
  \inferrule* [lab=process] {} {{P,Q} \bc M \;| \;P|Q \;|\; @{x}}
  \and
  \inferrule* [lab=name] {} {{x} \bc \quotep{P}}
\end{mathpar} 

Note that $\vec{x}$ (resp. $\vec{P}$) denotes a vector of names
(resp. processes) of length $|\vec{x}|$ (resp. $|\vec{P}|$). We adopt
the following useful abbreviations.

\begin{mathpar}
   x?(\vec{y}).P := x.(\vec{y})P \and  x\clift{\vec{P}} := x.\clift{\vec{P}}
   \and x!(y) := \lift{x}{\dropn{y}}
   \and \Pi_{i=0}^{n-1}P_i := P_0 | \ldots | P_{n-1}
\end{mathpar}

\subsubsection{Structural congruence}

\paragraph{Free and bound names and alpha-equivalence.} At the
core of structural equivalence is alpha-equivalence which identifies
process that are the same up to a change of variable. Formally, we
recognize the distinction between free and bound names. The free names
of a process, $\freenames{P}$, may be calculated recursively as
follows:

\begin{mathpar}
\freenames{\pzero} := \emptyset
  \and \\
  \freenames{x?(y).P} := \{ x \} \cup (\freenames{P} \setminus \{ y \})
  \and 
  \freenames{x!\langle P \rangle} := \{ x \} \cup \{ P \} 
  \and \\
  \freenames{P|Q} := \freenames{P} \cup \freenames{Q}
  \and \\
  \freenames{@{x}} := \{ x \}
\end{mathpar}

$\pi$
$\quotep{\pi}$

$\freenames{-} : \pi \to \mathcal{P}(\quotep{\pi})$

\begin{eqnarray*}
  \freenames{\pzero} & := & \emptyset \\
  \freenames{x?(y).P} & := & \{ x \} \cup (\freenames{P} \setminus \{ y \}) \\
  \freenames{x!\langle P \rangle} & := & \{ x \} \cup \{ P \} \\
  \freenames{P|Q} & := & \freenames{P} \cup \freenames{Q} \\
  \freenames{\dropn{x}} & := & \{ x \}
\end{eqnarray*}

The bound names of a process, $\boundnames{P}$, are those names occurring in $P$
that are not free. For example, in $x?(y).0$, the name $x$ is free, while $y$ is bound.

\begin{mathpar}
  \inferrule* [lab=monoidal-laws] {} { P|Q \equiv Q|P \and P|0 \equiv P \and P|(Q|R) \equiv (P|Q)|R }
\end{mathpar}

\begin{mathpar}
  \inferrule* [lab=alpha-equivalence] {} { (x)P \equiv (y)P\{y/x\} \and y \not\in \freenames{P} }
\end{mathpar}

\begin{definition}
Then two processes, $P,Q$, are alpha-equivalent if $P = Q\{\vec{y}/\vec{x}\}$ for
some $\vec{x} \in \boundnames{Q},\vec{y} \in \boundnames{P}$, where $Q\{\vec{y}/\vec{x}\}$
denotes the capture-avoiding substitution of $\vec{y}$ for $\vec{x}$ in $Q$.
\end{definition}

\begin{definition}
  The {\em structural congruence} \cite{SangiorgiWalker} , $\equiv$,
  between processes is the least congruence containing
  alpha-equivalence, satisfying the abelian monoid laws
  (associativity, commutativity and $\pzero$ as identity) for parallel
  composition $|$ and for summation $+$.
\end{definition}

\subsection{Name equivalence}

We take name equivalence, written $\nameeq$, to be the smallest
equivalence relation generated by the following rules.

\begin{mathpar}
\inferrule*[lab=Quote-drop]
{ }
{ \quotep{@{x}} \nameeq x }

\inferrule*[lab=Struct-equiv]
{ P \scong Q }
{ \quotep{P} \nameeq \quotep{Q} }
\end{mathpar}

The astute reader will have noticed that the mutual recursion of names
and processes imposes a mutual recursion on alpha-equivalence and
structural equivalence via name-equivalence. Fortunately, all of this
works out pleasantly and we may calculate in the natural way, free of
concern. The reader interested in the details is referred to the
appendix \ref{appendix:rho_details}.

\subsection{Substitution}

We use $\Proc$ for the set of processes, $\QProc$ for the set of
names, and $\id{\{}\vec{y} / \vec{x} \id{\}}$ to denote partial maps,
$s : \QProc \rightarrow \QProc$. A map, $s$ lifts, uniquely, to a map
on process terms, $\widehat{s} : \Proc \rightarrow \Proc$ by the
following equations.

\begin{mathpar}
  (0) \psubstp{Q}{P} := 0 \\
  (R \juxtap S) \psubstp{Q}{P}
  :=    
  (R)\psubstp{Q}{P} \juxtap (S) \psubstp{Q}{P} \\
  (x?(y).R) \psubstp{Q}{P}    
  :=    
  (x)\substp{Q}{P} (z)\concat( (R \psubstn{z}{y}) \psubstp{Q}{P} ) \\
  (\lift{x}{R}) \psubstp{Q}{P}  
  :=
  \lift{(x)\substp{Q}{P}}{ R \psubstp{Q}{P} } \\
%   (\dropn{x})  \psubstp{Q}{P}       
%   := 
%   \left\{ 
%     \begin{array}{ccc} 
%       \dropn{\quotep{Q}} & & x \nameeq \quotep{P} \\
%       \dropn{x} & & otherwise \\
%     \end{array}
%   \right. 
  (\dropn{x})  \psubstp{Q}{P}       
  := 
  \left\{ 
    \begin{array}{ccc} 
      Q & & x \nameeq \quotep{P} \\
      \dropn{x} & & otherwise \\
    \end{array}
  \right.
\end{mathpar}
 

where

\begin{eqnarray}
  (x)\id{\{} \lpquote Q \rpquote / \lpquote P \rpquote \id{\}}            = 
  \left\{ 
    \begin{array}{ccc}
      \lpquote Q \rpquote & & x \nameeq \lpquote P \rpquote \\
      x & & otherwise \\
    \end{array}
  \right. \nonumber
\end{eqnarray}

and $z$ is chosen distinct from $\quotep{P}$, $\quotep{Q}$, the free
names in $Q$, and all the names in $R$. Our $\alpha$-equivalence will
be built in the standard way from this substitution.

\begin{remark}\label{rem:no_self_referential_names}
  One consequence of these definitions is that $\forall P. \quotep{P}
  \not\in \freenames{P}$.
\end{remark}

\subsection{ Dynamic quote: an example }

Anticipating something of what's to come, consider applying the
substitution, $\widehat{\id{\{}u / z \id{\}}}$, to the following pair
of processes, $\lift{w}{y!(z)}$ and $w[ \lpquote y!(z) \rpquote ]$.

\begin{eqnarray}
	\lift{w}{y!(z)}\widehat{\id{\{}u / z \id{\}}}
		& = &
		\lift{w}{y!(u)} \nonumber\\
	w[ \lpquote y!(z) \rpquote ] \widehat{ \id{\{}u / z \id{\}} }
		& = &
		w[ \lpquote y!(z) \rpquote ] \nonumber
\end{eqnarray}

Because the body of the process between quotes is impervious to
substitution, we get radically different answers. In fact, by
examining the first process in an input context,
e.g. $x?(z).\lift{w}{y!(z)}$, we see that the process under the lift
operator may be shaped by prefixed inputs binding a name inside it. In
this sense, the lift operator will be seen as a way to dynamically
construct processes before reifying them as names.

Finally equipped with these standard features we can present the
dynamics of the calculus.

\subsubsection{Operational semantics} 

Finally, we introduce the computational dynamics. What marks these
algebras as distinct from other more traditionally studied algebraic
structures, e.g. vector spaces or polynomial rings, is the manner in
which dynamics is captured. In traditional structures, dynamics is typically
expressed through morphisms between such structures, as in linear maps
between vector spaces or morphisms between rings. In algebras
associated with the semantics of computation, the dynamics is
expressed as part of the algebraic structure itself, through a
reduction reduction relation typically denoted by $\red$. Below, we
give a recursive presentation of this relation for the calculus used
in the encoding.

$\red \subseteq \pi \times \pi$
$\red : \pi \to \mathcal{P}(\pi)$

\begin{mathpar}
  \inferrule* [lab=Comm] { \textsf{match}( x_{src}, x_{trgt} ) } { x_{trgt}?(y)P \; | \; x_{src}!\langle {Q} \rangle \red P\{\quotep{Q}/y}\} }
  \and \\
  \inferrule* [lab=Par] {{P} \red {P}'} {{{P} | {Q}} \red {{P}' | {Q}}}
  \and
  \inferrule* [lab=Equiv]{{{P} \scong {P}'} \andalso {{P}' \red {Q}'} \andalso {{Q}' \scong {Q}}}{{P} \red {Q}}
\end{mathpar}

\begin{eqnarray*}
  match_{\equiv} (\quotep{P},\quotep{Q}) & := & P \equiv Q \\
  match_{\dagger}(\quotep{P},\quotep{Q}) & := & \forall R. P|Q \red^{*} R => R \red^{*} 0 \\
  match_{K}(\quotep{P},\quotep{Q}) & := & K \mbox{ for some context } K
\end{eqnarray*}

$u?(x)P | u!\langle Q \rangle \red P\{\quotep{Q}/x\}$

%We write $\wred$ for $\red^*$, and $P\red$ if $\exists Q $ such that $ P \red Q$.
We write $P\red$ if $\exists Q $ such that $ P \red Q$ and $P\not\red$, otherwise.

\section{Replication}

As mentioned before, it is known that replication (and hence
recursion) can be implemented in a higher-order process algebra
\cite{SangiorgiWalker}. As our first example of calculation with the
machinery thus far presented we give the construction explicitly in
the {\rhoc}.

\begin{eqnarray}
	D_{x} & := & \prefix{x}{y}{(\binpar{\outputp{x}{y}}{@{y}})} \nonumber\\
	\bangp_{x}{P} & := & \binpar{{x}!\langle{\binpar{D_{x}}{P}}\rangle}{D_{x}} \nonumber
\end{eqnarray}

\begin{eqnarray}
	\bangp_{x}{P} & & \nonumber\\
	=
	& {x}!\langle{(\prefix{x}{y}{(\outputp{x}{y} | @{y})) | P}}\rangle 
	      | \prefix{x}{y}{(\outputp{x}{y} | @{y})} & \nonumber\\
	\red
	& (\outputp{x}{y} | @{y})\substn{\quotep{(\prefix{x}{y}{(@{y} | \outputp{x}{y})) | P}}}{y} & \nonumber\\
	=
	& \outputp{x}{\quotep{(\prefix{x}{y}{(\outputp{x}{y} | @{y})) | P}}}
	  | {(\prefix{x}{y}{(\outputp{x}{y} | @{y})) | P}} & \nonumber\\
	\red
	& \ldots & \nonumber\\
	\red^*
	& P | P | \ldots & \nonumber
\end{eqnarray}

Of course, this encoding, as an implementation, runs away, unfolding
$\bangp{P}$ eagerly. A lazier and more implementable replication
operator, restricted to input-guarded processes, may be obtained as follows.

\begin{eqnarray}
\bangp{\prefix{u}{v}{P}} 
	:= 
	\binpar{\lift{x}{\prefix{u}{v}{(\binpar{D(x)}{P})}}}{D(x)} \nonumber
\end{eqnarray}

\begin{remark}
  Note that the lazier definition still does not deal with summation
  or mixed summation (i.e. sums over input and output). The reader is
  invited to construct definitions of replication that deal with these
  features. 

  Further, the definitions are parameterized in a name, $x$. Can you,
  gentle reader, make a definition that eliminates this parameter and
  guarantees no accidental interaction between the replication
  machinery and the process being replicated -- i.e. no accidental
  sharing of names used by the process to get its work done and the
  name(s) used by the replication to effect copying. This latter
  revision of the definition of replication is crucial to obtaining
  the expected identity $!!P \sim !P$.
\end{remark}

\begin{remark}\label{rem:paradoxical_combinator}
  The reader familiar with the lambda calculus will have noticed the
  similarity between $D$ and the paradoxical combinator.

  [Ed. note: the existence of this seems to suggest we have to be more
  restrictive on the set of processes and names we admit if we are to
  support no-cloning.]
\end{remark}

\subsubsection{Bisimulation}

The computational dynamics gives rise to another kind of equivalence,
the equivalence of computational behavior. As previously mentioned
this is typically captured \emph{via} some form of bisimulation.

% The notion we use in this paper is weak barbed bisimulation
% \cite{milner91polyadicpi}.

The notion we use in this paper is derived from weak barbed
bisimulation \cite{milner91polyadicpi}. 

\begin{definition}
An \emph{observation relation}, $\downarrow_{\mathcal N}$, over a set
of names, $\mathcal N$, is the smallest relation satisfying the rules
below.

\infrule[Out-barb]{y \in {\mathcal N}, \; x \nameeq y}
		  {\outputp{x}{v} \downarrow_{\mathcal N} x}
\infrule[Par-barb]{\mbox{$P\downarrow_{\mathcal N} x$ or $Q\downarrow_{\mathcal N} x$}}
		  {\binpar{P}{Q} \downarrow_{\mathcal N} x}

We write $P \Downarrow_{\mathcal N} x$ if there is $Q$ such that 
$P \wred Q$ and $Q \downarrow_{\mathcal N} x$.
\end{definition}

\begin{definition}
%\label{def.bbisim}
An  ${\mathcal N}$-\emph{barbed bisimulation} over a set of names, ${\mathcal N}$, is a symmetric binary relation 
${\mathcal S}_{\mathcal N}$ between agents such that $P\rel{S}_{\mathcal N}Q$ implies:
\begin{enumerate}
\item If $P \red P'$ then $Q \wred Q'$ and $P'\rel{S}_{\mathcal N} Q'$.
\item If $P\downarrow_{\mathcal N} x$, then $Q\Downarrow_{\mathcal N} x$.
\end{enumerate}
$P$ is ${\mathcal N}$-barbed bisimilar to $Q$, written
$P \wbbisim_{\mathcal N} Q$, if $P \rel{S}_{\mathcal N} Q$ for some ${\mathcal N}$-barbed bisimulation ${\mathcal S}_{\mathcal N}$.
\end{definition}

$\mathcal{R} \subseteq \pi \times \pi$

$P \mathcal{R} Q => \forall P'. P \red P' \Rightarrow \exists Q'. Q \red Q', P' \mathcal{R} Q'$

$P \vdash x \Rightarrow Q \vdash x$

\begin{mathpar}
  \inferrule*[lab=Out-barb]{x \nameeq y}{{y}!\langle{Q}\rangle \vdash x}
  \and
  \inferrule*[lab=Par-barb]{\mbox{$P\vdash x$ or $Q\vdash x$}}{\binpar{P}{Q} \vdash x}
\end{mathpar}

\subsubsection{Contexts}

One of the principle advantages of computational calculi like the
$\pi$-calculus is a well-defined notion of context,
contextual-equivalence and a correlation between
contextual-equivalence and notions of bisimulation. The notion of
context allows the decomposition of a process into (sub-)process and
its syntactic environment, its context. Thus, a context may be
thought of as a process with a ``hole'' (written $\Box$) in it. The
application of a context $M$ to a process $P$, written $M[P]$, is
tantamount to filling the hole in $M$ with $P$. In this paper we do
not need the full weight of this theory, but do make use of the notion
of context in the proof the main theorem. 

\begin{mathpar}
  \inferrule* [lab=summation] {} {{M_{M},M_{N}} \bc \Box \;|\; x.M_{A} \;|\; M_{M}+M_{N}}
  \and
  \inferrule* [lab=agent] {} {{M_{A}} \bc (\vec{x})M_{P} \;| \; \clift{P_0,\ldots,M_{P},\ldots,P_N}}
  \and \\
  \inferrule* [lab=process] {} {{M_{P}} \bc M_{N} \;| \;P|M_{P} }
\end{mathpar} 

\begin{mathpar}
  \inferrule* [lab=sychronization] {} {M_{N} \bc \Box \;|\; x?M_{F} \;|\; x!M_{C}}
  \and
  \inferrule* [lab=abstraction] {} {{M_{F}} \bc (x)M_{P} }
  \and
  \inferrule* [lab=concretion] {} {{M_{C}} \bc \langle M_{P} \rangle }
  \and \\
  \inferrule* [lab=process] {} {{M_{P}} \bc M_{N} \;| \;P|M_{P} }
\end{mathpar}

\begin{definition}[contextual application] Given a context $M$, and
  process $P$, we define the \emph{contextual application}, $M[P] :=
  M\{P/\Box\}$. That is, the contextual application of M to P is the
  substitution of $P$ for $\Box$ in $M$.
\end{definition}

$\meaningof{-} : L \to \mathcal{P}(\pi)$

\begin{mathpar}
  \inferrule* [lab=collection] {} {\meaningof{true} = \pi, \and \meaningof{~E} = \pi \setminus \meaningof{E}, \and \meaningof{E_{1} \& E_{2}} = \meaningof{E_{1}} \cap \meaningof{E_{2}}}
\end{mathpar}

\begin{mathpar}
  \inferrule* [lab=structure] {} {\meaningof{0} = \{ P \in \pi | P \equiv 0 \}, \and \\ \meaningof{E_1 | E_2} = \{ P \in \pi | P \equiv P_{1} | P_{2}, P_{1} \in \meaningof{E_{1}}, P_{2} \in \meaningof{E_2}\} }
\end{mathpar}

\begin{mathpar}
 \inferrule* [lab=behavior] {} {\meaningof{\langle a?b \rangle E} = \{ P \in \pi | P \equiv Q | u?(y)P', \\ \and \\\\ \and \\ \;\;\; u \in \meaningof{a}, \forall z.P'\{z/y\} \in \meaningof{E\{z/b\}}\}, \and \\ \meaningof{a!E} = \{ P \in \pi | P \equiv Q | x!\langle P' \rangle, x \in \meaningof{a} P' \in \meaningof{E}\} }
\end{mathpar}

\begin{mathpar}
 \inferrule* [lab=nominal] {} {\meaningof{\quotep{E}} = \{ \quotep{P} \in \quotep{\pi} | P \in \meaningof{E} \}, \and \meaningof{\quotep{P}} = \{ \quotep{Q} \in \quotep{\pi} | P \equiv Q \} \and \\ \meaningof{@\quotep{E}} = \{ P \in \pi | P \equiv @x, x \in \meaningof{E} \}}
\end{mathpar}

\begin{eqnarray*}
  \\
  \meaningof{-} : TS \to ST
\end{eqnarray*}

\begin{eqnarray*}
  \\
  L : TS \to ST
\end{eqnarray*}

\begin{eqnarray*}
  \\
  P \models E \iff P \in \meaningof{E}
\end{eqnarray*}

\begin{eqnarray*}
  P \approx_{L} Q \iff \forall E \in L. P \models E \iff Q \models E
\end{eqnarray*}

\begin{eqnarray*}
  P \approx_{K} Q
\end{eqnarray*}

\begin{eqnarray*}
  P \approx Q
\end{eqnarray*}

$\approx_{K} = \approx = \approx_{L}$

\subsubsection{Contextual duality}

Note that contexts extend the quotation operation to a family of
operations from processes to names. Given a context, $M$, we can
define a \emph{nominal context}, $\quotep{M}$ by $\quotep{M}[P] :=
\quotep{M[P]}$. To foreshadow what is to come we observe that these
operations enjoy a duality with processes very much like the duality
between vectors and maps from vectors to scalars.

Further, because the calculus is essentially higher-order, we have a
correspondence between contexts and processes. More specifically,
given a name $x$ and a context $M$ we can construct $M^{*}_{x}$ such
that 

\begin{mathpar}
  M^{*}_{x} | \lift{x}{P} \red M[P]
\end{mathpar}

namely,

\begin{mathpar}
  M^{*}_{x} := x?(u).M[\dropn{u}]
\end{mathpar}

The dependence of $M^{*}_{x}$ on a name makes it an abstraction, 

\begin{mathpar}
  M^{*} := (x)x?(u).M[\dropn{u}]
\end{mathpar}

\subsection{Additional notation}

It will sometimes be convenient to denote the process a name
quotes. We already have the notation $x = \quotep{P}$, but it will be
convenient to introduce an alternate notation, $\procn{x}$, when we
want to emphasize the connection to the use of the name. Note that, by
virtue of name equivalence, $\quotep{\procn{x}} \nameeq x$; so, the
notation is consistent with previous definitions.

Further, because names have structure it is possible to effect
substitutions on the basis of that structure. This means we need to
upgrade our notation for substitutions, which we accomplish by
adapting comprehension notation. Thus,

\begin{mathpar}
  P\{ y / x : x \in S \}
\end{mathpar}

is interpreted to mean the process derived from P by replacing (in a
capture-avoiding manner) each occurrence of $x$ in $S$ by $y$. For example,

\begin{mathpar}
  P\{ \quotep{\procn{x}|\procn{x}} / x : x \in \freenames{P} \}
\end{mathpar}

will replace each (occurrence) of a free name $x$ in $P$ by
$\quotep{\procn{x}|\procn{x}}$.

Also, we will avail ourselves of the notation $x^{L}$ and $x^{R}$ to
denote injections of a name into disjoint copies of the name
space. There are numerous ways to accomplish this. One example can be
found in \cite{MeredithR05}. This notation overloads to vectors of
names: $\vec{x}^{\pi} := (x_{i}^{\pi} \; : \; 0 \leq i < |\vec{x}| )$ where $\pi \in \{L,R\}$.

We also use $P^{\Box} := P|\Box$.

In \cite{MeredithR05} an interpretation of the new operator is
given. It turns out that there are several possible interpretations
all enjoying the requisite algebraic properties of the operator (see
\cite{milner91polyadicpi}). We will therefore make liberal use of
$(\nu\; \vec{x})P$.

% subsection the_syntax_and_semantics_of_the_notation_system (end)   

\input{qm2pi.qmops} 

\input{qm2pi.sterngerlach} 

\input{qm2pi.metric} 

% section concurrent_process_calculi (end)

%\input{qm2pi.proofsketch}

% section proof sketch (end)

%\input{qm2pi.slviaknots} 

% section spatial logic via knots (end)

\input{qm2pi.conclusion}

% section conclusion (end)

%\input{qm2pi.dtcodes} 

% section wiring algorithm (end)

\input{qm2pi.ack} 

% section acknowledgments (end)

\newpage


\bibliographystyle{plain}   
\bibliography{../../biblios/main.bib}

\input{qm2pi.rhodetails}

\end{document}

 

%\ifpdf
%\usepackage[pdftex]{graphicx}
%\else
%\usepackage{graphicx}
%\fi

 % \ifpdf
%  \usepackage{pdfsync}
%  \if


%\title{Brief Article}
%\author{David F. Snyder}
%\author{L.G. Meredith}

%\address{Dept. of Math., Texas State University--San Marcos, San Marcos, TX 78666}
       
\pagestyle{empty}


\begin{document}

\lstset{language=[Objective]Caml,frame=shadowbox}

\documentclass[12pt]{llncs}
%\documentclass{jktr}

\usepackage[pdftex]{hyperref}                   
\usepackage {listings}
\usepackage {mathpartir}
\usepackage{bcprules}
%\usepackage{listings}
                       
\usepackage{graphicx} 
%\usepackage[margins=2.5cm,nohead,nofoot]{geometry}
%\usepackage{geometry}
\usepackage{amsfonts}
\usepackage{amstext}
\usepackage{latexsym}
\usepackage{amssymb}
\usepackage{color}


%\include{myPreamble}
\include{qm2pi.local} 

%\ifpdf
%\usepackage[pdftex]{graphicx}
%\else
%\usepackage{graphicx}
%\fi

 % \ifpdf
%  \usepackage{pdfsync}
%  \if


%\title{Brief Article}
%\author{David F. Snyder}
%\author{L.G. Meredith}

%\address{Dept. of Math., Texas State University--San Marcos, San Marcos, TX 78666}
       
\pagestyle{empty}


\begin{document}

\lstset{language=[Objective]Caml,frame=shadowbox}

\input{qm2pi.front}

% section front matter (end)

\input{qm2pi.intro} 
 
% section introduction (end)

% \input{qm2pi.knotations} 

% section notation (end)

\input{qm2pi.process.calculi} 

% section concurrent_process_calculi_and_spatial_logics_ (end)
    
%\input{qm2pi.knots2pi} 

%\input{qm2pi.trefoil} 

%\input{qm2pi.mainthm} 

% subsection basic_interpretation (end)

%\input{qm2pi.rho.presentation} 
\subsection{The syntax and semantics of the notation system}\label{sub:the_syntax_and_semantics_of_the_notation_system} % (fold)

We now summarize a technical presentation of the calculus that
embodies our theory of dynamics. The typical presentation of such a
calculus follows the style of giving generators and relations on
them. The grammar, below, describing term constructors, freely
generates the set of processes, $\Proc$. This set is then quotiented
by a relation known as structural congruence and it is over this set
that the notion of dynamics is expressed. This presentation is
essentially that of \cite{MeredithR05} with the addition of
polyadicity and summation. For readability we have relegated some of
the technical subtleties to an appendix.

\subsubsection{Process grammar}\label{subsub:process_grammar}

\begin{mathpar}
  \inferrule* [lab=synchronization] {} {{M} \bc \pzero \;|\; x?F \;|\; x!C }
  \and
  \inferrule* [lab=abstraction] {} {{F} \bc (x)P}
  \and
  \inferrule* [lab=concretion] {} {{C} \bc \langle Q \rangle}
  \and
  \inferrule* [lab=process] {} {{P,Q} \bc M \;| \;P|Q \;|\; @{x}}
  \and
  \inferrule* [lab=name] {} {{x} \bc \quotep{P}}
\end{mathpar} 

Note that $\vec{x}$ (resp. $\vec{P}$) denotes a vector of names
(resp. processes) of length $|\vec{x}|$ (resp. $|\vec{P}|$). We adopt
the following useful abbreviations.

\begin{mathpar}
   x?(\vec{y}).P := x.(\vec{y})P \and  x\clift{\vec{P}} := x.\clift{\vec{P}}
   \and x!(y) := \lift{x}{\dropn{y}}
   \and \Pi_{i=0}^{n-1}P_i := P_0 | \ldots | P_{n-1}
\end{mathpar}

\subsubsection{Structural congruence}

\paragraph{Free and bound names and alpha-equivalence.} At the
core of structural equivalence is alpha-equivalence which identifies
process that are the same up to a change of variable. Formally, we
recognize the distinction between free and bound names. The free names
of a process, $\freenames{P}$, may be calculated recursively as
follows:

\begin{mathpar}
\freenames{\pzero} := \emptyset
  \and \\
  \freenames{x?(y).P} := \{ x \} \cup (\freenames{P} \setminus \{ y \})
  \and 
  \freenames{x!\langle P \rangle} := \{ x \} \cup \{ P \} 
  \and \\
  \freenames{P|Q} := \freenames{P} \cup \freenames{Q}
  \and \\
  \freenames{@{x}} := \{ x \}
\end{mathpar}

$\pi$
$\quotep{\pi}$

$\freenames{-} : \pi \to \mathcal{P}(\quotep{\pi})$

\begin{eqnarray*}
  \freenames{\pzero} & := & \emptyset \\
  \freenames{x?(y).P} & := & \{ x \} \cup (\freenames{P} \setminus \{ y \}) \\
  \freenames{x!\langle P \rangle} & := & \{ x \} \cup \{ P \} \\
  \freenames{P|Q} & := & \freenames{P} \cup \freenames{Q} \\
  \freenames{\dropn{x}} & := & \{ x \}
\end{eqnarray*}

The bound names of a process, $\boundnames{P}$, are those names occurring in $P$
that are not free. For example, in $x?(y).0$, the name $x$ is free, while $y$ is bound.

\begin{mathpar}
  \inferrule* [lab=monoidal-laws] {} { P|Q \equiv Q|P \and P|0 \equiv P \and P|(Q|R) \equiv (P|Q)|R }
\end{mathpar}

\begin{mathpar}
  \inferrule* [lab=alpha-equivalence] {} { (x)P \equiv (y)P\{y/x\} \and y \not\in \freenames{P} }
\end{mathpar}

\begin{definition}
Then two processes, $P,Q$, are alpha-equivalent if $P = Q\{\vec{y}/\vec{x}\}$ for
some $\vec{x} \in \boundnames{Q},\vec{y} \in \boundnames{P}$, where $Q\{\vec{y}/\vec{x}\}$
denotes the capture-avoiding substitution of $\vec{y}$ for $\vec{x}$ in $Q$.
\end{definition}

\begin{definition}
  The {\em structural congruence} \cite{SangiorgiWalker} , $\equiv$,
  between processes is the least congruence containing
  alpha-equivalence, satisfying the abelian monoid laws
  (associativity, commutativity and $\pzero$ as identity) for parallel
  composition $|$ and for summation $+$.
\end{definition}

\subsection{Name equivalence}

We take name equivalence, written $\nameeq$, to be the smallest
equivalence relation generated by the following rules.

\begin{mathpar}
\inferrule*[lab=Quote-drop]
{ }
{ \quotep{@{x}} \nameeq x }

\inferrule*[lab=Struct-equiv]
{ P \scong Q }
{ \quotep{P} \nameeq \quotep{Q} }
\end{mathpar}

The astute reader will have noticed that the mutual recursion of names
and processes imposes a mutual recursion on alpha-equivalence and
structural equivalence via name-equivalence. Fortunately, all of this
works out pleasantly and we may calculate in the natural way, free of
concern. The reader interested in the details is referred to the
appendix \ref{appendix:rho_details}.

\subsection{Substitution}

We use $\Proc$ for the set of processes, $\QProc$ for the set of
names, and $\id{\{}\vec{y} / \vec{x} \id{\}}$ to denote partial maps,
$s : \QProc \rightarrow \QProc$. A map, $s$ lifts, uniquely, to a map
on process terms, $\widehat{s} : \Proc \rightarrow \Proc$ by the
following equations.

\begin{mathpar}
  (0) \psubstp{Q}{P} := 0 \\
  (R \juxtap S) \psubstp{Q}{P}
  :=    
  (R)\psubstp{Q}{P} \juxtap (S) \psubstp{Q}{P} \\
  (x?(y).R) \psubstp{Q}{P}    
  :=    
  (x)\substp{Q}{P} (z)\concat( (R \psubstn{z}{y}) \psubstp{Q}{P} ) \\
  (\lift{x}{R}) \psubstp{Q}{P}  
  :=
  \lift{(x)\substp{Q}{P}}{ R \psubstp{Q}{P} } \\
%   (\dropn{x})  \psubstp{Q}{P}       
%   := 
%   \left\{ 
%     \begin{array}{ccc} 
%       \dropn{\quotep{Q}} & & x \nameeq \quotep{P} \\
%       \dropn{x} & & otherwise \\
%     \end{array}
%   \right. 
  (\dropn{x})  \psubstp{Q}{P}       
  := 
  \left\{ 
    \begin{array}{ccc} 
      Q & & x \nameeq \quotep{P} \\
      \dropn{x} & & otherwise \\
    \end{array}
  \right.
\end{mathpar}
 

where

\begin{eqnarray}
  (x)\id{\{} \lpquote Q \rpquote / \lpquote P \rpquote \id{\}}            = 
  \left\{ 
    \begin{array}{ccc}
      \lpquote Q \rpquote & & x \nameeq \lpquote P \rpquote \\
      x & & otherwise \\
    \end{array}
  \right. \nonumber
\end{eqnarray}

and $z$ is chosen distinct from $\quotep{P}$, $\quotep{Q}$, the free
names in $Q$, and all the names in $R$. Our $\alpha$-equivalence will
be built in the standard way from this substitution.

\begin{remark}\label{rem:no_self_referential_names}
  One consequence of these definitions is that $\forall P. \quotep{P}
  \not\in \freenames{P}$.
\end{remark}

\subsection{ Dynamic quote: an example }

Anticipating something of what's to come, consider applying the
substitution, $\widehat{\id{\{}u / z \id{\}}}$, to the following pair
of processes, $\lift{w}{y!(z)}$ and $w[ \lpquote y!(z) \rpquote ]$.

\begin{eqnarray}
	\lift{w}{y!(z)}\widehat{\id{\{}u / z \id{\}}}
		& = &
		\lift{w}{y!(u)} \nonumber\\
	w[ \lpquote y!(z) \rpquote ] \widehat{ \id{\{}u / z \id{\}} }
		& = &
		w[ \lpquote y!(z) \rpquote ] \nonumber
\end{eqnarray}

Because the body of the process between quotes is impervious to
substitution, we get radically different answers. In fact, by
examining the first process in an input context,
e.g. $x?(z).\lift{w}{y!(z)}$, we see that the process under the lift
operator may be shaped by prefixed inputs binding a name inside it. In
this sense, the lift operator will be seen as a way to dynamically
construct processes before reifying them as names.

Finally equipped with these standard features we can present the
dynamics of the calculus.

\subsubsection{Operational semantics} 

Finally, we introduce the computational dynamics. What marks these
algebras as distinct from other more traditionally studied algebraic
structures, e.g. vector spaces or polynomial rings, is the manner in
which dynamics is captured. In traditional structures, dynamics is typically
expressed through morphisms between such structures, as in linear maps
between vector spaces or morphisms between rings. In algebras
associated with the semantics of computation, the dynamics is
expressed as part of the algebraic structure itself, through a
reduction reduction relation typically denoted by $\red$. Below, we
give a recursive presentation of this relation for the calculus used
in the encoding.

$\red \subseteq \pi \times \pi$
$\red : \pi \to \mathcal{P}(\pi)$

\begin{mathpar}
  \inferrule* [lab=Comm] { \textsf{match}( x_{src}, x_{trgt} ) } { x_{trgt}?(y)P \; | \; x_{src}!\langle {Q} \rangle \red P\{\quotep{Q}/y}\} }
  \and \\
  \inferrule* [lab=Par] {{P} \red {P}'} {{{P} | {Q}} \red {{P}' | {Q}}}
  \and
  \inferrule* [lab=Equiv]{{{P} \scong {P}'} \andalso {{P}' \red {Q}'} \andalso {{Q}' \scong {Q}}}{{P} \red {Q}}
\end{mathpar}

\begin{eqnarray*}
  match_{\equiv} (\quotep{P},\quotep{Q}) & := & P \equiv Q \\
  match_{\dagger}(\quotep{P},\quotep{Q}) & := & \forall R. P|Q \red^{*} R => R \red^{*} 0 \\
  match_{K}(\quotep{P},\quotep{Q}) & := & K \mbox{ for some context } K
\end{eqnarray*}

$u?(x)P | u!\langle Q \rangle \red P\{\quotep{Q}/x\}$

%We write $\wred$ for $\red^*$, and $P\red$ if $\exists Q $ such that $ P \red Q$.
We write $P\red$ if $\exists Q $ such that $ P \red Q$ and $P\not\red$, otherwise.

\section{Replication}

As mentioned before, it is known that replication (and hence
recursion) can be implemented in a higher-order process algebra
\cite{SangiorgiWalker}. As our first example of calculation with the
machinery thus far presented we give the construction explicitly in
the {\rhoc}.

\begin{eqnarray}
	D_{x} & := & \prefix{x}{y}{(\binpar{\outputp{x}{y}}{@{y}})} \nonumber\\
	\bangp_{x}{P} & := & \binpar{{x}!\langle{\binpar{D_{x}}{P}}\rangle}{D_{x}} \nonumber
\end{eqnarray}

\begin{eqnarray}
	\bangp_{x}{P} & & \nonumber\\
	=
	& {x}!\langle{(\prefix{x}{y}{(\outputp{x}{y} | @{y})) | P}}\rangle 
	      | \prefix{x}{y}{(\outputp{x}{y} | @{y})} & \nonumber\\
	\red
	& (\outputp{x}{y} | @{y})\substn{\quotep{(\prefix{x}{y}{(@{y} | \outputp{x}{y})) | P}}}{y} & \nonumber\\
	=
	& \outputp{x}{\quotep{(\prefix{x}{y}{(\outputp{x}{y} | @{y})) | P}}}
	  | {(\prefix{x}{y}{(\outputp{x}{y} | @{y})) | P}} & \nonumber\\
	\red
	& \ldots & \nonumber\\
	\red^*
	& P | P | \ldots & \nonumber
\end{eqnarray}

Of course, this encoding, as an implementation, runs away, unfolding
$\bangp{P}$ eagerly. A lazier and more implementable replication
operator, restricted to input-guarded processes, may be obtained as follows.

\begin{eqnarray}
\bangp{\prefix{u}{v}{P}} 
	:= 
	\binpar{\lift{x}{\prefix{u}{v}{(\binpar{D(x)}{P})}}}{D(x)} \nonumber
\end{eqnarray}

\begin{remark}
  Note that the lazier definition still does not deal with summation
  or mixed summation (i.e. sums over input and output). The reader is
  invited to construct definitions of replication that deal with these
  features. 

  Further, the definitions are parameterized in a name, $x$. Can you,
  gentle reader, make a definition that eliminates this parameter and
  guarantees no accidental interaction between the replication
  machinery and the process being replicated -- i.e. no accidental
  sharing of names used by the process to get its work done and the
  name(s) used by the replication to effect copying. This latter
  revision of the definition of replication is crucial to obtaining
  the expected identity $!!P \sim !P$.
\end{remark}

\begin{remark}\label{rem:paradoxical_combinator}
  The reader familiar with the lambda calculus will have noticed the
  similarity between $D$ and the paradoxical combinator.

  [Ed. note: the existence of this seems to suggest we have to be more
  restrictive on the set of processes and names we admit if we are to
  support no-cloning.]
\end{remark}

\subsubsection{Bisimulation}

The computational dynamics gives rise to another kind of equivalence,
the equivalence of computational behavior. As previously mentioned
this is typically captured \emph{via} some form of bisimulation.

% The notion we use in this paper is weak barbed bisimulation
% \cite{milner91polyadicpi}.

The notion we use in this paper is derived from weak barbed
bisimulation \cite{milner91polyadicpi}. 

\begin{definition}
An \emph{observation relation}, $\downarrow_{\mathcal N}$, over a set
of names, $\mathcal N$, is the smallest relation satisfying the rules
below.

\infrule[Out-barb]{y \in {\mathcal N}, \; x \nameeq y}
		  {\outputp{x}{v} \downarrow_{\mathcal N} x}
\infrule[Par-barb]{\mbox{$P\downarrow_{\mathcal N} x$ or $Q\downarrow_{\mathcal N} x$}}
		  {\binpar{P}{Q} \downarrow_{\mathcal N} x}

We write $P \Downarrow_{\mathcal N} x$ if there is $Q$ such that 
$P \wred Q$ and $Q \downarrow_{\mathcal N} x$.
\end{definition}

\begin{definition}
%\label{def.bbisim}
An  ${\mathcal N}$-\emph{barbed bisimulation} over a set of names, ${\mathcal N}$, is a symmetric binary relation 
${\mathcal S}_{\mathcal N}$ between agents such that $P\rel{S}_{\mathcal N}Q$ implies:
\begin{enumerate}
\item If $P \red P'$ then $Q \wred Q'$ and $P'\rel{S}_{\mathcal N} Q'$.
\item If $P\downarrow_{\mathcal N} x$, then $Q\Downarrow_{\mathcal N} x$.
\end{enumerate}
$P$ is ${\mathcal N}$-barbed bisimilar to $Q$, written
$P \wbbisim_{\mathcal N} Q$, if $P \rel{S}_{\mathcal N} Q$ for some ${\mathcal N}$-barbed bisimulation ${\mathcal S}_{\mathcal N}$.
\end{definition}

$\mathcal{R} \subseteq \pi \times \pi$

$P \mathcal{R} Q => \forall P'. P \red P' \Rightarrow \exists Q'. Q \red Q', P' \mathcal{R} Q'$

$P \vdash x \Rightarrow Q \vdash x$

\begin{mathpar}
  \inferrule*[lab=Out-barb]{x \nameeq y}{{y}!\langle{Q}\rangle \vdash x}
  \and
  \inferrule*[lab=Par-barb]{\mbox{$P\vdash x$ or $Q\vdash x$}}{\binpar{P}{Q} \vdash x}
\end{mathpar}

\subsubsection{Contexts}

One of the principle advantages of computational calculi like the
$\pi$-calculus is a well-defined notion of context,
contextual-equivalence and a correlation between
contextual-equivalence and notions of bisimulation. The notion of
context allows the decomposition of a process into (sub-)process and
its syntactic environment, its context. Thus, a context may be
thought of as a process with a ``hole'' (written $\Box$) in it. The
application of a context $M$ to a process $P$, written $M[P]$, is
tantamount to filling the hole in $M$ with $P$. In this paper we do
not need the full weight of this theory, but do make use of the notion
of context in the proof the main theorem. 

\begin{mathpar}
  \inferrule* [lab=summation] {} {{M_{M},M_{N}} \bc \Box \;|\; x.M_{A} \;|\; M_{M}+M_{N}}
  \and
  \inferrule* [lab=agent] {} {{M_{A}} \bc (\vec{x})M_{P} \;| \; \clift{P_0,\ldots,M_{P},\ldots,P_N}}
  \and \\
  \inferrule* [lab=process] {} {{M_{P}} \bc M_{N} \;| \;P|M_{P} }
\end{mathpar} 

\begin{mathpar}
  \inferrule* [lab=sychronization] {} {M_{N} \bc \Box \;|\; x?M_{F} \;|\; x!M_{C}}
  \and
  \inferrule* [lab=abstraction] {} {{M_{F}} \bc (x)M_{P} }
  \and
  \inferrule* [lab=concretion] {} {{M_{C}} \bc \langle M_{P} \rangle }
  \and \\
  \inferrule* [lab=process] {} {{M_{P}} \bc M_{N} \;| \;P|M_{P} }
\end{mathpar}

\begin{definition}[contextual application] Given a context $M$, and
  process $P$, we define the \emph{contextual application}, $M[P] :=
  M\{P/\Box\}$. That is, the contextual application of M to P is the
  substitution of $P$ for $\Box$ in $M$.
\end{definition}

$\meaningof{-} : L \to \mathcal{P}(\pi)$

\begin{mathpar}
  \inferrule* [lab=collection] {} {\meaningof{true} = \pi, \and \meaningof{~E} = \pi \setminus \meaningof{E}, \and \meaningof{E_{1} \& E_{2}} = \meaningof{E_{1}} \cap \meaningof{E_{2}}}
\end{mathpar}

\begin{mathpar}
  \inferrule* [lab=structure] {} {\meaningof{0} = \{ P \in \pi | P \equiv 0 \}, \and \\ \meaningof{E_1 | E_2} = \{ P \in \pi | P \equiv P_{1} | P_{2}, P_{1} \in \meaningof{E_{1}}, P_{2} \in \meaningof{E_2}\} }
\end{mathpar}

\begin{mathpar}
 \inferrule* [lab=behavior] {} {\meaningof{\langle a?b \rangle E} = \{ P \in \pi | P \equiv Q | u?(y)P', \\ \and \\\\ \and \\ \;\;\; u \in \meaningof{a}, \forall z.P'\{z/y\} \in \meaningof{E\{z/b\}}\}, \and \\ \meaningof{a!E} = \{ P \in \pi | P \equiv Q | x!\langle P' \rangle, x \in \meaningof{a} P' \in \meaningof{E}\} }
\end{mathpar}

\begin{mathpar}
 \inferrule* [lab=nominal] {} {\meaningof{\quotep{E}} = \{ \quotep{P} \in \quotep{\pi} | P \in \meaningof{E} \}, \and \meaningof{\quotep{P}} = \{ \quotep{Q} \in \quotep{\pi} | P \equiv Q \} \and \\ \meaningof{@\quotep{E}} = \{ P \in \pi | P \equiv @x, x \in \meaningof{E} \}}
\end{mathpar}

\begin{eqnarray*}
  \\
  \meaningof{-} : TS \to ST
\end{eqnarray*}

\begin{eqnarray*}
  \\
  L : TS \to ST
\end{eqnarray*}

\begin{eqnarray*}
  \\
  P \models E \iff P \in \meaningof{E}
\end{eqnarray*}

\begin{eqnarray*}
  P \approx_{L} Q \iff \forall E \in L. P \models E \iff Q \models E
\end{eqnarray*}

\begin{eqnarray*}
  P \approx_{K} Q
\end{eqnarray*}

\begin{eqnarray*}
  P \approx Q
\end{eqnarray*}

$\approx_{K} = \approx = \approx_{L}$

\subsubsection{Contextual duality}

Note that contexts extend the quotation operation to a family of
operations from processes to names. Given a context, $M$, we can
define a \emph{nominal context}, $\quotep{M}$ by $\quotep{M}[P] :=
\quotep{M[P]}$. To foreshadow what is to come we observe that these
operations enjoy a duality with processes very much like the duality
between vectors and maps from vectors to scalars.

Further, because the calculus is essentially higher-order, we have a
correspondence between contexts and processes. More specifically,
given a name $x$ and a context $M$ we can construct $M^{*}_{x}$ such
that 

\begin{mathpar}
  M^{*}_{x} | \lift{x}{P} \red M[P]
\end{mathpar}

namely,

\begin{mathpar}
  M^{*}_{x} := x?(u).M[\dropn{u}]
\end{mathpar}

The dependence of $M^{*}_{x}$ on a name makes it an abstraction, 

\begin{mathpar}
  M^{*} := (x)x?(u).M[\dropn{u}]
\end{mathpar}

\subsection{Additional notation}

It will sometimes be convenient to denote the process a name
quotes. We already have the notation $x = \quotep{P}$, but it will be
convenient to introduce an alternate notation, $\procn{x}$, when we
want to emphasize the connection to the use of the name. Note that, by
virtue of name equivalence, $\quotep{\procn{x}} \nameeq x$; so, the
notation is consistent with previous definitions.

Further, because names have structure it is possible to effect
substitutions on the basis of that structure. This means we need to
upgrade our notation for substitutions, which we accomplish by
adapting comprehension notation. Thus,

\begin{mathpar}
  P\{ y / x : x \in S \}
\end{mathpar}

is interpreted to mean the process derived from P by replacing (in a
capture-avoiding manner) each occurrence of $x$ in $S$ by $y$. For example,

\begin{mathpar}
  P\{ \quotep{\procn{x}|\procn{x}} / x : x \in \freenames{P} \}
\end{mathpar}

will replace each (occurrence) of a free name $x$ in $P$ by
$\quotep{\procn{x}|\procn{x}}$.

Also, we will avail ourselves of the notation $x^{L}$ and $x^{R}$ to
denote injections of a name into disjoint copies of the name
space. There are numerous ways to accomplish this. One example can be
found in \cite{MeredithR05}. This notation overloads to vectors of
names: $\vec{x}^{\pi} := (x_{i}^{\pi} \; : \; 0 \leq i < |\vec{x}| )$ where $\pi \in \{L,R\}$.

We also use $P^{\Box} := P|\Box$.

In \cite{MeredithR05} an interpretation of the new operator is
given. It turns out that there are several possible interpretations
all enjoying the requisite algebraic properties of the operator (see
\cite{milner91polyadicpi}). We will therefore make liberal use of
$(\nu\; \vec{x})P$.

% subsection the_syntax_and_semantics_of_the_notation_system (end)   

\input{qm2pi.qmops} 

\input{qm2pi.sterngerlach} 

\input{qm2pi.metric} 

% section concurrent_process_calculi (end)

%\input{qm2pi.proofsketch}

% section proof sketch (end)

%\input{qm2pi.slviaknots} 

% section spatial logic via knots (end)

\input{qm2pi.conclusion}

% section conclusion (end)

%\input{qm2pi.dtcodes} 

% section wiring algorithm (end)

\input{qm2pi.ack} 

% section acknowledgments (end)

\newpage


\bibliographystyle{plain}   
\bibliography{../../biblios/main.bib}

\input{qm2pi.rhodetails}

\end{document}



% section front matter (end)

\section{Introduction}\label{sec:introduction} % (fold)
In this draft of the material i am going to have to dispense with the
usual writing conventions adopted in papers on these topics. i'm going
to have adopt whatever tone i need at the time i'm writing up the
calculations. Sometimes this may be very conversational; others it may
be the barest mathematical grunts; others still it may be that i have
lifted text from one of my other papers because the exposition of some
point was better said there. i hope that my readers are not unduly put
out by this decision. i'm not doing this to flout convention or be
rebellious. i find these calculations very technically challenging. To
keep everything going technically, something has to give; i have to
let go of some cognitive burden. So, the academic writing style --
with all of its trade-offs in terms of facilitating technical
communication -- is what i'm letting go of. Perhaps subsequent drafts
can be tightened and polished, but for now, i'm going to speak as if
we were sitting together in a coffee shop with a laptop, wifi and a
pad of paper and a pencil.

So, here's what i have to say. We -- you and i, comfortably ensconced
in our coffee shop and well-equipped with our tools -- can realize and
carry out the calculations of quantum mechanics over a very different
formal theory of dynamics, a formal theory of dynamics that
corresponds to a theory of concurrent computation with
\emph{reflection}. It has the advantage that the underlying theory is
already `quantized', but supports analogues all of the continuuous
operations. Strikingly, this underlying theory has recently been
connected with a notion of metric that we can show, by calculating
together, coincides with the metric induced by the inner product.

There are a lot of reasons why you might be interested in seeing
calculations of this form. Here's why i'm interested. For the past
several centuries there has been no competitor to the ``Newtonian''
account of dynamics. As a result the predominant share of accounts of
dynamical systems and situations have had to be formulated in terms of
the Newtonian machinery. i view this as an intellectually dangerous
position to occupy. Everything, despite it's intrinsic shape, turns
into a nail to be hit with this hammer. Recently, however, the theory
of computation has matured to the point where we have candidates for
theories of dynamics that offer very different perspective on
reasoning about dynamical systems and situations. Testing these
candidates against very successful accounts of dynamical situations,
like quantum mechanics, is going to give us some sense of how mature
they are and some measure of the quality of these accounts of
dynamics.

\subsection{Summary of contributions and outline of paper}

So, we're going to develop an interpretation of the operations of
quantum mechanics normally interpreted by Hilbert spaces and
operators. We're going to do this over a theory of computation. Note
that this is very different than the usual quantum computation program
which develops notions of computation over quantum mechanics. Rather,
we are developing a story that aligns with Wheeler's slogan: It from
Bit. To do this we will first provide an account of the theory of
computation at play here. Then we will dive into a calculation-driven
interpretation of the operations of quantum mechanics.

The reason we take this approach is that -- until very recently --
there hasn't been an axiomatic account of quantum mechanics. As a
result there has been no sharp delineation of the mathematical theory
supporting interpretation of the physical theory and the physical
theory, itself. So, ambient features of the maths are free to be
exploited (or supressed) without a real accounting of their physical
relevance. There is no sharp statement ``here's the physical theory''
qua \emph{theory} and ``here's the mathematical interpretation''
enabling a judgment of how faithful the interpretation is -- apart
from experimental observation. When there is an axiomatic account we
can judge how well a given mathematical formalism supports an
interpretation of the axioms, independent of
experimentation. Likewise, we can judge how well we have captured our
physical evidence and experience with our axiomatics, independent of
any specific mathematical implementation, with accidental detail that
may or may not have physical significance. 

In lieu of a fully fleshed out and vetted axiomatic account of quantum
mechanics, interpreting the operational notions in service of modeling
physical systems will have to suffice. In other words, we are not in
the business of providing a model of Hilbert spaces and operators. We
are in the business of providing a model of quantum mechanics because
we are motivated by testing our notions of dynamics against physical
theory; and, the predictive calculations of the physical theory must
serve as the best formulation -- shy of a fully fleshed out axiomatic
account -- of the physical theory itself (as they have for scientific
theories since time immemorial). Put another way, despite a
whole-hearted commitment to an It-from-Bit ontology, we are firmly
aligned with the shut-up-and-calculate camp as the best way to obtain
results either from the physical perspective or as a quality assurance
measure of our fledgling theory of dynamics.

In detail, we present a reflective process calculus. Then we develop
intuitive correspondences between the notions available in this
calculus and the usual physical notions supporting quantum mechanical
calculations. Thus, 

\begin{table}[htp]
  \center{
    \fbox{
      \begin{tabular}{c|c}
        quantum mechanics & process calculus \\
        \hline
        scalar & name \\
        state vector & process \\
        dual & contextual duals \\
        matrix & formal sums of process-context-dual pairs \\
        orthogonality & process annihilation \\
        inner product & execution-formula + quoting
      \end{tabular}
    }
  }
  \caption{QM - process calculi correspondences}
\end{table}

Then we tighten up these intuitions to operational definitions. We
employ the Dirac notation as the best proxy we can find for an
abstract syntax of the quantum mechanical notions. The definitions we
develop put us in contact with equational constraints coming from the
theory that we demonstrate the definitions and calculations satisfy.

This puts us in a position to shut up and calculate for the
Stern-Gerlach experimental set up, showing how these predictive
calculations become calculations on processes in our theory of a
reflective process calculus.

Penultimately, we demonstrate that the notion of metric coming from
the inner product coincides with the notion of metric available from
the theory of bisimulation. This demonstration gives us the right to
think of space as arising from behavior. Finally, we consider where we
might go from the new vantage point we have obtained.

% section introduction (end) 
 
% section introduction (end)

% \documentclass[12pt]{llncs}
%\documentclass{jktr}

\usepackage[pdftex]{hyperref}                   
\usepackage {listings}
\usepackage {mathpartir}
\usepackage{bcprules}
%\usepackage{listings}
                       
\usepackage{graphicx} 
%\usepackage[margins=2.5cm,nohead,nofoot]{geometry}
%\usepackage{geometry}
\usepackage{amsfonts}
\usepackage{amstext}
\usepackage{latexsym}
\usepackage{amssymb}
\usepackage{color}


%\include{myPreamble}
\include{qm2pi.local} 

%\ifpdf
%\usepackage[pdftex]{graphicx}
%\else
%\usepackage{graphicx}
%\fi

 % \ifpdf
%  \usepackage{pdfsync}
%  \if


%\title{Brief Article}
%\author{David F. Snyder}
%\author{L.G. Meredith}

%\address{Dept. of Math., Texas State University--San Marcos, San Marcos, TX 78666}
       
\pagestyle{empty}


\begin{document}

\lstset{language=[Objective]Caml,frame=shadowbox}

\input{qm2pi.front}

% section front matter (end)

\input{qm2pi.intro} 
 
% section introduction (end)

% \input{qm2pi.knotations} 

% section notation (end)

\input{qm2pi.process.calculi} 

% section concurrent_process_calculi_and_spatial_logics_ (end)
    
%\input{qm2pi.knots2pi} 

%\input{qm2pi.trefoil} 

%\input{qm2pi.mainthm} 

% subsection basic_interpretation (end)

%\input{qm2pi.rho.presentation} 
\subsection{The syntax and semantics of the notation system}\label{sub:the_syntax_and_semantics_of_the_notation_system} % (fold)

We now summarize a technical presentation of the calculus that
embodies our theory of dynamics. The typical presentation of such a
calculus follows the style of giving generators and relations on
them. The grammar, below, describing term constructors, freely
generates the set of processes, $\Proc$. This set is then quotiented
by a relation known as structural congruence and it is over this set
that the notion of dynamics is expressed. This presentation is
essentially that of \cite{MeredithR05} with the addition of
polyadicity and summation. For readability we have relegated some of
the technical subtleties to an appendix.

\subsubsection{Process grammar}\label{subsub:process_grammar}

\begin{mathpar}
  \inferrule* [lab=synchronization] {} {{M} \bc \pzero \;|\; x?F \;|\; x!C }
  \and
  \inferrule* [lab=abstraction] {} {{F} \bc (x)P}
  \and
  \inferrule* [lab=concretion] {} {{C} \bc \langle Q \rangle}
  \and
  \inferrule* [lab=process] {} {{P,Q} \bc M \;| \;P|Q \;|\; @{x}}
  \and
  \inferrule* [lab=name] {} {{x} \bc \quotep{P}}
\end{mathpar} 

Note that $\vec{x}$ (resp. $\vec{P}$) denotes a vector of names
(resp. processes) of length $|\vec{x}|$ (resp. $|\vec{P}|$). We adopt
the following useful abbreviations.

\begin{mathpar}
   x?(\vec{y}).P := x.(\vec{y})P \and  x\clift{\vec{P}} := x.\clift{\vec{P}}
   \and x!(y) := \lift{x}{\dropn{y}}
   \and \Pi_{i=0}^{n-1}P_i := P_0 | \ldots | P_{n-1}
\end{mathpar}

\subsubsection{Structural congruence}

\paragraph{Free and bound names and alpha-equivalence.} At the
core of structural equivalence is alpha-equivalence which identifies
process that are the same up to a change of variable. Formally, we
recognize the distinction between free and bound names. The free names
of a process, $\freenames{P}$, may be calculated recursively as
follows:

\begin{mathpar}
\freenames{\pzero} := \emptyset
  \and \\
  \freenames{x?(y).P} := \{ x \} \cup (\freenames{P} \setminus \{ y \})
  \and 
  \freenames{x!\langle P \rangle} := \{ x \} \cup \{ P \} 
  \and \\
  \freenames{P|Q} := \freenames{P} \cup \freenames{Q}
  \and \\
  \freenames{@{x}} := \{ x \}
\end{mathpar}

$\pi$
$\quotep{\pi}$

$\freenames{-} : \pi \to \mathcal{P}(\quotep{\pi})$

\begin{eqnarray*}
  \freenames{\pzero} & := & \emptyset \\
  \freenames{x?(y).P} & := & \{ x \} \cup (\freenames{P} \setminus \{ y \}) \\
  \freenames{x!\langle P \rangle} & := & \{ x \} \cup \{ P \} \\
  \freenames{P|Q} & := & \freenames{P} \cup \freenames{Q} \\
  \freenames{\dropn{x}} & := & \{ x \}
\end{eqnarray*}

The bound names of a process, $\boundnames{P}$, are those names occurring in $P$
that are not free. For example, in $x?(y).0$, the name $x$ is free, while $y$ is bound.

\begin{mathpar}
  \inferrule* [lab=monoidal-laws] {} { P|Q \equiv Q|P \and P|0 \equiv P \and P|(Q|R) \equiv (P|Q)|R }
\end{mathpar}

\begin{mathpar}
  \inferrule* [lab=alpha-equivalence] {} { (x)P \equiv (y)P\{y/x\} \and y \not\in \freenames{P} }
\end{mathpar}

\begin{definition}
Then two processes, $P,Q$, are alpha-equivalent if $P = Q\{\vec{y}/\vec{x}\}$ for
some $\vec{x} \in \boundnames{Q},\vec{y} \in \boundnames{P}$, where $Q\{\vec{y}/\vec{x}\}$
denotes the capture-avoiding substitution of $\vec{y}$ for $\vec{x}$ in $Q$.
\end{definition}

\begin{definition}
  The {\em structural congruence} \cite{SangiorgiWalker} , $\equiv$,
  between processes is the least congruence containing
  alpha-equivalence, satisfying the abelian monoid laws
  (associativity, commutativity and $\pzero$ as identity) for parallel
  composition $|$ and for summation $+$.
\end{definition}

\subsection{Name equivalence}

We take name equivalence, written $\nameeq$, to be the smallest
equivalence relation generated by the following rules.

\begin{mathpar}
\inferrule*[lab=Quote-drop]
{ }
{ \quotep{@{x}} \nameeq x }

\inferrule*[lab=Struct-equiv]
{ P \scong Q }
{ \quotep{P} \nameeq \quotep{Q} }
\end{mathpar}

The astute reader will have noticed that the mutual recursion of names
and processes imposes a mutual recursion on alpha-equivalence and
structural equivalence via name-equivalence. Fortunately, all of this
works out pleasantly and we may calculate in the natural way, free of
concern. The reader interested in the details is referred to the
appendix \ref{appendix:rho_details}.

\subsection{Substitution}

We use $\Proc$ for the set of processes, $\QProc$ for the set of
names, and $\id{\{}\vec{y} / \vec{x} \id{\}}$ to denote partial maps,
$s : \QProc \rightarrow \QProc$. A map, $s$ lifts, uniquely, to a map
on process terms, $\widehat{s} : \Proc \rightarrow \Proc$ by the
following equations.

\begin{mathpar}
  (0) \psubstp{Q}{P} := 0 \\
  (R \juxtap S) \psubstp{Q}{P}
  :=    
  (R)\psubstp{Q}{P} \juxtap (S) \psubstp{Q}{P} \\
  (x?(y).R) \psubstp{Q}{P}    
  :=    
  (x)\substp{Q}{P} (z)\concat( (R \psubstn{z}{y}) \psubstp{Q}{P} ) \\
  (\lift{x}{R}) \psubstp{Q}{P}  
  :=
  \lift{(x)\substp{Q}{P}}{ R \psubstp{Q}{P} } \\
%   (\dropn{x})  \psubstp{Q}{P}       
%   := 
%   \left\{ 
%     \begin{array}{ccc} 
%       \dropn{\quotep{Q}} & & x \nameeq \quotep{P} \\
%       \dropn{x} & & otherwise \\
%     \end{array}
%   \right. 
  (\dropn{x})  \psubstp{Q}{P}       
  := 
  \left\{ 
    \begin{array}{ccc} 
      Q & & x \nameeq \quotep{P} \\
      \dropn{x} & & otherwise \\
    \end{array}
  \right.
\end{mathpar}
 

where

\begin{eqnarray}
  (x)\id{\{} \lpquote Q \rpquote / \lpquote P \rpquote \id{\}}            = 
  \left\{ 
    \begin{array}{ccc}
      \lpquote Q \rpquote & & x \nameeq \lpquote P \rpquote \\
      x & & otherwise \\
    \end{array}
  \right. \nonumber
\end{eqnarray}

and $z$ is chosen distinct from $\quotep{P}$, $\quotep{Q}$, the free
names in $Q$, and all the names in $R$. Our $\alpha$-equivalence will
be built in the standard way from this substitution.

\begin{remark}\label{rem:no_self_referential_names}
  One consequence of these definitions is that $\forall P. \quotep{P}
  \not\in \freenames{P}$.
\end{remark}

\subsection{ Dynamic quote: an example }

Anticipating something of what's to come, consider applying the
substitution, $\widehat{\id{\{}u / z \id{\}}}$, to the following pair
of processes, $\lift{w}{y!(z)}$ and $w[ \lpquote y!(z) \rpquote ]$.

\begin{eqnarray}
	\lift{w}{y!(z)}\widehat{\id{\{}u / z \id{\}}}
		& = &
		\lift{w}{y!(u)} \nonumber\\
	w[ \lpquote y!(z) \rpquote ] \widehat{ \id{\{}u / z \id{\}} }
		& = &
		w[ \lpquote y!(z) \rpquote ] \nonumber
\end{eqnarray}

Because the body of the process between quotes is impervious to
substitution, we get radically different answers. In fact, by
examining the first process in an input context,
e.g. $x?(z).\lift{w}{y!(z)}$, we see that the process under the lift
operator may be shaped by prefixed inputs binding a name inside it. In
this sense, the lift operator will be seen as a way to dynamically
construct processes before reifying them as names.

Finally equipped with these standard features we can present the
dynamics of the calculus.

\subsubsection{Operational semantics} 

Finally, we introduce the computational dynamics. What marks these
algebras as distinct from other more traditionally studied algebraic
structures, e.g. vector spaces or polynomial rings, is the manner in
which dynamics is captured. In traditional structures, dynamics is typically
expressed through morphisms between such structures, as in linear maps
between vector spaces or morphisms between rings. In algebras
associated with the semantics of computation, the dynamics is
expressed as part of the algebraic structure itself, through a
reduction reduction relation typically denoted by $\red$. Below, we
give a recursive presentation of this relation for the calculus used
in the encoding.

$\red \subseteq \pi \times \pi$
$\red : \pi \to \mathcal{P}(\pi)$

\begin{mathpar}
  \inferrule* [lab=Comm] { \textsf{match}( x_{src}, x_{trgt} ) } { x_{trgt}?(y)P \; | \; x_{src}!\langle {Q} \rangle \red P\{\quotep{Q}/y}\} }
  \and \\
  \inferrule* [lab=Par] {{P} \red {P}'} {{{P} | {Q}} \red {{P}' | {Q}}}
  \and
  \inferrule* [lab=Equiv]{{{P} \scong {P}'} \andalso {{P}' \red {Q}'} \andalso {{Q}' \scong {Q}}}{{P} \red {Q}}
\end{mathpar}

\begin{eqnarray*}
  match_{\equiv} (\quotep{P},\quotep{Q}) & := & P \equiv Q \\
  match_{\dagger}(\quotep{P},\quotep{Q}) & := & \forall R. P|Q \red^{*} R => R \red^{*} 0 \\
  match_{K}(\quotep{P},\quotep{Q}) & := & K \mbox{ for some context } K
\end{eqnarray*}

$u?(x)P | u!\langle Q \rangle \red P\{\quotep{Q}/x\}$

%We write $\wred$ for $\red^*$, and $P\red$ if $\exists Q $ such that $ P \red Q$.
We write $P\red$ if $\exists Q $ such that $ P \red Q$ and $P\not\red$, otherwise.

\section{Replication}

As mentioned before, it is known that replication (and hence
recursion) can be implemented in a higher-order process algebra
\cite{SangiorgiWalker}. As our first example of calculation with the
machinery thus far presented we give the construction explicitly in
the {\rhoc}.

\begin{eqnarray}
	D_{x} & := & \prefix{x}{y}{(\binpar{\outputp{x}{y}}{@{y}})} \nonumber\\
	\bangp_{x}{P} & := & \binpar{{x}!\langle{\binpar{D_{x}}{P}}\rangle}{D_{x}} \nonumber
\end{eqnarray}

\begin{eqnarray}
	\bangp_{x}{P} & & \nonumber\\
	=
	& {x}!\langle{(\prefix{x}{y}{(\outputp{x}{y} | @{y})) | P}}\rangle 
	      | \prefix{x}{y}{(\outputp{x}{y} | @{y})} & \nonumber\\
	\red
	& (\outputp{x}{y} | @{y})\substn{\quotep{(\prefix{x}{y}{(@{y} | \outputp{x}{y})) | P}}}{y} & \nonumber\\
	=
	& \outputp{x}{\quotep{(\prefix{x}{y}{(\outputp{x}{y} | @{y})) | P}}}
	  | {(\prefix{x}{y}{(\outputp{x}{y} | @{y})) | P}} & \nonumber\\
	\red
	& \ldots & \nonumber\\
	\red^*
	& P | P | \ldots & \nonumber
\end{eqnarray}

Of course, this encoding, as an implementation, runs away, unfolding
$\bangp{P}$ eagerly. A lazier and more implementable replication
operator, restricted to input-guarded processes, may be obtained as follows.

\begin{eqnarray}
\bangp{\prefix{u}{v}{P}} 
	:= 
	\binpar{\lift{x}{\prefix{u}{v}{(\binpar{D(x)}{P})}}}{D(x)} \nonumber
\end{eqnarray}

\begin{remark}
  Note that the lazier definition still does not deal with summation
  or mixed summation (i.e. sums over input and output). The reader is
  invited to construct definitions of replication that deal with these
  features. 

  Further, the definitions are parameterized in a name, $x$. Can you,
  gentle reader, make a definition that eliminates this parameter and
  guarantees no accidental interaction between the replication
  machinery and the process being replicated -- i.e. no accidental
  sharing of names used by the process to get its work done and the
  name(s) used by the replication to effect copying. This latter
  revision of the definition of replication is crucial to obtaining
  the expected identity $!!P \sim !P$.
\end{remark}

\begin{remark}\label{rem:paradoxical_combinator}
  The reader familiar with the lambda calculus will have noticed the
  similarity between $D$ and the paradoxical combinator.

  [Ed. note: the existence of this seems to suggest we have to be more
  restrictive on the set of processes and names we admit if we are to
  support no-cloning.]
\end{remark}

\subsubsection{Bisimulation}

The computational dynamics gives rise to another kind of equivalence,
the equivalence of computational behavior. As previously mentioned
this is typically captured \emph{via} some form of bisimulation.

% The notion we use in this paper is weak barbed bisimulation
% \cite{milner91polyadicpi}.

The notion we use in this paper is derived from weak barbed
bisimulation \cite{milner91polyadicpi}. 

\begin{definition}
An \emph{observation relation}, $\downarrow_{\mathcal N}$, over a set
of names, $\mathcal N$, is the smallest relation satisfying the rules
below.

\infrule[Out-barb]{y \in {\mathcal N}, \; x \nameeq y}
		  {\outputp{x}{v} \downarrow_{\mathcal N} x}
\infrule[Par-barb]{\mbox{$P\downarrow_{\mathcal N} x$ or $Q\downarrow_{\mathcal N} x$}}
		  {\binpar{P}{Q} \downarrow_{\mathcal N} x}

We write $P \Downarrow_{\mathcal N} x$ if there is $Q$ such that 
$P \wred Q$ and $Q \downarrow_{\mathcal N} x$.
\end{definition}

\begin{definition}
%\label{def.bbisim}
An  ${\mathcal N}$-\emph{barbed bisimulation} over a set of names, ${\mathcal N}$, is a symmetric binary relation 
${\mathcal S}_{\mathcal N}$ between agents such that $P\rel{S}_{\mathcal N}Q$ implies:
\begin{enumerate}
\item If $P \red P'$ then $Q \wred Q'$ and $P'\rel{S}_{\mathcal N} Q'$.
\item If $P\downarrow_{\mathcal N} x$, then $Q\Downarrow_{\mathcal N} x$.
\end{enumerate}
$P$ is ${\mathcal N}$-barbed bisimilar to $Q$, written
$P \wbbisim_{\mathcal N} Q$, if $P \rel{S}_{\mathcal N} Q$ for some ${\mathcal N}$-barbed bisimulation ${\mathcal S}_{\mathcal N}$.
\end{definition}

$\mathcal{R} \subseteq \pi \times \pi$

$P \mathcal{R} Q => \forall P'. P \red P' \Rightarrow \exists Q'. Q \red Q', P' \mathcal{R} Q'$

$P \vdash x \Rightarrow Q \vdash x$

\begin{mathpar}
  \inferrule*[lab=Out-barb]{x \nameeq y}{{y}!\langle{Q}\rangle \vdash x}
  \and
  \inferrule*[lab=Par-barb]{\mbox{$P\vdash x$ or $Q\vdash x$}}{\binpar{P}{Q} \vdash x}
\end{mathpar}

\subsubsection{Contexts}

One of the principle advantages of computational calculi like the
$\pi$-calculus is a well-defined notion of context,
contextual-equivalence and a correlation between
contextual-equivalence and notions of bisimulation. The notion of
context allows the decomposition of a process into (sub-)process and
its syntactic environment, its context. Thus, a context may be
thought of as a process with a ``hole'' (written $\Box$) in it. The
application of a context $M$ to a process $P$, written $M[P]$, is
tantamount to filling the hole in $M$ with $P$. In this paper we do
not need the full weight of this theory, but do make use of the notion
of context in the proof the main theorem. 

\begin{mathpar}
  \inferrule* [lab=summation] {} {{M_{M},M_{N}} \bc \Box \;|\; x.M_{A} \;|\; M_{M}+M_{N}}
  \and
  \inferrule* [lab=agent] {} {{M_{A}} \bc (\vec{x})M_{P} \;| \; \clift{P_0,\ldots,M_{P},\ldots,P_N}}
  \and \\
  \inferrule* [lab=process] {} {{M_{P}} \bc M_{N} \;| \;P|M_{P} }
\end{mathpar} 

\begin{mathpar}
  \inferrule* [lab=sychronization] {} {M_{N} \bc \Box \;|\; x?M_{F} \;|\; x!M_{C}}
  \and
  \inferrule* [lab=abstraction] {} {{M_{F}} \bc (x)M_{P} }
  \and
  \inferrule* [lab=concretion] {} {{M_{C}} \bc \langle M_{P} \rangle }
  \and \\
  \inferrule* [lab=process] {} {{M_{P}} \bc M_{N} \;| \;P|M_{P} }
\end{mathpar}

\begin{definition}[contextual application] Given a context $M$, and
  process $P$, we define the \emph{contextual application}, $M[P] :=
  M\{P/\Box\}$. That is, the contextual application of M to P is the
  substitution of $P$ for $\Box$ in $M$.
\end{definition}

$\meaningof{-} : L \to \mathcal{P}(\pi)$

\begin{mathpar}
  \inferrule* [lab=collection] {} {\meaningof{true} = \pi, \and \meaningof{~E} = \pi \setminus \meaningof{E}, \and \meaningof{E_{1} \& E_{2}} = \meaningof{E_{1}} \cap \meaningof{E_{2}}}
\end{mathpar}

\begin{mathpar}
  \inferrule* [lab=structure] {} {\meaningof{0} = \{ P \in \pi | P \equiv 0 \}, \and \\ \meaningof{E_1 | E_2} = \{ P \in \pi | P \equiv P_{1} | P_{2}, P_{1} \in \meaningof{E_{1}}, P_{2} \in \meaningof{E_2}\} }
\end{mathpar}

\begin{mathpar}
 \inferrule* [lab=behavior] {} {\meaningof{\langle a?b \rangle E} = \{ P \in \pi | P \equiv Q | u?(y)P', \\ \and \\\\ \and \\ \;\;\; u \in \meaningof{a}, \forall z.P'\{z/y\} \in \meaningof{E\{z/b\}}\}, \and \\ \meaningof{a!E} = \{ P \in \pi | P \equiv Q | x!\langle P' \rangle, x \in \meaningof{a} P' \in \meaningof{E}\} }
\end{mathpar}

\begin{mathpar}
 \inferrule* [lab=nominal] {} {\meaningof{\quotep{E}} = \{ \quotep{P} \in \quotep{\pi} | P \in \meaningof{E} \}, \and \meaningof{\quotep{P}} = \{ \quotep{Q} \in \quotep{\pi} | P \equiv Q \} \and \\ \meaningof{@\quotep{E}} = \{ P \in \pi | P \equiv @x, x \in \meaningof{E} \}}
\end{mathpar}

\begin{eqnarray*}
  \\
  \meaningof{-} : TS \to ST
\end{eqnarray*}

\begin{eqnarray*}
  \\
  L : TS \to ST
\end{eqnarray*}

\begin{eqnarray*}
  \\
  P \models E \iff P \in \meaningof{E}
\end{eqnarray*}

\begin{eqnarray*}
  P \approx_{L} Q \iff \forall E \in L. P \models E \iff Q \models E
\end{eqnarray*}

\begin{eqnarray*}
  P \approx_{K} Q
\end{eqnarray*}

\begin{eqnarray*}
  P \approx Q
\end{eqnarray*}

$\approx_{K} = \approx = \approx_{L}$

\subsubsection{Contextual duality}

Note that contexts extend the quotation operation to a family of
operations from processes to names. Given a context, $M$, we can
define a \emph{nominal context}, $\quotep{M}$ by $\quotep{M}[P] :=
\quotep{M[P]}$. To foreshadow what is to come we observe that these
operations enjoy a duality with processes very much like the duality
between vectors and maps from vectors to scalars.

Further, because the calculus is essentially higher-order, we have a
correspondence between contexts and processes. More specifically,
given a name $x$ and a context $M$ we can construct $M^{*}_{x}$ such
that 

\begin{mathpar}
  M^{*}_{x} | \lift{x}{P} \red M[P]
\end{mathpar}

namely,

\begin{mathpar}
  M^{*}_{x} := x?(u).M[\dropn{u}]
\end{mathpar}

The dependence of $M^{*}_{x}$ on a name makes it an abstraction, 

\begin{mathpar}
  M^{*} := (x)x?(u).M[\dropn{u}]
\end{mathpar}

\subsection{Additional notation}

It will sometimes be convenient to denote the process a name
quotes. We already have the notation $x = \quotep{P}$, but it will be
convenient to introduce an alternate notation, $\procn{x}$, when we
want to emphasize the connection to the use of the name. Note that, by
virtue of name equivalence, $\quotep{\procn{x}} \nameeq x$; so, the
notation is consistent with previous definitions.

Further, because names have structure it is possible to effect
substitutions on the basis of that structure. This means we need to
upgrade our notation for substitutions, which we accomplish by
adapting comprehension notation. Thus,

\begin{mathpar}
  P\{ y / x : x \in S \}
\end{mathpar}

is interpreted to mean the process derived from P by replacing (in a
capture-avoiding manner) each occurrence of $x$ in $S$ by $y$. For example,

\begin{mathpar}
  P\{ \quotep{\procn{x}|\procn{x}} / x : x \in \freenames{P} \}
\end{mathpar}

will replace each (occurrence) of a free name $x$ in $P$ by
$\quotep{\procn{x}|\procn{x}}$.

Also, we will avail ourselves of the notation $x^{L}$ and $x^{R}$ to
denote injections of a name into disjoint copies of the name
space. There are numerous ways to accomplish this. One example can be
found in \cite{MeredithR05}. This notation overloads to vectors of
names: $\vec{x}^{\pi} := (x_{i}^{\pi} \; : \; 0 \leq i < |\vec{x}| )$ where $\pi \in \{L,R\}$.

We also use $P^{\Box} := P|\Box$.

In \cite{MeredithR05} an interpretation of the new operator is
given. It turns out that there are several possible interpretations
all enjoying the requisite algebraic properties of the operator (see
\cite{milner91polyadicpi}). We will therefore make liberal use of
$(\nu\; \vec{x})P$.

% subsection the_syntax_and_semantics_of_the_notation_system (end)   

\input{qm2pi.qmops} 

\input{qm2pi.sterngerlach} 

\input{qm2pi.metric} 

% section concurrent_process_calculi (end)

%\input{qm2pi.proofsketch}

% section proof sketch (end)

%\input{qm2pi.slviaknots} 

% section spatial logic via knots (end)

\input{qm2pi.conclusion}

% section conclusion (end)

%\input{qm2pi.dtcodes} 

% section wiring algorithm (end)

\input{qm2pi.ack} 

% section acknowledgments (end)

\newpage


\bibliographystyle{plain}   
\bibliography{../../biblios/main.bib}

\input{qm2pi.rhodetails}

\end{document}

 

% section notation (end)

\input{qm2pi.process.calculi} 

% section concurrent_process_calculi_and_spatial_logics_ (end)
    
%\documentclass[12pt]{llncs}
%\documentclass{jktr}

\usepackage[pdftex]{hyperref}                   
\usepackage {listings}
\usepackage {mathpartir}
\usepackage{bcprules}
%\usepackage{listings}
                       
\usepackage{graphicx} 
%\usepackage[margins=2.5cm,nohead,nofoot]{geometry}
%\usepackage{geometry}
\usepackage{amsfonts}
\usepackage{amstext}
\usepackage{latexsym}
\usepackage{amssymb}
\usepackage{color}


%\include{myPreamble}
\include{qm2pi.local} 

%\ifpdf
%\usepackage[pdftex]{graphicx}
%\else
%\usepackage{graphicx}
%\fi

 % \ifpdf
%  \usepackage{pdfsync}
%  \if


%\title{Brief Article}
%\author{David F. Snyder}
%\author{L.G. Meredith}

%\address{Dept. of Math., Texas State University--San Marcos, San Marcos, TX 78666}
       
\pagestyle{empty}


\begin{document}

\lstset{language=[Objective]Caml,frame=shadowbox}

\input{qm2pi.front}

% section front matter (end)

\input{qm2pi.intro} 
 
% section introduction (end)

% \input{qm2pi.knotations} 

% section notation (end)

\input{qm2pi.process.calculi} 

% section concurrent_process_calculi_and_spatial_logics_ (end)
    
%\input{qm2pi.knots2pi} 

%\input{qm2pi.trefoil} 

%\input{qm2pi.mainthm} 

% subsection basic_interpretation (end)

%\input{qm2pi.rho.presentation} 
\subsection{The syntax and semantics of the notation system}\label{sub:the_syntax_and_semantics_of_the_notation_system} % (fold)

We now summarize a technical presentation of the calculus that
embodies our theory of dynamics. The typical presentation of such a
calculus follows the style of giving generators and relations on
them. The grammar, below, describing term constructors, freely
generates the set of processes, $\Proc$. This set is then quotiented
by a relation known as structural congruence and it is over this set
that the notion of dynamics is expressed. This presentation is
essentially that of \cite{MeredithR05} with the addition of
polyadicity and summation. For readability we have relegated some of
the technical subtleties to an appendix.

\subsubsection{Process grammar}\label{subsub:process_grammar}

\begin{mathpar}
  \inferrule* [lab=synchronization] {} {{M} \bc \pzero \;|\; x?F \;|\; x!C }
  \and
  \inferrule* [lab=abstraction] {} {{F} \bc (x)P}
  \and
  \inferrule* [lab=concretion] {} {{C} \bc \langle Q \rangle}
  \and
  \inferrule* [lab=process] {} {{P,Q} \bc M \;| \;P|Q \;|\; @{x}}
  \and
  \inferrule* [lab=name] {} {{x} \bc \quotep{P}}
\end{mathpar} 

Note that $\vec{x}$ (resp. $\vec{P}$) denotes a vector of names
(resp. processes) of length $|\vec{x}|$ (resp. $|\vec{P}|$). We adopt
the following useful abbreviations.

\begin{mathpar}
   x?(\vec{y}).P := x.(\vec{y})P \and  x\clift{\vec{P}} := x.\clift{\vec{P}}
   \and x!(y) := \lift{x}{\dropn{y}}
   \and \Pi_{i=0}^{n-1}P_i := P_0 | \ldots | P_{n-1}
\end{mathpar}

\subsubsection{Structural congruence}

\paragraph{Free and bound names and alpha-equivalence.} At the
core of structural equivalence is alpha-equivalence which identifies
process that are the same up to a change of variable. Formally, we
recognize the distinction between free and bound names. The free names
of a process, $\freenames{P}$, may be calculated recursively as
follows:

\begin{mathpar}
\freenames{\pzero} := \emptyset
  \and \\
  \freenames{x?(y).P} := \{ x \} \cup (\freenames{P} \setminus \{ y \})
  \and 
  \freenames{x!\langle P \rangle} := \{ x \} \cup \{ P \} 
  \and \\
  \freenames{P|Q} := \freenames{P} \cup \freenames{Q}
  \and \\
  \freenames{@{x}} := \{ x \}
\end{mathpar}

$\pi$
$\quotep{\pi}$

$\freenames{-} : \pi \to \mathcal{P}(\quotep{\pi})$

\begin{eqnarray*}
  \freenames{\pzero} & := & \emptyset \\
  \freenames{x?(y).P} & := & \{ x \} \cup (\freenames{P} \setminus \{ y \}) \\
  \freenames{x!\langle P \rangle} & := & \{ x \} \cup \{ P \} \\
  \freenames{P|Q} & := & \freenames{P} \cup \freenames{Q} \\
  \freenames{\dropn{x}} & := & \{ x \}
\end{eqnarray*}

The bound names of a process, $\boundnames{P}$, are those names occurring in $P$
that are not free. For example, in $x?(y).0$, the name $x$ is free, while $y$ is bound.

\begin{mathpar}
  \inferrule* [lab=monoidal-laws] {} { P|Q \equiv Q|P \and P|0 \equiv P \and P|(Q|R) \equiv (P|Q)|R }
\end{mathpar}

\begin{mathpar}
  \inferrule* [lab=alpha-equivalence] {} { (x)P \equiv (y)P\{y/x\} \and y \not\in \freenames{P} }
\end{mathpar}

\begin{definition}
Then two processes, $P,Q$, are alpha-equivalent if $P = Q\{\vec{y}/\vec{x}\}$ for
some $\vec{x} \in \boundnames{Q},\vec{y} \in \boundnames{P}$, where $Q\{\vec{y}/\vec{x}\}$
denotes the capture-avoiding substitution of $\vec{y}$ for $\vec{x}$ in $Q$.
\end{definition}

\begin{definition}
  The {\em structural congruence} \cite{SangiorgiWalker} , $\equiv$,
  between processes is the least congruence containing
  alpha-equivalence, satisfying the abelian monoid laws
  (associativity, commutativity and $\pzero$ as identity) for parallel
  composition $|$ and for summation $+$.
\end{definition}

\subsection{Name equivalence}

We take name equivalence, written $\nameeq$, to be the smallest
equivalence relation generated by the following rules.

\begin{mathpar}
\inferrule*[lab=Quote-drop]
{ }
{ \quotep{@{x}} \nameeq x }

\inferrule*[lab=Struct-equiv]
{ P \scong Q }
{ \quotep{P} \nameeq \quotep{Q} }
\end{mathpar}

The astute reader will have noticed that the mutual recursion of names
and processes imposes a mutual recursion on alpha-equivalence and
structural equivalence via name-equivalence. Fortunately, all of this
works out pleasantly and we may calculate in the natural way, free of
concern. The reader interested in the details is referred to the
appendix \ref{appendix:rho_details}.

\subsection{Substitution}

We use $\Proc$ for the set of processes, $\QProc$ for the set of
names, and $\id{\{}\vec{y} / \vec{x} \id{\}}$ to denote partial maps,
$s : \QProc \rightarrow \QProc$. A map, $s$ lifts, uniquely, to a map
on process terms, $\widehat{s} : \Proc \rightarrow \Proc$ by the
following equations.

\begin{mathpar}
  (0) \psubstp{Q}{P} := 0 \\
  (R \juxtap S) \psubstp{Q}{P}
  :=    
  (R)\psubstp{Q}{P} \juxtap (S) \psubstp{Q}{P} \\
  (x?(y).R) \psubstp{Q}{P}    
  :=    
  (x)\substp{Q}{P} (z)\concat( (R \psubstn{z}{y}) \psubstp{Q}{P} ) \\
  (\lift{x}{R}) \psubstp{Q}{P}  
  :=
  \lift{(x)\substp{Q}{P}}{ R \psubstp{Q}{P} } \\
%   (\dropn{x})  \psubstp{Q}{P}       
%   := 
%   \left\{ 
%     \begin{array}{ccc} 
%       \dropn{\quotep{Q}} & & x \nameeq \quotep{P} \\
%       \dropn{x} & & otherwise \\
%     \end{array}
%   \right. 
  (\dropn{x})  \psubstp{Q}{P}       
  := 
  \left\{ 
    \begin{array}{ccc} 
      Q & & x \nameeq \quotep{P} \\
      \dropn{x} & & otherwise \\
    \end{array}
  \right.
\end{mathpar}
 

where

\begin{eqnarray}
  (x)\id{\{} \lpquote Q \rpquote / \lpquote P \rpquote \id{\}}            = 
  \left\{ 
    \begin{array}{ccc}
      \lpquote Q \rpquote & & x \nameeq \lpquote P \rpquote \\
      x & & otherwise \\
    \end{array}
  \right. \nonumber
\end{eqnarray}

and $z$ is chosen distinct from $\quotep{P}$, $\quotep{Q}$, the free
names in $Q$, and all the names in $R$. Our $\alpha$-equivalence will
be built in the standard way from this substitution.

\begin{remark}\label{rem:no_self_referential_names}
  One consequence of these definitions is that $\forall P. \quotep{P}
  \not\in \freenames{P}$.
\end{remark}

\subsection{ Dynamic quote: an example }

Anticipating something of what's to come, consider applying the
substitution, $\widehat{\id{\{}u / z \id{\}}}$, to the following pair
of processes, $\lift{w}{y!(z)}$ and $w[ \lpquote y!(z) \rpquote ]$.

\begin{eqnarray}
	\lift{w}{y!(z)}\widehat{\id{\{}u / z \id{\}}}
		& = &
		\lift{w}{y!(u)} \nonumber\\
	w[ \lpquote y!(z) \rpquote ] \widehat{ \id{\{}u / z \id{\}} }
		& = &
		w[ \lpquote y!(z) \rpquote ] \nonumber
\end{eqnarray}

Because the body of the process between quotes is impervious to
substitution, we get radically different answers. In fact, by
examining the first process in an input context,
e.g. $x?(z).\lift{w}{y!(z)}$, we see that the process under the lift
operator may be shaped by prefixed inputs binding a name inside it. In
this sense, the lift operator will be seen as a way to dynamically
construct processes before reifying them as names.

Finally equipped with these standard features we can present the
dynamics of the calculus.

\subsubsection{Operational semantics} 

Finally, we introduce the computational dynamics. What marks these
algebras as distinct from other more traditionally studied algebraic
structures, e.g. vector spaces or polynomial rings, is the manner in
which dynamics is captured. In traditional structures, dynamics is typically
expressed through morphisms between such structures, as in linear maps
between vector spaces or morphisms between rings. In algebras
associated with the semantics of computation, the dynamics is
expressed as part of the algebraic structure itself, through a
reduction reduction relation typically denoted by $\red$. Below, we
give a recursive presentation of this relation for the calculus used
in the encoding.

$\red \subseteq \pi \times \pi$
$\red : \pi \to \mathcal{P}(\pi)$

\begin{mathpar}
  \inferrule* [lab=Comm] { \textsf{match}( x_{src}, x_{trgt} ) } { x_{trgt}?(y)P \; | \; x_{src}!\langle {Q} \rangle \red P\{\quotep{Q}/y}\} }
  \and \\
  \inferrule* [lab=Par] {{P} \red {P}'} {{{P} | {Q}} \red {{P}' | {Q}}}
  \and
  \inferrule* [lab=Equiv]{{{P} \scong {P}'} \andalso {{P}' \red {Q}'} \andalso {{Q}' \scong {Q}}}{{P} \red {Q}}
\end{mathpar}

\begin{eqnarray*}
  match_{\equiv} (\quotep{P},\quotep{Q}) & := & P \equiv Q \\
  match_{\dagger}(\quotep{P},\quotep{Q}) & := & \forall R. P|Q \red^{*} R => R \red^{*} 0 \\
  match_{K}(\quotep{P},\quotep{Q}) & := & K \mbox{ for some context } K
\end{eqnarray*}

$u?(x)P | u!\langle Q \rangle \red P\{\quotep{Q}/x\}$

%We write $\wred$ for $\red^*$, and $P\red$ if $\exists Q $ such that $ P \red Q$.
We write $P\red$ if $\exists Q $ such that $ P \red Q$ and $P\not\red$, otherwise.

\section{Replication}

As mentioned before, it is known that replication (and hence
recursion) can be implemented in a higher-order process algebra
\cite{SangiorgiWalker}. As our first example of calculation with the
machinery thus far presented we give the construction explicitly in
the {\rhoc}.

\begin{eqnarray}
	D_{x} & := & \prefix{x}{y}{(\binpar{\outputp{x}{y}}{@{y}})} \nonumber\\
	\bangp_{x}{P} & := & \binpar{{x}!\langle{\binpar{D_{x}}{P}}\rangle}{D_{x}} \nonumber
\end{eqnarray}

\begin{eqnarray}
	\bangp_{x}{P} & & \nonumber\\
	=
	& {x}!\langle{(\prefix{x}{y}{(\outputp{x}{y} | @{y})) | P}}\rangle 
	      | \prefix{x}{y}{(\outputp{x}{y} | @{y})} & \nonumber\\
	\red
	& (\outputp{x}{y} | @{y})\substn{\quotep{(\prefix{x}{y}{(@{y} | \outputp{x}{y})) | P}}}{y} & \nonumber\\
	=
	& \outputp{x}{\quotep{(\prefix{x}{y}{(\outputp{x}{y} | @{y})) | P}}}
	  | {(\prefix{x}{y}{(\outputp{x}{y} | @{y})) | P}} & \nonumber\\
	\red
	& \ldots & \nonumber\\
	\red^*
	& P | P | \ldots & \nonumber
\end{eqnarray}

Of course, this encoding, as an implementation, runs away, unfolding
$\bangp{P}$ eagerly. A lazier and more implementable replication
operator, restricted to input-guarded processes, may be obtained as follows.

\begin{eqnarray}
\bangp{\prefix{u}{v}{P}} 
	:= 
	\binpar{\lift{x}{\prefix{u}{v}{(\binpar{D(x)}{P})}}}{D(x)} \nonumber
\end{eqnarray}

\begin{remark}
  Note that the lazier definition still does not deal with summation
  or mixed summation (i.e. sums over input and output). The reader is
  invited to construct definitions of replication that deal with these
  features. 

  Further, the definitions are parameterized in a name, $x$. Can you,
  gentle reader, make a definition that eliminates this parameter and
  guarantees no accidental interaction between the replication
  machinery and the process being replicated -- i.e. no accidental
  sharing of names used by the process to get its work done and the
  name(s) used by the replication to effect copying. This latter
  revision of the definition of replication is crucial to obtaining
  the expected identity $!!P \sim !P$.
\end{remark}

\begin{remark}\label{rem:paradoxical_combinator}
  The reader familiar with the lambda calculus will have noticed the
  similarity between $D$ and the paradoxical combinator.

  [Ed. note: the existence of this seems to suggest we have to be more
  restrictive on the set of processes and names we admit if we are to
  support no-cloning.]
\end{remark}

\subsubsection{Bisimulation}

The computational dynamics gives rise to another kind of equivalence,
the equivalence of computational behavior. As previously mentioned
this is typically captured \emph{via} some form of bisimulation.

% The notion we use in this paper is weak barbed bisimulation
% \cite{milner91polyadicpi}.

The notion we use in this paper is derived from weak barbed
bisimulation \cite{milner91polyadicpi}. 

\begin{definition}
An \emph{observation relation}, $\downarrow_{\mathcal N}$, over a set
of names, $\mathcal N$, is the smallest relation satisfying the rules
below.

\infrule[Out-barb]{y \in {\mathcal N}, \; x \nameeq y}
		  {\outputp{x}{v} \downarrow_{\mathcal N} x}
\infrule[Par-barb]{\mbox{$P\downarrow_{\mathcal N} x$ or $Q\downarrow_{\mathcal N} x$}}
		  {\binpar{P}{Q} \downarrow_{\mathcal N} x}

We write $P \Downarrow_{\mathcal N} x$ if there is $Q$ such that 
$P \wred Q$ and $Q \downarrow_{\mathcal N} x$.
\end{definition}

\begin{definition}
%\label{def.bbisim}
An  ${\mathcal N}$-\emph{barbed bisimulation} over a set of names, ${\mathcal N}$, is a symmetric binary relation 
${\mathcal S}_{\mathcal N}$ between agents such that $P\rel{S}_{\mathcal N}Q$ implies:
\begin{enumerate}
\item If $P \red P'$ then $Q \wred Q'$ and $P'\rel{S}_{\mathcal N} Q'$.
\item If $P\downarrow_{\mathcal N} x$, then $Q\Downarrow_{\mathcal N} x$.
\end{enumerate}
$P$ is ${\mathcal N}$-barbed bisimilar to $Q$, written
$P \wbbisim_{\mathcal N} Q$, if $P \rel{S}_{\mathcal N} Q$ for some ${\mathcal N}$-barbed bisimulation ${\mathcal S}_{\mathcal N}$.
\end{definition}

$\mathcal{R} \subseteq \pi \times \pi$

$P \mathcal{R} Q => \forall P'. P \red P' \Rightarrow \exists Q'. Q \red Q', P' \mathcal{R} Q'$

$P \vdash x \Rightarrow Q \vdash x$

\begin{mathpar}
  \inferrule*[lab=Out-barb]{x \nameeq y}{{y}!\langle{Q}\rangle \vdash x}
  \and
  \inferrule*[lab=Par-barb]{\mbox{$P\vdash x$ or $Q\vdash x$}}{\binpar{P}{Q} \vdash x}
\end{mathpar}

\subsubsection{Contexts}

One of the principle advantages of computational calculi like the
$\pi$-calculus is a well-defined notion of context,
contextual-equivalence and a correlation between
contextual-equivalence and notions of bisimulation. The notion of
context allows the decomposition of a process into (sub-)process and
its syntactic environment, its context. Thus, a context may be
thought of as a process with a ``hole'' (written $\Box$) in it. The
application of a context $M$ to a process $P$, written $M[P]$, is
tantamount to filling the hole in $M$ with $P$. In this paper we do
not need the full weight of this theory, but do make use of the notion
of context in the proof the main theorem. 

\begin{mathpar}
  \inferrule* [lab=summation] {} {{M_{M},M_{N}} \bc \Box \;|\; x.M_{A} \;|\; M_{M}+M_{N}}
  \and
  \inferrule* [lab=agent] {} {{M_{A}} \bc (\vec{x})M_{P} \;| \; \clift{P_0,\ldots,M_{P},\ldots,P_N}}
  \and \\
  \inferrule* [lab=process] {} {{M_{P}} \bc M_{N} \;| \;P|M_{P} }
\end{mathpar} 

\begin{mathpar}
  \inferrule* [lab=sychronization] {} {M_{N} \bc \Box \;|\; x?M_{F} \;|\; x!M_{C}}
  \and
  \inferrule* [lab=abstraction] {} {{M_{F}} \bc (x)M_{P} }
  \and
  \inferrule* [lab=concretion] {} {{M_{C}} \bc \langle M_{P} \rangle }
  \and \\
  \inferrule* [lab=process] {} {{M_{P}} \bc M_{N} \;| \;P|M_{P} }
\end{mathpar}

\begin{definition}[contextual application] Given a context $M$, and
  process $P$, we define the \emph{contextual application}, $M[P] :=
  M\{P/\Box\}$. That is, the contextual application of M to P is the
  substitution of $P$ for $\Box$ in $M$.
\end{definition}

$\meaningof{-} : L \to \mathcal{P}(\pi)$

\begin{mathpar}
  \inferrule* [lab=collection] {} {\meaningof{true} = \pi, \and \meaningof{~E} = \pi \setminus \meaningof{E}, \and \meaningof{E_{1} \& E_{2}} = \meaningof{E_{1}} \cap \meaningof{E_{2}}}
\end{mathpar}

\begin{mathpar}
  \inferrule* [lab=structure] {} {\meaningof{0} = \{ P \in \pi | P \equiv 0 \}, \and \\ \meaningof{E_1 | E_2} = \{ P \in \pi | P \equiv P_{1} | P_{2}, P_{1} \in \meaningof{E_{1}}, P_{2} \in \meaningof{E_2}\} }
\end{mathpar}

\begin{mathpar}
 \inferrule* [lab=behavior] {} {\meaningof{\langle a?b \rangle E} = \{ P \in \pi | P \equiv Q | u?(y)P', \\ \and \\\\ \and \\ \;\;\; u \in \meaningof{a}, \forall z.P'\{z/y\} \in \meaningof{E\{z/b\}}\}, \and \\ \meaningof{a!E} = \{ P \in \pi | P \equiv Q | x!\langle P' \rangle, x \in \meaningof{a} P' \in \meaningof{E}\} }
\end{mathpar}

\begin{mathpar}
 \inferrule* [lab=nominal] {} {\meaningof{\quotep{E}} = \{ \quotep{P} \in \quotep{\pi} | P \in \meaningof{E} \}, \and \meaningof{\quotep{P}} = \{ \quotep{Q} \in \quotep{\pi} | P \equiv Q \} \and \\ \meaningof{@\quotep{E}} = \{ P \in \pi | P \equiv @x, x \in \meaningof{E} \}}
\end{mathpar}

\begin{eqnarray*}
  \\
  \meaningof{-} : TS \to ST
\end{eqnarray*}

\begin{eqnarray*}
  \\
  L : TS \to ST
\end{eqnarray*}

\begin{eqnarray*}
  \\
  P \models E \iff P \in \meaningof{E}
\end{eqnarray*}

\begin{eqnarray*}
  P \approx_{L} Q \iff \forall E \in L. P \models E \iff Q \models E
\end{eqnarray*}

\begin{eqnarray*}
  P \approx_{K} Q
\end{eqnarray*}

\begin{eqnarray*}
  P \approx Q
\end{eqnarray*}

$\approx_{K} = \approx = \approx_{L}$

\subsubsection{Contextual duality}

Note that contexts extend the quotation operation to a family of
operations from processes to names. Given a context, $M$, we can
define a \emph{nominal context}, $\quotep{M}$ by $\quotep{M}[P] :=
\quotep{M[P]}$. To foreshadow what is to come we observe that these
operations enjoy a duality with processes very much like the duality
between vectors and maps from vectors to scalars.

Further, because the calculus is essentially higher-order, we have a
correspondence between contexts and processes. More specifically,
given a name $x$ and a context $M$ we can construct $M^{*}_{x}$ such
that 

\begin{mathpar}
  M^{*}_{x} | \lift{x}{P} \red M[P]
\end{mathpar}

namely,

\begin{mathpar}
  M^{*}_{x} := x?(u).M[\dropn{u}]
\end{mathpar}

The dependence of $M^{*}_{x}$ on a name makes it an abstraction, 

\begin{mathpar}
  M^{*} := (x)x?(u).M[\dropn{u}]
\end{mathpar}

\subsection{Additional notation}

It will sometimes be convenient to denote the process a name
quotes. We already have the notation $x = \quotep{P}$, but it will be
convenient to introduce an alternate notation, $\procn{x}$, when we
want to emphasize the connection to the use of the name. Note that, by
virtue of name equivalence, $\quotep{\procn{x}} \nameeq x$; so, the
notation is consistent with previous definitions.

Further, because names have structure it is possible to effect
substitutions on the basis of that structure. This means we need to
upgrade our notation for substitutions, which we accomplish by
adapting comprehension notation. Thus,

\begin{mathpar}
  P\{ y / x : x \in S \}
\end{mathpar}

is interpreted to mean the process derived from P by replacing (in a
capture-avoiding manner) each occurrence of $x$ in $S$ by $y$. For example,

\begin{mathpar}
  P\{ \quotep{\procn{x}|\procn{x}} / x : x \in \freenames{P} \}
\end{mathpar}

will replace each (occurrence) of a free name $x$ in $P$ by
$\quotep{\procn{x}|\procn{x}}$.

Also, we will avail ourselves of the notation $x^{L}$ and $x^{R}$ to
denote injections of a name into disjoint copies of the name
space. There are numerous ways to accomplish this. One example can be
found in \cite{MeredithR05}. This notation overloads to vectors of
names: $\vec{x}^{\pi} := (x_{i}^{\pi} \; : \; 0 \leq i < |\vec{x}| )$ where $\pi \in \{L,R\}$.

We also use $P^{\Box} := P|\Box$.

In \cite{MeredithR05} an interpretation of the new operator is
given. It turns out that there are several possible interpretations
all enjoying the requisite algebraic properties of the operator (see
\cite{milner91polyadicpi}). We will therefore make liberal use of
$(\nu\; \vec{x})P$.

% subsection the_syntax_and_semantics_of_the_notation_system (end)   

\input{qm2pi.qmops} 

\input{qm2pi.sterngerlach} 

\input{qm2pi.metric} 

% section concurrent_process_calculi (end)

%\input{qm2pi.proofsketch}

% section proof sketch (end)

%\input{qm2pi.slviaknots} 

% section spatial logic via knots (end)

\input{qm2pi.conclusion}

% section conclusion (end)

%\input{qm2pi.dtcodes} 

% section wiring algorithm (end)

\input{qm2pi.ack} 

% section acknowledgments (end)

\newpage


\bibliographystyle{plain}   
\bibliography{../../biblios/main.bib}

\input{qm2pi.rhodetails}

\end{document}

 

%\documentclass[12pt]{llncs}
%\documentclass{jktr}

\usepackage[pdftex]{hyperref}                   
\usepackage {listings}
\usepackage {mathpartir}
\usepackage{bcprules}
%\usepackage{listings}
                       
\usepackage{graphicx} 
%\usepackage[margins=2.5cm,nohead,nofoot]{geometry}
%\usepackage{geometry}
\usepackage{amsfonts}
\usepackage{amstext}
\usepackage{latexsym}
\usepackage{amssymb}
\usepackage{color}


%\include{myPreamble}
\include{qm2pi.local} 

%\ifpdf
%\usepackage[pdftex]{graphicx}
%\else
%\usepackage{graphicx}
%\fi

 % \ifpdf
%  \usepackage{pdfsync}
%  \if


%\title{Brief Article}
%\author{David F. Snyder}
%\author{L.G. Meredith}

%\address{Dept. of Math., Texas State University--San Marcos, San Marcos, TX 78666}
       
\pagestyle{empty}


\begin{document}

\lstset{language=[Objective]Caml,frame=shadowbox}

\input{qm2pi.front}

% section front matter (end)

\input{qm2pi.intro} 
 
% section introduction (end)

% \input{qm2pi.knotations} 

% section notation (end)

\input{qm2pi.process.calculi} 

% section concurrent_process_calculi_and_spatial_logics_ (end)
    
%\input{qm2pi.knots2pi} 

%\input{qm2pi.trefoil} 

%\input{qm2pi.mainthm} 

% subsection basic_interpretation (end)

%\input{qm2pi.rho.presentation} 
\subsection{The syntax and semantics of the notation system}\label{sub:the_syntax_and_semantics_of_the_notation_system} % (fold)

We now summarize a technical presentation of the calculus that
embodies our theory of dynamics. The typical presentation of such a
calculus follows the style of giving generators and relations on
them. The grammar, below, describing term constructors, freely
generates the set of processes, $\Proc$. This set is then quotiented
by a relation known as structural congruence and it is over this set
that the notion of dynamics is expressed. This presentation is
essentially that of \cite{MeredithR05} with the addition of
polyadicity and summation. For readability we have relegated some of
the technical subtleties to an appendix.

\subsubsection{Process grammar}\label{subsub:process_grammar}

\begin{mathpar}
  \inferrule* [lab=synchronization] {} {{M} \bc \pzero \;|\; x?F \;|\; x!C }
  \and
  \inferrule* [lab=abstraction] {} {{F} \bc (x)P}
  \and
  \inferrule* [lab=concretion] {} {{C} \bc \langle Q \rangle}
  \and
  \inferrule* [lab=process] {} {{P,Q} \bc M \;| \;P|Q \;|\; @{x}}
  \and
  \inferrule* [lab=name] {} {{x} \bc \quotep{P}}
\end{mathpar} 

Note that $\vec{x}$ (resp. $\vec{P}$) denotes a vector of names
(resp. processes) of length $|\vec{x}|$ (resp. $|\vec{P}|$). We adopt
the following useful abbreviations.

\begin{mathpar}
   x?(\vec{y}).P := x.(\vec{y})P \and  x\clift{\vec{P}} := x.\clift{\vec{P}}
   \and x!(y) := \lift{x}{\dropn{y}}
   \and \Pi_{i=0}^{n-1}P_i := P_0 | \ldots | P_{n-1}
\end{mathpar}

\subsubsection{Structural congruence}

\paragraph{Free and bound names and alpha-equivalence.} At the
core of structural equivalence is alpha-equivalence which identifies
process that are the same up to a change of variable. Formally, we
recognize the distinction between free and bound names. The free names
of a process, $\freenames{P}$, may be calculated recursively as
follows:

\begin{mathpar}
\freenames{\pzero} := \emptyset
  \and \\
  \freenames{x?(y).P} := \{ x \} \cup (\freenames{P} \setminus \{ y \})
  \and 
  \freenames{x!\langle P \rangle} := \{ x \} \cup \{ P \} 
  \and \\
  \freenames{P|Q} := \freenames{P} \cup \freenames{Q}
  \and \\
  \freenames{@{x}} := \{ x \}
\end{mathpar}

$\pi$
$\quotep{\pi}$

$\freenames{-} : \pi \to \mathcal{P}(\quotep{\pi})$

\begin{eqnarray*}
  \freenames{\pzero} & := & \emptyset \\
  \freenames{x?(y).P} & := & \{ x \} \cup (\freenames{P} \setminus \{ y \}) \\
  \freenames{x!\langle P \rangle} & := & \{ x \} \cup \{ P \} \\
  \freenames{P|Q} & := & \freenames{P} \cup \freenames{Q} \\
  \freenames{\dropn{x}} & := & \{ x \}
\end{eqnarray*}

The bound names of a process, $\boundnames{P}$, are those names occurring in $P$
that are not free. For example, in $x?(y).0$, the name $x$ is free, while $y$ is bound.

\begin{mathpar}
  \inferrule* [lab=monoidal-laws] {} { P|Q \equiv Q|P \and P|0 \equiv P \and P|(Q|R) \equiv (P|Q)|R }
\end{mathpar}

\begin{mathpar}
  \inferrule* [lab=alpha-equivalence] {} { (x)P \equiv (y)P\{y/x\} \and y \not\in \freenames{P} }
\end{mathpar}

\begin{definition}
Then two processes, $P,Q$, are alpha-equivalent if $P = Q\{\vec{y}/\vec{x}\}$ for
some $\vec{x} \in \boundnames{Q},\vec{y} \in \boundnames{P}$, where $Q\{\vec{y}/\vec{x}\}$
denotes the capture-avoiding substitution of $\vec{y}$ for $\vec{x}$ in $Q$.
\end{definition}

\begin{definition}
  The {\em structural congruence} \cite{SangiorgiWalker} , $\equiv$,
  between processes is the least congruence containing
  alpha-equivalence, satisfying the abelian monoid laws
  (associativity, commutativity and $\pzero$ as identity) for parallel
  composition $|$ and for summation $+$.
\end{definition}

\subsection{Name equivalence}

We take name equivalence, written $\nameeq$, to be the smallest
equivalence relation generated by the following rules.

\begin{mathpar}
\inferrule*[lab=Quote-drop]
{ }
{ \quotep{@{x}} \nameeq x }

\inferrule*[lab=Struct-equiv]
{ P \scong Q }
{ \quotep{P} \nameeq \quotep{Q} }
\end{mathpar}

The astute reader will have noticed that the mutual recursion of names
and processes imposes a mutual recursion on alpha-equivalence and
structural equivalence via name-equivalence. Fortunately, all of this
works out pleasantly and we may calculate in the natural way, free of
concern. The reader interested in the details is referred to the
appendix \ref{appendix:rho_details}.

\subsection{Substitution}

We use $\Proc$ for the set of processes, $\QProc$ for the set of
names, and $\id{\{}\vec{y} / \vec{x} \id{\}}$ to denote partial maps,
$s : \QProc \rightarrow \QProc$. A map, $s$ lifts, uniquely, to a map
on process terms, $\widehat{s} : \Proc \rightarrow \Proc$ by the
following equations.

\begin{mathpar}
  (0) \psubstp{Q}{P} := 0 \\
  (R \juxtap S) \psubstp{Q}{P}
  :=    
  (R)\psubstp{Q}{P} \juxtap (S) \psubstp{Q}{P} \\
  (x?(y).R) \psubstp{Q}{P}    
  :=    
  (x)\substp{Q}{P} (z)\concat( (R \psubstn{z}{y}) \psubstp{Q}{P} ) \\
  (\lift{x}{R}) \psubstp{Q}{P}  
  :=
  \lift{(x)\substp{Q}{P}}{ R \psubstp{Q}{P} } \\
%   (\dropn{x})  \psubstp{Q}{P}       
%   := 
%   \left\{ 
%     \begin{array}{ccc} 
%       \dropn{\quotep{Q}} & & x \nameeq \quotep{P} \\
%       \dropn{x} & & otherwise \\
%     \end{array}
%   \right. 
  (\dropn{x})  \psubstp{Q}{P}       
  := 
  \left\{ 
    \begin{array}{ccc} 
      Q & & x \nameeq \quotep{P} \\
      \dropn{x} & & otherwise \\
    \end{array}
  \right.
\end{mathpar}
 

where

\begin{eqnarray}
  (x)\id{\{} \lpquote Q \rpquote / \lpquote P \rpquote \id{\}}            = 
  \left\{ 
    \begin{array}{ccc}
      \lpquote Q \rpquote & & x \nameeq \lpquote P \rpquote \\
      x & & otherwise \\
    \end{array}
  \right. \nonumber
\end{eqnarray}

and $z$ is chosen distinct from $\quotep{P}$, $\quotep{Q}$, the free
names in $Q$, and all the names in $R$. Our $\alpha$-equivalence will
be built in the standard way from this substitution.

\begin{remark}\label{rem:no_self_referential_names}
  One consequence of these definitions is that $\forall P. \quotep{P}
  \not\in \freenames{P}$.
\end{remark}

\subsection{ Dynamic quote: an example }

Anticipating something of what's to come, consider applying the
substitution, $\widehat{\id{\{}u / z \id{\}}}$, to the following pair
of processes, $\lift{w}{y!(z)}$ and $w[ \lpquote y!(z) \rpquote ]$.

\begin{eqnarray}
	\lift{w}{y!(z)}\widehat{\id{\{}u / z \id{\}}}
		& = &
		\lift{w}{y!(u)} \nonumber\\
	w[ \lpquote y!(z) \rpquote ] \widehat{ \id{\{}u / z \id{\}} }
		& = &
		w[ \lpquote y!(z) \rpquote ] \nonumber
\end{eqnarray}

Because the body of the process between quotes is impervious to
substitution, we get radically different answers. In fact, by
examining the first process in an input context,
e.g. $x?(z).\lift{w}{y!(z)}$, we see that the process under the lift
operator may be shaped by prefixed inputs binding a name inside it. In
this sense, the lift operator will be seen as a way to dynamically
construct processes before reifying them as names.

Finally equipped with these standard features we can present the
dynamics of the calculus.

\subsubsection{Operational semantics} 

Finally, we introduce the computational dynamics. What marks these
algebras as distinct from other more traditionally studied algebraic
structures, e.g. vector spaces or polynomial rings, is the manner in
which dynamics is captured. In traditional structures, dynamics is typically
expressed through morphisms between such structures, as in linear maps
between vector spaces or morphisms between rings. In algebras
associated with the semantics of computation, the dynamics is
expressed as part of the algebraic structure itself, through a
reduction reduction relation typically denoted by $\red$. Below, we
give a recursive presentation of this relation for the calculus used
in the encoding.

$\red \subseteq \pi \times \pi$
$\red : \pi \to \mathcal{P}(\pi)$

\begin{mathpar}
  \inferrule* [lab=Comm] { \textsf{match}( x_{src}, x_{trgt} ) } { x_{trgt}?(y)P \; | \; x_{src}!\langle {Q} \rangle \red P\{\quotep{Q}/y}\} }
  \and \\
  \inferrule* [lab=Par] {{P} \red {P}'} {{{P} | {Q}} \red {{P}' | {Q}}}
  \and
  \inferrule* [lab=Equiv]{{{P} \scong {P}'} \andalso {{P}' \red {Q}'} \andalso {{Q}' \scong {Q}}}{{P} \red {Q}}
\end{mathpar}

\begin{eqnarray*}
  match_{\equiv} (\quotep{P},\quotep{Q}) & := & P \equiv Q \\
  match_{\dagger}(\quotep{P},\quotep{Q}) & := & \forall R. P|Q \red^{*} R => R \red^{*} 0 \\
  match_{K}(\quotep{P},\quotep{Q}) & := & K \mbox{ for some context } K
\end{eqnarray*}

$u?(x)P | u!\langle Q \rangle \red P\{\quotep{Q}/x\}$

%We write $\wred$ for $\red^*$, and $P\red$ if $\exists Q $ such that $ P \red Q$.
We write $P\red$ if $\exists Q $ such that $ P \red Q$ and $P\not\red$, otherwise.

\section{Replication}

As mentioned before, it is known that replication (and hence
recursion) can be implemented in a higher-order process algebra
\cite{SangiorgiWalker}. As our first example of calculation with the
machinery thus far presented we give the construction explicitly in
the {\rhoc}.

\begin{eqnarray}
	D_{x} & := & \prefix{x}{y}{(\binpar{\outputp{x}{y}}{@{y}})} \nonumber\\
	\bangp_{x}{P} & := & \binpar{{x}!\langle{\binpar{D_{x}}{P}}\rangle}{D_{x}} \nonumber
\end{eqnarray}

\begin{eqnarray}
	\bangp_{x}{P} & & \nonumber\\
	=
	& {x}!\langle{(\prefix{x}{y}{(\outputp{x}{y} | @{y})) | P}}\rangle 
	      | \prefix{x}{y}{(\outputp{x}{y} | @{y})} & \nonumber\\
	\red
	& (\outputp{x}{y} | @{y})\substn{\quotep{(\prefix{x}{y}{(@{y} | \outputp{x}{y})) | P}}}{y} & \nonumber\\
	=
	& \outputp{x}{\quotep{(\prefix{x}{y}{(\outputp{x}{y} | @{y})) | P}}}
	  | {(\prefix{x}{y}{(\outputp{x}{y} | @{y})) | P}} & \nonumber\\
	\red
	& \ldots & \nonumber\\
	\red^*
	& P | P | \ldots & \nonumber
\end{eqnarray}

Of course, this encoding, as an implementation, runs away, unfolding
$\bangp{P}$ eagerly. A lazier and more implementable replication
operator, restricted to input-guarded processes, may be obtained as follows.

\begin{eqnarray}
\bangp{\prefix{u}{v}{P}} 
	:= 
	\binpar{\lift{x}{\prefix{u}{v}{(\binpar{D(x)}{P})}}}{D(x)} \nonumber
\end{eqnarray}

\begin{remark}
  Note that the lazier definition still does not deal with summation
  or mixed summation (i.e. sums over input and output). The reader is
  invited to construct definitions of replication that deal with these
  features. 

  Further, the definitions are parameterized in a name, $x$. Can you,
  gentle reader, make a definition that eliminates this parameter and
  guarantees no accidental interaction between the replication
  machinery and the process being replicated -- i.e. no accidental
  sharing of names used by the process to get its work done and the
  name(s) used by the replication to effect copying. This latter
  revision of the definition of replication is crucial to obtaining
  the expected identity $!!P \sim !P$.
\end{remark}

\begin{remark}\label{rem:paradoxical_combinator}
  The reader familiar with the lambda calculus will have noticed the
  similarity between $D$ and the paradoxical combinator.

  [Ed. note: the existence of this seems to suggest we have to be more
  restrictive on the set of processes and names we admit if we are to
  support no-cloning.]
\end{remark}

\subsubsection{Bisimulation}

The computational dynamics gives rise to another kind of equivalence,
the equivalence of computational behavior. As previously mentioned
this is typically captured \emph{via} some form of bisimulation.

% The notion we use in this paper is weak barbed bisimulation
% \cite{milner91polyadicpi}.

The notion we use in this paper is derived from weak barbed
bisimulation \cite{milner91polyadicpi}. 

\begin{definition}
An \emph{observation relation}, $\downarrow_{\mathcal N}$, over a set
of names, $\mathcal N$, is the smallest relation satisfying the rules
below.

\infrule[Out-barb]{y \in {\mathcal N}, \; x \nameeq y}
		  {\outputp{x}{v} \downarrow_{\mathcal N} x}
\infrule[Par-barb]{\mbox{$P\downarrow_{\mathcal N} x$ or $Q\downarrow_{\mathcal N} x$}}
		  {\binpar{P}{Q} \downarrow_{\mathcal N} x}

We write $P \Downarrow_{\mathcal N} x$ if there is $Q$ such that 
$P \wred Q$ and $Q \downarrow_{\mathcal N} x$.
\end{definition}

\begin{definition}
%\label{def.bbisim}
An  ${\mathcal N}$-\emph{barbed bisimulation} over a set of names, ${\mathcal N}$, is a symmetric binary relation 
${\mathcal S}_{\mathcal N}$ between agents such that $P\rel{S}_{\mathcal N}Q$ implies:
\begin{enumerate}
\item If $P \red P'$ then $Q \wred Q'$ and $P'\rel{S}_{\mathcal N} Q'$.
\item If $P\downarrow_{\mathcal N} x$, then $Q\Downarrow_{\mathcal N} x$.
\end{enumerate}
$P$ is ${\mathcal N}$-barbed bisimilar to $Q$, written
$P \wbbisim_{\mathcal N} Q$, if $P \rel{S}_{\mathcal N} Q$ for some ${\mathcal N}$-barbed bisimulation ${\mathcal S}_{\mathcal N}$.
\end{definition}

$\mathcal{R} \subseteq \pi \times \pi$

$P \mathcal{R} Q => \forall P'. P \red P' \Rightarrow \exists Q'. Q \red Q', P' \mathcal{R} Q'$

$P \vdash x \Rightarrow Q \vdash x$

\begin{mathpar}
  \inferrule*[lab=Out-barb]{x \nameeq y}{{y}!\langle{Q}\rangle \vdash x}
  \and
  \inferrule*[lab=Par-barb]{\mbox{$P\vdash x$ or $Q\vdash x$}}{\binpar{P}{Q} \vdash x}
\end{mathpar}

\subsubsection{Contexts}

One of the principle advantages of computational calculi like the
$\pi$-calculus is a well-defined notion of context,
contextual-equivalence and a correlation between
contextual-equivalence and notions of bisimulation. The notion of
context allows the decomposition of a process into (sub-)process and
its syntactic environment, its context. Thus, a context may be
thought of as a process with a ``hole'' (written $\Box$) in it. The
application of a context $M$ to a process $P$, written $M[P]$, is
tantamount to filling the hole in $M$ with $P$. In this paper we do
not need the full weight of this theory, but do make use of the notion
of context in the proof the main theorem. 

\begin{mathpar}
  \inferrule* [lab=summation] {} {{M_{M},M_{N}} \bc \Box \;|\; x.M_{A} \;|\; M_{M}+M_{N}}
  \and
  \inferrule* [lab=agent] {} {{M_{A}} \bc (\vec{x})M_{P} \;| \; \clift{P_0,\ldots,M_{P},\ldots,P_N}}
  \and \\
  \inferrule* [lab=process] {} {{M_{P}} \bc M_{N} \;| \;P|M_{P} }
\end{mathpar} 

\begin{mathpar}
  \inferrule* [lab=sychronization] {} {M_{N} \bc \Box \;|\; x?M_{F} \;|\; x!M_{C}}
  \and
  \inferrule* [lab=abstraction] {} {{M_{F}} \bc (x)M_{P} }
  \and
  \inferrule* [lab=concretion] {} {{M_{C}} \bc \langle M_{P} \rangle }
  \and \\
  \inferrule* [lab=process] {} {{M_{P}} \bc M_{N} \;| \;P|M_{P} }
\end{mathpar}

\begin{definition}[contextual application] Given a context $M$, and
  process $P$, we define the \emph{contextual application}, $M[P] :=
  M\{P/\Box\}$. That is, the contextual application of M to P is the
  substitution of $P$ for $\Box$ in $M$.
\end{definition}

$\meaningof{-} : L \to \mathcal{P}(\pi)$

\begin{mathpar}
  \inferrule* [lab=collection] {} {\meaningof{true} = \pi, \and \meaningof{~E} = \pi \setminus \meaningof{E}, \and \meaningof{E_{1} \& E_{2}} = \meaningof{E_{1}} \cap \meaningof{E_{2}}}
\end{mathpar}

\begin{mathpar}
  \inferrule* [lab=structure] {} {\meaningof{0} = \{ P \in \pi | P \equiv 0 \}, \and \\ \meaningof{E_1 | E_2} = \{ P \in \pi | P \equiv P_{1} | P_{2}, P_{1} \in \meaningof{E_{1}}, P_{2} \in \meaningof{E_2}\} }
\end{mathpar}

\begin{mathpar}
 \inferrule* [lab=behavior] {} {\meaningof{\langle a?b \rangle E} = \{ P \in \pi | P \equiv Q | u?(y)P', \\ \and \\\\ \and \\ \;\;\; u \in \meaningof{a}, \forall z.P'\{z/y\} \in \meaningof{E\{z/b\}}\}, \and \\ \meaningof{a!E} = \{ P \in \pi | P \equiv Q | x!\langle P' \rangle, x \in \meaningof{a} P' \in \meaningof{E}\} }
\end{mathpar}

\begin{mathpar}
 \inferrule* [lab=nominal] {} {\meaningof{\quotep{E}} = \{ \quotep{P} \in \quotep{\pi} | P \in \meaningof{E} \}, \and \meaningof{\quotep{P}} = \{ \quotep{Q} \in \quotep{\pi} | P \equiv Q \} \and \\ \meaningof{@\quotep{E}} = \{ P \in \pi | P \equiv @x, x \in \meaningof{E} \}}
\end{mathpar}

\begin{eqnarray*}
  \\
  \meaningof{-} : TS \to ST
\end{eqnarray*}

\begin{eqnarray*}
  \\
  L : TS \to ST
\end{eqnarray*}

\begin{eqnarray*}
  \\
  P \models E \iff P \in \meaningof{E}
\end{eqnarray*}

\begin{eqnarray*}
  P \approx_{L} Q \iff \forall E \in L. P \models E \iff Q \models E
\end{eqnarray*}

\begin{eqnarray*}
  P \approx_{K} Q
\end{eqnarray*}

\begin{eqnarray*}
  P \approx Q
\end{eqnarray*}

$\approx_{K} = \approx = \approx_{L}$

\subsubsection{Contextual duality}

Note that contexts extend the quotation operation to a family of
operations from processes to names. Given a context, $M$, we can
define a \emph{nominal context}, $\quotep{M}$ by $\quotep{M}[P] :=
\quotep{M[P]}$. To foreshadow what is to come we observe that these
operations enjoy a duality with processes very much like the duality
between vectors and maps from vectors to scalars.

Further, because the calculus is essentially higher-order, we have a
correspondence between contexts and processes. More specifically,
given a name $x$ and a context $M$ we can construct $M^{*}_{x}$ such
that 

\begin{mathpar}
  M^{*}_{x} | \lift{x}{P} \red M[P]
\end{mathpar}

namely,

\begin{mathpar}
  M^{*}_{x} := x?(u).M[\dropn{u}]
\end{mathpar}

The dependence of $M^{*}_{x}$ on a name makes it an abstraction, 

\begin{mathpar}
  M^{*} := (x)x?(u).M[\dropn{u}]
\end{mathpar}

\subsection{Additional notation}

It will sometimes be convenient to denote the process a name
quotes. We already have the notation $x = \quotep{P}$, but it will be
convenient to introduce an alternate notation, $\procn{x}$, when we
want to emphasize the connection to the use of the name. Note that, by
virtue of name equivalence, $\quotep{\procn{x}} \nameeq x$; so, the
notation is consistent with previous definitions.

Further, because names have structure it is possible to effect
substitutions on the basis of that structure. This means we need to
upgrade our notation for substitutions, which we accomplish by
adapting comprehension notation. Thus,

\begin{mathpar}
  P\{ y / x : x \in S \}
\end{mathpar}

is interpreted to mean the process derived from P by replacing (in a
capture-avoiding manner) each occurrence of $x$ in $S$ by $y$. For example,

\begin{mathpar}
  P\{ \quotep{\procn{x}|\procn{x}} / x : x \in \freenames{P} \}
\end{mathpar}

will replace each (occurrence) of a free name $x$ in $P$ by
$\quotep{\procn{x}|\procn{x}}$.

Also, we will avail ourselves of the notation $x^{L}$ and $x^{R}$ to
denote injections of a name into disjoint copies of the name
space. There are numerous ways to accomplish this. One example can be
found in \cite{MeredithR05}. This notation overloads to vectors of
names: $\vec{x}^{\pi} := (x_{i}^{\pi} \; : \; 0 \leq i < |\vec{x}| )$ where $\pi \in \{L,R\}$.

We also use $P^{\Box} := P|\Box$.

In \cite{MeredithR05} an interpretation of the new operator is
given. It turns out that there are several possible interpretations
all enjoying the requisite algebraic properties of the operator (see
\cite{milner91polyadicpi}). We will therefore make liberal use of
$(\nu\; \vec{x})P$.

% subsection the_syntax_and_semantics_of_the_notation_system (end)   

\input{qm2pi.qmops} 

\input{qm2pi.sterngerlach} 

\input{qm2pi.metric} 

% section concurrent_process_calculi (end)

%\input{qm2pi.proofsketch}

% section proof sketch (end)

%\input{qm2pi.slviaknots} 

% section spatial logic via knots (end)

\input{qm2pi.conclusion}

% section conclusion (end)

%\input{qm2pi.dtcodes} 

% section wiring algorithm (end)

\input{qm2pi.ack} 

% section acknowledgments (end)

\newpage


\bibliographystyle{plain}   
\bibliography{../../biblios/main.bib}

\input{qm2pi.rhodetails}

\end{document}

 

%\documentclass[12pt]{llncs}
%\documentclass{jktr}

\usepackage[pdftex]{hyperref}                   
\usepackage {listings}
\usepackage {mathpartir}
\usepackage{bcprules}
%\usepackage{listings}
                       
\usepackage{graphicx} 
%\usepackage[margins=2.5cm,nohead,nofoot]{geometry}
%\usepackage{geometry}
\usepackage{amsfonts}
\usepackage{amstext}
\usepackage{latexsym}
\usepackage{amssymb}
\usepackage{color}


%\include{myPreamble}
\include{qm2pi.local} 

%\ifpdf
%\usepackage[pdftex]{graphicx}
%\else
%\usepackage{graphicx}
%\fi

 % \ifpdf
%  \usepackage{pdfsync}
%  \if


%\title{Brief Article}
%\author{David F. Snyder}
%\author{L.G. Meredith}

%\address{Dept. of Math., Texas State University--San Marcos, San Marcos, TX 78666}
       
\pagestyle{empty}


\begin{document}

\lstset{language=[Objective]Caml,frame=shadowbox}

\input{qm2pi.front}

% section front matter (end)

\input{qm2pi.intro} 
 
% section introduction (end)

% \input{qm2pi.knotations} 

% section notation (end)

\input{qm2pi.process.calculi} 

% section concurrent_process_calculi_and_spatial_logics_ (end)
    
%\input{qm2pi.knots2pi} 

%\input{qm2pi.trefoil} 

%\input{qm2pi.mainthm} 

% subsection basic_interpretation (end)

%\input{qm2pi.rho.presentation} 
\subsection{The syntax and semantics of the notation system}\label{sub:the_syntax_and_semantics_of_the_notation_system} % (fold)

We now summarize a technical presentation of the calculus that
embodies our theory of dynamics. The typical presentation of such a
calculus follows the style of giving generators and relations on
them. The grammar, below, describing term constructors, freely
generates the set of processes, $\Proc$. This set is then quotiented
by a relation known as structural congruence and it is over this set
that the notion of dynamics is expressed. This presentation is
essentially that of \cite{MeredithR05} with the addition of
polyadicity and summation. For readability we have relegated some of
the technical subtleties to an appendix.

\subsubsection{Process grammar}\label{subsub:process_grammar}

\begin{mathpar}
  \inferrule* [lab=synchronization] {} {{M} \bc \pzero \;|\; x?F \;|\; x!C }
  \and
  \inferrule* [lab=abstraction] {} {{F} \bc (x)P}
  \and
  \inferrule* [lab=concretion] {} {{C} \bc \langle Q \rangle}
  \and
  \inferrule* [lab=process] {} {{P,Q} \bc M \;| \;P|Q \;|\; @{x}}
  \and
  \inferrule* [lab=name] {} {{x} \bc \quotep{P}}
\end{mathpar} 

Note that $\vec{x}$ (resp. $\vec{P}$) denotes a vector of names
(resp. processes) of length $|\vec{x}|$ (resp. $|\vec{P}|$). We adopt
the following useful abbreviations.

\begin{mathpar}
   x?(\vec{y}).P := x.(\vec{y})P \and  x\clift{\vec{P}} := x.\clift{\vec{P}}
   \and x!(y) := \lift{x}{\dropn{y}}
   \and \Pi_{i=0}^{n-1}P_i := P_0 | \ldots | P_{n-1}
\end{mathpar}

\subsubsection{Structural congruence}

\paragraph{Free and bound names and alpha-equivalence.} At the
core of structural equivalence is alpha-equivalence which identifies
process that are the same up to a change of variable. Formally, we
recognize the distinction between free and bound names. The free names
of a process, $\freenames{P}$, may be calculated recursively as
follows:

\begin{mathpar}
\freenames{\pzero} := \emptyset
  \and \\
  \freenames{x?(y).P} := \{ x \} \cup (\freenames{P} \setminus \{ y \})
  \and 
  \freenames{x!\langle P \rangle} := \{ x \} \cup \{ P \} 
  \and \\
  \freenames{P|Q} := \freenames{P} \cup \freenames{Q}
  \and \\
  \freenames{@{x}} := \{ x \}
\end{mathpar}

$\pi$
$\quotep{\pi}$

$\freenames{-} : \pi \to \mathcal{P}(\quotep{\pi})$

\begin{eqnarray*}
  \freenames{\pzero} & := & \emptyset \\
  \freenames{x?(y).P} & := & \{ x \} \cup (\freenames{P} \setminus \{ y \}) \\
  \freenames{x!\langle P \rangle} & := & \{ x \} \cup \{ P \} \\
  \freenames{P|Q} & := & \freenames{P} \cup \freenames{Q} \\
  \freenames{\dropn{x}} & := & \{ x \}
\end{eqnarray*}

The bound names of a process, $\boundnames{P}$, are those names occurring in $P$
that are not free. For example, in $x?(y).0$, the name $x$ is free, while $y$ is bound.

\begin{mathpar}
  \inferrule* [lab=monoidal-laws] {} { P|Q \equiv Q|P \and P|0 \equiv P \and P|(Q|R) \equiv (P|Q)|R }
\end{mathpar}

\begin{mathpar}
  \inferrule* [lab=alpha-equivalence] {} { (x)P \equiv (y)P\{y/x\} \and y \not\in \freenames{P} }
\end{mathpar}

\begin{definition}
Then two processes, $P,Q$, are alpha-equivalent if $P = Q\{\vec{y}/\vec{x}\}$ for
some $\vec{x} \in \boundnames{Q},\vec{y} \in \boundnames{P}$, where $Q\{\vec{y}/\vec{x}\}$
denotes the capture-avoiding substitution of $\vec{y}$ for $\vec{x}$ in $Q$.
\end{definition}

\begin{definition}
  The {\em structural congruence} \cite{SangiorgiWalker} , $\equiv$,
  between processes is the least congruence containing
  alpha-equivalence, satisfying the abelian monoid laws
  (associativity, commutativity and $\pzero$ as identity) for parallel
  composition $|$ and for summation $+$.
\end{definition}

\subsection{Name equivalence}

We take name equivalence, written $\nameeq$, to be the smallest
equivalence relation generated by the following rules.

\begin{mathpar}
\inferrule*[lab=Quote-drop]
{ }
{ \quotep{@{x}} \nameeq x }

\inferrule*[lab=Struct-equiv]
{ P \scong Q }
{ \quotep{P} \nameeq \quotep{Q} }
\end{mathpar}

The astute reader will have noticed that the mutual recursion of names
and processes imposes a mutual recursion on alpha-equivalence and
structural equivalence via name-equivalence. Fortunately, all of this
works out pleasantly and we may calculate in the natural way, free of
concern. The reader interested in the details is referred to the
appendix \ref{appendix:rho_details}.

\subsection{Substitution}

We use $\Proc$ for the set of processes, $\QProc$ for the set of
names, and $\id{\{}\vec{y} / \vec{x} \id{\}}$ to denote partial maps,
$s : \QProc \rightarrow \QProc$. A map, $s$ lifts, uniquely, to a map
on process terms, $\widehat{s} : \Proc \rightarrow \Proc$ by the
following equations.

\begin{mathpar}
  (0) \psubstp{Q}{P} := 0 \\
  (R \juxtap S) \psubstp{Q}{P}
  :=    
  (R)\psubstp{Q}{P} \juxtap (S) \psubstp{Q}{P} \\
  (x?(y).R) \psubstp{Q}{P}    
  :=    
  (x)\substp{Q}{P} (z)\concat( (R \psubstn{z}{y}) \psubstp{Q}{P} ) \\
  (\lift{x}{R}) \psubstp{Q}{P}  
  :=
  \lift{(x)\substp{Q}{P}}{ R \psubstp{Q}{P} } \\
%   (\dropn{x})  \psubstp{Q}{P}       
%   := 
%   \left\{ 
%     \begin{array}{ccc} 
%       \dropn{\quotep{Q}} & & x \nameeq \quotep{P} \\
%       \dropn{x} & & otherwise \\
%     \end{array}
%   \right. 
  (\dropn{x})  \psubstp{Q}{P}       
  := 
  \left\{ 
    \begin{array}{ccc} 
      Q & & x \nameeq \quotep{P} \\
      \dropn{x} & & otherwise \\
    \end{array}
  \right.
\end{mathpar}
 

where

\begin{eqnarray}
  (x)\id{\{} \lpquote Q \rpquote / \lpquote P \rpquote \id{\}}            = 
  \left\{ 
    \begin{array}{ccc}
      \lpquote Q \rpquote & & x \nameeq \lpquote P \rpquote \\
      x & & otherwise \\
    \end{array}
  \right. \nonumber
\end{eqnarray}

and $z$ is chosen distinct from $\quotep{P}$, $\quotep{Q}$, the free
names in $Q$, and all the names in $R$. Our $\alpha$-equivalence will
be built in the standard way from this substitution.

\begin{remark}\label{rem:no_self_referential_names}
  One consequence of these definitions is that $\forall P. \quotep{P}
  \not\in \freenames{P}$.
\end{remark}

\subsection{ Dynamic quote: an example }

Anticipating something of what's to come, consider applying the
substitution, $\widehat{\id{\{}u / z \id{\}}}$, to the following pair
of processes, $\lift{w}{y!(z)}$ and $w[ \lpquote y!(z) \rpquote ]$.

\begin{eqnarray}
	\lift{w}{y!(z)}\widehat{\id{\{}u / z \id{\}}}
		& = &
		\lift{w}{y!(u)} \nonumber\\
	w[ \lpquote y!(z) \rpquote ] \widehat{ \id{\{}u / z \id{\}} }
		& = &
		w[ \lpquote y!(z) \rpquote ] \nonumber
\end{eqnarray}

Because the body of the process between quotes is impervious to
substitution, we get radically different answers. In fact, by
examining the first process in an input context,
e.g. $x?(z).\lift{w}{y!(z)}$, we see that the process under the lift
operator may be shaped by prefixed inputs binding a name inside it. In
this sense, the lift operator will be seen as a way to dynamically
construct processes before reifying them as names.

Finally equipped with these standard features we can present the
dynamics of the calculus.

\subsubsection{Operational semantics} 

Finally, we introduce the computational dynamics. What marks these
algebras as distinct from other more traditionally studied algebraic
structures, e.g. vector spaces or polynomial rings, is the manner in
which dynamics is captured. In traditional structures, dynamics is typically
expressed through morphisms between such structures, as in linear maps
between vector spaces or morphisms between rings. In algebras
associated with the semantics of computation, the dynamics is
expressed as part of the algebraic structure itself, through a
reduction reduction relation typically denoted by $\red$. Below, we
give a recursive presentation of this relation for the calculus used
in the encoding.

$\red \subseteq \pi \times \pi$
$\red : \pi \to \mathcal{P}(\pi)$

\begin{mathpar}
  \inferrule* [lab=Comm] { \textsf{match}( x_{src}, x_{trgt} ) } { x_{trgt}?(y)P \; | \; x_{src}!\langle {Q} \rangle \red P\{\quotep{Q}/y}\} }
  \and \\
  \inferrule* [lab=Par] {{P} \red {P}'} {{{P} | {Q}} \red {{P}' | {Q}}}
  \and
  \inferrule* [lab=Equiv]{{{P} \scong {P}'} \andalso {{P}' \red {Q}'} \andalso {{Q}' \scong {Q}}}{{P} \red {Q}}
\end{mathpar}

\begin{eqnarray*}
  match_{\equiv} (\quotep{P},\quotep{Q}) & := & P \equiv Q \\
  match_{\dagger}(\quotep{P},\quotep{Q}) & := & \forall R. P|Q \red^{*} R => R \red^{*} 0 \\
  match_{K}(\quotep{P},\quotep{Q}) & := & K \mbox{ for some context } K
\end{eqnarray*}

$u?(x)P | u!\langle Q \rangle \red P\{\quotep{Q}/x\}$

%We write $\wred$ for $\red^*$, and $P\red$ if $\exists Q $ such that $ P \red Q$.
We write $P\red$ if $\exists Q $ such that $ P \red Q$ and $P\not\red$, otherwise.

\section{Replication}

As mentioned before, it is known that replication (and hence
recursion) can be implemented in a higher-order process algebra
\cite{SangiorgiWalker}. As our first example of calculation with the
machinery thus far presented we give the construction explicitly in
the {\rhoc}.

\begin{eqnarray}
	D_{x} & := & \prefix{x}{y}{(\binpar{\outputp{x}{y}}{@{y}})} \nonumber\\
	\bangp_{x}{P} & := & \binpar{{x}!\langle{\binpar{D_{x}}{P}}\rangle}{D_{x}} \nonumber
\end{eqnarray}

\begin{eqnarray}
	\bangp_{x}{P} & & \nonumber\\
	=
	& {x}!\langle{(\prefix{x}{y}{(\outputp{x}{y} | @{y})) | P}}\rangle 
	      | \prefix{x}{y}{(\outputp{x}{y} | @{y})} & \nonumber\\
	\red
	& (\outputp{x}{y} | @{y})\substn{\quotep{(\prefix{x}{y}{(@{y} | \outputp{x}{y})) | P}}}{y} & \nonumber\\
	=
	& \outputp{x}{\quotep{(\prefix{x}{y}{(\outputp{x}{y} | @{y})) | P}}}
	  | {(\prefix{x}{y}{(\outputp{x}{y} | @{y})) | P}} & \nonumber\\
	\red
	& \ldots & \nonumber\\
	\red^*
	& P | P | \ldots & \nonumber
\end{eqnarray}

Of course, this encoding, as an implementation, runs away, unfolding
$\bangp{P}$ eagerly. A lazier and more implementable replication
operator, restricted to input-guarded processes, may be obtained as follows.

\begin{eqnarray}
\bangp{\prefix{u}{v}{P}} 
	:= 
	\binpar{\lift{x}{\prefix{u}{v}{(\binpar{D(x)}{P})}}}{D(x)} \nonumber
\end{eqnarray}

\begin{remark}
  Note that the lazier definition still does not deal with summation
  or mixed summation (i.e. sums over input and output). The reader is
  invited to construct definitions of replication that deal with these
  features. 

  Further, the definitions are parameterized in a name, $x$. Can you,
  gentle reader, make a definition that eliminates this parameter and
  guarantees no accidental interaction between the replication
  machinery and the process being replicated -- i.e. no accidental
  sharing of names used by the process to get its work done and the
  name(s) used by the replication to effect copying. This latter
  revision of the definition of replication is crucial to obtaining
  the expected identity $!!P \sim !P$.
\end{remark}

\begin{remark}\label{rem:paradoxical_combinator}
  The reader familiar with the lambda calculus will have noticed the
  similarity between $D$ and the paradoxical combinator.

  [Ed. note: the existence of this seems to suggest we have to be more
  restrictive on the set of processes and names we admit if we are to
  support no-cloning.]
\end{remark}

\subsubsection{Bisimulation}

The computational dynamics gives rise to another kind of equivalence,
the equivalence of computational behavior. As previously mentioned
this is typically captured \emph{via} some form of bisimulation.

% The notion we use in this paper is weak barbed bisimulation
% \cite{milner91polyadicpi}.

The notion we use in this paper is derived from weak barbed
bisimulation \cite{milner91polyadicpi}. 

\begin{definition}
An \emph{observation relation}, $\downarrow_{\mathcal N}$, over a set
of names, $\mathcal N$, is the smallest relation satisfying the rules
below.

\infrule[Out-barb]{y \in {\mathcal N}, \; x \nameeq y}
		  {\outputp{x}{v} \downarrow_{\mathcal N} x}
\infrule[Par-barb]{\mbox{$P\downarrow_{\mathcal N} x$ or $Q\downarrow_{\mathcal N} x$}}
		  {\binpar{P}{Q} \downarrow_{\mathcal N} x}

We write $P \Downarrow_{\mathcal N} x$ if there is $Q$ such that 
$P \wred Q$ and $Q \downarrow_{\mathcal N} x$.
\end{definition}

\begin{definition}
%\label{def.bbisim}
An  ${\mathcal N}$-\emph{barbed bisimulation} over a set of names, ${\mathcal N}$, is a symmetric binary relation 
${\mathcal S}_{\mathcal N}$ between agents such that $P\rel{S}_{\mathcal N}Q$ implies:
\begin{enumerate}
\item If $P \red P'$ then $Q \wred Q'$ and $P'\rel{S}_{\mathcal N} Q'$.
\item If $P\downarrow_{\mathcal N} x$, then $Q\Downarrow_{\mathcal N} x$.
\end{enumerate}
$P$ is ${\mathcal N}$-barbed bisimilar to $Q$, written
$P \wbbisim_{\mathcal N} Q$, if $P \rel{S}_{\mathcal N} Q$ for some ${\mathcal N}$-barbed bisimulation ${\mathcal S}_{\mathcal N}$.
\end{definition}

$\mathcal{R} \subseteq \pi \times \pi$

$P \mathcal{R} Q => \forall P'. P \red P' \Rightarrow \exists Q'. Q \red Q', P' \mathcal{R} Q'$

$P \vdash x \Rightarrow Q \vdash x$

\begin{mathpar}
  \inferrule*[lab=Out-barb]{x \nameeq y}{{y}!\langle{Q}\rangle \vdash x}
  \and
  \inferrule*[lab=Par-barb]{\mbox{$P\vdash x$ or $Q\vdash x$}}{\binpar{P}{Q} \vdash x}
\end{mathpar}

\subsubsection{Contexts}

One of the principle advantages of computational calculi like the
$\pi$-calculus is a well-defined notion of context,
contextual-equivalence and a correlation between
contextual-equivalence and notions of bisimulation. The notion of
context allows the decomposition of a process into (sub-)process and
its syntactic environment, its context. Thus, a context may be
thought of as a process with a ``hole'' (written $\Box$) in it. The
application of a context $M$ to a process $P$, written $M[P]$, is
tantamount to filling the hole in $M$ with $P$. In this paper we do
not need the full weight of this theory, but do make use of the notion
of context in the proof the main theorem. 

\begin{mathpar}
  \inferrule* [lab=summation] {} {{M_{M},M_{N}} \bc \Box \;|\; x.M_{A} \;|\; M_{M}+M_{N}}
  \and
  \inferrule* [lab=agent] {} {{M_{A}} \bc (\vec{x})M_{P} \;| \; \clift{P_0,\ldots,M_{P},\ldots,P_N}}
  \and \\
  \inferrule* [lab=process] {} {{M_{P}} \bc M_{N} \;| \;P|M_{P} }
\end{mathpar} 

\begin{mathpar}
  \inferrule* [lab=sychronization] {} {M_{N} \bc \Box \;|\; x?M_{F} \;|\; x!M_{C}}
  \and
  \inferrule* [lab=abstraction] {} {{M_{F}} \bc (x)M_{P} }
  \and
  \inferrule* [lab=concretion] {} {{M_{C}} \bc \langle M_{P} \rangle }
  \and \\
  \inferrule* [lab=process] {} {{M_{P}} \bc M_{N} \;| \;P|M_{P} }
\end{mathpar}

\begin{definition}[contextual application] Given a context $M$, and
  process $P$, we define the \emph{contextual application}, $M[P] :=
  M\{P/\Box\}$. That is, the contextual application of M to P is the
  substitution of $P$ for $\Box$ in $M$.
\end{definition}

$\meaningof{-} : L \to \mathcal{P}(\pi)$

\begin{mathpar}
  \inferrule* [lab=collection] {} {\meaningof{true} = \pi, \and \meaningof{~E} = \pi \setminus \meaningof{E}, \and \meaningof{E_{1} \& E_{2}} = \meaningof{E_{1}} \cap \meaningof{E_{2}}}
\end{mathpar}

\begin{mathpar}
  \inferrule* [lab=structure] {} {\meaningof{0} = \{ P \in \pi | P \equiv 0 \}, \and \\ \meaningof{E_1 | E_2} = \{ P \in \pi | P \equiv P_{1} | P_{2}, P_{1} \in \meaningof{E_{1}}, P_{2} \in \meaningof{E_2}\} }
\end{mathpar}

\begin{mathpar}
 \inferrule* [lab=behavior] {} {\meaningof{\langle a?b \rangle E} = \{ P \in \pi | P \equiv Q | u?(y)P', \\ \and \\\\ \and \\ \;\;\; u \in \meaningof{a}, \forall z.P'\{z/y\} \in \meaningof{E\{z/b\}}\}, \and \\ \meaningof{a!E} = \{ P \in \pi | P \equiv Q | x!\langle P' \rangle, x \in \meaningof{a} P' \in \meaningof{E}\} }
\end{mathpar}

\begin{mathpar}
 \inferrule* [lab=nominal] {} {\meaningof{\quotep{E}} = \{ \quotep{P} \in \quotep{\pi} | P \in \meaningof{E} \}, \and \meaningof{\quotep{P}} = \{ \quotep{Q} \in \quotep{\pi} | P \equiv Q \} \and \\ \meaningof{@\quotep{E}} = \{ P \in \pi | P \equiv @x, x \in \meaningof{E} \}}
\end{mathpar}

\begin{eqnarray*}
  \\
  \meaningof{-} : TS \to ST
\end{eqnarray*}

\begin{eqnarray*}
  \\
  L : TS \to ST
\end{eqnarray*}

\begin{eqnarray*}
  \\
  P \models E \iff P \in \meaningof{E}
\end{eqnarray*}

\begin{eqnarray*}
  P \approx_{L} Q \iff \forall E \in L. P \models E \iff Q \models E
\end{eqnarray*}

\begin{eqnarray*}
  P \approx_{K} Q
\end{eqnarray*}

\begin{eqnarray*}
  P \approx Q
\end{eqnarray*}

$\approx_{K} = \approx = \approx_{L}$

\subsubsection{Contextual duality}

Note that contexts extend the quotation operation to a family of
operations from processes to names. Given a context, $M$, we can
define a \emph{nominal context}, $\quotep{M}$ by $\quotep{M}[P] :=
\quotep{M[P]}$. To foreshadow what is to come we observe that these
operations enjoy a duality with processes very much like the duality
between vectors and maps from vectors to scalars.

Further, because the calculus is essentially higher-order, we have a
correspondence between contexts and processes. More specifically,
given a name $x$ and a context $M$ we can construct $M^{*}_{x}$ such
that 

\begin{mathpar}
  M^{*}_{x} | \lift{x}{P} \red M[P]
\end{mathpar}

namely,

\begin{mathpar}
  M^{*}_{x} := x?(u).M[\dropn{u}]
\end{mathpar}

The dependence of $M^{*}_{x}$ on a name makes it an abstraction, 

\begin{mathpar}
  M^{*} := (x)x?(u).M[\dropn{u}]
\end{mathpar}

\subsection{Additional notation}

It will sometimes be convenient to denote the process a name
quotes. We already have the notation $x = \quotep{P}$, but it will be
convenient to introduce an alternate notation, $\procn{x}$, when we
want to emphasize the connection to the use of the name. Note that, by
virtue of name equivalence, $\quotep{\procn{x}} \nameeq x$; so, the
notation is consistent with previous definitions.

Further, because names have structure it is possible to effect
substitutions on the basis of that structure. This means we need to
upgrade our notation for substitutions, which we accomplish by
adapting comprehension notation. Thus,

\begin{mathpar}
  P\{ y / x : x \in S \}
\end{mathpar}

is interpreted to mean the process derived from P by replacing (in a
capture-avoiding manner) each occurrence of $x$ in $S$ by $y$. For example,

\begin{mathpar}
  P\{ \quotep{\procn{x}|\procn{x}} / x : x \in \freenames{P} \}
\end{mathpar}

will replace each (occurrence) of a free name $x$ in $P$ by
$\quotep{\procn{x}|\procn{x}}$.

Also, we will avail ourselves of the notation $x^{L}$ and $x^{R}$ to
denote injections of a name into disjoint copies of the name
space. There are numerous ways to accomplish this. One example can be
found in \cite{MeredithR05}. This notation overloads to vectors of
names: $\vec{x}^{\pi} := (x_{i}^{\pi} \; : \; 0 \leq i < |\vec{x}| )$ where $\pi \in \{L,R\}$.

We also use $P^{\Box} := P|\Box$.

In \cite{MeredithR05} an interpretation of the new operator is
given. It turns out that there are several possible interpretations
all enjoying the requisite algebraic properties of the operator (see
\cite{milner91polyadicpi}). We will therefore make liberal use of
$(\nu\; \vec{x})P$.

% subsection the_syntax_and_semantics_of_the_notation_system (end)   

\input{qm2pi.qmops} 

\input{qm2pi.sterngerlach} 

\input{qm2pi.metric} 

% section concurrent_process_calculi (end)

%\input{qm2pi.proofsketch}

% section proof sketch (end)

%\input{qm2pi.slviaknots} 

% section spatial logic via knots (end)

\input{qm2pi.conclusion}

% section conclusion (end)

%\input{qm2pi.dtcodes} 

% section wiring algorithm (end)

\input{qm2pi.ack} 

% section acknowledgments (end)

\newpage


\bibliographystyle{plain}   
\bibliography{../../biblios/main.bib}

\input{qm2pi.rhodetails}

\end{document}

 

% subsection basic_interpretation (end)

%\input{qm2pi.rho.presentation} 
\subsection{The syntax and semantics of the notation system}\label{sub:the_syntax_and_semantics_of_the_notation_system} % (fold)

We now summarize a technical presentation of the calculus that
embodies our theory of dynamics. The typical presentation of such a
calculus follows the style of giving generators and relations on
them. The grammar, below, describing term constructors, freely
generates the set of processes, $\Proc$. This set is then quotiented
by a relation known as structural congruence and it is over this set
that the notion of dynamics is expressed. This presentation is
essentially that of \cite{MeredithR05} with the addition of
polyadicity and summation. For readability we have relegated some of
the technical subtleties to an appendix.

\subsubsection{Process grammar}\label{subsub:process_grammar}

\begin{mathpar}
  \inferrule* [lab=synchronization] {} {{M} \bc \pzero \;|\; x?F \;|\; x!C }
  \and
  \inferrule* [lab=abstraction] {} {{F} \bc (x)P}
  \and
  \inferrule* [lab=concretion] {} {{C} \bc \langle Q \rangle}
  \and
  \inferrule* [lab=process] {} {{P,Q} \bc M \;| \;P|Q \;|\; @{x}}
  \and
  \inferrule* [lab=name] {} {{x} \bc \quotep{P}}
\end{mathpar} 

Note that $\vec{x}$ (resp. $\vec{P}$) denotes a vector of names
(resp. processes) of length $|\vec{x}|$ (resp. $|\vec{P}|$). We adopt
the following useful abbreviations.

\begin{mathpar}
   x?(\vec{y}).P := x.(\vec{y})P \and  x\clift{\vec{P}} := x.\clift{\vec{P}}
   \and x!(y) := \lift{x}{\dropn{y}}
   \and \Pi_{i=0}^{n-1}P_i := P_0 | \ldots | P_{n-1}
\end{mathpar}

\subsubsection{Structural congruence}

\paragraph{Free and bound names and alpha-equivalence.} At the
core of structural equivalence is alpha-equivalence which identifies
process that are the same up to a change of variable. Formally, we
recognize the distinction between free and bound names. The free names
of a process, $\freenames{P}$, may be calculated recursively as
follows:

\begin{mathpar}
\freenames{\pzero} := \emptyset
  \and \\
  \freenames{x?(y).P} := \{ x \} \cup (\freenames{P} \setminus \{ y \})
  \and 
  \freenames{x!\langle P \rangle} := \{ x \} \cup \{ P \} 
  \and \\
  \freenames{P|Q} := \freenames{P} \cup \freenames{Q}
  \and \\
  \freenames{@{x}} := \{ x \}
\end{mathpar}

$\pi$
$\quotep{\pi}$

$\freenames{-} : \pi \to \mathcal{P}(\quotep{\pi})$

\begin{eqnarray*}
  \freenames{\pzero} & := & \emptyset \\
  \freenames{x?(y).P} & := & \{ x \} \cup (\freenames{P} \setminus \{ y \}) \\
  \freenames{x!\langle P \rangle} & := & \{ x \} \cup \{ P \} \\
  \freenames{P|Q} & := & \freenames{P} \cup \freenames{Q} \\
  \freenames{\dropn{x}} & := & \{ x \}
\end{eqnarray*}

The bound names of a process, $\boundnames{P}$, are those names occurring in $P$
that are not free. For example, in $x?(y).0$, the name $x$ is free, while $y$ is bound.

\begin{mathpar}
  \inferrule* [lab=monoidal-laws] {} { P|Q \equiv Q|P \and P|0 \equiv P \and P|(Q|R) \equiv (P|Q)|R }
\end{mathpar}

\begin{mathpar}
  \inferrule* [lab=alpha-equivalence] {} { (x)P \equiv (y)P\{y/x\} \and y \not\in \freenames{P} }
\end{mathpar}

\begin{definition}
Then two processes, $P,Q$, are alpha-equivalent if $P = Q\{\vec{y}/\vec{x}\}$ for
some $\vec{x} \in \boundnames{Q},\vec{y} \in \boundnames{P}$, where $Q\{\vec{y}/\vec{x}\}$
denotes the capture-avoiding substitution of $\vec{y}$ for $\vec{x}$ in $Q$.
\end{definition}

\begin{definition}
  The {\em structural congruence} \cite{SangiorgiWalker} , $\equiv$,
  between processes is the least congruence containing
  alpha-equivalence, satisfying the abelian monoid laws
  (associativity, commutativity and $\pzero$ as identity) for parallel
  composition $|$ and for summation $+$.
\end{definition}

\subsection{Name equivalence}

We take name equivalence, written $\nameeq$, to be the smallest
equivalence relation generated by the following rules.

\begin{mathpar}
\inferrule*[lab=Quote-drop]
{ }
{ \quotep{@{x}} \nameeq x }

\inferrule*[lab=Struct-equiv]
{ P \scong Q }
{ \quotep{P} \nameeq \quotep{Q} }
\end{mathpar}

The astute reader will have noticed that the mutual recursion of names
and processes imposes a mutual recursion on alpha-equivalence and
structural equivalence via name-equivalence. Fortunately, all of this
works out pleasantly and we may calculate in the natural way, free of
concern. The reader interested in the details is referred to the
appendix \ref{appendix:rho_details}.

\subsection{Substitution}

We use $\Proc$ for the set of processes, $\QProc$ for the set of
names, and $\id{\{}\vec{y} / \vec{x} \id{\}}$ to denote partial maps,
$s : \QProc \rightarrow \QProc$. A map, $s$ lifts, uniquely, to a map
on process terms, $\widehat{s} : \Proc \rightarrow \Proc$ by the
following equations.

\begin{mathpar}
  (0) \psubstp{Q}{P} := 0 \\
  (R \juxtap S) \psubstp{Q}{P}
  :=    
  (R)\psubstp{Q}{P} \juxtap (S) \psubstp{Q}{P} \\
  (x?(y).R) \psubstp{Q}{P}    
  :=    
  (x)\substp{Q}{P} (z)\concat( (R \psubstn{z}{y}) \psubstp{Q}{P} ) \\
  (\lift{x}{R}) \psubstp{Q}{P}  
  :=
  \lift{(x)\substp{Q}{P}}{ R \psubstp{Q}{P} } \\
%   (\dropn{x})  \psubstp{Q}{P}       
%   := 
%   \left\{ 
%     \begin{array}{ccc} 
%       \dropn{\quotep{Q}} & & x \nameeq \quotep{P} \\
%       \dropn{x} & & otherwise \\
%     \end{array}
%   \right. 
  (\dropn{x})  \psubstp{Q}{P}       
  := 
  \left\{ 
    \begin{array}{ccc} 
      Q & & x \nameeq \quotep{P} \\
      \dropn{x} & & otherwise \\
    \end{array}
  \right.
\end{mathpar}
 

where

\begin{eqnarray}
  (x)\id{\{} \lpquote Q \rpquote / \lpquote P \rpquote \id{\}}            = 
  \left\{ 
    \begin{array}{ccc}
      \lpquote Q \rpquote & & x \nameeq \lpquote P \rpquote \\
      x & & otherwise \\
    \end{array}
  \right. \nonumber
\end{eqnarray}

and $z$ is chosen distinct from $\quotep{P}$, $\quotep{Q}$, the free
names in $Q$, and all the names in $R$. Our $\alpha$-equivalence will
be built in the standard way from this substitution.

\begin{remark}\label{rem:no_self_referential_names}
  One consequence of these definitions is that $\forall P. \quotep{P}
  \not\in \freenames{P}$.
\end{remark}

\subsection{ Dynamic quote: an example }

Anticipating something of what's to come, consider applying the
substitution, $\widehat{\id{\{}u / z \id{\}}}$, to the following pair
of processes, $\lift{w}{y!(z)}$ and $w[ \lpquote y!(z) \rpquote ]$.

\begin{eqnarray}
	\lift{w}{y!(z)}\widehat{\id{\{}u / z \id{\}}}
		& = &
		\lift{w}{y!(u)} \nonumber\\
	w[ \lpquote y!(z) \rpquote ] \widehat{ \id{\{}u / z \id{\}} }
		& = &
		w[ \lpquote y!(z) \rpquote ] \nonumber
\end{eqnarray}

Because the body of the process between quotes is impervious to
substitution, we get radically different answers. In fact, by
examining the first process in an input context,
e.g. $x?(z).\lift{w}{y!(z)}$, we see that the process under the lift
operator may be shaped by prefixed inputs binding a name inside it. In
this sense, the lift operator will be seen as a way to dynamically
construct processes before reifying them as names.

Finally equipped with these standard features we can present the
dynamics of the calculus.

\subsubsection{Operational semantics} 

Finally, we introduce the computational dynamics. What marks these
algebras as distinct from other more traditionally studied algebraic
structures, e.g. vector spaces or polynomial rings, is the manner in
which dynamics is captured. In traditional structures, dynamics is typically
expressed through morphisms between such structures, as in linear maps
between vector spaces or morphisms between rings. In algebras
associated with the semantics of computation, the dynamics is
expressed as part of the algebraic structure itself, through a
reduction reduction relation typically denoted by $\red$. Below, we
give a recursive presentation of this relation for the calculus used
in the encoding.

$\red \subseteq \pi \times \pi$
$\red : \pi \to \mathcal{P}(\pi)$

\begin{mathpar}
  \inferrule* [lab=Comm] { \textsf{match}( x_{src}, x_{trgt} ) } { x_{trgt}?(y)P \; | \; x_{src}!\langle {Q} \rangle \red P\{\quotep{Q}/y}\} }
  \and \\
  \inferrule* [lab=Par] {{P} \red {P}'} {{{P} | {Q}} \red {{P}' | {Q}}}
  \and
  \inferrule* [lab=Equiv]{{{P} \scong {P}'} \andalso {{P}' \red {Q}'} \andalso {{Q}' \scong {Q}}}{{P} \red {Q}}
\end{mathpar}

\begin{eqnarray*}
  match_{\equiv} (\quotep{P},\quotep{Q}) & := & P \equiv Q \\
  match_{\dagger}(\quotep{P},\quotep{Q}) & := & \forall R. P|Q \red^{*} R => R \red^{*} 0 \\
  match_{K}(\quotep{P},\quotep{Q}) & := & K \mbox{ for some context } K
\end{eqnarray*}

$u?(x)P | u!\langle Q \rangle \red P\{\quotep{Q}/x\}$

%We write $\wred$ for $\red^*$, and $P\red$ if $\exists Q $ such that $ P \red Q$.
We write $P\red$ if $\exists Q $ such that $ P \red Q$ and $P\not\red$, otherwise.

\section{Replication}

As mentioned before, it is known that replication (and hence
recursion) can be implemented in a higher-order process algebra
\cite{SangiorgiWalker}. As our first example of calculation with the
machinery thus far presented we give the construction explicitly in
the {\rhoc}.

\begin{eqnarray}
	D_{x} & := & \prefix{x}{y}{(\binpar{\outputp{x}{y}}{@{y}})} \nonumber\\
	\bangp_{x}{P} & := & \binpar{{x}!\langle{\binpar{D_{x}}{P}}\rangle}{D_{x}} \nonumber
\end{eqnarray}

\begin{eqnarray}
	\bangp_{x}{P} & & \nonumber\\
	=
	& {x}!\langle{(\prefix{x}{y}{(\outputp{x}{y} | @{y})) | P}}\rangle 
	      | \prefix{x}{y}{(\outputp{x}{y} | @{y})} & \nonumber\\
	\red
	& (\outputp{x}{y} | @{y})\substn{\quotep{(\prefix{x}{y}{(@{y} | \outputp{x}{y})) | P}}}{y} & \nonumber\\
	=
	& \outputp{x}{\quotep{(\prefix{x}{y}{(\outputp{x}{y} | @{y})) | P}}}
	  | {(\prefix{x}{y}{(\outputp{x}{y} | @{y})) | P}} & \nonumber\\
	\red
	& \ldots & \nonumber\\
	\red^*
	& P | P | \ldots & \nonumber
\end{eqnarray}

Of course, this encoding, as an implementation, runs away, unfolding
$\bangp{P}$ eagerly. A lazier and more implementable replication
operator, restricted to input-guarded processes, may be obtained as follows.

\begin{eqnarray}
\bangp{\prefix{u}{v}{P}} 
	:= 
	\binpar{\lift{x}{\prefix{u}{v}{(\binpar{D(x)}{P})}}}{D(x)} \nonumber
\end{eqnarray}

\begin{remark}
  Note that the lazier definition still does not deal with summation
  or mixed summation (i.e. sums over input and output). The reader is
  invited to construct definitions of replication that deal with these
  features. 

  Further, the definitions are parameterized in a name, $x$. Can you,
  gentle reader, make a definition that eliminates this parameter and
  guarantees no accidental interaction between the replication
  machinery and the process being replicated -- i.e. no accidental
  sharing of names used by the process to get its work done and the
  name(s) used by the replication to effect copying. This latter
  revision of the definition of replication is crucial to obtaining
  the expected identity $!!P \sim !P$.
\end{remark}

\begin{remark}\label{rem:paradoxical_combinator}
  The reader familiar with the lambda calculus will have noticed the
  similarity between $D$ and the paradoxical combinator.

  [Ed. note: the existence of this seems to suggest we have to be more
  restrictive on the set of processes and names we admit if we are to
  support no-cloning.]
\end{remark}

\subsubsection{Bisimulation}

The computational dynamics gives rise to another kind of equivalence,
the equivalence of computational behavior. As previously mentioned
this is typically captured \emph{via} some form of bisimulation.

% The notion we use in this paper is weak barbed bisimulation
% \cite{milner91polyadicpi}.

The notion we use in this paper is derived from weak barbed
bisimulation \cite{milner91polyadicpi}. 

\begin{definition}
An \emph{observation relation}, $\downarrow_{\mathcal N}$, over a set
of names, $\mathcal N$, is the smallest relation satisfying the rules
below.

\infrule[Out-barb]{y \in {\mathcal N}, \; x \nameeq y}
		  {\outputp{x}{v} \downarrow_{\mathcal N} x}
\infrule[Par-barb]{\mbox{$P\downarrow_{\mathcal N} x$ or $Q\downarrow_{\mathcal N} x$}}
		  {\binpar{P}{Q} \downarrow_{\mathcal N} x}

We write $P \Downarrow_{\mathcal N} x$ if there is $Q$ such that 
$P \wred Q$ and $Q \downarrow_{\mathcal N} x$.
\end{definition}

\begin{definition}
%\label{def.bbisim}
An  ${\mathcal N}$-\emph{barbed bisimulation} over a set of names, ${\mathcal N}$, is a symmetric binary relation 
${\mathcal S}_{\mathcal N}$ between agents such that $P\rel{S}_{\mathcal N}Q$ implies:
\begin{enumerate}
\item If $P \red P'$ then $Q \wred Q'$ and $P'\rel{S}_{\mathcal N} Q'$.
\item If $P\downarrow_{\mathcal N} x$, then $Q\Downarrow_{\mathcal N} x$.
\end{enumerate}
$P$ is ${\mathcal N}$-barbed bisimilar to $Q$, written
$P \wbbisim_{\mathcal N} Q$, if $P \rel{S}_{\mathcal N} Q$ for some ${\mathcal N}$-barbed bisimulation ${\mathcal S}_{\mathcal N}$.
\end{definition}

$\mathcal{R} \subseteq \pi \times \pi$

$P \mathcal{R} Q => \forall P'. P \red P' \Rightarrow \exists Q'. Q \red Q', P' \mathcal{R} Q'$

$P \vdash x \Rightarrow Q \vdash x$

\begin{mathpar}
  \inferrule*[lab=Out-barb]{x \nameeq y}{{y}!\langle{Q}\rangle \vdash x}
  \and
  \inferrule*[lab=Par-barb]{\mbox{$P\vdash x$ or $Q\vdash x$}}{\binpar{P}{Q} \vdash x}
\end{mathpar}

\subsubsection{Contexts}

One of the principle advantages of computational calculi like the
$\pi$-calculus is a well-defined notion of context,
contextual-equivalence and a correlation between
contextual-equivalence and notions of bisimulation. The notion of
context allows the decomposition of a process into (sub-)process and
its syntactic environment, its context. Thus, a context may be
thought of as a process with a ``hole'' (written $\Box$) in it. The
application of a context $M$ to a process $P$, written $M[P]$, is
tantamount to filling the hole in $M$ with $P$. In this paper we do
not need the full weight of this theory, but do make use of the notion
of context in the proof the main theorem. 

\begin{mathpar}
  \inferrule* [lab=summation] {} {{M_{M},M_{N}} \bc \Box \;|\; x.M_{A} \;|\; M_{M}+M_{N}}
  \and
  \inferrule* [lab=agent] {} {{M_{A}} \bc (\vec{x})M_{P} \;| \; \clift{P_0,\ldots,M_{P},\ldots,P_N}}
  \and \\
  \inferrule* [lab=process] {} {{M_{P}} \bc M_{N} \;| \;P|M_{P} }
\end{mathpar} 

\begin{mathpar}
  \inferrule* [lab=sychronization] {} {M_{N} \bc \Box \;|\; x?M_{F} \;|\; x!M_{C}}
  \and
  \inferrule* [lab=abstraction] {} {{M_{F}} \bc (x)M_{P} }
  \and
  \inferrule* [lab=concretion] {} {{M_{C}} \bc \langle M_{P} \rangle }
  \and \\
  \inferrule* [lab=process] {} {{M_{P}} \bc M_{N} \;| \;P|M_{P} }
\end{mathpar}

\begin{definition}[contextual application] Given a context $M$, and
  process $P$, we define the \emph{contextual application}, $M[P] :=
  M\{P/\Box\}$. That is, the contextual application of M to P is the
  substitution of $P$ for $\Box$ in $M$.
\end{definition}

$\meaningof{-} : L \to \mathcal{P}(\pi)$

\begin{mathpar}
  \inferrule* [lab=collection] {} {\meaningof{true} = \pi, \and \meaningof{~E} = \pi \setminus \meaningof{E}, \and \meaningof{E_{1} \& E_{2}} = \meaningof{E_{1}} \cap \meaningof{E_{2}}}
\end{mathpar}

\begin{mathpar}
  \inferrule* [lab=structure] {} {\meaningof{0} = \{ P \in \pi | P \equiv 0 \}, \and \\ \meaningof{E_1 | E_2} = \{ P \in \pi | P \equiv P_{1} | P_{2}, P_{1} \in \meaningof{E_{1}}, P_{2} \in \meaningof{E_2}\} }
\end{mathpar}

\begin{mathpar}
 \inferrule* [lab=behavior] {} {\meaningof{\langle a?b \rangle E} = \{ P \in \pi | P \equiv Q | u?(y)P', \\ \and \\\\ \and \\ \;\;\; u \in \meaningof{a}, \forall z.P'\{z/y\} \in \meaningof{E\{z/b\}}\}, \and \\ \meaningof{a!E} = \{ P \in \pi | P \equiv Q | x!\langle P' \rangle, x \in \meaningof{a} P' \in \meaningof{E}\} }
\end{mathpar}

\begin{mathpar}
 \inferrule* [lab=nominal] {} {\meaningof{\quotep{E}} = \{ \quotep{P} \in \quotep{\pi} | P \in \meaningof{E} \}, \and \meaningof{\quotep{P}} = \{ \quotep{Q} \in \quotep{\pi} | P \equiv Q \} \and \\ \meaningof{@\quotep{E}} = \{ P \in \pi | P \equiv @x, x \in \meaningof{E} \}}
\end{mathpar}

\begin{eqnarray*}
  \\
  \meaningof{-} : TS \to ST
\end{eqnarray*}

\begin{eqnarray*}
  \\
  L : TS \to ST
\end{eqnarray*}

\begin{eqnarray*}
  \\
  P \models E \iff P \in \meaningof{E}
\end{eqnarray*}

\begin{eqnarray*}
  P \approx_{L} Q \iff \forall E \in L. P \models E \iff Q \models E
\end{eqnarray*}

\begin{eqnarray*}
  P \approx_{K} Q
\end{eqnarray*}

\begin{eqnarray*}
  P \approx Q
\end{eqnarray*}

$\approx_{K} = \approx = \approx_{L}$

\subsubsection{Contextual duality}

Note that contexts extend the quotation operation to a family of
operations from processes to names. Given a context, $M$, we can
define a \emph{nominal context}, $\quotep{M}$ by $\quotep{M}[P] :=
\quotep{M[P]}$. To foreshadow what is to come we observe that these
operations enjoy a duality with processes very much like the duality
between vectors and maps from vectors to scalars.

Further, because the calculus is essentially higher-order, we have a
correspondence between contexts and processes. More specifically,
given a name $x$ and a context $M$ we can construct $M^{*}_{x}$ such
that 

\begin{mathpar}
  M^{*}_{x} | \lift{x}{P} \red M[P]
\end{mathpar}

namely,

\begin{mathpar}
  M^{*}_{x} := x?(u).M[\dropn{u}]
\end{mathpar}

The dependence of $M^{*}_{x}$ on a name makes it an abstraction, 

\begin{mathpar}
  M^{*} := (x)x?(u).M[\dropn{u}]
\end{mathpar}

\subsection{Additional notation}

It will sometimes be convenient to denote the process a name
quotes. We already have the notation $x = \quotep{P}$, but it will be
convenient to introduce an alternate notation, $\procn{x}$, when we
want to emphasize the connection to the use of the name. Note that, by
virtue of name equivalence, $\quotep{\procn{x}} \nameeq x$; so, the
notation is consistent with previous definitions.

Further, because names have structure it is possible to effect
substitutions on the basis of that structure. This means we need to
upgrade our notation for substitutions, which we accomplish by
adapting comprehension notation. Thus,

\begin{mathpar}
  P\{ y / x : x \in S \}
\end{mathpar}

is interpreted to mean the process derived from P by replacing (in a
capture-avoiding manner) each occurrence of $x$ in $S$ by $y$. For example,

\begin{mathpar}
  P\{ \quotep{\procn{x}|\procn{x}} / x : x \in \freenames{P} \}
\end{mathpar}

will replace each (occurrence) of a free name $x$ in $P$ by
$\quotep{\procn{x}|\procn{x}}$.

Also, we will avail ourselves of the notation $x^{L}$ and $x^{R}$ to
denote injections of a name into disjoint copies of the name
space. There are numerous ways to accomplish this. One example can be
found in \cite{MeredithR05}. This notation overloads to vectors of
names: $\vec{x}^{\pi} := (x_{i}^{\pi} \; : \; 0 \leq i < |\vec{x}| )$ where $\pi \in \{L,R\}$.

We also use $P^{\Box} := P|\Box$.

In \cite{MeredithR05} an interpretation of the new operator is
given. It turns out that there are several possible interpretations
all enjoying the requisite algebraic properties of the operator (see
\cite{milner91polyadicpi}). We will therefore make liberal use of
$(\nu\; \vec{x})P$.

% subsection the_syntax_and_semantics_of_the_notation_system (end)   

\section{Interpretation of QM}
\subsection{Supporting definitions}
\subsubsection{Multiplication}
\begin{mathpar}
  \quotep{Q} \cdot \quotep{R} := \quotep{Q|R}
  \and \\
  \quotep{Q} \cdot P := P\{ \quotep{Q|R} / \quotep{R} : \quotep{R} \in \freenames{P} \}
\end{mathpar}

\paragraph{Discussion}
The first line needs little explanation. The second line says that
each free name of the process is replaced with the multiplication of
that name by the scalar. Multiplication of a scalar (name) by a state
(process) results in a process all the names of which have been `moved
over' by parallel composition with the process the scalar
quotes. There is a subtlety that the bound names have to be
manipulated so that multiplied names aren't accidentally
captured. There are many ways to achieve this.

\begin{remark}\label{rem:multiplication_identities}
  The reader is invited to verify that for all $x,y,z \in \QProc$ and $P \in \Proc$
  \begin{mathpar}
    x \cdot \quotep{0} \equiv x 
    \and
    x \cdot y \equiv y \cdot x
    \and
    x \cdot (y \cdot z) \equiv (x \cdot y) \cdot z
    \and \\
    \quotep{0} \cdot P \equiv P
    \and \\
    x \cdot (y \cdot P) \equiv (x \cdot y) \cdot P
    \and \\
    x \cdot (P|Q) \equiv (x \cdot P) | (x \cdot Q)
    \and \\    
  \end{mathpar}
\end{remark}

\subsubsection{Tensor product}

We define a tensor product on processes by structural induction.

\paragraph{Tensor of sums} First note that all summations, including
$\pzero$ and sequence, can be written $\Sigma_{i} x_{i}.A_{i} +
\Sigma_{j} x_{j}.C_{j}$, where we have grouped input-guarded processes
together and output-guarded processes together.

Thus, we can define the tensor product of two summations, $N_{1}\otimes N_{2}$, where

\begin{mathpar}
  N_{1} := \Sigma_{i} x_{i}.A_{i} + \Sigma_{j} x_{j}.C_{j}
  \and
  N_{2} := \Sigma_{i'} y_{i'}.B_{i'} + \Sigma_{j'} y_{j'}.D_{j'} 
\end{mathpar}

as follows.

\begin{mathpar}
  \Sigma_{i} x_{i}.A_{i} + \Sigma_{j} x_{j}.C_{j} \otimes \Sigma_{i'}
  y_{i'}.B_{i'} + \Sigma_{j'} y_{j'}.D_{j'} 
  \and \\
  := \; \Sigma_{i} \Sigma_{i'} \quotep{\stackrel{\vee}{x_{i}}| \stackrel{\vee}{y_{i'}}}.(A_{i}\otimes B_{i'}) \; | \; \Sigma_{i'} \Sigma_{i} \quotep{\stackrel{\vee}{y_{i'}}|\stackrel{\vee}{x_{i}}}.(B_{i'}\otimes A_{i})
  \and
  \;\; | \;\; \Sigma_{j} \Sigma_{j'} \quotep{\stackrel{\vee}{x_{j}}|\stackrel{\vee}{y_{j'}}}.(A_{j}\otimes B_{j'}) \; | \; \Sigma_{j'} \Sigma_{j} \quotep{\stackrel{\vee}{y_{j'}}|\stackrel{\vee}{x_{j}}}.(B_{j'}\otimes A_{j})
\end{mathpar}

\begin{remark}
  Do we need to $x^{L}$ and $y^{R}$ for this construction as well?
\end{remark}

\paragraph{Tensor of parallel compositions} Next, we distribute tensor
over par.

\begin{mathpar}
  P_{1}|P_{2} \otimes Q_{1}|Q_{2} := (P_{1} \otimes Q_{1}) | (P_{1}
  \otimes Q_{2}) | (P_{2} \otimes Q_{1}) | (P_{2} \otimes Q_{2})
\end{mathpar}

\paragraph{Tensor with dropped names} We treat tensor of a
process with a dropped name as parallel composition.

\begin{mathpar}
  P \otimes \dropn{x} := P | \dropn{x}
\end{mathpar}

\paragraph{Tensor of agents}

Finally, we need to define tensor on agents. Note that the definition
of tensor on normal products only tensors inputs with inputs and
outputs with outputs. Thus, we only have to define the operation on
``homogeneous'' pairings.

\begin{mathpar}
  (\vec{x})P \otimes (\vec{y})Q
  \and \\
  := (x_{0}^{L}|y_{0}^{R},\ldots,x_{0}^{L}|y_{n}^{R},\ldots,x_{m}^{L}|y_{0}^{R},\ldots,x_{m}^{L}|y_{n}^R)(P\{ \vec{x}^{L}/\vec{x}\} \otimes Q \{ \vec{y}^{R}/\vec{y}\})
  \and \\
  \clift{\vec{P}} \otimes \clift{\vec{Q}}
  \and \\
  := \clift{P_{0}\otimes Q_{0},\ldots,P_{0}\otimes Q_{n},\ldots,P_{m}\otimes Q_{0},\ldots,P_{m}\otimes Q_{n}}
\end{mathpar}

\begin{remark}
  Observe that arities of tensored abstractions matches arities of
  tensored concretions if the original arities matched. Note also that
  the length of the arities corresponds to the increase in dimension
  we see in ordinary vector space tensor product.
\end{remark}

\begin{remark}
  Operationally, this definition distributes the tensor down to
  components ``linked'' by summation. Tensor over summation is
  intriguing in that it mixes names. Moreover, as a consequence of the
  way it mixes names we have the identities for all $x \in \QProc$ and
  $P,Q \in \Proc$

  \begin{mathpar}
    (x \cdot P) \otimes Q \equiv x \cdot (P \otimes Q) \equiv P \otimes (x \cdot Q)
    \and
    P \otimes \pzero \equiv P
  \end{mathpar}

  that the reader is invited to verify.
\end{remark}

\subsubsection{Annihilation}
\begin{mathpar}
  P^{\perp} := \{ Q | \forall R. P|Q \red^{*} R \Rightarrow R \red^{*} \pzero \}
  \and \\
  P^{\underline{\perp}} := \Sigma_{Q \in P^{\perp}} \quotep{Q}?(y).(\dropn{y}|Q) | \Sigma_{Q \in P^{\perp}} \quotep{Q}\clift{\Box}
\end{mathpar}

\paragraph{Discussion} The reader will note that $P^{\perp}$ is a
\emph{set} of processes, while $P^{\underline{\perp}}$ is a
\emph{context}. We call the set $P^{\perp}$ the \emph{annihilators} of
$P$. The parallel composition of a process in the annihilators of $P$
with $P$ will result in a process, the state space of which has all
paths eventually leading to $\pzero$. Execution may endure loops; but
under reasonable conditions of fairness (naturally guaranteed under
most notions of bisimulation) such a composite process cannot get
stuck in such a loop and will, eventually pop out and terminate.

The context $P^{\underline{\perp}}$ is ready and willing to ``take the
$P$ out of'' the process to which it is applied. It will effectively
transmit the code of the process to which it is applied to one of the
annihilators and run the process against it.

\subsubsection{Evaluation}
We fix $M$ a domain of fully abstract interpretation with an equality
coincident with bisimulation. We take $\meaningof{\cdot} : \Proc \to
M$ to be the map interpreting processes and $\nmeaningof{\cdot} : \M
\to Proc$ to be the map running the other way. Then we define

\begin{mathpar}
  \int P := \nmeaningof{\meaningof{P}}
\end{mathpar}

\paragraph{Discussion}
There are many fully abstract interpretations of Milner's
$\pi$-calculus. Any of them can be used as a basis for interpreting
the reflective calculus here. Equipped with such a domain it is
largely a matter of grinding through to check that the Yoneda
construction for the normalization-by-evaluation program can be
extended to this setting.

\begin{remark}
  The reader is invited to verify that $\int (P^{\underline{\perp}}[P]) = 0$.
\end{remark}

\subsection{Quantum mechanics}

Table \ref{tbl:core_qm_op_defns} gives the core operational definitions

\begin{table}[htp]\label{tbl:core_qm_op_defns}
  \center{
    \fbox{
      \begin{tabular}{c|c}
        quantum mechanics & process calculus \\
        \hline
        scalar & $x := \quotep{P}$ \\
        state vector & $\state{P} := P$ \\
        dual & $\state{P}^{*} := \event{P^{\underline{\perp}}} := \quotep{P^{\underline{\perp}}}[-]$ \\
        matrix & $ \Sigma_{\alpha} \state{P_{\alpha}}x_{\alpha}\event{Q_{\alpha}}$ \\
        vector addition & $\state{P} + \state{Q} := \state{P | Q}$ \\
        tensor product & $\state{P} \otimes \state{Q} := \state{P \otimes Q}$ \\
        inner product & $\innerprod{P}{Q} := \quotep{\int P^{\underline{\perp}}[Q]}$ \\
      \end{tabular}
    }
  }
  \caption{QM - operational definitions}
\end{table}

where

\begin{mathpar}
  \prmatrix{P}{Q} := \fprmatrix{P}{\quotep{\pzero}}{Q}
  \and
  \fprmatrix{P}{x}{Q} := (\state{P},x,\event{Q})
  \and
  (\fprmatrix{P}{x}{Q})(\state{R}) := x \cdot \innerprod{Q}{R} \cdot \state{P}
  \and
  (\fprmatrix{P}{x}{Q})(\event{R}) := x \cdot \innerprod{R}{P} \cdot \event{Q}
\end{mathpar}

\paragraph{Discussion}
As promised: vectors (aka states) are represented as processes; duals
as contextual duals; inner product definition should be compared with
standard inner product definition for ....

\begin{remark}
  Assuming $\int (P^{\underline{\perp}}[P]) = 0$, the reader is
  invited to verify that $(\fprmatrix{P}{x}{P})(\state{P}) = x \cdot \state{P}$.
\end{remark}

\begin{remark}
  The reader is invited to verify that $\innerprod{P}{Q}$ could
  equally well have been written $\quotep{\int \stackrel{\vee}{x}}$
  where $x = \event{P^{\underline{\perp}}}(Q)$.

  One of the motivations for this remark is that there is another way
  to factor these operations. We could package up evaluation in the dual:

  \begin{mathpar}
    \state{P}^{*} := \event{\int P^{\underline{\perp}}} := \quotep{\int P^{\underline{\perp}}}[-]
  \end{mathpar}

  and then have inner product defined by
  
  \begin{mathpar}
    \innerprod{P}{Q} := \event{P}(Q)
  \end{mathpar}

  Hopefully, experience with the calculations will provide guidance on
  the best factoring.
\end{remark}

\begin{remark}
  Assuming $\int (P^{\underline{\perp}}[P]) = 0$, the reader is
  invited to verify that $\forall P,Q. (\prmatrix{0}{Q})(\state{0}) =
  \state{0}$ and dually $(\prmatrix{P}{0})(\event{0}) = \event{0}$.
\end{remark}

\begin{remark}
  i'm a little worried that i don't (yet) have proper support for
  complex conjugacy. But, the observation above may give us a
  clue. According to Abramsky, it must be the case that the scalars
  are iso to the homset of the identity for the tensor -- which the
  observation above characterizes. 

  For now, we will simply bookmark the notion with $\overline{x}$.
\end{remark}

\subsubsection{Adjointness}

We need to give a definition of $(\cdot)^{\dagger}$ for matrices. The
obvious candidate definition is
\begin{mathpar}
(\Sigma_{\alpha}\fprmatrix{P_{\alpha}}{x_{\alpha}}{Q_{\alpha}})^{\dagger}
= \Sigma_{\alpha}\fprmatrix{(Q_{\alpha}^{\underline{\perp}})^{*}}{\overline{x}_{\alpha}}{P_{\alpha}^{\underline{\perp}}} 
\end{mathpar}

But, $(Q_{\alpha}^{\underline{\perp}})^{*}$ requires a name along
which to communicate the process to achieve the context application.

\subsubsection{Basis for a basis}
If processes label states and ``addition'' of states (a.k.a. vector
addition) is interpreted as parallel composition, what corresponds to
notions of linear independence and basis? Here, we recall that Yoshida
has developed a set of \emph{combinators} for an asynchronous verison
of Milner's $\pi$-calculus. These are a finite set of processes such
any process can be expressed as parallel composition of these
combinators together with liberal uses of the new operator and
replication. We can simply give a translation of these into the
present calculus and have reasonable expectation that the property
carries over. That is, that the resultant set allows to express all
processes via parallel composition. Note, however, that there is no
new operator or replication in this calculus. As a result, we expect
that the corresponding set is actually infinite. That is, we expect
that the space is actually infinite dimensional.

\begin{remark}
  The attentive reader may be a bit concerned. Certainly, the
  collection $S$, $K$ and $I$ is a finite set of
  combinators. Shouldn't we expect to see a finite set of combinators
  for an effectively equivalent system? i am very sympathetic to this
  critique and feel it warrants full attention. On the other hand, i
  also have in mind the following analogy. The natural numbers, as a
  monoid under addition, has exactly $1$ generator, while the natural
  numbers, as a monoid under multiplication, has countably many
  generators (the primes). We observe that the application of the
  lambda calculus is much less resource sensitive than the parallel
  composition of the $\pi$-calculus. Could it be the case that we have
  an analogy of the form
  
  \begin{mathpar}
    m + n : MN :: m*n : M|N
  \end{mathpar}

  giving a similar blow up in the set of ``primes''?  This is such a
  wonderful thought that, even if it's not true, i think it's worth
  writing down.
\end{remark}
 

\documentclass[12pt]{llncs}
%\documentclass{jktr}

\usepackage[pdftex]{hyperref}                   
\usepackage {listings}
\usepackage {mathpartir}
\usepackage{bcprules}
%\usepackage{listings}
                       
\usepackage{graphicx} 
%\usepackage[margins=2.5cm,nohead,nofoot]{geometry}
%\usepackage{geometry}
\usepackage{amsfonts}
\usepackage{amstext}
\usepackage{latexsym}
\usepackage{amssymb}
\usepackage{color}


%\include{myPreamble}
\include{qm2pi.local} 

%\ifpdf
%\usepackage[pdftex]{graphicx}
%\else
%\usepackage{graphicx}
%\fi

 % \ifpdf
%  \usepackage{pdfsync}
%  \if


%\title{Brief Article}
%\author{David F. Snyder}
%\author{L.G. Meredith}

%\address{Dept. of Math., Texas State University--San Marcos, San Marcos, TX 78666}
       
\pagestyle{empty}


\begin{document}

\lstset{language=[Objective]Caml,frame=shadowbox}

\input{qm2pi.front}

% section front matter (end)

\input{qm2pi.intro} 
 
% section introduction (end)

% \input{qm2pi.knotations} 

% section notation (end)

\input{qm2pi.process.calculi} 

% section concurrent_process_calculi_and_spatial_logics_ (end)
    
%\input{qm2pi.knots2pi} 

%\input{qm2pi.trefoil} 

%\input{qm2pi.mainthm} 

% subsection basic_interpretation (end)

%\input{qm2pi.rho.presentation} 
\subsection{The syntax and semantics of the notation system}\label{sub:the_syntax_and_semantics_of_the_notation_system} % (fold)

We now summarize a technical presentation of the calculus that
embodies our theory of dynamics. The typical presentation of such a
calculus follows the style of giving generators and relations on
them. The grammar, below, describing term constructors, freely
generates the set of processes, $\Proc$. This set is then quotiented
by a relation known as structural congruence and it is over this set
that the notion of dynamics is expressed. This presentation is
essentially that of \cite{MeredithR05} with the addition of
polyadicity and summation. For readability we have relegated some of
the technical subtleties to an appendix.

\subsubsection{Process grammar}\label{subsub:process_grammar}

\begin{mathpar}
  \inferrule* [lab=synchronization] {} {{M} \bc \pzero \;|\; x?F \;|\; x!C }
  \and
  \inferrule* [lab=abstraction] {} {{F} \bc (x)P}
  \and
  \inferrule* [lab=concretion] {} {{C} \bc \langle Q \rangle}
  \and
  \inferrule* [lab=process] {} {{P,Q} \bc M \;| \;P|Q \;|\; @{x}}
  \and
  \inferrule* [lab=name] {} {{x} \bc \quotep{P}}
\end{mathpar} 

Note that $\vec{x}$ (resp. $\vec{P}$) denotes a vector of names
(resp. processes) of length $|\vec{x}|$ (resp. $|\vec{P}|$). We adopt
the following useful abbreviations.

\begin{mathpar}
   x?(\vec{y}).P := x.(\vec{y})P \and  x\clift{\vec{P}} := x.\clift{\vec{P}}
   \and x!(y) := \lift{x}{\dropn{y}}
   \and \Pi_{i=0}^{n-1}P_i := P_0 | \ldots | P_{n-1}
\end{mathpar}

\subsubsection{Structural congruence}

\paragraph{Free and bound names and alpha-equivalence.} At the
core of structural equivalence is alpha-equivalence which identifies
process that are the same up to a change of variable. Formally, we
recognize the distinction between free and bound names. The free names
of a process, $\freenames{P}$, may be calculated recursively as
follows:

\begin{mathpar}
\freenames{\pzero} := \emptyset
  \and \\
  \freenames{x?(y).P} := \{ x \} \cup (\freenames{P} \setminus \{ y \})
  \and 
  \freenames{x!\langle P \rangle} := \{ x \} \cup \{ P \} 
  \and \\
  \freenames{P|Q} := \freenames{P} \cup \freenames{Q}
  \and \\
  \freenames{@{x}} := \{ x \}
\end{mathpar}

$\pi$
$\quotep{\pi}$

$\freenames{-} : \pi \to \mathcal{P}(\quotep{\pi})$

\begin{eqnarray*}
  \freenames{\pzero} & := & \emptyset \\
  \freenames{x?(y).P} & := & \{ x \} \cup (\freenames{P} \setminus \{ y \}) \\
  \freenames{x!\langle P \rangle} & := & \{ x \} \cup \{ P \} \\
  \freenames{P|Q} & := & \freenames{P} \cup \freenames{Q} \\
  \freenames{\dropn{x}} & := & \{ x \}
\end{eqnarray*}

The bound names of a process, $\boundnames{P}$, are those names occurring in $P$
that are not free. For example, in $x?(y).0$, the name $x$ is free, while $y$ is bound.

\begin{mathpar}
  \inferrule* [lab=monoidal-laws] {} { P|Q \equiv Q|P \and P|0 \equiv P \and P|(Q|R) \equiv (P|Q)|R }
\end{mathpar}

\begin{mathpar}
  \inferrule* [lab=alpha-equivalence] {} { (x)P \equiv (y)P\{y/x\} \and y \not\in \freenames{P} }
\end{mathpar}

\begin{definition}
Then two processes, $P,Q$, are alpha-equivalent if $P = Q\{\vec{y}/\vec{x}\}$ for
some $\vec{x} \in \boundnames{Q},\vec{y} \in \boundnames{P}$, where $Q\{\vec{y}/\vec{x}\}$
denotes the capture-avoiding substitution of $\vec{y}$ for $\vec{x}$ in $Q$.
\end{definition}

\begin{definition}
  The {\em structural congruence} \cite{SangiorgiWalker} , $\equiv$,
  between processes is the least congruence containing
  alpha-equivalence, satisfying the abelian monoid laws
  (associativity, commutativity and $\pzero$ as identity) for parallel
  composition $|$ and for summation $+$.
\end{definition}

\subsection{Name equivalence}

We take name equivalence, written $\nameeq$, to be the smallest
equivalence relation generated by the following rules.

\begin{mathpar}
\inferrule*[lab=Quote-drop]
{ }
{ \quotep{@{x}} \nameeq x }

\inferrule*[lab=Struct-equiv]
{ P \scong Q }
{ \quotep{P} \nameeq \quotep{Q} }
\end{mathpar}

The astute reader will have noticed that the mutual recursion of names
and processes imposes a mutual recursion on alpha-equivalence and
structural equivalence via name-equivalence. Fortunately, all of this
works out pleasantly and we may calculate in the natural way, free of
concern. The reader interested in the details is referred to the
appendix \ref{appendix:rho_details}.

\subsection{Substitution}

We use $\Proc$ for the set of processes, $\QProc$ for the set of
names, and $\id{\{}\vec{y} / \vec{x} \id{\}}$ to denote partial maps,
$s : \QProc \rightarrow \QProc$. A map, $s$ lifts, uniquely, to a map
on process terms, $\widehat{s} : \Proc \rightarrow \Proc$ by the
following equations.

\begin{mathpar}
  (0) \psubstp{Q}{P} := 0 \\
  (R \juxtap S) \psubstp{Q}{P}
  :=    
  (R)\psubstp{Q}{P} \juxtap (S) \psubstp{Q}{P} \\
  (x?(y).R) \psubstp{Q}{P}    
  :=    
  (x)\substp{Q}{P} (z)\concat( (R \psubstn{z}{y}) \psubstp{Q}{P} ) \\
  (\lift{x}{R}) \psubstp{Q}{P}  
  :=
  \lift{(x)\substp{Q}{P}}{ R \psubstp{Q}{P} } \\
%   (\dropn{x})  \psubstp{Q}{P}       
%   := 
%   \left\{ 
%     \begin{array}{ccc} 
%       \dropn{\quotep{Q}} & & x \nameeq \quotep{P} \\
%       \dropn{x} & & otherwise \\
%     \end{array}
%   \right. 
  (\dropn{x})  \psubstp{Q}{P}       
  := 
  \left\{ 
    \begin{array}{ccc} 
      Q & & x \nameeq \quotep{P} \\
      \dropn{x} & & otherwise \\
    \end{array}
  \right.
\end{mathpar}
 

where

\begin{eqnarray}
  (x)\id{\{} \lpquote Q \rpquote / \lpquote P \rpquote \id{\}}            = 
  \left\{ 
    \begin{array}{ccc}
      \lpquote Q \rpquote & & x \nameeq \lpquote P \rpquote \\
      x & & otherwise \\
    \end{array}
  \right. \nonumber
\end{eqnarray}

and $z$ is chosen distinct from $\quotep{P}$, $\quotep{Q}$, the free
names in $Q$, and all the names in $R$. Our $\alpha$-equivalence will
be built in the standard way from this substitution.

\begin{remark}\label{rem:no_self_referential_names}
  One consequence of these definitions is that $\forall P. \quotep{P}
  \not\in \freenames{P}$.
\end{remark}

\subsection{ Dynamic quote: an example }

Anticipating something of what's to come, consider applying the
substitution, $\widehat{\id{\{}u / z \id{\}}}$, to the following pair
of processes, $\lift{w}{y!(z)}$ and $w[ \lpquote y!(z) \rpquote ]$.

\begin{eqnarray}
	\lift{w}{y!(z)}\widehat{\id{\{}u / z \id{\}}}
		& = &
		\lift{w}{y!(u)} \nonumber\\
	w[ \lpquote y!(z) \rpquote ] \widehat{ \id{\{}u / z \id{\}} }
		& = &
		w[ \lpquote y!(z) \rpquote ] \nonumber
\end{eqnarray}

Because the body of the process between quotes is impervious to
substitution, we get radically different answers. In fact, by
examining the first process in an input context,
e.g. $x?(z).\lift{w}{y!(z)}$, we see that the process under the lift
operator may be shaped by prefixed inputs binding a name inside it. In
this sense, the lift operator will be seen as a way to dynamically
construct processes before reifying them as names.

Finally equipped with these standard features we can present the
dynamics of the calculus.

\subsubsection{Operational semantics} 

Finally, we introduce the computational dynamics. What marks these
algebras as distinct from other more traditionally studied algebraic
structures, e.g. vector spaces or polynomial rings, is the manner in
which dynamics is captured. In traditional structures, dynamics is typically
expressed through morphisms between such structures, as in linear maps
between vector spaces or morphisms between rings. In algebras
associated with the semantics of computation, the dynamics is
expressed as part of the algebraic structure itself, through a
reduction reduction relation typically denoted by $\red$. Below, we
give a recursive presentation of this relation for the calculus used
in the encoding.

$\red \subseteq \pi \times \pi$
$\red : \pi \to \mathcal{P}(\pi)$

\begin{mathpar}
  \inferrule* [lab=Comm] { \textsf{match}( x_{src}, x_{trgt} ) } { x_{trgt}?(y)P \; | \; x_{src}!\langle {Q} \rangle \red P\{\quotep{Q}/y}\} }
  \and \\
  \inferrule* [lab=Par] {{P} \red {P}'} {{{P} | {Q}} \red {{P}' | {Q}}}
  \and
  \inferrule* [lab=Equiv]{{{P} \scong {P}'} \andalso {{P}' \red {Q}'} \andalso {{Q}' \scong {Q}}}{{P} \red {Q}}
\end{mathpar}

\begin{eqnarray*}
  match_{\equiv} (\quotep{P},\quotep{Q}) & := & P \equiv Q \\
  match_{\dagger}(\quotep{P},\quotep{Q}) & := & \forall R. P|Q \red^{*} R => R \red^{*} 0 \\
  match_{K}(\quotep{P},\quotep{Q}) & := & K \mbox{ for some context } K
\end{eqnarray*}

$u?(x)P | u!\langle Q \rangle \red P\{\quotep{Q}/x\}$

%We write $\wred$ for $\red^*$, and $P\red$ if $\exists Q $ such that $ P \red Q$.
We write $P\red$ if $\exists Q $ such that $ P \red Q$ and $P\not\red$, otherwise.

\section{Replication}

As mentioned before, it is known that replication (and hence
recursion) can be implemented in a higher-order process algebra
\cite{SangiorgiWalker}. As our first example of calculation with the
machinery thus far presented we give the construction explicitly in
the {\rhoc}.

\begin{eqnarray}
	D_{x} & := & \prefix{x}{y}{(\binpar{\outputp{x}{y}}{@{y}})} \nonumber\\
	\bangp_{x}{P} & := & \binpar{{x}!\langle{\binpar{D_{x}}{P}}\rangle}{D_{x}} \nonumber
\end{eqnarray}

\begin{eqnarray}
	\bangp_{x}{P} & & \nonumber\\
	=
	& {x}!\langle{(\prefix{x}{y}{(\outputp{x}{y} | @{y})) | P}}\rangle 
	      | \prefix{x}{y}{(\outputp{x}{y} | @{y})} & \nonumber\\
	\red
	& (\outputp{x}{y} | @{y})\substn{\quotep{(\prefix{x}{y}{(@{y} | \outputp{x}{y})) | P}}}{y} & \nonumber\\
	=
	& \outputp{x}{\quotep{(\prefix{x}{y}{(\outputp{x}{y} | @{y})) | P}}}
	  | {(\prefix{x}{y}{(\outputp{x}{y} | @{y})) | P}} & \nonumber\\
	\red
	& \ldots & \nonumber\\
	\red^*
	& P | P | \ldots & \nonumber
\end{eqnarray}

Of course, this encoding, as an implementation, runs away, unfolding
$\bangp{P}$ eagerly. A lazier and more implementable replication
operator, restricted to input-guarded processes, may be obtained as follows.

\begin{eqnarray}
\bangp{\prefix{u}{v}{P}} 
	:= 
	\binpar{\lift{x}{\prefix{u}{v}{(\binpar{D(x)}{P})}}}{D(x)} \nonumber
\end{eqnarray}

\begin{remark}
  Note that the lazier definition still does not deal with summation
  or mixed summation (i.e. sums over input and output). The reader is
  invited to construct definitions of replication that deal with these
  features. 

  Further, the definitions are parameterized in a name, $x$. Can you,
  gentle reader, make a definition that eliminates this parameter and
  guarantees no accidental interaction between the replication
  machinery and the process being replicated -- i.e. no accidental
  sharing of names used by the process to get its work done and the
  name(s) used by the replication to effect copying. This latter
  revision of the definition of replication is crucial to obtaining
  the expected identity $!!P \sim !P$.
\end{remark}

\begin{remark}\label{rem:paradoxical_combinator}
  The reader familiar with the lambda calculus will have noticed the
  similarity between $D$ and the paradoxical combinator.

  [Ed. note: the existence of this seems to suggest we have to be more
  restrictive on the set of processes and names we admit if we are to
  support no-cloning.]
\end{remark}

\subsubsection{Bisimulation}

The computational dynamics gives rise to another kind of equivalence,
the equivalence of computational behavior. As previously mentioned
this is typically captured \emph{via} some form of bisimulation.

% The notion we use in this paper is weak barbed bisimulation
% \cite{milner91polyadicpi}.

The notion we use in this paper is derived from weak barbed
bisimulation \cite{milner91polyadicpi}. 

\begin{definition}
An \emph{observation relation}, $\downarrow_{\mathcal N}$, over a set
of names, $\mathcal N$, is the smallest relation satisfying the rules
below.

\infrule[Out-barb]{y \in {\mathcal N}, \; x \nameeq y}
		  {\outputp{x}{v} \downarrow_{\mathcal N} x}
\infrule[Par-barb]{\mbox{$P\downarrow_{\mathcal N} x$ or $Q\downarrow_{\mathcal N} x$}}
		  {\binpar{P}{Q} \downarrow_{\mathcal N} x}

We write $P \Downarrow_{\mathcal N} x$ if there is $Q$ such that 
$P \wred Q$ and $Q \downarrow_{\mathcal N} x$.
\end{definition}

\begin{definition}
%\label{def.bbisim}
An  ${\mathcal N}$-\emph{barbed bisimulation} over a set of names, ${\mathcal N}$, is a symmetric binary relation 
${\mathcal S}_{\mathcal N}$ between agents such that $P\rel{S}_{\mathcal N}Q$ implies:
\begin{enumerate}
\item If $P \red P'$ then $Q \wred Q'$ and $P'\rel{S}_{\mathcal N} Q'$.
\item If $P\downarrow_{\mathcal N} x$, then $Q\Downarrow_{\mathcal N} x$.
\end{enumerate}
$P$ is ${\mathcal N}$-barbed bisimilar to $Q$, written
$P \wbbisim_{\mathcal N} Q$, if $P \rel{S}_{\mathcal N} Q$ for some ${\mathcal N}$-barbed bisimulation ${\mathcal S}_{\mathcal N}$.
\end{definition}

$\mathcal{R} \subseteq \pi \times \pi$

$P \mathcal{R} Q => \forall P'. P \red P' \Rightarrow \exists Q'. Q \red Q', P' \mathcal{R} Q'$

$P \vdash x \Rightarrow Q \vdash x$

\begin{mathpar}
  \inferrule*[lab=Out-barb]{x \nameeq y}{{y}!\langle{Q}\rangle \vdash x}
  \and
  \inferrule*[lab=Par-barb]{\mbox{$P\vdash x$ or $Q\vdash x$}}{\binpar{P}{Q} \vdash x}
\end{mathpar}

\subsubsection{Contexts}

One of the principle advantages of computational calculi like the
$\pi$-calculus is a well-defined notion of context,
contextual-equivalence and a correlation between
contextual-equivalence and notions of bisimulation. The notion of
context allows the decomposition of a process into (sub-)process and
its syntactic environment, its context. Thus, a context may be
thought of as a process with a ``hole'' (written $\Box$) in it. The
application of a context $M$ to a process $P$, written $M[P]$, is
tantamount to filling the hole in $M$ with $P$. In this paper we do
not need the full weight of this theory, but do make use of the notion
of context in the proof the main theorem. 

\begin{mathpar}
  \inferrule* [lab=summation] {} {{M_{M},M_{N}} \bc \Box \;|\; x.M_{A} \;|\; M_{M}+M_{N}}
  \and
  \inferrule* [lab=agent] {} {{M_{A}} \bc (\vec{x})M_{P} \;| \; \clift{P_0,\ldots,M_{P},\ldots,P_N}}
  \and \\
  \inferrule* [lab=process] {} {{M_{P}} \bc M_{N} \;| \;P|M_{P} }
\end{mathpar} 

\begin{mathpar}
  \inferrule* [lab=sychronization] {} {M_{N} \bc \Box \;|\; x?M_{F} \;|\; x!M_{C}}
  \and
  \inferrule* [lab=abstraction] {} {{M_{F}} \bc (x)M_{P} }
  \and
  \inferrule* [lab=concretion] {} {{M_{C}} \bc \langle M_{P} \rangle }
  \and \\
  \inferrule* [lab=process] {} {{M_{P}} \bc M_{N} \;| \;P|M_{P} }
\end{mathpar}

\begin{definition}[contextual application] Given a context $M$, and
  process $P$, we define the \emph{contextual application}, $M[P] :=
  M\{P/\Box\}$. That is, the contextual application of M to P is the
  substitution of $P$ for $\Box$ in $M$.
\end{definition}

$\meaningof{-} : L \to \mathcal{P}(\pi)$

\begin{mathpar}
  \inferrule* [lab=collection] {} {\meaningof{true} = \pi, \and \meaningof{~E} = \pi \setminus \meaningof{E}, \and \meaningof{E_{1} \& E_{2}} = \meaningof{E_{1}} \cap \meaningof{E_{2}}}
\end{mathpar}

\begin{mathpar}
  \inferrule* [lab=structure] {} {\meaningof{0} = \{ P \in \pi | P \equiv 0 \}, \and \\ \meaningof{E_1 | E_2} = \{ P \in \pi | P \equiv P_{1} | P_{2}, P_{1} \in \meaningof{E_{1}}, P_{2} \in \meaningof{E_2}\} }
\end{mathpar}

\begin{mathpar}
 \inferrule* [lab=behavior] {} {\meaningof{\langle a?b \rangle E} = \{ P \in \pi | P \equiv Q | u?(y)P', \\ \and \\\\ \and \\ \;\;\; u \in \meaningof{a}, \forall z.P'\{z/y\} \in \meaningof{E\{z/b\}}\}, \and \\ \meaningof{a!E} = \{ P \in \pi | P \equiv Q | x!\langle P' \rangle, x \in \meaningof{a} P' \in \meaningof{E}\} }
\end{mathpar}

\begin{mathpar}
 \inferrule* [lab=nominal] {} {\meaningof{\quotep{E}} = \{ \quotep{P} \in \quotep{\pi} | P \in \meaningof{E} \}, \and \meaningof{\quotep{P}} = \{ \quotep{Q} \in \quotep{\pi} | P \equiv Q \} \and \\ \meaningof{@\quotep{E}} = \{ P \in \pi | P \equiv @x, x \in \meaningof{E} \}}
\end{mathpar}

\begin{eqnarray*}
  \\
  \meaningof{-} : TS \to ST
\end{eqnarray*}

\begin{eqnarray*}
  \\
  L : TS \to ST
\end{eqnarray*}

\begin{eqnarray*}
  \\
  P \models E \iff P \in \meaningof{E}
\end{eqnarray*}

\begin{eqnarray*}
  P \approx_{L} Q \iff \forall E \in L. P \models E \iff Q \models E
\end{eqnarray*}

\begin{eqnarray*}
  P \approx_{K} Q
\end{eqnarray*}

\begin{eqnarray*}
  P \approx Q
\end{eqnarray*}

$\approx_{K} = \approx = \approx_{L}$

\subsubsection{Contextual duality}

Note that contexts extend the quotation operation to a family of
operations from processes to names. Given a context, $M$, we can
define a \emph{nominal context}, $\quotep{M}$ by $\quotep{M}[P] :=
\quotep{M[P]}$. To foreshadow what is to come we observe that these
operations enjoy a duality with processes very much like the duality
between vectors and maps from vectors to scalars.

Further, because the calculus is essentially higher-order, we have a
correspondence between contexts and processes. More specifically,
given a name $x$ and a context $M$ we can construct $M^{*}_{x}$ such
that 

\begin{mathpar}
  M^{*}_{x} | \lift{x}{P} \red M[P]
\end{mathpar}

namely,

\begin{mathpar}
  M^{*}_{x} := x?(u).M[\dropn{u}]
\end{mathpar}

The dependence of $M^{*}_{x}$ on a name makes it an abstraction, 

\begin{mathpar}
  M^{*} := (x)x?(u).M[\dropn{u}]
\end{mathpar}

\subsection{Additional notation}

It will sometimes be convenient to denote the process a name
quotes. We already have the notation $x = \quotep{P}$, but it will be
convenient to introduce an alternate notation, $\procn{x}$, when we
want to emphasize the connection to the use of the name. Note that, by
virtue of name equivalence, $\quotep{\procn{x}} \nameeq x$; so, the
notation is consistent with previous definitions.

Further, because names have structure it is possible to effect
substitutions on the basis of that structure. This means we need to
upgrade our notation for substitutions, which we accomplish by
adapting comprehension notation. Thus,

\begin{mathpar}
  P\{ y / x : x \in S \}
\end{mathpar}

is interpreted to mean the process derived from P by replacing (in a
capture-avoiding manner) each occurrence of $x$ in $S$ by $y$. For example,

\begin{mathpar}
  P\{ \quotep{\procn{x}|\procn{x}} / x : x \in \freenames{P} \}
\end{mathpar}

will replace each (occurrence) of a free name $x$ in $P$ by
$\quotep{\procn{x}|\procn{x}}$.

Also, we will avail ourselves of the notation $x^{L}$ and $x^{R}$ to
denote injections of a name into disjoint copies of the name
space. There are numerous ways to accomplish this. One example can be
found in \cite{MeredithR05}. This notation overloads to vectors of
names: $\vec{x}^{\pi} := (x_{i}^{\pi} \; : \; 0 \leq i < |\vec{x}| )$ where $\pi \in \{L,R\}$.

We also use $P^{\Box} := P|\Box$.

In \cite{MeredithR05} an interpretation of the new operator is
given. It turns out that there are several possible interpretations
all enjoying the requisite algebraic properties of the operator (see
\cite{milner91polyadicpi}). We will therefore make liberal use of
$(\nu\; \vec{x})P$.

% subsection the_syntax_and_semantics_of_the_notation_system (end)   

\input{qm2pi.qmops} 

\input{qm2pi.sterngerlach} 

\input{qm2pi.metric} 

% section concurrent_process_calculi (end)

%\input{qm2pi.proofsketch}

% section proof sketch (end)

%\input{qm2pi.slviaknots} 

% section spatial logic via knots (end)

\input{qm2pi.conclusion}

% section conclusion (end)

%\input{qm2pi.dtcodes} 

% section wiring algorithm (end)

\input{qm2pi.ack} 

% section acknowledgments (end)

\newpage


\bibliographystyle{plain}   
\bibliography{../../biblios/main.bib}

\input{qm2pi.rhodetails}

\end{document}

 

\documentclass[12pt]{llncs}
%\documentclass{jktr}

\usepackage[pdftex]{hyperref}                   
\usepackage {listings}
\usepackage {mathpartir}
\usepackage{bcprules}
%\usepackage{listings}
                       
\usepackage{graphicx} 
%\usepackage[margins=2.5cm,nohead,nofoot]{geometry}
%\usepackage{geometry}
\usepackage{amsfonts}
\usepackage{amstext}
\usepackage{latexsym}
\usepackage{amssymb}
\usepackage{color}


%\include{myPreamble}
\include{qm2pi.local} 

%\ifpdf
%\usepackage[pdftex]{graphicx}
%\else
%\usepackage{graphicx}
%\fi

 % \ifpdf
%  \usepackage{pdfsync}
%  \if


%\title{Brief Article}
%\author{David F. Snyder}
%\author{L.G. Meredith}

%\address{Dept. of Math., Texas State University--San Marcos, San Marcos, TX 78666}
       
\pagestyle{empty}


\begin{document}

\lstset{language=[Objective]Caml,frame=shadowbox}

\input{qm2pi.front}

% section front matter (end)

\input{qm2pi.intro} 
 
% section introduction (end)

% \input{qm2pi.knotations} 

% section notation (end)

\input{qm2pi.process.calculi} 

% section concurrent_process_calculi_and_spatial_logics_ (end)
    
%\input{qm2pi.knots2pi} 

%\input{qm2pi.trefoil} 

%\input{qm2pi.mainthm} 

% subsection basic_interpretation (end)

%\input{qm2pi.rho.presentation} 
\subsection{The syntax and semantics of the notation system}\label{sub:the_syntax_and_semantics_of_the_notation_system} % (fold)

We now summarize a technical presentation of the calculus that
embodies our theory of dynamics. The typical presentation of such a
calculus follows the style of giving generators and relations on
them. The grammar, below, describing term constructors, freely
generates the set of processes, $\Proc$. This set is then quotiented
by a relation known as structural congruence and it is over this set
that the notion of dynamics is expressed. This presentation is
essentially that of \cite{MeredithR05} with the addition of
polyadicity and summation. For readability we have relegated some of
the technical subtleties to an appendix.

\subsubsection{Process grammar}\label{subsub:process_grammar}

\begin{mathpar}
  \inferrule* [lab=synchronization] {} {{M} \bc \pzero \;|\; x?F \;|\; x!C }
  \and
  \inferrule* [lab=abstraction] {} {{F} \bc (x)P}
  \and
  \inferrule* [lab=concretion] {} {{C} \bc \langle Q \rangle}
  \and
  \inferrule* [lab=process] {} {{P,Q} \bc M \;| \;P|Q \;|\; @{x}}
  \and
  \inferrule* [lab=name] {} {{x} \bc \quotep{P}}
\end{mathpar} 

Note that $\vec{x}$ (resp. $\vec{P}$) denotes a vector of names
(resp. processes) of length $|\vec{x}|$ (resp. $|\vec{P}|$). We adopt
the following useful abbreviations.

\begin{mathpar}
   x?(\vec{y}).P := x.(\vec{y})P \and  x\clift{\vec{P}} := x.\clift{\vec{P}}
   \and x!(y) := \lift{x}{\dropn{y}}
   \and \Pi_{i=0}^{n-1}P_i := P_0 | \ldots | P_{n-1}
\end{mathpar}

\subsubsection{Structural congruence}

\paragraph{Free and bound names and alpha-equivalence.} At the
core of structural equivalence is alpha-equivalence which identifies
process that are the same up to a change of variable. Formally, we
recognize the distinction between free and bound names. The free names
of a process, $\freenames{P}$, may be calculated recursively as
follows:

\begin{mathpar}
\freenames{\pzero} := \emptyset
  \and \\
  \freenames{x?(y).P} := \{ x \} \cup (\freenames{P} \setminus \{ y \})
  \and 
  \freenames{x!\langle P \rangle} := \{ x \} \cup \{ P \} 
  \and \\
  \freenames{P|Q} := \freenames{P} \cup \freenames{Q}
  \and \\
  \freenames{@{x}} := \{ x \}
\end{mathpar}

$\pi$
$\quotep{\pi}$

$\freenames{-} : \pi \to \mathcal{P}(\quotep{\pi})$

\begin{eqnarray*}
  \freenames{\pzero} & := & \emptyset \\
  \freenames{x?(y).P} & := & \{ x \} \cup (\freenames{P} \setminus \{ y \}) \\
  \freenames{x!\langle P \rangle} & := & \{ x \} \cup \{ P \} \\
  \freenames{P|Q} & := & \freenames{P} \cup \freenames{Q} \\
  \freenames{\dropn{x}} & := & \{ x \}
\end{eqnarray*}

The bound names of a process, $\boundnames{P}$, are those names occurring in $P$
that are not free. For example, in $x?(y).0$, the name $x$ is free, while $y$ is bound.

\begin{mathpar}
  \inferrule* [lab=monoidal-laws] {} { P|Q \equiv Q|P \and P|0 \equiv P \and P|(Q|R) \equiv (P|Q)|R }
\end{mathpar}

\begin{mathpar}
  \inferrule* [lab=alpha-equivalence] {} { (x)P \equiv (y)P\{y/x\} \and y \not\in \freenames{P} }
\end{mathpar}

\begin{definition}
Then two processes, $P,Q$, are alpha-equivalent if $P = Q\{\vec{y}/\vec{x}\}$ for
some $\vec{x} \in \boundnames{Q},\vec{y} \in \boundnames{P}$, where $Q\{\vec{y}/\vec{x}\}$
denotes the capture-avoiding substitution of $\vec{y}$ for $\vec{x}$ in $Q$.
\end{definition}

\begin{definition}
  The {\em structural congruence} \cite{SangiorgiWalker} , $\equiv$,
  between processes is the least congruence containing
  alpha-equivalence, satisfying the abelian monoid laws
  (associativity, commutativity and $\pzero$ as identity) for parallel
  composition $|$ and for summation $+$.
\end{definition}

\subsection{Name equivalence}

We take name equivalence, written $\nameeq$, to be the smallest
equivalence relation generated by the following rules.

\begin{mathpar}
\inferrule*[lab=Quote-drop]
{ }
{ \quotep{@{x}} \nameeq x }

\inferrule*[lab=Struct-equiv]
{ P \scong Q }
{ \quotep{P} \nameeq \quotep{Q} }
\end{mathpar}

The astute reader will have noticed that the mutual recursion of names
and processes imposes a mutual recursion on alpha-equivalence and
structural equivalence via name-equivalence. Fortunately, all of this
works out pleasantly and we may calculate in the natural way, free of
concern. The reader interested in the details is referred to the
appendix \ref{appendix:rho_details}.

\subsection{Substitution}

We use $\Proc$ for the set of processes, $\QProc$ for the set of
names, and $\id{\{}\vec{y} / \vec{x} \id{\}}$ to denote partial maps,
$s : \QProc \rightarrow \QProc$. A map, $s$ lifts, uniquely, to a map
on process terms, $\widehat{s} : \Proc \rightarrow \Proc$ by the
following equations.

\begin{mathpar}
  (0) \psubstp{Q}{P} := 0 \\
  (R \juxtap S) \psubstp{Q}{P}
  :=    
  (R)\psubstp{Q}{P} \juxtap (S) \psubstp{Q}{P} \\
  (x?(y).R) \psubstp{Q}{P}    
  :=    
  (x)\substp{Q}{P} (z)\concat( (R \psubstn{z}{y}) \psubstp{Q}{P} ) \\
  (\lift{x}{R}) \psubstp{Q}{P}  
  :=
  \lift{(x)\substp{Q}{P}}{ R \psubstp{Q}{P} } \\
%   (\dropn{x})  \psubstp{Q}{P}       
%   := 
%   \left\{ 
%     \begin{array}{ccc} 
%       \dropn{\quotep{Q}} & & x \nameeq \quotep{P} \\
%       \dropn{x} & & otherwise \\
%     \end{array}
%   \right. 
  (\dropn{x})  \psubstp{Q}{P}       
  := 
  \left\{ 
    \begin{array}{ccc} 
      Q & & x \nameeq \quotep{P} \\
      \dropn{x} & & otherwise \\
    \end{array}
  \right.
\end{mathpar}
 

where

\begin{eqnarray}
  (x)\id{\{} \lpquote Q \rpquote / \lpquote P \rpquote \id{\}}            = 
  \left\{ 
    \begin{array}{ccc}
      \lpquote Q \rpquote & & x \nameeq \lpquote P \rpquote \\
      x & & otherwise \\
    \end{array}
  \right. \nonumber
\end{eqnarray}

and $z$ is chosen distinct from $\quotep{P}$, $\quotep{Q}$, the free
names in $Q$, and all the names in $R$. Our $\alpha$-equivalence will
be built in the standard way from this substitution.

\begin{remark}\label{rem:no_self_referential_names}
  One consequence of these definitions is that $\forall P. \quotep{P}
  \not\in \freenames{P}$.
\end{remark}

\subsection{ Dynamic quote: an example }

Anticipating something of what's to come, consider applying the
substitution, $\widehat{\id{\{}u / z \id{\}}}$, to the following pair
of processes, $\lift{w}{y!(z)}$ and $w[ \lpquote y!(z) \rpquote ]$.

\begin{eqnarray}
	\lift{w}{y!(z)}\widehat{\id{\{}u / z \id{\}}}
		& = &
		\lift{w}{y!(u)} \nonumber\\
	w[ \lpquote y!(z) \rpquote ] \widehat{ \id{\{}u / z \id{\}} }
		& = &
		w[ \lpquote y!(z) \rpquote ] \nonumber
\end{eqnarray}

Because the body of the process between quotes is impervious to
substitution, we get radically different answers. In fact, by
examining the first process in an input context,
e.g. $x?(z).\lift{w}{y!(z)}$, we see that the process under the lift
operator may be shaped by prefixed inputs binding a name inside it. In
this sense, the lift operator will be seen as a way to dynamically
construct processes before reifying them as names.

Finally equipped with these standard features we can present the
dynamics of the calculus.

\subsubsection{Operational semantics} 

Finally, we introduce the computational dynamics. What marks these
algebras as distinct from other more traditionally studied algebraic
structures, e.g. vector spaces or polynomial rings, is the manner in
which dynamics is captured. In traditional structures, dynamics is typically
expressed through morphisms between such structures, as in linear maps
between vector spaces or morphisms between rings. In algebras
associated with the semantics of computation, the dynamics is
expressed as part of the algebraic structure itself, through a
reduction reduction relation typically denoted by $\red$. Below, we
give a recursive presentation of this relation for the calculus used
in the encoding.

$\red \subseteq \pi \times \pi$
$\red : \pi \to \mathcal{P}(\pi)$

\begin{mathpar}
  \inferrule* [lab=Comm] { \textsf{match}( x_{src}, x_{trgt} ) } { x_{trgt}?(y)P \; | \; x_{src}!\langle {Q} \rangle \red P\{\quotep{Q}/y}\} }
  \and \\
  \inferrule* [lab=Par] {{P} \red {P}'} {{{P} | {Q}} \red {{P}' | {Q}}}
  \and
  \inferrule* [lab=Equiv]{{{P} \scong {P}'} \andalso {{P}' \red {Q}'} \andalso {{Q}' \scong {Q}}}{{P} \red {Q}}
\end{mathpar}

\begin{eqnarray*}
  match_{\equiv} (\quotep{P},\quotep{Q}) & := & P \equiv Q \\
  match_{\dagger}(\quotep{P},\quotep{Q}) & := & \forall R. P|Q \red^{*} R => R \red^{*} 0 \\
  match_{K}(\quotep{P},\quotep{Q}) & := & K \mbox{ for some context } K
\end{eqnarray*}

$u?(x)P | u!\langle Q \rangle \red P\{\quotep{Q}/x\}$

%We write $\wred$ for $\red^*$, and $P\red$ if $\exists Q $ such that $ P \red Q$.
We write $P\red$ if $\exists Q $ such that $ P \red Q$ and $P\not\red$, otherwise.

\section{Replication}

As mentioned before, it is known that replication (and hence
recursion) can be implemented in a higher-order process algebra
\cite{SangiorgiWalker}. As our first example of calculation with the
machinery thus far presented we give the construction explicitly in
the {\rhoc}.

\begin{eqnarray}
	D_{x} & := & \prefix{x}{y}{(\binpar{\outputp{x}{y}}{@{y}})} \nonumber\\
	\bangp_{x}{P} & := & \binpar{{x}!\langle{\binpar{D_{x}}{P}}\rangle}{D_{x}} \nonumber
\end{eqnarray}

\begin{eqnarray}
	\bangp_{x}{P} & & \nonumber\\
	=
	& {x}!\langle{(\prefix{x}{y}{(\outputp{x}{y} | @{y})) | P}}\rangle 
	      | \prefix{x}{y}{(\outputp{x}{y} | @{y})} & \nonumber\\
	\red
	& (\outputp{x}{y} | @{y})\substn{\quotep{(\prefix{x}{y}{(@{y} | \outputp{x}{y})) | P}}}{y} & \nonumber\\
	=
	& \outputp{x}{\quotep{(\prefix{x}{y}{(\outputp{x}{y} | @{y})) | P}}}
	  | {(\prefix{x}{y}{(\outputp{x}{y} | @{y})) | P}} & \nonumber\\
	\red
	& \ldots & \nonumber\\
	\red^*
	& P | P | \ldots & \nonumber
\end{eqnarray}

Of course, this encoding, as an implementation, runs away, unfolding
$\bangp{P}$ eagerly. A lazier and more implementable replication
operator, restricted to input-guarded processes, may be obtained as follows.

\begin{eqnarray}
\bangp{\prefix{u}{v}{P}} 
	:= 
	\binpar{\lift{x}{\prefix{u}{v}{(\binpar{D(x)}{P})}}}{D(x)} \nonumber
\end{eqnarray}

\begin{remark}
  Note that the lazier definition still does not deal with summation
  or mixed summation (i.e. sums over input and output). The reader is
  invited to construct definitions of replication that deal with these
  features. 

  Further, the definitions are parameterized in a name, $x$. Can you,
  gentle reader, make a definition that eliminates this parameter and
  guarantees no accidental interaction between the replication
  machinery and the process being replicated -- i.e. no accidental
  sharing of names used by the process to get its work done and the
  name(s) used by the replication to effect copying. This latter
  revision of the definition of replication is crucial to obtaining
  the expected identity $!!P \sim !P$.
\end{remark}

\begin{remark}\label{rem:paradoxical_combinator}
  The reader familiar with the lambda calculus will have noticed the
  similarity between $D$ and the paradoxical combinator.

  [Ed. note: the existence of this seems to suggest we have to be more
  restrictive on the set of processes and names we admit if we are to
  support no-cloning.]
\end{remark}

\subsubsection{Bisimulation}

The computational dynamics gives rise to another kind of equivalence,
the equivalence of computational behavior. As previously mentioned
this is typically captured \emph{via} some form of bisimulation.

% The notion we use in this paper is weak barbed bisimulation
% \cite{milner91polyadicpi}.

The notion we use in this paper is derived from weak barbed
bisimulation \cite{milner91polyadicpi}. 

\begin{definition}
An \emph{observation relation}, $\downarrow_{\mathcal N}$, over a set
of names, $\mathcal N$, is the smallest relation satisfying the rules
below.

\infrule[Out-barb]{y \in {\mathcal N}, \; x \nameeq y}
		  {\outputp{x}{v} \downarrow_{\mathcal N} x}
\infrule[Par-barb]{\mbox{$P\downarrow_{\mathcal N} x$ or $Q\downarrow_{\mathcal N} x$}}
		  {\binpar{P}{Q} \downarrow_{\mathcal N} x}

We write $P \Downarrow_{\mathcal N} x$ if there is $Q$ such that 
$P \wred Q$ and $Q \downarrow_{\mathcal N} x$.
\end{definition}

\begin{definition}
%\label{def.bbisim}
An  ${\mathcal N}$-\emph{barbed bisimulation} over a set of names, ${\mathcal N}$, is a symmetric binary relation 
${\mathcal S}_{\mathcal N}$ between agents such that $P\rel{S}_{\mathcal N}Q$ implies:
\begin{enumerate}
\item If $P \red P'$ then $Q \wred Q'$ and $P'\rel{S}_{\mathcal N} Q'$.
\item If $P\downarrow_{\mathcal N} x$, then $Q\Downarrow_{\mathcal N} x$.
\end{enumerate}
$P$ is ${\mathcal N}$-barbed bisimilar to $Q$, written
$P \wbbisim_{\mathcal N} Q$, if $P \rel{S}_{\mathcal N} Q$ for some ${\mathcal N}$-barbed bisimulation ${\mathcal S}_{\mathcal N}$.
\end{definition}

$\mathcal{R} \subseteq \pi \times \pi$

$P \mathcal{R} Q => \forall P'. P \red P' \Rightarrow \exists Q'. Q \red Q', P' \mathcal{R} Q'$

$P \vdash x \Rightarrow Q \vdash x$

\begin{mathpar}
  \inferrule*[lab=Out-barb]{x \nameeq y}{{y}!\langle{Q}\rangle \vdash x}
  \and
  \inferrule*[lab=Par-barb]{\mbox{$P\vdash x$ or $Q\vdash x$}}{\binpar{P}{Q} \vdash x}
\end{mathpar}

\subsubsection{Contexts}

One of the principle advantages of computational calculi like the
$\pi$-calculus is a well-defined notion of context,
contextual-equivalence and a correlation between
contextual-equivalence and notions of bisimulation. The notion of
context allows the decomposition of a process into (sub-)process and
its syntactic environment, its context. Thus, a context may be
thought of as a process with a ``hole'' (written $\Box$) in it. The
application of a context $M$ to a process $P$, written $M[P]$, is
tantamount to filling the hole in $M$ with $P$. In this paper we do
not need the full weight of this theory, but do make use of the notion
of context in the proof the main theorem. 

\begin{mathpar}
  \inferrule* [lab=summation] {} {{M_{M},M_{N}} \bc \Box \;|\; x.M_{A} \;|\; M_{M}+M_{N}}
  \and
  \inferrule* [lab=agent] {} {{M_{A}} \bc (\vec{x})M_{P} \;| \; \clift{P_0,\ldots,M_{P},\ldots,P_N}}
  \and \\
  \inferrule* [lab=process] {} {{M_{P}} \bc M_{N} \;| \;P|M_{P} }
\end{mathpar} 

\begin{mathpar}
  \inferrule* [lab=sychronization] {} {M_{N} \bc \Box \;|\; x?M_{F} \;|\; x!M_{C}}
  \and
  \inferrule* [lab=abstraction] {} {{M_{F}} \bc (x)M_{P} }
  \and
  \inferrule* [lab=concretion] {} {{M_{C}} \bc \langle M_{P} \rangle }
  \and \\
  \inferrule* [lab=process] {} {{M_{P}} \bc M_{N} \;| \;P|M_{P} }
\end{mathpar}

\begin{definition}[contextual application] Given a context $M$, and
  process $P$, we define the \emph{contextual application}, $M[P] :=
  M\{P/\Box\}$. That is, the contextual application of M to P is the
  substitution of $P$ for $\Box$ in $M$.
\end{definition}

$\meaningof{-} : L \to \mathcal{P}(\pi)$

\begin{mathpar}
  \inferrule* [lab=collection] {} {\meaningof{true} = \pi, \and \meaningof{~E} = \pi \setminus \meaningof{E}, \and \meaningof{E_{1} \& E_{2}} = \meaningof{E_{1}} \cap \meaningof{E_{2}}}
\end{mathpar}

\begin{mathpar}
  \inferrule* [lab=structure] {} {\meaningof{0} = \{ P \in \pi | P \equiv 0 \}, \and \\ \meaningof{E_1 | E_2} = \{ P \in \pi | P \equiv P_{1} | P_{2}, P_{1} \in \meaningof{E_{1}}, P_{2} \in \meaningof{E_2}\} }
\end{mathpar}

\begin{mathpar}
 \inferrule* [lab=behavior] {} {\meaningof{\langle a?b \rangle E} = \{ P \in \pi | P \equiv Q | u?(y)P', \\ \and \\\\ \and \\ \;\;\; u \in \meaningof{a}, \forall z.P'\{z/y\} \in \meaningof{E\{z/b\}}\}, \and \\ \meaningof{a!E} = \{ P \in \pi | P \equiv Q | x!\langle P' \rangle, x \in \meaningof{a} P' \in \meaningof{E}\} }
\end{mathpar}

\begin{mathpar}
 \inferrule* [lab=nominal] {} {\meaningof{\quotep{E}} = \{ \quotep{P} \in \quotep{\pi} | P \in \meaningof{E} \}, \and \meaningof{\quotep{P}} = \{ \quotep{Q} \in \quotep{\pi} | P \equiv Q \} \and \\ \meaningof{@\quotep{E}} = \{ P \in \pi | P \equiv @x, x \in \meaningof{E} \}}
\end{mathpar}

\begin{eqnarray*}
  \\
  \meaningof{-} : TS \to ST
\end{eqnarray*}

\begin{eqnarray*}
  \\
  L : TS \to ST
\end{eqnarray*}

\begin{eqnarray*}
  \\
  P \models E \iff P \in \meaningof{E}
\end{eqnarray*}

\begin{eqnarray*}
  P \approx_{L} Q \iff \forall E \in L. P \models E \iff Q \models E
\end{eqnarray*}

\begin{eqnarray*}
  P \approx_{K} Q
\end{eqnarray*}

\begin{eqnarray*}
  P \approx Q
\end{eqnarray*}

$\approx_{K} = \approx = \approx_{L}$

\subsubsection{Contextual duality}

Note that contexts extend the quotation operation to a family of
operations from processes to names. Given a context, $M$, we can
define a \emph{nominal context}, $\quotep{M}$ by $\quotep{M}[P] :=
\quotep{M[P]}$. To foreshadow what is to come we observe that these
operations enjoy a duality with processes very much like the duality
between vectors and maps from vectors to scalars.

Further, because the calculus is essentially higher-order, we have a
correspondence between contexts and processes. More specifically,
given a name $x$ and a context $M$ we can construct $M^{*}_{x}$ such
that 

\begin{mathpar}
  M^{*}_{x} | \lift{x}{P} \red M[P]
\end{mathpar}

namely,

\begin{mathpar}
  M^{*}_{x} := x?(u).M[\dropn{u}]
\end{mathpar}

The dependence of $M^{*}_{x}$ on a name makes it an abstraction, 

\begin{mathpar}
  M^{*} := (x)x?(u).M[\dropn{u}]
\end{mathpar}

\subsection{Additional notation}

It will sometimes be convenient to denote the process a name
quotes. We already have the notation $x = \quotep{P}$, but it will be
convenient to introduce an alternate notation, $\procn{x}$, when we
want to emphasize the connection to the use of the name. Note that, by
virtue of name equivalence, $\quotep{\procn{x}} \nameeq x$; so, the
notation is consistent with previous definitions.

Further, because names have structure it is possible to effect
substitutions on the basis of that structure. This means we need to
upgrade our notation for substitutions, which we accomplish by
adapting comprehension notation. Thus,

\begin{mathpar}
  P\{ y / x : x \in S \}
\end{mathpar}

is interpreted to mean the process derived from P by replacing (in a
capture-avoiding manner) each occurrence of $x$ in $S$ by $y$. For example,

\begin{mathpar}
  P\{ \quotep{\procn{x}|\procn{x}} / x : x \in \freenames{P} \}
\end{mathpar}

will replace each (occurrence) of a free name $x$ in $P$ by
$\quotep{\procn{x}|\procn{x}}$.

Also, we will avail ourselves of the notation $x^{L}$ and $x^{R}$ to
denote injections of a name into disjoint copies of the name
space. There are numerous ways to accomplish this. One example can be
found in \cite{MeredithR05}. This notation overloads to vectors of
names: $\vec{x}^{\pi} := (x_{i}^{\pi} \; : \; 0 \leq i < |\vec{x}| )$ where $\pi \in \{L,R\}$.

We also use $P^{\Box} := P|\Box$.

In \cite{MeredithR05} an interpretation of the new operator is
given. It turns out that there are several possible interpretations
all enjoying the requisite algebraic properties of the operator (see
\cite{milner91polyadicpi}). We will therefore make liberal use of
$(\nu\; \vec{x})P$.

% subsection the_syntax_and_semantics_of_the_notation_system (end)   

\input{qm2pi.qmops} 

\input{qm2pi.sterngerlach} 

\input{qm2pi.metric} 

% section concurrent_process_calculi (end)

%\input{qm2pi.proofsketch}

% section proof sketch (end)

%\input{qm2pi.slviaknots} 

% section spatial logic via knots (end)

\input{qm2pi.conclusion}

% section conclusion (end)

%\input{qm2pi.dtcodes} 

% section wiring algorithm (end)

\input{qm2pi.ack} 

% section acknowledgments (end)

\newpage


\bibliographystyle{plain}   
\bibliography{../../biblios/main.bib}

\input{qm2pi.rhodetails}

\end{document}

 

% section concurrent_process_calculi (end)

%\documentclass[12pt]{llncs}
%\documentclass{jktr}

\usepackage[pdftex]{hyperref}                   
\usepackage {listings}
\usepackage {mathpartir}
\usepackage{bcprules}
%\usepackage{listings}
                       
\usepackage{graphicx} 
%\usepackage[margins=2.5cm,nohead,nofoot]{geometry}
%\usepackage{geometry}
\usepackage{amsfonts}
\usepackage{amstext}
\usepackage{latexsym}
\usepackage{amssymb}
\usepackage{color}


%\include{myPreamble}
\include{qm2pi.local} 

%\ifpdf
%\usepackage[pdftex]{graphicx}
%\else
%\usepackage{graphicx}
%\fi

 % \ifpdf
%  \usepackage{pdfsync}
%  \if


%\title{Brief Article}
%\author{David F. Snyder}
%\author{L.G. Meredith}

%\address{Dept. of Math., Texas State University--San Marcos, San Marcos, TX 78666}
       
\pagestyle{empty}


\begin{document}

\lstset{language=[Objective]Caml,frame=shadowbox}

\input{qm2pi.front}

% section front matter (end)

\input{qm2pi.intro} 
 
% section introduction (end)

% \input{qm2pi.knotations} 

% section notation (end)

\input{qm2pi.process.calculi} 

% section concurrent_process_calculi_and_spatial_logics_ (end)
    
%\input{qm2pi.knots2pi} 

%\input{qm2pi.trefoil} 

%\input{qm2pi.mainthm} 

% subsection basic_interpretation (end)

%\input{qm2pi.rho.presentation} 
\subsection{The syntax and semantics of the notation system}\label{sub:the_syntax_and_semantics_of_the_notation_system} % (fold)

We now summarize a technical presentation of the calculus that
embodies our theory of dynamics. The typical presentation of such a
calculus follows the style of giving generators and relations on
them. The grammar, below, describing term constructors, freely
generates the set of processes, $\Proc$. This set is then quotiented
by a relation known as structural congruence and it is over this set
that the notion of dynamics is expressed. This presentation is
essentially that of \cite{MeredithR05} with the addition of
polyadicity and summation. For readability we have relegated some of
the technical subtleties to an appendix.

\subsubsection{Process grammar}\label{subsub:process_grammar}

\begin{mathpar}
  \inferrule* [lab=synchronization] {} {{M} \bc \pzero \;|\; x?F \;|\; x!C }
  \and
  \inferrule* [lab=abstraction] {} {{F} \bc (x)P}
  \and
  \inferrule* [lab=concretion] {} {{C} \bc \langle Q \rangle}
  \and
  \inferrule* [lab=process] {} {{P,Q} \bc M \;| \;P|Q \;|\; @{x}}
  \and
  \inferrule* [lab=name] {} {{x} \bc \quotep{P}}
\end{mathpar} 

Note that $\vec{x}$ (resp. $\vec{P}$) denotes a vector of names
(resp. processes) of length $|\vec{x}|$ (resp. $|\vec{P}|$). We adopt
the following useful abbreviations.

\begin{mathpar}
   x?(\vec{y}).P := x.(\vec{y})P \and  x\clift{\vec{P}} := x.\clift{\vec{P}}
   \and x!(y) := \lift{x}{\dropn{y}}
   \and \Pi_{i=0}^{n-1}P_i := P_0 | \ldots | P_{n-1}
\end{mathpar}

\subsubsection{Structural congruence}

\paragraph{Free and bound names and alpha-equivalence.} At the
core of structural equivalence is alpha-equivalence which identifies
process that are the same up to a change of variable. Formally, we
recognize the distinction between free and bound names. The free names
of a process, $\freenames{P}$, may be calculated recursively as
follows:

\begin{mathpar}
\freenames{\pzero} := \emptyset
  \and \\
  \freenames{x?(y).P} := \{ x \} \cup (\freenames{P} \setminus \{ y \})
  \and 
  \freenames{x!\langle P \rangle} := \{ x \} \cup \{ P \} 
  \and \\
  \freenames{P|Q} := \freenames{P} \cup \freenames{Q}
  \and \\
  \freenames{@{x}} := \{ x \}
\end{mathpar}

$\pi$
$\quotep{\pi}$

$\freenames{-} : \pi \to \mathcal{P}(\quotep{\pi})$

\begin{eqnarray*}
  \freenames{\pzero} & := & \emptyset \\
  \freenames{x?(y).P} & := & \{ x \} \cup (\freenames{P} \setminus \{ y \}) \\
  \freenames{x!\langle P \rangle} & := & \{ x \} \cup \{ P \} \\
  \freenames{P|Q} & := & \freenames{P} \cup \freenames{Q} \\
  \freenames{\dropn{x}} & := & \{ x \}
\end{eqnarray*}

The bound names of a process, $\boundnames{P}$, are those names occurring in $P$
that are not free. For example, in $x?(y).0$, the name $x$ is free, while $y$ is bound.

\begin{mathpar}
  \inferrule* [lab=monoidal-laws] {} { P|Q \equiv Q|P \and P|0 \equiv P \and P|(Q|R) \equiv (P|Q)|R }
\end{mathpar}

\begin{mathpar}
  \inferrule* [lab=alpha-equivalence] {} { (x)P \equiv (y)P\{y/x\} \and y \not\in \freenames{P} }
\end{mathpar}

\begin{definition}
Then two processes, $P,Q$, are alpha-equivalent if $P = Q\{\vec{y}/\vec{x}\}$ for
some $\vec{x} \in \boundnames{Q},\vec{y} \in \boundnames{P}$, where $Q\{\vec{y}/\vec{x}\}$
denotes the capture-avoiding substitution of $\vec{y}$ for $\vec{x}$ in $Q$.
\end{definition}

\begin{definition}
  The {\em structural congruence} \cite{SangiorgiWalker} , $\equiv$,
  between processes is the least congruence containing
  alpha-equivalence, satisfying the abelian monoid laws
  (associativity, commutativity and $\pzero$ as identity) for parallel
  composition $|$ and for summation $+$.
\end{definition}

\subsection{Name equivalence}

We take name equivalence, written $\nameeq$, to be the smallest
equivalence relation generated by the following rules.

\begin{mathpar}
\inferrule*[lab=Quote-drop]
{ }
{ \quotep{@{x}} \nameeq x }

\inferrule*[lab=Struct-equiv]
{ P \scong Q }
{ \quotep{P} \nameeq \quotep{Q} }
\end{mathpar}

The astute reader will have noticed that the mutual recursion of names
and processes imposes a mutual recursion on alpha-equivalence and
structural equivalence via name-equivalence. Fortunately, all of this
works out pleasantly and we may calculate in the natural way, free of
concern. The reader interested in the details is referred to the
appendix \ref{appendix:rho_details}.

\subsection{Substitution}

We use $\Proc$ for the set of processes, $\QProc$ for the set of
names, and $\id{\{}\vec{y} / \vec{x} \id{\}}$ to denote partial maps,
$s : \QProc \rightarrow \QProc$. A map, $s$ lifts, uniquely, to a map
on process terms, $\widehat{s} : \Proc \rightarrow \Proc$ by the
following equations.

\begin{mathpar}
  (0) \psubstp{Q}{P} := 0 \\
  (R \juxtap S) \psubstp{Q}{P}
  :=    
  (R)\psubstp{Q}{P} \juxtap (S) \psubstp{Q}{P} \\
  (x?(y).R) \psubstp{Q}{P}    
  :=    
  (x)\substp{Q}{P} (z)\concat( (R \psubstn{z}{y}) \psubstp{Q}{P} ) \\
  (\lift{x}{R}) \psubstp{Q}{P}  
  :=
  \lift{(x)\substp{Q}{P}}{ R \psubstp{Q}{P} } \\
%   (\dropn{x})  \psubstp{Q}{P}       
%   := 
%   \left\{ 
%     \begin{array}{ccc} 
%       \dropn{\quotep{Q}} & & x \nameeq \quotep{P} \\
%       \dropn{x} & & otherwise \\
%     \end{array}
%   \right. 
  (\dropn{x})  \psubstp{Q}{P}       
  := 
  \left\{ 
    \begin{array}{ccc} 
      Q & & x \nameeq \quotep{P} \\
      \dropn{x} & & otherwise \\
    \end{array}
  \right.
\end{mathpar}
 

where

\begin{eqnarray}
  (x)\id{\{} \lpquote Q \rpquote / \lpquote P \rpquote \id{\}}            = 
  \left\{ 
    \begin{array}{ccc}
      \lpquote Q \rpquote & & x \nameeq \lpquote P \rpquote \\
      x & & otherwise \\
    \end{array}
  \right. \nonumber
\end{eqnarray}

and $z$ is chosen distinct from $\quotep{P}$, $\quotep{Q}$, the free
names in $Q$, and all the names in $R$. Our $\alpha$-equivalence will
be built in the standard way from this substitution.

\begin{remark}\label{rem:no_self_referential_names}
  One consequence of these definitions is that $\forall P. \quotep{P}
  \not\in \freenames{P}$.
\end{remark}

\subsection{ Dynamic quote: an example }

Anticipating something of what's to come, consider applying the
substitution, $\widehat{\id{\{}u / z \id{\}}}$, to the following pair
of processes, $\lift{w}{y!(z)}$ and $w[ \lpquote y!(z) \rpquote ]$.

\begin{eqnarray}
	\lift{w}{y!(z)}\widehat{\id{\{}u / z \id{\}}}
		& = &
		\lift{w}{y!(u)} \nonumber\\
	w[ \lpquote y!(z) \rpquote ] \widehat{ \id{\{}u / z \id{\}} }
		& = &
		w[ \lpquote y!(z) \rpquote ] \nonumber
\end{eqnarray}

Because the body of the process between quotes is impervious to
substitution, we get radically different answers. In fact, by
examining the first process in an input context,
e.g. $x?(z).\lift{w}{y!(z)}$, we see that the process under the lift
operator may be shaped by prefixed inputs binding a name inside it. In
this sense, the lift operator will be seen as a way to dynamically
construct processes before reifying them as names.

Finally equipped with these standard features we can present the
dynamics of the calculus.

\subsubsection{Operational semantics} 

Finally, we introduce the computational dynamics. What marks these
algebras as distinct from other more traditionally studied algebraic
structures, e.g. vector spaces or polynomial rings, is the manner in
which dynamics is captured. In traditional structures, dynamics is typically
expressed through morphisms between such structures, as in linear maps
between vector spaces or morphisms between rings. In algebras
associated with the semantics of computation, the dynamics is
expressed as part of the algebraic structure itself, through a
reduction reduction relation typically denoted by $\red$. Below, we
give a recursive presentation of this relation for the calculus used
in the encoding.

$\red \subseteq \pi \times \pi$
$\red : \pi \to \mathcal{P}(\pi)$

\begin{mathpar}
  \inferrule* [lab=Comm] { \textsf{match}( x_{src}, x_{trgt} ) } { x_{trgt}?(y)P \; | \; x_{src}!\langle {Q} \rangle \red P\{\quotep{Q}/y}\} }
  \and \\
  \inferrule* [lab=Par] {{P} \red {P}'} {{{P} | {Q}} \red {{P}' | {Q}}}
  \and
  \inferrule* [lab=Equiv]{{{P} \scong {P}'} \andalso {{P}' \red {Q}'} \andalso {{Q}' \scong {Q}}}{{P} \red {Q}}
\end{mathpar}

\begin{eqnarray*}
  match_{\equiv} (\quotep{P},\quotep{Q}) & := & P \equiv Q \\
  match_{\dagger}(\quotep{P},\quotep{Q}) & := & \forall R. P|Q \red^{*} R => R \red^{*} 0 \\
  match_{K}(\quotep{P},\quotep{Q}) & := & K \mbox{ for some context } K
\end{eqnarray*}

$u?(x)P | u!\langle Q \rangle \red P\{\quotep{Q}/x\}$

%We write $\wred$ for $\red^*$, and $P\red$ if $\exists Q $ such that $ P \red Q$.
We write $P\red$ if $\exists Q $ such that $ P \red Q$ and $P\not\red$, otherwise.

\section{Replication}

As mentioned before, it is known that replication (and hence
recursion) can be implemented in a higher-order process algebra
\cite{SangiorgiWalker}. As our first example of calculation with the
machinery thus far presented we give the construction explicitly in
the {\rhoc}.

\begin{eqnarray}
	D_{x} & := & \prefix{x}{y}{(\binpar{\outputp{x}{y}}{@{y}})} \nonumber\\
	\bangp_{x}{P} & := & \binpar{{x}!\langle{\binpar{D_{x}}{P}}\rangle}{D_{x}} \nonumber
\end{eqnarray}

\begin{eqnarray}
	\bangp_{x}{P} & & \nonumber\\
	=
	& {x}!\langle{(\prefix{x}{y}{(\outputp{x}{y} | @{y})) | P}}\rangle 
	      | \prefix{x}{y}{(\outputp{x}{y} | @{y})} & \nonumber\\
	\red
	& (\outputp{x}{y} | @{y})\substn{\quotep{(\prefix{x}{y}{(@{y} | \outputp{x}{y})) | P}}}{y} & \nonumber\\
	=
	& \outputp{x}{\quotep{(\prefix{x}{y}{(\outputp{x}{y} | @{y})) | P}}}
	  | {(\prefix{x}{y}{(\outputp{x}{y} | @{y})) | P}} & \nonumber\\
	\red
	& \ldots & \nonumber\\
	\red^*
	& P | P | \ldots & \nonumber
\end{eqnarray}

Of course, this encoding, as an implementation, runs away, unfolding
$\bangp{P}$ eagerly. A lazier and more implementable replication
operator, restricted to input-guarded processes, may be obtained as follows.

\begin{eqnarray}
\bangp{\prefix{u}{v}{P}} 
	:= 
	\binpar{\lift{x}{\prefix{u}{v}{(\binpar{D(x)}{P})}}}{D(x)} \nonumber
\end{eqnarray}

\begin{remark}
  Note that the lazier definition still does not deal with summation
  or mixed summation (i.e. sums over input and output). The reader is
  invited to construct definitions of replication that deal with these
  features. 

  Further, the definitions are parameterized in a name, $x$. Can you,
  gentle reader, make a definition that eliminates this parameter and
  guarantees no accidental interaction between the replication
  machinery and the process being replicated -- i.e. no accidental
  sharing of names used by the process to get its work done and the
  name(s) used by the replication to effect copying. This latter
  revision of the definition of replication is crucial to obtaining
  the expected identity $!!P \sim !P$.
\end{remark}

\begin{remark}\label{rem:paradoxical_combinator}
  The reader familiar with the lambda calculus will have noticed the
  similarity between $D$ and the paradoxical combinator.

  [Ed. note: the existence of this seems to suggest we have to be more
  restrictive on the set of processes and names we admit if we are to
  support no-cloning.]
\end{remark}

\subsubsection{Bisimulation}

The computational dynamics gives rise to another kind of equivalence,
the equivalence of computational behavior. As previously mentioned
this is typically captured \emph{via} some form of bisimulation.

% The notion we use in this paper is weak barbed bisimulation
% \cite{milner91polyadicpi}.

The notion we use in this paper is derived from weak barbed
bisimulation \cite{milner91polyadicpi}. 

\begin{definition}
An \emph{observation relation}, $\downarrow_{\mathcal N}$, over a set
of names, $\mathcal N$, is the smallest relation satisfying the rules
below.

\infrule[Out-barb]{y \in {\mathcal N}, \; x \nameeq y}
		  {\outputp{x}{v} \downarrow_{\mathcal N} x}
\infrule[Par-barb]{\mbox{$P\downarrow_{\mathcal N} x$ or $Q\downarrow_{\mathcal N} x$}}
		  {\binpar{P}{Q} \downarrow_{\mathcal N} x}

We write $P \Downarrow_{\mathcal N} x$ if there is $Q$ such that 
$P \wred Q$ and $Q \downarrow_{\mathcal N} x$.
\end{definition}

\begin{definition}
%\label{def.bbisim}
An  ${\mathcal N}$-\emph{barbed bisimulation} over a set of names, ${\mathcal N}$, is a symmetric binary relation 
${\mathcal S}_{\mathcal N}$ between agents such that $P\rel{S}_{\mathcal N}Q$ implies:
\begin{enumerate}
\item If $P \red P'$ then $Q \wred Q'$ and $P'\rel{S}_{\mathcal N} Q'$.
\item If $P\downarrow_{\mathcal N} x$, then $Q\Downarrow_{\mathcal N} x$.
\end{enumerate}
$P$ is ${\mathcal N}$-barbed bisimilar to $Q$, written
$P \wbbisim_{\mathcal N} Q$, if $P \rel{S}_{\mathcal N} Q$ for some ${\mathcal N}$-barbed bisimulation ${\mathcal S}_{\mathcal N}$.
\end{definition}

$\mathcal{R} \subseteq \pi \times \pi$

$P \mathcal{R} Q => \forall P'. P \red P' \Rightarrow \exists Q'. Q \red Q', P' \mathcal{R} Q'$

$P \vdash x \Rightarrow Q \vdash x$

\begin{mathpar}
  \inferrule*[lab=Out-barb]{x \nameeq y}{{y}!\langle{Q}\rangle \vdash x}
  \and
  \inferrule*[lab=Par-barb]{\mbox{$P\vdash x$ or $Q\vdash x$}}{\binpar{P}{Q} \vdash x}
\end{mathpar}

\subsubsection{Contexts}

One of the principle advantages of computational calculi like the
$\pi$-calculus is a well-defined notion of context,
contextual-equivalence and a correlation between
contextual-equivalence and notions of bisimulation. The notion of
context allows the decomposition of a process into (sub-)process and
its syntactic environment, its context. Thus, a context may be
thought of as a process with a ``hole'' (written $\Box$) in it. The
application of a context $M$ to a process $P$, written $M[P]$, is
tantamount to filling the hole in $M$ with $P$. In this paper we do
not need the full weight of this theory, but do make use of the notion
of context in the proof the main theorem. 

\begin{mathpar}
  \inferrule* [lab=summation] {} {{M_{M},M_{N}} \bc \Box \;|\; x.M_{A} \;|\; M_{M}+M_{N}}
  \and
  \inferrule* [lab=agent] {} {{M_{A}} \bc (\vec{x})M_{P} \;| \; \clift{P_0,\ldots,M_{P},\ldots,P_N}}
  \and \\
  \inferrule* [lab=process] {} {{M_{P}} \bc M_{N} \;| \;P|M_{P} }
\end{mathpar} 

\begin{mathpar}
  \inferrule* [lab=sychronization] {} {M_{N} \bc \Box \;|\; x?M_{F} \;|\; x!M_{C}}
  \and
  \inferrule* [lab=abstraction] {} {{M_{F}} \bc (x)M_{P} }
  \and
  \inferrule* [lab=concretion] {} {{M_{C}} \bc \langle M_{P} \rangle }
  \and \\
  \inferrule* [lab=process] {} {{M_{P}} \bc M_{N} \;| \;P|M_{P} }
\end{mathpar}

\begin{definition}[contextual application] Given a context $M$, and
  process $P$, we define the \emph{contextual application}, $M[P] :=
  M\{P/\Box\}$. That is, the contextual application of M to P is the
  substitution of $P$ for $\Box$ in $M$.
\end{definition}

$\meaningof{-} : L \to \mathcal{P}(\pi)$

\begin{mathpar}
  \inferrule* [lab=collection] {} {\meaningof{true} = \pi, \and \meaningof{~E} = \pi \setminus \meaningof{E}, \and \meaningof{E_{1} \& E_{2}} = \meaningof{E_{1}} \cap \meaningof{E_{2}}}
\end{mathpar}

\begin{mathpar}
  \inferrule* [lab=structure] {} {\meaningof{0} = \{ P \in \pi | P \equiv 0 \}, \and \\ \meaningof{E_1 | E_2} = \{ P \in \pi | P \equiv P_{1} | P_{2}, P_{1} \in \meaningof{E_{1}}, P_{2} \in \meaningof{E_2}\} }
\end{mathpar}

\begin{mathpar}
 \inferrule* [lab=behavior] {} {\meaningof{\langle a?b \rangle E} = \{ P \in \pi | P \equiv Q | u?(y)P', \\ \and \\\\ \and \\ \;\;\; u \in \meaningof{a}, \forall z.P'\{z/y\} \in \meaningof{E\{z/b\}}\}, \and \\ \meaningof{a!E} = \{ P \in \pi | P \equiv Q | x!\langle P' \rangle, x \in \meaningof{a} P' \in \meaningof{E}\} }
\end{mathpar}

\begin{mathpar}
 \inferrule* [lab=nominal] {} {\meaningof{\quotep{E}} = \{ \quotep{P} \in \quotep{\pi} | P \in \meaningof{E} \}, \and \meaningof{\quotep{P}} = \{ \quotep{Q} \in \quotep{\pi} | P \equiv Q \} \and \\ \meaningof{@\quotep{E}} = \{ P \in \pi | P \equiv @x, x \in \meaningof{E} \}}
\end{mathpar}

\begin{eqnarray*}
  \\
  \meaningof{-} : TS \to ST
\end{eqnarray*}

\begin{eqnarray*}
  \\
  L : TS \to ST
\end{eqnarray*}

\begin{eqnarray*}
  \\
  P \models E \iff P \in \meaningof{E}
\end{eqnarray*}

\begin{eqnarray*}
  P \approx_{L} Q \iff \forall E \in L. P \models E \iff Q \models E
\end{eqnarray*}

\begin{eqnarray*}
  P \approx_{K} Q
\end{eqnarray*}

\begin{eqnarray*}
  P \approx Q
\end{eqnarray*}

$\approx_{K} = \approx = \approx_{L}$

\subsubsection{Contextual duality}

Note that contexts extend the quotation operation to a family of
operations from processes to names. Given a context, $M$, we can
define a \emph{nominal context}, $\quotep{M}$ by $\quotep{M}[P] :=
\quotep{M[P]}$. To foreshadow what is to come we observe that these
operations enjoy a duality with processes very much like the duality
between vectors and maps from vectors to scalars.

Further, because the calculus is essentially higher-order, we have a
correspondence between contexts and processes. More specifically,
given a name $x$ and a context $M$ we can construct $M^{*}_{x}$ such
that 

\begin{mathpar}
  M^{*}_{x} | \lift{x}{P} \red M[P]
\end{mathpar}

namely,

\begin{mathpar}
  M^{*}_{x} := x?(u).M[\dropn{u}]
\end{mathpar}

The dependence of $M^{*}_{x}$ on a name makes it an abstraction, 

\begin{mathpar}
  M^{*} := (x)x?(u).M[\dropn{u}]
\end{mathpar}

\subsection{Additional notation}

It will sometimes be convenient to denote the process a name
quotes. We already have the notation $x = \quotep{P}$, but it will be
convenient to introduce an alternate notation, $\procn{x}$, when we
want to emphasize the connection to the use of the name. Note that, by
virtue of name equivalence, $\quotep{\procn{x}} \nameeq x$; so, the
notation is consistent with previous definitions.

Further, because names have structure it is possible to effect
substitutions on the basis of that structure. This means we need to
upgrade our notation for substitutions, which we accomplish by
adapting comprehension notation. Thus,

\begin{mathpar}
  P\{ y / x : x \in S \}
\end{mathpar}

is interpreted to mean the process derived from P by replacing (in a
capture-avoiding manner) each occurrence of $x$ in $S$ by $y$. For example,

\begin{mathpar}
  P\{ \quotep{\procn{x}|\procn{x}} / x : x \in \freenames{P} \}
\end{mathpar}

will replace each (occurrence) of a free name $x$ in $P$ by
$\quotep{\procn{x}|\procn{x}}$.

Also, we will avail ourselves of the notation $x^{L}$ and $x^{R}$ to
denote injections of a name into disjoint copies of the name
space. There are numerous ways to accomplish this. One example can be
found in \cite{MeredithR05}. This notation overloads to vectors of
names: $\vec{x}^{\pi} := (x_{i}^{\pi} \; : \; 0 \leq i < |\vec{x}| )$ where $\pi \in \{L,R\}$.

We also use $P^{\Box} := P|\Box$.

In \cite{MeredithR05} an interpretation of the new operator is
given. It turns out that there are several possible interpretations
all enjoying the requisite algebraic properties of the operator (see
\cite{milner91polyadicpi}). We will therefore make liberal use of
$(\nu\; \vec{x})P$.

% subsection the_syntax_and_semantics_of_the_notation_system (end)   

\input{qm2pi.qmops} 

\input{qm2pi.sterngerlach} 

\input{qm2pi.metric} 

% section concurrent_process_calculi (end)

%\input{qm2pi.proofsketch}

% section proof sketch (end)

%\input{qm2pi.slviaknots} 

% section spatial logic via knots (end)

\input{qm2pi.conclusion}

% section conclusion (end)

%\input{qm2pi.dtcodes} 

% section wiring algorithm (end)

\input{qm2pi.ack} 

% section acknowledgments (end)

\newpage


\bibliographystyle{plain}   
\bibliography{../../biblios/main.bib}

\input{qm2pi.rhodetails}

\end{document}



% section proof sketch (end)

%\section{Unlikely characters: spatial logic for
  knots}\label{sub:characteristic_formulae} % (fold)

Associated to the mobile process calculi are a family of logics known
as the Hennessy-Milner logics. These logics typically enjoy a
semantics interpreting formulae as sets of processes that when
factored through the encoding outlined above allows an identification
of classes of knots with logical formulae. In the context of this
encoding the sub-family known as the spatial logics \cite{CairesC03}
\cite{CairesC04} \cite{Caires04} are of particular interest providing
several important features for expressing and reasoning about
properties (i.e. classes) of knots. We hint here at how this may be done.

%\begin{description}
%\item [structural connectives] 
\subsubsection{Structural connectives} The spatial logics enjoy
structural connectives corresponding, at the logical level, to the
parallel composition ($P | Q$) and new name ($(\nu \; x)P$)
connectives for processes. As illustrated in the examples below, these
connectives are extremely expressive given the shape of our encoding.
%\item [decideable satisfaction]

\subsubsection{Decideable satisfaction}
In \cite{Caires04} the satisfaction relation is shown to be decideable
for a rich class of processes. It further turns out that the image of
the our encoding is a proper subset of that class. This result
provides the basis for an algorithm by which to search for knots
enjoying a given property.
%\item [characteristic formulae]

\subsubsection{Characteristic formulae}
In the same paper \cite{Caires04} , Caires presents a means of calculating
characteristic formulae, selecting equivalence classes of processes
up to a pre--specified depth limit on the support set of names. Composed with our
encoding, this characteristic formula can be used to select
characteristic formulae for knots.
%\end{description}

\subsubsection{Spatial logic formulae}

The grammar below (segmented for comprehension) summarizes the syntax
of spatial logic formulae. We employ illustrative examples in the
sequel to provide an intuitive understanding of their meaning
referring the reader to \cite{Caires04} for a more detailed explication
of the semantics.

\begin{mathpar}
  \inferrule* [lab=boolean] {} {{A,B} \bc T \;|\; \neg A \;|\; A \wedge B \;|\; \eta = \eta'}
  \and
  \inferrule* [lab=spatial] {} {|\; \pzero \;|\; A | B \;|\; x \text{\textregistered} A \;|\; \forall x . A \;|\;  H x . A}
  \and
  \inferrule* [lab=behavioral] {} {|\; \alpha . A}
  \and 
  \inferrule* [lab=recursion] {} {|\; X(\vec{u}) \;|\; \mu X(\vec{u}) . A}
  \and
  \inferrule* [lab=action] {} {\alpha \bc \langle x?(\vec{y}) \rangle \;|\; \langle x!(\vec{y}) \rangle \;|\; \langle \tau \rangle}
  \and 
  \inferrule* [lab=name] {} {\eta \bc x \;|\; \tau}
\end{mathpar} 

% subsection characteristic_formulae (end)   	 

\subsection{Example formulae}\label{sub:example_formulae_} % (fold)

\subsubsection{Crossing as formula.}
% 
% \begin{align*}
%   \frac{d}{dx} \sin x &= \cos x 
%   & \frac{d}{dx} e^x &= e^x \\
%   \frac{d}{dx} \cos x &= - \sin x 
%   & \frac{d}{dx} \log x &= \frac{1}{x} \\
% \end{align*} 

\begin{align*}
 \mu C(x_{0},x_{1},y_{0},y_{1},u).&(\langle x_{0}?(z) \rangle(\langle u! \rangle\langle y_{1}!z \rangle C(x_{0},x_{1},y_{0},y_{1},u)) & \\
  & \wedge \langle y_{1}?(z) \rangle (\langle u! \rangle \langle x_{0}!z \rangle C(x_{0},x_{1},y_{0},y_{1},u)) & \\
  & \wedge \langle x_{1}?(z) \rangle (\langle u? \rangle \langle y_{0}!z \rangle C(x_{0},x_{1},y_{0},y_{1},u)) & \\
  & \wedge \langle y_{0}?(z) \rangle (\langle u? \rangle \langle x_{1}!z \rangle C(x_{0},x_{1},y_{0},y_{1},u))) &
\end{align*}

The lexicographical similarity between the shape of this formulae and
the shape of definition of the process representing a crossing reveals
the intuitive meaning of this formulae. It describes the capabilities
of a process that has the right to represent a crossing. For example
it picks out processes that may perform an input on the port $x_0$ in
its initial menu of capabilities. What differentiates the formula
from the process, however, is that the crossing process is the
smallest candidate to satisfy the formula. Infinitely many other
processes -- with internal behavior hidden behind this interface, so
to speak -- also satisfy this formula. Even this simple formula,
then, can be seen to open a new view onto knots, providing a
computational interpretation of \emph{virtual} knots.

Note that this formula is derived by hand. A similar formula can be
derived by employing Caires' calculation of characteristic formula
\cite{Caires04} to the process representing a crossing. In light of
this discussion, we let
$\meaningof{C}_{\phi}(x0,x1,y0,y1,u)$ denote a formula specifying the
dynamics we wish to capture of a crossing. To guarantee we preserve
the shape of the interface and minimal semantics we demand that
$\meaningof{C}_{\phi}(x0,x1,y0,y1,u) \Rightarrow
\textbf{C}(x0,x1,y0,y1,u)$ where $\textbf{C}(x0,x1,y0,y1,u)$ denotes
the formula above.
                            
\subsubsection{Crossing number constraints.}
The moral content of the context lemma (Lemma \ref{context}) is that the notion of
``locality'' in the Reidemeister moves is effectively captured by the
parallel composition operator of the process calculus. This intuition
extends through the logic. Given a formula,
$\meaningof{C}_{\phi}(x0,x1,y0,y1,u)$, we can use the structural
connectives to specify constraints on crossing numbers, such as at
least $n$ crossings, or exactly $n$ crossings.
\begin{mathpar}
  \inferrule* [lab=at-least-n] {} { K^{\geq n}_{\phi}(\vec{xs},\vec{ys}) := \Pi_{i=0}^{n-1} Hu . \meaningof{C}_{\phi}(xs_i,ys_i,u) | T }
  \and 
  \inferrule* [lab=exactly-n] {} { K^{= n}_{\phi}(\vec{xs},\vec{ys}) := \Pi_{i=0}^{n-1} Hu . \meaningof{C}_{\phi}(xs_i,ys_i,u) | \neg (\forall x_0,y_0,x_1,y_1,u . \meaningof{C}_{\phi}(x_0,y_0,x_1,y_1,u) | T) }
\end{mathpar}

To round out this section, recall that the encoding of an $n$-crossing
knot decomposes into a parallel composition of $n$ \emph{copies} of a
crossing process together with a wiring harness. To specify different
knot classes with the same crossing number amounts to specifying
logical constraints on the wiring harness. In the interest of space,
we defer examples to a forthcoming paper. Suffice it to say that both
the conditions ``alternating knot'' and ``contains the tangle
corresponding to 5/3'' are expressible. For example, it is possible to
calculate the characteristic formula of a process corresponding to the
tangle 5/3 and conjoin it into the classifying formula via the
composition connective of the logic.

Finally, we wish to observe that it is entirely within reason to
contemplate a more domain-specific version of spatial logic tailored
to the shape of processes in the image of the encoding. Such a
domain-specific logic would have a better claim to the title formal
language of knot properties.

% subsection example_formulae_ (end)

% section knots_as_processes (end) 

% section spatial logic via knots (end)

\section{Conclusions and future work}

\paragraph{Testing physical space}
You, gentle reader, may wonder why of all the theorems to be proved
given this set up we pick the one above. In some sense it's hardly
central to quantum mechanics. We see it as central in the sense that
it firmly establishes a notion of physical space arising from a notion
of the equivalence of behavior. Relating bisimulation to a metric is a
big step forward, but one is faced with interpreting the relationship
of that metric space to something more physical. Quantum mechanical
notions of ``physical'' space are still far from intuitive, but by
relating this idea of distance as testing to calculations that predict
physical circumstances we are making a not insignificant step forward
toward an understanding of the physical space we inhabit as
essentially dynamic.

\paragraph{Effectivity and simulation}
One of the observations we have yet to make is that the entire program
spelled out here is effective. We have built various interpreters for
the reflective calculus at work in this interpretation. In principle,
then, we can simulate quantum mechanics on a computer. The place where
the simulation may lose fidelity is the infinitely branching summation
for the annihilator.

In this connection i also want to point out that the evaluation style
calculation of the inner product puts the non-determinism of the
summation right at the heart of measurement. This suggests that
Milner's original reduction-based formulation of the dynamics of his
calculi in terms of sums was not just notationally suggestive of a
notion of measure-and-continue but captured some significant part of
the physics.

\paragraph{Quantum continuations}
In light of this last observation i want to point out that the
predominant account of quantum mechanics is missing a key aspect of a
truly compositional story of the physical situation. In a real lab,
when a measurement is made the observation can be made to feed into
another device that then makes another measurement conditioned on the
results of the first. This means that after the superposition was
collapsed the entire experimental set up remained in
superposition. While QM offers a means of writing this down it doesn't
quite line up well with the well-trodden formulation of computation
and continuation that we see so succinctly expressed in Milner's
calculi. This suggests that there might be advantages to this account
of dynamics waiting to be explored.

\paragraph{Quantum logic}
In this connection, we also note that by virtue of having the
Hennessy-Milner construction, we can pull the construction through the
interpretation of QM. This gives us a natural candidate for a quantum
logic that enjoys an extremely tight connection with it's domain of
interpretation, making the construction much less ad hoc (rather it is
the image of functor!).

\paragraph{Quantum probabiity}
i have questions about the basis of the interpretation of inner
product as probability amplitude. In particular, using which
axiomatization of probability theory does the notion of probability
amplitude earn the right to be so dubbed? In other words, where is the
proof that the operation for calculating a probability amplitude (and
then squaring) satisfies the axioms of what it means to calculate a
probability? Even if such a proof exists (i have yet to find it in the
literature), i wonder if it might not be possible to turn things on
their heads. Can we view the calculation of the probability amplitude
as an axiomatization of probability? If so, then the definition we
give for calculating probability amplitude may provide the basis for
an \emph{effective} theory of probability.

\paragraph{Quantum vs ``biological'' information}
Finally, i want to conclude with a more philosophical observation. At
a recent workshop in which QM was a predominant topic i noticed
something about quantum information. The speaker was giving a riveting
discussion of axiomatic QM and showing how properties of ``no
cloning'' and ``no deleting'' emerged as consequences of the
axiomatization. Theorems of this form are necessary to give us a sense
of confidence that our axioms characterize the physical theory. What
struck me, though, was that if quantum information is neither erasable
nor replicable it is markedly different from \emph{life}. Two of the
things we know about life is that

\begin{itemize}
  \item it ends;
  \item to gain some measure of persistence, to transcend it's
    finitude it is imminently copyable.
\end{itemize}

Both of these qualities are summarized succinctly in the aphorism: all
flesh is grass. For me these two kinds of ``information'' -- call them
quantum and biological -- are end points on a spectrum of strategies
for persistence. At one end, we have those curious entities that enjoy
uniqueness and permanence; at the other, we have those who in the face
of a certain end and an uncertain present make a go of passing
something on. To me one of the more remarkable aspects of the latter
strategy is that in the presence of noise (and certain features of
copying) we get a kind of dynamism, a chance for improvement against a
given persistent condition.

% subsection other_calculi_other_bisimulations_and_geometry_as_behavior (end)




% section conclusion (end)

%\documentclass[12pt]{llncs}
%\documentclass{jktr}

\usepackage[pdftex]{hyperref}                   
\usepackage {listings}
\usepackage {mathpartir}
\usepackage{bcprules}
%\usepackage{listings}
                       
\usepackage{graphicx} 
%\usepackage[margins=2.5cm,nohead,nofoot]{geometry}
%\usepackage{geometry}
\usepackage{amsfonts}
\usepackage{amstext}
\usepackage{latexsym}
\usepackage{amssymb}
\usepackage{color}


%\include{myPreamble}
\include{qm2pi.local} 

%\ifpdf
%\usepackage[pdftex]{graphicx}
%\else
%\usepackage{graphicx}
%\fi

 % \ifpdf
%  \usepackage{pdfsync}
%  \if


%\title{Brief Article}
%\author{David F. Snyder}
%\author{L.G. Meredith}

%\address{Dept. of Math., Texas State University--San Marcos, San Marcos, TX 78666}
       
\pagestyle{empty}


\begin{document}

\lstset{language=[Objective]Caml,frame=shadowbox}

\input{qm2pi.front}

% section front matter (end)

\input{qm2pi.intro} 
 
% section introduction (end)

% \input{qm2pi.knotations} 

% section notation (end)

\input{qm2pi.process.calculi} 

% section concurrent_process_calculi_and_spatial_logics_ (end)
    
%\input{qm2pi.knots2pi} 

%\input{qm2pi.trefoil} 

%\input{qm2pi.mainthm} 

% subsection basic_interpretation (end)

%\input{qm2pi.rho.presentation} 
\subsection{The syntax and semantics of the notation system}\label{sub:the_syntax_and_semantics_of_the_notation_system} % (fold)

We now summarize a technical presentation of the calculus that
embodies our theory of dynamics. The typical presentation of such a
calculus follows the style of giving generators and relations on
them. The grammar, below, describing term constructors, freely
generates the set of processes, $\Proc$. This set is then quotiented
by a relation known as structural congruence and it is over this set
that the notion of dynamics is expressed. This presentation is
essentially that of \cite{MeredithR05} with the addition of
polyadicity and summation. For readability we have relegated some of
the technical subtleties to an appendix.

\subsubsection{Process grammar}\label{subsub:process_grammar}

\begin{mathpar}
  \inferrule* [lab=synchronization] {} {{M} \bc \pzero \;|\; x?F \;|\; x!C }
  \and
  \inferrule* [lab=abstraction] {} {{F} \bc (x)P}
  \and
  \inferrule* [lab=concretion] {} {{C} \bc \langle Q \rangle}
  \and
  \inferrule* [lab=process] {} {{P,Q} \bc M \;| \;P|Q \;|\; @{x}}
  \and
  \inferrule* [lab=name] {} {{x} \bc \quotep{P}}
\end{mathpar} 

Note that $\vec{x}$ (resp. $\vec{P}$) denotes a vector of names
(resp. processes) of length $|\vec{x}|$ (resp. $|\vec{P}|$). We adopt
the following useful abbreviations.

\begin{mathpar}
   x?(\vec{y}).P := x.(\vec{y})P \and  x\clift{\vec{P}} := x.\clift{\vec{P}}
   \and x!(y) := \lift{x}{\dropn{y}}
   \and \Pi_{i=0}^{n-1}P_i := P_0 | \ldots | P_{n-1}
\end{mathpar}

\subsubsection{Structural congruence}

\paragraph{Free and bound names and alpha-equivalence.} At the
core of structural equivalence is alpha-equivalence which identifies
process that are the same up to a change of variable. Formally, we
recognize the distinction between free and bound names. The free names
of a process, $\freenames{P}$, may be calculated recursively as
follows:

\begin{mathpar}
\freenames{\pzero} := \emptyset
  \and \\
  \freenames{x?(y).P} := \{ x \} \cup (\freenames{P} \setminus \{ y \})
  \and 
  \freenames{x!\langle P \rangle} := \{ x \} \cup \{ P \} 
  \and \\
  \freenames{P|Q} := \freenames{P} \cup \freenames{Q}
  \and \\
  \freenames{@{x}} := \{ x \}
\end{mathpar}

$\pi$
$\quotep{\pi}$

$\freenames{-} : \pi \to \mathcal{P}(\quotep{\pi})$

\begin{eqnarray*}
  \freenames{\pzero} & := & \emptyset \\
  \freenames{x?(y).P} & := & \{ x \} \cup (\freenames{P} \setminus \{ y \}) \\
  \freenames{x!\langle P \rangle} & := & \{ x \} \cup \{ P \} \\
  \freenames{P|Q} & := & \freenames{P} \cup \freenames{Q} \\
  \freenames{\dropn{x}} & := & \{ x \}
\end{eqnarray*}

The bound names of a process, $\boundnames{P}$, are those names occurring in $P$
that are not free. For example, in $x?(y).0$, the name $x$ is free, while $y$ is bound.

\begin{mathpar}
  \inferrule* [lab=monoidal-laws] {} { P|Q \equiv Q|P \and P|0 \equiv P \and P|(Q|R) \equiv (P|Q)|R }
\end{mathpar}

\begin{mathpar}
  \inferrule* [lab=alpha-equivalence] {} { (x)P \equiv (y)P\{y/x\} \and y \not\in \freenames{P} }
\end{mathpar}

\begin{definition}
Then two processes, $P,Q$, are alpha-equivalent if $P = Q\{\vec{y}/\vec{x}\}$ for
some $\vec{x} \in \boundnames{Q},\vec{y} \in \boundnames{P}$, where $Q\{\vec{y}/\vec{x}\}$
denotes the capture-avoiding substitution of $\vec{y}$ for $\vec{x}$ in $Q$.
\end{definition}

\begin{definition}
  The {\em structural congruence} \cite{SangiorgiWalker} , $\equiv$,
  between processes is the least congruence containing
  alpha-equivalence, satisfying the abelian monoid laws
  (associativity, commutativity and $\pzero$ as identity) for parallel
  composition $|$ and for summation $+$.
\end{definition}

\subsection{Name equivalence}

We take name equivalence, written $\nameeq$, to be the smallest
equivalence relation generated by the following rules.

\begin{mathpar}
\inferrule*[lab=Quote-drop]
{ }
{ \quotep{@{x}} \nameeq x }

\inferrule*[lab=Struct-equiv]
{ P \scong Q }
{ \quotep{P} \nameeq \quotep{Q} }
\end{mathpar}

The astute reader will have noticed that the mutual recursion of names
and processes imposes a mutual recursion on alpha-equivalence and
structural equivalence via name-equivalence. Fortunately, all of this
works out pleasantly and we may calculate in the natural way, free of
concern. The reader interested in the details is referred to the
appendix \ref{appendix:rho_details}.

\subsection{Substitution}

We use $\Proc$ for the set of processes, $\QProc$ for the set of
names, and $\id{\{}\vec{y} / \vec{x} \id{\}}$ to denote partial maps,
$s : \QProc \rightarrow \QProc$. A map, $s$ lifts, uniquely, to a map
on process terms, $\widehat{s} : \Proc \rightarrow \Proc$ by the
following equations.

\begin{mathpar}
  (0) \psubstp{Q}{P} := 0 \\
  (R \juxtap S) \psubstp{Q}{P}
  :=    
  (R)\psubstp{Q}{P} \juxtap (S) \psubstp{Q}{P} \\
  (x?(y).R) \psubstp{Q}{P}    
  :=    
  (x)\substp{Q}{P} (z)\concat( (R \psubstn{z}{y}) \psubstp{Q}{P} ) \\
  (\lift{x}{R}) \psubstp{Q}{P}  
  :=
  \lift{(x)\substp{Q}{P}}{ R \psubstp{Q}{P} } \\
%   (\dropn{x})  \psubstp{Q}{P}       
%   := 
%   \left\{ 
%     \begin{array}{ccc} 
%       \dropn{\quotep{Q}} & & x \nameeq \quotep{P} \\
%       \dropn{x} & & otherwise \\
%     \end{array}
%   \right. 
  (\dropn{x})  \psubstp{Q}{P}       
  := 
  \left\{ 
    \begin{array}{ccc} 
      Q & & x \nameeq \quotep{P} \\
      \dropn{x} & & otherwise \\
    \end{array}
  \right.
\end{mathpar}
 

where

\begin{eqnarray}
  (x)\id{\{} \lpquote Q \rpquote / \lpquote P \rpquote \id{\}}            = 
  \left\{ 
    \begin{array}{ccc}
      \lpquote Q \rpquote & & x \nameeq \lpquote P \rpquote \\
      x & & otherwise \\
    \end{array}
  \right. \nonumber
\end{eqnarray}

and $z$ is chosen distinct from $\quotep{P}$, $\quotep{Q}$, the free
names in $Q$, and all the names in $R$. Our $\alpha$-equivalence will
be built in the standard way from this substitution.

\begin{remark}\label{rem:no_self_referential_names}
  One consequence of these definitions is that $\forall P. \quotep{P}
  \not\in \freenames{P}$.
\end{remark}

\subsection{ Dynamic quote: an example }

Anticipating something of what's to come, consider applying the
substitution, $\widehat{\id{\{}u / z \id{\}}}$, to the following pair
of processes, $\lift{w}{y!(z)}$ and $w[ \lpquote y!(z) \rpquote ]$.

\begin{eqnarray}
	\lift{w}{y!(z)}\widehat{\id{\{}u / z \id{\}}}
		& = &
		\lift{w}{y!(u)} \nonumber\\
	w[ \lpquote y!(z) \rpquote ] \widehat{ \id{\{}u / z \id{\}} }
		& = &
		w[ \lpquote y!(z) \rpquote ] \nonumber
\end{eqnarray}

Because the body of the process between quotes is impervious to
substitution, we get radically different answers. In fact, by
examining the first process in an input context,
e.g. $x?(z).\lift{w}{y!(z)}$, we see that the process under the lift
operator may be shaped by prefixed inputs binding a name inside it. In
this sense, the lift operator will be seen as a way to dynamically
construct processes before reifying them as names.

Finally equipped with these standard features we can present the
dynamics of the calculus.

\subsubsection{Operational semantics} 

Finally, we introduce the computational dynamics. What marks these
algebras as distinct from other more traditionally studied algebraic
structures, e.g. vector spaces or polynomial rings, is the manner in
which dynamics is captured. In traditional structures, dynamics is typically
expressed through morphisms between such structures, as in linear maps
between vector spaces or morphisms between rings. In algebras
associated with the semantics of computation, the dynamics is
expressed as part of the algebraic structure itself, through a
reduction reduction relation typically denoted by $\red$. Below, we
give a recursive presentation of this relation for the calculus used
in the encoding.

$\red \subseteq \pi \times \pi$
$\red : \pi \to \mathcal{P}(\pi)$

\begin{mathpar}
  \inferrule* [lab=Comm] { \textsf{match}( x_{src}, x_{trgt} ) } { x_{trgt}?(y)P \; | \; x_{src}!\langle {Q} \rangle \red P\{\quotep{Q}/y}\} }
  \and \\
  \inferrule* [lab=Par] {{P} \red {P}'} {{{P} | {Q}} \red {{P}' | {Q}}}
  \and
  \inferrule* [lab=Equiv]{{{P} \scong {P}'} \andalso {{P}' \red {Q}'} \andalso {{Q}' \scong {Q}}}{{P} \red {Q}}
\end{mathpar}

\begin{eqnarray*}
  match_{\equiv} (\quotep{P},\quotep{Q}) & := & P \equiv Q \\
  match_{\dagger}(\quotep{P},\quotep{Q}) & := & \forall R. P|Q \red^{*} R => R \red^{*} 0 \\
  match_{K}(\quotep{P},\quotep{Q}) & := & K \mbox{ for some context } K
\end{eqnarray*}

$u?(x)P | u!\langle Q \rangle \red P\{\quotep{Q}/x\}$

%We write $\wred$ for $\red^*$, and $P\red$ if $\exists Q $ such that $ P \red Q$.
We write $P\red$ if $\exists Q $ such that $ P \red Q$ and $P\not\red$, otherwise.

\section{Replication}

As mentioned before, it is known that replication (and hence
recursion) can be implemented in a higher-order process algebra
\cite{SangiorgiWalker}. As our first example of calculation with the
machinery thus far presented we give the construction explicitly in
the {\rhoc}.

\begin{eqnarray}
	D_{x} & := & \prefix{x}{y}{(\binpar{\outputp{x}{y}}{@{y}})} \nonumber\\
	\bangp_{x}{P} & := & \binpar{{x}!\langle{\binpar{D_{x}}{P}}\rangle}{D_{x}} \nonumber
\end{eqnarray}

\begin{eqnarray}
	\bangp_{x}{P} & & \nonumber\\
	=
	& {x}!\langle{(\prefix{x}{y}{(\outputp{x}{y} | @{y})) | P}}\rangle 
	      | \prefix{x}{y}{(\outputp{x}{y} | @{y})} & \nonumber\\
	\red
	& (\outputp{x}{y} | @{y})\substn{\quotep{(\prefix{x}{y}{(@{y} | \outputp{x}{y})) | P}}}{y} & \nonumber\\
	=
	& \outputp{x}{\quotep{(\prefix{x}{y}{(\outputp{x}{y} | @{y})) | P}}}
	  | {(\prefix{x}{y}{(\outputp{x}{y} | @{y})) | P}} & \nonumber\\
	\red
	& \ldots & \nonumber\\
	\red^*
	& P | P | \ldots & \nonumber
\end{eqnarray}

Of course, this encoding, as an implementation, runs away, unfolding
$\bangp{P}$ eagerly. A lazier and more implementable replication
operator, restricted to input-guarded processes, may be obtained as follows.

\begin{eqnarray}
\bangp{\prefix{u}{v}{P}} 
	:= 
	\binpar{\lift{x}{\prefix{u}{v}{(\binpar{D(x)}{P})}}}{D(x)} \nonumber
\end{eqnarray}

\begin{remark}
  Note that the lazier definition still does not deal with summation
  or mixed summation (i.e. sums over input and output). The reader is
  invited to construct definitions of replication that deal with these
  features. 

  Further, the definitions are parameterized in a name, $x$. Can you,
  gentle reader, make a definition that eliminates this parameter and
  guarantees no accidental interaction between the replication
  machinery and the process being replicated -- i.e. no accidental
  sharing of names used by the process to get its work done and the
  name(s) used by the replication to effect copying. This latter
  revision of the definition of replication is crucial to obtaining
  the expected identity $!!P \sim !P$.
\end{remark}

\begin{remark}\label{rem:paradoxical_combinator}
  The reader familiar with the lambda calculus will have noticed the
  similarity between $D$ and the paradoxical combinator.

  [Ed. note: the existence of this seems to suggest we have to be more
  restrictive on the set of processes and names we admit if we are to
  support no-cloning.]
\end{remark}

\subsubsection{Bisimulation}

The computational dynamics gives rise to another kind of equivalence,
the equivalence of computational behavior. As previously mentioned
this is typically captured \emph{via} some form of bisimulation.

% The notion we use in this paper is weak barbed bisimulation
% \cite{milner91polyadicpi}.

The notion we use in this paper is derived from weak barbed
bisimulation \cite{milner91polyadicpi}. 

\begin{definition}
An \emph{observation relation}, $\downarrow_{\mathcal N}$, over a set
of names, $\mathcal N$, is the smallest relation satisfying the rules
below.

\infrule[Out-barb]{y \in {\mathcal N}, \; x \nameeq y}
		  {\outputp{x}{v} \downarrow_{\mathcal N} x}
\infrule[Par-barb]{\mbox{$P\downarrow_{\mathcal N} x$ or $Q\downarrow_{\mathcal N} x$}}
		  {\binpar{P}{Q} \downarrow_{\mathcal N} x}

We write $P \Downarrow_{\mathcal N} x$ if there is $Q$ such that 
$P \wred Q$ and $Q \downarrow_{\mathcal N} x$.
\end{definition}

\begin{definition}
%\label{def.bbisim}
An  ${\mathcal N}$-\emph{barbed bisimulation} over a set of names, ${\mathcal N}$, is a symmetric binary relation 
${\mathcal S}_{\mathcal N}$ between agents such that $P\rel{S}_{\mathcal N}Q$ implies:
\begin{enumerate}
\item If $P \red P'$ then $Q \wred Q'$ and $P'\rel{S}_{\mathcal N} Q'$.
\item If $P\downarrow_{\mathcal N} x$, then $Q\Downarrow_{\mathcal N} x$.
\end{enumerate}
$P$ is ${\mathcal N}$-barbed bisimilar to $Q$, written
$P \wbbisim_{\mathcal N} Q$, if $P \rel{S}_{\mathcal N} Q$ for some ${\mathcal N}$-barbed bisimulation ${\mathcal S}_{\mathcal N}$.
\end{definition}

$\mathcal{R} \subseteq \pi \times \pi$

$P \mathcal{R} Q => \forall P'. P \red P' \Rightarrow \exists Q'. Q \red Q', P' \mathcal{R} Q'$

$P \vdash x \Rightarrow Q \vdash x$

\begin{mathpar}
  \inferrule*[lab=Out-barb]{x \nameeq y}{{y}!\langle{Q}\rangle \vdash x}
  \and
  \inferrule*[lab=Par-barb]{\mbox{$P\vdash x$ or $Q\vdash x$}}{\binpar{P}{Q} \vdash x}
\end{mathpar}

\subsubsection{Contexts}

One of the principle advantages of computational calculi like the
$\pi$-calculus is a well-defined notion of context,
contextual-equivalence and a correlation between
contextual-equivalence and notions of bisimulation. The notion of
context allows the decomposition of a process into (sub-)process and
its syntactic environment, its context. Thus, a context may be
thought of as a process with a ``hole'' (written $\Box$) in it. The
application of a context $M$ to a process $P$, written $M[P]$, is
tantamount to filling the hole in $M$ with $P$. In this paper we do
not need the full weight of this theory, but do make use of the notion
of context in the proof the main theorem. 

\begin{mathpar}
  \inferrule* [lab=summation] {} {{M_{M},M_{N}} \bc \Box \;|\; x.M_{A} \;|\; M_{M}+M_{N}}
  \and
  \inferrule* [lab=agent] {} {{M_{A}} \bc (\vec{x})M_{P} \;| \; \clift{P_0,\ldots,M_{P},\ldots,P_N}}
  \and \\
  \inferrule* [lab=process] {} {{M_{P}} \bc M_{N} \;| \;P|M_{P} }
\end{mathpar} 

\begin{mathpar}
  \inferrule* [lab=sychronization] {} {M_{N} \bc \Box \;|\; x?M_{F} \;|\; x!M_{C}}
  \and
  \inferrule* [lab=abstraction] {} {{M_{F}} \bc (x)M_{P} }
  \and
  \inferrule* [lab=concretion] {} {{M_{C}} \bc \langle M_{P} \rangle }
  \and \\
  \inferrule* [lab=process] {} {{M_{P}} \bc M_{N} \;| \;P|M_{P} }
\end{mathpar}

\begin{definition}[contextual application] Given a context $M$, and
  process $P$, we define the \emph{contextual application}, $M[P] :=
  M\{P/\Box\}$. That is, the contextual application of M to P is the
  substitution of $P$ for $\Box$ in $M$.
\end{definition}

$\meaningof{-} : L \to \mathcal{P}(\pi)$

\begin{mathpar}
  \inferrule* [lab=collection] {} {\meaningof{true} = \pi, \and \meaningof{~E} = \pi \setminus \meaningof{E}, \and \meaningof{E_{1} \& E_{2}} = \meaningof{E_{1}} \cap \meaningof{E_{2}}}
\end{mathpar}

\begin{mathpar}
  \inferrule* [lab=structure] {} {\meaningof{0} = \{ P \in \pi | P \equiv 0 \}, \and \\ \meaningof{E_1 | E_2} = \{ P \in \pi | P \equiv P_{1} | P_{2}, P_{1} \in \meaningof{E_{1}}, P_{2} \in \meaningof{E_2}\} }
\end{mathpar}

\begin{mathpar}
 \inferrule* [lab=behavior] {} {\meaningof{\langle a?b \rangle E} = \{ P \in \pi | P \equiv Q | u?(y)P', \\ \and \\\\ \and \\ \;\;\; u \in \meaningof{a}, \forall z.P'\{z/y\} \in \meaningof{E\{z/b\}}\}, \and \\ \meaningof{a!E} = \{ P \in \pi | P \equiv Q | x!\langle P' \rangle, x \in \meaningof{a} P' \in \meaningof{E}\} }
\end{mathpar}

\begin{mathpar}
 \inferrule* [lab=nominal] {} {\meaningof{\quotep{E}} = \{ \quotep{P} \in \quotep{\pi} | P \in \meaningof{E} \}, \and \meaningof{\quotep{P}} = \{ \quotep{Q} \in \quotep{\pi} | P \equiv Q \} \and \\ \meaningof{@\quotep{E}} = \{ P \in \pi | P \equiv @x, x \in \meaningof{E} \}}
\end{mathpar}

\begin{eqnarray*}
  \\
  \meaningof{-} : TS \to ST
\end{eqnarray*}

\begin{eqnarray*}
  \\
  L : TS \to ST
\end{eqnarray*}

\begin{eqnarray*}
  \\
  P \models E \iff P \in \meaningof{E}
\end{eqnarray*}

\begin{eqnarray*}
  P \approx_{L} Q \iff \forall E \in L. P \models E \iff Q \models E
\end{eqnarray*}

\begin{eqnarray*}
  P \approx_{K} Q
\end{eqnarray*}

\begin{eqnarray*}
  P \approx Q
\end{eqnarray*}

$\approx_{K} = \approx = \approx_{L}$

\subsubsection{Contextual duality}

Note that contexts extend the quotation operation to a family of
operations from processes to names. Given a context, $M$, we can
define a \emph{nominal context}, $\quotep{M}$ by $\quotep{M}[P] :=
\quotep{M[P]}$. To foreshadow what is to come we observe that these
operations enjoy a duality with processes very much like the duality
between vectors and maps from vectors to scalars.

Further, because the calculus is essentially higher-order, we have a
correspondence between contexts and processes. More specifically,
given a name $x$ and a context $M$ we can construct $M^{*}_{x}$ such
that 

\begin{mathpar}
  M^{*}_{x} | \lift{x}{P} \red M[P]
\end{mathpar}

namely,

\begin{mathpar}
  M^{*}_{x} := x?(u).M[\dropn{u}]
\end{mathpar}

The dependence of $M^{*}_{x}$ on a name makes it an abstraction, 

\begin{mathpar}
  M^{*} := (x)x?(u).M[\dropn{u}]
\end{mathpar}

\subsection{Additional notation}

It will sometimes be convenient to denote the process a name
quotes. We already have the notation $x = \quotep{P}$, but it will be
convenient to introduce an alternate notation, $\procn{x}$, when we
want to emphasize the connection to the use of the name. Note that, by
virtue of name equivalence, $\quotep{\procn{x}} \nameeq x$; so, the
notation is consistent with previous definitions.

Further, because names have structure it is possible to effect
substitutions on the basis of that structure. This means we need to
upgrade our notation for substitutions, which we accomplish by
adapting comprehension notation. Thus,

\begin{mathpar}
  P\{ y / x : x \in S \}
\end{mathpar}

is interpreted to mean the process derived from P by replacing (in a
capture-avoiding manner) each occurrence of $x$ in $S$ by $y$. For example,

\begin{mathpar}
  P\{ \quotep{\procn{x}|\procn{x}} / x : x \in \freenames{P} \}
\end{mathpar}

will replace each (occurrence) of a free name $x$ in $P$ by
$\quotep{\procn{x}|\procn{x}}$.

Also, we will avail ourselves of the notation $x^{L}$ and $x^{R}$ to
denote injections of a name into disjoint copies of the name
space. There are numerous ways to accomplish this. One example can be
found in \cite{MeredithR05}. This notation overloads to vectors of
names: $\vec{x}^{\pi} := (x_{i}^{\pi} \; : \; 0 \leq i < |\vec{x}| )$ where $\pi \in \{L,R\}$.

We also use $P^{\Box} := P|\Box$.

In \cite{MeredithR05} an interpretation of the new operator is
given. It turns out that there are several possible interpretations
all enjoying the requisite algebraic properties of the operator (see
\cite{milner91polyadicpi}). We will therefore make liberal use of
$(\nu\; \vec{x})P$.

% subsection the_syntax_and_semantics_of_the_notation_system (end)   

\input{qm2pi.qmops} 

\input{qm2pi.sterngerlach} 

\input{qm2pi.metric} 

% section concurrent_process_calculi (end)

%\input{qm2pi.proofsketch}

% section proof sketch (end)

%\input{qm2pi.slviaknots} 

% section spatial logic via knots (end)

\input{qm2pi.conclusion}

% section conclusion (end)

%\input{qm2pi.dtcodes} 

% section wiring algorithm (end)

\input{qm2pi.ack} 

% section acknowledgments (end)

\newpage


\bibliographystyle{plain}   
\bibliography{../../biblios/main.bib}

\input{qm2pi.rhodetails}

\end{document}

 

% section wiring algorithm (end)

\documentclass[12pt]{llncs}
%\documentclass{jktr}

\usepackage[pdftex]{hyperref}                   
\usepackage {listings}
\usepackage {mathpartir}
\usepackage{bcprules}
%\usepackage{listings}
                       
\usepackage{graphicx} 
%\usepackage[margins=2.5cm,nohead,nofoot]{geometry}
%\usepackage{geometry}
\usepackage{amsfonts}
\usepackage{amstext}
\usepackage{latexsym}
\usepackage{amssymb}
\usepackage{color}


%\include{myPreamble}
\include{qm2pi.local} 

%\ifpdf
%\usepackage[pdftex]{graphicx}
%\else
%\usepackage{graphicx}
%\fi

 % \ifpdf
%  \usepackage{pdfsync}
%  \if


%\title{Brief Article}
%\author{David F. Snyder}
%\author{L.G. Meredith}

%\address{Dept. of Math., Texas State University--San Marcos, San Marcos, TX 78666}
       
\pagestyle{empty}


\begin{document}

\lstset{language=[Objective]Caml,frame=shadowbox}

\input{qm2pi.front}

% section front matter (end)

\input{qm2pi.intro} 
 
% section introduction (end)

% \input{qm2pi.knotations} 

% section notation (end)

\input{qm2pi.process.calculi} 

% section concurrent_process_calculi_and_spatial_logics_ (end)
    
%\input{qm2pi.knots2pi} 

%\input{qm2pi.trefoil} 

%\input{qm2pi.mainthm} 

% subsection basic_interpretation (end)

%\input{qm2pi.rho.presentation} 
\subsection{The syntax and semantics of the notation system}\label{sub:the_syntax_and_semantics_of_the_notation_system} % (fold)

We now summarize a technical presentation of the calculus that
embodies our theory of dynamics. The typical presentation of such a
calculus follows the style of giving generators and relations on
them. The grammar, below, describing term constructors, freely
generates the set of processes, $\Proc$. This set is then quotiented
by a relation known as structural congruence and it is over this set
that the notion of dynamics is expressed. This presentation is
essentially that of \cite{MeredithR05} with the addition of
polyadicity and summation. For readability we have relegated some of
the technical subtleties to an appendix.

\subsubsection{Process grammar}\label{subsub:process_grammar}

\begin{mathpar}
  \inferrule* [lab=synchronization] {} {{M} \bc \pzero \;|\; x?F \;|\; x!C }
  \and
  \inferrule* [lab=abstraction] {} {{F} \bc (x)P}
  \and
  \inferrule* [lab=concretion] {} {{C} \bc \langle Q \rangle}
  \and
  \inferrule* [lab=process] {} {{P,Q} \bc M \;| \;P|Q \;|\; @{x}}
  \and
  \inferrule* [lab=name] {} {{x} \bc \quotep{P}}
\end{mathpar} 

Note that $\vec{x}$ (resp. $\vec{P}$) denotes a vector of names
(resp. processes) of length $|\vec{x}|$ (resp. $|\vec{P}|$). We adopt
the following useful abbreviations.

\begin{mathpar}
   x?(\vec{y}).P := x.(\vec{y})P \and  x\clift{\vec{P}} := x.\clift{\vec{P}}
   \and x!(y) := \lift{x}{\dropn{y}}
   \and \Pi_{i=0}^{n-1}P_i := P_0 | \ldots | P_{n-1}
\end{mathpar}

\subsubsection{Structural congruence}

\paragraph{Free and bound names and alpha-equivalence.} At the
core of structural equivalence is alpha-equivalence which identifies
process that are the same up to a change of variable. Formally, we
recognize the distinction between free and bound names. The free names
of a process, $\freenames{P}$, may be calculated recursively as
follows:

\begin{mathpar}
\freenames{\pzero} := \emptyset
  \and \\
  \freenames{x?(y).P} := \{ x \} \cup (\freenames{P} \setminus \{ y \})
  \and 
  \freenames{x!\langle P \rangle} := \{ x \} \cup \{ P \} 
  \and \\
  \freenames{P|Q} := \freenames{P} \cup \freenames{Q}
  \and \\
  \freenames{@{x}} := \{ x \}
\end{mathpar}

$\pi$
$\quotep{\pi}$

$\freenames{-} : \pi \to \mathcal{P}(\quotep{\pi})$

\begin{eqnarray*}
  \freenames{\pzero} & := & \emptyset \\
  \freenames{x?(y).P} & := & \{ x \} \cup (\freenames{P} \setminus \{ y \}) \\
  \freenames{x!\langle P \rangle} & := & \{ x \} \cup \{ P \} \\
  \freenames{P|Q} & := & \freenames{P} \cup \freenames{Q} \\
  \freenames{\dropn{x}} & := & \{ x \}
\end{eqnarray*}

The bound names of a process, $\boundnames{P}$, are those names occurring in $P$
that are not free. For example, in $x?(y).0$, the name $x$ is free, while $y$ is bound.

\begin{mathpar}
  \inferrule* [lab=monoidal-laws] {} { P|Q \equiv Q|P \and P|0 \equiv P \and P|(Q|R) \equiv (P|Q)|R }
\end{mathpar}

\begin{mathpar}
  \inferrule* [lab=alpha-equivalence] {} { (x)P \equiv (y)P\{y/x\} \and y \not\in \freenames{P} }
\end{mathpar}

\begin{definition}
Then two processes, $P,Q$, are alpha-equivalent if $P = Q\{\vec{y}/\vec{x}\}$ for
some $\vec{x} \in \boundnames{Q},\vec{y} \in \boundnames{P}$, where $Q\{\vec{y}/\vec{x}\}$
denotes the capture-avoiding substitution of $\vec{y}$ for $\vec{x}$ in $Q$.
\end{definition}

\begin{definition}
  The {\em structural congruence} \cite{SangiorgiWalker} , $\equiv$,
  between processes is the least congruence containing
  alpha-equivalence, satisfying the abelian monoid laws
  (associativity, commutativity and $\pzero$ as identity) for parallel
  composition $|$ and for summation $+$.
\end{definition}

\subsection{Name equivalence}

We take name equivalence, written $\nameeq$, to be the smallest
equivalence relation generated by the following rules.

\begin{mathpar}
\inferrule*[lab=Quote-drop]
{ }
{ \quotep{@{x}} \nameeq x }

\inferrule*[lab=Struct-equiv]
{ P \scong Q }
{ \quotep{P} \nameeq \quotep{Q} }
\end{mathpar}

The astute reader will have noticed that the mutual recursion of names
and processes imposes a mutual recursion on alpha-equivalence and
structural equivalence via name-equivalence. Fortunately, all of this
works out pleasantly and we may calculate in the natural way, free of
concern. The reader interested in the details is referred to the
appendix \ref{appendix:rho_details}.

\subsection{Substitution}

We use $\Proc$ for the set of processes, $\QProc$ for the set of
names, and $\id{\{}\vec{y} / \vec{x} \id{\}}$ to denote partial maps,
$s : \QProc \rightarrow \QProc$. A map, $s$ lifts, uniquely, to a map
on process terms, $\widehat{s} : \Proc \rightarrow \Proc$ by the
following equations.

\begin{mathpar}
  (0) \psubstp{Q}{P} := 0 \\
  (R \juxtap S) \psubstp{Q}{P}
  :=    
  (R)\psubstp{Q}{P} \juxtap (S) \psubstp{Q}{P} \\
  (x?(y).R) \psubstp{Q}{P}    
  :=    
  (x)\substp{Q}{P} (z)\concat( (R \psubstn{z}{y}) \psubstp{Q}{P} ) \\
  (\lift{x}{R}) \psubstp{Q}{P}  
  :=
  \lift{(x)\substp{Q}{P}}{ R \psubstp{Q}{P} } \\
%   (\dropn{x})  \psubstp{Q}{P}       
%   := 
%   \left\{ 
%     \begin{array}{ccc} 
%       \dropn{\quotep{Q}} & & x \nameeq \quotep{P} \\
%       \dropn{x} & & otherwise \\
%     \end{array}
%   \right. 
  (\dropn{x})  \psubstp{Q}{P}       
  := 
  \left\{ 
    \begin{array}{ccc} 
      Q & & x \nameeq \quotep{P} \\
      \dropn{x} & & otherwise \\
    \end{array}
  \right.
\end{mathpar}
 

where

\begin{eqnarray}
  (x)\id{\{} \lpquote Q \rpquote / \lpquote P \rpquote \id{\}}            = 
  \left\{ 
    \begin{array}{ccc}
      \lpquote Q \rpquote & & x \nameeq \lpquote P \rpquote \\
      x & & otherwise \\
    \end{array}
  \right. \nonumber
\end{eqnarray}

and $z$ is chosen distinct from $\quotep{P}$, $\quotep{Q}$, the free
names in $Q$, and all the names in $R$. Our $\alpha$-equivalence will
be built in the standard way from this substitution.

\begin{remark}\label{rem:no_self_referential_names}
  One consequence of these definitions is that $\forall P. \quotep{P}
  \not\in \freenames{P}$.
\end{remark}

\subsection{ Dynamic quote: an example }

Anticipating something of what's to come, consider applying the
substitution, $\widehat{\id{\{}u / z \id{\}}}$, to the following pair
of processes, $\lift{w}{y!(z)}$ and $w[ \lpquote y!(z) \rpquote ]$.

\begin{eqnarray}
	\lift{w}{y!(z)}\widehat{\id{\{}u / z \id{\}}}
		& = &
		\lift{w}{y!(u)} \nonumber\\
	w[ \lpquote y!(z) \rpquote ] \widehat{ \id{\{}u / z \id{\}} }
		& = &
		w[ \lpquote y!(z) \rpquote ] \nonumber
\end{eqnarray}

Because the body of the process between quotes is impervious to
substitution, we get radically different answers. In fact, by
examining the first process in an input context,
e.g. $x?(z).\lift{w}{y!(z)}$, we see that the process under the lift
operator may be shaped by prefixed inputs binding a name inside it. In
this sense, the lift operator will be seen as a way to dynamically
construct processes before reifying them as names.

Finally equipped with these standard features we can present the
dynamics of the calculus.

\subsubsection{Operational semantics} 

Finally, we introduce the computational dynamics. What marks these
algebras as distinct from other more traditionally studied algebraic
structures, e.g. vector spaces or polynomial rings, is the manner in
which dynamics is captured. In traditional structures, dynamics is typically
expressed through morphisms between such structures, as in linear maps
between vector spaces or morphisms between rings. In algebras
associated with the semantics of computation, the dynamics is
expressed as part of the algebraic structure itself, through a
reduction reduction relation typically denoted by $\red$. Below, we
give a recursive presentation of this relation for the calculus used
in the encoding.

$\red \subseteq \pi \times \pi$
$\red : \pi \to \mathcal{P}(\pi)$

\begin{mathpar}
  \inferrule* [lab=Comm] { \textsf{match}( x_{src}, x_{trgt} ) } { x_{trgt}?(y)P \; | \; x_{src}!\langle {Q} \rangle \red P\{\quotep{Q}/y}\} }
  \and \\
  \inferrule* [lab=Par] {{P} \red {P}'} {{{P} | {Q}} \red {{P}' | {Q}}}
  \and
  \inferrule* [lab=Equiv]{{{P} \scong {P}'} \andalso {{P}' \red {Q}'} \andalso {{Q}' \scong {Q}}}{{P} \red {Q}}
\end{mathpar}

\begin{eqnarray*}
  match_{\equiv} (\quotep{P},\quotep{Q}) & := & P \equiv Q \\
  match_{\dagger}(\quotep{P},\quotep{Q}) & := & \forall R. P|Q \red^{*} R => R \red^{*} 0 \\
  match_{K}(\quotep{P},\quotep{Q}) & := & K \mbox{ for some context } K
\end{eqnarray*}

$u?(x)P | u!\langle Q \rangle \red P\{\quotep{Q}/x\}$

%We write $\wred$ for $\red^*$, and $P\red$ if $\exists Q $ such that $ P \red Q$.
We write $P\red$ if $\exists Q $ such that $ P \red Q$ and $P\not\red$, otherwise.

\section{Replication}

As mentioned before, it is known that replication (and hence
recursion) can be implemented in a higher-order process algebra
\cite{SangiorgiWalker}. As our first example of calculation with the
machinery thus far presented we give the construction explicitly in
the {\rhoc}.

\begin{eqnarray}
	D_{x} & := & \prefix{x}{y}{(\binpar{\outputp{x}{y}}{@{y}})} \nonumber\\
	\bangp_{x}{P} & := & \binpar{{x}!\langle{\binpar{D_{x}}{P}}\rangle}{D_{x}} \nonumber
\end{eqnarray}

\begin{eqnarray}
	\bangp_{x}{P} & & \nonumber\\
	=
	& {x}!\langle{(\prefix{x}{y}{(\outputp{x}{y} | @{y})) | P}}\rangle 
	      | \prefix{x}{y}{(\outputp{x}{y} | @{y})} & \nonumber\\
	\red
	& (\outputp{x}{y} | @{y})\substn{\quotep{(\prefix{x}{y}{(@{y} | \outputp{x}{y})) | P}}}{y} & \nonumber\\
	=
	& \outputp{x}{\quotep{(\prefix{x}{y}{(\outputp{x}{y} | @{y})) | P}}}
	  | {(\prefix{x}{y}{(\outputp{x}{y} | @{y})) | P}} & \nonumber\\
	\red
	& \ldots & \nonumber\\
	\red^*
	& P | P | \ldots & \nonumber
\end{eqnarray}

Of course, this encoding, as an implementation, runs away, unfolding
$\bangp{P}$ eagerly. A lazier and more implementable replication
operator, restricted to input-guarded processes, may be obtained as follows.

\begin{eqnarray}
\bangp{\prefix{u}{v}{P}} 
	:= 
	\binpar{\lift{x}{\prefix{u}{v}{(\binpar{D(x)}{P})}}}{D(x)} \nonumber
\end{eqnarray}

\begin{remark}
  Note that the lazier definition still does not deal with summation
  or mixed summation (i.e. sums over input and output). The reader is
  invited to construct definitions of replication that deal with these
  features. 

  Further, the definitions are parameterized in a name, $x$. Can you,
  gentle reader, make a definition that eliminates this parameter and
  guarantees no accidental interaction between the replication
  machinery and the process being replicated -- i.e. no accidental
  sharing of names used by the process to get its work done and the
  name(s) used by the replication to effect copying. This latter
  revision of the definition of replication is crucial to obtaining
  the expected identity $!!P \sim !P$.
\end{remark}

\begin{remark}\label{rem:paradoxical_combinator}
  The reader familiar with the lambda calculus will have noticed the
  similarity between $D$ and the paradoxical combinator.

  [Ed. note: the existence of this seems to suggest we have to be more
  restrictive on the set of processes and names we admit if we are to
  support no-cloning.]
\end{remark}

\subsubsection{Bisimulation}

The computational dynamics gives rise to another kind of equivalence,
the equivalence of computational behavior. As previously mentioned
this is typically captured \emph{via} some form of bisimulation.

% The notion we use in this paper is weak barbed bisimulation
% \cite{milner91polyadicpi}.

The notion we use in this paper is derived from weak barbed
bisimulation \cite{milner91polyadicpi}. 

\begin{definition}
An \emph{observation relation}, $\downarrow_{\mathcal N}$, over a set
of names, $\mathcal N$, is the smallest relation satisfying the rules
below.

\infrule[Out-barb]{y \in {\mathcal N}, \; x \nameeq y}
		  {\outputp{x}{v} \downarrow_{\mathcal N} x}
\infrule[Par-barb]{\mbox{$P\downarrow_{\mathcal N} x$ or $Q\downarrow_{\mathcal N} x$}}
		  {\binpar{P}{Q} \downarrow_{\mathcal N} x}

We write $P \Downarrow_{\mathcal N} x$ if there is $Q$ such that 
$P \wred Q$ and $Q \downarrow_{\mathcal N} x$.
\end{definition}

\begin{definition}
%\label{def.bbisim}
An  ${\mathcal N}$-\emph{barbed bisimulation} over a set of names, ${\mathcal N}$, is a symmetric binary relation 
${\mathcal S}_{\mathcal N}$ between agents such that $P\rel{S}_{\mathcal N}Q$ implies:
\begin{enumerate}
\item If $P \red P'$ then $Q \wred Q'$ and $P'\rel{S}_{\mathcal N} Q'$.
\item If $P\downarrow_{\mathcal N} x$, then $Q\Downarrow_{\mathcal N} x$.
\end{enumerate}
$P$ is ${\mathcal N}$-barbed bisimilar to $Q$, written
$P \wbbisim_{\mathcal N} Q$, if $P \rel{S}_{\mathcal N} Q$ for some ${\mathcal N}$-barbed bisimulation ${\mathcal S}_{\mathcal N}$.
\end{definition}

$\mathcal{R} \subseteq \pi \times \pi$

$P \mathcal{R} Q => \forall P'. P \red P' \Rightarrow \exists Q'. Q \red Q', P' \mathcal{R} Q'$

$P \vdash x \Rightarrow Q \vdash x$

\begin{mathpar}
  \inferrule*[lab=Out-barb]{x \nameeq y}{{y}!\langle{Q}\rangle \vdash x}
  \and
  \inferrule*[lab=Par-barb]{\mbox{$P\vdash x$ or $Q\vdash x$}}{\binpar{P}{Q} \vdash x}
\end{mathpar}

\subsubsection{Contexts}

One of the principle advantages of computational calculi like the
$\pi$-calculus is a well-defined notion of context,
contextual-equivalence and a correlation between
contextual-equivalence and notions of bisimulation. The notion of
context allows the decomposition of a process into (sub-)process and
its syntactic environment, its context. Thus, a context may be
thought of as a process with a ``hole'' (written $\Box$) in it. The
application of a context $M$ to a process $P$, written $M[P]$, is
tantamount to filling the hole in $M$ with $P$. In this paper we do
not need the full weight of this theory, but do make use of the notion
of context in the proof the main theorem. 

\begin{mathpar}
  \inferrule* [lab=summation] {} {{M_{M},M_{N}} \bc \Box \;|\; x.M_{A} \;|\; M_{M}+M_{N}}
  \and
  \inferrule* [lab=agent] {} {{M_{A}} \bc (\vec{x})M_{P} \;| \; \clift{P_0,\ldots,M_{P},\ldots,P_N}}
  \and \\
  \inferrule* [lab=process] {} {{M_{P}} \bc M_{N} \;| \;P|M_{P} }
\end{mathpar} 

\begin{mathpar}
  \inferrule* [lab=sychronization] {} {M_{N} \bc \Box \;|\; x?M_{F} \;|\; x!M_{C}}
  \and
  \inferrule* [lab=abstraction] {} {{M_{F}} \bc (x)M_{P} }
  \and
  \inferrule* [lab=concretion] {} {{M_{C}} \bc \langle M_{P} \rangle }
  \and \\
  \inferrule* [lab=process] {} {{M_{P}} \bc M_{N} \;| \;P|M_{P} }
\end{mathpar}

\begin{definition}[contextual application] Given a context $M$, and
  process $P$, we define the \emph{contextual application}, $M[P] :=
  M\{P/\Box\}$. That is, the contextual application of M to P is the
  substitution of $P$ for $\Box$ in $M$.
\end{definition}

$\meaningof{-} : L \to \mathcal{P}(\pi)$

\begin{mathpar}
  \inferrule* [lab=collection] {} {\meaningof{true} = \pi, \and \meaningof{~E} = \pi \setminus \meaningof{E}, \and \meaningof{E_{1} \& E_{2}} = \meaningof{E_{1}} \cap \meaningof{E_{2}}}
\end{mathpar}

\begin{mathpar}
  \inferrule* [lab=structure] {} {\meaningof{0} = \{ P \in \pi | P \equiv 0 \}, \and \\ \meaningof{E_1 | E_2} = \{ P \in \pi | P \equiv P_{1} | P_{2}, P_{1} \in \meaningof{E_{1}}, P_{2} \in \meaningof{E_2}\} }
\end{mathpar}

\begin{mathpar}
 \inferrule* [lab=behavior] {} {\meaningof{\langle a?b \rangle E} = \{ P \in \pi | P \equiv Q | u?(y)P', \\ \and \\\\ \and \\ \;\;\; u \in \meaningof{a}, \forall z.P'\{z/y\} \in \meaningof{E\{z/b\}}\}, \and \\ \meaningof{a!E} = \{ P \in \pi | P \equiv Q | x!\langle P' \rangle, x \in \meaningof{a} P' \in \meaningof{E}\} }
\end{mathpar}

\begin{mathpar}
 \inferrule* [lab=nominal] {} {\meaningof{\quotep{E}} = \{ \quotep{P} \in \quotep{\pi} | P \in \meaningof{E} \}, \and \meaningof{\quotep{P}} = \{ \quotep{Q} \in \quotep{\pi} | P \equiv Q \} \and \\ \meaningof{@\quotep{E}} = \{ P \in \pi | P \equiv @x, x \in \meaningof{E} \}}
\end{mathpar}

\begin{eqnarray*}
  \\
  \meaningof{-} : TS \to ST
\end{eqnarray*}

\begin{eqnarray*}
  \\
  L : TS \to ST
\end{eqnarray*}

\begin{eqnarray*}
  \\
  P \models E \iff P \in \meaningof{E}
\end{eqnarray*}

\begin{eqnarray*}
  P \approx_{L} Q \iff \forall E \in L. P \models E \iff Q \models E
\end{eqnarray*}

\begin{eqnarray*}
  P \approx_{K} Q
\end{eqnarray*}

\begin{eqnarray*}
  P \approx Q
\end{eqnarray*}

$\approx_{K} = \approx = \approx_{L}$

\subsubsection{Contextual duality}

Note that contexts extend the quotation operation to a family of
operations from processes to names. Given a context, $M$, we can
define a \emph{nominal context}, $\quotep{M}$ by $\quotep{M}[P] :=
\quotep{M[P]}$. To foreshadow what is to come we observe that these
operations enjoy a duality with processes very much like the duality
between vectors and maps from vectors to scalars.

Further, because the calculus is essentially higher-order, we have a
correspondence between contexts and processes. More specifically,
given a name $x$ and a context $M$ we can construct $M^{*}_{x}$ such
that 

\begin{mathpar}
  M^{*}_{x} | \lift{x}{P} \red M[P]
\end{mathpar}

namely,

\begin{mathpar}
  M^{*}_{x} := x?(u).M[\dropn{u}]
\end{mathpar}

The dependence of $M^{*}_{x}$ on a name makes it an abstraction, 

\begin{mathpar}
  M^{*} := (x)x?(u).M[\dropn{u}]
\end{mathpar}

\subsection{Additional notation}

It will sometimes be convenient to denote the process a name
quotes. We already have the notation $x = \quotep{P}$, but it will be
convenient to introduce an alternate notation, $\procn{x}$, when we
want to emphasize the connection to the use of the name. Note that, by
virtue of name equivalence, $\quotep{\procn{x}} \nameeq x$; so, the
notation is consistent with previous definitions.

Further, because names have structure it is possible to effect
substitutions on the basis of that structure. This means we need to
upgrade our notation for substitutions, which we accomplish by
adapting comprehension notation. Thus,

\begin{mathpar}
  P\{ y / x : x \in S \}
\end{mathpar}

is interpreted to mean the process derived from P by replacing (in a
capture-avoiding manner) each occurrence of $x$ in $S$ by $y$. For example,

\begin{mathpar}
  P\{ \quotep{\procn{x}|\procn{x}} / x : x \in \freenames{P} \}
\end{mathpar}

will replace each (occurrence) of a free name $x$ in $P$ by
$\quotep{\procn{x}|\procn{x}}$.

Also, we will avail ourselves of the notation $x^{L}$ and $x^{R}$ to
denote injections of a name into disjoint copies of the name
space. There are numerous ways to accomplish this. One example can be
found in \cite{MeredithR05}. This notation overloads to vectors of
names: $\vec{x}^{\pi} := (x_{i}^{\pi} \; : \; 0 \leq i < |\vec{x}| )$ where $\pi \in \{L,R\}$.

We also use $P^{\Box} := P|\Box$.

In \cite{MeredithR05} an interpretation of the new operator is
given. It turns out that there are several possible interpretations
all enjoying the requisite algebraic properties of the operator (see
\cite{milner91polyadicpi}). We will therefore make liberal use of
$(\nu\; \vec{x})P$.

% subsection the_syntax_and_semantics_of_the_notation_system (end)   

\input{qm2pi.qmops} 

\input{qm2pi.sterngerlach} 

\input{qm2pi.metric} 

% section concurrent_process_calculi (end)

%\input{qm2pi.proofsketch}

% section proof sketch (end)

%\input{qm2pi.slviaknots} 

% section spatial logic via knots (end)

\input{qm2pi.conclusion}

% section conclusion (end)

%\input{qm2pi.dtcodes} 

% section wiring algorithm (end)

\input{qm2pi.ack} 

% section acknowledgments (end)

\newpage


\bibliographystyle{plain}   
\bibliography{../../biblios/main.bib}

\input{qm2pi.rhodetails}

\end{document}

 

% section acknowledgments (end)

\newpage


\bibliographystyle{plain}   
\bibliography{../../biblios/main.bib}

\documentclass[12pt]{llncs}
%\documentclass{jktr}

\usepackage[pdftex]{hyperref}                   
\usepackage {listings}
\usepackage {mathpartir}
\usepackage{bcprules}
%\usepackage{listings}
                       
\usepackage{graphicx} 
%\usepackage[margins=2.5cm,nohead,nofoot]{geometry}
%\usepackage{geometry}
\usepackage{amsfonts}
\usepackage{amstext}
\usepackage{latexsym}
\usepackage{amssymb}
\usepackage{color}


%\include{myPreamble}
\include{qm2pi.local} 

%\ifpdf
%\usepackage[pdftex]{graphicx}
%\else
%\usepackage{graphicx}
%\fi

 % \ifpdf
%  \usepackage{pdfsync}
%  \if


%\title{Brief Article}
%\author{David F. Snyder}
%\author{L.G. Meredith}

%\address{Dept. of Math., Texas State University--San Marcos, San Marcos, TX 78666}
       
\pagestyle{empty}


\begin{document}

\lstset{language=[Objective]Caml,frame=shadowbox}

\input{qm2pi.front}

% section front matter (end)

\input{qm2pi.intro} 
 
% section introduction (end)

% \input{qm2pi.knotations} 

% section notation (end)

\input{qm2pi.process.calculi} 

% section concurrent_process_calculi_and_spatial_logics_ (end)
    
%\input{qm2pi.knots2pi} 

%\input{qm2pi.trefoil} 

%\input{qm2pi.mainthm} 

% subsection basic_interpretation (end)

%\input{qm2pi.rho.presentation} 
\subsection{The syntax and semantics of the notation system}\label{sub:the_syntax_and_semantics_of_the_notation_system} % (fold)

We now summarize a technical presentation of the calculus that
embodies our theory of dynamics. The typical presentation of such a
calculus follows the style of giving generators and relations on
them. The grammar, below, describing term constructors, freely
generates the set of processes, $\Proc$. This set is then quotiented
by a relation known as structural congruence and it is over this set
that the notion of dynamics is expressed. This presentation is
essentially that of \cite{MeredithR05} with the addition of
polyadicity and summation. For readability we have relegated some of
the technical subtleties to an appendix.

\subsubsection{Process grammar}\label{subsub:process_grammar}

\begin{mathpar}
  \inferrule* [lab=synchronization] {} {{M} \bc \pzero \;|\; x?F \;|\; x!C }
  \and
  \inferrule* [lab=abstraction] {} {{F} \bc (x)P}
  \and
  \inferrule* [lab=concretion] {} {{C} \bc \langle Q \rangle}
  \and
  \inferrule* [lab=process] {} {{P,Q} \bc M \;| \;P|Q \;|\; @{x}}
  \and
  \inferrule* [lab=name] {} {{x} \bc \quotep{P}}
\end{mathpar} 

Note that $\vec{x}$ (resp. $\vec{P}$) denotes a vector of names
(resp. processes) of length $|\vec{x}|$ (resp. $|\vec{P}|$). We adopt
the following useful abbreviations.

\begin{mathpar}
   x?(\vec{y}).P := x.(\vec{y})P \and  x\clift{\vec{P}} := x.\clift{\vec{P}}
   \and x!(y) := \lift{x}{\dropn{y}}
   \and \Pi_{i=0}^{n-1}P_i := P_0 | \ldots | P_{n-1}
\end{mathpar}

\subsubsection{Structural congruence}

\paragraph{Free and bound names and alpha-equivalence.} At the
core of structural equivalence is alpha-equivalence which identifies
process that are the same up to a change of variable. Formally, we
recognize the distinction between free and bound names. The free names
of a process, $\freenames{P}$, may be calculated recursively as
follows:

\begin{mathpar}
\freenames{\pzero} := \emptyset
  \and \\
  \freenames{x?(y).P} := \{ x \} \cup (\freenames{P} \setminus \{ y \})
  \and 
  \freenames{x!\langle P \rangle} := \{ x \} \cup \{ P \} 
  \and \\
  \freenames{P|Q} := \freenames{P} \cup \freenames{Q}
  \and \\
  \freenames{@{x}} := \{ x \}
\end{mathpar}

$\pi$
$\quotep{\pi}$

$\freenames{-} : \pi \to \mathcal{P}(\quotep{\pi})$

\begin{eqnarray*}
  \freenames{\pzero} & := & \emptyset \\
  \freenames{x?(y).P} & := & \{ x \} \cup (\freenames{P} \setminus \{ y \}) \\
  \freenames{x!\langle P \rangle} & := & \{ x \} \cup \{ P \} \\
  \freenames{P|Q} & := & \freenames{P} \cup \freenames{Q} \\
  \freenames{\dropn{x}} & := & \{ x \}
\end{eqnarray*}

The bound names of a process, $\boundnames{P}$, are those names occurring in $P$
that are not free. For example, in $x?(y).0$, the name $x$ is free, while $y$ is bound.

\begin{mathpar}
  \inferrule* [lab=monoidal-laws] {} { P|Q \equiv Q|P \and P|0 \equiv P \and P|(Q|R) \equiv (P|Q)|R }
\end{mathpar}

\begin{mathpar}
  \inferrule* [lab=alpha-equivalence] {} { (x)P \equiv (y)P\{y/x\} \and y \not\in \freenames{P} }
\end{mathpar}

\begin{definition}
Then two processes, $P,Q$, are alpha-equivalent if $P = Q\{\vec{y}/\vec{x}\}$ for
some $\vec{x} \in \boundnames{Q},\vec{y} \in \boundnames{P}$, where $Q\{\vec{y}/\vec{x}\}$
denotes the capture-avoiding substitution of $\vec{y}$ for $\vec{x}$ in $Q$.
\end{definition}

\begin{definition}
  The {\em structural congruence} \cite{SangiorgiWalker} , $\equiv$,
  between processes is the least congruence containing
  alpha-equivalence, satisfying the abelian monoid laws
  (associativity, commutativity and $\pzero$ as identity) for parallel
  composition $|$ and for summation $+$.
\end{definition}

\subsection{Name equivalence}

We take name equivalence, written $\nameeq$, to be the smallest
equivalence relation generated by the following rules.

\begin{mathpar}
\inferrule*[lab=Quote-drop]
{ }
{ \quotep{@{x}} \nameeq x }

\inferrule*[lab=Struct-equiv]
{ P \scong Q }
{ \quotep{P} \nameeq \quotep{Q} }
\end{mathpar}

The astute reader will have noticed that the mutual recursion of names
and processes imposes a mutual recursion on alpha-equivalence and
structural equivalence via name-equivalence. Fortunately, all of this
works out pleasantly and we may calculate in the natural way, free of
concern. The reader interested in the details is referred to the
appendix \ref{appendix:rho_details}.

\subsection{Substitution}

We use $\Proc$ for the set of processes, $\QProc$ for the set of
names, and $\id{\{}\vec{y} / \vec{x} \id{\}}$ to denote partial maps,
$s : \QProc \rightarrow \QProc$. A map, $s$ lifts, uniquely, to a map
on process terms, $\widehat{s} : \Proc \rightarrow \Proc$ by the
following equations.

\begin{mathpar}
  (0) \psubstp{Q}{P} := 0 \\
  (R \juxtap S) \psubstp{Q}{P}
  :=    
  (R)\psubstp{Q}{P} \juxtap (S) \psubstp{Q}{P} \\
  (x?(y).R) \psubstp{Q}{P}    
  :=    
  (x)\substp{Q}{P} (z)\concat( (R \psubstn{z}{y}) \psubstp{Q}{P} ) \\
  (\lift{x}{R}) \psubstp{Q}{P}  
  :=
  \lift{(x)\substp{Q}{P}}{ R \psubstp{Q}{P} } \\
%   (\dropn{x})  \psubstp{Q}{P}       
%   := 
%   \left\{ 
%     \begin{array}{ccc} 
%       \dropn{\quotep{Q}} & & x \nameeq \quotep{P} \\
%       \dropn{x} & & otherwise \\
%     \end{array}
%   \right. 
  (\dropn{x})  \psubstp{Q}{P}       
  := 
  \left\{ 
    \begin{array}{ccc} 
      Q & & x \nameeq \quotep{P} \\
      \dropn{x} & & otherwise \\
    \end{array}
  \right.
\end{mathpar}
 

where

\begin{eqnarray}
  (x)\id{\{} \lpquote Q \rpquote / \lpquote P \rpquote \id{\}}            = 
  \left\{ 
    \begin{array}{ccc}
      \lpquote Q \rpquote & & x \nameeq \lpquote P \rpquote \\
      x & & otherwise \\
    \end{array}
  \right. \nonumber
\end{eqnarray}

and $z$ is chosen distinct from $\quotep{P}$, $\quotep{Q}$, the free
names in $Q$, and all the names in $R$. Our $\alpha$-equivalence will
be built in the standard way from this substitution.

\begin{remark}\label{rem:no_self_referential_names}
  One consequence of these definitions is that $\forall P. \quotep{P}
  \not\in \freenames{P}$.
\end{remark}

\subsection{ Dynamic quote: an example }

Anticipating something of what's to come, consider applying the
substitution, $\widehat{\id{\{}u / z \id{\}}}$, to the following pair
of processes, $\lift{w}{y!(z)}$ and $w[ \lpquote y!(z) \rpquote ]$.

\begin{eqnarray}
	\lift{w}{y!(z)}\widehat{\id{\{}u / z \id{\}}}
		& = &
		\lift{w}{y!(u)} \nonumber\\
	w[ \lpquote y!(z) \rpquote ] \widehat{ \id{\{}u / z \id{\}} }
		& = &
		w[ \lpquote y!(z) \rpquote ] \nonumber
\end{eqnarray}

Because the body of the process between quotes is impervious to
substitution, we get radically different answers. In fact, by
examining the first process in an input context,
e.g. $x?(z).\lift{w}{y!(z)}$, we see that the process under the lift
operator may be shaped by prefixed inputs binding a name inside it. In
this sense, the lift operator will be seen as a way to dynamically
construct processes before reifying them as names.

Finally equipped with these standard features we can present the
dynamics of the calculus.

\subsubsection{Operational semantics} 

Finally, we introduce the computational dynamics. What marks these
algebras as distinct from other more traditionally studied algebraic
structures, e.g. vector spaces or polynomial rings, is the manner in
which dynamics is captured. In traditional structures, dynamics is typically
expressed through morphisms between such structures, as in linear maps
between vector spaces or morphisms between rings. In algebras
associated with the semantics of computation, the dynamics is
expressed as part of the algebraic structure itself, through a
reduction reduction relation typically denoted by $\red$. Below, we
give a recursive presentation of this relation for the calculus used
in the encoding.

$\red \subseteq \pi \times \pi$
$\red : \pi \to \mathcal{P}(\pi)$

\begin{mathpar}
  \inferrule* [lab=Comm] { \textsf{match}( x_{src}, x_{trgt} ) } { x_{trgt}?(y)P \; | \; x_{src}!\langle {Q} \rangle \red P\{\quotep{Q}/y}\} }
  \and \\
  \inferrule* [lab=Par] {{P} \red {P}'} {{{P} | {Q}} \red {{P}' | {Q}}}
  \and
  \inferrule* [lab=Equiv]{{{P} \scong {P}'} \andalso {{P}' \red {Q}'} \andalso {{Q}' \scong {Q}}}{{P} \red {Q}}
\end{mathpar}

\begin{eqnarray*}
  match_{\equiv} (\quotep{P},\quotep{Q}) & := & P \equiv Q \\
  match_{\dagger}(\quotep{P},\quotep{Q}) & := & \forall R. P|Q \red^{*} R => R \red^{*} 0 \\
  match_{K}(\quotep{P},\quotep{Q}) & := & K \mbox{ for some context } K
\end{eqnarray*}

$u?(x)P | u!\langle Q \rangle \red P\{\quotep{Q}/x\}$

%We write $\wred$ for $\red^*$, and $P\red$ if $\exists Q $ such that $ P \red Q$.
We write $P\red$ if $\exists Q $ such that $ P \red Q$ and $P\not\red$, otherwise.

\section{Replication}

As mentioned before, it is known that replication (and hence
recursion) can be implemented in a higher-order process algebra
\cite{SangiorgiWalker}. As our first example of calculation with the
machinery thus far presented we give the construction explicitly in
the {\rhoc}.

\begin{eqnarray}
	D_{x} & := & \prefix{x}{y}{(\binpar{\outputp{x}{y}}{@{y}})} \nonumber\\
	\bangp_{x}{P} & := & \binpar{{x}!\langle{\binpar{D_{x}}{P}}\rangle}{D_{x}} \nonumber
\end{eqnarray}

\begin{eqnarray}
	\bangp_{x}{P} & & \nonumber\\
	=
	& {x}!\langle{(\prefix{x}{y}{(\outputp{x}{y} | @{y})) | P}}\rangle 
	      | \prefix{x}{y}{(\outputp{x}{y} | @{y})} & \nonumber\\
	\red
	& (\outputp{x}{y} | @{y})\substn{\quotep{(\prefix{x}{y}{(@{y} | \outputp{x}{y})) | P}}}{y} & \nonumber\\
	=
	& \outputp{x}{\quotep{(\prefix{x}{y}{(\outputp{x}{y} | @{y})) | P}}}
	  | {(\prefix{x}{y}{(\outputp{x}{y} | @{y})) | P}} & \nonumber\\
	\red
	& \ldots & \nonumber\\
	\red^*
	& P | P | \ldots & \nonumber
\end{eqnarray}

Of course, this encoding, as an implementation, runs away, unfolding
$\bangp{P}$ eagerly. A lazier and more implementable replication
operator, restricted to input-guarded processes, may be obtained as follows.

\begin{eqnarray}
\bangp{\prefix{u}{v}{P}} 
	:= 
	\binpar{\lift{x}{\prefix{u}{v}{(\binpar{D(x)}{P})}}}{D(x)} \nonumber
\end{eqnarray}

\begin{remark}
  Note that the lazier definition still does not deal with summation
  or mixed summation (i.e. sums over input and output). The reader is
  invited to construct definitions of replication that deal with these
  features. 

  Further, the definitions are parameterized in a name, $x$. Can you,
  gentle reader, make a definition that eliminates this parameter and
  guarantees no accidental interaction between the replication
  machinery and the process being replicated -- i.e. no accidental
  sharing of names used by the process to get its work done and the
  name(s) used by the replication to effect copying. This latter
  revision of the definition of replication is crucial to obtaining
  the expected identity $!!P \sim !P$.
\end{remark}

\begin{remark}\label{rem:paradoxical_combinator}
  The reader familiar with the lambda calculus will have noticed the
  similarity between $D$ and the paradoxical combinator.

  [Ed. note: the existence of this seems to suggest we have to be more
  restrictive on the set of processes and names we admit if we are to
  support no-cloning.]
\end{remark}

\subsubsection{Bisimulation}

The computational dynamics gives rise to another kind of equivalence,
the equivalence of computational behavior. As previously mentioned
this is typically captured \emph{via} some form of bisimulation.

% The notion we use in this paper is weak barbed bisimulation
% \cite{milner91polyadicpi}.

The notion we use in this paper is derived from weak barbed
bisimulation \cite{milner91polyadicpi}. 

\begin{definition}
An \emph{observation relation}, $\downarrow_{\mathcal N}$, over a set
of names, $\mathcal N$, is the smallest relation satisfying the rules
below.

\infrule[Out-barb]{y \in {\mathcal N}, \; x \nameeq y}
		  {\outputp{x}{v} \downarrow_{\mathcal N} x}
\infrule[Par-barb]{\mbox{$P\downarrow_{\mathcal N} x$ or $Q\downarrow_{\mathcal N} x$}}
		  {\binpar{P}{Q} \downarrow_{\mathcal N} x}

We write $P \Downarrow_{\mathcal N} x$ if there is $Q$ such that 
$P \wred Q$ and $Q \downarrow_{\mathcal N} x$.
\end{definition}

\begin{definition}
%\label{def.bbisim}
An  ${\mathcal N}$-\emph{barbed bisimulation} over a set of names, ${\mathcal N}$, is a symmetric binary relation 
${\mathcal S}_{\mathcal N}$ between agents such that $P\rel{S}_{\mathcal N}Q$ implies:
\begin{enumerate}
\item If $P \red P'$ then $Q \wred Q'$ and $P'\rel{S}_{\mathcal N} Q'$.
\item If $P\downarrow_{\mathcal N} x$, then $Q\Downarrow_{\mathcal N} x$.
\end{enumerate}
$P$ is ${\mathcal N}$-barbed bisimilar to $Q$, written
$P \wbbisim_{\mathcal N} Q$, if $P \rel{S}_{\mathcal N} Q$ for some ${\mathcal N}$-barbed bisimulation ${\mathcal S}_{\mathcal N}$.
\end{definition}

$\mathcal{R} \subseteq \pi \times \pi$

$P \mathcal{R} Q => \forall P'. P \red P' \Rightarrow \exists Q'. Q \red Q', P' \mathcal{R} Q'$

$P \vdash x \Rightarrow Q \vdash x$

\begin{mathpar}
  \inferrule*[lab=Out-barb]{x \nameeq y}{{y}!\langle{Q}\rangle \vdash x}
  \and
  \inferrule*[lab=Par-barb]{\mbox{$P\vdash x$ or $Q\vdash x$}}{\binpar{P}{Q} \vdash x}
\end{mathpar}

\subsubsection{Contexts}

One of the principle advantages of computational calculi like the
$\pi$-calculus is a well-defined notion of context,
contextual-equivalence and a correlation between
contextual-equivalence and notions of bisimulation. The notion of
context allows the decomposition of a process into (sub-)process and
its syntactic environment, its context. Thus, a context may be
thought of as a process with a ``hole'' (written $\Box$) in it. The
application of a context $M$ to a process $P$, written $M[P]$, is
tantamount to filling the hole in $M$ with $P$. In this paper we do
not need the full weight of this theory, but do make use of the notion
of context in the proof the main theorem. 

\begin{mathpar}
  \inferrule* [lab=summation] {} {{M_{M},M_{N}} \bc \Box \;|\; x.M_{A} \;|\; M_{M}+M_{N}}
  \and
  \inferrule* [lab=agent] {} {{M_{A}} \bc (\vec{x})M_{P} \;| \; \clift{P_0,\ldots,M_{P},\ldots,P_N}}
  \and \\
  \inferrule* [lab=process] {} {{M_{P}} \bc M_{N} \;| \;P|M_{P} }
\end{mathpar} 

\begin{mathpar}
  \inferrule* [lab=sychronization] {} {M_{N} \bc \Box \;|\; x?M_{F} \;|\; x!M_{C}}
  \and
  \inferrule* [lab=abstraction] {} {{M_{F}} \bc (x)M_{P} }
  \and
  \inferrule* [lab=concretion] {} {{M_{C}} \bc \langle M_{P} \rangle }
  \and \\
  \inferrule* [lab=process] {} {{M_{P}} \bc M_{N} \;| \;P|M_{P} }
\end{mathpar}

\begin{definition}[contextual application] Given a context $M$, and
  process $P$, we define the \emph{contextual application}, $M[P] :=
  M\{P/\Box\}$. That is, the contextual application of M to P is the
  substitution of $P$ for $\Box$ in $M$.
\end{definition}

$\meaningof{-} : L \to \mathcal{P}(\pi)$

\begin{mathpar}
  \inferrule* [lab=collection] {} {\meaningof{true} = \pi, \and \meaningof{~E} = \pi \setminus \meaningof{E}, \and \meaningof{E_{1} \& E_{2}} = \meaningof{E_{1}} \cap \meaningof{E_{2}}}
\end{mathpar}

\begin{mathpar}
  \inferrule* [lab=structure] {} {\meaningof{0} = \{ P \in \pi | P \equiv 0 \}, \and \\ \meaningof{E_1 | E_2} = \{ P \in \pi | P \equiv P_{1} | P_{2}, P_{1} \in \meaningof{E_{1}}, P_{2} \in \meaningof{E_2}\} }
\end{mathpar}

\begin{mathpar}
 \inferrule* [lab=behavior] {} {\meaningof{\langle a?b \rangle E} = \{ P \in \pi | P \equiv Q | u?(y)P', \\ \and \\\\ \and \\ \;\;\; u \in \meaningof{a}, \forall z.P'\{z/y\} \in \meaningof{E\{z/b\}}\}, \and \\ \meaningof{a!E} = \{ P \in \pi | P \equiv Q | x!\langle P' \rangle, x \in \meaningof{a} P' \in \meaningof{E}\} }
\end{mathpar}

\begin{mathpar}
 \inferrule* [lab=nominal] {} {\meaningof{\quotep{E}} = \{ \quotep{P} \in \quotep{\pi} | P \in \meaningof{E} \}, \and \meaningof{\quotep{P}} = \{ \quotep{Q} \in \quotep{\pi} | P \equiv Q \} \and \\ \meaningof{@\quotep{E}} = \{ P \in \pi | P \equiv @x, x \in \meaningof{E} \}}
\end{mathpar}

\begin{eqnarray*}
  \\
  \meaningof{-} : TS \to ST
\end{eqnarray*}

\begin{eqnarray*}
  \\
  L : TS \to ST
\end{eqnarray*}

\begin{eqnarray*}
  \\
  P \models E \iff P \in \meaningof{E}
\end{eqnarray*}

\begin{eqnarray*}
  P \approx_{L} Q \iff \forall E \in L. P \models E \iff Q \models E
\end{eqnarray*}

\begin{eqnarray*}
  P \approx_{K} Q
\end{eqnarray*}

\begin{eqnarray*}
  P \approx Q
\end{eqnarray*}

$\approx_{K} = \approx = \approx_{L}$

\subsubsection{Contextual duality}

Note that contexts extend the quotation operation to a family of
operations from processes to names. Given a context, $M$, we can
define a \emph{nominal context}, $\quotep{M}$ by $\quotep{M}[P] :=
\quotep{M[P]}$. To foreshadow what is to come we observe that these
operations enjoy a duality with processes very much like the duality
between vectors and maps from vectors to scalars.

Further, because the calculus is essentially higher-order, we have a
correspondence between contexts and processes. More specifically,
given a name $x$ and a context $M$ we can construct $M^{*}_{x}$ such
that 

\begin{mathpar}
  M^{*}_{x} | \lift{x}{P} \red M[P]
\end{mathpar}

namely,

\begin{mathpar}
  M^{*}_{x} := x?(u).M[\dropn{u}]
\end{mathpar}

The dependence of $M^{*}_{x}$ on a name makes it an abstraction, 

\begin{mathpar}
  M^{*} := (x)x?(u).M[\dropn{u}]
\end{mathpar}

\subsection{Additional notation}

It will sometimes be convenient to denote the process a name
quotes. We already have the notation $x = \quotep{P}$, but it will be
convenient to introduce an alternate notation, $\procn{x}$, when we
want to emphasize the connection to the use of the name. Note that, by
virtue of name equivalence, $\quotep{\procn{x}} \nameeq x$; so, the
notation is consistent with previous definitions.

Further, because names have structure it is possible to effect
substitutions on the basis of that structure. This means we need to
upgrade our notation for substitutions, which we accomplish by
adapting comprehension notation. Thus,

\begin{mathpar}
  P\{ y / x : x \in S \}
\end{mathpar}

is interpreted to mean the process derived from P by replacing (in a
capture-avoiding manner) each occurrence of $x$ in $S$ by $y$. For example,

\begin{mathpar}
  P\{ \quotep{\procn{x}|\procn{x}} / x : x \in \freenames{P} \}
\end{mathpar}

will replace each (occurrence) of a free name $x$ in $P$ by
$\quotep{\procn{x}|\procn{x}}$.

Also, we will avail ourselves of the notation $x^{L}$ and $x^{R}$ to
denote injections of a name into disjoint copies of the name
space. There are numerous ways to accomplish this. One example can be
found in \cite{MeredithR05}. This notation overloads to vectors of
names: $\vec{x}^{\pi} := (x_{i}^{\pi} \; : \; 0 \leq i < |\vec{x}| )$ where $\pi \in \{L,R\}$.

We also use $P^{\Box} := P|\Box$.

In \cite{MeredithR05} an interpretation of the new operator is
given. It turns out that there are several possible interpretations
all enjoying the requisite algebraic properties of the operator (see
\cite{milner91polyadicpi}). We will therefore make liberal use of
$(\nu\; \vec{x})P$.

% subsection the_syntax_and_semantics_of_the_notation_system (end)   

\input{qm2pi.qmops} 

\input{qm2pi.sterngerlach} 

\input{qm2pi.metric} 

% section concurrent_process_calculi (end)

%\input{qm2pi.proofsketch}

% section proof sketch (end)

%\input{qm2pi.slviaknots} 

% section spatial logic via knots (end)

\input{qm2pi.conclusion}

% section conclusion (end)

%\input{qm2pi.dtcodes} 

% section wiring algorithm (end)

\input{qm2pi.ack} 

% section acknowledgments (end)

\newpage


\bibliographystyle{plain}   
\bibliography{../../biblios/main.bib}

\input{qm2pi.rhodetails}

\end{document}



\end{document}

 

% section concurrent_process_calculi (end)

%\documentclass[12pt]{llncs}
%\documentclass{jktr}

\usepackage[pdftex]{hyperref}                   
\usepackage {listings}
\usepackage {mathpartir}
\usepackage{bcprules}
%\usepackage{listings}
                       
\usepackage{graphicx} 
%\usepackage[margins=2.5cm,nohead,nofoot]{geometry}
%\usepackage{geometry}
\usepackage{amsfonts}
\usepackage{amstext}
\usepackage{latexsym}
\usepackage{amssymb}
\usepackage{color}


%\include{myPreamble}
\documentclass[12pt]{llncs}
%\documentclass{jktr}

\usepackage[pdftex]{hyperref}                   
\usepackage {listings}
\usepackage {mathpartir}
\usepackage{bcprules}
%\usepackage{listings}
                       
\usepackage{graphicx} 
%\usepackage[margins=2.5cm,nohead,nofoot]{geometry}
%\usepackage{geometry}
\usepackage{amsfonts}
\usepackage{amstext}
\usepackage{latexsym}
\usepackage{amssymb}
\usepackage{color}


%\include{myPreamble}
\include{qm2pi.local} 

%\ifpdf
%\usepackage[pdftex]{graphicx}
%\else
%\usepackage{graphicx}
%\fi

 % \ifpdf
%  \usepackage{pdfsync}
%  \if


%\title{Brief Article}
%\author{David F. Snyder}
%\author{L.G. Meredith}

%\address{Dept. of Math., Texas State University--San Marcos, San Marcos, TX 78666}
       
\pagestyle{empty}


\begin{document}

\lstset{language=[Objective]Caml,frame=shadowbox}

\input{qm2pi.front}

% section front matter (end)

\input{qm2pi.intro} 
 
% section introduction (end)

% \input{qm2pi.knotations} 

% section notation (end)

\input{qm2pi.process.calculi} 

% section concurrent_process_calculi_and_spatial_logics_ (end)
    
%\input{qm2pi.knots2pi} 

%\input{qm2pi.trefoil} 

%\input{qm2pi.mainthm} 

% subsection basic_interpretation (end)

%\input{qm2pi.rho.presentation} 
\subsection{The syntax and semantics of the notation system}\label{sub:the_syntax_and_semantics_of_the_notation_system} % (fold)

We now summarize a technical presentation of the calculus that
embodies our theory of dynamics. The typical presentation of such a
calculus follows the style of giving generators and relations on
them. The grammar, below, describing term constructors, freely
generates the set of processes, $\Proc$. This set is then quotiented
by a relation known as structural congruence and it is over this set
that the notion of dynamics is expressed. This presentation is
essentially that of \cite{MeredithR05} with the addition of
polyadicity and summation. For readability we have relegated some of
the technical subtleties to an appendix.

\subsubsection{Process grammar}\label{subsub:process_grammar}

\begin{mathpar}
  \inferrule* [lab=synchronization] {} {{M} \bc \pzero \;|\; x?F \;|\; x!C }
  \and
  \inferrule* [lab=abstraction] {} {{F} \bc (x)P}
  \and
  \inferrule* [lab=concretion] {} {{C} \bc \langle Q \rangle}
  \and
  \inferrule* [lab=process] {} {{P,Q} \bc M \;| \;P|Q \;|\; @{x}}
  \and
  \inferrule* [lab=name] {} {{x} \bc \quotep{P}}
\end{mathpar} 

Note that $\vec{x}$ (resp. $\vec{P}$) denotes a vector of names
(resp. processes) of length $|\vec{x}|$ (resp. $|\vec{P}|$). We adopt
the following useful abbreviations.

\begin{mathpar}
   x?(\vec{y}).P := x.(\vec{y})P \and  x\clift{\vec{P}} := x.\clift{\vec{P}}
   \and x!(y) := \lift{x}{\dropn{y}}
   \and \Pi_{i=0}^{n-1}P_i := P_0 | \ldots | P_{n-1}
\end{mathpar}

\subsubsection{Structural congruence}

\paragraph{Free and bound names and alpha-equivalence.} At the
core of structural equivalence is alpha-equivalence which identifies
process that are the same up to a change of variable. Formally, we
recognize the distinction between free and bound names. The free names
of a process, $\freenames{P}$, may be calculated recursively as
follows:

\begin{mathpar}
\freenames{\pzero} := \emptyset
  \and \\
  \freenames{x?(y).P} := \{ x \} \cup (\freenames{P} \setminus \{ y \})
  \and 
  \freenames{x!\langle P \rangle} := \{ x \} \cup \{ P \} 
  \and \\
  \freenames{P|Q} := \freenames{P} \cup \freenames{Q}
  \and \\
  \freenames{@{x}} := \{ x \}
\end{mathpar}

$\pi$
$\quotep{\pi}$

$\freenames{-} : \pi \to \mathcal{P}(\quotep{\pi})$

\begin{eqnarray*}
  \freenames{\pzero} & := & \emptyset \\
  \freenames{x?(y).P} & := & \{ x \} \cup (\freenames{P} \setminus \{ y \}) \\
  \freenames{x!\langle P \rangle} & := & \{ x \} \cup \{ P \} \\
  \freenames{P|Q} & := & \freenames{P} \cup \freenames{Q} \\
  \freenames{\dropn{x}} & := & \{ x \}
\end{eqnarray*}

The bound names of a process, $\boundnames{P}$, are those names occurring in $P$
that are not free. For example, in $x?(y).0$, the name $x$ is free, while $y$ is bound.

\begin{mathpar}
  \inferrule* [lab=monoidal-laws] {} { P|Q \equiv Q|P \and P|0 \equiv P \and P|(Q|R) \equiv (P|Q)|R }
\end{mathpar}

\begin{mathpar}
  \inferrule* [lab=alpha-equivalence] {} { (x)P \equiv (y)P\{y/x\} \and y \not\in \freenames{P} }
\end{mathpar}

\begin{definition}
Then two processes, $P,Q$, are alpha-equivalent if $P = Q\{\vec{y}/\vec{x}\}$ for
some $\vec{x} \in \boundnames{Q},\vec{y} \in \boundnames{P}$, where $Q\{\vec{y}/\vec{x}\}$
denotes the capture-avoiding substitution of $\vec{y}$ for $\vec{x}$ in $Q$.
\end{definition}

\begin{definition}
  The {\em structural congruence} \cite{SangiorgiWalker} , $\equiv$,
  between processes is the least congruence containing
  alpha-equivalence, satisfying the abelian monoid laws
  (associativity, commutativity and $\pzero$ as identity) for parallel
  composition $|$ and for summation $+$.
\end{definition}

\subsection{Name equivalence}

We take name equivalence, written $\nameeq$, to be the smallest
equivalence relation generated by the following rules.

\begin{mathpar}
\inferrule*[lab=Quote-drop]
{ }
{ \quotep{@{x}} \nameeq x }

\inferrule*[lab=Struct-equiv]
{ P \scong Q }
{ \quotep{P} \nameeq \quotep{Q} }
\end{mathpar}

The astute reader will have noticed that the mutual recursion of names
and processes imposes a mutual recursion on alpha-equivalence and
structural equivalence via name-equivalence. Fortunately, all of this
works out pleasantly and we may calculate in the natural way, free of
concern. The reader interested in the details is referred to the
appendix \ref{appendix:rho_details}.

\subsection{Substitution}

We use $\Proc$ for the set of processes, $\QProc$ for the set of
names, and $\id{\{}\vec{y} / \vec{x} \id{\}}$ to denote partial maps,
$s : \QProc \rightarrow \QProc$. A map, $s$ lifts, uniquely, to a map
on process terms, $\widehat{s} : \Proc \rightarrow \Proc$ by the
following equations.

\begin{mathpar}
  (0) \psubstp{Q}{P} := 0 \\
  (R \juxtap S) \psubstp{Q}{P}
  :=    
  (R)\psubstp{Q}{P} \juxtap (S) \psubstp{Q}{P} \\
  (x?(y).R) \psubstp{Q}{P}    
  :=    
  (x)\substp{Q}{P} (z)\concat( (R \psubstn{z}{y}) \psubstp{Q}{P} ) \\
  (\lift{x}{R}) \psubstp{Q}{P}  
  :=
  \lift{(x)\substp{Q}{P}}{ R \psubstp{Q}{P} } \\
%   (\dropn{x})  \psubstp{Q}{P}       
%   := 
%   \left\{ 
%     \begin{array}{ccc} 
%       \dropn{\quotep{Q}} & & x \nameeq \quotep{P} \\
%       \dropn{x} & & otherwise \\
%     \end{array}
%   \right. 
  (\dropn{x})  \psubstp{Q}{P}       
  := 
  \left\{ 
    \begin{array}{ccc} 
      Q & & x \nameeq \quotep{P} \\
      \dropn{x} & & otherwise \\
    \end{array}
  \right.
\end{mathpar}
 

where

\begin{eqnarray}
  (x)\id{\{} \lpquote Q \rpquote / \lpquote P \rpquote \id{\}}            = 
  \left\{ 
    \begin{array}{ccc}
      \lpquote Q \rpquote & & x \nameeq \lpquote P \rpquote \\
      x & & otherwise \\
    \end{array}
  \right. \nonumber
\end{eqnarray}

and $z$ is chosen distinct from $\quotep{P}$, $\quotep{Q}$, the free
names in $Q$, and all the names in $R$. Our $\alpha$-equivalence will
be built in the standard way from this substitution.

\begin{remark}\label{rem:no_self_referential_names}
  One consequence of these definitions is that $\forall P. \quotep{P}
  \not\in \freenames{P}$.
\end{remark}

\subsection{ Dynamic quote: an example }

Anticipating something of what's to come, consider applying the
substitution, $\widehat{\id{\{}u / z \id{\}}}$, to the following pair
of processes, $\lift{w}{y!(z)}$ and $w[ \lpquote y!(z) \rpquote ]$.

\begin{eqnarray}
	\lift{w}{y!(z)}\widehat{\id{\{}u / z \id{\}}}
		& = &
		\lift{w}{y!(u)} \nonumber\\
	w[ \lpquote y!(z) \rpquote ] \widehat{ \id{\{}u / z \id{\}} }
		& = &
		w[ \lpquote y!(z) \rpquote ] \nonumber
\end{eqnarray}

Because the body of the process between quotes is impervious to
substitution, we get radically different answers. In fact, by
examining the first process in an input context,
e.g. $x?(z).\lift{w}{y!(z)}$, we see that the process under the lift
operator may be shaped by prefixed inputs binding a name inside it. In
this sense, the lift operator will be seen as a way to dynamically
construct processes before reifying them as names.

Finally equipped with these standard features we can present the
dynamics of the calculus.

\subsubsection{Operational semantics} 

Finally, we introduce the computational dynamics. What marks these
algebras as distinct from other more traditionally studied algebraic
structures, e.g. vector spaces or polynomial rings, is the manner in
which dynamics is captured. In traditional structures, dynamics is typically
expressed through morphisms between such structures, as in linear maps
between vector spaces or morphisms between rings. In algebras
associated with the semantics of computation, the dynamics is
expressed as part of the algebraic structure itself, through a
reduction reduction relation typically denoted by $\red$. Below, we
give a recursive presentation of this relation for the calculus used
in the encoding.

$\red \subseteq \pi \times \pi$
$\red : \pi \to \mathcal{P}(\pi)$

\begin{mathpar}
  \inferrule* [lab=Comm] { \textsf{match}( x_{src}, x_{trgt} ) } { x_{trgt}?(y)P \; | \; x_{src}!\langle {Q} \rangle \red P\{\quotep{Q}/y}\} }
  \and \\
  \inferrule* [lab=Par] {{P} \red {P}'} {{{P} | {Q}} \red {{P}' | {Q}}}
  \and
  \inferrule* [lab=Equiv]{{{P} \scong {P}'} \andalso {{P}' \red {Q}'} \andalso {{Q}' \scong {Q}}}{{P} \red {Q}}
\end{mathpar}

\begin{eqnarray*}
  match_{\equiv} (\quotep{P},\quotep{Q}) & := & P \equiv Q \\
  match_{\dagger}(\quotep{P},\quotep{Q}) & := & \forall R. P|Q \red^{*} R => R \red^{*} 0 \\
  match_{K}(\quotep{P},\quotep{Q}) & := & K \mbox{ for some context } K
\end{eqnarray*}

$u?(x)P | u!\langle Q \rangle \red P\{\quotep{Q}/x\}$

%We write $\wred$ for $\red^*$, and $P\red$ if $\exists Q $ such that $ P \red Q$.
We write $P\red$ if $\exists Q $ such that $ P \red Q$ and $P\not\red$, otherwise.

\section{Replication}

As mentioned before, it is known that replication (and hence
recursion) can be implemented in a higher-order process algebra
\cite{SangiorgiWalker}. As our first example of calculation with the
machinery thus far presented we give the construction explicitly in
the {\rhoc}.

\begin{eqnarray}
	D_{x} & := & \prefix{x}{y}{(\binpar{\outputp{x}{y}}{@{y}})} \nonumber\\
	\bangp_{x}{P} & := & \binpar{{x}!\langle{\binpar{D_{x}}{P}}\rangle}{D_{x}} \nonumber
\end{eqnarray}

\begin{eqnarray}
	\bangp_{x}{P} & & \nonumber\\
	=
	& {x}!\langle{(\prefix{x}{y}{(\outputp{x}{y} | @{y})) | P}}\rangle 
	      | \prefix{x}{y}{(\outputp{x}{y} | @{y})} & \nonumber\\
	\red
	& (\outputp{x}{y} | @{y})\substn{\quotep{(\prefix{x}{y}{(@{y} | \outputp{x}{y})) | P}}}{y} & \nonumber\\
	=
	& \outputp{x}{\quotep{(\prefix{x}{y}{(\outputp{x}{y} | @{y})) | P}}}
	  | {(\prefix{x}{y}{(\outputp{x}{y} | @{y})) | P}} & \nonumber\\
	\red
	& \ldots & \nonumber\\
	\red^*
	& P | P | \ldots & \nonumber
\end{eqnarray}

Of course, this encoding, as an implementation, runs away, unfolding
$\bangp{P}$ eagerly. A lazier and more implementable replication
operator, restricted to input-guarded processes, may be obtained as follows.

\begin{eqnarray}
\bangp{\prefix{u}{v}{P}} 
	:= 
	\binpar{\lift{x}{\prefix{u}{v}{(\binpar{D(x)}{P})}}}{D(x)} \nonumber
\end{eqnarray}

\begin{remark}
  Note that the lazier definition still does not deal with summation
  or mixed summation (i.e. sums over input and output). The reader is
  invited to construct definitions of replication that deal with these
  features. 

  Further, the definitions are parameterized in a name, $x$. Can you,
  gentle reader, make a definition that eliminates this parameter and
  guarantees no accidental interaction between the replication
  machinery and the process being replicated -- i.e. no accidental
  sharing of names used by the process to get its work done and the
  name(s) used by the replication to effect copying. This latter
  revision of the definition of replication is crucial to obtaining
  the expected identity $!!P \sim !P$.
\end{remark}

\begin{remark}\label{rem:paradoxical_combinator}
  The reader familiar with the lambda calculus will have noticed the
  similarity between $D$ and the paradoxical combinator.

  [Ed. note: the existence of this seems to suggest we have to be more
  restrictive on the set of processes and names we admit if we are to
  support no-cloning.]
\end{remark}

\subsubsection{Bisimulation}

The computational dynamics gives rise to another kind of equivalence,
the equivalence of computational behavior. As previously mentioned
this is typically captured \emph{via} some form of bisimulation.

% The notion we use in this paper is weak barbed bisimulation
% \cite{milner91polyadicpi}.

The notion we use in this paper is derived from weak barbed
bisimulation \cite{milner91polyadicpi}. 

\begin{definition}
An \emph{observation relation}, $\downarrow_{\mathcal N}$, over a set
of names, $\mathcal N$, is the smallest relation satisfying the rules
below.

\infrule[Out-barb]{y \in {\mathcal N}, \; x \nameeq y}
		  {\outputp{x}{v} \downarrow_{\mathcal N} x}
\infrule[Par-barb]{\mbox{$P\downarrow_{\mathcal N} x$ or $Q\downarrow_{\mathcal N} x$}}
		  {\binpar{P}{Q} \downarrow_{\mathcal N} x}

We write $P \Downarrow_{\mathcal N} x$ if there is $Q$ such that 
$P \wred Q$ and $Q \downarrow_{\mathcal N} x$.
\end{definition}

\begin{definition}
%\label{def.bbisim}
An  ${\mathcal N}$-\emph{barbed bisimulation} over a set of names, ${\mathcal N}$, is a symmetric binary relation 
${\mathcal S}_{\mathcal N}$ between agents such that $P\rel{S}_{\mathcal N}Q$ implies:
\begin{enumerate}
\item If $P \red P'$ then $Q \wred Q'$ and $P'\rel{S}_{\mathcal N} Q'$.
\item If $P\downarrow_{\mathcal N} x$, then $Q\Downarrow_{\mathcal N} x$.
\end{enumerate}
$P$ is ${\mathcal N}$-barbed bisimilar to $Q$, written
$P \wbbisim_{\mathcal N} Q$, if $P \rel{S}_{\mathcal N} Q$ for some ${\mathcal N}$-barbed bisimulation ${\mathcal S}_{\mathcal N}$.
\end{definition}

$\mathcal{R} \subseteq \pi \times \pi$

$P \mathcal{R} Q => \forall P'. P \red P' \Rightarrow \exists Q'. Q \red Q', P' \mathcal{R} Q'$

$P \vdash x \Rightarrow Q \vdash x$

\begin{mathpar}
  \inferrule*[lab=Out-barb]{x \nameeq y}{{y}!\langle{Q}\rangle \vdash x}
  \and
  \inferrule*[lab=Par-barb]{\mbox{$P\vdash x$ or $Q\vdash x$}}{\binpar{P}{Q} \vdash x}
\end{mathpar}

\subsubsection{Contexts}

One of the principle advantages of computational calculi like the
$\pi$-calculus is a well-defined notion of context,
contextual-equivalence and a correlation between
contextual-equivalence and notions of bisimulation. The notion of
context allows the decomposition of a process into (sub-)process and
its syntactic environment, its context. Thus, a context may be
thought of as a process with a ``hole'' (written $\Box$) in it. The
application of a context $M$ to a process $P$, written $M[P]$, is
tantamount to filling the hole in $M$ with $P$. In this paper we do
not need the full weight of this theory, but do make use of the notion
of context in the proof the main theorem. 

\begin{mathpar}
  \inferrule* [lab=summation] {} {{M_{M},M_{N}} \bc \Box \;|\; x.M_{A} \;|\; M_{M}+M_{N}}
  \and
  \inferrule* [lab=agent] {} {{M_{A}} \bc (\vec{x})M_{P} \;| \; \clift{P_0,\ldots,M_{P},\ldots,P_N}}
  \and \\
  \inferrule* [lab=process] {} {{M_{P}} \bc M_{N} \;| \;P|M_{P} }
\end{mathpar} 

\begin{mathpar}
  \inferrule* [lab=sychronization] {} {M_{N} \bc \Box \;|\; x?M_{F} \;|\; x!M_{C}}
  \and
  \inferrule* [lab=abstraction] {} {{M_{F}} \bc (x)M_{P} }
  \and
  \inferrule* [lab=concretion] {} {{M_{C}} \bc \langle M_{P} \rangle }
  \and \\
  \inferrule* [lab=process] {} {{M_{P}} \bc M_{N} \;| \;P|M_{P} }
\end{mathpar}

\begin{definition}[contextual application] Given a context $M$, and
  process $P$, we define the \emph{contextual application}, $M[P] :=
  M\{P/\Box\}$. That is, the contextual application of M to P is the
  substitution of $P$ for $\Box$ in $M$.
\end{definition}

$\meaningof{-} : L \to \mathcal{P}(\pi)$

\begin{mathpar}
  \inferrule* [lab=collection] {} {\meaningof{true} = \pi, \and \meaningof{~E} = \pi \setminus \meaningof{E}, \and \meaningof{E_{1} \& E_{2}} = \meaningof{E_{1}} \cap \meaningof{E_{2}}}
\end{mathpar}

\begin{mathpar}
  \inferrule* [lab=structure] {} {\meaningof{0} = \{ P \in \pi | P \equiv 0 \}, \and \\ \meaningof{E_1 | E_2} = \{ P \in \pi | P \equiv P_{1} | P_{2}, P_{1} \in \meaningof{E_{1}}, P_{2} \in \meaningof{E_2}\} }
\end{mathpar}

\begin{mathpar}
 \inferrule* [lab=behavior] {} {\meaningof{\langle a?b \rangle E} = \{ P \in \pi | P \equiv Q | u?(y)P', \\ \and \\\\ \and \\ \;\;\; u \in \meaningof{a}, \forall z.P'\{z/y\} \in \meaningof{E\{z/b\}}\}, \and \\ \meaningof{a!E} = \{ P \in \pi | P \equiv Q | x!\langle P' \rangle, x \in \meaningof{a} P' \in \meaningof{E}\} }
\end{mathpar}

\begin{mathpar}
 \inferrule* [lab=nominal] {} {\meaningof{\quotep{E}} = \{ \quotep{P} \in \quotep{\pi} | P \in \meaningof{E} \}, \and \meaningof{\quotep{P}} = \{ \quotep{Q} \in \quotep{\pi} | P \equiv Q \} \and \\ \meaningof{@\quotep{E}} = \{ P \in \pi | P \equiv @x, x \in \meaningof{E} \}}
\end{mathpar}

\begin{eqnarray*}
  \\
  \meaningof{-} : TS \to ST
\end{eqnarray*}

\begin{eqnarray*}
  \\
  L : TS \to ST
\end{eqnarray*}

\begin{eqnarray*}
  \\
  P \models E \iff P \in \meaningof{E}
\end{eqnarray*}

\begin{eqnarray*}
  P \approx_{L} Q \iff \forall E \in L. P \models E \iff Q \models E
\end{eqnarray*}

\begin{eqnarray*}
  P \approx_{K} Q
\end{eqnarray*}

\begin{eqnarray*}
  P \approx Q
\end{eqnarray*}

$\approx_{K} = \approx = \approx_{L}$

\subsubsection{Contextual duality}

Note that contexts extend the quotation operation to a family of
operations from processes to names. Given a context, $M$, we can
define a \emph{nominal context}, $\quotep{M}$ by $\quotep{M}[P] :=
\quotep{M[P]}$. To foreshadow what is to come we observe that these
operations enjoy a duality with processes very much like the duality
between vectors and maps from vectors to scalars.

Further, because the calculus is essentially higher-order, we have a
correspondence between contexts and processes. More specifically,
given a name $x$ and a context $M$ we can construct $M^{*}_{x}$ such
that 

\begin{mathpar}
  M^{*}_{x} | \lift{x}{P} \red M[P]
\end{mathpar}

namely,

\begin{mathpar}
  M^{*}_{x} := x?(u).M[\dropn{u}]
\end{mathpar}

The dependence of $M^{*}_{x}$ on a name makes it an abstraction, 

\begin{mathpar}
  M^{*} := (x)x?(u).M[\dropn{u}]
\end{mathpar}

\subsection{Additional notation}

It will sometimes be convenient to denote the process a name
quotes. We already have the notation $x = \quotep{P}$, but it will be
convenient to introduce an alternate notation, $\procn{x}$, when we
want to emphasize the connection to the use of the name. Note that, by
virtue of name equivalence, $\quotep{\procn{x}} \nameeq x$; so, the
notation is consistent with previous definitions.

Further, because names have structure it is possible to effect
substitutions on the basis of that structure. This means we need to
upgrade our notation for substitutions, which we accomplish by
adapting comprehension notation. Thus,

\begin{mathpar}
  P\{ y / x : x \in S \}
\end{mathpar}

is interpreted to mean the process derived from P by replacing (in a
capture-avoiding manner) each occurrence of $x$ in $S$ by $y$. For example,

\begin{mathpar}
  P\{ \quotep{\procn{x}|\procn{x}} / x : x \in \freenames{P} \}
\end{mathpar}

will replace each (occurrence) of a free name $x$ in $P$ by
$\quotep{\procn{x}|\procn{x}}$.

Also, we will avail ourselves of the notation $x^{L}$ and $x^{R}$ to
denote injections of a name into disjoint copies of the name
space. There are numerous ways to accomplish this. One example can be
found in \cite{MeredithR05}. This notation overloads to vectors of
names: $\vec{x}^{\pi} := (x_{i}^{\pi} \; : \; 0 \leq i < |\vec{x}| )$ where $\pi \in \{L,R\}$.

We also use $P^{\Box} := P|\Box$.

In \cite{MeredithR05} an interpretation of the new operator is
given. It turns out that there are several possible interpretations
all enjoying the requisite algebraic properties of the operator (see
\cite{milner91polyadicpi}). We will therefore make liberal use of
$(\nu\; \vec{x})P$.

% subsection the_syntax_and_semantics_of_the_notation_system (end)   

\input{qm2pi.qmops} 

\input{qm2pi.sterngerlach} 

\input{qm2pi.metric} 

% section concurrent_process_calculi (end)

%\input{qm2pi.proofsketch}

% section proof sketch (end)

%\input{qm2pi.slviaknots} 

% section spatial logic via knots (end)

\input{qm2pi.conclusion}

% section conclusion (end)

%\input{qm2pi.dtcodes} 

% section wiring algorithm (end)

\input{qm2pi.ack} 

% section acknowledgments (end)

\newpage


\bibliographystyle{plain}   
\bibliography{../../biblios/main.bib}

\input{qm2pi.rhodetails}

\end{document}

 

%\ifpdf
%\usepackage[pdftex]{graphicx}
%\else
%\usepackage{graphicx}
%\fi

 % \ifpdf
%  \usepackage{pdfsync}
%  \if


%\title{Brief Article}
%\author{David F. Snyder}
%\author{L.G. Meredith}

%\address{Dept. of Math., Texas State University--San Marcos, San Marcos, TX 78666}
       
\pagestyle{empty}


\begin{document}

\lstset{language=[Objective]Caml,frame=shadowbox}

\documentclass[12pt]{llncs}
%\documentclass{jktr}

\usepackage[pdftex]{hyperref}                   
\usepackage {listings}
\usepackage {mathpartir}
\usepackage{bcprules}
%\usepackage{listings}
                       
\usepackage{graphicx} 
%\usepackage[margins=2.5cm,nohead,nofoot]{geometry}
%\usepackage{geometry}
\usepackage{amsfonts}
\usepackage{amstext}
\usepackage{latexsym}
\usepackage{amssymb}
\usepackage{color}


%\include{myPreamble}
\include{qm2pi.local} 

%\ifpdf
%\usepackage[pdftex]{graphicx}
%\else
%\usepackage{graphicx}
%\fi

 % \ifpdf
%  \usepackage{pdfsync}
%  \if


%\title{Brief Article}
%\author{David F. Snyder}
%\author{L.G. Meredith}

%\address{Dept. of Math., Texas State University--San Marcos, San Marcos, TX 78666}
       
\pagestyle{empty}


\begin{document}

\lstset{language=[Objective]Caml,frame=shadowbox}

\input{qm2pi.front}

% section front matter (end)

\input{qm2pi.intro} 
 
% section introduction (end)

% \input{qm2pi.knotations} 

% section notation (end)

\input{qm2pi.process.calculi} 

% section concurrent_process_calculi_and_spatial_logics_ (end)
    
%\input{qm2pi.knots2pi} 

%\input{qm2pi.trefoil} 

%\input{qm2pi.mainthm} 

% subsection basic_interpretation (end)

%\input{qm2pi.rho.presentation} 
\subsection{The syntax and semantics of the notation system}\label{sub:the_syntax_and_semantics_of_the_notation_system} % (fold)

We now summarize a technical presentation of the calculus that
embodies our theory of dynamics. The typical presentation of such a
calculus follows the style of giving generators and relations on
them. The grammar, below, describing term constructors, freely
generates the set of processes, $\Proc$. This set is then quotiented
by a relation known as structural congruence and it is over this set
that the notion of dynamics is expressed. This presentation is
essentially that of \cite{MeredithR05} with the addition of
polyadicity and summation. For readability we have relegated some of
the technical subtleties to an appendix.

\subsubsection{Process grammar}\label{subsub:process_grammar}

\begin{mathpar}
  \inferrule* [lab=synchronization] {} {{M} \bc \pzero \;|\; x?F \;|\; x!C }
  \and
  \inferrule* [lab=abstraction] {} {{F} \bc (x)P}
  \and
  \inferrule* [lab=concretion] {} {{C} \bc \langle Q \rangle}
  \and
  \inferrule* [lab=process] {} {{P,Q} \bc M \;| \;P|Q \;|\; @{x}}
  \and
  \inferrule* [lab=name] {} {{x} \bc \quotep{P}}
\end{mathpar} 

Note that $\vec{x}$ (resp. $\vec{P}$) denotes a vector of names
(resp. processes) of length $|\vec{x}|$ (resp. $|\vec{P}|$). We adopt
the following useful abbreviations.

\begin{mathpar}
   x?(\vec{y}).P := x.(\vec{y})P \and  x\clift{\vec{P}} := x.\clift{\vec{P}}
   \and x!(y) := \lift{x}{\dropn{y}}
   \and \Pi_{i=0}^{n-1}P_i := P_0 | \ldots | P_{n-1}
\end{mathpar}

\subsubsection{Structural congruence}

\paragraph{Free and bound names and alpha-equivalence.} At the
core of structural equivalence is alpha-equivalence which identifies
process that are the same up to a change of variable. Formally, we
recognize the distinction between free and bound names. The free names
of a process, $\freenames{P}$, may be calculated recursively as
follows:

\begin{mathpar}
\freenames{\pzero} := \emptyset
  \and \\
  \freenames{x?(y).P} := \{ x \} \cup (\freenames{P} \setminus \{ y \})
  \and 
  \freenames{x!\langle P \rangle} := \{ x \} \cup \{ P \} 
  \and \\
  \freenames{P|Q} := \freenames{P} \cup \freenames{Q}
  \and \\
  \freenames{@{x}} := \{ x \}
\end{mathpar}

$\pi$
$\quotep{\pi}$

$\freenames{-} : \pi \to \mathcal{P}(\quotep{\pi})$

\begin{eqnarray*}
  \freenames{\pzero} & := & \emptyset \\
  \freenames{x?(y).P} & := & \{ x \} \cup (\freenames{P} \setminus \{ y \}) \\
  \freenames{x!\langle P \rangle} & := & \{ x \} \cup \{ P \} \\
  \freenames{P|Q} & := & \freenames{P} \cup \freenames{Q} \\
  \freenames{\dropn{x}} & := & \{ x \}
\end{eqnarray*}

The bound names of a process, $\boundnames{P}$, are those names occurring in $P$
that are not free. For example, in $x?(y).0$, the name $x$ is free, while $y$ is bound.

\begin{mathpar}
  \inferrule* [lab=monoidal-laws] {} { P|Q \equiv Q|P \and P|0 \equiv P \and P|(Q|R) \equiv (P|Q)|R }
\end{mathpar}

\begin{mathpar}
  \inferrule* [lab=alpha-equivalence] {} { (x)P \equiv (y)P\{y/x\} \and y \not\in \freenames{P} }
\end{mathpar}

\begin{definition}
Then two processes, $P,Q$, are alpha-equivalent if $P = Q\{\vec{y}/\vec{x}\}$ for
some $\vec{x} \in \boundnames{Q},\vec{y} \in \boundnames{P}$, where $Q\{\vec{y}/\vec{x}\}$
denotes the capture-avoiding substitution of $\vec{y}$ for $\vec{x}$ in $Q$.
\end{definition}

\begin{definition}
  The {\em structural congruence} \cite{SangiorgiWalker} , $\equiv$,
  between processes is the least congruence containing
  alpha-equivalence, satisfying the abelian monoid laws
  (associativity, commutativity and $\pzero$ as identity) for parallel
  composition $|$ and for summation $+$.
\end{definition}

\subsection{Name equivalence}

We take name equivalence, written $\nameeq$, to be the smallest
equivalence relation generated by the following rules.

\begin{mathpar}
\inferrule*[lab=Quote-drop]
{ }
{ \quotep{@{x}} \nameeq x }

\inferrule*[lab=Struct-equiv]
{ P \scong Q }
{ \quotep{P} \nameeq \quotep{Q} }
\end{mathpar}

The astute reader will have noticed that the mutual recursion of names
and processes imposes a mutual recursion on alpha-equivalence and
structural equivalence via name-equivalence. Fortunately, all of this
works out pleasantly and we may calculate in the natural way, free of
concern. The reader interested in the details is referred to the
appendix \ref{appendix:rho_details}.

\subsection{Substitution}

We use $\Proc$ for the set of processes, $\QProc$ for the set of
names, and $\id{\{}\vec{y} / \vec{x} \id{\}}$ to denote partial maps,
$s : \QProc \rightarrow \QProc$. A map, $s$ lifts, uniquely, to a map
on process terms, $\widehat{s} : \Proc \rightarrow \Proc$ by the
following equations.

\begin{mathpar}
  (0) \psubstp{Q}{P} := 0 \\
  (R \juxtap S) \psubstp{Q}{P}
  :=    
  (R)\psubstp{Q}{P} \juxtap (S) \psubstp{Q}{P} \\
  (x?(y).R) \psubstp{Q}{P}    
  :=    
  (x)\substp{Q}{P} (z)\concat( (R \psubstn{z}{y}) \psubstp{Q}{P} ) \\
  (\lift{x}{R}) \psubstp{Q}{P}  
  :=
  \lift{(x)\substp{Q}{P}}{ R \psubstp{Q}{P} } \\
%   (\dropn{x})  \psubstp{Q}{P}       
%   := 
%   \left\{ 
%     \begin{array}{ccc} 
%       \dropn{\quotep{Q}} & & x \nameeq \quotep{P} \\
%       \dropn{x} & & otherwise \\
%     \end{array}
%   \right. 
  (\dropn{x})  \psubstp{Q}{P}       
  := 
  \left\{ 
    \begin{array}{ccc} 
      Q & & x \nameeq \quotep{P} \\
      \dropn{x} & & otherwise \\
    \end{array}
  \right.
\end{mathpar}
 

where

\begin{eqnarray}
  (x)\id{\{} \lpquote Q \rpquote / \lpquote P \rpquote \id{\}}            = 
  \left\{ 
    \begin{array}{ccc}
      \lpquote Q \rpquote & & x \nameeq \lpquote P \rpquote \\
      x & & otherwise \\
    \end{array}
  \right. \nonumber
\end{eqnarray}

and $z$ is chosen distinct from $\quotep{P}$, $\quotep{Q}$, the free
names in $Q$, and all the names in $R$. Our $\alpha$-equivalence will
be built in the standard way from this substitution.

\begin{remark}\label{rem:no_self_referential_names}
  One consequence of these definitions is that $\forall P. \quotep{P}
  \not\in \freenames{P}$.
\end{remark}

\subsection{ Dynamic quote: an example }

Anticipating something of what's to come, consider applying the
substitution, $\widehat{\id{\{}u / z \id{\}}}$, to the following pair
of processes, $\lift{w}{y!(z)}$ and $w[ \lpquote y!(z) \rpquote ]$.

\begin{eqnarray}
	\lift{w}{y!(z)}\widehat{\id{\{}u / z \id{\}}}
		& = &
		\lift{w}{y!(u)} \nonumber\\
	w[ \lpquote y!(z) \rpquote ] \widehat{ \id{\{}u / z \id{\}} }
		& = &
		w[ \lpquote y!(z) \rpquote ] \nonumber
\end{eqnarray}

Because the body of the process between quotes is impervious to
substitution, we get radically different answers. In fact, by
examining the first process in an input context,
e.g. $x?(z).\lift{w}{y!(z)}$, we see that the process under the lift
operator may be shaped by prefixed inputs binding a name inside it. In
this sense, the lift operator will be seen as a way to dynamically
construct processes before reifying them as names.

Finally equipped with these standard features we can present the
dynamics of the calculus.

\subsubsection{Operational semantics} 

Finally, we introduce the computational dynamics. What marks these
algebras as distinct from other more traditionally studied algebraic
structures, e.g. vector spaces or polynomial rings, is the manner in
which dynamics is captured. In traditional structures, dynamics is typically
expressed through morphisms between such structures, as in linear maps
between vector spaces or morphisms between rings. In algebras
associated with the semantics of computation, the dynamics is
expressed as part of the algebraic structure itself, through a
reduction reduction relation typically denoted by $\red$. Below, we
give a recursive presentation of this relation for the calculus used
in the encoding.

$\red \subseteq \pi \times \pi$
$\red : \pi \to \mathcal{P}(\pi)$

\begin{mathpar}
  \inferrule* [lab=Comm] { \textsf{match}( x_{src}, x_{trgt} ) } { x_{trgt}?(y)P \; | \; x_{src}!\langle {Q} \rangle \red P\{\quotep{Q}/y}\} }
  \and \\
  \inferrule* [lab=Par] {{P} \red {P}'} {{{P} | {Q}} \red {{P}' | {Q}}}
  \and
  \inferrule* [lab=Equiv]{{{P} \scong {P}'} \andalso {{P}' \red {Q}'} \andalso {{Q}' \scong {Q}}}{{P} \red {Q}}
\end{mathpar}

\begin{eqnarray*}
  match_{\equiv} (\quotep{P},\quotep{Q}) & := & P \equiv Q \\
  match_{\dagger}(\quotep{P},\quotep{Q}) & := & \forall R. P|Q \red^{*} R => R \red^{*} 0 \\
  match_{K}(\quotep{P},\quotep{Q}) & := & K \mbox{ for some context } K
\end{eqnarray*}

$u?(x)P | u!\langle Q \rangle \red P\{\quotep{Q}/x\}$

%We write $\wred$ for $\red^*$, and $P\red$ if $\exists Q $ such that $ P \red Q$.
We write $P\red$ if $\exists Q $ such that $ P \red Q$ and $P\not\red$, otherwise.

\section{Replication}

As mentioned before, it is known that replication (and hence
recursion) can be implemented in a higher-order process algebra
\cite{SangiorgiWalker}. As our first example of calculation with the
machinery thus far presented we give the construction explicitly in
the {\rhoc}.

\begin{eqnarray}
	D_{x} & := & \prefix{x}{y}{(\binpar{\outputp{x}{y}}{@{y}})} \nonumber\\
	\bangp_{x}{P} & := & \binpar{{x}!\langle{\binpar{D_{x}}{P}}\rangle}{D_{x}} \nonumber
\end{eqnarray}

\begin{eqnarray}
	\bangp_{x}{P} & & \nonumber\\
	=
	& {x}!\langle{(\prefix{x}{y}{(\outputp{x}{y} | @{y})) | P}}\rangle 
	      | \prefix{x}{y}{(\outputp{x}{y} | @{y})} & \nonumber\\
	\red
	& (\outputp{x}{y} | @{y})\substn{\quotep{(\prefix{x}{y}{(@{y} | \outputp{x}{y})) | P}}}{y} & \nonumber\\
	=
	& \outputp{x}{\quotep{(\prefix{x}{y}{(\outputp{x}{y} | @{y})) | P}}}
	  | {(\prefix{x}{y}{(\outputp{x}{y} | @{y})) | P}} & \nonumber\\
	\red
	& \ldots & \nonumber\\
	\red^*
	& P | P | \ldots & \nonumber
\end{eqnarray}

Of course, this encoding, as an implementation, runs away, unfolding
$\bangp{P}$ eagerly. A lazier and more implementable replication
operator, restricted to input-guarded processes, may be obtained as follows.

\begin{eqnarray}
\bangp{\prefix{u}{v}{P}} 
	:= 
	\binpar{\lift{x}{\prefix{u}{v}{(\binpar{D(x)}{P})}}}{D(x)} \nonumber
\end{eqnarray}

\begin{remark}
  Note that the lazier definition still does not deal with summation
  or mixed summation (i.e. sums over input and output). The reader is
  invited to construct definitions of replication that deal with these
  features. 

  Further, the definitions are parameterized in a name, $x$. Can you,
  gentle reader, make a definition that eliminates this parameter and
  guarantees no accidental interaction between the replication
  machinery and the process being replicated -- i.e. no accidental
  sharing of names used by the process to get its work done and the
  name(s) used by the replication to effect copying. This latter
  revision of the definition of replication is crucial to obtaining
  the expected identity $!!P \sim !P$.
\end{remark}

\begin{remark}\label{rem:paradoxical_combinator}
  The reader familiar with the lambda calculus will have noticed the
  similarity between $D$ and the paradoxical combinator.

  [Ed. note: the existence of this seems to suggest we have to be more
  restrictive on the set of processes and names we admit if we are to
  support no-cloning.]
\end{remark}

\subsubsection{Bisimulation}

The computational dynamics gives rise to another kind of equivalence,
the equivalence of computational behavior. As previously mentioned
this is typically captured \emph{via} some form of bisimulation.

% The notion we use in this paper is weak barbed bisimulation
% \cite{milner91polyadicpi}.

The notion we use in this paper is derived from weak barbed
bisimulation \cite{milner91polyadicpi}. 

\begin{definition}
An \emph{observation relation}, $\downarrow_{\mathcal N}$, over a set
of names, $\mathcal N$, is the smallest relation satisfying the rules
below.

\infrule[Out-barb]{y \in {\mathcal N}, \; x \nameeq y}
		  {\outputp{x}{v} \downarrow_{\mathcal N} x}
\infrule[Par-barb]{\mbox{$P\downarrow_{\mathcal N} x$ or $Q\downarrow_{\mathcal N} x$}}
		  {\binpar{P}{Q} \downarrow_{\mathcal N} x}

We write $P \Downarrow_{\mathcal N} x$ if there is $Q$ such that 
$P \wred Q$ and $Q \downarrow_{\mathcal N} x$.
\end{definition}

\begin{definition}
%\label{def.bbisim}
An  ${\mathcal N}$-\emph{barbed bisimulation} over a set of names, ${\mathcal N}$, is a symmetric binary relation 
${\mathcal S}_{\mathcal N}$ between agents such that $P\rel{S}_{\mathcal N}Q$ implies:
\begin{enumerate}
\item If $P \red P'$ then $Q \wred Q'$ and $P'\rel{S}_{\mathcal N} Q'$.
\item If $P\downarrow_{\mathcal N} x$, then $Q\Downarrow_{\mathcal N} x$.
\end{enumerate}
$P$ is ${\mathcal N}$-barbed bisimilar to $Q$, written
$P \wbbisim_{\mathcal N} Q$, if $P \rel{S}_{\mathcal N} Q$ for some ${\mathcal N}$-barbed bisimulation ${\mathcal S}_{\mathcal N}$.
\end{definition}

$\mathcal{R} \subseteq \pi \times \pi$

$P \mathcal{R} Q => \forall P'. P \red P' \Rightarrow \exists Q'. Q \red Q', P' \mathcal{R} Q'$

$P \vdash x \Rightarrow Q \vdash x$

\begin{mathpar}
  \inferrule*[lab=Out-barb]{x \nameeq y}{{y}!\langle{Q}\rangle \vdash x}
  \and
  \inferrule*[lab=Par-barb]{\mbox{$P\vdash x$ or $Q\vdash x$}}{\binpar{P}{Q} \vdash x}
\end{mathpar}

\subsubsection{Contexts}

One of the principle advantages of computational calculi like the
$\pi$-calculus is a well-defined notion of context,
contextual-equivalence and a correlation between
contextual-equivalence and notions of bisimulation. The notion of
context allows the decomposition of a process into (sub-)process and
its syntactic environment, its context. Thus, a context may be
thought of as a process with a ``hole'' (written $\Box$) in it. The
application of a context $M$ to a process $P$, written $M[P]$, is
tantamount to filling the hole in $M$ with $P$. In this paper we do
not need the full weight of this theory, but do make use of the notion
of context in the proof the main theorem. 

\begin{mathpar}
  \inferrule* [lab=summation] {} {{M_{M},M_{N}} \bc \Box \;|\; x.M_{A} \;|\; M_{M}+M_{N}}
  \and
  \inferrule* [lab=agent] {} {{M_{A}} \bc (\vec{x})M_{P} \;| \; \clift{P_0,\ldots,M_{P},\ldots,P_N}}
  \and \\
  \inferrule* [lab=process] {} {{M_{P}} \bc M_{N} \;| \;P|M_{P} }
\end{mathpar} 

\begin{mathpar}
  \inferrule* [lab=sychronization] {} {M_{N} \bc \Box \;|\; x?M_{F} \;|\; x!M_{C}}
  \and
  \inferrule* [lab=abstraction] {} {{M_{F}} \bc (x)M_{P} }
  \and
  \inferrule* [lab=concretion] {} {{M_{C}} \bc \langle M_{P} \rangle }
  \and \\
  \inferrule* [lab=process] {} {{M_{P}} \bc M_{N} \;| \;P|M_{P} }
\end{mathpar}

\begin{definition}[contextual application] Given a context $M$, and
  process $P$, we define the \emph{contextual application}, $M[P] :=
  M\{P/\Box\}$. That is, the contextual application of M to P is the
  substitution of $P$ for $\Box$ in $M$.
\end{definition}

$\meaningof{-} : L \to \mathcal{P}(\pi)$

\begin{mathpar}
  \inferrule* [lab=collection] {} {\meaningof{true} = \pi, \and \meaningof{~E} = \pi \setminus \meaningof{E}, \and \meaningof{E_{1} \& E_{2}} = \meaningof{E_{1}} \cap \meaningof{E_{2}}}
\end{mathpar}

\begin{mathpar}
  \inferrule* [lab=structure] {} {\meaningof{0} = \{ P \in \pi | P \equiv 0 \}, \and \\ \meaningof{E_1 | E_2} = \{ P \in \pi | P \equiv P_{1} | P_{2}, P_{1} \in \meaningof{E_{1}}, P_{2} \in \meaningof{E_2}\} }
\end{mathpar}

\begin{mathpar}
 \inferrule* [lab=behavior] {} {\meaningof{\langle a?b \rangle E} = \{ P \in \pi | P \equiv Q | u?(y)P', \\ \and \\\\ \and \\ \;\;\; u \in \meaningof{a}, \forall z.P'\{z/y\} \in \meaningof{E\{z/b\}}\}, \and \\ \meaningof{a!E} = \{ P \in \pi | P \equiv Q | x!\langle P' \rangle, x \in \meaningof{a} P' \in \meaningof{E}\} }
\end{mathpar}

\begin{mathpar}
 \inferrule* [lab=nominal] {} {\meaningof{\quotep{E}} = \{ \quotep{P} \in \quotep{\pi} | P \in \meaningof{E} \}, \and \meaningof{\quotep{P}} = \{ \quotep{Q} \in \quotep{\pi} | P \equiv Q \} \and \\ \meaningof{@\quotep{E}} = \{ P \in \pi | P \equiv @x, x \in \meaningof{E} \}}
\end{mathpar}

\begin{eqnarray*}
  \\
  \meaningof{-} : TS \to ST
\end{eqnarray*}

\begin{eqnarray*}
  \\
  L : TS \to ST
\end{eqnarray*}

\begin{eqnarray*}
  \\
  P \models E \iff P \in \meaningof{E}
\end{eqnarray*}

\begin{eqnarray*}
  P \approx_{L} Q \iff \forall E \in L. P \models E \iff Q \models E
\end{eqnarray*}

\begin{eqnarray*}
  P \approx_{K} Q
\end{eqnarray*}

\begin{eqnarray*}
  P \approx Q
\end{eqnarray*}

$\approx_{K} = \approx = \approx_{L}$

\subsubsection{Contextual duality}

Note that contexts extend the quotation operation to a family of
operations from processes to names. Given a context, $M$, we can
define a \emph{nominal context}, $\quotep{M}$ by $\quotep{M}[P] :=
\quotep{M[P]}$. To foreshadow what is to come we observe that these
operations enjoy a duality with processes very much like the duality
between vectors and maps from vectors to scalars.

Further, because the calculus is essentially higher-order, we have a
correspondence between contexts and processes. More specifically,
given a name $x$ and a context $M$ we can construct $M^{*}_{x}$ such
that 

\begin{mathpar}
  M^{*}_{x} | \lift{x}{P} \red M[P]
\end{mathpar}

namely,

\begin{mathpar}
  M^{*}_{x} := x?(u).M[\dropn{u}]
\end{mathpar}

The dependence of $M^{*}_{x}$ on a name makes it an abstraction, 

\begin{mathpar}
  M^{*} := (x)x?(u).M[\dropn{u}]
\end{mathpar}

\subsection{Additional notation}

It will sometimes be convenient to denote the process a name
quotes. We already have the notation $x = \quotep{P}$, but it will be
convenient to introduce an alternate notation, $\procn{x}$, when we
want to emphasize the connection to the use of the name. Note that, by
virtue of name equivalence, $\quotep{\procn{x}} \nameeq x$; so, the
notation is consistent with previous definitions.

Further, because names have structure it is possible to effect
substitutions on the basis of that structure. This means we need to
upgrade our notation for substitutions, which we accomplish by
adapting comprehension notation. Thus,

\begin{mathpar}
  P\{ y / x : x \in S \}
\end{mathpar}

is interpreted to mean the process derived from P by replacing (in a
capture-avoiding manner) each occurrence of $x$ in $S$ by $y$. For example,

\begin{mathpar}
  P\{ \quotep{\procn{x}|\procn{x}} / x : x \in \freenames{P} \}
\end{mathpar}

will replace each (occurrence) of a free name $x$ in $P$ by
$\quotep{\procn{x}|\procn{x}}$.

Also, we will avail ourselves of the notation $x^{L}$ and $x^{R}$ to
denote injections of a name into disjoint copies of the name
space. There are numerous ways to accomplish this. One example can be
found in \cite{MeredithR05}. This notation overloads to vectors of
names: $\vec{x}^{\pi} := (x_{i}^{\pi} \; : \; 0 \leq i < |\vec{x}| )$ where $\pi \in \{L,R\}$.

We also use $P^{\Box} := P|\Box$.

In \cite{MeredithR05} an interpretation of the new operator is
given. It turns out that there are several possible interpretations
all enjoying the requisite algebraic properties of the operator (see
\cite{milner91polyadicpi}). We will therefore make liberal use of
$(\nu\; \vec{x})P$.

% subsection the_syntax_and_semantics_of_the_notation_system (end)   

\input{qm2pi.qmops} 

\input{qm2pi.sterngerlach} 

\input{qm2pi.metric} 

% section concurrent_process_calculi (end)

%\input{qm2pi.proofsketch}

% section proof sketch (end)

%\input{qm2pi.slviaknots} 

% section spatial logic via knots (end)

\input{qm2pi.conclusion}

% section conclusion (end)

%\input{qm2pi.dtcodes} 

% section wiring algorithm (end)

\input{qm2pi.ack} 

% section acknowledgments (end)

\newpage


\bibliographystyle{plain}   
\bibliography{../../biblios/main.bib}

\input{qm2pi.rhodetails}

\end{document}



% section front matter (end)

\section{Introduction}\label{sec:introduction} % (fold)
In this draft of the material i am going to have to dispense with the
usual writing conventions adopted in papers on these topics. i'm going
to have adopt whatever tone i need at the time i'm writing up the
calculations. Sometimes this may be very conversational; others it may
be the barest mathematical grunts; others still it may be that i have
lifted text from one of my other papers because the exposition of some
point was better said there. i hope that my readers are not unduly put
out by this decision. i'm not doing this to flout convention or be
rebellious. i find these calculations very technically challenging. To
keep everything going technically, something has to give; i have to
let go of some cognitive burden. So, the academic writing style --
with all of its trade-offs in terms of facilitating technical
communication -- is what i'm letting go of. Perhaps subsequent drafts
can be tightened and polished, but for now, i'm going to speak as if
we were sitting together in a coffee shop with a laptop, wifi and a
pad of paper and a pencil.

So, here's what i have to say. We -- you and i, comfortably ensconced
in our coffee shop and well-equipped with our tools -- can realize and
carry out the calculations of quantum mechanics over a very different
formal theory of dynamics, a formal theory of dynamics that
corresponds to a theory of concurrent computation with
\emph{reflection}. It has the advantage that the underlying theory is
already `quantized', but supports analogues all of the continuuous
operations. Strikingly, this underlying theory has recently been
connected with a notion of metric that we can show, by calculating
together, coincides with the metric induced by the inner product.

There are a lot of reasons why you might be interested in seeing
calculations of this form. Here's why i'm interested. For the past
several centuries there has been no competitor to the ``Newtonian''
account of dynamics. As a result the predominant share of accounts of
dynamical systems and situations have had to be formulated in terms of
the Newtonian machinery. i view this as an intellectually dangerous
position to occupy. Everything, despite it's intrinsic shape, turns
into a nail to be hit with this hammer. Recently, however, the theory
of computation has matured to the point where we have candidates for
theories of dynamics that offer very different perspective on
reasoning about dynamical systems and situations. Testing these
candidates against very successful accounts of dynamical situations,
like quantum mechanics, is going to give us some sense of how mature
they are and some measure of the quality of these accounts of
dynamics.

\subsection{Summary of contributions and outline of paper}

So, we're going to develop an interpretation of the operations of
quantum mechanics normally interpreted by Hilbert spaces and
operators. We're going to do this over a theory of computation. Note
that this is very different than the usual quantum computation program
which develops notions of computation over quantum mechanics. Rather,
we are developing a story that aligns with Wheeler's slogan: It from
Bit. To do this we will first provide an account of the theory of
computation at play here. Then we will dive into a calculation-driven
interpretation of the operations of quantum mechanics.

The reason we take this approach is that -- until very recently --
there hasn't been an axiomatic account of quantum mechanics. As a
result there has been no sharp delineation of the mathematical theory
supporting interpretation of the physical theory and the physical
theory, itself. So, ambient features of the maths are free to be
exploited (or supressed) without a real accounting of their physical
relevance. There is no sharp statement ``here's the physical theory''
qua \emph{theory} and ``here's the mathematical interpretation''
enabling a judgment of how faithful the interpretation is -- apart
from experimental observation. When there is an axiomatic account we
can judge how well a given mathematical formalism supports an
interpretation of the axioms, independent of
experimentation. Likewise, we can judge how well we have captured our
physical evidence and experience with our axiomatics, independent of
any specific mathematical implementation, with accidental detail that
may or may not have physical significance. 

In lieu of a fully fleshed out and vetted axiomatic account of quantum
mechanics, interpreting the operational notions in service of modeling
physical systems will have to suffice. In other words, we are not in
the business of providing a model of Hilbert spaces and operators. We
are in the business of providing a model of quantum mechanics because
we are motivated by testing our notions of dynamics against physical
theory; and, the predictive calculations of the physical theory must
serve as the best formulation -- shy of a fully fleshed out axiomatic
account -- of the physical theory itself (as they have for scientific
theories since time immemorial). Put another way, despite a
whole-hearted commitment to an It-from-Bit ontology, we are firmly
aligned with the shut-up-and-calculate camp as the best way to obtain
results either from the physical perspective or as a quality assurance
measure of our fledgling theory of dynamics.

In detail, we present a reflective process calculus. Then we develop
intuitive correspondences between the notions available in this
calculus and the usual physical notions supporting quantum mechanical
calculations. Thus, 

\begin{table}[htp]
  \center{
    \fbox{
      \begin{tabular}{c|c}
        quantum mechanics & process calculus \\
        \hline
        scalar & name \\
        state vector & process \\
        dual & contextual duals \\
        matrix & formal sums of process-context-dual pairs \\
        orthogonality & process annihilation \\
        inner product & execution-formula + quoting
      \end{tabular}
    }
  }
  \caption{QM - process calculi correspondences}
\end{table}

Then we tighten up these intuitions to operational definitions. We
employ the Dirac notation as the best proxy we can find for an
abstract syntax of the quantum mechanical notions. The definitions we
develop put us in contact with equational constraints coming from the
theory that we demonstrate the definitions and calculations satisfy.

This puts us in a position to shut up and calculate for the
Stern-Gerlach experimental set up, showing how these predictive
calculations become calculations on processes in our theory of a
reflective process calculus.

Penultimately, we demonstrate that the notion of metric coming from
the inner product coincides with the notion of metric available from
the theory of bisimulation. This demonstration gives us the right to
think of space as arising from behavior. Finally, we consider where we
might go from the new vantage point we have obtained.

% section introduction (end) 
 
% section introduction (end)

% \documentclass[12pt]{llncs}
%\documentclass{jktr}

\usepackage[pdftex]{hyperref}                   
\usepackage {listings}
\usepackage {mathpartir}
\usepackage{bcprules}
%\usepackage{listings}
                       
\usepackage{graphicx} 
%\usepackage[margins=2.5cm,nohead,nofoot]{geometry}
%\usepackage{geometry}
\usepackage{amsfonts}
\usepackage{amstext}
\usepackage{latexsym}
\usepackage{amssymb}
\usepackage{color}


%\include{myPreamble}
\include{qm2pi.local} 

%\ifpdf
%\usepackage[pdftex]{graphicx}
%\else
%\usepackage{graphicx}
%\fi

 % \ifpdf
%  \usepackage{pdfsync}
%  \if


%\title{Brief Article}
%\author{David F. Snyder}
%\author{L.G. Meredith}

%\address{Dept. of Math., Texas State University--San Marcos, San Marcos, TX 78666}
       
\pagestyle{empty}


\begin{document}

\lstset{language=[Objective]Caml,frame=shadowbox}

\input{qm2pi.front}

% section front matter (end)

\input{qm2pi.intro} 
 
% section introduction (end)

% \input{qm2pi.knotations} 

% section notation (end)

\input{qm2pi.process.calculi} 

% section concurrent_process_calculi_and_spatial_logics_ (end)
    
%\input{qm2pi.knots2pi} 

%\input{qm2pi.trefoil} 

%\input{qm2pi.mainthm} 

% subsection basic_interpretation (end)

%\input{qm2pi.rho.presentation} 
\subsection{The syntax and semantics of the notation system}\label{sub:the_syntax_and_semantics_of_the_notation_system} % (fold)

We now summarize a technical presentation of the calculus that
embodies our theory of dynamics. The typical presentation of such a
calculus follows the style of giving generators and relations on
them. The grammar, below, describing term constructors, freely
generates the set of processes, $\Proc$. This set is then quotiented
by a relation known as structural congruence and it is over this set
that the notion of dynamics is expressed. This presentation is
essentially that of \cite{MeredithR05} with the addition of
polyadicity and summation. For readability we have relegated some of
the technical subtleties to an appendix.

\subsubsection{Process grammar}\label{subsub:process_grammar}

\begin{mathpar}
  \inferrule* [lab=synchronization] {} {{M} \bc \pzero \;|\; x?F \;|\; x!C }
  \and
  \inferrule* [lab=abstraction] {} {{F} \bc (x)P}
  \and
  \inferrule* [lab=concretion] {} {{C} \bc \langle Q \rangle}
  \and
  \inferrule* [lab=process] {} {{P,Q} \bc M \;| \;P|Q \;|\; @{x}}
  \and
  \inferrule* [lab=name] {} {{x} \bc \quotep{P}}
\end{mathpar} 

Note that $\vec{x}$ (resp. $\vec{P}$) denotes a vector of names
(resp. processes) of length $|\vec{x}|$ (resp. $|\vec{P}|$). We adopt
the following useful abbreviations.

\begin{mathpar}
   x?(\vec{y}).P := x.(\vec{y})P \and  x\clift{\vec{P}} := x.\clift{\vec{P}}
   \and x!(y) := \lift{x}{\dropn{y}}
   \and \Pi_{i=0}^{n-1}P_i := P_0 | \ldots | P_{n-1}
\end{mathpar}

\subsubsection{Structural congruence}

\paragraph{Free and bound names and alpha-equivalence.} At the
core of structural equivalence is alpha-equivalence which identifies
process that are the same up to a change of variable. Formally, we
recognize the distinction between free and bound names. The free names
of a process, $\freenames{P}$, may be calculated recursively as
follows:

\begin{mathpar}
\freenames{\pzero} := \emptyset
  \and \\
  \freenames{x?(y).P} := \{ x \} \cup (\freenames{P} \setminus \{ y \})
  \and 
  \freenames{x!\langle P \rangle} := \{ x \} \cup \{ P \} 
  \and \\
  \freenames{P|Q} := \freenames{P} \cup \freenames{Q}
  \and \\
  \freenames{@{x}} := \{ x \}
\end{mathpar}

$\pi$
$\quotep{\pi}$

$\freenames{-} : \pi \to \mathcal{P}(\quotep{\pi})$

\begin{eqnarray*}
  \freenames{\pzero} & := & \emptyset \\
  \freenames{x?(y).P} & := & \{ x \} \cup (\freenames{P} \setminus \{ y \}) \\
  \freenames{x!\langle P \rangle} & := & \{ x \} \cup \{ P \} \\
  \freenames{P|Q} & := & \freenames{P} \cup \freenames{Q} \\
  \freenames{\dropn{x}} & := & \{ x \}
\end{eqnarray*}

The bound names of a process, $\boundnames{P}$, are those names occurring in $P$
that are not free. For example, in $x?(y).0$, the name $x$ is free, while $y$ is bound.

\begin{mathpar}
  \inferrule* [lab=monoidal-laws] {} { P|Q \equiv Q|P \and P|0 \equiv P \and P|(Q|R) \equiv (P|Q)|R }
\end{mathpar}

\begin{mathpar}
  \inferrule* [lab=alpha-equivalence] {} { (x)P \equiv (y)P\{y/x\} \and y \not\in \freenames{P} }
\end{mathpar}

\begin{definition}
Then two processes, $P,Q$, are alpha-equivalent if $P = Q\{\vec{y}/\vec{x}\}$ for
some $\vec{x} \in \boundnames{Q},\vec{y} \in \boundnames{P}$, where $Q\{\vec{y}/\vec{x}\}$
denotes the capture-avoiding substitution of $\vec{y}$ for $\vec{x}$ in $Q$.
\end{definition}

\begin{definition}
  The {\em structural congruence} \cite{SangiorgiWalker} , $\equiv$,
  between processes is the least congruence containing
  alpha-equivalence, satisfying the abelian monoid laws
  (associativity, commutativity and $\pzero$ as identity) for parallel
  composition $|$ and for summation $+$.
\end{definition}

\subsection{Name equivalence}

We take name equivalence, written $\nameeq$, to be the smallest
equivalence relation generated by the following rules.

\begin{mathpar}
\inferrule*[lab=Quote-drop]
{ }
{ \quotep{@{x}} \nameeq x }

\inferrule*[lab=Struct-equiv]
{ P \scong Q }
{ \quotep{P} \nameeq \quotep{Q} }
\end{mathpar}

The astute reader will have noticed that the mutual recursion of names
and processes imposes a mutual recursion on alpha-equivalence and
structural equivalence via name-equivalence. Fortunately, all of this
works out pleasantly and we may calculate in the natural way, free of
concern. The reader interested in the details is referred to the
appendix \ref{appendix:rho_details}.

\subsection{Substitution}

We use $\Proc$ for the set of processes, $\QProc$ for the set of
names, and $\id{\{}\vec{y} / \vec{x} \id{\}}$ to denote partial maps,
$s : \QProc \rightarrow \QProc$. A map, $s$ lifts, uniquely, to a map
on process terms, $\widehat{s} : \Proc \rightarrow \Proc$ by the
following equations.

\begin{mathpar}
  (0) \psubstp{Q}{P} := 0 \\
  (R \juxtap S) \psubstp{Q}{P}
  :=    
  (R)\psubstp{Q}{P} \juxtap (S) \psubstp{Q}{P} \\
  (x?(y).R) \psubstp{Q}{P}    
  :=    
  (x)\substp{Q}{P} (z)\concat( (R \psubstn{z}{y}) \psubstp{Q}{P} ) \\
  (\lift{x}{R}) \psubstp{Q}{P}  
  :=
  \lift{(x)\substp{Q}{P}}{ R \psubstp{Q}{P} } \\
%   (\dropn{x})  \psubstp{Q}{P}       
%   := 
%   \left\{ 
%     \begin{array}{ccc} 
%       \dropn{\quotep{Q}} & & x \nameeq \quotep{P} \\
%       \dropn{x} & & otherwise \\
%     \end{array}
%   \right. 
  (\dropn{x})  \psubstp{Q}{P}       
  := 
  \left\{ 
    \begin{array}{ccc} 
      Q & & x \nameeq \quotep{P} \\
      \dropn{x} & & otherwise \\
    \end{array}
  \right.
\end{mathpar}
 

where

\begin{eqnarray}
  (x)\id{\{} \lpquote Q \rpquote / \lpquote P \rpquote \id{\}}            = 
  \left\{ 
    \begin{array}{ccc}
      \lpquote Q \rpquote & & x \nameeq \lpquote P \rpquote \\
      x & & otherwise \\
    \end{array}
  \right. \nonumber
\end{eqnarray}

and $z$ is chosen distinct from $\quotep{P}$, $\quotep{Q}$, the free
names in $Q$, and all the names in $R$. Our $\alpha$-equivalence will
be built in the standard way from this substitution.

\begin{remark}\label{rem:no_self_referential_names}
  One consequence of these definitions is that $\forall P. \quotep{P}
  \not\in \freenames{P}$.
\end{remark}

\subsection{ Dynamic quote: an example }

Anticipating something of what's to come, consider applying the
substitution, $\widehat{\id{\{}u / z \id{\}}}$, to the following pair
of processes, $\lift{w}{y!(z)}$ and $w[ \lpquote y!(z) \rpquote ]$.

\begin{eqnarray}
	\lift{w}{y!(z)}\widehat{\id{\{}u / z \id{\}}}
		& = &
		\lift{w}{y!(u)} \nonumber\\
	w[ \lpquote y!(z) \rpquote ] \widehat{ \id{\{}u / z \id{\}} }
		& = &
		w[ \lpquote y!(z) \rpquote ] \nonumber
\end{eqnarray}

Because the body of the process between quotes is impervious to
substitution, we get radically different answers. In fact, by
examining the first process in an input context,
e.g. $x?(z).\lift{w}{y!(z)}$, we see that the process under the lift
operator may be shaped by prefixed inputs binding a name inside it. In
this sense, the lift operator will be seen as a way to dynamically
construct processes before reifying them as names.

Finally equipped with these standard features we can present the
dynamics of the calculus.

\subsubsection{Operational semantics} 

Finally, we introduce the computational dynamics. What marks these
algebras as distinct from other more traditionally studied algebraic
structures, e.g. vector spaces or polynomial rings, is the manner in
which dynamics is captured. In traditional structures, dynamics is typically
expressed through morphisms between such structures, as in linear maps
between vector spaces or morphisms between rings. In algebras
associated with the semantics of computation, the dynamics is
expressed as part of the algebraic structure itself, through a
reduction reduction relation typically denoted by $\red$. Below, we
give a recursive presentation of this relation for the calculus used
in the encoding.

$\red \subseteq \pi \times \pi$
$\red : \pi \to \mathcal{P}(\pi)$

\begin{mathpar}
  \inferrule* [lab=Comm] { \textsf{match}( x_{src}, x_{trgt} ) } { x_{trgt}?(y)P \; | \; x_{src}!\langle {Q} \rangle \red P\{\quotep{Q}/y}\} }
  \and \\
  \inferrule* [lab=Par] {{P} \red {P}'} {{{P} | {Q}} \red {{P}' | {Q}}}
  \and
  \inferrule* [lab=Equiv]{{{P} \scong {P}'} \andalso {{P}' \red {Q}'} \andalso {{Q}' \scong {Q}}}{{P} \red {Q}}
\end{mathpar}

\begin{eqnarray*}
  match_{\equiv} (\quotep{P},\quotep{Q}) & := & P \equiv Q \\
  match_{\dagger}(\quotep{P},\quotep{Q}) & := & \forall R. P|Q \red^{*} R => R \red^{*} 0 \\
  match_{K}(\quotep{P},\quotep{Q}) & := & K \mbox{ for some context } K
\end{eqnarray*}

$u?(x)P | u!\langle Q \rangle \red P\{\quotep{Q}/x\}$

%We write $\wred$ for $\red^*$, and $P\red$ if $\exists Q $ such that $ P \red Q$.
We write $P\red$ if $\exists Q $ such that $ P \red Q$ and $P\not\red$, otherwise.

\section{Replication}

As mentioned before, it is known that replication (and hence
recursion) can be implemented in a higher-order process algebra
\cite{SangiorgiWalker}. As our first example of calculation with the
machinery thus far presented we give the construction explicitly in
the {\rhoc}.

\begin{eqnarray}
	D_{x} & := & \prefix{x}{y}{(\binpar{\outputp{x}{y}}{@{y}})} \nonumber\\
	\bangp_{x}{P} & := & \binpar{{x}!\langle{\binpar{D_{x}}{P}}\rangle}{D_{x}} \nonumber
\end{eqnarray}

\begin{eqnarray}
	\bangp_{x}{P} & & \nonumber\\
	=
	& {x}!\langle{(\prefix{x}{y}{(\outputp{x}{y} | @{y})) | P}}\rangle 
	      | \prefix{x}{y}{(\outputp{x}{y} | @{y})} & \nonumber\\
	\red
	& (\outputp{x}{y} | @{y})\substn{\quotep{(\prefix{x}{y}{(@{y} | \outputp{x}{y})) | P}}}{y} & \nonumber\\
	=
	& \outputp{x}{\quotep{(\prefix{x}{y}{(\outputp{x}{y} | @{y})) | P}}}
	  | {(\prefix{x}{y}{(\outputp{x}{y} | @{y})) | P}} & \nonumber\\
	\red
	& \ldots & \nonumber\\
	\red^*
	& P | P | \ldots & \nonumber
\end{eqnarray}

Of course, this encoding, as an implementation, runs away, unfolding
$\bangp{P}$ eagerly. A lazier and more implementable replication
operator, restricted to input-guarded processes, may be obtained as follows.

\begin{eqnarray}
\bangp{\prefix{u}{v}{P}} 
	:= 
	\binpar{\lift{x}{\prefix{u}{v}{(\binpar{D(x)}{P})}}}{D(x)} \nonumber
\end{eqnarray}

\begin{remark}
  Note that the lazier definition still does not deal with summation
  or mixed summation (i.e. sums over input and output). The reader is
  invited to construct definitions of replication that deal with these
  features. 

  Further, the definitions are parameterized in a name, $x$. Can you,
  gentle reader, make a definition that eliminates this parameter and
  guarantees no accidental interaction between the replication
  machinery and the process being replicated -- i.e. no accidental
  sharing of names used by the process to get its work done and the
  name(s) used by the replication to effect copying. This latter
  revision of the definition of replication is crucial to obtaining
  the expected identity $!!P \sim !P$.
\end{remark}

\begin{remark}\label{rem:paradoxical_combinator}
  The reader familiar with the lambda calculus will have noticed the
  similarity between $D$ and the paradoxical combinator.

  [Ed. note: the existence of this seems to suggest we have to be more
  restrictive on the set of processes and names we admit if we are to
  support no-cloning.]
\end{remark}

\subsubsection{Bisimulation}

The computational dynamics gives rise to another kind of equivalence,
the equivalence of computational behavior. As previously mentioned
this is typically captured \emph{via} some form of bisimulation.

% The notion we use in this paper is weak barbed bisimulation
% \cite{milner91polyadicpi}.

The notion we use in this paper is derived from weak barbed
bisimulation \cite{milner91polyadicpi}. 

\begin{definition}
An \emph{observation relation}, $\downarrow_{\mathcal N}$, over a set
of names, $\mathcal N$, is the smallest relation satisfying the rules
below.

\infrule[Out-barb]{y \in {\mathcal N}, \; x \nameeq y}
		  {\outputp{x}{v} \downarrow_{\mathcal N} x}
\infrule[Par-barb]{\mbox{$P\downarrow_{\mathcal N} x$ or $Q\downarrow_{\mathcal N} x$}}
		  {\binpar{P}{Q} \downarrow_{\mathcal N} x}

We write $P \Downarrow_{\mathcal N} x$ if there is $Q$ such that 
$P \wred Q$ and $Q \downarrow_{\mathcal N} x$.
\end{definition}

\begin{definition}
%\label{def.bbisim}
An  ${\mathcal N}$-\emph{barbed bisimulation} over a set of names, ${\mathcal N}$, is a symmetric binary relation 
${\mathcal S}_{\mathcal N}$ between agents such that $P\rel{S}_{\mathcal N}Q$ implies:
\begin{enumerate}
\item If $P \red P'$ then $Q \wred Q'$ and $P'\rel{S}_{\mathcal N} Q'$.
\item If $P\downarrow_{\mathcal N} x$, then $Q\Downarrow_{\mathcal N} x$.
\end{enumerate}
$P$ is ${\mathcal N}$-barbed bisimilar to $Q$, written
$P \wbbisim_{\mathcal N} Q$, if $P \rel{S}_{\mathcal N} Q$ for some ${\mathcal N}$-barbed bisimulation ${\mathcal S}_{\mathcal N}$.
\end{definition}

$\mathcal{R} \subseteq \pi \times \pi$

$P \mathcal{R} Q => \forall P'. P \red P' \Rightarrow \exists Q'. Q \red Q', P' \mathcal{R} Q'$

$P \vdash x \Rightarrow Q \vdash x$

\begin{mathpar}
  \inferrule*[lab=Out-barb]{x \nameeq y}{{y}!\langle{Q}\rangle \vdash x}
  \and
  \inferrule*[lab=Par-barb]{\mbox{$P\vdash x$ or $Q\vdash x$}}{\binpar{P}{Q} \vdash x}
\end{mathpar}

\subsubsection{Contexts}

One of the principle advantages of computational calculi like the
$\pi$-calculus is a well-defined notion of context,
contextual-equivalence and a correlation between
contextual-equivalence and notions of bisimulation. The notion of
context allows the decomposition of a process into (sub-)process and
its syntactic environment, its context. Thus, a context may be
thought of as a process with a ``hole'' (written $\Box$) in it. The
application of a context $M$ to a process $P$, written $M[P]$, is
tantamount to filling the hole in $M$ with $P$. In this paper we do
not need the full weight of this theory, but do make use of the notion
of context in the proof the main theorem. 

\begin{mathpar}
  \inferrule* [lab=summation] {} {{M_{M},M_{N}} \bc \Box \;|\; x.M_{A} \;|\; M_{M}+M_{N}}
  \and
  \inferrule* [lab=agent] {} {{M_{A}} \bc (\vec{x})M_{P} \;| \; \clift{P_0,\ldots,M_{P},\ldots,P_N}}
  \and \\
  \inferrule* [lab=process] {} {{M_{P}} \bc M_{N} \;| \;P|M_{P} }
\end{mathpar} 

\begin{mathpar}
  \inferrule* [lab=sychronization] {} {M_{N} \bc \Box \;|\; x?M_{F} \;|\; x!M_{C}}
  \and
  \inferrule* [lab=abstraction] {} {{M_{F}} \bc (x)M_{P} }
  \and
  \inferrule* [lab=concretion] {} {{M_{C}} \bc \langle M_{P} \rangle }
  \and \\
  \inferrule* [lab=process] {} {{M_{P}} \bc M_{N} \;| \;P|M_{P} }
\end{mathpar}

\begin{definition}[contextual application] Given a context $M$, and
  process $P$, we define the \emph{contextual application}, $M[P] :=
  M\{P/\Box\}$. That is, the contextual application of M to P is the
  substitution of $P$ for $\Box$ in $M$.
\end{definition}

$\meaningof{-} : L \to \mathcal{P}(\pi)$

\begin{mathpar}
  \inferrule* [lab=collection] {} {\meaningof{true} = \pi, \and \meaningof{~E} = \pi \setminus \meaningof{E}, \and \meaningof{E_{1} \& E_{2}} = \meaningof{E_{1}} \cap \meaningof{E_{2}}}
\end{mathpar}

\begin{mathpar}
  \inferrule* [lab=structure] {} {\meaningof{0} = \{ P \in \pi | P \equiv 0 \}, \and \\ \meaningof{E_1 | E_2} = \{ P \in \pi | P \equiv P_{1} | P_{2}, P_{1} \in \meaningof{E_{1}}, P_{2} \in \meaningof{E_2}\} }
\end{mathpar}

\begin{mathpar}
 \inferrule* [lab=behavior] {} {\meaningof{\langle a?b \rangle E} = \{ P \in \pi | P \equiv Q | u?(y)P', \\ \and \\\\ \and \\ \;\;\; u \in \meaningof{a}, \forall z.P'\{z/y\} \in \meaningof{E\{z/b\}}\}, \and \\ \meaningof{a!E} = \{ P \in \pi | P \equiv Q | x!\langle P' \rangle, x \in \meaningof{a} P' \in \meaningof{E}\} }
\end{mathpar}

\begin{mathpar}
 \inferrule* [lab=nominal] {} {\meaningof{\quotep{E}} = \{ \quotep{P} \in \quotep{\pi} | P \in \meaningof{E} \}, \and \meaningof{\quotep{P}} = \{ \quotep{Q} \in \quotep{\pi} | P \equiv Q \} \and \\ \meaningof{@\quotep{E}} = \{ P \in \pi | P \equiv @x, x \in \meaningof{E} \}}
\end{mathpar}

\begin{eqnarray*}
  \\
  \meaningof{-} : TS \to ST
\end{eqnarray*}

\begin{eqnarray*}
  \\
  L : TS \to ST
\end{eqnarray*}

\begin{eqnarray*}
  \\
  P \models E \iff P \in \meaningof{E}
\end{eqnarray*}

\begin{eqnarray*}
  P \approx_{L} Q \iff \forall E \in L. P \models E \iff Q \models E
\end{eqnarray*}

\begin{eqnarray*}
  P \approx_{K} Q
\end{eqnarray*}

\begin{eqnarray*}
  P \approx Q
\end{eqnarray*}

$\approx_{K} = \approx = \approx_{L}$

\subsubsection{Contextual duality}

Note that contexts extend the quotation operation to a family of
operations from processes to names. Given a context, $M$, we can
define a \emph{nominal context}, $\quotep{M}$ by $\quotep{M}[P] :=
\quotep{M[P]}$. To foreshadow what is to come we observe that these
operations enjoy a duality with processes very much like the duality
between vectors and maps from vectors to scalars.

Further, because the calculus is essentially higher-order, we have a
correspondence between contexts and processes. More specifically,
given a name $x$ and a context $M$ we can construct $M^{*}_{x}$ such
that 

\begin{mathpar}
  M^{*}_{x} | \lift{x}{P} \red M[P]
\end{mathpar}

namely,

\begin{mathpar}
  M^{*}_{x} := x?(u).M[\dropn{u}]
\end{mathpar}

The dependence of $M^{*}_{x}$ on a name makes it an abstraction, 

\begin{mathpar}
  M^{*} := (x)x?(u).M[\dropn{u}]
\end{mathpar}

\subsection{Additional notation}

It will sometimes be convenient to denote the process a name
quotes. We already have the notation $x = \quotep{P}$, but it will be
convenient to introduce an alternate notation, $\procn{x}$, when we
want to emphasize the connection to the use of the name. Note that, by
virtue of name equivalence, $\quotep{\procn{x}} \nameeq x$; so, the
notation is consistent with previous definitions.

Further, because names have structure it is possible to effect
substitutions on the basis of that structure. This means we need to
upgrade our notation for substitutions, which we accomplish by
adapting comprehension notation. Thus,

\begin{mathpar}
  P\{ y / x : x \in S \}
\end{mathpar}

is interpreted to mean the process derived from P by replacing (in a
capture-avoiding manner) each occurrence of $x$ in $S$ by $y$. For example,

\begin{mathpar}
  P\{ \quotep{\procn{x}|\procn{x}} / x : x \in \freenames{P} \}
\end{mathpar}

will replace each (occurrence) of a free name $x$ in $P$ by
$\quotep{\procn{x}|\procn{x}}$.

Also, we will avail ourselves of the notation $x^{L}$ and $x^{R}$ to
denote injections of a name into disjoint copies of the name
space. There are numerous ways to accomplish this. One example can be
found in \cite{MeredithR05}. This notation overloads to vectors of
names: $\vec{x}^{\pi} := (x_{i}^{\pi} \; : \; 0 \leq i < |\vec{x}| )$ where $\pi \in \{L,R\}$.

We also use $P^{\Box} := P|\Box$.

In \cite{MeredithR05} an interpretation of the new operator is
given. It turns out that there are several possible interpretations
all enjoying the requisite algebraic properties of the operator (see
\cite{milner91polyadicpi}). We will therefore make liberal use of
$(\nu\; \vec{x})P$.

% subsection the_syntax_and_semantics_of_the_notation_system (end)   

\input{qm2pi.qmops} 

\input{qm2pi.sterngerlach} 

\input{qm2pi.metric} 

% section concurrent_process_calculi (end)

%\input{qm2pi.proofsketch}

% section proof sketch (end)

%\input{qm2pi.slviaknots} 

% section spatial logic via knots (end)

\input{qm2pi.conclusion}

% section conclusion (end)

%\input{qm2pi.dtcodes} 

% section wiring algorithm (end)

\input{qm2pi.ack} 

% section acknowledgments (end)

\newpage


\bibliographystyle{plain}   
\bibliography{../../biblios/main.bib}

\input{qm2pi.rhodetails}

\end{document}

 

% section notation (end)

\input{qm2pi.process.calculi} 

% section concurrent_process_calculi_and_spatial_logics_ (end)
    
%\documentclass[12pt]{llncs}
%\documentclass{jktr}

\usepackage[pdftex]{hyperref}                   
\usepackage {listings}
\usepackage {mathpartir}
\usepackage{bcprules}
%\usepackage{listings}
                       
\usepackage{graphicx} 
%\usepackage[margins=2.5cm,nohead,nofoot]{geometry}
%\usepackage{geometry}
\usepackage{amsfonts}
\usepackage{amstext}
\usepackage{latexsym}
\usepackage{amssymb}
\usepackage{color}


%\include{myPreamble}
\include{qm2pi.local} 

%\ifpdf
%\usepackage[pdftex]{graphicx}
%\else
%\usepackage{graphicx}
%\fi

 % \ifpdf
%  \usepackage{pdfsync}
%  \if


%\title{Brief Article}
%\author{David F. Snyder}
%\author{L.G. Meredith}

%\address{Dept. of Math., Texas State University--San Marcos, San Marcos, TX 78666}
       
\pagestyle{empty}


\begin{document}

\lstset{language=[Objective]Caml,frame=shadowbox}

\input{qm2pi.front}

% section front matter (end)

\input{qm2pi.intro} 
 
% section introduction (end)

% \input{qm2pi.knotations} 

% section notation (end)

\input{qm2pi.process.calculi} 

% section concurrent_process_calculi_and_spatial_logics_ (end)
    
%\input{qm2pi.knots2pi} 

%\input{qm2pi.trefoil} 

%\input{qm2pi.mainthm} 

% subsection basic_interpretation (end)

%\input{qm2pi.rho.presentation} 
\subsection{The syntax and semantics of the notation system}\label{sub:the_syntax_and_semantics_of_the_notation_system} % (fold)

We now summarize a technical presentation of the calculus that
embodies our theory of dynamics. The typical presentation of such a
calculus follows the style of giving generators and relations on
them. The grammar, below, describing term constructors, freely
generates the set of processes, $\Proc$. This set is then quotiented
by a relation known as structural congruence and it is over this set
that the notion of dynamics is expressed. This presentation is
essentially that of \cite{MeredithR05} with the addition of
polyadicity and summation. For readability we have relegated some of
the technical subtleties to an appendix.

\subsubsection{Process grammar}\label{subsub:process_grammar}

\begin{mathpar}
  \inferrule* [lab=synchronization] {} {{M} \bc \pzero \;|\; x?F \;|\; x!C }
  \and
  \inferrule* [lab=abstraction] {} {{F} \bc (x)P}
  \and
  \inferrule* [lab=concretion] {} {{C} \bc \langle Q \rangle}
  \and
  \inferrule* [lab=process] {} {{P,Q} \bc M \;| \;P|Q \;|\; @{x}}
  \and
  \inferrule* [lab=name] {} {{x} \bc \quotep{P}}
\end{mathpar} 

Note that $\vec{x}$ (resp. $\vec{P}$) denotes a vector of names
(resp. processes) of length $|\vec{x}|$ (resp. $|\vec{P}|$). We adopt
the following useful abbreviations.

\begin{mathpar}
   x?(\vec{y}).P := x.(\vec{y})P \and  x\clift{\vec{P}} := x.\clift{\vec{P}}
   \and x!(y) := \lift{x}{\dropn{y}}
   \and \Pi_{i=0}^{n-1}P_i := P_0 | \ldots | P_{n-1}
\end{mathpar}

\subsubsection{Structural congruence}

\paragraph{Free and bound names and alpha-equivalence.} At the
core of structural equivalence is alpha-equivalence which identifies
process that are the same up to a change of variable. Formally, we
recognize the distinction between free and bound names. The free names
of a process, $\freenames{P}$, may be calculated recursively as
follows:

\begin{mathpar}
\freenames{\pzero} := \emptyset
  \and \\
  \freenames{x?(y).P} := \{ x \} \cup (\freenames{P} \setminus \{ y \})
  \and 
  \freenames{x!\langle P \rangle} := \{ x \} \cup \{ P \} 
  \and \\
  \freenames{P|Q} := \freenames{P} \cup \freenames{Q}
  \and \\
  \freenames{@{x}} := \{ x \}
\end{mathpar}

$\pi$
$\quotep{\pi}$

$\freenames{-} : \pi \to \mathcal{P}(\quotep{\pi})$

\begin{eqnarray*}
  \freenames{\pzero} & := & \emptyset \\
  \freenames{x?(y).P} & := & \{ x \} \cup (\freenames{P} \setminus \{ y \}) \\
  \freenames{x!\langle P \rangle} & := & \{ x \} \cup \{ P \} \\
  \freenames{P|Q} & := & \freenames{P} \cup \freenames{Q} \\
  \freenames{\dropn{x}} & := & \{ x \}
\end{eqnarray*}

The bound names of a process, $\boundnames{P}$, are those names occurring in $P$
that are not free. For example, in $x?(y).0$, the name $x$ is free, while $y$ is bound.

\begin{mathpar}
  \inferrule* [lab=monoidal-laws] {} { P|Q \equiv Q|P \and P|0 \equiv P \and P|(Q|R) \equiv (P|Q)|R }
\end{mathpar}

\begin{mathpar}
  \inferrule* [lab=alpha-equivalence] {} { (x)P \equiv (y)P\{y/x\} \and y \not\in \freenames{P} }
\end{mathpar}

\begin{definition}
Then two processes, $P,Q$, are alpha-equivalent if $P = Q\{\vec{y}/\vec{x}\}$ for
some $\vec{x} \in \boundnames{Q},\vec{y} \in \boundnames{P}$, where $Q\{\vec{y}/\vec{x}\}$
denotes the capture-avoiding substitution of $\vec{y}$ for $\vec{x}$ in $Q$.
\end{definition}

\begin{definition}
  The {\em structural congruence} \cite{SangiorgiWalker} , $\equiv$,
  between processes is the least congruence containing
  alpha-equivalence, satisfying the abelian monoid laws
  (associativity, commutativity and $\pzero$ as identity) for parallel
  composition $|$ and for summation $+$.
\end{definition}

\subsection{Name equivalence}

We take name equivalence, written $\nameeq$, to be the smallest
equivalence relation generated by the following rules.

\begin{mathpar}
\inferrule*[lab=Quote-drop]
{ }
{ \quotep{@{x}} \nameeq x }

\inferrule*[lab=Struct-equiv]
{ P \scong Q }
{ \quotep{P} \nameeq \quotep{Q} }
\end{mathpar}

The astute reader will have noticed that the mutual recursion of names
and processes imposes a mutual recursion on alpha-equivalence and
structural equivalence via name-equivalence. Fortunately, all of this
works out pleasantly and we may calculate in the natural way, free of
concern. The reader interested in the details is referred to the
appendix \ref{appendix:rho_details}.

\subsection{Substitution}

We use $\Proc$ for the set of processes, $\QProc$ for the set of
names, and $\id{\{}\vec{y} / \vec{x} \id{\}}$ to denote partial maps,
$s : \QProc \rightarrow \QProc$. A map, $s$ lifts, uniquely, to a map
on process terms, $\widehat{s} : \Proc \rightarrow \Proc$ by the
following equations.

\begin{mathpar}
  (0) \psubstp{Q}{P} := 0 \\
  (R \juxtap S) \psubstp{Q}{P}
  :=    
  (R)\psubstp{Q}{P} \juxtap (S) \psubstp{Q}{P} \\
  (x?(y).R) \psubstp{Q}{P}    
  :=    
  (x)\substp{Q}{P} (z)\concat( (R \psubstn{z}{y}) \psubstp{Q}{P} ) \\
  (\lift{x}{R}) \psubstp{Q}{P}  
  :=
  \lift{(x)\substp{Q}{P}}{ R \psubstp{Q}{P} } \\
%   (\dropn{x})  \psubstp{Q}{P}       
%   := 
%   \left\{ 
%     \begin{array}{ccc} 
%       \dropn{\quotep{Q}} & & x \nameeq \quotep{P} \\
%       \dropn{x} & & otherwise \\
%     \end{array}
%   \right. 
  (\dropn{x})  \psubstp{Q}{P}       
  := 
  \left\{ 
    \begin{array}{ccc} 
      Q & & x \nameeq \quotep{P} \\
      \dropn{x} & & otherwise \\
    \end{array}
  \right.
\end{mathpar}
 

where

\begin{eqnarray}
  (x)\id{\{} \lpquote Q \rpquote / \lpquote P \rpquote \id{\}}            = 
  \left\{ 
    \begin{array}{ccc}
      \lpquote Q \rpquote & & x \nameeq \lpquote P \rpquote \\
      x & & otherwise \\
    \end{array}
  \right. \nonumber
\end{eqnarray}

and $z$ is chosen distinct from $\quotep{P}$, $\quotep{Q}$, the free
names in $Q$, and all the names in $R$. Our $\alpha$-equivalence will
be built in the standard way from this substitution.

\begin{remark}\label{rem:no_self_referential_names}
  One consequence of these definitions is that $\forall P. \quotep{P}
  \not\in \freenames{P}$.
\end{remark}

\subsection{ Dynamic quote: an example }

Anticipating something of what's to come, consider applying the
substitution, $\widehat{\id{\{}u / z \id{\}}}$, to the following pair
of processes, $\lift{w}{y!(z)}$ and $w[ \lpquote y!(z) \rpquote ]$.

\begin{eqnarray}
	\lift{w}{y!(z)}\widehat{\id{\{}u / z \id{\}}}
		& = &
		\lift{w}{y!(u)} \nonumber\\
	w[ \lpquote y!(z) \rpquote ] \widehat{ \id{\{}u / z \id{\}} }
		& = &
		w[ \lpquote y!(z) \rpquote ] \nonumber
\end{eqnarray}

Because the body of the process between quotes is impervious to
substitution, we get radically different answers. In fact, by
examining the first process in an input context,
e.g. $x?(z).\lift{w}{y!(z)}$, we see that the process under the lift
operator may be shaped by prefixed inputs binding a name inside it. In
this sense, the lift operator will be seen as a way to dynamically
construct processes before reifying them as names.

Finally equipped with these standard features we can present the
dynamics of the calculus.

\subsubsection{Operational semantics} 

Finally, we introduce the computational dynamics. What marks these
algebras as distinct from other more traditionally studied algebraic
structures, e.g. vector spaces or polynomial rings, is the manner in
which dynamics is captured. In traditional structures, dynamics is typically
expressed through morphisms between such structures, as in linear maps
between vector spaces or morphisms between rings. In algebras
associated with the semantics of computation, the dynamics is
expressed as part of the algebraic structure itself, through a
reduction reduction relation typically denoted by $\red$. Below, we
give a recursive presentation of this relation for the calculus used
in the encoding.

$\red \subseteq \pi \times \pi$
$\red : \pi \to \mathcal{P}(\pi)$

\begin{mathpar}
  \inferrule* [lab=Comm] { \textsf{match}( x_{src}, x_{trgt} ) } { x_{trgt}?(y)P \; | \; x_{src}!\langle {Q} \rangle \red P\{\quotep{Q}/y}\} }
  \and \\
  \inferrule* [lab=Par] {{P} \red {P}'} {{{P} | {Q}} \red {{P}' | {Q}}}
  \and
  \inferrule* [lab=Equiv]{{{P} \scong {P}'} \andalso {{P}' \red {Q}'} \andalso {{Q}' \scong {Q}}}{{P} \red {Q}}
\end{mathpar}

\begin{eqnarray*}
  match_{\equiv} (\quotep{P},\quotep{Q}) & := & P \equiv Q \\
  match_{\dagger}(\quotep{P},\quotep{Q}) & := & \forall R. P|Q \red^{*} R => R \red^{*} 0 \\
  match_{K}(\quotep{P},\quotep{Q}) & := & K \mbox{ for some context } K
\end{eqnarray*}

$u?(x)P | u!\langle Q \rangle \red P\{\quotep{Q}/x\}$

%We write $\wred$ for $\red^*$, and $P\red$ if $\exists Q $ such that $ P \red Q$.
We write $P\red$ if $\exists Q $ such that $ P \red Q$ and $P\not\red$, otherwise.

\section{Replication}

As mentioned before, it is known that replication (and hence
recursion) can be implemented in a higher-order process algebra
\cite{SangiorgiWalker}. As our first example of calculation with the
machinery thus far presented we give the construction explicitly in
the {\rhoc}.

\begin{eqnarray}
	D_{x} & := & \prefix{x}{y}{(\binpar{\outputp{x}{y}}{@{y}})} \nonumber\\
	\bangp_{x}{P} & := & \binpar{{x}!\langle{\binpar{D_{x}}{P}}\rangle}{D_{x}} \nonumber
\end{eqnarray}

\begin{eqnarray}
	\bangp_{x}{P} & & \nonumber\\
	=
	& {x}!\langle{(\prefix{x}{y}{(\outputp{x}{y} | @{y})) | P}}\rangle 
	      | \prefix{x}{y}{(\outputp{x}{y} | @{y})} & \nonumber\\
	\red
	& (\outputp{x}{y} | @{y})\substn{\quotep{(\prefix{x}{y}{(@{y} | \outputp{x}{y})) | P}}}{y} & \nonumber\\
	=
	& \outputp{x}{\quotep{(\prefix{x}{y}{(\outputp{x}{y} | @{y})) | P}}}
	  | {(\prefix{x}{y}{(\outputp{x}{y} | @{y})) | P}} & \nonumber\\
	\red
	& \ldots & \nonumber\\
	\red^*
	& P | P | \ldots & \nonumber
\end{eqnarray}

Of course, this encoding, as an implementation, runs away, unfolding
$\bangp{P}$ eagerly. A lazier and more implementable replication
operator, restricted to input-guarded processes, may be obtained as follows.

\begin{eqnarray}
\bangp{\prefix{u}{v}{P}} 
	:= 
	\binpar{\lift{x}{\prefix{u}{v}{(\binpar{D(x)}{P})}}}{D(x)} \nonumber
\end{eqnarray}

\begin{remark}
  Note that the lazier definition still does not deal with summation
  or mixed summation (i.e. sums over input and output). The reader is
  invited to construct definitions of replication that deal with these
  features. 

  Further, the definitions are parameterized in a name, $x$. Can you,
  gentle reader, make a definition that eliminates this parameter and
  guarantees no accidental interaction between the replication
  machinery and the process being replicated -- i.e. no accidental
  sharing of names used by the process to get its work done and the
  name(s) used by the replication to effect copying. This latter
  revision of the definition of replication is crucial to obtaining
  the expected identity $!!P \sim !P$.
\end{remark}

\begin{remark}\label{rem:paradoxical_combinator}
  The reader familiar with the lambda calculus will have noticed the
  similarity between $D$ and the paradoxical combinator.

  [Ed. note: the existence of this seems to suggest we have to be more
  restrictive on the set of processes and names we admit if we are to
  support no-cloning.]
\end{remark}

\subsubsection{Bisimulation}

The computational dynamics gives rise to another kind of equivalence,
the equivalence of computational behavior. As previously mentioned
this is typically captured \emph{via} some form of bisimulation.

% The notion we use in this paper is weak barbed bisimulation
% \cite{milner91polyadicpi}.

The notion we use in this paper is derived from weak barbed
bisimulation \cite{milner91polyadicpi}. 

\begin{definition}
An \emph{observation relation}, $\downarrow_{\mathcal N}$, over a set
of names, $\mathcal N$, is the smallest relation satisfying the rules
below.

\infrule[Out-barb]{y \in {\mathcal N}, \; x \nameeq y}
		  {\outputp{x}{v} \downarrow_{\mathcal N} x}
\infrule[Par-barb]{\mbox{$P\downarrow_{\mathcal N} x$ or $Q\downarrow_{\mathcal N} x$}}
		  {\binpar{P}{Q} \downarrow_{\mathcal N} x}

We write $P \Downarrow_{\mathcal N} x$ if there is $Q$ such that 
$P \wred Q$ and $Q \downarrow_{\mathcal N} x$.
\end{definition}

\begin{definition}
%\label{def.bbisim}
An  ${\mathcal N}$-\emph{barbed bisimulation} over a set of names, ${\mathcal N}$, is a symmetric binary relation 
${\mathcal S}_{\mathcal N}$ between agents such that $P\rel{S}_{\mathcal N}Q$ implies:
\begin{enumerate}
\item If $P \red P'$ then $Q \wred Q'$ and $P'\rel{S}_{\mathcal N} Q'$.
\item If $P\downarrow_{\mathcal N} x$, then $Q\Downarrow_{\mathcal N} x$.
\end{enumerate}
$P$ is ${\mathcal N}$-barbed bisimilar to $Q$, written
$P \wbbisim_{\mathcal N} Q$, if $P \rel{S}_{\mathcal N} Q$ for some ${\mathcal N}$-barbed bisimulation ${\mathcal S}_{\mathcal N}$.
\end{definition}

$\mathcal{R} \subseteq \pi \times \pi$

$P \mathcal{R} Q => \forall P'. P \red P' \Rightarrow \exists Q'. Q \red Q', P' \mathcal{R} Q'$

$P \vdash x \Rightarrow Q \vdash x$

\begin{mathpar}
  \inferrule*[lab=Out-barb]{x \nameeq y}{{y}!\langle{Q}\rangle \vdash x}
  \and
  \inferrule*[lab=Par-barb]{\mbox{$P\vdash x$ or $Q\vdash x$}}{\binpar{P}{Q} \vdash x}
\end{mathpar}

\subsubsection{Contexts}

One of the principle advantages of computational calculi like the
$\pi$-calculus is a well-defined notion of context,
contextual-equivalence and a correlation between
contextual-equivalence and notions of bisimulation. The notion of
context allows the decomposition of a process into (sub-)process and
its syntactic environment, its context. Thus, a context may be
thought of as a process with a ``hole'' (written $\Box$) in it. The
application of a context $M$ to a process $P$, written $M[P]$, is
tantamount to filling the hole in $M$ with $P$. In this paper we do
not need the full weight of this theory, but do make use of the notion
of context in the proof the main theorem. 

\begin{mathpar}
  \inferrule* [lab=summation] {} {{M_{M},M_{N}} \bc \Box \;|\; x.M_{A} \;|\; M_{M}+M_{N}}
  \and
  \inferrule* [lab=agent] {} {{M_{A}} \bc (\vec{x})M_{P} \;| \; \clift{P_0,\ldots,M_{P},\ldots,P_N}}
  \and \\
  \inferrule* [lab=process] {} {{M_{P}} \bc M_{N} \;| \;P|M_{P} }
\end{mathpar} 

\begin{mathpar}
  \inferrule* [lab=sychronization] {} {M_{N} \bc \Box \;|\; x?M_{F} \;|\; x!M_{C}}
  \and
  \inferrule* [lab=abstraction] {} {{M_{F}} \bc (x)M_{P} }
  \and
  \inferrule* [lab=concretion] {} {{M_{C}} \bc \langle M_{P} \rangle }
  \and \\
  \inferrule* [lab=process] {} {{M_{P}} \bc M_{N} \;| \;P|M_{P} }
\end{mathpar}

\begin{definition}[contextual application] Given a context $M$, and
  process $P$, we define the \emph{contextual application}, $M[P] :=
  M\{P/\Box\}$. That is, the contextual application of M to P is the
  substitution of $P$ for $\Box$ in $M$.
\end{definition}

$\meaningof{-} : L \to \mathcal{P}(\pi)$

\begin{mathpar}
  \inferrule* [lab=collection] {} {\meaningof{true} = \pi, \and \meaningof{~E} = \pi \setminus \meaningof{E}, \and \meaningof{E_{1} \& E_{2}} = \meaningof{E_{1}} \cap \meaningof{E_{2}}}
\end{mathpar}

\begin{mathpar}
  \inferrule* [lab=structure] {} {\meaningof{0} = \{ P \in \pi | P \equiv 0 \}, \and \\ \meaningof{E_1 | E_2} = \{ P \in \pi | P \equiv P_{1} | P_{2}, P_{1} \in \meaningof{E_{1}}, P_{2} \in \meaningof{E_2}\} }
\end{mathpar}

\begin{mathpar}
 \inferrule* [lab=behavior] {} {\meaningof{\langle a?b \rangle E} = \{ P \in \pi | P \equiv Q | u?(y)P', \\ \and \\\\ \and \\ \;\;\; u \in \meaningof{a}, \forall z.P'\{z/y\} \in \meaningof{E\{z/b\}}\}, \and \\ \meaningof{a!E} = \{ P \in \pi | P \equiv Q | x!\langle P' \rangle, x \in \meaningof{a} P' \in \meaningof{E}\} }
\end{mathpar}

\begin{mathpar}
 \inferrule* [lab=nominal] {} {\meaningof{\quotep{E}} = \{ \quotep{P} \in \quotep{\pi} | P \in \meaningof{E} \}, \and \meaningof{\quotep{P}} = \{ \quotep{Q} \in \quotep{\pi} | P \equiv Q \} \and \\ \meaningof{@\quotep{E}} = \{ P \in \pi | P \equiv @x, x \in \meaningof{E} \}}
\end{mathpar}

\begin{eqnarray*}
  \\
  \meaningof{-} : TS \to ST
\end{eqnarray*}

\begin{eqnarray*}
  \\
  L : TS \to ST
\end{eqnarray*}

\begin{eqnarray*}
  \\
  P \models E \iff P \in \meaningof{E}
\end{eqnarray*}

\begin{eqnarray*}
  P \approx_{L} Q \iff \forall E \in L. P \models E \iff Q \models E
\end{eqnarray*}

\begin{eqnarray*}
  P \approx_{K} Q
\end{eqnarray*}

\begin{eqnarray*}
  P \approx Q
\end{eqnarray*}

$\approx_{K} = \approx = \approx_{L}$

\subsubsection{Contextual duality}

Note that contexts extend the quotation operation to a family of
operations from processes to names. Given a context, $M$, we can
define a \emph{nominal context}, $\quotep{M}$ by $\quotep{M}[P] :=
\quotep{M[P]}$. To foreshadow what is to come we observe that these
operations enjoy a duality with processes very much like the duality
between vectors and maps from vectors to scalars.

Further, because the calculus is essentially higher-order, we have a
correspondence between contexts and processes. More specifically,
given a name $x$ and a context $M$ we can construct $M^{*}_{x}$ such
that 

\begin{mathpar}
  M^{*}_{x} | \lift{x}{P} \red M[P]
\end{mathpar}

namely,

\begin{mathpar}
  M^{*}_{x} := x?(u).M[\dropn{u}]
\end{mathpar}

The dependence of $M^{*}_{x}$ on a name makes it an abstraction, 

\begin{mathpar}
  M^{*} := (x)x?(u).M[\dropn{u}]
\end{mathpar}

\subsection{Additional notation}

It will sometimes be convenient to denote the process a name
quotes. We already have the notation $x = \quotep{P}$, but it will be
convenient to introduce an alternate notation, $\procn{x}$, when we
want to emphasize the connection to the use of the name. Note that, by
virtue of name equivalence, $\quotep{\procn{x}} \nameeq x$; so, the
notation is consistent with previous definitions.

Further, because names have structure it is possible to effect
substitutions on the basis of that structure. This means we need to
upgrade our notation for substitutions, which we accomplish by
adapting comprehension notation. Thus,

\begin{mathpar}
  P\{ y / x : x \in S \}
\end{mathpar}

is interpreted to mean the process derived from P by replacing (in a
capture-avoiding manner) each occurrence of $x$ in $S$ by $y$. For example,

\begin{mathpar}
  P\{ \quotep{\procn{x}|\procn{x}} / x : x \in \freenames{P} \}
\end{mathpar}

will replace each (occurrence) of a free name $x$ in $P$ by
$\quotep{\procn{x}|\procn{x}}$.

Also, we will avail ourselves of the notation $x^{L}$ and $x^{R}$ to
denote injections of a name into disjoint copies of the name
space. There are numerous ways to accomplish this. One example can be
found in \cite{MeredithR05}. This notation overloads to vectors of
names: $\vec{x}^{\pi} := (x_{i}^{\pi} \; : \; 0 \leq i < |\vec{x}| )$ where $\pi \in \{L,R\}$.

We also use $P^{\Box} := P|\Box$.

In \cite{MeredithR05} an interpretation of the new operator is
given. It turns out that there are several possible interpretations
all enjoying the requisite algebraic properties of the operator (see
\cite{milner91polyadicpi}). We will therefore make liberal use of
$(\nu\; \vec{x})P$.

% subsection the_syntax_and_semantics_of_the_notation_system (end)   

\input{qm2pi.qmops} 

\input{qm2pi.sterngerlach} 

\input{qm2pi.metric} 

% section concurrent_process_calculi (end)

%\input{qm2pi.proofsketch}

% section proof sketch (end)

%\input{qm2pi.slviaknots} 

% section spatial logic via knots (end)

\input{qm2pi.conclusion}

% section conclusion (end)

%\input{qm2pi.dtcodes} 

% section wiring algorithm (end)

\input{qm2pi.ack} 

% section acknowledgments (end)

\newpage


\bibliographystyle{plain}   
\bibliography{../../biblios/main.bib}

\input{qm2pi.rhodetails}

\end{document}

 

%\documentclass[12pt]{llncs}
%\documentclass{jktr}

\usepackage[pdftex]{hyperref}                   
\usepackage {listings}
\usepackage {mathpartir}
\usepackage{bcprules}
%\usepackage{listings}
                       
\usepackage{graphicx} 
%\usepackage[margins=2.5cm,nohead,nofoot]{geometry}
%\usepackage{geometry}
\usepackage{amsfonts}
\usepackage{amstext}
\usepackage{latexsym}
\usepackage{amssymb}
\usepackage{color}


%\include{myPreamble}
\include{qm2pi.local} 

%\ifpdf
%\usepackage[pdftex]{graphicx}
%\else
%\usepackage{graphicx}
%\fi

 % \ifpdf
%  \usepackage{pdfsync}
%  \if


%\title{Brief Article}
%\author{David F. Snyder}
%\author{L.G. Meredith}

%\address{Dept. of Math., Texas State University--San Marcos, San Marcos, TX 78666}
       
\pagestyle{empty}


\begin{document}

\lstset{language=[Objective]Caml,frame=shadowbox}

\input{qm2pi.front}

% section front matter (end)

\input{qm2pi.intro} 
 
% section introduction (end)

% \input{qm2pi.knotations} 

% section notation (end)

\input{qm2pi.process.calculi} 

% section concurrent_process_calculi_and_spatial_logics_ (end)
    
%\input{qm2pi.knots2pi} 

%\input{qm2pi.trefoil} 

%\input{qm2pi.mainthm} 

% subsection basic_interpretation (end)

%\input{qm2pi.rho.presentation} 
\subsection{The syntax and semantics of the notation system}\label{sub:the_syntax_and_semantics_of_the_notation_system} % (fold)

We now summarize a technical presentation of the calculus that
embodies our theory of dynamics. The typical presentation of such a
calculus follows the style of giving generators and relations on
them. The grammar, below, describing term constructors, freely
generates the set of processes, $\Proc$. This set is then quotiented
by a relation known as structural congruence and it is over this set
that the notion of dynamics is expressed. This presentation is
essentially that of \cite{MeredithR05} with the addition of
polyadicity and summation. For readability we have relegated some of
the technical subtleties to an appendix.

\subsubsection{Process grammar}\label{subsub:process_grammar}

\begin{mathpar}
  \inferrule* [lab=synchronization] {} {{M} \bc \pzero \;|\; x?F \;|\; x!C }
  \and
  \inferrule* [lab=abstraction] {} {{F} \bc (x)P}
  \and
  \inferrule* [lab=concretion] {} {{C} \bc \langle Q \rangle}
  \and
  \inferrule* [lab=process] {} {{P,Q} \bc M \;| \;P|Q \;|\; @{x}}
  \and
  \inferrule* [lab=name] {} {{x} \bc \quotep{P}}
\end{mathpar} 

Note that $\vec{x}$ (resp. $\vec{P}$) denotes a vector of names
(resp. processes) of length $|\vec{x}|$ (resp. $|\vec{P}|$). We adopt
the following useful abbreviations.

\begin{mathpar}
   x?(\vec{y}).P := x.(\vec{y})P \and  x\clift{\vec{P}} := x.\clift{\vec{P}}
   \and x!(y) := \lift{x}{\dropn{y}}
   \and \Pi_{i=0}^{n-1}P_i := P_0 | \ldots | P_{n-1}
\end{mathpar}

\subsubsection{Structural congruence}

\paragraph{Free and bound names and alpha-equivalence.} At the
core of structural equivalence is alpha-equivalence which identifies
process that are the same up to a change of variable. Formally, we
recognize the distinction between free and bound names. The free names
of a process, $\freenames{P}$, may be calculated recursively as
follows:

\begin{mathpar}
\freenames{\pzero} := \emptyset
  \and \\
  \freenames{x?(y).P} := \{ x \} \cup (\freenames{P} \setminus \{ y \})
  \and 
  \freenames{x!\langle P \rangle} := \{ x \} \cup \{ P \} 
  \and \\
  \freenames{P|Q} := \freenames{P} \cup \freenames{Q}
  \and \\
  \freenames{@{x}} := \{ x \}
\end{mathpar}

$\pi$
$\quotep{\pi}$

$\freenames{-} : \pi \to \mathcal{P}(\quotep{\pi})$

\begin{eqnarray*}
  \freenames{\pzero} & := & \emptyset \\
  \freenames{x?(y).P} & := & \{ x \} \cup (\freenames{P} \setminus \{ y \}) \\
  \freenames{x!\langle P \rangle} & := & \{ x \} \cup \{ P \} \\
  \freenames{P|Q} & := & \freenames{P} \cup \freenames{Q} \\
  \freenames{\dropn{x}} & := & \{ x \}
\end{eqnarray*}

The bound names of a process, $\boundnames{P}$, are those names occurring in $P$
that are not free. For example, in $x?(y).0$, the name $x$ is free, while $y$ is bound.

\begin{mathpar}
  \inferrule* [lab=monoidal-laws] {} { P|Q \equiv Q|P \and P|0 \equiv P \and P|(Q|R) \equiv (P|Q)|R }
\end{mathpar}

\begin{mathpar}
  \inferrule* [lab=alpha-equivalence] {} { (x)P \equiv (y)P\{y/x\} \and y \not\in \freenames{P} }
\end{mathpar}

\begin{definition}
Then two processes, $P,Q$, are alpha-equivalent if $P = Q\{\vec{y}/\vec{x}\}$ for
some $\vec{x} \in \boundnames{Q},\vec{y} \in \boundnames{P}$, where $Q\{\vec{y}/\vec{x}\}$
denotes the capture-avoiding substitution of $\vec{y}$ for $\vec{x}$ in $Q$.
\end{definition}

\begin{definition}
  The {\em structural congruence} \cite{SangiorgiWalker} , $\equiv$,
  between processes is the least congruence containing
  alpha-equivalence, satisfying the abelian monoid laws
  (associativity, commutativity and $\pzero$ as identity) for parallel
  composition $|$ and for summation $+$.
\end{definition}

\subsection{Name equivalence}

We take name equivalence, written $\nameeq$, to be the smallest
equivalence relation generated by the following rules.

\begin{mathpar}
\inferrule*[lab=Quote-drop]
{ }
{ \quotep{@{x}} \nameeq x }

\inferrule*[lab=Struct-equiv]
{ P \scong Q }
{ \quotep{P} \nameeq \quotep{Q} }
\end{mathpar}

The astute reader will have noticed that the mutual recursion of names
and processes imposes a mutual recursion on alpha-equivalence and
structural equivalence via name-equivalence. Fortunately, all of this
works out pleasantly and we may calculate in the natural way, free of
concern. The reader interested in the details is referred to the
appendix \ref{appendix:rho_details}.

\subsection{Substitution}

We use $\Proc$ for the set of processes, $\QProc$ for the set of
names, and $\id{\{}\vec{y} / \vec{x} \id{\}}$ to denote partial maps,
$s : \QProc \rightarrow \QProc$. A map, $s$ lifts, uniquely, to a map
on process terms, $\widehat{s} : \Proc \rightarrow \Proc$ by the
following equations.

\begin{mathpar}
  (0) \psubstp{Q}{P} := 0 \\
  (R \juxtap S) \psubstp{Q}{P}
  :=    
  (R)\psubstp{Q}{P} \juxtap (S) \psubstp{Q}{P} \\
  (x?(y).R) \psubstp{Q}{P}    
  :=    
  (x)\substp{Q}{P} (z)\concat( (R \psubstn{z}{y}) \psubstp{Q}{P} ) \\
  (\lift{x}{R}) \psubstp{Q}{P}  
  :=
  \lift{(x)\substp{Q}{P}}{ R \psubstp{Q}{P} } \\
%   (\dropn{x})  \psubstp{Q}{P}       
%   := 
%   \left\{ 
%     \begin{array}{ccc} 
%       \dropn{\quotep{Q}} & & x \nameeq \quotep{P} \\
%       \dropn{x} & & otherwise \\
%     \end{array}
%   \right. 
  (\dropn{x})  \psubstp{Q}{P}       
  := 
  \left\{ 
    \begin{array}{ccc} 
      Q & & x \nameeq \quotep{P} \\
      \dropn{x} & & otherwise \\
    \end{array}
  \right.
\end{mathpar}
 

where

\begin{eqnarray}
  (x)\id{\{} \lpquote Q \rpquote / \lpquote P \rpquote \id{\}}            = 
  \left\{ 
    \begin{array}{ccc}
      \lpquote Q \rpquote & & x \nameeq \lpquote P \rpquote \\
      x & & otherwise \\
    \end{array}
  \right. \nonumber
\end{eqnarray}

and $z$ is chosen distinct from $\quotep{P}$, $\quotep{Q}$, the free
names in $Q$, and all the names in $R$. Our $\alpha$-equivalence will
be built in the standard way from this substitution.

\begin{remark}\label{rem:no_self_referential_names}
  One consequence of these definitions is that $\forall P. \quotep{P}
  \not\in \freenames{P}$.
\end{remark}

\subsection{ Dynamic quote: an example }

Anticipating something of what's to come, consider applying the
substitution, $\widehat{\id{\{}u / z \id{\}}}$, to the following pair
of processes, $\lift{w}{y!(z)}$ and $w[ \lpquote y!(z) \rpquote ]$.

\begin{eqnarray}
	\lift{w}{y!(z)}\widehat{\id{\{}u / z \id{\}}}
		& = &
		\lift{w}{y!(u)} \nonumber\\
	w[ \lpquote y!(z) \rpquote ] \widehat{ \id{\{}u / z \id{\}} }
		& = &
		w[ \lpquote y!(z) \rpquote ] \nonumber
\end{eqnarray}

Because the body of the process between quotes is impervious to
substitution, we get radically different answers. In fact, by
examining the first process in an input context,
e.g. $x?(z).\lift{w}{y!(z)}$, we see that the process under the lift
operator may be shaped by prefixed inputs binding a name inside it. In
this sense, the lift operator will be seen as a way to dynamically
construct processes before reifying them as names.

Finally equipped with these standard features we can present the
dynamics of the calculus.

\subsubsection{Operational semantics} 

Finally, we introduce the computational dynamics. What marks these
algebras as distinct from other more traditionally studied algebraic
structures, e.g. vector spaces or polynomial rings, is the manner in
which dynamics is captured. In traditional structures, dynamics is typically
expressed through morphisms between such structures, as in linear maps
between vector spaces or morphisms between rings. In algebras
associated with the semantics of computation, the dynamics is
expressed as part of the algebraic structure itself, through a
reduction reduction relation typically denoted by $\red$. Below, we
give a recursive presentation of this relation for the calculus used
in the encoding.

$\red \subseteq \pi \times \pi$
$\red : \pi \to \mathcal{P}(\pi)$

\begin{mathpar}
  \inferrule* [lab=Comm] { \textsf{match}( x_{src}, x_{trgt} ) } { x_{trgt}?(y)P \; | \; x_{src}!\langle {Q} \rangle \red P\{\quotep{Q}/y}\} }
  \and \\
  \inferrule* [lab=Par] {{P} \red {P}'} {{{P} | {Q}} \red {{P}' | {Q}}}
  \and
  \inferrule* [lab=Equiv]{{{P} \scong {P}'} \andalso {{P}' \red {Q}'} \andalso {{Q}' \scong {Q}}}{{P} \red {Q}}
\end{mathpar}

\begin{eqnarray*}
  match_{\equiv} (\quotep{P},\quotep{Q}) & := & P \equiv Q \\
  match_{\dagger}(\quotep{P},\quotep{Q}) & := & \forall R. P|Q \red^{*} R => R \red^{*} 0 \\
  match_{K}(\quotep{P},\quotep{Q}) & := & K \mbox{ for some context } K
\end{eqnarray*}

$u?(x)P | u!\langle Q \rangle \red P\{\quotep{Q}/x\}$

%We write $\wred$ for $\red^*$, and $P\red$ if $\exists Q $ such that $ P \red Q$.
We write $P\red$ if $\exists Q $ such that $ P \red Q$ and $P\not\red$, otherwise.

\section{Replication}

As mentioned before, it is known that replication (and hence
recursion) can be implemented in a higher-order process algebra
\cite{SangiorgiWalker}. As our first example of calculation with the
machinery thus far presented we give the construction explicitly in
the {\rhoc}.

\begin{eqnarray}
	D_{x} & := & \prefix{x}{y}{(\binpar{\outputp{x}{y}}{@{y}})} \nonumber\\
	\bangp_{x}{P} & := & \binpar{{x}!\langle{\binpar{D_{x}}{P}}\rangle}{D_{x}} \nonumber
\end{eqnarray}

\begin{eqnarray}
	\bangp_{x}{P} & & \nonumber\\
	=
	& {x}!\langle{(\prefix{x}{y}{(\outputp{x}{y} | @{y})) | P}}\rangle 
	      | \prefix{x}{y}{(\outputp{x}{y} | @{y})} & \nonumber\\
	\red
	& (\outputp{x}{y} | @{y})\substn{\quotep{(\prefix{x}{y}{(@{y} | \outputp{x}{y})) | P}}}{y} & \nonumber\\
	=
	& \outputp{x}{\quotep{(\prefix{x}{y}{(\outputp{x}{y} | @{y})) | P}}}
	  | {(\prefix{x}{y}{(\outputp{x}{y} | @{y})) | P}} & \nonumber\\
	\red
	& \ldots & \nonumber\\
	\red^*
	& P | P | \ldots & \nonumber
\end{eqnarray}

Of course, this encoding, as an implementation, runs away, unfolding
$\bangp{P}$ eagerly. A lazier and more implementable replication
operator, restricted to input-guarded processes, may be obtained as follows.

\begin{eqnarray}
\bangp{\prefix{u}{v}{P}} 
	:= 
	\binpar{\lift{x}{\prefix{u}{v}{(\binpar{D(x)}{P})}}}{D(x)} \nonumber
\end{eqnarray}

\begin{remark}
  Note that the lazier definition still does not deal with summation
  or mixed summation (i.e. sums over input and output). The reader is
  invited to construct definitions of replication that deal with these
  features. 

  Further, the definitions are parameterized in a name, $x$. Can you,
  gentle reader, make a definition that eliminates this parameter and
  guarantees no accidental interaction between the replication
  machinery and the process being replicated -- i.e. no accidental
  sharing of names used by the process to get its work done and the
  name(s) used by the replication to effect copying. This latter
  revision of the definition of replication is crucial to obtaining
  the expected identity $!!P \sim !P$.
\end{remark}

\begin{remark}\label{rem:paradoxical_combinator}
  The reader familiar with the lambda calculus will have noticed the
  similarity between $D$ and the paradoxical combinator.

  [Ed. note: the existence of this seems to suggest we have to be more
  restrictive on the set of processes and names we admit if we are to
  support no-cloning.]
\end{remark}

\subsubsection{Bisimulation}

The computational dynamics gives rise to another kind of equivalence,
the equivalence of computational behavior. As previously mentioned
this is typically captured \emph{via} some form of bisimulation.

% The notion we use in this paper is weak barbed bisimulation
% \cite{milner91polyadicpi}.

The notion we use in this paper is derived from weak barbed
bisimulation \cite{milner91polyadicpi}. 

\begin{definition}
An \emph{observation relation}, $\downarrow_{\mathcal N}$, over a set
of names, $\mathcal N$, is the smallest relation satisfying the rules
below.

\infrule[Out-barb]{y \in {\mathcal N}, \; x \nameeq y}
		  {\outputp{x}{v} \downarrow_{\mathcal N} x}
\infrule[Par-barb]{\mbox{$P\downarrow_{\mathcal N} x$ or $Q\downarrow_{\mathcal N} x$}}
		  {\binpar{P}{Q} \downarrow_{\mathcal N} x}

We write $P \Downarrow_{\mathcal N} x$ if there is $Q$ such that 
$P \wred Q$ and $Q \downarrow_{\mathcal N} x$.
\end{definition}

\begin{definition}
%\label{def.bbisim}
An  ${\mathcal N}$-\emph{barbed bisimulation} over a set of names, ${\mathcal N}$, is a symmetric binary relation 
${\mathcal S}_{\mathcal N}$ between agents such that $P\rel{S}_{\mathcal N}Q$ implies:
\begin{enumerate}
\item If $P \red P'$ then $Q \wred Q'$ and $P'\rel{S}_{\mathcal N} Q'$.
\item If $P\downarrow_{\mathcal N} x$, then $Q\Downarrow_{\mathcal N} x$.
\end{enumerate}
$P$ is ${\mathcal N}$-barbed bisimilar to $Q$, written
$P \wbbisim_{\mathcal N} Q$, if $P \rel{S}_{\mathcal N} Q$ for some ${\mathcal N}$-barbed bisimulation ${\mathcal S}_{\mathcal N}$.
\end{definition}

$\mathcal{R} \subseteq \pi \times \pi$

$P \mathcal{R} Q => \forall P'. P \red P' \Rightarrow \exists Q'. Q \red Q', P' \mathcal{R} Q'$

$P \vdash x \Rightarrow Q \vdash x$

\begin{mathpar}
  \inferrule*[lab=Out-barb]{x \nameeq y}{{y}!\langle{Q}\rangle \vdash x}
  \and
  \inferrule*[lab=Par-barb]{\mbox{$P\vdash x$ or $Q\vdash x$}}{\binpar{P}{Q} \vdash x}
\end{mathpar}

\subsubsection{Contexts}

One of the principle advantages of computational calculi like the
$\pi$-calculus is a well-defined notion of context,
contextual-equivalence and a correlation between
contextual-equivalence and notions of bisimulation. The notion of
context allows the decomposition of a process into (sub-)process and
its syntactic environment, its context. Thus, a context may be
thought of as a process with a ``hole'' (written $\Box$) in it. The
application of a context $M$ to a process $P$, written $M[P]$, is
tantamount to filling the hole in $M$ with $P$. In this paper we do
not need the full weight of this theory, but do make use of the notion
of context in the proof the main theorem. 

\begin{mathpar}
  \inferrule* [lab=summation] {} {{M_{M},M_{N}} \bc \Box \;|\; x.M_{A} \;|\; M_{M}+M_{N}}
  \and
  \inferrule* [lab=agent] {} {{M_{A}} \bc (\vec{x})M_{P} \;| \; \clift{P_0,\ldots,M_{P},\ldots,P_N}}
  \and \\
  \inferrule* [lab=process] {} {{M_{P}} \bc M_{N} \;| \;P|M_{P} }
\end{mathpar} 

\begin{mathpar}
  \inferrule* [lab=sychronization] {} {M_{N} \bc \Box \;|\; x?M_{F} \;|\; x!M_{C}}
  \and
  \inferrule* [lab=abstraction] {} {{M_{F}} \bc (x)M_{P} }
  \and
  \inferrule* [lab=concretion] {} {{M_{C}} \bc \langle M_{P} \rangle }
  \and \\
  \inferrule* [lab=process] {} {{M_{P}} \bc M_{N} \;| \;P|M_{P} }
\end{mathpar}

\begin{definition}[contextual application] Given a context $M$, and
  process $P$, we define the \emph{contextual application}, $M[P] :=
  M\{P/\Box\}$. That is, the contextual application of M to P is the
  substitution of $P$ for $\Box$ in $M$.
\end{definition}

$\meaningof{-} : L \to \mathcal{P}(\pi)$

\begin{mathpar}
  \inferrule* [lab=collection] {} {\meaningof{true} = \pi, \and \meaningof{~E} = \pi \setminus \meaningof{E}, \and \meaningof{E_{1} \& E_{2}} = \meaningof{E_{1}} \cap \meaningof{E_{2}}}
\end{mathpar}

\begin{mathpar}
  \inferrule* [lab=structure] {} {\meaningof{0} = \{ P \in \pi | P \equiv 0 \}, \and \\ \meaningof{E_1 | E_2} = \{ P \in \pi | P \equiv P_{1} | P_{2}, P_{1} \in \meaningof{E_{1}}, P_{2} \in \meaningof{E_2}\} }
\end{mathpar}

\begin{mathpar}
 \inferrule* [lab=behavior] {} {\meaningof{\langle a?b \rangle E} = \{ P \in \pi | P \equiv Q | u?(y)P', \\ \and \\\\ \and \\ \;\;\; u \in \meaningof{a}, \forall z.P'\{z/y\} \in \meaningof{E\{z/b\}}\}, \and \\ \meaningof{a!E} = \{ P \in \pi | P \equiv Q | x!\langle P' \rangle, x \in \meaningof{a} P' \in \meaningof{E}\} }
\end{mathpar}

\begin{mathpar}
 \inferrule* [lab=nominal] {} {\meaningof{\quotep{E}} = \{ \quotep{P} \in \quotep{\pi} | P \in \meaningof{E} \}, \and \meaningof{\quotep{P}} = \{ \quotep{Q} \in \quotep{\pi} | P \equiv Q \} \and \\ \meaningof{@\quotep{E}} = \{ P \in \pi | P \equiv @x, x \in \meaningof{E} \}}
\end{mathpar}

\begin{eqnarray*}
  \\
  \meaningof{-} : TS \to ST
\end{eqnarray*}

\begin{eqnarray*}
  \\
  L : TS \to ST
\end{eqnarray*}

\begin{eqnarray*}
  \\
  P \models E \iff P \in \meaningof{E}
\end{eqnarray*}

\begin{eqnarray*}
  P \approx_{L} Q \iff \forall E \in L. P \models E \iff Q \models E
\end{eqnarray*}

\begin{eqnarray*}
  P \approx_{K} Q
\end{eqnarray*}

\begin{eqnarray*}
  P \approx Q
\end{eqnarray*}

$\approx_{K} = \approx = \approx_{L}$

\subsubsection{Contextual duality}

Note that contexts extend the quotation operation to a family of
operations from processes to names. Given a context, $M$, we can
define a \emph{nominal context}, $\quotep{M}$ by $\quotep{M}[P] :=
\quotep{M[P]}$. To foreshadow what is to come we observe that these
operations enjoy a duality with processes very much like the duality
between vectors and maps from vectors to scalars.

Further, because the calculus is essentially higher-order, we have a
correspondence between contexts and processes. More specifically,
given a name $x$ and a context $M$ we can construct $M^{*}_{x}$ such
that 

\begin{mathpar}
  M^{*}_{x} | \lift{x}{P} \red M[P]
\end{mathpar}

namely,

\begin{mathpar}
  M^{*}_{x} := x?(u).M[\dropn{u}]
\end{mathpar}

The dependence of $M^{*}_{x}$ on a name makes it an abstraction, 

\begin{mathpar}
  M^{*} := (x)x?(u).M[\dropn{u}]
\end{mathpar}

\subsection{Additional notation}

It will sometimes be convenient to denote the process a name
quotes. We already have the notation $x = \quotep{P}$, but it will be
convenient to introduce an alternate notation, $\procn{x}$, when we
want to emphasize the connection to the use of the name. Note that, by
virtue of name equivalence, $\quotep{\procn{x}} \nameeq x$; so, the
notation is consistent with previous definitions.

Further, because names have structure it is possible to effect
substitutions on the basis of that structure. This means we need to
upgrade our notation for substitutions, which we accomplish by
adapting comprehension notation. Thus,

\begin{mathpar}
  P\{ y / x : x \in S \}
\end{mathpar}

is interpreted to mean the process derived from P by replacing (in a
capture-avoiding manner) each occurrence of $x$ in $S$ by $y$. For example,

\begin{mathpar}
  P\{ \quotep{\procn{x}|\procn{x}} / x : x \in \freenames{P} \}
\end{mathpar}

will replace each (occurrence) of a free name $x$ in $P$ by
$\quotep{\procn{x}|\procn{x}}$.

Also, we will avail ourselves of the notation $x^{L}$ and $x^{R}$ to
denote injections of a name into disjoint copies of the name
space. There are numerous ways to accomplish this. One example can be
found in \cite{MeredithR05}. This notation overloads to vectors of
names: $\vec{x}^{\pi} := (x_{i}^{\pi} \; : \; 0 \leq i < |\vec{x}| )$ where $\pi \in \{L,R\}$.

We also use $P^{\Box} := P|\Box$.

In \cite{MeredithR05} an interpretation of the new operator is
given. It turns out that there are several possible interpretations
all enjoying the requisite algebraic properties of the operator (see
\cite{milner91polyadicpi}). We will therefore make liberal use of
$(\nu\; \vec{x})P$.

% subsection the_syntax_and_semantics_of_the_notation_system (end)   

\input{qm2pi.qmops} 

\input{qm2pi.sterngerlach} 

\input{qm2pi.metric} 

% section concurrent_process_calculi (end)

%\input{qm2pi.proofsketch}

% section proof sketch (end)

%\input{qm2pi.slviaknots} 

% section spatial logic via knots (end)

\input{qm2pi.conclusion}

% section conclusion (end)

%\input{qm2pi.dtcodes} 

% section wiring algorithm (end)

\input{qm2pi.ack} 

% section acknowledgments (end)

\newpage


\bibliographystyle{plain}   
\bibliography{../../biblios/main.bib}

\input{qm2pi.rhodetails}

\end{document}

 

%\documentclass[12pt]{llncs}
%\documentclass{jktr}

\usepackage[pdftex]{hyperref}                   
\usepackage {listings}
\usepackage {mathpartir}
\usepackage{bcprules}
%\usepackage{listings}
                       
\usepackage{graphicx} 
%\usepackage[margins=2.5cm,nohead,nofoot]{geometry}
%\usepackage{geometry}
\usepackage{amsfonts}
\usepackage{amstext}
\usepackage{latexsym}
\usepackage{amssymb}
\usepackage{color}


%\include{myPreamble}
\include{qm2pi.local} 

%\ifpdf
%\usepackage[pdftex]{graphicx}
%\else
%\usepackage{graphicx}
%\fi

 % \ifpdf
%  \usepackage{pdfsync}
%  \if


%\title{Brief Article}
%\author{David F. Snyder}
%\author{L.G. Meredith}

%\address{Dept. of Math., Texas State University--San Marcos, San Marcos, TX 78666}
       
\pagestyle{empty}


\begin{document}

\lstset{language=[Objective]Caml,frame=shadowbox}

\input{qm2pi.front}

% section front matter (end)

\input{qm2pi.intro} 
 
% section introduction (end)

% \input{qm2pi.knotations} 

% section notation (end)

\input{qm2pi.process.calculi} 

% section concurrent_process_calculi_and_spatial_logics_ (end)
    
%\input{qm2pi.knots2pi} 

%\input{qm2pi.trefoil} 

%\input{qm2pi.mainthm} 

% subsection basic_interpretation (end)

%\input{qm2pi.rho.presentation} 
\subsection{The syntax and semantics of the notation system}\label{sub:the_syntax_and_semantics_of_the_notation_system} % (fold)

We now summarize a technical presentation of the calculus that
embodies our theory of dynamics. The typical presentation of such a
calculus follows the style of giving generators and relations on
them. The grammar, below, describing term constructors, freely
generates the set of processes, $\Proc$. This set is then quotiented
by a relation known as structural congruence and it is over this set
that the notion of dynamics is expressed. This presentation is
essentially that of \cite{MeredithR05} with the addition of
polyadicity and summation. For readability we have relegated some of
the technical subtleties to an appendix.

\subsubsection{Process grammar}\label{subsub:process_grammar}

\begin{mathpar}
  \inferrule* [lab=synchronization] {} {{M} \bc \pzero \;|\; x?F \;|\; x!C }
  \and
  \inferrule* [lab=abstraction] {} {{F} \bc (x)P}
  \and
  \inferrule* [lab=concretion] {} {{C} \bc \langle Q \rangle}
  \and
  \inferrule* [lab=process] {} {{P,Q} \bc M \;| \;P|Q \;|\; @{x}}
  \and
  \inferrule* [lab=name] {} {{x} \bc \quotep{P}}
\end{mathpar} 

Note that $\vec{x}$ (resp. $\vec{P}$) denotes a vector of names
(resp. processes) of length $|\vec{x}|$ (resp. $|\vec{P}|$). We adopt
the following useful abbreviations.

\begin{mathpar}
   x?(\vec{y}).P := x.(\vec{y})P \and  x\clift{\vec{P}} := x.\clift{\vec{P}}
   \and x!(y) := \lift{x}{\dropn{y}}
   \and \Pi_{i=0}^{n-1}P_i := P_0 | \ldots | P_{n-1}
\end{mathpar}

\subsubsection{Structural congruence}

\paragraph{Free and bound names and alpha-equivalence.} At the
core of structural equivalence is alpha-equivalence which identifies
process that are the same up to a change of variable. Formally, we
recognize the distinction between free and bound names. The free names
of a process, $\freenames{P}$, may be calculated recursively as
follows:

\begin{mathpar}
\freenames{\pzero} := \emptyset
  \and \\
  \freenames{x?(y).P} := \{ x \} \cup (\freenames{P} \setminus \{ y \})
  \and 
  \freenames{x!\langle P \rangle} := \{ x \} \cup \{ P \} 
  \and \\
  \freenames{P|Q} := \freenames{P} \cup \freenames{Q}
  \and \\
  \freenames{@{x}} := \{ x \}
\end{mathpar}

$\pi$
$\quotep{\pi}$

$\freenames{-} : \pi \to \mathcal{P}(\quotep{\pi})$

\begin{eqnarray*}
  \freenames{\pzero} & := & \emptyset \\
  \freenames{x?(y).P} & := & \{ x \} \cup (\freenames{P} \setminus \{ y \}) \\
  \freenames{x!\langle P \rangle} & := & \{ x \} \cup \{ P \} \\
  \freenames{P|Q} & := & \freenames{P} \cup \freenames{Q} \\
  \freenames{\dropn{x}} & := & \{ x \}
\end{eqnarray*}

The bound names of a process, $\boundnames{P}$, are those names occurring in $P$
that are not free. For example, in $x?(y).0$, the name $x$ is free, while $y$ is bound.

\begin{mathpar}
  \inferrule* [lab=monoidal-laws] {} { P|Q \equiv Q|P \and P|0 \equiv P \and P|(Q|R) \equiv (P|Q)|R }
\end{mathpar}

\begin{mathpar}
  \inferrule* [lab=alpha-equivalence] {} { (x)P \equiv (y)P\{y/x\} \and y \not\in \freenames{P} }
\end{mathpar}

\begin{definition}
Then two processes, $P,Q$, are alpha-equivalent if $P = Q\{\vec{y}/\vec{x}\}$ for
some $\vec{x} \in \boundnames{Q},\vec{y} \in \boundnames{P}$, where $Q\{\vec{y}/\vec{x}\}$
denotes the capture-avoiding substitution of $\vec{y}$ for $\vec{x}$ in $Q$.
\end{definition}

\begin{definition}
  The {\em structural congruence} \cite{SangiorgiWalker} , $\equiv$,
  between processes is the least congruence containing
  alpha-equivalence, satisfying the abelian monoid laws
  (associativity, commutativity and $\pzero$ as identity) for parallel
  composition $|$ and for summation $+$.
\end{definition}

\subsection{Name equivalence}

We take name equivalence, written $\nameeq$, to be the smallest
equivalence relation generated by the following rules.

\begin{mathpar}
\inferrule*[lab=Quote-drop]
{ }
{ \quotep{@{x}} \nameeq x }

\inferrule*[lab=Struct-equiv]
{ P \scong Q }
{ \quotep{P} \nameeq \quotep{Q} }
\end{mathpar}

The astute reader will have noticed that the mutual recursion of names
and processes imposes a mutual recursion on alpha-equivalence and
structural equivalence via name-equivalence. Fortunately, all of this
works out pleasantly and we may calculate in the natural way, free of
concern. The reader interested in the details is referred to the
appendix \ref{appendix:rho_details}.

\subsection{Substitution}

We use $\Proc$ for the set of processes, $\QProc$ for the set of
names, and $\id{\{}\vec{y} / \vec{x} \id{\}}$ to denote partial maps,
$s : \QProc \rightarrow \QProc$. A map, $s$ lifts, uniquely, to a map
on process terms, $\widehat{s} : \Proc \rightarrow \Proc$ by the
following equations.

\begin{mathpar}
  (0) \psubstp{Q}{P} := 0 \\
  (R \juxtap S) \psubstp{Q}{P}
  :=    
  (R)\psubstp{Q}{P} \juxtap (S) \psubstp{Q}{P} \\
  (x?(y).R) \psubstp{Q}{P}    
  :=    
  (x)\substp{Q}{P} (z)\concat( (R \psubstn{z}{y}) \psubstp{Q}{P} ) \\
  (\lift{x}{R}) \psubstp{Q}{P}  
  :=
  \lift{(x)\substp{Q}{P}}{ R \psubstp{Q}{P} } \\
%   (\dropn{x})  \psubstp{Q}{P}       
%   := 
%   \left\{ 
%     \begin{array}{ccc} 
%       \dropn{\quotep{Q}} & & x \nameeq \quotep{P} \\
%       \dropn{x} & & otherwise \\
%     \end{array}
%   \right. 
  (\dropn{x})  \psubstp{Q}{P}       
  := 
  \left\{ 
    \begin{array}{ccc} 
      Q & & x \nameeq \quotep{P} \\
      \dropn{x} & & otherwise \\
    \end{array}
  \right.
\end{mathpar}
 

where

\begin{eqnarray}
  (x)\id{\{} \lpquote Q \rpquote / \lpquote P \rpquote \id{\}}            = 
  \left\{ 
    \begin{array}{ccc}
      \lpquote Q \rpquote & & x \nameeq \lpquote P \rpquote \\
      x & & otherwise \\
    \end{array}
  \right. \nonumber
\end{eqnarray}

and $z$ is chosen distinct from $\quotep{P}$, $\quotep{Q}$, the free
names in $Q$, and all the names in $R$. Our $\alpha$-equivalence will
be built in the standard way from this substitution.

\begin{remark}\label{rem:no_self_referential_names}
  One consequence of these definitions is that $\forall P. \quotep{P}
  \not\in \freenames{P}$.
\end{remark}

\subsection{ Dynamic quote: an example }

Anticipating something of what's to come, consider applying the
substitution, $\widehat{\id{\{}u / z \id{\}}}$, to the following pair
of processes, $\lift{w}{y!(z)}$ and $w[ \lpquote y!(z) \rpquote ]$.

\begin{eqnarray}
	\lift{w}{y!(z)}\widehat{\id{\{}u / z \id{\}}}
		& = &
		\lift{w}{y!(u)} \nonumber\\
	w[ \lpquote y!(z) \rpquote ] \widehat{ \id{\{}u / z \id{\}} }
		& = &
		w[ \lpquote y!(z) \rpquote ] \nonumber
\end{eqnarray}

Because the body of the process between quotes is impervious to
substitution, we get radically different answers. In fact, by
examining the first process in an input context,
e.g. $x?(z).\lift{w}{y!(z)}$, we see that the process under the lift
operator may be shaped by prefixed inputs binding a name inside it. In
this sense, the lift operator will be seen as a way to dynamically
construct processes before reifying them as names.

Finally equipped with these standard features we can present the
dynamics of the calculus.

\subsubsection{Operational semantics} 

Finally, we introduce the computational dynamics. What marks these
algebras as distinct from other more traditionally studied algebraic
structures, e.g. vector spaces or polynomial rings, is the manner in
which dynamics is captured. In traditional structures, dynamics is typically
expressed through morphisms between such structures, as in linear maps
between vector spaces or morphisms between rings. In algebras
associated with the semantics of computation, the dynamics is
expressed as part of the algebraic structure itself, through a
reduction reduction relation typically denoted by $\red$. Below, we
give a recursive presentation of this relation for the calculus used
in the encoding.

$\red \subseteq \pi \times \pi$
$\red : \pi \to \mathcal{P}(\pi)$

\begin{mathpar}
  \inferrule* [lab=Comm] { \textsf{match}( x_{src}, x_{trgt} ) } { x_{trgt}?(y)P \; | \; x_{src}!\langle {Q} \rangle \red P\{\quotep{Q}/y}\} }
  \and \\
  \inferrule* [lab=Par] {{P} \red {P}'} {{{P} | {Q}} \red {{P}' | {Q}}}
  \and
  \inferrule* [lab=Equiv]{{{P} \scong {P}'} \andalso {{P}' \red {Q}'} \andalso {{Q}' \scong {Q}}}{{P} \red {Q}}
\end{mathpar}

\begin{eqnarray*}
  match_{\equiv} (\quotep{P},\quotep{Q}) & := & P \equiv Q \\
  match_{\dagger}(\quotep{P},\quotep{Q}) & := & \forall R. P|Q \red^{*} R => R \red^{*} 0 \\
  match_{K}(\quotep{P},\quotep{Q}) & := & K \mbox{ for some context } K
\end{eqnarray*}

$u?(x)P | u!\langle Q \rangle \red P\{\quotep{Q}/x\}$

%We write $\wred$ for $\red^*$, and $P\red$ if $\exists Q $ such that $ P \red Q$.
We write $P\red$ if $\exists Q $ such that $ P \red Q$ and $P\not\red$, otherwise.

\section{Replication}

As mentioned before, it is known that replication (and hence
recursion) can be implemented in a higher-order process algebra
\cite{SangiorgiWalker}. As our first example of calculation with the
machinery thus far presented we give the construction explicitly in
the {\rhoc}.

\begin{eqnarray}
	D_{x} & := & \prefix{x}{y}{(\binpar{\outputp{x}{y}}{@{y}})} \nonumber\\
	\bangp_{x}{P} & := & \binpar{{x}!\langle{\binpar{D_{x}}{P}}\rangle}{D_{x}} \nonumber
\end{eqnarray}

\begin{eqnarray}
	\bangp_{x}{P} & & \nonumber\\
	=
	& {x}!\langle{(\prefix{x}{y}{(\outputp{x}{y} | @{y})) | P}}\rangle 
	      | \prefix{x}{y}{(\outputp{x}{y} | @{y})} & \nonumber\\
	\red
	& (\outputp{x}{y} | @{y})\substn{\quotep{(\prefix{x}{y}{(@{y} | \outputp{x}{y})) | P}}}{y} & \nonumber\\
	=
	& \outputp{x}{\quotep{(\prefix{x}{y}{(\outputp{x}{y} | @{y})) | P}}}
	  | {(\prefix{x}{y}{(\outputp{x}{y} | @{y})) | P}} & \nonumber\\
	\red
	& \ldots & \nonumber\\
	\red^*
	& P | P | \ldots & \nonumber
\end{eqnarray}

Of course, this encoding, as an implementation, runs away, unfolding
$\bangp{P}$ eagerly. A lazier and more implementable replication
operator, restricted to input-guarded processes, may be obtained as follows.

\begin{eqnarray}
\bangp{\prefix{u}{v}{P}} 
	:= 
	\binpar{\lift{x}{\prefix{u}{v}{(\binpar{D(x)}{P})}}}{D(x)} \nonumber
\end{eqnarray}

\begin{remark}
  Note that the lazier definition still does not deal with summation
  or mixed summation (i.e. sums over input and output). The reader is
  invited to construct definitions of replication that deal with these
  features. 

  Further, the definitions are parameterized in a name, $x$. Can you,
  gentle reader, make a definition that eliminates this parameter and
  guarantees no accidental interaction between the replication
  machinery and the process being replicated -- i.e. no accidental
  sharing of names used by the process to get its work done and the
  name(s) used by the replication to effect copying. This latter
  revision of the definition of replication is crucial to obtaining
  the expected identity $!!P \sim !P$.
\end{remark}

\begin{remark}\label{rem:paradoxical_combinator}
  The reader familiar with the lambda calculus will have noticed the
  similarity between $D$ and the paradoxical combinator.

  [Ed. note: the existence of this seems to suggest we have to be more
  restrictive on the set of processes and names we admit if we are to
  support no-cloning.]
\end{remark}

\subsubsection{Bisimulation}

The computational dynamics gives rise to another kind of equivalence,
the equivalence of computational behavior. As previously mentioned
this is typically captured \emph{via} some form of bisimulation.

% The notion we use in this paper is weak barbed bisimulation
% \cite{milner91polyadicpi}.

The notion we use in this paper is derived from weak barbed
bisimulation \cite{milner91polyadicpi}. 

\begin{definition}
An \emph{observation relation}, $\downarrow_{\mathcal N}$, over a set
of names, $\mathcal N$, is the smallest relation satisfying the rules
below.

\infrule[Out-barb]{y \in {\mathcal N}, \; x \nameeq y}
		  {\outputp{x}{v} \downarrow_{\mathcal N} x}
\infrule[Par-barb]{\mbox{$P\downarrow_{\mathcal N} x$ or $Q\downarrow_{\mathcal N} x$}}
		  {\binpar{P}{Q} \downarrow_{\mathcal N} x}

We write $P \Downarrow_{\mathcal N} x$ if there is $Q$ such that 
$P \wred Q$ and $Q \downarrow_{\mathcal N} x$.
\end{definition}

\begin{definition}
%\label{def.bbisim}
An  ${\mathcal N}$-\emph{barbed bisimulation} over a set of names, ${\mathcal N}$, is a symmetric binary relation 
${\mathcal S}_{\mathcal N}$ between agents such that $P\rel{S}_{\mathcal N}Q$ implies:
\begin{enumerate}
\item If $P \red P'$ then $Q \wred Q'$ and $P'\rel{S}_{\mathcal N} Q'$.
\item If $P\downarrow_{\mathcal N} x$, then $Q\Downarrow_{\mathcal N} x$.
\end{enumerate}
$P$ is ${\mathcal N}$-barbed bisimilar to $Q$, written
$P \wbbisim_{\mathcal N} Q$, if $P \rel{S}_{\mathcal N} Q$ for some ${\mathcal N}$-barbed bisimulation ${\mathcal S}_{\mathcal N}$.
\end{definition}

$\mathcal{R} \subseteq \pi \times \pi$

$P \mathcal{R} Q => \forall P'. P \red P' \Rightarrow \exists Q'. Q \red Q', P' \mathcal{R} Q'$

$P \vdash x \Rightarrow Q \vdash x$

\begin{mathpar}
  \inferrule*[lab=Out-barb]{x \nameeq y}{{y}!\langle{Q}\rangle \vdash x}
  \and
  \inferrule*[lab=Par-barb]{\mbox{$P\vdash x$ or $Q\vdash x$}}{\binpar{P}{Q} \vdash x}
\end{mathpar}

\subsubsection{Contexts}

One of the principle advantages of computational calculi like the
$\pi$-calculus is a well-defined notion of context,
contextual-equivalence and a correlation between
contextual-equivalence and notions of bisimulation. The notion of
context allows the decomposition of a process into (sub-)process and
its syntactic environment, its context. Thus, a context may be
thought of as a process with a ``hole'' (written $\Box$) in it. The
application of a context $M$ to a process $P$, written $M[P]$, is
tantamount to filling the hole in $M$ with $P$. In this paper we do
not need the full weight of this theory, but do make use of the notion
of context in the proof the main theorem. 

\begin{mathpar}
  \inferrule* [lab=summation] {} {{M_{M},M_{N}} \bc \Box \;|\; x.M_{A} \;|\; M_{M}+M_{N}}
  \and
  \inferrule* [lab=agent] {} {{M_{A}} \bc (\vec{x})M_{P} \;| \; \clift{P_0,\ldots,M_{P},\ldots,P_N}}
  \and \\
  \inferrule* [lab=process] {} {{M_{P}} \bc M_{N} \;| \;P|M_{P} }
\end{mathpar} 

\begin{mathpar}
  \inferrule* [lab=sychronization] {} {M_{N} \bc \Box \;|\; x?M_{F} \;|\; x!M_{C}}
  \and
  \inferrule* [lab=abstraction] {} {{M_{F}} \bc (x)M_{P} }
  \and
  \inferrule* [lab=concretion] {} {{M_{C}} \bc \langle M_{P} \rangle }
  \and \\
  \inferrule* [lab=process] {} {{M_{P}} \bc M_{N} \;| \;P|M_{P} }
\end{mathpar}

\begin{definition}[contextual application] Given a context $M$, and
  process $P$, we define the \emph{contextual application}, $M[P] :=
  M\{P/\Box\}$. That is, the contextual application of M to P is the
  substitution of $P$ for $\Box$ in $M$.
\end{definition}

$\meaningof{-} : L \to \mathcal{P}(\pi)$

\begin{mathpar}
  \inferrule* [lab=collection] {} {\meaningof{true} = \pi, \and \meaningof{~E} = \pi \setminus \meaningof{E}, \and \meaningof{E_{1} \& E_{2}} = \meaningof{E_{1}} \cap \meaningof{E_{2}}}
\end{mathpar}

\begin{mathpar}
  \inferrule* [lab=structure] {} {\meaningof{0} = \{ P \in \pi | P \equiv 0 \}, \and \\ \meaningof{E_1 | E_2} = \{ P \in \pi | P \equiv P_{1} | P_{2}, P_{1} \in \meaningof{E_{1}}, P_{2} \in \meaningof{E_2}\} }
\end{mathpar}

\begin{mathpar}
 \inferrule* [lab=behavior] {} {\meaningof{\langle a?b \rangle E} = \{ P \in \pi | P \equiv Q | u?(y)P', \\ \and \\\\ \and \\ \;\;\; u \in \meaningof{a}, \forall z.P'\{z/y\} \in \meaningof{E\{z/b\}}\}, \and \\ \meaningof{a!E} = \{ P \in \pi | P \equiv Q | x!\langle P' \rangle, x \in \meaningof{a} P' \in \meaningof{E}\} }
\end{mathpar}

\begin{mathpar}
 \inferrule* [lab=nominal] {} {\meaningof{\quotep{E}} = \{ \quotep{P} \in \quotep{\pi} | P \in \meaningof{E} \}, \and \meaningof{\quotep{P}} = \{ \quotep{Q} \in \quotep{\pi} | P \equiv Q \} \and \\ \meaningof{@\quotep{E}} = \{ P \in \pi | P \equiv @x, x \in \meaningof{E} \}}
\end{mathpar}

\begin{eqnarray*}
  \\
  \meaningof{-} : TS \to ST
\end{eqnarray*}

\begin{eqnarray*}
  \\
  L : TS \to ST
\end{eqnarray*}

\begin{eqnarray*}
  \\
  P \models E \iff P \in \meaningof{E}
\end{eqnarray*}

\begin{eqnarray*}
  P \approx_{L} Q \iff \forall E \in L. P \models E \iff Q \models E
\end{eqnarray*}

\begin{eqnarray*}
  P \approx_{K} Q
\end{eqnarray*}

\begin{eqnarray*}
  P \approx Q
\end{eqnarray*}

$\approx_{K} = \approx = \approx_{L}$

\subsubsection{Contextual duality}

Note that contexts extend the quotation operation to a family of
operations from processes to names. Given a context, $M$, we can
define a \emph{nominal context}, $\quotep{M}$ by $\quotep{M}[P] :=
\quotep{M[P]}$. To foreshadow what is to come we observe that these
operations enjoy a duality with processes very much like the duality
between vectors and maps from vectors to scalars.

Further, because the calculus is essentially higher-order, we have a
correspondence between contexts and processes. More specifically,
given a name $x$ and a context $M$ we can construct $M^{*}_{x}$ such
that 

\begin{mathpar}
  M^{*}_{x} | \lift{x}{P} \red M[P]
\end{mathpar}

namely,

\begin{mathpar}
  M^{*}_{x} := x?(u).M[\dropn{u}]
\end{mathpar}

The dependence of $M^{*}_{x}$ on a name makes it an abstraction, 

\begin{mathpar}
  M^{*} := (x)x?(u).M[\dropn{u}]
\end{mathpar}

\subsection{Additional notation}

It will sometimes be convenient to denote the process a name
quotes. We already have the notation $x = \quotep{P}$, but it will be
convenient to introduce an alternate notation, $\procn{x}$, when we
want to emphasize the connection to the use of the name. Note that, by
virtue of name equivalence, $\quotep{\procn{x}} \nameeq x$; so, the
notation is consistent with previous definitions.

Further, because names have structure it is possible to effect
substitutions on the basis of that structure. This means we need to
upgrade our notation for substitutions, which we accomplish by
adapting comprehension notation. Thus,

\begin{mathpar}
  P\{ y / x : x \in S \}
\end{mathpar}

is interpreted to mean the process derived from P by replacing (in a
capture-avoiding manner) each occurrence of $x$ in $S$ by $y$. For example,

\begin{mathpar}
  P\{ \quotep{\procn{x}|\procn{x}} / x : x \in \freenames{P} \}
\end{mathpar}

will replace each (occurrence) of a free name $x$ in $P$ by
$\quotep{\procn{x}|\procn{x}}$.

Also, we will avail ourselves of the notation $x^{L}$ and $x^{R}$ to
denote injections of a name into disjoint copies of the name
space. There are numerous ways to accomplish this. One example can be
found in \cite{MeredithR05}. This notation overloads to vectors of
names: $\vec{x}^{\pi} := (x_{i}^{\pi} \; : \; 0 \leq i < |\vec{x}| )$ where $\pi \in \{L,R\}$.

We also use $P^{\Box} := P|\Box$.

In \cite{MeredithR05} an interpretation of the new operator is
given. It turns out that there are several possible interpretations
all enjoying the requisite algebraic properties of the operator (see
\cite{milner91polyadicpi}). We will therefore make liberal use of
$(\nu\; \vec{x})P$.

% subsection the_syntax_and_semantics_of_the_notation_system (end)   

\input{qm2pi.qmops} 

\input{qm2pi.sterngerlach} 

\input{qm2pi.metric} 

% section concurrent_process_calculi (end)

%\input{qm2pi.proofsketch}

% section proof sketch (end)

%\input{qm2pi.slviaknots} 

% section spatial logic via knots (end)

\input{qm2pi.conclusion}

% section conclusion (end)

%\input{qm2pi.dtcodes} 

% section wiring algorithm (end)

\input{qm2pi.ack} 

% section acknowledgments (end)

\newpage


\bibliographystyle{plain}   
\bibliography{../../biblios/main.bib}

\input{qm2pi.rhodetails}

\end{document}

 

% subsection basic_interpretation (end)

%\input{qm2pi.rho.presentation} 
\subsection{The syntax and semantics of the notation system}\label{sub:the_syntax_and_semantics_of_the_notation_system} % (fold)

We now summarize a technical presentation of the calculus that
embodies our theory of dynamics. The typical presentation of such a
calculus follows the style of giving generators and relations on
them. The grammar, below, describing term constructors, freely
generates the set of processes, $\Proc$. This set is then quotiented
by a relation known as structural congruence and it is over this set
that the notion of dynamics is expressed. This presentation is
essentially that of \cite{MeredithR05} with the addition of
polyadicity and summation. For readability we have relegated some of
the technical subtleties to an appendix.

\subsubsection{Process grammar}\label{subsub:process_grammar}

\begin{mathpar}
  \inferrule* [lab=synchronization] {} {{M} \bc \pzero \;|\; x?F \;|\; x!C }
  \and
  \inferrule* [lab=abstraction] {} {{F} \bc (x)P}
  \and
  \inferrule* [lab=concretion] {} {{C} \bc \langle Q \rangle}
  \and
  \inferrule* [lab=process] {} {{P,Q} \bc M \;| \;P|Q \;|\; @{x}}
  \and
  \inferrule* [lab=name] {} {{x} \bc \quotep{P}}
\end{mathpar} 

Note that $\vec{x}$ (resp. $\vec{P}$) denotes a vector of names
(resp. processes) of length $|\vec{x}|$ (resp. $|\vec{P}|$). We adopt
the following useful abbreviations.

\begin{mathpar}
   x?(\vec{y}).P := x.(\vec{y})P \and  x\clift{\vec{P}} := x.\clift{\vec{P}}
   \and x!(y) := \lift{x}{\dropn{y}}
   \and \Pi_{i=0}^{n-1}P_i := P_0 | \ldots | P_{n-1}
\end{mathpar}

\subsubsection{Structural congruence}

\paragraph{Free and bound names and alpha-equivalence.} At the
core of structural equivalence is alpha-equivalence which identifies
process that are the same up to a change of variable. Formally, we
recognize the distinction between free and bound names. The free names
of a process, $\freenames{P}$, may be calculated recursively as
follows:

\begin{mathpar}
\freenames{\pzero} := \emptyset
  \and \\
  \freenames{x?(y).P} := \{ x \} \cup (\freenames{P} \setminus \{ y \})
  \and 
  \freenames{x!\langle P \rangle} := \{ x \} \cup \{ P \} 
  \and \\
  \freenames{P|Q} := \freenames{P} \cup \freenames{Q}
  \and \\
  \freenames{@{x}} := \{ x \}
\end{mathpar}

$\pi$
$\quotep{\pi}$

$\freenames{-} : \pi \to \mathcal{P}(\quotep{\pi})$

\begin{eqnarray*}
  \freenames{\pzero} & := & \emptyset \\
  \freenames{x?(y).P} & := & \{ x \} \cup (\freenames{P} \setminus \{ y \}) \\
  \freenames{x!\langle P \rangle} & := & \{ x \} \cup \{ P \} \\
  \freenames{P|Q} & := & \freenames{P} \cup \freenames{Q} \\
  \freenames{\dropn{x}} & := & \{ x \}
\end{eqnarray*}

The bound names of a process, $\boundnames{P}$, are those names occurring in $P$
that are not free. For example, in $x?(y).0$, the name $x$ is free, while $y$ is bound.

\begin{mathpar}
  \inferrule* [lab=monoidal-laws] {} { P|Q \equiv Q|P \and P|0 \equiv P \and P|(Q|R) \equiv (P|Q)|R }
\end{mathpar}

\begin{mathpar}
  \inferrule* [lab=alpha-equivalence] {} { (x)P \equiv (y)P\{y/x\} \and y \not\in \freenames{P} }
\end{mathpar}

\begin{definition}
Then two processes, $P,Q$, are alpha-equivalent if $P = Q\{\vec{y}/\vec{x}\}$ for
some $\vec{x} \in \boundnames{Q},\vec{y} \in \boundnames{P}$, where $Q\{\vec{y}/\vec{x}\}$
denotes the capture-avoiding substitution of $\vec{y}$ for $\vec{x}$ in $Q$.
\end{definition}

\begin{definition}
  The {\em structural congruence} \cite{SangiorgiWalker} , $\equiv$,
  between processes is the least congruence containing
  alpha-equivalence, satisfying the abelian monoid laws
  (associativity, commutativity and $\pzero$ as identity) for parallel
  composition $|$ and for summation $+$.
\end{definition}

\subsection{Name equivalence}

We take name equivalence, written $\nameeq$, to be the smallest
equivalence relation generated by the following rules.

\begin{mathpar}
\inferrule*[lab=Quote-drop]
{ }
{ \quotep{@{x}} \nameeq x }

\inferrule*[lab=Struct-equiv]
{ P \scong Q }
{ \quotep{P} \nameeq \quotep{Q} }
\end{mathpar}

The astute reader will have noticed that the mutual recursion of names
and processes imposes a mutual recursion on alpha-equivalence and
structural equivalence via name-equivalence. Fortunately, all of this
works out pleasantly and we may calculate in the natural way, free of
concern. The reader interested in the details is referred to the
appendix \ref{appendix:rho_details}.

\subsection{Substitution}

We use $\Proc$ for the set of processes, $\QProc$ for the set of
names, and $\id{\{}\vec{y} / \vec{x} \id{\}}$ to denote partial maps,
$s : \QProc \rightarrow \QProc$. A map, $s$ lifts, uniquely, to a map
on process terms, $\widehat{s} : \Proc \rightarrow \Proc$ by the
following equations.

\begin{mathpar}
  (0) \psubstp{Q}{P} := 0 \\
  (R \juxtap S) \psubstp{Q}{P}
  :=    
  (R)\psubstp{Q}{P} \juxtap (S) \psubstp{Q}{P} \\
  (x?(y).R) \psubstp{Q}{P}    
  :=    
  (x)\substp{Q}{P} (z)\concat( (R \psubstn{z}{y}) \psubstp{Q}{P} ) \\
  (\lift{x}{R}) \psubstp{Q}{P}  
  :=
  \lift{(x)\substp{Q}{P}}{ R \psubstp{Q}{P} } \\
%   (\dropn{x})  \psubstp{Q}{P}       
%   := 
%   \left\{ 
%     \begin{array}{ccc} 
%       \dropn{\quotep{Q}} & & x \nameeq \quotep{P} \\
%       \dropn{x} & & otherwise \\
%     \end{array}
%   \right. 
  (\dropn{x})  \psubstp{Q}{P}       
  := 
  \left\{ 
    \begin{array}{ccc} 
      Q & & x \nameeq \quotep{P} \\
      \dropn{x} & & otherwise \\
    \end{array}
  \right.
\end{mathpar}
 

where

\begin{eqnarray}
  (x)\id{\{} \lpquote Q \rpquote / \lpquote P \rpquote \id{\}}            = 
  \left\{ 
    \begin{array}{ccc}
      \lpquote Q \rpquote & & x \nameeq \lpquote P \rpquote \\
      x & & otherwise \\
    \end{array}
  \right. \nonumber
\end{eqnarray}

and $z$ is chosen distinct from $\quotep{P}$, $\quotep{Q}$, the free
names in $Q$, and all the names in $R$. Our $\alpha$-equivalence will
be built in the standard way from this substitution.

\begin{remark}\label{rem:no_self_referential_names}
  One consequence of these definitions is that $\forall P. \quotep{P}
  \not\in \freenames{P}$.
\end{remark}

\subsection{ Dynamic quote: an example }

Anticipating something of what's to come, consider applying the
substitution, $\widehat{\id{\{}u / z \id{\}}}$, to the following pair
of processes, $\lift{w}{y!(z)}$ and $w[ \lpquote y!(z) \rpquote ]$.

\begin{eqnarray}
	\lift{w}{y!(z)}\widehat{\id{\{}u / z \id{\}}}
		& = &
		\lift{w}{y!(u)} \nonumber\\
	w[ \lpquote y!(z) \rpquote ] \widehat{ \id{\{}u / z \id{\}} }
		& = &
		w[ \lpquote y!(z) \rpquote ] \nonumber
\end{eqnarray}

Because the body of the process between quotes is impervious to
substitution, we get radically different answers. In fact, by
examining the first process in an input context,
e.g. $x?(z).\lift{w}{y!(z)}$, we see that the process under the lift
operator may be shaped by prefixed inputs binding a name inside it. In
this sense, the lift operator will be seen as a way to dynamically
construct processes before reifying them as names.

Finally equipped with these standard features we can present the
dynamics of the calculus.

\subsubsection{Operational semantics} 

Finally, we introduce the computational dynamics. What marks these
algebras as distinct from other more traditionally studied algebraic
structures, e.g. vector spaces or polynomial rings, is the manner in
which dynamics is captured. In traditional structures, dynamics is typically
expressed through morphisms between such structures, as in linear maps
between vector spaces or morphisms between rings. In algebras
associated with the semantics of computation, the dynamics is
expressed as part of the algebraic structure itself, through a
reduction reduction relation typically denoted by $\red$. Below, we
give a recursive presentation of this relation for the calculus used
in the encoding.

$\red \subseteq \pi \times \pi$
$\red : \pi \to \mathcal{P}(\pi)$

\begin{mathpar}
  \inferrule* [lab=Comm] { \textsf{match}( x_{src}, x_{trgt} ) } { x_{trgt}?(y)P \; | \; x_{src}!\langle {Q} \rangle \red P\{\quotep{Q}/y}\} }
  \and \\
  \inferrule* [lab=Par] {{P} \red {P}'} {{{P} | {Q}} \red {{P}' | {Q}}}
  \and
  \inferrule* [lab=Equiv]{{{P} \scong {P}'} \andalso {{P}' \red {Q}'} \andalso {{Q}' \scong {Q}}}{{P} \red {Q}}
\end{mathpar}

\begin{eqnarray*}
  match_{\equiv} (\quotep{P},\quotep{Q}) & := & P \equiv Q \\
  match_{\dagger}(\quotep{P},\quotep{Q}) & := & \forall R. P|Q \red^{*} R => R \red^{*} 0 \\
  match_{K}(\quotep{P},\quotep{Q}) & := & K \mbox{ for some context } K
\end{eqnarray*}

$u?(x)P | u!\langle Q \rangle \red P\{\quotep{Q}/x\}$

%We write $\wred$ for $\red^*$, and $P\red$ if $\exists Q $ such that $ P \red Q$.
We write $P\red$ if $\exists Q $ such that $ P \red Q$ and $P\not\red$, otherwise.

\section{Replication}

As mentioned before, it is known that replication (and hence
recursion) can be implemented in a higher-order process algebra
\cite{SangiorgiWalker}. As our first example of calculation with the
machinery thus far presented we give the construction explicitly in
the {\rhoc}.

\begin{eqnarray}
	D_{x} & := & \prefix{x}{y}{(\binpar{\outputp{x}{y}}{@{y}})} \nonumber\\
	\bangp_{x}{P} & := & \binpar{{x}!\langle{\binpar{D_{x}}{P}}\rangle}{D_{x}} \nonumber
\end{eqnarray}

\begin{eqnarray}
	\bangp_{x}{P} & & \nonumber\\
	=
	& {x}!\langle{(\prefix{x}{y}{(\outputp{x}{y} | @{y})) | P}}\rangle 
	      | \prefix{x}{y}{(\outputp{x}{y} | @{y})} & \nonumber\\
	\red
	& (\outputp{x}{y} | @{y})\substn{\quotep{(\prefix{x}{y}{(@{y} | \outputp{x}{y})) | P}}}{y} & \nonumber\\
	=
	& \outputp{x}{\quotep{(\prefix{x}{y}{(\outputp{x}{y} | @{y})) | P}}}
	  | {(\prefix{x}{y}{(\outputp{x}{y} | @{y})) | P}} & \nonumber\\
	\red
	& \ldots & \nonumber\\
	\red^*
	& P | P | \ldots & \nonumber
\end{eqnarray}

Of course, this encoding, as an implementation, runs away, unfolding
$\bangp{P}$ eagerly. A lazier and more implementable replication
operator, restricted to input-guarded processes, may be obtained as follows.

\begin{eqnarray}
\bangp{\prefix{u}{v}{P}} 
	:= 
	\binpar{\lift{x}{\prefix{u}{v}{(\binpar{D(x)}{P})}}}{D(x)} \nonumber
\end{eqnarray}

\begin{remark}
  Note that the lazier definition still does not deal with summation
  or mixed summation (i.e. sums over input and output). The reader is
  invited to construct definitions of replication that deal with these
  features. 

  Further, the definitions are parameterized in a name, $x$. Can you,
  gentle reader, make a definition that eliminates this parameter and
  guarantees no accidental interaction between the replication
  machinery and the process being replicated -- i.e. no accidental
  sharing of names used by the process to get its work done and the
  name(s) used by the replication to effect copying. This latter
  revision of the definition of replication is crucial to obtaining
  the expected identity $!!P \sim !P$.
\end{remark}

\begin{remark}\label{rem:paradoxical_combinator}
  The reader familiar with the lambda calculus will have noticed the
  similarity between $D$ and the paradoxical combinator.

  [Ed. note: the existence of this seems to suggest we have to be more
  restrictive on the set of processes and names we admit if we are to
  support no-cloning.]
\end{remark}

\subsubsection{Bisimulation}

The computational dynamics gives rise to another kind of equivalence,
the equivalence of computational behavior. As previously mentioned
this is typically captured \emph{via} some form of bisimulation.

% The notion we use in this paper is weak barbed bisimulation
% \cite{milner91polyadicpi}.

The notion we use in this paper is derived from weak barbed
bisimulation \cite{milner91polyadicpi}. 

\begin{definition}
An \emph{observation relation}, $\downarrow_{\mathcal N}$, over a set
of names, $\mathcal N$, is the smallest relation satisfying the rules
below.

\infrule[Out-barb]{y \in {\mathcal N}, \; x \nameeq y}
		  {\outputp{x}{v} \downarrow_{\mathcal N} x}
\infrule[Par-barb]{\mbox{$P\downarrow_{\mathcal N} x$ or $Q\downarrow_{\mathcal N} x$}}
		  {\binpar{P}{Q} \downarrow_{\mathcal N} x}

We write $P \Downarrow_{\mathcal N} x$ if there is $Q$ such that 
$P \wred Q$ and $Q \downarrow_{\mathcal N} x$.
\end{definition}

\begin{definition}
%\label{def.bbisim}
An  ${\mathcal N}$-\emph{barbed bisimulation} over a set of names, ${\mathcal N}$, is a symmetric binary relation 
${\mathcal S}_{\mathcal N}$ between agents such that $P\rel{S}_{\mathcal N}Q$ implies:
\begin{enumerate}
\item If $P \red P'$ then $Q \wred Q'$ and $P'\rel{S}_{\mathcal N} Q'$.
\item If $P\downarrow_{\mathcal N} x$, then $Q\Downarrow_{\mathcal N} x$.
\end{enumerate}
$P$ is ${\mathcal N}$-barbed bisimilar to $Q$, written
$P \wbbisim_{\mathcal N} Q$, if $P \rel{S}_{\mathcal N} Q$ for some ${\mathcal N}$-barbed bisimulation ${\mathcal S}_{\mathcal N}$.
\end{definition}

$\mathcal{R} \subseteq \pi \times \pi$

$P \mathcal{R} Q => \forall P'. P \red P' \Rightarrow \exists Q'. Q \red Q', P' \mathcal{R} Q'$

$P \vdash x \Rightarrow Q \vdash x$

\begin{mathpar}
  \inferrule*[lab=Out-barb]{x \nameeq y}{{y}!\langle{Q}\rangle \vdash x}
  \and
  \inferrule*[lab=Par-barb]{\mbox{$P\vdash x$ or $Q\vdash x$}}{\binpar{P}{Q} \vdash x}
\end{mathpar}

\subsubsection{Contexts}

One of the principle advantages of computational calculi like the
$\pi$-calculus is a well-defined notion of context,
contextual-equivalence and a correlation between
contextual-equivalence and notions of bisimulation. The notion of
context allows the decomposition of a process into (sub-)process and
its syntactic environment, its context. Thus, a context may be
thought of as a process with a ``hole'' (written $\Box$) in it. The
application of a context $M$ to a process $P$, written $M[P]$, is
tantamount to filling the hole in $M$ with $P$. In this paper we do
not need the full weight of this theory, but do make use of the notion
of context in the proof the main theorem. 

\begin{mathpar}
  \inferrule* [lab=summation] {} {{M_{M},M_{N}} \bc \Box \;|\; x.M_{A} \;|\; M_{M}+M_{N}}
  \and
  \inferrule* [lab=agent] {} {{M_{A}} \bc (\vec{x})M_{P} \;| \; \clift{P_0,\ldots,M_{P},\ldots,P_N}}
  \and \\
  \inferrule* [lab=process] {} {{M_{P}} \bc M_{N} \;| \;P|M_{P} }
\end{mathpar} 

\begin{mathpar}
  \inferrule* [lab=sychronization] {} {M_{N} \bc \Box \;|\; x?M_{F} \;|\; x!M_{C}}
  \and
  \inferrule* [lab=abstraction] {} {{M_{F}} \bc (x)M_{P} }
  \and
  \inferrule* [lab=concretion] {} {{M_{C}} \bc \langle M_{P} \rangle }
  \and \\
  \inferrule* [lab=process] {} {{M_{P}} \bc M_{N} \;| \;P|M_{P} }
\end{mathpar}

\begin{definition}[contextual application] Given a context $M$, and
  process $P$, we define the \emph{contextual application}, $M[P] :=
  M\{P/\Box\}$. That is, the contextual application of M to P is the
  substitution of $P$ for $\Box$ in $M$.
\end{definition}

$\meaningof{-} : L \to \mathcal{P}(\pi)$

\begin{mathpar}
  \inferrule* [lab=collection] {} {\meaningof{true} = \pi, \and \meaningof{~E} = \pi \setminus \meaningof{E}, \and \meaningof{E_{1} \& E_{2}} = \meaningof{E_{1}} \cap \meaningof{E_{2}}}
\end{mathpar}

\begin{mathpar}
  \inferrule* [lab=structure] {} {\meaningof{0} = \{ P \in \pi | P \equiv 0 \}, \and \\ \meaningof{E_1 | E_2} = \{ P \in \pi | P \equiv P_{1} | P_{2}, P_{1} \in \meaningof{E_{1}}, P_{2} \in \meaningof{E_2}\} }
\end{mathpar}

\begin{mathpar}
 \inferrule* [lab=behavior] {} {\meaningof{\langle a?b \rangle E} = \{ P \in \pi | P \equiv Q | u?(y)P', \\ \and \\\\ \and \\ \;\;\; u \in \meaningof{a}, \forall z.P'\{z/y\} \in \meaningof{E\{z/b\}}\}, \and \\ \meaningof{a!E} = \{ P \in \pi | P \equiv Q | x!\langle P' \rangle, x \in \meaningof{a} P' \in \meaningof{E}\} }
\end{mathpar}

\begin{mathpar}
 \inferrule* [lab=nominal] {} {\meaningof{\quotep{E}} = \{ \quotep{P} \in \quotep{\pi} | P \in \meaningof{E} \}, \and \meaningof{\quotep{P}} = \{ \quotep{Q} \in \quotep{\pi} | P \equiv Q \} \and \\ \meaningof{@\quotep{E}} = \{ P \in \pi | P \equiv @x, x \in \meaningof{E} \}}
\end{mathpar}

\begin{eqnarray*}
  \\
  \meaningof{-} : TS \to ST
\end{eqnarray*}

\begin{eqnarray*}
  \\
  L : TS \to ST
\end{eqnarray*}

\begin{eqnarray*}
  \\
  P \models E \iff P \in \meaningof{E}
\end{eqnarray*}

\begin{eqnarray*}
  P \approx_{L} Q \iff \forall E \in L. P \models E \iff Q \models E
\end{eqnarray*}

\begin{eqnarray*}
  P \approx_{K} Q
\end{eqnarray*}

\begin{eqnarray*}
  P \approx Q
\end{eqnarray*}

$\approx_{K} = \approx = \approx_{L}$

\subsubsection{Contextual duality}

Note that contexts extend the quotation operation to a family of
operations from processes to names. Given a context, $M$, we can
define a \emph{nominal context}, $\quotep{M}$ by $\quotep{M}[P] :=
\quotep{M[P]}$. To foreshadow what is to come we observe that these
operations enjoy a duality with processes very much like the duality
between vectors and maps from vectors to scalars.

Further, because the calculus is essentially higher-order, we have a
correspondence between contexts and processes. More specifically,
given a name $x$ and a context $M$ we can construct $M^{*}_{x}$ such
that 

\begin{mathpar}
  M^{*}_{x} | \lift{x}{P} \red M[P]
\end{mathpar}

namely,

\begin{mathpar}
  M^{*}_{x} := x?(u).M[\dropn{u}]
\end{mathpar}

The dependence of $M^{*}_{x}$ on a name makes it an abstraction, 

\begin{mathpar}
  M^{*} := (x)x?(u).M[\dropn{u}]
\end{mathpar}

\subsection{Additional notation}

It will sometimes be convenient to denote the process a name
quotes. We already have the notation $x = \quotep{P}$, but it will be
convenient to introduce an alternate notation, $\procn{x}$, when we
want to emphasize the connection to the use of the name. Note that, by
virtue of name equivalence, $\quotep{\procn{x}} \nameeq x$; so, the
notation is consistent with previous definitions.

Further, because names have structure it is possible to effect
substitutions on the basis of that structure. This means we need to
upgrade our notation for substitutions, which we accomplish by
adapting comprehension notation. Thus,

\begin{mathpar}
  P\{ y / x : x \in S \}
\end{mathpar}

is interpreted to mean the process derived from P by replacing (in a
capture-avoiding manner) each occurrence of $x$ in $S$ by $y$. For example,

\begin{mathpar}
  P\{ \quotep{\procn{x}|\procn{x}} / x : x \in \freenames{P} \}
\end{mathpar}

will replace each (occurrence) of a free name $x$ in $P$ by
$\quotep{\procn{x}|\procn{x}}$.

Also, we will avail ourselves of the notation $x^{L}$ and $x^{R}$ to
denote injections of a name into disjoint copies of the name
space. There are numerous ways to accomplish this. One example can be
found in \cite{MeredithR05}. This notation overloads to vectors of
names: $\vec{x}^{\pi} := (x_{i}^{\pi} \; : \; 0 \leq i < |\vec{x}| )$ where $\pi \in \{L,R\}$.

We also use $P^{\Box} := P|\Box$.

In \cite{MeredithR05} an interpretation of the new operator is
given. It turns out that there are several possible interpretations
all enjoying the requisite algebraic properties of the operator (see
\cite{milner91polyadicpi}). We will therefore make liberal use of
$(\nu\; \vec{x})P$.

% subsection the_syntax_and_semantics_of_the_notation_system (end)   

\section{Interpretation of QM}
\subsection{Supporting definitions}
\subsubsection{Multiplication}
\begin{mathpar}
  \quotep{Q} \cdot \quotep{R} := \quotep{Q|R}
  \and \\
  \quotep{Q} \cdot P := P\{ \quotep{Q|R} / \quotep{R} : \quotep{R} \in \freenames{P} \}
\end{mathpar}

\paragraph{Discussion}
The first line needs little explanation. The second line says that
each free name of the process is replaced with the multiplication of
that name by the scalar. Multiplication of a scalar (name) by a state
(process) results in a process all the names of which have been `moved
over' by parallel composition with the process the scalar
quotes. There is a subtlety that the bound names have to be
manipulated so that multiplied names aren't accidentally
captured. There are many ways to achieve this.

\begin{remark}\label{rem:multiplication_identities}
  The reader is invited to verify that for all $x,y,z \in \QProc$ and $P \in \Proc$
  \begin{mathpar}
    x \cdot \quotep{0} \equiv x 
    \and
    x \cdot y \equiv y \cdot x
    \and
    x \cdot (y \cdot z) \equiv (x \cdot y) \cdot z
    \and \\
    \quotep{0} \cdot P \equiv P
    \and \\
    x \cdot (y \cdot P) \equiv (x \cdot y) \cdot P
    \and \\
    x \cdot (P|Q) \equiv (x \cdot P) | (x \cdot Q)
    \and \\    
  \end{mathpar}
\end{remark}

\subsubsection{Tensor product}

We define a tensor product on processes by structural induction.

\paragraph{Tensor of sums} First note that all summations, including
$\pzero$ and sequence, can be written $\Sigma_{i} x_{i}.A_{i} +
\Sigma_{j} x_{j}.C_{j}$, where we have grouped input-guarded processes
together and output-guarded processes together.

Thus, we can define the tensor product of two summations, $N_{1}\otimes N_{2}$, where

\begin{mathpar}
  N_{1} := \Sigma_{i} x_{i}.A_{i} + \Sigma_{j} x_{j}.C_{j}
  \and
  N_{2} := \Sigma_{i'} y_{i'}.B_{i'} + \Sigma_{j'} y_{j'}.D_{j'} 
\end{mathpar}

as follows.

\begin{mathpar}
  \Sigma_{i} x_{i}.A_{i} + \Sigma_{j} x_{j}.C_{j} \otimes \Sigma_{i'}
  y_{i'}.B_{i'} + \Sigma_{j'} y_{j'}.D_{j'} 
  \and \\
  := \; \Sigma_{i} \Sigma_{i'} \quotep{\stackrel{\vee}{x_{i}}| \stackrel{\vee}{y_{i'}}}.(A_{i}\otimes B_{i'}) \; | \; \Sigma_{i'} \Sigma_{i} \quotep{\stackrel{\vee}{y_{i'}}|\stackrel{\vee}{x_{i}}}.(B_{i'}\otimes A_{i})
  \and
  \;\; | \;\; \Sigma_{j} \Sigma_{j'} \quotep{\stackrel{\vee}{x_{j}}|\stackrel{\vee}{y_{j'}}}.(A_{j}\otimes B_{j'}) \; | \; \Sigma_{j'} \Sigma_{j} \quotep{\stackrel{\vee}{y_{j'}}|\stackrel{\vee}{x_{j}}}.(B_{j'}\otimes A_{j})
\end{mathpar}

\begin{remark}
  Do we need to $x^{L}$ and $y^{R}$ for this construction as well?
\end{remark}

\paragraph{Tensor of parallel compositions} Next, we distribute tensor
over par.

\begin{mathpar}
  P_{1}|P_{2} \otimes Q_{1}|Q_{2} := (P_{1} \otimes Q_{1}) | (P_{1}
  \otimes Q_{2}) | (P_{2} \otimes Q_{1}) | (P_{2} \otimes Q_{2})
\end{mathpar}

\paragraph{Tensor with dropped names} We treat tensor of a
process with a dropped name as parallel composition.

\begin{mathpar}
  P \otimes \dropn{x} := P | \dropn{x}
\end{mathpar}

\paragraph{Tensor of agents}

Finally, we need to define tensor on agents. Note that the definition
of tensor on normal products only tensors inputs with inputs and
outputs with outputs. Thus, we only have to define the operation on
``homogeneous'' pairings.

\begin{mathpar}
  (\vec{x})P \otimes (\vec{y})Q
  \and \\
  := (x_{0}^{L}|y_{0}^{R},\ldots,x_{0}^{L}|y_{n}^{R},\ldots,x_{m}^{L}|y_{0}^{R},\ldots,x_{m}^{L}|y_{n}^R)(P\{ \vec{x}^{L}/\vec{x}\} \otimes Q \{ \vec{y}^{R}/\vec{y}\})
  \and \\
  \clift{\vec{P}} \otimes \clift{\vec{Q}}
  \and \\
  := \clift{P_{0}\otimes Q_{0},\ldots,P_{0}\otimes Q_{n},\ldots,P_{m}\otimes Q_{0},\ldots,P_{m}\otimes Q_{n}}
\end{mathpar}

\begin{remark}
  Observe that arities of tensored abstractions matches arities of
  tensored concretions if the original arities matched. Note also that
  the length of the arities corresponds to the increase in dimension
  we see in ordinary vector space tensor product.
\end{remark}

\begin{remark}
  Operationally, this definition distributes the tensor down to
  components ``linked'' by summation. Tensor over summation is
  intriguing in that it mixes names. Moreover, as a consequence of the
  way it mixes names we have the identities for all $x \in \QProc$ and
  $P,Q \in \Proc$

  \begin{mathpar}
    (x \cdot P) \otimes Q \equiv x \cdot (P \otimes Q) \equiv P \otimes (x \cdot Q)
    \and
    P \otimes \pzero \equiv P
  \end{mathpar}

  that the reader is invited to verify.
\end{remark}

\subsubsection{Annihilation}
\begin{mathpar}
  P^{\perp} := \{ Q | \forall R. P|Q \red^{*} R \Rightarrow R \red^{*} \pzero \}
  \and \\
  P^{\underline{\perp}} := \Sigma_{Q \in P^{\perp}} \quotep{Q}?(y).(\dropn{y}|Q) | \Sigma_{Q \in P^{\perp}} \quotep{Q}\clift{\Box}
\end{mathpar}

\paragraph{Discussion} The reader will note that $P^{\perp}$ is a
\emph{set} of processes, while $P^{\underline{\perp}}$ is a
\emph{context}. We call the set $P^{\perp}$ the \emph{annihilators} of
$P$. The parallel composition of a process in the annihilators of $P$
with $P$ will result in a process, the state space of which has all
paths eventually leading to $\pzero$. Execution may endure loops; but
under reasonable conditions of fairness (naturally guaranteed under
most notions of bisimulation) such a composite process cannot get
stuck in such a loop and will, eventually pop out and terminate.

The context $P^{\underline{\perp}}$ is ready and willing to ``take the
$P$ out of'' the process to which it is applied. It will effectively
transmit the code of the process to which it is applied to one of the
annihilators and run the process against it.

\subsubsection{Evaluation}
We fix $M$ a domain of fully abstract interpretation with an equality
coincident with bisimulation. We take $\meaningof{\cdot} : \Proc \to
M$ to be the map interpreting processes and $\nmeaningof{\cdot} : \M
\to Proc$ to be the map running the other way. Then we define

\begin{mathpar}
  \int P := \nmeaningof{\meaningof{P}}
\end{mathpar}

\paragraph{Discussion}
There are many fully abstract interpretations of Milner's
$\pi$-calculus. Any of them can be used as a basis for interpreting
the reflective calculus here. Equipped with such a domain it is
largely a matter of grinding through to check that the Yoneda
construction for the normalization-by-evaluation program can be
extended to this setting.

\begin{remark}
  The reader is invited to verify that $\int (P^{\underline{\perp}}[P]) = 0$.
\end{remark}

\subsection{Quantum mechanics}

Table \ref{tbl:core_qm_op_defns} gives the core operational definitions

\begin{table}[htp]\label{tbl:core_qm_op_defns}
  \center{
    \fbox{
      \begin{tabular}{c|c}
        quantum mechanics & process calculus \\
        \hline
        scalar & $x := \quotep{P}$ \\
        state vector & $\state{P} := P$ \\
        dual & $\state{P}^{*} := \event{P^{\underline{\perp}}} := \quotep{P^{\underline{\perp}}}[-]$ \\
        matrix & $ \Sigma_{\alpha} \state{P_{\alpha}}x_{\alpha}\event{Q_{\alpha}}$ \\
        vector addition & $\state{P} + \state{Q} := \state{P | Q}$ \\
        tensor product & $\state{P} \otimes \state{Q} := \state{P \otimes Q}$ \\
        inner product & $\innerprod{P}{Q} := \quotep{\int P^{\underline{\perp}}[Q]}$ \\
      \end{tabular}
    }
  }
  \caption{QM - operational definitions}
\end{table}

where

\begin{mathpar}
  \prmatrix{P}{Q} := \fprmatrix{P}{\quotep{\pzero}}{Q}
  \and
  \fprmatrix{P}{x}{Q} := (\state{P},x,\event{Q})
  \and
  (\fprmatrix{P}{x}{Q})(\state{R}) := x \cdot \innerprod{Q}{R} \cdot \state{P}
  \and
  (\fprmatrix{P}{x}{Q})(\event{R}) := x \cdot \innerprod{R}{P} \cdot \event{Q}
\end{mathpar}

\paragraph{Discussion}
As promised: vectors (aka states) are represented as processes; duals
as contextual duals; inner product definition should be compared with
standard inner product definition for ....

\begin{remark}
  Assuming $\int (P^{\underline{\perp}}[P]) = 0$, the reader is
  invited to verify that $(\fprmatrix{P}{x}{P})(\state{P}) = x \cdot \state{P}$.
\end{remark}

\begin{remark}
  The reader is invited to verify that $\innerprod{P}{Q}$ could
  equally well have been written $\quotep{\int \stackrel{\vee}{x}}$
  where $x = \event{P^{\underline{\perp}}}(Q)$.

  One of the motivations for this remark is that there is another way
  to factor these operations. We could package up evaluation in the dual:

  \begin{mathpar}
    \state{P}^{*} := \event{\int P^{\underline{\perp}}} := \quotep{\int P^{\underline{\perp}}}[-]
  \end{mathpar}

  and then have inner product defined by
  
  \begin{mathpar}
    \innerprod{P}{Q} := \event{P}(Q)
  \end{mathpar}

  Hopefully, experience with the calculations will provide guidance on
  the best factoring.
\end{remark}

\begin{remark}
  Assuming $\int (P^{\underline{\perp}}[P]) = 0$, the reader is
  invited to verify that $\forall P,Q. (\prmatrix{0}{Q})(\state{0}) =
  \state{0}$ and dually $(\prmatrix{P}{0})(\event{0}) = \event{0}$.
\end{remark}

\begin{remark}
  i'm a little worried that i don't (yet) have proper support for
  complex conjugacy. But, the observation above may give us a
  clue. According to Abramsky, it must be the case that the scalars
  are iso to the homset of the identity for the tensor -- which the
  observation above characterizes. 

  For now, we will simply bookmark the notion with $\overline{x}$.
\end{remark}

\subsubsection{Adjointness}

We need to give a definition of $(\cdot)^{\dagger}$ for matrices. The
obvious candidate definition is
\begin{mathpar}
(\Sigma_{\alpha}\fprmatrix{P_{\alpha}}{x_{\alpha}}{Q_{\alpha}})^{\dagger}
= \Sigma_{\alpha}\fprmatrix{(Q_{\alpha}^{\underline{\perp}})^{*}}{\overline{x}_{\alpha}}{P_{\alpha}^{\underline{\perp}}} 
\end{mathpar}

But, $(Q_{\alpha}^{\underline{\perp}})^{*}$ requires a name along
which to communicate the process to achieve the context application.

\subsubsection{Basis for a basis}
If processes label states and ``addition'' of states (a.k.a. vector
addition) is interpreted as parallel composition, what corresponds to
notions of linear independence and basis? Here, we recall that Yoshida
has developed a set of \emph{combinators} for an asynchronous verison
of Milner's $\pi$-calculus. These are a finite set of processes such
any process can be expressed as parallel composition of these
combinators together with liberal uses of the new operator and
replication. We can simply give a translation of these into the
present calculus and have reasonable expectation that the property
carries over. That is, that the resultant set allows to express all
processes via parallel composition. Note, however, that there is no
new operator or replication in this calculus. As a result, we expect
that the corresponding set is actually infinite. That is, we expect
that the space is actually infinite dimensional.

\begin{remark}
  The attentive reader may be a bit concerned. Certainly, the
  collection $S$, $K$ and $I$ is a finite set of
  combinators. Shouldn't we expect to see a finite set of combinators
  for an effectively equivalent system? i am very sympathetic to this
  critique and feel it warrants full attention. On the other hand, i
  also have in mind the following analogy. The natural numbers, as a
  monoid under addition, has exactly $1$ generator, while the natural
  numbers, as a monoid under multiplication, has countably many
  generators (the primes). We observe that the application of the
  lambda calculus is much less resource sensitive than the parallel
  composition of the $\pi$-calculus. Could it be the case that we have
  an analogy of the form
  
  \begin{mathpar}
    m + n : MN :: m*n : M|N
  \end{mathpar}

  giving a similar blow up in the set of ``primes''?  This is such a
  wonderful thought that, even if it's not true, i think it's worth
  writing down.
\end{remark}
 

\documentclass[12pt]{llncs}
%\documentclass{jktr}

\usepackage[pdftex]{hyperref}                   
\usepackage {listings}
\usepackage {mathpartir}
\usepackage{bcprules}
%\usepackage{listings}
                       
\usepackage{graphicx} 
%\usepackage[margins=2.5cm,nohead,nofoot]{geometry}
%\usepackage{geometry}
\usepackage{amsfonts}
\usepackage{amstext}
\usepackage{latexsym}
\usepackage{amssymb}
\usepackage{color}


%\include{myPreamble}
\include{qm2pi.local} 

%\ifpdf
%\usepackage[pdftex]{graphicx}
%\else
%\usepackage{graphicx}
%\fi

 % \ifpdf
%  \usepackage{pdfsync}
%  \if


%\title{Brief Article}
%\author{David F. Snyder}
%\author{L.G. Meredith}

%\address{Dept. of Math., Texas State University--San Marcos, San Marcos, TX 78666}
       
\pagestyle{empty}


\begin{document}

\lstset{language=[Objective]Caml,frame=shadowbox}

\input{qm2pi.front}

% section front matter (end)

\input{qm2pi.intro} 
 
% section introduction (end)

% \input{qm2pi.knotations} 

% section notation (end)

\input{qm2pi.process.calculi} 

% section concurrent_process_calculi_and_spatial_logics_ (end)
    
%\input{qm2pi.knots2pi} 

%\input{qm2pi.trefoil} 

%\input{qm2pi.mainthm} 

% subsection basic_interpretation (end)

%\input{qm2pi.rho.presentation} 
\subsection{The syntax and semantics of the notation system}\label{sub:the_syntax_and_semantics_of_the_notation_system} % (fold)

We now summarize a technical presentation of the calculus that
embodies our theory of dynamics. The typical presentation of such a
calculus follows the style of giving generators and relations on
them. The grammar, below, describing term constructors, freely
generates the set of processes, $\Proc$. This set is then quotiented
by a relation known as structural congruence and it is over this set
that the notion of dynamics is expressed. This presentation is
essentially that of \cite{MeredithR05} with the addition of
polyadicity and summation. For readability we have relegated some of
the technical subtleties to an appendix.

\subsubsection{Process grammar}\label{subsub:process_grammar}

\begin{mathpar}
  \inferrule* [lab=synchronization] {} {{M} \bc \pzero \;|\; x?F \;|\; x!C }
  \and
  \inferrule* [lab=abstraction] {} {{F} \bc (x)P}
  \and
  \inferrule* [lab=concretion] {} {{C} \bc \langle Q \rangle}
  \and
  \inferrule* [lab=process] {} {{P,Q} \bc M \;| \;P|Q \;|\; @{x}}
  \and
  \inferrule* [lab=name] {} {{x} \bc \quotep{P}}
\end{mathpar} 

Note that $\vec{x}$ (resp. $\vec{P}$) denotes a vector of names
(resp. processes) of length $|\vec{x}|$ (resp. $|\vec{P}|$). We adopt
the following useful abbreviations.

\begin{mathpar}
   x?(\vec{y}).P := x.(\vec{y})P \and  x\clift{\vec{P}} := x.\clift{\vec{P}}
   \and x!(y) := \lift{x}{\dropn{y}}
   \and \Pi_{i=0}^{n-1}P_i := P_0 | \ldots | P_{n-1}
\end{mathpar}

\subsubsection{Structural congruence}

\paragraph{Free and bound names and alpha-equivalence.} At the
core of structural equivalence is alpha-equivalence which identifies
process that are the same up to a change of variable. Formally, we
recognize the distinction between free and bound names. The free names
of a process, $\freenames{P}$, may be calculated recursively as
follows:

\begin{mathpar}
\freenames{\pzero} := \emptyset
  \and \\
  \freenames{x?(y).P} := \{ x \} \cup (\freenames{P} \setminus \{ y \})
  \and 
  \freenames{x!\langle P \rangle} := \{ x \} \cup \{ P \} 
  \and \\
  \freenames{P|Q} := \freenames{P} \cup \freenames{Q}
  \and \\
  \freenames{@{x}} := \{ x \}
\end{mathpar}

$\pi$
$\quotep{\pi}$

$\freenames{-} : \pi \to \mathcal{P}(\quotep{\pi})$

\begin{eqnarray*}
  \freenames{\pzero} & := & \emptyset \\
  \freenames{x?(y).P} & := & \{ x \} \cup (\freenames{P} \setminus \{ y \}) \\
  \freenames{x!\langle P \rangle} & := & \{ x \} \cup \{ P \} \\
  \freenames{P|Q} & := & \freenames{P} \cup \freenames{Q} \\
  \freenames{\dropn{x}} & := & \{ x \}
\end{eqnarray*}

The bound names of a process, $\boundnames{P}$, are those names occurring in $P$
that are not free. For example, in $x?(y).0$, the name $x$ is free, while $y$ is bound.

\begin{mathpar}
  \inferrule* [lab=monoidal-laws] {} { P|Q \equiv Q|P \and P|0 \equiv P \and P|(Q|R) \equiv (P|Q)|R }
\end{mathpar}

\begin{mathpar}
  \inferrule* [lab=alpha-equivalence] {} { (x)P \equiv (y)P\{y/x\} \and y \not\in \freenames{P} }
\end{mathpar}

\begin{definition}
Then two processes, $P,Q$, are alpha-equivalent if $P = Q\{\vec{y}/\vec{x}\}$ for
some $\vec{x} \in \boundnames{Q},\vec{y} \in \boundnames{P}$, where $Q\{\vec{y}/\vec{x}\}$
denotes the capture-avoiding substitution of $\vec{y}$ for $\vec{x}$ in $Q$.
\end{definition}

\begin{definition}
  The {\em structural congruence} \cite{SangiorgiWalker} , $\equiv$,
  between processes is the least congruence containing
  alpha-equivalence, satisfying the abelian monoid laws
  (associativity, commutativity and $\pzero$ as identity) for parallel
  composition $|$ and for summation $+$.
\end{definition}

\subsection{Name equivalence}

We take name equivalence, written $\nameeq$, to be the smallest
equivalence relation generated by the following rules.

\begin{mathpar}
\inferrule*[lab=Quote-drop]
{ }
{ \quotep{@{x}} \nameeq x }

\inferrule*[lab=Struct-equiv]
{ P \scong Q }
{ \quotep{P} \nameeq \quotep{Q} }
\end{mathpar}

The astute reader will have noticed that the mutual recursion of names
and processes imposes a mutual recursion on alpha-equivalence and
structural equivalence via name-equivalence. Fortunately, all of this
works out pleasantly and we may calculate in the natural way, free of
concern. The reader interested in the details is referred to the
appendix \ref{appendix:rho_details}.

\subsection{Substitution}

We use $\Proc$ for the set of processes, $\QProc$ for the set of
names, and $\id{\{}\vec{y} / \vec{x} \id{\}}$ to denote partial maps,
$s : \QProc \rightarrow \QProc$. A map, $s$ lifts, uniquely, to a map
on process terms, $\widehat{s} : \Proc \rightarrow \Proc$ by the
following equations.

\begin{mathpar}
  (0) \psubstp{Q}{P} := 0 \\
  (R \juxtap S) \psubstp{Q}{P}
  :=    
  (R)\psubstp{Q}{P} \juxtap (S) \psubstp{Q}{P} \\
  (x?(y).R) \psubstp{Q}{P}    
  :=    
  (x)\substp{Q}{P} (z)\concat( (R \psubstn{z}{y}) \psubstp{Q}{P} ) \\
  (\lift{x}{R}) \psubstp{Q}{P}  
  :=
  \lift{(x)\substp{Q}{P}}{ R \psubstp{Q}{P} } \\
%   (\dropn{x})  \psubstp{Q}{P}       
%   := 
%   \left\{ 
%     \begin{array}{ccc} 
%       \dropn{\quotep{Q}} & & x \nameeq \quotep{P} \\
%       \dropn{x} & & otherwise \\
%     \end{array}
%   \right. 
  (\dropn{x})  \psubstp{Q}{P}       
  := 
  \left\{ 
    \begin{array}{ccc} 
      Q & & x \nameeq \quotep{P} \\
      \dropn{x} & & otherwise \\
    \end{array}
  \right.
\end{mathpar}
 

where

\begin{eqnarray}
  (x)\id{\{} \lpquote Q \rpquote / \lpquote P \rpquote \id{\}}            = 
  \left\{ 
    \begin{array}{ccc}
      \lpquote Q \rpquote & & x \nameeq \lpquote P \rpquote \\
      x & & otherwise \\
    \end{array}
  \right. \nonumber
\end{eqnarray}

and $z$ is chosen distinct from $\quotep{P}$, $\quotep{Q}$, the free
names in $Q$, and all the names in $R$. Our $\alpha$-equivalence will
be built in the standard way from this substitution.

\begin{remark}\label{rem:no_self_referential_names}
  One consequence of these definitions is that $\forall P. \quotep{P}
  \not\in \freenames{P}$.
\end{remark}

\subsection{ Dynamic quote: an example }

Anticipating something of what's to come, consider applying the
substitution, $\widehat{\id{\{}u / z \id{\}}}$, to the following pair
of processes, $\lift{w}{y!(z)}$ and $w[ \lpquote y!(z) \rpquote ]$.

\begin{eqnarray}
	\lift{w}{y!(z)}\widehat{\id{\{}u / z \id{\}}}
		& = &
		\lift{w}{y!(u)} \nonumber\\
	w[ \lpquote y!(z) \rpquote ] \widehat{ \id{\{}u / z \id{\}} }
		& = &
		w[ \lpquote y!(z) \rpquote ] \nonumber
\end{eqnarray}

Because the body of the process between quotes is impervious to
substitution, we get radically different answers. In fact, by
examining the first process in an input context,
e.g. $x?(z).\lift{w}{y!(z)}$, we see that the process under the lift
operator may be shaped by prefixed inputs binding a name inside it. In
this sense, the lift operator will be seen as a way to dynamically
construct processes before reifying them as names.

Finally equipped with these standard features we can present the
dynamics of the calculus.

\subsubsection{Operational semantics} 

Finally, we introduce the computational dynamics. What marks these
algebras as distinct from other more traditionally studied algebraic
structures, e.g. vector spaces or polynomial rings, is the manner in
which dynamics is captured. In traditional structures, dynamics is typically
expressed through morphisms between such structures, as in linear maps
between vector spaces or morphisms between rings. In algebras
associated with the semantics of computation, the dynamics is
expressed as part of the algebraic structure itself, through a
reduction reduction relation typically denoted by $\red$. Below, we
give a recursive presentation of this relation for the calculus used
in the encoding.

$\red \subseteq \pi \times \pi$
$\red : \pi \to \mathcal{P}(\pi)$

\begin{mathpar}
  \inferrule* [lab=Comm] { \textsf{match}( x_{src}, x_{trgt} ) } { x_{trgt}?(y)P \; | \; x_{src}!\langle {Q} \rangle \red P\{\quotep{Q}/y}\} }
  \and \\
  \inferrule* [lab=Par] {{P} \red {P}'} {{{P} | {Q}} \red {{P}' | {Q}}}
  \and
  \inferrule* [lab=Equiv]{{{P} \scong {P}'} \andalso {{P}' \red {Q}'} \andalso {{Q}' \scong {Q}}}{{P} \red {Q}}
\end{mathpar}

\begin{eqnarray*}
  match_{\equiv} (\quotep{P},\quotep{Q}) & := & P \equiv Q \\
  match_{\dagger}(\quotep{P},\quotep{Q}) & := & \forall R. P|Q \red^{*} R => R \red^{*} 0 \\
  match_{K}(\quotep{P},\quotep{Q}) & := & K \mbox{ for some context } K
\end{eqnarray*}

$u?(x)P | u!\langle Q \rangle \red P\{\quotep{Q}/x\}$

%We write $\wred$ for $\red^*$, and $P\red$ if $\exists Q $ such that $ P \red Q$.
We write $P\red$ if $\exists Q $ such that $ P \red Q$ and $P\not\red$, otherwise.

\section{Replication}

As mentioned before, it is known that replication (and hence
recursion) can be implemented in a higher-order process algebra
\cite{SangiorgiWalker}. As our first example of calculation with the
machinery thus far presented we give the construction explicitly in
the {\rhoc}.

\begin{eqnarray}
	D_{x} & := & \prefix{x}{y}{(\binpar{\outputp{x}{y}}{@{y}})} \nonumber\\
	\bangp_{x}{P} & := & \binpar{{x}!\langle{\binpar{D_{x}}{P}}\rangle}{D_{x}} \nonumber
\end{eqnarray}

\begin{eqnarray}
	\bangp_{x}{P} & & \nonumber\\
	=
	& {x}!\langle{(\prefix{x}{y}{(\outputp{x}{y} | @{y})) | P}}\rangle 
	      | \prefix{x}{y}{(\outputp{x}{y} | @{y})} & \nonumber\\
	\red
	& (\outputp{x}{y} | @{y})\substn{\quotep{(\prefix{x}{y}{(@{y} | \outputp{x}{y})) | P}}}{y} & \nonumber\\
	=
	& \outputp{x}{\quotep{(\prefix{x}{y}{(\outputp{x}{y} | @{y})) | P}}}
	  | {(\prefix{x}{y}{(\outputp{x}{y} | @{y})) | P}} & \nonumber\\
	\red
	& \ldots & \nonumber\\
	\red^*
	& P | P | \ldots & \nonumber
\end{eqnarray}

Of course, this encoding, as an implementation, runs away, unfolding
$\bangp{P}$ eagerly. A lazier and more implementable replication
operator, restricted to input-guarded processes, may be obtained as follows.

\begin{eqnarray}
\bangp{\prefix{u}{v}{P}} 
	:= 
	\binpar{\lift{x}{\prefix{u}{v}{(\binpar{D(x)}{P})}}}{D(x)} \nonumber
\end{eqnarray}

\begin{remark}
  Note that the lazier definition still does not deal with summation
  or mixed summation (i.e. sums over input and output). The reader is
  invited to construct definitions of replication that deal with these
  features. 

  Further, the definitions are parameterized in a name, $x$. Can you,
  gentle reader, make a definition that eliminates this parameter and
  guarantees no accidental interaction between the replication
  machinery and the process being replicated -- i.e. no accidental
  sharing of names used by the process to get its work done and the
  name(s) used by the replication to effect copying. This latter
  revision of the definition of replication is crucial to obtaining
  the expected identity $!!P \sim !P$.
\end{remark}

\begin{remark}\label{rem:paradoxical_combinator}
  The reader familiar with the lambda calculus will have noticed the
  similarity between $D$ and the paradoxical combinator.

  [Ed. note: the existence of this seems to suggest we have to be more
  restrictive on the set of processes and names we admit if we are to
  support no-cloning.]
\end{remark}

\subsubsection{Bisimulation}

The computational dynamics gives rise to another kind of equivalence,
the equivalence of computational behavior. As previously mentioned
this is typically captured \emph{via} some form of bisimulation.

% The notion we use in this paper is weak barbed bisimulation
% \cite{milner91polyadicpi}.

The notion we use in this paper is derived from weak barbed
bisimulation \cite{milner91polyadicpi}. 

\begin{definition}
An \emph{observation relation}, $\downarrow_{\mathcal N}$, over a set
of names, $\mathcal N$, is the smallest relation satisfying the rules
below.

\infrule[Out-barb]{y \in {\mathcal N}, \; x \nameeq y}
		  {\outputp{x}{v} \downarrow_{\mathcal N} x}
\infrule[Par-barb]{\mbox{$P\downarrow_{\mathcal N} x$ or $Q\downarrow_{\mathcal N} x$}}
		  {\binpar{P}{Q} \downarrow_{\mathcal N} x}

We write $P \Downarrow_{\mathcal N} x$ if there is $Q$ such that 
$P \wred Q$ and $Q \downarrow_{\mathcal N} x$.
\end{definition}

\begin{definition}
%\label{def.bbisim}
An  ${\mathcal N}$-\emph{barbed bisimulation} over a set of names, ${\mathcal N}$, is a symmetric binary relation 
${\mathcal S}_{\mathcal N}$ between agents such that $P\rel{S}_{\mathcal N}Q$ implies:
\begin{enumerate}
\item If $P \red P'$ then $Q \wred Q'$ and $P'\rel{S}_{\mathcal N} Q'$.
\item If $P\downarrow_{\mathcal N} x$, then $Q\Downarrow_{\mathcal N} x$.
\end{enumerate}
$P$ is ${\mathcal N}$-barbed bisimilar to $Q$, written
$P \wbbisim_{\mathcal N} Q$, if $P \rel{S}_{\mathcal N} Q$ for some ${\mathcal N}$-barbed bisimulation ${\mathcal S}_{\mathcal N}$.
\end{definition}

$\mathcal{R} \subseteq \pi \times \pi$

$P \mathcal{R} Q => \forall P'. P \red P' \Rightarrow \exists Q'. Q \red Q', P' \mathcal{R} Q'$

$P \vdash x \Rightarrow Q \vdash x$

\begin{mathpar}
  \inferrule*[lab=Out-barb]{x \nameeq y}{{y}!\langle{Q}\rangle \vdash x}
  \and
  \inferrule*[lab=Par-barb]{\mbox{$P\vdash x$ or $Q\vdash x$}}{\binpar{P}{Q} \vdash x}
\end{mathpar}

\subsubsection{Contexts}

One of the principle advantages of computational calculi like the
$\pi$-calculus is a well-defined notion of context,
contextual-equivalence and a correlation between
contextual-equivalence and notions of bisimulation. The notion of
context allows the decomposition of a process into (sub-)process and
its syntactic environment, its context. Thus, a context may be
thought of as a process with a ``hole'' (written $\Box$) in it. The
application of a context $M$ to a process $P$, written $M[P]$, is
tantamount to filling the hole in $M$ with $P$. In this paper we do
not need the full weight of this theory, but do make use of the notion
of context in the proof the main theorem. 

\begin{mathpar}
  \inferrule* [lab=summation] {} {{M_{M},M_{N}} \bc \Box \;|\; x.M_{A} \;|\; M_{M}+M_{N}}
  \and
  \inferrule* [lab=agent] {} {{M_{A}} \bc (\vec{x})M_{P} \;| \; \clift{P_0,\ldots,M_{P},\ldots,P_N}}
  \and \\
  \inferrule* [lab=process] {} {{M_{P}} \bc M_{N} \;| \;P|M_{P} }
\end{mathpar} 

\begin{mathpar}
  \inferrule* [lab=sychronization] {} {M_{N} \bc \Box \;|\; x?M_{F} \;|\; x!M_{C}}
  \and
  \inferrule* [lab=abstraction] {} {{M_{F}} \bc (x)M_{P} }
  \and
  \inferrule* [lab=concretion] {} {{M_{C}} \bc \langle M_{P} \rangle }
  \and \\
  \inferrule* [lab=process] {} {{M_{P}} \bc M_{N} \;| \;P|M_{P} }
\end{mathpar}

\begin{definition}[contextual application] Given a context $M$, and
  process $P$, we define the \emph{contextual application}, $M[P] :=
  M\{P/\Box\}$. That is, the contextual application of M to P is the
  substitution of $P$ for $\Box$ in $M$.
\end{definition}

$\meaningof{-} : L \to \mathcal{P}(\pi)$

\begin{mathpar}
  \inferrule* [lab=collection] {} {\meaningof{true} = \pi, \and \meaningof{~E} = \pi \setminus \meaningof{E}, \and \meaningof{E_{1} \& E_{2}} = \meaningof{E_{1}} \cap \meaningof{E_{2}}}
\end{mathpar}

\begin{mathpar}
  \inferrule* [lab=structure] {} {\meaningof{0} = \{ P \in \pi | P \equiv 0 \}, \and \\ \meaningof{E_1 | E_2} = \{ P \in \pi | P \equiv P_{1} | P_{2}, P_{1} \in \meaningof{E_{1}}, P_{2} \in \meaningof{E_2}\} }
\end{mathpar}

\begin{mathpar}
 \inferrule* [lab=behavior] {} {\meaningof{\langle a?b \rangle E} = \{ P \in \pi | P \equiv Q | u?(y)P', \\ \and \\\\ \and \\ \;\;\; u \in \meaningof{a}, \forall z.P'\{z/y\} \in \meaningof{E\{z/b\}}\}, \and \\ \meaningof{a!E} = \{ P \in \pi | P \equiv Q | x!\langle P' \rangle, x \in \meaningof{a} P' \in \meaningof{E}\} }
\end{mathpar}

\begin{mathpar}
 \inferrule* [lab=nominal] {} {\meaningof{\quotep{E}} = \{ \quotep{P} \in \quotep{\pi} | P \in \meaningof{E} \}, \and \meaningof{\quotep{P}} = \{ \quotep{Q} \in \quotep{\pi} | P \equiv Q \} \and \\ \meaningof{@\quotep{E}} = \{ P \in \pi | P \equiv @x, x \in \meaningof{E} \}}
\end{mathpar}

\begin{eqnarray*}
  \\
  \meaningof{-} : TS \to ST
\end{eqnarray*}

\begin{eqnarray*}
  \\
  L : TS \to ST
\end{eqnarray*}

\begin{eqnarray*}
  \\
  P \models E \iff P \in \meaningof{E}
\end{eqnarray*}

\begin{eqnarray*}
  P \approx_{L} Q \iff \forall E \in L. P \models E \iff Q \models E
\end{eqnarray*}

\begin{eqnarray*}
  P \approx_{K} Q
\end{eqnarray*}

\begin{eqnarray*}
  P \approx Q
\end{eqnarray*}

$\approx_{K} = \approx = \approx_{L}$

\subsubsection{Contextual duality}

Note that contexts extend the quotation operation to a family of
operations from processes to names. Given a context, $M$, we can
define a \emph{nominal context}, $\quotep{M}$ by $\quotep{M}[P] :=
\quotep{M[P]}$. To foreshadow what is to come we observe that these
operations enjoy a duality with processes very much like the duality
between vectors and maps from vectors to scalars.

Further, because the calculus is essentially higher-order, we have a
correspondence between contexts and processes. More specifically,
given a name $x$ and a context $M$ we can construct $M^{*}_{x}$ such
that 

\begin{mathpar}
  M^{*}_{x} | \lift{x}{P} \red M[P]
\end{mathpar}

namely,

\begin{mathpar}
  M^{*}_{x} := x?(u).M[\dropn{u}]
\end{mathpar}

The dependence of $M^{*}_{x}$ on a name makes it an abstraction, 

\begin{mathpar}
  M^{*} := (x)x?(u).M[\dropn{u}]
\end{mathpar}

\subsection{Additional notation}

It will sometimes be convenient to denote the process a name
quotes. We already have the notation $x = \quotep{P}$, but it will be
convenient to introduce an alternate notation, $\procn{x}$, when we
want to emphasize the connection to the use of the name. Note that, by
virtue of name equivalence, $\quotep{\procn{x}} \nameeq x$; so, the
notation is consistent with previous definitions.

Further, because names have structure it is possible to effect
substitutions on the basis of that structure. This means we need to
upgrade our notation for substitutions, which we accomplish by
adapting comprehension notation. Thus,

\begin{mathpar}
  P\{ y / x : x \in S \}
\end{mathpar}

is interpreted to mean the process derived from P by replacing (in a
capture-avoiding manner) each occurrence of $x$ in $S$ by $y$. For example,

\begin{mathpar}
  P\{ \quotep{\procn{x}|\procn{x}} / x : x \in \freenames{P} \}
\end{mathpar}

will replace each (occurrence) of a free name $x$ in $P$ by
$\quotep{\procn{x}|\procn{x}}$.

Also, we will avail ourselves of the notation $x^{L}$ and $x^{R}$ to
denote injections of a name into disjoint copies of the name
space. There are numerous ways to accomplish this. One example can be
found in \cite{MeredithR05}. This notation overloads to vectors of
names: $\vec{x}^{\pi} := (x_{i}^{\pi} \; : \; 0 \leq i < |\vec{x}| )$ where $\pi \in \{L,R\}$.

We also use $P^{\Box} := P|\Box$.

In \cite{MeredithR05} an interpretation of the new operator is
given. It turns out that there are several possible interpretations
all enjoying the requisite algebraic properties of the operator (see
\cite{milner91polyadicpi}). We will therefore make liberal use of
$(\nu\; \vec{x})P$.

% subsection the_syntax_and_semantics_of_the_notation_system (end)   

\input{qm2pi.qmops} 

\input{qm2pi.sterngerlach} 

\input{qm2pi.metric} 

% section concurrent_process_calculi (end)

%\input{qm2pi.proofsketch}

% section proof sketch (end)

%\input{qm2pi.slviaknots} 

% section spatial logic via knots (end)

\input{qm2pi.conclusion}

% section conclusion (end)

%\input{qm2pi.dtcodes} 

% section wiring algorithm (end)

\input{qm2pi.ack} 

% section acknowledgments (end)

\newpage


\bibliographystyle{plain}   
\bibliography{../../biblios/main.bib}

\input{qm2pi.rhodetails}

\end{document}

 

\documentclass[12pt]{llncs}
%\documentclass{jktr}

\usepackage[pdftex]{hyperref}                   
\usepackage {listings}
\usepackage {mathpartir}
\usepackage{bcprules}
%\usepackage{listings}
                       
\usepackage{graphicx} 
%\usepackage[margins=2.5cm,nohead,nofoot]{geometry}
%\usepackage{geometry}
\usepackage{amsfonts}
\usepackage{amstext}
\usepackage{latexsym}
\usepackage{amssymb}
\usepackage{color}


%\include{myPreamble}
\include{qm2pi.local} 

%\ifpdf
%\usepackage[pdftex]{graphicx}
%\else
%\usepackage{graphicx}
%\fi

 % \ifpdf
%  \usepackage{pdfsync}
%  \if


%\title{Brief Article}
%\author{David F. Snyder}
%\author{L.G. Meredith}

%\address{Dept. of Math., Texas State University--San Marcos, San Marcos, TX 78666}
       
\pagestyle{empty}


\begin{document}

\lstset{language=[Objective]Caml,frame=shadowbox}

\input{qm2pi.front}

% section front matter (end)

\input{qm2pi.intro} 
 
% section introduction (end)

% \input{qm2pi.knotations} 

% section notation (end)

\input{qm2pi.process.calculi} 

% section concurrent_process_calculi_and_spatial_logics_ (end)
    
%\input{qm2pi.knots2pi} 

%\input{qm2pi.trefoil} 

%\input{qm2pi.mainthm} 

% subsection basic_interpretation (end)

%\input{qm2pi.rho.presentation} 
\subsection{The syntax and semantics of the notation system}\label{sub:the_syntax_and_semantics_of_the_notation_system} % (fold)

We now summarize a technical presentation of the calculus that
embodies our theory of dynamics. The typical presentation of such a
calculus follows the style of giving generators and relations on
them. The grammar, below, describing term constructors, freely
generates the set of processes, $\Proc$. This set is then quotiented
by a relation known as structural congruence and it is over this set
that the notion of dynamics is expressed. This presentation is
essentially that of \cite{MeredithR05} with the addition of
polyadicity and summation. For readability we have relegated some of
the technical subtleties to an appendix.

\subsubsection{Process grammar}\label{subsub:process_grammar}

\begin{mathpar}
  \inferrule* [lab=synchronization] {} {{M} \bc \pzero \;|\; x?F \;|\; x!C }
  \and
  \inferrule* [lab=abstraction] {} {{F} \bc (x)P}
  \and
  \inferrule* [lab=concretion] {} {{C} \bc \langle Q \rangle}
  \and
  \inferrule* [lab=process] {} {{P,Q} \bc M \;| \;P|Q \;|\; @{x}}
  \and
  \inferrule* [lab=name] {} {{x} \bc \quotep{P}}
\end{mathpar} 

Note that $\vec{x}$ (resp. $\vec{P}$) denotes a vector of names
(resp. processes) of length $|\vec{x}|$ (resp. $|\vec{P}|$). We adopt
the following useful abbreviations.

\begin{mathpar}
   x?(\vec{y}).P := x.(\vec{y})P \and  x\clift{\vec{P}} := x.\clift{\vec{P}}
   \and x!(y) := \lift{x}{\dropn{y}}
   \and \Pi_{i=0}^{n-1}P_i := P_0 | \ldots | P_{n-1}
\end{mathpar}

\subsubsection{Structural congruence}

\paragraph{Free and bound names and alpha-equivalence.} At the
core of structural equivalence is alpha-equivalence which identifies
process that are the same up to a change of variable. Formally, we
recognize the distinction between free and bound names. The free names
of a process, $\freenames{P}$, may be calculated recursively as
follows:

\begin{mathpar}
\freenames{\pzero} := \emptyset
  \and \\
  \freenames{x?(y).P} := \{ x \} \cup (\freenames{P} \setminus \{ y \})
  \and 
  \freenames{x!\langle P \rangle} := \{ x \} \cup \{ P \} 
  \and \\
  \freenames{P|Q} := \freenames{P} \cup \freenames{Q}
  \and \\
  \freenames{@{x}} := \{ x \}
\end{mathpar}

$\pi$
$\quotep{\pi}$

$\freenames{-} : \pi \to \mathcal{P}(\quotep{\pi})$

\begin{eqnarray*}
  \freenames{\pzero} & := & \emptyset \\
  \freenames{x?(y).P} & := & \{ x \} \cup (\freenames{P} \setminus \{ y \}) \\
  \freenames{x!\langle P \rangle} & := & \{ x \} \cup \{ P \} \\
  \freenames{P|Q} & := & \freenames{P} \cup \freenames{Q} \\
  \freenames{\dropn{x}} & := & \{ x \}
\end{eqnarray*}

The bound names of a process, $\boundnames{P}$, are those names occurring in $P$
that are not free. For example, in $x?(y).0$, the name $x$ is free, while $y$ is bound.

\begin{mathpar}
  \inferrule* [lab=monoidal-laws] {} { P|Q \equiv Q|P \and P|0 \equiv P \and P|(Q|R) \equiv (P|Q)|R }
\end{mathpar}

\begin{mathpar}
  \inferrule* [lab=alpha-equivalence] {} { (x)P \equiv (y)P\{y/x\} \and y \not\in \freenames{P} }
\end{mathpar}

\begin{definition}
Then two processes, $P,Q$, are alpha-equivalent if $P = Q\{\vec{y}/\vec{x}\}$ for
some $\vec{x} \in \boundnames{Q},\vec{y} \in \boundnames{P}$, where $Q\{\vec{y}/\vec{x}\}$
denotes the capture-avoiding substitution of $\vec{y}$ for $\vec{x}$ in $Q$.
\end{definition}

\begin{definition}
  The {\em structural congruence} \cite{SangiorgiWalker} , $\equiv$,
  between processes is the least congruence containing
  alpha-equivalence, satisfying the abelian monoid laws
  (associativity, commutativity and $\pzero$ as identity) for parallel
  composition $|$ and for summation $+$.
\end{definition}

\subsection{Name equivalence}

We take name equivalence, written $\nameeq$, to be the smallest
equivalence relation generated by the following rules.

\begin{mathpar}
\inferrule*[lab=Quote-drop]
{ }
{ \quotep{@{x}} \nameeq x }

\inferrule*[lab=Struct-equiv]
{ P \scong Q }
{ \quotep{P} \nameeq \quotep{Q} }
\end{mathpar}

The astute reader will have noticed that the mutual recursion of names
and processes imposes a mutual recursion on alpha-equivalence and
structural equivalence via name-equivalence. Fortunately, all of this
works out pleasantly and we may calculate in the natural way, free of
concern. The reader interested in the details is referred to the
appendix \ref{appendix:rho_details}.

\subsection{Substitution}

We use $\Proc$ for the set of processes, $\QProc$ for the set of
names, and $\id{\{}\vec{y} / \vec{x} \id{\}}$ to denote partial maps,
$s : \QProc \rightarrow \QProc$. A map, $s$ lifts, uniquely, to a map
on process terms, $\widehat{s} : \Proc \rightarrow \Proc$ by the
following equations.

\begin{mathpar}
  (0) \psubstp{Q}{P} := 0 \\
  (R \juxtap S) \psubstp{Q}{P}
  :=    
  (R)\psubstp{Q}{P} \juxtap (S) \psubstp{Q}{P} \\
  (x?(y).R) \psubstp{Q}{P}    
  :=    
  (x)\substp{Q}{P} (z)\concat( (R \psubstn{z}{y}) \psubstp{Q}{P} ) \\
  (\lift{x}{R}) \psubstp{Q}{P}  
  :=
  \lift{(x)\substp{Q}{P}}{ R \psubstp{Q}{P} } \\
%   (\dropn{x})  \psubstp{Q}{P}       
%   := 
%   \left\{ 
%     \begin{array}{ccc} 
%       \dropn{\quotep{Q}} & & x \nameeq \quotep{P} \\
%       \dropn{x} & & otherwise \\
%     \end{array}
%   \right. 
  (\dropn{x})  \psubstp{Q}{P}       
  := 
  \left\{ 
    \begin{array}{ccc} 
      Q & & x \nameeq \quotep{P} \\
      \dropn{x} & & otherwise \\
    \end{array}
  \right.
\end{mathpar}
 

where

\begin{eqnarray}
  (x)\id{\{} \lpquote Q \rpquote / \lpquote P \rpquote \id{\}}            = 
  \left\{ 
    \begin{array}{ccc}
      \lpquote Q \rpquote & & x \nameeq \lpquote P \rpquote \\
      x & & otherwise \\
    \end{array}
  \right. \nonumber
\end{eqnarray}

and $z$ is chosen distinct from $\quotep{P}$, $\quotep{Q}$, the free
names in $Q$, and all the names in $R$. Our $\alpha$-equivalence will
be built in the standard way from this substitution.

\begin{remark}\label{rem:no_self_referential_names}
  One consequence of these definitions is that $\forall P. \quotep{P}
  \not\in \freenames{P}$.
\end{remark}

\subsection{ Dynamic quote: an example }

Anticipating something of what's to come, consider applying the
substitution, $\widehat{\id{\{}u / z \id{\}}}$, to the following pair
of processes, $\lift{w}{y!(z)}$ and $w[ \lpquote y!(z) \rpquote ]$.

\begin{eqnarray}
	\lift{w}{y!(z)}\widehat{\id{\{}u / z \id{\}}}
		& = &
		\lift{w}{y!(u)} \nonumber\\
	w[ \lpquote y!(z) \rpquote ] \widehat{ \id{\{}u / z \id{\}} }
		& = &
		w[ \lpquote y!(z) \rpquote ] \nonumber
\end{eqnarray}

Because the body of the process between quotes is impervious to
substitution, we get radically different answers. In fact, by
examining the first process in an input context,
e.g. $x?(z).\lift{w}{y!(z)}$, we see that the process under the lift
operator may be shaped by prefixed inputs binding a name inside it. In
this sense, the lift operator will be seen as a way to dynamically
construct processes before reifying them as names.

Finally equipped with these standard features we can present the
dynamics of the calculus.

\subsubsection{Operational semantics} 

Finally, we introduce the computational dynamics. What marks these
algebras as distinct from other more traditionally studied algebraic
structures, e.g. vector spaces or polynomial rings, is the manner in
which dynamics is captured. In traditional structures, dynamics is typically
expressed through morphisms between such structures, as in linear maps
between vector spaces or morphisms between rings. In algebras
associated with the semantics of computation, the dynamics is
expressed as part of the algebraic structure itself, through a
reduction reduction relation typically denoted by $\red$. Below, we
give a recursive presentation of this relation for the calculus used
in the encoding.

$\red \subseteq \pi \times \pi$
$\red : \pi \to \mathcal{P}(\pi)$

\begin{mathpar}
  \inferrule* [lab=Comm] { \textsf{match}( x_{src}, x_{trgt} ) } { x_{trgt}?(y)P \; | \; x_{src}!\langle {Q} \rangle \red P\{\quotep{Q}/y}\} }
  \and \\
  \inferrule* [lab=Par] {{P} \red {P}'} {{{P} | {Q}} \red {{P}' | {Q}}}
  \and
  \inferrule* [lab=Equiv]{{{P} \scong {P}'} \andalso {{P}' \red {Q}'} \andalso {{Q}' \scong {Q}}}{{P} \red {Q}}
\end{mathpar}

\begin{eqnarray*}
  match_{\equiv} (\quotep{P},\quotep{Q}) & := & P \equiv Q \\
  match_{\dagger}(\quotep{P},\quotep{Q}) & := & \forall R. P|Q \red^{*} R => R \red^{*} 0 \\
  match_{K}(\quotep{P},\quotep{Q}) & := & K \mbox{ for some context } K
\end{eqnarray*}

$u?(x)P | u!\langle Q \rangle \red P\{\quotep{Q}/x\}$

%We write $\wred$ for $\red^*$, and $P\red$ if $\exists Q $ such that $ P \red Q$.
We write $P\red$ if $\exists Q $ such that $ P \red Q$ and $P\not\red$, otherwise.

\section{Replication}

As mentioned before, it is known that replication (and hence
recursion) can be implemented in a higher-order process algebra
\cite{SangiorgiWalker}. As our first example of calculation with the
machinery thus far presented we give the construction explicitly in
the {\rhoc}.

\begin{eqnarray}
	D_{x} & := & \prefix{x}{y}{(\binpar{\outputp{x}{y}}{@{y}})} \nonumber\\
	\bangp_{x}{P} & := & \binpar{{x}!\langle{\binpar{D_{x}}{P}}\rangle}{D_{x}} \nonumber
\end{eqnarray}

\begin{eqnarray}
	\bangp_{x}{P} & & \nonumber\\
	=
	& {x}!\langle{(\prefix{x}{y}{(\outputp{x}{y} | @{y})) | P}}\rangle 
	      | \prefix{x}{y}{(\outputp{x}{y} | @{y})} & \nonumber\\
	\red
	& (\outputp{x}{y} | @{y})\substn{\quotep{(\prefix{x}{y}{(@{y} | \outputp{x}{y})) | P}}}{y} & \nonumber\\
	=
	& \outputp{x}{\quotep{(\prefix{x}{y}{(\outputp{x}{y} | @{y})) | P}}}
	  | {(\prefix{x}{y}{(\outputp{x}{y} | @{y})) | P}} & \nonumber\\
	\red
	& \ldots & \nonumber\\
	\red^*
	& P | P | \ldots & \nonumber
\end{eqnarray}

Of course, this encoding, as an implementation, runs away, unfolding
$\bangp{P}$ eagerly. A lazier and more implementable replication
operator, restricted to input-guarded processes, may be obtained as follows.

\begin{eqnarray}
\bangp{\prefix{u}{v}{P}} 
	:= 
	\binpar{\lift{x}{\prefix{u}{v}{(\binpar{D(x)}{P})}}}{D(x)} \nonumber
\end{eqnarray}

\begin{remark}
  Note that the lazier definition still does not deal with summation
  or mixed summation (i.e. sums over input and output). The reader is
  invited to construct definitions of replication that deal with these
  features. 

  Further, the definitions are parameterized in a name, $x$. Can you,
  gentle reader, make a definition that eliminates this parameter and
  guarantees no accidental interaction between the replication
  machinery and the process being replicated -- i.e. no accidental
  sharing of names used by the process to get its work done and the
  name(s) used by the replication to effect copying. This latter
  revision of the definition of replication is crucial to obtaining
  the expected identity $!!P \sim !P$.
\end{remark}

\begin{remark}\label{rem:paradoxical_combinator}
  The reader familiar with the lambda calculus will have noticed the
  similarity between $D$ and the paradoxical combinator.

  [Ed. note: the existence of this seems to suggest we have to be more
  restrictive on the set of processes and names we admit if we are to
  support no-cloning.]
\end{remark}

\subsubsection{Bisimulation}

The computational dynamics gives rise to another kind of equivalence,
the equivalence of computational behavior. As previously mentioned
this is typically captured \emph{via} some form of bisimulation.

% The notion we use in this paper is weak barbed bisimulation
% \cite{milner91polyadicpi}.

The notion we use in this paper is derived from weak barbed
bisimulation \cite{milner91polyadicpi}. 

\begin{definition}
An \emph{observation relation}, $\downarrow_{\mathcal N}$, over a set
of names, $\mathcal N$, is the smallest relation satisfying the rules
below.

\infrule[Out-barb]{y \in {\mathcal N}, \; x \nameeq y}
		  {\outputp{x}{v} \downarrow_{\mathcal N} x}
\infrule[Par-barb]{\mbox{$P\downarrow_{\mathcal N} x$ or $Q\downarrow_{\mathcal N} x$}}
		  {\binpar{P}{Q} \downarrow_{\mathcal N} x}

We write $P \Downarrow_{\mathcal N} x$ if there is $Q$ such that 
$P \wred Q$ and $Q \downarrow_{\mathcal N} x$.
\end{definition}

\begin{definition}
%\label{def.bbisim}
An  ${\mathcal N}$-\emph{barbed bisimulation} over a set of names, ${\mathcal N}$, is a symmetric binary relation 
${\mathcal S}_{\mathcal N}$ between agents such that $P\rel{S}_{\mathcal N}Q$ implies:
\begin{enumerate}
\item If $P \red P'$ then $Q \wred Q'$ and $P'\rel{S}_{\mathcal N} Q'$.
\item If $P\downarrow_{\mathcal N} x$, then $Q\Downarrow_{\mathcal N} x$.
\end{enumerate}
$P$ is ${\mathcal N}$-barbed bisimilar to $Q$, written
$P \wbbisim_{\mathcal N} Q$, if $P \rel{S}_{\mathcal N} Q$ for some ${\mathcal N}$-barbed bisimulation ${\mathcal S}_{\mathcal N}$.
\end{definition}

$\mathcal{R} \subseteq \pi \times \pi$

$P \mathcal{R} Q => \forall P'. P \red P' \Rightarrow \exists Q'. Q \red Q', P' \mathcal{R} Q'$

$P \vdash x \Rightarrow Q \vdash x$

\begin{mathpar}
  \inferrule*[lab=Out-barb]{x \nameeq y}{{y}!\langle{Q}\rangle \vdash x}
  \and
  \inferrule*[lab=Par-barb]{\mbox{$P\vdash x$ or $Q\vdash x$}}{\binpar{P}{Q} \vdash x}
\end{mathpar}

\subsubsection{Contexts}

One of the principle advantages of computational calculi like the
$\pi$-calculus is a well-defined notion of context,
contextual-equivalence and a correlation between
contextual-equivalence and notions of bisimulation. The notion of
context allows the decomposition of a process into (sub-)process and
its syntactic environment, its context. Thus, a context may be
thought of as a process with a ``hole'' (written $\Box$) in it. The
application of a context $M$ to a process $P$, written $M[P]$, is
tantamount to filling the hole in $M$ with $P$. In this paper we do
not need the full weight of this theory, but do make use of the notion
of context in the proof the main theorem. 

\begin{mathpar}
  \inferrule* [lab=summation] {} {{M_{M},M_{N}} \bc \Box \;|\; x.M_{A} \;|\; M_{M}+M_{N}}
  \and
  \inferrule* [lab=agent] {} {{M_{A}} \bc (\vec{x})M_{P} \;| \; \clift{P_0,\ldots,M_{P},\ldots,P_N}}
  \and \\
  \inferrule* [lab=process] {} {{M_{P}} \bc M_{N} \;| \;P|M_{P} }
\end{mathpar} 

\begin{mathpar}
  \inferrule* [lab=sychronization] {} {M_{N} \bc \Box \;|\; x?M_{F} \;|\; x!M_{C}}
  \and
  \inferrule* [lab=abstraction] {} {{M_{F}} \bc (x)M_{P} }
  \and
  \inferrule* [lab=concretion] {} {{M_{C}} \bc \langle M_{P} \rangle }
  \and \\
  \inferrule* [lab=process] {} {{M_{P}} \bc M_{N} \;| \;P|M_{P} }
\end{mathpar}

\begin{definition}[contextual application] Given a context $M$, and
  process $P$, we define the \emph{contextual application}, $M[P] :=
  M\{P/\Box\}$. That is, the contextual application of M to P is the
  substitution of $P$ for $\Box$ in $M$.
\end{definition}

$\meaningof{-} : L \to \mathcal{P}(\pi)$

\begin{mathpar}
  \inferrule* [lab=collection] {} {\meaningof{true} = \pi, \and \meaningof{~E} = \pi \setminus \meaningof{E}, \and \meaningof{E_{1} \& E_{2}} = \meaningof{E_{1}} \cap \meaningof{E_{2}}}
\end{mathpar}

\begin{mathpar}
  \inferrule* [lab=structure] {} {\meaningof{0} = \{ P \in \pi | P \equiv 0 \}, \and \\ \meaningof{E_1 | E_2} = \{ P \in \pi | P \equiv P_{1} | P_{2}, P_{1} \in \meaningof{E_{1}}, P_{2} \in \meaningof{E_2}\} }
\end{mathpar}

\begin{mathpar}
 \inferrule* [lab=behavior] {} {\meaningof{\langle a?b \rangle E} = \{ P \in \pi | P \equiv Q | u?(y)P', \\ \and \\\\ \and \\ \;\;\; u \in \meaningof{a}, \forall z.P'\{z/y\} \in \meaningof{E\{z/b\}}\}, \and \\ \meaningof{a!E} = \{ P \in \pi | P \equiv Q | x!\langle P' \rangle, x \in \meaningof{a} P' \in \meaningof{E}\} }
\end{mathpar}

\begin{mathpar}
 \inferrule* [lab=nominal] {} {\meaningof{\quotep{E}} = \{ \quotep{P} \in \quotep{\pi} | P \in \meaningof{E} \}, \and \meaningof{\quotep{P}} = \{ \quotep{Q} \in \quotep{\pi} | P \equiv Q \} \and \\ \meaningof{@\quotep{E}} = \{ P \in \pi | P \equiv @x, x \in \meaningof{E} \}}
\end{mathpar}

\begin{eqnarray*}
  \\
  \meaningof{-} : TS \to ST
\end{eqnarray*}

\begin{eqnarray*}
  \\
  L : TS \to ST
\end{eqnarray*}

\begin{eqnarray*}
  \\
  P \models E \iff P \in \meaningof{E}
\end{eqnarray*}

\begin{eqnarray*}
  P \approx_{L} Q \iff \forall E \in L. P \models E \iff Q \models E
\end{eqnarray*}

\begin{eqnarray*}
  P \approx_{K} Q
\end{eqnarray*}

\begin{eqnarray*}
  P \approx Q
\end{eqnarray*}

$\approx_{K} = \approx = \approx_{L}$

\subsubsection{Contextual duality}

Note that contexts extend the quotation operation to a family of
operations from processes to names. Given a context, $M$, we can
define a \emph{nominal context}, $\quotep{M}$ by $\quotep{M}[P] :=
\quotep{M[P]}$. To foreshadow what is to come we observe that these
operations enjoy a duality with processes very much like the duality
between vectors and maps from vectors to scalars.

Further, because the calculus is essentially higher-order, we have a
correspondence between contexts and processes. More specifically,
given a name $x$ and a context $M$ we can construct $M^{*}_{x}$ such
that 

\begin{mathpar}
  M^{*}_{x} | \lift{x}{P} \red M[P]
\end{mathpar}

namely,

\begin{mathpar}
  M^{*}_{x} := x?(u).M[\dropn{u}]
\end{mathpar}

The dependence of $M^{*}_{x}$ on a name makes it an abstraction, 

\begin{mathpar}
  M^{*} := (x)x?(u).M[\dropn{u}]
\end{mathpar}

\subsection{Additional notation}

It will sometimes be convenient to denote the process a name
quotes. We already have the notation $x = \quotep{P}$, but it will be
convenient to introduce an alternate notation, $\procn{x}$, when we
want to emphasize the connection to the use of the name. Note that, by
virtue of name equivalence, $\quotep{\procn{x}} \nameeq x$; so, the
notation is consistent with previous definitions.

Further, because names have structure it is possible to effect
substitutions on the basis of that structure. This means we need to
upgrade our notation for substitutions, which we accomplish by
adapting comprehension notation. Thus,

\begin{mathpar}
  P\{ y / x : x \in S \}
\end{mathpar}

is interpreted to mean the process derived from P by replacing (in a
capture-avoiding manner) each occurrence of $x$ in $S$ by $y$. For example,

\begin{mathpar}
  P\{ \quotep{\procn{x}|\procn{x}} / x : x \in \freenames{P} \}
\end{mathpar}

will replace each (occurrence) of a free name $x$ in $P$ by
$\quotep{\procn{x}|\procn{x}}$.

Also, we will avail ourselves of the notation $x^{L}$ and $x^{R}$ to
denote injections of a name into disjoint copies of the name
space. There are numerous ways to accomplish this. One example can be
found in \cite{MeredithR05}. This notation overloads to vectors of
names: $\vec{x}^{\pi} := (x_{i}^{\pi} \; : \; 0 \leq i < |\vec{x}| )$ where $\pi \in \{L,R\}$.

We also use $P^{\Box} := P|\Box$.

In \cite{MeredithR05} an interpretation of the new operator is
given. It turns out that there are several possible interpretations
all enjoying the requisite algebraic properties of the operator (see
\cite{milner91polyadicpi}). We will therefore make liberal use of
$(\nu\; \vec{x})P$.

% subsection the_syntax_and_semantics_of_the_notation_system (end)   

\input{qm2pi.qmops} 

\input{qm2pi.sterngerlach} 

\input{qm2pi.metric} 

% section concurrent_process_calculi (end)

%\input{qm2pi.proofsketch}

% section proof sketch (end)

%\input{qm2pi.slviaknots} 

% section spatial logic via knots (end)

\input{qm2pi.conclusion}

% section conclusion (end)

%\input{qm2pi.dtcodes} 

% section wiring algorithm (end)

\input{qm2pi.ack} 

% section acknowledgments (end)

\newpage


\bibliographystyle{plain}   
\bibliography{../../biblios/main.bib}

\input{qm2pi.rhodetails}

\end{document}

 

% section concurrent_process_calculi (end)

%\documentclass[12pt]{llncs}
%\documentclass{jktr}

\usepackage[pdftex]{hyperref}                   
\usepackage {listings}
\usepackage {mathpartir}
\usepackage{bcprules}
%\usepackage{listings}
                       
\usepackage{graphicx} 
%\usepackage[margins=2.5cm,nohead,nofoot]{geometry}
%\usepackage{geometry}
\usepackage{amsfonts}
\usepackage{amstext}
\usepackage{latexsym}
\usepackage{amssymb}
\usepackage{color}


%\include{myPreamble}
\include{qm2pi.local} 

%\ifpdf
%\usepackage[pdftex]{graphicx}
%\else
%\usepackage{graphicx}
%\fi

 % \ifpdf
%  \usepackage{pdfsync}
%  \if


%\title{Brief Article}
%\author{David F. Snyder}
%\author{L.G. Meredith}

%\address{Dept. of Math., Texas State University--San Marcos, San Marcos, TX 78666}
       
\pagestyle{empty}


\begin{document}

\lstset{language=[Objective]Caml,frame=shadowbox}

\input{qm2pi.front}

% section front matter (end)

\input{qm2pi.intro} 
 
% section introduction (end)

% \input{qm2pi.knotations} 

% section notation (end)

\input{qm2pi.process.calculi} 

% section concurrent_process_calculi_and_spatial_logics_ (end)
    
%\input{qm2pi.knots2pi} 

%\input{qm2pi.trefoil} 

%\input{qm2pi.mainthm} 

% subsection basic_interpretation (end)

%\input{qm2pi.rho.presentation} 
\subsection{The syntax and semantics of the notation system}\label{sub:the_syntax_and_semantics_of_the_notation_system} % (fold)

We now summarize a technical presentation of the calculus that
embodies our theory of dynamics. The typical presentation of such a
calculus follows the style of giving generators and relations on
them. The grammar, below, describing term constructors, freely
generates the set of processes, $\Proc$. This set is then quotiented
by a relation known as structural congruence and it is over this set
that the notion of dynamics is expressed. This presentation is
essentially that of \cite{MeredithR05} with the addition of
polyadicity and summation. For readability we have relegated some of
the technical subtleties to an appendix.

\subsubsection{Process grammar}\label{subsub:process_grammar}

\begin{mathpar}
  \inferrule* [lab=synchronization] {} {{M} \bc \pzero \;|\; x?F \;|\; x!C }
  \and
  \inferrule* [lab=abstraction] {} {{F} \bc (x)P}
  \and
  \inferrule* [lab=concretion] {} {{C} \bc \langle Q \rangle}
  \and
  \inferrule* [lab=process] {} {{P,Q} \bc M \;| \;P|Q \;|\; @{x}}
  \and
  \inferrule* [lab=name] {} {{x} \bc \quotep{P}}
\end{mathpar} 

Note that $\vec{x}$ (resp. $\vec{P}$) denotes a vector of names
(resp. processes) of length $|\vec{x}|$ (resp. $|\vec{P}|$). We adopt
the following useful abbreviations.

\begin{mathpar}
   x?(\vec{y}).P := x.(\vec{y})P \and  x\clift{\vec{P}} := x.\clift{\vec{P}}
   \and x!(y) := \lift{x}{\dropn{y}}
   \and \Pi_{i=0}^{n-1}P_i := P_0 | \ldots | P_{n-1}
\end{mathpar}

\subsubsection{Structural congruence}

\paragraph{Free and bound names and alpha-equivalence.} At the
core of structural equivalence is alpha-equivalence which identifies
process that are the same up to a change of variable. Formally, we
recognize the distinction between free and bound names. The free names
of a process, $\freenames{P}$, may be calculated recursively as
follows:

\begin{mathpar}
\freenames{\pzero} := \emptyset
  \and \\
  \freenames{x?(y).P} := \{ x \} \cup (\freenames{P} \setminus \{ y \})
  \and 
  \freenames{x!\langle P \rangle} := \{ x \} \cup \{ P \} 
  \and \\
  \freenames{P|Q} := \freenames{P} \cup \freenames{Q}
  \and \\
  \freenames{@{x}} := \{ x \}
\end{mathpar}

$\pi$
$\quotep{\pi}$

$\freenames{-} : \pi \to \mathcal{P}(\quotep{\pi})$

\begin{eqnarray*}
  \freenames{\pzero} & := & \emptyset \\
  \freenames{x?(y).P} & := & \{ x \} \cup (\freenames{P} \setminus \{ y \}) \\
  \freenames{x!\langle P \rangle} & := & \{ x \} \cup \{ P \} \\
  \freenames{P|Q} & := & \freenames{P} \cup \freenames{Q} \\
  \freenames{\dropn{x}} & := & \{ x \}
\end{eqnarray*}

The bound names of a process, $\boundnames{P}$, are those names occurring in $P$
that are not free. For example, in $x?(y).0$, the name $x$ is free, while $y$ is bound.

\begin{mathpar}
  \inferrule* [lab=monoidal-laws] {} { P|Q \equiv Q|P \and P|0 \equiv P \and P|(Q|R) \equiv (P|Q)|R }
\end{mathpar}

\begin{mathpar}
  \inferrule* [lab=alpha-equivalence] {} { (x)P \equiv (y)P\{y/x\} \and y \not\in \freenames{P} }
\end{mathpar}

\begin{definition}
Then two processes, $P,Q$, are alpha-equivalent if $P = Q\{\vec{y}/\vec{x}\}$ for
some $\vec{x} \in \boundnames{Q},\vec{y} \in \boundnames{P}$, where $Q\{\vec{y}/\vec{x}\}$
denotes the capture-avoiding substitution of $\vec{y}$ for $\vec{x}$ in $Q$.
\end{definition}

\begin{definition}
  The {\em structural congruence} \cite{SangiorgiWalker} , $\equiv$,
  between processes is the least congruence containing
  alpha-equivalence, satisfying the abelian monoid laws
  (associativity, commutativity and $\pzero$ as identity) for parallel
  composition $|$ and for summation $+$.
\end{definition}

\subsection{Name equivalence}

We take name equivalence, written $\nameeq$, to be the smallest
equivalence relation generated by the following rules.

\begin{mathpar}
\inferrule*[lab=Quote-drop]
{ }
{ \quotep{@{x}} \nameeq x }

\inferrule*[lab=Struct-equiv]
{ P \scong Q }
{ \quotep{P} \nameeq \quotep{Q} }
\end{mathpar}

The astute reader will have noticed that the mutual recursion of names
and processes imposes a mutual recursion on alpha-equivalence and
structural equivalence via name-equivalence. Fortunately, all of this
works out pleasantly and we may calculate in the natural way, free of
concern. The reader interested in the details is referred to the
appendix \ref{appendix:rho_details}.

\subsection{Substitution}

We use $\Proc$ for the set of processes, $\QProc$ for the set of
names, and $\id{\{}\vec{y} / \vec{x} \id{\}}$ to denote partial maps,
$s : \QProc \rightarrow \QProc$. A map, $s$ lifts, uniquely, to a map
on process terms, $\widehat{s} : \Proc \rightarrow \Proc$ by the
following equations.

\begin{mathpar}
  (0) \psubstp{Q}{P} := 0 \\
  (R \juxtap S) \psubstp{Q}{P}
  :=    
  (R)\psubstp{Q}{P} \juxtap (S) \psubstp{Q}{P} \\
  (x?(y).R) \psubstp{Q}{P}    
  :=    
  (x)\substp{Q}{P} (z)\concat( (R \psubstn{z}{y}) \psubstp{Q}{P} ) \\
  (\lift{x}{R}) \psubstp{Q}{P}  
  :=
  \lift{(x)\substp{Q}{P}}{ R \psubstp{Q}{P} } \\
%   (\dropn{x})  \psubstp{Q}{P}       
%   := 
%   \left\{ 
%     \begin{array}{ccc} 
%       \dropn{\quotep{Q}} & & x \nameeq \quotep{P} \\
%       \dropn{x} & & otherwise \\
%     \end{array}
%   \right. 
  (\dropn{x})  \psubstp{Q}{P}       
  := 
  \left\{ 
    \begin{array}{ccc} 
      Q & & x \nameeq \quotep{P} \\
      \dropn{x} & & otherwise \\
    \end{array}
  \right.
\end{mathpar}
 

where

\begin{eqnarray}
  (x)\id{\{} \lpquote Q \rpquote / \lpquote P \rpquote \id{\}}            = 
  \left\{ 
    \begin{array}{ccc}
      \lpquote Q \rpquote & & x \nameeq \lpquote P \rpquote \\
      x & & otherwise \\
    \end{array}
  \right. \nonumber
\end{eqnarray}

and $z$ is chosen distinct from $\quotep{P}$, $\quotep{Q}$, the free
names in $Q$, and all the names in $R$. Our $\alpha$-equivalence will
be built in the standard way from this substitution.

\begin{remark}\label{rem:no_self_referential_names}
  One consequence of these definitions is that $\forall P. \quotep{P}
  \not\in \freenames{P}$.
\end{remark}

\subsection{ Dynamic quote: an example }

Anticipating something of what's to come, consider applying the
substitution, $\widehat{\id{\{}u / z \id{\}}}$, to the following pair
of processes, $\lift{w}{y!(z)}$ and $w[ \lpquote y!(z) \rpquote ]$.

\begin{eqnarray}
	\lift{w}{y!(z)}\widehat{\id{\{}u / z \id{\}}}
		& = &
		\lift{w}{y!(u)} \nonumber\\
	w[ \lpquote y!(z) \rpquote ] \widehat{ \id{\{}u / z \id{\}} }
		& = &
		w[ \lpquote y!(z) \rpquote ] \nonumber
\end{eqnarray}

Because the body of the process between quotes is impervious to
substitution, we get radically different answers. In fact, by
examining the first process in an input context,
e.g. $x?(z).\lift{w}{y!(z)}$, we see that the process under the lift
operator may be shaped by prefixed inputs binding a name inside it. In
this sense, the lift operator will be seen as a way to dynamically
construct processes before reifying them as names.

Finally equipped with these standard features we can present the
dynamics of the calculus.

\subsubsection{Operational semantics} 

Finally, we introduce the computational dynamics. What marks these
algebras as distinct from other more traditionally studied algebraic
structures, e.g. vector spaces or polynomial rings, is the manner in
which dynamics is captured. In traditional structures, dynamics is typically
expressed through morphisms between such structures, as in linear maps
between vector spaces or morphisms between rings. In algebras
associated with the semantics of computation, the dynamics is
expressed as part of the algebraic structure itself, through a
reduction reduction relation typically denoted by $\red$. Below, we
give a recursive presentation of this relation for the calculus used
in the encoding.

$\red \subseteq \pi \times \pi$
$\red : \pi \to \mathcal{P}(\pi)$

\begin{mathpar}
  \inferrule* [lab=Comm] { \textsf{match}( x_{src}, x_{trgt} ) } { x_{trgt}?(y)P \; | \; x_{src}!\langle {Q} \rangle \red P\{\quotep{Q}/y}\} }
  \and \\
  \inferrule* [lab=Par] {{P} \red {P}'} {{{P} | {Q}} \red {{P}' | {Q}}}
  \and
  \inferrule* [lab=Equiv]{{{P} \scong {P}'} \andalso {{P}' \red {Q}'} \andalso {{Q}' \scong {Q}}}{{P} \red {Q}}
\end{mathpar}

\begin{eqnarray*}
  match_{\equiv} (\quotep{P},\quotep{Q}) & := & P \equiv Q \\
  match_{\dagger}(\quotep{P},\quotep{Q}) & := & \forall R. P|Q \red^{*} R => R \red^{*} 0 \\
  match_{K}(\quotep{P},\quotep{Q}) & := & K \mbox{ for some context } K
\end{eqnarray*}

$u?(x)P | u!\langle Q \rangle \red P\{\quotep{Q}/x\}$

%We write $\wred$ for $\red^*$, and $P\red$ if $\exists Q $ such that $ P \red Q$.
We write $P\red$ if $\exists Q $ such that $ P \red Q$ and $P\not\red$, otherwise.

\section{Replication}

As mentioned before, it is known that replication (and hence
recursion) can be implemented in a higher-order process algebra
\cite{SangiorgiWalker}. As our first example of calculation with the
machinery thus far presented we give the construction explicitly in
the {\rhoc}.

\begin{eqnarray}
	D_{x} & := & \prefix{x}{y}{(\binpar{\outputp{x}{y}}{@{y}})} \nonumber\\
	\bangp_{x}{P} & := & \binpar{{x}!\langle{\binpar{D_{x}}{P}}\rangle}{D_{x}} \nonumber
\end{eqnarray}

\begin{eqnarray}
	\bangp_{x}{P} & & \nonumber\\
	=
	& {x}!\langle{(\prefix{x}{y}{(\outputp{x}{y} | @{y})) | P}}\rangle 
	      | \prefix{x}{y}{(\outputp{x}{y} | @{y})} & \nonumber\\
	\red
	& (\outputp{x}{y} | @{y})\substn{\quotep{(\prefix{x}{y}{(@{y} | \outputp{x}{y})) | P}}}{y} & \nonumber\\
	=
	& \outputp{x}{\quotep{(\prefix{x}{y}{(\outputp{x}{y} | @{y})) | P}}}
	  | {(\prefix{x}{y}{(\outputp{x}{y} | @{y})) | P}} & \nonumber\\
	\red
	& \ldots & \nonumber\\
	\red^*
	& P | P | \ldots & \nonumber
\end{eqnarray}

Of course, this encoding, as an implementation, runs away, unfolding
$\bangp{P}$ eagerly. A lazier and more implementable replication
operator, restricted to input-guarded processes, may be obtained as follows.

\begin{eqnarray}
\bangp{\prefix{u}{v}{P}} 
	:= 
	\binpar{\lift{x}{\prefix{u}{v}{(\binpar{D(x)}{P})}}}{D(x)} \nonumber
\end{eqnarray}

\begin{remark}
  Note that the lazier definition still does not deal with summation
  or mixed summation (i.e. sums over input and output). The reader is
  invited to construct definitions of replication that deal with these
  features. 

  Further, the definitions are parameterized in a name, $x$. Can you,
  gentle reader, make a definition that eliminates this parameter and
  guarantees no accidental interaction between the replication
  machinery and the process being replicated -- i.e. no accidental
  sharing of names used by the process to get its work done and the
  name(s) used by the replication to effect copying. This latter
  revision of the definition of replication is crucial to obtaining
  the expected identity $!!P \sim !P$.
\end{remark}

\begin{remark}\label{rem:paradoxical_combinator}
  The reader familiar with the lambda calculus will have noticed the
  similarity between $D$ and the paradoxical combinator.

  [Ed. note: the existence of this seems to suggest we have to be more
  restrictive on the set of processes and names we admit if we are to
  support no-cloning.]
\end{remark}

\subsubsection{Bisimulation}

The computational dynamics gives rise to another kind of equivalence,
the equivalence of computational behavior. As previously mentioned
this is typically captured \emph{via} some form of bisimulation.

% The notion we use in this paper is weak barbed bisimulation
% \cite{milner91polyadicpi}.

The notion we use in this paper is derived from weak barbed
bisimulation \cite{milner91polyadicpi}. 

\begin{definition}
An \emph{observation relation}, $\downarrow_{\mathcal N}$, over a set
of names, $\mathcal N$, is the smallest relation satisfying the rules
below.

\infrule[Out-barb]{y \in {\mathcal N}, \; x \nameeq y}
		  {\outputp{x}{v} \downarrow_{\mathcal N} x}
\infrule[Par-barb]{\mbox{$P\downarrow_{\mathcal N} x$ or $Q\downarrow_{\mathcal N} x$}}
		  {\binpar{P}{Q} \downarrow_{\mathcal N} x}

We write $P \Downarrow_{\mathcal N} x$ if there is $Q$ such that 
$P \wred Q$ and $Q \downarrow_{\mathcal N} x$.
\end{definition}

\begin{definition}
%\label{def.bbisim}
An  ${\mathcal N}$-\emph{barbed bisimulation} over a set of names, ${\mathcal N}$, is a symmetric binary relation 
${\mathcal S}_{\mathcal N}$ between agents such that $P\rel{S}_{\mathcal N}Q$ implies:
\begin{enumerate}
\item If $P \red P'$ then $Q \wred Q'$ and $P'\rel{S}_{\mathcal N} Q'$.
\item If $P\downarrow_{\mathcal N} x$, then $Q\Downarrow_{\mathcal N} x$.
\end{enumerate}
$P$ is ${\mathcal N}$-barbed bisimilar to $Q$, written
$P \wbbisim_{\mathcal N} Q$, if $P \rel{S}_{\mathcal N} Q$ for some ${\mathcal N}$-barbed bisimulation ${\mathcal S}_{\mathcal N}$.
\end{definition}

$\mathcal{R} \subseteq \pi \times \pi$

$P \mathcal{R} Q => \forall P'. P \red P' \Rightarrow \exists Q'. Q \red Q', P' \mathcal{R} Q'$

$P \vdash x \Rightarrow Q \vdash x$

\begin{mathpar}
  \inferrule*[lab=Out-barb]{x \nameeq y}{{y}!\langle{Q}\rangle \vdash x}
  \and
  \inferrule*[lab=Par-barb]{\mbox{$P\vdash x$ or $Q\vdash x$}}{\binpar{P}{Q} \vdash x}
\end{mathpar}

\subsubsection{Contexts}

One of the principle advantages of computational calculi like the
$\pi$-calculus is a well-defined notion of context,
contextual-equivalence and a correlation between
contextual-equivalence and notions of bisimulation. The notion of
context allows the decomposition of a process into (sub-)process and
its syntactic environment, its context. Thus, a context may be
thought of as a process with a ``hole'' (written $\Box$) in it. The
application of a context $M$ to a process $P$, written $M[P]$, is
tantamount to filling the hole in $M$ with $P$. In this paper we do
not need the full weight of this theory, but do make use of the notion
of context in the proof the main theorem. 

\begin{mathpar}
  \inferrule* [lab=summation] {} {{M_{M},M_{N}} \bc \Box \;|\; x.M_{A} \;|\; M_{M}+M_{N}}
  \and
  \inferrule* [lab=agent] {} {{M_{A}} \bc (\vec{x})M_{P} \;| \; \clift{P_0,\ldots,M_{P},\ldots,P_N}}
  \and \\
  \inferrule* [lab=process] {} {{M_{P}} \bc M_{N} \;| \;P|M_{P} }
\end{mathpar} 

\begin{mathpar}
  \inferrule* [lab=sychronization] {} {M_{N} \bc \Box \;|\; x?M_{F} \;|\; x!M_{C}}
  \and
  \inferrule* [lab=abstraction] {} {{M_{F}} \bc (x)M_{P} }
  \and
  \inferrule* [lab=concretion] {} {{M_{C}} \bc \langle M_{P} \rangle }
  \and \\
  \inferrule* [lab=process] {} {{M_{P}} \bc M_{N} \;| \;P|M_{P} }
\end{mathpar}

\begin{definition}[contextual application] Given a context $M$, and
  process $P$, we define the \emph{contextual application}, $M[P] :=
  M\{P/\Box\}$. That is, the contextual application of M to P is the
  substitution of $P$ for $\Box$ in $M$.
\end{definition}

$\meaningof{-} : L \to \mathcal{P}(\pi)$

\begin{mathpar}
  \inferrule* [lab=collection] {} {\meaningof{true} = \pi, \and \meaningof{~E} = \pi \setminus \meaningof{E}, \and \meaningof{E_{1} \& E_{2}} = \meaningof{E_{1}} \cap \meaningof{E_{2}}}
\end{mathpar}

\begin{mathpar}
  \inferrule* [lab=structure] {} {\meaningof{0} = \{ P \in \pi | P \equiv 0 \}, \and \\ \meaningof{E_1 | E_2} = \{ P \in \pi | P \equiv P_{1} | P_{2}, P_{1} \in \meaningof{E_{1}}, P_{2} \in \meaningof{E_2}\} }
\end{mathpar}

\begin{mathpar}
 \inferrule* [lab=behavior] {} {\meaningof{\langle a?b \rangle E} = \{ P \in \pi | P \equiv Q | u?(y)P', \\ \and \\\\ \and \\ \;\;\; u \in \meaningof{a}, \forall z.P'\{z/y\} \in \meaningof{E\{z/b\}}\}, \and \\ \meaningof{a!E} = \{ P \in \pi | P \equiv Q | x!\langle P' \rangle, x \in \meaningof{a} P' \in \meaningof{E}\} }
\end{mathpar}

\begin{mathpar}
 \inferrule* [lab=nominal] {} {\meaningof{\quotep{E}} = \{ \quotep{P} \in \quotep{\pi} | P \in \meaningof{E} \}, \and \meaningof{\quotep{P}} = \{ \quotep{Q} \in \quotep{\pi} | P \equiv Q \} \and \\ \meaningof{@\quotep{E}} = \{ P \in \pi | P \equiv @x, x \in \meaningof{E} \}}
\end{mathpar}

\begin{eqnarray*}
  \\
  \meaningof{-} : TS \to ST
\end{eqnarray*}

\begin{eqnarray*}
  \\
  L : TS \to ST
\end{eqnarray*}

\begin{eqnarray*}
  \\
  P \models E \iff P \in \meaningof{E}
\end{eqnarray*}

\begin{eqnarray*}
  P \approx_{L} Q \iff \forall E \in L. P \models E \iff Q \models E
\end{eqnarray*}

\begin{eqnarray*}
  P \approx_{K} Q
\end{eqnarray*}

\begin{eqnarray*}
  P \approx Q
\end{eqnarray*}

$\approx_{K} = \approx = \approx_{L}$

\subsubsection{Contextual duality}

Note that contexts extend the quotation operation to a family of
operations from processes to names. Given a context, $M$, we can
define a \emph{nominal context}, $\quotep{M}$ by $\quotep{M}[P] :=
\quotep{M[P]}$. To foreshadow what is to come we observe that these
operations enjoy a duality with processes very much like the duality
between vectors and maps from vectors to scalars.

Further, because the calculus is essentially higher-order, we have a
correspondence between contexts and processes. More specifically,
given a name $x$ and a context $M$ we can construct $M^{*}_{x}$ such
that 

\begin{mathpar}
  M^{*}_{x} | \lift{x}{P} \red M[P]
\end{mathpar}

namely,

\begin{mathpar}
  M^{*}_{x} := x?(u).M[\dropn{u}]
\end{mathpar}

The dependence of $M^{*}_{x}$ on a name makes it an abstraction, 

\begin{mathpar}
  M^{*} := (x)x?(u).M[\dropn{u}]
\end{mathpar}

\subsection{Additional notation}

It will sometimes be convenient to denote the process a name
quotes. We already have the notation $x = \quotep{P}$, but it will be
convenient to introduce an alternate notation, $\procn{x}$, when we
want to emphasize the connection to the use of the name. Note that, by
virtue of name equivalence, $\quotep{\procn{x}} \nameeq x$; so, the
notation is consistent with previous definitions.

Further, because names have structure it is possible to effect
substitutions on the basis of that structure. This means we need to
upgrade our notation for substitutions, which we accomplish by
adapting comprehension notation. Thus,

\begin{mathpar}
  P\{ y / x : x \in S \}
\end{mathpar}

is interpreted to mean the process derived from P by replacing (in a
capture-avoiding manner) each occurrence of $x$ in $S$ by $y$. For example,

\begin{mathpar}
  P\{ \quotep{\procn{x}|\procn{x}} / x : x \in \freenames{P} \}
\end{mathpar}

will replace each (occurrence) of a free name $x$ in $P$ by
$\quotep{\procn{x}|\procn{x}}$.

Also, we will avail ourselves of the notation $x^{L}$ and $x^{R}$ to
denote injections of a name into disjoint copies of the name
space. There are numerous ways to accomplish this. One example can be
found in \cite{MeredithR05}. This notation overloads to vectors of
names: $\vec{x}^{\pi} := (x_{i}^{\pi} \; : \; 0 \leq i < |\vec{x}| )$ where $\pi \in \{L,R\}$.

We also use $P^{\Box} := P|\Box$.

In \cite{MeredithR05} an interpretation of the new operator is
given. It turns out that there are several possible interpretations
all enjoying the requisite algebraic properties of the operator (see
\cite{milner91polyadicpi}). We will therefore make liberal use of
$(\nu\; \vec{x})P$.

% subsection the_syntax_and_semantics_of_the_notation_system (end)   

\input{qm2pi.qmops} 

\input{qm2pi.sterngerlach} 

\input{qm2pi.metric} 

% section concurrent_process_calculi (end)

%\input{qm2pi.proofsketch}

% section proof sketch (end)

%\input{qm2pi.slviaknots} 

% section spatial logic via knots (end)

\input{qm2pi.conclusion}

% section conclusion (end)

%\input{qm2pi.dtcodes} 

% section wiring algorithm (end)

\input{qm2pi.ack} 

% section acknowledgments (end)

\newpage


\bibliographystyle{plain}   
\bibliography{../../biblios/main.bib}

\input{qm2pi.rhodetails}

\end{document}



% section proof sketch (end)

%\section{Unlikely characters: spatial logic for
  knots}\label{sub:characteristic_formulae} % (fold)

Associated to the mobile process calculi are a family of logics known
as the Hennessy-Milner logics. These logics typically enjoy a
semantics interpreting formulae as sets of processes that when
factored through the encoding outlined above allows an identification
of classes of knots with logical formulae. In the context of this
encoding the sub-family known as the spatial logics \cite{CairesC03}
\cite{CairesC04} \cite{Caires04} are of particular interest providing
several important features for expressing and reasoning about
properties (i.e. classes) of knots. We hint here at how this may be done.

%\begin{description}
%\item [structural connectives] 
\subsubsection{Structural connectives} The spatial logics enjoy
structural connectives corresponding, at the logical level, to the
parallel composition ($P | Q$) and new name ($(\nu \; x)P$)
connectives for processes. As illustrated in the examples below, these
connectives are extremely expressive given the shape of our encoding.
%\item [decideable satisfaction]

\subsubsection{Decideable satisfaction}
In \cite{Caires04} the satisfaction relation is shown to be decideable
for a rich class of processes. It further turns out that the image of
the our encoding is a proper subset of that class. This result
provides the basis for an algorithm by which to search for knots
enjoying a given property.
%\item [characteristic formulae]

\subsubsection{Characteristic formulae}
In the same paper \cite{Caires04} , Caires presents a means of calculating
characteristic formulae, selecting equivalence classes of processes
up to a pre--specified depth limit on the support set of names. Composed with our
encoding, this characteristic formula can be used to select
characteristic formulae for knots.
%\end{description}

\subsubsection{Spatial logic formulae}

The grammar below (segmented for comprehension) summarizes the syntax
of spatial logic formulae. We employ illustrative examples in the
sequel to provide an intuitive understanding of their meaning
referring the reader to \cite{Caires04} for a more detailed explication
of the semantics.

\begin{mathpar}
  \inferrule* [lab=boolean] {} {{A,B} \bc T \;|\; \neg A \;|\; A \wedge B \;|\; \eta = \eta'}
  \and
  \inferrule* [lab=spatial] {} {|\; \pzero \;|\; A | B \;|\; x \text{\textregistered} A \;|\; \forall x . A \;|\;  H x . A}
  \and
  \inferrule* [lab=behavioral] {} {|\; \alpha . A}
  \and 
  \inferrule* [lab=recursion] {} {|\; X(\vec{u}) \;|\; \mu X(\vec{u}) . A}
  \and
  \inferrule* [lab=action] {} {\alpha \bc \langle x?(\vec{y}) \rangle \;|\; \langle x!(\vec{y}) \rangle \;|\; \langle \tau \rangle}
  \and 
  \inferrule* [lab=name] {} {\eta \bc x \;|\; \tau}
\end{mathpar} 

% subsection characteristic_formulae (end)   	 

\subsection{Example formulae}\label{sub:example_formulae_} % (fold)

\subsubsection{Crossing as formula.}
% 
% \begin{align*}
%   \frac{d}{dx} \sin x &= \cos x 
%   & \frac{d}{dx} e^x &= e^x \\
%   \frac{d}{dx} \cos x &= - \sin x 
%   & \frac{d}{dx} \log x &= \frac{1}{x} \\
% \end{align*} 

\begin{align*}
 \mu C(x_{0},x_{1},y_{0},y_{1},u).&(\langle x_{0}?(z) \rangle(\langle u! \rangle\langle y_{1}!z \rangle C(x_{0},x_{1},y_{0},y_{1},u)) & \\
  & \wedge \langle y_{1}?(z) \rangle (\langle u! \rangle \langle x_{0}!z \rangle C(x_{0},x_{1},y_{0},y_{1},u)) & \\
  & \wedge \langle x_{1}?(z) \rangle (\langle u? \rangle \langle y_{0}!z \rangle C(x_{0},x_{1},y_{0},y_{1},u)) & \\
  & \wedge \langle y_{0}?(z) \rangle (\langle u? \rangle \langle x_{1}!z \rangle C(x_{0},x_{1},y_{0},y_{1},u))) &
\end{align*}

The lexicographical similarity between the shape of this formulae and
the shape of definition of the process representing a crossing reveals
the intuitive meaning of this formulae. It describes the capabilities
of a process that has the right to represent a crossing. For example
it picks out processes that may perform an input on the port $x_0$ in
its initial menu of capabilities. What differentiates the formula
from the process, however, is that the crossing process is the
smallest candidate to satisfy the formula. Infinitely many other
processes -- with internal behavior hidden behind this interface, so
to speak -- also satisfy this formula. Even this simple formula,
then, can be seen to open a new view onto knots, providing a
computational interpretation of \emph{virtual} knots.

Note that this formula is derived by hand. A similar formula can be
derived by employing Caires' calculation of characteristic formula
\cite{Caires04} to the process representing a crossing. In light of
this discussion, we let
$\meaningof{C}_{\phi}(x0,x1,y0,y1,u)$ denote a formula specifying the
dynamics we wish to capture of a crossing. To guarantee we preserve
the shape of the interface and minimal semantics we demand that
$\meaningof{C}_{\phi}(x0,x1,y0,y1,u) \Rightarrow
\textbf{C}(x0,x1,y0,y1,u)$ where $\textbf{C}(x0,x1,y0,y1,u)$ denotes
the formula above.
                            
\subsubsection{Crossing number constraints.}
The moral content of the context lemma (Lemma \ref{context}) is that the notion of
``locality'' in the Reidemeister moves is effectively captured by the
parallel composition operator of the process calculus. This intuition
extends through the logic. Given a formula,
$\meaningof{C}_{\phi}(x0,x1,y0,y1,u)$, we can use the structural
connectives to specify constraints on crossing numbers, such as at
least $n$ crossings, or exactly $n$ crossings.
\begin{mathpar}
  \inferrule* [lab=at-least-n] {} { K^{\geq n}_{\phi}(\vec{xs},\vec{ys}) := \Pi_{i=0}^{n-1} Hu . \meaningof{C}_{\phi}(xs_i,ys_i,u) | T }
  \and 
  \inferrule* [lab=exactly-n] {} { K^{= n}_{\phi}(\vec{xs},\vec{ys}) := \Pi_{i=0}^{n-1} Hu . \meaningof{C}_{\phi}(xs_i,ys_i,u) | \neg (\forall x_0,y_0,x_1,y_1,u . \meaningof{C}_{\phi}(x_0,y_0,x_1,y_1,u) | T) }
\end{mathpar}

To round out this section, recall that the encoding of an $n$-crossing
knot decomposes into a parallel composition of $n$ \emph{copies} of a
crossing process together with a wiring harness. To specify different
knot classes with the same crossing number amounts to specifying
logical constraints on the wiring harness. In the interest of space,
we defer examples to a forthcoming paper. Suffice it to say that both
the conditions ``alternating knot'' and ``contains the tangle
corresponding to 5/3'' are expressible. For example, it is possible to
calculate the characteristic formula of a process corresponding to the
tangle 5/3 and conjoin it into the classifying formula via the
composition connective of the logic.

Finally, we wish to observe that it is entirely within reason to
contemplate a more domain-specific version of spatial logic tailored
to the shape of processes in the image of the encoding. Such a
domain-specific logic would have a better claim to the title formal
language of knot properties.

% subsection example_formulae_ (end)

% section knots_as_processes (end) 

% section spatial logic via knots (end)

\section{Conclusions and future work}

\paragraph{Testing physical space}
You, gentle reader, may wonder why of all the theorems to be proved
given this set up we pick the one above. In some sense it's hardly
central to quantum mechanics. We see it as central in the sense that
it firmly establishes a notion of physical space arising from a notion
of the equivalence of behavior. Relating bisimulation to a metric is a
big step forward, but one is faced with interpreting the relationship
of that metric space to something more physical. Quantum mechanical
notions of ``physical'' space are still far from intuitive, but by
relating this idea of distance as testing to calculations that predict
physical circumstances we are making a not insignificant step forward
toward an understanding of the physical space we inhabit as
essentially dynamic.

\paragraph{Effectivity and simulation}
One of the observations we have yet to make is that the entire program
spelled out here is effective. We have built various interpreters for
the reflective calculus at work in this interpretation. In principle,
then, we can simulate quantum mechanics on a computer. The place where
the simulation may lose fidelity is the infinitely branching summation
for the annihilator.

In this connection i also want to point out that the evaluation style
calculation of the inner product puts the non-determinism of the
summation right at the heart of measurement. This suggests that
Milner's original reduction-based formulation of the dynamics of his
calculi in terms of sums was not just notationally suggestive of a
notion of measure-and-continue but captured some significant part of
the physics.

\paragraph{Quantum continuations}
In light of this last observation i want to point out that the
predominant account of quantum mechanics is missing a key aspect of a
truly compositional story of the physical situation. In a real lab,
when a measurement is made the observation can be made to feed into
another device that then makes another measurement conditioned on the
results of the first. This means that after the superposition was
collapsed the entire experimental set up remained in
superposition. While QM offers a means of writing this down it doesn't
quite line up well with the well-trodden formulation of computation
and continuation that we see so succinctly expressed in Milner's
calculi. This suggests that there might be advantages to this account
of dynamics waiting to be explored.

\paragraph{Quantum logic}
In this connection, we also note that by virtue of having the
Hennessy-Milner construction, we can pull the construction through the
interpretation of QM. This gives us a natural candidate for a quantum
logic that enjoys an extremely tight connection with it's domain of
interpretation, making the construction much less ad hoc (rather it is
the image of functor!).

\paragraph{Quantum probabiity}
i have questions about the basis of the interpretation of inner
product as probability amplitude. In particular, using which
axiomatization of probability theory does the notion of probability
amplitude earn the right to be so dubbed? In other words, where is the
proof that the operation for calculating a probability amplitude (and
then squaring) satisfies the axioms of what it means to calculate a
probability? Even if such a proof exists (i have yet to find it in the
literature), i wonder if it might not be possible to turn things on
their heads. Can we view the calculation of the probability amplitude
as an axiomatization of probability? If so, then the definition we
give for calculating probability amplitude may provide the basis for
an \emph{effective} theory of probability.

\paragraph{Quantum vs ``biological'' information}
Finally, i want to conclude with a more philosophical observation. At
a recent workshop in which QM was a predominant topic i noticed
something about quantum information. The speaker was giving a riveting
discussion of axiomatic QM and showing how properties of ``no
cloning'' and ``no deleting'' emerged as consequences of the
axiomatization. Theorems of this form are necessary to give us a sense
of confidence that our axioms characterize the physical theory. What
struck me, though, was that if quantum information is neither erasable
nor replicable it is markedly different from \emph{life}. Two of the
things we know about life is that

\begin{itemize}
  \item it ends;
  \item to gain some measure of persistence, to transcend it's
    finitude it is imminently copyable.
\end{itemize}

Both of these qualities are summarized succinctly in the aphorism: all
flesh is grass. For me these two kinds of ``information'' -- call them
quantum and biological -- are end points on a spectrum of strategies
for persistence. At one end, we have those curious entities that enjoy
uniqueness and permanence; at the other, we have those who in the face
of a certain end and an uncertain present make a go of passing
something on. To me one of the more remarkable aspects of the latter
strategy is that in the presence of noise (and certain features of
copying) we get a kind of dynamism, a chance for improvement against a
given persistent condition.

% subsection other_calculi_other_bisimulations_and_geometry_as_behavior (end)




% section conclusion (end)

%\documentclass[12pt]{llncs}
%\documentclass{jktr}

\usepackage[pdftex]{hyperref}                   
\usepackage {listings}
\usepackage {mathpartir}
\usepackage{bcprules}
%\usepackage{listings}
                       
\usepackage{graphicx} 
%\usepackage[margins=2.5cm,nohead,nofoot]{geometry}
%\usepackage{geometry}
\usepackage{amsfonts}
\usepackage{amstext}
\usepackage{latexsym}
\usepackage{amssymb}
\usepackage{color}


%\include{myPreamble}
\include{qm2pi.local} 

%\ifpdf
%\usepackage[pdftex]{graphicx}
%\else
%\usepackage{graphicx}
%\fi

 % \ifpdf
%  \usepackage{pdfsync}
%  \if


%\title{Brief Article}
%\author{David F. Snyder}
%\author{L.G. Meredith}

%\address{Dept. of Math., Texas State University--San Marcos, San Marcos, TX 78666}
       
\pagestyle{empty}


\begin{document}

\lstset{language=[Objective]Caml,frame=shadowbox}

\input{qm2pi.front}

% section front matter (end)

\input{qm2pi.intro} 
 
% section introduction (end)

% \input{qm2pi.knotations} 

% section notation (end)

\input{qm2pi.process.calculi} 

% section concurrent_process_calculi_and_spatial_logics_ (end)
    
%\input{qm2pi.knots2pi} 

%\input{qm2pi.trefoil} 

%\input{qm2pi.mainthm} 

% subsection basic_interpretation (end)

%\input{qm2pi.rho.presentation} 
\subsection{The syntax and semantics of the notation system}\label{sub:the_syntax_and_semantics_of_the_notation_system} % (fold)

We now summarize a technical presentation of the calculus that
embodies our theory of dynamics. The typical presentation of such a
calculus follows the style of giving generators and relations on
them. The grammar, below, describing term constructors, freely
generates the set of processes, $\Proc$. This set is then quotiented
by a relation known as structural congruence and it is over this set
that the notion of dynamics is expressed. This presentation is
essentially that of \cite{MeredithR05} with the addition of
polyadicity and summation. For readability we have relegated some of
the technical subtleties to an appendix.

\subsubsection{Process grammar}\label{subsub:process_grammar}

\begin{mathpar}
  \inferrule* [lab=synchronization] {} {{M} \bc \pzero \;|\; x?F \;|\; x!C }
  \and
  \inferrule* [lab=abstraction] {} {{F} \bc (x)P}
  \and
  \inferrule* [lab=concretion] {} {{C} \bc \langle Q \rangle}
  \and
  \inferrule* [lab=process] {} {{P,Q} \bc M \;| \;P|Q \;|\; @{x}}
  \and
  \inferrule* [lab=name] {} {{x} \bc \quotep{P}}
\end{mathpar} 

Note that $\vec{x}$ (resp. $\vec{P}$) denotes a vector of names
(resp. processes) of length $|\vec{x}|$ (resp. $|\vec{P}|$). We adopt
the following useful abbreviations.

\begin{mathpar}
   x?(\vec{y}).P := x.(\vec{y})P \and  x\clift{\vec{P}} := x.\clift{\vec{P}}
   \and x!(y) := \lift{x}{\dropn{y}}
   \and \Pi_{i=0}^{n-1}P_i := P_0 | \ldots | P_{n-1}
\end{mathpar}

\subsubsection{Structural congruence}

\paragraph{Free and bound names and alpha-equivalence.} At the
core of structural equivalence is alpha-equivalence which identifies
process that are the same up to a change of variable. Formally, we
recognize the distinction between free and bound names. The free names
of a process, $\freenames{P}$, may be calculated recursively as
follows:

\begin{mathpar}
\freenames{\pzero} := \emptyset
  \and \\
  \freenames{x?(y).P} := \{ x \} \cup (\freenames{P} \setminus \{ y \})
  \and 
  \freenames{x!\langle P \rangle} := \{ x \} \cup \{ P \} 
  \and \\
  \freenames{P|Q} := \freenames{P} \cup \freenames{Q}
  \and \\
  \freenames{@{x}} := \{ x \}
\end{mathpar}

$\pi$
$\quotep{\pi}$

$\freenames{-} : \pi \to \mathcal{P}(\quotep{\pi})$

\begin{eqnarray*}
  \freenames{\pzero} & := & \emptyset \\
  \freenames{x?(y).P} & := & \{ x \} \cup (\freenames{P} \setminus \{ y \}) \\
  \freenames{x!\langle P \rangle} & := & \{ x \} \cup \{ P \} \\
  \freenames{P|Q} & := & \freenames{P} \cup \freenames{Q} \\
  \freenames{\dropn{x}} & := & \{ x \}
\end{eqnarray*}

The bound names of a process, $\boundnames{P}$, are those names occurring in $P$
that are not free. For example, in $x?(y).0$, the name $x$ is free, while $y$ is bound.

\begin{mathpar}
  \inferrule* [lab=monoidal-laws] {} { P|Q \equiv Q|P \and P|0 \equiv P \and P|(Q|R) \equiv (P|Q)|R }
\end{mathpar}

\begin{mathpar}
  \inferrule* [lab=alpha-equivalence] {} { (x)P \equiv (y)P\{y/x\} \and y \not\in \freenames{P} }
\end{mathpar}

\begin{definition}
Then two processes, $P,Q$, are alpha-equivalent if $P = Q\{\vec{y}/\vec{x}\}$ for
some $\vec{x} \in \boundnames{Q},\vec{y} \in \boundnames{P}$, where $Q\{\vec{y}/\vec{x}\}$
denotes the capture-avoiding substitution of $\vec{y}$ for $\vec{x}$ in $Q$.
\end{definition}

\begin{definition}
  The {\em structural congruence} \cite{SangiorgiWalker} , $\equiv$,
  between processes is the least congruence containing
  alpha-equivalence, satisfying the abelian monoid laws
  (associativity, commutativity and $\pzero$ as identity) for parallel
  composition $|$ and for summation $+$.
\end{definition}

\subsection{Name equivalence}

We take name equivalence, written $\nameeq$, to be the smallest
equivalence relation generated by the following rules.

\begin{mathpar}
\inferrule*[lab=Quote-drop]
{ }
{ \quotep{@{x}} \nameeq x }

\inferrule*[lab=Struct-equiv]
{ P \scong Q }
{ \quotep{P} \nameeq \quotep{Q} }
\end{mathpar}

The astute reader will have noticed that the mutual recursion of names
and processes imposes a mutual recursion on alpha-equivalence and
structural equivalence via name-equivalence. Fortunately, all of this
works out pleasantly and we may calculate in the natural way, free of
concern. The reader interested in the details is referred to the
appendix \ref{appendix:rho_details}.

\subsection{Substitution}

We use $\Proc$ for the set of processes, $\QProc$ for the set of
names, and $\id{\{}\vec{y} / \vec{x} \id{\}}$ to denote partial maps,
$s : \QProc \rightarrow \QProc$. A map, $s$ lifts, uniquely, to a map
on process terms, $\widehat{s} : \Proc \rightarrow \Proc$ by the
following equations.

\begin{mathpar}
  (0) \psubstp{Q}{P} := 0 \\
  (R \juxtap S) \psubstp{Q}{P}
  :=    
  (R)\psubstp{Q}{P} \juxtap (S) \psubstp{Q}{P} \\
  (x?(y).R) \psubstp{Q}{P}    
  :=    
  (x)\substp{Q}{P} (z)\concat( (R \psubstn{z}{y}) \psubstp{Q}{P} ) \\
  (\lift{x}{R}) \psubstp{Q}{P}  
  :=
  \lift{(x)\substp{Q}{P}}{ R \psubstp{Q}{P} } \\
%   (\dropn{x})  \psubstp{Q}{P}       
%   := 
%   \left\{ 
%     \begin{array}{ccc} 
%       \dropn{\quotep{Q}} & & x \nameeq \quotep{P} \\
%       \dropn{x} & & otherwise \\
%     \end{array}
%   \right. 
  (\dropn{x})  \psubstp{Q}{P}       
  := 
  \left\{ 
    \begin{array}{ccc} 
      Q & & x \nameeq \quotep{P} \\
      \dropn{x} & & otherwise \\
    \end{array}
  \right.
\end{mathpar}
 

where

\begin{eqnarray}
  (x)\id{\{} \lpquote Q \rpquote / \lpquote P \rpquote \id{\}}            = 
  \left\{ 
    \begin{array}{ccc}
      \lpquote Q \rpquote & & x \nameeq \lpquote P \rpquote \\
      x & & otherwise \\
    \end{array}
  \right. \nonumber
\end{eqnarray}

and $z$ is chosen distinct from $\quotep{P}$, $\quotep{Q}$, the free
names in $Q$, and all the names in $R$. Our $\alpha$-equivalence will
be built in the standard way from this substitution.

\begin{remark}\label{rem:no_self_referential_names}
  One consequence of these definitions is that $\forall P. \quotep{P}
  \not\in \freenames{P}$.
\end{remark}

\subsection{ Dynamic quote: an example }

Anticipating something of what's to come, consider applying the
substitution, $\widehat{\id{\{}u / z \id{\}}}$, to the following pair
of processes, $\lift{w}{y!(z)}$ and $w[ \lpquote y!(z) \rpquote ]$.

\begin{eqnarray}
	\lift{w}{y!(z)}\widehat{\id{\{}u / z \id{\}}}
		& = &
		\lift{w}{y!(u)} \nonumber\\
	w[ \lpquote y!(z) \rpquote ] \widehat{ \id{\{}u / z \id{\}} }
		& = &
		w[ \lpquote y!(z) \rpquote ] \nonumber
\end{eqnarray}

Because the body of the process between quotes is impervious to
substitution, we get radically different answers. In fact, by
examining the first process in an input context,
e.g. $x?(z).\lift{w}{y!(z)}$, we see that the process under the lift
operator may be shaped by prefixed inputs binding a name inside it. In
this sense, the lift operator will be seen as a way to dynamically
construct processes before reifying them as names.

Finally equipped with these standard features we can present the
dynamics of the calculus.

\subsubsection{Operational semantics} 

Finally, we introduce the computational dynamics. What marks these
algebras as distinct from other more traditionally studied algebraic
structures, e.g. vector spaces or polynomial rings, is the manner in
which dynamics is captured. In traditional structures, dynamics is typically
expressed through morphisms between such structures, as in linear maps
between vector spaces or morphisms between rings. In algebras
associated with the semantics of computation, the dynamics is
expressed as part of the algebraic structure itself, through a
reduction reduction relation typically denoted by $\red$. Below, we
give a recursive presentation of this relation for the calculus used
in the encoding.

$\red \subseteq \pi \times \pi$
$\red : \pi \to \mathcal{P}(\pi)$

\begin{mathpar}
  \inferrule* [lab=Comm] { \textsf{match}( x_{src}, x_{trgt} ) } { x_{trgt}?(y)P \; | \; x_{src}!\langle {Q} \rangle \red P\{\quotep{Q}/y}\} }
  \and \\
  \inferrule* [lab=Par] {{P} \red {P}'} {{{P} | {Q}} \red {{P}' | {Q}}}
  \and
  \inferrule* [lab=Equiv]{{{P} \scong {P}'} \andalso {{P}' \red {Q}'} \andalso {{Q}' \scong {Q}}}{{P} \red {Q}}
\end{mathpar}

\begin{eqnarray*}
  match_{\equiv} (\quotep{P},\quotep{Q}) & := & P \equiv Q \\
  match_{\dagger}(\quotep{P},\quotep{Q}) & := & \forall R. P|Q \red^{*} R => R \red^{*} 0 \\
  match_{K}(\quotep{P},\quotep{Q}) & := & K \mbox{ for some context } K
\end{eqnarray*}

$u?(x)P | u!\langle Q \rangle \red P\{\quotep{Q}/x\}$

%We write $\wred$ for $\red^*$, and $P\red$ if $\exists Q $ such that $ P \red Q$.
We write $P\red$ if $\exists Q $ such that $ P \red Q$ and $P\not\red$, otherwise.

\section{Replication}

As mentioned before, it is known that replication (and hence
recursion) can be implemented in a higher-order process algebra
\cite{SangiorgiWalker}. As our first example of calculation with the
machinery thus far presented we give the construction explicitly in
the {\rhoc}.

\begin{eqnarray}
	D_{x} & := & \prefix{x}{y}{(\binpar{\outputp{x}{y}}{@{y}})} \nonumber\\
	\bangp_{x}{P} & := & \binpar{{x}!\langle{\binpar{D_{x}}{P}}\rangle}{D_{x}} \nonumber
\end{eqnarray}

\begin{eqnarray}
	\bangp_{x}{P} & & \nonumber\\
	=
	& {x}!\langle{(\prefix{x}{y}{(\outputp{x}{y} | @{y})) | P}}\rangle 
	      | \prefix{x}{y}{(\outputp{x}{y} | @{y})} & \nonumber\\
	\red
	& (\outputp{x}{y} | @{y})\substn{\quotep{(\prefix{x}{y}{(@{y} | \outputp{x}{y})) | P}}}{y} & \nonumber\\
	=
	& \outputp{x}{\quotep{(\prefix{x}{y}{(\outputp{x}{y} | @{y})) | P}}}
	  | {(\prefix{x}{y}{(\outputp{x}{y} | @{y})) | P}} & \nonumber\\
	\red
	& \ldots & \nonumber\\
	\red^*
	& P | P | \ldots & \nonumber
\end{eqnarray}

Of course, this encoding, as an implementation, runs away, unfolding
$\bangp{P}$ eagerly. A lazier and more implementable replication
operator, restricted to input-guarded processes, may be obtained as follows.

\begin{eqnarray}
\bangp{\prefix{u}{v}{P}} 
	:= 
	\binpar{\lift{x}{\prefix{u}{v}{(\binpar{D(x)}{P})}}}{D(x)} \nonumber
\end{eqnarray}

\begin{remark}
  Note that the lazier definition still does not deal with summation
  or mixed summation (i.e. sums over input and output). The reader is
  invited to construct definitions of replication that deal with these
  features. 

  Further, the definitions are parameterized in a name, $x$. Can you,
  gentle reader, make a definition that eliminates this parameter and
  guarantees no accidental interaction between the replication
  machinery and the process being replicated -- i.e. no accidental
  sharing of names used by the process to get its work done and the
  name(s) used by the replication to effect copying. This latter
  revision of the definition of replication is crucial to obtaining
  the expected identity $!!P \sim !P$.
\end{remark}

\begin{remark}\label{rem:paradoxical_combinator}
  The reader familiar with the lambda calculus will have noticed the
  similarity between $D$ and the paradoxical combinator.

  [Ed. note: the existence of this seems to suggest we have to be more
  restrictive on the set of processes and names we admit if we are to
  support no-cloning.]
\end{remark}

\subsubsection{Bisimulation}

The computational dynamics gives rise to another kind of equivalence,
the equivalence of computational behavior. As previously mentioned
this is typically captured \emph{via} some form of bisimulation.

% The notion we use in this paper is weak barbed bisimulation
% \cite{milner91polyadicpi}.

The notion we use in this paper is derived from weak barbed
bisimulation \cite{milner91polyadicpi}. 

\begin{definition}
An \emph{observation relation}, $\downarrow_{\mathcal N}$, over a set
of names, $\mathcal N$, is the smallest relation satisfying the rules
below.

\infrule[Out-barb]{y \in {\mathcal N}, \; x \nameeq y}
		  {\outputp{x}{v} \downarrow_{\mathcal N} x}
\infrule[Par-barb]{\mbox{$P\downarrow_{\mathcal N} x$ or $Q\downarrow_{\mathcal N} x$}}
		  {\binpar{P}{Q} \downarrow_{\mathcal N} x}

We write $P \Downarrow_{\mathcal N} x$ if there is $Q$ such that 
$P \wred Q$ and $Q \downarrow_{\mathcal N} x$.
\end{definition}

\begin{definition}
%\label{def.bbisim}
An  ${\mathcal N}$-\emph{barbed bisimulation} over a set of names, ${\mathcal N}$, is a symmetric binary relation 
${\mathcal S}_{\mathcal N}$ between agents such that $P\rel{S}_{\mathcal N}Q$ implies:
\begin{enumerate}
\item If $P \red P'$ then $Q \wred Q'$ and $P'\rel{S}_{\mathcal N} Q'$.
\item If $P\downarrow_{\mathcal N} x$, then $Q\Downarrow_{\mathcal N} x$.
\end{enumerate}
$P$ is ${\mathcal N}$-barbed bisimilar to $Q$, written
$P \wbbisim_{\mathcal N} Q$, if $P \rel{S}_{\mathcal N} Q$ for some ${\mathcal N}$-barbed bisimulation ${\mathcal S}_{\mathcal N}$.
\end{definition}

$\mathcal{R} \subseteq \pi \times \pi$

$P \mathcal{R} Q => \forall P'. P \red P' \Rightarrow \exists Q'. Q \red Q', P' \mathcal{R} Q'$

$P \vdash x \Rightarrow Q \vdash x$

\begin{mathpar}
  \inferrule*[lab=Out-barb]{x \nameeq y}{{y}!\langle{Q}\rangle \vdash x}
  \and
  \inferrule*[lab=Par-barb]{\mbox{$P\vdash x$ or $Q\vdash x$}}{\binpar{P}{Q} \vdash x}
\end{mathpar}

\subsubsection{Contexts}

One of the principle advantages of computational calculi like the
$\pi$-calculus is a well-defined notion of context,
contextual-equivalence and a correlation between
contextual-equivalence and notions of bisimulation. The notion of
context allows the decomposition of a process into (sub-)process and
its syntactic environment, its context. Thus, a context may be
thought of as a process with a ``hole'' (written $\Box$) in it. The
application of a context $M$ to a process $P$, written $M[P]$, is
tantamount to filling the hole in $M$ with $P$. In this paper we do
not need the full weight of this theory, but do make use of the notion
of context in the proof the main theorem. 

\begin{mathpar}
  \inferrule* [lab=summation] {} {{M_{M},M_{N}} \bc \Box \;|\; x.M_{A} \;|\; M_{M}+M_{N}}
  \and
  \inferrule* [lab=agent] {} {{M_{A}} \bc (\vec{x})M_{P} \;| \; \clift{P_0,\ldots,M_{P},\ldots,P_N}}
  \and \\
  \inferrule* [lab=process] {} {{M_{P}} \bc M_{N} \;| \;P|M_{P} }
\end{mathpar} 

\begin{mathpar}
  \inferrule* [lab=sychronization] {} {M_{N} \bc \Box \;|\; x?M_{F} \;|\; x!M_{C}}
  \and
  \inferrule* [lab=abstraction] {} {{M_{F}} \bc (x)M_{P} }
  \and
  \inferrule* [lab=concretion] {} {{M_{C}} \bc \langle M_{P} \rangle }
  \and \\
  \inferrule* [lab=process] {} {{M_{P}} \bc M_{N} \;| \;P|M_{P} }
\end{mathpar}

\begin{definition}[contextual application] Given a context $M$, and
  process $P$, we define the \emph{contextual application}, $M[P] :=
  M\{P/\Box\}$. That is, the contextual application of M to P is the
  substitution of $P$ for $\Box$ in $M$.
\end{definition}

$\meaningof{-} : L \to \mathcal{P}(\pi)$

\begin{mathpar}
  \inferrule* [lab=collection] {} {\meaningof{true} = \pi, \and \meaningof{~E} = \pi \setminus \meaningof{E}, \and \meaningof{E_{1} \& E_{2}} = \meaningof{E_{1}} \cap \meaningof{E_{2}}}
\end{mathpar}

\begin{mathpar}
  \inferrule* [lab=structure] {} {\meaningof{0} = \{ P \in \pi | P \equiv 0 \}, \and \\ \meaningof{E_1 | E_2} = \{ P \in \pi | P \equiv P_{1} | P_{2}, P_{1} \in \meaningof{E_{1}}, P_{2} \in \meaningof{E_2}\} }
\end{mathpar}

\begin{mathpar}
 \inferrule* [lab=behavior] {} {\meaningof{\langle a?b \rangle E} = \{ P \in \pi | P \equiv Q | u?(y)P', \\ \and \\\\ \and \\ \;\;\; u \in \meaningof{a}, \forall z.P'\{z/y\} \in \meaningof{E\{z/b\}}\}, \and \\ \meaningof{a!E} = \{ P \in \pi | P \equiv Q | x!\langle P' \rangle, x \in \meaningof{a} P' \in \meaningof{E}\} }
\end{mathpar}

\begin{mathpar}
 \inferrule* [lab=nominal] {} {\meaningof{\quotep{E}} = \{ \quotep{P} \in \quotep{\pi} | P \in \meaningof{E} \}, \and \meaningof{\quotep{P}} = \{ \quotep{Q} \in \quotep{\pi} | P \equiv Q \} \and \\ \meaningof{@\quotep{E}} = \{ P \in \pi | P \equiv @x, x \in \meaningof{E} \}}
\end{mathpar}

\begin{eqnarray*}
  \\
  \meaningof{-} : TS \to ST
\end{eqnarray*}

\begin{eqnarray*}
  \\
  L : TS \to ST
\end{eqnarray*}

\begin{eqnarray*}
  \\
  P \models E \iff P \in \meaningof{E}
\end{eqnarray*}

\begin{eqnarray*}
  P \approx_{L} Q \iff \forall E \in L. P \models E \iff Q \models E
\end{eqnarray*}

\begin{eqnarray*}
  P \approx_{K} Q
\end{eqnarray*}

\begin{eqnarray*}
  P \approx Q
\end{eqnarray*}

$\approx_{K} = \approx = \approx_{L}$

\subsubsection{Contextual duality}

Note that contexts extend the quotation operation to a family of
operations from processes to names. Given a context, $M$, we can
define a \emph{nominal context}, $\quotep{M}$ by $\quotep{M}[P] :=
\quotep{M[P]}$. To foreshadow what is to come we observe that these
operations enjoy a duality with processes very much like the duality
between vectors and maps from vectors to scalars.

Further, because the calculus is essentially higher-order, we have a
correspondence between contexts and processes. More specifically,
given a name $x$ and a context $M$ we can construct $M^{*}_{x}$ such
that 

\begin{mathpar}
  M^{*}_{x} | \lift{x}{P} \red M[P]
\end{mathpar}

namely,

\begin{mathpar}
  M^{*}_{x} := x?(u).M[\dropn{u}]
\end{mathpar}

The dependence of $M^{*}_{x}$ on a name makes it an abstraction, 

\begin{mathpar}
  M^{*} := (x)x?(u).M[\dropn{u}]
\end{mathpar}

\subsection{Additional notation}

It will sometimes be convenient to denote the process a name
quotes. We already have the notation $x = \quotep{P}$, but it will be
convenient to introduce an alternate notation, $\procn{x}$, when we
want to emphasize the connection to the use of the name. Note that, by
virtue of name equivalence, $\quotep{\procn{x}} \nameeq x$; so, the
notation is consistent with previous definitions.

Further, because names have structure it is possible to effect
substitutions on the basis of that structure. This means we need to
upgrade our notation for substitutions, which we accomplish by
adapting comprehension notation. Thus,

\begin{mathpar}
  P\{ y / x : x \in S \}
\end{mathpar}

is interpreted to mean the process derived from P by replacing (in a
capture-avoiding manner) each occurrence of $x$ in $S$ by $y$. For example,

\begin{mathpar}
  P\{ \quotep{\procn{x}|\procn{x}} / x : x \in \freenames{P} \}
\end{mathpar}

will replace each (occurrence) of a free name $x$ in $P$ by
$\quotep{\procn{x}|\procn{x}}$.

Also, we will avail ourselves of the notation $x^{L}$ and $x^{R}$ to
denote injections of a name into disjoint copies of the name
space. There are numerous ways to accomplish this. One example can be
found in \cite{MeredithR05}. This notation overloads to vectors of
names: $\vec{x}^{\pi} := (x_{i}^{\pi} \; : \; 0 \leq i < |\vec{x}| )$ where $\pi \in \{L,R\}$.

We also use $P^{\Box} := P|\Box$.

In \cite{MeredithR05} an interpretation of the new operator is
given. It turns out that there are several possible interpretations
all enjoying the requisite algebraic properties of the operator (see
\cite{milner91polyadicpi}). We will therefore make liberal use of
$(\nu\; \vec{x})P$.

% subsection the_syntax_and_semantics_of_the_notation_system (end)   

\input{qm2pi.qmops} 

\input{qm2pi.sterngerlach} 

\input{qm2pi.metric} 

% section concurrent_process_calculi (end)

%\input{qm2pi.proofsketch}

% section proof sketch (end)

%\input{qm2pi.slviaknots} 

% section spatial logic via knots (end)

\input{qm2pi.conclusion}

% section conclusion (end)

%\input{qm2pi.dtcodes} 

% section wiring algorithm (end)

\input{qm2pi.ack} 

% section acknowledgments (end)

\newpage


\bibliographystyle{plain}   
\bibliography{../../biblios/main.bib}

\input{qm2pi.rhodetails}

\end{document}

 

% section wiring algorithm (end)

\documentclass[12pt]{llncs}
%\documentclass{jktr}

\usepackage[pdftex]{hyperref}                   
\usepackage {listings}
\usepackage {mathpartir}
\usepackage{bcprules}
%\usepackage{listings}
                       
\usepackage{graphicx} 
%\usepackage[margins=2.5cm,nohead,nofoot]{geometry}
%\usepackage{geometry}
\usepackage{amsfonts}
\usepackage{amstext}
\usepackage{latexsym}
\usepackage{amssymb}
\usepackage{color}


%\include{myPreamble}
\include{qm2pi.local} 

%\ifpdf
%\usepackage[pdftex]{graphicx}
%\else
%\usepackage{graphicx}
%\fi

 % \ifpdf
%  \usepackage{pdfsync}
%  \if


%\title{Brief Article}
%\author{David F. Snyder}
%\author{L.G. Meredith}

%\address{Dept. of Math., Texas State University--San Marcos, San Marcos, TX 78666}
       
\pagestyle{empty}


\begin{document}

\lstset{language=[Objective]Caml,frame=shadowbox}

\input{qm2pi.front}

% section front matter (end)

\input{qm2pi.intro} 
 
% section introduction (end)

% \input{qm2pi.knotations} 

% section notation (end)

\input{qm2pi.process.calculi} 

% section concurrent_process_calculi_and_spatial_logics_ (end)
    
%\input{qm2pi.knots2pi} 

%\input{qm2pi.trefoil} 

%\input{qm2pi.mainthm} 

% subsection basic_interpretation (end)

%\input{qm2pi.rho.presentation} 
\subsection{The syntax and semantics of the notation system}\label{sub:the_syntax_and_semantics_of_the_notation_system} % (fold)

We now summarize a technical presentation of the calculus that
embodies our theory of dynamics. The typical presentation of such a
calculus follows the style of giving generators and relations on
them. The grammar, below, describing term constructors, freely
generates the set of processes, $\Proc$. This set is then quotiented
by a relation known as structural congruence and it is over this set
that the notion of dynamics is expressed. This presentation is
essentially that of \cite{MeredithR05} with the addition of
polyadicity and summation. For readability we have relegated some of
the technical subtleties to an appendix.

\subsubsection{Process grammar}\label{subsub:process_grammar}

\begin{mathpar}
  \inferrule* [lab=synchronization] {} {{M} \bc \pzero \;|\; x?F \;|\; x!C }
  \and
  \inferrule* [lab=abstraction] {} {{F} \bc (x)P}
  \and
  \inferrule* [lab=concretion] {} {{C} \bc \langle Q \rangle}
  \and
  \inferrule* [lab=process] {} {{P,Q} \bc M \;| \;P|Q \;|\; @{x}}
  \and
  \inferrule* [lab=name] {} {{x} \bc \quotep{P}}
\end{mathpar} 

Note that $\vec{x}$ (resp. $\vec{P}$) denotes a vector of names
(resp. processes) of length $|\vec{x}|$ (resp. $|\vec{P}|$). We adopt
the following useful abbreviations.

\begin{mathpar}
   x?(\vec{y}).P := x.(\vec{y})P \and  x\clift{\vec{P}} := x.\clift{\vec{P}}
   \and x!(y) := \lift{x}{\dropn{y}}
   \and \Pi_{i=0}^{n-1}P_i := P_0 | \ldots | P_{n-1}
\end{mathpar}

\subsubsection{Structural congruence}

\paragraph{Free and bound names and alpha-equivalence.} At the
core of structural equivalence is alpha-equivalence which identifies
process that are the same up to a change of variable. Formally, we
recognize the distinction between free and bound names. The free names
of a process, $\freenames{P}$, may be calculated recursively as
follows:

\begin{mathpar}
\freenames{\pzero} := \emptyset
  \and \\
  \freenames{x?(y).P} := \{ x \} \cup (\freenames{P} \setminus \{ y \})
  \and 
  \freenames{x!\langle P \rangle} := \{ x \} \cup \{ P \} 
  \and \\
  \freenames{P|Q} := \freenames{P} \cup \freenames{Q}
  \and \\
  \freenames{@{x}} := \{ x \}
\end{mathpar}

$\pi$
$\quotep{\pi}$

$\freenames{-} : \pi \to \mathcal{P}(\quotep{\pi})$

\begin{eqnarray*}
  \freenames{\pzero} & := & \emptyset \\
  \freenames{x?(y).P} & := & \{ x \} \cup (\freenames{P} \setminus \{ y \}) \\
  \freenames{x!\langle P \rangle} & := & \{ x \} \cup \{ P \} \\
  \freenames{P|Q} & := & \freenames{P} \cup \freenames{Q} \\
  \freenames{\dropn{x}} & := & \{ x \}
\end{eqnarray*}

The bound names of a process, $\boundnames{P}$, are those names occurring in $P$
that are not free. For example, in $x?(y).0$, the name $x$ is free, while $y$ is bound.

\begin{mathpar}
  \inferrule* [lab=monoidal-laws] {} { P|Q \equiv Q|P \and P|0 \equiv P \and P|(Q|R) \equiv (P|Q)|R }
\end{mathpar}

\begin{mathpar}
  \inferrule* [lab=alpha-equivalence] {} { (x)P \equiv (y)P\{y/x\} \and y \not\in \freenames{P} }
\end{mathpar}

\begin{definition}
Then two processes, $P,Q$, are alpha-equivalent if $P = Q\{\vec{y}/\vec{x}\}$ for
some $\vec{x} \in \boundnames{Q},\vec{y} \in \boundnames{P}$, where $Q\{\vec{y}/\vec{x}\}$
denotes the capture-avoiding substitution of $\vec{y}$ for $\vec{x}$ in $Q$.
\end{definition}

\begin{definition}
  The {\em structural congruence} \cite{SangiorgiWalker} , $\equiv$,
  between processes is the least congruence containing
  alpha-equivalence, satisfying the abelian monoid laws
  (associativity, commutativity and $\pzero$ as identity) for parallel
  composition $|$ and for summation $+$.
\end{definition}

\subsection{Name equivalence}

We take name equivalence, written $\nameeq$, to be the smallest
equivalence relation generated by the following rules.

\begin{mathpar}
\inferrule*[lab=Quote-drop]
{ }
{ \quotep{@{x}} \nameeq x }

\inferrule*[lab=Struct-equiv]
{ P \scong Q }
{ \quotep{P} \nameeq \quotep{Q} }
\end{mathpar}

The astute reader will have noticed that the mutual recursion of names
and processes imposes a mutual recursion on alpha-equivalence and
structural equivalence via name-equivalence. Fortunately, all of this
works out pleasantly and we may calculate in the natural way, free of
concern. The reader interested in the details is referred to the
appendix \ref{appendix:rho_details}.

\subsection{Substitution}

We use $\Proc$ for the set of processes, $\QProc$ for the set of
names, and $\id{\{}\vec{y} / \vec{x} \id{\}}$ to denote partial maps,
$s : \QProc \rightarrow \QProc$. A map, $s$ lifts, uniquely, to a map
on process terms, $\widehat{s} : \Proc \rightarrow \Proc$ by the
following equations.

\begin{mathpar}
  (0) \psubstp{Q}{P} := 0 \\
  (R \juxtap S) \psubstp{Q}{P}
  :=    
  (R)\psubstp{Q}{P} \juxtap (S) \psubstp{Q}{P} \\
  (x?(y).R) \psubstp{Q}{P}    
  :=    
  (x)\substp{Q}{P} (z)\concat( (R \psubstn{z}{y}) \psubstp{Q}{P} ) \\
  (\lift{x}{R}) \psubstp{Q}{P}  
  :=
  \lift{(x)\substp{Q}{P}}{ R \psubstp{Q}{P} } \\
%   (\dropn{x})  \psubstp{Q}{P}       
%   := 
%   \left\{ 
%     \begin{array}{ccc} 
%       \dropn{\quotep{Q}} & & x \nameeq \quotep{P} \\
%       \dropn{x} & & otherwise \\
%     \end{array}
%   \right. 
  (\dropn{x})  \psubstp{Q}{P}       
  := 
  \left\{ 
    \begin{array}{ccc} 
      Q & & x \nameeq \quotep{P} \\
      \dropn{x} & & otherwise \\
    \end{array}
  \right.
\end{mathpar}
 

where

\begin{eqnarray}
  (x)\id{\{} \lpquote Q \rpquote / \lpquote P \rpquote \id{\}}            = 
  \left\{ 
    \begin{array}{ccc}
      \lpquote Q \rpquote & & x \nameeq \lpquote P \rpquote \\
      x & & otherwise \\
    \end{array}
  \right. \nonumber
\end{eqnarray}

and $z$ is chosen distinct from $\quotep{P}$, $\quotep{Q}$, the free
names in $Q$, and all the names in $R$. Our $\alpha$-equivalence will
be built in the standard way from this substitution.

\begin{remark}\label{rem:no_self_referential_names}
  One consequence of these definitions is that $\forall P. \quotep{P}
  \not\in \freenames{P}$.
\end{remark}

\subsection{ Dynamic quote: an example }

Anticipating something of what's to come, consider applying the
substitution, $\widehat{\id{\{}u / z \id{\}}}$, to the following pair
of processes, $\lift{w}{y!(z)}$ and $w[ \lpquote y!(z) \rpquote ]$.

\begin{eqnarray}
	\lift{w}{y!(z)}\widehat{\id{\{}u / z \id{\}}}
		& = &
		\lift{w}{y!(u)} \nonumber\\
	w[ \lpquote y!(z) \rpquote ] \widehat{ \id{\{}u / z \id{\}} }
		& = &
		w[ \lpquote y!(z) \rpquote ] \nonumber
\end{eqnarray}

Because the body of the process between quotes is impervious to
substitution, we get radically different answers. In fact, by
examining the first process in an input context,
e.g. $x?(z).\lift{w}{y!(z)}$, we see that the process under the lift
operator may be shaped by prefixed inputs binding a name inside it. In
this sense, the lift operator will be seen as a way to dynamically
construct processes before reifying them as names.

Finally equipped with these standard features we can present the
dynamics of the calculus.

\subsubsection{Operational semantics} 

Finally, we introduce the computational dynamics. What marks these
algebras as distinct from other more traditionally studied algebraic
structures, e.g. vector spaces or polynomial rings, is the manner in
which dynamics is captured. In traditional structures, dynamics is typically
expressed through morphisms between such structures, as in linear maps
between vector spaces or morphisms between rings. In algebras
associated with the semantics of computation, the dynamics is
expressed as part of the algebraic structure itself, through a
reduction reduction relation typically denoted by $\red$. Below, we
give a recursive presentation of this relation for the calculus used
in the encoding.

$\red \subseteq \pi \times \pi$
$\red : \pi \to \mathcal{P}(\pi)$

\begin{mathpar}
  \inferrule* [lab=Comm] { \textsf{match}( x_{src}, x_{trgt} ) } { x_{trgt}?(y)P \; | \; x_{src}!\langle {Q} \rangle \red P\{\quotep{Q}/y}\} }
  \and \\
  \inferrule* [lab=Par] {{P} \red {P}'} {{{P} | {Q}} \red {{P}' | {Q}}}
  \and
  \inferrule* [lab=Equiv]{{{P} \scong {P}'} \andalso {{P}' \red {Q}'} \andalso {{Q}' \scong {Q}}}{{P} \red {Q}}
\end{mathpar}

\begin{eqnarray*}
  match_{\equiv} (\quotep{P},\quotep{Q}) & := & P \equiv Q \\
  match_{\dagger}(\quotep{P},\quotep{Q}) & := & \forall R. P|Q \red^{*} R => R \red^{*} 0 \\
  match_{K}(\quotep{P},\quotep{Q}) & := & K \mbox{ for some context } K
\end{eqnarray*}

$u?(x)P | u!\langle Q \rangle \red P\{\quotep{Q}/x\}$

%We write $\wred$ for $\red^*$, and $P\red$ if $\exists Q $ such that $ P \red Q$.
We write $P\red$ if $\exists Q $ such that $ P \red Q$ and $P\not\red$, otherwise.

\section{Replication}

As mentioned before, it is known that replication (and hence
recursion) can be implemented in a higher-order process algebra
\cite{SangiorgiWalker}. As our first example of calculation with the
machinery thus far presented we give the construction explicitly in
the {\rhoc}.

\begin{eqnarray}
	D_{x} & := & \prefix{x}{y}{(\binpar{\outputp{x}{y}}{@{y}})} \nonumber\\
	\bangp_{x}{P} & := & \binpar{{x}!\langle{\binpar{D_{x}}{P}}\rangle}{D_{x}} \nonumber
\end{eqnarray}

\begin{eqnarray}
	\bangp_{x}{P} & & \nonumber\\
	=
	& {x}!\langle{(\prefix{x}{y}{(\outputp{x}{y} | @{y})) | P}}\rangle 
	      | \prefix{x}{y}{(\outputp{x}{y} | @{y})} & \nonumber\\
	\red
	& (\outputp{x}{y} | @{y})\substn{\quotep{(\prefix{x}{y}{(@{y} | \outputp{x}{y})) | P}}}{y} & \nonumber\\
	=
	& \outputp{x}{\quotep{(\prefix{x}{y}{(\outputp{x}{y} | @{y})) | P}}}
	  | {(\prefix{x}{y}{(\outputp{x}{y} | @{y})) | P}} & \nonumber\\
	\red
	& \ldots & \nonumber\\
	\red^*
	& P | P | \ldots & \nonumber
\end{eqnarray}

Of course, this encoding, as an implementation, runs away, unfolding
$\bangp{P}$ eagerly. A lazier and more implementable replication
operator, restricted to input-guarded processes, may be obtained as follows.

\begin{eqnarray}
\bangp{\prefix{u}{v}{P}} 
	:= 
	\binpar{\lift{x}{\prefix{u}{v}{(\binpar{D(x)}{P})}}}{D(x)} \nonumber
\end{eqnarray}

\begin{remark}
  Note that the lazier definition still does not deal with summation
  or mixed summation (i.e. sums over input and output). The reader is
  invited to construct definitions of replication that deal with these
  features. 

  Further, the definitions are parameterized in a name, $x$. Can you,
  gentle reader, make a definition that eliminates this parameter and
  guarantees no accidental interaction between the replication
  machinery and the process being replicated -- i.e. no accidental
  sharing of names used by the process to get its work done and the
  name(s) used by the replication to effect copying. This latter
  revision of the definition of replication is crucial to obtaining
  the expected identity $!!P \sim !P$.
\end{remark}

\begin{remark}\label{rem:paradoxical_combinator}
  The reader familiar with the lambda calculus will have noticed the
  similarity between $D$ and the paradoxical combinator.

  [Ed. note: the existence of this seems to suggest we have to be more
  restrictive on the set of processes and names we admit if we are to
  support no-cloning.]
\end{remark}

\subsubsection{Bisimulation}

The computational dynamics gives rise to another kind of equivalence,
the equivalence of computational behavior. As previously mentioned
this is typically captured \emph{via} some form of bisimulation.

% The notion we use in this paper is weak barbed bisimulation
% \cite{milner91polyadicpi}.

The notion we use in this paper is derived from weak barbed
bisimulation \cite{milner91polyadicpi}. 

\begin{definition}
An \emph{observation relation}, $\downarrow_{\mathcal N}$, over a set
of names, $\mathcal N$, is the smallest relation satisfying the rules
below.

\infrule[Out-barb]{y \in {\mathcal N}, \; x \nameeq y}
		  {\outputp{x}{v} \downarrow_{\mathcal N} x}
\infrule[Par-barb]{\mbox{$P\downarrow_{\mathcal N} x$ or $Q\downarrow_{\mathcal N} x$}}
		  {\binpar{P}{Q} \downarrow_{\mathcal N} x}

We write $P \Downarrow_{\mathcal N} x$ if there is $Q$ such that 
$P \wred Q$ and $Q \downarrow_{\mathcal N} x$.
\end{definition}

\begin{definition}
%\label{def.bbisim}
An  ${\mathcal N}$-\emph{barbed bisimulation} over a set of names, ${\mathcal N}$, is a symmetric binary relation 
${\mathcal S}_{\mathcal N}$ between agents such that $P\rel{S}_{\mathcal N}Q$ implies:
\begin{enumerate}
\item If $P \red P'$ then $Q \wred Q'$ and $P'\rel{S}_{\mathcal N} Q'$.
\item If $P\downarrow_{\mathcal N} x$, then $Q\Downarrow_{\mathcal N} x$.
\end{enumerate}
$P$ is ${\mathcal N}$-barbed bisimilar to $Q$, written
$P \wbbisim_{\mathcal N} Q$, if $P \rel{S}_{\mathcal N} Q$ for some ${\mathcal N}$-barbed bisimulation ${\mathcal S}_{\mathcal N}$.
\end{definition}

$\mathcal{R} \subseteq \pi \times \pi$

$P \mathcal{R} Q => \forall P'. P \red P' \Rightarrow \exists Q'. Q \red Q', P' \mathcal{R} Q'$

$P \vdash x \Rightarrow Q \vdash x$

\begin{mathpar}
  \inferrule*[lab=Out-barb]{x \nameeq y}{{y}!\langle{Q}\rangle \vdash x}
  \and
  \inferrule*[lab=Par-barb]{\mbox{$P\vdash x$ or $Q\vdash x$}}{\binpar{P}{Q} \vdash x}
\end{mathpar}

\subsubsection{Contexts}

One of the principle advantages of computational calculi like the
$\pi$-calculus is a well-defined notion of context,
contextual-equivalence and a correlation between
contextual-equivalence and notions of bisimulation. The notion of
context allows the decomposition of a process into (sub-)process and
its syntactic environment, its context. Thus, a context may be
thought of as a process with a ``hole'' (written $\Box$) in it. The
application of a context $M$ to a process $P$, written $M[P]$, is
tantamount to filling the hole in $M$ with $P$. In this paper we do
not need the full weight of this theory, but do make use of the notion
of context in the proof the main theorem. 

\begin{mathpar}
  \inferrule* [lab=summation] {} {{M_{M},M_{N}} \bc \Box \;|\; x.M_{A} \;|\; M_{M}+M_{N}}
  \and
  \inferrule* [lab=agent] {} {{M_{A}} \bc (\vec{x})M_{P} \;| \; \clift{P_0,\ldots,M_{P},\ldots,P_N}}
  \and \\
  \inferrule* [lab=process] {} {{M_{P}} \bc M_{N} \;| \;P|M_{P} }
\end{mathpar} 

\begin{mathpar}
  \inferrule* [lab=sychronization] {} {M_{N} \bc \Box \;|\; x?M_{F} \;|\; x!M_{C}}
  \and
  \inferrule* [lab=abstraction] {} {{M_{F}} \bc (x)M_{P} }
  \and
  \inferrule* [lab=concretion] {} {{M_{C}} \bc \langle M_{P} \rangle }
  \and \\
  \inferrule* [lab=process] {} {{M_{P}} \bc M_{N} \;| \;P|M_{P} }
\end{mathpar}

\begin{definition}[contextual application] Given a context $M$, and
  process $P$, we define the \emph{contextual application}, $M[P] :=
  M\{P/\Box\}$. That is, the contextual application of M to P is the
  substitution of $P$ for $\Box$ in $M$.
\end{definition}

$\meaningof{-} : L \to \mathcal{P}(\pi)$

\begin{mathpar}
  \inferrule* [lab=collection] {} {\meaningof{true} = \pi, \and \meaningof{~E} = \pi \setminus \meaningof{E}, \and \meaningof{E_{1} \& E_{2}} = \meaningof{E_{1}} \cap \meaningof{E_{2}}}
\end{mathpar}

\begin{mathpar}
  \inferrule* [lab=structure] {} {\meaningof{0} = \{ P \in \pi | P \equiv 0 \}, \and \\ \meaningof{E_1 | E_2} = \{ P \in \pi | P \equiv P_{1} | P_{2}, P_{1} \in \meaningof{E_{1}}, P_{2} \in \meaningof{E_2}\} }
\end{mathpar}

\begin{mathpar}
 \inferrule* [lab=behavior] {} {\meaningof{\langle a?b \rangle E} = \{ P \in \pi | P \equiv Q | u?(y)P', \\ \and \\\\ \and \\ \;\;\; u \in \meaningof{a}, \forall z.P'\{z/y\} \in \meaningof{E\{z/b\}}\}, \and \\ \meaningof{a!E} = \{ P \in \pi | P \equiv Q | x!\langle P' \rangle, x \in \meaningof{a} P' \in \meaningof{E}\} }
\end{mathpar}

\begin{mathpar}
 \inferrule* [lab=nominal] {} {\meaningof{\quotep{E}} = \{ \quotep{P} \in \quotep{\pi} | P \in \meaningof{E} \}, \and \meaningof{\quotep{P}} = \{ \quotep{Q} \in \quotep{\pi} | P \equiv Q \} \and \\ \meaningof{@\quotep{E}} = \{ P \in \pi | P \equiv @x, x \in \meaningof{E} \}}
\end{mathpar}

\begin{eqnarray*}
  \\
  \meaningof{-} : TS \to ST
\end{eqnarray*}

\begin{eqnarray*}
  \\
  L : TS \to ST
\end{eqnarray*}

\begin{eqnarray*}
  \\
  P \models E \iff P \in \meaningof{E}
\end{eqnarray*}

\begin{eqnarray*}
  P \approx_{L} Q \iff \forall E \in L. P \models E \iff Q \models E
\end{eqnarray*}

\begin{eqnarray*}
  P \approx_{K} Q
\end{eqnarray*}

\begin{eqnarray*}
  P \approx Q
\end{eqnarray*}

$\approx_{K} = \approx = \approx_{L}$

\subsubsection{Contextual duality}

Note that contexts extend the quotation operation to a family of
operations from processes to names. Given a context, $M$, we can
define a \emph{nominal context}, $\quotep{M}$ by $\quotep{M}[P] :=
\quotep{M[P]}$. To foreshadow what is to come we observe that these
operations enjoy a duality with processes very much like the duality
between vectors and maps from vectors to scalars.

Further, because the calculus is essentially higher-order, we have a
correspondence between contexts and processes. More specifically,
given a name $x$ and a context $M$ we can construct $M^{*}_{x}$ such
that 

\begin{mathpar}
  M^{*}_{x} | \lift{x}{P} \red M[P]
\end{mathpar}

namely,

\begin{mathpar}
  M^{*}_{x} := x?(u).M[\dropn{u}]
\end{mathpar}

The dependence of $M^{*}_{x}$ on a name makes it an abstraction, 

\begin{mathpar}
  M^{*} := (x)x?(u).M[\dropn{u}]
\end{mathpar}

\subsection{Additional notation}

It will sometimes be convenient to denote the process a name
quotes. We already have the notation $x = \quotep{P}$, but it will be
convenient to introduce an alternate notation, $\procn{x}$, when we
want to emphasize the connection to the use of the name. Note that, by
virtue of name equivalence, $\quotep{\procn{x}} \nameeq x$; so, the
notation is consistent with previous definitions.

Further, because names have structure it is possible to effect
substitutions on the basis of that structure. This means we need to
upgrade our notation for substitutions, which we accomplish by
adapting comprehension notation. Thus,

\begin{mathpar}
  P\{ y / x : x \in S \}
\end{mathpar}

is interpreted to mean the process derived from P by replacing (in a
capture-avoiding manner) each occurrence of $x$ in $S$ by $y$. For example,

\begin{mathpar}
  P\{ \quotep{\procn{x}|\procn{x}} / x : x \in \freenames{P} \}
\end{mathpar}

will replace each (occurrence) of a free name $x$ in $P$ by
$\quotep{\procn{x}|\procn{x}}$.

Also, we will avail ourselves of the notation $x^{L}$ and $x^{R}$ to
denote injections of a name into disjoint copies of the name
space. There are numerous ways to accomplish this. One example can be
found in \cite{MeredithR05}. This notation overloads to vectors of
names: $\vec{x}^{\pi} := (x_{i}^{\pi} \; : \; 0 \leq i < |\vec{x}| )$ where $\pi \in \{L,R\}$.

We also use $P^{\Box} := P|\Box$.

In \cite{MeredithR05} an interpretation of the new operator is
given. It turns out that there are several possible interpretations
all enjoying the requisite algebraic properties of the operator (see
\cite{milner91polyadicpi}). We will therefore make liberal use of
$(\nu\; \vec{x})P$.

% subsection the_syntax_and_semantics_of_the_notation_system (end)   

\input{qm2pi.qmops} 

\input{qm2pi.sterngerlach} 

\input{qm2pi.metric} 

% section concurrent_process_calculi (end)

%\input{qm2pi.proofsketch}

% section proof sketch (end)

%\input{qm2pi.slviaknots} 

% section spatial logic via knots (end)

\input{qm2pi.conclusion}

% section conclusion (end)

%\input{qm2pi.dtcodes} 

% section wiring algorithm (end)

\input{qm2pi.ack} 

% section acknowledgments (end)

\newpage


\bibliographystyle{plain}   
\bibliography{../../biblios/main.bib}

\input{qm2pi.rhodetails}

\end{document}

 

% section acknowledgments (end)

\newpage


\bibliographystyle{plain}   
\bibliography{../../biblios/main.bib}

\documentclass[12pt]{llncs}
%\documentclass{jktr}

\usepackage[pdftex]{hyperref}                   
\usepackage {listings}
\usepackage {mathpartir}
\usepackage{bcprules}
%\usepackage{listings}
                       
\usepackage{graphicx} 
%\usepackage[margins=2.5cm,nohead,nofoot]{geometry}
%\usepackage{geometry}
\usepackage{amsfonts}
\usepackage{amstext}
\usepackage{latexsym}
\usepackage{amssymb}
\usepackage{color}


%\include{myPreamble}
\include{qm2pi.local} 

%\ifpdf
%\usepackage[pdftex]{graphicx}
%\else
%\usepackage{graphicx}
%\fi

 % \ifpdf
%  \usepackage{pdfsync}
%  \if


%\title{Brief Article}
%\author{David F. Snyder}
%\author{L.G. Meredith}

%\address{Dept. of Math., Texas State University--San Marcos, San Marcos, TX 78666}
       
\pagestyle{empty}


\begin{document}

\lstset{language=[Objective]Caml,frame=shadowbox}

\input{qm2pi.front}

% section front matter (end)

\input{qm2pi.intro} 
 
% section introduction (end)

% \input{qm2pi.knotations} 

% section notation (end)

\input{qm2pi.process.calculi} 

% section concurrent_process_calculi_and_spatial_logics_ (end)
    
%\input{qm2pi.knots2pi} 

%\input{qm2pi.trefoil} 

%\input{qm2pi.mainthm} 

% subsection basic_interpretation (end)

%\input{qm2pi.rho.presentation} 
\subsection{The syntax and semantics of the notation system}\label{sub:the_syntax_and_semantics_of_the_notation_system} % (fold)

We now summarize a technical presentation of the calculus that
embodies our theory of dynamics. The typical presentation of such a
calculus follows the style of giving generators and relations on
them. The grammar, below, describing term constructors, freely
generates the set of processes, $\Proc$. This set is then quotiented
by a relation known as structural congruence and it is over this set
that the notion of dynamics is expressed. This presentation is
essentially that of \cite{MeredithR05} with the addition of
polyadicity and summation. For readability we have relegated some of
the technical subtleties to an appendix.

\subsubsection{Process grammar}\label{subsub:process_grammar}

\begin{mathpar}
  \inferrule* [lab=synchronization] {} {{M} \bc \pzero \;|\; x?F \;|\; x!C }
  \and
  \inferrule* [lab=abstraction] {} {{F} \bc (x)P}
  \and
  \inferrule* [lab=concretion] {} {{C} \bc \langle Q \rangle}
  \and
  \inferrule* [lab=process] {} {{P,Q} \bc M \;| \;P|Q \;|\; @{x}}
  \and
  \inferrule* [lab=name] {} {{x} \bc \quotep{P}}
\end{mathpar} 

Note that $\vec{x}$ (resp. $\vec{P}$) denotes a vector of names
(resp. processes) of length $|\vec{x}|$ (resp. $|\vec{P}|$). We adopt
the following useful abbreviations.

\begin{mathpar}
   x?(\vec{y}).P := x.(\vec{y})P \and  x\clift{\vec{P}} := x.\clift{\vec{P}}
   \and x!(y) := \lift{x}{\dropn{y}}
   \and \Pi_{i=0}^{n-1}P_i := P_0 | \ldots | P_{n-1}
\end{mathpar}

\subsubsection{Structural congruence}

\paragraph{Free and bound names and alpha-equivalence.} At the
core of structural equivalence is alpha-equivalence which identifies
process that are the same up to a change of variable. Formally, we
recognize the distinction between free and bound names. The free names
of a process, $\freenames{P}$, may be calculated recursively as
follows:

\begin{mathpar}
\freenames{\pzero} := \emptyset
  \and \\
  \freenames{x?(y).P} := \{ x \} \cup (\freenames{P} \setminus \{ y \})
  \and 
  \freenames{x!\langle P \rangle} := \{ x \} \cup \{ P \} 
  \and \\
  \freenames{P|Q} := \freenames{P} \cup \freenames{Q}
  \and \\
  \freenames{@{x}} := \{ x \}
\end{mathpar}

$\pi$
$\quotep{\pi}$

$\freenames{-} : \pi \to \mathcal{P}(\quotep{\pi})$

\begin{eqnarray*}
  \freenames{\pzero} & := & \emptyset \\
  \freenames{x?(y).P} & := & \{ x \} \cup (\freenames{P} \setminus \{ y \}) \\
  \freenames{x!\langle P \rangle} & := & \{ x \} \cup \{ P \} \\
  \freenames{P|Q} & := & \freenames{P} \cup \freenames{Q} \\
  \freenames{\dropn{x}} & := & \{ x \}
\end{eqnarray*}

The bound names of a process, $\boundnames{P}$, are those names occurring in $P$
that are not free. For example, in $x?(y).0$, the name $x$ is free, while $y$ is bound.

\begin{mathpar}
  \inferrule* [lab=monoidal-laws] {} { P|Q \equiv Q|P \and P|0 \equiv P \and P|(Q|R) \equiv (P|Q)|R }
\end{mathpar}

\begin{mathpar}
  \inferrule* [lab=alpha-equivalence] {} { (x)P \equiv (y)P\{y/x\} \and y \not\in \freenames{P} }
\end{mathpar}

\begin{definition}
Then two processes, $P,Q$, are alpha-equivalent if $P = Q\{\vec{y}/\vec{x}\}$ for
some $\vec{x} \in \boundnames{Q},\vec{y} \in \boundnames{P}$, where $Q\{\vec{y}/\vec{x}\}$
denotes the capture-avoiding substitution of $\vec{y}$ for $\vec{x}$ in $Q$.
\end{definition}

\begin{definition}
  The {\em structural congruence} \cite{SangiorgiWalker} , $\equiv$,
  between processes is the least congruence containing
  alpha-equivalence, satisfying the abelian monoid laws
  (associativity, commutativity and $\pzero$ as identity) for parallel
  composition $|$ and for summation $+$.
\end{definition}

\subsection{Name equivalence}

We take name equivalence, written $\nameeq$, to be the smallest
equivalence relation generated by the following rules.

\begin{mathpar}
\inferrule*[lab=Quote-drop]
{ }
{ \quotep{@{x}} \nameeq x }

\inferrule*[lab=Struct-equiv]
{ P \scong Q }
{ \quotep{P} \nameeq \quotep{Q} }
\end{mathpar}

The astute reader will have noticed that the mutual recursion of names
and processes imposes a mutual recursion on alpha-equivalence and
structural equivalence via name-equivalence. Fortunately, all of this
works out pleasantly and we may calculate in the natural way, free of
concern. The reader interested in the details is referred to the
appendix \ref{appendix:rho_details}.

\subsection{Substitution}

We use $\Proc$ for the set of processes, $\QProc$ for the set of
names, and $\id{\{}\vec{y} / \vec{x} \id{\}}$ to denote partial maps,
$s : \QProc \rightarrow \QProc$. A map, $s$ lifts, uniquely, to a map
on process terms, $\widehat{s} : \Proc \rightarrow \Proc$ by the
following equations.

\begin{mathpar}
  (0) \psubstp{Q}{P} := 0 \\
  (R \juxtap S) \psubstp{Q}{P}
  :=    
  (R)\psubstp{Q}{P} \juxtap (S) \psubstp{Q}{P} \\
  (x?(y).R) \psubstp{Q}{P}    
  :=    
  (x)\substp{Q}{P} (z)\concat( (R \psubstn{z}{y}) \psubstp{Q}{P} ) \\
  (\lift{x}{R}) \psubstp{Q}{P}  
  :=
  \lift{(x)\substp{Q}{P}}{ R \psubstp{Q}{P} } \\
%   (\dropn{x})  \psubstp{Q}{P}       
%   := 
%   \left\{ 
%     \begin{array}{ccc} 
%       \dropn{\quotep{Q}} & & x \nameeq \quotep{P} \\
%       \dropn{x} & & otherwise \\
%     \end{array}
%   \right. 
  (\dropn{x})  \psubstp{Q}{P}       
  := 
  \left\{ 
    \begin{array}{ccc} 
      Q & & x \nameeq \quotep{P} \\
      \dropn{x} & & otherwise \\
    \end{array}
  \right.
\end{mathpar}
 

where

\begin{eqnarray}
  (x)\id{\{} \lpquote Q \rpquote / \lpquote P \rpquote \id{\}}            = 
  \left\{ 
    \begin{array}{ccc}
      \lpquote Q \rpquote & & x \nameeq \lpquote P \rpquote \\
      x & & otherwise \\
    \end{array}
  \right. \nonumber
\end{eqnarray}

and $z$ is chosen distinct from $\quotep{P}$, $\quotep{Q}$, the free
names in $Q$, and all the names in $R$. Our $\alpha$-equivalence will
be built in the standard way from this substitution.

\begin{remark}\label{rem:no_self_referential_names}
  One consequence of these definitions is that $\forall P. \quotep{P}
  \not\in \freenames{P}$.
\end{remark}

\subsection{ Dynamic quote: an example }

Anticipating something of what's to come, consider applying the
substitution, $\widehat{\id{\{}u / z \id{\}}}$, to the following pair
of processes, $\lift{w}{y!(z)}$ and $w[ \lpquote y!(z) \rpquote ]$.

\begin{eqnarray}
	\lift{w}{y!(z)}\widehat{\id{\{}u / z \id{\}}}
		& = &
		\lift{w}{y!(u)} \nonumber\\
	w[ \lpquote y!(z) \rpquote ] \widehat{ \id{\{}u / z \id{\}} }
		& = &
		w[ \lpquote y!(z) \rpquote ] \nonumber
\end{eqnarray}

Because the body of the process between quotes is impervious to
substitution, we get radically different answers. In fact, by
examining the first process in an input context,
e.g. $x?(z).\lift{w}{y!(z)}$, we see that the process under the lift
operator may be shaped by prefixed inputs binding a name inside it. In
this sense, the lift operator will be seen as a way to dynamically
construct processes before reifying them as names.

Finally equipped with these standard features we can present the
dynamics of the calculus.

\subsubsection{Operational semantics} 

Finally, we introduce the computational dynamics. What marks these
algebras as distinct from other more traditionally studied algebraic
structures, e.g. vector spaces or polynomial rings, is the manner in
which dynamics is captured. In traditional structures, dynamics is typically
expressed through morphisms between such structures, as in linear maps
between vector spaces or morphisms between rings. In algebras
associated with the semantics of computation, the dynamics is
expressed as part of the algebraic structure itself, through a
reduction reduction relation typically denoted by $\red$. Below, we
give a recursive presentation of this relation for the calculus used
in the encoding.

$\red \subseteq \pi \times \pi$
$\red : \pi \to \mathcal{P}(\pi)$

\begin{mathpar}
  \inferrule* [lab=Comm] { \textsf{match}( x_{src}, x_{trgt} ) } { x_{trgt}?(y)P \; | \; x_{src}!\langle {Q} \rangle \red P\{\quotep{Q}/y}\} }
  \and \\
  \inferrule* [lab=Par] {{P} \red {P}'} {{{P} | {Q}} \red {{P}' | {Q}}}
  \and
  \inferrule* [lab=Equiv]{{{P} \scong {P}'} \andalso {{P}' \red {Q}'} \andalso {{Q}' \scong {Q}}}{{P} \red {Q}}
\end{mathpar}

\begin{eqnarray*}
  match_{\equiv} (\quotep{P},\quotep{Q}) & := & P \equiv Q \\
  match_{\dagger}(\quotep{P},\quotep{Q}) & := & \forall R. P|Q \red^{*} R => R \red^{*} 0 \\
  match_{K}(\quotep{P},\quotep{Q}) & := & K \mbox{ for some context } K
\end{eqnarray*}

$u?(x)P | u!\langle Q \rangle \red P\{\quotep{Q}/x\}$

%We write $\wred$ for $\red^*$, and $P\red$ if $\exists Q $ such that $ P \red Q$.
We write $P\red$ if $\exists Q $ such that $ P \red Q$ and $P\not\red$, otherwise.

\section{Replication}

As mentioned before, it is known that replication (and hence
recursion) can be implemented in a higher-order process algebra
\cite{SangiorgiWalker}. As our first example of calculation with the
machinery thus far presented we give the construction explicitly in
the {\rhoc}.

\begin{eqnarray}
	D_{x} & := & \prefix{x}{y}{(\binpar{\outputp{x}{y}}{@{y}})} \nonumber\\
	\bangp_{x}{P} & := & \binpar{{x}!\langle{\binpar{D_{x}}{P}}\rangle}{D_{x}} \nonumber
\end{eqnarray}

\begin{eqnarray}
	\bangp_{x}{P} & & \nonumber\\
	=
	& {x}!\langle{(\prefix{x}{y}{(\outputp{x}{y} | @{y})) | P}}\rangle 
	      | \prefix{x}{y}{(\outputp{x}{y} | @{y})} & \nonumber\\
	\red
	& (\outputp{x}{y} | @{y})\substn{\quotep{(\prefix{x}{y}{(@{y} | \outputp{x}{y})) | P}}}{y} & \nonumber\\
	=
	& \outputp{x}{\quotep{(\prefix{x}{y}{(\outputp{x}{y} | @{y})) | P}}}
	  | {(\prefix{x}{y}{(\outputp{x}{y} | @{y})) | P}} & \nonumber\\
	\red
	& \ldots & \nonumber\\
	\red^*
	& P | P | \ldots & \nonumber
\end{eqnarray}

Of course, this encoding, as an implementation, runs away, unfolding
$\bangp{P}$ eagerly. A lazier and more implementable replication
operator, restricted to input-guarded processes, may be obtained as follows.

\begin{eqnarray}
\bangp{\prefix{u}{v}{P}} 
	:= 
	\binpar{\lift{x}{\prefix{u}{v}{(\binpar{D(x)}{P})}}}{D(x)} \nonumber
\end{eqnarray}

\begin{remark}
  Note that the lazier definition still does not deal with summation
  or mixed summation (i.e. sums over input and output). The reader is
  invited to construct definitions of replication that deal with these
  features. 

  Further, the definitions are parameterized in a name, $x$. Can you,
  gentle reader, make a definition that eliminates this parameter and
  guarantees no accidental interaction between the replication
  machinery and the process being replicated -- i.e. no accidental
  sharing of names used by the process to get its work done and the
  name(s) used by the replication to effect copying. This latter
  revision of the definition of replication is crucial to obtaining
  the expected identity $!!P \sim !P$.
\end{remark}

\begin{remark}\label{rem:paradoxical_combinator}
  The reader familiar with the lambda calculus will have noticed the
  similarity between $D$ and the paradoxical combinator.

  [Ed. note: the existence of this seems to suggest we have to be more
  restrictive on the set of processes and names we admit if we are to
  support no-cloning.]
\end{remark}

\subsubsection{Bisimulation}

The computational dynamics gives rise to another kind of equivalence,
the equivalence of computational behavior. As previously mentioned
this is typically captured \emph{via} some form of bisimulation.

% The notion we use in this paper is weak barbed bisimulation
% \cite{milner91polyadicpi}.

The notion we use in this paper is derived from weak barbed
bisimulation \cite{milner91polyadicpi}. 

\begin{definition}
An \emph{observation relation}, $\downarrow_{\mathcal N}$, over a set
of names, $\mathcal N$, is the smallest relation satisfying the rules
below.

\infrule[Out-barb]{y \in {\mathcal N}, \; x \nameeq y}
		  {\outputp{x}{v} \downarrow_{\mathcal N} x}
\infrule[Par-barb]{\mbox{$P\downarrow_{\mathcal N} x$ or $Q\downarrow_{\mathcal N} x$}}
		  {\binpar{P}{Q} \downarrow_{\mathcal N} x}

We write $P \Downarrow_{\mathcal N} x$ if there is $Q$ such that 
$P \wred Q$ and $Q \downarrow_{\mathcal N} x$.
\end{definition}

\begin{definition}
%\label{def.bbisim}
An  ${\mathcal N}$-\emph{barbed bisimulation} over a set of names, ${\mathcal N}$, is a symmetric binary relation 
${\mathcal S}_{\mathcal N}$ between agents such that $P\rel{S}_{\mathcal N}Q$ implies:
\begin{enumerate}
\item If $P \red P'$ then $Q \wred Q'$ and $P'\rel{S}_{\mathcal N} Q'$.
\item If $P\downarrow_{\mathcal N} x$, then $Q\Downarrow_{\mathcal N} x$.
\end{enumerate}
$P$ is ${\mathcal N}$-barbed bisimilar to $Q$, written
$P \wbbisim_{\mathcal N} Q$, if $P \rel{S}_{\mathcal N} Q$ for some ${\mathcal N}$-barbed bisimulation ${\mathcal S}_{\mathcal N}$.
\end{definition}

$\mathcal{R} \subseteq \pi \times \pi$

$P \mathcal{R} Q => \forall P'. P \red P' \Rightarrow \exists Q'. Q \red Q', P' \mathcal{R} Q'$

$P \vdash x \Rightarrow Q \vdash x$

\begin{mathpar}
  \inferrule*[lab=Out-barb]{x \nameeq y}{{y}!\langle{Q}\rangle \vdash x}
  \and
  \inferrule*[lab=Par-barb]{\mbox{$P\vdash x$ or $Q\vdash x$}}{\binpar{P}{Q} \vdash x}
\end{mathpar}

\subsubsection{Contexts}

One of the principle advantages of computational calculi like the
$\pi$-calculus is a well-defined notion of context,
contextual-equivalence and a correlation between
contextual-equivalence and notions of bisimulation. The notion of
context allows the decomposition of a process into (sub-)process and
its syntactic environment, its context. Thus, a context may be
thought of as a process with a ``hole'' (written $\Box$) in it. The
application of a context $M$ to a process $P$, written $M[P]$, is
tantamount to filling the hole in $M$ with $P$. In this paper we do
not need the full weight of this theory, but do make use of the notion
of context in the proof the main theorem. 

\begin{mathpar}
  \inferrule* [lab=summation] {} {{M_{M},M_{N}} \bc \Box \;|\; x.M_{A} \;|\; M_{M}+M_{N}}
  \and
  \inferrule* [lab=agent] {} {{M_{A}} \bc (\vec{x})M_{P} \;| \; \clift{P_0,\ldots,M_{P},\ldots,P_N}}
  \and \\
  \inferrule* [lab=process] {} {{M_{P}} \bc M_{N} \;| \;P|M_{P} }
\end{mathpar} 

\begin{mathpar}
  \inferrule* [lab=sychronization] {} {M_{N} \bc \Box \;|\; x?M_{F} \;|\; x!M_{C}}
  \and
  \inferrule* [lab=abstraction] {} {{M_{F}} \bc (x)M_{P} }
  \and
  \inferrule* [lab=concretion] {} {{M_{C}} \bc \langle M_{P} \rangle }
  \and \\
  \inferrule* [lab=process] {} {{M_{P}} \bc M_{N} \;| \;P|M_{P} }
\end{mathpar}

\begin{definition}[contextual application] Given a context $M$, and
  process $P$, we define the \emph{contextual application}, $M[P] :=
  M\{P/\Box\}$. That is, the contextual application of M to P is the
  substitution of $P$ for $\Box$ in $M$.
\end{definition}

$\meaningof{-} : L \to \mathcal{P}(\pi)$

\begin{mathpar}
  \inferrule* [lab=collection] {} {\meaningof{true} = \pi, \and \meaningof{~E} = \pi \setminus \meaningof{E}, \and \meaningof{E_{1} \& E_{2}} = \meaningof{E_{1}} \cap \meaningof{E_{2}}}
\end{mathpar}

\begin{mathpar}
  \inferrule* [lab=structure] {} {\meaningof{0} = \{ P \in \pi | P \equiv 0 \}, \and \\ \meaningof{E_1 | E_2} = \{ P \in \pi | P \equiv P_{1} | P_{2}, P_{1} \in \meaningof{E_{1}}, P_{2} \in \meaningof{E_2}\} }
\end{mathpar}

\begin{mathpar}
 \inferrule* [lab=behavior] {} {\meaningof{\langle a?b \rangle E} = \{ P \in \pi | P \equiv Q | u?(y)P', \\ \and \\\\ \and \\ \;\;\; u \in \meaningof{a}, \forall z.P'\{z/y\} \in \meaningof{E\{z/b\}}\}, \and \\ \meaningof{a!E} = \{ P \in \pi | P \equiv Q | x!\langle P' \rangle, x \in \meaningof{a} P' \in \meaningof{E}\} }
\end{mathpar}

\begin{mathpar}
 \inferrule* [lab=nominal] {} {\meaningof{\quotep{E}} = \{ \quotep{P} \in \quotep{\pi} | P \in \meaningof{E} \}, \and \meaningof{\quotep{P}} = \{ \quotep{Q} \in \quotep{\pi} | P \equiv Q \} \and \\ \meaningof{@\quotep{E}} = \{ P \in \pi | P \equiv @x, x \in \meaningof{E} \}}
\end{mathpar}

\begin{eqnarray*}
  \\
  \meaningof{-} : TS \to ST
\end{eqnarray*}

\begin{eqnarray*}
  \\
  L : TS \to ST
\end{eqnarray*}

\begin{eqnarray*}
  \\
  P \models E \iff P \in \meaningof{E}
\end{eqnarray*}

\begin{eqnarray*}
  P \approx_{L} Q \iff \forall E \in L. P \models E \iff Q \models E
\end{eqnarray*}

\begin{eqnarray*}
  P \approx_{K} Q
\end{eqnarray*}

\begin{eqnarray*}
  P \approx Q
\end{eqnarray*}

$\approx_{K} = \approx = \approx_{L}$

\subsubsection{Contextual duality}

Note that contexts extend the quotation operation to a family of
operations from processes to names. Given a context, $M$, we can
define a \emph{nominal context}, $\quotep{M}$ by $\quotep{M}[P] :=
\quotep{M[P]}$. To foreshadow what is to come we observe that these
operations enjoy a duality with processes very much like the duality
between vectors and maps from vectors to scalars.

Further, because the calculus is essentially higher-order, we have a
correspondence between contexts and processes. More specifically,
given a name $x$ and a context $M$ we can construct $M^{*}_{x}$ such
that 

\begin{mathpar}
  M^{*}_{x} | \lift{x}{P} \red M[P]
\end{mathpar}

namely,

\begin{mathpar}
  M^{*}_{x} := x?(u).M[\dropn{u}]
\end{mathpar}

The dependence of $M^{*}_{x}$ on a name makes it an abstraction, 

\begin{mathpar}
  M^{*} := (x)x?(u).M[\dropn{u}]
\end{mathpar}

\subsection{Additional notation}

It will sometimes be convenient to denote the process a name
quotes. We already have the notation $x = \quotep{P}$, but it will be
convenient to introduce an alternate notation, $\procn{x}$, when we
want to emphasize the connection to the use of the name. Note that, by
virtue of name equivalence, $\quotep{\procn{x}} \nameeq x$; so, the
notation is consistent with previous definitions.

Further, because names have structure it is possible to effect
substitutions on the basis of that structure. This means we need to
upgrade our notation for substitutions, which we accomplish by
adapting comprehension notation. Thus,

\begin{mathpar}
  P\{ y / x : x \in S \}
\end{mathpar}

is interpreted to mean the process derived from P by replacing (in a
capture-avoiding manner) each occurrence of $x$ in $S$ by $y$. For example,

\begin{mathpar}
  P\{ \quotep{\procn{x}|\procn{x}} / x : x \in \freenames{P} \}
\end{mathpar}

will replace each (occurrence) of a free name $x$ in $P$ by
$\quotep{\procn{x}|\procn{x}}$.

Also, we will avail ourselves of the notation $x^{L}$ and $x^{R}$ to
denote injections of a name into disjoint copies of the name
space. There are numerous ways to accomplish this. One example can be
found in \cite{MeredithR05}. This notation overloads to vectors of
names: $\vec{x}^{\pi} := (x_{i}^{\pi} \; : \; 0 \leq i < |\vec{x}| )$ where $\pi \in \{L,R\}$.

We also use $P^{\Box} := P|\Box$.

In \cite{MeredithR05} an interpretation of the new operator is
given. It turns out that there are several possible interpretations
all enjoying the requisite algebraic properties of the operator (see
\cite{milner91polyadicpi}). We will therefore make liberal use of
$(\nu\; \vec{x})P$.

% subsection the_syntax_and_semantics_of_the_notation_system (end)   

\input{qm2pi.qmops} 

\input{qm2pi.sterngerlach} 

\input{qm2pi.metric} 

% section concurrent_process_calculi (end)

%\input{qm2pi.proofsketch}

% section proof sketch (end)

%\input{qm2pi.slviaknots} 

% section spatial logic via knots (end)

\input{qm2pi.conclusion}

% section conclusion (end)

%\input{qm2pi.dtcodes} 

% section wiring algorithm (end)

\input{qm2pi.ack} 

% section acknowledgments (end)

\newpage


\bibliographystyle{plain}   
\bibliography{../../biblios/main.bib}

\input{qm2pi.rhodetails}

\end{document}



\end{document}



% section proof sketch (end)

%\section{Unlikely characters: spatial logic for
  knots}\label{sub:characteristic_formulae} % (fold)

Associated to the mobile process calculi are a family of logics known
as the Hennessy-Milner logics. These logics typically enjoy a
semantics interpreting formulae as sets of processes that when
factored through the encoding outlined above allows an identification
of classes of knots with logical formulae. In the context of this
encoding the sub-family known as the spatial logics \cite{CairesC03}
\cite{CairesC04} \cite{Caires04} are of particular interest providing
several important features for expressing and reasoning about
properties (i.e. classes) of knots. We hint here at how this may be done.

%\begin{description}
%\item [structural connectives] 
\subsubsection{Structural connectives} The spatial logics enjoy
structural connectives corresponding, at the logical level, to the
parallel composition ($P | Q$) and new name ($(\nu \; x)P$)
connectives for processes. As illustrated in the examples below, these
connectives are extremely expressive given the shape of our encoding.
%\item [decideable satisfaction]

\subsubsection{Decideable satisfaction}
In \cite{Caires04} the satisfaction relation is shown to be decideable
for a rich class of processes. It further turns out that the image of
the our encoding is a proper subset of that class. This result
provides the basis for an algorithm by which to search for knots
enjoying a given property.
%\item [characteristic formulae]

\subsubsection{Characteristic formulae}
In the same paper \cite{Caires04} , Caires presents a means of calculating
characteristic formulae, selecting equivalence classes of processes
up to a pre--specified depth limit on the support set of names. Composed with our
encoding, this characteristic formula can be used to select
characteristic formulae for knots.
%\end{description}

\subsubsection{Spatial logic formulae}

The grammar below (segmented for comprehension) summarizes the syntax
of spatial logic formulae. We employ illustrative examples in the
sequel to provide an intuitive understanding of their meaning
referring the reader to \cite{Caires04} for a more detailed explication
of the semantics.

\begin{mathpar}
  \inferrule* [lab=boolean] {} {{A,B} \bc T \;|\; \neg A \;|\; A \wedge B \;|\; \eta = \eta'}
  \and
  \inferrule* [lab=spatial] {} {|\; \pzero \;|\; A | B \;|\; x \text{\textregistered} A \;|\; \forall x . A \;|\;  H x . A}
  \and
  \inferrule* [lab=behavioral] {} {|\; \alpha . A}
  \and 
  \inferrule* [lab=recursion] {} {|\; X(\vec{u}) \;|\; \mu X(\vec{u}) . A}
  \and
  \inferrule* [lab=action] {} {\alpha \bc \langle x?(\vec{y}) \rangle \;|\; \langle x!(\vec{y}) \rangle \;|\; \langle \tau \rangle}
  \and 
  \inferrule* [lab=name] {} {\eta \bc x \;|\; \tau}
\end{mathpar} 

% subsection characteristic_formulae (end)   	 

\subsection{Example formulae}\label{sub:example_formulae_} % (fold)

\subsubsection{Crossing as formula.}
% 
% \begin{align*}
%   \frac{d}{dx} \sin x &= \cos x 
%   & \frac{d}{dx} e^x &= e^x \\
%   \frac{d}{dx} \cos x &= - \sin x 
%   & \frac{d}{dx} \log x &= \frac{1}{x} \\
% \end{align*} 

\begin{align*}
 \mu C(x_{0},x_{1},y_{0},y_{1},u).&(\langle x_{0}?(z) \rangle(\langle u! \rangle\langle y_{1}!z \rangle C(x_{0},x_{1},y_{0},y_{1},u)) & \\
  & \wedge \langle y_{1}?(z) \rangle (\langle u! \rangle \langle x_{0}!z \rangle C(x_{0},x_{1},y_{0},y_{1},u)) & \\
  & \wedge \langle x_{1}?(z) \rangle (\langle u? \rangle \langle y_{0}!z \rangle C(x_{0},x_{1},y_{0},y_{1},u)) & \\
  & \wedge \langle y_{0}?(z) \rangle (\langle u? \rangle \langle x_{1}!z \rangle C(x_{0},x_{1},y_{0},y_{1},u))) &
\end{align*}

The lexicographical similarity between the shape of this formulae and
the shape of definition of the process representing a crossing reveals
the intuitive meaning of this formulae. It describes the capabilities
of a process that has the right to represent a crossing. For example
it picks out processes that may perform an input on the port $x_0$ in
its initial menu of capabilities. What differentiates the formula
from the process, however, is that the crossing process is the
smallest candidate to satisfy the formula. Infinitely many other
processes -- with internal behavior hidden behind this interface, so
to speak -- also satisfy this formula. Even this simple formula,
then, can be seen to open a new view onto knots, providing a
computational interpretation of \emph{virtual} knots.

Note that this formula is derived by hand. A similar formula can be
derived by employing Caires' calculation of characteristic formula
\cite{Caires04} to the process representing a crossing. In light of
this discussion, we let
$\meaningof{C}_{\phi}(x0,x1,y0,y1,u)$ denote a formula specifying the
dynamics we wish to capture of a crossing. To guarantee we preserve
the shape of the interface and minimal semantics we demand that
$\meaningof{C}_{\phi}(x0,x1,y0,y1,u) \Rightarrow
\textbf{C}(x0,x1,y0,y1,u)$ where $\textbf{C}(x0,x1,y0,y1,u)$ denotes
the formula above.
                            
\subsubsection{Crossing number constraints.}
The moral content of the context lemma (Lemma \ref{context}) is that the notion of
``locality'' in the Reidemeister moves is effectively captured by the
parallel composition operator of the process calculus. This intuition
extends through the logic. Given a formula,
$\meaningof{C}_{\phi}(x0,x1,y0,y1,u)$, we can use the structural
connectives to specify constraints on crossing numbers, such as at
least $n$ crossings, or exactly $n$ crossings.
\begin{mathpar}
  \inferrule* [lab=at-least-n] {} { K^{\geq n}_{\phi}(\vec{xs},\vec{ys}) := \Pi_{i=0}^{n-1} Hu . \meaningof{C}_{\phi}(xs_i,ys_i,u) | T }
  \and 
  \inferrule* [lab=exactly-n] {} { K^{= n}_{\phi}(\vec{xs},\vec{ys}) := \Pi_{i=0}^{n-1} Hu . \meaningof{C}_{\phi}(xs_i,ys_i,u) | \neg (\forall x_0,y_0,x_1,y_1,u . \meaningof{C}_{\phi}(x_0,y_0,x_1,y_1,u) | T) }
\end{mathpar}

To round out this section, recall that the encoding of an $n$-crossing
knot decomposes into a parallel composition of $n$ \emph{copies} of a
crossing process together with a wiring harness. To specify different
knot classes with the same crossing number amounts to specifying
logical constraints on the wiring harness. In the interest of space,
we defer examples to a forthcoming paper. Suffice it to say that both
the conditions ``alternating knot'' and ``contains the tangle
corresponding to 5/3'' are expressible. For example, it is possible to
calculate the characteristic formula of a process corresponding to the
tangle 5/3 and conjoin it into the classifying formula via the
composition connective of the logic.

Finally, we wish to observe that it is entirely within reason to
contemplate a more domain-specific version of spatial logic tailored
to the shape of processes in the image of the encoding. Such a
domain-specific logic would have a better claim to the title formal
language of knot properties.

% subsection example_formulae_ (end)

% section knots_as_processes (end) 

% section spatial logic via knots (end)

\section{Conclusions and future work}

\paragraph{Testing physical space}
You, gentle reader, may wonder why of all the theorems to be proved
given this set up we pick the one above. In some sense it's hardly
central to quantum mechanics. We see it as central in the sense that
it firmly establishes a notion of physical space arising from a notion
of the equivalence of behavior. Relating bisimulation to a metric is a
big step forward, but one is faced with interpreting the relationship
of that metric space to something more physical. Quantum mechanical
notions of ``physical'' space are still far from intuitive, but by
relating this idea of distance as testing to calculations that predict
physical circumstances we are making a not insignificant step forward
toward an understanding of the physical space we inhabit as
essentially dynamic.

\paragraph{Effectivity and simulation}
One of the observations we have yet to make is that the entire program
spelled out here is effective. We have built various interpreters for
the reflective calculus at work in this interpretation. In principle,
then, we can simulate quantum mechanics on a computer. The place where
the simulation may lose fidelity is the infinitely branching summation
for the annihilator.

In this connection i also want to point out that the evaluation style
calculation of the inner product puts the non-determinism of the
summation right at the heart of measurement. This suggests that
Milner's original reduction-based formulation of the dynamics of his
calculi in terms of sums was not just notationally suggestive of a
notion of measure-and-continue but captured some significant part of
the physics.

\paragraph{Quantum continuations}
In light of this last observation i want to point out that the
predominant account of quantum mechanics is missing a key aspect of a
truly compositional story of the physical situation. In a real lab,
when a measurement is made the observation can be made to feed into
another device that then makes another measurement conditioned on the
results of the first. This means that after the superposition was
collapsed the entire experimental set up remained in
superposition. While QM offers a means of writing this down it doesn't
quite line up well with the well-trodden formulation of computation
and continuation that we see so succinctly expressed in Milner's
calculi. This suggests that there might be advantages to this account
of dynamics waiting to be explored.

\paragraph{Quantum logic}
In this connection, we also note that by virtue of having the
Hennessy-Milner construction, we can pull the construction through the
interpretation of QM. This gives us a natural candidate for a quantum
logic that enjoys an extremely tight connection with it's domain of
interpretation, making the construction much less ad hoc (rather it is
the image of functor!).

\paragraph{Quantum probabiity}
i have questions about the basis of the interpretation of inner
product as probability amplitude. In particular, using which
axiomatization of probability theory does the notion of probability
amplitude earn the right to be so dubbed? In other words, where is the
proof that the operation for calculating a probability amplitude (and
then squaring) satisfies the axioms of what it means to calculate a
probability? Even if such a proof exists (i have yet to find it in the
literature), i wonder if it might not be possible to turn things on
their heads. Can we view the calculation of the probability amplitude
as an axiomatization of probability? If so, then the definition we
give for calculating probability amplitude may provide the basis for
an \emph{effective} theory of probability.

\paragraph{Quantum vs ``biological'' information}
Finally, i want to conclude with a more philosophical observation. At
a recent workshop in which QM was a predominant topic i noticed
something about quantum information. The speaker was giving a riveting
discussion of axiomatic QM and showing how properties of ``no
cloning'' and ``no deleting'' emerged as consequences of the
axiomatization. Theorems of this form are necessary to give us a sense
of confidence that our axioms characterize the physical theory. What
struck me, though, was that if quantum information is neither erasable
nor replicable it is markedly different from \emph{life}. Two of the
things we know about life is that

\begin{itemize}
  \item it ends;
  \item to gain some measure of persistence, to transcend it's
    finitude it is imminently copyable.
\end{itemize}

Both of these qualities are summarized succinctly in the aphorism: all
flesh is grass. For me these two kinds of ``information'' -- call them
quantum and biological -- are end points on a spectrum of strategies
for persistence. At one end, we have those curious entities that enjoy
uniqueness and permanence; at the other, we have those who in the face
of a certain end and an uncertain present make a go of passing
something on. To me one of the more remarkable aspects of the latter
strategy is that in the presence of noise (and certain features of
copying) we get a kind of dynamism, a chance for improvement against a
given persistent condition.

% subsection other_calculi_other_bisimulations_and_geometry_as_behavior (end)




% section conclusion (end)

%\documentclass[12pt]{llncs}
%\documentclass{jktr}

\usepackage[pdftex]{hyperref}                   
\usepackage {listings}
\usepackage {mathpartir}
\usepackage{bcprules}
%\usepackage{listings}
                       
\usepackage{graphicx} 
%\usepackage[margins=2.5cm,nohead,nofoot]{geometry}
%\usepackage{geometry}
\usepackage{amsfonts}
\usepackage{amstext}
\usepackage{latexsym}
\usepackage{amssymb}
\usepackage{color}


%\include{myPreamble}
\documentclass[12pt]{llncs}
%\documentclass{jktr}

\usepackage[pdftex]{hyperref}                   
\usepackage {listings}
\usepackage {mathpartir}
\usepackage{bcprules}
%\usepackage{listings}
                       
\usepackage{graphicx} 
%\usepackage[margins=2.5cm,nohead,nofoot]{geometry}
%\usepackage{geometry}
\usepackage{amsfonts}
\usepackage{amstext}
\usepackage{latexsym}
\usepackage{amssymb}
\usepackage{color}


%\include{myPreamble}
\include{qm2pi.local} 

%\ifpdf
%\usepackage[pdftex]{graphicx}
%\else
%\usepackage{graphicx}
%\fi

 % \ifpdf
%  \usepackage{pdfsync}
%  \if


%\title{Brief Article}
%\author{David F. Snyder}
%\author{L.G. Meredith}

%\address{Dept. of Math., Texas State University--San Marcos, San Marcos, TX 78666}
       
\pagestyle{empty}


\begin{document}

\lstset{language=[Objective]Caml,frame=shadowbox}

\input{qm2pi.front}

% section front matter (end)

\input{qm2pi.intro} 
 
% section introduction (end)

% \input{qm2pi.knotations} 

% section notation (end)

\input{qm2pi.process.calculi} 

% section concurrent_process_calculi_and_spatial_logics_ (end)
    
%\input{qm2pi.knots2pi} 

%\input{qm2pi.trefoil} 

%\input{qm2pi.mainthm} 

% subsection basic_interpretation (end)

%\input{qm2pi.rho.presentation} 
\subsection{The syntax and semantics of the notation system}\label{sub:the_syntax_and_semantics_of_the_notation_system} % (fold)

We now summarize a technical presentation of the calculus that
embodies our theory of dynamics. The typical presentation of such a
calculus follows the style of giving generators and relations on
them. The grammar, below, describing term constructors, freely
generates the set of processes, $\Proc$. This set is then quotiented
by a relation known as structural congruence and it is over this set
that the notion of dynamics is expressed. This presentation is
essentially that of \cite{MeredithR05} with the addition of
polyadicity and summation. For readability we have relegated some of
the technical subtleties to an appendix.

\subsubsection{Process grammar}\label{subsub:process_grammar}

\begin{mathpar}
  \inferrule* [lab=synchronization] {} {{M} \bc \pzero \;|\; x?F \;|\; x!C }
  \and
  \inferrule* [lab=abstraction] {} {{F} \bc (x)P}
  \and
  \inferrule* [lab=concretion] {} {{C} \bc \langle Q \rangle}
  \and
  \inferrule* [lab=process] {} {{P,Q} \bc M \;| \;P|Q \;|\; @{x}}
  \and
  \inferrule* [lab=name] {} {{x} \bc \quotep{P}}
\end{mathpar} 

Note that $\vec{x}$ (resp. $\vec{P}$) denotes a vector of names
(resp. processes) of length $|\vec{x}|$ (resp. $|\vec{P}|$). We adopt
the following useful abbreviations.

\begin{mathpar}
   x?(\vec{y}).P := x.(\vec{y})P \and  x\clift{\vec{P}} := x.\clift{\vec{P}}
   \and x!(y) := \lift{x}{\dropn{y}}
   \and \Pi_{i=0}^{n-1}P_i := P_0 | \ldots | P_{n-1}
\end{mathpar}

\subsubsection{Structural congruence}

\paragraph{Free and bound names and alpha-equivalence.} At the
core of structural equivalence is alpha-equivalence which identifies
process that are the same up to a change of variable. Formally, we
recognize the distinction between free and bound names. The free names
of a process, $\freenames{P}$, may be calculated recursively as
follows:

\begin{mathpar}
\freenames{\pzero} := \emptyset
  \and \\
  \freenames{x?(y).P} := \{ x \} \cup (\freenames{P} \setminus \{ y \})
  \and 
  \freenames{x!\langle P \rangle} := \{ x \} \cup \{ P \} 
  \and \\
  \freenames{P|Q} := \freenames{P} \cup \freenames{Q}
  \and \\
  \freenames{@{x}} := \{ x \}
\end{mathpar}

$\pi$
$\quotep{\pi}$

$\freenames{-} : \pi \to \mathcal{P}(\quotep{\pi})$

\begin{eqnarray*}
  \freenames{\pzero} & := & \emptyset \\
  \freenames{x?(y).P} & := & \{ x \} \cup (\freenames{P} \setminus \{ y \}) \\
  \freenames{x!\langle P \rangle} & := & \{ x \} \cup \{ P \} \\
  \freenames{P|Q} & := & \freenames{P} \cup \freenames{Q} \\
  \freenames{\dropn{x}} & := & \{ x \}
\end{eqnarray*}

The bound names of a process, $\boundnames{P}$, are those names occurring in $P$
that are not free. For example, in $x?(y).0$, the name $x$ is free, while $y$ is bound.

\begin{mathpar}
  \inferrule* [lab=monoidal-laws] {} { P|Q \equiv Q|P \and P|0 \equiv P \and P|(Q|R) \equiv (P|Q)|R }
\end{mathpar}

\begin{mathpar}
  \inferrule* [lab=alpha-equivalence] {} { (x)P \equiv (y)P\{y/x\} \and y \not\in \freenames{P} }
\end{mathpar}

\begin{definition}
Then two processes, $P,Q$, are alpha-equivalent if $P = Q\{\vec{y}/\vec{x}\}$ for
some $\vec{x} \in \boundnames{Q},\vec{y} \in \boundnames{P}$, where $Q\{\vec{y}/\vec{x}\}$
denotes the capture-avoiding substitution of $\vec{y}$ for $\vec{x}$ in $Q$.
\end{definition}

\begin{definition}
  The {\em structural congruence} \cite{SangiorgiWalker} , $\equiv$,
  between processes is the least congruence containing
  alpha-equivalence, satisfying the abelian monoid laws
  (associativity, commutativity and $\pzero$ as identity) for parallel
  composition $|$ and for summation $+$.
\end{definition}

\subsection{Name equivalence}

We take name equivalence, written $\nameeq$, to be the smallest
equivalence relation generated by the following rules.

\begin{mathpar}
\inferrule*[lab=Quote-drop]
{ }
{ \quotep{@{x}} \nameeq x }

\inferrule*[lab=Struct-equiv]
{ P \scong Q }
{ \quotep{P} \nameeq \quotep{Q} }
\end{mathpar}

The astute reader will have noticed that the mutual recursion of names
and processes imposes a mutual recursion on alpha-equivalence and
structural equivalence via name-equivalence. Fortunately, all of this
works out pleasantly and we may calculate in the natural way, free of
concern. The reader interested in the details is referred to the
appendix \ref{appendix:rho_details}.

\subsection{Substitution}

We use $\Proc$ for the set of processes, $\QProc$ for the set of
names, and $\id{\{}\vec{y} / \vec{x} \id{\}}$ to denote partial maps,
$s : \QProc \rightarrow \QProc$. A map, $s$ lifts, uniquely, to a map
on process terms, $\widehat{s} : \Proc \rightarrow \Proc$ by the
following equations.

\begin{mathpar}
  (0) \psubstp{Q}{P} := 0 \\
  (R \juxtap S) \psubstp{Q}{P}
  :=    
  (R)\psubstp{Q}{P} \juxtap (S) \psubstp{Q}{P} \\
  (x?(y).R) \psubstp{Q}{P}    
  :=    
  (x)\substp{Q}{P} (z)\concat( (R \psubstn{z}{y}) \psubstp{Q}{P} ) \\
  (\lift{x}{R}) \psubstp{Q}{P}  
  :=
  \lift{(x)\substp{Q}{P}}{ R \psubstp{Q}{P} } \\
%   (\dropn{x})  \psubstp{Q}{P}       
%   := 
%   \left\{ 
%     \begin{array}{ccc} 
%       \dropn{\quotep{Q}} & & x \nameeq \quotep{P} \\
%       \dropn{x} & & otherwise \\
%     \end{array}
%   \right. 
  (\dropn{x})  \psubstp{Q}{P}       
  := 
  \left\{ 
    \begin{array}{ccc} 
      Q & & x \nameeq \quotep{P} \\
      \dropn{x} & & otherwise \\
    \end{array}
  \right.
\end{mathpar}
 

where

\begin{eqnarray}
  (x)\id{\{} \lpquote Q \rpquote / \lpquote P \rpquote \id{\}}            = 
  \left\{ 
    \begin{array}{ccc}
      \lpquote Q \rpquote & & x \nameeq \lpquote P \rpquote \\
      x & & otherwise \\
    \end{array}
  \right. \nonumber
\end{eqnarray}

and $z$ is chosen distinct from $\quotep{P}$, $\quotep{Q}$, the free
names in $Q$, and all the names in $R$. Our $\alpha$-equivalence will
be built in the standard way from this substitution.

\begin{remark}\label{rem:no_self_referential_names}
  One consequence of these definitions is that $\forall P. \quotep{P}
  \not\in \freenames{P}$.
\end{remark}

\subsection{ Dynamic quote: an example }

Anticipating something of what's to come, consider applying the
substitution, $\widehat{\id{\{}u / z \id{\}}}$, to the following pair
of processes, $\lift{w}{y!(z)}$ and $w[ \lpquote y!(z) \rpquote ]$.

\begin{eqnarray}
	\lift{w}{y!(z)}\widehat{\id{\{}u / z \id{\}}}
		& = &
		\lift{w}{y!(u)} \nonumber\\
	w[ \lpquote y!(z) \rpquote ] \widehat{ \id{\{}u / z \id{\}} }
		& = &
		w[ \lpquote y!(z) \rpquote ] \nonumber
\end{eqnarray}

Because the body of the process between quotes is impervious to
substitution, we get radically different answers. In fact, by
examining the first process in an input context,
e.g. $x?(z).\lift{w}{y!(z)}$, we see that the process under the lift
operator may be shaped by prefixed inputs binding a name inside it. In
this sense, the lift operator will be seen as a way to dynamically
construct processes before reifying them as names.

Finally equipped with these standard features we can present the
dynamics of the calculus.

\subsubsection{Operational semantics} 

Finally, we introduce the computational dynamics. What marks these
algebras as distinct from other more traditionally studied algebraic
structures, e.g. vector spaces or polynomial rings, is the manner in
which dynamics is captured. In traditional structures, dynamics is typically
expressed through morphisms between such structures, as in linear maps
between vector spaces or morphisms between rings. In algebras
associated with the semantics of computation, the dynamics is
expressed as part of the algebraic structure itself, through a
reduction reduction relation typically denoted by $\red$. Below, we
give a recursive presentation of this relation for the calculus used
in the encoding.

$\red \subseteq \pi \times \pi$
$\red : \pi \to \mathcal{P}(\pi)$

\begin{mathpar}
  \inferrule* [lab=Comm] { \textsf{match}( x_{src}, x_{trgt} ) } { x_{trgt}?(y)P \; | \; x_{src}!\langle {Q} \rangle \red P\{\quotep{Q}/y}\} }
  \and \\
  \inferrule* [lab=Par] {{P} \red {P}'} {{{P} | {Q}} \red {{P}' | {Q}}}
  \and
  \inferrule* [lab=Equiv]{{{P} \scong {P}'} \andalso {{P}' \red {Q}'} \andalso {{Q}' \scong {Q}}}{{P} \red {Q}}
\end{mathpar}

\begin{eqnarray*}
  match_{\equiv} (\quotep{P},\quotep{Q}) & := & P \equiv Q \\
  match_{\dagger}(\quotep{P},\quotep{Q}) & := & \forall R. P|Q \red^{*} R => R \red^{*} 0 \\
  match_{K}(\quotep{P},\quotep{Q}) & := & K \mbox{ for some context } K
\end{eqnarray*}

$u?(x)P | u!\langle Q \rangle \red P\{\quotep{Q}/x\}$

%We write $\wred$ for $\red^*$, and $P\red$ if $\exists Q $ such that $ P \red Q$.
We write $P\red$ if $\exists Q $ such that $ P \red Q$ and $P\not\red$, otherwise.

\section{Replication}

As mentioned before, it is known that replication (and hence
recursion) can be implemented in a higher-order process algebra
\cite{SangiorgiWalker}. As our first example of calculation with the
machinery thus far presented we give the construction explicitly in
the {\rhoc}.

\begin{eqnarray}
	D_{x} & := & \prefix{x}{y}{(\binpar{\outputp{x}{y}}{@{y}})} \nonumber\\
	\bangp_{x}{P} & := & \binpar{{x}!\langle{\binpar{D_{x}}{P}}\rangle}{D_{x}} \nonumber
\end{eqnarray}

\begin{eqnarray}
	\bangp_{x}{P} & & \nonumber\\
	=
	& {x}!\langle{(\prefix{x}{y}{(\outputp{x}{y} | @{y})) | P}}\rangle 
	      | \prefix{x}{y}{(\outputp{x}{y} | @{y})} & \nonumber\\
	\red
	& (\outputp{x}{y} | @{y})\substn{\quotep{(\prefix{x}{y}{(@{y} | \outputp{x}{y})) | P}}}{y} & \nonumber\\
	=
	& \outputp{x}{\quotep{(\prefix{x}{y}{(\outputp{x}{y} | @{y})) | P}}}
	  | {(\prefix{x}{y}{(\outputp{x}{y} | @{y})) | P}} & \nonumber\\
	\red
	& \ldots & \nonumber\\
	\red^*
	& P | P | \ldots & \nonumber
\end{eqnarray}

Of course, this encoding, as an implementation, runs away, unfolding
$\bangp{P}$ eagerly. A lazier and more implementable replication
operator, restricted to input-guarded processes, may be obtained as follows.

\begin{eqnarray}
\bangp{\prefix{u}{v}{P}} 
	:= 
	\binpar{\lift{x}{\prefix{u}{v}{(\binpar{D(x)}{P})}}}{D(x)} \nonumber
\end{eqnarray}

\begin{remark}
  Note that the lazier definition still does not deal with summation
  or mixed summation (i.e. sums over input and output). The reader is
  invited to construct definitions of replication that deal with these
  features. 

  Further, the definitions are parameterized in a name, $x$. Can you,
  gentle reader, make a definition that eliminates this parameter and
  guarantees no accidental interaction between the replication
  machinery and the process being replicated -- i.e. no accidental
  sharing of names used by the process to get its work done and the
  name(s) used by the replication to effect copying. This latter
  revision of the definition of replication is crucial to obtaining
  the expected identity $!!P \sim !P$.
\end{remark}

\begin{remark}\label{rem:paradoxical_combinator}
  The reader familiar with the lambda calculus will have noticed the
  similarity between $D$ and the paradoxical combinator.

  [Ed. note: the existence of this seems to suggest we have to be more
  restrictive on the set of processes and names we admit if we are to
  support no-cloning.]
\end{remark}

\subsubsection{Bisimulation}

The computational dynamics gives rise to another kind of equivalence,
the equivalence of computational behavior. As previously mentioned
this is typically captured \emph{via} some form of bisimulation.

% The notion we use in this paper is weak barbed bisimulation
% \cite{milner91polyadicpi}.

The notion we use in this paper is derived from weak barbed
bisimulation \cite{milner91polyadicpi}. 

\begin{definition}
An \emph{observation relation}, $\downarrow_{\mathcal N}$, over a set
of names, $\mathcal N$, is the smallest relation satisfying the rules
below.

\infrule[Out-barb]{y \in {\mathcal N}, \; x \nameeq y}
		  {\outputp{x}{v} \downarrow_{\mathcal N} x}
\infrule[Par-barb]{\mbox{$P\downarrow_{\mathcal N} x$ or $Q\downarrow_{\mathcal N} x$}}
		  {\binpar{P}{Q} \downarrow_{\mathcal N} x}

We write $P \Downarrow_{\mathcal N} x$ if there is $Q$ such that 
$P \wred Q$ and $Q \downarrow_{\mathcal N} x$.
\end{definition}

\begin{definition}
%\label{def.bbisim}
An  ${\mathcal N}$-\emph{barbed bisimulation} over a set of names, ${\mathcal N}$, is a symmetric binary relation 
${\mathcal S}_{\mathcal N}$ between agents such that $P\rel{S}_{\mathcal N}Q$ implies:
\begin{enumerate}
\item If $P \red P'$ then $Q \wred Q'$ and $P'\rel{S}_{\mathcal N} Q'$.
\item If $P\downarrow_{\mathcal N} x$, then $Q\Downarrow_{\mathcal N} x$.
\end{enumerate}
$P$ is ${\mathcal N}$-barbed bisimilar to $Q$, written
$P \wbbisim_{\mathcal N} Q$, if $P \rel{S}_{\mathcal N} Q$ for some ${\mathcal N}$-barbed bisimulation ${\mathcal S}_{\mathcal N}$.
\end{definition}

$\mathcal{R} \subseteq \pi \times \pi$

$P \mathcal{R} Q => \forall P'. P \red P' \Rightarrow \exists Q'. Q \red Q', P' \mathcal{R} Q'$

$P \vdash x \Rightarrow Q \vdash x$

\begin{mathpar}
  \inferrule*[lab=Out-barb]{x \nameeq y}{{y}!\langle{Q}\rangle \vdash x}
  \and
  \inferrule*[lab=Par-barb]{\mbox{$P\vdash x$ or $Q\vdash x$}}{\binpar{P}{Q} \vdash x}
\end{mathpar}

\subsubsection{Contexts}

One of the principle advantages of computational calculi like the
$\pi$-calculus is a well-defined notion of context,
contextual-equivalence and a correlation between
contextual-equivalence and notions of bisimulation. The notion of
context allows the decomposition of a process into (sub-)process and
its syntactic environment, its context. Thus, a context may be
thought of as a process with a ``hole'' (written $\Box$) in it. The
application of a context $M$ to a process $P$, written $M[P]$, is
tantamount to filling the hole in $M$ with $P$. In this paper we do
not need the full weight of this theory, but do make use of the notion
of context in the proof the main theorem. 

\begin{mathpar}
  \inferrule* [lab=summation] {} {{M_{M},M_{N}} \bc \Box \;|\; x.M_{A} \;|\; M_{M}+M_{N}}
  \and
  \inferrule* [lab=agent] {} {{M_{A}} \bc (\vec{x})M_{P} \;| \; \clift{P_0,\ldots,M_{P},\ldots,P_N}}
  \and \\
  \inferrule* [lab=process] {} {{M_{P}} \bc M_{N} \;| \;P|M_{P} }
\end{mathpar} 

\begin{mathpar}
  \inferrule* [lab=sychronization] {} {M_{N} \bc \Box \;|\; x?M_{F} \;|\; x!M_{C}}
  \and
  \inferrule* [lab=abstraction] {} {{M_{F}} \bc (x)M_{P} }
  \and
  \inferrule* [lab=concretion] {} {{M_{C}} \bc \langle M_{P} \rangle }
  \and \\
  \inferrule* [lab=process] {} {{M_{P}} \bc M_{N} \;| \;P|M_{P} }
\end{mathpar}

\begin{definition}[contextual application] Given a context $M$, and
  process $P$, we define the \emph{contextual application}, $M[P] :=
  M\{P/\Box\}$. That is, the contextual application of M to P is the
  substitution of $P$ for $\Box$ in $M$.
\end{definition}

$\meaningof{-} : L \to \mathcal{P}(\pi)$

\begin{mathpar}
  \inferrule* [lab=collection] {} {\meaningof{true} = \pi, \and \meaningof{~E} = \pi \setminus \meaningof{E}, \and \meaningof{E_{1} \& E_{2}} = \meaningof{E_{1}} \cap \meaningof{E_{2}}}
\end{mathpar}

\begin{mathpar}
  \inferrule* [lab=structure] {} {\meaningof{0} = \{ P \in \pi | P \equiv 0 \}, \and \\ \meaningof{E_1 | E_2} = \{ P \in \pi | P \equiv P_{1} | P_{2}, P_{1} \in \meaningof{E_{1}}, P_{2} \in \meaningof{E_2}\} }
\end{mathpar}

\begin{mathpar}
 \inferrule* [lab=behavior] {} {\meaningof{\langle a?b \rangle E} = \{ P \in \pi | P \equiv Q | u?(y)P', \\ \and \\\\ \and \\ \;\;\; u \in \meaningof{a}, \forall z.P'\{z/y\} \in \meaningof{E\{z/b\}}\}, \and \\ \meaningof{a!E} = \{ P \in \pi | P \equiv Q | x!\langle P' \rangle, x \in \meaningof{a} P' \in \meaningof{E}\} }
\end{mathpar}

\begin{mathpar}
 \inferrule* [lab=nominal] {} {\meaningof{\quotep{E}} = \{ \quotep{P} \in \quotep{\pi} | P \in \meaningof{E} \}, \and \meaningof{\quotep{P}} = \{ \quotep{Q} \in \quotep{\pi} | P \equiv Q \} \and \\ \meaningof{@\quotep{E}} = \{ P \in \pi | P \equiv @x, x \in \meaningof{E} \}}
\end{mathpar}

\begin{eqnarray*}
  \\
  \meaningof{-} : TS \to ST
\end{eqnarray*}

\begin{eqnarray*}
  \\
  L : TS \to ST
\end{eqnarray*}

\begin{eqnarray*}
  \\
  P \models E \iff P \in \meaningof{E}
\end{eqnarray*}

\begin{eqnarray*}
  P \approx_{L} Q \iff \forall E \in L. P \models E \iff Q \models E
\end{eqnarray*}

\begin{eqnarray*}
  P \approx_{K} Q
\end{eqnarray*}

\begin{eqnarray*}
  P \approx Q
\end{eqnarray*}

$\approx_{K} = \approx = \approx_{L}$

\subsubsection{Contextual duality}

Note that contexts extend the quotation operation to a family of
operations from processes to names. Given a context, $M$, we can
define a \emph{nominal context}, $\quotep{M}$ by $\quotep{M}[P] :=
\quotep{M[P]}$. To foreshadow what is to come we observe that these
operations enjoy a duality with processes very much like the duality
between vectors and maps from vectors to scalars.

Further, because the calculus is essentially higher-order, we have a
correspondence between contexts and processes. More specifically,
given a name $x$ and a context $M$ we can construct $M^{*}_{x}$ such
that 

\begin{mathpar}
  M^{*}_{x} | \lift{x}{P} \red M[P]
\end{mathpar}

namely,

\begin{mathpar}
  M^{*}_{x} := x?(u).M[\dropn{u}]
\end{mathpar}

The dependence of $M^{*}_{x}$ on a name makes it an abstraction, 

\begin{mathpar}
  M^{*} := (x)x?(u).M[\dropn{u}]
\end{mathpar}

\subsection{Additional notation}

It will sometimes be convenient to denote the process a name
quotes. We already have the notation $x = \quotep{P}$, but it will be
convenient to introduce an alternate notation, $\procn{x}$, when we
want to emphasize the connection to the use of the name. Note that, by
virtue of name equivalence, $\quotep{\procn{x}} \nameeq x$; so, the
notation is consistent with previous definitions.

Further, because names have structure it is possible to effect
substitutions on the basis of that structure. This means we need to
upgrade our notation for substitutions, which we accomplish by
adapting comprehension notation. Thus,

\begin{mathpar}
  P\{ y / x : x \in S \}
\end{mathpar}

is interpreted to mean the process derived from P by replacing (in a
capture-avoiding manner) each occurrence of $x$ in $S$ by $y$. For example,

\begin{mathpar}
  P\{ \quotep{\procn{x}|\procn{x}} / x : x \in \freenames{P} \}
\end{mathpar}

will replace each (occurrence) of a free name $x$ in $P$ by
$\quotep{\procn{x}|\procn{x}}$.

Also, we will avail ourselves of the notation $x^{L}$ and $x^{R}$ to
denote injections of a name into disjoint copies of the name
space. There are numerous ways to accomplish this. One example can be
found in \cite{MeredithR05}. This notation overloads to vectors of
names: $\vec{x}^{\pi} := (x_{i}^{\pi} \; : \; 0 \leq i < |\vec{x}| )$ where $\pi \in \{L,R\}$.

We also use $P^{\Box} := P|\Box$.

In \cite{MeredithR05} an interpretation of the new operator is
given. It turns out that there are several possible interpretations
all enjoying the requisite algebraic properties of the operator (see
\cite{milner91polyadicpi}). We will therefore make liberal use of
$(\nu\; \vec{x})P$.

% subsection the_syntax_and_semantics_of_the_notation_system (end)   

\input{qm2pi.qmops} 

\input{qm2pi.sterngerlach} 

\input{qm2pi.metric} 

% section concurrent_process_calculi (end)

%\input{qm2pi.proofsketch}

% section proof sketch (end)

%\input{qm2pi.slviaknots} 

% section spatial logic via knots (end)

\input{qm2pi.conclusion}

% section conclusion (end)

%\input{qm2pi.dtcodes} 

% section wiring algorithm (end)

\input{qm2pi.ack} 

% section acknowledgments (end)

\newpage


\bibliographystyle{plain}   
\bibliography{../../biblios/main.bib}

\input{qm2pi.rhodetails}

\end{document}

 

%\ifpdf
%\usepackage[pdftex]{graphicx}
%\else
%\usepackage{graphicx}
%\fi

 % \ifpdf
%  \usepackage{pdfsync}
%  \if


%\title{Brief Article}
%\author{David F. Snyder}
%\author{L.G. Meredith}

%\address{Dept. of Math., Texas State University--San Marcos, San Marcos, TX 78666}
       
\pagestyle{empty}


\begin{document}

\lstset{language=[Objective]Caml,frame=shadowbox}

\documentclass[12pt]{llncs}
%\documentclass{jktr}

\usepackage[pdftex]{hyperref}                   
\usepackage {listings}
\usepackage {mathpartir}
\usepackage{bcprules}
%\usepackage{listings}
                       
\usepackage{graphicx} 
%\usepackage[margins=2.5cm,nohead,nofoot]{geometry}
%\usepackage{geometry}
\usepackage{amsfonts}
\usepackage{amstext}
\usepackage{latexsym}
\usepackage{amssymb}
\usepackage{color}


%\include{myPreamble}
\include{qm2pi.local} 

%\ifpdf
%\usepackage[pdftex]{graphicx}
%\else
%\usepackage{graphicx}
%\fi

 % \ifpdf
%  \usepackage{pdfsync}
%  \if


%\title{Brief Article}
%\author{David F. Snyder}
%\author{L.G. Meredith}

%\address{Dept. of Math., Texas State University--San Marcos, San Marcos, TX 78666}
       
\pagestyle{empty}


\begin{document}

\lstset{language=[Objective]Caml,frame=shadowbox}

\input{qm2pi.front}

% section front matter (end)

\input{qm2pi.intro} 
 
% section introduction (end)

% \input{qm2pi.knotations} 

% section notation (end)

\input{qm2pi.process.calculi} 

% section concurrent_process_calculi_and_spatial_logics_ (end)
    
%\input{qm2pi.knots2pi} 

%\input{qm2pi.trefoil} 

%\input{qm2pi.mainthm} 

% subsection basic_interpretation (end)

%\input{qm2pi.rho.presentation} 
\subsection{The syntax and semantics of the notation system}\label{sub:the_syntax_and_semantics_of_the_notation_system} % (fold)

We now summarize a technical presentation of the calculus that
embodies our theory of dynamics. The typical presentation of such a
calculus follows the style of giving generators and relations on
them. The grammar, below, describing term constructors, freely
generates the set of processes, $\Proc$. This set is then quotiented
by a relation known as structural congruence and it is over this set
that the notion of dynamics is expressed. This presentation is
essentially that of \cite{MeredithR05} with the addition of
polyadicity and summation. For readability we have relegated some of
the technical subtleties to an appendix.

\subsubsection{Process grammar}\label{subsub:process_grammar}

\begin{mathpar}
  \inferrule* [lab=synchronization] {} {{M} \bc \pzero \;|\; x?F \;|\; x!C }
  \and
  \inferrule* [lab=abstraction] {} {{F} \bc (x)P}
  \and
  \inferrule* [lab=concretion] {} {{C} \bc \langle Q \rangle}
  \and
  \inferrule* [lab=process] {} {{P,Q} \bc M \;| \;P|Q \;|\; @{x}}
  \and
  \inferrule* [lab=name] {} {{x} \bc \quotep{P}}
\end{mathpar} 

Note that $\vec{x}$ (resp. $\vec{P}$) denotes a vector of names
(resp. processes) of length $|\vec{x}|$ (resp. $|\vec{P}|$). We adopt
the following useful abbreviations.

\begin{mathpar}
   x?(\vec{y}).P := x.(\vec{y})P \and  x\clift{\vec{P}} := x.\clift{\vec{P}}
   \and x!(y) := \lift{x}{\dropn{y}}
   \and \Pi_{i=0}^{n-1}P_i := P_0 | \ldots | P_{n-1}
\end{mathpar}

\subsubsection{Structural congruence}

\paragraph{Free and bound names and alpha-equivalence.} At the
core of structural equivalence is alpha-equivalence which identifies
process that are the same up to a change of variable. Formally, we
recognize the distinction between free and bound names. The free names
of a process, $\freenames{P}$, may be calculated recursively as
follows:

\begin{mathpar}
\freenames{\pzero} := \emptyset
  \and \\
  \freenames{x?(y).P} := \{ x \} \cup (\freenames{P} \setminus \{ y \})
  \and 
  \freenames{x!\langle P \rangle} := \{ x \} \cup \{ P \} 
  \and \\
  \freenames{P|Q} := \freenames{P} \cup \freenames{Q}
  \and \\
  \freenames{@{x}} := \{ x \}
\end{mathpar}

$\pi$
$\quotep{\pi}$

$\freenames{-} : \pi \to \mathcal{P}(\quotep{\pi})$

\begin{eqnarray*}
  \freenames{\pzero} & := & \emptyset \\
  \freenames{x?(y).P} & := & \{ x \} \cup (\freenames{P} \setminus \{ y \}) \\
  \freenames{x!\langle P \rangle} & := & \{ x \} \cup \{ P \} \\
  \freenames{P|Q} & := & \freenames{P} \cup \freenames{Q} \\
  \freenames{\dropn{x}} & := & \{ x \}
\end{eqnarray*}

The bound names of a process, $\boundnames{P}$, are those names occurring in $P$
that are not free. For example, in $x?(y).0$, the name $x$ is free, while $y$ is bound.

\begin{mathpar}
  \inferrule* [lab=monoidal-laws] {} { P|Q \equiv Q|P \and P|0 \equiv P \and P|(Q|R) \equiv (P|Q)|R }
\end{mathpar}

\begin{mathpar}
  \inferrule* [lab=alpha-equivalence] {} { (x)P \equiv (y)P\{y/x\} \and y \not\in \freenames{P} }
\end{mathpar}

\begin{definition}
Then two processes, $P,Q$, are alpha-equivalent if $P = Q\{\vec{y}/\vec{x}\}$ for
some $\vec{x} \in \boundnames{Q},\vec{y} \in \boundnames{P}$, where $Q\{\vec{y}/\vec{x}\}$
denotes the capture-avoiding substitution of $\vec{y}$ for $\vec{x}$ in $Q$.
\end{definition}

\begin{definition}
  The {\em structural congruence} \cite{SangiorgiWalker} , $\equiv$,
  between processes is the least congruence containing
  alpha-equivalence, satisfying the abelian monoid laws
  (associativity, commutativity and $\pzero$ as identity) for parallel
  composition $|$ and for summation $+$.
\end{definition}

\subsection{Name equivalence}

We take name equivalence, written $\nameeq$, to be the smallest
equivalence relation generated by the following rules.

\begin{mathpar}
\inferrule*[lab=Quote-drop]
{ }
{ \quotep{@{x}} \nameeq x }

\inferrule*[lab=Struct-equiv]
{ P \scong Q }
{ \quotep{P} \nameeq \quotep{Q} }
\end{mathpar}

The astute reader will have noticed that the mutual recursion of names
and processes imposes a mutual recursion on alpha-equivalence and
structural equivalence via name-equivalence. Fortunately, all of this
works out pleasantly and we may calculate in the natural way, free of
concern. The reader interested in the details is referred to the
appendix \ref{appendix:rho_details}.

\subsection{Substitution}

We use $\Proc$ for the set of processes, $\QProc$ for the set of
names, and $\id{\{}\vec{y} / \vec{x} \id{\}}$ to denote partial maps,
$s : \QProc \rightarrow \QProc$. A map, $s$ lifts, uniquely, to a map
on process terms, $\widehat{s} : \Proc \rightarrow \Proc$ by the
following equations.

\begin{mathpar}
  (0) \psubstp{Q}{P} := 0 \\
  (R \juxtap S) \psubstp{Q}{P}
  :=    
  (R)\psubstp{Q}{P} \juxtap (S) \psubstp{Q}{P} \\
  (x?(y).R) \psubstp{Q}{P}    
  :=    
  (x)\substp{Q}{P} (z)\concat( (R \psubstn{z}{y}) \psubstp{Q}{P} ) \\
  (\lift{x}{R}) \psubstp{Q}{P}  
  :=
  \lift{(x)\substp{Q}{P}}{ R \psubstp{Q}{P} } \\
%   (\dropn{x})  \psubstp{Q}{P}       
%   := 
%   \left\{ 
%     \begin{array}{ccc} 
%       \dropn{\quotep{Q}} & & x \nameeq \quotep{P} \\
%       \dropn{x} & & otherwise \\
%     \end{array}
%   \right. 
  (\dropn{x})  \psubstp{Q}{P}       
  := 
  \left\{ 
    \begin{array}{ccc} 
      Q & & x \nameeq \quotep{P} \\
      \dropn{x} & & otherwise \\
    \end{array}
  \right.
\end{mathpar}
 

where

\begin{eqnarray}
  (x)\id{\{} \lpquote Q \rpquote / \lpquote P \rpquote \id{\}}            = 
  \left\{ 
    \begin{array}{ccc}
      \lpquote Q \rpquote & & x \nameeq \lpquote P \rpquote \\
      x & & otherwise \\
    \end{array}
  \right. \nonumber
\end{eqnarray}

and $z$ is chosen distinct from $\quotep{P}$, $\quotep{Q}$, the free
names in $Q$, and all the names in $R$. Our $\alpha$-equivalence will
be built in the standard way from this substitution.

\begin{remark}\label{rem:no_self_referential_names}
  One consequence of these definitions is that $\forall P. \quotep{P}
  \not\in \freenames{P}$.
\end{remark}

\subsection{ Dynamic quote: an example }

Anticipating something of what's to come, consider applying the
substitution, $\widehat{\id{\{}u / z \id{\}}}$, to the following pair
of processes, $\lift{w}{y!(z)}$ and $w[ \lpquote y!(z) \rpquote ]$.

\begin{eqnarray}
	\lift{w}{y!(z)}\widehat{\id{\{}u / z \id{\}}}
		& = &
		\lift{w}{y!(u)} \nonumber\\
	w[ \lpquote y!(z) \rpquote ] \widehat{ \id{\{}u / z \id{\}} }
		& = &
		w[ \lpquote y!(z) \rpquote ] \nonumber
\end{eqnarray}

Because the body of the process between quotes is impervious to
substitution, we get radically different answers. In fact, by
examining the first process in an input context,
e.g. $x?(z).\lift{w}{y!(z)}$, we see that the process under the lift
operator may be shaped by prefixed inputs binding a name inside it. In
this sense, the lift operator will be seen as a way to dynamically
construct processes before reifying them as names.

Finally equipped with these standard features we can present the
dynamics of the calculus.

\subsubsection{Operational semantics} 

Finally, we introduce the computational dynamics. What marks these
algebras as distinct from other more traditionally studied algebraic
structures, e.g. vector spaces or polynomial rings, is the manner in
which dynamics is captured. In traditional structures, dynamics is typically
expressed through morphisms between such structures, as in linear maps
between vector spaces or morphisms between rings. In algebras
associated with the semantics of computation, the dynamics is
expressed as part of the algebraic structure itself, through a
reduction reduction relation typically denoted by $\red$. Below, we
give a recursive presentation of this relation for the calculus used
in the encoding.

$\red \subseteq \pi \times \pi$
$\red : \pi \to \mathcal{P}(\pi)$

\begin{mathpar}
  \inferrule* [lab=Comm] { \textsf{match}( x_{src}, x_{trgt} ) } { x_{trgt}?(y)P \; | \; x_{src}!\langle {Q} \rangle \red P\{\quotep{Q}/y}\} }
  \and \\
  \inferrule* [lab=Par] {{P} \red {P}'} {{{P} | {Q}} \red {{P}' | {Q}}}
  \and
  \inferrule* [lab=Equiv]{{{P} \scong {P}'} \andalso {{P}' \red {Q}'} \andalso {{Q}' \scong {Q}}}{{P} \red {Q}}
\end{mathpar}

\begin{eqnarray*}
  match_{\equiv} (\quotep{P},\quotep{Q}) & := & P \equiv Q \\
  match_{\dagger}(\quotep{P},\quotep{Q}) & := & \forall R. P|Q \red^{*} R => R \red^{*} 0 \\
  match_{K}(\quotep{P},\quotep{Q}) & := & K \mbox{ for some context } K
\end{eqnarray*}

$u?(x)P | u!\langle Q \rangle \red P\{\quotep{Q}/x\}$

%We write $\wred$ for $\red^*$, and $P\red$ if $\exists Q $ such that $ P \red Q$.
We write $P\red$ if $\exists Q $ such that $ P \red Q$ and $P\not\red$, otherwise.

\section{Replication}

As mentioned before, it is known that replication (and hence
recursion) can be implemented in a higher-order process algebra
\cite{SangiorgiWalker}. As our first example of calculation with the
machinery thus far presented we give the construction explicitly in
the {\rhoc}.

\begin{eqnarray}
	D_{x} & := & \prefix{x}{y}{(\binpar{\outputp{x}{y}}{@{y}})} \nonumber\\
	\bangp_{x}{P} & := & \binpar{{x}!\langle{\binpar{D_{x}}{P}}\rangle}{D_{x}} \nonumber
\end{eqnarray}

\begin{eqnarray}
	\bangp_{x}{P} & & \nonumber\\
	=
	& {x}!\langle{(\prefix{x}{y}{(\outputp{x}{y} | @{y})) | P}}\rangle 
	      | \prefix{x}{y}{(\outputp{x}{y} | @{y})} & \nonumber\\
	\red
	& (\outputp{x}{y} | @{y})\substn{\quotep{(\prefix{x}{y}{(@{y} | \outputp{x}{y})) | P}}}{y} & \nonumber\\
	=
	& \outputp{x}{\quotep{(\prefix{x}{y}{(\outputp{x}{y} | @{y})) | P}}}
	  | {(\prefix{x}{y}{(\outputp{x}{y} | @{y})) | P}} & \nonumber\\
	\red
	& \ldots & \nonumber\\
	\red^*
	& P | P | \ldots & \nonumber
\end{eqnarray}

Of course, this encoding, as an implementation, runs away, unfolding
$\bangp{P}$ eagerly. A lazier and more implementable replication
operator, restricted to input-guarded processes, may be obtained as follows.

\begin{eqnarray}
\bangp{\prefix{u}{v}{P}} 
	:= 
	\binpar{\lift{x}{\prefix{u}{v}{(\binpar{D(x)}{P})}}}{D(x)} \nonumber
\end{eqnarray}

\begin{remark}
  Note that the lazier definition still does not deal with summation
  or mixed summation (i.e. sums over input and output). The reader is
  invited to construct definitions of replication that deal with these
  features. 

  Further, the definitions are parameterized in a name, $x$. Can you,
  gentle reader, make a definition that eliminates this parameter and
  guarantees no accidental interaction between the replication
  machinery and the process being replicated -- i.e. no accidental
  sharing of names used by the process to get its work done and the
  name(s) used by the replication to effect copying. This latter
  revision of the definition of replication is crucial to obtaining
  the expected identity $!!P \sim !P$.
\end{remark}

\begin{remark}\label{rem:paradoxical_combinator}
  The reader familiar with the lambda calculus will have noticed the
  similarity between $D$ and the paradoxical combinator.

  [Ed. note: the existence of this seems to suggest we have to be more
  restrictive on the set of processes and names we admit if we are to
  support no-cloning.]
\end{remark}

\subsubsection{Bisimulation}

The computational dynamics gives rise to another kind of equivalence,
the equivalence of computational behavior. As previously mentioned
this is typically captured \emph{via} some form of bisimulation.

% The notion we use in this paper is weak barbed bisimulation
% \cite{milner91polyadicpi}.

The notion we use in this paper is derived from weak barbed
bisimulation \cite{milner91polyadicpi}. 

\begin{definition}
An \emph{observation relation}, $\downarrow_{\mathcal N}$, over a set
of names, $\mathcal N$, is the smallest relation satisfying the rules
below.

\infrule[Out-barb]{y \in {\mathcal N}, \; x \nameeq y}
		  {\outputp{x}{v} \downarrow_{\mathcal N} x}
\infrule[Par-barb]{\mbox{$P\downarrow_{\mathcal N} x$ or $Q\downarrow_{\mathcal N} x$}}
		  {\binpar{P}{Q} \downarrow_{\mathcal N} x}

We write $P \Downarrow_{\mathcal N} x$ if there is $Q$ such that 
$P \wred Q$ and $Q \downarrow_{\mathcal N} x$.
\end{definition}

\begin{definition}
%\label{def.bbisim}
An  ${\mathcal N}$-\emph{barbed bisimulation} over a set of names, ${\mathcal N}$, is a symmetric binary relation 
${\mathcal S}_{\mathcal N}$ between agents such that $P\rel{S}_{\mathcal N}Q$ implies:
\begin{enumerate}
\item If $P \red P'$ then $Q \wred Q'$ and $P'\rel{S}_{\mathcal N} Q'$.
\item If $P\downarrow_{\mathcal N} x$, then $Q\Downarrow_{\mathcal N} x$.
\end{enumerate}
$P$ is ${\mathcal N}$-barbed bisimilar to $Q$, written
$P \wbbisim_{\mathcal N} Q$, if $P \rel{S}_{\mathcal N} Q$ for some ${\mathcal N}$-barbed bisimulation ${\mathcal S}_{\mathcal N}$.
\end{definition}

$\mathcal{R} \subseteq \pi \times \pi$

$P \mathcal{R} Q => \forall P'. P \red P' \Rightarrow \exists Q'. Q \red Q', P' \mathcal{R} Q'$

$P \vdash x \Rightarrow Q \vdash x$

\begin{mathpar}
  \inferrule*[lab=Out-barb]{x \nameeq y}{{y}!\langle{Q}\rangle \vdash x}
  \and
  \inferrule*[lab=Par-barb]{\mbox{$P\vdash x$ or $Q\vdash x$}}{\binpar{P}{Q} \vdash x}
\end{mathpar}

\subsubsection{Contexts}

One of the principle advantages of computational calculi like the
$\pi$-calculus is a well-defined notion of context,
contextual-equivalence and a correlation between
contextual-equivalence and notions of bisimulation. The notion of
context allows the decomposition of a process into (sub-)process and
its syntactic environment, its context. Thus, a context may be
thought of as a process with a ``hole'' (written $\Box$) in it. The
application of a context $M$ to a process $P$, written $M[P]$, is
tantamount to filling the hole in $M$ with $P$. In this paper we do
not need the full weight of this theory, but do make use of the notion
of context in the proof the main theorem. 

\begin{mathpar}
  \inferrule* [lab=summation] {} {{M_{M},M_{N}} \bc \Box \;|\; x.M_{A} \;|\; M_{M}+M_{N}}
  \and
  \inferrule* [lab=agent] {} {{M_{A}} \bc (\vec{x})M_{P} \;| \; \clift{P_0,\ldots,M_{P},\ldots,P_N}}
  \and \\
  \inferrule* [lab=process] {} {{M_{P}} \bc M_{N} \;| \;P|M_{P} }
\end{mathpar} 

\begin{mathpar}
  \inferrule* [lab=sychronization] {} {M_{N} \bc \Box \;|\; x?M_{F} \;|\; x!M_{C}}
  \and
  \inferrule* [lab=abstraction] {} {{M_{F}} \bc (x)M_{P} }
  \and
  \inferrule* [lab=concretion] {} {{M_{C}} \bc \langle M_{P} \rangle }
  \and \\
  \inferrule* [lab=process] {} {{M_{P}} \bc M_{N} \;| \;P|M_{P} }
\end{mathpar}

\begin{definition}[contextual application] Given a context $M$, and
  process $P$, we define the \emph{contextual application}, $M[P] :=
  M\{P/\Box\}$. That is, the contextual application of M to P is the
  substitution of $P$ for $\Box$ in $M$.
\end{definition}

$\meaningof{-} : L \to \mathcal{P}(\pi)$

\begin{mathpar}
  \inferrule* [lab=collection] {} {\meaningof{true} = \pi, \and \meaningof{~E} = \pi \setminus \meaningof{E}, \and \meaningof{E_{1} \& E_{2}} = \meaningof{E_{1}} \cap \meaningof{E_{2}}}
\end{mathpar}

\begin{mathpar}
  \inferrule* [lab=structure] {} {\meaningof{0} = \{ P \in \pi | P \equiv 0 \}, \and \\ \meaningof{E_1 | E_2} = \{ P \in \pi | P \equiv P_{1} | P_{2}, P_{1} \in \meaningof{E_{1}}, P_{2} \in \meaningof{E_2}\} }
\end{mathpar}

\begin{mathpar}
 \inferrule* [lab=behavior] {} {\meaningof{\langle a?b \rangle E} = \{ P \in \pi | P \equiv Q | u?(y)P', \\ \and \\\\ \and \\ \;\;\; u \in \meaningof{a}, \forall z.P'\{z/y\} \in \meaningof{E\{z/b\}}\}, \and \\ \meaningof{a!E} = \{ P \in \pi | P \equiv Q | x!\langle P' \rangle, x \in \meaningof{a} P' \in \meaningof{E}\} }
\end{mathpar}

\begin{mathpar}
 \inferrule* [lab=nominal] {} {\meaningof{\quotep{E}} = \{ \quotep{P} \in \quotep{\pi} | P \in \meaningof{E} \}, \and \meaningof{\quotep{P}} = \{ \quotep{Q} \in \quotep{\pi} | P \equiv Q \} \and \\ \meaningof{@\quotep{E}} = \{ P \in \pi | P \equiv @x, x \in \meaningof{E} \}}
\end{mathpar}

\begin{eqnarray*}
  \\
  \meaningof{-} : TS \to ST
\end{eqnarray*}

\begin{eqnarray*}
  \\
  L : TS \to ST
\end{eqnarray*}

\begin{eqnarray*}
  \\
  P \models E \iff P \in \meaningof{E}
\end{eqnarray*}

\begin{eqnarray*}
  P \approx_{L} Q \iff \forall E \in L. P \models E \iff Q \models E
\end{eqnarray*}

\begin{eqnarray*}
  P \approx_{K} Q
\end{eqnarray*}

\begin{eqnarray*}
  P \approx Q
\end{eqnarray*}

$\approx_{K} = \approx = \approx_{L}$

\subsubsection{Contextual duality}

Note that contexts extend the quotation operation to a family of
operations from processes to names. Given a context, $M$, we can
define a \emph{nominal context}, $\quotep{M}$ by $\quotep{M}[P] :=
\quotep{M[P]}$. To foreshadow what is to come we observe that these
operations enjoy a duality with processes very much like the duality
between vectors and maps from vectors to scalars.

Further, because the calculus is essentially higher-order, we have a
correspondence between contexts and processes. More specifically,
given a name $x$ and a context $M$ we can construct $M^{*}_{x}$ such
that 

\begin{mathpar}
  M^{*}_{x} | \lift{x}{P} \red M[P]
\end{mathpar}

namely,

\begin{mathpar}
  M^{*}_{x} := x?(u).M[\dropn{u}]
\end{mathpar}

The dependence of $M^{*}_{x}$ on a name makes it an abstraction, 

\begin{mathpar}
  M^{*} := (x)x?(u).M[\dropn{u}]
\end{mathpar}

\subsection{Additional notation}

It will sometimes be convenient to denote the process a name
quotes. We already have the notation $x = \quotep{P}$, but it will be
convenient to introduce an alternate notation, $\procn{x}$, when we
want to emphasize the connection to the use of the name. Note that, by
virtue of name equivalence, $\quotep{\procn{x}} \nameeq x$; so, the
notation is consistent with previous definitions.

Further, because names have structure it is possible to effect
substitutions on the basis of that structure. This means we need to
upgrade our notation for substitutions, which we accomplish by
adapting comprehension notation. Thus,

\begin{mathpar}
  P\{ y / x : x \in S \}
\end{mathpar}

is interpreted to mean the process derived from P by replacing (in a
capture-avoiding manner) each occurrence of $x$ in $S$ by $y$. For example,

\begin{mathpar}
  P\{ \quotep{\procn{x}|\procn{x}} / x : x \in \freenames{P} \}
\end{mathpar}

will replace each (occurrence) of a free name $x$ in $P$ by
$\quotep{\procn{x}|\procn{x}}$.

Also, we will avail ourselves of the notation $x^{L}$ and $x^{R}$ to
denote injections of a name into disjoint copies of the name
space. There are numerous ways to accomplish this. One example can be
found in \cite{MeredithR05}. This notation overloads to vectors of
names: $\vec{x}^{\pi} := (x_{i}^{\pi} \; : \; 0 \leq i < |\vec{x}| )$ where $\pi \in \{L,R\}$.

We also use $P^{\Box} := P|\Box$.

In \cite{MeredithR05} an interpretation of the new operator is
given. It turns out that there are several possible interpretations
all enjoying the requisite algebraic properties of the operator (see
\cite{milner91polyadicpi}). We will therefore make liberal use of
$(\nu\; \vec{x})P$.

% subsection the_syntax_and_semantics_of_the_notation_system (end)   

\input{qm2pi.qmops} 

\input{qm2pi.sterngerlach} 

\input{qm2pi.metric} 

% section concurrent_process_calculi (end)

%\input{qm2pi.proofsketch}

% section proof sketch (end)

%\input{qm2pi.slviaknots} 

% section spatial logic via knots (end)

\input{qm2pi.conclusion}

% section conclusion (end)

%\input{qm2pi.dtcodes} 

% section wiring algorithm (end)

\input{qm2pi.ack} 

% section acknowledgments (end)

\newpage


\bibliographystyle{plain}   
\bibliography{../../biblios/main.bib}

\input{qm2pi.rhodetails}

\end{document}



% section front matter (end)

\section{Introduction}\label{sec:introduction} % (fold)
In this draft of the material i am going to have to dispense with the
usual writing conventions adopted in papers on these topics. i'm going
to have adopt whatever tone i need at the time i'm writing up the
calculations. Sometimes this may be very conversational; others it may
be the barest mathematical grunts; others still it may be that i have
lifted text from one of my other papers because the exposition of some
point was better said there. i hope that my readers are not unduly put
out by this decision. i'm not doing this to flout convention or be
rebellious. i find these calculations very technically challenging. To
keep everything going technically, something has to give; i have to
let go of some cognitive burden. So, the academic writing style --
with all of its trade-offs in terms of facilitating technical
communication -- is what i'm letting go of. Perhaps subsequent drafts
can be tightened and polished, but for now, i'm going to speak as if
we were sitting together in a coffee shop with a laptop, wifi and a
pad of paper and a pencil.

So, here's what i have to say. We -- you and i, comfortably ensconced
in our coffee shop and well-equipped with our tools -- can realize and
carry out the calculations of quantum mechanics over a very different
formal theory of dynamics, a formal theory of dynamics that
corresponds to a theory of concurrent computation with
\emph{reflection}. It has the advantage that the underlying theory is
already `quantized', but supports analogues all of the continuuous
operations. Strikingly, this underlying theory has recently been
connected with a notion of metric that we can show, by calculating
together, coincides with the metric induced by the inner product.

There are a lot of reasons why you might be interested in seeing
calculations of this form. Here's why i'm interested. For the past
several centuries there has been no competitor to the ``Newtonian''
account of dynamics. As a result the predominant share of accounts of
dynamical systems and situations have had to be formulated in terms of
the Newtonian machinery. i view this as an intellectually dangerous
position to occupy. Everything, despite it's intrinsic shape, turns
into a nail to be hit with this hammer. Recently, however, the theory
of computation has matured to the point where we have candidates for
theories of dynamics that offer very different perspective on
reasoning about dynamical systems and situations. Testing these
candidates against very successful accounts of dynamical situations,
like quantum mechanics, is going to give us some sense of how mature
they are and some measure of the quality of these accounts of
dynamics.

\subsection{Summary of contributions and outline of paper}

So, we're going to develop an interpretation of the operations of
quantum mechanics normally interpreted by Hilbert spaces and
operators. We're going to do this over a theory of computation. Note
that this is very different than the usual quantum computation program
which develops notions of computation over quantum mechanics. Rather,
we are developing a story that aligns with Wheeler's slogan: It from
Bit. To do this we will first provide an account of the theory of
computation at play here. Then we will dive into a calculation-driven
interpretation of the operations of quantum mechanics.

The reason we take this approach is that -- until very recently --
there hasn't been an axiomatic account of quantum mechanics. As a
result there has been no sharp delineation of the mathematical theory
supporting interpretation of the physical theory and the physical
theory, itself. So, ambient features of the maths are free to be
exploited (or supressed) without a real accounting of their physical
relevance. There is no sharp statement ``here's the physical theory''
qua \emph{theory} and ``here's the mathematical interpretation''
enabling a judgment of how faithful the interpretation is -- apart
from experimental observation. When there is an axiomatic account we
can judge how well a given mathematical formalism supports an
interpretation of the axioms, independent of
experimentation. Likewise, we can judge how well we have captured our
physical evidence and experience with our axiomatics, independent of
any specific mathematical implementation, with accidental detail that
may or may not have physical significance. 

In lieu of a fully fleshed out and vetted axiomatic account of quantum
mechanics, interpreting the operational notions in service of modeling
physical systems will have to suffice. In other words, we are not in
the business of providing a model of Hilbert spaces and operators. We
are in the business of providing a model of quantum mechanics because
we are motivated by testing our notions of dynamics against physical
theory; and, the predictive calculations of the physical theory must
serve as the best formulation -- shy of a fully fleshed out axiomatic
account -- of the physical theory itself (as they have for scientific
theories since time immemorial). Put another way, despite a
whole-hearted commitment to an It-from-Bit ontology, we are firmly
aligned with the shut-up-and-calculate camp as the best way to obtain
results either from the physical perspective or as a quality assurance
measure of our fledgling theory of dynamics.

In detail, we present a reflective process calculus. Then we develop
intuitive correspondences between the notions available in this
calculus and the usual physical notions supporting quantum mechanical
calculations. Thus, 

\begin{table}[htp]
  \center{
    \fbox{
      \begin{tabular}{c|c}
        quantum mechanics & process calculus \\
        \hline
        scalar & name \\
        state vector & process \\
        dual & contextual duals \\
        matrix & formal sums of process-context-dual pairs \\
        orthogonality & process annihilation \\
        inner product & execution-formula + quoting
      \end{tabular}
    }
  }
  \caption{QM - process calculi correspondences}
\end{table}

Then we tighten up these intuitions to operational definitions. We
employ the Dirac notation as the best proxy we can find for an
abstract syntax of the quantum mechanical notions. The definitions we
develop put us in contact with equational constraints coming from the
theory that we demonstrate the definitions and calculations satisfy.

This puts us in a position to shut up and calculate for the
Stern-Gerlach experimental set up, showing how these predictive
calculations become calculations on processes in our theory of a
reflective process calculus.

Penultimately, we demonstrate that the notion of metric coming from
the inner product coincides with the notion of metric available from
the theory of bisimulation. This demonstration gives us the right to
think of space as arising from behavior. Finally, we consider where we
might go from the new vantage point we have obtained.

% section introduction (end) 
 
% section introduction (end)

% \documentclass[12pt]{llncs}
%\documentclass{jktr}

\usepackage[pdftex]{hyperref}                   
\usepackage {listings}
\usepackage {mathpartir}
\usepackage{bcprules}
%\usepackage{listings}
                       
\usepackage{graphicx} 
%\usepackage[margins=2.5cm,nohead,nofoot]{geometry}
%\usepackage{geometry}
\usepackage{amsfonts}
\usepackage{amstext}
\usepackage{latexsym}
\usepackage{amssymb}
\usepackage{color}


%\include{myPreamble}
\include{qm2pi.local} 

%\ifpdf
%\usepackage[pdftex]{graphicx}
%\else
%\usepackage{graphicx}
%\fi

 % \ifpdf
%  \usepackage{pdfsync}
%  \if


%\title{Brief Article}
%\author{David F. Snyder}
%\author{L.G. Meredith}

%\address{Dept. of Math., Texas State University--San Marcos, San Marcos, TX 78666}
       
\pagestyle{empty}


\begin{document}

\lstset{language=[Objective]Caml,frame=shadowbox}

\input{qm2pi.front}

% section front matter (end)

\input{qm2pi.intro} 
 
% section introduction (end)

% \input{qm2pi.knotations} 

% section notation (end)

\input{qm2pi.process.calculi} 

% section concurrent_process_calculi_and_spatial_logics_ (end)
    
%\input{qm2pi.knots2pi} 

%\input{qm2pi.trefoil} 

%\input{qm2pi.mainthm} 

% subsection basic_interpretation (end)

%\input{qm2pi.rho.presentation} 
\subsection{The syntax and semantics of the notation system}\label{sub:the_syntax_and_semantics_of_the_notation_system} % (fold)

We now summarize a technical presentation of the calculus that
embodies our theory of dynamics. The typical presentation of such a
calculus follows the style of giving generators and relations on
them. The grammar, below, describing term constructors, freely
generates the set of processes, $\Proc$. This set is then quotiented
by a relation known as structural congruence and it is over this set
that the notion of dynamics is expressed. This presentation is
essentially that of \cite{MeredithR05} with the addition of
polyadicity and summation. For readability we have relegated some of
the technical subtleties to an appendix.

\subsubsection{Process grammar}\label{subsub:process_grammar}

\begin{mathpar}
  \inferrule* [lab=synchronization] {} {{M} \bc \pzero \;|\; x?F \;|\; x!C }
  \and
  \inferrule* [lab=abstraction] {} {{F} \bc (x)P}
  \and
  \inferrule* [lab=concretion] {} {{C} \bc \langle Q \rangle}
  \and
  \inferrule* [lab=process] {} {{P,Q} \bc M \;| \;P|Q \;|\; @{x}}
  \and
  \inferrule* [lab=name] {} {{x} \bc \quotep{P}}
\end{mathpar} 

Note that $\vec{x}$ (resp. $\vec{P}$) denotes a vector of names
(resp. processes) of length $|\vec{x}|$ (resp. $|\vec{P}|$). We adopt
the following useful abbreviations.

\begin{mathpar}
   x?(\vec{y}).P := x.(\vec{y})P \and  x\clift{\vec{P}} := x.\clift{\vec{P}}
   \and x!(y) := \lift{x}{\dropn{y}}
   \and \Pi_{i=0}^{n-1}P_i := P_0 | \ldots | P_{n-1}
\end{mathpar}

\subsubsection{Structural congruence}

\paragraph{Free and bound names and alpha-equivalence.} At the
core of structural equivalence is alpha-equivalence which identifies
process that are the same up to a change of variable. Formally, we
recognize the distinction between free and bound names. The free names
of a process, $\freenames{P}$, may be calculated recursively as
follows:

\begin{mathpar}
\freenames{\pzero} := \emptyset
  \and \\
  \freenames{x?(y).P} := \{ x \} \cup (\freenames{P} \setminus \{ y \})
  \and 
  \freenames{x!\langle P \rangle} := \{ x \} \cup \{ P \} 
  \and \\
  \freenames{P|Q} := \freenames{P} \cup \freenames{Q}
  \and \\
  \freenames{@{x}} := \{ x \}
\end{mathpar}

$\pi$
$\quotep{\pi}$

$\freenames{-} : \pi \to \mathcal{P}(\quotep{\pi})$

\begin{eqnarray*}
  \freenames{\pzero} & := & \emptyset \\
  \freenames{x?(y).P} & := & \{ x \} \cup (\freenames{P} \setminus \{ y \}) \\
  \freenames{x!\langle P \rangle} & := & \{ x \} \cup \{ P \} \\
  \freenames{P|Q} & := & \freenames{P} \cup \freenames{Q} \\
  \freenames{\dropn{x}} & := & \{ x \}
\end{eqnarray*}

The bound names of a process, $\boundnames{P}$, are those names occurring in $P$
that are not free. For example, in $x?(y).0$, the name $x$ is free, while $y$ is bound.

\begin{mathpar}
  \inferrule* [lab=monoidal-laws] {} { P|Q \equiv Q|P \and P|0 \equiv P \and P|(Q|R) \equiv (P|Q)|R }
\end{mathpar}

\begin{mathpar}
  \inferrule* [lab=alpha-equivalence] {} { (x)P \equiv (y)P\{y/x\} \and y \not\in \freenames{P} }
\end{mathpar}

\begin{definition}
Then two processes, $P,Q$, are alpha-equivalent if $P = Q\{\vec{y}/\vec{x}\}$ for
some $\vec{x} \in \boundnames{Q},\vec{y} \in \boundnames{P}$, where $Q\{\vec{y}/\vec{x}\}$
denotes the capture-avoiding substitution of $\vec{y}$ for $\vec{x}$ in $Q$.
\end{definition}

\begin{definition}
  The {\em structural congruence} \cite{SangiorgiWalker} , $\equiv$,
  between processes is the least congruence containing
  alpha-equivalence, satisfying the abelian monoid laws
  (associativity, commutativity and $\pzero$ as identity) for parallel
  composition $|$ and for summation $+$.
\end{definition}

\subsection{Name equivalence}

We take name equivalence, written $\nameeq$, to be the smallest
equivalence relation generated by the following rules.

\begin{mathpar}
\inferrule*[lab=Quote-drop]
{ }
{ \quotep{@{x}} \nameeq x }

\inferrule*[lab=Struct-equiv]
{ P \scong Q }
{ \quotep{P} \nameeq \quotep{Q} }
\end{mathpar}

The astute reader will have noticed that the mutual recursion of names
and processes imposes a mutual recursion on alpha-equivalence and
structural equivalence via name-equivalence. Fortunately, all of this
works out pleasantly and we may calculate in the natural way, free of
concern. The reader interested in the details is referred to the
appendix \ref{appendix:rho_details}.

\subsection{Substitution}

We use $\Proc$ for the set of processes, $\QProc$ for the set of
names, and $\id{\{}\vec{y} / \vec{x} \id{\}}$ to denote partial maps,
$s : \QProc \rightarrow \QProc$. A map, $s$ lifts, uniquely, to a map
on process terms, $\widehat{s} : \Proc \rightarrow \Proc$ by the
following equations.

\begin{mathpar}
  (0) \psubstp{Q}{P} := 0 \\
  (R \juxtap S) \psubstp{Q}{P}
  :=    
  (R)\psubstp{Q}{P} \juxtap (S) \psubstp{Q}{P} \\
  (x?(y).R) \psubstp{Q}{P}    
  :=    
  (x)\substp{Q}{P} (z)\concat( (R \psubstn{z}{y}) \psubstp{Q}{P} ) \\
  (\lift{x}{R}) \psubstp{Q}{P}  
  :=
  \lift{(x)\substp{Q}{P}}{ R \psubstp{Q}{P} } \\
%   (\dropn{x})  \psubstp{Q}{P}       
%   := 
%   \left\{ 
%     \begin{array}{ccc} 
%       \dropn{\quotep{Q}} & & x \nameeq \quotep{P} \\
%       \dropn{x} & & otherwise \\
%     \end{array}
%   \right. 
  (\dropn{x})  \psubstp{Q}{P}       
  := 
  \left\{ 
    \begin{array}{ccc} 
      Q & & x \nameeq \quotep{P} \\
      \dropn{x} & & otherwise \\
    \end{array}
  \right.
\end{mathpar}
 

where

\begin{eqnarray}
  (x)\id{\{} \lpquote Q \rpquote / \lpquote P \rpquote \id{\}}            = 
  \left\{ 
    \begin{array}{ccc}
      \lpquote Q \rpquote & & x \nameeq \lpquote P \rpquote \\
      x & & otherwise \\
    \end{array}
  \right. \nonumber
\end{eqnarray}

and $z$ is chosen distinct from $\quotep{P}$, $\quotep{Q}$, the free
names in $Q$, and all the names in $R$. Our $\alpha$-equivalence will
be built in the standard way from this substitution.

\begin{remark}\label{rem:no_self_referential_names}
  One consequence of these definitions is that $\forall P. \quotep{P}
  \not\in \freenames{P}$.
\end{remark}

\subsection{ Dynamic quote: an example }

Anticipating something of what's to come, consider applying the
substitution, $\widehat{\id{\{}u / z \id{\}}}$, to the following pair
of processes, $\lift{w}{y!(z)}$ and $w[ \lpquote y!(z) \rpquote ]$.

\begin{eqnarray}
	\lift{w}{y!(z)}\widehat{\id{\{}u / z \id{\}}}
		& = &
		\lift{w}{y!(u)} \nonumber\\
	w[ \lpquote y!(z) \rpquote ] \widehat{ \id{\{}u / z \id{\}} }
		& = &
		w[ \lpquote y!(z) \rpquote ] \nonumber
\end{eqnarray}

Because the body of the process between quotes is impervious to
substitution, we get radically different answers. In fact, by
examining the first process in an input context,
e.g. $x?(z).\lift{w}{y!(z)}$, we see that the process under the lift
operator may be shaped by prefixed inputs binding a name inside it. In
this sense, the lift operator will be seen as a way to dynamically
construct processes before reifying them as names.

Finally equipped with these standard features we can present the
dynamics of the calculus.

\subsubsection{Operational semantics} 

Finally, we introduce the computational dynamics. What marks these
algebras as distinct from other more traditionally studied algebraic
structures, e.g. vector spaces or polynomial rings, is the manner in
which dynamics is captured. In traditional structures, dynamics is typically
expressed through morphisms between such structures, as in linear maps
between vector spaces or morphisms between rings. In algebras
associated with the semantics of computation, the dynamics is
expressed as part of the algebraic structure itself, through a
reduction reduction relation typically denoted by $\red$. Below, we
give a recursive presentation of this relation for the calculus used
in the encoding.

$\red \subseteq \pi \times \pi$
$\red : \pi \to \mathcal{P}(\pi)$

\begin{mathpar}
  \inferrule* [lab=Comm] { \textsf{match}( x_{src}, x_{trgt} ) } { x_{trgt}?(y)P \; | \; x_{src}!\langle {Q} \rangle \red P\{\quotep{Q}/y}\} }
  \and \\
  \inferrule* [lab=Par] {{P} \red {P}'} {{{P} | {Q}} \red {{P}' | {Q}}}
  \and
  \inferrule* [lab=Equiv]{{{P} \scong {P}'} \andalso {{P}' \red {Q}'} \andalso {{Q}' \scong {Q}}}{{P} \red {Q}}
\end{mathpar}

\begin{eqnarray*}
  match_{\equiv} (\quotep{P},\quotep{Q}) & := & P \equiv Q \\
  match_{\dagger}(\quotep{P},\quotep{Q}) & := & \forall R. P|Q \red^{*} R => R \red^{*} 0 \\
  match_{K}(\quotep{P},\quotep{Q}) & := & K \mbox{ for some context } K
\end{eqnarray*}

$u?(x)P | u!\langle Q \rangle \red P\{\quotep{Q}/x\}$

%We write $\wred$ for $\red^*$, and $P\red$ if $\exists Q $ such that $ P \red Q$.
We write $P\red$ if $\exists Q $ such that $ P \red Q$ and $P\not\red$, otherwise.

\section{Replication}

As mentioned before, it is known that replication (and hence
recursion) can be implemented in a higher-order process algebra
\cite{SangiorgiWalker}. As our first example of calculation with the
machinery thus far presented we give the construction explicitly in
the {\rhoc}.

\begin{eqnarray}
	D_{x} & := & \prefix{x}{y}{(\binpar{\outputp{x}{y}}{@{y}})} \nonumber\\
	\bangp_{x}{P} & := & \binpar{{x}!\langle{\binpar{D_{x}}{P}}\rangle}{D_{x}} \nonumber
\end{eqnarray}

\begin{eqnarray}
	\bangp_{x}{P} & & \nonumber\\
	=
	& {x}!\langle{(\prefix{x}{y}{(\outputp{x}{y} | @{y})) | P}}\rangle 
	      | \prefix{x}{y}{(\outputp{x}{y} | @{y})} & \nonumber\\
	\red
	& (\outputp{x}{y} | @{y})\substn{\quotep{(\prefix{x}{y}{(@{y} | \outputp{x}{y})) | P}}}{y} & \nonumber\\
	=
	& \outputp{x}{\quotep{(\prefix{x}{y}{(\outputp{x}{y} | @{y})) | P}}}
	  | {(\prefix{x}{y}{(\outputp{x}{y} | @{y})) | P}} & \nonumber\\
	\red
	& \ldots & \nonumber\\
	\red^*
	& P | P | \ldots & \nonumber
\end{eqnarray}

Of course, this encoding, as an implementation, runs away, unfolding
$\bangp{P}$ eagerly. A lazier and more implementable replication
operator, restricted to input-guarded processes, may be obtained as follows.

\begin{eqnarray}
\bangp{\prefix{u}{v}{P}} 
	:= 
	\binpar{\lift{x}{\prefix{u}{v}{(\binpar{D(x)}{P})}}}{D(x)} \nonumber
\end{eqnarray}

\begin{remark}
  Note that the lazier definition still does not deal with summation
  or mixed summation (i.e. sums over input and output). The reader is
  invited to construct definitions of replication that deal with these
  features. 

  Further, the definitions are parameterized in a name, $x$. Can you,
  gentle reader, make a definition that eliminates this parameter and
  guarantees no accidental interaction between the replication
  machinery and the process being replicated -- i.e. no accidental
  sharing of names used by the process to get its work done and the
  name(s) used by the replication to effect copying. This latter
  revision of the definition of replication is crucial to obtaining
  the expected identity $!!P \sim !P$.
\end{remark}

\begin{remark}\label{rem:paradoxical_combinator}
  The reader familiar with the lambda calculus will have noticed the
  similarity between $D$ and the paradoxical combinator.

  [Ed. note: the existence of this seems to suggest we have to be more
  restrictive on the set of processes and names we admit if we are to
  support no-cloning.]
\end{remark}

\subsubsection{Bisimulation}

The computational dynamics gives rise to another kind of equivalence,
the equivalence of computational behavior. As previously mentioned
this is typically captured \emph{via} some form of bisimulation.

% The notion we use in this paper is weak barbed bisimulation
% \cite{milner91polyadicpi}.

The notion we use in this paper is derived from weak barbed
bisimulation \cite{milner91polyadicpi}. 

\begin{definition}
An \emph{observation relation}, $\downarrow_{\mathcal N}$, over a set
of names, $\mathcal N$, is the smallest relation satisfying the rules
below.

\infrule[Out-barb]{y \in {\mathcal N}, \; x \nameeq y}
		  {\outputp{x}{v} \downarrow_{\mathcal N} x}
\infrule[Par-barb]{\mbox{$P\downarrow_{\mathcal N} x$ or $Q\downarrow_{\mathcal N} x$}}
		  {\binpar{P}{Q} \downarrow_{\mathcal N} x}

We write $P \Downarrow_{\mathcal N} x$ if there is $Q$ such that 
$P \wred Q$ and $Q \downarrow_{\mathcal N} x$.
\end{definition}

\begin{definition}
%\label{def.bbisim}
An  ${\mathcal N}$-\emph{barbed bisimulation} over a set of names, ${\mathcal N}$, is a symmetric binary relation 
${\mathcal S}_{\mathcal N}$ between agents such that $P\rel{S}_{\mathcal N}Q$ implies:
\begin{enumerate}
\item If $P \red P'$ then $Q \wred Q'$ and $P'\rel{S}_{\mathcal N} Q'$.
\item If $P\downarrow_{\mathcal N} x$, then $Q\Downarrow_{\mathcal N} x$.
\end{enumerate}
$P$ is ${\mathcal N}$-barbed bisimilar to $Q$, written
$P \wbbisim_{\mathcal N} Q$, if $P \rel{S}_{\mathcal N} Q$ for some ${\mathcal N}$-barbed bisimulation ${\mathcal S}_{\mathcal N}$.
\end{definition}

$\mathcal{R} \subseteq \pi \times \pi$

$P \mathcal{R} Q => \forall P'. P \red P' \Rightarrow \exists Q'. Q \red Q', P' \mathcal{R} Q'$

$P \vdash x \Rightarrow Q \vdash x$

\begin{mathpar}
  \inferrule*[lab=Out-barb]{x \nameeq y}{{y}!\langle{Q}\rangle \vdash x}
  \and
  \inferrule*[lab=Par-barb]{\mbox{$P\vdash x$ or $Q\vdash x$}}{\binpar{P}{Q} \vdash x}
\end{mathpar}

\subsubsection{Contexts}

One of the principle advantages of computational calculi like the
$\pi$-calculus is a well-defined notion of context,
contextual-equivalence and a correlation between
contextual-equivalence and notions of bisimulation. The notion of
context allows the decomposition of a process into (sub-)process and
its syntactic environment, its context. Thus, a context may be
thought of as a process with a ``hole'' (written $\Box$) in it. The
application of a context $M$ to a process $P$, written $M[P]$, is
tantamount to filling the hole in $M$ with $P$. In this paper we do
not need the full weight of this theory, but do make use of the notion
of context in the proof the main theorem. 

\begin{mathpar}
  \inferrule* [lab=summation] {} {{M_{M},M_{N}} \bc \Box \;|\; x.M_{A} \;|\; M_{M}+M_{N}}
  \and
  \inferrule* [lab=agent] {} {{M_{A}} \bc (\vec{x})M_{P} \;| \; \clift{P_0,\ldots,M_{P},\ldots,P_N}}
  \and \\
  \inferrule* [lab=process] {} {{M_{P}} \bc M_{N} \;| \;P|M_{P} }
\end{mathpar} 

\begin{mathpar}
  \inferrule* [lab=sychronization] {} {M_{N} \bc \Box \;|\; x?M_{F} \;|\; x!M_{C}}
  \and
  \inferrule* [lab=abstraction] {} {{M_{F}} \bc (x)M_{P} }
  \and
  \inferrule* [lab=concretion] {} {{M_{C}} \bc \langle M_{P} \rangle }
  \and \\
  \inferrule* [lab=process] {} {{M_{P}} \bc M_{N} \;| \;P|M_{P} }
\end{mathpar}

\begin{definition}[contextual application] Given a context $M$, and
  process $P$, we define the \emph{contextual application}, $M[P] :=
  M\{P/\Box\}$. That is, the contextual application of M to P is the
  substitution of $P$ for $\Box$ in $M$.
\end{definition}

$\meaningof{-} : L \to \mathcal{P}(\pi)$

\begin{mathpar}
  \inferrule* [lab=collection] {} {\meaningof{true} = \pi, \and \meaningof{~E} = \pi \setminus \meaningof{E}, \and \meaningof{E_{1} \& E_{2}} = \meaningof{E_{1}} \cap \meaningof{E_{2}}}
\end{mathpar}

\begin{mathpar}
  \inferrule* [lab=structure] {} {\meaningof{0} = \{ P \in \pi | P \equiv 0 \}, \and \\ \meaningof{E_1 | E_2} = \{ P \in \pi | P \equiv P_{1} | P_{2}, P_{1} \in \meaningof{E_{1}}, P_{2} \in \meaningof{E_2}\} }
\end{mathpar}

\begin{mathpar}
 \inferrule* [lab=behavior] {} {\meaningof{\langle a?b \rangle E} = \{ P \in \pi | P \equiv Q | u?(y)P', \\ \and \\\\ \and \\ \;\;\; u \in \meaningof{a}, \forall z.P'\{z/y\} \in \meaningof{E\{z/b\}}\}, \and \\ \meaningof{a!E} = \{ P \in \pi | P \equiv Q | x!\langle P' \rangle, x \in \meaningof{a} P' \in \meaningof{E}\} }
\end{mathpar}

\begin{mathpar}
 \inferrule* [lab=nominal] {} {\meaningof{\quotep{E}} = \{ \quotep{P} \in \quotep{\pi} | P \in \meaningof{E} \}, \and \meaningof{\quotep{P}} = \{ \quotep{Q} \in \quotep{\pi} | P \equiv Q \} \and \\ \meaningof{@\quotep{E}} = \{ P \in \pi | P \equiv @x, x \in \meaningof{E} \}}
\end{mathpar}

\begin{eqnarray*}
  \\
  \meaningof{-} : TS \to ST
\end{eqnarray*}

\begin{eqnarray*}
  \\
  L : TS \to ST
\end{eqnarray*}

\begin{eqnarray*}
  \\
  P \models E \iff P \in \meaningof{E}
\end{eqnarray*}

\begin{eqnarray*}
  P \approx_{L} Q \iff \forall E \in L. P \models E \iff Q \models E
\end{eqnarray*}

\begin{eqnarray*}
  P \approx_{K} Q
\end{eqnarray*}

\begin{eqnarray*}
  P \approx Q
\end{eqnarray*}

$\approx_{K} = \approx = \approx_{L}$

\subsubsection{Contextual duality}

Note that contexts extend the quotation operation to a family of
operations from processes to names. Given a context, $M$, we can
define a \emph{nominal context}, $\quotep{M}$ by $\quotep{M}[P] :=
\quotep{M[P]}$. To foreshadow what is to come we observe that these
operations enjoy a duality with processes very much like the duality
between vectors and maps from vectors to scalars.

Further, because the calculus is essentially higher-order, we have a
correspondence between contexts and processes. More specifically,
given a name $x$ and a context $M$ we can construct $M^{*}_{x}$ such
that 

\begin{mathpar}
  M^{*}_{x} | \lift{x}{P} \red M[P]
\end{mathpar}

namely,

\begin{mathpar}
  M^{*}_{x} := x?(u).M[\dropn{u}]
\end{mathpar}

The dependence of $M^{*}_{x}$ on a name makes it an abstraction, 

\begin{mathpar}
  M^{*} := (x)x?(u).M[\dropn{u}]
\end{mathpar}

\subsection{Additional notation}

It will sometimes be convenient to denote the process a name
quotes. We already have the notation $x = \quotep{P}$, but it will be
convenient to introduce an alternate notation, $\procn{x}$, when we
want to emphasize the connection to the use of the name. Note that, by
virtue of name equivalence, $\quotep{\procn{x}} \nameeq x$; so, the
notation is consistent with previous definitions.

Further, because names have structure it is possible to effect
substitutions on the basis of that structure. This means we need to
upgrade our notation for substitutions, which we accomplish by
adapting comprehension notation. Thus,

\begin{mathpar}
  P\{ y / x : x \in S \}
\end{mathpar}

is interpreted to mean the process derived from P by replacing (in a
capture-avoiding manner) each occurrence of $x$ in $S$ by $y$. For example,

\begin{mathpar}
  P\{ \quotep{\procn{x}|\procn{x}} / x : x \in \freenames{P} \}
\end{mathpar}

will replace each (occurrence) of a free name $x$ in $P$ by
$\quotep{\procn{x}|\procn{x}}$.

Also, we will avail ourselves of the notation $x^{L}$ and $x^{R}$ to
denote injections of a name into disjoint copies of the name
space. There are numerous ways to accomplish this. One example can be
found in \cite{MeredithR05}. This notation overloads to vectors of
names: $\vec{x}^{\pi} := (x_{i}^{\pi} \; : \; 0 \leq i < |\vec{x}| )$ where $\pi \in \{L,R\}$.

We also use $P^{\Box} := P|\Box$.

In \cite{MeredithR05} an interpretation of the new operator is
given. It turns out that there are several possible interpretations
all enjoying the requisite algebraic properties of the operator (see
\cite{milner91polyadicpi}). We will therefore make liberal use of
$(\nu\; \vec{x})P$.

% subsection the_syntax_and_semantics_of_the_notation_system (end)   

\input{qm2pi.qmops} 

\input{qm2pi.sterngerlach} 

\input{qm2pi.metric} 

% section concurrent_process_calculi (end)

%\input{qm2pi.proofsketch}

% section proof sketch (end)

%\input{qm2pi.slviaknots} 

% section spatial logic via knots (end)

\input{qm2pi.conclusion}

% section conclusion (end)

%\input{qm2pi.dtcodes} 

% section wiring algorithm (end)

\input{qm2pi.ack} 

% section acknowledgments (end)

\newpage


\bibliographystyle{plain}   
\bibliography{../../biblios/main.bib}

\input{qm2pi.rhodetails}

\end{document}

 

% section notation (end)

\input{qm2pi.process.calculi} 

% section concurrent_process_calculi_and_spatial_logics_ (end)
    
%\documentclass[12pt]{llncs}
%\documentclass{jktr}

\usepackage[pdftex]{hyperref}                   
\usepackage {listings}
\usepackage {mathpartir}
\usepackage{bcprules}
%\usepackage{listings}
                       
\usepackage{graphicx} 
%\usepackage[margins=2.5cm,nohead,nofoot]{geometry}
%\usepackage{geometry}
\usepackage{amsfonts}
\usepackage{amstext}
\usepackage{latexsym}
\usepackage{amssymb}
\usepackage{color}


%\include{myPreamble}
\include{qm2pi.local} 

%\ifpdf
%\usepackage[pdftex]{graphicx}
%\else
%\usepackage{graphicx}
%\fi

 % \ifpdf
%  \usepackage{pdfsync}
%  \if


%\title{Brief Article}
%\author{David F. Snyder}
%\author{L.G. Meredith}

%\address{Dept. of Math., Texas State University--San Marcos, San Marcos, TX 78666}
       
\pagestyle{empty}


\begin{document}

\lstset{language=[Objective]Caml,frame=shadowbox}

\input{qm2pi.front}

% section front matter (end)

\input{qm2pi.intro} 
 
% section introduction (end)

% \input{qm2pi.knotations} 

% section notation (end)

\input{qm2pi.process.calculi} 

% section concurrent_process_calculi_and_spatial_logics_ (end)
    
%\input{qm2pi.knots2pi} 

%\input{qm2pi.trefoil} 

%\input{qm2pi.mainthm} 

% subsection basic_interpretation (end)

%\input{qm2pi.rho.presentation} 
\subsection{The syntax and semantics of the notation system}\label{sub:the_syntax_and_semantics_of_the_notation_system} % (fold)

We now summarize a technical presentation of the calculus that
embodies our theory of dynamics. The typical presentation of such a
calculus follows the style of giving generators and relations on
them. The grammar, below, describing term constructors, freely
generates the set of processes, $\Proc$. This set is then quotiented
by a relation known as structural congruence and it is over this set
that the notion of dynamics is expressed. This presentation is
essentially that of \cite{MeredithR05} with the addition of
polyadicity and summation. For readability we have relegated some of
the technical subtleties to an appendix.

\subsubsection{Process grammar}\label{subsub:process_grammar}

\begin{mathpar}
  \inferrule* [lab=synchronization] {} {{M} \bc \pzero \;|\; x?F \;|\; x!C }
  \and
  \inferrule* [lab=abstraction] {} {{F} \bc (x)P}
  \and
  \inferrule* [lab=concretion] {} {{C} \bc \langle Q \rangle}
  \and
  \inferrule* [lab=process] {} {{P,Q} \bc M \;| \;P|Q \;|\; @{x}}
  \and
  \inferrule* [lab=name] {} {{x} \bc \quotep{P}}
\end{mathpar} 

Note that $\vec{x}$ (resp. $\vec{P}$) denotes a vector of names
(resp. processes) of length $|\vec{x}|$ (resp. $|\vec{P}|$). We adopt
the following useful abbreviations.

\begin{mathpar}
   x?(\vec{y}).P := x.(\vec{y})P \and  x\clift{\vec{P}} := x.\clift{\vec{P}}
   \and x!(y) := \lift{x}{\dropn{y}}
   \and \Pi_{i=0}^{n-1}P_i := P_0 | \ldots | P_{n-1}
\end{mathpar}

\subsubsection{Structural congruence}

\paragraph{Free and bound names and alpha-equivalence.} At the
core of structural equivalence is alpha-equivalence which identifies
process that are the same up to a change of variable. Formally, we
recognize the distinction between free and bound names. The free names
of a process, $\freenames{P}$, may be calculated recursively as
follows:

\begin{mathpar}
\freenames{\pzero} := \emptyset
  \and \\
  \freenames{x?(y).P} := \{ x \} \cup (\freenames{P} \setminus \{ y \})
  \and 
  \freenames{x!\langle P \rangle} := \{ x \} \cup \{ P \} 
  \and \\
  \freenames{P|Q} := \freenames{P} \cup \freenames{Q}
  \and \\
  \freenames{@{x}} := \{ x \}
\end{mathpar}

$\pi$
$\quotep{\pi}$

$\freenames{-} : \pi \to \mathcal{P}(\quotep{\pi})$

\begin{eqnarray*}
  \freenames{\pzero} & := & \emptyset \\
  \freenames{x?(y).P} & := & \{ x \} \cup (\freenames{P} \setminus \{ y \}) \\
  \freenames{x!\langle P \rangle} & := & \{ x \} \cup \{ P \} \\
  \freenames{P|Q} & := & \freenames{P} \cup \freenames{Q} \\
  \freenames{\dropn{x}} & := & \{ x \}
\end{eqnarray*}

The bound names of a process, $\boundnames{P}$, are those names occurring in $P$
that are not free. For example, in $x?(y).0$, the name $x$ is free, while $y$ is bound.

\begin{mathpar}
  \inferrule* [lab=monoidal-laws] {} { P|Q \equiv Q|P \and P|0 \equiv P \and P|(Q|R) \equiv (P|Q)|R }
\end{mathpar}

\begin{mathpar}
  \inferrule* [lab=alpha-equivalence] {} { (x)P \equiv (y)P\{y/x\} \and y \not\in \freenames{P} }
\end{mathpar}

\begin{definition}
Then two processes, $P,Q$, are alpha-equivalent if $P = Q\{\vec{y}/\vec{x}\}$ for
some $\vec{x} \in \boundnames{Q},\vec{y} \in \boundnames{P}$, where $Q\{\vec{y}/\vec{x}\}$
denotes the capture-avoiding substitution of $\vec{y}$ for $\vec{x}$ in $Q$.
\end{definition}

\begin{definition}
  The {\em structural congruence} \cite{SangiorgiWalker} , $\equiv$,
  between processes is the least congruence containing
  alpha-equivalence, satisfying the abelian monoid laws
  (associativity, commutativity and $\pzero$ as identity) for parallel
  composition $|$ and for summation $+$.
\end{definition}

\subsection{Name equivalence}

We take name equivalence, written $\nameeq$, to be the smallest
equivalence relation generated by the following rules.

\begin{mathpar}
\inferrule*[lab=Quote-drop]
{ }
{ \quotep{@{x}} \nameeq x }

\inferrule*[lab=Struct-equiv]
{ P \scong Q }
{ \quotep{P} \nameeq \quotep{Q} }
\end{mathpar}

The astute reader will have noticed that the mutual recursion of names
and processes imposes a mutual recursion on alpha-equivalence and
structural equivalence via name-equivalence. Fortunately, all of this
works out pleasantly and we may calculate in the natural way, free of
concern. The reader interested in the details is referred to the
appendix \ref{appendix:rho_details}.

\subsection{Substitution}

We use $\Proc$ for the set of processes, $\QProc$ for the set of
names, and $\id{\{}\vec{y} / \vec{x} \id{\}}$ to denote partial maps,
$s : \QProc \rightarrow \QProc$. A map, $s$ lifts, uniquely, to a map
on process terms, $\widehat{s} : \Proc \rightarrow \Proc$ by the
following equations.

\begin{mathpar}
  (0) \psubstp{Q}{P} := 0 \\
  (R \juxtap S) \psubstp{Q}{P}
  :=    
  (R)\psubstp{Q}{P} \juxtap (S) \psubstp{Q}{P} \\
  (x?(y).R) \psubstp{Q}{P}    
  :=    
  (x)\substp{Q}{P} (z)\concat( (R \psubstn{z}{y}) \psubstp{Q}{P} ) \\
  (\lift{x}{R}) \psubstp{Q}{P}  
  :=
  \lift{(x)\substp{Q}{P}}{ R \psubstp{Q}{P} } \\
%   (\dropn{x})  \psubstp{Q}{P}       
%   := 
%   \left\{ 
%     \begin{array}{ccc} 
%       \dropn{\quotep{Q}} & & x \nameeq \quotep{P} \\
%       \dropn{x} & & otherwise \\
%     \end{array}
%   \right. 
  (\dropn{x})  \psubstp{Q}{P}       
  := 
  \left\{ 
    \begin{array}{ccc} 
      Q & & x \nameeq \quotep{P} \\
      \dropn{x} & & otherwise \\
    \end{array}
  \right.
\end{mathpar}
 

where

\begin{eqnarray}
  (x)\id{\{} \lpquote Q \rpquote / \lpquote P \rpquote \id{\}}            = 
  \left\{ 
    \begin{array}{ccc}
      \lpquote Q \rpquote & & x \nameeq \lpquote P \rpquote \\
      x & & otherwise \\
    \end{array}
  \right. \nonumber
\end{eqnarray}

and $z$ is chosen distinct from $\quotep{P}$, $\quotep{Q}$, the free
names in $Q$, and all the names in $R$. Our $\alpha$-equivalence will
be built in the standard way from this substitution.

\begin{remark}\label{rem:no_self_referential_names}
  One consequence of these definitions is that $\forall P. \quotep{P}
  \not\in \freenames{P}$.
\end{remark}

\subsection{ Dynamic quote: an example }

Anticipating something of what's to come, consider applying the
substitution, $\widehat{\id{\{}u / z \id{\}}}$, to the following pair
of processes, $\lift{w}{y!(z)}$ and $w[ \lpquote y!(z) \rpquote ]$.

\begin{eqnarray}
	\lift{w}{y!(z)}\widehat{\id{\{}u / z \id{\}}}
		& = &
		\lift{w}{y!(u)} \nonumber\\
	w[ \lpquote y!(z) \rpquote ] \widehat{ \id{\{}u / z \id{\}} }
		& = &
		w[ \lpquote y!(z) \rpquote ] \nonumber
\end{eqnarray}

Because the body of the process between quotes is impervious to
substitution, we get radically different answers. In fact, by
examining the first process in an input context,
e.g. $x?(z).\lift{w}{y!(z)}$, we see that the process under the lift
operator may be shaped by prefixed inputs binding a name inside it. In
this sense, the lift operator will be seen as a way to dynamically
construct processes before reifying them as names.

Finally equipped with these standard features we can present the
dynamics of the calculus.

\subsubsection{Operational semantics} 

Finally, we introduce the computational dynamics. What marks these
algebras as distinct from other more traditionally studied algebraic
structures, e.g. vector spaces or polynomial rings, is the manner in
which dynamics is captured. In traditional structures, dynamics is typically
expressed through morphisms between such structures, as in linear maps
between vector spaces or morphisms between rings. In algebras
associated with the semantics of computation, the dynamics is
expressed as part of the algebraic structure itself, through a
reduction reduction relation typically denoted by $\red$. Below, we
give a recursive presentation of this relation for the calculus used
in the encoding.

$\red \subseteq \pi \times \pi$
$\red : \pi \to \mathcal{P}(\pi)$

\begin{mathpar}
  \inferrule* [lab=Comm] { \textsf{match}( x_{src}, x_{trgt} ) } { x_{trgt}?(y)P \; | \; x_{src}!\langle {Q} \rangle \red P\{\quotep{Q}/y}\} }
  \and \\
  \inferrule* [lab=Par] {{P} \red {P}'} {{{P} | {Q}} \red {{P}' | {Q}}}
  \and
  \inferrule* [lab=Equiv]{{{P} \scong {P}'} \andalso {{P}' \red {Q}'} \andalso {{Q}' \scong {Q}}}{{P} \red {Q}}
\end{mathpar}

\begin{eqnarray*}
  match_{\equiv} (\quotep{P},\quotep{Q}) & := & P \equiv Q \\
  match_{\dagger}(\quotep{P},\quotep{Q}) & := & \forall R. P|Q \red^{*} R => R \red^{*} 0 \\
  match_{K}(\quotep{P},\quotep{Q}) & := & K \mbox{ for some context } K
\end{eqnarray*}

$u?(x)P | u!\langle Q \rangle \red P\{\quotep{Q}/x\}$

%We write $\wred$ for $\red^*$, and $P\red$ if $\exists Q $ such that $ P \red Q$.
We write $P\red$ if $\exists Q $ such that $ P \red Q$ and $P\not\red$, otherwise.

\section{Replication}

As mentioned before, it is known that replication (and hence
recursion) can be implemented in a higher-order process algebra
\cite{SangiorgiWalker}. As our first example of calculation with the
machinery thus far presented we give the construction explicitly in
the {\rhoc}.

\begin{eqnarray}
	D_{x} & := & \prefix{x}{y}{(\binpar{\outputp{x}{y}}{@{y}})} \nonumber\\
	\bangp_{x}{P} & := & \binpar{{x}!\langle{\binpar{D_{x}}{P}}\rangle}{D_{x}} \nonumber
\end{eqnarray}

\begin{eqnarray}
	\bangp_{x}{P} & & \nonumber\\
	=
	& {x}!\langle{(\prefix{x}{y}{(\outputp{x}{y} | @{y})) | P}}\rangle 
	      | \prefix{x}{y}{(\outputp{x}{y} | @{y})} & \nonumber\\
	\red
	& (\outputp{x}{y} | @{y})\substn{\quotep{(\prefix{x}{y}{(@{y} | \outputp{x}{y})) | P}}}{y} & \nonumber\\
	=
	& \outputp{x}{\quotep{(\prefix{x}{y}{(\outputp{x}{y} | @{y})) | P}}}
	  | {(\prefix{x}{y}{(\outputp{x}{y} | @{y})) | P}} & \nonumber\\
	\red
	& \ldots & \nonumber\\
	\red^*
	& P | P | \ldots & \nonumber
\end{eqnarray}

Of course, this encoding, as an implementation, runs away, unfolding
$\bangp{P}$ eagerly. A lazier and more implementable replication
operator, restricted to input-guarded processes, may be obtained as follows.

\begin{eqnarray}
\bangp{\prefix{u}{v}{P}} 
	:= 
	\binpar{\lift{x}{\prefix{u}{v}{(\binpar{D(x)}{P})}}}{D(x)} \nonumber
\end{eqnarray}

\begin{remark}
  Note that the lazier definition still does not deal with summation
  or mixed summation (i.e. sums over input and output). The reader is
  invited to construct definitions of replication that deal with these
  features. 

  Further, the definitions are parameterized in a name, $x$. Can you,
  gentle reader, make a definition that eliminates this parameter and
  guarantees no accidental interaction between the replication
  machinery and the process being replicated -- i.e. no accidental
  sharing of names used by the process to get its work done and the
  name(s) used by the replication to effect copying. This latter
  revision of the definition of replication is crucial to obtaining
  the expected identity $!!P \sim !P$.
\end{remark}

\begin{remark}\label{rem:paradoxical_combinator}
  The reader familiar with the lambda calculus will have noticed the
  similarity between $D$ and the paradoxical combinator.

  [Ed. note: the existence of this seems to suggest we have to be more
  restrictive on the set of processes and names we admit if we are to
  support no-cloning.]
\end{remark}

\subsubsection{Bisimulation}

The computational dynamics gives rise to another kind of equivalence,
the equivalence of computational behavior. As previously mentioned
this is typically captured \emph{via} some form of bisimulation.

% The notion we use in this paper is weak barbed bisimulation
% \cite{milner91polyadicpi}.

The notion we use in this paper is derived from weak barbed
bisimulation \cite{milner91polyadicpi}. 

\begin{definition}
An \emph{observation relation}, $\downarrow_{\mathcal N}$, over a set
of names, $\mathcal N$, is the smallest relation satisfying the rules
below.

\infrule[Out-barb]{y \in {\mathcal N}, \; x \nameeq y}
		  {\outputp{x}{v} \downarrow_{\mathcal N} x}
\infrule[Par-barb]{\mbox{$P\downarrow_{\mathcal N} x$ or $Q\downarrow_{\mathcal N} x$}}
		  {\binpar{P}{Q} \downarrow_{\mathcal N} x}

We write $P \Downarrow_{\mathcal N} x$ if there is $Q$ such that 
$P \wred Q$ and $Q \downarrow_{\mathcal N} x$.
\end{definition}

\begin{definition}
%\label{def.bbisim}
An  ${\mathcal N}$-\emph{barbed bisimulation} over a set of names, ${\mathcal N}$, is a symmetric binary relation 
${\mathcal S}_{\mathcal N}$ between agents such that $P\rel{S}_{\mathcal N}Q$ implies:
\begin{enumerate}
\item If $P \red P'$ then $Q \wred Q'$ and $P'\rel{S}_{\mathcal N} Q'$.
\item If $P\downarrow_{\mathcal N} x$, then $Q\Downarrow_{\mathcal N} x$.
\end{enumerate}
$P$ is ${\mathcal N}$-barbed bisimilar to $Q$, written
$P \wbbisim_{\mathcal N} Q$, if $P \rel{S}_{\mathcal N} Q$ for some ${\mathcal N}$-barbed bisimulation ${\mathcal S}_{\mathcal N}$.
\end{definition}

$\mathcal{R} \subseteq \pi \times \pi$

$P \mathcal{R} Q => \forall P'. P \red P' \Rightarrow \exists Q'. Q \red Q', P' \mathcal{R} Q'$

$P \vdash x \Rightarrow Q \vdash x$

\begin{mathpar}
  \inferrule*[lab=Out-barb]{x \nameeq y}{{y}!\langle{Q}\rangle \vdash x}
  \and
  \inferrule*[lab=Par-barb]{\mbox{$P\vdash x$ or $Q\vdash x$}}{\binpar{P}{Q} \vdash x}
\end{mathpar}

\subsubsection{Contexts}

One of the principle advantages of computational calculi like the
$\pi$-calculus is a well-defined notion of context,
contextual-equivalence and a correlation between
contextual-equivalence and notions of bisimulation. The notion of
context allows the decomposition of a process into (sub-)process and
its syntactic environment, its context. Thus, a context may be
thought of as a process with a ``hole'' (written $\Box$) in it. The
application of a context $M$ to a process $P$, written $M[P]$, is
tantamount to filling the hole in $M$ with $P$. In this paper we do
not need the full weight of this theory, but do make use of the notion
of context in the proof the main theorem. 

\begin{mathpar}
  \inferrule* [lab=summation] {} {{M_{M},M_{N}} \bc \Box \;|\; x.M_{A} \;|\; M_{M}+M_{N}}
  \and
  \inferrule* [lab=agent] {} {{M_{A}} \bc (\vec{x})M_{P} \;| \; \clift{P_0,\ldots,M_{P},\ldots,P_N}}
  \and \\
  \inferrule* [lab=process] {} {{M_{P}} \bc M_{N} \;| \;P|M_{P} }
\end{mathpar} 

\begin{mathpar}
  \inferrule* [lab=sychronization] {} {M_{N} \bc \Box \;|\; x?M_{F} \;|\; x!M_{C}}
  \and
  \inferrule* [lab=abstraction] {} {{M_{F}} \bc (x)M_{P} }
  \and
  \inferrule* [lab=concretion] {} {{M_{C}} \bc \langle M_{P} \rangle }
  \and \\
  \inferrule* [lab=process] {} {{M_{P}} \bc M_{N} \;| \;P|M_{P} }
\end{mathpar}

\begin{definition}[contextual application] Given a context $M$, and
  process $P$, we define the \emph{contextual application}, $M[P] :=
  M\{P/\Box\}$. That is, the contextual application of M to P is the
  substitution of $P$ for $\Box$ in $M$.
\end{definition}

$\meaningof{-} : L \to \mathcal{P}(\pi)$

\begin{mathpar}
  \inferrule* [lab=collection] {} {\meaningof{true} = \pi, \and \meaningof{~E} = \pi \setminus \meaningof{E}, \and \meaningof{E_{1} \& E_{2}} = \meaningof{E_{1}} \cap \meaningof{E_{2}}}
\end{mathpar}

\begin{mathpar}
  \inferrule* [lab=structure] {} {\meaningof{0} = \{ P \in \pi | P \equiv 0 \}, \and \\ \meaningof{E_1 | E_2} = \{ P \in \pi | P \equiv P_{1} | P_{2}, P_{1} \in \meaningof{E_{1}}, P_{2} \in \meaningof{E_2}\} }
\end{mathpar}

\begin{mathpar}
 \inferrule* [lab=behavior] {} {\meaningof{\langle a?b \rangle E} = \{ P \in \pi | P \equiv Q | u?(y)P', \\ \and \\\\ \and \\ \;\;\; u \in \meaningof{a}, \forall z.P'\{z/y\} \in \meaningof{E\{z/b\}}\}, \and \\ \meaningof{a!E} = \{ P \in \pi | P \equiv Q | x!\langle P' \rangle, x \in \meaningof{a} P' \in \meaningof{E}\} }
\end{mathpar}

\begin{mathpar}
 \inferrule* [lab=nominal] {} {\meaningof{\quotep{E}} = \{ \quotep{P} \in \quotep{\pi} | P \in \meaningof{E} \}, \and \meaningof{\quotep{P}} = \{ \quotep{Q} \in \quotep{\pi} | P \equiv Q \} \and \\ \meaningof{@\quotep{E}} = \{ P \in \pi | P \equiv @x, x \in \meaningof{E} \}}
\end{mathpar}

\begin{eqnarray*}
  \\
  \meaningof{-} : TS \to ST
\end{eqnarray*}

\begin{eqnarray*}
  \\
  L : TS \to ST
\end{eqnarray*}

\begin{eqnarray*}
  \\
  P \models E \iff P \in \meaningof{E}
\end{eqnarray*}

\begin{eqnarray*}
  P \approx_{L} Q \iff \forall E \in L. P \models E \iff Q \models E
\end{eqnarray*}

\begin{eqnarray*}
  P \approx_{K} Q
\end{eqnarray*}

\begin{eqnarray*}
  P \approx Q
\end{eqnarray*}

$\approx_{K} = \approx = \approx_{L}$

\subsubsection{Contextual duality}

Note that contexts extend the quotation operation to a family of
operations from processes to names. Given a context, $M$, we can
define a \emph{nominal context}, $\quotep{M}$ by $\quotep{M}[P] :=
\quotep{M[P]}$. To foreshadow what is to come we observe that these
operations enjoy a duality with processes very much like the duality
between vectors and maps from vectors to scalars.

Further, because the calculus is essentially higher-order, we have a
correspondence between contexts and processes. More specifically,
given a name $x$ and a context $M$ we can construct $M^{*}_{x}$ such
that 

\begin{mathpar}
  M^{*}_{x} | \lift{x}{P} \red M[P]
\end{mathpar}

namely,

\begin{mathpar}
  M^{*}_{x} := x?(u).M[\dropn{u}]
\end{mathpar}

The dependence of $M^{*}_{x}$ on a name makes it an abstraction, 

\begin{mathpar}
  M^{*} := (x)x?(u).M[\dropn{u}]
\end{mathpar}

\subsection{Additional notation}

It will sometimes be convenient to denote the process a name
quotes. We already have the notation $x = \quotep{P}$, but it will be
convenient to introduce an alternate notation, $\procn{x}$, when we
want to emphasize the connection to the use of the name. Note that, by
virtue of name equivalence, $\quotep{\procn{x}} \nameeq x$; so, the
notation is consistent with previous definitions.

Further, because names have structure it is possible to effect
substitutions on the basis of that structure. This means we need to
upgrade our notation for substitutions, which we accomplish by
adapting comprehension notation. Thus,

\begin{mathpar}
  P\{ y / x : x \in S \}
\end{mathpar}

is interpreted to mean the process derived from P by replacing (in a
capture-avoiding manner) each occurrence of $x$ in $S$ by $y$. For example,

\begin{mathpar}
  P\{ \quotep{\procn{x}|\procn{x}} / x : x \in \freenames{P} \}
\end{mathpar}

will replace each (occurrence) of a free name $x$ in $P$ by
$\quotep{\procn{x}|\procn{x}}$.

Also, we will avail ourselves of the notation $x^{L}$ and $x^{R}$ to
denote injections of a name into disjoint copies of the name
space. There are numerous ways to accomplish this. One example can be
found in \cite{MeredithR05}. This notation overloads to vectors of
names: $\vec{x}^{\pi} := (x_{i}^{\pi} \; : \; 0 \leq i < |\vec{x}| )$ where $\pi \in \{L,R\}$.

We also use $P^{\Box} := P|\Box$.

In \cite{MeredithR05} an interpretation of the new operator is
given. It turns out that there are several possible interpretations
all enjoying the requisite algebraic properties of the operator (see
\cite{milner91polyadicpi}). We will therefore make liberal use of
$(\nu\; \vec{x})P$.

% subsection the_syntax_and_semantics_of_the_notation_system (end)   

\input{qm2pi.qmops} 

\input{qm2pi.sterngerlach} 

\input{qm2pi.metric} 

% section concurrent_process_calculi (end)

%\input{qm2pi.proofsketch}

% section proof sketch (end)

%\input{qm2pi.slviaknots} 

% section spatial logic via knots (end)

\input{qm2pi.conclusion}

% section conclusion (end)

%\input{qm2pi.dtcodes} 

% section wiring algorithm (end)

\input{qm2pi.ack} 

% section acknowledgments (end)

\newpage


\bibliographystyle{plain}   
\bibliography{../../biblios/main.bib}

\input{qm2pi.rhodetails}

\end{document}

 

%\documentclass[12pt]{llncs}
%\documentclass{jktr}

\usepackage[pdftex]{hyperref}                   
\usepackage {listings}
\usepackage {mathpartir}
\usepackage{bcprules}
%\usepackage{listings}
                       
\usepackage{graphicx} 
%\usepackage[margins=2.5cm,nohead,nofoot]{geometry}
%\usepackage{geometry}
\usepackage{amsfonts}
\usepackage{amstext}
\usepackage{latexsym}
\usepackage{amssymb}
\usepackage{color}


%\include{myPreamble}
\include{qm2pi.local} 

%\ifpdf
%\usepackage[pdftex]{graphicx}
%\else
%\usepackage{graphicx}
%\fi

 % \ifpdf
%  \usepackage{pdfsync}
%  \if


%\title{Brief Article}
%\author{David F. Snyder}
%\author{L.G. Meredith}

%\address{Dept. of Math., Texas State University--San Marcos, San Marcos, TX 78666}
       
\pagestyle{empty}


\begin{document}

\lstset{language=[Objective]Caml,frame=shadowbox}

\input{qm2pi.front}

% section front matter (end)

\input{qm2pi.intro} 
 
% section introduction (end)

% \input{qm2pi.knotations} 

% section notation (end)

\input{qm2pi.process.calculi} 

% section concurrent_process_calculi_and_spatial_logics_ (end)
    
%\input{qm2pi.knots2pi} 

%\input{qm2pi.trefoil} 

%\input{qm2pi.mainthm} 

% subsection basic_interpretation (end)

%\input{qm2pi.rho.presentation} 
\subsection{The syntax and semantics of the notation system}\label{sub:the_syntax_and_semantics_of_the_notation_system} % (fold)

We now summarize a technical presentation of the calculus that
embodies our theory of dynamics. The typical presentation of such a
calculus follows the style of giving generators and relations on
them. The grammar, below, describing term constructors, freely
generates the set of processes, $\Proc$. This set is then quotiented
by a relation known as structural congruence and it is over this set
that the notion of dynamics is expressed. This presentation is
essentially that of \cite{MeredithR05} with the addition of
polyadicity and summation. For readability we have relegated some of
the technical subtleties to an appendix.

\subsubsection{Process grammar}\label{subsub:process_grammar}

\begin{mathpar}
  \inferrule* [lab=synchronization] {} {{M} \bc \pzero \;|\; x?F \;|\; x!C }
  \and
  \inferrule* [lab=abstraction] {} {{F} \bc (x)P}
  \and
  \inferrule* [lab=concretion] {} {{C} \bc \langle Q \rangle}
  \and
  \inferrule* [lab=process] {} {{P,Q} \bc M \;| \;P|Q \;|\; @{x}}
  \and
  \inferrule* [lab=name] {} {{x} \bc \quotep{P}}
\end{mathpar} 

Note that $\vec{x}$ (resp. $\vec{P}$) denotes a vector of names
(resp. processes) of length $|\vec{x}|$ (resp. $|\vec{P}|$). We adopt
the following useful abbreviations.

\begin{mathpar}
   x?(\vec{y}).P := x.(\vec{y})P \and  x\clift{\vec{P}} := x.\clift{\vec{P}}
   \and x!(y) := \lift{x}{\dropn{y}}
   \and \Pi_{i=0}^{n-1}P_i := P_0 | \ldots | P_{n-1}
\end{mathpar}

\subsubsection{Structural congruence}

\paragraph{Free and bound names and alpha-equivalence.} At the
core of structural equivalence is alpha-equivalence which identifies
process that are the same up to a change of variable. Formally, we
recognize the distinction between free and bound names. The free names
of a process, $\freenames{P}$, may be calculated recursively as
follows:

\begin{mathpar}
\freenames{\pzero} := \emptyset
  \and \\
  \freenames{x?(y).P} := \{ x \} \cup (\freenames{P} \setminus \{ y \})
  \and 
  \freenames{x!\langle P \rangle} := \{ x \} \cup \{ P \} 
  \and \\
  \freenames{P|Q} := \freenames{P} \cup \freenames{Q}
  \and \\
  \freenames{@{x}} := \{ x \}
\end{mathpar}

$\pi$
$\quotep{\pi}$

$\freenames{-} : \pi \to \mathcal{P}(\quotep{\pi})$

\begin{eqnarray*}
  \freenames{\pzero} & := & \emptyset \\
  \freenames{x?(y).P} & := & \{ x \} \cup (\freenames{P} \setminus \{ y \}) \\
  \freenames{x!\langle P \rangle} & := & \{ x \} \cup \{ P \} \\
  \freenames{P|Q} & := & \freenames{P} \cup \freenames{Q} \\
  \freenames{\dropn{x}} & := & \{ x \}
\end{eqnarray*}

The bound names of a process, $\boundnames{P}$, are those names occurring in $P$
that are not free. For example, in $x?(y).0$, the name $x$ is free, while $y$ is bound.

\begin{mathpar}
  \inferrule* [lab=monoidal-laws] {} { P|Q \equiv Q|P \and P|0 \equiv P \and P|(Q|R) \equiv (P|Q)|R }
\end{mathpar}

\begin{mathpar}
  \inferrule* [lab=alpha-equivalence] {} { (x)P \equiv (y)P\{y/x\} \and y \not\in \freenames{P} }
\end{mathpar}

\begin{definition}
Then two processes, $P,Q$, are alpha-equivalent if $P = Q\{\vec{y}/\vec{x}\}$ for
some $\vec{x} \in \boundnames{Q},\vec{y} \in \boundnames{P}$, where $Q\{\vec{y}/\vec{x}\}$
denotes the capture-avoiding substitution of $\vec{y}$ for $\vec{x}$ in $Q$.
\end{definition}

\begin{definition}
  The {\em structural congruence} \cite{SangiorgiWalker} , $\equiv$,
  between processes is the least congruence containing
  alpha-equivalence, satisfying the abelian monoid laws
  (associativity, commutativity and $\pzero$ as identity) for parallel
  composition $|$ and for summation $+$.
\end{definition}

\subsection{Name equivalence}

We take name equivalence, written $\nameeq$, to be the smallest
equivalence relation generated by the following rules.

\begin{mathpar}
\inferrule*[lab=Quote-drop]
{ }
{ \quotep{@{x}} \nameeq x }

\inferrule*[lab=Struct-equiv]
{ P \scong Q }
{ \quotep{P} \nameeq \quotep{Q} }
\end{mathpar}

The astute reader will have noticed that the mutual recursion of names
and processes imposes a mutual recursion on alpha-equivalence and
structural equivalence via name-equivalence. Fortunately, all of this
works out pleasantly and we may calculate in the natural way, free of
concern. The reader interested in the details is referred to the
appendix \ref{appendix:rho_details}.

\subsection{Substitution}

We use $\Proc$ for the set of processes, $\QProc$ for the set of
names, and $\id{\{}\vec{y} / \vec{x} \id{\}}$ to denote partial maps,
$s : \QProc \rightarrow \QProc$. A map, $s$ lifts, uniquely, to a map
on process terms, $\widehat{s} : \Proc \rightarrow \Proc$ by the
following equations.

\begin{mathpar}
  (0) \psubstp{Q}{P} := 0 \\
  (R \juxtap S) \psubstp{Q}{P}
  :=    
  (R)\psubstp{Q}{P} \juxtap (S) \psubstp{Q}{P} \\
  (x?(y).R) \psubstp{Q}{P}    
  :=    
  (x)\substp{Q}{P} (z)\concat( (R \psubstn{z}{y}) \psubstp{Q}{P} ) \\
  (\lift{x}{R}) \psubstp{Q}{P}  
  :=
  \lift{(x)\substp{Q}{P}}{ R \psubstp{Q}{P} } \\
%   (\dropn{x})  \psubstp{Q}{P}       
%   := 
%   \left\{ 
%     \begin{array}{ccc} 
%       \dropn{\quotep{Q}} & & x \nameeq \quotep{P} \\
%       \dropn{x} & & otherwise \\
%     \end{array}
%   \right. 
  (\dropn{x})  \psubstp{Q}{P}       
  := 
  \left\{ 
    \begin{array}{ccc} 
      Q & & x \nameeq \quotep{P} \\
      \dropn{x} & & otherwise \\
    \end{array}
  \right.
\end{mathpar}
 

where

\begin{eqnarray}
  (x)\id{\{} \lpquote Q \rpquote / \lpquote P \rpquote \id{\}}            = 
  \left\{ 
    \begin{array}{ccc}
      \lpquote Q \rpquote & & x \nameeq \lpquote P \rpquote \\
      x & & otherwise \\
    \end{array}
  \right. \nonumber
\end{eqnarray}

and $z$ is chosen distinct from $\quotep{P}$, $\quotep{Q}$, the free
names in $Q$, and all the names in $R$. Our $\alpha$-equivalence will
be built in the standard way from this substitution.

\begin{remark}\label{rem:no_self_referential_names}
  One consequence of these definitions is that $\forall P. \quotep{P}
  \not\in \freenames{P}$.
\end{remark}

\subsection{ Dynamic quote: an example }

Anticipating something of what's to come, consider applying the
substitution, $\widehat{\id{\{}u / z \id{\}}}$, to the following pair
of processes, $\lift{w}{y!(z)}$ and $w[ \lpquote y!(z) \rpquote ]$.

\begin{eqnarray}
	\lift{w}{y!(z)}\widehat{\id{\{}u / z \id{\}}}
		& = &
		\lift{w}{y!(u)} \nonumber\\
	w[ \lpquote y!(z) \rpquote ] \widehat{ \id{\{}u / z \id{\}} }
		& = &
		w[ \lpquote y!(z) \rpquote ] \nonumber
\end{eqnarray}

Because the body of the process between quotes is impervious to
substitution, we get radically different answers. In fact, by
examining the first process in an input context,
e.g. $x?(z).\lift{w}{y!(z)}$, we see that the process under the lift
operator may be shaped by prefixed inputs binding a name inside it. In
this sense, the lift operator will be seen as a way to dynamically
construct processes before reifying them as names.

Finally equipped with these standard features we can present the
dynamics of the calculus.

\subsubsection{Operational semantics} 

Finally, we introduce the computational dynamics. What marks these
algebras as distinct from other more traditionally studied algebraic
structures, e.g. vector spaces or polynomial rings, is the manner in
which dynamics is captured. In traditional structures, dynamics is typically
expressed through morphisms between such structures, as in linear maps
between vector spaces or morphisms between rings. In algebras
associated with the semantics of computation, the dynamics is
expressed as part of the algebraic structure itself, through a
reduction reduction relation typically denoted by $\red$. Below, we
give a recursive presentation of this relation for the calculus used
in the encoding.

$\red \subseteq \pi \times \pi$
$\red : \pi \to \mathcal{P}(\pi)$

\begin{mathpar}
  \inferrule* [lab=Comm] { \textsf{match}( x_{src}, x_{trgt} ) } { x_{trgt}?(y)P \; | \; x_{src}!\langle {Q} \rangle \red P\{\quotep{Q}/y}\} }
  \and \\
  \inferrule* [lab=Par] {{P} \red {P}'} {{{P} | {Q}} \red {{P}' | {Q}}}
  \and
  \inferrule* [lab=Equiv]{{{P} \scong {P}'} \andalso {{P}' \red {Q}'} \andalso {{Q}' \scong {Q}}}{{P} \red {Q}}
\end{mathpar}

\begin{eqnarray*}
  match_{\equiv} (\quotep{P},\quotep{Q}) & := & P \equiv Q \\
  match_{\dagger}(\quotep{P},\quotep{Q}) & := & \forall R. P|Q \red^{*} R => R \red^{*} 0 \\
  match_{K}(\quotep{P},\quotep{Q}) & := & K \mbox{ for some context } K
\end{eqnarray*}

$u?(x)P | u!\langle Q \rangle \red P\{\quotep{Q}/x\}$

%We write $\wred$ for $\red^*$, and $P\red$ if $\exists Q $ such that $ P \red Q$.
We write $P\red$ if $\exists Q $ such that $ P \red Q$ and $P\not\red$, otherwise.

\section{Replication}

As mentioned before, it is known that replication (and hence
recursion) can be implemented in a higher-order process algebra
\cite{SangiorgiWalker}. As our first example of calculation with the
machinery thus far presented we give the construction explicitly in
the {\rhoc}.

\begin{eqnarray}
	D_{x} & := & \prefix{x}{y}{(\binpar{\outputp{x}{y}}{@{y}})} \nonumber\\
	\bangp_{x}{P} & := & \binpar{{x}!\langle{\binpar{D_{x}}{P}}\rangle}{D_{x}} \nonumber
\end{eqnarray}

\begin{eqnarray}
	\bangp_{x}{P} & & \nonumber\\
	=
	& {x}!\langle{(\prefix{x}{y}{(\outputp{x}{y} | @{y})) | P}}\rangle 
	      | \prefix{x}{y}{(\outputp{x}{y} | @{y})} & \nonumber\\
	\red
	& (\outputp{x}{y} | @{y})\substn{\quotep{(\prefix{x}{y}{(@{y} | \outputp{x}{y})) | P}}}{y} & \nonumber\\
	=
	& \outputp{x}{\quotep{(\prefix{x}{y}{(\outputp{x}{y} | @{y})) | P}}}
	  | {(\prefix{x}{y}{(\outputp{x}{y} | @{y})) | P}} & \nonumber\\
	\red
	& \ldots & \nonumber\\
	\red^*
	& P | P | \ldots & \nonumber
\end{eqnarray}

Of course, this encoding, as an implementation, runs away, unfolding
$\bangp{P}$ eagerly. A lazier and more implementable replication
operator, restricted to input-guarded processes, may be obtained as follows.

\begin{eqnarray}
\bangp{\prefix{u}{v}{P}} 
	:= 
	\binpar{\lift{x}{\prefix{u}{v}{(\binpar{D(x)}{P})}}}{D(x)} \nonumber
\end{eqnarray}

\begin{remark}
  Note that the lazier definition still does not deal with summation
  or mixed summation (i.e. sums over input and output). The reader is
  invited to construct definitions of replication that deal with these
  features. 

  Further, the definitions are parameterized in a name, $x$. Can you,
  gentle reader, make a definition that eliminates this parameter and
  guarantees no accidental interaction between the replication
  machinery and the process being replicated -- i.e. no accidental
  sharing of names used by the process to get its work done and the
  name(s) used by the replication to effect copying. This latter
  revision of the definition of replication is crucial to obtaining
  the expected identity $!!P \sim !P$.
\end{remark}

\begin{remark}\label{rem:paradoxical_combinator}
  The reader familiar with the lambda calculus will have noticed the
  similarity between $D$ and the paradoxical combinator.

  [Ed. note: the existence of this seems to suggest we have to be more
  restrictive on the set of processes and names we admit if we are to
  support no-cloning.]
\end{remark}

\subsubsection{Bisimulation}

The computational dynamics gives rise to another kind of equivalence,
the equivalence of computational behavior. As previously mentioned
this is typically captured \emph{via} some form of bisimulation.

% The notion we use in this paper is weak barbed bisimulation
% \cite{milner91polyadicpi}.

The notion we use in this paper is derived from weak barbed
bisimulation \cite{milner91polyadicpi}. 

\begin{definition}
An \emph{observation relation}, $\downarrow_{\mathcal N}$, over a set
of names, $\mathcal N$, is the smallest relation satisfying the rules
below.

\infrule[Out-barb]{y \in {\mathcal N}, \; x \nameeq y}
		  {\outputp{x}{v} \downarrow_{\mathcal N} x}
\infrule[Par-barb]{\mbox{$P\downarrow_{\mathcal N} x$ or $Q\downarrow_{\mathcal N} x$}}
		  {\binpar{P}{Q} \downarrow_{\mathcal N} x}

We write $P \Downarrow_{\mathcal N} x$ if there is $Q$ such that 
$P \wred Q$ and $Q \downarrow_{\mathcal N} x$.
\end{definition}

\begin{definition}
%\label{def.bbisim}
An  ${\mathcal N}$-\emph{barbed bisimulation} over a set of names, ${\mathcal N}$, is a symmetric binary relation 
${\mathcal S}_{\mathcal N}$ between agents such that $P\rel{S}_{\mathcal N}Q$ implies:
\begin{enumerate}
\item If $P \red P'$ then $Q \wred Q'$ and $P'\rel{S}_{\mathcal N} Q'$.
\item If $P\downarrow_{\mathcal N} x$, then $Q\Downarrow_{\mathcal N} x$.
\end{enumerate}
$P$ is ${\mathcal N}$-barbed bisimilar to $Q$, written
$P \wbbisim_{\mathcal N} Q$, if $P \rel{S}_{\mathcal N} Q$ for some ${\mathcal N}$-barbed bisimulation ${\mathcal S}_{\mathcal N}$.
\end{definition}

$\mathcal{R} \subseteq \pi \times \pi$

$P \mathcal{R} Q => \forall P'. P \red P' \Rightarrow \exists Q'. Q \red Q', P' \mathcal{R} Q'$

$P \vdash x \Rightarrow Q \vdash x$

\begin{mathpar}
  \inferrule*[lab=Out-barb]{x \nameeq y}{{y}!\langle{Q}\rangle \vdash x}
  \and
  \inferrule*[lab=Par-barb]{\mbox{$P\vdash x$ or $Q\vdash x$}}{\binpar{P}{Q} \vdash x}
\end{mathpar}

\subsubsection{Contexts}

One of the principle advantages of computational calculi like the
$\pi$-calculus is a well-defined notion of context,
contextual-equivalence and a correlation between
contextual-equivalence and notions of bisimulation. The notion of
context allows the decomposition of a process into (sub-)process and
its syntactic environment, its context. Thus, a context may be
thought of as a process with a ``hole'' (written $\Box$) in it. The
application of a context $M$ to a process $P$, written $M[P]$, is
tantamount to filling the hole in $M$ with $P$. In this paper we do
not need the full weight of this theory, but do make use of the notion
of context in the proof the main theorem. 

\begin{mathpar}
  \inferrule* [lab=summation] {} {{M_{M},M_{N}} \bc \Box \;|\; x.M_{A} \;|\; M_{M}+M_{N}}
  \and
  \inferrule* [lab=agent] {} {{M_{A}} \bc (\vec{x})M_{P} \;| \; \clift{P_0,\ldots,M_{P},\ldots,P_N}}
  \and \\
  \inferrule* [lab=process] {} {{M_{P}} \bc M_{N} \;| \;P|M_{P} }
\end{mathpar} 

\begin{mathpar}
  \inferrule* [lab=sychronization] {} {M_{N} \bc \Box \;|\; x?M_{F} \;|\; x!M_{C}}
  \and
  \inferrule* [lab=abstraction] {} {{M_{F}} \bc (x)M_{P} }
  \and
  \inferrule* [lab=concretion] {} {{M_{C}} \bc \langle M_{P} \rangle }
  \and \\
  \inferrule* [lab=process] {} {{M_{P}} \bc M_{N} \;| \;P|M_{P} }
\end{mathpar}

\begin{definition}[contextual application] Given a context $M$, and
  process $P$, we define the \emph{contextual application}, $M[P] :=
  M\{P/\Box\}$. That is, the contextual application of M to P is the
  substitution of $P$ for $\Box$ in $M$.
\end{definition}

$\meaningof{-} : L \to \mathcal{P}(\pi)$

\begin{mathpar}
  \inferrule* [lab=collection] {} {\meaningof{true} = \pi, \and \meaningof{~E} = \pi \setminus \meaningof{E}, \and \meaningof{E_{1} \& E_{2}} = \meaningof{E_{1}} \cap \meaningof{E_{2}}}
\end{mathpar}

\begin{mathpar}
  \inferrule* [lab=structure] {} {\meaningof{0} = \{ P \in \pi | P \equiv 0 \}, \and \\ \meaningof{E_1 | E_2} = \{ P \in \pi | P \equiv P_{1} | P_{2}, P_{1} \in \meaningof{E_{1}}, P_{2} \in \meaningof{E_2}\} }
\end{mathpar}

\begin{mathpar}
 \inferrule* [lab=behavior] {} {\meaningof{\langle a?b \rangle E} = \{ P \in \pi | P \equiv Q | u?(y)P', \\ \and \\\\ \and \\ \;\;\; u \in \meaningof{a}, \forall z.P'\{z/y\} \in \meaningof{E\{z/b\}}\}, \and \\ \meaningof{a!E} = \{ P \in \pi | P \equiv Q | x!\langle P' \rangle, x \in \meaningof{a} P' \in \meaningof{E}\} }
\end{mathpar}

\begin{mathpar}
 \inferrule* [lab=nominal] {} {\meaningof{\quotep{E}} = \{ \quotep{P} \in \quotep{\pi} | P \in \meaningof{E} \}, \and \meaningof{\quotep{P}} = \{ \quotep{Q} \in \quotep{\pi} | P \equiv Q \} \and \\ \meaningof{@\quotep{E}} = \{ P \in \pi | P \equiv @x, x \in \meaningof{E} \}}
\end{mathpar}

\begin{eqnarray*}
  \\
  \meaningof{-} : TS \to ST
\end{eqnarray*}

\begin{eqnarray*}
  \\
  L : TS \to ST
\end{eqnarray*}

\begin{eqnarray*}
  \\
  P \models E \iff P \in \meaningof{E}
\end{eqnarray*}

\begin{eqnarray*}
  P \approx_{L} Q \iff \forall E \in L. P \models E \iff Q \models E
\end{eqnarray*}

\begin{eqnarray*}
  P \approx_{K} Q
\end{eqnarray*}

\begin{eqnarray*}
  P \approx Q
\end{eqnarray*}

$\approx_{K} = \approx = \approx_{L}$

\subsubsection{Contextual duality}

Note that contexts extend the quotation operation to a family of
operations from processes to names. Given a context, $M$, we can
define a \emph{nominal context}, $\quotep{M}$ by $\quotep{M}[P] :=
\quotep{M[P]}$. To foreshadow what is to come we observe that these
operations enjoy a duality with processes very much like the duality
between vectors and maps from vectors to scalars.

Further, because the calculus is essentially higher-order, we have a
correspondence between contexts and processes. More specifically,
given a name $x$ and a context $M$ we can construct $M^{*}_{x}$ such
that 

\begin{mathpar}
  M^{*}_{x} | \lift{x}{P} \red M[P]
\end{mathpar}

namely,

\begin{mathpar}
  M^{*}_{x} := x?(u).M[\dropn{u}]
\end{mathpar}

The dependence of $M^{*}_{x}$ on a name makes it an abstraction, 

\begin{mathpar}
  M^{*} := (x)x?(u).M[\dropn{u}]
\end{mathpar}

\subsection{Additional notation}

It will sometimes be convenient to denote the process a name
quotes. We already have the notation $x = \quotep{P}$, but it will be
convenient to introduce an alternate notation, $\procn{x}$, when we
want to emphasize the connection to the use of the name. Note that, by
virtue of name equivalence, $\quotep{\procn{x}} \nameeq x$; so, the
notation is consistent with previous definitions.

Further, because names have structure it is possible to effect
substitutions on the basis of that structure. This means we need to
upgrade our notation for substitutions, which we accomplish by
adapting comprehension notation. Thus,

\begin{mathpar}
  P\{ y / x : x \in S \}
\end{mathpar}

is interpreted to mean the process derived from P by replacing (in a
capture-avoiding manner) each occurrence of $x$ in $S$ by $y$. For example,

\begin{mathpar}
  P\{ \quotep{\procn{x}|\procn{x}} / x : x \in \freenames{P} \}
\end{mathpar}

will replace each (occurrence) of a free name $x$ in $P$ by
$\quotep{\procn{x}|\procn{x}}$.

Also, we will avail ourselves of the notation $x^{L}$ and $x^{R}$ to
denote injections of a name into disjoint copies of the name
space. There are numerous ways to accomplish this. One example can be
found in \cite{MeredithR05}. This notation overloads to vectors of
names: $\vec{x}^{\pi} := (x_{i}^{\pi} \; : \; 0 \leq i < |\vec{x}| )$ where $\pi \in \{L,R\}$.

We also use $P^{\Box} := P|\Box$.

In \cite{MeredithR05} an interpretation of the new operator is
given. It turns out that there are several possible interpretations
all enjoying the requisite algebraic properties of the operator (see
\cite{milner91polyadicpi}). We will therefore make liberal use of
$(\nu\; \vec{x})P$.

% subsection the_syntax_and_semantics_of_the_notation_system (end)   

\input{qm2pi.qmops} 

\input{qm2pi.sterngerlach} 

\input{qm2pi.metric} 

% section concurrent_process_calculi (end)

%\input{qm2pi.proofsketch}

% section proof sketch (end)

%\input{qm2pi.slviaknots} 

% section spatial logic via knots (end)

\input{qm2pi.conclusion}

% section conclusion (end)

%\input{qm2pi.dtcodes} 

% section wiring algorithm (end)

\input{qm2pi.ack} 

% section acknowledgments (end)

\newpage


\bibliographystyle{plain}   
\bibliography{../../biblios/main.bib}

\input{qm2pi.rhodetails}

\end{document}

 

%\documentclass[12pt]{llncs}
%\documentclass{jktr}

\usepackage[pdftex]{hyperref}                   
\usepackage {listings}
\usepackage {mathpartir}
\usepackage{bcprules}
%\usepackage{listings}
                       
\usepackage{graphicx} 
%\usepackage[margins=2.5cm,nohead,nofoot]{geometry}
%\usepackage{geometry}
\usepackage{amsfonts}
\usepackage{amstext}
\usepackage{latexsym}
\usepackage{amssymb}
\usepackage{color}


%\include{myPreamble}
\include{qm2pi.local} 

%\ifpdf
%\usepackage[pdftex]{graphicx}
%\else
%\usepackage{graphicx}
%\fi

 % \ifpdf
%  \usepackage{pdfsync}
%  \if


%\title{Brief Article}
%\author{David F. Snyder}
%\author{L.G. Meredith}

%\address{Dept. of Math., Texas State University--San Marcos, San Marcos, TX 78666}
       
\pagestyle{empty}


\begin{document}

\lstset{language=[Objective]Caml,frame=shadowbox}

\input{qm2pi.front}

% section front matter (end)

\input{qm2pi.intro} 
 
% section introduction (end)

% \input{qm2pi.knotations} 

% section notation (end)

\input{qm2pi.process.calculi} 

% section concurrent_process_calculi_and_spatial_logics_ (end)
    
%\input{qm2pi.knots2pi} 

%\input{qm2pi.trefoil} 

%\input{qm2pi.mainthm} 

% subsection basic_interpretation (end)

%\input{qm2pi.rho.presentation} 
\subsection{The syntax and semantics of the notation system}\label{sub:the_syntax_and_semantics_of_the_notation_system} % (fold)

We now summarize a technical presentation of the calculus that
embodies our theory of dynamics. The typical presentation of such a
calculus follows the style of giving generators and relations on
them. The grammar, below, describing term constructors, freely
generates the set of processes, $\Proc$. This set is then quotiented
by a relation known as structural congruence and it is over this set
that the notion of dynamics is expressed. This presentation is
essentially that of \cite{MeredithR05} with the addition of
polyadicity and summation. For readability we have relegated some of
the technical subtleties to an appendix.

\subsubsection{Process grammar}\label{subsub:process_grammar}

\begin{mathpar}
  \inferrule* [lab=synchronization] {} {{M} \bc \pzero \;|\; x?F \;|\; x!C }
  \and
  \inferrule* [lab=abstraction] {} {{F} \bc (x)P}
  \and
  \inferrule* [lab=concretion] {} {{C} \bc \langle Q \rangle}
  \and
  \inferrule* [lab=process] {} {{P,Q} \bc M \;| \;P|Q \;|\; @{x}}
  \and
  \inferrule* [lab=name] {} {{x} \bc \quotep{P}}
\end{mathpar} 

Note that $\vec{x}$ (resp. $\vec{P}$) denotes a vector of names
(resp. processes) of length $|\vec{x}|$ (resp. $|\vec{P}|$). We adopt
the following useful abbreviations.

\begin{mathpar}
   x?(\vec{y}).P := x.(\vec{y})P \and  x\clift{\vec{P}} := x.\clift{\vec{P}}
   \and x!(y) := \lift{x}{\dropn{y}}
   \and \Pi_{i=0}^{n-1}P_i := P_0 | \ldots | P_{n-1}
\end{mathpar}

\subsubsection{Structural congruence}

\paragraph{Free and bound names and alpha-equivalence.} At the
core of structural equivalence is alpha-equivalence which identifies
process that are the same up to a change of variable. Formally, we
recognize the distinction between free and bound names. The free names
of a process, $\freenames{P}$, may be calculated recursively as
follows:

\begin{mathpar}
\freenames{\pzero} := \emptyset
  \and \\
  \freenames{x?(y).P} := \{ x \} \cup (\freenames{P} \setminus \{ y \})
  \and 
  \freenames{x!\langle P \rangle} := \{ x \} \cup \{ P \} 
  \and \\
  \freenames{P|Q} := \freenames{P} \cup \freenames{Q}
  \and \\
  \freenames{@{x}} := \{ x \}
\end{mathpar}

$\pi$
$\quotep{\pi}$

$\freenames{-} : \pi \to \mathcal{P}(\quotep{\pi})$

\begin{eqnarray*}
  \freenames{\pzero} & := & \emptyset \\
  \freenames{x?(y).P} & := & \{ x \} \cup (\freenames{P} \setminus \{ y \}) \\
  \freenames{x!\langle P \rangle} & := & \{ x \} \cup \{ P \} \\
  \freenames{P|Q} & := & \freenames{P} \cup \freenames{Q} \\
  \freenames{\dropn{x}} & := & \{ x \}
\end{eqnarray*}

The bound names of a process, $\boundnames{P}$, are those names occurring in $P$
that are not free. For example, in $x?(y).0$, the name $x$ is free, while $y$ is bound.

\begin{mathpar}
  \inferrule* [lab=monoidal-laws] {} { P|Q \equiv Q|P \and P|0 \equiv P \and P|(Q|R) \equiv (P|Q)|R }
\end{mathpar}

\begin{mathpar}
  \inferrule* [lab=alpha-equivalence] {} { (x)P \equiv (y)P\{y/x\} \and y \not\in \freenames{P} }
\end{mathpar}

\begin{definition}
Then two processes, $P,Q$, are alpha-equivalent if $P = Q\{\vec{y}/\vec{x}\}$ for
some $\vec{x} \in \boundnames{Q},\vec{y} \in \boundnames{P}$, where $Q\{\vec{y}/\vec{x}\}$
denotes the capture-avoiding substitution of $\vec{y}$ for $\vec{x}$ in $Q$.
\end{definition}

\begin{definition}
  The {\em structural congruence} \cite{SangiorgiWalker} , $\equiv$,
  between processes is the least congruence containing
  alpha-equivalence, satisfying the abelian monoid laws
  (associativity, commutativity and $\pzero$ as identity) for parallel
  composition $|$ and for summation $+$.
\end{definition}

\subsection{Name equivalence}

We take name equivalence, written $\nameeq$, to be the smallest
equivalence relation generated by the following rules.

\begin{mathpar}
\inferrule*[lab=Quote-drop]
{ }
{ \quotep{@{x}} \nameeq x }

\inferrule*[lab=Struct-equiv]
{ P \scong Q }
{ \quotep{P} \nameeq \quotep{Q} }
\end{mathpar}

The astute reader will have noticed that the mutual recursion of names
and processes imposes a mutual recursion on alpha-equivalence and
structural equivalence via name-equivalence. Fortunately, all of this
works out pleasantly and we may calculate in the natural way, free of
concern. The reader interested in the details is referred to the
appendix \ref{appendix:rho_details}.

\subsection{Substitution}

We use $\Proc$ for the set of processes, $\QProc$ for the set of
names, and $\id{\{}\vec{y} / \vec{x} \id{\}}$ to denote partial maps,
$s : \QProc \rightarrow \QProc$. A map, $s$ lifts, uniquely, to a map
on process terms, $\widehat{s} : \Proc \rightarrow \Proc$ by the
following equations.

\begin{mathpar}
  (0) \psubstp{Q}{P} := 0 \\
  (R \juxtap S) \psubstp{Q}{P}
  :=    
  (R)\psubstp{Q}{P} \juxtap (S) \psubstp{Q}{P} \\
  (x?(y).R) \psubstp{Q}{P}    
  :=    
  (x)\substp{Q}{P} (z)\concat( (R \psubstn{z}{y}) \psubstp{Q}{P} ) \\
  (\lift{x}{R}) \psubstp{Q}{P}  
  :=
  \lift{(x)\substp{Q}{P}}{ R \psubstp{Q}{P} } \\
%   (\dropn{x})  \psubstp{Q}{P}       
%   := 
%   \left\{ 
%     \begin{array}{ccc} 
%       \dropn{\quotep{Q}} & & x \nameeq \quotep{P} \\
%       \dropn{x} & & otherwise \\
%     \end{array}
%   \right. 
  (\dropn{x})  \psubstp{Q}{P}       
  := 
  \left\{ 
    \begin{array}{ccc} 
      Q & & x \nameeq \quotep{P} \\
      \dropn{x} & & otherwise \\
    \end{array}
  \right.
\end{mathpar}
 

where

\begin{eqnarray}
  (x)\id{\{} \lpquote Q \rpquote / \lpquote P \rpquote \id{\}}            = 
  \left\{ 
    \begin{array}{ccc}
      \lpquote Q \rpquote & & x \nameeq \lpquote P \rpquote \\
      x & & otherwise \\
    \end{array}
  \right. \nonumber
\end{eqnarray}

and $z$ is chosen distinct from $\quotep{P}$, $\quotep{Q}$, the free
names in $Q$, and all the names in $R$. Our $\alpha$-equivalence will
be built in the standard way from this substitution.

\begin{remark}\label{rem:no_self_referential_names}
  One consequence of these definitions is that $\forall P. \quotep{P}
  \not\in \freenames{P}$.
\end{remark}

\subsection{ Dynamic quote: an example }

Anticipating something of what's to come, consider applying the
substitution, $\widehat{\id{\{}u / z \id{\}}}$, to the following pair
of processes, $\lift{w}{y!(z)}$ and $w[ \lpquote y!(z) \rpquote ]$.

\begin{eqnarray}
	\lift{w}{y!(z)}\widehat{\id{\{}u / z \id{\}}}
		& = &
		\lift{w}{y!(u)} \nonumber\\
	w[ \lpquote y!(z) \rpquote ] \widehat{ \id{\{}u / z \id{\}} }
		& = &
		w[ \lpquote y!(z) \rpquote ] \nonumber
\end{eqnarray}

Because the body of the process between quotes is impervious to
substitution, we get radically different answers. In fact, by
examining the first process in an input context,
e.g. $x?(z).\lift{w}{y!(z)}$, we see that the process under the lift
operator may be shaped by prefixed inputs binding a name inside it. In
this sense, the lift operator will be seen as a way to dynamically
construct processes before reifying them as names.

Finally equipped with these standard features we can present the
dynamics of the calculus.

\subsubsection{Operational semantics} 

Finally, we introduce the computational dynamics. What marks these
algebras as distinct from other more traditionally studied algebraic
structures, e.g. vector spaces or polynomial rings, is the manner in
which dynamics is captured. In traditional structures, dynamics is typically
expressed through morphisms between such structures, as in linear maps
between vector spaces or morphisms between rings. In algebras
associated with the semantics of computation, the dynamics is
expressed as part of the algebraic structure itself, through a
reduction reduction relation typically denoted by $\red$. Below, we
give a recursive presentation of this relation for the calculus used
in the encoding.

$\red \subseteq \pi \times \pi$
$\red : \pi \to \mathcal{P}(\pi)$

\begin{mathpar}
  \inferrule* [lab=Comm] { \textsf{match}( x_{src}, x_{trgt} ) } { x_{trgt}?(y)P \; | \; x_{src}!\langle {Q} \rangle \red P\{\quotep{Q}/y}\} }
  \and \\
  \inferrule* [lab=Par] {{P} \red {P}'} {{{P} | {Q}} \red {{P}' | {Q}}}
  \and
  \inferrule* [lab=Equiv]{{{P} \scong {P}'} \andalso {{P}' \red {Q}'} \andalso {{Q}' \scong {Q}}}{{P} \red {Q}}
\end{mathpar}

\begin{eqnarray*}
  match_{\equiv} (\quotep{P},\quotep{Q}) & := & P \equiv Q \\
  match_{\dagger}(\quotep{P},\quotep{Q}) & := & \forall R. P|Q \red^{*} R => R \red^{*} 0 \\
  match_{K}(\quotep{P},\quotep{Q}) & := & K \mbox{ for some context } K
\end{eqnarray*}

$u?(x)P | u!\langle Q \rangle \red P\{\quotep{Q}/x\}$

%We write $\wred$ for $\red^*$, and $P\red$ if $\exists Q $ such that $ P \red Q$.
We write $P\red$ if $\exists Q $ such that $ P \red Q$ and $P\not\red$, otherwise.

\section{Replication}

As mentioned before, it is known that replication (and hence
recursion) can be implemented in a higher-order process algebra
\cite{SangiorgiWalker}. As our first example of calculation with the
machinery thus far presented we give the construction explicitly in
the {\rhoc}.

\begin{eqnarray}
	D_{x} & := & \prefix{x}{y}{(\binpar{\outputp{x}{y}}{@{y}})} \nonumber\\
	\bangp_{x}{P} & := & \binpar{{x}!\langle{\binpar{D_{x}}{P}}\rangle}{D_{x}} \nonumber
\end{eqnarray}

\begin{eqnarray}
	\bangp_{x}{P} & & \nonumber\\
	=
	& {x}!\langle{(\prefix{x}{y}{(\outputp{x}{y} | @{y})) | P}}\rangle 
	      | \prefix{x}{y}{(\outputp{x}{y} | @{y})} & \nonumber\\
	\red
	& (\outputp{x}{y} | @{y})\substn{\quotep{(\prefix{x}{y}{(@{y} | \outputp{x}{y})) | P}}}{y} & \nonumber\\
	=
	& \outputp{x}{\quotep{(\prefix{x}{y}{(\outputp{x}{y} | @{y})) | P}}}
	  | {(\prefix{x}{y}{(\outputp{x}{y} | @{y})) | P}} & \nonumber\\
	\red
	& \ldots & \nonumber\\
	\red^*
	& P | P | \ldots & \nonumber
\end{eqnarray}

Of course, this encoding, as an implementation, runs away, unfolding
$\bangp{P}$ eagerly. A lazier and more implementable replication
operator, restricted to input-guarded processes, may be obtained as follows.

\begin{eqnarray}
\bangp{\prefix{u}{v}{P}} 
	:= 
	\binpar{\lift{x}{\prefix{u}{v}{(\binpar{D(x)}{P})}}}{D(x)} \nonumber
\end{eqnarray}

\begin{remark}
  Note that the lazier definition still does not deal with summation
  or mixed summation (i.e. sums over input and output). The reader is
  invited to construct definitions of replication that deal with these
  features. 

  Further, the definitions are parameterized in a name, $x$. Can you,
  gentle reader, make a definition that eliminates this parameter and
  guarantees no accidental interaction between the replication
  machinery and the process being replicated -- i.e. no accidental
  sharing of names used by the process to get its work done and the
  name(s) used by the replication to effect copying. This latter
  revision of the definition of replication is crucial to obtaining
  the expected identity $!!P \sim !P$.
\end{remark}

\begin{remark}\label{rem:paradoxical_combinator}
  The reader familiar with the lambda calculus will have noticed the
  similarity between $D$ and the paradoxical combinator.

  [Ed. note: the existence of this seems to suggest we have to be more
  restrictive on the set of processes and names we admit if we are to
  support no-cloning.]
\end{remark}

\subsubsection{Bisimulation}

The computational dynamics gives rise to another kind of equivalence,
the equivalence of computational behavior. As previously mentioned
this is typically captured \emph{via} some form of bisimulation.

% The notion we use in this paper is weak barbed bisimulation
% \cite{milner91polyadicpi}.

The notion we use in this paper is derived from weak barbed
bisimulation \cite{milner91polyadicpi}. 

\begin{definition}
An \emph{observation relation}, $\downarrow_{\mathcal N}$, over a set
of names, $\mathcal N$, is the smallest relation satisfying the rules
below.

\infrule[Out-barb]{y \in {\mathcal N}, \; x \nameeq y}
		  {\outputp{x}{v} \downarrow_{\mathcal N} x}
\infrule[Par-barb]{\mbox{$P\downarrow_{\mathcal N} x$ or $Q\downarrow_{\mathcal N} x$}}
		  {\binpar{P}{Q} \downarrow_{\mathcal N} x}

We write $P \Downarrow_{\mathcal N} x$ if there is $Q$ such that 
$P \wred Q$ and $Q \downarrow_{\mathcal N} x$.
\end{definition}

\begin{definition}
%\label{def.bbisim}
An  ${\mathcal N}$-\emph{barbed bisimulation} over a set of names, ${\mathcal N}$, is a symmetric binary relation 
${\mathcal S}_{\mathcal N}$ between agents such that $P\rel{S}_{\mathcal N}Q$ implies:
\begin{enumerate}
\item If $P \red P'$ then $Q \wred Q'$ and $P'\rel{S}_{\mathcal N} Q'$.
\item If $P\downarrow_{\mathcal N} x$, then $Q\Downarrow_{\mathcal N} x$.
\end{enumerate}
$P$ is ${\mathcal N}$-barbed bisimilar to $Q$, written
$P \wbbisim_{\mathcal N} Q$, if $P \rel{S}_{\mathcal N} Q$ for some ${\mathcal N}$-barbed bisimulation ${\mathcal S}_{\mathcal N}$.
\end{definition}

$\mathcal{R} \subseteq \pi \times \pi$

$P \mathcal{R} Q => \forall P'. P \red P' \Rightarrow \exists Q'. Q \red Q', P' \mathcal{R} Q'$

$P \vdash x \Rightarrow Q \vdash x$

\begin{mathpar}
  \inferrule*[lab=Out-barb]{x \nameeq y}{{y}!\langle{Q}\rangle \vdash x}
  \and
  \inferrule*[lab=Par-barb]{\mbox{$P\vdash x$ or $Q\vdash x$}}{\binpar{P}{Q} \vdash x}
\end{mathpar}

\subsubsection{Contexts}

One of the principle advantages of computational calculi like the
$\pi$-calculus is a well-defined notion of context,
contextual-equivalence and a correlation between
contextual-equivalence and notions of bisimulation. The notion of
context allows the decomposition of a process into (sub-)process and
its syntactic environment, its context. Thus, a context may be
thought of as a process with a ``hole'' (written $\Box$) in it. The
application of a context $M$ to a process $P$, written $M[P]$, is
tantamount to filling the hole in $M$ with $P$. In this paper we do
not need the full weight of this theory, but do make use of the notion
of context in the proof the main theorem. 

\begin{mathpar}
  \inferrule* [lab=summation] {} {{M_{M},M_{N}} \bc \Box \;|\; x.M_{A} \;|\; M_{M}+M_{N}}
  \and
  \inferrule* [lab=agent] {} {{M_{A}} \bc (\vec{x})M_{P} \;| \; \clift{P_0,\ldots,M_{P},\ldots,P_N}}
  \and \\
  \inferrule* [lab=process] {} {{M_{P}} \bc M_{N} \;| \;P|M_{P} }
\end{mathpar} 

\begin{mathpar}
  \inferrule* [lab=sychronization] {} {M_{N} \bc \Box \;|\; x?M_{F} \;|\; x!M_{C}}
  \and
  \inferrule* [lab=abstraction] {} {{M_{F}} \bc (x)M_{P} }
  \and
  \inferrule* [lab=concretion] {} {{M_{C}} \bc \langle M_{P} \rangle }
  \and \\
  \inferrule* [lab=process] {} {{M_{P}} \bc M_{N} \;| \;P|M_{P} }
\end{mathpar}

\begin{definition}[contextual application] Given a context $M$, and
  process $P$, we define the \emph{contextual application}, $M[P] :=
  M\{P/\Box\}$. That is, the contextual application of M to P is the
  substitution of $P$ for $\Box$ in $M$.
\end{definition}

$\meaningof{-} : L \to \mathcal{P}(\pi)$

\begin{mathpar}
  \inferrule* [lab=collection] {} {\meaningof{true} = \pi, \and \meaningof{~E} = \pi \setminus \meaningof{E}, \and \meaningof{E_{1} \& E_{2}} = \meaningof{E_{1}} \cap \meaningof{E_{2}}}
\end{mathpar}

\begin{mathpar}
  \inferrule* [lab=structure] {} {\meaningof{0} = \{ P \in \pi | P \equiv 0 \}, \and \\ \meaningof{E_1 | E_2} = \{ P \in \pi | P \equiv P_{1} | P_{2}, P_{1} \in \meaningof{E_{1}}, P_{2} \in \meaningof{E_2}\} }
\end{mathpar}

\begin{mathpar}
 \inferrule* [lab=behavior] {} {\meaningof{\langle a?b \rangle E} = \{ P \in \pi | P \equiv Q | u?(y)P', \\ \and \\\\ \and \\ \;\;\; u \in \meaningof{a}, \forall z.P'\{z/y\} \in \meaningof{E\{z/b\}}\}, \and \\ \meaningof{a!E} = \{ P \in \pi | P \equiv Q | x!\langle P' \rangle, x \in \meaningof{a} P' \in \meaningof{E}\} }
\end{mathpar}

\begin{mathpar}
 \inferrule* [lab=nominal] {} {\meaningof{\quotep{E}} = \{ \quotep{P} \in \quotep{\pi} | P \in \meaningof{E} \}, \and \meaningof{\quotep{P}} = \{ \quotep{Q} \in \quotep{\pi} | P \equiv Q \} \and \\ \meaningof{@\quotep{E}} = \{ P \in \pi | P \equiv @x, x \in \meaningof{E} \}}
\end{mathpar}

\begin{eqnarray*}
  \\
  \meaningof{-} : TS \to ST
\end{eqnarray*}

\begin{eqnarray*}
  \\
  L : TS \to ST
\end{eqnarray*}

\begin{eqnarray*}
  \\
  P \models E \iff P \in \meaningof{E}
\end{eqnarray*}

\begin{eqnarray*}
  P \approx_{L} Q \iff \forall E \in L. P \models E \iff Q \models E
\end{eqnarray*}

\begin{eqnarray*}
  P \approx_{K} Q
\end{eqnarray*}

\begin{eqnarray*}
  P \approx Q
\end{eqnarray*}

$\approx_{K} = \approx = \approx_{L}$

\subsubsection{Contextual duality}

Note that contexts extend the quotation operation to a family of
operations from processes to names. Given a context, $M$, we can
define a \emph{nominal context}, $\quotep{M}$ by $\quotep{M}[P] :=
\quotep{M[P]}$. To foreshadow what is to come we observe that these
operations enjoy a duality with processes very much like the duality
between vectors and maps from vectors to scalars.

Further, because the calculus is essentially higher-order, we have a
correspondence between contexts and processes. More specifically,
given a name $x$ and a context $M$ we can construct $M^{*}_{x}$ such
that 

\begin{mathpar}
  M^{*}_{x} | \lift{x}{P} \red M[P]
\end{mathpar}

namely,

\begin{mathpar}
  M^{*}_{x} := x?(u).M[\dropn{u}]
\end{mathpar}

The dependence of $M^{*}_{x}$ on a name makes it an abstraction, 

\begin{mathpar}
  M^{*} := (x)x?(u).M[\dropn{u}]
\end{mathpar}

\subsection{Additional notation}

It will sometimes be convenient to denote the process a name
quotes. We already have the notation $x = \quotep{P}$, but it will be
convenient to introduce an alternate notation, $\procn{x}$, when we
want to emphasize the connection to the use of the name. Note that, by
virtue of name equivalence, $\quotep{\procn{x}} \nameeq x$; so, the
notation is consistent with previous definitions.

Further, because names have structure it is possible to effect
substitutions on the basis of that structure. This means we need to
upgrade our notation for substitutions, which we accomplish by
adapting comprehension notation. Thus,

\begin{mathpar}
  P\{ y / x : x \in S \}
\end{mathpar}

is interpreted to mean the process derived from P by replacing (in a
capture-avoiding manner) each occurrence of $x$ in $S$ by $y$. For example,

\begin{mathpar}
  P\{ \quotep{\procn{x}|\procn{x}} / x : x \in \freenames{P} \}
\end{mathpar}

will replace each (occurrence) of a free name $x$ in $P$ by
$\quotep{\procn{x}|\procn{x}}$.

Also, we will avail ourselves of the notation $x^{L}$ and $x^{R}$ to
denote injections of a name into disjoint copies of the name
space. There are numerous ways to accomplish this. One example can be
found in \cite{MeredithR05}. This notation overloads to vectors of
names: $\vec{x}^{\pi} := (x_{i}^{\pi} \; : \; 0 \leq i < |\vec{x}| )$ where $\pi \in \{L,R\}$.

We also use $P^{\Box} := P|\Box$.

In \cite{MeredithR05} an interpretation of the new operator is
given. It turns out that there are several possible interpretations
all enjoying the requisite algebraic properties of the operator (see
\cite{milner91polyadicpi}). We will therefore make liberal use of
$(\nu\; \vec{x})P$.

% subsection the_syntax_and_semantics_of_the_notation_system (end)   

\input{qm2pi.qmops} 

\input{qm2pi.sterngerlach} 

\input{qm2pi.metric} 

% section concurrent_process_calculi (end)

%\input{qm2pi.proofsketch}

% section proof sketch (end)

%\input{qm2pi.slviaknots} 

% section spatial logic via knots (end)

\input{qm2pi.conclusion}

% section conclusion (end)

%\input{qm2pi.dtcodes} 

% section wiring algorithm (end)

\input{qm2pi.ack} 

% section acknowledgments (end)

\newpage


\bibliographystyle{plain}   
\bibliography{../../biblios/main.bib}

\input{qm2pi.rhodetails}

\end{document}

 

% subsection basic_interpretation (end)

%\input{qm2pi.rho.presentation} 
\subsection{The syntax and semantics of the notation system}\label{sub:the_syntax_and_semantics_of_the_notation_system} % (fold)

We now summarize a technical presentation of the calculus that
embodies our theory of dynamics. The typical presentation of such a
calculus follows the style of giving generators and relations on
them. The grammar, below, describing term constructors, freely
generates the set of processes, $\Proc$. This set is then quotiented
by a relation known as structural congruence and it is over this set
that the notion of dynamics is expressed. This presentation is
essentially that of \cite{MeredithR05} with the addition of
polyadicity and summation. For readability we have relegated some of
the technical subtleties to an appendix.

\subsubsection{Process grammar}\label{subsub:process_grammar}

\begin{mathpar}
  \inferrule* [lab=synchronization] {} {{M} \bc \pzero \;|\; x?F \;|\; x!C }
  \and
  \inferrule* [lab=abstraction] {} {{F} \bc (x)P}
  \and
  \inferrule* [lab=concretion] {} {{C} \bc \langle Q \rangle}
  \and
  \inferrule* [lab=process] {} {{P,Q} \bc M \;| \;P|Q \;|\; @{x}}
  \and
  \inferrule* [lab=name] {} {{x} \bc \quotep{P}}
\end{mathpar} 

Note that $\vec{x}$ (resp. $\vec{P}$) denotes a vector of names
(resp. processes) of length $|\vec{x}|$ (resp. $|\vec{P}|$). We adopt
the following useful abbreviations.

\begin{mathpar}
   x?(\vec{y}).P := x.(\vec{y})P \and  x\clift{\vec{P}} := x.\clift{\vec{P}}
   \and x!(y) := \lift{x}{\dropn{y}}
   \and \Pi_{i=0}^{n-1}P_i := P_0 | \ldots | P_{n-1}
\end{mathpar}

\subsubsection{Structural congruence}

\paragraph{Free and bound names and alpha-equivalence.} At the
core of structural equivalence is alpha-equivalence which identifies
process that are the same up to a change of variable. Formally, we
recognize the distinction between free and bound names. The free names
of a process, $\freenames{P}$, may be calculated recursively as
follows:

\begin{mathpar}
\freenames{\pzero} := \emptyset
  \and \\
  \freenames{x?(y).P} := \{ x \} \cup (\freenames{P} \setminus \{ y \})
  \and 
  \freenames{x!\langle P \rangle} := \{ x \} \cup \{ P \} 
  \and \\
  \freenames{P|Q} := \freenames{P} \cup \freenames{Q}
  \and \\
  \freenames{@{x}} := \{ x \}
\end{mathpar}

$\pi$
$\quotep{\pi}$

$\freenames{-} : \pi \to \mathcal{P}(\quotep{\pi})$

\begin{eqnarray*}
  \freenames{\pzero} & := & \emptyset \\
  \freenames{x?(y).P} & := & \{ x \} \cup (\freenames{P} \setminus \{ y \}) \\
  \freenames{x!\langle P \rangle} & := & \{ x \} \cup \{ P \} \\
  \freenames{P|Q} & := & \freenames{P} \cup \freenames{Q} \\
  \freenames{\dropn{x}} & := & \{ x \}
\end{eqnarray*}

The bound names of a process, $\boundnames{P}$, are those names occurring in $P$
that are not free. For example, in $x?(y).0$, the name $x$ is free, while $y$ is bound.

\begin{mathpar}
  \inferrule* [lab=monoidal-laws] {} { P|Q \equiv Q|P \and P|0 \equiv P \and P|(Q|R) \equiv (P|Q)|R }
\end{mathpar}

\begin{mathpar}
  \inferrule* [lab=alpha-equivalence] {} { (x)P \equiv (y)P\{y/x\} \and y \not\in \freenames{P} }
\end{mathpar}

\begin{definition}
Then two processes, $P,Q$, are alpha-equivalent if $P = Q\{\vec{y}/\vec{x}\}$ for
some $\vec{x} \in \boundnames{Q},\vec{y} \in \boundnames{P}$, where $Q\{\vec{y}/\vec{x}\}$
denotes the capture-avoiding substitution of $\vec{y}$ for $\vec{x}$ in $Q$.
\end{definition}

\begin{definition}
  The {\em structural congruence} \cite{SangiorgiWalker} , $\equiv$,
  between processes is the least congruence containing
  alpha-equivalence, satisfying the abelian monoid laws
  (associativity, commutativity and $\pzero$ as identity) for parallel
  composition $|$ and for summation $+$.
\end{definition}

\subsection{Name equivalence}

We take name equivalence, written $\nameeq$, to be the smallest
equivalence relation generated by the following rules.

\begin{mathpar}
\inferrule*[lab=Quote-drop]
{ }
{ \quotep{@{x}} \nameeq x }

\inferrule*[lab=Struct-equiv]
{ P \scong Q }
{ \quotep{P} \nameeq \quotep{Q} }
\end{mathpar}

The astute reader will have noticed that the mutual recursion of names
and processes imposes a mutual recursion on alpha-equivalence and
structural equivalence via name-equivalence. Fortunately, all of this
works out pleasantly and we may calculate in the natural way, free of
concern. The reader interested in the details is referred to the
appendix \ref{appendix:rho_details}.

\subsection{Substitution}

We use $\Proc$ for the set of processes, $\QProc$ for the set of
names, and $\id{\{}\vec{y} / \vec{x} \id{\}}$ to denote partial maps,
$s : \QProc \rightarrow \QProc$. A map, $s$ lifts, uniquely, to a map
on process terms, $\widehat{s} : \Proc \rightarrow \Proc$ by the
following equations.

\begin{mathpar}
  (0) \psubstp{Q}{P} := 0 \\
  (R \juxtap S) \psubstp{Q}{P}
  :=    
  (R)\psubstp{Q}{P} \juxtap (S) \psubstp{Q}{P} \\
  (x?(y).R) \psubstp{Q}{P}    
  :=    
  (x)\substp{Q}{P} (z)\concat( (R \psubstn{z}{y}) \psubstp{Q}{P} ) \\
  (\lift{x}{R}) \psubstp{Q}{P}  
  :=
  \lift{(x)\substp{Q}{P}}{ R \psubstp{Q}{P} } \\
%   (\dropn{x})  \psubstp{Q}{P}       
%   := 
%   \left\{ 
%     \begin{array}{ccc} 
%       \dropn{\quotep{Q}} & & x \nameeq \quotep{P} \\
%       \dropn{x} & & otherwise \\
%     \end{array}
%   \right. 
  (\dropn{x})  \psubstp{Q}{P}       
  := 
  \left\{ 
    \begin{array}{ccc} 
      Q & & x \nameeq \quotep{P} \\
      \dropn{x} & & otherwise \\
    \end{array}
  \right.
\end{mathpar}
 

where

\begin{eqnarray}
  (x)\id{\{} \lpquote Q \rpquote / \lpquote P \rpquote \id{\}}            = 
  \left\{ 
    \begin{array}{ccc}
      \lpquote Q \rpquote & & x \nameeq \lpquote P \rpquote \\
      x & & otherwise \\
    \end{array}
  \right. \nonumber
\end{eqnarray}

and $z$ is chosen distinct from $\quotep{P}$, $\quotep{Q}$, the free
names in $Q$, and all the names in $R$. Our $\alpha$-equivalence will
be built in the standard way from this substitution.

\begin{remark}\label{rem:no_self_referential_names}
  One consequence of these definitions is that $\forall P. \quotep{P}
  \not\in \freenames{P}$.
\end{remark}

\subsection{ Dynamic quote: an example }

Anticipating something of what's to come, consider applying the
substitution, $\widehat{\id{\{}u / z \id{\}}}$, to the following pair
of processes, $\lift{w}{y!(z)}$ and $w[ \lpquote y!(z) \rpquote ]$.

\begin{eqnarray}
	\lift{w}{y!(z)}\widehat{\id{\{}u / z \id{\}}}
		& = &
		\lift{w}{y!(u)} \nonumber\\
	w[ \lpquote y!(z) \rpquote ] \widehat{ \id{\{}u / z \id{\}} }
		& = &
		w[ \lpquote y!(z) \rpquote ] \nonumber
\end{eqnarray}

Because the body of the process between quotes is impervious to
substitution, we get radically different answers. In fact, by
examining the first process in an input context,
e.g. $x?(z).\lift{w}{y!(z)}$, we see that the process under the lift
operator may be shaped by prefixed inputs binding a name inside it. In
this sense, the lift operator will be seen as a way to dynamically
construct processes before reifying them as names.

Finally equipped with these standard features we can present the
dynamics of the calculus.

\subsubsection{Operational semantics} 

Finally, we introduce the computational dynamics. What marks these
algebras as distinct from other more traditionally studied algebraic
structures, e.g. vector spaces or polynomial rings, is the manner in
which dynamics is captured. In traditional structures, dynamics is typically
expressed through morphisms between such structures, as in linear maps
between vector spaces or morphisms between rings. In algebras
associated with the semantics of computation, the dynamics is
expressed as part of the algebraic structure itself, through a
reduction reduction relation typically denoted by $\red$. Below, we
give a recursive presentation of this relation for the calculus used
in the encoding.

$\red \subseteq \pi \times \pi$
$\red : \pi \to \mathcal{P}(\pi)$

\begin{mathpar}
  \inferrule* [lab=Comm] { \textsf{match}( x_{src}, x_{trgt} ) } { x_{trgt}?(y)P \; | \; x_{src}!\langle {Q} \rangle \red P\{\quotep{Q}/y}\} }
  \and \\
  \inferrule* [lab=Par] {{P} \red {P}'} {{{P} | {Q}} \red {{P}' | {Q}}}
  \and
  \inferrule* [lab=Equiv]{{{P} \scong {P}'} \andalso {{P}' \red {Q}'} \andalso {{Q}' \scong {Q}}}{{P} \red {Q}}
\end{mathpar}

\begin{eqnarray*}
  match_{\equiv} (\quotep{P},\quotep{Q}) & := & P \equiv Q \\
  match_{\dagger}(\quotep{P},\quotep{Q}) & := & \forall R. P|Q \red^{*} R => R \red^{*} 0 \\
  match_{K}(\quotep{P},\quotep{Q}) & := & K \mbox{ for some context } K
\end{eqnarray*}

$u?(x)P | u!\langle Q \rangle \red P\{\quotep{Q}/x\}$

%We write $\wred$ for $\red^*$, and $P\red$ if $\exists Q $ such that $ P \red Q$.
We write $P\red$ if $\exists Q $ such that $ P \red Q$ and $P\not\red$, otherwise.

\section{Replication}

As mentioned before, it is known that replication (and hence
recursion) can be implemented in a higher-order process algebra
\cite{SangiorgiWalker}. As our first example of calculation with the
machinery thus far presented we give the construction explicitly in
the {\rhoc}.

\begin{eqnarray}
	D_{x} & := & \prefix{x}{y}{(\binpar{\outputp{x}{y}}{@{y}})} \nonumber\\
	\bangp_{x}{P} & := & \binpar{{x}!\langle{\binpar{D_{x}}{P}}\rangle}{D_{x}} \nonumber
\end{eqnarray}

\begin{eqnarray}
	\bangp_{x}{P} & & \nonumber\\
	=
	& {x}!\langle{(\prefix{x}{y}{(\outputp{x}{y} | @{y})) | P}}\rangle 
	      | \prefix{x}{y}{(\outputp{x}{y} | @{y})} & \nonumber\\
	\red
	& (\outputp{x}{y} | @{y})\substn{\quotep{(\prefix{x}{y}{(@{y} | \outputp{x}{y})) | P}}}{y} & \nonumber\\
	=
	& \outputp{x}{\quotep{(\prefix{x}{y}{(\outputp{x}{y} | @{y})) | P}}}
	  | {(\prefix{x}{y}{(\outputp{x}{y} | @{y})) | P}} & \nonumber\\
	\red
	& \ldots & \nonumber\\
	\red^*
	& P | P | \ldots & \nonumber
\end{eqnarray}

Of course, this encoding, as an implementation, runs away, unfolding
$\bangp{P}$ eagerly. A lazier and more implementable replication
operator, restricted to input-guarded processes, may be obtained as follows.

\begin{eqnarray}
\bangp{\prefix{u}{v}{P}} 
	:= 
	\binpar{\lift{x}{\prefix{u}{v}{(\binpar{D(x)}{P})}}}{D(x)} \nonumber
\end{eqnarray}

\begin{remark}
  Note that the lazier definition still does not deal with summation
  or mixed summation (i.e. sums over input and output). The reader is
  invited to construct definitions of replication that deal with these
  features. 

  Further, the definitions are parameterized in a name, $x$. Can you,
  gentle reader, make a definition that eliminates this parameter and
  guarantees no accidental interaction between the replication
  machinery and the process being replicated -- i.e. no accidental
  sharing of names used by the process to get its work done and the
  name(s) used by the replication to effect copying. This latter
  revision of the definition of replication is crucial to obtaining
  the expected identity $!!P \sim !P$.
\end{remark}

\begin{remark}\label{rem:paradoxical_combinator}
  The reader familiar with the lambda calculus will have noticed the
  similarity between $D$ and the paradoxical combinator.

  [Ed. note: the existence of this seems to suggest we have to be more
  restrictive on the set of processes and names we admit if we are to
  support no-cloning.]
\end{remark}

\subsubsection{Bisimulation}

The computational dynamics gives rise to another kind of equivalence,
the equivalence of computational behavior. As previously mentioned
this is typically captured \emph{via} some form of bisimulation.

% The notion we use in this paper is weak barbed bisimulation
% \cite{milner91polyadicpi}.

The notion we use in this paper is derived from weak barbed
bisimulation \cite{milner91polyadicpi}. 

\begin{definition}
An \emph{observation relation}, $\downarrow_{\mathcal N}$, over a set
of names, $\mathcal N$, is the smallest relation satisfying the rules
below.

\infrule[Out-barb]{y \in {\mathcal N}, \; x \nameeq y}
		  {\outputp{x}{v} \downarrow_{\mathcal N} x}
\infrule[Par-barb]{\mbox{$P\downarrow_{\mathcal N} x$ or $Q\downarrow_{\mathcal N} x$}}
		  {\binpar{P}{Q} \downarrow_{\mathcal N} x}

We write $P \Downarrow_{\mathcal N} x$ if there is $Q$ such that 
$P \wred Q$ and $Q \downarrow_{\mathcal N} x$.
\end{definition}

\begin{definition}
%\label{def.bbisim}
An  ${\mathcal N}$-\emph{barbed bisimulation} over a set of names, ${\mathcal N}$, is a symmetric binary relation 
${\mathcal S}_{\mathcal N}$ between agents such that $P\rel{S}_{\mathcal N}Q$ implies:
\begin{enumerate}
\item If $P \red P'$ then $Q \wred Q'$ and $P'\rel{S}_{\mathcal N} Q'$.
\item If $P\downarrow_{\mathcal N} x$, then $Q\Downarrow_{\mathcal N} x$.
\end{enumerate}
$P$ is ${\mathcal N}$-barbed bisimilar to $Q$, written
$P \wbbisim_{\mathcal N} Q$, if $P \rel{S}_{\mathcal N} Q$ for some ${\mathcal N}$-barbed bisimulation ${\mathcal S}_{\mathcal N}$.
\end{definition}

$\mathcal{R} \subseteq \pi \times \pi$

$P \mathcal{R} Q => \forall P'. P \red P' \Rightarrow \exists Q'. Q \red Q', P' \mathcal{R} Q'$

$P \vdash x \Rightarrow Q \vdash x$

\begin{mathpar}
  \inferrule*[lab=Out-barb]{x \nameeq y}{{y}!\langle{Q}\rangle \vdash x}
  \and
  \inferrule*[lab=Par-barb]{\mbox{$P\vdash x$ or $Q\vdash x$}}{\binpar{P}{Q} \vdash x}
\end{mathpar}

\subsubsection{Contexts}

One of the principle advantages of computational calculi like the
$\pi$-calculus is a well-defined notion of context,
contextual-equivalence and a correlation between
contextual-equivalence and notions of bisimulation. The notion of
context allows the decomposition of a process into (sub-)process and
its syntactic environment, its context. Thus, a context may be
thought of as a process with a ``hole'' (written $\Box$) in it. The
application of a context $M$ to a process $P$, written $M[P]$, is
tantamount to filling the hole in $M$ with $P$. In this paper we do
not need the full weight of this theory, but do make use of the notion
of context in the proof the main theorem. 

\begin{mathpar}
  \inferrule* [lab=summation] {} {{M_{M},M_{N}} \bc \Box \;|\; x.M_{A} \;|\; M_{M}+M_{N}}
  \and
  \inferrule* [lab=agent] {} {{M_{A}} \bc (\vec{x})M_{P} \;| \; \clift{P_0,\ldots,M_{P},\ldots,P_N}}
  \and \\
  \inferrule* [lab=process] {} {{M_{P}} \bc M_{N} \;| \;P|M_{P} }
\end{mathpar} 

\begin{mathpar}
  \inferrule* [lab=sychronization] {} {M_{N} \bc \Box \;|\; x?M_{F} \;|\; x!M_{C}}
  \and
  \inferrule* [lab=abstraction] {} {{M_{F}} \bc (x)M_{P} }
  \and
  \inferrule* [lab=concretion] {} {{M_{C}} \bc \langle M_{P} \rangle }
  \and \\
  \inferrule* [lab=process] {} {{M_{P}} \bc M_{N} \;| \;P|M_{P} }
\end{mathpar}

\begin{definition}[contextual application] Given a context $M$, and
  process $P$, we define the \emph{contextual application}, $M[P] :=
  M\{P/\Box\}$. That is, the contextual application of M to P is the
  substitution of $P$ for $\Box$ in $M$.
\end{definition}

$\meaningof{-} : L \to \mathcal{P}(\pi)$

\begin{mathpar}
  \inferrule* [lab=collection] {} {\meaningof{true} = \pi, \and \meaningof{~E} = \pi \setminus \meaningof{E}, \and \meaningof{E_{1} \& E_{2}} = \meaningof{E_{1}} \cap \meaningof{E_{2}}}
\end{mathpar}

\begin{mathpar}
  \inferrule* [lab=structure] {} {\meaningof{0} = \{ P \in \pi | P \equiv 0 \}, \and \\ \meaningof{E_1 | E_2} = \{ P \in \pi | P \equiv P_{1} | P_{2}, P_{1} \in \meaningof{E_{1}}, P_{2} \in \meaningof{E_2}\} }
\end{mathpar}

\begin{mathpar}
 \inferrule* [lab=behavior] {} {\meaningof{\langle a?b \rangle E} = \{ P \in \pi | P \equiv Q | u?(y)P', \\ \and \\\\ \and \\ \;\;\; u \in \meaningof{a}, \forall z.P'\{z/y\} \in \meaningof{E\{z/b\}}\}, \and \\ \meaningof{a!E} = \{ P \in \pi | P \equiv Q | x!\langle P' \rangle, x \in \meaningof{a} P' \in \meaningof{E}\} }
\end{mathpar}

\begin{mathpar}
 \inferrule* [lab=nominal] {} {\meaningof{\quotep{E}} = \{ \quotep{P} \in \quotep{\pi} | P \in \meaningof{E} \}, \and \meaningof{\quotep{P}} = \{ \quotep{Q} \in \quotep{\pi} | P \equiv Q \} \and \\ \meaningof{@\quotep{E}} = \{ P \in \pi | P \equiv @x, x \in \meaningof{E} \}}
\end{mathpar}

\begin{eqnarray*}
  \\
  \meaningof{-} : TS \to ST
\end{eqnarray*}

\begin{eqnarray*}
  \\
  L : TS \to ST
\end{eqnarray*}

\begin{eqnarray*}
  \\
  P \models E \iff P \in \meaningof{E}
\end{eqnarray*}

\begin{eqnarray*}
  P \approx_{L} Q \iff \forall E \in L. P \models E \iff Q \models E
\end{eqnarray*}

\begin{eqnarray*}
  P \approx_{K} Q
\end{eqnarray*}

\begin{eqnarray*}
  P \approx Q
\end{eqnarray*}

$\approx_{K} = \approx = \approx_{L}$

\subsubsection{Contextual duality}

Note that contexts extend the quotation operation to a family of
operations from processes to names. Given a context, $M$, we can
define a \emph{nominal context}, $\quotep{M}$ by $\quotep{M}[P] :=
\quotep{M[P]}$. To foreshadow what is to come we observe that these
operations enjoy a duality with processes very much like the duality
between vectors and maps from vectors to scalars.

Further, because the calculus is essentially higher-order, we have a
correspondence between contexts and processes. More specifically,
given a name $x$ and a context $M$ we can construct $M^{*}_{x}$ such
that 

\begin{mathpar}
  M^{*}_{x} | \lift{x}{P} \red M[P]
\end{mathpar}

namely,

\begin{mathpar}
  M^{*}_{x} := x?(u).M[\dropn{u}]
\end{mathpar}

The dependence of $M^{*}_{x}$ on a name makes it an abstraction, 

\begin{mathpar}
  M^{*} := (x)x?(u).M[\dropn{u}]
\end{mathpar}

\subsection{Additional notation}

It will sometimes be convenient to denote the process a name
quotes. We already have the notation $x = \quotep{P}$, but it will be
convenient to introduce an alternate notation, $\procn{x}$, when we
want to emphasize the connection to the use of the name. Note that, by
virtue of name equivalence, $\quotep{\procn{x}} \nameeq x$; so, the
notation is consistent with previous definitions.

Further, because names have structure it is possible to effect
substitutions on the basis of that structure. This means we need to
upgrade our notation for substitutions, which we accomplish by
adapting comprehension notation. Thus,

\begin{mathpar}
  P\{ y / x : x \in S \}
\end{mathpar}

is interpreted to mean the process derived from P by replacing (in a
capture-avoiding manner) each occurrence of $x$ in $S$ by $y$. For example,

\begin{mathpar}
  P\{ \quotep{\procn{x}|\procn{x}} / x : x \in \freenames{P} \}
\end{mathpar}

will replace each (occurrence) of a free name $x$ in $P$ by
$\quotep{\procn{x}|\procn{x}}$.

Also, we will avail ourselves of the notation $x^{L}$ and $x^{R}$ to
denote injections of a name into disjoint copies of the name
space. There are numerous ways to accomplish this. One example can be
found in \cite{MeredithR05}. This notation overloads to vectors of
names: $\vec{x}^{\pi} := (x_{i}^{\pi} \; : \; 0 \leq i < |\vec{x}| )$ where $\pi \in \{L,R\}$.

We also use $P^{\Box} := P|\Box$.

In \cite{MeredithR05} an interpretation of the new operator is
given. It turns out that there are several possible interpretations
all enjoying the requisite algebraic properties of the operator (see
\cite{milner91polyadicpi}). We will therefore make liberal use of
$(\nu\; \vec{x})P$.

% subsection the_syntax_and_semantics_of_the_notation_system (end)   

\section{Interpretation of QM}
\subsection{Supporting definitions}
\subsubsection{Multiplication}
\begin{mathpar}
  \quotep{Q} \cdot \quotep{R} := \quotep{Q|R}
  \and \\
  \quotep{Q} \cdot P := P\{ \quotep{Q|R} / \quotep{R} : \quotep{R} \in \freenames{P} \}
\end{mathpar}

\paragraph{Discussion}
The first line needs little explanation. The second line says that
each free name of the process is replaced with the multiplication of
that name by the scalar. Multiplication of a scalar (name) by a state
(process) results in a process all the names of which have been `moved
over' by parallel composition with the process the scalar
quotes. There is a subtlety that the bound names have to be
manipulated so that multiplied names aren't accidentally
captured. There are many ways to achieve this.

\begin{remark}\label{rem:multiplication_identities}
  The reader is invited to verify that for all $x,y,z \in \QProc$ and $P \in \Proc$
  \begin{mathpar}
    x \cdot \quotep{0} \equiv x 
    \and
    x \cdot y \equiv y \cdot x
    \and
    x \cdot (y \cdot z) \equiv (x \cdot y) \cdot z
    \and \\
    \quotep{0} \cdot P \equiv P
    \and \\
    x \cdot (y \cdot P) \equiv (x \cdot y) \cdot P
    \and \\
    x \cdot (P|Q) \equiv (x \cdot P) | (x \cdot Q)
    \and \\    
  \end{mathpar}
\end{remark}

\subsubsection{Tensor product}

We define a tensor product on processes by structural induction.

\paragraph{Tensor of sums} First note that all summations, including
$\pzero$ and sequence, can be written $\Sigma_{i} x_{i}.A_{i} +
\Sigma_{j} x_{j}.C_{j}$, where we have grouped input-guarded processes
together and output-guarded processes together.

Thus, we can define the tensor product of two summations, $N_{1}\otimes N_{2}$, where

\begin{mathpar}
  N_{1} := \Sigma_{i} x_{i}.A_{i} + \Sigma_{j} x_{j}.C_{j}
  \and
  N_{2} := \Sigma_{i'} y_{i'}.B_{i'} + \Sigma_{j'} y_{j'}.D_{j'} 
\end{mathpar}

as follows.

\begin{mathpar}
  \Sigma_{i} x_{i}.A_{i} + \Sigma_{j} x_{j}.C_{j} \otimes \Sigma_{i'}
  y_{i'}.B_{i'} + \Sigma_{j'} y_{j'}.D_{j'} 
  \and \\
  := \; \Sigma_{i} \Sigma_{i'} \quotep{\stackrel{\vee}{x_{i}}| \stackrel{\vee}{y_{i'}}}.(A_{i}\otimes B_{i'}) \; | \; \Sigma_{i'} \Sigma_{i} \quotep{\stackrel{\vee}{y_{i'}}|\stackrel{\vee}{x_{i}}}.(B_{i'}\otimes A_{i})
  \and
  \;\; | \;\; \Sigma_{j} \Sigma_{j'} \quotep{\stackrel{\vee}{x_{j}}|\stackrel{\vee}{y_{j'}}}.(A_{j}\otimes B_{j'}) \; | \; \Sigma_{j'} \Sigma_{j} \quotep{\stackrel{\vee}{y_{j'}}|\stackrel{\vee}{x_{j}}}.(B_{j'}\otimes A_{j})
\end{mathpar}

\begin{remark}
  Do we need to $x^{L}$ and $y^{R}$ for this construction as well?
\end{remark}

\paragraph{Tensor of parallel compositions} Next, we distribute tensor
over par.

\begin{mathpar}
  P_{1}|P_{2} \otimes Q_{1}|Q_{2} := (P_{1} \otimes Q_{1}) | (P_{1}
  \otimes Q_{2}) | (P_{2} \otimes Q_{1}) | (P_{2} \otimes Q_{2})
\end{mathpar}

\paragraph{Tensor with dropped names} We treat tensor of a
process with a dropped name as parallel composition.

\begin{mathpar}
  P \otimes \dropn{x} := P | \dropn{x}
\end{mathpar}

\paragraph{Tensor of agents}

Finally, we need to define tensor on agents. Note that the definition
of tensor on normal products only tensors inputs with inputs and
outputs with outputs. Thus, we only have to define the operation on
``homogeneous'' pairings.

\begin{mathpar}
  (\vec{x})P \otimes (\vec{y})Q
  \and \\
  := (x_{0}^{L}|y_{0}^{R},\ldots,x_{0}^{L}|y_{n}^{R},\ldots,x_{m}^{L}|y_{0}^{R},\ldots,x_{m}^{L}|y_{n}^R)(P\{ \vec{x}^{L}/\vec{x}\} \otimes Q \{ \vec{y}^{R}/\vec{y}\})
  \and \\
  \clift{\vec{P}} \otimes \clift{\vec{Q}}
  \and \\
  := \clift{P_{0}\otimes Q_{0},\ldots,P_{0}\otimes Q_{n},\ldots,P_{m}\otimes Q_{0},\ldots,P_{m}\otimes Q_{n}}
\end{mathpar}

\begin{remark}
  Observe that arities of tensored abstractions matches arities of
  tensored concretions if the original arities matched. Note also that
  the length of the arities corresponds to the increase in dimension
  we see in ordinary vector space tensor product.
\end{remark}

\begin{remark}
  Operationally, this definition distributes the tensor down to
  components ``linked'' by summation. Tensor over summation is
  intriguing in that it mixes names. Moreover, as a consequence of the
  way it mixes names we have the identities for all $x \in \QProc$ and
  $P,Q \in \Proc$

  \begin{mathpar}
    (x \cdot P) \otimes Q \equiv x \cdot (P \otimes Q) \equiv P \otimes (x \cdot Q)
    \and
    P \otimes \pzero \equiv P
  \end{mathpar}

  that the reader is invited to verify.
\end{remark}

\subsubsection{Annihilation}
\begin{mathpar}
  P^{\perp} := \{ Q | \forall R. P|Q \red^{*} R \Rightarrow R \red^{*} \pzero \}
  \and \\
  P^{\underline{\perp}} := \Sigma_{Q \in P^{\perp}} \quotep{Q}?(y).(\dropn{y}|Q) | \Sigma_{Q \in P^{\perp}} \quotep{Q}\clift{\Box}
\end{mathpar}

\paragraph{Discussion} The reader will note that $P^{\perp}$ is a
\emph{set} of processes, while $P^{\underline{\perp}}$ is a
\emph{context}. We call the set $P^{\perp}$ the \emph{annihilators} of
$P$. The parallel composition of a process in the annihilators of $P$
with $P$ will result in a process, the state space of which has all
paths eventually leading to $\pzero$. Execution may endure loops; but
under reasonable conditions of fairness (naturally guaranteed under
most notions of bisimulation) such a composite process cannot get
stuck in such a loop and will, eventually pop out and terminate.

The context $P^{\underline{\perp}}$ is ready and willing to ``take the
$P$ out of'' the process to which it is applied. It will effectively
transmit the code of the process to which it is applied to one of the
annihilators and run the process against it.

\subsubsection{Evaluation}
We fix $M$ a domain of fully abstract interpretation with an equality
coincident with bisimulation. We take $\meaningof{\cdot} : \Proc \to
M$ to be the map interpreting processes and $\nmeaningof{\cdot} : \M
\to Proc$ to be the map running the other way. Then we define

\begin{mathpar}
  \int P := \nmeaningof{\meaningof{P}}
\end{mathpar}

\paragraph{Discussion}
There are many fully abstract interpretations of Milner's
$\pi$-calculus. Any of them can be used as a basis for interpreting
the reflective calculus here. Equipped with such a domain it is
largely a matter of grinding through to check that the Yoneda
construction for the normalization-by-evaluation program can be
extended to this setting.

\begin{remark}
  The reader is invited to verify that $\int (P^{\underline{\perp}}[P]) = 0$.
\end{remark}

\subsection{Quantum mechanics}

Table \ref{tbl:core_qm_op_defns} gives the core operational definitions

\begin{table}[htp]\label{tbl:core_qm_op_defns}
  \center{
    \fbox{
      \begin{tabular}{c|c}
        quantum mechanics & process calculus \\
        \hline
        scalar & $x := \quotep{P}$ \\
        state vector & $\state{P} := P$ \\
        dual & $\state{P}^{*} := \event{P^{\underline{\perp}}} := \quotep{P^{\underline{\perp}}}[-]$ \\
        matrix & $ \Sigma_{\alpha} \state{P_{\alpha}}x_{\alpha}\event{Q_{\alpha}}$ \\
        vector addition & $\state{P} + \state{Q} := \state{P | Q}$ \\
        tensor product & $\state{P} \otimes \state{Q} := \state{P \otimes Q}$ \\
        inner product & $\innerprod{P}{Q} := \quotep{\int P^{\underline{\perp}}[Q]}$ \\
      \end{tabular}
    }
  }
  \caption{QM - operational definitions}
\end{table}

where

\begin{mathpar}
  \prmatrix{P}{Q} := \fprmatrix{P}{\quotep{\pzero}}{Q}
  \and
  \fprmatrix{P}{x}{Q} := (\state{P},x,\event{Q})
  \and
  (\fprmatrix{P}{x}{Q})(\state{R}) := x \cdot \innerprod{Q}{R} \cdot \state{P}
  \and
  (\fprmatrix{P}{x}{Q})(\event{R}) := x \cdot \innerprod{R}{P} \cdot \event{Q}
\end{mathpar}

\paragraph{Discussion}
As promised: vectors (aka states) are represented as processes; duals
as contextual duals; inner product definition should be compared with
standard inner product definition for ....

\begin{remark}
  Assuming $\int (P^{\underline{\perp}}[P]) = 0$, the reader is
  invited to verify that $(\fprmatrix{P}{x}{P})(\state{P}) = x \cdot \state{P}$.
\end{remark}

\begin{remark}
  The reader is invited to verify that $\innerprod{P}{Q}$ could
  equally well have been written $\quotep{\int \stackrel{\vee}{x}}$
  where $x = \event{P^{\underline{\perp}}}(Q)$.

  One of the motivations for this remark is that there is another way
  to factor these operations. We could package up evaluation in the dual:

  \begin{mathpar}
    \state{P}^{*} := \event{\int P^{\underline{\perp}}} := \quotep{\int P^{\underline{\perp}}}[-]
  \end{mathpar}

  and then have inner product defined by
  
  \begin{mathpar}
    \innerprod{P}{Q} := \event{P}(Q)
  \end{mathpar}

  Hopefully, experience with the calculations will provide guidance on
  the best factoring.
\end{remark}

\begin{remark}
  Assuming $\int (P^{\underline{\perp}}[P]) = 0$, the reader is
  invited to verify that $\forall P,Q. (\prmatrix{0}{Q})(\state{0}) =
  \state{0}$ and dually $(\prmatrix{P}{0})(\event{0}) = \event{0}$.
\end{remark}

\begin{remark}
  i'm a little worried that i don't (yet) have proper support for
  complex conjugacy. But, the observation above may give us a
  clue. According to Abramsky, it must be the case that the scalars
  are iso to the homset of the identity for the tensor -- which the
  observation above characterizes. 

  For now, we will simply bookmark the notion with $\overline{x}$.
\end{remark}

\subsubsection{Adjointness}

We need to give a definition of $(\cdot)^{\dagger}$ for matrices. The
obvious candidate definition is
\begin{mathpar}
(\Sigma_{\alpha}\fprmatrix{P_{\alpha}}{x_{\alpha}}{Q_{\alpha}})^{\dagger}
= \Sigma_{\alpha}\fprmatrix{(Q_{\alpha}^{\underline{\perp}})^{*}}{\overline{x}_{\alpha}}{P_{\alpha}^{\underline{\perp}}} 
\end{mathpar}

But, $(Q_{\alpha}^{\underline{\perp}})^{*}$ requires a name along
which to communicate the process to achieve the context application.

\subsubsection{Basis for a basis}
If processes label states and ``addition'' of states (a.k.a. vector
addition) is interpreted as parallel composition, what corresponds to
notions of linear independence and basis? Here, we recall that Yoshida
has developed a set of \emph{combinators} for an asynchronous verison
of Milner's $\pi$-calculus. These are a finite set of processes such
any process can be expressed as parallel composition of these
combinators together with liberal uses of the new operator and
replication. We can simply give a translation of these into the
present calculus and have reasonable expectation that the property
carries over. That is, that the resultant set allows to express all
processes via parallel composition. Note, however, that there is no
new operator or replication in this calculus. As a result, we expect
that the corresponding set is actually infinite. That is, we expect
that the space is actually infinite dimensional.

\begin{remark}
  The attentive reader may be a bit concerned. Certainly, the
  collection $S$, $K$ and $I$ is a finite set of
  combinators. Shouldn't we expect to see a finite set of combinators
  for an effectively equivalent system? i am very sympathetic to this
  critique and feel it warrants full attention. On the other hand, i
  also have in mind the following analogy. The natural numbers, as a
  monoid under addition, has exactly $1$ generator, while the natural
  numbers, as a monoid under multiplication, has countably many
  generators (the primes). We observe that the application of the
  lambda calculus is much less resource sensitive than the parallel
  composition of the $\pi$-calculus. Could it be the case that we have
  an analogy of the form
  
  \begin{mathpar}
    m + n : MN :: m*n : M|N
  \end{mathpar}

  giving a similar blow up in the set of ``primes''?  This is such a
  wonderful thought that, even if it's not true, i think it's worth
  writing down.
\end{remark}
 

\documentclass[12pt]{llncs}
%\documentclass{jktr}

\usepackage[pdftex]{hyperref}                   
\usepackage {listings}
\usepackage {mathpartir}
\usepackage{bcprules}
%\usepackage{listings}
                       
\usepackage{graphicx} 
%\usepackage[margins=2.5cm,nohead,nofoot]{geometry}
%\usepackage{geometry}
\usepackage{amsfonts}
\usepackage{amstext}
\usepackage{latexsym}
\usepackage{amssymb}
\usepackage{color}


%\include{myPreamble}
\include{qm2pi.local} 

%\ifpdf
%\usepackage[pdftex]{graphicx}
%\else
%\usepackage{graphicx}
%\fi

 % \ifpdf
%  \usepackage{pdfsync}
%  \if


%\title{Brief Article}
%\author{David F. Snyder}
%\author{L.G. Meredith}

%\address{Dept. of Math., Texas State University--San Marcos, San Marcos, TX 78666}
       
\pagestyle{empty}


\begin{document}

\lstset{language=[Objective]Caml,frame=shadowbox}

\input{qm2pi.front}

% section front matter (end)

\input{qm2pi.intro} 
 
% section introduction (end)

% \input{qm2pi.knotations} 

% section notation (end)

\input{qm2pi.process.calculi} 

% section concurrent_process_calculi_and_spatial_logics_ (end)
    
%\input{qm2pi.knots2pi} 

%\input{qm2pi.trefoil} 

%\input{qm2pi.mainthm} 

% subsection basic_interpretation (end)

%\input{qm2pi.rho.presentation} 
\subsection{The syntax and semantics of the notation system}\label{sub:the_syntax_and_semantics_of_the_notation_system} % (fold)

We now summarize a technical presentation of the calculus that
embodies our theory of dynamics. The typical presentation of such a
calculus follows the style of giving generators and relations on
them. The grammar, below, describing term constructors, freely
generates the set of processes, $\Proc$. This set is then quotiented
by a relation known as structural congruence and it is over this set
that the notion of dynamics is expressed. This presentation is
essentially that of \cite{MeredithR05} with the addition of
polyadicity and summation. For readability we have relegated some of
the technical subtleties to an appendix.

\subsubsection{Process grammar}\label{subsub:process_grammar}

\begin{mathpar}
  \inferrule* [lab=synchronization] {} {{M} \bc \pzero \;|\; x?F \;|\; x!C }
  \and
  \inferrule* [lab=abstraction] {} {{F} \bc (x)P}
  \and
  \inferrule* [lab=concretion] {} {{C} \bc \langle Q \rangle}
  \and
  \inferrule* [lab=process] {} {{P,Q} \bc M \;| \;P|Q \;|\; @{x}}
  \and
  \inferrule* [lab=name] {} {{x} \bc \quotep{P}}
\end{mathpar} 

Note that $\vec{x}$ (resp. $\vec{P}$) denotes a vector of names
(resp. processes) of length $|\vec{x}|$ (resp. $|\vec{P}|$). We adopt
the following useful abbreviations.

\begin{mathpar}
   x?(\vec{y}).P := x.(\vec{y})P \and  x\clift{\vec{P}} := x.\clift{\vec{P}}
   \and x!(y) := \lift{x}{\dropn{y}}
   \and \Pi_{i=0}^{n-1}P_i := P_0 | \ldots | P_{n-1}
\end{mathpar}

\subsubsection{Structural congruence}

\paragraph{Free and bound names and alpha-equivalence.} At the
core of structural equivalence is alpha-equivalence which identifies
process that are the same up to a change of variable. Formally, we
recognize the distinction between free and bound names. The free names
of a process, $\freenames{P}$, may be calculated recursively as
follows:

\begin{mathpar}
\freenames{\pzero} := \emptyset
  \and \\
  \freenames{x?(y).P} := \{ x \} \cup (\freenames{P} \setminus \{ y \})
  \and 
  \freenames{x!\langle P \rangle} := \{ x \} \cup \{ P \} 
  \and \\
  \freenames{P|Q} := \freenames{P} \cup \freenames{Q}
  \and \\
  \freenames{@{x}} := \{ x \}
\end{mathpar}

$\pi$
$\quotep{\pi}$

$\freenames{-} : \pi \to \mathcal{P}(\quotep{\pi})$

\begin{eqnarray*}
  \freenames{\pzero} & := & \emptyset \\
  \freenames{x?(y).P} & := & \{ x \} \cup (\freenames{P} \setminus \{ y \}) \\
  \freenames{x!\langle P \rangle} & := & \{ x \} \cup \{ P \} \\
  \freenames{P|Q} & := & \freenames{P} \cup \freenames{Q} \\
  \freenames{\dropn{x}} & := & \{ x \}
\end{eqnarray*}

The bound names of a process, $\boundnames{P}$, are those names occurring in $P$
that are not free. For example, in $x?(y).0$, the name $x$ is free, while $y$ is bound.

\begin{mathpar}
  \inferrule* [lab=monoidal-laws] {} { P|Q \equiv Q|P \and P|0 \equiv P \and P|(Q|R) \equiv (P|Q)|R }
\end{mathpar}

\begin{mathpar}
  \inferrule* [lab=alpha-equivalence] {} { (x)P \equiv (y)P\{y/x\} \and y \not\in \freenames{P} }
\end{mathpar}

\begin{definition}
Then two processes, $P,Q$, are alpha-equivalent if $P = Q\{\vec{y}/\vec{x}\}$ for
some $\vec{x} \in \boundnames{Q},\vec{y} \in \boundnames{P}$, where $Q\{\vec{y}/\vec{x}\}$
denotes the capture-avoiding substitution of $\vec{y}$ for $\vec{x}$ in $Q$.
\end{definition}

\begin{definition}
  The {\em structural congruence} \cite{SangiorgiWalker} , $\equiv$,
  between processes is the least congruence containing
  alpha-equivalence, satisfying the abelian monoid laws
  (associativity, commutativity and $\pzero$ as identity) for parallel
  composition $|$ and for summation $+$.
\end{definition}

\subsection{Name equivalence}

We take name equivalence, written $\nameeq$, to be the smallest
equivalence relation generated by the following rules.

\begin{mathpar}
\inferrule*[lab=Quote-drop]
{ }
{ \quotep{@{x}} \nameeq x }

\inferrule*[lab=Struct-equiv]
{ P \scong Q }
{ \quotep{P} \nameeq \quotep{Q} }
\end{mathpar}

The astute reader will have noticed that the mutual recursion of names
and processes imposes a mutual recursion on alpha-equivalence and
structural equivalence via name-equivalence. Fortunately, all of this
works out pleasantly and we may calculate in the natural way, free of
concern. The reader interested in the details is referred to the
appendix \ref{appendix:rho_details}.

\subsection{Substitution}

We use $\Proc$ for the set of processes, $\QProc$ for the set of
names, and $\id{\{}\vec{y} / \vec{x} \id{\}}$ to denote partial maps,
$s : \QProc \rightarrow \QProc$. A map, $s$ lifts, uniquely, to a map
on process terms, $\widehat{s} : \Proc \rightarrow \Proc$ by the
following equations.

\begin{mathpar}
  (0) \psubstp{Q}{P} := 0 \\
  (R \juxtap S) \psubstp{Q}{P}
  :=    
  (R)\psubstp{Q}{P} \juxtap (S) \psubstp{Q}{P} \\
  (x?(y).R) \psubstp{Q}{P}    
  :=    
  (x)\substp{Q}{P} (z)\concat( (R \psubstn{z}{y}) \psubstp{Q}{P} ) \\
  (\lift{x}{R}) \psubstp{Q}{P}  
  :=
  \lift{(x)\substp{Q}{P}}{ R \psubstp{Q}{P} } \\
%   (\dropn{x})  \psubstp{Q}{P}       
%   := 
%   \left\{ 
%     \begin{array}{ccc} 
%       \dropn{\quotep{Q}} & & x \nameeq \quotep{P} \\
%       \dropn{x} & & otherwise \\
%     \end{array}
%   \right. 
  (\dropn{x})  \psubstp{Q}{P}       
  := 
  \left\{ 
    \begin{array}{ccc} 
      Q & & x \nameeq \quotep{P} \\
      \dropn{x} & & otherwise \\
    \end{array}
  \right.
\end{mathpar}
 

where

\begin{eqnarray}
  (x)\id{\{} \lpquote Q \rpquote / \lpquote P \rpquote \id{\}}            = 
  \left\{ 
    \begin{array}{ccc}
      \lpquote Q \rpquote & & x \nameeq \lpquote P \rpquote \\
      x & & otherwise \\
    \end{array}
  \right. \nonumber
\end{eqnarray}

and $z$ is chosen distinct from $\quotep{P}$, $\quotep{Q}$, the free
names in $Q$, and all the names in $R$. Our $\alpha$-equivalence will
be built in the standard way from this substitution.

\begin{remark}\label{rem:no_self_referential_names}
  One consequence of these definitions is that $\forall P. \quotep{P}
  \not\in \freenames{P}$.
\end{remark}

\subsection{ Dynamic quote: an example }

Anticipating something of what's to come, consider applying the
substitution, $\widehat{\id{\{}u / z \id{\}}}$, to the following pair
of processes, $\lift{w}{y!(z)}$ and $w[ \lpquote y!(z) \rpquote ]$.

\begin{eqnarray}
	\lift{w}{y!(z)}\widehat{\id{\{}u / z \id{\}}}
		& = &
		\lift{w}{y!(u)} \nonumber\\
	w[ \lpquote y!(z) \rpquote ] \widehat{ \id{\{}u / z \id{\}} }
		& = &
		w[ \lpquote y!(z) \rpquote ] \nonumber
\end{eqnarray}

Because the body of the process between quotes is impervious to
substitution, we get radically different answers. In fact, by
examining the first process in an input context,
e.g. $x?(z).\lift{w}{y!(z)}$, we see that the process under the lift
operator may be shaped by prefixed inputs binding a name inside it. In
this sense, the lift operator will be seen as a way to dynamically
construct processes before reifying them as names.

Finally equipped with these standard features we can present the
dynamics of the calculus.

\subsubsection{Operational semantics} 

Finally, we introduce the computational dynamics. What marks these
algebras as distinct from other more traditionally studied algebraic
structures, e.g. vector spaces or polynomial rings, is the manner in
which dynamics is captured. In traditional structures, dynamics is typically
expressed through morphisms between such structures, as in linear maps
between vector spaces or morphisms between rings. In algebras
associated with the semantics of computation, the dynamics is
expressed as part of the algebraic structure itself, through a
reduction reduction relation typically denoted by $\red$. Below, we
give a recursive presentation of this relation for the calculus used
in the encoding.

$\red \subseteq \pi \times \pi$
$\red : \pi \to \mathcal{P}(\pi)$

\begin{mathpar}
  \inferrule* [lab=Comm] { \textsf{match}( x_{src}, x_{trgt} ) } { x_{trgt}?(y)P \; | \; x_{src}!\langle {Q} \rangle \red P\{\quotep{Q}/y}\} }
  \and \\
  \inferrule* [lab=Par] {{P} \red {P}'} {{{P} | {Q}} \red {{P}' | {Q}}}
  \and
  \inferrule* [lab=Equiv]{{{P} \scong {P}'} \andalso {{P}' \red {Q}'} \andalso {{Q}' \scong {Q}}}{{P} \red {Q}}
\end{mathpar}

\begin{eqnarray*}
  match_{\equiv} (\quotep{P},\quotep{Q}) & := & P \equiv Q \\
  match_{\dagger}(\quotep{P},\quotep{Q}) & := & \forall R. P|Q \red^{*} R => R \red^{*} 0 \\
  match_{K}(\quotep{P},\quotep{Q}) & := & K \mbox{ for some context } K
\end{eqnarray*}

$u?(x)P | u!\langle Q \rangle \red P\{\quotep{Q}/x\}$

%We write $\wred$ for $\red^*$, and $P\red$ if $\exists Q $ such that $ P \red Q$.
We write $P\red$ if $\exists Q $ such that $ P \red Q$ and $P\not\red$, otherwise.

\section{Replication}

As mentioned before, it is known that replication (and hence
recursion) can be implemented in a higher-order process algebra
\cite{SangiorgiWalker}. As our first example of calculation with the
machinery thus far presented we give the construction explicitly in
the {\rhoc}.

\begin{eqnarray}
	D_{x} & := & \prefix{x}{y}{(\binpar{\outputp{x}{y}}{@{y}})} \nonumber\\
	\bangp_{x}{P} & := & \binpar{{x}!\langle{\binpar{D_{x}}{P}}\rangle}{D_{x}} \nonumber
\end{eqnarray}

\begin{eqnarray}
	\bangp_{x}{P} & & \nonumber\\
	=
	& {x}!\langle{(\prefix{x}{y}{(\outputp{x}{y} | @{y})) | P}}\rangle 
	      | \prefix{x}{y}{(\outputp{x}{y} | @{y})} & \nonumber\\
	\red
	& (\outputp{x}{y} | @{y})\substn{\quotep{(\prefix{x}{y}{(@{y} | \outputp{x}{y})) | P}}}{y} & \nonumber\\
	=
	& \outputp{x}{\quotep{(\prefix{x}{y}{(\outputp{x}{y} | @{y})) | P}}}
	  | {(\prefix{x}{y}{(\outputp{x}{y} | @{y})) | P}} & \nonumber\\
	\red
	& \ldots & \nonumber\\
	\red^*
	& P | P | \ldots & \nonumber
\end{eqnarray}

Of course, this encoding, as an implementation, runs away, unfolding
$\bangp{P}$ eagerly. A lazier and more implementable replication
operator, restricted to input-guarded processes, may be obtained as follows.

\begin{eqnarray}
\bangp{\prefix{u}{v}{P}} 
	:= 
	\binpar{\lift{x}{\prefix{u}{v}{(\binpar{D(x)}{P})}}}{D(x)} \nonumber
\end{eqnarray}

\begin{remark}
  Note that the lazier definition still does not deal with summation
  or mixed summation (i.e. sums over input and output). The reader is
  invited to construct definitions of replication that deal with these
  features. 

  Further, the definitions are parameterized in a name, $x$. Can you,
  gentle reader, make a definition that eliminates this parameter and
  guarantees no accidental interaction between the replication
  machinery and the process being replicated -- i.e. no accidental
  sharing of names used by the process to get its work done and the
  name(s) used by the replication to effect copying. This latter
  revision of the definition of replication is crucial to obtaining
  the expected identity $!!P \sim !P$.
\end{remark}

\begin{remark}\label{rem:paradoxical_combinator}
  The reader familiar with the lambda calculus will have noticed the
  similarity between $D$ and the paradoxical combinator.

  [Ed. note: the existence of this seems to suggest we have to be more
  restrictive on the set of processes and names we admit if we are to
  support no-cloning.]
\end{remark}

\subsubsection{Bisimulation}

The computational dynamics gives rise to another kind of equivalence,
the equivalence of computational behavior. As previously mentioned
this is typically captured \emph{via} some form of bisimulation.

% The notion we use in this paper is weak barbed bisimulation
% \cite{milner91polyadicpi}.

The notion we use in this paper is derived from weak barbed
bisimulation \cite{milner91polyadicpi}. 

\begin{definition}
An \emph{observation relation}, $\downarrow_{\mathcal N}$, over a set
of names, $\mathcal N$, is the smallest relation satisfying the rules
below.

\infrule[Out-barb]{y \in {\mathcal N}, \; x \nameeq y}
		  {\outputp{x}{v} \downarrow_{\mathcal N} x}
\infrule[Par-barb]{\mbox{$P\downarrow_{\mathcal N} x$ or $Q\downarrow_{\mathcal N} x$}}
		  {\binpar{P}{Q} \downarrow_{\mathcal N} x}

We write $P \Downarrow_{\mathcal N} x$ if there is $Q$ such that 
$P \wred Q$ and $Q \downarrow_{\mathcal N} x$.
\end{definition}

\begin{definition}
%\label{def.bbisim}
An  ${\mathcal N}$-\emph{barbed bisimulation} over a set of names, ${\mathcal N}$, is a symmetric binary relation 
${\mathcal S}_{\mathcal N}$ between agents such that $P\rel{S}_{\mathcal N}Q$ implies:
\begin{enumerate}
\item If $P \red P'$ then $Q \wred Q'$ and $P'\rel{S}_{\mathcal N} Q'$.
\item If $P\downarrow_{\mathcal N} x$, then $Q\Downarrow_{\mathcal N} x$.
\end{enumerate}
$P$ is ${\mathcal N}$-barbed bisimilar to $Q$, written
$P \wbbisim_{\mathcal N} Q$, if $P \rel{S}_{\mathcal N} Q$ for some ${\mathcal N}$-barbed bisimulation ${\mathcal S}_{\mathcal N}$.
\end{definition}

$\mathcal{R} \subseteq \pi \times \pi$

$P \mathcal{R} Q => \forall P'. P \red P' \Rightarrow \exists Q'. Q \red Q', P' \mathcal{R} Q'$

$P \vdash x \Rightarrow Q \vdash x$

\begin{mathpar}
  \inferrule*[lab=Out-barb]{x \nameeq y}{{y}!\langle{Q}\rangle \vdash x}
  \and
  \inferrule*[lab=Par-barb]{\mbox{$P\vdash x$ or $Q\vdash x$}}{\binpar{P}{Q} \vdash x}
\end{mathpar}

\subsubsection{Contexts}

One of the principle advantages of computational calculi like the
$\pi$-calculus is a well-defined notion of context,
contextual-equivalence and a correlation between
contextual-equivalence and notions of bisimulation. The notion of
context allows the decomposition of a process into (sub-)process and
its syntactic environment, its context. Thus, a context may be
thought of as a process with a ``hole'' (written $\Box$) in it. The
application of a context $M$ to a process $P$, written $M[P]$, is
tantamount to filling the hole in $M$ with $P$. In this paper we do
not need the full weight of this theory, but do make use of the notion
of context in the proof the main theorem. 

\begin{mathpar}
  \inferrule* [lab=summation] {} {{M_{M},M_{N}} \bc \Box \;|\; x.M_{A} \;|\; M_{M}+M_{N}}
  \and
  \inferrule* [lab=agent] {} {{M_{A}} \bc (\vec{x})M_{P} \;| \; \clift{P_0,\ldots,M_{P},\ldots,P_N}}
  \and \\
  \inferrule* [lab=process] {} {{M_{P}} \bc M_{N} \;| \;P|M_{P} }
\end{mathpar} 

\begin{mathpar}
  \inferrule* [lab=sychronization] {} {M_{N} \bc \Box \;|\; x?M_{F} \;|\; x!M_{C}}
  \and
  \inferrule* [lab=abstraction] {} {{M_{F}} \bc (x)M_{P} }
  \and
  \inferrule* [lab=concretion] {} {{M_{C}} \bc \langle M_{P} \rangle }
  \and \\
  \inferrule* [lab=process] {} {{M_{P}} \bc M_{N} \;| \;P|M_{P} }
\end{mathpar}

\begin{definition}[contextual application] Given a context $M$, and
  process $P$, we define the \emph{contextual application}, $M[P] :=
  M\{P/\Box\}$. That is, the contextual application of M to P is the
  substitution of $P$ for $\Box$ in $M$.
\end{definition}

$\meaningof{-} : L \to \mathcal{P}(\pi)$

\begin{mathpar}
  \inferrule* [lab=collection] {} {\meaningof{true} = \pi, \and \meaningof{~E} = \pi \setminus \meaningof{E}, \and \meaningof{E_{1} \& E_{2}} = \meaningof{E_{1}} \cap \meaningof{E_{2}}}
\end{mathpar}

\begin{mathpar}
  \inferrule* [lab=structure] {} {\meaningof{0} = \{ P \in \pi | P \equiv 0 \}, \and \\ \meaningof{E_1 | E_2} = \{ P \in \pi | P \equiv P_{1} | P_{2}, P_{1} \in \meaningof{E_{1}}, P_{2} \in \meaningof{E_2}\} }
\end{mathpar}

\begin{mathpar}
 \inferrule* [lab=behavior] {} {\meaningof{\langle a?b \rangle E} = \{ P \in \pi | P \equiv Q | u?(y)P', \\ \and \\\\ \and \\ \;\;\; u \in \meaningof{a}, \forall z.P'\{z/y\} \in \meaningof{E\{z/b\}}\}, \and \\ \meaningof{a!E} = \{ P \in \pi | P \equiv Q | x!\langle P' \rangle, x \in \meaningof{a} P' \in \meaningof{E}\} }
\end{mathpar}

\begin{mathpar}
 \inferrule* [lab=nominal] {} {\meaningof{\quotep{E}} = \{ \quotep{P} \in \quotep{\pi} | P \in \meaningof{E} \}, \and \meaningof{\quotep{P}} = \{ \quotep{Q} \in \quotep{\pi} | P \equiv Q \} \and \\ \meaningof{@\quotep{E}} = \{ P \in \pi | P \equiv @x, x \in \meaningof{E} \}}
\end{mathpar}

\begin{eqnarray*}
  \\
  \meaningof{-} : TS \to ST
\end{eqnarray*}

\begin{eqnarray*}
  \\
  L : TS \to ST
\end{eqnarray*}

\begin{eqnarray*}
  \\
  P \models E \iff P \in \meaningof{E}
\end{eqnarray*}

\begin{eqnarray*}
  P \approx_{L} Q \iff \forall E \in L. P \models E \iff Q \models E
\end{eqnarray*}

\begin{eqnarray*}
  P \approx_{K} Q
\end{eqnarray*}

\begin{eqnarray*}
  P \approx Q
\end{eqnarray*}

$\approx_{K} = \approx = \approx_{L}$

\subsubsection{Contextual duality}

Note that contexts extend the quotation operation to a family of
operations from processes to names. Given a context, $M$, we can
define a \emph{nominal context}, $\quotep{M}$ by $\quotep{M}[P] :=
\quotep{M[P]}$. To foreshadow what is to come we observe that these
operations enjoy a duality with processes very much like the duality
between vectors and maps from vectors to scalars.

Further, because the calculus is essentially higher-order, we have a
correspondence between contexts and processes. More specifically,
given a name $x$ and a context $M$ we can construct $M^{*}_{x}$ such
that 

\begin{mathpar}
  M^{*}_{x} | \lift{x}{P} \red M[P]
\end{mathpar}

namely,

\begin{mathpar}
  M^{*}_{x} := x?(u).M[\dropn{u}]
\end{mathpar}

The dependence of $M^{*}_{x}$ on a name makes it an abstraction, 

\begin{mathpar}
  M^{*} := (x)x?(u).M[\dropn{u}]
\end{mathpar}

\subsection{Additional notation}

It will sometimes be convenient to denote the process a name
quotes. We already have the notation $x = \quotep{P}$, but it will be
convenient to introduce an alternate notation, $\procn{x}$, when we
want to emphasize the connection to the use of the name. Note that, by
virtue of name equivalence, $\quotep{\procn{x}} \nameeq x$; so, the
notation is consistent with previous definitions.

Further, because names have structure it is possible to effect
substitutions on the basis of that structure. This means we need to
upgrade our notation for substitutions, which we accomplish by
adapting comprehension notation. Thus,

\begin{mathpar}
  P\{ y / x : x \in S \}
\end{mathpar}

is interpreted to mean the process derived from P by replacing (in a
capture-avoiding manner) each occurrence of $x$ in $S$ by $y$. For example,

\begin{mathpar}
  P\{ \quotep{\procn{x}|\procn{x}} / x : x \in \freenames{P} \}
\end{mathpar}

will replace each (occurrence) of a free name $x$ in $P$ by
$\quotep{\procn{x}|\procn{x}}$.

Also, we will avail ourselves of the notation $x^{L}$ and $x^{R}$ to
denote injections of a name into disjoint copies of the name
space. There are numerous ways to accomplish this. One example can be
found in \cite{MeredithR05}. This notation overloads to vectors of
names: $\vec{x}^{\pi} := (x_{i}^{\pi} \; : \; 0 \leq i < |\vec{x}| )$ where $\pi \in \{L,R\}$.

We also use $P^{\Box} := P|\Box$.

In \cite{MeredithR05} an interpretation of the new operator is
given. It turns out that there are several possible interpretations
all enjoying the requisite algebraic properties of the operator (see
\cite{milner91polyadicpi}). We will therefore make liberal use of
$(\nu\; \vec{x})P$.

% subsection the_syntax_and_semantics_of_the_notation_system (end)   

\input{qm2pi.qmops} 

\input{qm2pi.sterngerlach} 

\input{qm2pi.metric} 

% section concurrent_process_calculi (end)

%\input{qm2pi.proofsketch}

% section proof sketch (end)

%\input{qm2pi.slviaknots} 

% section spatial logic via knots (end)

\input{qm2pi.conclusion}

% section conclusion (end)

%\input{qm2pi.dtcodes} 

% section wiring algorithm (end)

\input{qm2pi.ack} 

% section acknowledgments (end)

\newpage


\bibliographystyle{plain}   
\bibliography{../../biblios/main.bib}

\input{qm2pi.rhodetails}

\end{document}

 

\documentclass[12pt]{llncs}
%\documentclass{jktr}

\usepackage[pdftex]{hyperref}                   
\usepackage {listings}
\usepackage {mathpartir}
\usepackage{bcprules}
%\usepackage{listings}
                       
\usepackage{graphicx} 
%\usepackage[margins=2.5cm,nohead,nofoot]{geometry}
%\usepackage{geometry}
\usepackage{amsfonts}
\usepackage{amstext}
\usepackage{latexsym}
\usepackage{amssymb}
\usepackage{color}


%\include{myPreamble}
\include{qm2pi.local} 

%\ifpdf
%\usepackage[pdftex]{graphicx}
%\else
%\usepackage{graphicx}
%\fi

 % \ifpdf
%  \usepackage{pdfsync}
%  \if


%\title{Brief Article}
%\author{David F. Snyder}
%\author{L.G. Meredith}

%\address{Dept. of Math., Texas State University--San Marcos, San Marcos, TX 78666}
       
\pagestyle{empty}


\begin{document}

\lstset{language=[Objective]Caml,frame=shadowbox}

\input{qm2pi.front}

% section front matter (end)

\input{qm2pi.intro} 
 
% section introduction (end)

% \input{qm2pi.knotations} 

% section notation (end)

\input{qm2pi.process.calculi} 

% section concurrent_process_calculi_and_spatial_logics_ (end)
    
%\input{qm2pi.knots2pi} 

%\input{qm2pi.trefoil} 

%\input{qm2pi.mainthm} 

% subsection basic_interpretation (end)

%\input{qm2pi.rho.presentation} 
\subsection{The syntax and semantics of the notation system}\label{sub:the_syntax_and_semantics_of_the_notation_system} % (fold)

We now summarize a technical presentation of the calculus that
embodies our theory of dynamics. The typical presentation of such a
calculus follows the style of giving generators and relations on
them. The grammar, below, describing term constructors, freely
generates the set of processes, $\Proc$. This set is then quotiented
by a relation known as structural congruence and it is over this set
that the notion of dynamics is expressed. This presentation is
essentially that of \cite{MeredithR05} with the addition of
polyadicity and summation. For readability we have relegated some of
the technical subtleties to an appendix.

\subsubsection{Process grammar}\label{subsub:process_grammar}

\begin{mathpar}
  \inferrule* [lab=synchronization] {} {{M} \bc \pzero \;|\; x?F \;|\; x!C }
  \and
  \inferrule* [lab=abstraction] {} {{F} \bc (x)P}
  \and
  \inferrule* [lab=concretion] {} {{C} \bc \langle Q \rangle}
  \and
  \inferrule* [lab=process] {} {{P,Q} \bc M \;| \;P|Q \;|\; @{x}}
  \and
  \inferrule* [lab=name] {} {{x} \bc \quotep{P}}
\end{mathpar} 

Note that $\vec{x}$ (resp. $\vec{P}$) denotes a vector of names
(resp. processes) of length $|\vec{x}|$ (resp. $|\vec{P}|$). We adopt
the following useful abbreviations.

\begin{mathpar}
   x?(\vec{y}).P := x.(\vec{y})P \and  x\clift{\vec{P}} := x.\clift{\vec{P}}
   \and x!(y) := \lift{x}{\dropn{y}}
   \and \Pi_{i=0}^{n-1}P_i := P_0 | \ldots | P_{n-1}
\end{mathpar}

\subsubsection{Structural congruence}

\paragraph{Free and bound names and alpha-equivalence.} At the
core of structural equivalence is alpha-equivalence which identifies
process that are the same up to a change of variable. Formally, we
recognize the distinction between free and bound names. The free names
of a process, $\freenames{P}$, may be calculated recursively as
follows:

\begin{mathpar}
\freenames{\pzero} := \emptyset
  \and \\
  \freenames{x?(y).P} := \{ x \} \cup (\freenames{P} \setminus \{ y \})
  \and 
  \freenames{x!\langle P \rangle} := \{ x \} \cup \{ P \} 
  \and \\
  \freenames{P|Q} := \freenames{P} \cup \freenames{Q}
  \and \\
  \freenames{@{x}} := \{ x \}
\end{mathpar}

$\pi$
$\quotep{\pi}$

$\freenames{-} : \pi \to \mathcal{P}(\quotep{\pi})$

\begin{eqnarray*}
  \freenames{\pzero} & := & \emptyset \\
  \freenames{x?(y).P} & := & \{ x \} \cup (\freenames{P} \setminus \{ y \}) \\
  \freenames{x!\langle P \rangle} & := & \{ x \} \cup \{ P \} \\
  \freenames{P|Q} & := & \freenames{P} \cup \freenames{Q} \\
  \freenames{\dropn{x}} & := & \{ x \}
\end{eqnarray*}

The bound names of a process, $\boundnames{P}$, are those names occurring in $P$
that are not free. For example, in $x?(y).0$, the name $x$ is free, while $y$ is bound.

\begin{mathpar}
  \inferrule* [lab=monoidal-laws] {} { P|Q \equiv Q|P \and P|0 \equiv P \and P|(Q|R) \equiv (P|Q)|R }
\end{mathpar}

\begin{mathpar}
  \inferrule* [lab=alpha-equivalence] {} { (x)P \equiv (y)P\{y/x\} \and y \not\in \freenames{P} }
\end{mathpar}

\begin{definition}
Then two processes, $P,Q$, are alpha-equivalent if $P = Q\{\vec{y}/\vec{x}\}$ for
some $\vec{x} \in \boundnames{Q},\vec{y} \in \boundnames{P}$, where $Q\{\vec{y}/\vec{x}\}$
denotes the capture-avoiding substitution of $\vec{y}$ for $\vec{x}$ in $Q$.
\end{definition}

\begin{definition}
  The {\em structural congruence} \cite{SangiorgiWalker} , $\equiv$,
  between processes is the least congruence containing
  alpha-equivalence, satisfying the abelian monoid laws
  (associativity, commutativity and $\pzero$ as identity) for parallel
  composition $|$ and for summation $+$.
\end{definition}

\subsection{Name equivalence}

We take name equivalence, written $\nameeq$, to be the smallest
equivalence relation generated by the following rules.

\begin{mathpar}
\inferrule*[lab=Quote-drop]
{ }
{ \quotep{@{x}} \nameeq x }

\inferrule*[lab=Struct-equiv]
{ P \scong Q }
{ \quotep{P} \nameeq \quotep{Q} }
\end{mathpar}

The astute reader will have noticed that the mutual recursion of names
and processes imposes a mutual recursion on alpha-equivalence and
structural equivalence via name-equivalence. Fortunately, all of this
works out pleasantly and we may calculate in the natural way, free of
concern. The reader interested in the details is referred to the
appendix \ref{appendix:rho_details}.

\subsection{Substitution}

We use $\Proc$ for the set of processes, $\QProc$ for the set of
names, and $\id{\{}\vec{y} / \vec{x} \id{\}}$ to denote partial maps,
$s : \QProc \rightarrow \QProc$. A map, $s$ lifts, uniquely, to a map
on process terms, $\widehat{s} : \Proc \rightarrow \Proc$ by the
following equations.

\begin{mathpar}
  (0) \psubstp{Q}{P} := 0 \\
  (R \juxtap S) \psubstp{Q}{P}
  :=    
  (R)\psubstp{Q}{P} \juxtap (S) \psubstp{Q}{P} \\
  (x?(y).R) \psubstp{Q}{P}    
  :=    
  (x)\substp{Q}{P} (z)\concat( (R \psubstn{z}{y}) \psubstp{Q}{P} ) \\
  (\lift{x}{R}) \psubstp{Q}{P}  
  :=
  \lift{(x)\substp{Q}{P}}{ R \psubstp{Q}{P} } \\
%   (\dropn{x})  \psubstp{Q}{P}       
%   := 
%   \left\{ 
%     \begin{array}{ccc} 
%       \dropn{\quotep{Q}} & & x \nameeq \quotep{P} \\
%       \dropn{x} & & otherwise \\
%     \end{array}
%   \right. 
  (\dropn{x})  \psubstp{Q}{P}       
  := 
  \left\{ 
    \begin{array}{ccc} 
      Q & & x \nameeq \quotep{P} \\
      \dropn{x} & & otherwise \\
    \end{array}
  \right.
\end{mathpar}
 

where

\begin{eqnarray}
  (x)\id{\{} \lpquote Q \rpquote / \lpquote P \rpquote \id{\}}            = 
  \left\{ 
    \begin{array}{ccc}
      \lpquote Q \rpquote & & x \nameeq \lpquote P \rpquote \\
      x & & otherwise \\
    \end{array}
  \right. \nonumber
\end{eqnarray}

and $z$ is chosen distinct from $\quotep{P}$, $\quotep{Q}$, the free
names in $Q$, and all the names in $R$. Our $\alpha$-equivalence will
be built in the standard way from this substitution.

\begin{remark}\label{rem:no_self_referential_names}
  One consequence of these definitions is that $\forall P. \quotep{P}
  \not\in \freenames{P}$.
\end{remark}

\subsection{ Dynamic quote: an example }

Anticipating something of what's to come, consider applying the
substitution, $\widehat{\id{\{}u / z \id{\}}}$, to the following pair
of processes, $\lift{w}{y!(z)}$ and $w[ \lpquote y!(z) \rpquote ]$.

\begin{eqnarray}
	\lift{w}{y!(z)}\widehat{\id{\{}u / z \id{\}}}
		& = &
		\lift{w}{y!(u)} \nonumber\\
	w[ \lpquote y!(z) \rpquote ] \widehat{ \id{\{}u / z \id{\}} }
		& = &
		w[ \lpquote y!(z) \rpquote ] \nonumber
\end{eqnarray}

Because the body of the process between quotes is impervious to
substitution, we get radically different answers. In fact, by
examining the first process in an input context,
e.g. $x?(z).\lift{w}{y!(z)}$, we see that the process under the lift
operator may be shaped by prefixed inputs binding a name inside it. In
this sense, the lift operator will be seen as a way to dynamically
construct processes before reifying them as names.

Finally equipped with these standard features we can present the
dynamics of the calculus.

\subsubsection{Operational semantics} 

Finally, we introduce the computational dynamics. What marks these
algebras as distinct from other more traditionally studied algebraic
structures, e.g. vector spaces or polynomial rings, is the manner in
which dynamics is captured. In traditional structures, dynamics is typically
expressed through morphisms between such structures, as in linear maps
between vector spaces or morphisms between rings. In algebras
associated with the semantics of computation, the dynamics is
expressed as part of the algebraic structure itself, through a
reduction reduction relation typically denoted by $\red$. Below, we
give a recursive presentation of this relation for the calculus used
in the encoding.

$\red \subseteq \pi \times \pi$
$\red : \pi \to \mathcal{P}(\pi)$

\begin{mathpar}
  \inferrule* [lab=Comm] { \textsf{match}( x_{src}, x_{trgt} ) } { x_{trgt}?(y)P \; | \; x_{src}!\langle {Q} \rangle \red P\{\quotep{Q}/y}\} }
  \and \\
  \inferrule* [lab=Par] {{P} \red {P}'} {{{P} | {Q}} \red {{P}' | {Q}}}
  \and
  \inferrule* [lab=Equiv]{{{P} \scong {P}'} \andalso {{P}' \red {Q}'} \andalso {{Q}' \scong {Q}}}{{P} \red {Q}}
\end{mathpar}

\begin{eqnarray*}
  match_{\equiv} (\quotep{P},\quotep{Q}) & := & P \equiv Q \\
  match_{\dagger}(\quotep{P},\quotep{Q}) & := & \forall R. P|Q \red^{*} R => R \red^{*} 0 \\
  match_{K}(\quotep{P},\quotep{Q}) & := & K \mbox{ for some context } K
\end{eqnarray*}

$u?(x)P | u!\langle Q \rangle \red P\{\quotep{Q}/x\}$

%We write $\wred$ for $\red^*$, and $P\red$ if $\exists Q $ such that $ P \red Q$.
We write $P\red$ if $\exists Q $ such that $ P \red Q$ and $P\not\red$, otherwise.

\section{Replication}

As mentioned before, it is known that replication (and hence
recursion) can be implemented in a higher-order process algebra
\cite{SangiorgiWalker}. As our first example of calculation with the
machinery thus far presented we give the construction explicitly in
the {\rhoc}.

\begin{eqnarray}
	D_{x} & := & \prefix{x}{y}{(\binpar{\outputp{x}{y}}{@{y}})} \nonumber\\
	\bangp_{x}{P} & := & \binpar{{x}!\langle{\binpar{D_{x}}{P}}\rangle}{D_{x}} \nonumber
\end{eqnarray}

\begin{eqnarray}
	\bangp_{x}{P} & & \nonumber\\
	=
	& {x}!\langle{(\prefix{x}{y}{(\outputp{x}{y} | @{y})) | P}}\rangle 
	      | \prefix{x}{y}{(\outputp{x}{y} | @{y})} & \nonumber\\
	\red
	& (\outputp{x}{y} | @{y})\substn{\quotep{(\prefix{x}{y}{(@{y} | \outputp{x}{y})) | P}}}{y} & \nonumber\\
	=
	& \outputp{x}{\quotep{(\prefix{x}{y}{(\outputp{x}{y} | @{y})) | P}}}
	  | {(\prefix{x}{y}{(\outputp{x}{y} | @{y})) | P}} & \nonumber\\
	\red
	& \ldots & \nonumber\\
	\red^*
	& P | P | \ldots & \nonumber
\end{eqnarray}

Of course, this encoding, as an implementation, runs away, unfolding
$\bangp{P}$ eagerly. A lazier and more implementable replication
operator, restricted to input-guarded processes, may be obtained as follows.

\begin{eqnarray}
\bangp{\prefix{u}{v}{P}} 
	:= 
	\binpar{\lift{x}{\prefix{u}{v}{(\binpar{D(x)}{P})}}}{D(x)} \nonumber
\end{eqnarray}

\begin{remark}
  Note that the lazier definition still does not deal with summation
  or mixed summation (i.e. sums over input and output). The reader is
  invited to construct definitions of replication that deal with these
  features. 

  Further, the definitions are parameterized in a name, $x$. Can you,
  gentle reader, make a definition that eliminates this parameter and
  guarantees no accidental interaction between the replication
  machinery and the process being replicated -- i.e. no accidental
  sharing of names used by the process to get its work done and the
  name(s) used by the replication to effect copying. This latter
  revision of the definition of replication is crucial to obtaining
  the expected identity $!!P \sim !P$.
\end{remark}

\begin{remark}\label{rem:paradoxical_combinator}
  The reader familiar with the lambda calculus will have noticed the
  similarity between $D$ and the paradoxical combinator.

  [Ed. note: the existence of this seems to suggest we have to be more
  restrictive on the set of processes and names we admit if we are to
  support no-cloning.]
\end{remark}

\subsubsection{Bisimulation}

The computational dynamics gives rise to another kind of equivalence,
the equivalence of computational behavior. As previously mentioned
this is typically captured \emph{via} some form of bisimulation.

% The notion we use in this paper is weak barbed bisimulation
% \cite{milner91polyadicpi}.

The notion we use in this paper is derived from weak barbed
bisimulation \cite{milner91polyadicpi}. 

\begin{definition}
An \emph{observation relation}, $\downarrow_{\mathcal N}$, over a set
of names, $\mathcal N$, is the smallest relation satisfying the rules
below.

\infrule[Out-barb]{y \in {\mathcal N}, \; x \nameeq y}
		  {\outputp{x}{v} \downarrow_{\mathcal N} x}
\infrule[Par-barb]{\mbox{$P\downarrow_{\mathcal N} x$ or $Q\downarrow_{\mathcal N} x$}}
		  {\binpar{P}{Q} \downarrow_{\mathcal N} x}

We write $P \Downarrow_{\mathcal N} x$ if there is $Q$ such that 
$P \wred Q$ and $Q \downarrow_{\mathcal N} x$.
\end{definition}

\begin{definition}
%\label{def.bbisim}
An  ${\mathcal N}$-\emph{barbed bisimulation} over a set of names, ${\mathcal N}$, is a symmetric binary relation 
${\mathcal S}_{\mathcal N}$ between agents such that $P\rel{S}_{\mathcal N}Q$ implies:
\begin{enumerate}
\item If $P \red P'$ then $Q \wred Q'$ and $P'\rel{S}_{\mathcal N} Q'$.
\item If $P\downarrow_{\mathcal N} x$, then $Q\Downarrow_{\mathcal N} x$.
\end{enumerate}
$P$ is ${\mathcal N}$-barbed bisimilar to $Q$, written
$P \wbbisim_{\mathcal N} Q$, if $P \rel{S}_{\mathcal N} Q$ for some ${\mathcal N}$-barbed bisimulation ${\mathcal S}_{\mathcal N}$.
\end{definition}

$\mathcal{R} \subseteq \pi \times \pi$

$P \mathcal{R} Q => \forall P'. P \red P' \Rightarrow \exists Q'. Q \red Q', P' \mathcal{R} Q'$

$P \vdash x \Rightarrow Q \vdash x$

\begin{mathpar}
  \inferrule*[lab=Out-barb]{x \nameeq y}{{y}!\langle{Q}\rangle \vdash x}
  \and
  \inferrule*[lab=Par-barb]{\mbox{$P\vdash x$ or $Q\vdash x$}}{\binpar{P}{Q} \vdash x}
\end{mathpar}

\subsubsection{Contexts}

One of the principle advantages of computational calculi like the
$\pi$-calculus is a well-defined notion of context,
contextual-equivalence and a correlation between
contextual-equivalence and notions of bisimulation. The notion of
context allows the decomposition of a process into (sub-)process and
its syntactic environment, its context. Thus, a context may be
thought of as a process with a ``hole'' (written $\Box$) in it. The
application of a context $M$ to a process $P$, written $M[P]$, is
tantamount to filling the hole in $M$ with $P$. In this paper we do
not need the full weight of this theory, but do make use of the notion
of context in the proof the main theorem. 

\begin{mathpar}
  \inferrule* [lab=summation] {} {{M_{M},M_{N}} \bc \Box \;|\; x.M_{A} \;|\; M_{M}+M_{N}}
  \and
  \inferrule* [lab=agent] {} {{M_{A}} \bc (\vec{x})M_{P} \;| \; \clift{P_0,\ldots,M_{P},\ldots,P_N}}
  \and \\
  \inferrule* [lab=process] {} {{M_{P}} \bc M_{N} \;| \;P|M_{P} }
\end{mathpar} 

\begin{mathpar}
  \inferrule* [lab=sychronization] {} {M_{N} \bc \Box \;|\; x?M_{F} \;|\; x!M_{C}}
  \and
  \inferrule* [lab=abstraction] {} {{M_{F}} \bc (x)M_{P} }
  \and
  \inferrule* [lab=concretion] {} {{M_{C}} \bc \langle M_{P} \rangle }
  \and \\
  \inferrule* [lab=process] {} {{M_{P}} \bc M_{N} \;| \;P|M_{P} }
\end{mathpar}

\begin{definition}[contextual application] Given a context $M$, and
  process $P$, we define the \emph{contextual application}, $M[P] :=
  M\{P/\Box\}$. That is, the contextual application of M to P is the
  substitution of $P$ for $\Box$ in $M$.
\end{definition}

$\meaningof{-} : L \to \mathcal{P}(\pi)$

\begin{mathpar}
  \inferrule* [lab=collection] {} {\meaningof{true} = \pi, \and \meaningof{~E} = \pi \setminus \meaningof{E}, \and \meaningof{E_{1} \& E_{2}} = \meaningof{E_{1}} \cap \meaningof{E_{2}}}
\end{mathpar}

\begin{mathpar}
  \inferrule* [lab=structure] {} {\meaningof{0} = \{ P \in \pi | P \equiv 0 \}, \and \\ \meaningof{E_1 | E_2} = \{ P \in \pi | P \equiv P_{1} | P_{2}, P_{1} \in \meaningof{E_{1}}, P_{2} \in \meaningof{E_2}\} }
\end{mathpar}

\begin{mathpar}
 \inferrule* [lab=behavior] {} {\meaningof{\langle a?b \rangle E} = \{ P \in \pi | P \equiv Q | u?(y)P', \\ \and \\\\ \and \\ \;\;\; u \in \meaningof{a}, \forall z.P'\{z/y\} \in \meaningof{E\{z/b\}}\}, \and \\ \meaningof{a!E} = \{ P \in \pi | P \equiv Q | x!\langle P' \rangle, x \in \meaningof{a} P' \in \meaningof{E}\} }
\end{mathpar}

\begin{mathpar}
 \inferrule* [lab=nominal] {} {\meaningof{\quotep{E}} = \{ \quotep{P} \in \quotep{\pi} | P \in \meaningof{E} \}, \and \meaningof{\quotep{P}} = \{ \quotep{Q} \in \quotep{\pi} | P \equiv Q \} \and \\ \meaningof{@\quotep{E}} = \{ P \in \pi | P \equiv @x, x \in \meaningof{E} \}}
\end{mathpar}

\begin{eqnarray*}
  \\
  \meaningof{-} : TS \to ST
\end{eqnarray*}

\begin{eqnarray*}
  \\
  L : TS \to ST
\end{eqnarray*}

\begin{eqnarray*}
  \\
  P \models E \iff P \in \meaningof{E}
\end{eqnarray*}

\begin{eqnarray*}
  P \approx_{L} Q \iff \forall E \in L. P \models E \iff Q \models E
\end{eqnarray*}

\begin{eqnarray*}
  P \approx_{K} Q
\end{eqnarray*}

\begin{eqnarray*}
  P \approx Q
\end{eqnarray*}

$\approx_{K} = \approx = \approx_{L}$

\subsubsection{Contextual duality}

Note that contexts extend the quotation operation to a family of
operations from processes to names. Given a context, $M$, we can
define a \emph{nominal context}, $\quotep{M}$ by $\quotep{M}[P] :=
\quotep{M[P]}$. To foreshadow what is to come we observe that these
operations enjoy a duality with processes very much like the duality
between vectors and maps from vectors to scalars.

Further, because the calculus is essentially higher-order, we have a
correspondence between contexts and processes. More specifically,
given a name $x$ and a context $M$ we can construct $M^{*}_{x}$ such
that 

\begin{mathpar}
  M^{*}_{x} | \lift{x}{P} \red M[P]
\end{mathpar}

namely,

\begin{mathpar}
  M^{*}_{x} := x?(u).M[\dropn{u}]
\end{mathpar}

The dependence of $M^{*}_{x}$ on a name makes it an abstraction, 

\begin{mathpar}
  M^{*} := (x)x?(u).M[\dropn{u}]
\end{mathpar}

\subsection{Additional notation}

It will sometimes be convenient to denote the process a name
quotes. We already have the notation $x = \quotep{P}$, but it will be
convenient to introduce an alternate notation, $\procn{x}$, when we
want to emphasize the connection to the use of the name. Note that, by
virtue of name equivalence, $\quotep{\procn{x}} \nameeq x$; so, the
notation is consistent with previous definitions.

Further, because names have structure it is possible to effect
substitutions on the basis of that structure. This means we need to
upgrade our notation for substitutions, which we accomplish by
adapting comprehension notation. Thus,

\begin{mathpar}
  P\{ y / x : x \in S \}
\end{mathpar}

is interpreted to mean the process derived from P by replacing (in a
capture-avoiding manner) each occurrence of $x$ in $S$ by $y$. For example,

\begin{mathpar}
  P\{ \quotep{\procn{x}|\procn{x}} / x : x \in \freenames{P} \}
\end{mathpar}

will replace each (occurrence) of a free name $x$ in $P$ by
$\quotep{\procn{x}|\procn{x}}$.

Also, we will avail ourselves of the notation $x^{L}$ and $x^{R}$ to
denote injections of a name into disjoint copies of the name
space. There are numerous ways to accomplish this. One example can be
found in \cite{MeredithR05}. This notation overloads to vectors of
names: $\vec{x}^{\pi} := (x_{i}^{\pi} \; : \; 0 \leq i < |\vec{x}| )$ where $\pi \in \{L,R\}$.

We also use $P^{\Box} := P|\Box$.

In \cite{MeredithR05} an interpretation of the new operator is
given. It turns out that there are several possible interpretations
all enjoying the requisite algebraic properties of the operator (see
\cite{milner91polyadicpi}). We will therefore make liberal use of
$(\nu\; \vec{x})P$.

% subsection the_syntax_and_semantics_of_the_notation_system (end)   

\input{qm2pi.qmops} 

\input{qm2pi.sterngerlach} 

\input{qm2pi.metric} 

% section concurrent_process_calculi (end)

%\input{qm2pi.proofsketch}

% section proof sketch (end)

%\input{qm2pi.slviaknots} 

% section spatial logic via knots (end)

\input{qm2pi.conclusion}

% section conclusion (end)

%\input{qm2pi.dtcodes} 

% section wiring algorithm (end)

\input{qm2pi.ack} 

% section acknowledgments (end)

\newpage


\bibliographystyle{plain}   
\bibliography{../../biblios/main.bib}

\input{qm2pi.rhodetails}

\end{document}

 

% section concurrent_process_calculi (end)

%\documentclass[12pt]{llncs}
%\documentclass{jktr}

\usepackage[pdftex]{hyperref}                   
\usepackage {listings}
\usepackage {mathpartir}
\usepackage{bcprules}
%\usepackage{listings}
                       
\usepackage{graphicx} 
%\usepackage[margins=2.5cm,nohead,nofoot]{geometry}
%\usepackage{geometry}
\usepackage{amsfonts}
\usepackage{amstext}
\usepackage{latexsym}
\usepackage{amssymb}
\usepackage{color}


%\include{myPreamble}
\include{qm2pi.local} 

%\ifpdf
%\usepackage[pdftex]{graphicx}
%\else
%\usepackage{graphicx}
%\fi

 % \ifpdf
%  \usepackage{pdfsync}
%  \if


%\title{Brief Article}
%\author{David F. Snyder}
%\author{L.G. Meredith}

%\address{Dept. of Math., Texas State University--San Marcos, San Marcos, TX 78666}
       
\pagestyle{empty}


\begin{document}

\lstset{language=[Objective]Caml,frame=shadowbox}

\input{qm2pi.front}

% section front matter (end)

\input{qm2pi.intro} 
 
% section introduction (end)

% \input{qm2pi.knotations} 

% section notation (end)

\input{qm2pi.process.calculi} 

% section concurrent_process_calculi_and_spatial_logics_ (end)
    
%\input{qm2pi.knots2pi} 

%\input{qm2pi.trefoil} 

%\input{qm2pi.mainthm} 

% subsection basic_interpretation (end)

%\input{qm2pi.rho.presentation} 
\subsection{The syntax and semantics of the notation system}\label{sub:the_syntax_and_semantics_of_the_notation_system} % (fold)

We now summarize a technical presentation of the calculus that
embodies our theory of dynamics. The typical presentation of such a
calculus follows the style of giving generators and relations on
them. The grammar, below, describing term constructors, freely
generates the set of processes, $\Proc$. This set is then quotiented
by a relation known as structural congruence and it is over this set
that the notion of dynamics is expressed. This presentation is
essentially that of \cite{MeredithR05} with the addition of
polyadicity and summation. For readability we have relegated some of
the technical subtleties to an appendix.

\subsubsection{Process grammar}\label{subsub:process_grammar}

\begin{mathpar}
  \inferrule* [lab=synchronization] {} {{M} \bc \pzero \;|\; x?F \;|\; x!C }
  \and
  \inferrule* [lab=abstraction] {} {{F} \bc (x)P}
  \and
  \inferrule* [lab=concretion] {} {{C} \bc \langle Q \rangle}
  \and
  \inferrule* [lab=process] {} {{P,Q} \bc M \;| \;P|Q \;|\; @{x}}
  \and
  \inferrule* [lab=name] {} {{x} \bc \quotep{P}}
\end{mathpar} 

Note that $\vec{x}$ (resp. $\vec{P}$) denotes a vector of names
(resp. processes) of length $|\vec{x}|$ (resp. $|\vec{P}|$). We adopt
the following useful abbreviations.

\begin{mathpar}
   x?(\vec{y}).P := x.(\vec{y})P \and  x\clift{\vec{P}} := x.\clift{\vec{P}}
   \and x!(y) := \lift{x}{\dropn{y}}
   \and \Pi_{i=0}^{n-1}P_i := P_0 | \ldots | P_{n-1}
\end{mathpar}

\subsubsection{Structural congruence}

\paragraph{Free and bound names and alpha-equivalence.} At the
core of structural equivalence is alpha-equivalence which identifies
process that are the same up to a change of variable. Formally, we
recognize the distinction between free and bound names. The free names
of a process, $\freenames{P}$, may be calculated recursively as
follows:

\begin{mathpar}
\freenames{\pzero} := \emptyset
  \and \\
  \freenames{x?(y).P} := \{ x \} \cup (\freenames{P} \setminus \{ y \})
  \and 
  \freenames{x!\langle P \rangle} := \{ x \} \cup \{ P \} 
  \and \\
  \freenames{P|Q} := \freenames{P} \cup \freenames{Q}
  \and \\
  \freenames{@{x}} := \{ x \}
\end{mathpar}

$\pi$
$\quotep{\pi}$

$\freenames{-} : \pi \to \mathcal{P}(\quotep{\pi})$

\begin{eqnarray*}
  \freenames{\pzero} & := & \emptyset \\
  \freenames{x?(y).P} & := & \{ x \} \cup (\freenames{P} \setminus \{ y \}) \\
  \freenames{x!\langle P \rangle} & := & \{ x \} \cup \{ P \} \\
  \freenames{P|Q} & := & \freenames{P} \cup \freenames{Q} \\
  \freenames{\dropn{x}} & := & \{ x \}
\end{eqnarray*}

The bound names of a process, $\boundnames{P}$, are those names occurring in $P$
that are not free. For example, in $x?(y).0$, the name $x$ is free, while $y$ is bound.

\begin{mathpar}
  \inferrule* [lab=monoidal-laws] {} { P|Q \equiv Q|P \and P|0 \equiv P \and P|(Q|R) \equiv (P|Q)|R }
\end{mathpar}

\begin{mathpar}
  \inferrule* [lab=alpha-equivalence] {} { (x)P \equiv (y)P\{y/x\} \and y \not\in \freenames{P} }
\end{mathpar}

\begin{definition}
Then two processes, $P,Q$, are alpha-equivalent if $P = Q\{\vec{y}/\vec{x}\}$ for
some $\vec{x} \in \boundnames{Q},\vec{y} \in \boundnames{P}$, where $Q\{\vec{y}/\vec{x}\}$
denotes the capture-avoiding substitution of $\vec{y}$ for $\vec{x}$ in $Q$.
\end{definition}

\begin{definition}
  The {\em structural congruence} \cite{SangiorgiWalker} , $\equiv$,
  between processes is the least congruence containing
  alpha-equivalence, satisfying the abelian monoid laws
  (associativity, commutativity and $\pzero$ as identity) for parallel
  composition $|$ and for summation $+$.
\end{definition}

\subsection{Name equivalence}

We take name equivalence, written $\nameeq$, to be the smallest
equivalence relation generated by the following rules.

\begin{mathpar}
\inferrule*[lab=Quote-drop]
{ }
{ \quotep{@{x}} \nameeq x }

\inferrule*[lab=Struct-equiv]
{ P \scong Q }
{ \quotep{P} \nameeq \quotep{Q} }
\end{mathpar}

The astute reader will have noticed that the mutual recursion of names
and processes imposes a mutual recursion on alpha-equivalence and
structural equivalence via name-equivalence. Fortunately, all of this
works out pleasantly and we may calculate in the natural way, free of
concern. The reader interested in the details is referred to the
appendix \ref{appendix:rho_details}.

\subsection{Substitution}

We use $\Proc$ for the set of processes, $\QProc$ for the set of
names, and $\id{\{}\vec{y} / \vec{x} \id{\}}$ to denote partial maps,
$s : \QProc \rightarrow \QProc$. A map, $s$ lifts, uniquely, to a map
on process terms, $\widehat{s} : \Proc \rightarrow \Proc$ by the
following equations.

\begin{mathpar}
  (0) \psubstp{Q}{P} := 0 \\
  (R \juxtap S) \psubstp{Q}{P}
  :=    
  (R)\psubstp{Q}{P} \juxtap (S) \psubstp{Q}{P} \\
  (x?(y).R) \psubstp{Q}{P}    
  :=    
  (x)\substp{Q}{P} (z)\concat( (R \psubstn{z}{y}) \psubstp{Q}{P} ) \\
  (\lift{x}{R}) \psubstp{Q}{P}  
  :=
  \lift{(x)\substp{Q}{P}}{ R \psubstp{Q}{P} } \\
%   (\dropn{x})  \psubstp{Q}{P}       
%   := 
%   \left\{ 
%     \begin{array}{ccc} 
%       \dropn{\quotep{Q}} & & x \nameeq \quotep{P} \\
%       \dropn{x} & & otherwise \\
%     \end{array}
%   \right. 
  (\dropn{x})  \psubstp{Q}{P}       
  := 
  \left\{ 
    \begin{array}{ccc} 
      Q & & x \nameeq \quotep{P} \\
      \dropn{x} & & otherwise \\
    \end{array}
  \right.
\end{mathpar}
 

where

\begin{eqnarray}
  (x)\id{\{} \lpquote Q \rpquote / \lpquote P \rpquote \id{\}}            = 
  \left\{ 
    \begin{array}{ccc}
      \lpquote Q \rpquote & & x \nameeq \lpquote P \rpquote \\
      x & & otherwise \\
    \end{array}
  \right. \nonumber
\end{eqnarray}

and $z$ is chosen distinct from $\quotep{P}$, $\quotep{Q}$, the free
names in $Q$, and all the names in $R$. Our $\alpha$-equivalence will
be built in the standard way from this substitution.

\begin{remark}\label{rem:no_self_referential_names}
  One consequence of these definitions is that $\forall P. \quotep{P}
  \not\in \freenames{P}$.
\end{remark}

\subsection{ Dynamic quote: an example }

Anticipating something of what's to come, consider applying the
substitution, $\widehat{\id{\{}u / z \id{\}}}$, to the following pair
of processes, $\lift{w}{y!(z)}$ and $w[ \lpquote y!(z) \rpquote ]$.

\begin{eqnarray}
	\lift{w}{y!(z)}\widehat{\id{\{}u / z \id{\}}}
		& = &
		\lift{w}{y!(u)} \nonumber\\
	w[ \lpquote y!(z) \rpquote ] \widehat{ \id{\{}u / z \id{\}} }
		& = &
		w[ \lpquote y!(z) \rpquote ] \nonumber
\end{eqnarray}

Because the body of the process between quotes is impervious to
substitution, we get radically different answers. In fact, by
examining the first process in an input context,
e.g. $x?(z).\lift{w}{y!(z)}$, we see that the process under the lift
operator may be shaped by prefixed inputs binding a name inside it. In
this sense, the lift operator will be seen as a way to dynamically
construct processes before reifying them as names.

Finally equipped with these standard features we can present the
dynamics of the calculus.

\subsubsection{Operational semantics} 

Finally, we introduce the computational dynamics. What marks these
algebras as distinct from other more traditionally studied algebraic
structures, e.g. vector spaces or polynomial rings, is the manner in
which dynamics is captured. In traditional structures, dynamics is typically
expressed through morphisms between such structures, as in linear maps
between vector spaces or morphisms between rings. In algebras
associated with the semantics of computation, the dynamics is
expressed as part of the algebraic structure itself, through a
reduction reduction relation typically denoted by $\red$. Below, we
give a recursive presentation of this relation for the calculus used
in the encoding.

$\red \subseteq \pi \times \pi$
$\red : \pi \to \mathcal{P}(\pi)$

\begin{mathpar}
  \inferrule* [lab=Comm] { \textsf{match}( x_{src}, x_{trgt} ) } { x_{trgt}?(y)P \; | \; x_{src}!\langle {Q} \rangle \red P\{\quotep{Q}/y}\} }
  \and \\
  \inferrule* [lab=Par] {{P} \red {P}'} {{{P} | {Q}} \red {{P}' | {Q}}}
  \and
  \inferrule* [lab=Equiv]{{{P} \scong {P}'} \andalso {{P}' \red {Q}'} \andalso {{Q}' \scong {Q}}}{{P} \red {Q}}
\end{mathpar}

\begin{eqnarray*}
  match_{\equiv} (\quotep{P},\quotep{Q}) & := & P \equiv Q \\
  match_{\dagger}(\quotep{P},\quotep{Q}) & := & \forall R. P|Q \red^{*} R => R \red^{*} 0 \\
  match_{K}(\quotep{P},\quotep{Q}) & := & K \mbox{ for some context } K
\end{eqnarray*}

$u?(x)P | u!\langle Q \rangle \red P\{\quotep{Q}/x\}$

%We write $\wred$ for $\red^*$, and $P\red$ if $\exists Q $ such that $ P \red Q$.
We write $P\red$ if $\exists Q $ such that $ P \red Q$ and $P\not\red$, otherwise.

\section{Replication}

As mentioned before, it is known that replication (and hence
recursion) can be implemented in a higher-order process algebra
\cite{SangiorgiWalker}. As our first example of calculation with the
machinery thus far presented we give the construction explicitly in
the {\rhoc}.

\begin{eqnarray}
	D_{x} & := & \prefix{x}{y}{(\binpar{\outputp{x}{y}}{@{y}})} \nonumber\\
	\bangp_{x}{P} & := & \binpar{{x}!\langle{\binpar{D_{x}}{P}}\rangle}{D_{x}} \nonumber
\end{eqnarray}

\begin{eqnarray}
	\bangp_{x}{P} & & \nonumber\\
	=
	& {x}!\langle{(\prefix{x}{y}{(\outputp{x}{y} | @{y})) | P}}\rangle 
	      | \prefix{x}{y}{(\outputp{x}{y} | @{y})} & \nonumber\\
	\red
	& (\outputp{x}{y} | @{y})\substn{\quotep{(\prefix{x}{y}{(@{y} | \outputp{x}{y})) | P}}}{y} & \nonumber\\
	=
	& \outputp{x}{\quotep{(\prefix{x}{y}{(\outputp{x}{y} | @{y})) | P}}}
	  | {(\prefix{x}{y}{(\outputp{x}{y} | @{y})) | P}} & \nonumber\\
	\red
	& \ldots & \nonumber\\
	\red^*
	& P | P | \ldots & \nonumber
\end{eqnarray}

Of course, this encoding, as an implementation, runs away, unfolding
$\bangp{P}$ eagerly. A lazier and more implementable replication
operator, restricted to input-guarded processes, may be obtained as follows.

\begin{eqnarray}
\bangp{\prefix{u}{v}{P}} 
	:= 
	\binpar{\lift{x}{\prefix{u}{v}{(\binpar{D(x)}{P})}}}{D(x)} \nonumber
\end{eqnarray}

\begin{remark}
  Note that the lazier definition still does not deal with summation
  or mixed summation (i.e. sums over input and output). The reader is
  invited to construct definitions of replication that deal with these
  features. 

  Further, the definitions are parameterized in a name, $x$. Can you,
  gentle reader, make a definition that eliminates this parameter and
  guarantees no accidental interaction between the replication
  machinery and the process being replicated -- i.e. no accidental
  sharing of names used by the process to get its work done and the
  name(s) used by the replication to effect copying. This latter
  revision of the definition of replication is crucial to obtaining
  the expected identity $!!P \sim !P$.
\end{remark}

\begin{remark}\label{rem:paradoxical_combinator}
  The reader familiar with the lambda calculus will have noticed the
  similarity between $D$ and the paradoxical combinator.

  [Ed. note: the existence of this seems to suggest we have to be more
  restrictive on the set of processes and names we admit if we are to
  support no-cloning.]
\end{remark}

\subsubsection{Bisimulation}

The computational dynamics gives rise to another kind of equivalence,
the equivalence of computational behavior. As previously mentioned
this is typically captured \emph{via} some form of bisimulation.

% The notion we use in this paper is weak barbed bisimulation
% \cite{milner91polyadicpi}.

The notion we use in this paper is derived from weak barbed
bisimulation \cite{milner91polyadicpi}. 

\begin{definition}
An \emph{observation relation}, $\downarrow_{\mathcal N}$, over a set
of names, $\mathcal N$, is the smallest relation satisfying the rules
below.

\infrule[Out-barb]{y \in {\mathcal N}, \; x \nameeq y}
		  {\outputp{x}{v} \downarrow_{\mathcal N} x}
\infrule[Par-barb]{\mbox{$P\downarrow_{\mathcal N} x$ or $Q\downarrow_{\mathcal N} x$}}
		  {\binpar{P}{Q} \downarrow_{\mathcal N} x}

We write $P \Downarrow_{\mathcal N} x$ if there is $Q$ such that 
$P \wred Q$ and $Q \downarrow_{\mathcal N} x$.
\end{definition}

\begin{definition}
%\label{def.bbisim}
An  ${\mathcal N}$-\emph{barbed bisimulation} over a set of names, ${\mathcal N}$, is a symmetric binary relation 
${\mathcal S}_{\mathcal N}$ between agents such that $P\rel{S}_{\mathcal N}Q$ implies:
\begin{enumerate}
\item If $P \red P'$ then $Q \wred Q'$ and $P'\rel{S}_{\mathcal N} Q'$.
\item If $P\downarrow_{\mathcal N} x$, then $Q\Downarrow_{\mathcal N} x$.
\end{enumerate}
$P$ is ${\mathcal N}$-barbed bisimilar to $Q$, written
$P \wbbisim_{\mathcal N} Q$, if $P \rel{S}_{\mathcal N} Q$ for some ${\mathcal N}$-barbed bisimulation ${\mathcal S}_{\mathcal N}$.
\end{definition}

$\mathcal{R} \subseteq \pi \times \pi$

$P \mathcal{R} Q => \forall P'. P \red P' \Rightarrow \exists Q'. Q \red Q', P' \mathcal{R} Q'$

$P \vdash x \Rightarrow Q \vdash x$

\begin{mathpar}
  \inferrule*[lab=Out-barb]{x \nameeq y}{{y}!\langle{Q}\rangle \vdash x}
  \and
  \inferrule*[lab=Par-barb]{\mbox{$P\vdash x$ or $Q\vdash x$}}{\binpar{P}{Q} \vdash x}
\end{mathpar}

\subsubsection{Contexts}

One of the principle advantages of computational calculi like the
$\pi$-calculus is a well-defined notion of context,
contextual-equivalence and a correlation between
contextual-equivalence and notions of bisimulation. The notion of
context allows the decomposition of a process into (sub-)process and
its syntactic environment, its context. Thus, a context may be
thought of as a process with a ``hole'' (written $\Box$) in it. The
application of a context $M$ to a process $P$, written $M[P]$, is
tantamount to filling the hole in $M$ with $P$. In this paper we do
not need the full weight of this theory, but do make use of the notion
of context in the proof the main theorem. 

\begin{mathpar}
  \inferrule* [lab=summation] {} {{M_{M},M_{N}} \bc \Box \;|\; x.M_{A} \;|\; M_{M}+M_{N}}
  \and
  \inferrule* [lab=agent] {} {{M_{A}} \bc (\vec{x})M_{P} \;| \; \clift{P_0,\ldots,M_{P},\ldots,P_N}}
  \and \\
  \inferrule* [lab=process] {} {{M_{P}} \bc M_{N} \;| \;P|M_{P} }
\end{mathpar} 

\begin{mathpar}
  \inferrule* [lab=sychronization] {} {M_{N} \bc \Box \;|\; x?M_{F} \;|\; x!M_{C}}
  \and
  \inferrule* [lab=abstraction] {} {{M_{F}} \bc (x)M_{P} }
  \and
  \inferrule* [lab=concretion] {} {{M_{C}} \bc \langle M_{P} \rangle }
  \and \\
  \inferrule* [lab=process] {} {{M_{P}} \bc M_{N} \;| \;P|M_{P} }
\end{mathpar}

\begin{definition}[contextual application] Given a context $M$, and
  process $P$, we define the \emph{contextual application}, $M[P] :=
  M\{P/\Box\}$. That is, the contextual application of M to P is the
  substitution of $P$ for $\Box$ in $M$.
\end{definition}

$\meaningof{-} : L \to \mathcal{P}(\pi)$

\begin{mathpar}
  \inferrule* [lab=collection] {} {\meaningof{true} = \pi, \and \meaningof{~E} = \pi \setminus \meaningof{E}, \and \meaningof{E_{1} \& E_{2}} = \meaningof{E_{1}} \cap \meaningof{E_{2}}}
\end{mathpar}

\begin{mathpar}
  \inferrule* [lab=structure] {} {\meaningof{0} = \{ P \in \pi | P \equiv 0 \}, \and \\ \meaningof{E_1 | E_2} = \{ P \in \pi | P \equiv P_{1} | P_{2}, P_{1} \in \meaningof{E_{1}}, P_{2} \in \meaningof{E_2}\} }
\end{mathpar}

\begin{mathpar}
 \inferrule* [lab=behavior] {} {\meaningof{\langle a?b \rangle E} = \{ P \in \pi | P \equiv Q | u?(y)P', \\ \and \\\\ \and \\ \;\;\; u \in \meaningof{a}, \forall z.P'\{z/y\} \in \meaningof{E\{z/b\}}\}, \and \\ \meaningof{a!E} = \{ P \in \pi | P \equiv Q | x!\langle P' \rangle, x \in \meaningof{a} P' \in \meaningof{E}\} }
\end{mathpar}

\begin{mathpar}
 \inferrule* [lab=nominal] {} {\meaningof{\quotep{E}} = \{ \quotep{P} \in \quotep{\pi} | P \in \meaningof{E} \}, \and \meaningof{\quotep{P}} = \{ \quotep{Q} \in \quotep{\pi} | P \equiv Q \} \and \\ \meaningof{@\quotep{E}} = \{ P \in \pi | P \equiv @x, x \in \meaningof{E} \}}
\end{mathpar}

\begin{eqnarray*}
  \\
  \meaningof{-} : TS \to ST
\end{eqnarray*}

\begin{eqnarray*}
  \\
  L : TS \to ST
\end{eqnarray*}

\begin{eqnarray*}
  \\
  P \models E \iff P \in \meaningof{E}
\end{eqnarray*}

\begin{eqnarray*}
  P \approx_{L} Q \iff \forall E \in L. P \models E \iff Q \models E
\end{eqnarray*}

\begin{eqnarray*}
  P \approx_{K} Q
\end{eqnarray*}

\begin{eqnarray*}
  P \approx Q
\end{eqnarray*}

$\approx_{K} = \approx = \approx_{L}$

\subsubsection{Contextual duality}

Note that contexts extend the quotation operation to a family of
operations from processes to names. Given a context, $M$, we can
define a \emph{nominal context}, $\quotep{M}$ by $\quotep{M}[P] :=
\quotep{M[P]}$. To foreshadow what is to come we observe that these
operations enjoy a duality with processes very much like the duality
between vectors and maps from vectors to scalars.

Further, because the calculus is essentially higher-order, we have a
correspondence between contexts and processes. More specifically,
given a name $x$ and a context $M$ we can construct $M^{*}_{x}$ such
that 

\begin{mathpar}
  M^{*}_{x} | \lift{x}{P} \red M[P]
\end{mathpar}

namely,

\begin{mathpar}
  M^{*}_{x} := x?(u).M[\dropn{u}]
\end{mathpar}

The dependence of $M^{*}_{x}$ on a name makes it an abstraction, 

\begin{mathpar}
  M^{*} := (x)x?(u).M[\dropn{u}]
\end{mathpar}

\subsection{Additional notation}

It will sometimes be convenient to denote the process a name
quotes. We already have the notation $x = \quotep{P}$, but it will be
convenient to introduce an alternate notation, $\procn{x}$, when we
want to emphasize the connection to the use of the name. Note that, by
virtue of name equivalence, $\quotep{\procn{x}} \nameeq x$; so, the
notation is consistent with previous definitions.

Further, because names have structure it is possible to effect
substitutions on the basis of that structure. This means we need to
upgrade our notation for substitutions, which we accomplish by
adapting comprehension notation. Thus,

\begin{mathpar}
  P\{ y / x : x \in S \}
\end{mathpar}

is interpreted to mean the process derived from P by replacing (in a
capture-avoiding manner) each occurrence of $x$ in $S$ by $y$. For example,

\begin{mathpar}
  P\{ \quotep{\procn{x}|\procn{x}} / x : x \in \freenames{P} \}
\end{mathpar}

will replace each (occurrence) of a free name $x$ in $P$ by
$\quotep{\procn{x}|\procn{x}}$.

Also, we will avail ourselves of the notation $x^{L}$ and $x^{R}$ to
denote injections of a name into disjoint copies of the name
space. There are numerous ways to accomplish this. One example can be
found in \cite{MeredithR05}. This notation overloads to vectors of
names: $\vec{x}^{\pi} := (x_{i}^{\pi} \; : \; 0 \leq i < |\vec{x}| )$ where $\pi \in \{L,R\}$.

We also use $P^{\Box} := P|\Box$.

In \cite{MeredithR05} an interpretation of the new operator is
given. It turns out that there are several possible interpretations
all enjoying the requisite algebraic properties of the operator (see
\cite{milner91polyadicpi}). We will therefore make liberal use of
$(\nu\; \vec{x})P$.

% subsection the_syntax_and_semantics_of_the_notation_system (end)   

\input{qm2pi.qmops} 

\input{qm2pi.sterngerlach} 

\input{qm2pi.metric} 

% section concurrent_process_calculi (end)

%\input{qm2pi.proofsketch}

% section proof sketch (end)

%\input{qm2pi.slviaknots} 

% section spatial logic via knots (end)

\input{qm2pi.conclusion}

% section conclusion (end)

%\input{qm2pi.dtcodes} 

% section wiring algorithm (end)

\input{qm2pi.ack} 

% section acknowledgments (end)

\newpage


\bibliographystyle{plain}   
\bibliography{../../biblios/main.bib}

\input{qm2pi.rhodetails}

\end{document}



% section proof sketch (end)

%\section{Unlikely characters: spatial logic for
  knots}\label{sub:characteristic_formulae} % (fold)

Associated to the mobile process calculi are a family of logics known
as the Hennessy-Milner logics. These logics typically enjoy a
semantics interpreting formulae as sets of processes that when
factored through the encoding outlined above allows an identification
of classes of knots with logical formulae. In the context of this
encoding the sub-family known as the spatial logics \cite{CairesC03}
\cite{CairesC04} \cite{Caires04} are of particular interest providing
several important features for expressing and reasoning about
properties (i.e. classes) of knots. We hint here at how this may be done.

%\begin{description}
%\item [structural connectives] 
\subsubsection{Structural connectives} The spatial logics enjoy
structural connectives corresponding, at the logical level, to the
parallel composition ($P | Q$) and new name ($(\nu \; x)P$)
connectives for processes. As illustrated in the examples below, these
connectives are extremely expressive given the shape of our encoding.
%\item [decideable satisfaction]

\subsubsection{Decideable satisfaction}
In \cite{Caires04} the satisfaction relation is shown to be decideable
for a rich class of processes. It further turns out that the image of
the our encoding is a proper subset of that class. This result
provides the basis for an algorithm by which to search for knots
enjoying a given property.
%\item [characteristic formulae]

\subsubsection{Characteristic formulae}
In the same paper \cite{Caires04} , Caires presents a means of calculating
characteristic formulae, selecting equivalence classes of processes
up to a pre--specified depth limit on the support set of names. Composed with our
encoding, this characteristic formula can be used to select
characteristic formulae for knots.
%\end{description}

\subsubsection{Spatial logic formulae}

The grammar below (segmented for comprehension) summarizes the syntax
of spatial logic formulae. We employ illustrative examples in the
sequel to provide an intuitive understanding of their meaning
referring the reader to \cite{Caires04} for a more detailed explication
of the semantics.

\begin{mathpar}
  \inferrule* [lab=boolean] {} {{A,B} \bc T \;|\; \neg A \;|\; A \wedge B \;|\; \eta = \eta'}
  \and
  \inferrule* [lab=spatial] {} {|\; \pzero \;|\; A | B \;|\; x \text{\textregistered} A \;|\; \forall x . A \;|\;  H x . A}
  \and
  \inferrule* [lab=behavioral] {} {|\; \alpha . A}
  \and 
  \inferrule* [lab=recursion] {} {|\; X(\vec{u}) \;|\; \mu X(\vec{u}) . A}
  \and
  \inferrule* [lab=action] {} {\alpha \bc \langle x?(\vec{y}) \rangle \;|\; \langle x!(\vec{y}) \rangle \;|\; \langle \tau \rangle}
  \and 
  \inferrule* [lab=name] {} {\eta \bc x \;|\; \tau}
\end{mathpar} 

% subsection characteristic_formulae (end)   	 

\subsection{Example formulae}\label{sub:example_formulae_} % (fold)

\subsubsection{Crossing as formula.}
% 
% \begin{align*}
%   \frac{d}{dx} \sin x &= \cos x 
%   & \frac{d}{dx} e^x &= e^x \\
%   \frac{d}{dx} \cos x &= - \sin x 
%   & \frac{d}{dx} \log x &= \frac{1}{x} \\
% \end{align*} 

\begin{align*}
 \mu C(x_{0},x_{1},y_{0},y_{1},u).&(\langle x_{0}?(z) \rangle(\langle u! \rangle\langle y_{1}!z \rangle C(x_{0},x_{1},y_{0},y_{1},u)) & \\
  & \wedge \langle y_{1}?(z) \rangle (\langle u! \rangle \langle x_{0}!z \rangle C(x_{0},x_{1},y_{0},y_{1},u)) & \\
  & \wedge \langle x_{1}?(z) \rangle (\langle u? \rangle \langle y_{0}!z \rangle C(x_{0},x_{1},y_{0},y_{1},u)) & \\
  & \wedge \langle y_{0}?(z) \rangle (\langle u? \rangle \langle x_{1}!z \rangle C(x_{0},x_{1},y_{0},y_{1},u))) &
\end{align*}

The lexicographical similarity between the shape of this formulae and
the shape of definition of the process representing a crossing reveals
the intuitive meaning of this formulae. It describes the capabilities
of a process that has the right to represent a crossing. For example
it picks out processes that may perform an input on the port $x_0$ in
its initial menu of capabilities. What differentiates the formula
from the process, however, is that the crossing process is the
smallest candidate to satisfy the formula. Infinitely many other
processes -- with internal behavior hidden behind this interface, so
to speak -- also satisfy this formula. Even this simple formula,
then, can be seen to open a new view onto knots, providing a
computational interpretation of \emph{virtual} knots.

Note that this formula is derived by hand. A similar formula can be
derived by employing Caires' calculation of characteristic formula
\cite{Caires04} to the process representing a crossing. In light of
this discussion, we let
$\meaningof{C}_{\phi}(x0,x1,y0,y1,u)$ denote a formula specifying the
dynamics we wish to capture of a crossing. To guarantee we preserve
the shape of the interface and minimal semantics we demand that
$\meaningof{C}_{\phi}(x0,x1,y0,y1,u) \Rightarrow
\textbf{C}(x0,x1,y0,y1,u)$ where $\textbf{C}(x0,x1,y0,y1,u)$ denotes
the formula above.
                            
\subsubsection{Crossing number constraints.}
The moral content of the context lemma (Lemma \ref{context}) is that the notion of
``locality'' in the Reidemeister moves is effectively captured by the
parallel composition operator of the process calculus. This intuition
extends through the logic. Given a formula,
$\meaningof{C}_{\phi}(x0,x1,y0,y1,u)$, we can use the structural
connectives to specify constraints on crossing numbers, such as at
least $n$ crossings, or exactly $n$ crossings.
\begin{mathpar}
  \inferrule* [lab=at-least-n] {} { K^{\geq n}_{\phi}(\vec{xs},\vec{ys}) := \Pi_{i=0}^{n-1} Hu . \meaningof{C}_{\phi}(xs_i,ys_i,u) | T }
  \and 
  \inferrule* [lab=exactly-n] {} { K^{= n}_{\phi}(\vec{xs},\vec{ys}) := \Pi_{i=0}^{n-1} Hu . \meaningof{C}_{\phi}(xs_i,ys_i,u) | \neg (\forall x_0,y_0,x_1,y_1,u . \meaningof{C}_{\phi}(x_0,y_0,x_1,y_1,u) | T) }
\end{mathpar}

To round out this section, recall that the encoding of an $n$-crossing
knot decomposes into a parallel composition of $n$ \emph{copies} of a
crossing process together with a wiring harness. To specify different
knot classes with the same crossing number amounts to specifying
logical constraints on the wiring harness. In the interest of space,
we defer examples to a forthcoming paper. Suffice it to say that both
the conditions ``alternating knot'' and ``contains the tangle
corresponding to 5/3'' are expressible. For example, it is possible to
calculate the characteristic formula of a process corresponding to the
tangle 5/3 and conjoin it into the classifying formula via the
composition connective of the logic.

Finally, we wish to observe that it is entirely within reason to
contemplate a more domain-specific version of spatial logic tailored
to the shape of processes in the image of the encoding. Such a
domain-specific logic would have a better claim to the title formal
language of knot properties.

% subsection example_formulae_ (end)

% section knots_as_processes (end) 

% section spatial logic via knots (end)

\section{Conclusions and future work}

\paragraph{Testing physical space}
You, gentle reader, may wonder why of all the theorems to be proved
given this set up we pick the one above. In some sense it's hardly
central to quantum mechanics. We see it as central in the sense that
it firmly establishes a notion of physical space arising from a notion
of the equivalence of behavior. Relating bisimulation to a metric is a
big step forward, but one is faced with interpreting the relationship
of that metric space to something more physical. Quantum mechanical
notions of ``physical'' space are still far from intuitive, but by
relating this idea of distance as testing to calculations that predict
physical circumstances we are making a not insignificant step forward
toward an understanding of the physical space we inhabit as
essentially dynamic.

\paragraph{Effectivity and simulation}
One of the observations we have yet to make is that the entire program
spelled out here is effective. We have built various interpreters for
the reflective calculus at work in this interpretation. In principle,
then, we can simulate quantum mechanics on a computer. The place where
the simulation may lose fidelity is the infinitely branching summation
for the annihilator.

In this connection i also want to point out that the evaluation style
calculation of the inner product puts the non-determinism of the
summation right at the heart of measurement. This suggests that
Milner's original reduction-based formulation of the dynamics of his
calculi in terms of sums was not just notationally suggestive of a
notion of measure-and-continue but captured some significant part of
the physics.

\paragraph{Quantum continuations}
In light of this last observation i want to point out that the
predominant account of quantum mechanics is missing a key aspect of a
truly compositional story of the physical situation. In a real lab,
when a measurement is made the observation can be made to feed into
another device that then makes another measurement conditioned on the
results of the first. This means that after the superposition was
collapsed the entire experimental set up remained in
superposition. While QM offers a means of writing this down it doesn't
quite line up well with the well-trodden formulation of computation
and continuation that we see so succinctly expressed in Milner's
calculi. This suggests that there might be advantages to this account
of dynamics waiting to be explored.

\paragraph{Quantum logic}
In this connection, we also note that by virtue of having the
Hennessy-Milner construction, we can pull the construction through the
interpretation of QM. This gives us a natural candidate for a quantum
logic that enjoys an extremely tight connection with it's domain of
interpretation, making the construction much less ad hoc (rather it is
the image of functor!).

\paragraph{Quantum probabiity}
i have questions about the basis of the interpretation of inner
product as probability amplitude. In particular, using which
axiomatization of probability theory does the notion of probability
amplitude earn the right to be so dubbed? In other words, where is the
proof that the operation for calculating a probability amplitude (and
then squaring) satisfies the axioms of what it means to calculate a
probability? Even if such a proof exists (i have yet to find it in the
literature), i wonder if it might not be possible to turn things on
their heads. Can we view the calculation of the probability amplitude
as an axiomatization of probability? If so, then the definition we
give for calculating probability amplitude may provide the basis for
an \emph{effective} theory of probability.

\paragraph{Quantum vs ``biological'' information}
Finally, i want to conclude with a more philosophical observation. At
a recent workshop in which QM was a predominant topic i noticed
something about quantum information. The speaker was giving a riveting
discussion of axiomatic QM and showing how properties of ``no
cloning'' and ``no deleting'' emerged as consequences of the
axiomatization. Theorems of this form are necessary to give us a sense
of confidence that our axioms characterize the physical theory. What
struck me, though, was that if quantum information is neither erasable
nor replicable it is markedly different from \emph{life}. Two of the
things we know about life is that

\begin{itemize}
  \item it ends;
  \item to gain some measure of persistence, to transcend it's
    finitude it is imminently copyable.
\end{itemize}

Both of these qualities are summarized succinctly in the aphorism: all
flesh is grass. For me these two kinds of ``information'' -- call them
quantum and biological -- are end points on a spectrum of strategies
for persistence. At one end, we have those curious entities that enjoy
uniqueness and permanence; at the other, we have those who in the face
of a certain end and an uncertain present make a go of passing
something on. To me one of the more remarkable aspects of the latter
strategy is that in the presence of noise (and certain features of
copying) we get a kind of dynamism, a chance for improvement against a
given persistent condition.

% subsection other_calculi_other_bisimulations_and_geometry_as_behavior (end)




% section conclusion (end)

%\documentclass[12pt]{llncs}
%\documentclass{jktr}

\usepackage[pdftex]{hyperref}                   
\usepackage {listings}
\usepackage {mathpartir}
\usepackage{bcprules}
%\usepackage{listings}
                       
\usepackage{graphicx} 
%\usepackage[margins=2.5cm,nohead,nofoot]{geometry}
%\usepackage{geometry}
\usepackage{amsfonts}
\usepackage{amstext}
\usepackage{latexsym}
\usepackage{amssymb}
\usepackage{color}


%\include{myPreamble}
\include{qm2pi.local} 

%\ifpdf
%\usepackage[pdftex]{graphicx}
%\else
%\usepackage{graphicx}
%\fi

 % \ifpdf
%  \usepackage{pdfsync}
%  \if


%\title{Brief Article}
%\author{David F. Snyder}
%\author{L.G. Meredith}

%\address{Dept. of Math., Texas State University--San Marcos, San Marcos, TX 78666}
       
\pagestyle{empty}


\begin{document}

\lstset{language=[Objective]Caml,frame=shadowbox}

\input{qm2pi.front}

% section front matter (end)

\input{qm2pi.intro} 
 
% section introduction (end)

% \input{qm2pi.knotations} 

% section notation (end)

\input{qm2pi.process.calculi} 

% section concurrent_process_calculi_and_spatial_logics_ (end)
    
%\input{qm2pi.knots2pi} 

%\input{qm2pi.trefoil} 

%\input{qm2pi.mainthm} 

% subsection basic_interpretation (end)

%\input{qm2pi.rho.presentation} 
\subsection{The syntax and semantics of the notation system}\label{sub:the_syntax_and_semantics_of_the_notation_system} % (fold)

We now summarize a technical presentation of the calculus that
embodies our theory of dynamics. The typical presentation of such a
calculus follows the style of giving generators and relations on
them. The grammar, below, describing term constructors, freely
generates the set of processes, $\Proc$. This set is then quotiented
by a relation known as structural congruence and it is over this set
that the notion of dynamics is expressed. This presentation is
essentially that of \cite{MeredithR05} with the addition of
polyadicity and summation. For readability we have relegated some of
the technical subtleties to an appendix.

\subsubsection{Process grammar}\label{subsub:process_grammar}

\begin{mathpar}
  \inferrule* [lab=synchronization] {} {{M} \bc \pzero \;|\; x?F \;|\; x!C }
  \and
  \inferrule* [lab=abstraction] {} {{F} \bc (x)P}
  \and
  \inferrule* [lab=concretion] {} {{C} \bc \langle Q \rangle}
  \and
  \inferrule* [lab=process] {} {{P,Q} \bc M \;| \;P|Q \;|\; @{x}}
  \and
  \inferrule* [lab=name] {} {{x} \bc \quotep{P}}
\end{mathpar} 

Note that $\vec{x}$ (resp. $\vec{P}$) denotes a vector of names
(resp. processes) of length $|\vec{x}|$ (resp. $|\vec{P}|$). We adopt
the following useful abbreviations.

\begin{mathpar}
   x?(\vec{y}).P := x.(\vec{y})P \and  x\clift{\vec{P}} := x.\clift{\vec{P}}
   \and x!(y) := \lift{x}{\dropn{y}}
   \and \Pi_{i=0}^{n-1}P_i := P_0 | \ldots | P_{n-1}
\end{mathpar}

\subsubsection{Structural congruence}

\paragraph{Free and bound names and alpha-equivalence.} At the
core of structural equivalence is alpha-equivalence which identifies
process that are the same up to a change of variable. Formally, we
recognize the distinction between free and bound names. The free names
of a process, $\freenames{P}$, may be calculated recursively as
follows:

\begin{mathpar}
\freenames{\pzero} := \emptyset
  \and \\
  \freenames{x?(y).P} := \{ x \} \cup (\freenames{P} \setminus \{ y \})
  \and 
  \freenames{x!\langle P \rangle} := \{ x \} \cup \{ P \} 
  \and \\
  \freenames{P|Q} := \freenames{P} \cup \freenames{Q}
  \and \\
  \freenames{@{x}} := \{ x \}
\end{mathpar}

$\pi$
$\quotep{\pi}$

$\freenames{-} : \pi \to \mathcal{P}(\quotep{\pi})$

\begin{eqnarray*}
  \freenames{\pzero} & := & \emptyset \\
  \freenames{x?(y).P} & := & \{ x \} \cup (\freenames{P} \setminus \{ y \}) \\
  \freenames{x!\langle P \rangle} & := & \{ x \} \cup \{ P \} \\
  \freenames{P|Q} & := & \freenames{P} \cup \freenames{Q} \\
  \freenames{\dropn{x}} & := & \{ x \}
\end{eqnarray*}

The bound names of a process, $\boundnames{P}$, are those names occurring in $P$
that are not free. For example, in $x?(y).0$, the name $x$ is free, while $y$ is bound.

\begin{mathpar}
  \inferrule* [lab=monoidal-laws] {} { P|Q \equiv Q|P \and P|0 \equiv P \and P|(Q|R) \equiv (P|Q)|R }
\end{mathpar}

\begin{mathpar}
  \inferrule* [lab=alpha-equivalence] {} { (x)P \equiv (y)P\{y/x\} \and y \not\in \freenames{P} }
\end{mathpar}

\begin{definition}
Then two processes, $P,Q$, are alpha-equivalent if $P = Q\{\vec{y}/\vec{x}\}$ for
some $\vec{x} \in \boundnames{Q},\vec{y} \in \boundnames{P}$, where $Q\{\vec{y}/\vec{x}\}$
denotes the capture-avoiding substitution of $\vec{y}$ for $\vec{x}$ in $Q$.
\end{definition}

\begin{definition}
  The {\em structural congruence} \cite{SangiorgiWalker} , $\equiv$,
  between processes is the least congruence containing
  alpha-equivalence, satisfying the abelian monoid laws
  (associativity, commutativity and $\pzero$ as identity) for parallel
  composition $|$ and for summation $+$.
\end{definition}

\subsection{Name equivalence}

We take name equivalence, written $\nameeq$, to be the smallest
equivalence relation generated by the following rules.

\begin{mathpar}
\inferrule*[lab=Quote-drop]
{ }
{ \quotep{@{x}} \nameeq x }

\inferrule*[lab=Struct-equiv]
{ P \scong Q }
{ \quotep{P} \nameeq \quotep{Q} }
\end{mathpar}

The astute reader will have noticed that the mutual recursion of names
and processes imposes a mutual recursion on alpha-equivalence and
structural equivalence via name-equivalence. Fortunately, all of this
works out pleasantly and we may calculate in the natural way, free of
concern. The reader interested in the details is referred to the
appendix \ref{appendix:rho_details}.

\subsection{Substitution}

We use $\Proc$ for the set of processes, $\QProc$ for the set of
names, and $\id{\{}\vec{y} / \vec{x} \id{\}}$ to denote partial maps,
$s : \QProc \rightarrow \QProc$. A map, $s$ lifts, uniquely, to a map
on process terms, $\widehat{s} : \Proc \rightarrow \Proc$ by the
following equations.

\begin{mathpar}
  (0) \psubstp{Q}{P} := 0 \\
  (R \juxtap S) \psubstp{Q}{P}
  :=    
  (R)\psubstp{Q}{P} \juxtap (S) \psubstp{Q}{P} \\
  (x?(y).R) \psubstp{Q}{P}    
  :=    
  (x)\substp{Q}{P} (z)\concat( (R \psubstn{z}{y}) \psubstp{Q}{P} ) \\
  (\lift{x}{R}) \psubstp{Q}{P}  
  :=
  \lift{(x)\substp{Q}{P}}{ R \psubstp{Q}{P} } \\
%   (\dropn{x})  \psubstp{Q}{P}       
%   := 
%   \left\{ 
%     \begin{array}{ccc} 
%       \dropn{\quotep{Q}} & & x \nameeq \quotep{P} \\
%       \dropn{x} & & otherwise \\
%     \end{array}
%   \right. 
  (\dropn{x})  \psubstp{Q}{P}       
  := 
  \left\{ 
    \begin{array}{ccc} 
      Q & & x \nameeq \quotep{P} \\
      \dropn{x} & & otherwise \\
    \end{array}
  \right.
\end{mathpar}
 

where

\begin{eqnarray}
  (x)\id{\{} \lpquote Q \rpquote / \lpquote P \rpquote \id{\}}            = 
  \left\{ 
    \begin{array}{ccc}
      \lpquote Q \rpquote & & x \nameeq \lpquote P \rpquote \\
      x & & otherwise \\
    \end{array}
  \right. \nonumber
\end{eqnarray}

and $z$ is chosen distinct from $\quotep{P}$, $\quotep{Q}$, the free
names in $Q$, and all the names in $R$. Our $\alpha$-equivalence will
be built in the standard way from this substitution.

\begin{remark}\label{rem:no_self_referential_names}
  One consequence of these definitions is that $\forall P. \quotep{P}
  \not\in \freenames{P}$.
\end{remark}

\subsection{ Dynamic quote: an example }

Anticipating something of what's to come, consider applying the
substitution, $\widehat{\id{\{}u / z \id{\}}}$, to the following pair
of processes, $\lift{w}{y!(z)}$ and $w[ \lpquote y!(z) \rpquote ]$.

\begin{eqnarray}
	\lift{w}{y!(z)}\widehat{\id{\{}u / z \id{\}}}
		& = &
		\lift{w}{y!(u)} \nonumber\\
	w[ \lpquote y!(z) \rpquote ] \widehat{ \id{\{}u / z \id{\}} }
		& = &
		w[ \lpquote y!(z) \rpquote ] \nonumber
\end{eqnarray}

Because the body of the process between quotes is impervious to
substitution, we get radically different answers. In fact, by
examining the first process in an input context,
e.g. $x?(z).\lift{w}{y!(z)}$, we see that the process under the lift
operator may be shaped by prefixed inputs binding a name inside it. In
this sense, the lift operator will be seen as a way to dynamically
construct processes before reifying them as names.

Finally equipped with these standard features we can present the
dynamics of the calculus.

\subsubsection{Operational semantics} 

Finally, we introduce the computational dynamics. What marks these
algebras as distinct from other more traditionally studied algebraic
structures, e.g. vector spaces or polynomial rings, is the manner in
which dynamics is captured. In traditional structures, dynamics is typically
expressed through morphisms between such structures, as in linear maps
between vector spaces or morphisms between rings. In algebras
associated with the semantics of computation, the dynamics is
expressed as part of the algebraic structure itself, through a
reduction reduction relation typically denoted by $\red$. Below, we
give a recursive presentation of this relation for the calculus used
in the encoding.

$\red \subseteq \pi \times \pi$
$\red : \pi \to \mathcal{P}(\pi)$

\begin{mathpar}
  \inferrule* [lab=Comm] { \textsf{match}( x_{src}, x_{trgt} ) } { x_{trgt}?(y)P \; | \; x_{src}!\langle {Q} \rangle \red P\{\quotep{Q}/y}\} }
  \and \\
  \inferrule* [lab=Par] {{P} \red {P}'} {{{P} | {Q}} \red {{P}' | {Q}}}
  \and
  \inferrule* [lab=Equiv]{{{P} \scong {P}'} \andalso {{P}' \red {Q}'} \andalso {{Q}' \scong {Q}}}{{P} \red {Q}}
\end{mathpar}

\begin{eqnarray*}
  match_{\equiv} (\quotep{P},\quotep{Q}) & := & P \equiv Q \\
  match_{\dagger}(\quotep{P},\quotep{Q}) & := & \forall R. P|Q \red^{*} R => R \red^{*} 0 \\
  match_{K}(\quotep{P},\quotep{Q}) & := & K \mbox{ for some context } K
\end{eqnarray*}

$u?(x)P | u!\langle Q \rangle \red P\{\quotep{Q}/x\}$

%We write $\wred$ for $\red^*$, and $P\red$ if $\exists Q $ such that $ P \red Q$.
We write $P\red$ if $\exists Q $ such that $ P \red Q$ and $P\not\red$, otherwise.

\section{Replication}

As mentioned before, it is known that replication (and hence
recursion) can be implemented in a higher-order process algebra
\cite{SangiorgiWalker}. As our first example of calculation with the
machinery thus far presented we give the construction explicitly in
the {\rhoc}.

\begin{eqnarray}
	D_{x} & := & \prefix{x}{y}{(\binpar{\outputp{x}{y}}{@{y}})} \nonumber\\
	\bangp_{x}{P} & := & \binpar{{x}!\langle{\binpar{D_{x}}{P}}\rangle}{D_{x}} \nonumber
\end{eqnarray}

\begin{eqnarray}
	\bangp_{x}{P} & & \nonumber\\
	=
	& {x}!\langle{(\prefix{x}{y}{(\outputp{x}{y} | @{y})) | P}}\rangle 
	      | \prefix{x}{y}{(\outputp{x}{y} | @{y})} & \nonumber\\
	\red
	& (\outputp{x}{y} | @{y})\substn{\quotep{(\prefix{x}{y}{(@{y} | \outputp{x}{y})) | P}}}{y} & \nonumber\\
	=
	& \outputp{x}{\quotep{(\prefix{x}{y}{(\outputp{x}{y} | @{y})) | P}}}
	  | {(\prefix{x}{y}{(\outputp{x}{y} | @{y})) | P}} & \nonumber\\
	\red
	& \ldots & \nonumber\\
	\red^*
	& P | P | \ldots & \nonumber
\end{eqnarray}

Of course, this encoding, as an implementation, runs away, unfolding
$\bangp{P}$ eagerly. A lazier and more implementable replication
operator, restricted to input-guarded processes, may be obtained as follows.

\begin{eqnarray}
\bangp{\prefix{u}{v}{P}} 
	:= 
	\binpar{\lift{x}{\prefix{u}{v}{(\binpar{D(x)}{P})}}}{D(x)} \nonumber
\end{eqnarray}

\begin{remark}
  Note that the lazier definition still does not deal with summation
  or mixed summation (i.e. sums over input and output). The reader is
  invited to construct definitions of replication that deal with these
  features. 

  Further, the definitions are parameterized in a name, $x$. Can you,
  gentle reader, make a definition that eliminates this parameter and
  guarantees no accidental interaction between the replication
  machinery and the process being replicated -- i.e. no accidental
  sharing of names used by the process to get its work done and the
  name(s) used by the replication to effect copying. This latter
  revision of the definition of replication is crucial to obtaining
  the expected identity $!!P \sim !P$.
\end{remark}

\begin{remark}\label{rem:paradoxical_combinator}
  The reader familiar with the lambda calculus will have noticed the
  similarity between $D$ and the paradoxical combinator.

  [Ed. note: the existence of this seems to suggest we have to be more
  restrictive on the set of processes and names we admit if we are to
  support no-cloning.]
\end{remark}

\subsubsection{Bisimulation}

The computational dynamics gives rise to another kind of equivalence,
the equivalence of computational behavior. As previously mentioned
this is typically captured \emph{via} some form of bisimulation.

% The notion we use in this paper is weak barbed bisimulation
% \cite{milner91polyadicpi}.

The notion we use in this paper is derived from weak barbed
bisimulation \cite{milner91polyadicpi}. 

\begin{definition}
An \emph{observation relation}, $\downarrow_{\mathcal N}$, over a set
of names, $\mathcal N$, is the smallest relation satisfying the rules
below.

\infrule[Out-barb]{y \in {\mathcal N}, \; x \nameeq y}
		  {\outputp{x}{v} \downarrow_{\mathcal N} x}
\infrule[Par-barb]{\mbox{$P\downarrow_{\mathcal N} x$ or $Q\downarrow_{\mathcal N} x$}}
		  {\binpar{P}{Q} \downarrow_{\mathcal N} x}

We write $P \Downarrow_{\mathcal N} x$ if there is $Q$ such that 
$P \wred Q$ and $Q \downarrow_{\mathcal N} x$.
\end{definition}

\begin{definition}
%\label{def.bbisim}
An  ${\mathcal N}$-\emph{barbed bisimulation} over a set of names, ${\mathcal N}$, is a symmetric binary relation 
${\mathcal S}_{\mathcal N}$ between agents such that $P\rel{S}_{\mathcal N}Q$ implies:
\begin{enumerate}
\item If $P \red P'$ then $Q \wred Q'$ and $P'\rel{S}_{\mathcal N} Q'$.
\item If $P\downarrow_{\mathcal N} x$, then $Q\Downarrow_{\mathcal N} x$.
\end{enumerate}
$P$ is ${\mathcal N}$-barbed bisimilar to $Q$, written
$P \wbbisim_{\mathcal N} Q$, if $P \rel{S}_{\mathcal N} Q$ for some ${\mathcal N}$-barbed bisimulation ${\mathcal S}_{\mathcal N}$.
\end{definition}

$\mathcal{R} \subseteq \pi \times \pi$

$P \mathcal{R} Q => \forall P'. P \red P' \Rightarrow \exists Q'. Q \red Q', P' \mathcal{R} Q'$

$P \vdash x \Rightarrow Q \vdash x$

\begin{mathpar}
  \inferrule*[lab=Out-barb]{x \nameeq y}{{y}!\langle{Q}\rangle \vdash x}
  \and
  \inferrule*[lab=Par-barb]{\mbox{$P\vdash x$ or $Q\vdash x$}}{\binpar{P}{Q} \vdash x}
\end{mathpar}

\subsubsection{Contexts}

One of the principle advantages of computational calculi like the
$\pi$-calculus is a well-defined notion of context,
contextual-equivalence and a correlation between
contextual-equivalence and notions of bisimulation. The notion of
context allows the decomposition of a process into (sub-)process and
its syntactic environment, its context. Thus, a context may be
thought of as a process with a ``hole'' (written $\Box$) in it. The
application of a context $M$ to a process $P$, written $M[P]$, is
tantamount to filling the hole in $M$ with $P$. In this paper we do
not need the full weight of this theory, but do make use of the notion
of context in the proof the main theorem. 

\begin{mathpar}
  \inferrule* [lab=summation] {} {{M_{M},M_{N}} \bc \Box \;|\; x.M_{A} \;|\; M_{M}+M_{N}}
  \and
  \inferrule* [lab=agent] {} {{M_{A}} \bc (\vec{x})M_{P} \;| \; \clift{P_0,\ldots,M_{P},\ldots,P_N}}
  \and \\
  \inferrule* [lab=process] {} {{M_{P}} \bc M_{N} \;| \;P|M_{P} }
\end{mathpar} 

\begin{mathpar}
  \inferrule* [lab=sychronization] {} {M_{N} \bc \Box \;|\; x?M_{F} \;|\; x!M_{C}}
  \and
  \inferrule* [lab=abstraction] {} {{M_{F}} \bc (x)M_{P} }
  \and
  \inferrule* [lab=concretion] {} {{M_{C}} \bc \langle M_{P} \rangle }
  \and \\
  \inferrule* [lab=process] {} {{M_{P}} \bc M_{N} \;| \;P|M_{P} }
\end{mathpar}

\begin{definition}[contextual application] Given a context $M$, and
  process $P$, we define the \emph{contextual application}, $M[P] :=
  M\{P/\Box\}$. That is, the contextual application of M to P is the
  substitution of $P$ for $\Box$ in $M$.
\end{definition}

$\meaningof{-} : L \to \mathcal{P}(\pi)$

\begin{mathpar}
  \inferrule* [lab=collection] {} {\meaningof{true} = \pi, \and \meaningof{~E} = \pi \setminus \meaningof{E}, \and \meaningof{E_{1} \& E_{2}} = \meaningof{E_{1}} \cap \meaningof{E_{2}}}
\end{mathpar}

\begin{mathpar}
  \inferrule* [lab=structure] {} {\meaningof{0} = \{ P \in \pi | P \equiv 0 \}, \and \\ \meaningof{E_1 | E_2} = \{ P \in \pi | P \equiv P_{1} | P_{2}, P_{1} \in \meaningof{E_{1}}, P_{2} \in \meaningof{E_2}\} }
\end{mathpar}

\begin{mathpar}
 \inferrule* [lab=behavior] {} {\meaningof{\langle a?b \rangle E} = \{ P \in \pi | P \equiv Q | u?(y)P', \\ \and \\\\ \and \\ \;\;\; u \in \meaningof{a}, \forall z.P'\{z/y\} \in \meaningof{E\{z/b\}}\}, \and \\ \meaningof{a!E} = \{ P \in \pi | P \equiv Q | x!\langle P' \rangle, x \in \meaningof{a} P' \in \meaningof{E}\} }
\end{mathpar}

\begin{mathpar}
 \inferrule* [lab=nominal] {} {\meaningof{\quotep{E}} = \{ \quotep{P} \in \quotep{\pi} | P \in \meaningof{E} \}, \and \meaningof{\quotep{P}} = \{ \quotep{Q} \in \quotep{\pi} | P \equiv Q \} \and \\ \meaningof{@\quotep{E}} = \{ P \in \pi | P \equiv @x, x \in \meaningof{E} \}}
\end{mathpar}

\begin{eqnarray*}
  \\
  \meaningof{-} : TS \to ST
\end{eqnarray*}

\begin{eqnarray*}
  \\
  L : TS \to ST
\end{eqnarray*}

\begin{eqnarray*}
  \\
  P \models E \iff P \in \meaningof{E}
\end{eqnarray*}

\begin{eqnarray*}
  P \approx_{L} Q \iff \forall E \in L. P \models E \iff Q \models E
\end{eqnarray*}

\begin{eqnarray*}
  P \approx_{K} Q
\end{eqnarray*}

\begin{eqnarray*}
  P \approx Q
\end{eqnarray*}

$\approx_{K} = \approx = \approx_{L}$

\subsubsection{Contextual duality}

Note that contexts extend the quotation operation to a family of
operations from processes to names. Given a context, $M$, we can
define a \emph{nominal context}, $\quotep{M}$ by $\quotep{M}[P] :=
\quotep{M[P]}$. To foreshadow what is to come we observe that these
operations enjoy a duality with processes very much like the duality
between vectors and maps from vectors to scalars.

Further, because the calculus is essentially higher-order, we have a
correspondence between contexts and processes. More specifically,
given a name $x$ and a context $M$ we can construct $M^{*}_{x}$ such
that 

\begin{mathpar}
  M^{*}_{x} | \lift{x}{P} \red M[P]
\end{mathpar}

namely,

\begin{mathpar}
  M^{*}_{x} := x?(u).M[\dropn{u}]
\end{mathpar}

The dependence of $M^{*}_{x}$ on a name makes it an abstraction, 

\begin{mathpar}
  M^{*} := (x)x?(u).M[\dropn{u}]
\end{mathpar}

\subsection{Additional notation}

It will sometimes be convenient to denote the process a name
quotes. We already have the notation $x = \quotep{P}$, but it will be
convenient to introduce an alternate notation, $\procn{x}$, when we
want to emphasize the connection to the use of the name. Note that, by
virtue of name equivalence, $\quotep{\procn{x}} \nameeq x$; so, the
notation is consistent with previous definitions.

Further, because names have structure it is possible to effect
substitutions on the basis of that structure. This means we need to
upgrade our notation for substitutions, which we accomplish by
adapting comprehension notation. Thus,

\begin{mathpar}
  P\{ y / x : x \in S \}
\end{mathpar}

is interpreted to mean the process derived from P by replacing (in a
capture-avoiding manner) each occurrence of $x$ in $S$ by $y$. For example,

\begin{mathpar}
  P\{ \quotep{\procn{x}|\procn{x}} / x : x \in \freenames{P} \}
\end{mathpar}

will replace each (occurrence) of a free name $x$ in $P$ by
$\quotep{\procn{x}|\procn{x}}$.

Also, we will avail ourselves of the notation $x^{L}$ and $x^{R}$ to
denote injections of a name into disjoint copies of the name
space. There are numerous ways to accomplish this. One example can be
found in \cite{MeredithR05}. This notation overloads to vectors of
names: $\vec{x}^{\pi} := (x_{i}^{\pi} \; : \; 0 \leq i < |\vec{x}| )$ where $\pi \in \{L,R\}$.

We also use $P^{\Box} := P|\Box$.

In \cite{MeredithR05} an interpretation of the new operator is
given. It turns out that there are several possible interpretations
all enjoying the requisite algebraic properties of the operator (see
\cite{milner91polyadicpi}). We will therefore make liberal use of
$(\nu\; \vec{x})P$.

% subsection the_syntax_and_semantics_of_the_notation_system (end)   

\input{qm2pi.qmops} 

\input{qm2pi.sterngerlach} 

\input{qm2pi.metric} 

% section concurrent_process_calculi (end)

%\input{qm2pi.proofsketch}

% section proof sketch (end)

%\input{qm2pi.slviaknots} 

% section spatial logic via knots (end)

\input{qm2pi.conclusion}

% section conclusion (end)

%\input{qm2pi.dtcodes} 

% section wiring algorithm (end)

\input{qm2pi.ack} 

% section acknowledgments (end)

\newpage


\bibliographystyle{plain}   
\bibliography{../../biblios/main.bib}

\input{qm2pi.rhodetails}

\end{document}

 

% section wiring algorithm (end)

\documentclass[12pt]{llncs}
%\documentclass{jktr}

\usepackage[pdftex]{hyperref}                   
\usepackage {listings}
\usepackage {mathpartir}
\usepackage{bcprules}
%\usepackage{listings}
                       
\usepackage{graphicx} 
%\usepackage[margins=2.5cm,nohead,nofoot]{geometry}
%\usepackage{geometry}
\usepackage{amsfonts}
\usepackage{amstext}
\usepackage{latexsym}
\usepackage{amssymb}
\usepackage{color}


%\include{myPreamble}
\include{qm2pi.local} 

%\ifpdf
%\usepackage[pdftex]{graphicx}
%\else
%\usepackage{graphicx}
%\fi

 % \ifpdf
%  \usepackage{pdfsync}
%  \if


%\title{Brief Article}
%\author{David F. Snyder}
%\author{L.G. Meredith}

%\address{Dept. of Math., Texas State University--San Marcos, San Marcos, TX 78666}
       
\pagestyle{empty}


\begin{document}

\lstset{language=[Objective]Caml,frame=shadowbox}

\input{qm2pi.front}

% section front matter (end)

\input{qm2pi.intro} 
 
% section introduction (end)

% \input{qm2pi.knotations} 

% section notation (end)

\input{qm2pi.process.calculi} 

% section concurrent_process_calculi_and_spatial_logics_ (end)
    
%\input{qm2pi.knots2pi} 

%\input{qm2pi.trefoil} 

%\input{qm2pi.mainthm} 

% subsection basic_interpretation (end)

%\input{qm2pi.rho.presentation} 
\subsection{The syntax and semantics of the notation system}\label{sub:the_syntax_and_semantics_of_the_notation_system} % (fold)

We now summarize a technical presentation of the calculus that
embodies our theory of dynamics. The typical presentation of such a
calculus follows the style of giving generators and relations on
them. The grammar, below, describing term constructors, freely
generates the set of processes, $\Proc$. This set is then quotiented
by a relation known as structural congruence and it is over this set
that the notion of dynamics is expressed. This presentation is
essentially that of \cite{MeredithR05} with the addition of
polyadicity and summation. For readability we have relegated some of
the technical subtleties to an appendix.

\subsubsection{Process grammar}\label{subsub:process_grammar}

\begin{mathpar}
  \inferrule* [lab=synchronization] {} {{M} \bc \pzero \;|\; x?F \;|\; x!C }
  \and
  \inferrule* [lab=abstraction] {} {{F} \bc (x)P}
  \and
  \inferrule* [lab=concretion] {} {{C} \bc \langle Q \rangle}
  \and
  \inferrule* [lab=process] {} {{P,Q} \bc M \;| \;P|Q \;|\; @{x}}
  \and
  \inferrule* [lab=name] {} {{x} \bc \quotep{P}}
\end{mathpar} 

Note that $\vec{x}$ (resp. $\vec{P}$) denotes a vector of names
(resp. processes) of length $|\vec{x}|$ (resp. $|\vec{P}|$). We adopt
the following useful abbreviations.

\begin{mathpar}
   x?(\vec{y}).P := x.(\vec{y})P \and  x\clift{\vec{P}} := x.\clift{\vec{P}}
   \and x!(y) := \lift{x}{\dropn{y}}
   \and \Pi_{i=0}^{n-1}P_i := P_0 | \ldots | P_{n-1}
\end{mathpar}

\subsubsection{Structural congruence}

\paragraph{Free and bound names and alpha-equivalence.} At the
core of structural equivalence is alpha-equivalence which identifies
process that are the same up to a change of variable. Formally, we
recognize the distinction between free and bound names. The free names
of a process, $\freenames{P}$, may be calculated recursively as
follows:

\begin{mathpar}
\freenames{\pzero} := \emptyset
  \and \\
  \freenames{x?(y).P} := \{ x \} \cup (\freenames{P} \setminus \{ y \})
  \and 
  \freenames{x!\langle P \rangle} := \{ x \} \cup \{ P \} 
  \and \\
  \freenames{P|Q} := \freenames{P} \cup \freenames{Q}
  \and \\
  \freenames{@{x}} := \{ x \}
\end{mathpar}

$\pi$
$\quotep{\pi}$

$\freenames{-} : \pi \to \mathcal{P}(\quotep{\pi})$

\begin{eqnarray*}
  \freenames{\pzero} & := & \emptyset \\
  \freenames{x?(y).P} & := & \{ x \} \cup (\freenames{P} \setminus \{ y \}) \\
  \freenames{x!\langle P \rangle} & := & \{ x \} \cup \{ P \} \\
  \freenames{P|Q} & := & \freenames{P} \cup \freenames{Q} \\
  \freenames{\dropn{x}} & := & \{ x \}
\end{eqnarray*}

The bound names of a process, $\boundnames{P}$, are those names occurring in $P$
that are not free. For example, in $x?(y).0$, the name $x$ is free, while $y$ is bound.

\begin{mathpar}
  \inferrule* [lab=monoidal-laws] {} { P|Q \equiv Q|P \and P|0 \equiv P \and P|(Q|R) \equiv (P|Q)|R }
\end{mathpar}

\begin{mathpar}
  \inferrule* [lab=alpha-equivalence] {} { (x)P \equiv (y)P\{y/x\} \and y \not\in \freenames{P} }
\end{mathpar}

\begin{definition}
Then two processes, $P,Q$, are alpha-equivalent if $P = Q\{\vec{y}/\vec{x}\}$ for
some $\vec{x} \in \boundnames{Q},\vec{y} \in \boundnames{P}$, where $Q\{\vec{y}/\vec{x}\}$
denotes the capture-avoiding substitution of $\vec{y}$ for $\vec{x}$ in $Q$.
\end{definition}

\begin{definition}
  The {\em structural congruence} \cite{SangiorgiWalker} , $\equiv$,
  between processes is the least congruence containing
  alpha-equivalence, satisfying the abelian monoid laws
  (associativity, commutativity and $\pzero$ as identity) for parallel
  composition $|$ and for summation $+$.
\end{definition}

\subsection{Name equivalence}

We take name equivalence, written $\nameeq$, to be the smallest
equivalence relation generated by the following rules.

\begin{mathpar}
\inferrule*[lab=Quote-drop]
{ }
{ \quotep{@{x}} \nameeq x }

\inferrule*[lab=Struct-equiv]
{ P \scong Q }
{ \quotep{P} \nameeq \quotep{Q} }
\end{mathpar}

The astute reader will have noticed that the mutual recursion of names
and processes imposes a mutual recursion on alpha-equivalence and
structural equivalence via name-equivalence. Fortunately, all of this
works out pleasantly and we may calculate in the natural way, free of
concern. The reader interested in the details is referred to the
appendix \ref{appendix:rho_details}.

\subsection{Substitution}

We use $\Proc$ for the set of processes, $\QProc$ for the set of
names, and $\id{\{}\vec{y} / \vec{x} \id{\}}$ to denote partial maps,
$s : \QProc \rightarrow \QProc$. A map, $s$ lifts, uniquely, to a map
on process terms, $\widehat{s} : \Proc \rightarrow \Proc$ by the
following equations.

\begin{mathpar}
  (0) \psubstp{Q}{P} := 0 \\
  (R \juxtap S) \psubstp{Q}{P}
  :=    
  (R)\psubstp{Q}{P} \juxtap (S) \psubstp{Q}{P} \\
  (x?(y).R) \psubstp{Q}{P}    
  :=    
  (x)\substp{Q}{P} (z)\concat( (R \psubstn{z}{y}) \psubstp{Q}{P} ) \\
  (\lift{x}{R}) \psubstp{Q}{P}  
  :=
  \lift{(x)\substp{Q}{P}}{ R \psubstp{Q}{P} } \\
%   (\dropn{x})  \psubstp{Q}{P}       
%   := 
%   \left\{ 
%     \begin{array}{ccc} 
%       \dropn{\quotep{Q}} & & x \nameeq \quotep{P} \\
%       \dropn{x} & & otherwise \\
%     \end{array}
%   \right. 
  (\dropn{x})  \psubstp{Q}{P}       
  := 
  \left\{ 
    \begin{array}{ccc} 
      Q & & x \nameeq \quotep{P} \\
      \dropn{x} & & otherwise \\
    \end{array}
  \right.
\end{mathpar}
 

where

\begin{eqnarray}
  (x)\id{\{} \lpquote Q \rpquote / \lpquote P \rpquote \id{\}}            = 
  \left\{ 
    \begin{array}{ccc}
      \lpquote Q \rpquote & & x \nameeq \lpquote P \rpquote \\
      x & & otherwise \\
    \end{array}
  \right. \nonumber
\end{eqnarray}

and $z$ is chosen distinct from $\quotep{P}$, $\quotep{Q}$, the free
names in $Q$, and all the names in $R$. Our $\alpha$-equivalence will
be built in the standard way from this substitution.

\begin{remark}\label{rem:no_self_referential_names}
  One consequence of these definitions is that $\forall P. \quotep{P}
  \not\in \freenames{P}$.
\end{remark}

\subsection{ Dynamic quote: an example }

Anticipating something of what's to come, consider applying the
substitution, $\widehat{\id{\{}u / z \id{\}}}$, to the following pair
of processes, $\lift{w}{y!(z)}$ and $w[ \lpquote y!(z) \rpquote ]$.

\begin{eqnarray}
	\lift{w}{y!(z)}\widehat{\id{\{}u / z \id{\}}}
		& = &
		\lift{w}{y!(u)} \nonumber\\
	w[ \lpquote y!(z) \rpquote ] \widehat{ \id{\{}u / z \id{\}} }
		& = &
		w[ \lpquote y!(z) \rpquote ] \nonumber
\end{eqnarray}

Because the body of the process between quotes is impervious to
substitution, we get radically different answers. In fact, by
examining the first process in an input context,
e.g. $x?(z).\lift{w}{y!(z)}$, we see that the process under the lift
operator may be shaped by prefixed inputs binding a name inside it. In
this sense, the lift operator will be seen as a way to dynamically
construct processes before reifying them as names.

Finally equipped with these standard features we can present the
dynamics of the calculus.

\subsubsection{Operational semantics} 

Finally, we introduce the computational dynamics. What marks these
algebras as distinct from other more traditionally studied algebraic
structures, e.g. vector spaces or polynomial rings, is the manner in
which dynamics is captured. In traditional structures, dynamics is typically
expressed through morphisms between such structures, as in linear maps
between vector spaces or morphisms between rings. In algebras
associated with the semantics of computation, the dynamics is
expressed as part of the algebraic structure itself, through a
reduction reduction relation typically denoted by $\red$. Below, we
give a recursive presentation of this relation for the calculus used
in the encoding.

$\red \subseteq \pi \times \pi$
$\red : \pi \to \mathcal{P}(\pi)$

\begin{mathpar}
  \inferrule* [lab=Comm] { \textsf{match}( x_{src}, x_{trgt} ) } { x_{trgt}?(y)P \; | \; x_{src}!\langle {Q} \rangle \red P\{\quotep{Q}/y}\} }
  \and \\
  \inferrule* [lab=Par] {{P} \red {P}'} {{{P} | {Q}} \red {{P}' | {Q}}}
  \and
  \inferrule* [lab=Equiv]{{{P} \scong {P}'} \andalso {{P}' \red {Q}'} \andalso {{Q}' \scong {Q}}}{{P} \red {Q}}
\end{mathpar}

\begin{eqnarray*}
  match_{\equiv} (\quotep{P},\quotep{Q}) & := & P \equiv Q \\
  match_{\dagger}(\quotep{P},\quotep{Q}) & := & \forall R. P|Q \red^{*} R => R \red^{*} 0 \\
  match_{K}(\quotep{P},\quotep{Q}) & := & K \mbox{ for some context } K
\end{eqnarray*}

$u?(x)P | u!\langle Q \rangle \red P\{\quotep{Q}/x\}$

%We write $\wred$ for $\red^*$, and $P\red$ if $\exists Q $ such that $ P \red Q$.
We write $P\red$ if $\exists Q $ such that $ P \red Q$ and $P\not\red$, otherwise.

\section{Replication}

As mentioned before, it is known that replication (and hence
recursion) can be implemented in a higher-order process algebra
\cite{SangiorgiWalker}. As our first example of calculation with the
machinery thus far presented we give the construction explicitly in
the {\rhoc}.

\begin{eqnarray}
	D_{x} & := & \prefix{x}{y}{(\binpar{\outputp{x}{y}}{@{y}})} \nonumber\\
	\bangp_{x}{P} & := & \binpar{{x}!\langle{\binpar{D_{x}}{P}}\rangle}{D_{x}} \nonumber
\end{eqnarray}

\begin{eqnarray}
	\bangp_{x}{P} & & \nonumber\\
	=
	& {x}!\langle{(\prefix{x}{y}{(\outputp{x}{y} | @{y})) | P}}\rangle 
	      | \prefix{x}{y}{(\outputp{x}{y} | @{y})} & \nonumber\\
	\red
	& (\outputp{x}{y} | @{y})\substn{\quotep{(\prefix{x}{y}{(@{y} | \outputp{x}{y})) | P}}}{y} & \nonumber\\
	=
	& \outputp{x}{\quotep{(\prefix{x}{y}{(\outputp{x}{y} | @{y})) | P}}}
	  | {(\prefix{x}{y}{(\outputp{x}{y} | @{y})) | P}} & \nonumber\\
	\red
	& \ldots & \nonumber\\
	\red^*
	& P | P | \ldots & \nonumber
\end{eqnarray}

Of course, this encoding, as an implementation, runs away, unfolding
$\bangp{P}$ eagerly. A lazier and more implementable replication
operator, restricted to input-guarded processes, may be obtained as follows.

\begin{eqnarray}
\bangp{\prefix{u}{v}{P}} 
	:= 
	\binpar{\lift{x}{\prefix{u}{v}{(\binpar{D(x)}{P})}}}{D(x)} \nonumber
\end{eqnarray}

\begin{remark}
  Note that the lazier definition still does not deal with summation
  or mixed summation (i.e. sums over input and output). The reader is
  invited to construct definitions of replication that deal with these
  features. 

  Further, the definitions are parameterized in a name, $x$. Can you,
  gentle reader, make a definition that eliminates this parameter and
  guarantees no accidental interaction between the replication
  machinery and the process being replicated -- i.e. no accidental
  sharing of names used by the process to get its work done and the
  name(s) used by the replication to effect copying. This latter
  revision of the definition of replication is crucial to obtaining
  the expected identity $!!P \sim !P$.
\end{remark}

\begin{remark}\label{rem:paradoxical_combinator}
  The reader familiar with the lambda calculus will have noticed the
  similarity between $D$ and the paradoxical combinator.

  [Ed. note: the existence of this seems to suggest we have to be more
  restrictive on the set of processes and names we admit if we are to
  support no-cloning.]
\end{remark}

\subsubsection{Bisimulation}

The computational dynamics gives rise to another kind of equivalence,
the equivalence of computational behavior. As previously mentioned
this is typically captured \emph{via} some form of bisimulation.

% The notion we use in this paper is weak barbed bisimulation
% \cite{milner91polyadicpi}.

The notion we use in this paper is derived from weak barbed
bisimulation \cite{milner91polyadicpi}. 

\begin{definition}
An \emph{observation relation}, $\downarrow_{\mathcal N}$, over a set
of names, $\mathcal N$, is the smallest relation satisfying the rules
below.

\infrule[Out-barb]{y \in {\mathcal N}, \; x \nameeq y}
		  {\outputp{x}{v} \downarrow_{\mathcal N} x}
\infrule[Par-barb]{\mbox{$P\downarrow_{\mathcal N} x$ or $Q\downarrow_{\mathcal N} x$}}
		  {\binpar{P}{Q} \downarrow_{\mathcal N} x}

We write $P \Downarrow_{\mathcal N} x$ if there is $Q$ such that 
$P \wred Q$ and $Q \downarrow_{\mathcal N} x$.
\end{definition}

\begin{definition}
%\label{def.bbisim}
An  ${\mathcal N}$-\emph{barbed bisimulation} over a set of names, ${\mathcal N}$, is a symmetric binary relation 
${\mathcal S}_{\mathcal N}$ between agents such that $P\rel{S}_{\mathcal N}Q$ implies:
\begin{enumerate}
\item If $P \red P'$ then $Q \wred Q'$ and $P'\rel{S}_{\mathcal N} Q'$.
\item If $P\downarrow_{\mathcal N} x$, then $Q\Downarrow_{\mathcal N} x$.
\end{enumerate}
$P$ is ${\mathcal N}$-barbed bisimilar to $Q$, written
$P \wbbisim_{\mathcal N} Q$, if $P \rel{S}_{\mathcal N} Q$ for some ${\mathcal N}$-barbed bisimulation ${\mathcal S}_{\mathcal N}$.
\end{definition}

$\mathcal{R} \subseteq \pi \times \pi$

$P \mathcal{R} Q => \forall P'. P \red P' \Rightarrow \exists Q'. Q \red Q', P' \mathcal{R} Q'$

$P \vdash x \Rightarrow Q \vdash x$

\begin{mathpar}
  \inferrule*[lab=Out-barb]{x \nameeq y}{{y}!\langle{Q}\rangle \vdash x}
  \and
  \inferrule*[lab=Par-barb]{\mbox{$P\vdash x$ or $Q\vdash x$}}{\binpar{P}{Q} \vdash x}
\end{mathpar}

\subsubsection{Contexts}

One of the principle advantages of computational calculi like the
$\pi$-calculus is a well-defined notion of context,
contextual-equivalence and a correlation between
contextual-equivalence and notions of bisimulation. The notion of
context allows the decomposition of a process into (sub-)process and
its syntactic environment, its context. Thus, a context may be
thought of as a process with a ``hole'' (written $\Box$) in it. The
application of a context $M$ to a process $P$, written $M[P]$, is
tantamount to filling the hole in $M$ with $P$. In this paper we do
not need the full weight of this theory, but do make use of the notion
of context in the proof the main theorem. 

\begin{mathpar}
  \inferrule* [lab=summation] {} {{M_{M},M_{N}} \bc \Box \;|\; x.M_{A} \;|\; M_{M}+M_{N}}
  \and
  \inferrule* [lab=agent] {} {{M_{A}} \bc (\vec{x})M_{P} \;| \; \clift{P_0,\ldots,M_{P},\ldots,P_N}}
  \and \\
  \inferrule* [lab=process] {} {{M_{P}} \bc M_{N} \;| \;P|M_{P} }
\end{mathpar} 

\begin{mathpar}
  \inferrule* [lab=sychronization] {} {M_{N} \bc \Box \;|\; x?M_{F} \;|\; x!M_{C}}
  \and
  \inferrule* [lab=abstraction] {} {{M_{F}} \bc (x)M_{P} }
  \and
  \inferrule* [lab=concretion] {} {{M_{C}} \bc \langle M_{P} \rangle }
  \and \\
  \inferrule* [lab=process] {} {{M_{P}} \bc M_{N} \;| \;P|M_{P} }
\end{mathpar}

\begin{definition}[contextual application] Given a context $M$, and
  process $P$, we define the \emph{contextual application}, $M[P] :=
  M\{P/\Box\}$. That is, the contextual application of M to P is the
  substitution of $P$ for $\Box$ in $M$.
\end{definition}

$\meaningof{-} : L \to \mathcal{P}(\pi)$

\begin{mathpar}
  \inferrule* [lab=collection] {} {\meaningof{true} = \pi, \and \meaningof{~E} = \pi \setminus \meaningof{E}, \and \meaningof{E_{1} \& E_{2}} = \meaningof{E_{1}} \cap \meaningof{E_{2}}}
\end{mathpar}

\begin{mathpar}
  \inferrule* [lab=structure] {} {\meaningof{0} = \{ P \in \pi | P \equiv 0 \}, \and \\ \meaningof{E_1 | E_2} = \{ P \in \pi | P \equiv P_{1} | P_{2}, P_{1} \in \meaningof{E_{1}}, P_{2} \in \meaningof{E_2}\} }
\end{mathpar}

\begin{mathpar}
 \inferrule* [lab=behavior] {} {\meaningof{\langle a?b \rangle E} = \{ P \in \pi | P \equiv Q | u?(y)P', \\ \and \\\\ \and \\ \;\;\; u \in \meaningof{a}, \forall z.P'\{z/y\} \in \meaningof{E\{z/b\}}\}, \and \\ \meaningof{a!E} = \{ P \in \pi | P \equiv Q | x!\langle P' \rangle, x \in \meaningof{a} P' \in \meaningof{E}\} }
\end{mathpar}

\begin{mathpar}
 \inferrule* [lab=nominal] {} {\meaningof{\quotep{E}} = \{ \quotep{P} \in \quotep{\pi} | P \in \meaningof{E} \}, \and \meaningof{\quotep{P}} = \{ \quotep{Q} \in \quotep{\pi} | P \equiv Q \} \and \\ \meaningof{@\quotep{E}} = \{ P \in \pi | P \equiv @x, x \in \meaningof{E} \}}
\end{mathpar}

\begin{eqnarray*}
  \\
  \meaningof{-} : TS \to ST
\end{eqnarray*}

\begin{eqnarray*}
  \\
  L : TS \to ST
\end{eqnarray*}

\begin{eqnarray*}
  \\
  P \models E \iff P \in \meaningof{E}
\end{eqnarray*}

\begin{eqnarray*}
  P \approx_{L} Q \iff \forall E \in L. P \models E \iff Q \models E
\end{eqnarray*}

\begin{eqnarray*}
  P \approx_{K} Q
\end{eqnarray*}

\begin{eqnarray*}
  P \approx Q
\end{eqnarray*}

$\approx_{K} = \approx = \approx_{L}$

\subsubsection{Contextual duality}

Note that contexts extend the quotation operation to a family of
operations from processes to names. Given a context, $M$, we can
define a \emph{nominal context}, $\quotep{M}$ by $\quotep{M}[P] :=
\quotep{M[P]}$. To foreshadow what is to come we observe that these
operations enjoy a duality with processes very much like the duality
between vectors and maps from vectors to scalars.

Further, because the calculus is essentially higher-order, we have a
correspondence between contexts and processes. More specifically,
given a name $x$ and a context $M$ we can construct $M^{*}_{x}$ such
that 

\begin{mathpar}
  M^{*}_{x} | \lift{x}{P} \red M[P]
\end{mathpar}

namely,

\begin{mathpar}
  M^{*}_{x} := x?(u).M[\dropn{u}]
\end{mathpar}

The dependence of $M^{*}_{x}$ on a name makes it an abstraction, 

\begin{mathpar}
  M^{*} := (x)x?(u).M[\dropn{u}]
\end{mathpar}

\subsection{Additional notation}

It will sometimes be convenient to denote the process a name
quotes. We already have the notation $x = \quotep{P}$, but it will be
convenient to introduce an alternate notation, $\procn{x}$, when we
want to emphasize the connection to the use of the name. Note that, by
virtue of name equivalence, $\quotep{\procn{x}} \nameeq x$; so, the
notation is consistent with previous definitions.

Further, because names have structure it is possible to effect
substitutions on the basis of that structure. This means we need to
upgrade our notation for substitutions, which we accomplish by
adapting comprehension notation. Thus,

\begin{mathpar}
  P\{ y / x : x \in S \}
\end{mathpar}

is interpreted to mean the process derived from P by replacing (in a
capture-avoiding manner) each occurrence of $x$ in $S$ by $y$. For example,

\begin{mathpar}
  P\{ \quotep{\procn{x}|\procn{x}} / x : x \in \freenames{P} \}
\end{mathpar}

will replace each (occurrence) of a free name $x$ in $P$ by
$\quotep{\procn{x}|\procn{x}}$.

Also, we will avail ourselves of the notation $x^{L}$ and $x^{R}$ to
denote injections of a name into disjoint copies of the name
space. There are numerous ways to accomplish this. One example can be
found in \cite{MeredithR05}. This notation overloads to vectors of
names: $\vec{x}^{\pi} := (x_{i}^{\pi} \; : \; 0 \leq i < |\vec{x}| )$ where $\pi \in \{L,R\}$.

We also use $P^{\Box} := P|\Box$.

In \cite{MeredithR05} an interpretation of the new operator is
given. It turns out that there are several possible interpretations
all enjoying the requisite algebraic properties of the operator (see
\cite{milner91polyadicpi}). We will therefore make liberal use of
$(\nu\; \vec{x})P$.

% subsection the_syntax_and_semantics_of_the_notation_system (end)   

\input{qm2pi.qmops} 

\input{qm2pi.sterngerlach} 

\input{qm2pi.metric} 

% section concurrent_process_calculi (end)

%\input{qm2pi.proofsketch}

% section proof sketch (end)

%\input{qm2pi.slviaknots} 

% section spatial logic via knots (end)

\input{qm2pi.conclusion}

% section conclusion (end)

%\input{qm2pi.dtcodes} 

% section wiring algorithm (end)

\input{qm2pi.ack} 

% section acknowledgments (end)

\newpage


\bibliographystyle{plain}   
\bibliography{../../biblios/main.bib}

\input{qm2pi.rhodetails}

\end{document}

 

% section acknowledgments (end)

\newpage


\bibliographystyle{plain}   
\bibliography{../../biblios/main.bib}

\documentclass[12pt]{llncs}
%\documentclass{jktr}

\usepackage[pdftex]{hyperref}                   
\usepackage {listings}
\usepackage {mathpartir}
\usepackage{bcprules}
%\usepackage{listings}
                       
\usepackage{graphicx} 
%\usepackage[margins=2.5cm,nohead,nofoot]{geometry}
%\usepackage{geometry}
\usepackage{amsfonts}
\usepackage{amstext}
\usepackage{latexsym}
\usepackage{amssymb}
\usepackage{color}


%\include{myPreamble}
\include{qm2pi.local} 

%\ifpdf
%\usepackage[pdftex]{graphicx}
%\else
%\usepackage{graphicx}
%\fi

 % \ifpdf
%  \usepackage{pdfsync}
%  \if


%\title{Brief Article}
%\author{David F. Snyder}
%\author{L.G. Meredith}

%\address{Dept. of Math., Texas State University--San Marcos, San Marcos, TX 78666}
       
\pagestyle{empty}


\begin{document}

\lstset{language=[Objective]Caml,frame=shadowbox}

\input{qm2pi.front}

% section front matter (end)

\input{qm2pi.intro} 
 
% section introduction (end)

% \input{qm2pi.knotations} 

% section notation (end)

\input{qm2pi.process.calculi} 

% section concurrent_process_calculi_and_spatial_logics_ (end)
    
%\input{qm2pi.knots2pi} 

%\input{qm2pi.trefoil} 

%\input{qm2pi.mainthm} 

% subsection basic_interpretation (end)

%\input{qm2pi.rho.presentation} 
\subsection{The syntax and semantics of the notation system}\label{sub:the_syntax_and_semantics_of_the_notation_system} % (fold)

We now summarize a technical presentation of the calculus that
embodies our theory of dynamics. The typical presentation of such a
calculus follows the style of giving generators and relations on
them. The grammar, below, describing term constructors, freely
generates the set of processes, $\Proc$. This set is then quotiented
by a relation known as structural congruence and it is over this set
that the notion of dynamics is expressed. This presentation is
essentially that of \cite{MeredithR05} with the addition of
polyadicity and summation. For readability we have relegated some of
the technical subtleties to an appendix.

\subsubsection{Process grammar}\label{subsub:process_grammar}

\begin{mathpar}
  \inferrule* [lab=synchronization] {} {{M} \bc \pzero \;|\; x?F \;|\; x!C }
  \and
  \inferrule* [lab=abstraction] {} {{F} \bc (x)P}
  \and
  \inferrule* [lab=concretion] {} {{C} \bc \langle Q \rangle}
  \and
  \inferrule* [lab=process] {} {{P,Q} \bc M \;| \;P|Q \;|\; @{x}}
  \and
  \inferrule* [lab=name] {} {{x} \bc \quotep{P}}
\end{mathpar} 

Note that $\vec{x}$ (resp. $\vec{P}$) denotes a vector of names
(resp. processes) of length $|\vec{x}|$ (resp. $|\vec{P}|$). We adopt
the following useful abbreviations.

\begin{mathpar}
   x?(\vec{y}).P := x.(\vec{y})P \and  x\clift{\vec{P}} := x.\clift{\vec{P}}
   \and x!(y) := \lift{x}{\dropn{y}}
   \and \Pi_{i=0}^{n-1}P_i := P_0 | \ldots | P_{n-1}
\end{mathpar}

\subsubsection{Structural congruence}

\paragraph{Free and bound names and alpha-equivalence.} At the
core of structural equivalence is alpha-equivalence which identifies
process that are the same up to a change of variable. Formally, we
recognize the distinction between free and bound names. The free names
of a process, $\freenames{P}$, may be calculated recursively as
follows:

\begin{mathpar}
\freenames{\pzero} := \emptyset
  \and \\
  \freenames{x?(y).P} := \{ x \} \cup (\freenames{P} \setminus \{ y \})
  \and 
  \freenames{x!\langle P \rangle} := \{ x \} \cup \{ P \} 
  \and \\
  \freenames{P|Q} := \freenames{P} \cup \freenames{Q}
  \and \\
  \freenames{@{x}} := \{ x \}
\end{mathpar}

$\pi$
$\quotep{\pi}$

$\freenames{-} : \pi \to \mathcal{P}(\quotep{\pi})$

\begin{eqnarray*}
  \freenames{\pzero} & := & \emptyset \\
  \freenames{x?(y).P} & := & \{ x \} \cup (\freenames{P} \setminus \{ y \}) \\
  \freenames{x!\langle P \rangle} & := & \{ x \} \cup \{ P \} \\
  \freenames{P|Q} & := & \freenames{P} \cup \freenames{Q} \\
  \freenames{\dropn{x}} & := & \{ x \}
\end{eqnarray*}

The bound names of a process, $\boundnames{P}$, are those names occurring in $P$
that are not free. For example, in $x?(y).0$, the name $x$ is free, while $y$ is bound.

\begin{mathpar}
  \inferrule* [lab=monoidal-laws] {} { P|Q \equiv Q|P \and P|0 \equiv P \and P|(Q|R) \equiv (P|Q)|R }
\end{mathpar}

\begin{mathpar}
  \inferrule* [lab=alpha-equivalence] {} { (x)P \equiv (y)P\{y/x\} \and y \not\in \freenames{P} }
\end{mathpar}

\begin{definition}
Then two processes, $P,Q$, are alpha-equivalent if $P = Q\{\vec{y}/\vec{x}\}$ for
some $\vec{x} \in \boundnames{Q},\vec{y} \in \boundnames{P}$, where $Q\{\vec{y}/\vec{x}\}$
denotes the capture-avoiding substitution of $\vec{y}$ for $\vec{x}$ in $Q$.
\end{definition}

\begin{definition}
  The {\em structural congruence} \cite{SangiorgiWalker} , $\equiv$,
  between processes is the least congruence containing
  alpha-equivalence, satisfying the abelian monoid laws
  (associativity, commutativity and $\pzero$ as identity) for parallel
  composition $|$ and for summation $+$.
\end{definition}

\subsection{Name equivalence}

We take name equivalence, written $\nameeq$, to be the smallest
equivalence relation generated by the following rules.

\begin{mathpar}
\inferrule*[lab=Quote-drop]
{ }
{ \quotep{@{x}} \nameeq x }

\inferrule*[lab=Struct-equiv]
{ P \scong Q }
{ \quotep{P} \nameeq \quotep{Q} }
\end{mathpar}

The astute reader will have noticed that the mutual recursion of names
and processes imposes a mutual recursion on alpha-equivalence and
structural equivalence via name-equivalence. Fortunately, all of this
works out pleasantly and we may calculate in the natural way, free of
concern. The reader interested in the details is referred to the
appendix \ref{appendix:rho_details}.

\subsection{Substitution}

We use $\Proc$ for the set of processes, $\QProc$ for the set of
names, and $\id{\{}\vec{y} / \vec{x} \id{\}}$ to denote partial maps,
$s : \QProc \rightarrow \QProc$. A map, $s$ lifts, uniquely, to a map
on process terms, $\widehat{s} : \Proc \rightarrow \Proc$ by the
following equations.

\begin{mathpar}
  (0) \psubstp{Q}{P} := 0 \\
  (R \juxtap S) \psubstp{Q}{P}
  :=    
  (R)\psubstp{Q}{P} \juxtap (S) \psubstp{Q}{P} \\
  (x?(y).R) \psubstp{Q}{P}    
  :=    
  (x)\substp{Q}{P} (z)\concat( (R \psubstn{z}{y}) \psubstp{Q}{P} ) \\
  (\lift{x}{R}) \psubstp{Q}{P}  
  :=
  \lift{(x)\substp{Q}{P}}{ R \psubstp{Q}{P} } \\
%   (\dropn{x})  \psubstp{Q}{P}       
%   := 
%   \left\{ 
%     \begin{array}{ccc} 
%       \dropn{\quotep{Q}} & & x \nameeq \quotep{P} \\
%       \dropn{x} & & otherwise \\
%     \end{array}
%   \right. 
  (\dropn{x})  \psubstp{Q}{P}       
  := 
  \left\{ 
    \begin{array}{ccc} 
      Q & & x \nameeq \quotep{P} \\
      \dropn{x} & & otherwise \\
    \end{array}
  \right.
\end{mathpar}
 

where

\begin{eqnarray}
  (x)\id{\{} \lpquote Q \rpquote / \lpquote P \rpquote \id{\}}            = 
  \left\{ 
    \begin{array}{ccc}
      \lpquote Q \rpquote & & x \nameeq \lpquote P \rpquote \\
      x & & otherwise \\
    \end{array}
  \right. \nonumber
\end{eqnarray}

and $z$ is chosen distinct from $\quotep{P}$, $\quotep{Q}$, the free
names in $Q$, and all the names in $R$. Our $\alpha$-equivalence will
be built in the standard way from this substitution.

\begin{remark}\label{rem:no_self_referential_names}
  One consequence of these definitions is that $\forall P. \quotep{P}
  \not\in \freenames{P}$.
\end{remark}

\subsection{ Dynamic quote: an example }

Anticipating something of what's to come, consider applying the
substitution, $\widehat{\id{\{}u / z \id{\}}}$, to the following pair
of processes, $\lift{w}{y!(z)}$ and $w[ \lpquote y!(z) \rpquote ]$.

\begin{eqnarray}
	\lift{w}{y!(z)}\widehat{\id{\{}u / z \id{\}}}
		& = &
		\lift{w}{y!(u)} \nonumber\\
	w[ \lpquote y!(z) \rpquote ] \widehat{ \id{\{}u / z \id{\}} }
		& = &
		w[ \lpquote y!(z) \rpquote ] \nonumber
\end{eqnarray}

Because the body of the process between quotes is impervious to
substitution, we get radically different answers. In fact, by
examining the first process in an input context,
e.g. $x?(z).\lift{w}{y!(z)}$, we see that the process under the lift
operator may be shaped by prefixed inputs binding a name inside it. In
this sense, the lift operator will be seen as a way to dynamically
construct processes before reifying them as names.

Finally equipped with these standard features we can present the
dynamics of the calculus.

\subsubsection{Operational semantics} 

Finally, we introduce the computational dynamics. What marks these
algebras as distinct from other more traditionally studied algebraic
structures, e.g. vector spaces or polynomial rings, is the manner in
which dynamics is captured. In traditional structures, dynamics is typically
expressed through morphisms between such structures, as in linear maps
between vector spaces or morphisms between rings. In algebras
associated with the semantics of computation, the dynamics is
expressed as part of the algebraic structure itself, through a
reduction reduction relation typically denoted by $\red$. Below, we
give a recursive presentation of this relation for the calculus used
in the encoding.

$\red \subseteq \pi \times \pi$
$\red : \pi \to \mathcal{P}(\pi)$

\begin{mathpar}
  \inferrule* [lab=Comm] { \textsf{match}( x_{src}, x_{trgt} ) } { x_{trgt}?(y)P \; | \; x_{src}!\langle {Q} \rangle \red P\{\quotep{Q}/y}\} }
  \and \\
  \inferrule* [lab=Par] {{P} \red {P}'} {{{P} | {Q}} \red {{P}' | {Q}}}
  \and
  \inferrule* [lab=Equiv]{{{P} \scong {P}'} \andalso {{P}' \red {Q}'} \andalso {{Q}' \scong {Q}}}{{P} \red {Q}}
\end{mathpar}

\begin{eqnarray*}
  match_{\equiv} (\quotep{P},\quotep{Q}) & := & P \equiv Q \\
  match_{\dagger}(\quotep{P},\quotep{Q}) & := & \forall R. P|Q \red^{*} R => R \red^{*} 0 \\
  match_{K}(\quotep{P},\quotep{Q}) & := & K \mbox{ for some context } K
\end{eqnarray*}

$u?(x)P | u!\langle Q \rangle \red P\{\quotep{Q}/x\}$

%We write $\wred$ for $\red^*$, and $P\red$ if $\exists Q $ such that $ P \red Q$.
We write $P\red$ if $\exists Q $ such that $ P \red Q$ and $P\not\red$, otherwise.

\section{Replication}

As mentioned before, it is known that replication (and hence
recursion) can be implemented in a higher-order process algebra
\cite{SangiorgiWalker}. As our first example of calculation with the
machinery thus far presented we give the construction explicitly in
the {\rhoc}.

\begin{eqnarray}
	D_{x} & := & \prefix{x}{y}{(\binpar{\outputp{x}{y}}{@{y}})} \nonumber\\
	\bangp_{x}{P} & := & \binpar{{x}!\langle{\binpar{D_{x}}{P}}\rangle}{D_{x}} \nonumber
\end{eqnarray}

\begin{eqnarray}
	\bangp_{x}{P} & & \nonumber\\
	=
	& {x}!\langle{(\prefix{x}{y}{(\outputp{x}{y} | @{y})) | P}}\rangle 
	      | \prefix{x}{y}{(\outputp{x}{y} | @{y})} & \nonumber\\
	\red
	& (\outputp{x}{y} | @{y})\substn{\quotep{(\prefix{x}{y}{(@{y} | \outputp{x}{y})) | P}}}{y} & \nonumber\\
	=
	& \outputp{x}{\quotep{(\prefix{x}{y}{(\outputp{x}{y} | @{y})) | P}}}
	  | {(\prefix{x}{y}{(\outputp{x}{y} | @{y})) | P}} & \nonumber\\
	\red
	& \ldots & \nonumber\\
	\red^*
	& P | P | \ldots & \nonumber
\end{eqnarray}

Of course, this encoding, as an implementation, runs away, unfolding
$\bangp{P}$ eagerly. A lazier and more implementable replication
operator, restricted to input-guarded processes, may be obtained as follows.

\begin{eqnarray}
\bangp{\prefix{u}{v}{P}} 
	:= 
	\binpar{\lift{x}{\prefix{u}{v}{(\binpar{D(x)}{P})}}}{D(x)} \nonumber
\end{eqnarray}

\begin{remark}
  Note that the lazier definition still does not deal with summation
  or mixed summation (i.e. sums over input and output). The reader is
  invited to construct definitions of replication that deal with these
  features. 

  Further, the definitions are parameterized in a name, $x$. Can you,
  gentle reader, make a definition that eliminates this parameter and
  guarantees no accidental interaction between the replication
  machinery and the process being replicated -- i.e. no accidental
  sharing of names used by the process to get its work done and the
  name(s) used by the replication to effect copying. This latter
  revision of the definition of replication is crucial to obtaining
  the expected identity $!!P \sim !P$.
\end{remark}

\begin{remark}\label{rem:paradoxical_combinator}
  The reader familiar with the lambda calculus will have noticed the
  similarity between $D$ and the paradoxical combinator.

  [Ed. note: the existence of this seems to suggest we have to be more
  restrictive on the set of processes and names we admit if we are to
  support no-cloning.]
\end{remark}

\subsubsection{Bisimulation}

The computational dynamics gives rise to another kind of equivalence,
the equivalence of computational behavior. As previously mentioned
this is typically captured \emph{via} some form of bisimulation.

% The notion we use in this paper is weak barbed bisimulation
% \cite{milner91polyadicpi}.

The notion we use in this paper is derived from weak barbed
bisimulation \cite{milner91polyadicpi}. 

\begin{definition}
An \emph{observation relation}, $\downarrow_{\mathcal N}$, over a set
of names, $\mathcal N$, is the smallest relation satisfying the rules
below.

\infrule[Out-barb]{y \in {\mathcal N}, \; x \nameeq y}
		  {\outputp{x}{v} \downarrow_{\mathcal N} x}
\infrule[Par-barb]{\mbox{$P\downarrow_{\mathcal N} x$ or $Q\downarrow_{\mathcal N} x$}}
		  {\binpar{P}{Q} \downarrow_{\mathcal N} x}

We write $P \Downarrow_{\mathcal N} x$ if there is $Q$ such that 
$P \wred Q$ and $Q \downarrow_{\mathcal N} x$.
\end{definition}

\begin{definition}
%\label{def.bbisim}
An  ${\mathcal N}$-\emph{barbed bisimulation} over a set of names, ${\mathcal N}$, is a symmetric binary relation 
${\mathcal S}_{\mathcal N}$ between agents such that $P\rel{S}_{\mathcal N}Q$ implies:
\begin{enumerate}
\item If $P \red P'$ then $Q \wred Q'$ and $P'\rel{S}_{\mathcal N} Q'$.
\item If $P\downarrow_{\mathcal N} x$, then $Q\Downarrow_{\mathcal N} x$.
\end{enumerate}
$P$ is ${\mathcal N}$-barbed bisimilar to $Q$, written
$P \wbbisim_{\mathcal N} Q$, if $P \rel{S}_{\mathcal N} Q$ for some ${\mathcal N}$-barbed bisimulation ${\mathcal S}_{\mathcal N}$.
\end{definition}

$\mathcal{R} \subseteq \pi \times \pi$

$P \mathcal{R} Q => \forall P'. P \red P' \Rightarrow \exists Q'. Q \red Q', P' \mathcal{R} Q'$

$P \vdash x \Rightarrow Q \vdash x$

\begin{mathpar}
  \inferrule*[lab=Out-barb]{x \nameeq y}{{y}!\langle{Q}\rangle \vdash x}
  \and
  \inferrule*[lab=Par-barb]{\mbox{$P\vdash x$ or $Q\vdash x$}}{\binpar{P}{Q} \vdash x}
\end{mathpar}

\subsubsection{Contexts}

One of the principle advantages of computational calculi like the
$\pi$-calculus is a well-defined notion of context,
contextual-equivalence and a correlation between
contextual-equivalence and notions of bisimulation. The notion of
context allows the decomposition of a process into (sub-)process and
its syntactic environment, its context. Thus, a context may be
thought of as a process with a ``hole'' (written $\Box$) in it. The
application of a context $M$ to a process $P$, written $M[P]$, is
tantamount to filling the hole in $M$ with $P$. In this paper we do
not need the full weight of this theory, but do make use of the notion
of context in the proof the main theorem. 

\begin{mathpar}
  \inferrule* [lab=summation] {} {{M_{M},M_{N}} \bc \Box \;|\; x.M_{A} \;|\; M_{M}+M_{N}}
  \and
  \inferrule* [lab=agent] {} {{M_{A}} \bc (\vec{x})M_{P} \;| \; \clift{P_0,\ldots,M_{P},\ldots,P_N}}
  \and \\
  \inferrule* [lab=process] {} {{M_{P}} \bc M_{N} \;| \;P|M_{P} }
\end{mathpar} 

\begin{mathpar}
  \inferrule* [lab=sychronization] {} {M_{N} \bc \Box \;|\; x?M_{F} \;|\; x!M_{C}}
  \and
  \inferrule* [lab=abstraction] {} {{M_{F}} \bc (x)M_{P} }
  \and
  \inferrule* [lab=concretion] {} {{M_{C}} \bc \langle M_{P} \rangle }
  \and \\
  \inferrule* [lab=process] {} {{M_{P}} \bc M_{N} \;| \;P|M_{P} }
\end{mathpar}

\begin{definition}[contextual application] Given a context $M$, and
  process $P$, we define the \emph{contextual application}, $M[P] :=
  M\{P/\Box\}$. That is, the contextual application of M to P is the
  substitution of $P$ for $\Box$ in $M$.
\end{definition}

$\meaningof{-} : L \to \mathcal{P}(\pi)$

\begin{mathpar}
  \inferrule* [lab=collection] {} {\meaningof{true} = \pi, \and \meaningof{~E} = \pi \setminus \meaningof{E}, \and \meaningof{E_{1} \& E_{2}} = \meaningof{E_{1}} \cap \meaningof{E_{2}}}
\end{mathpar}

\begin{mathpar}
  \inferrule* [lab=structure] {} {\meaningof{0} = \{ P \in \pi | P \equiv 0 \}, \and \\ \meaningof{E_1 | E_2} = \{ P \in \pi | P \equiv P_{1} | P_{2}, P_{1} \in \meaningof{E_{1}}, P_{2} \in \meaningof{E_2}\} }
\end{mathpar}

\begin{mathpar}
 \inferrule* [lab=behavior] {} {\meaningof{\langle a?b \rangle E} = \{ P \in \pi | P \equiv Q | u?(y)P', \\ \and \\\\ \and \\ \;\;\; u \in \meaningof{a}, \forall z.P'\{z/y\} \in \meaningof{E\{z/b\}}\}, \and \\ \meaningof{a!E} = \{ P \in \pi | P \equiv Q | x!\langle P' \rangle, x \in \meaningof{a} P' \in \meaningof{E}\} }
\end{mathpar}

\begin{mathpar}
 \inferrule* [lab=nominal] {} {\meaningof{\quotep{E}} = \{ \quotep{P} \in \quotep{\pi} | P \in \meaningof{E} \}, \and \meaningof{\quotep{P}} = \{ \quotep{Q} \in \quotep{\pi} | P \equiv Q \} \and \\ \meaningof{@\quotep{E}} = \{ P \in \pi | P \equiv @x, x \in \meaningof{E} \}}
\end{mathpar}

\begin{eqnarray*}
  \\
  \meaningof{-} : TS \to ST
\end{eqnarray*}

\begin{eqnarray*}
  \\
  L : TS \to ST
\end{eqnarray*}

\begin{eqnarray*}
  \\
  P \models E \iff P \in \meaningof{E}
\end{eqnarray*}

\begin{eqnarray*}
  P \approx_{L} Q \iff \forall E \in L. P \models E \iff Q \models E
\end{eqnarray*}

\begin{eqnarray*}
  P \approx_{K} Q
\end{eqnarray*}

\begin{eqnarray*}
  P \approx Q
\end{eqnarray*}

$\approx_{K} = \approx = \approx_{L}$

\subsubsection{Contextual duality}

Note that contexts extend the quotation operation to a family of
operations from processes to names. Given a context, $M$, we can
define a \emph{nominal context}, $\quotep{M}$ by $\quotep{M}[P] :=
\quotep{M[P]}$. To foreshadow what is to come we observe that these
operations enjoy a duality with processes very much like the duality
between vectors and maps from vectors to scalars.

Further, because the calculus is essentially higher-order, we have a
correspondence between contexts and processes. More specifically,
given a name $x$ and a context $M$ we can construct $M^{*}_{x}$ such
that 

\begin{mathpar}
  M^{*}_{x} | \lift{x}{P} \red M[P]
\end{mathpar}

namely,

\begin{mathpar}
  M^{*}_{x} := x?(u).M[\dropn{u}]
\end{mathpar}

The dependence of $M^{*}_{x}$ on a name makes it an abstraction, 

\begin{mathpar}
  M^{*} := (x)x?(u).M[\dropn{u}]
\end{mathpar}

\subsection{Additional notation}

It will sometimes be convenient to denote the process a name
quotes. We already have the notation $x = \quotep{P}$, but it will be
convenient to introduce an alternate notation, $\procn{x}$, when we
want to emphasize the connection to the use of the name. Note that, by
virtue of name equivalence, $\quotep{\procn{x}} \nameeq x$; so, the
notation is consistent with previous definitions.

Further, because names have structure it is possible to effect
substitutions on the basis of that structure. This means we need to
upgrade our notation for substitutions, which we accomplish by
adapting comprehension notation. Thus,

\begin{mathpar}
  P\{ y / x : x \in S \}
\end{mathpar}

is interpreted to mean the process derived from P by replacing (in a
capture-avoiding manner) each occurrence of $x$ in $S$ by $y$. For example,

\begin{mathpar}
  P\{ \quotep{\procn{x}|\procn{x}} / x : x \in \freenames{P} \}
\end{mathpar}

will replace each (occurrence) of a free name $x$ in $P$ by
$\quotep{\procn{x}|\procn{x}}$.

Also, we will avail ourselves of the notation $x^{L}$ and $x^{R}$ to
denote injections of a name into disjoint copies of the name
space. There are numerous ways to accomplish this. One example can be
found in \cite{MeredithR05}. This notation overloads to vectors of
names: $\vec{x}^{\pi} := (x_{i}^{\pi} \; : \; 0 \leq i < |\vec{x}| )$ where $\pi \in \{L,R\}$.

We also use $P^{\Box} := P|\Box$.

In \cite{MeredithR05} an interpretation of the new operator is
given. It turns out that there are several possible interpretations
all enjoying the requisite algebraic properties of the operator (see
\cite{milner91polyadicpi}). We will therefore make liberal use of
$(\nu\; \vec{x})P$.

% subsection the_syntax_and_semantics_of_the_notation_system (end)   

\input{qm2pi.qmops} 

\input{qm2pi.sterngerlach} 

\input{qm2pi.metric} 

% section concurrent_process_calculi (end)

%\input{qm2pi.proofsketch}

% section proof sketch (end)

%\input{qm2pi.slviaknots} 

% section spatial logic via knots (end)

\input{qm2pi.conclusion}

% section conclusion (end)

%\input{qm2pi.dtcodes} 

% section wiring algorithm (end)

\input{qm2pi.ack} 

% section acknowledgments (end)

\newpage


\bibliographystyle{plain}   
\bibliography{../../biblios/main.bib}

\input{qm2pi.rhodetails}

\end{document}



\end{document}

 

% section wiring algorithm (end)

\documentclass[12pt]{llncs}
%\documentclass{jktr}

\usepackage[pdftex]{hyperref}                   
\usepackage {listings}
\usepackage {mathpartir}
\usepackage{bcprules}
%\usepackage{listings}
                       
\usepackage{graphicx} 
%\usepackage[margins=2.5cm,nohead,nofoot]{geometry}
%\usepackage{geometry}
\usepackage{amsfonts}
\usepackage{amstext}
\usepackage{latexsym}
\usepackage{amssymb}
\usepackage{color}


%\include{myPreamble}
\documentclass[12pt]{llncs}
%\documentclass{jktr}

\usepackage[pdftex]{hyperref}                   
\usepackage {listings}
\usepackage {mathpartir}
\usepackage{bcprules}
%\usepackage{listings}
                       
\usepackage{graphicx} 
%\usepackage[margins=2.5cm,nohead,nofoot]{geometry}
%\usepackage{geometry}
\usepackage{amsfonts}
\usepackage{amstext}
\usepackage{latexsym}
\usepackage{amssymb}
\usepackage{color}


%\include{myPreamble}
\include{qm2pi.local} 

%\ifpdf
%\usepackage[pdftex]{graphicx}
%\else
%\usepackage{graphicx}
%\fi

 % \ifpdf
%  \usepackage{pdfsync}
%  \if


%\title{Brief Article}
%\author{David F. Snyder}
%\author{L.G. Meredith}

%\address{Dept. of Math., Texas State University--San Marcos, San Marcos, TX 78666}
       
\pagestyle{empty}


\begin{document}

\lstset{language=[Objective]Caml,frame=shadowbox}

\input{qm2pi.front}

% section front matter (end)

\input{qm2pi.intro} 
 
% section introduction (end)

% \input{qm2pi.knotations} 

% section notation (end)

\input{qm2pi.process.calculi} 

% section concurrent_process_calculi_and_spatial_logics_ (end)
    
%\input{qm2pi.knots2pi} 

%\input{qm2pi.trefoil} 

%\input{qm2pi.mainthm} 

% subsection basic_interpretation (end)

%\input{qm2pi.rho.presentation} 
\subsection{The syntax and semantics of the notation system}\label{sub:the_syntax_and_semantics_of_the_notation_system} % (fold)

We now summarize a technical presentation of the calculus that
embodies our theory of dynamics. The typical presentation of such a
calculus follows the style of giving generators and relations on
them. The grammar, below, describing term constructors, freely
generates the set of processes, $\Proc$. This set is then quotiented
by a relation known as structural congruence and it is over this set
that the notion of dynamics is expressed. This presentation is
essentially that of \cite{MeredithR05} with the addition of
polyadicity and summation. For readability we have relegated some of
the technical subtleties to an appendix.

\subsubsection{Process grammar}\label{subsub:process_grammar}

\begin{mathpar}
  \inferrule* [lab=synchronization] {} {{M} \bc \pzero \;|\; x?F \;|\; x!C }
  \and
  \inferrule* [lab=abstraction] {} {{F} \bc (x)P}
  \and
  \inferrule* [lab=concretion] {} {{C} \bc \langle Q \rangle}
  \and
  \inferrule* [lab=process] {} {{P,Q} \bc M \;| \;P|Q \;|\; @{x}}
  \and
  \inferrule* [lab=name] {} {{x} \bc \quotep{P}}
\end{mathpar} 

Note that $\vec{x}$ (resp. $\vec{P}$) denotes a vector of names
(resp. processes) of length $|\vec{x}|$ (resp. $|\vec{P}|$). We adopt
the following useful abbreviations.

\begin{mathpar}
   x?(\vec{y}).P := x.(\vec{y})P \and  x\clift{\vec{P}} := x.\clift{\vec{P}}
   \and x!(y) := \lift{x}{\dropn{y}}
   \and \Pi_{i=0}^{n-1}P_i := P_0 | \ldots | P_{n-1}
\end{mathpar}

\subsubsection{Structural congruence}

\paragraph{Free and bound names and alpha-equivalence.} At the
core of structural equivalence is alpha-equivalence which identifies
process that are the same up to a change of variable. Formally, we
recognize the distinction between free and bound names. The free names
of a process, $\freenames{P}$, may be calculated recursively as
follows:

\begin{mathpar}
\freenames{\pzero} := \emptyset
  \and \\
  \freenames{x?(y).P} := \{ x \} \cup (\freenames{P} \setminus \{ y \})
  \and 
  \freenames{x!\langle P \rangle} := \{ x \} \cup \{ P \} 
  \and \\
  \freenames{P|Q} := \freenames{P} \cup \freenames{Q}
  \and \\
  \freenames{@{x}} := \{ x \}
\end{mathpar}

$\pi$
$\quotep{\pi}$

$\freenames{-} : \pi \to \mathcal{P}(\quotep{\pi})$

\begin{eqnarray*}
  \freenames{\pzero} & := & \emptyset \\
  \freenames{x?(y).P} & := & \{ x \} \cup (\freenames{P} \setminus \{ y \}) \\
  \freenames{x!\langle P \rangle} & := & \{ x \} \cup \{ P \} \\
  \freenames{P|Q} & := & \freenames{P} \cup \freenames{Q} \\
  \freenames{\dropn{x}} & := & \{ x \}
\end{eqnarray*}

The bound names of a process, $\boundnames{P}$, are those names occurring in $P$
that are not free. For example, in $x?(y).0$, the name $x$ is free, while $y$ is bound.

\begin{mathpar}
  \inferrule* [lab=monoidal-laws] {} { P|Q \equiv Q|P \and P|0 \equiv P \and P|(Q|R) \equiv (P|Q)|R }
\end{mathpar}

\begin{mathpar}
  \inferrule* [lab=alpha-equivalence] {} { (x)P \equiv (y)P\{y/x\} \and y \not\in \freenames{P} }
\end{mathpar}

\begin{definition}
Then two processes, $P,Q$, are alpha-equivalent if $P = Q\{\vec{y}/\vec{x}\}$ for
some $\vec{x} \in \boundnames{Q},\vec{y} \in \boundnames{P}$, where $Q\{\vec{y}/\vec{x}\}$
denotes the capture-avoiding substitution of $\vec{y}$ for $\vec{x}$ in $Q$.
\end{definition}

\begin{definition}
  The {\em structural congruence} \cite{SangiorgiWalker} , $\equiv$,
  between processes is the least congruence containing
  alpha-equivalence, satisfying the abelian monoid laws
  (associativity, commutativity and $\pzero$ as identity) for parallel
  composition $|$ and for summation $+$.
\end{definition}

\subsection{Name equivalence}

We take name equivalence, written $\nameeq$, to be the smallest
equivalence relation generated by the following rules.

\begin{mathpar}
\inferrule*[lab=Quote-drop]
{ }
{ \quotep{@{x}} \nameeq x }

\inferrule*[lab=Struct-equiv]
{ P \scong Q }
{ \quotep{P} \nameeq \quotep{Q} }
\end{mathpar}

The astute reader will have noticed that the mutual recursion of names
and processes imposes a mutual recursion on alpha-equivalence and
structural equivalence via name-equivalence. Fortunately, all of this
works out pleasantly and we may calculate in the natural way, free of
concern. The reader interested in the details is referred to the
appendix \ref{appendix:rho_details}.

\subsection{Substitution}

We use $\Proc$ for the set of processes, $\QProc$ for the set of
names, and $\id{\{}\vec{y} / \vec{x} \id{\}}$ to denote partial maps,
$s : \QProc \rightarrow \QProc$. A map, $s$ lifts, uniquely, to a map
on process terms, $\widehat{s} : \Proc \rightarrow \Proc$ by the
following equations.

\begin{mathpar}
  (0) \psubstp{Q}{P} := 0 \\
  (R \juxtap S) \psubstp{Q}{P}
  :=    
  (R)\psubstp{Q}{P} \juxtap (S) \psubstp{Q}{P} \\
  (x?(y).R) \psubstp{Q}{P}    
  :=    
  (x)\substp{Q}{P} (z)\concat( (R \psubstn{z}{y}) \psubstp{Q}{P} ) \\
  (\lift{x}{R}) \psubstp{Q}{P}  
  :=
  \lift{(x)\substp{Q}{P}}{ R \psubstp{Q}{P} } \\
%   (\dropn{x})  \psubstp{Q}{P}       
%   := 
%   \left\{ 
%     \begin{array}{ccc} 
%       \dropn{\quotep{Q}} & & x \nameeq \quotep{P} \\
%       \dropn{x} & & otherwise \\
%     \end{array}
%   \right. 
  (\dropn{x})  \psubstp{Q}{P}       
  := 
  \left\{ 
    \begin{array}{ccc} 
      Q & & x \nameeq \quotep{P} \\
      \dropn{x} & & otherwise \\
    \end{array}
  \right.
\end{mathpar}
 

where

\begin{eqnarray}
  (x)\id{\{} \lpquote Q \rpquote / \lpquote P \rpquote \id{\}}            = 
  \left\{ 
    \begin{array}{ccc}
      \lpquote Q \rpquote & & x \nameeq \lpquote P \rpquote \\
      x & & otherwise \\
    \end{array}
  \right. \nonumber
\end{eqnarray}

and $z$ is chosen distinct from $\quotep{P}$, $\quotep{Q}$, the free
names in $Q$, and all the names in $R$. Our $\alpha$-equivalence will
be built in the standard way from this substitution.

\begin{remark}\label{rem:no_self_referential_names}
  One consequence of these definitions is that $\forall P. \quotep{P}
  \not\in \freenames{P}$.
\end{remark}

\subsection{ Dynamic quote: an example }

Anticipating something of what's to come, consider applying the
substitution, $\widehat{\id{\{}u / z \id{\}}}$, to the following pair
of processes, $\lift{w}{y!(z)}$ and $w[ \lpquote y!(z) \rpquote ]$.

\begin{eqnarray}
	\lift{w}{y!(z)}\widehat{\id{\{}u / z \id{\}}}
		& = &
		\lift{w}{y!(u)} \nonumber\\
	w[ \lpquote y!(z) \rpquote ] \widehat{ \id{\{}u / z \id{\}} }
		& = &
		w[ \lpquote y!(z) \rpquote ] \nonumber
\end{eqnarray}

Because the body of the process between quotes is impervious to
substitution, we get radically different answers. In fact, by
examining the first process in an input context,
e.g. $x?(z).\lift{w}{y!(z)}$, we see that the process under the lift
operator may be shaped by prefixed inputs binding a name inside it. In
this sense, the lift operator will be seen as a way to dynamically
construct processes before reifying them as names.

Finally equipped with these standard features we can present the
dynamics of the calculus.

\subsubsection{Operational semantics} 

Finally, we introduce the computational dynamics. What marks these
algebras as distinct from other more traditionally studied algebraic
structures, e.g. vector spaces or polynomial rings, is the manner in
which dynamics is captured. In traditional structures, dynamics is typically
expressed through morphisms between such structures, as in linear maps
between vector spaces or morphisms between rings. In algebras
associated with the semantics of computation, the dynamics is
expressed as part of the algebraic structure itself, through a
reduction reduction relation typically denoted by $\red$. Below, we
give a recursive presentation of this relation for the calculus used
in the encoding.

$\red \subseteq \pi \times \pi$
$\red : \pi \to \mathcal{P}(\pi)$

\begin{mathpar}
  \inferrule* [lab=Comm] { \textsf{match}( x_{src}, x_{trgt} ) } { x_{trgt}?(y)P \; | \; x_{src}!\langle {Q} \rangle \red P\{\quotep{Q}/y}\} }
  \and \\
  \inferrule* [lab=Par] {{P} \red {P}'} {{{P} | {Q}} \red {{P}' | {Q}}}
  \and
  \inferrule* [lab=Equiv]{{{P} \scong {P}'} \andalso {{P}' \red {Q}'} \andalso {{Q}' \scong {Q}}}{{P} \red {Q}}
\end{mathpar}

\begin{eqnarray*}
  match_{\equiv} (\quotep{P},\quotep{Q}) & := & P \equiv Q \\
  match_{\dagger}(\quotep{P},\quotep{Q}) & := & \forall R. P|Q \red^{*} R => R \red^{*} 0 \\
  match_{K}(\quotep{P},\quotep{Q}) & := & K \mbox{ for some context } K
\end{eqnarray*}

$u?(x)P | u!\langle Q \rangle \red P\{\quotep{Q}/x\}$

%We write $\wred$ for $\red^*$, and $P\red$ if $\exists Q $ such that $ P \red Q$.
We write $P\red$ if $\exists Q $ such that $ P \red Q$ and $P\not\red$, otherwise.

\section{Replication}

As mentioned before, it is known that replication (and hence
recursion) can be implemented in a higher-order process algebra
\cite{SangiorgiWalker}. As our first example of calculation with the
machinery thus far presented we give the construction explicitly in
the {\rhoc}.

\begin{eqnarray}
	D_{x} & := & \prefix{x}{y}{(\binpar{\outputp{x}{y}}{@{y}})} \nonumber\\
	\bangp_{x}{P} & := & \binpar{{x}!\langle{\binpar{D_{x}}{P}}\rangle}{D_{x}} \nonumber
\end{eqnarray}

\begin{eqnarray}
	\bangp_{x}{P} & & \nonumber\\
	=
	& {x}!\langle{(\prefix{x}{y}{(\outputp{x}{y} | @{y})) | P}}\rangle 
	      | \prefix{x}{y}{(\outputp{x}{y} | @{y})} & \nonumber\\
	\red
	& (\outputp{x}{y} | @{y})\substn{\quotep{(\prefix{x}{y}{(@{y} | \outputp{x}{y})) | P}}}{y} & \nonumber\\
	=
	& \outputp{x}{\quotep{(\prefix{x}{y}{(\outputp{x}{y} | @{y})) | P}}}
	  | {(\prefix{x}{y}{(\outputp{x}{y} | @{y})) | P}} & \nonumber\\
	\red
	& \ldots & \nonumber\\
	\red^*
	& P | P | \ldots & \nonumber
\end{eqnarray}

Of course, this encoding, as an implementation, runs away, unfolding
$\bangp{P}$ eagerly. A lazier and more implementable replication
operator, restricted to input-guarded processes, may be obtained as follows.

\begin{eqnarray}
\bangp{\prefix{u}{v}{P}} 
	:= 
	\binpar{\lift{x}{\prefix{u}{v}{(\binpar{D(x)}{P})}}}{D(x)} \nonumber
\end{eqnarray}

\begin{remark}
  Note that the lazier definition still does not deal with summation
  or mixed summation (i.e. sums over input and output). The reader is
  invited to construct definitions of replication that deal with these
  features. 

  Further, the definitions are parameterized in a name, $x$. Can you,
  gentle reader, make a definition that eliminates this parameter and
  guarantees no accidental interaction between the replication
  machinery and the process being replicated -- i.e. no accidental
  sharing of names used by the process to get its work done and the
  name(s) used by the replication to effect copying. This latter
  revision of the definition of replication is crucial to obtaining
  the expected identity $!!P \sim !P$.
\end{remark}

\begin{remark}\label{rem:paradoxical_combinator}
  The reader familiar with the lambda calculus will have noticed the
  similarity between $D$ and the paradoxical combinator.

  [Ed. note: the existence of this seems to suggest we have to be more
  restrictive on the set of processes and names we admit if we are to
  support no-cloning.]
\end{remark}

\subsubsection{Bisimulation}

The computational dynamics gives rise to another kind of equivalence,
the equivalence of computational behavior. As previously mentioned
this is typically captured \emph{via} some form of bisimulation.

% The notion we use in this paper is weak barbed bisimulation
% \cite{milner91polyadicpi}.

The notion we use in this paper is derived from weak barbed
bisimulation \cite{milner91polyadicpi}. 

\begin{definition}
An \emph{observation relation}, $\downarrow_{\mathcal N}$, over a set
of names, $\mathcal N$, is the smallest relation satisfying the rules
below.

\infrule[Out-barb]{y \in {\mathcal N}, \; x \nameeq y}
		  {\outputp{x}{v} \downarrow_{\mathcal N} x}
\infrule[Par-barb]{\mbox{$P\downarrow_{\mathcal N} x$ or $Q\downarrow_{\mathcal N} x$}}
		  {\binpar{P}{Q} \downarrow_{\mathcal N} x}

We write $P \Downarrow_{\mathcal N} x$ if there is $Q$ such that 
$P \wred Q$ and $Q \downarrow_{\mathcal N} x$.
\end{definition}

\begin{definition}
%\label{def.bbisim}
An  ${\mathcal N}$-\emph{barbed bisimulation} over a set of names, ${\mathcal N}$, is a symmetric binary relation 
${\mathcal S}_{\mathcal N}$ between agents such that $P\rel{S}_{\mathcal N}Q$ implies:
\begin{enumerate}
\item If $P \red P'$ then $Q \wred Q'$ and $P'\rel{S}_{\mathcal N} Q'$.
\item If $P\downarrow_{\mathcal N} x$, then $Q\Downarrow_{\mathcal N} x$.
\end{enumerate}
$P$ is ${\mathcal N}$-barbed bisimilar to $Q$, written
$P \wbbisim_{\mathcal N} Q$, if $P \rel{S}_{\mathcal N} Q$ for some ${\mathcal N}$-barbed bisimulation ${\mathcal S}_{\mathcal N}$.
\end{definition}

$\mathcal{R} \subseteq \pi \times \pi$

$P \mathcal{R} Q => \forall P'. P \red P' \Rightarrow \exists Q'. Q \red Q', P' \mathcal{R} Q'$

$P \vdash x \Rightarrow Q \vdash x$

\begin{mathpar}
  \inferrule*[lab=Out-barb]{x \nameeq y}{{y}!\langle{Q}\rangle \vdash x}
  \and
  \inferrule*[lab=Par-barb]{\mbox{$P\vdash x$ or $Q\vdash x$}}{\binpar{P}{Q} \vdash x}
\end{mathpar}

\subsubsection{Contexts}

One of the principle advantages of computational calculi like the
$\pi$-calculus is a well-defined notion of context,
contextual-equivalence and a correlation between
contextual-equivalence and notions of bisimulation. The notion of
context allows the decomposition of a process into (sub-)process and
its syntactic environment, its context. Thus, a context may be
thought of as a process with a ``hole'' (written $\Box$) in it. The
application of a context $M$ to a process $P$, written $M[P]$, is
tantamount to filling the hole in $M$ with $P$. In this paper we do
not need the full weight of this theory, but do make use of the notion
of context in the proof the main theorem. 

\begin{mathpar}
  \inferrule* [lab=summation] {} {{M_{M},M_{N}} \bc \Box \;|\; x.M_{A} \;|\; M_{M}+M_{N}}
  \and
  \inferrule* [lab=agent] {} {{M_{A}} \bc (\vec{x})M_{P} \;| \; \clift{P_0,\ldots,M_{P},\ldots,P_N}}
  \and \\
  \inferrule* [lab=process] {} {{M_{P}} \bc M_{N} \;| \;P|M_{P} }
\end{mathpar} 

\begin{mathpar}
  \inferrule* [lab=sychronization] {} {M_{N} \bc \Box \;|\; x?M_{F} \;|\; x!M_{C}}
  \and
  \inferrule* [lab=abstraction] {} {{M_{F}} \bc (x)M_{P} }
  \and
  \inferrule* [lab=concretion] {} {{M_{C}} \bc \langle M_{P} \rangle }
  \and \\
  \inferrule* [lab=process] {} {{M_{P}} \bc M_{N} \;| \;P|M_{P} }
\end{mathpar}

\begin{definition}[contextual application] Given a context $M$, and
  process $P$, we define the \emph{contextual application}, $M[P] :=
  M\{P/\Box\}$. That is, the contextual application of M to P is the
  substitution of $P$ for $\Box$ in $M$.
\end{definition}

$\meaningof{-} : L \to \mathcal{P}(\pi)$

\begin{mathpar}
  \inferrule* [lab=collection] {} {\meaningof{true} = \pi, \and \meaningof{~E} = \pi \setminus \meaningof{E}, \and \meaningof{E_{1} \& E_{2}} = \meaningof{E_{1}} \cap \meaningof{E_{2}}}
\end{mathpar}

\begin{mathpar}
  \inferrule* [lab=structure] {} {\meaningof{0} = \{ P \in \pi | P \equiv 0 \}, \and \\ \meaningof{E_1 | E_2} = \{ P \in \pi | P \equiv P_{1} | P_{2}, P_{1} \in \meaningof{E_{1}}, P_{2} \in \meaningof{E_2}\} }
\end{mathpar}

\begin{mathpar}
 \inferrule* [lab=behavior] {} {\meaningof{\langle a?b \rangle E} = \{ P \in \pi | P \equiv Q | u?(y)P', \\ \and \\\\ \and \\ \;\;\; u \in \meaningof{a}, \forall z.P'\{z/y\} \in \meaningof{E\{z/b\}}\}, \and \\ \meaningof{a!E} = \{ P \in \pi | P \equiv Q | x!\langle P' \rangle, x \in \meaningof{a} P' \in \meaningof{E}\} }
\end{mathpar}

\begin{mathpar}
 \inferrule* [lab=nominal] {} {\meaningof{\quotep{E}} = \{ \quotep{P} \in \quotep{\pi} | P \in \meaningof{E} \}, \and \meaningof{\quotep{P}} = \{ \quotep{Q} \in \quotep{\pi} | P \equiv Q \} \and \\ \meaningof{@\quotep{E}} = \{ P \in \pi | P \equiv @x, x \in \meaningof{E} \}}
\end{mathpar}

\begin{eqnarray*}
  \\
  \meaningof{-} : TS \to ST
\end{eqnarray*}

\begin{eqnarray*}
  \\
  L : TS \to ST
\end{eqnarray*}

\begin{eqnarray*}
  \\
  P \models E \iff P \in \meaningof{E}
\end{eqnarray*}

\begin{eqnarray*}
  P \approx_{L} Q \iff \forall E \in L. P \models E \iff Q \models E
\end{eqnarray*}

\begin{eqnarray*}
  P \approx_{K} Q
\end{eqnarray*}

\begin{eqnarray*}
  P \approx Q
\end{eqnarray*}

$\approx_{K} = \approx = \approx_{L}$

\subsubsection{Contextual duality}

Note that contexts extend the quotation operation to a family of
operations from processes to names. Given a context, $M$, we can
define a \emph{nominal context}, $\quotep{M}$ by $\quotep{M}[P] :=
\quotep{M[P]}$. To foreshadow what is to come we observe that these
operations enjoy a duality with processes very much like the duality
between vectors and maps from vectors to scalars.

Further, because the calculus is essentially higher-order, we have a
correspondence between contexts and processes. More specifically,
given a name $x$ and a context $M$ we can construct $M^{*}_{x}$ such
that 

\begin{mathpar}
  M^{*}_{x} | \lift{x}{P} \red M[P]
\end{mathpar}

namely,

\begin{mathpar}
  M^{*}_{x} := x?(u).M[\dropn{u}]
\end{mathpar}

The dependence of $M^{*}_{x}$ on a name makes it an abstraction, 

\begin{mathpar}
  M^{*} := (x)x?(u).M[\dropn{u}]
\end{mathpar}

\subsection{Additional notation}

It will sometimes be convenient to denote the process a name
quotes. We already have the notation $x = \quotep{P}$, but it will be
convenient to introduce an alternate notation, $\procn{x}$, when we
want to emphasize the connection to the use of the name. Note that, by
virtue of name equivalence, $\quotep{\procn{x}} \nameeq x$; so, the
notation is consistent with previous definitions.

Further, because names have structure it is possible to effect
substitutions on the basis of that structure. This means we need to
upgrade our notation for substitutions, which we accomplish by
adapting comprehension notation. Thus,

\begin{mathpar}
  P\{ y / x : x \in S \}
\end{mathpar}

is interpreted to mean the process derived from P by replacing (in a
capture-avoiding manner) each occurrence of $x$ in $S$ by $y$. For example,

\begin{mathpar}
  P\{ \quotep{\procn{x}|\procn{x}} / x : x \in \freenames{P} \}
\end{mathpar}

will replace each (occurrence) of a free name $x$ in $P$ by
$\quotep{\procn{x}|\procn{x}}$.

Also, we will avail ourselves of the notation $x^{L}$ and $x^{R}$ to
denote injections of a name into disjoint copies of the name
space. There are numerous ways to accomplish this. One example can be
found in \cite{MeredithR05}. This notation overloads to vectors of
names: $\vec{x}^{\pi} := (x_{i}^{\pi} \; : \; 0 \leq i < |\vec{x}| )$ where $\pi \in \{L,R\}$.

We also use $P^{\Box} := P|\Box$.

In \cite{MeredithR05} an interpretation of the new operator is
given. It turns out that there are several possible interpretations
all enjoying the requisite algebraic properties of the operator (see
\cite{milner91polyadicpi}). We will therefore make liberal use of
$(\nu\; \vec{x})P$.

% subsection the_syntax_and_semantics_of_the_notation_system (end)   

\input{qm2pi.qmops} 

\input{qm2pi.sterngerlach} 

\input{qm2pi.metric} 

% section concurrent_process_calculi (end)

%\input{qm2pi.proofsketch}

% section proof sketch (end)

%\input{qm2pi.slviaknots} 

% section spatial logic via knots (end)

\input{qm2pi.conclusion}

% section conclusion (end)

%\input{qm2pi.dtcodes} 

% section wiring algorithm (end)

\input{qm2pi.ack} 

% section acknowledgments (end)

\newpage


\bibliographystyle{plain}   
\bibliography{../../biblios/main.bib}

\input{qm2pi.rhodetails}

\end{document}

 

%\ifpdf
%\usepackage[pdftex]{graphicx}
%\else
%\usepackage{graphicx}
%\fi

 % \ifpdf
%  \usepackage{pdfsync}
%  \if


%\title{Brief Article}
%\author{David F. Snyder}
%\author{L.G. Meredith}

%\address{Dept. of Math., Texas State University--San Marcos, San Marcos, TX 78666}
       
\pagestyle{empty}


\begin{document}

\lstset{language=[Objective]Caml,frame=shadowbox}

\documentclass[12pt]{llncs}
%\documentclass{jktr}

\usepackage[pdftex]{hyperref}                   
\usepackage {listings}
\usepackage {mathpartir}
\usepackage{bcprules}
%\usepackage{listings}
                       
\usepackage{graphicx} 
%\usepackage[margins=2.5cm,nohead,nofoot]{geometry}
%\usepackage{geometry}
\usepackage{amsfonts}
\usepackage{amstext}
\usepackage{latexsym}
\usepackage{amssymb}
\usepackage{color}


%\include{myPreamble}
\include{qm2pi.local} 

%\ifpdf
%\usepackage[pdftex]{graphicx}
%\else
%\usepackage{graphicx}
%\fi

 % \ifpdf
%  \usepackage{pdfsync}
%  \if


%\title{Brief Article}
%\author{David F. Snyder}
%\author{L.G. Meredith}

%\address{Dept. of Math., Texas State University--San Marcos, San Marcos, TX 78666}
       
\pagestyle{empty}


\begin{document}

\lstset{language=[Objective]Caml,frame=shadowbox}

\input{qm2pi.front}

% section front matter (end)

\input{qm2pi.intro} 
 
% section introduction (end)

% \input{qm2pi.knotations} 

% section notation (end)

\input{qm2pi.process.calculi} 

% section concurrent_process_calculi_and_spatial_logics_ (end)
    
%\input{qm2pi.knots2pi} 

%\input{qm2pi.trefoil} 

%\input{qm2pi.mainthm} 

% subsection basic_interpretation (end)

%\input{qm2pi.rho.presentation} 
\subsection{The syntax and semantics of the notation system}\label{sub:the_syntax_and_semantics_of_the_notation_system} % (fold)

We now summarize a technical presentation of the calculus that
embodies our theory of dynamics. The typical presentation of such a
calculus follows the style of giving generators and relations on
them. The grammar, below, describing term constructors, freely
generates the set of processes, $\Proc$. This set is then quotiented
by a relation known as structural congruence and it is over this set
that the notion of dynamics is expressed. This presentation is
essentially that of \cite{MeredithR05} with the addition of
polyadicity and summation. For readability we have relegated some of
the technical subtleties to an appendix.

\subsubsection{Process grammar}\label{subsub:process_grammar}

\begin{mathpar}
  \inferrule* [lab=synchronization] {} {{M} \bc \pzero \;|\; x?F \;|\; x!C }
  \and
  \inferrule* [lab=abstraction] {} {{F} \bc (x)P}
  \and
  \inferrule* [lab=concretion] {} {{C} \bc \langle Q \rangle}
  \and
  \inferrule* [lab=process] {} {{P,Q} \bc M \;| \;P|Q \;|\; @{x}}
  \and
  \inferrule* [lab=name] {} {{x} \bc \quotep{P}}
\end{mathpar} 

Note that $\vec{x}$ (resp. $\vec{P}$) denotes a vector of names
(resp. processes) of length $|\vec{x}|$ (resp. $|\vec{P}|$). We adopt
the following useful abbreviations.

\begin{mathpar}
   x?(\vec{y}).P := x.(\vec{y})P \and  x\clift{\vec{P}} := x.\clift{\vec{P}}
   \and x!(y) := \lift{x}{\dropn{y}}
   \and \Pi_{i=0}^{n-1}P_i := P_0 | \ldots | P_{n-1}
\end{mathpar}

\subsubsection{Structural congruence}

\paragraph{Free and bound names and alpha-equivalence.} At the
core of structural equivalence is alpha-equivalence which identifies
process that are the same up to a change of variable. Formally, we
recognize the distinction between free and bound names. The free names
of a process, $\freenames{P}$, may be calculated recursively as
follows:

\begin{mathpar}
\freenames{\pzero} := \emptyset
  \and \\
  \freenames{x?(y).P} := \{ x \} \cup (\freenames{P} \setminus \{ y \})
  \and 
  \freenames{x!\langle P \rangle} := \{ x \} \cup \{ P \} 
  \and \\
  \freenames{P|Q} := \freenames{P} \cup \freenames{Q}
  \and \\
  \freenames{@{x}} := \{ x \}
\end{mathpar}

$\pi$
$\quotep{\pi}$

$\freenames{-} : \pi \to \mathcal{P}(\quotep{\pi})$

\begin{eqnarray*}
  \freenames{\pzero} & := & \emptyset \\
  \freenames{x?(y).P} & := & \{ x \} \cup (\freenames{P} \setminus \{ y \}) \\
  \freenames{x!\langle P \rangle} & := & \{ x \} \cup \{ P \} \\
  \freenames{P|Q} & := & \freenames{P} \cup \freenames{Q} \\
  \freenames{\dropn{x}} & := & \{ x \}
\end{eqnarray*}

The bound names of a process, $\boundnames{P}$, are those names occurring in $P$
that are not free. For example, in $x?(y).0$, the name $x$ is free, while $y$ is bound.

\begin{mathpar}
  \inferrule* [lab=monoidal-laws] {} { P|Q \equiv Q|P \and P|0 \equiv P \and P|(Q|R) \equiv (P|Q)|R }
\end{mathpar}

\begin{mathpar}
  \inferrule* [lab=alpha-equivalence] {} { (x)P \equiv (y)P\{y/x\} \and y \not\in \freenames{P} }
\end{mathpar}

\begin{definition}
Then two processes, $P,Q$, are alpha-equivalent if $P = Q\{\vec{y}/\vec{x}\}$ for
some $\vec{x} \in \boundnames{Q},\vec{y} \in \boundnames{P}$, where $Q\{\vec{y}/\vec{x}\}$
denotes the capture-avoiding substitution of $\vec{y}$ for $\vec{x}$ in $Q$.
\end{definition}

\begin{definition}
  The {\em structural congruence} \cite{SangiorgiWalker} , $\equiv$,
  between processes is the least congruence containing
  alpha-equivalence, satisfying the abelian monoid laws
  (associativity, commutativity and $\pzero$ as identity) for parallel
  composition $|$ and for summation $+$.
\end{definition}

\subsection{Name equivalence}

We take name equivalence, written $\nameeq$, to be the smallest
equivalence relation generated by the following rules.

\begin{mathpar}
\inferrule*[lab=Quote-drop]
{ }
{ \quotep{@{x}} \nameeq x }

\inferrule*[lab=Struct-equiv]
{ P \scong Q }
{ \quotep{P} \nameeq \quotep{Q} }
\end{mathpar}

The astute reader will have noticed that the mutual recursion of names
and processes imposes a mutual recursion on alpha-equivalence and
structural equivalence via name-equivalence. Fortunately, all of this
works out pleasantly and we may calculate in the natural way, free of
concern. The reader interested in the details is referred to the
appendix \ref{appendix:rho_details}.

\subsection{Substitution}

We use $\Proc$ for the set of processes, $\QProc$ for the set of
names, and $\id{\{}\vec{y} / \vec{x} \id{\}}$ to denote partial maps,
$s : \QProc \rightarrow \QProc$. A map, $s$ lifts, uniquely, to a map
on process terms, $\widehat{s} : \Proc \rightarrow \Proc$ by the
following equations.

\begin{mathpar}
  (0) \psubstp{Q}{P} := 0 \\
  (R \juxtap S) \psubstp{Q}{P}
  :=    
  (R)\psubstp{Q}{P} \juxtap (S) \psubstp{Q}{P} \\
  (x?(y).R) \psubstp{Q}{P}    
  :=    
  (x)\substp{Q}{P} (z)\concat( (R \psubstn{z}{y}) \psubstp{Q}{P} ) \\
  (\lift{x}{R}) \psubstp{Q}{P}  
  :=
  \lift{(x)\substp{Q}{P}}{ R \psubstp{Q}{P} } \\
%   (\dropn{x})  \psubstp{Q}{P}       
%   := 
%   \left\{ 
%     \begin{array}{ccc} 
%       \dropn{\quotep{Q}} & & x \nameeq \quotep{P} \\
%       \dropn{x} & & otherwise \\
%     \end{array}
%   \right. 
  (\dropn{x})  \psubstp{Q}{P}       
  := 
  \left\{ 
    \begin{array}{ccc} 
      Q & & x \nameeq \quotep{P} \\
      \dropn{x} & & otherwise \\
    \end{array}
  \right.
\end{mathpar}
 

where

\begin{eqnarray}
  (x)\id{\{} \lpquote Q \rpquote / \lpquote P \rpquote \id{\}}            = 
  \left\{ 
    \begin{array}{ccc}
      \lpquote Q \rpquote & & x \nameeq \lpquote P \rpquote \\
      x & & otherwise \\
    \end{array}
  \right. \nonumber
\end{eqnarray}

and $z$ is chosen distinct from $\quotep{P}$, $\quotep{Q}$, the free
names in $Q$, and all the names in $R$. Our $\alpha$-equivalence will
be built in the standard way from this substitution.

\begin{remark}\label{rem:no_self_referential_names}
  One consequence of these definitions is that $\forall P. \quotep{P}
  \not\in \freenames{P}$.
\end{remark}

\subsection{ Dynamic quote: an example }

Anticipating something of what's to come, consider applying the
substitution, $\widehat{\id{\{}u / z \id{\}}}$, to the following pair
of processes, $\lift{w}{y!(z)}$ and $w[ \lpquote y!(z) \rpquote ]$.

\begin{eqnarray}
	\lift{w}{y!(z)}\widehat{\id{\{}u / z \id{\}}}
		& = &
		\lift{w}{y!(u)} \nonumber\\
	w[ \lpquote y!(z) \rpquote ] \widehat{ \id{\{}u / z \id{\}} }
		& = &
		w[ \lpquote y!(z) \rpquote ] \nonumber
\end{eqnarray}

Because the body of the process between quotes is impervious to
substitution, we get radically different answers. In fact, by
examining the first process in an input context,
e.g. $x?(z).\lift{w}{y!(z)}$, we see that the process under the lift
operator may be shaped by prefixed inputs binding a name inside it. In
this sense, the lift operator will be seen as a way to dynamically
construct processes before reifying them as names.

Finally equipped with these standard features we can present the
dynamics of the calculus.

\subsubsection{Operational semantics} 

Finally, we introduce the computational dynamics. What marks these
algebras as distinct from other more traditionally studied algebraic
structures, e.g. vector spaces or polynomial rings, is the manner in
which dynamics is captured. In traditional structures, dynamics is typically
expressed through morphisms between such structures, as in linear maps
between vector spaces or morphisms between rings. In algebras
associated with the semantics of computation, the dynamics is
expressed as part of the algebraic structure itself, through a
reduction reduction relation typically denoted by $\red$. Below, we
give a recursive presentation of this relation for the calculus used
in the encoding.

$\red \subseteq \pi \times \pi$
$\red : \pi \to \mathcal{P}(\pi)$

\begin{mathpar}
  \inferrule* [lab=Comm] { \textsf{match}( x_{src}, x_{trgt} ) } { x_{trgt}?(y)P \; | \; x_{src}!\langle {Q} \rangle \red P\{\quotep{Q}/y}\} }
  \and \\
  \inferrule* [lab=Par] {{P} \red {P}'} {{{P} | {Q}} \red {{P}' | {Q}}}
  \and
  \inferrule* [lab=Equiv]{{{P} \scong {P}'} \andalso {{P}' \red {Q}'} \andalso {{Q}' \scong {Q}}}{{P} \red {Q}}
\end{mathpar}

\begin{eqnarray*}
  match_{\equiv} (\quotep{P},\quotep{Q}) & := & P \equiv Q \\
  match_{\dagger}(\quotep{P},\quotep{Q}) & := & \forall R. P|Q \red^{*} R => R \red^{*} 0 \\
  match_{K}(\quotep{P},\quotep{Q}) & := & K \mbox{ for some context } K
\end{eqnarray*}

$u?(x)P | u!\langle Q \rangle \red P\{\quotep{Q}/x\}$

%We write $\wred$ for $\red^*$, and $P\red$ if $\exists Q $ such that $ P \red Q$.
We write $P\red$ if $\exists Q $ such that $ P \red Q$ and $P\not\red$, otherwise.

\section{Replication}

As mentioned before, it is known that replication (and hence
recursion) can be implemented in a higher-order process algebra
\cite{SangiorgiWalker}. As our first example of calculation with the
machinery thus far presented we give the construction explicitly in
the {\rhoc}.

\begin{eqnarray}
	D_{x} & := & \prefix{x}{y}{(\binpar{\outputp{x}{y}}{@{y}})} \nonumber\\
	\bangp_{x}{P} & := & \binpar{{x}!\langle{\binpar{D_{x}}{P}}\rangle}{D_{x}} \nonumber
\end{eqnarray}

\begin{eqnarray}
	\bangp_{x}{P} & & \nonumber\\
	=
	& {x}!\langle{(\prefix{x}{y}{(\outputp{x}{y} | @{y})) | P}}\rangle 
	      | \prefix{x}{y}{(\outputp{x}{y} | @{y})} & \nonumber\\
	\red
	& (\outputp{x}{y} | @{y})\substn{\quotep{(\prefix{x}{y}{(@{y} | \outputp{x}{y})) | P}}}{y} & \nonumber\\
	=
	& \outputp{x}{\quotep{(\prefix{x}{y}{(\outputp{x}{y} | @{y})) | P}}}
	  | {(\prefix{x}{y}{(\outputp{x}{y} | @{y})) | P}} & \nonumber\\
	\red
	& \ldots & \nonumber\\
	\red^*
	& P | P | \ldots & \nonumber
\end{eqnarray}

Of course, this encoding, as an implementation, runs away, unfolding
$\bangp{P}$ eagerly. A lazier and more implementable replication
operator, restricted to input-guarded processes, may be obtained as follows.

\begin{eqnarray}
\bangp{\prefix{u}{v}{P}} 
	:= 
	\binpar{\lift{x}{\prefix{u}{v}{(\binpar{D(x)}{P})}}}{D(x)} \nonumber
\end{eqnarray}

\begin{remark}
  Note that the lazier definition still does not deal with summation
  or mixed summation (i.e. sums over input and output). The reader is
  invited to construct definitions of replication that deal with these
  features. 

  Further, the definitions are parameterized in a name, $x$. Can you,
  gentle reader, make a definition that eliminates this parameter and
  guarantees no accidental interaction between the replication
  machinery and the process being replicated -- i.e. no accidental
  sharing of names used by the process to get its work done and the
  name(s) used by the replication to effect copying. This latter
  revision of the definition of replication is crucial to obtaining
  the expected identity $!!P \sim !P$.
\end{remark}

\begin{remark}\label{rem:paradoxical_combinator}
  The reader familiar with the lambda calculus will have noticed the
  similarity between $D$ and the paradoxical combinator.

  [Ed. note: the existence of this seems to suggest we have to be more
  restrictive on the set of processes and names we admit if we are to
  support no-cloning.]
\end{remark}

\subsubsection{Bisimulation}

The computational dynamics gives rise to another kind of equivalence,
the equivalence of computational behavior. As previously mentioned
this is typically captured \emph{via} some form of bisimulation.

% The notion we use in this paper is weak barbed bisimulation
% \cite{milner91polyadicpi}.

The notion we use in this paper is derived from weak barbed
bisimulation \cite{milner91polyadicpi}. 

\begin{definition}
An \emph{observation relation}, $\downarrow_{\mathcal N}$, over a set
of names, $\mathcal N$, is the smallest relation satisfying the rules
below.

\infrule[Out-barb]{y \in {\mathcal N}, \; x \nameeq y}
		  {\outputp{x}{v} \downarrow_{\mathcal N} x}
\infrule[Par-barb]{\mbox{$P\downarrow_{\mathcal N} x$ or $Q\downarrow_{\mathcal N} x$}}
		  {\binpar{P}{Q} \downarrow_{\mathcal N} x}

We write $P \Downarrow_{\mathcal N} x$ if there is $Q$ such that 
$P \wred Q$ and $Q \downarrow_{\mathcal N} x$.
\end{definition}

\begin{definition}
%\label{def.bbisim}
An  ${\mathcal N}$-\emph{barbed bisimulation} over a set of names, ${\mathcal N}$, is a symmetric binary relation 
${\mathcal S}_{\mathcal N}$ between agents such that $P\rel{S}_{\mathcal N}Q$ implies:
\begin{enumerate}
\item If $P \red P'$ then $Q \wred Q'$ and $P'\rel{S}_{\mathcal N} Q'$.
\item If $P\downarrow_{\mathcal N} x$, then $Q\Downarrow_{\mathcal N} x$.
\end{enumerate}
$P$ is ${\mathcal N}$-barbed bisimilar to $Q$, written
$P \wbbisim_{\mathcal N} Q$, if $P \rel{S}_{\mathcal N} Q$ for some ${\mathcal N}$-barbed bisimulation ${\mathcal S}_{\mathcal N}$.
\end{definition}

$\mathcal{R} \subseteq \pi \times \pi$

$P \mathcal{R} Q => \forall P'. P \red P' \Rightarrow \exists Q'. Q \red Q', P' \mathcal{R} Q'$

$P \vdash x \Rightarrow Q \vdash x$

\begin{mathpar}
  \inferrule*[lab=Out-barb]{x \nameeq y}{{y}!\langle{Q}\rangle \vdash x}
  \and
  \inferrule*[lab=Par-barb]{\mbox{$P\vdash x$ or $Q\vdash x$}}{\binpar{P}{Q} \vdash x}
\end{mathpar}

\subsubsection{Contexts}

One of the principle advantages of computational calculi like the
$\pi$-calculus is a well-defined notion of context,
contextual-equivalence and a correlation between
contextual-equivalence and notions of bisimulation. The notion of
context allows the decomposition of a process into (sub-)process and
its syntactic environment, its context. Thus, a context may be
thought of as a process with a ``hole'' (written $\Box$) in it. The
application of a context $M$ to a process $P$, written $M[P]$, is
tantamount to filling the hole in $M$ with $P$. In this paper we do
not need the full weight of this theory, but do make use of the notion
of context in the proof the main theorem. 

\begin{mathpar}
  \inferrule* [lab=summation] {} {{M_{M},M_{N}} \bc \Box \;|\; x.M_{A} \;|\; M_{M}+M_{N}}
  \and
  \inferrule* [lab=agent] {} {{M_{A}} \bc (\vec{x})M_{P} \;| \; \clift{P_0,\ldots,M_{P},\ldots,P_N}}
  \and \\
  \inferrule* [lab=process] {} {{M_{P}} \bc M_{N} \;| \;P|M_{P} }
\end{mathpar} 

\begin{mathpar}
  \inferrule* [lab=sychronization] {} {M_{N} \bc \Box \;|\; x?M_{F} \;|\; x!M_{C}}
  \and
  \inferrule* [lab=abstraction] {} {{M_{F}} \bc (x)M_{P} }
  \and
  \inferrule* [lab=concretion] {} {{M_{C}} \bc \langle M_{P} \rangle }
  \and \\
  \inferrule* [lab=process] {} {{M_{P}} \bc M_{N} \;| \;P|M_{P} }
\end{mathpar}

\begin{definition}[contextual application] Given a context $M$, and
  process $P$, we define the \emph{contextual application}, $M[P] :=
  M\{P/\Box\}$. That is, the contextual application of M to P is the
  substitution of $P$ for $\Box$ in $M$.
\end{definition}

$\meaningof{-} : L \to \mathcal{P}(\pi)$

\begin{mathpar}
  \inferrule* [lab=collection] {} {\meaningof{true} = \pi, \and \meaningof{~E} = \pi \setminus \meaningof{E}, \and \meaningof{E_{1} \& E_{2}} = \meaningof{E_{1}} \cap \meaningof{E_{2}}}
\end{mathpar}

\begin{mathpar}
  \inferrule* [lab=structure] {} {\meaningof{0} = \{ P \in \pi | P \equiv 0 \}, \and \\ \meaningof{E_1 | E_2} = \{ P \in \pi | P \equiv P_{1} | P_{2}, P_{1} \in \meaningof{E_{1}}, P_{2} \in \meaningof{E_2}\} }
\end{mathpar}

\begin{mathpar}
 \inferrule* [lab=behavior] {} {\meaningof{\langle a?b \rangle E} = \{ P \in \pi | P \equiv Q | u?(y)P', \\ \and \\\\ \and \\ \;\;\; u \in \meaningof{a}, \forall z.P'\{z/y\} \in \meaningof{E\{z/b\}}\}, \and \\ \meaningof{a!E} = \{ P \in \pi | P \equiv Q | x!\langle P' \rangle, x \in \meaningof{a} P' \in \meaningof{E}\} }
\end{mathpar}

\begin{mathpar}
 \inferrule* [lab=nominal] {} {\meaningof{\quotep{E}} = \{ \quotep{P} \in \quotep{\pi} | P \in \meaningof{E} \}, \and \meaningof{\quotep{P}} = \{ \quotep{Q} \in \quotep{\pi} | P \equiv Q \} \and \\ \meaningof{@\quotep{E}} = \{ P \in \pi | P \equiv @x, x \in \meaningof{E} \}}
\end{mathpar}

\begin{eqnarray*}
  \\
  \meaningof{-} : TS \to ST
\end{eqnarray*}

\begin{eqnarray*}
  \\
  L : TS \to ST
\end{eqnarray*}

\begin{eqnarray*}
  \\
  P \models E \iff P \in \meaningof{E}
\end{eqnarray*}

\begin{eqnarray*}
  P \approx_{L} Q \iff \forall E \in L. P \models E \iff Q \models E
\end{eqnarray*}

\begin{eqnarray*}
  P \approx_{K} Q
\end{eqnarray*}

\begin{eqnarray*}
  P \approx Q
\end{eqnarray*}

$\approx_{K} = \approx = \approx_{L}$

\subsubsection{Contextual duality}

Note that contexts extend the quotation operation to a family of
operations from processes to names. Given a context, $M$, we can
define a \emph{nominal context}, $\quotep{M}$ by $\quotep{M}[P] :=
\quotep{M[P]}$. To foreshadow what is to come we observe that these
operations enjoy a duality with processes very much like the duality
between vectors and maps from vectors to scalars.

Further, because the calculus is essentially higher-order, we have a
correspondence between contexts and processes. More specifically,
given a name $x$ and a context $M$ we can construct $M^{*}_{x}$ such
that 

\begin{mathpar}
  M^{*}_{x} | \lift{x}{P} \red M[P]
\end{mathpar}

namely,

\begin{mathpar}
  M^{*}_{x} := x?(u).M[\dropn{u}]
\end{mathpar}

The dependence of $M^{*}_{x}$ on a name makes it an abstraction, 

\begin{mathpar}
  M^{*} := (x)x?(u).M[\dropn{u}]
\end{mathpar}

\subsection{Additional notation}

It will sometimes be convenient to denote the process a name
quotes. We already have the notation $x = \quotep{P}$, but it will be
convenient to introduce an alternate notation, $\procn{x}$, when we
want to emphasize the connection to the use of the name. Note that, by
virtue of name equivalence, $\quotep{\procn{x}} \nameeq x$; so, the
notation is consistent with previous definitions.

Further, because names have structure it is possible to effect
substitutions on the basis of that structure. This means we need to
upgrade our notation for substitutions, which we accomplish by
adapting comprehension notation. Thus,

\begin{mathpar}
  P\{ y / x : x \in S \}
\end{mathpar}

is interpreted to mean the process derived from P by replacing (in a
capture-avoiding manner) each occurrence of $x$ in $S$ by $y$. For example,

\begin{mathpar}
  P\{ \quotep{\procn{x}|\procn{x}} / x : x \in \freenames{P} \}
\end{mathpar}

will replace each (occurrence) of a free name $x$ in $P$ by
$\quotep{\procn{x}|\procn{x}}$.

Also, we will avail ourselves of the notation $x^{L}$ and $x^{R}$ to
denote injections of a name into disjoint copies of the name
space. There are numerous ways to accomplish this. One example can be
found in \cite{MeredithR05}. This notation overloads to vectors of
names: $\vec{x}^{\pi} := (x_{i}^{\pi} \; : \; 0 \leq i < |\vec{x}| )$ where $\pi \in \{L,R\}$.

We also use $P^{\Box} := P|\Box$.

In \cite{MeredithR05} an interpretation of the new operator is
given. It turns out that there are several possible interpretations
all enjoying the requisite algebraic properties of the operator (see
\cite{milner91polyadicpi}). We will therefore make liberal use of
$(\nu\; \vec{x})P$.

% subsection the_syntax_and_semantics_of_the_notation_system (end)   

\input{qm2pi.qmops} 

\input{qm2pi.sterngerlach} 

\input{qm2pi.metric} 

% section concurrent_process_calculi (end)

%\input{qm2pi.proofsketch}

% section proof sketch (end)

%\input{qm2pi.slviaknots} 

% section spatial logic via knots (end)

\input{qm2pi.conclusion}

% section conclusion (end)

%\input{qm2pi.dtcodes} 

% section wiring algorithm (end)

\input{qm2pi.ack} 

% section acknowledgments (end)

\newpage


\bibliographystyle{plain}   
\bibliography{../../biblios/main.bib}

\input{qm2pi.rhodetails}

\end{document}



% section front matter (end)

\section{Introduction}\label{sec:introduction} % (fold)
In this draft of the material i am going to have to dispense with the
usual writing conventions adopted in papers on these topics. i'm going
to have adopt whatever tone i need at the time i'm writing up the
calculations. Sometimes this may be very conversational; others it may
be the barest mathematical grunts; others still it may be that i have
lifted text from one of my other papers because the exposition of some
point was better said there. i hope that my readers are not unduly put
out by this decision. i'm not doing this to flout convention or be
rebellious. i find these calculations very technically challenging. To
keep everything going technically, something has to give; i have to
let go of some cognitive burden. So, the academic writing style --
with all of its trade-offs in terms of facilitating technical
communication -- is what i'm letting go of. Perhaps subsequent drafts
can be tightened and polished, but for now, i'm going to speak as if
we were sitting together in a coffee shop with a laptop, wifi and a
pad of paper and a pencil.

So, here's what i have to say. We -- you and i, comfortably ensconced
in our coffee shop and well-equipped with our tools -- can realize and
carry out the calculations of quantum mechanics over a very different
formal theory of dynamics, a formal theory of dynamics that
corresponds to a theory of concurrent computation with
\emph{reflection}. It has the advantage that the underlying theory is
already `quantized', but supports analogues all of the continuuous
operations. Strikingly, this underlying theory has recently been
connected with a notion of metric that we can show, by calculating
together, coincides with the metric induced by the inner product.

There are a lot of reasons why you might be interested in seeing
calculations of this form. Here's why i'm interested. For the past
several centuries there has been no competitor to the ``Newtonian''
account of dynamics. As a result the predominant share of accounts of
dynamical systems and situations have had to be formulated in terms of
the Newtonian machinery. i view this as an intellectually dangerous
position to occupy. Everything, despite it's intrinsic shape, turns
into a nail to be hit with this hammer. Recently, however, the theory
of computation has matured to the point where we have candidates for
theories of dynamics that offer very different perspective on
reasoning about dynamical systems and situations. Testing these
candidates against very successful accounts of dynamical situations,
like quantum mechanics, is going to give us some sense of how mature
they are and some measure of the quality of these accounts of
dynamics.

\subsection{Summary of contributions and outline of paper}

So, we're going to develop an interpretation of the operations of
quantum mechanics normally interpreted by Hilbert spaces and
operators. We're going to do this over a theory of computation. Note
that this is very different than the usual quantum computation program
which develops notions of computation over quantum mechanics. Rather,
we are developing a story that aligns with Wheeler's slogan: It from
Bit. To do this we will first provide an account of the theory of
computation at play here. Then we will dive into a calculation-driven
interpretation of the operations of quantum mechanics.

The reason we take this approach is that -- until very recently --
there hasn't been an axiomatic account of quantum mechanics. As a
result there has been no sharp delineation of the mathematical theory
supporting interpretation of the physical theory and the physical
theory, itself. So, ambient features of the maths are free to be
exploited (or supressed) without a real accounting of their physical
relevance. There is no sharp statement ``here's the physical theory''
qua \emph{theory} and ``here's the mathematical interpretation''
enabling a judgment of how faithful the interpretation is -- apart
from experimental observation. When there is an axiomatic account we
can judge how well a given mathematical formalism supports an
interpretation of the axioms, independent of
experimentation. Likewise, we can judge how well we have captured our
physical evidence and experience with our axiomatics, independent of
any specific mathematical implementation, with accidental detail that
may or may not have physical significance. 

In lieu of a fully fleshed out and vetted axiomatic account of quantum
mechanics, interpreting the operational notions in service of modeling
physical systems will have to suffice. In other words, we are not in
the business of providing a model of Hilbert spaces and operators. We
are in the business of providing a model of quantum mechanics because
we are motivated by testing our notions of dynamics against physical
theory; and, the predictive calculations of the physical theory must
serve as the best formulation -- shy of a fully fleshed out axiomatic
account -- of the physical theory itself (as they have for scientific
theories since time immemorial). Put another way, despite a
whole-hearted commitment to an It-from-Bit ontology, we are firmly
aligned with the shut-up-and-calculate camp as the best way to obtain
results either from the physical perspective or as a quality assurance
measure of our fledgling theory of dynamics.

In detail, we present a reflective process calculus. Then we develop
intuitive correspondences between the notions available in this
calculus and the usual physical notions supporting quantum mechanical
calculations. Thus, 

\begin{table}[htp]
  \center{
    \fbox{
      \begin{tabular}{c|c}
        quantum mechanics & process calculus \\
        \hline
        scalar & name \\
        state vector & process \\
        dual & contextual duals \\
        matrix & formal sums of process-context-dual pairs \\
        orthogonality & process annihilation \\
        inner product & execution-formula + quoting
      \end{tabular}
    }
  }
  \caption{QM - process calculi correspondences}
\end{table}

Then we tighten up these intuitions to operational definitions. We
employ the Dirac notation as the best proxy we can find for an
abstract syntax of the quantum mechanical notions. The definitions we
develop put us in contact with equational constraints coming from the
theory that we demonstrate the definitions and calculations satisfy.

This puts us in a position to shut up and calculate for the
Stern-Gerlach experimental set up, showing how these predictive
calculations become calculations on processes in our theory of a
reflective process calculus.

Penultimately, we demonstrate that the notion of metric coming from
the inner product coincides with the notion of metric available from
the theory of bisimulation. This demonstration gives us the right to
think of space as arising from behavior. Finally, we consider where we
might go from the new vantage point we have obtained.

% section introduction (end) 
 
% section introduction (end)

% \documentclass[12pt]{llncs}
%\documentclass{jktr}

\usepackage[pdftex]{hyperref}                   
\usepackage {listings}
\usepackage {mathpartir}
\usepackage{bcprules}
%\usepackage{listings}
                       
\usepackage{graphicx} 
%\usepackage[margins=2.5cm,nohead,nofoot]{geometry}
%\usepackage{geometry}
\usepackage{amsfonts}
\usepackage{amstext}
\usepackage{latexsym}
\usepackage{amssymb}
\usepackage{color}


%\include{myPreamble}
\include{qm2pi.local} 

%\ifpdf
%\usepackage[pdftex]{graphicx}
%\else
%\usepackage{graphicx}
%\fi

 % \ifpdf
%  \usepackage{pdfsync}
%  \if


%\title{Brief Article}
%\author{David F. Snyder}
%\author{L.G. Meredith}

%\address{Dept. of Math., Texas State University--San Marcos, San Marcos, TX 78666}
       
\pagestyle{empty}


\begin{document}

\lstset{language=[Objective]Caml,frame=shadowbox}

\input{qm2pi.front}

% section front matter (end)

\input{qm2pi.intro} 
 
% section introduction (end)

% \input{qm2pi.knotations} 

% section notation (end)

\input{qm2pi.process.calculi} 

% section concurrent_process_calculi_and_spatial_logics_ (end)
    
%\input{qm2pi.knots2pi} 

%\input{qm2pi.trefoil} 

%\input{qm2pi.mainthm} 

% subsection basic_interpretation (end)

%\input{qm2pi.rho.presentation} 
\subsection{The syntax and semantics of the notation system}\label{sub:the_syntax_and_semantics_of_the_notation_system} % (fold)

We now summarize a technical presentation of the calculus that
embodies our theory of dynamics. The typical presentation of such a
calculus follows the style of giving generators and relations on
them. The grammar, below, describing term constructors, freely
generates the set of processes, $\Proc$. This set is then quotiented
by a relation known as structural congruence and it is over this set
that the notion of dynamics is expressed. This presentation is
essentially that of \cite{MeredithR05} with the addition of
polyadicity and summation. For readability we have relegated some of
the technical subtleties to an appendix.

\subsubsection{Process grammar}\label{subsub:process_grammar}

\begin{mathpar}
  \inferrule* [lab=synchronization] {} {{M} \bc \pzero \;|\; x?F \;|\; x!C }
  \and
  \inferrule* [lab=abstraction] {} {{F} \bc (x)P}
  \and
  \inferrule* [lab=concretion] {} {{C} \bc \langle Q \rangle}
  \and
  \inferrule* [lab=process] {} {{P,Q} \bc M \;| \;P|Q \;|\; @{x}}
  \and
  \inferrule* [lab=name] {} {{x} \bc \quotep{P}}
\end{mathpar} 

Note that $\vec{x}$ (resp. $\vec{P}$) denotes a vector of names
(resp. processes) of length $|\vec{x}|$ (resp. $|\vec{P}|$). We adopt
the following useful abbreviations.

\begin{mathpar}
   x?(\vec{y}).P := x.(\vec{y})P \and  x\clift{\vec{P}} := x.\clift{\vec{P}}
   \and x!(y) := \lift{x}{\dropn{y}}
   \and \Pi_{i=0}^{n-1}P_i := P_0 | \ldots | P_{n-1}
\end{mathpar}

\subsubsection{Structural congruence}

\paragraph{Free and bound names and alpha-equivalence.} At the
core of structural equivalence is alpha-equivalence which identifies
process that are the same up to a change of variable. Formally, we
recognize the distinction between free and bound names. The free names
of a process, $\freenames{P}$, may be calculated recursively as
follows:

\begin{mathpar}
\freenames{\pzero} := \emptyset
  \and \\
  \freenames{x?(y).P} := \{ x \} \cup (\freenames{P} \setminus \{ y \})
  \and 
  \freenames{x!\langle P \rangle} := \{ x \} \cup \{ P \} 
  \and \\
  \freenames{P|Q} := \freenames{P} \cup \freenames{Q}
  \and \\
  \freenames{@{x}} := \{ x \}
\end{mathpar}

$\pi$
$\quotep{\pi}$

$\freenames{-} : \pi \to \mathcal{P}(\quotep{\pi})$

\begin{eqnarray*}
  \freenames{\pzero} & := & \emptyset \\
  \freenames{x?(y).P} & := & \{ x \} \cup (\freenames{P} \setminus \{ y \}) \\
  \freenames{x!\langle P \rangle} & := & \{ x \} \cup \{ P \} \\
  \freenames{P|Q} & := & \freenames{P} \cup \freenames{Q} \\
  \freenames{\dropn{x}} & := & \{ x \}
\end{eqnarray*}

The bound names of a process, $\boundnames{P}$, are those names occurring in $P$
that are not free. For example, in $x?(y).0$, the name $x$ is free, while $y$ is bound.

\begin{mathpar}
  \inferrule* [lab=monoidal-laws] {} { P|Q \equiv Q|P \and P|0 \equiv P \and P|(Q|R) \equiv (P|Q)|R }
\end{mathpar}

\begin{mathpar}
  \inferrule* [lab=alpha-equivalence] {} { (x)P \equiv (y)P\{y/x\} \and y \not\in \freenames{P} }
\end{mathpar}

\begin{definition}
Then two processes, $P,Q$, are alpha-equivalent if $P = Q\{\vec{y}/\vec{x}\}$ for
some $\vec{x} \in \boundnames{Q},\vec{y} \in \boundnames{P}$, where $Q\{\vec{y}/\vec{x}\}$
denotes the capture-avoiding substitution of $\vec{y}$ for $\vec{x}$ in $Q$.
\end{definition}

\begin{definition}
  The {\em structural congruence} \cite{SangiorgiWalker} , $\equiv$,
  between processes is the least congruence containing
  alpha-equivalence, satisfying the abelian monoid laws
  (associativity, commutativity and $\pzero$ as identity) for parallel
  composition $|$ and for summation $+$.
\end{definition}

\subsection{Name equivalence}

We take name equivalence, written $\nameeq$, to be the smallest
equivalence relation generated by the following rules.

\begin{mathpar}
\inferrule*[lab=Quote-drop]
{ }
{ \quotep{@{x}} \nameeq x }

\inferrule*[lab=Struct-equiv]
{ P \scong Q }
{ \quotep{P} \nameeq \quotep{Q} }
\end{mathpar}

The astute reader will have noticed that the mutual recursion of names
and processes imposes a mutual recursion on alpha-equivalence and
structural equivalence via name-equivalence. Fortunately, all of this
works out pleasantly and we may calculate in the natural way, free of
concern. The reader interested in the details is referred to the
appendix \ref{appendix:rho_details}.

\subsection{Substitution}

We use $\Proc$ for the set of processes, $\QProc$ for the set of
names, and $\id{\{}\vec{y} / \vec{x} \id{\}}$ to denote partial maps,
$s : \QProc \rightarrow \QProc$. A map, $s$ lifts, uniquely, to a map
on process terms, $\widehat{s} : \Proc \rightarrow \Proc$ by the
following equations.

\begin{mathpar}
  (0) \psubstp{Q}{P} := 0 \\
  (R \juxtap S) \psubstp{Q}{P}
  :=    
  (R)\psubstp{Q}{P} \juxtap (S) \psubstp{Q}{P} \\
  (x?(y).R) \psubstp{Q}{P}    
  :=    
  (x)\substp{Q}{P} (z)\concat( (R \psubstn{z}{y}) \psubstp{Q}{P} ) \\
  (\lift{x}{R}) \psubstp{Q}{P}  
  :=
  \lift{(x)\substp{Q}{P}}{ R \psubstp{Q}{P} } \\
%   (\dropn{x})  \psubstp{Q}{P}       
%   := 
%   \left\{ 
%     \begin{array}{ccc} 
%       \dropn{\quotep{Q}} & & x \nameeq \quotep{P} \\
%       \dropn{x} & & otherwise \\
%     \end{array}
%   \right. 
  (\dropn{x})  \psubstp{Q}{P}       
  := 
  \left\{ 
    \begin{array}{ccc} 
      Q & & x \nameeq \quotep{P} \\
      \dropn{x} & & otherwise \\
    \end{array}
  \right.
\end{mathpar}
 

where

\begin{eqnarray}
  (x)\id{\{} \lpquote Q \rpquote / \lpquote P \rpquote \id{\}}            = 
  \left\{ 
    \begin{array}{ccc}
      \lpquote Q \rpquote & & x \nameeq \lpquote P \rpquote \\
      x & & otherwise \\
    \end{array}
  \right. \nonumber
\end{eqnarray}

and $z$ is chosen distinct from $\quotep{P}$, $\quotep{Q}$, the free
names in $Q$, and all the names in $R$. Our $\alpha$-equivalence will
be built in the standard way from this substitution.

\begin{remark}\label{rem:no_self_referential_names}
  One consequence of these definitions is that $\forall P. \quotep{P}
  \not\in \freenames{P}$.
\end{remark}

\subsection{ Dynamic quote: an example }

Anticipating something of what's to come, consider applying the
substitution, $\widehat{\id{\{}u / z \id{\}}}$, to the following pair
of processes, $\lift{w}{y!(z)}$ and $w[ \lpquote y!(z) \rpquote ]$.

\begin{eqnarray}
	\lift{w}{y!(z)}\widehat{\id{\{}u / z \id{\}}}
		& = &
		\lift{w}{y!(u)} \nonumber\\
	w[ \lpquote y!(z) \rpquote ] \widehat{ \id{\{}u / z \id{\}} }
		& = &
		w[ \lpquote y!(z) \rpquote ] \nonumber
\end{eqnarray}

Because the body of the process between quotes is impervious to
substitution, we get radically different answers. In fact, by
examining the first process in an input context,
e.g. $x?(z).\lift{w}{y!(z)}$, we see that the process under the lift
operator may be shaped by prefixed inputs binding a name inside it. In
this sense, the lift operator will be seen as a way to dynamically
construct processes before reifying them as names.

Finally equipped with these standard features we can present the
dynamics of the calculus.

\subsubsection{Operational semantics} 

Finally, we introduce the computational dynamics. What marks these
algebras as distinct from other more traditionally studied algebraic
structures, e.g. vector spaces or polynomial rings, is the manner in
which dynamics is captured. In traditional structures, dynamics is typically
expressed through morphisms between such structures, as in linear maps
between vector spaces or morphisms between rings. In algebras
associated with the semantics of computation, the dynamics is
expressed as part of the algebraic structure itself, through a
reduction reduction relation typically denoted by $\red$. Below, we
give a recursive presentation of this relation for the calculus used
in the encoding.

$\red \subseteq \pi \times \pi$
$\red : \pi \to \mathcal{P}(\pi)$

\begin{mathpar}
  \inferrule* [lab=Comm] { \textsf{match}( x_{src}, x_{trgt} ) } { x_{trgt}?(y)P \; | \; x_{src}!\langle {Q} \rangle \red P\{\quotep{Q}/y}\} }
  \and \\
  \inferrule* [lab=Par] {{P} \red {P}'} {{{P} | {Q}} \red {{P}' | {Q}}}
  \and
  \inferrule* [lab=Equiv]{{{P} \scong {P}'} \andalso {{P}' \red {Q}'} \andalso {{Q}' \scong {Q}}}{{P} \red {Q}}
\end{mathpar}

\begin{eqnarray*}
  match_{\equiv} (\quotep{P},\quotep{Q}) & := & P \equiv Q \\
  match_{\dagger}(\quotep{P},\quotep{Q}) & := & \forall R. P|Q \red^{*} R => R \red^{*} 0 \\
  match_{K}(\quotep{P},\quotep{Q}) & := & K \mbox{ for some context } K
\end{eqnarray*}

$u?(x)P | u!\langle Q \rangle \red P\{\quotep{Q}/x\}$

%We write $\wred$ for $\red^*$, and $P\red$ if $\exists Q $ such that $ P \red Q$.
We write $P\red$ if $\exists Q $ such that $ P \red Q$ and $P\not\red$, otherwise.

\section{Replication}

As mentioned before, it is known that replication (and hence
recursion) can be implemented in a higher-order process algebra
\cite{SangiorgiWalker}. As our first example of calculation with the
machinery thus far presented we give the construction explicitly in
the {\rhoc}.

\begin{eqnarray}
	D_{x} & := & \prefix{x}{y}{(\binpar{\outputp{x}{y}}{@{y}})} \nonumber\\
	\bangp_{x}{P} & := & \binpar{{x}!\langle{\binpar{D_{x}}{P}}\rangle}{D_{x}} \nonumber
\end{eqnarray}

\begin{eqnarray}
	\bangp_{x}{P} & & \nonumber\\
	=
	& {x}!\langle{(\prefix{x}{y}{(\outputp{x}{y} | @{y})) | P}}\rangle 
	      | \prefix{x}{y}{(\outputp{x}{y} | @{y})} & \nonumber\\
	\red
	& (\outputp{x}{y} | @{y})\substn{\quotep{(\prefix{x}{y}{(@{y} | \outputp{x}{y})) | P}}}{y} & \nonumber\\
	=
	& \outputp{x}{\quotep{(\prefix{x}{y}{(\outputp{x}{y} | @{y})) | P}}}
	  | {(\prefix{x}{y}{(\outputp{x}{y} | @{y})) | P}} & \nonumber\\
	\red
	& \ldots & \nonumber\\
	\red^*
	& P | P | \ldots & \nonumber
\end{eqnarray}

Of course, this encoding, as an implementation, runs away, unfolding
$\bangp{P}$ eagerly. A lazier and more implementable replication
operator, restricted to input-guarded processes, may be obtained as follows.

\begin{eqnarray}
\bangp{\prefix{u}{v}{P}} 
	:= 
	\binpar{\lift{x}{\prefix{u}{v}{(\binpar{D(x)}{P})}}}{D(x)} \nonumber
\end{eqnarray}

\begin{remark}
  Note that the lazier definition still does not deal with summation
  or mixed summation (i.e. sums over input and output). The reader is
  invited to construct definitions of replication that deal with these
  features. 

  Further, the definitions are parameterized in a name, $x$. Can you,
  gentle reader, make a definition that eliminates this parameter and
  guarantees no accidental interaction between the replication
  machinery and the process being replicated -- i.e. no accidental
  sharing of names used by the process to get its work done and the
  name(s) used by the replication to effect copying. This latter
  revision of the definition of replication is crucial to obtaining
  the expected identity $!!P \sim !P$.
\end{remark}

\begin{remark}\label{rem:paradoxical_combinator}
  The reader familiar with the lambda calculus will have noticed the
  similarity between $D$ and the paradoxical combinator.

  [Ed. note: the existence of this seems to suggest we have to be more
  restrictive on the set of processes and names we admit if we are to
  support no-cloning.]
\end{remark}

\subsubsection{Bisimulation}

The computational dynamics gives rise to another kind of equivalence,
the equivalence of computational behavior. As previously mentioned
this is typically captured \emph{via} some form of bisimulation.

% The notion we use in this paper is weak barbed bisimulation
% \cite{milner91polyadicpi}.

The notion we use in this paper is derived from weak barbed
bisimulation \cite{milner91polyadicpi}. 

\begin{definition}
An \emph{observation relation}, $\downarrow_{\mathcal N}$, over a set
of names, $\mathcal N$, is the smallest relation satisfying the rules
below.

\infrule[Out-barb]{y \in {\mathcal N}, \; x \nameeq y}
		  {\outputp{x}{v} \downarrow_{\mathcal N} x}
\infrule[Par-barb]{\mbox{$P\downarrow_{\mathcal N} x$ or $Q\downarrow_{\mathcal N} x$}}
		  {\binpar{P}{Q} \downarrow_{\mathcal N} x}

We write $P \Downarrow_{\mathcal N} x$ if there is $Q$ such that 
$P \wred Q$ and $Q \downarrow_{\mathcal N} x$.
\end{definition}

\begin{definition}
%\label{def.bbisim}
An  ${\mathcal N}$-\emph{barbed bisimulation} over a set of names, ${\mathcal N}$, is a symmetric binary relation 
${\mathcal S}_{\mathcal N}$ between agents such that $P\rel{S}_{\mathcal N}Q$ implies:
\begin{enumerate}
\item If $P \red P'$ then $Q \wred Q'$ and $P'\rel{S}_{\mathcal N} Q'$.
\item If $P\downarrow_{\mathcal N} x$, then $Q\Downarrow_{\mathcal N} x$.
\end{enumerate}
$P$ is ${\mathcal N}$-barbed bisimilar to $Q$, written
$P \wbbisim_{\mathcal N} Q$, if $P \rel{S}_{\mathcal N} Q$ for some ${\mathcal N}$-barbed bisimulation ${\mathcal S}_{\mathcal N}$.
\end{definition}

$\mathcal{R} \subseteq \pi \times \pi$

$P \mathcal{R} Q => \forall P'. P \red P' \Rightarrow \exists Q'. Q \red Q', P' \mathcal{R} Q'$

$P \vdash x \Rightarrow Q \vdash x$

\begin{mathpar}
  \inferrule*[lab=Out-barb]{x \nameeq y}{{y}!\langle{Q}\rangle \vdash x}
  \and
  \inferrule*[lab=Par-barb]{\mbox{$P\vdash x$ or $Q\vdash x$}}{\binpar{P}{Q} \vdash x}
\end{mathpar}

\subsubsection{Contexts}

One of the principle advantages of computational calculi like the
$\pi$-calculus is a well-defined notion of context,
contextual-equivalence and a correlation between
contextual-equivalence and notions of bisimulation. The notion of
context allows the decomposition of a process into (sub-)process and
its syntactic environment, its context. Thus, a context may be
thought of as a process with a ``hole'' (written $\Box$) in it. The
application of a context $M$ to a process $P$, written $M[P]$, is
tantamount to filling the hole in $M$ with $P$. In this paper we do
not need the full weight of this theory, but do make use of the notion
of context in the proof the main theorem. 

\begin{mathpar}
  \inferrule* [lab=summation] {} {{M_{M},M_{N}} \bc \Box \;|\; x.M_{A} \;|\; M_{M}+M_{N}}
  \and
  \inferrule* [lab=agent] {} {{M_{A}} \bc (\vec{x})M_{P} \;| \; \clift{P_0,\ldots,M_{P},\ldots,P_N}}
  \and \\
  \inferrule* [lab=process] {} {{M_{P}} \bc M_{N} \;| \;P|M_{P} }
\end{mathpar} 

\begin{mathpar}
  \inferrule* [lab=sychronization] {} {M_{N} \bc \Box \;|\; x?M_{F} \;|\; x!M_{C}}
  \and
  \inferrule* [lab=abstraction] {} {{M_{F}} \bc (x)M_{P} }
  \and
  \inferrule* [lab=concretion] {} {{M_{C}} \bc \langle M_{P} \rangle }
  \and \\
  \inferrule* [lab=process] {} {{M_{P}} \bc M_{N} \;| \;P|M_{P} }
\end{mathpar}

\begin{definition}[contextual application] Given a context $M$, and
  process $P$, we define the \emph{contextual application}, $M[P] :=
  M\{P/\Box\}$. That is, the contextual application of M to P is the
  substitution of $P$ for $\Box$ in $M$.
\end{definition}

$\meaningof{-} : L \to \mathcal{P}(\pi)$

\begin{mathpar}
  \inferrule* [lab=collection] {} {\meaningof{true} = \pi, \and \meaningof{~E} = \pi \setminus \meaningof{E}, \and \meaningof{E_{1} \& E_{2}} = \meaningof{E_{1}} \cap \meaningof{E_{2}}}
\end{mathpar}

\begin{mathpar}
  \inferrule* [lab=structure] {} {\meaningof{0} = \{ P \in \pi | P \equiv 0 \}, \and \\ \meaningof{E_1 | E_2} = \{ P \in \pi | P \equiv P_{1} | P_{2}, P_{1} \in \meaningof{E_{1}}, P_{2} \in \meaningof{E_2}\} }
\end{mathpar}

\begin{mathpar}
 \inferrule* [lab=behavior] {} {\meaningof{\langle a?b \rangle E} = \{ P \in \pi | P \equiv Q | u?(y)P', \\ \and \\\\ \and \\ \;\;\; u \in \meaningof{a}, \forall z.P'\{z/y\} \in \meaningof{E\{z/b\}}\}, \and \\ \meaningof{a!E} = \{ P \in \pi | P \equiv Q | x!\langle P' \rangle, x \in \meaningof{a} P' \in \meaningof{E}\} }
\end{mathpar}

\begin{mathpar}
 \inferrule* [lab=nominal] {} {\meaningof{\quotep{E}} = \{ \quotep{P} \in \quotep{\pi} | P \in \meaningof{E} \}, \and \meaningof{\quotep{P}} = \{ \quotep{Q} \in \quotep{\pi} | P \equiv Q \} \and \\ \meaningof{@\quotep{E}} = \{ P \in \pi | P \equiv @x, x \in \meaningof{E} \}}
\end{mathpar}

\begin{eqnarray*}
  \\
  \meaningof{-} : TS \to ST
\end{eqnarray*}

\begin{eqnarray*}
  \\
  L : TS \to ST
\end{eqnarray*}

\begin{eqnarray*}
  \\
  P \models E \iff P \in \meaningof{E}
\end{eqnarray*}

\begin{eqnarray*}
  P \approx_{L} Q \iff \forall E \in L. P \models E \iff Q \models E
\end{eqnarray*}

\begin{eqnarray*}
  P \approx_{K} Q
\end{eqnarray*}

\begin{eqnarray*}
  P \approx Q
\end{eqnarray*}

$\approx_{K} = \approx = \approx_{L}$

\subsubsection{Contextual duality}

Note that contexts extend the quotation operation to a family of
operations from processes to names. Given a context, $M$, we can
define a \emph{nominal context}, $\quotep{M}$ by $\quotep{M}[P] :=
\quotep{M[P]}$. To foreshadow what is to come we observe that these
operations enjoy a duality with processes very much like the duality
between vectors and maps from vectors to scalars.

Further, because the calculus is essentially higher-order, we have a
correspondence between contexts and processes. More specifically,
given a name $x$ and a context $M$ we can construct $M^{*}_{x}$ such
that 

\begin{mathpar}
  M^{*}_{x} | \lift{x}{P} \red M[P]
\end{mathpar}

namely,

\begin{mathpar}
  M^{*}_{x} := x?(u).M[\dropn{u}]
\end{mathpar}

The dependence of $M^{*}_{x}$ on a name makes it an abstraction, 

\begin{mathpar}
  M^{*} := (x)x?(u).M[\dropn{u}]
\end{mathpar}

\subsection{Additional notation}

It will sometimes be convenient to denote the process a name
quotes. We already have the notation $x = \quotep{P}$, but it will be
convenient to introduce an alternate notation, $\procn{x}$, when we
want to emphasize the connection to the use of the name. Note that, by
virtue of name equivalence, $\quotep{\procn{x}} \nameeq x$; so, the
notation is consistent with previous definitions.

Further, because names have structure it is possible to effect
substitutions on the basis of that structure. This means we need to
upgrade our notation for substitutions, which we accomplish by
adapting comprehension notation. Thus,

\begin{mathpar}
  P\{ y / x : x \in S \}
\end{mathpar}

is interpreted to mean the process derived from P by replacing (in a
capture-avoiding manner) each occurrence of $x$ in $S$ by $y$. For example,

\begin{mathpar}
  P\{ \quotep{\procn{x}|\procn{x}} / x : x \in \freenames{P} \}
\end{mathpar}

will replace each (occurrence) of a free name $x$ in $P$ by
$\quotep{\procn{x}|\procn{x}}$.

Also, we will avail ourselves of the notation $x^{L}$ and $x^{R}$ to
denote injections of a name into disjoint copies of the name
space. There are numerous ways to accomplish this. One example can be
found in \cite{MeredithR05}. This notation overloads to vectors of
names: $\vec{x}^{\pi} := (x_{i}^{\pi} \; : \; 0 \leq i < |\vec{x}| )$ where $\pi \in \{L,R\}$.

We also use $P^{\Box} := P|\Box$.

In \cite{MeredithR05} an interpretation of the new operator is
given. It turns out that there are several possible interpretations
all enjoying the requisite algebraic properties of the operator (see
\cite{milner91polyadicpi}). We will therefore make liberal use of
$(\nu\; \vec{x})P$.

% subsection the_syntax_and_semantics_of_the_notation_system (end)   

\input{qm2pi.qmops} 

\input{qm2pi.sterngerlach} 

\input{qm2pi.metric} 

% section concurrent_process_calculi (end)

%\input{qm2pi.proofsketch}

% section proof sketch (end)

%\input{qm2pi.slviaknots} 

% section spatial logic via knots (end)

\input{qm2pi.conclusion}

% section conclusion (end)

%\input{qm2pi.dtcodes} 

% section wiring algorithm (end)

\input{qm2pi.ack} 

% section acknowledgments (end)

\newpage


\bibliographystyle{plain}   
\bibliography{../../biblios/main.bib}

\input{qm2pi.rhodetails}

\end{document}

 

% section notation (end)

\input{qm2pi.process.calculi} 

% section concurrent_process_calculi_and_spatial_logics_ (end)
    
%\documentclass[12pt]{llncs}
%\documentclass{jktr}

\usepackage[pdftex]{hyperref}                   
\usepackage {listings}
\usepackage {mathpartir}
\usepackage{bcprules}
%\usepackage{listings}
                       
\usepackage{graphicx} 
%\usepackage[margins=2.5cm,nohead,nofoot]{geometry}
%\usepackage{geometry}
\usepackage{amsfonts}
\usepackage{amstext}
\usepackage{latexsym}
\usepackage{amssymb}
\usepackage{color}


%\include{myPreamble}
\include{qm2pi.local} 

%\ifpdf
%\usepackage[pdftex]{graphicx}
%\else
%\usepackage{graphicx}
%\fi

 % \ifpdf
%  \usepackage{pdfsync}
%  \if


%\title{Brief Article}
%\author{David F. Snyder}
%\author{L.G. Meredith}

%\address{Dept. of Math., Texas State University--San Marcos, San Marcos, TX 78666}
       
\pagestyle{empty}


\begin{document}

\lstset{language=[Objective]Caml,frame=shadowbox}

\input{qm2pi.front}

% section front matter (end)

\input{qm2pi.intro} 
 
% section introduction (end)

% \input{qm2pi.knotations} 

% section notation (end)

\input{qm2pi.process.calculi} 

% section concurrent_process_calculi_and_spatial_logics_ (end)
    
%\input{qm2pi.knots2pi} 

%\input{qm2pi.trefoil} 

%\input{qm2pi.mainthm} 

% subsection basic_interpretation (end)

%\input{qm2pi.rho.presentation} 
\subsection{The syntax and semantics of the notation system}\label{sub:the_syntax_and_semantics_of_the_notation_system} % (fold)

We now summarize a technical presentation of the calculus that
embodies our theory of dynamics. The typical presentation of such a
calculus follows the style of giving generators and relations on
them. The grammar, below, describing term constructors, freely
generates the set of processes, $\Proc$. This set is then quotiented
by a relation known as structural congruence and it is over this set
that the notion of dynamics is expressed. This presentation is
essentially that of \cite{MeredithR05} with the addition of
polyadicity and summation. For readability we have relegated some of
the technical subtleties to an appendix.

\subsubsection{Process grammar}\label{subsub:process_grammar}

\begin{mathpar}
  \inferrule* [lab=synchronization] {} {{M} \bc \pzero \;|\; x?F \;|\; x!C }
  \and
  \inferrule* [lab=abstraction] {} {{F} \bc (x)P}
  \and
  \inferrule* [lab=concretion] {} {{C} \bc \langle Q \rangle}
  \and
  \inferrule* [lab=process] {} {{P,Q} \bc M \;| \;P|Q \;|\; @{x}}
  \and
  \inferrule* [lab=name] {} {{x} \bc \quotep{P}}
\end{mathpar} 

Note that $\vec{x}$ (resp. $\vec{P}$) denotes a vector of names
(resp. processes) of length $|\vec{x}|$ (resp. $|\vec{P}|$). We adopt
the following useful abbreviations.

\begin{mathpar}
   x?(\vec{y}).P := x.(\vec{y})P \and  x\clift{\vec{P}} := x.\clift{\vec{P}}
   \and x!(y) := \lift{x}{\dropn{y}}
   \and \Pi_{i=0}^{n-1}P_i := P_0 | \ldots | P_{n-1}
\end{mathpar}

\subsubsection{Structural congruence}

\paragraph{Free and bound names and alpha-equivalence.} At the
core of structural equivalence is alpha-equivalence which identifies
process that are the same up to a change of variable. Formally, we
recognize the distinction between free and bound names. The free names
of a process, $\freenames{P}$, may be calculated recursively as
follows:

\begin{mathpar}
\freenames{\pzero} := \emptyset
  \and \\
  \freenames{x?(y).P} := \{ x \} \cup (\freenames{P} \setminus \{ y \})
  \and 
  \freenames{x!\langle P \rangle} := \{ x \} \cup \{ P \} 
  \and \\
  \freenames{P|Q} := \freenames{P} \cup \freenames{Q}
  \and \\
  \freenames{@{x}} := \{ x \}
\end{mathpar}

$\pi$
$\quotep{\pi}$

$\freenames{-} : \pi \to \mathcal{P}(\quotep{\pi})$

\begin{eqnarray*}
  \freenames{\pzero} & := & \emptyset \\
  \freenames{x?(y).P} & := & \{ x \} \cup (\freenames{P} \setminus \{ y \}) \\
  \freenames{x!\langle P \rangle} & := & \{ x \} \cup \{ P \} \\
  \freenames{P|Q} & := & \freenames{P} \cup \freenames{Q} \\
  \freenames{\dropn{x}} & := & \{ x \}
\end{eqnarray*}

The bound names of a process, $\boundnames{P}$, are those names occurring in $P$
that are not free. For example, in $x?(y).0$, the name $x$ is free, while $y$ is bound.

\begin{mathpar}
  \inferrule* [lab=monoidal-laws] {} { P|Q \equiv Q|P \and P|0 \equiv P \and P|(Q|R) \equiv (P|Q)|R }
\end{mathpar}

\begin{mathpar}
  \inferrule* [lab=alpha-equivalence] {} { (x)P \equiv (y)P\{y/x\} \and y \not\in \freenames{P} }
\end{mathpar}

\begin{definition}
Then two processes, $P,Q$, are alpha-equivalent if $P = Q\{\vec{y}/\vec{x}\}$ for
some $\vec{x} \in \boundnames{Q},\vec{y} \in \boundnames{P}$, where $Q\{\vec{y}/\vec{x}\}$
denotes the capture-avoiding substitution of $\vec{y}$ for $\vec{x}$ in $Q$.
\end{definition}

\begin{definition}
  The {\em structural congruence} \cite{SangiorgiWalker} , $\equiv$,
  between processes is the least congruence containing
  alpha-equivalence, satisfying the abelian monoid laws
  (associativity, commutativity and $\pzero$ as identity) for parallel
  composition $|$ and for summation $+$.
\end{definition}

\subsection{Name equivalence}

We take name equivalence, written $\nameeq$, to be the smallest
equivalence relation generated by the following rules.

\begin{mathpar}
\inferrule*[lab=Quote-drop]
{ }
{ \quotep{@{x}} \nameeq x }

\inferrule*[lab=Struct-equiv]
{ P \scong Q }
{ \quotep{P} \nameeq \quotep{Q} }
\end{mathpar}

The astute reader will have noticed that the mutual recursion of names
and processes imposes a mutual recursion on alpha-equivalence and
structural equivalence via name-equivalence. Fortunately, all of this
works out pleasantly and we may calculate in the natural way, free of
concern. The reader interested in the details is referred to the
appendix \ref{appendix:rho_details}.

\subsection{Substitution}

We use $\Proc$ for the set of processes, $\QProc$ for the set of
names, and $\id{\{}\vec{y} / \vec{x} \id{\}}$ to denote partial maps,
$s : \QProc \rightarrow \QProc$. A map, $s$ lifts, uniquely, to a map
on process terms, $\widehat{s} : \Proc \rightarrow \Proc$ by the
following equations.

\begin{mathpar}
  (0) \psubstp{Q}{P} := 0 \\
  (R \juxtap S) \psubstp{Q}{P}
  :=    
  (R)\psubstp{Q}{P} \juxtap (S) \psubstp{Q}{P} \\
  (x?(y).R) \psubstp{Q}{P}    
  :=    
  (x)\substp{Q}{P} (z)\concat( (R \psubstn{z}{y}) \psubstp{Q}{P} ) \\
  (\lift{x}{R}) \psubstp{Q}{P}  
  :=
  \lift{(x)\substp{Q}{P}}{ R \psubstp{Q}{P} } \\
%   (\dropn{x})  \psubstp{Q}{P}       
%   := 
%   \left\{ 
%     \begin{array}{ccc} 
%       \dropn{\quotep{Q}} & & x \nameeq \quotep{P} \\
%       \dropn{x} & & otherwise \\
%     \end{array}
%   \right. 
  (\dropn{x})  \psubstp{Q}{P}       
  := 
  \left\{ 
    \begin{array}{ccc} 
      Q & & x \nameeq \quotep{P} \\
      \dropn{x} & & otherwise \\
    \end{array}
  \right.
\end{mathpar}
 

where

\begin{eqnarray}
  (x)\id{\{} \lpquote Q \rpquote / \lpquote P \rpquote \id{\}}            = 
  \left\{ 
    \begin{array}{ccc}
      \lpquote Q \rpquote & & x \nameeq \lpquote P \rpquote \\
      x & & otherwise \\
    \end{array}
  \right. \nonumber
\end{eqnarray}

and $z$ is chosen distinct from $\quotep{P}$, $\quotep{Q}$, the free
names in $Q$, and all the names in $R$. Our $\alpha$-equivalence will
be built in the standard way from this substitution.

\begin{remark}\label{rem:no_self_referential_names}
  One consequence of these definitions is that $\forall P. \quotep{P}
  \not\in \freenames{P}$.
\end{remark}

\subsection{ Dynamic quote: an example }

Anticipating something of what's to come, consider applying the
substitution, $\widehat{\id{\{}u / z \id{\}}}$, to the following pair
of processes, $\lift{w}{y!(z)}$ and $w[ \lpquote y!(z) \rpquote ]$.

\begin{eqnarray}
	\lift{w}{y!(z)}\widehat{\id{\{}u / z \id{\}}}
		& = &
		\lift{w}{y!(u)} \nonumber\\
	w[ \lpquote y!(z) \rpquote ] \widehat{ \id{\{}u / z \id{\}} }
		& = &
		w[ \lpquote y!(z) \rpquote ] \nonumber
\end{eqnarray}

Because the body of the process between quotes is impervious to
substitution, we get radically different answers. In fact, by
examining the first process in an input context,
e.g. $x?(z).\lift{w}{y!(z)}$, we see that the process under the lift
operator may be shaped by prefixed inputs binding a name inside it. In
this sense, the lift operator will be seen as a way to dynamically
construct processes before reifying them as names.

Finally equipped with these standard features we can present the
dynamics of the calculus.

\subsubsection{Operational semantics} 

Finally, we introduce the computational dynamics. What marks these
algebras as distinct from other more traditionally studied algebraic
structures, e.g. vector spaces or polynomial rings, is the manner in
which dynamics is captured. In traditional structures, dynamics is typically
expressed through morphisms between such structures, as in linear maps
between vector spaces or morphisms between rings. In algebras
associated with the semantics of computation, the dynamics is
expressed as part of the algebraic structure itself, through a
reduction reduction relation typically denoted by $\red$. Below, we
give a recursive presentation of this relation for the calculus used
in the encoding.

$\red \subseteq \pi \times \pi$
$\red : \pi \to \mathcal{P}(\pi)$

\begin{mathpar}
  \inferrule* [lab=Comm] { \textsf{match}( x_{src}, x_{trgt} ) } { x_{trgt}?(y)P \; | \; x_{src}!\langle {Q} \rangle \red P\{\quotep{Q}/y}\} }
  \and \\
  \inferrule* [lab=Par] {{P} \red {P}'} {{{P} | {Q}} \red {{P}' | {Q}}}
  \and
  \inferrule* [lab=Equiv]{{{P} \scong {P}'} \andalso {{P}' \red {Q}'} \andalso {{Q}' \scong {Q}}}{{P} \red {Q}}
\end{mathpar}

\begin{eqnarray*}
  match_{\equiv} (\quotep{P},\quotep{Q}) & := & P \equiv Q \\
  match_{\dagger}(\quotep{P},\quotep{Q}) & := & \forall R. P|Q \red^{*} R => R \red^{*} 0 \\
  match_{K}(\quotep{P},\quotep{Q}) & := & K \mbox{ for some context } K
\end{eqnarray*}

$u?(x)P | u!\langle Q \rangle \red P\{\quotep{Q}/x\}$

%We write $\wred$ for $\red^*$, and $P\red$ if $\exists Q $ such that $ P \red Q$.
We write $P\red$ if $\exists Q $ such that $ P \red Q$ and $P\not\red$, otherwise.

\section{Replication}

As mentioned before, it is known that replication (and hence
recursion) can be implemented in a higher-order process algebra
\cite{SangiorgiWalker}. As our first example of calculation with the
machinery thus far presented we give the construction explicitly in
the {\rhoc}.

\begin{eqnarray}
	D_{x} & := & \prefix{x}{y}{(\binpar{\outputp{x}{y}}{@{y}})} \nonumber\\
	\bangp_{x}{P} & := & \binpar{{x}!\langle{\binpar{D_{x}}{P}}\rangle}{D_{x}} \nonumber
\end{eqnarray}

\begin{eqnarray}
	\bangp_{x}{P} & & \nonumber\\
	=
	& {x}!\langle{(\prefix{x}{y}{(\outputp{x}{y} | @{y})) | P}}\rangle 
	      | \prefix{x}{y}{(\outputp{x}{y} | @{y})} & \nonumber\\
	\red
	& (\outputp{x}{y} | @{y})\substn{\quotep{(\prefix{x}{y}{(@{y} | \outputp{x}{y})) | P}}}{y} & \nonumber\\
	=
	& \outputp{x}{\quotep{(\prefix{x}{y}{(\outputp{x}{y} | @{y})) | P}}}
	  | {(\prefix{x}{y}{(\outputp{x}{y} | @{y})) | P}} & \nonumber\\
	\red
	& \ldots & \nonumber\\
	\red^*
	& P | P | \ldots & \nonumber
\end{eqnarray}

Of course, this encoding, as an implementation, runs away, unfolding
$\bangp{P}$ eagerly. A lazier and more implementable replication
operator, restricted to input-guarded processes, may be obtained as follows.

\begin{eqnarray}
\bangp{\prefix{u}{v}{P}} 
	:= 
	\binpar{\lift{x}{\prefix{u}{v}{(\binpar{D(x)}{P})}}}{D(x)} \nonumber
\end{eqnarray}

\begin{remark}
  Note that the lazier definition still does not deal with summation
  or mixed summation (i.e. sums over input and output). The reader is
  invited to construct definitions of replication that deal with these
  features. 

  Further, the definitions are parameterized in a name, $x$. Can you,
  gentle reader, make a definition that eliminates this parameter and
  guarantees no accidental interaction between the replication
  machinery and the process being replicated -- i.e. no accidental
  sharing of names used by the process to get its work done and the
  name(s) used by the replication to effect copying. This latter
  revision of the definition of replication is crucial to obtaining
  the expected identity $!!P \sim !P$.
\end{remark}

\begin{remark}\label{rem:paradoxical_combinator}
  The reader familiar with the lambda calculus will have noticed the
  similarity between $D$ and the paradoxical combinator.

  [Ed. note: the existence of this seems to suggest we have to be more
  restrictive on the set of processes and names we admit if we are to
  support no-cloning.]
\end{remark}

\subsubsection{Bisimulation}

The computational dynamics gives rise to another kind of equivalence,
the equivalence of computational behavior. As previously mentioned
this is typically captured \emph{via} some form of bisimulation.

% The notion we use in this paper is weak barbed bisimulation
% \cite{milner91polyadicpi}.

The notion we use in this paper is derived from weak barbed
bisimulation \cite{milner91polyadicpi}. 

\begin{definition}
An \emph{observation relation}, $\downarrow_{\mathcal N}$, over a set
of names, $\mathcal N$, is the smallest relation satisfying the rules
below.

\infrule[Out-barb]{y \in {\mathcal N}, \; x \nameeq y}
		  {\outputp{x}{v} \downarrow_{\mathcal N} x}
\infrule[Par-barb]{\mbox{$P\downarrow_{\mathcal N} x$ or $Q\downarrow_{\mathcal N} x$}}
		  {\binpar{P}{Q} \downarrow_{\mathcal N} x}

We write $P \Downarrow_{\mathcal N} x$ if there is $Q$ such that 
$P \wred Q$ and $Q \downarrow_{\mathcal N} x$.
\end{definition}

\begin{definition}
%\label{def.bbisim}
An  ${\mathcal N}$-\emph{barbed bisimulation} over a set of names, ${\mathcal N}$, is a symmetric binary relation 
${\mathcal S}_{\mathcal N}$ between agents such that $P\rel{S}_{\mathcal N}Q$ implies:
\begin{enumerate}
\item If $P \red P'$ then $Q \wred Q'$ and $P'\rel{S}_{\mathcal N} Q'$.
\item If $P\downarrow_{\mathcal N} x$, then $Q\Downarrow_{\mathcal N} x$.
\end{enumerate}
$P$ is ${\mathcal N}$-barbed bisimilar to $Q$, written
$P \wbbisim_{\mathcal N} Q$, if $P \rel{S}_{\mathcal N} Q$ for some ${\mathcal N}$-barbed bisimulation ${\mathcal S}_{\mathcal N}$.
\end{definition}

$\mathcal{R} \subseteq \pi \times \pi$

$P \mathcal{R} Q => \forall P'. P \red P' \Rightarrow \exists Q'. Q \red Q', P' \mathcal{R} Q'$

$P \vdash x \Rightarrow Q \vdash x$

\begin{mathpar}
  \inferrule*[lab=Out-barb]{x \nameeq y}{{y}!\langle{Q}\rangle \vdash x}
  \and
  \inferrule*[lab=Par-barb]{\mbox{$P\vdash x$ or $Q\vdash x$}}{\binpar{P}{Q} \vdash x}
\end{mathpar}

\subsubsection{Contexts}

One of the principle advantages of computational calculi like the
$\pi$-calculus is a well-defined notion of context,
contextual-equivalence and a correlation between
contextual-equivalence and notions of bisimulation. The notion of
context allows the decomposition of a process into (sub-)process and
its syntactic environment, its context. Thus, a context may be
thought of as a process with a ``hole'' (written $\Box$) in it. The
application of a context $M$ to a process $P$, written $M[P]$, is
tantamount to filling the hole in $M$ with $P$. In this paper we do
not need the full weight of this theory, but do make use of the notion
of context in the proof the main theorem. 

\begin{mathpar}
  \inferrule* [lab=summation] {} {{M_{M},M_{N}} \bc \Box \;|\; x.M_{A} \;|\; M_{M}+M_{N}}
  \and
  \inferrule* [lab=agent] {} {{M_{A}} \bc (\vec{x})M_{P} \;| \; \clift{P_0,\ldots,M_{P},\ldots,P_N}}
  \and \\
  \inferrule* [lab=process] {} {{M_{P}} \bc M_{N} \;| \;P|M_{P} }
\end{mathpar} 

\begin{mathpar}
  \inferrule* [lab=sychronization] {} {M_{N} \bc \Box \;|\; x?M_{F} \;|\; x!M_{C}}
  \and
  \inferrule* [lab=abstraction] {} {{M_{F}} \bc (x)M_{P} }
  \and
  \inferrule* [lab=concretion] {} {{M_{C}} \bc \langle M_{P} \rangle }
  \and \\
  \inferrule* [lab=process] {} {{M_{P}} \bc M_{N} \;| \;P|M_{P} }
\end{mathpar}

\begin{definition}[contextual application] Given a context $M$, and
  process $P$, we define the \emph{contextual application}, $M[P] :=
  M\{P/\Box\}$. That is, the contextual application of M to P is the
  substitution of $P$ for $\Box$ in $M$.
\end{definition}

$\meaningof{-} : L \to \mathcal{P}(\pi)$

\begin{mathpar}
  \inferrule* [lab=collection] {} {\meaningof{true} = \pi, \and \meaningof{~E} = \pi \setminus \meaningof{E}, \and \meaningof{E_{1} \& E_{2}} = \meaningof{E_{1}} \cap \meaningof{E_{2}}}
\end{mathpar}

\begin{mathpar}
  \inferrule* [lab=structure] {} {\meaningof{0} = \{ P \in \pi | P \equiv 0 \}, \and \\ \meaningof{E_1 | E_2} = \{ P \in \pi | P \equiv P_{1} | P_{2}, P_{1} \in \meaningof{E_{1}}, P_{2} \in \meaningof{E_2}\} }
\end{mathpar}

\begin{mathpar}
 \inferrule* [lab=behavior] {} {\meaningof{\langle a?b \rangle E} = \{ P \in \pi | P \equiv Q | u?(y)P', \\ \and \\\\ \and \\ \;\;\; u \in \meaningof{a}, \forall z.P'\{z/y\} \in \meaningof{E\{z/b\}}\}, \and \\ \meaningof{a!E} = \{ P \in \pi | P \equiv Q | x!\langle P' \rangle, x \in \meaningof{a} P' \in \meaningof{E}\} }
\end{mathpar}

\begin{mathpar}
 \inferrule* [lab=nominal] {} {\meaningof{\quotep{E}} = \{ \quotep{P} \in \quotep{\pi} | P \in \meaningof{E} \}, \and \meaningof{\quotep{P}} = \{ \quotep{Q} \in \quotep{\pi} | P \equiv Q \} \and \\ \meaningof{@\quotep{E}} = \{ P \in \pi | P \equiv @x, x \in \meaningof{E} \}}
\end{mathpar}

\begin{eqnarray*}
  \\
  \meaningof{-} : TS \to ST
\end{eqnarray*}

\begin{eqnarray*}
  \\
  L : TS \to ST
\end{eqnarray*}

\begin{eqnarray*}
  \\
  P \models E \iff P \in \meaningof{E}
\end{eqnarray*}

\begin{eqnarray*}
  P \approx_{L} Q \iff \forall E \in L. P \models E \iff Q \models E
\end{eqnarray*}

\begin{eqnarray*}
  P \approx_{K} Q
\end{eqnarray*}

\begin{eqnarray*}
  P \approx Q
\end{eqnarray*}

$\approx_{K} = \approx = \approx_{L}$

\subsubsection{Contextual duality}

Note that contexts extend the quotation operation to a family of
operations from processes to names. Given a context, $M$, we can
define a \emph{nominal context}, $\quotep{M}$ by $\quotep{M}[P] :=
\quotep{M[P]}$. To foreshadow what is to come we observe that these
operations enjoy a duality with processes very much like the duality
between vectors and maps from vectors to scalars.

Further, because the calculus is essentially higher-order, we have a
correspondence between contexts and processes. More specifically,
given a name $x$ and a context $M$ we can construct $M^{*}_{x}$ such
that 

\begin{mathpar}
  M^{*}_{x} | \lift{x}{P} \red M[P]
\end{mathpar}

namely,

\begin{mathpar}
  M^{*}_{x} := x?(u).M[\dropn{u}]
\end{mathpar}

The dependence of $M^{*}_{x}$ on a name makes it an abstraction, 

\begin{mathpar}
  M^{*} := (x)x?(u).M[\dropn{u}]
\end{mathpar}

\subsection{Additional notation}

It will sometimes be convenient to denote the process a name
quotes. We already have the notation $x = \quotep{P}$, but it will be
convenient to introduce an alternate notation, $\procn{x}$, when we
want to emphasize the connection to the use of the name. Note that, by
virtue of name equivalence, $\quotep{\procn{x}} \nameeq x$; so, the
notation is consistent with previous definitions.

Further, because names have structure it is possible to effect
substitutions on the basis of that structure. This means we need to
upgrade our notation for substitutions, which we accomplish by
adapting comprehension notation. Thus,

\begin{mathpar}
  P\{ y / x : x \in S \}
\end{mathpar}

is interpreted to mean the process derived from P by replacing (in a
capture-avoiding manner) each occurrence of $x$ in $S$ by $y$. For example,

\begin{mathpar}
  P\{ \quotep{\procn{x}|\procn{x}} / x : x \in \freenames{P} \}
\end{mathpar}

will replace each (occurrence) of a free name $x$ in $P$ by
$\quotep{\procn{x}|\procn{x}}$.

Also, we will avail ourselves of the notation $x^{L}$ and $x^{R}$ to
denote injections of a name into disjoint copies of the name
space. There are numerous ways to accomplish this. One example can be
found in \cite{MeredithR05}. This notation overloads to vectors of
names: $\vec{x}^{\pi} := (x_{i}^{\pi} \; : \; 0 \leq i < |\vec{x}| )$ where $\pi \in \{L,R\}$.

We also use $P^{\Box} := P|\Box$.

In \cite{MeredithR05} an interpretation of the new operator is
given. It turns out that there are several possible interpretations
all enjoying the requisite algebraic properties of the operator (see
\cite{milner91polyadicpi}). We will therefore make liberal use of
$(\nu\; \vec{x})P$.

% subsection the_syntax_and_semantics_of_the_notation_system (end)   

\input{qm2pi.qmops} 

\input{qm2pi.sterngerlach} 

\input{qm2pi.metric} 

% section concurrent_process_calculi (end)

%\input{qm2pi.proofsketch}

% section proof sketch (end)

%\input{qm2pi.slviaknots} 

% section spatial logic via knots (end)

\input{qm2pi.conclusion}

% section conclusion (end)

%\input{qm2pi.dtcodes} 

% section wiring algorithm (end)

\input{qm2pi.ack} 

% section acknowledgments (end)

\newpage


\bibliographystyle{plain}   
\bibliography{../../biblios/main.bib}

\input{qm2pi.rhodetails}

\end{document}

 

%\documentclass[12pt]{llncs}
%\documentclass{jktr}

\usepackage[pdftex]{hyperref}                   
\usepackage {listings}
\usepackage {mathpartir}
\usepackage{bcprules}
%\usepackage{listings}
                       
\usepackage{graphicx} 
%\usepackage[margins=2.5cm,nohead,nofoot]{geometry}
%\usepackage{geometry}
\usepackage{amsfonts}
\usepackage{amstext}
\usepackage{latexsym}
\usepackage{amssymb}
\usepackage{color}


%\include{myPreamble}
\include{qm2pi.local} 

%\ifpdf
%\usepackage[pdftex]{graphicx}
%\else
%\usepackage{graphicx}
%\fi

 % \ifpdf
%  \usepackage{pdfsync}
%  \if


%\title{Brief Article}
%\author{David F. Snyder}
%\author{L.G. Meredith}

%\address{Dept. of Math., Texas State University--San Marcos, San Marcos, TX 78666}
       
\pagestyle{empty}


\begin{document}

\lstset{language=[Objective]Caml,frame=shadowbox}

\input{qm2pi.front}

% section front matter (end)

\input{qm2pi.intro} 
 
% section introduction (end)

% \input{qm2pi.knotations} 

% section notation (end)

\input{qm2pi.process.calculi} 

% section concurrent_process_calculi_and_spatial_logics_ (end)
    
%\input{qm2pi.knots2pi} 

%\input{qm2pi.trefoil} 

%\input{qm2pi.mainthm} 

% subsection basic_interpretation (end)

%\input{qm2pi.rho.presentation} 
\subsection{The syntax and semantics of the notation system}\label{sub:the_syntax_and_semantics_of_the_notation_system} % (fold)

We now summarize a technical presentation of the calculus that
embodies our theory of dynamics. The typical presentation of such a
calculus follows the style of giving generators and relations on
them. The grammar, below, describing term constructors, freely
generates the set of processes, $\Proc$. This set is then quotiented
by a relation known as structural congruence and it is over this set
that the notion of dynamics is expressed. This presentation is
essentially that of \cite{MeredithR05} with the addition of
polyadicity and summation. For readability we have relegated some of
the technical subtleties to an appendix.

\subsubsection{Process grammar}\label{subsub:process_grammar}

\begin{mathpar}
  \inferrule* [lab=synchronization] {} {{M} \bc \pzero \;|\; x?F \;|\; x!C }
  \and
  \inferrule* [lab=abstraction] {} {{F} \bc (x)P}
  \and
  \inferrule* [lab=concretion] {} {{C} \bc \langle Q \rangle}
  \and
  \inferrule* [lab=process] {} {{P,Q} \bc M \;| \;P|Q \;|\; @{x}}
  \and
  \inferrule* [lab=name] {} {{x} \bc \quotep{P}}
\end{mathpar} 

Note that $\vec{x}$ (resp. $\vec{P}$) denotes a vector of names
(resp. processes) of length $|\vec{x}|$ (resp. $|\vec{P}|$). We adopt
the following useful abbreviations.

\begin{mathpar}
   x?(\vec{y}).P := x.(\vec{y})P \and  x\clift{\vec{P}} := x.\clift{\vec{P}}
   \and x!(y) := \lift{x}{\dropn{y}}
   \and \Pi_{i=0}^{n-1}P_i := P_0 | \ldots | P_{n-1}
\end{mathpar}

\subsubsection{Structural congruence}

\paragraph{Free and bound names and alpha-equivalence.} At the
core of structural equivalence is alpha-equivalence which identifies
process that are the same up to a change of variable. Formally, we
recognize the distinction between free and bound names. The free names
of a process, $\freenames{P}$, may be calculated recursively as
follows:

\begin{mathpar}
\freenames{\pzero} := \emptyset
  \and \\
  \freenames{x?(y).P} := \{ x \} \cup (\freenames{P} \setminus \{ y \})
  \and 
  \freenames{x!\langle P \rangle} := \{ x \} \cup \{ P \} 
  \and \\
  \freenames{P|Q} := \freenames{P} \cup \freenames{Q}
  \and \\
  \freenames{@{x}} := \{ x \}
\end{mathpar}

$\pi$
$\quotep{\pi}$

$\freenames{-} : \pi \to \mathcal{P}(\quotep{\pi})$

\begin{eqnarray*}
  \freenames{\pzero} & := & \emptyset \\
  \freenames{x?(y).P} & := & \{ x \} \cup (\freenames{P} \setminus \{ y \}) \\
  \freenames{x!\langle P \rangle} & := & \{ x \} \cup \{ P \} \\
  \freenames{P|Q} & := & \freenames{P} \cup \freenames{Q} \\
  \freenames{\dropn{x}} & := & \{ x \}
\end{eqnarray*}

The bound names of a process, $\boundnames{P}$, are those names occurring in $P$
that are not free. For example, in $x?(y).0$, the name $x$ is free, while $y$ is bound.

\begin{mathpar}
  \inferrule* [lab=monoidal-laws] {} { P|Q \equiv Q|P \and P|0 \equiv P \and P|(Q|R) \equiv (P|Q)|R }
\end{mathpar}

\begin{mathpar}
  \inferrule* [lab=alpha-equivalence] {} { (x)P \equiv (y)P\{y/x\} \and y \not\in \freenames{P} }
\end{mathpar}

\begin{definition}
Then two processes, $P,Q$, are alpha-equivalent if $P = Q\{\vec{y}/\vec{x}\}$ for
some $\vec{x} \in \boundnames{Q},\vec{y} \in \boundnames{P}$, where $Q\{\vec{y}/\vec{x}\}$
denotes the capture-avoiding substitution of $\vec{y}$ for $\vec{x}$ in $Q$.
\end{definition}

\begin{definition}
  The {\em structural congruence} \cite{SangiorgiWalker} , $\equiv$,
  between processes is the least congruence containing
  alpha-equivalence, satisfying the abelian monoid laws
  (associativity, commutativity and $\pzero$ as identity) for parallel
  composition $|$ and for summation $+$.
\end{definition}

\subsection{Name equivalence}

We take name equivalence, written $\nameeq$, to be the smallest
equivalence relation generated by the following rules.

\begin{mathpar}
\inferrule*[lab=Quote-drop]
{ }
{ \quotep{@{x}} \nameeq x }

\inferrule*[lab=Struct-equiv]
{ P \scong Q }
{ \quotep{P} \nameeq \quotep{Q} }
\end{mathpar}

The astute reader will have noticed that the mutual recursion of names
and processes imposes a mutual recursion on alpha-equivalence and
structural equivalence via name-equivalence. Fortunately, all of this
works out pleasantly and we may calculate in the natural way, free of
concern. The reader interested in the details is referred to the
appendix \ref{appendix:rho_details}.

\subsection{Substitution}

We use $\Proc$ for the set of processes, $\QProc$ for the set of
names, and $\id{\{}\vec{y} / \vec{x} \id{\}}$ to denote partial maps,
$s : \QProc \rightarrow \QProc$. A map, $s$ lifts, uniquely, to a map
on process terms, $\widehat{s} : \Proc \rightarrow \Proc$ by the
following equations.

\begin{mathpar}
  (0) \psubstp{Q}{P} := 0 \\
  (R \juxtap S) \psubstp{Q}{P}
  :=    
  (R)\psubstp{Q}{P} \juxtap (S) \psubstp{Q}{P} \\
  (x?(y).R) \psubstp{Q}{P}    
  :=    
  (x)\substp{Q}{P} (z)\concat( (R \psubstn{z}{y}) \psubstp{Q}{P} ) \\
  (\lift{x}{R}) \psubstp{Q}{P}  
  :=
  \lift{(x)\substp{Q}{P}}{ R \psubstp{Q}{P} } \\
%   (\dropn{x})  \psubstp{Q}{P}       
%   := 
%   \left\{ 
%     \begin{array}{ccc} 
%       \dropn{\quotep{Q}} & & x \nameeq \quotep{P} \\
%       \dropn{x} & & otherwise \\
%     \end{array}
%   \right. 
  (\dropn{x})  \psubstp{Q}{P}       
  := 
  \left\{ 
    \begin{array}{ccc} 
      Q & & x \nameeq \quotep{P} \\
      \dropn{x} & & otherwise \\
    \end{array}
  \right.
\end{mathpar}
 

where

\begin{eqnarray}
  (x)\id{\{} \lpquote Q \rpquote / \lpquote P \rpquote \id{\}}            = 
  \left\{ 
    \begin{array}{ccc}
      \lpquote Q \rpquote & & x \nameeq \lpquote P \rpquote \\
      x & & otherwise \\
    \end{array}
  \right. \nonumber
\end{eqnarray}

and $z$ is chosen distinct from $\quotep{P}$, $\quotep{Q}$, the free
names in $Q$, and all the names in $R$. Our $\alpha$-equivalence will
be built in the standard way from this substitution.

\begin{remark}\label{rem:no_self_referential_names}
  One consequence of these definitions is that $\forall P. \quotep{P}
  \not\in \freenames{P}$.
\end{remark}

\subsection{ Dynamic quote: an example }

Anticipating something of what's to come, consider applying the
substitution, $\widehat{\id{\{}u / z \id{\}}}$, to the following pair
of processes, $\lift{w}{y!(z)}$ and $w[ \lpquote y!(z) \rpquote ]$.

\begin{eqnarray}
	\lift{w}{y!(z)}\widehat{\id{\{}u / z \id{\}}}
		& = &
		\lift{w}{y!(u)} \nonumber\\
	w[ \lpquote y!(z) \rpquote ] \widehat{ \id{\{}u / z \id{\}} }
		& = &
		w[ \lpquote y!(z) \rpquote ] \nonumber
\end{eqnarray}

Because the body of the process between quotes is impervious to
substitution, we get radically different answers. In fact, by
examining the first process in an input context,
e.g. $x?(z).\lift{w}{y!(z)}$, we see that the process under the lift
operator may be shaped by prefixed inputs binding a name inside it. In
this sense, the lift operator will be seen as a way to dynamically
construct processes before reifying them as names.

Finally equipped with these standard features we can present the
dynamics of the calculus.

\subsubsection{Operational semantics} 

Finally, we introduce the computational dynamics. What marks these
algebras as distinct from other more traditionally studied algebraic
structures, e.g. vector spaces or polynomial rings, is the manner in
which dynamics is captured. In traditional structures, dynamics is typically
expressed through morphisms between such structures, as in linear maps
between vector spaces or morphisms between rings. In algebras
associated with the semantics of computation, the dynamics is
expressed as part of the algebraic structure itself, through a
reduction reduction relation typically denoted by $\red$. Below, we
give a recursive presentation of this relation for the calculus used
in the encoding.

$\red \subseteq \pi \times \pi$
$\red : \pi \to \mathcal{P}(\pi)$

\begin{mathpar}
  \inferrule* [lab=Comm] { \textsf{match}( x_{src}, x_{trgt} ) } { x_{trgt}?(y)P \; | \; x_{src}!\langle {Q} \rangle \red P\{\quotep{Q}/y}\} }
  \and \\
  \inferrule* [lab=Par] {{P} \red {P}'} {{{P} | {Q}} \red {{P}' | {Q}}}
  \and
  \inferrule* [lab=Equiv]{{{P} \scong {P}'} \andalso {{P}' \red {Q}'} \andalso {{Q}' \scong {Q}}}{{P} \red {Q}}
\end{mathpar}

\begin{eqnarray*}
  match_{\equiv} (\quotep{P},\quotep{Q}) & := & P \equiv Q \\
  match_{\dagger}(\quotep{P},\quotep{Q}) & := & \forall R. P|Q \red^{*} R => R \red^{*} 0 \\
  match_{K}(\quotep{P},\quotep{Q}) & := & K \mbox{ for some context } K
\end{eqnarray*}

$u?(x)P | u!\langle Q \rangle \red P\{\quotep{Q}/x\}$

%We write $\wred$ for $\red^*$, and $P\red$ if $\exists Q $ such that $ P \red Q$.
We write $P\red$ if $\exists Q $ such that $ P \red Q$ and $P\not\red$, otherwise.

\section{Replication}

As mentioned before, it is known that replication (and hence
recursion) can be implemented in a higher-order process algebra
\cite{SangiorgiWalker}. As our first example of calculation with the
machinery thus far presented we give the construction explicitly in
the {\rhoc}.

\begin{eqnarray}
	D_{x} & := & \prefix{x}{y}{(\binpar{\outputp{x}{y}}{@{y}})} \nonumber\\
	\bangp_{x}{P} & := & \binpar{{x}!\langle{\binpar{D_{x}}{P}}\rangle}{D_{x}} \nonumber
\end{eqnarray}

\begin{eqnarray}
	\bangp_{x}{P} & & \nonumber\\
	=
	& {x}!\langle{(\prefix{x}{y}{(\outputp{x}{y} | @{y})) | P}}\rangle 
	      | \prefix{x}{y}{(\outputp{x}{y} | @{y})} & \nonumber\\
	\red
	& (\outputp{x}{y} | @{y})\substn{\quotep{(\prefix{x}{y}{(@{y} | \outputp{x}{y})) | P}}}{y} & \nonumber\\
	=
	& \outputp{x}{\quotep{(\prefix{x}{y}{(\outputp{x}{y} | @{y})) | P}}}
	  | {(\prefix{x}{y}{(\outputp{x}{y} | @{y})) | P}} & \nonumber\\
	\red
	& \ldots & \nonumber\\
	\red^*
	& P | P | \ldots & \nonumber
\end{eqnarray}

Of course, this encoding, as an implementation, runs away, unfolding
$\bangp{P}$ eagerly. A lazier and more implementable replication
operator, restricted to input-guarded processes, may be obtained as follows.

\begin{eqnarray}
\bangp{\prefix{u}{v}{P}} 
	:= 
	\binpar{\lift{x}{\prefix{u}{v}{(\binpar{D(x)}{P})}}}{D(x)} \nonumber
\end{eqnarray}

\begin{remark}
  Note that the lazier definition still does not deal with summation
  or mixed summation (i.e. sums over input and output). The reader is
  invited to construct definitions of replication that deal with these
  features. 

  Further, the definitions are parameterized in a name, $x$. Can you,
  gentle reader, make a definition that eliminates this parameter and
  guarantees no accidental interaction between the replication
  machinery and the process being replicated -- i.e. no accidental
  sharing of names used by the process to get its work done and the
  name(s) used by the replication to effect copying. This latter
  revision of the definition of replication is crucial to obtaining
  the expected identity $!!P \sim !P$.
\end{remark}

\begin{remark}\label{rem:paradoxical_combinator}
  The reader familiar with the lambda calculus will have noticed the
  similarity between $D$ and the paradoxical combinator.

  [Ed. note: the existence of this seems to suggest we have to be more
  restrictive on the set of processes and names we admit if we are to
  support no-cloning.]
\end{remark}

\subsubsection{Bisimulation}

The computational dynamics gives rise to another kind of equivalence,
the equivalence of computational behavior. As previously mentioned
this is typically captured \emph{via} some form of bisimulation.

% The notion we use in this paper is weak barbed bisimulation
% \cite{milner91polyadicpi}.

The notion we use in this paper is derived from weak barbed
bisimulation \cite{milner91polyadicpi}. 

\begin{definition}
An \emph{observation relation}, $\downarrow_{\mathcal N}$, over a set
of names, $\mathcal N$, is the smallest relation satisfying the rules
below.

\infrule[Out-barb]{y \in {\mathcal N}, \; x \nameeq y}
		  {\outputp{x}{v} \downarrow_{\mathcal N} x}
\infrule[Par-barb]{\mbox{$P\downarrow_{\mathcal N} x$ or $Q\downarrow_{\mathcal N} x$}}
		  {\binpar{P}{Q} \downarrow_{\mathcal N} x}

We write $P \Downarrow_{\mathcal N} x$ if there is $Q$ such that 
$P \wred Q$ and $Q \downarrow_{\mathcal N} x$.
\end{definition}

\begin{definition}
%\label{def.bbisim}
An  ${\mathcal N}$-\emph{barbed bisimulation} over a set of names, ${\mathcal N}$, is a symmetric binary relation 
${\mathcal S}_{\mathcal N}$ between agents such that $P\rel{S}_{\mathcal N}Q$ implies:
\begin{enumerate}
\item If $P \red P'$ then $Q \wred Q'$ and $P'\rel{S}_{\mathcal N} Q'$.
\item If $P\downarrow_{\mathcal N} x$, then $Q\Downarrow_{\mathcal N} x$.
\end{enumerate}
$P$ is ${\mathcal N}$-barbed bisimilar to $Q$, written
$P \wbbisim_{\mathcal N} Q$, if $P \rel{S}_{\mathcal N} Q$ for some ${\mathcal N}$-barbed bisimulation ${\mathcal S}_{\mathcal N}$.
\end{definition}

$\mathcal{R} \subseteq \pi \times \pi$

$P \mathcal{R} Q => \forall P'. P \red P' \Rightarrow \exists Q'. Q \red Q', P' \mathcal{R} Q'$

$P \vdash x \Rightarrow Q \vdash x$

\begin{mathpar}
  \inferrule*[lab=Out-barb]{x \nameeq y}{{y}!\langle{Q}\rangle \vdash x}
  \and
  \inferrule*[lab=Par-barb]{\mbox{$P\vdash x$ or $Q\vdash x$}}{\binpar{P}{Q} \vdash x}
\end{mathpar}

\subsubsection{Contexts}

One of the principle advantages of computational calculi like the
$\pi$-calculus is a well-defined notion of context,
contextual-equivalence and a correlation between
contextual-equivalence and notions of bisimulation. The notion of
context allows the decomposition of a process into (sub-)process and
its syntactic environment, its context. Thus, a context may be
thought of as a process with a ``hole'' (written $\Box$) in it. The
application of a context $M$ to a process $P$, written $M[P]$, is
tantamount to filling the hole in $M$ with $P$. In this paper we do
not need the full weight of this theory, but do make use of the notion
of context in the proof the main theorem. 

\begin{mathpar}
  \inferrule* [lab=summation] {} {{M_{M},M_{N}} \bc \Box \;|\; x.M_{A} \;|\; M_{M}+M_{N}}
  \and
  \inferrule* [lab=agent] {} {{M_{A}} \bc (\vec{x})M_{P} \;| \; \clift{P_0,\ldots,M_{P},\ldots,P_N}}
  \and \\
  \inferrule* [lab=process] {} {{M_{P}} \bc M_{N} \;| \;P|M_{P} }
\end{mathpar} 

\begin{mathpar}
  \inferrule* [lab=sychronization] {} {M_{N} \bc \Box \;|\; x?M_{F} \;|\; x!M_{C}}
  \and
  \inferrule* [lab=abstraction] {} {{M_{F}} \bc (x)M_{P} }
  \and
  \inferrule* [lab=concretion] {} {{M_{C}} \bc \langle M_{P} \rangle }
  \and \\
  \inferrule* [lab=process] {} {{M_{P}} \bc M_{N} \;| \;P|M_{P} }
\end{mathpar}

\begin{definition}[contextual application] Given a context $M$, and
  process $P$, we define the \emph{contextual application}, $M[P] :=
  M\{P/\Box\}$. That is, the contextual application of M to P is the
  substitution of $P$ for $\Box$ in $M$.
\end{definition}

$\meaningof{-} : L \to \mathcal{P}(\pi)$

\begin{mathpar}
  \inferrule* [lab=collection] {} {\meaningof{true} = \pi, \and \meaningof{~E} = \pi \setminus \meaningof{E}, \and \meaningof{E_{1} \& E_{2}} = \meaningof{E_{1}} \cap \meaningof{E_{2}}}
\end{mathpar}

\begin{mathpar}
  \inferrule* [lab=structure] {} {\meaningof{0} = \{ P \in \pi | P \equiv 0 \}, \and \\ \meaningof{E_1 | E_2} = \{ P \in \pi | P \equiv P_{1} | P_{2}, P_{1} \in \meaningof{E_{1}}, P_{2} \in \meaningof{E_2}\} }
\end{mathpar}

\begin{mathpar}
 \inferrule* [lab=behavior] {} {\meaningof{\langle a?b \rangle E} = \{ P \in \pi | P \equiv Q | u?(y)P', \\ \and \\\\ \and \\ \;\;\; u \in \meaningof{a}, \forall z.P'\{z/y\} \in \meaningof{E\{z/b\}}\}, \and \\ \meaningof{a!E} = \{ P \in \pi | P \equiv Q | x!\langle P' \rangle, x \in \meaningof{a} P' \in \meaningof{E}\} }
\end{mathpar}

\begin{mathpar}
 \inferrule* [lab=nominal] {} {\meaningof{\quotep{E}} = \{ \quotep{P} \in \quotep{\pi} | P \in \meaningof{E} \}, \and \meaningof{\quotep{P}} = \{ \quotep{Q} \in \quotep{\pi} | P \equiv Q \} \and \\ \meaningof{@\quotep{E}} = \{ P \in \pi | P \equiv @x, x \in \meaningof{E} \}}
\end{mathpar}

\begin{eqnarray*}
  \\
  \meaningof{-} : TS \to ST
\end{eqnarray*}

\begin{eqnarray*}
  \\
  L : TS \to ST
\end{eqnarray*}

\begin{eqnarray*}
  \\
  P \models E \iff P \in \meaningof{E}
\end{eqnarray*}

\begin{eqnarray*}
  P \approx_{L} Q \iff \forall E \in L. P \models E \iff Q \models E
\end{eqnarray*}

\begin{eqnarray*}
  P \approx_{K} Q
\end{eqnarray*}

\begin{eqnarray*}
  P \approx Q
\end{eqnarray*}

$\approx_{K} = \approx = \approx_{L}$

\subsubsection{Contextual duality}

Note that contexts extend the quotation operation to a family of
operations from processes to names. Given a context, $M$, we can
define a \emph{nominal context}, $\quotep{M}$ by $\quotep{M}[P] :=
\quotep{M[P]}$. To foreshadow what is to come we observe that these
operations enjoy a duality with processes very much like the duality
between vectors and maps from vectors to scalars.

Further, because the calculus is essentially higher-order, we have a
correspondence between contexts and processes. More specifically,
given a name $x$ and a context $M$ we can construct $M^{*}_{x}$ such
that 

\begin{mathpar}
  M^{*}_{x} | \lift{x}{P} \red M[P]
\end{mathpar}

namely,

\begin{mathpar}
  M^{*}_{x} := x?(u).M[\dropn{u}]
\end{mathpar}

The dependence of $M^{*}_{x}$ on a name makes it an abstraction, 

\begin{mathpar}
  M^{*} := (x)x?(u).M[\dropn{u}]
\end{mathpar}

\subsection{Additional notation}

It will sometimes be convenient to denote the process a name
quotes. We already have the notation $x = \quotep{P}$, but it will be
convenient to introduce an alternate notation, $\procn{x}$, when we
want to emphasize the connection to the use of the name. Note that, by
virtue of name equivalence, $\quotep{\procn{x}} \nameeq x$; so, the
notation is consistent with previous definitions.

Further, because names have structure it is possible to effect
substitutions on the basis of that structure. This means we need to
upgrade our notation for substitutions, which we accomplish by
adapting comprehension notation. Thus,

\begin{mathpar}
  P\{ y / x : x \in S \}
\end{mathpar}

is interpreted to mean the process derived from P by replacing (in a
capture-avoiding manner) each occurrence of $x$ in $S$ by $y$. For example,

\begin{mathpar}
  P\{ \quotep{\procn{x}|\procn{x}} / x : x \in \freenames{P} \}
\end{mathpar}

will replace each (occurrence) of a free name $x$ in $P$ by
$\quotep{\procn{x}|\procn{x}}$.

Also, we will avail ourselves of the notation $x^{L}$ and $x^{R}$ to
denote injections of a name into disjoint copies of the name
space. There are numerous ways to accomplish this. One example can be
found in \cite{MeredithR05}. This notation overloads to vectors of
names: $\vec{x}^{\pi} := (x_{i}^{\pi} \; : \; 0 \leq i < |\vec{x}| )$ where $\pi \in \{L,R\}$.

We also use $P^{\Box} := P|\Box$.

In \cite{MeredithR05} an interpretation of the new operator is
given. It turns out that there are several possible interpretations
all enjoying the requisite algebraic properties of the operator (see
\cite{milner91polyadicpi}). We will therefore make liberal use of
$(\nu\; \vec{x})P$.

% subsection the_syntax_and_semantics_of_the_notation_system (end)   

\input{qm2pi.qmops} 

\input{qm2pi.sterngerlach} 

\input{qm2pi.metric} 

% section concurrent_process_calculi (end)

%\input{qm2pi.proofsketch}

% section proof sketch (end)

%\input{qm2pi.slviaknots} 

% section spatial logic via knots (end)

\input{qm2pi.conclusion}

% section conclusion (end)

%\input{qm2pi.dtcodes} 

% section wiring algorithm (end)

\input{qm2pi.ack} 

% section acknowledgments (end)

\newpage


\bibliographystyle{plain}   
\bibliography{../../biblios/main.bib}

\input{qm2pi.rhodetails}

\end{document}

 

%\documentclass[12pt]{llncs}
%\documentclass{jktr}

\usepackage[pdftex]{hyperref}                   
\usepackage {listings}
\usepackage {mathpartir}
\usepackage{bcprules}
%\usepackage{listings}
                       
\usepackage{graphicx} 
%\usepackage[margins=2.5cm,nohead,nofoot]{geometry}
%\usepackage{geometry}
\usepackage{amsfonts}
\usepackage{amstext}
\usepackage{latexsym}
\usepackage{amssymb}
\usepackage{color}


%\include{myPreamble}
\include{qm2pi.local} 

%\ifpdf
%\usepackage[pdftex]{graphicx}
%\else
%\usepackage{graphicx}
%\fi

 % \ifpdf
%  \usepackage{pdfsync}
%  \if


%\title{Brief Article}
%\author{David F. Snyder}
%\author{L.G. Meredith}

%\address{Dept. of Math., Texas State University--San Marcos, San Marcos, TX 78666}
       
\pagestyle{empty}


\begin{document}

\lstset{language=[Objective]Caml,frame=shadowbox}

\input{qm2pi.front}

% section front matter (end)

\input{qm2pi.intro} 
 
% section introduction (end)

% \input{qm2pi.knotations} 

% section notation (end)

\input{qm2pi.process.calculi} 

% section concurrent_process_calculi_and_spatial_logics_ (end)
    
%\input{qm2pi.knots2pi} 

%\input{qm2pi.trefoil} 

%\input{qm2pi.mainthm} 

% subsection basic_interpretation (end)

%\input{qm2pi.rho.presentation} 
\subsection{The syntax and semantics of the notation system}\label{sub:the_syntax_and_semantics_of_the_notation_system} % (fold)

We now summarize a technical presentation of the calculus that
embodies our theory of dynamics. The typical presentation of such a
calculus follows the style of giving generators and relations on
them. The grammar, below, describing term constructors, freely
generates the set of processes, $\Proc$. This set is then quotiented
by a relation known as structural congruence and it is over this set
that the notion of dynamics is expressed. This presentation is
essentially that of \cite{MeredithR05} with the addition of
polyadicity and summation. For readability we have relegated some of
the technical subtleties to an appendix.

\subsubsection{Process grammar}\label{subsub:process_grammar}

\begin{mathpar}
  \inferrule* [lab=synchronization] {} {{M} \bc \pzero \;|\; x?F \;|\; x!C }
  \and
  \inferrule* [lab=abstraction] {} {{F} \bc (x)P}
  \and
  \inferrule* [lab=concretion] {} {{C} \bc \langle Q \rangle}
  \and
  \inferrule* [lab=process] {} {{P,Q} \bc M \;| \;P|Q \;|\; @{x}}
  \and
  \inferrule* [lab=name] {} {{x} \bc \quotep{P}}
\end{mathpar} 

Note that $\vec{x}$ (resp. $\vec{P}$) denotes a vector of names
(resp. processes) of length $|\vec{x}|$ (resp. $|\vec{P}|$). We adopt
the following useful abbreviations.

\begin{mathpar}
   x?(\vec{y}).P := x.(\vec{y})P \and  x\clift{\vec{P}} := x.\clift{\vec{P}}
   \and x!(y) := \lift{x}{\dropn{y}}
   \and \Pi_{i=0}^{n-1}P_i := P_0 | \ldots | P_{n-1}
\end{mathpar}

\subsubsection{Structural congruence}

\paragraph{Free and bound names and alpha-equivalence.} At the
core of structural equivalence is alpha-equivalence which identifies
process that are the same up to a change of variable. Formally, we
recognize the distinction between free and bound names. The free names
of a process, $\freenames{P}$, may be calculated recursively as
follows:

\begin{mathpar}
\freenames{\pzero} := \emptyset
  \and \\
  \freenames{x?(y).P} := \{ x \} \cup (\freenames{P} \setminus \{ y \})
  \and 
  \freenames{x!\langle P \rangle} := \{ x \} \cup \{ P \} 
  \and \\
  \freenames{P|Q} := \freenames{P} \cup \freenames{Q}
  \and \\
  \freenames{@{x}} := \{ x \}
\end{mathpar}

$\pi$
$\quotep{\pi}$

$\freenames{-} : \pi \to \mathcal{P}(\quotep{\pi})$

\begin{eqnarray*}
  \freenames{\pzero} & := & \emptyset \\
  \freenames{x?(y).P} & := & \{ x \} \cup (\freenames{P} \setminus \{ y \}) \\
  \freenames{x!\langle P \rangle} & := & \{ x \} \cup \{ P \} \\
  \freenames{P|Q} & := & \freenames{P} \cup \freenames{Q} \\
  \freenames{\dropn{x}} & := & \{ x \}
\end{eqnarray*}

The bound names of a process, $\boundnames{P}$, are those names occurring in $P$
that are not free. For example, in $x?(y).0$, the name $x$ is free, while $y$ is bound.

\begin{mathpar}
  \inferrule* [lab=monoidal-laws] {} { P|Q \equiv Q|P \and P|0 \equiv P \and P|(Q|R) \equiv (P|Q)|R }
\end{mathpar}

\begin{mathpar}
  \inferrule* [lab=alpha-equivalence] {} { (x)P \equiv (y)P\{y/x\} \and y \not\in \freenames{P} }
\end{mathpar}

\begin{definition}
Then two processes, $P,Q$, are alpha-equivalent if $P = Q\{\vec{y}/\vec{x}\}$ for
some $\vec{x} \in \boundnames{Q},\vec{y} \in \boundnames{P}$, where $Q\{\vec{y}/\vec{x}\}$
denotes the capture-avoiding substitution of $\vec{y}$ for $\vec{x}$ in $Q$.
\end{definition}

\begin{definition}
  The {\em structural congruence} \cite{SangiorgiWalker} , $\equiv$,
  between processes is the least congruence containing
  alpha-equivalence, satisfying the abelian monoid laws
  (associativity, commutativity and $\pzero$ as identity) for parallel
  composition $|$ and for summation $+$.
\end{definition}

\subsection{Name equivalence}

We take name equivalence, written $\nameeq$, to be the smallest
equivalence relation generated by the following rules.

\begin{mathpar}
\inferrule*[lab=Quote-drop]
{ }
{ \quotep{@{x}} \nameeq x }

\inferrule*[lab=Struct-equiv]
{ P \scong Q }
{ \quotep{P} \nameeq \quotep{Q} }
\end{mathpar}

The astute reader will have noticed that the mutual recursion of names
and processes imposes a mutual recursion on alpha-equivalence and
structural equivalence via name-equivalence. Fortunately, all of this
works out pleasantly and we may calculate in the natural way, free of
concern. The reader interested in the details is referred to the
appendix \ref{appendix:rho_details}.

\subsection{Substitution}

We use $\Proc$ for the set of processes, $\QProc$ for the set of
names, and $\id{\{}\vec{y} / \vec{x} \id{\}}$ to denote partial maps,
$s : \QProc \rightarrow \QProc$. A map, $s$ lifts, uniquely, to a map
on process terms, $\widehat{s} : \Proc \rightarrow \Proc$ by the
following equations.

\begin{mathpar}
  (0) \psubstp{Q}{P} := 0 \\
  (R \juxtap S) \psubstp{Q}{P}
  :=    
  (R)\psubstp{Q}{P} \juxtap (S) \psubstp{Q}{P} \\
  (x?(y).R) \psubstp{Q}{P}    
  :=    
  (x)\substp{Q}{P} (z)\concat( (R \psubstn{z}{y}) \psubstp{Q}{P} ) \\
  (\lift{x}{R}) \psubstp{Q}{P}  
  :=
  \lift{(x)\substp{Q}{P}}{ R \psubstp{Q}{P} } \\
%   (\dropn{x})  \psubstp{Q}{P}       
%   := 
%   \left\{ 
%     \begin{array}{ccc} 
%       \dropn{\quotep{Q}} & & x \nameeq \quotep{P} \\
%       \dropn{x} & & otherwise \\
%     \end{array}
%   \right. 
  (\dropn{x})  \psubstp{Q}{P}       
  := 
  \left\{ 
    \begin{array}{ccc} 
      Q & & x \nameeq \quotep{P} \\
      \dropn{x} & & otherwise \\
    \end{array}
  \right.
\end{mathpar}
 

where

\begin{eqnarray}
  (x)\id{\{} \lpquote Q \rpquote / \lpquote P \rpquote \id{\}}            = 
  \left\{ 
    \begin{array}{ccc}
      \lpquote Q \rpquote & & x \nameeq \lpquote P \rpquote \\
      x & & otherwise \\
    \end{array}
  \right. \nonumber
\end{eqnarray}

and $z$ is chosen distinct from $\quotep{P}$, $\quotep{Q}$, the free
names in $Q$, and all the names in $R$. Our $\alpha$-equivalence will
be built in the standard way from this substitution.

\begin{remark}\label{rem:no_self_referential_names}
  One consequence of these definitions is that $\forall P. \quotep{P}
  \not\in \freenames{P}$.
\end{remark}

\subsection{ Dynamic quote: an example }

Anticipating something of what's to come, consider applying the
substitution, $\widehat{\id{\{}u / z \id{\}}}$, to the following pair
of processes, $\lift{w}{y!(z)}$ and $w[ \lpquote y!(z) \rpquote ]$.

\begin{eqnarray}
	\lift{w}{y!(z)}\widehat{\id{\{}u / z \id{\}}}
		& = &
		\lift{w}{y!(u)} \nonumber\\
	w[ \lpquote y!(z) \rpquote ] \widehat{ \id{\{}u / z \id{\}} }
		& = &
		w[ \lpquote y!(z) \rpquote ] \nonumber
\end{eqnarray}

Because the body of the process between quotes is impervious to
substitution, we get radically different answers. In fact, by
examining the first process in an input context,
e.g. $x?(z).\lift{w}{y!(z)}$, we see that the process under the lift
operator may be shaped by prefixed inputs binding a name inside it. In
this sense, the lift operator will be seen as a way to dynamically
construct processes before reifying them as names.

Finally equipped with these standard features we can present the
dynamics of the calculus.

\subsubsection{Operational semantics} 

Finally, we introduce the computational dynamics. What marks these
algebras as distinct from other more traditionally studied algebraic
structures, e.g. vector spaces or polynomial rings, is the manner in
which dynamics is captured. In traditional structures, dynamics is typically
expressed through morphisms between such structures, as in linear maps
between vector spaces or morphisms between rings. In algebras
associated with the semantics of computation, the dynamics is
expressed as part of the algebraic structure itself, through a
reduction reduction relation typically denoted by $\red$. Below, we
give a recursive presentation of this relation for the calculus used
in the encoding.

$\red \subseteq \pi \times \pi$
$\red : \pi \to \mathcal{P}(\pi)$

\begin{mathpar}
  \inferrule* [lab=Comm] { \textsf{match}( x_{src}, x_{trgt} ) } { x_{trgt}?(y)P \; | \; x_{src}!\langle {Q} \rangle \red P\{\quotep{Q}/y}\} }
  \and \\
  \inferrule* [lab=Par] {{P} \red {P}'} {{{P} | {Q}} \red {{P}' | {Q}}}
  \and
  \inferrule* [lab=Equiv]{{{P} \scong {P}'} \andalso {{P}' \red {Q}'} \andalso {{Q}' \scong {Q}}}{{P} \red {Q}}
\end{mathpar}

\begin{eqnarray*}
  match_{\equiv} (\quotep{P},\quotep{Q}) & := & P \equiv Q \\
  match_{\dagger}(\quotep{P},\quotep{Q}) & := & \forall R. P|Q \red^{*} R => R \red^{*} 0 \\
  match_{K}(\quotep{P},\quotep{Q}) & := & K \mbox{ for some context } K
\end{eqnarray*}

$u?(x)P | u!\langle Q \rangle \red P\{\quotep{Q}/x\}$

%We write $\wred$ for $\red^*$, and $P\red$ if $\exists Q $ such that $ P \red Q$.
We write $P\red$ if $\exists Q $ such that $ P \red Q$ and $P\not\red$, otherwise.

\section{Replication}

As mentioned before, it is known that replication (and hence
recursion) can be implemented in a higher-order process algebra
\cite{SangiorgiWalker}. As our first example of calculation with the
machinery thus far presented we give the construction explicitly in
the {\rhoc}.

\begin{eqnarray}
	D_{x} & := & \prefix{x}{y}{(\binpar{\outputp{x}{y}}{@{y}})} \nonumber\\
	\bangp_{x}{P} & := & \binpar{{x}!\langle{\binpar{D_{x}}{P}}\rangle}{D_{x}} \nonumber
\end{eqnarray}

\begin{eqnarray}
	\bangp_{x}{P} & & \nonumber\\
	=
	& {x}!\langle{(\prefix{x}{y}{(\outputp{x}{y} | @{y})) | P}}\rangle 
	      | \prefix{x}{y}{(\outputp{x}{y} | @{y})} & \nonumber\\
	\red
	& (\outputp{x}{y} | @{y})\substn{\quotep{(\prefix{x}{y}{(@{y} | \outputp{x}{y})) | P}}}{y} & \nonumber\\
	=
	& \outputp{x}{\quotep{(\prefix{x}{y}{(\outputp{x}{y} | @{y})) | P}}}
	  | {(\prefix{x}{y}{(\outputp{x}{y} | @{y})) | P}} & \nonumber\\
	\red
	& \ldots & \nonumber\\
	\red^*
	& P | P | \ldots & \nonumber
\end{eqnarray}

Of course, this encoding, as an implementation, runs away, unfolding
$\bangp{P}$ eagerly. A lazier and more implementable replication
operator, restricted to input-guarded processes, may be obtained as follows.

\begin{eqnarray}
\bangp{\prefix{u}{v}{P}} 
	:= 
	\binpar{\lift{x}{\prefix{u}{v}{(\binpar{D(x)}{P})}}}{D(x)} \nonumber
\end{eqnarray}

\begin{remark}
  Note that the lazier definition still does not deal with summation
  or mixed summation (i.e. sums over input and output). The reader is
  invited to construct definitions of replication that deal with these
  features. 

  Further, the definitions are parameterized in a name, $x$. Can you,
  gentle reader, make a definition that eliminates this parameter and
  guarantees no accidental interaction between the replication
  machinery and the process being replicated -- i.e. no accidental
  sharing of names used by the process to get its work done and the
  name(s) used by the replication to effect copying. This latter
  revision of the definition of replication is crucial to obtaining
  the expected identity $!!P \sim !P$.
\end{remark}

\begin{remark}\label{rem:paradoxical_combinator}
  The reader familiar with the lambda calculus will have noticed the
  similarity between $D$ and the paradoxical combinator.

  [Ed. note: the existence of this seems to suggest we have to be more
  restrictive on the set of processes and names we admit if we are to
  support no-cloning.]
\end{remark}

\subsubsection{Bisimulation}

The computational dynamics gives rise to another kind of equivalence,
the equivalence of computational behavior. As previously mentioned
this is typically captured \emph{via} some form of bisimulation.

% The notion we use in this paper is weak barbed bisimulation
% \cite{milner91polyadicpi}.

The notion we use in this paper is derived from weak barbed
bisimulation \cite{milner91polyadicpi}. 

\begin{definition}
An \emph{observation relation}, $\downarrow_{\mathcal N}$, over a set
of names, $\mathcal N$, is the smallest relation satisfying the rules
below.

\infrule[Out-barb]{y \in {\mathcal N}, \; x \nameeq y}
		  {\outputp{x}{v} \downarrow_{\mathcal N} x}
\infrule[Par-barb]{\mbox{$P\downarrow_{\mathcal N} x$ or $Q\downarrow_{\mathcal N} x$}}
		  {\binpar{P}{Q} \downarrow_{\mathcal N} x}

We write $P \Downarrow_{\mathcal N} x$ if there is $Q$ such that 
$P \wred Q$ and $Q \downarrow_{\mathcal N} x$.
\end{definition}

\begin{definition}
%\label{def.bbisim}
An  ${\mathcal N}$-\emph{barbed bisimulation} over a set of names, ${\mathcal N}$, is a symmetric binary relation 
${\mathcal S}_{\mathcal N}$ between agents such that $P\rel{S}_{\mathcal N}Q$ implies:
\begin{enumerate}
\item If $P \red P'$ then $Q \wred Q'$ and $P'\rel{S}_{\mathcal N} Q'$.
\item If $P\downarrow_{\mathcal N} x$, then $Q\Downarrow_{\mathcal N} x$.
\end{enumerate}
$P$ is ${\mathcal N}$-barbed bisimilar to $Q$, written
$P \wbbisim_{\mathcal N} Q$, if $P \rel{S}_{\mathcal N} Q$ for some ${\mathcal N}$-barbed bisimulation ${\mathcal S}_{\mathcal N}$.
\end{definition}

$\mathcal{R} \subseteq \pi \times \pi$

$P \mathcal{R} Q => \forall P'. P \red P' \Rightarrow \exists Q'. Q \red Q', P' \mathcal{R} Q'$

$P \vdash x \Rightarrow Q \vdash x$

\begin{mathpar}
  \inferrule*[lab=Out-barb]{x \nameeq y}{{y}!\langle{Q}\rangle \vdash x}
  \and
  \inferrule*[lab=Par-barb]{\mbox{$P\vdash x$ or $Q\vdash x$}}{\binpar{P}{Q} \vdash x}
\end{mathpar}

\subsubsection{Contexts}

One of the principle advantages of computational calculi like the
$\pi$-calculus is a well-defined notion of context,
contextual-equivalence and a correlation between
contextual-equivalence and notions of bisimulation. The notion of
context allows the decomposition of a process into (sub-)process and
its syntactic environment, its context. Thus, a context may be
thought of as a process with a ``hole'' (written $\Box$) in it. The
application of a context $M$ to a process $P$, written $M[P]$, is
tantamount to filling the hole in $M$ with $P$. In this paper we do
not need the full weight of this theory, but do make use of the notion
of context in the proof the main theorem. 

\begin{mathpar}
  \inferrule* [lab=summation] {} {{M_{M},M_{N}} \bc \Box \;|\; x.M_{A} \;|\; M_{M}+M_{N}}
  \and
  \inferrule* [lab=agent] {} {{M_{A}} \bc (\vec{x})M_{P} \;| \; \clift{P_0,\ldots,M_{P},\ldots,P_N}}
  \and \\
  \inferrule* [lab=process] {} {{M_{P}} \bc M_{N} \;| \;P|M_{P} }
\end{mathpar} 

\begin{mathpar}
  \inferrule* [lab=sychronization] {} {M_{N} \bc \Box \;|\; x?M_{F} \;|\; x!M_{C}}
  \and
  \inferrule* [lab=abstraction] {} {{M_{F}} \bc (x)M_{P} }
  \and
  \inferrule* [lab=concretion] {} {{M_{C}} \bc \langle M_{P} \rangle }
  \and \\
  \inferrule* [lab=process] {} {{M_{P}} \bc M_{N} \;| \;P|M_{P} }
\end{mathpar}

\begin{definition}[contextual application] Given a context $M$, and
  process $P$, we define the \emph{contextual application}, $M[P] :=
  M\{P/\Box\}$. That is, the contextual application of M to P is the
  substitution of $P$ for $\Box$ in $M$.
\end{definition}

$\meaningof{-} : L \to \mathcal{P}(\pi)$

\begin{mathpar}
  \inferrule* [lab=collection] {} {\meaningof{true} = \pi, \and \meaningof{~E} = \pi \setminus \meaningof{E}, \and \meaningof{E_{1} \& E_{2}} = \meaningof{E_{1}} \cap \meaningof{E_{2}}}
\end{mathpar}

\begin{mathpar}
  \inferrule* [lab=structure] {} {\meaningof{0} = \{ P \in \pi | P \equiv 0 \}, \and \\ \meaningof{E_1 | E_2} = \{ P \in \pi | P \equiv P_{1} | P_{2}, P_{1} \in \meaningof{E_{1}}, P_{2} \in \meaningof{E_2}\} }
\end{mathpar}

\begin{mathpar}
 \inferrule* [lab=behavior] {} {\meaningof{\langle a?b \rangle E} = \{ P \in \pi | P \equiv Q | u?(y)P', \\ \and \\\\ \and \\ \;\;\; u \in \meaningof{a}, \forall z.P'\{z/y\} \in \meaningof{E\{z/b\}}\}, \and \\ \meaningof{a!E} = \{ P \in \pi | P \equiv Q | x!\langle P' \rangle, x \in \meaningof{a} P' \in \meaningof{E}\} }
\end{mathpar}

\begin{mathpar}
 \inferrule* [lab=nominal] {} {\meaningof{\quotep{E}} = \{ \quotep{P} \in \quotep{\pi} | P \in \meaningof{E} \}, \and \meaningof{\quotep{P}} = \{ \quotep{Q} \in \quotep{\pi} | P \equiv Q \} \and \\ \meaningof{@\quotep{E}} = \{ P \in \pi | P \equiv @x, x \in \meaningof{E} \}}
\end{mathpar}

\begin{eqnarray*}
  \\
  \meaningof{-} : TS \to ST
\end{eqnarray*}

\begin{eqnarray*}
  \\
  L : TS \to ST
\end{eqnarray*}

\begin{eqnarray*}
  \\
  P \models E \iff P \in \meaningof{E}
\end{eqnarray*}

\begin{eqnarray*}
  P \approx_{L} Q \iff \forall E \in L. P \models E \iff Q \models E
\end{eqnarray*}

\begin{eqnarray*}
  P \approx_{K} Q
\end{eqnarray*}

\begin{eqnarray*}
  P \approx Q
\end{eqnarray*}

$\approx_{K} = \approx = \approx_{L}$

\subsubsection{Contextual duality}

Note that contexts extend the quotation operation to a family of
operations from processes to names. Given a context, $M$, we can
define a \emph{nominal context}, $\quotep{M}$ by $\quotep{M}[P] :=
\quotep{M[P]}$. To foreshadow what is to come we observe that these
operations enjoy a duality with processes very much like the duality
between vectors and maps from vectors to scalars.

Further, because the calculus is essentially higher-order, we have a
correspondence between contexts and processes. More specifically,
given a name $x$ and a context $M$ we can construct $M^{*}_{x}$ such
that 

\begin{mathpar}
  M^{*}_{x} | \lift{x}{P} \red M[P]
\end{mathpar}

namely,

\begin{mathpar}
  M^{*}_{x} := x?(u).M[\dropn{u}]
\end{mathpar}

The dependence of $M^{*}_{x}$ on a name makes it an abstraction, 

\begin{mathpar}
  M^{*} := (x)x?(u).M[\dropn{u}]
\end{mathpar}

\subsection{Additional notation}

It will sometimes be convenient to denote the process a name
quotes. We already have the notation $x = \quotep{P}$, but it will be
convenient to introduce an alternate notation, $\procn{x}$, when we
want to emphasize the connection to the use of the name. Note that, by
virtue of name equivalence, $\quotep{\procn{x}} \nameeq x$; so, the
notation is consistent with previous definitions.

Further, because names have structure it is possible to effect
substitutions on the basis of that structure. This means we need to
upgrade our notation for substitutions, which we accomplish by
adapting comprehension notation. Thus,

\begin{mathpar}
  P\{ y / x : x \in S \}
\end{mathpar}

is interpreted to mean the process derived from P by replacing (in a
capture-avoiding manner) each occurrence of $x$ in $S$ by $y$. For example,

\begin{mathpar}
  P\{ \quotep{\procn{x}|\procn{x}} / x : x \in \freenames{P} \}
\end{mathpar}

will replace each (occurrence) of a free name $x$ in $P$ by
$\quotep{\procn{x}|\procn{x}}$.

Also, we will avail ourselves of the notation $x^{L}$ and $x^{R}$ to
denote injections of a name into disjoint copies of the name
space. There are numerous ways to accomplish this. One example can be
found in \cite{MeredithR05}. This notation overloads to vectors of
names: $\vec{x}^{\pi} := (x_{i}^{\pi} \; : \; 0 \leq i < |\vec{x}| )$ where $\pi \in \{L,R\}$.

We also use $P^{\Box} := P|\Box$.

In \cite{MeredithR05} an interpretation of the new operator is
given. It turns out that there are several possible interpretations
all enjoying the requisite algebraic properties of the operator (see
\cite{milner91polyadicpi}). We will therefore make liberal use of
$(\nu\; \vec{x})P$.

% subsection the_syntax_and_semantics_of_the_notation_system (end)   

\input{qm2pi.qmops} 

\input{qm2pi.sterngerlach} 

\input{qm2pi.metric} 

% section concurrent_process_calculi (end)

%\input{qm2pi.proofsketch}

% section proof sketch (end)

%\input{qm2pi.slviaknots} 

% section spatial logic via knots (end)

\input{qm2pi.conclusion}

% section conclusion (end)

%\input{qm2pi.dtcodes} 

% section wiring algorithm (end)

\input{qm2pi.ack} 

% section acknowledgments (end)

\newpage


\bibliographystyle{plain}   
\bibliography{../../biblios/main.bib}

\input{qm2pi.rhodetails}

\end{document}

 

% subsection basic_interpretation (end)

%\input{qm2pi.rho.presentation} 
\subsection{The syntax and semantics of the notation system}\label{sub:the_syntax_and_semantics_of_the_notation_system} % (fold)

We now summarize a technical presentation of the calculus that
embodies our theory of dynamics. The typical presentation of such a
calculus follows the style of giving generators and relations on
them. The grammar, below, describing term constructors, freely
generates the set of processes, $\Proc$. This set is then quotiented
by a relation known as structural congruence and it is over this set
that the notion of dynamics is expressed. This presentation is
essentially that of \cite{MeredithR05} with the addition of
polyadicity and summation. For readability we have relegated some of
the technical subtleties to an appendix.

\subsubsection{Process grammar}\label{subsub:process_grammar}

\begin{mathpar}
  \inferrule* [lab=synchronization] {} {{M} \bc \pzero \;|\; x?F \;|\; x!C }
  \and
  \inferrule* [lab=abstraction] {} {{F} \bc (x)P}
  \and
  \inferrule* [lab=concretion] {} {{C} \bc \langle Q \rangle}
  \and
  \inferrule* [lab=process] {} {{P,Q} \bc M \;| \;P|Q \;|\; @{x}}
  \and
  \inferrule* [lab=name] {} {{x} \bc \quotep{P}}
\end{mathpar} 

Note that $\vec{x}$ (resp. $\vec{P}$) denotes a vector of names
(resp. processes) of length $|\vec{x}|$ (resp. $|\vec{P}|$). We adopt
the following useful abbreviations.

\begin{mathpar}
   x?(\vec{y}).P := x.(\vec{y})P \and  x\clift{\vec{P}} := x.\clift{\vec{P}}
   \and x!(y) := \lift{x}{\dropn{y}}
   \and \Pi_{i=0}^{n-1}P_i := P_0 | \ldots | P_{n-1}
\end{mathpar}

\subsubsection{Structural congruence}

\paragraph{Free and bound names and alpha-equivalence.} At the
core of structural equivalence is alpha-equivalence which identifies
process that are the same up to a change of variable. Formally, we
recognize the distinction between free and bound names. The free names
of a process, $\freenames{P}$, may be calculated recursively as
follows:

\begin{mathpar}
\freenames{\pzero} := \emptyset
  \and \\
  \freenames{x?(y).P} := \{ x \} \cup (\freenames{P} \setminus \{ y \})
  \and 
  \freenames{x!\langle P \rangle} := \{ x \} \cup \{ P \} 
  \and \\
  \freenames{P|Q} := \freenames{P} \cup \freenames{Q}
  \and \\
  \freenames{@{x}} := \{ x \}
\end{mathpar}

$\pi$
$\quotep{\pi}$

$\freenames{-} : \pi \to \mathcal{P}(\quotep{\pi})$

\begin{eqnarray*}
  \freenames{\pzero} & := & \emptyset \\
  \freenames{x?(y).P} & := & \{ x \} \cup (\freenames{P} \setminus \{ y \}) \\
  \freenames{x!\langle P \rangle} & := & \{ x \} \cup \{ P \} \\
  \freenames{P|Q} & := & \freenames{P} \cup \freenames{Q} \\
  \freenames{\dropn{x}} & := & \{ x \}
\end{eqnarray*}

The bound names of a process, $\boundnames{P}$, are those names occurring in $P$
that are not free. For example, in $x?(y).0$, the name $x$ is free, while $y$ is bound.

\begin{mathpar}
  \inferrule* [lab=monoidal-laws] {} { P|Q \equiv Q|P \and P|0 \equiv P \and P|(Q|R) \equiv (P|Q)|R }
\end{mathpar}

\begin{mathpar}
  \inferrule* [lab=alpha-equivalence] {} { (x)P \equiv (y)P\{y/x\} \and y \not\in \freenames{P} }
\end{mathpar}

\begin{definition}
Then two processes, $P,Q$, are alpha-equivalent if $P = Q\{\vec{y}/\vec{x}\}$ for
some $\vec{x} \in \boundnames{Q},\vec{y} \in \boundnames{P}$, where $Q\{\vec{y}/\vec{x}\}$
denotes the capture-avoiding substitution of $\vec{y}$ for $\vec{x}$ in $Q$.
\end{definition}

\begin{definition}
  The {\em structural congruence} \cite{SangiorgiWalker} , $\equiv$,
  between processes is the least congruence containing
  alpha-equivalence, satisfying the abelian monoid laws
  (associativity, commutativity and $\pzero$ as identity) for parallel
  composition $|$ and for summation $+$.
\end{definition}

\subsection{Name equivalence}

We take name equivalence, written $\nameeq$, to be the smallest
equivalence relation generated by the following rules.

\begin{mathpar}
\inferrule*[lab=Quote-drop]
{ }
{ \quotep{@{x}} \nameeq x }

\inferrule*[lab=Struct-equiv]
{ P \scong Q }
{ \quotep{P} \nameeq \quotep{Q} }
\end{mathpar}

The astute reader will have noticed that the mutual recursion of names
and processes imposes a mutual recursion on alpha-equivalence and
structural equivalence via name-equivalence. Fortunately, all of this
works out pleasantly and we may calculate in the natural way, free of
concern. The reader interested in the details is referred to the
appendix \ref{appendix:rho_details}.

\subsection{Substitution}

We use $\Proc$ for the set of processes, $\QProc$ for the set of
names, and $\id{\{}\vec{y} / \vec{x} \id{\}}$ to denote partial maps,
$s : \QProc \rightarrow \QProc$. A map, $s$ lifts, uniquely, to a map
on process terms, $\widehat{s} : \Proc \rightarrow \Proc$ by the
following equations.

\begin{mathpar}
  (0) \psubstp{Q}{P} := 0 \\
  (R \juxtap S) \psubstp{Q}{P}
  :=    
  (R)\psubstp{Q}{P} \juxtap (S) \psubstp{Q}{P} \\
  (x?(y).R) \psubstp{Q}{P}    
  :=    
  (x)\substp{Q}{P} (z)\concat( (R \psubstn{z}{y}) \psubstp{Q}{P} ) \\
  (\lift{x}{R}) \psubstp{Q}{P}  
  :=
  \lift{(x)\substp{Q}{P}}{ R \psubstp{Q}{P} } \\
%   (\dropn{x})  \psubstp{Q}{P}       
%   := 
%   \left\{ 
%     \begin{array}{ccc} 
%       \dropn{\quotep{Q}} & & x \nameeq \quotep{P} \\
%       \dropn{x} & & otherwise \\
%     \end{array}
%   \right. 
  (\dropn{x})  \psubstp{Q}{P}       
  := 
  \left\{ 
    \begin{array}{ccc} 
      Q & & x \nameeq \quotep{P} \\
      \dropn{x} & & otherwise \\
    \end{array}
  \right.
\end{mathpar}
 

where

\begin{eqnarray}
  (x)\id{\{} \lpquote Q \rpquote / \lpquote P \rpquote \id{\}}            = 
  \left\{ 
    \begin{array}{ccc}
      \lpquote Q \rpquote & & x \nameeq \lpquote P \rpquote \\
      x & & otherwise \\
    \end{array}
  \right. \nonumber
\end{eqnarray}

and $z$ is chosen distinct from $\quotep{P}$, $\quotep{Q}$, the free
names in $Q$, and all the names in $R$. Our $\alpha$-equivalence will
be built in the standard way from this substitution.

\begin{remark}\label{rem:no_self_referential_names}
  One consequence of these definitions is that $\forall P. \quotep{P}
  \not\in \freenames{P}$.
\end{remark}

\subsection{ Dynamic quote: an example }

Anticipating something of what's to come, consider applying the
substitution, $\widehat{\id{\{}u / z \id{\}}}$, to the following pair
of processes, $\lift{w}{y!(z)}$ and $w[ \lpquote y!(z) \rpquote ]$.

\begin{eqnarray}
	\lift{w}{y!(z)}\widehat{\id{\{}u / z \id{\}}}
		& = &
		\lift{w}{y!(u)} \nonumber\\
	w[ \lpquote y!(z) \rpquote ] \widehat{ \id{\{}u / z \id{\}} }
		& = &
		w[ \lpquote y!(z) \rpquote ] \nonumber
\end{eqnarray}

Because the body of the process between quotes is impervious to
substitution, we get radically different answers. In fact, by
examining the first process in an input context,
e.g. $x?(z).\lift{w}{y!(z)}$, we see that the process under the lift
operator may be shaped by prefixed inputs binding a name inside it. In
this sense, the lift operator will be seen as a way to dynamically
construct processes before reifying them as names.

Finally equipped with these standard features we can present the
dynamics of the calculus.

\subsubsection{Operational semantics} 

Finally, we introduce the computational dynamics. What marks these
algebras as distinct from other more traditionally studied algebraic
structures, e.g. vector spaces or polynomial rings, is the manner in
which dynamics is captured. In traditional structures, dynamics is typically
expressed through morphisms between such structures, as in linear maps
between vector spaces or morphisms between rings. In algebras
associated with the semantics of computation, the dynamics is
expressed as part of the algebraic structure itself, through a
reduction reduction relation typically denoted by $\red$. Below, we
give a recursive presentation of this relation for the calculus used
in the encoding.

$\red \subseteq \pi \times \pi$
$\red : \pi \to \mathcal{P}(\pi)$

\begin{mathpar}
  \inferrule* [lab=Comm] { \textsf{match}( x_{src}, x_{trgt} ) } { x_{trgt}?(y)P \; | \; x_{src}!\langle {Q} \rangle \red P\{\quotep{Q}/y}\} }
  \and \\
  \inferrule* [lab=Par] {{P} \red {P}'} {{{P} | {Q}} \red {{P}' | {Q}}}
  \and
  \inferrule* [lab=Equiv]{{{P} \scong {P}'} \andalso {{P}' \red {Q}'} \andalso {{Q}' \scong {Q}}}{{P} \red {Q}}
\end{mathpar}

\begin{eqnarray*}
  match_{\equiv} (\quotep{P},\quotep{Q}) & := & P \equiv Q \\
  match_{\dagger}(\quotep{P},\quotep{Q}) & := & \forall R. P|Q \red^{*} R => R \red^{*} 0 \\
  match_{K}(\quotep{P},\quotep{Q}) & := & K \mbox{ for some context } K
\end{eqnarray*}

$u?(x)P | u!\langle Q \rangle \red P\{\quotep{Q}/x\}$

%We write $\wred$ for $\red^*$, and $P\red$ if $\exists Q $ such that $ P \red Q$.
We write $P\red$ if $\exists Q $ such that $ P \red Q$ and $P\not\red$, otherwise.

\section{Replication}

As mentioned before, it is known that replication (and hence
recursion) can be implemented in a higher-order process algebra
\cite{SangiorgiWalker}. As our first example of calculation with the
machinery thus far presented we give the construction explicitly in
the {\rhoc}.

\begin{eqnarray}
	D_{x} & := & \prefix{x}{y}{(\binpar{\outputp{x}{y}}{@{y}})} \nonumber\\
	\bangp_{x}{P} & := & \binpar{{x}!\langle{\binpar{D_{x}}{P}}\rangle}{D_{x}} \nonumber
\end{eqnarray}

\begin{eqnarray}
	\bangp_{x}{P} & & \nonumber\\
	=
	& {x}!\langle{(\prefix{x}{y}{(\outputp{x}{y} | @{y})) | P}}\rangle 
	      | \prefix{x}{y}{(\outputp{x}{y} | @{y})} & \nonumber\\
	\red
	& (\outputp{x}{y} | @{y})\substn{\quotep{(\prefix{x}{y}{(@{y} | \outputp{x}{y})) | P}}}{y} & \nonumber\\
	=
	& \outputp{x}{\quotep{(\prefix{x}{y}{(\outputp{x}{y} | @{y})) | P}}}
	  | {(\prefix{x}{y}{(\outputp{x}{y} | @{y})) | P}} & \nonumber\\
	\red
	& \ldots & \nonumber\\
	\red^*
	& P | P | \ldots & \nonumber
\end{eqnarray}

Of course, this encoding, as an implementation, runs away, unfolding
$\bangp{P}$ eagerly. A lazier and more implementable replication
operator, restricted to input-guarded processes, may be obtained as follows.

\begin{eqnarray}
\bangp{\prefix{u}{v}{P}} 
	:= 
	\binpar{\lift{x}{\prefix{u}{v}{(\binpar{D(x)}{P})}}}{D(x)} \nonumber
\end{eqnarray}

\begin{remark}
  Note that the lazier definition still does not deal with summation
  or mixed summation (i.e. sums over input and output). The reader is
  invited to construct definitions of replication that deal with these
  features. 

  Further, the definitions are parameterized in a name, $x$. Can you,
  gentle reader, make a definition that eliminates this parameter and
  guarantees no accidental interaction between the replication
  machinery and the process being replicated -- i.e. no accidental
  sharing of names used by the process to get its work done and the
  name(s) used by the replication to effect copying. This latter
  revision of the definition of replication is crucial to obtaining
  the expected identity $!!P \sim !P$.
\end{remark}

\begin{remark}\label{rem:paradoxical_combinator}
  The reader familiar with the lambda calculus will have noticed the
  similarity between $D$ and the paradoxical combinator.

  [Ed. note: the existence of this seems to suggest we have to be more
  restrictive on the set of processes and names we admit if we are to
  support no-cloning.]
\end{remark}

\subsubsection{Bisimulation}

The computational dynamics gives rise to another kind of equivalence,
the equivalence of computational behavior. As previously mentioned
this is typically captured \emph{via} some form of bisimulation.

% The notion we use in this paper is weak barbed bisimulation
% \cite{milner91polyadicpi}.

The notion we use in this paper is derived from weak barbed
bisimulation \cite{milner91polyadicpi}. 

\begin{definition}
An \emph{observation relation}, $\downarrow_{\mathcal N}$, over a set
of names, $\mathcal N$, is the smallest relation satisfying the rules
below.

\infrule[Out-barb]{y \in {\mathcal N}, \; x \nameeq y}
		  {\outputp{x}{v} \downarrow_{\mathcal N} x}
\infrule[Par-barb]{\mbox{$P\downarrow_{\mathcal N} x$ or $Q\downarrow_{\mathcal N} x$}}
		  {\binpar{P}{Q} \downarrow_{\mathcal N} x}

We write $P \Downarrow_{\mathcal N} x$ if there is $Q$ such that 
$P \wred Q$ and $Q \downarrow_{\mathcal N} x$.
\end{definition}

\begin{definition}
%\label{def.bbisim}
An  ${\mathcal N}$-\emph{barbed bisimulation} over a set of names, ${\mathcal N}$, is a symmetric binary relation 
${\mathcal S}_{\mathcal N}$ between agents such that $P\rel{S}_{\mathcal N}Q$ implies:
\begin{enumerate}
\item If $P \red P'$ then $Q \wred Q'$ and $P'\rel{S}_{\mathcal N} Q'$.
\item If $P\downarrow_{\mathcal N} x$, then $Q\Downarrow_{\mathcal N} x$.
\end{enumerate}
$P$ is ${\mathcal N}$-barbed bisimilar to $Q$, written
$P \wbbisim_{\mathcal N} Q$, if $P \rel{S}_{\mathcal N} Q$ for some ${\mathcal N}$-barbed bisimulation ${\mathcal S}_{\mathcal N}$.
\end{definition}

$\mathcal{R} \subseteq \pi \times \pi$

$P \mathcal{R} Q => \forall P'. P \red P' \Rightarrow \exists Q'. Q \red Q', P' \mathcal{R} Q'$

$P \vdash x \Rightarrow Q \vdash x$

\begin{mathpar}
  \inferrule*[lab=Out-barb]{x \nameeq y}{{y}!\langle{Q}\rangle \vdash x}
  \and
  \inferrule*[lab=Par-barb]{\mbox{$P\vdash x$ or $Q\vdash x$}}{\binpar{P}{Q} \vdash x}
\end{mathpar}

\subsubsection{Contexts}

One of the principle advantages of computational calculi like the
$\pi$-calculus is a well-defined notion of context,
contextual-equivalence and a correlation between
contextual-equivalence and notions of bisimulation. The notion of
context allows the decomposition of a process into (sub-)process and
its syntactic environment, its context. Thus, a context may be
thought of as a process with a ``hole'' (written $\Box$) in it. The
application of a context $M$ to a process $P$, written $M[P]$, is
tantamount to filling the hole in $M$ with $P$. In this paper we do
not need the full weight of this theory, but do make use of the notion
of context in the proof the main theorem. 

\begin{mathpar}
  \inferrule* [lab=summation] {} {{M_{M},M_{N}} \bc \Box \;|\; x.M_{A} \;|\; M_{M}+M_{N}}
  \and
  \inferrule* [lab=agent] {} {{M_{A}} \bc (\vec{x})M_{P} \;| \; \clift{P_0,\ldots,M_{P},\ldots,P_N}}
  \and \\
  \inferrule* [lab=process] {} {{M_{P}} \bc M_{N} \;| \;P|M_{P} }
\end{mathpar} 

\begin{mathpar}
  \inferrule* [lab=sychronization] {} {M_{N} \bc \Box \;|\; x?M_{F} \;|\; x!M_{C}}
  \and
  \inferrule* [lab=abstraction] {} {{M_{F}} \bc (x)M_{P} }
  \and
  \inferrule* [lab=concretion] {} {{M_{C}} \bc \langle M_{P} \rangle }
  \and \\
  \inferrule* [lab=process] {} {{M_{P}} \bc M_{N} \;| \;P|M_{P} }
\end{mathpar}

\begin{definition}[contextual application] Given a context $M$, and
  process $P$, we define the \emph{contextual application}, $M[P] :=
  M\{P/\Box\}$. That is, the contextual application of M to P is the
  substitution of $P$ for $\Box$ in $M$.
\end{definition}

$\meaningof{-} : L \to \mathcal{P}(\pi)$

\begin{mathpar}
  \inferrule* [lab=collection] {} {\meaningof{true} = \pi, \and \meaningof{~E} = \pi \setminus \meaningof{E}, \and \meaningof{E_{1} \& E_{2}} = \meaningof{E_{1}} \cap \meaningof{E_{2}}}
\end{mathpar}

\begin{mathpar}
  \inferrule* [lab=structure] {} {\meaningof{0} = \{ P \in \pi | P \equiv 0 \}, \and \\ \meaningof{E_1 | E_2} = \{ P \in \pi | P \equiv P_{1} | P_{2}, P_{1} \in \meaningof{E_{1}}, P_{2} \in \meaningof{E_2}\} }
\end{mathpar}

\begin{mathpar}
 \inferrule* [lab=behavior] {} {\meaningof{\langle a?b \rangle E} = \{ P \in \pi | P \equiv Q | u?(y)P', \\ \and \\\\ \and \\ \;\;\; u \in \meaningof{a}, \forall z.P'\{z/y\} \in \meaningof{E\{z/b\}}\}, \and \\ \meaningof{a!E} = \{ P \in \pi | P \equiv Q | x!\langle P' \rangle, x \in \meaningof{a} P' \in \meaningof{E}\} }
\end{mathpar}

\begin{mathpar}
 \inferrule* [lab=nominal] {} {\meaningof{\quotep{E}} = \{ \quotep{P} \in \quotep{\pi} | P \in \meaningof{E} \}, \and \meaningof{\quotep{P}} = \{ \quotep{Q} \in \quotep{\pi} | P \equiv Q \} \and \\ \meaningof{@\quotep{E}} = \{ P \in \pi | P \equiv @x, x \in \meaningof{E} \}}
\end{mathpar}

\begin{eqnarray*}
  \\
  \meaningof{-} : TS \to ST
\end{eqnarray*}

\begin{eqnarray*}
  \\
  L : TS \to ST
\end{eqnarray*}

\begin{eqnarray*}
  \\
  P \models E \iff P \in \meaningof{E}
\end{eqnarray*}

\begin{eqnarray*}
  P \approx_{L} Q \iff \forall E \in L. P \models E \iff Q \models E
\end{eqnarray*}

\begin{eqnarray*}
  P \approx_{K} Q
\end{eqnarray*}

\begin{eqnarray*}
  P \approx Q
\end{eqnarray*}

$\approx_{K} = \approx = \approx_{L}$

\subsubsection{Contextual duality}

Note that contexts extend the quotation operation to a family of
operations from processes to names. Given a context, $M$, we can
define a \emph{nominal context}, $\quotep{M}$ by $\quotep{M}[P] :=
\quotep{M[P]}$. To foreshadow what is to come we observe that these
operations enjoy a duality with processes very much like the duality
between vectors and maps from vectors to scalars.

Further, because the calculus is essentially higher-order, we have a
correspondence between contexts and processes. More specifically,
given a name $x$ and a context $M$ we can construct $M^{*}_{x}$ such
that 

\begin{mathpar}
  M^{*}_{x} | \lift{x}{P} \red M[P]
\end{mathpar}

namely,

\begin{mathpar}
  M^{*}_{x} := x?(u).M[\dropn{u}]
\end{mathpar}

The dependence of $M^{*}_{x}$ on a name makes it an abstraction, 

\begin{mathpar}
  M^{*} := (x)x?(u).M[\dropn{u}]
\end{mathpar}

\subsection{Additional notation}

It will sometimes be convenient to denote the process a name
quotes. We already have the notation $x = \quotep{P}$, but it will be
convenient to introduce an alternate notation, $\procn{x}$, when we
want to emphasize the connection to the use of the name. Note that, by
virtue of name equivalence, $\quotep{\procn{x}} \nameeq x$; so, the
notation is consistent with previous definitions.

Further, because names have structure it is possible to effect
substitutions on the basis of that structure. This means we need to
upgrade our notation for substitutions, which we accomplish by
adapting comprehension notation. Thus,

\begin{mathpar}
  P\{ y / x : x \in S \}
\end{mathpar}

is interpreted to mean the process derived from P by replacing (in a
capture-avoiding manner) each occurrence of $x$ in $S$ by $y$. For example,

\begin{mathpar}
  P\{ \quotep{\procn{x}|\procn{x}} / x : x \in \freenames{P} \}
\end{mathpar}

will replace each (occurrence) of a free name $x$ in $P$ by
$\quotep{\procn{x}|\procn{x}}$.

Also, we will avail ourselves of the notation $x^{L}$ and $x^{R}$ to
denote injections of a name into disjoint copies of the name
space. There are numerous ways to accomplish this. One example can be
found in \cite{MeredithR05}. This notation overloads to vectors of
names: $\vec{x}^{\pi} := (x_{i}^{\pi} \; : \; 0 \leq i < |\vec{x}| )$ where $\pi \in \{L,R\}$.

We also use $P^{\Box} := P|\Box$.

In \cite{MeredithR05} an interpretation of the new operator is
given. It turns out that there are several possible interpretations
all enjoying the requisite algebraic properties of the operator (see
\cite{milner91polyadicpi}). We will therefore make liberal use of
$(\nu\; \vec{x})P$.

% subsection the_syntax_and_semantics_of_the_notation_system (end)   

\section{Interpretation of QM}
\subsection{Supporting definitions}
\subsubsection{Multiplication}
\begin{mathpar}
  \quotep{Q} \cdot \quotep{R} := \quotep{Q|R}
  \and \\
  \quotep{Q} \cdot P := P\{ \quotep{Q|R} / \quotep{R} : \quotep{R} \in \freenames{P} \}
\end{mathpar}

\paragraph{Discussion}
The first line needs little explanation. The second line says that
each free name of the process is replaced with the multiplication of
that name by the scalar. Multiplication of a scalar (name) by a state
(process) results in a process all the names of which have been `moved
over' by parallel composition with the process the scalar
quotes. There is a subtlety that the bound names have to be
manipulated so that multiplied names aren't accidentally
captured. There are many ways to achieve this.

\begin{remark}\label{rem:multiplication_identities}
  The reader is invited to verify that for all $x,y,z \in \QProc$ and $P \in \Proc$
  \begin{mathpar}
    x \cdot \quotep{0} \equiv x 
    \and
    x \cdot y \equiv y \cdot x
    \and
    x \cdot (y \cdot z) \equiv (x \cdot y) \cdot z
    \and \\
    \quotep{0} \cdot P \equiv P
    \and \\
    x \cdot (y \cdot P) \equiv (x \cdot y) \cdot P
    \and \\
    x \cdot (P|Q) \equiv (x \cdot P) | (x \cdot Q)
    \and \\    
  \end{mathpar}
\end{remark}

\subsubsection{Tensor product}

We define a tensor product on processes by structural induction.

\paragraph{Tensor of sums} First note that all summations, including
$\pzero$ and sequence, can be written $\Sigma_{i} x_{i}.A_{i} +
\Sigma_{j} x_{j}.C_{j}$, where we have grouped input-guarded processes
together and output-guarded processes together.

Thus, we can define the tensor product of two summations, $N_{1}\otimes N_{2}$, where

\begin{mathpar}
  N_{1} := \Sigma_{i} x_{i}.A_{i} + \Sigma_{j} x_{j}.C_{j}
  \and
  N_{2} := \Sigma_{i'} y_{i'}.B_{i'} + \Sigma_{j'} y_{j'}.D_{j'} 
\end{mathpar}

as follows.

\begin{mathpar}
  \Sigma_{i} x_{i}.A_{i} + \Sigma_{j} x_{j}.C_{j} \otimes \Sigma_{i'}
  y_{i'}.B_{i'} + \Sigma_{j'} y_{j'}.D_{j'} 
  \and \\
  := \; \Sigma_{i} \Sigma_{i'} \quotep{\stackrel{\vee}{x_{i}}| \stackrel{\vee}{y_{i'}}}.(A_{i}\otimes B_{i'}) \; | \; \Sigma_{i'} \Sigma_{i} \quotep{\stackrel{\vee}{y_{i'}}|\stackrel{\vee}{x_{i}}}.(B_{i'}\otimes A_{i})
  \and
  \;\; | \;\; \Sigma_{j} \Sigma_{j'} \quotep{\stackrel{\vee}{x_{j}}|\stackrel{\vee}{y_{j'}}}.(A_{j}\otimes B_{j'}) \; | \; \Sigma_{j'} \Sigma_{j} \quotep{\stackrel{\vee}{y_{j'}}|\stackrel{\vee}{x_{j}}}.(B_{j'}\otimes A_{j})
\end{mathpar}

\begin{remark}
  Do we need to $x^{L}$ and $y^{R}$ for this construction as well?
\end{remark}

\paragraph{Tensor of parallel compositions} Next, we distribute tensor
over par.

\begin{mathpar}
  P_{1}|P_{2} \otimes Q_{1}|Q_{2} := (P_{1} \otimes Q_{1}) | (P_{1}
  \otimes Q_{2}) | (P_{2} \otimes Q_{1}) | (P_{2} \otimes Q_{2})
\end{mathpar}

\paragraph{Tensor with dropped names} We treat tensor of a
process with a dropped name as parallel composition.

\begin{mathpar}
  P \otimes \dropn{x} := P | \dropn{x}
\end{mathpar}

\paragraph{Tensor of agents}

Finally, we need to define tensor on agents. Note that the definition
of tensor on normal products only tensors inputs with inputs and
outputs with outputs. Thus, we only have to define the operation on
``homogeneous'' pairings.

\begin{mathpar}
  (\vec{x})P \otimes (\vec{y})Q
  \and \\
  := (x_{0}^{L}|y_{0}^{R},\ldots,x_{0}^{L}|y_{n}^{R},\ldots,x_{m}^{L}|y_{0}^{R},\ldots,x_{m}^{L}|y_{n}^R)(P\{ \vec{x}^{L}/\vec{x}\} \otimes Q \{ \vec{y}^{R}/\vec{y}\})
  \and \\
  \clift{\vec{P}} \otimes \clift{\vec{Q}}
  \and \\
  := \clift{P_{0}\otimes Q_{0},\ldots,P_{0}\otimes Q_{n},\ldots,P_{m}\otimes Q_{0},\ldots,P_{m}\otimes Q_{n}}
\end{mathpar}

\begin{remark}
  Observe that arities of tensored abstractions matches arities of
  tensored concretions if the original arities matched. Note also that
  the length of the arities corresponds to the increase in dimension
  we see in ordinary vector space tensor product.
\end{remark}

\begin{remark}
  Operationally, this definition distributes the tensor down to
  components ``linked'' by summation. Tensor over summation is
  intriguing in that it mixes names. Moreover, as a consequence of the
  way it mixes names we have the identities for all $x \in \QProc$ and
  $P,Q \in \Proc$

  \begin{mathpar}
    (x \cdot P) \otimes Q \equiv x \cdot (P \otimes Q) \equiv P \otimes (x \cdot Q)
    \and
    P \otimes \pzero \equiv P
  \end{mathpar}

  that the reader is invited to verify.
\end{remark}

\subsubsection{Annihilation}
\begin{mathpar}
  P^{\perp} := \{ Q | \forall R. P|Q \red^{*} R \Rightarrow R \red^{*} \pzero \}
  \and \\
  P^{\underline{\perp}} := \Sigma_{Q \in P^{\perp}} \quotep{Q}?(y).(\dropn{y}|Q) | \Sigma_{Q \in P^{\perp}} \quotep{Q}\clift{\Box}
\end{mathpar}

\paragraph{Discussion} The reader will note that $P^{\perp}$ is a
\emph{set} of processes, while $P^{\underline{\perp}}$ is a
\emph{context}. We call the set $P^{\perp}$ the \emph{annihilators} of
$P$. The parallel composition of a process in the annihilators of $P$
with $P$ will result in a process, the state space of which has all
paths eventually leading to $\pzero$. Execution may endure loops; but
under reasonable conditions of fairness (naturally guaranteed under
most notions of bisimulation) such a composite process cannot get
stuck in such a loop and will, eventually pop out and terminate.

The context $P^{\underline{\perp}}$ is ready and willing to ``take the
$P$ out of'' the process to which it is applied. It will effectively
transmit the code of the process to which it is applied to one of the
annihilators and run the process against it.

\subsubsection{Evaluation}
We fix $M$ a domain of fully abstract interpretation with an equality
coincident with bisimulation. We take $\meaningof{\cdot} : \Proc \to
M$ to be the map interpreting processes and $\nmeaningof{\cdot} : \M
\to Proc$ to be the map running the other way. Then we define

\begin{mathpar}
  \int P := \nmeaningof{\meaningof{P}}
\end{mathpar}

\paragraph{Discussion}
There are many fully abstract interpretations of Milner's
$\pi$-calculus. Any of them can be used as a basis for interpreting
the reflective calculus here. Equipped with such a domain it is
largely a matter of grinding through to check that the Yoneda
construction for the normalization-by-evaluation program can be
extended to this setting.

\begin{remark}
  The reader is invited to verify that $\int (P^{\underline{\perp}}[P]) = 0$.
\end{remark}

\subsection{Quantum mechanics}

Table \ref{tbl:core_qm_op_defns} gives the core operational definitions

\begin{table}[htp]\label{tbl:core_qm_op_defns}
  \center{
    \fbox{
      \begin{tabular}{c|c}
        quantum mechanics & process calculus \\
        \hline
        scalar & $x := \quotep{P}$ \\
        state vector & $\state{P} := P$ \\
        dual & $\state{P}^{*} := \event{P^{\underline{\perp}}} := \quotep{P^{\underline{\perp}}}[-]$ \\
        matrix & $ \Sigma_{\alpha} \state{P_{\alpha}}x_{\alpha}\event{Q_{\alpha}}$ \\
        vector addition & $\state{P} + \state{Q} := \state{P | Q}$ \\
        tensor product & $\state{P} \otimes \state{Q} := \state{P \otimes Q}$ \\
        inner product & $\innerprod{P}{Q} := \quotep{\int P^{\underline{\perp}}[Q]}$ \\
      \end{tabular}
    }
  }
  \caption{QM - operational definitions}
\end{table}

where

\begin{mathpar}
  \prmatrix{P}{Q} := \fprmatrix{P}{\quotep{\pzero}}{Q}
  \and
  \fprmatrix{P}{x}{Q} := (\state{P},x,\event{Q})
  \and
  (\fprmatrix{P}{x}{Q})(\state{R}) := x \cdot \innerprod{Q}{R} \cdot \state{P}
  \and
  (\fprmatrix{P}{x}{Q})(\event{R}) := x \cdot \innerprod{R}{P} \cdot \event{Q}
\end{mathpar}

\paragraph{Discussion}
As promised: vectors (aka states) are represented as processes; duals
as contextual duals; inner product definition should be compared with
standard inner product definition for ....

\begin{remark}
  Assuming $\int (P^{\underline{\perp}}[P]) = 0$, the reader is
  invited to verify that $(\fprmatrix{P}{x}{P})(\state{P}) = x \cdot \state{P}$.
\end{remark}

\begin{remark}
  The reader is invited to verify that $\innerprod{P}{Q}$ could
  equally well have been written $\quotep{\int \stackrel{\vee}{x}}$
  where $x = \event{P^{\underline{\perp}}}(Q)$.

  One of the motivations for this remark is that there is another way
  to factor these operations. We could package up evaluation in the dual:

  \begin{mathpar}
    \state{P}^{*} := \event{\int P^{\underline{\perp}}} := \quotep{\int P^{\underline{\perp}}}[-]
  \end{mathpar}

  and then have inner product defined by
  
  \begin{mathpar}
    \innerprod{P}{Q} := \event{P}(Q)
  \end{mathpar}

  Hopefully, experience with the calculations will provide guidance on
  the best factoring.
\end{remark}

\begin{remark}
  Assuming $\int (P^{\underline{\perp}}[P]) = 0$, the reader is
  invited to verify that $\forall P,Q. (\prmatrix{0}{Q})(\state{0}) =
  \state{0}$ and dually $(\prmatrix{P}{0})(\event{0}) = \event{0}$.
\end{remark}

\begin{remark}
  i'm a little worried that i don't (yet) have proper support for
  complex conjugacy. But, the observation above may give us a
  clue. According to Abramsky, it must be the case that the scalars
  are iso to the homset of the identity for the tensor -- which the
  observation above characterizes. 

  For now, we will simply bookmark the notion with $\overline{x}$.
\end{remark}

\subsubsection{Adjointness}

We need to give a definition of $(\cdot)^{\dagger}$ for matrices. The
obvious candidate definition is
\begin{mathpar}
(\Sigma_{\alpha}\fprmatrix{P_{\alpha}}{x_{\alpha}}{Q_{\alpha}})^{\dagger}
= \Sigma_{\alpha}\fprmatrix{(Q_{\alpha}^{\underline{\perp}})^{*}}{\overline{x}_{\alpha}}{P_{\alpha}^{\underline{\perp}}} 
\end{mathpar}

But, $(Q_{\alpha}^{\underline{\perp}})^{*}$ requires a name along
which to communicate the process to achieve the context application.

\subsubsection{Basis for a basis}
If processes label states and ``addition'' of states (a.k.a. vector
addition) is interpreted as parallel composition, what corresponds to
notions of linear independence and basis? Here, we recall that Yoshida
has developed a set of \emph{combinators} for an asynchronous verison
of Milner's $\pi$-calculus. These are a finite set of processes such
any process can be expressed as parallel composition of these
combinators together with liberal uses of the new operator and
replication. We can simply give a translation of these into the
present calculus and have reasonable expectation that the property
carries over. That is, that the resultant set allows to express all
processes via parallel composition. Note, however, that there is no
new operator or replication in this calculus. As a result, we expect
that the corresponding set is actually infinite. That is, we expect
that the space is actually infinite dimensional.

\begin{remark}
  The attentive reader may be a bit concerned. Certainly, the
  collection $S$, $K$ and $I$ is a finite set of
  combinators. Shouldn't we expect to see a finite set of combinators
  for an effectively equivalent system? i am very sympathetic to this
  critique and feel it warrants full attention. On the other hand, i
  also have in mind the following analogy. The natural numbers, as a
  monoid under addition, has exactly $1$ generator, while the natural
  numbers, as a monoid under multiplication, has countably many
  generators (the primes). We observe that the application of the
  lambda calculus is much less resource sensitive than the parallel
  composition of the $\pi$-calculus. Could it be the case that we have
  an analogy of the form
  
  \begin{mathpar}
    m + n : MN :: m*n : M|N
  \end{mathpar}

  giving a similar blow up in the set of ``primes''?  This is such a
  wonderful thought that, even if it's not true, i think it's worth
  writing down.
\end{remark}
 

\documentclass[12pt]{llncs}
%\documentclass{jktr}

\usepackage[pdftex]{hyperref}                   
\usepackage {listings}
\usepackage {mathpartir}
\usepackage{bcprules}
%\usepackage{listings}
                       
\usepackage{graphicx} 
%\usepackage[margins=2.5cm,nohead,nofoot]{geometry}
%\usepackage{geometry}
\usepackage{amsfonts}
\usepackage{amstext}
\usepackage{latexsym}
\usepackage{amssymb}
\usepackage{color}


%\include{myPreamble}
\include{qm2pi.local} 

%\ifpdf
%\usepackage[pdftex]{graphicx}
%\else
%\usepackage{graphicx}
%\fi

 % \ifpdf
%  \usepackage{pdfsync}
%  \if


%\title{Brief Article}
%\author{David F. Snyder}
%\author{L.G. Meredith}

%\address{Dept. of Math., Texas State University--San Marcos, San Marcos, TX 78666}
       
\pagestyle{empty}


\begin{document}

\lstset{language=[Objective]Caml,frame=shadowbox}

\input{qm2pi.front}

% section front matter (end)

\input{qm2pi.intro} 
 
% section introduction (end)

% \input{qm2pi.knotations} 

% section notation (end)

\input{qm2pi.process.calculi} 

% section concurrent_process_calculi_and_spatial_logics_ (end)
    
%\input{qm2pi.knots2pi} 

%\input{qm2pi.trefoil} 

%\input{qm2pi.mainthm} 

% subsection basic_interpretation (end)

%\input{qm2pi.rho.presentation} 
\subsection{The syntax and semantics of the notation system}\label{sub:the_syntax_and_semantics_of_the_notation_system} % (fold)

We now summarize a technical presentation of the calculus that
embodies our theory of dynamics. The typical presentation of such a
calculus follows the style of giving generators and relations on
them. The grammar, below, describing term constructors, freely
generates the set of processes, $\Proc$. This set is then quotiented
by a relation known as structural congruence and it is over this set
that the notion of dynamics is expressed. This presentation is
essentially that of \cite{MeredithR05} with the addition of
polyadicity and summation. For readability we have relegated some of
the technical subtleties to an appendix.

\subsubsection{Process grammar}\label{subsub:process_grammar}

\begin{mathpar}
  \inferrule* [lab=synchronization] {} {{M} \bc \pzero \;|\; x?F \;|\; x!C }
  \and
  \inferrule* [lab=abstraction] {} {{F} \bc (x)P}
  \and
  \inferrule* [lab=concretion] {} {{C} \bc \langle Q \rangle}
  \and
  \inferrule* [lab=process] {} {{P,Q} \bc M \;| \;P|Q \;|\; @{x}}
  \and
  \inferrule* [lab=name] {} {{x} \bc \quotep{P}}
\end{mathpar} 

Note that $\vec{x}$ (resp. $\vec{P}$) denotes a vector of names
(resp. processes) of length $|\vec{x}|$ (resp. $|\vec{P}|$). We adopt
the following useful abbreviations.

\begin{mathpar}
   x?(\vec{y}).P := x.(\vec{y})P \and  x\clift{\vec{P}} := x.\clift{\vec{P}}
   \and x!(y) := \lift{x}{\dropn{y}}
   \and \Pi_{i=0}^{n-1}P_i := P_0 | \ldots | P_{n-1}
\end{mathpar}

\subsubsection{Structural congruence}

\paragraph{Free and bound names and alpha-equivalence.} At the
core of structural equivalence is alpha-equivalence which identifies
process that are the same up to a change of variable. Formally, we
recognize the distinction between free and bound names. The free names
of a process, $\freenames{P}$, may be calculated recursively as
follows:

\begin{mathpar}
\freenames{\pzero} := \emptyset
  \and \\
  \freenames{x?(y).P} := \{ x \} \cup (\freenames{P} \setminus \{ y \})
  \and 
  \freenames{x!\langle P \rangle} := \{ x \} \cup \{ P \} 
  \and \\
  \freenames{P|Q} := \freenames{P} \cup \freenames{Q}
  \and \\
  \freenames{@{x}} := \{ x \}
\end{mathpar}

$\pi$
$\quotep{\pi}$

$\freenames{-} : \pi \to \mathcal{P}(\quotep{\pi})$

\begin{eqnarray*}
  \freenames{\pzero} & := & \emptyset \\
  \freenames{x?(y).P} & := & \{ x \} \cup (\freenames{P} \setminus \{ y \}) \\
  \freenames{x!\langle P \rangle} & := & \{ x \} \cup \{ P \} \\
  \freenames{P|Q} & := & \freenames{P} \cup \freenames{Q} \\
  \freenames{\dropn{x}} & := & \{ x \}
\end{eqnarray*}

The bound names of a process, $\boundnames{P}$, are those names occurring in $P$
that are not free. For example, in $x?(y).0$, the name $x$ is free, while $y$ is bound.

\begin{mathpar}
  \inferrule* [lab=monoidal-laws] {} { P|Q \equiv Q|P \and P|0 \equiv P \and P|(Q|R) \equiv (P|Q)|R }
\end{mathpar}

\begin{mathpar}
  \inferrule* [lab=alpha-equivalence] {} { (x)P \equiv (y)P\{y/x\} \and y \not\in \freenames{P} }
\end{mathpar}

\begin{definition}
Then two processes, $P,Q$, are alpha-equivalent if $P = Q\{\vec{y}/\vec{x}\}$ for
some $\vec{x} \in \boundnames{Q},\vec{y} \in \boundnames{P}$, where $Q\{\vec{y}/\vec{x}\}$
denotes the capture-avoiding substitution of $\vec{y}$ for $\vec{x}$ in $Q$.
\end{definition}

\begin{definition}
  The {\em structural congruence} \cite{SangiorgiWalker} , $\equiv$,
  between processes is the least congruence containing
  alpha-equivalence, satisfying the abelian monoid laws
  (associativity, commutativity and $\pzero$ as identity) for parallel
  composition $|$ and for summation $+$.
\end{definition}

\subsection{Name equivalence}

We take name equivalence, written $\nameeq$, to be the smallest
equivalence relation generated by the following rules.

\begin{mathpar}
\inferrule*[lab=Quote-drop]
{ }
{ \quotep{@{x}} \nameeq x }

\inferrule*[lab=Struct-equiv]
{ P \scong Q }
{ \quotep{P} \nameeq \quotep{Q} }
\end{mathpar}

The astute reader will have noticed that the mutual recursion of names
and processes imposes a mutual recursion on alpha-equivalence and
structural equivalence via name-equivalence. Fortunately, all of this
works out pleasantly and we may calculate in the natural way, free of
concern. The reader interested in the details is referred to the
appendix \ref{appendix:rho_details}.

\subsection{Substitution}

We use $\Proc$ for the set of processes, $\QProc$ for the set of
names, and $\id{\{}\vec{y} / \vec{x} \id{\}}$ to denote partial maps,
$s : \QProc \rightarrow \QProc$. A map, $s$ lifts, uniquely, to a map
on process terms, $\widehat{s} : \Proc \rightarrow \Proc$ by the
following equations.

\begin{mathpar}
  (0) \psubstp{Q}{P} := 0 \\
  (R \juxtap S) \psubstp{Q}{P}
  :=    
  (R)\psubstp{Q}{P} \juxtap (S) \psubstp{Q}{P} \\
  (x?(y).R) \psubstp{Q}{P}    
  :=    
  (x)\substp{Q}{P} (z)\concat( (R \psubstn{z}{y}) \psubstp{Q}{P} ) \\
  (\lift{x}{R}) \psubstp{Q}{P}  
  :=
  \lift{(x)\substp{Q}{P}}{ R \psubstp{Q}{P} } \\
%   (\dropn{x})  \psubstp{Q}{P}       
%   := 
%   \left\{ 
%     \begin{array}{ccc} 
%       \dropn{\quotep{Q}} & & x \nameeq \quotep{P} \\
%       \dropn{x} & & otherwise \\
%     \end{array}
%   \right. 
  (\dropn{x})  \psubstp{Q}{P}       
  := 
  \left\{ 
    \begin{array}{ccc} 
      Q & & x \nameeq \quotep{P} \\
      \dropn{x} & & otherwise \\
    \end{array}
  \right.
\end{mathpar}
 

where

\begin{eqnarray}
  (x)\id{\{} \lpquote Q \rpquote / \lpquote P \rpquote \id{\}}            = 
  \left\{ 
    \begin{array}{ccc}
      \lpquote Q \rpquote & & x \nameeq \lpquote P \rpquote \\
      x & & otherwise \\
    \end{array}
  \right. \nonumber
\end{eqnarray}

and $z$ is chosen distinct from $\quotep{P}$, $\quotep{Q}$, the free
names in $Q$, and all the names in $R$. Our $\alpha$-equivalence will
be built in the standard way from this substitution.

\begin{remark}\label{rem:no_self_referential_names}
  One consequence of these definitions is that $\forall P. \quotep{P}
  \not\in \freenames{P}$.
\end{remark}

\subsection{ Dynamic quote: an example }

Anticipating something of what's to come, consider applying the
substitution, $\widehat{\id{\{}u / z \id{\}}}$, to the following pair
of processes, $\lift{w}{y!(z)}$ and $w[ \lpquote y!(z) \rpquote ]$.

\begin{eqnarray}
	\lift{w}{y!(z)}\widehat{\id{\{}u / z \id{\}}}
		& = &
		\lift{w}{y!(u)} \nonumber\\
	w[ \lpquote y!(z) \rpquote ] \widehat{ \id{\{}u / z \id{\}} }
		& = &
		w[ \lpquote y!(z) \rpquote ] \nonumber
\end{eqnarray}

Because the body of the process between quotes is impervious to
substitution, we get radically different answers. In fact, by
examining the first process in an input context,
e.g. $x?(z).\lift{w}{y!(z)}$, we see that the process under the lift
operator may be shaped by prefixed inputs binding a name inside it. In
this sense, the lift operator will be seen as a way to dynamically
construct processes before reifying them as names.

Finally equipped with these standard features we can present the
dynamics of the calculus.

\subsubsection{Operational semantics} 

Finally, we introduce the computational dynamics. What marks these
algebras as distinct from other more traditionally studied algebraic
structures, e.g. vector spaces or polynomial rings, is the manner in
which dynamics is captured. In traditional structures, dynamics is typically
expressed through morphisms between such structures, as in linear maps
between vector spaces or morphisms between rings. In algebras
associated with the semantics of computation, the dynamics is
expressed as part of the algebraic structure itself, through a
reduction reduction relation typically denoted by $\red$. Below, we
give a recursive presentation of this relation for the calculus used
in the encoding.

$\red \subseteq \pi \times \pi$
$\red : \pi \to \mathcal{P}(\pi)$

\begin{mathpar}
  \inferrule* [lab=Comm] { \textsf{match}( x_{src}, x_{trgt} ) } { x_{trgt}?(y)P \; | \; x_{src}!\langle {Q} \rangle \red P\{\quotep{Q}/y}\} }
  \and \\
  \inferrule* [lab=Par] {{P} \red {P}'} {{{P} | {Q}} \red {{P}' | {Q}}}
  \and
  \inferrule* [lab=Equiv]{{{P} \scong {P}'} \andalso {{P}' \red {Q}'} \andalso {{Q}' \scong {Q}}}{{P} \red {Q}}
\end{mathpar}

\begin{eqnarray*}
  match_{\equiv} (\quotep{P},\quotep{Q}) & := & P \equiv Q \\
  match_{\dagger}(\quotep{P},\quotep{Q}) & := & \forall R. P|Q \red^{*} R => R \red^{*} 0 \\
  match_{K}(\quotep{P},\quotep{Q}) & := & K \mbox{ for some context } K
\end{eqnarray*}

$u?(x)P | u!\langle Q \rangle \red P\{\quotep{Q}/x\}$

%We write $\wred$ for $\red^*$, and $P\red$ if $\exists Q $ such that $ P \red Q$.
We write $P\red$ if $\exists Q $ such that $ P \red Q$ and $P\not\red$, otherwise.

\section{Replication}

As mentioned before, it is known that replication (and hence
recursion) can be implemented in a higher-order process algebra
\cite{SangiorgiWalker}. As our first example of calculation with the
machinery thus far presented we give the construction explicitly in
the {\rhoc}.

\begin{eqnarray}
	D_{x} & := & \prefix{x}{y}{(\binpar{\outputp{x}{y}}{@{y}})} \nonumber\\
	\bangp_{x}{P} & := & \binpar{{x}!\langle{\binpar{D_{x}}{P}}\rangle}{D_{x}} \nonumber
\end{eqnarray}

\begin{eqnarray}
	\bangp_{x}{P} & & \nonumber\\
	=
	& {x}!\langle{(\prefix{x}{y}{(\outputp{x}{y} | @{y})) | P}}\rangle 
	      | \prefix{x}{y}{(\outputp{x}{y} | @{y})} & \nonumber\\
	\red
	& (\outputp{x}{y} | @{y})\substn{\quotep{(\prefix{x}{y}{(@{y} | \outputp{x}{y})) | P}}}{y} & \nonumber\\
	=
	& \outputp{x}{\quotep{(\prefix{x}{y}{(\outputp{x}{y} | @{y})) | P}}}
	  | {(\prefix{x}{y}{(\outputp{x}{y} | @{y})) | P}} & \nonumber\\
	\red
	& \ldots & \nonumber\\
	\red^*
	& P | P | \ldots & \nonumber
\end{eqnarray}

Of course, this encoding, as an implementation, runs away, unfolding
$\bangp{P}$ eagerly. A lazier and more implementable replication
operator, restricted to input-guarded processes, may be obtained as follows.

\begin{eqnarray}
\bangp{\prefix{u}{v}{P}} 
	:= 
	\binpar{\lift{x}{\prefix{u}{v}{(\binpar{D(x)}{P})}}}{D(x)} \nonumber
\end{eqnarray}

\begin{remark}
  Note that the lazier definition still does not deal with summation
  or mixed summation (i.e. sums over input and output). The reader is
  invited to construct definitions of replication that deal with these
  features. 

  Further, the definitions are parameterized in a name, $x$. Can you,
  gentle reader, make a definition that eliminates this parameter and
  guarantees no accidental interaction between the replication
  machinery and the process being replicated -- i.e. no accidental
  sharing of names used by the process to get its work done and the
  name(s) used by the replication to effect copying. This latter
  revision of the definition of replication is crucial to obtaining
  the expected identity $!!P \sim !P$.
\end{remark}

\begin{remark}\label{rem:paradoxical_combinator}
  The reader familiar with the lambda calculus will have noticed the
  similarity between $D$ and the paradoxical combinator.

  [Ed. note: the existence of this seems to suggest we have to be more
  restrictive on the set of processes and names we admit if we are to
  support no-cloning.]
\end{remark}

\subsubsection{Bisimulation}

The computational dynamics gives rise to another kind of equivalence,
the equivalence of computational behavior. As previously mentioned
this is typically captured \emph{via} some form of bisimulation.

% The notion we use in this paper is weak barbed bisimulation
% \cite{milner91polyadicpi}.

The notion we use in this paper is derived from weak barbed
bisimulation \cite{milner91polyadicpi}. 

\begin{definition}
An \emph{observation relation}, $\downarrow_{\mathcal N}$, over a set
of names, $\mathcal N$, is the smallest relation satisfying the rules
below.

\infrule[Out-barb]{y \in {\mathcal N}, \; x \nameeq y}
		  {\outputp{x}{v} \downarrow_{\mathcal N} x}
\infrule[Par-barb]{\mbox{$P\downarrow_{\mathcal N} x$ or $Q\downarrow_{\mathcal N} x$}}
		  {\binpar{P}{Q} \downarrow_{\mathcal N} x}

We write $P \Downarrow_{\mathcal N} x$ if there is $Q$ such that 
$P \wred Q$ and $Q \downarrow_{\mathcal N} x$.
\end{definition}

\begin{definition}
%\label{def.bbisim}
An  ${\mathcal N}$-\emph{barbed bisimulation} over a set of names, ${\mathcal N}$, is a symmetric binary relation 
${\mathcal S}_{\mathcal N}$ between agents such that $P\rel{S}_{\mathcal N}Q$ implies:
\begin{enumerate}
\item If $P \red P'$ then $Q \wred Q'$ and $P'\rel{S}_{\mathcal N} Q'$.
\item If $P\downarrow_{\mathcal N} x$, then $Q\Downarrow_{\mathcal N} x$.
\end{enumerate}
$P$ is ${\mathcal N}$-barbed bisimilar to $Q$, written
$P \wbbisim_{\mathcal N} Q$, if $P \rel{S}_{\mathcal N} Q$ for some ${\mathcal N}$-barbed bisimulation ${\mathcal S}_{\mathcal N}$.
\end{definition}

$\mathcal{R} \subseteq \pi \times \pi$

$P \mathcal{R} Q => \forall P'. P \red P' \Rightarrow \exists Q'. Q \red Q', P' \mathcal{R} Q'$

$P \vdash x \Rightarrow Q \vdash x$

\begin{mathpar}
  \inferrule*[lab=Out-barb]{x \nameeq y}{{y}!\langle{Q}\rangle \vdash x}
  \and
  \inferrule*[lab=Par-barb]{\mbox{$P\vdash x$ or $Q\vdash x$}}{\binpar{P}{Q} \vdash x}
\end{mathpar}

\subsubsection{Contexts}

One of the principle advantages of computational calculi like the
$\pi$-calculus is a well-defined notion of context,
contextual-equivalence and a correlation between
contextual-equivalence and notions of bisimulation. The notion of
context allows the decomposition of a process into (sub-)process and
its syntactic environment, its context. Thus, a context may be
thought of as a process with a ``hole'' (written $\Box$) in it. The
application of a context $M$ to a process $P$, written $M[P]$, is
tantamount to filling the hole in $M$ with $P$. In this paper we do
not need the full weight of this theory, but do make use of the notion
of context in the proof the main theorem. 

\begin{mathpar}
  \inferrule* [lab=summation] {} {{M_{M},M_{N}} \bc \Box \;|\; x.M_{A} \;|\; M_{M}+M_{N}}
  \and
  \inferrule* [lab=agent] {} {{M_{A}} \bc (\vec{x})M_{P} \;| \; \clift{P_0,\ldots,M_{P},\ldots,P_N}}
  \and \\
  \inferrule* [lab=process] {} {{M_{P}} \bc M_{N} \;| \;P|M_{P} }
\end{mathpar} 

\begin{mathpar}
  \inferrule* [lab=sychronization] {} {M_{N} \bc \Box \;|\; x?M_{F} \;|\; x!M_{C}}
  \and
  \inferrule* [lab=abstraction] {} {{M_{F}} \bc (x)M_{P} }
  \and
  \inferrule* [lab=concretion] {} {{M_{C}} \bc \langle M_{P} \rangle }
  \and \\
  \inferrule* [lab=process] {} {{M_{P}} \bc M_{N} \;| \;P|M_{P} }
\end{mathpar}

\begin{definition}[contextual application] Given a context $M$, and
  process $P$, we define the \emph{contextual application}, $M[P] :=
  M\{P/\Box\}$. That is, the contextual application of M to P is the
  substitution of $P$ for $\Box$ in $M$.
\end{definition}

$\meaningof{-} : L \to \mathcal{P}(\pi)$

\begin{mathpar}
  \inferrule* [lab=collection] {} {\meaningof{true} = \pi, \and \meaningof{~E} = \pi \setminus \meaningof{E}, \and \meaningof{E_{1} \& E_{2}} = \meaningof{E_{1}} \cap \meaningof{E_{2}}}
\end{mathpar}

\begin{mathpar}
  \inferrule* [lab=structure] {} {\meaningof{0} = \{ P \in \pi | P \equiv 0 \}, \and \\ \meaningof{E_1 | E_2} = \{ P \in \pi | P \equiv P_{1} | P_{2}, P_{1} \in \meaningof{E_{1}}, P_{2} \in \meaningof{E_2}\} }
\end{mathpar}

\begin{mathpar}
 \inferrule* [lab=behavior] {} {\meaningof{\langle a?b \rangle E} = \{ P \in \pi | P \equiv Q | u?(y)P', \\ \and \\\\ \and \\ \;\;\; u \in \meaningof{a}, \forall z.P'\{z/y\} \in \meaningof{E\{z/b\}}\}, \and \\ \meaningof{a!E} = \{ P \in \pi | P \equiv Q | x!\langle P' \rangle, x \in \meaningof{a} P' \in \meaningof{E}\} }
\end{mathpar}

\begin{mathpar}
 \inferrule* [lab=nominal] {} {\meaningof{\quotep{E}} = \{ \quotep{P} \in \quotep{\pi} | P \in \meaningof{E} \}, \and \meaningof{\quotep{P}} = \{ \quotep{Q} \in \quotep{\pi} | P \equiv Q \} \and \\ \meaningof{@\quotep{E}} = \{ P \in \pi | P \equiv @x, x \in \meaningof{E} \}}
\end{mathpar}

\begin{eqnarray*}
  \\
  \meaningof{-} : TS \to ST
\end{eqnarray*}

\begin{eqnarray*}
  \\
  L : TS \to ST
\end{eqnarray*}

\begin{eqnarray*}
  \\
  P \models E \iff P \in \meaningof{E}
\end{eqnarray*}

\begin{eqnarray*}
  P \approx_{L} Q \iff \forall E \in L. P \models E \iff Q \models E
\end{eqnarray*}

\begin{eqnarray*}
  P \approx_{K} Q
\end{eqnarray*}

\begin{eqnarray*}
  P \approx Q
\end{eqnarray*}

$\approx_{K} = \approx = \approx_{L}$

\subsubsection{Contextual duality}

Note that contexts extend the quotation operation to a family of
operations from processes to names. Given a context, $M$, we can
define a \emph{nominal context}, $\quotep{M}$ by $\quotep{M}[P] :=
\quotep{M[P]}$. To foreshadow what is to come we observe that these
operations enjoy a duality with processes very much like the duality
between vectors and maps from vectors to scalars.

Further, because the calculus is essentially higher-order, we have a
correspondence between contexts and processes. More specifically,
given a name $x$ and a context $M$ we can construct $M^{*}_{x}$ such
that 

\begin{mathpar}
  M^{*}_{x} | \lift{x}{P} \red M[P]
\end{mathpar}

namely,

\begin{mathpar}
  M^{*}_{x} := x?(u).M[\dropn{u}]
\end{mathpar}

The dependence of $M^{*}_{x}$ on a name makes it an abstraction, 

\begin{mathpar}
  M^{*} := (x)x?(u).M[\dropn{u}]
\end{mathpar}

\subsection{Additional notation}

It will sometimes be convenient to denote the process a name
quotes. We already have the notation $x = \quotep{P}$, but it will be
convenient to introduce an alternate notation, $\procn{x}$, when we
want to emphasize the connection to the use of the name. Note that, by
virtue of name equivalence, $\quotep{\procn{x}} \nameeq x$; so, the
notation is consistent with previous definitions.

Further, because names have structure it is possible to effect
substitutions on the basis of that structure. This means we need to
upgrade our notation for substitutions, which we accomplish by
adapting comprehension notation. Thus,

\begin{mathpar}
  P\{ y / x : x \in S \}
\end{mathpar}

is interpreted to mean the process derived from P by replacing (in a
capture-avoiding manner) each occurrence of $x$ in $S$ by $y$. For example,

\begin{mathpar}
  P\{ \quotep{\procn{x}|\procn{x}} / x : x \in \freenames{P} \}
\end{mathpar}

will replace each (occurrence) of a free name $x$ in $P$ by
$\quotep{\procn{x}|\procn{x}}$.

Also, we will avail ourselves of the notation $x^{L}$ and $x^{R}$ to
denote injections of a name into disjoint copies of the name
space. There are numerous ways to accomplish this. One example can be
found in \cite{MeredithR05}. This notation overloads to vectors of
names: $\vec{x}^{\pi} := (x_{i}^{\pi} \; : \; 0 \leq i < |\vec{x}| )$ where $\pi \in \{L,R\}$.

We also use $P^{\Box} := P|\Box$.

In \cite{MeredithR05} an interpretation of the new operator is
given. It turns out that there are several possible interpretations
all enjoying the requisite algebraic properties of the operator (see
\cite{milner91polyadicpi}). We will therefore make liberal use of
$(\nu\; \vec{x})P$.

% subsection the_syntax_and_semantics_of_the_notation_system (end)   

\input{qm2pi.qmops} 

\input{qm2pi.sterngerlach} 

\input{qm2pi.metric} 

% section concurrent_process_calculi (end)

%\input{qm2pi.proofsketch}

% section proof sketch (end)

%\input{qm2pi.slviaknots} 

% section spatial logic via knots (end)

\input{qm2pi.conclusion}

% section conclusion (end)

%\input{qm2pi.dtcodes} 

% section wiring algorithm (end)

\input{qm2pi.ack} 

% section acknowledgments (end)

\newpage


\bibliographystyle{plain}   
\bibliography{../../biblios/main.bib}

\input{qm2pi.rhodetails}

\end{document}

 

\documentclass[12pt]{llncs}
%\documentclass{jktr}

\usepackage[pdftex]{hyperref}                   
\usepackage {listings}
\usepackage {mathpartir}
\usepackage{bcprules}
%\usepackage{listings}
                       
\usepackage{graphicx} 
%\usepackage[margins=2.5cm,nohead,nofoot]{geometry}
%\usepackage{geometry}
\usepackage{amsfonts}
\usepackage{amstext}
\usepackage{latexsym}
\usepackage{amssymb}
\usepackage{color}


%\include{myPreamble}
\include{qm2pi.local} 

%\ifpdf
%\usepackage[pdftex]{graphicx}
%\else
%\usepackage{graphicx}
%\fi

 % \ifpdf
%  \usepackage{pdfsync}
%  \if


%\title{Brief Article}
%\author{David F. Snyder}
%\author{L.G. Meredith}

%\address{Dept. of Math., Texas State University--San Marcos, San Marcos, TX 78666}
       
\pagestyle{empty}


\begin{document}

\lstset{language=[Objective]Caml,frame=shadowbox}

\input{qm2pi.front}

% section front matter (end)

\input{qm2pi.intro} 
 
% section introduction (end)

% \input{qm2pi.knotations} 

% section notation (end)

\input{qm2pi.process.calculi} 

% section concurrent_process_calculi_and_spatial_logics_ (end)
    
%\input{qm2pi.knots2pi} 

%\input{qm2pi.trefoil} 

%\input{qm2pi.mainthm} 

% subsection basic_interpretation (end)

%\input{qm2pi.rho.presentation} 
\subsection{The syntax and semantics of the notation system}\label{sub:the_syntax_and_semantics_of_the_notation_system} % (fold)

We now summarize a technical presentation of the calculus that
embodies our theory of dynamics. The typical presentation of such a
calculus follows the style of giving generators and relations on
them. The grammar, below, describing term constructors, freely
generates the set of processes, $\Proc$. This set is then quotiented
by a relation known as structural congruence and it is over this set
that the notion of dynamics is expressed. This presentation is
essentially that of \cite{MeredithR05} with the addition of
polyadicity and summation. For readability we have relegated some of
the technical subtleties to an appendix.

\subsubsection{Process grammar}\label{subsub:process_grammar}

\begin{mathpar}
  \inferrule* [lab=synchronization] {} {{M} \bc \pzero \;|\; x?F \;|\; x!C }
  \and
  \inferrule* [lab=abstraction] {} {{F} \bc (x)P}
  \and
  \inferrule* [lab=concretion] {} {{C} \bc \langle Q \rangle}
  \and
  \inferrule* [lab=process] {} {{P,Q} \bc M \;| \;P|Q \;|\; @{x}}
  \and
  \inferrule* [lab=name] {} {{x} \bc \quotep{P}}
\end{mathpar} 

Note that $\vec{x}$ (resp. $\vec{P}$) denotes a vector of names
(resp. processes) of length $|\vec{x}|$ (resp. $|\vec{P}|$). We adopt
the following useful abbreviations.

\begin{mathpar}
   x?(\vec{y}).P := x.(\vec{y})P \and  x\clift{\vec{P}} := x.\clift{\vec{P}}
   \and x!(y) := \lift{x}{\dropn{y}}
   \and \Pi_{i=0}^{n-1}P_i := P_0 | \ldots | P_{n-1}
\end{mathpar}

\subsubsection{Structural congruence}

\paragraph{Free and bound names and alpha-equivalence.} At the
core of structural equivalence is alpha-equivalence which identifies
process that are the same up to a change of variable. Formally, we
recognize the distinction between free and bound names. The free names
of a process, $\freenames{P}$, may be calculated recursively as
follows:

\begin{mathpar}
\freenames{\pzero} := \emptyset
  \and \\
  \freenames{x?(y).P} := \{ x \} \cup (\freenames{P} \setminus \{ y \})
  \and 
  \freenames{x!\langle P \rangle} := \{ x \} \cup \{ P \} 
  \and \\
  \freenames{P|Q} := \freenames{P} \cup \freenames{Q}
  \and \\
  \freenames{@{x}} := \{ x \}
\end{mathpar}

$\pi$
$\quotep{\pi}$

$\freenames{-} : \pi \to \mathcal{P}(\quotep{\pi})$

\begin{eqnarray*}
  \freenames{\pzero} & := & \emptyset \\
  \freenames{x?(y).P} & := & \{ x \} \cup (\freenames{P} \setminus \{ y \}) \\
  \freenames{x!\langle P \rangle} & := & \{ x \} \cup \{ P \} \\
  \freenames{P|Q} & := & \freenames{P} \cup \freenames{Q} \\
  \freenames{\dropn{x}} & := & \{ x \}
\end{eqnarray*}

The bound names of a process, $\boundnames{P}$, are those names occurring in $P$
that are not free. For example, in $x?(y).0$, the name $x$ is free, while $y$ is bound.

\begin{mathpar}
  \inferrule* [lab=monoidal-laws] {} { P|Q \equiv Q|P \and P|0 \equiv P \and P|(Q|R) \equiv (P|Q)|R }
\end{mathpar}

\begin{mathpar}
  \inferrule* [lab=alpha-equivalence] {} { (x)P \equiv (y)P\{y/x\} \and y \not\in \freenames{P} }
\end{mathpar}

\begin{definition}
Then two processes, $P,Q$, are alpha-equivalent if $P = Q\{\vec{y}/\vec{x}\}$ for
some $\vec{x} \in \boundnames{Q},\vec{y} \in \boundnames{P}$, where $Q\{\vec{y}/\vec{x}\}$
denotes the capture-avoiding substitution of $\vec{y}$ for $\vec{x}$ in $Q$.
\end{definition}

\begin{definition}
  The {\em structural congruence} \cite{SangiorgiWalker} , $\equiv$,
  between processes is the least congruence containing
  alpha-equivalence, satisfying the abelian monoid laws
  (associativity, commutativity and $\pzero$ as identity) for parallel
  composition $|$ and for summation $+$.
\end{definition}

\subsection{Name equivalence}

We take name equivalence, written $\nameeq$, to be the smallest
equivalence relation generated by the following rules.

\begin{mathpar}
\inferrule*[lab=Quote-drop]
{ }
{ \quotep{@{x}} \nameeq x }

\inferrule*[lab=Struct-equiv]
{ P \scong Q }
{ \quotep{P} \nameeq \quotep{Q} }
\end{mathpar}

The astute reader will have noticed that the mutual recursion of names
and processes imposes a mutual recursion on alpha-equivalence and
structural equivalence via name-equivalence. Fortunately, all of this
works out pleasantly and we may calculate in the natural way, free of
concern. The reader interested in the details is referred to the
appendix \ref{appendix:rho_details}.

\subsection{Substitution}

We use $\Proc$ for the set of processes, $\QProc$ for the set of
names, and $\id{\{}\vec{y} / \vec{x} \id{\}}$ to denote partial maps,
$s : \QProc \rightarrow \QProc$. A map, $s$ lifts, uniquely, to a map
on process terms, $\widehat{s} : \Proc \rightarrow \Proc$ by the
following equations.

\begin{mathpar}
  (0) \psubstp{Q}{P} := 0 \\
  (R \juxtap S) \psubstp{Q}{P}
  :=    
  (R)\psubstp{Q}{P} \juxtap (S) \psubstp{Q}{P} \\
  (x?(y).R) \psubstp{Q}{P}    
  :=    
  (x)\substp{Q}{P} (z)\concat( (R \psubstn{z}{y}) \psubstp{Q}{P} ) \\
  (\lift{x}{R}) \psubstp{Q}{P}  
  :=
  \lift{(x)\substp{Q}{P}}{ R \psubstp{Q}{P} } \\
%   (\dropn{x})  \psubstp{Q}{P}       
%   := 
%   \left\{ 
%     \begin{array}{ccc} 
%       \dropn{\quotep{Q}} & & x \nameeq \quotep{P} \\
%       \dropn{x} & & otherwise \\
%     \end{array}
%   \right. 
  (\dropn{x})  \psubstp{Q}{P}       
  := 
  \left\{ 
    \begin{array}{ccc} 
      Q & & x \nameeq \quotep{P} \\
      \dropn{x} & & otherwise \\
    \end{array}
  \right.
\end{mathpar}
 

where

\begin{eqnarray}
  (x)\id{\{} \lpquote Q \rpquote / \lpquote P \rpquote \id{\}}            = 
  \left\{ 
    \begin{array}{ccc}
      \lpquote Q \rpquote & & x \nameeq \lpquote P \rpquote \\
      x & & otherwise \\
    \end{array}
  \right. \nonumber
\end{eqnarray}

and $z$ is chosen distinct from $\quotep{P}$, $\quotep{Q}$, the free
names in $Q$, and all the names in $R$. Our $\alpha$-equivalence will
be built in the standard way from this substitution.

\begin{remark}\label{rem:no_self_referential_names}
  One consequence of these definitions is that $\forall P. \quotep{P}
  \not\in \freenames{P}$.
\end{remark}

\subsection{ Dynamic quote: an example }

Anticipating something of what's to come, consider applying the
substitution, $\widehat{\id{\{}u / z \id{\}}}$, to the following pair
of processes, $\lift{w}{y!(z)}$ and $w[ \lpquote y!(z) \rpquote ]$.

\begin{eqnarray}
	\lift{w}{y!(z)}\widehat{\id{\{}u / z \id{\}}}
		& = &
		\lift{w}{y!(u)} \nonumber\\
	w[ \lpquote y!(z) \rpquote ] \widehat{ \id{\{}u / z \id{\}} }
		& = &
		w[ \lpquote y!(z) \rpquote ] \nonumber
\end{eqnarray}

Because the body of the process between quotes is impervious to
substitution, we get radically different answers. In fact, by
examining the first process in an input context,
e.g. $x?(z).\lift{w}{y!(z)}$, we see that the process under the lift
operator may be shaped by prefixed inputs binding a name inside it. In
this sense, the lift operator will be seen as a way to dynamically
construct processes before reifying them as names.

Finally equipped with these standard features we can present the
dynamics of the calculus.

\subsubsection{Operational semantics} 

Finally, we introduce the computational dynamics. What marks these
algebras as distinct from other more traditionally studied algebraic
structures, e.g. vector spaces or polynomial rings, is the manner in
which dynamics is captured. In traditional structures, dynamics is typically
expressed through morphisms between such structures, as in linear maps
between vector spaces or morphisms between rings. In algebras
associated with the semantics of computation, the dynamics is
expressed as part of the algebraic structure itself, through a
reduction reduction relation typically denoted by $\red$. Below, we
give a recursive presentation of this relation for the calculus used
in the encoding.

$\red \subseteq \pi \times \pi$
$\red : \pi \to \mathcal{P}(\pi)$

\begin{mathpar}
  \inferrule* [lab=Comm] { \textsf{match}( x_{src}, x_{trgt} ) } { x_{trgt}?(y)P \; | \; x_{src}!\langle {Q} \rangle \red P\{\quotep{Q}/y}\} }
  \and \\
  \inferrule* [lab=Par] {{P} \red {P}'} {{{P} | {Q}} \red {{P}' | {Q}}}
  \and
  \inferrule* [lab=Equiv]{{{P} \scong {P}'} \andalso {{P}' \red {Q}'} \andalso {{Q}' \scong {Q}}}{{P} \red {Q}}
\end{mathpar}

\begin{eqnarray*}
  match_{\equiv} (\quotep{P},\quotep{Q}) & := & P \equiv Q \\
  match_{\dagger}(\quotep{P},\quotep{Q}) & := & \forall R. P|Q \red^{*} R => R \red^{*} 0 \\
  match_{K}(\quotep{P},\quotep{Q}) & := & K \mbox{ for some context } K
\end{eqnarray*}

$u?(x)P | u!\langle Q \rangle \red P\{\quotep{Q}/x\}$

%We write $\wred$ for $\red^*$, and $P\red$ if $\exists Q $ such that $ P \red Q$.
We write $P\red$ if $\exists Q $ such that $ P \red Q$ and $P\not\red$, otherwise.

\section{Replication}

As mentioned before, it is known that replication (and hence
recursion) can be implemented in a higher-order process algebra
\cite{SangiorgiWalker}. As our first example of calculation with the
machinery thus far presented we give the construction explicitly in
the {\rhoc}.

\begin{eqnarray}
	D_{x} & := & \prefix{x}{y}{(\binpar{\outputp{x}{y}}{@{y}})} \nonumber\\
	\bangp_{x}{P} & := & \binpar{{x}!\langle{\binpar{D_{x}}{P}}\rangle}{D_{x}} \nonumber
\end{eqnarray}

\begin{eqnarray}
	\bangp_{x}{P} & & \nonumber\\
	=
	& {x}!\langle{(\prefix{x}{y}{(\outputp{x}{y} | @{y})) | P}}\rangle 
	      | \prefix{x}{y}{(\outputp{x}{y} | @{y})} & \nonumber\\
	\red
	& (\outputp{x}{y} | @{y})\substn{\quotep{(\prefix{x}{y}{(@{y} | \outputp{x}{y})) | P}}}{y} & \nonumber\\
	=
	& \outputp{x}{\quotep{(\prefix{x}{y}{(\outputp{x}{y} | @{y})) | P}}}
	  | {(\prefix{x}{y}{(\outputp{x}{y} | @{y})) | P}} & \nonumber\\
	\red
	& \ldots & \nonumber\\
	\red^*
	& P | P | \ldots & \nonumber
\end{eqnarray}

Of course, this encoding, as an implementation, runs away, unfolding
$\bangp{P}$ eagerly. A lazier and more implementable replication
operator, restricted to input-guarded processes, may be obtained as follows.

\begin{eqnarray}
\bangp{\prefix{u}{v}{P}} 
	:= 
	\binpar{\lift{x}{\prefix{u}{v}{(\binpar{D(x)}{P})}}}{D(x)} \nonumber
\end{eqnarray}

\begin{remark}
  Note that the lazier definition still does not deal with summation
  or mixed summation (i.e. sums over input and output). The reader is
  invited to construct definitions of replication that deal with these
  features. 

  Further, the definitions are parameterized in a name, $x$. Can you,
  gentle reader, make a definition that eliminates this parameter and
  guarantees no accidental interaction between the replication
  machinery and the process being replicated -- i.e. no accidental
  sharing of names used by the process to get its work done and the
  name(s) used by the replication to effect copying. This latter
  revision of the definition of replication is crucial to obtaining
  the expected identity $!!P \sim !P$.
\end{remark}

\begin{remark}\label{rem:paradoxical_combinator}
  The reader familiar with the lambda calculus will have noticed the
  similarity between $D$ and the paradoxical combinator.

  [Ed. note: the existence of this seems to suggest we have to be more
  restrictive on the set of processes and names we admit if we are to
  support no-cloning.]
\end{remark}

\subsubsection{Bisimulation}

The computational dynamics gives rise to another kind of equivalence,
the equivalence of computational behavior. As previously mentioned
this is typically captured \emph{via} some form of bisimulation.

% The notion we use in this paper is weak barbed bisimulation
% \cite{milner91polyadicpi}.

The notion we use in this paper is derived from weak barbed
bisimulation \cite{milner91polyadicpi}. 

\begin{definition}
An \emph{observation relation}, $\downarrow_{\mathcal N}$, over a set
of names, $\mathcal N$, is the smallest relation satisfying the rules
below.

\infrule[Out-barb]{y \in {\mathcal N}, \; x \nameeq y}
		  {\outputp{x}{v} \downarrow_{\mathcal N} x}
\infrule[Par-barb]{\mbox{$P\downarrow_{\mathcal N} x$ or $Q\downarrow_{\mathcal N} x$}}
		  {\binpar{P}{Q} \downarrow_{\mathcal N} x}

We write $P \Downarrow_{\mathcal N} x$ if there is $Q$ such that 
$P \wred Q$ and $Q \downarrow_{\mathcal N} x$.
\end{definition}

\begin{definition}
%\label{def.bbisim}
An  ${\mathcal N}$-\emph{barbed bisimulation} over a set of names, ${\mathcal N}$, is a symmetric binary relation 
${\mathcal S}_{\mathcal N}$ between agents such that $P\rel{S}_{\mathcal N}Q$ implies:
\begin{enumerate}
\item If $P \red P'$ then $Q \wred Q'$ and $P'\rel{S}_{\mathcal N} Q'$.
\item If $P\downarrow_{\mathcal N} x$, then $Q\Downarrow_{\mathcal N} x$.
\end{enumerate}
$P$ is ${\mathcal N}$-barbed bisimilar to $Q$, written
$P \wbbisim_{\mathcal N} Q$, if $P \rel{S}_{\mathcal N} Q$ for some ${\mathcal N}$-barbed bisimulation ${\mathcal S}_{\mathcal N}$.
\end{definition}

$\mathcal{R} \subseteq \pi \times \pi$

$P \mathcal{R} Q => \forall P'. P \red P' \Rightarrow \exists Q'. Q \red Q', P' \mathcal{R} Q'$

$P \vdash x \Rightarrow Q \vdash x$

\begin{mathpar}
  \inferrule*[lab=Out-barb]{x \nameeq y}{{y}!\langle{Q}\rangle \vdash x}
  \and
  \inferrule*[lab=Par-barb]{\mbox{$P\vdash x$ or $Q\vdash x$}}{\binpar{P}{Q} \vdash x}
\end{mathpar}

\subsubsection{Contexts}

One of the principle advantages of computational calculi like the
$\pi$-calculus is a well-defined notion of context,
contextual-equivalence and a correlation between
contextual-equivalence and notions of bisimulation. The notion of
context allows the decomposition of a process into (sub-)process and
its syntactic environment, its context. Thus, a context may be
thought of as a process with a ``hole'' (written $\Box$) in it. The
application of a context $M$ to a process $P$, written $M[P]$, is
tantamount to filling the hole in $M$ with $P$. In this paper we do
not need the full weight of this theory, but do make use of the notion
of context in the proof the main theorem. 

\begin{mathpar}
  \inferrule* [lab=summation] {} {{M_{M},M_{N}} \bc \Box \;|\; x.M_{A} \;|\; M_{M}+M_{N}}
  \and
  \inferrule* [lab=agent] {} {{M_{A}} \bc (\vec{x})M_{P} \;| \; \clift{P_0,\ldots,M_{P},\ldots,P_N}}
  \and \\
  \inferrule* [lab=process] {} {{M_{P}} \bc M_{N} \;| \;P|M_{P} }
\end{mathpar} 

\begin{mathpar}
  \inferrule* [lab=sychronization] {} {M_{N} \bc \Box \;|\; x?M_{F} \;|\; x!M_{C}}
  \and
  \inferrule* [lab=abstraction] {} {{M_{F}} \bc (x)M_{P} }
  \and
  \inferrule* [lab=concretion] {} {{M_{C}} \bc \langle M_{P} \rangle }
  \and \\
  \inferrule* [lab=process] {} {{M_{P}} \bc M_{N} \;| \;P|M_{P} }
\end{mathpar}

\begin{definition}[contextual application] Given a context $M$, and
  process $P$, we define the \emph{contextual application}, $M[P] :=
  M\{P/\Box\}$. That is, the contextual application of M to P is the
  substitution of $P$ for $\Box$ in $M$.
\end{definition}

$\meaningof{-} : L \to \mathcal{P}(\pi)$

\begin{mathpar}
  \inferrule* [lab=collection] {} {\meaningof{true} = \pi, \and \meaningof{~E} = \pi \setminus \meaningof{E}, \and \meaningof{E_{1} \& E_{2}} = \meaningof{E_{1}} \cap \meaningof{E_{2}}}
\end{mathpar}

\begin{mathpar}
  \inferrule* [lab=structure] {} {\meaningof{0} = \{ P \in \pi | P \equiv 0 \}, \and \\ \meaningof{E_1 | E_2} = \{ P \in \pi | P \equiv P_{1} | P_{2}, P_{1} \in \meaningof{E_{1}}, P_{2} \in \meaningof{E_2}\} }
\end{mathpar}

\begin{mathpar}
 \inferrule* [lab=behavior] {} {\meaningof{\langle a?b \rangle E} = \{ P \in \pi | P \equiv Q | u?(y)P', \\ \and \\\\ \and \\ \;\;\; u \in \meaningof{a}, \forall z.P'\{z/y\} \in \meaningof{E\{z/b\}}\}, \and \\ \meaningof{a!E} = \{ P \in \pi | P \equiv Q | x!\langle P' \rangle, x \in \meaningof{a} P' \in \meaningof{E}\} }
\end{mathpar}

\begin{mathpar}
 \inferrule* [lab=nominal] {} {\meaningof{\quotep{E}} = \{ \quotep{P} \in \quotep{\pi} | P \in \meaningof{E} \}, \and \meaningof{\quotep{P}} = \{ \quotep{Q} \in \quotep{\pi} | P \equiv Q \} \and \\ \meaningof{@\quotep{E}} = \{ P \in \pi | P \equiv @x, x \in \meaningof{E} \}}
\end{mathpar}

\begin{eqnarray*}
  \\
  \meaningof{-} : TS \to ST
\end{eqnarray*}

\begin{eqnarray*}
  \\
  L : TS \to ST
\end{eqnarray*}

\begin{eqnarray*}
  \\
  P \models E \iff P \in \meaningof{E}
\end{eqnarray*}

\begin{eqnarray*}
  P \approx_{L} Q \iff \forall E \in L. P \models E \iff Q \models E
\end{eqnarray*}

\begin{eqnarray*}
  P \approx_{K} Q
\end{eqnarray*}

\begin{eqnarray*}
  P \approx Q
\end{eqnarray*}

$\approx_{K} = \approx = \approx_{L}$

\subsubsection{Contextual duality}

Note that contexts extend the quotation operation to a family of
operations from processes to names. Given a context, $M$, we can
define a \emph{nominal context}, $\quotep{M}$ by $\quotep{M}[P] :=
\quotep{M[P]}$. To foreshadow what is to come we observe that these
operations enjoy a duality with processes very much like the duality
between vectors and maps from vectors to scalars.

Further, because the calculus is essentially higher-order, we have a
correspondence between contexts and processes. More specifically,
given a name $x$ and a context $M$ we can construct $M^{*}_{x}$ such
that 

\begin{mathpar}
  M^{*}_{x} | \lift{x}{P} \red M[P]
\end{mathpar}

namely,

\begin{mathpar}
  M^{*}_{x} := x?(u).M[\dropn{u}]
\end{mathpar}

The dependence of $M^{*}_{x}$ on a name makes it an abstraction, 

\begin{mathpar}
  M^{*} := (x)x?(u).M[\dropn{u}]
\end{mathpar}

\subsection{Additional notation}

It will sometimes be convenient to denote the process a name
quotes. We already have the notation $x = \quotep{P}$, but it will be
convenient to introduce an alternate notation, $\procn{x}$, when we
want to emphasize the connection to the use of the name. Note that, by
virtue of name equivalence, $\quotep{\procn{x}} \nameeq x$; so, the
notation is consistent with previous definitions.

Further, because names have structure it is possible to effect
substitutions on the basis of that structure. This means we need to
upgrade our notation for substitutions, which we accomplish by
adapting comprehension notation. Thus,

\begin{mathpar}
  P\{ y / x : x \in S \}
\end{mathpar}

is interpreted to mean the process derived from P by replacing (in a
capture-avoiding manner) each occurrence of $x$ in $S$ by $y$. For example,

\begin{mathpar}
  P\{ \quotep{\procn{x}|\procn{x}} / x : x \in \freenames{P} \}
\end{mathpar}

will replace each (occurrence) of a free name $x$ in $P$ by
$\quotep{\procn{x}|\procn{x}}$.

Also, we will avail ourselves of the notation $x^{L}$ and $x^{R}$ to
denote injections of a name into disjoint copies of the name
space. There are numerous ways to accomplish this. One example can be
found in \cite{MeredithR05}. This notation overloads to vectors of
names: $\vec{x}^{\pi} := (x_{i}^{\pi} \; : \; 0 \leq i < |\vec{x}| )$ where $\pi \in \{L,R\}$.

We also use $P^{\Box} := P|\Box$.

In \cite{MeredithR05} an interpretation of the new operator is
given. It turns out that there are several possible interpretations
all enjoying the requisite algebraic properties of the operator (see
\cite{milner91polyadicpi}). We will therefore make liberal use of
$(\nu\; \vec{x})P$.

% subsection the_syntax_and_semantics_of_the_notation_system (end)   

\input{qm2pi.qmops} 

\input{qm2pi.sterngerlach} 

\input{qm2pi.metric} 

% section concurrent_process_calculi (end)

%\input{qm2pi.proofsketch}

% section proof sketch (end)

%\input{qm2pi.slviaknots} 

% section spatial logic via knots (end)

\input{qm2pi.conclusion}

% section conclusion (end)

%\input{qm2pi.dtcodes} 

% section wiring algorithm (end)

\input{qm2pi.ack} 

% section acknowledgments (end)

\newpage


\bibliographystyle{plain}   
\bibliography{../../biblios/main.bib}

\input{qm2pi.rhodetails}

\end{document}

 

% section concurrent_process_calculi (end)

%\documentclass[12pt]{llncs}
%\documentclass{jktr}

\usepackage[pdftex]{hyperref}                   
\usepackage {listings}
\usepackage {mathpartir}
\usepackage{bcprules}
%\usepackage{listings}
                       
\usepackage{graphicx} 
%\usepackage[margins=2.5cm,nohead,nofoot]{geometry}
%\usepackage{geometry}
\usepackage{amsfonts}
\usepackage{amstext}
\usepackage{latexsym}
\usepackage{amssymb}
\usepackage{color}


%\include{myPreamble}
\include{qm2pi.local} 

%\ifpdf
%\usepackage[pdftex]{graphicx}
%\else
%\usepackage{graphicx}
%\fi

 % \ifpdf
%  \usepackage{pdfsync}
%  \if


%\title{Brief Article}
%\author{David F. Snyder}
%\author{L.G. Meredith}

%\address{Dept. of Math., Texas State University--San Marcos, San Marcos, TX 78666}
       
\pagestyle{empty}


\begin{document}

\lstset{language=[Objective]Caml,frame=shadowbox}

\input{qm2pi.front}

% section front matter (end)

\input{qm2pi.intro} 
 
% section introduction (end)

% \input{qm2pi.knotations} 

% section notation (end)

\input{qm2pi.process.calculi} 

% section concurrent_process_calculi_and_spatial_logics_ (end)
    
%\input{qm2pi.knots2pi} 

%\input{qm2pi.trefoil} 

%\input{qm2pi.mainthm} 

% subsection basic_interpretation (end)

%\input{qm2pi.rho.presentation} 
\subsection{The syntax and semantics of the notation system}\label{sub:the_syntax_and_semantics_of_the_notation_system} % (fold)

We now summarize a technical presentation of the calculus that
embodies our theory of dynamics. The typical presentation of such a
calculus follows the style of giving generators and relations on
them. The grammar, below, describing term constructors, freely
generates the set of processes, $\Proc$. This set is then quotiented
by a relation known as structural congruence and it is over this set
that the notion of dynamics is expressed. This presentation is
essentially that of \cite{MeredithR05} with the addition of
polyadicity and summation. For readability we have relegated some of
the technical subtleties to an appendix.

\subsubsection{Process grammar}\label{subsub:process_grammar}

\begin{mathpar}
  \inferrule* [lab=synchronization] {} {{M} \bc \pzero \;|\; x?F \;|\; x!C }
  \and
  \inferrule* [lab=abstraction] {} {{F} \bc (x)P}
  \and
  \inferrule* [lab=concretion] {} {{C} \bc \langle Q \rangle}
  \and
  \inferrule* [lab=process] {} {{P,Q} \bc M \;| \;P|Q \;|\; @{x}}
  \and
  \inferrule* [lab=name] {} {{x} \bc \quotep{P}}
\end{mathpar} 

Note that $\vec{x}$ (resp. $\vec{P}$) denotes a vector of names
(resp. processes) of length $|\vec{x}|$ (resp. $|\vec{P}|$). We adopt
the following useful abbreviations.

\begin{mathpar}
   x?(\vec{y}).P := x.(\vec{y})P \and  x\clift{\vec{P}} := x.\clift{\vec{P}}
   \and x!(y) := \lift{x}{\dropn{y}}
   \and \Pi_{i=0}^{n-1}P_i := P_0 | \ldots | P_{n-1}
\end{mathpar}

\subsubsection{Structural congruence}

\paragraph{Free and bound names and alpha-equivalence.} At the
core of structural equivalence is alpha-equivalence which identifies
process that are the same up to a change of variable. Formally, we
recognize the distinction between free and bound names. The free names
of a process, $\freenames{P}$, may be calculated recursively as
follows:

\begin{mathpar}
\freenames{\pzero} := \emptyset
  \and \\
  \freenames{x?(y).P} := \{ x \} \cup (\freenames{P} \setminus \{ y \})
  \and 
  \freenames{x!\langle P \rangle} := \{ x \} \cup \{ P \} 
  \and \\
  \freenames{P|Q} := \freenames{P} \cup \freenames{Q}
  \and \\
  \freenames{@{x}} := \{ x \}
\end{mathpar}

$\pi$
$\quotep{\pi}$

$\freenames{-} : \pi \to \mathcal{P}(\quotep{\pi})$

\begin{eqnarray*}
  \freenames{\pzero} & := & \emptyset \\
  \freenames{x?(y).P} & := & \{ x \} \cup (\freenames{P} \setminus \{ y \}) \\
  \freenames{x!\langle P \rangle} & := & \{ x \} \cup \{ P \} \\
  \freenames{P|Q} & := & \freenames{P} \cup \freenames{Q} \\
  \freenames{\dropn{x}} & := & \{ x \}
\end{eqnarray*}

The bound names of a process, $\boundnames{P}$, are those names occurring in $P$
that are not free. For example, in $x?(y).0$, the name $x$ is free, while $y$ is bound.

\begin{mathpar}
  \inferrule* [lab=monoidal-laws] {} { P|Q \equiv Q|P \and P|0 \equiv P \and P|(Q|R) \equiv (P|Q)|R }
\end{mathpar}

\begin{mathpar}
  \inferrule* [lab=alpha-equivalence] {} { (x)P \equiv (y)P\{y/x\} \and y \not\in \freenames{P} }
\end{mathpar}

\begin{definition}
Then two processes, $P,Q$, are alpha-equivalent if $P = Q\{\vec{y}/\vec{x}\}$ for
some $\vec{x} \in \boundnames{Q},\vec{y} \in \boundnames{P}$, where $Q\{\vec{y}/\vec{x}\}$
denotes the capture-avoiding substitution of $\vec{y}$ for $\vec{x}$ in $Q$.
\end{definition}

\begin{definition}
  The {\em structural congruence} \cite{SangiorgiWalker} , $\equiv$,
  between processes is the least congruence containing
  alpha-equivalence, satisfying the abelian monoid laws
  (associativity, commutativity and $\pzero$ as identity) for parallel
  composition $|$ and for summation $+$.
\end{definition}

\subsection{Name equivalence}

We take name equivalence, written $\nameeq$, to be the smallest
equivalence relation generated by the following rules.

\begin{mathpar}
\inferrule*[lab=Quote-drop]
{ }
{ \quotep{@{x}} \nameeq x }

\inferrule*[lab=Struct-equiv]
{ P \scong Q }
{ \quotep{P} \nameeq \quotep{Q} }
\end{mathpar}

The astute reader will have noticed that the mutual recursion of names
and processes imposes a mutual recursion on alpha-equivalence and
structural equivalence via name-equivalence. Fortunately, all of this
works out pleasantly and we may calculate in the natural way, free of
concern. The reader interested in the details is referred to the
appendix \ref{appendix:rho_details}.

\subsection{Substitution}

We use $\Proc$ for the set of processes, $\QProc$ for the set of
names, and $\id{\{}\vec{y} / \vec{x} \id{\}}$ to denote partial maps,
$s : \QProc \rightarrow \QProc$. A map, $s$ lifts, uniquely, to a map
on process terms, $\widehat{s} : \Proc \rightarrow \Proc$ by the
following equations.

\begin{mathpar}
  (0) \psubstp{Q}{P} := 0 \\
  (R \juxtap S) \psubstp{Q}{P}
  :=    
  (R)\psubstp{Q}{P} \juxtap (S) \psubstp{Q}{P} \\
  (x?(y).R) \psubstp{Q}{P}    
  :=    
  (x)\substp{Q}{P} (z)\concat( (R \psubstn{z}{y}) \psubstp{Q}{P} ) \\
  (\lift{x}{R}) \psubstp{Q}{P}  
  :=
  \lift{(x)\substp{Q}{P}}{ R \psubstp{Q}{P} } \\
%   (\dropn{x})  \psubstp{Q}{P}       
%   := 
%   \left\{ 
%     \begin{array}{ccc} 
%       \dropn{\quotep{Q}} & & x \nameeq \quotep{P} \\
%       \dropn{x} & & otherwise \\
%     \end{array}
%   \right. 
  (\dropn{x})  \psubstp{Q}{P}       
  := 
  \left\{ 
    \begin{array}{ccc} 
      Q & & x \nameeq \quotep{P} \\
      \dropn{x} & & otherwise \\
    \end{array}
  \right.
\end{mathpar}
 

where

\begin{eqnarray}
  (x)\id{\{} \lpquote Q \rpquote / \lpquote P \rpquote \id{\}}            = 
  \left\{ 
    \begin{array}{ccc}
      \lpquote Q \rpquote & & x \nameeq \lpquote P \rpquote \\
      x & & otherwise \\
    \end{array}
  \right. \nonumber
\end{eqnarray}

and $z$ is chosen distinct from $\quotep{P}$, $\quotep{Q}$, the free
names in $Q$, and all the names in $R$. Our $\alpha$-equivalence will
be built in the standard way from this substitution.

\begin{remark}\label{rem:no_self_referential_names}
  One consequence of these definitions is that $\forall P. \quotep{P}
  \not\in \freenames{P}$.
\end{remark}

\subsection{ Dynamic quote: an example }

Anticipating something of what's to come, consider applying the
substitution, $\widehat{\id{\{}u / z \id{\}}}$, to the following pair
of processes, $\lift{w}{y!(z)}$ and $w[ \lpquote y!(z) \rpquote ]$.

\begin{eqnarray}
	\lift{w}{y!(z)}\widehat{\id{\{}u / z \id{\}}}
		& = &
		\lift{w}{y!(u)} \nonumber\\
	w[ \lpquote y!(z) \rpquote ] \widehat{ \id{\{}u / z \id{\}} }
		& = &
		w[ \lpquote y!(z) \rpquote ] \nonumber
\end{eqnarray}

Because the body of the process between quotes is impervious to
substitution, we get radically different answers. In fact, by
examining the first process in an input context,
e.g. $x?(z).\lift{w}{y!(z)}$, we see that the process under the lift
operator may be shaped by prefixed inputs binding a name inside it. In
this sense, the lift operator will be seen as a way to dynamically
construct processes before reifying them as names.

Finally equipped with these standard features we can present the
dynamics of the calculus.

\subsubsection{Operational semantics} 

Finally, we introduce the computational dynamics. What marks these
algebras as distinct from other more traditionally studied algebraic
structures, e.g. vector spaces or polynomial rings, is the manner in
which dynamics is captured. In traditional structures, dynamics is typically
expressed through morphisms between such structures, as in linear maps
between vector spaces or morphisms between rings. In algebras
associated with the semantics of computation, the dynamics is
expressed as part of the algebraic structure itself, through a
reduction reduction relation typically denoted by $\red$. Below, we
give a recursive presentation of this relation for the calculus used
in the encoding.

$\red \subseteq \pi \times \pi$
$\red : \pi \to \mathcal{P}(\pi)$

\begin{mathpar}
  \inferrule* [lab=Comm] { \textsf{match}( x_{src}, x_{trgt} ) } { x_{trgt}?(y)P \; | \; x_{src}!\langle {Q} \rangle \red P\{\quotep{Q}/y}\} }
  \and \\
  \inferrule* [lab=Par] {{P} \red {P}'} {{{P} | {Q}} \red {{P}' | {Q}}}
  \and
  \inferrule* [lab=Equiv]{{{P} \scong {P}'} \andalso {{P}' \red {Q}'} \andalso {{Q}' \scong {Q}}}{{P} \red {Q}}
\end{mathpar}

\begin{eqnarray*}
  match_{\equiv} (\quotep{P},\quotep{Q}) & := & P \equiv Q \\
  match_{\dagger}(\quotep{P},\quotep{Q}) & := & \forall R. P|Q \red^{*} R => R \red^{*} 0 \\
  match_{K}(\quotep{P},\quotep{Q}) & := & K \mbox{ for some context } K
\end{eqnarray*}

$u?(x)P | u!\langle Q \rangle \red P\{\quotep{Q}/x\}$

%We write $\wred$ for $\red^*$, and $P\red$ if $\exists Q $ such that $ P \red Q$.
We write $P\red$ if $\exists Q $ such that $ P \red Q$ and $P\not\red$, otherwise.

\section{Replication}

As mentioned before, it is known that replication (and hence
recursion) can be implemented in a higher-order process algebra
\cite{SangiorgiWalker}. As our first example of calculation with the
machinery thus far presented we give the construction explicitly in
the {\rhoc}.

\begin{eqnarray}
	D_{x} & := & \prefix{x}{y}{(\binpar{\outputp{x}{y}}{@{y}})} \nonumber\\
	\bangp_{x}{P} & := & \binpar{{x}!\langle{\binpar{D_{x}}{P}}\rangle}{D_{x}} \nonumber
\end{eqnarray}

\begin{eqnarray}
	\bangp_{x}{P} & & \nonumber\\
	=
	& {x}!\langle{(\prefix{x}{y}{(\outputp{x}{y} | @{y})) | P}}\rangle 
	      | \prefix{x}{y}{(\outputp{x}{y} | @{y})} & \nonumber\\
	\red
	& (\outputp{x}{y} | @{y})\substn{\quotep{(\prefix{x}{y}{(@{y} | \outputp{x}{y})) | P}}}{y} & \nonumber\\
	=
	& \outputp{x}{\quotep{(\prefix{x}{y}{(\outputp{x}{y} | @{y})) | P}}}
	  | {(\prefix{x}{y}{(\outputp{x}{y} | @{y})) | P}} & \nonumber\\
	\red
	& \ldots & \nonumber\\
	\red^*
	& P | P | \ldots & \nonumber
\end{eqnarray}

Of course, this encoding, as an implementation, runs away, unfolding
$\bangp{P}$ eagerly. A lazier and more implementable replication
operator, restricted to input-guarded processes, may be obtained as follows.

\begin{eqnarray}
\bangp{\prefix{u}{v}{P}} 
	:= 
	\binpar{\lift{x}{\prefix{u}{v}{(\binpar{D(x)}{P})}}}{D(x)} \nonumber
\end{eqnarray}

\begin{remark}
  Note that the lazier definition still does not deal with summation
  or mixed summation (i.e. sums over input and output). The reader is
  invited to construct definitions of replication that deal with these
  features. 

  Further, the definitions are parameterized in a name, $x$. Can you,
  gentle reader, make a definition that eliminates this parameter and
  guarantees no accidental interaction between the replication
  machinery and the process being replicated -- i.e. no accidental
  sharing of names used by the process to get its work done and the
  name(s) used by the replication to effect copying. This latter
  revision of the definition of replication is crucial to obtaining
  the expected identity $!!P \sim !P$.
\end{remark}

\begin{remark}\label{rem:paradoxical_combinator}
  The reader familiar with the lambda calculus will have noticed the
  similarity between $D$ and the paradoxical combinator.

  [Ed. note: the existence of this seems to suggest we have to be more
  restrictive on the set of processes and names we admit if we are to
  support no-cloning.]
\end{remark}

\subsubsection{Bisimulation}

The computational dynamics gives rise to another kind of equivalence,
the equivalence of computational behavior. As previously mentioned
this is typically captured \emph{via} some form of bisimulation.

% The notion we use in this paper is weak barbed bisimulation
% \cite{milner91polyadicpi}.

The notion we use in this paper is derived from weak barbed
bisimulation \cite{milner91polyadicpi}. 

\begin{definition}
An \emph{observation relation}, $\downarrow_{\mathcal N}$, over a set
of names, $\mathcal N$, is the smallest relation satisfying the rules
below.

\infrule[Out-barb]{y \in {\mathcal N}, \; x \nameeq y}
		  {\outputp{x}{v} \downarrow_{\mathcal N} x}
\infrule[Par-barb]{\mbox{$P\downarrow_{\mathcal N} x$ or $Q\downarrow_{\mathcal N} x$}}
		  {\binpar{P}{Q} \downarrow_{\mathcal N} x}

We write $P \Downarrow_{\mathcal N} x$ if there is $Q$ such that 
$P \wred Q$ and $Q \downarrow_{\mathcal N} x$.
\end{definition}

\begin{definition}
%\label{def.bbisim}
An  ${\mathcal N}$-\emph{barbed bisimulation} over a set of names, ${\mathcal N}$, is a symmetric binary relation 
${\mathcal S}_{\mathcal N}$ between agents such that $P\rel{S}_{\mathcal N}Q$ implies:
\begin{enumerate}
\item If $P \red P'$ then $Q \wred Q'$ and $P'\rel{S}_{\mathcal N} Q'$.
\item If $P\downarrow_{\mathcal N} x$, then $Q\Downarrow_{\mathcal N} x$.
\end{enumerate}
$P$ is ${\mathcal N}$-barbed bisimilar to $Q$, written
$P \wbbisim_{\mathcal N} Q$, if $P \rel{S}_{\mathcal N} Q$ for some ${\mathcal N}$-barbed bisimulation ${\mathcal S}_{\mathcal N}$.
\end{definition}

$\mathcal{R} \subseteq \pi \times \pi$

$P \mathcal{R} Q => \forall P'. P \red P' \Rightarrow \exists Q'. Q \red Q', P' \mathcal{R} Q'$

$P \vdash x \Rightarrow Q \vdash x$

\begin{mathpar}
  \inferrule*[lab=Out-barb]{x \nameeq y}{{y}!\langle{Q}\rangle \vdash x}
  \and
  \inferrule*[lab=Par-barb]{\mbox{$P\vdash x$ or $Q\vdash x$}}{\binpar{P}{Q} \vdash x}
\end{mathpar}

\subsubsection{Contexts}

One of the principle advantages of computational calculi like the
$\pi$-calculus is a well-defined notion of context,
contextual-equivalence and a correlation between
contextual-equivalence and notions of bisimulation. The notion of
context allows the decomposition of a process into (sub-)process and
its syntactic environment, its context. Thus, a context may be
thought of as a process with a ``hole'' (written $\Box$) in it. The
application of a context $M$ to a process $P$, written $M[P]$, is
tantamount to filling the hole in $M$ with $P$. In this paper we do
not need the full weight of this theory, but do make use of the notion
of context in the proof the main theorem. 

\begin{mathpar}
  \inferrule* [lab=summation] {} {{M_{M},M_{N}} \bc \Box \;|\; x.M_{A} \;|\; M_{M}+M_{N}}
  \and
  \inferrule* [lab=agent] {} {{M_{A}} \bc (\vec{x})M_{P} \;| \; \clift{P_0,\ldots,M_{P},\ldots,P_N}}
  \and \\
  \inferrule* [lab=process] {} {{M_{P}} \bc M_{N} \;| \;P|M_{P} }
\end{mathpar} 

\begin{mathpar}
  \inferrule* [lab=sychronization] {} {M_{N} \bc \Box \;|\; x?M_{F} \;|\; x!M_{C}}
  \and
  \inferrule* [lab=abstraction] {} {{M_{F}} \bc (x)M_{P} }
  \and
  \inferrule* [lab=concretion] {} {{M_{C}} \bc \langle M_{P} \rangle }
  \and \\
  \inferrule* [lab=process] {} {{M_{P}} \bc M_{N} \;| \;P|M_{P} }
\end{mathpar}

\begin{definition}[contextual application] Given a context $M$, and
  process $P$, we define the \emph{contextual application}, $M[P] :=
  M\{P/\Box\}$. That is, the contextual application of M to P is the
  substitution of $P$ for $\Box$ in $M$.
\end{definition}

$\meaningof{-} : L \to \mathcal{P}(\pi)$

\begin{mathpar}
  \inferrule* [lab=collection] {} {\meaningof{true} = \pi, \and \meaningof{~E} = \pi \setminus \meaningof{E}, \and \meaningof{E_{1} \& E_{2}} = \meaningof{E_{1}} \cap \meaningof{E_{2}}}
\end{mathpar}

\begin{mathpar}
  \inferrule* [lab=structure] {} {\meaningof{0} = \{ P \in \pi | P \equiv 0 \}, \and \\ \meaningof{E_1 | E_2} = \{ P \in \pi | P \equiv P_{1} | P_{2}, P_{1} \in \meaningof{E_{1}}, P_{2} \in \meaningof{E_2}\} }
\end{mathpar}

\begin{mathpar}
 \inferrule* [lab=behavior] {} {\meaningof{\langle a?b \rangle E} = \{ P \in \pi | P \equiv Q | u?(y)P', \\ \and \\\\ \and \\ \;\;\; u \in \meaningof{a}, \forall z.P'\{z/y\} \in \meaningof{E\{z/b\}}\}, \and \\ \meaningof{a!E} = \{ P \in \pi | P \equiv Q | x!\langle P' \rangle, x \in \meaningof{a} P' \in \meaningof{E}\} }
\end{mathpar}

\begin{mathpar}
 \inferrule* [lab=nominal] {} {\meaningof{\quotep{E}} = \{ \quotep{P} \in \quotep{\pi} | P \in \meaningof{E} \}, \and \meaningof{\quotep{P}} = \{ \quotep{Q} \in \quotep{\pi} | P \equiv Q \} \and \\ \meaningof{@\quotep{E}} = \{ P \in \pi | P \equiv @x, x \in \meaningof{E} \}}
\end{mathpar}

\begin{eqnarray*}
  \\
  \meaningof{-} : TS \to ST
\end{eqnarray*}

\begin{eqnarray*}
  \\
  L : TS \to ST
\end{eqnarray*}

\begin{eqnarray*}
  \\
  P \models E \iff P \in \meaningof{E}
\end{eqnarray*}

\begin{eqnarray*}
  P \approx_{L} Q \iff \forall E \in L. P \models E \iff Q \models E
\end{eqnarray*}

\begin{eqnarray*}
  P \approx_{K} Q
\end{eqnarray*}

\begin{eqnarray*}
  P \approx Q
\end{eqnarray*}

$\approx_{K} = \approx = \approx_{L}$

\subsubsection{Contextual duality}

Note that contexts extend the quotation operation to a family of
operations from processes to names. Given a context, $M$, we can
define a \emph{nominal context}, $\quotep{M}$ by $\quotep{M}[P] :=
\quotep{M[P]}$. To foreshadow what is to come we observe that these
operations enjoy a duality with processes very much like the duality
between vectors and maps from vectors to scalars.

Further, because the calculus is essentially higher-order, we have a
correspondence between contexts and processes. More specifically,
given a name $x$ and a context $M$ we can construct $M^{*}_{x}$ such
that 

\begin{mathpar}
  M^{*}_{x} | \lift{x}{P} \red M[P]
\end{mathpar}

namely,

\begin{mathpar}
  M^{*}_{x} := x?(u).M[\dropn{u}]
\end{mathpar}

The dependence of $M^{*}_{x}$ on a name makes it an abstraction, 

\begin{mathpar}
  M^{*} := (x)x?(u).M[\dropn{u}]
\end{mathpar}

\subsection{Additional notation}

It will sometimes be convenient to denote the process a name
quotes. We already have the notation $x = \quotep{P}$, but it will be
convenient to introduce an alternate notation, $\procn{x}$, when we
want to emphasize the connection to the use of the name. Note that, by
virtue of name equivalence, $\quotep{\procn{x}} \nameeq x$; so, the
notation is consistent with previous definitions.

Further, because names have structure it is possible to effect
substitutions on the basis of that structure. This means we need to
upgrade our notation for substitutions, which we accomplish by
adapting comprehension notation. Thus,

\begin{mathpar}
  P\{ y / x : x \in S \}
\end{mathpar}

is interpreted to mean the process derived from P by replacing (in a
capture-avoiding manner) each occurrence of $x$ in $S$ by $y$. For example,

\begin{mathpar}
  P\{ \quotep{\procn{x}|\procn{x}} / x : x \in \freenames{P} \}
\end{mathpar}

will replace each (occurrence) of a free name $x$ in $P$ by
$\quotep{\procn{x}|\procn{x}}$.

Also, we will avail ourselves of the notation $x^{L}$ and $x^{R}$ to
denote injections of a name into disjoint copies of the name
space. There are numerous ways to accomplish this. One example can be
found in \cite{MeredithR05}. This notation overloads to vectors of
names: $\vec{x}^{\pi} := (x_{i}^{\pi} \; : \; 0 \leq i < |\vec{x}| )$ where $\pi \in \{L,R\}$.

We also use $P^{\Box} := P|\Box$.

In \cite{MeredithR05} an interpretation of the new operator is
given. It turns out that there are several possible interpretations
all enjoying the requisite algebraic properties of the operator (see
\cite{milner91polyadicpi}). We will therefore make liberal use of
$(\nu\; \vec{x})P$.

% subsection the_syntax_and_semantics_of_the_notation_system (end)   

\input{qm2pi.qmops} 

\input{qm2pi.sterngerlach} 

\input{qm2pi.metric} 

% section concurrent_process_calculi (end)

%\input{qm2pi.proofsketch}

% section proof sketch (end)

%\input{qm2pi.slviaknots} 

% section spatial logic via knots (end)

\input{qm2pi.conclusion}

% section conclusion (end)

%\input{qm2pi.dtcodes} 

% section wiring algorithm (end)

\input{qm2pi.ack} 

% section acknowledgments (end)

\newpage


\bibliographystyle{plain}   
\bibliography{../../biblios/main.bib}

\input{qm2pi.rhodetails}

\end{document}



% section proof sketch (end)

%\section{Unlikely characters: spatial logic for
  knots}\label{sub:characteristic_formulae} % (fold)

Associated to the mobile process calculi are a family of logics known
as the Hennessy-Milner logics. These logics typically enjoy a
semantics interpreting formulae as sets of processes that when
factored through the encoding outlined above allows an identification
of classes of knots with logical formulae. In the context of this
encoding the sub-family known as the spatial logics \cite{CairesC03}
\cite{CairesC04} \cite{Caires04} are of particular interest providing
several important features for expressing and reasoning about
properties (i.e. classes) of knots. We hint here at how this may be done.

%\begin{description}
%\item [structural connectives] 
\subsubsection{Structural connectives} The spatial logics enjoy
structural connectives corresponding, at the logical level, to the
parallel composition ($P | Q$) and new name ($(\nu \; x)P$)
connectives for processes. As illustrated in the examples below, these
connectives are extremely expressive given the shape of our encoding.
%\item [decideable satisfaction]

\subsubsection{Decideable satisfaction}
In \cite{Caires04} the satisfaction relation is shown to be decideable
for a rich class of processes. It further turns out that the image of
the our encoding is a proper subset of that class. This result
provides the basis for an algorithm by which to search for knots
enjoying a given property.
%\item [characteristic formulae]

\subsubsection{Characteristic formulae}
In the same paper \cite{Caires04} , Caires presents a means of calculating
characteristic formulae, selecting equivalence classes of processes
up to a pre--specified depth limit on the support set of names. Composed with our
encoding, this characteristic formula can be used to select
characteristic formulae for knots.
%\end{description}

\subsubsection{Spatial logic formulae}

The grammar below (segmented for comprehension) summarizes the syntax
of spatial logic formulae. We employ illustrative examples in the
sequel to provide an intuitive understanding of their meaning
referring the reader to \cite{Caires04} for a more detailed explication
of the semantics.

\begin{mathpar}
  \inferrule* [lab=boolean] {} {{A,B} \bc T \;|\; \neg A \;|\; A \wedge B \;|\; \eta = \eta'}
  \and
  \inferrule* [lab=spatial] {} {|\; \pzero \;|\; A | B \;|\; x \text{\textregistered} A \;|\; \forall x . A \;|\;  H x . A}
  \and
  \inferrule* [lab=behavioral] {} {|\; \alpha . A}
  \and 
  \inferrule* [lab=recursion] {} {|\; X(\vec{u}) \;|\; \mu X(\vec{u}) . A}
  \and
  \inferrule* [lab=action] {} {\alpha \bc \langle x?(\vec{y}) \rangle \;|\; \langle x!(\vec{y}) \rangle \;|\; \langle \tau \rangle}
  \and 
  \inferrule* [lab=name] {} {\eta \bc x \;|\; \tau}
\end{mathpar} 

% subsection characteristic_formulae (end)   	 

\subsection{Example formulae}\label{sub:example_formulae_} % (fold)

\subsubsection{Crossing as formula.}
% 
% \begin{align*}
%   \frac{d}{dx} \sin x &= \cos x 
%   & \frac{d}{dx} e^x &= e^x \\
%   \frac{d}{dx} \cos x &= - \sin x 
%   & \frac{d}{dx} \log x &= \frac{1}{x} \\
% \end{align*} 

\begin{align*}
 \mu C(x_{0},x_{1},y_{0},y_{1},u).&(\langle x_{0}?(z) \rangle(\langle u! \rangle\langle y_{1}!z \rangle C(x_{0},x_{1},y_{0},y_{1},u)) & \\
  & \wedge \langle y_{1}?(z) \rangle (\langle u! \rangle \langle x_{0}!z \rangle C(x_{0},x_{1},y_{0},y_{1},u)) & \\
  & \wedge \langle x_{1}?(z) \rangle (\langle u? \rangle \langle y_{0}!z \rangle C(x_{0},x_{1},y_{0},y_{1},u)) & \\
  & \wedge \langle y_{0}?(z) \rangle (\langle u? \rangle \langle x_{1}!z \rangle C(x_{0},x_{1},y_{0},y_{1},u))) &
\end{align*}

The lexicographical similarity between the shape of this formulae and
the shape of definition of the process representing a crossing reveals
the intuitive meaning of this formulae. It describes the capabilities
of a process that has the right to represent a crossing. For example
it picks out processes that may perform an input on the port $x_0$ in
its initial menu of capabilities. What differentiates the formula
from the process, however, is that the crossing process is the
smallest candidate to satisfy the formula. Infinitely many other
processes -- with internal behavior hidden behind this interface, so
to speak -- also satisfy this formula. Even this simple formula,
then, can be seen to open a new view onto knots, providing a
computational interpretation of \emph{virtual} knots.

Note that this formula is derived by hand. A similar formula can be
derived by employing Caires' calculation of characteristic formula
\cite{Caires04} to the process representing a crossing. In light of
this discussion, we let
$\meaningof{C}_{\phi}(x0,x1,y0,y1,u)$ denote a formula specifying the
dynamics we wish to capture of a crossing. To guarantee we preserve
the shape of the interface and minimal semantics we demand that
$\meaningof{C}_{\phi}(x0,x1,y0,y1,u) \Rightarrow
\textbf{C}(x0,x1,y0,y1,u)$ where $\textbf{C}(x0,x1,y0,y1,u)$ denotes
the formula above.
                            
\subsubsection{Crossing number constraints.}
The moral content of the context lemma (Lemma \ref{context}) is that the notion of
``locality'' in the Reidemeister moves is effectively captured by the
parallel composition operator of the process calculus. This intuition
extends through the logic. Given a formula,
$\meaningof{C}_{\phi}(x0,x1,y0,y1,u)$, we can use the structural
connectives to specify constraints on crossing numbers, such as at
least $n$ crossings, or exactly $n$ crossings.
\begin{mathpar}
  \inferrule* [lab=at-least-n] {} { K^{\geq n}_{\phi}(\vec{xs},\vec{ys}) := \Pi_{i=0}^{n-1} Hu . \meaningof{C}_{\phi}(xs_i,ys_i,u) | T }
  \and 
  \inferrule* [lab=exactly-n] {} { K^{= n}_{\phi}(\vec{xs},\vec{ys}) := \Pi_{i=0}^{n-1} Hu . \meaningof{C}_{\phi}(xs_i,ys_i,u) | \neg (\forall x_0,y_0,x_1,y_1,u . \meaningof{C}_{\phi}(x_0,y_0,x_1,y_1,u) | T) }
\end{mathpar}

To round out this section, recall that the encoding of an $n$-crossing
knot decomposes into a parallel composition of $n$ \emph{copies} of a
crossing process together with a wiring harness. To specify different
knot classes with the same crossing number amounts to specifying
logical constraints on the wiring harness. In the interest of space,
we defer examples to a forthcoming paper. Suffice it to say that both
the conditions ``alternating knot'' and ``contains the tangle
corresponding to 5/3'' are expressible. For example, it is possible to
calculate the characteristic formula of a process corresponding to the
tangle 5/3 and conjoin it into the classifying formula via the
composition connective of the logic.

Finally, we wish to observe that it is entirely within reason to
contemplate a more domain-specific version of spatial logic tailored
to the shape of processes in the image of the encoding. Such a
domain-specific logic would have a better claim to the title formal
language of knot properties.

% subsection example_formulae_ (end)

% section knots_as_processes (end) 

% section spatial logic via knots (end)

\section{Conclusions and future work}

\paragraph{Testing physical space}
You, gentle reader, may wonder why of all the theorems to be proved
given this set up we pick the one above. In some sense it's hardly
central to quantum mechanics. We see it as central in the sense that
it firmly establishes a notion of physical space arising from a notion
of the equivalence of behavior. Relating bisimulation to a metric is a
big step forward, but one is faced with interpreting the relationship
of that metric space to something more physical. Quantum mechanical
notions of ``physical'' space are still far from intuitive, but by
relating this idea of distance as testing to calculations that predict
physical circumstances we are making a not insignificant step forward
toward an understanding of the physical space we inhabit as
essentially dynamic.

\paragraph{Effectivity and simulation}
One of the observations we have yet to make is that the entire program
spelled out here is effective. We have built various interpreters for
the reflective calculus at work in this interpretation. In principle,
then, we can simulate quantum mechanics on a computer. The place where
the simulation may lose fidelity is the infinitely branching summation
for the annihilator.

In this connection i also want to point out that the evaluation style
calculation of the inner product puts the non-determinism of the
summation right at the heart of measurement. This suggests that
Milner's original reduction-based formulation of the dynamics of his
calculi in terms of sums was not just notationally suggestive of a
notion of measure-and-continue but captured some significant part of
the physics.

\paragraph{Quantum continuations}
In light of this last observation i want to point out that the
predominant account of quantum mechanics is missing a key aspect of a
truly compositional story of the physical situation. In a real lab,
when a measurement is made the observation can be made to feed into
another device that then makes another measurement conditioned on the
results of the first. This means that after the superposition was
collapsed the entire experimental set up remained in
superposition. While QM offers a means of writing this down it doesn't
quite line up well with the well-trodden formulation of computation
and continuation that we see so succinctly expressed in Milner's
calculi. This suggests that there might be advantages to this account
of dynamics waiting to be explored.

\paragraph{Quantum logic}
In this connection, we also note that by virtue of having the
Hennessy-Milner construction, we can pull the construction through the
interpretation of QM. This gives us a natural candidate for a quantum
logic that enjoys an extremely tight connection with it's domain of
interpretation, making the construction much less ad hoc (rather it is
the image of functor!).

\paragraph{Quantum probabiity}
i have questions about the basis of the interpretation of inner
product as probability amplitude. In particular, using which
axiomatization of probability theory does the notion of probability
amplitude earn the right to be so dubbed? In other words, where is the
proof that the operation for calculating a probability amplitude (and
then squaring) satisfies the axioms of what it means to calculate a
probability? Even if such a proof exists (i have yet to find it in the
literature), i wonder if it might not be possible to turn things on
their heads. Can we view the calculation of the probability amplitude
as an axiomatization of probability? If so, then the definition we
give for calculating probability amplitude may provide the basis for
an \emph{effective} theory of probability.

\paragraph{Quantum vs ``biological'' information}
Finally, i want to conclude with a more philosophical observation. At
a recent workshop in which QM was a predominant topic i noticed
something about quantum information. The speaker was giving a riveting
discussion of axiomatic QM and showing how properties of ``no
cloning'' and ``no deleting'' emerged as consequences of the
axiomatization. Theorems of this form are necessary to give us a sense
of confidence that our axioms characterize the physical theory. What
struck me, though, was that if quantum information is neither erasable
nor replicable it is markedly different from \emph{life}. Two of the
things we know about life is that

\begin{itemize}
  \item it ends;
  \item to gain some measure of persistence, to transcend it's
    finitude it is imminently copyable.
\end{itemize}

Both of these qualities are summarized succinctly in the aphorism: all
flesh is grass. For me these two kinds of ``information'' -- call them
quantum and biological -- are end points on a spectrum of strategies
for persistence. At one end, we have those curious entities that enjoy
uniqueness and permanence; at the other, we have those who in the face
of a certain end and an uncertain present make a go of passing
something on. To me one of the more remarkable aspects of the latter
strategy is that in the presence of noise (and certain features of
copying) we get a kind of dynamism, a chance for improvement against a
given persistent condition.

% subsection other_calculi_other_bisimulations_and_geometry_as_behavior (end)




% section conclusion (end)

%\documentclass[12pt]{llncs}
%\documentclass{jktr}

\usepackage[pdftex]{hyperref}                   
\usepackage {listings}
\usepackage {mathpartir}
\usepackage{bcprules}
%\usepackage{listings}
                       
\usepackage{graphicx} 
%\usepackage[margins=2.5cm,nohead,nofoot]{geometry}
%\usepackage{geometry}
\usepackage{amsfonts}
\usepackage{amstext}
\usepackage{latexsym}
\usepackage{amssymb}
\usepackage{color}


%\include{myPreamble}
\include{qm2pi.local} 

%\ifpdf
%\usepackage[pdftex]{graphicx}
%\else
%\usepackage{graphicx}
%\fi

 % \ifpdf
%  \usepackage{pdfsync}
%  \if


%\title{Brief Article}
%\author{David F. Snyder}
%\author{L.G. Meredith}

%\address{Dept. of Math., Texas State University--San Marcos, San Marcos, TX 78666}
       
\pagestyle{empty}


\begin{document}

\lstset{language=[Objective]Caml,frame=shadowbox}

\input{qm2pi.front}

% section front matter (end)

\input{qm2pi.intro} 
 
% section introduction (end)

% \input{qm2pi.knotations} 

% section notation (end)

\input{qm2pi.process.calculi} 

% section concurrent_process_calculi_and_spatial_logics_ (end)
    
%\input{qm2pi.knots2pi} 

%\input{qm2pi.trefoil} 

%\input{qm2pi.mainthm} 

% subsection basic_interpretation (end)

%\input{qm2pi.rho.presentation} 
\subsection{The syntax and semantics of the notation system}\label{sub:the_syntax_and_semantics_of_the_notation_system} % (fold)

We now summarize a technical presentation of the calculus that
embodies our theory of dynamics. The typical presentation of such a
calculus follows the style of giving generators and relations on
them. The grammar, below, describing term constructors, freely
generates the set of processes, $\Proc$. This set is then quotiented
by a relation known as structural congruence and it is over this set
that the notion of dynamics is expressed. This presentation is
essentially that of \cite{MeredithR05} with the addition of
polyadicity and summation. For readability we have relegated some of
the technical subtleties to an appendix.

\subsubsection{Process grammar}\label{subsub:process_grammar}

\begin{mathpar}
  \inferrule* [lab=synchronization] {} {{M} \bc \pzero \;|\; x?F \;|\; x!C }
  \and
  \inferrule* [lab=abstraction] {} {{F} \bc (x)P}
  \and
  \inferrule* [lab=concretion] {} {{C} \bc \langle Q \rangle}
  \and
  \inferrule* [lab=process] {} {{P,Q} \bc M \;| \;P|Q \;|\; @{x}}
  \and
  \inferrule* [lab=name] {} {{x} \bc \quotep{P}}
\end{mathpar} 

Note that $\vec{x}$ (resp. $\vec{P}$) denotes a vector of names
(resp. processes) of length $|\vec{x}|$ (resp. $|\vec{P}|$). We adopt
the following useful abbreviations.

\begin{mathpar}
   x?(\vec{y}).P := x.(\vec{y})P \and  x\clift{\vec{P}} := x.\clift{\vec{P}}
   \and x!(y) := \lift{x}{\dropn{y}}
   \and \Pi_{i=0}^{n-1}P_i := P_0 | \ldots | P_{n-1}
\end{mathpar}

\subsubsection{Structural congruence}

\paragraph{Free and bound names and alpha-equivalence.} At the
core of structural equivalence is alpha-equivalence which identifies
process that are the same up to a change of variable. Formally, we
recognize the distinction between free and bound names. The free names
of a process, $\freenames{P}$, may be calculated recursively as
follows:

\begin{mathpar}
\freenames{\pzero} := \emptyset
  \and \\
  \freenames{x?(y).P} := \{ x \} \cup (\freenames{P} \setminus \{ y \})
  \and 
  \freenames{x!\langle P \rangle} := \{ x \} \cup \{ P \} 
  \and \\
  \freenames{P|Q} := \freenames{P} \cup \freenames{Q}
  \and \\
  \freenames{@{x}} := \{ x \}
\end{mathpar}

$\pi$
$\quotep{\pi}$

$\freenames{-} : \pi \to \mathcal{P}(\quotep{\pi})$

\begin{eqnarray*}
  \freenames{\pzero} & := & \emptyset \\
  \freenames{x?(y).P} & := & \{ x \} \cup (\freenames{P} \setminus \{ y \}) \\
  \freenames{x!\langle P \rangle} & := & \{ x \} \cup \{ P \} \\
  \freenames{P|Q} & := & \freenames{P} \cup \freenames{Q} \\
  \freenames{\dropn{x}} & := & \{ x \}
\end{eqnarray*}

The bound names of a process, $\boundnames{P}$, are those names occurring in $P$
that are not free. For example, in $x?(y).0$, the name $x$ is free, while $y$ is bound.

\begin{mathpar}
  \inferrule* [lab=monoidal-laws] {} { P|Q \equiv Q|P \and P|0 \equiv P \and P|(Q|R) \equiv (P|Q)|R }
\end{mathpar}

\begin{mathpar}
  \inferrule* [lab=alpha-equivalence] {} { (x)P \equiv (y)P\{y/x\} \and y \not\in \freenames{P} }
\end{mathpar}

\begin{definition}
Then two processes, $P,Q$, are alpha-equivalent if $P = Q\{\vec{y}/\vec{x}\}$ for
some $\vec{x} \in \boundnames{Q},\vec{y} \in \boundnames{P}$, where $Q\{\vec{y}/\vec{x}\}$
denotes the capture-avoiding substitution of $\vec{y}$ for $\vec{x}$ in $Q$.
\end{definition}

\begin{definition}
  The {\em structural congruence} \cite{SangiorgiWalker} , $\equiv$,
  between processes is the least congruence containing
  alpha-equivalence, satisfying the abelian monoid laws
  (associativity, commutativity and $\pzero$ as identity) for parallel
  composition $|$ and for summation $+$.
\end{definition}

\subsection{Name equivalence}

We take name equivalence, written $\nameeq$, to be the smallest
equivalence relation generated by the following rules.

\begin{mathpar}
\inferrule*[lab=Quote-drop]
{ }
{ \quotep{@{x}} \nameeq x }

\inferrule*[lab=Struct-equiv]
{ P \scong Q }
{ \quotep{P} \nameeq \quotep{Q} }
\end{mathpar}

The astute reader will have noticed that the mutual recursion of names
and processes imposes a mutual recursion on alpha-equivalence and
structural equivalence via name-equivalence. Fortunately, all of this
works out pleasantly and we may calculate in the natural way, free of
concern. The reader interested in the details is referred to the
appendix \ref{appendix:rho_details}.

\subsection{Substitution}

We use $\Proc$ for the set of processes, $\QProc$ for the set of
names, and $\id{\{}\vec{y} / \vec{x} \id{\}}$ to denote partial maps,
$s : \QProc \rightarrow \QProc$. A map, $s$ lifts, uniquely, to a map
on process terms, $\widehat{s} : \Proc \rightarrow \Proc$ by the
following equations.

\begin{mathpar}
  (0) \psubstp{Q}{P} := 0 \\
  (R \juxtap S) \psubstp{Q}{P}
  :=    
  (R)\psubstp{Q}{P} \juxtap (S) \psubstp{Q}{P} \\
  (x?(y).R) \psubstp{Q}{P}    
  :=    
  (x)\substp{Q}{P} (z)\concat( (R \psubstn{z}{y}) \psubstp{Q}{P} ) \\
  (\lift{x}{R}) \psubstp{Q}{P}  
  :=
  \lift{(x)\substp{Q}{P}}{ R \psubstp{Q}{P} } \\
%   (\dropn{x})  \psubstp{Q}{P}       
%   := 
%   \left\{ 
%     \begin{array}{ccc} 
%       \dropn{\quotep{Q}} & & x \nameeq \quotep{P} \\
%       \dropn{x} & & otherwise \\
%     \end{array}
%   \right. 
  (\dropn{x})  \psubstp{Q}{P}       
  := 
  \left\{ 
    \begin{array}{ccc} 
      Q & & x \nameeq \quotep{P} \\
      \dropn{x} & & otherwise \\
    \end{array}
  \right.
\end{mathpar}
 

where

\begin{eqnarray}
  (x)\id{\{} \lpquote Q \rpquote / \lpquote P \rpquote \id{\}}            = 
  \left\{ 
    \begin{array}{ccc}
      \lpquote Q \rpquote & & x \nameeq \lpquote P \rpquote \\
      x & & otherwise \\
    \end{array}
  \right. \nonumber
\end{eqnarray}

and $z$ is chosen distinct from $\quotep{P}$, $\quotep{Q}$, the free
names in $Q$, and all the names in $R$. Our $\alpha$-equivalence will
be built in the standard way from this substitution.

\begin{remark}\label{rem:no_self_referential_names}
  One consequence of these definitions is that $\forall P. \quotep{P}
  \not\in \freenames{P}$.
\end{remark}

\subsection{ Dynamic quote: an example }

Anticipating something of what's to come, consider applying the
substitution, $\widehat{\id{\{}u / z \id{\}}}$, to the following pair
of processes, $\lift{w}{y!(z)}$ and $w[ \lpquote y!(z) \rpquote ]$.

\begin{eqnarray}
	\lift{w}{y!(z)}\widehat{\id{\{}u / z \id{\}}}
		& = &
		\lift{w}{y!(u)} \nonumber\\
	w[ \lpquote y!(z) \rpquote ] \widehat{ \id{\{}u / z \id{\}} }
		& = &
		w[ \lpquote y!(z) \rpquote ] \nonumber
\end{eqnarray}

Because the body of the process between quotes is impervious to
substitution, we get radically different answers. In fact, by
examining the first process in an input context,
e.g. $x?(z).\lift{w}{y!(z)}$, we see that the process under the lift
operator may be shaped by prefixed inputs binding a name inside it. In
this sense, the lift operator will be seen as a way to dynamically
construct processes before reifying them as names.

Finally equipped with these standard features we can present the
dynamics of the calculus.

\subsubsection{Operational semantics} 

Finally, we introduce the computational dynamics. What marks these
algebras as distinct from other more traditionally studied algebraic
structures, e.g. vector spaces or polynomial rings, is the manner in
which dynamics is captured. In traditional structures, dynamics is typically
expressed through morphisms between such structures, as in linear maps
between vector spaces or morphisms between rings. In algebras
associated with the semantics of computation, the dynamics is
expressed as part of the algebraic structure itself, through a
reduction reduction relation typically denoted by $\red$. Below, we
give a recursive presentation of this relation for the calculus used
in the encoding.

$\red \subseteq \pi \times \pi$
$\red : \pi \to \mathcal{P}(\pi)$

\begin{mathpar}
  \inferrule* [lab=Comm] { \textsf{match}( x_{src}, x_{trgt} ) } { x_{trgt}?(y)P \; | \; x_{src}!\langle {Q} \rangle \red P\{\quotep{Q}/y}\} }
  \and \\
  \inferrule* [lab=Par] {{P} \red {P}'} {{{P} | {Q}} \red {{P}' | {Q}}}
  \and
  \inferrule* [lab=Equiv]{{{P} \scong {P}'} \andalso {{P}' \red {Q}'} \andalso {{Q}' \scong {Q}}}{{P} \red {Q}}
\end{mathpar}

\begin{eqnarray*}
  match_{\equiv} (\quotep{P},\quotep{Q}) & := & P \equiv Q \\
  match_{\dagger}(\quotep{P},\quotep{Q}) & := & \forall R. P|Q \red^{*} R => R \red^{*} 0 \\
  match_{K}(\quotep{P},\quotep{Q}) & := & K \mbox{ for some context } K
\end{eqnarray*}

$u?(x)P | u!\langle Q \rangle \red P\{\quotep{Q}/x\}$

%We write $\wred$ for $\red^*$, and $P\red$ if $\exists Q $ such that $ P \red Q$.
We write $P\red$ if $\exists Q $ such that $ P \red Q$ and $P\not\red$, otherwise.

\section{Replication}

As mentioned before, it is known that replication (and hence
recursion) can be implemented in a higher-order process algebra
\cite{SangiorgiWalker}. As our first example of calculation with the
machinery thus far presented we give the construction explicitly in
the {\rhoc}.

\begin{eqnarray}
	D_{x} & := & \prefix{x}{y}{(\binpar{\outputp{x}{y}}{@{y}})} \nonumber\\
	\bangp_{x}{P} & := & \binpar{{x}!\langle{\binpar{D_{x}}{P}}\rangle}{D_{x}} \nonumber
\end{eqnarray}

\begin{eqnarray}
	\bangp_{x}{P} & & \nonumber\\
	=
	& {x}!\langle{(\prefix{x}{y}{(\outputp{x}{y} | @{y})) | P}}\rangle 
	      | \prefix{x}{y}{(\outputp{x}{y} | @{y})} & \nonumber\\
	\red
	& (\outputp{x}{y} | @{y})\substn{\quotep{(\prefix{x}{y}{(@{y} | \outputp{x}{y})) | P}}}{y} & \nonumber\\
	=
	& \outputp{x}{\quotep{(\prefix{x}{y}{(\outputp{x}{y} | @{y})) | P}}}
	  | {(\prefix{x}{y}{(\outputp{x}{y} | @{y})) | P}} & \nonumber\\
	\red
	& \ldots & \nonumber\\
	\red^*
	& P | P | \ldots & \nonumber
\end{eqnarray}

Of course, this encoding, as an implementation, runs away, unfolding
$\bangp{P}$ eagerly. A lazier and more implementable replication
operator, restricted to input-guarded processes, may be obtained as follows.

\begin{eqnarray}
\bangp{\prefix{u}{v}{P}} 
	:= 
	\binpar{\lift{x}{\prefix{u}{v}{(\binpar{D(x)}{P})}}}{D(x)} \nonumber
\end{eqnarray}

\begin{remark}
  Note that the lazier definition still does not deal with summation
  or mixed summation (i.e. sums over input and output). The reader is
  invited to construct definitions of replication that deal with these
  features. 

  Further, the definitions are parameterized in a name, $x$. Can you,
  gentle reader, make a definition that eliminates this parameter and
  guarantees no accidental interaction between the replication
  machinery and the process being replicated -- i.e. no accidental
  sharing of names used by the process to get its work done and the
  name(s) used by the replication to effect copying. This latter
  revision of the definition of replication is crucial to obtaining
  the expected identity $!!P \sim !P$.
\end{remark}

\begin{remark}\label{rem:paradoxical_combinator}
  The reader familiar with the lambda calculus will have noticed the
  similarity between $D$ and the paradoxical combinator.

  [Ed. note: the existence of this seems to suggest we have to be more
  restrictive on the set of processes and names we admit if we are to
  support no-cloning.]
\end{remark}

\subsubsection{Bisimulation}

The computational dynamics gives rise to another kind of equivalence,
the equivalence of computational behavior. As previously mentioned
this is typically captured \emph{via} some form of bisimulation.

% The notion we use in this paper is weak barbed bisimulation
% \cite{milner91polyadicpi}.

The notion we use in this paper is derived from weak barbed
bisimulation \cite{milner91polyadicpi}. 

\begin{definition}
An \emph{observation relation}, $\downarrow_{\mathcal N}$, over a set
of names, $\mathcal N$, is the smallest relation satisfying the rules
below.

\infrule[Out-barb]{y \in {\mathcal N}, \; x \nameeq y}
		  {\outputp{x}{v} \downarrow_{\mathcal N} x}
\infrule[Par-barb]{\mbox{$P\downarrow_{\mathcal N} x$ or $Q\downarrow_{\mathcal N} x$}}
		  {\binpar{P}{Q} \downarrow_{\mathcal N} x}

We write $P \Downarrow_{\mathcal N} x$ if there is $Q$ such that 
$P \wred Q$ and $Q \downarrow_{\mathcal N} x$.
\end{definition}

\begin{definition}
%\label{def.bbisim}
An  ${\mathcal N}$-\emph{barbed bisimulation} over a set of names, ${\mathcal N}$, is a symmetric binary relation 
${\mathcal S}_{\mathcal N}$ between agents such that $P\rel{S}_{\mathcal N}Q$ implies:
\begin{enumerate}
\item If $P \red P'$ then $Q \wred Q'$ and $P'\rel{S}_{\mathcal N} Q'$.
\item If $P\downarrow_{\mathcal N} x$, then $Q\Downarrow_{\mathcal N} x$.
\end{enumerate}
$P$ is ${\mathcal N}$-barbed bisimilar to $Q$, written
$P \wbbisim_{\mathcal N} Q$, if $P \rel{S}_{\mathcal N} Q$ for some ${\mathcal N}$-barbed bisimulation ${\mathcal S}_{\mathcal N}$.
\end{definition}

$\mathcal{R} \subseteq \pi \times \pi$

$P \mathcal{R} Q => \forall P'. P \red P' \Rightarrow \exists Q'. Q \red Q', P' \mathcal{R} Q'$

$P \vdash x \Rightarrow Q \vdash x$

\begin{mathpar}
  \inferrule*[lab=Out-barb]{x \nameeq y}{{y}!\langle{Q}\rangle \vdash x}
  \and
  \inferrule*[lab=Par-barb]{\mbox{$P\vdash x$ or $Q\vdash x$}}{\binpar{P}{Q} \vdash x}
\end{mathpar}

\subsubsection{Contexts}

One of the principle advantages of computational calculi like the
$\pi$-calculus is a well-defined notion of context,
contextual-equivalence and a correlation between
contextual-equivalence and notions of bisimulation. The notion of
context allows the decomposition of a process into (sub-)process and
its syntactic environment, its context. Thus, a context may be
thought of as a process with a ``hole'' (written $\Box$) in it. The
application of a context $M$ to a process $P$, written $M[P]$, is
tantamount to filling the hole in $M$ with $P$. In this paper we do
not need the full weight of this theory, but do make use of the notion
of context in the proof the main theorem. 

\begin{mathpar}
  \inferrule* [lab=summation] {} {{M_{M},M_{N}} \bc \Box \;|\; x.M_{A} \;|\; M_{M}+M_{N}}
  \and
  \inferrule* [lab=agent] {} {{M_{A}} \bc (\vec{x})M_{P} \;| \; \clift{P_0,\ldots,M_{P},\ldots,P_N}}
  \and \\
  \inferrule* [lab=process] {} {{M_{P}} \bc M_{N} \;| \;P|M_{P} }
\end{mathpar} 

\begin{mathpar}
  \inferrule* [lab=sychronization] {} {M_{N} \bc \Box \;|\; x?M_{F} \;|\; x!M_{C}}
  \and
  \inferrule* [lab=abstraction] {} {{M_{F}} \bc (x)M_{P} }
  \and
  \inferrule* [lab=concretion] {} {{M_{C}} \bc \langle M_{P} \rangle }
  \and \\
  \inferrule* [lab=process] {} {{M_{P}} \bc M_{N} \;| \;P|M_{P} }
\end{mathpar}

\begin{definition}[contextual application] Given a context $M$, and
  process $P$, we define the \emph{contextual application}, $M[P] :=
  M\{P/\Box\}$. That is, the contextual application of M to P is the
  substitution of $P$ for $\Box$ in $M$.
\end{definition}

$\meaningof{-} : L \to \mathcal{P}(\pi)$

\begin{mathpar}
  \inferrule* [lab=collection] {} {\meaningof{true} = \pi, \and \meaningof{~E} = \pi \setminus \meaningof{E}, \and \meaningof{E_{1} \& E_{2}} = \meaningof{E_{1}} \cap \meaningof{E_{2}}}
\end{mathpar}

\begin{mathpar}
  \inferrule* [lab=structure] {} {\meaningof{0} = \{ P \in \pi | P \equiv 0 \}, \and \\ \meaningof{E_1 | E_2} = \{ P \in \pi | P \equiv P_{1} | P_{2}, P_{1} \in \meaningof{E_{1}}, P_{2} \in \meaningof{E_2}\} }
\end{mathpar}

\begin{mathpar}
 \inferrule* [lab=behavior] {} {\meaningof{\langle a?b \rangle E} = \{ P \in \pi | P \equiv Q | u?(y)P', \\ \and \\\\ \and \\ \;\;\; u \in \meaningof{a}, \forall z.P'\{z/y\} \in \meaningof{E\{z/b\}}\}, \and \\ \meaningof{a!E} = \{ P \in \pi | P \equiv Q | x!\langle P' \rangle, x \in \meaningof{a} P' \in \meaningof{E}\} }
\end{mathpar}

\begin{mathpar}
 \inferrule* [lab=nominal] {} {\meaningof{\quotep{E}} = \{ \quotep{P} \in \quotep{\pi} | P \in \meaningof{E} \}, \and \meaningof{\quotep{P}} = \{ \quotep{Q} \in \quotep{\pi} | P \equiv Q \} \and \\ \meaningof{@\quotep{E}} = \{ P \in \pi | P \equiv @x, x \in \meaningof{E} \}}
\end{mathpar}

\begin{eqnarray*}
  \\
  \meaningof{-} : TS \to ST
\end{eqnarray*}

\begin{eqnarray*}
  \\
  L : TS \to ST
\end{eqnarray*}

\begin{eqnarray*}
  \\
  P \models E \iff P \in \meaningof{E}
\end{eqnarray*}

\begin{eqnarray*}
  P \approx_{L} Q \iff \forall E \in L. P \models E \iff Q \models E
\end{eqnarray*}

\begin{eqnarray*}
  P \approx_{K} Q
\end{eqnarray*}

\begin{eqnarray*}
  P \approx Q
\end{eqnarray*}

$\approx_{K} = \approx = \approx_{L}$

\subsubsection{Contextual duality}

Note that contexts extend the quotation operation to a family of
operations from processes to names. Given a context, $M$, we can
define a \emph{nominal context}, $\quotep{M}$ by $\quotep{M}[P] :=
\quotep{M[P]}$. To foreshadow what is to come we observe that these
operations enjoy a duality with processes very much like the duality
between vectors and maps from vectors to scalars.

Further, because the calculus is essentially higher-order, we have a
correspondence between contexts and processes. More specifically,
given a name $x$ and a context $M$ we can construct $M^{*}_{x}$ such
that 

\begin{mathpar}
  M^{*}_{x} | \lift{x}{P} \red M[P]
\end{mathpar}

namely,

\begin{mathpar}
  M^{*}_{x} := x?(u).M[\dropn{u}]
\end{mathpar}

The dependence of $M^{*}_{x}$ on a name makes it an abstraction, 

\begin{mathpar}
  M^{*} := (x)x?(u).M[\dropn{u}]
\end{mathpar}

\subsection{Additional notation}

It will sometimes be convenient to denote the process a name
quotes. We already have the notation $x = \quotep{P}$, but it will be
convenient to introduce an alternate notation, $\procn{x}$, when we
want to emphasize the connection to the use of the name. Note that, by
virtue of name equivalence, $\quotep{\procn{x}} \nameeq x$; so, the
notation is consistent with previous definitions.

Further, because names have structure it is possible to effect
substitutions on the basis of that structure. This means we need to
upgrade our notation for substitutions, which we accomplish by
adapting comprehension notation. Thus,

\begin{mathpar}
  P\{ y / x : x \in S \}
\end{mathpar}

is interpreted to mean the process derived from P by replacing (in a
capture-avoiding manner) each occurrence of $x$ in $S$ by $y$. For example,

\begin{mathpar}
  P\{ \quotep{\procn{x}|\procn{x}} / x : x \in \freenames{P} \}
\end{mathpar}

will replace each (occurrence) of a free name $x$ in $P$ by
$\quotep{\procn{x}|\procn{x}}$.

Also, we will avail ourselves of the notation $x^{L}$ and $x^{R}$ to
denote injections of a name into disjoint copies of the name
space. There are numerous ways to accomplish this. One example can be
found in \cite{MeredithR05}. This notation overloads to vectors of
names: $\vec{x}^{\pi} := (x_{i}^{\pi} \; : \; 0 \leq i < |\vec{x}| )$ where $\pi \in \{L,R\}$.

We also use $P^{\Box} := P|\Box$.

In \cite{MeredithR05} an interpretation of the new operator is
given. It turns out that there are several possible interpretations
all enjoying the requisite algebraic properties of the operator (see
\cite{milner91polyadicpi}). We will therefore make liberal use of
$(\nu\; \vec{x})P$.

% subsection the_syntax_and_semantics_of_the_notation_system (end)   

\input{qm2pi.qmops} 

\input{qm2pi.sterngerlach} 

\input{qm2pi.metric} 

% section concurrent_process_calculi (end)

%\input{qm2pi.proofsketch}

% section proof sketch (end)

%\input{qm2pi.slviaknots} 

% section spatial logic via knots (end)

\input{qm2pi.conclusion}

% section conclusion (end)

%\input{qm2pi.dtcodes} 

% section wiring algorithm (end)

\input{qm2pi.ack} 

% section acknowledgments (end)

\newpage


\bibliographystyle{plain}   
\bibliography{../../biblios/main.bib}

\input{qm2pi.rhodetails}

\end{document}

 

% section wiring algorithm (end)

\documentclass[12pt]{llncs}
%\documentclass{jktr}

\usepackage[pdftex]{hyperref}                   
\usepackage {listings}
\usepackage {mathpartir}
\usepackage{bcprules}
%\usepackage{listings}
                       
\usepackage{graphicx} 
%\usepackage[margins=2.5cm,nohead,nofoot]{geometry}
%\usepackage{geometry}
\usepackage{amsfonts}
\usepackage{amstext}
\usepackage{latexsym}
\usepackage{amssymb}
\usepackage{color}


%\include{myPreamble}
\include{qm2pi.local} 

%\ifpdf
%\usepackage[pdftex]{graphicx}
%\else
%\usepackage{graphicx}
%\fi

 % \ifpdf
%  \usepackage{pdfsync}
%  \if


%\title{Brief Article}
%\author{David F. Snyder}
%\author{L.G. Meredith}

%\address{Dept. of Math., Texas State University--San Marcos, San Marcos, TX 78666}
       
\pagestyle{empty}


\begin{document}

\lstset{language=[Objective]Caml,frame=shadowbox}

\input{qm2pi.front}

% section front matter (end)

\input{qm2pi.intro} 
 
% section introduction (end)

% \input{qm2pi.knotations} 

% section notation (end)

\input{qm2pi.process.calculi} 

% section concurrent_process_calculi_and_spatial_logics_ (end)
    
%\input{qm2pi.knots2pi} 

%\input{qm2pi.trefoil} 

%\input{qm2pi.mainthm} 

% subsection basic_interpretation (end)

%\input{qm2pi.rho.presentation} 
\subsection{The syntax and semantics of the notation system}\label{sub:the_syntax_and_semantics_of_the_notation_system} % (fold)

We now summarize a technical presentation of the calculus that
embodies our theory of dynamics. The typical presentation of such a
calculus follows the style of giving generators and relations on
them. The grammar, below, describing term constructors, freely
generates the set of processes, $\Proc$. This set is then quotiented
by a relation known as structural congruence and it is over this set
that the notion of dynamics is expressed. This presentation is
essentially that of \cite{MeredithR05} with the addition of
polyadicity and summation. For readability we have relegated some of
the technical subtleties to an appendix.

\subsubsection{Process grammar}\label{subsub:process_grammar}

\begin{mathpar}
  \inferrule* [lab=synchronization] {} {{M} \bc \pzero \;|\; x?F \;|\; x!C }
  \and
  \inferrule* [lab=abstraction] {} {{F} \bc (x)P}
  \and
  \inferrule* [lab=concretion] {} {{C} \bc \langle Q \rangle}
  \and
  \inferrule* [lab=process] {} {{P,Q} \bc M \;| \;P|Q \;|\; @{x}}
  \and
  \inferrule* [lab=name] {} {{x} \bc \quotep{P}}
\end{mathpar} 

Note that $\vec{x}$ (resp. $\vec{P}$) denotes a vector of names
(resp. processes) of length $|\vec{x}|$ (resp. $|\vec{P}|$). We adopt
the following useful abbreviations.

\begin{mathpar}
   x?(\vec{y}).P := x.(\vec{y})P \and  x\clift{\vec{P}} := x.\clift{\vec{P}}
   \and x!(y) := \lift{x}{\dropn{y}}
   \and \Pi_{i=0}^{n-1}P_i := P_0 | \ldots | P_{n-1}
\end{mathpar}

\subsubsection{Structural congruence}

\paragraph{Free and bound names and alpha-equivalence.} At the
core of structural equivalence is alpha-equivalence which identifies
process that are the same up to a change of variable. Formally, we
recognize the distinction between free and bound names. The free names
of a process, $\freenames{P}$, may be calculated recursively as
follows:

\begin{mathpar}
\freenames{\pzero} := \emptyset
  \and \\
  \freenames{x?(y).P} := \{ x \} \cup (\freenames{P} \setminus \{ y \})
  \and 
  \freenames{x!\langle P \rangle} := \{ x \} \cup \{ P \} 
  \and \\
  \freenames{P|Q} := \freenames{P} \cup \freenames{Q}
  \and \\
  \freenames{@{x}} := \{ x \}
\end{mathpar}

$\pi$
$\quotep{\pi}$

$\freenames{-} : \pi \to \mathcal{P}(\quotep{\pi})$

\begin{eqnarray*}
  \freenames{\pzero} & := & \emptyset \\
  \freenames{x?(y).P} & := & \{ x \} \cup (\freenames{P} \setminus \{ y \}) \\
  \freenames{x!\langle P \rangle} & := & \{ x \} \cup \{ P \} \\
  \freenames{P|Q} & := & \freenames{P} \cup \freenames{Q} \\
  \freenames{\dropn{x}} & := & \{ x \}
\end{eqnarray*}

The bound names of a process, $\boundnames{P}$, are those names occurring in $P$
that are not free. For example, in $x?(y).0$, the name $x$ is free, while $y$ is bound.

\begin{mathpar}
  \inferrule* [lab=monoidal-laws] {} { P|Q \equiv Q|P \and P|0 \equiv P \and P|(Q|R) \equiv (P|Q)|R }
\end{mathpar}

\begin{mathpar}
  \inferrule* [lab=alpha-equivalence] {} { (x)P \equiv (y)P\{y/x\} \and y \not\in \freenames{P} }
\end{mathpar}

\begin{definition}
Then two processes, $P,Q$, are alpha-equivalent if $P = Q\{\vec{y}/\vec{x}\}$ for
some $\vec{x} \in \boundnames{Q},\vec{y} \in \boundnames{P}$, where $Q\{\vec{y}/\vec{x}\}$
denotes the capture-avoiding substitution of $\vec{y}$ for $\vec{x}$ in $Q$.
\end{definition}

\begin{definition}
  The {\em structural congruence} \cite{SangiorgiWalker} , $\equiv$,
  between processes is the least congruence containing
  alpha-equivalence, satisfying the abelian monoid laws
  (associativity, commutativity and $\pzero$ as identity) for parallel
  composition $|$ and for summation $+$.
\end{definition}

\subsection{Name equivalence}

We take name equivalence, written $\nameeq$, to be the smallest
equivalence relation generated by the following rules.

\begin{mathpar}
\inferrule*[lab=Quote-drop]
{ }
{ \quotep{@{x}} \nameeq x }

\inferrule*[lab=Struct-equiv]
{ P \scong Q }
{ \quotep{P} \nameeq \quotep{Q} }
\end{mathpar}

The astute reader will have noticed that the mutual recursion of names
and processes imposes a mutual recursion on alpha-equivalence and
structural equivalence via name-equivalence. Fortunately, all of this
works out pleasantly and we may calculate in the natural way, free of
concern. The reader interested in the details is referred to the
appendix \ref{appendix:rho_details}.

\subsection{Substitution}

We use $\Proc$ for the set of processes, $\QProc$ for the set of
names, and $\id{\{}\vec{y} / \vec{x} \id{\}}$ to denote partial maps,
$s : \QProc \rightarrow \QProc$. A map, $s$ lifts, uniquely, to a map
on process terms, $\widehat{s} : \Proc \rightarrow \Proc$ by the
following equations.

\begin{mathpar}
  (0) \psubstp{Q}{P} := 0 \\
  (R \juxtap S) \psubstp{Q}{P}
  :=    
  (R)\psubstp{Q}{P} \juxtap (S) \psubstp{Q}{P} \\
  (x?(y).R) \psubstp{Q}{P}    
  :=    
  (x)\substp{Q}{P} (z)\concat( (R \psubstn{z}{y}) \psubstp{Q}{P} ) \\
  (\lift{x}{R}) \psubstp{Q}{P}  
  :=
  \lift{(x)\substp{Q}{P}}{ R \psubstp{Q}{P} } \\
%   (\dropn{x})  \psubstp{Q}{P}       
%   := 
%   \left\{ 
%     \begin{array}{ccc} 
%       \dropn{\quotep{Q}} & & x \nameeq \quotep{P} \\
%       \dropn{x} & & otherwise \\
%     \end{array}
%   \right. 
  (\dropn{x})  \psubstp{Q}{P}       
  := 
  \left\{ 
    \begin{array}{ccc} 
      Q & & x \nameeq \quotep{P} \\
      \dropn{x} & & otherwise \\
    \end{array}
  \right.
\end{mathpar}
 

where

\begin{eqnarray}
  (x)\id{\{} \lpquote Q \rpquote / \lpquote P \rpquote \id{\}}            = 
  \left\{ 
    \begin{array}{ccc}
      \lpquote Q \rpquote & & x \nameeq \lpquote P \rpquote \\
      x & & otherwise \\
    \end{array}
  \right. \nonumber
\end{eqnarray}

and $z$ is chosen distinct from $\quotep{P}$, $\quotep{Q}$, the free
names in $Q$, and all the names in $R$. Our $\alpha$-equivalence will
be built in the standard way from this substitution.

\begin{remark}\label{rem:no_self_referential_names}
  One consequence of these definitions is that $\forall P. \quotep{P}
  \not\in \freenames{P}$.
\end{remark}

\subsection{ Dynamic quote: an example }

Anticipating something of what's to come, consider applying the
substitution, $\widehat{\id{\{}u / z \id{\}}}$, to the following pair
of processes, $\lift{w}{y!(z)}$ and $w[ \lpquote y!(z) \rpquote ]$.

\begin{eqnarray}
	\lift{w}{y!(z)}\widehat{\id{\{}u / z \id{\}}}
		& = &
		\lift{w}{y!(u)} \nonumber\\
	w[ \lpquote y!(z) \rpquote ] \widehat{ \id{\{}u / z \id{\}} }
		& = &
		w[ \lpquote y!(z) \rpquote ] \nonumber
\end{eqnarray}

Because the body of the process between quotes is impervious to
substitution, we get radically different answers. In fact, by
examining the first process in an input context,
e.g. $x?(z).\lift{w}{y!(z)}$, we see that the process under the lift
operator may be shaped by prefixed inputs binding a name inside it. In
this sense, the lift operator will be seen as a way to dynamically
construct processes before reifying them as names.

Finally equipped with these standard features we can present the
dynamics of the calculus.

\subsubsection{Operational semantics} 

Finally, we introduce the computational dynamics. What marks these
algebras as distinct from other more traditionally studied algebraic
structures, e.g. vector spaces or polynomial rings, is the manner in
which dynamics is captured. In traditional structures, dynamics is typically
expressed through morphisms between such structures, as in linear maps
between vector spaces or morphisms between rings. In algebras
associated with the semantics of computation, the dynamics is
expressed as part of the algebraic structure itself, through a
reduction reduction relation typically denoted by $\red$. Below, we
give a recursive presentation of this relation for the calculus used
in the encoding.

$\red \subseteq \pi \times \pi$
$\red : \pi \to \mathcal{P}(\pi)$

\begin{mathpar}
  \inferrule* [lab=Comm] { \textsf{match}( x_{src}, x_{trgt} ) } { x_{trgt}?(y)P \; | \; x_{src}!\langle {Q} \rangle \red P\{\quotep{Q}/y}\} }
  \and \\
  \inferrule* [lab=Par] {{P} \red {P}'} {{{P} | {Q}} \red {{P}' | {Q}}}
  \and
  \inferrule* [lab=Equiv]{{{P} \scong {P}'} \andalso {{P}' \red {Q}'} \andalso {{Q}' \scong {Q}}}{{P} \red {Q}}
\end{mathpar}

\begin{eqnarray*}
  match_{\equiv} (\quotep{P},\quotep{Q}) & := & P \equiv Q \\
  match_{\dagger}(\quotep{P},\quotep{Q}) & := & \forall R. P|Q \red^{*} R => R \red^{*} 0 \\
  match_{K}(\quotep{P},\quotep{Q}) & := & K \mbox{ for some context } K
\end{eqnarray*}

$u?(x)P | u!\langle Q \rangle \red P\{\quotep{Q}/x\}$

%We write $\wred$ for $\red^*$, and $P\red$ if $\exists Q $ such that $ P \red Q$.
We write $P\red$ if $\exists Q $ such that $ P \red Q$ and $P\not\red$, otherwise.

\section{Replication}

As mentioned before, it is known that replication (and hence
recursion) can be implemented in a higher-order process algebra
\cite{SangiorgiWalker}. As our first example of calculation with the
machinery thus far presented we give the construction explicitly in
the {\rhoc}.

\begin{eqnarray}
	D_{x} & := & \prefix{x}{y}{(\binpar{\outputp{x}{y}}{@{y}})} \nonumber\\
	\bangp_{x}{P} & := & \binpar{{x}!\langle{\binpar{D_{x}}{P}}\rangle}{D_{x}} \nonumber
\end{eqnarray}

\begin{eqnarray}
	\bangp_{x}{P} & & \nonumber\\
	=
	& {x}!\langle{(\prefix{x}{y}{(\outputp{x}{y} | @{y})) | P}}\rangle 
	      | \prefix{x}{y}{(\outputp{x}{y} | @{y})} & \nonumber\\
	\red
	& (\outputp{x}{y} | @{y})\substn{\quotep{(\prefix{x}{y}{(@{y} | \outputp{x}{y})) | P}}}{y} & \nonumber\\
	=
	& \outputp{x}{\quotep{(\prefix{x}{y}{(\outputp{x}{y} | @{y})) | P}}}
	  | {(\prefix{x}{y}{(\outputp{x}{y} | @{y})) | P}} & \nonumber\\
	\red
	& \ldots & \nonumber\\
	\red^*
	& P | P | \ldots & \nonumber
\end{eqnarray}

Of course, this encoding, as an implementation, runs away, unfolding
$\bangp{P}$ eagerly. A lazier and more implementable replication
operator, restricted to input-guarded processes, may be obtained as follows.

\begin{eqnarray}
\bangp{\prefix{u}{v}{P}} 
	:= 
	\binpar{\lift{x}{\prefix{u}{v}{(\binpar{D(x)}{P})}}}{D(x)} \nonumber
\end{eqnarray}

\begin{remark}
  Note that the lazier definition still does not deal with summation
  or mixed summation (i.e. sums over input and output). The reader is
  invited to construct definitions of replication that deal with these
  features. 

  Further, the definitions are parameterized in a name, $x$. Can you,
  gentle reader, make a definition that eliminates this parameter and
  guarantees no accidental interaction between the replication
  machinery and the process being replicated -- i.e. no accidental
  sharing of names used by the process to get its work done and the
  name(s) used by the replication to effect copying. This latter
  revision of the definition of replication is crucial to obtaining
  the expected identity $!!P \sim !P$.
\end{remark}

\begin{remark}\label{rem:paradoxical_combinator}
  The reader familiar with the lambda calculus will have noticed the
  similarity between $D$ and the paradoxical combinator.

  [Ed. note: the existence of this seems to suggest we have to be more
  restrictive on the set of processes and names we admit if we are to
  support no-cloning.]
\end{remark}

\subsubsection{Bisimulation}

The computational dynamics gives rise to another kind of equivalence,
the equivalence of computational behavior. As previously mentioned
this is typically captured \emph{via} some form of bisimulation.

% The notion we use in this paper is weak barbed bisimulation
% \cite{milner91polyadicpi}.

The notion we use in this paper is derived from weak barbed
bisimulation \cite{milner91polyadicpi}. 

\begin{definition}
An \emph{observation relation}, $\downarrow_{\mathcal N}$, over a set
of names, $\mathcal N$, is the smallest relation satisfying the rules
below.

\infrule[Out-barb]{y \in {\mathcal N}, \; x \nameeq y}
		  {\outputp{x}{v} \downarrow_{\mathcal N} x}
\infrule[Par-barb]{\mbox{$P\downarrow_{\mathcal N} x$ or $Q\downarrow_{\mathcal N} x$}}
		  {\binpar{P}{Q} \downarrow_{\mathcal N} x}

We write $P \Downarrow_{\mathcal N} x$ if there is $Q$ such that 
$P \wred Q$ and $Q \downarrow_{\mathcal N} x$.
\end{definition}

\begin{definition}
%\label{def.bbisim}
An  ${\mathcal N}$-\emph{barbed bisimulation} over a set of names, ${\mathcal N}$, is a symmetric binary relation 
${\mathcal S}_{\mathcal N}$ between agents such that $P\rel{S}_{\mathcal N}Q$ implies:
\begin{enumerate}
\item If $P \red P'$ then $Q \wred Q'$ and $P'\rel{S}_{\mathcal N} Q'$.
\item If $P\downarrow_{\mathcal N} x$, then $Q\Downarrow_{\mathcal N} x$.
\end{enumerate}
$P$ is ${\mathcal N}$-barbed bisimilar to $Q$, written
$P \wbbisim_{\mathcal N} Q$, if $P \rel{S}_{\mathcal N} Q$ for some ${\mathcal N}$-barbed bisimulation ${\mathcal S}_{\mathcal N}$.
\end{definition}

$\mathcal{R} \subseteq \pi \times \pi$

$P \mathcal{R} Q => \forall P'. P \red P' \Rightarrow \exists Q'. Q \red Q', P' \mathcal{R} Q'$

$P \vdash x \Rightarrow Q \vdash x$

\begin{mathpar}
  \inferrule*[lab=Out-barb]{x \nameeq y}{{y}!\langle{Q}\rangle \vdash x}
  \and
  \inferrule*[lab=Par-barb]{\mbox{$P\vdash x$ or $Q\vdash x$}}{\binpar{P}{Q} \vdash x}
\end{mathpar}

\subsubsection{Contexts}

One of the principle advantages of computational calculi like the
$\pi$-calculus is a well-defined notion of context,
contextual-equivalence and a correlation between
contextual-equivalence and notions of bisimulation. The notion of
context allows the decomposition of a process into (sub-)process and
its syntactic environment, its context. Thus, a context may be
thought of as a process with a ``hole'' (written $\Box$) in it. The
application of a context $M$ to a process $P$, written $M[P]$, is
tantamount to filling the hole in $M$ with $P$. In this paper we do
not need the full weight of this theory, but do make use of the notion
of context in the proof the main theorem. 

\begin{mathpar}
  \inferrule* [lab=summation] {} {{M_{M},M_{N}} \bc \Box \;|\; x.M_{A} \;|\; M_{M}+M_{N}}
  \and
  \inferrule* [lab=agent] {} {{M_{A}} \bc (\vec{x})M_{P} \;| \; \clift{P_0,\ldots,M_{P},\ldots,P_N}}
  \and \\
  \inferrule* [lab=process] {} {{M_{P}} \bc M_{N} \;| \;P|M_{P} }
\end{mathpar} 

\begin{mathpar}
  \inferrule* [lab=sychronization] {} {M_{N} \bc \Box \;|\; x?M_{F} \;|\; x!M_{C}}
  \and
  \inferrule* [lab=abstraction] {} {{M_{F}} \bc (x)M_{P} }
  \and
  \inferrule* [lab=concretion] {} {{M_{C}} \bc \langle M_{P} \rangle }
  \and \\
  \inferrule* [lab=process] {} {{M_{P}} \bc M_{N} \;| \;P|M_{P} }
\end{mathpar}

\begin{definition}[contextual application] Given a context $M$, and
  process $P$, we define the \emph{contextual application}, $M[P] :=
  M\{P/\Box\}$. That is, the contextual application of M to P is the
  substitution of $P$ for $\Box$ in $M$.
\end{definition}

$\meaningof{-} : L \to \mathcal{P}(\pi)$

\begin{mathpar}
  \inferrule* [lab=collection] {} {\meaningof{true} = \pi, \and \meaningof{~E} = \pi \setminus \meaningof{E}, \and \meaningof{E_{1} \& E_{2}} = \meaningof{E_{1}} \cap \meaningof{E_{2}}}
\end{mathpar}

\begin{mathpar}
  \inferrule* [lab=structure] {} {\meaningof{0} = \{ P \in \pi | P \equiv 0 \}, \and \\ \meaningof{E_1 | E_2} = \{ P \in \pi | P \equiv P_{1} | P_{2}, P_{1} \in \meaningof{E_{1}}, P_{2} \in \meaningof{E_2}\} }
\end{mathpar}

\begin{mathpar}
 \inferrule* [lab=behavior] {} {\meaningof{\langle a?b \rangle E} = \{ P \in \pi | P \equiv Q | u?(y)P', \\ \and \\\\ \and \\ \;\;\; u \in \meaningof{a}, \forall z.P'\{z/y\} \in \meaningof{E\{z/b\}}\}, \and \\ \meaningof{a!E} = \{ P \in \pi | P \equiv Q | x!\langle P' \rangle, x \in \meaningof{a} P' \in \meaningof{E}\} }
\end{mathpar}

\begin{mathpar}
 \inferrule* [lab=nominal] {} {\meaningof{\quotep{E}} = \{ \quotep{P} \in \quotep{\pi} | P \in \meaningof{E} \}, \and \meaningof{\quotep{P}} = \{ \quotep{Q} \in \quotep{\pi} | P \equiv Q \} \and \\ \meaningof{@\quotep{E}} = \{ P \in \pi | P \equiv @x, x \in \meaningof{E} \}}
\end{mathpar}

\begin{eqnarray*}
  \\
  \meaningof{-} : TS \to ST
\end{eqnarray*}

\begin{eqnarray*}
  \\
  L : TS \to ST
\end{eqnarray*}

\begin{eqnarray*}
  \\
  P \models E \iff P \in \meaningof{E}
\end{eqnarray*}

\begin{eqnarray*}
  P \approx_{L} Q \iff \forall E \in L. P \models E \iff Q \models E
\end{eqnarray*}

\begin{eqnarray*}
  P \approx_{K} Q
\end{eqnarray*}

\begin{eqnarray*}
  P \approx Q
\end{eqnarray*}

$\approx_{K} = \approx = \approx_{L}$

\subsubsection{Contextual duality}

Note that contexts extend the quotation operation to a family of
operations from processes to names. Given a context, $M$, we can
define a \emph{nominal context}, $\quotep{M}$ by $\quotep{M}[P] :=
\quotep{M[P]}$. To foreshadow what is to come we observe that these
operations enjoy a duality with processes very much like the duality
between vectors and maps from vectors to scalars.

Further, because the calculus is essentially higher-order, we have a
correspondence between contexts and processes. More specifically,
given a name $x$ and a context $M$ we can construct $M^{*}_{x}$ such
that 

\begin{mathpar}
  M^{*}_{x} | \lift{x}{P} \red M[P]
\end{mathpar}

namely,

\begin{mathpar}
  M^{*}_{x} := x?(u).M[\dropn{u}]
\end{mathpar}

The dependence of $M^{*}_{x}$ on a name makes it an abstraction, 

\begin{mathpar}
  M^{*} := (x)x?(u).M[\dropn{u}]
\end{mathpar}

\subsection{Additional notation}

It will sometimes be convenient to denote the process a name
quotes. We already have the notation $x = \quotep{P}$, but it will be
convenient to introduce an alternate notation, $\procn{x}$, when we
want to emphasize the connection to the use of the name. Note that, by
virtue of name equivalence, $\quotep{\procn{x}} \nameeq x$; so, the
notation is consistent with previous definitions.

Further, because names have structure it is possible to effect
substitutions on the basis of that structure. This means we need to
upgrade our notation for substitutions, which we accomplish by
adapting comprehension notation. Thus,

\begin{mathpar}
  P\{ y / x : x \in S \}
\end{mathpar}

is interpreted to mean the process derived from P by replacing (in a
capture-avoiding manner) each occurrence of $x$ in $S$ by $y$. For example,

\begin{mathpar}
  P\{ \quotep{\procn{x}|\procn{x}} / x : x \in \freenames{P} \}
\end{mathpar}

will replace each (occurrence) of a free name $x$ in $P$ by
$\quotep{\procn{x}|\procn{x}}$.

Also, we will avail ourselves of the notation $x^{L}$ and $x^{R}$ to
denote injections of a name into disjoint copies of the name
space. There are numerous ways to accomplish this. One example can be
found in \cite{MeredithR05}. This notation overloads to vectors of
names: $\vec{x}^{\pi} := (x_{i}^{\pi} \; : \; 0 \leq i < |\vec{x}| )$ where $\pi \in \{L,R\}$.

We also use $P^{\Box} := P|\Box$.

In \cite{MeredithR05} an interpretation of the new operator is
given. It turns out that there are several possible interpretations
all enjoying the requisite algebraic properties of the operator (see
\cite{milner91polyadicpi}). We will therefore make liberal use of
$(\nu\; \vec{x})P$.

% subsection the_syntax_and_semantics_of_the_notation_system (end)   

\input{qm2pi.qmops} 

\input{qm2pi.sterngerlach} 

\input{qm2pi.metric} 

% section concurrent_process_calculi (end)

%\input{qm2pi.proofsketch}

% section proof sketch (end)

%\input{qm2pi.slviaknots} 

% section spatial logic via knots (end)

\input{qm2pi.conclusion}

% section conclusion (end)

%\input{qm2pi.dtcodes} 

% section wiring algorithm (end)

\input{qm2pi.ack} 

% section acknowledgments (end)

\newpage


\bibliographystyle{plain}   
\bibliography{../../biblios/main.bib}

\input{qm2pi.rhodetails}

\end{document}

 

% section acknowledgments (end)

\newpage


\bibliographystyle{plain}   
\bibliography{../../biblios/main.bib}

\documentclass[12pt]{llncs}
%\documentclass{jktr}

\usepackage[pdftex]{hyperref}                   
\usepackage {listings}
\usepackage {mathpartir}
\usepackage{bcprules}
%\usepackage{listings}
                       
\usepackage{graphicx} 
%\usepackage[margins=2.5cm,nohead,nofoot]{geometry}
%\usepackage{geometry}
\usepackage{amsfonts}
\usepackage{amstext}
\usepackage{latexsym}
\usepackage{amssymb}
\usepackage{color}


%\include{myPreamble}
\include{qm2pi.local} 

%\ifpdf
%\usepackage[pdftex]{graphicx}
%\else
%\usepackage{graphicx}
%\fi

 % \ifpdf
%  \usepackage{pdfsync}
%  \if


%\title{Brief Article}
%\author{David F. Snyder}
%\author{L.G. Meredith}

%\address{Dept. of Math., Texas State University--San Marcos, San Marcos, TX 78666}
       
\pagestyle{empty}


\begin{document}

\lstset{language=[Objective]Caml,frame=shadowbox}

\input{qm2pi.front}

% section front matter (end)

\input{qm2pi.intro} 
 
% section introduction (end)

% \input{qm2pi.knotations} 

% section notation (end)

\input{qm2pi.process.calculi} 

% section concurrent_process_calculi_and_spatial_logics_ (end)
    
%\input{qm2pi.knots2pi} 

%\input{qm2pi.trefoil} 

%\input{qm2pi.mainthm} 

% subsection basic_interpretation (end)

%\input{qm2pi.rho.presentation} 
\subsection{The syntax and semantics of the notation system}\label{sub:the_syntax_and_semantics_of_the_notation_system} % (fold)

We now summarize a technical presentation of the calculus that
embodies our theory of dynamics. The typical presentation of such a
calculus follows the style of giving generators and relations on
them. The grammar, below, describing term constructors, freely
generates the set of processes, $\Proc$. This set is then quotiented
by a relation known as structural congruence and it is over this set
that the notion of dynamics is expressed. This presentation is
essentially that of \cite{MeredithR05} with the addition of
polyadicity and summation. For readability we have relegated some of
the technical subtleties to an appendix.

\subsubsection{Process grammar}\label{subsub:process_grammar}

\begin{mathpar}
  \inferrule* [lab=synchronization] {} {{M} \bc \pzero \;|\; x?F \;|\; x!C }
  \and
  \inferrule* [lab=abstraction] {} {{F} \bc (x)P}
  \and
  \inferrule* [lab=concretion] {} {{C} \bc \langle Q \rangle}
  \and
  \inferrule* [lab=process] {} {{P,Q} \bc M \;| \;P|Q \;|\; @{x}}
  \and
  \inferrule* [lab=name] {} {{x} \bc \quotep{P}}
\end{mathpar} 

Note that $\vec{x}$ (resp. $\vec{P}$) denotes a vector of names
(resp. processes) of length $|\vec{x}|$ (resp. $|\vec{P}|$). We adopt
the following useful abbreviations.

\begin{mathpar}
   x?(\vec{y}).P := x.(\vec{y})P \and  x\clift{\vec{P}} := x.\clift{\vec{P}}
   \and x!(y) := \lift{x}{\dropn{y}}
   \and \Pi_{i=0}^{n-1}P_i := P_0 | \ldots | P_{n-1}
\end{mathpar}

\subsubsection{Structural congruence}

\paragraph{Free and bound names and alpha-equivalence.} At the
core of structural equivalence is alpha-equivalence which identifies
process that are the same up to a change of variable. Formally, we
recognize the distinction between free and bound names. The free names
of a process, $\freenames{P}$, may be calculated recursively as
follows:

\begin{mathpar}
\freenames{\pzero} := \emptyset
  \and \\
  \freenames{x?(y).P} := \{ x \} \cup (\freenames{P} \setminus \{ y \})
  \and 
  \freenames{x!\langle P \rangle} := \{ x \} \cup \{ P \} 
  \and \\
  \freenames{P|Q} := \freenames{P} \cup \freenames{Q}
  \and \\
  \freenames{@{x}} := \{ x \}
\end{mathpar}

$\pi$
$\quotep{\pi}$

$\freenames{-} : \pi \to \mathcal{P}(\quotep{\pi})$

\begin{eqnarray*}
  \freenames{\pzero} & := & \emptyset \\
  \freenames{x?(y).P} & := & \{ x \} \cup (\freenames{P} \setminus \{ y \}) \\
  \freenames{x!\langle P \rangle} & := & \{ x \} \cup \{ P \} \\
  \freenames{P|Q} & := & \freenames{P} \cup \freenames{Q} \\
  \freenames{\dropn{x}} & := & \{ x \}
\end{eqnarray*}

The bound names of a process, $\boundnames{P}$, are those names occurring in $P$
that are not free. For example, in $x?(y).0$, the name $x$ is free, while $y$ is bound.

\begin{mathpar}
  \inferrule* [lab=monoidal-laws] {} { P|Q \equiv Q|P \and P|0 \equiv P \and P|(Q|R) \equiv (P|Q)|R }
\end{mathpar}

\begin{mathpar}
  \inferrule* [lab=alpha-equivalence] {} { (x)P \equiv (y)P\{y/x\} \and y \not\in \freenames{P} }
\end{mathpar}

\begin{definition}
Then two processes, $P,Q$, are alpha-equivalent if $P = Q\{\vec{y}/\vec{x}\}$ for
some $\vec{x} \in \boundnames{Q},\vec{y} \in \boundnames{P}$, where $Q\{\vec{y}/\vec{x}\}$
denotes the capture-avoiding substitution of $\vec{y}$ for $\vec{x}$ in $Q$.
\end{definition}

\begin{definition}
  The {\em structural congruence} \cite{SangiorgiWalker} , $\equiv$,
  between processes is the least congruence containing
  alpha-equivalence, satisfying the abelian monoid laws
  (associativity, commutativity and $\pzero$ as identity) for parallel
  composition $|$ and for summation $+$.
\end{definition}

\subsection{Name equivalence}

We take name equivalence, written $\nameeq$, to be the smallest
equivalence relation generated by the following rules.

\begin{mathpar}
\inferrule*[lab=Quote-drop]
{ }
{ \quotep{@{x}} \nameeq x }

\inferrule*[lab=Struct-equiv]
{ P \scong Q }
{ \quotep{P} \nameeq \quotep{Q} }
\end{mathpar}

The astute reader will have noticed that the mutual recursion of names
and processes imposes a mutual recursion on alpha-equivalence and
structural equivalence via name-equivalence. Fortunately, all of this
works out pleasantly and we may calculate in the natural way, free of
concern. The reader interested in the details is referred to the
appendix \ref{appendix:rho_details}.

\subsection{Substitution}

We use $\Proc$ for the set of processes, $\QProc$ for the set of
names, and $\id{\{}\vec{y} / \vec{x} \id{\}}$ to denote partial maps,
$s : \QProc \rightarrow \QProc$. A map, $s$ lifts, uniquely, to a map
on process terms, $\widehat{s} : \Proc \rightarrow \Proc$ by the
following equations.

\begin{mathpar}
  (0) \psubstp{Q}{P} := 0 \\
  (R \juxtap S) \psubstp{Q}{P}
  :=    
  (R)\psubstp{Q}{P} \juxtap (S) \psubstp{Q}{P} \\
  (x?(y).R) \psubstp{Q}{P}    
  :=    
  (x)\substp{Q}{P} (z)\concat( (R \psubstn{z}{y}) \psubstp{Q}{P} ) \\
  (\lift{x}{R}) \psubstp{Q}{P}  
  :=
  \lift{(x)\substp{Q}{P}}{ R \psubstp{Q}{P} } \\
%   (\dropn{x})  \psubstp{Q}{P}       
%   := 
%   \left\{ 
%     \begin{array}{ccc} 
%       \dropn{\quotep{Q}} & & x \nameeq \quotep{P} \\
%       \dropn{x} & & otherwise \\
%     \end{array}
%   \right. 
  (\dropn{x})  \psubstp{Q}{P}       
  := 
  \left\{ 
    \begin{array}{ccc} 
      Q & & x \nameeq \quotep{P} \\
      \dropn{x} & & otherwise \\
    \end{array}
  \right.
\end{mathpar}
 

where

\begin{eqnarray}
  (x)\id{\{} \lpquote Q \rpquote / \lpquote P \rpquote \id{\}}            = 
  \left\{ 
    \begin{array}{ccc}
      \lpquote Q \rpquote & & x \nameeq \lpquote P \rpquote \\
      x & & otherwise \\
    \end{array}
  \right. \nonumber
\end{eqnarray}

and $z$ is chosen distinct from $\quotep{P}$, $\quotep{Q}$, the free
names in $Q$, and all the names in $R$. Our $\alpha$-equivalence will
be built in the standard way from this substitution.

\begin{remark}\label{rem:no_self_referential_names}
  One consequence of these definitions is that $\forall P. \quotep{P}
  \not\in \freenames{P}$.
\end{remark}

\subsection{ Dynamic quote: an example }

Anticipating something of what's to come, consider applying the
substitution, $\widehat{\id{\{}u / z \id{\}}}$, to the following pair
of processes, $\lift{w}{y!(z)}$ and $w[ \lpquote y!(z) \rpquote ]$.

\begin{eqnarray}
	\lift{w}{y!(z)}\widehat{\id{\{}u / z \id{\}}}
		& = &
		\lift{w}{y!(u)} \nonumber\\
	w[ \lpquote y!(z) \rpquote ] \widehat{ \id{\{}u / z \id{\}} }
		& = &
		w[ \lpquote y!(z) \rpquote ] \nonumber
\end{eqnarray}

Because the body of the process between quotes is impervious to
substitution, we get radically different answers. In fact, by
examining the first process in an input context,
e.g. $x?(z).\lift{w}{y!(z)}$, we see that the process under the lift
operator may be shaped by prefixed inputs binding a name inside it. In
this sense, the lift operator will be seen as a way to dynamically
construct processes before reifying them as names.

Finally equipped with these standard features we can present the
dynamics of the calculus.

\subsubsection{Operational semantics} 

Finally, we introduce the computational dynamics. What marks these
algebras as distinct from other more traditionally studied algebraic
structures, e.g. vector spaces or polynomial rings, is the manner in
which dynamics is captured. In traditional structures, dynamics is typically
expressed through morphisms between such structures, as in linear maps
between vector spaces or morphisms between rings. In algebras
associated with the semantics of computation, the dynamics is
expressed as part of the algebraic structure itself, through a
reduction reduction relation typically denoted by $\red$. Below, we
give a recursive presentation of this relation for the calculus used
in the encoding.

$\red \subseteq \pi \times \pi$
$\red : \pi \to \mathcal{P}(\pi)$

\begin{mathpar}
  \inferrule* [lab=Comm] { \textsf{match}( x_{src}, x_{trgt} ) } { x_{trgt}?(y)P \; | \; x_{src}!\langle {Q} \rangle \red P\{\quotep{Q}/y}\} }
  \and \\
  \inferrule* [lab=Par] {{P} \red {P}'} {{{P} | {Q}} \red {{P}' | {Q}}}
  \and
  \inferrule* [lab=Equiv]{{{P} \scong {P}'} \andalso {{P}' \red {Q}'} \andalso {{Q}' \scong {Q}}}{{P} \red {Q}}
\end{mathpar}

\begin{eqnarray*}
  match_{\equiv} (\quotep{P},\quotep{Q}) & := & P \equiv Q \\
  match_{\dagger}(\quotep{P},\quotep{Q}) & := & \forall R. P|Q \red^{*} R => R \red^{*} 0 \\
  match_{K}(\quotep{P},\quotep{Q}) & := & K \mbox{ for some context } K
\end{eqnarray*}

$u?(x)P | u!\langle Q \rangle \red P\{\quotep{Q}/x\}$

%We write $\wred$ for $\red^*$, and $P\red$ if $\exists Q $ such that $ P \red Q$.
We write $P\red$ if $\exists Q $ such that $ P \red Q$ and $P\not\red$, otherwise.

\section{Replication}

As mentioned before, it is known that replication (and hence
recursion) can be implemented in a higher-order process algebra
\cite{SangiorgiWalker}. As our first example of calculation with the
machinery thus far presented we give the construction explicitly in
the {\rhoc}.

\begin{eqnarray}
	D_{x} & := & \prefix{x}{y}{(\binpar{\outputp{x}{y}}{@{y}})} \nonumber\\
	\bangp_{x}{P} & := & \binpar{{x}!\langle{\binpar{D_{x}}{P}}\rangle}{D_{x}} \nonumber
\end{eqnarray}

\begin{eqnarray}
	\bangp_{x}{P} & & \nonumber\\
	=
	& {x}!\langle{(\prefix{x}{y}{(\outputp{x}{y} | @{y})) | P}}\rangle 
	      | \prefix{x}{y}{(\outputp{x}{y} | @{y})} & \nonumber\\
	\red
	& (\outputp{x}{y} | @{y})\substn{\quotep{(\prefix{x}{y}{(@{y} | \outputp{x}{y})) | P}}}{y} & \nonumber\\
	=
	& \outputp{x}{\quotep{(\prefix{x}{y}{(\outputp{x}{y} | @{y})) | P}}}
	  | {(\prefix{x}{y}{(\outputp{x}{y} | @{y})) | P}} & \nonumber\\
	\red
	& \ldots & \nonumber\\
	\red^*
	& P | P | \ldots & \nonumber
\end{eqnarray}

Of course, this encoding, as an implementation, runs away, unfolding
$\bangp{P}$ eagerly. A lazier and more implementable replication
operator, restricted to input-guarded processes, may be obtained as follows.

\begin{eqnarray}
\bangp{\prefix{u}{v}{P}} 
	:= 
	\binpar{\lift{x}{\prefix{u}{v}{(\binpar{D(x)}{P})}}}{D(x)} \nonumber
\end{eqnarray}

\begin{remark}
  Note that the lazier definition still does not deal with summation
  or mixed summation (i.e. sums over input and output). The reader is
  invited to construct definitions of replication that deal with these
  features. 

  Further, the definitions are parameterized in a name, $x$. Can you,
  gentle reader, make a definition that eliminates this parameter and
  guarantees no accidental interaction between the replication
  machinery and the process being replicated -- i.e. no accidental
  sharing of names used by the process to get its work done and the
  name(s) used by the replication to effect copying. This latter
  revision of the definition of replication is crucial to obtaining
  the expected identity $!!P \sim !P$.
\end{remark}

\begin{remark}\label{rem:paradoxical_combinator}
  The reader familiar with the lambda calculus will have noticed the
  similarity between $D$ and the paradoxical combinator.

  [Ed. note: the existence of this seems to suggest we have to be more
  restrictive on the set of processes and names we admit if we are to
  support no-cloning.]
\end{remark}

\subsubsection{Bisimulation}

The computational dynamics gives rise to another kind of equivalence,
the equivalence of computational behavior. As previously mentioned
this is typically captured \emph{via} some form of bisimulation.

% The notion we use in this paper is weak barbed bisimulation
% \cite{milner91polyadicpi}.

The notion we use in this paper is derived from weak barbed
bisimulation \cite{milner91polyadicpi}. 

\begin{definition}
An \emph{observation relation}, $\downarrow_{\mathcal N}$, over a set
of names, $\mathcal N$, is the smallest relation satisfying the rules
below.

\infrule[Out-barb]{y \in {\mathcal N}, \; x \nameeq y}
		  {\outputp{x}{v} \downarrow_{\mathcal N} x}
\infrule[Par-barb]{\mbox{$P\downarrow_{\mathcal N} x$ or $Q\downarrow_{\mathcal N} x$}}
		  {\binpar{P}{Q} \downarrow_{\mathcal N} x}

We write $P \Downarrow_{\mathcal N} x$ if there is $Q$ such that 
$P \wred Q$ and $Q \downarrow_{\mathcal N} x$.
\end{definition}

\begin{definition}
%\label{def.bbisim}
An  ${\mathcal N}$-\emph{barbed bisimulation} over a set of names, ${\mathcal N}$, is a symmetric binary relation 
${\mathcal S}_{\mathcal N}$ between agents such that $P\rel{S}_{\mathcal N}Q$ implies:
\begin{enumerate}
\item If $P \red P'$ then $Q \wred Q'$ and $P'\rel{S}_{\mathcal N} Q'$.
\item If $P\downarrow_{\mathcal N} x$, then $Q\Downarrow_{\mathcal N} x$.
\end{enumerate}
$P$ is ${\mathcal N}$-barbed bisimilar to $Q$, written
$P \wbbisim_{\mathcal N} Q$, if $P \rel{S}_{\mathcal N} Q$ for some ${\mathcal N}$-barbed bisimulation ${\mathcal S}_{\mathcal N}$.
\end{definition}

$\mathcal{R} \subseteq \pi \times \pi$

$P \mathcal{R} Q => \forall P'. P \red P' \Rightarrow \exists Q'. Q \red Q', P' \mathcal{R} Q'$

$P \vdash x \Rightarrow Q \vdash x$

\begin{mathpar}
  \inferrule*[lab=Out-barb]{x \nameeq y}{{y}!\langle{Q}\rangle \vdash x}
  \and
  \inferrule*[lab=Par-barb]{\mbox{$P\vdash x$ or $Q\vdash x$}}{\binpar{P}{Q} \vdash x}
\end{mathpar}

\subsubsection{Contexts}

One of the principle advantages of computational calculi like the
$\pi$-calculus is a well-defined notion of context,
contextual-equivalence and a correlation between
contextual-equivalence and notions of bisimulation. The notion of
context allows the decomposition of a process into (sub-)process and
its syntactic environment, its context. Thus, a context may be
thought of as a process with a ``hole'' (written $\Box$) in it. The
application of a context $M$ to a process $P$, written $M[P]$, is
tantamount to filling the hole in $M$ with $P$. In this paper we do
not need the full weight of this theory, but do make use of the notion
of context in the proof the main theorem. 

\begin{mathpar}
  \inferrule* [lab=summation] {} {{M_{M},M_{N}} \bc \Box \;|\; x.M_{A} \;|\; M_{M}+M_{N}}
  \and
  \inferrule* [lab=agent] {} {{M_{A}} \bc (\vec{x})M_{P} \;| \; \clift{P_0,\ldots,M_{P},\ldots,P_N}}
  \and \\
  \inferrule* [lab=process] {} {{M_{P}} \bc M_{N} \;| \;P|M_{P} }
\end{mathpar} 

\begin{mathpar}
  \inferrule* [lab=sychronization] {} {M_{N} \bc \Box \;|\; x?M_{F} \;|\; x!M_{C}}
  \and
  \inferrule* [lab=abstraction] {} {{M_{F}} \bc (x)M_{P} }
  \and
  \inferrule* [lab=concretion] {} {{M_{C}} \bc \langle M_{P} \rangle }
  \and \\
  \inferrule* [lab=process] {} {{M_{P}} \bc M_{N} \;| \;P|M_{P} }
\end{mathpar}

\begin{definition}[contextual application] Given a context $M$, and
  process $P$, we define the \emph{contextual application}, $M[P] :=
  M\{P/\Box\}$. That is, the contextual application of M to P is the
  substitution of $P$ for $\Box$ in $M$.
\end{definition}

$\meaningof{-} : L \to \mathcal{P}(\pi)$

\begin{mathpar}
  \inferrule* [lab=collection] {} {\meaningof{true} = \pi, \and \meaningof{~E} = \pi \setminus \meaningof{E}, \and \meaningof{E_{1} \& E_{2}} = \meaningof{E_{1}} \cap \meaningof{E_{2}}}
\end{mathpar}

\begin{mathpar}
  \inferrule* [lab=structure] {} {\meaningof{0} = \{ P \in \pi | P \equiv 0 \}, \and \\ \meaningof{E_1 | E_2} = \{ P \in \pi | P \equiv P_{1} | P_{2}, P_{1} \in \meaningof{E_{1}}, P_{2} \in \meaningof{E_2}\} }
\end{mathpar}

\begin{mathpar}
 \inferrule* [lab=behavior] {} {\meaningof{\langle a?b \rangle E} = \{ P \in \pi | P \equiv Q | u?(y)P', \\ \and \\\\ \and \\ \;\;\; u \in \meaningof{a}, \forall z.P'\{z/y\} \in \meaningof{E\{z/b\}}\}, \and \\ \meaningof{a!E} = \{ P \in \pi | P \equiv Q | x!\langle P' \rangle, x \in \meaningof{a} P' \in \meaningof{E}\} }
\end{mathpar}

\begin{mathpar}
 \inferrule* [lab=nominal] {} {\meaningof{\quotep{E}} = \{ \quotep{P} \in \quotep{\pi} | P \in \meaningof{E} \}, \and \meaningof{\quotep{P}} = \{ \quotep{Q} \in \quotep{\pi} | P \equiv Q \} \and \\ \meaningof{@\quotep{E}} = \{ P \in \pi | P \equiv @x, x \in \meaningof{E} \}}
\end{mathpar}

\begin{eqnarray*}
  \\
  \meaningof{-} : TS \to ST
\end{eqnarray*}

\begin{eqnarray*}
  \\
  L : TS \to ST
\end{eqnarray*}

\begin{eqnarray*}
  \\
  P \models E \iff P \in \meaningof{E}
\end{eqnarray*}

\begin{eqnarray*}
  P \approx_{L} Q \iff \forall E \in L. P \models E \iff Q \models E
\end{eqnarray*}

\begin{eqnarray*}
  P \approx_{K} Q
\end{eqnarray*}

\begin{eqnarray*}
  P \approx Q
\end{eqnarray*}

$\approx_{K} = \approx = \approx_{L}$

\subsubsection{Contextual duality}

Note that contexts extend the quotation operation to a family of
operations from processes to names. Given a context, $M$, we can
define a \emph{nominal context}, $\quotep{M}$ by $\quotep{M}[P] :=
\quotep{M[P]}$. To foreshadow what is to come we observe that these
operations enjoy a duality with processes very much like the duality
between vectors and maps from vectors to scalars.

Further, because the calculus is essentially higher-order, we have a
correspondence between contexts and processes. More specifically,
given a name $x$ and a context $M$ we can construct $M^{*}_{x}$ such
that 

\begin{mathpar}
  M^{*}_{x} | \lift{x}{P} \red M[P]
\end{mathpar}

namely,

\begin{mathpar}
  M^{*}_{x} := x?(u).M[\dropn{u}]
\end{mathpar}

The dependence of $M^{*}_{x}$ on a name makes it an abstraction, 

\begin{mathpar}
  M^{*} := (x)x?(u).M[\dropn{u}]
\end{mathpar}

\subsection{Additional notation}

It will sometimes be convenient to denote the process a name
quotes. We already have the notation $x = \quotep{P}$, but it will be
convenient to introduce an alternate notation, $\procn{x}$, when we
want to emphasize the connection to the use of the name. Note that, by
virtue of name equivalence, $\quotep{\procn{x}} \nameeq x$; so, the
notation is consistent with previous definitions.

Further, because names have structure it is possible to effect
substitutions on the basis of that structure. This means we need to
upgrade our notation for substitutions, which we accomplish by
adapting comprehension notation. Thus,

\begin{mathpar}
  P\{ y / x : x \in S \}
\end{mathpar}

is interpreted to mean the process derived from P by replacing (in a
capture-avoiding manner) each occurrence of $x$ in $S$ by $y$. For example,

\begin{mathpar}
  P\{ \quotep{\procn{x}|\procn{x}} / x : x \in \freenames{P} \}
\end{mathpar}

will replace each (occurrence) of a free name $x$ in $P$ by
$\quotep{\procn{x}|\procn{x}}$.

Also, we will avail ourselves of the notation $x^{L}$ and $x^{R}$ to
denote injections of a name into disjoint copies of the name
space. There are numerous ways to accomplish this. One example can be
found in \cite{MeredithR05}. This notation overloads to vectors of
names: $\vec{x}^{\pi} := (x_{i}^{\pi} \; : \; 0 \leq i < |\vec{x}| )$ where $\pi \in \{L,R\}$.

We also use $P^{\Box} := P|\Box$.

In \cite{MeredithR05} an interpretation of the new operator is
given. It turns out that there are several possible interpretations
all enjoying the requisite algebraic properties of the operator (see
\cite{milner91polyadicpi}). We will therefore make liberal use of
$(\nu\; \vec{x})P$.

% subsection the_syntax_and_semantics_of_the_notation_system (end)   

\input{qm2pi.qmops} 

\input{qm2pi.sterngerlach} 

\input{qm2pi.metric} 

% section concurrent_process_calculi (end)

%\input{qm2pi.proofsketch}

% section proof sketch (end)

%\input{qm2pi.slviaknots} 

% section spatial logic via knots (end)

\input{qm2pi.conclusion}

% section conclusion (end)

%\input{qm2pi.dtcodes} 

% section wiring algorithm (end)

\input{qm2pi.ack} 

% section acknowledgments (end)

\newpage


\bibliographystyle{plain}   
\bibliography{../../biblios/main.bib}

\input{qm2pi.rhodetails}

\end{document}



\end{document}

 

% section acknowledgments (end)

\newpage


\bibliographystyle{plain}   
\bibliography{../../biblios/main.bib}

\documentclass[12pt]{llncs}
%\documentclass{jktr}

\usepackage[pdftex]{hyperref}                   
\usepackage {listings}
\usepackage {mathpartir}
\usepackage{bcprules}
%\usepackage{listings}
                       
\usepackage{graphicx} 
%\usepackage[margins=2.5cm,nohead,nofoot]{geometry}
%\usepackage{geometry}
\usepackage{amsfonts}
\usepackage{amstext}
\usepackage{latexsym}
\usepackage{amssymb}
\usepackage{color}


%\include{myPreamble}
\documentclass[12pt]{llncs}
%\documentclass{jktr}

\usepackage[pdftex]{hyperref}                   
\usepackage {listings}
\usepackage {mathpartir}
\usepackage{bcprules}
%\usepackage{listings}
                       
\usepackage{graphicx} 
%\usepackage[margins=2.5cm,nohead,nofoot]{geometry}
%\usepackage{geometry}
\usepackage{amsfonts}
\usepackage{amstext}
\usepackage{latexsym}
\usepackage{amssymb}
\usepackage{color}


%\include{myPreamble}
\include{qm2pi.local} 

%\ifpdf
%\usepackage[pdftex]{graphicx}
%\else
%\usepackage{graphicx}
%\fi

 % \ifpdf
%  \usepackage{pdfsync}
%  \if


%\title{Brief Article}
%\author{David F. Snyder}
%\author{L.G. Meredith}

%\address{Dept. of Math., Texas State University--San Marcos, San Marcos, TX 78666}
       
\pagestyle{empty}


\begin{document}

\lstset{language=[Objective]Caml,frame=shadowbox}

\input{qm2pi.front}

% section front matter (end)

\input{qm2pi.intro} 
 
% section introduction (end)

% \input{qm2pi.knotations} 

% section notation (end)

\input{qm2pi.process.calculi} 

% section concurrent_process_calculi_and_spatial_logics_ (end)
    
%\input{qm2pi.knots2pi} 

%\input{qm2pi.trefoil} 

%\input{qm2pi.mainthm} 

% subsection basic_interpretation (end)

%\input{qm2pi.rho.presentation} 
\subsection{The syntax and semantics of the notation system}\label{sub:the_syntax_and_semantics_of_the_notation_system} % (fold)

We now summarize a technical presentation of the calculus that
embodies our theory of dynamics. The typical presentation of such a
calculus follows the style of giving generators and relations on
them. The grammar, below, describing term constructors, freely
generates the set of processes, $\Proc$. This set is then quotiented
by a relation known as structural congruence and it is over this set
that the notion of dynamics is expressed. This presentation is
essentially that of \cite{MeredithR05} with the addition of
polyadicity and summation. For readability we have relegated some of
the technical subtleties to an appendix.

\subsubsection{Process grammar}\label{subsub:process_grammar}

\begin{mathpar}
  \inferrule* [lab=synchronization] {} {{M} \bc \pzero \;|\; x?F \;|\; x!C }
  \and
  \inferrule* [lab=abstraction] {} {{F} \bc (x)P}
  \and
  \inferrule* [lab=concretion] {} {{C} \bc \langle Q \rangle}
  \and
  \inferrule* [lab=process] {} {{P,Q} \bc M \;| \;P|Q \;|\; @{x}}
  \and
  \inferrule* [lab=name] {} {{x} \bc \quotep{P}}
\end{mathpar} 

Note that $\vec{x}$ (resp. $\vec{P}$) denotes a vector of names
(resp. processes) of length $|\vec{x}|$ (resp. $|\vec{P}|$). We adopt
the following useful abbreviations.

\begin{mathpar}
   x?(\vec{y}).P := x.(\vec{y})P \and  x\clift{\vec{P}} := x.\clift{\vec{P}}
   \and x!(y) := \lift{x}{\dropn{y}}
   \and \Pi_{i=0}^{n-1}P_i := P_0 | \ldots | P_{n-1}
\end{mathpar}

\subsubsection{Structural congruence}

\paragraph{Free and bound names and alpha-equivalence.} At the
core of structural equivalence is alpha-equivalence which identifies
process that are the same up to a change of variable. Formally, we
recognize the distinction between free and bound names. The free names
of a process, $\freenames{P}$, may be calculated recursively as
follows:

\begin{mathpar}
\freenames{\pzero} := \emptyset
  \and \\
  \freenames{x?(y).P} := \{ x \} \cup (\freenames{P} \setminus \{ y \})
  \and 
  \freenames{x!\langle P \rangle} := \{ x \} \cup \{ P \} 
  \and \\
  \freenames{P|Q} := \freenames{P} \cup \freenames{Q}
  \and \\
  \freenames{@{x}} := \{ x \}
\end{mathpar}

$\pi$
$\quotep{\pi}$

$\freenames{-} : \pi \to \mathcal{P}(\quotep{\pi})$

\begin{eqnarray*}
  \freenames{\pzero} & := & \emptyset \\
  \freenames{x?(y).P} & := & \{ x \} \cup (\freenames{P} \setminus \{ y \}) \\
  \freenames{x!\langle P \rangle} & := & \{ x \} \cup \{ P \} \\
  \freenames{P|Q} & := & \freenames{P} \cup \freenames{Q} \\
  \freenames{\dropn{x}} & := & \{ x \}
\end{eqnarray*}

The bound names of a process, $\boundnames{P}$, are those names occurring in $P$
that are not free. For example, in $x?(y).0$, the name $x$ is free, while $y$ is bound.

\begin{mathpar}
  \inferrule* [lab=monoidal-laws] {} { P|Q \equiv Q|P \and P|0 \equiv P \and P|(Q|R) \equiv (P|Q)|R }
\end{mathpar}

\begin{mathpar}
  \inferrule* [lab=alpha-equivalence] {} { (x)P \equiv (y)P\{y/x\} \and y \not\in \freenames{P} }
\end{mathpar}

\begin{definition}
Then two processes, $P,Q$, are alpha-equivalent if $P = Q\{\vec{y}/\vec{x}\}$ for
some $\vec{x} \in \boundnames{Q},\vec{y} \in \boundnames{P}$, where $Q\{\vec{y}/\vec{x}\}$
denotes the capture-avoiding substitution of $\vec{y}$ for $\vec{x}$ in $Q$.
\end{definition}

\begin{definition}
  The {\em structural congruence} \cite{SangiorgiWalker} , $\equiv$,
  between processes is the least congruence containing
  alpha-equivalence, satisfying the abelian monoid laws
  (associativity, commutativity and $\pzero$ as identity) for parallel
  composition $|$ and for summation $+$.
\end{definition}

\subsection{Name equivalence}

We take name equivalence, written $\nameeq$, to be the smallest
equivalence relation generated by the following rules.

\begin{mathpar}
\inferrule*[lab=Quote-drop]
{ }
{ \quotep{@{x}} \nameeq x }

\inferrule*[lab=Struct-equiv]
{ P \scong Q }
{ \quotep{P} \nameeq \quotep{Q} }
\end{mathpar}

The astute reader will have noticed that the mutual recursion of names
and processes imposes a mutual recursion on alpha-equivalence and
structural equivalence via name-equivalence. Fortunately, all of this
works out pleasantly and we may calculate in the natural way, free of
concern. The reader interested in the details is referred to the
appendix \ref{appendix:rho_details}.

\subsection{Substitution}

We use $\Proc$ for the set of processes, $\QProc$ for the set of
names, and $\id{\{}\vec{y} / \vec{x} \id{\}}$ to denote partial maps,
$s : \QProc \rightarrow \QProc$. A map, $s$ lifts, uniquely, to a map
on process terms, $\widehat{s} : \Proc \rightarrow \Proc$ by the
following equations.

\begin{mathpar}
  (0) \psubstp{Q}{P} := 0 \\
  (R \juxtap S) \psubstp{Q}{P}
  :=    
  (R)\psubstp{Q}{P} \juxtap (S) \psubstp{Q}{P} \\
  (x?(y).R) \psubstp{Q}{P}    
  :=    
  (x)\substp{Q}{P} (z)\concat( (R \psubstn{z}{y}) \psubstp{Q}{P} ) \\
  (\lift{x}{R}) \psubstp{Q}{P}  
  :=
  \lift{(x)\substp{Q}{P}}{ R \psubstp{Q}{P} } \\
%   (\dropn{x})  \psubstp{Q}{P}       
%   := 
%   \left\{ 
%     \begin{array}{ccc} 
%       \dropn{\quotep{Q}} & & x \nameeq \quotep{P} \\
%       \dropn{x} & & otherwise \\
%     \end{array}
%   \right. 
  (\dropn{x})  \psubstp{Q}{P}       
  := 
  \left\{ 
    \begin{array}{ccc} 
      Q & & x \nameeq \quotep{P} \\
      \dropn{x} & & otherwise \\
    \end{array}
  \right.
\end{mathpar}
 

where

\begin{eqnarray}
  (x)\id{\{} \lpquote Q \rpquote / \lpquote P \rpquote \id{\}}            = 
  \left\{ 
    \begin{array}{ccc}
      \lpquote Q \rpquote & & x \nameeq \lpquote P \rpquote \\
      x & & otherwise \\
    \end{array}
  \right. \nonumber
\end{eqnarray}

and $z$ is chosen distinct from $\quotep{P}$, $\quotep{Q}$, the free
names in $Q$, and all the names in $R$. Our $\alpha$-equivalence will
be built in the standard way from this substitution.

\begin{remark}\label{rem:no_self_referential_names}
  One consequence of these definitions is that $\forall P. \quotep{P}
  \not\in \freenames{P}$.
\end{remark}

\subsection{ Dynamic quote: an example }

Anticipating something of what's to come, consider applying the
substitution, $\widehat{\id{\{}u / z \id{\}}}$, to the following pair
of processes, $\lift{w}{y!(z)}$ and $w[ \lpquote y!(z) \rpquote ]$.

\begin{eqnarray}
	\lift{w}{y!(z)}\widehat{\id{\{}u / z \id{\}}}
		& = &
		\lift{w}{y!(u)} \nonumber\\
	w[ \lpquote y!(z) \rpquote ] \widehat{ \id{\{}u / z \id{\}} }
		& = &
		w[ \lpquote y!(z) \rpquote ] \nonumber
\end{eqnarray}

Because the body of the process between quotes is impervious to
substitution, we get radically different answers. In fact, by
examining the first process in an input context,
e.g. $x?(z).\lift{w}{y!(z)}$, we see that the process under the lift
operator may be shaped by prefixed inputs binding a name inside it. In
this sense, the lift operator will be seen as a way to dynamically
construct processes before reifying them as names.

Finally equipped with these standard features we can present the
dynamics of the calculus.

\subsubsection{Operational semantics} 

Finally, we introduce the computational dynamics. What marks these
algebras as distinct from other more traditionally studied algebraic
structures, e.g. vector spaces or polynomial rings, is the manner in
which dynamics is captured. In traditional structures, dynamics is typically
expressed through morphisms between such structures, as in linear maps
between vector spaces or morphisms between rings. In algebras
associated with the semantics of computation, the dynamics is
expressed as part of the algebraic structure itself, through a
reduction reduction relation typically denoted by $\red$. Below, we
give a recursive presentation of this relation for the calculus used
in the encoding.

$\red \subseteq \pi \times \pi$
$\red : \pi \to \mathcal{P}(\pi)$

\begin{mathpar}
  \inferrule* [lab=Comm] { \textsf{match}( x_{src}, x_{trgt} ) } { x_{trgt}?(y)P \; | \; x_{src}!\langle {Q} \rangle \red P\{\quotep{Q}/y}\} }
  \and \\
  \inferrule* [lab=Par] {{P} \red {P}'} {{{P} | {Q}} \red {{P}' | {Q}}}
  \and
  \inferrule* [lab=Equiv]{{{P} \scong {P}'} \andalso {{P}' \red {Q}'} \andalso {{Q}' \scong {Q}}}{{P} \red {Q}}
\end{mathpar}

\begin{eqnarray*}
  match_{\equiv} (\quotep{P},\quotep{Q}) & := & P \equiv Q \\
  match_{\dagger}(\quotep{P},\quotep{Q}) & := & \forall R. P|Q \red^{*} R => R \red^{*} 0 \\
  match_{K}(\quotep{P},\quotep{Q}) & := & K \mbox{ for some context } K
\end{eqnarray*}

$u?(x)P | u!\langle Q \rangle \red P\{\quotep{Q}/x\}$

%We write $\wred$ for $\red^*$, and $P\red$ if $\exists Q $ such that $ P \red Q$.
We write $P\red$ if $\exists Q $ such that $ P \red Q$ and $P\not\red$, otherwise.

\section{Replication}

As mentioned before, it is known that replication (and hence
recursion) can be implemented in a higher-order process algebra
\cite{SangiorgiWalker}. As our first example of calculation with the
machinery thus far presented we give the construction explicitly in
the {\rhoc}.

\begin{eqnarray}
	D_{x} & := & \prefix{x}{y}{(\binpar{\outputp{x}{y}}{@{y}})} \nonumber\\
	\bangp_{x}{P} & := & \binpar{{x}!\langle{\binpar{D_{x}}{P}}\rangle}{D_{x}} \nonumber
\end{eqnarray}

\begin{eqnarray}
	\bangp_{x}{P} & & \nonumber\\
	=
	& {x}!\langle{(\prefix{x}{y}{(\outputp{x}{y} | @{y})) | P}}\rangle 
	      | \prefix{x}{y}{(\outputp{x}{y} | @{y})} & \nonumber\\
	\red
	& (\outputp{x}{y} | @{y})\substn{\quotep{(\prefix{x}{y}{(@{y} | \outputp{x}{y})) | P}}}{y} & \nonumber\\
	=
	& \outputp{x}{\quotep{(\prefix{x}{y}{(\outputp{x}{y} | @{y})) | P}}}
	  | {(\prefix{x}{y}{(\outputp{x}{y} | @{y})) | P}} & \nonumber\\
	\red
	& \ldots & \nonumber\\
	\red^*
	& P | P | \ldots & \nonumber
\end{eqnarray}

Of course, this encoding, as an implementation, runs away, unfolding
$\bangp{P}$ eagerly. A lazier and more implementable replication
operator, restricted to input-guarded processes, may be obtained as follows.

\begin{eqnarray}
\bangp{\prefix{u}{v}{P}} 
	:= 
	\binpar{\lift{x}{\prefix{u}{v}{(\binpar{D(x)}{P})}}}{D(x)} \nonumber
\end{eqnarray}

\begin{remark}
  Note that the lazier definition still does not deal with summation
  or mixed summation (i.e. sums over input and output). The reader is
  invited to construct definitions of replication that deal with these
  features. 

  Further, the definitions are parameterized in a name, $x$. Can you,
  gentle reader, make a definition that eliminates this parameter and
  guarantees no accidental interaction between the replication
  machinery and the process being replicated -- i.e. no accidental
  sharing of names used by the process to get its work done and the
  name(s) used by the replication to effect copying. This latter
  revision of the definition of replication is crucial to obtaining
  the expected identity $!!P \sim !P$.
\end{remark}

\begin{remark}\label{rem:paradoxical_combinator}
  The reader familiar with the lambda calculus will have noticed the
  similarity between $D$ and the paradoxical combinator.

  [Ed. note: the existence of this seems to suggest we have to be more
  restrictive on the set of processes and names we admit if we are to
  support no-cloning.]
\end{remark}

\subsubsection{Bisimulation}

The computational dynamics gives rise to another kind of equivalence,
the equivalence of computational behavior. As previously mentioned
this is typically captured \emph{via} some form of bisimulation.

% The notion we use in this paper is weak barbed bisimulation
% \cite{milner91polyadicpi}.

The notion we use in this paper is derived from weak barbed
bisimulation \cite{milner91polyadicpi}. 

\begin{definition}
An \emph{observation relation}, $\downarrow_{\mathcal N}$, over a set
of names, $\mathcal N$, is the smallest relation satisfying the rules
below.

\infrule[Out-barb]{y \in {\mathcal N}, \; x \nameeq y}
		  {\outputp{x}{v} \downarrow_{\mathcal N} x}
\infrule[Par-barb]{\mbox{$P\downarrow_{\mathcal N} x$ or $Q\downarrow_{\mathcal N} x$}}
		  {\binpar{P}{Q} \downarrow_{\mathcal N} x}

We write $P \Downarrow_{\mathcal N} x$ if there is $Q$ such that 
$P \wred Q$ and $Q \downarrow_{\mathcal N} x$.
\end{definition}

\begin{definition}
%\label{def.bbisim}
An  ${\mathcal N}$-\emph{barbed bisimulation} over a set of names, ${\mathcal N}$, is a symmetric binary relation 
${\mathcal S}_{\mathcal N}$ between agents such that $P\rel{S}_{\mathcal N}Q$ implies:
\begin{enumerate}
\item If $P \red P'$ then $Q \wred Q'$ and $P'\rel{S}_{\mathcal N} Q'$.
\item If $P\downarrow_{\mathcal N} x$, then $Q\Downarrow_{\mathcal N} x$.
\end{enumerate}
$P$ is ${\mathcal N}$-barbed bisimilar to $Q$, written
$P \wbbisim_{\mathcal N} Q$, if $P \rel{S}_{\mathcal N} Q$ for some ${\mathcal N}$-barbed bisimulation ${\mathcal S}_{\mathcal N}$.
\end{definition}

$\mathcal{R} \subseteq \pi \times \pi$

$P \mathcal{R} Q => \forall P'. P \red P' \Rightarrow \exists Q'. Q \red Q', P' \mathcal{R} Q'$

$P \vdash x \Rightarrow Q \vdash x$

\begin{mathpar}
  \inferrule*[lab=Out-barb]{x \nameeq y}{{y}!\langle{Q}\rangle \vdash x}
  \and
  \inferrule*[lab=Par-barb]{\mbox{$P\vdash x$ or $Q\vdash x$}}{\binpar{P}{Q} \vdash x}
\end{mathpar}

\subsubsection{Contexts}

One of the principle advantages of computational calculi like the
$\pi$-calculus is a well-defined notion of context,
contextual-equivalence and a correlation between
contextual-equivalence and notions of bisimulation. The notion of
context allows the decomposition of a process into (sub-)process and
its syntactic environment, its context. Thus, a context may be
thought of as a process with a ``hole'' (written $\Box$) in it. The
application of a context $M$ to a process $P$, written $M[P]$, is
tantamount to filling the hole in $M$ with $P$. In this paper we do
not need the full weight of this theory, but do make use of the notion
of context in the proof the main theorem. 

\begin{mathpar}
  \inferrule* [lab=summation] {} {{M_{M},M_{N}} \bc \Box \;|\; x.M_{A} \;|\; M_{M}+M_{N}}
  \and
  \inferrule* [lab=agent] {} {{M_{A}} \bc (\vec{x})M_{P} \;| \; \clift{P_0,\ldots,M_{P},\ldots,P_N}}
  \and \\
  \inferrule* [lab=process] {} {{M_{P}} \bc M_{N} \;| \;P|M_{P} }
\end{mathpar} 

\begin{mathpar}
  \inferrule* [lab=sychronization] {} {M_{N} \bc \Box \;|\; x?M_{F} \;|\; x!M_{C}}
  \and
  \inferrule* [lab=abstraction] {} {{M_{F}} \bc (x)M_{P} }
  \and
  \inferrule* [lab=concretion] {} {{M_{C}} \bc \langle M_{P} \rangle }
  \and \\
  \inferrule* [lab=process] {} {{M_{P}} \bc M_{N} \;| \;P|M_{P} }
\end{mathpar}

\begin{definition}[contextual application] Given a context $M$, and
  process $P$, we define the \emph{contextual application}, $M[P] :=
  M\{P/\Box\}$. That is, the contextual application of M to P is the
  substitution of $P$ for $\Box$ in $M$.
\end{definition}

$\meaningof{-} : L \to \mathcal{P}(\pi)$

\begin{mathpar}
  \inferrule* [lab=collection] {} {\meaningof{true} = \pi, \and \meaningof{~E} = \pi \setminus \meaningof{E}, \and \meaningof{E_{1} \& E_{2}} = \meaningof{E_{1}} \cap \meaningof{E_{2}}}
\end{mathpar}

\begin{mathpar}
  \inferrule* [lab=structure] {} {\meaningof{0} = \{ P \in \pi | P \equiv 0 \}, \and \\ \meaningof{E_1 | E_2} = \{ P \in \pi | P \equiv P_{1} | P_{2}, P_{1} \in \meaningof{E_{1}}, P_{2} \in \meaningof{E_2}\} }
\end{mathpar}

\begin{mathpar}
 \inferrule* [lab=behavior] {} {\meaningof{\langle a?b \rangle E} = \{ P \in \pi | P \equiv Q | u?(y)P', \\ \and \\\\ \and \\ \;\;\; u \in \meaningof{a}, \forall z.P'\{z/y\} \in \meaningof{E\{z/b\}}\}, \and \\ \meaningof{a!E} = \{ P \in \pi | P \equiv Q | x!\langle P' \rangle, x \in \meaningof{a} P' \in \meaningof{E}\} }
\end{mathpar}

\begin{mathpar}
 \inferrule* [lab=nominal] {} {\meaningof{\quotep{E}} = \{ \quotep{P} \in \quotep{\pi} | P \in \meaningof{E} \}, \and \meaningof{\quotep{P}} = \{ \quotep{Q} \in \quotep{\pi} | P \equiv Q \} \and \\ \meaningof{@\quotep{E}} = \{ P \in \pi | P \equiv @x, x \in \meaningof{E} \}}
\end{mathpar}

\begin{eqnarray*}
  \\
  \meaningof{-} : TS \to ST
\end{eqnarray*}

\begin{eqnarray*}
  \\
  L : TS \to ST
\end{eqnarray*}

\begin{eqnarray*}
  \\
  P \models E \iff P \in \meaningof{E}
\end{eqnarray*}

\begin{eqnarray*}
  P \approx_{L} Q \iff \forall E \in L. P \models E \iff Q \models E
\end{eqnarray*}

\begin{eqnarray*}
  P \approx_{K} Q
\end{eqnarray*}

\begin{eqnarray*}
  P \approx Q
\end{eqnarray*}

$\approx_{K} = \approx = \approx_{L}$

\subsubsection{Contextual duality}

Note that contexts extend the quotation operation to a family of
operations from processes to names. Given a context, $M$, we can
define a \emph{nominal context}, $\quotep{M}$ by $\quotep{M}[P] :=
\quotep{M[P]}$. To foreshadow what is to come we observe that these
operations enjoy a duality with processes very much like the duality
between vectors and maps from vectors to scalars.

Further, because the calculus is essentially higher-order, we have a
correspondence between contexts and processes. More specifically,
given a name $x$ and a context $M$ we can construct $M^{*}_{x}$ such
that 

\begin{mathpar}
  M^{*}_{x} | \lift{x}{P} \red M[P]
\end{mathpar}

namely,

\begin{mathpar}
  M^{*}_{x} := x?(u).M[\dropn{u}]
\end{mathpar}

The dependence of $M^{*}_{x}$ on a name makes it an abstraction, 

\begin{mathpar}
  M^{*} := (x)x?(u).M[\dropn{u}]
\end{mathpar}

\subsection{Additional notation}

It will sometimes be convenient to denote the process a name
quotes. We already have the notation $x = \quotep{P}$, but it will be
convenient to introduce an alternate notation, $\procn{x}$, when we
want to emphasize the connection to the use of the name. Note that, by
virtue of name equivalence, $\quotep{\procn{x}} \nameeq x$; so, the
notation is consistent with previous definitions.

Further, because names have structure it is possible to effect
substitutions on the basis of that structure. This means we need to
upgrade our notation for substitutions, which we accomplish by
adapting comprehension notation. Thus,

\begin{mathpar}
  P\{ y / x : x \in S \}
\end{mathpar}

is interpreted to mean the process derived from P by replacing (in a
capture-avoiding manner) each occurrence of $x$ in $S$ by $y$. For example,

\begin{mathpar}
  P\{ \quotep{\procn{x}|\procn{x}} / x : x \in \freenames{P} \}
\end{mathpar}

will replace each (occurrence) of a free name $x$ in $P$ by
$\quotep{\procn{x}|\procn{x}}$.

Also, we will avail ourselves of the notation $x^{L}$ and $x^{R}$ to
denote injections of a name into disjoint copies of the name
space. There are numerous ways to accomplish this. One example can be
found in \cite{MeredithR05}. This notation overloads to vectors of
names: $\vec{x}^{\pi} := (x_{i}^{\pi} \; : \; 0 \leq i < |\vec{x}| )$ where $\pi \in \{L,R\}$.

We also use $P^{\Box} := P|\Box$.

In \cite{MeredithR05} an interpretation of the new operator is
given. It turns out that there are several possible interpretations
all enjoying the requisite algebraic properties of the operator (see
\cite{milner91polyadicpi}). We will therefore make liberal use of
$(\nu\; \vec{x})P$.

% subsection the_syntax_and_semantics_of_the_notation_system (end)   

\input{qm2pi.qmops} 

\input{qm2pi.sterngerlach} 

\input{qm2pi.metric} 

% section concurrent_process_calculi (end)

%\input{qm2pi.proofsketch}

% section proof sketch (end)

%\input{qm2pi.slviaknots} 

% section spatial logic via knots (end)

\input{qm2pi.conclusion}

% section conclusion (end)

%\input{qm2pi.dtcodes} 

% section wiring algorithm (end)

\input{qm2pi.ack} 

% section acknowledgments (end)

\newpage


\bibliographystyle{plain}   
\bibliography{../../biblios/main.bib}

\input{qm2pi.rhodetails}

\end{document}

 

%\ifpdf
%\usepackage[pdftex]{graphicx}
%\else
%\usepackage{graphicx}
%\fi

 % \ifpdf
%  \usepackage{pdfsync}
%  \if


%\title{Brief Article}
%\author{David F. Snyder}
%\author{L.G. Meredith}

%\address{Dept. of Math., Texas State University--San Marcos, San Marcos, TX 78666}
       
\pagestyle{empty}


\begin{document}

\lstset{language=[Objective]Caml,frame=shadowbox}

\documentclass[12pt]{llncs}
%\documentclass{jktr}

\usepackage[pdftex]{hyperref}                   
\usepackage {listings}
\usepackage {mathpartir}
\usepackage{bcprules}
%\usepackage{listings}
                       
\usepackage{graphicx} 
%\usepackage[margins=2.5cm,nohead,nofoot]{geometry}
%\usepackage{geometry}
\usepackage{amsfonts}
\usepackage{amstext}
\usepackage{latexsym}
\usepackage{amssymb}
\usepackage{color}


%\include{myPreamble}
\include{qm2pi.local} 

%\ifpdf
%\usepackage[pdftex]{graphicx}
%\else
%\usepackage{graphicx}
%\fi

 % \ifpdf
%  \usepackage{pdfsync}
%  \if


%\title{Brief Article}
%\author{David F. Snyder}
%\author{L.G. Meredith}

%\address{Dept. of Math., Texas State University--San Marcos, San Marcos, TX 78666}
       
\pagestyle{empty}


\begin{document}

\lstset{language=[Objective]Caml,frame=shadowbox}

\input{qm2pi.front}

% section front matter (end)

\input{qm2pi.intro} 
 
% section introduction (end)

% \input{qm2pi.knotations} 

% section notation (end)

\input{qm2pi.process.calculi} 

% section concurrent_process_calculi_and_spatial_logics_ (end)
    
%\input{qm2pi.knots2pi} 

%\input{qm2pi.trefoil} 

%\input{qm2pi.mainthm} 

% subsection basic_interpretation (end)

%\input{qm2pi.rho.presentation} 
\subsection{The syntax and semantics of the notation system}\label{sub:the_syntax_and_semantics_of_the_notation_system} % (fold)

We now summarize a technical presentation of the calculus that
embodies our theory of dynamics. The typical presentation of such a
calculus follows the style of giving generators and relations on
them. The grammar, below, describing term constructors, freely
generates the set of processes, $\Proc$. This set is then quotiented
by a relation known as structural congruence and it is over this set
that the notion of dynamics is expressed. This presentation is
essentially that of \cite{MeredithR05} with the addition of
polyadicity and summation. For readability we have relegated some of
the technical subtleties to an appendix.

\subsubsection{Process grammar}\label{subsub:process_grammar}

\begin{mathpar}
  \inferrule* [lab=synchronization] {} {{M} \bc \pzero \;|\; x?F \;|\; x!C }
  \and
  \inferrule* [lab=abstraction] {} {{F} \bc (x)P}
  \and
  \inferrule* [lab=concretion] {} {{C} \bc \langle Q \rangle}
  \and
  \inferrule* [lab=process] {} {{P,Q} \bc M \;| \;P|Q \;|\; @{x}}
  \and
  \inferrule* [lab=name] {} {{x} \bc \quotep{P}}
\end{mathpar} 

Note that $\vec{x}$ (resp. $\vec{P}$) denotes a vector of names
(resp. processes) of length $|\vec{x}|$ (resp. $|\vec{P}|$). We adopt
the following useful abbreviations.

\begin{mathpar}
   x?(\vec{y}).P := x.(\vec{y})P \and  x\clift{\vec{P}} := x.\clift{\vec{P}}
   \and x!(y) := \lift{x}{\dropn{y}}
   \and \Pi_{i=0}^{n-1}P_i := P_0 | \ldots | P_{n-1}
\end{mathpar}

\subsubsection{Structural congruence}

\paragraph{Free and bound names and alpha-equivalence.} At the
core of structural equivalence is alpha-equivalence which identifies
process that are the same up to a change of variable. Formally, we
recognize the distinction between free and bound names. The free names
of a process, $\freenames{P}$, may be calculated recursively as
follows:

\begin{mathpar}
\freenames{\pzero} := \emptyset
  \and \\
  \freenames{x?(y).P} := \{ x \} \cup (\freenames{P} \setminus \{ y \})
  \and 
  \freenames{x!\langle P \rangle} := \{ x \} \cup \{ P \} 
  \and \\
  \freenames{P|Q} := \freenames{P} \cup \freenames{Q}
  \and \\
  \freenames{@{x}} := \{ x \}
\end{mathpar}

$\pi$
$\quotep{\pi}$

$\freenames{-} : \pi \to \mathcal{P}(\quotep{\pi})$

\begin{eqnarray*}
  \freenames{\pzero} & := & \emptyset \\
  \freenames{x?(y).P} & := & \{ x \} \cup (\freenames{P} \setminus \{ y \}) \\
  \freenames{x!\langle P \rangle} & := & \{ x \} \cup \{ P \} \\
  \freenames{P|Q} & := & \freenames{P} \cup \freenames{Q} \\
  \freenames{\dropn{x}} & := & \{ x \}
\end{eqnarray*}

The bound names of a process, $\boundnames{P}$, are those names occurring in $P$
that are not free. For example, in $x?(y).0$, the name $x$ is free, while $y$ is bound.

\begin{mathpar}
  \inferrule* [lab=monoidal-laws] {} { P|Q \equiv Q|P \and P|0 \equiv P \and P|(Q|R) \equiv (P|Q)|R }
\end{mathpar}

\begin{mathpar}
  \inferrule* [lab=alpha-equivalence] {} { (x)P \equiv (y)P\{y/x\} \and y \not\in \freenames{P} }
\end{mathpar}

\begin{definition}
Then two processes, $P,Q$, are alpha-equivalent if $P = Q\{\vec{y}/\vec{x}\}$ for
some $\vec{x} \in \boundnames{Q},\vec{y} \in \boundnames{P}$, where $Q\{\vec{y}/\vec{x}\}$
denotes the capture-avoiding substitution of $\vec{y}$ for $\vec{x}$ in $Q$.
\end{definition}

\begin{definition}
  The {\em structural congruence} \cite{SangiorgiWalker} , $\equiv$,
  between processes is the least congruence containing
  alpha-equivalence, satisfying the abelian monoid laws
  (associativity, commutativity and $\pzero$ as identity) for parallel
  composition $|$ and for summation $+$.
\end{definition}

\subsection{Name equivalence}

We take name equivalence, written $\nameeq$, to be the smallest
equivalence relation generated by the following rules.

\begin{mathpar}
\inferrule*[lab=Quote-drop]
{ }
{ \quotep{@{x}} \nameeq x }

\inferrule*[lab=Struct-equiv]
{ P \scong Q }
{ \quotep{P} \nameeq \quotep{Q} }
\end{mathpar}

The astute reader will have noticed that the mutual recursion of names
and processes imposes a mutual recursion on alpha-equivalence and
structural equivalence via name-equivalence. Fortunately, all of this
works out pleasantly and we may calculate in the natural way, free of
concern. The reader interested in the details is referred to the
appendix \ref{appendix:rho_details}.

\subsection{Substitution}

We use $\Proc$ for the set of processes, $\QProc$ for the set of
names, and $\id{\{}\vec{y} / \vec{x} \id{\}}$ to denote partial maps,
$s : \QProc \rightarrow \QProc$. A map, $s$ lifts, uniquely, to a map
on process terms, $\widehat{s} : \Proc \rightarrow \Proc$ by the
following equations.

\begin{mathpar}
  (0) \psubstp{Q}{P} := 0 \\
  (R \juxtap S) \psubstp{Q}{P}
  :=    
  (R)\psubstp{Q}{P} \juxtap (S) \psubstp{Q}{P} \\
  (x?(y).R) \psubstp{Q}{P}    
  :=    
  (x)\substp{Q}{P} (z)\concat( (R \psubstn{z}{y}) \psubstp{Q}{P} ) \\
  (\lift{x}{R}) \psubstp{Q}{P}  
  :=
  \lift{(x)\substp{Q}{P}}{ R \psubstp{Q}{P} } \\
%   (\dropn{x})  \psubstp{Q}{P}       
%   := 
%   \left\{ 
%     \begin{array}{ccc} 
%       \dropn{\quotep{Q}} & & x \nameeq \quotep{P} \\
%       \dropn{x} & & otherwise \\
%     \end{array}
%   \right. 
  (\dropn{x})  \psubstp{Q}{P}       
  := 
  \left\{ 
    \begin{array}{ccc} 
      Q & & x \nameeq \quotep{P} \\
      \dropn{x} & & otherwise \\
    \end{array}
  \right.
\end{mathpar}
 

where

\begin{eqnarray}
  (x)\id{\{} \lpquote Q \rpquote / \lpquote P \rpquote \id{\}}            = 
  \left\{ 
    \begin{array}{ccc}
      \lpquote Q \rpquote & & x \nameeq \lpquote P \rpquote \\
      x & & otherwise \\
    \end{array}
  \right. \nonumber
\end{eqnarray}

and $z$ is chosen distinct from $\quotep{P}$, $\quotep{Q}$, the free
names in $Q$, and all the names in $R$. Our $\alpha$-equivalence will
be built in the standard way from this substitution.

\begin{remark}\label{rem:no_self_referential_names}
  One consequence of these definitions is that $\forall P. \quotep{P}
  \not\in \freenames{P}$.
\end{remark}

\subsection{ Dynamic quote: an example }

Anticipating something of what's to come, consider applying the
substitution, $\widehat{\id{\{}u / z \id{\}}}$, to the following pair
of processes, $\lift{w}{y!(z)}$ and $w[ \lpquote y!(z) \rpquote ]$.

\begin{eqnarray}
	\lift{w}{y!(z)}\widehat{\id{\{}u / z \id{\}}}
		& = &
		\lift{w}{y!(u)} \nonumber\\
	w[ \lpquote y!(z) \rpquote ] \widehat{ \id{\{}u / z \id{\}} }
		& = &
		w[ \lpquote y!(z) \rpquote ] \nonumber
\end{eqnarray}

Because the body of the process between quotes is impervious to
substitution, we get radically different answers. In fact, by
examining the first process in an input context,
e.g. $x?(z).\lift{w}{y!(z)}$, we see that the process under the lift
operator may be shaped by prefixed inputs binding a name inside it. In
this sense, the lift operator will be seen as a way to dynamically
construct processes before reifying them as names.

Finally equipped with these standard features we can present the
dynamics of the calculus.

\subsubsection{Operational semantics} 

Finally, we introduce the computational dynamics. What marks these
algebras as distinct from other more traditionally studied algebraic
structures, e.g. vector spaces or polynomial rings, is the manner in
which dynamics is captured. In traditional structures, dynamics is typically
expressed through morphisms between such structures, as in linear maps
between vector spaces or morphisms between rings. In algebras
associated with the semantics of computation, the dynamics is
expressed as part of the algebraic structure itself, through a
reduction reduction relation typically denoted by $\red$. Below, we
give a recursive presentation of this relation for the calculus used
in the encoding.

$\red \subseteq \pi \times \pi$
$\red : \pi \to \mathcal{P}(\pi)$

\begin{mathpar}
  \inferrule* [lab=Comm] { \textsf{match}( x_{src}, x_{trgt} ) } { x_{trgt}?(y)P \; | \; x_{src}!\langle {Q} \rangle \red P\{\quotep{Q}/y}\} }
  \and \\
  \inferrule* [lab=Par] {{P} \red {P}'} {{{P} | {Q}} \red {{P}' | {Q}}}
  \and
  \inferrule* [lab=Equiv]{{{P} \scong {P}'} \andalso {{P}' \red {Q}'} \andalso {{Q}' \scong {Q}}}{{P} \red {Q}}
\end{mathpar}

\begin{eqnarray*}
  match_{\equiv} (\quotep{P},\quotep{Q}) & := & P \equiv Q \\
  match_{\dagger}(\quotep{P},\quotep{Q}) & := & \forall R. P|Q \red^{*} R => R \red^{*} 0 \\
  match_{K}(\quotep{P},\quotep{Q}) & := & K \mbox{ for some context } K
\end{eqnarray*}

$u?(x)P | u!\langle Q \rangle \red P\{\quotep{Q}/x\}$

%We write $\wred$ for $\red^*$, and $P\red$ if $\exists Q $ such that $ P \red Q$.
We write $P\red$ if $\exists Q $ such that $ P \red Q$ and $P\not\red$, otherwise.

\section{Replication}

As mentioned before, it is known that replication (and hence
recursion) can be implemented in a higher-order process algebra
\cite{SangiorgiWalker}. As our first example of calculation with the
machinery thus far presented we give the construction explicitly in
the {\rhoc}.

\begin{eqnarray}
	D_{x} & := & \prefix{x}{y}{(\binpar{\outputp{x}{y}}{@{y}})} \nonumber\\
	\bangp_{x}{P} & := & \binpar{{x}!\langle{\binpar{D_{x}}{P}}\rangle}{D_{x}} \nonumber
\end{eqnarray}

\begin{eqnarray}
	\bangp_{x}{P} & & \nonumber\\
	=
	& {x}!\langle{(\prefix{x}{y}{(\outputp{x}{y} | @{y})) | P}}\rangle 
	      | \prefix{x}{y}{(\outputp{x}{y} | @{y})} & \nonumber\\
	\red
	& (\outputp{x}{y} | @{y})\substn{\quotep{(\prefix{x}{y}{(@{y} | \outputp{x}{y})) | P}}}{y} & \nonumber\\
	=
	& \outputp{x}{\quotep{(\prefix{x}{y}{(\outputp{x}{y} | @{y})) | P}}}
	  | {(\prefix{x}{y}{(\outputp{x}{y} | @{y})) | P}} & \nonumber\\
	\red
	& \ldots & \nonumber\\
	\red^*
	& P | P | \ldots & \nonumber
\end{eqnarray}

Of course, this encoding, as an implementation, runs away, unfolding
$\bangp{P}$ eagerly. A lazier and more implementable replication
operator, restricted to input-guarded processes, may be obtained as follows.

\begin{eqnarray}
\bangp{\prefix{u}{v}{P}} 
	:= 
	\binpar{\lift{x}{\prefix{u}{v}{(\binpar{D(x)}{P})}}}{D(x)} \nonumber
\end{eqnarray}

\begin{remark}
  Note that the lazier definition still does not deal with summation
  or mixed summation (i.e. sums over input and output). The reader is
  invited to construct definitions of replication that deal with these
  features. 

  Further, the definitions are parameterized in a name, $x$. Can you,
  gentle reader, make a definition that eliminates this parameter and
  guarantees no accidental interaction between the replication
  machinery and the process being replicated -- i.e. no accidental
  sharing of names used by the process to get its work done and the
  name(s) used by the replication to effect copying. This latter
  revision of the definition of replication is crucial to obtaining
  the expected identity $!!P \sim !P$.
\end{remark}

\begin{remark}\label{rem:paradoxical_combinator}
  The reader familiar with the lambda calculus will have noticed the
  similarity between $D$ and the paradoxical combinator.

  [Ed. note: the existence of this seems to suggest we have to be more
  restrictive on the set of processes and names we admit if we are to
  support no-cloning.]
\end{remark}

\subsubsection{Bisimulation}

The computational dynamics gives rise to another kind of equivalence,
the equivalence of computational behavior. As previously mentioned
this is typically captured \emph{via} some form of bisimulation.

% The notion we use in this paper is weak barbed bisimulation
% \cite{milner91polyadicpi}.

The notion we use in this paper is derived from weak barbed
bisimulation \cite{milner91polyadicpi}. 

\begin{definition}
An \emph{observation relation}, $\downarrow_{\mathcal N}$, over a set
of names, $\mathcal N$, is the smallest relation satisfying the rules
below.

\infrule[Out-barb]{y \in {\mathcal N}, \; x \nameeq y}
		  {\outputp{x}{v} \downarrow_{\mathcal N} x}
\infrule[Par-barb]{\mbox{$P\downarrow_{\mathcal N} x$ or $Q\downarrow_{\mathcal N} x$}}
		  {\binpar{P}{Q} \downarrow_{\mathcal N} x}

We write $P \Downarrow_{\mathcal N} x$ if there is $Q$ such that 
$P \wred Q$ and $Q \downarrow_{\mathcal N} x$.
\end{definition}

\begin{definition}
%\label{def.bbisim}
An  ${\mathcal N}$-\emph{barbed bisimulation} over a set of names, ${\mathcal N}$, is a symmetric binary relation 
${\mathcal S}_{\mathcal N}$ between agents such that $P\rel{S}_{\mathcal N}Q$ implies:
\begin{enumerate}
\item If $P \red P'$ then $Q \wred Q'$ and $P'\rel{S}_{\mathcal N} Q'$.
\item If $P\downarrow_{\mathcal N} x$, then $Q\Downarrow_{\mathcal N} x$.
\end{enumerate}
$P$ is ${\mathcal N}$-barbed bisimilar to $Q$, written
$P \wbbisim_{\mathcal N} Q$, if $P \rel{S}_{\mathcal N} Q$ for some ${\mathcal N}$-barbed bisimulation ${\mathcal S}_{\mathcal N}$.
\end{definition}

$\mathcal{R} \subseteq \pi \times \pi$

$P \mathcal{R} Q => \forall P'. P \red P' \Rightarrow \exists Q'. Q \red Q', P' \mathcal{R} Q'$

$P \vdash x \Rightarrow Q \vdash x$

\begin{mathpar}
  \inferrule*[lab=Out-barb]{x \nameeq y}{{y}!\langle{Q}\rangle \vdash x}
  \and
  \inferrule*[lab=Par-barb]{\mbox{$P\vdash x$ or $Q\vdash x$}}{\binpar{P}{Q} \vdash x}
\end{mathpar}

\subsubsection{Contexts}

One of the principle advantages of computational calculi like the
$\pi$-calculus is a well-defined notion of context,
contextual-equivalence and a correlation between
contextual-equivalence and notions of bisimulation. The notion of
context allows the decomposition of a process into (sub-)process and
its syntactic environment, its context. Thus, a context may be
thought of as a process with a ``hole'' (written $\Box$) in it. The
application of a context $M$ to a process $P$, written $M[P]$, is
tantamount to filling the hole in $M$ with $P$. In this paper we do
not need the full weight of this theory, but do make use of the notion
of context in the proof the main theorem. 

\begin{mathpar}
  \inferrule* [lab=summation] {} {{M_{M},M_{N}} \bc \Box \;|\; x.M_{A} \;|\; M_{M}+M_{N}}
  \and
  \inferrule* [lab=agent] {} {{M_{A}} \bc (\vec{x})M_{P} \;| \; \clift{P_0,\ldots,M_{P},\ldots,P_N}}
  \and \\
  \inferrule* [lab=process] {} {{M_{P}} \bc M_{N} \;| \;P|M_{P} }
\end{mathpar} 

\begin{mathpar}
  \inferrule* [lab=sychronization] {} {M_{N} \bc \Box \;|\; x?M_{F} \;|\; x!M_{C}}
  \and
  \inferrule* [lab=abstraction] {} {{M_{F}} \bc (x)M_{P} }
  \and
  \inferrule* [lab=concretion] {} {{M_{C}} \bc \langle M_{P} \rangle }
  \and \\
  \inferrule* [lab=process] {} {{M_{P}} \bc M_{N} \;| \;P|M_{P} }
\end{mathpar}

\begin{definition}[contextual application] Given a context $M$, and
  process $P$, we define the \emph{contextual application}, $M[P] :=
  M\{P/\Box\}$. That is, the contextual application of M to P is the
  substitution of $P$ for $\Box$ in $M$.
\end{definition}

$\meaningof{-} : L \to \mathcal{P}(\pi)$

\begin{mathpar}
  \inferrule* [lab=collection] {} {\meaningof{true} = \pi, \and \meaningof{~E} = \pi \setminus \meaningof{E}, \and \meaningof{E_{1} \& E_{2}} = \meaningof{E_{1}} \cap \meaningof{E_{2}}}
\end{mathpar}

\begin{mathpar}
  \inferrule* [lab=structure] {} {\meaningof{0} = \{ P \in \pi | P \equiv 0 \}, \and \\ \meaningof{E_1 | E_2} = \{ P \in \pi | P \equiv P_{1} | P_{2}, P_{1} \in \meaningof{E_{1}}, P_{2} \in \meaningof{E_2}\} }
\end{mathpar}

\begin{mathpar}
 \inferrule* [lab=behavior] {} {\meaningof{\langle a?b \rangle E} = \{ P \in \pi | P \equiv Q | u?(y)P', \\ \and \\\\ \and \\ \;\;\; u \in \meaningof{a}, \forall z.P'\{z/y\} \in \meaningof{E\{z/b\}}\}, \and \\ \meaningof{a!E} = \{ P \in \pi | P \equiv Q | x!\langle P' \rangle, x \in \meaningof{a} P' \in \meaningof{E}\} }
\end{mathpar}

\begin{mathpar}
 \inferrule* [lab=nominal] {} {\meaningof{\quotep{E}} = \{ \quotep{P} \in \quotep{\pi} | P \in \meaningof{E} \}, \and \meaningof{\quotep{P}} = \{ \quotep{Q} \in \quotep{\pi} | P \equiv Q \} \and \\ \meaningof{@\quotep{E}} = \{ P \in \pi | P \equiv @x, x \in \meaningof{E} \}}
\end{mathpar}

\begin{eqnarray*}
  \\
  \meaningof{-} : TS \to ST
\end{eqnarray*}

\begin{eqnarray*}
  \\
  L : TS \to ST
\end{eqnarray*}

\begin{eqnarray*}
  \\
  P \models E \iff P \in \meaningof{E}
\end{eqnarray*}

\begin{eqnarray*}
  P \approx_{L} Q \iff \forall E \in L. P \models E \iff Q \models E
\end{eqnarray*}

\begin{eqnarray*}
  P \approx_{K} Q
\end{eqnarray*}

\begin{eqnarray*}
  P \approx Q
\end{eqnarray*}

$\approx_{K} = \approx = \approx_{L}$

\subsubsection{Contextual duality}

Note that contexts extend the quotation operation to a family of
operations from processes to names. Given a context, $M$, we can
define a \emph{nominal context}, $\quotep{M}$ by $\quotep{M}[P] :=
\quotep{M[P]}$. To foreshadow what is to come we observe that these
operations enjoy a duality with processes very much like the duality
between vectors and maps from vectors to scalars.

Further, because the calculus is essentially higher-order, we have a
correspondence between contexts and processes. More specifically,
given a name $x$ and a context $M$ we can construct $M^{*}_{x}$ such
that 

\begin{mathpar}
  M^{*}_{x} | \lift{x}{P} \red M[P]
\end{mathpar}

namely,

\begin{mathpar}
  M^{*}_{x} := x?(u).M[\dropn{u}]
\end{mathpar}

The dependence of $M^{*}_{x}$ on a name makes it an abstraction, 

\begin{mathpar}
  M^{*} := (x)x?(u).M[\dropn{u}]
\end{mathpar}

\subsection{Additional notation}

It will sometimes be convenient to denote the process a name
quotes. We already have the notation $x = \quotep{P}$, but it will be
convenient to introduce an alternate notation, $\procn{x}$, when we
want to emphasize the connection to the use of the name. Note that, by
virtue of name equivalence, $\quotep{\procn{x}} \nameeq x$; so, the
notation is consistent with previous definitions.

Further, because names have structure it is possible to effect
substitutions on the basis of that structure. This means we need to
upgrade our notation for substitutions, which we accomplish by
adapting comprehension notation. Thus,

\begin{mathpar}
  P\{ y / x : x \in S \}
\end{mathpar}

is interpreted to mean the process derived from P by replacing (in a
capture-avoiding manner) each occurrence of $x$ in $S$ by $y$. For example,

\begin{mathpar}
  P\{ \quotep{\procn{x}|\procn{x}} / x : x \in \freenames{P} \}
\end{mathpar}

will replace each (occurrence) of a free name $x$ in $P$ by
$\quotep{\procn{x}|\procn{x}}$.

Also, we will avail ourselves of the notation $x^{L}$ and $x^{R}$ to
denote injections of a name into disjoint copies of the name
space. There are numerous ways to accomplish this. One example can be
found in \cite{MeredithR05}. This notation overloads to vectors of
names: $\vec{x}^{\pi} := (x_{i}^{\pi} \; : \; 0 \leq i < |\vec{x}| )$ where $\pi \in \{L,R\}$.

We also use $P^{\Box} := P|\Box$.

In \cite{MeredithR05} an interpretation of the new operator is
given. It turns out that there are several possible interpretations
all enjoying the requisite algebraic properties of the operator (see
\cite{milner91polyadicpi}). We will therefore make liberal use of
$(\nu\; \vec{x})P$.

% subsection the_syntax_and_semantics_of_the_notation_system (end)   

\input{qm2pi.qmops} 

\input{qm2pi.sterngerlach} 

\input{qm2pi.metric} 

% section concurrent_process_calculi (end)

%\input{qm2pi.proofsketch}

% section proof sketch (end)

%\input{qm2pi.slviaknots} 

% section spatial logic via knots (end)

\input{qm2pi.conclusion}

% section conclusion (end)

%\input{qm2pi.dtcodes} 

% section wiring algorithm (end)

\input{qm2pi.ack} 

% section acknowledgments (end)

\newpage


\bibliographystyle{plain}   
\bibliography{../../biblios/main.bib}

\input{qm2pi.rhodetails}

\end{document}



% section front matter (end)

\section{Introduction}\label{sec:introduction} % (fold)
In this draft of the material i am going to have to dispense with the
usual writing conventions adopted in papers on these topics. i'm going
to have adopt whatever tone i need at the time i'm writing up the
calculations. Sometimes this may be very conversational; others it may
be the barest mathematical grunts; others still it may be that i have
lifted text from one of my other papers because the exposition of some
point was better said there. i hope that my readers are not unduly put
out by this decision. i'm not doing this to flout convention or be
rebellious. i find these calculations very technically challenging. To
keep everything going technically, something has to give; i have to
let go of some cognitive burden. So, the academic writing style --
with all of its trade-offs in terms of facilitating technical
communication -- is what i'm letting go of. Perhaps subsequent drafts
can be tightened and polished, but for now, i'm going to speak as if
we were sitting together in a coffee shop with a laptop, wifi and a
pad of paper and a pencil.

So, here's what i have to say. We -- you and i, comfortably ensconced
in our coffee shop and well-equipped with our tools -- can realize and
carry out the calculations of quantum mechanics over a very different
formal theory of dynamics, a formal theory of dynamics that
corresponds to a theory of concurrent computation with
\emph{reflection}. It has the advantage that the underlying theory is
already `quantized', but supports analogues all of the continuuous
operations. Strikingly, this underlying theory has recently been
connected with a notion of metric that we can show, by calculating
together, coincides with the metric induced by the inner product.

There are a lot of reasons why you might be interested in seeing
calculations of this form. Here's why i'm interested. For the past
several centuries there has been no competitor to the ``Newtonian''
account of dynamics. As a result the predominant share of accounts of
dynamical systems and situations have had to be formulated in terms of
the Newtonian machinery. i view this as an intellectually dangerous
position to occupy. Everything, despite it's intrinsic shape, turns
into a nail to be hit with this hammer. Recently, however, the theory
of computation has matured to the point where we have candidates for
theories of dynamics that offer very different perspective on
reasoning about dynamical systems and situations. Testing these
candidates against very successful accounts of dynamical situations,
like quantum mechanics, is going to give us some sense of how mature
they are and some measure of the quality of these accounts of
dynamics.

\subsection{Summary of contributions and outline of paper}

So, we're going to develop an interpretation of the operations of
quantum mechanics normally interpreted by Hilbert spaces and
operators. We're going to do this over a theory of computation. Note
that this is very different than the usual quantum computation program
which develops notions of computation over quantum mechanics. Rather,
we are developing a story that aligns with Wheeler's slogan: It from
Bit. To do this we will first provide an account of the theory of
computation at play here. Then we will dive into a calculation-driven
interpretation of the operations of quantum mechanics.

The reason we take this approach is that -- until very recently --
there hasn't been an axiomatic account of quantum mechanics. As a
result there has been no sharp delineation of the mathematical theory
supporting interpretation of the physical theory and the physical
theory, itself. So, ambient features of the maths are free to be
exploited (or supressed) without a real accounting of their physical
relevance. There is no sharp statement ``here's the physical theory''
qua \emph{theory} and ``here's the mathematical interpretation''
enabling a judgment of how faithful the interpretation is -- apart
from experimental observation. When there is an axiomatic account we
can judge how well a given mathematical formalism supports an
interpretation of the axioms, independent of
experimentation. Likewise, we can judge how well we have captured our
physical evidence and experience with our axiomatics, independent of
any specific mathematical implementation, with accidental detail that
may or may not have physical significance. 

In lieu of a fully fleshed out and vetted axiomatic account of quantum
mechanics, interpreting the operational notions in service of modeling
physical systems will have to suffice. In other words, we are not in
the business of providing a model of Hilbert spaces and operators. We
are in the business of providing a model of quantum mechanics because
we are motivated by testing our notions of dynamics against physical
theory; and, the predictive calculations of the physical theory must
serve as the best formulation -- shy of a fully fleshed out axiomatic
account -- of the physical theory itself (as they have for scientific
theories since time immemorial). Put another way, despite a
whole-hearted commitment to an It-from-Bit ontology, we are firmly
aligned with the shut-up-and-calculate camp as the best way to obtain
results either from the physical perspective or as a quality assurance
measure of our fledgling theory of dynamics.

In detail, we present a reflective process calculus. Then we develop
intuitive correspondences between the notions available in this
calculus and the usual physical notions supporting quantum mechanical
calculations. Thus, 

\begin{table}[htp]
  \center{
    \fbox{
      \begin{tabular}{c|c}
        quantum mechanics & process calculus \\
        \hline
        scalar & name \\
        state vector & process \\
        dual & contextual duals \\
        matrix & formal sums of process-context-dual pairs \\
        orthogonality & process annihilation \\
        inner product & execution-formula + quoting
      \end{tabular}
    }
  }
  \caption{QM - process calculi correspondences}
\end{table}

Then we tighten up these intuitions to operational definitions. We
employ the Dirac notation as the best proxy we can find for an
abstract syntax of the quantum mechanical notions. The definitions we
develop put us in contact with equational constraints coming from the
theory that we demonstrate the definitions and calculations satisfy.

This puts us in a position to shut up and calculate for the
Stern-Gerlach experimental set up, showing how these predictive
calculations become calculations on processes in our theory of a
reflective process calculus.

Penultimately, we demonstrate that the notion of metric coming from
the inner product coincides with the notion of metric available from
the theory of bisimulation. This demonstration gives us the right to
think of space as arising from behavior. Finally, we consider where we
might go from the new vantage point we have obtained.

% section introduction (end) 
 
% section introduction (end)

% \documentclass[12pt]{llncs}
%\documentclass{jktr}

\usepackage[pdftex]{hyperref}                   
\usepackage {listings}
\usepackage {mathpartir}
\usepackage{bcprules}
%\usepackage{listings}
                       
\usepackage{graphicx} 
%\usepackage[margins=2.5cm,nohead,nofoot]{geometry}
%\usepackage{geometry}
\usepackage{amsfonts}
\usepackage{amstext}
\usepackage{latexsym}
\usepackage{amssymb}
\usepackage{color}


%\include{myPreamble}
\include{qm2pi.local} 

%\ifpdf
%\usepackage[pdftex]{graphicx}
%\else
%\usepackage{graphicx}
%\fi

 % \ifpdf
%  \usepackage{pdfsync}
%  \if


%\title{Brief Article}
%\author{David F. Snyder}
%\author{L.G. Meredith}

%\address{Dept. of Math., Texas State University--San Marcos, San Marcos, TX 78666}
       
\pagestyle{empty}


\begin{document}

\lstset{language=[Objective]Caml,frame=shadowbox}

\input{qm2pi.front}

% section front matter (end)

\input{qm2pi.intro} 
 
% section introduction (end)

% \input{qm2pi.knotations} 

% section notation (end)

\input{qm2pi.process.calculi} 

% section concurrent_process_calculi_and_spatial_logics_ (end)
    
%\input{qm2pi.knots2pi} 

%\input{qm2pi.trefoil} 

%\input{qm2pi.mainthm} 

% subsection basic_interpretation (end)

%\input{qm2pi.rho.presentation} 
\subsection{The syntax and semantics of the notation system}\label{sub:the_syntax_and_semantics_of_the_notation_system} % (fold)

We now summarize a technical presentation of the calculus that
embodies our theory of dynamics. The typical presentation of such a
calculus follows the style of giving generators and relations on
them. The grammar, below, describing term constructors, freely
generates the set of processes, $\Proc$. This set is then quotiented
by a relation known as structural congruence and it is over this set
that the notion of dynamics is expressed. This presentation is
essentially that of \cite{MeredithR05} with the addition of
polyadicity and summation. For readability we have relegated some of
the technical subtleties to an appendix.

\subsubsection{Process grammar}\label{subsub:process_grammar}

\begin{mathpar}
  \inferrule* [lab=synchronization] {} {{M} \bc \pzero \;|\; x?F \;|\; x!C }
  \and
  \inferrule* [lab=abstraction] {} {{F} \bc (x)P}
  \and
  \inferrule* [lab=concretion] {} {{C} \bc \langle Q \rangle}
  \and
  \inferrule* [lab=process] {} {{P,Q} \bc M \;| \;P|Q \;|\; @{x}}
  \and
  \inferrule* [lab=name] {} {{x} \bc \quotep{P}}
\end{mathpar} 

Note that $\vec{x}$ (resp. $\vec{P}$) denotes a vector of names
(resp. processes) of length $|\vec{x}|$ (resp. $|\vec{P}|$). We adopt
the following useful abbreviations.

\begin{mathpar}
   x?(\vec{y}).P := x.(\vec{y})P \and  x\clift{\vec{P}} := x.\clift{\vec{P}}
   \and x!(y) := \lift{x}{\dropn{y}}
   \and \Pi_{i=0}^{n-1}P_i := P_0 | \ldots | P_{n-1}
\end{mathpar}

\subsubsection{Structural congruence}

\paragraph{Free and bound names and alpha-equivalence.} At the
core of structural equivalence is alpha-equivalence which identifies
process that are the same up to a change of variable. Formally, we
recognize the distinction between free and bound names. The free names
of a process, $\freenames{P}$, may be calculated recursively as
follows:

\begin{mathpar}
\freenames{\pzero} := \emptyset
  \and \\
  \freenames{x?(y).P} := \{ x \} \cup (\freenames{P} \setminus \{ y \})
  \and 
  \freenames{x!\langle P \rangle} := \{ x \} \cup \{ P \} 
  \and \\
  \freenames{P|Q} := \freenames{P} \cup \freenames{Q}
  \and \\
  \freenames{@{x}} := \{ x \}
\end{mathpar}

$\pi$
$\quotep{\pi}$

$\freenames{-} : \pi \to \mathcal{P}(\quotep{\pi})$

\begin{eqnarray*}
  \freenames{\pzero} & := & \emptyset \\
  \freenames{x?(y).P} & := & \{ x \} \cup (\freenames{P} \setminus \{ y \}) \\
  \freenames{x!\langle P \rangle} & := & \{ x \} \cup \{ P \} \\
  \freenames{P|Q} & := & \freenames{P} \cup \freenames{Q} \\
  \freenames{\dropn{x}} & := & \{ x \}
\end{eqnarray*}

The bound names of a process, $\boundnames{P}$, are those names occurring in $P$
that are not free. For example, in $x?(y).0$, the name $x$ is free, while $y$ is bound.

\begin{mathpar}
  \inferrule* [lab=monoidal-laws] {} { P|Q \equiv Q|P \and P|0 \equiv P \and P|(Q|R) \equiv (P|Q)|R }
\end{mathpar}

\begin{mathpar}
  \inferrule* [lab=alpha-equivalence] {} { (x)P \equiv (y)P\{y/x\} \and y \not\in \freenames{P} }
\end{mathpar}

\begin{definition}
Then two processes, $P,Q$, are alpha-equivalent if $P = Q\{\vec{y}/\vec{x}\}$ for
some $\vec{x} \in \boundnames{Q},\vec{y} \in \boundnames{P}$, where $Q\{\vec{y}/\vec{x}\}$
denotes the capture-avoiding substitution of $\vec{y}$ for $\vec{x}$ in $Q$.
\end{definition}

\begin{definition}
  The {\em structural congruence} \cite{SangiorgiWalker} , $\equiv$,
  between processes is the least congruence containing
  alpha-equivalence, satisfying the abelian monoid laws
  (associativity, commutativity and $\pzero$ as identity) for parallel
  composition $|$ and for summation $+$.
\end{definition}

\subsection{Name equivalence}

We take name equivalence, written $\nameeq$, to be the smallest
equivalence relation generated by the following rules.

\begin{mathpar}
\inferrule*[lab=Quote-drop]
{ }
{ \quotep{@{x}} \nameeq x }

\inferrule*[lab=Struct-equiv]
{ P \scong Q }
{ \quotep{P} \nameeq \quotep{Q} }
\end{mathpar}

The astute reader will have noticed that the mutual recursion of names
and processes imposes a mutual recursion on alpha-equivalence and
structural equivalence via name-equivalence. Fortunately, all of this
works out pleasantly and we may calculate in the natural way, free of
concern. The reader interested in the details is referred to the
appendix \ref{appendix:rho_details}.

\subsection{Substitution}

We use $\Proc$ for the set of processes, $\QProc$ for the set of
names, and $\id{\{}\vec{y} / \vec{x} \id{\}}$ to denote partial maps,
$s : \QProc \rightarrow \QProc$. A map, $s$ lifts, uniquely, to a map
on process terms, $\widehat{s} : \Proc \rightarrow \Proc$ by the
following equations.

\begin{mathpar}
  (0) \psubstp{Q}{P} := 0 \\
  (R \juxtap S) \psubstp{Q}{P}
  :=    
  (R)\psubstp{Q}{P} \juxtap (S) \psubstp{Q}{P} \\
  (x?(y).R) \psubstp{Q}{P}    
  :=    
  (x)\substp{Q}{P} (z)\concat( (R \psubstn{z}{y}) \psubstp{Q}{P} ) \\
  (\lift{x}{R}) \psubstp{Q}{P}  
  :=
  \lift{(x)\substp{Q}{P}}{ R \psubstp{Q}{P} } \\
%   (\dropn{x})  \psubstp{Q}{P}       
%   := 
%   \left\{ 
%     \begin{array}{ccc} 
%       \dropn{\quotep{Q}} & & x \nameeq \quotep{P} \\
%       \dropn{x} & & otherwise \\
%     \end{array}
%   \right. 
  (\dropn{x})  \psubstp{Q}{P}       
  := 
  \left\{ 
    \begin{array}{ccc} 
      Q & & x \nameeq \quotep{P} \\
      \dropn{x} & & otherwise \\
    \end{array}
  \right.
\end{mathpar}
 

where

\begin{eqnarray}
  (x)\id{\{} \lpquote Q \rpquote / \lpquote P \rpquote \id{\}}            = 
  \left\{ 
    \begin{array}{ccc}
      \lpquote Q \rpquote & & x \nameeq \lpquote P \rpquote \\
      x & & otherwise \\
    \end{array}
  \right. \nonumber
\end{eqnarray}

and $z$ is chosen distinct from $\quotep{P}$, $\quotep{Q}$, the free
names in $Q$, and all the names in $R$. Our $\alpha$-equivalence will
be built in the standard way from this substitution.

\begin{remark}\label{rem:no_self_referential_names}
  One consequence of these definitions is that $\forall P. \quotep{P}
  \not\in \freenames{P}$.
\end{remark}

\subsection{ Dynamic quote: an example }

Anticipating something of what's to come, consider applying the
substitution, $\widehat{\id{\{}u / z \id{\}}}$, to the following pair
of processes, $\lift{w}{y!(z)}$ and $w[ \lpquote y!(z) \rpquote ]$.

\begin{eqnarray}
	\lift{w}{y!(z)}\widehat{\id{\{}u / z \id{\}}}
		& = &
		\lift{w}{y!(u)} \nonumber\\
	w[ \lpquote y!(z) \rpquote ] \widehat{ \id{\{}u / z \id{\}} }
		& = &
		w[ \lpquote y!(z) \rpquote ] \nonumber
\end{eqnarray}

Because the body of the process between quotes is impervious to
substitution, we get radically different answers. In fact, by
examining the first process in an input context,
e.g. $x?(z).\lift{w}{y!(z)}$, we see that the process under the lift
operator may be shaped by prefixed inputs binding a name inside it. In
this sense, the lift operator will be seen as a way to dynamically
construct processes before reifying them as names.

Finally equipped with these standard features we can present the
dynamics of the calculus.

\subsubsection{Operational semantics} 

Finally, we introduce the computational dynamics. What marks these
algebras as distinct from other more traditionally studied algebraic
structures, e.g. vector spaces or polynomial rings, is the manner in
which dynamics is captured. In traditional structures, dynamics is typically
expressed through morphisms between such structures, as in linear maps
between vector spaces or morphisms between rings. In algebras
associated with the semantics of computation, the dynamics is
expressed as part of the algebraic structure itself, through a
reduction reduction relation typically denoted by $\red$. Below, we
give a recursive presentation of this relation for the calculus used
in the encoding.

$\red \subseteq \pi \times \pi$
$\red : \pi \to \mathcal{P}(\pi)$

\begin{mathpar}
  \inferrule* [lab=Comm] { \textsf{match}( x_{src}, x_{trgt} ) } { x_{trgt}?(y)P \; | \; x_{src}!\langle {Q} \rangle \red P\{\quotep{Q}/y}\} }
  \and \\
  \inferrule* [lab=Par] {{P} \red {P}'} {{{P} | {Q}} \red {{P}' | {Q}}}
  \and
  \inferrule* [lab=Equiv]{{{P} \scong {P}'} \andalso {{P}' \red {Q}'} \andalso {{Q}' \scong {Q}}}{{P} \red {Q}}
\end{mathpar}

\begin{eqnarray*}
  match_{\equiv} (\quotep{P},\quotep{Q}) & := & P \equiv Q \\
  match_{\dagger}(\quotep{P},\quotep{Q}) & := & \forall R. P|Q \red^{*} R => R \red^{*} 0 \\
  match_{K}(\quotep{P},\quotep{Q}) & := & K \mbox{ for some context } K
\end{eqnarray*}

$u?(x)P | u!\langle Q \rangle \red P\{\quotep{Q}/x\}$

%We write $\wred$ for $\red^*$, and $P\red$ if $\exists Q $ such that $ P \red Q$.
We write $P\red$ if $\exists Q $ such that $ P \red Q$ and $P\not\red$, otherwise.

\section{Replication}

As mentioned before, it is known that replication (and hence
recursion) can be implemented in a higher-order process algebra
\cite{SangiorgiWalker}. As our first example of calculation with the
machinery thus far presented we give the construction explicitly in
the {\rhoc}.

\begin{eqnarray}
	D_{x} & := & \prefix{x}{y}{(\binpar{\outputp{x}{y}}{@{y}})} \nonumber\\
	\bangp_{x}{P} & := & \binpar{{x}!\langle{\binpar{D_{x}}{P}}\rangle}{D_{x}} \nonumber
\end{eqnarray}

\begin{eqnarray}
	\bangp_{x}{P} & & \nonumber\\
	=
	& {x}!\langle{(\prefix{x}{y}{(\outputp{x}{y} | @{y})) | P}}\rangle 
	      | \prefix{x}{y}{(\outputp{x}{y} | @{y})} & \nonumber\\
	\red
	& (\outputp{x}{y} | @{y})\substn{\quotep{(\prefix{x}{y}{(@{y} | \outputp{x}{y})) | P}}}{y} & \nonumber\\
	=
	& \outputp{x}{\quotep{(\prefix{x}{y}{(\outputp{x}{y} | @{y})) | P}}}
	  | {(\prefix{x}{y}{(\outputp{x}{y} | @{y})) | P}} & \nonumber\\
	\red
	& \ldots & \nonumber\\
	\red^*
	& P | P | \ldots & \nonumber
\end{eqnarray}

Of course, this encoding, as an implementation, runs away, unfolding
$\bangp{P}$ eagerly. A lazier and more implementable replication
operator, restricted to input-guarded processes, may be obtained as follows.

\begin{eqnarray}
\bangp{\prefix{u}{v}{P}} 
	:= 
	\binpar{\lift{x}{\prefix{u}{v}{(\binpar{D(x)}{P})}}}{D(x)} \nonumber
\end{eqnarray}

\begin{remark}
  Note that the lazier definition still does not deal with summation
  or mixed summation (i.e. sums over input and output). The reader is
  invited to construct definitions of replication that deal with these
  features. 

  Further, the definitions are parameterized in a name, $x$. Can you,
  gentle reader, make a definition that eliminates this parameter and
  guarantees no accidental interaction between the replication
  machinery and the process being replicated -- i.e. no accidental
  sharing of names used by the process to get its work done and the
  name(s) used by the replication to effect copying. This latter
  revision of the definition of replication is crucial to obtaining
  the expected identity $!!P \sim !P$.
\end{remark}

\begin{remark}\label{rem:paradoxical_combinator}
  The reader familiar with the lambda calculus will have noticed the
  similarity between $D$ and the paradoxical combinator.

  [Ed. note: the existence of this seems to suggest we have to be more
  restrictive on the set of processes and names we admit if we are to
  support no-cloning.]
\end{remark}

\subsubsection{Bisimulation}

The computational dynamics gives rise to another kind of equivalence,
the equivalence of computational behavior. As previously mentioned
this is typically captured \emph{via} some form of bisimulation.

% The notion we use in this paper is weak barbed bisimulation
% \cite{milner91polyadicpi}.

The notion we use in this paper is derived from weak barbed
bisimulation \cite{milner91polyadicpi}. 

\begin{definition}
An \emph{observation relation}, $\downarrow_{\mathcal N}$, over a set
of names, $\mathcal N$, is the smallest relation satisfying the rules
below.

\infrule[Out-barb]{y \in {\mathcal N}, \; x \nameeq y}
		  {\outputp{x}{v} \downarrow_{\mathcal N} x}
\infrule[Par-barb]{\mbox{$P\downarrow_{\mathcal N} x$ or $Q\downarrow_{\mathcal N} x$}}
		  {\binpar{P}{Q} \downarrow_{\mathcal N} x}

We write $P \Downarrow_{\mathcal N} x$ if there is $Q$ such that 
$P \wred Q$ and $Q \downarrow_{\mathcal N} x$.
\end{definition}

\begin{definition}
%\label{def.bbisim}
An  ${\mathcal N}$-\emph{barbed bisimulation} over a set of names, ${\mathcal N}$, is a symmetric binary relation 
${\mathcal S}_{\mathcal N}$ between agents such that $P\rel{S}_{\mathcal N}Q$ implies:
\begin{enumerate}
\item If $P \red P'$ then $Q \wred Q'$ and $P'\rel{S}_{\mathcal N} Q'$.
\item If $P\downarrow_{\mathcal N} x$, then $Q\Downarrow_{\mathcal N} x$.
\end{enumerate}
$P$ is ${\mathcal N}$-barbed bisimilar to $Q$, written
$P \wbbisim_{\mathcal N} Q$, if $P \rel{S}_{\mathcal N} Q$ for some ${\mathcal N}$-barbed bisimulation ${\mathcal S}_{\mathcal N}$.
\end{definition}

$\mathcal{R} \subseteq \pi \times \pi$

$P \mathcal{R} Q => \forall P'. P \red P' \Rightarrow \exists Q'. Q \red Q', P' \mathcal{R} Q'$

$P \vdash x \Rightarrow Q \vdash x$

\begin{mathpar}
  \inferrule*[lab=Out-barb]{x \nameeq y}{{y}!\langle{Q}\rangle \vdash x}
  \and
  \inferrule*[lab=Par-barb]{\mbox{$P\vdash x$ or $Q\vdash x$}}{\binpar{P}{Q} \vdash x}
\end{mathpar}

\subsubsection{Contexts}

One of the principle advantages of computational calculi like the
$\pi$-calculus is a well-defined notion of context,
contextual-equivalence and a correlation between
contextual-equivalence and notions of bisimulation. The notion of
context allows the decomposition of a process into (sub-)process and
its syntactic environment, its context. Thus, a context may be
thought of as a process with a ``hole'' (written $\Box$) in it. The
application of a context $M$ to a process $P$, written $M[P]$, is
tantamount to filling the hole in $M$ with $P$. In this paper we do
not need the full weight of this theory, but do make use of the notion
of context in the proof the main theorem. 

\begin{mathpar}
  \inferrule* [lab=summation] {} {{M_{M},M_{N}} \bc \Box \;|\; x.M_{A} \;|\; M_{M}+M_{N}}
  \and
  \inferrule* [lab=agent] {} {{M_{A}} \bc (\vec{x})M_{P} \;| \; \clift{P_0,\ldots,M_{P},\ldots,P_N}}
  \and \\
  \inferrule* [lab=process] {} {{M_{P}} \bc M_{N} \;| \;P|M_{P} }
\end{mathpar} 

\begin{mathpar}
  \inferrule* [lab=sychronization] {} {M_{N} \bc \Box \;|\; x?M_{F} \;|\; x!M_{C}}
  \and
  \inferrule* [lab=abstraction] {} {{M_{F}} \bc (x)M_{P} }
  \and
  \inferrule* [lab=concretion] {} {{M_{C}} \bc \langle M_{P} \rangle }
  \and \\
  \inferrule* [lab=process] {} {{M_{P}} \bc M_{N} \;| \;P|M_{P} }
\end{mathpar}

\begin{definition}[contextual application] Given a context $M$, and
  process $P$, we define the \emph{contextual application}, $M[P] :=
  M\{P/\Box\}$. That is, the contextual application of M to P is the
  substitution of $P$ for $\Box$ in $M$.
\end{definition}

$\meaningof{-} : L \to \mathcal{P}(\pi)$

\begin{mathpar}
  \inferrule* [lab=collection] {} {\meaningof{true} = \pi, \and \meaningof{~E} = \pi \setminus \meaningof{E}, \and \meaningof{E_{1} \& E_{2}} = \meaningof{E_{1}} \cap \meaningof{E_{2}}}
\end{mathpar}

\begin{mathpar}
  \inferrule* [lab=structure] {} {\meaningof{0} = \{ P \in \pi | P \equiv 0 \}, \and \\ \meaningof{E_1 | E_2} = \{ P \in \pi | P \equiv P_{1} | P_{2}, P_{1} \in \meaningof{E_{1}}, P_{2} \in \meaningof{E_2}\} }
\end{mathpar}

\begin{mathpar}
 \inferrule* [lab=behavior] {} {\meaningof{\langle a?b \rangle E} = \{ P \in \pi | P \equiv Q | u?(y)P', \\ \and \\\\ \and \\ \;\;\; u \in \meaningof{a}, \forall z.P'\{z/y\} \in \meaningof{E\{z/b\}}\}, \and \\ \meaningof{a!E} = \{ P \in \pi | P \equiv Q | x!\langle P' \rangle, x \in \meaningof{a} P' \in \meaningof{E}\} }
\end{mathpar}

\begin{mathpar}
 \inferrule* [lab=nominal] {} {\meaningof{\quotep{E}} = \{ \quotep{P} \in \quotep{\pi} | P \in \meaningof{E} \}, \and \meaningof{\quotep{P}} = \{ \quotep{Q} \in \quotep{\pi} | P \equiv Q \} \and \\ \meaningof{@\quotep{E}} = \{ P \in \pi | P \equiv @x, x \in \meaningof{E} \}}
\end{mathpar}

\begin{eqnarray*}
  \\
  \meaningof{-} : TS \to ST
\end{eqnarray*}

\begin{eqnarray*}
  \\
  L : TS \to ST
\end{eqnarray*}

\begin{eqnarray*}
  \\
  P \models E \iff P \in \meaningof{E}
\end{eqnarray*}

\begin{eqnarray*}
  P \approx_{L} Q \iff \forall E \in L. P \models E \iff Q \models E
\end{eqnarray*}

\begin{eqnarray*}
  P \approx_{K} Q
\end{eqnarray*}

\begin{eqnarray*}
  P \approx Q
\end{eqnarray*}

$\approx_{K} = \approx = \approx_{L}$

\subsubsection{Contextual duality}

Note that contexts extend the quotation operation to a family of
operations from processes to names. Given a context, $M$, we can
define a \emph{nominal context}, $\quotep{M}$ by $\quotep{M}[P] :=
\quotep{M[P]}$. To foreshadow what is to come we observe that these
operations enjoy a duality with processes very much like the duality
between vectors and maps from vectors to scalars.

Further, because the calculus is essentially higher-order, we have a
correspondence between contexts and processes. More specifically,
given a name $x$ and a context $M$ we can construct $M^{*}_{x}$ such
that 

\begin{mathpar}
  M^{*}_{x} | \lift{x}{P} \red M[P]
\end{mathpar}

namely,

\begin{mathpar}
  M^{*}_{x} := x?(u).M[\dropn{u}]
\end{mathpar}

The dependence of $M^{*}_{x}$ on a name makes it an abstraction, 

\begin{mathpar}
  M^{*} := (x)x?(u).M[\dropn{u}]
\end{mathpar}

\subsection{Additional notation}

It will sometimes be convenient to denote the process a name
quotes. We already have the notation $x = \quotep{P}$, but it will be
convenient to introduce an alternate notation, $\procn{x}$, when we
want to emphasize the connection to the use of the name. Note that, by
virtue of name equivalence, $\quotep{\procn{x}} \nameeq x$; so, the
notation is consistent with previous definitions.

Further, because names have structure it is possible to effect
substitutions on the basis of that structure. This means we need to
upgrade our notation for substitutions, which we accomplish by
adapting comprehension notation. Thus,

\begin{mathpar}
  P\{ y / x : x \in S \}
\end{mathpar}

is interpreted to mean the process derived from P by replacing (in a
capture-avoiding manner) each occurrence of $x$ in $S$ by $y$. For example,

\begin{mathpar}
  P\{ \quotep{\procn{x}|\procn{x}} / x : x \in \freenames{P} \}
\end{mathpar}

will replace each (occurrence) of a free name $x$ in $P$ by
$\quotep{\procn{x}|\procn{x}}$.

Also, we will avail ourselves of the notation $x^{L}$ and $x^{R}$ to
denote injections of a name into disjoint copies of the name
space. There are numerous ways to accomplish this. One example can be
found in \cite{MeredithR05}. This notation overloads to vectors of
names: $\vec{x}^{\pi} := (x_{i}^{\pi} \; : \; 0 \leq i < |\vec{x}| )$ where $\pi \in \{L,R\}$.

We also use $P^{\Box} := P|\Box$.

In \cite{MeredithR05} an interpretation of the new operator is
given. It turns out that there are several possible interpretations
all enjoying the requisite algebraic properties of the operator (see
\cite{milner91polyadicpi}). We will therefore make liberal use of
$(\nu\; \vec{x})P$.

% subsection the_syntax_and_semantics_of_the_notation_system (end)   

\input{qm2pi.qmops} 

\input{qm2pi.sterngerlach} 

\input{qm2pi.metric} 

% section concurrent_process_calculi (end)

%\input{qm2pi.proofsketch}

% section proof sketch (end)

%\input{qm2pi.slviaknots} 

% section spatial logic via knots (end)

\input{qm2pi.conclusion}

% section conclusion (end)

%\input{qm2pi.dtcodes} 

% section wiring algorithm (end)

\input{qm2pi.ack} 

% section acknowledgments (end)

\newpage


\bibliographystyle{plain}   
\bibliography{../../biblios/main.bib}

\input{qm2pi.rhodetails}

\end{document}

 

% section notation (end)

\input{qm2pi.process.calculi} 

% section concurrent_process_calculi_and_spatial_logics_ (end)
    
%\documentclass[12pt]{llncs}
%\documentclass{jktr}

\usepackage[pdftex]{hyperref}                   
\usepackage {listings}
\usepackage {mathpartir}
\usepackage{bcprules}
%\usepackage{listings}
                       
\usepackage{graphicx} 
%\usepackage[margins=2.5cm,nohead,nofoot]{geometry}
%\usepackage{geometry}
\usepackage{amsfonts}
\usepackage{amstext}
\usepackage{latexsym}
\usepackage{amssymb}
\usepackage{color}


%\include{myPreamble}
\include{qm2pi.local} 

%\ifpdf
%\usepackage[pdftex]{graphicx}
%\else
%\usepackage{graphicx}
%\fi

 % \ifpdf
%  \usepackage{pdfsync}
%  \if


%\title{Brief Article}
%\author{David F. Snyder}
%\author{L.G. Meredith}

%\address{Dept. of Math., Texas State University--San Marcos, San Marcos, TX 78666}
       
\pagestyle{empty}


\begin{document}

\lstset{language=[Objective]Caml,frame=shadowbox}

\input{qm2pi.front}

% section front matter (end)

\input{qm2pi.intro} 
 
% section introduction (end)

% \input{qm2pi.knotations} 

% section notation (end)

\input{qm2pi.process.calculi} 

% section concurrent_process_calculi_and_spatial_logics_ (end)
    
%\input{qm2pi.knots2pi} 

%\input{qm2pi.trefoil} 

%\input{qm2pi.mainthm} 

% subsection basic_interpretation (end)

%\input{qm2pi.rho.presentation} 
\subsection{The syntax and semantics of the notation system}\label{sub:the_syntax_and_semantics_of_the_notation_system} % (fold)

We now summarize a technical presentation of the calculus that
embodies our theory of dynamics. The typical presentation of such a
calculus follows the style of giving generators and relations on
them. The grammar, below, describing term constructors, freely
generates the set of processes, $\Proc$. This set is then quotiented
by a relation known as structural congruence and it is over this set
that the notion of dynamics is expressed. This presentation is
essentially that of \cite{MeredithR05} with the addition of
polyadicity and summation. For readability we have relegated some of
the technical subtleties to an appendix.

\subsubsection{Process grammar}\label{subsub:process_grammar}

\begin{mathpar}
  \inferrule* [lab=synchronization] {} {{M} \bc \pzero \;|\; x?F \;|\; x!C }
  \and
  \inferrule* [lab=abstraction] {} {{F} \bc (x)P}
  \and
  \inferrule* [lab=concretion] {} {{C} \bc \langle Q \rangle}
  \and
  \inferrule* [lab=process] {} {{P,Q} \bc M \;| \;P|Q \;|\; @{x}}
  \and
  \inferrule* [lab=name] {} {{x} \bc \quotep{P}}
\end{mathpar} 

Note that $\vec{x}$ (resp. $\vec{P}$) denotes a vector of names
(resp. processes) of length $|\vec{x}|$ (resp. $|\vec{P}|$). We adopt
the following useful abbreviations.

\begin{mathpar}
   x?(\vec{y}).P := x.(\vec{y})P \and  x\clift{\vec{P}} := x.\clift{\vec{P}}
   \and x!(y) := \lift{x}{\dropn{y}}
   \and \Pi_{i=0}^{n-1}P_i := P_0 | \ldots | P_{n-1}
\end{mathpar}

\subsubsection{Structural congruence}

\paragraph{Free and bound names and alpha-equivalence.} At the
core of structural equivalence is alpha-equivalence which identifies
process that are the same up to a change of variable. Formally, we
recognize the distinction between free and bound names. The free names
of a process, $\freenames{P}$, may be calculated recursively as
follows:

\begin{mathpar}
\freenames{\pzero} := \emptyset
  \and \\
  \freenames{x?(y).P} := \{ x \} \cup (\freenames{P} \setminus \{ y \})
  \and 
  \freenames{x!\langle P \rangle} := \{ x \} \cup \{ P \} 
  \and \\
  \freenames{P|Q} := \freenames{P} \cup \freenames{Q}
  \and \\
  \freenames{@{x}} := \{ x \}
\end{mathpar}

$\pi$
$\quotep{\pi}$

$\freenames{-} : \pi \to \mathcal{P}(\quotep{\pi})$

\begin{eqnarray*}
  \freenames{\pzero} & := & \emptyset \\
  \freenames{x?(y).P} & := & \{ x \} \cup (\freenames{P} \setminus \{ y \}) \\
  \freenames{x!\langle P \rangle} & := & \{ x \} \cup \{ P \} \\
  \freenames{P|Q} & := & \freenames{P} \cup \freenames{Q} \\
  \freenames{\dropn{x}} & := & \{ x \}
\end{eqnarray*}

The bound names of a process, $\boundnames{P}$, are those names occurring in $P$
that are not free. For example, in $x?(y).0$, the name $x$ is free, while $y$ is bound.

\begin{mathpar}
  \inferrule* [lab=monoidal-laws] {} { P|Q \equiv Q|P \and P|0 \equiv P \and P|(Q|R) \equiv (P|Q)|R }
\end{mathpar}

\begin{mathpar}
  \inferrule* [lab=alpha-equivalence] {} { (x)P \equiv (y)P\{y/x\} \and y \not\in \freenames{P} }
\end{mathpar}

\begin{definition}
Then two processes, $P,Q$, are alpha-equivalent if $P = Q\{\vec{y}/\vec{x}\}$ for
some $\vec{x} \in \boundnames{Q},\vec{y} \in \boundnames{P}$, where $Q\{\vec{y}/\vec{x}\}$
denotes the capture-avoiding substitution of $\vec{y}$ for $\vec{x}$ in $Q$.
\end{definition}

\begin{definition}
  The {\em structural congruence} \cite{SangiorgiWalker} , $\equiv$,
  between processes is the least congruence containing
  alpha-equivalence, satisfying the abelian monoid laws
  (associativity, commutativity and $\pzero$ as identity) for parallel
  composition $|$ and for summation $+$.
\end{definition}

\subsection{Name equivalence}

We take name equivalence, written $\nameeq$, to be the smallest
equivalence relation generated by the following rules.

\begin{mathpar}
\inferrule*[lab=Quote-drop]
{ }
{ \quotep{@{x}} \nameeq x }

\inferrule*[lab=Struct-equiv]
{ P \scong Q }
{ \quotep{P} \nameeq \quotep{Q} }
\end{mathpar}

The astute reader will have noticed that the mutual recursion of names
and processes imposes a mutual recursion on alpha-equivalence and
structural equivalence via name-equivalence. Fortunately, all of this
works out pleasantly and we may calculate in the natural way, free of
concern. The reader interested in the details is referred to the
appendix \ref{appendix:rho_details}.

\subsection{Substitution}

We use $\Proc$ for the set of processes, $\QProc$ for the set of
names, and $\id{\{}\vec{y} / \vec{x} \id{\}}$ to denote partial maps,
$s : \QProc \rightarrow \QProc$. A map, $s$ lifts, uniquely, to a map
on process terms, $\widehat{s} : \Proc \rightarrow \Proc$ by the
following equations.

\begin{mathpar}
  (0) \psubstp{Q}{P} := 0 \\
  (R \juxtap S) \psubstp{Q}{P}
  :=    
  (R)\psubstp{Q}{P} \juxtap (S) \psubstp{Q}{P} \\
  (x?(y).R) \psubstp{Q}{P}    
  :=    
  (x)\substp{Q}{P} (z)\concat( (R \psubstn{z}{y}) \psubstp{Q}{P} ) \\
  (\lift{x}{R}) \psubstp{Q}{P}  
  :=
  \lift{(x)\substp{Q}{P}}{ R \psubstp{Q}{P} } \\
%   (\dropn{x})  \psubstp{Q}{P}       
%   := 
%   \left\{ 
%     \begin{array}{ccc} 
%       \dropn{\quotep{Q}} & & x \nameeq \quotep{P} \\
%       \dropn{x} & & otherwise \\
%     \end{array}
%   \right. 
  (\dropn{x})  \psubstp{Q}{P}       
  := 
  \left\{ 
    \begin{array}{ccc} 
      Q & & x \nameeq \quotep{P} \\
      \dropn{x} & & otherwise \\
    \end{array}
  \right.
\end{mathpar}
 

where

\begin{eqnarray}
  (x)\id{\{} \lpquote Q \rpquote / \lpquote P \rpquote \id{\}}            = 
  \left\{ 
    \begin{array}{ccc}
      \lpquote Q \rpquote & & x \nameeq \lpquote P \rpquote \\
      x & & otherwise \\
    \end{array}
  \right. \nonumber
\end{eqnarray}

and $z$ is chosen distinct from $\quotep{P}$, $\quotep{Q}$, the free
names in $Q$, and all the names in $R$. Our $\alpha$-equivalence will
be built in the standard way from this substitution.

\begin{remark}\label{rem:no_self_referential_names}
  One consequence of these definitions is that $\forall P. \quotep{P}
  \not\in \freenames{P}$.
\end{remark}

\subsection{ Dynamic quote: an example }

Anticipating something of what's to come, consider applying the
substitution, $\widehat{\id{\{}u / z \id{\}}}$, to the following pair
of processes, $\lift{w}{y!(z)}$ and $w[ \lpquote y!(z) \rpquote ]$.

\begin{eqnarray}
	\lift{w}{y!(z)}\widehat{\id{\{}u / z \id{\}}}
		& = &
		\lift{w}{y!(u)} \nonumber\\
	w[ \lpquote y!(z) \rpquote ] \widehat{ \id{\{}u / z \id{\}} }
		& = &
		w[ \lpquote y!(z) \rpquote ] \nonumber
\end{eqnarray}

Because the body of the process between quotes is impervious to
substitution, we get radically different answers. In fact, by
examining the first process in an input context,
e.g. $x?(z).\lift{w}{y!(z)}$, we see that the process under the lift
operator may be shaped by prefixed inputs binding a name inside it. In
this sense, the lift operator will be seen as a way to dynamically
construct processes before reifying them as names.

Finally equipped with these standard features we can present the
dynamics of the calculus.

\subsubsection{Operational semantics} 

Finally, we introduce the computational dynamics. What marks these
algebras as distinct from other more traditionally studied algebraic
structures, e.g. vector spaces or polynomial rings, is the manner in
which dynamics is captured. In traditional structures, dynamics is typically
expressed through morphisms between such structures, as in linear maps
between vector spaces or morphisms between rings. In algebras
associated with the semantics of computation, the dynamics is
expressed as part of the algebraic structure itself, through a
reduction reduction relation typically denoted by $\red$. Below, we
give a recursive presentation of this relation for the calculus used
in the encoding.

$\red \subseteq \pi \times \pi$
$\red : \pi \to \mathcal{P}(\pi)$

\begin{mathpar}
  \inferrule* [lab=Comm] { \textsf{match}( x_{src}, x_{trgt} ) } { x_{trgt}?(y)P \; | \; x_{src}!\langle {Q} \rangle \red P\{\quotep{Q}/y}\} }
  \and \\
  \inferrule* [lab=Par] {{P} \red {P}'} {{{P} | {Q}} \red {{P}' | {Q}}}
  \and
  \inferrule* [lab=Equiv]{{{P} \scong {P}'} \andalso {{P}' \red {Q}'} \andalso {{Q}' \scong {Q}}}{{P} \red {Q}}
\end{mathpar}

\begin{eqnarray*}
  match_{\equiv} (\quotep{P},\quotep{Q}) & := & P \equiv Q \\
  match_{\dagger}(\quotep{P},\quotep{Q}) & := & \forall R. P|Q \red^{*} R => R \red^{*} 0 \\
  match_{K}(\quotep{P},\quotep{Q}) & := & K \mbox{ for some context } K
\end{eqnarray*}

$u?(x)P | u!\langle Q \rangle \red P\{\quotep{Q}/x\}$

%We write $\wred$ for $\red^*$, and $P\red$ if $\exists Q $ such that $ P \red Q$.
We write $P\red$ if $\exists Q $ such that $ P \red Q$ and $P\not\red$, otherwise.

\section{Replication}

As mentioned before, it is known that replication (and hence
recursion) can be implemented in a higher-order process algebra
\cite{SangiorgiWalker}. As our first example of calculation with the
machinery thus far presented we give the construction explicitly in
the {\rhoc}.

\begin{eqnarray}
	D_{x} & := & \prefix{x}{y}{(\binpar{\outputp{x}{y}}{@{y}})} \nonumber\\
	\bangp_{x}{P} & := & \binpar{{x}!\langle{\binpar{D_{x}}{P}}\rangle}{D_{x}} \nonumber
\end{eqnarray}

\begin{eqnarray}
	\bangp_{x}{P} & & \nonumber\\
	=
	& {x}!\langle{(\prefix{x}{y}{(\outputp{x}{y} | @{y})) | P}}\rangle 
	      | \prefix{x}{y}{(\outputp{x}{y} | @{y})} & \nonumber\\
	\red
	& (\outputp{x}{y} | @{y})\substn{\quotep{(\prefix{x}{y}{(@{y} | \outputp{x}{y})) | P}}}{y} & \nonumber\\
	=
	& \outputp{x}{\quotep{(\prefix{x}{y}{(\outputp{x}{y} | @{y})) | P}}}
	  | {(\prefix{x}{y}{(\outputp{x}{y} | @{y})) | P}} & \nonumber\\
	\red
	& \ldots & \nonumber\\
	\red^*
	& P | P | \ldots & \nonumber
\end{eqnarray}

Of course, this encoding, as an implementation, runs away, unfolding
$\bangp{P}$ eagerly. A lazier and more implementable replication
operator, restricted to input-guarded processes, may be obtained as follows.

\begin{eqnarray}
\bangp{\prefix{u}{v}{P}} 
	:= 
	\binpar{\lift{x}{\prefix{u}{v}{(\binpar{D(x)}{P})}}}{D(x)} \nonumber
\end{eqnarray}

\begin{remark}
  Note that the lazier definition still does not deal with summation
  or mixed summation (i.e. sums over input and output). The reader is
  invited to construct definitions of replication that deal with these
  features. 

  Further, the definitions are parameterized in a name, $x$. Can you,
  gentle reader, make a definition that eliminates this parameter and
  guarantees no accidental interaction between the replication
  machinery and the process being replicated -- i.e. no accidental
  sharing of names used by the process to get its work done and the
  name(s) used by the replication to effect copying. This latter
  revision of the definition of replication is crucial to obtaining
  the expected identity $!!P \sim !P$.
\end{remark}

\begin{remark}\label{rem:paradoxical_combinator}
  The reader familiar with the lambda calculus will have noticed the
  similarity between $D$ and the paradoxical combinator.

  [Ed. note: the existence of this seems to suggest we have to be more
  restrictive on the set of processes and names we admit if we are to
  support no-cloning.]
\end{remark}

\subsubsection{Bisimulation}

The computational dynamics gives rise to another kind of equivalence,
the equivalence of computational behavior. As previously mentioned
this is typically captured \emph{via} some form of bisimulation.

% The notion we use in this paper is weak barbed bisimulation
% \cite{milner91polyadicpi}.

The notion we use in this paper is derived from weak barbed
bisimulation \cite{milner91polyadicpi}. 

\begin{definition}
An \emph{observation relation}, $\downarrow_{\mathcal N}$, over a set
of names, $\mathcal N$, is the smallest relation satisfying the rules
below.

\infrule[Out-barb]{y \in {\mathcal N}, \; x \nameeq y}
		  {\outputp{x}{v} \downarrow_{\mathcal N} x}
\infrule[Par-barb]{\mbox{$P\downarrow_{\mathcal N} x$ or $Q\downarrow_{\mathcal N} x$}}
		  {\binpar{P}{Q} \downarrow_{\mathcal N} x}

We write $P \Downarrow_{\mathcal N} x$ if there is $Q$ such that 
$P \wred Q$ and $Q \downarrow_{\mathcal N} x$.
\end{definition}

\begin{definition}
%\label{def.bbisim}
An  ${\mathcal N}$-\emph{barbed bisimulation} over a set of names, ${\mathcal N}$, is a symmetric binary relation 
${\mathcal S}_{\mathcal N}$ between agents such that $P\rel{S}_{\mathcal N}Q$ implies:
\begin{enumerate}
\item If $P \red P'$ then $Q \wred Q'$ and $P'\rel{S}_{\mathcal N} Q'$.
\item If $P\downarrow_{\mathcal N} x$, then $Q\Downarrow_{\mathcal N} x$.
\end{enumerate}
$P$ is ${\mathcal N}$-barbed bisimilar to $Q$, written
$P \wbbisim_{\mathcal N} Q$, if $P \rel{S}_{\mathcal N} Q$ for some ${\mathcal N}$-barbed bisimulation ${\mathcal S}_{\mathcal N}$.
\end{definition}

$\mathcal{R} \subseteq \pi \times \pi$

$P \mathcal{R} Q => \forall P'. P \red P' \Rightarrow \exists Q'. Q \red Q', P' \mathcal{R} Q'$

$P \vdash x \Rightarrow Q \vdash x$

\begin{mathpar}
  \inferrule*[lab=Out-barb]{x \nameeq y}{{y}!\langle{Q}\rangle \vdash x}
  \and
  \inferrule*[lab=Par-barb]{\mbox{$P\vdash x$ or $Q\vdash x$}}{\binpar{P}{Q} \vdash x}
\end{mathpar}

\subsubsection{Contexts}

One of the principle advantages of computational calculi like the
$\pi$-calculus is a well-defined notion of context,
contextual-equivalence and a correlation between
contextual-equivalence and notions of bisimulation. The notion of
context allows the decomposition of a process into (sub-)process and
its syntactic environment, its context. Thus, a context may be
thought of as a process with a ``hole'' (written $\Box$) in it. The
application of a context $M$ to a process $P$, written $M[P]$, is
tantamount to filling the hole in $M$ with $P$. In this paper we do
not need the full weight of this theory, but do make use of the notion
of context in the proof the main theorem. 

\begin{mathpar}
  \inferrule* [lab=summation] {} {{M_{M},M_{N}} \bc \Box \;|\; x.M_{A} \;|\; M_{M}+M_{N}}
  \and
  \inferrule* [lab=agent] {} {{M_{A}} \bc (\vec{x})M_{P} \;| \; \clift{P_0,\ldots,M_{P},\ldots,P_N}}
  \and \\
  \inferrule* [lab=process] {} {{M_{P}} \bc M_{N} \;| \;P|M_{P} }
\end{mathpar} 

\begin{mathpar}
  \inferrule* [lab=sychronization] {} {M_{N} \bc \Box \;|\; x?M_{F} \;|\; x!M_{C}}
  \and
  \inferrule* [lab=abstraction] {} {{M_{F}} \bc (x)M_{P} }
  \and
  \inferrule* [lab=concretion] {} {{M_{C}} \bc \langle M_{P} \rangle }
  \and \\
  \inferrule* [lab=process] {} {{M_{P}} \bc M_{N} \;| \;P|M_{P} }
\end{mathpar}

\begin{definition}[contextual application] Given a context $M$, and
  process $P$, we define the \emph{contextual application}, $M[P] :=
  M\{P/\Box\}$. That is, the contextual application of M to P is the
  substitution of $P$ for $\Box$ in $M$.
\end{definition}

$\meaningof{-} : L \to \mathcal{P}(\pi)$

\begin{mathpar}
  \inferrule* [lab=collection] {} {\meaningof{true} = \pi, \and \meaningof{~E} = \pi \setminus \meaningof{E}, \and \meaningof{E_{1} \& E_{2}} = \meaningof{E_{1}} \cap \meaningof{E_{2}}}
\end{mathpar}

\begin{mathpar}
  \inferrule* [lab=structure] {} {\meaningof{0} = \{ P \in \pi | P \equiv 0 \}, \and \\ \meaningof{E_1 | E_2} = \{ P \in \pi | P \equiv P_{1} | P_{2}, P_{1} \in \meaningof{E_{1}}, P_{2} \in \meaningof{E_2}\} }
\end{mathpar}

\begin{mathpar}
 \inferrule* [lab=behavior] {} {\meaningof{\langle a?b \rangle E} = \{ P \in \pi | P \equiv Q | u?(y)P', \\ \and \\\\ \and \\ \;\;\; u \in \meaningof{a}, \forall z.P'\{z/y\} \in \meaningof{E\{z/b\}}\}, \and \\ \meaningof{a!E} = \{ P \in \pi | P \equiv Q | x!\langle P' \rangle, x \in \meaningof{a} P' \in \meaningof{E}\} }
\end{mathpar}

\begin{mathpar}
 \inferrule* [lab=nominal] {} {\meaningof{\quotep{E}} = \{ \quotep{P} \in \quotep{\pi} | P \in \meaningof{E} \}, \and \meaningof{\quotep{P}} = \{ \quotep{Q} \in \quotep{\pi} | P \equiv Q \} \and \\ \meaningof{@\quotep{E}} = \{ P \in \pi | P \equiv @x, x \in \meaningof{E} \}}
\end{mathpar}

\begin{eqnarray*}
  \\
  \meaningof{-} : TS \to ST
\end{eqnarray*}

\begin{eqnarray*}
  \\
  L : TS \to ST
\end{eqnarray*}

\begin{eqnarray*}
  \\
  P \models E \iff P \in \meaningof{E}
\end{eqnarray*}

\begin{eqnarray*}
  P \approx_{L} Q \iff \forall E \in L. P \models E \iff Q \models E
\end{eqnarray*}

\begin{eqnarray*}
  P \approx_{K} Q
\end{eqnarray*}

\begin{eqnarray*}
  P \approx Q
\end{eqnarray*}

$\approx_{K} = \approx = \approx_{L}$

\subsubsection{Contextual duality}

Note that contexts extend the quotation operation to a family of
operations from processes to names. Given a context, $M$, we can
define a \emph{nominal context}, $\quotep{M}$ by $\quotep{M}[P] :=
\quotep{M[P]}$. To foreshadow what is to come we observe that these
operations enjoy a duality with processes very much like the duality
between vectors and maps from vectors to scalars.

Further, because the calculus is essentially higher-order, we have a
correspondence between contexts and processes. More specifically,
given a name $x$ and a context $M$ we can construct $M^{*}_{x}$ such
that 

\begin{mathpar}
  M^{*}_{x} | \lift{x}{P} \red M[P]
\end{mathpar}

namely,

\begin{mathpar}
  M^{*}_{x} := x?(u).M[\dropn{u}]
\end{mathpar}

The dependence of $M^{*}_{x}$ on a name makes it an abstraction, 

\begin{mathpar}
  M^{*} := (x)x?(u).M[\dropn{u}]
\end{mathpar}

\subsection{Additional notation}

It will sometimes be convenient to denote the process a name
quotes. We already have the notation $x = \quotep{P}$, but it will be
convenient to introduce an alternate notation, $\procn{x}$, when we
want to emphasize the connection to the use of the name. Note that, by
virtue of name equivalence, $\quotep{\procn{x}} \nameeq x$; so, the
notation is consistent with previous definitions.

Further, because names have structure it is possible to effect
substitutions on the basis of that structure. This means we need to
upgrade our notation for substitutions, which we accomplish by
adapting comprehension notation. Thus,

\begin{mathpar}
  P\{ y / x : x \in S \}
\end{mathpar}

is interpreted to mean the process derived from P by replacing (in a
capture-avoiding manner) each occurrence of $x$ in $S$ by $y$. For example,

\begin{mathpar}
  P\{ \quotep{\procn{x}|\procn{x}} / x : x \in \freenames{P} \}
\end{mathpar}

will replace each (occurrence) of a free name $x$ in $P$ by
$\quotep{\procn{x}|\procn{x}}$.

Also, we will avail ourselves of the notation $x^{L}$ and $x^{R}$ to
denote injections of a name into disjoint copies of the name
space. There are numerous ways to accomplish this. One example can be
found in \cite{MeredithR05}. This notation overloads to vectors of
names: $\vec{x}^{\pi} := (x_{i}^{\pi} \; : \; 0 \leq i < |\vec{x}| )$ where $\pi \in \{L,R\}$.

We also use $P^{\Box} := P|\Box$.

In \cite{MeredithR05} an interpretation of the new operator is
given. It turns out that there are several possible interpretations
all enjoying the requisite algebraic properties of the operator (see
\cite{milner91polyadicpi}). We will therefore make liberal use of
$(\nu\; \vec{x})P$.

% subsection the_syntax_and_semantics_of_the_notation_system (end)   

\input{qm2pi.qmops} 

\input{qm2pi.sterngerlach} 

\input{qm2pi.metric} 

% section concurrent_process_calculi (end)

%\input{qm2pi.proofsketch}

% section proof sketch (end)

%\input{qm2pi.slviaknots} 

% section spatial logic via knots (end)

\input{qm2pi.conclusion}

% section conclusion (end)

%\input{qm2pi.dtcodes} 

% section wiring algorithm (end)

\input{qm2pi.ack} 

% section acknowledgments (end)

\newpage


\bibliographystyle{plain}   
\bibliography{../../biblios/main.bib}

\input{qm2pi.rhodetails}

\end{document}

 

%\documentclass[12pt]{llncs}
%\documentclass{jktr}

\usepackage[pdftex]{hyperref}                   
\usepackage {listings}
\usepackage {mathpartir}
\usepackage{bcprules}
%\usepackage{listings}
                       
\usepackage{graphicx} 
%\usepackage[margins=2.5cm,nohead,nofoot]{geometry}
%\usepackage{geometry}
\usepackage{amsfonts}
\usepackage{amstext}
\usepackage{latexsym}
\usepackage{amssymb}
\usepackage{color}


%\include{myPreamble}
\include{qm2pi.local} 

%\ifpdf
%\usepackage[pdftex]{graphicx}
%\else
%\usepackage{graphicx}
%\fi

 % \ifpdf
%  \usepackage{pdfsync}
%  \if


%\title{Brief Article}
%\author{David F. Snyder}
%\author{L.G. Meredith}

%\address{Dept. of Math., Texas State University--San Marcos, San Marcos, TX 78666}
       
\pagestyle{empty}


\begin{document}

\lstset{language=[Objective]Caml,frame=shadowbox}

\input{qm2pi.front}

% section front matter (end)

\input{qm2pi.intro} 
 
% section introduction (end)

% \input{qm2pi.knotations} 

% section notation (end)

\input{qm2pi.process.calculi} 

% section concurrent_process_calculi_and_spatial_logics_ (end)
    
%\input{qm2pi.knots2pi} 

%\input{qm2pi.trefoil} 

%\input{qm2pi.mainthm} 

% subsection basic_interpretation (end)

%\input{qm2pi.rho.presentation} 
\subsection{The syntax and semantics of the notation system}\label{sub:the_syntax_and_semantics_of_the_notation_system} % (fold)

We now summarize a technical presentation of the calculus that
embodies our theory of dynamics. The typical presentation of such a
calculus follows the style of giving generators and relations on
them. The grammar, below, describing term constructors, freely
generates the set of processes, $\Proc$. This set is then quotiented
by a relation known as structural congruence and it is over this set
that the notion of dynamics is expressed. This presentation is
essentially that of \cite{MeredithR05} with the addition of
polyadicity and summation. For readability we have relegated some of
the technical subtleties to an appendix.

\subsubsection{Process grammar}\label{subsub:process_grammar}

\begin{mathpar}
  \inferrule* [lab=synchronization] {} {{M} \bc \pzero \;|\; x?F \;|\; x!C }
  \and
  \inferrule* [lab=abstraction] {} {{F} \bc (x)P}
  \and
  \inferrule* [lab=concretion] {} {{C} \bc \langle Q \rangle}
  \and
  \inferrule* [lab=process] {} {{P,Q} \bc M \;| \;P|Q \;|\; @{x}}
  \and
  \inferrule* [lab=name] {} {{x} \bc \quotep{P}}
\end{mathpar} 

Note that $\vec{x}$ (resp. $\vec{P}$) denotes a vector of names
(resp. processes) of length $|\vec{x}|$ (resp. $|\vec{P}|$). We adopt
the following useful abbreviations.

\begin{mathpar}
   x?(\vec{y}).P := x.(\vec{y})P \and  x\clift{\vec{P}} := x.\clift{\vec{P}}
   \and x!(y) := \lift{x}{\dropn{y}}
   \and \Pi_{i=0}^{n-1}P_i := P_0 | \ldots | P_{n-1}
\end{mathpar}

\subsubsection{Structural congruence}

\paragraph{Free and bound names and alpha-equivalence.} At the
core of structural equivalence is alpha-equivalence which identifies
process that are the same up to a change of variable. Formally, we
recognize the distinction between free and bound names. The free names
of a process, $\freenames{P}$, may be calculated recursively as
follows:

\begin{mathpar}
\freenames{\pzero} := \emptyset
  \and \\
  \freenames{x?(y).P} := \{ x \} \cup (\freenames{P} \setminus \{ y \})
  \and 
  \freenames{x!\langle P \rangle} := \{ x \} \cup \{ P \} 
  \and \\
  \freenames{P|Q} := \freenames{P} \cup \freenames{Q}
  \and \\
  \freenames{@{x}} := \{ x \}
\end{mathpar}

$\pi$
$\quotep{\pi}$

$\freenames{-} : \pi \to \mathcal{P}(\quotep{\pi})$

\begin{eqnarray*}
  \freenames{\pzero} & := & \emptyset \\
  \freenames{x?(y).P} & := & \{ x \} \cup (\freenames{P} \setminus \{ y \}) \\
  \freenames{x!\langle P \rangle} & := & \{ x \} \cup \{ P \} \\
  \freenames{P|Q} & := & \freenames{P} \cup \freenames{Q} \\
  \freenames{\dropn{x}} & := & \{ x \}
\end{eqnarray*}

The bound names of a process, $\boundnames{P}$, are those names occurring in $P$
that are not free. For example, in $x?(y).0$, the name $x$ is free, while $y$ is bound.

\begin{mathpar}
  \inferrule* [lab=monoidal-laws] {} { P|Q \equiv Q|P \and P|0 \equiv P \and P|(Q|R) \equiv (P|Q)|R }
\end{mathpar}

\begin{mathpar}
  \inferrule* [lab=alpha-equivalence] {} { (x)P \equiv (y)P\{y/x\} \and y \not\in \freenames{P} }
\end{mathpar}

\begin{definition}
Then two processes, $P,Q$, are alpha-equivalent if $P = Q\{\vec{y}/\vec{x}\}$ for
some $\vec{x} \in \boundnames{Q},\vec{y} \in \boundnames{P}$, where $Q\{\vec{y}/\vec{x}\}$
denotes the capture-avoiding substitution of $\vec{y}$ for $\vec{x}$ in $Q$.
\end{definition}

\begin{definition}
  The {\em structural congruence} \cite{SangiorgiWalker} , $\equiv$,
  between processes is the least congruence containing
  alpha-equivalence, satisfying the abelian monoid laws
  (associativity, commutativity and $\pzero$ as identity) for parallel
  composition $|$ and for summation $+$.
\end{definition}

\subsection{Name equivalence}

We take name equivalence, written $\nameeq$, to be the smallest
equivalence relation generated by the following rules.

\begin{mathpar}
\inferrule*[lab=Quote-drop]
{ }
{ \quotep{@{x}} \nameeq x }

\inferrule*[lab=Struct-equiv]
{ P \scong Q }
{ \quotep{P} \nameeq \quotep{Q} }
\end{mathpar}

The astute reader will have noticed that the mutual recursion of names
and processes imposes a mutual recursion on alpha-equivalence and
structural equivalence via name-equivalence. Fortunately, all of this
works out pleasantly and we may calculate in the natural way, free of
concern. The reader interested in the details is referred to the
appendix \ref{appendix:rho_details}.

\subsection{Substitution}

We use $\Proc$ for the set of processes, $\QProc$ for the set of
names, and $\id{\{}\vec{y} / \vec{x} \id{\}}$ to denote partial maps,
$s : \QProc \rightarrow \QProc$. A map, $s$ lifts, uniquely, to a map
on process terms, $\widehat{s} : \Proc \rightarrow \Proc$ by the
following equations.

\begin{mathpar}
  (0) \psubstp{Q}{P} := 0 \\
  (R \juxtap S) \psubstp{Q}{P}
  :=    
  (R)\psubstp{Q}{P} \juxtap (S) \psubstp{Q}{P} \\
  (x?(y).R) \psubstp{Q}{P}    
  :=    
  (x)\substp{Q}{P} (z)\concat( (R \psubstn{z}{y}) \psubstp{Q}{P} ) \\
  (\lift{x}{R}) \psubstp{Q}{P}  
  :=
  \lift{(x)\substp{Q}{P}}{ R \psubstp{Q}{P} } \\
%   (\dropn{x})  \psubstp{Q}{P}       
%   := 
%   \left\{ 
%     \begin{array}{ccc} 
%       \dropn{\quotep{Q}} & & x \nameeq \quotep{P} \\
%       \dropn{x} & & otherwise \\
%     \end{array}
%   \right. 
  (\dropn{x})  \psubstp{Q}{P}       
  := 
  \left\{ 
    \begin{array}{ccc} 
      Q & & x \nameeq \quotep{P} \\
      \dropn{x} & & otherwise \\
    \end{array}
  \right.
\end{mathpar}
 

where

\begin{eqnarray}
  (x)\id{\{} \lpquote Q \rpquote / \lpquote P \rpquote \id{\}}            = 
  \left\{ 
    \begin{array}{ccc}
      \lpquote Q \rpquote & & x \nameeq \lpquote P \rpquote \\
      x & & otherwise \\
    \end{array}
  \right. \nonumber
\end{eqnarray}

and $z$ is chosen distinct from $\quotep{P}$, $\quotep{Q}$, the free
names in $Q$, and all the names in $R$. Our $\alpha$-equivalence will
be built in the standard way from this substitution.

\begin{remark}\label{rem:no_self_referential_names}
  One consequence of these definitions is that $\forall P. \quotep{P}
  \not\in \freenames{P}$.
\end{remark}

\subsection{ Dynamic quote: an example }

Anticipating something of what's to come, consider applying the
substitution, $\widehat{\id{\{}u / z \id{\}}}$, to the following pair
of processes, $\lift{w}{y!(z)}$ and $w[ \lpquote y!(z) \rpquote ]$.

\begin{eqnarray}
	\lift{w}{y!(z)}\widehat{\id{\{}u / z \id{\}}}
		& = &
		\lift{w}{y!(u)} \nonumber\\
	w[ \lpquote y!(z) \rpquote ] \widehat{ \id{\{}u / z \id{\}} }
		& = &
		w[ \lpquote y!(z) \rpquote ] \nonumber
\end{eqnarray}

Because the body of the process between quotes is impervious to
substitution, we get radically different answers. In fact, by
examining the first process in an input context,
e.g. $x?(z).\lift{w}{y!(z)}$, we see that the process under the lift
operator may be shaped by prefixed inputs binding a name inside it. In
this sense, the lift operator will be seen as a way to dynamically
construct processes before reifying them as names.

Finally equipped with these standard features we can present the
dynamics of the calculus.

\subsubsection{Operational semantics} 

Finally, we introduce the computational dynamics. What marks these
algebras as distinct from other more traditionally studied algebraic
structures, e.g. vector spaces or polynomial rings, is the manner in
which dynamics is captured. In traditional structures, dynamics is typically
expressed through morphisms between such structures, as in linear maps
between vector spaces or morphisms between rings. In algebras
associated with the semantics of computation, the dynamics is
expressed as part of the algebraic structure itself, through a
reduction reduction relation typically denoted by $\red$. Below, we
give a recursive presentation of this relation for the calculus used
in the encoding.

$\red \subseteq \pi \times \pi$
$\red : \pi \to \mathcal{P}(\pi)$

\begin{mathpar}
  \inferrule* [lab=Comm] { \textsf{match}( x_{src}, x_{trgt} ) } { x_{trgt}?(y)P \; | \; x_{src}!\langle {Q} \rangle \red P\{\quotep{Q}/y}\} }
  \and \\
  \inferrule* [lab=Par] {{P} \red {P}'} {{{P} | {Q}} \red {{P}' | {Q}}}
  \and
  \inferrule* [lab=Equiv]{{{P} \scong {P}'} \andalso {{P}' \red {Q}'} \andalso {{Q}' \scong {Q}}}{{P} \red {Q}}
\end{mathpar}

\begin{eqnarray*}
  match_{\equiv} (\quotep{P},\quotep{Q}) & := & P \equiv Q \\
  match_{\dagger}(\quotep{P},\quotep{Q}) & := & \forall R. P|Q \red^{*} R => R \red^{*} 0 \\
  match_{K}(\quotep{P},\quotep{Q}) & := & K \mbox{ for some context } K
\end{eqnarray*}

$u?(x)P | u!\langle Q \rangle \red P\{\quotep{Q}/x\}$

%We write $\wred$ for $\red^*$, and $P\red$ if $\exists Q $ such that $ P \red Q$.
We write $P\red$ if $\exists Q $ such that $ P \red Q$ and $P\not\red$, otherwise.

\section{Replication}

As mentioned before, it is known that replication (and hence
recursion) can be implemented in a higher-order process algebra
\cite{SangiorgiWalker}. As our first example of calculation with the
machinery thus far presented we give the construction explicitly in
the {\rhoc}.

\begin{eqnarray}
	D_{x} & := & \prefix{x}{y}{(\binpar{\outputp{x}{y}}{@{y}})} \nonumber\\
	\bangp_{x}{P} & := & \binpar{{x}!\langle{\binpar{D_{x}}{P}}\rangle}{D_{x}} \nonumber
\end{eqnarray}

\begin{eqnarray}
	\bangp_{x}{P} & & \nonumber\\
	=
	& {x}!\langle{(\prefix{x}{y}{(\outputp{x}{y} | @{y})) | P}}\rangle 
	      | \prefix{x}{y}{(\outputp{x}{y} | @{y})} & \nonumber\\
	\red
	& (\outputp{x}{y} | @{y})\substn{\quotep{(\prefix{x}{y}{(@{y} | \outputp{x}{y})) | P}}}{y} & \nonumber\\
	=
	& \outputp{x}{\quotep{(\prefix{x}{y}{(\outputp{x}{y} | @{y})) | P}}}
	  | {(\prefix{x}{y}{(\outputp{x}{y} | @{y})) | P}} & \nonumber\\
	\red
	& \ldots & \nonumber\\
	\red^*
	& P | P | \ldots & \nonumber
\end{eqnarray}

Of course, this encoding, as an implementation, runs away, unfolding
$\bangp{P}$ eagerly. A lazier and more implementable replication
operator, restricted to input-guarded processes, may be obtained as follows.

\begin{eqnarray}
\bangp{\prefix{u}{v}{P}} 
	:= 
	\binpar{\lift{x}{\prefix{u}{v}{(\binpar{D(x)}{P})}}}{D(x)} \nonumber
\end{eqnarray}

\begin{remark}
  Note that the lazier definition still does not deal with summation
  or mixed summation (i.e. sums over input and output). The reader is
  invited to construct definitions of replication that deal with these
  features. 

  Further, the definitions are parameterized in a name, $x$. Can you,
  gentle reader, make a definition that eliminates this parameter and
  guarantees no accidental interaction between the replication
  machinery and the process being replicated -- i.e. no accidental
  sharing of names used by the process to get its work done and the
  name(s) used by the replication to effect copying. This latter
  revision of the definition of replication is crucial to obtaining
  the expected identity $!!P \sim !P$.
\end{remark}

\begin{remark}\label{rem:paradoxical_combinator}
  The reader familiar with the lambda calculus will have noticed the
  similarity between $D$ and the paradoxical combinator.

  [Ed. note: the existence of this seems to suggest we have to be more
  restrictive on the set of processes and names we admit if we are to
  support no-cloning.]
\end{remark}

\subsubsection{Bisimulation}

The computational dynamics gives rise to another kind of equivalence,
the equivalence of computational behavior. As previously mentioned
this is typically captured \emph{via} some form of bisimulation.

% The notion we use in this paper is weak barbed bisimulation
% \cite{milner91polyadicpi}.

The notion we use in this paper is derived from weak barbed
bisimulation \cite{milner91polyadicpi}. 

\begin{definition}
An \emph{observation relation}, $\downarrow_{\mathcal N}$, over a set
of names, $\mathcal N$, is the smallest relation satisfying the rules
below.

\infrule[Out-barb]{y \in {\mathcal N}, \; x \nameeq y}
		  {\outputp{x}{v} \downarrow_{\mathcal N} x}
\infrule[Par-barb]{\mbox{$P\downarrow_{\mathcal N} x$ or $Q\downarrow_{\mathcal N} x$}}
		  {\binpar{P}{Q} \downarrow_{\mathcal N} x}

We write $P \Downarrow_{\mathcal N} x$ if there is $Q$ such that 
$P \wred Q$ and $Q \downarrow_{\mathcal N} x$.
\end{definition}

\begin{definition}
%\label{def.bbisim}
An  ${\mathcal N}$-\emph{barbed bisimulation} over a set of names, ${\mathcal N}$, is a symmetric binary relation 
${\mathcal S}_{\mathcal N}$ between agents such that $P\rel{S}_{\mathcal N}Q$ implies:
\begin{enumerate}
\item If $P \red P'$ then $Q \wred Q'$ and $P'\rel{S}_{\mathcal N} Q'$.
\item If $P\downarrow_{\mathcal N} x$, then $Q\Downarrow_{\mathcal N} x$.
\end{enumerate}
$P$ is ${\mathcal N}$-barbed bisimilar to $Q$, written
$P \wbbisim_{\mathcal N} Q$, if $P \rel{S}_{\mathcal N} Q$ for some ${\mathcal N}$-barbed bisimulation ${\mathcal S}_{\mathcal N}$.
\end{definition}

$\mathcal{R} \subseteq \pi \times \pi$

$P \mathcal{R} Q => \forall P'. P \red P' \Rightarrow \exists Q'. Q \red Q', P' \mathcal{R} Q'$

$P \vdash x \Rightarrow Q \vdash x$

\begin{mathpar}
  \inferrule*[lab=Out-barb]{x \nameeq y}{{y}!\langle{Q}\rangle \vdash x}
  \and
  \inferrule*[lab=Par-barb]{\mbox{$P\vdash x$ or $Q\vdash x$}}{\binpar{P}{Q} \vdash x}
\end{mathpar}

\subsubsection{Contexts}

One of the principle advantages of computational calculi like the
$\pi$-calculus is a well-defined notion of context,
contextual-equivalence and a correlation between
contextual-equivalence and notions of bisimulation. The notion of
context allows the decomposition of a process into (sub-)process and
its syntactic environment, its context. Thus, a context may be
thought of as a process with a ``hole'' (written $\Box$) in it. The
application of a context $M$ to a process $P$, written $M[P]$, is
tantamount to filling the hole in $M$ with $P$. In this paper we do
not need the full weight of this theory, but do make use of the notion
of context in the proof the main theorem. 

\begin{mathpar}
  \inferrule* [lab=summation] {} {{M_{M},M_{N}} \bc \Box \;|\; x.M_{A} \;|\; M_{M}+M_{N}}
  \and
  \inferrule* [lab=agent] {} {{M_{A}} \bc (\vec{x})M_{P} \;| \; \clift{P_0,\ldots,M_{P},\ldots,P_N}}
  \and \\
  \inferrule* [lab=process] {} {{M_{P}} \bc M_{N} \;| \;P|M_{P} }
\end{mathpar} 

\begin{mathpar}
  \inferrule* [lab=sychronization] {} {M_{N} \bc \Box \;|\; x?M_{F} \;|\; x!M_{C}}
  \and
  \inferrule* [lab=abstraction] {} {{M_{F}} \bc (x)M_{P} }
  \and
  \inferrule* [lab=concretion] {} {{M_{C}} \bc \langle M_{P} \rangle }
  \and \\
  \inferrule* [lab=process] {} {{M_{P}} \bc M_{N} \;| \;P|M_{P} }
\end{mathpar}

\begin{definition}[contextual application] Given a context $M$, and
  process $P$, we define the \emph{contextual application}, $M[P] :=
  M\{P/\Box\}$. That is, the contextual application of M to P is the
  substitution of $P$ for $\Box$ in $M$.
\end{definition}

$\meaningof{-} : L \to \mathcal{P}(\pi)$

\begin{mathpar}
  \inferrule* [lab=collection] {} {\meaningof{true} = \pi, \and \meaningof{~E} = \pi \setminus \meaningof{E}, \and \meaningof{E_{1} \& E_{2}} = \meaningof{E_{1}} \cap \meaningof{E_{2}}}
\end{mathpar}

\begin{mathpar}
  \inferrule* [lab=structure] {} {\meaningof{0} = \{ P \in \pi | P \equiv 0 \}, \and \\ \meaningof{E_1 | E_2} = \{ P \in \pi | P \equiv P_{1} | P_{2}, P_{1} \in \meaningof{E_{1}}, P_{2} \in \meaningof{E_2}\} }
\end{mathpar}

\begin{mathpar}
 \inferrule* [lab=behavior] {} {\meaningof{\langle a?b \rangle E} = \{ P \in \pi | P \equiv Q | u?(y)P', \\ \and \\\\ \and \\ \;\;\; u \in \meaningof{a}, \forall z.P'\{z/y\} \in \meaningof{E\{z/b\}}\}, \and \\ \meaningof{a!E} = \{ P \in \pi | P \equiv Q | x!\langle P' \rangle, x \in \meaningof{a} P' \in \meaningof{E}\} }
\end{mathpar}

\begin{mathpar}
 \inferrule* [lab=nominal] {} {\meaningof{\quotep{E}} = \{ \quotep{P} \in \quotep{\pi} | P \in \meaningof{E} \}, \and \meaningof{\quotep{P}} = \{ \quotep{Q} \in \quotep{\pi} | P \equiv Q \} \and \\ \meaningof{@\quotep{E}} = \{ P \in \pi | P \equiv @x, x \in \meaningof{E} \}}
\end{mathpar}

\begin{eqnarray*}
  \\
  \meaningof{-} : TS \to ST
\end{eqnarray*}

\begin{eqnarray*}
  \\
  L : TS \to ST
\end{eqnarray*}

\begin{eqnarray*}
  \\
  P \models E \iff P \in \meaningof{E}
\end{eqnarray*}

\begin{eqnarray*}
  P \approx_{L} Q \iff \forall E \in L. P \models E \iff Q \models E
\end{eqnarray*}

\begin{eqnarray*}
  P \approx_{K} Q
\end{eqnarray*}

\begin{eqnarray*}
  P \approx Q
\end{eqnarray*}

$\approx_{K} = \approx = \approx_{L}$

\subsubsection{Contextual duality}

Note that contexts extend the quotation operation to a family of
operations from processes to names. Given a context, $M$, we can
define a \emph{nominal context}, $\quotep{M}$ by $\quotep{M}[P] :=
\quotep{M[P]}$. To foreshadow what is to come we observe that these
operations enjoy a duality with processes very much like the duality
between vectors and maps from vectors to scalars.

Further, because the calculus is essentially higher-order, we have a
correspondence between contexts and processes. More specifically,
given a name $x$ and a context $M$ we can construct $M^{*}_{x}$ such
that 

\begin{mathpar}
  M^{*}_{x} | \lift{x}{P} \red M[P]
\end{mathpar}

namely,

\begin{mathpar}
  M^{*}_{x} := x?(u).M[\dropn{u}]
\end{mathpar}

The dependence of $M^{*}_{x}$ on a name makes it an abstraction, 

\begin{mathpar}
  M^{*} := (x)x?(u).M[\dropn{u}]
\end{mathpar}

\subsection{Additional notation}

It will sometimes be convenient to denote the process a name
quotes. We already have the notation $x = \quotep{P}$, but it will be
convenient to introduce an alternate notation, $\procn{x}$, when we
want to emphasize the connection to the use of the name. Note that, by
virtue of name equivalence, $\quotep{\procn{x}} \nameeq x$; so, the
notation is consistent with previous definitions.

Further, because names have structure it is possible to effect
substitutions on the basis of that structure. This means we need to
upgrade our notation for substitutions, which we accomplish by
adapting comprehension notation. Thus,

\begin{mathpar}
  P\{ y / x : x \in S \}
\end{mathpar}

is interpreted to mean the process derived from P by replacing (in a
capture-avoiding manner) each occurrence of $x$ in $S$ by $y$. For example,

\begin{mathpar}
  P\{ \quotep{\procn{x}|\procn{x}} / x : x \in \freenames{P} \}
\end{mathpar}

will replace each (occurrence) of a free name $x$ in $P$ by
$\quotep{\procn{x}|\procn{x}}$.

Also, we will avail ourselves of the notation $x^{L}$ and $x^{R}$ to
denote injections of a name into disjoint copies of the name
space. There are numerous ways to accomplish this. One example can be
found in \cite{MeredithR05}. This notation overloads to vectors of
names: $\vec{x}^{\pi} := (x_{i}^{\pi} \; : \; 0 \leq i < |\vec{x}| )$ where $\pi \in \{L,R\}$.

We also use $P^{\Box} := P|\Box$.

In \cite{MeredithR05} an interpretation of the new operator is
given. It turns out that there are several possible interpretations
all enjoying the requisite algebraic properties of the operator (see
\cite{milner91polyadicpi}). We will therefore make liberal use of
$(\nu\; \vec{x})P$.

% subsection the_syntax_and_semantics_of_the_notation_system (end)   

\input{qm2pi.qmops} 

\input{qm2pi.sterngerlach} 

\input{qm2pi.metric} 

% section concurrent_process_calculi (end)

%\input{qm2pi.proofsketch}

% section proof sketch (end)

%\input{qm2pi.slviaknots} 

% section spatial logic via knots (end)

\input{qm2pi.conclusion}

% section conclusion (end)

%\input{qm2pi.dtcodes} 

% section wiring algorithm (end)

\input{qm2pi.ack} 

% section acknowledgments (end)

\newpage


\bibliographystyle{plain}   
\bibliography{../../biblios/main.bib}

\input{qm2pi.rhodetails}

\end{document}

 

%\documentclass[12pt]{llncs}
%\documentclass{jktr}

\usepackage[pdftex]{hyperref}                   
\usepackage {listings}
\usepackage {mathpartir}
\usepackage{bcprules}
%\usepackage{listings}
                       
\usepackage{graphicx} 
%\usepackage[margins=2.5cm,nohead,nofoot]{geometry}
%\usepackage{geometry}
\usepackage{amsfonts}
\usepackage{amstext}
\usepackage{latexsym}
\usepackage{amssymb}
\usepackage{color}


%\include{myPreamble}
\include{qm2pi.local} 

%\ifpdf
%\usepackage[pdftex]{graphicx}
%\else
%\usepackage{graphicx}
%\fi

 % \ifpdf
%  \usepackage{pdfsync}
%  \if


%\title{Brief Article}
%\author{David F. Snyder}
%\author{L.G. Meredith}

%\address{Dept. of Math., Texas State University--San Marcos, San Marcos, TX 78666}
       
\pagestyle{empty}


\begin{document}

\lstset{language=[Objective]Caml,frame=shadowbox}

\input{qm2pi.front}

% section front matter (end)

\input{qm2pi.intro} 
 
% section introduction (end)

% \input{qm2pi.knotations} 

% section notation (end)

\input{qm2pi.process.calculi} 

% section concurrent_process_calculi_and_spatial_logics_ (end)
    
%\input{qm2pi.knots2pi} 

%\input{qm2pi.trefoil} 

%\input{qm2pi.mainthm} 

% subsection basic_interpretation (end)

%\input{qm2pi.rho.presentation} 
\subsection{The syntax and semantics of the notation system}\label{sub:the_syntax_and_semantics_of_the_notation_system} % (fold)

We now summarize a technical presentation of the calculus that
embodies our theory of dynamics. The typical presentation of such a
calculus follows the style of giving generators and relations on
them. The grammar, below, describing term constructors, freely
generates the set of processes, $\Proc$. This set is then quotiented
by a relation known as structural congruence and it is over this set
that the notion of dynamics is expressed. This presentation is
essentially that of \cite{MeredithR05} with the addition of
polyadicity and summation. For readability we have relegated some of
the technical subtleties to an appendix.

\subsubsection{Process grammar}\label{subsub:process_grammar}

\begin{mathpar}
  \inferrule* [lab=synchronization] {} {{M} \bc \pzero \;|\; x?F \;|\; x!C }
  \and
  \inferrule* [lab=abstraction] {} {{F} \bc (x)P}
  \and
  \inferrule* [lab=concretion] {} {{C} \bc \langle Q \rangle}
  \and
  \inferrule* [lab=process] {} {{P,Q} \bc M \;| \;P|Q \;|\; @{x}}
  \and
  \inferrule* [lab=name] {} {{x} \bc \quotep{P}}
\end{mathpar} 

Note that $\vec{x}$ (resp. $\vec{P}$) denotes a vector of names
(resp. processes) of length $|\vec{x}|$ (resp. $|\vec{P}|$). We adopt
the following useful abbreviations.

\begin{mathpar}
   x?(\vec{y}).P := x.(\vec{y})P \and  x\clift{\vec{P}} := x.\clift{\vec{P}}
   \and x!(y) := \lift{x}{\dropn{y}}
   \and \Pi_{i=0}^{n-1}P_i := P_0 | \ldots | P_{n-1}
\end{mathpar}

\subsubsection{Structural congruence}

\paragraph{Free and bound names and alpha-equivalence.} At the
core of structural equivalence is alpha-equivalence which identifies
process that are the same up to a change of variable. Formally, we
recognize the distinction between free and bound names. The free names
of a process, $\freenames{P}$, may be calculated recursively as
follows:

\begin{mathpar}
\freenames{\pzero} := \emptyset
  \and \\
  \freenames{x?(y).P} := \{ x \} \cup (\freenames{P} \setminus \{ y \})
  \and 
  \freenames{x!\langle P \rangle} := \{ x \} \cup \{ P \} 
  \and \\
  \freenames{P|Q} := \freenames{P} \cup \freenames{Q}
  \and \\
  \freenames{@{x}} := \{ x \}
\end{mathpar}

$\pi$
$\quotep{\pi}$

$\freenames{-} : \pi \to \mathcal{P}(\quotep{\pi})$

\begin{eqnarray*}
  \freenames{\pzero} & := & \emptyset \\
  \freenames{x?(y).P} & := & \{ x \} \cup (\freenames{P} \setminus \{ y \}) \\
  \freenames{x!\langle P \rangle} & := & \{ x \} \cup \{ P \} \\
  \freenames{P|Q} & := & \freenames{P} \cup \freenames{Q} \\
  \freenames{\dropn{x}} & := & \{ x \}
\end{eqnarray*}

The bound names of a process, $\boundnames{P}$, are those names occurring in $P$
that are not free. For example, in $x?(y).0$, the name $x$ is free, while $y$ is bound.

\begin{mathpar}
  \inferrule* [lab=monoidal-laws] {} { P|Q \equiv Q|P \and P|0 \equiv P \and P|(Q|R) \equiv (P|Q)|R }
\end{mathpar}

\begin{mathpar}
  \inferrule* [lab=alpha-equivalence] {} { (x)P \equiv (y)P\{y/x\} \and y \not\in \freenames{P} }
\end{mathpar}

\begin{definition}
Then two processes, $P,Q$, are alpha-equivalent if $P = Q\{\vec{y}/\vec{x}\}$ for
some $\vec{x} \in \boundnames{Q},\vec{y} \in \boundnames{P}$, where $Q\{\vec{y}/\vec{x}\}$
denotes the capture-avoiding substitution of $\vec{y}$ for $\vec{x}$ in $Q$.
\end{definition}

\begin{definition}
  The {\em structural congruence} \cite{SangiorgiWalker} , $\equiv$,
  between processes is the least congruence containing
  alpha-equivalence, satisfying the abelian monoid laws
  (associativity, commutativity and $\pzero$ as identity) for parallel
  composition $|$ and for summation $+$.
\end{definition}

\subsection{Name equivalence}

We take name equivalence, written $\nameeq$, to be the smallest
equivalence relation generated by the following rules.

\begin{mathpar}
\inferrule*[lab=Quote-drop]
{ }
{ \quotep{@{x}} \nameeq x }

\inferrule*[lab=Struct-equiv]
{ P \scong Q }
{ \quotep{P} \nameeq \quotep{Q} }
\end{mathpar}

The astute reader will have noticed that the mutual recursion of names
and processes imposes a mutual recursion on alpha-equivalence and
structural equivalence via name-equivalence. Fortunately, all of this
works out pleasantly and we may calculate in the natural way, free of
concern. The reader interested in the details is referred to the
appendix \ref{appendix:rho_details}.

\subsection{Substitution}

We use $\Proc$ for the set of processes, $\QProc$ for the set of
names, and $\id{\{}\vec{y} / \vec{x} \id{\}}$ to denote partial maps,
$s : \QProc \rightarrow \QProc$. A map, $s$ lifts, uniquely, to a map
on process terms, $\widehat{s} : \Proc \rightarrow \Proc$ by the
following equations.

\begin{mathpar}
  (0) \psubstp{Q}{P} := 0 \\
  (R \juxtap S) \psubstp{Q}{P}
  :=    
  (R)\psubstp{Q}{P} \juxtap (S) \psubstp{Q}{P} \\
  (x?(y).R) \psubstp{Q}{P}    
  :=    
  (x)\substp{Q}{P} (z)\concat( (R \psubstn{z}{y}) \psubstp{Q}{P} ) \\
  (\lift{x}{R}) \psubstp{Q}{P}  
  :=
  \lift{(x)\substp{Q}{P}}{ R \psubstp{Q}{P} } \\
%   (\dropn{x})  \psubstp{Q}{P}       
%   := 
%   \left\{ 
%     \begin{array}{ccc} 
%       \dropn{\quotep{Q}} & & x \nameeq \quotep{P} \\
%       \dropn{x} & & otherwise \\
%     \end{array}
%   \right. 
  (\dropn{x})  \psubstp{Q}{P}       
  := 
  \left\{ 
    \begin{array}{ccc} 
      Q & & x \nameeq \quotep{P} \\
      \dropn{x} & & otherwise \\
    \end{array}
  \right.
\end{mathpar}
 

where

\begin{eqnarray}
  (x)\id{\{} \lpquote Q \rpquote / \lpquote P \rpquote \id{\}}            = 
  \left\{ 
    \begin{array}{ccc}
      \lpquote Q \rpquote & & x \nameeq \lpquote P \rpquote \\
      x & & otherwise \\
    \end{array}
  \right. \nonumber
\end{eqnarray}

and $z$ is chosen distinct from $\quotep{P}$, $\quotep{Q}$, the free
names in $Q$, and all the names in $R$. Our $\alpha$-equivalence will
be built in the standard way from this substitution.

\begin{remark}\label{rem:no_self_referential_names}
  One consequence of these definitions is that $\forall P. \quotep{P}
  \not\in \freenames{P}$.
\end{remark}

\subsection{ Dynamic quote: an example }

Anticipating something of what's to come, consider applying the
substitution, $\widehat{\id{\{}u / z \id{\}}}$, to the following pair
of processes, $\lift{w}{y!(z)}$ and $w[ \lpquote y!(z) \rpquote ]$.

\begin{eqnarray}
	\lift{w}{y!(z)}\widehat{\id{\{}u / z \id{\}}}
		& = &
		\lift{w}{y!(u)} \nonumber\\
	w[ \lpquote y!(z) \rpquote ] \widehat{ \id{\{}u / z \id{\}} }
		& = &
		w[ \lpquote y!(z) \rpquote ] \nonumber
\end{eqnarray}

Because the body of the process between quotes is impervious to
substitution, we get radically different answers. In fact, by
examining the first process in an input context,
e.g. $x?(z).\lift{w}{y!(z)}$, we see that the process under the lift
operator may be shaped by prefixed inputs binding a name inside it. In
this sense, the lift operator will be seen as a way to dynamically
construct processes before reifying them as names.

Finally equipped with these standard features we can present the
dynamics of the calculus.

\subsubsection{Operational semantics} 

Finally, we introduce the computational dynamics. What marks these
algebras as distinct from other more traditionally studied algebraic
structures, e.g. vector spaces or polynomial rings, is the manner in
which dynamics is captured. In traditional structures, dynamics is typically
expressed through morphisms between such structures, as in linear maps
between vector spaces or morphisms between rings. In algebras
associated with the semantics of computation, the dynamics is
expressed as part of the algebraic structure itself, through a
reduction reduction relation typically denoted by $\red$. Below, we
give a recursive presentation of this relation for the calculus used
in the encoding.

$\red \subseteq \pi \times \pi$
$\red : \pi \to \mathcal{P}(\pi)$

\begin{mathpar}
  \inferrule* [lab=Comm] { \textsf{match}( x_{src}, x_{trgt} ) } { x_{trgt}?(y)P \; | \; x_{src}!\langle {Q} \rangle \red P\{\quotep{Q}/y}\} }
  \and \\
  \inferrule* [lab=Par] {{P} \red {P}'} {{{P} | {Q}} \red {{P}' | {Q}}}
  \and
  \inferrule* [lab=Equiv]{{{P} \scong {P}'} \andalso {{P}' \red {Q}'} \andalso {{Q}' \scong {Q}}}{{P} \red {Q}}
\end{mathpar}

\begin{eqnarray*}
  match_{\equiv} (\quotep{P},\quotep{Q}) & := & P \equiv Q \\
  match_{\dagger}(\quotep{P},\quotep{Q}) & := & \forall R. P|Q \red^{*} R => R \red^{*} 0 \\
  match_{K}(\quotep{P},\quotep{Q}) & := & K \mbox{ for some context } K
\end{eqnarray*}

$u?(x)P | u!\langle Q \rangle \red P\{\quotep{Q}/x\}$

%We write $\wred$ for $\red^*$, and $P\red$ if $\exists Q $ such that $ P \red Q$.
We write $P\red$ if $\exists Q $ such that $ P \red Q$ and $P\not\red$, otherwise.

\section{Replication}

As mentioned before, it is known that replication (and hence
recursion) can be implemented in a higher-order process algebra
\cite{SangiorgiWalker}. As our first example of calculation with the
machinery thus far presented we give the construction explicitly in
the {\rhoc}.

\begin{eqnarray}
	D_{x} & := & \prefix{x}{y}{(\binpar{\outputp{x}{y}}{@{y}})} \nonumber\\
	\bangp_{x}{P} & := & \binpar{{x}!\langle{\binpar{D_{x}}{P}}\rangle}{D_{x}} \nonumber
\end{eqnarray}

\begin{eqnarray}
	\bangp_{x}{P} & & \nonumber\\
	=
	& {x}!\langle{(\prefix{x}{y}{(\outputp{x}{y} | @{y})) | P}}\rangle 
	      | \prefix{x}{y}{(\outputp{x}{y} | @{y})} & \nonumber\\
	\red
	& (\outputp{x}{y} | @{y})\substn{\quotep{(\prefix{x}{y}{(@{y} | \outputp{x}{y})) | P}}}{y} & \nonumber\\
	=
	& \outputp{x}{\quotep{(\prefix{x}{y}{(\outputp{x}{y} | @{y})) | P}}}
	  | {(\prefix{x}{y}{(\outputp{x}{y} | @{y})) | P}} & \nonumber\\
	\red
	& \ldots & \nonumber\\
	\red^*
	& P | P | \ldots & \nonumber
\end{eqnarray}

Of course, this encoding, as an implementation, runs away, unfolding
$\bangp{P}$ eagerly. A lazier and more implementable replication
operator, restricted to input-guarded processes, may be obtained as follows.

\begin{eqnarray}
\bangp{\prefix{u}{v}{P}} 
	:= 
	\binpar{\lift{x}{\prefix{u}{v}{(\binpar{D(x)}{P})}}}{D(x)} \nonumber
\end{eqnarray}

\begin{remark}
  Note that the lazier definition still does not deal with summation
  or mixed summation (i.e. sums over input and output). The reader is
  invited to construct definitions of replication that deal with these
  features. 

  Further, the definitions are parameterized in a name, $x$. Can you,
  gentle reader, make a definition that eliminates this parameter and
  guarantees no accidental interaction between the replication
  machinery and the process being replicated -- i.e. no accidental
  sharing of names used by the process to get its work done and the
  name(s) used by the replication to effect copying. This latter
  revision of the definition of replication is crucial to obtaining
  the expected identity $!!P \sim !P$.
\end{remark}

\begin{remark}\label{rem:paradoxical_combinator}
  The reader familiar with the lambda calculus will have noticed the
  similarity between $D$ and the paradoxical combinator.

  [Ed. note: the existence of this seems to suggest we have to be more
  restrictive on the set of processes and names we admit if we are to
  support no-cloning.]
\end{remark}

\subsubsection{Bisimulation}

The computational dynamics gives rise to another kind of equivalence,
the equivalence of computational behavior. As previously mentioned
this is typically captured \emph{via} some form of bisimulation.

% The notion we use in this paper is weak barbed bisimulation
% \cite{milner91polyadicpi}.

The notion we use in this paper is derived from weak barbed
bisimulation \cite{milner91polyadicpi}. 

\begin{definition}
An \emph{observation relation}, $\downarrow_{\mathcal N}$, over a set
of names, $\mathcal N$, is the smallest relation satisfying the rules
below.

\infrule[Out-barb]{y \in {\mathcal N}, \; x \nameeq y}
		  {\outputp{x}{v} \downarrow_{\mathcal N} x}
\infrule[Par-barb]{\mbox{$P\downarrow_{\mathcal N} x$ or $Q\downarrow_{\mathcal N} x$}}
		  {\binpar{P}{Q} \downarrow_{\mathcal N} x}

We write $P \Downarrow_{\mathcal N} x$ if there is $Q$ such that 
$P \wred Q$ and $Q \downarrow_{\mathcal N} x$.
\end{definition}

\begin{definition}
%\label{def.bbisim}
An  ${\mathcal N}$-\emph{barbed bisimulation} over a set of names, ${\mathcal N}$, is a symmetric binary relation 
${\mathcal S}_{\mathcal N}$ between agents such that $P\rel{S}_{\mathcal N}Q$ implies:
\begin{enumerate}
\item If $P \red P'$ then $Q \wred Q'$ and $P'\rel{S}_{\mathcal N} Q'$.
\item If $P\downarrow_{\mathcal N} x$, then $Q\Downarrow_{\mathcal N} x$.
\end{enumerate}
$P$ is ${\mathcal N}$-barbed bisimilar to $Q$, written
$P \wbbisim_{\mathcal N} Q$, if $P \rel{S}_{\mathcal N} Q$ for some ${\mathcal N}$-barbed bisimulation ${\mathcal S}_{\mathcal N}$.
\end{definition}

$\mathcal{R} \subseteq \pi \times \pi$

$P \mathcal{R} Q => \forall P'. P \red P' \Rightarrow \exists Q'. Q \red Q', P' \mathcal{R} Q'$

$P \vdash x \Rightarrow Q \vdash x$

\begin{mathpar}
  \inferrule*[lab=Out-barb]{x \nameeq y}{{y}!\langle{Q}\rangle \vdash x}
  \and
  \inferrule*[lab=Par-barb]{\mbox{$P\vdash x$ or $Q\vdash x$}}{\binpar{P}{Q} \vdash x}
\end{mathpar}

\subsubsection{Contexts}

One of the principle advantages of computational calculi like the
$\pi$-calculus is a well-defined notion of context,
contextual-equivalence and a correlation between
contextual-equivalence and notions of bisimulation. The notion of
context allows the decomposition of a process into (sub-)process and
its syntactic environment, its context. Thus, a context may be
thought of as a process with a ``hole'' (written $\Box$) in it. The
application of a context $M$ to a process $P$, written $M[P]$, is
tantamount to filling the hole in $M$ with $P$. In this paper we do
not need the full weight of this theory, but do make use of the notion
of context in the proof the main theorem. 

\begin{mathpar}
  \inferrule* [lab=summation] {} {{M_{M},M_{N}} \bc \Box \;|\; x.M_{A} \;|\; M_{M}+M_{N}}
  \and
  \inferrule* [lab=agent] {} {{M_{A}} \bc (\vec{x})M_{P} \;| \; \clift{P_0,\ldots,M_{P},\ldots,P_N}}
  \and \\
  \inferrule* [lab=process] {} {{M_{P}} \bc M_{N} \;| \;P|M_{P} }
\end{mathpar} 

\begin{mathpar}
  \inferrule* [lab=sychronization] {} {M_{N} \bc \Box \;|\; x?M_{F} \;|\; x!M_{C}}
  \and
  \inferrule* [lab=abstraction] {} {{M_{F}} \bc (x)M_{P} }
  \and
  \inferrule* [lab=concretion] {} {{M_{C}} \bc \langle M_{P} \rangle }
  \and \\
  \inferrule* [lab=process] {} {{M_{P}} \bc M_{N} \;| \;P|M_{P} }
\end{mathpar}

\begin{definition}[contextual application] Given a context $M$, and
  process $P$, we define the \emph{contextual application}, $M[P] :=
  M\{P/\Box\}$. That is, the contextual application of M to P is the
  substitution of $P$ for $\Box$ in $M$.
\end{definition}

$\meaningof{-} : L \to \mathcal{P}(\pi)$

\begin{mathpar}
  \inferrule* [lab=collection] {} {\meaningof{true} = \pi, \and \meaningof{~E} = \pi \setminus \meaningof{E}, \and \meaningof{E_{1} \& E_{2}} = \meaningof{E_{1}} \cap \meaningof{E_{2}}}
\end{mathpar}

\begin{mathpar}
  \inferrule* [lab=structure] {} {\meaningof{0} = \{ P \in \pi | P \equiv 0 \}, \and \\ \meaningof{E_1 | E_2} = \{ P \in \pi | P \equiv P_{1} | P_{2}, P_{1} \in \meaningof{E_{1}}, P_{2} \in \meaningof{E_2}\} }
\end{mathpar}

\begin{mathpar}
 \inferrule* [lab=behavior] {} {\meaningof{\langle a?b \rangle E} = \{ P \in \pi | P \equiv Q | u?(y)P', \\ \and \\\\ \and \\ \;\;\; u \in \meaningof{a}, \forall z.P'\{z/y\} \in \meaningof{E\{z/b\}}\}, \and \\ \meaningof{a!E} = \{ P \in \pi | P \equiv Q | x!\langle P' \rangle, x \in \meaningof{a} P' \in \meaningof{E}\} }
\end{mathpar}

\begin{mathpar}
 \inferrule* [lab=nominal] {} {\meaningof{\quotep{E}} = \{ \quotep{P} \in \quotep{\pi} | P \in \meaningof{E} \}, \and \meaningof{\quotep{P}} = \{ \quotep{Q} \in \quotep{\pi} | P \equiv Q \} \and \\ \meaningof{@\quotep{E}} = \{ P \in \pi | P \equiv @x, x \in \meaningof{E} \}}
\end{mathpar}

\begin{eqnarray*}
  \\
  \meaningof{-} : TS \to ST
\end{eqnarray*}

\begin{eqnarray*}
  \\
  L : TS \to ST
\end{eqnarray*}

\begin{eqnarray*}
  \\
  P \models E \iff P \in \meaningof{E}
\end{eqnarray*}

\begin{eqnarray*}
  P \approx_{L} Q \iff \forall E \in L. P \models E \iff Q \models E
\end{eqnarray*}

\begin{eqnarray*}
  P \approx_{K} Q
\end{eqnarray*}

\begin{eqnarray*}
  P \approx Q
\end{eqnarray*}

$\approx_{K} = \approx = \approx_{L}$

\subsubsection{Contextual duality}

Note that contexts extend the quotation operation to a family of
operations from processes to names. Given a context, $M$, we can
define a \emph{nominal context}, $\quotep{M}$ by $\quotep{M}[P] :=
\quotep{M[P]}$. To foreshadow what is to come we observe that these
operations enjoy a duality with processes very much like the duality
between vectors and maps from vectors to scalars.

Further, because the calculus is essentially higher-order, we have a
correspondence between contexts and processes. More specifically,
given a name $x$ and a context $M$ we can construct $M^{*}_{x}$ such
that 

\begin{mathpar}
  M^{*}_{x} | \lift{x}{P} \red M[P]
\end{mathpar}

namely,

\begin{mathpar}
  M^{*}_{x} := x?(u).M[\dropn{u}]
\end{mathpar}

The dependence of $M^{*}_{x}$ on a name makes it an abstraction, 

\begin{mathpar}
  M^{*} := (x)x?(u).M[\dropn{u}]
\end{mathpar}

\subsection{Additional notation}

It will sometimes be convenient to denote the process a name
quotes. We already have the notation $x = \quotep{P}$, but it will be
convenient to introduce an alternate notation, $\procn{x}$, when we
want to emphasize the connection to the use of the name. Note that, by
virtue of name equivalence, $\quotep{\procn{x}} \nameeq x$; so, the
notation is consistent with previous definitions.

Further, because names have structure it is possible to effect
substitutions on the basis of that structure. This means we need to
upgrade our notation for substitutions, which we accomplish by
adapting comprehension notation. Thus,

\begin{mathpar}
  P\{ y / x : x \in S \}
\end{mathpar}

is interpreted to mean the process derived from P by replacing (in a
capture-avoiding manner) each occurrence of $x$ in $S$ by $y$. For example,

\begin{mathpar}
  P\{ \quotep{\procn{x}|\procn{x}} / x : x \in \freenames{P} \}
\end{mathpar}

will replace each (occurrence) of a free name $x$ in $P$ by
$\quotep{\procn{x}|\procn{x}}$.

Also, we will avail ourselves of the notation $x^{L}$ and $x^{R}$ to
denote injections of a name into disjoint copies of the name
space. There are numerous ways to accomplish this. One example can be
found in \cite{MeredithR05}. This notation overloads to vectors of
names: $\vec{x}^{\pi} := (x_{i}^{\pi} \; : \; 0 \leq i < |\vec{x}| )$ where $\pi \in \{L,R\}$.

We also use $P^{\Box} := P|\Box$.

In \cite{MeredithR05} an interpretation of the new operator is
given. It turns out that there are several possible interpretations
all enjoying the requisite algebraic properties of the operator (see
\cite{milner91polyadicpi}). We will therefore make liberal use of
$(\nu\; \vec{x})P$.

% subsection the_syntax_and_semantics_of_the_notation_system (end)   

\input{qm2pi.qmops} 

\input{qm2pi.sterngerlach} 

\input{qm2pi.metric} 

% section concurrent_process_calculi (end)

%\input{qm2pi.proofsketch}

% section proof sketch (end)

%\input{qm2pi.slviaknots} 

% section spatial logic via knots (end)

\input{qm2pi.conclusion}

% section conclusion (end)

%\input{qm2pi.dtcodes} 

% section wiring algorithm (end)

\input{qm2pi.ack} 

% section acknowledgments (end)

\newpage


\bibliographystyle{plain}   
\bibliography{../../biblios/main.bib}

\input{qm2pi.rhodetails}

\end{document}

 

% subsection basic_interpretation (end)

%\input{qm2pi.rho.presentation} 
\subsection{The syntax and semantics of the notation system}\label{sub:the_syntax_and_semantics_of_the_notation_system} % (fold)

We now summarize a technical presentation of the calculus that
embodies our theory of dynamics. The typical presentation of such a
calculus follows the style of giving generators and relations on
them. The grammar, below, describing term constructors, freely
generates the set of processes, $\Proc$. This set is then quotiented
by a relation known as structural congruence and it is over this set
that the notion of dynamics is expressed. This presentation is
essentially that of \cite{MeredithR05} with the addition of
polyadicity and summation. For readability we have relegated some of
the technical subtleties to an appendix.

\subsubsection{Process grammar}\label{subsub:process_grammar}

\begin{mathpar}
  \inferrule* [lab=synchronization] {} {{M} \bc \pzero \;|\; x?F \;|\; x!C }
  \and
  \inferrule* [lab=abstraction] {} {{F} \bc (x)P}
  \and
  \inferrule* [lab=concretion] {} {{C} \bc \langle Q \rangle}
  \and
  \inferrule* [lab=process] {} {{P,Q} \bc M \;| \;P|Q \;|\; @{x}}
  \and
  \inferrule* [lab=name] {} {{x} \bc \quotep{P}}
\end{mathpar} 

Note that $\vec{x}$ (resp. $\vec{P}$) denotes a vector of names
(resp. processes) of length $|\vec{x}|$ (resp. $|\vec{P}|$). We adopt
the following useful abbreviations.

\begin{mathpar}
   x?(\vec{y}).P := x.(\vec{y})P \and  x\clift{\vec{P}} := x.\clift{\vec{P}}
   \and x!(y) := \lift{x}{\dropn{y}}
   \and \Pi_{i=0}^{n-1}P_i := P_0 | \ldots | P_{n-1}
\end{mathpar}

\subsubsection{Structural congruence}

\paragraph{Free and bound names and alpha-equivalence.} At the
core of structural equivalence is alpha-equivalence which identifies
process that are the same up to a change of variable. Formally, we
recognize the distinction between free and bound names. The free names
of a process, $\freenames{P}$, may be calculated recursively as
follows:

\begin{mathpar}
\freenames{\pzero} := \emptyset
  \and \\
  \freenames{x?(y).P} := \{ x \} \cup (\freenames{P} \setminus \{ y \})
  \and 
  \freenames{x!\langle P \rangle} := \{ x \} \cup \{ P \} 
  \and \\
  \freenames{P|Q} := \freenames{P} \cup \freenames{Q}
  \and \\
  \freenames{@{x}} := \{ x \}
\end{mathpar}

$\pi$
$\quotep{\pi}$

$\freenames{-} : \pi \to \mathcal{P}(\quotep{\pi})$

\begin{eqnarray*}
  \freenames{\pzero} & := & \emptyset \\
  \freenames{x?(y).P} & := & \{ x \} \cup (\freenames{P} \setminus \{ y \}) \\
  \freenames{x!\langle P \rangle} & := & \{ x \} \cup \{ P \} \\
  \freenames{P|Q} & := & \freenames{P} \cup \freenames{Q} \\
  \freenames{\dropn{x}} & := & \{ x \}
\end{eqnarray*}

The bound names of a process, $\boundnames{P}$, are those names occurring in $P$
that are not free. For example, in $x?(y).0$, the name $x$ is free, while $y$ is bound.

\begin{mathpar}
  \inferrule* [lab=monoidal-laws] {} { P|Q \equiv Q|P \and P|0 \equiv P \and P|(Q|R) \equiv (P|Q)|R }
\end{mathpar}

\begin{mathpar}
  \inferrule* [lab=alpha-equivalence] {} { (x)P \equiv (y)P\{y/x\} \and y \not\in \freenames{P} }
\end{mathpar}

\begin{definition}
Then two processes, $P,Q$, are alpha-equivalent if $P = Q\{\vec{y}/\vec{x}\}$ for
some $\vec{x} \in \boundnames{Q},\vec{y} \in \boundnames{P}$, where $Q\{\vec{y}/\vec{x}\}$
denotes the capture-avoiding substitution of $\vec{y}$ for $\vec{x}$ in $Q$.
\end{definition}

\begin{definition}
  The {\em structural congruence} \cite{SangiorgiWalker} , $\equiv$,
  between processes is the least congruence containing
  alpha-equivalence, satisfying the abelian monoid laws
  (associativity, commutativity and $\pzero$ as identity) for parallel
  composition $|$ and for summation $+$.
\end{definition}

\subsection{Name equivalence}

We take name equivalence, written $\nameeq$, to be the smallest
equivalence relation generated by the following rules.

\begin{mathpar}
\inferrule*[lab=Quote-drop]
{ }
{ \quotep{@{x}} \nameeq x }

\inferrule*[lab=Struct-equiv]
{ P \scong Q }
{ \quotep{P} \nameeq \quotep{Q} }
\end{mathpar}

The astute reader will have noticed that the mutual recursion of names
and processes imposes a mutual recursion on alpha-equivalence and
structural equivalence via name-equivalence. Fortunately, all of this
works out pleasantly and we may calculate in the natural way, free of
concern. The reader interested in the details is referred to the
appendix \ref{appendix:rho_details}.

\subsection{Substitution}

We use $\Proc$ for the set of processes, $\QProc$ for the set of
names, and $\id{\{}\vec{y} / \vec{x} \id{\}}$ to denote partial maps,
$s : \QProc \rightarrow \QProc$. A map, $s$ lifts, uniquely, to a map
on process terms, $\widehat{s} : \Proc \rightarrow \Proc$ by the
following equations.

\begin{mathpar}
  (0) \psubstp{Q}{P} := 0 \\
  (R \juxtap S) \psubstp{Q}{P}
  :=    
  (R)\psubstp{Q}{P} \juxtap (S) \psubstp{Q}{P} \\
  (x?(y).R) \psubstp{Q}{P}    
  :=    
  (x)\substp{Q}{P} (z)\concat( (R \psubstn{z}{y}) \psubstp{Q}{P} ) \\
  (\lift{x}{R}) \psubstp{Q}{P}  
  :=
  \lift{(x)\substp{Q}{P}}{ R \psubstp{Q}{P} } \\
%   (\dropn{x})  \psubstp{Q}{P}       
%   := 
%   \left\{ 
%     \begin{array}{ccc} 
%       \dropn{\quotep{Q}} & & x \nameeq \quotep{P} \\
%       \dropn{x} & & otherwise \\
%     \end{array}
%   \right. 
  (\dropn{x})  \psubstp{Q}{P}       
  := 
  \left\{ 
    \begin{array}{ccc} 
      Q & & x \nameeq \quotep{P} \\
      \dropn{x} & & otherwise \\
    \end{array}
  \right.
\end{mathpar}
 

where

\begin{eqnarray}
  (x)\id{\{} \lpquote Q \rpquote / \lpquote P \rpquote \id{\}}            = 
  \left\{ 
    \begin{array}{ccc}
      \lpquote Q \rpquote & & x \nameeq \lpquote P \rpquote \\
      x & & otherwise \\
    \end{array}
  \right. \nonumber
\end{eqnarray}

and $z$ is chosen distinct from $\quotep{P}$, $\quotep{Q}$, the free
names in $Q$, and all the names in $R$. Our $\alpha$-equivalence will
be built in the standard way from this substitution.

\begin{remark}\label{rem:no_self_referential_names}
  One consequence of these definitions is that $\forall P. \quotep{P}
  \not\in \freenames{P}$.
\end{remark}

\subsection{ Dynamic quote: an example }

Anticipating something of what's to come, consider applying the
substitution, $\widehat{\id{\{}u / z \id{\}}}$, to the following pair
of processes, $\lift{w}{y!(z)}$ and $w[ \lpquote y!(z) \rpquote ]$.

\begin{eqnarray}
	\lift{w}{y!(z)}\widehat{\id{\{}u / z \id{\}}}
		& = &
		\lift{w}{y!(u)} \nonumber\\
	w[ \lpquote y!(z) \rpquote ] \widehat{ \id{\{}u / z \id{\}} }
		& = &
		w[ \lpquote y!(z) \rpquote ] \nonumber
\end{eqnarray}

Because the body of the process between quotes is impervious to
substitution, we get radically different answers. In fact, by
examining the first process in an input context,
e.g. $x?(z).\lift{w}{y!(z)}$, we see that the process under the lift
operator may be shaped by prefixed inputs binding a name inside it. In
this sense, the lift operator will be seen as a way to dynamically
construct processes before reifying them as names.

Finally equipped with these standard features we can present the
dynamics of the calculus.

\subsubsection{Operational semantics} 

Finally, we introduce the computational dynamics. What marks these
algebras as distinct from other more traditionally studied algebraic
structures, e.g. vector spaces or polynomial rings, is the manner in
which dynamics is captured. In traditional structures, dynamics is typically
expressed through morphisms between such structures, as in linear maps
between vector spaces or morphisms between rings. In algebras
associated with the semantics of computation, the dynamics is
expressed as part of the algebraic structure itself, through a
reduction reduction relation typically denoted by $\red$. Below, we
give a recursive presentation of this relation for the calculus used
in the encoding.

$\red \subseteq \pi \times \pi$
$\red : \pi \to \mathcal{P}(\pi)$

\begin{mathpar}
  \inferrule* [lab=Comm] { \textsf{match}( x_{src}, x_{trgt} ) } { x_{trgt}?(y)P \; | \; x_{src}!\langle {Q} \rangle \red P\{\quotep{Q}/y}\} }
  \and \\
  \inferrule* [lab=Par] {{P} \red {P}'} {{{P} | {Q}} \red {{P}' | {Q}}}
  \and
  \inferrule* [lab=Equiv]{{{P} \scong {P}'} \andalso {{P}' \red {Q}'} \andalso {{Q}' \scong {Q}}}{{P} \red {Q}}
\end{mathpar}

\begin{eqnarray*}
  match_{\equiv} (\quotep{P},\quotep{Q}) & := & P \equiv Q \\
  match_{\dagger}(\quotep{P},\quotep{Q}) & := & \forall R. P|Q \red^{*} R => R \red^{*} 0 \\
  match_{K}(\quotep{P},\quotep{Q}) & := & K \mbox{ for some context } K
\end{eqnarray*}

$u?(x)P | u!\langle Q \rangle \red P\{\quotep{Q}/x\}$

%We write $\wred$ for $\red^*$, and $P\red$ if $\exists Q $ such that $ P \red Q$.
We write $P\red$ if $\exists Q $ such that $ P \red Q$ and $P\not\red$, otherwise.

\section{Replication}

As mentioned before, it is known that replication (and hence
recursion) can be implemented in a higher-order process algebra
\cite{SangiorgiWalker}. As our first example of calculation with the
machinery thus far presented we give the construction explicitly in
the {\rhoc}.

\begin{eqnarray}
	D_{x} & := & \prefix{x}{y}{(\binpar{\outputp{x}{y}}{@{y}})} \nonumber\\
	\bangp_{x}{P} & := & \binpar{{x}!\langle{\binpar{D_{x}}{P}}\rangle}{D_{x}} \nonumber
\end{eqnarray}

\begin{eqnarray}
	\bangp_{x}{P} & & \nonumber\\
	=
	& {x}!\langle{(\prefix{x}{y}{(\outputp{x}{y} | @{y})) | P}}\rangle 
	      | \prefix{x}{y}{(\outputp{x}{y} | @{y})} & \nonumber\\
	\red
	& (\outputp{x}{y} | @{y})\substn{\quotep{(\prefix{x}{y}{(@{y} | \outputp{x}{y})) | P}}}{y} & \nonumber\\
	=
	& \outputp{x}{\quotep{(\prefix{x}{y}{(\outputp{x}{y} | @{y})) | P}}}
	  | {(\prefix{x}{y}{(\outputp{x}{y} | @{y})) | P}} & \nonumber\\
	\red
	& \ldots & \nonumber\\
	\red^*
	& P | P | \ldots & \nonumber
\end{eqnarray}

Of course, this encoding, as an implementation, runs away, unfolding
$\bangp{P}$ eagerly. A lazier and more implementable replication
operator, restricted to input-guarded processes, may be obtained as follows.

\begin{eqnarray}
\bangp{\prefix{u}{v}{P}} 
	:= 
	\binpar{\lift{x}{\prefix{u}{v}{(\binpar{D(x)}{P})}}}{D(x)} \nonumber
\end{eqnarray}

\begin{remark}
  Note that the lazier definition still does not deal with summation
  or mixed summation (i.e. sums over input and output). The reader is
  invited to construct definitions of replication that deal with these
  features. 

  Further, the definitions are parameterized in a name, $x$. Can you,
  gentle reader, make a definition that eliminates this parameter and
  guarantees no accidental interaction between the replication
  machinery and the process being replicated -- i.e. no accidental
  sharing of names used by the process to get its work done and the
  name(s) used by the replication to effect copying. This latter
  revision of the definition of replication is crucial to obtaining
  the expected identity $!!P \sim !P$.
\end{remark}

\begin{remark}\label{rem:paradoxical_combinator}
  The reader familiar with the lambda calculus will have noticed the
  similarity between $D$ and the paradoxical combinator.

  [Ed. note: the existence of this seems to suggest we have to be more
  restrictive on the set of processes and names we admit if we are to
  support no-cloning.]
\end{remark}

\subsubsection{Bisimulation}

The computational dynamics gives rise to another kind of equivalence,
the equivalence of computational behavior. As previously mentioned
this is typically captured \emph{via} some form of bisimulation.

% The notion we use in this paper is weak barbed bisimulation
% \cite{milner91polyadicpi}.

The notion we use in this paper is derived from weak barbed
bisimulation \cite{milner91polyadicpi}. 

\begin{definition}
An \emph{observation relation}, $\downarrow_{\mathcal N}$, over a set
of names, $\mathcal N$, is the smallest relation satisfying the rules
below.

\infrule[Out-barb]{y \in {\mathcal N}, \; x \nameeq y}
		  {\outputp{x}{v} \downarrow_{\mathcal N} x}
\infrule[Par-barb]{\mbox{$P\downarrow_{\mathcal N} x$ or $Q\downarrow_{\mathcal N} x$}}
		  {\binpar{P}{Q} \downarrow_{\mathcal N} x}

We write $P \Downarrow_{\mathcal N} x$ if there is $Q$ such that 
$P \wred Q$ and $Q \downarrow_{\mathcal N} x$.
\end{definition}

\begin{definition}
%\label{def.bbisim}
An  ${\mathcal N}$-\emph{barbed bisimulation} over a set of names, ${\mathcal N}$, is a symmetric binary relation 
${\mathcal S}_{\mathcal N}$ between agents such that $P\rel{S}_{\mathcal N}Q$ implies:
\begin{enumerate}
\item If $P \red P'$ then $Q \wred Q'$ and $P'\rel{S}_{\mathcal N} Q'$.
\item If $P\downarrow_{\mathcal N} x$, then $Q\Downarrow_{\mathcal N} x$.
\end{enumerate}
$P$ is ${\mathcal N}$-barbed bisimilar to $Q$, written
$P \wbbisim_{\mathcal N} Q$, if $P \rel{S}_{\mathcal N} Q$ for some ${\mathcal N}$-barbed bisimulation ${\mathcal S}_{\mathcal N}$.
\end{definition}

$\mathcal{R} \subseteq \pi \times \pi$

$P \mathcal{R} Q => \forall P'. P \red P' \Rightarrow \exists Q'. Q \red Q', P' \mathcal{R} Q'$

$P \vdash x \Rightarrow Q \vdash x$

\begin{mathpar}
  \inferrule*[lab=Out-barb]{x \nameeq y}{{y}!\langle{Q}\rangle \vdash x}
  \and
  \inferrule*[lab=Par-barb]{\mbox{$P\vdash x$ or $Q\vdash x$}}{\binpar{P}{Q} \vdash x}
\end{mathpar}

\subsubsection{Contexts}

One of the principle advantages of computational calculi like the
$\pi$-calculus is a well-defined notion of context,
contextual-equivalence and a correlation between
contextual-equivalence and notions of bisimulation. The notion of
context allows the decomposition of a process into (sub-)process and
its syntactic environment, its context. Thus, a context may be
thought of as a process with a ``hole'' (written $\Box$) in it. The
application of a context $M$ to a process $P$, written $M[P]$, is
tantamount to filling the hole in $M$ with $P$. In this paper we do
not need the full weight of this theory, but do make use of the notion
of context in the proof the main theorem. 

\begin{mathpar}
  \inferrule* [lab=summation] {} {{M_{M},M_{N}} \bc \Box \;|\; x.M_{A} \;|\; M_{M}+M_{N}}
  \and
  \inferrule* [lab=agent] {} {{M_{A}} \bc (\vec{x})M_{P} \;| \; \clift{P_0,\ldots,M_{P},\ldots,P_N}}
  \and \\
  \inferrule* [lab=process] {} {{M_{P}} \bc M_{N} \;| \;P|M_{P} }
\end{mathpar} 

\begin{mathpar}
  \inferrule* [lab=sychronization] {} {M_{N} \bc \Box \;|\; x?M_{F} \;|\; x!M_{C}}
  \and
  \inferrule* [lab=abstraction] {} {{M_{F}} \bc (x)M_{P} }
  \and
  \inferrule* [lab=concretion] {} {{M_{C}} \bc \langle M_{P} \rangle }
  \and \\
  \inferrule* [lab=process] {} {{M_{P}} \bc M_{N} \;| \;P|M_{P} }
\end{mathpar}

\begin{definition}[contextual application] Given a context $M$, and
  process $P$, we define the \emph{contextual application}, $M[P] :=
  M\{P/\Box\}$. That is, the contextual application of M to P is the
  substitution of $P$ for $\Box$ in $M$.
\end{definition}

$\meaningof{-} : L \to \mathcal{P}(\pi)$

\begin{mathpar}
  \inferrule* [lab=collection] {} {\meaningof{true} = \pi, \and \meaningof{~E} = \pi \setminus \meaningof{E}, \and \meaningof{E_{1} \& E_{2}} = \meaningof{E_{1}} \cap \meaningof{E_{2}}}
\end{mathpar}

\begin{mathpar}
  \inferrule* [lab=structure] {} {\meaningof{0} = \{ P \in \pi | P \equiv 0 \}, \and \\ \meaningof{E_1 | E_2} = \{ P \in \pi | P \equiv P_{1} | P_{2}, P_{1} \in \meaningof{E_{1}}, P_{2} \in \meaningof{E_2}\} }
\end{mathpar}

\begin{mathpar}
 \inferrule* [lab=behavior] {} {\meaningof{\langle a?b \rangle E} = \{ P \in \pi | P \equiv Q | u?(y)P', \\ \and \\\\ \and \\ \;\;\; u \in \meaningof{a}, \forall z.P'\{z/y\} \in \meaningof{E\{z/b\}}\}, \and \\ \meaningof{a!E} = \{ P \in \pi | P \equiv Q | x!\langle P' \rangle, x \in \meaningof{a} P' \in \meaningof{E}\} }
\end{mathpar}

\begin{mathpar}
 \inferrule* [lab=nominal] {} {\meaningof{\quotep{E}} = \{ \quotep{P} \in \quotep{\pi} | P \in \meaningof{E} \}, \and \meaningof{\quotep{P}} = \{ \quotep{Q} \in \quotep{\pi} | P \equiv Q \} \and \\ \meaningof{@\quotep{E}} = \{ P \in \pi | P \equiv @x, x \in \meaningof{E} \}}
\end{mathpar}

\begin{eqnarray*}
  \\
  \meaningof{-} : TS \to ST
\end{eqnarray*}

\begin{eqnarray*}
  \\
  L : TS \to ST
\end{eqnarray*}

\begin{eqnarray*}
  \\
  P \models E \iff P \in \meaningof{E}
\end{eqnarray*}

\begin{eqnarray*}
  P \approx_{L} Q \iff \forall E \in L. P \models E \iff Q \models E
\end{eqnarray*}

\begin{eqnarray*}
  P \approx_{K} Q
\end{eqnarray*}

\begin{eqnarray*}
  P \approx Q
\end{eqnarray*}

$\approx_{K} = \approx = \approx_{L}$

\subsubsection{Contextual duality}

Note that contexts extend the quotation operation to a family of
operations from processes to names. Given a context, $M$, we can
define a \emph{nominal context}, $\quotep{M}$ by $\quotep{M}[P] :=
\quotep{M[P]}$. To foreshadow what is to come we observe that these
operations enjoy a duality with processes very much like the duality
between vectors and maps from vectors to scalars.

Further, because the calculus is essentially higher-order, we have a
correspondence between contexts and processes. More specifically,
given a name $x$ and a context $M$ we can construct $M^{*}_{x}$ such
that 

\begin{mathpar}
  M^{*}_{x} | \lift{x}{P} \red M[P]
\end{mathpar}

namely,

\begin{mathpar}
  M^{*}_{x} := x?(u).M[\dropn{u}]
\end{mathpar}

The dependence of $M^{*}_{x}$ on a name makes it an abstraction, 

\begin{mathpar}
  M^{*} := (x)x?(u).M[\dropn{u}]
\end{mathpar}

\subsection{Additional notation}

It will sometimes be convenient to denote the process a name
quotes. We already have the notation $x = \quotep{P}$, but it will be
convenient to introduce an alternate notation, $\procn{x}$, when we
want to emphasize the connection to the use of the name. Note that, by
virtue of name equivalence, $\quotep{\procn{x}} \nameeq x$; so, the
notation is consistent with previous definitions.

Further, because names have structure it is possible to effect
substitutions on the basis of that structure. This means we need to
upgrade our notation for substitutions, which we accomplish by
adapting comprehension notation. Thus,

\begin{mathpar}
  P\{ y / x : x \in S \}
\end{mathpar}

is interpreted to mean the process derived from P by replacing (in a
capture-avoiding manner) each occurrence of $x$ in $S$ by $y$. For example,

\begin{mathpar}
  P\{ \quotep{\procn{x}|\procn{x}} / x : x \in \freenames{P} \}
\end{mathpar}

will replace each (occurrence) of a free name $x$ in $P$ by
$\quotep{\procn{x}|\procn{x}}$.

Also, we will avail ourselves of the notation $x^{L}$ and $x^{R}$ to
denote injections of a name into disjoint copies of the name
space. There are numerous ways to accomplish this. One example can be
found in \cite{MeredithR05}. This notation overloads to vectors of
names: $\vec{x}^{\pi} := (x_{i}^{\pi} \; : \; 0 \leq i < |\vec{x}| )$ where $\pi \in \{L,R\}$.

We also use $P^{\Box} := P|\Box$.

In \cite{MeredithR05} an interpretation of the new operator is
given. It turns out that there are several possible interpretations
all enjoying the requisite algebraic properties of the operator (see
\cite{milner91polyadicpi}). We will therefore make liberal use of
$(\nu\; \vec{x})P$.

% subsection the_syntax_and_semantics_of_the_notation_system (end)   

\section{Interpretation of QM}
\subsection{Supporting definitions}
\subsubsection{Multiplication}
\begin{mathpar}
  \quotep{Q} \cdot \quotep{R} := \quotep{Q|R}
  \and \\
  \quotep{Q} \cdot P := P\{ \quotep{Q|R} / \quotep{R} : \quotep{R} \in \freenames{P} \}
\end{mathpar}

\paragraph{Discussion}
The first line needs little explanation. The second line says that
each free name of the process is replaced with the multiplication of
that name by the scalar. Multiplication of a scalar (name) by a state
(process) results in a process all the names of which have been `moved
over' by parallel composition with the process the scalar
quotes. There is a subtlety that the bound names have to be
manipulated so that multiplied names aren't accidentally
captured. There are many ways to achieve this.

\begin{remark}\label{rem:multiplication_identities}
  The reader is invited to verify that for all $x,y,z \in \QProc$ and $P \in \Proc$
  \begin{mathpar}
    x \cdot \quotep{0} \equiv x 
    \and
    x \cdot y \equiv y \cdot x
    \and
    x \cdot (y \cdot z) \equiv (x \cdot y) \cdot z
    \and \\
    \quotep{0} \cdot P \equiv P
    \and \\
    x \cdot (y \cdot P) \equiv (x \cdot y) \cdot P
    \and \\
    x \cdot (P|Q) \equiv (x \cdot P) | (x \cdot Q)
    \and \\    
  \end{mathpar}
\end{remark}

\subsubsection{Tensor product}

We define a tensor product on processes by structural induction.

\paragraph{Tensor of sums} First note that all summations, including
$\pzero$ and sequence, can be written $\Sigma_{i} x_{i}.A_{i} +
\Sigma_{j} x_{j}.C_{j}$, where we have grouped input-guarded processes
together and output-guarded processes together.

Thus, we can define the tensor product of two summations, $N_{1}\otimes N_{2}$, where

\begin{mathpar}
  N_{1} := \Sigma_{i} x_{i}.A_{i} + \Sigma_{j} x_{j}.C_{j}
  \and
  N_{2} := \Sigma_{i'} y_{i'}.B_{i'} + \Sigma_{j'} y_{j'}.D_{j'} 
\end{mathpar}

as follows.

\begin{mathpar}
  \Sigma_{i} x_{i}.A_{i} + \Sigma_{j} x_{j}.C_{j} \otimes \Sigma_{i'}
  y_{i'}.B_{i'} + \Sigma_{j'} y_{j'}.D_{j'} 
  \and \\
  := \; \Sigma_{i} \Sigma_{i'} \quotep{\stackrel{\vee}{x_{i}}| \stackrel{\vee}{y_{i'}}}.(A_{i}\otimes B_{i'}) \; | \; \Sigma_{i'} \Sigma_{i} \quotep{\stackrel{\vee}{y_{i'}}|\stackrel{\vee}{x_{i}}}.(B_{i'}\otimes A_{i})
  \and
  \;\; | \;\; \Sigma_{j} \Sigma_{j'} \quotep{\stackrel{\vee}{x_{j}}|\stackrel{\vee}{y_{j'}}}.(A_{j}\otimes B_{j'}) \; | \; \Sigma_{j'} \Sigma_{j} \quotep{\stackrel{\vee}{y_{j'}}|\stackrel{\vee}{x_{j}}}.(B_{j'}\otimes A_{j})
\end{mathpar}

\begin{remark}
  Do we need to $x^{L}$ and $y^{R}$ for this construction as well?
\end{remark}

\paragraph{Tensor of parallel compositions} Next, we distribute tensor
over par.

\begin{mathpar}
  P_{1}|P_{2} \otimes Q_{1}|Q_{2} := (P_{1} \otimes Q_{1}) | (P_{1}
  \otimes Q_{2}) | (P_{2} \otimes Q_{1}) | (P_{2} \otimes Q_{2})
\end{mathpar}

\paragraph{Tensor with dropped names} We treat tensor of a
process with a dropped name as parallel composition.

\begin{mathpar}
  P \otimes \dropn{x} := P | \dropn{x}
\end{mathpar}

\paragraph{Tensor of agents}

Finally, we need to define tensor on agents. Note that the definition
of tensor on normal products only tensors inputs with inputs and
outputs with outputs. Thus, we only have to define the operation on
``homogeneous'' pairings.

\begin{mathpar}
  (\vec{x})P \otimes (\vec{y})Q
  \and \\
  := (x_{0}^{L}|y_{0}^{R},\ldots,x_{0}^{L}|y_{n}^{R},\ldots,x_{m}^{L}|y_{0}^{R},\ldots,x_{m}^{L}|y_{n}^R)(P\{ \vec{x}^{L}/\vec{x}\} \otimes Q \{ \vec{y}^{R}/\vec{y}\})
  \and \\
  \clift{\vec{P}} \otimes \clift{\vec{Q}}
  \and \\
  := \clift{P_{0}\otimes Q_{0},\ldots,P_{0}\otimes Q_{n},\ldots,P_{m}\otimes Q_{0},\ldots,P_{m}\otimes Q_{n}}
\end{mathpar}

\begin{remark}
  Observe that arities of tensored abstractions matches arities of
  tensored concretions if the original arities matched. Note also that
  the length of the arities corresponds to the increase in dimension
  we see in ordinary vector space tensor product.
\end{remark}

\begin{remark}
  Operationally, this definition distributes the tensor down to
  components ``linked'' by summation. Tensor over summation is
  intriguing in that it mixes names. Moreover, as a consequence of the
  way it mixes names we have the identities for all $x \in \QProc$ and
  $P,Q \in \Proc$

  \begin{mathpar}
    (x \cdot P) \otimes Q \equiv x \cdot (P \otimes Q) \equiv P \otimes (x \cdot Q)
    \and
    P \otimes \pzero \equiv P
  \end{mathpar}

  that the reader is invited to verify.
\end{remark}

\subsubsection{Annihilation}
\begin{mathpar}
  P^{\perp} := \{ Q | \forall R. P|Q \red^{*} R \Rightarrow R \red^{*} \pzero \}
  \and \\
  P^{\underline{\perp}} := \Sigma_{Q \in P^{\perp}} \quotep{Q}?(y).(\dropn{y}|Q) | \Sigma_{Q \in P^{\perp}} \quotep{Q}\clift{\Box}
\end{mathpar}

\paragraph{Discussion} The reader will note that $P^{\perp}$ is a
\emph{set} of processes, while $P^{\underline{\perp}}$ is a
\emph{context}. We call the set $P^{\perp}$ the \emph{annihilators} of
$P$. The parallel composition of a process in the annihilators of $P$
with $P$ will result in a process, the state space of which has all
paths eventually leading to $\pzero$. Execution may endure loops; but
under reasonable conditions of fairness (naturally guaranteed under
most notions of bisimulation) such a composite process cannot get
stuck in such a loop and will, eventually pop out and terminate.

The context $P^{\underline{\perp}}$ is ready and willing to ``take the
$P$ out of'' the process to which it is applied. It will effectively
transmit the code of the process to which it is applied to one of the
annihilators and run the process against it.

\subsubsection{Evaluation}
We fix $M$ a domain of fully abstract interpretation with an equality
coincident with bisimulation. We take $\meaningof{\cdot} : \Proc \to
M$ to be the map interpreting processes and $\nmeaningof{\cdot} : \M
\to Proc$ to be the map running the other way. Then we define

\begin{mathpar}
  \int P := \nmeaningof{\meaningof{P}}
\end{mathpar}

\paragraph{Discussion}
There are many fully abstract interpretations of Milner's
$\pi$-calculus. Any of them can be used as a basis for interpreting
the reflective calculus here. Equipped with such a domain it is
largely a matter of grinding through to check that the Yoneda
construction for the normalization-by-evaluation program can be
extended to this setting.

\begin{remark}
  The reader is invited to verify that $\int (P^{\underline{\perp}}[P]) = 0$.
\end{remark}

\subsection{Quantum mechanics}

Table \ref{tbl:core_qm_op_defns} gives the core operational definitions

\begin{table}[htp]\label{tbl:core_qm_op_defns}
  \center{
    \fbox{
      \begin{tabular}{c|c}
        quantum mechanics & process calculus \\
        \hline
        scalar & $x := \quotep{P}$ \\
        state vector & $\state{P} := P$ \\
        dual & $\state{P}^{*} := \event{P^{\underline{\perp}}} := \quotep{P^{\underline{\perp}}}[-]$ \\
        matrix & $ \Sigma_{\alpha} \state{P_{\alpha}}x_{\alpha}\event{Q_{\alpha}}$ \\
        vector addition & $\state{P} + \state{Q} := \state{P | Q}$ \\
        tensor product & $\state{P} \otimes \state{Q} := \state{P \otimes Q}$ \\
        inner product & $\innerprod{P}{Q} := \quotep{\int P^{\underline{\perp}}[Q]}$ \\
      \end{tabular}
    }
  }
  \caption{QM - operational definitions}
\end{table}

where

\begin{mathpar}
  \prmatrix{P}{Q} := \fprmatrix{P}{\quotep{\pzero}}{Q}
  \and
  \fprmatrix{P}{x}{Q} := (\state{P},x,\event{Q})
  \and
  (\fprmatrix{P}{x}{Q})(\state{R}) := x \cdot \innerprod{Q}{R} \cdot \state{P}
  \and
  (\fprmatrix{P}{x}{Q})(\event{R}) := x \cdot \innerprod{R}{P} \cdot \event{Q}
\end{mathpar}

\paragraph{Discussion}
As promised: vectors (aka states) are represented as processes; duals
as contextual duals; inner product definition should be compared with
standard inner product definition for ....

\begin{remark}
  Assuming $\int (P^{\underline{\perp}}[P]) = 0$, the reader is
  invited to verify that $(\fprmatrix{P}{x}{P})(\state{P}) = x \cdot \state{P}$.
\end{remark}

\begin{remark}
  The reader is invited to verify that $\innerprod{P}{Q}$ could
  equally well have been written $\quotep{\int \stackrel{\vee}{x}}$
  where $x = \event{P^{\underline{\perp}}}(Q)$.

  One of the motivations for this remark is that there is another way
  to factor these operations. We could package up evaluation in the dual:

  \begin{mathpar}
    \state{P}^{*} := \event{\int P^{\underline{\perp}}} := \quotep{\int P^{\underline{\perp}}}[-]
  \end{mathpar}

  and then have inner product defined by
  
  \begin{mathpar}
    \innerprod{P}{Q} := \event{P}(Q)
  \end{mathpar}

  Hopefully, experience with the calculations will provide guidance on
  the best factoring.
\end{remark}

\begin{remark}
  Assuming $\int (P^{\underline{\perp}}[P]) = 0$, the reader is
  invited to verify that $\forall P,Q. (\prmatrix{0}{Q})(\state{0}) =
  \state{0}$ and dually $(\prmatrix{P}{0})(\event{0}) = \event{0}$.
\end{remark}

\begin{remark}
  i'm a little worried that i don't (yet) have proper support for
  complex conjugacy. But, the observation above may give us a
  clue. According to Abramsky, it must be the case that the scalars
  are iso to the homset of the identity for the tensor -- which the
  observation above characterizes. 

  For now, we will simply bookmark the notion with $\overline{x}$.
\end{remark}

\subsubsection{Adjointness}

We need to give a definition of $(\cdot)^{\dagger}$ for matrices. The
obvious candidate definition is
\begin{mathpar}
(\Sigma_{\alpha}\fprmatrix{P_{\alpha}}{x_{\alpha}}{Q_{\alpha}})^{\dagger}
= \Sigma_{\alpha}\fprmatrix{(Q_{\alpha}^{\underline{\perp}})^{*}}{\overline{x}_{\alpha}}{P_{\alpha}^{\underline{\perp}}} 
\end{mathpar}

But, $(Q_{\alpha}^{\underline{\perp}})^{*}$ requires a name along
which to communicate the process to achieve the context application.

\subsubsection{Basis for a basis}
If processes label states and ``addition'' of states (a.k.a. vector
addition) is interpreted as parallel composition, what corresponds to
notions of linear independence and basis? Here, we recall that Yoshida
has developed a set of \emph{combinators} for an asynchronous verison
of Milner's $\pi$-calculus. These are a finite set of processes such
any process can be expressed as parallel composition of these
combinators together with liberal uses of the new operator and
replication. We can simply give a translation of these into the
present calculus and have reasonable expectation that the property
carries over. That is, that the resultant set allows to express all
processes via parallel composition. Note, however, that there is no
new operator or replication in this calculus. As a result, we expect
that the corresponding set is actually infinite. That is, we expect
that the space is actually infinite dimensional.

\begin{remark}
  The attentive reader may be a bit concerned. Certainly, the
  collection $S$, $K$ and $I$ is a finite set of
  combinators. Shouldn't we expect to see a finite set of combinators
  for an effectively equivalent system? i am very sympathetic to this
  critique and feel it warrants full attention. On the other hand, i
  also have in mind the following analogy. The natural numbers, as a
  monoid under addition, has exactly $1$ generator, while the natural
  numbers, as a monoid under multiplication, has countably many
  generators (the primes). We observe that the application of the
  lambda calculus is much less resource sensitive than the parallel
  composition of the $\pi$-calculus. Could it be the case that we have
  an analogy of the form
  
  \begin{mathpar}
    m + n : MN :: m*n : M|N
  \end{mathpar}

  giving a similar blow up in the set of ``primes''?  This is such a
  wonderful thought that, even if it's not true, i think it's worth
  writing down.
\end{remark}
 

\documentclass[12pt]{llncs}
%\documentclass{jktr}

\usepackage[pdftex]{hyperref}                   
\usepackage {listings}
\usepackage {mathpartir}
\usepackage{bcprules}
%\usepackage{listings}
                       
\usepackage{graphicx} 
%\usepackage[margins=2.5cm,nohead,nofoot]{geometry}
%\usepackage{geometry}
\usepackage{amsfonts}
\usepackage{amstext}
\usepackage{latexsym}
\usepackage{amssymb}
\usepackage{color}


%\include{myPreamble}
\include{qm2pi.local} 

%\ifpdf
%\usepackage[pdftex]{graphicx}
%\else
%\usepackage{graphicx}
%\fi

 % \ifpdf
%  \usepackage{pdfsync}
%  \if


%\title{Brief Article}
%\author{David F. Snyder}
%\author{L.G. Meredith}

%\address{Dept. of Math., Texas State University--San Marcos, San Marcos, TX 78666}
       
\pagestyle{empty}


\begin{document}

\lstset{language=[Objective]Caml,frame=shadowbox}

\input{qm2pi.front}

% section front matter (end)

\input{qm2pi.intro} 
 
% section introduction (end)

% \input{qm2pi.knotations} 

% section notation (end)

\input{qm2pi.process.calculi} 

% section concurrent_process_calculi_and_spatial_logics_ (end)
    
%\input{qm2pi.knots2pi} 

%\input{qm2pi.trefoil} 

%\input{qm2pi.mainthm} 

% subsection basic_interpretation (end)

%\input{qm2pi.rho.presentation} 
\subsection{The syntax and semantics of the notation system}\label{sub:the_syntax_and_semantics_of_the_notation_system} % (fold)

We now summarize a technical presentation of the calculus that
embodies our theory of dynamics. The typical presentation of such a
calculus follows the style of giving generators and relations on
them. The grammar, below, describing term constructors, freely
generates the set of processes, $\Proc$. This set is then quotiented
by a relation known as structural congruence and it is over this set
that the notion of dynamics is expressed. This presentation is
essentially that of \cite{MeredithR05} with the addition of
polyadicity and summation. For readability we have relegated some of
the technical subtleties to an appendix.

\subsubsection{Process grammar}\label{subsub:process_grammar}

\begin{mathpar}
  \inferrule* [lab=synchronization] {} {{M} \bc \pzero \;|\; x?F \;|\; x!C }
  \and
  \inferrule* [lab=abstraction] {} {{F} \bc (x)P}
  \and
  \inferrule* [lab=concretion] {} {{C} \bc \langle Q \rangle}
  \and
  \inferrule* [lab=process] {} {{P,Q} \bc M \;| \;P|Q \;|\; @{x}}
  \and
  \inferrule* [lab=name] {} {{x} \bc \quotep{P}}
\end{mathpar} 

Note that $\vec{x}$ (resp. $\vec{P}$) denotes a vector of names
(resp. processes) of length $|\vec{x}|$ (resp. $|\vec{P}|$). We adopt
the following useful abbreviations.

\begin{mathpar}
   x?(\vec{y}).P := x.(\vec{y})P \and  x\clift{\vec{P}} := x.\clift{\vec{P}}
   \and x!(y) := \lift{x}{\dropn{y}}
   \and \Pi_{i=0}^{n-1}P_i := P_0 | \ldots | P_{n-1}
\end{mathpar}

\subsubsection{Structural congruence}

\paragraph{Free and bound names and alpha-equivalence.} At the
core of structural equivalence is alpha-equivalence which identifies
process that are the same up to a change of variable. Formally, we
recognize the distinction between free and bound names. The free names
of a process, $\freenames{P}$, may be calculated recursively as
follows:

\begin{mathpar}
\freenames{\pzero} := \emptyset
  \and \\
  \freenames{x?(y).P} := \{ x \} \cup (\freenames{P} \setminus \{ y \})
  \and 
  \freenames{x!\langle P \rangle} := \{ x \} \cup \{ P \} 
  \and \\
  \freenames{P|Q} := \freenames{P} \cup \freenames{Q}
  \and \\
  \freenames{@{x}} := \{ x \}
\end{mathpar}

$\pi$
$\quotep{\pi}$

$\freenames{-} : \pi \to \mathcal{P}(\quotep{\pi})$

\begin{eqnarray*}
  \freenames{\pzero} & := & \emptyset \\
  \freenames{x?(y).P} & := & \{ x \} \cup (\freenames{P} \setminus \{ y \}) \\
  \freenames{x!\langle P \rangle} & := & \{ x \} \cup \{ P \} \\
  \freenames{P|Q} & := & \freenames{P} \cup \freenames{Q} \\
  \freenames{\dropn{x}} & := & \{ x \}
\end{eqnarray*}

The bound names of a process, $\boundnames{P}$, are those names occurring in $P$
that are not free. For example, in $x?(y).0$, the name $x$ is free, while $y$ is bound.

\begin{mathpar}
  \inferrule* [lab=monoidal-laws] {} { P|Q \equiv Q|P \and P|0 \equiv P \and P|(Q|R) \equiv (P|Q)|R }
\end{mathpar}

\begin{mathpar}
  \inferrule* [lab=alpha-equivalence] {} { (x)P \equiv (y)P\{y/x\} \and y \not\in \freenames{P} }
\end{mathpar}

\begin{definition}
Then two processes, $P,Q$, are alpha-equivalent if $P = Q\{\vec{y}/\vec{x}\}$ for
some $\vec{x} \in \boundnames{Q},\vec{y} \in \boundnames{P}$, where $Q\{\vec{y}/\vec{x}\}$
denotes the capture-avoiding substitution of $\vec{y}$ for $\vec{x}$ in $Q$.
\end{definition}

\begin{definition}
  The {\em structural congruence} \cite{SangiorgiWalker} , $\equiv$,
  between processes is the least congruence containing
  alpha-equivalence, satisfying the abelian monoid laws
  (associativity, commutativity and $\pzero$ as identity) for parallel
  composition $|$ and for summation $+$.
\end{definition}

\subsection{Name equivalence}

We take name equivalence, written $\nameeq$, to be the smallest
equivalence relation generated by the following rules.

\begin{mathpar}
\inferrule*[lab=Quote-drop]
{ }
{ \quotep{@{x}} \nameeq x }

\inferrule*[lab=Struct-equiv]
{ P \scong Q }
{ \quotep{P} \nameeq \quotep{Q} }
\end{mathpar}

The astute reader will have noticed that the mutual recursion of names
and processes imposes a mutual recursion on alpha-equivalence and
structural equivalence via name-equivalence. Fortunately, all of this
works out pleasantly and we may calculate in the natural way, free of
concern. The reader interested in the details is referred to the
appendix \ref{appendix:rho_details}.

\subsection{Substitution}

We use $\Proc$ for the set of processes, $\QProc$ for the set of
names, and $\id{\{}\vec{y} / \vec{x} \id{\}}$ to denote partial maps,
$s : \QProc \rightarrow \QProc$. A map, $s$ lifts, uniquely, to a map
on process terms, $\widehat{s} : \Proc \rightarrow \Proc$ by the
following equations.

\begin{mathpar}
  (0) \psubstp{Q}{P} := 0 \\
  (R \juxtap S) \psubstp{Q}{P}
  :=    
  (R)\psubstp{Q}{P} \juxtap (S) \psubstp{Q}{P} \\
  (x?(y).R) \psubstp{Q}{P}    
  :=    
  (x)\substp{Q}{P} (z)\concat( (R \psubstn{z}{y}) \psubstp{Q}{P} ) \\
  (\lift{x}{R}) \psubstp{Q}{P}  
  :=
  \lift{(x)\substp{Q}{P}}{ R \psubstp{Q}{P} } \\
%   (\dropn{x})  \psubstp{Q}{P}       
%   := 
%   \left\{ 
%     \begin{array}{ccc} 
%       \dropn{\quotep{Q}} & & x \nameeq \quotep{P} \\
%       \dropn{x} & & otherwise \\
%     \end{array}
%   \right. 
  (\dropn{x})  \psubstp{Q}{P}       
  := 
  \left\{ 
    \begin{array}{ccc} 
      Q & & x \nameeq \quotep{P} \\
      \dropn{x} & & otherwise \\
    \end{array}
  \right.
\end{mathpar}
 

where

\begin{eqnarray}
  (x)\id{\{} \lpquote Q \rpquote / \lpquote P \rpquote \id{\}}            = 
  \left\{ 
    \begin{array}{ccc}
      \lpquote Q \rpquote & & x \nameeq \lpquote P \rpquote \\
      x & & otherwise \\
    \end{array}
  \right. \nonumber
\end{eqnarray}

and $z$ is chosen distinct from $\quotep{P}$, $\quotep{Q}$, the free
names in $Q$, and all the names in $R$. Our $\alpha$-equivalence will
be built in the standard way from this substitution.

\begin{remark}\label{rem:no_self_referential_names}
  One consequence of these definitions is that $\forall P. \quotep{P}
  \not\in \freenames{P}$.
\end{remark}

\subsection{ Dynamic quote: an example }

Anticipating something of what's to come, consider applying the
substitution, $\widehat{\id{\{}u / z \id{\}}}$, to the following pair
of processes, $\lift{w}{y!(z)}$ and $w[ \lpquote y!(z) \rpquote ]$.

\begin{eqnarray}
	\lift{w}{y!(z)}\widehat{\id{\{}u / z \id{\}}}
		& = &
		\lift{w}{y!(u)} \nonumber\\
	w[ \lpquote y!(z) \rpquote ] \widehat{ \id{\{}u / z \id{\}} }
		& = &
		w[ \lpquote y!(z) \rpquote ] \nonumber
\end{eqnarray}

Because the body of the process between quotes is impervious to
substitution, we get radically different answers. In fact, by
examining the first process in an input context,
e.g. $x?(z).\lift{w}{y!(z)}$, we see that the process under the lift
operator may be shaped by prefixed inputs binding a name inside it. In
this sense, the lift operator will be seen as a way to dynamically
construct processes before reifying them as names.

Finally equipped with these standard features we can present the
dynamics of the calculus.

\subsubsection{Operational semantics} 

Finally, we introduce the computational dynamics. What marks these
algebras as distinct from other more traditionally studied algebraic
structures, e.g. vector spaces or polynomial rings, is the manner in
which dynamics is captured. In traditional structures, dynamics is typically
expressed through morphisms between such structures, as in linear maps
between vector spaces or morphisms between rings. In algebras
associated with the semantics of computation, the dynamics is
expressed as part of the algebraic structure itself, through a
reduction reduction relation typically denoted by $\red$. Below, we
give a recursive presentation of this relation for the calculus used
in the encoding.

$\red \subseteq \pi \times \pi$
$\red : \pi \to \mathcal{P}(\pi)$

\begin{mathpar}
  \inferrule* [lab=Comm] { \textsf{match}( x_{src}, x_{trgt} ) } { x_{trgt}?(y)P \; | \; x_{src}!\langle {Q} \rangle \red P\{\quotep{Q}/y}\} }
  \and \\
  \inferrule* [lab=Par] {{P} \red {P}'} {{{P} | {Q}} \red {{P}' | {Q}}}
  \and
  \inferrule* [lab=Equiv]{{{P} \scong {P}'} \andalso {{P}' \red {Q}'} \andalso {{Q}' \scong {Q}}}{{P} \red {Q}}
\end{mathpar}

\begin{eqnarray*}
  match_{\equiv} (\quotep{P},\quotep{Q}) & := & P \equiv Q \\
  match_{\dagger}(\quotep{P},\quotep{Q}) & := & \forall R. P|Q \red^{*} R => R \red^{*} 0 \\
  match_{K}(\quotep{P},\quotep{Q}) & := & K \mbox{ for some context } K
\end{eqnarray*}

$u?(x)P | u!\langle Q \rangle \red P\{\quotep{Q}/x\}$

%We write $\wred$ for $\red^*$, and $P\red$ if $\exists Q $ such that $ P \red Q$.
We write $P\red$ if $\exists Q $ such that $ P \red Q$ and $P\not\red$, otherwise.

\section{Replication}

As mentioned before, it is known that replication (and hence
recursion) can be implemented in a higher-order process algebra
\cite{SangiorgiWalker}. As our first example of calculation with the
machinery thus far presented we give the construction explicitly in
the {\rhoc}.

\begin{eqnarray}
	D_{x} & := & \prefix{x}{y}{(\binpar{\outputp{x}{y}}{@{y}})} \nonumber\\
	\bangp_{x}{P} & := & \binpar{{x}!\langle{\binpar{D_{x}}{P}}\rangle}{D_{x}} \nonumber
\end{eqnarray}

\begin{eqnarray}
	\bangp_{x}{P} & & \nonumber\\
	=
	& {x}!\langle{(\prefix{x}{y}{(\outputp{x}{y} | @{y})) | P}}\rangle 
	      | \prefix{x}{y}{(\outputp{x}{y} | @{y})} & \nonumber\\
	\red
	& (\outputp{x}{y} | @{y})\substn{\quotep{(\prefix{x}{y}{(@{y} | \outputp{x}{y})) | P}}}{y} & \nonumber\\
	=
	& \outputp{x}{\quotep{(\prefix{x}{y}{(\outputp{x}{y} | @{y})) | P}}}
	  | {(\prefix{x}{y}{(\outputp{x}{y} | @{y})) | P}} & \nonumber\\
	\red
	& \ldots & \nonumber\\
	\red^*
	& P | P | \ldots & \nonumber
\end{eqnarray}

Of course, this encoding, as an implementation, runs away, unfolding
$\bangp{P}$ eagerly. A lazier and more implementable replication
operator, restricted to input-guarded processes, may be obtained as follows.

\begin{eqnarray}
\bangp{\prefix{u}{v}{P}} 
	:= 
	\binpar{\lift{x}{\prefix{u}{v}{(\binpar{D(x)}{P})}}}{D(x)} \nonumber
\end{eqnarray}

\begin{remark}
  Note that the lazier definition still does not deal with summation
  or mixed summation (i.e. sums over input and output). The reader is
  invited to construct definitions of replication that deal with these
  features. 

  Further, the definitions are parameterized in a name, $x$. Can you,
  gentle reader, make a definition that eliminates this parameter and
  guarantees no accidental interaction between the replication
  machinery and the process being replicated -- i.e. no accidental
  sharing of names used by the process to get its work done and the
  name(s) used by the replication to effect copying. This latter
  revision of the definition of replication is crucial to obtaining
  the expected identity $!!P \sim !P$.
\end{remark}

\begin{remark}\label{rem:paradoxical_combinator}
  The reader familiar with the lambda calculus will have noticed the
  similarity between $D$ and the paradoxical combinator.

  [Ed. note: the existence of this seems to suggest we have to be more
  restrictive on the set of processes and names we admit if we are to
  support no-cloning.]
\end{remark}

\subsubsection{Bisimulation}

The computational dynamics gives rise to another kind of equivalence,
the equivalence of computational behavior. As previously mentioned
this is typically captured \emph{via} some form of bisimulation.

% The notion we use in this paper is weak barbed bisimulation
% \cite{milner91polyadicpi}.

The notion we use in this paper is derived from weak barbed
bisimulation \cite{milner91polyadicpi}. 

\begin{definition}
An \emph{observation relation}, $\downarrow_{\mathcal N}$, over a set
of names, $\mathcal N$, is the smallest relation satisfying the rules
below.

\infrule[Out-barb]{y \in {\mathcal N}, \; x \nameeq y}
		  {\outputp{x}{v} \downarrow_{\mathcal N} x}
\infrule[Par-barb]{\mbox{$P\downarrow_{\mathcal N} x$ or $Q\downarrow_{\mathcal N} x$}}
		  {\binpar{P}{Q} \downarrow_{\mathcal N} x}

We write $P \Downarrow_{\mathcal N} x$ if there is $Q$ such that 
$P \wred Q$ and $Q \downarrow_{\mathcal N} x$.
\end{definition}

\begin{definition}
%\label{def.bbisim}
An  ${\mathcal N}$-\emph{barbed bisimulation} over a set of names, ${\mathcal N}$, is a symmetric binary relation 
${\mathcal S}_{\mathcal N}$ between agents such that $P\rel{S}_{\mathcal N}Q$ implies:
\begin{enumerate}
\item If $P \red P'$ then $Q \wred Q'$ and $P'\rel{S}_{\mathcal N} Q'$.
\item If $P\downarrow_{\mathcal N} x$, then $Q\Downarrow_{\mathcal N} x$.
\end{enumerate}
$P$ is ${\mathcal N}$-barbed bisimilar to $Q$, written
$P \wbbisim_{\mathcal N} Q$, if $P \rel{S}_{\mathcal N} Q$ for some ${\mathcal N}$-barbed bisimulation ${\mathcal S}_{\mathcal N}$.
\end{definition}

$\mathcal{R} \subseteq \pi \times \pi$

$P \mathcal{R} Q => \forall P'. P \red P' \Rightarrow \exists Q'. Q \red Q', P' \mathcal{R} Q'$

$P \vdash x \Rightarrow Q \vdash x$

\begin{mathpar}
  \inferrule*[lab=Out-barb]{x \nameeq y}{{y}!\langle{Q}\rangle \vdash x}
  \and
  \inferrule*[lab=Par-barb]{\mbox{$P\vdash x$ or $Q\vdash x$}}{\binpar{P}{Q} \vdash x}
\end{mathpar}

\subsubsection{Contexts}

One of the principle advantages of computational calculi like the
$\pi$-calculus is a well-defined notion of context,
contextual-equivalence and a correlation between
contextual-equivalence and notions of bisimulation. The notion of
context allows the decomposition of a process into (sub-)process and
its syntactic environment, its context. Thus, a context may be
thought of as a process with a ``hole'' (written $\Box$) in it. The
application of a context $M$ to a process $P$, written $M[P]$, is
tantamount to filling the hole in $M$ with $P$. In this paper we do
not need the full weight of this theory, but do make use of the notion
of context in the proof the main theorem. 

\begin{mathpar}
  \inferrule* [lab=summation] {} {{M_{M},M_{N}} \bc \Box \;|\; x.M_{A} \;|\; M_{M}+M_{N}}
  \and
  \inferrule* [lab=agent] {} {{M_{A}} \bc (\vec{x})M_{P} \;| \; \clift{P_0,\ldots,M_{P},\ldots,P_N}}
  \and \\
  \inferrule* [lab=process] {} {{M_{P}} \bc M_{N} \;| \;P|M_{P} }
\end{mathpar} 

\begin{mathpar}
  \inferrule* [lab=sychronization] {} {M_{N} \bc \Box \;|\; x?M_{F} \;|\; x!M_{C}}
  \and
  \inferrule* [lab=abstraction] {} {{M_{F}} \bc (x)M_{P} }
  \and
  \inferrule* [lab=concretion] {} {{M_{C}} \bc \langle M_{P} \rangle }
  \and \\
  \inferrule* [lab=process] {} {{M_{P}} \bc M_{N} \;| \;P|M_{P} }
\end{mathpar}

\begin{definition}[contextual application] Given a context $M$, and
  process $P$, we define the \emph{contextual application}, $M[P] :=
  M\{P/\Box\}$. That is, the contextual application of M to P is the
  substitution of $P$ for $\Box$ in $M$.
\end{definition}

$\meaningof{-} : L \to \mathcal{P}(\pi)$

\begin{mathpar}
  \inferrule* [lab=collection] {} {\meaningof{true} = \pi, \and \meaningof{~E} = \pi \setminus \meaningof{E}, \and \meaningof{E_{1} \& E_{2}} = \meaningof{E_{1}} \cap \meaningof{E_{2}}}
\end{mathpar}

\begin{mathpar}
  \inferrule* [lab=structure] {} {\meaningof{0} = \{ P \in \pi | P \equiv 0 \}, \and \\ \meaningof{E_1 | E_2} = \{ P \in \pi | P \equiv P_{1} | P_{2}, P_{1} \in \meaningof{E_{1}}, P_{2} \in \meaningof{E_2}\} }
\end{mathpar}

\begin{mathpar}
 \inferrule* [lab=behavior] {} {\meaningof{\langle a?b \rangle E} = \{ P \in \pi | P \equiv Q | u?(y)P', \\ \and \\\\ \and \\ \;\;\; u \in \meaningof{a}, \forall z.P'\{z/y\} \in \meaningof{E\{z/b\}}\}, \and \\ \meaningof{a!E} = \{ P \in \pi | P \equiv Q | x!\langle P' \rangle, x \in \meaningof{a} P' \in \meaningof{E}\} }
\end{mathpar}

\begin{mathpar}
 \inferrule* [lab=nominal] {} {\meaningof{\quotep{E}} = \{ \quotep{P} \in \quotep{\pi} | P \in \meaningof{E} \}, \and \meaningof{\quotep{P}} = \{ \quotep{Q} \in \quotep{\pi} | P \equiv Q \} \and \\ \meaningof{@\quotep{E}} = \{ P \in \pi | P \equiv @x, x \in \meaningof{E} \}}
\end{mathpar}

\begin{eqnarray*}
  \\
  \meaningof{-} : TS \to ST
\end{eqnarray*}

\begin{eqnarray*}
  \\
  L : TS \to ST
\end{eqnarray*}

\begin{eqnarray*}
  \\
  P \models E \iff P \in \meaningof{E}
\end{eqnarray*}

\begin{eqnarray*}
  P \approx_{L} Q \iff \forall E \in L. P \models E \iff Q \models E
\end{eqnarray*}

\begin{eqnarray*}
  P \approx_{K} Q
\end{eqnarray*}

\begin{eqnarray*}
  P \approx Q
\end{eqnarray*}

$\approx_{K} = \approx = \approx_{L}$

\subsubsection{Contextual duality}

Note that contexts extend the quotation operation to a family of
operations from processes to names. Given a context, $M$, we can
define a \emph{nominal context}, $\quotep{M}$ by $\quotep{M}[P] :=
\quotep{M[P]}$. To foreshadow what is to come we observe that these
operations enjoy a duality with processes very much like the duality
between vectors and maps from vectors to scalars.

Further, because the calculus is essentially higher-order, we have a
correspondence between contexts and processes. More specifically,
given a name $x$ and a context $M$ we can construct $M^{*}_{x}$ such
that 

\begin{mathpar}
  M^{*}_{x} | \lift{x}{P} \red M[P]
\end{mathpar}

namely,

\begin{mathpar}
  M^{*}_{x} := x?(u).M[\dropn{u}]
\end{mathpar}

The dependence of $M^{*}_{x}$ on a name makes it an abstraction, 

\begin{mathpar}
  M^{*} := (x)x?(u).M[\dropn{u}]
\end{mathpar}

\subsection{Additional notation}

It will sometimes be convenient to denote the process a name
quotes. We already have the notation $x = \quotep{P}$, but it will be
convenient to introduce an alternate notation, $\procn{x}$, when we
want to emphasize the connection to the use of the name. Note that, by
virtue of name equivalence, $\quotep{\procn{x}} \nameeq x$; so, the
notation is consistent with previous definitions.

Further, because names have structure it is possible to effect
substitutions on the basis of that structure. This means we need to
upgrade our notation for substitutions, which we accomplish by
adapting comprehension notation. Thus,

\begin{mathpar}
  P\{ y / x : x \in S \}
\end{mathpar}

is interpreted to mean the process derived from P by replacing (in a
capture-avoiding manner) each occurrence of $x$ in $S$ by $y$. For example,

\begin{mathpar}
  P\{ \quotep{\procn{x}|\procn{x}} / x : x \in \freenames{P} \}
\end{mathpar}

will replace each (occurrence) of a free name $x$ in $P$ by
$\quotep{\procn{x}|\procn{x}}$.

Also, we will avail ourselves of the notation $x^{L}$ and $x^{R}$ to
denote injections of a name into disjoint copies of the name
space. There are numerous ways to accomplish this. One example can be
found in \cite{MeredithR05}. This notation overloads to vectors of
names: $\vec{x}^{\pi} := (x_{i}^{\pi} \; : \; 0 \leq i < |\vec{x}| )$ where $\pi \in \{L,R\}$.

We also use $P^{\Box} := P|\Box$.

In \cite{MeredithR05} an interpretation of the new operator is
given. It turns out that there are several possible interpretations
all enjoying the requisite algebraic properties of the operator (see
\cite{milner91polyadicpi}). We will therefore make liberal use of
$(\nu\; \vec{x})P$.

% subsection the_syntax_and_semantics_of_the_notation_system (end)   

\input{qm2pi.qmops} 

\input{qm2pi.sterngerlach} 

\input{qm2pi.metric} 

% section concurrent_process_calculi (end)

%\input{qm2pi.proofsketch}

% section proof sketch (end)

%\input{qm2pi.slviaknots} 

% section spatial logic via knots (end)

\input{qm2pi.conclusion}

% section conclusion (end)

%\input{qm2pi.dtcodes} 

% section wiring algorithm (end)

\input{qm2pi.ack} 

% section acknowledgments (end)

\newpage


\bibliographystyle{plain}   
\bibliography{../../biblios/main.bib}

\input{qm2pi.rhodetails}

\end{document}

 

\documentclass[12pt]{llncs}
%\documentclass{jktr}

\usepackage[pdftex]{hyperref}                   
\usepackage {listings}
\usepackage {mathpartir}
\usepackage{bcprules}
%\usepackage{listings}
                       
\usepackage{graphicx} 
%\usepackage[margins=2.5cm,nohead,nofoot]{geometry}
%\usepackage{geometry}
\usepackage{amsfonts}
\usepackage{amstext}
\usepackage{latexsym}
\usepackage{amssymb}
\usepackage{color}


%\include{myPreamble}
\include{qm2pi.local} 

%\ifpdf
%\usepackage[pdftex]{graphicx}
%\else
%\usepackage{graphicx}
%\fi

 % \ifpdf
%  \usepackage{pdfsync}
%  \if


%\title{Brief Article}
%\author{David F. Snyder}
%\author{L.G. Meredith}

%\address{Dept. of Math., Texas State University--San Marcos, San Marcos, TX 78666}
       
\pagestyle{empty}


\begin{document}

\lstset{language=[Objective]Caml,frame=shadowbox}

\input{qm2pi.front}

% section front matter (end)

\input{qm2pi.intro} 
 
% section introduction (end)

% \input{qm2pi.knotations} 

% section notation (end)

\input{qm2pi.process.calculi} 

% section concurrent_process_calculi_and_spatial_logics_ (end)
    
%\input{qm2pi.knots2pi} 

%\input{qm2pi.trefoil} 

%\input{qm2pi.mainthm} 

% subsection basic_interpretation (end)

%\input{qm2pi.rho.presentation} 
\subsection{The syntax and semantics of the notation system}\label{sub:the_syntax_and_semantics_of_the_notation_system} % (fold)

We now summarize a technical presentation of the calculus that
embodies our theory of dynamics. The typical presentation of such a
calculus follows the style of giving generators and relations on
them. The grammar, below, describing term constructors, freely
generates the set of processes, $\Proc$. This set is then quotiented
by a relation known as structural congruence and it is over this set
that the notion of dynamics is expressed. This presentation is
essentially that of \cite{MeredithR05} with the addition of
polyadicity and summation. For readability we have relegated some of
the technical subtleties to an appendix.

\subsubsection{Process grammar}\label{subsub:process_grammar}

\begin{mathpar}
  \inferrule* [lab=synchronization] {} {{M} \bc \pzero \;|\; x?F \;|\; x!C }
  \and
  \inferrule* [lab=abstraction] {} {{F} \bc (x)P}
  \and
  \inferrule* [lab=concretion] {} {{C} \bc \langle Q \rangle}
  \and
  \inferrule* [lab=process] {} {{P,Q} \bc M \;| \;P|Q \;|\; @{x}}
  \and
  \inferrule* [lab=name] {} {{x} \bc \quotep{P}}
\end{mathpar} 

Note that $\vec{x}$ (resp. $\vec{P}$) denotes a vector of names
(resp. processes) of length $|\vec{x}|$ (resp. $|\vec{P}|$). We adopt
the following useful abbreviations.

\begin{mathpar}
   x?(\vec{y}).P := x.(\vec{y})P \and  x\clift{\vec{P}} := x.\clift{\vec{P}}
   \and x!(y) := \lift{x}{\dropn{y}}
   \and \Pi_{i=0}^{n-1}P_i := P_0 | \ldots | P_{n-1}
\end{mathpar}

\subsubsection{Structural congruence}

\paragraph{Free and bound names and alpha-equivalence.} At the
core of structural equivalence is alpha-equivalence which identifies
process that are the same up to a change of variable. Formally, we
recognize the distinction between free and bound names. The free names
of a process, $\freenames{P}$, may be calculated recursively as
follows:

\begin{mathpar}
\freenames{\pzero} := \emptyset
  \and \\
  \freenames{x?(y).P} := \{ x \} \cup (\freenames{P} \setminus \{ y \})
  \and 
  \freenames{x!\langle P \rangle} := \{ x \} \cup \{ P \} 
  \and \\
  \freenames{P|Q} := \freenames{P} \cup \freenames{Q}
  \and \\
  \freenames{@{x}} := \{ x \}
\end{mathpar}

$\pi$
$\quotep{\pi}$

$\freenames{-} : \pi \to \mathcal{P}(\quotep{\pi})$

\begin{eqnarray*}
  \freenames{\pzero} & := & \emptyset \\
  \freenames{x?(y).P} & := & \{ x \} \cup (\freenames{P} \setminus \{ y \}) \\
  \freenames{x!\langle P \rangle} & := & \{ x \} \cup \{ P \} \\
  \freenames{P|Q} & := & \freenames{P} \cup \freenames{Q} \\
  \freenames{\dropn{x}} & := & \{ x \}
\end{eqnarray*}

The bound names of a process, $\boundnames{P}$, are those names occurring in $P$
that are not free. For example, in $x?(y).0$, the name $x$ is free, while $y$ is bound.

\begin{mathpar}
  \inferrule* [lab=monoidal-laws] {} { P|Q \equiv Q|P \and P|0 \equiv P \and P|(Q|R) \equiv (P|Q)|R }
\end{mathpar}

\begin{mathpar}
  \inferrule* [lab=alpha-equivalence] {} { (x)P \equiv (y)P\{y/x\} \and y \not\in \freenames{P} }
\end{mathpar}

\begin{definition}
Then two processes, $P,Q$, are alpha-equivalent if $P = Q\{\vec{y}/\vec{x}\}$ for
some $\vec{x} \in \boundnames{Q},\vec{y} \in \boundnames{P}$, where $Q\{\vec{y}/\vec{x}\}$
denotes the capture-avoiding substitution of $\vec{y}$ for $\vec{x}$ in $Q$.
\end{definition}

\begin{definition}
  The {\em structural congruence} \cite{SangiorgiWalker} , $\equiv$,
  between processes is the least congruence containing
  alpha-equivalence, satisfying the abelian monoid laws
  (associativity, commutativity and $\pzero$ as identity) for parallel
  composition $|$ and for summation $+$.
\end{definition}

\subsection{Name equivalence}

We take name equivalence, written $\nameeq$, to be the smallest
equivalence relation generated by the following rules.

\begin{mathpar}
\inferrule*[lab=Quote-drop]
{ }
{ \quotep{@{x}} \nameeq x }

\inferrule*[lab=Struct-equiv]
{ P \scong Q }
{ \quotep{P} \nameeq \quotep{Q} }
\end{mathpar}

The astute reader will have noticed that the mutual recursion of names
and processes imposes a mutual recursion on alpha-equivalence and
structural equivalence via name-equivalence. Fortunately, all of this
works out pleasantly and we may calculate in the natural way, free of
concern. The reader interested in the details is referred to the
appendix \ref{appendix:rho_details}.

\subsection{Substitution}

We use $\Proc$ for the set of processes, $\QProc$ for the set of
names, and $\id{\{}\vec{y} / \vec{x} \id{\}}$ to denote partial maps,
$s : \QProc \rightarrow \QProc$. A map, $s$ lifts, uniquely, to a map
on process terms, $\widehat{s} : \Proc \rightarrow \Proc$ by the
following equations.

\begin{mathpar}
  (0) \psubstp{Q}{P} := 0 \\
  (R \juxtap S) \psubstp{Q}{P}
  :=    
  (R)\psubstp{Q}{P} \juxtap (S) \psubstp{Q}{P} \\
  (x?(y).R) \psubstp{Q}{P}    
  :=    
  (x)\substp{Q}{P} (z)\concat( (R \psubstn{z}{y}) \psubstp{Q}{P} ) \\
  (\lift{x}{R}) \psubstp{Q}{P}  
  :=
  \lift{(x)\substp{Q}{P}}{ R \psubstp{Q}{P} } \\
%   (\dropn{x})  \psubstp{Q}{P}       
%   := 
%   \left\{ 
%     \begin{array}{ccc} 
%       \dropn{\quotep{Q}} & & x \nameeq \quotep{P} \\
%       \dropn{x} & & otherwise \\
%     \end{array}
%   \right. 
  (\dropn{x})  \psubstp{Q}{P}       
  := 
  \left\{ 
    \begin{array}{ccc} 
      Q & & x \nameeq \quotep{P} \\
      \dropn{x} & & otherwise \\
    \end{array}
  \right.
\end{mathpar}
 

where

\begin{eqnarray}
  (x)\id{\{} \lpquote Q \rpquote / \lpquote P \rpquote \id{\}}            = 
  \left\{ 
    \begin{array}{ccc}
      \lpquote Q \rpquote & & x \nameeq \lpquote P \rpquote \\
      x & & otherwise \\
    \end{array}
  \right. \nonumber
\end{eqnarray}

and $z$ is chosen distinct from $\quotep{P}$, $\quotep{Q}$, the free
names in $Q$, and all the names in $R$. Our $\alpha$-equivalence will
be built in the standard way from this substitution.

\begin{remark}\label{rem:no_self_referential_names}
  One consequence of these definitions is that $\forall P. \quotep{P}
  \not\in \freenames{P}$.
\end{remark}

\subsection{ Dynamic quote: an example }

Anticipating something of what's to come, consider applying the
substitution, $\widehat{\id{\{}u / z \id{\}}}$, to the following pair
of processes, $\lift{w}{y!(z)}$ and $w[ \lpquote y!(z) \rpquote ]$.

\begin{eqnarray}
	\lift{w}{y!(z)}\widehat{\id{\{}u / z \id{\}}}
		& = &
		\lift{w}{y!(u)} \nonumber\\
	w[ \lpquote y!(z) \rpquote ] \widehat{ \id{\{}u / z \id{\}} }
		& = &
		w[ \lpquote y!(z) \rpquote ] \nonumber
\end{eqnarray}

Because the body of the process between quotes is impervious to
substitution, we get radically different answers. In fact, by
examining the first process in an input context,
e.g. $x?(z).\lift{w}{y!(z)}$, we see that the process under the lift
operator may be shaped by prefixed inputs binding a name inside it. In
this sense, the lift operator will be seen as a way to dynamically
construct processes before reifying them as names.

Finally equipped with these standard features we can present the
dynamics of the calculus.

\subsubsection{Operational semantics} 

Finally, we introduce the computational dynamics. What marks these
algebras as distinct from other more traditionally studied algebraic
structures, e.g. vector spaces or polynomial rings, is the manner in
which dynamics is captured. In traditional structures, dynamics is typically
expressed through morphisms between such structures, as in linear maps
between vector spaces or morphisms between rings. In algebras
associated with the semantics of computation, the dynamics is
expressed as part of the algebraic structure itself, through a
reduction reduction relation typically denoted by $\red$. Below, we
give a recursive presentation of this relation for the calculus used
in the encoding.

$\red \subseteq \pi \times \pi$
$\red : \pi \to \mathcal{P}(\pi)$

\begin{mathpar}
  \inferrule* [lab=Comm] { \textsf{match}( x_{src}, x_{trgt} ) } { x_{trgt}?(y)P \; | \; x_{src}!\langle {Q} \rangle \red P\{\quotep{Q}/y}\} }
  \and \\
  \inferrule* [lab=Par] {{P} \red {P}'} {{{P} | {Q}} \red {{P}' | {Q}}}
  \and
  \inferrule* [lab=Equiv]{{{P} \scong {P}'} \andalso {{P}' \red {Q}'} \andalso {{Q}' \scong {Q}}}{{P} \red {Q}}
\end{mathpar}

\begin{eqnarray*}
  match_{\equiv} (\quotep{P},\quotep{Q}) & := & P \equiv Q \\
  match_{\dagger}(\quotep{P},\quotep{Q}) & := & \forall R. P|Q \red^{*} R => R \red^{*} 0 \\
  match_{K}(\quotep{P},\quotep{Q}) & := & K \mbox{ for some context } K
\end{eqnarray*}

$u?(x)P | u!\langle Q \rangle \red P\{\quotep{Q}/x\}$

%We write $\wred$ for $\red^*$, and $P\red$ if $\exists Q $ such that $ P \red Q$.
We write $P\red$ if $\exists Q $ such that $ P \red Q$ and $P\not\red$, otherwise.

\section{Replication}

As mentioned before, it is known that replication (and hence
recursion) can be implemented in a higher-order process algebra
\cite{SangiorgiWalker}. As our first example of calculation with the
machinery thus far presented we give the construction explicitly in
the {\rhoc}.

\begin{eqnarray}
	D_{x} & := & \prefix{x}{y}{(\binpar{\outputp{x}{y}}{@{y}})} \nonumber\\
	\bangp_{x}{P} & := & \binpar{{x}!\langle{\binpar{D_{x}}{P}}\rangle}{D_{x}} \nonumber
\end{eqnarray}

\begin{eqnarray}
	\bangp_{x}{P} & & \nonumber\\
	=
	& {x}!\langle{(\prefix{x}{y}{(\outputp{x}{y} | @{y})) | P}}\rangle 
	      | \prefix{x}{y}{(\outputp{x}{y} | @{y})} & \nonumber\\
	\red
	& (\outputp{x}{y} | @{y})\substn{\quotep{(\prefix{x}{y}{(@{y} | \outputp{x}{y})) | P}}}{y} & \nonumber\\
	=
	& \outputp{x}{\quotep{(\prefix{x}{y}{(\outputp{x}{y} | @{y})) | P}}}
	  | {(\prefix{x}{y}{(\outputp{x}{y} | @{y})) | P}} & \nonumber\\
	\red
	& \ldots & \nonumber\\
	\red^*
	& P | P | \ldots & \nonumber
\end{eqnarray}

Of course, this encoding, as an implementation, runs away, unfolding
$\bangp{P}$ eagerly. A lazier and more implementable replication
operator, restricted to input-guarded processes, may be obtained as follows.

\begin{eqnarray}
\bangp{\prefix{u}{v}{P}} 
	:= 
	\binpar{\lift{x}{\prefix{u}{v}{(\binpar{D(x)}{P})}}}{D(x)} \nonumber
\end{eqnarray}

\begin{remark}
  Note that the lazier definition still does not deal with summation
  or mixed summation (i.e. sums over input and output). The reader is
  invited to construct definitions of replication that deal with these
  features. 

  Further, the definitions are parameterized in a name, $x$. Can you,
  gentle reader, make a definition that eliminates this parameter and
  guarantees no accidental interaction between the replication
  machinery and the process being replicated -- i.e. no accidental
  sharing of names used by the process to get its work done and the
  name(s) used by the replication to effect copying. This latter
  revision of the definition of replication is crucial to obtaining
  the expected identity $!!P \sim !P$.
\end{remark}

\begin{remark}\label{rem:paradoxical_combinator}
  The reader familiar with the lambda calculus will have noticed the
  similarity between $D$ and the paradoxical combinator.

  [Ed. note: the existence of this seems to suggest we have to be more
  restrictive on the set of processes and names we admit if we are to
  support no-cloning.]
\end{remark}

\subsubsection{Bisimulation}

The computational dynamics gives rise to another kind of equivalence,
the equivalence of computational behavior. As previously mentioned
this is typically captured \emph{via} some form of bisimulation.

% The notion we use in this paper is weak barbed bisimulation
% \cite{milner91polyadicpi}.

The notion we use in this paper is derived from weak barbed
bisimulation \cite{milner91polyadicpi}. 

\begin{definition}
An \emph{observation relation}, $\downarrow_{\mathcal N}$, over a set
of names, $\mathcal N$, is the smallest relation satisfying the rules
below.

\infrule[Out-barb]{y \in {\mathcal N}, \; x \nameeq y}
		  {\outputp{x}{v} \downarrow_{\mathcal N} x}
\infrule[Par-barb]{\mbox{$P\downarrow_{\mathcal N} x$ or $Q\downarrow_{\mathcal N} x$}}
		  {\binpar{P}{Q} \downarrow_{\mathcal N} x}

We write $P \Downarrow_{\mathcal N} x$ if there is $Q$ such that 
$P \wred Q$ and $Q \downarrow_{\mathcal N} x$.
\end{definition}

\begin{definition}
%\label{def.bbisim}
An  ${\mathcal N}$-\emph{barbed bisimulation} over a set of names, ${\mathcal N}$, is a symmetric binary relation 
${\mathcal S}_{\mathcal N}$ between agents such that $P\rel{S}_{\mathcal N}Q$ implies:
\begin{enumerate}
\item If $P \red P'$ then $Q \wred Q'$ and $P'\rel{S}_{\mathcal N} Q'$.
\item If $P\downarrow_{\mathcal N} x$, then $Q\Downarrow_{\mathcal N} x$.
\end{enumerate}
$P$ is ${\mathcal N}$-barbed bisimilar to $Q$, written
$P \wbbisim_{\mathcal N} Q$, if $P \rel{S}_{\mathcal N} Q$ for some ${\mathcal N}$-barbed bisimulation ${\mathcal S}_{\mathcal N}$.
\end{definition}

$\mathcal{R} \subseteq \pi \times \pi$

$P \mathcal{R} Q => \forall P'. P \red P' \Rightarrow \exists Q'. Q \red Q', P' \mathcal{R} Q'$

$P \vdash x \Rightarrow Q \vdash x$

\begin{mathpar}
  \inferrule*[lab=Out-barb]{x \nameeq y}{{y}!\langle{Q}\rangle \vdash x}
  \and
  \inferrule*[lab=Par-barb]{\mbox{$P\vdash x$ or $Q\vdash x$}}{\binpar{P}{Q} \vdash x}
\end{mathpar}

\subsubsection{Contexts}

One of the principle advantages of computational calculi like the
$\pi$-calculus is a well-defined notion of context,
contextual-equivalence and a correlation between
contextual-equivalence and notions of bisimulation. The notion of
context allows the decomposition of a process into (sub-)process and
its syntactic environment, its context. Thus, a context may be
thought of as a process with a ``hole'' (written $\Box$) in it. The
application of a context $M$ to a process $P$, written $M[P]$, is
tantamount to filling the hole in $M$ with $P$. In this paper we do
not need the full weight of this theory, but do make use of the notion
of context in the proof the main theorem. 

\begin{mathpar}
  \inferrule* [lab=summation] {} {{M_{M},M_{N}} \bc \Box \;|\; x.M_{A} \;|\; M_{M}+M_{N}}
  \and
  \inferrule* [lab=agent] {} {{M_{A}} \bc (\vec{x})M_{P} \;| \; \clift{P_0,\ldots,M_{P},\ldots,P_N}}
  \and \\
  \inferrule* [lab=process] {} {{M_{P}} \bc M_{N} \;| \;P|M_{P} }
\end{mathpar} 

\begin{mathpar}
  \inferrule* [lab=sychronization] {} {M_{N} \bc \Box \;|\; x?M_{F} \;|\; x!M_{C}}
  \and
  \inferrule* [lab=abstraction] {} {{M_{F}} \bc (x)M_{P} }
  \and
  \inferrule* [lab=concretion] {} {{M_{C}} \bc \langle M_{P} \rangle }
  \and \\
  \inferrule* [lab=process] {} {{M_{P}} \bc M_{N} \;| \;P|M_{P} }
\end{mathpar}

\begin{definition}[contextual application] Given a context $M$, and
  process $P$, we define the \emph{contextual application}, $M[P] :=
  M\{P/\Box\}$. That is, the contextual application of M to P is the
  substitution of $P$ for $\Box$ in $M$.
\end{definition}

$\meaningof{-} : L \to \mathcal{P}(\pi)$

\begin{mathpar}
  \inferrule* [lab=collection] {} {\meaningof{true} = \pi, \and \meaningof{~E} = \pi \setminus \meaningof{E}, \and \meaningof{E_{1} \& E_{2}} = \meaningof{E_{1}} \cap \meaningof{E_{2}}}
\end{mathpar}

\begin{mathpar}
  \inferrule* [lab=structure] {} {\meaningof{0} = \{ P \in \pi | P \equiv 0 \}, \and \\ \meaningof{E_1 | E_2} = \{ P \in \pi | P \equiv P_{1} | P_{2}, P_{1} \in \meaningof{E_{1}}, P_{2} \in \meaningof{E_2}\} }
\end{mathpar}

\begin{mathpar}
 \inferrule* [lab=behavior] {} {\meaningof{\langle a?b \rangle E} = \{ P \in \pi | P \equiv Q | u?(y)P', \\ \and \\\\ \and \\ \;\;\; u \in \meaningof{a}, \forall z.P'\{z/y\} \in \meaningof{E\{z/b\}}\}, \and \\ \meaningof{a!E} = \{ P \in \pi | P \equiv Q | x!\langle P' \rangle, x \in \meaningof{a} P' \in \meaningof{E}\} }
\end{mathpar}

\begin{mathpar}
 \inferrule* [lab=nominal] {} {\meaningof{\quotep{E}} = \{ \quotep{P} \in \quotep{\pi} | P \in \meaningof{E} \}, \and \meaningof{\quotep{P}} = \{ \quotep{Q} \in \quotep{\pi} | P \equiv Q \} \and \\ \meaningof{@\quotep{E}} = \{ P \in \pi | P \equiv @x, x \in \meaningof{E} \}}
\end{mathpar}

\begin{eqnarray*}
  \\
  \meaningof{-} : TS \to ST
\end{eqnarray*}

\begin{eqnarray*}
  \\
  L : TS \to ST
\end{eqnarray*}

\begin{eqnarray*}
  \\
  P \models E \iff P \in \meaningof{E}
\end{eqnarray*}

\begin{eqnarray*}
  P \approx_{L} Q \iff \forall E \in L. P \models E \iff Q \models E
\end{eqnarray*}

\begin{eqnarray*}
  P \approx_{K} Q
\end{eqnarray*}

\begin{eqnarray*}
  P \approx Q
\end{eqnarray*}

$\approx_{K} = \approx = \approx_{L}$

\subsubsection{Contextual duality}

Note that contexts extend the quotation operation to a family of
operations from processes to names. Given a context, $M$, we can
define a \emph{nominal context}, $\quotep{M}$ by $\quotep{M}[P] :=
\quotep{M[P]}$. To foreshadow what is to come we observe that these
operations enjoy a duality with processes very much like the duality
between vectors and maps from vectors to scalars.

Further, because the calculus is essentially higher-order, we have a
correspondence between contexts and processes. More specifically,
given a name $x$ and a context $M$ we can construct $M^{*}_{x}$ such
that 

\begin{mathpar}
  M^{*}_{x} | \lift{x}{P} \red M[P]
\end{mathpar}

namely,

\begin{mathpar}
  M^{*}_{x} := x?(u).M[\dropn{u}]
\end{mathpar}

The dependence of $M^{*}_{x}$ on a name makes it an abstraction, 

\begin{mathpar}
  M^{*} := (x)x?(u).M[\dropn{u}]
\end{mathpar}

\subsection{Additional notation}

It will sometimes be convenient to denote the process a name
quotes. We already have the notation $x = \quotep{P}$, but it will be
convenient to introduce an alternate notation, $\procn{x}$, when we
want to emphasize the connection to the use of the name. Note that, by
virtue of name equivalence, $\quotep{\procn{x}} \nameeq x$; so, the
notation is consistent with previous definitions.

Further, because names have structure it is possible to effect
substitutions on the basis of that structure. This means we need to
upgrade our notation for substitutions, which we accomplish by
adapting comprehension notation. Thus,

\begin{mathpar}
  P\{ y / x : x \in S \}
\end{mathpar}

is interpreted to mean the process derived from P by replacing (in a
capture-avoiding manner) each occurrence of $x$ in $S$ by $y$. For example,

\begin{mathpar}
  P\{ \quotep{\procn{x}|\procn{x}} / x : x \in \freenames{P} \}
\end{mathpar}

will replace each (occurrence) of a free name $x$ in $P$ by
$\quotep{\procn{x}|\procn{x}}$.

Also, we will avail ourselves of the notation $x^{L}$ and $x^{R}$ to
denote injections of a name into disjoint copies of the name
space. There are numerous ways to accomplish this. One example can be
found in \cite{MeredithR05}. This notation overloads to vectors of
names: $\vec{x}^{\pi} := (x_{i}^{\pi} \; : \; 0 \leq i < |\vec{x}| )$ where $\pi \in \{L,R\}$.

We also use $P^{\Box} := P|\Box$.

In \cite{MeredithR05} an interpretation of the new operator is
given. It turns out that there are several possible interpretations
all enjoying the requisite algebraic properties of the operator (see
\cite{milner91polyadicpi}). We will therefore make liberal use of
$(\nu\; \vec{x})P$.

% subsection the_syntax_and_semantics_of_the_notation_system (end)   

\input{qm2pi.qmops} 

\input{qm2pi.sterngerlach} 

\input{qm2pi.metric} 

% section concurrent_process_calculi (end)

%\input{qm2pi.proofsketch}

% section proof sketch (end)

%\input{qm2pi.slviaknots} 

% section spatial logic via knots (end)

\input{qm2pi.conclusion}

% section conclusion (end)

%\input{qm2pi.dtcodes} 

% section wiring algorithm (end)

\input{qm2pi.ack} 

% section acknowledgments (end)

\newpage


\bibliographystyle{plain}   
\bibliography{../../biblios/main.bib}

\input{qm2pi.rhodetails}

\end{document}

 

% section concurrent_process_calculi (end)

%\documentclass[12pt]{llncs}
%\documentclass{jktr}

\usepackage[pdftex]{hyperref}                   
\usepackage {listings}
\usepackage {mathpartir}
\usepackage{bcprules}
%\usepackage{listings}
                       
\usepackage{graphicx} 
%\usepackage[margins=2.5cm,nohead,nofoot]{geometry}
%\usepackage{geometry}
\usepackage{amsfonts}
\usepackage{amstext}
\usepackage{latexsym}
\usepackage{amssymb}
\usepackage{color}


%\include{myPreamble}
\include{qm2pi.local} 

%\ifpdf
%\usepackage[pdftex]{graphicx}
%\else
%\usepackage{graphicx}
%\fi

 % \ifpdf
%  \usepackage{pdfsync}
%  \if


%\title{Brief Article}
%\author{David F. Snyder}
%\author{L.G. Meredith}

%\address{Dept. of Math., Texas State University--San Marcos, San Marcos, TX 78666}
       
\pagestyle{empty}


\begin{document}

\lstset{language=[Objective]Caml,frame=shadowbox}

\input{qm2pi.front}

% section front matter (end)

\input{qm2pi.intro} 
 
% section introduction (end)

% \input{qm2pi.knotations} 

% section notation (end)

\input{qm2pi.process.calculi} 

% section concurrent_process_calculi_and_spatial_logics_ (end)
    
%\input{qm2pi.knots2pi} 

%\input{qm2pi.trefoil} 

%\input{qm2pi.mainthm} 

% subsection basic_interpretation (end)

%\input{qm2pi.rho.presentation} 
\subsection{The syntax and semantics of the notation system}\label{sub:the_syntax_and_semantics_of_the_notation_system} % (fold)

We now summarize a technical presentation of the calculus that
embodies our theory of dynamics. The typical presentation of such a
calculus follows the style of giving generators and relations on
them. The grammar, below, describing term constructors, freely
generates the set of processes, $\Proc$. This set is then quotiented
by a relation known as structural congruence and it is over this set
that the notion of dynamics is expressed. This presentation is
essentially that of \cite{MeredithR05} with the addition of
polyadicity and summation. For readability we have relegated some of
the technical subtleties to an appendix.

\subsubsection{Process grammar}\label{subsub:process_grammar}

\begin{mathpar}
  \inferrule* [lab=synchronization] {} {{M} \bc \pzero \;|\; x?F \;|\; x!C }
  \and
  \inferrule* [lab=abstraction] {} {{F} \bc (x)P}
  \and
  \inferrule* [lab=concretion] {} {{C} \bc \langle Q \rangle}
  \and
  \inferrule* [lab=process] {} {{P,Q} \bc M \;| \;P|Q \;|\; @{x}}
  \and
  \inferrule* [lab=name] {} {{x} \bc \quotep{P}}
\end{mathpar} 

Note that $\vec{x}$ (resp. $\vec{P}$) denotes a vector of names
(resp. processes) of length $|\vec{x}|$ (resp. $|\vec{P}|$). We adopt
the following useful abbreviations.

\begin{mathpar}
   x?(\vec{y}).P := x.(\vec{y})P \and  x\clift{\vec{P}} := x.\clift{\vec{P}}
   \and x!(y) := \lift{x}{\dropn{y}}
   \and \Pi_{i=0}^{n-1}P_i := P_0 | \ldots | P_{n-1}
\end{mathpar}

\subsubsection{Structural congruence}

\paragraph{Free and bound names and alpha-equivalence.} At the
core of structural equivalence is alpha-equivalence which identifies
process that are the same up to a change of variable. Formally, we
recognize the distinction between free and bound names. The free names
of a process, $\freenames{P}$, may be calculated recursively as
follows:

\begin{mathpar}
\freenames{\pzero} := \emptyset
  \and \\
  \freenames{x?(y).P} := \{ x \} \cup (\freenames{P} \setminus \{ y \})
  \and 
  \freenames{x!\langle P \rangle} := \{ x \} \cup \{ P \} 
  \and \\
  \freenames{P|Q} := \freenames{P} \cup \freenames{Q}
  \and \\
  \freenames{@{x}} := \{ x \}
\end{mathpar}

$\pi$
$\quotep{\pi}$

$\freenames{-} : \pi \to \mathcal{P}(\quotep{\pi})$

\begin{eqnarray*}
  \freenames{\pzero} & := & \emptyset \\
  \freenames{x?(y).P} & := & \{ x \} \cup (\freenames{P} \setminus \{ y \}) \\
  \freenames{x!\langle P \rangle} & := & \{ x \} \cup \{ P \} \\
  \freenames{P|Q} & := & \freenames{P} \cup \freenames{Q} \\
  \freenames{\dropn{x}} & := & \{ x \}
\end{eqnarray*}

The bound names of a process, $\boundnames{P}$, are those names occurring in $P$
that are not free. For example, in $x?(y).0$, the name $x$ is free, while $y$ is bound.

\begin{mathpar}
  \inferrule* [lab=monoidal-laws] {} { P|Q \equiv Q|P \and P|0 \equiv P \and P|(Q|R) \equiv (P|Q)|R }
\end{mathpar}

\begin{mathpar}
  \inferrule* [lab=alpha-equivalence] {} { (x)P \equiv (y)P\{y/x\} \and y \not\in \freenames{P} }
\end{mathpar}

\begin{definition}
Then two processes, $P,Q$, are alpha-equivalent if $P = Q\{\vec{y}/\vec{x}\}$ for
some $\vec{x} \in \boundnames{Q},\vec{y} \in \boundnames{P}$, where $Q\{\vec{y}/\vec{x}\}$
denotes the capture-avoiding substitution of $\vec{y}$ for $\vec{x}$ in $Q$.
\end{definition}

\begin{definition}
  The {\em structural congruence} \cite{SangiorgiWalker} , $\equiv$,
  between processes is the least congruence containing
  alpha-equivalence, satisfying the abelian monoid laws
  (associativity, commutativity and $\pzero$ as identity) for parallel
  composition $|$ and for summation $+$.
\end{definition}

\subsection{Name equivalence}

We take name equivalence, written $\nameeq$, to be the smallest
equivalence relation generated by the following rules.

\begin{mathpar}
\inferrule*[lab=Quote-drop]
{ }
{ \quotep{@{x}} \nameeq x }

\inferrule*[lab=Struct-equiv]
{ P \scong Q }
{ \quotep{P} \nameeq \quotep{Q} }
\end{mathpar}

The astute reader will have noticed that the mutual recursion of names
and processes imposes a mutual recursion on alpha-equivalence and
structural equivalence via name-equivalence. Fortunately, all of this
works out pleasantly and we may calculate in the natural way, free of
concern. The reader interested in the details is referred to the
appendix \ref{appendix:rho_details}.

\subsection{Substitution}

We use $\Proc$ for the set of processes, $\QProc$ for the set of
names, and $\id{\{}\vec{y} / \vec{x} \id{\}}$ to denote partial maps,
$s : \QProc \rightarrow \QProc$. A map, $s$ lifts, uniquely, to a map
on process terms, $\widehat{s} : \Proc \rightarrow \Proc$ by the
following equations.

\begin{mathpar}
  (0) \psubstp{Q}{P} := 0 \\
  (R \juxtap S) \psubstp{Q}{P}
  :=    
  (R)\psubstp{Q}{P} \juxtap (S) \psubstp{Q}{P} \\
  (x?(y).R) \psubstp{Q}{P}    
  :=    
  (x)\substp{Q}{P} (z)\concat( (R \psubstn{z}{y}) \psubstp{Q}{P} ) \\
  (\lift{x}{R}) \psubstp{Q}{P}  
  :=
  \lift{(x)\substp{Q}{P}}{ R \psubstp{Q}{P} } \\
%   (\dropn{x})  \psubstp{Q}{P}       
%   := 
%   \left\{ 
%     \begin{array}{ccc} 
%       \dropn{\quotep{Q}} & & x \nameeq \quotep{P} \\
%       \dropn{x} & & otherwise \\
%     \end{array}
%   \right. 
  (\dropn{x})  \psubstp{Q}{P}       
  := 
  \left\{ 
    \begin{array}{ccc} 
      Q & & x \nameeq \quotep{P} \\
      \dropn{x} & & otherwise \\
    \end{array}
  \right.
\end{mathpar}
 

where

\begin{eqnarray}
  (x)\id{\{} \lpquote Q \rpquote / \lpquote P \rpquote \id{\}}            = 
  \left\{ 
    \begin{array}{ccc}
      \lpquote Q \rpquote & & x \nameeq \lpquote P \rpquote \\
      x & & otherwise \\
    \end{array}
  \right. \nonumber
\end{eqnarray}

and $z$ is chosen distinct from $\quotep{P}$, $\quotep{Q}$, the free
names in $Q$, and all the names in $R$. Our $\alpha$-equivalence will
be built in the standard way from this substitution.

\begin{remark}\label{rem:no_self_referential_names}
  One consequence of these definitions is that $\forall P. \quotep{P}
  \not\in \freenames{P}$.
\end{remark}

\subsection{ Dynamic quote: an example }

Anticipating something of what's to come, consider applying the
substitution, $\widehat{\id{\{}u / z \id{\}}}$, to the following pair
of processes, $\lift{w}{y!(z)}$ and $w[ \lpquote y!(z) \rpquote ]$.

\begin{eqnarray}
	\lift{w}{y!(z)}\widehat{\id{\{}u / z \id{\}}}
		& = &
		\lift{w}{y!(u)} \nonumber\\
	w[ \lpquote y!(z) \rpquote ] \widehat{ \id{\{}u / z \id{\}} }
		& = &
		w[ \lpquote y!(z) \rpquote ] \nonumber
\end{eqnarray}

Because the body of the process between quotes is impervious to
substitution, we get radically different answers. In fact, by
examining the first process in an input context,
e.g. $x?(z).\lift{w}{y!(z)}$, we see that the process under the lift
operator may be shaped by prefixed inputs binding a name inside it. In
this sense, the lift operator will be seen as a way to dynamically
construct processes before reifying them as names.

Finally equipped with these standard features we can present the
dynamics of the calculus.

\subsubsection{Operational semantics} 

Finally, we introduce the computational dynamics. What marks these
algebras as distinct from other more traditionally studied algebraic
structures, e.g. vector spaces or polynomial rings, is the manner in
which dynamics is captured. In traditional structures, dynamics is typically
expressed through morphisms between such structures, as in linear maps
between vector spaces or morphisms between rings. In algebras
associated with the semantics of computation, the dynamics is
expressed as part of the algebraic structure itself, through a
reduction reduction relation typically denoted by $\red$. Below, we
give a recursive presentation of this relation for the calculus used
in the encoding.

$\red \subseteq \pi \times \pi$
$\red : \pi \to \mathcal{P}(\pi)$

\begin{mathpar}
  \inferrule* [lab=Comm] { \textsf{match}( x_{src}, x_{trgt} ) } { x_{trgt}?(y)P \; | \; x_{src}!\langle {Q} \rangle \red P\{\quotep{Q}/y}\} }
  \and \\
  \inferrule* [lab=Par] {{P} \red {P}'} {{{P} | {Q}} \red {{P}' | {Q}}}
  \and
  \inferrule* [lab=Equiv]{{{P} \scong {P}'} \andalso {{P}' \red {Q}'} \andalso {{Q}' \scong {Q}}}{{P} \red {Q}}
\end{mathpar}

\begin{eqnarray*}
  match_{\equiv} (\quotep{P},\quotep{Q}) & := & P \equiv Q \\
  match_{\dagger}(\quotep{P},\quotep{Q}) & := & \forall R. P|Q \red^{*} R => R \red^{*} 0 \\
  match_{K}(\quotep{P},\quotep{Q}) & := & K \mbox{ for some context } K
\end{eqnarray*}

$u?(x)P | u!\langle Q \rangle \red P\{\quotep{Q}/x\}$

%We write $\wred$ for $\red^*$, and $P\red$ if $\exists Q $ such that $ P \red Q$.
We write $P\red$ if $\exists Q $ such that $ P \red Q$ and $P\not\red$, otherwise.

\section{Replication}

As mentioned before, it is known that replication (and hence
recursion) can be implemented in a higher-order process algebra
\cite{SangiorgiWalker}. As our first example of calculation with the
machinery thus far presented we give the construction explicitly in
the {\rhoc}.

\begin{eqnarray}
	D_{x} & := & \prefix{x}{y}{(\binpar{\outputp{x}{y}}{@{y}})} \nonumber\\
	\bangp_{x}{P} & := & \binpar{{x}!\langle{\binpar{D_{x}}{P}}\rangle}{D_{x}} \nonumber
\end{eqnarray}

\begin{eqnarray}
	\bangp_{x}{P} & & \nonumber\\
	=
	& {x}!\langle{(\prefix{x}{y}{(\outputp{x}{y} | @{y})) | P}}\rangle 
	      | \prefix{x}{y}{(\outputp{x}{y} | @{y})} & \nonumber\\
	\red
	& (\outputp{x}{y} | @{y})\substn{\quotep{(\prefix{x}{y}{(@{y} | \outputp{x}{y})) | P}}}{y} & \nonumber\\
	=
	& \outputp{x}{\quotep{(\prefix{x}{y}{(\outputp{x}{y} | @{y})) | P}}}
	  | {(\prefix{x}{y}{(\outputp{x}{y} | @{y})) | P}} & \nonumber\\
	\red
	& \ldots & \nonumber\\
	\red^*
	& P | P | \ldots & \nonumber
\end{eqnarray}

Of course, this encoding, as an implementation, runs away, unfolding
$\bangp{P}$ eagerly. A lazier and more implementable replication
operator, restricted to input-guarded processes, may be obtained as follows.

\begin{eqnarray}
\bangp{\prefix{u}{v}{P}} 
	:= 
	\binpar{\lift{x}{\prefix{u}{v}{(\binpar{D(x)}{P})}}}{D(x)} \nonumber
\end{eqnarray}

\begin{remark}
  Note that the lazier definition still does not deal with summation
  or mixed summation (i.e. sums over input and output). The reader is
  invited to construct definitions of replication that deal with these
  features. 

  Further, the definitions are parameterized in a name, $x$. Can you,
  gentle reader, make a definition that eliminates this parameter and
  guarantees no accidental interaction between the replication
  machinery and the process being replicated -- i.e. no accidental
  sharing of names used by the process to get its work done and the
  name(s) used by the replication to effect copying. This latter
  revision of the definition of replication is crucial to obtaining
  the expected identity $!!P \sim !P$.
\end{remark}

\begin{remark}\label{rem:paradoxical_combinator}
  The reader familiar with the lambda calculus will have noticed the
  similarity between $D$ and the paradoxical combinator.

  [Ed. note: the existence of this seems to suggest we have to be more
  restrictive on the set of processes and names we admit if we are to
  support no-cloning.]
\end{remark}

\subsubsection{Bisimulation}

The computational dynamics gives rise to another kind of equivalence,
the equivalence of computational behavior. As previously mentioned
this is typically captured \emph{via} some form of bisimulation.

% The notion we use in this paper is weak barbed bisimulation
% \cite{milner91polyadicpi}.

The notion we use in this paper is derived from weak barbed
bisimulation \cite{milner91polyadicpi}. 

\begin{definition}
An \emph{observation relation}, $\downarrow_{\mathcal N}$, over a set
of names, $\mathcal N$, is the smallest relation satisfying the rules
below.

\infrule[Out-barb]{y \in {\mathcal N}, \; x \nameeq y}
		  {\outputp{x}{v} \downarrow_{\mathcal N} x}
\infrule[Par-barb]{\mbox{$P\downarrow_{\mathcal N} x$ or $Q\downarrow_{\mathcal N} x$}}
		  {\binpar{P}{Q} \downarrow_{\mathcal N} x}

We write $P \Downarrow_{\mathcal N} x$ if there is $Q$ such that 
$P \wred Q$ and $Q \downarrow_{\mathcal N} x$.
\end{definition}

\begin{definition}
%\label{def.bbisim}
An  ${\mathcal N}$-\emph{barbed bisimulation} over a set of names, ${\mathcal N}$, is a symmetric binary relation 
${\mathcal S}_{\mathcal N}$ between agents such that $P\rel{S}_{\mathcal N}Q$ implies:
\begin{enumerate}
\item If $P \red P'$ then $Q \wred Q'$ and $P'\rel{S}_{\mathcal N} Q'$.
\item If $P\downarrow_{\mathcal N} x$, then $Q\Downarrow_{\mathcal N} x$.
\end{enumerate}
$P$ is ${\mathcal N}$-barbed bisimilar to $Q$, written
$P \wbbisim_{\mathcal N} Q$, if $P \rel{S}_{\mathcal N} Q$ for some ${\mathcal N}$-barbed bisimulation ${\mathcal S}_{\mathcal N}$.
\end{definition}

$\mathcal{R} \subseteq \pi \times \pi$

$P \mathcal{R} Q => \forall P'. P \red P' \Rightarrow \exists Q'. Q \red Q', P' \mathcal{R} Q'$

$P \vdash x \Rightarrow Q \vdash x$

\begin{mathpar}
  \inferrule*[lab=Out-barb]{x \nameeq y}{{y}!\langle{Q}\rangle \vdash x}
  \and
  \inferrule*[lab=Par-barb]{\mbox{$P\vdash x$ or $Q\vdash x$}}{\binpar{P}{Q} \vdash x}
\end{mathpar}

\subsubsection{Contexts}

One of the principle advantages of computational calculi like the
$\pi$-calculus is a well-defined notion of context,
contextual-equivalence and a correlation between
contextual-equivalence and notions of bisimulation. The notion of
context allows the decomposition of a process into (sub-)process and
its syntactic environment, its context. Thus, a context may be
thought of as a process with a ``hole'' (written $\Box$) in it. The
application of a context $M$ to a process $P$, written $M[P]$, is
tantamount to filling the hole in $M$ with $P$. In this paper we do
not need the full weight of this theory, but do make use of the notion
of context in the proof the main theorem. 

\begin{mathpar}
  \inferrule* [lab=summation] {} {{M_{M},M_{N}} \bc \Box \;|\; x.M_{A} \;|\; M_{M}+M_{N}}
  \and
  \inferrule* [lab=agent] {} {{M_{A}} \bc (\vec{x})M_{P} \;| \; \clift{P_0,\ldots,M_{P},\ldots,P_N}}
  \and \\
  \inferrule* [lab=process] {} {{M_{P}} \bc M_{N} \;| \;P|M_{P} }
\end{mathpar} 

\begin{mathpar}
  \inferrule* [lab=sychronization] {} {M_{N} \bc \Box \;|\; x?M_{F} \;|\; x!M_{C}}
  \and
  \inferrule* [lab=abstraction] {} {{M_{F}} \bc (x)M_{P} }
  \and
  \inferrule* [lab=concretion] {} {{M_{C}} \bc \langle M_{P} \rangle }
  \and \\
  \inferrule* [lab=process] {} {{M_{P}} \bc M_{N} \;| \;P|M_{P} }
\end{mathpar}

\begin{definition}[contextual application] Given a context $M$, and
  process $P$, we define the \emph{contextual application}, $M[P] :=
  M\{P/\Box\}$. That is, the contextual application of M to P is the
  substitution of $P$ for $\Box$ in $M$.
\end{definition}

$\meaningof{-} : L \to \mathcal{P}(\pi)$

\begin{mathpar}
  \inferrule* [lab=collection] {} {\meaningof{true} = \pi, \and \meaningof{~E} = \pi \setminus \meaningof{E}, \and \meaningof{E_{1} \& E_{2}} = \meaningof{E_{1}} \cap \meaningof{E_{2}}}
\end{mathpar}

\begin{mathpar}
  \inferrule* [lab=structure] {} {\meaningof{0} = \{ P \in \pi | P \equiv 0 \}, \and \\ \meaningof{E_1 | E_2} = \{ P \in \pi | P \equiv P_{1} | P_{2}, P_{1} \in \meaningof{E_{1}}, P_{2} \in \meaningof{E_2}\} }
\end{mathpar}

\begin{mathpar}
 \inferrule* [lab=behavior] {} {\meaningof{\langle a?b \rangle E} = \{ P \in \pi | P \equiv Q | u?(y)P', \\ \and \\\\ \and \\ \;\;\; u \in \meaningof{a}, \forall z.P'\{z/y\} \in \meaningof{E\{z/b\}}\}, \and \\ \meaningof{a!E} = \{ P \in \pi | P \equiv Q | x!\langle P' \rangle, x \in \meaningof{a} P' \in \meaningof{E}\} }
\end{mathpar}

\begin{mathpar}
 \inferrule* [lab=nominal] {} {\meaningof{\quotep{E}} = \{ \quotep{P} \in \quotep{\pi} | P \in \meaningof{E} \}, \and \meaningof{\quotep{P}} = \{ \quotep{Q} \in \quotep{\pi} | P \equiv Q \} \and \\ \meaningof{@\quotep{E}} = \{ P \in \pi | P \equiv @x, x \in \meaningof{E} \}}
\end{mathpar}

\begin{eqnarray*}
  \\
  \meaningof{-} : TS \to ST
\end{eqnarray*}

\begin{eqnarray*}
  \\
  L : TS \to ST
\end{eqnarray*}

\begin{eqnarray*}
  \\
  P \models E \iff P \in \meaningof{E}
\end{eqnarray*}

\begin{eqnarray*}
  P \approx_{L} Q \iff \forall E \in L. P \models E \iff Q \models E
\end{eqnarray*}

\begin{eqnarray*}
  P \approx_{K} Q
\end{eqnarray*}

\begin{eqnarray*}
  P \approx Q
\end{eqnarray*}

$\approx_{K} = \approx = \approx_{L}$

\subsubsection{Contextual duality}

Note that contexts extend the quotation operation to a family of
operations from processes to names. Given a context, $M$, we can
define a \emph{nominal context}, $\quotep{M}$ by $\quotep{M}[P] :=
\quotep{M[P]}$. To foreshadow what is to come we observe that these
operations enjoy a duality with processes very much like the duality
between vectors and maps from vectors to scalars.

Further, because the calculus is essentially higher-order, we have a
correspondence between contexts and processes. More specifically,
given a name $x$ and a context $M$ we can construct $M^{*}_{x}$ such
that 

\begin{mathpar}
  M^{*}_{x} | \lift{x}{P} \red M[P]
\end{mathpar}

namely,

\begin{mathpar}
  M^{*}_{x} := x?(u).M[\dropn{u}]
\end{mathpar}

The dependence of $M^{*}_{x}$ on a name makes it an abstraction, 

\begin{mathpar}
  M^{*} := (x)x?(u).M[\dropn{u}]
\end{mathpar}

\subsection{Additional notation}

It will sometimes be convenient to denote the process a name
quotes. We already have the notation $x = \quotep{P}$, but it will be
convenient to introduce an alternate notation, $\procn{x}$, when we
want to emphasize the connection to the use of the name. Note that, by
virtue of name equivalence, $\quotep{\procn{x}} \nameeq x$; so, the
notation is consistent with previous definitions.

Further, because names have structure it is possible to effect
substitutions on the basis of that structure. This means we need to
upgrade our notation for substitutions, which we accomplish by
adapting comprehension notation. Thus,

\begin{mathpar}
  P\{ y / x : x \in S \}
\end{mathpar}

is interpreted to mean the process derived from P by replacing (in a
capture-avoiding manner) each occurrence of $x$ in $S$ by $y$. For example,

\begin{mathpar}
  P\{ \quotep{\procn{x}|\procn{x}} / x : x \in \freenames{P} \}
\end{mathpar}

will replace each (occurrence) of a free name $x$ in $P$ by
$\quotep{\procn{x}|\procn{x}}$.

Also, we will avail ourselves of the notation $x^{L}$ and $x^{R}$ to
denote injections of a name into disjoint copies of the name
space. There are numerous ways to accomplish this. One example can be
found in \cite{MeredithR05}. This notation overloads to vectors of
names: $\vec{x}^{\pi} := (x_{i}^{\pi} \; : \; 0 \leq i < |\vec{x}| )$ where $\pi \in \{L,R\}$.

We also use $P^{\Box} := P|\Box$.

In \cite{MeredithR05} an interpretation of the new operator is
given. It turns out that there are several possible interpretations
all enjoying the requisite algebraic properties of the operator (see
\cite{milner91polyadicpi}). We will therefore make liberal use of
$(\nu\; \vec{x})P$.

% subsection the_syntax_and_semantics_of_the_notation_system (end)   

\input{qm2pi.qmops} 

\input{qm2pi.sterngerlach} 

\input{qm2pi.metric} 

% section concurrent_process_calculi (end)

%\input{qm2pi.proofsketch}

% section proof sketch (end)

%\input{qm2pi.slviaknots} 

% section spatial logic via knots (end)

\input{qm2pi.conclusion}

% section conclusion (end)

%\input{qm2pi.dtcodes} 

% section wiring algorithm (end)

\input{qm2pi.ack} 

% section acknowledgments (end)

\newpage


\bibliographystyle{plain}   
\bibliography{../../biblios/main.bib}

\input{qm2pi.rhodetails}

\end{document}



% section proof sketch (end)

%\section{Unlikely characters: spatial logic for
  knots}\label{sub:characteristic_formulae} % (fold)

Associated to the mobile process calculi are a family of logics known
as the Hennessy-Milner logics. These logics typically enjoy a
semantics interpreting formulae as sets of processes that when
factored through the encoding outlined above allows an identification
of classes of knots with logical formulae. In the context of this
encoding the sub-family known as the spatial logics \cite{CairesC03}
\cite{CairesC04} \cite{Caires04} are of particular interest providing
several important features for expressing and reasoning about
properties (i.e. classes) of knots. We hint here at how this may be done.

%\begin{description}
%\item [structural connectives] 
\subsubsection{Structural connectives} The spatial logics enjoy
structural connectives corresponding, at the logical level, to the
parallel composition ($P | Q$) and new name ($(\nu \; x)P$)
connectives for processes. As illustrated in the examples below, these
connectives are extremely expressive given the shape of our encoding.
%\item [decideable satisfaction]

\subsubsection{Decideable satisfaction}
In \cite{Caires04} the satisfaction relation is shown to be decideable
for a rich class of processes. It further turns out that the image of
the our encoding is a proper subset of that class. This result
provides the basis for an algorithm by which to search for knots
enjoying a given property.
%\item [characteristic formulae]

\subsubsection{Characteristic formulae}
In the same paper \cite{Caires04} , Caires presents a means of calculating
characteristic formulae, selecting equivalence classes of processes
up to a pre--specified depth limit on the support set of names. Composed with our
encoding, this characteristic formula can be used to select
characteristic formulae for knots.
%\end{description}

\subsubsection{Spatial logic formulae}

The grammar below (segmented for comprehension) summarizes the syntax
of spatial logic formulae. We employ illustrative examples in the
sequel to provide an intuitive understanding of their meaning
referring the reader to \cite{Caires04} for a more detailed explication
of the semantics.

\begin{mathpar}
  \inferrule* [lab=boolean] {} {{A,B} \bc T \;|\; \neg A \;|\; A \wedge B \;|\; \eta = \eta'}
  \and
  \inferrule* [lab=spatial] {} {|\; \pzero \;|\; A | B \;|\; x \text{\textregistered} A \;|\; \forall x . A \;|\;  H x . A}
  \and
  \inferrule* [lab=behavioral] {} {|\; \alpha . A}
  \and 
  \inferrule* [lab=recursion] {} {|\; X(\vec{u}) \;|\; \mu X(\vec{u}) . A}
  \and
  \inferrule* [lab=action] {} {\alpha \bc \langle x?(\vec{y}) \rangle \;|\; \langle x!(\vec{y}) \rangle \;|\; \langle \tau \rangle}
  \and 
  \inferrule* [lab=name] {} {\eta \bc x \;|\; \tau}
\end{mathpar} 

% subsection characteristic_formulae (end)   	 

\subsection{Example formulae}\label{sub:example_formulae_} % (fold)

\subsubsection{Crossing as formula.}
% 
% \begin{align*}
%   \frac{d}{dx} \sin x &= \cos x 
%   & \frac{d}{dx} e^x &= e^x \\
%   \frac{d}{dx} \cos x &= - \sin x 
%   & \frac{d}{dx} \log x &= \frac{1}{x} \\
% \end{align*} 

\begin{align*}
 \mu C(x_{0},x_{1},y_{0},y_{1},u).&(\langle x_{0}?(z) \rangle(\langle u! \rangle\langle y_{1}!z \rangle C(x_{0},x_{1},y_{0},y_{1},u)) & \\
  & \wedge \langle y_{1}?(z) \rangle (\langle u! \rangle \langle x_{0}!z \rangle C(x_{0},x_{1},y_{0},y_{1},u)) & \\
  & \wedge \langle x_{1}?(z) \rangle (\langle u? \rangle \langle y_{0}!z \rangle C(x_{0},x_{1},y_{0},y_{1},u)) & \\
  & \wedge \langle y_{0}?(z) \rangle (\langle u? \rangle \langle x_{1}!z \rangle C(x_{0},x_{1},y_{0},y_{1},u))) &
\end{align*}

The lexicographical similarity between the shape of this formulae and
the shape of definition of the process representing a crossing reveals
the intuitive meaning of this formulae. It describes the capabilities
of a process that has the right to represent a crossing. For example
it picks out processes that may perform an input on the port $x_0$ in
its initial menu of capabilities. What differentiates the formula
from the process, however, is that the crossing process is the
smallest candidate to satisfy the formula. Infinitely many other
processes -- with internal behavior hidden behind this interface, so
to speak -- also satisfy this formula. Even this simple formula,
then, can be seen to open a new view onto knots, providing a
computational interpretation of \emph{virtual} knots.

Note that this formula is derived by hand. A similar formula can be
derived by employing Caires' calculation of characteristic formula
\cite{Caires04} to the process representing a crossing. In light of
this discussion, we let
$\meaningof{C}_{\phi}(x0,x1,y0,y1,u)$ denote a formula specifying the
dynamics we wish to capture of a crossing. To guarantee we preserve
the shape of the interface and minimal semantics we demand that
$\meaningof{C}_{\phi}(x0,x1,y0,y1,u) \Rightarrow
\textbf{C}(x0,x1,y0,y1,u)$ where $\textbf{C}(x0,x1,y0,y1,u)$ denotes
the formula above.
                            
\subsubsection{Crossing number constraints.}
The moral content of the context lemma (Lemma \ref{context}) is that the notion of
``locality'' in the Reidemeister moves is effectively captured by the
parallel composition operator of the process calculus. This intuition
extends through the logic. Given a formula,
$\meaningof{C}_{\phi}(x0,x1,y0,y1,u)$, we can use the structural
connectives to specify constraints on crossing numbers, such as at
least $n$ crossings, or exactly $n$ crossings.
\begin{mathpar}
  \inferrule* [lab=at-least-n] {} { K^{\geq n}_{\phi}(\vec{xs},\vec{ys}) := \Pi_{i=0}^{n-1} Hu . \meaningof{C}_{\phi}(xs_i,ys_i,u) | T }
  \and 
  \inferrule* [lab=exactly-n] {} { K^{= n}_{\phi}(\vec{xs},\vec{ys}) := \Pi_{i=0}^{n-1} Hu . \meaningof{C}_{\phi}(xs_i,ys_i,u) | \neg (\forall x_0,y_0,x_1,y_1,u . \meaningof{C}_{\phi}(x_0,y_0,x_1,y_1,u) | T) }
\end{mathpar}

To round out this section, recall that the encoding of an $n$-crossing
knot decomposes into a parallel composition of $n$ \emph{copies} of a
crossing process together with a wiring harness. To specify different
knot classes with the same crossing number amounts to specifying
logical constraints on the wiring harness. In the interest of space,
we defer examples to a forthcoming paper. Suffice it to say that both
the conditions ``alternating knot'' and ``contains the tangle
corresponding to 5/3'' are expressible. For example, it is possible to
calculate the characteristic formula of a process corresponding to the
tangle 5/3 and conjoin it into the classifying formula via the
composition connective of the logic.

Finally, we wish to observe that it is entirely within reason to
contemplate a more domain-specific version of spatial logic tailored
to the shape of processes in the image of the encoding. Such a
domain-specific logic would have a better claim to the title formal
language of knot properties.

% subsection example_formulae_ (end)

% section knots_as_processes (end) 

% section spatial logic via knots (end)

\section{Conclusions and future work}

\paragraph{Testing physical space}
You, gentle reader, may wonder why of all the theorems to be proved
given this set up we pick the one above. In some sense it's hardly
central to quantum mechanics. We see it as central in the sense that
it firmly establishes a notion of physical space arising from a notion
of the equivalence of behavior. Relating bisimulation to a metric is a
big step forward, but one is faced with interpreting the relationship
of that metric space to something more physical. Quantum mechanical
notions of ``physical'' space are still far from intuitive, but by
relating this idea of distance as testing to calculations that predict
physical circumstances we are making a not insignificant step forward
toward an understanding of the physical space we inhabit as
essentially dynamic.

\paragraph{Effectivity and simulation}
One of the observations we have yet to make is that the entire program
spelled out here is effective. We have built various interpreters for
the reflective calculus at work in this interpretation. In principle,
then, we can simulate quantum mechanics on a computer. The place where
the simulation may lose fidelity is the infinitely branching summation
for the annihilator.

In this connection i also want to point out that the evaluation style
calculation of the inner product puts the non-determinism of the
summation right at the heart of measurement. This suggests that
Milner's original reduction-based formulation of the dynamics of his
calculi in terms of sums was not just notationally suggestive of a
notion of measure-and-continue but captured some significant part of
the physics.

\paragraph{Quantum continuations}
In light of this last observation i want to point out that the
predominant account of quantum mechanics is missing a key aspect of a
truly compositional story of the physical situation. In a real lab,
when a measurement is made the observation can be made to feed into
another device that then makes another measurement conditioned on the
results of the first. This means that after the superposition was
collapsed the entire experimental set up remained in
superposition. While QM offers a means of writing this down it doesn't
quite line up well with the well-trodden formulation of computation
and continuation that we see so succinctly expressed in Milner's
calculi. This suggests that there might be advantages to this account
of dynamics waiting to be explored.

\paragraph{Quantum logic}
In this connection, we also note that by virtue of having the
Hennessy-Milner construction, we can pull the construction through the
interpretation of QM. This gives us a natural candidate for a quantum
logic that enjoys an extremely tight connection with it's domain of
interpretation, making the construction much less ad hoc (rather it is
the image of functor!).

\paragraph{Quantum probabiity}
i have questions about the basis of the interpretation of inner
product as probability amplitude. In particular, using which
axiomatization of probability theory does the notion of probability
amplitude earn the right to be so dubbed? In other words, where is the
proof that the operation for calculating a probability amplitude (and
then squaring) satisfies the axioms of what it means to calculate a
probability? Even if such a proof exists (i have yet to find it in the
literature), i wonder if it might not be possible to turn things on
their heads. Can we view the calculation of the probability amplitude
as an axiomatization of probability? If so, then the definition we
give for calculating probability amplitude may provide the basis for
an \emph{effective} theory of probability.

\paragraph{Quantum vs ``biological'' information}
Finally, i want to conclude with a more philosophical observation. At
a recent workshop in which QM was a predominant topic i noticed
something about quantum information. The speaker was giving a riveting
discussion of axiomatic QM and showing how properties of ``no
cloning'' and ``no deleting'' emerged as consequences of the
axiomatization. Theorems of this form are necessary to give us a sense
of confidence that our axioms characterize the physical theory. What
struck me, though, was that if quantum information is neither erasable
nor replicable it is markedly different from \emph{life}. Two of the
things we know about life is that

\begin{itemize}
  \item it ends;
  \item to gain some measure of persistence, to transcend it's
    finitude it is imminently copyable.
\end{itemize}

Both of these qualities are summarized succinctly in the aphorism: all
flesh is grass. For me these two kinds of ``information'' -- call them
quantum and biological -- are end points on a spectrum of strategies
for persistence. At one end, we have those curious entities that enjoy
uniqueness and permanence; at the other, we have those who in the face
of a certain end and an uncertain present make a go of passing
something on. To me one of the more remarkable aspects of the latter
strategy is that in the presence of noise (and certain features of
copying) we get a kind of dynamism, a chance for improvement against a
given persistent condition.

% subsection other_calculi_other_bisimulations_and_geometry_as_behavior (end)




% section conclusion (end)

%\documentclass[12pt]{llncs}
%\documentclass{jktr}

\usepackage[pdftex]{hyperref}                   
\usepackage {listings}
\usepackage {mathpartir}
\usepackage{bcprules}
%\usepackage{listings}
                       
\usepackage{graphicx} 
%\usepackage[margins=2.5cm,nohead,nofoot]{geometry}
%\usepackage{geometry}
\usepackage{amsfonts}
\usepackage{amstext}
\usepackage{latexsym}
\usepackage{amssymb}
\usepackage{color}


%\include{myPreamble}
\include{qm2pi.local} 

%\ifpdf
%\usepackage[pdftex]{graphicx}
%\else
%\usepackage{graphicx}
%\fi

 % \ifpdf
%  \usepackage{pdfsync}
%  \if


%\title{Brief Article}
%\author{David F. Snyder}
%\author{L.G. Meredith}

%\address{Dept. of Math., Texas State University--San Marcos, San Marcos, TX 78666}
       
\pagestyle{empty}


\begin{document}

\lstset{language=[Objective]Caml,frame=shadowbox}

\input{qm2pi.front}

% section front matter (end)

\input{qm2pi.intro} 
 
% section introduction (end)

% \input{qm2pi.knotations} 

% section notation (end)

\input{qm2pi.process.calculi} 

% section concurrent_process_calculi_and_spatial_logics_ (end)
    
%\input{qm2pi.knots2pi} 

%\input{qm2pi.trefoil} 

%\input{qm2pi.mainthm} 

% subsection basic_interpretation (end)

%\input{qm2pi.rho.presentation} 
\subsection{The syntax and semantics of the notation system}\label{sub:the_syntax_and_semantics_of_the_notation_system} % (fold)

We now summarize a technical presentation of the calculus that
embodies our theory of dynamics. The typical presentation of such a
calculus follows the style of giving generators and relations on
them. The grammar, below, describing term constructors, freely
generates the set of processes, $\Proc$. This set is then quotiented
by a relation known as structural congruence and it is over this set
that the notion of dynamics is expressed. This presentation is
essentially that of \cite{MeredithR05} with the addition of
polyadicity and summation. For readability we have relegated some of
the technical subtleties to an appendix.

\subsubsection{Process grammar}\label{subsub:process_grammar}

\begin{mathpar}
  \inferrule* [lab=synchronization] {} {{M} \bc \pzero \;|\; x?F \;|\; x!C }
  \and
  \inferrule* [lab=abstraction] {} {{F} \bc (x)P}
  \and
  \inferrule* [lab=concretion] {} {{C} \bc \langle Q \rangle}
  \and
  \inferrule* [lab=process] {} {{P,Q} \bc M \;| \;P|Q \;|\; @{x}}
  \and
  \inferrule* [lab=name] {} {{x} \bc \quotep{P}}
\end{mathpar} 

Note that $\vec{x}$ (resp. $\vec{P}$) denotes a vector of names
(resp. processes) of length $|\vec{x}|$ (resp. $|\vec{P}|$). We adopt
the following useful abbreviations.

\begin{mathpar}
   x?(\vec{y}).P := x.(\vec{y})P \and  x\clift{\vec{P}} := x.\clift{\vec{P}}
   \and x!(y) := \lift{x}{\dropn{y}}
   \and \Pi_{i=0}^{n-1}P_i := P_0 | \ldots | P_{n-1}
\end{mathpar}

\subsubsection{Structural congruence}

\paragraph{Free and bound names and alpha-equivalence.} At the
core of structural equivalence is alpha-equivalence which identifies
process that are the same up to a change of variable. Formally, we
recognize the distinction between free and bound names. The free names
of a process, $\freenames{P}$, may be calculated recursively as
follows:

\begin{mathpar}
\freenames{\pzero} := \emptyset
  \and \\
  \freenames{x?(y).P} := \{ x \} \cup (\freenames{P} \setminus \{ y \})
  \and 
  \freenames{x!\langle P \rangle} := \{ x \} \cup \{ P \} 
  \and \\
  \freenames{P|Q} := \freenames{P} \cup \freenames{Q}
  \and \\
  \freenames{@{x}} := \{ x \}
\end{mathpar}

$\pi$
$\quotep{\pi}$

$\freenames{-} : \pi \to \mathcal{P}(\quotep{\pi})$

\begin{eqnarray*}
  \freenames{\pzero} & := & \emptyset \\
  \freenames{x?(y).P} & := & \{ x \} \cup (\freenames{P} \setminus \{ y \}) \\
  \freenames{x!\langle P \rangle} & := & \{ x \} \cup \{ P \} \\
  \freenames{P|Q} & := & \freenames{P} \cup \freenames{Q} \\
  \freenames{\dropn{x}} & := & \{ x \}
\end{eqnarray*}

The bound names of a process, $\boundnames{P}$, are those names occurring in $P$
that are not free. For example, in $x?(y).0$, the name $x$ is free, while $y$ is bound.

\begin{mathpar}
  \inferrule* [lab=monoidal-laws] {} { P|Q \equiv Q|P \and P|0 \equiv P \and P|(Q|R) \equiv (P|Q)|R }
\end{mathpar}

\begin{mathpar}
  \inferrule* [lab=alpha-equivalence] {} { (x)P \equiv (y)P\{y/x\} \and y \not\in \freenames{P} }
\end{mathpar}

\begin{definition}
Then two processes, $P,Q$, are alpha-equivalent if $P = Q\{\vec{y}/\vec{x}\}$ for
some $\vec{x} \in \boundnames{Q},\vec{y} \in \boundnames{P}$, where $Q\{\vec{y}/\vec{x}\}$
denotes the capture-avoiding substitution of $\vec{y}$ for $\vec{x}$ in $Q$.
\end{definition}

\begin{definition}
  The {\em structural congruence} \cite{SangiorgiWalker} , $\equiv$,
  between processes is the least congruence containing
  alpha-equivalence, satisfying the abelian monoid laws
  (associativity, commutativity and $\pzero$ as identity) for parallel
  composition $|$ and for summation $+$.
\end{definition}

\subsection{Name equivalence}

We take name equivalence, written $\nameeq$, to be the smallest
equivalence relation generated by the following rules.

\begin{mathpar}
\inferrule*[lab=Quote-drop]
{ }
{ \quotep{@{x}} \nameeq x }

\inferrule*[lab=Struct-equiv]
{ P \scong Q }
{ \quotep{P} \nameeq \quotep{Q} }
\end{mathpar}

The astute reader will have noticed that the mutual recursion of names
and processes imposes a mutual recursion on alpha-equivalence and
structural equivalence via name-equivalence. Fortunately, all of this
works out pleasantly and we may calculate in the natural way, free of
concern. The reader interested in the details is referred to the
appendix \ref{appendix:rho_details}.

\subsection{Substitution}

We use $\Proc$ for the set of processes, $\QProc$ for the set of
names, and $\id{\{}\vec{y} / \vec{x} \id{\}}$ to denote partial maps,
$s : \QProc \rightarrow \QProc$. A map, $s$ lifts, uniquely, to a map
on process terms, $\widehat{s} : \Proc \rightarrow \Proc$ by the
following equations.

\begin{mathpar}
  (0) \psubstp{Q}{P} := 0 \\
  (R \juxtap S) \psubstp{Q}{P}
  :=    
  (R)\psubstp{Q}{P} \juxtap (S) \psubstp{Q}{P} \\
  (x?(y).R) \psubstp{Q}{P}    
  :=    
  (x)\substp{Q}{P} (z)\concat( (R \psubstn{z}{y}) \psubstp{Q}{P} ) \\
  (\lift{x}{R}) \psubstp{Q}{P}  
  :=
  \lift{(x)\substp{Q}{P}}{ R \psubstp{Q}{P} } \\
%   (\dropn{x})  \psubstp{Q}{P}       
%   := 
%   \left\{ 
%     \begin{array}{ccc} 
%       \dropn{\quotep{Q}} & & x \nameeq \quotep{P} \\
%       \dropn{x} & & otherwise \\
%     \end{array}
%   \right. 
  (\dropn{x})  \psubstp{Q}{P}       
  := 
  \left\{ 
    \begin{array}{ccc} 
      Q & & x \nameeq \quotep{P} \\
      \dropn{x} & & otherwise \\
    \end{array}
  \right.
\end{mathpar}
 

where

\begin{eqnarray}
  (x)\id{\{} \lpquote Q \rpquote / \lpquote P \rpquote \id{\}}            = 
  \left\{ 
    \begin{array}{ccc}
      \lpquote Q \rpquote & & x \nameeq \lpquote P \rpquote \\
      x & & otherwise \\
    \end{array}
  \right. \nonumber
\end{eqnarray}

and $z$ is chosen distinct from $\quotep{P}$, $\quotep{Q}$, the free
names in $Q$, and all the names in $R$. Our $\alpha$-equivalence will
be built in the standard way from this substitution.

\begin{remark}\label{rem:no_self_referential_names}
  One consequence of these definitions is that $\forall P. \quotep{P}
  \not\in \freenames{P}$.
\end{remark}

\subsection{ Dynamic quote: an example }

Anticipating something of what's to come, consider applying the
substitution, $\widehat{\id{\{}u / z \id{\}}}$, to the following pair
of processes, $\lift{w}{y!(z)}$ and $w[ \lpquote y!(z) \rpquote ]$.

\begin{eqnarray}
	\lift{w}{y!(z)}\widehat{\id{\{}u / z \id{\}}}
		& = &
		\lift{w}{y!(u)} \nonumber\\
	w[ \lpquote y!(z) \rpquote ] \widehat{ \id{\{}u / z \id{\}} }
		& = &
		w[ \lpquote y!(z) \rpquote ] \nonumber
\end{eqnarray}

Because the body of the process between quotes is impervious to
substitution, we get radically different answers. In fact, by
examining the first process in an input context,
e.g. $x?(z).\lift{w}{y!(z)}$, we see that the process under the lift
operator may be shaped by prefixed inputs binding a name inside it. In
this sense, the lift operator will be seen as a way to dynamically
construct processes before reifying them as names.

Finally equipped with these standard features we can present the
dynamics of the calculus.

\subsubsection{Operational semantics} 

Finally, we introduce the computational dynamics. What marks these
algebras as distinct from other more traditionally studied algebraic
structures, e.g. vector spaces or polynomial rings, is the manner in
which dynamics is captured. In traditional structures, dynamics is typically
expressed through morphisms between such structures, as in linear maps
between vector spaces or morphisms between rings. In algebras
associated with the semantics of computation, the dynamics is
expressed as part of the algebraic structure itself, through a
reduction reduction relation typically denoted by $\red$. Below, we
give a recursive presentation of this relation for the calculus used
in the encoding.

$\red \subseteq \pi \times \pi$
$\red : \pi \to \mathcal{P}(\pi)$

\begin{mathpar}
  \inferrule* [lab=Comm] { \textsf{match}( x_{src}, x_{trgt} ) } { x_{trgt}?(y)P \; | \; x_{src}!\langle {Q} \rangle \red P\{\quotep{Q}/y}\} }
  \and \\
  \inferrule* [lab=Par] {{P} \red {P}'} {{{P} | {Q}} \red {{P}' | {Q}}}
  \and
  \inferrule* [lab=Equiv]{{{P} \scong {P}'} \andalso {{P}' \red {Q}'} \andalso {{Q}' \scong {Q}}}{{P} \red {Q}}
\end{mathpar}

\begin{eqnarray*}
  match_{\equiv} (\quotep{P},\quotep{Q}) & := & P \equiv Q \\
  match_{\dagger}(\quotep{P},\quotep{Q}) & := & \forall R. P|Q \red^{*} R => R \red^{*} 0 \\
  match_{K}(\quotep{P},\quotep{Q}) & := & K \mbox{ for some context } K
\end{eqnarray*}

$u?(x)P | u!\langle Q \rangle \red P\{\quotep{Q}/x\}$

%We write $\wred$ for $\red^*$, and $P\red$ if $\exists Q $ such that $ P \red Q$.
We write $P\red$ if $\exists Q $ such that $ P \red Q$ and $P\not\red$, otherwise.

\section{Replication}

As mentioned before, it is known that replication (and hence
recursion) can be implemented in a higher-order process algebra
\cite{SangiorgiWalker}. As our first example of calculation with the
machinery thus far presented we give the construction explicitly in
the {\rhoc}.

\begin{eqnarray}
	D_{x} & := & \prefix{x}{y}{(\binpar{\outputp{x}{y}}{@{y}})} \nonumber\\
	\bangp_{x}{P} & := & \binpar{{x}!\langle{\binpar{D_{x}}{P}}\rangle}{D_{x}} \nonumber
\end{eqnarray}

\begin{eqnarray}
	\bangp_{x}{P} & & \nonumber\\
	=
	& {x}!\langle{(\prefix{x}{y}{(\outputp{x}{y} | @{y})) | P}}\rangle 
	      | \prefix{x}{y}{(\outputp{x}{y} | @{y})} & \nonumber\\
	\red
	& (\outputp{x}{y} | @{y})\substn{\quotep{(\prefix{x}{y}{(@{y} | \outputp{x}{y})) | P}}}{y} & \nonumber\\
	=
	& \outputp{x}{\quotep{(\prefix{x}{y}{(\outputp{x}{y} | @{y})) | P}}}
	  | {(\prefix{x}{y}{(\outputp{x}{y} | @{y})) | P}} & \nonumber\\
	\red
	& \ldots & \nonumber\\
	\red^*
	& P | P | \ldots & \nonumber
\end{eqnarray}

Of course, this encoding, as an implementation, runs away, unfolding
$\bangp{P}$ eagerly. A lazier and more implementable replication
operator, restricted to input-guarded processes, may be obtained as follows.

\begin{eqnarray}
\bangp{\prefix{u}{v}{P}} 
	:= 
	\binpar{\lift{x}{\prefix{u}{v}{(\binpar{D(x)}{P})}}}{D(x)} \nonumber
\end{eqnarray}

\begin{remark}
  Note that the lazier definition still does not deal with summation
  or mixed summation (i.e. sums over input and output). The reader is
  invited to construct definitions of replication that deal with these
  features. 

  Further, the definitions are parameterized in a name, $x$. Can you,
  gentle reader, make a definition that eliminates this parameter and
  guarantees no accidental interaction between the replication
  machinery and the process being replicated -- i.e. no accidental
  sharing of names used by the process to get its work done and the
  name(s) used by the replication to effect copying. This latter
  revision of the definition of replication is crucial to obtaining
  the expected identity $!!P \sim !P$.
\end{remark}

\begin{remark}\label{rem:paradoxical_combinator}
  The reader familiar with the lambda calculus will have noticed the
  similarity between $D$ and the paradoxical combinator.

  [Ed. note: the existence of this seems to suggest we have to be more
  restrictive on the set of processes and names we admit if we are to
  support no-cloning.]
\end{remark}

\subsubsection{Bisimulation}

The computational dynamics gives rise to another kind of equivalence,
the equivalence of computational behavior. As previously mentioned
this is typically captured \emph{via} some form of bisimulation.

% The notion we use in this paper is weak barbed bisimulation
% \cite{milner91polyadicpi}.

The notion we use in this paper is derived from weak barbed
bisimulation \cite{milner91polyadicpi}. 

\begin{definition}
An \emph{observation relation}, $\downarrow_{\mathcal N}$, over a set
of names, $\mathcal N$, is the smallest relation satisfying the rules
below.

\infrule[Out-barb]{y \in {\mathcal N}, \; x \nameeq y}
		  {\outputp{x}{v} \downarrow_{\mathcal N} x}
\infrule[Par-barb]{\mbox{$P\downarrow_{\mathcal N} x$ or $Q\downarrow_{\mathcal N} x$}}
		  {\binpar{P}{Q} \downarrow_{\mathcal N} x}

We write $P \Downarrow_{\mathcal N} x$ if there is $Q$ such that 
$P \wred Q$ and $Q \downarrow_{\mathcal N} x$.
\end{definition}

\begin{definition}
%\label{def.bbisim}
An  ${\mathcal N}$-\emph{barbed bisimulation} over a set of names, ${\mathcal N}$, is a symmetric binary relation 
${\mathcal S}_{\mathcal N}$ between agents such that $P\rel{S}_{\mathcal N}Q$ implies:
\begin{enumerate}
\item If $P \red P'$ then $Q \wred Q'$ and $P'\rel{S}_{\mathcal N} Q'$.
\item If $P\downarrow_{\mathcal N} x$, then $Q\Downarrow_{\mathcal N} x$.
\end{enumerate}
$P$ is ${\mathcal N}$-barbed bisimilar to $Q$, written
$P \wbbisim_{\mathcal N} Q$, if $P \rel{S}_{\mathcal N} Q$ for some ${\mathcal N}$-barbed bisimulation ${\mathcal S}_{\mathcal N}$.
\end{definition}

$\mathcal{R} \subseteq \pi \times \pi$

$P \mathcal{R} Q => \forall P'. P \red P' \Rightarrow \exists Q'. Q \red Q', P' \mathcal{R} Q'$

$P \vdash x \Rightarrow Q \vdash x$

\begin{mathpar}
  \inferrule*[lab=Out-barb]{x \nameeq y}{{y}!\langle{Q}\rangle \vdash x}
  \and
  \inferrule*[lab=Par-barb]{\mbox{$P\vdash x$ or $Q\vdash x$}}{\binpar{P}{Q} \vdash x}
\end{mathpar}

\subsubsection{Contexts}

One of the principle advantages of computational calculi like the
$\pi$-calculus is a well-defined notion of context,
contextual-equivalence and a correlation between
contextual-equivalence and notions of bisimulation. The notion of
context allows the decomposition of a process into (sub-)process and
its syntactic environment, its context. Thus, a context may be
thought of as a process with a ``hole'' (written $\Box$) in it. The
application of a context $M$ to a process $P$, written $M[P]$, is
tantamount to filling the hole in $M$ with $P$. In this paper we do
not need the full weight of this theory, but do make use of the notion
of context in the proof the main theorem. 

\begin{mathpar}
  \inferrule* [lab=summation] {} {{M_{M},M_{N}} \bc \Box \;|\; x.M_{A} \;|\; M_{M}+M_{N}}
  \and
  \inferrule* [lab=agent] {} {{M_{A}} \bc (\vec{x})M_{P} \;| \; \clift{P_0,\ldots,M_{P},\ldots,P_N}}
  \and \\
  \inferrule* [lab=process] {} {{M_{P}} \bc M_{N} \;| \;P|M_{P} }
\end{mathpar} 

\begin{mathpar}
  \inferrule* [lab=sychronization] {} {M_{N} \bc \Box \;|\; x?M_{F} \;|\; x!M_{C}}
  \and
  \inferrule* [lab=abstraction] {} {{M_{F}} \bc (x)M_{P} }
  \and
  \inferrule* [lab=concretion] {} {{M_{C}} \bc \langle M_{P} \rangle }
  \and \\
  \inferrule* [lab=process] {} {{M_{P}} \bc M_{N} \;| \;P|M_{P} }
\end{mathpar}

\begin{definition}[contextual application] Given a context $M$, and
  process $P$, we define the \emph{contextual application}, $M[P] :=
  M\{P/\Box\}$. That is, the contextual application of M to P is the
  substitution of $P$ for $\Box$ in $M$.
\end{definition}

$\meaningof{-} : L \to \mathcal{P}(\pi)$

\begin{mathpar}
  \inferrule* [lab=collection] {} {\meaningof{true} = \pi, \and \meaningof{~E} = \pi \setminus \meaningof{E}, \and \meaningof{E_{1} \& E_{2}} = \meaningof{E_{1}} \cap \meaningof{E_{2}}}
\end{mathpar}

\begin{mathpar}
  \inferrule* [lab=structure] {} {\meaningof{0} = \{ P \in \pi | P \equiv 0 \}, \and \\ \meaningof{E_1 | E_2} = \{ P \in \pi | P \equiv P_{1} | P_{2}, P_{1} \in \meaningof{E_{1}}, P_{2} \in \meaningof{E_2}\} }
\end{mathpar}

\begin{mathpar}
 \inferrule* [lab=behavior] {} {\meaningof{\langle a?b \rangle E} = \{ P \in \pi | P \equiv Q | u?(y)P', \\ \and \\\\ \and \\ \;\;\; u \in \meaningof{a}, \forall z.P'\{z/y\} \in \meaningof{E\{z/b\}}\}, \and \\ \meaningof{a!E} = \{ P \in \pi | P \equiv Q | x!\langle P' \rangle, x \in \meaningof{a} P' \in \meaningof{E}\} }
\end{mathpar}

\begin{mathpar}
 \inferrule* [lab=nominal] {} {\meaningof{\quotep{E}} = \{ \quotep{P} \in \quotep{\pi} | P \in \meaningof{E} \}, \and \meaningof{\quotep{P}} = \{ \quotep{Q} \in \quotep{\pi} | P \equiv Q \} \and \\ \meaningof{@\quotep{E}} = \{ P \in \pi | P \equiv @x, x \in \meaningof{E} \}}
\end{mathpar}

\begin{eqnarray*}
  \\
  \meaningof{-} : TS \to ST
\end{eqnarray*}

\begin{eqnarray*}
  \\
  L : TS \to ST
\end{eqnarray*}

\begin{eqnarray*}
  \\
  P \models E \iff P \in \meaningof{E}
\end{eqnarray*}

\begin{eqnarray*}
  P \approx_{L} Q \iff \forall E \in L. P \models E \iff Q \models E
\end{eqnarray*}

\begin{eqnarray*}
  P \approx_{K} Q
\end{eqnarray*}

\begin{eqnarray*}
  P \approx Q
\end{eqnarray*}

$\approx_{K} = \approx = \approx_{L}$

\subsubsection{Contextual duality}

Note that contexts extend the quotation operation to a family of
operations from processes to names. Given a context, $M$, we can
define a \emph{nominal context}, $\quotep{M}$ by $\quotep{M}[P] :=
\quotep{M[P]}$. To foreshadow what is to come we observe that these
operations enjoy a duality with processes very much like the duality
between vectors and maps from vectors to scalars.

Further, because the calculus is essentially higher-order, we have a
correspondence between contexts and processes. More specifically,
given a name $x$ and a context $M$ we can construct $M^{*}_{x}$ such
that 

\begin{mathpar}
  M^{*}_{x} | \lift{x}{P} \red M[P]
\end{mathpar}

namely,

\begin{mathpar}
  M^{*}_{x} := x?(u).M[\dropn{u}]
\end{mathpar}

The dependence of $M^{*}_{x}$ on a name makes it an abstraction, 

\begin{mathpar}
  M^{*} := (x)x?(u).M[\dropn{u}]
\end{mathpar}

\subsection{Additional notation}

It will sometimes be convenient to denote the process a name
quotes. We already have the notation $x = \quotep{P}$, but it will be
convenient to introduce an alternate notation, $\procn{x}$, when we
want to emphasize the connection to the use of the name. Note that, by
virtue of name equivalence, $\quotep{\procn{x}} \nameeq x$; so, the
notation is consistent with previous definitions.

Further, because names have structure it is possible to effect
substitutions on the basis of that structure. This means we need to
upgrade our notation for substitutions, which we accomplish by
adapting comprehension notation. Thus,

\begin{mathpar}
  P\{ y / x : x \in S \}
\end{mathpar}

is interpreted to mean the process derived from P by replacing (in a
capture-avoiding manner) each occurrence of $x$ in $S$ by $y$. For example,

\begin{mathpar}
  P\{ \quotep{\procn{x}|\procn{x}} / x : x \in \freenames{P} \}
\end{mathpar}

will replace each (occurrence) of a free name $x$ in $P$ by
$\quotep{\procn{x}|\procn{x}}$.

Also, we will avail ourselves of the notation $x^{L}$ and $x^{R}$ to
denote injections of a name into disjoint copies of the name
space. There are numerous ways to accomplish this. One example can be
found in \cite{MeredithR05}. This notation overloads to vectors of
names: $\vec{x}^{\pi} := (x_{i}^{\pi} \; : \; 0 \leq i < |\vec{x}| )$ where $\pi \in \{L,R\}$.

We also use $P^{\Box} := P|\Box$.

In \cite{MeredithR05} an interpretation of the new operator is
given. It turns out that there are several possible interpretations
all enjoying the requisite algebraic properties of the operator (see
\cite{milner91polyadicpi}). We will therefore make liberal use of
$(\nu\; \vec{x})P$.

% subsection the_syntax_and_semantics_of_the_notation_system (end)   

\input{qm2pi.qmops} 

\input{qm2pi.sterngerlach} 

\input{qm2pi.metric} 

% section concurrent_process_calculi (end)

%\input{qm2pi.proofsketch}

% section proof sketch (end)

%\input{qm2pi.slviaknots} 

% section spatial logic via knots (end)

\input{qm2pi.conclusion}

% section conclusion (end)

%\input{qm2pi.dtcodes} 

% section wiring algorithm (end)

\input{qm2pi.ack} 

% section acknowledgments (end)

\newpage


\bibliographystyle{plain}   
\bibliography{../../biblios/main.bib}

\input{qm2pi.rhodetails}

\end{document}

 

% section wiring algorithm (end)

\documentclass[12pt]{llncs}
%\documentclass{jktr}

\usepackage[pdftex]{hyperref}                   
\usepackage {listings}
\usepackage {mathpartir}
\usepackage{bcprules}
%\usepackage{listings}
                       
\usepackage{graphicx} 
%\usepackage[margins=2.5cm,nohead,nofoot]{geometry}
%\usepackage{geometry}
\usepackage{amsfonts}
\usepackage{amstext}
\usepackage{latexsym}
\usepackage{amssymb}
\usepackage{color}


%\include{myPreamble}
\include{qm2pi.local} 

%\ifpdf
%\usepackage[pdftex]{graphicx}
%\else
%\usepackage{graphicx}
%\fi

 % \ifpdf
%  \usepackage{pdfsync}
%  \if


%\title{Brief Article}
%\author{David F. Snyder}
%\author{L.G. Meredith}

%\address{Dept. of Math., Texas State University--San Marcos, San Marcos, TX 78666}
       
\pagestyle{empty}


\begin{document}

\lstset{language=[Objective]Caml,frame=shadowbox}

\input{qm2pi.front}

% section front matter (end)

\input{qm2pi.intro} 
 
% section introduction (end)

% \input{qm2pi.knotations} 

% section notation (end)

\input{qm2pi.process.calculi} 

% section concurrent_process_calculi_and_spatial_logics_ (end)
    
%\input{qm2pi.knots2pi} 

%\input{qm2pi.trefoil} 

%\input{qm2pi.mainthm} 

% subsection basic_interpretation (end)

%\input{qm2pi.rho.presentation} 
\subsection{The syntax and semantics of the notation system}\label{sub:the_syntax_and_semantics_of_the_notation_system} % (fold)

We now summarize a technical presentation of the calculus that
embodies our theory of dynamics. The typical presentation of such a
calculus follows the style of giving generators and relations on
them. The grammar, below, describing term constructors, freely
generates the set of processes, $\Proc$. This set is then quotiented
by a relation known as structural congruence and it is over this set
that the notion of dynamics is expressed. This presentation is
essentially that of \cite{MeredithR05} with the addition of
polyadicity and summation. For readability we have relegated some of
the technical subtleties to an appendix.

\subsubsection{Process grammar}\label{subsub:process_grammar}

\begin{mathpar}
  \inferrule* [lab=synchronization] {} {{M} \bc \pzero \;|\; x?F \;|\; x!C }
  \and
  \inferrule* [lab=abstraction] {} {{F} \bc (x)P}
  \and
  \inferrule* [lab=concretion] {} {{C} \bc \langle Q \rangle}
  \and
  \inferrule* [lab=process] {} {{P,Q} \bc M \;| \;P|Q \;|\; @{x}}
  \and
  \inferrule* [lab=name] {} {{x} \bc \quotep{P}}
\end{mathpar} 

Note that $\vec{x}$ (resp. $\vec{P}$) denotes a vector of names
(resp. processes) of length $|\vec{x}|$ (resp. $|\vec{P}|$). We adopt
the following useful abbreviations.

\begin{mathpar}
   x?(\vec{y}).P := x.(\vec{y})P \and  x\clift{\vec{P}} := x.\clift{\vec{P}}
   \and x!(y) := \lift{x}{\dropn{y}}
   \and \Pi_{i=0}^{n-1}P_i := P_0 | \ldots | P_{n-1}
\end{mathpar}

\subsubsection{Structural congruence}

\paragraph{Free and bound names and alpha-equivalence.} At the
core of structural equivalence is alpha-equivalence which identifies
process that are the same up to a change of variable. Formally, we
recognize the distinction between free and bound names. The free names
of a process, $\freenames{P}$, may be calculated recursively as
follows:

\begin{mathpar}
\freenames{\pzero} := \emptyset
  \and \\
  \freenames{x?(y).P} := \{ x \} \cup (\freenames{P} \setminus \{ y \})
  \and 
  \freenames{x!\langle P \rangle} := \{ x \} \cup \{ P \} 
  \and \\
  \freenames{P|Q} := \freenames{P} \cup \freenames{Q}
  \and \\
  \freenames{@{x}} := \{ x \}
\end{mathpar}

$\pi$
$\quotep{\pi}$

$\freenames{-} : \pi \to \mathcal{P}(\quotep{\pi})$

\begin{eqnarray*}
  \freenames{\pzero} & := & \emptyset \\
  \freenames{x?(y).P} & := & \{ x \} \cup (\freenames{P} \setminus \{ y \}) \\
  \freenames{x!\langle P \rangle} & := & \{ x \} \cup \{ P \} \\
  \freenames{P|Q} & := & \freenames{P} \cup \freenames{Q} \\
  \freenames{\dropn{x}} & := & \{ x \}
\end{eqnarray*}

The bound names of a process, $\boundnames{P}$, are those names occurring in $P$
that are not free. For example, in $x?(y).0$, the name $x$ is free, while $y$ is bound.

\begin{mathpar}
  \inferrule* [lab=monoidal-laws] {} { P|Q \equiv Q|P \and P|0 \equiv P \and P|(Q|R) \equiv (P|Q)|R }
\end{mathpar}

\begin{mathpar}
  \inferrule* [lab=alpha-equivalence] {} { (x)P \equiv (y)P\{y/x\} \and y \not\in \freenames{P} }
\end{mathpar}

\begin{definition}
Then two processes, $P,Q$, are alpha-equivalent if $P = Q\{\vec{y}/\vec{x}\}$ for
some $\vec{x} \in \boundnames{Q},\vec{y} \in \boundnames{P}$, where $Q\{\vec{y}/\vec{x}\}$
denotes the capture-avoiding substitution of $\vec{y}$ for $\vec{x}$ in $Q$.
\end{definition}

\begin{definition}
  The {\em structural congruence} \cite{SangiorgiWalker} , $\equiv$,
  between processes is the least congruence containing
  alpha-equivalence, satisfying the abelian monoid laws
  (associativity, commutativity and $\pzero$ as identity) for parallel
  composition $|$ and for summation $+$.
\end{definition}

\subsection{Name equivalence}

We take name equivalence, written $\nameeq$, to be the smallest
equivalence relation generated by the following rules.

\begin{mathpar}
\inferrule*[lab=Quote-drop]
{ }
{ \quotep{@{x}} \nameeq x }

\inferrule*[lab=Struct-equiv]
{ P \scong Q }
{ \quotep{P} \nameeq \quotep{Q} }
\end{mathpar}

The astute reader will have noticed that the mutual recursion of names
and processes imposes a mutual recursion on alpha-equivalence and
structural equivalence via name-equivalence. Fortunately, all of this
works out pleasantly and we may calculate in the natural way, free of
concern. The reader interested in the details is referred to the
appendix \ref{appendix:rho_details}.

\subsection{Substitution}

We use $\Proc$ for the set of processes, $\QProc$ for the set of
names, and $\id{\{}\vec{y} / \vec{x} \id{\}}$ to denote partial maps,
$s : \QProc \rightarrow \QProc$. A map, $s$ lifts, uniquely, to a map
on process terms, $\widehat{s} : \Proc \rightarrow \Proc$ by the
following equations.

\begin{mathpar}
  (0) \psubstp{Q}{P} := 0 \\
  (R \juxtap S) \psubstp{Q}{P}
  :=    
  (R)\psubstp{Q}{P} \juxtap (S) \psubstp{Q}{P} \\
  (x?(y).R) \psubstp{Q}{P}    
  :=    
  (x)\substp{Q}{P} (z)\concat( (R \psubstn{z}{y}) \psubstp{Q}{P} ) \\
  (\lift{x}{R}) \psubstp{Q}{P}  
  :=
  \lift{(x)\substp{Q}{P}}{ R \psubstp{Q}{P} } \\
%   (\dropn{x})  \psubstp{Q}{P}       
%   := 
%   \left\{ 
%     \begin{array}{ccc} 
%       \dropn{\quotep{Q}} & & x \nameeq \quotep{P} \\
%       \dropn{x} & & otherwise \\
%     \end{array}
%   \right. 
  (\dropn{x})  \psubstp{Q}{P}       
  := 
  \left\{ 
    \begin{array}{ccc} 
      Q & & x \nameeq \quotep{P} \\
      \dropn{x} & & otherwise \\
    \end{array}
  \right.
\end{mathpar}
 

where

\begin{eqnarray}
  (x)\id{\{} \lpquote Q \rpquote / \lpquote P \rpquote \id{\}}            = 
  \left\{ 
    \begin{array}{ccc}
      \lpquote Q \rpquote & & x \nameeq \lpquote P \rpquote \\
      x & & otherwise \\
    \end{array}
  \right. \nonumber
\end{eqnarray}

and $z$ is chosen distinct from $\quotep{P}$, $\quotep{Q}$, the free
names in $Q$, and all the names in $R$. Our $\alpha$-equivalence will
be built in the standard way from this substitution.

\begin{remark}\label{rem:no_self_referential_names}
  One consequence of these definitions is that $\forall P. \quotep{P}
  \not\in \freenames{P}$.
\end{remark}

\subsection{ Dynamic quote: an example }

Anticipating something of what's to come, consider applying the
substitution, $\widehat{\id{\{}u / z \id{\}}}$, to the following pair
of processes, $\lift{w}{y!(z)}$ and $w[ \lpquote y!(z) \rpquote ]$.

\begin{eqnarray}
	\lift{w}{y!(z)}\widehat{\id{\{}u / z \id{\}}}
		& = &
		\lift{w}{y!(u)} \nonumber\\
	w[ \lpquote y!(z) \rpquote ] \widehat{ \id{\{}u / z \id{\}} }
		& = &
		w[ \lpquote y!(z) \rpquote ] \nonumber
\end{eqnarray}

Because the body of the process between quotes is impervious to
substitution, we get radically different answers. In fact, by
examining the first process in an input context,
e.g. $x?(z).\lift{w}{y!(z)}$, we see that the process under the lift
operator may be shaped by prefixed inputs binding a name inside it. In
this sense, the lift operator will be seen as a way to dynamically
construct processes before reifying them as names.

Finally equipped with these standard features we can present the
dynamics of the calculus.

\subsubsection{Operational semantics} 

Finally, we introduce the computational dynamics. What marks these
algebras as distinct from other more traditionally studied algebraic
structures, e.g. vector spaces or polynomial rings, is the manner in
which dynamics is captured. In traditional structures, dynamics is typically
expressed through morphisms between such structures, as in linear maps
between vector spaces or morphisms between rings. In algebras
associated with the semantics of computation, the dynamics is
expressed as part of the algebraic structure itself, through a
reduction reduction relation typically denoted by $\red$. Below, we
give a recursive presentation of this relation for the calculus used
in the encoding.

$\red \subseteq \pi \times \pi$
$\red : \pi \to \mathcal{P}(\pi)$

\begin{mathpar}
  \inferrule* [lab=Comm] { \textsf{match}( x_{src}, x_{trgt} ) } { x_{trgt}?(y)P \; | \; x_{src}!\langle {Q} \rangle \red P\{\quotep{Q}/y}\} }
  \and \\
  \inferrule* [lab=Par] {{P} \red {P}'} {{{P} | {Q}} \red {{P}' | {Q}}}
  \and
  \inferrule* [lab=Equiv]{{{P} \scong {P}'} \andalso {{P}' \red {Q}'} \andalso {{Q}' \scong {Q}}}{{P} \red {Q}}
\end{mathpar}

\begin{eqnarray*}
  match_{\equiv} (\quotep{P},\quotep{Q}) & := & P \equiv Q \\
  match_{\dagger}(\quotep{P},\quotep{Q}) & := & \forall R. P|Q \red^{*} R => R \red^{*} 0 \\
  match_{K}(\quotep{P},\quotep{Q}) & := & K \mbox{ for some context } K
\end{eqnarray*}

$u?(x)P | u!\langle Q \rangle \red P\{\quotep{Q}/x\}$

%We write $\wred$ for $\red^*$, and $P\red$ if $\exists Q $ such that $ P \red Q$.
We write $P\red$ if $\exists Q $ such that $ P \red Q$ and $P\not\red$, otherwise.

\section{Replication}

As mentioned before, it is known that replication (and hence
recursion) can be implemented in a higher-order process algebra
\cite{SangiorgiWalker}. As our first example of calculation with the
machinery thus far presented we give the construction explicitly in
the {\rhoc}.

\begin{eqnarray}
	D_{x} & := & \prefix{x}{y}{(\binpar{\outputp{x}{y}}{@{y}})} \nonumber\\
	\bangp_{x}{P} & := & \binpar{{x}!\langle{\binpar{D_{x}}{P}}\rangle}{D_{x}} \nonumber
\end{eqnarray}

\begin{eqnarray}
	\bangp_{x}{P} & & \nonumber\\
	=
	& {x}!\langle{(\prefix{x}{y}{(\outputp{x}{y} | @{y})) | P}}\rangle 
	      | \prefix{x}{y}{(\outputp{x}{y} | @{y})} & \nonumber\\
	\red
	& (\outputp{x}{y} | @{y})\substn{\quotep{(\prefix{x}{y}{(@{y} | \outputp{x}{y})) | P}}}{y} & \nonumber\\
	=
	& \outputp{x}{\quotep{(\prefix{x}{y}{(\outputp{x}{y} | @{y})) | P}}}
	  | {(\prefix{x}{y}{(\outputp{x}{y} | @{y})) | P}} & \nonumber\\
	\red
	& \ldots & \nonumber\\
	\red^*
	& P | P | \ldots & \nonumber
\end{eqnarray}

Of course, this encoding, as an implementation, runs away, unfolding
$\bangp{P}$ eagerly. A lazier and more implementable replication
operator, restricted to input-guarded processes, may be obtained as follows.

\begin{eqnarray}
\bangp{\prefix{u}{v}{P}} 
	:= 
	\binpar{\lift{x}{\prefix{u}{v}{(\binpar{D(x)}{P})}}}{D(x)} \nonumber
\end{eqnarray}

\begin{remark}
  Note that the lazier definition still does not deal with summation
  or mixed summation (i.e. sums over input and output). The reader is
  invited to construct definitions of replication that deal with these
  features. 

  Further, the definitions are parameterized in a name, $x$. Can you,
  gentle reader, make a definition that eliminates this parameter and
  guarantees no accidental interaction between the replication
  machinery and the process being replicated -- i.e. no accidental
  sharing of names used by the process to get its work done and the
  name(s) used by the replication to effect copying. This latter
  revision of the definition of replication is crucial to obtaining
  the expected identity $!!P \sim !P$.
\end{remark}

\begin{remark}\label{rem:paradoxical_combinator}
  The reader familiar with the lambda calculus will have noticed the
  similarity between $D$ and the paradoxical combinator.

  [Ed. note: the existence of this seems to suggest we have to be more
  restrictive on the set of processes and names we admit if we are to
  support no-cloning.]
\end{remark}

\subsubsection{Bisimulation}

The computational dynamics gives rise to another kind of equivalence,
the equivalence of computational behavior. As previously mentioned
this is typically captured \emph{via} some form of bisimulation.

% The notion we use in this paper is weak barbed bisimulation
% \cite{milner91polyadicpi}.

The notion we use in this paper is derived from weak barbed
bisimulation \cite{milner91polyadicpi}. 

\begin{definition}
An \emph{observation relation}, $\downarrow_{\mathcal N}$, over a set
of names, $\mathcal N$, is the smallest relation satisfying the rules
below.

\infrule[Out-barb]{y \in {\mathcal N}, \; x \nameeq y}
		  {\outputp{x}{v} \downarrow_{\mathcal N} x}
\infrule[Par-barb]{\mbox{$P\downarrow_{\mathcal N} x$ or $Q\downarrow_{\mathcal N} x$}}
		  {\binpar{P}{Q} \downarrow_{\mathcal N} x}

We write $P \Downarrow_{\mathcal N} x$ if there is $Q$ such that 
$P \wred Q$ and $Q \downarrow_{\mathcal N} x$.
\end{definition}

\begin{definition}
%\label{def.bbisim}
An  ${\mathcal N}$-\emph{barbed bisimulation} over a set of names, ${\mathcal N}$, is a symmetric binary relation 
${\mathcal S}_{\mathcal N}$ between agents such that $P\rel{S}_{\mathcal N}Q$ implies:
\begin{enumerate}
\item If $P \red P'$ then $Q \wred Q'$ and $P'\rel{S}_{\mathcal N} Q'$.
\item If $P\downarrow_{\mathcal N} x$, then $Q\Downarrow_{\mathcal N} x$.
\end{enumerate}
$P$ is ${\mathcal N}$-barbed bisimilar to $Q$, written
$P \wbbisim_{\mathcal N} Q$, if $P \rel{S}_{\mathcal N} Q$ for some ${\mathcal N}$-barbed bisimulation ${\mathcal S}_{\mathcal N}$.
\end{definition}

$\mathcal{R} \subseteq \pi \times \pi$

$P \mathcal{R} Q => \forall P'. P \red P' \Rightarrow \exists Q'. Q \red Q', P' \mathcal{R} Q'$

$P \vdash x \Rightarrow Q \vdash x$

\begin{mathpar}
  \inferrule*[lab=Out-barb]{x \nameeq y}{{y}!\langle{Q}\rangle \vdash x}
  \and
  \inferrule*[lab=Par-barb]{\mbox{$P\vdash x$ or $Q\vdash x$}}{\binpar{P}{Q} \vdash x}
\end{mathpar}

\subsubsection{Contexts}

One of the principle advantages of computational calculi like the
$\pi$-calculus is a well-defined notion of context,
contextual-equivalence and a correlation between
contextual-equivalence and notions of bisimulation. The notion of
context allows the decomposition of a process into (sub-)process and
its syntactic environment, its context. Thus, a context may be
thought of as a process with a ``hole'' (written $\Box$) in it. The
application of a context $M$ to a process $P$, written $M[P]$, is
tantamount to filling the hole in $M$ with $P$. In this paper we do
not need the full weight of this theory, but do make use of the notion
of context in the proof the main theorem. 

\begin{mathpar}
  \inferrule* [lab=summation] {} {{M_{M},M_{N}} \bc \Box \;|\; x.M_{A} \;|\; M_{M}+M_{N}}
  \and
  \inferrule* [lab=agent] {} {{M_{A}} \bc (\vec{x})M_{P} \;| \; \clift{P_0,\ldots,M_{P},\ldots,P_N}}
  \and \\
  \inferrule* [lab=process] {} {{M_{P}} \bc M_{N} \;| \;P|M_{P} }
\end{mathpar} 

\begin{mathpar}
  \inferrule* [lab=sychronization] {} {M_{N} \bc \Box \;|\; x?M_{F} \;|\; x!M_{C}}
  \and
  \inferrule* [lab=abstraction] {} {{M_{F}} \bc (x)M_{P} }
  \and
  \inferrule* [lab=concretion] {} {{M_{C}} \bc \langle M_{P} \rangle }
  \and \\
  \inferrule* [lab=process] {} {{M_{P}} \bc M_{N} \;| \;P|M_{P} }
\end{mathpar}

\begin{definition}[contextual application] Given a context $M$, and
  process $P$, we define the \emph{contextual application}, $M[P] :=
  M\{P/\Box\}$. That is, the contextual application of M to P is the
  substitution of $P$ for $\Box$ in $M$.
\end{definition}

$\meaningof{-} : L \to \mathcal{P}(\pi)$

\begin{mathpar}
  \inferrule* [lab=collection] {} {\meaningof{true} = \pi, \and \meaningof{~E} = \pi \setminus \meaningof{E}, \and \meaningof{E_{1} \& E_{2}} = \meaningof{E_{1}} \cap \meaningof{E_{2}}}
\end{mathpar}

\begin{mathpar}
  \inferrule* [lab=structure] {} {\meaningof{0} = \{ P \in \pi | P \equiv 0 \}, \and \\ \meaningof{E_1 | E_2} = \{ P \in \pi | P \equiv P_{1} | P_{2}, P_{1} \in \meaningof{E_{1}}, P_{2} \in \meaningof{E_2}\} }
\end{mathpar}

\begin{mathpar}
 \inferrule* [lab=behavior] {} {\meaningof{\langle a?b \rangle E} = \{ P \in \pi | P \equiv Q | u?(y)P', \\ \and \\\\ \and \\ \;\;\; u \in \meaningof{a}, \forall z.P'\{z/y\} \in \meaningof{E\{z/b\}}\}, \and \\ \meaningof{a!E} = \{ P \in \pi | P \equiv Q | x!\langle P' \rangle, x \in \meaningof{a} P' \in \meaningof{E}\} }
\end{mathpar}

\begin{mathpar}
 \inferrule* [lab=nominal] {} {\meaningof{\quotep{E}} = \{ \quotep{P} \in \quotep{\pi} | P \in \meaningof{E} \}, \and \meaningof{\quotep{P}} = \{ \quotep{Q} \in \quotep{\pi} | P \equiv Q \} \and \\ \meaningof{@\quotep{E}} = \{ P \in \pi | P \equiv @x, x \in \meaningof{E} \}}
\end{mathpar}

\begin{eqnarray*}
  \\
  \meaningof{-} : TS \to ST
\end{eqnarray*}

\begin{eqnarray*}
  \\
  L : TS \to ST
\end{eqnarray*}

\begin{eqnarray*}
  \\
  P \models E \iff P \in \meaningof{E}
\end{eqnarray*}

\begin{eqnarray*}
  P \approx_{L} Q \iff \forall E \in L. P \models E \iff Q \models E
\end{eqnarray*}

\begin{eqnarray*}
  P \approx_{K} Q
\end{eqnarray*}

\begin{eqnarray*}
  P \approx Q
\end{eqnarray*}

$\approx_{K} = \approx = \approx_{L}$

\subsubsection{Contextual duality}

Note that contexts extend the quotation operation to a family of
operations from processes to names. Given a context, $M$, we can
define a \emph{nominal context}, $\quotep{M}$ by $\quotep{M}[P] :=
\quotep{M[P]}$. To foreshadow what is to come we observe that these
operations enjoy a duality with processes very much like the duality
between vectors and maps from vectors to scalars.

Further, because the calculus is essentially higher-order, we have a
correspondence between contexts and processes. More specifically,
given a name $x$ and a context $M$ we can construct $M^{*}_{x}$ such
that 

\begin{mathpar}
  M^{*}_{x} | \lift{x}{P} \red M[P]
\end{mathpar}

namely,

\begin{mathpar}
  M^{*}_{x} := x?(u).M[\dropn{u}]
\end{mathpar}

The dependence of $M^{*}_{x}$ on a name makes it an abstraction, 

\begin{mathpar}
  M^{*} := (x)x?(u).M[\dropn{u}]
\end{mathpar}

\subsection{Additional notation}

It will sometimes be convenient to denote the process a name
quotes. We already have the notation $x = \quotep{P}$, but it will be
convenient to introduce an alternate notation, $\procn{x}$, when we
want to emphasize the connection to the use of the name. Note that, by
virtue of name equivalence, $\quotep{\procn{x}} \nameeq x$; so, the
notation is consistent with previous definitions.

Further, because names have structure it is possible to effect
substitutions on the basis of that structure. This means we need to
upgrade our notation for substitutions, which we accomplish by
adapting comprehension notation. Thus,

\begin{mathpar}
  P\{ y / x : x \in S \}
\end{mathpar}

is interpreted to mean the process derived from P by replacing (in a
capture-avoiding manner) each occurrence of $x$ in $S$ by $y$. For example,

\begin{mathpar}
  P\{ \quotep{\procn{x}|\procn{x}} / x : x \in \freenames{P} \}
\end{mathpar}

will replace each (occurrence) of a free name $x$ in $P$ by
$\quotep{\procn{x}|\procn{x}}$.

Also, we will avail ourselves of the notation $x^{L}$ and $x^{R}$ to
denote injections of a name into disjoint copies of the name
space. There are numerous ways to accomplish this. One example can be
found in \cite{MeredithR05}. This notation overloads to vectors of
names: $\vec{x}^{\pi} := (x_{i}^{\pi} \; : \; 0 \leq i < |\vec{x}| )$ where $\pi \in \{L,R\}$.

We also use $P^{\Box} := P|\Box$.

In \cite{MeredithR05} an interpretation of the new operator is
given. It turns out that there are several possible interpretations
all enjoying the requisite algebraic properties of the operator (see
\cite{milner91polyadicpi}). We will therefore make liberal use of
$(\nu\; \vec{x})P$.

% subsection the_syntax_and_semantics_of_the_notation_system (end)   

\input{qm2pi.qmops} 

\input{qm2pi.sterngerlach} 

\input{qm2pi.metric} 

% section concurrent_process_calculi (end)

%\input{qm2pi.proofsketch}

% section proof sketch (end)

%\input{qm2pi.slviaknots} 

% section spatial logic via knots (end)

\input{qm2pi.conclusion}

% section conclusion (end)

%\input{qm2pi.dtcodes} 

% section wiring algorithm (end)

\input{qm2pi.ack} 

% section acknowledgments (end)

\newpage


\bibliographystyle{plain}   
\bibliography{../../biblios/main.bib}

\input{qm2pi.rhodetails}

\end{document}

 

% section acknowledgments (end)

\newpage


\bibliographystyle{plain}   
\bibliography{../../biblios/main.bib}

\documentclass[12pt]{llncs}
%\documentclass{jktr}

\usepackage[pdftex]{hyperref}                   
\usepackage {listings}
\usepackage {mathpartir}
\usepackage{bcprules}
%\usepackage{listings}
                       
\usepackage{graphicx} 
%\usepackage[margins=2.5cm,nohead,nofoot]{geometry}
%\usepackage{geometry}
\usepackage{amsfonts}
\usepackage{amstext}
\usepackage{latexsym}
\usepackage{amssymb}
\usepackage{color}


%\include{myPreamble}
\include{qm2pi.local} 

%\ifpdf
%\usepackage[pdftex]{graphicx}
%\else
%\usepackage{graphicx}
%\fi

 % \ifpdf
%  \usepackage{pdfsync}
%  \if


%\title{Brief Article}
%\author{David F. Snyder}
%\author{L.G. Meredith}

%\address{Dept. of Math., Texas State University--San Marcos, San Marcos, TX 78666}
       
\pagestyle{empty}


\begin{document}

\lstset{language=[Objective]Caml,frame=shadowbox}

\input{qm2pi.front}

% section front matter (end)

\input{qm2pi.intro} 
 
% section introduction (end)

% \input{qm2pi.knotations} 

% section notation (end)

\input{qm2pi.process.calculi} 

% section concurrent_process_calculi_and_spatial_logics_ (end)
    
%\input{qm2pi.knots2pi} 

%\input{qm2pi.trefoil} 

%\input{qm2pi.mainthm} 

% subsection basic_interpretation (end)

%\input{qm2pi.rho.presentation} 
\subsection{The syntax and semantics of the notation system}\label{sub:the_syntax_and_semantics_of_the_notation_system} % (fold)

We now summarize a technical presentation of the calculus that
embodies our theory of dynamics. The typical presentation of such a
calculus follows the style of giving generators and relations on
them. The grammar, below, describing term constructors, freely
generates the set of processes, $\Proc$. This set is then quotiented
by a relation known as structural congruence and it is over this set
that the notion of dynamics is expressed. This presentation is
essentially that of \cite{MeredithR05} with the addition of
polyadicity and summation. For readability we have relegated some of
the technical subtleties to an appendix.

\subsubsection{Process grammar}\label{subsub:process_grammar}

\begin{mathpar}
  \inferrule* [lab=synchronization] {} {{M} \bc \pzero \;|\; x?F \;|\; x!C }
  \and
  \inferrule* [lab=abstraction] {} {{F} \bc (x)P}
  \and
  \inferrule* [lab=concretion] {} {{C} \bc \langle Q \rangle}
  \and
  \inferrule* [lab=process] {} {{P,Q} \bc M \;| \;P|Q \;|\; @{x}}
  \and
  \inferrule* [lab=name] {} {{x} \bc \quotep{P}}
\end{mathpar} 

Note that $\vec{x}$ (resp. $\vec{P}$) denotes a vector of names
(resp. processes) of length $|\vec{x}|$ (resp. $|\vec{P}|$). We adopt
the following useful abbreviations.

\begin{mathpar}
   x?(\vec{y}).P := x.(\vec{y})P \and  x\clift{\vec{P}} := x.\clift{\vec{P}}
   \and x!(y) := \lift{x}{\dropn{y}}
   \and \Pi_{i=0}^{n-1}P_i := P_0 | \ldots | P_{n-1}
\end{mathpar}

\subsubsection{Structural congruence}

\paragraph{Free and bound names and alpha-equivalence.} At the
core of structural equivalence is alpha-equivalence which identifies
process that are the same up to a change of variable. Formally, we
recognize the distinction between free and bound names. The free names
of a process, $\freenames{P}$, may be calculated recursively as
follows:

\begin{mathpar}
\freenames{\pzero} := \emptyset
  \and \\
  \freenames{x?(y).P} := \{ x \} \cup (\freenames{P} \setminus \{ y \})
  \and 
  \freenames{x!\langle P \rangle} := \{ x \} \cup \{ P \} 
  \and \\
  \freenames{P|Q} := \freenames{P} \cup \freenames{Q}
  \and \\
  \freenames{@{x}} := \{ x \}
\end{mathpar}

$\pi$
$\quotep{\pi}$

$\freenames{-} : \pi \to \mathcal{P}(\quotep{\pi})$

\begin{eqnarray*}
  \freenames{\pzero} & := & \emptyset \\
  \freenames{x?(y).P} & := & \{ x \} \cup (\freenames{P} \setminus \{ y \}) \\
  \freenames{x!\langle P \rangle} & := & \{ x \} \cup \{ P \} \\
  \freenames{P|Q} & := & \freenames{P} \cup \freenames{Q} \\
  \freenames{\dropn{x}} & := & \{ x \}
\end{eqnarray*}

The bound names of a process, $\boundnames{P}$, are those names occurring in $P$
that are not free. For example, in $x?(y).0$, the name $x$ is free, while $y$ is bound.

\begin{mathpar}
  \inferrule* [lab=monoidal-laws] {} { P|Q \equiv Q|P \and P|0 \equiv P \and P|(Q|R) \equiv (P|Q)|R }
\end{mathpar}

\begin{mathpar}
  \inferrule* [lab=alpha-equivalence] {} { (x)P \equiv (y)P\{y/x\} \and y \not\in \freenames{P} }
\end{mathpar}

\begin{definition}
Then two processes, $P,Q$, are alpha-equivalent if $P = Q\{\vec{y}/\vec{x}\}$ for
some $\vec{x} \in \boundnames{Q},\vec{y} \in \boundnames{P}$, where $Q\{\vec{y}/\vec{x}\}$
denotes the capture-avoiding substitution of $\vec{y}$ for $\vec{x}$ in $Q$.
\end{definition}

\begin{definition}
  The {\em structural congruence} \cite{SangiorgiWalker} , $\equiv$,
  between processes is the least congruence containing
  alpha-equivalence, satisfying the abelian monoid laws
  (associativity, commutativity and $\pzero$ as identity) for parallel
  composition $|$ and for summation $+$.
\end{definition}

\subsection{Name equivalence}

We take name equivalence, written $\nameeq$, to be the smallest
equivalence relation generated by the following rules.

\begin{mathpar}
\inferrule*[lab=Quote-drop]
{ }
{ \quotep{@{x}} \nameeq x }

\inferrule*[lab=Struct-equiv]
{ P \scong Q }
{ \quotep{P} \nameeq \quotep{Q} }
\end{mathpar}

The astute reader will have noticed that the mutual recursion of names
and processes imposes a mutual recursion on alpha-equivalence and
structural equivalence via name-equivalence. Fortunately, all of this
works out pleasantly and we may calculate in the natural way, free of
concern. The reader interested in the details is referred to the
appendix \ref{appendix:rho_details}.

\subsection{Substitution}

We use $\Proc$ for the set of processes, $\QProc$ for the set of
names, and $\id{\{}\vec{y} / \vec{x} \id{\}}$ to denote partial maps,
$s : \QProc \rightarrow \QProc$. A map, $s$ lifts, uniquely, to a map
on process terms, $\widehat{s} : \Proc \rightarrow \Proc$ by the
following equations.

\begin{mathpar}
  (0) \psubstp{Q}{P} := 0 \\
  (R \juxtap S) \psubstp{Q}{P}
  :=    
  (R)\psubstp{Q}{P} \juxtap (S) \psubstp{Q}{P} \\
  (x?(y).R) \psubstp{Q}{P}    
  :=    
  (x)\substp{Q}{P} (z)\concat( (R \psubstn{z}{y}) \psubstp{Q}{P} ) \\
  (\lift{x}{R}) \psubstp{Q}{P}  
  :=
  \lift{(x)\substp{Q}{P}}{ R \psubstp{Q}{P} } \\
%   (\dropn{x})  \psubstp{Q}{P}       
%   := 
%   \left\{ 
%     \begin{array}{ccc} 
%       \dropn{\quotep{Q}} & & x \nameeq \quotep{P} \\
%       \dropn{x} & & otherwise \\
%     \end{array}
%   \right. 
  (\dropn{x})  \psubstp{Q}{P}       
  := 
  \left\{ 
    \begin{array}{ccc} 
      Q & & x \nameeq \quotep{P} \\
      \dropn{x} & & otherwise \\
    \end{array}
  \right.
\end{mathpar}
 

where

\begin{eqnarray}
  (x)\id{\{} \lpquote Q \rpquote / \lpquote P \rpquote \id{\}}            = 
  \left\{ 
    \begin{array}{ccc}
      \lpquote Q \rpquote & & x \nameeq \lpquote P \rpquote \\
      x & & otherwise \\
    \end{array}
  \right. \nonumber
\end{eqnarray}

and $z$ is chosen distinct from $\quotep{P}$, $\quotep{Q}$, the free
names in $Q$, and all the names in $R$. Our $\alpha$-equivalence will
be built in the standard way from this substitution.

\begin{remark}\label{rem:no_self_referential_names}
  One consequence of these definitions is that $\forall P. \quotep{P}
  \not\in \freenames{P}$.
\end{remark}

\subsection{ Dynamic quote: an example }

Anticipating something of what's to come, consider applying the
substitution, $\widehat{\id{\{}u / z \id{\}}}$, to the following pair
of processes, $\lift{w}{y!(z)}$ and $w[ \lpquote y!(z) \rpquote ]$.

\begin{eqnarray}
	\lift{w}{y!(z)}\widehat{\id{\{}u / z \id{\}}}
		& = &
		\lift{w}{y!(u)} \nonumber\\
	w[ \lpquote y!(z) \rpquote ] \widehat{ \id{\{}u / z \id{\}} }
		& = &
		w[ \lpquote y!(z) \rpquote ] \nonumber
\end{eqnarray}

Because the body of the process between quotes is impervious to
substitution, we get radically different answers. In fact, by
examining the first process in an input context,
e.g. $x?(z).\lift{w}{y!(z)}$, we see that the process under the lift
operator may be shaped by prefixed inputs binding a name inside it. In
this sense, the lift operator will be seen as a way to dynamically
construct processes before reifying them as names.

Finally equipped with these standard features we can present the
dynamics of the calculus.

\subsubsection{Operational semantics} 

Finally, we introduce the computational dynamics. What marks these
algebras as distinct from other more traditionally studied algebraic
structures, e.g. vector spaces or polynomial rings, is the manner in
which dynamics is captured. In traditional structures, dynamics is typically
expressed through morphisms between such structures, as in linear maps
between vector spaces or morphisms between rings. In algebras
associated with the semantics of computation, the dynamics is
expressed as part of the algebraic structure itself, through a
reduction reduction relation typically denoted by $\red$. Below, we
give a recursive presentation of this relation for the calculus used
in the encoding.

$\red \subseteq \pi \times \pi$
$\red : \pi \to \mathcal{P}(\pi)$

\begin{mathpar}
  \inferrule* [lab=Comm] { \textsf{match}( x_{src}, x_{trgt} ) } { x_{trgt}?(y)P \; | \; x_{src}!\langle {Q} \rangle \red P\{\quotep{Q}/y}\} }
  \and \\
  \inferrule* [lab=Par] {{P} \red {P}'} {{{P} | {Q}} \red {{P}' | {Q}}}
  \and
  \inferrule* [lab=Equiv]{{{P} \scong {P}'} \andalso {{P}' \red {Q}'} \andalso {{Q}' \scong {Q}}}{{P} \red {Q}}
\end{mathpar}

\begin{eqnarray*}
  match_{\equiv} (\quotep{P},\quotep{Q}) & := & P \equiv Q \\
  match_{\dagger}(\quotep{P},\quotep{Q}) & := & \forall R. P|Q \red^{*} R => R \red^{*} 0 \\
  match_{K}(\quotep{P},\quotep{Q}) & := & K \mbox{ for some context } K
\end{eqnarray*}

$u?(x)P | u!\langle Q \rangle \red P\{\quotep{Q}/x\}$

%We write $\wred$ for $\red^*$, and $P\red$ if $\exists Q $ such that $ P \red Q$.
We write $P\red$ if $\exists Q $ such that $ P \red Q$ and $P\not\red$, otherwise.

\section{Replication}

As mentioned before, it is known that replication (and hence
recursion) can be implemented in a higher-order process algebra
\cite{SangiorgiWalker}. As our first example of calculation with the
machinery thus far presented we give the construction explicitly in
the {\rhoc}.

\begin{eqnarray}
	D_{x} & := & \prefix{x}{y}{(\binpar{\outputp{x}{y}}{@{y}})} \nonumber\\
	\bangp_{x}{P} & := & \binpar{{x}!\langle{\binpar{D_{x}}{P}}\rangle}{D_{x}} \nonumber
\end{eqnarray}

\begin{eqnarray}
	\bangp_{x}{P} & & \nonumber\\
	=
	& {x}!\langle{(\prefix{x}{y}{(\outputp{x}{y} | @{y})) | P}}\rangle 
	      | \prefix{x}{y}{(\outputp{x}{y} | @{y})} & \nonumber\\
	\red
	& (\outputp{x}{y} | @{y})\substn{\quotep{(\prefix{x}{y}{(@{y} | \outputp{x}{y})) | P}}}{y} & \nonumber\\
	=
	& \outputp{x}{\quotep{(\prefix{x}{y}{(\outputp{x}{y} | @{y})) | P}}}
	  | {(\prefix{x}{y}{(\outputp{x}{y} | @{y})) | P}} & \nonumber\\
	\red
	& \ldots & \nonumber\\
	\red^*
	& P | P | \ldots & \nonumber
\end{eqnarray}

Of course, this encoding, as an implementation, runs away, unfolding
$\bangp{P}$ eagerly. A lazier and more implementable replication
operator, restricted to input-guarded processes, may be obtained as follows.

\begin{eqnarray}
\bangp{\prefix{u}{v}{P}} 
	:= 
	\binpar{\lift{x}{\prefix{u}{v}{(\binpar{D(x)}{P})}}}{D(x)} \nonumber
\end{eqnarray}

\begin{remark}
  Note that the lazier definition still does not deal with summation
  or mixed summation (i.e. sums over input and output). The reader is
  invited to construct definitions of replication that deal with these
  features. 

  Further, the definitions are parameterized in a name, $x$. Can you,
  gentle reader, make a definition that eliminates this parameter and
  guarantees no accidental interaction between the replication
  machinery and the process being replicated -- i.e. no accidental
  sharing of names used by the process to get its work done and the
  name(s) used by the replication to effect copying. This latter
  revision of the definition of replication is crucial to obtaining
  the expected identity $!!P \sim !P$.
\end{remark}

\begin{remark}\label{rem:paradoxical_combinator}
  The reader familiar with the lambda calculus will have noticed the
  similarity between $D$ and the paradoxical combinator.

  [Ed. note: the existence of this seems to suggest we have to be more
  restrictive on the set of processes and names we admit if we are to
  support no-cloning.]
\end{remark}

\subsubsection{Bisimulation}

The computational dynamics gives rise to another kind of equivalence,
the equivalence of computational behavior. As previously mentioned
this is typically captured \emph{via} some form of bisimulation.

% The notion we use in this paper is weak barbed bisimulation
% \cite{milner91polyadicpi}.

The notion we use in this paper is derived from weak barbed
bisimulation \cite{milner91polyadicpi}. 

\begin{definition}
An \emph{observation relation}, $\downarrow_{\mathcal N}$, over a set
of names, $\mathcal N$, is the smallest relation satisfying the rules
below.

\infrule[Out-barb]{y \in {\mathcal N}, \; x \nameeq y}
		  {\outputp{x}{v} \downarrow_{\mathcal N} x}
\infrule[Par-barb]{\mbox{$P\downarrow_{\mathcal N} x$ or $Q\downarrow_{\mathcal N} x$}}
		  {\binpar{P}{Q} \downarrow_{\mathcal N} x}

We write $P \Downarrow_{\mathcal N} x$ if there is $Q$ such that 
$P \wred Q$ and $Q \downarrow_{\mathcal N} x$.
\end{definition}

\begin{definition}
%\label{def.bbisim}
An  ${\mathcal N}$-\emph{barbed bisimulation} over a set of names, ${\mathcal N}$, is a symmetric binary relation 
${\mathcal S}_{\mathcal N}$ between agents such that $P\rel{S}_{\mathcal N}Q$ implies:
\begin{enumerate}
\item If $P \red P'$ then $Q \wred Q'$ and $P'\rel{S}_{\mathcal N} Q'$.
\item If $P\downarrow_{\mathcal N} x$, then $Q\Downarrow_{\mathcal N} x$.
\end{enumerate}
$P$ is ${\mathcal N}$-barbed bisimilar to $Q$, written
$P \wbbisim_{\mathcal N} Q$, if $P \rel{S}_{\mathcal N} Q$ for some ${\mathcal N}$-barbed bisimulation ${\mathcal S}_{\mathcal N}$.
\end{definition}

$\mathcal{R} \subseteq \pi \times \pi$

$P \mathcal{R} Q => \forall P'. P \red P' \Rightarrow \exists Q'. Q \red Q', P' \mathcal{R} Q'$

$P \vdash x \Rightarrow Q \vdash x$

\begin{mathpar}
  \inferrule*[lab=Out-barb]{x \nameeq y}{{y}!\langle{Q}\rangle \vdash x}
  \and
  \inferrule*[lab=Par-barb]{\mbox{$P\vdash x$ or $Q\vdash x$}}{\binpar{P}{Q} \vdash x}
\end{mathpar}

\subsubsection{Contexts}

One of the principle advantages of computational calculi like the
$\pi$-calculus is a well-defined notion of context,
contextual-equivalence and a correlation between
contextual-equivalence and notions of bisimulation. The notion of
context allows the decomposition of a process into (sub-)process and
its syntactic environment, its context. Thus, a context may be
thought of as a process with a ``hole'' (written $\Box$) in it. The
application of a context $M$ to a process $P$, written $M[P]$, is
tantamount to filling the hole in $M$ with $P$. In this paper we do
not need the full weight of this theory, but do make use of the notion
of context in the proof the main theorem. 

\begin{mathpar}
  \inferrule* [lab=summation] {} {{M_{M},M_{N}} \bc \Box \;|\; x.M_{A} \;|\; M_{M}+M_{N}}
  \and
  \inferrule* [lab=agent] {} {{M_{A}} \bc (\vec{x})M_{P} \;| \; \clift{P_0,\ldots,M_{P},\ldots,P_N}}
  \and \\
  \inferrule* [lab=process] {} {{M_{P}} \bc M_{N} \;| \;P|M_{P} }
\end{mathpar} 

\begin{mathpar}
  \inferrule* [lab=sychronization] {} {M_{N} \bc \Box \;|\; x?M_{F} \;|\; x!M_{C}}
  \and
  \inferrule* [lab=abstraction] {} {{M_{F}} \bc (x)M_{P} }
  \and
  \inferrule* [lab=concretion] {} {{M_{C}} \bc \langle M_{P} \rangle }
  \and \\
  \inferrule* [lab=process] {} {{M_{P}} \bc M_{N} \;| \;P|M_{P} }
\end{mathpar}

\begin{definition}[contextual application] Given a context $M$, and
  process $P$, we define the \emph{contextual application}, $M[P] :=
  M\{P/\Box\}$. That is, the contextual application of M to P is the
  substitution of $P$ for $\Box$ in $M$.
\end{definition}

$\meaningof{-} : L \to \mathcal{P}(\pi)$

\begin{mathpar}
  \inferrule* [lab=collection] {} {\meaningof{true} = \pi, \and \meaningof{~E} = \pi \setminus \meaningof{E}, \and \meaningof{E_{1} \& E_{2}} = \meaningof{E_{1}} \cap \meaningof{E_{2}}}
\end{mathpar}

\begin{mathpar}
  \inferrule* [lab=structure] {} {\meaningof{0} = \{ P \in \pi | P \equiv 0 \}, \and \\ \meaningof{E_1 | E_2} = \{ P \in \pi | P \equiv P_{1} | P_{2}, P_{1} \in \meaningof{E_{1}}, P_{2} \in \meaningof{E_2}\} }
\end{mathpar}

\begin{mathpar}
 \inferrule* [lab=behavior] {} {\meaningof{\langle a?b \rangle E} = \{ P \in \pi | P \equiv Q | u?(y)P', \\ \and \\\\ \and \\ \;\;\; u \in \meaningof{a}, \forall z.P'\{z/y\} \in \meaningof{E\{z/b\}}\}, \and \\ \meaningof{a!E} = \{ P \in \pi | P \equiv Q | x!\langle P' \rangle, x \in \meaningof{a} P' \in \meaningof{E}\} }
\end{mathpar}

\begin{mathpar}
 \inferrule* [lab=nominal] {} {\meaningof{\quotep{E}} = \{ \quotep{P} \in \quotep{\pi} | P \in \meaningof{E} \}, \and \meaningof{\quotep{P}} = \{ \quotep{Q} \in \quotep{\pi} | P \equiv Q \} \and \\ \meaningof{@\quotep{E}} = \{ P \in \pi | P \equiv @x, x \in \meaningof{E} \}}
\end{mathpar}

\begin{eqnarray*}
  \\
  \meaningof{-} : TS \to ST
\end{eqnarray*}

\begin{eqnarray*}
  \\
  L : TS \to ST
\end{eqnarray*}

\begin{eqnarray*}
  \\
  P \models E \iff P \in \meaningof{E}
\end{eqnarray*}

\begin{eqnarray*}
  P \approx_{L} Q \iff \forall E \in L. P \models E \iff Q \models E
\end{eqnarray*}

\begin{eqnarray*}
  P \approx_{K} Q
\end{eqnarray*}

\begin{eqnarray*}
  P \approx Q
\end{eqnarray*}

$\approx_{K} = \approx = \approx_{L}$

\subsubsection{Contextual duality}

Note that contexts extend the quotation operation to a family of
operations from processes to names. Given a context, $M$, we can
define a \emph{nominal context}, $\quotep{M}$ by $\quotep{M}[P] :=
\quotep{M[P]}$. To foreshadow what is to come we observe that these
operations enjoy a duality with processes very much like the duality
between vectors and maps from vectors to scalars.

Further, because the calculus is essentially higher-order, we have a
correspondence between contexts and processes. More specifically,
given a name $x$ and a context $M$ we can construct $M^{*}_{x}$ such
that 

\begin{mathpar}
  M^{*}_{x} | \lift{x}{P} \red M[P]
\end{mathpar}

namely,

\begin{mathpar}
  M^{*}_{x} := x?(u).M[\dropn{u}]
\end{mathpar}

The dependence of $M^{*}_{x}$ on a name makes it an abstraction, 

\begin{mathpar}
  M^{*} := (x)x?(u).M[\dropn{u}]
\end{mathpar}

\subsection{Additional notation}

It will sometimes be convenient to denote the process a name
quotes. We already have the notation $x = \quotep{P}$, but it will be
convenient to introduce an alternate notation, $\procn{x}$, when we
want to emphasize the connection to the use of the name. Note that, by
virtue of name equivalence, $\quotep{\procn{x}} \nameeq x$; so, the
notation is consistent with previous definitions.

Further, because names have structure it is possible to effect
substitutions on the basis of that structure. This means we need to
upgrade our notation for substitutions, which we accomplish by
adapting comprehension notation. Thus,

\begin{mathpar}
  P\{ y / x : x \in S \}
\end{mathpar}

is interpreted to mean the process derived from P by replacing (in a
capture-avoiding manner) each occurrence of $x$ in $S$ by $y$. For example,

\begin{mathpar}
  P\{ \quotep{\procn{x}|\procn{x}} / x : x \in \freenames{P} \}
\end{mathpar}

will replace each (occurrence) of a free name $x$ in $P$ by
$\quotep{\procn{x}|\procn{x}}$.

Also, we will avail ourselves of the notation $x^{L}$ and $x^{R}$ to
denote injections of a name into disjoint copies of the name
space. There are numerous ways to accomplish this. One example can be
found in \cite{MeredithR05}. This notation overloads to vectors of
names: $\vec{x}^{\pi} := (x_{i}^{\pi} \; : \; 0 \leq i < |\vec{x}| )$ where $\pi \in \{L,R\}$.

We also use $P^{\Box} := P|\Box$.

In \cite{MeredithR05} an interpretation of the new operator is
given. It turns out that there are several possible interpretations
all enjoying the requisite algebraic properties of the operator (see
\cite{milner91polyadicpi}). We will therefore make liberal use of
$(\nu\; \vec{x})P$.

% subsection the_syntax_and_semantics_of_the_notation_system (end)   

\input{qm2pi.qmops} 

\input{qm2pi.sterngerlach} 

\input{qm2pi.metric} 

% section concurrent_process_calculi (end)

%\input{qm2pi.proofsketch}

% section proof sketch (end)

%\input{qm2pi.slviaknots} 

% section spatial logic via knots (end)

\input{qm2pi.conclusion}

% section conclusion (end)

%\input{qm2pi.dtcodes} 

% section wiring algorithm (end)

\input{qm2pi.ack} 

% section acknowledgments (end)

\newpage


\bibliographystyle{plain}   
\bibliography{../../biblios/main.bib}

\input{qm2pi.rhodetails}

\end{document}



\end{document}



\end{document}

 

% subsection basic_interpretation (end)

%\input{qm2pi.rho.presentation} 
\subsection{The syntax and semantics of the notation system}\label{sub:the_syntax_and_semantics_of_the_notation_system} % (fold)

We now summarize a technical presentation of the calculus that
embodies our theory of dynamics. The typical presentation of such a
calculus follows the style of giving generators and relations on
them. The grammar, below, describing term constructors, freely
generates the set of processes, $\Proc$. This set is then quotiented
by a relation known as structural congruence and it is over this set
that the notion of dynamics is expressed. This presentation is
essentially that of \cite{MeredithR05} with the addition of
polyadicity and summation. For readability we have relegated some of
the technical subtleties to an appendix.

\subsubsection{Process grammar}\label{subsub:process_grammar}

\begin{mathpar}
  \inferrule* [lab=synchronization] {} {{M} \bc \pzero \;|\; x?F \;|\; x!C }
  \and
  \inferrule* [lab=abstraction] {} {{F} \bc (x)P}
  \and
  \inferrule* [lab=concretion] {} {{C} \bc \langle Q \rangle}
  \and
  \inferrule* [lab=process] {} {{P,Q} \bc M \;| \;P|Q \;|\; @{x}}
  \and
  \inferrule* [lab=name] {} {{x} \bc \quotep{P}}
\end{mathpar} 

Note that $\vec{x}$ (resp. $\vec{P}$) denotes a vector of names
(resp. processes) of length $|\vec{x}|$ (resp. $|\vec{P}|$). We adopt
the following useful abbreviations.

\begin{mathpar}
   x?(\vec{y}).P := x.(\vec{y})P \and  x\clift{\vec{P}} := x.\clift{\vec{P}}
   \and x!(y) := \lift{x}{\dropn{y}}
   \and \Pi_{i=0}^{n-1}P_i := P_0 | \ldots | P_{n-1}
\end{mathpar}

\subsubsection{Structural congruence}

\paragraph{Free and bound names and alpha-equivalence.} At the
core of structural equivalence is alpha-equivalence which identifies
process that are the same up to a change of variable. Formally, we
recognize the distinction between free and bound names. The free names
of a process, $\freenames{P}$, may be calculated recursively as
follows:

\begin{mathpar}
\freenames{\pzero} := \emptyset
  \and \\
  \freenames{x?(y).P} := \{ x \} \cup (\freenames{P} \setminus \{ y \})
  \and 
  \freenames{x!\langle P \rangle} := \{ x \} \cup \{ P \} 
  \and \\
  \freenames{P|Q} := \freenames{P} \cup \freenames{Q}
  \and \\
  \freenames{@{x}} := \{ x \}
\end{mathpar}

$\pi$
$\quotep{\pi}$

$\freenames{-} : \pi \to \mathcal{P}(\quotep{\pi})$

\begin{eqnarray*}
  \freenames{\pzero} & := & \emptyset \\
  \freenames{x?(y).P} & := & \{ x \} \cup (\freenames{P} \setminus \{ y \}) \\
  \freenames{x!\langle P \rangle} & := & \{ x \} \cup \{ P \} \\
  \freenames{P|Q} & := & \freenames{P} \cup \freenames{Q} \\
  \freenames{\dropn{x}} & := & \{ x \}
\end{eqnarray*}

The bound names of a process, $\boundnames{P}$, are those names occurring in $P$
that are not free. For example, in $x?(y).0$, the name $x$ is free, while $y$ is bound.

\begin{mathpar}
  \inferrule* [lab=monoidal-laws] {} { P|Q \equiv Q|P \and P|0 \equiv P \and P|(Q|R) \equiv (P|Q)|R }
\end{mathpar}

\begin{mathpar}
  \inferrule* [lab=alpha-equivalence] {} { (x)P \equiv (y)P\{y/x\} \and y \not\in \freenames{P} }
\end{mathpar}

\begin{definition}
Then two processes, $P,Q$, are alpha-equivalent if $P = Q\{\vec{y}/\vec{x}\}$ for
some $\vec{x} \in \boundnames{Q},\vec{y} \in \boundnames{P}$, where $Q\{\vec{y}/\vec{x}\}$
denotes the capture-avoiding substitution of $\vec{y}$ for $\vec{x}$ in $Q$.
\end{definition}

\begin{definition}
  The {\em structural congruence} \cite{SangiorgiWalker} , $\equiv$,
  between processes is the least congruence containing
  alpha-equivalence, satisfying the abelian monoid laws
  (associativity, commutativity and $\pzero$ as identity) for parallel
  composition $|$ and for summation $+$.
\end{definition}

\subsection{Name equivalence}

We take name equivalence, written $\nameeq$, to be the smallest
equivalence relation generated by the following rules.

\begin{mathpar}
\inferrule*[lab=Quote-drop]
{ }
{ \quotep{@{x}} \nameeq x }

\inferrule*[lab=Struct-equiv]
{ P \scong Q }
{ \quotep{P} \nameeq \quotep{Q} }
\end{mathpar}

The astute reader will have noticed that the mutual recursion of names
and processes imposes a mutual recursion on alpha-equivalence and
structural equivalence via name-equivalence. Fortunately, all of this
works out pleasantly and we may calculate in the natural way, free of
concern. The reader interested in the details is referred to the
appendix \ref{appendix:rho_details}.

\subsection{Substitution}

We use $\Proc$ for the set of processes, $\QProc$ for the set of
names, and $\id{\{}\vec{y} / \vec{x} \id{\}}$ to denote partial maps,
$s : \QProc \rightarrow \QProc$. A map, $s$ lifts, uniquely, to a map
on process terms, $\widehat{s} : \Proc \rightarrow \Proc$ by the
following equations.

\begin{mathpar}
  (0) \psubstp{Q}{P} := 0 \\
  (R \juxtap S) \psubstp{Q}{P}
  :=    
  (R)\psubstp{Q}{P} \juxtap (S) \psubstp{Q}{P} \\
  (x?(y).R) \psubstp{Q}{P}    
  :=    
  (x)\substp{Q}{P} (z)\concat( (R \psubstn{z}{y}) \psubstp{Q}{P} ) \\
  (\lift{x}{R}) \psubstp{Q}{P}  
  :=
  \lift{(x)\substp{Q}{P}}{ R \psubstp{Q}{P} } \\
%   (\dropn{x})  \psubstp{Q}{P}       
%   := 
%   \left\{ 
%     \begin{array}{ccc} 
%       \dropn{\quotep{Q}} & & x \nameeq \quotep{P} \\
%       \dropn{x} & & otherwise \\
%     \end{array}
%   \right. 
  (\dropn{x})  \psubstp{Q}{P}       
  := 
  \left\{ 
    \begin{array}{ccc} 
      Q & & x \nameeq \quotep{P} \\
      \dropn{x} & & otherwise \\
    \end{array}
  \right.
\end{mathpar}
 

where

\begin{eqnarray}
  (x)\id{\{} \lpquote Q \rpquote / \lpquote P \rpquote \id{\}}            = 
  \left\{ 
    \begin{array}{ccc}
      \lpquote Q \rpquote & & x \nameeq \lpquote P \rpquote \\
      x & & otherwise \\
    \end{array}
  \right. \nonumber
\end{eqnarray}

and $z$ is chosen distinct from $\quotep{P}$, $\quotep{Q}$, the free
names in $Q$, and all the names in $R$. Our $\alpha$-equivalence will
be built in the standard way from this substitution.

\begin{remark}\label{rem:no_self_referential_names}
  One consequence of these definitions is that $\forall P. \quotep{P}
  \not\in \freenames{P}$.
\end{remark}

\subsection{ Dynamic quote: an example }

Anticipating something of what's to come, consider applying the
substitution, $\widehat{\id{\{}u / z \id{\}}}$, to the following pair
of processes, $\lift{w}{y!(z)}$ and $w[ \lpquote y!(z) \rpquote ]$.

\begin{eqnarray}
	\lift{w}{y!(z)}\widehat{\id{\{}u / z \id{\}}}
		& = &
		\lift{w}{y!(u)} \nonumber\\
	w[ \lpquote y!(z) \rpquote ] \widehat{ \id{\{}u / z \id{\}} }
		& = &
		w[ \lpquote y!(z) \rpquote ] \nonumber
\end{eqnarray}

Because the body of the process between quotes is impervious to
substitution, we get radically different answers. In fact, by
examining the first process in an input context,
e.g. $x?(z).\lift{w}{y!(z)}$, we see that the process under the lift
operator may be shaped by prefixed inputs binding a name inside it. In
this sense, the lift operator will be seen as a way to dynamically
construct processes before reifying them as names.

Finally equipped with these standard features we can present the
dynamics of the calculus.

\subsubsection{Operational semantics} 

Finally, we introduce the computational dynamics. What marks these
algebras as distinct from other more traditionally studied algebraic
structures, e.g. vector spaces or polynomial rings, is the manner in
which dynamics is captured. In traditional structures, dynamics is typically
expressed through morphisms between such structures, as in linear maps
between vector spaces or morphisms between rings. In algebras
associated with the semantics of computation, the dynamics is
expressed as part of the algebraic structure itself, through a
reduction reduction relation typically denoted by $\red$. Below, we
give a recursive presentation of this relation for the calculus used
in the encoding.

$\red \subseteq \pi \times \pi$
$\red : \pi \to \mathcal{P}(\pi)$

\begin{mathpar}
  \inferrule* [lab=Comm] { \textsf{match}( x_{src}, x_{trgt} ) } { x_{trgt}?(y)P \; | \; x_{src}!\langle {Q} \rangle \red P\{\quotep{Q}/y}\} }
  \and \\
  \inferrule* [lab=Par] {{P} \red {P}'} {{{P} | {Q}} \red {{P}' | {Q}}}
  \and
  \inferrule* [lab=Equiv]{{{P} \scong {P}'} \andalso {{P}' \red {Q}'} \andalso {{Q}' \scong {Q}}}{{P} \red {Q}}
\end{mathpar}

\begin{eqnarray*}
  match_{\equiv} (\quotep{P},\quotep{Q}) & := & P \equiv Q \\
  match_{\dagger}(\quotep{P},\quotep{Q}) & := & \forall R. P|Q \red^{*} R => R \red^{*} 0 \\
  match_{K}(\quotep{P},\quotep{Q}) & := & K \mbox{ for some context } K
\end{eqnarray*}

$u?(x)P | u!\langle Q \rangle \red P\{\quotep{Q}/x\}$

%We write $\wred$ for $\red^*$, and $P\red$ if $\exists Q $ such that $ P \red Q$.
We write $P\red$ if $\exists Q $ such that $ P \red Q$ and $P\not\red$, otherwise.

\section{Replication}

As mentioned before, it is known that replication (and hence
recursion) can be implemented in a higher-order process algebra
\cite{SangiorgiWalker}. As our first example of calculation with the
machinery thus far presented we give the construction explicitly in
the {\rhoc}.

\begin{eqnarray}
	D_{x} & := & \prefix{x}{y}{(\binpar{\outputp{x}{y}}{@{y}})} \nonumber\\
	\bangp_{x}{P} & := & \binpar{{x}!\langle{\binpar{D_{x}}{P}}\rangle}{D_{x}} \nonumber
\end{eqnarray}

\begin{eqnarray}
	\bangp_{x}{P} & & \nonumber\\
	=
	& {x}!\langle{(\prefix{x}{y}{(\outputp{x}{y} | @{y})) | P}}\rangle 
	      | \prefix{x}{y}{(\outputp{x}{y} | @{y})} & \nonumber\\
	\red
	& (\outputp{x}{y} | @{y})\substn{\quotep{(\prefix{x}{y}{(@{y} | \outputp{x}{y})) | P}}}{y} & \nonumber\\
	=
	& \outputp{x}{\quotep{(\prefix{x}{y}{(\outputp{x}{y} | @{y})) | P}}}
	  | {(\prefix{x}{y}{(\outputp{x}{y} | @{y})) | P}} & \nonumber\\
	\red
	& \ldots & \nonumber\\
	\red^*
	& P | P | \ldots & \nonumber
\end{eqnarray}

Of course, this encoding, as an implementation, runs away, unfolding
$\bangp{P}$ eagerly. A lazier and more implementable replication
operator, restricted to input-guarded processes, may be obtained as follows.

\begin{eqnarray}
\bangp{\prefix{u}{v}{P}} 
	:= 
	\binpar{\lift{x}{\prefix{u}{v}{(\binpar{D(x)}{P})}}}{D(x)} \nonumber
\end{eqnarray}

\begin{remark}
  Note that the lazier definition still does not deal with summation
  or mixed summation (i.e. sums over input and output). The reader is
  invited to construct definitions of replication that deal with these
  features. 

  Further, the definitions are parameterized in a name, $x$. Can you,
  gentle reader, make a definition that eliminates this parameter and
  guarantees no accidental interaction between the replication
  machinery and the process being replicated -- i.e. no accidental
  sharing of names used by the process to get its work done and the
  name(s) used by the replication to effect copying. This latter
  revision of the definition of replication is crucial to obtaining
  the expected identity $!!P \sim !P$.
\end{remark}

\begin{remark}\label{rem:paradoxical_combinator}
  The reader familiar with the lambda calculus will have noticed the
  similarity between $D$ and the paradoxical combinator.

  [Ed. note: the existence of this seems to suggest we have to be more
  restrictive on the set of processes and names we admit if we are to
  support no-cloning.]
\end{remark}

\subsubsection{Bisimulation}

The computational dynamics gives rise to another kind of equivalence,
the equivalence of computational behavior. As previously mentioned
this is typically captured \emph{via} some form of bisimulation.

% The notion we use in this paper is weak barbed bisimulation
% \cite{milner91polyadicpi}.

The notion we use in this paper is derived from weak barbed
bisimulation \cite{milner91polyadicpi}. 

\begin{definition}
An \emph{observation relation}, $\downarrow_{\mathcal N}$, over a set
of names, $\mathcal N$, is the smallest relation satisfying the rules
below.

\infrule[Out-barb]{y \in {\mathcal N}, \; x \nameeq y}
		  {\outputp{x}{v} \downarrow_{\mathcal N} x}
\infrule[Par-barb]{\mbox{$P\downarrow_{\mathcal N} x$ or $Q\downarrow_{\mathcal N} x$}}
		  {\binpar{P}{Q} \downarrow_{\mathcal N} x}

We write $P \Downarrow_{\mathcal N} x$ if there is $Q$ such that 
$P \wred Q$ and $Q \downarrow_{\mathcal N} x$.
\end{definition}

\begin{definition}
%\label{def.bbisim}
An  ${\mathcal N}$-\emph{barbed bisimulation} over a set of names, ${\mathcal N}$, is a symmetric binary relation 
${\mathcal S}_{\mathcal N}$ between agents such that $P\rel{S}_{\mathcal N}Q$ implies:
\begin{enumerate}
\item If $P \red P'$ then $Q \wred Q'$ and $P'\rel{S}_{\mathcal N} Q'$.
\item If $P\downarrow_{\mathcal N} x$, then $Q\Downarrow_{\mathcal N} x$.
\end{enumerate}
$P$ is ${\mathcal N}$-barbed bisimilar to $Q$, written
$P \wbbisim_{\mathcal N} Q$, if $P \rel{S}_{\mathcal N} Q$ for some ${\mathcal N}$-barbed bisimulation ${\mathcal S}_{\mathcal N}$.
\end{definition}

$\mathcal{R} \subseteq \pi \times \pi$

$P \mathcal{R} Q => \forall P'. P \red P' \Rightarrow \exists Q'. Q \red Q', P' \mathcal{R} Q'$

$P \vdash x \Rightarrow Q \vdash x$

\begin{mathpar}
  \inferrule*[lab=Out-barb]{x \nameeq y}{{y}!\langle{Q}\rangle \vdash x}
  \and
  \inferrule*[lab=Par-barb]{\mbox{$P\vdash x$ or $Q\vdash x$}}{\binpar{P}{Q} \vdash x}
\end{mathpar}

\subsubsection{Contexts}

One of the principle advantages of computational calculi like the
$\pi$-calculus is a well-defined notion of context,
contextual-equivalence and a correlation between
contextual-equivalence and notions of bisimulation. The notion of
context allows the decomposition of a process into (sub-)process and
its syntactic environment, its context. Thus, a context may be
thought of as a process with a ``hole'' (written $\Box$) in it. The
application of a context $M$ to a process $P$, written $M[P]$, is
tantamount to filling the hole in $M$ with $P$. In this paper we do
not need the full weight of this theory, but do make use of the notion
of context in the proof the main theorem. 

\begin{mathpar}
  \inferrule* [lab=summation] {} {{M_{M},M_{N}} \bc \Box \;|\; x.M_{A} \;|\; M_{M}+M_{N}}
  \and
  \inferrule* [lab=agent] {} {{M_{A}} \bc (\vec{x})M_{P} \;| \; \clift{P_0,\ldots,M_{P},\ldots,P_N}}
  \and \\
  \inferrule* [lab=process] {} {{M_{P}} \bc M_{N} \;| \;P|M_{P} }
\end{mathpar} 

\begin{mathpar}
  \inferrule* [lab=sychronization] {} {M_{N} \bc \Box \;|\; x?M_{F} \;|\; x!M_{C}}
  \and
  \inferrule* [lab=abstraction] {} {{M_{F}} \bc (x)M_{P} }
  \and
  \inferrule* [lab=concretion] {} {{M_{C}} \bc \langle M_{P} \rangle }
  \and \\
  \inferrule* [lab=process] {} {{M_{P}} \bc M_{N} \;| \;P|M_{P} }
\end{mathpar}

\begin{definition}[contextual application] Given a context $M$, and
  process $P$, we define the \emph{contextual application}, $M[P] :=
  M\{P/\Box\}$. That is, the contextual application of M to P is the
  substitution of $P$ for $\Box$ in $M$.
\end{definition}

$\meaningof{-} : L \to \mathcal{P}(\pi)$

\begin{mathpar}
  \inferrule* [lab=collection] {} {\meaningof{true} = \pi, \and \meaningof{~E} = \pi \setminus \meaningof{E}, \and \meaningof{E_{1} \& E_{2}} = \meaningof{E_{1}} \cap \meaningof{E_{2}}}
\end{mathpar}

\begin{mathpar}
  \inferrule* [lab=structure] {} {\meaningof{0} = \{ P \in \pi | P \equiv 0 \}, \and \\ \meaningof{E_1 | E_2} = \{ P \in \pi | P \equiv P_{1} | P_{2}, P_{1} \in \meaningof{E_{1}}, P_{2} \in \meaningof{E_2}\} }
\end{mathpar}

\begin{mathpar}
 \inferrule* [lab=behavior] {} {\meaningof{\langle a?b \rangle E} = \{ P \in \pi | P \equiv Q | u?(y)P', \\ \and \\\\ \and \\ \;\;\; u \in \meaningof{a}, \forall z.P'\{z/y\} \in \meaningof{E\{z/b\}}\}, \and \\ \meaningof{a!E} = \{ P \in \pi | P \equiv Q | x!\langle P' \rangle, x \in \meaningof{a} P' \in \meaningof{E}\} }
\end{mathpar}

\begin{mathpar}
 \inferrule* [lab=nominal] {} {\meaningof{\quotep{E}} = \{ \quotep{P} \in \quotep{\pi} | P \in \meaningof{E} \}, \and \meaningof{\quotep{P}} = \{ \quotep{Q} \in \quotep{\pi} | P \equiv Q \} \and \\ \meaningof{@\quotep{E}} = \{ P \in \pi | P \equiv @x, x \in \meaningof{E} \}}
\end{mathpar}

\begin{eqnarray*}
  \\
  \meaningof{-} : TS \to ST
\end{eqnarray*}

\begin{eqnarray*}
  \\
  L : TS \to ST
\end{eqnarray*}

\begin{eqnarray*}
  \\
  P \models E \iff P \in \meaningof{E}
\end{eqnarray*}

\begin{eqnarray*}
  P \approx_{L} Q \iff \forall E \in L. P \models E \iff Q \models E
\end{eqnarray*}

\begin{eqnarray*}
  P \approx_{K} Q
\end{eqnarray*}

\begin{eqnarray*}
  P \approx Q
\end{eqnarray*}

$\approx_{K} = \approx = \approx_{L}$

\subsubsection{Contextual duality}

Note that contexts extend the quotation operation to a family of
operations from processes to names. Given a context, $M$, we can
define a \emph{nominal context}, $\quotep{M}$ by $\quotep{M}[P] :=
\quotep{M[P]}$. To foreshadow what is to come we observe that these
operations enjoy a duality with processes very much like the duality
between vectors and maps from vectors to scalars.

Further, because the calculus is essentially higher-order, we have a
correspondence between contexts and processes. More specifically,
given a name $x$ and a context $M$ we can construct $M^{*}_{x}$ such
that 

\begin{mathpar}
  M^{*}_{x} | \lift{x}{P} \red M[P]
\end{mathpar}

namely,

\begin{mathpar}
  M^{*}_{x} := x?(u).M[\dropn{u}]
\end{mathpar}

The dependence of $M^{*}_{x}$ on a name makes it an abstraction, 

\begin{mathpar}
  M^{*} := (x)x?(u).M[\dropn{u}]
\end{mathpar}

\subsection{Additional notation}

It will sometimes be convenient to denote the process a name
quotes. We already have the notation $x = \quotep{P}$, but it will be
convenient to introduce an alternate notation, $\procn{x}$, when we
want to emphasize the connection to the use of the name. Note that, by
virtue of name equivalence, $\quotep{\procn{x}} \nameeq x$; so, the
notation is consistent with previous definitions.

Further, because names have structure it is possible to effect
substitutions on the basis of that structure. This means we need to
upgrade our notation for substitutions, which we accomplish by
adapting comprehension notation. Thus,

\begin{mathpar}
  P\{ y / x : x \in S \}
\end{mathpar}

is interpreted to mean the process derived from P by replacing (in a
capture-avoiding manner) each occurrence of $x$ in $S$ by $y$. For example,

\begin{mathpar}
  P\{ \quotep{\procn{x}|\procn{x}} / x : x \in \freenames{P} \}
\end{mathpar}

will replace each (occurrence) of a free name $x$ in $P$ by
$\quotep{\procn{x}|\procn{x}}$.

Also, we will avail ourselves of the notation $x^{L}$ and $x^{R}$ to
denote injections of a name into disjoint copies of the name
space. There are numerous ways to accomplish this. One example can be
found in \cite{MeredithR05}. This notation overloads to vectors of
names: $\vec{x}^{\pi} := (x_{i}^{\pi} \; : \; 0 \leq i < |\vec{x}| )$ where $\pi \in \{L,R\}$.

We also use $P^{\Box} := P|\Box$.

In \cite{MeredithR05} an interpretation of the new operator is
given. It turns out that there are several possible interpretations
all enjoying the requisite algebraic properties of the operator (see
\cite{milner91polyadicpi}). We will therefore make liberal use of
$(\nu\; \vec{x})P$.

% subsection the_syntax_and_semantics_of_the_notation_system (end)   

\section{Interpretation of QM}
\subsection{Supporting definitions}
\subsubsection{Multiplication}
\begin{mathpar}
  \quotep{Q} \cdot \quotep{R} := \quotep{Q|R}
  \and \\
  \quotep{Q} \cdot P := P\{ \quotep{Q|R} / \quotep{R} : \quotep{R} \in \freenames{P} \}
\end{mathpar}

\paragraph{Discussion}
The first line needs little explanation. The second line says that
each free name of the process is replaced with the multiplication of
that name by the scalar. Multiplication of a scalar (name) by a state
(process) results in a process all the names of which have been `moved
over' by parallel composition with the process the scalar
quotes. There is a subtlety that the bound names have to be
manipulated so that multiplied names aren't accidentally
captured. There are many ways to achieve this.

\begin{remark}\label{rem:multiplication_identities}
  The reader is invited to verify that for all $x,y,z \in \QProc$ and $P \in \Proc$
  \begin{mathpar}
    x \cdot \quotep{0} \equiv x 
    \and
    x \cdot y \equiv y \cdot x
    \and
    x \cdot (y \cdot z) \equiv (x \cdot y) \cdot z
    \and \\
    \quotep{0} \cdot P \equiv P
    \and \\
    x \cdot (y \cdot P) \equiv (x \cdot y) \cdot P
    \and \\
    x \cdot (P|Q) \equiv (x \cdot P) | (x \cdot Q)
    \and \\    
  \end{mathpar}
\end{remark}

\subsubsection{Tensor product}

We define a tensor product on processes by structural induction.

\paragraph{Tensor of sums} First note that all summations, including
$\pzero$ and sequence, can be written $\Sigma_{i} x_{i}.A_{i} +
\Sigma_{j} x_{j}.C_{j}$, where we have grouped input-guarded processes
together and output-guarded processes together.

Thus, we can define the tensor product of two summations, $N_{1}\otimes N_{2}$, where

\begin{mathpar}
  N_{1} := \Sigma_{i} x_{i}.A_{i} + \Sigma_{j} x_{j}.C_{j}
  \and
  N_{2} := \Sigma_{i'} y_{i'}.B_{i'} + \Sigma_{j'} y_{j'}.D_{j'} 
\end{mathpar}

as follows.

\begin{mathpar}
  \Sigma_{i} x_{i}.A_{i} + \Sigma_{j} x_{j}.C_{j} \otimes \Sigma_{i'}
  y_{i'}.B_{i'} + \Sigma_{j'} y_{j'}.D_{j'} 
  \and \\
  := \; \Sigma_{i} \Sigma_{i'} \quotep{\stackrel{\vee}{x_{i}}| \stackrel{\vee}{y_{i'}}}.(A_{i}\otimes B_{i'}) \; | \; \Sigma_{i'} \Sigma_{i} \quotep{\stackrel{\vee}{y_{i'}}|\stackrel{\vee}{x_{i}}}.(B_{i'}\otimes A_{i})
  \and
  \;\; | \;\; \Sigma_{j} \Sigma_{j'} \quotep{\stackrel{\vee}{x_{j}}|\stackrel{\vee}{y_{j'}}}.(A_{j}\otimes B_{j'}) \; | \; \Sigma_{j'} \Sigma_{j} \quotep{\stackrel{\vee}{y_{j'}}|\stackrel{\vee}{x_{j}}}.(B_{j'}\otimes A_{j})
\end{mathpar}

\begin{remark}
  Do we need to $x^{L}$ and $y^{R}$ for this construction as well?
\end{remark}

\paragraph{Tensor of parallel compositions} Next, we distribute tensor
over par.

\begin{mathpar}
  P_{1}|P_{2} \otimes Q_{1}|Q_{2} := (P_{1} \otimes Q_{1}) | (P_{1}
  \otimes Q_{2}) | (P_{2} \otimes Q_{1}) | (P_{2} \otimes Q_{2})
\end{mathpar}

\paragraph{Tensor with dropped names} We treat tensor of a
process with a dropped name as parallel composition.

\begin{mathpar}
  P \otimes \dropn{x} := P | \dropn{x}
\end{mathpar}

\paragraph{Tensor of agents}

Finally, we need to define tensor on agents. Note that the definition
of tensor on normal products only tensors inputs with inputs and
outputs with outputs. Thus, we only have to define the operation on
``homogeneous'' pairings.

\begin{mathpar}
  (\vec{x})P \otimes (\vec{y})Q
  \and \\
  := (x_{0}^{L}|y_{0}^{R},\ldots,x_{0}^{L}|y_{n}^{R},\ldots,x_{m}^{L}|y_{0}^{R},\ldots,x_{m}^{L}|y_{n}^R)(P\{ \vec{x}^{L}/\vec{x}\} \otimes Q \{ \vec{y}^{R}/\vec{y}\})
  \and \\
  \clift{\vec{P}} \otimes \clift{\vec{Q}}
  \and \\
  := \clift{P_{0}\otimes Q_{0},\ldots,P_{0}\otimes Q_{n},\ldots,P_{m}\otimes Q_{0},\ldots,P_{m}\otimes Q_{n}}
\end{mathpar}

\begin{remark}
  Observe that arities of tensored abstractions matches arities of
  tensored concretions if the original arities matched. Note also that
  the length of the arities corresponds to the increase in dimension
  we see in ordinary vector space tensor product.
\end{remark}

\begin{remark}
  Operationally, this definition distributes the tensor down to
  components ``linked'' by summation. Tensor over summation is
  intriguing in that it mixes names. Moreover, as a consequence of the
  way it mixes names we have the identities for all $x \in \QProc$ and
  $P,Q \in \Proc$

  \begin{mathpar}
    (x \cdot P) \otimes Q \equiv x \cdot (P \otimes Q) \equiv P \otimes (x \cdot Q)
    \and
    P \otimes \pzero \equiv P
  \end{mathpar}

  that the reader is invited to verify.
\end{remark}

\subsubsection{Annihilation}
\begin{mathpar}
  P^{\perp} := \{ Q | \forall R. P|Q \red^{*} R \Rightarrow R \red^{*} \pzero \}
  \and \\
  P^{\underline{\perp}} := \Sigma_{Q \in P^{\perp}} \quotep{Q}?(y).(\dropn{y}|Q) | \Sigma_{Q \in P^{\perp}} \quotep{Q}\clift{\Box}
\end{mathpar}

\paragraph{Discussion} The reader will note that $P^{\perp}$ is a
\emph{set} of processes, while $P^{\underline{\perp}}$ is a
\emph{context}. We call the set $P^{\perp}$ the \emph{annihilators} of
$P$. The parallel composition of a process in the annihilators of $P$
with $P$ will result in a process, the state space of which has all
paths eventually leading to $\pzero$. Execution may endure loops; but
under reasonable conditions of fairness (naturally guaranteed under
most notions of bisimulation) such a composite process cannot get
stuck in such a loop and will, eventually pop out and terminate.

The context $P^{\underline{\perp}}$ is ready and willing to ``take the
$P$ out of'' the process to which it is applied. It will effectively
transmit the code of the process to which it is applied to one of the
annihilators and run the process against it.

\subsubsection{Evaluation}
We fix $M$ a domain of fully abstract interpretation with an equality
coincident with bisimulation. We take $\meaningof{\cdot} : \Proc \to
M$ to be the map interpreting processes and $\nmeaningof{\cdot} : \M
\to Proc$ to be the map running the other way. Then we define

\begin{mathpar}
  \int P := \nmeaningof{\meaningof{P}}
\end{mathpar}

\paragraph{Discussion}
There are many fully abstract interpretations of Milner's
$\pi$-calculus. Any of them can be used as a basis for interpreting
the reflective calculus here. Equipped with such a domain it is
largely a matter of grinding through to check that the Yoneda
construction for the normalization-by-evaluation program can be
extended to this setting.

\begin{remark}
  The reader is invited to verify that $\int (P^{\underline{\perp}}[P]) = 0$.
\end{remark}

\subsection{Quantum mechanics}

Table \ref{tbl:core_qm_op_defns} gives the core operational definitions

\begin{table}[htp]\label{tbl:core_qm_op_defns}
  \center{
    \fbox{
      \begin{tabular}{c|c}
        quantum mechanics & process calculus \\
        \hline
        scalar & $x := \quotep{P}$ \\
        state vector & $\state{P} := P$ \\
        dual & $\state{P}^{*} := \event{P^{\underline{\perp}}} := \quotep{P^{\underline{\perp}}}[-]$ \\
        matrix & $ \Sigma_{\alpha} \state{P_{\alpha}}x_{\alpha}\event{Q_{\alpha}}$ \\
        vector addition & $\state{P} + \state{Q} := \state{P | Q}$ \\
        tensor product & $\state{P} \otimes \state{Q} := \state{P \otimes Q}$ \\
        inner product & $\innerprod{P}{Q} := \quotep{\int P^{\underline{\perp}}[Q]}$ \\
      \end{tabular}
    }
  }
  \caption{QM - operational definitions}
\end{table}

where

\begin{mathpar}
  \prmatrix{P}{Q} := \fprmatrix{P}{\quotep{\pzero}}{Q}
  \and
  \fprmatrix{P}{x}{Q} := (\state{P},x,\event{Q})
  \and
  (\fprmatrix{P}{x}{Q})(\state{R}) := x \cdot \innerprod{Q}{R} \cdot \state{P}
  \and
  (\fprmatrix{P}{x}{Q})(\event{R}) := x \cdot \innerprod{R}{P} \cdot \event{Q}
\end{mathpar}

\paragraph{Discussion}
As promised: vectors (aka states) are represented as processes; duals
as contextual duals; inner product definition should be compared with
standard inner product definition for ....

\begin{remark}
  Assuming $\int (P^{\underline{\perp}}[P]) = 0$, the reader is
  invited to verify that $(\fprmatrix{P}{x}{P})(\state{P}) = x \cdot \state{P}$.
\end{remark}

\begin{remark}
  The reader is invited to verify that $\innerprod{P}{Q}$ could
  equally well have been written $\quotep{\int \stackrel{\vee}{x}}$
  where $x = \event{P^{\underline{\perp}}}(Q)$.

  One of the motivations for this remark is that there is another way
  to factor these operations. We could package up evaluation in the dual:

  \begin{mathpar}
    \state{P}^{*} := \event{\int P^{\underline{\perp}}} := \quotep{\int P^{\underline{\perp}}}[-]
  \end{mathpar}

  and then have inner product defined by
  
  \begin{mathpar}
    \innerprod{P}{Q} := \event{P}(Q)
  \end{mathpar}

  Hopefully, experience with the calculations will provide guidance on
  the best factoring.
\end{remark}

\begin{remark}
  Assuming $\int (P^{\underline{\perp}}[P]) = 0$, the reader is
  invited to verify that $\forall P,Q. (\prmatrix{0}{Q})(\state{0}) =
  \state{0}$ and dually $(\prmatrix{P}{0})(\event{0}) = \event{0}$.
\end{remark}

\begin{remark}
  i'm a little worried that i don't (yet) have proper support for
  complex conjugacy. But, the observation above may give us a
  clue. According to Abramsky, it must be the case that the scalars
  are iso to the homset of the identity for the tensor -- which the
  observation above characterizes. 

  For now, we will simply bookmark the notion with $\overline{x}$.
\end{remark}

\subsubsection{Adjointness}

We need to give a definition of $(\cdot)^{\dagger}$ for matrices. The
obvious candidate definition is
\begin{mathpar}
(\Sigma_{\alpha}\fprmatrix{P_{\alpha}}{x_{\alpha}}{Q_{\alpha}})^{\dagger}
= \Sigma_{\alpha}\fprmatrix{(Q_{\alpha}^{\underline{\perp}})^{*}}{\overline{x}_{\alpha}}{P_{\alpha}^{\underline{\perp}}} 
\end{mathpar}

But, $(Q_{\alpha}^{\underline{\perp}})^{*}$ requires a name along
which to communicate the process to achieve the context application.

\subsubsection{Basis for a basis}
If processes label states and ``addition'' of states (a.k.a. vector
addition) is interpreted as parallel composition, what corresponds to
notions of linear independence and basis? Here, we recall that Yoshida
has developed a set of \emph{combinators} for an asynchronous verison
of Milner's $\pi$-calculus. These are a finite set of processes such
any process can be expressed as parallel composition of these
combinators together with liberal uses of the new operator and
replication. We can simply give a translation of these into the
present calculus and have reasonable expectation that the property
carries over. That is, that the resultant set allows to express all
processes via parallel composition. Note, however, that there is no
new operator or replication in this calculus. As a result, we expect
that the corresponding set is actually infinite. That is, we expect
that the space is actually infinite dimensional.

\begin{remark}
  The attentive reader may be a bit concerned. Certainly, the
  collection $S$, $K$ and $I$ is a finite set of
  combinators. Shouldn't we expect to see a finite set of combinators
  for an effectively equivalent system? i am very sympathetic to this
  critique and feel it warrants full attention. On the other hand, i
  also have in mind the following analogy. The natural numbers, as a
  monoid under addition, has exactly $1$ generator, while the natural
  numbers, as a monoid under multiplication, has countably many
  generators (the primes). We observe that the application of the
  lambda calculus is much less resource sensitive than the parallel
  composition of the $\pi$-calculus. Could it be the case that we have
  an analogy of the form
  
  \begin{mathpar}
    m + n : MN :: m*n : M|N
  \end{mathpar}

  giving a similar blow up in the set of ``primes''?  This is such a
  wonderful thought that, even if it's not true, i think it's worth
  writing down.
\end{remark}
 

\documentclass[12pt]{llncs}
%\documentclass{jktr}

\usepackage[pdftex]{hyperref}                   
\usepackage {listings}
\usepackage {mathpartir}
\usepackage{bcprules}
%\usepackage{listings}
                       
\usepackage{graphicx} 
%\usepackage[margins=2.5cm,nohead,nofoot]{geometry}
%\usepackage{geometry}
\usepackage{amsfonts}
\usepackage{amstext}
\usepackage{latexsym}
\usepackage{amssymb}
\usepackage{color}


%\include{myPreamble}
\documentclass[12pt]{llncs}
%\documentclass{jktr}

\usepackage[pdftex]{hyperref}                   
\usepackage {listings}
\usepackage {mathpartir}
\usepackage{bcprules}
%\usepackage{listings}
                       
\usepackage{graphicx} 
%\usepackage[margins=2.5cm,nohead,nofoot]{geometry}
%\usepackage{geometry}
\usepackage{amsfonts}
\usepackage{amstext}
\usepackage{latexsym}
\usepackage{amssymb}
\usepackage{color}


%\include{myPreamble}
\documentclass[12pt]{llncs}
%\documentclass{jktr}

\usepackage[pdftex]{hyperref}                   
\usepackage {listings}
\usepackage {mathpartir}
\usepackage{bcprules}
%\usepackage{listings}
                       
\usepackage{graphicx} 
%\usepackage[margins=2.5cm,nohead,nofoot]{geometry}
%\usepackage{geometry}
\usepackage{amsfonts}
\usepackage{amstext}
\usepackage{latexsym}
\usepackage{amssymb}
\usepackage{color}


%\include{myPreamble}
\include{qm2pi.local} 

%\ifpdf
%\usepackage[pdftex]{graphicx}
%\else
%\usepackage{graphicx}
%\fi

 % \ifpdf
%  \usepackage{pdfsync}
%  \if


%\title{Brief Article}
%\author{David F. Snyder}
%\author{L.G. Meredith}

%\address{Dept. of Math., Texas State University--San Marcos, San Marcos, TX 78666}
       
\pagestyle{empty}


\begin{document}

\lstset{language=[Objective]Caml,frame=shadowbox}

\input{qm2pi.front}

% section front matter (end)

\input{qm2pi.intro} 
 
% section introduction (end)

% \input{qm2pi.knotations} 

% section notation (end)

\input{qm2pi.process.calculi} 

% section concurrent_process_calculi_and_spatial_logics_ (end)
    
%\input{qm2pi.knots2pi} 

%\input{qm2pi.trefoil} 

%\input{qm2pi.mainthm} 

% subsection basic_interpretation (end)

%\input{qm2pi.rho.presentation} 
\subsection{The syntax and semantics of the notation system}\label{sub:the_syntax_and_semantics_of_the_notation_system} % (fold)

We now summarize a technical presentation of the calculus that
embodies our theory of dynamics. The typical presentation of such a
calculus follows the style of giving generators and relations on
them. The grammar, below, describing term constructors, freely
generates the set of processes, $\Proc$. This set is then quotiented
by a relation known as structural congruence and it is over this set
that the notion of dynamics is expressed. This presentation is
essentially that of \cite{MeredithR05} with the addition of
polyadicity and summation. For readability we have relegated some of
the technical subtleties to an appendix.

\subsubsection{Process grammar}\label{subsub:process_grammar}

\begin{mathpar}
  \inferrule* [lab=synchronization] {} {{M} \bc \pzero \;|\; x?F \;|\; x!C }
  \and
  \inferrule* [lab=abstraction] {} {{F} \bc (x)P}
  \and
  \inferrule* [lab=concretion] {} {{C} \bc \langle Q \rangle}
  \and
  \inferrule* [lab=process] {} {{P,Q} \bc M \;| \;P|Q \;|\; @{x}}
  \and
  \inferrule* [lab=name] {} {{x} \bc \quotep{P}}
\end{mathpar} 

Note that $\vec{x}$ (resp. $\vec{P}$) denotes a vector of names
(resp. processes) of length $|\vec{x}|$ (resp. $|\vec{P}|$). We adopt
the following useful abbreviations.

\begin{mathpar}
   x?(\vec{y}).P := x.(\vec{y})P \and  x\clift{\vec{P}} := x.\clift{\vec{P}}
   \and x!(y) := \lift{x}{\dropn{y}}
   \and \Pi_{i=0}^{n-1}P_i := P_0 | \ldots | P_{n-1}
\end{mathpar}

\subsubsection{Structural congruence}

\paragraph{Free and bound names and alpha-equivalence.} At the
core of structural equivalence is alpha-equivalence which identifies
process that are the same up to a change of variable. Formally, we
recognize the distinction between free and bound names. The free names
of a process, $\freenames{P}$, may be calculated recursively as
follows:

\begin{mathpar}
\freenames{\pzero} := \emptyset
  \and \\
  \freenames{x?(y).P} := \{ x \} \cup (\freenames{P} \setminus \{ y \})
  \and 
  \freenames{x!\langle P \rangle} := \{ x \} \cup \{ P \} 
  \and \\
  \freenames{P|Q} := \freenames{P} \cup \freenames{Q}
  \and \\
  \freenames{@{x}} := \{ x \}
\end{mathpar}

$\pi$
$\quotep{\pi}$

$\freenames{-} : \pi \to \mathcal{P}(\quotep{\pi})$

\begin{eqnarray*}
  \freenames{\pzero} & := & \emptyset \\
  \freenames{x?(y).P} & := & \{ x \} \cup (\freenames{P} \setminus \{ y \}) \\
  \freenames{x!\langle P \rangle} & := & \{ x \} \cup \{ P \} \\
  \freenames{P|Q} & := & \freenames{P} \cup \freenames{Q} \\
  \freenames{\dropn{x}} & := & \{ x \}
\end{eqnarray*}

The bound names of a process, $\boundnames{P}$, are those names occurring in $P$
that are not free. For example, in $x?(y).0$, the name $x$ is free, while $y$ is bound.

\begin{mathpar}
  \inferrule* [lab=monoidal-laws] {} { P|Q \equiv Q|P \and P|0 \equiv P \and P|(Q|R) \equiv (P|Q)|R }
\end{mathpar}

\begin{mathpar}
  \inferrule* [lab=alpha-equivalence] {} { (x)P \equiv (y)P\{y/x\} \and y \not\in \freenames{P} }
\end{mathpar}

\begin{definition}
Then two processes, $P,Q$, are alpha-equivalent if $P = Q\{\vec{y}/\vec{x}\}$ for
some $\vec{x} \in \boundnames{Q},\vec{y} \in \boundnames{P}$, where $Q\{\vec{y}/\vec{x}\}$
denotes the capture-avoiding substitution of $\vec{y}$ for $\vec{x}$ in $Q$.
\end{definition}

\begin{definition}
  The {\em structural congruence} \cite{SangiorgiWalker} , $\equiv$,
  between processes is the least congruence containing
  alpha-equivalence, satisfying the abelian monoid laws
  (associativity, commutativity and $\pzero$ as identity) for parallel
  composition $|$ and for summation $+$.
\end{definition}

\subsection{Name equivalence}

We take name equivalence, written $\nameeq$, to be the smallest
equivalence relation generated by the following rules.

\begin{mathpar}
\inferrule*[lab=Quote-drop]
{ }
{ \quotep{@{x}} \nameeq x }

\inferrule*[lab=Struct-equiv]
{ P \scong Q }
{ \quotep{P} \nameeq \quotep{Q} }
\end{mathpar}

The astute reader will have noticed that the mutual recursion of names
and processes imposes a mutual recursion on alpha-equivalence and
structural equivalence via name-equivalence. Fortunately, all of this
works out pleasantly and we may calculate in the natural way, free of
concern. The reader interested in the details is referred to the
appendix \ref{appendix:rho_details}.

\subsection{Substitution}

We use $\Proc$ for the set of processes, $\QProc$ for the set of
names, and $\id{\{}\vec{y} / \vec{x} \id{\}}$ to denote partial maps,
$s : \QProc \rightarrow \QProc$. A map, $s$ lifts, uniquely, to a map
on process terms, $\widehat{s} : \Proc \rightarrow \Proc$ by the
following equations.

\begin{mathpar}
  (0) \psubstp{Q}{P} := 0 \\
  (R \juxtap S) \psubstp{Q}{P}
  :=    
  (R)\psubstp{Q}{P} \juxtap (S) \psubstp{Q}{P} \\
  (x?(y).R) \psubstp{Q}{P}    
  :=    
  (x)\substp{Q}{P} (z)\concat( (R \psubstn{z}{y}) \psubstp{Q}{P} ) \\
  (\lift{x}{R}) \psubstp{Q}{P}  
  :=
  \lift{(x)\substp{Q}{P}}{ R \psubstp{Q}{P} } \\
%   (\dropn{x})  \psubstp{Q}{P}       
%   := 
%   \left\{ 
%     \begin{array}{ccc} 
%       \dropn{\quotep{Q}} & & x \nameeq \quotep{P} \\
%       \dropn{x} & & otherwise \\
%     \end{array}
%   \right. 
  (\dropn{x})  \psubstp{Q}{P}       
  := 
  \left\{ 
    \begin{array}{ccc} 
      Q & & x \nameeq \quotep{P} \\
      \dropn{x} & & otherwise \\
    \end{array}
  \right.
\end{mathpar}
 

where

\begin{eqnarray}
  (x)\id{\{} \lpquote Q \rpquote / \lpquote P \rpquote \id{\}}            = 
  \left\{ 
    \begin{array}{ccc}
      \lpquote Q \rpquote & & x \nameeq \lpquote P \rpquote \\
      x & & otherwise \\
    \end{array}
  \right. \nonumber
\end{eqnarray}

and $z$ is chosen distinct from $\quotep{P}$, $\quotep{Q}$, the free
names in $Q$, and all the names in $R$. Our $\alpha$-equivalence will
be built in the standard way from this substitution.

\begin{remark}\label{rem:no_self_referential_names}
  One consequence of these definitions is that $\forall P. \quotep{P}
  \not\in \freenames{P}$.
\end{remark}

\subsection{ Dynamic quote: an example }

Anticipating something of what's to come, consider applying the
substitution, $\widehat{\id{\{}u / z \id{\}}}$, to the following pair
of processes, $\lift{w}{y!(z)}$ and $w[ \lpquote y!(z) \rpquote ]$.

\begin{eqnarray}
	\lift{w}{y!(z)}\widehat{\id{\{}u / z \id{\}}}
		& = &
		\lift{w}{y!(u)} \nonumber\\
	w[ \lpquote y!(z) \rpquote ] \widehat{ \id{\{}u / z \id{\}} }
		& = &
		w[ \lpquote y!(z) \rpquote ] \nonumber
\end{eqnarray}

Because the body of the process between quotes is impervious to
substitution, we get radically different answers. In fact, by
examining the first process in an input context,
e.g. $x?(z).\lift{w}{y!(z)}$, we see that the process under the lift
operator may be shaped by prefixed inputs binding a name inside it. In
this sense, the lift operator will be seen as a way to dynamically
construct processes before reifying them as names.

Finally equipped with these standard features we can present the
dynamics of the calculus.

\subsubsection{Operational semantics} 

Finally, we introduce the computational dynamics. What marks these
algebras as distinct from other more traditionally studied algebraic
structures, e.g. vector spaces or polynomial rings, is the manner in
which dynamics is captured. In traditional structures, dynamics is typically
expressed through morphisms between such structures, as in linear maps
between vector spaces or morphisms between rings. In algebras
associated with the semantics of computation, the dynamics is
expressed as part of the algebraic structure itself, through a
reduction reduction relation typically denoted by $\red$. Below, we
give a recursive presentation of this relation for the calculus used
in the encoding.

$\red \subseteq \pi \times \pi$
$\red : \pi \to \mathcal{P}(\pi)$

\begin{mathpar}
  \inferrule* [lab=Comm] { \textsf{match}( x_{src}, x_{trgt} ) } { x_{trgt}?(y)P \; | \; x_{src}!\langle {Q} \rangle \red P\{\quotep{Q}/y}\} }
  \and \\
  \inferrule* [lab=Par] {{P} \red {P}'} {{{P} | {Q}} \red {{P}' | {Q}}}
  \and
  \inferrule* [lab=Equiv]{{{P} \scong {P}'} \andalso {{P}' \red {Q}'} \andalso {{Q}' \scong {Q}}}{{P} \red {Q}}
\end{mathpar}

\begin{eqnarray*}
  match_{\equiv} (\quotep{P},\quotep{Q}) & := & P \equiv Q \\
  match_{\dagger}(\quotep{P},\quotep{Q}) & := & \forall R. P|Q \red^{*} R => R \red^{*} 0 \\
  match_{K}(\quotep{P},\quotep{Q}) & := & K \mbox{ for some context } K
\end{eqnarray*}

$u?(x)P | u!\langle Q \rangle \red P\{\quotep{Q}/x\}$

%We write $\wred$ for $\red^*$, and $P\red$ if $\exists Q $ such that $ P \red Q$.
We write $P\red$ if $\exists Q $ such that $ P \red Q$ and $P\not\red$, otherwise.

\section{Replication}

As mentioned before, it is known that replication (and hence
recursion) can be implemented in a higher-order process algebra
\cite{SangiorgiWalker}. As our first example of calculation with the
machinery thus far presented we give the construction explicitly in
the {\rhoc}.

\begin{eqnarray}
	D_{x} & := & \prefix{x}{y}{(\binpar{\outputp{x}{y}}{@{y}})} \nonumber\\
	\bangp_{x}{P} & := & \binpar{{x}!\langle{\binpar{D_{x}}{P}}\rangle}{D_{x}} \nonumber
\end{eqnarray}

\begin{eqnarray}
	\bangp_{x}{P} & & \nonumber\\
	=
	& {x}!\langle{(\prefix{x}{y}{(\outputp{x}{y} | @{y})) | P}}\rangle 
	      | \prefix{x}{y}{(\outputp{x}{y} | @{y})} & \nonumber\\
	\red
	& (\outputp{x}{y} | @{y})\substn{\quotep{(\prefix{x}{y}{(@{y} | \outputp{x}{y})) | P}}}{y} & \nonumber\\
	=
	& \outputp{x}{\quotep{(\prefix{x}{y}{(\outputp{x}{y} | @{y})) | P}}}
	  | {(\prefix{x}{y}{(\outputp{x}{y} | @{y})) | P}} & \nonumber\\
	\red
	& \ldots & \nonumber\\
	\red^*
	& P | P | \ldots & \nonumber
\end{eqnarray}

Of course, this encoding, as an implementation, runs away, unfolding
$\bangp{P}$ eagerly. A lazier and more implementable replication
operator, restricted to input-guarded processes, may be obtained as follows.

\begin{eqnarray}
\bangp{\prefix{u}{v}{P}} 
	:= 
	\binpar{\lift{x}{\prefix{u}{v}{(\binpar{D(x)}{P})}}}{D(x)} \nonumber
\end{eqnarray}

\begin{remark}
  Note that the lazier definition still does not deal with summation
  or mixed summation (i.e. sums over input and output). The reader is
  invited to construct definitions of replication that deal with these
  features. 

  Further, the definitions are parameterized in a name, $x$. Can you,
  gentle reader, make a definition that eliminates this parameter and
  guarantees no accidental interaction between the replication
  machinery and the process being replicated -- i.e. no accidental
  sharing of names used by the process to get its work done and the
  name(s) used by the replication to effect copying. This latter
  revision of the definition of replication is crucial to obtaining
  the expected identity $!!P \sim !P$.
\end{remark}

\begin{remark}\label{rem:paradoxical_combinator}
  The reader familiar with the lambda calculus will have noticed the
  similarity between $D$ and the paradoxical combinator.

  [Ed. note: the existence of this seems to suggest we have to be more
  restrictive on the set of processes and names we admit if we are to
  support no-cloning.]
\end{remark}

\subsubsection{Bisimulation}

The computational dynamics gives rise to another kind of equivalence,
the equivalence of computational behavior. As previously mentioned
this is typically captured \emph{via} some form of bisimulation.

% The notion we use in this paper is weak barbed bisimulation
% \cite{milner91polyadicpi}.

The notion we use in this paper is derived from weak barbed
bisimulation \cite{milner91polyadicpi}. 

\begin{definition}
An \emph{observation relation}, $\downarrow_{\mathcal N}$, over a set
of names, $\mathcal N$, is the smallest relation satisfying the rules
below.

\infrule[Out-barb]{y \in {\mathcal N}, \; x \nameeq y}
		  {\outputp{x}{v} \downarrow_{\mathcal N} x}
\infrule[Par-barb]{\mbox{$P\downarrow_{\mathcal N} x$ or $Q\downarrow_{\mathcal N} x$}}
		  {\binpar{P}{Q} \downarrow_{\mathcal N} x}

We write $P \Downarrow_{\mathcal N} x$ if there is $Q$ such that 
$P \wred Q$ and $Q \downarrow_{\mathcal N} x$.
\end{definition}

\begin{definition}
%\label{def.bbisim}
An  ${\mathcal N}$-\emph{barbed bisimulation} over a set of names, ${\mathcal N}$, is a symmetric binary relation 
${\mathcal S}_{\mathcal N}$ between agents such that $P\rel{S}_{\mathcal N}Q$ implies:
\begin{enumerate}
\item If $P \red P'$ then $Q \wred Q'$ and $P'\rel{S}_{\mathcal N} Q'$.
\item If $P\downarrow_{\mathcal N} x$, then $Q\Downarrow_{\mathcal N} x$.
\end{enumerate}
$P$ is ${\mathcal N}$-barbed bisimilar to $Q$, written
$P \wbbisim_{\mathcal N} Q$, if $P \rel{S}_{\mathcal N} Q$ for some ${\mathcal N}$-barbed bisimulation ${\mathcal S}_{\mathcal N}$.
\end{definition}

$\mathcal{R} \subseteq \pi \times \pi$

$P \mathcal{R} Q => \forall P'. P \red P' \Rightarrow \exists Q'. Q \red Q', P' \mathcal{R} Q'$

$P \vdash x \Rightarrow Q \vdash x$

\begin{mathpar}
  \inferrule*[lab=Out-barb]{x \nameeq y}{{y}!\langle{Q}\rangle \vdash x}
  \and
  \inferrule*[lab=Par-barb]{\mbox{$P\vdash x$ or $Q\vdash x$}}{\binpar{P}{Q} \vdash x}
\end{mathpar}

\subsubsection{Contexts}

One of the principle advantages of computational calculi like the
$\pi$-calculus is a well-defined notion of context,
contextual-equivalence and a correlation between
contextual-equivalence and notions of bisimulation. The notion of
context allows the decomposition of a process into (sub-)process and
its syntactic environment, its context. Thus, a context may be
thought of as a process with a ``hole'' (written $\Box$) in it. The
application of a context $M$ to a process $P$, written $M[P]$, is
tantamount to filling the hole in $M$ with $P$. In this paper we do
not need the full weight of this theory, but do make use of the notion
of context in the proof the main theorem. 

\begin{mathpar}
  \inferrule* [lab=summation] {} {{M_{M},M_{N}} \bc \Box \;|\; x.M_{A} \;|\; M_{M}+M_{N}}
  \and
  \inferrule* [lab=agent] {} {{M_{A}} \bc (\vec{x})M_{P} \;| \; \clift{P_0,\ldots,M_{P},\ldots,P_N}}
  \and \\
  \inferrule* [lab=process] {} {{M_{P}} \bc M_{N} \;| \;P|M_{P} }
\end{mathpar} 

\begin{mathpar}
  \inferrule* [lab=sychronization] {} {M_{N} \bc \Box \;|\; x?M_{F} \;|\; x!M_{C}}
  \and
  \inferrule* [lab=abstraction] {} {{M_{F}} \bc (x)M_{P} }
  \and
  \inferrule* [lab=concretion] {} {{M_{C}} \bc \langle M_{P} \rangle }
  \and \\
  \inferrule* [lab=process] {} {{M_{P}} \bc M_{N} \;| \;P|M_{P} }
\end{mathpar}

\begin{definition}[contextual application] Given a context $M$, and
  process $P$, we define the \emph{contextual application}, $M[P] :=
  M\{P/\Box\}$. That is, the contextual application of M to P is the
  substitution of $P$ for $\Box$ in $M$.
\end{definition}

$\meaningof{-} : L \to \mathcal{P}(\pi)$

\begin{mathpar}
  \inferrule* [lab=collection] {} {\meaningof{true} = \pi, \and \meaningof{~E} = \pi \setminus \meaningof{E}, \and \meaningof{E_{1} \& E_{2}} = \meaningof{E_{1}} \cap \meaningof{E_{2}}}
\end{mathpar}

\begin{mathpar}
  \inferrule* [lab=structure] {} {\meaningof{0} = \{ P \in \pi | P \equiv 0 \}, \and \\ \meaningof{E_1 | E_2} = \{ P \in \pi | P \equiv P_{1} | P_{2}, P_{1} \in \meaningof{E_{1}}, P_{2} \in \meaningof{E_2}\} }
\end{mathpar}

\begin{mathpar}
 \inferrule* [lab=behavior] {} {\meaningof{\langle a?b \rangle E} = \{ P \in \pi | P \equiv Q | u?(y)P', \\ \and \\\\ \and \\ \;\;\; u \in \meaningof{a}, \forall z.P'\{z/y\} \in \meaningof{E\{z/b\}}\}, \and \\ \meaningof{a!E} = \{ P \in \pi | P \equiv Q | x!\langle P' \rangle, x \in \meaningof{a} P' \in \meaningof{E}\} }
\end{mathpar}

\begin{mathpar}
 \inferrule* [lab=nominal] {} {\meaningof{\quotep{E}} = \{ \quotep{P} \in \quotep{\pi} | P \in \meaningof{E} \}, \and \meaningof{\quotep{P}} = \{ \quotep{Q} \in \quotep{\pi} | P \equiv Q \} \and \\ \meaningof{@\quotep{E}} = \{ P \in \pi | P \equiv @x, x \in \meaningof{E} \}}
\end{mathpar}

\begin{eqnarray*}
  \\
  \meaningof{-} : TS \to ST
\end{eqnarray*}

\begin{eqnarray*}
  \\
  L : TS \to ST
\end{eqnarray*}

\begin{eqnarray*}
  \\
  P \models E \iff P \in \meaningof{E}
\end{eqnarray*}

\begin{eqnarray*}
  P \approx_{L} Q \iff \forall E \in L. P \models E \iff Q \models E
\end{eqnarray*}

\begin{eqnarray*}
  P \approx_{K} Q
\end{eqnarray*}

\begin{eqnarray*}
  P \approx Q
\end{eqnarray*}

$\approx_{K} = \approx = \approx_{L}$

\subsubsection{Contextual duality}

Note that contexts extend the quotation operation to a family of
operations from processes to names. Given a context, $M$, we can
define a \emph{nominal context}, $\quotep{M}$ by $\quotep{M}[P] :=
\quotep{M[P]}$. To foreshadow what is to come we observe that these
operations enjoy a duality with processes very much like the duality
between vectors and maps from vectors to scalars.

Further, because the calculus is essentially higher-order, we have a
correspondence between contexts and processes. More specifically,
given a name $x$ and a context $M$ we can construct $M^{*}_{x}$ such
that 

\begin{mathpar}
  M^{*}_{x} | \lift{x}{P} \red M[P]
\end{mathpar}

namely,

\begin{mathpar}
  M^{*}_{x} := x?(u).M[\dropn{u}]
\end{mathpar}

The dependence of $M^{*}_{x}$ on a name makes it an abstraction, 

\begin{mathpar}
  M^{*} := (x)x?(u).M[\dropn{u}]
\end{mathpar}

\subsection{Additional notation}

It will sometimes be convenient to denote the process a name
quotes. We already have the notation $x = \quotep{P}$, but it will be
convenient to introduce an alternate notation, $\procn{x}$, when we
want to emphasize the connection to the use of the name. Note that, by
virtue of name equivalence, $\quotep{\procn{x}} \nameeq x$; so, the
notation is consistent with previous definitions.

Further, because names have structure it is possible to effect
substitutions on the basis of that structure. This means we need to
upgrade our notation for substitutions, which we accomplish by
adapting comprehension notation. Thus,

\begin{mathpar}
  P\{ y / x : x \in S \}
\end{mathpar}

is interpreted to mean the process derived from P by replacing (in a
capture-avoiding manner) each occurrence of $x$ in $S$ by $y$. For example,

\begin{mathpar}
  P\{ \quotep{\procn{x}|\procn{x}} / x : x \in \freenames{P} \}
\end{mathpar}

will replace each (occurrence) of a free name $x$ in $P$ by
$\quotep{\procn{x}|\procn{x}}$.

Also, we will avail ourselves of the notation $x^{L}$ and $x^{R}$ to
denote injections of a name into disjoint copies of the name
space. There are numerous ways to accomplish this. One example can be
found in \cite{MeredithR05}. This notation overloads to vectors of
names: $\vec{x}^{\pi} := (x_{i}^{\pi} \; : \; 0 \leq i < |\vec{x}| )$ where $\pi \in \{L,R\}$.

We also use $P^{\Box} := P|\Box$.

In \cite{MeredithR05} an interpretation of the new operator is
given. It turns out that there are several possible interpretations
all enjoying the requisite algebraic properties of the operator (see
\cite{milner91polyadicpi}). We will therefore make liberal use of
$(\nu\; \vec{x})P$.

% subsection the_syntax_and_semantics_of_the_notation_system (end)   

\input{qm2pi.qmops} 

\input{qm2pi.sterngerlach} 

\input{qm2pi.metric} 

% section concurrent_process_calculi (end)

%\input{qm2pi.proofsketch}

% section proof sketch (end)

%\input{qm2pi.slviaknots} 

% section spatial logic via knots (end)

\input{qm2pi.conclusion}

% section conclusion (end)

%\input{qm2pi.dtcodes} 

% section wiring algorithm (end)

\input{qm2pi.ack} 

% section acknowledgments (end)

\newpage


\bibliographystyle{plain}   
\bibliography{../../biblios/main.bib}

\input{qm2pi.rhodetails}

\end{document}

 

%\ifpdf
%\usepackage[pdftex]{graphicx}
%\else
%\usepackage{graphicx}
%\fi

 % \ifpdf
%  \usepackage{pdfsync}
%  \if


%\title{Brief Article}
%\author{David F. Snyder}
%\author{L.G. Meredith}

%\address{Dept. of Math., Texas State University--San Marcos, San Marcos, TX 78666}
       
\pagestyle{empty}


\begin{document}

\lstset{language=[Objective]Caml,frame=shadowbox}

\documentclass[12pt]{llncs}
%\documentclass{jktr}

\usepackage[pdftex]{hyperref}                   
\usepackage {listings}
\usepackage {mathpartir}
\usepackage{bcprules}
%\usepackage{listings}
                       
\usepackage{graphicx} 
%\usepackage[margins=2.5cm,nohead,nofoot]{geometry}
%\usepackage{geometry}
\usepackage{amsfonts}
\usepackage{amstext}
\usepackage{latexsym}
\usepackage{amssymb}
\usepackage{color}


%\include{myPreamble}
\include{qm2pi.local} 

%\ifpdf
%\usepackage[pdftex]{graphicx}
%\else
%\usepackage{graphicx}
%\fi

 % \ifpdf
%  \usepackage{pdfsync}
%  \if


%\title{Brief Article}
%\author{David F. Snyder}
%\author{L.G. Meredith}

%\address{Dept. of Math., Texas State University--San Marcos, San Marcos, TX 78666}
       
\pagestyle{empty}


\begin{document}

\lstset{language=[Objective]Caml,frame=shadowbox}

\input{qm2pi.front}

% section front matter (end)

\input{qm2pi.intro} 
 
% section introduction (end)

% \input{qm2pi.knotations} 

% section notation (end)

\input{qm2pi.process.calculi} 

% section concurrent_process_calculi_and_spatial_logics_ (end)
    
%\input{qm2pi.knots2pi} 

%\input{qm2pi.trefoil} 

%\input{qm2pi.mainthm} 

% subsection basic_interpretation (end)

%\input{qm2pi.rho.presentation} 
\subsection{The syntax and semantics of the notation system}\label{sub:the_syntax_and_semantics_of_the_notation_system} % (fold)

We now summarize a technical presentation of the calculus that
embodies our theory of dynamics. The typical presentation of such a
calculus follows the style of giving generators and relations on
them. The grammar, below, describing term constructors, freely
generates the set of processes, $\Proc$. This set is then quotiented
by a relation known as structural congruence and it is over this set
that the notion of dynamics is expressed. This presentation is
essentially that of \cite{MeredithR05} with the addition of
polyadicity and summation. For readability we have relegated some of
the technical subtleties to an appendix.

\subsubsection{Process grammar}\label{subsub:process_grammar}

\begin{mathpar}
  \inferrule* [lab=synchronization] {} {{M} \bc \pzero \;|\; x?F \;|\; x!C }
  \and
  \inferrule* [lab=abstraction] {} {{F} \bc (x)P}
  \and
  \inferrule* [lab=concretion] {} {{C} \bc \langle Q \rangle}
  \and
  \inferrule* [lab=process] {} {{P,Q} \bc M \;| \;P|Q \;|\; @{x}}
  \and
  \inferrule* [lab=name] {} {{x} \bc \quotep{P}}
\end{mathpar} 

Note that $\vec{x}$ (resp. $\vec{P}$) denotes a vector of names
(resp. processes) of length $|\vec{x}|$ (resp. $|\vec{P}|$). We adopt
the following useful abbreviations.

\begin{mathpar}
   x?(\vec{y}).P := x.(\vec{y})P \and  x\clift{\vec{P}} := x.\clift{\vec{P}}
   \and x!(y) := \lift{x}{\dropn{y}}
   \and \Pi_{i=0}^{n-1}P_i := P_0 | \ldots | P_{n-1}
\end{mathpar}

\subsubsection{Structural congruence}

\paragraph{Free and bound names and alpha-equivalence.} At the
core of structural equivalence is alpha-equivalence which identifies
process that are the same up to a change of variable. Formally, we
recognize the distinction between free and bound names. The free names
of a process, $\freenames{P}$, may be calculated recursively as
follows:

\begin{mathpar}
\freenames{\pzero} := \emptyset
  \and \\
  \freenames{x?(y).P} := \{ x \} \cup (\freenames{P} \setminus \{ y \})
  \and 
  \freenames{x!\langle P \rangle} := \{ x \} \cup \{ P \} 
  \and \\
  \freenames{P|Q} := \freenames{P} \cup \freenames{Q}
  \and \\
  \freenames{@{x}} := \{ x \}
\end{mathpar}

$\pi$
$\quotep{\pi}$

$\freenames{-} : \pi \to \mathcal{P}(\quotep{\pi})$

\begin{eqnarray*}
  \freenames{\pzero} & := & \emptyset \\
  \freenames{x?(y).P} & := & \{ x \} \cup (\freenames{P} \setminus \{ y \}) \\
  \freenames{x!\langle P \rangle} & := & \{ x \} \cup \{ P \} \\
  \freenames{P|Q} & := & \freenames{P} \cup \freenames{Q} \\
  \freenames{\dropn{x}} & := & \{ x \}
\end{eqnarray*}

The bound names of a process, $\boundnames{P}$, are those names occurring in $P$
that are not free. For example, in $x?(y).0$, the name $x$ is free, while $y$ is bound.

\begin{mathpar}
  \inferrule* [lab=monoidal-laws] {} { P|Q \equiv Q|P \and P|0 \equiv P \and P|(Q|R) \equiv (P|Q)|R }
\end{mathpar}

\begin{mathpar}
  \inferrule* [lab=alpha-equivalence] {} { (x)P \equiv (y)P\{y/x\} \and y \not\in \freenames{P} }
\end{mathpar}

\begin{definition}
Then two processes, $P,Q$, are alpha-equivalent if $P = Q\{\vec{y}/\vec{x}\}$ for
some $\vec{x} \in \boundnames{Q},\vec{y} \in \boundnames{P}$, where $Q\{\vec{y}/\vec{x}\}$
denotes the capture-avoiding substitution of $\vec{y}$ for $\vec{x}$ in $Q$.
\end{definition}

\begin{definition}
  The {\em structural congruence} \cite{SangiorgiWalker} , $\equiv$,
  between processes is the least congruence containing
  alpha-equivalence, satisfying the abelian monoid laws
  (associativity, commutativity and $\pzero$ as identity) for parallel
  composition $|$ and for summation $+$.
\end{definition}

\subsection{Name equivalence}

We take name equivalence, written $\nameeq$, to be the smallest
equivalence relation generated by the following rules.

\begin{mathpar}
\inferrule*[lab=Quote-drop]
{ }
{ \quotep{@{x}} \nameeq x }

\inferrule*[lab=Struct-equiv]
{ P \scong Q }
{ \quotep{P} \nameeq \quotep{Q} }
\end{mathpar}

The astute reader will have noticed that the mutual recursion of names
and processes imposes a mutual recursion on alpha-equivalence and
structural equivalence via name-equivalence. Fortunately, all of this
works out pleasantly and we may calculate in the natural way, free of
concern. The reader interested in the details is referred to the
appendix \ref{appendix:rho_details}.

\subsection{Substitution}

We use $\Proc$ for the set of processes, $\QProc$ for the set of
names, and $\id{\{}\vec{y} / \vec{x} \id{\}}$ to denote partial maps,
$s : \QProc \rightarrow \QProc$. A map, $s$ lifts, uniquely, to a map
on process terms, $\widehat{s} : \Proc \rightarrow \Proc$ by the
following equations.

\begin{mathpar}
  (0) \psubstp{Q}{P} := 0 \\
  (R \juxtap S) \psubstp{Q}{P}
  :=    
  (R)\psubstp{Q}{P} \juxtap (S) \psubstp{Q}{P} \\
  (x?(y).R) \psubstp{Q}{P}    
  :=    
  (x)\substp{Q}{P} (z)\concat( (R \psubstn{z}{y}) \psubstp{Q}{P} ) \\
  (\lift{x}{R}) \psubstp{Q}{P}  
  :=
  \lift{(x)\substp{Q}{P}}{ R \psubstp{Q}{P} } \\
%   (\dropn{x})  \psubstp{Q}{P}       
%   := 
%   \left\{ 
%     \begin{array}{ccc} 
%       \dropn{\quotep{Q}} & & x \nameeq \quotep{P} \\
%       \dropn{x} & & otherwise \\
%     \end{array}
%   \right. 
  (\dropn{x})  \psubstp{Q}{P}       
  := 
  \left\{ 
    \begin{array}{ccc} 
      Q & & x \nameeq \quotep{P} \\
      \dropn{x} & & otherwise \\
    \end{array}
  \right.
\end{mathpar}
 

where

\begin{eqnarray}
  (x)\id{\{} \lpquote Q \rpquote / \lpquote P \rpquote \id{\}}            = 
  \left\{ 
    \begin{array}{ccc}
      \lpquote Q \rpquote & & x \nameeq \lpquote P \rpquote \\
      x & & otherwise \\
    \end{array}
  \right. \nonumber
\end{eqnarray}

and $z$ is chosen distinct from $\quotep{P}$, $\quotep{Q}$, the free
names in $Q$, and all the names in $R$. Our $\alpha$-equivalence will
be built in the standard way from this substitution.

\begin{remark}\label{rem:no_self_referential_names}
  One consequence of these definitions is that $\forall P. \quotep{P}
  \not\in \freenames{P}$.
\end{remark}

\subsection{ Dynamic quote: an example }

Anticipating something of what's to come, consider applying the
substitution, $\widehat{\id{\{}u / z \id{\}}}$, to the following pair
of processes, $\lift{w}{y!(z)}$ and $w[ \lpquote y!(z) \rpquote ]$.

\begin{eqnarray}
	\lift{w}{y!(z)}\widehat{\id{\{}u / z \id{\}}}
		& = &
		\lift{w}{y!(u)} \nonumber\\
	w[ \lpquote y!(z) \rpquote ] \widehat{ \id{\{}u / z \id{\}} }
		& = &
		w[ \lpquote y!(z) \rpquote ] \nonumber
\end{eqnarray}

Because the body of the process between quotes is impervious to
substitution, we get radically different answers. In fact, by
examining the first process in an input context,
e.g. $x?(z).\lift{w}{y!(z)}$, we see that the process under the lift
operator may be shaped by prefixed inputs binding a name inside it. In
this sense, the lift operator will be seen as a way to dynamically
construct processes before reifying them as names.

Finally equipped with these standard features we can present the
dynamics of the calculus.

\subsubsection{Operational semantics} 

Finally, we introduce the computational dynamics. What marks these
algebras as distinct from other more traditionally studied algebraic
structures, e.g. vector spaces or polynomial rings, is the manner in
which dynamics is captured. In traditional structures, dynamics is typically
expressed through morphisms between such structures, as in linear maps
between vector spaces or morphisms between rings. In algebras
associated with the semantics of computation, the dynamics is
expressed as part of the algebraic structure itself, through a
reduction reduction relation typically denoted by $\red$. Below, we
give a recursive presentation of this relation for the calculus used
in the encoding.

$\red \subseteq \pi \times \pi$
$\red : \pi \to \mathcal{P}(\pi)$

\begin{mathpar}
  \inferrule* [lab=Comm] { \textsf{match}( x_{src}, x_{trgt} ) } { x_{trgt}?(y)P \; | \; x_{src}!\langle {Q} \rangle \red P\{\quotep{Q}/y}\} }
  \and \\
  \inferrule* [lab=Par] {{P} \red {P}'} {{{P} | {Q}} \red {{P}' | {Q}}}
  \and
  \inferrule* [lab=Equiv]{{{P} \scong {P}'} \andalso {{P}' \red {Q}'} \andalso {{Q}' \scong {Q}}}{{P} \red {Q}}
\end{mathpar}

\begin{eqnarray*}
  match_{\equiv} (\quotep{P},\quotep{Q}) & := & P \equiv Q \\
  match_{\dagger}(\quotep{P},\quotep{Q}) & := & \forall R. P|Q \red^{*} R => R \red^{*} 0 \\
  match_{K}(\quotep{P},\quotep{Q}) & := & K \mbox{ for some context } K
\end{eqnarray*}

$u?(x)P | u!\langle Q \rangle \red P\{\quotep{Q}/x\}$

%We write $\wred$ for $\red^*$, and $P\red$ if $\exists Q $ such that $ P \red Q$.
We write $P\red$ if $\exists Q $ such that $ P \red Q$ and $P\not\red$, otherwise.

\section{Replication}

As mentioned before, it is known that replication (and hence
recursion) can be implemented in a higher-order process algebra
\cite{SangiorgiWalker}. As our first example of calculation with the
machinery thus far presented we give the construction explicitly in
the {\rhoc}.

\begin{eqnarray}
	D_{x} & := & \prefix{x}{y}{(\binpar{\outputp{x}{y}}{@{y}})} \nonumber\\
	\bangp_{x}{P} & := & \binpar{{x}!\langle{\binpar{D_{x}}{P}}\rangle}{D_{x}} \nonumber
\end{eqnarray}

\begin{eqnarray}
	\bangp_{x}{P} & & \nonumber\\
	=
	& {x}!\langle{(\prefix{x}{y}{(\outputp{x}{y} | @{y})) | P}}\rangle 
	      | \prefix{x}{y}{(\outputp{x}{y} | @{y})} & \nonumber\\
	\red
	& (\outputp{x}{y} | @{y})\substn{\quotep{(\prefix{x}{y}{(@{y} | \outputp{x}{y})) | P}}}{y} & \nonumber\\
	=
	& \outputp{x}{\quotep{(\prefix{x}{y}{(\outputp{x}{y} | @{y})) | P}}}
	  | {(\prefix{x}{y}{(\outputp{x}{y} | @{y})) | P}} & \nonumber\\
	\red
	& \ldots & \nonumber\\
	\red^*
	& P | P | \ldots & \nonumber
\end{eqnarray}

Of course, this encoding, as an implementation, runs away, unfolding
$\bangp{P}$ eagerly. A lazier and more implementable replication
operator, restricted to input-guarded processes, may be obtained as follows.

\begin{eqnarray}
\bangp{\prefix{u}{v}{P}} 
	:= 
	\binpar{\lift{x}{\prefix{u}{v}{(\binpar{D(x)}{P})}}}{D(x)} \nonumber
\end{eqnarray}

\begin{remark}
  Note that the lazier definition still does not deal with summation
  or mixed summation (i.e. sums over input and output). The reader is
  invited to construct definitions of replication that deal with these
  features. 

  Further, the definitions are parameterized in a name, $x$. Can you,
  gentle reader, make a definition that eliminates this parameter and
  guarantees no accidental interaction between the replication
  machinery and the process being replicated -- i.e. no accidental
  sharing of names used by the process to get its work done and the
  name(s) used by the replication to effect copying. This latter
  revision of the definition of replication is crucial to obtaining
  the expected identity $!!P \sim !P$.
\end{remark}

\begin{remark}\label{rem:paradoxical_combinator}
  The reader familiar with the lambda calculus will have noticed the
  similarity between $D$ and the paradoxical combinator.

  [Ed. note: the existence of this seems to suggest we have to be more
  restrictive on the set of processes and names we admit if we are to
  support no-cloning.]
\end{remark}

\subsubsection{Bisimulation}

The computational dynamics gives rise to another kind of equivalence,
the equivalence of computational behavior. As previously mentioned
this is typically captured \emph{via} some form of bisimulation.

% The notion we use in this paper is weak barbed bisimulation
% \cite{milner91polyadicpi}.

The notion we use in this paper is derived from weak barbed
bisimulation \cite{milner91polyadicpi}. 

\begin{definition}
An \emph{observation relation}, $\downarrow_{\mathcal N}$, over a set
of names, $\mathcal N$, is the smallest relation satisfying the rules
below.

\infrule[Out-barb]{y \in {\mathcal N}, \; x \nameeq y}
		  {\outputp{x}{v} \downarrow_{\mathcal N} x}
\infrule[Par-barb]{\mbox{$P\downarrow_{\mathcal N} x$ or $Q\downarrow_{\mathcal N} x$}}
		  {\binpar{P}{Q} \downarrow_{\mathcal N} x}

We write $P \Downarrow_{\mathcal N} x$ if there is $Q$ such that 
$P \wred Q$ and $Q \downarrow_{\mathcal N} x$.
\end{definition}

\begin{definition}
%\label{def.bbisim}
An  ${\mathcal N}$-\emph{barbed bisimulation} over a set of names, ${\mathcal N}$, is a symmetric binary relation 
${\mathcal S}_{\mathcal N}$ between agents such that $P\rel{S}_{\mathcal N}Q$ implies:
\begin{enumerate}
\item If $P \red P'$ then $Q \wred Q'$ and $P'\rel{S}_{\mathcal N} Q'$.
\item If $P\downarrow_{\mathcal N} x$, then $Q\Downarrow_{\mathcal N} x$.
\end{enumerate}
$P$ is ${\mathcal N}$-barbed bisimilar to $Q$, written
$P \wbbisim_{\mathcal N} Q$, if $P \rel{S}_{\mathcal N} Q$ for some ${\mathcal N}$-barbed bisimulation ${\mathcal S}_{\mathcal N}$.
\end{definition}

$\mathcal{R} \subseteq \pi \times \pi$

$P \mathcal{R} Q => \forall P'. P \red P' \Rightarrow \exists Q'. Q \red Q', P' \mathcal{R} Q'$

$P \vdash x \Rightarrow Q \vdash x$

\begin{mathpar}
  \inferrule*[lab=Out-barb]{x \nameeq y}{{y}!\langle{Q}\rangle \vdash x}
  \and
  \inferrule*[lab=Par-barb]{\mbox{$P\vdash x$ or $Q\vdash x$}}{\binpar{P}{Q} \vdash x}
\end{mathpar}

\subsubsection{Contexts}

One of the principle advantages of computational calculi like the
$\pi$-calculus is a well-defined notion of context,
contextual-equivalence and a correlation between
contextual-equivalence and notions of bisimulation. The notion of
context allows the decomposition of a process into (sub-)process and
its syntactic environment, its context. Thus, a context may be
thought of as a process with a ``hole'' (written $\Box$) in it. The
application of a context $M$ to a process $P$, written $M[P]$, is
tantamount to filling the hole in $M$ with $P$. In this paper we do
not need the full weight of this theory, but do make use of the notion
of context in the proof the main theorem. 

\begin{mathpar}
  \inferrule* [lab=summation] {} {{M_{M},M_{N}} \bc \Box \;|\; x.M_{A} \;|\; M_{M}+M_{N}}
  \and
  \inferrule* [lab=agent] {} {{M_{A}} \bc (\vec{x})M_{P} \;| \; \clift{P_0,\ldots,M_{P},\ldots,P_N}}
  \and \\
  \inferrule* [lab=process] {} {{M_{P}} \bc M_{N} \;| \;P|M_{P} }
\end{mathpar} 

\begin{mathpar}
  \inferrule* [lab=sychronization] {} {M_{N} \bc \Box \;|\; x?M_{F} \;|\; x!M_{C}}
  \and
  \inferrule* [lab=abstraction] {} {{M_{F}} \bc (x)M_{P} }
  \and
  \inferrule* [lab=concretion] {} {{M_{C}} \bc \langle M_{P} \rangle }
  \and \\
  \inferrule* [lab=process] {} {{M_{P}} \bc M_{N} \;| \;P|M_{P} }
\end{mathpar}

\begin{definition}[contextual application] Given a context $M$, and
  process $P$, we define the \emph{contextual application}, $M[P] :=
  M\{P/\Box\}$. That is, the contextual application of M to P is the
  substitution of $P$ for $\Box$ in $M$.
\end{definition}

$\meaningof{-} : L \to \mathcal{P}(\pi)$

\begin{mathpar}
  \inferrule* [lab=collection] {} {\meaningof{true} = \pi, \and \meaningof{~E} = \pi \setminus \meaningof{E}, \and \meaningof{E_{1} \& E_{2}} = \meaningof{E_{1}} \cap \meaningof{E_{2}}}
\end{mathpar}

\begin{mathpar}
  \inferrule* [lab=structure] {} {\meaningof{0} = \{ P \in \pi | P \equiv 0 \}, \and \\ \meaningof{E_1 | E_2} = \{ P \in \pi | P \equiv P_{1} | P_{2}, P_{1} \in \meaningof{E_{1}}, P_{2} \in \meaningof{E_2}\} }
\end{mathpar}

\begin{mathpar}
 \inferrule* [lab=behavior] {} {\meaningof{\langle a?b \rangle E} = \{ P \in \pi | P \equiv Q | u?(y)P', \\ \and \\\\ \and \\ \;\;\; u \in \meaningof{a}, \forall z.P'\{z/y\} \in \meaningof{E\{z/b\}}\}, \and \\ \meaningof{a!E} = \{ P \in \pi | P \equiv Q | x!\langle P' \rangle, x \in \meaningof{a} P' \in \meaningof{E}\} }
\end{mathpar}

\begin{mathpar}
 \inferrule* [lab=nominal] {} {\meaningof{\quotep{E}} = \{ \quotep{P} \in \quotep{\pi} | P \in \meaningof{E} \}, \and \meaningof{\quotep{P}} = \{ \quotep{Q} \in \quotep{\pi} | P \equiv Q \} \and \\ \meaningof{@\quotep{E}} = \{ P \in \pi | P \equiv @x, x \in \meaningof{E} \}}
\end{mathpar}

\begin{eqnarray*}
  \\
  \meaningof{-} : TS \to ST
\end{eqnarray*}

\begin{eqnarray*}
  \\
  L : TS \to ST
\end{eqnarray*}

\begin{eqnarray*}
  \\
  P \models E \iff P \in \meaningof{E}
\end{eqnarray*}

\begin{eqnarray*}
  P \approx_{L} Q \iff \forall E \in L. P \models E \iff Q \models E
\end{eqnarray*}

\begin{eqnarray*}
  P \approx_{K} Q
\end{eqnarray*}

\begin{eqnarray*}
  P \approx Q
\end{eqnarray*}

$\approx_{K} = \approx = \approx_{L}$

\subsubsection{Contextual duality}

Note that contexts extend the quotation operation to a family of
operations from processes to names. Given a context, $M$, we can
define a \emph{nominal context}, $\quotep{M}$ by $\quotep{M}[P] :=
\quotep{M[P]}$. To foreshadow what is to come we observe that these
operations enjoy a duality with processes very much like the duality
between vectors and maps from vectors to scalars.

Further, because the calculus is essentially higher-order, we have a
correspondence between contexts and processes. More specifically,
given a name $x$ and a context $M$ we can construct $M^{*}_{x}$ such
that 

\begin{mathpar}
  M^{*}_{x} | \lift{x}{P} \red M[P]
\end{mathpar}

namely,

\begin{mathpar}
  M^{*}_{x} := x?(u).M[\dropn{u}]
\end{mathpar}

The dependence of $M^{*}_{x}$ on a name makes it an abstraction, 

\begin{mathpar}
  M^{*} := (x)x?(u).M[\dropn{u}]
\end{mathpar}

\subsection{Additional notation}

It will sometimes be convenient to denote the process a name
quotes. We already have the notation $x = \quotep{P}$, but it will be
convenient to introduce an alternate notation, $\procn{x}$, when we
want to emphasize the connection to the use of the name. Note that, by
virtue of name equivalence, $\quotep{\procn{x}} \nameeq x$; so, the
notation is consistent with previous definitions.

Further, because names have structure it is possible to effect
substitutions on the basis of that structure. This means we need to
upgrade our notation for substitutions, which we accomplish by
adapting comprehension notation. Thus,

\begin{mathpar}
  P\{ y / x : x \in S \}
\end{mathpar}

is interpreted to mean the process derived from P by replacing (in a
capture-avoiding manner) each occurrence of $x$ in $S$ by $y$. For example,

\begin{mathpar}
  P\{ \quotep{\procn{x}|\procn{x}} / x : x \in \freenames{P} \}
\end{mathpar}

will replace each (occurrence) of a free name $x$ in $P$ by
$\quotep{\procn{x}|\procn{x}}$.

Also, we will avail ourselves of the notation $x^{L}$ and $x^{R}$ to
denote injections of a name into disjoint copies of the name
space. There are numerous ways to accomplish this. One example can be
found in \cite{MeredithR05}. This notation overloads to vectors of
names: $\vec{x}^{\pi} := (x_{i}^{\pi} \; : \; 0 \leq i < |\vec{x}| )$ where $\pi \in \{L,R\}$.

We also use $P^{\Box} := P|\Box$.

In \cite{MeredithR05} an interpretation of the new operator is
given. It turns out that there are several possible interpretations
all enjoying the requisite algebraic properties of the operator (see
\cite{milner91polyadicpi}). We will therefore make liberal use of
$(\nu\; \vec{x})P$.

% subsection the_syntax_and_semantics_of_the_notation_system (end)   

\input{qm2pi.qmops} 

\input{qm2pi.sterngerlach} 

\input{qm2pi.metric} 

% section concurrent_process_calculi (end)

%\input{qm2pi.proofsketch}

% section proof sketch (end)

%\input{qm2pi.slviaknots} 

% section spatial logic via knots (end)

\input{qm2pi.conclusion}

% section conclusion (end)

%\input{qm2pi.dtcodes} 

% section wiring algorithm (end)

\input{qm2pi.ack} 

% section acknowledgments (end)

\newpage


\bibliographystyle{plain}   
\bibliography{../../biblios/main.bib}

\input{qm2pi.rhodetails}

\end{document}



% section front matter (end)

\section{Introduction}\label{sec:introduction} % (fold)
In this draft of the material i am going to have to dispense with the
usual writing conventions adopted in papers on these topics. i'm going
to have adopt whatever tone i need at the time i'm writing up the
calculations. Sometimes this may be very conversational; others it may
be the barest mathematical grunts; others still it may be that i have
lifted text from one of my other papers because the exposition of some
point was better said there. i hope that my readers are not unduly put
out by this decision. i'm not doing this to flout convention or be
rebellious. i find these calculations very technically challenging. To
keep everything going technically, something has to give; i have to
let go of some cognitive burden. So, the academic writing style --
with all of its trade-offs in terms of facilitating technical
communication -- is what i'm letting go of. Perhaps subsequent drafts
can be tightened and polished, but for now, i'm going to speak as if
we were sitting together in a coffee shop with a laptop, wifi and a
pad of paper and a pencil.

So, here's what i have to say. We -- you and i, comfortably ensconced
in our coffee shop and well-equipped with our tools -- can realize and
carry out the calculations of quantum mechanics over a very different
formal theory of dynamics, a formal theory of dynamics that
corresponds to a theory of concurrent computation with
\emph{reflection}. It has the advantage that the underlying theory is
already `quantized', but supports analogues all of the continuuous
operations. Strikingly, this underlying theory has recently been
connected with a notion of metric that we can show, by calculating
together, coincides with the metric induced by the inner product.

There are a lot of reasons why you might be interested in seeing
calculations of this form. Here's why i'm interested. For the past
several centuries there has been no competitor to the ``Newtonian''
account of dynamics. As a result the predominant share of accounts of
dynamical systems and situations have had to be formulated in terms of
the Newtonian machinery. i view this as an intellectually dangerous
position to occupy. Everything, despite it's intrinsic shape, turns
into a nail to be hit with this hammer. Recently, however, the theory
of computation has matured to the point where we have candidates for
theories of dynamics that offer very different perspective on
reasoning about dynamical systems and situations. Testing these
candidates against very successful accounts of dynamical situations,
like quantum mechanics, is going to give us some sense of how mature
they are and some measure of the quality of these accounts of
dynamics.

\subsection{Summary of contributions and outline of paper}

So, we're going to develop an interpretation of the operations of
quantum mechanics normally interpreted by Hilbert spaces and
operators. We're going to do this over a theory of computation. Note
that this is very different than the usual quantum computation program
which develops notions of computation over quantum mechanics. Rather,
we are developing a story that aligns with Wheeler's slogan: It from
Bit. To do this we will first provide an account of the theory of
computation at play here. Then we will dive into a calculation-driven
interpretation of the operations of quantum mechanics.

The reason we take this approach is that -- until very recently --
there hasn't been an axiomatic account of quantum mechanics. As a
result there has been no sharp delineation of the mathematical theory
supporting interpretation of the physical theory and the physical
theory, itself. So, ambient features of the maths are free to be
exploited (or supressed) without a real accounting of their physical
relevance. There is no sharp statement ``here's the physical theory''
qua \emph{theory} and ``here's the mathematical interpretation''
enabling a judgment of how faithful the interpretation is -- apart
from experimental observation. When there is an axiomatic account we
can judge how well a given mathematical formalism supports an
interpretation of the axioms, independent of
experimentation. Likewise, we can judge how well we have captured our
physical evidence and experience with our axiomatics, independent of
any specific mathematical implementation, with accidental detail that
may or may not have physical significance. 

In lieu of a fully fleshed out and vetted axiomatic account of quantum
mechanics, interpreting the operational notions in service of modeling
physical systems will have to suffice. In other words, we are not in
the business of providing a model of Hilbert spaces and operators. We
are in the business of providing a model of quantum mechanics because
we are motivated by testing our notions of dynamics against physical
theory; and, the predictive calculations of the physical theory must
serve as the best formulation -- shy of a fully fleshed out axiomatic
account -- of the physical theory itself (as they have for scientific
theories since time immemorial). Put another way, despite a
whole-hearted commitment to an It-from-Bit ontology, we are firmly
aligned with the shut-up-and-calculate camp as the best way to obtain
results either from the physical perspective or as a quality assurance
measure of our fledgling theory of dynamics.

In detail, we present a reflective process calculus. Then we develop
intuitive correspondences between the notions available in this
calculus and the usual physical notions supporting quantum mechanical
calculations. Thus, 

\begin{table}[htp]
  \center{
    \fbox{
      \begin{tabular}{c|c}
        quantum mechanics & process calculus \\
        \hline
        scalar & name \\
        state vector & process \\
        dual & contextual duals \\
        matrix & formal sums of process-context-dual pairs \\
        orthogonality & process annihilation \\
        inner product & execution-formula + quoting
      \end{tabular}
    }
  }
  \caption{QM - process calculi correspondences}
\end{table}

Then we tighten up these intuitions to operational definitions. We
employ the Dirac notation as the best proxy we can find for an
abstract syntax of the quantum mechanical notions. The definitions we
develop put us in contact with equational constraints coming from the
theory that we demonstrate the definitions and calculations satisfy.

This puts us in a position to shut up and calculate for the
Stern-Gerlach experimental set up, showing how these predictive
calculations become calculations on processes in our theory of a
reflective process calculus.

Penultimately, we demonstrate that the notion of metric coming from
the inner product coincides with the notion of metric available from
the theory of bisimulation. This demonstration gives us the right to
think of space as arising from behavior. Finally, we consider where we
might go from the new vantage point we have obtained.

% section introduction (end) 
 
% section introduction (end)

% \documentclass[12pt]{llncs}
%\documentclass{jktr}

\usepackage[pdftex]{hyperref}                   
\usepackage {listings}
\usepackage {mathpartir}
\usepackage{bcprules}
%\usepackage{listings}
                       
\usepackage{graphicx} 
%\usepackage[margins=2.5cm,nohead,nofoot]{geometry}
%\usepackage{geometry}
\usepackage{amsfonts}
\usepackage{amstext}
\usepackage{latexsym}
\usepackage{amssymb}
\usepackage{color}


%\include{myPreamble}
\include{qm2pi.local} 

%\ifpdf
%\usepackage[pdftex]{graphicx}
%\else
%\usepackage{graphicx}
%\fi

 % \ifpdf
%  \usepackage{pdfsync}
%  \if


%\title{Brief Article}
%\author{David F. Snyder}
%\author{L.G. Meredith}

%\address{Dept. of Math., Texas State University--San Marcos, San Marcos, TX 78666}
       
\pagestyle{empty}


\begin{document}

\lstset{language=[Objective]Caml,frame=shadowbox}

\input{qm2pi.front}

% section front matter (end)

\input{qm2pi.intro} 
 
% section introduction (end)

% \input{qm2pi.knotations} 

% section notation (end)

\input{qm2pi.process.calculi} 

% section concurrent_process_calculi_and_spatial_logics_ (end)
    
%\input{qm2pi.knots2pi} 

%\input{qm2pi.trefoil} 

%\input{qm2pi.mainthm} 

% subsection basic_interpretation (end)

%\input{qm2pi.rho.presentation} 
\subsection{The syntax and semantics of the notation system}\label{sub:the_syntax_and_semantics_of_the_notation_system} % (fold)

We now summarize a technical presentation of the calculus that
embodies our theory of dynamics. The typical presentation of such a
calculus follows the style of giving generators and relations on
them. The grammar, below, describing term constructors, freely
generates the set of processes, $\Proc$. This set is then quotiented
by a relation known as structural congruence and it is over this set
that the notion of dynamics is expressed. This presentation is
essentially that of \cite{MeredithR05} with the addition of
polyadicity and summation. For readability we have relegated some of
the technical subtleties to an appendix.

\subsubsection{Process grammar}\label{subsub:process_grammar}

\begin{mathpar}
  \inferrule* [lab=synchronization] {} {{M} \bc \pzero \;|\; x?F \;|\; x!C }
  \and
  \inferrule* [lab=abstraction] {} {{F} \bc (x)P}
  \and
  \inferrule* [lab=concretion] {} {{C} \bc \langle Q \rangle}
  \and
  \inferrule* [lab=process] {} {{P,Q} \bc M \;| \;P|Q \;|\; @{x}}
  \and
  \inferrule* [lab=name] {} {{x} \bc \quotep{P}}
\end{mathpar} 

Note that $\vec{x}$ (resp. $\vec{P}$) denotes a vector of names
(resp. processes) of length $|\vec{x}|$ (resp. $|\vec{P}|$). We adopt
the following useful abbreviations.

\begin{mathpar}
   x?(\vec{y}).P := x.(\vec{y})P \and  x\clift{\vec{P}} := x.\clift{\vec{P}}
   \and x!(y) := \lift{x}{\dropn{y}}
   \and \Pi_{i=0}^{n-1}P_i := P_0 | \ldots | P_{n-1}
\end{mathpar}

\subsubsection{Structural congruence}

\paragraph{Free and bound names and alpha-equivalence.} At the
core of structural equivalence is alpha-equivalence which identifies
process that are the same up to a change of variable. Formally, we
recognize the distinction between free and bound names. The free names
of a process, $\freenames{P}$, may be calculated recursively as
follows:

\begin{mathpar}
\freenames{\pzero} := \emptyset
  \and \\
  \freenames{x?(y).P} := \{ x \} \cup (\freenames{P} \setminus \{ y \})
  \and 
  \freenames{x!\langle P \rangle} := \{ x \} \cup \{ P \} 
  \and \\
  \freenames{P|Q} := \freenames{P} \cup \freenames{Q}
  \and \\
  \freenames{@{x}} := \{ x \}
\end{mathpar}

$\pi$
$\quotep{\pi}$

$\freenames{-} : \pi \to \mathcal{P}(\quotep{\pi})$

\begin{eqnarray*}
  \freenames{\pzero} & := & \emptyset \\
  \freenames{x?(y).P} & := & \{ x \} \cup (\freenames{P} \setminus \{ y \}) \\
  \freenames{x!\langle P \rangle} & := & \{ x \} \cup \{ P \} \\
  \freenames{P|Q} & := & \freenames{P} \cup \freenames{Q} \\
  \freenames{\dropn{x}} & := & \{ x \}
\end{eqnarray*}

The bound names of a process, $\boundnames{P}$, are those names occurring in $P$
that are not free. For example, in $x?(y).0$, the name $x$ is free, while $y$ is bound.

\begin{mathpar}
  \inferrule* [lab=monoidal-laws] {} { P|Q \equiv Q|P \and P|0 \equiv P \and P|(Q|R) \equiv (P|Q)|R }
\end{mathpar}

\begin{mathpar}
  \inferrule* [lab=alpha-equivalence] {} { (x)P \equiv (y)P\{y/x\} \and y \not\in \freenames{P} }
\end{mathpar}

\begin{definition}
Then two processes, $P,Q$, are alpha-equivalent if $P = Q\{\vec{y}/\vec{x}\}$ for
some $\vec{x} \in \boundnames{Q},\vec{y} \in \boundnames{P}$, where $Q\{\vec{y}/\vec{x}\}$
denotes the capture-avoiding substitution of $\vec{y}$ for $\vec{x}$ in $Q$.
\end{definition}

\begin{definition}
  The {\em structural congruence} \cite{SangiorgiWalker} , $\equiv$,
  between processes is the least congruence containing
  alpha-equivalence, satisfying the abelian monoid laws
  (associativity, commutativity and $\pzero$ as identity) for parallel
  composition $|$ and for summation $+$.
\end{definition}

\subsection{Name equivalence}

We take name equivalence, written $\nameeq$, to be the smallest
equivalence relation generated by the following rules.

\begin{mathpar}
\inferrule*[lab=Quote-drop]
{ }
{ \quotep{@{x}} \nameeq x }

\inferrule*[lab=Struct-equiv]
{ P \scong Q }
{ \quotep{P} \nameeq \quotep{Q} }
\end{mathpar}

The astute reader will have noticed that the mutual recursion of names
and processes imposes a mutual recursion on alpha-equivalence and
structural equivalence via name-equivalence. Fortunately, all of this
works out pleasantly and we may calculate in the natural way, free of
concern. The reader interested in the details is referred to the
appendix \ref{appendix:rho_details}.

\subsection{Substitution}

We use $\Proc$ for the set of processes, $\QProc$ for the set of
names, and $\id{\{}\vec{y} / \vec{x} \id{\}}$ to denote partial maps,
$s : \QProc \rightarrow \QProc$. A map, $s$ lifts, uniquely, to a map
on process terms, $\widehat{s} : \Proc \rightarrow \Proc$ by the
following equations.

\begin{mathpar}
  (0) \psubstp{Q}{P} := 0 \\
  (R \juxtap S) \psubstp{Q}{P}
  :=    
  (R)\psubstp{Q}{P} \juxtap (S) \psubstp{Q}{P} \\
  (x?(y).R) \psubstp{Q}{P}    
  :=    
  (x)\substp{Q}{P} (z)\concat( (R \psubstn{z}{y}) \psubstp{Q}{P} ) \\
  (\lift{x}{R}) \psubstp{Q}{P}  
  :=
  \lift{(x)\substp{Q}{P}}{ R \psubstp{Q}{P} } \\
%   (\dropn{x})  \psubstp{Q}{P}       
%   := 
%   \left\{ 
%     \begin{array}{ccc} 
%       \dropn{\quotep{Q}} & & x \nameeq \quotep{P} \\
%       \dropn{x} & & otherwise \\
%     \end{array}
%   \right. 
  (\dropn{x})  \psubstp{Q}{P}       
  := 
  \left\{ 
    \begin{array}{ccc} 
      Q & & x \nameeq \quotep{P} \\
      \dropn{x} & & otherwise \\
    \end{array}
  \right.
\end{mathpar}
 

where

\begin{eqnarray}
  (x)\id{\{} \lpquote Q \rpquote / \lpquote P \rpquote \id{\}}            = 
  \left\{ 
    \begin{array}{ccc}
      \lpquote Q \rpquote & & x \nameeq \lpquote P \rpquote \\
      x & & otherwise \\
    \end{array}
  \right. \nonumber
\end{eqnarray}

and $z$ is chosen distinct from $\quotep{P}$, $\quotep{Q}$, the free
names in $Q$, and all the names in $R$. Our $\alpha$-equivalence will
be built in the standard way from this substitution.

\begin{remark}\label{rem:no_self_referential_names}
  One consequence of these definitions is that $\forall P. \quotep{P}
  \not\in \freenames{P}$.
\end{remark}

\subsection{ Dynamic quote: an example }

Anticipating something of what's to come, consider applying the
substitution, $\widehat{\id{\{}u / z \id{\}}}$, to the following pair
of processes, $\lift{w}{y!(z)}$ and $w[ \lpquote y!(z) \rpquote ]$.

\begin{eqnarray}
	\lift{w}{y!(z)}\widehat{\id{\{}u / z \id{\}}}
		& = &
		\lift{w}{y!(u)} \nonumber\\
	w[ \lpquote y!(z) \rpquote ] \widehat{ \id{\{}u / z \id{\}} }
		& = &
		w[ \lpquote y!(z) \rpquote ] \nonumber
\end{eqnarray}

Because the body of the process between quotes is impervious to
substitution, we get radically different answers. In fact, by
examining the first process in an input context,
e.g. $x?(z).\lift{w}{y!(z)}$, we see that the process under the lift
operator may be shaped by prefixed inputs binding a name inside it. In
this sense, the lift operator will be seen as a way to dynamically
construct processes before reifying them as names.

Finally equipped with these standard features we can present the
dynamics of the calculus.

\subsubsection{Operational semantics} 

Finally, we introduce the computational dynamics. What marks these
algebras as distinct from other more traditionally studied algebraic
structures, e.g. vector spaces or polynomial rings, is the manner in
which dynamics is captured. In traditional structures, dynamics is typically
expressed through morphisms between such structures, as in linear maps
between vector spaces or morphisms between rings. In algebras
associated with the semantics of computation, the dynamics is
expressed as part of the algebraic structure itself, through a
reduction reduction relation typically denoted by $\red$. Below, we
give a recursive presentation of this relation for the calculus used
in the encoding.

$\red \subseteq \pi \times \pi$
$\red : \pi \to \mathcal{P}(\pi)$

\begin{mathpar}
  \inferrule* [lab=Comm] { \textsf{match}( x_{src}, x_{trgt} ) } { x_{trgt}?(y)P \; | \; x_{src}!\langle {Q} \rangle \red P\{\quotep{Q}/y}\} }
  \and \\
  \inferrule* [lab=Par] {{P} \red {P}'} {{{P} | {Q}} \red {{P}' | {Q}}}
  \and
  \inferrule* [lab=Equiv]{{{P} \scong {P}'} \andalso {{P}' \red {Q}'} \andalso {{Q}' \scong {Q}}}{{P} \red {Q}}
\end{mathpar}

\begin{eqnarray*}
  match_{\equiv} (\quotep{P},\quotep{Q}) & := & P \equiv Q \\
  match_{\dagger}(\quotep{P},\quotep{Q}) & := & \forall R. P|Q \red^{*} R => R \red^{*} 0 \\
  match_{K}(\quotep{P},\quotep{Q}) & := & K \mbox{ for some context } K
\end{eqnarray*}

$u?(x)P | u!\langle Q \rangle \red P\{\quotep{Q}/x\}$

%We write $\wred$ for $\red^*$, and $P\red$ if $\exists Q $ such that $ P \red Q$.
We write $P\red$ if $\exists Q $ such that $ P \red Q$ and $P\not\red$, otherwise.

\section{Replication}

As mentioned before, it is known that replication (and hence
recursion) can be implemented in a higher-order process algebra
\cite{SangiorgiWalker}. As our first example of calculation with the
machinery thus far presented we give the construction explicitly in
the {\rhoc}.

\begin{eqnarray}
	D_{x} & := & \prefix{x}{y}{(\binpar{\outputp{x}{y}}{@{y}})} \nonumber\\
	\bangp_{x}{P} & := & \binpar{{x}!\langle{\binpar{D_{x}}{P}}\rangle}{D_{x}} \nonumber
\end{eqnarray}

\begin{eqnarray}
	\bangp_{x}{P} & & \nonumber\\
	=
	& {x}!\langle{(\prefix{x}{y}{(\outputp{x}{y} | @{y})) | P}}\rangle 
	      | \prefix{x}{y}{(\outputp{x}{y} | @{y})} & \nonumber\\
	\red
	& (\outputp{x}{y} | @{y})\substn{\quotep{(\prefix{x}{y}{(@{y} | \outputp{x}{y})) | P}}}{y} & \nonumber\\
	=
	& \outputp{x}{\quotep{(\prefix{x}{y}{(\outputp{x}{y} | @{y})) | P}}}
	  | {(\prefix{x}{y}{(\outputp{x}{y} | @{y})) | P}} & \nonumber\\
	\red
	& \ldots & \nonumber\\
	\red^*
	& P | P | \ldots & \nonumber
\end{eqnarray}

Of course, this encoding, as an implementation, runs away, unfolding
$\bangp{P}$ eagerly. A lazier and more implementable replication
operator, restricted to input-guarded processes, may be obtained as follows.

\begin{eqnarray}
\bangp{\prefix{u}{v}{P}} 
	:= 
	\binpar{\lift{x}{\prefix{u}{v}{(\binpar{D(x)}{P})}}}{D(x)} \nonumber
\end{eqnarray}

\begin{remark}
  Note that the lazier definition still does not deal with summation
  or mixed summation (i.e. sums over input and output). The reader is
  invited to construct definitions of replication that deal with these
  features. 

  Further, the definitions are parameterized in a name, $x$. Can you,
  gentle reader, make a definition that eliminates this parameter and
  guarantees no accidental interaction between the replication
  machinery and the process being replicated -- i.e. no accidental
  sharing of names used by the process to get its work done and the
  name(s) used by the replication to effect copying. This latter
  revision of the definition of replication is crucial to obtaining
  the expected identity $!!P \sim !P$.
\end{remark}

\begin{remark}\label{rem:paradoxical_combinator}
  The reader familiar with the lambda calculus will have noticed the
  similarity between $D$ and the paradoxical combinator.

  [Ed. note: the existence of this seems to suggest we have to be more
  restrictive on the set of processes and names we admit if we are to
  support no-cloning.]
\end{remark}

\subsubsection{Bisimulation}

The computational dynamics gives rise to another kind of equivalence,
the equivalence of computational behavior. As previously mentioned
this is typically captured \emph{via} some form of bisimulation.

% The notion we use in this paper is weak barbed bisimulation
% \cite{milner91polyadicpi}.

The notion we use in this paper is derived from weak barbed
bisimulation \cite{milner91polyadicpi}. 

\begin{definition}
An \emph{observation relation}, $\downarrow_{\mathcal N}$, over a set
of names, $\mathcal N$, is the smallest relation satisfying the rules
below.

\infrule[Out-barb]{y \in {\mathcal N}, \; x \nameeq y}
		  {\outputp{x}{v} \downarrow_{\mathcal N} x}
\infrule[Par-barb]{\mbox{$P\downarrow_{\mathcal N} x$ or $Q\downarrow_{\mathcal N} x$}}
		  {\binpar{P}{Q} \downarrow_{\mathcal N} x}

We write $P \Downarrow_{\mathcal N} x$ if there is $Q$ such that 
$P \wred Q$ and $Q \downarrow_{\mathcal N} x$.
\end{definition}

\begin{definition}
%\label{def.bbisim}
An  ${\mathcal N}$-\emph{barbed bisimulation} over a set of names, ${\mathcal N}$, is a symmetric binary relation 
${\mathcal S}_{\mathcal N}$ between agents such that $P\rel{S}_{\mathcal N}Q$ implies:
\begin{enumerate}
\item If $P \red P'$ then $Q \wred Q'$ and $P'\rel{S}_{\mathcal N} Q'$.
\item If $P\downarrow_{\mathcal N} x$, then $Q\Downarrow_{\mathcal N} x$.
\end{enumerate}
$P$ is ${\mathcal N}$-barbed bisimilar to $Q$, written
$P \wbbisim_{\mathcal N} Q$, if $P \rel{S}_{\mathcal N} Q$ for some ${\mathcal N}$-barbed bisimulation ${\mathcal S}_{\mathcal N}$.
\end{definition}

$\mathcal{R} \subseteq \pi \times \pi$

$P \mathcal{R} Q => \forall P'. P \red P' \Rightarrow \exists Q'. Q \red Q', P' \mathcal{R} Q'$

$P \vdash x \Rightarrow Q \vdash x$

\begin{mathpar}
  \inferrule*[lab=Out-barb]{x \nameeq y}{{y}!\langle{Q}\rangle \vdash x}
  \and
  \inferrule*[lab=Par-barb]{\mbox{$P\vdash x$ or $Q\vdash x$}}{\binpar{P}{Q} \vdash x}
\end{mathpar}

\subsubsection{Contexts}

One of the principle advantages of computational calculi like the
$\pi$-calculus is a well-defined notion of context,
contextual-equivalence and a correlation between
contextual-equivalence and notions of bisimulation. The notion of
context allows the decomposition of a process into (sub-)process and
its syntactic environment, its context. Thus, a context may be
thought of as a process with a ``hole'' (written $\Box$) in it. The
application of a context $M$ to a process $P$, written $M[P]$, is
tantamount to filling the hole in $M$ with $P$. In this paper we do
not need the full weight of this theory, but do make use of the notion
of context in the proof the main theorem. 

\begin{mathpar}
  \inferrule* [lab=summation] {} {{M_{M},M_{N}} \bc \Box \;|\; x.M_{A} \;|\; M_{M}+M_{N}}
  \and
  \inferrule* [lab=agent] {} {{M_{A}} \bc (\vec{x})M_{P} \;| \; \clift{P_0,\ldots,M_{P},\ldots,P_N}}
  \and \\
  \inferrule* [lab=process] {} {{M_{P}} \bc M_{N} \;| \;P|M_{P} }
\end{mathpar} 

\begin{mathpar}
  \inferrule* [lab=sychronization] {} {M_{N} \bc \Box \;|\; x?M_{F} \;|\; x!M_{C}}
  \and
  \inferrule* [lab=abstraction] {} {{M_{F}} \bc (x)M_{P} }
  \and
  \inferrule* [lab=concretion] {} {{M_{C}} \bc \langle M_{P} \rangle }
  \and \\
  \inferrule* [lab=process] {} {{M_{P}} \bc M_{N} \;| \;P|M_{P} }
\end{mathpar}

\begin{definition}[contextual application] Given a context $M$, and
  process $P$, we define the \emph{contextual application}, $M[P] :=
  M\{P/\Box\}$. That is, the contextual application of M to P is the
  substitution of $P$ for $\Box$ in $M$.
\end{definition}

$\meaningof{-} : L \to \mathcal{P}(\pi)$

\begin{mathpar}
  \inferrule* [lab=collection] {} {\meaningof{true} = \pi, \and \meaningof{~E} = \pi \setminus \meaningof{E}, \and \meaningof{E_{1} \& E_{2}} = \meaningof{E_{1}} \cap \meaningof{E_{2}}}
\end{mathpar}

\begin{mathpar}
  \inferrule* [lab=structure] {} {\meaningof{0} = \{ P \in \pi | P \equiv 0 \}, \and \\ \meaningof{E_1 | E_2} = \{ P \in \pi | P \equiv P_{1} | P_{2}, P_{1} \in \meaningof{E_{1}}, P_{2} \in \meaningof{E_2}\} }
\end{mathpar}

\begin{mathpar}
 \inferrule* [lab=behavior] {} {\meaningof{\langle a?b \rangle E} = \{ P \in \pi | P \equiv Q | u?(y)P', \\ \and \\\\ \and \\ \;\;\; u \in \meaningof{a}, \forall z.P'\{z/y\} \in \meaningof{E\{z/b\}}\}, \and \\ \meaningof{a!E} = \{ P \in \pi | P \equiv Q | x!\langle P' \rangle, x \in \meaningof{a} P' \in \meaningof{E}\} }
\end{mathpar}

\begin{mathpar}
 \inferrule* [lab=nominal] {} {\meaningof{\quotep{E}} = \{ \quotep{P} \in \quotep{\pi} | P \in \meaningof{E} \}, \and \meaningof{\quotep{P}} = \{ \quotep{Q} \in \quotep{\pi} | P \equiv Q \} \and \\ \meaningof{@\quotep{E}} = \{ P \in \pi | P \equiv @x, x \in \meaningof{E} \}}
\end{mathpar}

\begin{eqnarray*}
  \\
  \meaningof{-} : TS \to ST
\end{eqnarray*}

\begin{eqnarray*}
  \\
  L : TS \to ST
\end{eqnarray*}

\begin{eqnarray*}
  \\
  P \models E \iff P \in \meaningof{E}
\end{eqnarray*}

\begin{eqnarray*}
  P \approx_{L} Q \iff \forall E \in L. P \models E \iff Q \models E
\end{eqnarray*}

\begin{eqnarray*}
  P \approx_{K} Q
\end{eqnarray*}

\begin{eqnarray*}
  P \approx Q
\end{eqnarray*}

$\approx_{K} = \approx = \approx_{L}$

\subsubsection{Contextual duality}

Note that contexts extend the quotation operation to a family of
operations from processes to names. Given a context, $M$, we can
define a \emph{nominal context}, $\quotep{M}$ by $\quotep{M}[P] :=
\quotep{M[P]}$. To foreshadow what is to come we observe that these
operations enjoy a duality with processes very much like the duality
between vectors and maps from vectors to scalars.

Further, because the calculus is essentially higher-order, we have a
correspondence between contexts and processes. More specifically,
given a name $x$ and a context $M$ we can construct $M^{*}_{x}$ such
that 

\begin{mathpar}
  M^{*}_{x} | \lift{x}{P} \red M[P]
\end{mathpar}

namely,

\begin{mathpar}
  M^{*}_{x} := x?(u).M[\dropn{u}]
\end{mathpar}

The dependence of $M^{*}_{x}$ on a name makes it an abstraction, 

\begin{mathpar}
  M^{*} := (x)x?(u).M[\dropn{u}]
\end{mathpar}

\subsection{Additional notation}

It will sometimes be convenient to denote the process a name
quotes. We already have the notation $x = \quotep{P}$, but it will be
convenient to introduce an alternate notation, $\procn{x}$, when we
want to emphasize the connection to the use of the name. Note that, by
virtue of name equivalence, $\quotep{\procn{x}} \nameeq x$; so, the
notation is consistent with previous definitions.

Further, because names have structure it is possible to effect
substitutions on the basis of that structure. This means we need to
upgrade our notation for substitutions, which we accomplish by
adapting comprehension notation. Thus,

\begin{mathpar}
  P\{ y / x : x \in S \}
\end{mathpar}

is interpreted to mean the process derived from P by replacing (in a
capture-avoiding manner) each occurrence of $x$ in $S$ by $y$. For example,

\begin{mathpar}
  P\{ \quotep{\procn{x}|\procn{x}} / x : x \in \freenames{P} \}
\end{mathpar}

will replace each (occurrence) of a free name $x$ in $P$ by
$\quotep{\procn{x}|\procn{x}}$.

Also, we will avail ourselves of the notation $x^{L}$ and $x^{R}$ to
denote injections of a name into disjoint copies of the name
space. There are numerous ways to accomplish this. One example can be
found in \cite{MeredithR05}. This notation overloads to vectors of
names: $\vec{x}^{\pi} := (x_{i}^{\pi} \; : \; 0 \leq i < |\vec{x}| )$ where $\pi \in \{L,R\}$.

We also use $P^{\Box} := P|\Box$.

In \cite{MeredithR05} an interpretation of the new operator is
given. It turns out that there are several possible interpretations
all enjoying the requisite algebraic properties of the operator (see
\cite{milner91polyadicpi}). We will therefore make liberal use of
$(\nu\; \vec{x})P$.

% subsection the_syntax_and_semantics_of_the_notation_system (end)   

\input{qm2pi.qmops} 

\input{qm2pi.sterngerlach} 

\input{qm2pi.metric} 

% section concurrent_process_calculi (end)

%\input{qm2pi.proofsketch}

% section proof sketch (end)

%\input{qm2pi.slviaknots} 

% section spatial logic via knots (end)

\input{qm2pi.conclusion}

% section conclusion (end)

%\input{qm2pi.dtcodes} 

% section wiring algorithm (end)

\input{qm2pi.ack} 

% section acknowledgments (end)

\newpage


\bibliographystyle{plain}   
\bibliography{../../biblios/main.bib}

\input{qm2pi.rhodetails}

\end{document}

 

% section notation (end)

\input{qm2pi.process.calculi} 

% section concurrent_process_calculi_and_spatial_logics_ (end)
    
%\documentclass[12pt]{llncs}
%\documentclass{jktr}

\usepackage[pdftex]{hyperref}                   
\usepackage {listings}
\usepackage {mathpartir}
\usepackage{bcprules}
%\usepackage{listings}
                       
\usepackage{graphicx} 
%\usepackage[margins=2.5cm,nohead,nofoot]{geometry}
%\usepackage{geometry}
\usepackage{amsfonts}
\usepackage{amstext}
\usepackage{latexsym}
\usepackage{amssymb}
\usepackage{color}


%\include{myPreamble}
\include{qm2pi.local} 

%\ifpdf
%\usepackage[pdftex]{graphicx}
%\else
%\usepackage{graphicx}
%\fi

 % \ifpdf
%  \usepackage{pdfsync}
%  \if


%\title{Brief Article}
%\author{David F. Snyder}
%\author{L.G. Meredith}

%\address{Dept. of Math., Texas State University--San Marcos, San Marcos, TX 78666}
       
\pagestyle{empty}


\begin{document}

\lstset{language=[Objective]Caml,frame=shadowbox}

\input{qm2pi.front}

% section front matter (end)

\input{qm2pi.intro} 
 
% section introduction (end)

% \input{qm2pi.knotations} 

% section notation (end)

\input{qm2pi.process.calculi} 

% section concurrent_process_calculi_and_spatial_logics_ (end)
    
%\input{qm2pi.knots2pi} 

%\input{qm2pi.trefoil} 

%\input{qm2pi.mainthm} 

% subsection basic_interpretation (end)

%\input{qm2pi.rho.presentation} 
\subsection{The syntax and semantics of the notation system}\label{sub:the_syntax_and_semantics_of_the_notation_system} % (fold)

We now summarize a technical presentation of the calculus that
embodies our theory of dynamics. The typical presentation of such a
calculus follows the style of giving generators and relations on
them. The grammar, below, describing term constructors, freely
generates the set of processes, $\Proc$. This set is then quotiented
by a relation known as structural congruence and it is over this set
that the notion of dynamics is expressed. This presentation is
essentially that of \cite{MeredithR05} with the addition of
polyadicity and summation. For readability we have relegated some of
the technical subtleties to an appendix.

\subsubsection{Process grammar}\label{subsub:process_grammar}

\begin{mathpar}
  \inferrule* [lab=synchronization] {} {{M} \bc \pzero \;|\; x?F \;|\; x!C }
  \and
  \inferrule* [lab=abstraction] {} {{F} \bc (x)P}
  \and
  \inferrule* [lab=concretion] {} {{C} \bc \langle Q \rangle}
  \and
  \inferrule* [lab=process] {} {{P,Q} \bc M \;| \;P|Q \;|\; @{x}}
  \and
  \inferrule* [lab=name] {} {{x} \bc \quotep{P}}
\end{mathpar} 

Note that $\vec{x}$ (resp. $\vec{P}$) denotes a vector of names
(resp. processes) of length $|\vec{x}|$ (resp. $|\vec{P}|$). We adopt
the following useful abbreviations.

\begin{mathpar}
   x?(\vec{y}).P := x.(\vec{y})P \and  x\clift{\vec{P}} := x.\clift{\vec{P}}
   \and x!(y) := \lift{x}{\dropn{y}}
   \and \Pi_{i=0}^{n-1}P_i := P_0 | \ldots | P_{n-1}
\end{mathpar}

\subsubsection{Structural congruence}

\paragraph{Free and bound names and alpha-equivalence.} At the
core of structural equivalence is alpha-equivalence which identifies
process that are the same up to a change of variable. Formally, we
recognize the distinction between free and bound names. The free names
of a process, $\freenames{P}$, may be calculated recursively as
follows:

\begin{mathpar}
\freenames{\pzero} := \emptyset
  \and \\
  \freenames{x?(y).P} := \{ x \} \cup (\freenames{P} \setminus \{ y \})
  \and 
  \freenames{x!\langle P \rangle} := \{ x \} \cup \{ P \} 
  \and \\
  \freenames{P|Q} := \freenames{P} \cup \freenames{Q}
  \and \\
  \freenames{@{x}} := \{ x \}
\end{mathpar}

$\pi$
$\quotep{\pi}$

$\freenames{-} : \pi \to \mathcal{P}(\quotep{\pi})$

\begin{eqnarray*}
  \freenames{\pzero} & := & \emptyset \\
  \freenames{x?(y).P} & := & \{ x \} \cup (\freenames{P} \setminus \{ y \}) \\
  \freenames{x!\langle P \rangle} & := & \{ x \} \cup \{ P \} \\
  \freenames{P|Q} & := & \freenames{P} \cup \freenames{Q} \\
  \freenames{\dropn{x}} & := & \{ x \}
\end{eqnarray*}

The bound names of a process, $\boundnames{P}$, are those names occurring in $P$
that are not free. For example, in $x?(y).0$, the name $x$ is free, while $y$ is bound.

\begin{mathpar}
  \inferrule* [lab=monoidal-laws] {} { P|Q \equiv Q|P \and P|0 \equiv P \and P|(Q|R) \equiv (P|Q)|R }
\end{mathpar}

\begin{mathpar}
  \inferrule* [lab=alpha-equivalence] {} { (x)P \equiv (y)P\{y/x\} \and y \not\in \freenames{P} }
\end{mathpar}

\begin{definition}
Then two processes, $P,Q$, are alpha-equivalent if $P = Q\{\vec{y}/\vec{x}\}$ for
some $\vec{x} \in \boundnames{Q},\vec{y} \in \boundnames{P}$, where $Q\{\vec{y}/\vec{x}\}$
denotes the capture-avoiding substitution of $\vec{y}$ for $\vec{x}$ in $Q$.
\end{definition}

\begin{definition}
  The {\em structural congruence} \cite{SangiorgiWalker} , $\equiv$,
  between processes is the least congruence containing
  alpha-equivalence, satisfying the abelian monoid laws
  (associativity, commutativity and $\pzero$ as identity) for parallel
  composition $|$ and for summation $+$.
\end{definition}

\subsection{Name equivalence}

We take name equivalence, written $\nameeq$, to be the smallest
equivalence relation generated by the following rules.

\begin{mathpar}
\inferrule*[lab=Quote-drop]
{ }
{ \quotep{@{x}} \nameeq x }

\inferrule*[lab=Struct-equiv]
{ P \scong Q }
{ \quotep{P} \nameeq \quotep{Q} }
\end{mathpar}

The astute reader will have noticed that the mutual recursion of names
and processes imposes a mutual recursion on alpha-equivalence and
structural equivalence via name-equivalence. Fortunately, all of this
works out pleasantly and we may calculate in the natural way, free of
concern. The reader interested in the details is referred to the
appendix \ref{appendix:rho_details}.

\subsection{Substitution}

We use $\Proc$ for the set of processes, $\QProc$ for the set of
names, and $\id{\{}\vec{y} / \vec{x} \id{\}}$ to denote partial maps,
$s : \QProc \rightarrow \QProc$. A map, $s$ lifts, uniquely, to a map
on process terms, $\widehat{s} : \Proc \rightarrow \Proc$ by the
following equations.

\begin{mathpar}
  (0) \psubstp{Q}{P} := 0 \\
  (R \juxtap S) \psubstp{Q}{P}
  :=    
  (R)\psubstp{Q}{P} \juxtap (S) \psubstp{Q}{P} \\
  (x?(y).R) \psubstp{Q}{P}    
  :=    
  (x)\substp{Q}{P} (z)\concat( (R \psubstn{z}{y}) \psubstp{Q}{P} ) \\
  (\lift{x}{R}) \psubstp{Q}{P}  
  :=
  \lift{(x)\substp{Q}{P}}{ R \psubstp{Q}{P} } \\
%   (\dropn{x})  \psubstp{Q}{P}       
%   := 
%   \left\{ 
%     \begin{array}{ccc} 
%       \dropn{\quotep{Q}} & & x \nameeq \quotep{P} \\
%       \dropn{x} & & otherwise \\
%     \end{array}
%   \right. 
  (\dropn{x})  \psubstp{Q}{P}       
  := 
  \left\{ 
    \begin{array}{ccc} 
      Q & & x \nameeq \quotep{P} \\
      \dropn{x} & & otherwise \\
    \end{array}
  \right.
\end{mathpar}
 

where

\begin{eqnarray}
  (x)\id{\{} \lpquote Q \rpquote / \lpquote P \rpquote \id{\}}            = 
  \left\{ 
    \begin{array}{ccc}
      \lpquote Q \rpquote & & x \nameeq \lpquote P \rpquote \\
      x & & otherwise \\
    \end{array}
  \right. \nonumber
\end{eqnarray}

and $z$ is chosen distinct from $\quotep{P}$, $\quotep{Q}$, the free
names in $Q$, and all the names in $R$. Our $\alpha$-equivalence will
be built in the standard way from this substitution.

\begin{remark}\label{rem:no_self_referential_names}
  One consequence of these definitions is that $\forall P. \quotep{P}
  \not\in \freenames{P}$.
\end{remark}

\subsection{ Dynamic quote: an example }

Anticipating something of what's to come, consider applying the
substitution, $\widehat{\id{\{}u / z \id{\}}}$, to the following pair
of processes, $\lift{w}{y!(z)}$ and $w[ \lpquote y!(z) \rpquote ]$.

\begin{eqnarray}
	\lift{w}{y!(z)}\widehat{\id{\{}u / z \id{\}}}
		& = &
		\lift{w}{y!(u)} \nonumber\\
	w[ \lpquote y!(z) \rpquote ] \widehat{ \id{\{}u / z \id{\}} }
		& = &
		w[ \lpquote y!(z) \rpquote ] \nonumber
\end{eqnarray}

Because the body of the process between quotes is impervious to
substitution, we get radically different answers. In fact, by
examining the first process in an input context,
e.g. $x?(z).\lift{w}{y!(z)}$, we see that the process under the lift
operator may be shaped by prefixed inputs binding a name inside it. In
this sense, the lift operator will be seen as a way to dynamically
construct processes before reifying them as names.

Finally equipped with these standard features we can present the
dynamics of the calculus.

\subsubsection{Operational semantics} 

Finally, we introduce the computational dynamics. What marks these
algebras as distinct from other more traditionally studied algebraic
structures, e.g. vector spaces or polynomial rings, is the manner in
which dynamics is captured. In traditional structures, dynamics is typically
expressed through morphisms between such structures, as in linear maps
between vector spaces or morphisms between rings. In algebras
associated with the semantics of computation, the dynamics is
expressed as part of the algebraic structure itself, through a
reduction reduction relation typically denoted by $\red$. Below, we
give a recursive presentation of this relation for the calculus used
in the encoding.

$\red \subseteq \pi \times \pi$
$\red : \pi \to \mathcal{P}(\pi)$

\begin{mathpar}
  \inferrule* [lab=Comm] { \textsf{match}( x_{src}, x_{trgt} ) } { x_{trgt}?(y)P \; | \; x_{src}!\langle {Q} \rangle \red P\{\quotep{Q}/y}\} }
  \and \\
  \inferrule* [lab=Par] {{P} \red {P}'} {{{P} | {Q}} \red {{P}' | {Q}}}
  \and
  \inferrule* [lab=Equiv]{{{P} \scong {P}'} \andalso {{P}' \red {Q}'} \andalso {{Q}' \scong {Q}}}{{P} \red {Q}}
\end{mathpar}

\begin{eqnarray*}
  match_{\equiv} (\quotep{P},\quotep{Q}) & := & P \equiv Q \\
  match_{\dagger}(\quotep{P},\quotep{Q}) & := & \forall R. P|Q \red^{*} R => R \red^{*} 0 \\
  match_{K}(\quotep{P},\quotep{Q}) & := & K \mbox{ for some context } K
\end{eqnarray*}

$u?(x)P | u!\langle Q \rangle \red P\{\quotep{Q}/x\}$

%We write $\wred$ for $\red^*$, and $P\red$ if $\exists Q $ such that $ P \red Q$.
We write $P\red$ if $\exists Q $ such that $ P \red Q$ and $P\not\red$, otherwise.

\section{Replication}

As mentioned before, it is known that replication (and hence
recursion) can be implemented in a higher-order process algebra
\cite{SangiorgiWalker}. As our first example of calculation with the
machinery thus far presented we give the construction explicitly in
the {\rhoc}.

\begin{eqnarray}
	D_{x} & := & \prefix{x}{y}{(\binpar{\outputp{x}{y}}{@{y}})} \nonumber\\
	\bangp_{x}{P} & := & \binpar{{x}!\langle{\binpar{D_{x}}{P}}\rangle}{D_{x}} \nonumber
\end{eqnarray}

\begin{eqnarray}
	\bangp_{x}{P} & & \nonumber\\
	=
	& {x}!\langle{(\prefix{x}{y}{(\outputp{x}{y} | @{y})) | P}}\rangle 
	      | \prefix{x}{y}{(\outputp{x}{y} | @{y})} & \nonumber\\
	\red
	& (\outputp{x}{y} | @{y})\substn{\quotep{(\prefix{x}{y}{(@{y} | \outputp{x}{y})) | P}}}{y} & \nonumber\\
	=
	& \outputp{x}{\quotep{(\prefix{x}{y}{(\outputp{x}{y} | @{y})) | P}}}
	  | {(\prefix{x}{y}{(\outputp{x}{y} | @{y})) | P}} & \nonumber\\
	\red
	& \ldots & \nonumber\\
	\red^*
	& P | P | \ldots & \nonumber
\end{eqnarray}

Of course, this encoding, as an implementation, runs away, unfolding
$\bangp{P}$ eagerly. A lazier and more implementable replication
operator, restricted to input-guarded processes, may be obtained as follows.

\begin{eqnarray}
\bangp{\prefix{u}{v}{P}} 
	:= 
	\binpar{\lift{x}{\prefix{u}{v}{(\binpar{D(x)}{P})}}}{D(x)} \nonumber
\end{eqnarray}

\begin{remark}
  Note that the lazier definition still does not deal with summation
  or mixed summation (i.e. sums over input and output). The reader is
  invited to construct definitions of replication that deal with these
  features. 

  Further, the definitions are parameterized in a name, $x$. Can you,
  gentle reader, make a definition that eliminates this parameter and
  guarantees no accidental interaction between the replication
  machinery and the process being replicated -- i.e. no accidental
  sharing of names used by the process to get its work done and the
  name(s) used by the replication to effect copying. This latter
  revision of the definition of replication is crucial to obtaining
  the expected identity $!!P \sim !P$.
\end{remark}

\begin{remark}\label{rem:paradoxical_combinator}
  The reader familiar with the lambda calculus will have noticed the
  similarity between $D$ and the paradoxical combinator.

  [Ed. note: the existence of this seems to suggest we have to be more
  restrictive on the set of processes and names we admit if we are to
  support no-cloning.]
\end{remark}

\subsubsection{Bisimulation}

The computational dynamics gives rise to another kind of equivalence,
the equivalence of computational behavior. As previously mentioned
this is typically captured \emph{via} some form of bisimulation.

% The notion we use in this paper is weak barbed bisimulation
% \cite{milner91polyadicpi}.

The notion we use in this paper is derived from weak barbed
bisimulation \cite{milner91polyadicpi}. 

\begin{definition}
An \emph{observation relation}, $\downarrow_{\mathcal N}$, over a set
of names, $\mathcal N$, is the smallest relation satisfying the rules
below.

\infrule[Out-barb]{y \in {\mathcal N}, \; x \nameeq y}
		  {\outputp{x}{v} \downarrow_{\mathcal N} x}
\infrule[Par-barb]{\mbox{$P\downarrow_{\mathcal N} x$ or $Q\downarrow_{\mathcal N} x$}}
		  {\binpar{P}{Q} \downarrow_{\mathcal N} x}

We write $P \Downarrow_{\mathcal N} x$ if there is $Q$ such that 
$P \wred Q$ and $Q \downarrow_{\mathcal N} x$.
\end{definition}

\begin{definition}
%\label{def.bbisim}
An  ${\mathcal N}$-\emph{barbed bisimulation} over a set of names, ${\mathcal N}$, is a symmetric binary relation 
${\mathcal S}_{\mathcal N}$ between agents such that $P\rel{S}_{\mathcal N}Q$ implies:
\begin{enumerate}
\item If $P \red P'$ then $Q \wred Q'$ and $P'\rel{S}_{\mathcal N} Q'$.
\item If $P\downarrow_{\mathcal N} x$, then $Q\Downarrow_{\mathcal N} x$.
\end{enumerate}
$P$ is ${\mathcal N}$-barbed bisimilar to $Q$, written
$P \wbbisim_{\mathcal N} Q$, if $P \rel{S}_{\mathcal N} Q$ for some ${\mathcal N}$-barbed bisimulation ${\mathcal S}_{\mathcal N}$.
\end{definition}

$\mathcal{R} \subseteq \pi \times \pi$

$P \mathcal{R} Q => \forall P'. P \red P' \Rightarrow \exists Q'. Q \red Q', P' \mathcal{R} Q'$

$P \vdash x \Rightarrow Q \vdash x$

\begin{mathpar}
  \inferrule*[lab=Out-barb]{x \nameeq y}{{y}!\langle{Q}\rangle \vdash x}
  \and
  \inferrule*[lab=Par-barb]{\mbox{$P\vdash x$ or $Q\vdash x$}}{\binpar{P}{Q} \vdash x}
\end{mathpar}

\subsubsection{Contexts}

One of the principle advantages of computational calculi like the
$\pi$-calculus is a well-defined notion of context,
contextual-equivalence and a correlation between
contextual-equivalence and notions of bisimulation. The notion of
context allows the decomposition of a process into (sub-)process and
its syntactic environment, its context. Thus, a context may be
thought of as a process with a ``hole'' (written $\Box$) in it. The
application of a context $M$ to a process $P$, written $M[P]$, is
tantamount to filling the hole in $M$ with $P$. In this paper we do
not need the full weight of this theory, but do make use of the notion
of context in the proof the main theorem. 

\begin{mathpar}
  \inferrule* [lab=summation] {} {{M_{M},M_{N}} \bc \Box \;|\; x.M_{A} \;|\; M_{M}+M_{N}}
  \and
  \inferrule* [lab=agent] {} {{M_{A}} \bc (\vec{x})M_{P} \;| \; \clift{P_0,\ldots,M_{P},\ldots,P_N}}
  \and \\
  \inferrule* [lab=process] {} {{M_{P}} \bc M_{N} \;| \;P|M_{P} }
\end{mathpar} 

\begin{mathpar}
  \inferrule* [lab=sychronization] {} {M_{N} \bc \Box \;|\; x?M_{F} \;|\; x!M_{C}}
  \and
  \inferrule* [lab=abstraction] {} {{M_{F}} \bc (x)M_{P} }
  \and
  \inferrule* [lab=concretion] {} {{M_{C}} \bc \langle M_{P} \rangle }
  \and \\
  \inferrule* [lab=process] {} {{M_{P}} \bc M_{N} \;| \;P|M_{P} }
\end{mathpar}

\begin{definition}[contextual application] Given a context $M$, and
  process $P$, we define the \emph{contextual application}, $M[P] :=
  M\{P/\Box\}$. That is, the contextual application of M to P is the
  substitution of $P$ for $\Box$ in $M$.
\end{definition}

$\meaningof{-} : L \to \mathcal{P}(\pi)$

\begin{mathpar}
  \inferrule* [lab=collection] {} {\meaningof{true} = \pi, \and \meaningof{~E} = \pi \setminus \meaningof{E}, \and \meaningof{E_{1} \& E_{2}} = \meaningof{E_{1}} \cap \meaningof{E_{2}}}
\end{mathpar}

\begin{mathpar}
  \inferrule* [lab=structure] {} {\meaningof{0} = \{ P \in \pi | P \equiv 0 \}, \and \\ \meaningof{E_1 | E_2} = \{ P \in \pi | P \equiv P_{1} | P_{2}, P_{1} \in \meaningof{E_{1}}, P_{2} \in \meaningof{E_2}\} }
\end{mathpar}

\begin{mathpar}
 \inferrule* [lab=behavior] {} {\meaningof{\langle a?b \rangle E} = \{ P \in \pi | P \equiv Q | u?(y)P', \\ \and \\\\ \and \\ \;\;\; u \in \meaningof{a}, \forall z.P'\{z/y\} \in \meaningof{E\{z/b\}}\}, \and \\ \meaningof{a!E} = \{ P \in \pi | P \equiv Q | x!\langle P' \rangle, x \in \meaningof{a} P' \in \meaningof{E}\} }
\end{mathpar}

\begin{mathpar}
 \inferrule* [lab=nominal] {} {\meaningof{\quotep{E}} = \{ \quotep{P} \in \quotep{\pi} | P \in \meaningof{E} \}, \and \meaningof{\quotep{P}} = \{ \quotep{Q} \in \quotep{\pi} | P \equiv Q \} \and \\ \meaningof{@\quotep{E}} = \{ P \in \pi | P \equiv @x, x \in \meaningof{E} \}}
\end{mathpar}

\begin{eqnarray*}
  \\
  \meaningof{-} : TS \to ST
\end{eqnarray*}

\begin{eqnarray*}
  \\
  L : TS \to ST
\end{eqnarray*}

\begin{eqnarray*}
  \\
  P \models E \iff P \in \meaningof{E}
\end{eqnarray*}

\begin{eqnarray*}
  P \approx_{L} Q \iff \forall E \in L. P \models E \iff Q \models E
\end{eqnarray*}

\begin{eqnarray*}
  P \approx_{K} Q
\end{eqnarray*}

\begin{eqnarray*}
  P \approx Q
\end{eqnarray*}

$\approx_{K} = \approx = \approx_{L}$

\subsubsection{Contextual duality}

Note that contexts extend the quotation operation to a family of
operations from processes to names. Given a context, $M$, we can
define a \emph{nominal context}, $\quotep{M}$ by $\quotep{M}[P] :=
\quotep{M[P]}$. To foreshadow what is to come we observe that these
operations enjoy a duality with processes very much like the duality
between vectors and maps from vectors to scalars.

Further, because the calculus is essentially higher-order, we have a
correspondence between contexts and processes. More specifically,
given a name $x$ and a context $M$ we can construct $M^{*}_{x}$ such
that 

\begin{mathpar}
  M^{*}_{x} | \lift{x}{P} \red M[P]
\end{mathpar}

namely,

\begin{mathpar}
  M^{*}_{x} := x?(u).M[\dropn{u}]
\end{mathpar}

The dependence of $M^{*}_{x}$ on a name makes it an abstraction, 

\begin{mathpar}
  M^{*} := (x)x?(u).M[\dropn{u}]
\end{mathpar}

\subsection{Additional notation}

It will sometimes be convenient to denote the process a name
quotes. We already have the notation $x = \quotep{P}$, but it will be
convenient to introduce an alternate notation, $\procn{x}$, when we
want to emphasize the connection to the use of the name. Note that, by
virtue of name equivalence, $\quotep{\procn{x}} \nameeq x$; so, the
notation is consistent with previous definitions.

Further, because names have structure it is possible to effect
substitutions on the basis of that structure. This means we need to
upgrade our notation for substitutions, which we accomplish by
adapting comprehension notation. Thus,

\begin{mathpar}
  P\{ y / x : x \in S \}
\end{mathpar}

is interpreted to mean the process derived from P by replacing (in a
capture-avoiding manner) each occurrence of $x$ in $S$ by $y$. For example,

\begin{mathpar}
  P\{ \quotep{\procn{x}|\procn{x}} / x : x \in \freenames{P} \}
\end{mathpar}

will replace each (occurrence) of a free name $x$ in $P$ by
$\quotep{\procn{x}|\procn{x}}$.

Also, we will avail ourselves of the notation $x^{L}$ and $x^{R}$ to
denote injections of a name into disjoint copies of the name
space. There are numerous ways to accomplish this. One example can be
found in \cite{MeredithR05}. This notation overloads to vectors of
names: $\vec{x}^{\pi} := (x_{i}^{\pi} \; : \; 0 \leq i < |\vec{x}| )$ where $\pi \in \{L,R\}$.

We also use $P^{\Box} := P|\Box$.

In \cite{MeredithR05} an interpretation of the new operator is
given. It turns out that there are several possible interpretations
all enjoying the requisite algebraic properties of the operator (see
\cite{milner91polyadicpi}). We will therefore make liberal use of
$(\nu\; \vec{x})P$.

% subsection the_syntax_and_semantics_of_the_notation_system (end)   

\input{qm2pi.qmops} 

\input{qm2pi.sterngerlach} 

\input{qm2pi.metric} 

% section concurrent_process_calculi (end)

%\input{qm2pi.proofsketch}

% section proof sketch (end)

%\input{qm2pi.slviaknots} 

% section spatial logic via knots (end)

\input{qm2pi.conclusion}

% section conclusion (end)

%\input{qm2pi.dtcodes} 

% section wiring algorithm (end)

\input{qm2pi.ack} 

% section acknowledgments (end)

\newpage


\bibliographystyle{plain}   
\bibliography{../../biblios/main.bib}

\input{qm2pi.rhodetails}

\end{document}

 

%\documentclass[12pt]{llncs}
%\documentclass{jktr}

\usepackage[pdftex]{hyperref}                   
\usepackage {listings}
\usepackage {mathpartir}
\usepackage{bcprules}
%\usepackage{listings}
                       
\usepackage{graphicx} 
%\usepackage[margins=2.5cm,nohead,nofoot]{geometry}
%\usepackage{geometry}
\usepackage{amsfonts}
\usepackage{amstext}
\usepackage{latexsym}
\usepackage{amssymb}
\usepackage{color}


%\include{myPreamble}
\include{qm2pi.local} 

%\ifpdf
%\usepackage[pdftex]{graphicx}
%\else
%\usepackage{graphicx}
%\fi

 % \ifpdf
%  \usepackage{pdfsync}
%  \if


%\title{Brief Article}
%\author{David F. Snyder}
%\author{L.G. Meredith}

%\address{Dept. of Math., Texas State University--San Marcos, San Marcos, TX 78666}
       
\pagestyle{empty}


\begin{document}

\lstset{language=[Objective]Caml,frame=shadowbox}

\input{qm2pi.front}

% section front matter (end)

\input{qm2pi.intro} 
 
% section introduction (end)

% \input{qm2pi.knotations} 

% section notation (end)

\input{qm2pi.process.calculi} 

% section concurrent_process_calculi_and_spatial_logics_ (end)
    
%\input{qm2pi.knots2pi} 

%\input{qm2pi.trefoil} 

%\input{qm2pi.mainthm} 

% subsection basic_interpretation (end)

%\input{qm2pi.rho.presentation} 
\subsection{The syntax and semantics of the notation system}\label{sub:the_syntax_and_semantics_of_the_notation_system} % (fold)

We now summarize a technical presentation of the calculus that
embodies our theory of dynamics. The typical presentation of such a
calculus follows the style of giving generators and relations on
them. The grammar, below, describing term constructors, freely
generates the set of processes, $\Proc$. This set is then quotiented
by a relation known as structural congruence and it is over this set
that the notion of dynamics is expressed. This presentation is
essentially that of \cite{MeredithR05} with the addition of
polyadicity and summation. For readability we have relegated some of
the technical subtleties to an appendix.

\subsubsection{Process grammar}\label{subsub:process_grammar}

\begin{mathpar}
  \inferrule* [lab=synchronization] {} {{M} \bc \pzero \;|\; x?F \;|\; x!C }
  \and
  \inferrule* [lab=abstraction] {} {{F} \bc (x)P}
  \and
  \inferrule* [lab=concretion] {} {{C} \bc \langle Q \rangle}
  \and
  \inferrule* [lab=process] {} {{P,Q} \bc M \;| \;P|Q \;|\; @{x}}
  \and
  \inferrule* [lab=name] {} {{x} \bc \quotep{P}}
\end{mathpar} 

Note that $\vec{x}$ (resp. $\vec{P}$) denotes a vector of names
(resp. processes) of length $|\vec{x}|$ (resp. $|\vec{P}|$). We adopt
the following useful abbreviations.

\begin{mathpar}
   x?(\vec{y}).P := x.(\vec{y})P \and  x\clift{\vec{P}} := x.\clift{\vec{P}}
   \and x!(y) := \lift{x}{\dropn{y}}
   \and \Pi_{i=0}^{n-1}P_i := P_0 | \ldots | P_{n-1}
\end{mathpar}

\subsubsection{Structural congruence}

\paragraph{Free and bound names and alpha-equivalence.} At the
core of structural equivalence is alpha-equivalence which identifies
process that are the same up to a change of variable. Formally, we
recognize the distinction between free and bound names. The free names
of a process, $\freenames{P}$, may be calculated recursively as
follows:

\begin{mathpar}
\freenames{\pzero} := \emptyset
  \and \\
  \freenames{x?(y).P} := \{ x \} \cup (\freenames{P} \setminus \{ y \})
  \and 
  \freenames{x!\langle P \rangle} := \{ x \} \cup \{ P \} 
  \and \\
  \freenames{P|Q} := \freenames{P} \cup \freenames{Q}
  \and \\
  \freenames{@{x}} := \{ x \}
\end{mathpar}

$\pi$
$\quotep{\pi}$

$\freenames{-} : \pi \to \mathcal{P}(\quotep{\pi})$

\begin{eqnarray*}
  \freenames{\pzero} & := & \emptyset \\
  \freenames{x?(y).P} & := & \{ x \} \cup (\freenames{P} \setminus \{ y \}) \\
  \freenames{x!\langle P \rangle} & := & \{ x \} \cup \{ P \} \\
  \freenames{P|Q} & := & \freenames{P} \cup \freenames{Q} \\
  \freenames{\dropn{x}} & := & \{ x \}
\end{eqnarray*}

The bound names of a process, $\boundnames{P}$, are those names occurring in $P$
that are not free. For example, in $x?(y).0$, the name $x$ is free, while $y$ is bound.

\begin{mathpar}
  \inferrule* [lab=monoidal-laws] {} { P|Q \equiv Q|P \and P|0 \equiv P \and P|(Q|R) \equiv (P|Q)|R }
\end{mathpar}

\begin{mathpar}
  \inferrule* [lab=alpha-equivalence] {} { (x)P \equiv (y)P\{y/x\} \and y \not\in \freenames{P} }
\end{mathpar}

\begin{definition}
Then two processes, $P,Q$, are alpha-equivalent if $P = Q\{\vec{y}/\vec{x}\}$ for
some $\vec{x} \in \boundnames{Q},\vec{y} \in \boundnames{P}$, where $Q\{\vec{y}/\vec{x}\}$
denotes the capture-avoiding substitution of $\vec{y}$ for $\vec{x}$ in $Q$.
\end{definition}

\begin{definition}
  The {\em structural congruence} \cite{SangiorgiWalker} , $\equiv$,
  between processes is the least congruence containing
  alpha-equivalence, satisfying the abelian monoid laws
  (associativity, commutativity and $\pzero$ as identity) for parallel
  composition $|$ and for summation $+$.
\end{definition}

\subsection{Name equivalence}

We take name equivalence, written $\nameeq$, to be the smallest
equivalence relation generated by the following rules.

\begin{mathpar}
\inferrule*[lab=Quote-drop]
{ }
{ \quotep{@{x}} \nameeq x }

\inferrule*[lab=Struct-equiv]
{ P \scong Q }
{ \quotep{P} \nameeq \quotep{Q} }
\end{mathpar}

The astute reader will have noticed that the mutual recursion of names
and processes imposes a mutual recursion on alpha-equivalence and
structural equivalence via name-equivalence. Fortunately, all of this
works out pleasantly and we may calculate in the natural way, free of
concern. The reader interested in the details is referred to the
appendix \ref{appendix:rho_details}.

\subsection{Substitution}

We use $\Proc$ for the set of processes, $\QProc$ for the set of
names, and $\id{\{}\vec{y} / \vec{x} \id{\}}$ to denote partial maps,
$s : \QProc \rightarrow \QProc$. A map, $s$ lifts, uniquely, to a map
on process terms, $\widehat{s} : \Proc \rightarrow \Proc$ by the
following equations.

\begin{mathpar}
  (0) \psubstp{Q}{P} := 0 \\
  (R \juxtap S) \psubstp{Q}{P}
  :=    
  (R)\psubstp{Q}{P} \juxtap (S) \psubstp{Q}{P} \\
  (x?(y).R) \psubstp{Q}{P}    
  :=    
  (x)\substp{Q}{P} (z)\concat( (R \psubstn{z}{y}) \psubstp{Q}{P} ) \\
  (\lift{x}{R}) \psubstp{Q}{P}  
  :=
  \lift{(x)\substp{Q}{P}}{ R \psubstp{Q}{P} } \\
%   (\dropn{x})  \psubstp{Q}{P}       
%   := 
%   \left\{ 
%     \begin{array}{ccc} 
%       \dropn{\quotep{Q}} & & x \nameeq \quotep{P} \\
%       \dropn{x} & & otherwise \\
%     \end{array}
%   \right. 
  (\dropn{x})  \psubstp{Q}{P}       
  := 
  \left\{ 
    \begin{array}{ccc} 
      Q & & x \nameeq \quotep{P} \\
      \dropn{x} & & otherwise \\
    \end{array}
  \right.
\end{mathpar}
 

where

\begin{eqnarray}
  (x)\id{\{} \lpquote Q \rpquote / \lpquote P \rpquote \id{\}}            = 
  \left\{ 
    \begin{array}{ccc}
      \lpquote Q \rpquote & & x \nameeq \lpquote P \rpquote \\
      x & & otherwise \\
    \end{array}
  \right. \nonumber
\end{eqnarray}

and $z$ is chosen distinct from $\quotep{P}$, $\quotep{Q}$, the free
names in $Q$, and all the names in $R$. Our $\alpha$-equivalence will
be built in the standard way from this substitution.

\begin{remark}\label{rem:no_self_referential_names}
  One consequence of these definitions is that $\forall P. \quotep{P}
  \not\in \freenames{P}$.
\end{remark}

\subsection{ Dynamic quote: an example }

Anticipating something of what's to come, consider applying the
substitution, $\widehat{\id{\{}u / z \id{\}}}$, to the following pair
of processes, $\lift{w}{y!(z)}$ and $w[ \lpquote y!(z) \rpquote ]$.

\begin{eqnarray}
	\lift{w}{y!(z)}\widehat{\id{\{}u / z \id{\}}}
		& = &
		\lift{w}{y!(u)} \nonumber\\
	w[ \lpquote y!(z) \rpquote ] \widehat{ \id{\{}u / z \id{\}} }
		& = &
		w[ \lpquote y!(z) \rpquote ] \nonumber
\end{eqnarray}

Because the body of the process between quotes is impervious to
substitution, we get radically different answers. In fact, by
examining the first process in an input context,
e.g. $x?(z).\lift{w}{y!(z)}$, we see that the process under the lift
operator may be shaped by prefixed inputs binding a name inside it. In
this sense, the lift operator will be seen as a way to dynamically
construct processes before reifying them as names.

Finally equipped with these standard features we can present the
dynamics of the calculus.

\subsubsection{Operational semantics} 

Finally, we introduce the computational dynamics. What marks these
algebras as distinct from other more traditionally studied algebraic
structures, e.g. vector spaces or polynomial rings, is the manner in
which dynamics is captured. In traditional structures, dynamics is typically
expressed through morphisms between such structures, as in linear maps
between vector spaces or morphisms between rings. In algebras
associated with the semantics of computation, the dynamics is
expressed as part of the algebraic structure itself, through a
reduction reduction relation typically denoted by $\red$. Below, we
give a recursive presentation of this relation for the calculus used
in the encoding.

$\red \subseteq \pi \times \pi$
$\red : \pi \to \mathcal{P}(\pi)$

\begin{mathpar}
  \inferrule* [lab=Comm] { \textsf{match}( x_{src}, x_{trgt} ) } { x_{trgt}?(y)P \; | \; x_{src}!\langle {Q} \rangle \red P\{\quotep{Q}/y}\} }
  \and \\
  \inferrule* [lab=Par] {{P} \red {P}'} {{{P} | {Q}} \red {{P}' | {Q}}}
  \and
  \inferrule* [lab=Equiv]{{{P} \scong {P}'} \andalso {{P}' \red {Q}'} \andalso {{Q}' \scong {Q}}}{{P} \red {Q}}
\end{mathpar}

\begin{eqnarray*}
  match_{\equiv} (\quotep{P},\quotep{Q}) & := & P \equiv Q \\
  match_{\dagger}(\quotep{P},\quotep{Q}) & := & \forall R. P|Q \red^{*} R => R \red^{*} 0 \\
  match_{K}(\quotep{P},\quotep{Q}) & := & K \mbox{ for some context } K
\end{eqnarray*}

$u?(x)P | u!\langle Q \rangle \red P\{\quotep{Q}/x\}$

%We write $\wred$ for $\red^*$, and $P\red$ if $\exists Q $ such that $ P \red Q$.
We write $P\red$ if $\exists Q $ such that $ P \red Q$ and $P\not\red$, otherwise.

\section{Replication}

As mentioned before, it is known that replication (and hence
recursion) can be implemented in a higher-order process algebra
\cite{SangiorgiWalker}. As our first example of calculation with the
machinery thus far presented we give the construction explicitly in
the {\rhoc}.

\begin{eqnarray}
	D_{x} & := & \prefix{x}{y}{(\binpar{\outputp{x}{y}}{@{y}})} \nonumber\\
	\bangp_{x}{P} & := & \binpar{{x}!\langle{\binpar{D_{x}}{P}}\rangle}{D_{x}} \nonumber
\end{eqnarray}

\begin{eqnarray}
	\bangp_{x}{P} & & \nonumber\\
	=
	& {x}!\langle{(\prefix{x}{y}{(\outputp{x}{y} | @{y})) | P}}\rangle 
	      | \prefix{x}{y}{(\outputp{x}{y} | @{y})} & \nonumber\\
	\red
	& (\outputp{x}{y} | @{y})\substn{\quotep{(\prefix{x}{y}{(@{y} | \outputp{x}{y})) | P}}}{y} & \nonumber\\
	=
	& \outputp{x}{\quotep{(\prefix{x}{y}{(\outputp{x}{y} | @{y})) | P}}}
	  | {(\prefix{x}{y}{(\outputp{x}{y} | @{y})) | P}} & \nonumber\\
	\red
	& \ldots & \nonumber\\
	\red^*
	& P | P | \ldots & \nonumber
\end{eqnarray}

Of course, this encoding, as an implementation, runs away, unfolding
$\bangp{P}$ eagerly. A lazier and more implementable replication
operator, restricted to input-guarded processes, may be obtained as follows.

\begin{eqnarray}
\bangp{\prefix{u}{v}{P}} 
	:= 
	\binpar{\lift{x}{\prefix{u}{v}{(\binpar{D(x)}{P})}}}{D(x)} \nonumber
\end{eqnarray}

\begin{remark}
  Note that the lazier definition still does not deal with summation
  or mixed summation (i.e. sums over input and output). The reader is
  invited to construct definitions of replication that deal with these
  features. 

  Further, the definitions are parameterized in a name, $x$. Can you,
  gentle reader, make a definition that eliminates this parameter and
  guarantees no accidental interaction between the replication
  machinery and the process being replicated -- i.e. no accidental
  sharing of names used by the process to get its work done and the
  name(s) used by the replication to effect copying. This latter
  revision of the definition of replication is crucial to obtaining
  the expected identity $!!P \sim !P$.
\end{remark}

\begin{remark}\label{rem:paradoxical_combinator}
  The reader familiar with the lambda calculus will have noticed the
  similarity between $D$ and the paradoxical combinator.

  [Ed. note: the existence of this seems to suggest we have to be more
  restrictive on the set of processes and names we admit if we are to
  support no-cloning.]
\end{remark}

\subsubsection{Bisimulation}

The computational dynamics gives rise to another kind of equivalence,
the equivalence of computational behavior. As previously mentioned
this is typically captured \emph{via} some form of bisimulation.

% The notion we use in this paper is weak barbed bisimulation
% \cite{milner91polyadicpi}.

The notion we use in this paper is derived from weak barbed
bisimulation \cite{milner91polyadicpi}. 

\begin{definition}
An \emph{observation relation}, $\downarrow_{\mathcal N}$, over a set
of names, $\mathcal N$, is the smallest relation satisfying the rules
below.

\infrule[Out-barb]{y \in {\mathcal N}, \; x \nameeq y}
		  {\outputp{x}{v} \downarrow_{\mathcal N} x}
\infrule[Par-barb]{\mbox{$P\downarrow_{\mathcal N} x$ or $Q\downarrow_{\mathcal N} x$}}
		  {\binpar{P}{Q} \downarrow_{\mathcal N} x}

We write $P \Downarrow_{\mathcal N} x$ if there is $Q$ such that 
$P \wred Q$ and $Q \downarrow_{\mathcal N} x$.
\end{definition}

\begin{definition}
%\label{def.bbisim}
An  ${\mathcal N}$-\emph{barbed bisimulation} over a set of names, ${\mathcal N}$, is a symmetric binary relation 
${\mathcal S}_{\mathcal N}$ between agents such that $P\rel{S}_{\mathcal N}Q$ implies:
\begin{enumerate}
\item If $P \red P'$ then $Q \wred Q'$ and $P'\rel{S}_{\mathcal N} Q'$.
\item If $P\downarrow_{\mathcal N} x$, then $Q\Downarrow_{\mathcal N} x$.
\end{enumerate}
$P$ is ${\mathcal N}$-barbed bisimilar to $Q$, written
$P \wbbisim_{\mathcal N} Q$, if $P \rel{S}_{\mathcal N} Q$ for some ${\mathcal N}$-barbed bisimulation ${\mathcal S}_{\mathcal N}$.
\end{definition}

$\mathcal{R} \subseteq \pi \times \pi$

$P \mathcal{R} Q => \forall P'. P \red P' \Rightarrow \exists Q'. Q \red Q', P' \mathcal{R} Q'$

$P \vdash x \Rightarrow Q \vdash x$

\begin{mathpar}
  \inferrule*[lab=Out-barb]{x \nameeq y}{{y}!\langle{Q}\rangle \vdash x}
  \and
  \inferrule*[lab=Par-barb]{\mbox{$P\vdash x$ or $Q\vdash x$}}{\binpar{P}{Q} \vdash x}
\end{mathpar}

\subsubsection{Contexts}

One of the principle advantages of computational calculi like the
$\pi$-calculus is a well-defined notion of context,
contextual-equivalence and a correlation between
contextual-equivalence and notions of bisimulation. The notion of
context allows the decomposition of a process into (sub-)process and
its syntactic environment, its context. Thus, a context may be
thought of as a process with a ``hole'' (written $\Box$) in it. The
application of a context $M$ to a process $P$, written $M[P]$, is
tantamount to filling the hole in $M$ with $P$. In this paper we do
not need the full weight of this theory, but do make use of the notion
of context in the proof the main theorem. 

\begin{mathpar}
  \inferrule* [lab=summation] {} {{M_{M},M_{N}} \bc \Box \;|\; x.M_{A} \;|\; M_{M}+M_{N}}
  \and
  \inferrule* [lab=agent] {} {{M_{A}} \bc (\vec{x})M_{P} \;| \; \clift{P_0,\ldots,M_{P},\ldots,P_N}}
  \and \\
  \inferrule* [lab=process] {} {{M_{P}} \bc M_{N} \;| \;P|M_{P} }
\end{mathpar} 

\begin{mathpar}
  \inferrule* [lab=sychronization] {} {M_{N} \bc \Box \;|\; x?M_{F} \;|\; x!M_{C}}
  \and
  \inferrule* [lab=abstraction] {} {{M_{F}} \bc (x)M_{P} }
  \and
  \inferrule* [lab=concretion] {} {{M_{C}} \bc \langle M_{P} \rangle }
  \and \\
  \inferrule* [lab=process] {} {{M_{P}} \bc M_{N} \;| \;P|M_{P} }
\end{mathpar}

\begin{definition}[contextual application] Given a context $M$, and
  process $P$, we define the \emph{contextual application}, $M[P] :=
  M\{P/\Box\}$. That is, the contextual application of M to P is the
  substitution of $P$ for $\Box$ in $M$.
\end{definition}

$\meaningof{-} : L \to \mathcal{P}(\pi)$

\begin{mathpar}
  \inferrule* [lab=collection] {} {\meaningof{true} = \pi, \and \meaningof{~E} = \pi \setminus \meaningof{E}, \and \meaningof{E_{1} \& E_{2}} = \meaningof{E_{1}} \cap \meaningof{E_{2}}}
\end{mathpar}

\begin{mathpar}
  \inferrule* [lab=structure] {} {\meaningof{0} = \{ P \in \pi | P \equiv 0 \}, \and \\ \meaningof{E_1 | E_2} = \{ P \in \pi | P \equiv P_{1} | P_{2}, P_{1} \in \meaningof{E_{1}}, P_{2} \in \meaningof{E_2}\} }
\end{mathpar}

\begin{mathpar}
 \inferrule* [lab=behavior] {} {\meaningof{\langle a?b \rangle E} = \{ P \in \pi | P \equiv Q | u?(y)P', \\ \and \\\\ \and \\ \;\;\; u \in \meaningof{a}, \forall z.P'\{z/y\} \in \meaningof{E\{z/b\}}\}, \and \\ \meaningof{a!E} = \{ P \in \pi | P \equiv Q | x!\langle P' \rangle, x \in \meaningof{a} P' \in \meaningof{E}\} }
\end{mathpar}

\begin{mathpar}
 \inferrule* [lab=nominal] {} {\meaningof{\quotep{E}} = \{ \quotep{P} \in \quotep{\pi} | P \in \meaningof{E} \}, \and \meaningof{\quotep{P}} = \{ \quotep{Q} \in \quotep{\pi} | P \equiv Q \} \and \\ \meaningof{@\quotep{E}} = \{ P \in \pi | P \equiv @x, x \in \meaningof{E} \}}
\end{mathpar}

\begin{eqnarray*}
  \\
  \meaningof{-} : TS \to ST
\end{eqnarray*}

\begin{eqnarray*}
  \\
  L : TS \to ST
\end{eqnarray*}

\begin{eqnarray*}
  \\
  P \models E \iff P \in \meaningof{E}
\end{eqnarray*}

\begin{eqnarray*}
  P \approx_{L} Q \iff \forall E \in L. P \models E \iff Q \models E
\end{eqnarray*}

\begin{eqnarray*}
  P \approx_{K} Q
\end{eqnarray*}

\begin{eqnarray*}
  P \approx Q
\end{eqnarray*}

$\approx_{K} = \approx = \approx_{L}$

\subsubsection{Contextual duality}

Note that contexts extend the quotation operation to a family of
operations from processes to names. Given a context, $M$, we can
define a \emph{nominal context}, $\quotep{M}$ by $\quotep{M}[P] :=
\quotep{M[P]}$. To foreshadow what is to come we observe that these
operations enjoy a duality with processes very much like the duality
between vectors and maps from vectors to scalars.

Further, because the calculus is essentially higher-order, we have a
correspondence between contexts and processes. More specifically,
given a name $x$ and a context $M$ we can construct $M^{*}_{x}$ such
that 

\begin{mathpar}
  M^{*}_{x} | \lift{x}{P} \red M[P]
\end{mathpar}

namely,

\begin{mathpar}
  M^{*}_{x} := x?(u).M[\dropn{u}]
\end{mathpar}

The dependence of $M^{*}_{x}$ on a name makes it an abstraction, 

\begin{mathpar}
  M^{*} := (x)x?(u).M[\dropn{u}]
\end{mathpar}

\subsection{Additional notation}

It will sometimes be convenient to denote the process a name
quotes. We already have the notation $x = \quotep{P}$, but it will be
convenient to introduce an alternate notation, $\procn{x}$, when we
want to emphasize the connection to the use of the name. Note that, by
virtue of name equivalence, $\quotep{\procn{x}} \nameeq x$; so, the
notation is consistent with previous definitions.

Further, because names have structure it is possible to effect
substitutions on the basis of that structure. This means we need to
upgrade our notation for substitutions, which we accomplish by
adapting comprehension notation. Thus,

\begin{mathpar}
  P\{ y / x : x \in S \}
\end{mathpar}

is interpreted to mean the process derived from P by replacing (in a
capture-avoiding manner) each occurrence of $x$ in $S$ by $y$. For example,

\begin{mathpar}
  P\{ \quotep{\procn{x}|\procn{x}} / x : x \in \freenames{P} \}
\end{mathpar}

will replace each (occurrence) of a free name $x$ in $P$ by
$\quotep{\procn{x}|\procn{x}}$.

Also, we will avail ourselves of the notation $x^{L}$ and $x^{R}$ to
denote injections of a name into disjoint copies of the name
space. There are numerous ways to accomplish this. One example can be
found in \cite{MeredithR05}. This notation overloads to vectors of
names: $\vec{x}^{\pi} := (x_{i}^{\pi} \; : \; 0 \leq i < |\vec{x}| )$ where $\pi \in \{L,R\}$.

We also use $P^{\Box} := P|\Box$.

In \cite{MeredithR05} an interpretation of the new operator is
given. It turns out that there are several possible interpretations
all enjoying the requisite algebraic properties of the operator (see
\cite{milner91polyadicpi}). We will therefore make liberal use of
$(\nu\; \vec{x})P$.

% subsection the_syntax_and_semantics_of_the_notation_system (end)   

\input{qm2pi.qmops} 

\input{qm2pi.sterngerlach} 

\input{qm2pi.metric} 

% section concurrent_process_calculi (end)

%\input{qm2pi.proofsketch}

% section proof sketch (end)

%\input{qm2pi.slviaknots} 

% section spatial logic via knots (end)

\input{qm2pi.conclusion}

% section conclusion (end)

%\input{qm2pi.dtcodes} 

% section wiring algorithm (end)

\input{qm2pi.ack} 

% section acknowledgments (end)

\newpage


\bibliographystyle{plain}   
\bibliography{../../biblios/main.bib}

\input{qm2pi.rhodetails}

\end{document}

 

%\documentclass[12pt]{llncs}
%\documentclass{jktr}

\usepackage[pdftex]{hyperref}                   
\usepackage {listings}
\usepackage {mathpartir}
\usepackage{bcprules}
%\usepackage{listings}
                       
\usepackage{graphicx} 
%\usepackage[margins=2.5cm,nohead,nofoot]{geometry}
%\usepackage{geometry}
\usepackage{amsfonts}
\usepackage{amstext}
\usepackage{latexsym}
\usepackage{amssymb}
\usepackage{color}


%\include{myPreamble}
\include{qm2pi.local} 

%\ifpdf
%\usepackage[pdftex]{graphicx}
%\else
%\usepackage{graphicx}
%\fi

 % \ifpdf
%  \usepackage{pdfsync}
%  \if


%\title{Brief Article}
%\author{David F. Snyder}
%\author{L.G. Meredith}

%\address{Dept. of Math., Texas State University--San Marcos, San Marcos, TX 78666}
       
\pagestyle{empty}


\begin{document}

\lstset{language=[Objective]Caml,frame=shadowbox}

\input{qm2pi.front}

% section front matter (end)

\input{qm2pi.intro} 
 
% section introduction (end)

% \input{qm2pi.knotations} 

% section notation (end)

\input{qm2pi.process.calculi} 

% section concurrent_process_calculi_and_spatial_logics_ (end)
    
%\input{qm2pi.knots2pi} 

%\input{qm2pi.trefoil} 

%\input{qm2pi.mainthm} 

% subsection basic_interpretation (end)

%\input{qm2pi.rho.presentation} 
\subsection{The syntax and semantics of the notation system}\label{sub:the_syntax_and_semantics_of_the_notation_system} % (fold)

We now summarize a technical presentation of the calculus that
embodies our theory of dynamics. The typical presentation of such a
calculus follows the style of giving generators and relations on
them. The grammar, below, describing term constructors, freely
generates the set of processes, $\Proc$. This set is then quotiented
by a relation known as structural congruence and it is over this set
that the notion of dynamics is expressed. This presentation is
essentially that of \cite{MeredithR05} with the addition of
polyadicity and summation. For readability we have relegated some of
the technical subtleties to an appendix.

\subsubsection{Process grammar}\label{subsub:process_grammar}

\begin{mathpar}
  \inferrule* [lab=synchronization] {} {{M} \bc \pzero \;|\; x?F \;|\; x!C }
  \and
  \inferrule* [lab=abstraction] {} {{F} \bc (x)P}
  \and
  \inferrule* [lab=concretion] {} {{C} \bc \langle Q \rangle}
  \and
  \inferrule* [lab=process] {} {{P,Q} \bc M \;| \;P|Q \;|\; @{x}}
  \and
  \inferrule* [lab=name] {} {{x} \bc \quotep{P}}
\end{mathpar} 

Note that $\vec{x}$ (resp. $\vec{P}$) denotes a vector of names
(resp. processes) of length $|\vec{x}|$ (resp. $|\vec{P}|$). We adopt
the following useful abbreviations.

\begin{mathpar}
   x?(\vec{y}).P := x.(\vec{y})P \and  x\clift{\vec{P}} := x.\clift{\vec{P}}
   \and x!(y) := \lift{x}{\dropn{y}}
   \and \Pi_{i=0}^{n-1}P_i := P_0 | \ldots | P_{n-1}
\end{mathpar}

\subsubsection{Structural congruence}

\paragraph{Free and bound names and alpha-equivalence.} At the
core of structural equivalence is alpha-equivalence which identifies
process that are the same up to a change of variable. Formally, we
recognize the distinction between free and bound names. The free names
of a process, $\freenames{P}$, may be calculated recursively as
follows:

\begin{mathpar}
\freenames{\pzero} := \emptyset
  \and \\
  \freenames{x?(y).P} := \{ x \} \cup (\freenames{P} \setminus \{ y \})
  \and 
  \freenames{x!\langle P \rangle} := \{ x \} \cup \{ P \} 
  \and \\
  \freenames{P|Q} := \freenames{P} \cup \freenames{Q}
  \and \\
  \freenames{@{x}} := \{ x \}
\end{mathpar}

$\pi$
$\quotep{\pi}$

$\freenames{-} : \pi \to \mathcal{P}(\quotep{\pi})$

\begin{eqnarray*}
  \freenames{\pzero} & := & \emptyset \\
  \freenames{x?(y).P} & := & \{ x \} \cup (\freenames{P} \setminus \{ y \}) \\
  \freenames{x!\langle P \rangle} & := & \{ x \} \cup \{ P \} \\
  \freenames{P|Q} & := & \freenames{P} \cup \freenames{Q} \\
  \freenames{\dropn{x}} & := & \{ x \}
\end{eqnarray*}

The bound names of a process, $\boundnames{P}$, are those names occurring in $P$
that are not free. For example, in $x?(y).0$, the name $x$ is free, while $y$ is bound.

\begin{mathpar}
  \inferrule* [lab=monoidal-laws] {} { P|Q \equiv Q|P \and P|0 \equiv P \and P|(Q|R) \equiv (P|Q)|R }
\end{mathpar}

\begin{mathpar}
  \inferrule* [lab=alpha-equivalence] {} { (x)P \equiv (y)P\{y/x\} \and y \not\in \freenames{P} }
\end{mathpar}

\begin{definition}
Then two processes, $P,Q$, are alpha-equivalent if $P = Q\{\vec{y}/\vec{x}\}$ for
some $\vec{x} \in \boundnames{Q},\vec{y} \in \boundnames{P}$, where $Q\{\vec{y}/\vec{x}\}$
denotes the capture-avoiding substitution of $\vec{y}$ for $\vec{x}$ in $Q$.
\end{definition}

\begin{definition}
  The {\em structural congruence} \cite{SangiorgiWalker} , $\equiv$,
  between processes is the least congruence containing
  alpha-equivalence, satisfying the abelian monoid laws
  (associativity, commutativity and $\pzero$ as identity) for parallel
  composition $|$ and for summation $+$.
\end{definition}

\subsection{Name equivalence}

We take name equivalence, written $\nameeq$, to be the smallest
equivalence relation generated by the following rules.

\begin{mathpar}
\inferrule*[lab=Quote-drop]
{ }
{ \quotep{@{x}} \nameeq x }

\inferrule*[lab=Struct-equiv]
{ P \scong Q }
{ \quotep{P} \nameeq \quotep{Q} }
\end{mathpar}

The astute reader will have noticed that the mutual recursion of names
and processes imposes a mutual recursion on alpha-equivalence and
structural equivalence via name-equivalence. Fortunately, all of this
works out pleasantly and we may calculate in the natural way, free of
concern. The reader interested in the details is referred to the
appendix \ref{appendix:rho_details}.

\subsection{Substitution}

We use $\Proc$ for the set of processes, $\QProc$ for the set of
names, and $\id{\{}\vec{y} / \vec{x} \id{\}}$ to denote partial maps,
$s : \QProc \rightarrow \QProc$. A map, $s$ lifts, uniquely, to a map
on process terms, $\widehat{s} : \Proc \rightarrow \Proc$ by the
following equations.

\begin{mathpar}
  (0) \psubstp{Q}{P} := 0 \\
  (R \juxtap S) \psubstp{Q}{P}
  :=    
  (R)\psubstp{Q}{P} \juxtap (S) \psubstp{Q}{P} \\
  (x?(y).R) \psubstp{Q}{P}    
  :=    
  (x)\substp{Q}{P} (z)\concat( (R \psubstn{z}{y}) \psubstp{Q}{P} ) \\
  (\lift{x}{R}) \psubstp{Q}{P}  
  :=
  \lift{(x)\substp{Q}{P}}{ R \psubstp{Q}{P} } \\
%   (\dropn{x})  \psubstp{Q}{P}       
%   := 
%   \left\{ 
%     \begin{array}{ccc} 
%       \dropn{\quotep{Q}} & & x \nameeq \quotep{P} \\
%       \dropn{x} & & otherwise \\
%     \end{array}
%   \right. 
  (\dropn{x})  \psubstp{Q}{P}       
  := 
  \left\{ 
    \begin{array}{ccc} 
      Q & & x \nameeq \quotep{P} \\
      \dropn{x} & & otherwise \\
    \end{array}
  \right.
\end{mathpar}
 

where

\begin{eqnarray}
  (x)\id{\{} \lpquote Q \rpquote / \lpquote P \rpquote \id{\}}            = 
  \left\{ 
    \begin{array}{ccc}
      \lpquote Q \rpquote & & x \nameeq \lpquote P \rpquote \\
      x & & otherwise \\
    \end{array}
  \right. \nonumber
\end{eqnarray}

and $z$ is chosen distinct from $\quotep{P}$, $\quotep{Q}$, the free
names in $Q$, and all the names in $R$. Our $\alpha$-equivalence will
be built in the standard way from this substitution.

\begin{remark}\label{rem:no_self_referential_names}
  One consequence of these definitions is that $\forall P. \quotep{P}
  \not\in \freenames{P}$.
\end{remark}

\subsection{ Dynamic quote: an example }

Anticipating something of what's to come, consider applying the
substitution, $\widehat{\id{\{}u / z \id{\}}}$, to the following pair
of processes, $\lift{w}{y!(z)}$ and $w[ \lpquote y!(z) \rpquote ]$.

\begin{eqnarray}
	\lift{w}{y!(z)}\widehat{\id{\{}u / z \id{\}}}
		& = &
		\lift{w}{y!(u)} \nonumber\\
	w[ \lpquote y!(z) \rpquote ] \widehat{ \id{\{}u / z \id{\}} }
		& = &
		w[ \lpquote y!(z) \rpquote ] \nonumber
\end{eqnarray}

Because the body of the process between quotes is impervious to
substitution, we get radically different answers. In fact, by
examining the first process in an input context,
e.g. $x?(z).\lift{w}{y!(z)}$, we see that the process under the lift
operator may be shaped by prefixed inputs binding a name inside it. In
this sense, the lift operator will be seen as a way to dynamically
construct processes before reifying them as names.

Finally equipped with these standard features we can present the
dynamics of the calculus.

\subsubsection{Operational semantics} 

Finally, we introduce the computational dynamics. What marks these
algebras as distinct from other more traditionally studied algebraic
structures, e.g. vector spaces or polynomial rings, is the manner in
which dynamics is captured. In traditional structures, dynamics is typically
expressed through morphisms between such structures, as in linear maps
between vector spaces or morphisms between rings. In algebras
associated with the semantics of computation, the dynamics is
expressed as part of the algebraic structure itself, through a
reduction reduction relation typically denoted by $\red$. Below, we
give a recursive presentation of this relation for the calculus used
in the encoding.

$\red \subseteq \pi \times \pi$
$\red : \pi \to \mathcal{P}(\pi)$

\begin{mathpar}
  \inferrule* [lab=Comm] { \textsf{match}( x_{src}, x_{trgt} ) } { x_{trgt}?(y)P \; | \; x_{src}!\langle {Q} \rangle \red P\{\quotep{Q}/y}\} }
  \and \\
  \inferrule* [lab=Par] {{P} \red {P}'} {{{P} | {Q}} \red {{P}' | {Q}}}
  \and
  \inferrule* [lab=Equiv]{{{P} \scong {P}'} \andalso {{P}' \red {Q}'} \andalso {{Q}' \scong {Q}}}{{P} \red {Q}}
\end{mathpar}

\begin{eqnarray*}
  match_{\equiv} (\quotep{P},\quotep{Q}) & := & P \equiv Q \\
  match_{\dagger}(\quotep{P},\quotep{Q}) & := & \forall R. P|Q \red^{*} R => R \red^{*} 0 \\
  match_{K}(\quotep{P},\quotep{Q}) & := & K \mbox{ for some context } K
\end{eqnarray*}

$u?(x)P | u!\langle Q \rangle \red P\{\quotep{Q}/x\}$

%We write $\wred$ for $\red^*$, and $P\red$ if $\exists Q $ such that $ P \red Q$.
We write $P\red$ if $\exists Q $ such that $ P \red Q$ and $P\not\red$, otherwise.

\section{Replication}

As mentioned before, it is known that replication (and hence
recursion) can be implemented in a higher-order process algebra
\cite{SangiorgiWalker}. As our first example of calculation with the
machinery thus far presented we give the construction explicitly in
the {\rhoc}.

\begin{eqnarray}
	D_{x} & := & \prefix{x}{y}{(\binpar{\outputp{x}{y}}{@{y}})} \nonumber\\
	\bangp_{x}{P} & := & \binpar{{x}!\langle{\binpar{D_{x}}{P}}\rangle}{D_{x}} \nonumber
\end{eqnarray}

\begin{eqnarray}
	\bangp_{x}{P} & & \nonumber\\
	=
	& {x}!\langle{(\prefix{x}{y}{(\outputp{x}{y} | @{y})) | P}}\rangle 
	      | \prefix{x}{y}{(\outputp{x}{y} | @{y})} & \nonumber\\
	\red
	& (\outputp{x}{y} | @{y})\substn{\quotep{(\prefix{x}{y}{(@{y} | \outputp{x}{y})) | P}}}{y} & \nonumber\\
	=
	& \outputp{x}{\quotep{(\prefix{x}{y}{(\outputp{x}{y} | @{y})) | P}}}
	  | {(\prefix{x}{y}{(\outputp{x}{y} | @{y})) | P}} & \nonumber\\
	\red
	& \ldots & \nonumber\\
	\red^*
	& P | P | \ldots & \nonumber
\end{eqnarray}

Of course, this encoding, as an implementation, runs away, unfolding
$\bangp{P}$ eagerly. A lazier and more implementable replication
operator, restricted to input-guarded processes, may be obtained as follows.

\begin{eqnarray}
\bangp{\prefix{u}{v}{P}} 
	:= 
	\binpar{\lift{x}{\prefix{u}{v}{(\binpar{D(x)}{P})}}}{D(x)} \nonumber
\end{eqnarray}

\begin{remark}
  Note that the lazier definition still does not deal with summation
  or mixed summation (i.e. sums over input and output). The reader is
  invited to construct definitions of replication that deal with these
  features. 

  Further, the definitions are parameterized in a name, $x$. Can you,
  gentle reader, make a definition that eliminates this parameter and
  guarantees no accidental interaction between the replication
  machinery and the process being replicated -- i.e. no accidental
  sharing of names used by the process to get its work done and the
  name(s) used by the replication to effect copying. This latter
  revision of the definition of replication is crucial to obtaining
  the expected identity $!!P \sim !P$.
\end{remark}

\begin{remark}\label{rem:paradoxical_combinator}
  The reader familiar with the lambda calculus will have noticed the
  similarity between $D$ and the paradoxical combinator.

  [Ed. note: the existence of this seems to suggest we have to be more
  restrictive on the set of processes and names we admit if we are to
  support no-cloning.]
\end{remark}

\subsubsection{Bisimulation}

The computational dynamics gives rise to another kind of equivalence,
the equivalence of computational behavior. As previously mentioned
this is typically captured \emph{via} some form of bisimulation.

% The notion we use in this paper is weak barbed bisimulation
% \cite{milner91polyadicpi}.

The notion we use in this paper is derived from weak barbed
bisimulation \cite{milner91polyadicpi}. 

\begin{definition}
An \emph{observation relation}, $\downarrow_{\mathcal N}$, over a set
of names, $\mathcal N$, is the smallest relation satisfying the rules
below.

\infrule[Out-barb]{y \in {\mathcal N}, \; x \nameeq y}
		  {\outputp{x}{v} \downarrow_{\mathcal N} x}
\infrule[Par-barb]{\mbox{$P\downarrow_{\mathcal N} x$ or $Q\downarrow_{\mathcal N} x$}}
		  {\binpar{P}{Q} \downarrow_{\mathcal N} x}

We write $P \Downarrow_{\mathcal N} x$ if there is $Q$ such that 
$P \wred Q$ and $Q \downarrow_{\mathcal N} x$.
\end{definition}

\begin{definition}
%\label{def.bbisim}
An  ${\mathcal N}$-\emph{barbed bisimulation} over a set of names, ${\mathcal N}$, is a symmetric binary relation 
${\mathcal S}_{\mathcal N}$ between agents such that $P\rel{S}_{\mathcal N}Q$ implies:
\begin{enumerate}
\item If $P \red P'$ then $Q \wred Q'$ and $P'\rel{S}_{\mathcal N} Q'$.
\item If $P\downarrow_{\mathcal N} x$, then $Q\Downarrow_{\mathcal N} x$.
\end{enumerate}
$P$ is ${\mathcal N}$-barbed bisimilar to $Q$, written
$P \wbbisim_{\mathcal N} Q$, if $P \rel{S}_{\mathcal N} Q$ for some ${\mathcal N}$-barbed bisimulation ${\mathcal S}_{\mathcal N}$.
\end{definition}

$\mathcal{R} \subseteq \pi \times \pi$

$P \mathcal{R} Q => \forall P'. P \red P' \Rightarrow \exists Q'. Q \red Q', P' \mathcal{R} Q'$

$P \vdash x \Rightarrow Q \vdash x$

\begin{mathpar}
  \inferrule*[lab=Out-barb]{x \nameeq y}{{y}!\langle{Q}\rangle \vdash x}
  \and
  \inferrule*[lab=Par-barb]{\mbox{$P\vdash x$ or $Q\vdash x$}}{\binpar{P}{Q} \vdash x}
\end{mathpar}

\subsubsection{Contexts}

One of the principle advantages of computational calculi like the
$\pi$-calculus is a well-defined notion of context,
contextual-equivalence and a correlation between
contextual-equivalence and notions of bisimulation. The notion of
context allows the decomposition of a process into (sub-)process and
its syntactic environment, its context. Thus, a context may be
thought of as a process with a ``hole'' (written $\Box$) in it. The
application of a context $M$ to a process $P$, written $M[P]$, is
tantamount to filling the hole in $M$ with $P$. In this paper we do
not need the full weight of this theory, but do make use of the notion
of context in the proof the main theorem. 

\begin{mathpar}
  \inferrule* [lab=summation] {} {{M_{M},M_{N}} \bc \Box \;|\; x.M_{A} \;|\; M_{M}+M_{N}}
  \and
  \inferrule* [lab=agent] {} {{M_{A}} \bc (\vec{x})M_{P} \;| \; \clift{P_0,\ldots,M_{P},\ldots,P_N}}
  \and \\
  \inferrule* [lab=process] {} {{M_{P}} \bc M_{N} \;| \;P|M_{P} }
\end{mathpar} 

\begin{mathpar}
  \inferrule* [lab=sychronization] {} {M_{N} \bc \Box \;|\; x?M_{F} \;|\; x!M_{C}}
  \and
  \inferrule* [lab=abstraction] {} {{M_{F}} \bc (x)M_{P} }
  \and
  \inferrule* [lab=concretion] {} {{M_{C}} \bc \langle M_{P} \rangle }
  \and \\
  \inferrule* [lab=process] {} {{M_{P}} \bc M_{N} \;| \;P|M_{P} }
\end{mathpar}

\begin{definition}[contextual application] Given a context $M$, and
  process $P$, we define the \emph{contextual application}, $M[P] :=
  M\{P/\Box\}$. That is, the contextual application of M to P is the
  substitution of $P$ for $\Box$ in $M$.
\end{definition}

$\meaningof{-} : L \to \mathcal{P}(\pi)$

\begin{mathpar}
  \inferrule* [lab=collection] {} {\meaningof{true} = \pi, \and \meaningof{~E} = \pi \setminus \meaningof{E}, \and \meaningof{E_{1} \& E_{2}} = \meaningof{E_{1}} \cap \meaningof{E_{2}}}
\end{mathpar}

\begin{mathpar}
  \inferrule* [lab=structure] {} {\meaningof{0} = \{ P \in \pi | P \equiv 0 \}, \and \\ \meaningof{E_1 | E_2} = \{ P \in \pi | P \equiv P_{1} | P_{2}, P_{1} \in \meaningof{E_{1}}, P_{2} \in \meaningof{E_2}\} }
\end{mathpar}

\begin{mathpar}
 \inferrule* [lab=behavior] {} {\meaningof{\langle a?b \rangle E} = \{ P \in \pi | P \equiv Q | u?(y)P', \\ \and \\\\ \and \\ \;\;\; u \in \meaningof{a}, \forall z.P'\{z/y\} \in \meaningof{E\{z/b\}}\}, \and \\ \meaningof{a!E} = \{ P \in \pi | P \equiv Q | x!\langle P' \rangle, x \in \meaningof{a} P' \in \meaningof{E}\} }
\end{mathpar}

\begin{mathpar}
 \inferrule* [lab=nominal] {} {\meaningof{\quotep{E}} = \{ \quotep{P} \in \quotep{\pi} | P \in \meaningof{E} \}, \and \meaningof{\quotep{P}} = \{ \quotep{Q} \in \quotep{\pi} | P \equiv Q \} \and \\ \meaningof{@\quotep{E}} = \{ P \in \pi | P \equiv @x, x \in \meaningof{E} \}}
\end{mathpar}

\begin{eqnarray*}
  \\
  \meaningof{-} : TS \to ST
\end{eqnarray*}

\begin{eqnarray*}
  \\
  L : TS \to ST
\end{eqnarray*}

\begin{eqnarray*}
  \\
  P \models E \iff P \in \meaningof{E}
\end{eqnarray*}

\begin{eqnarray*}
  P \approx_{L} Q \iff \forall E \in L. P \models E \iff Q \models E
\end{eqnarray*}

\begin{eqnarray*}
  P \approx_{K} Q
\end{eqnarray*}

\begin{eqnarray*}
  P \approx Q
\end{eqnarray*}

$\approx_{K} = \approx = \approx_{L}$

\subsubsection{Contextual duality}

Note that contexts extend the quotation operation to a family of
operations from processes to names. Given a context, $M$, we can
define a \emph{nominal context}, $\quotep{M}$ by $\quotep{M}[P] :=
\quotep{M[P]}$. To foreshadow what is to come we observe that these
operations enjoy a duality with processes very much like the duality
between vectors and maps from vectors to scalars.

Further, because the calculus is essentially higher-order, we have a
correspondence between contexts and processes. More specifically,
given a name $x$ and a context $M$ we can construct $M^{*}_{x}$ such
that 

\begin{mathpar}
  M^{*}_{x} | \lift{x}{P} \red M[P]
\end{mathpar}

namely,

\begin{mathpar}
  M^{*}_{x} := x?(u).M[\dropn{u}]
\end{mathpar}

The dependence of $M^{*}_{x}$ on a name makes it an abstraction, 

\begin{mathpar}
  M^{*} := (x)x?(u).M[\dropn{u}]
\end{mathpar}

\subsection{Additional notation}

It will sometimes be convenient to denote the process a name
quotes. We already have the notation $x = \quotep{P}$, but it will be
convenient to introduce an alternate notation, $\procn{x}$, when we
want to emphasize the connection to the use of the name. Note that, by
virtue of name equivalence, $\quotep{\procn{x}} \nameeq x$; so, the
notation is consistent with previous definitions.

Further, because names have structure it is possible to effect
substitutions on the basis of that structure. This means we need to
upgrade our notation for substitutions, which we accomplish by
adapting comprehension notation. Thus,

\begin{mathpar}
  P\{ y / x : x \in S \}
\end{mathpar}

is interpreted to mean the process derived from P by replacing (in a
capture-avoiding manner) each occurrence of $x$ in $S$ by $y$. For example,

\begin{mathpar}
  P\{ \quotep{\procn{x}|\procn{x}} / x : x \in \freenames{P} \}
\end{mathpar}

will replace each (occurrence) of a free name $x$ in $P$ by
$\quotep{\procn{x}|\procn{x}}$.

Also, we will avail ourselves of the notation $x^{L}$ and $x^{R}$ to
denote injections of a name into disjoint copies of the name
space. There are numerous ways to accomplish this. One example can be
found in \cite{MeredithR05}. This notation overloads to vectors of
names: $\vec{x}^{\pi} := (x_{i}^{\pi} \; : \; 0 \leq i < |\vec{x}| )$ where $\pi \in \{L,R\}$.

We also use $P^{\Box} := P|\Box$.

In \cite{MeredithR05} an interpretation of the new operator is
given. It turns out that there are several possible interpretations
all enjoying the requisite algebraic properties of the operator (see
\cite{milner91polyadicpi}). We will therefore make liberal use of
$(\nu\; \vec{x})P$.

% subsection the_syntax_and_semantics_of_the_notation_system (end)   

\input{qm2pi.qmops} 

\input{qm2pi.sterngerlach} 

\input{qm2pi.metric} 

% section concurrent_process_calculi (end)

%\input{qm2pi.proofsketch}

% section proof sketch (end)

%\input{qm2pi.slviaknots} 

% section spatial logic via knots (end)

\input{qm2pi.conclusion}

% section conclusion (end)

%\input{qm2pi.dtcodes} 

% section wiring algorithm (end)

\input{qm2pi.ack} 

% section acknowledgments (end)

\newpage


\bibliographystyle{plain}   
\bibliography{../../biblios/main.bib}

\input{qm2pi.rhodetails}

\end{document}

 

% subsection basic_interpretation (end)

%\input{qm2pi.rho.presentation} 
\subsection{The syntax and semantics of the notation system}\label{sub:the_syntax_and_semantics_of_the_notation_system} % (fold)

We now summarize a technical presentation of the calculus that
embodies our theory of dynamics. The typical presentation of such a
calculus follows the style of giving generators and relations on
them. The grammar, below, describing term constructors, freely
generates the set of processes, $\Proc$. This set is then quotiented
by a relation known as structural congruence and it is over this set
that the notion of dynamics is expressed. This presentation is
essentially that of \cite{MeredithR05} with the addition of
polyadicity and summation. For readability we have relegated some of
the technical subtleties to an appendix.

\subsubsection{Process grammar}\label{subsub:process_grammar}

\begin{mathpar}
  \inferrule* [lab=synchronization] {} {{M} \bc \pzero \;|\; x?F \;|\; x!C }
  \and
  \inferrule* [lab=abstraction] {} {{F} \bc (x)P}
  \and
  \inferrule* [lab=concretion] {} {{C} \bc \langle Q \rangle}
  \and
  \inferrule* [lab=process] {} {{P,Q} \bc M \;| \;P|Q \;|\; @{x}}
  \and
  \inferrule* [lab=name] {} {{x} \bc \quotep{P}}
\end{mathpar} 

Note that $\vec{x}$ (resp. $\vec{P}$) denotes a vector of names
(resp. processes) of length $|\vec{x}|$ (resp. $|\vec{P}|$). We adopt
the following useful abbreviations.

\begin{mathpar}
   x?(\vec{y}).P := x.(\vec{y})P \and  x\clift{\vec{P}} := x.\clift{\vec{P}}
   \and x!(y) := \lift{x}{\dropn{y}}
   \and \Pi_{i=0}^{n-1}P_i := P_0 | \ldots | P_{n-1}
\end{mathpar}

\subsubsection{Structural congruence}

\paragraph{Free and bound names and alpha-equivalence.} At the
core of structural equivalence is alpha-equivalence which identifies
process that are the same up to a change of variable. Formally, we
recognize the distinction between free and bound names. The free names
of a process, $\freenames{P}$, may be calculated recursively as
follows:

\begin{mathpar}
\freenames{\pzero} := \emptyset
  \and \\
  \freenames{x?(y).P} := \{ x \} \cup (\freenames{P} \setminus \{ y \})
  \and 
  \freenames{x!\langle P \rangle} := \{ x \} \cup \{ P \} 
  \and \\
  \freenames{P|Q} := \freenames{P} \cup \freenames{Q}
  \and \\
  \freenames{@{x}} := \{ x \}
\end{mathpar}

$\pi$
$\quotep{\pi}$

$\freenames{-} : \pi \to \mathcal{P}(\quotep{\pi})$

\begin{eqnarray*}
  \freenames{\pzero} & := & \emptyset \\
  \freenames{x?(y).P} & := & \{ x \} \cup (\freenames{P} \setminus \{ y \}) \\
  \freenames{x!\langle P \rangle} & := & \{ x \} \cup \{ P \} \\
  \freenames{P|Q} & := & \freenames{P} \cup \freenames{Q} \\
  \freenames{\dropn{x}} & := & \{ x \}
\end{eqnarray*}

The bound names of a process, $\boundnames{P}$, are those names occurring in $P$
that are not free. For example, in $x?(y).0$, the name $x$ is free, while $y$ is bound.

\begin{mathpar}
  \inferrule* [lab=monoidal-laws] {} { P|Q \equiv Q|P \and P|0 \equiv P \and P|(Q|R) \equiv (P|Q)|R }
\end{mathpar}

\begin{mathpar}
  \inferrule* [lab=alpha-equivalence] {} { (x)P \equiv (y)P\{y/x\} \and y \not\in \freenames{P} }
\end{mathpar}

\begin{definition}
Then two processes, $P,Q$, are alpha-equivalent if $P = Q\{\vec{y}/\vec{x}\}$ for
some $\vec{x} \in \boundnames{Q},\vec{y} \in \boundnames{P}$, where $Q\{\vec{y}/\vec{x}\}$
denotes the capture-avoiding substitution of $\vec{y}$ for $\vec{x}$ in $Q$.
\end{definition}

\begin{definition}
  The {\em structural congruence} \cite{SangiorgiWalker} , $\equiv$,
  between processes is the least congruence containing
  alpha-equivalence, satisfying the abelian monoid laws
  (associativity, commutativity and $\pzero$ as identity) for parallel
  composition $|$ and for summation $+$.
\end{definition}

\subsection{Name equivalence}

We take name equivalence, written $\nameeq$, to be the smallest
equivalence relation generated by the following rules.

\begin{mathpar}
\inferrule*[lab=Quote-drop]
{ }
{ \quotep{@{x}} \nameeq x }

\inferrule*[lab=Struct-equiv]
{ P \scong Q }
{ \quotep{P} \nameeq \quotep{Q} }
\end{mathpar}

The astute reader will have noticed that the mutual recursion of names
and processes imposes a mutual recursion on alpha-equivalence and
structural equivalence via name-equivalence. Fortunately, all of this
works out pleasantly and we may calculate in the natural way, free of
concern. The reader interested in the details is referred to the
appendix \ref{appendix:rho_details}.

\subsection{Substitution}

We use $\Proc$ for the set of processes, $\QProc$ for the set of
names, and $\id{\{}\vec{y} / \vec{x} \id{\}}$ to denote partial maps,
$s : \QProc \rightarrow \QProc$. A map, $s$ lifts, uniquely, to a map
on process terms, $\widehat{s} : \Proc \rightarrow \Proc$ by the
following equations.

\begin{mathpar}
  (0) \psubstp{Q}{P} := 0 \\
  (R \juxtap S) \psubstp{Q}{P}
  :=    
  (R)\psubstp{Q}{P} \juxtap (S) \psubstp{Q}{P} \\
  (x?(y).R) \psubstp{Q}{P}    
  :=    
  (x)\substp{Q}{P} (z)\concat( (R \psubstn{z}{y}) \psubstp{Q}{P} ) \\
  (\lift{x}{R}) \psubstp{Q}{P}  
  :=
  \lift{(x)\substp{Q}{P}}{ R \psubstp{Q}{P} } \\
%   (\dropn{x})  \psubstp{Q}{P}       
%   := 
%   \left\{ 
%     \begin{array}{ccc} 
%       \dropn{\quotep{Q}} & & x \nameeq \quotep{P} \\
%       \dropn{x} & & otherwise \\
%     \end{array}
%   \right. 
  (\dropn{x})  \psubstp{Q}{P}       
  := 
  \left\{ 
    \begin{array}{ccc} 
      Q & & x \nameeq \quotep{P} \\
      \dropn{x} & & otherwise \\
    \end{array}
  \right.
\end{mathpar}
 

where

\begin{eqnarray}
  (x)\id{\{} \lpquote Q \rpquote / \lpquote P \rpquote \id{\}}            = 
  \left\{ 
    \begin{array}{ccc}
      \lpquote Q \rpquote & & x \nameeq \lpquote P \rpquote \\
      x & & otherwise \\
    \end{array}
  \right. \nonumber
\end{eqnarray}

and $z$ is chosen distinct from $\quotep{P}$, $\quotep{Q}$, the free
names in $Q$, and all the names in $R$. Our $\alpha$-equivalence will
be built in the standard way from this substitution.

\begin{remark}\label{rem:no_self_referential_names}
  One consequence of these definitions is that $\forall P. \quotep{P}
  \not\in \freenames{P}$.
\end{remark}

\subsection{ Dynamic quote: an example }

Anticipating something of what's to come, consider applying the
substitution, $\widehat{\id{\{}u / z \id{\}}}$, to the following pair
of processes, $\lift{w}{y!(z)}$ and $w[ \lpquote y!(z) \rpquote ]$.

\begin{eqnarray}
	\lift{w}{y!(z)}\widehat{\id{\{}u / z \id{\}}}
		& = &
		\lift{w}{y!(u)} \nonumber\\
	w[ \lpquote y!(z) \rpquote ] \widehat{ \id{\{}u / z \id{\}} }
		& = &
		w[ \lpquote y!(z) \rpquote ] \nonumber
\end{eqnarray}

Because the body of the process between quotes is impervious to
substitution, we get radically different answers. In fact, by
examining the first process in an input context,
e.g. $x?(z).\lift{w}{y!(z)}$, we see that the process under the lift
operator may be shaped by prefixed inputs binding a name inside it. In
this sense, the lift operator will be seen as a way to dynamically
construct processes before reifying them as names.

Finally equipped with these standard features we can present the
dynamics of the calculus.

\subsubsection{Operational semantics} 

Finally, we introduce the computational dynamics. What marks these
algebras as distinct from other more traditionally studied algebraic
structures, e.g. vector spaces or polynomial rings, is the manner in
which dynamics is captured. In traditional structures, dynamics is typically
expressed through morphisms between such structures, as in linear maps
between vector spaces or morphisms between rings. In algebras
associated with the semantics of computation, the dynamics is
expressed as part of the algebraic structure itself, through a
reduction reduction relation typically denoted by $\red$. Below, we
give a recursive presentation of this relation for the calculus used
in the encoding.

$\red \subseteq \pi \times \pi$
$\red : \pi \to \mathcal{P}(\pi)$

\begin{mathpar}
  \inferrule* [lab=Comm] { \textsf{match}( x_{src}, x_{trgt} ) } { x_{trgt}?(y)P \; | \; x_{src}!\langle {Q} \rangle \red P\{\quotep{Q}/y}\} }
  \and \\
  \inferrule* [lab=Par] {{P} \red {P}'} {{{P} | {Q}} \red {{P}' | {Q}}}
  \and
  \inferrule* [lab=Equiv]{{{P} \scong {P}'} \andalso {{P}' \red {Q}'} \andalso {{Q}' \scong {Q}}}{{P} \red {Q}}
\end{mathpar}

\begin{eqnarray*}
  match_{\equiv} (\quotep{P},\quotep{Q}) & := & P \equiv Q \\
  match_{\dagger}(\quotep{P},\quotep{Q}) & := & \forall R. P|Q \red^{*} R => R \red^{*} 0 \\
  match_{K}(\quotep{P},\quotep{Q}) & := & K \mbox{ for some context } K
\end{eqnarray*}

$u?(x)P | u!\langle Q \rangle \red P\{\quotep{Q}/x\}$

%We write $\wred$ for $\red^*$, and $P\red$ if $\exists Q $ such that $ P \red Q$.
We write $P\red$ if $\exists Q $ such that $ P \red Q$ and $P\not\red$, otherwise.

\section{Replication}

As mentioned before, it is known that replication (and hence
recursion) can be implemented in a higher-order process algebra
\cite{SangiorgiWalker}. As our first example of calculation with the
machinery thus far presented we give the construction explicitly in
the {\rhoc}.

\begin{eqnarray}
	D_{x} & := & \prefix{x}{y}{(\binpar{\outputp{x}{y}}{@{y}})} \nonumber\\
	\bangp_{x}{P} & := & \binpar{{x}!\langle{\binpar{D_{x}}{P}}\rangle}{D_{x}} \nonumber
\end{eqnarray}

\begin{eqnarray}
	\bangp_{x}{P} & & \nonumber\\
	=
	& {x}!\langle{(\prefix{x}{y}{(\outputp{x}{y} | @{y})) | P}}\rangle 
	      | \prefix{x}{y}{(\outputp{x}{y} | @{y})} & \nonumber\\
	\red
	& (\outputp{x}{y} | @{y})\substn{\quotep{(\prefix{x}{y}{(@{y} | \outputp{x}{y})) | P}}}{y} & \nonumber\\
	=
	& \outputp{x}{\quotep{(\prefix{x}{y}{(\outputp{x}{y} | @{y})) | P}}}
	  | {(\prefix{x}{y}{(\outputp{x}{y} | @{y})) | P}} & \nonumber\\
	\red
	& \ldots & \nonumber\\
	\red^*
	& P | P | \ldots & \nonumber
\end{eqnarray}

Of course, this encoding, as an implementation, runs away, unfolding
$\bangp{P}$ eagerly. A lazier and more implementable replication
operator, restricted to input-guarded processes, may be obtained as follows.

\begin{eqnarray}
\bangp{\prefix{u}{v}{P}} 
	:= 
	\binpar{\lift{x}{\prefix{u}{v}{(\binpar{D(x)}{P})}}}{D(x)} \nonumber
\end{eqnarray}

\begin{remark}
  Note that the lazier definition still does not deal with summation
  or mixed summation (i.e. sums over input and output). The reader is
  invited to construct definitions of replication that deal with these
  features. 

  Further, the definitions are parameterized in a name, $x$. Can you,
  gentle reader, make a definition that eliminates this parameter and
  guarantees no accidental interaction between the replication
  machinery and the process being replicated -- i.e. no accidental
  sharing of names used by the process to get its work done and the
  name(s) used by the replication to effect copying. This latter
  revision of the definition of replication is crucial to obtaining
  the expected identity $!!P \sim !P$.
\end{remark}

\begin{remark}\label{rem:paradoxical_combinator}
  The reader familiar with the lambda calculus will have noticed the
  similarity between $D$ and the paradoxical combinator.

  [Ed. note: the existence of this seems to suggest we have to be more
  restrictive on the set of processes and names we admit if we are to
  support no-cloning.]
\end{remark}

\subsubsection{Bisimulation}

The computational dynamics gives rise to another kind of equivalence,
the equivalence of computational behavior. As previously mentioned
this is typically captured \emph{via} some form of bisimulation.

% The notion we use in this paper is weak barbed bisimulation
% \cite{milner91polyadicpi}.

The notion we use in this paper is derived from weak barbed
bisimulation \cite{milner91polyadicpi}. 

\begin{definition}
An \emph{observation relation}, $\downarrow_{\mathcal N}$, over a set
of names, $\mathcal N$, is the smallest relation satisfying the rules
below.

\infrule[Out-barb]{y \in {\mathcal N}, \; x \nameeq y}
		  {\outputp{x}{v} \downarrow_{\mathcal N} x}
\infrule[Par-barb]{\mbox{$P\downarrow_{\mathcal N} x$ or $Q\downarrow_{\mathcal N} x$}}
		  {\binpar{P}{Q} \downarrow_{\mathcal N} x}

We write $P \Downarrow_{\mathcal N} x$ if there is $Q$ such that 
$P \wred Q$ and $Q \downarrow_{\mathcal N} x$.
\end{definition}

\begin{definition}
%\label{def.bbisim}
An  ${\mathcal N}$-\emph{barbed bisimulation} over a set of names, ${\mathcal N}$, is a symmetric binary relation 
${\mathcal S}_{\mathcal N}$ between agents such that $P\rel{S}_{\mathcal N}Q$ implies:
\begin{enumerate}
\item If $P \red P'$ then $Q \wred Q'$ and $P'\rel{S}_{\mathcal N} Q'$.
\item If $P\downarrow_{\mathcal N} x$, then $Q\Downarrow_{\mathcal N} x$.
\end{enumerate}
$P$ is ${\mathcal N}$-barbed bisimilar to $Q$, written
$P \wbbisim_{\mathcal N} Q$, if $P \rel{S}_{\mathcal N} Q$ for some ${\mathcal N}$-barbed bisimulation ${\mathcal S}_{\mathcal N}$.
\end{definition}

$\mathcal{R} \subseteq \pi \times \pi$

$P \mathcal{R} Q => \forall P'. P \red P' \Rightarrow \exists Q'. Q \red Q', P' \mathcal{R} Q'$

$P \vdash x \Rightarrow Q \vdash x$

\begin{mathpar}
  \inferrule*[lab=Out-barb]{x \nameeq y}{{y}!\langle{Q}\rangle \vdash x}
  \and
  \inferrule*[lab=Par-barb]{\mbox{$P\vdash x$ or $Q\vdash x$}}{\binpar{P}{Q} \vdash x}
\end{mathpar}

\subsubsection{Contexts}

One of the principle advantages of computational calculi like the
$\pi$-calculus is a well-defined notion of context,
contextual-equivalence and a correlation between
contextual-equivalence and notions of bisimulation. The notion of
context allows the decomposition of a process into (sub-)process and
its syntactic environment, its context. Thus, a context may be
thought of as a process with a ``hole'' (written $\Box$) in it. The
application of a context $M$ to a process $P$, written $M[P]$, is
tantamount to filling the hole in $M$ with $P$. In this paper we do
not need the full weight of this theory, but do make use of the notion
of context in the proof the main theorem. 

\begin{mathpar}
  \inferrule* [lab=summation] {} {{M_{M},M_{N}} \bc \Box \;|\; x.M_{A} \;|\; M_{M}+M_{N}}
  \and
  \inferrule* [lab=agent] {} {{M_{A}} \bc (\vec{x})M_{P} \;| \; \clift{P_0,\ldots,M_{P},\ldots,P_N}}
  \and \\
  \inferrule* [lab=process] {} {{M_{P}} \bc M_{N} \;| \;P|M_{P} }
\end{mathpar} 

\begin{mathpar}
  \inferrule* [lab=sychronization] {} {M_{N} \bc \Box \;|\; x?M_{F} \;|\; x!M_{C}}
  \and
  \inferrule* [lab=abstraction] {} {{M_{F}} \bc (x)M_{P} }
  \and
  \inferrule* [lab=concretion] {} {{M_{C}} \bc \langle M_{P} \rangle }
  \and \\
  \inferrule* [lab=process] {} {{M_{P}} \bc M_{N} \;| \;P|M_{P} }
\end{mathpar}

\begin{definition}[contextual application] Given a context $M$, and
  process $P$, we define the \emph{contextual application}, $M[P] :=
  M\{P/\Box\}$. That is, the contextual application of M to P is the
  substitution of $P$ for $\Box$ in $M$.
\end{definition}

$\meaningof{-} : L \to \mathcal{P}(\pi)$

\begin{mathpar}
  \inferrule* [lab=collection] {} {\meaningof{true} = \pi, \and \meaningof{~E} = \pi \setminus \meaningof{E}, \and \meaningof{E_{1} \& E_{2}} = \meaningof{E_{1}} \cap \meaningof{E_{2}}}
\end{mathpar}

\begin{mathpar}
  \inferrule* [lab=structure] {} {\meaningof{0} = \{ P \in \pi | P \equiv 0 \}, \and \\ \meaningof{E_1 | E_2} = \{ P \in \pi | P \equiv P_{1} | P_{2}, P_{1} \in \meaningof{E_{1}}, P_{2} \in \meaningof{E_2}\} }
\end{mathpar}

\begin{mathpar}
 \inferrule* [lab=behavior] {} {\meaningof{\langle a?b \rangle E} = \{ P \in \pi | P \equiv Q | u?(y)P', \\ \and \\\\ \and \\ \;\;\; u \in \meaningof{a}, \forall z.P'\{z/y\} \in \meaningof{E\{z/b\}}\}, \and \\ \meaningof{a!E} = \{ P \in \pi | P \equiv Q | x!\langle P' \rangle, x \in \meaningof{a} P' \in \meaningof{E}\} }
\end{mathpar}

\begin{mathpar}
 \inferrule* [lab=nominal] {} {\meaningof{\quotep{E}} = \{ \quotep{P} \in \quotep{\pi} | P \in \meaningof{E} \}, \and \meaningof{\quotep{P}} = \{ \quotep{Q} \in \quotep{\pi} | P \equiv Q \} \and \\ \meaningof{@\quotep{E}} = \{ P \in \pi | P \equiv @x, x \in \meaningof{E} \}}
\end{mathpar}

\begin{eqnarray*}
  \\
  \meaningof{-} : TS \to ST
\end{eqnarray*}

\begin{eqnarray*}
  \\
  L : TS \to ST
\end{eqnarray*}

\begin{eqnarray*}
  \\
  P \models E \iff P \in \meaningof{E}
\end{eqnarray*}

\begin{eqnarray*}
  P \approx_{L} Q \iff \forall E \in L. P \models E \iff Q \models E
\end{eqnarray*}

\begin{eqnarray*}
  P \approx_{K} Q
\end{eqnarray*}

\begin{eqnarray*}
  P \approx Q
\end{eqnarray*}

$\approx_{K} = \approx = \approx_{L}$

\subsubsection{Contextual duality}

Note that contexts extend the quotation operation to a family of
operations from processes to names. Given a context, $M$, we can
define a \emph{nominal context}, $\quotep{M}$ by $\quotep{M}[P] :=
\quotep{M[P]}$. To foreshadow what is to come we observe that these
operations enjoy a duality with processes very much like the duality
between vectors and maps from vectors to scalars.

Further, because the calculus is essentially higher-order, we have a
correspondence between contexts and processes. More specifically,
given a name $x$ and a context $M$ we can construct $M^{*}_{x}$ such
that 

\begin{mathpar}
  M^{*}_{x} | \lift{x}{P} \red M[P]
\end{mathpar}

namely,

\begin{mathpar}
  M^{*}_{x} := x?(u).M[\dropn{u}]
\end{mathpar}

The dependence of $M^{*}_{x}$ on a name makes it an abstraction, 

\begin{mathpar}
  M^{*} := (x)x?(u).M[\dropn{u}]
\end{mathpar}

\subsection{Additional notation}

It will sometimes be convenient to denote the process a name
quotes. We already have the notation $x = \quotep{P}$, but it will be
convenient to introduce an alternate notation, $\procn{x}$, when we
want to emphasize the connection to the use of the name. Note that, by
virtue of name equivalence, $\quotep{\procn{x}} \nameeq x$; so, the
notation is consistent with previous definitions.

Further, because names have structure it is possible to effect
substitutions on the basis of that structure. This means we need to
upgrade our notation for substitutions, which we accomplish by
adapting comprehension notation. Thus,

\begin{mathpar}
  P\{ y / x : x \in S \}
\end{mathpar}

is interpreted to mean the process derived from P by replacing (in a
capture-avoiding manner) each occurrence of $x$ in $S$ by $y$. For example,

\begin{mathpar}
  P\{ \quotep{\procn{x}|\procn{x}} / x : x \in \freenames{P} \}
\end{mathpar}

will replace each (occurrence) of a free name $x$ in $P$ by
$\quotep{\procn{x}|\procn{x}}$.

Also, we will avail ourselves of the notation $x^{L}$ and $x^{R}$ to
denote injections of a name into disjoint copies of the name
space. There are numerous ways to accomplish this. One example can be
found in \cite{MeredithR05}. This notation overloads to vectors of
names: $\vec{x}^{\pi} := (x_{i}^{\pi} \; : \; 0 \leq i < |\vec{x}| )$ where $\pi \in \{L,R\}$.

We also use $P^{\Box} := P|\Box$.

In \cite{MeredithR05} an interpretation of the new operator is
given. It turns out that there are several possible interpretations
all enjoying the requisite algebraic properties of the operator (see
\cite{milner91polyadicpi}). We will therefore make liberal use of
$(\nu\; \vec{x})P$.

% subsection the_syntax_and_semantics_of_the_notation_system (end)   

\section{Interpretation of QM}
\subsection{Supporting definitions}
\subsubsection{Multiplication}
\begin{mathpar}
  \quotep{Q} \cdot \quotep{R} := \quotep{Q|R}
  \and \\
  \quotep{Q} \cdot P := P\{ \quotep{Q|R} / \quotep{R} : \quotep{R} \in \freenames{P} \}
\end{mathpar}

\paragraph{Discussion}
The first line needs little explanation. The second line says that
each free name of the process is replaced with the multiplication of
that name by the scalar. Multiplication of a scalar (name) by a state
(process) results in a process all the names of which have been `moved
over' by parallel composition with the process the scalar
quotes. There is a subtlety that the bound names have to be
manipulated so that multiplied names aren't accidentally
captured. There are many ways to achieve this.

\begin{remark}\label{rem:multiplication_identities}
  The reader is invited to verify that for all $x,y,z \in \QProc$ and $P \in \Proc$
  \begin{mathpar}
    x \cdot \quotep{0} \equiv x 
    \and
    x \cdot y \equiv y \cdot x
    \and
    x \cdot (y \cdot z) \equiv (x \cdot y) \cdot z
    \and \\
    \quotep{0} \cdot P \equiv P
    \and \\
    x \cdot (y \cdot P) \equiv (x \cdot y) \cdot P
    \and \\
    x \cdot (P|Q) \equiv (x \cdot P) | (x \cdot Q)
    \and \\    
  \end{mathpar}
\end{remark}

\subsubsection{Tensor product}

We define a tensor product on processes by structural induction.

\paragraph{Tensor of sums} First note that all summations, including
$\pzero$ and sequence, can be written $\Sigma_{i} x_{i}.A_{i} +
\Sigma_{j} x_{j}.C_{j}$, where we have grouped input-guarded processes
together and output-guarded processes together.

Thus, we can define the tensor product of two summations, $N_{1}\otimes N_{2}$, where

\begin{mathpar}
  N_{1} := \Sigma_{i} x_{i}.A_{i} + \Sigma_{j} x_{j}.C_{j}
  \and
  N_{2} := \Sigma_{i'} y_{i'}.B_{i'} + \Sigma_{j'} y_{j'}.D_{j'} 
\end{mathpar}

as follows.

\begin{mathpar}
  \Sigma_{i} x_{i}.A_{i} + \Sigma_{j} x_{j}.C_{j} \otimes \Sigma_{i'}
  y_{i'}.B_{i'} + \Sigma_{j'} y_{j'}.D_{j'} 
  \and \\
  := \; \Sigma_{i} \Sigma_{i'} \quotep{\stackrel{\vee}{x_{i}}| \stackrel{\vee}{y_{i'}}}.(A_{i}\otimes B_{i'}) \; | \; \Sigma_{i'} \Sigma_{i} \quotep{\stackrel{\vee}{y_{i'}}|\stackrel{\vee}{x_{i}}}.(B_{i'}\otimes A_{i})
  \and
  \;\; | \;\; \Sigma_{j} \Sigma_{j'} \quotep{\stackrel{\vee}{x_{j}}|\stackrel{\vee}{y_{j'}}}.(A_{j}\otimes B_{j'}) \; | \; \Sigma_{j'} \Sigma_{j} \quotep{\stackrel{\vee}{y_{j'}}|\stackrel{\vee}{x_{j}}}.(B_{j'}\otimes A_{j})
\end{mathpar}

\begin{remark}
  Do we need to $x^{L}$ and $y^{R}$ for this construction as well?
\end{remark}

\paragraph{Tensor of parallel compositions} Next, we distribute tensor
over par.

\begin{mathpar}
  P_{1}|P_{2} \otimes Q_{1}|Q_{2} := (P_{1} \otimes Q_{1}) | (P_{1}
  \otimes Q_{2}) | (P_{2} \otimes Q_{1}) | (P_{2} \otimes Q_{2})
\end{mathpar}

\paragraph{Tensor with dropped names} We treat tensor of a
process with a dropped name as parallel composition.

\begin{mathpar}
  P \otimes \dropn{x} := P | \dropn{x}
\end{mathpar}

\paragraph{Tensor of agents}

Finally, we need to define tensor on agents. Note that the definition
of tensor on normal products only tensors inputs with inputs and
outputs with outputs. Thus, we only have to define the operation on
``homogeneous'' pairings.

\begin{mathpar}
  (\vec{x})P \otimes (\vec{y})Q
  \and \\
  := (x_{0}^{L}|y_{0}^{R},\ldots,x_{0}^{L}|y_{n}^{R},\ldots,x_{m}^{L}|y_{0}^{R},\ldots,x_{m}^{L}|y_{n}^R)(P\{ \vec{x}^{L}/\vec{x}\} \otimes Q \{ \vec{y}^{R}/\vec{y}\})
  \and \\
  \clift{\vec{P}} \otimes \clift{\vec{Q}}
  \and \\
  := \clift{P_{0}\otimes Q_{0},\ldots,P_{0}\otimes Q_{n},\ldots,P_{m}\otimes Q_{0},\ldots,P_{m}\otimes Q_{n}}
\end{mathpar}

\begin{remark}
  Observe that arities of tensored abstractions matches arities of
  tensored concretions if the original arities matched. Note also that
  the length of the arities corresponds to the increase in dimension
  we see in ordinary vector space tensor product.
\end{remark}

\begin{remark}
  Operationally, this definition distributes the tensor down to
  components ``linked'' by summation. Tensor over summation is
  intriguing in that it mixes names. Moreover, as a consequence of the
  way it mixes names we have the identities for all $x \in \QProc$ and
  $P,Q \in \Proc$

  \begin{mathpar}
    (x \cdot P) \otimes Q \equiv x \cdot (P \otimes Q) \equiv P \otimes (x \cdot Q)
    \and
    P \otimes \pzero \equiv P
  \end{mathpar}

  that the reader is invited to verify.
\end{remark}

\subsubsection{Annihilation}
\begin{mathpar}
  P^{\perp} := \{ Q | \forall R. P|Q \red^{*} R \Rightarrow R \red^{*} \pzero \}
  \and \\
  P^{\underline{\perp}} := \Sigma_{Q \in P^{\perp}} \quotep{Q}?(y).(\dropn{y}|Q) | \Sigma_{Q \in P^{\perp}} \quotep{Q}\clift{\Box}
\end{mathpar}

\paragraph{Discussion} The reader will note that $P^{\perp}$ is a
\emph{set} of processes, while $P^{\underline{\perp}}$ is a
\emph{context}. We call the set $P^{\perp}$ the \emph{annihilators} of
$P$. The parallel composition of a process in the annihilators of $P$
with $P$ will result in a process, the state space of which has all
paths eventually leading to $\pzero$. Execution may endure loops; but
under reasonable conditions of fairness (naturally guaranteed under
most notions of bisimulation) such a composite process cannot get
stuck in such a loop and will, eventually pop out and terminate.

The context $P^{\underline{\perp}}$ is ready and willing to ``take the
$P$ out of'' the process to which it is applied. It will effectively
transmit the code of the process to which it is applied to one of the
annihilators and run the process against it.

\subsubsection{Evaluation}
We fix $M$ a domain of fully abstract interpretation with an equality
coincident with bisimulation. We take $\meaningof{\cdot} : \Proc \to
M$ to be the map interpreting processes and $\nmeaningof{\cdot} : \M
\to Proc$ to be the map running the other way. Then we define

\begin{mathpar}
  \int P := \nmeaningof{\meaningof{P}}
\end{mathpar}

\paragraph{Discussion}
There are many fully abstract interpretations of Milner's
$\pi$-calculus. Any of them can be used as a basis for interpreting
the reflective calculus here. Equipped with such a domain it is
largely a matter of grinding through to check that the Yoneda
construction for the normalization-by-evaluation program can be
extended to this setting.

\begin{remark}
  The reader is invited to verify that $\int (P^{\underline{\perp}}[P]) = 0$.
\end{remark}

\subsection{Quantum mechanics}

Table \ref{tbl:core_qm_op_defns} gives the core operational definitions

\begin{table}[htp]\label{tbl:core_qm_op_defns}
  \center{
    \fbox{
      \begin{tabular}{c|c}
        quantum mechanics & process calculus \\
        \hline
        scalar & $x := \quotep{P}$ \\
        state vector & $\state{P} := P$ \\
        dual & $\state{P}^{*} := \event{P^{\underline{\perp}}} := \quotep{P^{\underline{\perp}}}[-]$ \\
        matrix & $ \Sigma_{\alpha} \state{P_{\alpha}}x_{\alpha}\event{Q_{\alpha}}$ \\
        vector addition & $\state{P} + \state{Q} := \state{P | Q}$ \\
        tensor product & $\state{P} \otimes \state{Q} := \state{P \otimes Q}$ \\
        inner product & $\innerprod{P}{Q} := \quotep{\int P^{\underline{\perp}}[Q]}$ \\
      \end{tabular}
    }
  }
  \caption{QM - operational definitions}
\end{table}

where

\begin{mathpar}
  \prmatrix{P}{Q} := \fprmatrix{P}{\quotep{\pzero}}{Q}
  \and
  \fprmatrix{P}{x}{Q} := (\state{P},x,\event{Q})
  \and
  (\fprmatrix{P}{x}{Q})(\state{R}) := x \cdot \innerprod{Q}{R} \cdot \state{P}
  \and
  (\fprmatrix{P}{x}{Q})(\event{R}) := x \cdot \innerprod{R}{P} \cdot \event{Q}
\end{mathpar}

\paragraph{Discussion}
As promised: vectors (aka states) are represented as processes; duals
as contextual duals; inner product definition should be compared with
standard inner product definition for ....

\begin{remark}
  Assuming $\int (P^{\underline{\perp}}[P]) = 0$, the reader is
  invited to verify that $(\fprmatrix{P}{x}{P})(\state{P}) = x \cdot \state{P}$.
\end{remark}

\begin{remark}
  The reader is invited to verify that $\innerprod{P}{Q}$ could
  equally well have been written $\quotep{\int \stackrel{\vee}{x}}$
  where $x = \event{P^{\underline{\perp}}}(Q)$.

  One of the motivations for this remark is that there is another way
  to factor these operations. We could package up evaluation in the dual:

  \begin{mathpar}
    \state{P}^{*} := \event{\int P^{\underline{\perp}}} := \quotep{\int P^{\underline{\perp}}}[-]
  \end{mathpar}

  and then have inner product defined by
  
  \begin{mathpar}
    \innerprod{P}{Q} := \event{P}(Q)
  \end{mathpar}

  Hopefully, experience with the calculations will provide guidance on
  the best factoring.
\end{remark}

\begin{remark}
  Assuming $\int (P^{\underline{\perp}}[P]) = 0$, the reader is
  invited to verify that $\forall P,Q. (\prmatrix{0}{Q})(\state{0}) =
  \state{0}$ and dually $(\prmatrix{P}{0})(\event{0}) = \event{0}$.
\end{remark}

\begin{remark}
  i'm a little worried that i don't (yet) have proper support for
  complex conjugacy. But, the observation above may give us a
  clue. According to Abramsky, it must be the case that the scalars
  are iso to the homset of the identity for the tensor -- which the
  observation above characterizes. 

  For now, we will simply bookmark the notion with $\overline{x}$.
\end{remark}

\subsubsection{Adjointness}

We need to give a definition of $(\cdot)^{\dagger}$ for matrices. The
obvious candidate definition is
\begin{mathpar}
(\Sigma_{\alpha}\fprmatrix{P_{\alpha}}{x_{\alpha}}{Q_{\alpha}})^{\dagger}
= \Sigma_{\alpha}\fprmatrix{(Q_{\alpha}^{\underline{\perp}})^{*}}{\overline{x}_{\alpha}}{P_{\alpha}^{\underline{\perp}}} 
\end{mathpar}

But, $(Q_{\alpha}^{\underline{\perp}})^{*}$ requires a name along
which to communicate the process to achieve the context application.

\subsubsection{Basis for a basis}
If processes label states and ``addition'' of states (a.k.a. vector
addition) is interpreted as parallel composition, what corresponds to
notions of linear independence and basis? Here, we recall that Yoshida
has developed a set of \emph{combinators} for an asynchronous verison
of Milner's $\pi$-calculus. These are a finite set of processes such
any process can be expressed as parallel composition of these
combinators together with liberal uses of the new operator and
replication. We can simply give a translation of these into the
present calculus and have reasonable expectation that the property
carries over. That is, that the resultant set allows to express all
processes via parallel composition. Note, however, that there is no
new operator or replication in this calculus. As a result, we expect
that the corresponding set is actually infinite. That is, we expect
that the space is actually infinite dimensional.

\begin{remark}
  The attentive reader may be a bit concerned. Certainly, the
  collection $S$, $K$ and $I$ is a finite set of
  combinators. Shouldn't we expect to see a finite set of combinators
  for an effectively equivalent system? i am very sympathetic to this
  critique and feel it warrants full attention. On the other hand, i
  also have in mind the following analogy. The natural numbers, as a
  monoid under addition, has exactly $1$ generator, while the natural
  numbers, as a monoid under multiplication, has countably many
  generators (the primes). We observe that the application of the
  lambda calculus is much less resource sensitive than the parallel
  composition of the $\pi$-calculus. Could it be the case that we have
  an analogy of the form
  
  \begin{mathpar}
    m + n : MN :: m*n : M|N
  \end{mathpar}

  giving a similar blow up in the set of ``primes''?  This is such a
  wonderful thought that, even if it's not true, i think it's worth
  writing down.
\end{remark}
 

\documentclass[12pt]{llncs}
%\documentclass{jktr}

\usepackage[pdftex]{hyperref}                   
\usepackage {listings}
\usepackage {mathpartir}
\usepackage{bcprules}
%\usepackage{listings}
                       
\usepackage{graphicx} 
%\usepackage[margins=2.5cm,nohead,nofoot]{geometry}
%\usepackage{geometry}
\usepackage{amsfonts}
\usepackage{amstext}
\usepackage{latexsym}
\usepackage{amssymb}
\usepackage{color}


%\include{myPreamble}
\include{qm2pi.local} 

%\ifpdf
%\usepackage[pdftex]{graphicx}
%\else
%\usepackage{graphicx}
%\fi

 % \ifpdf
%  \usepackage{pdfsync}
%  \if


%\title{Brief Article}
%\author{David F. Snyder}
%\author{L.G. Meredith}

%\address{Dept. of Math., Texas State University--San Marcos, San Marcos, TX 78666}
       
\pagestyle{empty}


\begin{document}

\lstset{language=[Objective]Caml,frame=shadowbox}

\input{qm2pi.front}

% section front matter (end)

\input{qm2pi.intro} 
 
% section introduction (end)

% \input{qm2pi.knotations} 

% section notation (end)

\input{qm2pi.process.calculi} 

% section concurrent_process_calculi_and_spatial_logics_ (end)
    
%\input{qm2pi.knots2pi} 

%\input{qm2pi.trefoil} 

%\input{qm2pi.mainthm} 

% subsection basic_interpretation (end)

%\input{qm2pi.rho.presentation} 
\subsection{The syntax and semantics of the notation system}\label{sub:the_syntax_and_semantics_of_the_notation_system} % (fold)

We now summarize a technical presentation of the calculus that
embodies our theory of dynamics. The typical presentation of such a
calculus follows the style of giving generators and relations on
them. The grammar, below, describing term constructors, freely
generates the set of processes, $\Proc$. This set is then quotiented
by a relation known as structural congruence and it is over this set
that the notion of dynamics is expressed. This presentation is
essentially that of \cite{MeredithR05} with the addition of
polyadicity and summation. For readability we have relegated some of
the technical subtleties to an appendix.

\subsubsection{Process grammar}\label{subsub:process_grammar}

\begin{mathpar}
  \inferrule* [lab=synchronization] {} {{M} \bc \pzero \;|\; x?F \;|\; x!C }
  \and
  \inferrule* [lab=abstraction] {} {{F} \bc (x)P}
  \and
  \inferrule* [lab=concretion] {} {{C} \bc \langle Q \rangle}
  \and
  \inferrule* [lab=process] {} {{P,Q} \bc M \;| \;P|Q \;|\; @{x}}
  \and
  \inferrule* [lab=name] {} {{x} \bc \quotep{P}}
\end{mathpar} 

Note that $\vec{x}$ (resp. $\vec{P}$) denotes a vector of names
(resp. processes) of length $|\vec{x}|$ (resp. $|\vec{P}|$). We adopt
the following useful abbreviations.

\begin{mathpar}
   x?(\vec{y}).P := x.(\vec{y})P \and  x\clift{\vec{P}} := x.\clift{\vec{P}}
   \and x!(y) := \lift{x}{\dropn{y}}
   \and \Pi_{i=0}^{n-1}P_i := P_0 | \ldots | P_{n-1}
\end{mathpar}

\subsubsection{Structural congruence}

\paragraph{Free and bound names and alpha-equivalence.} At the
core of structural equivalence is alpha-equivalence which identifies
process that are the same up to a change of variable. Formally, we
recognize the distinction between free and bound names. The free names
of a process, $\freenames{P}$, may be calculated recursively as
follows:

\begin{mathpar}
\freenames{\pzero} := \emptyset
  \and \\
  \freenames{x?(y).P} := \{ x \} \cup (\freenames{P} \setminus \{ y \})
  \and 
  \freenames{x!\langle P \rangle} := \{ x \} \cup \{ P \} 
  \and \\
  \freenames{P|Q} := \freenames{P} \cup \freenames{Q}
  \and \\
  \freenames{@{x}} := \{ x \}
\end{mathpar}

$\pi$
$\quotep{\pi}$

$\freenames{-} : \pi \to \mathcal{P}(\quotep{\pi})$

\begin{eqnarray*}
  \freenames{\pzero} & := & \emptyset \\
  \freenames{x?(y).P} & := & \{ x \} \cup (\freenames{P} \setminus \{ y \}) \\
  \freenames{x!\langle P \rangle} & := & \{ x \} \cup \{ P \} \\
  \freenames{P|Q} & := & \freenames{P} \cup \freenames{Q} \\
  \freenames{\dropn{x}} & := & \{ x \}
\end{eqnarray*}

The bound names of a process, $\boundnames{P}$, are those names occurring in $P$
that are not free. For example, in $x?(y).0$, the name $x$ is free, while $y$ is bound.

\begin{mathpar}
  \inferrule* [lab=monoidal-laws] {} { P|Q \equiv Q|P \and P|0 \equiv P \and P|(Q|R) \equiv (P|Q)|R }
\end{mathpar}

\begin{mathpar}
  \inferrule* [lab=alpha-equivalence] {} { (x)P \equiv (y)P\{y/x\} \and y \not\in \freenames{P} }
\end{mathpar}

\begin{definition}
Then two processes, $P,Q$, are alpha-equivalent if $P = Q\{\vec{y}/\vec{x}\}$ for
some $\vec{x} \in \boundnames{Q},\vec{y} \in \boundnames{P}$, where $Q\{\vec{y}/\vec{x}\}$
denotes the capture-avoiding substitution of $\vec{y}$ for $\vec{x}$ in $Q$.
\end{definition}

\begin{definition}
  The {\em structural congruence} \cite{SangiorgiWalker} , $\equiv$,
  between processes is the least congruence containing
  alpha-equivalence, satisfying the abelian monoid laws
  (associativity, commutativity and $\pzero$ as identity) for parallel
  composition $|$ and for summation $+$.
\end{definition}

\subsection{Name equivalence}

We take name equivalence, written $\nameeq$, to be the smallest
equivalence relation generated by the following rules.

\begin{mathpar}
\inferrule*[lab=Quote-drop]
{ }
{ \quotep{@{x}} \nameeq x }

\inferrule*[lab=Struct-equiv]
{ P \scong Q }
{ \quotep{P} \nameeq \quotep{Q} }
\end{mathpar}

The astute reader will have noticed that the mutual recursion of names
and processes imposes a mutual recursion on alpha-equivalence and
structural equivalence via name-equivalence. Fortunately, all of this
works out pleasantly and we may calculate in the natural way, free of
concern. The reader interested in the details is referred to the
appendix \ref{appendix:rho_details}.

\subsection{Substitution}

We use $\Proc$ for the set of processes, $\QProc$ for the set of
names, and $\id{\{}\vec{y} / \vec{x} \id{\}}$ to denote partial maps,
$s : \QProc \rightarrow \QProc$. A map, $s$ lifts, uniquely, to a map
on process terms, $\widehat{s} : \Proc \rightarrow \Proc$ by the
following equations.

\begin{mathpar}
  (0) \psubstp{Q}{P} := 0 \\
  (R \juxtap S) \psubstp{Q}{P}
  :=    
  (R)\psubstp{Q}{P} \juxtap (S) \psubstp{Q}{P} \\
  (x?(y).R) \psubstp{Q}{P}    
  :=    
  (x)\substp{Q}{P} (z)\concat( (R \psubstn{z}{y}) \psubstp{Q}{P} ) \\
  (\lift{x}{R}) \psubstp{Q}{P}  
  :=
  \lift{(x)\substp{Q}{P}}{ R \psubstp{Q}{P} } \\
%   (\dropn{x})  \psubstp{Q}{P}       
%   := 
%   \left\{ 
%     \begin{array}{ccc} 
%       \dropn{\quotep{Q}} & & x \nameeq \quotep{P} \\
%       \dropn{x} & & otherwise \\
%     \end{array}
%   \right. 
  (\dropn{x})  \psubstp{Q}{P}       
  := 
  \left\{ 
    \begin{array}{ccc} 
      Q & & x \nameeq \quotep{P} \\
      \dropn{x} & & otherwise \\
    \end{array}
  \right.
\end{mathpar}
 

where

\begin{eqnarray}
  (x)\id{\{} \lpquote Q \rpquote / \lpquote P \rpquote \id{\}}            = 
  \left\{ 
    \begin{array}{ccc}
      \lpquote Q \rpquote & & x \nameeq \lpquote P \rpquote \\
      x & & otherwise \\
    \end{array}
  \right. \nonumber
\end{eqnarray}

and $z$ is chosen distinct from $\quotep{P}$, $\quotep{Q}$, the free
names in $Q$, and all the names in $R$. Our $\alpha$-equivalence will
be built in the standard way from this substitution.

\begin{remark}\label{rem:no_self_referential_names}
  One consequence of these definitions is that $\forall P. \quotep{P}
  \not\in \freenames{P}$.
\end{remark}

\subsection{ Dynamic quote: an example }

Anticipating something of what's to come, consider applying the
substitution, $\widehat{\id{\{}u / z \id{\}}}$, to the following pair
of processes, $\lift{w}{y!(z)}$ and $w[ \lpquote y!(z) \rpquote ]$.

\begin{eqnarray}
	\lift{w}{y!(z)}\widehat{\id{\{}u / z \id{\}}}
		& = &
		\lift{w}{y!(u)} \nonumber\\
	w[ \lpquote y!(z) \rpquote ] \widehat{ \id{\{}u / z \id{\}} }
		& = &
		w[ \lpquote y!(z) \rpquote ] \nonumber
\end{eqnarray}

Because the body of the process between quotes is impervious to
substitution, we get radically different answers. In fact, by
examining the first process in an input context,
e.g. $x?(z).\lift{w}{y!(z)}$, we see that the process under the lift
operator may be shaped by prefixed inputs binding a name inside it. In
this sense, the lift operator will be seen as a way to dynamically
construct processes before reifying them as names.

Finally equipped with these standard features we can present the
dynamics of the calculus.

\subsubsection{Operational semantics} 

Finally, we introduce the computational dynamics. What marks these
algebras as distinct from other more traditionally studied algebraic
structures, e.g. vector spaces or polynomial rings, is the manner in
which dynamics is captured. In traditional structures, dynamics is typically
expressed through morphisms between such structures, as in linear maps
between vector spaces or morphisms between rings. In algebras
associated with the semantics of computation, the dynamics is
expressed as part of the algebraic structure itself, through a
reduction reduction relation typically denoted by $\red$. Below, we
give a recursive presentation of this relation for the calculus used
in the encoding.

$\red \subseteq \pi \times \pi$
$\red : \pi \to \mathcal{P}(\pi)$

\begin{mathpar}
  \inferrule* [lab=Comm] { \textsf{match}( x_{src}, x_{trgt} ) } { x_{trgt}?(y)P \; | \; x_{src}!\langle {Q} \rangle \red P\{\quotep{Q}/y}\} }
  \and \\
  \inferrule* [lab=Par] {{P} \red {P}'} {{{P} | {Q}} \red {{P}' | {Q}}}
  \and
  \inferrule* [lab=Equiv]{{{P} \scong {P}'} \andalso {{P}' \red {Q}'} \andalso {{Q}' \scong {Q}}}{{P} \red {Q}}
\end{mathpar}

\begin{eqnarray*}
  match_{\equiv} (\quotep{P},\quotep{Q}) & := & P \equiv Q \\
  match_{\dagger}(\quotep{P},\quotep{Q}) & := & \forall R. P|Q \red^{*} R => R \red^{*} 0 \\
  match_{K}(\quotep{P},\quotep{Q}) & := & K \mbox{ for some context } K
\end{eqnarray*}

$u?(x)P | u!\langle Q \rangle \red P\{\quotep{Q}/x\}$

%We write $\wred$ for $\red^*$, and $P\red$ if $\exists Q $ such that $ P \red Q$.
We write $P\red$ if $\exists Q $ such that $ P \red Q$ and $P\not\red$, otherwise.

\section{Replication}

As mentioned before, it is known that replication (and hence
recursion) can be implemented in a higher-order process algebra
\cite{SangiorgiWalker}. As our first example of calculation with the
machinery thus far presented we give the construction explicitly in
the {\rhoc}.

\begin{eqnarray}
	D_{x} & := & \prefix{x}{y}{(\binpar{\outputp{x}{y}}{@{y}})} \nonumber\\
	\bangp_{x}{P} & := & \binpar{{x}!\langle{\binpar{D_{x}}{P}}\rangle}{D_{x}} \nonumber
\end{eqnarray}

\begin{eqnarray}
	\bangp_{x}{P} & & \nonumber\\
	=
	& {x}!\langle{(\prefix{x}{y}{(\outputp{x}{y} | @{y})) | P}}\rangle 
	      | \prefix{x}{y}{(\outputp{x}{y} | @{y})} & \nonumber\\
	\red
	& (\outputp{x}{y} | @{y})\substn{\quotep{(\prefix{x}{y}{(@{y} | \outputp{x}{y})) | P}}}{y} & \nonumber\\
	=
	& \outputp{x}{\quotep{(\prefix{x}{y}{(\outputp{x}{y} | @{y})) | P}}}
	  | {(\prefix{x}{y}{(\outputp{x}{y} | @{y})) | P}} & \nonumber\\
	\red
	& \ldots & \nonumber\\
	\red^*
	& P | P | \ldots & \nonumber
\end{eqnarray}

Of course, this encoding, as an implementation, runs away, unfolding
$\bangp{P}$ eagerly. A lazier and more implementable replication
operator, restricted to input-guarded processes, may be obtained as follows.

\begin{eqnarray}
\bangp{\prefix{u}{v}{P}} 
	:= 
	\binpar{\lift{x}{\prefix{u}{v}{(\binpar{D(x)}{P})}}}{D(x)} \nonumber
\end{eqnarray}

\begin{remark}
  Note that the lazier definition still does not deal with summation
  or mixed summation (i.e. sums over input and output). The reader is
  invited to construct definitions of replication that deal with these
  features. 

  Further, the definitions are parameterized in a name, $x$. Can you,
  gentle reader, make a definition that eliminates this parameter and
  guarantees no accidental interaction between the replication
  machinery and the process being replicated -- i.e. no accidental
  sharing of names used by the process to get its work done and the
  name(s) used by the replication to effect copying. This latter
  revision of the definition of replication is crucial to obtaining
  the expected identity $!!P \sim !P$.
\end{remark}

\begin{remark}\label{rem:paradoxical_combinator}
  The reader familiar with the lambda calculus will have noticed the
  similarity between $D$ and the paradoxical combinator.

  [Ed. note: the existence of this seems to suggest we have to be more
  restrictive on the set of processes and names we admit if we are to
  support no-cloning.]
\end{remark}

\subsubsection{Bisimulation}

The computational dynamics gives rise to another kind of equivalence,
the equivalence of computational behavior. As previously mentioned
this is typically captured \emph{via} some form of bisimulation.

% The notion we use in this paper is weak barbed bisimulation
% \cite{milner91polyadicpi}.

The notion we use in this paper is derived from weak barbed
bisimulation \cite{milner91polyadicpi}. 

\begin{definition}
An \emph{observation relation}, $\downarrow_{\mathcal N}$, over a set
of names, $\mathcal N$, is the smallest relation satisfying the rules
below.

\infrule[Out-barb]{y \in {\mathcal N}, \; x \nameeq y}
		  {\outputp{x}{v} \downarrow_{\mathcal N} x}
\infrule[Par-barb]{\mbox{$P\downarrow_{\mathcal N} x$ or $Q\downarrow_{\mathcal N} x$}}
		  {\binpar{P}{Q} \downarrow_{\mathcal N} x}

We write $P \Downarrow_{\mathcal N} x$ if there is $Q$ such that 
$P \wred Q$ and $Q \downarrow_{\mathcal N} x$.
\end{definition}

\begin{definition}
%\label{def.bbisim}
An  ${\mathcal N}$-\emph{barbed bisimulation} over a set of names, ${\mathcal N}$, is a symmetric binary relation 
${\mathcal S}_{\mathcal N}$ between agents such that $P\rel{S}_{\mathcal N}Q$ implies:
\begin{enumerate}
\item If $P \red P'$ then $Q \wred Q'$ and $P'\rel{S}_{\mathcal N} Q'$.
\item If $P\downarrow_{\mathcal N} x$, then $Q\Downarrow_{\mathcal N} x$.
\end{enumerate}
$P$ is ${\mathcal N}$-barbed bisimilar to $Q$, written
$P \wbbisim_{\mathcal N} Q$, if $P \rel{S}_{\mathcal N} Q$ for some ${\mathcal N}$-barbed bisimulation ${\mathcal S}_{\mathcal N}$.
\end{definition}

$\mathcal{R} \subseteq \pi \times \pi$

$P \mathcal{R} Q => \forall P'. P \red P' \Rightarrow \exists Q'. Q \red Q', P' \mathcal{R} Q'$

$P \vdash x \Rightarrow Q \vdash x$

\begin{mathpar}
  \inferrule*[lab=Out-barb]{x \nameeq y}{{y}!\langle{Q}\rangle \vdash x}
  \and
  \inferrule*[lab=Par-barb]{\mbox{$P\vdash x$ or $Q\vdash x$}}{\binpar{P}{Q} \vdash x}
\end{mathpar}

\subsubsection{Contexts}

One of the principle advantages of computational calculi like the
$\pi$-calculus is a well-defined notion of context,
contextual-equivalence and a correlation between
contextual-equivalence and notions of bisimulation. The notion of
context allows the decomposition of a process into (sub-)process and
its syntactic environment, its context. Thus, a context may be
thought of as a process with a ``hole'' (written $\Box$) in it. The
application of a context $M$ to a process $P$, written $M[P]$, is
tantamount to filling the hole in $M$ with $P$. In this paper we do
not need the full weight of this theory, but do make use of the notion
of context in the proof the main theorem. 

\begin{mathpar}
  \inferrule* [lab=summation] {} {{M_{M},M_{N}} \bc \Box \;|\; x.M_{A} \;|\; M_{M}+M_{N}}
  \and
  \inferrule* [lab=agent] {} {{M_{A}} \bc (\vec{x})M_{P} \;| \; \clift{P_0,\ldots,M_{P},\ldots,P_N}}
  \and \\
  \inferrule* [lab=process] {} {{M_{P}} \bc M_{N} \;| \;P|M_{P} }
\end{mathpar} 

\begin{mathpar}
  \inferrule* [lab=sychronization] {} {M_{N} \bc \Box \;|\; x?M_{F} \;|\; x!M_{C}}
  \and
  \inferrule* [lab=abstraction] {} {{M_{F}} \bc (x)M_{P} }
  \and
  \inferrule* [lab=concretion] {} {{M_{C}} \bc \langle M_{P} \rangle }
  \and \\
  \inferrule* [lab=process] {} {{M_{P}} \bc M_{N} \;| \;P|M_{P} }
\end{mathpar}

\begin{definition}[contextual application] Given a context $M$, and
  process $P$, we define the \emph{contextual application}, $M[P] :=
  M\{P/\Box\}$. That is, the contextual application of M to P is the
  substitution of $P$ for $\Box$ in $M$.
\end{definition}

$\meaningof{-} : L \to \mathcal{P}(\pi)$

\begin{mathpar}
  \inferrule* [lab=collection] {} {\meaningof{true} = \pi, \and \meaningof{~E} = \pi \setminus \meaningof{E}, \and \meaningof{E_{1} \& E_{2}} = \meaningof{E_{1}} \cap \meaningof{E_{2}}}
\end{mathpar}

\begin{mathpar}
  \inferrule* [lab=structure] {} {\meaningof{0} = \{ P \in \pi | P \equiv 0 \}, \and \\ \meaningof{E_1 | E_2} = \{ P \in \pi | P \equiv P_{1} | P_{2}, P_{1} \in \meaningof{E_{1}}, P_{2} \in \meaningof{E_2}\} }
\end{mathpar}

\begin{mathpar}
 \inferrule* [lab=behavior] {} {\meaningof{\langle a?b \rangle E} = \{ P \in \pi | P \equiv Q | u?(y)P', \\ \and \\\\ \and \\ \;\;\; u \in \meaningof{a}, \forall z.P'\{z/y\} \in \meaningof{E\{z/b\}}\}, \and \\ \meaningof{a!E} = \{ P \in \pi | P \equiv Q | x!\langle P' \rangle, x \in \meaningof{a} P' \in \meaningof{E}\} }
\end{mathpar}

\begin{mathpar}
 \inferrule* [lab=nominal] {} {\meaningof{\quotep{E}} = \{ \quotep{P} \in \quotep{\pi} | P \in \meaningof{E} \}, \and \meaningof{\quotep{P}} = \{ \quotep{Q} \in \quotep{\pi} | P \equiv Q \} \and \\ \meaningof{@\quotep{E}} = \{ P \in \pi | P \equiv @x, x \in \meaningof{E} \}}
\end{mathpar}

\begin{eqnarray*}
  \\
  \meaningof{-} : TS \to ST
\end{eqnarray*}

\begin{eqnarray*}
  \\
  L : TS \to ST
\end{eqnarray*}

\begin{eqnarray*}
  \\
  P \models E \iff P \in \meaningof{E}
\end{eqnarray*}

\begin{eqnarray*}
  P \approx_{L} Q \iff \forall E \in L. P \models E \iff Q \models E
\end{eqnarray*}

\begin{eqnarray*}
  P \approx_{K} Q
\end{eqnarray*}

\begin{eqnarray*}
  P \approx Q
\end{eqnarray*}

$\approx_{K} = \approx = \approx_{L}$

\subsubsection{Contextual duality}

Note that contexts extend the quotation operation to a family of
operations from processes to names. Given a context, $M$, we can
define a \emph{nominal context}, $\quotep{M}$ by $\quotep{M}[P] :=
\quotep{M[P]}$. To foreshadow what is to come we observe that these
operations enjoy a duality with processes very much like the duality
between vectors and maps from vectors to scalars.

Further, because the calculus is essentially higher-order, we have a
correspondence between contexts and processes. More specifically,
given a name $x$ and a context $M$ we can construct $M^{*}_{x}$ such
that 

\begin{mathpar}
  M^{*}_{x} | \lift{x}{P} \red M[P]
\end{mathpar}

namely,

\begin{mathpar}
  M^{*}_{x} := x?(u).M[\dropn{u}]
\end{mathpar}

The dependence of $M^{*}_{x}$ on a name makes it an abstraction, 

\begin{mathpar}
  M^{*} := (x)x?(u).M[\dropn{u}]
\end{mathpar}

\subsection{Additional notation}

It will sometimes be convenient to denote the process a name
quotes. We already have the notation $x = \quotep{P}$, but it will be
convenient to introduce an alternate notation, $\procn{x}$, when we
want to emphasize the connection to the use of the name. Note that, by
virtue of name equivalence, $\quotep{\procn{x}} \nameeq x$; so, the
notation is consistent with previous definitions.

Further, because names have structure it is possible to effect
substitutions on the basis of that structure. This means we need to
upgrade our notation for substitutions, which we accomplish by
adapting comprehension notation. Thus,

\begin{mathpar}
  P\{ y / x : x \in S \}
\end{mathpar}

is interpreted to mean the process derived from P by replacing (in a
capture-avoiding manner) each occurrence of $x$ in $S$ by $y$. For example,

\begin{mathpar}
  P\{ \quotep{\procn{x}|\procn{x}} / x : x \in \freenames{P} \}
\end{mathpar}

will replace each (occurrence) of a free name $x$ in $P$ by
$\quotep{\procn{x}|\procn{x}}$.

Also, we will avail ourselves of the notation $x^{L}$ and $x^{R}$ to
denote injections of a name into disjoint copies of the name
space. There are numerous ways to accomplish this. One example can be
found in \cite{MeredithR05}. This notation overloads to vectors of
names: $\vec{x}^{\pi} := (x_{i}^{\pi} \; : \; 0 \leq i < |\vec{x}| )$ where $\pi \in \{L,R\}$.

We also use $P^{\Box} := P|\Box$.

In \cite{MeredithR05} an interpretation of the new operator is
given. It turns out that there are several possible interpretations
all enjoying the requisite algebraic properties of the operator (see
\cite{milner91polyadicpi}). We will therefore make liberal use of
$(\nu\; \vec{x})P$.

% subsection the_syntax_and_semantics_of_the_notation_system (end)   

\input{qm2pi.qmops} 

\input{qm2pi.sterngerlach} 

\input{qm2pi.metric} 

% section concurrent_process_calculi (end)

%\input{qm2pi.proofsketch}

% section proof sketch (end)

%\input{qm2pi.slviaknots} 

% section spatial logic via knots (end)

\input{qm2pi.conclusion}

% section conclusion (end)

%\input{qm2pi.dtcodes} 

% section wiring algorithm (end)

\input{qm2pi.ack} 

% section acknowledgments (end)

\newpage


\bibliographystyle{plain}   
\bibliography{../../biblios/main.bib}

\input{qm2pi.rhodetails}

\end{document}

 

\documentclass[12pt]{llncs}
%\documentclass{jktr}

\usepackage[pdftex]{hyperref}                   
\usepackage {listings}
\usepackage {mathpartir}
\usepackage{bcprules}
%\usepackage{listings}
                       
\usepackage{graphicx} 
%\usepackage[margins=2.5cm,nohead,nofoot]{geometry}
%\usepackage{geometry}
\usepackage{amsfonts}
\usepackage{amstext}
\usepackage{latexsym}
\usepackage{amssymb}
\usepackage{color}


%\include{myPreamble}
\include{qm2pi.local} 

%\ifpdf
%\usepackage[pdftex]{graphicx}
%\else
%\usepackage{graphicx}
%\fi

 % \ifpdf
%  \usepackage{pdfsync}
%  \if


%\title{Brief Article}
%\author{David F. Snyder}
%\author{L.G. Meredith}

%\address{Dept. of Math., Texas State University--San Marcos, San Marcos, TX 78666}
       
\pagestyle{empty}


\begin{document}

\lstset{language=[Objective]Caml,frame=shadowbox}

\input{qm2pi.front}

% section front matter (end)

\input{qm2pi.intro} 
 
% section introduction (end)

% \input{qm2pi.knotations} 

% section notation (end)

\input{qm2pi.process.calculi} 

% section concurrent_process_calculi_and_spatial_logics_ (end)
    
%\input{qm2pi.knots2pi} 

%\input{qm2pi.trefoil} 

%\input{qm2pi.mainthm} 

% subsection basic_interpretation (end)

%\input{qm2pi.rho.presentation} 
\subsection{The syntax and semantics of the notation system}\label{sub:the_syntax_and_semantics_of_the_notation_system} % (fold)

We now summarize a technical presentation of the calculus that
embodies our theory of dynamics. The typical presentation of such a
calculus follows the style of giving generators and relations on
them. The grammar, below, describing term constructors, freely
generates the set of processes, $\Proc$. This set is then quotiented
by a relation known as structural congruence and it is over this set
that the notion of dynamics is expressed. This presentation is
essentially that of \cite{MeredithR05} with the addition of
polyadicity and summation. For readability we have relegated some of
the technical subtleties to an appendix.

\subsubsection{Process grammar}\label{subsub:process_grammar}

\begin{mathpar}
  \inferrule* [lab=synchronization] {} {{M} \bc \pzero \;|\; x?F \;|\; x!C }
  \and
  \inferrule* [lab=abstraction] {} {{F} \bc (x)P}
  \and
  \inferrule* [lab=concretion] {} {{C} \bc \langle Q \rangle}
  \and
  \inferrule* [lab=process] {} {{P,Q} \bc M \;| \;P|Q \;|\; @{x}}
  \and
  \inferrule* [lab=name] {} {{x} \bc \quotep{P}}
\end{mathpar} 

Note that $\vec{x}$ (resp. $\vec{P}$) denotes a vector of names
(resp. processes) of length $|\vec{x}|$ (resp. $|\vec{P}|$). We adopt
the following useful abbreviations.

\begin{mathpar}
   x?(\vec{y}).P := x.(\vec{y})P \and  x\clift{\vec{P}} := x.\clift{\vec{P}}
   \and x!(y) := \lift{x}{\dropn{y}}
   \and \Pi_{i=0}^{n-1}P_i := P_0 | \ldots | P_{n-1}
\end{mathpar}

\subsubsection{Structural congruence}

\paragraph{Free and bound names and alpha-equivalence.} At the
core of structural equivalence is alpha-equivalence which identifies
process that are the same up to a change of variable. Formally, we
recognize the distinction between free and bound names. The free names
of a process, $\freenames{P}$, may be calculated recursively as
follows:

\begin{mathpar}
\freenames{\pzero} := \emptyset
  \and \\
  \freenames{x?(y).P} := \{ x \} \cup (\freenames{P} \setminus \{ y \})
  \and 
  \freenames{x!\langle P \rangle} := \{ x \} \cup \{ P \} 
  \and \\
  \freenames{P|Q} := \freenames{P} \cup \freenames{Q}
  \and \\
  \freenames{@{x}} := \{ x \}
\end{mathpar}

$\pi$
$\quotep{\pi}$

$\freenames{-} : \pi \to \mathcal{P}(\quotep{\pi})$

\begin{eqnarray*}
  \freenames{\pzero} & := & \emptyset \\
  \freenames{x?(y).P} & := & \{ x \} \cup (\freenames{P} \setminus \{ y \}) \\
  \freenames{x!\langle P \rangle} & := & \{ x \} \cup \{ P \} \\
  \freenames{P|Q} & := & \freenames{P} \cup \freenames{Q} \\
  \freenames{\dropn{x}} & := & \{ x \}
\end{eqnarray*}

The bound names of a process, $\boundnames{P}$, are those names occurring in $P$
that are not free. For example, in $x?(y).0$, the name $x$ is free, while $y$ is bound.

\begin{mathpar}
  \inferrule* [lab=monoidal-laws] {} { P|Q \equiv Q|P \and P|0 \equiv P \and P|(Q|R) \equiv (P|Q)|R }
\end{mathpar}

\begin{mathpar}
  \inferrule* [lab=alpha-equivalence] {} { (x)P \equiv (y)P\{y/x\} \and y \not\in \freenames{P} }
\end{mathpar}

\begin{definition}
Then two processes, $P,Q$, are alpha-equivalent if $P = Q\{\vec{y}/\vec{x}\}$ for
some $\vec{x} \in \boundnames{Q},\vec{y} \in \boundnames{P}$, where $Q\{\vec{y}/\vec{x}\}$
denotes the capture-avoiding substitution of $\vec{y}$ for $\vec{x}$ in $Q$.
\end{definition}

\begin{definition}
  The {\em structural congruence} \cite{SangiorgiWalker} , $\equiv$,
  between processes is the least congruence containing
  alpha-equivalence, satisfying the abelian monoid laws
  (associativity, commutativity and $\pzero$ as identity) for parallel
  composition $|$ and for summation $+$.
\end{definition}

\subsection{Name equivalence}

We take name equivalence, written $\nameeq$, to be the smallest
equivalence relation generated by the following rules.

\begin{mathpar}
\inferrule*[lab=Quote-drop]
{ }
{ \quotep{@{x}} \nameeq x }

\inferrule*[lab=Struct-equiv]
{ P \scong Q }
{ \quotep{P} \nameeq \quotep{Q} }
\end{mathpar}

The astute reader will have noticed that the mutual recursion of names
and processes imposes a mutual recursion on alpha-equivalence and
structural equivalence via name-equivalence. Fortunately, all of this
works out pleasantly and we may calculate in the natural way, free of
concern. The reader interested in the details is referred to the
appendix \ref{appendix:rho_details}.

\subsection{Substitution}

We use $\Proc$ for the set of processes, $\QProc$ for the set of
names, and $\id{\{}\vec{y} / \vec{x} \id{\}}$ to denote partial maps,
$s : \QProc \rightarrow \QProc$. A map, $s$ lifts, uniquely, to a map
on process terms, $\widehat{s} : \Proc \rightarrow \Proc$ by the
following equations.

\begin{mathpar}
  (0) \psubstp{Q}{P} := 0 \\
  (R \juxtap S) \psubstp{Q}{P}
  :=    
  (R)\psubstp{Q}{P} \juxtap (S) \psubstp{Q}{P} \\
  (x?(y).R) \psubstp{Q}{P}    
  :=    
  (x)\substp{Q}{P} (z)\concat( (R \psubstn{z}{y}) \psubstp{Q}{P} ) \\
  (\lift{x}{R}) \psubstp{Q}{P}  
  :=
  \lift{(x)\substp{Q}{P}}{ R \psubstp{Q}{P} } \\
%   (\dropn{x})  \psubstp{Q}{P}       
%   := 
%   \left\{ 
%     \begin{array}{ccc} 
%       \dropn{\quotep{Q}} & & x \nameeq \quotep{P} \\
%       \dropn{x} & & otherwise \\
%     \end{array}
%   \right. 
  (\dropn{x})  \psubstp{Q}{P}       
  := 
  \left\{ 
    \begin{array}{ccc} 
      Q & & x \nameeq \quotep{P} \\
      \dropn{x} & & otherwise \\
    \end{array}
  \right.
\end{mathpar}
 

where

\begin{eqnarray}
  (x)\id{\{} \lpquote Q \rpquote / \lpquote P \rpquote \id{\}}            = 
  \left\{ 
    \begin{array}{ccc}
      \lpquote Q \rpquote & & x \nameeq \lpquote P \rpquote \\
      x & & otherwise \\
    \end{array}
  \right. \nonumber
\end{eqnarray}

and $z$ is chosen distinct from $\quotep{P}$, $\quotep{Q}$, the free
names in $Q$, and all the names in $R$. Our $\alpha$-equivalence will
be built in the standard way from this substitution.

\begin{remark}\label{rem:no_self_referential_names}
  One consequence of these definitions is that $\forall P. \quotep{P}
  \not\in \freenames{P}$.
\end{remark}

\subsection{ Dynamic quote: an example }

Anticipating something of what's to come, consider applying the
substitution, $\widehat{\id{\{}u / z \id{\}}}$, to the following pair
of processes, $\lift{w}{y!(z)}$ and $w[ \lpquote y!(z) \rpquote ]$.

\begin{eqnarray}
	\lift{w}{y!(z)}\widehat{\id{\{}u / z \id{\}}}
		& = &
		\lift{w}{y!(u)} \nonumber\\
	w[ \lpquote y!(z) \rpquote ] \widehat{ \id{\{}u / z \id{\}} }
		& = &
		w[ \lpquote y!(z) \rpquote ] \nonumber
\end{eqnarray}

Because the body of the process between quotes is impervious to
substitution, we get radically different answers. In fact, by
examining the first process in an input context,
e.g. $x?(z).\lift{w}{y!(z)}$, we see that the process under the lift
operator may be shaped by prefixed inputs binding a name inside it. In
this sense, the lift operator will be seen as a way to dynamically
construct processes before reifying them as names.

Finally equipped with these standard features we can present the
dynamics of the calculus.

\subsubsection{Operational semantics} 

Finally, we introduce the computational dynamics. What marks these
algebras as distinct from other more traditionally studied algebraic
structures, e.g. vector spaces or polynomial rings, is the manner in
which dynamics is captured. In traditional structures, dynamics is typically
expressed through morphisms between such structures, as in linear maps
between vector spaces or morphisms between rings. In algebras
associated with the semantics of computation, the dynamics is
expressed as part of the algebraic structure itself, through a
reduction reduction relation typically denoted by $\red$. Below, we
give a recursive presentation of this relation for the calculus used
in the encoding.

$\red \subseteq \pi \times \pi$
$\red : \pi \to \mathcal{P}(\pi)$

\begin{mathpar}
  \inferrule* [lab=Comm] { \textsf{match}( x_{src}, x_{trgt} ) } { x_{trgt}?(y)P \; | \; x_{src}!\langle {Q} \rangle \red P\{\quotep{Q}/y}\} }
  \and \\
  \inferrule* [lab=Par] {{P} \red {P}'} {{{P} | {Q}} \red {{P}' | {Q}}}
  \and
  \inferrule* [lab=Equiv]{{{P} \scong {P}'} \andalso {{P}' \red {Q}'} \andalso {{Q}' \scong {Q}}}{{P} \red {Q}}
\end{mathpar}

\begin{eqnarray*}
  match_{\equiv} (\quotep{P},\quotep{Q}) & := & P \equiv Q \\
  match_{\dagger}(\quotep{P},\quotep{Q}) & := & \forall R. P|Q \red^{*} R => R \red^{*} 0 \\
  match_{K}(\quotep{P},\quotep{Q}) & := & K \mbox{ for some context } K
\end{eqnarray*}

$u?(x)P | u!\langle Q \rangle \red P\{\quotep{Q}/x\}$

%We write $\wred$ for $\red^*$, and $P\red$ if $\exists Q $ such that $ P \red Q$.
We write $P\red$ if $\exists Q $ such that $ P \red Q$ and $P\not\red$, otherwise.

\section{Replication}

As mentioned before, it is known that replication (and hence
recursion) can be implemented in a higher-order process algebra
\cite{SangiorgiWalker}. As our first example of calculation with the
machinery thus far presented we give the construction explicitly in
the {\rhoc}.

\begin{eqnarray}
	D_{x} & := & \prefix{x}{y}{(\binpar{\outputp{x}{y}}{@{y}})} \nonumber\\
	\bangp_{x}{P} & := & \binpar{{x}!\langle{\binpar{D_{x}}{P}}\rangle}{D_{x}} \nonumber
\end{eqnarray}

\begin{eqnarray}
	\bangp_{x}{P} & & \nonumber\\
	=
	& {x}!\langle{(\prefix{x}{y}{(\outputp{x}{y} | @{y})) | P}}\rangle 
	      | \prefix{x}{y}{(\outputp{x}{y} | @{y})} & \nonumber\\
	\red
	& (\outputp{x}{y} | @{y})\substn{\quotep{(\prefix{x}{y}{(@{y} | \outputp{x}{y})) | P}}}{y} & \nonumber\\
	=
	& \outputp{x}{\quotep{(\prefix{x}{y}{(\outputp{x}{y} | @{y})) | P}}}
	  | {(\prefix{x}{y}{(\outputp{x}{y} | @{y})) | P}} & \nonumber\\
	\red
	& \ldots & \nonumber\\
	\red^*
	& P | P | \ldots & \nonumber
\end{eqnarray}

Of course, this encoding, as an implementation, runs away, unfolding
$\bangp{P}$ eagerly. A lazier and more implementable replication
operator, restricted to input-guarded processes, may be obtained as follows.

\begin{eqnarray}
\bangp{\prefix{u}{v}{P}} 
	:= 
	\binpar{\lift{x}{\prefix{u}{v}{(\binpar{D(x)}{P})}}}{D(x)} \nonumber
\end{eqnarray}

\begin{remark}
  Note that the lazier definition still does not deal with summation
  or mixed summation (i.e. sums over input and output). The reader is
  invited to construct definitions of replication that deal with these
  features. 

  Further, the definitions are parameterized in a name, $x$. Can you,
  gentle reader, make a definition that eliminates this parameter and
  guarantees no accidental interaction between the replication
  machinery and the process being replicated -- i.e. no accidental
  sharing of names used by the process to get its work done and the
  name(s) used by the replication to effect copying. This latter
  revision of the definition of replication is crucial to obtaining
  the expected identity $!!P \sim !P$.
\end{remark}

\begin{remark}\label{rem:paradoxical_combinator}
  The reader familiar with the lambda calculus will have noticed the
  similarity between $D$ and the paradoxical combinator.

  [Ed. note: the existence of this seems to suggest we have to be more
  restrictive on the set of processes and names we admit if we are to
  support no-cloning.]
\end{remark}

\subsubsection{Bisimulation}

The computational dynamics gives rise to another kind of equivalence,
the equivalence of computational behavior. As previously mentioned
this is typically captured \emph{via} some form of bisimulation.

% The notion we use in this paper is weak barbed bisimulation
% \cite{milner91polyadicpi}.

The notion we use in this paper is derived from weak barbed
bisimulation \cite{milner91polyadicpi}. 

\begin{definition}
An \emph{observation relation}, $\downarrow_{\mathcal N}$, over a set
of names, $\mathcal N$, is the smallest relation satisfying the rules
below.

\infrule[Out-barb]{y \in {\mathcal N}, \; x \nameeq y}
		  {\outputp{x}{v} \downarrow_{\mathcal N} x}
\infrule[Par-barb]{\mbox{$P\downarrow_{\mathcal N} x$ or $Q\downarrow_{\mathcal N} x$}}
		  {\binpar{P}{Q} \downarrow_{\mathcal N} x}

We write $P \Downarrow_{\mathcal N} x$ if there is $Q$ such that 
$P \wred Q$ and $Q \downarrow_{\mathcal N} x$.
\end{definition}

\begin{definition}
%\label{def.bbisim}
An  ${\mathcal N}$-\emph{barbed bisimulation} over a set of names, ${\mathcal N}$, is a symmetric binary relation 
${\mathcal S}_{\mathcal N}$ between agents such that $P\rel{S}_{\mathcal N}Q$ implies:
\begin{enumerate}
\item If $P \red P'$ then $Q \wred Q'$ and $P'\rel{S}_{\mathcal N} Q'$.
\item If $P\downarrow_{\mathcal N} x$, then $Q\Downarrow_{\mathcal N} x$.
\end{enumerate}
$P$ is ${\mathcal N}$-barbed bisimilar to $Q$, written
$P \wbbisim_{\mathcal N} Q$, if $P \rel{S}_{\mathcal N} Q$ for some ${\mathcal N}$-barbed bisimulation ${\mathcal S}_{\mathcal N}$.
\end{definition}

$\mathcal{R} \subseteq \pi \times \pi$

$P \mathcal{R} Q => \forall P'. P \red P' \Rightarrow \exists Q'. Q \red Q', P' \mathcal{R} Q'$

$P \vdash x \Rightarrow Q \vdash x$

\begin{mathpar}
  \inferrule*[lab=Out-barb]{x \nameeq y}{{y}!\langle{Q}\rangle \vdash x}
  \and
  \inferrule*[lab=Par-barb]{\mbox{$P\vdash x$ or $Q\vdash x$}}{\binpar{P}{Q} \vdash x}
\end{mathpar}

\subsubsection{Contexts}

One of the principle advantages of computational calculi like the
$\pi$-calculus is a well-defined notion of context,
contextual-equivalence and a correlation between
contextual-equivalence and notions of bisimulation. The notion of
context allows the decomposition of a process into (sub-)process and
its syntactic environment, its context. Thus, a context may be
thought of as a process with a ``hole'' (written $\Box$) in it. The
application of a context $M$ to a process $P$, written $M[P]$, is
tantamount to filling the hole in $M$ with $P$. In this paper we do
not need the full weight of this theory, but do make use of the notion
of context in the proof the main theorem. 

\begin{mathpar}
  \inferrule* [lab=summation] {} {{M_{M},M_{N}} \bc \Box \;|\; x.M_{A} \;|\; M_{M}+M_{N}}
  \and
  \inferrule* [lab=agent] {} {{M_{A}} \bc (\vec{x})M_{P} \;| \; \clift{P_0,\ldots,M_{P},\ldots,P_N}}
  \and \\
  \inferrule* [lab=process] {} {{M_{P}} \bc M_{N} \;| \;P|M_{P} }
\end{mathpar} 

\begin{mathpar}
  \inferrule* [lab=sychronization] {} {M_{N} \bc \Box \;|\; x?M_{F} \;|\; x!M_{C}}
  \and
  \inferrule* [lab=abstraction] {} {{M_{F}} \bc (x)M_{P} }
  \and
  \inferrule* [lab=concretion] {} {{M_{C}} \bc \langle M_{P} \rangle }
  \and \\
  \inferrule* [lab=process] {} {{M_{P}} \bc M_{N} \;| \;P|M_{P} }
\end{mathpar}

\begin{definition}[contextual application] Given a context $M$, and
  process $P$, we define the \emph{contextual application}, $M[P] :=
  M\{P/\Box\}$. That is, the contextual application of M to P is the
  substitution of $P$ for $\Box$ in $M$.
\end{definition}

$\meaningof{-} : L \to \mathcal{P}(\pi)$

\begin{mathpar}
  \inferrule* [lab=collection] {} {\meaningof{true} = \pi, \and \meaningof{~E} = \pi \setminus \meaningof{E}, \and \meaningof{E_{1} \& E_{2}} = \meaningof{E_{1}} \cap \meaningof{E_{2}}}
\end{mathpar}

\begin{mathpar}
  \inferrule* [lab=structure] {} {\meaningof{0} = \{ P \in \pi | P \equiv 0 \}, \and \\ \meaningof{E_1 | E_2} = \{ P \in \pi | P \equiv P_{1} | P_{2}, P_{1} \in \meaningof{E_{1}}, P_{2} \in \meaningof{E_2}\} }
\end{mathpar}

\begin{mathpar}
 \inferrule* [lab=behavior] {} {\meaningof{\langle a?b \rangle E} = \{ P \in \pi | P \equiv Q | u?(y)P', \\ \and \\\\ \and \\ \;\;\; u \in \meaningof{a}, \forall z.P'\{z/y\} \in \meaningof{E\{z/b\}}\}, \and \\ \meaningof{a!E} = \{ P \in \pi | P \equiv Q | x!\langle P' \rangle, x \in \meaningof{a} P' \in \meaningof{E}\} }
\end{mathpar}

\begin{mathpar}
 \inferrule* [lab=nominal] {} {\meaningof{\quotep{E}} = \{ \quotep{P} \in \quotep{\pi} | P \in \meaningof{E} \}, \and \meaningof{\quotep{P}} = \{ \quotep{Q} \in \quotep{\pi} | P \equiv Q \} \and \\ \meaningof{@\quotep{E}} = \{ P \in \pi | P \equiv @x, x \in \meaningof{E} \}}
\end{mathpar}

\begin{eqnarray*}
  \\
  \meaningof{-} : TS \to ST
\end{eqnarray*}

\begin{eqnarray*}
  \\
  L : TS \to ST
\end{eqnarray*}

\begin{eqnarray*}
  \\
  P \models E \iff P \in \meaningof{E}
\end{eqnarray*}

\begin{eqnarray*}
  P \approx_{L} Q \iff \forall E \in L. P \models E \iff Q \models E
\end{eqnarray*}

\begin{eqnarray*}
  P \approx_{K} Q
\end{eqnarray*}

\begin{eqnarray*}
  P \approx Q
\end{eqnarray*}

$\approx_{K} = \approx = \approx_{L}$

\subsubsection{Contextual duality}

Note that contexts extend the quotation operation to a family of
operations from processes to names. Given a context, $M$, we can
define a \emph{nominal context}, $\quotep{M}$ by $\quotep{M}[P] :=
\quotep{M[P]}$. To foreshadow what is to come we observe that these
operations enjoy a duality with processes very much like the duality
between vectors and maps from vectors to scalars.

Further, because the calculus is essentially higher-order, we have a
correspondence between contexts and processes. More specifically,
given a name $x$ and a context $M$ we can construct $M^{*}_{x}$ such
that 

\begin{mathpar}
  M^{*}_{x} | \lift{x}{P} \red M[P]
\end{mathpar}

namely,

\begin{mathpar}
  M^{*}_{x} := x?(u).M[\dropn{u}]
\end{mathpar}

The dependence of $M^{*}_{x}$ on a name makes it an abstraction, 

\begin{mathpar}
  M^{*} := (x)x?(u).M[\dropn{u}]
\end{mathpar}

\subsection{Additional notation}

It will sometimes be convenient to denote the process a name
quotes. We already have the notation $x = \quotep{P}$, but it will be
convenient to introduce an alternate notation, $\procn{x}$, when we
want to emphasize the connection to the use of the name. Note that, by
virtue of name equivalence, $\quotep{\procn{x}} \nameeq x$; so, the
notation is consistent with previous definitions.

Further, because names have structure it is possible to effect
substitutions on the basis of that structure. This means we need to
upgrade our notation for substitutions, which we accomplish by
adapting comprehension notation. Thus,

\begin{mathpar}
  P\{ y / x : x \in S \}
\end{mathpar}

is interpreted to mean the process derived from P by replacing (in a
capture-avoiding manner) each occurrence of $x$ in $S$ by $y$. For example,

\begin{mathpar}
  P\{ \quotep{\procn{x}|\procn{x}} / x : x \in \freenames{P} \}
\end{mathpar}

will replace each (occurrence) of a free name $x$ in $P$ by
$\quotep{\procn{x}|\procn{x}}$.

Also, we will avail ourselves of the notation $x^{L}$ and $x^{R}$ to
denote injections of a name into disjoint copies of the name
space. There are numerous ways to accomplish this. One example can be
found in \cite{MeredithR05}. This notation overloads to vectors of
names: $\vec{x}^{\pi} := (x_{i}^{\pi} \; : \; 0 \leq i < |\vec{x}| )$ where $\pi \in \{L,R\}$.

We also use $P^{\Box} := P|\Box$.

In \cite{MeredithR05} an interpretation of the new operator is
given. It turns out that there are several possible interpretations
all enjoying the requisite algebraic properties of the operator (see
\cite{milner91polyadicpi}). We will therefore make liberal use of
$(\nu\; \vec{x})P$.

% subsection the_syntax_and_semantics_of_the_notation_system (end)   

\input{qm2pi.qmops} 

\input{qm2pi.sterngerlach} 

\input{qm2pi.metric} 

% section concurrent_process_calculi (end)

%\input{qm2pi.proofsketch}

% section proof sketch (end)

%\input{qm2pi.slviaknots} 

% section spatial logic via knots (end)

\input{qm2pi.conclusion}

% section conclusion (end)

%\input{qm2pi.dtcodes} 

% section wiring algorithm (end)

\input{qm2pi.ack} 

% section acknowledgments (end)

\newpage


\bibliographystyle{plain}   
\bibliography{../../biblios/main.bib}

\input{qm2pi.rhodetails}

\end{document}

 

% section concurrent_process_calculi (end)

%\documentclass[12pt]{llncs}
%\documentclass{jktr}

\usepackage[pdftex]{hyperref}                   
\usepackage {listings}
\usepackage {mathpartir}
\usepackage{bcprules}
%\usepackage{listings}
                       
\usepackage{graphicx} 
%\usepackage[margins=2.5cm,nohead,nofoot]{geometry}
%\usepackage{geometry}
\usepackage{amsfonts}
\usepackage{amstext}
\usepackage{latexsym}
\usepackage{amssymb}
\usepackage{color}


%\include{myPreamble}
\include{qm2pi.local} 

%\ifpdf
%\usepackage[pdftex]{graphicx}
%\else
%\usepackage{graphicx}
%\fi

 % \ifpdf
%  \usepackage{pdfsync}
%  \if


%\title{Brief Article}
%\author{David F. Snyder}
%\author{L.G. Meredith}

%\address{Dept. of Math., Texas State University--San Marcos, San Marcos, TX 78666}
       
\pagestyle{empty}


\begin{document}

\lstset{language=[Objective]Caml,frame=shadowbox}

\input{qm2pi.front}

% section front matter (end)

\input{qm2pi.intro} 
 
% section introduction (end)

% \input{qm2pi.knotations} 

% section notation (end)

\input{qm2pi.process.calculi} 

% section concurrent_process_calculi_and_spatial_logics_ (end)
    
%\input{qm2pi.knots2pi} 

%\input{qm2pi.trefoil} 

%\input{qm2pi.mainthm} 

% subsection basic_interpretation (end)

%\input{qm2pi.rho.presentation} 
\subsection{The syntax and semantics of the notation system}\label{sub:the_syntax_and_semantics_of_the_notation_system} % (fold)

We now summarize a technical presentation of the calculus that
embodies our theory of dynamics. The typical presentation of such a
calculus follows the style of giving generators and relations on
them. The grammar, below, describing term constructors, freely
generates the set of processes, $\Proc$. This set is then quotiented
by a relation known as structural congruence and it is over this set
that the notion of dynamics is expressed. This presentation is
essentially that of \cite{MeredithR05} with the addition of
polyadicity and summation. For readability we have relegated some of
the technical subtleties to an appendix.

\subsubsection{Process grammar}\label{subsub:process_grammar}

\begin{mathpar}
  \inferrule* [lab=synchronization] {} {{M} \bc \pzero \;|\; x?F \;|\; x!C }
  \and
  \inferrule* [lab=abstraction] {} {{F} \bc (x)P}
  \and
  \inferrule* [lab=concretion] {} {{C} \bc \langle Q \rangle}
  \and
  \inferrule* [lab=process] {} {{P,Q} \bc M \;| \;P|Q \;|\; @{x}}
  \and
  \inferrule* [lab=name] {} {{x} \bc \quotep{P}}
\end{mathpar} 

Note that $\vec{x}$ (resp. $\vec{P}$) denotes a vector of names
(resp. processes) of length $|\vec{x}|$ (resp. $|\vec{P}|$). We adopt
the following useful abbreviations.

\begin{mathpar}
   x?(\vec{y}).P := x.(\vec{y})P \and  x\clift{\vec{P}} := x.\clift{\vec{P}}
   \and x!(y) := \lift{x}{\dropn{y}}
   \and \Pi_{i=0}^{n-1}P_i := P_0 | \ldots | P_{n-1}
\end{mathpar}

\subsubsection{Structural congruence}

\paragraph{Free and bound names and alpha-equivalence.} At the
core of structural equivalence is alpha-equivalence which identifies
process that are the same up to a change of variable. Formally, we
recognize the distinction between free and bound names. The free names
of a process, $\freenames{P}$, may be calculated recursively as
follows:

\begin{mathpar}
\freenames{\pzero} := \emptyset
  \and \\
  \freenames{x?(y).P} := \{ x \} \cup (\freenames{P} \setminus \{ y \})
  \and 
  \freenames{x!\langle P \rangle} := \{ x \} \cup \{ P \} 
  \and \\
  \freenames{P|Q} := \freenames{P} \cup \freenames{Q}
  \and \\
  \freenames{@{x}} := \{ x \}
\end{mathpar}

$\pi$
$\quotep{\pi}$

$\freenames{-} : \pi \to \mathcal{P}(\quotep{\pi})$

\begin{eqnarray*}
  \freenames{\pzero} & := & \emptyset \\
  \freenames{x?(y).P} & := & \{ x \} \cup (\freenames{P} \setminus \{ y \}) \\
  \freenames{x!\langle P \rangle} & := & \{ x \} \cup \{ P \} \\
  \freenames{P|Q} & := & \freenames{P} \cup \freenames{Q} \\
  \freenames{\dropn{x}} & := & \{ x \}
\end{eqnarray*}

The bound names of a process, $\boundnames{P}$, are those names occurring in $P$
that are not free. For example, in $x?(y).0$, the name $x$ is free, while $y$ is bound.

\begin{mathpar}
  \inferrule* [lab=monoidal-laws] {} { P|Q \equiv Q|P \and P|0 \equiv P \and P|(Q|R) \equiv (P|Q)|R }
\end{mathpar}

\begin{mathpar}
  \inferrule* [lab=alpha-equivalence] {} { (x)P \equiv (y)P\{y/x\} \and y \not\in \freenames{P} }
\end{mathpar}

\begin{definition}
Then two processes, $P,Q$, are alpha-equivalent if $P = Q\{\vec{y}/\vec{x}\}$ for
some $\vec{x} \in \boundnames{Q},\vec{y} \in \boundnames{P}$, where $Q\{\vec{y}/\vec{x}\}$
denotes the capture-avoiding substitution of $\vec{y}$ for $\vec{x}$ in $Q$.
\end{definition}

\begin{definition}
  The {\em structural congruence} \cite{SangiorgiWalker} , $\equiv$,
  between processes is the least congruence containing
  alpha-equivalence, satisfying the abelian monoid laws
  (associativity, commutativity and $\pzero$ as identity) for parallel
  composition $|$ and for summation $+$.
\end{definition}

\subsection{Name equivalence}

We take name equivalence, written $\nameeq$, to be the smallest
equivalence relation generated by the following rules.

\begin{mathpar}
\inferrule*[lab=Quote-drop]
{ }
{ \quotep{@{x}} \nameeq x }

\inferrule*[lab=Struct-equiv]
{ P \scong Q }
{ \quotep{P} \nameeq \quotep{Q} }
\end{mathpar}

The astute reader will have noticed that the mutual recursion of names
and processes imposes a mutual recursion on alpha-equivalence and
structural equivalence via name-equivalence. Fortunately, all of this
works out pleasantly and we may calculate in the natural way, free of
concern. The reader interested in the details is referred to the
appendix \ref{appendix:rho_details}.

\subsection{Substitution}

We use $\Proc$ for the set of processes, $\QProc$ for the set of
names, and $\id{\{}\vec{y} / \vec{x} \id{\}}$ to denote partial maps,
$s : \QProc \rightarrow \QProc$. A map, $s$ lifts, uniquely, to a map
on process terms, $\widehat{s} : \Proc \rightarrow \Proc$ by the
following equations.

\begin{mathpar}
  (0) \psubstp{Q}{P} := 0 \\
  (R \juxtap S) \psubstp{Q}{P}
  :=    
  (R)\psubstp{Q}{P} \juxtap (S) \psubstp{Q}{P} \\
  (x?(y).R) \psubstp{Q}{P}    
  :=    
  (x)\substp{Q}{P} (z)\concat( (R \psubstn{z}{y}) \psubstp{Q}{P} ) \\
  (\lift{x}{R}) \psubstp{Q}{P}  
  :=
  \lift{(x)\substp{Q}{P}}{ R \psubstp{Q}{P} } \\
%   (\dropn{x})  \psubstp{Q}{P}       
%   := 
%   \left\{ 
%     \begin{array}{ccc} 
%       \dropn{\quotep{Q}} & & x \nameeq \quotep{P} \\
%       \dropn{x} & & otherwise \\
%     \end{array}
%   \right. 
  (\dropn{x})  \psubstp{Q}{P}       
  := 
  \left\{ 
    \begin{array}{ccc} 
      Q & & x \nameeq \quotep{P} \\
      \dropn{x} & & otherwise \\
    \end{array}
  \right.
\end{mathpar}
 

where

\begin{eqnarray}
  (x)\id{\{} \lpquote Q \rpquote / \lpquote P \rpquote \id{\}}            = 
  \left\{ 
    \begin{array}{ccc}
      \lpquote Q \rpquote & & x \nameeq \lpquote P \rpquote \\
      x & & otherwise \\
    \end{array}
  \right. \nonumber
\end{eqnarray}

and $z$ is chosen distinct from $\quotep{P}$, $\quotep{Q}$, the free
names in $Q$, and all the names in $R$. Our $\alpha$-equivalence will
be built in the standard way from this substitution.

\begin{remark}\label{rem:no_self_referential_names}
  One consequence of these definitions is that $\forall P. \quotep{P}
  \not\in \freenames{P}$.
\end{remark}

\subsection{ Dynamic quote: an example }

Anticipating something of what's to come, consider applying the
substitution, $\widehat{\id{\{}u / z \id{\}}}$, to the following pair
of processes, $\lift{w}{y!(z)}$ and $w[ \lpquote y!(z) \rpquote ]$.

\begin{eqnarray}
	\lift{w}{y!(z)}\widehat{\id{\{}u / z \id{\}}}
		& = &
		\lift{w}{y!(u)} \nonumber\\
	w[ \lpquote y!(z) \rpquote ] \widehat{ \id{\{}u / z \id{\}} }
		& = &
		w[ \lpquote y!(z) \rpquote ] \nonumber
\end{eqnarray}

Because the body of the process between quotes is impervious to
substitution, we get radically different answers. In fact, by
examining the first process in an input context,
e.g. $x?(z).\lift{w}{y!(z)}$, we see that the process under the lift
operator may be shaped by prefixed inputs binding a name inside it. In
this sense, the lift operator will be seen as a way to dynamically
construct processes before reifying them as names.

Finally equipped with these standard features we can present the
dynamics of the calculus.

\subsubsection{Operational semantics} 

Finally, we introduce the computational dynamics. What marks these
algebras as distinct from other more traditionally studied algebraic
structures, e.g. vector spaces or polynomial rings, is the manner in
which dynamics is captured. In traditional structures, dynamics is typically
expressed through morphisms between such structures, as in linear maps
between vector spaces or morphisms between rings. In algebras
associated with the semantics of computation, the dynamics is
expressed as part of the algebraic structure itself, through a
reduction reduction relation typically denoted by $\red$. Below, we
give a recursive presentation of this relation for the calculus used
in the encoding.

$\red \subseteq \pi \times \pi$
$\red : \pi \to \mathcal{P}(\pi)$

\begin{mathpar}
  \inferrule* [lab=Comm] { \textsf{match}( x_{src}, x_{trgt} ) } { x_{trgt}?(y)P \; | \; x_{src}!\langle {Q} \rangle \red P\{\quotep{Q}/y}\} }
  \and \\
  \inferrule* [lab=Par] {{P} \red {P}'} {{{P} | {Q}} \red {{P}' | {Q}}}
  \and
  \inferrule* [lab=Equiv]{{{P} \scong {P}'} \andalso {{P}' \red {Q}'} \andalso {{Q}' \scong {Q}}}{{P} \red {Q}}
\end{mathpar}

\begin{eqnarray*}
  match_{\equiv} (\quotep{P},\quotep{Q}) & := & P \equiv Q \\
  match_{\dagger}(\quotep{P},\quotep{Q}) & := & \forall R. P|Q \red^{*} R => R \red^{*} 0 \\
  match_{K}(\quotep{P},\quotep{Q}) & := & K \mbox{ for some context } K
\end{eqnarray*}

$u?(x)P | u!\langle Q \rangle \red P\{\quotep{Q}/x\}$

%We write $\wred$ for $\red^*$, and $P\red$ if $\exists Q $ such that $ P \red Q$.
We write $P\red$ if $\exists Q $ such that $ P \red Q$ and $P\not\red$, otherwise.

\section{Replication}

As mentioned before, it is known that replication (and hence
recursion) can be implemented in a higher-order process algebra
\cite{SangiorgiWalker}. As our first example of calculation with the
machinery thus far presented we give the construction explicitly in
the {\rhoc}.

\begin{eqnarray}
	D_{x} & := & \prefix{x}{y}{(\binpar{\outputp{x}{y}}{@{y}})} \nonumber\\
	\bangp_{x}{P} & := & \binpar{{x}!\langle{\binpar{D_{x}}{P}}\rangle}{D_{x}} \nonumber
\end{eqnarray}

\begin{eqnarray}
	\bangp_{x}{P} & & \nonumber\\
	=
	& {x}!\langle{(\prefix{x}{y}{(\outputp{x}{y} | @{y})) | P}}\rangle 
	      | \prefix{x}{y}{(\outputp{x}{y} | @{y})} & \nonumber\\
	\red
	& (\outputp{x}{y} | @{y})\substn{\quotep{(\prefix{x}{y}{(@{y} | \outputp{x}{y})) | P}}}{y} & \nonumber\\
	=
	& \outputp{x}{\quotep{(\prefix{x}{y}{(\outputp{x}{y} | @{y})) | P}}}
	  | {(\prefix{x}{y}{(\outputp{x}{y} | @{y})) | P}} & \nonumber\\
	\red
	& \ldots & \nonumber\\
	\red^*
	& P | P | \ldots & \nonumber
\end{eqnarray}

Of course, this encoding, as an implementation, runs away, unfolding
$\bangp{P}$ eagerly. A lazier and more implementable replication
operator, restricted to input-guarded processes, may be obtained as follows.

\begin{eqnarray}
\bangp{\prefix{u}{v}{P}} 
	:= 
	\binpar{\lift{x}{\prefix{u}{v}{(\binpar{D(x)}{P})}}}{D(x)} \nonumber
\end{eqnarray}

\begin{remark}
  Note that the lazier definition still does not deal with summation
  or mixed summation (i.e. sums over input and output). The reader is
  invited to construct definitions of replication that deal with these
  features. 

  Further, the definitions are parameterized in a name, $x$. Can you,
  gentle reader, make a definition that eliminates this parameter and
  guarantees no accidental interaction between the replication
  machinery and the process being replicated -- i.e. no accidental
  sharing of names used by the process to get its work done and the
  name(s) used by the replication to effect copying. This latter
  revision of the definition of replication is crucial to obtaining
  the expected identity $!!P \sim !P$.
\end{remark}

\begin{remark}\label{rem:paradoxical_combinator}
  The reader familiar with the lambda calculus will have noticed the
  similarity between $D$ and the paradoxical combinator.

  [Ed. note: the existence of this seems to suggest we have to be more
  restrictive on the set of processes and names we admit if we are to
  support no-cloning.]
\end{remark}

\subsubsection{Bisimulation}

The computational dynamics gives rise to another kind of equivalence,
the equivalence of computational behavior. As previously mentioned
this is typically captured \emph{via} some form of bisimulation.

% The notion we use in this paper is weak barbed bisimulation
% \cite{milner91polyadicpi}.

The notion we use in this paper is derived from weak barbed
bisimulation \cite{milner91polyadicpi}. 

\begin{definition}
An \emph{observation relation}, $\downarrow_{\mathcal N}$, over a set
of names, $\mathcal N$, is the smallest relation satisfying the rules
below.

\infrule[Out-barb]{y \in {\mathcal N}, \; x \nameeq y}
		  {\outputp{x}{v} \downarrow_{\mathcal N} x}
\infrule[Par-barb]{\mbox{$P\downarrow_{\mathcal N} x$ or $Q\downarrow_{\mathcal N} x$}}
		  {\binpar{P}{Q} \downarrow_{\mathcal N} x}

We write $P \Downarrow_{\mathcal N} x$ if there is $Q$ such that 
$P \wred Q$ and $Q \downarrow_{\mathcal N} x$.
\end{definition}

\begin{definition}
%\label{def.bbisim}
An  ${\mathcal N}$-\emph{barbed bisimulation} over a set of names, ${\mathcal N}$, is a symmetric binary relation 
${\mathcal S}_{\mathcal N}$ between agents such that $P\rel{S}_{\mathcal N}Q$ implies:
\begin{enumerate}
\item If $P \red P'$ then $Q \wred Q'$ and $P'\rel{S}_{\mathcal N} Q'$.
\item If $P\downarrow_{\mathcal N} x$, then $Q\Downarrow_{\mathcal N} x$.
\end{enumerate}
$P$ is ${\mathcal N}$-barbed bisimilar to $Q$, written
$P \wbbisim_{\mathcal N} Q$, if $P \rel{S}_{\mathcal N} Q$ for some ${\mathcal N}$-barbed bisimulation ${\mathcal S}_{\mathcal N}$.
\end{definition}

$\mathcal{R} \subseteq \pi \times \pi$

$P \mathcal{R} Q => \forall P'. P \red P' \Rightarrow \exists Q'. Q \red Q', P' \mathcal{R} Q'$

$P \vdash x \Rightarrow Q \vdash x$

\begin{mathpar}
  \inferrule*[lab=Out-barb]{x \nameeq y}{{y}!\langle{Q}\rangle \vdash x}
  \and
  \inferrule*[lab=Par-barb]{\mbox{$P\vdash x$ or $Q\vdash x$}}{\binpar{P}{Q} \vdash x}
\end{mathpar}

\subsubsection{Contexts}

One of the principle advantages of computational calculi like the
$\pi$-calculus is a well-defined notion of context,
contextual-equivalence and a correlation between
contextual-equivalence and notions of bisimulation. The notion of
context allows the decomposition of a process into (sub-)process and
its syntactic environment, its context. Thus, a context may be
thought of as a process with a ``hole'' (written $\Box$) in it. The
application of a context $M$ to a process $P$, written $M[P]$, is
tantamount to filling the hole in $M$ with $P$. In this paper we do
not need the full weight of this theory, but do make use of the notion
of context in the proof the main theorem. 

\begin{mathpar}
  \inferrule* [lab=summation] {} {{M_{M},M_{N}} \bc \Box \;|\; x.M_{A} \;|\; M_{M}+M_{N}}
  \and
  \inferrule* [lab=agent] {} {{M_{A}} \bc (\vec{x})M_{P} \;| \; \clift{P_0,\ldots,M_{P},\ldots,P_N}}
  \and \\
  \inferrule* [lab=process] {} {{M_{P}} \bc M_{N} \;| \;P|M_{P} }
\end{mathpar} 

\begin{mathpar}
  \inferrule* [lab=sychronization] {} {M_{N} \bc \Box \;|\; x?M_{F} \;|\; x!M_{C}}
  \and
  \inferrule* [lab=abstraction] {} {{M_{F}} \bc (x)M_{P} }
  \and
  \inferrule* [lab=concretion] {} {{M_{C}} \bc \langle M_{P} \rangle }
  \and \\
  \inferrule* [lab=process] {} {{M_{P}} \bc M_{N} \;| \;P|M_{P} }
\end{mathpar}

\begin{definition}[contextual application] Given a context $M$, and
  process $P$, we define the \emph{contextual application}, $M[P] :=
  M\{P/\Box\}$. That is, the contextual application of M to P is the
  substitution of $P$ for $\Box$ in $M$.
\end{definition}

$\meaningof{-} : L \to \mathcal{P}(\pi)$

\begin{mathpar}
  \inferrule* [lab=collection] {} {\meaningof{true} = \pi, \and \meaningof{~E} = \pi \setminus \meaningof{E}, \and \meaningof{E_{1} \& E_{2}} = \meaningof{E_{1}} \cap \meaningof{E_{2}}}
\end{mathpar}

\begin{mathpar}
  \inferrule* [lab=structure] {} {\meaningof{0} = \{ P \in \pi | P \equiv 0 \}, \and \\ \meaningof{E_1 | E_2} = \{ P \in \pi | P \equiv P_{1} | P_{2}, P_{1} \in \meaningof{E_{1}}, P_{2} \in \meaningof{E_2}\} }
\end{mathpar}

\begin{mathpar}
 \inferrule* [lab=behavior] {} {\meaningof{\langle a?b \rangle E} = \{ P \in \pi | P \equiv Q | u?(y)P', \\ \and \\\\ \and \\ \;\;\; u \in \meaningof{a}, \forall z.P'\{z/y\} \in \meaningof{E\{z/b\}}\}, \and \\ \meaningof{a!E} = \{ P \in \pi | P \equiv Q | x!\langle P' \rangle, x \in \meaningof{a} P' \in \meaningof{E}\} }
\end{mathpar}

\begin{mathpar}
 \inferrule* [lab=nominal] {} {\meaningof{\quotep{E}} = \{ \quotep{P} \in \quotep{\pi} | P \in \meaningof{E} \}, \and \meaningof{\quotep{P}} = \{ \quotep{Q} \in \quotep{\pi} | P \equiv Q \} \and \\ \meaningof{@\quotep{E}} = \{ P \in \pi | P \equiv @x, x \in \meaningof{E} \}}
\end{mathpar}

\begin{eqnarray*}
  \\
  \meaningof{-} : TS \to ST
\end{eqnarray*}

\begin{eqnarray*}
  \\
  L : TS \to ST
\end{eqnarray*}

\begin{eqnarray*}
  \\
  P \models E \iff P \in \meaningof{E}
\end{eqnarray*}

\begin{eqnarray*}
  P \approx_{L} Q \iff \forall E \in L. P \models E \iff Q \models E
\end{eqnarray*}

\begin{eqnarray*}
  P \approx_{K} Q
\end{eqnarray*}

\begin{eqnarray*}
  P \approx Q
\end{eqnarray*}

$\approx_{K} = \approx = \approx_{L}$

\subsubsection{Contextual duality}

Note that contexts extend the quotation operation to a family of
operations from processes to names. Given a context, $M$, we can
define a \emph{nominal context}, $\quotep{M}$ by $\quotep{M}[P] :=
\quotep{M[P]}$. To foreshadow what is to come we observe that these
operations enjoy a duality with processes very much like the duality
between vectors and maps from vectors to scalars.

Further, because the calculus is essentially higher-order, we have a
correspondence between contexts and processes. More specifically,
given a name $x$ and a context $M$ we can construct $M^{*}_{x}$ such
that 

\begin{mathpar}
  M^{*}_{x} | \lift{x}{P} \red M[P]
\end{mathpar}

namely,

\begin{mathpar}
  M^{*}_{x} := x?(u).M[\dropn{u}]
\end{mathpar}

The dependence of $M^{*}_{x}$ on a name makes it an abstraction, 

\begin{mathpar}
  M^{*} := (x)x?(u).M[\dropn{u}]
\end{mathpar}

\subsection{Additional notation}

It will sometimes be convenient to denote the process a name
quotes. We already have the notation $x = \quotep{P}$, but it will be
convenient to introduce an alternate notation, $\procn{x}$, when we
want to emphasize the connection to the use of the name. Note that, by
virtue of name equivalence, $\quotep{\procn{x}} \nameeq x$; so, the
notation is consistent with previous definitions.

Further, because names have structure it is possible to effect
substitutions on the basis of that structure. This means we need to
upgrade our notation for substitutions, which we accomplish by
adapting comprehension notation. Thus,

\begin{mathpar}
  P\{ y / x : x \in S \}
\end{mathpar}

is interpreted to mean the process derived from P by replacing (in a
capture-avoiding manner) each occurrence of $x$ in $S$ by $y$. For example,

\begin{mathpar}
  P\{ \quotep{\procn{x}|\procn{x}} / x : x \in \freenames{P} \}
\end{mathpar}

will replace each (occurrence) of a free name $x$ in $P$ by
$\quotep{\procn{x}|\procn{x}}$.

Also, we will avail ourselves of the notation $x^{L}$ and $x^{R}$ to
denote injections of a name into disjoint copies of the name
space. There are numerous ways to accomplish this. One example can be
found in \cite{MeredithR05}. This notation overloads to vectors of
names: $\vec{x}^{\pi} := (x_{i}^{\pi} \; : \; 0 \leq i < |\vec{x}| )$ where $\pi \in \{L,R\}$.

We also use $P^{\Box} := P|\Box$.

In \cite{MeredithR05} an interpretation of the new operator is
given. It turns out that there are several possible interpretations
all enjoying the requisite algebraic properties of the operator (see
\cite{milner91polyadicpi}). We will therefore make liberal use of
$(\nu\; \vec{x})P$.

% subsection the_syntax_and_semantics_of_the_notation_system (end)   

\input{qm2pi.qmops} 

\input{qm2pi.sterngerlach} 

\input{qm2pi.metric} 

% section concurrent_process_calculi (end)

%\input{qm2pi.proofsketch}

% section proof sketch (end)

%\input{qm2pi.slviaknots} 

% section spatial logic via knots (end)

\input{qm2pi.conclusion}

% section conclusion (end)

%\input{qm2pi.dtcodes} 

% section wiring algorithm (end)

\input{qm2pi.ack} 

% section acknowledgments (end)

\newpage


\bibliographystyle{plain}   
\bibliography{../../biblios/main.bib}

\input{qm2pi.rhodetails}

\end{document}



% section proof sketch (end)

%\section{Unlikely characters: spatial logic for
  knots}\label{sub:characteristic_formulae} % (fold)

Associated to the mobile process calculi are a family of logics known
as the Hennessy-Milner logics. These logics typically enjoy a
semantics interpreting formulae as sets of processes that when
factored through the encoding outlined above allows an identification
of classes of knots with logical formulae. In the context of this
encoding the sub-family known as the spatial logics \cite{CairesC03}
\cite{CairesC04} \cite{Caires04} are of particular interest providing
several important features for expressing and reasoning about
properties (i.e. classes) of knots. We hint here at how this may be done.

%\begin{description}
%\item [structural connectives] 
\subsubsection{Structural connectives} The spatial logics enjoy
structural connectives corresponding, at the logical level, to the
parallel composition ($P | Q$) and new name ($(\nu \; x)P$)
connectives for processes. As illustrated in the examples below, these
connectives are extremely expressive given the shape of our encoding.
%\item [decideable satisfaction]

\subsubsection{Decideable satisfaction}
In \cite{Caires04} the satisfaction relation is shown to be decideable
for a rich class of processes. It further turns out that the image of
the our encoding is a proper subset of that class. This result
provides the basis for an algorithm by which to search for knots
enjoying a given property.
%\item [characteristic formulae]

\subsubsection{Characteristic formulae}
In the same paper \cite{Caires04} , Caires presents a means of calculating
characteristic formulae, selecting equivalence classes of processes
up to a pre--specified depth limit on the support set of names. Composed with our
encoding, this characteristic formula can be used to select
characteristic formulae for knots.
%\end{description}

\subsubsection{Spatial logic formulae}

The grammar below (segmented for comprehension) summarizes the syntax
of spatial logic formulae. We employ illustrative examples in the
sequel to provide an intuitive understanding of their meaning
referring the reader to \cite{Caires04} for a more detailed explication
of the semantics.

\begin{mathpar}
  \inferrule* [lab=boolean] {} {{A,B} \bc T \;|\; \neg A \;|\; A \wedge B \;|\; \eta = \eta'}
  \and
  \inferrule* [lab=spatial] {} {|\; \pzero \;|\; A | B \;|\; x \text{\textregistered} A \;|\; \forall x . A \;|\;  H x . A}
  \and
  \inferrule* [lab=behavioral] {} {|\; \alpha . A}
  \and 
  \inferrule* [lab=recursion] {} {|\; X(\vec{u}) \;|\; \mu X(\vec{u}) . A}
  \and
  \inferrule* [lab=action] {} {\alpha \bc \langle x?(\vec{y}) \rangle \;|\; \langle x!(\vec{y}) \rangle \;|\; \langle \tau \rangle}
  \and 
  \inferrule* [lab=name] {} {\eta \bc x \;|\; \tau}
\end{mathpar} 

% subsection characteristic_formulae (end)   	 

\subsection{Example formulae}\label{sub:example_formulae_} % (fold)

\subsubsection{Crossing as formula.}
% 
% \begin{align*}
%   \frac{d}{dx} \sin x &= \cos x 
%   & \frac{d}{dx} e^x &= e^x \\
%   \frac{d}{dx} \cos x &= - \sin x 
%   & \frac{d}{dx} \log x &= \frac{1}{x} \\
% \end{align*} 

\begin{align*}
 \mu C(x_{0},x_{1},y_{0},y_{1},u).&(\langle x_{0}?(z) \rangle(\langle u! \rangle\langle y_{1}!z \rangle C(x_{0},x_{1},y_{0},y_{1},u)) & \\
  & \wedge \langle y_{1}?(z) \rangle (\langle u! \rangle \langle x_{0}!z \rangle C(x_{0},x_{1},y_{0},y_{1},u)) & \\
  & \wedge \langle x_{1}?(z) \rangle (\langle u? \rangle \langle y_{0}!z \rangle C(x_{0},x_{1},y_{0},y_{1},u)) & \\
  & \wedge \langle y_{0}?(z) \rangle (\langle u? \rangle \langle x_{1}!z \rangle C(x_{0},x_{1},y_{0},y_{1},u))) &
\end{align*}

The lexicographical similarity between the shape of this formulae and
the shape of definition of the process representing a crossing reveals
the intuitive meaning of this formulae. It describes the capabilities
of a process that has the right to represent a crossing. For example
it picks out processes that may perform an input on the port $x_0$ in
its initial menu of capabilities. What differentiates the formula
from the process, however, is that the crossing process is the
smallest candidate to satisfy the formula. Infinitely many other
processes -- with internal behavior hidden behind this interface, so
to speak -- also satisfy this formula. Even this simple formula,
then, can be seen to open a new view onto knots, providing a
computational interpretation of \emph{virtual} knots.

Note that this formula is derived by hand. A similar formula can be
derived by employing Caires' calculation of characteristic formula
\cite{Caires04} to the process representing a crossing. In light of
this discussion, we let
$\meaningof{C}_{\phi}(x0,x1,y0,y1,u)$ denote a formula specifying the
dynamics we wish to capture of a crossing. To guarantee we preserve
the shape of the interface and minimal semantics we demand that
$\meaningof{C}_{\phi}(x0,x1,y0,y1,u) \Rightarrow
\textbf{C}(x0,x1,y0,y1,u)$ where $\textbf{C}(x0,x1,y0,y1,u)$ denotes
the formula above.
                            
\subsubsection{Crossing number constraints.}
The moral content of the context lemma (Lemma \ref{context}) is that the notion of
``locality'' in the Reidemeister moves is effectively captured by the
parallel composition operator of the process calculus. This intuition
extends through the logic. Given a formula,
$\meaningof{C}_{\phi}(x0,x1,y0,y1,u)$, we can use the structural
connectives to specify constraints on crossing numbers, such as at
least $n$ crossings, or exactly $n$ crossings.
\begin{mathpar}
  \inferrule* [lab=at-least-n] {} { K^{\geq n}_{\phi}(\vec{xs},\vec{ys}) := \Pi_{i=0}^{n-1} Hu . \meaningof{C}_{\phi}(xs_i,ys_i,u) | T }
  \and 
  \inferrule* [lab=exactly-n] {} { K^{= n}_{\phi}(\vec{xs},\vec{ys}) := \Pi_{i=0}^{n-1} Hu . \meaningof{C}_{\phi}(xs_i,ys_i,u) | \neg (\forall x_0,y_0,x_1,y_1,u . \meaningof{C}_{\phi}(x_0,y_0,x_1,y_1,u) | T) }
\end{mathpar}

To round out this section, recall that the encoding of an $n$-crossing
knot decomposes into a parallel composition of $n$ \emph{copies} of a
crossing process together with a wiring harness. To specify different
knot classes with the same crossing number amounts to specifying
logical constraints on the wiring harness. In the interest of space,
we defer examples to a forthcoming paper. Suffice it to say that both
the conditions ``alternating knot'' and ``contains the tangle
corresponding to 5/3'' are expressible. For example, it is possible to
calculate the characteristic formula of a process corresponding to the
tangle 5/3 and conjoin it into the classifying formula via the
composition connective of the logic.

Finally, we wish to observe that it is entirely within reason to
contemplate a more domain-specific version of spatial logic tailored
to the shape of processes in the image of the encoding. Such a
domain-specific logic would have a better claim to the title formal
language of knot properties.

% subsection example_formulae_ (end)

% section knots_as_processes (end) 

% section spatial logic via knots (end)

\section{Conclusions and future work}

\paragraph{Testing physical space}
You, gentle reader, may wonder why of all the theorems to be proved
given this set up we pick the one above. In some sense it's hardly
central to quantum mechanics. We see it as central in the sense that
it firmly establishes a notion of physical space arising from a notion
of the equivalence of behavior. Relating bisimulation to a metric is a
big step forward, but one is faced with interpreting the relationship
of that metric space to something more physical. Quantum mechanical
notions of ``physical'' space are still far from intuitive, but by
relating this idea of distance as testing to calculations that predict
physical circumstances we are making a not insignificant step forward
toward an understanding of the physical space we inhabit as
essentially dynamic.

\paragraph{Effectivity and simulation}
One of the observations we have yet to make is that the entire program
spelled out here is effective. We have built various interpreters for
the reflective calculus at work in this interpretation. In principle,
then, we can simulate quantum mechanics on a computer. The place where
the simulation may lose fidelity is the infinitely branching summation
for the annihilator.

In this connection i also want to point out that the evaluation style
calculation of the inner product puts the non-determinism of the
summation right at the heart of measurement. This suggests that
Milner's original reduction-based formulation of the dynamics of his
calculi in terms of sums was not just notationally suggestive of a
notion of measure-and-continue but captured some significant part of
the physics.

\paragraph{Quantum continuations}
In light of this last observation i want to point out that the
predominant account of quantum mechanics is missing a key aspect of a
truly compositional story of the physical situation. In a real lab,
when a measurement is made the observation can be made to feed into
another device that then makes another measurement conditioned on the
results of the first. This means that after the superposition was
collapsed the entire experimental set up remained in
superposition. While QM offers a means of writing this down it doesn't
quite line up well with the well-trodden formulation of computation
and continuation that we see so succinctly expressed in Milner's
calculi. This suggests that there might be advantages to this account
of dynamics waiting to be explored.

\paragraph{Quantum logic}
In this connection, we also note that by virtue of having the
Hennessy-Milner construction, we can pull the construction through the
interpretation of QM. This gives us a natural candidate for a quantum
logic that enjoys an extremely tight connection with it's domain of
interpretation, making the construction much less ad hoc (rather it is
the image of functor!).

\paragraph{Quantum probabiity}
i have questions about the basis of the interpretation of inner
product as probability amplitude. In particular, using which
axiomatization of probability theory does the notion of probability
amplitude earn the right to be so dubbed? In other words, where is the
proof that the operation for calculating a probability amplitude (and
then squaring) satisfies the axioms of what it means to calculate a
probability? Even if such a proof exists (i have yet to find it in the
literature), i wonder if it might not be possible to turn things on
their heads. Can we view the calculation of the probability amplitude
as an axiomatization of probability? If so, then the definition we
give for calculating probability amplitude may provide the basis for
an \emph{effective} theory of probability.

\paragraph{Quantum vs ``biological'' information}
Finally, i want to conclude with a more philosophical observation. At
a recent workshop in which QM was a predominant topic i noticed
something about quantum information. The speaker was giving a riveting
discussion of axiomatic QM and showing how properties of ``no
cloning'' and ``no deleting'' emerged as consequences of the
axiomatization. Theorems of this form are necessary to give us a sense
of confidence that our axioms characterize the physical theory. What
struck me, though, was that if quantum information is neither erasable
nor replicable it is markedly different from \emph{life}. Two of the
things we know about life is that

\begin{itemize}
  \item it ends;
  \item to gain some measure of persistence, to transcend it's
    finitude it is imminently copyable.
\end{itemize}

Both of these qualities are summarized succinctly in the aphorism: all
flesh is grass. For me these two kinds of ``information'' -- call them
quantum and biological -- are end points on a spectrum of strategies
for persistence. At one end, we have those curious entities that enjoy
uniqueness and permanence; at the other, we have those who in the face
of a certain end and an uncertain present make a go of passing
something on. To me one of the more remarkable aspects of the latter
strategy is that in the presence of noise (and certain features of
copying) we get a kind of dynamism, a chance for improvement against a
given persistent condition.

% subsection other_calculi_other_bisimulations_and_geometry_as_behavior (end)




% section conclusion (end)

%\documentclass[12pt]{llncs}
%\documentclass{jktr}

\usepackage[pdftex]{hyperref}                   
\usepackage {listings}
\usepackage {mathpartir}
\usepackage{bcprules}
%\usepackage{listings}
                       
\usepackage{graphicx} 
%\usepackage[margins=2.5cm,nohead,nofoot]{geometry}
%\usepackage{geometry}
\usepackage{amsfonts}
\usepackage{amstext}
\usepackage{latexsym}
\usepackage{amssymb}
\usepackage{color}


%\include{myPreamble}
\include{qm2pi.local} 

%\ifpdf
%\usepackage[pdftex]{graphicx}
%\else
%\usepackage{graphicx}
%\fi

 % \ifpdf
%  \usepackage{pdfsync}
%  \if


%\title{Brief Article}
%\author{David F. Snyder}
%\author{L.G. Meredith}

%\address{Dept. of Math., Texas State University--San Marcos, San Marcos, TX 78666}
       
\pagestyle{empty}


\begin{document}

\lstset{language=[Objective]Caml,frame=shadowbox}

\input{qm2pi.front}

% section front matter (end)

\input{qm2pi.intro} 
 
% section introduction (end)

% \input{qm2pi.knotations} 

% section notation (end)

\input{qm2pi.process.calculi} 

% section concurrent_process_calculi_and_spatial_logics_ (end)
    
%\input{qm2pi.knots2pi} 

%\input{qm2pi.trefoil} 

%\input{qm2pi.mainthm} 

% subsection basic_interpretation (end)

%\input{qm2pi.rho.presentation} 
\subsection{The syntax and semantics of the notation system}\label{sub:the_syntax_and_semantics_of_the_notation_system} % (fold)

We now summarize a technical presentation of the calculus that
embodies our theory of dynamics. The typical presentation of such a
calculus follows the style of giving generators and relations on
them. The grammar, below, describing term constructors, freely
generates the set of processes, $\Proc$. This set is then quotiented
by a relation known as structural congruence and it is over this set
that the notion of dynamics is expressed. This presentation is
essentially that of \cite{MeredithR05} with the addition of
polyadicity and summation. For readability we have relegated some of
the technical subtleties to an appendix.

\subsubsection{Process grammar}\label{subsub:process_grammar}

\begin{mathpar}
  \inferrule* [lab=synchronization] {} {{M} \bc \pzero \;|\; x?F \;|\; x!C }
  \and
  \inferrule* [lab=abstraction] {} {{F} \bc (x)P}
  \and
  \inferrule* [lab=concretion] {} {{C} \bc \langle Q \rangle}
  \and
  \inferrule* [lab=process] {} {{P,Q} \bc M \;| \;P|Q \;|\; @{x}}
  \and
  \inferrule* [lab=name] {} {{x} \bc \quotep{P}}
\end{mathpar} 

Note that $\vec{x}$ (resp. $\vec{P}$) denotes a vector of names
(resp. processes) of length $|\vec{x}|$ (resp. $|\vec{P}|$). We adopt
the following useful abbreviations.

\begin{mathpar}
   x?(\vec{y}).P := x.(\vec{y})P \and  x\clift{\vec{P}} := x.\clift{\vec{P}}
   \and x!(y) := \lift{x}{\dropn{y}}
   \and \Pi_{i=0}^{n-1}P_i := P_0 | \ldots | P_{n-1}
\end{mathpar}

\subsubsection{Structural congruence}

\paragraph{Free and bound names and alpha-equivalence.} At the
core of structural equivalence is alpha-equivalence which identifies
process that are the same up to a change of variable. Formally, we
recognize the distinction between free and bound names. The free names
of a process, $\freenames{P}$, may be calculated recursively as
follows:

\begin{mathpar}
\freenames{\pzero} := \emptyset
  \and \\
  \freenames{x?(y).P} := \{ x \} \cup (\freenames{P} \setminus \{ y \})
  \and 
  \freenames{x!\langle P \rangle} := \{ x \} \cup \{ P \} 
  \and \\
  \freenames{P|Q} := \freenames{P} \cup \freenames{Q}
  \and \\
  \freenames{@{x}} := \{ x \}
\end{mathpar}

$\pi$
$\quotep{\pi}$

$\freenames{-} : \pi \to \mathcal{P}(\quotep{\pi})$

\begin{eqnarray*}
  \freenames{\pzero} & := & \emptyset \\
  \freenames{x?(y).P} & := & \{ x \} \cup (\freenames{P} \setminus \{ y \}) \\
  \freenames{x!\langle P \rangle} & := & \{ x \} \cup \{ P \} \\
  \freenames{P|Q} & := & \freenames{P} \cup \freenames{Q} \\
  \freenames{\dropn{x}} & := & \{ x \}
\end{eqnarray*}

The bound names of a process, $\boundnames{P}$, are those names occurring in $P$
that are not free. For example, in $x?(y).0$, the name $x$ is free, while $y$ is bound.

\begin{mathpar}
  \inferrule* [lab=monoidal-laws] {} { P|Q \equiv Q|P \and P|0 \equiv P \and P|(Q|R) \equiv (P|Q)|R }
\end{mathpar}

\begin{mathpar}
  \inferrule* [lab=alpha-equivalence] {} { (x)P \equiv (y)P\{y/x\} \and y \not\in \freenames{P} }
\end{mathpar}

\begin{definition}
Then two processes, $P,Q$, are alpha-equivalent if $P = Q\{\vec{y}/\vec{x}\}$ for
some $\vec{x} \in \boundnames{Q},\vec{y} \in \boundnames{P}$, where $Q\{\vec{y}/\vec{x}\}$
denotes the capture-avoiding substitution of $\vec{y}$ for $\vec{x}$ in $Q$.
\end{definition}

\begin{definition}
  The {\em structural congruence} \cite{SangiorgiWalker} , $\equiv$,
  between processes is the least congruence containing
  alpha-equivalence, satisfying the abelian monoid laws
  (associativity, commutativity and $\pzero$ as identity) for parallel
  composition $|$ and for summation $+$.
\end{definition}

\subsection{Name equivalence}

We take name equivalence, written $\nameeq$, to be the smallest
equivalence relation generated by the following rules.

\begin{mathpar}
\inferrule*[lab=Quote-drop]
{ }
{ \quotep{@{x}} \nameeq x }

\inferrule*[lab=Struct-equiv]
{ P \scong Q }
{ \quotep{P} \nameeq \quotep{Q} }
\end{mathpar}

The astute reader will have noticed that the mutual recursion of names
and processes imposes a mutual recursion on alpha-equivalence and
structural equivalence via name-equivalence. Fortunately, all of this
works out pleasantly and we may calculate in the natural way, free of
concern. The reader interested in the details is referred to the
appendix \ref{appendix:rho_details}.

\subsection{Substitution}

We use $\Proc$ for the set of processes, $\QProc$ for the set of
names, and $\id{\{}\vec{y} / \vec{x} \id{\}}$ to denote partial maps,
$s : \QProc \rightarrow \QProc$. A map, $s$ lifts, uniquely, to a map
on process terms, $\widehat{s} : \Proc \rightarrow \Proc$ by the
following equations.

\begin{mathpar}
  (0) \psubstp{Q}{P} := 0 \\
  (R \juxtap S) \psubstp{Q}{P}
  :=    
  (R)\psubstp{Q}{P} \juxtap (S) \psubstp{Q}{P} \\
  (x?(y).R) \psubstp{Q}{P}    
  :=    
  (x)\substp{Q}{P} (z)\concat( (R \psubstn{z}{y}) \psubstp{Q}{P} ) \\
  (\lift{x}{R}) \psubstp{Q}{P}  
  :=
  \lift{(x)\substp{Q}{P}}{ R \psubstp{Q}{P} } \\
%   (\dropn{x})  \psubstp{Q}{P}       
%   := 
%   \left\{ 
%     \begin{array}{ccc} 
%       \dropn{\quotep{Q}} & & x \nameeq \quotep{P} \\
%       \dropn{x} & & otherwise \\
%     \end{array}
%   \right. 
  (\dropn{x})  \psubstp{Q}{P}       
  := 
  \left\{ 
    \begin{array}{ccc} 
      Q & & x \nameeq \quotep{P} \\
      \dropn{x} & & otherwise \\
    \end{array}
  \right.
\end{mathpar}
 

where

\begin{eqnarray}
  (x)\id{\{} \lpquote Q \rpquote / \lpquote P \rpquote \id{\}}            = 
  \left\{ 
    \begin{array}{ccc}
      \lpquote Q \rpquote & & x \nameeq \lpquote P \rpquote \\
      x & & otherwise \\
    \end{array}
  \right. \nonumber
\end{eqnarray}

and $z$ is chosen distinct from $\quotep{P}$, $\quotep{Q}$, the free
names in $Q$, and all the names in $R$. Our $\alpha$-equivalence will
be built in the standard way from this substitution.

\begin{remark}\label{rem:no_self_referential_names}
  One consequence of these definitions is that $\forall P. \quotep{P}
  \not\in \freenames{P}$.
\end{remark}

\subsection{ Dynamic quote: an example }

Anticipating something of what's to come, consider applying the
substitution, $\widehat{\id{\{}u / z \id{\}}}$, to the following pair
of processes, $\lift{w}{y!(z)}$ and $w[ \lpquote y!(z) \rpquote ]$.

\begin{eqnarray}
	\lift{w}{y!(z)}\widehat{\id{\{}u / z \id{\}}}
		& = &
		\lift{w}{y!(u)} \nonumber\\
	w[ \lpquote y!(z) \rpquote ] \widehat{ \id{\{}u / z \id{\}} }
		& = &
		w[ \lpquote y!(z) \rpquote ] \nonumber
\end{eqnarray}

Because the body of the process between quotes is impervious to
substitution, we get radically different answers. In fact, by
examining the first process in an input context,
e.g. $x?(z).\lift{w}{y!(z)}$, we see that the process under the lift
operator may be shaped by prefixed inputs binding a name inside it. In
this sense, the lift operator will be seen as a way to dynamically
construct processes before reifying them as names.

Finally equipped with these standard features we can present the
dynamics of the calculus.

\subsubsection{Operational semantics} 

Finally, we introduce the computational dynamics. What marks these
algebras as distinct from other more traditionally studied algebraic
structures, e.g. vector spaces or polynomial rings, is the manner in
which dynamics is captured. In traditional structures, dynamics is typically
expressed through morphisms between such structures, as in linear maps
between vector spaces or morphisms between rings. In algebras
associated with the semantics of computation, the dynamics is
expressed as part of the algebraic structure itself, through a
reduction reduction relation typically denoted by $\red$. Below, we
give a recursive presentation of this relation for the calculus used
in the encoding.

$\red \subseteq \pi \times \pi$
$\red : \pi \to \mathcal{P}(\pi)$

\begin{mathpar}
  \inferrule* [lab=Comm] { \textsf{match}( x_{src}, x_{trgt} ) } { x_{trgt}?(y)P \; | \; x_{src}!\langle {Q} \rangle \red P\{\quotep{Q}/y}\} }
  \and \\
  \inferrule* [lab=Par] {{P} \red {P}'} {{{P} | {Q}} \red {{P}' | {Q}}}
  \and
  \inferrule* [lab=Equiv]{{{P} \scong {P}'} \andalso {{P}' \red {Q}'} \andalso {{Q}' \scong {Q}}}{{P} \red {Q}}
\end{mathpar}

\begin{eqnarray*}
  match_{\equiv} (\quotep{P},\quotep{Q}) & := & P \equiv Q \\
  match_{\dagger}(\quotep{P},\quotep{Q}) & := & \forall R. P|Q \red^{*} R => R \red^{*} 0 \\
  match_{K}(\quotep{P},\quotep{Q}) & := & K \mbox{ for some context } K
\end{eqnarray*}

$u?(x)P | u!\langle Q \rangle \red P\{\quotep{Q}/x\}$

%We write $\wred$ for $\red^*$, and $P\red$ if $\exists Q $ such that $ P \red Q$.
We write $P\red$ if $\exists Q $ such that $ P \red Q$ and $P\not\red$, otherwise.

\section{Replication}

As mentioned before, it is known that replication (and hence
recursion) can be implemented in a higher-order process algebra
\cite{SangiorgiWalker}. As our first example of calculation with the
machinery thus far presented we give the construction explicitly in
the {\rhoc}.

\begin{eqnarray}
	D_{x} & := & \prefix{x}{y}{(\binpar{\outputp{x}{y}}{@{y}})} \nonumber\\
	\bangp_{x}{P} & := & \binpar{{x}!\langle{\binpar{D_{x}}{P}}\rangle}{D_{x}} \nonumber
\end{eqnarray}

\begin{eqnarray}
	\bangp_{x}{P} & & \nonumber\\
	=
	& {x}!\langle{(\prefix{x}{y}{(\outputp{x}{y} | @{y})) | P}}\rangle 
	      | \prefix{x}{y}{(\outputp{x}{y} | @{y})} & \nonumber\\
	\red
	& (\outputp{x}{y} | @{y})\substn{\quotep{(\prefix{x}{y}{(@{y} | \outputp{x}{y})) | P}}}{y} & \nonumber\\
	=
	& \outputp{x}{\quotep{(\prefix{x}{y}{(\outputp{x}{y} | @{y})) | P}}}
	  | {(\prefix{x}{y}{(\outputp{x}{y} | @{y})) | P}} & \nonumber\\
	\red
	& \ldots & \nonumber\\
	\red^*
	& P | P | \ldots & \nonumber
\end{eqnarray}

Of course, this encoding, as an implementation, runs away, unfolding
$\bangp{P}$ eagerly. A lazier and more implementable replication
operator, restricted to input-guarded processes, may be obtained as follows.

\begin{eqnarray}
\bangp{\prefix{u}{v}{P}} 
	:= 
	\binpar{\lift{x}{\prefix{u}{v}{(\binpar{D(x)}{P})}}}{D(x)} \nonumber
\end{eqnarray}

\begin{remark}
  Note that the lazier definition still does not deal with summation
  or mixed summation (i.e. sums over input and output). The reader is
  invited to construct definitions of replication that deal with these
  features. 

  Further, the definitions are parameterized in a name, $x$. Can you,
  gentle reader, make a definition that eliminates this parameter and
  guarantees no accidental interaction between the replication
  machinery and the process being replicated -- i.e. no accidental
  sharing of names used by the process to get its work done and the
  name(s) used by the replication to effect copying. This latter
  revision of the definition of replication is crucial to obtaining
  the expected identity $!!P \sim !P$.
\end{remark}

\begin{remark}\label{rem:paradoxical_combinator}
  The reader familiar with the lambda calculus will have noticed the
  similarity between $D$ and the paradoxical combinator.

  [Ed. note: the existence of this seems to suggest we have to be more
  restrictive on the set of processes and names we admit if we are to
  support no-cloning.]
\end{remark}

\subsubsection{Bisimulation}

The computational dynamics gives rise to another kind of equivalence,
the equivalence of computational behavior. As previously mentioned
this is typically captured \emph{via} some form of bisimulation.

% The notion we use in this paper is weak barbed bisimulation
% \cite{milner91polyadicpi}.

The notion we use in this paper is derived from weak barbed
bisimulation \cite{milner91polyadicpi}. 

\begin{definition}
An \emph{observation relation}, $\downarrow_{\mathcal N}$, over a set
of names, $\mathcal N$, is the smallest relation satisfying the rules
below.

\infrule[Out-barb]{y \in {\mathcal N}, \; x \nameeq y}
		  {\outputp{x}{v} \downarrow_{\mathcal N} x}
\infrule[Par-barb]{\mbox{$P\downarrow_{\mathcal N} x$ or $Q\downarrow_{\mathcal N} x$}}
		  {\binpar{P}{Q} \downarrow_{\mathcal N} x}

We write $P \Downarrow_{\mathcal N} x$ if there is $Q$ such that 
$P \wred Q$ and $Q \downarrow_{\mathcal N} x$.
\end{definition}

\begin{definition}
%\label{def.bbisim}
An  ${\mathcal N}$-\emph{barbed bisimulation} over a set of names, ${\mathcal N}$, is a symmetric binary relation 
${\mathcal S}_{\mathcal N}$ between agents such that $P\rel{S}_{\mathcal N}Q$ implies:
\begin{enumerate}
\item If $P \red P'$ then $Q \wred Q'$ and $P'\rel{S}_{\mathcal N} Q'$.
\item If $P\downarrow_{\mathcal N} x$, then $Q\Downarrow_{\mathcal N} x$.
\end{enumerate}
$P$ is ${\mathcal N}$-barbed bisimilar to $Q$, written
$P \wbbisim_{\mathcal N} Q$, if $P \rel{S}_{\mathcal N} Q$ for some ${\mathcal N}$-barbed bisimulation ${\mathcal S}_{\mathcal N}$.
\end{definition}

$\mathcal{R} \subseteq \pi \times \pi$

$P \mathcal{R} Q => \forall P'. P \red P' \Rightarrow \exists Q'. Q \red Q', P' \mathcal{R} Q'$

$P \vdash x \Rightarrow Q \vdash x$

\begin{mathpar}
  \inferrule*[lab=Out-barb]{x \nameeq y}{{y}!\langle{Q}\rangle \vdash x}
  \and
  \inferrule*[lab=Par-barb]{\mbox{$P\vdash x$ or $Q\vdash x$}}{\binpar{P}{Q} \vdash x}
\end{mathpar}

\subsubsection{Contexts}

One of the principle advantages of computational calculi like the
$\pi$-calculus is a well-defined notion of context,
contextual-equivalence and a correlation between
contextual-equivalence and notions of bisimulation. The notion of
context allows the decomposition of a process into (sub-)process and
its syntactic environment, its context. Thus, a context may be
thought of as a process with a ``hole'' (written $\Box$) in it. The
application of a context $M$ to a process $P$, written $M[P]$, is
tantamount to filling the hole in $M$ with $P$. In this paper we do
not need the full weight of this theory, but do make use of the notion
of context in the proof the main theorem. 

\begin{mathpar}
  \inferrule* [lab=summation] {} {{M_{M},M_{N}} \bc \Box \;|\; x.M_{A} \;|\; M_{M}+M_{N}}
  \and
  \inferrule* [lab=agent] {} {{M_{A}} \bc (\vec{x})M_{P} \;| \; \clift{P_0,\ldots,M_{P},\ldots,P_N}}
  \and \\
  \inferrule* [lab=process] {} {{M_{P}} \bc M_{N} \;| \;P|M_{P} }
\end{mathpar} 

\begin{mathpar}
  \inferrule* [lab=sychronization] {} {M_{N} \bc \Box \;|\; x?M_{F} \;|\; x!M_{C}}
  \and
  \inferrule* [lab=abstraction] {} {{M_{F}} \bc (x)M_{P} }
  \and
  \inferrule* [lab=concretion] {} {{M_{C}} \bc \langle M_{P} \rangle }
  \and \\
  \inferrule* [lab=process] {} {{M_{P}} \bc M_{N} \;| \;P|M_{P} }
\end{mathpar}

\begin{definition}[contextual application] Given a context $M$, and
  process $P$, we define the \emph{contextual application}, $M[P] :=
  M\{P/\Box\}$. That is, the contextual application of M to P is the
  substitution of $P$ for $\Box$ in $M$.
\end{definition}

$\meaningof{-} : L \to \mathcal{P}(\pi)$

\begin{mathpar}
  \inferrule* [lab=collection] {} {\meaningof{true} = \pi, \and \meaningof{~E} = \pi \setminus \meaningof{E}, \and \meaningof{E_{1} \& E_{2}} = \meaningof{E_{1}} \cap \meaningof{E_{2}}}
\end{mathpar}

\begin{mathpar}
  \inferrule* [lab=structure] {} {\meaningof{0} = \{ P \in \pi | P \equiv 0 \}, \and \\ \meaningof{E_1 | E_2} = \{ P \in \pi | P \equiv P_{1} | P_{2}, P_{1} \in \meaningof{E_{1}}, P_{2} \in \meaningof{E_2}\} }
\end{mathpar}

\begin{mathpar}
 \inferrule* [lab=behavior] {} {\meaningof{\langle a?b \rangle E} = \{ P \in \pi | P \equiv Q | u?(y)P', \\ \and \\\\ \and \\ \;\;\; u \in \meaningof{a}, \forall z.P'\{z/y\} \in \meaningof{E\{z/b\}}\}, \and \\ \meaningof{a!E} = \{ P \in \pi | P \equiv Q | x!\langle P' \rangle, x \in \meaningof{a} P' \in \meaningof{E}\} }
\end{mathpar}

\begin{mathpar}
 \inferrule* [lab=nominal] {} {\meaningof{\quotep{E}} = \{ \quotep{P} \in \quotep{\pi} | P \in \meaningof{E} \}, \and \meaningof{\quotep{P}} = \{ \quotep{Q} \in \quotep{\pi} | P \equiv Q \} \and \\ \meaningof{@\quotep{E}} = \{ P \in \pi | P \equiv @x, x \in \meaningof{E} \}}
\end{mathpar}

\begin{eqnarray*}
  \\
  \meaningof{-} : TS \to ST
\end{eqnarray*}

\begin{eqnarray*}
  \\
  L : TS \to ST
\end{eqnarray*}

\begin{eqnarray*}
  \\
  P \models E \iff P \in \meaningof{E}
\end{eqnarray*}

\begin{eqnarray*}
  P \approx_{L} Q \iff \forall E \in L. P \models E \iff Q \models E
\end{eqnarray*}

\begin{eqnarray*}
  P \approx_{K} Q
\end{eqnarray*}

\begin{eqnarray*}
  P \approx Q
\end{eqnarray*}

$\approx_{K} = \approx = \approx_{L}$

\subsubsection{Contextual duality}

Note that contexts extend the quotation operation to a family of
operations from processes to names. Given a context, $M$, we can
define a \emph{nominal context}, $\quotep{M}$ by $\quotep{M}[P] :=
\quotep{M[P]}$. To foreshadow what is to come we observe that these
operations enjoy a duality with processes very much like the duality
between vectors and maps from vectors to scalars.

Further, because the calculus is essentially higher-order, we have a
correspondence between contexts and processes. More specifically,
given a name $x$ and a context $M$ we can construct $M^{*}_{x}$ such
that 

\begin{mathpar}
  M^{*}_{x} | \lift{x}{P} \red M[P]
\end{mathpar}

namely,

\begin{mathpar}
  M^{*}_{x} := x?(u).M[\dropn{u}]
\end{mathpar}

The dependence of $M^{*}_{x}$ on a name makes it an abstraction, 

\begin{mathpar}
  M^{*} := (x)x?(u).M[\dropn{u}]
\end{mathpar}

\subsection{Additional notation}

It will sometimes be convenient to denote the process a name
quotes. We already have the notation $x = \quotep{P}$, but it will be
convenient to introduce an alternate notation, $\procn{x}$, when we
want to emphasize the connection to the use of the name. Note that, by
virtue of name equivalence, $\quotep{\procn{x}} \nameeq x$; so, the
notation is consistent with previous definitions.

Further, because names have structure it is possible to effect
substitutions on the basis of that structure. This means we need to
upgrade our notation for substitutions, which we accomplish by
adapting comprehension notation. Thus,

\begin{mathpar}
  P\{ y / x : x \in S \}
\end{mathpar}

is interpreted to mean the process derived from P by replacing (in a
capture-avoiding manner) each occurrence of $x$ in $S$ by $y$. For example,

\begin{mathpar}
  P\{ \quotep{\procn{x}|\procn{x}} / x : x \in \freenames{P} \}
\end{mathpar}

will replace each (occurrence) of a free name $x$ in $P$ by
$\quotep{\procn{x}|\procn{x}}$.

Also, we will avail ourselves of the notation $x^{L}$ and $x^{R}$ to
denote injections of a name into disjoint copies of the name
space. There are numerous ways to accomplish this. One example can be
found in \cite{MeredithR05}. This notation overloads to vectors of
names: $\vec{x}^{\pi} := (x_{i}^{\pi} \; : \; 0 \leq i < |\vec{x}| )$ where $\pi \in \{L,R\}$.

We also use $P^{\Box} := P|\Box$.

In \cite{MeredithR05} an interpretation of the new operator is
given. It turns out that there are several possible interpretations
all enjoying the requisite algebraic properties of the operator (see
\cite{milner91polyadicpi}). We will therefore make liberal use of
$(\nu\; \vec{x})P$.

% subsection the_syntax_and_semantics_of_the_notation_system (end)   

\input{qm2pi.qmops} 

\input{qm2pi.sterngerlach} 

\input{qm2pi.metric} 

% section concurrent_process_calculi (end)

%\input{qm2pi.proofsketch}

% section proof sketch (end)

%\input{qm2pi.slviaknots} 

% section spatial logic via knots (end)

\input{qm2pi.conclusion}

% section conclusion (end)

%\input{qm2pi.dtcodes} 

% section wiring algorithm (end)

\input{qm2pi.ack} 

% section acknowledgments (end)

\newpage


\bibliographystyle{plain}   
\bibliography{../../biblios/main.bib}

\input{qm2pi.rhodetails}

\end{document}

 

% section wiring algorithm (end)

\documentclass[12pt]{llncs}
%\documentclass{jktr}

\usepackage[pdftex]{hyperref}                   
\usepackage {listings}
\usepackage {mathpartir}
\usepackage{bcprules}
%\usepackage{listings}
                       
\usepackage{graphicx} 
%\usepackage[margins=2.5cm,nohead,nofoot]{geometry}
%\usepackage{geometry}
\usepackage{amsfonts}
\usepackage{amstext}
\usepackage{latexsym}
\usepackage{amssymb}
\usepackage{color}


%\include{myPreamble}
\include{qm2pi.local} 

%\ifpdf
%\usepackage[pdftex]{graphicx}
%\else
%\usepackage{graphicx}
%\fi

 % \ifpdf
%  \usepackage{pdfsync}
%  \if


%\title{Brief Article}
%\author{David F. Snyder}
%\author{L.G. Meredith}

%\address{Dept. of Math., Texas State University--San Marcos, San Marcos, TX 78666}
       
\pagestyle{empty}


\begin{document}

\lstset{language=[Objective]Caml,frame=shadowbox}

\input{qm2pi.front}

% section front matter (end)

\input{qm2pi.intro} 
 
% section introduction (end)

% \input{qm2pi.knotations} 

% section notation (end)

\input{qm2pi.process.calculi} 

% section concurrent_process_calculi_and_spatial_logics_ (end)
    
%\input{qm2pi.knots2pi} 

%\input{qm2pi.trefoil} 

%\input{qm2pi.mainthm} 

% subsection basic_interpretation (end)

%\input{qm2pi.rho.presentation} 
\subsection{The syntax and semantics of the notation system}\label{sub:the_syntax_and_semantics_of_the_notation_system} % (fold)

We now summarize a technical presentation of the calculus that
embodies our theory of dynamics. The typical presentation of such a
calculus follows the style of giving generators and relations on
them. The grammar, below, describing term constructors, freely
generates the set of processes, $\Proc$. This set is then quotiented
by a relation known as structural congruence and it is over this set
that the notion of dynamics is expressed. This presentation is
essentially that of \cite{MeredithR05} with the addition of
polyadicity and summation. For readability we have relegated some of
the technical subtleties to an appendix.

\subsubsection{Process grammar}\label{subsub:process_grammar}

\begin{mathpar}
  \inferrule* [lab=synchronization] {} {{M} \bc \pzero \;|\; x?F \;|\; x!C }
  \and
  \inferrule* [lab=abstraction] {} {{F} \bc (x)P}
  \and
  \inferrule* [lab=concretion] {} {{C} \bc \langle Q \rangle}
  \and
  \inferrule* [lab=process] {} {{P,Q} \bc M \;| \;P|Q \;|\; @{x}}
  \and
  \inferrule* [lab=name] {} {{x} \bc \quotep{P}}
\end{mathpar} 

Note that $\vec{x}$ (resp. $\vec{P}$) denotes a vector of names
(resp. processes) of length $|\vec{x}|$ (resp. $|\vec{P}|$). We adopt
the following useful abbreviations.

\begin{mathpar}
   x?(\vec{y}).P := x.(\vec{y})P \and  x\clift{\vec{P}} := x.\clift{\vec{P}}
   \and x!(y) := \lift{x}{\dropn{y}}
   \and \Pi_{i=0}^{n-1}P_i := P_0 | \ldots | P_{n-1}
\end{mathpar}

\subsubsection{Structural congruence}

\paragraph{Free and bound names and alpha-equivalence.} At the
core of structural equivalence is alpha-equivalence which identifies
process that are the same up to a change of variable. Formally, we
recognize the distinction between free and bound names. The free names
of a process, $\freenames{P}$, may be calculated recursively as
follows:

\begin{mathpar}
\freenames{\pzero} := \emptyset
  \and \\
  \freenames{x?(y).P} := \{ x \} \cup (\freenames{P} \setminus \{ y \})
  \and 
  \freenames{x!\langle P \rangle} := \{ x \} \cup \{ P \} 
  \and \\
  \freenames{P|Q} := \freenames{P} \cup \freenames{Q}
  \and \\
  \freenames{@{x}} := \{ x \}
\end{mathpar}

$\pi$
$\quotep{\pi}$

$\freenames{-} : \pi \to \mathcal{P}(\quotep{\pi})$

\begin{eqnarray*}
  \freenames{\pzero} & := & \emptyset \\
  \freenames{x?(y).P} & := & \{ x \} \cup (\freenames{P} \setminus \{ y \}) \\
  \freenames{x!\langle P \rangle} & := & \{ x \} \cup \{ P \} \\
  \freenames{P|Q} & := & \freenames{P} \cup \freenames{Q} \\
  \freenames{\dropn{x}} & := & \{ x \}
\end{eqnarray*}

The bound names of a process, $\boundnames{P}$, are those names occurring in $P$
that are not free. For example, in $x?(y).0$, the name $x$ is free, while $y$ is bound.

\begin{mathpar}
  \inferrule* [lab=monoidal-laws] {} { P|Q \equiv Q|P \and P|0 \equiv P \and P|(Q|R) \equiv (P|Q)|R }
\end{mathpar}

\begin{mathpar}
  \inferrule* [lab=alpha-equivalence] {} { (x)P \equiv (y)P\{y/x\} \and y \not\in \freenames{P} }
\end{mathpar}

\begin{definition}
Then two processes, $P,Q$, are alpha-equivalent if $P = Q\{\vec{y}/\vec{x}\}$ for
some $\vec{x} \in \boundnames{Q},\vec{y} \in \boundnames{P}$, where $Q\{\vec{y}/\vec{x}\}$
denotes the capture-avoiding substitution of $\vec{y}$ for $\vec{x}$ in $Q$.
\end{definition}

\begin{definition}
  The {\em structural congruence} \cite{SangiorgiWalker} , $\equiv$,
  between processes is the least congruence containing
  alpha-equivalence, satisfying the abelian monoid laws
  (associativity, commutativity and $\pzero$ as identity) for parallel
  composition $|$ and for summation $+$.
\end{definition}

\subsection{Name equivalence}

We take name equivalence, written $\nameeq$, to be the smallest
equivalence relation generated by the following rules.

\begin{mathpar}
\inferrule*[lab=Quote-drop]
{ }
{ \quotep{@{x}} \nameeq x }

\inferrule*[lab=Struct-equiv]
{ P \scong Q }
{ \quotep{P} \nameeq \quotep{Q} }
\end{mathpar}

The astute reader will have noticed that the mutual recursion of names
and processes imposes a mutual recursion on alpha-equivalence and
structural equivalence via name-equivalence. Fortunately, all of this
works out pleasantly and we may calculate in the natural way, free of
concern. The reader interested in the details is referred to the
appendix \ref{appendix:rho_details}.

\subsection{Substitution}

We use $\Proc$ for the set of processes, $\QProc$ for the set of
names, and $\id{\{}\vec{y} / \vec{x} \id{\}}$ to denote partial maps,
$s : \QProc \rightarrow \QProc$. A map, $s$ lifts, uniquely, to a map
on process terms, $\widehat{s} : \Proc \rightarrow \Proc$ by the
following equations.

\begin{mathpar}
  (0) \psubstp{Q}{P} := 0 \\
  (R \juxtap S) \psubstp{Q}{P}
  :=    
  (R)\psubstp{Q}{P} \juxtap (S) \psubstp{Q}{P} \\
  (x?(y).R) \psubstp{Q}{P}    
  :=    
  (x)\substp{Q}{P} (z)\concat( (R \psubstn{z}{y}) \psubstp{Q}{P} ) \\
  (\lift{x}{R}) \psubstp{Q}{P}  
  :=
  \lift{(x)\substp{Q}{P}}{ R \psubstp{Q}{P} } \\
%   (\dropn{x})  \psubstp{Q}{P}       
%   := 
%   \left\{ 
%     \begin{array}{ccc} 
%       \dropn{\quotep{Q}} & & x \nameeq \quotep{P} \\
%       \dropn{x} & & otherwise \\
%     \end{array}
%   \right. 
  (\dropn{x})  \psubstp{Q}{P}       
  := 
  \left\{ 
    \begin{array}{ccc} 
      Q & & x \nameeq \quotep{P} \\
      \dropn{x} & & otherwise \\
    \end{array}
  \right.
\end{mathpar}
 

where

\begin{eqnarray}
  (x)\id{\{} \lpquote Q \rpquote / \lpquote P \rpquote \id{\}}            = 
  \left\{ 
    \begin{array}{ccc}
      \lpquote Q \rpquote & & x \nameeq \lpquote P \rpquote \\
      x & & otherwise \\
    \end{array}
  \right. \nonumber
\end{eqnarray}

and $z$ is chosen distinct from $\quotep{P}$, $\quotep{Q}$, the free
names in $Q$, and all the names in $R$. Our $\alpha$-equivalence will
be built in the standard way from this substitution.

\begin{remark}\label{rem:no_self_referential_names}
  One consequence of these definitions is that $\forall P. \quotep{P}
  \not\in \freenames{P}$.
\end{remark}

\subsection{ Dynamic quote: an example }

Anticipating something of what's to come, consider applying the
substitution, $\widehat{\id{\{}u / z \id{\}}}$, to the following pair
of processes, $\lift{w}{y!(z)}$ and $w[ \lpquote y!(z) \rpquote ]$.

\begin{eqnarray}
	\lift{w}{y!(z)}\widehat{\id{\{}u / z \id{\}}}
		& = &
		\lift{w}{y!(u)} \nonumber\\
	w[ \lpquote y!(z) \rpquote ] \widehat{ \id{\{}u / z \id{\}} }
		& = &
		w[ \lpquote y!(z) \rpquote ] \nonumber
\end{eqnarray}

Because the body of the process between quotes is impervious to
substitution, we get radically different answers. In fact, by
examining the first process in an input context,
e.g. $x?(z).\lift{w}{y!(z)}$, we see that the process under the lift
operator may be shaped by prefixed inputs binding a name inside it. In
this sense, the lift operator will be seen as a way to dynamically
construct processes before reifying them as names.

Finally equipped with these standard features we can present the
dynamics of the calculus.

\subsubsection{Operational semantics} 

Finally, we introduce the computational dynamics. What marks these
algebras as distinct from other more traditionally studied algebraic
structures, e.g. vector spaces or polynomial rings, is the manner in
which dynamics is captured. In traditional structures, dynamics is typically
expressed through morphisms between such structures, as in linear maps
between vector spaces or morphisms between rings. In algebras
associated with the semantics of computation, the dynamics is
expressed as part of the algebraic structure itself, through a
reduction reduction relation typically denoted by $\red$. Below, we
give a recursive presentation of this relation for the calculus used
in the encoding.

$\red \subseteq \pi \times \pi$
$\red : \pi \to \mathcal{P}(\pi)$

\begin{mathpar}
  \inferrule* [lab=Comm] { \textsf{match}( x_{src}, x_{trgt} ) } { x_{trgt}?(y)P \; | \; x_{src}!\langle {Q} \rangle \red P\{\quotep{Q}/y}\} }
  \and \\
  \inferrule* [lab=Par] {{P} \red {P}'} {{{P} | {Q}} \red {{P}' | {Q}}}
  \and
  \inferrule* [lab=Equiv]{{{P} \scong {P}'} \andalso {{P}' \red {Q}'} \andalso {{Q}' \scong {Q}}}{{P} \red {Q}}
\end{mathpar}

\begin{eqnarray*}
  match_{\equiv} (\quotep{P},\quotep{Q}) & := & P \equiv Q \\
  match_{\dagger}(\quotep{P},\quotep{Q}) & := & \forall R. P|Q \red^{*} R => R \red^{*} 0 \\
  match_{K}(\quotep{P},\quotep{Q}) & := & K \mbox{ for some context } K
\end{eqnarray*}

$u?(x)P | u!\langle Q \rangle \red P\{\quotep{Q}/x\}$

%We write $\wred$ for $\red^*$, and $P\red$ if $\exists Q $ such that $ P \red Q$.
We write $P\red$ if $\exists Q $ such that $ P \red Q$ and $P\not\red$, otherwise.

\section{Replication}

As mentioned before, it is known that replication (and hence
recursion) can be implemented in a higher-order process algebra
\cite{SangiorgiWalker}. As our first example of calculation with the
machinery thus far presented we give the construction explicitly in
the {\rhoc}.

\begin{eqnarray}
	D_{x} & := & \prefix{x}{y}{(\binpar{\outputp{x}{y}}{@{y}})} \nonumber\\
	\bangp_{x}{P} & := & \binpar{{x}!\langle{\binpar{D_{x}}{P}}\rangle}{D_{x}} \nonumber
\end{eqnarray}

\begin{eqnarray}
	\bangp_{x}{P} & & \nonumber\\
	=
	& {x}!\langle{(\prefix{x}{y}{(\outputp{x}{y} | @{y})) | P}}\rangle 
	      | \prefix{x}{y}{(\outputp{x}{y} | @{y})} & \nonumber\\
	\red
	& (\outputp{x}{y} | @{y})\substn{\quotep{(\prefix{x}{y}{(@{y} | \outputp{x}{y})) | P}}}{y} & \nonumber\\
	=
	& \outputp{x}{\quotep{(\prefix{x}{y}{(\outputp{x}{y} | @{y})) | P}}}
	  | {(\prefix{x}{y}{(\outputp{x}{y} | @{y})) | P}} & \nonumber\\
	\red
	& \ldots & \nonumber\\
	\red^*
	& P | P | \ldots & \nonumber
\end{eqnarray}

Of course, this encoding, as an implementation, runs away, unfolding
$\bangp{P}$ eagerly. A lazier and more implementable replication
operator, restricted to input-guarded processes, may be obtained as follows.

\begin{eqnarray}
\bangp{\prefix{u}{v}{P}} 
	:= 
	\binpar{\lift{x}{\prefix{u}{v}{(\binpar{D(x)}{P})}}}{D(x)} \nonumber
\end{eqnarray}

\begin{remark}
  Note that the lazier definition still does not deal with summation
  or mixed summation (i.e. sums over input and output). The reader is
  invited to construct definitions of replication that deal with these
  features. 

  Further, the definitions are parameterized in a name, $x$. Can you,
  gentle reader, make a definition that eliminates this parameter and
  guarantees no accidental interaction between the replication
  machinery and the process being replicated -- i.e. no accidental
  sharing of names used by the process to get its work done and the
  name(s) used by the replication to effect copying. This latter
  revision of the definition of replication is crucial to obtaining
  the expected identity $!!P \sim !P$.
\end{remark}

\begin{remark}\label{rem:paradoxical_combinator}
  The reader familiar with the lambda calculus will have noticed the
  similarity between $D$ and the paradoxical combinator.

  [Ed. note: the existence of this seems to suggest we have to be more
  restrictive on the set of processes and names we admit if we are to
  support no-cloning.]
\end{remark}

\subsubsection{Bisimulation}

The computational dynamics gives rise to another kind of equivalence,
the equivalence of computational behavior. As previously mentioned
this is typically captured \emph{via} some form of bisimulation.

% The notion we use in this paper is weak barbed bisimulation
% \cite{milner91polyadicpi}.

The notion we use in this paper is derived from weak barbed
bisimulation \cite{milner91polyadicpi}. 

\begin{definition}
An \emph{observation relation}, $\downarrow_{\mathcal N}$, over a set
of names, $\mathcal N$, is the smallest relation satisfying the rules
below.

\infrule[Out-barb]{y \in {\mathcal N}, \; x \nameeq y}
		  {\outputp{x}{v} \downarrow_{\mathcal N} x}
\infrule[Par-barb]{\mbox{$P\downarrow_{\mathcal N} x$ or $Q\downarrow_{\mathcal N} x$}}
		  {\binpar{P}{Q} \downarrow_{\mathcal N} x}

We write $P \Downarrow_{\mathcal N} x$ if there is $Q$ such that 
$P \wred Q$ and $Q \downarrow_{\mathcal N} x$.
\end{definition}

\begin{definition}
%\label{def.bbisim}
An  ${\mathcal N}$-\emph{barbed bisimulation} over a set of names, ${\mathcal N}$, is a symmetric binary relation 
${\mathcal S}_{\mathcal N}$ between agents such that $P\rel{S}_{\mathcal N}Q$ implies:
\begin{enumerate}
\item If $P \red P'$ then $Q \wred Q'$ and $P'\rel{S}_{\mathcal N} Q'$.
\item If $P\downarrow_{\mathcal N} x$, then $Q\Downarrow_{\mathcal N} x$.
\end{enumerate}
$P$ is ${\mathcal N}$-barbed bisimilar to $Q$, written
$P \wbbisim_{\mathcal N} Q$, if $P \rel{S}_{\mathcal N} Q$ for some ${\mathcal N}$-barbed bisimulation ${\mathcal S}_{\mathcal N}$.
\end{definition}

$\mathcal{R} \subseteq \pi \times \pi$

$P \mathcal{R} Q => \forall P'. P \red P' \Rightarrow \exists Q'. Q \red Q', P' \mathcal{R} Q'$

$P \vdash x \Rightarrow Q \vdash x$

\begin{mathpar}
  \inferrule*[lab=Out-barb]{x \nameeq y}{{y}!\langle{Q}\rangle \vdash x}
  \and
  \inferrule*[lab=Par-barb]{\mbox{$P\vdash x$ or $Q\vdash x$}}{\binpar{P}{Q} \vdash x}
\end{mathpar}

\subsubsection{Contexts}

One of the principle advantages of computational calculi like the
$\pi$-calculus is a well-defined notion of context,
contextual-equivalence and a correlation between
contextual-equivalence and notions of bisimulation. The notion of
context allows the decomposition of a process into (sub-)process and
its syntactic environment, its context. Thus, a context may be
thought of as a process with a ``hole'' (written $\Box$) in it. The
application of a context $M$ to a process $P$, written $M[P]$, is
tantamount to filling the hole in $M$ with $P$. In this paper we do
not need the full weight of this theory, but do make use of the notion
of context in the proof the main theorem. 

\begin{mathpar}
  \inferrule* [lab=summation] {} {{M_{M},M_{N}} \bc \Box \;|\; x.M_{A} \;|\; M_{M}+M_{N}}
  \and
  \inferrule* [lab=agent] {} {{M_{A}} \bc (\vec{x})M_{P} \;| \; \clift{P_0,\ldots,M_{P},\ldots,P_N}}
  \and \\
  \inferrule* [lab=process] {} {{M_{P}} \bc M_{N} \;| \;P|M_{P} }
\end{mathpar} 

\begin{mathpar}
  \inferrule* [lab=sychronization] {} {M_{N} \bc \Box \;|\; x?M_{F} \;|\; x!M_{C}}
  \and
  \inferrule* [lab=abstraction] {} {{M_{F}} \bc (x)M_{P} }
  \and
  \inferrule* [lab=concretion] {} {{M_{C}} \bc \langle M_{P} \rangle }
  \and \\
  \inferrule* [lab=process] {} {{M_{P}} \bc M_{N} \;| \;P|M_{P} }
\end{mathpar}

\begin{definition}[contextual application] Given a context $M$, and
  process $P$, we define the \emph{contextual application}, $M[P] :=
  M\{P/\Box\}$. That is, the contextual application of M to P is the
  substitution of $P$ for $\Box$ in $M$.
\end{definition}

$\meaningof{-} : L \to \mathcal{P}(\pi)$

\begin{mathpar}
  \inferrule* [lab=collection] {} {\meaningof{true} = \pi, \and \meaningof{~E} = \pi \setminus \meaningof{E}, \and \meaningof{E_{1} \& E_{2}} = \meaningof{E_{1}} \cap \meaningof{E_{2}}}
\end{mathpar}

\begin{mathpar}
  \inferrule* [lab=structure] {} {\meaningof{0} = \{ P \in \pi | P \equiv 0 \}, \and \\ \meaningof{E_1 | E_2} = \{ P \in \pi | P \equiv P_{1} | P_{2}, P_{1} \in \meaningof{E_{1}}, P_{2} \in \meaningof{E_2}\} }
\end{mathpar}

\begin{mathpar}
 \inferrule* [lab=behavior] {} {\meaningof{\langle a?b \rangle E} = \{ P \in \pi | P \equiv Q | u?(y)P', \\ \and \\\\ \and \\ \;\;\; u \in \meaningof{a}, \forall z.P'\{z/y\} \in \meaningof{E\{z/b\}}\}, \and \\ \meaningof{a!E} = \{ P \in \pi | P \equiv Q | x!\langle P' \rangle, x \in \meaningof{a} P' \in \meaningof{E}\} }
\end{mathpar}

\begin{mathpar}
 \inferrule* [lab=nominal] {} {\meaningof{\quotep{E}} = \{ \quotep{P} \in \quotep{\pi} | P \in \meaningof{E} \}, \and \meaningof{\quotep{P}} = \{ \quotep{Q} \in \quotep{\pi} | P \equiv Q \} \and \\ \meaningof{@\quotep{E}} = \{ P \in \pi | P \equiv @x, x \in \meaningof{E} \}}
\end{mathpar}

\begin{eqnarray*}
  \\
  \meaningof{-} : TS \to ST
\end{eqnarray*}

\begin{eqnarray*}
  \\
  L : TS \to ST
\end{eqnarray*}

\begin{eqnarray*}
  \\
  P \models E \iff P \in \meaningof{E}
\end{eqnarray*}

\begin{eqnarray*}
  P \approx_{L} Q \iff \forall E \in L. P \models E \iff Q \models E
\end{eqnarray*}

\begin{eqnarray*}
  P \approx_{K} Q
\end{eqnarray*}

\begin{eqnarray*}
  P \approx Q
\end{eqnarray*}

$\approx_{K} = \approx = \approx_{L}$

\subsubsection{Contextual duality}

Note that contexts extend the quotation operation to a family of
operations from processes to names. Given a context, $M$, we can
define a \emph{nominal context}, $\quotep{M}$ by $\quotep{M}[P] :=
\quotep{M[P]}$. To foreshadow what is to come we observe that these
operations enjoy a duality with processes very much like the duality
between vectors and maps from vectors to scalars.

Further, because the calculus is essentially higher-order, we have a
correspondence between contexts and processes. More specifically,
given a name $x$ and a context $M$ we can construct $M^{*}_{x}$ such
that 

\begin{mathpar}
  M^{*}_{x} | \lift{x}{P} \red M[P]
\end{mathpar}

namely,

\begin{mathpar}
  M^{*}_{x} := x?(u).M[\dropn{u}]
\end{mathpar}

The dependence of $M^{*}_{x}$ on a name makes it an abstraction, 

\begin{mathpar}
  M^{*} := (x)x?(u).M[\dropn{u}]
\end{mathpar}

\subsection{Additional notation}

It will sometimes be convenient to denote the process a name
quotes. We already have the notation $x = \quotep{P}$, but it will be
convenient to introduce an alternate notation, $\procn{x}$, when we
want to emphasize the connection to the use of the name. Note that, by
virtue of name equivalence, $\quotep{\procn{x}} \nameeq x$; so, the
notation is consistent with previous definitions.

Further, because names have structure it is possible to effect
substitutions on the basis of that structure. This means we need to
upgrade our notation for substitutions, which we accomplish by
adapting comprehension notation. Thus,

\begin{mathpar}
  P\{ y / x : x \in S \}
\end{mathpar}

is interpreted to mean the process derived from P by replacing (in a
capture-avoiding manner) each occurrence of $x$ in $S$ by $y$. For example,

\begin{mathpar}
  P\{ \quotep{\procn{x}|\procn{x}} / x : x \in \freenames{P} \}
\end{mathpar}

will replace each (occurrence) of a free name $x$ in $P$ by
$\quotep{\procn{x}|\procn{x}}$.

Also, we will avail ourselves of the notation $x^{L}$ and $x^{R}$ to
denote injections of a name into disjoint copies of the name
space. There are numerous ways to accomplish this. One example can be
found in \cite{MeredithR05}. This notation overloads to vectors of
names: $\vec{x}^{\pi} := (x_{i}^{\pi} \; : \; 0 \leq i < |\vec{x}| )$ where $\pi \in \{L,R\}$.

We also use $P^{\Box} := P|\Box$.

In \cite{MeredithR05} an interpretation of the new operator is
given. It turns out that there are several possible interpretations
all enjoying the requisite algebraic properties of the operator (see
\cite{milner91polyadicpi}). We will therefore make liberal use of
$(\nu\; \vec{x})P$.

% subsection the_syntax_and_semantics_of_the_notation_system (end)   

\input{qm2pi.qmops} 

\input{qm2pi.sterngerlach} 

\input{qm2pi.metric} 

% section concurrent_process_calculi (end)

%\input{qm2pi.proofsketch}

% section proof sketch (end)

%\input{qm2pi.slviaknots} 

% section spatial logic via knots (end)

\input{qm2pi.conclusion}

% section conclusion (end)

%\input{qm2pi.dtcodes} 

% section wiring algorithm (end)

\input{qm2pi.ack} 

% section acknowledgments (end)

\newpage


\bibliographystyle{plain}   
\bibliography{../../biblios/main.bib}

\input{qm2pi.rhodetails}

\end{document}

 

% section acknowledgments (end)

\newpage


\bibliographystyle{plain}   
\bibliography{../../biblios/main.bib}

\documentclass[12pt]{llncs}
%\documentclass{jktr}

\usepackage[pdftex]{hyperref}                   
\usepackage {listings}
\usepackage {mathpartir}
\usepackage{bcprules}
%\usepackage{listings}
                       
\usepackage{graphicx} 
%\usepackage[margins=2.5cm,nohead,nofoot]{geometry}
%\usepackage{geometry}
\usepackage{amsfonts}
\usepackage{amstext}
\usepackage{latexsym}
\usepackage{amssymb}
\usepackage{color}


%\include{myPreamble}
\include{qm2pi.local} 

%\ifpdf
%\usepackage[pdftex]{graphicx}
%\else
%\usepackage{graphicx}
%\fi

 % \ifpdf
%  \usepackage{pdfsync}
%  \if


%\title{Brief Article}
%\author{David F. Snyder}
%\author{L.G. Meredith}

%\address{Dept. of Math., Texas State University--San Marcos, San Marcos, TX 78666}
       
\pagestyle{empty}


\begin{document}

\lstset{language=[Objective]Caml,frame=shadowbox}

\input{qm2pi.front}

% section front matter (end)

\input{qm2pi.intro} 
 
% section introduction (end)

% \input{qm2pi.knotations} 

% section notation (end)

\input{qm2pi.process.calculi} 

% section concurrent_process_calculi_and_spatial_logics_ (end)
    
%\input{qm2pi.knots2pi} 

%\input{qm2pi.trefoil} 

%\input{qm2pi.mainthm} 

% subsection basic_interpretation (end)

%\input{qm2pi.rho.presentation} 
\subsection{The syntax and semantics of the notation system}\label{sub:the_syntax_and_semantics_of_the_notation_system} % (fold)

We now summarize a technical presentation of the calculus that
embodies our theory of dynamics. The typical presentation of such a
calculus follows the style of giving generators and relations on
them. The grammar, below, describing term constructors, freely
generates the set of processes, $\Proc$. This set is then quotiented
by a relation known as structural congruence and it is over this set
that the notion of dynamics is expressed. This presentation is
essentially that of \cite{MeredithR05} with the addition of
polyadicity and summation. For readability we have relegated some of
the technical subtleties to an appendix.

\subsubsection{Process grammar}\label{subsub:process_grammar}

\begin{mathpar}
  \inferrule* [lab=synchronization] {} {{M} \bc \pzero \;|\; x?F \;|\; x!C }
  \and
  \inferrule* [lab=abstraction] {} {{F} \bc (x)P}
  \and
  \inferrule* [lab=concretion] {} {{C} \bc \langle Q \rangle}
  \and
  \inferrule* [lab=process] {} {{P,Q} \bc M \;| \;P|Q \;|\; @{x}}
  \and
  \inferrule* [lab=name] {} {{x} \bc \quotep{P}}
\end{mathpar} 

Note that $\vec{x}$ (resp. $\vec{P}$) denotes a vector of names
(resp. processes) of length $|\vec{x}|$ (resp. $|\vec{P}|$). We adopt
the following useful abbreviations.

\begin{mathpar}
   x?(\vec{y}).P := x.(\vec{y})P \and  x\clift{\vec{P}} := x.\clift{\vec{P}}
   \and x!(y) := \lift{x}{\dropn{y}}
   \and \Pi_{i=0}^{n-1}P_i := P_0 | \ldots | P_{n-1}
\end{mathpar}

\subsubsection{Structural congruence}

\paragraph{Free and bound names and alpha-equivalence.} At the
core of structural equivalence is alpha-equivalence which identifies
process that are the same up to a change of variable. Formally, we
recognize the distinction between free and bound names. The free names
of a process, $\freenames{P}$, may be calculated recursively as
follows:

\begin{mathpar}
\freenames{\pzero} := \emptyset
  \and \\
  \freenames{x?(y).P} := \{ x \} \cup (\freenames{P} \setminus \{ y \})
  \and 
  \freenames{x!\langle P \rangle} := \{ x \} \cup \{ P \} 
  \and \\
  \freenames{P|Q} := \freenames{P} \cup \freenames{Q}
  \and \\
  \freenames{@{x}} := \{ x \}
\end{mathpar}

$\pi$
$\quotep{\pi}$

$\freenames{-} : \pi \to \mathcal{P}(\quotep{\pi})$

\begin{eqnarray*}
  \freenames{\pzero} & := & \emptyset \\
  \freenames{x?(y).P} & := & \{ x \} \cup (\freenames{P} \setminus \{ y \}) \\
  \freenames{x!\langle P \rangle} & := & \{ x \} \cup \{ P \} \\
  \freenames{P|Q} & := & \freenames{P} \cup \freenames{Q} \\
  \freenames{\dropn{x}} & := & \{ x \}
\end{eqnarray*}

The bound names of a process, $\boundnames{P}$, are those names occurring in $P$
that are not free. For example, in $x?(y).0$, the name $x$ is free, while $y$ is bound.

\begin{mathpar}
  \inferrule* [lab=monoidal-laws] {} { P|Q \equiv Q|P \and P|0 \equiv P \and P|(Q|R) \equiv (P|Q)|R }
\end{mathpar}

\begin{mathpar}
  \inferrule* [lab=alpha-equivalence] {} { (x)P \equiv (y)P\{y/x\} \and y \not\in \freenames{P} }
\end{mathpar}

\begin{definition}
Then two processes, $P,Q$, are alpha-equivalent if $P = Q\{\vec{y}/\vec{x}\}$ for
some $\vec{x} \in \boundnames{Q},\vec{y} \in \boundnames{P}$, where $Q\{\vec{y}/\vec{x}\}$
denotes the capture-avoiding substitution of $\vec{y}$ for $\vec{x}$ in $Q$.
\end{definition}

\begin{definition}
  The {\em structural congruence} \cite{SangiorgiWalker} , $\equiv$,
  between processes is the least congruence containing
  alpha-equivalence, satisfying the abelian monoid laws
  (associativity, commutativity and $\pzero$ as identity) for parallel
  composition $|$ and for summation $+$.
\end{definition}

\subsection{Name equivalence}

We take name equivalence, written $\nameeq$, to be the smallest
equivalence relation generated by the following rules.

\begin{mathpar}
\inferrule*[lab=Quote-drop]
{ }
{ \quotep{@{x}} \nameeq x }

\inferrule*[lab=Struct-equiv]
{ P \scong Q }
{ \quotep{P} \nameeq \quotep{Q} }
\end{mathpar}

The astute reader will have noticed that the mutual recursion of names
and processes imposes a mutual recursion on alpha-equivalence and
structural equivalence via name-equivalence. Fortunately, all of this
works out pleasantly and we may calculate in the natural way, free of
concern. The reader interested in the details is referred to the
appendix \ref{appendix:rho_details}.

\subsection{Substitution}

We use $\Proc$ for the set of processes, $\QProc$ for the set of
names, and $\id{\{}\vec{y} / \vec{x} \id{\}}$ to denote partial maps,
$s : \QProc \rightarrow \QProc$. A map, $s$ lifts, uniquely, to a map
on process terms, $\widehat{s} : \Proc \rightarrow \Proc$ by the
following equations.

\begin{mathpar}
  (0) \psubstp{Q}{P} := 0 \\
  (R \juxtap S) \psubstp{Q}{P}
  :=    
  (R)\psubstp{Q}{P} \juxtap (S) \psubstp{Q}{P} \\
  (x?(y).R) \psubstp{Q}{P}    
  :=    
  (x)\substp{Q}{P} (z)\concat( (R \psubstn{z}{y}) \psubstp{Q}{P} ) \\
  (\lift{x}{R}) \psubstp{Q}{P}  
  :=
  \lift{(x)\substp{Q}{P}}{ R \psubstp{Q}{P} } \\
%   (\dropn{x})  \psubstp{Q}{P}       
%   := 
%   \left\{ 
%     \begin{array}{ccc} 
%       \dropn{\quotep{Q}} & & x \nameeq \quotep{P} \\
%       \dropn{x} & & otherwise \\
%     \end{array}
%   \right. 
  (\dropn{x})  \psubstp{Q}{P}       
  := 
  \left\{ 
    \begin{array}{ccc} 
      Q & & x \nameeq \quotep{P} \\
      \dropn{x} & & otherwise \\
    \end{array}
  \right.
\end{mathpar}
 

where

\begin{eqnarray}
  (x)\id{\{} \lpquote Q \rpquote / \lpquote P \rpquote \id{\}}            = 
  \left\{ 
    \begin{array}{ccc}
      \lpquote Q \rpquote & & x \nameeq \lpquote P \rpquote \\
      x & & otherwise \\
    \end{array}
  \right. \nonumber
\end{eqnarray}

and $z$ is chosen distinct from $\quotep{P}$, $\quotep{Q}$, the free
names in $Q$, and all the names in $R$. Our $\alpha$-equivalence will
be built in the standard way from this substitution.

\begin{remark}\label{rem:no_self_referential_names}
  One consequence of these definitions is that $\forall P. \quotep{P}
  \not\in \freenames{P}$.
\end{remark}

\subsection{ Dynamic quote: an example }

Anticipating something of what's to come, consider applying the
substitution, $\widehat{\id{\{}u / z \id{\}}}$, to the following pair
of processes, $\lift{w}{y!(z)}$ and $w[ \lpquote y!(z) \rpquote ]$.

\begin{eqnarray}
	\lift{w}{y!(z)}\widehat{\id{\{}u / z \id{\}}}
		& = &
		\lift{w}{y!(u)} \nonumber\\
	w[ \lpquote y!(z) \rpquote ] \widehat{ \id{\{}u / z \id{\}} }
		& = &
		w[ \lpquote y!(z) \rpquote ] \nonumber
\end{eqnarray}

Because the body of the process between quotes is impervious to
substitution, we get radically different answers. In fact, by
examining the first process in an input context,
e.g. $x?(z).\lift{w}{y!(z)}$, we see that the process under the lift
operator may be shaped by prefixed inputs binding a name inside it. In
this sense, the lift operator will be seen as a way to dynamically
construct processes before reifying them as names.

Finally equipped with these standard features we can present the
dynamics of the calculus.

\subsubsection{Operational semantics} 

Finally, we introduce the computational dynamics. What marks these
algebras as distinct from other more traditionally studied algebraic
structures, e.g. vector spaces or polynomial rings, is the manner in
which dynamics is captured. In traditional structures, dynamics is typically
expressed through morphisms between such structures, as in linear maps
between vector spaces or morphisms between rings. In algebras
associated with the semantics of computation, the dynamics is
expressed as part of the algebraic structure itself, through a
reduction reduction relation typically denoted by $\red$. Below, we
give a recursive presentation of this relation for the calculus used
in the encoding.

$\red \subseteq \pi \times \pi$
$\red : \pi \to \mathcal{P}(\pi)$

\begin{mathpar}
  \inferrule* [lab=Comm] { \textsf{match}( x_{src}, x_{trgt} ) } { x_{trgt}?(y)P \; | \; x_{src}!\langle {Q} \rangle \red P\{\quotep{Q}/y}\} }
  \and \\
  \inferrule* [lab=Par] {{P} \red {P}'} {{{P} | {Q}} \red {{P}' | {Q}}}
  \and
  \inferrule* [lab=Equiv]{{{P} \scong {P}'} \andalso {{P}' \red {Q}'} \andalso {{Q}' \scong {Q}}}{{P} \red {Q}}
\end{mathpar}

\begin{eqnarray*}
  match_{\equiv} (\quotep{P},\quotep{Q}) & := & P \equiv Q \\
  match_{\dagger}(\quotep{P},\quotep{Q}) & := & \forall R. P|Q \red^{*} R => R \red^{*} 0 \\
  match_{K}(\quotep{P},\quotep{Q}) & := & K \mbox{ for some context } K
\end{eqnarray*}

$u?(x)P | u!\langle Q \rangle \red P\{\quotep{Q}/x\}$

%We write $\wred$ for $\red^*$, and $P\red$ if $\exists Q $ such that $ P \red Q$.
We write $P\red$ if $\exists Q $ such that $ P \red Q$ and $P\not\red$, otherwise.

\section{Replication}

As mentioned before, it is known that replication (and hence
recursion) can be implemented in a higher-order process algebra
\cite{SangiorgiWalker}. As our first example of calculation with the
machinery thus far presented we give the construction explicitly in
the {\rhoc}.

\begin{eqnarray}
	D_{x} & := & \prefix{x}{y}{(\binpar{\outputp{x}{y}}{@{y}})} \nonumber\\
	\bangp_{x}{P} & := & \binpar{{x}!\langle{\binpar{D_{x}}{P}}\rangle}{D_{x}} \nonumber
\end{eqnarray}

\begin{eqnarray}
	\bangp_{x}{P} & & \nonumber\\
	=
	& {x}!\langle{(\prefix{x}{y}{(\outputp{x}{y} | @{y})) | P}}\rangle 
	      | \prefix{x}{y}{(\outputp{x}{y} | @{y})} & \nonumber\\
	\red
	& (\outputp{x}{y} | @{y})\substn{\quotep{(\prefix{x}{y}{(@{y} | \outputp{x}{y})) | P}}}{y} & \nonumber\\
	=
	& \outputp{x}{\quotep{(\prefix{x}{y}{(\outputp{x}{y} | @{y})) | P}}}
	  | {(\prefix{x}{y}{(\outputp{x}{y} | @{y})) | P}} & \nonumber\\
	\red
	& \ldots & \nonumber\\
	\red^*
	& P | P | \ldots & \nonumber
\end{eqnarray}

Of course, this encoding, as an implementation, runs away, unfolding
$\bangp{P}$ eagerly. A lazier and more implementable replication
operator, restricted to input-guarded processes, may be obtained as follows.

\begin{eqnarray}
\bangp{\prefix{u}{v}{P}} 
	:= 
	\binpar{\lift{x}{\prefix{u}{v}{(\binpar{D(x)}{P})}}}{D(x)} \nonumber
\end{eqnarray}

\begin{remark}
  Note that the lazier definition still does not deal with summation
  or mixed summation (i.e. sums over input and output). The reader is
  invited to construct definitions of replication that deal with these
  features. 

  Further, the definitions are parameterized in a name, $x$. Can you,
  gentle reader, make a definition that eliminates this parameter and
  guarantees no accidental interaction between the replication
  machinery and the process being replicated -- i.e. no accidental
  sharing of names used by the process to get its work done and the
  name(s) used by the replication to effect copying. This latter
  revision of the definition of replication is crucial to obtaining
  the expected identity $!!P \sim !P$.
\end{remark}

\begin{remark}\label{rem:paradoxical_combinator}
  The reader familiar with the lambda calculus will have noticed the
  similarity between $D$ and the paradoxical combinator.

  [Ed. note: the existence of this seems to suggest we have to be more
  restrictive on the set of processes and names we admit if we are to
  support no-cloning.]
\end{remark}

\subsubsection{Bisimulation}

The computational dynamics gives rise to another kind of equivalence,
the equivalence of computational behavior. As previously mentioned
this is typically captured \emph{via} some form of bisimulation.

% The notion we use in this paper is weak barbed bisimulation
% \cite{milner91polyadicpi}.

The notion we use in this paper is derived from weak barbed
bisimulation \cite{milner91polyadicpi}. 

\begin{definition}
An \emph{observation relation}, $\downarrow_{\mathcal N}$, over a set
of names, $\mathcal N$, is the smallest relation satisfying the rules
below.

\infrule[Out-barb]{y \in {\mathcal N}, \; x \nameeq y}
		  {\outputp{x}{v} \downarrow_{\mathcal N} x}
\infrule[Par-barb]{\mbox{$P\downarrow_{\mathcal N} x$ or $Q\downarrow_{\mathcal N} x$}}
		  {\binpar{P}{Q} \downarrow_{\mathcal N} x}

We write $P \Downarrow_{\mathcal N} x$ if there is $Q$ such that 
$P \wred Q$ and $Q \downarrow_{\mathcal N} x$.
\end{definition}

\begin{definition}
%\label{def.bbisim}
An  ${\mathcal N}$-\emph{barbed bisimulation} over a set of names, ${\mathcal N}$, is a symmetric binary relation 
${\mathcal S}_{\mathcal N}$ between agents such that $P\rel{S}_{\mathcal N}Q$ implies:
\begin{enumerate}
\item If $P \red P'$ then $Q \wred Q'$ and $P'\rel{S}_{\mathcal N} Q'$.
\item If $P\downarrow_{\mathcal N} x$, then $Q\Downarrow_{\mathcal N} x$.
\end{enumerate}
$P$ is ${\mathcal N}$-barbed bisimilar to $Q$, written
$P \wbbisim_{\mathcal N} Q$, if $P \rel{S}_{\mathcal N} Q$ for some ${\mathcal N}$-barbed bisimulation ${\mathcal S}_{\mathcal N}$.
\end{definition}

$\mathcal{R} \subseteq \pi \times \pi$

$P \mathcal{R} Q => \forall P'. P \red P' \Rightarrow \exists Q'. Q \red Q', P' \mathcal{R} Q'$

$P \vdash x \Rightarrow Q \vdash x$

\begin{mathpar}
  \inferrule*[lab=Out-barb]{x \nameeq y}{{y}!\langle{Q}\rangle \vdash x}
  \and
  \inferrule*[lab=Par-barb]{\mbox{$P\vdash x$ or $Q\vdash x$}}{\binpar{P}{Q} \vdash x}
\end{mathpar}

\subsubsection{Contexts}

One of the principle advantages of computational calculi like the
$\pi$-calculus is a well-defined notion of context,
contextual-equivalence and a correlation between
contextual-equivalence and notions of bisimulation. The notion of
context allows the decomposition of a process into (sub-)process and
its syntactic environment, its context. Thus, a context may be
thought of as a process with a ``hole'' (written $\Box$) in it. The
application of a context $M$ to a process $P$, written $M[P]$, is
tantamount to filling the hole in $M$ with $P$. In this paper we do
not need the full weight of this theory, but do make use of the notion
of context in the proof the main theorem. 

\begin{mathpar}
  \inferrule* [lab=summation] {} {{M_{M},M_{N}} \bc \Box \;|\; x.M_{A} \;|\; M_{M}+M_{N}}
  \and
  \inferrule* [lab=agent] {} {{M_{A}} \bc (\vec{x})M_{P} \;| \; \clift{P_0,\ldots,M_{P},\ldots,P_N}}
  \and \\
  \inferrule* [lab=process] {} {{M_{P}} \bc M_{N} \;| \;P|M_{P} }
\end{mathpar} 

\begin{mathpar}
  \inferrule* [lab=sychronization] {} {M_{N} \bc \Box \;|\; x?M_{F} \;|\; x!M_{C}}
  \and
  \inferrule* [lab=abstraction] {} {{M_{F}} \bc (x)M_{P} }
  \and
  \inferrule* [lab=concretion] {} {{M_{C}} \bc \langle M_{P} \rangle }
  \and \\
  \inferrule* [lab=process] {} {{M_{P}} \bc M_{N} \;| \;P|M_{P} }
\end{mathpar}

\begin{definition}[contextual application] Given a context $M$, and
  process $P$, we define the \emph{contextual application}, $M[P] :=
  M\{P/\Box\}$. That is, the contextual application of M to P is the
  substitution of $P$ for $\Box$ in $M$.
\end{definition}

$\meaningof{-} : L \to \mathcal{P}(\pi)$

\begin{mathpar}
  \inferrule* [lab=collection] {} {\meaningof{true} = \pi, \and \meaningof{~E} = \pi \setminus \meaningof{E}, \and \meaningof{E_{1} \& E_{2}} = \meaningof{E_{1}} \cap \meaningof{E_{2}}}
\end{mathpar}

\begin{mathpar}
  \inferrule* [lab=structure] {} {\meaningof{0} = \{ P \in \pi | P \equiv 0 \}, \and \\ \meaningof{E_1 | E_2} = \{ P \in \pi | P \equiv P_{1} | P_{2}, P_{1} \in \meaningof{E_{1}}, P_{2} \in \meaningof{E_2}\} }
\end{mathpar}

\begin{mathpar}
 \inferrule* [lab=behavior] {} {\meaningof{\langle a?b \rangle E} = \{ P \in \pi | P \equiv Q | u?(y)P', \\ \and \\\\ \and \\ \;\;\; u \in \meaningof{a}, \forall z.P'\{z/y\} \in \meaningof{E\{z/b\}}\}, \and \\ \meaningof{a!E} = \{ P \in \pi | P \equiv Q | x!\langle P' \rangle, x \in \meaningof{a} P' \in \meaningof{E}\} }
\end{mathpar}

\begin{mathpar}
 \inferrule* [lab=nominal] {} {\meaningof{\quotep{E}} = \{ \quotep{P} \in \quotep{\pi} | P \in \meaningof{E} \}, \and \meaningof{\quotep{P}} = \{ \quotep{Q} \in \quotep{\pi} | P \equiv Q \} \and \\ \meaningof{@\quotep{E}} = \{ P \in \pi | P \equiv @x, x \in \meaningof{E} \}}
\end{mathpar}

\begin{eqnarray*}
  \\
  \meaningof{-} : TS \to ST
\end{eqnarray*}

\begin{eqnarray*}
  \\
  L : TS \to ST
\end{eqnarray*}

\begin{eqnarray*}
  \\
  P \models E \iff P \in \meaningof{E}
\end{eqnarray*}

\begin{eqnarray*}
  P \approx_{L} Q \iff \forall E \in L. P \models E \iff Q \models E
\end{eqnarray*}

\begin{eqnarray*}
  P \approx_{K} Q
\end{eqnarray*}

\begin{eqnarray*}
  P \approx Q
\end{eqnarray*}

$\approx_{K} = \approx = \approx_{L}$

\subsubsection{Contextual duality}

Note that contexts extend the quotation operation to a family of
operations from processes to names. Given a context, $M$, we can
define a \emph{nominal context}, $\quotep{M}$ by $\quotep{M}[P] :=
\quotep{M[P]}$. To foreshadow what is to come we observe that these
operations enjoy a duality with processes very much like the duality
between vectors and maps from vectors to scalars.

Further, because the calculus is essentially higher-order, we have a
correspondence between contexts and processes. More specifically,
given a name $x$ and a context $M$ we can construct $M^{*}_{x}$ such
that 

\begin{mathpar}
  M^{*}_{x} | \lift{x}{P} \red M[P]
\end{mathpar}

namely,

\begin{mathpar}
  M^{*}_{x} := x?(u).M[\dropn{u}]
\end{mathpar}

The dependence of $M^{*}_{x}$ on a name makes it an abstraction, 

\begin{mathpar}
  M^{*} := (x)x?(u).M[\dropn{u}]
\end{mathpar}

\subsection{Additional notation}

It will sometimes be convenient to denote the process a name
quotes. We already have the notation $x = \quotep{P}$, but it will be
convenient to introduce an alternate notation, $\procn{x}$, when we
want to emphasize the connection to the use of the name. Note that, by
virtue of name equivalence, $\quotep{\procn{x}} \nameeq x$; so, the
notation is consistent with previous definitions.

Further, because names have structure it is possible to effect
substitutions on the basis of that structure. This means we need to
upgrade our notation for substitutions, which we accomplish by
adapting comprehension notation. Thus,

\begin{mathpar}
  P\{ y / x : x \in S \}
\end{mathpar}

is interpreted to mean the process derived from P by replacing (in a
capture-avoiding manner) each occurrence of $x$ in $S$ by $y$. For example,

\begin{mathpar}
  P\{ \quotep{\procn{x}|\procn{x}} / x : x \in \freenames{P} \}
\end{mathpar}

will replace each (occurrence) of a free name $x$ in $P$ by
$\quotep{\procn{x}|\procn{x}}$.

Also, we will avail ourselves of the notation $x^{L}$ and $x^{R}$ to
denote injections of a name into disjoint copies of the name
space. There are numerous ways to accomplish this. One example can be
found in \cite{MeredithR05}. This notation overloads to vectors of
names: $\vec{x}^{\pi} := (x_{i}^{\pi} \; : \; 0 \leq i < |\vec{x}| )$ where $\pi \in \{L,R\}$.

We also use $P^{\Box} := P|\Box$.

In \cite{MeredithR05} an interpretation of the new operator is
given. It turns out that there are several possible interpretations
all enjoying the requisite algebraic properties of the operator (see
\cite{milner91polyadicpi}). We will therefore make liberal use of
$(\nu\; \vec{x})P$.

% subsection the_syntax_and_semantics_of_the_notation_system (end)   

\input{qm2pi.qmops} 

\input{qm2pi.sterngerlach} 

\input{qm2pi.metric} 

% section concurrent_process_calculi (end)

%\input{qm2pi.proofsketch}

% section proof sketch (end)

%\input{qm2pi.slviaknots} 

% section spatial logic via knots (end)

\input{qm2pi.conclusion}

% section conclusion (end)

%\input{qm2pi.dtcodes} 

% section wiring algorithm (end)

\input{qm2pi.ack} 

% section acknowledgments (end)

\newpage


\bibliographystyle{plain}   
\bibliography{../../biblios/main.bib}

\input{qm2pi.rhodetails}

\end{document}



\end{document}

 

%\ifpdf
%\usepackage[pdftex]{graphicx}
%\else
%\usepackage{graphicx}
%\fi

 % \ifpdf
%  \usepackage{pdfsync}
%  \if


%\title{Brief Article}
%\author{David F. Snyder}
%\author{L.G. Meredith}

%\address{Dept. of Math., Texas State University--San Marcos, San Marcos, TX 78666}
       
\pagestyle{empty}


\begin{document}

\lstset{language=[Objective]Caml,frame=shadowbox}

\documentclass[12pt]{llncs}
%\documentclass{jktr}

\usepackage[pdftex]{hyperref}                   
\usepackage {listings}
\usepackage {mathpartir}
\usepackage{bcprules}
%\usepackage{listings}
                       
\usepackage{graphicx} 
%\usepackage[margins=2.5cm,nohead,nofoot]{geometry}
%\usepackage{geometry}
\usepackage{amsfonts}
\usepackage{amstext}
\usepackage{latexsym}
\usepackage{amssymb}
\usepackage{color}


%\include{myPreamble}
\documentclass[12pt]{llncs}
%\documentclass{jktr}

\usepackage[pdftex]{hyperref}                   
\usepackage {listings}
\usepackage {mathpartir}
\usepackage{bcprules}
%\usepackage{listings}
                       
\usepackage{graphicx} 
%\usepackage[margins=2.5cm,nohead,nofoot]{geometry}
%\usepackage{geometry}
\usepackage{amsfonts}
\usepackage{amstext}
\usepackage{latexsym}
\usepackage{amssymb}
\usepackage{color}


%\include{myPreamble}
\include{qm2pi.local} 

%\ifpdf
%\usepackage[pdftex]{graphicx}
%\else
%\usepackage{graphicx}
%\fi

 % \ifpdf
%  \usepackage{pdfsync}
%  \if


%\title{Brief Article}
%\author{David F. Snyder}
%\author{L.G. Meredith}

%\address{Dept. of Math., Texas State University--San Marcos, San Marcos, TX 78666}
       
\pagestyle{empty}


\begin{document}

\lstset{language=[Objective]Caml,frame=shadowbox}

\input{qm2pi.front}

% section front matter (end)

\input{qm2pi.intro} 
 
% section introduction (end)

% \input{qm2pi.knotations} 

% section notation (end)

\input{qm2pi.process.calculi} 

% section concurrent_process_calculi_and_spatial_logics_ (end)
    
%\input{qm2pi.knots2pi} 

%\input{qm2pi.trefoil} 

%\input{qm2pi.mainthm} 

% subsection basic_interpretation (end)

%\input{qm2pi.rho.presentation} 
\subsection{The syntax and semantics of the notation system}\label{sub:the_syntax_and_semantics_of_the_notation_system} % (fold)

We now summarize a technical presentation of the calculus that
embodies our theory of dynamics. The typical presentation of such a
calculus follows the style of giving generators and relations on
them. The grammar, below, describing term constructors, freely
generates the set of processes, $\Proc$. This set is then quotiented
by a relation known as structural congruence and it is over this set
that the notion of dynamics is expressed. This presentation is
essentially that of \cite{MeredithR05} with the addition of
polyadicity and summation. For readability we have relegated some of
the technical subtleties to an appendix.

\subsubsection{Process grammar}\label{subsub:process_grammar}

\begin{mathpar}
  \inferrule* [lab=synchronization] {} {{M} \bc \pzero \;|\; x?F \;|\; x!C }
  \and
  \inferrule* [lab=abstraction] {} {{F} \bc (x)P}
  \and
  \inferrule* [lab=concretion] {} {{C} \bc \langle Q \rangle}
  \and
  \inferrule* [lab=process] {} {{P,Q} \bc M \;| \;P|Q \;|\; @{x}}
  \and
  \inferrule* [lab=name] {} {{x} \bc \quotep{P}}
\end{mathpar} 

Note that $\vec{x}$ (resp. $\vec{P}$) denotes a vector of names
(resp. processes) of length $|\vec{x}|$ (resp. $|\vec{P}|$). We adopt
the following useful abbreviations.

\begin{mathpar}
   x?(\vec{y}).P := x.(\vec{y})P \and  x\clift{\vec{P}} := x.\clift{\vec{P}}
   \and x!(y) := \lift{x}{\dropn{y}}
   \and \Pi_{i=0}^{n-1}P_i := P_0 | \ldots | P_{n-1}
\end{mathpar}

\subsubsection{Structural congruence}

\paragraph{Free and bound names and alpha-equivalence.} At the
core of structural equivalence is alpha-equivalence which identifies
process that are the same up to a change of variable. Formally, we
recognize the distinction between free and bound names. The free names
of a process, $\freenames{P}$, may be calculated recursively as
follows:

\begin{mathpar}
\freenames{\pzero} := \emptyset
  \and \\
  \freenames{x?(y).P} := \{ x \} \cup (\freenames{P} \setminus \{ y \})
  \and 
  \freenames{x!\langle P \rangle} := \{ x \} \cup \{ P \} 
  \and \\
  \freenames{P|Q} := \freenames{P} \cup \freenames{Q}
  \and \\
  \freenames{@{x}} := \{ x \}
\end{mathpar}

$\pi$
$\quotep{\pi}$

$\freenames{-} : \pi \to \mathcal{P}(\quotep{\pi})$

\begin{eqnarray*}
  \freenames{\pzero} & := & \emptyset \\
  \freenames{x?(y).P} & := & \{ x \} \cup (\freenames{P} \setminus \{ y \}) \\
  \freenames{x!\langle P \rangle} & := & \{ x \} \cup \{ P \} \\
  \freenames{P|Q} & := & \freenames{P} \cup \freenames{Q} \\
  \freenames{\dropn{x}} & := & \{ x \}
\end{eqnarray*}

The bound names of a process, $\boundnames{P}$, are those names occurring in $P$
that are not free. For example, in $x?(y).0$, the name $x$ is free, while $y$ is bound.

\begin{mathpar}
  \inferrule* [lab=monoidal-laws] {} { P|Q \equiv Q|P \and P|0 \equiv P \and P|(Q|R) \equiv (P|Q)|R }
\end{mathpar}

\begin{mathpar}
  \inferrule* [lab=alpha-equivalence] {} { (x)P \equiv (y)P\{y/x\} \and y \not\in \freenames{P} }
\end{mathpar}

\begin{definition}
Then two processes, $P,Q$, are alpha-equivalent if $P = Q\{\vec{y}/\vec{x}\}$ for
some $\vec{x} \in \boundnames{Q},\vec{y} \in \boundnames{P}$, where $Q\{\vec{y}/\vec{x}\}$
denotes the capture-avoiding substitution of $\vec{y}$ for $\vec{x}$ in $Q$.
\end{definition}

\begin{definition}
  The {\em structural congruence} \cite{SangiorgiWalker} , $\equiv$,
  between processes is the least congruence containing
  alpha-equivalence, satisfying the abelian monoid laws
  (associativity, commutativity and $\pzero$ as identity) for parallel
  composition $|$ and for summation $+$.
\end{definition}

\subsection{Name equivalence}

We take name equivalence, written $\nameeq$, to be the smallest
equivalence relation generated by the following rules.

\begin{mathpar}
\inferrule*[lab=Quote-drop]
{ }
{ \quotep{@{x}} \nameeq x }

\inferrule*[lab=Struct-equiv]
{ P \scong Q }
{ \quotep{P} \nameeq \quotep{Q} }
\end{mathpar}

The astute reader will have noticed that the mutual recursion of names
and processes imposes a mutual recursion on alpha-equivalence and
structural equivalence via name-equivalence. Fortunately, all of this
works out pleasantly and we may calculate in the natural way, free of
concern. The reader interested in the details is referred to the
appendix \ref{appendix:rho_details}.

\subsection{Substitution}

We use $\Proc$ for the set of processes, $\QProc$ for the set of
names, and $\id{\{}\vec{y} / \vec{x} \id{\}}$ to denote partial maps,
$s : \QProc \rightarrow \QProc$. A map, $s$ lifts, uniquely, to a map
on process terms, $\widehat{s} : \Proc \rightarrow \Proc$ by the
following equations.

\begin{mathpar}
  (0) \psubstp{Q}{P} := 0 \\
  (R \juxtap S) \psubstp{Q}{P}
  :=    
  (R)\psubstp{Q}{P} \juxtap (S) \psubstp{Q}{P} \\
  (x?(y).R) \psubstp{Q}{P}    
  :=    
  (x)\substp{Q}{P} (z)\concat( (R \psubstn{z}{y}) \psubstp{Q}{P} ) \\
  (\lift{x}{R}) \psubstp{Q}{P}  
  :=
  \lift{(x)\substp{Q}{P}}{ R \psubstp{Q}{P} } \\
%   (\dropn{x})  \psubstp{Q}{P}       
%   := 
%   \left\{ 
%     \begin{array}{ccc} 
%       \dropn{\quotep{Q}} & & x \nameeq \quotep{P} \\
%       \dropn{x} & & otherwise \\
%     \end{array}
%   \right. 
  (\dropn{x})  \psubstp{Q}{P}       
  := 
  \left\{ 
    \begin{array}{ccc} 
      Q & & x \nameeq \quotep{P} \\
      \dropn{x} & & otherwise \\
    \end{array}
  \right.
\end{mathpar}
 

where

\begin{eqnarray}
  (x)\id{\{} \lpquote Q \rpquote / \lpquote P \rpquote \id{\}}            = 
  \left\{ 
    \begin{array}{ccc}
      \lpquote Q \rpquote & & x \nameeq \lpquote P \rpquote \\
      x & & otherwise \\
    \end{array}
  \right. \nonumber
\end{eqnarray}

and $z$ is chosen distinct from $\quotep{P}$, $\quotep{Q}$, the free
names in $Q$, and all the names in $R$. Our $\alpha$-equivalence will
be built in the standard way from this substitution.

\begin{remark}\label{rem:no_self_referential_names}
  One consequence of these definitions is that $\forall P. \quotep{P}
  \not\in \freenames{P}$.
\end{remark}

\subsection{ Dynamic quote: an example }

Anticipating something of what's to come, consider applying the
substitution, $\widehat{\id{\{}u / z \id{\}}}$, to the following pair
of processes, $\lift{w}{y!(z)}$ and $w[ \lpquote y!(z) \rpquote ]$.

\begin{eqnarray}
	\lift{w}{y!(z)}\widehat{\id{\{}u / z \id{\}}}
		& = &
		\lift{w}{y!(u)} \nonumber\\
	w[ \lpquote y!(z) \rpquote ] \widehat{ \id{\{}u / z \id{\}} }
		& = &
		w[ \lpquote y!(z) \rpquote ] \nonumber
\end{eqnarray}

Because the body of the process between quotes is impervious to
substitution, we get radically different answers. In fact, by
examining the first process in an input context,
e.g. $x?(z).\lift{w}{y!(z)}$, we see that the process under the lift
operator may be shaped by prefixed inputs binding a name inside it. In
this sense, the lift operator will be seen as a way to dynamically
construct processes before reifying them as names.

Finally equipped with these standard features we can present the
dynamics of the calculus.

\subsubsection{Operational semantics} 

Finally, we introduce the computational dynamics. What marks these
algebras as distinct from other more traditionally studied algebraic
structures, e.g. vector spaces or polynomial rings, is the manner in
which dynamics is captured. In traditional structures, dynamics is typically
expressed through morphisms between such structures, as in linear maps
between vector spaces or morphisms between rings. In algebras
associated with the semantics of computation, the dynamics is
expressed as part of the algebraic structure itself, through a
reduction reduction relation typically denoted by $\red$. Below, we
give a recursive presentation of this relation for the calculus used
in the encoding.

$\red \subseteq \pi \times \pi$
$\red : \pi \to \mathcal{P}(\pi)$

\begin{mathpar}
  \inferrule* [lab=Comm] { \textsf{match}( x_{src}, x_{trgt} ) } { x_{trgt}?(y)P \; | \; x_{src}!\langle {Q} \rangle \red P\{\quotep{Q}/y}\} }
  \and \\
  \inferrule* [lab=Par] {{P} \red {P}'} {{{P} | {Q}} \red {{P}' | {Q}}}
  \and
  \inferrule* [lab=Equiv]{{{P} \scong {P}'} \andalso {{P}' \red {Q}'} \andalso {{Q}' \scong {Q}}}{{P} \red {Q}}
\end{mathpar}

\begin{eqnarray*}
  match_{\equiv} (\quotep{P},\quotep{Q}) & := & P \equiv Q \\
  match_{\dagger}(\quotep{P},\quotep{Q}) & := & \forall R. P|Q \red^{*} R => R \red^{*} 0 \\
  match_{K}(\quotep{P},\quotep{Q}) & := & K \mbox{ for some context } K
\end{eqnarray*}

$u?(x)P | u!\langle Q \rangle \red P\{\quotep{Q}/x\}$

%We write $\wred$ for $\red^*$, and $P\red$ if $\exists Q $ such that $ P \red Q$.
We write $P\red$ if $\exists Q $ such that $ P \red Q$ and $P\not\red$, otherwise.

\section{Replication}

As mentioned before, it is known that replication (and hence
recursion) can be implemented in a higher-order process algebra
\cite{SangiorgiWalker}. As our first example of calculation with the
machinery thus far presented we give the construction explicitly in
the {\rhoc}.

\begin{eqnarray}
	D_{x} & := & \prefix{x}{y}{(\binpar{\outputp{x}{y}}{@{y}})} \nonumber\\
	\bangp_{x}{P} & := & \binpar{{x}!\langle{\binpar{D_{x}}{P}}\rangle}{D_{x}} \nonumber
\end{eqnarray}

\begin{eqnarray}
	\bangp_{x}{P} & & \nonumber\\
	=
	& {x}!\langle{(\prefix{x}{y}{(\outputp{x}{y} | @{y})) | P}}\rangle 
	      | \prefix{x}{y}{(\outputp{x}{y} | @{y})} & \nonumber\\
	\red
	& (\outputp{x}{y} | @{y})\substn{\quotep{(\prefix{x}{y}{(@{y} | \outputp{x}{y})) | P}}}{y} & \nonumber\\
	=
	& \outputp{x}{\quotep{(\prefix{x}{y}{(\outputp{x}{y} | @{y})) | P}}}
	  | {(\prefix{x}{y}{(\outputp{x}{y} | @{y})) | P}} & \nonumber\\
	\red
	& \ldots & \nonumber\\
	\red^*
	& P | P | \ldots & \nonumber
\end{eqnarray}

Of course, this encoding, as an implementation, runs away, unfolding
$\bangp{P}$ eagerly. A lazier and more implementable replication
operator, restricted to input-guarded processes, may be obtained as follows.

\begin{eqnarray}
\bangp{\prefix{u}{v}{P}} 
	:= 
	\binpar{\lift{x}{\prefix{u}{v}{(\binpar{D(x)}{P})}}}{D(x)} \nonumber
\end{eqnarray}

\begin{remark}
  Note that the lazier definition still does not deal with summation
  or mixed summation (i.e. sums over input and output). The reader is
  invited to construct definitions of replication that deal with these
  features. 

  Further, the definitions are parameterized in a name, $x$. Can you,
  gentle reader, make a definition that eliminates this parameter and
  guarantees no accidental interaction between the replication
  machinery and the process being replicated -- i.e. no accidental
  sharing of names used by the process to get its work done and the
  name(s) used by the replication to effect copying. This latter
  revision of the definition of replication is crucial to obtaining
  the expected identity $!!P \sim !P$.
\end{remark}

\begin{remark}\label{rem:paradoxical_combinator}
  The reader familiar with the lambda calculus will have noticed the
  similarity between $D$ and the paradoxical combinator.

  [Ed. note: the existence of this seems to suggest we have to be more
  restrictive on the set of processes and names we admit if we are to
  support no-cloning.]
\end{remark}

\subsubsection{Bisimulation}

The computational dynamics gives rise to another kind of equivalence,
the equivalence of computational behavior. As previously mentioned
this is typically captured \emph{via} some form of bisimulation.

% The notion we use in this paper is weak barbed bisimulation
% \cite{milner91polyadicpi}.

The notion we use in this paper is derived from weak barbed
bisimulation \cite{milner91polyadicpi}. 

\begin{definition}
An \emph{observation relation}, $\downarrow_{\mathcal N}$, over a set
of names, $\mathcal N$, is the smallest relation satisfying the rules
below.

\infrule[Out-barb]{y \in {\mathcal N}, \; x \nameeq y}
		  {\outputp{x}{v} \downarrow_{\mathcal N} x}
\infrule[Par-barb]{\mbox{$P\downarrow_{\mathcal N} x$ or $Q\downarrow_{\mathcal N} x$}}
		  {\binpar{P}{Q} \downarrow_{\mathcal N} x}

We write $P \Downarrow_{\mathcal N} x$ if there is $Q$ such that 
$P \wred Q$ and $Q \downarrow_{\mathcal N} x$.
\end{definition}

\begin{definition}
%\label{def.bbisim}
An  ${\mathcal N}$-\emph{barbed bisimulation} over a set of names, ${\mathcal N}$, is a symmetric binary relation 
${\mathcal S}_{\mathcal N}$ between agents such that $P\rel{S}_{\mathcal N}Q$ implies:
\begin{enumerate}
\item If $P \red P'$ then $Q \wred Q'$ and $P'\rel{S}_{\mathcal N} Q'$.
\item If $P\downarrow_{\mathcal N} x$, then $Q\Downarrow_{\mathcal N} x$.
\end{enumerate}
$P$ is ${\mathcal N}$-barbed bisimilar to $Q$, written
$P \wbbisim_{\mathcal N} Q$, if $P \rel{S}_{\mathcal N} Q$ for some ${\mathcal N}$-barbed bisimulation ${\mathcal S}_{\mathcal N}$.
\end{definition}

$\mathcal{R} \subseteq \pi \times \pi$

$P \mathcal{R} Q => \forall P'. P \red P' \Rightarrow \exists Q'. Q \red Q', P' \mathcal{R} Q'$

$P \vdash x \Rightarrow Q \vdash x$

\begin{mathpar}
  \inferrule*[lab=Out-barb]{x \nameeq y}{{y}!\langle{Q}\rangle \vdash x}
  \and
  \inferrule*[lab=Par-barb]{\mbox{$P\vdash x$ or $Q\vdash x$}}{\binpar{P}{Q} \vdash x}
\end{mathpar}

\subsubsection{Contexts}

One of the principle advantages of computational calculi like the
$\pi$-calculus is a well-defined notion of context,
contextual-equivalence and a correlation between
contextual-equivalence and notions of bisimulation. The notion of
context allows the decomposition of a process into (sub-)process and
its syntactic environment, its context. Thus, a context may be
thought of as a process with a ``hole'' (written $\Box$) in it. The
application of a context $M$ to a process $P$, written $M[P]$, is
tantamount to filling the hole in $M$ with $P$. In this paper we do
not need the full weight of this theory, but do make use of the notion
of context in the proof the main theorem. 

\begin{mathpar}
  \inferrule* [lab=summation] {} {{M_{M},M_{N}} \bc \Box \;|\; x.M_{A} \;|\; M_{M}+M_{N}}
  \and
  \inferrule* [lab=agent] {} {{M_{A}} \bc (\vec{x})M_{P} \;| \; \clift{P_0,\ldots,M_{P},\ldots,P_N}}
  \and \\
  \inferrule* [lab=process] {} {{M_{P}} \bc M_{N} \;| \;P|M_{P} }
\end{mathpar} 

\begin{mathpar}
  \inferrule* [lab=sychronization] {} {M_{N} \bc \Box \;|\; x?M_{F} \;|\; x!M_{C}}
  \and
  \inferrule* [lab=abstraction] {} {{M_{F}} \bc (x)M_{P} }
  \and
  \inferrule* [lab=concretion] {} {{M_{C}} \bc \langle M_{P} \rangle }
  \and \\
  \inferrule* [lab=process] {} {{M_{P}} \bc M_{N} \;| \;P|M_{P} }
\end{mathpar}

\begin{definition}[contextual application] Given a context $M$, and
  process $P$, we define the \emph{contextual application}, $M[P] :=
  M\{P/\Box\}$. That is, the contextual application of M to P is the
  substitution of $P$ for $\Box$ in $M$.
\end{definition}

$\meaningof{-} : L \to \mathcal{P}(\pi)$

\begin{mathpar}
  \inferrule* [lab=collection] {} {\meaningof{true} = \pi, \and \meaningof{~E} = \pi \setminus \meaningof{E}, \and \meaningof{E_{1} \& E_{2}} = \meaningof{E_{1}} \cap \meaningof{E_{2}}}
\end{mathpar}

\begin{mathpar}
  \inferrule* [lab=structure] {} {\meaningof{0} = \{ P \in \pi | P \equiv 0 \}, \and \\ \meaningof{E_1 | E_2} = \{ P \in \pi | P \equiv P_{1} | P_{2}, P_{1} \in \meaningof{E_{1}}, P_{2} \in \meaningof{E_2}\} }
\end{mathpar}

\begin{mathpar}
 \inferrule* [lab=behavior] {} {\meaningof{\langle a?b \rangle E} = \{ P \in \pi | P \equiv Q | u?(y)P', \\ \and \\\\ \and \\ \;\;\; u \in \meaningof{a}, \forall z.P'\{z/y\} \in \meaningof{E\{z/b\}}\}, \and \\ \meaningof{a!E} = \{ P \in \pi | P \equiv Q | x!\langle P' \rangle, x \in \meaningof{a} P' \in \meaningof{E}\} }
\end{mathpar}

\begin{mathpar}
 \inferrule* [lab=nominal] {} {\meaningof{\quotep{E}} = \{ \quotep{P} \in \quotep{\pi} | P \in \meaningof{E} \}, \and \meaningof{\quotep{P}} = \{ \quotep{Q} \in \quotep{\pi} | P \equiv Q \} \and \\ \meaningof{@\quotep{E}} = \{ P \in \pi | P \equiv @x, x \in \meaningof{E} \}}
\end{mathpar}

\begin{eqnarray*}
  \\
  \meaningof{-} : TS \to ST
\end{eqnarray*}

\begin{eqnarray*}
  \\
  L : TS \to ST
\end{eqnarray*}

\begin{eqnarray*}
  \\
  P \models E \iff P \in \meaningof{E}
\end{eqnarray*}

\begin{eqnarray*}
  P \approx_{L} Q \iff \forall E \in L. P \models E \iff Q \models E
\end{eqnarray*}

\begin{eqnarray*}
  P \approx_{K} Q
\end{eqnarray*}

\begin{eqnarray*}
  P \approx Q
\end{eqnarray*}

$\approx_{K} = \approx = \approx_{L}$

\subsubsection{Contextual duality}

Note that contexts extend the quotation operation to a family of
operations from processes to names. Given a context, $M$, we can
define a \emph{nominal context}, $\quotep{M}$ by $\quotep{M}[P] :=
\quotep{M[P]}$. To foreshadow what is to come we observe that these
operations enjoy a duality with processes very much like the duality
between vectors and maps from vectors to scalars.

Further, because the calculus is essentially higher-order, we have a
correspondence between contexts and processes. More specifically,
given a name $x$ and a context $M$ we can construct $M^{*}_{x}$ such
that 

\begin{mathpar}
  M^{*}_{x} | \lift{x}{P} \red M[P]
\end{mathpar}

namely,

\begin{mathpar}
  M^{*}_{x} := x?(u).M[\dropn{u}]
\end{mathpar}

The dependence of $M^{*}_{x}$ on a name makes it an abstraction, 

\begin{mathpar}
  M^{*} := (x)x?(u).M[\dropn{u}]
\end{mathpar}

\subsection{Additional notation}

It will sometimes be convenient to denote the process a name
quotes. We already have the notation $x = \quotep{P}$, but it will be
convenient to introduce an alternate notation, $\procn{x}$, when we
want to emphasize the connection to the use of the name. Note that, by
virtue of name equivalence, $\quotep{\procn{x}} \nameeq x$; so, the
notation is consistent with previous definitions.

Further, because names have structure it is possible to effect
substitutions on the basis of that structure. This means we need to
upgrade our notation for substitutions, which we accomplish by
adapting comprehension notation. Thus,

\begin{mathpar}
  P\{ y / x : x \in S \}
\end{mathpar}

is interpreted to mean the process derived from P by replacing (in a
capture-avoiding manner) each occurrence of $x$ in $S$ by $y$. For example,

\begin{mathpar}
  P\{ \quotep{\procn{x}|\procn{x}} / x : x \in \freenames{P} \}
\end{mathpar}

will replace each (occurrence) of a free name $x$ in $P$ by
$\quotep{\procn{x}|\procn{x}}$.

Also, we will avail ourselves of the notation $x^{L}$ and $x^{R}$ to
denote injections of a name into disjoint copies of the name
space. There are numerous ways to accomplish this. One example can be
found in \cite{MeredithR05}. This notation overloads to vectors of
names: $\vec{x}^{\pi} := (x_{i}^{\pi} \; : \; 0 \leq i < |\vec{x}| )$ where $\pi \in \{L,R\}$.

We also use $P^{\Box} := P|\Box$.

In \cite{MeredithR05} an interpretation of the new operator is
given. It turns out that there are several possible interpretations
all enjoying the requisite algebraic properties of the operator (see
\cite{milner91polyadicpi}). We will therefore make liberal use of
$(\nu\; \vec{x})P$.

% subsection the_syntax_and_semantics_of_the_notation_system (end)   

\input{qm2pi.qmops} 

\input{qm2pi.sterngerlach} 

\input{qm2pi.metric} 

% section concurrent_process_calculi (end)

%\input{qm2pi.proofsketch}

% section proof sketch (end)

%\input{qm2pi.slviaknots} 

% section spatial logic via knots (end)

\input{qm2pi.conclusion}

% section conclusion (end)

%\input{qm2pi.dtcodes} 

% section wiring algorithm (end)

\input{qm2pi.ack} 

% section acknowledgments (end)

\newpage


\bibliographystyle{plain}   
\bibliography{../../biblios/main.bib}

\input{qm2pi.rhodetails}

\end{document}

 

%\ifpdf
%\usepackage[pdftex]{graphicx}
%\else
%\usepackage{graphicx}
%\fi

 % \ifpdf
%  \usepackage{pdfsync}
%  \if


%\title{Brief Article}
%\author{David F. Snyder}
%\author{L.G. Meredith}

%\address{Dept. of Math., Texas State University--San Marcos, San Marcos, TX 78666}
       
\pagestyle{empty}


\begin{document}

\lstset{language=[Objective]Caml,frame=shadowbox}

\documentclass[12pt]{llncs}
%\documentclass{jktr}

\usepackage[pdftex]{hyperref}                   
\usepackage {listings}
\usepackage {mathpartir}
\usepackage{bcprules}
%\usepackage{listings}
                       
\usepackage{graphicx} 
%\usepackage[margins=2.5cm,nohead,nofoot]{geometry}
%\usepackage{geometry}
\usepackage{amsfonts}
\usepackage{amstext}
\usepackage{latexsym}
\usepackage{amssymb}
\usepackage{color}


%\include{myPreamble}
\include{qm2pi.local} 

%\ifpdf
%\usepackage[pdftex]{graphicx}
%\else
%\usepackage{graphicx}
%\fi

 % \ifpdf
%  \usepackage{pdfsync}
%  \if


%\title{Brief Article}
%\author{David F. Snyder}
%\author{L.G. Meredith}

%\address{Dept. of Math., Texas State University--San Marcos, San Marcos, TX 78666}
       
\pagestyle{empty}


\begin{document}

\lstset{language=[Objective]Caml,frame=shadowbox}

\input{qm2pi.front}

% section front matter (end)

\input{qm2pi.intro} 
 
% section introduction (end)

% \input{qm2pi.knotations} 

% section notation (end)

\input{qm2pi.process.calculi} 

% section concurrent_process_calculi_and_spatial_logics_ (end)
    
%\input{qm2pi.knots2pi} 

%\input{qm2pi.trefoil} 

%\input{qm2pi.mainthm} 

% subsection basic_interpretation (end)

%\input{qm2pi.rho.presentation} 
\subsection{The syntax and semantics of the notation system}\label{sub:the_syntax_and_semantics_of_the_notation_system} % (fold)

We now summarize a technical presentation of the calculus that
embodies our theory of dynamics. The typical presentation of such a
calculus follows the style of giving generators and relations on
them. The grammar, below, describing term constructors, freely
generates the set of processes, $\Proc$. This set is then quotiented
by a relation known as structural congruence and it is over this set
that the notion of dynamics is expressed. This presentation is
essentially that of \cite{MeredithR05} with the addition of
polyadicity and summation. For readability we have relegated some of
the technical subtleties to an appendix.

\subsubsection{Process grammar}\label{subsub:process_grammar}

\begin{mathpar}
  \inferrule* [lab=synchronization] {} {{M} \bc \pzero \;|\; x?F \;|\; x!C }
  \and
  \inferrule* [lab=abstraction] {} {{F} \bc (x)P}
  \and
  \inferrule* [lab=concretion] {} {{C} \bc \langle Q \rangle}
  \and
  \inferrule* [lab=process] {} {{P,Q} \bc M \;| \;P|Q \;|\; @{x}}
  \and
  \inferrule* [lab=name] {} {{x} \bc \quotep{P}}
\end{mathpar} 

Note that $\vec{x}$ (resp. $\vec{P}$) denotes a vector of names
(resp. processes) of length $|\vec{x}|$ (resp. $|\vec{P}|$). We adopt
the following useful abbreviations.

\begin{mathpar}
   x?(\vec{y}).P := x.(\vec{y})P \and  x\clift{\vec{P}} := x.\clift{\vec{P}}
   \and x!(y) := \lift{x}{\dropn{y}}
   \and \Pi_{i=0}^{n-1}P_i := P_0 | \ldots | P_{n-1}
\end{mathpar}

\subsubsection{Structural congruence}

\paragraph{Free and bound names and alpha-equivalence.} At the
core of structural equivalence is alpha-equivalence which identifies
process that are the same up to a change of variable. Formally, we
recognize the distinction between free and bound names. The free names
of a process, $\freenames{P}$, may be calculated recursively as
follows:

\begin{mathpar}
\freenames{\pzero} := \emptyset
  \and \\
  \freenames{x?(y).P} := \{ x \} \cup (\freenames{P} \setminus \{ y \})
  \and 
  \freenames{x!\langle P \rangle} := \{ x \} \cup \{ P \} 
  \and \\
  \freenames{P|Q} := \freenames{P} \cup \freenames{Q}
  \and \\
  \freenames{@{x}} := \{ x \}
\end{mathpar}

$\pi$
$\quotep{\pi}$

$\freenames{-} : \pi \to \mathcal{P}(\quotep{\pi})$

\begin{eqnarray*}
  \freenames{\pzero} & := & \emptyset \\
  \freenames{x?(y).P} & := & \{ x \} \cup (\freenames{P} \setminus \{ y \}) \\
  \freenames{x!\langle P \rangle} & := & \{ x \} \cup \{ P \} \\
  \freenames{P|Q} & := & \freenames{P} \cup \freenames{Q} \\
  \freenames{\dropn{x}} & := & \{ x \}
\end{eqnarray*}

The bound names of a process, $\boundnames{P}$, are those names occurring in $P$
that are not free. For example, in $x?(y).0$, the name $x$ is free, while $y$ is bound.

\begin{mathpar}
  \inferrule* [lab=monoidal-laws] {} { P|Q \equiv Q|P \and P|0 \equiv P \and P|(Q|R) \equiv (P|Q)|R }
\end{mathpar}

\begin{mathpar}
  \inferrule* [lab=alpha-equivalence] {} { (x)P \equiv (y)P\{y/x\} \and y \not\in \freenames{P} }
\end{mathpar}

\begin{definition}
Then two processes, $P,Q$, are alpha-equivalent if $P = Q\{\vec{y}/\vec{x}\}$ for
some $\vec{x} \in \boundnames{Q},\vec{y} \in \boundnames{P}$, where $Q\{\vec{y}/\vec{x}\}$
denotes the capture-avoiding substitution of $\vec{y}$ for $\vec{x}$ in $Q$.
\end{definition}

\begin{definition}
  The {\em structural congruence} \cite{SangiorgiWalker} , $\equiv$,
  between processes is the least congruence containing
  alpha-equivalence, satisfying the abelian monoid laws
  (associativity, commutativity and $\pzero$ as identity) for parallel
  composition $|$ and for summation $+$.
\end{definition}

\subsection{Name equivalence}

We take name equivalence, written $\nameeq$, to be the smallest
equivalence relation generated by the following rules.

\begin{mathpar}
\inferrule*[lab=Quote-drop]
{ }
{ \quotep{@{x}} \nameeq x }

\inferrule*[lab=Struct-equiv]
{ P \scong Q }
{ \quotep{P} \nameeq \quotep{Q} }
\end{mathpar}

The astute reader will have noticed that the mutual recursion of names
and processes imposes a mutual recursion on alpha-equivalence and
structural equivalence via name-equivalence. Fortunately, all of this
works out pleasantly and we may calculate in the natural way, free of
concern. The reader interested in the details is referred to the
appendix \ref{appendix:rho_details}.

\subsection{Substitution}

We use $\Proc$ for the set of processes, $\QProc$ for the set of
names, and $\id{\{}\vec{y} / \vec{x} \id{\}}$ to denote partial maps,
$s : \QProc \rightarrow \QProc$. A map, $s$ lifts, uniquely, to a map
on process terms, $\widehat{s} : \Proc \rightarrow \Proc$ by the
following equations.

\begin{mathpar}
  (0) \psubstp{Q}{P} := 0 \\
  (R \juxtap S) \psubstp{Q}{P}
  :=    
  (R)\psubstp{Q}{P} \juxtap (S) \psubstp{Q}{P} \\
  (x?(y).R) \psubstp{Q}{P}    
  :=    
  (x)\substp{Q}{P} (z)\concat( (R \psubstn{z}{y}) \psubstp{Q}{P} ) \\
  (\lift{x}{R}) \psubstp{Q}{P}  
  :=
  \lift{(x)\substp{Q}{P}}{ R \psubstp{Q}{P} } \\
%   (\dropn{x})  \psubstp{Q}{P}       
%   := 
%   \left\{ 
%     \begin{array}{ccc} 
%       \dropn{\quotep{Q}} & & x \nameeq \quotep{P} \\
%       \dropn{x} & & otherwise \\
%     \end{array}
%   \right. 
  (\dropn{x})  \psubstp{Q}{P}       
  := 
  \left\{ 
    \begin{array}{ccc} 
      Q & & x \nameeq \quotep{P} \\
      \dropn{x} & & otherwise \\
    \end{array}
  \right.
\end{mathpar}
 

where

\begin{eqnarray}
  (x)\id{\{} \lpquote Q \rpquote / \lpquote P \rpquote \id{\}}            = 
  \left\{ 
    \begin{array}{ccc}
      \lpquote Q \rpquote & & x \nameeq \lpquote P \rpquote \\
      x & & otherwise \\
    \end{array}
  \right. \nonumber
\end{eqnarray}

and $z$ is chosen distinct from $\quotep{P}$, $\quotep{Q}$, the free
names in $Q$, and all the names in $R$. Our $\alpha$-equivalence will
be built in the standard way from this substitution.

\begin{remark}\label{rem:no_self_referential_names}
  One consequence of these definitions is that $\forall P. \quotep{P}
  \not\in \freenames{P}$.
\end{remark}

\subsection{ Dynamic quote: an example }

Anticipating something of what's to come, consider applying the
substitution, $\widehat{\id{\{}u / z \id{\}}}$, to the following pair
of processes, $\lift{w}{y!(z)}$ and $w[ \lpquote y!(z) \rpquote ]$.

\begin{eqnarray}
	\lift{w}{y!(z)}\widehat{\id{\{}u / z \id{\}}}
		& = &
		\lift{w}{y!(u)} \nonumber\\
	w[ \lpquote y!(z) \rpquote ] \widehat{ \id{\{}u / z \id{\}} }
		& = &
		w[ \lpquote y!(z) \rpquote ] \nonumber
\end{eqnarray}

Because the body of the process between quotes is impervious to
substitution, we get radically different answers. In fact, by
examining the first process in an input context,
e.g. $x?(z).\lift{w}{y!(z)}$, we see that the process under the lift
operator may be shaped by prefixed inputs binding a name inside it. In
this sense, the lift operator will be seen as a way to dynamically
construct processes before reifying them as names.

Finally equipped with these standard features we can present the
dynamics of the calculus.

\subsubsection{Operational semantics} 

Finally, we introduce the computational dynamics. What marks these
algebras as distinct from other more traditionally studied algebraic
structures, e.g. vector spaces or polynomial rings, is the manner in
which dynamics is captured. In traditional structures, dynamics is typically
expressed through morphisms between such structures, as in linear maps
between vector spaces or morphisms between rings. In algebras
associated with the semantics of computation, the dynamics is
expressed as part of the algebraic structure itself, through a
reduction reduction relation typically denoted by $\red$. Below, we
give a recursive presentation of this relation for the calculus used
in the encoding.

$\red \subseteq \pi \times \pi$
$\red : \pi \to \mathcal{P}(\pi)$

\begin{mathpar}
  \inferrule* [lab=Comm] { \textsf{match}( x_{src}, x_{trgt} ) } { x_{trgt}?(y)P \; | \; x_{src}!\langle {Q} \rangle \red P\{\quotep{Q}/y}\} }
  \and \\
  \inferrule* [lab=Par] {{P} \red {P}'} {{{P} | {Q}} \red {{P}' | {Q}}}
  \and
  \inferrule* [lab=Equiv]{{{P} \scong {P}'} \andalso {{P}' \red {Q}'} \andalso {{Q}' \scong {Q}}}{{P} \red {Q}}
\end{mathpar}

\begin{eqnarray*}
  match_{\equiv} (\quotep{P},\quotep{Q}) & := & P \equiv Q \\
  match_{\dagger}(\quotep{P},\quotep{Q}) & := & \forall R. P|Q \red^{*} R => R \red^{*} 0 \\
  match_{K}(\quotep{P},\quotep{Q}) & := & K \mbox{ for some context } K
\end{eqnarray*}

$u?(x)P | u!\langle Q \rangle \red P\{\quotep{Q}/x\}$

%We write $\wred$ for $\red^*$, and $P\red$ if $\exists Q $ such that $ P \red Q$.
We write $P\red$ if $\exists Q $ such that $ P \red Q$ and $P\not\red$, otherwise.

\section{Replication}

As mentioned before, it is known that replication (and hence
recursion) can be implemented in a higher-order process algebra
\cite{SangiorgiWalker}. As our first example of calculation with the
machinery thus far presented we give the construction explicitly in
the {\rhoc}.

\begin{eqnarray}
	D_{x} & := & \prefix{x}{y}{(\binpar{\outputp{x}{y}}{@{y}})} \nonumber\\
	\bangp_{x}{P} & := & \binpar{{x}!\langle{\binpar{D_{x}}{P}}\rangle}{D_{x}} \nonumber
\end{eqnarray}

\begin{eqnarray}
	\bangp_{x}{P} & & \nonumber\\
	=
	& {x}!\langle{(\prefix{x}{y}{(\outputp{x}{y} | @{y})) | P}}\rangle 
	      | \prefix{x}{y}{(\outputp{x}{y} | @{y})} & \nonumber\\
	\red
	& (\outputp{x}{y} | @{y})\substn{\quotep{(\prefix{x}{y}{(@{y} | \outputp{x}{y})) | P}}}{y} & \nonumber\\
	=
	& \outputp{x}{\quotep{(\prefix{x}{y}{(\outputp{x}{y} | @{y})) | P}}}
	  | {(\prefix{x}{y}{(\outputp{x}{y} | @{y})) | P}} & \nonumber\\
	\red
	& \ldots & \nonumber\\
	\red^*
	& P | P | \ldots & \nonumber
\end{eqnarray}

Of course, this encoding, as an implementation, runs away, unfolding
$\bangp{P}$ eagerly. A lazier and more implementable replication
operator, restricted to input-guarded processes, may be obtained as follows.

\begin{eqnarray}
\bangp{\prefix{u}{v}{P}} 
	:= 
	\binpar{\lift{x}{\prefix{u}{v}{(\binpar{D(x)}{P})}}}{D(x)} \nonumber
\end{eqnarray}

\begin{remark}
  Note that the lazier definition still does not deal with summation
  or mixed summation (i.e. sums over input and output). The reader is
  invited to construct definitions of replication that deal with these
  features. 

  Further, the definitions are parameterized in a name, $x$. Can you,
  gentle reader, make a definition that eliminates this parameter and
  guarantees no accidental interaction between the replication
  machinery and the process being replicated -- i.e. no accidental
  sharing of names used by the process to get its work done and the
  name(s) used by the replication to effect copying. This latter
  revision of the definition of replication is crucial to obtaining
  the expected identity $!!P \sim !P$.
\end{remark}

\begin{remark}\label{rem:paradoxical_combinator}
  The reader familiar with the lambda calculus will have noticed the
  similarity between $D$ and the paradoxical combinator.

  [Ed. note: the existence of this seems to suggest we have to be more
  restrictive on the set of processes and names we admit if we are to
  support no-cloning.]
\end{remark}

\subsubsection{Bisimulation}

The computational dynamics gives rise to another kind of equivalence,
the equivalence of computational behavior. As previously mentioned
this is typically captured \emph{via} some form of bisimulation.

% The notion we use in this paper is weak barbed bisimulation
% \cite{milner91polyadicpi}.

The notion we use in this paper is derived from weak barbed
bisimulation \cite{milner91polyadicpi}. 

\begin{definition}
An \emph{observation relation}, $\downarrow_{\mathcal N}$, over a set
of names, $\mathcal N$, is the smallest relation satisfying the rules
below.

\infrule[Out-barb]{y \in {\mathcal N}, \; x \nameeq y}
		  {\outputp{x}{v} \downarrow_{\mathcal N} x}
\infrule[Par-barb]{\mbox{$P\downarrow_{\mathcal N} x$ or $Q\downarrow_{\mathcal N} x$}}
		  {\binpar{P}{Q} \downarrow_{\mathcal N} x}

We write $P \Downarrow_{\mathcal N} x$ if there is $Q$ such that 
$P \wred Q$ and $Q \downarrow_{\mathcal N} x$.
\end{definition}

\begin{definition}
%\label{def.bbisim}
An  ${\mathcal N}$-\emph{barbed bisimulation} over a set of names, ${\mathcal N}$, is a symmetric binary relation 
${\mathcal S}_{\mathcal N}$ between agents such that $P\rel{S}_{\mathcal N}Q$ implies:
\begin{enumerate}
\item If $P \red P'$ then $Q \wred Q'$ and $P'\rel{S}_{\mathcal N} Q'$.
\item If $P\downarrow_{\mathcal N} x$, then $Q\Downarrow_{\mathcal N} x$.
\end{enumerate}
$P$ is ${\mathcal N}$-barbed bisimilar to $Q$, written
$P \wbbisim_{\mathcal N} Q$, if $P \rel{S}_{\mathcal N} Q$ for some ${\mathcal N}$-barbed bisimulation ${\mathcal S}_{\mathcal N}$.
\end{definition}

$\mathcal{R} \subseteq \pi \times \pi$

$P \mathcal{R} Q => \forall P'. P \red P' \Rightarrow \exists Q'. Q \red Q', P' \mathcal{R} Q'$

$P \vdash x \Rightarrow Q \vdash x$

\begin{mathpar}
  \inferrule*[lab=Out-barb]{x \nameeq y}{{y}!\langle{Q}\rangle \vdash x}
  \and
  \inferrule*[lab=Par-barb]{\mbox{$P\vdash x$ or $Q\vdash x$}}{\binpar{P}{Q} \vdash x}
\end{mathpar}

\subsubsection{Contexts}

One of the principle advantages of computational calculi like the
$\pi$-calculus is a well-defined notion of context,
contextual-equivalence and a correlation between
contextual-equivalence and notions of bisimulation. The notion of
context allows the decomposition of a process into (sub-)process and
its syntactic environment, its context. Thus, a context may be
thought of as a process with a ``hole'' (written $\Box$) in it. The
application of a context $M$ to a process $P$, written $M[P]$, is
tantamount to filling the hole in $M$ with $P$. In this paper we do
not need the full weight of this theory, but do make use of the notion
of context in the proof the main theorem. 

\begin{mathpar}
  \inferrule* [lab=summation] {} {{M_{M},M_{N}} \bc \Box \;|\; x.M_{A} \;|\; M_{M}+M_{N}}
  \and
  \inferrule* [lab=agent] {} {{M_{A}} \bc (\vec{x})M_{P} \;| \; \clift{P_0,\ldots,M_{P},\ldots,P_N}}
  \and \\
  \inferrule* [lab=process] {} {{M_{P}} \bc M_{N} \;| \;P|M_{P} }
\end{mathpar} 

\begin{mathpar}
  \inferrule* [lab=sychronization] {} {M_{N} \bc \Box \;|\; x?M_{F} \;|\; x!M_{C}}
  \and
  \inferrule* [lab=abstraction] {} {{M_{F}} \bc (x)M_{P} }
  \and
  \inferrule* [lab=concretion] {} {{M_{C}} \bc \langle M_{P} \rangle }
  \and \\
  \inferrule* [lab=process] {} {{M_{P}} \bc M_{N} \;| \;P|M_{P} }
\end{mathpar}

\begin{definition}[contextual application] Given a context $M$, and
  process $P$, we define the \emph{contextual application}, $M[P] :=
  M\{P/\Box\}$. That is, the contextual application of M to P is the
  substitution of $P$ for $\Box$ in $M$.
\end{definition}

$\meaningof{-} : L \to \mathcal{P}(\pi)$

\begin{mathpar}
  \inferrule* [lab=collection] {} {\meaningof{true} = \pi, \and \meaningof{~E} = \pi \setminus \meaningof{E}, \and \meaningof{E_{1} \& E_{2}} = \meaningof{E_{1}} \cap \meaningof{E_{2}}}
\end{mathpar}

\begin{mathpar}
  \inferrule* [lab=structure] {} {\meaningof{0} = \{ P \in \pi | P \equiv 0 \}, \and \\ \meaningof{E_1 | E_2} = \{ P \in \pi | P \equiv P_{1} | P_{2}, P_{1} \in \meaningof{E_{1}}, P_{2} \in \meaningof{E_2}\} }
\end{mathpar}

\begin{mathpar}
 \inferrule* [lab=behavior] {} {\meaningof{\langle a?b \rangle E} = \{ P \in \pi | P \equiv Q | u?(y)P', \\ \and \\\\ \and \\ \;\;\; u \in \meaningof{a}, \forall z.P'\{z/y\} \in \meaningof{E\{z/b\}}\}, \and \\ \meaningof{a!E} = \{ P \in \pi | P \equiv Q | x!\langle P' \rangle, x \in \meaningof{a} P' \in \meaningof{E}\} }
\end{mathpar}

\begin{mathpar}
 \inferrule* [lab=nominal] {} {\meaningof{\quotep{E}} = \{ \quotep{P} \in \quotep{\pi} | P \in \meaningof{E} \}, \and \meaningof{\quotep{P}} = \{ \quotep{Q} \in \quotep{\pi} | P \equiv Q \} \and \\ \meaningof{@\quotep{E}} = \{ P \in \pi | P \equiv @x, x \in \meaningof{E} \}}
\end{mathpar}

\begin{eqnarray*}
  \\
  \meaningof{-} : TS \to ST
\end{eqnarray*}

\begin{eqnarray*}
  \\
  L : TS \to ST
\end{eqnarray*}

\begin{eqnarray*}
  \\
  P \models E \iff P \in \meaningof{E}
\end{eqnarray*}

\begin{eqnarray*}
  P \approx_{L} Q \iff \forall E \in L. P \models E \iff Q \models E
\end{eqnarray*}

\begin{eqnarray*}
  P \approx_{K} Q
\end{eqnarray*}

\begin{eqnarray*}
  P \approx Q
\end{eqnarray*}

$\approx_{K} = \approx = \approx_{L}$

\subsubsection{Contextual duality}

Note that contexts extend the quotation operation to a family of
operations from processes to names. Given a context, $M$, we can
define a \emph{nominal context}, $\quotep{M}$ by $\quotep{M}[P] :=
\quotep{M[P]}$. To foreshadow what is to come we observe that these
operations enjoy a duality with processes very much like the duality
between vectors and maps from vectors to scalars.

Further, because the calculus is essentially higher-order, we have a
correspondence between contexts and processes. More specifically,
given a name $x$ and a context $M$ we can construct $M^{*}_{x}$ such
that 

\begin{mathpar}
  M^{*}_{x} | \lift{x}{P} \red M[P]
\end{mathpar}

namely,

\begin{mathpar}
  M^{*}_{x} := x?(u).M[\dropn{u}]
\end{mathpar}

The dependence of $M^{*}_{x}$ on a name makes it an abstraction, 

\begin{mathpar}
  M^{*} := (x)x?(u).M[\dropn{u}]
\end{mathpar}

\subsection{Additional notation}

It will sometimes be convenient to denote the process a name
quotes. We already have the notation $x = \quotep{P}$, but it will be
convenient to introduce an alternate notation, $\procn{x}$, when we
want to emphasize the connection to the use of the name. Note that, by
virtue of name equivalence, $\quotep{\procn{x}} \nameeq x$; so, the
notation is consistent with previous definitions.

Further, because names have structure it is possible to effect
substitutions on the basis of that structure. This means we need to
upgrade our notation for substitutions, which we accomplish by
adapting comprehension notation. Thus,

\begin{mathpar}
  P\{ y / x : x \in S \}
\end{mathpar}

is interpreted to mean the process derived from P by replacing (in a
capture-avoiding manner) each occurrence of $x$ in $S$ by $y$. For example,

\begin{mathpar}
  P\{ \quotep{\procn{x}|\procn{x}} / x : x \in \freenames{P} \}
\end{mathpar}

will replace each (occurrence) of a free name $x$ in $P$ by
$\quotep{\procn{x}|\procn{x}}$.

Also, we will avail ourselves of the notation $x^{L}$ and $x^{R}$ to
denote injections of a name into disjoint copies of the name
space. There are numerous ways to accomplish this. One example can be
found in \cite{MeredithR05}. This notation overloads to vectors of
names: $\vec{x}^{\pi} := (x_{i}^{\pi} \; : \; 0 \leq i < |\vec{x}| )$ where $\pi \in \{L,R\}$.

We also use $P^{\Box} := P|\Box$.

In \cite{MeredithR05} an interpretation of the new operator is
given. It turns out that there are several possible interpretations
all enjoying the requisite algebraic properties of the operator (see
\cite{milner91polyadicpi}). We will therefore make liberal use of
$(\nu\; \vec{x})P$.

% subsection the_syntax_and_semantics_of_the_notation_system (end)   

\input{qm2pi.qmops} 

\input{qm2pi.sterngerlach} 

\input{qm2pi.metric} 

% section concurrent_process_calculi (end)

%\input{qm2pi.proofsketch}

% section proof sketch (end)

%\input{qm2pi.slviaknots} 

% section spatial logic via knots (end)

\input{qm2pi.conclusion}

% section conclusion (end)

%\input{qm2pi.dtcodes} 

% section wiring algorithm (end)

\input{qm2pi.ack} 

% section acknowledgments (end)

\newpage


\bibliographystyle{plain}   
\bibliography{../../biblios/main.bib}

\input{qm2pi.rhodetails}

\end{document}



% section front matter (end)

\section{Introduction}\label{sec:introduction} % (fold)
In this draft of the material i am going to have to dispense with the
usual writing conventions adopted in papers on these topics. i'm going
to have adopt whatever tone i need at the time i'm writing up the
calculations. Sometimes this may be very conversational; others it may
be the barest mathematical grunts; others still it may be that i have
lifted text from one of my other papers because the exposition of some
point was better said there. i hope that my readers are not unduly put
out by this decision. i'm not doing this to flout convention or be
rebellious. i find these calculations very technically challenging. To
keep everything going technically, something has to give; i have to
let go of some cognitive burden. So, the academic writing style --
with all of its trade-offs in terms of facilitating technical
communication -- is what i'm letting go of. Perhaps subsequent drafts
can be tightened and polished, but for now, i'm going to speak as if
we were sitting together in a coffee shop with a laptop, wifi and a
pad of paper and a pencil.

So, here's what i have to say. We -- you and i, comfortably ensconced
in our coffee shop and well-equipped with our tools -- can realize and
carry out the calculations of quantum mechanics over a very different
formal theory of dynamics, a formal theory of dynamics that
corresponds to a theory of concurrent computation with
\emph{reflection}. It has the advantage that the underlying theory is
already `quantized', but supports analogues all of the continuuous
operations. Strikingly, this underlying theory has recently been
connected with a notion of metric that we can show, by calculating
together, coincides with the metric induced by the inner product.

There are a lot of reasons why you might be interested in seeing
calculations of this form. Here's why i'm interested. For the past
several centuries there has been no competitor to the ``Newtonian''
account of dynamics. As a result the predominant share of accounts of
dynamical systems and situations have had to be formulated in terms of
the Newtonian machinery. i view this as an intellectually dangerous
position to occupy. Everything, despite it's intrinsic shape, turns
into a nail to be hit with this hammer. Recently, however, the theory
of computation has matured to the point where we have candidates for
theories of dynamics that offer very different perspective on
reasoning about dynamical systems and situations. Testing these
candidates against very successful accounts of dynamical situations,
like quantum mechanics, is going to give us some sense of how mature
they are and some measure of the quality of these accounts of
dynamics.

\subsection{Summary of contributions and outline of paper}

So, we're going to develop an interpretation of the operations of
quantum mechanics normally interpreted by Hilbert spaces and
operators. We're going to do this over a theory of computation. Note
that this is very different than the usual quantum computation program
which develops notions of computation over quantum mechanics. Rather,
we are developing a story that aligns with Wheeler's slogan: It from
Bit. To do this we will first provide an account of the theory of
computation at play here. Then we will dive into a calculation-driven
interpretation of the operations of quantum mechanics.

The reason we take this approach is that -- until very recently --
there hasn't been an axiomatic account of quantum mechanics. As a
result there has been no sharp delineation of the mathematical theory
supporting interpretation of the physical theory and the physical
theory, itself. So, ambient features of the maths are free to be
exploited (or supressed) without a real accounting of their physical
relevance. There is no sharp statement ``here's the physical theory''
qua \emph{theory} and ``here's the mathematical interpretation''
enabling a judgment of how faithful the interpretation is -- apart
from experimental observation. When there is an axiomatic account we
can judge how well a given mathematical formalism supports an
interpretation of the axioms, independent of
experimentation. Likewise, we can judge how well we have captured our
physical evidence and experience with our axiomatics, independent of
any specific mathematical implementation, with accidental detail that
may or may not have physical significance. 

In lieu of a fully fleshed out and vetted axiomatic account of quantum
mechanics, interpreting the operational notions in service of modeling
physical systems will have to suffice. In other words, we are not in
the business of providing a model of Hilbert spaces and operators. We
are in the business of providing a model of quantum mechanics because
we are motivated by testing our notions of dynamics against physical
theory; and, the predictive calculations of the physical theory must
serve as the best formulation -- shy of a fully fleshed out axiomatic
account -- of the physical theory itself (as they have for scientific
theories since time immemorial). Put another way, despite a
whole-hearted commitment to an It-from-Bit ontology, we are firmly
aligned with the shut-up-and-calculate camp as the best way to obtain
results either from the physical perspective or as a quality assurance
measure of our fledgling theory of dynamics.

In detail, we present a reflective process calculus. Then we develop
intuitive correspondences between the notions available in this
calculus and the usual physical notions supporting quantum mechanical
calculations. Thus, 

\begin{table}[htp]
  \center{
    \fbox{
      \begin{tabular}{c|c}
        quantum mechanics & process calculus \\
        \hline
        scalar & name \\
        state vector & process \\
        dual & contextual duals \\
        matrix & formal sums of process-context-dual pairs \\
        orthogonality & process annihilation \\
        inner product & execution-formula + quoting
      \end{tabular}
    }
  }
  \caption{QM - process calculi correspondences}
\end{table}

Then we tighten up these intuitions to operational definitions. We
employ the Dirac notation as the best proxy we can find for an
abstract syntax of the quantum mechanical notions. The definitions we
develop put us in contact with equational constraints coming from the
theory that we demonstrate the definitions and calculations satisfy.

This puts us in a position to shut up and calculate for the
Stern-Gerlach experimental set up, showing how these predictive
calculations become calculations on processes in our theory of a
reflective process calculus.

Penultimately, we demonstrate that the notion of metric coming from
the inner product coincides with the notion of metric available from
the theory of bisimulation. This demonstration gives us the right to
think of space as arising from behavior. Finally, we consider where we
might go from the new vantage point we have obtained.

% section introduction (end) 
 
% section introduction (end)

% \documentclass[12pt]{llncs}
%\documentclass{jktr}

\usepackage[pdftex]{hyperref}                   
\usepackage {listings}
\usepackage {mathpartir}
\usepackage{bcprules}
%\usepackage{listings}
                       
\usepackage{graphicx} 
%\usepackage[margins=2.5cm,nohead,nofoot]{geometry}
%\usepackage{geometry}
\usepackage{amsfonts}
\usepackage{amstext}
\usepackage{latexsym}
\usepackage{amssymb}
\usepackage{color}


%\include{myPreamble}
\include{qm2pi.local} 

%\ifpdf
%\usepackage[pdftex]{graphicx}
%\else
%\usepackage{graphicx}
%\fi

 % \ifpdf
%  \usepackage{pdfsync}
%  \if


%\title{Brief Article}
%\author{David F. Snyder}
%\author{L.G. Meredith}

%\address{Dept. of Math., Texas State University--San Marcos, San Marcos, TX 78666}
       
\pagestyle{empty}


\begin{document}

\lstset{language=[Objective]Caml,frame=shadowbox}

\input{qm2pi.front}

% section front matter (end)

\input{qm2pi.intro} 
 
% section introduction (end)

% \input{qm2pi.knotations} 

% section notation (end)

\input{qm2pi.process.calculi} 

% section concurrent_process_calculi_and_spatial_logics_ (end)
    
%\input{qm2pi.knots2pi} 

%\input{qm2pi.trefoil} 

%\input{qm2pi.mainthm} 

% subsection basic_interpretation (end)

%\input{qm2pi.rho.presentation} 
\subsection{The syntax and semantics of the notation system}\label{sub:the_syntax_and_semantics_of_the_notation_system} % (fold)

We now summarize a technical presentation of the calculus that
embodies our theory of dynamics. The typical presentation of such a
calculus follows the style of giving generators and relations on
them. The grammar, below, describing term constructors, freely
generates the set of processes, $\Proc$. This set is then quotiented
by a relation known as structural congruence and it is over this set
that the notion of dynamics is expressed. This presentation is
essentially that of \cite{MeredithR05} with the addition of
polyadicity and summation. For readability we have relegated some of
the technical subtleties to an appendix.

\subsubsection{Process grammar}\label{subsub:process_grammar}

\begin{mathpar}
  \inferrule* [lab=synchronization] {} {{M} \bc \pzero \;|\; x?F \;|\; x!C }
  \and
  \inferrule* [lab=abstraction] {} {{F} \bc (x)P}
  \and
  \inferrule* [lab=concretion] {} {{C} \bc \langle Q \rangle}
  \and
  \inferrule* [lab=process] {} {{P,Q} \bc M \;| \;P|Q \;|\; @{x}}
  \and
  \inferrule* [lab=name] {} {{x} \bc \quotep{P}}
\end{mathpar} 

Note that $\vec{x}$ (resp. $\vec{P}$) denotes a vector of names
(resp. processes) of length $|\vec{x}|$ (resp. $|\vec{P}|$). We adopt
the following useful abbreviations.

\begin{mathpar}
   x?(\vec{y}).P := x.(\vec{y})P \and  x\clift{\vec{P}} := x.\clift{\vec{P}}
   \and x!(y) := \lift{x}{\dropn{y}}
   \and \Pi_{i=0}^{n-1}P_i := P_0 | \ldots | P_{n-1}
\end{mathpar}

\subsubsection{Structural congruence}

\paragraph{Free and bound names and alpha-equivalence.} At the
core of structural equivalence is alpha-equivalence which identifies
process that are the same up to a change of variable. Formally, we
recognize the distinction between free and bound names. The free names
of a process, $\freenames{P}$, may be calculated recursively as
follows:

\begin{mathpar}
\freenames{\pzero} := \emptyset
  \and \\
  \freenames{x?(y).P} := \{ x \} \cup (\freenames{P} \setminus \{ y \})
  \and 
  \freenames{x!\langle P \rangle} := \{ x \} \cup \{ P \} 
  \and \\
  \freenames{P|Q} := \freenames{P} \cup \freenames{Q}
  \and \\
  \freenames{@{x}} := \{ x \}
\end{mathpar}

$\pi$
$\quotep{\pi}$

$\freenames{-} : \pi \to \mathcal{P}(\quotep{\pi})$

\begin{eqnarray*}
  \freenames{\pzero} & := & \emptyset \\
  \freenames{x?(y).P} & := & \{ x \} \cup (\freenames{P} \setminus \{ y \}) \\
  \freenames{x!\langle P \rangle} & := & \{ x \} \cup \{ P \} \\
  \freenames{P|Q} & := & \freenames{P} \cup \freenames{Q} \\
  \freenames{\dropn{x}} & := & \{ x \}
\end{eqnarray*}

The bound names of a process, $\boundnames{P}$, are those names occurring in $P$
that are not free. For example, in $x?(y).0$, the name $x$ is free, while $y$ is bound.

\begin{mathpar}
  \inferrule* [lab=monoidal-laws] {} { P|Q \equiv Q|P \and P|0 \equiv P \and P|(Q|R) \equiv (P|Q)|R }
\end{mathpar}

\begin{mathpar}
  \inferrule* [lab=alpha-equivalence] {} { (x)P \equiv (y)P\{y/x\} \and y \not\in \freenames{P} }
\end{mathpar}

\begin{definition}
Then two processes, $P,Q$, are alpha-equivalent if $P = Q\{\vec{y}/\vec{x}\}$ for
some $\vec{x} \in \boundnames{Q},\vec{y} \in \boundnames{P}$, where $Q\{\vec{y}/\vec{x}\}$
denotes the capture-avoiding substitution of $\vec{y}$ for $\vec{x}$ in $Q$.
\end{definition}

\begin{definition}
  The {\em structural congruence} \cite{SangiorgiWalker} , $\equiv$,
  between processes is the least congruence containing
  alpha-equivalence, satisfying the abelian monoid laws
  (associativity, commutativity and $\pzero$ as identity) for parallel
  composition $|$ and for summation $+$.
\end{definition}

\subsection{Name equivalence}

We take name equivalence, written $\nameeq$, to be the smallest
equivalence relation generated by the following rules.

\begin{mathpar}
\inferrule*[lab=Quote-drop]
{ }
{ \quotep{@{x}} \nameeq x }

\inferrule*[lab=Struct-equiv]
{ P \scong Q }
{ \quotep{P} \nameeq \quotep{Q} }
\end{mathpar}

The astute reader will have noticed that the mutual recursion of names
and processes imposes a mutual recursion on alpha-equivalence and
structural equivalence via name-equivalence. Fortunately, all of this
works out pleasantly and we may calculate in the natural way, free of
concern. The reader interested in the details is referred to the
appendix \ref{appendix:rho_details}.

\subsection{Substitution}

We use $\Proc$ for the set of processes, $\QProc$ for the set of
names, and $\id{\{}\vec{y} / \vec{x} \id{\}}$ to denote partial maps,
$s : \QProc \rightarrow \QProc$. A map, $s$ lifts, uniquely, to a map
on process terms, $\widehat{s} : \Proc \rightarrow \Proc$ by the
following equations.

\begin{mathpar}
  (0) \psubstp{Q}{P} := 0 \\
  (R \juxtap S) \psubstp{Q}{P}
  :=    
  (R)\psubstp{Q}{P} \juxtap (S) \psubstp{Q}{P} \\
  (x?(y).R) \psubstp{Q}{P}    
  :=    
  (x)\substp{Q}{P} (z)\concat( (R \psubstn{z}{y}) \psubstp{Q}{P} ) \\
  (\lift{x}{R}) \psubstp{Q}{P}  
  :=
  \lift{(x)\substp{Q}{P}}{ R \psubstp{Q}{P} } \\
%   (\dropn{x})  \psubstp{Q}{P}       
%   := 
%   \left\{ 
%     \begin{array}{ccc} 
%       \dropn{\quotep{Q}} & & x \nameeq \quotep{P} \\
%       \dropn{x} & & otherwise \\
%     \end{array}
%   \right. 
  (\dropn{x})  \psubstp{Q}{P}       
  := 
  \left\{ 
    \begin{array}{ccc} 
      Q & & x \nameeq \quotep{P} \\
      \dropn{x} & & otherwise \\
    \end{array}
  \right.
\end{mathpar}
 

where

\begin{eqnarray}
  (x)\id{\{} \lpquote Q \rpquote / \lpquote P \rpquote \id{\}}            = 
  \left\{ 
    \begin{array}{ccc}
      \lpquote Q \rpquote & & x \nameeq \lpquote P \rpquote \\
      x & & otherwise \\
    \end{array}
  \right. \nonumber
\end{eqnarray}

and $z$ is chosen distinct from $\quotep{P}$, $\quotep{Q}$, the free
names in $Q$, and all the names in $R$. Our $\alpha$-equivalence will
be built in the standard way from this substitution.

\begin{remark}\label{rem:no_self_referential_names}
  One consequence of these definitions is that $\forall P. \quotep{P}
  \not\in \freenames{P}$.
\end{remark}

\subsection{ Dynamic quote: an example }

Anticipating something of what's to come, consider applying the
substitution, $\widehat{\id{\{}u / z \id{\}}}$, to the following pair
of processes, $\lift{w}{y!(z)}$ and $w[ \lpquote y!(z) \rpquote ]$.

\begin{eqnarray}
	\lift{w}{y!(z)}\widehat{\id{\{}u / z \id{\}}}
		& = &
		\lift{w}{y!(u)} \nonumber\\
	w[ \lpquote y!(z) \rpquote ] \widehat{ \id{\{}u / z \id{\}} }
		& = &
		w[ \lpquote y!(z) \rpquote ] \nonumber
\end{eqnarray}

Because the body of the process between quotes is impervious to
substitution, we get radically different answers. In fact, by
examining the first process in an input context,
e.g. $x?(z).\lift{w}{y!(z)}$, we see that the process under the lift
operator may be shaped by prefixed inputs binding a name inside it. In
this sense, the lift operator will be seen as a way to dynamically
construct processes before reifying them as names.

Finally equipped with these standard features we can present the
dynamics of the calculus.

\subsubsection{Operational semantics} 

Finally, we introduce the computational dynamics. What marks these
algebras as distinct from other more traditionally studied algebraic
structures, e.g. vector spaces or polynomial rings, is the manner in
which dynamics is captured. In traditional structures, dynamics is typically
expressed through morphisms between such structures, as in linear maps
between vector spaces or morphisms between rings. In algebras
associated with the semantics of computation, the dynamics is
expressed as part of the algebraic structure itself, through a
reduction reduction relation typically denoted by $\red$. Below, we
give a recursive presentation of this relation for the calculus used
in the encoding.

$\red \subseteq \pi \times \pi$
$\red : \pi \to \mathcal{P}(\pi)$

\begin{mathpar}
  \inferrule* [lab=Comm] { \textsf{match}( x_{src}, x_{trgt} ) } { x_{trgt}?(y)P \; | \; x_{src}!\langle {Q} \rangle \red P\{\quotep{Q}/y}\} }
  \and \\
  \inferrule* [lab=Par] {{P} \red {P}'} {{{P} | {Q}} \red {{P}' | {Q}}}
  \and
  \inferrule* [lab=Equiv]{{{P} \scong {P}'} \andalso {{P}' \red {Q}'} \andalso {{Q}' \scong {Q}}}{{P} \red {Q}}
\end{mathpar}

\begin{eqnarray*}
  match_{\equiv} (\quotep{P},\quotep{Q}) & := & P \equiv Q \\
  match_{\dagger}(\quotep{P},\quotep{Q}) & := & \forall R. P|Q \red^{*} R => R \red^{*} 0 \\
  match_{K}(\quotep{P},\quotep{Q}) & := & K \mbox{ for some context } K
\end{eqnarray*}

$u?(x)P | u!\langle Q \rangle \red P\{\quotep{Q}/x\}$

%We write $\wred$ for $\red^*$, and $P\red$ if $\exists Q $ such that $ P \red Q$.
We write $P\red$ if $\exists Q $ such that $ P \red Q$ and $P\not\red$, otherwise.

\section{Replication}

As mentioned before, it is known that replication (and hence
recursion) can be implemented in a higher-order process algebra
\cite{SangiorgiWalker}. As our first example of calculation with the
machinery thus far presented we give the construction explicitly in
the {\rhoc}.

\begin{eqnarray}
	D_{x} & := & \prefix{x}{y}{(\binpar{\outputp{x}{y}}{@{y}})} \nonumber\\
	\bangp_{x}{P} & := & \binpar{{x}!\langle{\binpar{D_{x}}{P}}\rangle}{D_{x}} \nonumber
\end{eqnarray}

\begin{eqnarray}
	\bangp_{x}{P} & & \nonumber\\
	=
	& {x}!\langle{(\prefix{x}{y}{(\outputp{x}{y} | @{y})) | P}}\rangle 
	      | \prefix{x}{y}{(\outputp{x}{y} | @{y})} & \nonumber\\
	\red
	& (\outputp{x}{y} | @{y})\substn{\quotep{(\prefix{x}{y}{(@{y} | \outputp{x}{y})) | P}}}{y} & \nonumber\\
	=
	& \outputp{x}{\quotep{(\prefix{x}{y}{(\outputp{x}{y} | @{y})) | P}}}
	  | {(\prefix{x}{y}{(\outputp{x}{y} | @{y})) | P}} & \nonumber\\
	\red
	& \ldots & \nonumber\\
	\red^*
	& P | P | \ldots & \nonumber
\end{eqnarray}

Of course, this encoding, as an implementation, runs away, unfolding
$\bangp{P}$ eagerly. A lazier and more implementable replication
operator, restricted to input-guarded processes, may be obtained as follows.

\begin{eqnarray}
\bangp{\prefix{u}{v}{P}} 
	:= 
	\binpar{\lift{x}{\prefix{u}{v}{(\binpar{D(x)}{P})}}}{D(x)} \nonumber
\end{eqnarray}

\begin{remark}
  Note that the lazier definition still does not deal with summation
  or mixed summation (i.e. sums over input and output). The reader is
  invited to construct definitions of replication that deal with these
  features. 

  Further, the definitions are parameterized in a name, $x$. Can you,
  gentle reader, make a definition that eliminates this parameter and
  guarantees no accidental interaction between the replication
  machinery and the process being replicated -- i.e. no accidental
  sharing of names used by the process to get its work done and the
  name(s) used by the replication to effect copying. This latter
  revision of the definition of replication is crucial to obtaining
  the expected identity $!!P \sim !P$.
\end{remark}

\begin{remark}\label{rem:paradoxical_combinator}
  The reader familiar with the lambda calculus will have noticed the
  similarity between $D$ and the paradoxical combinator.

  [Ed. note: the existence of this seems to suggest we have to be more
  restrictive on the set of processes and names we admit if we are to
  support no-cloning.]
\end{remark}

\subsubsection{Bisimulation}

The computational dynamics gives rise to another kind of equivalence,
the equivalence of computational behavior. As previously mentioned
this is typically captured \emph{via} some form of bisimulation.

% The notion we use in this paper is weak barbed bisimulation
% \cite{milner91polyadicpi}.

The notion we use in this paper is derived from weak barbed
bisimulation \cite{milner91polyadicpi}. 

\begin{definition}
An \emph{observation relation}, $\downarrow_{\mathcal N}$, over a set
of names, $\mathcal N$, is the smallest relation satisfying the rules
below.

\infrule[Out-barb]{y \in {\mathcal N}, \; x \nameeq y}
		  {\outputp{x}{v} \downarrow_{\mathcal N} x}
\infrule[Par-barb]{\mbox{$P\downarrow_{\mathcal N} x$ or $Q\downarrow_{\mathcal N} x$}}
		  {\binpar{P}{Q} \downarrow_{\mathcal N} x}

We write $P \Downarrow_{\mathcal N} x$ if there is $Q$ such that 
$P \wred Q$ and $Q \downarrow_{\mathcal N} x$.
\end{definition}

\begin{definition}
%\label{def.bbisim}
An  ${\mathcal N}$-\emph{barbed bisimulation} over a set of names, ${\mathcal N}$, is a symmetric binary relation 
${\mathcal S}_{\mathcal N}$ between agents such that $P\rel{S}_{\mathcal N}Q$ implies:
\begin{enumerate}
\item If $P \red P'$ then $Q \wred Q'$ and $P'\rel{S}_{\mathcal N} Q'$.
\item If $P\downarrow_{\mathcal N} x$, then $Q\Downarrow_{\mathcal N} x$.
\end{enumerate}
$P$ is ${\mathcal N}$-barbed bisimilar to $Q$, written
$P \wbbisim_{\mathcal N} Q$, if $P \rel{S}_{\mathcal N} Q$ for some ${\mathcal N}$-barbed bisimulation ${\mathcal S}_{\mathcal N}$.
\end{definition}

$\mathcal{R} \subseteq \pi \times \pi$

$P \mathcal{R} Q => \forall P'. P \red P' \Rightarrow \exists Q'. Q \red Q', P' \mathcal{R} Q'$

$P \vdash x \Rightarrow Q \vdash x$

\begin{mathpar}
  \inferrule*[lab=Out-barb]{x \nameeq y}{{y}!\langle{Q}\rangle \vdash x}
  \and
  \inferrule*[lab=Par-barb]{\mbox{$P\vdash x$ or $Q\vdash x$}}{\binpar{P}{Q} \vdash x}
\end{mathpar}

\subsubsection{Contexts}

One of the principle advantages of computational calculi like the
$\pi$-calculus is a well-defined notion of context,
contextual-equivalence and a correlation between
contextual-equivalence and notions of bisimulation. The notion of
context allows the decomposition of a process into (sub-)process and
its syntactic environment, its context. Thus, a context may be
thought of as a process with a ``hole'' (written $\Box$) in it. The
application of a context $M$ to a process $P$, written $M[P]$, is
tantamount to filling the hole in $M$ with $P$. In this paper we do
not need the full weight of this theory, but do make use of the notion
of context in the proof the main theorem. 

\begin{mathpar}
  \inferrule* [lab=summation] {} {{M_{M},M_{N}} \bc \Box \;|\; x.M_{A} \;|\; M_{M}+M_{N}}
  \and
  \inferrule* [lab=agent] {} {{M_{A}} \bc (\vec{x})M_{P} \;| \; \clift{P_0,\ldots,M_{P},\ldots,P_N}}
  \and \\
  \inferrule* [lab=process] {} {{M_{P}} \bc M_{N} \;| \;P|M_{P} }
\end{mathpar} 

\begin{mathpar}
  \inferrule* [lab=sychronization] {} {M_{N} \bc \Box \;|\; x?M_{F} \;|\; x!M_{C}}
  \and
  \inferrule* [lab=abstraction] {} {{M_{F}} \bc (x)M_{P} }
  \and
  \inferrule* [lab=concretion] {} {{M_{C}} \bc \langle M_{P} \rangle }
  \and \\
  \inferrule* [lab=process] {} {{M_{P}} \bc M_{N} \;| \;P|M_{P} }
\end{mathpar}

\begin{definition}[contextual application] Given a context $M$, and
  process $P$, we define the \emph{contextual application}, $M[P] :=
  M\{P/\Box\}$. That is, the contextual application of M to P is the
  substitution of $P$ for $\Box$ in $M$.
\end{definition}

$\meaningof{-} : L \to \mathcal{P}(\pi)$

\begin{mathpar}
  \inferrule* [lab=collection] {} {\meaningof{true} = \pi, \and \meaningof{~E} = \pi \setminus \meaningof{E}, \and \meaningof{E_{1} \& E_{2}} = \meaningof{E_{1}} \cap \meaningof{E_{2}}}
\end{mathpar}

\begin{mathpar}
  \inferrule* [lab=structure] {} {\meaningof{0} = \{ P \in \pi | P \equiv 0 \}, \and \\ \meaningof{E_1 | E_2} = \{ P \in \pi | P \equiv P_{1} | P_{2}, P_{1} \in \meaningof{E_{1}}, P_{2} \in \meaningof{E_2}\} }
\end{mathpar}

\begin{mathpar}
 \inferrule* [lab=behavior] {} {\meaningof{\langle a?b \rangle E} = \{ P \in \pi | P \equiv Q | u?(y)P', \\ \and \\\\ \and \\ \;\;\; u \in \meaningof{a}, \forall z.P'\{z/y\} \in \meaningof{E\{z/b\}}\}, \and \\ \meaningof{a!E} = \{ P \in \pi | P \equiv Q | x!\langle P' \rangle, x \in \meaningof{a} P' \in \meaningof{E}\} }
\end{mathpar}

\begin{mathpar}
 \inferrule* [lab=nominal] {} {\meaningof{\quotep{E}} = \{ \quotep{P} \in \quotep{\pi} | P \in \meaningof{E} \}, \and \meaningof{\quotep{P}} = \{ \quotep{Q} \in \quotep{\pi} | P \equiv Q \} \and \\ \meaningof{@\quotep{E}} = \{ P \in \pi | P \equiv @x, x \in \meaningof{E} \}}
\end{mathpar}

\begin{eqnarray*}
  \\
  \meaningof{-} : TS \to ST
\end{eqnarray*}

\begin{eqnarray*}
  \\
  L : TS \to ST
\end{eqnarray*}

\begin{eqnarray*}
  \\
  P \models E \iff P \in \meaningof{E}
\end{eqnarray*}

\begin{eqnarray*}
  P \approx_{L} Q \iff \forall E \in L. P \models E \iff Q \models E
\end{eqnarray*}

\begin{eqnarray*}
  P \approx_{K} Q
\end{eqnarray*}

\begin{eqnarray*}
  P \approx Q
\end{eqnarray*}

$\approx_{K} = \approx = \approx_{L}$

\subsubsection{Contextual duality}

Note that contexts extend the quotation operation to a family of
operations from processes to names. Given a context, $M$, we can
define a \emph{nominal context}, $\quotep{M}$ by $\quotep{M}[P] :=
\quotep{M[P]}$. To foreshadow what is to come we observe that these
operations enjoy a duality with processes very much like the duality
between vectors and maps from vectors to scalars.

Further, because the calculus is essentially higher-order, we have a
correspondence between contexts and processes. More specifically,
given a name $x$ and a context $M$ we can construct $M^{*}_{x}$ such
that 

\begin{mathpar}
  M^{*}_{x} | \lift{x}{P} \red M[P]
\end{mathpar}

namely,

\begin{mathpar}
  M^{*}_{x} := x?(u).M[\dropn{u}]
\end{mathpar}

The dependence of $M^{*}_{x}$ on a name makes it an abstraction, 

\begin{mathpar}
  M^{*} := (x)x?(u).M[\dropn{u}]
\end{mathpar}

\subsection{Additional notation}

It will sometimes be convenient to denote the process a name
quotes. We already have the notation $x = \quotep{P}$, but it will be
convenient to introduce an alternate notation, $\procn{x}$, when we
want to emphasize the connection to the use of the name. Note that, by
virtue of name equivalence, $\quotep{\procn{x}} \nameeq x$; so, the
notation is consistent with previous definitions.

Further, because names have structure it is possible to effect
substitutions on the basis of that structure. This means we need to
upgrade our notation for substitutions, which we accomplish by
adapting comprehension notation. Thus,

\begin{mathpar}
  P\{ y / x : x \in S \}
\end{mathpar}

is interpreted to mean the process derived from P by replacing (in a
capture-avoiding manner) each occurrence of $x$ in $S$ by $y$. For example,

\begin{mathpar}
  P\{ \quotep{\procn{x}|\procn{x}} / x : x \in \freenames{P} \}
\end{mathpar}

will replace each (occurrence) of a free name $x$ in $P$ by
$\quotep{\procn{x}|\procn{x}}$.

Also, we will avail ourselves of the notation $x^{L}$ and $x^{R}$ to
denote injections of a name into disjoint copies of the name
space. There are numerous ways to accomplish this. One example can be
found in \cite{MeredithR05}. This notation overloads to vectors of
names: $\vec{x}^{\pi} := (x_{i}^{\pi} \; : \; 0 \leq i < |\vec{x}| )$ where $\pi \in \{L,R\}$.

We also use $P^{\Box} := P|\Box$.

In \cite{MeredithR05} an interpretation of the new operator is
given. It turns out that there are several possible interpretations
all enjoying the requisite algebraic properties of the operator (see
\cite{milner91polyadicpi}). We will therefore make liberal use of
$(\nu\; \vec{x})P$.

% subsection the_syntax_and_semantics_of_the_notation_system (end)   

\input{qm2pi.qmops} 

\input{qm2pi.sterngerlach} 

\input{qm2pi.metric} 

% section concurrent_process_calculi (end)

%\input{qm2pi.proofsketch}

% section proof sketch (end)

%\input{qm2pi.slviaknots} 

% section spatial logic via knots (end)

\input{qm2pi.conclusion}

% section conclusion (end)

%\input{qm2pi.dtcodes} 

% section wiring algorithm (end)

\input{qm2pi.ack} 

% section acknowledgments (end)

\newpage


\bibliographystyle{plain}   
\bibliography{../../biblios/main.bib}

\input{qm2pi.rhodetails}

\end{document}

 

% section notation (end)

\input{qm2pi.process.calculi} 

% section concurrent_process_calculi_and_spatial_logics_ (end)
    
%\documentclass[12pt]{llncs}
%\documentclass{jktr}

\usepackage[pdftex]{hyperref}                   
\usepackage {listings}
\usepackage {mathpartir}
\usepackage{bcprules}
%\usepackage{listings}
                       
\usepackage{graphicx} 
%\usepackage[margins=2.5cm,nohead,nofoot]{geometry}
%\usepackage{geometry}
\usepackage{amsfonts}
\usepackage{amstext}
\usepackage{latexsym}
\usepackage{amssymb}
\usepackage{color}


%\include{myPreamble}
\include{qm2pi.local} 

%\ifpdf
%\usepackage[pdftex]{graphicx}
%\else
%\usepackage{graphicx}
%\fi

 % \ifpdf
%  \usepackage{pdfsync}
%  \if


%\title{Brief Article}
%\author{David F. Snyder}
%\author{L.G. Meredith}

%\address{Dept. of Math., Texas State University--San Marcos, San Marcos, TX 78666}
       
\pagestyle{empty}


\begin{document}

\lstset{language=[Objective]Caml,frame=shadowbox}

\input{qm2pi.front}

% section front matter (end)

\input{qm2pi.intro} 
 
% section introduction (end)

% \input{qm2pi.knotations} 

% section notation (end)

\input{qm2pi.process.calculi} 

% section concurrent_process_calculi_and_spatial_logics_ (end)
    
%\input{qm2pi.knots2pi} 

%\input{qm2pi.trefoil} 

%\input{qm2pi.mainthm} 

% subsection basic_interpretation (end)

%\input{qm2pi.rho.presentation} 
\subsection{The syntax and semantics of the notation system}\label{sub:the_syntax_and_semantics_of_the_notation_system} % (fold)

We now summarize a technical presentation of the calculus that
embodies our theory of dynamics. The typical presentation of such a
calculus follows the style of giving generators and relations on
them. The grammar, below, describing term constructors, freely
generates the set of processes, $\Proc$. This set is then quotiented
by a relation known as structural congruence and it is over this set
that the notion of dynamics is expressed. This presentation is
essentially that of \cite{MeredithR05} with the addition of
polyadicity and summation. For readability we have relegated some of
the technical subtleties to an appendix.

\subsubsection{Process grammar}\label{subsub:process_grammar}

\begin{mathpar}
  \inferrule* [lab=synchronization] {} {{M} \bc \pzero \;|\; x?F \;|\; x!C }
  \and
  \inferrule* [lab=abstraction] {} {{F} \bc (x)P}
  \and
  \inferrule* [lab=concretion] {} {{C} \bc \langle Q \rangle}
  \and
  \inferrule* [lab=process] {} {{P,Q} \bc M \;| \;P|Q \;|\; @{x}}
  \and
  \inferrule* [lab=name] {} {{x} \bc \quotep{P}}
\end{mathpar} 

Note that $\vec{x}$ (resp. $\vec{P}$) denotes a vector of names
(resp. processes) of length $|\vec{x}|$ (resp. $|\vec{P}|$). We adopt
the following useful abbreviations.

\begin{mathpar}
   x?(\vec{y}).P := x.(\vec{y})P \and  x\clift{\vec{P}} := x.\clift{\vec{P}}
   \and x!(y) := \lift{x}{\dropn{y}}
   \and \Pi_{i=0}^{n-1}P_i := P_0 | \ldots | P_{n-1}
\end{mathpar}

\subsubsection{Structural congruence}

\paragraph{Free and bound names and alpha-equivalence.} At the
core of structural equivalence is alpha-equivalence which identifies
process that are the same up to a change of variable. Formally, we
recognize the distinction between free and bound names. The free names
of a process, $\freenames{P}$, may be calculated recursively as
follows:

\begin{mathpar}
\freenames{\pzero} := \emptyset
  \and \\
  \freenames{x?(y).P} := \{ x \} \cup (\freenames{P} \setminus \{ y \})
  \and 
  \freenames{x!\langle P \rangle} := \{ x \} \cup \{ P \} 
  \and \\
  \freenames{P|Q} := \freenames{P} \cup \freenames{Q}
  \and \\
  \freenames{@{x}} := \{ x \}
\end{mathpar}

$\pi$
$\quotep{\pi}$

$\freenames{-} : \pi \to \mathcal{P}(\quotep{\pi})$

\begin{eqnarray*}
  \freenames{\pzero} & := & \emptyset \\
  \freenames{x?(y).P} & := & \{ x \} \cup (\freenames{P} \setminus \{ y \}) \\
  \freenames{x!\langle P \rangle} & := & \{ x \} \cup \{ P \} \\
  \freenames{P|Q} & := & \freenames{P} \cup \freenames{Q} \\
  \freenames{\dropn{x}} & := & \{ x \}
\end{eqnarray*}

The bound names of a process, $\boundnames{P}$, are those names occurring in $P$
that are not free. For example, in $x?(y).0$, the name $x$ is free, while $y$ is bound.

\begin{mathpar}
  \inferrule* [lab=monoidal-laws] {} { P|Q \equiv Q|P \and P|0 \equiv P \and P|(Q|R) \equiv (P|Q)|R }
\end{mathpar}

\begin{mathpar}
  \inferrule* [lab=alpha-equivalence] {} { (x)P \equiv (y)P\{y/x\} \and y \not\in \freenames{P} }
\end{mathpar}

\begin{definition}
Then two processes, $P,Q$, are alpha-equivalent if $P = Q\{\vec{y}/\vec{x}\}$ for
some $\vec{x} \in \boundnames{Q},\vec{y} \in \boundnames{P}$, where $Q\{\vec{y}/\vec{x}\}$
denotes the capture-avoiding substitution of $\vec{y}$ for $\vec{x}$ in $Q$.
\end{definition}

\begin{definition}
  The {\em structural congruence} \cite{SangiorgiWalker} , $\equiv$,
  between processes is the least congruence containing
  alpha-equivalence, satisfying the abelian monoid laws
  (associativity, commutativity and $\pzero$ as identity) for parallel
  composition $|$ and for summation $+$.
\end{definition}

\subsection{Name equivalence}

We take name equivalence, written $\nameeq$, to be the smallest
equivalence relation generated by the following rules.

\begin{mathpar}
\inferrule*[lab=Quote-drop]
{ }
{ \quotep{@{x}} \nameeq x }

\inferrule*[lab=Struct-equiv]
{ P \scong Q }
{ \quotep{P} \nameeq \quotep{Q} }
\end{mathpar}

The astute reader will have noticed that the mutual recursion of names
and processes imposes a mutual recursion on alpha-equivalence and
structural equivalence via name-equivalence. Fortunately, all of this
works out pleasantly and we may calculate in the natural way, free of
concern. The reader interested in the details is referred to the
appendix \ref{appendix:rho_details}.

\subsection{Substitution}

We use $\Proc$ for the set of processes, $\QProc$ for the set of
names, and $\id{\{}\vec{y} / \vec{x} \id{\}}$ to denote partial maps,
$s : \QProc \rightarrow \QProc$. A map, $s$ lifts, uniquely, to a map
on process terms, $\widehat{s} : \Proc \rightarrow \Proc$ by the
following equations.

\begin{mathpar}
  (0) \psubstp{Q}{P} := 0 \\
  (R \juxtap S) \psubstp{Q}{P}
  :=    
  (R)\psubstp{Q}{P} \juxtap (S) \psubstp{Q}{P} \\
  (x?(y).R) \psubstp{Q}{P}    
  :=    
  (x)\substp{Q}{P} (z)\concat( (R \psubstn{z}{y}) \psubstp{Q}{P} ) \\
  (\lift{x}{R}) \psubstp{Q}{P}  
  :=
  \lift{(x)\substp{Q}{P}}{ R \psubstp{Q}{P} } \\
%   (\dropn{x})  \psubstp{Q}{P}       
%   := 
%   \left\{ 
%     \begin{array}{ccc} 
%       \dropn{\quotep{Q}} & & x \nameeq \quotep{P} \\
%       \dropn{x} & & otherwise \\
%     \end{array}
%   \right. 
  (\dropn{x})  \psubstp{Q}{P}       
  := 
  \left\{ 
    \begin{array}{ccc} 
      Q & & x \nameeq \quotep{P} \\
      \dropn{x} & & otherwise \\
    \end{array}
  \right.
\end{mathpar}
 

where

\begin{eqnarray}
  (x)\id{\{} \lpquote Q \rpquote / \lpquote P \rpquote \id{\}}            = 
  \left\{ 
    \begin{array}{ccc}
      \lpquote Q \rpquote & & x \nameeq \lpquote P \rpquote \\
      x & & otherwise \\
    \end{array}
  \right. \nonumber
\end{eqnarray}

and $z$ is chosen distinct from $\quotep{P}$, $\quotep{Q}$, the free
names in $Q$, and all the names in $R$. Our $\alpha$-equivalence will
be built in the standard way from this substitution.

\begin{remark}\label{rem:no_self_referential_names}
  One consequence of these definitions is that $\forall P. \quotep{P}
  \not\in \freenames{P}$.
\end{remark}

\subsection{ Dynamic quote: an example }

Anticipating something of what's to come, consider applying the
substitution, $\widehat{\id{\{}u / z \id{\}}}$, to the following pair
of processes, $\lift{w}{y!(z)}$ and $w[ \lpquote y!(z) \rpquote ]$.

\begin{eqnarray}
	\lift{w}{y!(z)}\widehat{\id{\{}u / z \id{\}}}
		& = &
		\lift{w}{y!(u)} \nonumber\\
	w[ \lpquote y!(z) \rpquote ] \widehat{ \id{\{}u / z \id{\}} }
		& = &
		w[ \lpquote y!(z) \rpquote ] \nonumber
\end{eqnarray}

Because the body of the process between quotes is impervious to
substitution, we get radically different answers. In fact, by
examining the first process in an input context,
e.g. $x?(z).\lift{w}{y!(z)}$, we see that the process under the lift
operator may be shaped by prefixed inputs binding a name inside it. In
this sense, the lift operator will be seen as a way to dynamically
construct processes before reifying them as names.

Finally equipped with these standard features we can present the
dynamics of the calculus.

\subsubsection{Operational semantics} 

Finally, we introduce the computational dynamics. What marks these
algebras as distinct from other more traditionally studied algebraic
structures, e.g. vector spaces or polynomial rings, is the manner in
which dynamics is captured. In traditional structures, dynamics is typically
expressed through morphisms between such structures, as in linear maps
between vector spaces or morphisms between rings. In algebras
associated with the semantics of computation, the dynamics is
expressed as part of the algebraic structure itself, through a
reduction reduction relation typically denoted by $\red$. Below, we
give a recursive presentation of this relation for the calculus used
in the encoding.

$\red \subseteq \pi \times \pi$
$\red : \pi \to \mathcal{P}(\pi)$

\begin{mathpar}
  \inferrule* [lab=Comm] { \textsf{match}( x_{src}, x_{trgt} ) } { x_{trgt}?(y)P \; | \; x_{src}!\langle {Q} \rangle \red P\{\quotep{Q}/y}\} }
  \and \\
  \inferrule* [lab=Par] {{P} \red {P}'} {{{P} | {Q}} \red {{P}' | {Q}}}
  \and
  \inferrule* [lab=Equiv]{{{P} \scong {P}'} \andalso {{P}' \red {Q}'} \andalso {{Q}' \scong {Q}}}{{P} \red {Q}}
\end{mathpar}

\begin{eqnarray*}
  match_{\equiv} (\quotep{P},\quotep{Q}) & := & P \equiv Q \\
  match_{\dagger}(\quotep{P},\quotep{Q}) & := & \forall R. P|Q \red^{*} R => R \red^{*} 0 \\
  match_{K}(\quotep{P},\quotep{Q}) & := & K \mbox{ for some context } K
\end{eqnarray*}

$u?(x)P | u!\langle Q \rangle \red P\{\quotep{Q}/x\}$

%We write $\wred$ for $\red^*$, and $P\red$ if $\exists Q $ such that $ P \red Q$.
We write $P\red$ if $\exists Q $ such that $ P \red Q$ and $P\not\red$, otherwise.

\section{Replication}

As mentioned before, it is known that replication (and hence
recursion) can be implemented in a higher-order process algebra
\cite{SangiorgiWalker}. As our first example of calculation with the
machinery thus far presented we give the construction explicitly in
the {\rhoc}.

\begin{eqnarray}
	D_{x} & := & \prefix{x}{y}{(\binpar{\outputp{x}{y}}{@{y}})} \nonumber\\
	\bangp_{x}{P} & := & \binpar{{x}!\langle{\binpar{D_{x}}{P}}\rangle}{D_{x}} \nonumber
\end{eqnarray}

\begin{eqnarray}
	\bangp_{x}{P} & & \nonumber\\
	=
	& {x}!\langle{(\prefix{x}{y}{(\outputp{x}{y} | @{y})) | P}}\rangle 
	      | \prefix{x}{y}{(\outputp{x}{y} | @{y})} & \nonumber\\
	\red
	& (\outputp{x}{y} | @{y})\substn{\quotep{(\prefix{x}{y}{(@{y} | \outputp{x}{y})) | P}}}{y} & \nonumber\\
	=
	& \outputp{x}{\quotep{(\prefix{x}{y}{(\outputp{x}{y} | @{y})) | P}}}
	  | {(\prefix{x}{y}{(\outputp{x}{y} | @{y})) | P}} & \nonumber\\
	\red
	& \ldots & \nonumber\\
	\red^*
	& P | P | \ldots & \nonumber
\end{eqnarray}

Of course, this encoding, as an implementation, runs away, unfolding
$\bangp{P}$ eagerly. A lazier and more implementable replication
operator, restricted to input-guarded processes, may be obtained as follows.

\begin{eqnarray}
\bangp{\prefix{u}{v}{P}} 
	:= 
	\binpar{\lift{x}{\prefix{u}{v}{(\binpar{D(x)}{P})}}}{D(x)} \nonumber
\end{eqnarray}

\begin{remark}
  Note that the lazier definition still does not deal with summation
  or mixed summation (i.e. sums over input and output). The reader is
  invited to construct definitions of replication that deal with these
  features. 

  Further, the definitions are parameterized in a name, $x$. Can you,
  gentle reader, make a definition that eliminates this parameter and
  guarantees no accidental interaction between the replication
  machinery and the process being replicated -- i.e. no accidental
  sharing of names used by the process to get its work done and the
  name(s) used by the replication to effect copying. This latter
  revision of the definition of replication is crucial to obtaining
  the expected identity $!!P \sim !P$.
\end{remark}

\begin{remark}\label{rem:paradoxical_combinator}
  The reader familiar with the lambda calculus will have noticed the
  similarity between $D$ and the paradoxical combinator.

  [Ed. note: the existence of this seems to suggest we have to be more
  restrictive on the set of processes and names we admit if we are to
  support no-cloning.]
\end{remark}

\subsubsection{Bisimulation}

The computational dynamics gives rise to another kind of equivalence,
the equivalence of computational behavior. As previously mentioned
this is typically captured \emph{via} some form of bisimulation.

% The notion we use in this paper is weak barbed bisimulation
% \cite{milner91polyadicpi}.

The notion we use in this paper is derived from weak barbed
bisimulation \cite{milner91polyadicpi}. 

\begin{definition}
An \emph{observation relation}, $\downarrow_{\mathcal N}$, over a set
of names, $\mathcal N$, is the smallest relation satisfying the rules
below.

\infrule[Out-barb]{y \in {\mathcal N}, \; x \nameeq y}
		  {\outputp{x}{v} \downarrow_{\mathcal N} x}
\infrule[Par-barb]{\mbox{$P\downarrow_{\mathcal N} x$ or $Q\downarrow_{\mathcal N} x$}}
		  {\binpar{P}{Q} \downarrow_{\mathcal N} x}

We write $P \Downarrow_{\mathcal N} x$ if there is $Q$ such that 
$P \wred Q$ and $Q \downarrow_{\mathcal N} x$.
\end{definition}

\begin{definition}
%\label{def.bbisim}
An  ${\mathcal N}$-\emph{barbed bisimulation} over a set of names, ${\mathcal N}$, is a symmetric binary relation 
${\mathcal S}_{\mathcal N}$ between agents such that $P\rel{S}_{\mathcal N}Q$ implies:
\begin{enumerate}
\item If $P \red P'$ then $Q \wred Q'$ and $P'\rel{S}_{\mathcal N} Q'$.
\item If $P\downarrow_{\mathcal N} x$, then $Q\Downarrow_{\mathcal N} x$.
\end{enumerate}
$P$ is ${\mathcal N}$-barbed bisimilar to $Q$, written
$P \wbbisim_{\mathcal N} Q$, if $P \rel{S}_{\mathcal N} Q$ for some ${\mathcal N}$-barbed bisimulation ${\mathcal S}_{\mathcal N}$.
\end{definition}

$\mathcal{R} \subseteq \pi \times \pi$

$P \mathcal{R} Q => \forall P'. P \red P' \Rightarrow \exists Q'. Q \red Q', P' \mathcal{R} Q'$

$P \vdash x \Rightarrow Q \vdash x$

\begin{mathpar}
  \inferrule*[lab=Out-barb]{x \nameeq y}{{y}!\langle{Q}\rangle \vdash x}
  \and
  \inferrule*[lab=Par-barb]{\mbox{$P\vdash x$ or $Q\vdash x$}}{\binpar{P}{Q} \vdash x}
\end{mathpar}

\subsubsection{Contexts}

One of the principle advantages of computational calculi like the
$\pi$-calculus is a well-defined notion of context,
contextual-equivalence and a correlation between
contextual-equivalence and notions of bisimulation. The notion of
context allows the decomposition of a process into (sub-)process and
its syntactic environment, its context. Thus, a context may be
thought of as a process with a ``hole'' (written $\Box$) in it. The
application of a context $M$ to a process $P$, written $M[P]$, is
tantamount to filling the hole in $M$ with $P$. In this paper we do
not need the full weight of this theory, but do make use of the notion
of context in the proof the main theorem. 

\begin{mathpar}
  \inferrule* [lab=summation] {} {{M_{M},M_{N}} \bc \Box \;|\; x.M_{A} \;|\; M_{M}+M_{N}}
  \and
  \inferrule* [lab=agent] {} {{M_{A}} \bc (\vec{x})M_{P} \;| \; \clift{P_0,\ldots,M_{P},\ldots,P_N}}
  \and \\
  \inferrule* [lab=process] {} {{M_{P}} \bc M_{N} \;| \;P|M_{P} }
\end{mathpar} 

\begin{mathpar}
  \inferrule* [lab=sychronization] {} {M_{N} \bc \Box \;|\; x?M_{F} \;|\; x!M_{C}}
  \and
  \inferrule* [lab=abstraction] {} {{M_{F}} \bc (x)M_{P} }
  \and
  \inferrule* [lab=concretion] {} {{M_{C}} \bc \langle M_{P} \rangle }
  \and \\
  \inferrule* [lab=process] {} {{M_{P}} \bc M_{N} \;| \;P|M_{P} }
\end{mathpar}

\begin{definition}[contextual application] Given a context $M$, and
  process $P$, we define the \emph{contextual application}, $M[P] :=
  M\{P/\Box\}$. That is, the contextual application of M to P is the
  substitution of $P$ for $\Box$ in $M$.
\end{definition}

$\meaningof{-} : L \to \mathcal{P}(\pi)$

\begin{mathpar}
  \inferrule* [lab=collection] {} {\meaningof{true} = \pi, \and \meaningof{~E} = \pi \setminus \meaningof{E}, \and \meaningof{E_{1} \& E_{2}} = \meaningof{E_{1}} \cap \meaningof{E_{2}}}
\end{mathpar}

\begin{mathpar}
  \inferrule* [lab=structure] {} {\meaningof{0} = \{ P \in \pi | P \equiv 0 \}, \and \\ \meaningof{E_1 | E_2} = \{ P \in \pi | P \equiv P_{1} | P_{2}, P_{1} \in \meaningof{E_{1}}, P_{2} \in \meaningof{E_2}\} }
\end{mathpar}

\begin{mathpar}
 \inferrule* [lab=behavior] {} {\meaningof{\langle a?b \rangle E} = \{ P \in \pi | P \equiv Q | u?(y)P', \\ \and \\\\ \and \\ \;\;\; u \in \meaningof{a}, \forall z.P'\{z/y\} \in \meaningof{E\{z/b\}}\}, \and \\ \meaningof{a!E} = \{ P \in \pi | P \equiv Q | x!\langle P' \rangle, x \in \meaningof{a} P' \in \meaningof{E}\} }
\end{mathpar}

\begin{mathpar}
 \inferrule* [lab=nominal] {} {\meaningof{\quotep{E}} = \{ \quotep{P} \in \quotep{\pi} | P \in \meaningof{E} \}, \and \meaningof{\quotep{P}} = \{ \quotep{Q} \in \quotep{\pi} | P \equiv Q \} \and \\ \meaningof{@\quotep{E}} = \{ P \in \pi | P \equiv @x, x \in \meaningof{E} \}}
\end{mathpar}

\begin{eqnarray*}
  \\
  \meaningof{-} : TS \to ST
\end{eqnarray*}

\begin{eqnarray*}
  \\
  L : TS \to ST
\end{eqnarray*}

\begin{eqnarray*}
  \\
  P \models E \iff P \in \meaningof{E}
\end{eqnarray*}

\begin{eqnarray*}
  P \approx_{L} Q \iff \forall E \in L. P \models E \iff Q \models E
\end{eqnarray*}

\begin{eqnarray*}
  P \approx_{K} Q
\end{eqnarray*}

\begin{eqnarray*}
  P \approx Q
\end{eqnarray*}

$\approx_{K} = \approx = \approx_{L}$

\subsubsection{Contextual duality}

Note that contexts extend the quotation operation to a family of
operations from processes to names. Given a context, $M$, we can
define a \emph{nominal context}, $\quotep{M}$ by $\quotep{M}[P] :=
\quotep{M[P]}$. To foreshadow what is to come we observe that these
operations enjoy a duality with processes very much like the duality
between vectors and maps from vectors to scalars.

Further, because the calculus is essentially higher-order, we have a
correspondence between contexts and processes. More specifically,
given a name $x$ and a context $M$ we can construct $M^{*}_{x}$ such
that 

\begin{mathpar}
  M^{*}_{x} | \lift{x}{P} \red M[P]
\end{mathpar}

namely,

\begin{mathpar}
  M^{*}_{x} := x?(u).M[\dropn{u}]
\end{mathpar}

The dependence of $M^{*}_{x}$ on a name makes it an abstraction, 

\begin{mathpar}
  M^{*} := (x)x?(u).M[\dropn{u}]
\end{mathpar}

\subsection{Additional notation}

It will sometimes be convenient to denote the process a name
quotes. We already have the notation $x = \quotep{P}$, but it will be
convenient to introduce an alternate notation, $\procn{x}$, when we
want to emphasize the connection to the use of the name. Note that, by
virtue of name equivalence, $\quotep{\procn{x}} \nameeq x$; so, the
notation is consistent with previous definitions.

Further, because names have structure it is possible to effect
substitutions on the basis of that structure. This means we need to
upgrade our notation for substitutions, which we accomplish by
adapting comprehension notation. Thus,

\begin{mathpar}
  P\{ y / x : x \in S \}
\end{mathpar}

is interpreted to mean the process derived from P by replacing (in a
capture-avoiding manner) each occurrence of $x$ in $S$ by $y$. For example,

\begin{mathpar}
  P\{ \quotep{\procn{x}|\procn{x}} / x : x \in \freenames{P} \}
\end{mathpar}

will replace each (occurrence) of a free name $x$ in $P$ by
$\quotep{\procn{x}|\procn{x}}$.

Also, we will avail ourselves of the notation $x^{L}$ and $x^{R}$ to
denote injections of a name into disjoint copies of the name
space. There are numerous ways to accomplish this. One example can be
found in \cite{MeredithR05}. This notation overloads to vectors of
names: $\vec{x}^{\pi} := (x_{i}^{\pi} \; : \; 0 \leq i < |\vec{x}| )$ where $\pi \in \{L,R\}$.

We also use $P^{\Box} := P|\Box$.

In \cite{MeredithR05} an interpretation of the new operator is
given. It turns out that there are several possible interpretations
all enjoying the requisite algebraic properties of the operator (see
\cite{milner91polyadicpi}). We will therefore make liberal use of
$(\nu\; \vec{x})P$.

% subsection the_syntax_and_semantics_of_the_notation_system (end)   

\input{qm2pi.qmops} 

\input{qm2pi.sterngerlach} 

\input{qm2pi.metric} 

% section concurrent_process_calculi (end)

%\input{qm2pi.proofsketch}

% section proof sketch (end)

%\input{qm2pi.slviaknots} 

% section spatial logic via knots (end)

\input{qm2pi.conclusion}

% section conclusion (end)

%\input{qm2pi.dtcodes} 

% section wiring algorithm (end)

\input{qm2pi.ack} 

% section acknowledgments (end)

\newpage


\bibliographystyle{plain}   
\bibliography{../../biblios/main.bib}

\input{qm2pi.rhodetails}

\end{document}

 

%\documentclass[12pt]{llncs}
%\documentclass{jktr}

\usepackage[pdftex]{hyperref}                   
\usepackage {listings}
\usepackage {mathpartir}
\usepackage{bcprules}
%\usepackage{listings}
                       
\usepackage{graphicx} 
%\usepackage[margins=2.5cm,nohead,nofoot]{geometry}
%\usepackage{geometry}
\usepackage{amsfonts}
\usepackage{amstext}
\usepackage{latexsym}
\usepackage{amssymb}
\usepackage{color}


%\include{myPreamble}
\include{qm2pi.local} 

%\ifpdf
%\usepackage[pdftex]{graphicx}
%\else
%\usepackage{graphicx}
%\fi

 % \ifpdf
%  \usepackage{pdfsync}
%  \if


%\title{Brief Article}
%\author{David F. Snyder}
%\author{L.G. Meredith}

%\address{Dept. of Math., Texas State University--San Marcos, San Marcos, TX 78666}
       
\pagestyle{empty}


\begin{document}

\lstset{language=[Objective]Caml,frame=shadowbox}

\input{qm2pi.front}

% section front matter (end)

\input{qm2pi.intro} 
 
% section introduction (end)

% \input{qm2pi.knotations} 

% section notation (end)

\input{qm2pi.process.calculi} 

% section concurrent_process_calculi_and_spatial_logics_ (end)
    
%\input{qm2pi.knots2pi} 

%\input{qm2pi.trefoil} 

%\input{qm2pi.mainthm} 

% subsection basic_interpretation (end)

%\input{qm2pi.rho.presentation} 
\subsection{The syntax and semantics of the notation system}\label{sub:the_syntax_and_semantics_of_the_notation_system} % (fold)

We now summarize a technical presentation of the calculus that
embodies our theory of dynamics. The typical presentation of such a
calculus follows the style of giving generators and relations on
them. The grammar, below, describing term constructors, freely
generates the set of processes, $\Proc$. This set is then quotiented
by a relation known as structural congruence and it is over this set
that the notion of dynamics is expressed. This presentation is
essentially that of \cite{MeredithR05} with the addition of
polyadicity and summation. For readability we have relegated some of
the technical subtleties to an appendix.

\subsubsection{Process grammar}\label{subsub:process_grammar}

\begin{mathpar}
  \inferrule* [lab=synchronization] {} {{M} \bc \pzero \;|\; x?F \;|\; x!C }
  \and
  \inferrule* [lab=abstraction] {} {{F} \bc (x)P}
  \and
  \inferrule* [lab=concretion] {} {{C} \bc \langle Q \rangle}
  \and
  \inferrule* [lab=process] {} {{P,Q} \bc M \;| \;P|Q \;|\; @{x}}
  \and
  \inferrule* [lab=name] {} {{x} \bc \quotep{P}}
\end{mathpar} 

Note that $\vec{x}$ (resp. $\vec{P}$) denotes a vector of names
(resp. processes) of length $|\vec{x}|$ (resp. $|\vec{P}|$). We adopt
the following useful abbreviations.

\begin{mathpar}
   x?(\vec{y}).P := x.(\vec{y})P \and  x\clift{\vec{P}} := x.\clift{\vec{P}}
   \and x!(y) := \lift{x}{\dropn{y}}
   \and \Pi_{i=0}^{n-1}P_i := P_0 | \ldots | P_{n-1}
\end{mathpar}

\subsubsection{Structural congruence}

\paragraph{Free and bound names and alpha-equivalence.} At the
core of structural equivalence is alpha-equivalence which identifies
process that are the same up to a change of variable. Formally, we
recognize the distinction between free and bound names. The free names
of a process, $\freenames{P}$, may be calculated recursively as
follows:

\begin{mathpar}
\freenames{\pzero} := \emptyset
  \and \\
  \freenames{x?(y).P} := \{ x \} \cup (\freenames{P} \setminus \{ y \})
  \and 
  \freenames{x!\langle P \rangle} := \{ x \} \cup \{ P \} 
  \and \\
  \freenames{P|Q} := \freenames{P} \cup \freenames{Q}
  \and \\
  \freenames{@{x}} := \{ x \}
\end{mathpar}

$\pi$
$\quotep{\pi}$

$\freenames{-} : \pi \to \mathcal{P}(\quotep{\pi})$

\begin{eqnarray*}
  \freenames{\pzero} & := & \emptyset \\
  \freenames{x?(y).P} & := & \{ x \} \cup (\freenames{P} \setminus \{ y \}) \\
  \freenames{x!\langle P \rangle} & := & \{ x \} \cup \{ P \} \\
  \freenames{P|Q} & := & \freenames{P} \cup \freenames{Q} \\
  \freenames{\dropn{x}} & := & \{ x \}
\end{eqnarray*}

The bound names of a process, $\boundnames{P}$, are those names occurring in $P$
that are not free. For example, in $x?(y).0$, the name $x$ is free, while $y$ is bound.

\begin{mathpar}
  \inferrule* [lab=monoidal-laws] {} { P|Q \equiv Q|P \and P|0 \equiv P \and P|(Q|R) \equiv (P|Q)|R }
\end{mathpar}

\begin{mathpar}
  \inferrule* [lab=alpha-equivalence] {} { (x)P \equiv (y)P\{y/x\} \and y \not\in \freenames{P} }
\end{mathpar}

\begin{definition}
Then two processes, $P,Q$, are alpha-equivalent if $P = Q\{\vec{y}/\vec{x}\}$ for
some $\vec{x} \in \boundnames{Q},\vec{y} \in \boundnames{P}$, where $Q\{\vec{y}/\vec{x}\}$
denotes the capture-avoiding substitution of $\vec{y}$ for $\vec{x}$ in $Q$.
\end{definition}

\begin{definition}
  The {\em structural congruence} \cite{SangiorgiWalker} , $\equiv$,
  between processes is the least congruence containing
  alpha-equivalence, satisfying the abelian monoid laws
  (associativity, commutativity and $\pzero$ as identity) for parallel
  composition $|$ and for summation $+$.
\end{definition}

\subsection{Name equivalence}

We take name equivalence, written $\nameeq$, to be the smallest
equivalence relation generated by the following rules.

\begin{mathpar}
\inferrule*[lab=Quote-drop]
{ }
{ \quotep{@{x}} \nameeq x }

\inferrule*[lab=Struct-equiv]
{ P \scong Q }
{ \quotep{P} \nameeq \quotep{Q} }
\end{mathpar}

The astute reader will have noticed that the mutual recursion of names
and processes imposes a mutual recursion on alpha-equivalence and
structural equivalence via name-equivalence. Fortunately, all of this
works out pleasantly and we may calculate in the natural way, free of
concern. The reader interested in the details is referred to the
appendix \ref{appendix:rho_details}.

\subsection{Substitution}

We use $\Proc$ for the set of processes, $\QProc$ for the set of
names, and $\id{\{}\vec{y} / \vec{x} \id{\}}$ to denote partial maps,
$s : \QProc \rightarrow \QProc$. A map, $s$ lifts, uniquely, to a map
on process terms, $\widehat{s} : \Proc \rightarrow \Proc$ by the
following equations.

\begin{mathpar}
  (0) \psubstp{Q}{P} := 0 \\
  (R \juxtap S) \psubstp{Q}{P}
  :=    
  (R)\psubstp{Q}{P} \juxtap (S) \psubstp{Q}{P} \\
  (x?(y).R) \psubstp{Q}{P}    
  :=    
  (x)\substp{Q}{P} (z)\concat( (R \psubstn{z}{y}) \psubstp{Q}{P} ) \\
  (\lift{x}{R}) \psubstp{Q}{P}  
  :=
  \lift{(x)\substp{Q}{P}}{ R \psubstp{Q}{P} } \\
%   (\dropn{x})  \psubstp{Q}{P}       
%   := 
%   \left\{ 
%     \begin{array}{ccc} 
%       \dropn{\quotep{Q}} & & x \nameeq \quotep{P} \\
%       \dropn{x} & & otherwise \\
%     \end{array}
%   \right. 
  (\dropn{x})  \psubstp{Q}{P}       
  := 
  \left\{ 
    \begin{array}{ccc} 
      Q & & x \nameeq \quotep{P} \\
      \dropn{x} & & otherwise \\
    \end{array}
  \right.
\end{mathpar}
 

where

\begin{eqnarray}
  (x)\id{\{} \lpquote Q \rpquote / \lpquote P \rpquote \id{\}}            = 
  \left\{ 
    \begin{array}{ccc}
      \lpquote Q \rpquote & & x \nameeq \lpquote P \rpquote \\
      x & & otherwise \\
    \end{array}
  \right. \nonumber
\end{eqnarray}

and $z$ is chosen distinct from $\quotep{P}$, $\quotep{Q}$, the free
names in $Q$, and all the names in $R$. Our $\alpha$-equivalence will
be built in the standard way from this substitution.

\begin{remark}\label{rem:no_self_referential_names}
  One consequence of these definitions is that $\forall P. \quotep{P}
  \not\in \freenames{P}$.
\end{remark}

\subsection{ Dynamic quote: an example }

Anticipating something of what's to come, consider applying the
substitution, $\widehat{\id{\{}u / z \id{\}}}$, to the following pair
of processes, $\lift{w}{y!(z)}$ and $w[ \lpquote y!(z) \rpquote ]$.

\begin{eqnarray}
	\lift{w}{y!(z)}\widehat{\id{\{}u / z \id{\}}}
		& = &
		\lift{w}{y!(u)} \nonumber\\
	w[ \lpquote y!(z) \rpquote ] \widehat{ \id{\{}u / z \id{\}} }
		& = &
		w[ \lpquote y!(z) \rpquote ] \nonumber
\end{eqnarray}

Because the body of the process between quotes is impervious to
substitution, we get radically different answers. In fact, by
examining the first process in an input context,
e.g. $x?(z).\lift{w}{y!(z)}$, we see that the process under the lift
operator may be shaped by prefixed inputs binding a name inside it. In
this sense, the lift operator will be seen as a way to dynamically
construct processes before reifying them as names.

Finally equipped with these standard features we can present the
dynamics of the calculus.

\subsubsection{Operational semantics} 

Finally, we introduce the computational dynamics. What marks these
algebras as distinct from other more traditionally studied algebraic
structures, e.g. vector spaces or polynomial rings, is the manner in
which dynamics is captured. In traditional structures, dynamics is typically
expressed through morphisms between such structures, as in linear maps
between vector spaces or morphisms between rings. In algebras
associated with the semantics of computation, the dynamics is
expressed as part of the algebraic structure itself, through a
reduction reduction relation typically denoted by $\red$. Below, we
give a recursive presentation of this relation for the calculus used
in the encoding.

$\red \subseteq \pi \times \pi$
$\red : \pi \to \mathcal{P}(\pi)$

\begin{mathpar}
  \inferrule* [lab=Comm] { \textsf{match}( x_{src}, x_{trgt} ) } { x_{trgt}?(y)P \; | \; x_{src}!\langle {Q} \rangle \red P\{\quotep{Q}/y}\} }
  \and \\
  \inferrule* [lab=Par] {{P} \red {P}'} {{{P} | {Q}} \red {{P}' | {Q}}}
  \and
  \inferrule* [lab=Equiv]{{{P} \scong {P}'} \andalso {{P}' \red {Q}'} \andalso {{Q}' \scong {Q}}}{{P} \red {Q}}
\end{mathpar}

\begin{eqnarray*}
  match_{\equiv} (\quotep{P},\quotep{Q}) & := & P \equiv Q \\
  match_{\dagger}(\quotep{P},\quotep{Q}) & := & \forall R. P|Q \red^{*} R => R \red^{*} 0 \\
  match_{K}(\quotep{P},\quotep{Q}) & := & K \mbox{ for some context } K
\end{eqnarray*}

$u?(x)P | u!\langle Q \rangle \red P\{\quotep{Q}/x\}$

%We write $\wred$ for $\red^*$, and $P\red$ if $\exists Q $ such that $ P \red Q$.
We write $P\red$ if $\exists Q $ such that $ P \red Q$ and $P\not\red$, otherwise.

\section{Replication}

As mentioned before, it is known that replication (and hence
recursion) can be implemented in a higher-order process algebra
\cite{SangiorgiWalker}. As our first example of calculation with the
machinery thus far presented we give the construction explicitly in
the {\rhoc}.

\begin{eqnarray}
	D_{x} & := & \prefix{x}{y}{(\binpar{\outputp{x}{y}}{@{y}})} \nonumber\\
	\bangp_{x}{P} & := & \binpar{{x}!\langle{\binpar{D_{x}}{P}}\rangle}{D_{x}} \nonumber
\end{eqnarray}

\begin{eqnarray}
	\bangp_{x}{P} & & \nonumber\\
	=
	& {x}!\langle{(\prefix{x}{y}{(\outputp{x}{y} | @{y})) | P}}\rangle 
	      | \prefix{x}{y}{(\outputp{x}{y} | @{y})} & \nonumber\\
	\red
	& (\outputp{x}{y} | @{y})\substn{\quotep{(\prefix{x}{y}{(@{y} | \outputp{x}{y})) | P}}}{y} & \nonumber\\
	=
	& \outputp{x}{\quotep{(\prefix{x}{y}{(\outputp{x}{y} | @{y})) | P}}}
	  | {(\prefix{x}{y}{(\outputp{x}{y} | @{y})) | P}} & \nonumber\\
	\red
	& \ldots & \nonumber\\
	\red^*
	& P | P | \ldots & \nonumber
\end{eqnarray}

Of course, this encoding, as an implementation, runs away, unfolding
$\bangp{P}$ eagerly. A lazier and more implementable replication
operator, restricted to input-guarded processes, may be obtained as follows.

\begin{eqnarray}
\bangp{\prefix{u}{v}{P}} 
	:= 
	\binpar{\lift{x}{\prefix{u}{v}{(\binpar{D(x)}{P})}}}{D(x)} \nonumber
\end{eqnarray}

\begin{remark}
  Note that the lazier definition still does not deal with summation
  or mixed summation (i.e. sums over input and output). The reader is
  invited to construct definitions of replication that deal with these
  features. 

  Further, the definitions are parameterized in a name, $x$. Can you,
  gentle reader, make a definition that eliminates this parameter and
  guarantees no accidental interaction between the replication
  machinery and the process being replicated -- i.e. no accidental
  sharing of names used by the process to get its work done and the
  name(s) used by the replication to effect copying. This latter
  revision of the definition of replication is crucial to obtaining
  the expected identity $!!P \sim !P$.
\end{remark}

\begin{remark}\label{rem:paradoxical_combinator}
  The reader familiar with the lambda calculus will have noticed the
  similarity between $D$ and the paradoxical combinator.

  [Ed. note: the existence of this seems to suggest we have to be more
  restrictive on the set of processes and names we admit if we are to
  support no-cloning.]
\end{remark}

\subsubsection{Bisimulation}

The computational dynamics gives rise to another kind of equivalence,
the equivalence of computational behavior. As previously mentioned
this is typically captured \emph{via} some form of bisimulation.

% The notion we use in this paper is weak barbed bisimulation
% \cite{milner91polyadicpi}.

The notion we use in this paper is derived from weak barbed
bisimulation \cite{milner91polyadicpi}. 

\begin{definition}
An \emph{observation relation}, $\downarrow_{\mathcal N}$, over a set
of names, $\mathcal N$, is the smallest relation satisfying the rules
below.

\infrule[Out-barb]{y \in {\mathcal N}, \; x \nameeq y}
		  {\outputp{x}{v} \downarrow_{\mathcal N} x}
\infrule[Par-barb]{\mbox{$P\downarrow_{\mathcal N} x$ or $Q\downarrow_{\mathcal N} x$}}
		  {\binpar{P}{Q} \downarrow_{\mathcal N} x}

We write $P \Downarrow_{\mathcal N} x$ if there is $Q$ such that 
$P \wred Q$ and $Q \downarrow_{\mathcal N} x$.
\end{definition}

\begin{definition}
%\label{def.bbisim}
An  ${\mathcal N}$-\emph{barbed bisimulation} over a set of names, ${\mathcal N}$, is a symmetric binary relation 
${\mathcal S}_{\mathcal N}$ between agents such that $P\rel{S}_{\mathcal N}Q$ implies:
\begin{enumerate}
\item If $P \red P'$ then $Q \wred Q'$ and $P'\rel{S}_{\mathcal N} Q'$.
\item If $P\downarrow_{\mathcal N} x$, then $Q\Downarrow_{\mathcal N} x$.
\end{enumerate}
$P$ is ${\mathcal N}$-barbed bisimilar to $Q$, written
$P \wbbisim_{\mathcal N} Q$, if $P \rel{S}_{\mathcal N} Q$ for some ${\mathcal N}$-barbed bisimulation ${\mathcal S}_{\mathcal N}$.
\end{definition}

$\mathcal{R} \subseteq \pi \times \pi$

$P \mathcal{R} Q => \forall P'. P \red P' \Rightarrow \exists Q'. Q \red Q', P' \mathcal{R} Q'$

$P \vdash x \Rightarrow Q \vdash x$

\begin{mathpar}
  \inferrule*[lab=Out-barb]{x \nameeq y}{{y}!\langle{Q}\rangle \vdash x}
  \and
  \inferrule*[lab=Par-barb]{\mbox{$P\vdash x$ or $Q\vdash x$}}{\binpar{P}{Q} \vdash x}
\end{mathpar}

\subsubsection{Contexts}

One of the principle advantages of computational calculi like the
$\pi$-calculus is a well-defined notion of context,
contextual-equivalence and a correlation between
contextual-equivalence and notions of bisimulation. The notion of
context allows the decomposition of a process into (sub-)process and
its syntactic environment, its context. Thus, a context may be
thought of as a process with a ``hole'' (written $\Box$) in it. The
application of a context $M$ to a process $P$, written $M[P]$, is
tantamount to filling the hole in $M$ with $P$. In this paper we do
not need the full weight of this theory, but do make use of the notion
of context in the proof the main theorem. 

\begin{mathpar}
  \inferrule* [lab=summation] {} {{M_{M},M_{N}} \bc \Box \;|\; x.M_{A} \;|\; M_{M}+M_{N}}
  \and
  \inferrule* [lab=agent] {} {{M_{A}} \bc (\vec{x})M_{P} \;| \; \clift{P_0,\ldots,M_{P},\ldots,P_N}}
  \and \\
  \inferrule* [lab=process] {} {{M_{P}} \bc M_{N} \;| \;P|M_{P} }
\end{mathpar} 

\begin{mathpar}
  \inferrule* [lab=sychronization] {} {M_{N} \bc \Box \;|\; x?M_{F} \;|\; x!M_{C}}
  \and
  \inferrule* [lab=abstraction] {} {{M_{F}} \bc (x)M_{P} }
  \and
  \inferrule* [lab=concretion] {} {{M_{C}} \bc \langle M_{P} \rangle }
  \and \\
  \inferrule* [lab=process] {} {{M_{P}} \bc M_{N} \;| \;P|M_{P} }
\end{mathpar}

\begin{definition}[contextual application] Given a context $M$, and
  process $P$, we define the \emph{contextual application}, $M[P] :=
  M\{P/\Box\}$. That is, the contextual application of M to P is the
  substitution of $P$ for $\Box$ in $M$.
\end{definition}

$\meaningof{-} : L \to \mathcal{P}(\pi)$

\begin{mathpar}
  \inferrule* [lab=collection] {} {\meaningof{true} = \pi, \and \meaningof{~E} = \pi \setminus \meaningof{E}, \and \meaningof{E_{1} \& E_{2}} = \meaningof{E_{1}} \cap \meaningof{E_{2}}}
\end{mathpar}

\begin{mathpar}
  \inferrule* [lab=structure] {} {\meaningof{0} = \{ P \in \pi | P \equiv 0 \}, \and \\ \meaningof{E_1 | E_2} = \{ P \in \pi | P \equiv P_{1} | P_{2}, P_{1} \in \meaningof{E_{1}}, P_{2} \in \meaningof{E_2}\} }
\end{mathpar}

\begin{mathpar}
 \inferrule* [lab=behavior] {} {\meaningof{\langle a?b \rangle E} = \{ P \in \pi | P \equiv Q | u?(y)P', \\ \and \\\\ \and \\ \;\;\; u \in \meaningof{a}, \forall z.P'\{z/y\} \in \meaningof{E\{z/b\}}\}, \and \\ \meaningof{a!E} = \{ P \in \pi | P \equiv Q | x!\langle P' \rangle, x \in \meaningof{a} P' \in \meaningof{E}\} }
\end{mathpar}

\begin{mathpar}
 \inferrule* [lab=nominal] {} {\meaningof{\quotep{E}} = \{ \quotep{P} \in \quotep{\pi} | P \in \meaningof{E} \}, \and \meaningof{\quotep{P}} = \{ \quotep{Q} \in \quotep{\pi} | P \equiv Q \} \and \\ \meaningof{@\quotep{E}} = \{ P \in \pi | P \equiv @x, x \in \meaningof{E} \}}
\end{mathpar}

\begin{eqnarray*}
  \\
  \meaningof{-} : TS \to ST
\end{eqnarray*}

\begin{eqnarray*}
  \\
  L : TS \to ST
\end{eqnarray*}

\begin{eqnarray*}
  \\
  P \models E \iff P \in \meaningof{E}
\end{eqnarray*}

\begin{eqnarray*}
  P \approx_{L} Q \iff \forall E \in L. P \models E \iff Q \models E
\end{eqnarray*}

\begin{eqnarray*}
  P \approx_{K} Q
\end{eqnarray*}

\begin{eqnarray*}
  P \approx Q
\end{eqnarray*}

$\approx_{K} = \approx = \approx_{L}$

\subsubsection{Contextual duality}

Note that contexts extend the quotation operation to a family of
operations from processes to names. Given a context, $M$, we can
define a \emph{nominal context}, $\quotep{M}$ by $\quotep{M}[P] :=
\quotep{M[P]}$. To foreshadow what is to come we observe that these
operations enjoy a duality with processes very much like the duality
between vectors and maps from vectors to scalars.

Further, because the calculus is essentially higher-order, we have a
correspondence between contexts and processes. More specifically,
given a name $x$ and a context $M$ we can construct $M^{*}_{x}$ such
that 

\begin{mathpar}
  M^{*}_{x} | \lift{x}{P} \red M[P]
\end{mathpar}

namely,

\begin{mathpar}
  M^{*}_{x} := x?(u).M[\dropn{u}]
\end{mathpar}

The dependence of $M^{*}_{x}$ on a name makes it an abstraction, 

\begin{mathpar}
  M^{*} := (x)x?(u).M[\dropn{u}]
\end{mathpar}

\subsection{Additional notation}

It will sometimes be convenient to denote the process a name
quotes. We already have the notation $x = \quotep{P}$, but it will be
convenient to introduce an alternate notation, $\procn{x}$, when we
want to emphasize the connection to the use of the name. Note that, by
virtue of name equivalence, $\quotep{\procn{x}} \nameeq x$; so, the
notation is consistent with previous definitions.

Further, because names have structure it is possible to effect
substitutions on the basis of that structure. This means we need to
upgrade our notation for substitutions, which we accomplish by
adapting comprehension notation. Thus,

\begin{mathpar}
  P\{ y / x : x \in S \}
\end{mathpar}

is interpreted to mean the process derived from P by replacing (in a
capture-avoiding manner) each occurrence of $x$ in $S$ by $y$. For example,

\begin{mathpar}
  P\{ \quotep{\procn{x}|\procn{x}} / x : x \in \freenames{P} \}
\end{mathpar}

will replace each (occurrence) of a free name $x$ in $P$ by
$\quotep{\procn{x}|\procn{x}}$.

Also, we will avail ourselves of the notation $x^{L}$ and $x^{R}$ to
denote injections of a name into disjoint copies of the name
space. There are numerous ways to accomplish this. One example can be
found in \cite{MeredithR05}. This notation overloads to vectors of
names: $\vec{x}^{\pi} := (x_{i}^{\pi} \; : \; 0 \leq i < |\vec{x}| )$ where $\pi \in \{L,R\}$.

We also use $P^{\Box} := P|\Box$.

In \cite{MeredithR05} an interpretation of the new operator is
given. It turns out that there are several possible interpretations
all enjoying the requisite algebraic properties of the operator (see
\cite{milner91polyadicpi}). We will therefore make liberal use of
$(\nu\; \vec{x})P$.

% subsection the_syntax_and_semantics_of_the_notation_system (end)   

\input{qm2pi.qmops} 

\input{qm2pi.sterngerlach} 

\input{qm2pi.metric} 

% section concurrent_process_calculi (end)

%\input{qm2pi.proofsketch}

% section proof sketch (end)

%\input{qm2pi.slviaknots} 

% section spatial logic via knots (end)

\input{qm2pi.conclusion}

% section conclusion (end)

%\input{qm2pi.dtcodes} 

% section wiring algorithm (end)

\input{qm2pi.ack} 

% section acknowledgments (end)

\newpage


\bibliographystyle{plain}   
\bibliography{../../biblios/main.bib}

\input{qm2pi.rhodetails}

\end{document}

 

%\documentclass[12pt]{llncs}
%\documentclass{jktr}

\usepackage[pdftex]{hyperref}                   
\usepackage {listings}
\usepackage {mathpartir}
\usepackage{bcprules}
%\usepackage{listings}
                       
\usepackage{graphicx} 
%\usepackage[margins=2.5cm,nohead,nofoot]{geometry}
%\usepackage{geometry}
\usepackage{amsfonts}
\usepackage{amstext}
\usepackage{latexsym}
\usepackage{amssymb}
\usepackage{color}


%\include{myPreamble}
\include{qm2pi.local} 

%\ifpdf
%\usepackage[pdftex]{graphicx}
%\else
%\usepackage{graphicx}
%\fi

 % \ifpdf
%  \usepackage{pdfsync}
%  \if


%\title{Brief Article}
%\author{David F. Snyder}
%\author{L.G. Meredith}

%\address{Dept. of Math., Texas State University--San Marcos, San Marcos, TX 78666}
       
\pagestyle{empty}


\begin{document}

\lstset{language=[Objective]Caml,frame=shadowbox}

\input{qm2pi.front}

% section front matter (end)

\input{qm2pi.intro} 
 
% section introduction (end)

% \input{qm2pi.knotations} 

% section notation (end)

\input{qm2pi.process.calculi} 

% section concurrent_process_calculi_and_spatial_logics_ (end)
    
%\input{qm2pi.knots2pi} 

%\input{qm2pi.trefoil} 

%\input{qm2pi.mainthm} 

% subsection basic_interpretation (end)

%\input{qm2pi.rho.presentation} 
\subsection{The syntax and semantics of the notation system}\label{sub:the_syntax_and_semantics_of_the_notation_system} % (fold)

We now summarize a technical presentation of the calculus that
embodies our theory of dynamics. The typical presentation of such a
calculus follows the style of giving generators and relations on
them. The grammar, below, describing term constructors, freely
generates the set of processes, $\Proc$. This set is then quotiented
by a relation known as structural congruence and it is over this set
that the notion of dynamics is expressed. This presentation is
essentially that of \cite{MeredithR05} with the addition of
polyadicity and summation. For readability we have relegated some of
the technical subtleties to an appendix.

\subsubsection{Process grammar}\label{subsub:process_grammar}

\begin{mathpar}
  \inferrule* [lab=synchronization] {} {{M} \bc \pzero \;|\; x?F \;|\; x!C }
  \and
  \inferrule* [lab=abstraction] {} {{F} \bc (x)P}
  \and
  \inferrule* [lab=concretion] {} {{C} \bc \langle Q \rangle}
  \and
  \inferrule* [lab=process] {} {{P,Q} \bc M \;| \;P|Q \;|\; @{x}}
  \and
  \inferrule* [lab=name] {} {{x} \bc \quotep{P}}
\end{mathpar} 

Note that $\vec{x}$ (resp. $\vec{P}$) denotes a vector of names
(resp. processes) of length $|\vec{x}|$ (resp. $|\vec{P}|$). We adopt
the following useful abbreviations.

\begin{mathpar}
   x?(\vec{y}).P := x.(\vec{y})P \and  x\clift{\vec{P}} := x.\clift{\vec{P}}
   \and x!(y) := \lift{x}{\dropn{y}}
   \and \Pi_{i=0}^{n-1}P_i := P_0 | \ldots | P_{n-1}
\end{mathpar}

\subsubsection{Structural congruence}

\paragraph{Free and bound names and alpha-equivalence.} At the
core of structural equivalence is alpha-equivalence which identifies
process that are the same up to a change of variable. Formally, we
recognize the distinction between free and bound names. The free names
of a process, $\freenames{P}$, may be calculated recursively as
follows:

\begin{mathpar}
\freenames{\pzero} := \emptyset
  \and \\
  \freenames{x?(y).P} := \{ x \} \cup (\freenames{P} \setminus \{ y \})
  \and 
  \freenames{x!\langle P \rangle} := \{ x \} \cup \{ P \} 
  \and \\
  \freenames{P|Q} := \freenames{P} \cup \freenames{Q}
  \and \\
  \freenames{@{x}} := \{ x \}
\end{mathpar}

$\pi$
$\quotep{\pi}$

$\freenames{-} : \pi \to \mathcal{P}(\quotep{\pi})$

\begin{eqnarray*}
  \freenames{\pzero} & := & \emptyset \\
  \freenames{x?(y).P} & := & \{ x \} \cup (\freenames{P} \setminus \{ y \}) \\
  \freenames{x!\langle P \rangle} & := & \{ x \} \cup \{ P \} \\
  \freenames{P|Q} & := & \freenames{P} \cup \freenames{Q} \\
  \freenames{\dropn{x}} & := & \{ x \}
\end{eqnarray*}

The bound names of a process, $\boundnames{P}$, are those names occurring in $P$
that are not free. For example, in $x?(y).0$, the name $x$ is free, while $y$ is bound.

\begin{mathpar}
  \inferrule* [lab=monoidal-laws] {} { P|Q \equiv Q|P \and P|0 \equiv P \and P|(Q|R) \equiv (P|Q)|R }
\end{mathpar}

\begin{mathpar}
  \inferrule* [lab=alpha-equivalence] {} { (x)P \equiv (y)P\{y/x\} \and y \not\in \freenames{P} }
\end{mathpar}

\begin{definition}
Then two processes, $P,Q$, are alpha-equivalent if $P = Q\{\vec{y}/\vec{x}\}$ for
some $\vec{x} \in \boundnames{Q},\vec{y} \in \boundnames{P}$, where $Q\{\vec{y}/\vec{x}\}$
denotes the capture-avoiding substitution of $\vec{y}$ for $\vec{x}$ in $Q$.
\end{definition}

\begin{definition}
  The {\em structural congruence} \cite{SangiorgiWalker} , $\equiv$,
  between processes is the least congruence containing
  alpha-equivalence, satisfying the abelian monoid laws
  (associativity, commutativity and $\pzero$ as identity) for parallel
  composition $|$ and for summation $+$.
\end{definition}

\subsection{Name equivalence}

We take name equivalence, written $\nameeq$, to be the smallest
equivalence relation generated by the following rules.

\begin{mathpar}
\inferrule*[lab=Quote-drop]
{ }
{ \quotep{@{x}} \nameeq x }

\inferrule*[lab=Struct-equiv]
{ P \scong Q }
{ \quotep{P} \nameeq \quotep{Q} }
\end{mathpar}

The astute reader will have noticed that the mutual recursion of names
and processes imposes a mutual recursion on alpha-equivalence and
structural equivalence via name-equivalence. Fortunately, all of this
works out pleasantly and we may calculate in the natural way, free of
concern. The reader interested in the details is referred to the
appendix \ref{appendix:rho_details}.

\subsection{Substitution}

We use $\Proc$ for the set of processes, $\QProc$ for the set of
names, and $\id{\{}\vec{y} / \vec{x} \id{\}}$ to denote partial maps,
$s : \QProc \rightarrow \QProc$. A map, $s$ lifts, uniquely, to a map
on process terms, $\widehat{s} : \Proc \rightarrow \Proc$ by the
following equations.

\begin{mathpar}
  (0) \psubstp{Q}{P} := 0 \\
  (R \juxtap S) \psubstp{Q}{P}
  :=    
  (R)\psubstp{Q}{P} \juxtap (S) \psubstp{Q}{P} \\
  (x?(y).R) \psubstp{Q}{P}    
  :=    
  (x)\substp{Q}{P} (z)\concat( (R \psubstn{z}{y}) \psubstp{Q}{P} ) \\
  (\lift{x}{R}) \psubstp{Q}{P}  
  :=
  \lift{(x)\substp{Q}{P}}{ R \psubstp{Q}{P} } \\
%   (\dropn{x})  \psubstp{Q}{P}       
%   := 
%   \left\{ 
%     \begin{array}{ccc} 
%       \dropn{\quotep{Q}} & & x \nameeq \quotep{P} \\
%       \dropn{x} & & otherwise \\
%     \end{array}
%   \right. 
  (\dropn{x})  \psubstp{Q}{P}       
  := 
  \left\{ 
    \begin{array}{ccc} 
      Q & & x \nameeq \quotep{P} \\
      \dropn{x} & & otherwise \\
    \end{array}
  \right.
\end{mathpar}
 

where

\begin{eqnarray}
  (x)\id{\{} \lpquote Q \rpquote / \lpquote P \rpquote \id{\}}            = 
  \left\{ 
    \begin{array}{ccc}
      \lpquote Q \rpquote & & x \nameeq \lpquote P \rpquote \\
      x & & otherwise \\
    \end{array}
  \right. \nonumber
\end{eqnarray}

and $z$ is chosen distinct from $\quotep{P}$, $\quotep{Q}$, the free
names in $Q$, and all the names in $R$. Our $\alpha$-equivalence will
be built in the standard way from this substitution.

\begin{remark}\label{rem:no_self_referential_names}
  One consequence of these definitions is that $\forall P. \quotep{P}
  \not\in \freenames{P}$.
\end{remark}

\subsection{ Dynamic quote: an example }

Anticipating something of what's to come, consider applying the
substitution, $\widehat{\id{\{}u / z \id{\}}}$, to the following pair
of processes, $\lift{w}{y!(z)}$ and $w[ \lpquote y!(z) \rpquote ]$.

\begin{eqnarray}
	\lift{w}{y!(z)}\widehat{\id{\{}u / z \id{\}}}
		& = &
		\lift{w}{y!(u)} \nonumber\\
	w[ \lpquote y!(z) \rpquote ] \widehat{ \id{\{}u / z \id{\}} }
		& = &
		w[ \lpquote y!(z) \rpquote ] \nonumber
\end{eqnarray}

Because the body of the process between quotes is impervious to
substitution, we get radically different answers. In fact, by
examining the first process in an input context,
e.g. $x?(z).\lift{w}{y!(z)}$, we see that the process under the lift
operator may be shaped by prefixed inputs binding a name inside it. In
this sense, the lift operator will be seen as a way to dynamically
construct processes before reifying them as names.

Finally equipped with these standard features we can present the
dynamics of the calculus.

\subsubsection{Operational semantics} 

Finally, we introduce the computational dynamics. What marks these
algebras as distinct from other more traditionally studied algebraic
structures, e.g. vector spaces or polynomial rings, is the manner in
which dynamics is captured. In traditional structures, dynamics is typically
expressed through morphisms between such structures, as in linear maps
between vector spaces or morphisms between rings. In algebras
associated with the semantics of computation, the dynamics is
expressed as part of the algebraic structure itself, through a
reduction reduction relation typically denoted by $\red$. Below, we
give a recursive presentation of this relation for the calculus used
in the encoding.

$\red \subseteq \pi \times \pi$
$\red : \pi \to \mathcal{P}(\pi)$

\begin{mathpar}
  \inferrule* [lab=Comm] { \textsf{match}( x_{src}, x_{trgt} ) } { x_{trgt}?(y)P \; | \; x_{src}!\langle {Q} \rangle \red P\{\quotep{Q}/y}\} }
  \and \\
  \inferrule* [lab=Par] {{P} \red {P}'} {{{P} | {Q}} \red {{P}' | {Q}}}
  \and
  \inferrule* [lab=Equiv]{{{P} \scong {P}'} \andalso {{P}' \red {Q}'} \andalso {{Q}' \scong {Q}}}{{P} \red {Q}}
\end{mathpar}

\begin{eqnarray*}
  match_{\equiv} (\quotep{P},\quotep{Q}) & := & P \equiv Q \\
  match_{\dagger}(\quotep{P},\quotep{Q}) & := & \forall R. P|Q \red^{*} R => R \red^{*} 0 \\
  match_{K}(\quotep{P},\quotep{Q}) & := & K \mbox{ for some context } K
\end{eqnarray*}

$u?(x)P | u!\langle Q \rangle \red P\{\quotep{Q}/x\}$

%We write $\wred$ for $\red^*$, and $P\red$ if $\exists Q $ such that $ P \red Q$.
We write $P\red$ if $\exists Q $ such that $ P \red Q$ and $P\not\red$, otherwise.

\section{Replication}

As mentioned before, it is known that replication (and hence
recursion) can be implemented in a higher-order process algebra
\cite{SangiorgiWalker}. As our first example of calculation with the
machinery thus far presented we give the construction explicitly in
the {\rhoc}.

\begin{eqnarray}
	D_{x} & := & \prefix{x}{y}{(\binpar{\outputp{x}{y}}{@{y}})} \nonumber\\
	\bangp_{x}{P} & := & \binpar{{x}!\langle{\binpar{D_{x}}{P}}\rangle}{D_{x}} \nonumber
\end{eqnarray}

\begin{eqnarray}
	\bangp_{x}{P} & & \nonumber\\
	=
	& {x}!\langle{(\prefix{x}{y}{(\outputp{x}{y} | @{y})) | P}}\rangle 
	      | \prefix{x}{y}{(\outputp{x}{y} | @{y})} & \nonumber\\
	\red
	& (\outputp{x}{y} | @{y})\substn{\quotep{(\prefix{x}{y}{(@{y} | \outputp{x}{y})) | P}}}{y} & \nonumber\\
	=
	& \outputp{x}{\quotep{(\prefix{x}{y}{(\outputp{x}{y} | @{y})) | P}}}
	  | {(\prefix{x}{y}{(\outputp{x}{y} | @{y})) | P}} & \nonumber\\
	\red
	& \ldots & \nonumber\\
	\red^*
	& P | P | \ldots & \nonumber
\end{eqnarray}

Of course, this encoding, as an implementation, runs away, unfolding
$\bangp{P}$ eagerly. A lazier and more implementable replication
operator, restricted to input-guarded processes, may be obtained as follows.

\begin{eqnarray}
\bangp{\prefix{u}{v}{P}} 
	:= 
	\binpar{\lift{x}{\prefix{u}{v}{(\binpar{D(x)}{P})}}}{D(x)} \nonumber
\end{eqnarray}

\begin{remark}
  Note that the lazier definition still does not deal with summation
  or mixed summation (i.e. sums over input and output). The reader is
  invited to construct definitions of replication that deal with these
  features. 

  Further, the definitions are parameterized in a name, $x$. Can you,
  gentle reader, make a definition that eliminates this parameter and
  guarantees no accidental interaction between the replication
  machinery and the process being replicated -- i.e. no accidental
  sharing of names used by the process to get its work done and the
  name(s) used by the replication to effect copying. This latter
  revision of the definition of replication is crucial to obtaining
  the expected identity $!!P \sim !P$.
\end{remark}

\begin{remark}\label{rem:paradoxical_combinator}
  The reader familiar with the lambda calculus will have noticed the
  similarity between $D$ and the paradoxical combinator.

  [Ed. note: the existence of this seems to suggest we have to be more
  restrictive on the set of processes and names we admit if we are to
  support no-cloning.]
\end{remark}

\subsubsection{Bisimulation}

The computational dynamics gives rise to another kind of equivalence,
the equivalence of computational behavior. As previously mentioned
this is typically captured \emph{via} some form of bisimulation.

% The notion we use in this paper is weak barbed bisimulation
% \cite{milner91polyadicpi}.

The notion we use in this paper is derived from weak barbed
bisimulation \cite{milner91polyadicpi}. 

\begin{definition}
An \emph{observation relation}, $\downarrow_{\mathcal N}$, over a set
of names, $\mathcal N$, is the smallest relation satisfying the rules
below.

\infrule[Out-barb]{y \in {\mathcal N}, \; x \nameeq y}
		  {\outputp{x}{v} \downarrow_{\mathcal N} x}
\infrule[Par-barb]{\mbox{$P\downarrow_{\mathcal N} x$ or $Q\downarrow_{\mathcal N} x$}}
		  {\binpar{P}{Q} \downarrow_{\mathcal N} x}

We write $P \Downarrow_{\mathcal N} x$ if there is $Q$ such that 
$P \wred Q$ and $Q \downarrow_{\mathcal N} x$.
\end{definition}

\begin{definition}
%\label{def.bbisim}
An  ${\mathcal N}$-\emph{barbed bisimulation} over a set of names, ${\mathcal N}$, is a symmetric binary relation 
${\mathcal S}_{\mathcal N}$ between agents such that $P\rel{S}_{\mathcal N}Q$ implies:
\begin{enumerate}
\item If $P \red P'$ then $Q \wred Q'$ and $P'\rel{S}_{\mathcal N} Q'$.
\item If $P\downarrow_{\mathcal N} x$, then $Q\Downarrow_{\mathcal N} x$.
\end{enumerate}
$P$ is ${\mathcal N}$-barbed bisimilar to $Q$, written
$P \wbbisim_{\mathcal N} Q$, if $P \rel{S}_{\mathcal N} Q$ for some ${\mathcal N}$-barbed bisimulation ${\mathcal S}_{\mathcal N}$.
\end{definition}

$\mathcal{R} \subseteq \pi \times \pi$

$P \mathcal{R} Q => \forall P'. P \red P' \Rightarrow \exists Q'. Q \red Q', P' \mathcal{R} Q'$

$P \vdash x \Rightarrow Q \vdash x$

\begin{mathpar}
  \inferrule*[lab=Out-barb]{x \nameeq y}{{y}!\langle{Q}\rangle \vdash x}
  \and
  \inferrule*[lab=Par-barb]{\mbox{$P\vdash x$ or $Q\vdash x$}}{\binpar{P}{Q} \vdash x}
\end{mathpar}

\subsubsection{Contexts}

One of the principle advantages of computational calculi like the
$\pi$-calculus is a well-defined notion of context,
contextual-equivalence and a correlation between
contextual-equivalence and notions of bisimulation. The notion of
context allows the decomposition of a process into (sub-)process and
its syntactic environment, its context. Thus, a context may be
thought of as a process with a ``hole'' (written $\Box$) in it. The
application of a context $M$ to a process $P$, written $M[P]$, is
tantamount to filling the hole in $M$ with $P$. In this paper we do
not need the full weight of this theory, but do make use of the notion
of context in the proof the main theorem. 

\begin{mathpar}
  \inferrule* [lab=summation] {} {{M_{M},M_{N}} \bc \Box \;|\; x.M_{A} \;|\; M_{M}+M_{N}}
  \and
  \inferrule* [lab=agent] {} {{M_{A}} \bc (\vec{x})M_{P} \;| \; \clift{P_0,\ldots,M_{P},\ldots,P_N}}
  \and \\
  \inferrule* [lab=process] {} {{M_{P}} \bc M_{N} \;| \;P|M_{P} }
\end{mathpar} 

\begin{mathpar}
  \inferrule* [lab=sychronization] {} {M_{N} \bc \Box \;|\; x?M_{F} \;|\; x!M_{C}}
  \and
  \inferrule* [lab=abstraction] {} {{M_{F}} \bc (x)M_{P} }
  \and
  \inferrule* [lab=concretion] {} {{M_{C}} \bc \langle M_{P} \rangle }
  \and \\
  \inferrule* [lab=process] {} {{M_{P}} \bc M_{N} \;| \;P|M_{P} }
\end{mathpar}

\begin{definition}[contextual application] Given a context $M$, and
  process $P$, we define the \emph{contextual application}, $M[P] :=
  M\{P/\Box\}$. That is, the contextual application of M to P is the
  substitution of $P$ for $\Box$ in $M$.
\end{definition}

$\meaningof{-} : L \to \mathcal{P}(\pi)$

\begin{mathpar}
  \inferrule* [lab=collection] {} {\meaningof{true} = \pi, \and \meaningof{~E} = \pi \setminus \meaningof{E}, \and \meaningof{E_{1} \& E_{2}} = \meaningof{E_{1}} \cap \meaningof{E_{2}}}
\end{mathpar}

\begin{mathpar}
  \inferrule* [lab=structure] {} {\meaningof{0} = \{ P \in \pi | P \equiv 0 \}, \and \\ \meaningof{E_1 | E_2} = \{ P \in \pi | P \equiv P_{1} | P_{2}, P_{1} \in \meaningof{E_{1}}, P_{2} \in \meaningof{E_2}\} }
\end{mathpar}

\begin{mathpar}
 \inferrule* [lab=behavior] {} {\meaningof{\langle a?b \rangle E} = \{ P \in \pi | P \equiv Q | u?(y)P', \\ \and \\\\ \and \\ \;\;\; u \in \meaningof{a}, \forall z.P'\{z/y\} \in \meaningof{E\{z/b\}}\}, \and \\ \meaningof{a!E} = \{ P \in \pi | P \equiv Q | x!\langle P' \rangle, x \in \meaningof{a} P' \in \meaningof{E}\} }
\end{mathpar}

\begin{mathpar}
 \inferrule* [lab=nominal] {} {\meaningof{\quotep{E}} = \{ \quotep{P} \in \quotep{\pi} | P \in \meaningof{E} \}, \and \meaningof{\quotep{P}} = \{ \quotep{Q} \in \quotep{\pi} | P \equiv Q \} \and \\ \meaningof{@\quotep{E}} = \{ P \in \pi | P \equiv @x, x \in \meaningof{E} \}}
\end{mathpar}

\begin{eqnarray*}
  \\
  \meaningof{-} : TS \to ST
\end{eqnarray*}

\begin{eqnarray*}
  \\
  L : TS \to ST
\end{eqnarray*}

\begin{eqnarray*}
  \\
  P \models E \iff P \in \meaningof{E}
\end{eqnarray*}

\begin{eqnarray*}
  P \approx_{L} Q \iff \forall E \in L. P \models E \iff Q \models E
\end{eqnarray*}

\begin{eqnarray*}
  P \approx_{K} Q
\end{eqnarray*}

\begin{eqnarray*}
  P \approx Q
\end{eqnarray*}

$\approx_{K} = \approx = \approx_{L}$

\subsubsection{Contextual duality}

Note that contexts extend the quotation operation to a family of
operations from processes to names. Given a context, $M$, we can
define a \emph{nominal context}, $\quotep{M}$ by $\quotep{M}[P] :=
\quotep{M[P]}$. To foreshadow what is to come we observe that these
operations enjoy a duality with processes very much like the duality
between vectors and maps from vectors to scalars.

Further, because the calculus is essentially higher-order, we have a
correspondence between contexts and processes. More specifically,
given a name $x$ and a context $M$ we can construct $M^{*}_{x}$ such
that 

\begin{mathpar}
  M^{*}_{x} | \lift{x}{P} \red M[P]
\end{mathpar}

namely,

\begin{mathpar}
  M^{*}_{x} := x?(u).M[\dropn{u}]
\end{mathpar}

The dependence of $M^{*}_{x}$ on a name makes it an abstraction, 

\begin{mathpar}
  M^{*} := (x)x?(u).M[\dropn{u}]
\end{mathpar}

\subsection{Additional notation}

It will sometimes be convenient to denote the process a name
quotes. We already have the notation $x = \quotep{P}$, but it will be
convenient to introduce an alternate notation, $\procn{x}$, when we
want to emphasize the connection to the use of the name. Note that, by
virtue of name equivalence, $\quotep{\procn{x}} \nameeq x$; so, the
notation is consistent with previous definitions.

Further, because names have structure it is possible to effect
substitutions on the basis of that structure. This means we need to
upgrade our notation for substitutions, which we accomplish by
adapting comprehension notation. Thus,

\begin{mathpar}
  P\{ y / x : x \in S \}
\end{mathpar}

is interpreted to mean the process derived from P by replacing (in a
capture-avoiding manner) each occurrence of $x$ in $S$ by $y$. For example,

\begin{mathpar}
  P\{ \quotep{\procn{x}|\procn{x}} / x : x \in \freenames{P} \}
\end{mathpar}

will replace each (occurrence) of a free name $x$ in $P$ by
$\quotep{\procn{x}|\procn{x}}$.

Also, we will avail ourselves of the notation $x^{L}$ and $x^{R}$ to
denote injections of a name into disjoint copies of the name
space. There are numerous ways to accomplish this. One example can be
found in \cite{MeredithR05}. This notation overloads to vectors of
names: $\vec{x}^{\pi} := (x_{i}^{\pi} \; : \; 0 \leq i < |\vec{x}| )$ where $\pi \in \{L,R\}$.

We also use $P^{\Box} := P|\Box$.

In \cite{MeredithR05} an interpretation of the new operator is
given. It turns out that there are several possible interpretations
all enjoying the requisite algebraic properties of the operator (see
\cite{milner91polyadicpi}). We will therefore make liberal use of
$(\nu\; \vec{x})P$.

% subsection the_syntax_and_semantics_of_the_notation_system (end)   

\input{qm2pi.qmops} 

\input{qm2pi.sterngerlach} 

\input{qm2pi.metric} 

% section concurrent_process_calculi (end)

%\input{qm2pi.proofsketch}

% section proof sketch (end)

%\input{qm2pi.slviaknots} 

% section spatial logic via knots (end)

\input{qm2pi.conclusion}

% section conclusion (end)

%\input{qm2pi.dtcodes} 

% section wiring algorithm (end)

\input{qm2pi.ack} 

% section acknowledgments (end)

\newpage


\bibliographystyle{plain}   
\bibliography{../../biblios/main.bib}

\input{qm2pi.rhodetails}

\end{document}

 

% subsection basic_interpretation (end)

%\input{qm2pi.rho.presentation} 
\subsection{The syntax and semantics of the notation system}\label{sub:the_syntax_and_semantics_of_the_notation_system} % (fold)

We now summarize a technical presentation of the calculus that
embodies our theory of dynamics. The typical presentation of such a
calculus follows the style of giving generators and relations on
them. The grammar, below, describing term constructors, freely
generates the set of processes, $\Proc$. This set is then quotiented
by a relation known as structural congruence and it is over this set
that the notion of dynamics is expressed. This presentation is
essentially that of \cite{MeredithR05} with the addition of
polyadicity and summation. For readability we have relegated some of
the technical subtleties to an appendix.

\subsubsection{Process grammar}\label{subsub:process_grammar}

\begin{mathpar}
  \inferrule* [lab=synchronization] {} {{M} \bc \pzero \;|\; x?F \;|\; x!C }
  \and
  \inferrule* [lab=abstraction] {} {{F} \bc (x)P}
  \and
  \inferrule* [lab=concretion] {} {{C} \bc \langle Q \rangle}
  \and
  \inferrule* [lab=process] {} {{P,Q} \bc M \;| \;P|Q \;|\; @{x}}
  \and
  \inferrule* [lab=name] {} {{x} \bc \quotep{P}}
\end{mathpar} 

Note that $\vec{x}$ (resp. $\vec{P}$) denotes a vector of names
(resp. processes) of length $|\vec{x}|$ (resp. $|\vec{P}|$). We adopt
the following useful abbreviations.

\begin{mathpar}
   x?(\vec{y}).P := x.(\vec{y})P \and  x\clift{\vec{P}} := x.\clift{\vec{P}}
   \and x!(y) := \lift{x}{\dropn{y}}
   \and \Pi_{i=0}^{n-1}P_i := P_0 | \ldots | P_{n-1}
\end{mathpar}

\subsubsection{Structural congruence}

\paragraph{Free and bound names and alpha-equivalence.} At the
core of structural equivalence is alpha-equivalence which identifies
process that are the same up to a change of variable. Formally, we
recognize the distinction between free and bound names. The free names
of a process, $\freenames{P}$, may be calculated recursively as
follows:

\begin{mathpar}
\freenames{\pzero} := \emptyset
  \and \\
  \freenames{x?(y).P} := \{ x \} \cup (\freenames{P} \setminus \{ y \})
  \and 
  \freenames{x!\langle P \rangle} := \{ x \} \cup \{ P \} 
  \and \\
  \freenames{P|Q} := \freenames{P} \cup \freenames{Q}
  \and \\
  \freenames{@{x}} := \{ x \}
\end{mathpar}

$\pi$
$\quotep{\pi}$

$\freenames{-} : \pi \to \mathcal{P}(\quotep{\pi})$

\begin{eqnarray*}
  \freenames{\pzero} & := & \emptyset \\
  \freenames{x?(y).P} & := & \{ x \} \cup (\freenames{P} \setminus \{ y \}) \\
  \freenames{x!\langle P \rangle} & := & \{ x \} \cup \{ P \} \\
  \freenames{P|Q} & := & \freenames{P} \cup \freenames{Q} \\
  \freenames{\dropn{x}} & := & \{ x \}
\end{eqnarray*}

The bound names of a process, $\boundnames{P}$, are those names occurring in $P$
that are not free. For example, in $x?(y).0$, the name $x$ is free, while $y$ is bound.

\begin{mathpar}
  \inferrule* [lab=monoidal-laws] {} { P|Q \equiv Q|P \and P|0 \equiv P \and P|(Q|R) \equiv (P|Q)|R }
\end{mathpar}

\begin{mathpar}
  \inferrule* [lab=alpha-equivalence] {} { (x)P \equiv (y)P\{y/x\} \and y \not\in \freenames{P} }
\end{mathpar}

\begin{definition}
Then two processes, $P,Q$, are alpha-equivalent if $P = Q\{\vec{y}/\vec{x}\}$ for
some $\vec{x} \in \boundnames{Q},\vec{y} \in \boundnames{P}$, where $Q\{\vec{y}/\vec{x}\}$
denotes the capture-avoiding substitution of $\vec{y}$ for $\vec{x}$ in $Q$.
\end{definition}

\begin{definition}
  The {\em structural congruence} \cite{SangiorgiWalker} , $\equiv$,
  between processes is the least congruence containing
  alpha-equivalence, satisfying the abelian monoid laws
  (associativity, commutativity and $\pzero$ as identity) for parallel
  composition $|$ and for summation $+$.
\end{definition}

\subsection{Name equivalence}

We take name equivalence, written $\nameeq$, to be the smallest
equivalence relation generated by the following rules.

\begin{mathpar}
\inferrule*[lab=Quote-drop]
{ }
{ \quotep{@{x}} \nameeq x }

\inferrule*[lab=Struct-equiv]
{ P \scong Q }
{ \quotep{P} \nameeq \quotep{Q} }
\end{mathpar}

The astute reader will have noticed that the mutual recursion of names
and processes imposes a mutual recursion on alpha-equivalence and
structural equivalence via name-equivalence. Fortunately, all of this
works out pleasantly and we may calculate in the natural way, free of
concern. The reader interested in the details is referred to the
appendix \ref{appendix:rho_details}.

\subsection{Substitution}

We use $\Proc$ for the set of processes, $\QProc$ for the set of
names, and $\id{\{}\vec{y} / \vec{x} \id{\}}$ to denote partial maps,
$s : \QProc \rightarrow \QProc$. A map, $s$ lifts, uniquely, to a map
on process terms, $\widehat{s} : \Proc \rightarrow \Proc$ by the
following equations.

\begin{mathpar}
  (0) \psubstp{Q}{P} := 0 \\
  (R \juxtap S) \psubstp{Q}{P}
  :=    
  (R)\psubstp{Q}{P} \juxtap (S) \psubstp{Q}{P} \\
  (x?(y).R) \psubstp{Q}{P}    
  :=    
  (x)\substp{Q}{P} (z)\concat( (R \psubstn{z}{y}) \psubstp{Q}{P} ) \\
  (\lift{x}{R}) \psubstp{Q}{P}  
  :=
  \lift{(x)\substp{Q}{P}}{ R \psubstp{Q}{P} } \\
%   (\dropn{x})  \psubstp{Q}{P}       
%   := 
%   \left\{ 
%     \begin{array}{ccc} 
%       \dropn{\quotep{Q}} & & x \nameeq \quotep{P} \\
%       \dropn{x} & & otherwise \\
%     \end{array}
%   \right. 
  (\dropn{x})  \psubstp{Q}{P}       
  := 
  \left\{ 
    \begin{array}{ccc} 
      Q & & x \nameeq \quotep{P} \\
      \dropn{x} & & otherwise \\
    \end{array}
  \right.
\end{mathpar}
 

where

\begin{eqnarray}
  (x)\id{\{} \lpquote Q \rpquote / \lpquote P \rpquote \id{\}}            = 
  \left\{ 
    \begin{array}{ccc}
      \lpquote Q \rpquote & & x \nameeq \lpquote P \rpquote \\
      x & & otherwise \\
    \end{array}
  \right. \nonumber
\end{eqnarray}

and $z$ is chosen distinct from $\quotep{P}$, $\quotep{Q}$, the free
names in $Q$, and all the names in $R$. Our $\alpha$-equivalence will
be built in the standard way from this substitution.

\begin{remark}\label{rem:no_self_referential_names}
  One consequence of these definitions is that $\forall P. \quotep{P}
  \not\in \freenames{P}$.
\end{remark}

\subsection{ Dynamic quote: an example }

Anticipating something of what's to come, consider applying the
substitution, $\widehat{\id{\{}u / z \id{\}}}$, to the following pair
of processes, $\lift{w}{y!(z)}$ and $w[ \lpquote y!(z) \rpquote ]$.

\begin{eqnarray}
	\lift{w}{y!(z)}\widehat{\id{\{}u / z \id{\}}}
		& = &
		\lift{w}{y!(u)} \nonumber\\
	w[ \lpquote y!(z) \rpquote ] \widehat{ \id{\{}u / z \id{\}} }
		& = &
		w[ \lpquote y!(z) \rpquote ] \nonumber
\end{eqnarray}

Because the body of the process between quotes is impervious to
substitution, we get radically different answers. In fact, by
examining the first process in an input context,
e.g. $x?(z).\lift{w}{y!(z)}$, we see that the process under the lift
operator may be shaped by prefixed inputs binding a name inside it. In
this sense, the lift operator will be seen as a way to dynamically
construct processes before reifying them as names.

Finally equipped with these standard features we can present the
dynamics of the calculus.

\subsubsection{Operational semantics} 

Finally, we introduce the computational dynamics. What marks these
algebras as distinct from other more traditionally studied algebraic
structures, e.g. vector spaces or polynomial rings, is the manner in
which dynamics is captured. In traditional structures, dynamics is typically
expressed through morphisms between such structures, as in linear maps
between vector spaces or morphisms between rings. In algebras
associated with the semantics of computation, the dynamics is
expressed as part of the algebraic structure itself, through a
reduction reduction relation typically denoted by $\red$. Below, we
give a recursive presentation of this relation for the calculus used
in the encoding.

$\red \subseteq \pi \times \pi$
$\red : \pi \to \mathcal{P}(\pi)$

\begin{mathpar}
  \inferrule* [lab=Comm] { \textsf{match}( x_{src}, x_{trgt} ) } { x_{trgt}?(y)P \; | \; x_{src}!\langle {Q} \rangle \red P\{\quotep{Q}/y}\} }
  \and \\
  \inferrule* [lab=Par] {{P} \red {P}'} {{{P} | {Q}} \red {{P}' | {Q}}}
  \and
  \inferrule* [lab=Equiv]{{{P} \scong {P}'} \andalso {{P}' \red {Q}'} \andalso {{Q}' \scong {Q}}}{{P} \red {Q}}
\end{mathpar}

\begin{eqnarray*}
  match_{\equiv} (\quotep{P},\quotep{Q}) & := & P \equiv Q \\
  match_{\dagger}(\quotep{P},\quotep{Q}) & := & \forall R. P|Q \red^{*} R => R \red^{*} 0 \\
  match_{K}(\quotep{P},\quotep{Q}) & := & K \mbox{ for some context } K
\end{eqnarray*}

$u?(x)P | u!\langle Q \rangle \red P\{\quotep{Q}/x\}$

%We write $\wred$ for $\red^*$, and $P\red$ if $\exists Q $ such that $ P \red Q$.
We write $P\red$ if $\exists Q $ such that $ P \red Q$ and $P\not\red$, otherwise.

\section{Replication}

As mentioned before, it is known that replication (and hence
recursion) can be implemented in a higher-order process algebra
\cite{SangiorgiWalker}. As our first example of calculation with the
machinery thus far presented we give the construction explicitly in
the {\rhoc}.

\begin{eqnarray}
	D_{x} & := & \prefix{x}{y}{(\binpar{\outputp{x}{y}}{@{y}})} \nonumber\\
	\bangp_{x}{P} & := & \binpar{{x}!\langle{\binpar{D_{x}}{P}}\rangle}{D_{x}} \nonumber
\end{eqnarray}

\begin{eqnarray}
	\bangp_{x}{P} & & \nonumber\\
	=
	& {x}!\langle{(\prefix{x}{y}{(\outputp{x}{y} | @{y})) | P}}\rangle 
	      | \prefix{x}{y}{(\outputp{x}{y} | @{y})} & \nonumber\\
	\red
	& (\outputp{x}{y} | @{y})\substn{\quotep{(\prefix{x}{y}{(@{y} | \outputp{x}{y})) | P}}}{y} & \nonumber\\
	=
	& \outputp{x}{\quotep{(\prefix{x}{y}{(\outputp{x}{y} | @{y})) | P}}}
	  | {(\prefix{x}{y}{(\outputp{x}{y} | @{y})) | P}} & \nonumber\\
	\red
	& \ldots & \nonumber\\
	\red^*
	& P | P | \ldots & \nonumber
\end{eqnarray}

Of course, this encoding, as an implementation, runs away, unfolding
$\bangp{P}$ eagerly. A lazier and more implementable replication
operator, restricted to input-guarded processes, may be obtained as follows.

\begin{eqnarray}
\bangp{\prefix{u}{v}{P}} 
	:= 
	\binpar{\lift{x}{\prefix{u}{v}{(\binpar{D(x)}{P})}}}{D(x)} \nonumber
\end{eqnarray}

\begin{remark}
  Note that the lazier definition still does not deal with summation
  or mixed summation (i.e. sums over input and output). The reader is
  invited to construct definitions of replication that deal with these
  features. 

  Further, the definitions are parameterized in a name, $x$. Can you,
  gentle reader, make a definition that eliminates this parameter and
  guarantees no accidental interaction between the replication
  machinery and the process being replicated -- i.e. no accidental
  sharing of names used by the process to get its work done and the
  name(s) used by the replication to effect copying. This latter
  revision of the definition of replication is crucial to obtaining
  the expected identity $!!P \sim !P$.
\end{remark}

\begin{remark}\label{rem:paradoxical_combinator}
  The reader familiar with the lambda calculus will have noticed the
  similarity between $D$ and the paradoxical combinator.

  [Ed. note: the existence of this seems to suggest we have to be more
  restrictive on the set of processes and names we admit if we are to
  support no-cloning.]
\end{remark}

\subsubsection{Bisimulation}

The computational dynamics gives rise to another kind of equivalence,
the equivalence of computational behavior. As previously mentioned
this is typically captured \emph{via} some form of bisimulation.

% The notion we use in this paper is weak barbed bisimulation
% \cite{milner91polyadicpi}.

The notion we use in this paper is derived from weak barbed
bisimulation \cite{milner91polyadicpi}. 

\begin{definition}
An \emph{observation relation}, $\downarrow_{\mathcal N}$, over a set
of names, $\mathcal N$, is the smallest relation satisfying the rules
below.

\infrule[Out-barb]{y \in {\mathcal N}, \; x \nameeq y}
		  {\outputp{x}{v} \downarrow_{\mathcal N} x}
\infrule[Par-barb]{\mbox{$P\downarrow_{\mathcal N} x$ or $Q\downarrow_{\mathcal N} x$}}
		  {\binpar{P}{Q} \downarrow_{\mathcal N} x}

We write $P \Downarrow_{\mathcal N} x$ if there is $Q$ such that 
$P \wred Q$ and $Q \downarrow_{\mathcal N} x$.
\end{definition}

\begin{definition}
%\label{def.bbisim}
An  ${\mathcal N}$-\emph{barbed bisimulation} over a set of names, ${\mathcal N}$, is a symmetric binary relation 
${\mathcal S}_{\mathcal N}$ between agents such that $P\rel{S}_{\mathcal N}Q$ implies:
\begin{enumerate}
\item If $P \red P'$ then $Q \wred Q'$ and $P'\rel{S}_{\mathcal N} Q'$.
\item If $P\downarrow_{\mathcal N} x$, then $Q\Downarrow_{\mathcal N} x$.
\end{enumerate}
$P$ is ${\mathcal N}$-barbed bisimilar to $Q$, written
$P \wbbisim_{\mathcal N} Q$, if $P \rel{S}_{\mathcal N} Q$ for some ${\mathcal N}$-barbed bisimulation ${\mathcal S}_{\mathcal N}$.
\end{definition}

$\mathcal{R} \subseteq \pi \times \pi$

$P \mathcal{R} Q => \forall P'. P \red P' \Rightarrow \exists Q'. Q \red Q', P' \mathcal{R} Q'$

$P \vdash x \Rightarrow Q \vdash x$

\begin{mathpar}
  \inferrule*[lab=Out-barb]{x \nameeq y}{{y}!\langle{Q}\rangle \vdash x}
  \and
  \inferrule*[lab=Par-barb]{\mbox{$P\vdash x$ or $Q\vdash x$}}{\binpar{P}{Q} \vdash x}
\end{mathpar}

\subsubsection{Contexts}

One of the principle advantages of computational calculi like the
$\pi$-calculus is a well-defined notion of context,
contextual-equivalence and a correlation between
contextual-equivalence and notions of bisimulation. The notion of
context allows the decomposition of a process into (sub-)process and
its syntactic environment, its context. Thus, a context may be
thought of as a process with a ``hole'' (written $\Box$) in it. The
application of a context $M$ to a process $P$, written $M[P]$, is
tantamount to filling the hole in $M$ with $P$. In this paper we do
not need the full weight of this theory, but do make use of the notion
of context in the proof the main theorem. 

\begin{mathpar}
  \inferrule* [lab=summation] {} {{M_{M},M_{N}} \bc \Box \;|\; x.M_{A} \;|\; M_{M}+M_{N}}
  \and
  \inferrule* [lab=agent] {} {{M_{A}} \bc (\vec{x})M_{P} \;| \; \clift{P_0,\ldots,M_{P},\ldots,P_N}}
  \and \\
  \inferrule* [lab=process] {} {{M_{P}} \bc M_{N} \;| \;P|M_{P} }
\end{mathpar} 

\begin{mathpar}
  \inferrule* [lab=sychronization] {} {M_{N} \bc \Box \;|\; x?M_{F} \;|\; x!M_{C}}
  \and
  \inferrule* [lab=abstraction] {} {{M_{F}} \bc (x)M_{P} }
  \and
  \inferrule* [lab=concretion] {} {{M_{C}} \bc \langle M_{P} \rangle }
  \and \\
  \inferrule* [lab=process] {} {{M_{P}} \bc M_{N} \;| \;P|M_{P} }
\end{mathpar}

\begin{definition}[contextual application] Given a context $M$, and
  process $P$, we define the \emph{contextual application}, $M[P] :=
  M\{P/\Box\}$. That is, the contextual application of M to P is the
  substitution of $P$ for $\Box$ in $M$.
\end{definition}

$\meaningof{-} : L \to \mathcal{P}(\pi)$

\begin{mathpar}
  \inferrule* [lab=collection] {} {\meaningof{true} = \pi, \and \meaningof{~E} = \pi \setminus \meaningof{E}, \and \meaningof{E_{1} \& E_{2}} = \meaningof{E_{1}} \cap \meaningof{E_{2}}}
\end{mathpar}

\begin{mathpar}
  \inferrule* [lab=structure] {} {\meaningof{0} = \{ P \in \pi | P \equiv 0 \}, \and \\ \meaningof{E_1 | E_2} = \{ P \in \pi | P \equiv P_{1} | P_{2}, P_{1} \in \meaningof{E_{1}}, P_{2} \in \meaningof{E_2}\} }
\end{mathpar}

\begin{mathpar}
 \inferrule* [lab=behavior] {} {\meaningof{\langle a?b \rangle E} = \{ P \in \pi | P \equiv Q | u?(y)P', \\ \and \\\\ \and \\ \;\;\; u \in \meaningof{a}, \forall z.P'\{z/y\} \in \meaningof{E\{z/b\}}\}, \and \\ \meaningof{a!E} = \{ P \in \pi | P \equiv Q | x!\langle P' \rangle, x \in \meaningof{a} P' \in \meaningof{E}\} }
\end{mathpar}

\begin{mathpar}
 \inferrule* [lab=nominal] {} {\meaningof{\quotep{E}} = \{ \quotep{P} \in \quotep{\pi} | P \in \meaningof{E} \}, \and \meaningof{\quotep{P}} = \{ \quotep{Q} \in \quotep{\pi} | P \equiv Q \} \and \\ \meaningof{@\quotep{E}} = \{ P \in \pi | P \equiv @x, x \in \meaningof{E} \}}
\end{mathpar}

\begin{eqnarray*}
  \\
  \meaningof{-} : TS \to ST
\end{eqnarray*}

\begin{eqnarray*}
  \\
  L : TS \to ST
\end{eqnarray*}

\begin{eqnarray*}
  \\
  P \models E \iff P \in \meaningof{E}
\end{eqnarray*}

\begin{eqnarray*}
  P \approx_{L} Q \iff \forall E \in L. P \models E \iff Q \models E
\end{eqnarray*}

\begin{eqnarray*}
  P \approx_{K} Q
\end{eqnarray*}

\begin{eqnarray*}
  P \approx Q
\end{eqnarray*}

$\approx_{K} = \approx = \approx_{L}$

\subsubsection{Contextual duality}

Note that contexts extend the quotation operation to a family of
operations from processes to names. Given a context, $M$, we can
define a \emph{nominal context}, $\quotep{M}$ by $\quotep{M}[P] :=
\quotep{M[P]}$. To foreshadow what is to come we observe that these
operations enjoy a duality with processes very much like the duality
between vectors and maps from vectors to scalars.

Further, because the calculus is essentially higher-order, we have a
correspondence between contexts and processes. More specifically,
given a name $x$ and a context $M$ we can construct $M^{*}_{x}$ such
that 

\begin{mathpar}
  M^{*}_{x} | \lift{x}{P} \red M[P]
\end{mathpar}

namely,

\begin{mathpar}
  M^{*}_{x} := x?(u).M[\dropn{u}]
\end{mathpar}

The dependence of $M^{*}_{x}$ on a name makes it an abstraction, 

\begin{mathpar}
  M^{*} := (x)x?(u).M[\dropn{u}]
\end{mathpar}

\subsection{Additional notation}

It will sometimes be convenient to denote the process a name
quotes. We already have the notation $x = \quotep{P}$, but it will be
convenient to introduce an alternate notation, $\procn{x}$, when we
want to emphasize the connection to the use of the name. Note that, by
virtue of name equivalence, $\quotep{\procn{x}} \nameeq x$; so, the
notation is consistent with previous definitions.

Further, because names have structure it is possible to effect
substitutions on the basis of that structure. This means we need to
upgrade our notation for substitutions, which we accomplish by
adapting comprehension notation. Thus,

\begin{mathpar}
  P\{ y / x : x \in S \}
\end{mathpar}

is interpreted to mean the process derived from P by replacing (in a
capture-avoiding manner) each occurrence of $x$ in $S$ by $y$. For example,

\begin{mathpar}
  P\{ \quotep{\procn{x}|\procn{x}} / x : x \in \freenames{P} \}
\end{mathpar}

will replace each (occurrence) of a free name $x$ in $P$ by
$\quotep{\procn{x}|\procn{x}}$.

Also, we will avail ourselves of the notation $x^{L}$ and $x^{R}$ to
denote injections of a name into disjoint copies of the name
space. There are numerous ways to accomplish this. One example can be
found in \cite{MeredithR05}. This notation overloads to vectors of
names: $\vec{x}^{\pi} := (x_{i}^{\pi} \; : \; 0 \leq i < |\vec{x}| )$ where $\pi \in \{L,R\}$.

We also use $P^{\Box} := P|\Box$.

In \cite{MeredithR05} an interpretation of the new operator is
given. It turns out that there are several possible interpretations
all enjoying the requisite algebraic properties of the operator (see
\cite{milner91polyadicpi}). We will therefore make liberal use of
$(\nu\; \vec{x})P$.

% subsection the_syntax_and_semantics_of_the_notation_system (end)   

\section{Interpretation of QM}
\subsection{Supporting definitions}
\subsubsection{Multiplication}
\begin{mathpar}
  \quotep{Q} \cdot \quotep{R} := \quotep{Q|R}
  \and \\
  \quotep{Q} \cdot P := P\{ \quotep{Q|R} / \quotep{R} : \quotep{R} \in \freenames{P} \}
\end{mathpar}

\paragraph{Discussion}
The first line needs little explanation. The second line says that
each free name of the process is replaced with the multiplication of
that name by the scalar. Multiplication of a scalar (name) by a state
(process) results in a process all the names of which have been `moved
over' by parallel composition with the process the scalar
quotes. There is a subtlety that the bound names have to be
manipulated so that multiplied names aren't accidentally
captured. There are many ways to achieve this.

\begin{remark}\label{rem:multiplication_identities}
  The reader is invited to verify that for all $x,y,z \in \QProc$ and $P \in \Proc$
  \begin{mathpar}
    x \cdot \quotep{0} \equiv x 
    \and
    x \cdot y \equiv y \cdot x
    \and
    x \cdot (y \cdot z) \equiv (x \cdot y) \cdot z
    \and \\
    \quotep{0} \cdot P \equiv P
    \and \\
    x \cdot (y \cdot P) \equiv (x \cdot y) \cdot P
    \and \\
    x \cdot (P|Q) \equiv (x \cdot P) | (x \cdot Q)
    \and \\    
  \end{mathpar}
\end{remark}

\subsubsection{Tensor product}

We define a tensor product on processes by structural induction.

\paragraph{Tensor of sums} First note that all summations, including
$\pzero$ and sequence, can be written $\Sigma_{i} x_{i}.A_{i} +
\Sigma_{j} x_{j}.C_{j}$, where we have grouped input-guarded processes
together and output-guarded processes together.

Thus, we can define the tensor product of two summations, $N_{1}\otimes N_{2}$, where

\begin{mathpar}
  N_{1} := \Sigma_{i} x_{i}.A_{i} + \Sigma_{j} x_{j}.C_{j}
  \and
  N_{2} := \Sigma_{i'} y_{i'}.B_{i'} + \Sigma_{j'} y_{j'}.D_{j'} 
\end{mathpar}

as follows.

\begin{mathpar}
  \Sigma_{i} x_{i}.A_{i} + \Sigma_{j} x_{j}.C_{j} \otimes \Sigma_{i'}
  y_{i'}.B_{i'} + \Sigma_{j'} y_{j'}.D_{j'} 
  \and \\
  := \; \Sigma_{i} \Sigma_{i'} \quotep{\stackrel{\vee}{x_{i}}| \stackrel{\vee}{y_{i'}}}.(A_{i}\otimes B_{i'}) \; | \; \Sigma_{i'} \Sigma_{i} \quotep{\stackrel{\vee}{y_{i'}}|\stackrel{\vee}{x_{i}}}.(B_{i'}\otimes A_{i})
  \and
  \;\; | \;\; \Sigma_{j} \Sigma_{j'} \quotep{\stackrel{\vee}{x_{j}}|\stackrel{\vee}{y_{j'}}}.(A_{j}\otimes B_{j'}) \; | \; \Sigma_{j'} \Sigma_{j} \quotep{\stackrel{\vee}{y_{j'}}|\stackrel{\vee}{x_{j}}}.(B_{j'}\otimes A_{j})
\end{mathpar}

\begin{remark}
  Do we need to $x^{L}$ and $y^{R}$ for this construction as well?
\end{remark}

\paragraph{Tensor of parallel compositions} Next, we distribute tensor
over par.

\begin{mathpar}
  P_{1}|P_{2} \otimes Q_{1}|Q_{2} := (P_{1} \otimes Q_{1}) | (P_{1}
  \otimes Q_{2}) | (P_{2} \otimes Q_{1}) | (P_{2} \otimes Q_{2})
\end{mathpar}

\paragraph{Tensor with dropped names} We treat tensor of a
process with a dropped name as parallel composition.

\begin{mathpar}
  P \otimes \dropn{x} := P | \dropn{x}
\end{mathpar}

\paragraph{Tensor of agents}

Finally, we need to define tensor on agents. Note that the definition
of tensor on normal products only tensors inputs with inputs and
outputs with outputs. Thus, we only have to define the operation on
``homogeneous'' pairings.

\begin{mathpar}
  (\vec{x})P \otimes (\vec{y})Q
  \and \\
  := (x_{0}^{L}|y_{0}^{R},\ldots,x_{0}^{L}|y_{n}^{R},\ldots,x_{m}^{L}|y_{0}^{R},\ldots,x_{m}^{L}|y_{n}^R)(P\{ \vec{x}^{L}/\vec{x}\} \otimes Q \{ \vec{y}^{R}/\vec{y}\})
  \and \\
  \clift{\vec{P}} \otimes \clift{\vec{Q}}
  \and \\
  := \clift{P_{0}\otimes Q_{0},\ldots,P_{0}\otimes Q_{n},\ldots,P_{m}\otimes Q_{0},\ldots,P_{m}\otimes Q_{n}}
\end{mathpar}

\begin{remark}
  Observe that arities of tensored abstractions matches arities of
  tensored concretions if the original arities matched. Note also that
  the length of the arities corresponds to the increase in dimension
  we see in ordinary vector space tensor product.
\end{remark}

\begin{remark}
  Operationally, this definition distributes the tensor down to
  components ``linked'' by summation. Tensor over summation is
  intriguing in that it mixes names. Moreover, as a consequence of the
  way it mixes names we have the identities for all $x \in \QProc$ and
  $P,Q \in \Proc$

  \begin{mathpar}
    (x \cdot P) \otimes Q \equiv x \cdot (P \otimes Q) \equiv P \otimes (x \cdot Q)
    \and
    P \otimes \pzero \equiv P
  \end{mathpar}

  that the reader is invited to verify.
\end{remark}

\subsubsection{Annihilation}
\begin{mathpar}
  P^{\perp} := \{ Q | \forall R. P|Q \red^{*} R \Rightarrow R \red^{*} \pzero \}
  \and \\
  P^{\underline{\perp}} := \Sigma_{Q \in P^{\perp}} \quotep{Q}?(y).(\dropn{y}|Q) | \Sigma_{Q \in P^{\perp}} \quotep{Q}\clift{\Box}
\end{mathpar}

\paragraph{Discussion} The reader will note that $P^{\perp}$ is a
\emph{set} of processes, while $P^{\underline{\perp}}$ is a
\emph{context}. We call the set $P^{\perp}$ the \emph{annihilators} of
$P$. The parallel composition of a process in the annihilators of $P$
with $P$ will result in a process, the state space of which has all
paths eventually leading to $\pzero$. Execution may endure loops; but
under reasonable conditions of fairness (naturally guaranteed under
most notions of bisimulation) such a composite process cannot get
stuck in such a loop and will, eventually pop out and terminate.

The context $P^{\underline{\perp}}$ is ready and willing to ``take the
$P$ out of'' the process to which it is applied. It will effectively
transmit the code of the process to which it is applied to one of the
annihilators and run the process against it.

\subsubsection{Evaluation}
We fix $M$ a domain of fully abstract interpretation with an equality
coincident with bisimulation. We take $\meaningof{\cdot} : \Proc \to
M$ to be the map interpreting processes and $\nmeaningof{\cdot} : \M
\to Proc$ to be the map running the other way. Then we define

\begin{mathpar}
  \int P := \nmeaningof{\meaningof{P}}
\end{mathpar}

\paragraph{Discussion}
There are many fully abstract interpretations of Milner's
$\pi$-calculus. Any of them can be used as a basis for interpreting
the reflective calculus here. Equipped with such a domain it is
largely a matter of grinding through to check that the Yoneda
construction for the normalization-by-evaluation program can be
extended to this setting.

\begin{remark}
  The reader is invited to verify that $\int (P^{\underline{\perp}}[P]) = 0$.
\end{remark}

\subsection{Quantum mechanics}

Table \ref{tbl:core_qm_op_defns} gives the core operational definitions

\begin{table}[htp]\label{tbl:core_qm_op_defns}
  \center{
    \fbox{
      \begin{tabular}{c|c}
        quantum mechanics & process calculus \\
        \hline
        scalar & $x := \quotep{P}$ \\
        state vector & $\state{P} := P$ \\
        dual & $\state{P}^{*} := \event{P^{\underline{\perp}}} := \quotep{P^{\underline{\perp}}}[-]$ \\
        matrix & $ \Sigma_{\alpha} \state{P_{\alpha}}x_{\alpha}\event{Q_{\alpha}}$ \\
        vector addition & $\state{P} + \state{Q} := \state{P | Q}$ \\
        tensor product & $\state{P} \otimes \state{Q} := \state{P \otimes Q}$ \\
        inner product & $\innerprod{P}{Q} := \quotep{\int P^{\underline{\perp}}[Q]}$ \\
      \end{tabular}
    }
  }
  \caption{QM - operational definitions}
\end{table}

where

\begin{mathpar}
  \prmatrix{P}{Q} := \fprmatrix{P}{\quotep{\pzero}}{Q}
  \and
  \fprmatrix{P}{x}{Q} := (\state{P},x,\event{Q})
  \and
  (\fprmatrix{P}{x}{Q})(\state{R}) := x \cdot \innerprod{Q}{R} \cdot \state{P}
  \and
  (\fprmatrix{P}{x}{Q})(\event{R}) := x \cdot \innerprod{R}{P} \cdot \event{Q}
\end{mathpar}

\paragraph{Discussion}
As promised: vectors (aka states) are represented as processes; duals
as contextual duals; inner product definition should be compared with
standard inner product definition for ....

\begin{remark}
  Assuming $\int (P^{\underline{\perp}}[P]) = 0$, the reader is
  invited to verify that $(\fprmatrix{P}{x}{P})(\state{P}) = x \cdot \state{P}$.
\end{remark}

\begin{remark}
  The reader is invited to verify that $\innerprod{P}{Q}$ could
  equally well have been written $\quotep{\int \stackrel{\vee}{x}}$
  where $x = \event{P^{\underline{\perp}}}(Q)$.

  One of the motivations for this remark is that there is another way
  to factor these operations. We could package up evaluation in the dual:

  \begin{mathpar}
    \state{P}^{*} := \event{\int P^{\underline{\perp}}} := \quotep{\int P^{\underline{\perp}}}[-]
  \end{mathpar}

  and then have inner product defined by
  
  \begin{mathpar}
    \innerprod{P}{Q} := \event{P}(Q)
  \end{mathpar}

  Hopefully, experience with the calculations will provide guidance on
  the best factoring.
\end{remark}

\begin{remark}
  Assuming $\int (P^{\underline{\perp}}[P]) = 0$, the reader is
  invited to verify that $\forall P,Q. (\prmatrix{0}{Q})(\state{0}) =
  \state{0}$ and dually $(\prmatrix{P}{0})(\event{0}) = \event{0}$.
\end{remark}

\begin{remark}
  i'm a little worried that i don't (yet) have proper support for
  complex conjugacy. But, the observation above may give us a
  clue. According to Abramsky, it must be the case that the scalars
  are iso to the homset of the identity for the tensor -- which the
  observation above characterizes. 

  For now, we will simply bookmark the notion with $\overline{x}$.
\end{remark}

\subsubsection{Adjointness}

We need to give a definition of $(\cdot)^{\dagger}$ for matrices. The
obvious candidate definition is
\begin{mathpar}
(\Sigma_{\alpha}\fprmatrix{P_{\alpha}}{x_{\alpha}}{Q_{\alpha}})^{\dagger}
= \Sigma_{\alpha}\fprmatrix{(Q_{\alpha}^{\underline{\perp}})^{*}}{\overline{x}_{\alpha}}{P_{\alpha}^{\underline{\perp}}} 
\end{mathpar}

But, $(Q_{\alpha}^{\underline{\perp}})^{*}$ requires a name along
which to communicate the process to achieve the context application.

\subsubsection{Basis for a basis}
If processes label states and ``addition'' of states (a.k.a. vector
addition) is interpreted as parallel composition, what corresponds to
notions of linear independence and basis? Here, we recall that Yoshida
has developed a set of \emph{combinators} for an asynchronous verison
of Milner's $\pi$-calculus. These are a finite set of processes such
any process can be expressed as parallel composition of these
combinators together with liberal uses of the new operator and
replication. We can simply give a translation of these into the
present calculus and have reasonable expectation that the property
carries over. That is, that the resultant set allows to express all
processes via parallel composition. Note, however, that there is no
new operator or replication in this calculus. As a result, we expect
that the corresponding set is actually infinite. That is, we expect
that the space is actually infinite dimensional.

\begin{remark}
  The attentive reader may be a bit concerned. Certainly, the
  collection $S$, $K$ and $I$ is a finite set of
  combinators. Shouldn't we expect to see a finite set of combinators
  for an effectively equivalent system? i am very sympathetic to this
  critique and feel it warrants full attention. On the other hand, i
  also have in mind the following analogy. The natural numbers, as a
  monoid under addition, has exactly $1$ generator, while the natural
  numbers, as a monoid under multiplication, has countably many
  generators (the primes). We observe that the application of the
  lambda calculus is much less resource sensitive than the parallel
  composition of the $\pi$-calculus. Could it be the case that we have
  an analogy of the form
  
  \begin{mathpar}
    m + n : MN :: m*n : M|N
  \end{mathpar}

  giving a similar blow up in the set of ``primes''?  This is such a
  wonderful thought that, even if it's not true, i think it's worth
  writing down.
\end{remark}
 

\documentclass[12pt]{llncs}
%\documentclass{jktr}

\usepackage[pdftex]{hyperref}                   
\usepackage {listings}
\usepackage {mathpartir}
\usepackage{bcprules}
%\usepackage{listings}
                       
\usepackage{graphicx} 
%\usepackage[margins=2.5cm,nohead,nofoot]{geometry}
%\usepackage{geometry}
\usepackage{amsfonts}
\usepackage{amstext}
\usepackage{latexsym}
\usepackage{amssymb}
\usepackage{color}


%\include{myPreamble}
\include{qm2pi.local} 

%\ifpdf
%\usepackage[pdftex]{graphicx}
%\else
%\usepackage{graphicx}
%\fi

 % \ifpdf
%  \usepackage{pdfsync}
%  \if


%\title{Brief Article}
%\author{David F. Snyder}
%\author{L.G. Meredith}

%\address{Dept. of Math., Texas State University--San Marcos, San Marcos, TX 78666}
       
\pagestyle{empty}


\begin{document}

\lstset{language=[Objective]Caml,frame=shadowbox}

\input{qm2pi.front}

% section front matter (end)

\input{qm2pi.intro} 
 
% section introduction (end)

% \input{qm2pi.knotations} 

% section notation (end)

\input{qm2pi.process.calculi} 

% section concurrent_process_calculi_and_spatial_logics_ (end)
    
%\input{qm2pi.knots2pi} 

%\input{qm2pi.trefoil} 

%\input{qm2pi.mainthm} 

% subsection basic_interpretation (end)

%\input{qm2pi.rho.presentation} 
\subsection{The syntax and semantics of the notation system}\label{sub:the_syntax_and_semantics_of_the_notation_system} % (fold)

We now summarize a technical presentation of the calculus that
embodies our theory of dynamics. The typical presentation of such a
calculus follows the style of giving generators and relations on
them. The grammar, below, describing term constructors, freely
generates the set of processes, $\Proc$. This set is then quotiented
by a relation known as structural congruence and it is over this set
that the notion of dynamics is expressed. This presentation is
essentially that of \cite{MeredithR05} with the addition of
polyadicity and summation. For readability we have relegated some of
the technical subtleties to an appendix.

\subsubsection{Process grammar}\label{subsub:process_grammar}

\begin{mathpar}
  \inferrule* [lab=synchronization] {} {{M} \bc \pzero \;|\; x?F \;|\; x!C }
  \and
  \inferrule* [lab=abstraction] {} {{F} \bc (x)P}
  \and
  \inferrule* [lab=concretion] {} {{C} \bc \langle Q \rangle}
  \and
  \inferrule* [lab=process] {} {{P,Q} \bc M \;| \;P|Q \;|\; @{x}}
  \and
  \inferrule* [lab=name] {} {{x} \bc \quotep{P}}
\end{mathpar} 

Note that $\vec{x}$ (resp. $\vec{P}$) denotes a vector of names
(resp. processes) of length $|\vec{x}|$ (resp. $|\vec{P}|$). We adopt
the following useful abbreviations.

\begin{mathpar}
   x?(\vec{y}).P := x.(\vec{y})P \and  x\clift{\vec{P}} := x.\clift{\vec{P}}
   \and x!(y) := \lift{x}{\dropn{y}}
   \and \Pi_{i=0}^{n-1}P_i := P_0 | \ldots | P_{n-1}
\end{mathpar}

\subsubsection{Structural congruence}

\paragraph{Free and bound names and alpha-equivalence.} At the
core of structural equivalence is alpha-equivalence which identifies
process that are the same up to a change of variable. Formally, we
recognize the distinction between free and bound names. The free names
of a process, $\freenames{P}$, may be calculated recursively as
follows:

\begin{mathpar}
\freenames{\pzero} := \emptyset
  \and \\
  \freenames{x?(y).P} := \{ x \} \cup (\freenames{P} \setminus \{ y \})
  \and 
  \freenames{x!\langle P \rangle} := \{ x \} \cup \{ P \} 
  \and \\
  \freenames{P|Q} := \freenames{P} \cup \freenames{Q}
  \and \\
  \freenames{@{x}} := \{ x \}
\end{mathpar}

$\pi$
$\quotep{\pi}$

$\freenames{-} : \pi \to \mathcal{P}(\quotep{\pi})$

\begin{eqnarray*}
  \freenames{\pzero} & := & \emptyset \\
  \freenames{x?(y).P} & := & \{ x \} \cup (\freenames{P} \setminus \{ y \}) \\
  \freenames{x!\langle P \rangle} & := & \{ x \} \cup \{ P \} \\
  \freenames{P|Q} & := & \freenames{P} \cup \freenames{Q} \\
  \freenames{\dropn{x}} & := & \{ x \}
\end{eqnarray*}

The bound names of a process, $\boundnames{P}$, are those names occurring in $P$
that are not free. For example, in $x?(y).0$, the name $x$ is free, while $y$ is bound.

\begin{mathpar}
  \inferrule* [lab=monoidal-laws] {} { P|Q \equiv Q|P \and P|0 \equiv P \and P|(Q|R) \equiv (P|Q)|R }
\end{mathpar}

\begin{mathpar}
  \inferrule* [lab=alpha-equivalence] {} { (x)P \equiv (y)P\{y/x\} \and y \not\in \freenames{P} }
\end{mathpar}

\begin{definition}
Then two processes, $P,Q$, are alpha-equivalent if $P = Q\{\vec{y}/\vec{x}\}$ for
some $\vec{x} \in \boundnames{Q},\vec{y} \in \boundnames{P}$, where $Q\{\vec{y}/\vec{x}\}$
denotes the capture-avoiding substitution of $\vec{y}$ for $\vec{x}$ in $Q$.
\end{definition}

\begin{definition}
  The {\em structural congruence} \cite{SangiorgiWalker} , $\equiv$,
  between processes is the least congruence containing
  alpha-equivalence, satisfying the abelian monoid laws
  (associativity, commutativity and $\pzero$ as identity) for parallel
  composition $|$ and for summation $+$.
\end{definition}

\subsection{Name equivalence}

We take name equivalence, written $\nameeq$, to be the smallest
equivalence relation generated by the following rules.

\begin{mathpar}
\inferrule*[lab=Quote-drop]
{ }
{ \quotep{@{x}} \nameeq x }

\inferrule*[lab=Struct-equiv]
{ P \scong Q }
{ \quotep{P} \nameeq \quotep{Q} }
\end{mathpar}

The astute reader will have noticed that the mutual recursion of names
and processes imposes a mutual recursion on alpha-equivalence and
structural equivalence via name-equivalence. Fortunately, all of this
works out pleasantly and we may calculate in the natural way, free of
concern. The reader interested in the details is referred to the
appendix \ref{appendix:rho_details}.

\subsection{Substitution}

We use $\Proc$ for the set of processes, $\QProc$ for the set of
names, and $\id{\{}\vec{y} / \vec{x} \id{\}}$ to denote partial maps,
$s : \QProc \rightarrow \QProc$. A map, $s$ lifts, uniquely, to a map
on process terms, $\widehat{s} : \Proc \rightarrow \Proc$ by the
following equations.

\begin{mathpar}
  (0) \psubstp{Q}{P} := 0 \\
  (R \juxtap S) \psubstp{Q}{P}
  :=    
  (R)\psubstp{Q}{P} \juxtap (S) \psubstp{Q}{P} \\
  (x?(y).R) \psubstp{Q}{P}    
  :=    
  (x)\substp{Q}{P} (z)\concat( (R \psubstn{z}{y}) \psubstp{Q}{P} ) \\
  (\lift{x}{R}) \psubstp{Q}{P}  
  :=
  \lift{(x)\substp{Q}{P}}{ R \psubstp{Q}{P} } \\
%   (\dropn{x})  \psubstp{Q}{P}       
%   := 
%   \left\{ 
%     \begin{array}{ccc} 
%       \dropn{\quotep{Q}} & & x \nameeq \quotep{P} \\
%       \dropn{x} & & otherwise \\
%     \end{array}
%   \right. 
  (\dropn{x})  \psubstp{Q}{P}       
  := 
  \left\{ 
    \begin{array}{ccc} 
      Q & & x \nameeq \quotep{P} \\
      \dropn{x} & & otherwise \\
    \end{array}
  \right.
\end{mathpar}
 

where

\begin{eqnarray}
  (x)\id{\{} \lpquote Q \rpquote / \lpquote P \rpquote \id{\}}            = 
  \left\{ 
    \begin{array}{ccc}
      \lpquote Q \rpquote & & x \nameeq \lpquote P \rpquote \\
      x & & otherwise \\
    \end{array}
  \right. \nonumber
\end{eqnarray}

and $z$ is chosen distinct from $\quotep{P}$, $\quotep{Q}$, the free
names in $Q$, and all the names in $R$. Our $\alpha$-equivalence will
be built in the standard way from this substitution.

\begin{remark}\label{rem:no_self_referential_names}
  One consequence of these definitions is that $\forall P. \quotep{P}
  \not\in \freenames{P}$.
\end{remark}

\subsection{ Dynamic quote: an example }

Anticipating something of what's to come, consider applying the
substitution, $\widehat{\id{\{}u / z \id{\}}}$, to the following pair
of processes, $\lift{w}{y!(z)}$ and $w[ \lpquote y!(z) \rpquote ]$.

\begin{eqnarray}
	\lift{w}{y!(z)}\widehat{\id{\{}u / z \id{\}}}
		& = &
		\lift{w}{y!(u)} \nonumber\\
	w[ \lpquote y!(z) \rpquote ] \widehat{ \id{\{}u / z \id{\}} }
		& = &
		w[ \lpquote y!(z) \rpquote ] \nonumber
\end{eqnarray}

Because the body of the process between quotes is impervious to
substitution, we get radically different answers. In fact, by
examining the first process in an input context,
e.g. $x?(z).\lift{w}{y!(z)}$, we see that the process under the lift
operator may be shaped by prefixed inputs binding a name inside it. In
this sense, the lift operator will be seen as a way to dynamically
construct processes before reifying them as names.

Finally equipped with these standard features we can present the
dynamics of the calculus.

\subsubsection{Operational semantics} 

Finally, we introduce the computational dynamics. What marks these
algebras as distinct from other more traditionally studied algebraic
structures, e.g. vector spaces or polynomial rings, is the manner in
which dynamics is captured. In traditional structures, dynamics is typically
expressed through morphisms between such structures, as in linear maps
between vector spaces or morphisms between rings. In algebras
associated with the semantics of computation, the dynamics is
expressed as part of the algebraic structure itself, through a
reduction reduction relation typically denoted by $\red$. Below, we
give a recursive presentation of this relation for the calculus used
in the encoding.

$\red \subseteq \pi \times \pi$
$\red : \pi \to \mathcal{P}(\pi)$

\begin{mathpar}
  \inferrule* [lab=Comm] { \textsf{match}( x_{src}, x_{trgt} ) } { x_{trgt}?(y)P \; | \; x_{src}!\langle {Q} \rangle \red P\{\quotep{Q}/y}\} }
  \and \\
  \inferrule* [lab=Par] {{P} \red {P}'} {{{P} | {Q}} \red {{P}' | {Q}}}
  \and
  \inferrule* [lab=Equiv]{{{P} \scong {P}'} \andalso {{P}' \red {Q}'} \andalso {{Q}' \scong {Q}}}{{P} \red {Q}}
\end{mathpar}

\begin{eqnarray*}
  match_{\equiv} (\quotep{P},\quotep{Q}) & := & P \equiv Q \\
  match_{\dagger}(\quotep{P},\quotep{Q}) & := & \forall R. P|Q \red^{*} R => R \red^{*} 0 \\
  match_{K}(\quotep{P},\quotep{Q}) & := & K \mbox{ for some context } K
\end{eqnarray*}

$u?(x)P | u!\langle Q \rangle \red P\{\quotep{Q}/x\}$

%We write $\wred$ for $\red^*$, and $P\red$ if $\exists Q $ such that $ P \red Q$.
We write $P\red$ if $\exists Q $ such that $ P \red Q$ and $P\not\red$, otherwise.

\section{Replication}

As mentioned before, it is known that replication (and hence
recursion) can be implemented in a higher-order process algebra
\cite{SangiorgiWalker}. As our first example of calculation with the
machinery thus far presented we give the construction explicitly in
the {\rhoc}.

\begin{eqnarray}
	D_{x} & := & \prefix{x}{y}{(\binpar{\outputp{x}{y}}{@{y}})} \nonumber\\
	\bangp_{x}{P} & := & \binpar{{x}!\langle{\binpar{D_{x}}{P}}\rangle}{D_{x}} \nonumber
\end{eqnarray}

\begin{eqnarray}
	\bangp_{x}{P} & & \nonumber\\
	=
	& {x}!\langle{(\prefix{x}{y}{(\outputp{x}{y} | @{y})) | P}}\rangle 
	      | \prefix{x}{y}{(\outputp{x}{y} | @{y})} & \nonumber\\
	\red
	& (\outputp{x}{y} | @{y})\substn{\quotep{(\prefix{x}{y}{(@{y} | \outputp{x}{y})) | P}}}{y} & \nonumber\\
	=
	& \outputp{x}{\quotep{(\prefix{x}{y}{(\outputp{x}{y} | @{y})) | P}}}
	  | {(\prefix{x}{y}{(\outputp{x}{y} | @{y})) | P}} & \nonumber\\
	\red
	& \ldots & \nonumber\\
	\red^*
	& P | P | \ldots & \nonumber
\end{eqnarray}

Of course, this encoding, as an implementation, runs away, unfolding
$\bangp{P}$ eagerly. A lazier and more implementable replication
operator, restricted to input-guarded processes, may be obtained as follows.

\begin{eqnarray}
\bangp{\prefix{u}{v}{P}} 
	:= 
	\binpar{\lift{x}{\prefix{u}{v}{(\binpar{D(x)}{P})}}}{D(x)} \nonumber
\end{eqnarray}

\begin{remark}
  Note that the lazier definition still does not deal with summation
  or mixed summation (i.e. sums over input and output). The reader is
  invited to construct definitions of replication that deal with these
  features. 

  Further, the definitions are parameterized in a name, $x$. Can you,
  gentle reader, make a definition that eliminates this parameter and
  guarantees no accidental interaction between the replication
  machinery and the process being replicated -- i.e. no accidental
  sharing of names used by the process to get its work done and the
  name(s) used by the replication to effect copying. This latter
  revision of the definition of replication is crucial to obtaining
  the expected identity $!!P \sim !P$.
\end{remark}

\begin{remark}\label{rem:paradoxical_combinator}
  The reader familiar with the lambda calculus will have noticed the
  similarity between $D$ and the paradoxical combinator.

  [Ed. note: the existence of this seems to suggest we have to be more
  restrictive on the set of processes and names we admit if we are to
  support no-cloning.]
\end{remark}

\subsubsection{Bisimulation}

The computational dynamics gives rise to another kind of equivalence,
the equivalence of computational behavior. As previously mentioned
this is typically captured \emph{via} some form of bisimulation.

% The notion we use in this paper is weak barbed bisimulation
% \cite{milner91polyadicpi}.

The notion we use in this paper is derived from weak barbed
bisimulation \cite{milner91polyadicpi}. 

\begin{definition}
An \emph{observation relation}, $\downarrow_{\mathcal N}$, over a set
of names, $\mathcal N$, is the smallest relation satisfying the rules
below.

\infrule[Out-barb]{y \in {\mathcal N}, \; x \nameeq y}
		  {\outputp{x}{v} \downarrow_{\mathcal N} x}
\infrule[Par-barb]{\mbox{$P\downarrow_{\mathcal N} x$ or $Q\downarrow_{\mathcal N} x$}}
		  {\binpar{P}{Q} \downarrow_{\mathcal N} x}

We write $P \Downarrow_{\mathcal N} x$ if there is $Q$ such that 
$P \wred Q$ and $Q \downarrow_{\mathcal N} x$.
\end{definition}

\begin{definition}
%\label{def.bbisim}
An  ${\mathcal N}$-\emph{barbed bisimulation} over a set of names, ${\mathcal N}$, is a symmetric binary relation 
${\mathcal S}_{\mathcal N}$ between agents such that $P\rel{S}_{\mathcal N}Q$ implies:
\begin{enumerate}
\item If $P \red P'$ then $Q \wred Q'$ and $P'\rel{S}_{\mathcal N} Q'$.
\item If $P\downarrow_{\mathcal N} x$, then $Q\Downarrow_{\mathcal N} x$.
\end{enumerate}
$P$ is ${\mathcal N}$-barbed bisimilar to $Q$, written
$P \wbbisim_{\mathcal N} Q$, if $P \rel{S}_{\mathcal N} Q$ for some ${\mathcal N}$-barbed bisimulation ${\mathcal S}_{\mathcal N}$.
\end{definition}

$\mathcal{R} \subseteq \pi \times \pi$

$P \mathcal{R} Q => \forall P'. P \red P' \Rightarrow \exists Q'. Q \red Q', P' \mathcal{R} Q'$

$P \vdash x \Rightarrow Q \vdash x$

\begin{mathpar}
  \inferrule*[lab=Out-barb]{x \nameeq y}{{y}!\langle{Q}\rangle \vdash x}
  \and
  \inferrule*[lab=Par-barb]{\mbox{$P\vdash x$ or $Q\vdash x$}}{\binpar{P}{Q} \vdash x}
\end{mathpar}

\subsubsection{Contexts}

One of the principle advantages of computational calculi like the
$\pi$-calculus is a well-defined notion of context,
contextual-equivalence and a correlation between
contextual-equivalence and notions of bisimulation. The notion of
context allows the decomposition of a process into (sub-)process and
its syntactic environment, its context. Thus, a context may be
thought of as a process with a ``hole'' (written $\Box$) in it. The
application of a context $M$ to a process $P$, written $M[P]$, is
tantamount to filling the hole in $M$ with $P$. In this paper we do
not need the full weight of this theory, but do make use of the notion
of context in the proof the main theorem. 

\begin{mathpar}
  \inferrule* [lab=summation] {} {{M_{M},M_{N}} \bc \Box \;|\; x.M_{A} \;|\; M_{M}+M_{N}}
  \and
  \inferrule* [lab=agent] {} {{M_{A}} \bc (\vec{x})M_{P} \;| \; \clift{P_0,\ldots,M_{P},\ldots,P_N}}
  \and \\
  \inferrule* [lab=process] {} {{M_{P}} \bc M_{N} \;| \;P|M_{P} }
\end{mathpar} 

\begin{mathpar}
  \inferrule* [lab=sychronization] {} {M_{N} \bc \Box \;|\; x?M_{F} \;|\; x!M_{C}}
  \and
  \inferrule* [lab=abstraction] {} {{M_{F}} \bc (x)M_{P} }
  \and
  \inferrule* [lab=concretion] {} {{M_{C}} \bc \langle M_{P} \rangle }
  \and \\
  \inferrule* [lab=process] {} {{M_{P}} \bc M_{N} \;| \;P|M_{P} }
\end{mathpar}

\begin{definition}[contextual application] Given a context $M$, and
  process $P$, we define the \emph{contextual application}, $M[P] :=
  M\{P/\Box\}$. That is, the contextual application of M to P is the
  substitution of $P$ for $\Box$ in $M$.
\end{definition}

$\meaningof{-} : L \to \mathcal{P}(\pi)$

\begin{mathpar}
  \inferrule* [lab=collection] {} {\meaningof{true} = \pi, \and \meaningof{~E} = \pi \setminus \meaningof{E}, \and \meaningof{E_{1} \& E_{2}} = \meaningof{E_{1}} \cap \meaningof{E_{2}}}
\end{mathpar}

\begin{mathpar}
  \inferrule* [lab=structure] {} {\meaningof{0} = \{ P \in \pi | P \equiv 0 \}, \and \\ \meaningof{E_1 | E_2} = \{ P \in \pi | P \equiv P_{1} | P_{2}, P_{1} \in \meaningof{E_{1}}, P_{2} \in \meaningof{E_2}\} }
\end{mathpar}

\begin{mathpar}
 \inferrule* [lab=behavior] {} {\meaningof{\langle a?b \rangle E} = \{ P \in \pi | P \equiv Q | u?(y)P', \\ \and \\\\ \and \\ \;\;\; u \in \meaningof{a}, \forall z.P'\{z/y\} \in \meaningof{E\{z/b\}}\}, \and \\ \meaningof{a!E} = \{ P \in \pi | P \equiv Q | x!\langle P' \rangle, x \in \meaningof{a} P' \in \meaningof{E}\} }
\end{mathpar}

\begin{mathpar}
 \inferrule* [lab=nominal] {} {\meaningof{\quotep{E}} = \{ \quotep{P} \in \quotep{\pi} | P \in \meaningof{E} \}, \and \meaningof{\quotep{P}} = \{ \quotep{Q} \in \quotep{\pi} | P \equiv Q \} \and \\ \meaningof{@\quotep{E}} = \{ P \in \pi | P \equiv @x, x \in \meaningof{E} \}}
\end{mathpar}

\begin{eqnarray*}
  \\
  \meaningof{-} : TS \to ST
\end{eqnarray*}

\begin{eqnarray*}
  \\
  L : TS \to ST
\end{eqnarray*}

\begin{eqnarray*}
  \\
  P \models E \iff P \in \meaningof{E}
\end{eqnarray*}

\begin{eqnarray*}
  P \approx_{L} Q \iff \forall E \in L. P \models E \iff Q \models E
\end{eqnarray*}

\begin{eqnarray*}
  P \approx_{K} Q
\end{eqnarray*}

\begin{eqnarray*}
  P \approx Q
\end{eqnarray*}

$\approx_{K} = \approx = \approx_{L}$

\subsubsection{Contextual duality}

Note that contexts extend the quotation operation to a family of
operations from processes to names. Given a context, $M$, we can
define a \emph{nominal context}, $\quotep{M}$ by $\quotep{M}[P] :=
\quotep{M[P]}$. To foreshadow what is to come we observe that these
operations enjoy a duality with processes very much like the duality
between vectors and maps from vectors to scalars.

Further, because the calculus is essentially higher-order, we have a
correspondence between contexts and processes. More specifically,
given a name $x$ and a context $M$ we can construct $M^{*}_{x}$ such
that 

\begin{mathpar}
  M^{*}_{x} | \lift{x}{P} \red M[P]
\end{mathpar}

namely,

\begin{mathpar}
  M^{*}_{x} := x?(u).M[\dropn{u}]
\end{mathpar}

The dependence of $M^{*}_{x}$ on a name makes it an abstraction, 

\begin{mathpar}
  M^{*} := (x)x?(u).M[\dropn{u}]
\end{mathpar}

\subsection{Additional notation}

It will sometimes be convenient to denote the process a name
quotes. We already have the notation $x = \quotep{P}$, but it will be
convenient to introduce an alternate notation, $\procn{x}$, when we
want to emphasize the connection to the use of the name. Note that, by
virtue of name equivalence, $\quotep{\procn{x}} \nameeq x$; so, the
notation is consistent with previous definitions.

Further, because names have structure it is possible to effect
substitutions on the basis of that structure. This means we need to
upgrade our notation for substitutions, which we accomplish by
adapting comprehension notation. Thus,

\begin{mathpar}
  P\{ y / x : x \in S \}
\end{mathpar}

is interpreted to mean the process derived from P by replacing (in a
capture-avoiding manner) each occurrence of $x$ in $S$ by $y$. For example,

\begin{mathpar}
  P\{ \quotep{\procn{x}|\procn{x}} / x : x \in \freenames{P} \}
\end{mathpar}

will replace each (occurrence) of a free name $x$ in $P$ by
$\quotep{\procn{x}|\procn{x}}$.

Also, we will avail ourselves of the notation $x^{L}$ and $x^{R}$ to
denote injections of a name into disjoint copies of the name
space. There are numerous ways to accomplish this. One example can be
found in \cite{MeredithR05}. This notation overloads to vectors of
names: $\vec{x}^{\pi} := (x_{i}^{\pi} \; : \; 0 \leq i < |\vec{x}| )$ where $\pi \in \{L,R\}$.

We also use $P^{\Box} := P|\Box$.

In \cite{MeredithR05} an interpretation of the new operator is
given. It turns out that there are several possible interpretations
all enjoying the requisite algebraic properties of the operator (see
\cite{milner91polyadicpi}). We will therefore make liberal use of
$(\nu\; \vec{x})P$.

% subsection the_syntax_and_semantics_of_the_notation_system (end)   

\input{qm2pi.qmops} 

\input{qm2pi.sterngerlach} 

\input{qm2pi.metric} 

% section concurrent_process_calculi (end)

%\input{qm2pi.proofsketch}

% section proof sketch (end)

%\input{qm2pi.slviaknots} 

% section spatial logic via knots (end)

\input{qm2pi.conclusion}

% section conclusion (end)

%\input{qm2pi.dtcodes} 

% section wiring algorithm (end)

\input{qm2pi.ack} 

% section acknowledgments (end)

\newpage


\bibliographystyle{plain}   
\bibliography{../../biblios/main.bib}

\input{qm2pi.rhodetails}

\end{document}

 

\documentclass[12pt]{llncs}
%\documentclass{jktr}

\usepackage[pdftex]{hyperref}                   
\usepackage {listings}
\usepackage {mathpartir}
\usepackage{bcprules}
%\usepackage{listings}
                       
\usepackage{graphicx} 
%\usepackage[margins=2.5cm,nohead,nofoot]{geometry}
%\usepackage{geometry}
\usepackage{amsfonts}
\usepackage{amstext}
\usepackage{latexsym}
\usepackage{amssymb}
\usepackage{color}


%\include{myPreamble}
\include{qm2pi.local} 

%\ifpdf
%\usepackage[pdftex]{graphicx}
%\else
%\usepackage{graphicx}
%\fi

 % \ifpdf
%  \usepackage{pdfsync}
%  \if


%\title{Brief Article}
%\author{David F. Snyder}
%\author{L.G. Meredith}

%\address{Dept. of Math., Texas State University--San Marcos, San Marcos, TX 78666}
       
\pagestyle{empty}


\begin{document}

\lstset{language=[Objective]Caml,frame=shadowbox}

\input{qm2pi.front}

% section front matter (end)

\input{qm2pi.intro} 
 
% section introduction (end)

% \input{qm2pi.knotations} 

% section notation (end)

\input{qm2pi.process.calculi} 

% section concurrent_process_calculi_and_spatial_logics_ (end)
    
%\input{qm2pi.knots2pi} 

%\input{qm2pi.trefoil} 

%\input{qm2pi.mainthm} 

% subsection basic_interpretation (end)

%\input{qm2pi.rho.presentation} 
\subsection{The syntax and semantics of the notation system}\label{sub:the_syntax_and_semantics_of_the_notation_system} % (fold)

We now summarize a technical presentation of the calculus that
embodies our theory of dynamics. The typical presentation of such a
calculus follows the style of giving generators and relations on
them. The grammar, below, describing term constructors, freely
generates the set of processes, $\Proc$. This set is then quotiented
by a relation known as structural congruence and it is over this set
that the notion of dynamics is expressed. This presentation is
essentially that of \cite{MeredithR05} with the addition of
polyadicity and summation. For readability we have relegated some of
the technical subtleties to an appendix.

\subsubsection{Process grammar}\label{subsub:process_grammar}

\begin{mathpar}
  \inferrule* [lab=synchronization] {} {{M} \bc \pzero \;|\; x?F \;|\; x!C }
  \and
  \inferrule* [lab=abstraction] {} {{F} \bc (x)P}
  \and
  \inferrule* [lab=concretion] {} {{C} \bc \langle Q \rangle}
  \and
  \inferrule* [lab=process] {} {{P,Q} \bc M \;| \;P|Q \;|\; @{x}}
  \and
  \inferrule* [lab=name] {} {{x} \bc \quotep{P}}
\end{mathpar} 

Note that $\vec{x}$ (resp. $\vec{P}$) denotes a vector of names
(resp. processes) of length $|\vec{x}|$ (resp. $|\vec{P}|$). We adopt
the following useful abbreviations.

\begin{mathpar}
   x?(\vec{y}).P := x.(\vec{y})P \and  x\clift{\vec{P}} := x.\clift{\vec{P}}
   \and x!(y) := \lift{x}{\dropn{y}}
   \and \Pi_{i=0}^{n-1}P_i := P_0 | \ldots | P_{n-1}
\end{mathpar}

\subsubsection{Structural congruence}

\paragraph{Free and bound names and alpha-equivalence.} At the
core of structural equivalence is alpha-equivalence which identifies
process that are the same up to a change of variable. Formally, we
recognize the distinction between free and bound names. The free names
of a process, $\freenames{P}$, may be calculated recursively as
follows:

\begin{mathpar}
\freenames{\pzero} := \emptyset
  \and \\
  \freenames{x?(y).P} := \{ x \} \cup (\freenames{P} \setminus \{ y \})
  \and 
  \freenames{x!\langle P \rangle} := \{ x \} \cup \{ P \} 
  \and \\
  \freenames{P|Q} := \freenames{P} \cup \freenames{Q}
  \and \\
  \freenames{@{x}} := \{ x \}
\end{mathpar}

$\pi$
$\quotep{\pi}$

$\freenames{-} : \pi \to \mathcal{P}(\quotep{\pi})$

\begin{eqnarray*}
  \freenames{\pzero} & := & \emptyset \\
  \freenames{x?(y).P} & := & \{ x \} \cup (\freenames{P} \setminus \{ y \}) \\
  \freenames{x!\langle P \rangle} & := & \{ x \} \cup \{ P \} \\
  \freenames{P|Q} & := & \freenames{P} \cup \freenames{Q} \\
  \freenames{\dropn{x}} & := & \{ x \}
\end{eqnarray*}

The bound names of a process, $\boundnames{P}$, are those names occurring in $P$
that are not free. For example, in $x?(y).0$, the name $x$ is free, while $y$ is bound.

\begin{mathpar}
  \inferrule* [lab=monoidal-laws] {} { P|Q \equiv Q|P \and P|0 \equiv P \and P|(Q|R) \equiv (P|Q)|R }
\end{mathpar}

\begin{mathpar}
  \inferrule* [lab=alpha-equivalence] {} { (x)P \equiv (y)P\{y/x\} \and y \not\in \freenames{P} }
\end{mathpar}

\begin{definition}
Then two processes, $P,Q$, are alpha-equivalent if $P = Q\{\vec{y}/\vec{x}\}$ for
some $\vec{x} \in \boundnames{Q},\vec{y} \in \boundnames{P}$, where $Q\{\vec{y}/\vec{x}\}$
denotes the capture-avoiding substitution of $\vec{y}$ for $\vec{x}$ in $Q$.
\end{definition}

\begin{definition}
  The {\em structural congruence} \cite{SangiorgiWalker} , $\equiv$,
  between processes is the least congruence containing
  alpha-equivalence, satisfying the abelian monoid laws
  (associativity, commutativity and $\pzero$ as identity) for parallel
  composition $|$ and for summation $+$.
\end{definition}

\subsection{Name equivalence}

We take name equivalence, written $\nameeq$, to be the smallest
equivalence relation generated by the following rules.

\begin{mathpar}
\inferrule*[lab=Quote-drop]
{ }
{ \quotep{@{x}} \nameeq x }

\inferrule*[lab=Struct-equiv]
{ P \scong Q }
{ \quotep{P} \nameeq \quotep{Q} }
\end{mathpar}

The astute reader will have noticed that the mutual recursion of names
and processes imposes a mutual recursion on alpha-equivalence and
structural equivalence via name-equivalence. Fortunately, all of this
works out pleasantly and we may calculate in the natural way, free of
concern. The reader interested in the details is referred to the
appendix \ref{appendix:rho_details}.

\subsection{Substitution}

We use $\Proc$ for the set of processes, $\QProc$ for the set of
names, and $\id{\{}\vec{y} / \vec{x} \id{\}}$ to denote partial maps,
$s : \QProc \rightarrow \QProc$. A map, $s$ lifts, uniquely, to a map
on process terms, $\widehat{s} : \Proc \rightarrow \Proc$ by the
following equations.

\begin{mathpar}
  (0) \psubstp{Q}{P} := 0 \\
  (R \juxtap S) \psubstp{Q}{P}
  :=    
  (R)\psubstp{Q}{P} \juxtap (S) \psubstp{Q}{P} \\
  (x?(y).R) \psubstp{Q}{P}    
  :=    
  (x)\substp{Q}{P} (z)\concat( (R \psubstn{z}{y}) \psubstp{Q}{P} ) \\
  (\lift{x}{R}) \psubstp{Q}{P}  
  :=
  \lift{(x)\substp{Q}{P}}{ R \psubstp{Q}{P} } \\
%   (\dropn{x})  \psubstp{Q}{P}       
%   := 
%   \left\{ 
%     \begin{array}{ccc} 
%       \dropn{\quotep{Q}} & & x \nameeq \quotep{P} \\
%       \dropn{x} & & otherwise \\
%     \end{array}
%   \right. 
  (\dropn{x})  \psubstp{Q}{P}       
  := 
  \left\{ 
    \begin{array}{ccc} 
      Q & & x \nameeq \quotep{P} \\
      \dropn{x} & & otherwise \\
    \end{array}
  \right.
\end{mathpar}
 

where

\begin{eqnarray}
  (x)\id{\{} \lpquote Q \rpquote / \lpquote P \rpquote \id{\}}            = 
  \left\{ 
    \begin{array}{ccc}
      \lpquote Q \rpquote & & x \nameeq \lpquote P \rpquote \\
      x & & otherwise \\
    \end{array}
  \right. \nonumber
\end{eqnarray}

and $z$ is chosen distinct from $\quotep{P}$, $\quotep{Q}$, the free
names in $Q$, and all the names in $R$. Our $\alpha$-equivalence will
be built in the standard way from this substitution.

\begin{remark}\label{rem:no_self_referential_names}
  One consequence of these definitions is that $\forall P. \quotep{P}
  \not\in \freenames{P}$.
\end{remark}

\subsection{ Dynamic quote: an example }

Anticipating something of what's to come, consider applying the
substitution, $\widehat{\id{\{}u / z \id{\}}}$, to the following pair
of processes, $\lift{w}{y!(z)}$ and $w[ \lpquote y!(z) \rpquote ]$.

\begin{eqnarray}
	\lift{w}{y!(z)}\widehat{\id{\{}u / z \id{\}}}
		& = &
		\lift{w}{y!(u)} \nonumber\\
	w[ \lpquote y!(z) \rpquote ] \widehat{ \id{\{}u / z \id{\}} }
		& = &
		w[ \lpquote y!(z) \rpquote ] \nonumber
\end{eqnarray}

Because the body of the process between quotes is impervious to
substitution, we get radically different answers. In fact, by
examining the first process in an input context,
e.g. $x?(z).\lift{w}{y!(z)}$, we see that the process under the lift
operator may be shaped by prefixed inputs binding a name inside it. In
this sense, the lift operator will be seen as a way to dynamically
construct processes before reifying them as names.

Finally equipped with these standard features we can present the
dynamics of the calculus.

\subsubsection{Operational semantics} 

Finally, we introduce the computational dynamics. What marks these
algebras as distinct from other more traditionally studied algebraic
structures, e.g. vector spaces or polynomial rings, is the manner in
which dynamics is captured. In traditional structures, dynamics is typically
expressed through morphisms between such structures, as in linear maps
between vector spaces or morphisms between rings. In algebras
associated with the semantics of computation, the dynamics is
expressed as part of the algebraic structure itself, through a
reduction reduction relation typically denoted by $\red$. Below, we
give a recursive presentation of this relation for the calculus used
in the encoding.

$\red \subseteq \pi \times \pi$
$\red : \pi \to \mathcal{P}(\pi)$

\begin{mathpar}
  \inferrule* [lab=Comm] { \textsf{match}( x_{src}, x_{trgt} ) } { x_{trgt}?(y)P \; | \; x_{src}!\langle {Q} \rangle \red P\{\quotep{Q}/y}\} }
  \and \\
  \inferrule* [lab=Par] {{P} \red {P}'} {{{P} | {Q}} \red {{P}' | {Q}}}
  \and
  \inferrule* [lab=Equiv]{{{P} \scong {P}'} \andalso {{P}' \red {Q}'} \andalso {{Q}' \scong {Q}}}{{P} \red {Q}}
\end{mathpar}

\begin{eqnarray*}
  match_{\equiv} (\quotep{P},\quotep{Q}) & := & P \equiv Q \\
  match_{\dagger}(\quotep{P},\quotep{Q}) & := & \forall R. P|Q \red^{*} R => R \red^{*} 0 \\
  match_{K}(\quotep{P},\quotep{Q}) & := & K \mbox{ for some context } K
\end{eqnarray*}

$u?(x)P | u!\langle Q \rangle \red P\{\quotep{Q}/x\}$

%We write $\wred$ for $\red^*$, and $P\red$ if $\exists Q $ such that $ P \red Q$.
We write $P\red$ if $\exists Q $ such that $ P \red Q$ and $P\not\red$, otherwise.

\section{Replication}

As mentioned before, it is known that replication (and hence
recursion) can be implemented in a higher-order process algebra
\cite{SangiorgiWalker}. As our first example of calculation with the
machinery thus far presented we give the construction explicitly in
the {\rhoc}.

\begin{eqnarray}
	D_{x} & := & \prefix{x}{y}{(\binpar{\outputp{x}{y}}{@{y}})} \nonumber\\
	\bangp_{x}{P} & := & \binpar{{x}!\langle{\binpar{D_{x}}{P}}\rangle}{D_{x}} \nonumber
\end{eqnarray}

\begin{eqnarray}
	\bangp_{x}{P} & & \nonumber\\
	=
	& {x}!\langle{(\prefix{x}{y}{(\outputp{x}{y} | @{y})) | P}}\rangle 
	      | \prefix{x}{y}{(\outputp{x}{y} | @{y})} & \nonumber\\
	\red
	& (\outputp{x}{y} | @{y})\substn{\quotep{(\prefix{x}{y}{(@{y} | \outputp{x}{y})) | P}}}{y} & \nonumber\\
	=
	& \outputp{x}{\quotep{(\prefix{x}{y}{(\outputp{x}{y} | @{y})) | P}}}
	  | {(\prefix{x}{y}{(\outputp{x}{y} | @{y})) | P}} & \nonumber\\
	\red
	& \ldots & \nonumber\\
	\red^*
	& P | P | \ldots & \nonumber
\end{eqnarray}

Of course, this encoding, as an implementation, runs away, unfolding
$\bangp{P}$ eagerly. A lazier and more implementable replication
operator, restricted to input-guarded processes, may be obtained as follows.

\begin{eqnarray}
\bangp{\prefix{u}{v}{P}} 
	:= 
	\binpar{\lift{x}{\prefix{u}{v}{(\binpar{D(x)}{P})}}}{D(x)} \nonumber
\end{eqnarray}

\begin{remark}
  Note that the lazier definition still does not deal with summation
  or mixed summation (i.e. sums over input and output). The reader is
  invited to construct definitions of replication that deal with these
  features. 

  Further, the definitions are parameterized in a name, $x$. Can you,
  gentle reader, make a definition that eliminates this parameter and
  guarantees no accidental interaction between the replication
  machinery and the process being replicated -- i.e. no accidental
  sharing of names used by the process to get its work done and the
  name(s) used by the replication to effect copying. This latter
  revision of the definition of replication is crucial to obtaining
  the expected identity $!!P \sim !P$.
\end{remark}

\begin{remark}\label{rem:paradoxical_combinator}
  The reader familiar with the lambda calculus will have noticed the
  similarity between $D$ and the paradoxical combinator.

  [Ed. note: the existence of this seems to suggest we have to be more
  restrictive on the set of processes and names we admit if we are to
  support no-cloning.]
\end{remark}

\subsubsection{Bisimulation}

The computational dynamics gives rise to another kind of equivalence,
the equivalence of computational behavior. As previously mentioned
this is typically captured \emph{via} some form of bisimulation.

% The notion we use in this paper is weak barbed bisimulation
% \cite{milner91polyadicpi}.

The notion we use in this paper is derived from weak barbed
bisimulation \cite{milner91polyadicpi}. 

\begin{definition}
An \emph{observation relation}, $\downarrow_{\mathcal N}$, over a set
of names, $\mathcal N$, is the smallest relation satisfying the rules
below.

\infrule[Out-barb]{y \in {\mathcal N}, \; x \nameeq y}
		  {\outputp{x}{v} \downarrow_{\mathcal N} x}
\infrule[Par-barb]{\mbox{$P\downarrow_{\mathcal N} x$ or $Q\downarrow_{\mathcal N} x$}}
		  {\binpar{P}{Q} \downarrow_{\mathcal N} x}

We write $P \Downarrow_{\mathcal N} x$ if there is $Q$ such that 
$P \wred Q$ and $Q \downarrow_{\mathcal N} x$.
\end{definition}

\begin{definition}
%\label{def.bbisim}
An  ${\mathcal N}$-\emph{barbed bisimulation} over a set of names, ${\mathcal N}$, is a symmetric binary relation 
${\mathcal S}_{\mathcal N}$ between agents such that $P\rel{S}_{\mathcal N}Q$ implies:
\begin{enumerate}
\item If $P \red P'$ then $Q \wred Q'$ and $P'\rel{S}_{\mathcal N} Q'$.
\item If $P\downarrow_{\mathcal N} x$, then $Q\Downarrow_{\mathcal N} x$.
\end{enumerate}
$P$ is ${\mathcal N}$-barbed bisimilar to $Q$, written
$P \wbbisim_{\mathcal N} Q$, if $P \rel{S}_{\mathcal N} Q$ for some ${\mathcal N}$-barbed bisimulation ${\mathcal S}_{\mathcal N}$.
\end{definition}

$\mathcal{R} \subseteq \pi \times \pi$

$P \mathcal{R} Q => \forall P'. P \red P' \Rightarrow \exists Q'. Q \red Q', P' \mathcal{R} Q'$

$P \vdash x \Rightarrow Q \vdash x$

\begin{mathpar}
  \inferrule*[lab=Out-barb]{x \nameeq y}{{y}!\langle{Q}\rangle \vdash x}
  \and
  \inferrule*[lab=Par-barb]{\mbox{$P\vdash x$ or $Q\vdash x$}}{\binpar{P}{Q} \vdash x}
\end{mathpar}

\subsubsection{Contexts}

One of the principle advantages of computational calculi like the
$\pi$-calculus is a well-defined notion of context,
contextual-equivalence and a correlation between
contextual-equivalence and notions of bisimulation. The notion of
context allows the decomposition of a process into (sub-)process and
its syntactic environment, its context. Thus, a context may be
thought of as a process with a ``hole'' (written $\Box$) in it. The
application of a context $M$ to a process $P$, written $M[P]$, is
tantamount to filling the hole in $M$ with $P$. In this paper we do
not need the full weight of this theory, but do make use of the notion
of context in the proof the main theorem. 

\begin{mathpar}
  \inferrule* [lab=summation] {} {{M_{M},M_{N}} \bc \Box \;|\; x.M_{A} \;|\; M_{M}+M_{N}}
  \and
  \inferrule* [lab=agent] {} {{M_{A}} \bc (\vec{x})M_{P} \;| \; \clift{P_0,\ldots,M_{P},\ldots,P_N}}
  \and \\
  \inferrule* [lab=process] {} {{M_{P}} \bc M_{N} \;| \;P|M_{P} }
\end{mathpar} 

\begin{mathpar}
  \inferrule* [lab=sychronization] {} {M_{N} \bc \Box \;|\; x?M_{F} \;|\; x!M_{C}}
  \and
  \inferrule* [lab=abstraction] {} {{M_{F}} \bc (x)M_{P} }
  \and
  \inferrule* [lab=concretion] {} {{M_{C}} \bc \langle M_{P} \rangle }
  \and \\
  \inferrule* [lab=process] {} {{M_{P}} \bc M_{N} \;| \;P|M_{P} }
\end{mathpar}

\begin{definition}[contextual application] Given a context $M$, and
  process $P$, we define the \emph{contextual application}, $M[P] :=
  M\{P/\Box\}$. That is, the contextual application of M to P is the
  substitution of $P$ for $\Box$ in $M$.
\end{definition}

$\meaningof{-} : L \to \mathcal{P}(\pi)$

\begin{mathpar}
  \inferrule* [lab=collection] {} {\meaningof{true} = \pi, \and \meaningof{~E} = \pi \setminus \meaningof{E}, \and \meaningof{E_{1} \& E_{2}} = \meaningof{E_{1}} \cap \meaningof{E_{2}}}
\end{mathpar}

\begin{mathpar}
  \inferrule* [lab=structure] {} {\meaningof{0} = \{ P \in \pi | P \equiv 0 \}, \and \\ \meaningof{E_1 | E_2} = \{ P \in \pi | P \equiv P_{1} | P_{2}, P_{1} \in \meaningof{E_{1}}, P_{2} \in \meaningof{E_2}\} }
\end{mathpar}

\begin{mathpar}
 \inferrule* [lab=behavior] {} {\meaningof{\langle a?b \rangle E} = \{ P \in \pi | P \equiv Q | u?(y)P', \\ \and \\\\ \and \\ \;\;\; u \in \meaningof{a}, \forall z.P'\{z/y\} \in \meaningof{E\{z/b\}}\}, \and \\ \meaningof{a!E} = \{ P \in \pi | P \equiv Q | x!\langle P' \rangle, x \in \meaningof{a} P' \in \meaningof{E}\} }
\end{mathpar}

\begin{mathpar}
 \inferrule* [lab=nominal] {} {\meaningof{\quotep{E}} = \{ \quotep{P} \in \quotep{\pi} | P \in \meaningof{E} \}, \and \meaningof{\quotep{P}} = \{ \quotep{Q} \in \quotep{\pi} | P \equiv Q \} \and \\ \meaningof{@\quotep{E}} = \{ P \in \pi | P \equiv @x, x \in \meaningof{E} \}}
\end{mathpar}

\begin{eqnarray*}
  \\
  \meaningof{-} : TS \to ST
\end{eqnarray*}

\begin{eqnarray*}
  \\
  L : TS \to ST
\end{eqnarray*}

\begin{eqnarray*}
  \\
  P \models E \iff P \in \meaningof{E}
\end{eqnarray*}

\begin{eqnarray*}
  P \approx_{L} Q \iff \forall E \in L. P \models E \iff Q \models E
\end{eqnarray*}

\begin{eqnarray*}
  P \approx_{K} Q
\end{eqnarray*}

\begin{eqnarray*}
  P \approx Q
\end{eqnarray*}

$\approx_{K} = \approx = \approx_{L}$

\subsubsection{Contextual duality}

Note that contexts extend the quotation operation to a family of
operations from processes to names. Given a context, $M$, we can
define a \emph{nominal context}, $\quotep{M}$ by $\quotep{M}[P] :=
\quotep{M[P]}$. To foreshadow what is to come we observe that these
operations enjoy a duality with processes very much like the duality
between vectors and maps from vectors to scalars.

Further, because the calculus is essentially higher-order, we have a
correspondence between contexts and processes. More specifically,
given a name $x$ and a context $M$ we can construct $M^{*}_{x}$ such
that 

\begin{mathpar}
  M^{*}_{x} | \lift{x}{P} \red M[P]
\end{mathpar}

namely,

\begin{mathpar}
  M^{*}_{x} := x?(u).M[\dropn{u}]
\end{mathpar}

The dependence of $M^{*}_{x}$ on a name makes it an abstraction, 

\begin{mathpar}
  M^{*} := (x)x?(u).M[\dropn{u}]
\end{mathpar}

\subsection{Additional notation}

It will sometimes be convenient to denote the process a name
quotes. We already have the notation $x = \quotep{P}$, but it will be
convenient to introduce an alternate notation, $\procn{x}$, when we
want to emphasize the connection to the use of the name. Note that, by
virtue of name equivalence, $\quotep{\procn{x}} \nameeq x$; so, the
notation is consistent with previous definitions.

Further, because names have structure it is possible to effect
substitutions on the basis of that structure. This means we need to
upgrade our notation for substitutions, which we accomplish by
adapting comprehension notation. Thus,

\begin{mathpar}
  P\{ y / x : x \in S \}
\end{mathpar}

is interpreted to mean the process derived from P by replacing (in a
capture-avoiding manner) each occurrence of $x$ in $S$ by $y$. For example,

\begin{mathpar}
  P\{ \quotep{\procn{x}|\procn{x}} / x : x \in \freenames{P} \}
\end{mathpar}

will replace each (occurrence) of a free name $x$ in $P$ by
$\quotep{\procn{x}|\procn{x}}$.

Also, we will avail ourselves of the notation $x^{L}$ and $x^{R}$ to
denote injections of a name into disjoint copies of the name
space. There are numerous ways to accomplish this. One example can be
found in \cite{MeredithR05}. This notation overloads to vectors of
names: $\vec{x}^{\pi} := (x_{i}^{\pi} \; : \; 0 \leq i < |\vec{x}| )$ where $\pi \in \{L,R\}$.

We also use $P^{\Box} := P|\Box$.

In \cite{MeredithR05} an interpretation of the new operator is
given. It turns out that there are several possible interpretations
all enjoying the requisite algebraic properties of the operator (see
\cite{milner91polyadicpi}). We will therefore make liberal use of
$(\nu\; \vec{x})P$.

% subsection the_syntax_and_semantics_of_the_notation_system (end)   

\input{qm2pi.qmops} 

\input{qm2pi.sterngerlach} 

\input{qm2pi.metric} 

% section concurrent_process_calculi (end)

%\input{qm2pi.proofsketch}

% section proof sketch (end)

%\input{qm2pi.slviaknots} 

% section spatial logic via knots (end)

\input{qm2pi.conclusion}

% section conclusion (end)

%\input{qm2pi.dtcodes} 

% section wiring algorithm (end)

\input{qm2pi.ack} 

% section acknowledgments (end)

\newpage


\bibliographystyle{plain}   
\bibliography{../../biblios/main.bib}

\input{qm2pi.rhodetails}

\end{document}

 

% section concurrent_process_calculi (end)

%\documentclass[12pt]{llncs}
%\documentclass{jktr}

\usepackage[pdftex]{hyperref}                   
\usepackage {listings}
\usepackage {mathpartir}
\usepackage{bcprules}
%\usepackage{listings}
                       
\usepackage{graphicx} 
%\usepackage[margins=2.5cm,nohead,nofoot]{geometry}
%\usepackage{geometry}
\usepackage{amsfonts}
\usepackage{amstext}
\usepackage{latexsym}
\usepackage{amssymb}
\usepackage{color}


%\include{myPreamble}
\include{qm2pi.local} 

%\ifpdf
%\usepackage[pdftex]{graphicx}
%\else
%\usepackage{graphicx}
%\fi

 % \ifpdf
%  \usepackage{pdfsync}
%  \if


%\title{Brief Article}
%\author{David F. Snyder}
%\author{L.G. Meredith}

%\address{Dept. of Math., Texas State University--San Marcos, San Marcos, TX 78666}
       
\pagestyle{empty}


\begin{document}

\lstset{language=[Objective]Caml,frame=shadowbox}

\input{qm2pi.front}

% section front matter (end)

\input{qm2pi.intro} 
 
% section introduction (end)

% \input{qm2pi.knotations} 

% section notation (end)

\input{qm2pi.process.calculi} 

% section concurrent_process_calculi_and_spatial_logics_ (end)
    
%\input{qm2pi.knots2pi} 

%\input{qm2pi.trefoil} 

%\input{qm2pi.mainthm} 

% subsection basic_interpretation (end)

%\input{qm2pi.rho.presentation} 
\subsection{The syntax and semantics of the notation system}\label{sub:the_syntax_and_semantics_of_the_notation_system} % (fold)

We now summarize a technical presentation of the calculus that
embodies our theory of dynamics. The typical presentation of such a
calculus follows the style of giving generators and relations on
them. The grammar, below, describing term constructors, freely
generates the set of processes, $\Proc$. This set is then quotiented
by a relation known as structural congruence and it is over this set
that the notion of dynamics is expressed. This presentation is
essentially that of \cite{MeredithR05} with the addition of
polyadicity and summation. For readability we have relegated some of
the technical subtleties to an appendix.

\subsubsection{Process grammar}\label{subsub:process_grammar}

\begin{mathpar}
  \inferrule* [lab=synchronization] {} {{M} \bc \pzero \;|\; x?F \;|\; x!C }
  \and
  \inferrule* [lab=abstraction] {} {{F} \bc (x)P}
  \and
  \inferrule* [lab=concretion] {} {{C} \bc \langle Q \rangle}
  \and
  \inferrule* [lab=process] {} {{P,Q} \bc M \;| \;P|Q \;|\; @{x}}
  \and
  \inferrule* [lab=name] {} {{x} \bc \quotep{P}}
\end{mathpar} 

Note that $\vec{x}$ (resp. $\vec{P}$) denotes a vector of names
(resp. processes) of length $|\vec{x}|$ (resp. $|\vec{P}|$). We adopt
the following useful abbreviations.

\begin{mathpar}
   x?(\vec{y}).P := x.(\vec{y})P \and  x\clift{\vec{P}} := x.\clift{\vec{P}}
   \and x!(y) := \lift{x}{\dropn{y}}
   \and \Pi_{i=0}^{n-1}P_i := P_0 | \ldots | P_{n-1}
\end{mathpar}

\subsubsection{Structural congruence}

\paragraph{Free and bound names and alpha-equivalence.} At the
core of structural equivalence is alpha-equivalence which identifies
process that are the same up to a change of variable. Formally, we
recognize the distinction between free and bound names. The free names
of a process, $\freenames{P}$, may be calculated recursively as
follows:

\begin{mathpar}
\freenames{\pzero} := \emptyset
  \and \\
  \freenames{x?(y).P} := \{ x \} \cup (\freenames{P} \setminus \{ y \})
  \and 
  \freenames{x!\langle P \rangle} := \{ x \} \cup \{ P \} 
  \and \\
  \freenames{P|Q} := \freenames{P} \cup \freenames{Q}
  \and \\
  \freenames{@{x}} := \{ x \}
\end{mathpar}

$\pi$
$\quotep{\pi}$

$\freenames{-} : \pi \to \mathcal{P}(\quotep{\pi})$

\begin{eqnarray*}
  \freenames{\pzero} & := & \emptyset \\
  \freenames{x?(y).P} & := & \{ x \} \cup (\freenames{P} \setminus \{ y \}) \\
  \freenames{x!\langle P \rangle} & := & \{ x \} \cup \{ P \} \\
  \freenames{P|Q} & := & \freenames{P} \cup \freenames{Q} \\
  \freenames{\dropn{x}} & := & \{ x \}
\end{eqnarray*}

The bound names of a process, $\boundnames{P}$, are those names occurring in $P$
that are not free. For example, in $x?(y).0$, the name $x$ is free, while $y$ is bound.

\begin{mathpar}
  \inferrule* [lab=monoidal-laws] {} { P|Q \equiv Q|P \and P|0 \equiv P \and P|(Q|R) \equiv (P|Q)|R }
\end{mathpar}

\begin{mathpar}
  \inferrule* [lab=alpha-equivalence] {} { (x)P \equiv (y)P\{y/x\} \and y \not\in \freenames{P} }
\end{mathpar}

\begin{definition}
Then two processes, $P,Q$, are alpha-equivalent if $P = Q\{\vec{y}/\vec{x}\}$ for
some $\vec{x} \in \boundnames{Q},\vec{y} \in \boundnames{P}$, where $Q\{\vec{y}/\vec{x}\}$
denotes the capture-avoiding substitution of $\vec{y}$ for $\vec{x}$ in $Q$.
\end{definition}

\begin{definition}
  The {\em structural congruence} \cite{SangiorgiWalker} , $\equiv$,
  between processes is the least congruence containing
  alpha-equivalence, satisfying the abelian monoid laws
  (associativity, commutativity and $\pzero$ as identity) for parallel
  composition $|$ and for summation $+$.
\end{definition}

\subsection{Name equivalence}

We take name equivalence, written $\nameeq$, to be the smallest
equivalence relation generated by the following rules.

\begin{mathpar}
\inferrule*[lab=Quote-drop]
{ }
{ \quotep{@{x}} \nameeq x }

\inferrule*[lab=Struct-equiv]
{ P \scong Q }
{ \quotep{P} \nameeq \quotep{Q} }
\end{mathpar}

The astute reader will have noticed that the mutual recursion of names
and processes imposes a mutual recursion on alpha-equivalence and
structural equivalence via name-equivalence. Fortunately, all of this
works out pleasantly and we may calculate in the natural way, free of
concern. The reader interested in the details is referred to the
appendix \ref{appendix:rho_details}.

\subsection{Substitution}

We use $\Proc$ for the set of processes, $\QProc$ for the set of
names, and $\id{\{}\vec{y} / \vec{x} \id{\}}$ to denote partial maps,
$s : \QProc \rightarrow \QProc$. A map, $s$ lifts, uniquely, to a map
on process terms, $\widehat{s} : \Proc \rightarrow \Proc$ by the
following equations.

\begin{mathpar}
  (0) \psubstp{Q}{P} := 0 \\
  (R \juxtap S) \psubstp{Q}{P}
  :=    
  (R)\psubstp{Q}{P} \juxtap (S) \psubstp{Q}{P} \\
  (x?(y).R) \psubstp{Q}{P}    
  :=    
  (x)\substp{Q}{P} (z)\concat( (R \psubstn{z}{y}) \psubstp{Q}{P} ) \\
  (\lift{x}{R}) \psubstp{Q}{P}  
  :=
  \lift{(x)\substp{Q}{P}}{ R \psubstp{Q}{P} } \\
%   (\dropn{x})  \psubstp{Q}{P}       
%   := 
%   \left\{ 
%     \begin{array}{ccc} 
%       \dropn{\quotep{Q}} & & x \nameeq \quotep{P} \\
%       \dropn{x} & & otherwise \\
%     \end{array}
%   \right. 
  (\dropn{x})  \psubstp{Q}{P}       
  := 
  \left\{ 
    \begin{array}{ccc} 
      Q & & x \nameeq \quotep{P} \\
      \dropn{x} & & otherwise \\
    \end{array}
  \right.
\end{mathpar}
 

where

\begin{eqnarray}
  (x)\id{\{} \lpquote Q \rpquote / \lpquote P \rpquote \id{\}}            = 
  \left\{ 
    \begin{array}{ccc}
      \lpquote Q \rpquote & & x \nameeq \lpquote P \rpquote \\
      x & & otherwise \\
    \end{array}
  \right. \nonumber
\end{eqnarray}

and $z$ is chosen distinct from $\quotep{P}$, $\quotep{Q}$, the free
names in $Q$, and all the names in $R$. Our $\alpha$-equivalence will
be built in the standard way from this substitution.

\begin{remark}\label{rem:no_self_referential_names}
  One consequence of these definitions is that $\forall P. \quotep{P}
  \not\in \freenames{P}$.
\end{remark}

\subsection{ Dynamic quote: an example }

Anticipating something of what's to come, consider applying the
substitution, $\widehat{\id{\{}u / z \id{\}}}$, to the following pair
of processes, $\lift{w}{y!(z)}$ and $w[ \lpquote y!(z) \rpquote ]$.

\begin{eqnarray}
	\lift{w}{y!(z)}\widehat{\id{\{}u / z \id{\}}}
		& = &
		\lift{w}{y!(u)} \nonumber\\
	w[ \lpquote y!(z) \rpquote ] \widehat{ \id{\{}u / z \id{\}} }
		& = &
		w[ \lpquote y!(z) \rpquote ] \nonumber
\end{eqnarray}

Because the body of the process between quotes is impervious to
substitution, we get radically different answers. In fact, by
examining the first process in an input context,
e.g. $x?(z).\lift{w}{y!(z)}$, we see that the process under the lift
operator may be shaped by prefixed inputs binding a name inside it. In
this sense, the lift operator will be seen as a way to dynamically
construct processes before reifying them as names.

Finally equipped with these standard features we can present the
dynamics of the calculus.

\subsubsection{Operational semantics} 

Finally, we introduce the computational dynamics. What marks these
algebras as distinct from other more traditionally studied algebraic
structures, e.g. vector spaces or polynomial rings, is the manner in
which dynamics is captured. In traditional structures, dynamics is typically
expressed through morphisms between such structures, as in linear maps
between vector spaces or morphisms between rings. In algebras
associated with the semantics of computation, the dynamics is
expressed as part of the algebraic structure itself, through a
reduction reduction relation typically denoted by $\red$. Below, we
give a recursive presentation of this relation for the calculus used
in the encoding.

$\red \subseteq \pi \times \pi$
$\red : \pi \to \mathcal{P}(\pi)$

\begin{mathpar}
  \inferrule* [lab=Comm] { \textsf{match}( x_{src}, x_{trgt} ) } { x_{trgt}?(y)P \; | \; x_{src}!\langle {Q} \rangle \red P\{\quotep{Q}/y}\} }
  \and \\
  \inferrule* [lab=Par] {{P} \red {P}'} {{{P} | {Q}} \red {{P}' | {Q}}}
  \and
  \inferrule* [lab=Equiv]{{{P} \scong {P}'} \andalso {{P}' \red {Q}'} \andalso {{Q}' \scong {Q}}}{{P} \red {Q}}
\end{mathpar}

\begin{eqnarray*}
  match_{\equiv} (\quotep{P},\quotep{Q}) & := & P \equiv Q \\
  match_{\dagger}(\quotep{P},\quotep{Q}) & := & \forall R. P|Q \red^{*} R => R \red^{*} 0 \\
  match_{K}(\quotep{P},\quotep{Q}) & := & K \mbox{ for some context } K
\end{eqnarray*}

$u?(x)P | u!\langle Q \rangle \red P\{\quotep{Q}/x\}$

%We write $\wred$ for $\red^*$, and $P\red$ if $\exists Q $ such that $ P \red Q$.
We write $P\red$ if $\exists Q $ such that $ P \red Q$ and $P\not\red$, otherwise.

\section{Replication}

As mentioned before, it is known that replication (and hence
recursion) can be implemented in a higher-order process algebra
\cite{SangiorgiWalker}. As our first example of calculation with the
machinery thus far presented we give the construction explicitly in
the {\rhoc}.

\begin{eqnarray}
	D_{x} & := & \prefix{x}{y}{(\binpar{\outputp{x}{y}}{@{y}})} \nonumber\\
	\bangp_{x}{P} & := & \binpar{{x}!\langle{\binpar{D_{x}}{P}}\rangle}{D_{x}} \nonumber
\end{eqnarray}

\begin{eqnarray}
	\bangp_{x}{P} & & \nonumber\\
	=
	& {x}!\langle{(\prefix{x}{y}{(\outputp{x}{y} | @{y})) | P}}\rangle 
	      | \prefix{x}{y}{(\outputp{x}{y} | @{y})} & \nonumber\\
	\red
	& (\outputp{x}{y} | @{y})\substn{\quotep{(\prefix{x}{y}{(@{y} | \outputp{x}{y})) | P}}}{y} & \nonumber\\
	=
	& \outputp{x}{\quotep{(\prefix{x}{y}{(\outputp{x}{y} | @{y})) | P}}}
	  | {(\prefix{x}{y}{(\outputp{x}{y} | @{y})) | P}} & \nonumber\\
	\red
	& \ldots & \nonumber\\
	\red^*
	& P | P | \ldots & \nonumber
\end{eqnarray}

Of course, this encoding, as an implementation, runs away, unfolding
$\bangp{P}$ eagerly. A lazier and more implementable replication
operator, restricted to input-guarded processes, may be obtained as follows.

\begin{eqnarray}
\bangp{\prefix{u}{v}{P}} 
	:= 
	\binpar{\lift{x}{\prefix{u}{v}{(\binpar{D(x)}{P})}}}{D(x)} \nonumber
\end{eqnarray}

\begin{remark}
  Note that the lazier definition still does not deal with summation
  or mixed summation (i.e. sums over input and output). The reader is
  invited to construct definitions of replication that deal with these
  features. 

  Further, the definitions are parameterized in a name, $x$. Can you,
  gentle reader, make a definition that eliminates this parameter and
  guarantees no accidental interaction between the replication
  machinery and the process being replicated -- i.e. no accidental
  sharing of names used by the process to get its work done and the
  name(s) used by the replication to effect copying. This latter
  revision of the definition of replication is crucial to obtaining
  the expected identity $!!P \sim !P$.
\end{remark}

\begin{remark}\label{rem:paradoxical_combinator}
  The reader familiar with the lambda calculus will have noticed the
  similarity between $D$ and the paradoxical combinator.

  [Ed. note: the existence of this seems to suggest we have to be more
  restrictive on the set of processes and names we admit if we are to
  support no-cloning.]
\end{remark}

\subsubsection{Bisimulation}

The computational dynamics gives rise to another kind of equivalence,
the equivalence of computational behavior. As previously mentioned
this is typically captured \emph{via} some form of bisimulation.

% The notion we use in this paper is weak barbed bisimulation
% \cite{milner91polyadicpi}.

The notion we use in this paper is derived from weak barbed
bisimulation \cite{milner91polyadicpi}. 

\begin{definition}
An \emph{observation relation}, $\downarrow_{\mathcal N}$, over a set
of names, $\mathcal N$, is the smallest relation satisfying the rules
below.

\infrule[Out-barb]{y \in {\mathcal N}, \; x \nameeq y}
		  {\outputp{x}{v} \downarrow_{\mathcal N} x}
\infrule[Par-barb]{\mbox{$P\downarrow_{\mathcal N} x$ or $Q\downarrow_{\mathcal N} x$}}
		  {\binpar{P}{Q} \downarrow_{\mathcal N} x}

We write $P \Downarrow_{\mathcal N} x$ if there is $Q$ such that 
$P \wred Q$ and $Q \downarrow_{\mathcal N} x$.
\end{definition}

\begin{definition}
%\label{def.bbisim}
An  ${\mathcal N}$-\emph{barbed bisimulation} over a set of names, ${\mathcal N}$, is a symmetric binary relation 
${\mathcal S}_{\mathcal N}$ between agents such that $P\rel{S}_{\mathcal N}Q$ implies:
\begin{enumerate}
\item If $P \red P'$ then $Q \wred Q'$ and $P'\rel{S}_{\mathcal N} Q'$.
\item If $P\downarrow_{\mathcal N} x$, then $Q\Downarrow_{\mathcal N} x$.
\end{enumerate}
$P$ is ${\mathcal N}$-barbed bisimilar to $Q$, written
$P \wbbisim_{\mathcal N} Q$, if $P \rel{S}_{\mathcal N} Q$ for some ${\mathcal N}$-barbed bisimulation ${\mathcal S}_{\mathcal N}$.
\end{definition}

$\mathcal{R} \subseteq \pi \times \pi$

$P \mathcal{R} Q => \forall P'. P \red P' \Rightarrow \exists Q'. Q \red Q', P' \mathcal{R} Q'$

$P \vdash x \Rightarrow Q \vdash x$

\begin{mathpar}
  \inferrule*[lab=Out-barb]{x \nameeq y}{{y}!\langle{Q}\rangle \vdash x}
  \and
  \inferrule*[lab=Par-barb]{\mbox{$P\vdash x$ or $Q\vdash x$}}{\binpar{P}{Q} \vdash x}
\end{mathpar}

\subsubsection{Contexts}

One of the principle advantages of computational calculi like the
$\pi$-calculus is a well-defined notion of context,
contextual-equivalence and a correlation between
contextual-equivalence and notions of bisimulation. The notion of
context allows the decomposition of a process into (sub-)process and
its syntactic environment, its context. Thus, a context may be
thought of as a process with a ``hole'' (written $\Box$) in it. The
application of a context $M$ to a process $P$, written $M[P]$, is
tantamount to filling the hole in $M$ with $P$. In this paper we do
not need the full weight of this theory, but do make use of the notion
of context in the proof the main theorem. 

\begin{mathpar}
  \inferrule* [lab=summation] {} {{M_{M},M_{N}} \bc \Box \;|\; x.M_{A} \;|\; M_{M}+M_{N}}
  \and
  \inferrule* [lab=agent] {} {{M_{A}} \bc (\vec{x})M_{P} \;| \; \clift{P_0,\ldots,M_{P},\ldots,P_N}}
  \and \\
  \inferrule* [lab=process] {} {{M_{P}} \bc M_{N} \;| \;P|M_{P} }
\end{mathpar} 

\begin{mathpar}
  \inferrule* [lab=sychronization] {} {M_{N} \bc \Box \;|\; x?M_{F} \;|\; x!M_{C}}
  \and
  \inferrule* [lab=abstraction] {} {{M_{F}} \bc (x)M_{P} }
  \and
  \inferrule* [lab=concretion] {} {{M_{C}} \bc \langle M_{P} \rangle }
  \and \\
  \inferrule* [lab=process] {} {{M_{P}} \bc M_{N} \;| \;P|M_{P} }
\end{mathpar}

\begin{definition}[contextual application] Given a context $M$, and
  process $P$, we define the \emph{contextual application}, $M[P] :=
  M\{P/\Box\}$. That is, the contextual application of M to P is the
  substitution of $P$ for $\Box$ in $M$.
\end{definition}

$\meaningof{-} : L \to \mathcal{P}(\pi)$

\begin{mathpar}
  \inferrule* [lab=collection] {} {\meaningof{true} = \pi, \and \meaningof{~E} = \pi \setminus \meaningof{E}, \and \meaningof{E_{1} \& E_{2}} = \meaningof{E_{1}} \cap \meaningof{E_{2}}}
\end{mathpar}

\begin{mathpar}
  \inferrule* [lab=structure] {} {\meaningof{0} = \{ P \in \pi | P \equiv 0 \}, \and \\ \meaningof{E_1 | E_2} = \{ P \in \pi | P \equiv P_{1} | P_{2}, P_{1} \in \meaningof{E_{1}}, P_{2} \in \meaningof{E_2}\} }
\end{mathpar}

\begin{mathpar}
 \inferrule* [lab=behavior] {} {\meaningof{\langle a?b \rangle E} = \{ P \in \pi | P \equiv Q | u?(y)P', \\ \and \\\\ \and \\ \;\;\; u \in \meaningof{a}, \forall z.P'\{z/y\} \in \meaningof{E\{z/b\}}\}, \and \\ \meaningof{a!E} = \{ P \in \pi | P \equiv Q | x!\langle P' \rangle, x \in \meaningof{a} P' \in \meaningof{E}\} }
\end{mathpar}

\begin{mathpar}
 \inferrule* [lab=nominal] {} {\meaningof{\quotep{E}} = \{ \quotep{P} \in \quotep{\pi} | P \in \meaningof{E} \}, \and \meaningof{\quotep{P}} = \{ \quotep{Q} \in \quotep{\pi} | P \equiv Q \} \and \\ \meaningof{@\quotep{E}} = \{ P \in \pi | P \equiv @x, x \in \meaningof{E} \}}
\end{mathpar}

\begin{eqnarray*}
  \\
  \meaningof{-} : TS \to ST
\end{eqnarray*}

\begin{eqnarray*}
  \\
  L : TS \to ST
\end{eqnarray*}

\begin{eqnarray*}
  \\
  P \models E \iff P \in \meaningof{E}
\end{eqnarray*}

\begin{eqnarray*}
  P \approx_{L} Q \iff \forall E \in L. P \models E \iff Q \models E
\end{eqnarray*}

\begin{eqnarray*}
  P \approx_{K} Q
\end{eqnarray*}

\begin{eqnarray*}
  P \approx Q
\end{eqnarray*}

$\approx_{K} = \approx = \approx_{L}$

\subsubsection{Contextual duality}

Note that contexts extend the quotation operation to a family of
operations from processes to names. Given a context, $M$, we can
define a \emph{nominal context}, $\quotep{M}$ by $\quotep{M}[P] :=
\quotep{M[P]}$. To foreshadow what is to come we observe that these
operations enjoy a duality with processes very much like the duality
between vectors and maps from vectors to scalars.

Further, because the calculus is essentially higher-order, we have a
correspondence between contexts and processes. More specifically,
given a name $x$ and a context $M$ we can construct $M^{*}_{x}$ such
that 

\begin{mathpar}
  M^{*}_{x} | \lift{x}{P} \red M[P]
\end{mathpar}

namely,

\begin{mathpar}
  M^{*}_{x} := x?(u).M[\dropn{u}]
\end{mathpar}

The dependence of $M^{*}_{x}$ on a name makes it an abstraction, 

\begin{mathpar}
  M^{*} := (x)x?(u).M[\dropn{u}]
\end{mathpar}

\subsection{Additional notation}

It will sometimes be convenient to denote the process a name
quotes. We already have the notation $x = \quotep{P}$, but it will be
convenient to introduce an alternate notation, $\procn{x}$, when we
want to emphasize the connection to the use of the name. Note that, by
virtue of name equivalence, $\quotep{\procn{x}} \nameeq x$; so, the
notation is consistent with previous definitions.

Further, because names have structure it is possible to effect
substitutions on the basis of that structure. This means we need to
upgrade our notation for substitutions, which we accomplish by
adapting comprehension notation. Thus,

\begin{mathpar}
  P\{ y / x : x \in S \}
\end{mathpar}

is interpreted to mean the process derived from P by replacing (in a
capture-avoiding manner) each occurrence of $x$ in $S$ by $y$. For example,

\begin{mathpar}
  P\{ \quotep{\procn{x}|\procn{x}} / x : x \in \freenames{P} \}
\end{mathpar}

will replace each (occurrence) of a free name $x$ in $P$ by
$\quotep{\procn{x}|\procn{x}}$.

Also, we will avail ourselves of the notation $x^{L}$ and $x^{R}$ to
denote injections of a name into disjoint copies of the name
space. There are numerous ways to accomplish this. One example can be
found in \cite{MeredithR05}. This notation overloads to vectors of
names: $\vec{x}^{\pi} := (x_{i}^{\pi} \; : \; 0 \leq i < |\vec{x}| )$ where $\pi \in \{L,R\}$.

We also use $P^{\Box} := P|\Box$.

In \cite{MeredithR05} an interpretation of the new operator is
given. It turns out that there are several possible interpretations
all enjoying the requisite algebraic properties of the operator (see
\cite{milner91polyadicpi}). We will therefore make liberal use of
$(\nu\; \vec{x})P$.

% subsection the_syntax_and_semantics_of_the_notation_system (end)   

\input{qm2pi.qmops} 

\input{qm2pi.sterngerlach} 

\input{qm2pi.metric} 

% section concurrent_process_calculi (end)

%\input{qm2pi.proofsketch}

% section proof sketch (end)

%\input{qm2pi.slviaknots} 

% section spatial logic via knots (end)

\input{qm2pi.conclusion}

% section conclusion (end)

%\input{qm2pi.dtcodes} 

% section wiring algorithm (end)

\input{qm2pi.ack} 

% section acknowledgments (end)

\newpage


\bibliographystyle{plain}   
\bibliography{../../biblios/main.bib}

\input{qm2pi.rhodetails}

\end{document}



% section proof sketch (end)

%\section{Unlikely characters: spatial logic for
  knots}\label{sub:characteristic_formulae} % (fold)

Associated to the mobile process calculi are a family of logics known
as the Hennessy-Milner logics. These logics typically enjoy a
semantics interpreting formulae as sets of processes that when
factored through the encoding outlined above allows an identification
of classes of knots with logical formulae. In the context of this
encoding the sub-family known as the spatial logics \cite{CairesC03}
\cite{CairesC04} \cite{Caires04} are of particular interest providing
several important features for expressing and reasoning about
properties (i.e. classes) of knots. We hint here at how this may be done.

%\begin{description}
%\item [structural connectives] 
\subsubsection{Structural connectives} The spatial logics enjoy
structural connectives corresponding, at the logical level, to the
parallel composition ($P | Q$) and new name ($(\nu \; x)P$)
connectives for processes. As illustrated in the examples below, these
connectives are extremely expressive given the shape of our encoding.
%\item [decideable satisfaction]

\subsubsection{Decideable satisfaction}
In \cite{Caires04} the satisfaction relation is shown to be decideable
for a rich class of processes. It further turns out that the image of
the our encoding is a proper subset of that class. This result
provides the basis for an algorithm by which to search for knots
enjoying a given property.
%\item [characteristic formulae]

\subsubsection{Characteristic formulae}
In the same paper \cite{Caires04} , Caires presents a means of calculating
characteristic formulae, selecting equivalence classes of processes
up to a pre--specified depth limit on the support set of names. Composed with our
encoding, this characteristic formula can be used to select
characteristic formulae for knots.
%\end{description}

\subsubsection{Spatial logic formulae}

The grammar below (segmented for comprehension) summarizes the syntax
of spatial logic formulae. We employ illustrative examples in the
sequel to provide an intuitive understanding of their meaning
referring the reader to \cite{Caires04} for a more detailed explication
of the semantics.

\begin{mathpar}
  \inferrule* [lab=boolean] {} {{A,B} \bc T \;|\; \neg A \;|\; A \wedge B \;|\; \eta = \eta'}
  \and
  \inferrule* [lab=spatial] {} {|\; \pzero \;|\; A | B \;|\; x \text{\textregistered} A \;|\; \forall x . A \;|\;  H x . A}
  \and
  \inferrule* [lab=behavioral] {} {|\; \alpha . A}
  \and 
  \inferrule* [lab=recursion] {} {|\; X(\vec{u}) \;|\; \mu X(\vec{u}) . A}
  \and
  \inferrule* [lab=action] {} {\alpha \bc \langle x?(\vec{y}) \rangle \;|\; \langle x!(\vec{y}) \rangle \;|\; \langle \tau \rangle}
  \and 
  \inferrule* [lab=name] {} {\eta \bc x \;|\; \tau}
\end{mathpar} 

% subsection characteristic_formulae (end)   	 

\subsection{Example formulae}\label{sub:example_formulae_} % (fold)

\subsubsection{Crossing as formula.}
% 
% \begin{align*}
%   \frac{d}{dx} \sin x &= \cos x 
%   & \frac{d}{dx} e^x &= e^x \\
%   \frac{d}{dx} \cos x &= - \sin x 
%   & \frac{d}{dx} \log x &= \frac{1}{x} \\
% \end{align*} 

\begin{align*}
 \mu C(x_{0},x_{1},y_{0},y_{1},u).&(\langle x_{0}?(z) \rangle(\langle u! \rangle\langle y_{1}!z \rangle C(x_{0},x_{1},y_{0},y_{1},u)) & \\
  & \wedge \langle y_{1}?(z) \rangle (\langle u! \rangle \langle x_{0}!z \rangle C(x_{0},x_{1},y_{0},y_{1},u)) & \\
  & \wedge \langle x_{1}?(z) \rangle (\langle u? \rangle \langle y_{0}!z \rangle C(x_{0},x_{1},y_{0},y_{1},u)) & \\
  & \wedge \langle y_{0}?(z) \rangle (\langle u? \rangle \langle x_{1}!z \rangle C(x_{0},x_{1},y_{0},y_{1},u))) &
\end{align*}

The lexicographical similarity between the shape of this formulae and
the shape of definition of the process representing a crossing reveals
the intuitive meaning of this formulae. It describes the capabilities
of a process that has the right to represent a crossing. For example
it picks out processes that may perform an input on the port $x_0$ in
its initial menu of capabilities. What differentiates the formula
from the process, however, is that the crossing process is the
smallest candidate to satisfy the formula. Infinitely many other
processes -- with internal behavior hidden behind this interface, so
to speak -- also satisfy this formula. Even this simple formula,
then, can be seen to open a new view onto knots, providing a
computational interpretation of \emph{virtual} knots.

Note that this formula is derived by hand. A similar formula can be
derived by employing Caires' calculation of characteristic formula
\cite{Caires04} to the process representing a crossing. In light of
this discussion, we let
$\meaningof{C}_{\phi}(x0,x1,y0,y1,u)$ denote a formula specifying the
dynamics we wish to capture of a crossing. To guarantee we preserve
the shape of the interface and minimal semantics we demand that
$\meaningof{C}_{\phi}(x0,x1,y0,y1,u) \Rightarrow
\textbf{C}(x0,x1,y0,y1,u)$ where $\textbf{C}(x0,x1,y0,y1,u)$ denotes
the formula above.
                            
\subsubsection{Crossing number constraints.}
The moral content of the context lemma (Lemma \ref{context}) is that the notion of
``locality'' in the Reidemeister moves is effectively captured by the
parallel composition operator of the process calculus. This intuition
extends through the logic. Given a formula,
$\meaningof{C}_{\phi}(x0,x1,y0,y1,u)$, we can use the structural
connectives to specify constraints on crossing numbers, such as at
least $n$ crossings, or exactly $n$ crossings.
\begin{mathpar}
  \inferrule* [lab=at-least-n] {} { K^{\geq n}_{\phi}(\vec{xs},\vec{ys}) := \Pi_{i=0}^{n-1} Hu . \meaningof{C}_{\phi}(xs_i,ys_i,u) | T }
  \and 
  \inferrule* [lab=exactly-n] {} { K^{= n}_{\phi}(\vec{xs},\vec{ys}) := \Pi_{i=0}^{n-1} Hu . \meaningof{C}_{\phi}(xs_i,ys_i,u) | \neg (\forall x_0,y_0,x_1,y_1,u . \meaningof{C}_{\phi}(x_0,y_0,x_1,y_1,u) | T) }
\end{mathpar}

To round out this section, recall that the encoding of an $n$-crossing
knot decomposes into a parallel composition of $n$ \emph{copies} of a
crossing process together with a wiring harness. To specify different
knot classes with the same crossing number amounts to specifying
logical constraints on the wiring harness. In the interest of space,
we defer examples to a forthcoming paper. Suffice it to say that both
the conditions ``alternating knot'' and ``contains the tangle
corresponding to 5/3'' are expressible. For example, it is possible to
calculate the characteristic formula of a process corresponding to the
tangle 5/3 and conjoin it into the classifying formula via the
composition connective of the logic.

Finally, we wish to observe that it is entirely within reason to
contemplate a more domain-specific version of spatial logic tailored
to the shape of processes in the image of the encoding. Such a
domain-specific logic would have a better claim to the title formal
language of knot properties.

% subsection example_formulae_ (end)

% section knots_as_processes (end) 

% section spatial logic via knots (end)

\section{Conclusions and future work}

\paragraph{Testing physical space}
You, gentle reader, may wonder why of all the theorems to be proved
given this set up we pick the one above. In some sense it's hardly
central to quantum mechanics. We see it as central in the sense that
it firmly establishes a notion of physical space arising from a notion
of the equivalence of behavior. Relating bisimulation to a metric is a
big step forward, but one is faced with interpreting the relationship
of that metric space to something more physical. Quantum mechanical
notions of ``physical'' space are still far from intuitive, but by
relating this idea of distance as testing to calculations that predict
physical circumstances we are making a not insignificant step forward
toward an understanding of the physical space we inhabit as
essentially dynamic.

\paragraph{Effectivity and simulation}
One of the observations we have yet to make is that the entire program
spelled out here is effective. We have built various interpreters for
the reflective calculus at work in this interpretation. In principle,
then, we can simulate quantum mechanics on a computer. The place where
the simulation may lose fidelity is the infinitely branching summation
for the annihilator.

In this connection i also want to point out that the evaluation style
calculation of the inner product puts the non-determinism of the
summation right at the heart of measurement. This suggests that
Milner's original reduction-based formulation of the dynamics of his
calculi in terms of sums was not just notationally suggestive of a
notion of measure-and-continue but captured some significant part of
the physics.

\paragraph{Quantum continuations}
In light of this last observation i want to point out that the
predominant account of quantum mechanics is missing a key aspect of a
truly compositional story of the physical situation. In a real lab,
when a measurement is made the observation can be made to feed into
another device that then makes another measurement conditioned on the
results of the first. This means that after the superposition was
collapsed the entire experimental set up remained in
superposition. While QM offers a means of writing this down it doesn't
quite line up well with the well-trodden formulation of computation
and continuation that we see so succinctly expressed in Milner's
calculi. This suggests that there might be advantages to this account
of dynamics waiting to be explored.

\paragraph{Quantum logic}
In this connection, we also note that by virtue of having the
Hennessy-Milner construction, we can pull the construction through the
interpretation of QM. This gives us a natural candidate for a quantum
logic that enjoys an extremely tight connection with it's domain of
interpretation, making the construction much less ad hoc (rather it is
the image of functor!).

\paragraph{Quantum probabiity}
i have questions about the basis of the interpretation of inner
product as probability amplitude. In particular, using which
axiomatization of probability theory does the notion of probability
amplitude earn the right to be so dubbed? In other words, where is the
proof that the operation for calculating a probability amplitude (and
then squaring) satisfies the axioms of what it means to calculate a
probability? Even if such a proof exists (i have yet to find it in the
literature), i wonder if it might not be possible to turn things on
their heads. Can we view the calculation of the probability amplitude
as an axiomatization of probability? If so, then the definition we
give for calculating probability amplitude may provide the basis for
an \emph{effective} theory of probability.

\paragraph{Quantum vs ``biological'' information}
Finally, i want to conclude with a more philosophical observation. At
a recent workshop in which QM was a predominant topic i noticed
something about quantum information. The speaker was giving a riveting
discussion of axiomatic QM and showing how properties of ``no
cloning'' and ``no deleting'' emerged as consequences of the
axiomatization. Theorems of this form are necessary to give us a sense
of confidence that our axioms characterize the physical theory. What
struck me, though, was that if quantum information is neither erasable
nor replicable it is markedly different from \emph{life}. Two of the
things we know about life is that

\begin{itemize}
  \item it ends;
  \item to gain some measure of persistence, to transcend it's
    finitude it is imminently copyable.
\end{itemize}

Both of these qualities are summarized succinctly in the aphorism: all
flesh is grass. For me these two kinds of ``information'' -- call them
quantum and biological -- are end points on a spectrum of strategies
for persistence. At one end, we have those curious entities that enjoy
uniqueness and permanence; at the other, we have those who in the face
of a certain end and an uncertain present make a go of passing
something on. To me one of the more remarkable aspects of the latter
strategy is that in the presence of noise (and certain features of
copying) we get a kind of dynamism, a chance for improvement against a
given persistent condition.

% subsection other_calculi_other_bisimulations_and_geometry_as_behavior (end)




% section conclusion (end)

%\documentclass[12pt]{llncs}
%\documentclass{jktr}

\usepackage[pdftex]{hyperref}                   
\usepackage {listings}
\usepackage {mathpartir}
\usepackage{bcprules}
%\usepackage{listings}
                       
\usepackage{graphicx} 
%\usepackage[margins=2.5cm,nohead,nofoot]{geometry}
%\usepackage{geometry}
\usepackage{amsfonts}
\usepackage{amstext}
\usepackage{latexsym}
\usepackage{amssymb}
\usepackage{color}


%\include{myPreamble}
\include{qm2pi.local} 

%\ifpdf
%\usepackage[pdftex]{graphicx}
%\else
%\usepackage{graphicx}
%\fi

 % \ifpdf
%  \usepackage{pdfsync}
%  \if


%\title{Brief Article}
%\author{David F. Snyder}
%\author{L.G. Meredith}

%\address{Dept. of Math., Texas State University--San Marcos, San Marcos, TX 78666}
       
\pagestyle{empty}


\begin{document}

\lstset{language=[Objective]Caml,frame=shadowbox}

\input{qm2pi.front}

% section front matter (end)

\input{qm2pi.intro} 
 
% section introduction (end)

% \input{qm2pi.knotations} 

% section notation (end)

\input{qm2pi.process.calculi} 

% section concurrent_process_calculi_and_spatial_logics_ (end)
    
%\input{qm2pi.knots2pi} 

%\input{qm2pi.trefoil} 

%\input{qm2pi.mainthm} 

% subsection basic_interpretation (end)

%\input{qm2pi.rho.presentation} 
\subsection{The syntax and semantics of the notation system}\label{sub:the_syntax_and_semantics_of_the_notation_system} % (fold)

We now summarize a technical presentation of the calculus that
embodies our theory of dynamics. The typical presentation of such a
calculus follows the style of giving generators and relations on
them. The grammar, below, describing term constructors, freely
generates the set of processes, $\Proc$. This set is then quotiented
by a relation known as structural congruence and it is over this set
that the notion of dynamics is expressed. This presentation is
essentially that of \cite{MeredithR05} with the addition of
polyadicity and summation. For readability we have relegated some of
the technical subtleties to an appendix.

\subsubsection{Process grammar}\label{subsub:process_grammar}

\begin{mathpar}
  \inferrule* [lab=synchronization] {} {{M} \bc \pzero \;|\; x?F \;|\; x!C }
  \and
  \inferrule* [lab=abstraction] {} {{F} \bc (x)P}
  \and
  \inferrule* [lab=concretion] {} {{C} \bc \langle Q \rangle}
  \and
  \inferrule* [lab=process] {} {{P,Q} \bc M \;| \;P|Q \;|\; @{x}}
  \and
  \inferrule* [lab=name] {} {{x} \bc \quotep{P}}
\end{mathpar} 

Note that $\vec{x}$ (resp. $\vec{P}$) denotes a vector of names
(resp. processes) of length $|\vec{x}|$ (resp. $|\vec{P}|$). We adopt
the following useful abbreviations.

\begin{mathpar}
   x?(\vec{y}).P := x.(\vec{y})P \and  x\clift{\vec{P}} := x.\clift{\vec{P}}
   \and x!(y) := \lift{x}{\dropn{y}}
   \and \Pi_{i=0}^{n-1}P_i := P_0 | \ldots | P_{n-1}
\end{mathpar}

\subsubsection{Structural congruence}

\paragraph{Free and bound names and alpha-equivalence.} At the
core of structural equivalence is alpha-equivalence which identifies
process that are the same up to a change of variable. Formally, we
recognize the distinction between free and bound names. The free names
of a process, $\freenames{P}$, may be calculated recursively as
follows:

\begin{mathpar}
\freenames{\pzero} := \emptyset
  \and \\
  \freenames{x?(y).P} := \{ x \} \cup (\freenames{P} \setminus \{ y \})
  \and 
  \freenames{x!\langle P \rangle} := \{ x \} \cup \{ P \} 
  \and \\
  \freenames{P|Q} := \freenames{P} \cup \freenames{Q}
  \and \\
  \freenames{@{x}} := \{ x \}
\end{mathpar}

$\pi$
$\quotep{\pi}$

$\freenames{-} : \pi \to \mathcal{P}(\quotep{\pi})$

\begin{eqnarray*}
  \freenames{\pzero} & := & \emptyset \\
  \freenames{x?(y).P} & := & \{ x \} \cup (\freenames{P} \setminus \{ y \}) \\
  \freenames{x!\langle P \rangle} & := & \{ x \} \cup \{ P \} \\
  \freenames{P|Q} & := & \freenames{P} \cup \freenames{Q} \\
  \freenames{\dropn{x}} & := & \{ x \}
\end{eqnarray*}

The bound names of a process, $\boundnames{P}$, are those names occurring in $P$
that are not free. For example, in $x?(y).0$, the name $x$ is free, while $y$ is bound.

\begin{mathpar}
  \inferrule* [lab=monoidal-laws] {} { P|Q \equiv Q|P \and P|0 \equiv P \and P|(Q|R) \equiv (P|Q)|R }
\end{mathpar}

\begin{mathpar}
  \inferrule* [lab=alpha-equivalence] {} { (x)P \equiv (y)P\{y/x\} \and y \not\in \freenames{P} }
\end{mathpar}

\begin{definition}
Then two processes, $P,Q$, are alpha-equivalent if $P = Q\{\vec{y}/\vec{x}\}$ for
some $\vec{x} \in \boundnames{Q},\vec{y} \in \boundnames{P}$, where $Q\{\vec{y}/\vec{x}\}$
denotes the capture-avoiding substitution of $\vec{y}$ for $\vec{x}$ in $Q$.
\end{definition}

\begin{definition}
  The {\em structural congruence} \cite{SangiorgiWalker} , $\equiv$,
  between processes is the least congruence containing
  alpha-equivalence, satisfying the abelian monoid laws
  (associativity, commutativity and $\pzero$ as identity) for parallel
  composition $|$ and for summation $+$.
\end{definition}

\subsection{Name equivalence}

We take name equivalence, written $\nameeq$, to be the smallest
equivalence relation generated by the following rules.

\begin{mathpar}
\inferrule*[lab=Quote-drop]
{ }
{ \quotep{@{x}} \nameeq x }

\inferrule*[lab=Struct-equiv]
{ P \scong Q }
{ \quotep{P} \nameeq \quotep{Q} }
\end{mathpar}

The astute reader will have noticed that the mutual recursion of names
and processes imposes a mutual recursion on alpha-equivalence and
structural equivalence via name-equivalence. Fortunately, all of this
works out pleasantly and we may calculate in the natural way, free of
concern. The reader interested in the details is referred to the
appendix \ref{appendix:rho_details}.

\subsection{Substitution}

We use $\Proc$ for the set of processes, $\QProc$ for the set of
names, and $\id{\{}\vec{y} / \vec{x} \id{\}}$ to denote partial maps,
$s : \QProc \rightarrow \QProc$. A map, $s$ lifts, uniquely, to a map
on process terms, $\widehat{s} : \Proc \rightarrow \Proc$ by the
following equations.

\begin{mathpar}
  (0) \psubstp{Q}{P} := 0 \\
  (R \juxtap S) \psubstp{Q}{P}
  :=    
  (R)\psubstp{Q}{P} \juxtap (S) \psubstp{Q}{P} \\
  (x?(y).R) \psubstp{Q}{P}    
  :=    
  (x)\substp{Q}{P} (z)\concat( (R \psubstn{z}{y}) \psubstp{Q}{P} ) \\
  (\lift{x}{R}) \psubstp{Q}{P}  
  :=
  \lift{(x)\substp{Q}{P}}{ R \psubstp{Q}{P} } \\
%   (\dropn{x})  \psubstp{Q}{P}       
%   := 
%   \left\{ 
%     \begin{array}{ccc} 
%       \dropn{\quotep{Q}} & & x \nameeq \quotep{P} \\
%       \dropn{x} & & otherwise \\
%     \end{array}
%   \right. 
  (\dropn{x})  \psubstp{Q}{P}       
  := 
  \left\{ 
    \begin{array}{ccc} 
      Q & & x \nameeq \quotep{P} \\
      \dropn{x} & & otherwise \\
    \end{array}
  \right.
\end{mathpar}
 

where

\begin{eqnarray}
  (x)\id{\{} \lpquote Q \rpquote / \lpquote P \rpquote \id{\}}            = 
  \left\{ 
    \begin{array}{ccc}
      \lpquote Q \rpquote & & x \nameeq \lpquote P \rpquote \\
      x & & otherwise \\
    \end{array}
  \right. \nonumber
\end{eqnarray}

and $z$ is chosen distinct from $\quotep{P}$, $\quotep{Q}$, the free
names in $Q$, and all the names in $R$. Our $\alpha$-equivalence will
be built in the standard way from this substitution.

\begin{remark}\label{rem:no_self_referential_names}
  One consequence of these definitions is that $\forall P. \quotep{P}
  \not\in \freenames{P}$.
\end{remark}

\subsection{ Dynamic quote: an example }

Anticipating something of what's to come, consider applying the
substitution, $\widehat{\id{\{}u / z \id{\}}}$, to the following pair
of processes, $\lift{w}{y!(z)}$ and $w[ \lpquote y!(z) \rpquote ]$.

\begin{eqnarray}
	\lift{w}{y!(z)}\widehat{\id{\{}u / z \id{\}}}
		& = &
		\lift{w}{y!(u)} \nonumber\\
	w[ \lpquote y!(z) \rpquote ] \widehat{ \id{\{}u / z \id{\}} }
		& = &
		w[ \lpquote y!(z) \rpquote ] \nonumber
\end{eqnarray}

Because the body of the process between quotes is impervious to
substitution, we get radically different answers. In fact, by
examining the first process in an input context,
e.g. $x?(z).\lift{w}{y!(z)}$, we see that the process under the lift
operator may be shaped by prefixed inputs binding a name inside it. In
this sense, the lift operator will be seen as a way to dynamically
construct processes before reifying them as names.

Finally equipped with these standard features we can present the
dynamics of the calculus.

\subsubsection{Operational semantics} 

Finally, we introduce the computational dynamics. What marks these
algebras as distinct from other more traditionally studied algebraic
structures, e.g. vector spaces or polynomial rings, is the manner in
which dynamics is captured. In traditional structures, dynamics is typically
expressed through morphisms between such structures, as in linear maps
between vector spaces or morphisms between rings. In algebras
associated with the semantics of computation, the dynamics is
expressed as part of the algebraic structure itself, through a
reduction reduction relation typically denoted by $\red$. Below, we
give a recursive presentation of this relation for the calculus used
in the encoding.

$\red \subseteq \pi \times \pi$
$\red : \pi \to \mathcal{P}(\pi)$

\begin{mathpar}
  \inferrule* [lab=Comm] { \textsf{match}( x_{src}, x_{trgt} ) } { x_{trgt}?(y)P \; | \; x_{src}!\langle {Q} \rangle \red P\{\quotep{Q}/y}\} }
  \and \\
  \inferrule* [lab=Par] {{P} \red {P}'} {{{P} | {Q}} \red {{P}' | {Q}}}
  \and
  \inferrule* [lab=Equiv]{{{P} \scong {P}'} \andalso {{P}' \red {Q}'} \andalso {{Q}' \scong {Q}}}{{P} \red {Q}}
\end{mathpar}

\begin{eqnarray*}
  match_{\equiv} (\quotep{P},\quotep{Q}) & := & P \equiv Q \\
  match_{\dagger}(\quotep{P},\quotep{Q}) & := & \forall R. P|Q \red^{*} R => R \red^{*} 0 \\
  match_{K}(\quotep{P},\quotep{Q}) & := & K \mbox{ for some context } K
\end{eqnarray*}

$u?(x)P | u!\langle Q \rangle \red P\{\quotep{Q}/x\}$

%We write $\wred$ for $\red^*$, and $P\red$ if $\exists Q $ such that $ P \red Q$.
We write $P\red$ if $\exists Q $ such that $ P \red Q$ and $P\not\red$, otherwise.

\section{Replication}

As mentioned before, it is known that replication (and hence
recursion) can be implemented in a higher-order process algebra
\cite{SangiorgiWalker}. As our first example of calculation with the
machinery thus far presented we give the construction explicitly in
the {\rhoc}.

\begin{eqnarray}
	D_{x} & := & \prefix{x}{y}{(\binpar{\outputp{x}{y}}{@{y}})} \nonumber\\
	\bangp_{x}{P} & := & \binpar{{x}!\langle{\binpar{D_{x}}{P}}\rangle}{D_{x}} \nonumber
\end{eqnarray}

\begin{eqnarray}
	\bangp_{x}{P} & & \nonumber\\
	=
	& {x}!\langle{(\prefix{x}{y}{(\outputp{x}{y} | @{y})) | P}}\rangle 
	      | \prefix{x}{y}{(\outputp{x}{y} | @{y})} & \nonumber\\
	\red
	& (\outputp{x}{y} | @{y})\substn{\quotep{(\prefix{x}{y}{(@{y} | \outputp{x}{y})) | P}}}{y} & \nonumber\\
	=
	& \outputp{x}{\quotep{(\prefix{x}{y}{(\outputp{x}{y} | @{y})) | P}}}
	  | {(\prefix{x}{y}{(\outputp{x}{y} | @{y})) | P}} & \nonumber\\
	\red
	& \ldots & \nonumber\\
	\red^*
	& P | P | \ldots & \nonumber
\end{eqnarray}

Of course, this encoding, as an implementation, runs away, unfolding
$\bangp{P}$ eagerly. A lazier and more implementable replication
operator, restricted to input-guarded processes, may be obtained as follows.

\begin{eqnarray}
\bangp{\prefix{u}{v}{P}} 
	:= 
	\binpar{\lift{x}{\prefix{u}{v}{(\binpar{D(x)}{P})}}}{D(x)} \nonumber
\end{eqnarray}

\begin{remark}
  Note that the lazier definition still does not deal with summation
  or mixed summation (i.e. sums over input and output). The reader is
  invited to construct definitions of replication that deal with these
  features. 

  Further, the definitions are parameterized in a name, $x$. Can you,
  gentle reader, make a definition that eliminates this parameter and
  guarantees no accidental interaction between the replication
  machinery and the process being replicated -- i.e. no accidental
  sharing of names used by the process to get its work done and the
  name(s) used by the replication to effect copying. This latter
  revision of the definition of replication is crucial to obtaining
  the expected identity $!!P \sim !P$.
\end{remark}

\begin{remark}\label{rem:paradoxical_combinator}
  The reader familiar with the lambda calculus will have noticed the
  similarity between $D$ and the paradoxical combinator.

  [Ed. note: the existence of this seems to suggest we have to be more
  restrictive on the set of processes and names we admit if we are to
  support no-cloning.]
\end{remark}

\subsubsection{Bisimulation}

The computational dynamics gives rise to another kind of equivalence,
the equivalence of computational behavior. As previously mentioned
this is typically captured \emph{via} some form of bisimulation.

% The notion we use in this paper is weak barbed bisimulation
% \cite{milner91polyadicpi}.

The notion we use in this paper is derived from weak barbed
bisimulation \cite{milner91polyadicpi}. 

\begin{definition}
An \emph{observation relation}, $\downarrow_{\mathcal N}$, over a set
of names, $\mathcal N$, is the smallest relation satisfying the rules
below.

\infrule[Out-barb]{y \in {\mathcal N}, \; x \nameeq y}
		  {\outputp{x}{v} \downarrow_{\mathcal N} x}
\infrule[Par-barb]{\mbox{$P\downarrow_{\mathcal N} x$ or $Q\downarrow_{\mathcal N} x$}}
		  {\binpar{P}{Q} \downarrow_{\mathcal N} x}

We write $P \Downarrow_{\mathcal N} x$ if there is $Q$ such that 
$P \wred Q$ and $Q \downarrow_{\mathcal N} x$.
\end{definition}

\begin{definition}
%\label{def.bbisim}
An  ${\mathcal N}$-\emph{barbed bisimulation} over a set of names, ${\mathcal N}$, is a symmetric binary relation 
${\mathcal S}_{\mathcal N}$ between agents such that $P\rel{S}_{\mathcal N}Q$ implies:
\begin{enumerate}
\item If $P \red P'$ then $Q \wred Q'$ and $P'\rel{S}_{\mathcal N} Q'$.
\item If $P\downarrow_{\mathcal N} x$, then $Q\Downarrow_{\mathcal N} x$.
\end{enumerate}
$P$ is ${\mathcal N}$-barbed bisimilar to $Q$, written
$P \wbbisim_{\mathcal N} Q$, if $P \rel{S}_{\mathcal N} Q$ for some ${\mathcal N}$-barbed bisimulation ${\mathcal S}_{\mathcal N}$.
\end{definition}

$\mathcal{R} \subseteq \pi \times \pi$

$P \mathcal{R} Q => \forall P'. P \red P' \Rightarrow \exists Q'. Q \red Q', P' \mathcal{R} Q'$

$P \vdash x \Rightarrow Q \vdash x$

\begin{mathpar}
  \inferrule*[lab=Out-barb]{x \nameeq y}{{y}!\langle{Q}\rangle \vdash x}
  \and
  \inferrule*[lab=Par-barb]{\mbox{$P\vdash x$ or $Q\vdash x$}}{\binpar{P}{Q} \vdash x}
\end{mathpar}

\subsubsection{Contexts}

One of the principle advantages of computational calculi like the
$\pi$-calculus is a well-defined notion of context,
contextual-equivalence and a correlation between
contextual-equivalence and notions of bisimulation. The notion of
context allows the decomposition of a process into (sub-)process and
its syntactic environment, its context. Thus, a context may be
thought of as a process with a ``hole'' (written $\Box$) in it. The
application of a context $M$ to a process $P$, written $M[P]$, is
tantamount to filling the hole in $M$ with $P$. In this paper we do
not need the full weight of this theory, but do make use of the notion
of context in the proof the main theorem. 

\begin{mathpar}
  \inferrule* [lab=summation] {} {{M_{M},M_{N}} \bc \Box \;|\; x.M_{A} \;|\; M_{M}+M_{N}}
  \and
  \inferrule* [lab=agent] {} {{M_{A}} \bc (\vec{x})M_{P} \;| \; \clift{P_0,\ldots,M_{P},\ldots,P_N}}
  \and \\
  \inferrule* [lab=process] {} {{M_{P}} \bc M_{N} \;| \;P|M_{P} }
\end{mathpar} 

\begin{mathpar}
  \inferrule* [lab=sychronization] {} {M_{N} \bc \Box \;|\; x?M_{F} \;|\; x!M_{C}}
  \and
  \inferrule* [lab=abstraction] {} {{M_{F}} \bc (x)M_{P} }
  \and
  \inferrule* [lab=concretion] {} {{M_{C}} \bc \langle M_{P} \rangle }
  \and \\
  \inferrule* [lab=process] {} {{M_{P}} \bc M_{N} \;| \;P|M_{P} }
\end{mathpar}

\begin{definition}[contextual application] Given a context $M$, and
  process $P$, we define the \emph{contextual application}, $M[P] :=
  M\{P/\Box\}$. That is, the contextual application of M to P is the
  substitution of $P$ for $\Box$ in $M$.
\end{definition}

$\meaningof{-} : L \to \mathcal{P}(\pi)$

\begin{mathpar}
  \inferrule* [lab=collection] {} {\meaningof{true} = \pi, \and \meaningof{~E} = \pi \setminus \meaningof{E}, \and \meaningof{E_{1} \& E_{2}} = \meaningof{E_{1}} \cap \meaningof{E_{2}}}
\end{mathpar}

\begin{mathpar}
  \inferrule* [lab=structure] {} {\meaningof{0} = \{ P \in \pi | P \equiv 0 \}, \and \\ \meaningof{E_1 | E_2} = \{ P \in \pi | P \equiv P_{1} | P_{2}, P_{1} \in \meaningof{E_{1}}, P_{2} \in \meaningof{E_2}\} }
\end{mathpar}

\begin{mathpar}
 \inferrule* [lab=behavior] {} {\meaningof{\langle a?b \rangle E} = \{ P \in \pi | P \equiv Q | u?(y)P', \\ \and \\\\ \and \\ \;\;\; u \in \meaningof{a}, \forall z.P'\{z/y\} \in \meaningof{E\{z/b\}}\}, \and \\ \meaningof{a!E} = \{ P \in \pi | P \equiv Q | x!\langle P' \rangle, x \in \meaningof{a} P' \in \meaningof{E}\} }
\end{mathpar}

\begin{mathpar}
 \inferrule* [lab=nominal] {} {\meaningof{\quotep{E}} = \{ \quotep{P} \in \quotep{\pi} | P \in \meaningof{E} \}, \and \meaningof{\quotep{P}} = \{ \quotep{Q} \in \quotep{\pi} | P \equiv Q \} \and \\ \meaningof{@\quotep{E}} = \{ P \in \pi | P \equiv @x, x \in \meaningof{E} \}}
\end{mathpar}

\begin{eqnarray*}
  \\
  \meaningof{-} : TS \to ST
\end{eqnarray*}

\begin{eqnarray*}
  \\
  L : TS \to ST
\end{eqnarray*}

\begin{eqnarray*}
  \\
  P \models E \iff P \in \meaningof{E}
\end{eqnarray*}

\begin{eqnarray*}
  P \approx_{L} Q \iff \forall E \in L. P \models E \iff Q \models E
\end{eqnarray*}

\begin{eqnarray*}
  P \approx_{K} Q
\end{eqnarray*}

\begin{eqnarray*}
  P \approx Q
\end{eqnarray*}

$\approx_{K} = \approx = \approx_{L}$

\subsubsection{Contextual duality}

Note that contexts extend the quotation operation to a family of
operations from processes to names. Given a context, $M$, we can
define a \emph{nominal context}, $\quotep{M}$ by $\quotep{M}[P] :=
\quotep{M[P]}$. To foreshadow what is to come we observe that these
operations enjoy a duality with processes very much like the duality
between vectors and maps from vectors to scalars.

Further, because the calculus is essentially higher-order, we have a
correspondence between contexts and processes. More specifically,
given a name $x$ and a context $M$ we can construct $M^{*}_{x}$ such
that 

\begin{mathpar}
  M^{*}_{x} | \lift{x}{P} \red M[P]
\end{mathpar}

namely,

\begin{mathpar}
  M^{*}_{x} := x?(u).M[\dropn{u}]
\end{mathpar}

The dependence of $M^{*}_{x}$ on a name makes it an abstraction, 

\begin{mathpar}
  M^{*} := (x)x?(u).M[\dropn{u}]
\end{mathpar}

\subsection{Additional notation}

It will sometimes be convenient to denote the process a name
quotes. We already have the notation $x = \quotep{P}$, but it will be
convenient to introduce an alternate notation, $\procn{x}$, when we
want to emphasize the connection to the use of the name. Note that, by
virtue of name equivalence, $\quotep{\procn{x}} \nameeq x$; so, the
notation is consistent with previous definitions.

Further, because names have structure it is possible to effect
substitutions on the basis of that structure. This means we need to
upgrade our notation for substitutions, which we accomplish by
adapting comprehension notation. Thus,

\begin{mathpar}
  P\{ y / x : x \in S \}
\end{mathpar}

is interpreted to mean the process derived from P by replacing (in a
capture-avoiding manner) each occurrence of $x$ in $S$ by $y$. For example,

\begin{mathpar}
  P\{ \quotep{\procn{x}|\procn{x}} / x : x \in \freenames{P} \}
\end{mathpar}

will replace each (occurrence) of a free name $x$ in $P$ by
$\quotep{\procn{x}|\procn{x}}$.

Also, we will avail ourselves of the notation $x^{L}$ and $x^{R}$ to
denote injections of a name into disjoint copies of the name
space. There are numerous ways to accomplish this. One example can be
found in \cite{MeredithR05}. This notation overloads to vectors of
names: $\vec{x}^{\pi} := (x_{i}^{\pi} \; : \; 0 \leq i < |\vec{x}| )$ where $\pi \in \{L,R\}$.

We also use $P^{\Box} := P|\Box$.

In \cite{MeredithR05} an interpretation of the new operator is
given. It turns out that there are several possible interpretations
all enjoying the requisite algebraic properties of the operator (see
\cite{milner91polyadicpi}). We will therefore make liberal use of
$(\nu\; \vec{x})P$.

% subsection the_syntax_and_semantics_of_the_notation_system (end)   

\input{qm2pi.qmops} 

\input{qm2pi.sterngerlach} 

\input{qm2pi.metric} 

% section concurrent_process_calculi (end)

%\input{qm2pi.proofsketch}

% section proof sketch (end)

%\input{qm2pi.slviaknots} 

% section spatial logic via knots (end)

\input{qm2pi.conclusion}

% section conclusion (end)

%\input{qm2pi.dtcodes} 

% section wiring algorithm (end)

\input{qm2pi.ack} 

% section acknowledgments (end)

\newpage


\bibliographystyle{plain}   
\bibliography{../../biblios/main.bib}

\input{qm2pi.rhodetails}

\end{document}

 

% section wiring algorithm (end)

\documentclass[12pt]{llncs}
%\documentclass{jktr}

\usepackage[pdftex]{hyperref}                   
\usepackage {listings}
\usepackage {mathpartir}
\usepackage{bcprules}
%\usepackage{listings}
                       
\usepackage{graphicx} 
%\usepackage[margins=2.5cm,nohead,nofoot]{geometry}
%\usepackage{geometry}
\usepackage{amsfonts}
\usepackage{amstext}
\usepackage{latexsym}
\usepackage{amssymb}
\usepackage{color}


%\include{myPreamble}
\include{qm2pi.local} 

%\ifpdf
%\usepackage[pdftex]{graphicx}
%\else
%\usepackage{graphicx}
%\fi

 % \ifpdf
%  \usepackage{pdfsync}
%  \if


%\title{Brief Article}
%\author{David F. Snyder}
%\author{L.G. Meredith}

%\address{Dept. of Math., Texas State University--San Marcos, San Marcos, TX 78666}
       
\pagestyle{empty}


\begin{document}

\lstset{language=[Objective]Caml,frame=shadowbox}

\input{qm2pi.front}

% section front matter (end)

\input{qm2pi.intro} 
 
% section introduction (end)

% \input{qm2pi.knotations} 

% section notation (end)

\input{qm2pi.process.calculi} 

% section concurrent_process_calculi_and_spatial_logics_ (end)
    
%\input{qm2pi.knots2pi} 

%\input{qm2pi.trefoil} 

%\input{qm2pi.mainthm} 

% subsection basic_interpretation (end)

%\input{qm2pi.rho.presentation} 
\subsection{The syntax and semantics of the notation system}\label{sub:the_syntax_and_semantics_of_the_notation_system} % (fold)

We now summarize a technical presentation of the calculus that
embodies our theory of dynamics. The typical presentation of such a
calculus follows the style of giving generators and relations on
them. The grammar, below, describing term constructors, freely
generates the set of processes, $\Proc$. This set is then quotiented
by a relation known as structural congruence and it is over this set
that the notion of dynamics is expressed. This presentation is
essentially that of \cite{MeredithR05} with the addition of
polyadicity and summation. For readability we have relegated some of
the technical subtleties to an appendix.

\subsubsection{Process grammar}\label{subsub:process_grammar}

\begin{mathpar}
  \inferrule* [lab=synchronization] {} {{M} \bc \pzero \;|\; x?F \;|\; x!C }
  \and
  \inferrule* [lab=abstraction] {} {{F} \bc (x)P}
  \and
  \inferrule* [lab=concretion] {} {{C} \bc \langle Q \rangle}
  \and
  \inferrule* [lab=process] {} {{P,Q} \bc M \;| \;P|Q \;|\; @{x}}
  \and
  \inferrule* [lab=name] {} {{x} \bc \quotep{P}}
\end{mathpar} 

Note that $\vec{x}$ (resp. $\vec{P}$) denotes a vector of names
(resp. processes) of length $|\vec{x}|$ (resp. $|\vec{P}|$). We adopt
the following useful abbreviations.

\begin{mathpar}
   x?(\vec{y}).P := x.(\vec{y})P \and  x\clift{\vec{P}} := x.\clift{\vec{P}}
   \and x!(y) := \lift{x}{\dropn{y}}
   \and \Pi_{i=0}^{n-1}P_i := P_0 | \ldots | P_{n-1}
\end{mathpar}

\subsubsection{Structural congruence}

\paragraph{Free and bound names and alpha-equivalence.} At the
core of structural equivalence is alpha-equivalence which identifies
process that are the same up to a change of variable. Formally, we
recognize the distinction between free and bound names. The free names
of a process, $\freenames{P}$, may be calculated recursively as
follows:

\begin{mathpar}
\freenames{\pzero} := \emptyset
  \and \\
  \freenames{x?(y).P} := \{ x \} \cup (\freenames{P} \setminus \{ y \})
  \and 
  \freenames{x!\langle P \rangle} := \{ x \} \cup \{ P \} 
  \and \\
  \freenames{P|Q} := \freenames{P} \cup \freenames{Q}
  \and \\
  \freenames{@{x}} := \{ x \}
\end{mathpar}

$\pi$
$\quotep{\pi}$

$\freenames{-} : \pi \to \mathcal{P}(\quotep{\pi})$

\begin{eqnarray*}
  \freenames{\pzero} & := & \emptyset \\
  \freenames{x?(y).P} & := & \{ x \} \cup (\freenames{P} \setminus \{ y \}) \\
  \freenames{x!\langle P \rangle} & := & \{ x \} \cup \{ P \} \\
  \freenames{P|Q} & := & \freenames{P} \cup \freenames{Q} \\
  \freenames{\dropn{x}} & := & \{ x \}
\end{eqnarray*}

The bound names of a process, $\boundnames{P}$, are those names occurring in $P$
that are not free. For example, in $x?(y).0$, the name $x$ is free, while $y$ is bound.

\begin{mathpar}
  \inferrule* [lab=monoidal-laws] {} { P|Q \equiv Q|P \and P|0 \equiv P \and P|(Q|R) \equiv (P|Q)|R }
\end{mathpar}

\begin{mathpar}
  \inferrule* [lab=alpha-equivalence] {} { (x)P \equiv (y)P\{y/x\} \and y \not\in \freenames{P} }
\end{mathpar}

\begin{definition}
Then two processes, $P,Q$, are alpha-equivalent if $P = Q\{\vec{y}/\vec{x}\}$ for
some $\vec{x} \in \boundnames{Q},\vec{y} \in \boundnames{P}$, where $Q\{\vec{y}/\vec{x}\}$
denotes the capture-avoiding substitution of $\vec{y}$ for $\vec{x}$ in $Q$.
\end{definition}

\begin{definition}
  The {\em structural congruence} \cite{SangiorgiWalker} , $\equiv$,
  between processes is the least congruence containing
  alpha-equivalence, satisfying the abelian monoid laws
  (associativity, commutativity and $\pzero$ as identity) for parallel
  composition $|$ and for summation $+$.
\end{definition}

\subsection{Name equivalence}

We take name equivalence, written $\nameeq$, to be the smallest
equivalence relation generated by the following rules.

\begin{mathpar}
\inferrule*[lab=Quote-drop]
{ }
{ \quotep{@{x}} \nameeq x }

\inferrule*[lab=Struct-equiv]
{ P \scong Q }
{ \quotep{P} \nameeq \quotep{Q} }
\end{mathpar}

The astute reader will have noticed that the mutual recursion of names
and processes imposes a mutual recursion on alpha-equivalence and
structural equivalence via name-equivalence. Fortunately, all of this
works out pleasantly and we may calculate in the natural way, free of
concern. The reader interested in the details is referred to the
appendix \ref{appendix:rho_details}.

\subsection{Substitution}

We use $\Proc$ for the set of processes, $\QProc$ for the set of
names, and $\id{\{}\vec{y} / \vec{x} \id{\}}$ to denote partial maps,
$s : \QProc \rightarrow \QProc$. A map, $s$ lifts, uniquely, to a map
on process terms, $\widehat{s} : \Proc \rightarrow \Proc$ by the
following equations.

\begin{mathpar}
  (0) \psubstp{Q}{P} := 0 \\
  (R \juxtap S) \psubstp{Q}{P}
  :=    
  (R)\psubstp{Q}{P} \juxtap (S) \psubstp{Q}{P} \\
  (x?(y).R) \psubstp{Q}{P}    
  :=    
  (x)\substp{Q}{P} (z)\concat( (R \psubstn{z}{y}) \psubstp{Q}{P} ) \\
  (\lift{x}{R}) \psubstp{Q}{P}  
  :=
  \lift{(x)\substp{Q}{P}}{ R \psubstp{Q}{P} } \\
%   (\dropn{x})  \psubstp{Q}{P}       
%   := 
%   \left\{ 
%     \begin{array}{ccc} 
%       \dropn{\quotep{Q}} & & x \nameeq \quotep{P} \\
%       \dropn{x} & & otherwise \\
%     \end{array}
%   \right. 
  (\dropn{x})  \psubstp{Q}{P}       
  := 
  \left\{ 
    \begin{array}{ccc} 
      Q & & x \nameeq \quotep{P} \\
      \dropn{x} & & otherwise \\
    \end{array}
  \right.
\end{mathpar}
 

where

\begin{eqnarray}
  (x)\id{\{} \lpquote Q \rpquote / \lpquote P \rpquote \id{\}}            = 
  \left\{ 
    \begin{array}{ccc}
      \lpquote Q \rpquote & & x \nameeq \lpquote P \rpquote \\
      x & & otherwise \\
    \end{array}
  \right. \nonumber
\end{eqnarray}

and $z$ is chosen distinct from $\quotep{P}$, $\quotep{Q}$, the free
names in $Q$, and all the names in $R$. Our $\alpha$-equivalence will
be built in the standard way from this substitution.

\begin{remark}\label{rem:no_self_referential_names}
  One consequence of these definitions is that $\forall P. \quotep{P}
  \not\in \freenames{P}$.
\end{remark}

\subsection{ Dynamic quote: an example }

Anticipating something of what's to come, consider applying the
substitution, $\widehat{\id{\{}u / z \id{\}}}$, to the following pair
of processes, $\lift{w}{y!(z)}$ and $w[ \lpquote y!(z) \rpquote ]$.

\begin{eqnarray}
	\lift{w}{y!(z)}\widehat{\id{\{}u / z \id{\}}}
		& = &
		\lift{w}{y!(u)} \nonumber\\
	w[ \lpquote y!(z) \rpquote ] \widehat{ \id{\{}u / z \id{\}} }
		& = &
		w[ \lpquote y!(z) \rpquote ] \nonumber
\end{eqnarray}

Because the body of the process between quotes is impervious to
substitution, we get radically different answers. In fact, by
examining the first process in an input context,
e.g. $x?(z).\lift{w}{y!(z)}$, we see that the process under the lift
operator may be shaped by prefixed inputs binding a name inside it. In
this sense, the lift operator will be seen as a way to dynamically
construct processes before reifying them as names.

Finally equipped with these standard features we can present the
dynamics of the calculus.

\subsubsection{Operational semantics} 

Finally, we introduce the computational dynamics. What marks these
algebras as distinct from other more traditionally studied algebraic
structures, e.g. vector spaces or polynomial rings, is the manner in
which dynamics is captured. In traditional structures, dynamics is typically
expressed through morphisms between such structures, as in linear maps
between vector spaces or morphisms between rings. In algebras
associated with the semantics of computation, the dynamics is
expressed as part of the algebraic structure itself, through a
reduction reduction relation typically denoted by $\red$. Below, we
give a recursive presentation of this relation for the calculus used
in the encoding.

$\red \subseteq \pi \times \pi$
$\red : \pi \to \mathcal{P}(\pi)$

\begin{mathpar}
  \inferrule* [lab=Comm] { \textsf{match}( x_{src}, x_{trgt} ) } { x_{trgt}?(y)P \; | \; x_{src}!\langle {Q} \rangle \red P\{\quotep{Q}/y}\} }
  \and \\
  \inferrule* [lab=Par] {{P} \red {P}'} {{{P} | {Q}} \red {{P}' | {Q}}}
  \and
  \inferrule* [lab=Equiv]{{{P} \scong {P}'} \andalso {{P}' \red {Q}'} \andalso {{Q}' \scong {Q}}}{{P} \red {Q}}
\end{mathpar}

\begin{eqnarray*}
  match_{\equiv} (\quotep{P},\quotep{Q}) & := & P \equiv Q \\
  match_{\dagger}(\quotep{P},\quotep{Q}) & := & \forall R. P|Q \red^{*} R => R \red^{*} 0 \\
  match_{K}(\quotep{P},\quotep{Q}) & := & K \mbox{ for some context } K
\end{eqnarray*}

$u?(x)P | u!\langle Q \rangle \red P\{\quotep{Q}/x\}$

%We write $\wred$ for $\red^*$, and $P\red$ if $\exists Q $ such that $ P \red Q$.
We write $P\red$ if $\exists Q $ such that $ P \red Q$ and $P\not\red$, otherwise.

\section{Replication}

As mentioned before, it is known that replication (and hence
recursion) can be implemented in a higher-order process algebra
\cite{SangiorgiWalker}. As our first example of calculation with the
machinery thus far presented we give the construction explicitly in
the {\rhoc}.

\begin{eqnarray}
	D_{x} & := & \prefix{x}{y}{(\binpar{\outputp{x}{y}}{@{y}})} \nonumber\\
	\bangp_{x}{P} & := & \binpar{{x}!\langle{\binpar{D_{x}}{P}}\rangle}{D_{x}} \nonumber
\end{eqnarray}

\begin{eqnarray}
	\bangp_{x}{P} & & \nonumber\\
	=
	& {x}!\langle{(\prefix{x}{y}{(\outputp{x}{y} | @{y})) | P}}\rangle 
	      | \prefix{x}{y}{(\outputp{x}{y} | @{y})} & \nonumber\\
	\red
	& (\outputp{x}{y} | @{y})\substn{\quotep{(\prefix{x}{y}{(@{y} | \outputp{x}{y})) | P}}}{y} & \nonumber\\
	=
	& \outputp{x}{\quotep{(\prefix{x}{y}{(\outputp{x}{y} | @{y})) | P}}}
	  | {(\prefix{x}{y}{(\outputp{x}{y} | @{y})) | P}} & \nonumber\\
	\red
	& \ldots & \nonumber\\
	\red^*
	& P | P | \ldots & \nonumber
\end{eqnarray}

Of course, this encoding, as an implementation, runs away, unfolding
$\bangp{P}$ eagerly. A lazier and more implementable replication
operator, restricted to input-guarded processes, may be obtained as follows.

\begin{eqnarray}
\bangp{\prefix{u}{v}{P}} 
	:= 
	\binpar{\lift{x}{\prefix{u}{v}{(\binpar{D(x)}{P})}}}{D(x)} \nonumber
\end{eqnarray}

\begin{remark}
  Note that the lazier definition still does not deal with summation
  or mixed summation (i.e. sums over input and output). The reader is
  invited to construct definitions of replication that deal with these
  features. 

  Further, the definitions are parameterized in a name, $x$. Can you,
  gentle reader, make a definition that eliminates this parameter and
  guarantees no accidental interaction between the replication
  machinery and the process being replicated -- i.e. no accidental
  sharing of names used by the process to get its work done and the
  name(s) used by the replication to effect copying. This latter
  revision of the definition of replication is crucial to obtaining
  the expected identity $!!P \sim !P$.
\end{remark}

\begin{remark}\label{rem:paradoxical_combinator}
  The reader familiar with the lambda calculus will have noticed the
  similarity between $D$ and the paradoxical combinator.

  [Ed. note: the existence of this seems to suggest we have to be more
  restrictive on the set of processes and names we admit if we are to
  support no-cloning.]
\end{remark}

\subsubsection{Bisimulation}

The computational dynamics gives rise to another kind of equivalence,
the equivalence of computational behavior. As previously mentioned
this is typically captured \emph{via} some form of bisimulation.

% The notion we use in this paper is weak barbed bisimulation
% \cite{milner91polyadicpi}.

The notion we use in this paper is derived from weak barbed
bisimulation \cite{milner91polyadicpi}. 

\begin{definition}
An \emph{observation relation}, $\downarrow_{\mathcal N}$, over a set
of names, $\mathcal N$, is the smallest relation satisfying the rules
below.

\infrule[Out-barb]{y \in {\mathcal N}, \; x \nameeq y}
		  {\outputp{x}{v} \downarrow_{\mathcal N} x}
\infrule[Par-barb]{\mbox{$P\downarrow_{\mathcal N} x$ or $Q\downarrow_{\mathcal N} x$}}
		  {\binpar{P}{Q} \downarrow_{\mathcal N} x}

We write $P \Downarrow_{\mathcal N} x$ if there is $Q$ such that 
$P \wred Q$ and $Q \downarrow_{\mathcal N} x$.
\end{definition}

\begin{definition}
%\label{def.bbisim}
An  ${\mathcal N}$-\emph{barbed bisimulation} over a set of names, ${\mathcal N}$, is a symmetric binary relation 
${\mathcal S}_{\mathcal N}$ between agents such that $P\rel{S}_{\mathcal N}Q$ implies:
\begin{enumerate}
\item If $P \red P'$ then $Q \wred Q'$ and $P'\rel{S}_{\mathcal N} Q'$.
\item If $P\downarrow_{\mathcal N} x$, then $Q\Downarrow_{\mathcal N} x$.
\end{enumerate}
$P$ is ${\mathcal N}$-barbed bisimilar to $Q$, written
$P \wbbisim_{\mathcal N} Q$, if $P \rel{S}_{\mathcal N} Q$ for some ${\mathcal N}$-barbed bisimulation ${\mathcal S}_{\mathcal N}$.
\end{definition}

$\mathcal{R} \subseteq \pi \times \pi$

$P \mathcal{R} Q => \forall P'. P \red P' \Rightarrow \exists Q'. Q \red Q', P' \mathcal{R} Q'$

$P \vdash x \Rightarrow Q \vdash x$

\begin{mathpar}
  \inferrule*[lab=Out-barb]{x \nameeq y}{{y}!\langle{Q}\rangle \vdash x}
  \and
  \inferrule*[lab=Par-barb]{\mbox{$P\vdash x$ or $Q\vdash x$}}{\binpar{P}{Q} \vdash x}
\end{mathpar}

\subsubsection{Contexts}

One of the principle advantages of computational calculi like the
$\pi$-calculus is a well-defined notion of context,
contextual-equivalence and a correlation between
contextual-equivalence and notions of bisimulation. The notion of
context allows the decomposition of a process into (sub-)process and
its syntactic environment, its context. Thus, a context may be
thought of as a process with a ``hole'' (written $\Box$) in it. The
application of a context $M$ to a process $P$, written $M[P]$, is
tantamount to filling the hole in $M$ with $P$. In this paper we do
not need the full weight of this theory, but do make use of the notion
of context in the proof the main theorem. 

\begin{mathpar}
  \inferrule* [lab=summation] {} {{M_{M},M_{N}} \bc \Box \;|\; x.M_{A} \;|\; M_{M}+M_{N}}
  \and
  \inferrule* [lab=agent] {} {{M_{A}} \bc (\vec{x})M_{P} \;| \; \clift{P_0,\ldots,M_{P},\ldots,P_N}}
  \and \\
  \inferrule* [lab=process] {} {{M_{P}} \bc M_{N} \;| \;P|M_{P} }
\end{mathpar} 

\begin{mathpar}
  \inferrule* [lab=sychronization] {} {M_{N} \bc \Box \;|\; x?M_{F} \;|\; x!M_{C}}
  \and
  \inferrule* [lab=abstraction] {} {{M_{F}} \bc (x)M_{P} }
  \and
  \inferrule* [lab=concretion] {} {{M_{C}} \bc \langle M_{P} \rangle }
  \and \\
  \inferrule* [lab=process] {} {{M_{P}} \bc M_{N} \;| \;P|M_{P} }
\end{mathpar}

\begin{definition}[contextual application] Given a context $M$, and
  process $P$, we define the \emph{contextual application}, $M[P] :=
  M\{P/\Box\}$. That is, the contextual application of M to P is the
  substitution of $P$ for $\Box$ in $M$.
\end{definition}

$\meaningof{-} : L \to \mathcal{P}(\pi)$

\begin{mathpar}
  \inferrule* [lab=collection] {} {\meaningof{true} = \pi, \and \meaningof{~E} = \pi \setminus \meaningof{E}, \and \meaningof{E_{1} \& E_{2}} = \meaningof{E_{1}} \cap \meaningof{E_{2}}}
\end{mathpar}

\begin{mathpar}
  \inferrule* [lab=structure] {} {\meaningof{0} = \{ P \in \pi | P \equiv 0 \}, \and \\ \meaningof{E_1 | E_2} = \{ P \in \pi | P \equiv P_{1} | P_{2}, P_{1} \in \meaningof{E_{1}}, P_{2} \in \meaningof{E_2}\} }
\end{mathpar}

\begin{mathpar}
 \inferrule* [lab=behavior] {} {\meaningof{\langle a?b \rangle E} = \{ P \in \pi | P \equiv Q | u?(y)P', \\ \and \\\\ \and \\ \;\;\; u \in \meaningof{a}, \forall z.P'\{z/y\} \in \meaningof{E\{z/b\}}\}, \and \\ \meaningof{a!E} = \{ P \in \pi | P \equiv Q | x!\langle P' \rangle, x \in \meaningof{a} P' \in \meaningof{E}\} }
\end{mathpar}

\begin{mathpar}
 \inferrule* [lab=nominal] {} {\meaningof{\quotep{E}} = \{ \quotep{P} \in \quotep{\pi} | P \in \meaningof{E} \}, \and \meaningof{\quotep{P}} = \{ \quotep{Q} \in \quotep{\pi} | P \equiv Q \} \and \\ \meaningof{@\quotep{E}} = \{ P \in \pi | P \equiv @x, x \in \meaningof{E} \}}
\end{mathpar}

\begin{eqnarray*}
  \\
  \meaningof{-} : TS \to ST
\end{eqnarray*}

\begin{eqnarray*}
  \\
  L : TS \to ST
\end{eqnarray*}

\begin{eqnarray*}
  \\
  P \models E \iff P \in \meaningof{E}
\end{eqnarray*}

\begin{eqnarray*}
  P \approx_{L} Q \iff \forall E \in L. P \models E \iff Q \models E
\end{eqnarray*}

\begin{eqnarray*}
  P \approx_{K} Q
\end{eqnarray*}

\begin{eqnarray*}
  P \approx Q
\end{eqnarray*}

$\approx_{K} = \approx = \approx_{L}$

\subsubsection{Contextual duality}

Note that contexts extend the quotation operation to a family of
operations from processes to names. Given a context, $M$, we can
define a \emph{nominal context}, $\quotep{M}$ by $\quotep{M}[P] :=
\quotep{M[P]}$. To foreshadow what is to come we observe that these
operations enjoy a duality with processes very much like the duality
between vectors and maps from vectors to scalars.

Further, because the calculus is essentially higher-order, we have a
correspondence between contexts and processes. More specifically,
given a name $x$ and a context $M$ we can construct $M^{*}_{x}$ such
that 

\begin{mathpar}
  M^{*}_{x} | \lift{x}{P} \red M[P]
\end{mathpar}

namely,

\begin{mathpar}
  M^{*}_{x} := x?(u).M[\dropn{u}]
\end{mathpar}

The dependence of $M^{*}_{x}$ on a name makes it an abstraction, 

\begin{mathpar}
  M^{*} := (x)x?(u).M[\dropn{u}]
\end{mathpar}

\subsection{Additional notation}

It will sometimes be convenient to denote the process a name
quotes. We already have the notation $x = \quotep{P}$, but it will be
convenient to introduce an alternate notation, $\procn{x}$, when we
want to emphasize the connection to the use of the name. Note that, by
virtue of name equivalence, $\quotep{\procn{x}} \nameeq x$; so, the
notation is consistent with previous definitions.

Further, because names have structure it is possible to effect
substitutions on the basis of that structure. This means we need to
upgrade our notation for substitutions, which we accomplish by
adapting comprehension notation. Thus,

\begin{mathpar}
  P\{ y / x : x \in S \}
\end{mathpar}

is interpreted to mean the process derived from P by replacing (in a
capture-avoiding manner) each occurrence of $x$ in $S$ by $y$. For example,

\begin{mathpar}
  P\{ \quotep{\procn{x}|\procn{x}} / x : x \in \freenames{P} \}
\end{mathpar}

will replace each (occurrence) of a free name $x$ in $P$ by
$\quotep{\procn{x}|\procn{x}}$.

Also, we will avail ourselves of the notation $x^{L}$ and $x^{R}$ to
denote injections of a name into disjoint copies of the name
space. There are numerous ways to accomplish this. One example can be
found in \cite{MeredithR05}. This notation overloads to vectors of
names: $\vec{x}^{\pi} := (x_{i}^{\pi} \; : \; 0 \leq i < |\vec{x}| )$ where $\pi \in \{L,R\}$.

We also use $P^{\Box} := P|\Box$.

In \cite{MeredithR05} an interpretation of the new operator is
given. It turns out that there are several possible interpretations
all enjoying the requisite algebraic properties of the operator (see
\cite{milner91polyadicpi}). We will therefore make liberal use of
$(\nu\; \vec{x})P$.

% subsection the_syntax_and_semantics_of_the_notation_system (end)   

\input{qm2pi.qmops} 

\input{qm2pi.sterngerlach} 

\input{qm2pi.metric} 

% section concurrent_process_calculi (end)

%\input{qm2pi.proofsketch}

% section proof sketch (end)

%\input{qm2pi.slviaknots} 

% section spatial logic via knots (end)

\input{qm2pi.conclusion}

% section conclusion (end)

%\input{qm2pi.dtcodes} 

% section wiring algorithm (end)

\input{qm2pi.ack} 

% section acknowledgments (end)

\newpage


\bibliographystyle{plain}   
\bibliography{../../biblios/main.bib}

\input{qm2pi.rhodetails}

\end{document}

 

% section acknowledgments (end)

\newpage


\bibliographystyle{plain}   
\bibliography{../../biblios/main.bib}

\documentclass[12pt]{llncs}
%\documentclass{jktr}

\usepackage[pdftex]{hyperref}                   
\usepackage {listings}
\usepackage {mathpartir}
\usepackage{bcprules}
%\usepackage{listings}
                       
\usepackage{graphicx} 
%\usepackage[margins=2.5cm,nohead,nofoot]{geometry}
%\usepackage{geometry}
\usepackage{amsfonts}
\usepackage{amstext}
\usepackage{latexsym}
\usepackage{amssymb}
\usepackage{color}


%\include{myPreamble}
\include{qm2pi.local} 

%\ifpdf
%\usepackage[pdftex]{graphicx}
%\else
%\usepackage{graphicx}
%\fi

 % \ifpdf
%  \usepackage{pdfsync}
%  \if


%\title{Brief Article}
%\author{David F. Snyder}
%\author{L.G. Meredith}

%\address{Dept. of Math., Texas State University--San Marcos, San Marcos, TX 78666}
       
\pagestyle{empty}


\begin{document}

\lstset{language=[Objective]Caml,frame=shadowbox}

\input{qm2pi.front}

% section front matter (end)

\input{qm2pi.intro} 
 
% section introduction (end)

% \input{qm2pi.knotations} 

% section notation (end)

\input{qm2pi.process.calculi} 

% section concurrent_process_calculi_and_spatial_logics_ (end)
    
%\input{qm2pi.knots2pi} 

%\input{qm2pi.trefoil} 

%\input{qm2pi.mainthm} 

% subsection basic_interpretation (end)

%\input{qm2pi.rho.presentation} 
\subsection{The syntax and semantics of the notation system}\label{sub:the_syntax_and_semantics_of_the_notation_system} % (fold)

We now summarize a technical presentation of the calculus that
embodies our theory of dynamics. The typical presentation of such a
calculus follows the style of giving generators and relations on
them. The grammar, below, describing term constructors, freely
generates the set of processes, $\Proc$. This set is then quotiented
by a relation known as structural congruence and it is over this set
that the notion of dynamics is expressed. This presentation is
essentially that of \cite{MeredithR05} with the addition of
polyadicity and summation. For readability we have relegated some of
the technical subtleties to an appendix.

\subsubsection{Process grammar}\label{subsub:process_grammar}

\begin{mathpar}
  \inferrule* [lab=synchronization] {} {{M} \bc \pzero \;|\; x?F \;|\; x!C }
  \and
  \inferrule* [lab=abstraction] {} {{F} \bc (x)P}
  \and
  \inferrule* [lab=concretion] {} {{C} \bc \langle Q \rangle}
  \and
  \inferrule* [lab=process] {} {{P,Q} \bc M \;| \;P|Q \;|\; @{x}}
  \and
  \inferrule* [lab=name] {} {{x} \bc \quotep{P}}
\end{mathpar} 

Note that $\vec{x}$ (resp. $\vec{P}$) denotes a vector of names
(resp. processes) of length $|\vec{x}|$ (resp. $|\vec{P}|$). We adopt
the following useful abbreviations.

\begin{mathpar}
   x?(\vec{y}).P := x.(\vec{y})P \and  x\clift{\vec{P}} := x.\clift{\vec{P}}
   \and x!(y) := \lift{x}{\dropn{y}}
   \and \Pi_{i=0}^{n-1}P_i := P_0 | \ldots | P_{n-1}
\end{mathpar}

\subsubsection{Structural congruence}

\paragraph{Free and bound names and alpha-equivalence.} At the
core of structural equivalence is alpha-equivalence which identifies
process that are the same up to a change of variable. Formally, we
recognize the distinction between free and bound names. The free names
of a process, $\freenames{P}$, may be calculated recursively as
follows:

\begin{mathpar}
\freenames{\pzero} := \emptyset
  \and \\
  \freenames{x?(y).P} := \{ x \} \cup (\freenames{P} \setminus \{ y \})
  \and 
  \freenames{x!\langle P \rangle} := \{ x \} \cup \{ P \} 
  \and \\
  \freenames{P|Q} := \freenames{P} \cup \freenames{Q}
  \and \\
  \freenames{@{x}} := \{ x \}
\end{mathpar}

$\pi$
$\quotep{\pi}$

$\freenames{-} : \pi \to \mathcal{P}(\quotep{\pi})$

\begin{eqnarray*}
  \freenames{\pzero} & := & \emptyset \\
  \freenames{x?(y).P} & := & \{ x \} \cup (\freenames{P} \setminus \{ y \}) \\
  \freenames{x!\langle P \rangle} & := & \{ x \} \cup \{ P \} \\
  \freenames{P|Q} & := & \freenames{P} \cup \freenames{Q} \\
  \freenames{\dropn{x}} & := & \{ x \}
\end{eqnarray*}

The bound names of a process, $\boundnames{P}$, are those names occurring in $P$
that are not free. For example, in $x?(y).0$, the name $x$ is free, while $y$ is bound.

\begin{mathpar}
  \inferrule* [lab=monoidal-laws] {} { P|Q \equiv Q|P \and P|0 \equiv P \and P|(Q|R) \equiv (P|Q)|R }
\end{mathpar}

\begin{mathpar}
  \inferrule* [lab=alpha-equivalence] {} { (x)P \equiv (y)P\{y/x\} \and y \not\in \freenames{P} }
\end{mathpar}

\begin{definition}
Then two processes, $P,Q$, are alpha-equivalent if $P = Q\{\vec{y}/\vec{x}\}$ for
some $\vec{x} \in \boundnames{Q},\vec{y} \in \boundnames{P}$, where $Q\{\vec{y}/\vec{x}\}$
denotes the capture-avoiding substitution of $\vec{y}$ for $\vec{x}$ in $Q$.
\end{definition}

\begin{definition}
  The {\em structural congruence} \cite{SangiorgiWalker} , $\equiv$,
  between processes is the least congruence containing
  alpha-equivalence, satisfying the abelian monoid laws
  (associativity, commutativity and $\pzero$ as identity) for parallel
  composition $|$ and for summation $+$.
\end{definition}

\subsection{Name equivalence}

We take name equivalence, written $\nameeq$, to be the smallest
equivalence relation generated by the following rules.

\begin{mathpar}
\inferrule*[lab=Quote-drop]
{ }
{ \quotep{@{x}} \nameeq x }

\inferrule*[lab=Struct-equiv]
{ P \scong Q }
{ \quotep{P} \nameeq \quotep{Q} }
\end{mathpar}

The astute reader will have noticed that the mutual recursion of names
and processes imposes a mutual recursion on alpha-equivalence and
structural equivalence via name-equivalence. Fortunately, all of this
works out pleasantly and we may calculate in the natural way, free of
concern. The reader interested in the details is referred to the
appendix \ref{appendix:rho_details}.

\subsection{Substitution}

We use $\Proc$ for the set of processes, $\QProc$ for the set of
names, and $\id{\{}\vec{y} / \vec{x} \id{\}}$ to denote partial maps,
$s : \QProc \rightarrow \QProc$. A map, $s$ lifts, uniquely, to a map
on process terms, $\widehat{s} : \Proc \rightarrow \Proc$ by the
following equations.

\begin{mathpar}
  (0) \psubstp{Q}{P} := 0 \\
  (R \juxtap S) \psubstp{Q}{P}
  :=    
  (R)\psubstp{Q}{P} \juxtap (S) \psubstp{Q}{P} \\
  (x?(y).R) \psubstp{Q}{P}    
  :=    
  (x)\substp{Q}{P} (z)\concat( (R \psubstn{z}{y}) \psubstp{Q}{P} ) \\
  (\lift{x}{R}) \psubstp{Q}{P}  
  :=
  \lift{(x)\substp{Q}{P}}{ R \psubstp{Q}{P} } \\
%   (\dropn{x})  \psubstp{Q}{P}       
%   := 
%   \left\{ 
%     \begin{array}{ccc} 
%       \dropn{\quotep{Q}} & & x \nameeq \quotep{P} \\
%       \dropn{x} & & otherwise \\
%     \end{array}
%   \right. 
  (\dropn{x})  \psubstp{Q}{P}       
  := 
  \left\{ 
    \begin{array}{ccc} 
      Q & & x \nameeq \quotep{P} \\
      \dropn{x} & & otherwise \\
    \end{array}
  \right.
\end{mathpar}
 

where

\begin{eqnarray}
  (x)\id{\{} \lpquote Q \rpquote / \lpquote P \rpquote \id{\}}            = 
  \left\{ 
    \begin{array}{ccc}
      \lpquote Q \rpquote & & x \nameeq \lpquote P \rpquote \\
      x & & otherwise \\
    \end{array}
  \right. \nonumber
\end{eqnarray}

and $z$ is chosen distinct from $\quotep{P}$, $\quotep{Q}$, the free
names in $Q$, and all the names in $R$. Our $\alpha$-equivalence will
be built in the standard way from this substitution.

\begin{remark}\label{rem:no_self_referential_names}
  One consequence of these definitions is that $\forall P. \quotep{P}
  \not\in \freenames{P}$.
\end{remark}

\subsection{ Dynamic quote: an example }

Anticipating something of what's to come, consider applying the
substitution, $\widehat{\id{\{}u / z \id{\}}}$, to the following pair
of processes, $\lift{w}{y!(z)}$ and $w[ \lpquote y!(z) \rpquote ]$.

\begin{eqnarray}
	\lift{w}{y!(z)}\widehat{\id{\{}u / z \id{\}}}
		& = &
		\lift{w}{y!(u)} \nonumber\\
	w[ \lpquote y!(z) \rpquote ] \widehat{ \id{\{}u / z \id{\}} }
		& = &
		w[ \lpquote y!(z) \rpquote ] \nonumber
\end{eqnarray}

Because the body of the process between quotes is impervious to
substitution, we get radically different answers. In fact, by
examining the first process in an input context,
e.g. $x?(z).\lift{w}{y!(z)}$, we see that the process under the lift
operator may be shaped by prefixed inputs binding a name inside it. In
this sense, the lift operator will be seen as a way to dynamically
construct processes before reifying them as names.

Finally equipped with these standard features we can present the
dynamics of the calculus.

\subsubsection{Operational semantics} 

Finally, we introduce the computational dynamics. What marks these
algebras as distinct from other more traditionally studied algebraic
structures, e.g. vector spaces or polynomial rings, is the manner in
which dynamics is captured. In traditional structures, dynamics is typically
expressed through morphisms between such structures, as in linear maps
between vector spaces or morphisms between rings. In algebras
associated with the semantics of computation, the dynamics is
expressed as part of the algebraic structure itself, through a
reduction reduction relation typically denoted by $\red$. Below, we
give a recursive presentation of this relation for the calculus used
in the encoding.

$\red \subseteq \pi \times \pi$
$\red : \pi \to \mathcal{P}(\pi)$

\begin{mathpar}
  \inferrule* [lab=Comm] { \textsf{match}( x_{src}, x_{trgt} ) } { x_{trgt}?(y)P \; | \; x_{src}!\langle {Q} \rangle \red P\{\quotep{Q}/y}\} }
  \and \\
  \inferrule* [lab=Par] {{P} \red {P}'} {{{P} | {Q}} \red {{P}' | {Q}}}
  \and
  \inferrule* [lab=Equiv]{{{P} \scong {P}'} \andalso {{P}' \red {Q}'} \andalso {{Q}' \scong {Q}}}{{P} \red {Q}}
\end{mathpar}

\begin{eqnarray*}
  match_{\equiv} (\quotep{P},\quotep{Q}) & := & P \equiv Q \\
  match_{\dagger}(\quotep{P},\quotep{Q}) & := & \forall R. P|Q \red^{*} R => R \red^{*} 0 \\
  match_{K}(\quotep{P},\quotep{Q}) & := & K \mbox{ for some context } K
\end{eqnarray*}

$u?(x)P | u!\langle Q \rangle \red P\{\quotep{Q}/x\}$

%We write $\wred$ for $\red^*$, and $P\red$ if $\exists Q $ such that $ P \red Q$.
We write $P\red$ if $\exists Q $ such that $ P \red Q$ and $P\not\red$, otherwise.

\section{Replication}

As mentioned before, it is known that replication (and hence
recursion) can be implemented in a higher-order process algebra
\cite{SangiorgiWalker}. As our first example of calculation with the
machinery thus far presented we give the construction explicitly in
the {\rhoc}.

\begin{eqnarray}
	D_{x} & := & \prefix{x}{y}{(\binpar{\outputp{x}{y}}{@{y}})} \nonumber\\
	\bangp_{x}{P} & := & \binpar{{x}!\langle{\binpar{D_{x}}{P}}\rangle}{D_{x}} \nonumber
\end{eqnarray}

\begin{eqnarray}
	\bangp_{x}{P} & & \nonumber\\
	=
	& {x}!\langle{(\prefix{x}{y}{(\outputp{x}{y} | @{y})) | P}}\rangle 
	      | \prefix{x}{y}{(\outputp{x}{y} | @{y})} & \nonumber\\
	\red
	& (\outputp{x}{y} | @{y})\substn{\quotep{(\prefix{x}{y}{(@{y} | \outputp{x}{y})) | P}}}{y} & \nonumber\\
	=
	& \outputp{x}{\quotep{(\prefix{x}{y}{(\outputp{x}{y} | @{y})) | P}}}
	  | {(\prefix{x}{y}{(\outputp{x}{y} | @{y})) | P}} & \nonumber\\
	\red
	& \ldots & \nonumber\\
	\red^*
	& P | P | \ldots & \nonumber
\end{eqnarray}

Of course, this encoding, as an implementation, runs away, unfolding
$\bangp{P}$ eagerly. A lazier and more implementable replication
operator, restricted to input-guarded processes, may be obtained as follows.

\begin{eqnarray}
\bangp{\prefix{u}{v}{P}} 
	:= 
	\binpar{\lift{x}{\prefix{u}{v}{(\binpar{D(x)}{P})}}}{D(x)} \nonumber
\end{eqnarray}

\begin{remark}
  Note that the lazier definition still does not deal with summation
  or mixed summation (i.e. sums over input and output). The reader is
  invited to construct definitions of replication that deal with these
  features. 

  Further, the definitions are parameterized in a name, $x$. Can you,
  gentle reader, make a definition that eliminates this parameter and
  guarantees no accidental interaction between the replication
  machinery and the process being replicated -- i.e. no accidental
  sharing of names used by the process to get its work done and the
  name(s) used by the replication to effect copying. This latter
  revision of the definition of replication is crucial to obtaining
  the expected identity $!!P \sim !P$.
\end{remark}

\begin{remark}\label{rem:paradoxical_combinator}
  The reader familiar with the lambda calculus will have noticed the
  similarity between $D$ and the paradoxical combinator.

  [Ed. note: the existence of this seems to suggest we have to be more
  restrictive on the set of processes and names we admit if we are to
  support no-cloning.]
\end{remark}

\subsubsection{Bisimulation}

The computational dynamics gives rise to another kind of equivalence,
the equivalence of computational behavior. As previously mentioned
this is typically captured \emph{via} some form of bisimulation.

% The notion we use in this paper is weak barbed bisimulation
% \cite{milner91polyadicpi}.

The notion we use in this paper is derived from weak barbed
bisimulation \cite{milner91polyadicpi}. 

\begin{definition}
An \emph{observation relation}, $\downarrow_{\mathcal N}$, over a set
of names, $\mathcal N$, is the smallest relation satisfying the rules
below.

\infrule[Out-barb]{y \in {\mathcal N}, \; x \nameeq y}
		  {\outputp{x}{v} \downarrow_{\mathcal N} x}
\infrule[Par-barb]{\mbox{$P\downarrow_{\mathcal N} x$ or $Q\downarrow_{\mathcal N} x$}}
		  {\binpar{P}{Q} \downarrow_{\mathcal N} x}

We write $P \Downarrow_{\mathcal N} x$ if there is $Q$ such that 
$P \wred Q$ and $Q \downarrow_{\mathcal N} x$.
\end{definition}

\begin{definition}
%\label{def.bbisim}
An  ${\mathcal N}$-\emph{barbed bisimulation} over a set of names, ${\mathcal N}$, is a symmetric binary relation 
${\mathcal S}_{\mathcal N}$ between agents such that $P\rel{S}_{\mathcal N}Q$ implies:
\begin{enumerate}
\item If $P \red P'$ then $Q \wred Q'$ and $P'\rel{S}_{\mathcal N} Q'$.
\item If $P\downarrow_{\mathcal N} x$, then $Q\Downarrow_{\mathcal N} x$.
\end{enumerate}
$P$ is ${\mathcal N}$-barbed bisimilar to $Q$, written
$P \wbbisim_{\mathcal N} Q$, if $P \rel{S}_{\mathcal N} Q$ for some ${\mathcal N}$-barbed bisimulation ${\mathcal S}_{\mathcal N}$.
\end{definition}

$\mathcal{R} \subseteq \pi \times \pi$

$P \mathcal{R} Q => \forall P'. P \red P' \Rightarrow \exists Q'. Q \red Q', P' \mathcal{R} Q'$

$P \vdash x \Rightarrow Q \vdash x$

\begin{mathpar}
  \inferrule*[lab=Out-barb]{x \nameeq y}{{y}!\langle{Q}\rangle \vdash x}
  \and
  \inferrule*[lab=Par-barb]{\mbox{$P\vdash x$ or $Q\vdash x$}}{\binpar{P}{Q} \vdash x}
\end{mathpar}

\subsubsection{Contexts}

One of the principle advantages of computational calculi like the
$\pi$-calculus is a well-defined notion of context,
contextual-equivalence and a correlation between
contextual-equivalence and notions of bisimulation. The notion of
context allows the decomposition of a process into (sub-)process and
its syntactic environment, its context. Thus, a context may be
thought of as a process with a ``hole'' (written $\Box$) in it. The
application of a context $M$ to a process $P$, written $M[P]$, is
tantamount to filling the hole in $M$ with $P$. In this paper we do
not need the full weight of this theory, but do make use of the notion
of context in the proof the main theorem. 

\begin{mathpar}
  \inferrule* [lab=summation] {} {{M_{M},M_{N}} \bc \Box \;|\; x.M_{A} \;|\; M_{M}+M_{N}}
  \and
  \inferrule* [lab=agent] {} {{M_{A}} \bc (\vec{x})M_{P} \;| \; \clift{P_0,\ldots,M_{P},\ldots,P_N}}
  \and \\
  \inferrule* [lab=process] {} {{M_{P}} \bc M_{N} \;| \;P|M_{P} }
\end{mathpar} 

\begin{mathpar}
  \inferrule* [lab=sychronization] {} {M_{N} \bc \Box \;|\; x?M_{F} \;|\; x!M_{C}}
  \and
  \inferrule* [lab=abstraction] {} {{M_{F}} \bc (x)M_{P} }
  \and
  \inferrule* [lab=concretion] {} {{M_{C}} \bc \langle M_{P} \rangle }
  \and \\
  \inferrule* [lab=process] {} {{M_{P}} \bc M_{N} \;| \;P|M_{P} }
\end{mathpar}

\begin{definition}[contextual application] Given a context $M$, and
  process $P$, we define the \emph{contextual application}, $M[P] :=
  M\{P/\Box\}$. That is, the contextual application of M to P is the
  substitution of $P$ for $\Box$ in $M$.
\end{definition}

$\meaningof{-} : L \to \mathcal{P}(\pi)$

\begin{mathpar}
  \inferrule* [lab=collection] {} {\meaningof{true} = \pi, \and \meaningof{~E} = \pi \setminus \meaningof{E}, \and \meaningof{E_{1} \& E_{2}} = \meaningof{E_{1}} \cap \meaningof{E_{2}}}
\end{mathpar}

\begin{mathpar}
  \inferrule* [lab=structure] {} {\meaningof{0} = \{ P \in \pi | P \equiv 0 \}, \and \\ \meaningof{E_1 | E_2} = \{ P \in \pi | P \equiv P_{1} | P_{2}, P_{1} \in \meaningof{E_{1}}, P_{2} \in \meaningof{E_2}\} }
\end{mathpar}

\begin{mathpar}
 \inferrule* [lab=behavior] {} {\meaningof{\langle a?b \rangle E} = \{ P \in \pi | P \equiv Q | u?(y)P', \\ \and \\\\ \and \\ \;\;\; u \in \meaningof{a}, \forall z.P'\{z/y\} \in \meaningof{E\{z/b\}}\}, \and \\ \meaningof{a!E} = \{ P \in \pi | P \equiv Q | x!\langle P' \rangle, x \in \meaningof{a} P' \in \meaningof{E}\} }
\end{mathpar}

\begin{mathpar}
 \inferrule* [lab=nominal] {} {\meaningof{\quotep{E}} = \{ \quotep{P} \in \quotep{\pi} | P \in \meaningof{E} \}, \and \meaningof{\quotep{P}} = \{ \quotep{Q} \in \quotep{\pi} | P \equiv Q \} \and \\ \meaningof{@\quotep{E}} = \{ P \in \pi | P \equiv @x, x \in \meaningof{E} \}}
\end{mathpar}

\begin{eqnarray*}
  \\
  \meaningof{-} : TS \to ST
\end{eqnarray*}

\begin{eqnarray*}
  \\
  L : TS \to ST
\end{eqnarray*}

\begin{eqnarray*}
  \\
  P \models E \iff P \in \meaningof{E}
\end{eqnarray*}

\begin{eqnarray*}
  P \approx_{L} Q \iff \forall E \in L. P \models E \iff Q \models E
\end{eqnarray*}

\begin{eqnarray*}
  P \approx_{K} Q
\end{eqnarray*}

\begin{eqnarray*}
  P \approx Q
\end{eqnarray*}

$\approx_{K} = \approx = \approx_{L}$

\subsubsection{Contextual duality}

Note that contexts extend the quotation operation to a family of
operations from processes to names. Given a context, $M$, we can
define a \emph{nominal context}, $\quotep{M}$ by $\quotep{M}[P] :=
\quotep{M[P]}$. To foreshadow what is to come we observe that these
operations enjoy a duality with processes very much like the duality
between vectors and maps from vectors to scalars.

Further, because the calculus is essentially higher-order, we have a
correspondence between contexts and processes. More specifically,
given a name $x$ and a context $M$ we can construct $M^{*}_{x}$ such
that 

\begin{mathpar}
  M^{*}_{x} | \lift{x}{P} \red M[P]
\end{mathpar}

namely,

\begin{mathpar}
  M^{*}_{x} := x?(u).M[\dropn{u}]
\end{mathpar}

The dependence of $M^{*}_{x}$ on a name makes it an abstraction, 

\begin{mathpar}
  M^{*} := (x)x?(u).M[\dropn{u}]
\end{mathpar}

\subsection{Additional notation}

It will sometimes be convenient to denote the process a name
quotes. We already have the notation $x = \quotep{P}$, but it will be
convenient to introduce an alternate notation, $\procn{x}$, when we
want to emphasize the connection to the use of the name. Note that, by
virtue of name equivalence, $\quotep{\procn{x}} \nameeq x$; so, the
notation is consistent with previous definitions.

Further, because names have structure it is possible to effect
substitutions on the basis of that structure. This means we need to
upgrade our notation for substitutions, which we accomplish by
adapting comprehension notation. Thus,

\begin{mathpar}
  P\{ y / x : x \in S \}
\end{mathpar}

is interpreted to mean the process derived from P by replacing (in a
capture-avoiding manner) each occurrence of $x$ in $S$ by $y$. For example,

\begin{mathpar}
  P\{ \quotep{\procn{x}|\procn{x}} / x : x \in \freenames{P} \}
\end{mathpar}

will replace each (occurrence) of a free name $x$ in $P$ by
$\quotep{\procn{x}|\procn{x}}$.

Also, we will avail ourselves of the notation $x^{L}$ and $x^{R}$ to
denote injections of a name into disjoint copies of the name
space. There are numerous ways to accomplish this. One example can be
found in \cite{MeredithR05}. This notation overloads to vectors of
names: $\vec{x}^{\pi} := (x_{i}^{\pi} \; : \; 0 \leq i < |\vec{x}| )$ where $\pi \in \{L,R\}$.

We also use $P^{\Box} := P|\Box$.

In \cite{MeredithR05} an interpretation of the new operator is
given. It turns out that there are several possible interpretations
all enjoying the requisite algebraic properties of the operator (see
\cite{milner91polyadicpi}). We will therefore make liberal use of
$(\nu\; \vec{x})P$.

% subsection the_syntax_and_semantics_of_the_notation_system (end)   

\input{qm2pi.qmops} 

\input{qm2pi.sterngerlach} 

\input{qm2pi.metric} 

% section concurrent_process_calculi (end)

%\input{qm2pi.proofsketch}

% section proof sketch (end)

%\input{qm2pi.slviaknots} 

% section spatial logic via knots (end)

\input{qm2pi.conclusion}

% section conclusion (end)

%\input{qm2pi.dtcodes} 

% section wiring algorithm (end)

\input{qm2pi.ack} 

% section acknowledgments (end)

\newpage


\bibliographystyle{plain}   
\bibliography{../../biblios/main.bib}

\input{qm2pi.rhodetails}

\end{document}



\end{document}



% section front matter (end)

\section{Introduction}\label{sec:introduction} % (fold)
In this draft of the material i am going to have to dispense with the
usual writing conventions adopted in papers on these topics. i'm going
to have adopt whatever tone i need at the time i'm writing up the
calculations. Sometimes this may be very conversational; others it may
be the barest mathematical grunts; others still it may be that i have
lifted text from one of my other papers because the exposition of some
point was better said there. i hope that my readers are not unduly put
out by this decision. i'm not doing this to flout convention or be
rebellious. i find these calculations very technically challenging. To
keep everything going technically, something has to give; i have to
let go of some cognitive burden. So, the academic writing style --
with all of its trade-offs in terms of facilitating technical
communication -- is what i'm letting go of. Perhaps subsequent drafts
can be tightened and polished, but for now, i'm going to speak as if
we were sitting together in a coffee shop with a laptop, wifi and a
pad of paper and a pencil.

So, here's what i have to say. We -- you and i, comfortably ensconced
in our coffee shop and well-equipped with our tools -- can realize and
carry out the calculations of quantum mechanics over a very different
formal theory of dynamics, a formal theory of dynamics that
corresponds to a theory of concurrent computation with
\emph{reflection}. It has the advantage that the underlying theory is
already `quantized', but supports analogues all of the continuuous
operations. Strikingly, this underlying theory has recently been
connected with a notion of metric that we can show, by calculating
together, coincides with the metric induced by the inner product.

There are a lot of reasons why you might be interested in seeing
calculations of this form. Here's why i'm interested. For the past
several centuries there has been no competitor to the ``Newtonian''
account of dynamics. As a result the predominant share of accounts of
dynamical systems and situations have had to be formulated in terms of
the Newtonian machinery. i view this as an intellectually dangerous
position to occupy. Everything, despite it's intrinsic shape, turns
into a nail to be hit with this hammer. Recently, however, the theory
of computation has matured to the point where we have candidates for
theories of dynamics that offer very different perspective on
reasoning about dynamical systems and situations. Testing these
candidates against very successful accounts of dynamical situations,
like quantum mechanics, is going to give us some sense of how mature
they are and some measure of the quality of these accounts of
dynamics.

\subsection{Summary of contributions and outline of paper}

So, we're going to develop an interpretation of the operations of
quantum mechanics normally interpreted by Hilbert spaces and
operators. We're going to do this over a theory of computation. Note
that this is very different than the usual quantum computation program
which develops notions of computation over quantum mechanics. Rather,
we are developing a story that aligns with Wheeler's slogan: It from
Bit. To do this we will first provide an account of the theory of
computation at play here. Then we will dive into a calculation-driven
interpretation of the operations of quantum mechanics.

The reason we take this approach is that -- until very recently --
there hasn't been an axiomatic account of quantum mechanics. As a
result there has been no sharp delineation of the mathematical theory
supporting interpretation of the physical theory and the physical
theory, itself. So, ambient features of the maths are free to be
exploited (or supressed) without a real accounting of their physical
relevance. There is no sharp statement ``here's the physical theory''
qua \emph{theory} and ``here's the mathematical interpretation''
enabling a judgment of how faithful the interpretation is -- apart
from experimental observation. When there is an axiomatic account we
can judge how well a given mathematical formalism supports an
interpretation of the axioms, independent of
experimentation. Likewise, we can judge how well we have captured our
physical evidence and experience with our axiomatics, independent of
any specific mathematical implementation, with accidental detail that
may or may not have physical significance. 

In lieu of a fully fleshed out and vetted axiomatic account of quantum
mechanics, interpreting the operational notions in service of modeling
physical systems will have to suffice. In other words, we are not in
the business of providing a model of Hilbert spaces and operators. We
are in the business of providing a model of quantum mechanics because
we are motivated by testing our notions of dynamics against physical
theory; and, the predictive calculations of the physical theory must
serve as the best formulation -- shy of a fully fleshed out axiomatic
account -- of the physical theory itself (as they have for scientific
theories since time immemorial). Put another way, despite a
whole-hearted commitment to an It-from-Bit ontology, we are firmly
aligned with the shut-up-and-calculate camp as the best way to obtain
results either from the physical perspective or as a quality assurance
measure of our fledgling theory of dynamics.

In detail, we present a reflective process calculus. Then we develop
intuitive correspondences between the notions available in this
calculus and the usual physical notions supporting quantum mechanical
calculations. Thus, 

\begin{table}[htp]
  \center{
    \fbox{
      \begin{tabular}{c|c}
        quantum mechanics & process calculus \\
        \hline
        scalar & name \\
        state vector & process \\
        dual & contextual duals \\
        matrix & formal sums of process-context-dual pairs \\
        orthogonality & process annihilation \\
        inner product & execution-formula + quoting
      \end{tabular}
    }
  }
  \caption{QM - process calculi correspondences}
\end{table}

Then we tighten up these intuitions to operational definitions. We
employ the Dirac notation as the best proxy we can find for an
abstract syntax of the quantum mechanical notions. The definitions we
develop put us in contact with equational constraints coming from the
theory that we demonstrate the definitions and calculations satisfy.

This puts us in a position to shut up and calculate for the
Stern-Gerlach experimental set up, showing how these predictive
calculations become calculations on processes in our theory of a
reflective process calculus.

Penultimately, we demonstrate that the notion of metric coming from
the inner product coincides with the notion of metric available from
the theory of bisimulation. This demonstration gives us the right to
think of space as arising from behavior. Finally, we consider where we
might go from the new vantage point we have obtained.

% section introduction (end) 
 
% section introduction (end)

% \documentclass[12pt]{llncs}
%\documentclass{jktr}

\usepackage[pdftex]{hyperref}                   
\usepackage {listings}
\usepackage {mathpartir}
\usepackage{bcprules}
%\usepackage{listings}
                       
\usepackage{graphicx} 
%\usepackage[margins=2.5cm,nohead,nofoot]{geometry}
%\usepackage{geometry}
\usepackage{amsfonts}
\usepackage{amstext}
\usepackage{latexsym}
\usepackage{amssymb}
\usepackage{color}


%\include{myPreamble}
\documentclass[12pt]{llncs}
%\documentclass{jktr}

\usepackage[pdftex]{hyperref}                   
\usepackage {listings}
\usepackage {mathpartir}
\usepackage{bcprules}
%\usepackage{listings}
                       
\usepackage{graphicx} 
%\usepackage[margins=2.5cm,nohead,nofoot]{geometry}
%\usepackage{geometry}
\usepackage{amsfonts}
\usepackage{amstext}
\usepackage{latexsym}
\usepackage{amssymb}
\usepackage{color}


%\include{myPreamble}
\include{qm2pi.local} 

%\ifpdf
%\usepackage[pdftex]{graphicx}
%\else
%\usepackage{graphicx}
%\fi

 % \ifpdf
%  \usepackage{pdfsync}
%  \if


%\title{Brief Article}
%\author{David F. Snyder}
%\author{L.G. Meredith}

%\address{Dept. of Math., Texas State University--San Marcos, San Marcos, TX 78666}
       
\pagestyle{empty}


\begin{document}

\lstset{language=[Objective]Caml,frame=shadowbox}

\input{qm2pi.front}

% section front matter (end)

\input{qm2pi.intro} 
 
% section introduction (end)

% \input{qm2pi.knotations} 

% section notation (end)

\input{qm2pi.process.calculi} 

% section concurrent_process_calculi_and_spatial_logics_ (end)
    
%\input{qm2pi.knots2pi} 

%\input{qm2pi.trefoil} 

%\input{qm2pi.mainthm} 

% subsection basic_interpretation (end)

%\input{qm2pi.rho.presentation} 
\subsection{The syntax and semantics of the notation system}\label{sub:the_syntax_and_semantics_of_the_notation_system} % (fold)

We now summarize a technical presentation of the calculus that
embodies our theory of dynamics. The typical presentation of such a
calculus follows the style of giving generators and relations on
them. The grammar, below, describing term constructors, freely
generates the set of processes, $\Proc$. This set is then quotiented
by a relation known as structural congruence and it is over this set
that the notion of dynamics is expressed. This presentation is
essentially that of \cite{MeredithR05} with the addition of
polyadicity and summation. For readability we have relegated some of
the technical subtleties to an appendix.

\subsubsection{Process grammar}\label{subsub:process_grammar}

\begin{mathpar}
  \inferrule* [lab=synchronization] {} {{M} \bc \pzero \;|\; x?F \;|\; x!C }
  \and
  \inferrule* [lab=abstraction] {} {{F} \bc (x)P}
  \and
  \inferrule* [lab=concretion] {} {{C} \bc \langle Q \rangle}
  \and
  \inferrule* [lab=process] {} {{P,Q} \bc M \;| \;P|Q \;|\; @{x}}
  \and
  \inferrule* [lab=name] {} {{x} \bc \quotep{P}}
\end{mathpar} 

Note that $\vec{x}$ (resp. $\vec{P}$) denotes a vector of names
(resp. processes) of length $|\vec{x}|$ (resp. $|\vec{P}|$). We adopt
the following useful abbreviations.

\begin{mathpar}
   x?(\vec{y}).P := x.(\vec{y})P \and  x\clift{\vec{P}} := x.\clift{\vec{P}}
   \and x!(y) := \lift{x}{\dropn{y}}
   \and \Pi_{i=0}^{n-1}P_i := P_0 | \ldots | P_{n-1}
\end{mathpar}

\subsubsection{Structural congruence}

\paragraph{Free and bound names and alpha-equivalence.} At the
core of structural equivalence is alpha-equivalence which identifies
process that are the same up to a change of variable. Formally, we
recognize the distinction between free and bound names. The free names
of a process, $\freenames{P}$, may be calculated recursively as
follows:

\begin{mathpar}
\freenames{\pzero} := \emptyset
  \and \\
  \freenames{x?(y).P} := \{ x \} \cup (\freenames{P} \setminus \{ y \})
  \and 
  \freenames{x!\langle P \rangle} := \{ x \} \cup \{ P \} 
  \and \\
  \freenames{P|Q} := \freenames{P} \cup \freenames{Q}
  \and \\
  \freenames{@{x}} := \{ x \}
\end{mathpar}

$\pi$
$\quotep{\pi}$

$\freenames{-} : \pi \to \mathcal{P}(\quotep{\pi})$

\begin{eqnarray*}
  \freenames{\pzero} & := & \emptyset \\
  \freenames{x?(y).P} & := & \{ x \} \cup (\freenames{P} \setminus \{ y \}) \\
  \freenames{x!\langle P \rangle} & := & \{ x \} \cup \{ P \} \\
  \freenames{P|Q} & := & \freenames{P} \cup \freenames{Q} \\
  \freenames{\dropn{x}} & := & \{ x \}
\end{eqnarray*}

The bound names of a process, $\boundnames{P}$, are those names occurring in $P$
that are not free. For example, in $x?(y).0$, the name $x$ is free, while $y$ is bound.

\begin{mathpar}
  \inferrule* [lab=monoidal-laws] {} { P|Q \equiv Q|P \and P|0 \equiv P \and P|(Q|R) \equiv (P|Q)|R }
\end{mathpar}

\begin{mathpar}
  \inferrule* [lab=alpha-equivalence] {} { (x)P \equiv (y)P\{y/x\} \and y \not\in \freenames{P} }
\end{mathpar}

\begin{definition}
Then two processes, $P,Q$, are alpha-equivalent if $P = Q\{\vec{y}/\vec{x}\}$ for
some $\vec{x} \in \boundnames{Q},\vec{y} \in \boundnames{P}$, where $Q\{\vec{y}/\vec{x}\}$
denotes the capture-avoiding substitution of $\vec{y}$ for $\vec{x}$ in $Q$.
\end{definition}

\begin{definition}
  The {\em structural congruence} \cite{SangiorgiWalker} , $\equiv$,
  between processes is the least congruence containing
  alpha-equivalence, satisfying the abelian monoid laws
  (associativity, commutativity and $\pzero$ as identity) for parallel
  composition $|$ and for summation $+$.
\end{definition}

\subsection{Name equivalence}

We take name equivalence, written $\nameeq$, to be the smallest
equivalence relation generated by the following rules.

\begin{mathpar}
\inferrule*[lab=Quote-drop]
{ }
{ \quotep{@{x}} \nameeq x }

\inferrule*[lab=Struct-equiv]
{ P \scong Q }
{ \quotep{P} \nameeq \quotep{Q} }
\end{mathpar}

The astute reader will have noticed that the mutual recursion of names
and processes imposes a mutual recursion on alpha-equivalence and
structural equivalence via name-equivalence. Fortunately, all of this
works out pleasantly and we may calculate in the natural way, free of
concern. The reader interested in the details is referred to the
appendix \ref{appendix:rho_details}.

\subsection{Substitution}

We use $\Proc$ for the set of processes, $\QProc$ for the set of
names, and $\id{\{}\vec{y} / \vec{x} \id{\}}$ to denote partial maps,
$s : \QProc \rightarrow \QProc$. A map, $s$ lifts, uniquely, to a map
on process terms, $\widehat{s} : \Proc \rightarrow \Proc$ by the
following equations.

\begin{mathpar}
  (0) \psubstp{Q}{P} := 0 \\
  (R \juxtap S) \psubstp{Q}{P}
  :=    
  (R)\psubstp{Q}{P} \juxtap (S) \psubstp{Q}{P} \\
  (x?(y).R) \psubstp{Q}{P}    
  :=    
  (x)\substp{Q}{P} (z)\concat( (R \psubstn{z}{y}) \psubstp{Q}{P} ) \\
  (\lift{x}{R}) \psubstp{Q}{P}  
  :=
  \lift{(x)\substp{Q}{P}}{ R \psubstp{Q}{P} } \\
%   (\dropn{x})  \psubstp{Q}{P}       
%   := 
%   \left\{ 
%     \begin{array}{ccc} 
%       \dropn{\quotep{Q}} & & x \nameeq \quotep{P} \\
%       \dropn{x} & & otherwise \\
%     \end{array}
%   \right. 
  (\dropn{x})  \psubstp{Q}{P}       
  := 
  \left\{ 
    \begin{array}{ccc} 
      Q & & x \nameeq \quotep{P} \\
      \dropn{x} & & otherwise \\
    \end{array}
  \right.
\end{mathpar}
 

where

\begin{eqnarray}
  (x)\id{\{} \lpquote Q \rpquote / \lpquote P \rpquote \id{\}}            = 
  \left\{ 
    \begin{array}{ccc}
      \lpquote Q \rpquote & & x \nameeq \lpquote P \rpquote \\
      x & & otherwise \\
    \end{array}
  \right. \nonumber
\end{eqnarray}

and $z$ is chosen distinct from $\quotep{P}$, $\quotep{Q}$, the free
names in $Q$, and all the names in $R$. Our $\alpha$-equivalence will
be built in the standard way from this substitution.

\begin{remark}\label{rem:no_self_referential_names}
  One consequence of these definitions is that $\forall P. \quotep{P}
  \not\in \freenames{P}$.
\end{remark}

\subsection{ Dynamic quote: an example }

Anticipating something of what's to come, consider applying the
substitution, $\widehat{\id{\{}u / z \id{\}}}$, to the following pair
of processes, $\lift{w}{y!(z)}$ and $w[ \lpquote y!(z) \rpquote ]$.

\begin{eqnarray}
	\lift{w}{y!(z)}\widehat{\id{\{}u / z \id{\}}}
		& = &
		\lift{w}{y!(u)} \nonumber\\
	w[ \lpquote y!(z) \rpquote ] \widehat{ \id{\{}u / z \id{\}} }
		& = &
		w[ \lpquote y!(z) \rpquote ] \nonumber
\end{eqnarray}

Because the body of the process between quotes is impervious to
substitution, we get radically different answers. In fact, by
examining the first process in an input context,
e.g. $x?(z).\lift{w}{y!(z)}$, we see that the process under the lift
operator may be shaped by prefixed inputs binding a name inside it. In
this sense, the lift operator will be seen as a way to dynamically
construct processes before reifying them as names.

Finally equipped with these standard features we can present the
dynamics of the calculus.

\subsubsection{Operational semantics} 

Finally, we introduce the computational dynamics. What marks these
algebras as distinct from other more traditionally studied algebraic
structures, e.g. vector spaces or polynomial rings, is the manner in
which dynamics is captured. In traditional structures, dynamics is typically
expressed through morphisms between such structures, as in linear maps
between vector spaces or morphisms between rings. In algebras
associated with the semantics of computation, the dynamics is
expressed as part of the algebraic structure itself, through a
reduction reduction relation typically denoted by $\red$. Below, we
give a recursive presentation of this relation for the calculus used
in the encoding.

$\red \subseteq \pi \times \pi$
$\red : \pi \to \mathcal{P}(\pi)$

\begin{mathpar}
  \inferrule* [lab=Comm] { \textsf{match}( x_{src}, x_{trgt} ) } { x_{trgt}?(y)P \; | \; x_{src}!\langle {Q} \rangle \red P\{\quotep{Q}/y}\} }
  \and \\
  \inferrule* [lab=Par] {{P} \red {P}'} {{{P} | {Q}} \red {{P}' | {Q}}}
  \and
  \inferrule* [lab=Equiv]{{{P} \scong {P}'} \andalso {{P}' \red {Q}'} \andalso {{Q}' \scong {Q}}}{{P} \red {Q}}
\end{mathpar}

\begin{eqnarray*}
  match_{\equiv} (\quotep{P},\quotep{Q}) & := & P \equiv Q \\
  match_{\dagger}(\quotep{P},\quotep{Q}) & := & \forall R. P|Q \red^{*} R => R \red^{*} 0 \\
  match_{K}(\quotep{P},\quotep{Q}) & := & K \mbox{ for some context } K
\end{eqnarray*}

$u?(x)P | u!\langle Q \rangle \red P\{\quotep{Q}/x\}$

%We write $\wred$ for $\red^*$, and $P\red$ if $\exists Q $ such that $ P \red Q$.
We write $P\red$ if $\exists Q $ such that $ P \red Q$ and $P\not\red$, otherwise.

\section{Replication}

As mentioned before, it is known that replication (and hence
recursion) can be implemented in a higher-order process algebra
\cite{SangiorgiWalker}. As our first example of calculation with the
machinery thus far presented we give the construction explicitly in
the {\rhoc}.

\begin{eqnarray}
	D_{x} & := & \prefix{x}{y}{(\binpar{\outputp{x}{y}}{@{y}})} \nonumber\\
	\bangp_{x}{P} & := & \binpar{{x}!\langle{\binpar{D_{x}}{P}}\rangle}{D_{x}} \nonumber
\end{eqnarray}

\begin{eqnarray}
	\bangp_{x}{P} & & \nonumber\\
	=
	& {x}!\langle{(\prefix{x}{y}{(\outputp{x}{y} | @{y})) | P}}\rangle 
	      | \prefix{x}{y}{(\outputp{x}{y} | @{y})} & \nonumber\\
	\red
	& (\outputp{x}{y} | @{y})\substn{\quotep{(\prefix{x}{y}{(@{y} | \outputp{x}{y})) | P}}}{y} & \nonumber\\
	=
	& \outputp{x}{\quotep{(\prefix{x}{y}{(\outputp{x}{y} | @{y})) | P}}}
	  | {(\prefix{x}{y}{(\outputp{x}{y} | @{y})) | P}} & \nonumber\\
	\red
	& \ldots & \nonumber\\
	\red^*
	& P | P | \ldots & \nonumber
\end{eqnarray}

Of course, this encoding, as an implementation, runs away, unfolding
$\bangp{P}$ eagerly. A lazier and more implementable replication
operator, restricted to input-guarded processes, may be obtained as follows.

\begin{eqnarray}
\bangp{\prefix{u}{v}{P}} 
	:= 
	\binpar{\lift{x}{\prefix{u}{v}{(\binpar{D(x)}{P})}}}{D(x)} \nonumber
\end{eqnarray}

\begin{remark}
  Note that the lazier definition still does not deal with summation
  or mixed summation (i.e. sums over input and output). The reader is
  invited to construct definitions of replication that deal with these
  features. 

  Further, the definitions are parameterized in a name, $x$. Can you,
  gentle reader, make a definition that eliminates this parameter and
  guarantees no accidental interaction between the replication
  machinery and the process being replicated -- i.e. no accidental
  sharing of names used by the process to get its work done and the
  name(s) used by the replication to effect copying. This latter
  revision of the definition of replication is crucial to obtaining
  the expected identity $!!P \sim !P$.
\end{remark}

\begin{remark}\label{rem:paradoxical_combinator}
  The reader familiar with the lambda calculus will have noticed the
  similarity between $D$ and the paradoxical combinator.

  [Ed. note: the existence of this seems to suggest we have to be more
  restrictive on the set of processes and names we admit if we are to
  support no-cloning.]
\end{remark}

\subsubsection{Bisimulation}

The computational dynamics gives rise to another kind of equivalence,
the equivalence of computational behavior. As previously mentioned
this is typically captured \emph{via} some form of bisimulation.

% The notion we use in this paper is weak barbed bisimulation
% \cite{milner91polyadicpi}.

The notion we use in this paper is derived from weak barbed
bisimulation \cite{milner91polyadicpi}. 

\begin{definition}
An \emph{observation relation}, $\downarrow_{\mathcal N}$, over a set
of names, $\mathcal N$, is the smallest relation satisfying the rules
below.

\infrule[Out-barb]{y \in {\mathcal N}, \; x \nameeq y}
		  {\outputp{x}{v} \downarrow_{\mathcal N} x}
\infrule[Par-barb]{\mbox{$P\downarrow_{\mathcal N} x$ or $Q\downarrow_{\mathcal N} x$}}
		  {\binpar{P}{Q} \downarrow_{\mathcal N} x}

We write $P \Downarrow_{\mathcal N} x$ if there is $Q$ such that 
$P \wred Q$ and $Q \downarrow_{\mathcal N} x$.
\end{definition}

\begin{definition}
%\label{def.bbisim}
An  ${\mathcal N}$-\emph{barbed bisimulation} over a set of names, ${\mathcal N}$, is a symmetric binary relation 
${\mathcal S}_{\mathcal N}$ between agents such that $P\rel{S}_{\mathcal N}Q$ implies:
\begin{enumerate}
\item If $P \red P'$ then $Q \wred Q'$ and $P'\rel{S}_{\mathcal N} Q'$.
\item If $P\downarrow_{\mathcal N} x$, then $Q\Downarrow_{\mathcal N} x$.
\end{enumerate}
$P$ is ${\mathcal N}$-barbed bisimilar to $Q$, written
$P \wbbisim_{\mathcal N} Q$, if $P \rel{S}_{\mathcal N} Q$ for some ${\mathcal N}$-barbed bisimulation ${\mathcal S}_{\mathcal N}$.
\end{definition}

$\mathcal{R} \subseteq \pi \times \pi$

$P \mathcal{R} Q => \forall P'. P \red P' \Rightarrow \exists Q'. Q \red Q', P' \mathcal{R} Q'$

$P \vdash x \Rightarrow Q \vdash x$

\begin{mathpar}
  \inferrule*[lab=Out-barb]{x \nameeq y}{{y}!\langle{Q}\rangle \vdash x}
  \and
  \inferrule*[lab=Par-barb]{\mbox{$P\vdash x$ or $Q\vdash x$}}{\binpar{P}{Q} \vdash x}
\end{mathpar}

\subsubsection{Contexts}

One of the principle advantages of computational calculi like the
$\pi$-calculus is a well-defined notion of context,
contextual-equivalence and a correlation between
contextual-equivalence and notions of bisimulation. The notion of
context allows the decomposition of a process into (sub-)process and
its syntactic environment, its context. Thus, a context may be
thought of as a process with a ``hole'' (written $\Box$) in it. The
application of a context $M$ to a process $P$, written $M[P]$, is
tantamount to filling the hole in $M$ with $P$. In this paper we do
not need the full weight of this theory, but do make use of the notion
of context in the proof the main theorem. 

\begin{mathpar}
  \inferrule* [lab=summation] {} {{M_{M},M_{N}} \bc \Box \;|\; x.M_{A} \;|\; M_{M}+M_{N}}
  \and
  \inferrule* [lab=agent] {} {{M_{A}} \bc (\vec{x})M_{P} \;| \; \clift{P_0,\ldots,M_{P},\ldots,P_N}}
  \and \\
  \inferrule* [lab=process] {} {{M_{P}} \bc M_{N} \;| \;P|M_{P} }
\end{mathpar} 

\begin{mathpar}
  \inferrule* [lab=sychronization] {} {M_{N} \bc \Box \;|\; x?M_{F} \;|\; x!M_{C}}
  \and
  \inferrule* [lab=abstraction] {} {{M_{F}} \bc (x)M_{P} }
  \and
  \inferrule* [lab=concretion] {} {{M_{C}} \bc \langle M_{P} \rangle }
  \and \\
  \inferrule* [lab=process] {} {{M_{P}} \bc M_{N} \;| \;P|M_{P} }
\end{mathpar}

\begin{definition}[contextual application] Given a context $M$, and
  process $P$, we define the \emph{contextual application}, $M[P] :=
  M\{P/\Box\}$. That is, the contextual application of M to P is the
  substitution of $P$ for $\Box$ in $M$.
\end{definition}

$\meaningof{-} : L \to \mathcal{P}(\pi)$

\begin{mathpar}
  \inferrule* [lab=collection] {} {\meaningof{true} = \pi, \and \meaningof{~E} = \pi \setminus \meaningof{E}, \and \meaningof{E_{1} \& E_{2}} = \meaningof{E_{1}} \cap \meaningof{E_{2}}}
\end{mathpar}

\begin{mathpar}
  \inferrule* [lab=structure] {} {\meaningof{0} = \{ P \in \pi | P \equiv 0 \}, \and \\ \meaningof{E_1 | E_2} = \{ P \in \pi | P \equiv P_{1} | P_{2}, P_{1} \in \meaningof{E_{1}}, P_{2} \in \meaningof{E_2}\} }
\end{mathpar}

\begin{mathpar}
 \inferrule* [lab=behavior] {} {\meaningof{\langle a?b \rangle E} = \{ P \in \pi | P \equiv Q | u?(y)P', \\ \and \\\\ \and \\ \;\;\; u \in \meaningof{a}, \forall z.P'\{z/y\} \in \meaningof{E\{z/b\}}\}, \and \\ \meaningof{a!E} = \{ P \in \pi | P \equiv Q | x!\langle P' \rangle, x \in \meaningof{a} P' \in \meaningof{E}\} }
\end{mathpar}

\begin{mathpar}
 \inferrule* [lab=nominal] {} {\meaningof{\quotep{E}} = \{ \quotep{P} \in \quotep{\pi} | P \in \meaningof{E} \}, \and \meaningof{\quotep{P}} = \{ \quotep{Q} \in \quotep{\pi} | P \equiv Q \} \and \\ \meaningof{@\quotep{E}} = \{ P \in \pi | P \equiv @x, x \in \meaningof{E} \}}
\end{mathpar}

\begin{eqnarray*}
  \\
  \meaningof{-} : TS \to ST
\end{eqnarray*}

\begin{eqnarray*}
  \\
  L : TS \to ST
\end{eqnarray*}

\begin{eqnarray*}
  \\
  P \models E \iff P \in \meaningof{E}
\end{eqnarray*}

\begin{eqnarray*}
  P \approx_{L} Q \iff \forall E \in L. P \models E \iff Q \models E
\end{eqnarray*}

\begin{eqnarray*}
  P \approx_{K} Q
\end{eqnarray*}

\begin{eqnarray*}
  P \approx Q
\end{eqnarray*}

$\approx_{K} = \approx = \approx_{L}$

\subsubsection{Contextual duality}

Note that contexts extend the quotation operation to a family of
operations from processes to names. Given a context, $M$, we can
define a \emph{nominal context}, $\quotep{M}$ by $\quotep{M}[P] :=
\quotep{M[P]}$. To foreshadow what is to come we observe that these
operations enjoy a duality with processes very much like the duality
between vectors and maps from vectors to scalars.

Further, because the calculus is essentially higher-order, we have a
correspondence between contexts and processes. More specifically,
given a name $x$ and a context $M$ we can construct $M^{*}_{x}$ such
that 

\begin{mathpar}
  M^{*}_{x} | \lift{x}{P} \red M[P]
\end{mathpar}

namely,

\begin{mathpar}
  M^{*}_{x} := x?(u).M[\dropn{u}]
\end{mathpar}

The dependence of $M^{*}_{x}$ on a name makes it an abstraction, 

\begin{mathpar}
  M^{*} := (x)x?(u).M[\dropn{u}]
\end{mathpar}

\subsection{Additional notation}

It will sometimes be convenient to denote the process a name
quotes. We already have the notation $x = \quotep{P}$, but it will be
convenient to introduce an alternate notation, $\procn{x}$, when we
want to emphasize the connection to the use of the name. Note that, by
virtue of name equivalence, $\quotep{\procn{x}} \nameeq x$; so, the
notation is consistent with previous definitions.

Further, because names have structure it is possible to effect
substitutions on the basis of that structure. This means we need to
upgrade our notation for substitutions, which we accomplish by
adapting comprehension notation. Thus,

\begin{mathpar}
  P\{ y / x : x \in S \}
\end{mathpar}

is interpreted to mean the process derived from P by replacing (in a
capture-avoiding manner) each occurrence of $x$ in $S$ by $y$. For example,

\begin{mathpar}
  P\{ \quotep{\procn{x}|\procn{x}} / x : x \in \freenames{P} \}
\end{mathpar}

will replace each (occurrence) of a free name $x$ in $P$ by
$\quotep{\procn{x}|\procn{x}}$.

Also, we will avail ourselves of the notation $x^{L}$ and $x^{R}$ to
denote injections of a name into disjoint copies of the name
space. There are numerous ways to accomplish this. One example can be
found in \cite{MeredithR05}. This notation overloads to vectors of
names: $\vec{x}^{\pi} := (x_{i}^{\pi} \; : \; 0 \leq i < |\vec{x}| )$ where $\pi \in \{L,R\}$.

We also use $P^{\Box} := P|\Box$.

In \cite{MeredithR05} an interpretation of the new operator is
given. It turns out that there are several possible interpretations
all enjoying the requisite algebraic properties of the operator (see
\cite{milner91polyadicpi}). We will therefore make liberal use of
$(\nu\; \vec{x})P$.

% subsection the_syntax_and_semantics_of_the_notation_system (end)   

\input{qm2pi.qmops} 

\input{qm2pi.sterngerlach} 

\input{qm2pi.metric} 

% section concurrent_process_calculi (end)

%\input{qm2pi.proofsketch}

% section proof sketch (end)

%\input{qm2pi.slviaknots} 

% section spatial logic via knots (end)

\input{qm2pi.conclusion}

% section conclusion (end)

%\input{qm2pi.dtcodes} 

% section wiring algorithm (end)

\input{qm2pi.ack} 

% section acknowledgments (end)

\newpage


\bibliographystyle{plain}   
\bibliography{../../biblios/main.bib}

\input{qm2pi.rhodetails}

\end{document}

 

%\ifpdf
%\usepackage[pdftex]{graphicx}
%\else
%\usepackage{graphicx}
%\fi

 % \ifpdf
%  \usepackage{pdfsync}
%  \if


%\title{Brief Article}
%\author{David F. Snyder}
%\author{L.G. Meredith}

%\address{Dept. of Math., Texas State University--San Marcos, San Marcos, TX 78666}
       
\pagestyle{empty}


\begin{document}

\lstset{language=[Objective]Caml,frame=shadowbox}

\documentclass[12pt]{llncs}
%\documentclass{jktr}

\usepackage[pdftex]{hyperref}                   
\usepackage {listings}
\usepackage {mathpartir}
\usepackage{bcprules}
%\usepackage{listings}
                       
\usepackage{graphicx} 
%\usepackage[margins=2.5cm,nohead,nofoot]{geometry}
%\usepackage{geometry}
\usepackage{amsfonts}
\usepackage{amstext}
\usepackage{latexsym}
\usepackage{amssymb}
\usepackage{color}


%\include{myPreamble}
\include{qm2pi.local} 

%\ifpdf
%\usepackage[pdftex]{graphicx}
%\else
%\usepackage{graphicx}
%\fi

 % \ifpdf
%  \usepackage{pdfsync}
%  \if


%\title{Brief Article}
%\author{David F. Snyder}
%\author{L.G. Meredith}

%\address{Dept. of Math., Texas State University--San Marcos, San Marcos, TX 78666}
       
\pagestyle{empty}


\begin{document}

\lstset{language=[Objective]Caml,frame=shadowbox}

\input{qm2pi.front}

% section front matter (end)

\input{qm2pi.intro} 
 
% section introduction (end)

% \input{qm2pi.knotations} 

% section notation (end)

\input{qm2pi.process.calculi} 

% section concurrent_process_calculi_and_spatial_logics_ (end)
    
%\input{qm2pi.knots2pi} 

%\input{qm2pi.trefoil} 

%\input{qm2pi.mainthm} 

% subsection basic_interpretation (end)

%\input{qm2pi.rho.presentation} 
\subsection{The syntax and semantics of the notation system}\label{sub:the_syntax_and_semantics_of_the_notation_system} % (fold)

We now summarize a technical presentation of the calculus that
embodies our theory of dynamics. The typical presentation of such a
calculus follows the style of giving generators and relations on
them. The grammar, below, describing term constructors, freely
generates the set of processes, $\Proc$. This set is then quotiented
by a relation known as structural congruence and it is over this set
that the notion of dynamics is expressed. This presentation is
essentially that of \cite{MeredithR05} with the addition of
polyadicity and summation. For readability we have relegated some of
the technical subtleties to an appendix.

\subsubsection{Process grammar}\label{subsub:process_grammar}

\begin{mathpar}
  \inferrule* [lab=synchronization] {} {{M} \bc \pzero \;|\; x?F \;|\; x!C }
  \and
  \inferrule* [lab=abstraction] {} {{F} \bc (x)P}
  \and
  \inferrule* [lab=concretion] {} {{C} \bc \langle Q \rangle}
  \and
  \inferrule* [lab=process] {} {{P,Q} \bc M \;| \;P|Q \;|\; @{x}}
  \and
  \inferrule* [lab=name] {} {{x} \bc \quotep{P}}
\end{mathpar} 

Note that $\vec{x}$ (resp. $\vec{P}$) denotes a vector of names
(resp. processes) of length $|\vec{x}|$ (resp. $|\vec{P}|$). We adopt
the following useful abbreviations.

\begin{mathpar}
   x?(\vec{y}).P := x.(\vec{y})P \and  x\clift{\vec{P}} := x.\clift{\vec{P}}
   \and x!(y) := \lift{x}{\dropn{y}}
   \and \Pi_{i=0}^{n-1}P_i := P_0 | \ldots | P_{n-1}
\end{mathpar}

\subsubsection{Structural congruence}

\paragraph{Free and bound names and alpha-equivalence.} At the
core of structural equivalence is alpha-equivalence which identifies
process that are the same up to a change of variable. Formally, we
recognize the distinction between free and bound names. The free names
of a process, $\freenames{P}$, may be calculated recursively as
follows:

\begin{mathpar}
\freenames{\pzero} := \emptyset
  \and \\
  \freenames{x?(y).P} := \{ x \} \cup (\freenames{P} \setminus \{ y \})
  \and 
  \freenames{x!\langle P \rangle} := \{ x \} \cup \{ P \} 
  \and \\
  \freenames{P|Q} := \freenames{P} \cup \freenames{Q}
  \and \\
  \freenames{@{x}} := \{ x \}
\end{mathpar}

$\pi$
$\quotep{\pi}$

$\freenames{-} : \pi \to \mathcal{P}(\quotep{\pi})$

\begin{eqnarray*}
  \freenames{\pzero} & := & \emptyset \\
  \freenames{x?(y).P} & := & \{ x \} \cup (\freenames{P} \setminus \{ y \}) \\
  \freenames{x!\langle P \rangle} & := & \{ x \} \cup \{ P \} \\
  \freenames{P|Q} & := & \freenames{P} \cup \freenames{Q} \\
  \freenames{\dropn{x}} & := & \{ x \}
\end{eqnarray*}

The bound names of a process, $\boundnames{P}$, are those names occurring in $P$
that are not free. For example, in $x?(y).0$, the name $x$ is free, while $y$ is bound.

\begin{mathpar}
  \inferrule* [lab=monoidal-laws] {} { P|Q \equiv Q|P \and P|0 \equiv P \and P|(Q|R) \equiv (P|Q)|R }
\end{mathpar}

\begin{mathpar}
  \inferrule* [lab=alpha-equivalence] {} { (x)P \equiv (y)P\{y/x\} \and y \not\in \freenames{P} }
\end{mathpar}

\begin{definition}
Then two processes, $P,Q$, are alpha-equivalent if $P = Q\{\vec{y}/\vec{x}\}$ for
some $\vec{x} \in \boundnames{Q},\vec{y} \in \boundnames{P}$, where $Q\{\vec{y}/\vec{x}\}$
denotes the capture-avoiding substitution of $\vec{y}$ for $\vec{x}$ in $Q$.
\end{definition}

\begin{definition}
  The {\em structural congruence} \cite{SangiorgiWalker} , $\equiv$,
  between processes is the least congruence containing
  alpha-equivalence, satisfying the abelian monoid laws
  (associativity, commutativity and $\pzero$ as identity) for parallel
  composition $|$ and for summation $+$.
\end{definition}

\subsection{Name equivalence}

We take name equivalence, written $\nameeq$, to be the smallest
equivalence relation generated by the following rules.

\begin{mathpar}
\inferrule*[lab=Quote-drop]
{ }
{ \quotep{@{x}} \nameeq x }

\inferrule*[lab=Struct-equiv]
{ P \scong Q }
{ \quotep{P} \nameeq \quotep{Q} }
\end{mathpar}

The astute reader will have noticed that the mutual recursion of names
and processes imposes a mutual recursion on alpha-equivalence and
structural equivalence via name-equivalence. Fortunately, all of this
works out pleasantly and we may calculate in the natural way, free of
concern. The reader interested in the details is referred to the
appendix \ref{appendix:rho_details}.

\subsection{Substitution}

We use $\Proc$ for the set of processes, $\QProc$ for the set of
names, and $\id{\{}\vec{y} / \vec{x} \id{\}}$ to denote partial maps,
$s : \QProc \rightarrow \QProc$. A map, $s$ lifts, uniquely, to a map
on process terms, $\widehat{s} : \Proc \rightarrow \Proc$ by the
following equations.

\begin{mathpar}
  (0) \psubstp{Q}{P} := 0 \\
  (R \juxtap S) \psubstp{Q}{P}
  :=    
  (R)\psubstp{Q}{P} \juxtap (S) \psubstp{Q}{P} \\
  (x?(y).R) \psubstp{Q}{P}    
  :=    
  (x)\substp{Q}{P} (z)\concat( (R \psubstn{z}{y}) \psubstp{Q}{P} ) \\
  (\lift{x}{R}) \psubstp{Q}{P}  
  :=
  \lift{(x)\substp{Q}{P}}{ R \psubstp{Q}{P} } \\
%   (\dropn{x})  \psubstp{Q}{P}       
%   := 
%   \left\{ 
%     \begin{array}{ccc} 
%       \dropn{\quotep{Q}} & & x \nameeq \quotep{P} \\
%       \dropn{x} & & otherwise \\
%     \end{array}
%   \right. 
  (\dropn{x})  \psubstp{Q}{P}       
  := 
  \left\{ 
    \begin{array}{ccc} 
      Q & & x \nameeq \quotep{P} \\
      \dropn{x} & & otherwise \\
    \end{array}
  \right.
\end{mathpar}
 

where

\begin{eqnarray}
  (x)\id{\{} \lpquote Q \rpquote / \lpquote P \rpquote \id{\}}            = 
  \left\{ 
    \begin{array}{ccc}
      \lpquote Q \rpquote & & x \nameeq \lpquote P \rpquote \\
      x & & otherwise \\
    \end{array}
  \right. \nonumber
\end{eqnarray}

and $z$ is chosen distinct from $\quotep{P}$, $\quotep{Q}$, the free
names in $Q$, and all the names in $R$. Our $\alpha$-equivalence will
be built in the standard way from this substitution.

\begin{remark}\label{rem:no_self_referential_names}
  One consequence of these definitions is that $\forall P. \quotep{P}
  \not\in \freenames{P}$.
\end{remark}

\subsection{ Dynamic quote: an example }

Anticipating something of what's to come, consider applying the
substitution, $\widehat{\id{\{}u / z \id{\}}}$, to the following pair
of processes, $\lift{w}{y!(z)}$ and $w[ \lpquote y!(z) \rpquote ]$.

\begin{eqnarray}
	\lift{w}{y!(z)}\widehat{\id{\{}u / z \id{\}}}
		& = &
		\lift{w}{y!(u)} \nonumber\\
	w[ \lpquote y!(z) \rpquote ] \widehat{ \id{\{}u / z \id{\}} }
		& = &
		w[ \lpquote y!(z) \rpquote ] \nonumber
\end{eqnarray}

Because the body of the process between quotes is impervious to
substitution, we get radically different answers. In fact, by
examining the first process in an input context,
e.g. $x?(z).\lift{w}{y!(z)}$, we see that the process under the lift
operator may be shaped by prefixed inputs binding a name inside it. In
this sense, the lift operator will be seen as a way to dynamically
construct processes before reifying them as names.

Finally equipped with these standard features we can present the
dynamics of the calculus.

\subsubsection{Operational semantics} 

Finally, we introduce the computational dynamics. What marks these
algebras as distinct from other more traditionally studied algebraic
structures, e.g. vector spaces or polynomial rings, is the manner in
which dynamics is captured. In traditional structures, dynamics is typically
expressed through morphisms between such structures, as in linear maps
between vector spaces or morphisms between rings. In algebras
associated with the semantics of computation, the dynamics is
expressed as part of the algebraic structure itself, through a
reduction reduction relation typically denoted by $\red$. Below, we
give a recursive presentation of this relation for the calculus used
in the encoding.

$\red \subseteq \pi \times \pi$
$\red : \pi \to \mathcal{P}(\pi)$

\begin{mathpar}
  \inferrule* [lab=Comm] { \textsf{match}( x_{src}, x_{trgt} ) } { x_{trgt}?(y)P \; | \; x_{src}!\langle {Q} \rangle \red P\{\quotep{Q}/y}\} }
  \and \\
  \inferrule* [lab=Par] {{P} \red {P}'} {{{P} | {Q}} \red {{P}' | {Q}}}
  \and
  \inferrule* [lab=Equiv]{{{P} \scong {P}'} \andalso {{P}' \red {Q}'} \andalso {{Q}' \scong {Q}}}{{P} \red {Q}}
\end{mathpar}

\begin{eqnarray*}
  match_{\equiv} (\quotep{P},\quotep{Q}) & := & P \equiv Q \\
  match_{\dagger}(\quotep{P},\quotep{Q}) & := & \forall R. P|Q \red^{*} R => R \red^{*} 0 \\
  match_{K}(\quotep{P},\quotep{Q}) & := & K \mbox{ for some context } K
\end{eqnarray*}

$u?(x)P | u!\langle Q \rangle \red P\{\quotep{Q}/x\}$

%We write $\wred$ for $\red^*$, and $P\red$ if $\exists Q $ such that $ P \red Q$.
We write $P\red$ if $\exists Q $ such that $ P \red Q$ and $P\not\red$, otherwise.

\section{Replication}

As mentioned before, it is known that replication (and hence
recursion) can be implemented in a higher-order process algebra
\cite{SangiorgiWalker}. As our first example of calculation with the
machinery thus far presented we give the construction explicitly in
the {\rhoc}.

\begin{eqnarray}
	D_{x} & := & \prefix{x}{y}{(\binpar{\outputp{x}{y}}{@{y}})} \nonumber\\
	\bangp_{x}{P} & := & \binpar{{x}!\langle{\binpar{D_{x}}{P}}\rangle}{D_{x}} \nonumber
\end{eqnarray}

\begin{eqnarray}
	\bangp_{x}{P} & & \nonumber\\
	=
	& {x}!\langle{(\prefix{x}{y}{(\outputp{x}{y} | @{y})) | P}}\rangle 
	      | \prefix{x}{y}{(\outputp{x}{y} | @{y})} & \nonumber\\
	\red
	& (\outputp{x}{y} | @{y})\substn{\quotep{(\prefix{x}{y}{(@{y} | \outputp{x}{y})) | P}}}{y} & \nonumber\\
	=
	& \outputp{x}{\quotep{(\prefix{x}{y}{(\outputp{x}{y} | @{y})) | P}}}
	  | {(\prefix{x}{y}{(\outputp{x}{y} | @{y})) | P}} & \nonumber\\
	\red
	& \ldots & \nonumber\\
	\red^*
	& P | P | \ldots & \nonumber
\end{eqnarray}

Of course, this encoding, as an implementation, runs away, unfolding
$\bangp{P}$ eagerly. A lazier and more implementable replication
operator, restricted to input-guarded processes, may be obtained as follows.

\begin{eqnarray}
\bangp{\prefix{u}{v}{P}} 
	:= 
	\binpar{\lift{x}{\prefix{u}{v}{(\binpar{D(x)}{P})}}}{D(x)} \nonumber
\end{eqnarray}

\begin{remark}
  Note that the lazier definition still does not deal with summation
  or mixed summation (i.e. sums over input and output). The reader is
  invited to construct definitions of replication that deal with these
  features. 

  Further, the definitions are parameterized in a name, $x$. Can you,
  gentle reader, make a definition that eliminates this parameter and
  guarantees no accidental interaction between the replication
  machinery and the process being replicated -- i.e. no accidental
  sharing of names used by the process to get its work done and the
  name(s) used by the replication to effect copying. This latter
  revision of the definition of replication is crucial to obtaining
  the expected identity $!!P \sim !P$.
\end{remark}

\begin{remark}\label{rem:paradoxical_combinator}
  The reader familiar with the lambda calculus will have noticed the
  similarity between $D$ and the paradoxical combinator.

  [Ed. note: the existence of this seems to suggest we have to be more
  restrictive on the set of processes and names we admit if we are to
  support no-cloning.]
\end{remark}

\subsubsection{Bisimulation}

The computational dynamics gives rise to another kind of equivalence,
the equivalence of computational behavior. As previously mentioned
this is typically captured \emph{via} some form of bisimulation.

% The notion we use in this paper is weak barbed bisimulation
% \cite{milner91polyadicpi}.

The notion we use in this paper is derived from weak barbed
bisimulation \cite{milner91polyadicpi}. 

\begin{definition}
An \emph{observation relation}, $\downarrow_{\mathcal N}$, over a set
of names, $\mathcal N$, is the smallest relation satisfying the rules
below.

\infrule[Out-barb]{y \in {\mathcal N}, \; x \nameeq y}
		  {\outputp{x}{v} \downarrow_{\mathcal N} x}
\infrule[Par-barb]{\mbox{$P\downarrow_{\mathcal N} x$ or $Q\downarrow_{\mathcal N} x$}}
		  {\binpar{P}{Q} \downarrow_{\mathcal N} x}

We write $P \Downarrow_{\mathcal N} x$ if there is $Q$ such that 
$P \wred Q$ and $Q \downarrow_{\mathcal N} x$.
\end{definition}

\begin{definition}
%\label{def.bbisim}
An  ${\mathcal N}$-\emph{barbed bisimulation} over a set of names, ${\mathcal N}$, is a symmetric binary relation 
${\mathcal S}_{\mathcal N}$ between agents such that $P\rel{S}_{\mathcal N}Q$ implies:
\begin{enumerate}
\item If $P \red P'$ then $Q \wred Q'$ and $P'\rel{S}_{\mathcal N} Q'$.
\item If $P\downarrow_{\mathcal N} x$, then $Q\Downarrow_{\mathcal N} x$.
\end{enumerate}
$P$ is ${\mathcal N}$-barbed bisimilar to $Q$, written
$P \wbbisim_{\mathcal N} Q$, if $P \rel{S}_{\mathcal N} Q$ for some ${\mathcal N}$-barbed bisimulation ${\mathcal S}_{\mathcal N}$.
\end{definition}

$\mathcal{R} \subseteq \pi \times \pi$

$P \mathcal{R} Q => \forall P'. P \red P' \Rightarrow \exists Q'. Q \red Q', P' \mathcal{R} Q'$

$P \vdash x \Rightarrow Q \vdash x$

\begin{mathpar}
  \inferrule*[lab=Out-barb]{x \nameeq y}{{y}!\langle{Q}\rangle \vdash x}
  \and
  \inferrule*[lab=Par-barb]{\mbox{$P\vdash x$ or $Q\vdash x$}}{\binpar{P}{Q} \vdash x}
\end{mathpar}

\subsubsection{Contexts}

One of the principle advantages of computational calculi like the
$\pi$-calculus is a well-defined notion of context,
contextual-equivalence and a correlation between
contextual-equivalence and notions of bisimulation. The notion of
context allows the decomposition of a process into (sub-)process and
its syntactic environment, its context. Thus, a context may be
thought of as a process with a ``hole'' (written $\Box$) in it. The
application of a context $M$ to a process $P$, written $M[P]$, is
tantamount to filling the hole in $M$ with $P$. In this paper we do
not need the full weight of this theory, but do make use of the notion
of context in the proof the main theorem. 

\begin{mathpar}
  \inferrule* [lab=summation] {} {{M_{M},M_{N}} \bc \Box \;|\; x.M_{A} \;|\; M_{M}+M_{N}}
  \and
  \inferrule* [lab=agent] {} {{M_{A}} \bc (\vec{x})M_{P} \;| \; \clift{P_0,\ldots,M_{P},\ldots,P_N}}
  \and \\
  \inferrule* [lab=process] {} {{M_{P}} \bc M_{N} \;| \;P|M_{P} }
\end{mathpar} 

\begin{mathpar}
  \inferrule* [lab=sychronization] {} {M_{N} \bc \Box \;|\; x?M_{F} \;|\; x!M_{C}}
  \and
  \inferrule* [lab=abstraction] {} {{M_{F}} \bc (x)M_{P} }
  \and
  \inferrule* [lab=concretion] {} {{M_{C}} \bc \langle M_{P} \rangle }
  \and \\
  \inferrule* [lab=process] {} {{M_{P}} \bc M_{N} \;| \;P|M_{P} }
\end{mathpar}

\begin{definition}[contextual application] Given a context $M$, and
  process $P$, we define the \emph{contextual application}, $M[P] :=
  M\{P/\Box\}$. That is, the contextual application of M to P is the
  substitution of $P$ for $\Box$ in $M$.
\end{definition}

$\meaningof{-} : L \to \mathcal{P}(\pi)$

\begin{mathpar}
  \inferrule* [lab=collection] {} {\meaningof{true} = \pi, \and \meaningof{~E} = \pi \setminus \meaningof{E}, \and \meaningof{E_{1} \& E_{2}} = \meaningof{E_{1}} \cap \meaningof{E_{2}}}
\end{mathpar}

\begin{mathpar}
  \inferrule* [lab=structure] {} {\meaningof{0} = \{ P \in \pi | P \equiv 0 \}, \and \\ \meaningof{E_1 | E_2} = \{ P \in \pi | P \equiv P_{1} | P_{2}, P_{1} \in \meaningof{E_{1}}, P_{2} \in \meaningof{E_2}\} }
\end{mathpar}

\begin{mathpar}
 \inferrule* [lab=behavior] {} {\meaningof{\langle a?b \rangle E} = \{ P \in \pi | P \equiv Q | u?(y)P', \\ \and \\\\ \and \\ \;\;\; u \in \meaningof{a}, \forall z.P'\{z/y\} \in \meaningof{E\{z/b\}}\}, \and \\ \meaningof{a!E} = \{ P \in \pi | P \equiv Q | x!\langle P' \rangle, x \in \meaningof{a} P' \in \meaningof{E}\} }
\end{mathpar}

\begin{mathpar}
 \inferrule* [lab=nominal] {} {\meaningof{\quotep{E}} = \{ \quotep{P} \in \quotep{\pi} | P \in \meaningof{E} \}, \and \meaningof{\quotep{P}} = \{ \quotep{Q} \in \quotep{\pi} | P \equiv Q \} \and \\ \meaningof{@\quotep{E}} = \{ P \in \pi | P \equiv @x, x \in \meaningof{E} \}}
\end{mathpar}

\begin{eqnarray*}
  \\
  \meaningof{-} : TS \to ST
\end{eqnarray*}

\begin{eqnarray*}
  \\
  L : TS \to ST
\end{eqnarray*}

\begin{eqnarray*}
  \\
  P \models E \iff P \in \meaningof{E}
\end{eqnarray*}

\begin{eqnarray*}
  P \approx_{L} Q \iff \forall E \in L. P \models E \iff Q \models E
\end{eqnarray*}

\begin{eqnarray*}
  P \approx_{K} Q
\end{eqnarray*}

\begin{eqnarray*}
  P \approx Q
\end{eqnarray*}

$\approx_{K} = \approx = \approx_{L}$

\subsubsection{Contextual duality}

Note that contexts extend the quotation operation to a family of
operations from processes to names. Given a context, $M$, we can
define a \emph{nominal context}, $\quotep{M}$ by $\quotep{M}[P] :=
\quotep{M[P]}$. To foreshadow what is to come we observe that these
operations enjoy a duality with processes very much like the duality
between vectors and maps from vectors to scalars.

Further, because the calculus is essentially higher-order, we have a
correspondence between contexts and processes. More specifically,
given a name $x$ and a context $M$ we can construct $M^{*}_{x}$ such
that 

\begin{mathpar}
  M^{*}_{x} | \lift{x}{P} \red M[P]
\end{mathpar}

namely,

\begin{mathpar}
  M^{*}_{x} := x?(u).M[\dropn{u}]
\end{mathpar}

The dependence of $M^{*}_{x}$ on a name makes it an abstraction, 

\begin{mathpar}
  M^{*} := (x)x?(u).M[\dropn{u}]
\end{mathpar}

\subsection{Additional notation}

It will sometimes be convenient to denote the process a name
quotes. We already have the notation $x = \quotep{P}$, but it will be
convenient to introduce an alternate notation, $\procn{x}$, when we
want to emphasize the connection to the use of the name. Note that, by
virtue of name equivalence, $\quotep{\procn{x}} \nameeq x$; so, the
notation is consistent with previous definitions.

Further, because names have structure it is possible to effect
substitutions on the basis of that structure. This means we need to
upgrade our notation for substitutions, which we accomplish by
adapting comprehension notation. Thus,

\begin{mathpar}
  P\{ y / x : x \in S \}
\end{mathpar}

is interpreted to mean the process derived from P by replacing (in a
capture-avoiding manner) each occurrence of $x$ in $S$ by $y$. For example,

\begin{mathpar}
  P\{ \quotep{\procn{x}|\procn{x}} / x : x \in \freenames{P} \}
\end{mathpar}

will replace each (occurrence) of a free name $x$ in $P$ by
$\quotep{\procn{x}|\procn{x}}$.

Also, we will avail ourselves of the notation $x^{L}$ and $x^{R}$ to
denote injections of a name into disjoint copies of the name
space. There are numerous ways to accomplish this. One example can be
found in \cite{MeredithR05}. This notation overloads to vectors of
names: $\vec{x}^{\pi} := (x_{i}^{\pi} \; : \; 0 \leq i < |\vec{x}| )$ where $\pi \in \{L,R\}$.

We also use $P^{\Box} := P|\Box$.

In \cite{MeredithR05} an interpretation of the new operator is
given. It turns out that there are several possible interpretations
all enjoying the requisite algebraic properties of the operator (see
\cite{milner91polyadicpi}). We will therefore make liberal use of
$(\nu\; \vec{x})P$.

% subsection the_syntax_and_semantics_of_the_notation_system (end)   

\input{qm2pi.qmops} 

\input{qm2pi.sterngerlach} 

\input{qm2pi.metric} 

% section concurrent_process_calculi (end)

%\input{qm2pi.proofsketch}

% section proof sketch (end)

%\input{qm2pi.slviaknots} 

% section spatial logic via knots (end)

\input{qm2pi.conclusion}

% section conclusion (end)

%\input{qm2pi.dtcodes} 

% section wiring algorithm (end)

\input{qm2pi.ack} 

% section acknowledgments (end)

\newpage


\bibliographystyle{plain}   
\bibliography{../../biblios/main.bib}

\input{qm2pi.rhodetails}

\end{document}



% section front matter (end)

\section{Introduction}\label{sec:introduction} % (fold)
In this draft of the material i am going to have to dispense with the
usual writing conventions adopted in papers on these topics. i'm going
to have adopt whatever tone i need at the time i'm writing up the
calculations. Sometimes this may be very conversational; others it may
be the barest mathematical grunts; others still it may be that i have
lifted text from one of my other papers because the exposition of some
point was better said there. i hope that my readers are not unduly put
out by this decision. i'm not doing this to flout convention or be
rebellious. i find these calculations very technically challenging. To
keep everything going technically, something has to give; i have to
let go of some cognitive burden. So, the academic writing style --
with all of its trade-offs in terms of facilitating technical
communication -- is what i'm letting go of. Perhaps subsequent drafts
can be tightened and polished, but for now, i'm going to speak as if
we were sitting together in a coffee shop with a laptop, wifi and a
pad of paper and a pencil.

So, here's what i have to say. We -- you and i, comfortably ensconced
in our coffee shop and well-equipped with our tools -- can realize and
carry out the calculations of quantum mechanics over a very different
formal theory of dynamics, a formal theory of dynamics that
corresponds to a theory of concurrent computation with
\emph{reflection}. It has the advantage that the underlying theory is
already `quantized', but supports analogues all of the continuuous
operations. Strikingly, this underlying theory has recently been
connected with a notion of metric that we can show, by calculating
together, coincides with the metric induced by the inner product.

There are a lot of reasons why you might be interested in seeing
calculations of this form. Here's why i'm interested. For the past
several centuries there has been no competitor to the ``Newtonian''
account of dynamics. As a result the predominant share of accounts of
dynamical systems and situations have had to be formulated in terms of
the Newtonian machinery. i view this as an intellectually dangerous
position to occupy. Everything, despite it's intrinsic shape, turns
into a nail to be hit with this hammer. Recently, however, the theory
of computation has matured to the point where we have candidates for
theories of dynamics that offer very different perspective on
reasoning about dynamical systems and situations. Testing these
candidates against very successful accounts of dynamical situations,
like quantum mechanics, is going to give us some sense of how mature
they are and some measure of the quality of these accounts of
dynamics.

\subsection{Summary of contributions and outline of paper}

So, we're going to develop an interpretation of the operations of
quantum mechanics normally interpreted by Hilbert spaces and
operators. We're going to do this over a theory of computation. Note
that this is very different than the usual quantum computation program
which develops notions of computation over quantum mechanics. Rather,
we are developing a story that aligns with Wheeler's slogan: It from
Bit. To do this we will first provide an account of the theory of
computation at play here. Then we will dive into a calculation-driven
interpretation of the operations of quantum mechanics.

The reason we take this approach is that -- until very recently --
there hasn't been an axiomatic account of quantum mechanics. As a
result there has been no sharp delineation of the mathematical theory
supporting interpretation of the physical theory and the physical
theory, itself. So, ambient features of the maths are free to be
exploited (or supressed) without a real accounting of their physical
relevance. There is no sharp statement ``here's the physical theory''
qua \emph{theory} and ``here's the mathematical interpretation''
enabling a judgment of how faithful the interpretation is -- apart
from experimental observation. When there is an axiomatic account we
can judge how well a given mathematical formalism supports an
interpretation of the axioms, independent of
experimentation. Likewise, we can judge how well we have captured our
physical evidence and experience with our axiomatics, independent of
any specific mathematical implementation, with accidental detail that
may or may not have physical significance. 

In lieu of a fully fleshed out and vetted axiomatic account of quantum
mechanics, interpreting the operational notions in service of modeling
physical systems will have to suffice. In other words, we are not in
the business of providing a model of Hilbert spaces and operators. We
are in the business of providing a model of quantum mechanics because
we are motivated by testing our notions of dynamics against physical
theory; and, the predictive calculations of the physical theory must
serve as the best formulation -- shy of a fully fleshed out axiomatic
account -- of the physical theory itself (as they have for scientific
theories since time immemorial). Put another way, despite a
whole-hearted commitment to an It-from-Bit ontology, we are firmly
aligned with the shut-up-and-calculate camp as the best way to obtain
results either from the physical perspective or as a quality assurance
measure of our fledgling theory of dynamics.

In detail, we present a reflective process calculus. Then we develop
intuitive correspondences between the notions available in this
calculus and the usual physical notions supporting quantum mechanical
calculations. Thus, 

\begin{table}[htp]
  \center{
    \fbox{
      \begin{tabular}{c|c}
        quantum mechanics & process calculus \\
        \hline
        scalar & name \\
        state vector & process \\
        dual & contextual duals \\
        matrix & formal sums of process-context-dual pairs \\
        orthogonality & process annihilation \\
        inner product & execution-formula + quoting
      \end{tabular}
    }
  }
  \caption{QM - process calculi correspondences}
\end{table}

Then we tighten up these intuitions to operational definitions. We
employ the Dirac notation as the best proxy we can find for an
abstract syntax of the quantum mechanical notions. The definitions we
develop put us in contact with equational constraints coming from the
theory that we demonstrate the definitions and calculations satisfy.

This puts us in a position to shut up and calculate for the
Stern-Gerlach experimental set up, showing how these predictive
calculations become calculations on processes in our theory of a
reflective process calculus.

Penultimately, we demonstrate that the notion of metric coming from
the inner product coincides with the notion of metric available from
the theory of bisimulation. This demonstration gives us the right to
think of space as arising from behavior. Finally, we consider where we
might go from the new vantage point we have obtained.

% section introduction (end) 
 
% section introduction (end)

% \documentclass[12pt]{llncs}
%\documentclass{jktr}

\usepackage[pdftex]{hyperref}                   
\usepackage {listings}
\usepackage {mathpartir}
\usepackage{bcprules}
%\usepackage{listings}
                       
\usepackage{graphicx} 
%\usepackage[margins=2.5cm,nohead,nofoot]{geometry}
%\usepackage{geometry}
\usepackage{amsfonts}
\usepackage{amstext}
\usepackage{latexsym}
\usepackage{amssymb}
\usepackage{color}


%\include{myPreamble}
\include{qm2pi.local} 

%\ifpdf
%\usepackage[pdftex]{graphicx}
%\else
%\usepackage{graphicx}
%\fi

 % \ifpdf
%  \usepackage{pdfsync}
%  \if


%\title{Brief Article}
%\author{David F. Snyder}
%\author{L.G. Meredith}

%\address{Dept. of Math., Texas State University--San Marcos, San Marcos, TX 78666}
       
\pagestyle{empty}


\begin{document}

\lstset{language=[Objective]Caml,frame=shadowbox}

\input{qm2pi.front}

% section front matter (end)

\input{qm2pi.intro} 
 
% section introduction (end)

% \input{qm2pi.knotations} 

% section notation (end)

\input{qm2pi.process.calculi} 

% section concurrent_process_calculi_and_spatial_logics_ (end)
    
%\input{qm2pi.knots2pi} 

%\input{qm2pi.trefoil} 

%\input{qm2pi.mainthm} 

% subsection basic_interpretation (end)

%\input{qm2pi.rho.presentation} 
\subsection{The syntax and semantics of the notation system}\label{sub:the_syntax_and_semantics_of_the_notation_system} % (fold)

We now summarize a technical presentation of the calculus that
embodies our theory of dynamics. The typical presentation of such a
calculus follows the style of giving generators and relations on
them. The grammar, below, describing term constructors, freely
generates the set of processes, $\Proc$. This set is then quotiented
by a relation known as structural congruence and it is over this set
that the notion of dynamics is expressed. This presentation is
essentially that of \cite{MeredithR05} with the addition of
polyadicity and summation. For readability we have relegated some of
the technical subtleties to an appendix.

\subsubsection{Process grammar}\label{subsub:process_grammar}

\begin{mathpar}
  \inferrule* [lab=synchronization] {} {{M} \bc \pzero \;|\; x?F \;|\; x!C }
  \and
  \inferrule* [lab=abstraction] {} {{F} \bc (x)P}
  \and
  \inferrule* [lab=concretion] {} {{C} \bc \langle Q \rangle}
  \and
  \inferrule* [lab=process] {} {{P,Q} \bc M \;| \;P|Q \;|\; @{x}}
  \and
  \inferrule* [lab=name] {} {{x} \bc \quotep{P}}
\end{mathpar} 

Note that $\vec{x}$ (resp. $\vec{P}$) denotes a vector of names
(resp. processes) of length $|\vec{x}|$ (resp. $|\vec{P}|$). We adopt
the following useful abbreviations.

\begin{mathpar}
   x?(\vec{y}).P := x.(\vec{y})P \and  x\clift{\vec{P}} := x.\clift{\vec{P}}
   \and x!(y) := \lift{x}{\dropn{y}}
   \and \Pi_{i=0}^{n-1}P_i := P_0 | \ldots | P_{n-1}
\end{mathpar}

\subsubsection{Structural congruence}

\paragraph{Free and bound names and alpha-equivalence.} At the
core of structural equivalence is alpha-equivalence which identifies
process that are the same up to a change of variable. Formally, we
recognize the distinction between free and bound names. The free names
of a process, $\freenames{P}$, may be calculated recursively as
follows:

\begin{mathpar}
\freenames{\pzero} := \emptyset
  \and \\
  \freenames{x?(y).P} := \{ x \} \cup (\freenames{P} \setminus \{ y \})
  \and 
  \freenames{x!\langle P \rangle} := \{ x \} \cup \{ P \} 
  \and \\
  \freenames{P|Q} := \freenames{P} \cup \freenames{Q}
  \and \\
  \freenames{@{x}} := \{ x \}
\end{mathpar}

$\pi$
$\quotep{\pi}$

$\freenames{-} : \pi \to \mathcal{P}(\quotep{\pi})$

\begin{eqnarray*}
  \freenames{\pzero} & := & \emptyset \\
  \freenames{x?(y).P} & := & \{ x \} \cup (\freenames{P} \setminus \{ y \}) \\
  \freenames{x!\langle P \rangle} & := & \{ x \} \cup \{ P \} \\
  \freenames{P|Q} & := & \freenames{P} \cup \freenames{Q} \\
  \freenames{\dropn{x}} & := & \{ x \}
\end{eqnarray*}

The bound names of a process, $\boundnames{P}$, are those names occurring in $P$
that are not free. For example, in $x?(y).0$, the name $x$ is free, while $y$ is bound.

\begin{mathpar}
  \inferrule* [lab=monoidal-laws] {} { P|Q \equiv Q|P \and P|0 \equiv P \and P|(Q|R) \equiv (P|Q)|R }
\end{mathpar}

\begin{mathpar}
  \inferrule* [lab=alpha-equivalence] {} { (x)P \equiv (y)P\{y/x\} \and y \not\in \freenames{P} }
\end{mathpar}

\begin{definition}
Then two processes, $P,Q$, are alpha-equivalent if $P = Q\{\vec{y}/\vec{x}\}$ for
some $\vec{x} \in \boundnames{Q},\vec{y} \in \boundnames{P}$, where $Q\{\vec{y}/\vec{x}\}$
denotes the capture-avoiding substitution of $\vec{y}$ for $\vec{x}$ in $Q$.
\end{definition}

\begin{definition}
  The {\em structural congruence} \cite{SangiorgiWalker} , $\equiv$,
  between processes is the least congruence containing
  alpha-equivalence, satisfying the abelian monoid laws
  (associativity, commutativity and $\pzero$ as identity) for parallel
  composition $|$ and for summation $+$.
\end{definition}

\subsection{Name equivalence}

We take name equivalence, written $\nameeq$, to be the smallest
equivalence relation generated by the following rules.

\begin{mathpar}
\inferrule*[lab=Quote-drop]
{ }
{ \quotep{@{x}} \nameeq x }

\inferrule*[lab=Struct-equiv]
{ P \scong Q }
{ \quotep{P} \nameeq \quotep{Q} }
\end{mathpar}

The astute reader will have noticed that the mutual recursion of names
and processes imposes a mutual recursion on alpha-equivalence and
structural equivalence via name-equivalence. Fortunately, all of this
works out pleasantly and we may calculate in the natural way, free of
concern. The reader interested in the details is referred to the
appendix \ref{appendix:rho_details}.

\subsection{Substitution}

We use $\Proc$ for the set of processes, $\QProc$ for the set of
names, and $\id{\{}\vec{y} / \vec{x} \id{\}}$ to denote partial maps,
$s : \QProc \rightarrow \QProc$. A map, $s$ lifts, uniquely, to a map
on process terms, $\widehat{s} : \Proc \rightarrow \Proc$ by the
following equations.

\begin{mathpar}
  (0) \psubstp{Q}{P} := 0 \\
  (R \juxtap S) \psubstp{Q}{P}
  :=    
  (R)\psubstp{Q}{P} \juxtap (S) \psubstp{Q}{P} \\
  (x?(y).R) \psubstp{Q}{P}    
  :=    
  (x)\substp{Q}{P} (z)\concat( (R \psubstn{z}{y}) \psubstp{Q}{P} ) \\
  (\lift{x}{R}) \psubstp{Q}{P}  
  :=
  \lift{(x)\substp{Q}{P}}{ R \psubstp{Q}{P} } \\
%   (\dropn{x})  \psubstp{Q}{P}       
%   := 
%   \left\{ 
%     \begin{array}{ccc} 
%       \dropn{\quotep{Q}} & & x \nameeq \quotep{P} \\
%       \dropn{x} & & otherwise \\
%     \end{array}
%   \right. 
  (\dropn{x})  \psubstp{Q}{P}       
  := 
  \left\{ 
    \begin{array}{ccc} 
      Q & & x \nameeq \quotep{P} \\
      \dropn{x} & & otherwise \\
    \end{array}
  \right.
\end{mathpar}
 

where

\begin{eqnarray}
  (x)\id{\{} \lpquote Q \rpquote / \lpquote P \rpquote \id{\}}            = 
  \left\{ 
    \begin{array}{ccc}
      \lpquote Q \rpquote & & x \nameeq \lpquote P \rpquote \\
      x & & otherwise \\
    \end{array}
  \right. \nonumber
\end{eqnarray}

and $z$ is chosen distinct from $\quotep{P}$, $\quotep{Q}$, the free
names in $Q$, and all the names in $R$. Our $\alpha$-equivalence will
be built in the standard way from this substitution.

\begin{remark}\label{rem:no_self_referential_names}
  One consequence of these definitions is that $\forall P. \quotep{P}
  \not\in \freenames{P}$.
\end{remark}

\subsection{ Dynamic quote: an example }

Anticipating something of what's to come, consider applying the
substitution, $\widehat{\id{\{}u / z \id{\}}}$, to the following pair
of processes, $\lift{w}{y!(z)}$ and $w[ \lpquote y!(z) \rpquote ]$.

\begin{eqnarray}
	\lift{w}{y!(z)}\widehat{\id{\{}u / z \id{\}}}
		& = &
		\lift{w}{y!(u)} \nonumber\\
	w[ \lpquote y!(z) \rpquote ] \widehat{ \id{\{}u / z \id{\}} }
		& = &
		w[ \lpquote y!(z) \rpquote ] \nonumber
\end{eqnarray}

Because the body of the process between quotes is impervious to
substitution, we get radically different answers. In fact, by
examining the first process in an input context,
e.g. $x?(z).\lift{w}{y!(z)}$, we see that the process under the lift
operator may be shaped by prefixed inputs binding a name inside it. In
this sense, the lift operator will be seen as a way to dynamically
construct processes before reifying them as names.

Finally equipped with these standard features we can present the
dynamics of the calculus.

\subsubsection{Operational semantics} 

Finally, we introduce the computational dynamics. What marks these
algebras as distinct from other more traditionally studied algebraic
structures, e.g. vector spaces or polynomial rings, is the manner in
which dynamics is captured. In traditional structures, dynamics is typically
expressed through morphisms between such structures, as in linear maps
between vector spaces or morphisms between rings. In algebras
associated with the semantics of computation, the dynamics is
expressed as part of the algebraic structure itself, through a
reduction reduction relation typically denoted by $\red$. Below, we
give a recursive presentation of this relation for the calculus used
in the encoding.

$\red \subseteq \pi \times \pi$
$\red : \pi \to \mathcal{P}(\pi)$

\begin{mathpar}
  \inferrule* [lab=Comm] { \textsf{match}( x_{src}, x_{trgt} ) } { x_{trgt}?(y)P \; | \; x_{src}!\langle {Q} \rangle \red P\{\quotep{Q}/y}\} }
  \and \\
  \inferrule* [lab=Par] {{P} \red {P}'} {{{P} | {Q}} \red {{P}' | {Q}}}
  \and
  \inferrule* [lab=Equiv]{{{P} \scong {P}'} \andalso {{P}' \red {Q}'} \andalso {{Q}' \scong {Q}}}{{P} \red {Q}}
\end{mathpar}

\begin{eqnarray*}
  match_{\equiv} (\quotep{P},\quotep{Q}) & := & P \equiv Q \\
  match_{\dagger}(\quotep{P},\quotep{Q}) & := & \forall R. P|Q \red^{*} R => R \red^{*} 0 \\
  match_{K}(\quotep{P},\quotep{Q}) & := & K \mbox{ for some context } K
\end{eqnarray*}

$u?(x)P | u!\langle Q \rangle \red P\{\quotep{Q}/x\}$

%We write $\wred$ for $\red^*$, and $P\red$ if $\exists Q $ such that $ P \red Q$.
We write $P\red$ if $\exists Q $ such that $ P \red Q$ and $P\not\red$, otherwise.

\section{Replication}

As mentioned before, it is known that replication (and hence
recursion) can be implemented in a higher-order process algebra
\cite{SangiorgiWalker}. As our first example of calculation with the
machinery thus far presented we give the construction explicitly in
the {\rhoc}.

\begin{eqnarray}
	D_{x} & := & \prefix{x}{y}{(\binpar{\outputp{x}{y}}{@{y}})} \nonumber\\
	\bangp_{x}{P} & := & \binpar{{x}!\langle{\binpar{D_{x}}{P}}\rangle}{D_{x}} \nonumber
\end{eqnarray}

\begin{eqnarray}
	\bangp_{x}{P} & & \nonumber\\
	=
	& {x}!\langle{(\prefix{x}{y}{(\outputp{x}{y} | @{y})) | P}}\rangle 
	      | \prefix{x}{y}{(\outputp{x}{y} | @{y})} & \nonumber\\
	\red
	& (\outputp{x}{y} | @{y})\substn{\quotep{(\prefix{x}{y}{(@{y} | \outputp{x}{y})) | P}}}{y} & \nonumber\\
	=
	& \outputp{x}{\quotep{(\prefix{x}{y}{(\outputp{x}{y} | @{y})) | P}}}
	  | {(\prefix{x}{y}{(\outputp{x}{y} | @{y})) | P}} & \nonumber\\
	\red
	& \ldots & \nonumber\\
	\red^*
	& P | P | \ldots & \nonumber
\end{eqnarray}

Of course, this encoding, as an implementation, runs away, unfolding
$\bangp{P}$ eagerly. A lazier and more implementable replication
operator, restricted to input-guarded processes, may be obtained as follows.

\begin{eqnarray}
\bangp{\prefix{u}{v}{P}} 
	:= 
	\binpar{\lift{x}{\prefix{u}{v}{(\binpar{D(x)}{P})}}}{D(x)} \nonumber
\end{eqnarray}

\begin{remark}
  Note that the lazier definition still does not deal with summation
  or mixed summation (i.e. sums over input and output). The reader is
  invited to construct definitions of replication that deal with these
  features. 

  Further, the definitions are parameterized in a name, $x$. Can you,
  gentle reader, make a definition that eliminates this parameter and
  guarantees no accidental interaction between the replication
  machinery and the process being replicated -- i.e. no accidental
  sharing of names used by the process to get its work done and the
  name(s) used by the replication to effect copying. This latter
  revision of the definition of replication is crucial to obtaining
  the expected identity $!!P \sim !P$.
\end{remark}

\begin{remark}\label{rem:paradoxical_combinator}
  The reader familiar with the lambda calculus will have noticed the
  similarity between $D$ and the paradoxical combinator.

  [Ed. note: the existence of this seems to suggest we have to be more
  restrictive on the set of processes and names we admit if we are to
  support no-cloning.]
\end{remark}

\subsubsection{Bisimulation}

The computational dynamics gives rise to another kind of equivalence,
the equivalence of computational behavior. As previously mentioned
this is typically captured \emph{via} some form of bisimulation.

% The notion we use in this paper is weak barbed bisimulation
% \cite{milner91polyadicpi}.

The notion we use in this paper is derived from weak barbed
bisimulation \cite{milner91polyadicpi}. 

\begin{definition}
An \emph{observation relation}, $\downarrow_{\mathcal N}$, over a set
of names, $\mathcal N$, is the smallest relation satisfying the rules
below.

\infrule[Out-barb]{y \in {\mathcal N}, \; x \nameeq y}
		  {\outputp{x}{v} \downarrow_{\mathcal N} x}
\infrule[Par-barb]{\mbox{$P\downarrow_{\mathcal N} x$ or $Q\downarrow_{\mathcal N} x$}}
		  {\binpar{P}{Q} \downarrow_{\mathcal N} x}

We write $P \Downarrow_{\mathcal N} x$ if there is $Q$ such that 
$P \wred Q$ and $Q \downarrow_{\mathcal N} x$.
\end{definition}

\begin{definition}
%\label{def.bbisim}
An  ${\mathcal N}$-\emph{barbed bisimulation} over a set of names, ${\mathcal N}$, is a symmetric binary relation 
${\mathcal S}_{\mathcal N}$ between agents such that $P\rel{S}_{\mathcal N}Q$ implies:
\begin{enumerate}
\item If $P \red P'$ then $Q \wred Q'$ and $P'\rel{S}_{\mathcal N} Q'$.
\item If $P\downarrow_{\mathcal N} x$, then $Q\Downarrow_{\mathcal N} x$.
\end{enumerate}
$P$ is ${\mathcal N}$-barbed bisimilar to $Q$, written
$P \wbbisim_{\mathcal N} Q$, if $P \rel{S}_{\mathcal N} Q$ for some ${\mathcal N}$-barbed bisimulation ${\mathcal S}_{\mathcal N}$.
\end{definition}

$\mathcal{R} \subseteq \pi \times \pi$

$P \mathcal{R} Q => \forall P'. P \red P' \Rightarrow \exists Q'. Q \red Q', P' \mathcal{R} Q'$

$P \vdash x \Rightarrow Q \vdash x$

\begin{mathpar}
  \inferrule*[lab=Out-barb]{x \nameeq y}{{y}!\langle{Q}\rangle \vdash x}
  \and
  \inferrule*[lab=Par-barb]{\mbox{$P\vdash x$ or $Q\vdash x$}}{\binpar{P}{Q} \vdash x}
\end{mathpar}

\subsubsection{Contexts}

One of the principle advantages of computational calculi like the
$\pi$-calculus is a well-defined notion of context,
contextual-equivalence and a correlation between
contextual-equivalence and notions of bisimulation. The notion of
context allows the decomposition of a process into (sub-)process and
its syntactic environment, its context. Thus, a context may be
thought of as a process with a ``hole'' (written $\Box$) in it. The
application of a context $M$ to a process $P$, written $M[P]$, is
tantamount to filling the hole in $M$ with $P$. In this paper we do
not need the full weight of this theory, but do make use of the notion
of context in the proof the main theorem. 

\begin{mathpar}
  \inferrule* [lab=summation] {} {{M_{M},M_{N}} \bc \Box \;|\; x.M_{A} \;|\; M_{M}+M_{N}}
  \and
  \inferrule* [lab=agent] {} {{M_{A}} \bc (\vec{x})M_{P} \;| \; \clift{P_0,\ldots,M_{P},\ldots,P_N}}
  \and \\
  \inferrule* [lab=process] {} {{M_{P}} \bc M_{N} \;| \;P|M_{P} }
\end{mathpar} 

\begin{mathpar}
  \inferrule* [lab=sychronization] {} {M_{N} \bc \Box \;|\; x?M_{F} \;|\; x!M_{C}}
  \and
  \inferrule* [lab=abstraction] {} {{M_{F}} \bc (x)M_{P} }
  \and
  \inferrule* [lab=concretion] {} {{M_{C}} \bc \langle M_{P} \rangle }
  \and \\
  \inferrule* [lab=process] {} {{M_{P}} \bc M_{N} \;| \;P|M_{P} }
\end{mathpar}

\begin{definition}[contextual application] Given a context $M$, and
  process $P$, we define the \emph{contextual application}, $M[P] :=
  M\{P/\Box\}$. That is, the contextual application of M to P is the
  substitution of $P$ for $\Box$ in $M$.
\end{definition}

$\meaningof{-} : L \to \mathcal{P}(\pi)$

\begin{mathpar}
  \inferrule* [lab=collection] {} {\meaningof{true} = \pi, \and \meaningof{~E} = \pi \setminus \meaningof{E}, \and \meaningof{E_{1} \& E_{2}} = \meaningof{E_{1}} \cap \meaningof{E_{2}}}
\end{mathpar}

\begin{mathpar}
  \inferrule* [lab=structure] {} {\meaningof{0} = \{ P \in \pi | P \equiv 0 \}, \and \\ \meaningof{E_1 | E_2} = \{ P \in \pi | P \equiv P_{1} | P_{2}, P_{1} \in \meaningof{E_{1}}, P_{2} \in \meaningof{E_2}\} }
\end{mathpar}

\begin{mathpar}
 \inferrule* [lab=behavior] {} {\meaningof{\langle a?b \rangle E} = \{ P \in \pi | P \equiv Q | u?(y)P', \\ \and \\\\ \and \\ \;\;\; u \in \meaningof{a}, \forall z.P'\{z/y\} \in \meaningof{E\{z/b\}}\}, \and \\ \meaningof{a!E} = \{ P \in \pi | P \equiv Q | x!\langle P' \rangle, x \in \meaningof{a} P' \in \meaningof{E}\} }
\end{mathpar}

\begin{mathpar}
 \inferrule* [lab=nominal] {} {\meaningof{\quotep{E}} = \{ \quotep{P} \in \quotep{\pi} | P \in \meaningof{E} \}, \and \meaningof{\quotep{P}} = \{ \quotep{Q} \in \quotep{\pi} | P \equiv Q \} \and \\ \meaningof{@\quotep{E}} = \{ P \in \pi | P \equiv @x, x \in \meaningof{E} \}}
\end{mathpar}

\begin{eqnarray*}
  \\
  \meaningof{-} : TS \to ST
\end{eqnarray*}

\begin{eqnarray*}
  \\
  L : TS \to ST
\end{eqnarray*}

\begin{eqnarray*}
  \\
  P \models E \iff P \in \meaningof{E}
\end{eqnarray*}

\begin{eqnarray*}
  P \approx_{L} Q \iff \forall E \in L. P \models E \iff Q \models E
\end{eqnarray*}

\begin{eqnarray*}
  P \approx_{K} Q
\end{eqnarray*}

\begin{eqnarray*}
  P \approx Q
\end{eqnarray*}

$\approx_{K} = \approx = \approx_{L}$

\subsubsection{Contextual duality}

Note that contexts extend the quotation operation to a family of
operations from processes to names. Given a context, $M$, we can
define a \emph{nominal context}, $\quotep{M}$ by $\quotep{M}[P] :=
\quotep{M[P]}$. To foreshadow what is to come we observe that these
operations enjoy a duality with processes very much like the duality
between vectors and maps from vectors to scalars.

Further, because the calculus is essentially higher-order, we have a
correspondence between contexts and processes. More specifically,
given a name $x$ and a context $M$ we can construct $M^{*}_{x}$ such
that 

\begin{mathpar}
  M^{*}_{x} | \lift{x}{P} \red M[P]
\end{mathpar}

namely,

\begin{mathpar}
  M^{*}_{x} := x?(u).M[\dropn{u}]
\end{mathpar}

The dependence of $M^{*}_{x}$ on a name makes it an abstraction, 

\begin{mathpar}
  M^{*} := (x)x?(u).M[\dropn{u}]
\end{mathpar}

\subsection{Additional notation}

It will sometimes be convenient to denote the process a name
quotes. We already have the notation $x = \quotep{P}$, but it will be
convenient to introduce an alternate notation, $\procn{x}$, when we
want to emphasize the connection to the use of the name. Note that, by
virtue of name equivalence, $\quotep{\procn{x}} \nameeq x$; so, the
notation is consistent with previous definitions.

Further, because names have structure it is possible to effect
substitutions on the basis of that structure. This means we need to
upgrade our notation for substitutions, which we accomplish by
adapting comprehension notation. Thus,

\begin{mathpar}
  P\{ y / x : x \in S \}
\end{mathpar}

is interpreted to mean the process derived from P by replacing (in a
capture-avoiding manner) each occurrence of $x$ in $S$ by $y$. For example,

\begin{mathpar}
  P\{ \quotep{\procn{x}|\procn{x}} / x : x \in \freenames{P} \}
\end{mathpar}

will replace each (occurrence) of a free name $x$ in $P$ by
$\quotep{\procn{x}|\procn{x}}$.

Also, we will avail ourselves of the notation $x^{L}$ and $x^{R}$ to
denote injections of a name into disjoint copies of the name
space. There are numerous ways to accomplish this. One example can be
found in \cite{MeredithR05}. This notation overloads to vectors of
names: $\vec{x}^{\pi} := (x_{i}^{\pi} \; : \; 0 \leq i < |\vec{x}| )$ where $\pi \in \{L,R\}$.

We also use $P^{\Box} := P|\Box$.

In \cite{MeredithR05} an interpretation of the new operator is
given. It turns out that there are several possible interpretations
all enjoying the requisite algebraic properties of the operator (see
\cite{milner91polyadicpi}). We will therefore make liberal use of
$(\nu\; \vec{x})P$.

% subsection the_syntax_and_semantics_of_the_notation_system (end)   

\input{qm2pi.qmops} 

\input{qm2pi.sterngerlach} 

\input{qm2pi.metric} 

% section concurrent_process_calculi (end)

%\input{qm2pi.proofsketch}

% section proof sketch (end)

%\input{qm2pi.slviaknots} 

% section spatial logic via knots (end)

\input{qm2pi.conclusion}

% section conclusion (end)

%\input{qm2pi.dtcodes} 

% section wiring algorithm (end)

\input{qm2pi.ack} 

% section acknowledgments (end)

\newpage


\bibliographystyle{plain}   
\bibliography{../../biblios/main.bib}

\input{qm2pi.rhodetails}

\end{document}

 

% section notation (end)

\input{qm2pi.process.calculi} 

% section concurrent_process_calculi_and_spatial_logics_ (end)
    
%\documentclass[12pt]{llncs}
%\documentclass{jktr}

\usepackage[pdftex]{hyperref}                   
\usepackage {listings}
\usepackage {mathpartir}
\usepackage{bcprules}
%\usepackage{listings}
                       
\usepackage{graphicx} 
%\usepackage[margins=2.5cm,nohead,nofoot]{geometry}
%\usepackage{geometry}
\usepackage{amsfonts}
\usepackage{amstext}
\usepackage{latexsym}
\usepackage{amssymb}
\usepackage{color}


%\include{myPreamble}
\include{qm2pi.local} 

%\ifpdf
%\usepackage[pdftex]{graphicx}
%\else
%\usepackage{graphicx}
%\fi

 % \ifpdf
%  \usepackage{pdfsync}
%  \if


%\title{Brief Article}
%\author{David F. Snyder}
%\author{L.G. Meredith}

%\address{Dept. of Math., Texas State University--San Marcos, San Marcos, TX 78666}
       
\pagestyle{empty}


\begin{document}

\lstset{language=[Objective]Caml,frame=shadowbox}

\input{qm2pi.front}

% section front matter (end)

\input{qm2pi.intro} 
 
% section introduction (end)

% \input{qm2pi.knotations} 

% section notation (end)

\input{qm2pi.process.calculi} 

% section concurrent_process_calculi_and_spatial_logics_ (end)
    
%\input{qm2pi.knots2pi} 

%\input{qm2pi.trefoil} 

%\input{qm2pi.mainthm} 

% subsection basic_interpretation (end)

%\input{qm2pi.rho.presentation} 
\subsection{The syntax and semantics of the notation system}\label{sub:the_syntax_and_semantics_of_the_notation_system} % (fold)

We now summarize a technical presentation of the calculus that
embodies our theory of dynamics. The typical presentation of such a
calculus follows the style of giving generators and relations on
them. The grammar, below, describing term constructors, freely
generates the set of processes, $\Proc$. This set is then quotiented
by a relation known as structural congruence and it is over this set
that the notion of dynamics is expressed. This presentation is
essentially that of \cite{MeredithR05} with the addition of
polyadicity and summation. For readability we have relegated some of
the technical subtleties to an appendix.

\subsubsection{Process grammar}\label{subsub:process_grammar}

\begin{mathpar}
  \inferrule* [lab=synchronization] {} {{M} \bc \pzero \;|\; x?F \;|\; x!C }
  \and
  \inferrule* [lab=abstraction] {} {{F} \bc (x)P}
  \and
  \inferrule* [lab=concretion] {} {{C} \bc \langle Q \rangle}
  \and
  \inferrule* [lab=process] {} {{P,Q} \bc M \;| \;P|Q \;|\; @{x}}
  \and
  \inferrule* [lab=name] {} {{x} \bc \quotep{P}}
\end{mathpar} 

Note that $\vec{x}$ (resp. $\vec{P}$) denotes a vector of names
(resp. processes) of length $|\vec{x}|$ (resp. $|\vec{P}|$). We adopt
the following useful abbreviations.

\begin{mathpar}
   x?(\vec{y}).P := x.(\vec{y})P \and  x\clift{\vec{P}} := x.\clift{\vec{P}}
   \and x!(y) := \lift{x}{\dropn{y}}
   \and \Pi_{i=0}^{n-1}P_i := P_0 | \ldots | P_{n-1}
\end{mathpar}

\subsubsection{Structural congruence}

\paragraph{Free and bound names and alpha-equivalence.} At the
core of structural equivalence is alpha-equivalence which identifies
process that are the same up to a change of variable. Formally, we
recognize the distinction between free and bound names. The free names
of a process, $\freenames{P}$, may be calculated recursively as
follows:

\begin{mathpar}
\freenames{\pzero} := \emptyset
  \and \\
  \freenames{x?(y).P} := \{ x \} \cup (\freenames{P} \setminus \{ y \})
  \and 
  \freenames{x!\langle P \rangle} := \{ x \} \cup \{ P \} 
  \and \\
  \freenames{P|Q} := \freenames{P} \cup \freenames{Q}
  \and \\
  \freenames{@{x}} := \{ x \}
\end{mathpar}

$\pi$
$\quotep{\pi}$

$\freenames{-} : \pi \to \mathcal{P}(\quotep{\pi})$

\begin{eqnarray*}
  \freenames{\pzero} & := & \emptyset \\
  \freenames{x?(y).P} & := & \{ x \} \cup (\freenames{P} \setminus \{ y \}) \\
  \freenames{x!\langle P \rangle} & := & \{ x \} \cup \{ P \} \\
  \freenames{P|Q} & := & \freenames{P} \cup \freenames{Q} \\
  \freenames{\dropn{x}} & := & \{ x \}
\end{eqnarray*}

The bound names of a process, $\boundnames{P}$, are those names occurring in $P$
that are not free. For example, in $x?(y).0$, the name $x$ is free, while $y$ is bound.

\begin{mathpar}
  \inferrule* [lab=monoidal-laws] {} { P|Q \equiv Q|P \and P|0 \equiv P \and P|(Q|R) \equiv (P|Q)|R }
\end{mathpar}

\begin{mathpar}
  \inferrule* [lab=alpha-equivalence] {} { (x)P \equiv (y)P\{y/x\} \and y \not\in \freenames{P} }
\end{mathpar}

\begin{definition}
Then two processes, $P,Q$, are alpha-equivalent if $P = Q\{\vec{y}/\vec{x}\}$ for
some $\vec{x} \in \boundnames{Q},\vec{y} \in \boundnames{P}$, where $Q\{\vec{y}/\vec{x}\}$
denotes the capture-avoiding substitution of $\vec{y}$ for $\vec{x}$ in $Q$.
\end{definition}

\begin{definition}
  The {\em structural congruence} \cite{SangiorgiWalker} , $\equiv$,
  between processes is the least congruence containing
  alpha-equivalence, satisfying the abelian monoid laws
  (associativity, commutativity and $\pzero$ as identity) for parallel
  composition $|$ and for summation $+$.
\end{definition}

\subsection{Name equivalence}

We take name equivalence, written $\nameeq$, to be the smallest
equivalence relation generated by the following rules.

\begin{mathpar}
\inferrule*[lab=Quote-drop]
{ }
{ \quotep{@{x}} \nameeq x }

\inferrule*[lab=Struct-equiv]
{ P \scong Q }
{ \quotep{P} \nameeq \quotep{Q} }
\end{mathpar}

The astute reader will have noticed that the mutual recursion of names
and processes imposes a mutual recursion on alpha-equivalence and
structural equivalence via name-equivalence. Fortunately, all of this
works out pleasantly and we may calculate in the natural way, free of
concern. The reader interested in the details is referred to the
appendix \ref{appendix:rho_details}.

\subsection{Substitution}

We use $\Proc$ for the set of processes, $\QProc$ for the set of
names, and $\id{\{}\vec{y} / \vec{x} \id{\}}$ to denote partial maps,
$s : \QProc \rightarrow \QProc$. A map, $s$ lifts, uniquely, to a map
on process terms, $\widehat{s} : \Proc \rightarrow \Proc$ by the
following equations.

\begin{mathpar}
  (0) \psubstp{Q}{P} := 0 \\
  (R \juxtap S) \psubstp{Q}{P}
  :=    
  (R)\psubstp{Q}{P} \juxtap (S) \psubstp{Q}{P} \\
  (x?(y).R) \psubstp{Q}{P}    
  :=    
  (x)\substp{Q}{P} (z)\concat( (R \psubstn{z}{y}) \psubstp{Q}{P} ) \\
  (\lift{x}{R}) \psubstp{Q}{P}  
  :=
  \lift{(x)\substp{Q}{P}}{ R \psubstp{Q}{P} } \\
%   (\dropn{x})  \psubstp{Q}{P}       
%   := 
%   \left\{ 
%     \begin{array}{ccc} 
%       \dropn{\quotep{Q}} & & x \nameeq \quotep{P} \\
%       \dropn{x} & & otherwise \\
%     \end{array}
%   \right. 
  (\dropn{x})  \psubstp{Q}{P}       
  := 
  \left\{ 
    \begin{array}{ccc} 
      Q & & x \nameeq \quotep{P} \\
      \dropn{x} & & otherwise \\
    \end{array}
  \right.
\end{mathpar}
 

where

\begin{eqnarray}
  (x)\id{\{} \lpquote Q \rpquote / \lpquote P \rpquote \id{\}}            = 
  \left\{ 
    \begin{array}{ccc}
      \lpquote Q \rpquote & & x \nameeq \lpquote P \rpquote \\
      x & & otherwise \\
    \end{array}
  \right. \nonumber
\end{eqnarray}

and $z$ is chosen distinct from $\quotep{P}$, $\quotep{Q}$, the free
names in $Q$, and all the names in $R$. Our $\alpha$-equivalence will
be built in the standard way from this substitution.

\begin{remark}\label{rem:no_self_referential_names}
  One consequence of these definitions is that $\forall P. \quotep{P}
  \not\in \freenames{P}$.
\end{remark}

\subsection{ Dynamic quote: an example }

Anticipating something of what's to come, consider applying the
substitution, $\widehat{\id{\{}u / z \id{\}}}$, to the following pair
of processes, $\lift{w}{y!(z)}$ and $w[ \lpquote y!(z) \rpquote ]$.

\begin{eqnarray}
	\lift{w}{y!(z)}\widehat{\id{\{}u / z \id{\}}}
		& = &
		\lift{w}{y!(u)} \nonumber\\
	w[ \lpquote y!(z) \rpquote ] \widehat{ \id{\{}u / z \id{\}} }
		& = &
		w[ \lpquote y!(z) \rpquote ] \nonumber
\end{eqnarray}

Because the body of the process between quotes is impervious to
substitution, we get radically different answers. In fact, by
examining the first process in an input context,
e.g. $x?(z).\lift{w}{y!(z)}$, we see that the process under the lift
operator may be shaped by prefixed inputs binding a name inside it. In
this sense, the lift operator will be seen as a way to dynamically
construct processes before reifying them as names.

Finally equipped with these standard features we can present the
dynamics of the calculus.

\subsubsection{Operational semantics} 

Finally, we introduce the computational dynamics. What marks these
algebras as distinct from other more traditionally studied algebraic
structures, e.g. vector spaces or polynomial rings, is the manner in
which dynamics is captured. In traditional structures, dynamics is typically
expressed through morphisms between such structures, as in linear maps
between vector spaces or morphisms between rings. In algebras
associated with the semantics of computation, the dynamics is
expressed as part of the algebraic structure itself, through a
reduction reduction relation typically denoted by $\red$. Below, we
give a recursive presentation of this relation for the calculus used
in the encoding.

$\red \subseteq \pi \times \pi$
$\red : \pi \to \mathcal{P}(\pi)$

\begin{mathpar}
  \inferrule* [lab=Comm] { \textsf{match}( x_{src}, x_{trgt} ) } { x_{trgt}?(y)P \; | \; x_{src}!\langle {Q} \rangle \red P\{\quotep{Q}/y}\} }
  \and \\
  \inferrule* [lab=Par] {{P} \red {P}'} {{{P} | {Q}} \red {{P}' | {Q}}}
  \and
  \inferrule* [lab=Equiv]{{{P} \scong {P}'} \andalso {{P}' \red {Q}'} \andalso {{Q}' \scong {Q}}}{{P} \red {Q}}
\end{mathpar}

\begin{eqnarray*}
  match_{\equiv} (\quotep{P},\quotep{Q}) & := & P \equiv Q \\
  match_{\dagger}(\quotep{P},\quotep{Q}) & := & \forall R. P|Q \red^{*} R => R \red^{*} 0 \\
  match_{K}(\quotep{P},\quotep{Q}) & := & K \mbox{ for some context } K
\end{eqnarray*}

$u?(x)P | u!\langle Q \rangle \red P\{\quotep{Q}/x\}$

%We write $\wred$ for $\red^*$, and $P\red$ if $\exists Q $ such that $ P \red Q$.
We write $P\red$ if $\exists Q $ such that $ P \red Q$ and $P\not\red$, otherwise.

\section{Replication}

As mentioned before, it is known that replication (and hence
recursion) can be implemented in a higher-order process algebra
\cite{SangiorgiWalker}. As our first example of calculation with the
machinery thus far presented we give the construction explicitly in
the {\rhoc}.

\begin{eqnarray}
	D_{x} & := & \prefix{x}{y}{(\binpar{\outputp{x}{y}}{@{y}})} \nonumber\\
	\bangp_{x}{P} & := & \binpar{{x}!\langle{\binpar{D_{x}}{P}}\rangle}{D_{x}} \nonumber
\end{eqnarray}

\begin{eqnarray}
	\bangp_{x}{P} & & \nonumber\\
	=
	& {x}!\langle{(\prefix{x}{y}{(\outputp{x}{y} | @{y})) | P}}\rangle 
	      | \prefix{x}{y}{(\outputp{x}{y} | @{y})} & \nonumber\\
	\red
	& (\outputp{x}{y} | @{y})\substn{\quotep{(\prefix{x}{y}{(@{y} | \outputp{x}{y})) | P}}}{y} & \nonumber\\
	=
	& \outputp{x}{\quotep{(\prefix{x}{y}{(\outputp{x}{y} | @{y})) | P}}}
	  | {(\prefix{x}{y}{(\outputp{x}{y} | @{y})) | P}} & \nonumber\\
	\red
	& \ldots & \nonumber\\
	\red^*
	& P | P | \ldots & \nonumber
\end{eqnarray}

Of course, this encoding, as an implementation, runs away, unfolding
$\bangp{P}$ eagerly. A lazier and more implementable replication
operator, restricted to input-guarded processes, may be obtained as follows.

\begin{eqnarray}
\bangp{\prefix{u}{v}{P}} 
	:= 
	\binpar{\lift{x}{\prefix{u}{v}{(\binpar{D(x)}{P})}}}{D(x)} \nonumber
\end{eqnarray}

\begin{remark}
  Note that the lazier definition still does not deal with summation
  or mixed summation (i.e. sums over input and output). The reader is
  invited to construct definitions of replication that deal with these
  features. 

  Further, the definitions are parameterized in a name, $x$. Can you,
  gentle reader, make a definition that eliminates this parameter and
  guarantees no accidental interaction between the replication
  machinery and the process being replicated -- i.e. no accidental
  sharing of names used by the process to get its work done and the
  name(s) used by the replication to effect copying. This latter
  revision of the definition of replication is crucial to obtaining
  the expected identity $!!P \sim !P$.
\end{remark}

\begin{remark}\label{rem:paradoxical_combinator}
  The reader familiar with the lambda calculus will have noticed the
  similarity between $D$ and the paradoxical combinator.

  [Ed. note: the existence of this seems to suggest we have to be more
  restrictive on the set of processes and names we admit if we are to
  support no-cloning.]
\end{remark}

\subsubsection{Bisimulation}

The computational dynamics gives rise to another kind of equivalence,
the equivalence of computational behavior. As previously mentioned
this is typically captured \emph{via} some form of bisimulation.

% The notion we use in this paper is weak barbed bisimulation
% \cite{milner91polyadicpi}.

The notion we use in this paper is derived from weak barbed
bisimulation \cite{milner91polyadicpi}. 

\begin{definition}
An \emph{observation relation}, $\downarrow_{\mathcal N}$, over a set
of names, $\mathcal N$, is the smallest relation satisfying the rules
below.

\infrule[Out-barb]{y \in {\mathcal N}, \; x \nameeq y}
		  {\outputp{x}{v} \downarrow_{\mathcal N} x}
\infrule[Par-barb]{\mbox{$P\downarrow_{\mathcal N} x$ or $Q\downarrow_{\mathcal N} x$}}
		  {\binpar{P}{Q} \downarrow_{\mathcal N} x}

We write $P \Downarrow_{\mathcal N} x$ if there is $Q$ such that 
$P \wred Q$ and $Q \downarrow_{\mathcal N} x$.
\end{definition}

\begin{definition}
%\label{def.bbisim}
An  ${\mathcal N}$-\emph{barbed bisimulation} over a set of names, ${\mathcal N}$, is a symmetric binary relation 
${\mathcal S}_{\mathcal N}$ between agents such that $P\rel{S}_{\mathcal N}Q$ implies:
\begin{enumerate}
\item If $P \red P'$ then $Q \wred Q'$ and $P'\rel{S}_{\mathcal N} Q'$.
\item If $P\downarrow_{\mathcal N} x$, then $Q\Downarrow_{\mathcal N} x$.
\end{enumerate}
$P$ is ${\mathcal N}$-barbed bisimilar to $Q$, written
$P \wbbisim_{\mathcal N} Q$, if $P \rel{S}_{\mathcal N} Q$ for some ${\mathcal N}$-barbed bisimulation ${\mathcal S}_{\mathcal N}$.
\end{definition}

$\mathcal{R} \subseteq \pi \times \pi$

$P \mathcal{R} Q => \forall P'. P \red P' \Rightarrow \exists Q'. Q \red Q', P' \mathcal{R} Q'$

$P \vdash x \Rightarrow Q \vdash x$

\begin{mathpar}
  \inferrule*[lab=Out-barb]{x \nameeq y}{{y}!\langle{Q}\rangle \vdash x}
  \and
  \inferrule*[lab=Par-barb]{\mbox{$P\vdash x$ or $Q\vdash x$}}{\binpar{P}{Q} \vdash x}
\end{mathpar}

\subsubsection{Contexts}

One of the principle advantages of computational calculi like the
$\pi$-calculus is a well-defined notion of context,
contextual-equivalence and a correlation between
contextual-equivalence and notions of bisimulation. The notion of
context allows the decomposition of a process into (sub-)process and
its syntactic environment, its context. Thus, a context may be
thought of as a process with a ``hole'' (written $\Box$) in it. The
application of a context $M$ to a process $P$, written $M[P]$, is
tantamount to filling the hole in $M$ with $P$. In this paper we do
not need the full weight of this theory, but do make use of the notion
of context in the proof the main theorem. 

\begin{mathpar}
  \inferrule* [lab=summation] {} {{M_{M},M_{N}} \bc \Box \;|\; x.M_{A} \;|\; M_{M}+M_{N}}
  \and
  \inferrule* [lab=agent] {} {{M_{A}} \bc (\vec{x})M_{P} \;| \; \clift{P_0,\ldots,M_{P},\ldots,P_N}}
  \and \\
  \inferrule* [lab=process] {} {{M_{P}} \bc M_{N} \;| \;P|M_{P} }
\end{mathpar} 

\begin{mathpar}
  \inferrule* [lab=sychronization] {} {M_{N} \bc \Box \;|\; x?M_{F} \;|\; x!M_{C}}
  \and
  \inferrule* [lab=abstraction] {} {{M_{F}} \bc (x)M_{P} }
  \and
  \inferrule* [lab=concretion] {} {{M_{C}} \bc \langle M_{P} \rangle }
  \and \\
  \inferrule* [lab=process] {} {{M_{P}} \bc M_{N} \;| \;P|M_{P} }
\end{mathpar}

\begin{definition}[contextual application] Given a context $M$, and
  process $P$, we define the \emph{contextual application}, $M[P] :=
  M\{P/\Box\}$. That is, the contextual application of M to P is the
  substitution of $P$ for $\Box$ in $M$.
\end{definition}

$\meaningof{-} : L \to \mathcal{P}(\pi)$

\begin{mathpar}
  \inferrule* [lab=collection] {} {\meaningof{true} = \pi, \and \meaningof{~E} = \pi \setminus \meaningof{E}, \and \meaningof{E_{1} \& E_{2}} = \meaningof{E_{1}} \cap \meaningof{E_{2}}}
\end{mathpar}

\begin{mathpar}
  \inferrule* [lab=structure] {} {\meaningof{0} = \{ P \in \pi | P \equiv 0 \}, \and \\ \meaningof{E_1 | E_2} = \{ P \in \pi | P \equiv P_{1} | P_{2}, P_{1} \in \meaningof{E_{1}}, P_{2} \in \meaningof{E_2}\} }
\end{mathpar}

\begin{mathpar}
 \inferrule* [lab=behavior] {} {\meaningof{\langle a?b \rangle E} = \{ P \in \pi | P \equiv Q | u?(y)P', \\ \and \\\\ \and \\ \;\;\; u \in \meaningof{a}, \forall z.P'\{z/y\} \in \meaningof{E\{z/b\}}\}, \and \\ \meaningof{a!E} = \{ P \in \pi | P \equiv Q | x!\langle P' \rangle, x \in \meaningof{a} P' \in \meaningof{E}\} }
\end{mathpar}

\begin{mathpar}
 \inferrule* [lab=nominal] {} {\meaningof{\quotep{E}} = \{ \quotep{P} \in \quotep{\pi} | P \in \meaningof{E} \}, \and \meaningof{\quotep{P}} = \{ \quotep{Q} \in \quotep{\pi} | P \equiv Q \} \and \\ \meaningof{@\quotep{E}} = \{ P \in \pi | P \equiv @x, x \in \meaningof{E} \}}
\end{mathpar}

\begin{eqnarray*}
  \\
  \meaningof{-} : TS \to ST
\end{eqnarray*}

\begin{eqnarray*}
  \\
  L : TS \to ST
\end{eqnarray*}

\begin{eqnarray*}
  \\
  P \models E \iff P \in \meaningof{E}
\end{eqnarray*}

\begin{eqnarray*}
  P \approx_{L} Q \iff \forall E \in L. P \models E \iff Q \models E
\end{eqnarray*}

\begin{eqnarray*}
  P \approx_{K} Q
\end{eqnarray*}

\begin{eqnarray*}
  P \approx Q
\end{eqnarray*}

$\approx_{K} = \approx = \approx_{L}$

\subsubsection{Contextual duality}

Note that contexts extend the quotation operation to a family of
operations from processes to names. Given a context, $M$, we can
define a \emph{nominal context}, $\quotep{M}$ by $\quotep{M}[P] :=
\quotep{M[P]}$. To foreshadow what is to come we observe that these
operations enjoy a duality with processes very much like the duality
between vectors and maps from vectors to scalars.

Further, because the calculus is essentially higher-order, we have a
correspondence between contexts and processes. More specifically,
given a name $x$ and a context $M$ we can construct $M^{*}_{x}$ such
that 

\begin{mathpar}
  M^{*}_{x} | \lift{x}{P} \red M[P]
\end{mathpar}

namely,

\begin{mathpar}
  M^{*}_{x} := x?(u).M[\dropn{u}]
\end{mathpar}

The dependence of $M^{*}_{x}$ on a name makes it an abstraction, 

\begin{mathpar}
  M^{*} := (x)x?(u).M[\dropn{u}]
\end{mathpar}

\subsection{Additional notation}

It will sometimes be convenient to denote the process a name
quotes. We already have the notation $x = \quotep{P}$, but it will be
convenient to introduce an alternate notation, $\procn{x}$, when we
want to emphasize the connection to the use of the name. Note that, by
virtue of name equivalence, $\quotep{\procn{x}} \nameeq x$; so, the
notation is consistent with previous definitions.

Further, because names have structure it is possible to effect
substitutions on the basis of that structure. This means we need to
upgrade our notation for substitutions, which we accomplish by
adapting comprehension notation. Thus,

\begin{mathpar}
  P\{ y / x : x \in S \}
\end{mathpar}

is interpreted to mean the process derived from P by replacing (in a
capture-avoiding manner) each occurrence of $x$ in $S$ by $y$. For example,

\begin{mathpar}
  P\{ \quotep{\procn{x}|\procn{x}} / x : x \in \freenames{P} \}
\end{mathpar}

will replace each (occurrence) of a free name $x$ in $P$ by
$\quotep{\procn{x}|\procn{x}}$.

Also, we will avail ourselves of the notation $x^{L}$ and $x^{R}$ to
denote injections of a name into disjoint copies of the name
space. There are numerous ways to accomplish this. One example can be
found in \cite{MeredithR05}. This notation overloads to vectors of
names: $\vec{x}^{\pi} := (x_{i}^{\pi} \; : \; 0 \leq i < |\vec{x}| )$ where $\pi \in \{L,R\}$.

We also use $P^{\Box} := P|\Box$.

In \cite{MeredithR05} an interpretation of the new operator is
given. It turns out that there are several possible interpretations
all enjoying the requisite algebraic properties of the operator (see
\cite{milner91polyadicpi}). We will therefore make liberal use of
$(\nu\; \vec{x})P$.

% subsection the_syntax_and_semantics_of_the_notation_system (end)   

\input{qm2pi.qmops} 

\input{qm2pi.sterngerlach} 

\input{qm2pi.metric} 

% section concurrent_process_calculi (end)

%\input{qm2pi.proofsketch}

% section proof sketch (end)

%\input{qm2pi.slviaknots} 

% section spatial logic via knots (end)

\input{qm2pi.conclusion}

% section conclusion (end)

%\input{qm2pi.dtcodes} 

% section wiring algorithm (end)

\input{qm2pi.ack} 

% section acknowledgments (end)

\newpage


\bibliographystyle{plain}   
\bibliography{../../biblios/main.bib}

\input{qm2pi.rhodetails}

\end{document}

 

%\documentclass[12pt]{llncs}
%\documentclass{jktr}

\usepackage[pdftex]{hyperref}                   
\usepackage {listings}
\usepackage {mathpartir}
\usepackage{bcprules}
%\usepackage{listings}
                       
\usepackage{graphicx} 
%\usepackage[margins=2.5cm,nohead,nofoot]{geometry}
%\usepackage{geometry}
\usepackage{amsfonts}
\usepackage{amstext}
\usepackage{latexsym}
\usepackage{amssymb}
\usepackage{color}


%\include{myPreamble}
\include{qm2pi.local} 

%\ifpdf
%\usepackage[pdftex]{graphicx}
%\else
%\usepackage{graphicx}
%\fi

 % \ifpdf
%  \usepackage{pdfsync}
%  \if


%\title{Brief Article}
%\author{David F. Snyder}
%\author{L.G. Meredith}

%\address{Dept. of Math., Texas State University--San Marcos, San Marcos, TX 78666}
       
\pagestyle{empty}


\begin{document}

\lstset{language=[Objective]Caml,frame=shadowbox}

\input{qm2pi.front}

% section front matter (end)

\input{qm2pi.intro} 
 
% section introduction (end)

% \input{qm2pi.knotations} 

% section notation (end)

\input{qm2pi.process.calculi} 

% section concurrent_process_calculi_and_spatial_logics_ (end)
    
%\input{qm2pi.knots2pi} 

%\input{qm2pi.trefoil} 

%\input{qm2pi.mainthm} 

% subsection basic_interpretation (end)

%\input{qm2pi.rho.presentation} 
\subsection{The syntax and semantics of the notation system}\label{sub:the_syntax_and_semantics_of_the_notation_system} % (fold)

We now summarize a technical presentation of the calculus that
embodies our theory of dynamics. The typical presentation of such a
calculus follows the style of giving generators and relations on
them. The grammar, below, describing term constructors, freely
generates the set of processes, $\Proc$. This set is then quotiented
by a relation known as structural congruence and it is over this set
that the notion of dynamics is expressed. This presentation is
essentially that of \cite{MeredithR05} with the addition of
polyadicity and summation. For readability we have relegated some of
the technical subtleties to an appendix.

\subsubsection{Process grammar}\label{subsub:process_grammar}

\begin{mathpar}
  \inferrule* [lab=synchronization] {} {{M} \bc \pzero \;|\; x?F \;|\; x!C }
  \and
  \inferrule* [lab=abstraction] {} {{F} \bc (x)P}
  \and
  \inferrule* [lab=concretion] {} {{C} \bc \langle Q \rangle}
  \and
  \inferrule* [lab=process] {} {{P,Q} \bc M \;| \;P|Q \;|\; @{x}}
  \and
  \inferrule* [lab=name] {} {{x} \bc \quotep{P}}
\end{mathpar} 

Note that $\vec{x}$ (resp. $\vec{P}$) denotes a vector of names
(resp. processes) of length $|\vec{x}|$ (resp. $|\vec{P}|$). We adopt
the following useful abbreviations.

\begin{mathpar}
   x?(\vec{y}).P := x.(\vec{y})P \and  x\clift{\vec{P}} := x.\clift{\vec{P}}
   \and x!(y) := \lift{x}{\dropn{y}}
   \and \Pi_{i=0}^{n-1}P_i := P_0 | \ldots | P_{n-1}
\end{mathpar}

\subsubsection{Structural congruence}

\paragraph{Free and bound names and alpha-equivalence.} At the
core of structural equivalence is alpha-equivalence which identifies
process that are the same up to a change of variable. Formally, we
recognize the distinction between free and bound names. The free names
of a process, $\freenames{P}$, may be calculated recursively as
follows:

\begin{mathpar}
\freenames{\pzero} := \emptyset
  \and \\
  \freenames{x?(y).P} := \{ x \} \cup (\freenames{P} \setminus \{ y \})
  \and 
  \freenames{x!\langle P \rangle} := \{ x \} \cup \{ P \} 
  \and \\
  \freenames{P|Q} := \freenames{P} \cup \freenames{Q}
  \and \\
  \freenames{@{x}} := \{ x \}
\end{mathpar}

$\pi$
$\quotep{\pi}$

$\freenames{-} : \pi \to \mathcal{P}(\quotep{\pi})$

\begin{eqnarray*}
  \freenames{\pzero} & := & \emptyset \\
  \freenames{x?(y).P} & := & \{ x \} \cup (\freenames{P} \setminus \{ y \}) \\
  \freenames{x!\langle P \rangle} & := & \{ x \} \cup \{ P \} \\
  \freenames{P|Q} & := & \freenames{P} \cup \freenames{Q} \\
  \freenames{\dropn{x}} & := & \{ x \}
\end{eqnarray*}

The bound names of a process, $\boundnames{P}$, are those names occurring in $P$
that are not free. For example, in $x?(y).0$, the name $x$ is free, while $y$ is bound.

\begin{mathpar}
  \inferrule* [lab=monoidal-laws] {} { P|Q \equiv Q|P \and P|0 \equiv P \and P|(Q|R) \equiv (P|Q)|R }
\end{mathpar}

\begin{mathpar}
  \inferrule* [lab=alpha-equivalence] {} { (x)P \equiv (y)P\{y/x\} \and y \not\in \freenames{P} }
\end{mathpar}

\begin{definition}
Then two processes, $P,Q$, are alpha-equivalent if $P = Q\{\vec{y}/\vec{x}\}$ for
some $\vec{x} \in \boundnames{Q},\vec{y} \in \boundnames{P}$, where $Q\{\vec{y}/\vec{x}\}$
denotes the capture-avoiding substitution of $\vec{y}$ for $\vec{x}$ in $Q$.
\end{definition}

\begin{definition}
  The {\em structural congruence} \cite{SangiorgiWalker} , $\equiv$,
  between processes is the least congruence containing
  alpha-equivalence, satisfying the abelian monoid laws
  (associativity, commutativity and $\pzero$ as identity) for parallel
  composition $|$ and for summation $+$.
\end{definition}

\subsection{Name equivalence}

We take name equivalence, written $\nameeq$, to be the smallest
equivalence relation generated by the following rules.

\begin{mathpar}
\inferrule*[lab=Quote-drop]
{ }
{ \quotep{@{x}} \nameeq x }

\inferrule*[lab=Struct-equiv]
{ P \scong Q }
{ \quotep{P} \nameeq \quotep{Q} }
\end{mathpar}

The astute reader will have noticed that the mutual recursion of names
and processes imposes a mutual recursion on alpha-equivalence and
structural equivalence via name-equivalence. Fortunately, all of this
works out pleasantly and we may calculate in the natural way, free of
concern. The reader interested in the details is referred to the
appendix \ref{appendix:rho_details}.

\subsection{Substitution}

We use $\Proc$ for the set of processes, $\QProc$ for the set of
names, and $\id{\{}\vec{y} / \vec{x} \id{\}}$ to denote partial maps,
$s : \QProc \rightarrow \QProc$. A map, $s$ lifts, uniquely, to a map
on process terms, $\widehat{s} : \Proc \rightarrow \Proc$ by the
following equations.

\begin{mathpar}
  (0) \psubstp{Q}{P} := 0 \\
  (R \juxtap S) \psubstp{Q}{P}
  :=    
  (R)\psubstp{Q}{P} \juxtap (S) \psubstp{Q}{P} \\
  (x?(y).R) \psubstp{Q}{P}    
  :=    
  (x)\substp{Q}{P} (z)\concat( (R \psubstn{z}{y}) \psubstp{Q}{P} ) \\
  (\lift{x}{R}) \psubstp{Q}{P}  
  :=
  \lift{(x)\substp{Q}{P}}{ R \psubstp{Q}{P} } \\
%   (\dropn{x})  \psubstp{Q}{P}       
%   := 
%   \left\{ 
%     \begin{array}{ccc} 
%       \dropn{\quotep{Q}} & & x \nameeq \quotep{P} \\
%       \dropn{x} & & otherwise \\
%     \end{array}
%   \right. 
  (\dropn{x})  \psubstp{Q}{P}       
  := 
  \left\{ 
    \begin{array}{ccc} 
      Q & & x \nameeq \quotep{P} \\
      \dropn{x} & & otherwise \\
    \end{array}
  \right.
\end{mathpar}
 

where

\begin{eqnarray}
  (x)\id{\{} \lpquote Q \rpquote / \lpquote P \rpquote \id{\}}            = 
  \left\{ 
    \begin{array}{ccc}
      \lpquote Q \rpquote & & x \nameeq \lpquote P \rpquote \\
      x & & otherwise \\
    \end{array}
  \right. \nonumber
\end{eqnarray}

and $z$ is chosen distinct from $\quotep{P}$, $\quotep{Q}$, the free
names in $Q$, and all the names in $R$. Our $\alpha$-equivalence will
be built in the standard way from this substitution.

\begin{remark}\label{rem:no_self_referential_names}
  One consequence of these definitions is that $\forall P. \quotep{P}
  \not\in \freenames{P}$.
\end{remark}

\subsection{ Dynamic quote: an example }

Anticipating something of what's to come, consider applying the
substitution, $\widehat{\id{\{}u / z \id{\}}}$, to the following pair
of processes, $\lift{w}{y!(z)}$ and $w[ \lpquote y!(z) \rpquote ]$.

\begin{eqnarray}
	\lift{w}{y!(z)}\widehat{\id{\{}u / z \id{\}}}
		& = &
		\lift{w}{y!(u)} \nonumber\\
	w[ \lpquote y!(z) \rpquote ] \widehat{ \id{\{}u / z \id{\}} }
		& = &
		w[ \lpquote y!(z) \rpquote ] \nonumber
\end{eqnarray}

Because the body of the process between quotes is impervious to
substitution, we get radically different answers. In fact, by
examining the first process in an input context,
e.g. $x?(z).\lift{w}{y!(z)}$, we see that the process under the lift
operator may be shaped by prefixed inputs binding a name inside it. In
this sense, the lift operator will be seen as a way to dynamically
construct processes before reifying them as names.

Finally equipped with these standard features we can present the
dynamics of the calculus.

\subsubsection{Operational semantics} 

Finally, we introduce the computational dynamics. What marks these
algebras as distinct from other more traditionally studied algebraic
structures, e.g. vector spaces or polynomial rings, is the manner in
which dynamics is captured. In traditional structures, dynamics is typically
expressed through morphisms between such structures, as in linear maps
between vector spaces or morphisms between rings. In algebras
associated with the semantics of computation, the dynamics is
expressed as part of the algebraic structure itself, through a
reduction reduction relation typically denoted by $\red$. Below, we
give a recursive presentation of this relation for the calculus used
in the encoding.

$\red \subseteq \pi \times \pi$
$\red : \pi \to \mathcal{P}(\pi)$

\begin{mathpar}
  \inferrule* [lab=Comm] { \textsf{match}( x_{src}, x_{trgt} ) } { x_{trgt}?(y)P \; | \; x_{src}!\langle {Q} \rangle \red P\{\quotep{Q}/y}\} }
  \and \\
  \inferrule* [lab=Par] {{P} \red {P}'} {{{P} | {Q}} \red {{P}' | {Q}}}
  \and
  \inferrule* [lab=Equiv]{{{P} \scong {P}'} \andalso {{P}' \red {Q}'} \andalso {{Q}' \scong {Q}}}{{P} \red {Q}}
\end{mathpar}

\begin{eqnarray*}
  match_{\equiv} (\quotep{P},\quotep{Q}) & := & P \equiv Q \\
  match_{\dagger}(\quotep{P},\quotep{Q}) & := & \forall R. P|Q \red^{*} R => R \red^{*} 0 \\
  match_{K}(\quotep{P},\quotep{Q}) & := & K \mbox{ for some context } K
\end{eqnarray*}

$u?(x)P | u!\langle Q \rangle \red P\{\quotep{Q}/x\}$

%We write $\wred$ for $\red^*$, and $P\red$ if $\exists Q $ such that $ P \red Q$.
We write $P\red$ if $\exists Q $ such that $ P \red Q$ and $P\not\red$, otherwise.

\section{Replication}

As mentioned before, it is known that replication (and hence
recursion) can be implemented in a higher-order process algebra
\cite{SangiorgiWalker}. As our first example of calculation with the
machinery thus far presented we give the construction explicitly in
the {\rhoc}.

\begin{eqnarray}
	D_{x} & := & \prefix{x}{y}{(\binpar{\outputp{x}{y}}{@{y}})} \nonumber\\
	\bangp_{x}{P} & := & \binpar{{x}!\langle{\binpar{D_{x}}{P}}\rangle}{D_{x}} \nonumber
\end{eqnarray}

\begin{eqnarray}
	\bangp_{x}{P} & & \nonumber\\
	=
	& {x}!\langle{(\prefix{x}{y}{(\outputp{x}{y} | @{y})) | P}}\rangle 
	      | \prefix{x}{y}{(\outputp{x}{y} | @{y})} & \nonumber\\
	\red
	& (\outputp{x}{y} | @{y})\substn{\quotep{(\prefix{x}{y}{(@{y} | \outputp{x}{y})) | P}}}{y} & \nonumber\\
	=
	& \outputp{x}{\quotep{(\prefix{x}{y}{(\outputp{x}{y} | @{y})) | P}}}
	  | {(\prefix{x}{y}{(\outputp{x}{y} | @{y})) | P}} & \nonumber\\
	\red
	& \ldots & \nonumber\\
	\red^*
	& P | P | \ldots & \nonumber
\end{eqnarray}

Of course, this encoding, as an implementation, runs away, unfolding
$\bangp{P}$ eagerly. A lazier and more implementable replication
operator, restricted to input-guarded processes, may be obtained as follows.

\begin{eqnarray}
\bangp{\prefix{u}{v}{P}} 
	:= 
	\binpar{\lift{x}{\prefix{u}{v}{(\binpar{D(x)}{P})}}}{D(x)} \nonumber
\end{eqnarray}

\begin{remark}
  Note that the lazier definition still does not deal with summation
  or mixed summation (i.e. sums over input and output). The reader is
  invited to construct definitions of replication that deal with these
  features. 

  Further, the definitions are parameterized in a name, $x$. Can you,
  gentle reader, make a definition that eliminates this parameter and
  guarantees no accidental interaction between the replication
  machinery and the process being replicated -- i.e. no accidental
  sharing of names used by the process to get its work done and the
  name(s) used by the replication to effect copying. This latter
  revision of the definition of replication is crucial to obtaining
  the expected identity $!!P \sim !P$.
\end{remark}

\begin{remark}\label{rem:paradoxical_combinator}
  The reader familiar with the lambda calculus will have noticed the
  similarity between $D$ and the paradoxical combinator.

  [Ed. note: the existence of this seems to suggest we have to be more
  restrictive on the set of processes and names we admit if we are to
  support no-cloning.]
\end{remark}

\subsubsection{Bisimulation}

The computational dynamics gives rise to another kind of equivalence,
the equivalence of computational behavior. As previously mentioned
this is typically captured \emph{via} some form of bisimulation.

% The notion we use in this paper is weak barbed bisimulation
% \cite{milner91polyadicpi}.

The notion we use in this paper is derived from weak barbed
bisimulation \cite{milner91polyadicpi}. 

\begin{definition}
An \emph{observation relation}, $\downarrow_{\mathcal N}$, over a set
of names, $\mathcal N$, is the smallest relation satisfying the rules
below.

\infrule[Out-barb]{y \in {\mathcal N}, \; x \nameeq y}
		  {\outputp{x}{v} \downarrow_{\mathcal N} x}
\infrule[Par-barb]{\mbox{$P\downarrow_{\mathcal N} x$ or $Q\downarrow_{\mathcal N} x$}}
		  {\binpar{P}{Q} \downarrow_{\mathcal N} x}

We write $P \Downarrow_{\mathcal N} x$ if there is $Q$ such that 
$P \wred Q$ and $Q \downarrow_{\mathcal N} x$.
\end{definition}

\begin{definition}
%\label{def.bbisim}
An  ${\mathcal N}$-\emph{barbed bisimulation} over a set of names, ${\mathcal N}$, is a symmetric binary relation 
${\mathcal S}_{\mathcal N}$ between agents such that $P\rel{S}_{\mathcal N}Q$ implies:
\begin{enumerate}
\item If $P \red P'$ then $Q \wred Q'$ and $P'\rel{S}_{\mathcal N} Q'$.
\item If $P\downarrow_{\mathcal N} x$, then $Q\Downarrow_{\mathcal N} x$.
\end{enumerate}
$P$ is ${\mathcal N}$-barbed bisimilar to $Q$, written
$P \wbbisim_{\mathcal N} Q$, if $P \rel{S}_{\mathcal N} Q$ for some ${\mathcal N}$-barbed bisimulation ${\mathcal S}_{\mathcal N}$.
\end{definition}

$\mathcal{R} \subseteq \pi \times \pi$

$P \mathcal{R} Q => \forall P'. P \red P' \Rightarrow \exists Q'. Q \red Q', P' \mathcal{R} Q'$

$P \vdash x \Rightarrow Q \vdash x$

\begin{mathpar}
  \inferrule*[lab=Out-barb]{x \nameeq y}{{y}!\langle{Q}\rangle \vdash x}
  \and
  \inferrule*[lab=Par-barb]{\mbox{$P\vdash x$ or $Q\vdash x$}}{\binpar{P}{Q} \vdash x}
\end{mathpar}

\subsubsection{Contexts}

One of the principle advantages of computational calculi like the
$\pi$-calculus is a well-defined notion of context,
contextual-equivalence and a correlation between
contextual-equivalence and notions of bisimulation. The notion of
context allows the decomposition of a process into (sub-)process and
its syntactic environment, its context. Thus, a context may be
thought of as a process with a ``hole'' (written $\Box$) in it. The
application of a context $M$ to a process $P$, written $M[P]$, is
tantamount to filling the hole in $M$ with $P$. In this paper we do
not need the full weight of this theory, but do make use of the notion
of context in the proof the main theorem. 

\begin{mathpar}
  \inferrule* [lab=summation] {} {{M_{M},M_{N}} \bc \Box \;|\; x.M_{A} \;|\; M_{M}+M_{N}}
  \and
  \inferrule* [lab=agent] {} {{M_{A}} \bc (\vec{x})M_{P} \;| \; \clift{P_0,\ldots,M_{P},\ldots,P_N}}
  \and \\
  \inferrule* [lab=process] {} {{M_{P}} \bc M_{N} \;| \;P|M_{P} }
\end{mathpar} 

\begin{mathpar}
  \inferrule* [lab=sychronization] {} {M_{N} \bc \Box \;|\; x?M_{F} \;|\; x!M_{C}}
  \and
  \inferrule* [lab=abstraction] {} {{M_{F}} \bc (x)M_{P} }
  \and
  \inferrule* [lab=concretion] {} {{M_{C}} \bc \langle M_{P} \rangle }
  \and \\
  \inferrule* [lab=process] {} {{M_{P}} \bc M_{N} \;| \;P|M_{P} }
\end{mathpar}

\begin{definition}[contextual application] Given a context $M$, and
  process $P$, we define the \emph{contextual application}, $M[P] :=
  M\{P/\Box\}$. That is, the contextual application of M to P is the
  substitution of $P$ for $\Box$ in $M$.
\end{definition}

$\meaningof{-} : L \to \mathcal{P}(\pi)$

\begin{mathpar}
  \inferrule* [lab=collection] {} {\meaningof{true} = \pi, \and \meaningof{~E} = \pi \setminus \meaningof{E}, \and \meaningof{E_{1} \& E_{2}} = \meaningof{E_{1}} \cap \meaningof{E_{2}}}
\end{mathpar}

\begin{mathpar}
  \inferrule* [lab=structure] {} {\meaningof{0} = \{ P \in \pi | P \equiv 0 \}, \and \\ \meaningof{E_1 | E_2} = \{ P \in \pi | P \equiv P_{1} | P_{2}, P_{1} \in \meaningof{E_{1}}, P_{2} \in \meaningof{E_2}\} }
\end{mathpar}

\begin{mathpar}
 \inferrule* [lab=behavior] {} {\meaningof{\langle a?b \rangle E} = \{ P \in \pi | P \equiv Q | u?(y)P', \\ \and \\\\ \and \\ \;\;\; u \in \meaningof{a}, \forall z.P'\{z/y\} \in \meaningof{E\{z/b\}}\}, \and \\ \meaningof{a!E} = \{ P \in \pi | P \equiv Q | x!\langle P' \rangle, x \in \meaningof{a} P' \in \meaningof{E}\} }
\end{mathpar}

\begin{mathpar}
 \inferrule* [lab=nominal] {} {\meaningof{\quotep{E}} = \{ \quotep{P} \in \quotep{\pi} | P \in \meaningof{E} \}, \and \meaningof{\quotep{P}} = \{ \quotep{Q} \in \quotep{\pi} | P \equiv Q \} \and \\ \meaningof{@\quotep{E}} = \{ P \in \pi | P \equiv @x, x \in \meaningof{E} \}}
\end{mathpar}

\begin{eqnarray*}
  \\
  \meaningof{-} : TS \to ST
\end{eqnarray*}

\begin{eqnarray*}
  \\
  L : TS \to ST
\end{eqnarray*}

\begin{eqnarray*}
  \\
  P \models E \iff P \in \meaningof{E}
\end{eqnarray*}

\begin{eqnarray*}
  P \approx_{L} Q \iff \forall E \in L. P \models E \iff Q \models E
\end{eqnarray*}

\begin{eqnarray*}
  P \approx_{K} Q
\end{eqnarray*}

\begin{eqnarray*}
  P \approx Q
\end{eqnarray*}

$\approx_{K} = \approx = \approx_{L}$

\subsubsection{Contextual duality}

Note that contexts extend the quotation operation to a family of
operations from processes to names. Given a context, $M$, we can
define a \emph{nominal context}, $\quotep{M}$ by $\quotep{M}[P] :=
\quotep{M[P]}$. To foreshadow what is to come we observe that these
operations enjoy a duality with processes very much like the duality
between vectors and maps from vectors to scalars.

Further, because the calculus is essentially higher-order, we have a
correspondence between contexts and processes. More specifically,
given a name $x$ and a context $M$ we can construct $M^{*}_{x}$ such
that 

\begin{mathpar}
  M^{*}_{x} | \lift{x}{P} \red M[P]
\end{mathpar}

namely,

\begin{mathpar}
  M^{*}_{x} := x?(u).M[\dropn{u}]
\end{mathpar}

The dependence of $M^{*}_{x}$ on a name makes it an abstraction, 

\begin{mathpar}
  M^{*} := (x)x?(u).M[\dropn{u}]
\end{mathpar}

\subsection{Additional notation}

It will sometimes be convenient to denote the process a name
quotes. We already have the notation $x = \quotep{P}$, but it will be
convenient to introduce an alternate notation, $\procn{x}$, when we
want to emphasize the connection to the use of the name. Note that, by
virtue of name equivalence, $\quotep{\procn{x}} \nameeq x$; so, the
notation is consistent with previous definitions.

Further, because names have structure it is possible to effect
substitutions on the basis of that structure. This means we need to
upgrade our notation for substitutions, which we accomplish by
adapting comprehension notation. Thus,

\begin{mathpar}
  P\{ y / x : x \in S \}
\end{mathpar}

is interpreted to mean the process derived from P by replacing (in a
capture-avoiding manner) each occurrence of $x$ in $S$ by $y$. For example,

\begin{mathpar}
  P\{ \quotep{\procn{x}|\procn{x}} / x : x \in \freenames{P} \}
\end{mathpar}

will replace each (occurrence) of a free name $x$ in $P$ by
$\quotep{\procn{x}|\procn{x}}$.

Also, we will avail ourselves of the notation $x^{L}$ and $x^{R}$ to
denote injections of a name into disjoint copies of the name
space. There are numerous ways to accomplish this. One example can be
found in \cite{MeredithR05}. This notation overloads to vectors of
names: $\vec{x}^{\pi} := (x_{i}^{\pi} \; : \; 0 \leq i < |\vec{x}| )$ where $\pi \in \{L,R\}$.

We also use $P^{\Box} := P|\Box$.

In \cite{MeredithR05} an interpretation of the new operator is
given. It turns out that there are several possible interpretations
all enjoying the requisite algebraic properties of the operator (see
\cite{milner91polyadicpi}). We will therefore make liberal use of
$(\nu\; \vec{x})P$.

% subsection the_syntax_and_semantics_of_the_notation_system (end)   

\input{qm2pi.qmops} 

\input{qm2pi.sterngerlach} 

\input{qm2pi.metric} 

% section concurrent_process_calculi (end)

%\input{qm2pi.proofsketch}

% section proof sketch (end)

%\input{qm2pi.slviaknots} 

% section spatial logic via knots (end)

\input{qm2pi.conclusion}

% section conclusion (end)

%\input{qm2pi.dtcodes} 

% section wiring algorithm (end)

\input{qm2pi.ack} 

% section acknowledgments (end)

\newpage


\bibliographystyle{plain}   
\bibliography{../../biblios/main.bib}

\input{qm2pi.rhodetails}

\end{document}

 

%\documentclass[12pt]{llncs}
%\documentclass{jktr}

\usepackage[pdftex]{hyperref}                   
\usepackage {listings}
\usepackage {mathpartir}
\usepackage{bcprules}
%\usepackage{listings}
                       
\usepackage{graphicx} 
%\usepackage[margins=2.5cm,nohead,nofoot]{geometry}
%\usepackage{geometry}
\usepackage{amsfonts}
\usepackage{amstext}
\usepackage{latexsym}
\usepackage{amssymb}
\usepackage{color}


%\include{myPreamble}
\include{qm2pi.local} 

%\ifpdf
%\usepackage[pdftex]{graphicx}
%\else
%\usepackage{graphicx}
%\fi

 % \ifpdf
%  \usepackage{pdfsync}
%  \if


%\title{Brief Article}
%\author{David F. Snyder}
%\author{L.G. Meredith}

%\address{Dept. of Math., Texas State University--San Marcos, San Marcos, TX 78666}
       
\pagestyle{empty}


\begin{document}

\lstset{language=[Objective]Caml,frame=shadowbox}

\input{qm2pi.front}

% section front matter (end)

\input{qm2pi.intro} 
 
% section introduction (end)

% \input{qm2pi.knotations} 

% section notation (end)

\input{qm2pi.process.calculi} 

% section concurrent_process_calculi_and_spatial_logics_ (end)
    
%\input{qm2pi.knots2pi} 

%\input{qm2pi.trefoil} 

%\input{qm2pi.mainthm} 

% subsection basic_interpretation (end)

%\input{qm2pi.rho.presentation} 
\subsection{The syntax and semantics of the notation system}\label{sub:the_syntax_and_semantics_of_the_notation_system} % (fold)

We now summarize a technical presentation of the calculus that
embodies our theory of dynamics. The typical presentation of such a
calculus follows the style of giving generators and relations on
them. The grammar, below, describing term constructors, freely
generates the set of processes, $\Proc$. This set is then quotiented
by a relation known as structural congruence and it is over this set
that the notion of dynamics is expressed. This presentation is
essentially that of \cite{MeredithR05} with the addition of
polyadicity and summation. For readability we have relegated some of
the technical subtleties to an appendix.

\subsubsection{Process grammar}\label{subsub:process_grammar}

\begin{mathpar}
  \inferrule* [lab=synchronization] {} {{M} \bc \pzero \;|\; x?F \;|\; x!C }
  \and
  \inferrule* [lab=abstraction] {} {{F} \bc (x)P}
  \and
  \inferrule* [lab=concretion] {} {{C} \bc \langle Q \rangle}
  \and
  \inferrule* [lab=process] {} {{P,Q} \bc M \;| \;P|Q \;|\; @{x}}
  \and
  \inferrule* [lab=name] {} {{x} \bc \quotep{P}}
\end{mathpar} 

Note that $\vec{x}$ (resp. $\vec{P}$) denotes a vector of names
(resp. processes) of length $|\vec{x}|$ (resp. $|\vec{P}|$). We adopt
the following useful abbreviations.

\begin{mathpar}
   x?(\vec{y}).P := x.(\vec{y})P \and  x\clift{\vec{P}} := x.\clift{\vec{P}}
   \and x!(y) := \lift{x}{\dropn{y}}
   \and \Pi_{i=0}^{n-1}P_i := P_0 | \ldots | P_{n-1}
\end{mathpar}

\subsubsection{Structural congruence}

\paragraph{Free and bound names and alpha-equivalence.} At the
core of structural equivalence is alpha-equivalence which identifies
process that are the same up to a change of variable. Formally, we
recognize the distinction between free and bound names. The free names
of a process, $\freenames{P}$, may be calculated recursively as
follows:

\begin{mathpar}
\freenames{\pzero} := \emptyset
  \and \\
  \freenames{x?(y).P} := \{ x \} \cup (\freenames{P} \setminus \{ y \})
  \and 
  \freenames{x!\langle P \rangle} := \{ x \} \cup \{ P \} 
  \and \\
  \freenames{P|Q} := \freenames{P} \cup \freenames{Q}
  \and \\
  \freenames{@{x}} := \{ x \}
\end{mathpar}

$\pi$
$\quotep{\pi}$

$\freenames{-} : \pi \to \mathcal{P}(\quotep{\pi})$

\begin{eqnarray*}
  \freenames{\pzero} & := & \emptyset \\
  \freenames{x?(y).P} & := & \{ x \} \cup (\freenames{P} \setminus \{ y \}) \\
  \freenames{x!\langle P \rangle} & := & \{ x \} \cup \{ P \} \\
  \freenames{P|Q} & := & \freenames{P} \cup \freenames{Q} \\
  \freenames{\dropn{x}} & := & \{ x \}
\end{eqnarray*}

The bound names of a process, $\boundnames{P}$, are those names occurring in $P$
that are not free. For example, in $x?(y).0$, the name $x$ is free, while $y$ is bound.

\begin{mathpar}
  \inferrule* [lab=monoidal-laws] {} { P|Q \equiv Q|P \and P|0 \equiv P \and P|(Q|R) \equiv (P|Q)|R }
\end{mathpar}

\begin{mathpar}
  \inferrule* [lab=alpha-equivalence] {} { (x)P \equiv (y)P\{y/x\} \and y \not\in \freenames{P} }
\end{mathpar}

\begin{definition}
Then two processes, $P,Q$, are alpha-equivalent if $P = Q\{\vec{y}/\vec{x}\}$ for
some $\vec{x} \in \boundnames{Q},\vec{y} \in \boundnames{P}$, where $Q\{\vec{y}/\vec{x}\}$
denotes the capture-avoiding substitution of $\vec{y}$ for $\vec{x}$ in $Q$.
\end{definition}

\begin{definition}
  The {\em structural congruence} \cite{SangiorgiWalker} , $\equiv$,
  between processes is the least congruence containing
  alpha-equivalence, satisfying the abelian monoid laws
  (associativity, commutativity and $\pzero$ as identity) for parallel
  composition $|$ and for summation $+$.
\end{definition}

\subsection{Name equivalence}

We take name equivalence, written $\nameeq$, to be the smallest
equivalence relation generated by the following rules.

\begin{mathpar}
\inferrule*[lab=Quote-drop]
{ }
{ \quotep{@{x}} \nameeq x }

\inferrule*[lab=Struct-equiv]
{ P \scong Q }
{ \quotep{P} \nameeq \quotep{Q} }
\end{mathpar}

The astute reader will have noticed that the mutual recursion of names
and processes imposes a mutual recursion on alpha-equivalence and
structural equivalence via name-equivalence. Fortunately, all of this
works out pleasantly and we may calculate in the natural way, free of
concern. The reader interested in the details is referred to the
appendix \ref{appendix:rho_details}.

\subsection{Substitution}

We use $\Proc$ for the set of processes, $\QProc$ for the set of
names, and $\id{\{}\vec{y} / \vec{x} \id{\}}$ to denote partial maps,
$s : \QProc \rightarrow \QProc$. A map, $s$ lifts, uniquely, to a map
on process terms, $\widehat{s} : \Proc \rightarrow \Proc$ by the
following equations.

\begin{mathpar}
  (0) \psubstp{Q}{P} := 0 \\
  (R \juxtap S) \psubstp{Q}{P}
  :=    
  (R)\psubstp{Q}{P} \juxtap (S) \psubstp{Q}{P} \\
  (x?(y).R) \psubstp{Q}{P}    
  :=    
  (x)\substp{Q}{P} (z)\concat( (R \psubstn{z}{y}) \psubstp{Q}{P} ) \\
  (\lift{x}{R}) \psubstp{Q}{P}  
  :=
  \lift{(x)\substp{Q}{P}}{ R \psubstp{Q}{P} } \\
%   (\dropn{x})  \psubstp{Q}{P}       
%   := 
%   \left\{ 
%     \begin{array}{ccc} 
%       \dropn{\quotep{Q}} & & x \nameeq \quotep{P} \\
%       \dropn{x} & & otherwise \\
%     \end{array}
%   \right. 
  (\dropn{x})  \psubstp{Q}{P}       
  := 
  \left\{ 
    \begin{array}{ccc} 
      Q & & x \nameeq \quotep{P} \\
      \dropn{x} & & otherwise \\
    \end{array}
  \right.
\end{mathpar}
 

where

\begin{eqnarray}
  (x)\id{\{} \lpquote Q \rpquote / \lpquote P \rpquote \id{\}}            = 
  \left\{ 
    \begin{array}{ccc}
      \lpquote Q \rpquote & & x \nameeq \lpquote P \rpquote \\
      x & & otherwise \\
    \end{array}
  \right. \nonumber
\end{eqnarray}

and $z$ is chosen distinct from $\quotep{P}$, $\quotep{Q}$, the free
names in $Q$, and all the names in $R$. Our $\alpha$-equivalence will
be built in the standard way from this substitution.

\begin{remark}\label{rem:no_self_referential_names}
  One consequence of these definitions is that $\forall P. \quotep{P}
  \not\in \freenames{P}$.
\end{remark}

\subsection{ Dynamic quote: an example }

Anticipating something of what's to come, consider applying the
substitution, $\widehat{\id{\{}u / z \id{\}}}$, to the following pair
of processes, $\lift{w}{y!(z)}$ and $w[ \lpquote y!(z) \rpquote ]$.

\begin{eqnarray}
	\lift{w}{y!(z)}\widehat{\id{\{}u / z \id{\}}}
		& = &
		\lift{w}{y!(u)} \nonumber\\
	w[ \lpquote y!(z) \rpquote ] \widehat{ \id{\{}u / z \id{\}} }
		& = &
		w[ \lpquote y!(z) \rpquote ] \nonumber
\end{eqnarray}

Because the body of the process between quotes is impervious to
substitution, we get radically different answers. In fact, by
examining the first process in an input context,
e.g. $x?(z).\lift{w}{y!(z)}$, we see that the process under the lift
operator may be shaped by prefixed inputs binding a name inside it. In
this sense, the lift operator will be seen as a way to dynamically
construct processes before reifying them as names.

Finally equipped with these standard features we can present the
dynamics of the calculus.

\subsubsection{Operational semantics} 

Finally, we introduce the computational dynamics. What marks these
algebras as distinct from other more traditionally studied algebraic
structures, e.g. vector spaces or polynomial rings, is the manner in
which dynamics is captured. In traditional structures, dynamics is typically
expressed through morphisms between such structures, as in linear maps
between vector spaces or morphisms between rings. In algebras
associated with the semantics of computation, the dynamics is
expressed as part of the algebraic structure itself, through a
reduction reduction relation typically denoted by $\red$. Below, we
give a recursive presentation of this relation for the calculus used
in the encoding.

$\red \subseteq \pi \times \pi$
$\red : \pi \to \mathcal{P}(\pi)$

\begin{mathpar}
  \inferrule* [lab=Comm] { \textsf{match}( x_{src}, x_{trgt} ) } { x_{trgt}?(y)P \; | \; x_{src}!\langle {Q} \rangle \red P\{\quotep{Q}/y}\} }
  \and \\
  \inferrule* [lab=Par] {{P} \red {P}'} {{{P} | {Q}} \red {{P}' | {Q}}}
  \and
  \inferrule* [lab=Equiv]{{{P} \scong {P}'} \andalso {{P}' \red {Q}'} \andalso {{Q}' \scong {Q}}}{{P} \red {Q}}
\end{mathpar}

\begin{eqnarray*}
  match_{\equiv} (\quotep{P},\quotep{Q}) & := & P \equiv Q \\
  match_{\dagger}(\quotep{P},\quotep{Q}) & := & \forall R. P|Q \red^{*} R => R \red^{*} 0 \\
  match_{K}(\quotep{P},\quotep{Q}) & := & K \mbox{ for some context } K
\end{eqnarray*}

$u?(x)P | u!\langle Q \rangle \red P\{\quotep{Q}/x\}$

%We write $\wred$ for $\red^*$, and $P\red$ if $\exists Q $ such that $ P \red Q$.
We write $P\red$ if $\exists Q $ such that $ P \red Q$ and $P\not\red$, otherwise.

\section{Replication}

As mentioned before, it is known that replication (and hence
recursion) can be implemented in a higher-order process algebra
\cite{SangiorgiWalker}. As our first example of calculation with the
machinery thus far presented we give the construction explicitly in
the {\rhoc}.

\begin{eqnarray}
	D_{x} & := & \prefix{x}{y}{(\binpar{\outputp{x}{y}}{@{y}})} \nonumber\\
	\bangp_{x}{P} & := & \binpar{{x}!\langle{\binpar{D_{x}}{P}}\rangle}{D_{x}} \nonumber
\end{eqnarray}

\begin{eqnarray}
	\bangp_{x}{P} & & \nonumber\\
	=
	& {x}!\langle{(\prefix{x}{y}{(\outputp{x}{y} | @{y})) | P}}\rangle 
	      | \prefix{x}{y}{(\outputp{x}{y} | @{y})} & \nonumber\\
	\red
	& (\outputp{x}{y} | @{y})\substn{\quotep{(\prefix{x}{y}{(@{y} | \outputp{x}{y})) | P}}}{y} & \nonumber\\
	=
	& \outputp{x}{\quotep{(\prefix{x}{y}{(\outputp{x}{y} | @{y})) | P}}}
	  | {(\prefix{x}{y}{(\outputp{x}{y} | @{y})) | P}} & \nonumber\\
	\red
	& \ldots & \nonumber\\
	\red^*
	& P | P | \ldots & \nonumber
\end{eqnarray}

Of course, this encoding, as an implementation, runs away, unfolding
$\bangp{P}$ eagerly. A lazier and more implementable replication
operator, restricted to input-guarded processes, may be obtained as follows.

\begin{eqnarray}
\bangp{\prefix{u}{v}{P}} 
	:= 
	\binpar{\lift{x}{\prefix{u}{v}{(\binpar{D(x)}{P})}}}{D(x)} \nonumber
\end{eqnarray}

\begin{remark}
  Note that the lazier definition still does not deal with summation
  or mixed summation (i.e. sums over input and output). The reader is
  invited to construct definitions of replication that deal with these
  features. 

  Further, the definitions are parameterized in a name, $x$. Can you,
  gentle reader, make a definition that eliminates this parameter and
  guarantees no accidental interaction between the replication
  machinery and the process being replicated -- i.e. no accidental
  sharing of names used by the process to get its work done and the
  name(s) used by the replication to effect copying. This latter
  revision of the definition of replication is crucial to obtaining
  the expected identity $!!P \sim !P$.
\end{remark}

\begin{remark}\label{rem:paradoxical_combinator}
  The reader familiar with the lambda calculus will have noticed the
  similarity between $D$ and the paradoxical combinator.

  [Ed. note: the existence of this seems to suggest we have to be more
  restrictive on the set of processes and names we admit if we are to
  support no-cloning.]
\end{remark}

\subsubsection{Bisimulation}

The computational dynamics gives rise to another kind of equivalence,
the equivalence of computational behavior. As previously mentioned
this is typically captured \emph{via} some form of bisimulation.

% The notion we use in this paper is weak barbed bisimulation
% \cite{milner91polyadicpi}.

The notion we use in this paper is derived from weak barbed
bisimulation \cite{milner91polyadicpi}. 

\begin{definition}
An \emph{observation relation}, $\downarrow_{\mathcal N}$, over a set
of names, $\mathcal N$, is the smallest relation satisfying the rules
below.

\infrule[Out-barb]{y \in {\mathcal N}, \; x \nameeq y}
		  {\outputp{x}{v} \downarrow_{\mathcal N} x}
\infrule[Par-barb]{\mbox{$P\downarrow_{\mathcal N} x$ or $Q\downarrow_{\mathcal N} x$}}
		  {\binpar{P}{Q} \downarrow_{\mathcal N} x}

We write $P \Downarrow_{\mathcal N} x$ if there is $Q$ such that 
$P \wred Q$ and $Q \downarrow_{\mathcal N} x$.
\end{definition}

\begin{definition}
%\label{def.bbisim}
An  ${\mathcal N}$-\emph{barbed bisimulation} over a set of names, ${\mathcal N}$, is a symmetric binary relation 
${\mathcal S}_{\mathcal N}$ between agents such that $P\rel{S}_{\mathcal N}Q$ implies:
\begin{enumerate}
\item If $P \red P'$ then $Q \wred Q'$ and $P'\rel{S}_{\mathcal N} Q'$.
\item If $P\downarrow_{\mathcal N} x$, then $Q\Downarrow_{\mathcal N} x$.
\end{enumerate}
$P$ is ${\mathcal N}$-barbed bisimilar to $Q$, written
$P \wbbisim_{\mathcal N} Q$, if $P \rel{S}_{\mathcal N} Q$ for some ${\mathcal N}$-barbed bisimulation ${\mathcal S}_{\mathcal N}$.
\end{definition}

$\mathcal{R} \subseteq \pi \times \pi$

$P \mathcal{R} Q => \forall P'. P \red P' \Rightarrow \exists Q'. Q \red Q', P' \mathcal{R} Q'$

$P \vdash x \Rightarrow Q \vdash x$

\begin{mathpar}
  \inferrule*[lab=Out-barb]{x \nameeq y}{{y}!\langle{Q}\rangle \vdash x}
  \and
  \inferrule*[lab=Par-barb]{\mbox{$P\vdash x$ or $Q\vdash x$}}{\binpar{P}{Q} \vdash x}
\end{mathpar}

\subsubsection{Contexts}

One of the principle advantages of computational calculi like the
$\pi$-calculus is a well-defined notion of context,
contextual-equivalence and a correlation between
contextual-equivalence and notions of bisimulation. The notion of
context allows the decomposition of a process into (sub-)process and
its syntactic environment, its context. Thus, a context may be
thought of as a process with a ``hole'' (written $\Box$) in it. The
application of a context $M$ to a process $P$, written $M[P]$, is
tantamount to filling the hole in $M$ with $P$. In this paper we do
not need the full weight of this theory, but do make use of the notion
of context in the proof the main theorem. 

\begin{mathpar}
  \inferrule* [lab=summation] {} {{M_{M},M_{N}} \bc \Box \;|\; x.M_{A} \;|\; M_{M}+M_{N}}
  \and
  \inferrule* [lab=agent] {} {{M_{A}} \bc (\vec{x})M_{P} \;| \; \clift{P_0,\ldots,M_{P},\ldots,P_N}}
  \and \\
  \inferrule* [lab=process] {} {{M_{P}} \bc M_{N} \;| \;P|M_{P} }
\end{mathpar} 

\begin{mathpar}
  \inferrule* [lab=sychronization] {} {M_{N} \bc \Box \;|\; x?M_{F} \;|\; x!M_{C}}
  \and
  \inferrule* [lab=abstraction] {} {{M_{F}} \bc (x)M_{P} }
  \and
  \inferrule* [lab=concretion] {} {{M_{C}} \bc \langle M_{P} \rangle }
  \and \\
  \inferrule* [lab=process] {} {{M_{P}} \bc M_{N} \;| \;P|M_{P} }
\end{mathpar}

\begin{definition}[contextual application] Given a context $M$, and
  process $P$, we define the \emph{contextual application}, $M[P] :=
  M\{P/\Box\}$. That is, the contextual application of M to P is the
  substitution of $P$ for $\Box$ in $M$.
\end{definition}

$\meaningof{-} : L \to \mathcal{P}(\pi)$

\begin{mathpar}
  \inferrule* [lab=collection] {} {\meaningof{true} = \pi, \and \meaningof{~E} = \pi \setminus \meaningof{E}, \and \meaningof{E_{1} \& E_{2}} = \meaningof{E_{1}} \cap \meaningof{E_{2}}}
\end{mathpar}

\begin{mathpar}
  \inferrule* [lab=structure] {} {\meaningof{0} = \{ P \in \pi | P \equiv 0 \}, \and \\ \meaningof{E_1 | E_2} = \{ P \in \pi | P \equiv P_{1} | P_{2}, P_{1} \in \meaningof{E_{1}}, P_{2} \in \meaningof{E_2}\} }
\end{mathpar}

\begin{mathpar}
 \inferrule* [lab=behavior] {} {\meaningof{\langle a?b \rangle E} = \{ P \in \pi | P \equiv Q | u?(y)P', \\ \and \\\\ \and \\ \;\;\; u \in \meaningof{a}, \forall z.P'\{z/y\} \in \meaningof{E\{z/b\}}\}, \and \\ \meaningof{a!E} = \{ P \in \pi | P \equiv Q | x!\langle P' \rangle, x \in \meaningof{a} P' \in \meaningof{E}\} }
\end{mathpar}

\begin{mathpar}
 \inferrule* [lab=nominal] {} {\meaningof{\quotep{E}} = \{ \quotep{P} \in \quotep{\pi} | P \in \meaningof{E} \}, \and \meaningof{\quotep{P}} = \{ \quotep{Q} \in \quotep{\pi} | P \equiv Q \} \and \\ \meaningof{@\quotep{E}} = \{ P \in \pi | P \equiv @x, x \in \meaningof{E} \}}
\end{mathpar}

\begin{eqnarray*}
  \\
  \meaningof{-} : TS \to ST
\end{eqnarray*}

\begin{eqnarray*}
  \\
  L : TS \to ST
\end{eqnarray*}

\begin{eqnarray*}
  \\
  P \models E \iff P \in \meaningof{E}
\end{eqnarray*}

\begin{eqnarray*}
  P \approx_{L} Q \iff \forall E \in L. P \models E \iff Q \models E
\end{eqnarray*}

\begin{eqnarray*}
  P \approx_{K} Q
\end{eqnarray*}

\begin{eqnarray*}
  P \approx Q
\end{eqnarray*}

$\approx_{K} = \approx = \approx_{L}$

\subsubsection{Contextual duality}

Note that contexts extend the quotation operation to a family of
operations from processes to names. Given a context, $M$, we can
define a \emph{nominal context}, $\quotep{M}$ by $\quotep{M}[P] :=
\quotep{M[P]}$. To foreshadow what is to come we observe that these
operations enjoy a duality with processes very much like the duality
between vectors and maps from vectors to scalars.

Further, because the calculus is essentially higher-order, we have a
correspondence between contexts and processes. More specifically,
given a name $x$ and a context $M$ we can construct $M^{*}_{x}$ such
that 

\begin{mathpar}
  M^{*}_{x} | \lift{x}{P} \red M[P]
\end{mathpar}

namely,

\begin{mathpar}
  M^{*}_{x} := x?(u).M[\dropn{u}]
\end{mathpar}

The dependence of $M^{*}_{x}$ on a name makes it an abstraction, 

\begin{mathpar}
  M^{*} := (x)x?(u).M[\dropn{u}]
\end{mathpar}

\subsection{Additional notation}

It will sometimes be convenient to denote the process a name
quotes. We already have the notation $x = \quotep{P}$, but it will be
convenient to introduce an alternate notation, $\procn{x}$, when we
want to emphasize the connection to the use of the name. Note that, by
virtue of name equivalence, $\quotep{\procn{x}} \nameeq x$; so, the
notation is consistent with previous definitions.

Further, because names have structure it is possible to effect
substitutions on the basis of that structure. This means we need to
upgrade our notation for substitutions, which we accomplish by
adapting comprehension notation. Thus,

\begin{mathpar}
  P\{ y / x : x \in S \}
\end{mathpar}

is interpreted to mean the process derived from P by replacing (in a
capture-avoiding manner) each occurrence of $x$ in $S$ by $y$. For example,

\begin{mathpar}
  P\{ \quotep{\procn{x}|\procn{x}} / x : x \in \freenames{P} \}
\end{mathpar}

will replace each (occurrence) of a free name $x$ in $P$ by
$\quotep{\procn{x}|\procn{x}}$.

Also, we will avail ourselves of the notation $x^{L}$ and $x^{R}$ to
denote injections of a name into disjoint copies of the name
space. There are numerous ways to accomplish this. One example can be
found in \cite{MeredithR05}. This notation overloads to vectors of
names: $\vec{x}^{\pi} := (x_{i}^{\pi} \; : \; 0 \leq i < |\vec{x}| )$ where $\pi \in \{L,R\}$.

We also use $P^{\Box} := P|\Box$.

In \cite{MeredithR05} an interpretation of the new operator is
given. It turns out that there are several possible interpretations
all enjoying the requisite algebraic properties of the operator (see
\cite{milner91polyadicpi}). We will therefore make liberal use of
$(\nu\; \vec{x})P$.

% subsection the_syntax_and_semantics_of_the_notation_system (end)   

\input{qm2pi.qmops} 

\input{qm2pi.sterngerlach} 

\input{qm2pi.metric} 

% section concurrent_process_calculi (end)

%\input{qm2pi.proofsketch}

% section proof sketch (end)

%\input{qm2pi.slviaknots} 

% section spatial logic via knots (end)

\input{qm2pi.conclusion}

% section conclusion (end)

%\input{qm2pi.dtcodes} 

% section wiring algorithm (end)

\input{qm2pi.ack} 

% section acknowledgments (end)

\newpage


\bibliographystyle{plain}   
\bibliography{../../biblios/main.bib}

\input{qm2pi.rhodetails}

\end{document}

 

% subsection basic_interpretation (end)

%\input{qm2pi.rho.presentation} 
\subsection{The syntax and semantics of the notation system}\label{sub:the_syntax_and_semantics_of_the_notation_system} % (fold)

We now summarize a technical presentation of the calculus that
embodies our theory of dynamics. The typical presentation of such a
calculus follows the style of giving generators and relations on
them. The grammar, below, describing term constructors, freely
generates the set of processes, $\Proc$. This set is then quotiented
by a relation known as structural congruence and it is over this set
that the notion of dynamics is expressed. This presentation is
essentially that of \cite{MeredithR05} with the addition of
polyadicity and summation. For readability we have relegated some of
the technical subtleties to an appendix.

\subsubsection{Process grammar}\label{subsub:process_grammar}

\begin{mathpar}
  \inferrule* [lab=synchronization] {} {{M} \bc \pzero \;|\; x?F \;|\; x!C }
  \and
  \inferrule* [lab=abstraction] {} {{F} \bc (x)P}
  \and
  \inferrule* [lab=concretion] {} {{C} \bc \langle Q \rangle}
  \and
  \inferrule* [lab=process] {} {{P,Q} \bc M \;| \;P|Q \;|\; @{x}}
  \and
  \inferrule* [lab=name] {} {{x} \bc \quotep{P}}
\end{mathpar} 

Note that $\vec{x}$ (resp. $\vec{P}$) denotes a vector of names
(resp. processes) of length $|\vec{x}|$ (resp. $|\vec{P}|$). We adopt
the following useful abbreviations.

\begin{mathpar}
   x?(\vec{y}).P := x.(\vec{y})P \and  x\clift{\vec{P}} := x.\clift{\vec{P}}
   \and x!(y) := \lift{x}{\dropn{y}}
   \and \Pi_{i=0}^{n-1}P_i := P_0 | \ldots | P_{n-1}
\end{mathpar}

\subsubsection{Structural congruence}

\paragraph{Free and bound names and alpha-equivalence.} At the
core of structural equivalence is alpha-equivalence which identifies
process that are the same up to a change of variable. Formally, we
recognize the distinction between free and bound names. The free names
of a process, $\freenames{P}$, may be calculated recursively as
follows:

\begin{mathpar}
\freenames{\pzero} := \emptyset
  \and \\
  \freenames{x?(y).P} := \{ x \} \cup (\freenames{P} \setminus \{ y \})
  \and 
  \freenames{x!\langle P \rangle} := \{ x \} \cup \{ P \} 
  \and \\
  \freenames{P|Q} := \freenames{P} \cup \freenames{Q}
  \and \\
  \freenames{@{x}} := \{ x \}
\end{mathpar}

$\pi$
$\quotep{\pi}$

$\freenames{-} : \pi \to \mathcal{P}(\quotep{\pi})$

\begin{eqnarray*}
  \freenames{\pzero} & := & \emptyset \\
  \freenames{x?(y).P} & := & \{ x \} \cup (\freenames{P} \setminus \{ y \}) \\
  \freenames{x!\langle P \rangle} & := & \{ x \} \cup \{ P \} \\
  \freenames{P|Q} & := & \freenames{P} \cup \freenames{Q} \\
  \freenames{\dropn{x}} & := & \{ x \}
\end{eqnarray*}

The bound names of a process, $\boundnames{P}$, are those names occurring in $P$
that are not free. For example, in $x?(y).0$, the name $x$ is free, while $y$ is bound.

\begin{mathpar}
  \inferrule* [lab=monoidal-laws] {} { P|Q \equiv Q|P \and P|0 \equiv P \and P|(Q|R) \equiv (P|Q)|R }
\end{mathpar}

\begin{mathpar}
  \inferrule* [lab=alpha-equivalence] {} { (x)P \equiv (y)P\{y/x\} \and y \not\in \freenames{P} }
\end{mathpar}

\begin{definition}
Then two processes, $P,Q$, are alpha-equivalent if $P = Q\{\vec{y}/\vec{x}\}$ for
some $\vec{x} \in \boundnames{Q},\vec{y} \in \boundnames{P}$, where $Q\{\vec{y}/\vec{x}\}$
denotes the capture-avoiding substitution of $\vec{y}$ for $\vec{x}$ in $Q$.
\end{definition}

\begin{definition}
  The {\em structural congruence} \cite{SangiorgiWalker} , $\equiv$,
  between processes is the least congruence containing
  alpha-equivalence, satisfying the abelian monoid laws
  (associativity, commutativity and $\pzero$ as identity) for parallel
  composition $|$ and for summation $+$.
\end{definition}

\subsection{Name equivalence}

We take name equivalence, written $\nameeq$, to be the smallest
equivalence relation generated by the following rules.

\begin{mathpar}
\inferrule*[lab=Quote-drop]
{ }
{ \quotep{@{x}} \nameeq x }

\inferrule*[lab=Struct-equiv]
{ P \scong Q }
{ \quotep{P} \nameeq \quotep{Q} }
\end{mathpar}

The astute reader will have noticed that the mutual recursion of names
and processes imposes a mutual recursion on alpha-equivalence and
structural equivalence via name-equivalence. Fortunately, all of this
works out pleasantly and we may calculate in the natural way, free of
concern. The reader interested in the details is referred to the
appendix \ref{appendix:rho_details}.

\subsection{Substitution}

We use $\Proc$ for the set of processes, $\QProc$ for the set of
names, and $\id{\{}\vec{y} / \vec{x} \id{\}}$ to denote partial maps,
$s : \QProc \rightarrow \QProc$. A map, $s$ lifts, uniquely, to a map
on process terms, $\widehat{s} : \Proc \rightarrow \Proc$ by the
following equations.

\begin{mathpar}
  (0) \psubstp{Q}{P} := 0 \\
  (R \juxtap S) \psubstp{Q}{P}
  :=    
  (R)\psubstp{Q}{P} \juxtap (S) \psubstp{Q}{P} \\
  (x?(y).R) \psubstp{Q}{P}    
  :=    
  (x)\substp{Q}{P} (z)\concat( (R \psubstn{z}{y}) \psubstp{Q}{P} ) \\
  (\lift{x}{R}) \psubstp{Q}{P}  
  :=
  \lift{(x)\substp{Q}{P}}{ R \psubstp{Q}{P} } \\
%   (\dropn{x})  \psubstp{Q}{P}       
%   := 
%   \left\{ 
%     \begin{array}{ccc} 
%       \dropn{\quotep{Q}} & & x \nameeq \quotep{P} \\
%       \dropn{x} & & otherwise \\
%     \end{array}
%   \right. 
  (\dropn{x})  \psubstp{Q}{P}       
  := 
  \left\{ 
    \begin{array}{ccc} 
      Q & & x \nameeq \quotep{P} \\
      \dropn{x} & & otherwise \\
    \end{array}
  \right.
\end{mathpar}
 

where

\begin{eqnarray}
  (x)\id{\{} \lpquote Q \rpquote / \lpquote P \rpquote \id{\}}            = 
  \left\{ 
    \begin{array}{ccc}
      \lpquote Q \rpquote & & x \nameeq \lpquote P \rpquote \\
      x & & otherwise \\
    \end{array}
  \right. \nonumber
\end{eqnarray}

and $z$ is chosen distinct from $\quotep{P}$, $\quotep{Q}$, the free
names in $Q$, and all the names in $R$. Our $\alpha$-equivalence will
be built in the standard way from this substitution.

\begin{remark}\label{rem:no_self_referential_names}
  One consequence of these definitions is that $\forall P. \quotep{P}
  \not\in \freenames{P}$.
\end{remark}

\subsection{ Dynamic quote: an example }

Anticipating something of what's to come, consider applying the
substitution, $\widehat{\id{\{}u / z \id{\}}}$, to the following pair
of processes, $\lift{w}{y!(z)}$ and $w[ \lpquote y!(z) \rpquote ]$.

\begin{eqnarray}
	\lift{w}{y!(z)}\widehat{\id{\{}u / z \id{\}}}
		& = &
		\lift{w}{y!(u)} \nonumber\\
	w[ \lpquote y!(z) \rpquote ] \widehat{ \id{\{}u / z \id{\}} }
		& = &
		w[ \lpquote y!(z) \rpquote ] \nonumber
\end{eqnarray}

Because the body of the process between quotes is impervious to
substitution, we get radically different answers. In fact, by
examining the first process in an input context,
e.g. $x?(z).\lift{w}{y!(z)}$, we see that the process under the lift
operator may be shaped by prefixed inputs binding a name inside it. In
this sense, the lift operator will be seen as a way to dynamically
construct processes before reifying them as names.

Finally equipped with these standard features we can present the
dynamics of the calculus.

\subsubsection{Operational semantics} 

Finally, we introduce the computational dynamics. What marks these
algebras as distinct from other more traditionally studied algebraic
structures, e.g. vector spaces or polynomial rings, is the manner in
which dynamics is captured. In traditional structures, dynamics is typically
expressed through morphisms between such structures, as in linear maps
between vector spaces or morphisms between rings. In algebras
associated with the semantics of computation, the dynamics is
expressed as part of the algebraic structure itself, through a
reduction reduction relation typically denoted by $\red$. Below, we
give a recursive presentation of this relation for the calculus used
in the encoding.

$\red \subseteq \pi \times \pi$
$\red : \pi \to \mathcal{P}(\pi)$

\begin{mathpar}
  \inferrule* [lab=Comm] { \textsf{match}( x_{src}, x_{trgt} ) } { x_{trgt}?(y)P \; | \; x_{src}!\langle {Q} \rangle \red P\{\quotep{Q}/y}\} }
  \and \\
  \inferrule* [lab=Par] {{P} \red {P}'} {{{P} | {Q}} \red {{P}' | {Q}}}
  \and
  \inferrule* [lab=Equiv]{{{P} \scong {P}'} \andalso {{P}' \red {Q}'} \andalso {{Q}' \scong {Q}}}{{P} \red {Q}}
\end{mathpar}

\begin{eqnarray*}
  match_{\equiv} (\quotep{P},\quotep{Q}) & := & P \equiv Q \\
  match_{\dagger}(\quotep{P},\quotep{Q}) & := & \forall R. P|Q \red^{*} R => R \red^{*} 0 \\
  match_{K}(\quotep{P},\quotep{Q}) & := & K \mbox{ for some context } K
\end{eqnarray*}

$u?(x)P | u!\langle Q \rangle \red P\{\quotep{Q}/x\}$

%We write $\wred$ for $\red^*$, and $P\red$ if $\exists Q $ such that $ P \red Q$.
We write $P\red$ if $\exists Q $ such that $ P \red Q$ and $P\not\red$, otherwise.

\section{Replication}

As mentioned before, it is known that replication (and hence
recursion) can be implemented in a higher-order process algebra
\cite{SangiorgiWalker}. As our first example of calculation with the
machinery thus far presented we give the construction explicitly in
the {\rhoc}.

\begin{eqnarray}
	D_{x} & := & \prefix{x}{y}{(\binpar{\outputp{x}{y}}{@{y}})} \nonumber\\
	\bangp_{x}{P} & := & \binpar{{x}!\langle{\binpar{D_{x}}{P}}\rangle}{D_{x}} \nonumber
\end{eqnarray}

\begin{eqnarray}
	\bangp_{x}{P} & & \nonumber\\
	=
	& {x}!\langle{(\prefix{x}{y}{(\outputp{x}{y} | @{y})) | P}}\rangle 
	      | \prefix{x}{y}{(\outputp{x}{y} | @{y})} & \nonumber\\
	\red
	& (\outputp{x}{y} | @{y})\substn{\quotep{(\prefix{x}{y}{(@{y} | \outputp{x}{y})) | P}}}{y} & \nonumber\\
	=
	& \outputp{x}{\quotep{(\prefix{x}{y}{(\outputp{x}{y} | @{y})) | P}}}
	  | {(\prefix{x}{y}{(\outputp{x}{y} | @{y})) | P}} & \nonumber\\
	\red
	& \ldots & \nonumber\\
	\red^*
	& P | P | \ldots & \nonumber
\end{eqnarray}

Of course, this encoding, as an implementation, runs away, unfolding
$\bangp{P}$ eagerly. A lazier and more implementable replication
operator, restricted to input-guarded processes, may be obtained as follows.

\begin{eqnarray}
\bangp{\prefix{u}{v}{P}} 
	:= 
	\binpar{\lift{x}{\prefix{u}{v}{(\binpar{D(x)}{P})}}}{D(x)} \nonumber
\end{eqnarray}

\begin{remark}
  Note that the lazier definition still does not deal with summation
  or mixed summation (i.e. sums over input and output). The reader is
  invited to construct definitions of replication that deal with these
  features. 

  Further, the definitions are parameterized in a name, $x$. Can you,
  gentle reader, make a definition that eliminates this parameter and
  guarantees no accidental interaction between the replication
  machinery and the process being replicated -- i.e. no accidental
  sharing of names used by the process to get its work done and the
  name(s) used by the replication to effect copying. This latter
  revision of the definition of replication is crucial to obtaining
  the expected identity $!!P \sim !P$.
\end{remark}

\begin{remark}\label{rem:paradoxical_combinator}
  The reader familiar with the lambda calculus will have noticed the
  similarity between $D$ and the paradoxical combinator.

  [Ed. note: the existence of this seems to suggest we have to be more
  restrictive on the set of processes and names we admit if we are to
  support no-cloning.]
\end{remark}

\subsubsection{Bisimulation}

The computational dynamics gives rise to another kind of equivalence,
the equivalence of computational behavior. As previously mentioned
this is typically captured \emph{via} some form of bisimulation.

% The notion we use in this paper is weak barbed bisimulation
% \cite{milner91polyadicpi}.

The notion we use in this paper is derived from weak barbed
bisimulation \cite{milner91polyadicpi}. 

\begin{definition}
An \emph{observation relation}, $\downarrow_{\mathcal N}$, over a set
of names, $\mathcal N$, is the smallest relation satisfying the rules
below.

\infrule[Out-barb]{y \in {\mathcal N}, \; x \nameeq y}
		  {\outputp{x}{v} \downarrow_{\mathcal N} x}
\infrule[Par-barb]{\mbox{$P\downarrow_{\mathcal N} x$ or $Q\downarrow_{\mathcal N} x$}}
		  {\binpar{P}{Q} \downarrow_{\mathcal N} x}

We write $P \Downarrow_{\mathcal N} x$ if there is $Q$ such that 
$P \wred Q$ and $Q \downarrow_{\mathcal N} x$.
\end{definition}

\begin{definition}
%\label{def.bbisim}
An  ${\mathcal N}$-\emph{barbed bisimulation} over a set of names, ${\mathcal N}$, is a symmetric binary relation 
${\mathcal S}_{\mathcal N}$ between agents such that $P\rel{S}_{\mathcal N}Q$ implies:
\begin{enumerate}
\item If $P \red P'$ then $Q \wred Q'$ and $P'\rel{S}_{\mathcal N} Q'$.
\item If $P\downarrow_{\mathcal N} x$, then $Q\Downarrow_{\mathcal N} x$.
\end{enumerate}
$P$ is ${\mathcal N}$-barbed bisimilar to $Q$, written
$P \wbbisim_{\mathcal N} Q$, if $P \rel{S}_{\mathcal N} Q$ for some ${\mathcal N}$-barbed bisimulation ${\mathcal S}_{\mathcal N}$.
\end{definition}

$\mathcal{R} \subseteq \pi \times \pi$

$P \mathcal{R} Q => \forall P'. P \red P' \Rightarrow \exists Q'. Q \red Q', P' \mathcal{R} Q'$

$P \vdash x \Rightarrow Q \vdash x$

\begin{mathpar}
  \inferrule*[lab=Out-barb]{x \nameeq y}{{y}!\langle{Q}\rangle \vdash x}
  \and
  \inferrule*[lab=Par-barb]{\mbox{$P\vdash x$ or $Q\vdash x$}}{\binpar{P}{Q} \vdash x}
\end{mathpar}

\subsubsection{Contexts}

One of the principle advantages of computational calculi like the
$\pi$-calculus is a well-defined notion of context,
contextual-equivalence and a correlation between
contextual-equivalence and notions of bisimulation. The notion of
context allows the decomposition of a process into (sub-)process and
its syntactic environment, its context. Thus, a context may be
thought of as a process with a ``hole'' (written $\Box$) in it. The
application of a context $M$ to a process $P$, written $M[P]$, is
tantamount to filling the hole in $M$ with $P$. In this paper we do
not need the full weight of this theory, but do make use of the notion
of context in the proof the main theorem. 

\begin{mathpar}
  \inferrule* [lab=summation] {} {{M_{M},M_{N}} \bc \Box \;|\; x.M_{A} \;|\; M_{M}+M_{N}}
  \and
  \inferrule* [lab=agent] {} {{M_{A}} \bc (\vec{x})M_{P} \;| \; \clift{P_0,\ldots,M_{P},\ldots,P_N}}
  \and \\
  \inferrule* [lab=process] {} {{M_{P}} \bc M_{N} \;| \;P|M_{P} }
\end{mathpar} 

\begin{mathpar}
  \inferrule* [lab=sychronization] {} {M_{N} \bc \Box \;|\; x?M_{F} \;|\; x!M_{C}}
  \and
  \inferrule* [lab=abstraction] {} {{M_{F}} \bc (x)M_{P} }
  \and
  \inferrule* [lab=concretion] {} {{M_{C}} \bc \langle M_{P} \rangle }
  \and \\
  \inferrule* [lab=process] {} {{M_{P}} \bc M_{N} \;| \;P|M_{P} }
\end{mathpar}

\begin{definition}[contextual application] Given a context $M$, and
  process $P$, we define the \emph{contextual application}, $M[P] :=
  M\{P/\Box\}$. That is, the contextual application of M to P is the
  substitution of $P$ for $\Box$ in $M$.
\end{definition}

$\meaningof{-} : L \to \mathcal{P}(\pi)$

\begin{mathpar}
  \inferrule* [lab=collection] {} {\meaningof{true} = \pi, \and \meaningof{~E} = \pi \setminus \meaningof{E}, \and \meaningof{E_{1} \& E_{2}} = \meaningof{E_{1}} \cap \meaningof{E_{2}}}
\end{mathpar}

\begin{mathpar}
  \inferrule* [lab=structure] {} {\meaningof{0} = \{ P \in \pi | P \equiv 0 \}, \and \\ \meaningof{E_1 | E_2} = \{ P \in \pi | P \equiv P_{1} | P_{2}, P_{1} \in \meaningof{E_{1}}, P_{2} \in \meaningof{E_2}\} }
\end{mathpar}

\begin{mathpar}
 \inferrule* [lab=behavior] {} {\meaningof{\langle a?b \rangle E} = \{ P \in \pi | P \equiv Q | u?(y)P', \\ \and \\\\ \and \\ \;\;\; u \in \meaningof{a}, \forall z.P'\{z/y\} \in \meaningof{E\{z/b\}}\}, \and \\ \meaningof{a!E} = \{ P \in \pi | P \equiv Q | x!\langle P' \rangle, x \in \meaningof{a} P' \in \meaningof{E}\} }
\end{mathpar}

\begin{mathpar}
 \inferrule* [lab=nominal] {} {\meaningof{\quotep{E}} = \{ \quotep{P} \in \quotep{\pi} | P \in \meaningof{E} \}, \and \meaningof{\quotep{P}} = \{ \quotep{Q} \in \quotep{\pi} | P \equiv Q \} \and \\ \meaningof{@\quotep{E}} = \{ P \in \pi | P \equiv @x, x \in \meaningof{E} \}}
\end{mathpar}

\begin{eqnarray*}
  \\
  \meaningof{-} : TS \to ST
\end{eqnarray*}

\begin{eqnarray*}
  \\
  L : TS \to ST
\end{eqnarray*}

\begin{eqnarray*}
  \\
  P \models E \iff P \in \meaningof{E}
\end{eqnarray*}

\begin{eqnarray*}
  P \approx_{L} Q \iff \forall E \in L. P \models E \iff Q \models E
\end{eqnarray*}

\begin{eqnarray*}
  P \approx_{K} Q
\end{eqnarray*}

\begin{eqnarray*}
  P \approx Q
\end{eqnarray*}

$\approx_{K} = \approx = \approx_{L}$

\subsubsection{Contextual duality}

Note that contexts extend the quotation operation to a family of
operations from processes to names. Given a context, $M$, we can
define a \emph{nominal context}, $\quotep{M}$ by $\quotep{M}[P] :=
\quotep{M[P]}$. To foreshadow what is to come we observe that these
operations enjoy a duality with processes very much like the duality
between vectors and maps from vectors to scalars.

Further, because the calculus is essentially higher-order, we have a
correspondence between contexts and processes. More specifically,
given a name $x$ and a context $M$ we can construct $M^{*}_{x}$ such
that 

\begin{mathpar}
  M^{*}_{x} | \lift{x}{P} \red M[P]
\end{mathpar}

namely,

\begin{mathpar}
  M^{*}_{x} := x?(u).M[\dropn{u}]
\end{mathpar}

The dependence of $M^{*}_{x}$ on a name makes it an abstraction, 

\begin{mathpar}
  M^{*} := (x)x?(u).M[\dropn{u}]
\end{mathpar}

\subsection{Additional notation}

It will sometimes be convenient to denote the process a name
quotes. We already have the notation $x = \quotep{P}$, but it will be
convenient to introduce an alternate notation, $\procn{x}$, when we
want to emphasize the connection to the use of the name. Note that, by
virtue of name equivalence, $\quotep{\procn{x}} \nameeq x$; so, the
notation is consistent with previous definitions.

Further, because names have structure it is possible to effect
substitutions on the basis of that structure. This means we need to
upgrade our notation for substitutions, which we accomplish by
adapting comprehension notation. Thus,

\begin{mathpar}
  P\{ y / x : x \in S \}
\end{mathpar}

is interpreted to mean the process derived from P by replacing (in a
capture-avoiding manner) each occurrence of $x$ in $S$ by $y$. For example,

\begin{mathpar}
  P\{ \quotep{\procn{x}|\procn{x}} / x : x \in \freenames{P} \}
\end{mathpar}

will replace each (occurrence) of a free name $x$ in $P$ by
$\quotep{\procn{x}|\procn{x}}$.

Also, we will avail ourselves of the notation $x^{L}$ and $x^{R}$ to
denote injections of a name into disjoint copies of the name
space. There are numerous ways to accomplish this. One example can be
found in \cite{MeredithR05}. This notation overloads to vectors of
names: $\vec{x}^{\pi} := (x_{i}^{\pi} \; : \; 0 \leq i < |\vec{x}| )$ where $\pi \in \{L,R\}$.

We also use $P^{\Box} := P|\Box$.

In \cite{MeredithR05} an interpretation of the new operator is
given. It turns out that there are several possible interpretations
all enjoying the requisite algebraic properties of the operator (see
\cite{milner91polyadicpi}). We will therefore make liberal use of
$(\nu\; \vec{x})P$.

% subsection the_syntax_and_semantics_of_the_notation_system (end)   

\section{Interpretation of QM}
\subsection{Supporting definitions}
\subsubsection{Multiplication}
\begin{mathpar}
  \quotep{Q} \cdot \quotep{R} := \quotep{Q|R}
  \and \\
  \quotep{Q} \cdot P := P\{ \quotep{Q|R} / \quotep{R} : \quotep{R} \in \freenames{P} \}
\end{mathpar}

\paragraph{Discussion}
The first line needs little explanation. The second line says that
each free name of the process is replaced with the multiplication of
that name by the scalar. Multiplication of a scalar (name) by a state
(process) results in a process all the names of which have been `moved
over' by parallel composition with the process the scalar
quotes. There is a subtlety that the bound names have to be
manipulated so that multiplied names aren't accidentally
captured. There are many ways to achieve this.

\begin{remark}\label{rem:multiplication_identities}
  The reader is invited to verify that for all $x,y,z \in \QProc$ and $P \in \Proc$
  \begin{mathpar}
    x \cdot \quotep{0} \equiv x 
    \and
    x \cdot y \equiv y \cdot x
    \and
    x \cdot (y \cdot z) \equiv (x \cdot y) \cdot z
    \and \\
    \quotep{0} \cdot P \equiv P
    \and \\
    x \cdot (y \cdot P) \equiv (x \cdot y) \cdot P
    \and \\
    x \cdot (P|Q) \equiv (x \cdot P) | (x \cdot Q)
    \and \\    
  \end{mathpar}
\end{remark}

\subsubsection{Tensor product}

We define a tensor product on processes by structural induction.

\paragraph{Tensor of sums} First note that all summations, including
$\pzero$ and sequence, can be written $\Sigma_{i} x_{i}.A_{i} +
\Sigma_{j} x_{j}.C_{j}$, where we have grouped input-guarded processes
together and output-guarded processes together.

Thus, we can define the tensor product of two summations, $N_{1}\otimes N_{2}$, where

\begin{mathpar}
  N_{1} := \Sigma_{i} x_{i}.A_{i} + \Sigma_{j} x_{j}.C_{j}
  \and
  N_{2} := \Sigma_{i'} y_{i'}.B_{i'} + \Sigma_{j'} y_{j'}.D_{j'} 
\end{mathpar}

as follows.

\begin{mathpar}
  \Sigma_{i} x_{i}.A_{i} + \Sigma_{j} x_{j}.C_{j} \otimes \Sigma_{i'}
  y_{i'}.B_{i'} + \Sigma_{j'} y_{j'}.D_{j'} 
  \and \\
  := \; \Sigma_{i} \Sigma_{i'} \quotep{\stackrel{\vee}{x_{i}}| \stackrel{\vee}{y_{i'}}}.(A_{i}\otimes B_{i'}) \; | \; \Sigma_{i'} \Sigma_{i} \quotep{\stackrel{\vee}{y_{i'}}|\stackrel{\vee}{x_{i}}}.(B_{i'}\otimes A_{i})
  \and
  \;\; | \;\; \Sigma_{j} \Sigma_{j'} \quotep{\stackrel{\vee}{x_{j}}|\stackrel{\vee}{y_{j'}}}.(A_{j}\otimes B_{j'}) \; | \; \Sigma_{j'} \Sigma_{j} \quotep{\stackrel{\vee}{y_{j'}}|\stackrel{\vee}{x_{j}}}.(B_{j'}\otimes A_{j})
\end{mathpar}

\begin{remark}
  Do we need to $x^{L}$ and $y^{R}$ for this construction as well?
\end{remark}

\paragraph{Tensor of parallel compositions} Next, we distribute tensor
over par.

\begin{mathpar}
  P_{1}|P_{2} \otimes Q_{1}|Q_{2} := (P_{1} \otimes Q_{1}) | (P_{1}
  \otimes Q_{2}) | (P_{2} \otimes Q_{1}) | (P_{2} \otimes Q_{2})
\end{mathpar}

\paragraph{Tensor with dropped names} We treat tensor of a
process with a dropped name as parallel composition.

\begin{mathpar}
  P \otimes \dropn{x} := P | \dropn{x}
\end{mathpar}

\paragraph{Tensor of agents}

Finally, we need to define tensor on agents. Note that the definition
of tensor on normal products only tensors inputs with inputs and
outputs with outputs. Thus, we only have to define the operation on
``homogeneous'' pairings.

\begin{mathpar}
  (\vec{x})P \otimes (\vec{y})Q
  \and \\
  := (x_{0}^{L}|y_{0}^{R},\ldots,x_{0}^{L}|y_{n}^{R},\ldots,x_{m}^{L}|y_{0}^{R},\ldots,x_{m}^{L}|y_{n}^R)(P\{ \vec{x}^{L}/\vec{x}\} \otimes Q \{ \vec{y}^{R}/\vec{y}\})
  \and \\
  \clift{\vec{P}} \otimes \clift{\vec{Q}}
  \and \\
  := \clift{P_{0}\otimes Q_{0},\ldots,P_{0}\otimes Q_{n},\ldots,P_{m}\otimes Q_{0},\ldots,P_{m}\otimes Q_{n}}
\end{mathpar}

\begin{remark}
  Observe that arities of tensored abstractions matches arities of
  tensored concretions if the original arities matched. Note also that
  the length of the arities corresponds to the increase in dimension
  we see in ordinary vector space tensor product.
\end{remark}

\begin{remark}
  Operationally, this definition distributes the tensor down to
  components ``linked'' by summation. Tensor over summation is
  intriguing in that it mixes names. Moreover, as a consequence of the
  way it mixes names we have the identities for all $x \in \QProc$ and
  $P,Q \in \Proc$

  \begin{mathpar}
    (x \cdot P) \otimes Q \equiv x \cdot (P \otimes Q) \equiv P \otimes (x \cdot Q)
    \and
    P \otimes \pzero \equiv P
  \end{mathpar}

  that the reader is invited to verify.
\end{remark}

\subsubsection{Annihilation}
\begin{mathpar}
  P^{\perp} := \{ Q | \forall R. P|Q \red^{*} R \Rightarrow R \red^{*} \pzero \}
  \and \\
  P^{\underline{\perp}} := \Sigma_{Q \in P^{\perp}} \quotep{Q}?(y).(\dropn{y}|Q) | \Sigma_{Q \in P^{\perp}} \quotep{Q}\clift{\Box}
\end{mathpar}

\paragraph{Discussion} The reader will note that $P^{\perp}$ is a
\emph{set} of processes, while $P^{\underline{\perp}}$ is a
\emph{context}. We call the set $P^{\perp}$ the \emph{annihilators} of
$P$. The parallel composition of a process in the annihilators of $P$
with $P$ will result in a process, the state space of which has all
paths eventually leading to $\pzero$. Execution may endure loops; but
under reasonable conditions of fairness (naturally guaranteed under
most notions of bisimulation) such a composite process cannot get
stuck in such a loop and will, eventually pop out and terminate.

The context $P^{\underline{\perp}}$ is ready and willing to ``take the
$P$ out of'' the process to which it is applied. It will effectively
transmit the code of the process to which it is applied to one of the
annihilators and run the process against it.

\subsubsection{Evaluation}
We fix $M$ a domain of fully abstract interpretation with an equality
coincident with bisimulation. We take $\meaningof{\cdot} : \Proc \to
M$ to be the map interpreting processes and $\nmeaningof{\cdot} : \M
\to Proc$ to be the map running the other way. Then we define

\begin{mathpar}
  \int P := \nmeaningof{\meaningof{P}}
\end{mathpar}

\paragraph{Discussion}
There are many fully abstract interpretations of Milner's
$\pi$-calculus. Any of them can be used as a basis for interpreting
the reflective calculus here. Equipped with such a domain it is
largely a matter of grinding through to check that the Yoneda
construction for the normalization-by-evaluation program can be
extended to this setting.

\begin{remark}
  The reader is invited to verify that $\int (P^{\underline{\perp}}[P]) = 0$.
\end{remark}

\subsection{Quantum mechanics}

Table \ref{tbl:core_qm_op_defns} gives the core operational definitions

\begin{table}[htp]\label{tbl:core_qm_op_defns}
  \center{
    \fbox{
      \begin{tabular}{c|c}
        quantum mechanics & process calculus \\
        \hline
        scalar & $x := \quotep{P}$ \\
        state vector & $\state{P} := P$ \\
        dual & $\state{P}^{*} := \event{P^{\underline{\perp}}} := \quotep{P^{\underline{\perp}}}[-]$ \\
        matrix & $ \Sigma_{\alpha} \state{P_{\alpha}}x_{\alpha}\event{Q_{\alpha}}$ \\
        vector addition & $\state{P} + \state{Q} := \state{P | Q}$ \\
        tensor product & $\state{P} \otimes \state{Q} := \state{P \otimes Q}$ \\
        inner product & $\innerprod{P}{Q} := \quotep{\int P^{\underline{\perp}}[Q]}$ \\
      \end{tabular}
    }
  }
  \caption{QM - operational definitions}
\end{table}

where

\begin{mathpar}
  \prmatrix{P}{Q} := \fprmatrix{P}{\quotep{\pzero}}{Q}
  \and
  \fprmatrix{P}{x}{Q} := (\state{P},x,\event{Q})
  \and
  (\fprmatrix{P}{x}{Q})(\state{R}) := x \cdot \innerprod{Q}{R} \cdot \state{P}
  \and
  (\fprmatrix{P}{x}{Q})(\event{R}) := x \cdot \innerprod{R}{P} \cdot \event{Q}
\end{mathpar}

\paragraph{Discussion}
As promised: vectors (aka states) are represented as processes; duals
as contextual duals; inner product definition should be compared with
standard inner product definition for ....

\begin{remark}
  Assuming $\int (P^{\underline{\perp}}[P]) = 0$, the reader is
  invited to verify that $(\fprmatrix{P}{x}{P})(\state{P}) = x \cdot \state{P}$.
\end{remark}

\begin{remark}
  The reader is invited to verify that $\innerprod{P}{Q}$ could
  equally well have been written $\quotep{\int \stackrel{\vee}{x}}$
  where $x = \event{P^{\underline{\perp}}}(Q)$.

  One of the motivations for this remark is that there is another way
  to factor these operations. We could package up evaluation in the dual:

  \begin{mathpar}
    \state{P}^{*} := \event{\int P^{\underline{\perp}}} := \quotep{\int P^{\underline{\perp}}}[-]
  \end{mathpar}

  and then have inner product defined by
  
  \begin{mathpar}
    \innerprod{P}{Q} := \event{P}(Q)
  \end{mathpar}

  Hopefully, experience with the calculations will provide guidance on
  the best factoring.
\end{remark}

\begin{remark}
  Assuming $\int (P^{\underline{\perp}}[P]) = 0$, the reader is
  invited to verify that $\forall P,Q. (\prmatrix{0}{Q})(\state{0}) =
  \state{0}$ and dually $(\prmatrix{P}{0})(\event{0}) = \event{0}$.
\end{remark}

\begin{remark}
  i'm a little worried that i don't (yet) have proper support for
  complex conjugacy. But, the observation above may give us a
  clue. According to Abramsky, it must be the case that the scalars
  are iso to the homset of the identity for the tensor -- which the
  observation above characterizes. 

  For now, we will simply bookmark the notion with $\overline{x}$.
\end{remark}

\subsubsection{Adjointness}

We need to give a definition of $(\cdot)^{\dagger}$ for matrices. The
obvious candidate definition is
\begin{mathpar}
(\Sigma_{\alpha}\fprmatrix{P_{\alpha}}{x_{\alpha}}{Q_{\alpha}})^{\dagger}
= \Sigma_{\alpha}\fprmatrix{(Q_{\alpha}^{\underline{\perp}})^{*}}{\overline{x}_{\alpha}}{P_{\alpha}^{\underline{\perp}}} 
\end{mathpar}

But, $(Q_{\alpha}^{\underline{\perp}})^{*}$ requires a name along
which to communicate the process to achieve the context application.

\subsubsection{Basis for a basis}
If processes label states and ``addition'' of states (a.k.a. vector
addition) is interpreted as parallel composition, what corresponds to
notions of linear independence and basis? Here, we recall that Yoshida
has developed a set of \emph{combinators} for an asynchronous verison
of Milner's $\pi$-calculus. These are a finite set of processes such
any process can be expressed as parallel composition of these
combinators together with liberal uses of the new operator and
replication. We can simply give a translation of these into the
present calculus and have reasonable expectation that the property
carries over. That is, that the resultant set allows to express all
processes via parallel composition. Note, however, that there is no
new operator or replication in this calculus. As a result, we expect
that the corresponding set is actually infinite. That is, we expect
that the space is actually infinite dimensional.

\begin{remark}
  The attentive reader may be a bit concerned. Certainly, the
  collection $S$, $K$ and $I$ is a finite set of
  combinators. Shouldn't we expect to see a finite set of combinators
  for an effectively equivalent system? i am very sympathetic to this
  critique and feel it warrants full attention. On the other hand, i
  also have in mind the following analogy. The natural numbers, as a
  monoid under addition, has exactly $1$ generator, while the natural
  numbers, as a monoid under multiplication, has countably many
  generators (the primes). We observe that the application of the
  lambda calculus is much less resource sensitive than the parallel
  composition of the $\pi$-calculus. Could it be the case that we have
  an analogy of the form
  
  \begin{mathpar}
    m + n : MN :: m*n : M|N
  \end{mathpar}

  giving a similar blow up in the set of ``primes''?  This is such a
  wonderful thought that, even if it's not true, i think it's worth
  writing down.
\end{remark}
 

\documentclass[12pt]{llncs}
%\documentclass{jktr}

\usepackage[pdftex]{hyperref}                   
\usepackage {listings}
\usepackage {mathpartir}
\usepackage{bcprules}
%\usepackage{listings}
                       
\usepackage{graphicx} 
%\usepackage[margins=2.5cm,nohead,nofoot]{geometry}
%\usepackage{geometry}
\usepackage{amsfonts}
\usepackage{amstext}
\usepackage{latexsym}
\usepackage{amssymb}
\usepackage{color}


%\include{myPreamble}
\include{qm2pi.local} 

%\ifpdf
%\usepackage[pdftex]{graphicx}
%\else
%\usepackage{graphicx}
%\fi

 % \ifpdf
%  \usepackage{pdfsync}
%  \if


%\title{Brief Article}
%\author{David F. Snyder}
%\author{L.G. Meredith}

%\address{Dept. of Math., Texas State University--San Marcos, San Marcos, TX 78666}
       
\pagestyle{empty}


\begin{document}

\lstset{language=[Objective]Caml,frame=shadowbox}

\input{qm2pi.front}

% section front matter (end)

\input{qm2pi.intro} 
 
% section introduction (end)

% \input{qm2pi.knotations} 

% section notation (end)

\input{qm2pi.process.calculi} 

% section concurrent_process_calculi_and_spatial_logics_ (end)
    
%\input{qm2pi.knots2pi} 

%\input{qm2pi.trefoil} 

%\input{qm2pi.mainthm} 

% subsection basic_interpretation (end)

%\input{qm2pi.rho.presentation} 
\subsection{The syntax and semantics of the notation system}\label{sub:the_syntax_and_semantics_of_the_notation_system} % (fold)

We now summarize a technical presentation of the calculus that
embodies our theory of dynamics. The typical presentation of such a
calculus follows the style of giving generators and relations on
them. The grammar, below, describing term constructors, freely
generates the set of processes, $\Proc$. This set is then quotiented
by a relation known as structural congruence and it is over this set
that the notion of dynamics is expressed. This presentation is
essentially that of \cite{MeredithR05} with the addition of
polyadicity and summation. For readability we have relegated some of
the technical subtleties to an appendix.

\subsubsection{Process grammar}\label{subsub:process_grammar}

\begin{mathpar}
  \inferrule* [lab=synchronization] {} {{M} \bc \pzero \;|\; x?F \;|\; x!C }
  \and
  \inferrule* [lab=abstraction] {} {{F} \bc (x)P}
  \and
  \inferrule* [lab=concretion] {} {{C} \bc \langle Q \rangle}
  \and
  \inferrule* [lab=process] {} {{P,Q} \bc M \;| \;P|Q \;|\; @{x}}
  \and
  \inferrule* [lab=name] {} {{x} \bc \quotep{P}}
\end{mathpar} 

Note that $\vec{x}$ (resp. $\vec{P}$) denotes a vector of names
(resp. processes) of length $|\vec{x}|$ (resp. $|\vec{P}|$). We adopt
the following useful abbreviations.

\begin{mathpar}
   x?(\vec{y}).P := x.(\vec{y})P \and  x\clift{\vec{P}} := x.\clift{\vec{P}}
   \and x!(y) := \lift{x}{\dropn{y}}
   \and \Pi_{i=0}^{n-1}P_i := P_0 | \ldots | P_{n-1}
\end{mathpar}

\subsubsection{Structural congruence}

\paragraph{Free and bound names and alpha-equivalence.} At the
core of structural equivalence is alpha-equivalence which identifies
process that are the same up to a change of variable. Formally, we
recognize the distinction between free and bound names. The free names
of a process, $\freenames{P}$, may be calculated recursively as
follows:

\begin{mathpar}
\freenames{\pzero} := \emptyset
  \and \\
  \freenames{x?(y).P} := \{ x \} \cup (\freenames{P} \setminus \{ y \})
  \and 
  \freenames{x!\langle P \rangle} := \{ x \} \cup \{ P \} 
  \and \\
  \freenames{P|Q} := \freenames{P} \cup \freenames{Q}
  \and \\
  \freenames{@{x}} := \{ x \}
\end{mathpar}

$\pi$
$\quotep{\pi}$

$\freenames{-} : \pi \to \mathcal{P}(\quotep{\pi})$

\begin{eqnarray*}
  \freenames{\pzero} & := & \emptyset \\
  \freenames{x?(y).P} & := & \{ x \} \cup (\freenames{P} \setminus \{ y \}) \\
  \freenames{x!\langle P \rangle} & := & \{ x \} \cup \{ P \} \\
  \freenames{P|Q} & := & \freenames{P} \cup \freenames{Q} \\
  \freenames{\dropn{x}} & := & \{ x \}
\end{eqnarray*}

The bound names of a process, $\boundnames{P}$, are those names occurring in $P$
that are not free. For example, in $x?(y).0$, the name $x$ is free, while $y$ is bound.

\begin{mathpar}
  \inferrule* [lab=monoidal-laws] {} { P|Q \equiv Q|P \and P|0 \equiv P \and P|(Q|R) \equiv (P|Q)|R }
\end{mathpar}

\begin{mathpar}
  \inferrule* [lab=alpha-equivalence] {} { (x)P \equiv (y)P\{y/x\} \and y \not\in \freenames{P} }
\end{mathpar}

\begin{definition}
Then two processes, $P,Q$, are alpha-equivalent if $P = Q\{\vec{y}/\vec{x}\}$ for
some $\vec{x} \in \boundnames{Q},\vec{y} \in \boundnames{P}$, where $Q\{\vec{y}/\vec{x}\}$
denotes the capture-avoiding substitution of $\vec{y}$ for $\vec{x}$ in $Q$.
\end{definition}

\begin{definition}
  The {\em structural congruence} \cite{SangiorgiWalker} , $\equiv$,
  between processes is the least congruence containing
  alpha-equivalence, satisfying the abelian monoid laws
  (associativity, commutativity and $\pzero$ as identity) for parallel
  composition $|$ and for summation $+$.
\end{definition}

\subsection{Name equivalence}

We take name equivalence, written $\nameeq$, to be the smallest
equivalence relation generated by the following rules.

\begin{mathpar}
\inferrule*[lab=Quote-drop]
{ }
{ \quotep{@{x}} \nameeq x }

\inferrule*[lab=Struct-equiv]
{ P \scong Q }
{ \quotep{P} \nameeq \quotep{Q} }
\end{mathpar}

The astute reader will have noticed that the mutual recursion of names
and processes imposes a mutual recursion on alpha-equivalence and
structural equivalence via name-equivalence. Fortunately, all of this
works out pleasantly and we may calculate in the natural way, free of
concern. The reader interested in the details is referred to the
appendix \ref{appendix:rho_details}.

\subsection{Substitution}

We use $\Proc$ for the set of processes, $\QProc$ for the set of
names, and $\id{\{}\vec{y} / \vec{x} \id{\}}$ to denote partial maps,
$s : \QProc \rightarrow \QProc$. A map, $s$ lifts, uniquely, to a map
on process terms, $\widehat{s} : \Proc \rightarrow \Proc$ by the
following equations.

\begin{mathpar}
  (0) \psubstp{Q}{P} := 0 \\
  (R \juxtap S) \psubstp{Q}{P}
  :=    
  (R)\psubstp{Q}{P} \juxtap (S) \psubstp{Q}{P} \\
  (x?(y).R) \psubstp{Q}{P}    
  :=    
  (x)\substp{Q}{P} (z)\concat( (R \psubstn{z}{y}) \psubstp{Q}{P} ) \\
  (\lift{x}{R}) \psubstp{Q}{P}  
  :=
  \lift{(x)\substp{Q}{P}}{ R \psubstp{Q}{P} } \\
%   (\dropn{x})  \psubstp{Q}{P}       
%   := 
%   \left\{ 
%     \begin{array}{ccc} 
%       \dropn{\quotep{Q}} & & x \nameeq \quotep{P} \\
%       \dropn{x} & & otherwise \\
%     \end{array}
%   \right. 
  (\dropn{x})  \psubstp{Q}{P}       
  := 
  \left\{ 
    \begin{array}{ccc} 
      Q & & x \nameeq \quotep{P} \\
      \dropn{x} & & otherwise \\
    \end{array}
  \right.
\end{mathpar}
 

where

\begin{eqnarray}
  (x)\id{\{} \lpquote Q \rpquote / \lpquote P \rpquote \id{\}}            = 
  \left\{ 
    \begin{array}{ccc}
      \lpquote Q \rpquote & & x \nameeq \lpquote P \rpquote \\
      x & & otherwise \\
    \end{array}
  \right. \nonumber
\end{eqnarray}

and $z$ is chosen distinct from $\quotep{P}$, $\quotep{Q}$, the free
names in $Q$, and all the names in $R$. Our $\alpha$-equivalence will
be built in the standard way from this substitution.

\begin{remark}\label{rem:no_self_referential_names}
  One consequence of these definitions is that $\forall P. \quotep{P}
  \not\in \freenames{P}$.
\end{remark}

\subsection{ Dynamic quote: an example }

Anticipating something of what's to come, consider applying the
substitution, $\widehat{\id{\{}u / z \id{\}}}$, to the following pair
of processes, $\lift{w}{y!(z)}$ and $w[ \lpquote y!(z) \rpquote ]$.

\begin{eqnarray}
	\lift{w}{y!(z)}\widehat{\id{\{}u / z \id{\}}}
		& = &
		\lift{w}{y!(u)} \nonumber\\
	w[ \lpquote y!(z) \rpquote ] \widehat{ \id{\{}u / z \id{\}} }
		& = &
		w[ \lpquote y!(z) \rpquote ] \nonumber
\end{eqnarray}

Because the body of the process between quotes is impervious to
substitution, we get radically different answers. In fact, by
examining the first process in an input context,
e.g. $x?(z).\lift{w}{y!(z)}$, we see that the process under the lift
operator may be shaped by prefixed inputs binding a name inside it. In
this sense, the lift operator will be seen as a way to dynamically
construct processes before reifying them as names.

Finally equipped with these standard features we can present the
dynamics of the calculus.

\subsubsection{Operational semantics} 

Finally, we introduce the computational dynamics. What marks these
algebras as distinct from other more traditionally studied algebraic
structures, e.g. vector spaces or polynomial rings, is the manner in
which dynamics is captured. In traditional structures, dynamics is typically
expressed through morphisms between such structures, as in linear maps
between vector spaces or morphisms between rings. In algebras
associated with the semantics of computation, the dynamics is
expressed as part of the algebraic structure itself, through a
reduction reduction relation typically denoted by $\red$. Below, we
give a recursive presentation of this relation for the calculus used
in the encoding.

$\red \subseteq \pi \times \pi$
$\red : \pi \to \mathcal{P}(\pi)$

\begin{mathpar}
  \inferrule* [lab=Comm] { \textsf{match}( x_{src}, x_{trgt} ) } { x_{trgt}?(y)P \; | \; x_{src}!\langle {Q} \rangle \red P\{\quotep{Q}/y}\} }
  \and \\
  \inferrule* [lab=Par] {{P} \red {P}'} {{{P} | {Q}} \red {{P}' | {Q}}}
  \and
  \inferrule* [lab=Equiv]{{{P} \scong {P}'} \andalso {{P}' \red {Q}'} \andalso {{Q}' \scong {Q}}}{{P} \red {Q}}
\end{mathpar}

\begin{eqnarray*}
  match_{\equiv} (\quotep{P},\quotep{Q}) & := & P \equiv Q \\
  match_{\dagger}(\quotep{P},\quotep{Q}) & := & \forall R. P|Q \red^{*} R => R \red^{*} 0 \\
  match_{K}(\quotep{P},\quotep{Q}) & := & K \mbox{ for some context } K
\end{eqnarray*}

$u?(x)P | u!\langle Q \rangle \red P\{\quotep{Q}/x\}$

%We write $\wred$ for $\red^*$, and $P\red$ if $\exists Q $ such that $ P \red Q$.
We write $P\red$ if $\exists Q $ such that $ P \red Q$ and $P\not\red$, otherwise.

\section{Replication}

As mentioned before, it is known that replication (and hence
recursion) can be implemented in a higher-order process algebra
\cite{SangiorgiWalker}. As our first example of calculation with the
machinery thus far presented we give the construction explicitly in
the {\rhoc}.

\begin{eqnarray}
	D_{x} & := & \prefix{x}{y}{(\binpar{\outputp{x}{y}}{@{y}})} \nonumber\\
	\bangp_{x}{P} & := & \binpar{{x}!\langle{\binpar{D_{x}}{P}}\rangle}{D_{x}} \nonumber
\end{eqnarray}

\begin{eqnarray}
	\bangp_{x}{P} & & \nonumber\\
	=
	& {x}!\langle{(\prefix{x}{y}{(\outputp{x}{y} | @{y})) | P}}\rangle 
	      | \prefix{x}{y}{(\outputp{x}{y} | @{y})} & \nonumber\\
	\red
	& (\outputp{x}{y} | @{y})\substn{\quotep{(\prefix{x}{y}{(@{y} | \outputp{x}{y})) | P}}}{y} & \nonumber\\
	=
	& \outputp{x}{\quotep{(\prefix{x}{y}{(\outputp{x}{y} | @{y})) | P}}}
	  | {(\prefix{x}{y}{(\outputp{x}{y} | @{y})) | P}} & \nonumber\\
	\red
	& \ldots & \nonumber\\
	\red^*
	& P | P | \ldots & \nonumber
\end{eqnarray}

Of course, this encoding, as an implementation, runs away, unfolding
$\bangp{P}$ eagerly. A lazier and more implementable replication
operator, restricted to input-guarded processes, may be obtained as follows.

\begin{eqnarray}
\bangp{\prefix{u}{v}{P}} 
	:= 
	\binpar{\lift{x}{\prefix{u}{v}{(\binpar{D(x)}{P})}}}{D(x)} \nonumber
\end{eqnarray}

\begin{remark}
  Note that the lazier definition still does not deal with summation
  or mixed summation (i.e. sums over input and output). The reader is
  invited to construct definitions of replication that deal with these
  features. 

  Further, the definitions are parameterized in a name, $x$. Can you,
  gentle reader, make a definition that eliminates this parameter and
  guarantees no accidental interaction between the replication
  machinery and the process being replicated -- i.e. no accidental
  sharing of names used by the process to get its work done and the
  name(s) used by the replication to effect copying. This latter
  revision of the definition of replication is crucial to obtaining
  the expected identity $!!P \sim !P$.
\end{remark}

\begin{remark}\label{rem:paradoxical_combinator}
  The reader familiar with the lambda calculus will have noticed the
  similarity between $D$ and the paradoxical combinator.

  [Ed. note: the existence of this seems to suggest we have to be more
  restrictive on the set of processes and names we admit if we are to
  support no-cloning.]
\end{remark}

\subsubsection{Bisimulation}

The computational dynamics gives rise to another kind of equivalence,
the equivalence of computational behavior. As previously mentioned
this is typically captured \emph{via} some form of bisimulation.

% The notion we use in this paper is weak barbed bisimulation
% \cite{milner91polyadicpi}.

The notion we use in this paper is derived from weak barbed
bisimulation \cite{milner91polyadicpi}. 

\begin{definition}
An \emph{observation relation}, $\downarrow_{\mathcal N}$, over a set
of names, $\mathcal N$, is the smallest relation satisfying the rules
below.

\infrule[Out-barb]{y \in {\mathcal N}, \; x \nameeq y}
		  {\outputp{x}{v} \downarrow_{\mathcal N} x}
\infrule[Par-barb]{\mbox{$P\downarrow_{\mathcal N} x$ or $Q\downarrow_{\mathcal N} x$}}
		  {\binpar{P}{Q} \downarrow_{\mathcal N} x}

We write $P \Downarrow_{\mathcal N} x$ if there is $Q$ such that 
$P \wred Q$ and $Q \downarrow_{\mathcal N} x$.
\end{definition}

\begin{definition}
%\label{def.bbisim}
An  ${\mathcal N}$-\emph{barbed bisimulation} over a set of names, ${\mathcal N}$, is a symmetric binary relation 
${\mathcal S}_{\mathcal N}$ between agents such that $P\rel{S}_{\mathcal N}Q$ implies:
\begin{enumerate}
\item If $P \red P'$ then $Q \wred Q'$ and $P'\rel{S}_{\mathcal N} Q'$.
\item If $P\downarrow_{\mathcal N} x$, then $Q\Downarrow_{\mathcal N} x$.
\end{enumerate}
$P$ is ${\mathcal N}$-barbed bisimilar to $Q$, written
$P \wbbisim_{\mathcal N} Q$, if $P \rel{S}_{\mathcal N} Q$ for some ${\mathcal N}$-barbed bisimulation ${\mathcal S}_{\mathcal N}$.
\end{definition}

$\mathcal{R} \subseteq \pi \times \pi$

$P \mathcal{R} Q => \forall P'. P \red P' \Rightarrow \exists Q'. Q \red Q', P' \mathcal{R} Q'$

$P \vdash x \Rightarrow Q \vdash x$

\begin{mathpar}
  \inferrule*[lab=Out-barb]{x \nameeq y}{{y}!\langle{Q}\rangle \vdash x}
  \and
  \inferrule*[lab=Par-barb]{\mbox{$P\vdash x$ or $Q\vdash x$}}{\binpar{P}{Q} \vdash x}
\end{mathpar}

\subsubsection{Contexts}

One of the principle advantages of computational calculi like the
$\pi$-calculus is a well-defined notion of context,
contextual-equivalence and a correlation between
contextual-equivalence and notions of bisimulation. The notion of
context allows the decomposition of a process into (sub-)process and
its syntactic environment, its context. Thus, a context may be
thought of as a process with a ``hole'' (written $\Box$) in it. The
application of a context $M$ to a process $P$, written $M[P]$, is
tantamount to filling the hole in $M$ with $P$. In this paper we do
not need the full weight of this theory, but do make use of the notion
of context in the proof the main theorem. 

\begin{mathpar}
  \inferrule* [lab=summation] {} {{M_{M},M_{N}} \bc \Box \;|\; x.M_{A} \;|\; M_{M}+M_{N}}
  \and
  \inferrule* [lab=agent] {} {{M_{A}} \bc (\vec{x})M_{P} \;| \; \clift{P_0,\ldots,M_{P},\ldots,P_N}}
  \and \\
  \inferrule* [lab=process] {} {{M_{P}} \bc M_{N} \;| \;P|M_{P} }
\end{mathpar} 

\begin{mathpar}
  \inferrule* [lab=sychronization] {} {M_{N} \bc \Box \;|\; x?M_{F} \;|\; x!M_{C}}
  \and
  \inferrule* [lab=abstraction] {} {{M_{F}} \bc (x)M_{P} }
  \and
  \inferrule* [lab=concretion] {} {{M_{C}} \bc \langle M_{P} \rangle }
  \and \\
  \inferrule* [lab=process] {} {{M_{P}} \bc M_{N} \;| \;P|M_{P} }
\end{mathpar}

\begin{definition}[contextual application] Given a context $M$, and
  process $P$, we define the \emph{contextual application}, $M[P] :=
  M\{P/\Box\}$. That is, the contextual application of M to P is the
  substitution of $P$ for $\Box$ in $M$.
\end{definition}

$\meaningof{-} : L \to \mathcal{P}(\pi)$

\begin{mathpar}
  \inferrule* [lab=collection] {} {\meaningof{true} = \pi, \and \meaningof{~E} = \pi \setminus \meaningof{E}, \and \meaningof{E_{1} \& E_{2}} = \meaningof{E_{1}} \cap \meaningof{E_{2}}}
\end{mathpar}

\begin{mathpar}
  \inferrule* [lab=structure] {} {\meaningof{0} = \{ P \in \pi | P \equiv 0 \}, \and \\ \meaningof{E_1 | E_2} = \{ P \in \pi | P \equiv P_{1} | P_{2}, P_{1} \in \meaningof{E_{1}}, P_{2} \in \meaningof{E_2}\} }
\end{mathpar}

\begin{mathpar}
 \inferrule* [lab=behavior] {} {\meaningof{\langle a?b \rangle E} = \{ P \in \pi | P \equiv Q | u?(y)P', \\ \and \\\\ \and \\ \;\;\; u \in \meaningof{a}, \forall z.P'\{z/y\} \in \meaningof{E\{z/b\}}\}, \and \\ \meaningof{a!E} = \{ P \in \pi | P \equiv Q | x!\langle P' \rangle, x \in \meaningof{a} P' \in \meaningof{E}\} }
\end{mathpar}

\begin{mathpar}
 \inferrule* [lab=nominal] {} {\meaningof{\quotep{E}} = \{ \quotep{P} \in \quotep{\pi} | P \in \meaningof{E} \}, \and \meaningof{\quotep{P}} = \{ \quotep{Q} \in \quotep{\pi} | P \equiv Q \} \and \\ \meaningof{@\quotep{E}} = \{ P \in \pi | P \equiv @x, x \in \meaningof{E} \}}
\end{mathpar}

\begin{eqnarray*}
  \\
  \meaningof{-} : TS \to ST
\end{eqnarray*}

\begin{eqnarray*}
  \\
  L : TS \to ST
\end{eqnarray*}

\begin{eqnarray*}
  \\
  P \models E \iff P \in \meaningof{E}
\end{eqnarray*}

\begin{eqnarray*}
  P \approx_{L} Q \iff \forall E \in L. P \models E \iff Q \models E
\end{eqnarray*}

\begin{eqnarray*}
  P \approx_{K} Q
\end{eqnarray*}

\begin{eqnarray*}
  P \approx Q
\end{eqnarray*}

$\approx_{K} = \approx = \approx_{L}$

\subsubsection{Contextual duality}

Note that contexts extend the quotation operation to a family of
operations from processes to names. Given a context, $M$, we can
define a \emph{nominal context}, $\quotep{M}$ by $\quotep{M}[P] :=
\quotep{M[P]}$. To foreshadow what is to come we observe that these
operations enjoy a duality with processes very much like the duality
between vectors and maps from vectors to scalars.

Further, because the calculus is essentially higher-order, we have a
correspondence between contexts and processes. More specifically,
given a name $x$ and a context $M$ we can construct $M^{*}_{x}$ such
that 

\begin{mathpar}
  M^{*}_{x} | \lift{x}{P} \red M[P]
\end{mathpar}

namely,

\begin{mathpar}
  M^{*}_{x} := x?(u).M[\dropn{u}]
\end{mathpar}

The dependence of $M^{*}_{x}$ on a name makes it an abstraction, 

\begin{mathpar}
  M^{*} := (x)x?(u).M[\dropn{u}]
\end{mathpar}

\subsection{Additional notation}

It will sometimes be convenient to denote the process a name
quotes. We already have the notation $x = \quotep{P}$, but it will be
convenient to introduce an alternate notation, $\procn{x}$, when we
want to emphasize the connection to the use of the name. Note that, by
virtue of name equivalence, $\quotep{\procn{x}} \nameeq x$; so, the
notation is consistent with previous definitions.

Further, because names have structure it is possible to effect
substitutions on the basis of that structure. This means we need to
upgrade our notation for substitutions, which we accomplish by
adapting comprehension notation. Thus,

\begin{mathpar}
  P\{ y / x : x \in S \}
\end{mathpar}

is interpreted to mean the process derived from P by replacing (in a
capture-avoiding manner) each occurrence of $x$ in $S$ by $y$. For example,

\begin{mathpar}
  P\{ \quotep{\procn{x}|\procn{x}} / x : x \in \freenames{P} \}
\end{mathpar}

will replace each (occurrence) of a free name $x$ in $P$ by
$\quotep{\procn{x}|\procn{x}}$.

Also, we will avail ourselves of the notation $x^{L}$ and $x^{R}$ to
denote injections of a name into disjoint copies of the name
space. There are numerous ways to accomplish this. One example can be
found in \cite{MeredithR05}. This notation overloads to vectors of
names: $\vec{x}^{\pi} := (x_{i}^{\pi} \; : \; 0 \leq i < |\vec{x}| )$ where $\pi \in \{L,R\}$.

We also use $P^{\Box} := P|\Box$.

In \cite{MeredithR05} an interpretation of the new operator is
given. It turns out that there are several possible interpretations
all enjoying the requisite algebraic properties of the operator (see
\cite{milner91polyadicpi}). We will therefore make liberal use of
$(\nu\; \vec{x})P$.

% subsection the_syntax_and_semantics_of_the_notation_system (end)   

\input{qm2pi.qmops} 

\input{qm2pi.sterngerlach} 

\input{qm2pi.metric} 

% section concurrent_process_calculi (end)

%\input{qm2pi.proofsketch}

% section proof sketch (end)

%\input{qm2pi.slviaknots} 

% section spatial logic via knots (end)

\input{qm2pi.conclusion}

% section conclusion (end)

%\input{qm2pi.dtcodes} 

% section wiring algorithm (end)

\input{qm2pi.ack} 

% section acknowledgments (end)

\newpage


\bibliographystyle{plain}   
\bibliography{../../biblios/main.bib}

\input{qm2pi.rhodetails}

\end{document}

 

\documentclass[12pt]{llncs}
%\documentclass{jktr}

\usepackage[pdftex]{hyperref}                   
\usepackage {listings}
\usepackage {mathpartir}
\usepackage{bcprules}
%\usepackage{listings}
                       
\usepackage{graphicx} 
%\usepackage[margins=2.5cm,nohead,nofoot]{geometry}
%\usepackage{geometry}
\usepackage{amsfonts}
\usepackage{amstext}
\usepackage{latexsym}
\usepackage{amssymb}
\usepackage{color}


%\include{myPreamble}
\include{qm2pi.local} 

%\ifpdf
%\usepackage[pdftex]{graphicx}
%\else
%\usepackage{graphicx}
%\fi

 % \ifpdf
%  \usepackage{pdfsync}
%  \if


%\title{Brief Article}
%\author{David F. Snyder}
%\author{L.G. Meredith}

%\address{Dept. of Math., Texas State University--San Marcos, San Marcos, TX 78666}
       
\pagestyle{empty}


\begin{document}

\lstset{language=[Objective]Caml,frame=shadowbox}

\input{qm2pi.front}

% section front matter (end)

\input{qm2pi.intro} 
 
% section introduction (end)

% \input{qm2pi.knotations} 

% section notation (end)

\input{qm2pi.process.calculi} 

% section concurrent_process_calculi_and_spatial_logics_ (end)
    
%\input{qm2pi.knots2pi} 

%\input{qm2pi.trefoil} 

%\input{qm2pi.mainthm} 

% subsection basic_interpretation (end)

%\input{qm2pi.rho.presentation} 
\subsection{The syntax and semantics of the notation system}\label{sub:the_syntax_and_semantics_of_the_notation_system} % (fold)

We now summarize a technical presentation of the calculus that
embodies our theory of dynamics. The typical presentation of such a
calculus follows the style of giving generators and relations on
them. The grammar, below, describing term constructors, freely
generates the set of processes, $\Proc$. This set is then quotiented
by a relation known as structural congruence and it is over this set
that the notion of dynamics is expressed. This presentation is
essentially that of \cite{MeredithR05} with the addition of
polyadicity and summation. For readability we have relegated some of
the technical subtleties to an appendix.

\subsubsection{Process grammar}\label{subsub:process_grammar}

\begin{mathpar}
  \inferrule* [lab=synchronization] {} {{M} \bc \pzero \;|\; x?F \;|\; x!C }
  \and
  \inferrule* [lab=abstraction] {} {{F} \bc (x)P}
  \and
  \inferrule* [lab=concretion] {} {{C} \bc \langle Q \rangle}
  \and
  \inferrule* [lab=process] {} {{P,Q} \bc M \;| \;P|Q \;|\; @{x}}
  \and
  \inferrule* [lab=name] {} {{x} \bc \quotep{P}}
\end{mathpar} 

Note that $\vec{x}$ (resp. $\vec{P}$) denotes a vector of names
(resp. processes) of length $|\vec{x}|$ (resp. $|\vec{P}|$). We adopt
the following useful abbreviations.

\begin{mathpar}
   x?(\vec{y}).P := x.(\vec{y})P \and  x\clift{\vec{P}} := x.\clift{\vec{P}}
   \and x!(y) := \lift{x}{\dropn{y}}
   \and \Pi_{i=0}^{n-1}P_i := P_0 | \ldots | P_{n-1}
\end{mathpar}

\subsubsection{Structural congruence}

\paragraph{Free and bound names and alpha-equivalence.} At the
core of structural equivalence is alpha-equivalence which identifies
process that are the same up to a change of variable. Formally, we
recognize the distinction between free and bound names. The free names
of a process, $\freenames{P}$, may be calculated recursively as
follows:

\begin{mathpar}
\freenames{\pzero} := \emptyset
  \and \\
  \freenames{x?(y).P} := \{ x \} \cup (\freenames{P} \setminus \{ y \})
  \and 
  \freenames{x!\langle P \rangle} := \{ x \} \cup \{ P \} 
  \and \\
  \freenames{P|Q} := \freenames{P} \cup \freenames{Q}
  \and \\
  \freenames{@{x}} := \{ x \}
\end{mathpar}

$\pi$
$\quotep{\pi}$

$\freenames{-} : \pi \to \mathcal{P}(\quotep{\pi})$

\begin{eqnarray*}
  \freenames{\pzero} & := & \emptyset \\
  \freenames{x?(y).P} & := & \{ x \} \cup (\freenames{P} \setminus \{ y \}) \\
  \freenames{x!\langle P \rangle} & := & \{ x \} \cup \{ P \} \\
  \freenames{P|Q} & := & \freenames{P} \cup \freenames{Q} \\
  \freenames{\dropn{x}} & := & \{ x \}
\end{eqnarray*}

The bound names of a process, $\boundnames{P}$, are those names occurring in $P$
that are not free. For example, in $x?(y).0$, the name $x$ is free, while $y$ is bound.

\begin{mathpar}
  \inferrule* [lab=monoidal-laws] {} { P|Q \equiv Q|P \and P|0 \equiv P \and P|(Q|R) \equiv (P|Q)|R }
\end{mathpar}

\begin{mathpar}
  \inferrule* [lab=alpha-equivalence] {} { (x)P \equiv (y)P\{y/x\} \and y \not\in \freenames{P} }
\end{mathpar}

\begin{definition}
Then two processes, $P,Q$, are alpha-equivalent if $P = Q\{\vec{y}/\vec{x}\}$ for
some $\vec{x} \in \boundnames{Q},\vec{y} \in \boundnames{P}$, where $Q\{\vec{y}/\vec{x}\}$
denotes the capture-avoiding substitution of $\vec{y}$ for $\vec{x}$ in $Q$.
\end{definition}

\begin{definition}
  The {\em structural congruence} \cite{SangiorgiWalker} , $\equiv$,
  between processes is the least congruence containing
  alpha-equivalence, satisfying the abelian monoid laws
  (associativity, commutativity and $\pzero$ as identity) for parallel
  composition $|$ and for summation $+$.
\end{definition}

\subsection{Name equivalence}

We take name equivalence, written $\nameeq$, to be the smallest
equivalence relation generated by the following rules.

\begin{mathpar}
\inferrule*[lab=Quote-drop]
{ }
{ \quotep{@{x}} \nameeq x }

\inferrule*[lab=Struct-equiv]
{ P \scong Q }
{ \quotep{P} \nameeq \quotep{Q} }
\end{mathpar}

The astute reader will have noticed that the mutual recursion of names
and processes imposes a mutual recursion on alpha-equivalence and
structural equivalence via name-equivalence. Fortunately, all of this
works out pleasantly and we may calculate in the natural way, free of
concern. The reader interested in the details is referred to the
appendix \ref{appendix:rho_details}.

\subsection{Substitution}

We use $\Proc$ for the set of processes, $\QProc$ for the set of
names, and $\id{\{}\vec{y} / \vec{x} \id{\}}$ to denote partial maps,
$s : \QProc \rightarrow \QProc$. A map, $s$ lifts, uniquely, to a map
on process terms, $\widehat{s} : \Proc \rightarrow \Proc$ by the
following equations.

\begin{mathpar}
  (0) \psubstp{Q}{P} := 0 \\
  (R \juxtap S) \psubstp{Q}{P}
  :=    
  (R)\psubstp{Q}{P} \juxtap (S) \psubstp{Q}{P} \\
  (x?(y).R) \psubstp{Q}{P}    
  :=    
  (x)\substp{Q}{P} (z)\concat( (R \psubstn{z}{y}) \psubstp{Q}{P} ) \\
  (\lift{x}{R}) \psubstp{Q}{P}  
  :=
  \lift{(x)\substp{Q}{P}}{ R \psubstp{Q}{P} } \\
%   (\dropn{x})  \psubstp{Q}{P}       
%   := 
%   \left\{ 
%     \begin{array}{ccc} 
%       \dropn{\quotep{Q}} & & x \nameeq \quotep{P} \\
%       \dropn{x} & & otherwise \\
%     \end{array}
%   \right. 
  (\dropn{x})  \psubstp{Q}{P}       
  := 
  \left\{ 
    \begin{array}{ccc} 
      Q & & x \nameeq \quotep{P} \\
      \dropn{x} & & otherwise \\
    \end{array}
  \right.
\end{mathpar}
 

where

\begin{eqnarray}
  (x)\id{\{} \lpquote Q \rpquote / \lpquote P \rpquote \id{\}}            = 
  \left\{ 
    \begin{array}{ccc}
      \lpquote Q \rpquote & & x \nameeq \lpquote P \rpquote \\
      x & & otherwise \\
    \end{array}
  \right. \nonumber
\end{eqnarray}

and $z$ is chosen distinct from $\quotep{P}$, $\quotep{Q}$, the free
names in $Q$, and all the names in $R$. Our $\alpha$-equivalence will
be built in the standard way from this substitution.

\begin{remark}\label{rem:no_self_referential_names}
  One consequence of these definitions is that $\forall P. \quotep{P}
  \not\in \freenames{P}$.
\end{remark}

\subsection{ Dynamic quote: an example }

Anticipating something of what's to come, consider applying the
substitution, $\widehat{\id{\{}u / z \id{\}}}$, to the following pair
of processes, $\lift{w}{y!(z)}$ and $w[ \lpquote y!(z) \rpquote ]$.

\begin{eqnarray}
	\lift{w}{y!(z)}\widehat{\id{\{}u / z \id{\}}}
		& = &
		\lift{w}{y!(u)} \nonumber\\
	w[ \lpquote y!(z) \rpquote ] \widehat{ \id{\{}u / z \id{\}} }
		& = &
		w[ \lpquote y!(z) \rpquote ] \nonumber
\end{eqnarray}

Because the body of the process between quotes is impervious to
substitution, we get radically different answers. In fact, by
examining the first process in an input context,
e.g. $x?(z).\lift{w}{y!(z)}$, we see that the process under the lift
operator may be shaped by prefixed inputs binding a name inside it. In
this sense, the lift operator will be seen as a way to dynamically
construct processes before reifying them as names.

Finally equipped with these standard features we can present the
dynamics of the calculus.

\subsubsection{Operational semantics} 

Finally, we introduce the computational dynamics. What marks these
algebras as distinct from other more traditionally studied algebraic
structures, e.g. vector spaces or polynomial rings, is the manner in
which dynamics is captured. In traditional structures, dynamics is typically
expressed through morphisms between such structures, as in linear maps
between vector spaces or morphisms between rings. In algebras
associated with the semantics of computation, the dynamics is
expressed as part of the algebraic structure itself, through a
reduction reduction relation typically denoted by $\red$. Below, we
give a recursive presentation of this relation for the calculus used
in the encoding.

$\red \subseteq \pi \times \pi$
$\red : \pi \to \mathcal{P}(\pi)$

\begin{mathpar}
  \inferrule* [lab=Comm] { \textsf{match}( x_{src}, x_{trgt} ) } { x_{trgt}?(y)P \; | \; x_{src}!\langle {Q} \rangle \red P\{\quotep{Q}/y}\} }
  \and \\
  \inferrule* [lab=Par] {{P} \red {P}'} {{{P} | {Q}} \red {{P}' | {Q}}}
  \and
  \inferrule* [lab=Equiv]{{{P} \scong {P}'} \andalso {{P}' \red {Q}'} \andalso {{Q}' \scong {Q}}}{{P} \red {Q}}
\end{mathpar}

\begin{eqnarray*}
  match_{\equiv} (\quotep{P},\quotep{Q}) & := & P \equiv Q \\
  match_{\dagger}(\quotep{P},\quotep{Q}) & := & \forall R. P|Q \red^{*} R => R \red^{*} 0 \\
  match_{K}(\quotep{P},\quotep{Q}) & := & K \mbox{ for some context } K
\end{eqnarray*}

$u?(x)P | u!\langle Q \rangle \red P\{\quotep{Q}/x\}$

%We write $\wred$ for $\red^*$, and $P\red$ if $\exists Q $ such that $ P \red Q$.
We write $P\red$ if $\exists Q $ such that $ P \red Q$ and $P\not\red$, otherwise.

\section{Replication}

As mentioned before, it is known that replication (and hence
recursion) can be implemented in a higher-order process algebra
\cite{SangiorgiWalker}. As our first example of calculation with the
machinery thus far presented we give the construction explicitly in
the {\rhoc}.

\begin{eqnarray}
	D_{x} & := & \prefix{x}{y}{(\binpar{\outputp{x}{y}}{@{y}})} \nonumber\\
	\bangp_{x}{P} & := & \binpar{{x}!\langle{\binpar{D_{x}}{P}}\rangle}{D_{x}} \nonumber
\end{eqnarray}

\begin{eqnarray}
	\bangp_{x}{P} & & \nonumber\\
	=
	& {x}!\langle{(\prefix{x}{y}{(\outputp{x}{y} | @{y})) | P}}\rangle 
	      | \prefix{x}{y}{(\outputp{x}{y} | @{y})} & \nonumber\\
	\red
	& (\outputp{x}{y} | @{y})\substn{\quotep{(\prefix{x}{y}{(@{y} | \outputp{x}{y})) | P}}}{y} & \nonumber\\
	=
	& \outputp{x}{\quotep{(\prefix{x}{y}{(\outputp{x}{y} | @{y})) | P}}}
	  | {(\prefix{x}{y}{(\outputp{x}{y} | @{y})) | P}} & \nonumber\\
	\red
	& \ldots & \nonumber\\
	\red^*
	& P | P | \ldots & \nonumber
\end{eqnarray}

Of course, this encoding, as an implementation, runs away, unfolding
$\bangp{P}$ eagerly. A lazier and more implementable replication
operator, restricted to input-guarded processes, may be obtained as follows.

\begin{eqnarray}
\bangp{\prefix{u}{v}{P}} 
	:= 
	\binpar{\lift{x}{\prefix{u}{v}{(\binpar{D(x)}{P})}}}{D(x)} \nonumber
\end{eqnarray}

\begin{remark}
  Note that the lazier definition still does not deal with summation
  or mixed summation (i.e. sums over input and output). The reader is
  invited to construct definitions of replication that deal with these
  features. 

  Further, the definitions are parameterized in a name, $x$. Can you,
  gentle reader, make a definition that eliminates this parameter and
  guarantees no accidental interaction between the replication
  machinery and the process being replicated -- i.e. no accidental
  sharing of names used by the process to get its work done and the
  name(s) used by the replication to effect copying. This latter
  revision of the definition of replication is crucial to obtaining
  the expected identity $!!P \sim !P$.
\end{remark}

\begin{remark}\label{rem:paradoxical_combinator}
  The reader familiar with the lambda calculus will have noticed the
  similarity between $D$ and the paradoxical combinator.

  [Ed. note: the existence of this seems to suggest we have to be more
  restrictive on the set of processes and names we admit if we are to
  support no-cloning.]
\end{remark}

\subsubsection{Bisimulation}

The computational dynamics gives rise to another kind of equivalence,
the equivalence of computational behavior. As previously mentioned
this is typically captured \emph{via} some form of bisimulation.

% The notion we use in this paper is weak barbed bisimulation
% \cite{milner91polyadicpi}.

The notion we use in this paper is derived from weak barbed
bisimulation \cite{milner91polyadicpi}. 

\begin{definition}
An \emph{observation relation}, $\downarrow_{\mathcal N}$, over a set
of names, $\mathcal N$, is the smallest relation satisfying the rules
below.

\infrule[Out-barb]{y \in {\mathcal N}, \; x \nameeq y}
		  {\outputp{x}{v} \downarrow_{\mathcal N} x}
\infrule[Par-barb]{\mbox{$P\downarrow_{\mathcal N} x$ or $Q\downarrow_{\mathcal N} x$}}
		  {\binpar{P}{Q} \downarrow_{\mathcal N} x}

We write $P \Downarrow_{\mathcal N} x$ if there is $Q$ such that 
$P \wred Q$ and $Q \downarrow_{\mathcal N} x$.
\end{definition}

\begin{definition}
%\label{def.bbisim}
An  ${\mathcal N}$-\emph{barbed bisimulation} over a set of names, ${\mathcal N}$, is a symmetric binary relation 
${\mathcal S}_{\mathcal N}$ between agents such that $P\rel{S}_{\mathcal N}Q$ implies:
\begin{enumerate}
\item If $P \red P'$ then $Q \wred Q'$ and $P'\rel{S}_{\mathcal N} Q'$.
\item If $P\downarrow_{\mathcal N} x$, then $Q\Downarrow_{\mathcal N} x$.
\end{enumerate}
$P$ is ${\mathcal N}$-barbed bisimilar to $Q$, written
$P \wbbisim_{\mathcal N} Q$, if $P \rel{S}_{\mathcal N} Q$ for some ${\mathcal N}$-barbed bisimulation ${\mathcal S}_{\mathcal N}$.
\end{definition}

$\mathcal{R} \subseteq \pi \times \pi$

$P \mathcal{R} Q => \forall P'. P \red P' \Rightarrow \exists Q'. Q \red Q', P' \mathcal{R} Q'$

$P \vdash x \Rightarrow Q \vdash x$

\begin{mathpar}
  \inferrule*[lab=Out-barb]{x \nameeq y}{{y}!\langle{Q}\rangle \vdash x}
  \and
  \inferrule*[lab=Par-barb]{\mbox{$P\vdash x$ or $Q\vdash x$}}{\binpar{P}{Q} \vdash x}
\end{mathpar}

\subsubsection{Contexts}

One of the principle advantages of computational calculi like the
$\pi$-calculus is a well-defined notion of context,
contextual-equivalence and a correlation between
contextual-equivalence and notions of bisimulation. The notion of
context allows the decomposition of a process into (sub-)process and
its syntactic environment, its context. Thus, a context may be
thought of as a process with a ``hole'' (written $\Box$) in it. The
application of a context $M$ to a process $P$, written $M[P]$, is
tantamount to filling the hole in $M$ with $P$. In this paper we do
not need the full weight of this theory, but do make use of the notion
of context in the proof the main theorem. 

\begin{mathpar}
  \inferrule* [lab=summation] {} {{M_{M},M_{N}} \bc \Box \;|\; x.M_{A} \;|\; M_{M}+M_{N}}
  \and
  \inferrule* [lab=agent] {} {{M_{A}} \bc (\vec{x})M_{P} \;| \; \clift{P_0,\ldots,M_{P},\ldots,P_N}}
  \and \\
  \inferrule* [lab=process] {} {{M_{P}} \bc M_{N} \;| \;P|M_{P} }
\end{mathpar} 

\begin{mathpar}
  \inferrule* [lab=sychronization] {} {M_{N} \bc \Box \;|\; x?M_{F} \;|\; x!M_{C}}
  \and
  \inferrule* [lab=abstraction] {} {{M_{F}} \bc (x)M_{P} }
  \and
  \inferrule* [lab=concretion] {} {{M_{C}} \bc \langle M_{P} \rangle }
  \and \\
  \inferrule* [lab=process] {} {{M_{P}} \bc M_{N} \;| \;P|M_{P} }
\end{mathpar}

\begin{definition}[contextual application] Given a context $M$, and
  process $P$, we define the \emph{contextual application}, $M[P] :=
  M\{P/\Box\}$. That is, the contextual application of M to P is the
  substitution of $P$ for $\Box$ in $M$.
\end{definition}

$\meaningof{-} : L \to \mathcal{P}(\pi)$

\begin{mathpar}
  \inferrule* [lab=collection] {} {\meaningof{true} = \pi, \and \meaningof{~E} = \pi \setminus \meaningof{E}, \and \meaningof{E_{1} \& E_{2}} = \meaningof{E_{1}} \cap \meaningof{E_{2}}}
\end{mathpar}

\begin{mathpar}
  \inferrule* [lab=structure] {} {\meaningof{0} = \{ P \in \pi | P \equiv 0 \}, \and \\ \meaningof{E_1 | E_2} = \{ P \in \pi | P \equiv P_{1} | P_{2}, P_{1} \in \meaningof{E_{1}}, P_{2} \in \meaningof{E_2}\} }
\end{mathpar}

\begin{mathpar}
 \inferrule* [lab=behavior] {} {\meaningof{\langle a?b \rangle E} = \{ P \in \pi | P \equiv Q | u?(y)P', \\ \and \\\\ \and \\ \;\;\; u \in \meaningof{a}, \forall z.P'\{z/y\} \in \meaningof{E\{z/b\}}\}, \and \\ \meaningof{a!E} = \{ P \in \pi | P \equiv Q | x!\langle P' \rangle, x \in \meaningof{a} P' \in \meaningof{E}\} }
\end{mathpar}

\begin{mathpar}
 \inferrule* [lab=nominal] {} {\meaningof{\quotep{E}} = \{ \quotep{P} \in \quotep{\pi} | P \in \meaningof{E} \}, \and \meaningof{\quotep{P}} = \{ \quotep{Q} \in \quotep{\pi} | P \equiv Q \} \and \\ \meaningof{@\quotep{E}} = \{ P \in \pi | P \equiv @x, x \in \meaningof{E} \}}
\end{mathpar}

\begin{eqnarray*}
  \\
  \meaningof{-} : TS \to ST
\end{eqnarray*}

\begin{eqnarray*}
  \\
  L : TS \to ST
\end{eqnarray*}

\begin{eqnarray*}
  \\
  P \models E \iff P \in \meaningof{E}
\end{eqnarray*}

\begin{eqnarray*}
  P \approx_{L} Q \iff \forall E \in L. P \models E \iff Q \models E
\end{eqnarray*}

\begin{eqnarray*}
  P \approx_{K} Q
\end{eqnarray*}

\begin{eqnarray*}
  P \approx Q
\end{eqnarray*}

$\approx_{K} = \approx = \approx_{L}$

\subsubsection{Contextual duality}

Note that contexts extend the quotation operation to a family of
operations from processes to names. Given a context, $M$, we can
define a \emph{nominal context}, $\quotep{M}$ by $\quotep{M}[P] :=
\quotep{M[P]}$. To foreshadow what is to come we observe that these
operations enjoy a duality with processes very much like the duality
between vectors and maps from vectors to scalars.

Further, because the calculus is essentially higher-order, we have a
correspondence between contexts and processes. More specifically,
given a name $x$ and a context $M$ we can construct $M^{*}_{x}$ such
that 

\begin{mathpar}
  M^{*}_{x} | \lift{x}{P} \red M[P]
\end{mathpar}

namely,

\begin{mathpar}
  M^{*}_{x} := x?(u).M[\dropn{u}]
\end{mathpar}

The dependence of $M^{*}_{x}$ on a name makes it an abstraction, 

\begin{mathpar}
  M^{*} := (x)x?(u).M[\dropn{u}]
\end{mathpar}

\subsection{Additional notation}

It will sometimes be convenient to denote the process a name
quotes. We already have the notation $x = \quotep{P}$, but it will be
convenient to introduce an alternate notation, $\procn{x}$, when we
want to emphasize the connection to the use of the name. Note that, by
virtue of name equivalence, $\quotep{\procn{x}} \nameeq x$; so, the
notation is consistent with previous definitions.

Further, because names have structure it is possible to effect
substitutions on the basis of that structure. This means we need to
upgrade our notation for substitutions, which we accomplish by
adapting comprehension notation. Thus,

\begin{mathpar}
  P\{ y / x : x \in S \}
\end{mathpar}

is interpreted to mean the process derived from P by replacing (in a
capture-avoiding manner) each occurrence of $x$ in $S$ by $y$. For example,

\begin{mathpar}
  P\{ \quotep{\procn{x}|\procn{x}} / x : x \in \freenames{P} \}
\end{mathpar}

will replace each (occurrence) of a free name $x$ in $P$ by
$\quotep{\procn{x}|\procn{x}}$.

Also, we will avail ourselves of the notation $x^{L}$ and $x^{R}$ to
denote injections of a name into disjoint copies of the name
space. There are numerous ways to accomplish this. One example can be
found in \cite{MeredithR05}. This notation overloads to vectors of
names: $\vec{x}^{\pi} := (x_{i}^{\pi} \; : \; 0 \leq i < |\vec{x}| )$ where $\pi \in \{L,R\}$.

We also use $P^{\Box} := P|\Box$.

In \cite{MeredithR05} an interpretation of the new operator is
given. It turns out that there are several possible interpretations
all enjoying the requisite algebraic properties of the operator (see
\cite{milner91polyadicpi}). We will therefore make liberal use of
$(\nu\; \vec{x})P$.

% subsection the_syntax_and_semantics_of_the_notation_system (end)   

\input{qm2pi.qmops} 

\input{qm2pi.sterngerlach} 

\input{qm2pi.metric} 

% section concurrent_process_calculi (end)

%\input{qm2pi.proofsketch}

% section proof sketch (end)

%\input{qm2pi.slviaknots} 

% section spatial logic via knots (end)

\input{qm2pi.conclusion}

% section conclusion (end)

%\input{qm2pi.dtcodes} 

% section wiring algorithm (end)

\input{qm2pi.ack} 

% section acknowledgments (end)

\newpage


\bibliographystyle{plain}   
\bibliography{../../biblios/main.bib}

\input{qm2pi.rhodetails}

\end{document}

 

% section concurrent_process_calculi (end)

%\documentclass[12pt]{llncs}
%\documentclass{jktr}

\usepackage[pdftex]{hyperref}                   
\usepackage {listings}
\usepackage {mathpartir}
\usepackage{bcprules}
%\usepackage{listings}
                       
\usepackage{graphicx} 
%\usepackage[margins=2.5cm,nohead,nofoot]{geometry}
%\usepackage{geometry}
\usepackage{amsfonts}
\usepackage{amstext}
\usepackage{latexsym}
\usepackage{amssymb}
\usepackage{color}


%\include{myPreamble}
\include{qm2pi.local} 

%\ifpdf
%\usepackage[pdftex]{graphicx}
%\else
%\usepackage{graphicx}
%\fi

 % \ifpdf
%  \usepackage{pdfsync}
%  \if


%\title{Brief Article}
%\author{David F. Snyder}
%\author{L.G. Meredith}

%\address{Dept. of Math., Texas State University--San Marcos, San Marcos, TX 78666}
       
\pagestyle{empty}


\begin{document}

\lstset{language=[Objective]Caml,frame=shadowbox}

\input{qm2pi.front}

% section front matter (end)

\input{qm2pi.intro} 
 
% section introduction (end)

% \input{qm2pi.knotations} 

% section notation (end)

\input{qm2pi.process.calculi} 

% section concurrent_process_calculi_and_spatial_logics_ (end)
    
%\input{qm2pi.knots2pi} 

%\input{qm2pi.trefoil} 

%\input{qm2pi.mainthm} 

% subsection basic_interpretation (end)

%\input{qm2pi.rho.presentation} 
\subsection{The syntax and semantics of the notation system}\label{sub:the_syntax_and_semantics_of_the_notation_system} % (fold)

We now summarize a technical presentation of the calculus that
embodies our theory of dynamics. The typical presentation of such a
calculus follows the style of giving generators and relations on
them. The grammar, below, describing term constructors, freely
generates the set of processes, $\Proc$. This set is then quotiented
by a relation known as structural congruence and it is over this set
that the notion of dynamics is expressed. This presentation is
essentially that of \cite{MeredithR05} with the addition of
polyadicity and summation. For readability we have relegated some of
the technical subtleties to an appendix.

\subsubsection{Process grammar}\label{subsub:process_grammar}

\begin{mathpar}
  \inferrule* [lab=synchronization] {} {{M} \bc \pzero \;|\; x?F \;|\; x!C }
  \and
  \inferrule* [lab=abstraction] {} {{F} \bc (x)P}
  \and
  \inferrule* [lab=concretion] {} {{C} \bc \langle Q \rangle}
  \and
  \inferrule* [lab=process] {} {{P,Q} \bc M \;| \;P|Q \;|\; @{x}}
  \and
  \inferrule* [lab=name] {} {{x} \bc \quotep{P}}
\end{mathpar} 

Note that $\vec{x}$ (resp. $\vec{P}$) denotes a vector of names
(resp. processes) of length $|\vec{x}|$ (resp. $|\vec{P}|$). We adopt
the following useful abbreviations.

\begin{mathpar}
   x?(\vec{y}).P := x.(\vec{y})P \and  x\clift{\vec{P}} := x.\clift{\vec{P}}
   \and x!(y) := \lift{x}{\dropn{y}}
   \and \Pi_{i=0}^{n-1}P_i := P_0 | \ldots | P_{n-1}
\end{mathpar}

\subsubsection{Structural congruence}

\paragraph{Free and bound names and alpha-equivalence.} At the
core of structural equivalence is alpha-equivalence which identifies
process that are the same up to a change of variable. Formally, we
recognize the distinction between free and bound names. The free names
of a process, $\freenames{P}$, may be calculated recursively as
follows:

\begin{mathpar}
\freenames{\pzero} := \emptyset
  \and \\
  \freenames{x?(y).P} := \{ x \} \cup (\freenames{P} \setminus \{ y \})
  \and 
  \freenames{x!\langle P \rangle} := \{ x \} \cup \{ P \} 
  \and \\
  \freenames{P|Q} := \freenames{P} \cup \freenames{Q}
  \and \\
  \freenames{@{x}} := \{ x \}
\end{mathpar}

$\pi$
$\quotep{\pi}$

$\freenames{-} : \pi \to \mathcal{P}(\quotep{\pi})$

\begin{eqnarray*}
  \freenames{\pzero} & := & \emptyset \\
  \freenames{x?(y).P} & := & \{ x \} \cup (\freenames{P} \setminus \{ y \}) \\
  \freenames{x!\langle P \rangle} & := & \{ x \} \cup \{ P \} \\
  \freenames{P|Q} & := & \freenames{P} \cup \freenames{Q} \\
  \freenames{\dropn{x}} & := & \{ x \}
\end{eqnarray*}

The bound names of a process, $\boundnames{P}$, are those names occurring in $P$
that are not free. For example, in $x?(y).0$, the name $x$ is free, while $y$ is bound.

\begin{mathpar}
  \inferrule* [lab=monoidal-laws] {} { P|Q \equiv Q|P \and P|0 \equiv P \and P|(Q|R) \equiv (P|Q)|R }
\end{mathpar}

\begin{mathpar}
  \inferrule* [lab=alpha-equivalence] {} { (x)P \equiv (y)P\{y/x\} \and y \not\in \freenames{P} }
\end{mathpar}

\begin{definition}
Then two processes, $P,Q$, are alpha-equivalent if $P = Q\{\vec{y}/\vec{x}\}$ for
some $\vec{x} \in \boundnames{Q},\vec{y} \in \boundnames{P}$, where $Q\{\vec{y}/\vec{x}\}$
denotes the capture-avoiding substitution of $\vec{y}$ for $\vec{x}$ in $Q$.
\end{definition}

\begin{definition}
  The {\em structural congruence} \cite{SangiorgiWalker} , $\equiv$,
  between processes is the least congruence containing
  alpha-equivalence, satisfying the abelian monoid laws
  (associativity, commutativity and $\pzero$ as identity) for parallel
  composition $|$ and for summation $+$.
\end{definition}

\subsection{Name equivalence}

We take name equivalence, written $\nameeq$, to be the smallest
equivalence relation generated by the following rules.

\begin{mathpar}
\inferrule*[lab=Quote-drop]
{ }
{ \quotep{@{x}} \nameeq x }

\inferrule*[lab=Struct-equiv]
{ P \scong Q }
{ \quotep{P} \nameeq \quotep{Q} }
\end{mathpar}

The astute reader will have noticed that the mutual recursion of names
and processes imposes a mutual recursion on alpha-equivalence and
structural equivalence via name-equivalence. Fortunately, all of this
works out pleasantly and we may calculate in the natural way, free of
concern. The reader interested in the details is referred to the
appendix \ref{appendix:rho_details}.

\subsection{Substitution}

We use $\Proc$ for the set of processes, $\QProc$ for the set of
names, and $\id{\{}\vec{y} / \vec{x} \id{\}}$ to denote partial maps,
$s : \QProc \rightarrow \QProc$. A map, $s$ lifts, uniquely, to a map
on process terms, $\widehat{s} : \Proc \rightarrow \Proc$ by the
following equations.

\begin{mathpar}
  (0) \psubstp{Q}{P} := 0 \\
  (R \juxtap S) \psubstp{Q}{P}
  :=    
  (R)\psubstp{Q}{P} \juxtap (S) \psubstp{Q}{P} \\
  (x?(y).R) \psubstp{Q}{P}    
  :=    
  (x)\substp{Q}{P} (z)\concat( (R \psubstn{z}{y}) \psubstp{Q}{P} ) \\
  (\lift{x}{R}) \psubstp{Q}{P}  
  :=
  \lift{(x)\substp{Q}{P}}{ R \psubstp{Q}{P} } \\
%   (\dropn{x})  \psubstp{Q}{P}       
%   := 
%   \left\{ 
%     \begin{array}{ccc} 
%       \dropn{\quotep{Q}} & & x \nameeq \quotep{P} \\
%       \dropn{x} & & otherwise \\
%     \end{array}
%   \right. 
  (\dropn{x})  \psubstp{Q}{P}       
  := 
  \left\{ 
    \begin{array}{ccc} 
      Q & & x \nameeq \quotep{P} \\
      \dropn{x} & & otherwise \\
    \end{array}
  \right.
\end{mathpar}
 

where

\begin{eqnarray}
  (x)\id{\{} \lpquote Q \rpquote / \lpquote P \rpquote \id{\}}            = 
  \left\{ 
    \begin{array}{ccc}
      \lpquote Q \rpquote & & x \nameeq \lpquote P \rpquote \\
      x & & otherwise \\
    \end{array}
  \right. \nonumber
\end{eqnarray}

and $z$ is chosen distinct from $\quotep{P}$, $\quotep{Q}$, the free
names in $Q$, and all the names in $R$. Our $\alpha$-equivalence will
be built in the standard way from this substitution.

\begin{remark}\label{rem:no_self_referential_names}
  One consequence of these definitions is that $\forall P. \quotep{P}
  \not\in \freenames{P}$.
\end{remark}

\subsection{ Dynamic quote: an example }

Anticipating something of what's to come, consider applying the
substitution, $\widehat{\id{\{}u / z \id{\}}}$, to the following pair
of processes, $\lift{w}{y!(z)}$ and $w[ \lpquote y!(z) \rpquote ]$.

\begin{eqnarray}
	\lift{w}{y!(z)}\widehat{\id{\{}u / z \id{\}}}
		& = &
		\lift{w}{y!(u)} \nonumber\\
	w[ \lpquote y!(z) \rpquote ] \widehat{ \id{\{}u / z \id{\}} }
		& = &
		w[ \lpquote y!(z) \rpquote ] \nonumber
\end{eqnarray}

Because the body of the process between quotes is impervious to
substitution, we get radically different answers. In fact, by
examining the first process in an input context,
e.g. $x?(z).\lift{w}{y!(z)}$, we see that the process under the lift
operator may be shaped by prefixed inputs binding a name inside it. In
this sense, the lift operator will be seen as a way to dynamically
construct processes before reifying them as names.

Finally equipped with these standard features we can present the
dynamics of the calculus.

\subsubsection{Operational semantics} 

Finally, we introduce the computational dynamics. What marks these
algebras as distinct from other more traditionally studied algebraic
structures, e.g. vector spaces or polynomial rings, is the manner in
which dynamics is captured. In traditional structures, dynamics is typically
expressed through morphisms between such structures, as in linear maps
between vector spaces or morphisms between rings. In algebras
associated with the semantics of computation, the dynamics is
expressed as part of the algebraic structure itself, through a
reduction reduction relation typically denoted by $\red$. Below, we
give a recursive presentation of this relation for the calculus used
in the encoding.

$\red \subseteq \pi \times \pi$
$\red : \pi \to \mathcal{P}(\pi)$

\begin{mathpar}
  \inferrule* [lab=Comm] { \textsf{match}( x_{src}, x_{trgt} ) } { x_{trgt}?(y)P \; | \; x_{src}!\langle {Q} \rangle \red P\{\quotep{Q}/y}\} }
  \and \\
  \inferrule* [lab=Par] {{P} \red {P}'} {{{P} | {Q}} \red {{P}' | {Q}}}
  \and
  \inferrule* [lab=Equiv]{{{P} \scong {P}'} \andalso {{P}' \red {Q}'} \andalso {{Q}' \scong {Q}}}{{P} \red {Q}}
\end{mathpar}

\begin{eqnarray*}
  match_{\equiv} (\quotep{P},\quotep{Q}) & := & P \equiv Q \\
  match_{\dagger}(\quotep{P},\quotep{Q}) & := & \forall R. P|Q \red^{*} R => R \red^{*} 0 \\
  match_{K}(\quotep{P},\quotep{Q}) & := & K \mbox{ for some context } K
\end{eqnarray*}

$u?(x)P | u!\langle Q \rangle \red P\{\quotep{Q}/x\}$

%We write $\wred$ for $\red^*$, and $P\red$ if $\exists Q $ such that $ P \red Q$.
We write $P\red$ if $\exists Q $ such that $ P \red Q$ and $P\not\red$, otherwise.

\section{Replication}

As mentioned before, it is known that replication (and hence
recursion) can be implemented in a higher-order process algebra
\cite{SangiorgiWalker}. As our first example of calculation with the
machinery thus far presented we give the construction explicitly in
the {\rhoc}.

\begin{eqnarray}
	D_{x} & := & \prefix{x}{y}{(\binpar{\outputp{x}{y}}{@{y}})} \nonumber\\
	\bangp_{x}{P} & := & \binpar{{x}!\langle{\binpar{D_{x}}{P}}\rangle}{D_{x}} \nonumber
\end{eqnarray}

\begin{eqnarray}
	\bangp_{x}{P} & & \nonumber\\
	=
	& {x}!\langle{(\prefix{x}{y}{(\outputp{x}{y} | @{y})) | P}}\rangle 
	      | \prefix{x}{y}{(\outputp{x}{y} | @{y})} & \nonumber\\
	\red
	& (\outputp{x}{y} | @{y})\substn{\quotep{(\prefix{x}{y}{(@{y} | \outputp{x}{y})) | P}}}{y} & \nonumber\\
	=
	& \outputp{x}{\quotep{(\prefix{x}{y}{(\outputp{x}{y} | @{y})) | P}}}
	  | {(\prefix{x}{y}{(\outputp{x}{y} | @{y})) | P}} & \nonumber\\
	\red
	& \ldots & \nonumber\\
	\red^*
	& P | P | \ldots & \nonumber
\end{eqnarray}

Of course, this encoding, as an implementation, runs away, unfolding
$\bangp{P}$ eagerly. A lazier and more implementable replication
operator, restricted to input-guarded processes, may be obtained as follows.

\begin{eqnarray}
\bangp{\prefix{u}{v}{P}} 
	:= 
	\binpar{\lift{x}{\prefix{u}{v}{(\binpar{D(x)}{P})}}}{D(x)} \nonumber
\end{eqnarray}

\begin{remark}
  Note that the lazier definition still does not deal with summation
  or mixed summation (i.e. sums over input and output). The reader is
  invited to construct definitions of replication that deal with these
  features. 

  Further, the definitions are parameterized in a name, $x$. Can you,
  gentle reader, make a definition that eliminates this parameter and
  guarantees no accidental interaction between the replication
  machinery and the process being replicated -- i.e. no accidental
  sharing of names used by the process to get its work done and the
  name(s) used by the replication to effect copying. This latter
  revision of the definition of replication is crucial to obtaining
  the expected identity $!!P \sim !P$.
\end{remark}

\begin{remark}\label{rem:paradoxical_combinator}
  The reader familiar with the lambda calculus will have noticed the
  similarity between $D$ and the paradoxical combinator.

  [Ed. note: the existence of this seems to suggest we have to be more
  restrictive on the set of processes and names we admit if we are to
  support no-cloning.]
\end{remark}

\subsubsection{Bisimulation}

The computational dynamics gives rise to another kind of equivalence,
the equivalence of computational behavior. As previously mentioned
this is typically captured \emph{via} some form of bisimulation.

% The notion we use in this paper is weak barbed bisimulation
% \cite{milner91polyadicpi}.

The notion we use in this paper is derived from weak barbed
bisimulation \cite{milner91polyadicpi}. 

\begin{definition}
An \emph{observation relation}, $\downarrow_{\mathcal N}$, over a set
of names, $\mathcal N$, is the smallest relation satisfying the rules
below.

\infrule[Out-barb]{y \in {\mathcal N}, \; x \nameeq y}
		  {\outputp{x}{v} \downarrow_{\mathcal N} x}
\infrule[Par-barb]{\mbox{$P\downarrow_{\mathcal N} x$ or $Q\downarrow_{\mathcal N} x$}}
		  {\binpar{P}{Q} \downarrow_{\mathcal N} x}

We write $P \Downarrow_{\mathcal N} x$ if there is $Q$ such that 
$P \wred Q$ and $Q \downarrow_{\mathcal N} x$.
\end{definition}

\begin{definition}
%\label{def.bbisim}
An  ${\mathcal N}$-\emph{barbed bisimulation} over a set of names, ${\mathcal N}$, is a symmetric binary relation 
${\mathcal S}_{\mathcal N}$ between agents such that $P\rel{S}_{\mathcal N}Q$ implies:
\begin{enumerate}
\item If $P \red P'$ then $Q \wred Q'$ and $P'\rel{S}_{\mathcal N} Q'$.
\item If $P\downarrow_{\mathcal N} x$, then $Q\Downarrow_{\mathcal N} x$.
\end{enumerate}
$P$ is ${\mathcal N}$-barbed bisimilar to $Q$, written
$P \wbbisim_{\mathcal N} Q$, if $P \rel{S}_{\mathcal N} Q$ for some ${\mathcal N}$-barbed bisimulation ${\mathcal S}_{\mathcal N}$.
\end{definition}

$\mathcal{R} \subseteq \pi \times \pi$

$P \mathcal{R} Q => \forall P'. P \red P' \Rightarrow \exists Q'. Q \red Q', P' \mathcal{R} Q'$

$P \vdash x \Rightarrow Q \vdash x$

\begin{mathpar}
  \inferrule*[lab=Out-barb]{x \nameeq y}{{y}!\langle{Q}\rangle \vdash x}
  \and
  \inferrule*[lab=Par-barb]{\mbox{$P\vdash x$ or $Q\vdash x$}}{\binpar{P}{Q} \vdash x}
\end{mathpar}

\subsubsection{Contexts}

One of the principle advantages of computational calculi like the
$\pi$-calculus is a well-defined notion of context,
contextual-equivalence and a correlation between
contextual-equivalence and notions of bisimulation. The notion of
context allows the decomposition of a process into (sub-)process and
its syntactic environment, its context. Thus, a context may be
thought of as a process with a ``hole'' (written $\Box$) in it. The
application of a context $M$ to a process $P$, written $M[P]$, is
tantamount to filling the hole in $M$ with $P$. In this paper we do
not need the full weight of this theory, but do make use of the notion
of context in the proof the main theorem. 

\begin{mathpar}
  \inferrule* [lab=summation] {} {{M_{M},M_{N}} \bc \Box \;|\; x.M_{A} \;|\; M_{M}+M_{N}}
  \and
  \inferrule* [lab=agent] {} {{M_{A}} \bc (\vec{x})M_{P} \;| \; \clift{P_0,\ldots,M_{P},\ldots,P_N}}
  \and \\
  \inferrule* [lab=process] {} {{M_{P}} \bc M_{N} \;| \;P|M_{P} }
\end{mathpar} 

\begin{mathpar}
  \inferrule* [lab=sychronization] {} {M_{N} \bc \Box \;|\; x?M_{F} \;|\; x!M_{C}}
  \and
  \inferrule* [lab=abstraction] {} {{M_{F}} \bc (x)M_{P} }
  \and
  \inferrule* [lab=concretion] {} {{M_{C}} \bc \langle M_{P} \rangle }
  \and \\
  \inferrule* [lab=process] {} {{M_{P}} \bc M_{N} \;| \;P|M_{P} }
\end{mathpar}

\begin{definition}[contextual application] Given a context $M$, and
  process $P$, we define the \emph{contextual application}, $M[P] :=
  M\{P/\Box\}$. That is, the contextual application of M to P is the
  substitution of $P$ for $\Box$ in $M$.
\end{definition}

$\meaningof{-} : L \to \mathcal{P}(\pi)$

\begin{mathpar}
  \inferrule* [lab=collection] {} {\meaningof{true} = \pi, \and \meaningof{~E} = \pi \setminus \meaningof{E}, \and \meaningof{E_{1} \& E_{2}} = \meaningof{E_{1}} \cap \meaningof{E_{2}}}
\end{mathpar}

\begin{mathpar}
  \inferrule* [lab=structure] {} {\meaningof{0} = \{ P \in \pi | P \equiv 0 \}, \and \\ \meaningof{E_1 | E_2} = \{ P \in \pi | P \equiv P_{1} | P_{2}, P_{1} \in \meaningof{E_{1}}, P_{2} \in \meaningof{E_2}\} }
\end{mathpar}

\begin{mathpar}
 \inferrule* [lab=behavior] {} {\meaningof{\langle a?b \rangle E} = \{ P \in \pi | P \equiv Q | u?(y)P', \\ \and \\\\ \and \\ \;\;\; u \in \meaningof{a}, \forall z.P'\{z/y\} \in \meaningof{E\{z/b\}}\}, \and \\ \meaningof{a!E} = \{ P \in \pi | P \equiv Q | x!\langle P' \rangle, x \in \meaningof{a} P' \in \meaningof{E}\} }
\end{mathpar}

\begin{mathpar}
 \inferrule* [lab=nominal] {} {\meaningof{\quotep{E}} = \{ \quotep{P} \in \quotep{\pi} | P \in \meaningof{E} \}, \and \meaningof{\quotep{P}} = \{ \quotep{Q} \in \quotep{\pi} | P \equiv Q \} \and \\ \meaningof{@\quotep{E}} = \{ P \in \pi | P \equiv @x, x \in \meaningof{E} \}}
\end{mathpar}

\begin{eqnarray*}
  \\
  \meaningof{-} : TS \to ST
\end{eqnarray*}

\begin{eqnarray*}
  \\
  L : TS \to ST
\end{eqnarray*}

\begin{eqnarray*}
  \\
  P \models E \iff P \in \meaningof{E}
\end{eqnarray*}

\begin{eqnarray*}
  P \approx_{L} Q \iff \forall E \in L. P \models E \iff Q \models E
\end{eqnarray*}

\begin{eqnarray*}
  P \approx_{K} Q
\end{eqnarray*}

\begin{eqnarray*}
  P \approx Q
\end{eqnarray*}

$\approx_{K} = \approx = \approx_{L}$

\subsubsection{Contextual duality}

Note that contexts extend the quotation operation to a family of
operations from processes to names. Given a context, $M$, we can
define a \emph{nominal context}, $\quotep{M}$ by $\quotep{M}[P] :=
\quotep{M[P]}$. To foreshadow what is to come we observe that these
operations enjoy a duality with processes very much like the duality
between vectors and maps from vectors to scalars.

Further, because the calculus is essentially higher-order, we have a
correspondence between contexts and processes. More specifically,
given a name $x$ and a context $M$ we can construct $M^{*}_{x}$ such
that 

\begin{mathpar}
  M^{*}_{x} | \lift{x}{P} \red M[P]
\end{mathpar}

namely,

\begin{mathpar}
  M^{*}_{x} := x?(u).M[\dropn{u}]
\end{mathpar}

The dependence of $M^{*}_{x}$ on a name makes it an abstraction, 

\begin{mathpar}
  M^{*} := (x)x?(u).M[\dropn{u}]
\end{mathpar}

\subsection{Additional notation}

It will sometimes be convenient to denote the process a name
quotes. We already have the notation $x = \quotep{P}$, but it will be
convenient to introduce an alternate notation, $\procn{x}$, when we
want to emphasize the connection to the use of the name. Note that, by
virtue of name equivalence, $\quotep{\procn{x}} \nameeq x$; so, the
notation is consistent with previous definitions.

Further, because names have structure it is possible to effect
substitutions on the basis of that structure. This means we need to
upgrade our notation for substitutions, which we accomplish by
adapting comprehension notation. Thus,

\begin{mathpar}
  P\{ y / x : x \in S \}
\end{mathpar}

is interpreted to mean the process derived from P by replacing (in a
capture-avoiding manner) each occurrence of $x$ in $S$ by $y$. For example,

\begin{mathpar}
  P\{ \quotep{\procn{x}|\procn{x}} / x : x \in \freenames{P} \}
\end{mathpar}

will replace each (occurrence) of a free name $x$ in $P$ by
$\quotep{\procn{x}|\procn{x}}$.

Also, we will avail ourselves of the notation $x^{L}$ and $x^{R}$ to
denote injections of a name into disjoint copies of the name
space. There are numerous ways to accomplish this. One example can be
found in \cite{MeredithR05}. This notation overloads to vectors of
names: $\vec{x}^{\pi} := (x_{i}^{\pi} \; : \; 0 \leq i < |\vec{x}| )$ where $\pi \in \{L,R\}$.

We also use $P^{\Box} := P|\Box$.

In \cite{MeredithR05} an interpretation of the new operator is
given. It turns out that there are several possible interpretations
all enjoying the requisite algebraic properties of the operator (see
\cite{milner91polyadicpi}). We will therefore make liberal use of
$(\nu\; \vec{x})P$.

% subsection the_syntax_and_semantics_of_the_notation_system (end)   

\input{qm2pi.qmops} 

\input{qm2pi.sterngerlach} 

\input{qm2pi.metric} 

% section concurrent_process_calculi (end)

%\input{qm2pi.proofsketch}

% section proof sketch (end)

%\input{qm2pi.slviaknots} 

% section spatial logic via knots (end)

\input{qm2pi.conclusion}

% section conclusion (end)

%\input{qm2pi.dtcodes} 

% section wiring algorithm (end)

\input{qm2pi.ack} 

% section acknowledgments (end)

\newpage


\bibliographystyle{plain}   
\bibliography{../../biblios/main.bib}

\input{qm2pi.rhodetails}

\end{document}



% section proof sketch (end)

%\section{Unlikely characters: spatial logic for
  knots}\label{sub:characteristic_formulae} % (fold)

Associated to the mobile process calculi are a family of logics known
as the Hennessy-Milner logics. These logics typically enjoy a
semantics interpreting formulae as sets of processes that when
factored through the encoding outlined above allows an identification
of classes of knots with logical formulae. In the context of this
encoding the sub-family known as the spatial logics \cite{CairesC03}
\cite{CairesC04} \cite{Caires04} are of particular interest providing
several important features for expressing and reasoning about
properties (i.e. classes) of knots. We hint here at how this may be done.

%\begin{description}
%\item [structural connectives] 
\subsubsection{Structural connectives} The spatial logics enjoy
structural connectives corresponding, at the logical level, to the
parallel composition ($P | Q$) and new name ($(\nu \; x)P$)
connectives for processes. As illustrated in the examples below, these
connectives are extremely expressive given the shape of our encoding.
%\item [decideable satisfaction]

\subsubsection{Decideable satisfaction}
In \cite{Caires04} the satisfaction relation is shown to be decideable
for a rich class of processes. It further turns out that the image of
the our encoding is a proper subset of that class. This result
provides the basis for an algorithm by which to search for knots
enjoying a given property.
%\item [characteristic formulae]

\subsubsection{Characteristic formulae}
In the same paper \cite{Caires04} , Caires presents a means of calculating
characteristic formulae, selecting equivalence classes of processes
up to a pre--specified depth limit on the support set of names. Composed with our
encoding, this characteristic formula can be used to select
characteristic formulae for knots.
%\end{description}

\subsubsection{Spatial logic formulae}

The grammar below (segmented for comprehension) summarizes the syntax
of spatial logic formulae. We employ illustrative examples in the
sequel to provide an intuitive understanding of their meaning
referring the reader to \cite{Caires04} for a more detailed explication
of the semantics.

\begin{mathpar}
  \inferrule* [lab=boolean] {} {{A,B} \bc T \;|\; \neg A \;|\; A \wedge B \;|\; \eta = \eta'}
  \and
  \inferrule* [lab=spatial] {} {|\; \pzero \;|\; A | B \;|\; x \text{\textregistered} A \;|\; \forall x . A \;|\;  H x . A}
  \and
  \inferrule* [lab=behavioral] {} {|\; \alpha . A}
  \and 
  \inferrule* [lab=recursion] {} {|\; X(\vec{u}) \;|\; \mu X(\vec{u}) . A}
  \and
  \inferrule* [lab=action] {} {\alpha \bc \langle x?(\vec{y}) \rangle \;|\; \langle x!(\vec{y}) \rangle \;|\; \langle \tau \rangle}
  \and 
  \inferrule* [lab=name] {} {\eta \bc x \;|\; \tau}
\end{mathpar} 

% subsection characteristic_formulae (end)   	 

\subsection{Example formulae}\label{sub:example_formulae_} % (fold)

\subsubsection{Crossing as formula.}
% 
% \begin{align*}
%   \frac{d}{dx} \sin x &= \cos x 
%   & \frac{d}{dx} e^x &= e^x \\
%   \frac{d}{dx} \cos x &= - \sin x 
%   & \frac{d}{dx} \log x &= \frac{1}{x} \\
% \end{align*} 

\begin{align*}
 \mu C(x_{0},x_{1},y_{0},y_{1},u).&(\langle x_{0}?(z) \rangle(\langle u! \rangle\langle y_{1}!z \rangle C(x_{0},x_{1},y_{0},y_{1},u)) & \\
  & \wedge \langle y_{1}?(z) \rangle (\langle u! \rangle \langle x_{0}!z \rangle C(x_{0},x_{1},y_{0},y_{1},u)) & \\
  & \wedge \langle x_{1}?(z) \rangle (\langle u? \rangle \langle y_{0}!z \rangle C(x_{0},x_{1},y_{0},y_{1},u)) & \\
  & \wedge \langle y_{0}?(z) \rangle (\langle u? \rangle \langle x_{1}!z \rangle C(x_{0},x_{1},y_{0},y_{1},u))) &
\end{align*}

The lexicographical similarity between the shape of this formulae and
the shape of definition of the process representing a crossing reveals
the intuitive meaning of this formulae. It describes the capabilities
of a process that has the right to represent a crossing. For example
it picks out processes that may perform an input on the port $x_0$ in
its initial menu of capabilities. What differentiates the formula
from the process, however, is that the crossing process is the
smallest candidate to satisfy the formula. Infinitely many other
processes -- with internal behavior hidden behind this interface, so
to speak -- also satisfy this formula. Even this simple formula,
then, can be seen to open a new view onto knots, providing a
computational interpretation of \emph{virtual} knots.

Note that this formula is derived by hand. A similar formula can be
derived by employing Caires' calculation of characteristic formula
\cite{Caires04} to the process representing a crossing. In light of
this discussion, we let
$\meaningof{C}_{\phi}(x0,x1,y0,y1,u)$ denote a formula specifying the
dynamics we wish to capture of a crossing. To guarantee we preserve
the shape of the interface and minimal semantics we demand that
$\meaningof{C}_{\phi}(x0,x1,y0,y1,u) \Rightarrow
\textbf{C}(x0,x1,y0,y1,u)$ where $\textbf{C}(x0,x1,y0,y1,u)$ denotes
the formula above.
                            
\subsubsection{Crossing number constraints.}
The moral content of the context lemma (Lemma \ref{context}) is that the notion of
``locality'' in the Reidemeister moves is effectively captured by the
parallel composition operator of the process calculus. This intuition
extends through the logic. Given a formula,
$\meaningof{C}_{\phi}(x0,x1,y0,y1,u)$, we can use the structural
connectives to specify constraints on crossing numbers, such as at
least $n$ crossings, or exactly $n$ crossings.
\begin{mathpar}
  \inferrule* [lab=at-least-n] {} { K^{\geq n}_{\phi}(\vec{xs},\vec{ys}) := \Pi_{i=0}^{n-1} Hu . \meaningof{C}_{\phi}(xs_i,ys_i,u) | T }
  \and 
  \inferrule* [lab=exactly-n] {} { K^{= n}_{\phi}(\vec{xs},\vec{ys}) := \Pi_{i=0}^{n-1} Hu . \meaningof{C}_{\phi}(xs_i,ys_i,u) | \neg (\forall x_0,y_0,x_1,y_1,u . \meaningof{C}_{\phi}(x_0,y_0,x_1,y_1,u) | T) }
\end{mathpar}

To round out this section, recall that the encoding of an $n$-crossing
knot decomposes into a parallel composition of $n$ \emph{copies} of a
crossing process together with a wiring harness. To specify different
knot classes with the same crossing number amounts to specifying
logical constraints on the wiring harness. In the interest of space,
we defer examples to a forthcoming paper. Suffice it to say that both
the conditions ``alternating knot'' and ``contains the tangle
corresponding to 5/3'' are expressible. For example, it is possible to
calculate the characteristic formula of a process corresponding to the
tangle 5/3 and conjoin it into the classifying formula via the
composition connective of the logic.

Finally, we wish to observe that it is entirely within reason to
contemplate a more domain-specific version of spatial logic tailored
to the shape of processes in the image of the encoding. Such a
domain-specific logic would have a better claim to the title formal
language of knot properties.

% subsection example_formulae_ (end)

% section knots_as_processes (end) 

% section spatial logic via knots (end)

\section{Conclusions and future work}

\paragraph{Testing physical space}
You, gentle reader, may wonder why of all the theorems to be proved
given this set up we pick the one above. In some sense it's hardly
central to quantum mechanics. We see it as central in the sense that
it firmly establishes a notion of physical space arising from a notion
of the equivalence of behavior. Relating bisimulation to a metric is a
big step forward, but one is faced with interpreting the relationship
of that metric space to something more physical. Quantum mechanical
notions of ``physical'' space are still far from intuitive, but by
relating this idea of distance as testing to calculations that predict
physical circumstances we are making a not insignificant step forward
toward an understanding of the physical space we inhabit as
essentially dynamic.

\paragraph{Effectivity and simulation}
One of the observations we have yet to make is that the entire program
spelled out here is effective. We have built various interpreters for
the reflective calculus at work in this interpretation. In principle,
then, we can simulate quantum mechanics on a computer. The place where
the simulation may lose fidelity is the infinitely branching summation
for the annihilator.

In this connection i also want to point out that the evaluation style
calculation of the inner product puts the non-determinism of the
summation right at the heart of measurement. This suggests that
Milner's original reduction-based formulation of the dynamics of his
calculi in terms of sums was not just notationally suggestive of a
notion of measure-and-continue but captured some significant part of
the physics.

\paragraph{Quantum continuations}
In light of this last observation i want to point out that the
predominant account of quantum mechanics is missing a key aspect of a
truly compositional story of the physical situation. In a real lab,
when a measurement is made the observation can be made to feed into
another device that then makes another measurement conditioned on the
results of the first. This means that after the superposition was
collapsed the entire experimental set up remained in
superposition. While QM offers a means of writing this down it doesn't
quite line up well with the well-trodden formulation of computation
and continuation that we see so succinctly expressed in Milner's
calculi. This suggests that there might be advantages to this account
of dynamics waiting to be explored.

\paragraph{Quantum logic}
In this connection, we also note that by virtue of having the
Hennessy-Milner construction, we can pull the construction through the
interpretation of QM. This gives us a natural candidate for a quantum
logic that enjoys an extremely tight connection with it's domain of
interpretation, making the construction much less ad hoc (rather it is
the image of functor!).

\paragraph{Quantum probabiity}
i have questions about the basis of the interpretation of inner
product as probability amplitude. In particular, using which
axiomatization of probability theory does the notion of probability
amplitude earn the right to be so dubbed? In other words, where is the
proof that the operation for calculating a probability amplitude (and
then squaring) satisfies the axioms of what it means to calculate a
probability? Even if such a proof exists (i have yet to find it in the
literature), i wonder if it might not be possible to turn things on
their heads. Can we view the calculation of the probability amplitude
as an axiomatization of probability? If so, then the definition we
give for calculating probability amplitude may provide the basis for
an \emph{effective} theory of probability.

\paragraph{Quantum vs ``biological'' information}
Finally, i want to conclude with a more philosophical observation. At
a recent workshop in which QM was a predominant topic i noticed
something about quantum information. The speaker was giving a riveting
discussion of axiomatic QM and showing how properties of ``no
cloning'' and ``no deleting'' emerged as consequences of the
axiomatization. Theorems of this form are necessary to give us a sense
of confidence that our axioms characterize the physical theory. What
struck me, though, was that if quantum information is neither erasable
nor replicable it is markedly different from \emph{life}. Two of the
things we know about life is that

\begin{itemize}
  \item it ends;
  \item to gain some measure of persistence, to transcend it's
    finitude it is imminently copyable.
\end{itemize}

Both of these qualities are summarized succinctly in the aphorism: all
flesh is grass. For me these two kinds of ``information'' -- call them
quantum and biological -- are end points on a spectrum of strategies
for persistence. At one end, we have those curious entities that enjoy
uniqueness and permanence; at the other, we have those who in the face
of a certain end and an uncertain present make a go of passing
something on. To me one of the more remarkable aspects of the latter
strategy is that in the presence of noise (and certain features of
copying) we get a kind of dynamism, a chance for improvement against a
given persistent condition.

% subsection other_calculi_other_bisimulations_and_geometry_as_behavior (end)




% section conclusion (end)

%\documentclass[12pt]{llncs}
%\documentclass{jktr}

\usepackage[pdftex]{hyperref}                   
\usepackage {listings}
\usepackage {mathpartir}
\usepackage{bcprules}
%\usepackage{listings}
                       
\usepackage{graphicx} 
%\usepackage[margins=2.5cm,nohead,nofoot]{geometry}
%\usepackage{geometry}
\usepackage{amsfonts}
\usepackage{amstext}
\usepackage{latexsym}
\usepackage{amssymb}
\usepackage{color}


%\include{myPreamble}
\include{qm2pi.local} 

%\ifpdf
%\usepackage[pdftex]{graphicx}
%\else
%\usepackage{graphicx}
%\fi

 % \ifpdf
%  \usepackage{pdfsync}
%  \if


%\title{Brief Article}
%\author{David F. Snyder}
%\author{L.G. Meredith}

%\address{Dept. of Math., Texas State University--San Marcos, San Marcos, TX 78666}
       
\pagestyle{empty}


\begin{document}

\lstset{language=[Objective]Caml,frame=shadowbox}

\input{qm2pi.front}

% section front matter (end)

\input{qm2pi.intro} 
 
% section introduction (end)

% \input{qm2pi.knotations} 

% section notation (end)

\input{qm2pi.process.calculi} 

% section concurrent_process_calculi_and_spatial_logics_ (end)
    
%\input{qm2pi.knots2pi} 

%\input{qm2pi.trefoil} 

%\input{qm2pi.mainthm} 

% subsection basic_interpretation (end)

%\input{qm2pi.rho.presentation} 
\subsection{The syntax and semantics of the notation system}\label{sub:the_syntax_and_semantics_of_the_notation_system} % (fold)

We now summarize a technical presentation of the calculus that
embodies our theory of dynamics. The typical presentation of such a
calculus follows the style of giving generators and relations on
them. The grammar, below, describing term constructors, freely
generates the set of processes, $\Proc$. This set is then quotiented
by a relation known as structural congruence and it is over this set
that the notion of dynamics is expressed. This presentation is
essentially that of \cite{MeredithR05} with the addition of
polyadicity and summation. For readability we have relegated some of
the technical subtleties to an appendix.

\subsubsection{Process grammar}\label{subsub:process_grammar}

\begin{mathpar}
  \inferrule* [lab=synchronization] {} {{M} \bc \pzero \;|\; x?F \;|\; x!C }
  \and
  \inferrule* [lab=abstraction] {} {{F} \bc (x)P}
  \and
  \inferrule* [lab=concretion] {} {{C} \bc \langle Q \rangle}
  \and
  \inferrule* [lab=process] {} {{P,Q} \bc M \;| \;P|Q \;|\; @{x}}
  \and
  \inferrule* [lab=name] {} {{x} \bc \quotep{P}}
\end{mathpar} 

Note that $\vec{x}$ (resp. $\vec{P}$) denotes a vector of names
(resp. processes) of length $|\vec{x}|$ (resp. $|\vec{P}|$). We adopt
the following useful abbreviations.

\begin{mathpar}
   x?(\vec{y}).P := x.(\vec{y})P \and  x\clift{\vec{P}} := x.\clift{\vec{P}}
   \and x!(y) := \lift{x}{\dropn{y}}
   \and \Pi_{i=0}^{n-1}P_i := P_0 | \ldots | P_{n-1}
\end{mathpar}

\subsubsection{Structural congruence}

\paragraph{Free and bound names and alpha-equivalence.} At the
core of structural equivalence is alpha-equivalence which identifies
process that are the same up to a change of variable. Formally, we
recognize the distinction between free and bound names. The free names
of a process, $\freenames{P}$, may be calculated recursively as
follows:

\begin{mathpar}
\freenames{\pzero} := \emptyset
  \and \\
  \freenames{x?(y).P} := \{ x \} \cup (\freenames{P} \setminus \{ y \})
  \and 
  \freenames{x!\langle P \rangle} := \{ x \} \cup \{ P \} 
  \and \\
  \freenames{P|Q} := \freenames{P} \cup \freenames{Q}
  \and \\
  \freenames{@{x}} := \{ x \}
\end{mathpar}

$\pi$
$\quotep{\pi}$

$\freenames{-} : \pi \to \mathcal{P}(\quotep{\pi})$

\begin{eqnarray*}
  \freenames{\pzero} & := & \emptyset \\
  \freenames{x?(y).P} & := & \{ x \} \cup (\freenames{P} \setminus \{ y \}) \\
  \freenames{x!\langle P \rangle} & := & \{ x \} \cup \{ P \} \\
  \freenames{P|Q} & := & \freenames{P} \cup \freenames{Q} \\
  \freenames{\dropn{x}} & := & \{ x \}
\end{eqnarray*}

The bound names of a process, $\boundnames{P}$, are those names occurring in $P$
that are not free. For example, in $x?(y).0$, the name $x$ is free, while $y$ is bound.

\begin{mathpar}
  \inferrule* [lab=monoidal-laws] {} { P|Q \equiv Q|P \and P|0 \equiv P \and P|(Q|R) \equiv (P|Q)|R }
\end{mathpar}

\begin{mathpar}
  \inferrule* [lab=alpha-equivalence] {} { (x)P \equiv (y)P\{y/x\} \and y \not\in \freenames{P} }
\end{mathpar}

\begin{definition}
Then two processes, $P,Q$, are alpha-equivalent if $P = Q\{\vec{y}/\vec{x}\}$ for
some $\vec{x} \in \boundnames{Q},\vec{y} \in \boundnames{P}$, where $Q\{\vec{y}/\vec{x}\}$
denotes the capture-avoiding substitution of $\vec{y}$ for $\vec{x}$ in $Q$.
\end{definition}

\begin{definition}
  The {\em structural congruence} \cite{SangiorgiWalker} , $\equiv$,
  between processes is the least congruence containing
  alpha-equivalence, satisfying the abelian monoid laws
  (associativity, commutativity and $\pzero$ as identity) for parallel
  composition $|$ and for summation $+$.
\end{definition}

\subsection{Name equivalence}

We take name equivalence, written $\nameeq$, to be the smallest
equivalence relation generated by the following rules.

\begin{mathpar}
\inferrule*[lab=Quote-drop]
{ }
{ \quotep{@{x}} \nameeq x }

\inferrule*[lab=Struct-equiv]
{ P \scong Q }
{ \quotep{P} \nameeq \quotep{Q} }
\end{mathpar}

The astute reader will have noticed that the mutual recursion of names
and processes imposes a mutual recursion on alpha-equivalence and
structural equivalence via name-equivalence. Fortunately, all of this
works out pleasantly and we may calculate in the natural way, free of
concern. The reader interested in the details is referred to the
appendix \ref{appendix:rho_details}.

\subsection{Substitution}

We use $\Proc$ for the set of processes, $\QProc$ for the set of
names, and $\id{\{}\vec{y} / \vec{x} \id{\}}$ to denote partial maps,
$s : \QProc \rightarrow \QProc$. A map, $s$ lifts, uniquely, to a map
on process terms, $\widehat{s} : \Proc \rightarrow \Proc$ by the
following equations.

\begin{mathpar}
  (0) \psubstp{Q}{P} := 0 \\
  (R \juxtap S) \psubstp{Q}{P}
  :=    
  (R)\psubstp{Q}{P} \juxtap (S) \psubstp{Q}{P} \\
  (x?(y).R) \psubstp{Q}{P}    
  :=    
  (x)\substp{Q}{P} (z)\concat( (R \psubstn{z}{y}) \psubstp{Q}{P} ) \\
  (\lift{x}{R}) \psubstp{Q}{P}  
  :=
  \lift{(x)\substp{Q}{P}}{ R \psubstp{Q}{P} } \\
%   (\dropn{x})  \psubstp{Q}{P}       
%   := 
%   \left\{ 
%     \begin{array}{ccc} 
%       \dropn{\quotep{Q}} & & x \nameeq \quotep{P} \\
%       \dropn{x} & & otherwise \\
%     \end{array}
%   \right. 
  (\dropn{x})  \psubstp{Q}{P}       
  := 
  \left\{ 
    \begin{array}{ccc} 
      Q & & x \nameeq \quotep{P} \\
      \dropn{x} & & otherwise \\
    \end{array}
  \right.
\end{mathpar}
 

where

\begin{eqnarray}
  (x)\id{\{} \lpquote Q \rpquote / \lpquote P \rpquote \id{\}}            = 
  \left\{ 
    \begin{array}{ccc}
      \lpquote Q \rpquote & & x \nameeq \lpquote P \rpquote \\
      x & & otherwise \\
    \end{array}
  \right. \nonumber
\end{eqnarray}

and $z$ is chosen distinct from $\quotep{P}$, $\quotep{Q}$, the free
names in $Q$, and all the names in $R$. Our $\alpha$-equivalence will
be built in the standard way from this substitution.

\begin{remark}\label{rem:no_self_referential_names}
  One consequence of these definitions is that $\forall P. \quotep{P}
  \not\in \freenames{P}$.
\end{remark}

\subsection{ Dynamic quote: an example }

Anticipating something of what's to come, consider applying the
substitution, $\widehat{\id{\{}u / z \id{\}}}$, to the following pair
of processes, $\lift{w}{y!(z)}$ and $w[ \lpquote y!(z) \rpquote ]$.

\begin{eqnarray}
	\lift{w}{y!(z)}\widehat{\id{\{}u / z \id{\}}}
		& = &
		\lift{w}{y!(u)} \nonumber\\
	w[ \lpquote y!(z) \rpquote ] \widehat{ \id{\{}u / z \id{\}} }
		& = &
		w[ \lpquote y!(z) \rpquote ] \nonumber
\end{eqnarray}

Because the body of the process between quotes is impervious to
substitution, we get radically different answers. In fact, by
examining the first process in an input context,
e.g. $x?(z).\lift{w}{y!(z)}$, we see that the process under the lift
operator may be shaped by prefixed inputs binding a name inside it. In
this sense, the lift operator will be seen as a way to dynamically
construct processes before reifying them as names.

Finally equipped with these standard features we can present the
dynamics of the calculus.

\subsubsection{Operational semantics} 

Finally, we introduce the computational dynamics. What marks these
algebras as distinct from other more traditionally studied algebraic
structures, e.g. vector spaces or polynomial rings, is the manner in
which dynamics is captured. In traditional structures, dynamics is typically
expressed through morphisms between such structures, as in linear maps
between vector spaces or morphisms between rings. In algebras
associated with the semantics of computation, the dynamics is
expressed as part of the algebraic structure itself, through a
reduction reduction relation typically denoted by $\red$. Below, we
give a recursive presentation of this relation for the calculus used
in the encoding.

$\red \subseteq \pi \times \pi$
$\red : \pi \to \mathcal{P}(\pi)$

\begin{mathpar}
  \inferrule* [lab=Comm] { \textsf{match}( x_{src}, x_{trgt} ) } { x_{trgt}?(y)P \; | \; x_{src}!\langle {Q} \rangle \red P\{\quotep{Q}/y}\} }
  \and \\
  \inferrule* [lab=Par] {{P} \red {P}'} {{{P} | {Q}} \red {{P}' | {Q}}}
  \and
  \inferrule* [lab=Equiv]{{{P} \scong {P}'} \andalso {{P}' \red {Q}'} \andalso {{Q}' \scong {Q}}}{{P} \red {Q}}
\end{mathpar}

\begin{eqnarray*}
  match_{\equiv} (\quotep{P},\quotep{Q}) & := & P \equiv Q \\
  match_{\dagger}(\quotep{P},\quotep{Q}) & := & \forall R. P|Q \red^{*} R => R \red^{*} 0 \\
  match_{K}(\quotep{P},\quotep{Q}) & := & K \mbox{ for some context } K
\end{eqnarray*}

$u?(x)P | u!\langle Q \rangle \red P\{\quotep{Q}/x\}$

%We write $\wred$ for $\red^*$, and $P\red$ if $\exists Q $ such that $ P \red Q$.
We write $P\red$ if $\exists Q $ such that $ P \red Q$ and $P\not\red$, otherwise.

\section{Replication}

As mentioned before, it is known that replication (and hence
recursion) can be implemented in a higher-order process algebra
\cite{SangiorgiWalker}. As our first example of calculation with the
machinery thus far presented we give the construction explicitly in
the {\rhoc}.

\begin{eqnarray}
	D_{x} & := & \prefix{x}{y}{(\binpar{\outputp{x}{y}}{@{y}})} \nonumber\\
	\bangp_{x}{P} & := & \binpar{{x}!\langle{\binpar{D_{x}}{P}}\rangle}{D_{x}} \nonumber
\end{eqnarray}

\begin{eqnarray}
	\bangp_{x}{P} & & \nonumber\\
	=
	& {x}!\langle{(\prefix{x}{y}{(\outputp{x}{y} | @{y})) | P}}\rangle 
	      | \prefix{x}{y}{(\outputp{x}{y} | @{y})} & \nonumber\\
	\red
	& (\outputp{x}{y} | @{y})\substn{\quotep{(\prefix{x}{y}{(@{y} | \outputp{x}{y})) | P}}}{y} & \nonumber\\
	=
	& \outputp{x}{\quotep{(\prefix{x}{y}{(\outputp{x}{y} | @{y})) | P}}}
	  | {(\prefix{x}{y}{(\outputp{x}{y} | @{y})) | P}} & \nonumber\\
	\red
	& \ldots & \nonumber\\
	\red^*
	& P | P | \ldots & \nonumber
\end{eqnarray}

Of course, this encoding, as an implementation, runs away, unfolding
$\bangp{P}$ eagerly. A lazier and more implementable replication
operator, restricted to input-guarded processes, may be obtained as follows.

\begin{eqnarray}
\bangp{\prefix{u}{v}{P}} 
	:= 
	\binpar{\lift{x}{\prefix{u}{v}{(\binpar{D(x)}{P})}}}{D(x)} \nonumber
\end{eqnarray}

\begin{remark}
  Note that the lazier definition still does not deal with summation
  or mixed summation (i.e. sums over input and output). The reader is
  invited to construct definitions of replication that deal with these
  features. 

  Further, the definitions are parameterized in a name, $x$. Can you,
  gentle reader, make a definition that eliminates this parameter and
  guarantees no accidental interaction between the replication
  machinery and the process being replicated -- i.e. no accidental
  sharing of names used by the process to get its work done and the
  name(s) used by the replication to effect copying. This latter
  revision of the definition of replication is crucial to obtaining
  the expected identity $!!P \sim !P$.
\end{remark}

\begin{remark}\label{rem:paradoxical_combinator}
  The reader familiar with the lambda calculus will have noticed the
  similarity between $D$ and the paradoxical combinator.

  [Ed. note: the existence of this seems to suggest we have to be more
  restrictive on the set of processes and names we admit if we are to
  support no-cloning.]
\end{remark}

\subsubsection{Bisimulation}

The computational dynamics gives rise to another kind of equivalence,
the equivalence of computational behavior. As previously mentioned
this is typically captured \emph{via} some form of bisimulation.

% The notion we use in this paper is weak barbed bisimulation
% \cite{milner91polyadicpi}.

The notion we use in this paper is derived from weak barbed
bisimulation \cite{milner91polyadicpi}. 

\begin{definition}
An \emph{observation relation}, $\downarrow_{\mathcal N}$, over a set
of names, $\mathcal N$, is the smallest relation satisfying the rules
below.

\infrule[Out-barb]{y \in {\mathcal N}, \; x \nameeq y}
		  {\outputp{x}{v} \downarrow_{\mathcal N} x}
\infrule[Par-barb]{\mbox{$P\downarrow_{\mathcal N} x$ or $Q\downarrow_{\mathcal N} x$}}
		  {\binpar{P}{Q} \downarrow_{\mathcal N} x}

We write $P \Downarrow_{\mathcal N} x$ if there is $Q$ such that 
$P \wred Q$ and $Q \downarrow_{\mathcal N} x$.
\end{definition}

\begin{definition}
%\label{def.bbisim}
An  ${\mathcal N}$-\emph{barbed bisimulation} over a set of names, ${\mathcal N}$, is a symmetric binary relation 
${\mathcal S}_{\mathcal N}$ between agents such that $P\rel{S}_{\mathcal N}Q$ implies:
\begin{enumerate}
\item If $P \red P'$ then $Q \wred Q'$ and $P'\rel{S}_{\mathcal N} Q'$.
\item If $P\downarrow_{\mathcal N} x$, then $Q\Downarrow_{\mathcal N} x$.
\end{enumerate}
$P$ is ${\mathcal N}$-barbed bisimilar to $Q$, written
$P \wbbisim_{\mathcal N} Q$, if $P \rel{S}_{\mathcal N} Q$ for some ${\mathcal N}$-barbed bisimulation ${\mathcal S}_{\mathcal N}$.
\end{definition}

$\mathcal{R} \subseteq \pi \times \pi$

$P \mathcal{R} Q => \forall P'. P \red P' \Rightarrow \exists Q'. Q \red Q', P' \mathcal{R} Q'$

$P \vdash x \Rightarrow Q \vdash x$

\begin{mathpar}
  \inferrule*[lab=Out-barb]{x \nameeq y}{{y}!\langle{Q}\rangle \vdash x}
  \and
  \inferrule*[lab=Par-barb]{\mbox{$P\vdash x$ or $Q\vdash x$}}{\binpar{P}{Q} \vdash x}
\end{mathpar}

\subsubsection{Contexts}

One of the principle advantages of computational calculi like the
$\pi$-calculus is a well-defined notion of context,
contextual-equivalence and a correlation between
contextual-equivalence and notions of bisimulation. The notion of
context allows the decomposition of a process into (sub-)process and
its syntactic environment, its context. Thus, a context may be
thought of as a process with a ``hole'' (written $\Box$) in it. The
application of a context $M$ to a process $P$, written $M[P]$, is
tantamount to filling the hole in $M$ with $P$. In this paper we do
not need the full weight of this theory, but do make use of the notion
of context in the proof the main theorem. 

\begin{mathpar}
  \inferrule* [lab=summation] {} {{M_{M},M_{N}} \bc \Box \;|\; x.M_{A} \;|\; M_{M}+M_{N}}
  \and
  \inferrule* [lab=agent] {} {{M_{A}} \bc (\vec{x})M_{P} \;| \; \clift{P_0,\ldots,M_{P},\ldots,P_N}}
  \and \\
  \inferrule* [lab=process] {} {{M_{P}} \bc M_{N} \;| \;P|M_{P} }
\end{mathpar} 

\begin{mathpar}
  \inferrule* [lab=sychronization] {} {M_{N} \bc \Box \;|\; x?M_{F} \;|\; x!M_{C}}
  \and
  \inferrule* [lab=abstraction] {} {{M_{F}} \bc (x)M_{P} }
  \and
  \inferrule* [lab=concretion] {} {{M_{C}} \bc \langle M_{P} \rangle }
  \and \\
  \inferrule* [lab=process] {} {{M_{P}} \bc M_{N} \;| \;P|M_{P} }
\end{mathpar}

\begin{definition}[contextual application] Given a context $M$, and
  process $P$, we define the \emph{contextual application}, $M[P] :=
  M\{P/\Box\}$. That is, the contextual application of M to P is the
  substitution of $P$ for $\Box$ in $M$.
\end{definition}

$\meaningof{-} : L \to \mathcal{P}(\pi)$

\begin{mathpar}
  \inferrule* [lab=collection] {} {\meaningof{true} = \pi, \and \meaningof{~E} = \pi \setminus \meaningof{E}, \and \meaningof{E_{1} \& E_{2}} = \meaningof{E_{1}} \cap \meaningof{E_{2}}}
\end{mathpar}

\begin{mathpar}
  \inferrule* [lab=structure] {} {\meaningof{0} = \{ P \in \pi | P \equiv 0 \}, \and \\ \meaningof{E_1 | E_2} = \{ P \in \pi | P \equiv P_{1} | P_{2}, P_{1} \in \meaningof{E_{1}}, P_{2} \in \meaningof{E_2}\} }
\end{mathpar}

\begin{mathpar}
 \inferrule* [lab=behavior] {} {\meaningof{\langle a?b \rangle E} = \{ P \in \pi | P \equiv Q | u?(y)P', \\ \and \\\\ \and \\ \;\;\; u \in \meaningof{a}, \forall z.P'\{z/y\} \in \meaningof{E\{z/b\}}\}, \and \\ \meaningof{a!E} = \{ P \in \pi | P \equiv Q | x!\langle P' \rangle, x \in \meaningof{a} P' \in \meaningof{E}\} }
\end{mathpar}

\begin{mathpar}
 \inferrule* [lab=nominal] {} {\meaningof{\quotep{E}} = \{ \quotep{P} \in \quotep{\pi} | P \in \meaningof{E} \}, \and \meaningof{\quotep{P}} = \{ \quotep{Q} \in \quotep{\pi} | P \equiv Q \} \and \\ \meaningof{@\quotep{E}} = \{ P \in \pi | P \equiv @x, x \in \meaningof{E} \}}
\end{mathpar}

\begin{eqnarray*}
  \\
  \meaningof{-} : TS \to ST
\end{eqnarray*}

\begin{eqnarray*}
  \\
  L : TS \to ST
\end{eqnarray*}

\begin{eqnarray*}
  \\
  P \models E \iff P \in \meaningof{E}
\end{eqnarray*}

\begin{eqnarray*}
  P \approx_{L} Q \iff \forall E \in L. P \models E \iff Q \models E
\end{eqnarray*}

\begin{eqnarray*}
  P \approx_{K} Q
\end{eqnarray*}

\begin{eqnarray*}
  P \approx Q
\end{eqnarray*}

$\approx_{K} = \approx = \approx_{L}$

\subsubsection{Contextual duality}

Note that contexts extend the quotation operation to a family of
operations from processes to names. Given a context, $M$, we can
define a \emph{nominal context}, $\quotep{M}$ by $\quotep{M}[P] :=
\quotep{M[P]}$. To foreshadow what is to come we observe that these
operations enjoy a duality with processes very much like the duality
between vectors and maps from vectors to scalars.

Further, because the calculus is essentially higher-order, we have a
correspondence between contexts and processes. More specifically,
given a name $x$ and a context $M$ we can construct $M^{*}_{x}$ such
that 

\begin{mathpar}
  M^{*}_{x} | \lift{x}{P} \red M[P]
\end{mathpar}

namely,

\begin{mathpar}
  M^{*}_{x} := x?(u).M[\dropn{u}]
\end{mathpar}

The dependence of $M^{*}_{x}$ on a name makes it an abstraction, 

\begin{mathpar}
  M^{*} := (x)x?(u).M[\dropn{u}]
\end{mathpar}

\subsection{Additional notation}

It will sometimes be convenient to denote the process a name
quotes. We already have the notation $x = \quotep{P}$, but it will be
convenient to introduce an alternate notation, $\procn{x}$, when we
want to emphasize the connection to the use of the name. Note that, by
virtue of name equivalence, $\quotep{\procn{x}} \nameeq x$; so, the
notation is consistent with previous definitions.

Further, because names have structure it is possible to effect
substitutions on the basis of that structure. This means we need to
upgrade our notation for substitutions, which we accomplish by
adapting comprehension notation. Thus,

\begin{mathpar}
  P\{ y / x : x \in S \}
\end{mathpar}

is interpreted to mean the process derived from P by replacing (in a
capture-avoiding manner) each occurrence of $x$ in $S$ by $y$. For example,

\begin{mathpar}
  P\{ \quotep{\procn{x}|\procn{x}} / x : x \in \freenames{P} \}
\end{mathpar}

will replace each (occurrence) of a free name $x$ in $P$ by
$\quotep{\procn{x}|\procn{x}}$.

Also, we will avail ourselves of the notation $x^{L}$ and $x^{R}$ to
denote injections of a name into disjoint copies of the name
space. There are numerous ways to accomplish this. One example can be
found in \cite{MeredithR05}. This notation overloads to vectors of
names: $\vec{x}^{\pi} := (x_{i}^{\pi} \; : \; 0 \leq i < |\vec{x}| )$ where $\pi \in \{L,R\}$.

We also use $P^{\Box} := P|\Box$.

In \cite{MeredithR05} an interpretation of the new operator is
given. It turns out that there are several possible interpretations
all enjoying the requisite algebraic properties of the operator (see
\cite{milner91polyadicpi}). We will therefore make liberal use of
$(\nu\; \vec{x})P$.

% subsection the_syntax_and_semantics_of_the_notation_system (end)   

\input{qm2pi.qmops} 

\input{qm2pi.sterngerlach} 

\input{qm2pi.metric} 

% section concurrent_process_calculi (end)

%\input{qm2pi.proofsketch}

% section proof sketch (end)

%\input{qm2pi.slviaknots} 

% section spatial logic via knots (end)

\input{qm2pi.conclusion}

% section conclusion (end)

%\input{qm2pi.dtcodes} 

% section wiring algorithm (end)

\input{qm2pi.ack} 

% section acknowledgments (end)

\newpage


\bibliographystyle{plain}   
\bibliography{../../biblios/main.bib}

\input{qm2pi.rhodetails}

\end{document}

 

% section wiring algorithm (end)

\documentclass[12pt]{llncs}
%\documentclass{jktr}

\usepackage[pdftex]{hyperref}                   
\usepackage {listings}
\usepackage {mathpartir}
\usepackage{bcprules}
%\usepackage{listings}
                       
\usepackage{graphicx} 
%\usepackage[margins=2.5cm,nohead,nofoot]{geometry}
%\usepackage{geometry}
\usepackage{amsfonts}
\usepackage{amstext}
\usepackage{latexsym}
\usepackage{amssymb}
\usepackage{color}


%\include{myPreamble}
\include{qm2pi.local} 

%\ifpdf
%\usepackage[pdftex]{graphicx}
%\else
%\usepackage{graphicx}
%\fi

 % \ifpdf
%  \usepackage{pdfsync}
%  \if


%\title{Brief Article}
%\author{David F. Snyder}
%\author{L.G. Meredith}

%\address{Dept. of Math., Texas State University--San Marcos, San Marcos, TX 78666}
       
\pagestyle{empty}


\begin{document}

\lstset{language=[Objective]Caml,frame=shadowbox}

\input{qm2pi.front}

% section front matter (end)

\input{qm2pi.intro} 
 
% section introduction (end)

% \input{qm2pi.knotations} 

% section notation (end)

\input{qm2pi.process.calculi} 

% section concurrent_process_calculi_and_spatial_logics_ (end)
    
%\input{qm2pi.knots2pi} 

%\input{qm2pi.trefoil} 

%\input{qm2pi.mainthm} 

% subsection basic_interpretation (end)

%\input{qm2pi.rho.presentation} 
\subsection{The syntax and semantics of the notation system}\label{sub:the_syntax_and_semantics_of_the_notation_system} % (fold)

We now summarize a technical presentation of the calculus that
embodies our theory of dynamics. The typical presentation of such a
calculus follows the style of giving generators and relations on
them. The grammar, below, describing term constructors, freely
generates the set of processes, $\Proc$. This set is then quotiented
by a relation known as structural congruence and it is over this set
that the notion of dynamics is expressed. This presentation is
essentially that of \cite{MeredithR05} with the addition of
polyadicity and summation. For readability we have relegated some of
the technical subtleties to an appendix.

\subsubsection{Process grammar}\label{subsub:process_grammar}

\begin{mathpar}
  \inferrule* [lab=synchronization] {} {{M} \bc \pzero \;|\; x?F \;|\; x!C }
  \and
  \inferrule* [lab=abstraction] {} {{F} \bc (x)P}
  \and
  \inferrule* [lab=concretion] {} {{C} \bc \langle Q \rangle}
  \and
  \inferrule* [lab=process] {} {{P,Q} \bc M \;| \;P|Q \;|\; @{x}}
  \and
  \inferrule* [lab=name] {} {{x} \bc \quotep{P}}
\end{mathpar} 

Note that $\vec{x}$ (resp. $\vec{P}$) denotes a vector of names
(resp. processes) of length $|\vec{x}|$ (resp. $|\vec{P}|$). We adopt
the following useful abbreviations.

\begin{mathpar}
   x?(\vec{y}).P := x.(\vec{y})P \and  x\clift{\vec{P}} := x.\clift{\vec{P}}
   \and x!(y) := \lift{x}{\dropn{y}}
   \and \Pi_{i=0}^{n-1}P_i := P_0 | \ldots | P_{n-1}
\end{mathpar}

\subsubsection{Structural congruence}

\paragraph{Free and bound names and alpha-equivalence.} At the
core of structural equivalence is alpha-equivalence which identifies
process that are the same up to a change of variable. Formally, we
recognize the distinction between free and bound names. The free names
of a process, $\freenames{P}$, may be calculated recursively as
follows:

\begin{mathpar}
\freenames{\pzero} := \emptyset
  \and \\
  \freenames{x?(y).P} := \{ x \} \cup (\freenames{P} \setminus \{ y \})
  \and 
  \freenames{x!\langle P \rangle} := \{ x \} \cup \{ P \} 
  \and \\
  \freenames{P|Q} := \freenames{P} \cup \freenames{Q}
  \and \\
  \freenames{@{x}} := \{ x \}
\end{mathpar}

$\pi$
$\quotep{\pi}$

$\freenames{-} : \pi \to \mathcal{P}(\quotep{\pi})$

\begin{eqnarray*}
  \freenames{\pzero} & := & \emptyset \\
  \freenames{x?(y).P} & := & \{ x \} \cup (\freenames{P} \setminus \{ y \}) \\
  \freenames{x!\langle P \rangle} & := & \{ x \} \cup \{ P \} \\
  \freenames{P|Q} & := & \freenames{P} \cup \freenames{Q} \\
  \freenames{\dropn{x}} & := & \{ x \}
\end{eqnarray*}

The bound names of a process, $\boundnames{P}$, are those names occurring in $P$
that are not free. For example, in $x?(y).0$, the name $x$ is free, while $y$ is bound.

\begin{mathpar}
  \inferrule* [lab=monoidal-laws] {} { P|Q \equiv Q|P \and P|0 \equiv P \and P|(Q|R) \equiv (P|Q)|R }
\end{mathpar}

\begin{mathpar}
  \inferrule* [lab=alpha-equivalence] {} { (x)P \equiv (y)P\{y/x\} \and y \not\in \freenames{P} }
\end{mathpar}

\begin{definition}
Then two processes, $P,Q$, are alpha-equivalent if $P = Q\{\vec{y}/\vec{x}\}$ for
some $\vec{x} \in \boundnames{Q},\vec{y} \in \boundnames{P}$, where $Q\{\vec{y}/\vec{x}\}$
denotes the capture-avoiding substitution of $\vec{y}$ for $\vec{x}$ in $Q$.
\end{definition}

\begin{definition}
  The {\em structural congruence} \cite{SangiorgiWalker} , $\equiv$,
  between processes is the least congruence containing
  alpha-equivalence, satisfying the abelian monoid laws
  (associativity, commutativity and $\pzero$ as identity) for parallel
  composition $|$ and for summation $+$.
\end{definition}

\subsection{Name equivalence}

We take name equivalence, written $\nameeq$, to be the smallest
equivalence relation generated by the following rules.

\begin{mathpar}
\inferrule*[lab=Quote-drop]
{ }
{ \quotep{@{x}} \nameeq x }

\inferrule*[lab=Struct-equiv]
{ P \scong Q }
{ \quotep{P} \nameeq \quotep{Q} }
\end{mathpar}

The astute reader will have noticed that the mutual recursion of names
and processes imposes a mutual recursion on alpha-equivalence and
structural equivalence via name-equivalence. Fortunately, all of this
works out pleasantly and we may calculate in the natural way, free of
concern. The reader interested in the details is referred to the
appendix \ref{appendix:rho_details}.

\subsection{Substitution}

We use $\Proc$ for the set of processes, $\QProc$ for the set of
names, and $\id{\{}\vec{y} / \vec{x} \id{\}}$ to denote partial maps,
$s : \QProc \rightarrow \QProc$. A map, $s$ lifts, uniquely, to a map
on process terms, $\widehat{s} : \Proc \rightarrow \Proc$ by the
following equations.

\begin{mathpar}
  (0) \psubstp{Q}{P} := 0 \\
  (R \juxtap S) \psubstp{Q}{P}
  :=    
  (R)\psubstp{Q}{P} \juxtap (S) \psubstp{Q}{P} \\
  (x?(y).R) \psubstp{Q}{P}    
  :=    
  (x)\substp{Q}{P} (z)\concat( (R \psubstn{z}{y}) \psubstp{Q}{P} ) \\
  (\lift{x}{R}) \psubstp{Q}{P}  
  :=
  \lift{(x)\substp{Q}{P}}{ R \psubstp{Q}{P} } \\
%   (\dropn{x})  \psubstp{Q}{P}       
%   := 
%   \left\{ 
%     \begin{array}{ccc} 
%       \dropn{\quotep{Q}} & & x \nameeq \quotep{P} \\
%       \dropn{x} & & otherwise \\
%     \end{array}
%   \right. 
  (\dropn{x})  \psubstp{Q}{P}       
  := 
  \left\{ 
    \begin{array}{ccc} 
      Q & & x \nameeq \quotep{P} \\
      \dropn{x} & & otherwise \\
    \end{array}
  \right.
\end{mathpar}
 

where

\begin{eqnarray}
  (x)\id{\{} \lpquote Q \rpquote / \lpquote P \rpquote \id{\}}            = 
  \left\{ 
    \begin{array}{ccc}
      \lpquote Q \rpquote & & x \nameeq \lpquote P \rpquote \\
      x & & otherwise \\
    \end{array}
  \right. \nonumber
\end{eqnarray}

and $z$ is chosen distinct from $\quotep{P}$, $\quotep{Q}$, the free
names in $Q$, and all the names in $R$. Our $\alpha$-equivalence will
be built in the standard way from this substitution.

\begin{remark}\label{rem:no_self_referential_names}
  One consequence of these definitions is that $\forall P. \quotep{P}
  \not\in \freenames{P}$.
\end{remark}

\subsection{ Dynamic quote: an example }

Anticipating something of what's to come, consider applying the
substitution, $\widehat{\id{\{}u / z \id{\}}}$, to the following pair
of processes, $\lift{w}{y!(z)}$ and $w[ \lpquote y!(z) \rpquote ]$.

\begin{eqnarray}
	\lift{w}{y!(z)}\widehat{\id{\{}u / z \id{\}}}
		& = &
		\lift{w}{y!(u)} \nonumber\\
	w[ \lpquote y!(z) \rpquote ] \widehat{ \id{\{}u / z \id{\}} }
		& = &
		w[ \lpquote y!(z) \rpquote ] \nonumber
\end{eqnarray}

Because the body of the process between quotes is impervious to
substitution, we get radically different answers. In fact, by
examining the first process in an input context,
e.g. $x?(z).\lift{w}{y!(z)}$, we see that the process under the lift
operator may be shaped by prefixed inputs binding a name inside it. In
this sense, the lift operator will be seen as a way to dynamically
construct processes before reifying them as names.

Finally equipped with these standard features we can present the
dynamics of the calculus.

\subsubsection{Operational semantics} 

Finally, we introduce the computational dynamics. What marks these
algebras as distinct from other more traditionally studied algebraic
structures, e.g. vector spaces or polynomial rings, is the manner in
which dynamics is captured. In traditional structures, dynamics is typically
expressed through morphisms between such structures, as in linear maps
between vector spaces or morphisms between rings. In algebras
associated with the semantics of computation, the dynamics is
expressed as part of the algebraic structure itself, through a
reduction reduction relation typically denoted by $\red$. Below, we
give a recursive presentation of this relation for the calculus used
in the encoding.

$\red \subseteq \pi \times \pi$
$\red : \pi \to \mathcal{P}(\pi)$

\begin{mathpar}
  \inferrule* [lab=Comm] { \textsf{match}( x_{src}, x_{trgt} ) } { x_{trgt}?(y)P \; | \; x_{src}!\langle {Q} \rangle \red P\{\quotep{Q}/y}\} }
  \and \\
  \inferrule* [lab=Par] {{P} \red {P}'} {{{P} | {Q}} \red {{P}' | {Q}}}
  \and
  \inferrule* [lab=Equiv]{{{P} \scong {P}'} \andalso {{P}' \red {Q}'} \andalso {{Q}' \scong {Q}}}{{P} \red {Q}}
\end{mathpar}

\begin{eqnarray*}
  match_{\equiv} (\quotep{P},\quotep{Q}) & := & P \equiv Q \\
  match_{\dagger}(\quotep{P},\quotep{Q}) & := & \forall R. P|Q \red^{*} R => R \red^{*} 0 \\
  match_{K}(\quotep{P},\quotep{Q}) & := & K \mbox{ for some context } K
\end{eqnarray*}

$u?(x)P | u!\langle Q \rangle \red P\{\quotep{Q}/x\}$

%We write $\wred$ for $\red^*$, and $P\red$ if $\exists Q $ such that $ P \red Q$.
We write $P\red$ if $\exists Q $ such that $ P \red Q$ and $P\not\red$, otherwise.

\section{Replication}

As mentioned before, it is known that replication (and hence
recursion) can be implemented in a higher-order process algebra
\cite{SangiorgiWalker}. As our first example of calculation with the
machinery thus far presented we give the construction explicitly in
the {\rhoc}.

\begin{eqnarray}
	D_{x} & := & \prefix{x}{y}{(\binpar{\outputp{x}{y}}{@{y}})} \nonumber\\
	\bangp_{x}{P} & := & \binpar{{x}!\langle{\binpar{D_{x}}{P}}\rangle}{D_{x}} \nonumber
\end{eqnarray}

\begin{eqnarray}
	\bangp_{x}{P} & & \nonumber\\
	=
	& {x}!\langle{(\prefix{x}{y}{(\outputp{x}{y} | @{y})) | P}}\rangle 
	      | \prefix{x}{y}{(\outputp{x}{y} | @{y})} & \nonumber\\
	\red
	& (\outputp{x}{y} | @{y})\substn{\quotep{(\prefix{x}{y}{(@{y} | \outputp{x}{y})) | P}}}{y} & \nonumber\\
	=
	& \outputp{x}{\quotep{(\prefix{x}{y}{(\outputp{x}{y} | @{y})) | P}}}
	  | {(\prefix{x}{y}{(\outputp{x}{y} | @{y})) | P}} & \nonumber\\
	\red
	& \ldots & \nonumber\\
	\red^*
	& P | P | \ldots & \nonumber
\end{eqnarray}

Of course, this encoding, as an implementation, runs away, unfolding
$\bangp{P}$ eagerly. A lazier and more implementable replication
operator, restricted to input-guarded processes, may be obtained as follows.

\begin{eqnarray}
\bangp{\prefix{u}{v}{P}} 
	:= 
	\binpar{\lift{x}{\prefix{u}{v}{(\binpar{D(x)}{P})}}}{D(x)} \nonumber
\end{eqnarray}

\begin{remark}
  Note that the lazier definition still does not deal with summation
  or mixed summation (i.e. sums over input and output). The reader is
  invited to construct definitions of replication that deal with these
  features. 

  Further, the definitions are parameterized in a name, $x$. Can you,
  gentle reader, make a definition that eliminates this parameter and
  guarantees no accidental interaction between the replication
  machinery and the process being replicated -- i.e. no accidental
  sharing of names used by the process to get its work done and the
  name(s) used by the replication to effect copying. This latter
  revision of the definition of replication is crucial to obtaining
  the expected identity $!!P \sim !P$.
\end{remark}

\begin{remark}\label{rem:paradoxical_combinator}
  The reader familiar with the lambda calculus will have noticed the
  similarity between $D$ and the paradoxical combinator.

  [Ed. note: the existence of this seems to suggest we have to be more
  restrictive on the set of processes and names we admit if we are to
  support no-cloning.]
\end{remark}

\subsubsection{Bisimulation}

The computational dynamics gives rise to another kind of equivalence,
the equivalence of computational behavior. As previously mentioned
this is typically captured \emph{via} some form of bisimulation.

% The notion we use in this paper is weak barbed bisimulation
% \cite{milner91polyadicpi}.

The notion we use in this paper is derived from weak barbed
bisimulation \cite{milner91polyadicpi}. 

\begin{definition}
An \emph{observation relation}, $\downarrow_{\mathcal N}$, over a set
of names, $\mathcal N$, is the smallest relation satisfying the rules
below.

\infrule[Out-barb]{y \in {\mathcal N}, \; x \nameeq y}
		  {\outputp{x}{v} \downarrow_{\mathcal N} x}
\infrule[Par-barb]{\mbox{$P\downarrow_{\mathcal N} x$ or $Q\downarrow_{\mathcal N} x$}}
		  {\binpar{P}{Q} \downarrow_{\mathcal N} x}

We write $P \Downarrow_{\mathcal N} x$ if there is $Q$ such that 
$P \wred Q$ and $Q \downarrow_{\mathcal N} x$.
\end{definition}

\begin{definition}
%\label{def.bbisim}
An  ${\mathcal N}$-\emph{barbed bisimulation} over a set of names, ${\mathcal N}$, is a symmetric binary relation 
${\mathcal S}_{\mathcal N}$ between agents such that $P\rel{S}_{\mathcal N}Q$ implies:
\begin{enumerate}
\item If $P \red P'$ then $Q \wred Q'$ and $P'\rel{S}_{\mathcal N} Q'$.
\item If $P\downarrow_{\mathcal N} x$, then $Q\Downarrow_{\mathcal N} x$.
\end{enumerate}
$P$ is ${\mathcal N}$-barbed bisimilar to $Q$, written
$P \wbbisim_{\mathcal N} Q$, if $P \rel{S}_{\mathcal N} Q$ for some ${\mathcal N}$-barbed bisimulation ${\mathcal S}_{\mathcal N}$.
\end{definition}

$\mathcal{R} \subseteq \pi \times \pi$

$P \mathcal{R} Q => \forall P'. P \red P' \Rightarrow \exists Q'. Q \red Q', P' \mathcal{R} Q'$

$P \vdash x \Rightarrow Q \vdash x$

\begin{mathpar}
  \inferrule*[lab=Out-barb]{x \nameeq y}{{y}!\langle{Q}\rangle \vdash x}
  \and
  \inferrule*[lab=Par-barb]{\mbox{$P\vdash x$ or $Q\vdash x$}}{\binpar{P}{Q} \vdash x}
\end{mathpar}

\subsubsection{Contexts}

One of the principle advantages of computational calculi like the
$\pi$-calculus is a well-defined notion of context,
contextual-equivalence and a correlation between
contextual-equivalence and notions of bisimulation. The notion of
context allows the decomposition of a process into (sub-)process and
its syntactic environment, its context. Thus, a context may be
thought of as a process with a ``hole'' (written $\Box$) in it. The
application of a context $M$ to a process $P$, written $M[P]$, is
tantamount to filling the hole in $M$ with $P$. In this paper we do
not need the full weight of this theory, but do make use of the notion
of context in the proof the main theorem. 

\begin{mathpar}
  \inferrule* [lab=summation] {} {{M_{M},M_{N}} \bc \Box \;|\; x.M_{A} \;|\; M_{M}+M_{N}}
  \and
  \inferrule* [lab=agent] {} {{M_{A}} \bc (\vec{x})M_{P} \;| \; \clift{P_0,\ldots,M_{P},\ldots,P_N}}
  \and \\
  \inferrule* [lab=process] {} {{M_{P}} \bc M_{N} \;| \;P|M_{P} }
\end{mathpar} 

\begin{mathpar}
  \inferrule* [lab=sychronization] {} {M_{N} \bc \Box \;|\; x?M_{F} \;|\; x!M_{C}}
  \and
  \inferrule* [lab=abstraction] {} {{M_{F}} \bc (x)M_{P} }
  \and
  \inferrule* [lab=concretion] {} {{M_{C}} \bc \langle M_{P} \rangle }
  \and \\
  \inferrule* [lab=process] {} {{M_{P}} \bc M_{N} \;| \;P|M_{P} }
\end{mathpar}

\begin{definition}[contextual application] Given a context $M$, and
  process $P$, we define the \emph{contextual application}, $M[P] :=
  M\{P/\Box\}$. That is, the contextual application of M to P is the
  substitution of $P$ for $\Box$ in $M$.
\end{definition}

$\meaningof{-} : L \to \mathcal{P}(\pi)$

\begin{mathpar}
  \inferrule* [lab=collection] {} {\meaningof{true} = \pi, \and \meaningof{~E} = \pi \setminus \meaningof{E}, \and \meaningof{E_{1} \& E_{2}} = \meaningof{E_{1}} \cap \meaningof{E_{2}}}
\end{mathpar}

\begin{mathpar}
  \inferrule* [lab=structure] {} {\meaningof{0} = \{ P \in \pi | P \equiv 0 \}, \and \\ \meaningof{E_1 | E_2} = \{ P \in \pi | P \equiv P_{1} | P_{2}, P_{1} \in \meaningof{E_{1}}, P_{2} \in \meaningof{E_2}\} }
\end{mathpar}

\begin{mathpar}
 \inferrule* [lab=behavior] {} {\meaningof{\langle a?b \rangle E} = \{ P \in \pi | P \equiv Q | u?(y)P', \\ \and \\\\ \and \\ \;\;\; u \in \meaningof{a}, \forall z.P'\{z/y\} \in \meaningof{E\{z/b\}}\}, \and \\ \meaningof{a!E} = \{ P \in \pi | P \equiv Q | x!\langle P' \rangle, x \in \meaningof{a} P' \in \meaningof{E}\} }
\end{mathpar}

\begin{mathpar}
 \inferrule* [lab=nominal] {} {\meaningof{\quotep{E}} = \{ \quotep{P} \in \quotep{\pi} | P \in \meaningof{E} \}, \and \meaningof{\quotep{P}} = \{ \quotep{Q} \in \quotep{\pi} | P \equiv Q \} \and \\ \meaningof{@\quotep{E}} = \{ P \in \pi | P \equiv @x, x \in \meaningof{E} \}}
\end{mathpar}

\begin{eqnarray*}
  \\
  \meaningof{-} : TS \to ST
\end{eqnarray*}

\begin{eqnarray*}
  \\
  L : TS \to ST
\end{eqnarray*}

\begin{eqnarray*}
  \\
  P \models E \iff P \in \meaningof{E}
\end{eqnarray*}

\begin{eqnarray*}
  P \approx_{L} Q \iff \forall E \in L. P \models E \iff Q \models E
\end{eqnarray*}

\begin{eqnarray*}
  P \approx_{K} Q
\end{eqnarray*}

\begin{eqnarray*}
  P \approx Q
\end{eqnarray*}

$\approx_{K} = \approx = \approx_{L}$

\subsubsection{Contextual duality}

Note that contexts extend the quotation operation to a family of
operations from processes to names. Given a context, $M$, we can
define a \emph{nominal context}, $\quotep{M}$ by $\quotep{M}[P] :=
\quotep{M[P]}$. To foreshadow what is to come we observe that these
operations enjoy a duality with processes very much like the duality
between vectors and maps from vectors to scalars.

Further, because the calculus is essentially higher-order, we have a
correspondence between contexts and processes. More specifically,
given a name $x$ and a context $M$ we can construct $M^{*}_{x}$ such
that 

\begin{mathpar}
  M^{*}_{x} | \lift{x}{P} \red M[P]
\end{mathpar}

namely,

\begin{mathpar}
  M^{*}_{x} := x?(u).M[\dropn{u}]
\end{mathpar}

The dependence of $M^{*}_{x}$ on a name makes it an abstraction, 

\begin{mathpar}
  M^{*} := (x)x?(u).M[\dropn{u}]
\end{mathpar}

\subsection{Additional notation}

It will sometimes be convenient to denote the process a name
quotes. We already have the notation $x = \quotep{P}$, but it will be
convenient to introduce an alternate notation, $\procn{x}$, when we
want to emphasize the connection to the use of the name. Note that, by
virtue of name equivalence, $\quotep{\procn{x}} \nameeq x$; so, the
notation is consistent with previous definitions.

Further, because names have structure it is possible to effect
substitutions on the basis of that structure. This means we need to
upgrade our notation for substitutions, which we accomplish by
adapting comprehension notation. Thus,

\begin{mathpar}
  P\{ y / x : x \in S \}
\end{mathpar}

is interpreted to mean the process derived from P by replacing (in a
capture-avoiding manner) each occurrence of $x$ in $S$ by $y$. For example,

\begin{mathpar}
  P\{ \quotep{\procn{x}|\procn{x}} / x : x \in \freenames{P} \}
\end{mathpar}

will replace each (occurrence) of a free name $x$ in $P$ by
$\quotep{\procn{x}|\procn{x}}$.

Also, we will avail ourselves of the notation $x^{L}$ and $x^{R}$ to
denote injections of a name into disjoint copies of the name
space. There are numerous ways to accomplish this. One example can be
found in \cite{MeredithR05}. This notation overloads to vectors of
names: $\vec{x}^{\pi} := (x_{i}^{\pi} \; : \; 0 \leq i < |\vec{x}| )$ where $\pi \in \{L,R\}$.

We also use $P^{\Box} := P|\Box$.

In \cite{MeredithR05} an interpretation of the new operator is
given. It turns out that there are several possible interpretations
all enjoying the requisite algebraic properties of the operator (see
\cite{milner91polyadicpi}). We will therefore make liberal use of
$(\nu\; \vec{x})P$.

% subsection the_syntax_and_semantics_of_the_notation_system (end)   

\input{qm2pi.qmops} 

\input{qm2pi.sterngerlach} 

\input{qm2pi.metric} 

% section concurrent_process_calculi (end)

%\input{qm2pi.proofsketch}

% section proof sketch (end)

%\input{qm2pi.slviaknots} 

% section spatial logic via knots (end)

\input{qm2pi.conclusion}

% section conclusion (end)

%\input{qm2pi.dtcodes} 

% section wiring algorithm (end)

\input{qm2pi.ack} 

% section acknowledgments (end)

\newpage


\bibliographystyle{plain}   
\bibliography{../../biblios/main.bib}

\input{qm2pi.rhodetails}

\end{document}

 

% section acknowledgments (end)

\newpage


\bibliographystyle{plain}   
\bibliography{../../biblios/main.bib}

\documentclass[12pt]{llncs}
%\documentclass{jktr}

\usepackage[pdftex]{hyperref}                   
\usepackage {listings}
\usepackage {mathpartir}
\usepackage{bcprules}
%\usepackage{listings}
                       
\usepackage{graphicx} 
%\usepackage[margins=2.5cm,nohead,nofoot]{geometry}
%\usepackage{geometry}
\usepackage{amsfonts}
\usepackage{amstext}
\usepackage{latexsym}
\usepackage{amssymb}
\usepackage{color}


%\include{myPreamble}
\include{qm2pi.local} 

%\ifpdf
%\usepackage[pdftex]{graphicx}
%\else
%\usepackage{graphicx}
%\fi

 % \ifpdf
%  \usepackage{pdfsync}
%  \if


%\title{Brief Article}
%\author{David F. Snyder}
%\author{L.G. Meredith}

%\address{Dept. of Math., Texas State University--San Marcos, San Marcos, TX 78666}
       
\pagestyle{empty}


\begin{document}

\lstset{language=[Objective]Caml,frame=shadowbox}

\input{qm2pi.front}

% section front matter (end)

\input{qm2pi.intro} 
 
% section introduction (end)

% \input{qm2pi.knotations} 

% section notation (end)

\input{qm2pi.process.calculi} 

% section concurrent_process_calculi_and_spatial_logics_ (end)
    
%\input{qm2pi.knots2pi} 

%\input{qm2pi.trefoil} 

%\input{qm2pi.mainthm} 

% subsection basic_interpretation (end)

%\input{qm2pi.rho.presentation} 
\subsection{The syntax and semantics of the notation system}\label{sub:the_syntax_and_semantics_of_the_notation_system} % (fold)

We now summarize a technical presentation of the calculus that
embodies our theory of dynamics. The typical presentation of such a
calculus follows the style of giving generators and relations on
them. The grammar, below, describing term constructors, freely
generates the set of processes, $\Proc$. This set is then quotiented
by a relation known as structural congruence and it is over this set
that the notion of dynamics is expressed. This presentation is
essentially that of \cite{MeredithR05} with the addition of
polyadicity and summation. For readability we have relegated some of
the technical subtleties to an appendix.

\subsubsection{Process grammar}\label{subsub:process_grammar}

\begin{mathpar}
  \inferrule* [lab=synchronization] {} {{M} \bc \pzero \;|\; x?F \;|\; x!C }
  \and
  \inferrule* [lab=abstraction] {} {{F} \bc (x)P}
  \and
  \inferrule* [lab=concretion] {} {{C} \bc \langle Q \rangle}
  \and
  \inferrule* [lab=process] {} {{P,Q} \bc M \;| \;P|Q \;|\; @{x}}
  \and
  \inferrule* [lab=name] {} {{x} \bc \quotep{P}}
\end{mathpar} 

Note that $\vec{x}$ (resp. $\vec{P}$) denotes a vector of names
(resp. processes) of length $|\vec{x}|$ (resp. $|\vec{P}|$). We adopt
the following useful abbreviations.

\begin{mathpar}
   x?(\vec{y}).P := x.(\vec{y})P \and  x\clift{\vec{P}} := x.\clift{\vec{P}}
   \and x!(y) := \lift{x}{\dropn{y}}
   \and \Pi_{i=0}^{n-1}P_i := P_0 | \ldots | P_{n-1}
\end{mathpar}

\subsubsection{Structural congruence}

\paragraph{Free and bound names and alpha-equivalence.} At the
core of structural equivalence is alpha-equivalence which identifies
process that are the same up to a change of variable. Formally, we
recognize the distinction between free and bound names. The free names
of a process, $\freenames{P}$, may be calculated recursively as
follows:

\begin{mathpar}
\freenames{\pzero} := \emptyset
  \and \\
  \freenames{x?(y).P} := \{ x \} \cup (\freenames{P} \setminus \{ y \})
  \and 
  \freenames{x!\langle P \rangle} := \{ x \} \cup \{ P \} 
  \and \\
  \freenames{P|Q} := \freenames{P} \cup \freenames{Q}
  \and \\
  \freenames{@{x}} := \{ x \}
\end{mathpar}

$\pi$
$\quotep{\pi}$

$\freenames{-} : \pi \to \mathcal{P}(\quotep{\pi})$

\begin{eqnarray*}
  \freenames{\pzero} & := & \emptyset \\
  \freenames{x?(y).P} & := & \{ x \} \cup (\freenames{P} \setminus \{ y \}) \\
  \freenames{x!\langle P \rangle} & := & \{ x \} \cup \{ P \} \\
  \freenames{P|Q} & := & \freenames{P} \cup \freenames{Q} \\
  \freenames{\dropn{x}} & := & \{ x \}
\end{eqnarray*}

The bound names of a process, $\boundnames{P}$, are those names occurring in $P$
that are not free. For example, in $x?(y).0$, the name $x$ is free, while $y$ is bound.

\begin{mathpar}
  \inferrule* [lab=monoidal-laws] {} { P|Q \equiv Q|P \and P|0 \equiv P \and P|(Q|R) \equiv (P|Q)|R }
\end{mathpar}

\begin{mathpar}
  \inferrule* [lab=alpha-equivalence] {} { (x)P \equiv (y)P\{y/x\} \and y \not\in \freenames{P} }
\end{mathpar}

\begin{definition}
Then two processes, $P,Q$, are alpha-equivalent if $P = Q\{\vec{y}/\vec{x}\}$ for
some $\vec{x} \in \boundnames{Q},\vec{y} \in \boundnames{P}$, where $Q\{\vec{y}/\vec{x}\}$
denotes the capture-avoiding substitution of $\vec{y}$ for $\vec{x}$ in $Q$.
\end{definition}

\begin{definition}
  The {\em structural congruence} \cite{SangiorgiWalker} , $\equiv$,
  between processes is the least congruence containing
  alpha-equivalence, satisfying the abelian monoid laws
  (associativity, commutativity and $\pzero$ as identity) for parallel
  composition $|$ and for summation $+$.
\end{definition}

\subsection{Name equivalence}

We take name equivalence, written $\nameeq$, to be the smallest
equivalence relation generated by the following rules.

\begin{mathpar}
\inferrule*[lab=Quote-drop]
{ }
{ \quotep{@{x}} \nameeq x }

\inferrule*[lab=Struct-equiv]
{ P \scong Q }
{ \quotep{P} \nameeq \quotep{Q} }
\end{mathpar}

The astute reader will have noticed that the mutual recursion of names
and processes imposes a mutual recursion on alpha-equivalence and
structural equivalence via name-equivalence. Fortunately, all of this
works out pleasantly and we may calculate in the natural way, free of
concern. The reader interested in the details is referred to the
appendix \ref{appendix:rho_details}.

\subsection{Substitution}

We use $\Proc$ for the set of processes, $\QProc$ for the set of
names, and $\id{\{}\vec{y} / \vec{x} \id{\}}$ to denote partial maps,
$s : \QProc \rightarrow \QProc$. A map, $s$ lifts, uniquely, to a map
on process terms, $\widehat{s} : \Proc \rightarrow \Proc$ by the
following equations.

\begin{mathpar}
  (0) \psubstp{Q}{P} := 0 \\
  (R \juxtap S) \psubstp{Q}{P}
  :=    
  (R)\psubstp{Q}{P} \juxtap (S) \psubstp{Q}{P} \\
  (x?(y).R) \psubstp{Q}{P}    
  :=    
  (x)\substp{Q}{P} (z)\concat( (R \psubstn{z}{y}) \psubstp{Q}{P} ) \\
  (\lift{x}{R}) \psubstp{Q}{P}  
  :=
  \lift{(x)\substp{Q}{P}}{ R \psubstp{Q}{P} } \\
%   (\dropn{x})  \psubstp{Q}{P}       
%   := 
%   \left\{ 
%     \begin{array}{ccc} 
%       \dropn{\quotep{Q}} & & x \nameeq \quotep{P} \\
%       \dropn{x} & & otherwise \\
%     \end{array}
%   \right. 
  (\dropn{x})  \psubstp{Q}{P}       
  := 
  \left\{ 
    \begin{array}{ccc} 
      Q & & x \nameeq \quotep{P} \\
      \dropn{x} & & otherwise \\
    \end{array}
  \right.
\end{mathpar}
 

where

\begin{eqnarray}
  (x)\id{\{} \lpquote Q \rpquote / \lpquote P \rpquote \id{\}}            = 
  \left\{ 
    \begin{array}{ccc}
      \lpquote Q \rpquote & & x \nameeq \lpquote P \rpquote \\
      x & & otherwise \\
    \end{array}
  \right. \nonumber
\end{eqnarray}

and $z$ is chosen distinct from $\quotep{P}$, $\quotep{Q}$, the free
names in $Q$, and all the names in $R$. Our $\alpha$-equivalence will
be built in the standard way from this substitution.

\begin{remark}\label{rem:no_self_referential_names}
  One consequence of these definitions is that $\forall P. \quotep{P}
  \not\in \freenames{P}$.
\end{remark}

\subsection{ Dynamic quote: an example }

Anticipating something of what's to come, consider applying the
substitution, $\widehat{\id{\{}u / z \id{\}}}$, to the following pair
of processes, $\lift{w}{y!(z)}$ and $w[ \lpquote y!(z) \rpquote ]$.

\begin{eqnarray}
	\lift{w}{y!(z)}\widehat{\id{\{}u / z \id{\}}}
		& = &
		\lift{w}{y!(u)} \nonumber\\
	w[ \lpquote y!(z) \rpquote ] \widehat{ \id{\{}u / z \id{\}} }
		& = &
		w[ \lpquote y!(z) \rpquote ] \nonumber
\end{eqnarray}

Because the body of the process between quotes is impervious to
substitution, we get radically different answers. In fact, by
examining the first process in an input context,
e.g. $x?(z).\lift{w}{y!(z)}$, we see that the process under the lift
operator may be shaped by prefixed inputs binding a name inside it. In
this sense, the lift operator will be seen as a way to dynamically
construct processes before reifying them as names.

Finally equipped with these standard features we can present the
dynamics of the calculus.

\subsubsection{Operational semantics} 

Finally, we introduce the computational dynamics. What marks these
algebras as distinct from other more traditionally studied algebraic
structures, e.g. vector spaces or polynomial rings, is the manner in
which dynamics is captured. In traditional structures, dynamics is typically
expressed through morphisms between such structures, as in linear maps
between vector spaces or morphisms between rings. In algebras
associated with the semantics of computation, the dynamics is
expressed as part of the algebraic structure itself, through a
reduction reduction relation typically denoted by $\red$. Below, we
give a recursive presentation of this relation for the calculus used
in the encoding.

$\red \subseteq \pi \times \pi$
$\red : \pi \to \mathcal{P}(\pi)$

\begin{mathpar}
  \inferrule* [lab=Comm] { \textsf{match}( x_{src}, x_{trgt} ) } { x_{trgt}?(y)P \; | \; x_{src}!\langle {Q} \rangle \red P\{\quotep{Q}/y}\} }
  \and \\
  \inferrule* [lab=Par] {{P} \red {P}'} {{{P} | {Q}} \red {{P}' | {Q}}}
  \and
  \inferrule* [lab=Equiv]{{{P} \scong {P}'} \andalso {{P}' \red {Q}'} \andalso {{Q}' \scong {Q}}}{{P} \red {Q}}
\end{mathpar}

\begin{eqnarray*}
  match_{\equiv} (\quotep{P},\quotep{Q}) & := & P \equiv Q \\
  match_{\dagger}(\quotep{P},\quotep{Q}) & := & \forall R. P|Q \red^{*} R => R \red^{*} 0 \\
  match_{K}(\quotep{P},\quotep{Q}) & := & K \mbox{ for some context } K
\end{eqnarray*}

$u?(x)P | u!\langle Q \rangle \red P\{\quotep{Q}/x\}$

%We write $\wred$ for $\red^*$, and $P\red$ if $\exists Q $ such that $ P \red Q$.
We write $P\red$ if $\exists Q $ such that $ P \red Q$ and $P\not\red$, otherwise.

\section{Replication}

As mentioned before, it is known that replication (and hence
recursion) can be implemented in a higher-order process algebra
\cite{SangiorgiWalker}. As our first example of calculation with the
machinery thus far presented we give the construction explicitly in
the {\rhoc}.

\begin{eqnarray}
	D_{x} & := & \prefix{x}{y}{(\binpar{\outputp{x}{y}}{@{y}})} \nonumber\\
	\bangp_{x}{P} & := & \binpar{{x}!\langle{\binpar{D_{x}}{P}}\rangle}{D_{x}} \nonumber
\end{eqnarray}

\begin{eqnarray}
	\bangp_{x}{P} & & \nonumber\\
	=
	& {x}!\langle{(\prefix{x}{y}{(\outputp{x}{y} | @{y})) | P}}\rangle 
	      | \prefix{x}{y}{(\outputp{x}{y} | @{y})} & \nonumber\\
	\red
	& (\outputp{x}{y} | @{y})\substn{\quotep{(\prefix{x}{y}{(@{y} | \outputp{x}{y})) | P}}}{y} & \nonumber\\
	=
	& \outputp{x}{\quotep{(\prefix{x}{y}{(\outputp{x}{y} | @{y})) | P}}}
	  | {(\prefix{x}{y}{(\outputp{x}{y} | @{y})) | P}} & \nonumber\\
	\red
	& \ldots & \nonumber\\
	\red^*
	& P | P | \ldots & \nonumber
\end{eqnarray}

Of course, this encoding, as an implementation, runs away, unfolding
$\bangp{P}$ eagerly. A lazier and more implementable replication
operator, restricted to input-guarded processes, may be obtained as follows.

\begin{eqnarray}
\bangp{\prefix{u}{v}{P}} 
	:= 
	\binpar{\lift{x}{\prefix{u}{v}{(\binpar{D(x)}{P})}}}{D(x)} \nonumber
\end{eqnarray}

\begin{remark}
  Note that the lazier definition still does not deal with summation
  or mixed summation (i.e. sums over input and output). The reader is
  invited to construct definitions of replication that deal with these
  features. 

  Further, the definitions are parameterized in a name, $x$. Can you,
  gentle reader, make a definition that eliminates this parameter and
  guarantees no accidental interaction between the replication
  machinery and the process being replicated -- i.e. no accidental
  sharing of names used by the process to get its work done and the
  name(s) used by the replication to effect copying. This latter
  revision of the definition of replication is crucial to obtaining
  the expected identity $!!P \sim !P$.
\end{remark}

\begin{remark}\label{rem:paradoxical_combinator}
  The reader familiar with the lambda calculus will have noticed the
  similarity between $D$ and the paradoxical combinator.

  [Ed. note: the existence of this seems to suggest we have to be more
  restrictive on the set of processes and names we admit if we are to
  support no-cloning.]
\end{remark}

\subsubsection{Bisimulation}

The computational dynamics gives rise to another kind of equivalence,
the equivalence of computational behavior. As previously mentioned
this is typically captured \emph{via} some form of bisimulation.

% The notion we use in this paper is weak barbed bisimulation
% \cite{milner91polyadicpi}.

The notion we use in this paper is derived from weak barbed
bisimulation \cite{milner91polyadicpi}. 

\begin{definition}
An \emph{observation relation}, $\downarrow_{\mathcal N}$, over a set
of names, $\mathcal N$, is the smallest relation satisfying the rules
below.

\infrule[Out-barb]{y \in {\mathcal N}, \; x \nameeq y}
		  {\outputp{x}{v} \downarrow_{\mathcal N} x}
\infrule[Par-barb]{\mbox{$P\downarrow_{\mathcal N} x$ or $Q\downarrow_{\mathcal N} x$}}
		  {\binpar{P}{Q} \downarrow_{\mathcal N} x}

We write $P \Downarrow_{\mathcal N} x$ if there is $Q$ such that 
$P \wred Q$ and $Q \downarrow_{\mathcal N} x$.
\end{definition}

\begin{definition}
%\label{def.bbisim}
An  ${\mathcal N}$-\emph{barbed bisimulation} over a set of names, ${\mathcal N}$, is a symmetric binary relation 
${\mathcal S}_{\mathcal N}$ between agents such that $P\rel{S}_{\mathcal N}Q$ implies:
\begin{enumerate}
\item If $P \red P'$ then $Q \wred Q'$ and $P'\rel{S}_{\mathcal N} Q'$.
\item If $P\downarrow_{\mathcal N} x$, then $Q\Downarrow_{\mathcal N} x$.
\end{enumerate}
$P$ is ${\mathcal N}$-barbed bisimilar to $Q$, written
$P \wbbisim_{\mathcal N} Q$, if $P \rel{S}_{\mathcal N} Q$ for some ${\mathcal N}$-barbed bisimulation ${\mathcal S}_{\mathcal N}$.
\end{definition}

$\mathcal{R} \subseteq \pi \times \pi$

$P \mathcal{R} Q => \forall P'. P \red P' \Rightarrow \exists Q'. Q \red Q', P' \mathcal{R} Q'$

$P \vdash x \Rightarrow Q \vdash x$

\begin{mathpar}
  \inferrule*[lab=Out-barb]{x \nameeq y}{{y}!\langle{Q}\rangle \vdash x}
  \and
  \inferrule*[lab=Par-barb]{\mbox{$P\vdash x$ or $Q\vdash x$}}{\binpar{P}{Q} \vdash x}
\end{mathpar}

\subsubsection{Contexts}

One of the principle advantages of computational calculi like the
$\pi$-calculus is a well-defined notion of context,
contextual-equivalence and a correlation between
contextual-equivalence and notions of bisimulation. The notion of
context allows the decomposition of a process into (sub-)process and
its syntactic environment, its context. Thus, a context may be
thought of as a process with a ``hole'' (written $\Box$) in it. The
application of a context $M$ to a process $P$, written $M[P]$, is
tantamount to filling the hole in $M$ with $P$. In this paper we do
not need the full weight of this theory, but do make use of the notion
of context in the proof the main theorem. 

\begin{mathpar}
  \inferrule* [lab=summation] {} {{M_{M},M_{N}} \bc \Box \;|\; x.M_{A} \;|\; M_{M}+M_{N}}
  \and
  \inferrule* [lab=agent] {} {{M_{A}} \bc (\vec{x})M_{P} \;| \; \clift{P_0,\ldots,M_{P},\ldots,P_N}}
  \and \\
  \inferrule* [lab=process] {} {{M_{P}} \bc M_{N} \;| \;P|M_{P} }
\end{mathpar} 

\begin{mathpar}
  \inferrule* [lab=sychronization] {} {M_{N} \bc \Box \;|\; x?M_{F} \;|\; x!M_{C}}
  \and
  \inferrule* [lab=abstraction] {} {{M_{F}} \bc (x)M_{P} }
  \and
  \inferrule* [lab=concretion] {} {{M_{C}} \bc \langle M_{P} \rangle }
  \and \\
  \inferrule* [lab=process] {} {{M_{P}} \bc M_{N} \;| \;P|M_{P} }
\end{mathpar}

\begin{definition}[contextual application] Given a context $M$, and
  process $P$, we define the \emph{contextual application}, $M[P] :=
  M\{P/\Box\}$. That is, the contextual application of M to P is the
  substitution of $P$ for $\Box$ in $M$.
\end{definition}

$\meaningof{-} : L \to \mathcal{P}(\pi)$

\begin{mathpar}
  \inferrule* [lab=collection] {} {\meaningof{true} = \pi, \and \meaningof{~E} = \pi \setminus \meaningof{E}, \and \meaningof{E_{1} \& E_{2}} = \meaningof{E_{1}} \cap \meaningof{E_{2}}}
\end{mathpar}

\begin{mathpar}
  \inferrule* [lab=structure] {} {\meaningof{0} = \{ P \in \pi | P \equiv 0 \}, \and \\ \meaningof{E_1 | E_2} = \{ P \in \pi | P \equiv P_{1} | P_{2}, P_{1} \in \meaningof{E_{1}}, P_{2} \in \meaningof{E_2}\} }
\end{mathpar}

\begin{mathpar}
 \inferrule* [lab=behavior] {} {\meaningof{\langle a?b \rangle E} = \{ P \in \pi | P \equiv Q | u?(y)P', \\ \and \\\\ \and \\ \;\;\; u \in \meaningof{a}, \forall z.P'\{z/y\} \in \meaningof{E\{z/b\}}\}, \and \\ \meaningof{a!E} = \{ P \in \pi | P \equiv Q | x!\langle P' \rangle, x \in \meaningof{a} P' \in \meaningof{E}\} }
\end{mathpar}

\begin{mathpar}
 \inferrule* [lab=nominal] {} {\meaningof{\quotep{E}} = \{ \quotep{P} \in \quotep{\pi} | P \in \meaningof{E} \}, \and \meaningof{\quotep{P}} = \{ \quotep{Q} \in \quotep{\pi} | P \equiv Q \} \and \\ \meaningof{@\quotep{E}} = \{ P \in \pi | P \equiv @x, x \in \meaningof{E} \}}
\end{mathpar}

\begin{eqnarray*}
  \\
  \meaningof{-} : TS \to ST
\end{eqnarray*}

\begin{eqnarray*}
  \\
  L : TS \to ST
\end{eqnarray*}

\begin{eqnarray*}
  \\
  P \models E \iff P \in \meaningof{E}
\end{eqnarray*}

\begin{eqnarray*}
  P \approx_{L} Q \iff \forall E \in L. P \models E \iff Q \models E
\end{eqnarray*}

\begin{eqnarray*}
  P \approx_{K} Q
\end{eqnarray*}

\begin{eqnarray*}
  P \approx Q
\end{eqnarray*}

$\approx_{K} = \approx = \approx_{L}$

\subsubsection{Contextual duality}

Note that contexts extend the quotation operation to a family of
operations from processes to names. Given a context, $M$, we can
define a \emph{nominal context}, $\quotep{M}$ by $\quotep{M}[P] :=
\quotep{M[P]}$. To foreshadow what is to come we observe that these
operations enjoy a duality with processes very much like the duality
between vectors and maps from vectors to scalars.

Further, because the calculus is essentially higher-order, we have a
correspondence between contexts and processes. More specifically,
given a name $x$ and a context $M$ we can construct $M^{*}_{x}$ such
that 

\begin{mathpar}
  M^{*}_{x} | \lift{x}{P} \red M[P]
\end{mathpar}

namely,

\begin{mathpar}
  M^{*}_{x} := x?(u).M[\dropn{u}]
\end{mathpar}

The dependence of $M^{*}_{x}$ on a name makes it an abstraction, 

\begin{mathpar}
  M^{*} := (x)x?(u).M[\dropn{u}]
\end{mathpar}

\subsection{Additional notation}

It will sometimes be convenient to denote the process a name
quotes. We already have the notation $x = \quotep{P}$, but it will be
convenient to introduce an alternate notation, $\procn{x}$, when we
want to emphasize the connection to the use of the name. Note that, by
virtue of name equivalence, $\quotep{\procn{x}} \nameeq x$; so, the
notation is consistent with previous definitions.

Further, because names have structure it is possible to effect
substitutions on the basis of that structure. This means we need to
upgrade our notation for substitutions, which we accomplish by
adapting comprehension notation. Thus,

\begin{mathpar}
  P\{ y / x : x \in S \}
\end{mathpar}

is interpreted to mean the process derived from P by replacing (in a
capture-avoiding manner) each occurrence of $x$ in $S$ by $y$. For example,

\begin{mathpar}
  P\{ \quotep{\procn{x}|\procn{x}} / x : x \in \freenames{P} \}
\end{mathpar}

will replace each (occurrence) of a free name $x$ in $P$ by
$\quotep{\procn{x}|\procn{x}}$.

Also, we will avail ourselves of the notation $x^{L}$ and $x^{R}$ to
denote injections of a name into disjoint copies of the name
space. There are numerous ways to accomplish this. One example can be
found in \cite{MeredithR05}. This notation overloads to vectors of
names: $\vec{x}^{\pi} := (x_{i}^{\pi} \; : \; 0 \leq i < |\vec{x}| )$ where $\pi \in \{L,R\}$.

We also use $P^{\Box} := P|\Box$.

In \cite{MeredithR05} an interpretation of the new operator is
given. It turns out that there are several possible interpretations
all enjoying the requisite algebraic properties of the operator (see
\cite{milner91polyadicpi}). We will therefore make liberal use of
$(\nu\; \vec{x})P$.

% subsection the_syntax_and_semantics_of_the_notation_system (end)   

\input{qm2pi.qmops} 

\input{qm2pi.sterngerlach} 

\input{qm2pi.metric} 

% section concurrent_process_calculi (end)

%\input{qm2pi.proofsketch}

% section proof sketch (end)

%\input{qm2pi.slviaknots} 

% section spatial logic via knots (end)

\input{qm2pi.conclusion}

% section conclusion (end)

%\input{qm2pi.dtcodes} 

% section wiring algorithm (end)

\input{qm2pi.ack} 

% section acknowledgments (end)

\newpage


\bibliographystyle{plain}   
\bibliography{../../biblios/main.bib}

\input{qm2pi.rhodetails}

\end{document}



\end{document}

 

% section notation (end)

\input{qm2pi.process.calculi} 

% section concurrent_process_calculi_and_spatial_logics_ (end)
    
%\documentclass[12pt]{llncs}
%\documentclass{jktr}

\usepackage[pdftex]{hyperref}                   
\usepackage {listings}
\usepackage {mathpartir}
\usepackage{bcprules}
%\usepackage{listings}
                       
\usepackage{graphicx} 
%\usepackage[margins=2.5cm,nohead,nofoot]{geometry}
%\usepackage{geometry}
\usepackage{amsfonts}
\usepackage{amstext}
\usepackage{latexsym}
\usepackage{amssymb}
\usepackage{color}


%\include{myPreamble}
\documentclass[12pt]{llncs}
%\documentclass{jktr}

\usepackage[pdftex]{hyperref}                   
\usepackage {listings}
\usepackage {mathpartir}
\usepackage{bcprules}
%\usepackage{listings}
                       
\usepackage{graphicx} 
%\usepackage[margins=2.5cm,nohead,nofoot]{geometry}
%\usepackage{geometry}
\usepackage{amsfonts}
\usepackage{amstext}
\usepackage{latexsym}
\usepackage{amssymb}
\usepackage{color}


%\include{myPreamble}
\include{qm2pi.local} 

%\ifpdf
%\usepackage[pdftex]{graphicx}
%\else
%\usepackage{graphicx}
%\fi

 % \ifpdf
%  \usepackage{pdfsync}
%  \if


%\title{Brief Article}
%\author{David F. Snyder}
%\author{L.G. Meredith}

%\address{Dept. of Math., Texas State University--San Marcos, San Marcos, TX 78666}
       
\pagestyle{empty}


\begin{document}

\lstset{language=[Objective]Caml,frame=shadowbox}

\input{qm2pi.front}

% section front matter (end)

\input{qm2pi.intro} 
 
% section introduction (end)

% \input{qm2pi.knotations} 

% section notation (end)

\input{qm2pi.process.calculi} 

% section concurrent_process_calculi_and_spatial_logics_ (end)
    
%\input{qm2pi.knots2pi} 

%\input{qm2pi.trefoil} 

%\input{qm2pi.mainthm} 

% subsection basic_interpretation (end)

%\input{qm2pi.rho.presentation} 
\subsection{The syntax and semantics of the notation system}\label{sub:the_syntax_and_semantics_of_the_notation_system} % (fold)

We now summarize a technical presentation of the calculus that
embodies our theory of dynamics. The typical presentation of such a
calculus follows the style of giving generators and relations on
them. The grammar, below, describing term constructors, freely
generates the set of processes, $\Proc$. This set is then quotiented
by a relation known as structural congruence and it is over this set
that the notion of dynamics is expressed. This presentation is
essentially that of \cite{MeredithR05} with the addition of
polyadicity and summation. For readability we have relegated some of
the technical subtleties to an appendix.

\subsubsection{Process grammar}\label{subsub:process_grammar}

\begin{mathpar}
  \inferrule* [lab=synchronization] {} {{M} \bc \pzero \;|\; x?F \;|\; x!C }
  \and
  \inferrule* [lab=abstraction] {} {{F} \bc (x)P}
  \and
  \inferrule* [lab=concretion] {} {{C} \bc \langle Q \rangle}
  \and
  \inferrule* [lab=process] {} {{P,Q} \bc M \;| \;P|Q \;|\; @{x}}
  \and
  \inferrule* [lab=name] {} {{x} \bc \quotep{P}}
\end{mathpar} 

Note that $\vec{x}$ (resp. $\vec{P}$) denotes a vector of names
(resp. processes) of length $|\vec{x}|$ (resp. $|\vec{P}|$). We adopt
the following useful abbreviations.

\begin{mathpar}
   x?(\vec{y}).P := x.(\vec{y})P \and  x\clift{\vec{P}} := x.\clift{\vec{P}}
   \and x!(y) := \lift{x}{\dropn{y}}
   \and \Pi_{i=0}^{n-1}P_i := P_0 | \ldots | P_{n-1}
\end{mathpar}

\subsubsection{Structural congruence}

\paragraph{Free and bound names and alpha-equivalence.} At the
core of structural equivalence is alpha-equivalence which identifies
process that are the same up to a change of variable. Formally, we
recognize the distinction between free and bound names. The free names
of a process, $\freenames{P}$, may be calculated recursively as
follows:

\begin{mathpar}
\freenames{\pzero} := \emptyset
  \and \\
  \freenames{x?(y).P} := \{ x \} \cup (\freenames{P} \setminus \{ y \})
  \and 
  \freenames{x!\langle P \rangle} := \{ x \} \cup \{ P \} 
  \and \\
  \freenames{P|Q} := \freenames{P} \cup \freenames{Q}
  \and \\
  \freenames{@{x}} := \{ x \}
\end{mathpar}

$\pi$
$\quotep{\pi}$

$\freenames{-} : \pi \to \mathcal{P}(\quotep{\pi})$

\begin{eqnarray*}
  \freenames{\pzero} & := & \emptyset \\
  \freenames{x?(y).P} & := & \{ x \} \cup (\freenames{P} \setminus \{ y \}) \\
  \freenames{x!\langle P \rangle} & := & \{ x \} \cup \{ P \} \\
  \freenames{P|Q} & := & \freenames{P} \cup \freenames{Q} \\
  \freenames{\dropn{x}} & := & \{ x \}
\end{eqnarray*}

The bound names of a process, $\boundnames{P}$, are those names occurring in $P$
that are not free. For example, in $x?(y).0$, the name $x$ is free, while $y$ is bound.

\begin{mathpar}
  \inferrule* [lab=monoidal-laws] {} { P|Q \equiv Q|P \and P|0 \equiv P \and P|(Q|R) \equiv (P|Q)|R }
\end{mathpar}

\begin{mathpar}
  \inferrule* [lab=alpha-equivalence] {} { (x)P \equiv (y)P\{y/x\} \and y \not\in \freenames{P} }
\end{mathpar}

\begin{definition}
Then two processes, $P,Q$, are alpha-equivalent if $P = Q\{\vec{y}/\vec{x}\}$ for
some $\vec{x} \in \boundnames{Q},\vec{y} \in \boundnames{P}$, where $Q\{\vec{y}/\vec{x}\}$
denotes the capture-avoiding substitution of $\vec{y}$ for $\vec{x}$ in $Q$.
\end{definition}

\begin{definition}
  The {\em structural congruence} \cite{SangiorgiWalker} , $\equiv$,
  between processes is the least congruence containing
  alpha-equivalence, satisfying the abelian monoid laws
  (associativity, commutativity and $\pzero$ as identity) for parallel
  composition $|$ and for summation $+$.
\end{definition}

\subsection{Name equivalence}

We take name equivalence, written $\nameeq$, to be the smallest
equivalence relation generated by the following rules.

\begin{mathpar}
\inferrule*[lab=Quote-drop]
{ }
{ \quotep{@{x}} \nameeq x }

\inferrule*[lab=Struct-equiv]
{ P \scong Q }
{ \quotep{P} \nameeq \quotep{Q} }
\end{mathpar}

The astute reader will have noticed that the mutual recursion of names
and processes imposes a mutual recursion on alpha-equivalence and
structural equivalence via name-equivalence. Fortunately, all of this
works out pleasantly and we may calculate in the natural way, free of
concern. The reader interested in the details is referred to the
appendix \ref{appendix:rho_details}.

\subsection{Substitution}

We use $\Proc$ for the set of processes, $\QProc$ for the set of
names, and $\id{\{}\vec{y} / \vec{x} \id{\}}$ to denote partial maps,
$s : \QProc \rightarrow \QProc$. A map, $s$ lifts, uniquely, to a map
on process terms, $\widehat{s} : \Proc \rightarrow \Proc$ by the
following equations.

\begin{mathpar}
  (0) \psubstp{Q}{P} := 0 \\
  (R \juxtap S) \psubstp{Q}{P}
  :=    
  (R)\psubstp{Q}{P} \juxtap (S) \psubstp{Q}{P} \\
  (x?(y).R) \psubstp{Q}{P}    
  :=    
  (x)\substp{Q}{P} (z)\concat( (R \psubstn{z}{y}) \psubstp{Q}{P} ) \\
  (\lift{x}{R}) \psubstp{Q}{P}  
  :=
  \lift{(x)\substp{Q}{P}}{ R \psubstp{Q}{P} } \\
%   (\dropn{x})  \psubstp{Q}{P}       
%   := 
%   \left\{ 
%     \begin{array}{ccc} 
%       \dropn{\quotep{Q}} & & x \nameeq \quotep{P} \\
%       \dropn{x} & & otherwise \\
%     \end{array}
%   \right. 
  (\dropn{x})  \psubstp{Q}{P}       
  := 
  \left\{ 
    \begin{array}{ccc} 
      Q & & x \nameeq \quotep{P} \\
      \dropn{x} & & otherwise \\
    \end{array}
  \right.
\end{mathpar}
 

where

\begin{eqnarray}
  (x)\id{\{} \lpquote Q \rpquote / \lpquote P \rpquote \id{\}}            = 
  \left\{ 
    \begin{array}{ccc}
      \lpquote Q \rpquote & & x \nameeq \lpquote P \rpquote \\
      x & & otherwise \\
    \end{array}
  \right. \nonumber
\end{eqnarray}

and $z$ is chosen distinct from $\quotep{P}$, $\quotep{Q}$, the free
names in $Q$, and all the names in $R$. Our $\alpha$-equivalence will
be built in the standard way from this substitution.

\begin{remark}\label{rem:no_self_referential_names}
  One consequence of these definitions is that $\forall P. \quotep{P}
  \not\in \freenames{P}$.
\end{remark}

\subsection{ Dynamic quote: an example }

Anticipating something of what's to come, consider applying the
substitution, $\widehat{\id{\{}u / z \id{\}}}$, to the following pair
of processes, $\lift{w}{y!(z)}$ and $w[ \lpquote y!(z) \rpquote ]$.

\begin{eqnarray}
	\lift{w}{y!(z)}\widehat{\id{\{}u / z \id{\}}}
		& = &
		\lift{w}{y!(u)} \nonumber\\
	w[ \lpquote y!(z) \rpquote ] \widehat{ \id{\{}u / z \id{\}} }
		& = &
		w[ \lpquote y!(z) \rpquote ] \nonumber
\end{eqnarray}

Because the body of the process between quotes is impervious to
substitution, we get radically different answers. In fact, by
examining the first process in an input context,
e.g. $x?(z).\lift{w}{y!(z)}$, we see that the process under the lift
operator may be shaped by prefixed inputs binding a name inside it. In
this sense, the lift operator will be seen as a way to dynamically
construct processes before reifying them as names.

Finally equipped with these standard features we can present the
dynamics of the calculus.

\subsubsection{Operational semantics} 

Finally, we introduce the computational dynamics. What marks these
algebras as distinct from other more traditionally studied algebraic
structures, e.g. vector spaces or polynomial rings, is the manner in
which dynamics is captured. In traditional structures, dynamics is typically
expressed through morphisms between such structures, as in linear maps
between vector spaces or morphisms between rings. In algebras
associated with the semantics of computation, the dynamics is
expressed as part of the algebraic structure itself, through a
reduction reduction relation typically denoted by $\red$. Below, we
give a recursive presentation of this relation for the calculus used
in the encoding.

$\red \subseteq \pi \times \pi$
$\red : \pi \to \mathcal{P}(\pi)$

\begin{mathpar}
  \inferrule* [lab=Comm] { \textsf{match}( x_{src}, x_{trgt} ) } { x_{trgt}?(y)P \; | \; x_{src}!\langle {Q} \rangle \red P\{\quotep{Q}/y}\} }
  \and \\
  \inferrule* [lab=Par] {{P} \red {P}'} {{{P} | {Q}} \red {{P}' | {Q}}}
  \and
  \inferrule* [lab=Equiv]{{{P} \scong {P}'} \andalso {{P}' \red {Q}'} \andalso {{Q}' \scong {Q}}}{{P} \red {Q}}
\end{mathpar}

\begin{eqnarray*}
  match_{\equiv} (\quotep{P},\quotep{Q}) & := & P \equiv Q \\
  match_{\dagger}(\quotep{P},\quotep{Q}) & := & \forall R. P|Q \red^{*} R => R \red^{*} 0 \\
  match_{K}(\quotep{P},\quotep{Q}) & := & K \mbox{ for some context } K
\end{eqnarray*}

$u?(x)P | u!\langle Q \rangle \red P\{\quotep{Q}/x\}$

%We write $\wred$ for $\red^*$, and $P\red$ if $\exists Q $ such that $ P \red Q$.
We write $P\red$ if $\exists Q $ such that $ P \red Q$ and $P\not\red$, otherwise.

\section{Replication}

As mentioned before, it is known that replication (and hence
recursion) can be implemented in a higher-order process algebra
\cite{SangiorgiWalker}. As our first example of calculation with the
machinery thus far presented we give the construction explicitly in
the {\rhoc}.

\begin{eqnarray}
	D_{x} & := & \prefix{x}{y}{(\binpar{\outputp{x}{y}}{@{y}})} \nonumber\\
	\bangp_{x}{P} & := & \binpar{{x}!\langle{\binpar{D_{x}}{P}}\rangle}{D_{x}} \nonumber
\end{eqnarray}

\begin{eqnarray}
	\bangp_{x}{P} & & \nonumber\\
	=
	& {x}!\langle{(\prefix{x}{y}{(\outputp{x}{y} | @{y})) | P}}\rangle 
	      | \prefix{x}{y}{(\outputp{x}{y} | @{y})} & \nonumber\\
	\red
	& (\outputp{x}{y} | @{y})\substn{\quotep{(\prefix{x}{y}{(@{y} | \outputp{x}{y})) | P}}}{y} & \nonumber\\
	=
	& \outputp{x}{\quotep{(\prefix{x}{y}{(\outputp{x}{y} | @{y})) | P}}}
	  | {(\prefix{x}{y}{(\outputp{x}{y} | @{y})) | P}} & \nonumber\\
	\red
	& \ldots & \nonumber\\
	\red^*
	& P | P | \ldots & \nonumber
\end{eqnarray}

Of course, this encoding, as an implementation, runs away, unfolding
$\bangp{P}$ eagerly. A lazier and more implementable replication
operator, restricted to input-guarded processes, may be obtained as follows.

\begin{eqnarray}
\bangp{\prefix{u}{v}{P}} 
	:= 
	\binpar{\lift{x}{\prefix{u}{v}{(\binpar{D(x)}{P})}}}{D(x)} \nonumber
\end{eqnarray}

\begin{remark}
  Note that the lazier definition still does not deal with summation
  or mixed summation (i.e. sums over input and output). The reader is
  invited to construct definitions of replication that deal with these
  features. 

  Further, the definitions are parameterized in a name, $x$. Can you,
  gentle reader, make a definition that eliminates this parameter and
  guarantees no accidental interaction between the replication
  machinery and the process being replicated -- i.e. no accidental
  sharing of names used by the process to get its work done and the
  name(s) used by the replication to effect copying. This latter
  revision of the definition of replication is crucial to obtaining
  the expected identity $!!P \sim !P$.
\end{remark}

\begin{remark}\label{rem:paradoxical_combinator}
  The reader familiar with the lambda calculus will have noticed the
  similarity between $D$ and the paradoxical combinator.

  [Ed. note: the existence of this seems to suggest we have to be more
  restrictive on the set of processes and names we admit if we are to
  support no-cloning.]
\end{remark}

\subsubsection{Bisimulation}

The computational dynamics gives rise to another kind of equivalence,
the equivalence of computational behavior. As previously mentioned
this is typically captured \emph{via} some form of bisimulation.

% The notion we use in this paper is weak barbed bisimulation
% \cite{milner91polyadicpi}.

The notion we use in this paper is derived from weak barbed
bisimulation \cite{milner91polyadicpi}. 

\begin{definition}
An \emph{observation relation}, $\downarrow_{\mathcal N}$, over a set
of names, $\mathcal N$, is the smallest relation satisfying the rules
below.

\infrule[Out-barb]{y \in {\mathcal N}, \; x \nameeq y}
		  {\outputp{x}{v} \downarrow_{\mathcal N} x}
\infrule[Par-barb]{\mbox{$P\downarrow_{\mathcal N} x$ or $Q\downarrow_{\mathcal N} x$}}
		  {\binpar{P}{Q} \downarrow_{\mathcal N} x}

We write $P \Downarrow_{\mathcal N} x$ if there is $Q$ such that 
$P \wred Q$ and $Q \downarrow_{\mathcal N} x$.
\end{definition}

\begin{definition}
%\label{def.bbisim}
An  ${\mathcal N}$-\emph{barbed bisimulation} over a set of names, ${\mathcal N}$, is a symmetric binary relation 
${\mathcal S}_{\mathcal N}$ between agents such that $P\rel{S}_{\mathcal N}Q$ implies:
\begin{enumerate}
\item If $P \red P'$ then $Q \wred Q'$ and $P'\rel{S}_{\mathcal N} Q'$.
\item If $P\downarrow_{\mathcal N} x$, then $Q\Downarrow_{\mathcal N} x$.
\end{enumerate}
$P$ is ${\mathcal N}$-barbed bisimilar to $Q$, written
$P \wbbisim_{\mathcal N} Q$, if $P \rel{S}_{\mathcal N} Q$ for some ${\mathcal N}$-barbed bisimulation ${\mathcal S}_{\mathcal N}$.
\end{definition}

$\mathcal{R} \subseteq \pi \times \pi$

$P \mathcal{R} Q => \forall P'. P \red P' \Rightarrow \exists Q'. Q \red Q', P' \mathcal{R} Q'$

$P \vdash x \Rightarrow Q \vdash x$

\begin{mathpar}
  \inferrule*[lab=Out-barb]{x \nameeq y}{{y}!\langle{Q}\rangle \vdash x}
  \and
  \inferrule*[lab=Par-barb]{\mbox{$P\vdash x$ or $Q\vdash x$}}{\binpar{P}{Q} \vdash x}
\end{mathpar}

\subsubsection{Contexts}

One of the principle advantages of computational calculi like the
$\pi$-calculus is a well-defined notion of context,
contextual-equivalence and a correlation between
contextual-equivalence and notions of bisimulation. The notion of
context allows the decomposition of a process into (sub-)process and
its syntactic environment, its context. Thus, a context may be
thought of as a process with a ``hole'' (written $\Box$) in it. The
application of a context $M$ to a process $P$, written $M[P]$, is
tantamount to filling the hole in $M$ with $P$. In this paper we do
not need the full weight of this theory, but do make use of the notion
of context in the proof the main theorem. 

\begin{mathpar}
  \inferrule* [lab=summation] {} {{M_{M},M_{N}} \bc \Box \;|\; x.M_{A} \;|\; M_{M}+M_{N}}
  \and
  \inferrule* [lab=agent] {} {{M_{A}} \bc (\vec{x})M_{P} \;| \; \clift{P_0,\ldots,M_{P},\ldots,P_N}}
  \and \\
  \inferrule* [lab=process] {} {{M_{P}} \bc M_{N} \;| \;P|M_{P} }
\end{mathpar} 

\begin{mathpar}
  \inferrule* [lab=sychronization] {} {M_{N} \bc \Box \;|\; x?M_{F} \;|\; x!M_{C}}
  \and
  \inferrule* [lab=abstraction] {} {{M_{F}} \bc (x)M_{P} }
  \and
  \inferrule* [lab=concretion] {} {{M_{C}} \bc \langle M_{P} \rangle }
  \and \\
  \inferrule* [lab=process] {} {{M_{P}} \bc M_{N} \;| \;P|M_{P} }
\end{mathpar}

\begin{definition}[contextual application] Given a context $M$, and
  process $P$, we define the \emph{contextual application}, $M[P] :=
  M\{P/\Box\}$. That is, the contextual application of M to P is the
  substitution of $P$ for $\Box$ in $M$.
\end{definition}

$\meaningof{-} : L \to \mathcal{P}(\pi)$

\begin{mathpar}
  \inferrule* [lab=collection] {} {\meaningof{true} = \pi, \and \meaningof{~E} = \pi \setminus \meaningof{E}, \and \meaningof{E_{1} \& E_{2}} = \meaningof{E_{1}} \cap \meaningof{E_{2}}}
\end{mathpar}

\begin{mathpar}
  \inferrule* [lab=structure] {} {\meaningof{0} = \{ P \in \pi | P \equiv 0 \}, \and \\ \meaningof{E_1 | E_2} = \{ P \in \pi | P \equiv P_{1} | P_{2}, P_{1} \in \meaningof{E_{1}}, P_{2} \in \meaningof{E_2}\} }
\end{mathpar}

\begin{mathpar}
 \inferrule* [lab=behavior] {} {\meaningof{\langle a?b \rangle E} = \{ P \in \pi | P \equiv Q | u?(y)P', \\ \and \\\\ \and \\ \;\;\; u \in \meaningof{a}, \forall z.P'\{z/y\} \in \meaningof{E\{z/b\}}\}, \and \\ \meaningof{a!E} = \{ P \in \pi | P \equiv Q | x!\langle P' \rangle, x \in \meaningof{a} P' \in \meaningof{E}\} }
\end{mathpar}

\begin{mathpar}
 \inferrule* [lab=nominal] {} {\meaningof{\quotep{E}} = \{ \quotep{P} \in \quotep{\pi} | P \in \meaningof{E} \}, \and \meaningof{\quotep{P}} = \{ \quotep{Q} \in \quotep{\pi} | P \equiv Q \} \and \\ \meaningof{@\quotep{E}} = \{ P \in \pi | P \equiv @x, x \in \meaningof{E} \}}
\end{mathpar}

\begin{eqnarray*}
  \\
  \meaningof{-} : TS \to ST
\end{eqnarray*}

\begin{eqnarray*}
  \\
  L : TS \to ST
\end{eqnarray*}

\begin{eqnarray*}
  \\
  P \models E \iff P \in \meaningof{E}
\end{eqnarray*}

\begin{eqnarray*}
  P \approx_{L} Q \iff \forall E \in L. P \models E \iff Q \models E
\end{eqnarray*}

\begin{eqnarray*}
  P \approx_{K} Q
\end{eqnarray*}

\begin{eqnarray*}
  P \approx Q
\end{eqnarray*}

$\approx_{K} = \approx = \approx_{L}$

\subsubsection{Contextual duality}

Note that contexts extend the quotation operation to a family of
operations from processes to names. Given a context, $M$, we can
define a \emph{nominal context}, $\quotep{M}$ by $\quotep{M}[P] :=
\quotep{M[P]}$. To foreshadow what is to come we observe that these
operations enjoy a duality with processes very much like the duality
between vectors and maps from vectors to scalars.

Further, because the calculus is essentially higher-order, we have a
correspondence between contexts and processes. More specifically,
given a name $x$ and a context $M$ we can construct $M^{*}_{x}$ such
that 

\begin{mathpar}
  M^{*}_{x} | \lift{x}{P} \red M[P]
\end{mathpar}

namely,

\begin{mathpar}
  M^{*}_{x} := x?(u).M[\dropn{u}]
\end{mathpar}

The dependence of $M^{*}_{x}$ on a name makes it an abstraction, 

\begin{mathpar}
  M^{*} := (x)x?(u).M[\dropn{u}]
\end{mathpar}

\subsection{Additional notation}

It will sometimes be convenient to denote the process a name
quotes. We already have the notation $x = \quotep{P}$, but it will be
convenient to introduce an alternate notation, $\procn{x}$, when we
want to emphasize the connection to the use of the name. Note that, by
virtue of name equivalence, $\quotep{\procn{x}} \nameeq x$; so, the
notation is consistent with previous definitions.

Further, because names have structure it is possible to effect
substitutions on the basis of that structure. This means we need to
upgrade our notation for substitutions, which we accomplish by
adapting comprehension notation. Thus,

\begin{mathpar}
  P\{ y / x : x \in S \}
\end{mathpar}

is interpreted to mean the process derived from P by replacing (in a
capture-avoiding manner) each occurrence of $x$ in $S$ by $y$. For example,

\begin{mathpar}
  P\{ \quotep{\procn{x}|\procn{x}} / x : x \in \freenames{P} \}
\end{mathpar}

will replace each (occurrence) of a free name $x$ in $P$ by
$\quotep{\procn{x}|\procn{x}}$.

Also, we will avail ourselves of the notation $x^{L}$ and $x^{R}$ to
denote injections of a name into disjoint copies of the name
space. There are numerous ways to accomplish this. One example can be
found in \cite{MeredithR05}. This notation overloads to vectors of
names: $\vec{x}^{\pi} := (x_{i}^{\pi} \; : \; 0 \leq i < |\vec{x}| )$ where $\pi \in \{L,R\}$.

We also use $P^{\Box} := P|\Box$.

In \cite{MeredithR05} an interpretation of the new operator is
given. It turns out that there are several possible interpretations
all enjoying the requisite algebraic properties of the operator (see
\cite{milner91polyadicpi}). We will therefore make liberal use of
$(\nu\; \vec{x})P$.

% subsection the_syntax_and_semantics_of_the_notation_system (end)   

\input{qm2pi.qmops} 

\input{qm2pi.sterngerlach} 

\input{qm2pi.metric} 

% section concurrent_process_calculi (end)

%\input{qm2pi.proofsketch}

% section proof sketch (end)

%\input{qm2pi.slviaknots} 

% section spatial logic via knots (end)

\input{qm2pi.conclusion}

% section conclusion (end)

%\input{qm2pi.dtcodes} 

% section wiring algorithm (end)

\input{qm2pi.ack} 

% section acknowledgments (end)

\newpage


\bibliographystyle{plain}   
\bibliography{../../biblios/main.bib}

\input{qm2pi.rhodetails}

\end{document}

 

%\ifpdf
%\usepackage[pdftex]{graphicx}
%\else
%\usepackage{graphicx}
%\fi

 % \ifpdf
%  \usepackage{pdfsync}
%  \if


%\title{Brief Article}
%\author{David F. Snyder}
%\author{L.G. Meredith}

%\address{Dept. of Math., Texas State University--San Marcos, San Marcos, TX 78666}
       
\pagestyle{empty}


\begin{document}

\lstset{language=[Objective]Caml,frame=shadowbox}

\documentclass[12pt]{llncs}
%\documentclass{jktr}

\usepackage[pdftex]{hyperref}                   
\usepackage {listings}
\usepackage {mathpartir}
\usepackage{bcprules}
%\usepackage{listings}
                       
\usepackage{graphicx} 
%\usepackage[margins=2.5cm,nohead,nofoot]{geometry}
%\usepackage{geometry}
\usepackage{amsfonts}
\usepackage{amstext}
\usepackage{latexsym}
\usepackage{amssymb}
\usepackage{color}


%\include{myPreamble}
\include{qm2pi.local} 

%\ifpdf
%\usepackage[pdftex]{graphicx}
%\else
%\usepackage{graphicx}
%\fi

 % \ifpdf
%  \usepackage{pdfsync}
%  \if


%\title{Brief Article}
%\author{David F. Snyder}
%\author{L.G. Meredith}

%\address{Dept. of Math., Texas State University--San Marcos, San Marcos, TX 78666}
       
\pagestyle{empty}


\begin{document}

\lstset{language=[Objective]Caml,frame=shadowbox}

\input{qm2pi.front}

% section front matter (end)

\input{qm2pi.intro} 
 
% section introduction (end)

% \input{qm2pi.knotations} 

% section notation (end)

\input{qm2pi.process.calculi} 

% section concurrent_process_calculi_and_spatial_logics_ (end)
    
%\input{qm2pi.knots2pi} 

%\input{qm2pi.trefoil} 

%\input{qm2pi.mainthm} 

% subsection basic_interpretation (end)

%\input{qm2pi.rho.presentation} 
\subsection{The syntax and semantics of the notation system}\label{sub:the_syntax_and_semantics_of_the_notation_system} % (fold)

We now summarize a technical presentation of the calculus that
embodies our theory of dynamics. The typical presentation of such a
calculus follows the style of giving generators and relations on
them. The grammar, below, describing term constructors, freely
generates the set of processes, $\Proc$. This set is then quotiented
by a relation known as structural congruence and it is over this set
that the notion of dynamics is expressed. This presentation is
essentially that of \cite{MeredithR05} with the addition of
polyadicity and summation. For readability we have relegated some of
the technical subtleties to an appendix.

\subsubsection{Process grammar}\label{subsub:process_grammar}

\begin{mathpar}
  \inferrule* [lab=synchronization] {} {{M} \bc \pzero \;|\; x?F \;|\; x!C }
  \and
  \inferrule* [lab=abstraction] {} {{F} \bc (x)P}
  \and
  \inferrule* [lab=concretion] {} {{C} \bc \langle Q \rangle}
  \and
  \inferrule* [lab=process] {} {{P,Q} \bc M \;| \;P|Q \;|\; @{x}}
  \and
  \inferrule* [lab=name] {} {{x} \bc \quotep{P}}
\end{mathpar} 

Note that $\vec{x}$ (resp. $\vec{P}$) denotes a vector of names
(resp. processes) of length $|\vec{x}|$ (resp. $|\vec{P}|$). We adopt
the following useful abbreviations.

\begin{mathpar}
   x?(\vec{y}).P := x.(\vec{y})P \and  x\clift{\vec{P}} := x.\clift{\vec{P}}
   \and x!(y) := \lift{x}{\dropn{y}}
   \and \Pi_{i=0}^{n-1}P_i := P_0 | \ldots | P_{n-1}
\end{mathpar}

\subsubsection{Structural congruence}

\paragraph{Free and bound names and alpha-equivalence.} At the
core of structural equivalence is alpha-equivalence which identifies
process that are the same up to a change of variable. Formally, we
recognize the distinction between free and bound names. The free names
of a process, $\freenames{P}$, may be calculated recursively as
follows:

\begin{mathpar}
\freenames{\pzero} := \emptyset
  \and \\
  \freenames{x?(y).P} := \{ x \} \cup (\freenames{P} \setminus \{ y \})
  \and 
  \freenames{x!\langle P \rangle} := \{ x \} \cup \{ P \} 
  \and \\
  \freenames{P|Q} := \freenames{P} \cup \freenames{Q}
  \and \\
  \freenames{@{x}} := \{ x \}
\end{mathpar}

$\pi$
$\quotep{\pi}$

$\freenames{-} : \pi \to \mathcal{P}(\quotep{\pi})$

\begin{eqnarray*}
  \freenames{\pzero} & := & \emptyset \\
  \freenames{x?(y).P} & := & \{ x \} \cup (\freenames{P} \setminus \{ y \}) \\
  \freenames{x!\langle P \rangle} & := & \{ x \} \cup \{ P \} \\
  \freenames{P|Q} & := & \freenames{P} \cup \freenames{Q} \\
  \freenames{\dropn{x}} & := & \{ x \}
\end{eqnarray*}

The bound names of a process, $\boundnames{P}$, are those names occurring in $P$
that are not free. For example, in $x?(y).0$, the name $x$ is free, while $y$ is bound.

\begin{mathpar}
  \inferrule* [lab=monoidal-laws] {} { P|Q \equiv Q|P \and P|0 \equiv P \and P|(Q|R) \equiv (P|Q)|R }
\end{mathpar}

\begin{mathpar}
  \inferrule* [lab=alpha-equivalence] {} { (x)P \equiv (y)P\{y/x\} \and y \not\in \freenames{P} }
\end{mathpar}

\begin{definition}
Then two processes, $P,Q$, are alpha-equivalent if $P = Q\{\vec{y}/\vec{x}\}$ for
some $\vec{x} \in \boundnames{Q},\vec{y} \in \boundnames{P}$, where $Q\{\vec{y}/\vec{x}\}$
denotes the capture-avoiding substitution of $\vec{y}$ for $\vec{x}$ in $Q$.
\end{definition}

\begin{definition}
  The {\em structural congruence} \cite{SangiorgiWalker} , $\equiv$,
  between processes is the least congruence containing
  alpha-equivalence, satisfying the abelian monoid laws
  (associativity, commutativity and $\pzero$ as identity) for parallel
  composition $|$ and for summation $+$.
\end{definition}

\subsection{Name equivalence}

We take name equivalence, written $\nameeq$, to be the smallest
equivalence relation generated by the following rules.

\begin{mathpar}
\inferrule*[lab=Quote-drop]
{ }
{ \quotep{@{x}} \nameeq x }

\inferrule*[lab=Struct-equiv]
{ P \scong Q }
{ \quotep{P} \nameeq \quotep{Q} }
\end{mathpar}

The astute reader will have noticed that the mutual recursion of names
and processes imposes a mutual recursion on alpha-equivalence and
structural equivalence via name-equivalence. Fortunately, all of this
works out pleasantly and we may calculate in the natural way, free of
concern. The reader interested in the details is referred to the
appendix \ref{appendix:rho_details}.

\subsection{Substitution}

We use $\Proc$ for the set of processes, $\QProc$ for the set of
names, and $\id{\{}\vec{y} / \vec{x} \id{\}}$ to denote partial maps,
$s : \QProc \rightarrow \QProc$. A map, $s$ lifts, uniquely, to a map
on process terms, $\widehat{s} : \Proc \rightarrow \Proc$ by the
following equations.

\begin{mathpar}
  (0) \psubstp{Q}{P} := 0 \\
  (R \juxtap S) \psubstp{Q}{P}
  :=    
  (R)\psubstp{Q}{P} \juxtap (S) \psubstp{Q}{P} \\
  (x?(y).R) \psubstp{Q}{P}    
  :=    
  (x)\substp{Q}{P} (z)\concat( (R \psubstn{z}{y}) \psubstp{Q}{P} ) \\
  (\lift{x}{R}) \psubstp{Q}{P}  
  :=
  \lift{(x)\substp{Q}{P}}{ R \psubstp{Q}{P} } \\
%   (\dropn{x})  \psubstp{Q}{P}       
%   := 
%   \left\{ 
%     \begin{array}{ccc} 
%       \dropn{\quotep{Q}} & & x \nameeq \quotep{P} \\
%       \dropn{x} & & otherwise \\
%     \end{array}
%   \right. 
  (\dropn{x})  \psubstp{Q}{P}       
  := 
  \left\{ 
    \begin{array}{ccc} 
      Q & & x \nameeq \quotep{P} \\
      \dropn{x} & & otherwise \\
    \end{array}
  \right.
\end{mathpar}
 

where

\begin{eqnarray}
  (x)\id{\{} \lpquote Q \rpquote / \lpquote P \rpquote \id{\}}            = 
  \left\{ 
    \begin{array}{ccc}
      \lpquote Q \rpquote & & x \nameeq \lpquote P \rpquote \\
      x & & otherwise \\
    \end{array}
  \right. \nonumber
\end{eqnarray}

and $z$ is chosen distinct from $\quotep{P}$, $\quotep{Q}$, the free
names in $Q$, and all the names in $R$. Our $\alpha$-equivalence will
be built in the standard way from this substitution.

\begin{remark}\label{rem:no_self_referential_names}
  One consequence of these definitions is that $\forall P. \quotep{P}
  \not\in \freenames{P}$.
\end{remark}

\subsection{ Dynamic quote: an example }

Anticipating something of what's to come, consider applying the
substitution, $\widehat{\id{\{}u / z \id{\}}}$, to the following pair
of processes, $\lift{w}{y!(z)}$ and $w[ \lpquote y!(z) \rpquote ]$.

\begin{eqnarray}
	\lift{w}{y!(z)}\widehat{\id{\{}u / z \id{\}}}
		& = &
		\lift{w}{y!(u)} \nonumber\\
	w[ \lpquote y!(z) \rpquote ] \widehat{ \id{\{}u / z \id{\}} }
		& = &
		w[ \lpquote y!(z) \rpquote ] \nonumber
\end{eqnarray}

Because the body of the process between quotes is impervious to
substitution, we get radically different answers. In fact, by
examining the first process in an input context,
e.g. $x?(z).\lift{w}{y!(z)}$, we see that the process under the lift
operator may be shaped by prefixed inputs binding a name inside it. In
this sense, the lift operator will be seen as a way to dynamically
construct processes before reifying them as names.

Finally equipped with these standard features we can present the
dynamics of the calculus.

\subsubsection{Operational semantics} 

Finally, we introduce the computational dynamics. What marks these
algebras as distinct from other more traditionally studied algebraic
structures, e.g. vector spaces or polynomial rings, is the manner in
which dynamics is captured. In traditional structures, dynamics is typically
expressed through morphisms between such structures, as in linear maps
between vector spaces or morphisms between rings. In algebras
associated with the semantics of computation, the dynamics is
expressed as part of the algebraic structure itself, through a
reduction reduction relation typically denoted by $\red$. Below, we
give a recursive presentation of this relation for the calculus used
in the encoding.

$\red \subseteq \pi \times \pi$
$\red : \pi \to \mathcal{P}(\pi)$

\begin{mathpar}
  \inferrule* [lab=Comm] { \textsf{match}( x_{src}, x_{trgt} ) } { x_{trgt}?(y)P \; | \; x_{src}!\langle {Q} \rangle \red P\{\quotep{Q}/y}\} }
  \and \\
  \inferrule* [lab=Par] {{P} \red {P}'} {{{P} | {Q}} \red {{P}' | {Q}}}
  \and
  \inferrule* [lab=Equiv]{{{P} \scong {P}'} \andalso {{P}' \red {Q}'} \andalso {{Q}' \scong {Q}}}{{P} \red {Q}}
\end{mathpar}

\begin{eqnarray*}
  match_{\equiv} (\quotep{P},\quotep{Q}) & := & P \equiv Q \\
  match_{\dagger}(\quotep{P},\quotep{Q}) & := & \forall R. P|Q \red^{*} R => R \red^{*} 0 \\
  match_{K}(\quotep{P},\quotep{Q}) & := & K \mbox{ for some context } K
\end{eqnarray*}

$u?(x)P | u!\langle Q \rangle \red P\{\quotep{Q}/x\}$

%We write $\wred$ for $\red^*$, and $P\red$ if $\exists Q $ such that $ P \red Q$.
We write $P\red$ if $\exists Q $ such that $ P \red Q$ and $P\not\red$, otherwise.

\section{Replication}

As mentioned before, it is known that replication (and hence
recursion) can be implemented in a higher-order process algebra
\cite{SangiorgiWalker}. As our first example of calculation with the
machinery thus far presented we give the construction explicitly in
the {\rhoc}.

\begin{eqnarray}
	D_{x} & := & \prefix{x}{y}{(\binpar{\outputp{x}{y}}{@{y}})} \nonumber\\
	\bangp_{x}{P} & := & \binpar{{x}!\langle{\binpar{D_{x}}{P}}\rangle}{D_{x}} \nonumber
\end{eqnarray}

\begin{eqnarray}
	\bangp_{x}{P} & & \nonumber\\
	=
	& {x}!\langle{(\prefix{x}{y}{(\outputp{x}{y} | @{y})) | P}}\rangle 
	      | \prefix{x}{y}{(\outputp{x}{y} | @{y})} & \nonumber\\
	\red
	& (\outputp{x}{y} | @{y})\substn{\quotep{(\prefix{x}{y}{(@{y} | \outputp{x}{y})) | P}}}{y} & \nonumber\\
	=
	& \outputp{x}{\quotep{(\prefix{x}{y}{(\outputp{x}{y} | @{y})) | P}}}
	  | {(\prefix{x}{y}{(\outputp{x}{y} | @{y})) | P}} & \nonumber\\
	\red
	& \ldots & \nonumber\\
	\red^*
	& P | P | \ldots & \nonumber
\end{eqnarray}

Of course, this encoding, as an implementation, runs away, unfolding
$\bangp{P}$ eagerly. A lazier and more implementable replication
operator, restricted to input-guarded processes, may be obtained as follows.

\begin{eqnarray}
\bangp{\prefix{u}{v}{P}} 
	:= 
	\binpar{\lift{x}{\prefix{u}{v}{(\binpar{D(x)}{P})}}}{D(x)} \nonumber
\end{eqnarray}

\begin{remark}
  Note that the lazier definition still does not deal with summation
  or mixed summation (i.e. sums over input and output). The reader is
  invited to construct definitions of replication that deal with these
  features. 

  Further, the definitions are parameterized in a name, $x$. Can you,
  gentle reader, make a definition that eliminates this parameter and
  guarantees no accidental interaction between the replication
  machinery and the process being replicated -- i.e. no accidental
  sharing of names used by the process to get its work done and the
  name(s) used by the replication to effect copying. This latter
  revision of the definition of replication is crucial to obtaining
  the expected identity $!!P \sim !P$.
\end{remark}

\begin{remark}\label{rem:paradoxical_combinator}
  The reader familiar with the lambda calculus will have noticed the
  similarity between $D$ and the paradoxical combinator.

  [Ed. note: the existence of this seems to suggest we have to be more
  restrictive on the set of processes and names we admit if we are to
  support no-cloning.]
\end{remark}

\subsubsection{Bisimulation}

The computational dynamics gives rise to another kind of equivalence,
the equivalence of computational behavior. As previously mentioned
this is typically captured \emph{via} some form of bisimulation.

% The notion we use in this paper is weak barbed bisimulation
% \cite{milner91polyadicpi}.

The notion we use in this paper is derived from weak barbed
bisimulation \cite{milner91polyadicpi}. 

\begin{definition}
An \emph{observation relation}, $\downarrow_{\mathcal N}$, over a set
of names, $\mathcal N$, is the smallest relation satisfying the rules
below.

\infrule[Out-barb]{y \in {\mathcal N}, \; x \nameeq y}
		  {\outputp{x}{v} \downarrow_{\mathcal N} x}
\infrule[Par-barb]{\mbox{$P\downarrow_{\mathcal N} x$ or $Q\downarrow_{\mathcal N} x$}}
		  {\binpar{P}{Q} \downarrow_{\mathcal N} x}

We write $P \Downarrow_{\mathcal N} x$ if there is $Q$ such that 
$P \wred Q$ and $Q \downarrow_{\mathcal N} x$.
\end{definition}

\begin{definition}
%\label{def.bbisim}
An  ${\mathcal N}$-\emph{barbed bisimulation} over a set of names, ${\mathcal N}$, is a symmetric binary relation 
${\mathcal S}_{\mathcal N}$ between agents such that $P\rel{S}_{\mathcal N}Q$ implies:
\begin{enumerate}
\item If $P \red P'$ then $Q \wred Q'$ and $P'\rel{S}_{\mathcal N} Q'$.
\item If $P\downarrow_{\mathcal N} x$, then $Q\Downarrow_{\mathcal N} x$.
\end{enumerate}
$P$ is ${\mathcal N}$-barbed bisimilar to $Q$, written
$P \wbbisim_{\mathcal N} Q$, if $P \rel{S}_{\mathcal N} Q$ for some ${\mathcal N}$-barbed bisimulation ${\mathcal S}_{\mathcal N}$.
\end{definition}

$\mathcal{R} \subseteq \pi \times \pi$

$P \mathcal{R} Q => \forall P'. P \red P' \Rightarrow \exists Q'. Q \red Q', P' \mathcal{R} Q'$

$P \vdash x \Rightarrow Q \vdash x$

\begin{mathpar}
  \inferrule*[lab=Out-barb]{x \nameeq y}{{y}!\langle{Q}\rangle \vdash x}
  \and
  \inferrule*[lab=Par-barb]{\mbox{$P\vdash x$ or $Q\vdash x$}}{\binpar{P}{Q} \vdash x}
\end{mathpar}

\subsubsection{Contexts}

One of the principle advantages of computational calculi like the
$\pi$-calculus is a well-defined notion of context,
contextual-equivalence and a correlation between
contextual-equivalence and notions of bisimulation. The notion of
context allows the decomposition of a process into (sub-)process and
its syntactic environment, its context. Thus, a context may be
thought of as a process with a ``hole'' (written $\Box$) in it. The
application of a context $M$ to a process $P$, written $M[P]$, is
tantamount to filling the hole in $M$ with $P$. In this paper we do
not need the full weight of this theory, but do make use of the notion
of context in the proof the main theorem. 

\begin{mathpar}
  \inferrule* [lab=summation] {} {{M_{M},M_{N}} \bc \Box \;|\; x.M_{A} \;|\; M_{M}+M_{N}}
  \and
  \inferrule* [lab=agent] {} {{M_{A}} \bc (\vec{x})M_{P} \;| \; \clift{P_0,\ldots,M_{P},\ldots,P_N}}
  \and \\
  \inferrule* [lab=process] {} {{M_{P}} \bc M_{N} \;| \;P|M_{P} }
\end{mathpar} 

\begin{mathpar}
  \inferrule* [lab=sychronization] {} {M_{N} \bc \Box \;|\; x?M_{F} \;|\; x!M_{C}}
  \and
  \inferrule* [lab=abstraction] {} {{M_{F}} \bc (x)M_{P} }
  \and
  \inferrule* [lab=concretion] {} {{M_{C}} \bc \langle M_{P} \rangle }
  \and \\
  \inferrule* [lab=process] {} {{M_{P}} \bc M_{N} \;| \;P|M_{P} }
\end{mathpar}

\begin{definition}[contextual application] Given a context $M$, and
  process $P$, we define the \emph{contextual application}, $M[P] :=
  M\{P/\Box\}$. That is, the contextual application of M to P is the
  substitution of $P$ for $\Box$ in $M$.
\end{definition}

$\meaningof{-} : L \to \mathcal{P}(\pi)$

\begin{mathpar}
  \inferrule* [lab=collection] {} {\meaningof{true} = \pi, \and \meaningof{~E} = \pi \setminus \meaningof{E}, \and \meaningof{E_{1} \& E_{2}} = \meaningof{E_{1}} \cap \meaningof{E_{2}}}
\end{mathpar}

\begin{mathpar}
  \inferrule* [lab=structure] {} {\meaningof{0} = \{ P \in \pi | P \equiv 0 \}, \and \\ \meaningof{E_1 | E_2} = \{ P \in \pi | P \equiv P_{1} | P_{2}, P_{1} \in \meaningof{E_{1}}, P_{2} \in \meaningof{E_2}\} }
\end{mathpar}

\begin{mathpar}
 \inferrule* [lab=behavior] {} {\meaningof{\langle a?b \rangle E} = \{ P \in \pi | P \equiv Q | u?(y)P', \\ \and \\\\ \and \\ \;\;\; u \in \meaningof{a}, \forall z.P'\{z/y\} \in \meaningof{E\{z/b\}}\}, \and \\ \meaningof{a!E} = \{ P \in \pi | P \equiv Q | x!\langle P' \rangle, x \in \meaningof{a} P' \in \meaningof{E}\} }
\end{mathpar}

\begin{mathpar}
 \inferrule* [lab=nominal] {} {\meaningof{\quotep{E}} = \{ \quotep{P} \in \quotep{\pi} | P \in \meaningof{E} \}, \and \meaningof{\quotep{P}} = \{ \quotep{Q} \in \quotep{\pi} | P \equiv Q \} \and \\ \meaningof{@\quotep{E}} = \{ P \in \pi | P \equiv @x, x \in \meaningof{E} \}}
\end{mathpar}

\begin{eqnarray*}
  \\
  \meaningof{-} : TS \to ST
\end{eqnarray*}

\begin{eqnarray*}
  \\
  L : TS \to ST
\end{eqnarray*}

\begin{eqnarray*}
  \\
  P \models E \iff P \in \meaningof{E}
\end{eqnarray*}

\begin{eqnarray*}
  P \approx_{L} Q \iff \forall E \in L. P \models E \iff Q \models E
\end{eqnarray*}

\begin{eqnarray*}
  P \approx_{K} Q
\end{eqnarray*}

\begin{eqnarray*}
  P \approx Q
\end{eqnarray*}

$\approx_{K} = \approx = \approx_{L}$

\subsubsection{Contextual duality}

Note that contexts extend the quotation operation to a family of
operations from processes to names. Given a context, $M$, we can
define a \emph{nominal context}, $\quotep{M}$ by $\quotep{M}[P] :=
\quotep{M[P]}$. To foreshadow what is to come we observe that these
operations enjoy a duality with processes very much like the duality
between vectors and maps from vectors to scalars.

Further, because the calculus is essentially higher-order, we have a
correspondence between contexts and processes. More specifically,
given a name $x$ and a context $M$ we can construct $M^{*}_{x}$ such
that 

\begin{mathpar}
  M^{*}_{x} | \lift{x}{P} \red M[P]
\end{mathpar}

namely,

\begin{mathpar}
  M^{*}_{x} := x?(u).M[\dropn{u}]
\end{mathpar}

The dependence of $M^{*}_{x}$ on a name makes it an abstraction, 

\begin{mathpar}
  M^{*} := (x)x?(u).M[\dropn{u}]
\end{mathpar}

\subsection{Additional notation}

It will sometimes be convenient to denote the process a name
quotes. We already have the notation $x = \quotep{P}$, but it will be
convenient to introduce an alternate notation, $\procn{x}$, when we
want to emphasize the connection to the use of the name. Note that, by
virtue of name equivalence, $\quotep{\procn{x}} \nameeq x$; so, the
notation is consistent with previous definitions.

Further, because names have structure it is possible to effect
substitutions on the basis of that structure. This means we need to
upgrade our notation for substitutions, which we accomplish by
adapting comprehension notation. Thus,

\begin{mathpar}
  P\{ y / x : x \in S \}
\end{mathpar}

is interpreted to mean the process derived from P by replacing (in a
capture-avoiding manner) each occurrence of $x$ in $S$ by $y$. For example,

\begin{mathpar}
  P\{ \quotep{\procn{x}|\procn{x}} / x : x \in \freenames{P} \}
\end{mathpar}

will replace each (occurrence) of a free name $x$ in $P$ by
$\quotep{\procn{x}|\procn{x}}$.

Also, we will avail ourselves of the notation $x^{L}$ and $x^{R}$ to
denote injections of a name into disjoint copies of the name
space. There are numerous ways to accomplish this. One example can be
found in \cite{MeredithR05}. This notation overloads to vectors of
names: $\vec{x}^{\pi} := (x_{i}^{\pi} \; : \; 0 \leq i < |\vec{x}| )$ where $\pi \in \{L,R\}$.

We also use $P^{\Box} := P|\Box$.

In \cite{MeredithR05} an interpretation of the new operator is
given. It turns out that there are several possible interpretations
all enjoying the requisite algebraic properties of the operator (see
\cite{milner91polyadicpi}). We will therefore make liberal use of
$(\nu\; \vec{x})P$.

% subsection the_syntax_and_semantics_of_the_notation_system (end)   

\input{qm2pi.qmops} 

\input{qm2pi.sterngerlach} 

\input{qm2pi.metric} 

% section concurrent_process_calculi (end)

%\input{qm2pi.proofsketch}

% section proof sketch (end)

%\input{qm2pi.slviaknots} 

% section spatial logic via knots (end)

\input{qm2pi.conclusion}

% section conclusion (end)

%\input{qm2pi.dtcodes} 

% section wiring algorithm (end)

\input{qm2pi.ack} 

% section acknowledgments (end)

\newpage


\bibliographystyle{plain}   
\bibliography{../../biblios/main.bib}

\input{qm2pi.rhodetails}

\end{document}



% section front matter (end)

\section{Introduction}\label{sec:introduction} % (fold)
In this draft of the material i am going to have to dispense with the
usual writing conventions adopted in papers on these topics. i'm going
to have adopt whatever tone i need at the time i'm writing up the
calculations. Sometimes this may be very conversational; others it may
be the barest mathematical grunts; others still it may be that i have
lifted text from one of my other papers because the exposition of some
point was better said there. i hope that my readers are not unduly put
out by this decision. i'm not doing this to flout convention or be
rebellious. i find these calculations very technically challenging. To
keep everything going technically, something has to give; i have to
let go of some cognitive burden. So, the academic writing style --
with all of its trade-offs in terms of facilitating technical
communication -- is what i'm letting go of. Perhaps subsequent drafts
can be tightened and polished, but for now, i'm going to speak as if
we were sitting together in a coffee shop with a laptop, wifi and a
pad of paper and a pencil.

So, here's what i have to say. We -- you and i, comfortably ensconced
in our coffee shop and well-equipped with our tools -- can realize and
carry out the calculations of quantum mechanics over a very different
formal theory of dynamics, a formal theory of dynamics that
corresponds to a theory of concurrent computation with
\emph{reflection}. It has the advantage that the underlying theory is
already `quantized', but supports analogues all of the continuuous
operations. Strikingly, this underlying theory has recently been
connected with a notion of metric that we can show, by calculating
together, coincides with the metric induced by the inner product.

There are a lot of reasons why you might be interested in seeing
calculations of this form. Here's why i'm interested. For the past
several centuries there has been no competitor to the ``Newtonian''
account of dynamics. As a result the predominant share of accounts of
dynamical systems and situations have had to be formulated in terms of
the Newtonian machinery. i view this as an intellectually dangerous
position to occupy. Everything, despite it's intrinsic shape, turns
into a nail to be hit with this hammer. Recently, however, the theory
of computation has matured to the point where we have candidates for
theories of dynamics that offer very different perspective on
reasoning about dynamical systems and situations. Testing these
candidates against very successful accounts of dynamical situations,
like quantum mechanics, is going to give us some sense of how mature
they are and some measure of the quality of these accounts of
dynamics.

\subsection{Summary of contributions and outline of paper}

So, we're going to develop an interpretation of the operations of
quantum mechanics normally interpreted by Hilbert spaces and
operators. We're going to do this over a theory of computation. Note
that this is very different than the usual quantum computation program
which develops notions of computation over quantum mechanics. Rather,
we are developing a story that aligns with Wheeler's slogan: It from
Bit. To do this we will first provide an account of the theory of
computation at play here. Then we will dive into a calculation-driven
interpretation of the operations of quantum mechanics.

The reason we take this approach is that -- until very recently --
there hasn't been an axiomatic account of quantum mechanics. As a
result there has been no sharp delineation of the mathematical theory
supporting interpretation of the physical theory and the physical
theory, itself. So, ambient features of the maths are free to be
exploited (or supressed) without a real accounting of their physical
relevance. There is no sharp statement ``here's the physical theory''
qua \emph{theory} and ``here's the mathematical interpretation''
enabling a judgment of how faithful the interpretation is -- apart
from experimental observation. When there is an axiomatic account we
can judge how well a given mathematical formalism supports an
interpretation of the axioms, independent of
experimentation. Likewise, we can judge how well we have captured our
physical evidence and experience with our axiomatics, independent of
any specific mathematical implementation, with accidental detail that
may or may not have physical significance. 

In lieu of a fully fleshed out and vetted axiomatic account of quantum
mechanics, interpreting the operational notions in service of modeling
physical systems will have to suffice. In other words, we are not in
the business of providing a model of Hilbert spaces and operators. We
are in the business of providing a model of quantum mechanics because
we are motivated by testing our notions of dynamics against physical
theory; and, the predictive calculations of the physical theory must
serve as the best formulation -- shy of a fully fleshed out axiomatic
account -- of the physical theory itself (as they have for scientific
theories since time immemorial). Put another way, despite a
whole-hearted commitment to an It-from-Bit ontology, we are firmly
aligned with the shut-up-and-calculate camp as the best way to obtain
results either from the physical perspective or as a quality assurance
measure of our fledgling theory of dynamics.

In detail, we present a reflective process calculus. Then we develop
intuitive correspondences between the notions available in this
calculus and the usual physical notions supporting quantum mechanical
calculations. Thus, 

\begin{table}[htp]
  \center{
    \fbox{
      \begin{tabular}{c|c}
        quantum mechanics & process calculus \\
        \hline
        scalar & name \\
        state vector & process \\
        dual & contextual duals \\
        matrix & formal sums of process-context-dual pairs \\
        orthogonality & process annihilation \\
        inner product & execution-formula + quoting
      \end{tabular}
    }
  }
  \caption{QM - process calculi correspondences}
\end{table}

Then we tighten up these intuitions to operational definitions. We
employ the Dirac notation as the best proxy we can find for an
abstract syntax of the quantum mechanical notions. The definitions we
develop put us in contact with equational constraints coming from the
theory that we demonstrate the definitions and calculations satisfy.

This puts us in a position to shut up and calculate for the
Stern-Gerlach experimental set up, showing how these predictive
calculations become calculations on processes in our theory of a
reflective process calculus.

Penultimately, we demonstrate that the notion of metric coming from
the inner product coincides with the notion of metric available from
the theory of bisimulation. This demonstration gives us the right to
think of space as arising from behavior. Finally, we consider where we
might go from the new vantage point we have obtained.

% section introduction (end) 
 
% section introduction (end)

% \documentclass[12pt]{llncs}
%\documentclass{jktr}

\usepackage[pdftex]{hyperref}                   
\usepackage {listings}
\usepackage {mathpartir}
\usepackage{bcprules}
%\usepackage{listings}
                       
\usepackage{graphicx} 
%\usepackage[margins=2.5cm,nohead,nofoot]{geometry}
%\usepackage{geometry}
\usepackage{amsfonts}
\usepackage{amstext}
\usepackage{latexsym}
\usepackage{amssymb}
\usepackage{color}


%\include{myPreamble}
\include{qm2pi.local} 

%\ifpdf
%\usepackage[pdftex]{graphicx}
%\else
%\usepackage{graphicx}
%\fi

 % \ifpdf
%  \usepackage{pdfsync}
%  \if


%\title{Brief Article}
%\author{David F. Snyder}
%\author{L.G. Meredith}

%\address{Dept. of Math., Texas State University--San Marcos, San Marcos, TX 78666}
       
\pagestyle{empty}


\begin{document}

\lstset{language=[Objective]Caml,frame=shadowbox}

\input{qm2pi.front}

% section front matter (end)

\input{qm2pi.intro} 
 
% section introduction (end)

% \input{qm2pi.knotations} 

% section notation (end)

\input{qm2pi.process.calculi} 

% section concurrent_process_calculi_and_spatial_logics_ (end)
    
%\input{qm2pi.knots2pi} 

%\input{qm2pi.trefoil} 

%\input{qm2pi.mainthm} 

% subsection basic_interpretation (end)

%\input{qm2pi.rho.presentation} 
\subsection{The syntax and semantics of the notation system}\label{sub:the_syntax_and_semantics_of_the_notation_system} % (fold)

We now summarize a technical presentation of the calculus that
embodies our theory of dynamics. The typical presentation of such a
calculus follows the style of giving generators and relations on
them. The grammar, below, describing term constructors, freely
generates the set of processes, $\Proc$. This set is then quotiented
by a relation known as structural congruence and it is over this set
that the notion of dynamics is expressed. This presentation is
essentially that of \cite{MeredithR05} with the addition of
polyadicity and summation. For readability we have relegated some of
the technical subtleties to an appendix.

\subsubsection{Process grammar}\label{subsub:process_grammar}

\begin{mathpar}
  \inferrule* [lab=synchronization] {} {{M} \bc \pzero \;|\; x?F \;|\; x!C }
  \and
  \inferrule* [lab=abstraction] {} {{F} \bc (x)P}
  \and
  \inferrule* [lab=concretion] {} {{C} \bc \langle Q \rangle}
  \and
  \inferrule* [lab=process] {} {{P,Q} \bc M \;| \;P|Q \;|\; @{x}}
  \and
  \inferrule* [lab=name] {} {{x} \bc \quotep{P}}
\end{mathpar} 

Note that $\vec{x}$ (resp. $\vec{P}$) denotes a vector of names
(resp. processes) of length $|\vec{x}|$ (resp. $|\vec{P}|$). We adopt
the following useful abbreviations.

\begin{mathpar}
   x?(\vec{y}).P := x.(\vec{y})P \and  x\clift{\vec{P}} := x.\clift{\vec{P}}
   \and x!(y) := \lift{x}{\dropn{y}}
   \and \Pi_{i=0}^{n-1}P_i := P_0 | \ldots | P_{n-1}
\end{mathpar}

\subsubsection{Structural congruence}

\paragraph{Free and bound names and alpha-equivalence.} At the
core of structural equivalence is alpha-equivalence which identifies
process that are the same up to a change of variable. Formally, we
recognize the distinction between free and bound names. The free names
of a process, $\freenames{P}$, may be calculated recursively as
follows:

\begin{mathpar}
\freenames{\pzero} := \emptyset
  \and \\
  \freenames{x?(y).P} := \{ x \} \cup (\freenames{P} \setminus \{ y \})
  \and 
  \freenames{x!\langle P \rangle} := \{ x \} \cup \{ P \} 
  \and \\
  \freenames{P|Q} := \freenames{P} \cup \freenames{Q}
  \and \\
  \freenames{@{x}} := \{ x \}
\end{mathpar}

$\pi$
$\quotep{\pi}$

$\freenames{-} : \pi \to \mathcal{P}(\quotep{\pi})$

\begin{eqnarray*}
  \freenames{\pzero} & := & \emptyset \\
  \freenames{x?(y).P} & := & \{ x \} \cup (\freenames{P} \setminus \{ y \}) \\
  \freenames{x!\langle P \rangle} & := & \{ x \} \cup \{ P \} \\
  \freenames{P|Q} & := & \freenames{P} \cup \freenames{Q} \\
  \freenames{\dropn{x}} & := & \{ x \}
\end{eqnarray*}

The bound names of a process, $\boundnames{P}$, are those names occurring in $P$
that are not free. For example, in $x?(y).0$, the name $x$ is free, while $y$ is bound.

\begin{mathpar}
  \inferrule* [lab=monoidal-laws] {} { P|Q \equiv Q|P \and P|0 \equiv P \and P|(Q|R) \equiv (P|Q)|R }
\end{mathpar}

\begin{mathpar}
  \inferrule* [lab=alpha-equivalence] {} { (x)P \equiv (y)P\{y/x\} \and y \not\in \freenames{P} }
\end{mathpar}

\begin{definition}
Then two processes, $P,Q$, are alpha-equivalent if $P = Q\{\vec{y}/\vec{x}\}$ for
some $\vec{x} \in \boundnames{Q},\vec{y} \in \boundnames{P}$, where $Q\{\vec{y}/\vec{x}\}$
denotes the capture-avoiding substitution of $\vec{y}$ for $\vec{x}$ in $Q$.
\end{definition}

\begin{definition}
  The {\em structural congruence} \cite{SangiorgiWalker} , $\equiv$,
  between processes is the least congruence containing
  alpha-equivalence, satisfying the abelian monoid laws
  (associativity, commutativity and $\pzero$ as identity) for parallel
  composition $|$ and for summation $+$.
\end{definition}

\subsection{Name equivalence}

We take name equivalence, written $\nameeq$, to be the smallest
equivalence relation generated by the following rules.

\begin{mathpar}
\inferrule*[lab=Quote-drop]
{ }
{ \quotep{@{x}} \nameeq x }

\inferrule*[lab=Struct-equiv]
{ P \scong Q }
{ \quotep{P} \nameeq \quotep{Q} }
\end{mathpar}

The astute reader will have noticed that the mutual recursion of names
and processes imposes a mutual recursion on alpha-equivalence and
structural equivalence via name-equivalence. Fortunately, all of this
works out pleasantly and we may calculate in the natural way, free of
concern. The reader interested in the details is referred to the
appendix \ref{appendix:rho_details}.

\subsection{Substitution}

We use $\Proc$ for the set of processes, $\QProc$ for the set of
names, and $\id{\{}\vec{y} / \vec{x} \id{\}}$ to denote partial maps,
$s : \QProc \rightarrow \QProc$. A map, $s$ lifts, uniquely, to a map
on process terms, $\widehat{s} : \Proc \rightarrow \Proc$ by the
following equations.

\begin{mathpar}
  (0) \psubstp{Q}{P} := 0 \\
  (R \juxtap S) \psubstp{Q}{P}
  :=    
  (R)\psubstp{Q}{P} \juxtap (S) \psubstp{Q}{P} \\
  (x?(y).R) \psubstp{Q}{P}    
  :=    
  (x)\substp{Q}{P} (z)\concat( (R \psubstn{z}{y}) \psubstp{Q}{P} ) \\
  (\lift{x}{R}) \psubstp{Q}{P}  
  :=
  \lift{(x)\substp{Q}{P}}{ R \psubstp{Q}{P} } \\
%   (\dropn{x})  \psubstp{Q}{P}       
%   := 
%   \left\{ 
%     \begin{array}{ccc} 
%       \dropn{\quotep{Q}} & & x \nameeq \quotep{P} \\
%       \dropn{x} & & otherwise \\
%     \end{array}
%   \right. 
  (\dropn{x})  \psubstp{Q}{P}       
  := 
  \left\{ 
    \begin{array}{ccc} 
      Q & & x \nameeq \quotep{P} \\
      \dropn{x} & & otherwise \\
    \end{array}
  \right.
\end{mathpar}
 

where

\begin{eqnarray}
  (x)\id{\{} \lpquote Q \rpquote / \lpquote P \rpquote \id{\}}            = 
  \left\{ 
    \begin{array}{ccc}
      \lpquote Q \rpquote & & x \nameeq \lpquote P \rpquote \\
      x & & otherwise \\
    \end{array}
  \right. \nonumber
\end{eqnarray}

and $z$ is chosen distinct from $\quotep{P}$, $\quotep{Q}$, the free
names in $Q$, and all the names in $R$. Our $\alpha$-equivalence will
be built in the standard way from this substitution.

\begin{remark}\label{rem:no_self_referential_names}
  One consequence of these definitions is that $\forall P. \quotep{P}
  \not\in \freenames{P}$.
\end{remark}

\subsection{ Dynamic quote: an example }

Anticipating something of what's to come, consider applying the
substitution, $\widehat{\id{\{}u / z \id{\}}}$, to the following pair
of processes, $\lift{w}{y!(z)}$ and $w[ \lpquote y!(z) \rpquote ]$.

\begin{eqnarray}
	\lift{w}{y!(z)}\widehat{\id{\{}u / z \id{\}}}
		& = &
		\lift{w}{y!(u)} \nonumber\\
	w[ \lpquote y!(z) \rpquote ] \widehat{ \id{\{}u / z \id{\}} }
		& = &
		w[ \lpquote y!(z) \rpquote ] \nonumber
\end{eqnarray}

Because the body of the process between quotes is impervious to
substitution, we get radically different answers. In fact, by
examining the first process in an input context,
e.g. $x?(z).\lift{w}{y!(z)}$, we see that the process under the lift
operator may be shaped by prefixed inputs binding a name inside it. In
this sense, the lift operator will be seen as a way to dynamically
construct processes before reifying them as names.

Finally equipped with these standard features we can present the
dynamics of the calculus.

\subsubsection{Operational semantics} 

Finally, we introduce the computational dynamics. What marks these
algebras as distinct from other more traditionally studied algebraic
structures, e.g. vector spaces or polynomial rings, is the manner in
which dynamics is captured. In traditional structures, dynamics is typically
expressed through morphisms between such structures, as in linear maps
between vector spaces or morphisms between rings. In algebras
associated with the semantics of computation, the dynamics is
expressed as part of the algebraic structure itself, through a
reduction reduction relation typically denoted by $\red$. Below, we
give a recursive presentation of this relation for the calculus used
in the encoding.

$\red \subseteq \pi \times \pi$
$\red : \pi \to \mathcal{P}(\pi)$

\begin{mathpar}
  \inferrule* [lab=Comm] { \textsf{match}( x_{src}, x_{trgt} ) } { x_{trgt}?(y)P \; | \; x_{src}!\langle {Q} \rangle \red P\{\quotep{Q}/y}\} }
  \and \\
  \inferrule* [lab=Par] {{P} \red {P}'} {{{P} | {Q}} \red {{P}' | {Q}}}
  \and
  \inferrule* [lab=Equiv]{{{P} \scong {P}'} \andalso {{P}' \red {Q}'} \andalso {{Q}' \scong {Q}}}{{P} \red {Q}}
\end{mathpar}

\begin{eqnarray*}
  match_{\equiv} (\quotep{P},\quotep{Q}) & := & P \equiv Q \\
  match_{\dagger}(\quotep{P},\quotep{Q}) & := & \forall R. P|Q \red^{*} R => R \red^{*} 0 \\
  match_{K}(\quotep{P},\quotep{Q}) & := & K \mbox{ for some context } K
\end{eqnarray*}

$u?(x)P | u!\langle Q \rangle \red P\{\quotep{Q}/x\}$

%We write $\wred$ for $\red^*$, and $P\red$ if $\exists Q $ such that $ P \red Q$.
We write $P\red$ if $\exists Q $ such that $ P \red Q$ and $P\not\red$, otherwise.

\section{Replication}

As mentioned before, it is known that replication (and hence
recursion) can be implemented in a higher-order process algebra
\cite{SangiorgiWalker}. As our first example of calculation with the
machinery thus far presented we give the construction explicitly in
the {\rhoc}.

\begin{eqnarray}
	D_{x} & := & \prefix{x}{y}{(\binpar{\outputp{x}{y}}{@{y}})} \nonumber\\
	\bangp_{x}{P} & := & \binpar{{x}!\langle{\binpar{D_{x}}{P}}\rangle}{D_{x}} \nonumber
\end{eqnarray}

\begin{eqnarray}
	\bangp_{x}{P} & & \nonumber\\
	=
	& {x}!\langle{(\prefix{x}{y}{(\outputp{x}{y} | @{y})) | P}}\rangle 
	      | \prefix{x}{y}{(\outputp{x}{y} | @{y})} & \nonumber\\
	\red
	& (\outputp{x}{y} | @{y})\substn{\quotep{(\prefix{x}{y}{(@{y} | \outputp{x}{y})) | P}}}{y} & \nonumber\\
	=
	& \outputp{x}{\quotep{(\prefix{x}{y}{(\outputp{x}{y} | @{y})) | P}}}
	  | {(\prefix{x}{y}{(\outputp{x}{y} | @{y})) | P}} & \nonumber\\
	\red
	& \ldots & \nonumber\\
	\red^*
	& P | P | \ldots & \nonumber
\end{eqnarray}

Of course, this encoding, as an implementation, runs away, unfolding
$\bangp{P}$ eagerly. A lazier and more implementable replication
operator, restricted to input-guarded processes, may be obtained as follows.

\begin{eqnarray}
\bangp{\prefix{u}{v}{P}} 
	:= 
	\binpar{\lift{x}{\prefix{u}{v}{(\binpar{D(x)}{P})}}}{D(x)} \nonumber
\end{eqnarray}

\begin{remark}
  Note that the lazier definition still does not deal with summation
  or mixed summation (i.e. sums over input and output). The reader is
  invited to construct definitions of replication that deal with these
  features. 

  Further, the definitions are parameterized in a name, $x$. Can you,
  gentle reader, make a definition that eliminates this parameter and
  guarantees no accidental interaction between the replication
  machinery and the process being replicated -- i.e. no accidental
  sharing of names used by the process to get its work done and the
  name(s) used by the replication to effect copying. This latter
  revision of the definition of replication is crucial to obtaining
  the expected identity $!!P \sim !P$.
\end{remark}

\begin{remark}\label{rem:paradoxical_combinator}
  The reader familiar with the lambda calculus will have noticed the
  similarity between $D$ and the paradoxical combinator.

  [Ed. note: the existence of this seems to suggest we have to be more
  restrictive on the set of processes and names we admit if we are to
  support no-cloning.]
\end{remark}

\subsubsection{Bisimulation}

The computational dynamics gives rise to another kind of equivalence,
the equivalence of computational behavior. As previously mentioned
this is typically captured \emph{via} some form of bisimulation.

% The notion we use in this paper is weak barbed bisimulation
% \cite{milner91polyadicpi}.

The notion we use in this paper is derived from weak barbed
bisimulation \cite{milner91polyadicpi}. 

\begin{definition}
An \emph{observation relation}, $\downarrow_{\mathcal N}$, over a set
of names, $\mathcal N$, is the smallest relation satisfying the rules
below.

\infrule[Out-barb]{y \in {\mathcal N}, \; x \nameeq y}
		  {\outputp{x}{v} \downarrow_{\mathcal N} x}
\infrule[Par-barb]{\mbox{$P\downarrow_{\mathcal N} x$ or $Q\downarrow_{\mathcal N} x$}}
		  {\binpar{P}{Q} \downarrow_{\mathcal N} x}

We write $P \Downarrow_{\mathcal N} x$ if there is $Q$ such that 
$P \wred Q$ and $Q \downarrow_{\mathcal N} x$.
\end{definition}

\begin{definition}
%\label{def.bbisim}
An  ${\mathcal N}$-\emph{barbed bisimulation} over a set of names, ${\mathcal N}$, is a symmetric binary relation 
${\mathcal S}_{\mathcal N}$ between agents such that $P\rel{S}_{\mathcal N}Q$ implies:
\begin{enumerate}
\item If $P \red P'$ then $Q \wred Q'$ and $P'\rel{S}_{\mathcal N} Q'$.
\item If $P\downarrow_{\mathcal N} x$, then $Q\Downarrow_{\mathcal N} x$.
\end{enumerate}
$P$ is ${\mathcal N}$-barbed bisimilar to $Q$, written
$P \wbbisim_{\mathcal N} Q$, if $P \rel{S}_{\mathcal N} Q$ for some ${\mathcal N}$-barbed bisimulation ${\mathcal S}_{\mathcal N}$.
\end{definition}

$\mathcal{R} \subseteq \pi \times \pi$

$P \mathcal{R} Q => \forall P'. P \red P' \Rightarrow \exists Q'. Q \red Q', P' \mathcal{R} Q'$

$P \vdash x \Rightarrow Q \vdash x$

\begin{mathpar}
  \inferrule*[lab=Out-barb]{x \nameeq y}{{y}!\langle{Q}\rangle \vdash x}
  \and
  \inferrule*[lab=Par-barb]{\mbox{$P\vdash x$ or $Q\vdash x$}}{\binpar{P}{Q} \vdash x}
\end{mathpar}

\subsubsection{Contexts}

One of the principle advantages of computational calculi like the
$\pi$-calculus is a well-defined notion of context,
contextual-equivalence and a correlation between
contextual-equivalence and notions of bisimulation. The notion of
context allows the decomposition of a process into (sub-)process and
its syntactic environment, its context. Thus, a context may be
thought of as a process with a ``hole'' (written $\Box$) in it. The
application of a context $M$ to a process $P$, written $M[P]$, is
tantamount to filling the hole in $M$ with $P$. In this paper we do
not need the full weight of this theory, but do make use of the notion
of context in the proof the main theorem. 

\begin{mathpar}
  \inferrule* [lab=summation] {} {{M_{M},M_{N}} \bc \Box \;|\; x.M_{A} \;|\; M_{M}+M_{N}}
  \and
  \inferrule* [lab=agent] {} {{M_{A}} \bc (\vec{x})M_{P} \;| \; \clift{P_0,\ldots,M_{P},\ldots,P_N}}
  \and \\
  \inferrule* [lab=process] {} {{M_{P}} \bc M_{N} \;| \;P|M_{P} }
\end{mathpar} 

\begin{mathpar}
  \inferrule* [lab=sychronization] {} {M_{N} \bc \Box \;|\; x?M_{F} \;|\; x!M_{C}}
  \and
  \inferrule* [lab=abstraction] {} {{M_{F}} \bc (x)M_{P} }
  \and
  \inferrule* [lab=concretion] {} {{M_{C}} \bc \langle M_{P} \rangle }
  \and \\
  \inferrule* [lab=process] {} {{M_{P}} \bc M_{N} \;| \;P|M_{P} }
\end{mathpar}

\begin{definition}[contextual application] Given a context $M$, and
  process $P$, we define the \emph{contextual application}, $M[P] :=
  M\{P/\Box\}$. That is, the contextual application of M to P is the
  substitution of $P$ for $\Box$ in $M$.
\end{definition}

$\meaningof{-} : L \to \mathcal{P}(\pi)$

\begin{mathpar}
  \inferrule* [lab=collection] {} {\meaningof{true} = \pi, \and \meaningof{~E} = \pi \setminus \meaningof{E}, \and \meaningof{E_{1} \& E_{2}} = \meaningof{E_{1}} \cap \meaningof{E_{2}}}
\end{mathpar}

\begin{mathpar}
  \inferrule* [lab=structure] {} {\meaningof{0} = \{ P \in \pi | P \equiv 0 \}, \and \\ \meaningof{E_1 | E_2} = \{ P \in \pi | P \equiv P_{1} | P_{2}, P_{1} \in \meaningof{E_{1}}, P_{2} \in \meaningof{E_2}\} }
\end{mathpar}

\begin{mathpar}
 \inferrule* [lab=behavior] {} {\meaningof{\langle a?b \rangle E} = \{ P \in \pi | P \equiv Q | u?(y)P', \\ \and \\\\ \and \\ \;\;\; u \in \meaningof{a}, \forall z.P'\{z/y\} \in \meaningof{E\{z/b\}}\}, \and \\ \meaningof{a!E} = \{ P \in \pi | P \equiv Q | x!\langle P' \rangle, x \in \meaningof{a} P' \in \meaningof{E}\} }
\end{mathpar}

\begin{mathpar}
 \inferrule* [lab=nominal] {} {\meaningof{\quotep{E}} = \{ \quotep{P} \in \quotep{\pi} | P \in \meaningof{E} \}, \and \meaningof{\quotep{P}} = \{ \quotep{Q} \in \quotep{\pi} | P \equiv Q \} \and \\ \meaningof{@\quotep{E}} = \{ P \in \pi | P \equiv @x, x \in \meaningof{E} \}}
\end{mathpar}

\begin{eqnarray*}
  \\
  \meaningof{-} : TS \to ST
\end{eqnarray*}

\begin{eqnarray*}
  \\
  L : TS \to ST
\end{eqnarray*}

\begin{eqnarray*}
  \\
  P \models E \iff P \in \meaningof{E}
\end{eqnarray*}

\begin{eqnarray*}
  P \approx_{L} Q \iff \forall E \in L. P \models E \iff Q \models E
\end{eqnarray*}

\begin{eqnarray*}
  P \approx_{K} Q
\end{eqnarray*}

\begin{eqnarray*}
  P \approx Q
\end{eqnarray*}

$\approx_{K} = \approx = \approx_{L}$

\subsubsection{Contextual duality}

Note that contexts extend the quotation operation to a family of
operations from processes to names. Given a context, $M$, we can
define a \emph{nominal context}, $\quotep{M}$ by $\quotep{M}[P] :=
\quotep{M[P]}$. To foreshadow what is to come we observe that these
operations enjoy a duality with processes very much like the duality
between vectors and maps from vectors to scalars.

Further, because the calculus is essentially higher-order, we have a
correspondence between contexts and processes. More specifically,
given a name $x$ and a context $M$ we can construct $M^{*}_{x}$ such
that 

\begin{mathpar}
  M^{*}_{x} | \lift{x}{P} \red M[P]
\end{mathpar}

namely,

\begin{mathpar}
  M^{*}_{x} := x?(u).M[\dropn{u}]
\end{mathpar}

The dependence of $M^{*}_{x}$ on a name makes it an abstraction, 

\begin{mathpar}
  M^{*} := (x)x?(u).M[\dropn{u}]
\end{mathpar}

\subsection{Additional notation}

It will sometimes be convenient to denote the process a name
quotes. We already have the notation $x = \quotep{P}$, but it will be
convenient to introduce an alternate notation, $\procn{x}$, when we
want to emphasize the connection to the use of the name. Note that, by
virtue of name equivalence, $\quotep{\procn{x}} \nameeq x$; so, the
notation is consistent with previous definitions.

Further, because names have structure it is possible to effect
substitutions on the basis of that structure. This means we need to
upgrade our notation for substitutions, which we accomplish by
adapting comprehension notation. Thus,

\begin{mathpar}
  P\{ y / x : x \in S \}
\end{mathpar}

is interpreted to mean the process derived from P by replacing (in a
capture-avoiding manner) each occurrence of $x$ in $S$ by $y$. For example,

\begin{mathpar}
  P\{ \quotep{\procn{x}|\procn{x}} / x : x \in \freenames{P} \}
\end{mathpar}

will replace each (occurrence) of a free name $x$ in $P$ by
$\quotep{\procn{x}|\procn{x}}$.

Also, we will avail ourselves of the notation $x^{L}$ and $x^{R}$ to
denote injections of a name into disjoint copies of the name
space. There are numerous ways to accomplish this. One example can be
found in \cite{MeredithR05}. This notation overloads to vectors of
names: $\vec{x}^{\pi} := (x_{i}^{\pi} \; : \; 0 \leq i < |\vec{x}| )$ where $\pi \in \{L,R\}$.

We also use $P^{\Box} := P|\Box$.

In \cite{MeredithR05} an interpretation of the new operator is
given. It turns out that there are several possible interpretations
all enjoying the requisite algebraic properties of the operator (see
\cite{milner91polyadicpi}). We will therefore make liberal use of
$(\nu\; \vec{x})P$.

% subsection the_syntax_and_semantics_of_the_notation_system (end)   

\input{qm2pi.qmops} 

\input{qm2pi.sterngerlach} 

\input{qm2pi.metric} 

% section concurrent_process_calculi (end)

%\input{qm2pi.proofsketch}

% section proof sketch (end)

%\input{qm2pi.slviaknots} 

% section spatial logic via knots (end)

\input{qm2pi.conclusion}

% section conclusion (end)

%\input{qm2pi.dtcodes} 

% section wiring algorithm (end)

\input{qm2pi.ack} 

% section acknowledgments (end)

\newpage


\bibliographystyle{plain}   
\bibliography{../../biblios/main.bib}

\input{qm2pi.rhodetails}

\end{document}

 

% section notation (end)

\input{qm2pi.process.calculi} 

% section concurrent_process_calculi_and_spatial_logics_ (end)
    
%\documentclass[12pt]{llncs}
%\documentclass{jktr}

\usepackage[pdftex]{hyperref}                   
\usepackage {listings}
\usepackage {mathpartir}
\usepackage{bcprules}
%\usepackage{listings}
                       
\usepackage{graphicx} 
%\usepackage[margins=2.5cm,nohead,nofoot]{geometry}
%\usepackage{geometry}
\usepackage{amsfonts}
\usepackage{amstext}
\usepackage{latexsym}
\usepackage{amssymb}
\usepackage{color}


%\include{myPreamble}
\include{qm2pi.local} 

%\ifpdf
%\usepackage[pdftex]{graphicx}
%\else
%\usepackage{graphicx}
%\fi

 % \ifpdf
%  \usepackage{pdfsync}
%  \if


%\title{Brief Article}
%\author{David F. Snyder}
%\author{L.G. Meredith}

%\address{Dept. of Math., Texas State University--San Marcos, San Marcos, TX 78666}
       
\pagestyle{empty}


\begin{document}

\lstset{language=[Objective]Caml,frame=shadowbox}

\input{qm2pi.front}

% section front matter (end)

\input{qm2pi.intro} 
 
% section introduction (end)

% \input{qm2pi.knotations} 

% section notation (end)

\input{qm2pi.process.calculi} 

% section concurrent_process_calculi_and_spatial_logics_ (end)
    
%\input{qm2pi.knots2pi} 

%\input{qm2pi.trefoil} 

%\input{qm2pi.mainthm} 

% subsection basic_interpretation (end)

%\input{qm2pi.rho.presentation} 
\subsection{The syntax and semantics of the notation system}\label{sub:the_syntax_and_semantics_of_the_notation_system} % (fold)

We now summarize a technical presentation of the calculus that
embodies our theory of dynamics. The typical presentation of such a
calculus follows the style of giving generators and relations on
them. The grammar, below, describing term constructors, freely
generates the set of processes, $\Proc$. This set is then quotiented
by a relation known as structural congruence and it is over this set
that the notion of dynamics is expressed. This presentation is
essentially that of \cite{MeredithR05} with the addition of
polyadicity and summation. For readability we have relegated some of
the technical subtleties to an appendix.

\subsubsection{Process grammar}\label{subsub:process_grammar}

\begin{mathpar}
  \inferrule* [lab=synchronization] {} {{M} \bc \pzero \;|\; x?F \;|\; x!C }
  \and
  \inferrule* [lab=abstraction] {} {{F} \bc (x)P}
  \and
  \inferrule* [lab=concretion] {} {{C} \bc \langle Q \rangle}
  \and
  \inferrule* [lab=process] {} {{P,Q} \bc M \;| \;P|Q \;|\; @{x}}
  \and
  \inferrule* [lab=name] {} {{x} \bc \quotep{P}}
\end{mathpar} 

Note that $\vec{x}$ (resp. $\vec{P}$) denotes a vector of names
(resp. processes) of length $|\vec{x}|$ (resp. $|\vec{P}|$). We adopt
the following useful abbreviations.

\begin{mathpar}
   x?(\vec{y}).P := x.(\vec{y})P \and  x\clift{\vec{P}} := x.\clift{\vec{P}}
   \and x!(y) := \lift{x}{\dropn{y}}
   \and \Pi_{i=0}^{n-1}P_i := P_0 | \ldots | P_{n-1}
\end{mathpar}

\subsubsection{Structural congruence}

\paragraph{Free and bound names and alpha-equivalence.} At the
core of structural equivalence is alpha-equivalence which identifies
process that are the same up to a change of variable. Formally, we
recognize the distinction between free and bound names. The free names
of a process, $\freenames{P}$, may be calculated recursively as
follows:

\begin{mathpar}
\freenames{\pzero} := \emptyset
  \and \\
  \freenames{x?(y).P} := \{ x \} \cup (\freenames{P} \setminus \{ y \})
  \and 
  \freenames{x!\langle P \rangle} := \{ x \} \cup \{ P \} 
  \and \\
  \freenames{P|Q} := \freenames{P} \cup \freenames{Q}
  \and \\
  \freenames{@{x}} := \{ x \}
\end{mathpar}

$\pi$
$\quotep{\pi}$

$\freenames{-} : \pi \to \mathcal{P}(\quotep{\pi})$

\begin{eqnarray*}
  \freenames{\pzero} & := & \emptyset \\
  \freenames{x?(y).P} & := & \{ x \} \cup (\freenames{P} \setminus \{ y \}) \\
  \freenames{x!\langle P \rangle} & := & \{ x \} \cup \{ P \} \\
  \freenames{P|Q} & := & \freenames{P} \cup \freenames{Q} \\
  \freenames{\dropn{x}} & := & \{ x \}
\end{eqnarray*}

The bound names of a process, $\boundnames{P}$, are those names occurring in $P$
that are not free. For example, in $x?(y).0$, the name $x$ is free, while $y$ is bound.

\begin{mathpar}
  \inferrule* [lab=monoidal-laws] {} { P|Q \equiv Q|P \and P|0 \equiv P \and P|(Q|R) \equiv (P|Q)|R }
\end{mathpar}

\begin{mathpar}
  \inferrule* [lab=alpha-equivalence] {} { (x)P \equiv (y)P\{y/x\} \and y \not\in \freenames{P} }
\end{mathpar}

\begin{definition}
Then two processes, $P,Q$, are alpha-equivalent if $P = Q\{\vec{y}/\vec{x}\}$ for
some $\vec{x} \in \boundnames{Q},\vec{y} \in \boundnames{P}$, where $Q\{\vec{y}/\vec{x}\}$
denotes the capture-avoiding substitution of $\vec{y}$ for $\vec{x}$ in $Q$.
\end{definition}

\begin{definition}
  The {\em structural congruence} \cite{SangiorgiWalker} , $\equiv$,
  between processes is the least congruence containing
  alpha-equivalence, satisfying the abelian monoid laws
  (associativity, commutativity and $\pzero$ as identity) for parallel
  composition $|$ and for summation $+$.
\end{definition}

\subsection{Name equivalence}

We take name equivalence, written $\nameeq$, to be the smallest
equivalence relation generated by the following rules.

\begin{mathpar}
\inferrule*[lab=Quote-drop]
{ }
{ \quotep{@{x}} \nameeq x }

\inferrule*[lab=Struct-equiv]
{ P \scong Q }
{ \quotep{P} \nameeq \quotep{Q} }
\end{mathpar}

The astute reader will have noticed that the mutual recursion of names
and processes imposes a mutual recursion on alpha-equivalence and
structural equivalence via name-equivalence. Fortunately, all of this
works out pleasantly and we may calculate in the natural way, free of
concern. The reader interested in the details is referred to the
appendix \ref{appendix:rho_details}.

\subsection{Substitution}

We use $\Proc$ for the set of processes, $\QProc$ for the set of
names, and $\id{\{}\vec{y} / \vec{x} \id{\}}$ to denote partial maps,
$s : \QProc \rightarrow \QProc$. A map, $s$ lifts, uniquely, to a map
on process terms, $\widehat{s} : \Proc \rightarrow \Proc$ by the
following equations.

\begin{mathpar}
  (0) \psubstp{Q}{P} := 0 \\
  (R \juxtap S) \psubstp{Q}{P}
  :=    
  (R)\psubstp{Q}{P} \juxtap (S) \psubstp{Q}{P} \\
  (x?(y).R) \psubstp{Q}{P}    
  :=    
  (x)\substp{Q}{P} (z)\concat( (R \psubstn{z}{y}) \psubstp{Q}{P} ) \\
  (\lift{x}{R}) \psubstp{Q}{P}  
  :=
  \lift{(x)\substp{Q}{P}}{ R \psubstp{Q}{P} } \\
%   (\dropn{x})  \psubstp{Q}{P}       
%   := 
%   \left\{ 
%     \begin{array}{ccc} 
%       \dropn{\quotep{Q}} & & x \nameeq \quotep{P} \\
%       \dropn{x} & & otherwise \\
%     \end{array}
%   \right. 
  (\dropn{x})  \psubstp{Q}{P}       
  := 
  \left\{ 
    \begin{array}{ccc} 
      Q & & x \nameeq \quotep{P} \\
      \dropn{x} & & otherwise \\
    \end{array}
  \right.
\end{mathpar}
 

where

\begin{eqnarray}
  (x)\id{\{} \lpquote Q \rpquote / \lpquote P \rpquote \id{\}}            = 
  \left\{ 
    \begin{array}{ccc}
      \lpquote Q \rpquote & & x \nameeq \lpquote P \rpquote \\
      x & & otherwise \\
    \end{array}
  \right. \nonumber
\end{eqnarray}

and $z$ is chosen distinct from $\quotep{P}$, $\quotep{Q}$, the free
names in $Q$, and all the names in $R$. Our $\alpha$-equivalence will
be built in the standard way from this substitution.

\begin{remark}\label{rem:no_self_referential_names}
  One consequence of these definitions is that $\forall P. \quotep{P}
  \not\in \freenames{P}$.
\end{remark}

\subsection{ Dynamic quote: an example }

Anticipating something of what's to come, consider applying the
substitution, $\widehat{\id{\{}u / z \id{\}}}$, to the following pair
of processes, $\lift{w}{y!(z)}$ and $w[ \lpquote y!(z) \rpquote ]$.

\begin{eqnarray}
	\lift{w}{y!(z)}\widehat{\id{\{}u / z \id{\}}}
		& = &
		\lift{w}{y!(u)} \nonumber\\
	w[ \lpquote y!(z) \rpquote ] \widehat{ \id{\{}u / z \id{\}} }
		& = &
		w[ \lpquote y!(z) \rpquote ] \nonumber
\end{eqnarray}

Because the body of the process between quotes is impervious to
substitution, we get radically different answers. In fact, by
examining the first process in an input context,
e.g. $x?(z).\lift{w}{y!(z)}$, we see that the process under the lift
operator may be shaped by prefixed inputs binding a name inside it. In
this sense, the lift operator will be seen as a way to dynamically
construct processes before reifying them as names.

Finally equipped with these standard features we can present the
dynamics of the calculus.

\subsubsection{Operational semantics} 

Finally, we introduce the computational dynamics. What marks these
algebras as distinct from other more traditionally studied algebraic
structures, e.g. vector spaces or polynomial rings, is the manner in
which dynamics is captured. In traditional structures, dynamics is typically
expressed through morphisms between such structures, as in linear maps
between vector spaces or morphisms between rings. In algebras
associated with the semantics of computation, the dynamics is
expressed as part of the algebraic structure itself, through a
reduction reduction relation typically denoted by $\red$. Below, we
give a recursive presentation of this relation for the calculus used
in the encoding.

$\red \subseteq \pi \times \pi$
$\red : \pi \to \mathcal{P}(\pi)$

\begin{mathpar}
  \inferrule* [lab=Comm] { \textsf{match}( x_{src}, x_{trgt} ) } { x_{trgt}?(y)P \; | \; x_{src}!\langle {Q} \rangle \red P\{\quotep{Q}/y}\} }
  \and \\
  \inferrule* [lab=Par] {{P} \red {P}'} {{{P} | {Q}} \red {{P}' | {Q}}}
  \and
  \inferrule* [lab=Equiv]{{{P} \scong {P}'} \andalso {{P}' \red {Q}'} \andalso {{Q}' \scong {Q}}}{{P} \red {Q}}
\end{mathpar}

\begin{eqnarray*}
  match_{\equiv} (\quotep{P},\quotep{Q}) & := & P \equiv Q \\
  match_{\dagger}(\quotep{P},\quotep{Q}) & := & \forall R. P|Q \red^{*} R => R \red^{*} 0 \\
  match_{K}(\quotep{P},\quotep{Q}) & := & K \mbox{ for some context } K
\end{eqnarray*}

$u?(x)P | u!\langle Q \rangle \red P\{\quotep{Q}/x\}$

%We write $\wred$ for $\red^*$, and $P\red$ if $\exists Q $ such that $ P \red Q$.
We write $P\red$ if $\exists Q $ such that $ P \red Q$ and $P\not\red$, otherwise.

\section{Replication}

As mentioned before, it is known that replication (and hence
recursion) can be implemented in a higher-order process algebra
\cite{SangiorgiWalker}. As our first example of calculation with the
machinery thus far presented we give the construction explicitly in
the {\rhoc}.

\begin{eqnarray}
	D_{x} & := & \prefix{x}{y}{(\binpar{\outputp{x}{y}}{@{y}})} \nonumber\\
	\bangp_{x}{P} & := & \binpar{{x}!\langle{\binpar{D_{x}}{P}}\rangle}{D_{x}} \nonumber
\end{eqnarray}

\begin{eqnarray}
	\bangp_{x}{P} & & \nonumber\\
	=
	& {x}!\langle{(\prefix{x}{y}{(\outputp{x}{y} | @{y})) | P}}\rangle 
	      | \prefix{x}{y}{(\outputp{x}{y} | @{y})} & \nonumber\\
	\red
	& (\outputp{x}{y} | @{y})\substn{\quotep{(\prefix{x}{y}{(@{y} | \outputp{x}{y})) | P}}}{y} & \nonumber\\
	=
	& \outputp{x}{\quotep{(\prefix{x}{y}{(\outputp{x}{y} | @{y})) | P}}}
	  | {(\prefix{x}{y}{(\outputp{x}{y} | @{y})) | P}} & \nonumber\\
	\red
	& \ldots & \nonumber\\
	\red^*
	& P | P | \ldots & \nonumber
\end{eqnarray}

Of course, this encoding, as an implementation, runs away, unfolding
$\bangp{P}$ eagerly. A lazier and more implementable replication
operator, restricted to input-guarded processes, may be obtained as follows.

\begin{eqnarray}
\bangp{\prefix{u}{v}{P}} 
	:= 
	\binpar{\lift{x}{\prefix{u}{v}{(\binpar{D(x)}{P})}}}{D(x)} \nonumber
\end{eqnarray}

\begin{remark}
  Note that the lazier definition still does not deal with summation
  or mixed summation (i.e. sums over input and output). The reader is
  invited to construct definitions of replication that deal with these
  features. 

  Further, the definitions are parameterized in a name, $x$. Can you,
  gentle reader, make a definition that eliminates this parameter and
  guarantees no accidental interaction between the replication
  machinery and the process being replicated -- i.e. no accidental
  sharing of names used by the process to get its work done and the
  name(s) used by the replication to effect copying. This latter
  revision of the definition of replication is crucial to obtaining
  the expected identity $!!P \sim !P$.
\end{remark}

\begin{remark}\label{rem:paradoxical_combinator}
  The reader familiar with the lambda calculus will have noticed the
  similarity between $D$ and the paradoxical combinator.

  [Ed. note: the existence of this seems to suggest we have to be more
  restrictive on the set of processes and names we admit if we are to
  support no-cloning.]
\end{remark}

\subsubsection{Bisimulation}

The computational dynamics gives rise to another kind of equivalence,
the equivalence of computational behavior. As previously mentioned
this is typically captured \emph{via} some form of bisimulation.

% The notion we use in this paper is weak barbed bisimulation
% \cite{milner91polyadicpi}.

The notion we use in this paper is derived from weak barbed
bisimulation \cite{milner91polyadicpi}. 

\begin{definition}
An \emph{observation relation}, $\downarrow_{\mathcal N}$, over a set
of names, $\mathcal N$, is the smallest relation satisfying the rules
below.

\infrule[Out-barb]{y \in {\mathcal N}, \; x \nameeq y}
		  {\outputp{x}{v} \downarrow_{\mathcal N} x}
\infrule[Par-barb]{\mbox{$P\downarrow_{\mathcal N} x$ or $Q\downarrow_{\mathcal N} x$}}
		  {\binpar{P}{Q} \downarrow_{\mathcal N} x}

We write $P \Downarrow_{\mathcal N} x$ if there is $Q$ such that 
$P \wred Q$ and $Q \downarrow_{\mathcal N} x$.
\end{definition}

\begin{definition}
%\label{def.bbisim}
An  ${\mathcal N}$-\emph{barbed bisimulation} over a set of names, ${\mathcal N}$, is a symmetric binary relation 
${\mathcal S}_{\mathcal N}$ between agents such that $P\rel{S}_{\mathcal N}Q$ implies:
\begin{enumerate}
\item If $P \red P'$ then $Q \wred Q'$ and $P'\rel{S}_{\mathcal N} Q'$.
\item If $P\downarrow_{\mathcal N} x$, then $Q\Downarrow_{\mathcal N} x$.
\end{enumerate}
$P$ is ${\mathcal N}$-barbed bisimilar to $Q$, written
$P \wbbisim_{\mathcal N} Q$, if $P \rel{S}_{\mathcal N} Q$ for some ${\mathcal N}$-barbed bisimulation ${\mathcal S}_{\mathcal N}$.
\end{definition}

$\mathcal{R} \subseteq \pi \times \pi$

$P \mathcal{R} Q => \forall P'. P \red P' \Rightarrow \exists Q'. Q \red Q', P' \mathcal{R} Q'$

$P \vdash x \Rightarrow Q \vdash x$

\begin{mathpar}
  \inferrule*[lab=Out-barb]{x \nameeq y}{{y}!\langle{Q}\rangle \vdash x}
  \and
  \inferrule*[lab=Par-barb]{\mbox{$P\vdash x$ or $Q\vdash x$}}{\binpar{P}{Q} \vdash x}
\end{mathpar}

\subsubsection{Contexts}

One of the principle advantages of computational calculi like the
$\pi$-calculus is a well-defined notion of context,
contextual-equivalence and a correlation between
contextual-equivalence and notions of bisimulation. The notion of
context allows the decomposition of a process into (sub-)process and
its syntactic environment, its context. Thus, a context may be
thought of as a process with a ``hole'' (written $\Box$) in it. The
application of a context $M$ to a process $P$, written $M[P]$, is
tantamount to filling the hole in $M$ with $P$. In this paper we do
not need the full weight of this theory, but do make use of the notion
of context in the proof the main theorem. 

\begin{mathpar}
  \inferrule* [lab=summation] {} {{M_{M},M_{N}} \bc \Box \;|\; x.M_{A} \;|\; M_{M}+M_{N}}
  \and
  \inferrule* [lab=agent] {} {{M_{A}} \bc (\vec{x})M_{P} \;| \; \clift{P_0,\ldots,M_{P},\ldots,P_N}}
  \and \\
  \inferrule* [lab=process] {} {{M_{P}} \bc M_{N} \;| \;P|M_{P} }
\end{mathpar} 

\begin{mathpar}
  \inferrule* [lab=sychronization] {} {M_{N} \bc \Box \;|\; x?M_{F} \;|\; x!M_{C}}
  \and
  \inferrule* [lab=abstraction] {} {{M_{F}} \bc (x)M_{P} }
  \and
  \inferrule* [lab=concretion] {} {{M_{C}} \bc \langle M_{P} \rangle }
  \and \\
  \inferrule* [lab=process] {} {{M_{P}} \bc M_{N} \;| \;P|M_{P} }
\end{mathpar}

\begin{definition}[contextual application] Given a context $M$, and
  process $P$, we define the \emph{contextual application}, $M[P] :=
  M\{P/\Box\}$. That is, the contextual application of M to P is the
  substitution of $P$ for $\Box$ in $M$.
\end{definition}

$\meaningof{-} : L \to \mathcal{P}(\pi)$

\begin{mathpar}
  \inferrule* [lab=collection] {} {\meaningof{true} = \pi, \and \meaningof{~E} = \pi \setminus \meaningof{E}, \and \meaningof{E_{1} \& E_{2}} = \meaningof{E_{1}} \cap \meaningof{E_{2}}}
\end{mathpar}

\begin{mathpar}
  \inferrule* [lab=structure] {} {\meaningof{0} = \{ P \in \pi | P \equiv 0 \}, \and \\ \meaningof{E_1 | E_2} = \{ P \in \pi | P \equiv P_{1} | P_{2}, P_{1} \in \meaningof{E_{1}}, P_{2} \in \meaningof{E_2}\} }
\end{mathpar}

\begin{mathpar}
 \inferrule* [lab=behavior] {} {\meaningof{\langle a?b \rangle E} = \{ P \in \pi | P \equiv Q | u?(y)P', \\ \and \\\\ \and \\ \;\;\; u \in \meaningof{a}, \forall z.P'\{z/y\} \in \meaningof{E\{z/b\}}\}, \and \\ \meaningof{a!E} = \{ P \in \pi | P \equiv Q | x!\langle P' \rangle, x \in \meaningof{a} P' \in \meaningof{E}\} }
\end{mathpar}

\begin{mathpar}
 \inferrule* [lab=nominal] {} {\meaningof{\quotep{E}} = \{ \quotep{P} \in \quotep{\pi} | P \in \meaningof{E} \}, \and \meaningof{\quotep{P}} = \{ \quotep{Q} \in \quotep{\pi} | P \equiv Q \} \and \\ \meaningof{@\quotep{E}} = \{ P \in \pi | P \equiv @x, x \in \meaningof{E} \}}
\end{mathpar}

\begin{eqnarray*}
  \\
  \meaningof{-} : TS \to ST
\end{eqnarray*}

\begin{eqnarray*}
  \\
  L : TS \to ST
\end{eqnarray*}

\begin{eqnarray*}
  \\
  P \models E \iff P \in \meaningof{E}
\end{eqnarray*}

\begin{eqnarray*}
  P \approx_{L} Q \iff \forall E \in L. P \models E \iff Q \models E
\end{eqnarray*}

\begin{eqnarray*}
  P \approx_{K} Q
\end{eqnarray*}

\begin{eqnarray*}
  P \approx Q
\end{eqnarray*}

$\approx_{K} = \approx = \approx_{L}$

\subsubsection{Contextual duality}

Note that contexts extend the quotation operation to a family of
operations from processes to names. Given a context, $M$, we can
define a \emph{nominal context}, $\quotep{M}$ by $\quotep{M}[P] :=
\quotep{M[P]}$. To foreshadow what is to come we observe that these
operations enjoy a duality with processes very much like the duality
between vectors and maps from vectors to scalars.

Further, because the calculus is essentially higher-order, we have a
correspondence between contexts and processes. More specifically,
given a name $x$ and a context $M$ we can construct $M^{*}_{x}$ such
that 

\begin{mathpar}
  M^{*}_{x} | \lift{x}{P} \red M[P]
\end{mathpar}

namely,

\begin{mathpar}
  M^{*}_{x} := x?(u).M[\dropn{u}]
\end{mathpar}

The dependence of $M^{*}_{x}$ on a name makes it an abstraction, 

\begin{mathpar}
  M^{*} := (x)x?(u).M[\dropn{u}]
\end{mathpar}

\subsection{Additional notation}

It will sometimes be convenient to denote the process a name
quotes. We already have the notation $x = \quotep{P}$, but it will be
convenient to introduce an alternate notation, $\procn{x}$, when we
want to emphasize the connection to the use of the name. Note that, by
virtue of name equivalence, $\quotep{\procn{x}} \nameeq x$; so, the
notation is consistent with previous definitions.

Further, because names have structure it is possible to effect
substitutions on the basis of that structure. This means we need to
upgrade our notation for substitutions, which we accomplish by
adapting comprehension notation. Thus,

\begin{mathpar}
  P\{ y / x : x \in S \}
\end{mathpar}

is interpreted to mean the process derived from P by replacing (in a
capture-avoiding manner) each occurrence of $x$ in $S$ by $y$. For example,

\begin{mathpar}
  P\{ \quotep{\procn{x}|\procn{x}} / x : x \in \freenames{P} \}
\end{mathpar}

will replace each (occurrence) of a free name $x$ in $P$ by
$\quotep{\procn{x}|\procn{x}}$.

Also, we will avail ourselves of the notation $x^{L}$ and $x^{R}$ to
denote injections of a name into disjoint copies of the name
space. There are numerous ways to accomplish this. One example can be
found in \cite{MeredithR05}. This notation overloads to vectors of
names: $\vec{x}^{\pi} := (x_{i}^{\pi} \; : \; 0 \leq i < |\vec{x}| )$ where $\pi \in \{L,R\}$.

We also use $P^{\Box} := P|\Box$.

In \cite{MeredithR05} an interpretation of the new operator is
given. It turns out that there are several possible interpretations
all enjoying the requisite algebraic properties of the operator (see
\cite{milner91polyadicpi}). We will therefore make liberal use of
$(\nu\; \vec{x})P$.

% subsection the_syntax_and_semantics_of_the_notation_system (end)   

\input{qm2pi.qmops} 

\input{qm2pi.sterngerlach} 

\input{qm2pi.metric} 

% section concurrent_process_calculi (end)

%\input{qm2pi.proofsketch}

% section proof sketch (end)

%\input{qm2pi.slviaknots} 

% section spatial logic via knots (end)

\input{qm2pi.conclusion}

% section conclusion (end)

%\input{qm2pi.dtcodes} 

% section wiring algorithm (end)

\input{qm2pi.ack} 

% section acknowledgments (end)

\newpage


\bibliographystyle{plain}   
\bibliography{../../biblios/main.bib}

\input{qm2pi.rhodetails}

\end{document}

 

%\documentclass[12pt]{llncs}
%\documentclass{jktr}

\usepackage[pdftex]{hyperref}                   
\usepackage {listings}
\usepackage {mathpartir}
\usepackage{bcprules}
%\usepackage{listings}
                       
\usepackage{graphicx} 
%\usepackage[margins=2.5cm,nohead,nofoot]{geometry}
%\usepackage{geometry}
\usepackage{amsfonts}
\usepackage{amstext}
\usepackage{latexsym}
\usepackage{amssymb}
\usepackage{color}


%\include{myPreamble}
\include{qm2pi.local} 

%\ifpdf
%\usepackage[pdftex]{graphicx}
%\else
%\usepackage{graphicx}
%\fi

 % \ifpdf
%  \usepackage{pdfsync}
%  \if


%\title{Brief Article}
%\author{David F. Snyder}
%\author{L.G. Meredith}

%\address{Dept. of Math., Texas State University--San Marcos, San Marcos, TX 78666}
       
\pagestyle{empty}


\begin{document}

\lstset{language=[Objective]Caml,frame=shadowbox}

\input{qm2pi.front}

% section front matter (end)

\input{qm2pi.intro} 
 
% section introduction (end)

% \input{qm2pi.knotations} 

% section notation (end)

\input{qm2pi.process.calculi} 

% section concurrent_process_calculi_and_spatial_logics_ (end)
    
%\input{qm2pi.knots2pi} 

%\input{qm2pi.trefoil} 

%\input{qm2pi.mainthm} 

% subsection basic_interpretation (end)

%\input{qm2pi.rho.presentation} 
\subsection{The syntax and semantics of the notation system}\label{sub:the_syntax_and_semantics_of_the_notation_system} % (fold)

We now summarize a technical presentation of the calculus that
embodies our theory of dynamics. The typical presentation of such a
calculus follows the style of giving generators and relations on
them. The grammar, below, describing term constructors, freely
generates the set of processes, $\Proc$. This set is then quotiented
by a relation known as structural congruence and it is over this set
that the notion of dynamics is expressed. This presentation is
essentially that of \cite{MeredithR05} with the addition of
polyadicity and summation. For readability we have relegated some of
the technical subtleties to an appendix.

\subsubsection{Process grammar}\label{subsub:process_grammar}

\begin{mathpar}
  \inferrule* [lab=synchronization] {} {{M} \bc \pzero \;|\; x?F \;|\; x!C }
  \and
  \inferrule* [lab=abstraction] {} {{F} \bc (x)P}
  \and
  \inferrule* [lab=concretion] {} {{C} \bc \langle Q \rangle}
  \and
  \inferrule* [lab=process] {} {{P,Q} \bc M \;| \;P|Q \;|\; @{x}}
  \and
  \inferrule* [lab=name] {} {{x} \bc \quotep{P}}
\end{mathpar} 

Note that $\vec{x}$ (resp. $\vec{P}$) denotes a vector of names
(resp. processes) of length $|\vec{x}|$ (resp. $|\vec{P}|$). We adopt
the following useful abbreviations.

\begin{mathpar}
   x?(\vec{y}).P := x.(\vec{y})P \and  x\clift{\vec{P}} := x.\clift{\vec{P}}
   \and x!(y) := \lift{x}{\dropn{y}}
   \and \Pi_{i=0}^{n-1}P_i := P_0 | \ldots | P_{n-1}
\end{mathpar}

\subsubsection{Structural congruence}

\paragraph{Free and bound names and alpha-equivalence.} At the
core of structural equivalence is alpha-equivalence which identifies
process that are the same up to a change of variable. Formally, we
recognize the distinction between free and bound names. The free names
of a process, $\freenames{P}$, may be calculated recursively as
follows:

\begin{mathpar}
\freenames{\pzero} := \emptyset
  \and \\
  \freenames{x?(y).P} := \{ x \} \cup (\freenames{P} \setminus \{ y \})
  \and 
  \freenames{x!\langle P \rangle} := \{ x \} \cup \{ P \} 
  \and \\
  \freenames{P|Q} := \freenames{P} \cup \freenames{Q}
  \and \\
  \freenames{@{x}} := \{ x \}
\end{mathpar}

$\pi$
$\quotep{\pi}$

$\freenames{-} : \pi \to \mathcal{P}(\quotep{\pi})$

\begin{eqnarray*}
  \freenames{\pzero} & := & \emptyset \\
  \freenames{x?(y).P} & := & \{ x \} \cup (\freenames{P} \setminus \{ y \}) \\
  \freenames{x!\langle P \rangle} & := & \{ x \} \cup \{ P \} \\
  \freenames{P|Q} & := & \freenames{P} \cup \freenames{Q} \\
  \freenames{\dropn{x}} & := & \{ x \}
\end{eqnarray*}

The bound names of a process, $\boundnames{P}$, are those names occurring in $P$
that are not free. For example, in $x?(y).0$, the name $x$ is free, while $y$ is bound.

\begin{mathpar}
  \inferrule* [lab=monoidal-laws] {} { P|Q \equiv Q|P \and P|0 \equiv P \and P|(Q|R) \equiv (P|Q)|R }
\end{mathpar}

\begin{mathpar}
  \inferrule* [lab=alpha-equivalence] {} { (x)P \equiv (y)P\{y/x\} \and y \not\in \freenames{P} }
\end{mathpar}

\begin{definition}
Then two processes, $P,Q$, are alpha-equivalent if $P = Q\{\vec{y}/\vec{x}\}$ for
some $\vec{x} \in \boundnames{Q},\vec{y} \in \boundnames{P}$, where $Q\{\vec{y}/\vec{x}\}$
denotes the capture-avoiding substitution of $\vec{y}$ for $\vec{x}$ in $Q$.
\end{definition}

\begin{definition}
  The {\em structural congruence} \cite{SangiorgiWalker} , $\equiv$,
  between processes is the least congruence containing
  alpha-equivalence, satisfying the abelian monoid laws
  (associativity, commutativity and $\pzero$ as identity) for parallel
  composition $|$ and for summation $+$.
\end{definition}

\subsection{Name equivalence}

We take name equivalence, written $\nameeq$, to be the smallest
equivalence relation generated by the following rules.

\begin{mathpar}
\inferrule*[lab=Quote-drop]
{ }
{ \quotep{@{x}} \nameeq x }

\inferrule*[lab=Struct-equiv]
{ P \scong Q }
{ \quotep{P} \nameeq \quotep{Q} }
\end{mathpar}

The astute reader will have noticed that the mutual recursion of names
and processes imposes a mutual recursion on alpha-equivalence and
structural equivalence via name-equivalence. Fortunately, all of this
works out pleasantly and we may calculate in the natural way, free of
concern. The reader interested in the details is referred to the
appendix \ref{appendix:rho_details}.

\subsection{Substitution}

We use $\Proc$ for the set of processes, $\QProc$ for the set of
names, and $\id{\{}\vec{y} / \vec{x} \id{\}}$ to denote partial maps,
$s : \QProc \rightarrow \QProc$. A map, $s$ lifts, uniquely, to a map
on process terms, $\widehat{s} : \Proc \rightarrow \Proc$ by the
following equations.

\begin{mathpar}
  (0) \psubstp{Q}{P} := 0 \\
  (R \juxtap S) \psubstp{Q}{P}
  :=    
  (R)\psubstp{Q}{P} \juxtap (S) \psubstp{Q}{P} \\
  (x?(y).R) \psubstp{Q}{P}    
  :=    
  (x)\substp{Q}{P} (z)\concat( (R \psubstn{z}{y}) \psubstp{Q}{P} ) \\
  (\lift{x}{R}) \psubstp{Q}{P}  
  :=
  \lift{(x)\substp{Q}{P}}{ R \psubstp{Q}{P} } \\
%   (\dropn{x})  \psubstp{Q}{P}       
%   := 
%   \left\{ 
%     \begin{array}{ccc} 
%       \dropn{\quotep{Q}} & & x \nameeq \quotep{P} \\
%       \dropn{x} & & otherwise \\
%     \end{array}
%   \right. 
  (\dropn{x})  \psubstp{Q}{P}       
  := 
  \left\{ 
    \begin{array}{ccc} 
      Q & & x \nameeq \quotep{P} \\
      \dropn{x} & & otherwise \\
    \end{array}
  \right.
\end{mathpar}
 

where

\begin{eqnarray}
  (x)\id{\{} \lpquote Q \rpquote / \lpquote P \rpquote \id{\}}            = 
  \left\{ 
    \begin{array}{ccc}
      \lpquote Q \rpquote & & x \nameeq \lpquote P \rpquote \\
      x & & otherwise \\
    \end{array}
  \right. \nonumber
\end{eqnarray}

and $z$ is chosen distinct from $\quotep{P}$, $\quotep{Q}$, the free
names in $Q$, and all the names in $R$. Our $\alpha$-equivalence will
be built in the standard way from this substitution.

\begin{remark}\label{rem:no_self_referential_names}
  One consequence of these definitions is that $\forall P. \quotep{P}
  \not\in \freenames{P}$.
\end{remark}

\subsection{ Dynamic quote: an example }

Anticipating something of what's to come, consider applying the
substitution, $\widehat{\id{\{}u / z \id{\}}}$, to the following pair
of processes, $\lift{w}{y!(z)}$ and $w[ \lpquote y!(z) \rpquote ]$.

\begin{eqnarray}
	\lift{w}{y!(z)}\widehat{\id{\{}u / z \id{\}}}
		& = &
		\lift{w}{y!(u)} \nonumber\\
	w[ \lpquote y!(z) \rpquote ] \widehat{ \id{\{}u / z \id{\}} }
		& = &
		w[ \lpquote y!(z) \rpquote ] \nonumber
\end{eqnarray}

Because the body of the process between quotes is impervious to
substitution, we get radically different answers. In fact, by
examining the first process in an input context,
e.g. $x?(z).\lift{w}{y!(z)}$, we see that the process under the lift
operator may be shaped by prefixed inputs binding a name inside it. In
this sense, the lift operator will be seen as a way to dynamically
construct processes before reifying them as names.

Finally equipped with these standard features we can present the
dynamics of the calculus.

\subsubsection{Operational semantics} 

Finally, we introduce the computational dynamics. What marks these
algebras as distinct from other more traditionally studied algebraic
structures, e.g. vector spaces or polynomial rings, is the manner in
which dynamics is captured. In traditional structures, dynamics is typically
expressed through morphisms between such structures, as in linear maps
between vector spaces or morphisms between rings. In algebras
associated with the semantics of computation, the dynamics is
expressed as part of the algebraic structure itself, through a
reduction reduction relation typically denoted by $\red$. Below, we
give a recursive presentation of this relation for the calculus used
in the encoding.

$\red \subseteq \pi \times \pi$
$\red : \pi \to \mathcal{P}(\pi)$

\begin{mathpar}
  \inferrule* [lab=Comm] { \textsf{match}( x_{src}, x_{trgt} ) } { x_{trgt}?(y)P \; | \; x_{src}!\langle {Q} \rangle \red P\{\quotep{Q}/y}\} }
  \and \\
  \inferrule* [lab=Par] {{P} \red {P}'} {{{P} | {Q}} \red {{P}' | {Q}}}
  \and
  \inferrule* [lab=Equiv]{{{P} \scong {P}'} \andalso {{P}' \red {Q}'} \andalso {{Q}' \scong {Q}}}{{P} \red {Q}}
\end{mathpar}

\begin{eqnarray*}
  match_{\equiv} (\quotep{P},\quotep{Q}) & := & P \equiv Q \\
  match_{\dagger}(\quotep{P},\quotep{Q}) & := & \forall R. P|Q \red^{*} R => R \red^{*} 0 \\
  match_{K}(\quotep{P},\quotep{Q}) & := & K \mbox{ for some context } K
\end{eqnarray*}

$u?(x)P | u!\langle Q \rangle \red P\{\quotep{Q}/x\}$

%We write $\wred$ for $\red^*$, and $P\red$ if $\exists Q $ such that $ P \red Q$.
We write $P\red$ if $\exists Q $ such that $ P \red Q$ and $P\not\red$, otherwise.

\section{Replication}

As mentioned before, it is known that replication (and hence
recursion) can be implemented in a higher-order process algebra
\cite{SangiorgiWalker}. As our first example of calculation with the
machinery thus far presented we give the construction explicitly in
the {\rhoc}.

\begin{eqnarray}
	D_{x} & := & \prefix{x}{y}{(\binpar{\outputp{x}{y}}{@{y}})} \nonumber\\
	\bangp_{x}{P} & := & \binpar{{x}!\langle{\binpar{D_{x}}{P}}\rangle}{D_{x}} \nonumber
\end{eqnarray}

\begin{eqnarray}
	\bangp_{x}{P} & & \nonumber\\
	=
	& {x}!\langle{(\prefix{x}{y}{(\outputp{x}{y} | @{y})) | P}}\rangle 
	      | \prefix{x}{y}{(\outputp{x}{y} | @{y})} & \nonumber\\
	\red
	& (\outputp{x}{y} | @{y})\substn{\quotep{(\prefix{x}{y}{(@{y} | \outputp{x}{y})) | P}}}{y} & \nonumber\\
	=
	& \outputp{x}{\quotep{(\prefix{x}{y}{(\outputp{x}{y} | @{y})) | P}}}
	  | {(\prefix{x}{y}{(\outputp{x}{y} | @{y})) | P}} & \nonumber\\
	\red
	& \ldots & \nonumber\\
	\red^*
	& P | P | \ldots & \nonumber
\end{eqnarray}

Of course, this encoding, as an implementation, runs away, unfolding
$\bangp{P}$ eagerly. A lazier and more implementable replication
operator, restricted to input-guarded processes, may be obtained as follows.

\begin{eqnarray}
\bangp{\prefix{u}{v}{P}} 
	:= 
	\binpar{\lift{x}{\prefix{u}{v}{(\binpar{D(x)}{P})}}}{D(x)} \nonumber
\end{eqnarray}

\begin{remark}
  Note that the lazier definition still does not deal with summation
  or mixed summation (i.e. sums over input and output). The reader is
  invited to construct definitions of replication that deal with these
  features. 

  Further, the definitions are parameterized in a name, $x$. Can you,
  gentle reader, make a definition that eliminates this parameter and
  guarantees no accidental interaction between the replication
  machinery and the process being replicated -- i.e. no accidental
  sharing of names used by the process to get its work done and the
  name(s) used by the replication to effect copying. This latter
  revision of the definition of replication is crucial to obtaining
  the expected identity $!!P \sim !P$.
\end{remark}

\begin{remark}\label{rem:paradoxical_combinator}
  The reader familiar with the lambda calculus will have noticed the
  similarity between $D$ and the paradoxical combinator.

  [Ed. note: the existence of this seems to suggest we have to be more
  restrictive on the set of processes and names we admit if we are to
  support no-cloning.]
\end{remark}

\subsubsection{Bisimulation}

The computational dynamics gives rise to another kind of equivalence,
the equivalence of computational behavior. As previously mentioned
this is typically captured \emph{via} some form of bisimulation.

% The notion we use in this paper is weak barbed bisimulation
% \cite{milner91polyadicpi}.

The notion we use in this paper is derived from weak barbed
bisimulation \cite{milner91polyadicpi}. 

\begin{definition}
An \emph{observation relation}, $\downarrow_{\mathcal N}$, over a set
of names, $\mathcal N$, is the smallest relation satisfying the rules
below.

\infrule[Out-barb]{y \in {\mathcal N}, \; x \nameeq y}
		  {\outputp{x}{v} \downarrow_{\mathcal N} x}
\infrule[Par-barb]{\mbox{$P\downarrow_{\mathcal N} x$ or $Q\downarrow_{\mathcal N} x$}}
		  {\binpar{P}{Q} \downarrow_{\mathcal N} x}

We write $P \Downarrow_{\mathcal N} x$ if there is $Q$ such that 
$P \wred Q$ and $Q \downarrow_{\mathcal N} x$.
\end{definition}

\begin{definition}
%\label{def.bbisim}
An  ${\mathcal N}$-\emph{barbed bisimulation} over a set of names, ${\mathcal N}$, is a symmetric binary relation 
${\mathcal S}_{\mathcal N}$ between agents such that $P\rel{S}_{\mathcal N}Q$ implies:
\begin{enumerate}
\item If $P \red P'$ then $Q \wred Q'$ and $P'\rel{S}_{\mathcal N} Q'$.
\item If $P\downarrow_{\mathcal N} x$, then $Q\Downarrow_{\mathcal N} x$.
\end{enumerate}
$P$ is ${\mathcal N}$-barbed bisimilar to $Q$, written
$P \wbbisim_{\mathcal N} Q$, if $P \rel{S}_{\mathcal N} Q$ for some ${\mathcal N}$-barbed bisimulation ${\mathcal S}_{\mathcal N}$.
\end{definition}

$\mathcal{R} \subseteq \pi \times \pi$

$P \mathcal{R} Q => \forall P'. P \red P' \Rightarrow \exists Q'. Q \red Q', P' \mathcal{R} Q'$

$P \vdash x \Rightarrow Q \vdash x$

\begin{mathpar}
  \inferrule*[lab=Out-barb]{x \nameeq y}{{y}!\langle{Q}\rangle \vdash x}
  \and
  \inferrule*[lab=Par-barb]{\mbox{$P\vdash x$ or $Q\vdash x$}}{\binpar{P}{Q} \vdash x}
\end{mathpar}

\subsubsection{Contexts}

One of the principle advantages of computational calculi like the
$\pi$-calculus is a well-defined notion of context,
contextual-equivalence and a correlation between
contextual-equivalence and notions of bisimulation. The notion of
context allows the decomposition of a process into (sub-)process and
its syntactic environment, its context. Thus, a context may be
thought of as a process with a ``hole'' (written $\Box$) in it. The
application of a context $M$ to a process $P$, written $M[P]$, is
tantamount to filling the hole in $M$ with $P$. In this paper we do
not need the full weight of this theory, but do make use of the notion
of context in the proof the main theorem. 

\begin{mathpar}
  \inferrule* [lab=summation] {} {{M_{M},M_{N}} \bc \Box \;|\; x.M_{A} \;|\; M_{M}+M_{N}}
  \and
  \inferrule* [lab=agent] {} {{M_{A}} \bc (\vec{x})M_{P} \;| \; \clift{P_0,\ldots,M_{P},\ldots,P_N}}
  \and \\
  \inferrule* [lab=process] {} {{M_{P}} \bc M_{N} \;| \;P|M_{P} }
\end{mathpar} 

\begin{mathpar}
  \inferrule* [lab=sychronization] {} {M_{N} \bc \Box \;|\; x?M_{F} \;|\; x!M_{C}}
  \and
  \inferrule* [lab=abstraction] {} {{M_{F}} \bc (x)M_{P} }
  \and
  \inferrule* [lab=concretion] {} {{M_{C}} \bc \langle M_{P} \rangle }
  \and \\
  \inferrule* [lab=process] {} {{M_{P}} \bc M_{N} \;| \;P|M_{P} }
\end{mathpar}

\begin{definition}[contextual application] Given a context $M$, and
  process $P$, we define the \emph{contextual application}, $M[P] :=
  M\{P/\Box\}$. That is, the contextual application of M to P is the
  substitution of $P$ for $\Box$ in $M$.
\end{definition}

$\meaningof{-} : L \to \mathcal{P}(\pi)$

\begin{mathpar}
  \inferrule* [lab=collection] {} {\meaningof{true} = \pi, \and \meaningof{~E} = \pi \setminus \meaningof{E}, \and \meaningof{E_{1} \& E_{2}} = \meaningof{E_{1}} \cap \meaningof{E_{2}}}
\end{mathpar}

\begin{mathpar}
  \inferrule* [lab=structure] {} {\meaningof{0} = \{ P \in \pi | P \equiv 0 \}, \and \\ \meaningof{E_1 | E_2} = \{ P \in \pi | P \equiv P_{1} | P_{2}, P_{1} \in \meaningof{E_{1}}, P_{2} \in \meaningof{E_2}\} }
\end{mathpar}

\begin{mathpar}
 \inferrule* [lab=behavior] {} {\meaningof{\langle a?b \rangle E} = \{ P \in \pi | P \equiv Q | u?(y)P', \\ \and \\\\ \and \\ \;\;\; u \in \meaningof{a}, \forall z.P'\{z/y\} \in \meaningof{E\{z/b\}}\}, \and \\ \meaningof{a!E} = \{ P \in \pi | P \equiv Q | x!\langle P' \rangle, x \in \meaningof{a} P' \in \meaningof{E}\} }
\end{mathpar}

\begin{mathpar}
 \inferrule* [lab=nominal] {} {\meaningof{\quotep{E}} = \{ \quotep{P} \in \quotep{\pi} | P \in \meaningof{E} \}, \and \meaningof{\quotep{P}} = \{ \quotep{Q} \in \quotep{\pi} | P \equiv Q \} \and \\ \meaningof{@\quotep{E}} = \{ P \in \pi | P \equiv @x, x \in \meaningof{E} \}}
\end{mathpar}

\begin{eqnarray*}
  \\
  \meaningof{-} : TS \to ST
\end{eqnarray*}

\begin{eqnarray*}
  \\
  L : TS \to ST
\end{eqnarray*}

\begin{eqnarray*}
  \\
  P \models E \iff P \in \meaningof{E}
\end{eqnarray*}

\begin{eqnarray*}
  P \approx_{L} Q \iff \forall E \in L. P \models E \iff Q \models E
\end{eqnarray*}

\begin{eqnarray*}
  P \approx_{K} Q
\end{eqnarray*}

\begin{eqnarray*}
  P \approx Q
\end{eqnarray*}

$\approx_{K} = \approx = \approx_{L}$

\subsubsection{Contextual duality}

Note that contexts extend the quotation operation to a family of
operations from processes to names. Given a context, $M$, we can
define a \emph{nominal context}, $\quotep{M}$ by $\quotep{M}[P] :=
\quotep{M[P]}$. To foreshadow what is to come we observe that these
operations enjoy a duality with processes very much like the duality
between vectors and maps from vectors to scalars.

Further, because the calculus is essentially higher-order, we have a
correspondence between contexts and processes. More specifically,
given a name $x$ and a context $M$ we can construct $M^{*}_{x}$ such
that 

\begin{mathpar}
  M^{*}_{x} | \lift{x}{P} \red M[P]
\end{mathpar}

namely,

\begin{mathpar}
  M^{*}_{x} := x?(u).M[\dropn{u}]
\end{mathpar}

The dependence of $M^{*}_{x}$ on a name makes it an abstraction, 

\begin{mathpar}
  M^{*} := (x)x?(u).M[\dropn{u}]
\end{mathpar}

\subsection{Additional notation}

It will sometimes be convenient to denote the process a name
quotes. We already have the notation $x = \quotep{P}$, but it will be
convenient to introduce an alternate notation, $\procn{x}$, when we
want to emphasize the connection to the use of the name. Note that, by
virtue of name equivalence, $\quotep{\procn{x}} \nameeq x$; so, the
notation is consistent with previous definitions.

Further, because names have structure it is possible to effect
substitutions on the basis of that structure. This means we need to
upgrade our notation for substitutions, which we accomplish by
adapting comprehension notation. Thus,

\begin{mathpar}
  P\{ y / x : x \in S \}
\end{mathpar}

is interpreted to mean the process derived from P by replacing (in a
capture-avoiding manner) each occurrence of $x$ in $S$ by $y$. For example,

\begin{mathpar}
  P\{ \quotep{\procn{x}|\procn{x}} / x : x \in \freenames{P} \}
\end{mathpar}

will replace each (occurrence) of a free name $x$ in $P$ by
$\quotep{\procn{x}|\procn{x}}$.

Also, we will avail ourselves of the notation $x^{L}$ and $x^{R}$ to
denote injections of a name into disjoint copies of the name
space. There are numerous ways to accomplish this. One example can be
found in \cite{MeredithR05}. This notation overloads to vectors of
names: $\vec{x}^{\pi} := (x_{i}^{\pi} \; : \; 0 \leq i < |\vec{x}| )$ where $\pi \in \{L,R\}$.

We also use $P^{\Box} := P|\Box$.

In \cite{MeredithR05} an interpretation of the new operator is
given. It turns out that there are several possible interpretations
all enjoying the requisite algebraic properties of the operator (see
\cite{milner91polyadicpi}). We will therefore make liberal use of
$(\nu\; \vec{x})P$.

% subsection the_syntax_and_semantics_of_the_notation_system (end)   

\input{qm2pi.qmops} 

\input{qm2pi.sterngerlach} 

\input{qm2pi.metric} 

% section concurrent_process_calculi (end)

%\input{qm2pi.proofsketch}

% section proof sketch (end)

%\input{qm2pi.slviaknots} 

% section spatial logic via knots (end)

\input{qm2pi.conclusion}

% section conclusion (end)

%\input{qm2pi.dtcodes} 

% section wiring algorithm (end)

\input{qm2pi.ack} 

% section acknowledgments (end)

\newpage


\bibliographystyle{plain}   
\bibliography{../../biblios/main.bib}

\input{qm2pi.rhodetails}

\end{document}

 

%\documentclass[12pt]{llncs}
%\documentclass{jktr}

\usepackage[pdftex]{hyperref}                   
\usepackage {listings}
\usepackage {mathpartir}
\usepackage{bcprules}
%\usepackage{listings}
                       
\usepackage{graphicx} 
%\usepackage[margins=2.5cm,nohead,nofoot]{geometry}
%\usepackage{geometry}
\usepackage{amsfonts}
\usepackage{amstext}
\usepackage{latexsym}
\usepackage{amssymb}
\usepackage{color}


%\include{myPreamble}
\include{qm2pi.local} 

%\ifpdf
%\usepackage[pdftex]{graphicx}
%\else
%\usepackage{graphicx}
%\fi

 % \ifpdf
%  \usepackage{pdfsync}
%  \if


%\title{Brief Article}
%\author{David F. Snyder}
%\author{L.G. Meredith}

%\address{Dept. of Math., Texas State University--San Marcos, San Marcos, TX 78666}
       
\pagestyle{empty}


\begin{document}

\lstset{language=[Objective]Caml,frame=shadowbox}

\input{qm2pi.front}

% section front matter (end)

\input{qm2pi.intro} 
 
% section introduction (end)

% \input{qm2pi.knotations} 

% section notation (end)

\input{qm2pi.process.calculi} 

% section concurrent_process_calculi_and_spatial_logics_ (end)
    
%\input{qm2pi.knots2pi} 

%\input{qm2pi.trefoil} 

%\input{qm2pi.mainthm} 

% subsection basic_interpretation (end)

%\input{qm2pi.rho.presentation} 
\subsection{The syntax and semantics of the notation system}\label{sub:the_syntax_and_semantics_of_the_notation_system} % (fold)

We now summarize a technical presentation of the calculus that
embodies our theory of dynamics. The typical presentation of such a
calculus follows the style of giving generators and relations on
them. The grammar, below, describing term constructors, freely
generates the set of processes, $\Proc$. This set is then quotiented
by a relation known as structural congruence and it is over this set
that the notion of dynamics is expressed. This presentation is
essentially that of \cite{MeredithR05} with the addition of
polyadicity and summation. For readability we have relegated some of
the technical subtleties to an appendix.

\subsubsection{Process grammar}\label{subsub:process_grammar}

\begin{mathpar}
  \inferrule* [lab=synchronization] {} {{M} \bc \pzero \;|\; x?F \;|\; x!C }
  \and
  \inferrule* [lab=abstraction] {} {{F} \bc (x)P}
  \and
  \inferrule* [lab=concretion] {} {{C} \bc \langle Q \rangle}
  \and
  \inferrule* [lab=process] {} {{P,Q} \bc M \;| \;P|Q \;|\; @{x}}
  \and
  \inferrule* [lab=name] {} {{x} \bc \quotep{P}}
\end{mathpar} 

Note that $\vec{x}$ (resp. $\vec{P}$) denotes a vector of names
(resp. processes) of length $|\vec{x}|$ (resp. $|\vec{P}|$). We adopt
the following useful abbreviations.

\begin{mathpar}
   x?(\vec{y}).P := x.(\vec{y})P \and  x\clift{\vec{P}} := x.\clift{\vec{P}}
   \and x!(y) := \lift{x}{\dropn{y}}
   \and \Pi_{i=0}^{n-1}P_i := P_0 | \ldots | P_{n-1}
\end{mathpar}

\subsubsection{Structural congruence}

\paragraph{Free and bound names and alpha-equivalence.} At the
core of structural equivalence is alpha-equivalence which identifies
process that are the same up to a change of variable. Formally, we
recognize the distinction between free and bound names. The free names
of a process, $\freenames{P}$, may be calculated recursively as
follows:

\begin{mathpar}
\freenames{\pzero} := \emptyset
  \and \\
  \freenames{x?(y).P} := \{ x \} \cup (\freenames{P} \setminus \{ y \})
  \and 
  \freenames{x!\langle P \rangle} := \{ x \} \cup \{ P \} 
  \and \\
  \freenames{P|Q} := \freenames{P} \cup \freenames{Q}
  \and \\
  \freenames{@{x}} := \{ x \}
\end{mathpar}

$\pi$
$\quotep{\pi}$

$\freenames{-} : \pi \to \mathcal{P}(\quotep{\pi})$

\begin{eqnarray*}
  \freenames{\pzero} & := & \emptyset \\
  \freenames{x?(y).P} & := & \{ x \} \cup (\freenames{P} \setminus \{ y \}) \\
  \freenames{x!\langle P \rangle} & := & \{ x \} \cup \{ P \} \\
  \freenames{P|Q} & := & \freenames{P} \cup \freenames{Q} \\
  \freenames{\dropn{x}} & := & \{ x \}
\end{eqnarray*}

The bound names of a process, $\boundnames{P}$, are those names occurring in $P$
that are not free. For example, in $x?(y).0$, the name $x$ is free, while $y$ is bound.

\begin{mathpar}
  \inferrule* [lab=monoidal-laws] {} { P|Q \equiv Q|P \and P|0 \equiv P \and P|(Q|R) \equiv (P|Q)|R }
\end{mathpar}

\begin{mathpar}
  \inferrule* [lab=alpha-equivalence] {} { (x)P \equiv (y)P\{y/x\} \and y \not\in \freenames{P} }
\end{mathpar}

\begin{definition}
Then two processes, $P,Q$, are alpha-equivalent if $P = Q\{\vec{y}/\vec{x}\}$ for
some $\vec{x} \in \boundnames{Q},\vec{y} \in \boundnames{P}$, where $Q\{\vec{y}/\vec{x}\}$
denotes the capture-avoiding substitution of $\vec{y}$ for $\vec{x}$ in $Q$.
\end{definition}

\begin{definition}
  The {\em structural congruence} \cite{SangiorgiWalker} , $\equiv$,
  between processes is the least congruence containing
  alpha-equivalence, satisfying the abelian monoid laws
  (associativity, commutativity and $\pzero$ as identity) for parallel
  composition $|$ and for summation $+$.
\end{definition}

\subsection{Name equivalence}

We take name equivalence, written $\nameeq$, to be the smallest
equivalence relation generated by the following rules.

\begin{mathpar}
\inferrule*[lab=Quote-drop]
{ }
{ \quotep{@{x}} \nameeq x }

\inferrule*[lab=Struct-equiv]
{ P \scong Q }
{ \quotep{P} \nameeq \quotep{Q} }
\end{mathpar}

The astute reader will have noticed that the mutual recursion of names
and processes imposes a mutual recursion on alpha-equivalence and
structural equivalence via name-equivalence. Fortunately, all of this
works out pleasantly and we may calculate in the natural way, free of
concern. The reader interested in the details is referred to the
appendix \ref{appendix:rho_details}.

\subsection{Substitution}

We use $\Proc$ for the set of processes, $\QProc$ for the set of
names, and $\id{\{}\vec{y} / \vec{x} \id{\}}$ to denote partial maps,
$s : \QProc \rightarrow \QProc$. A map, $s$ lifts, uniquely, to a map
on process terms, $\widehat{s} : \Proc \rightarrow \Proc$ by the
following equations.

\begin{mathpar}
  (0) \psubstp{Q}{P} := 0 \\
  (R \juxtap S) \psubstp{Q}{P}
  :=    
  (R)\psubstp{Q}{P} \juxtap (S) \psubstp{Q}{P} \\
  (x?(y).R) \psubstp{Q}{P}    
  :=    
  (x)\substp{Q}{P} (z)\concat( (R \psubstn{z}{y}) \psubstp{Q}{P} ) \\
  (\lift{x}{R}) \psubstp{Q}{P}  
  :=
  \lift{(x)\substp{Q}{P}}{ R \psubstp{Q}{P} } \\
%   (\dropn{x})  \psubstp{Q}{P}       
%   := 
%   \left\{ 
%     \begin{array}{ccc} 
%       \dropn{\quotep{Q}} & & x \nameeq \quotep{P} \\
%       \dropn{x} & & otherwise \\
%     \end{array}
%   \right. 
  (\dropn{x})  \psubstp{Q}{P}       
  := 
  \left\{ 
    \begin{array}{ccc} 
      Q & & x \nameeq \quotep{P} \\
      \dropn{x} & & otherwise \\
    \end{array}
  \right.
\end{mathpar}
 

where

\begin{eqnarray}
  (x)\id{\{} \lpquote Q \rpquote / \lpquote P \rpquote \id{\}}            = 
  \left\{ 
    \begin{array}{ccc}
      \lpquote Q \rpquote & & x \nameeq \lpquote P \rpquote \\
      x & & otherwise \\
    \end{array}
  \right. \nonumber
\end{eqnarray}

and $z$ is chosen distinct from $\quotep{P}$, $\quotep{Q}$, the free
names in $Q$, and all the names in $R$. Our $\alpha$-equivalence will
be built in the standard way from this substitution.

\begin{remark}\label{rem:no_self_referential_names}
  One consequence of these definitions is that $\forall P. \quotep{P}
  \not\in \freenames{P}$.
\end{remark}

\subsection{ Dynamic quote: an example }

Anticipating something of what's to come, consider applying the
substitution, $\widehat{\id{\{}u / z \id{\}}}$, to the following pair
of processes, $\lift{w}{y!(z)}$ and $w[ \lpquote y!(z) \rpquote ]$.

\begin{eqnarray}
	\lift{w}{y!(z)}\widehat{\id{\{}u / z \id{\}}}
		& = &
		\lift{w}{y!(u)} \nonumber\\
	w[ \lpquote y!(z) \rpquote ] \widehat{ \id{\{}u / z \id{\}} }
		& = &
		w[ \lpquote y!(z) \rpquote ] \nonumber
\end{eqnarray}

Because the body of the process between quotes is impervious to
substitution, we get radically different answers. In fact, by
examining the first process in an input context,
e.g. $x?(z).\lift{w}{y!(z)}$, we see that the process under the lift
operator may be shaped by prefixed inputs binding a name inside it. In
this sense, the lift operator will be seen as a way to dynamically
construct processes before reifying them as names.

Finally equipped with these standard features we can present the
dynamics of the calculus.

\subsubsection{Operational semantics} 

Finally, we introduce the computational dynamics. What marks these
algebras as distinct from other more traditionally studied algebraic
structures, e.g. vector spaces or polynomial rings, is the manner in
which dynamics is captured. In traditional structures, dynamics is typically
expressed through morphisms between such structures, as in linear maps
between vector spaces or morphisms between rings. In algebras
associated with the semantics of computation, the dynamics is
expressed as part of the algebraic structure itself, through a
reduction reduction relation typically denoted by $\red$. Below, we
give a recursive presentation of this relation for the calculus used
in the encoding.

$\red \subseteq \pi \times \pi$
$\red : \pi \to \mathcal{P}(\pi)$

\begin{mathpar}
  \inferrule* [lab=Comm] { \textsf{match}( x_{src}, x_{trgt} ) } { x_{trgt}?(y)P \; | \; x_{src}!\langle {Q} \rangle \red P\{\quotep{Q}/y}\} }
  \and \\
  \inferrule* [lab=Par] {{P} \red {P}'} {{{P} | {Q}} \red {{P}' | {Q}}}
  \and
  \inferrule* [lab=Equiv]{{{P} \scong {P}'} \andalso {{P}' \red {Q}'} \andalso {{Q}' \scong {Q}}}{{P} \red {Q}}
\end{mathpar}

\begin{eqnarray*}
  match_{\equiv} (\quotep{P},\quotep{Q}) & := & P \equiv Q \\
  match_{\dagger}(\quotep{P},\quotep{Q}) & := & \forall R. P|Q \red^{*} R => R \red^{*} 0 \\
  match_{K}(\quotep{P},\quotep{Q}) & := & K \mbox{ for some context } K
\end{eqnarray*}

$u?(x)P | u!\langle Q \rangle \red P\{\quotep{Q}/x\}$

%We write $\wred$ for $\red^*$, and $P\red$ if $\exists Q $ such that $ P \red Q$.
We write $P\red$ if $\exists Q $ such that $ P \red Q$ and $P\not\red$, otherwise.

\section{Replication}

As mentioned before, it is known that replication (and hence
recursion) can be implemented in a higher-order process algebra
\cite{SangiorgiWalker}. As our first example of calculation with the
machinery thus far presented we give the construction explicitly in
the {\rhoc}.

\begin{eqnarray}
	D_{x} & := & \prefix{x}{y}{(\binpar{\outputp{x}{y}}{@{y}})} \nonumber\\
	\bangp_{x}{P} & := & \binpar{{x}!\langle{\binpar{D_{x}}{P}}\rangle}{D_{x}} \nonumber
\end{eqnarray}

\begin{eqnarray}
	\bangp_{x}{P} & & \nonumber\\
	=
	& {x}!\langle{(\prefix{x}{y}{(\outputp{x}{y} | @{y})) | P}}\rangle 
	      | \prefix{x}{y}{(\outputp{x}{y} | @{y})} & \nonumber\\
	\red
	& (\outputp{x}{y} | @{y})\substn{\quotep{(\prefix{x}{y}{(@{y} | \outputp{x}{y})) | P}}}{y} & \nonumber\\
	=
	& \outputp{x}{\quotep{(\prefix{x}{y}{(\outputp{x}{y} | @{y})) | P}}}
	  | {(\prefix{x}{y}{(\outputp{x}{y} | @{y})) | P}} & \nonumber\\
	\red
	& \ldots & \nonumber\\
	\red^*
	& P | P | \ldots & \nonumber
\end{eqnarray}

Of course, this encoding, as an implementation, runs away, unfolding
$\bangp{P}$ eagerly. A lazier and more implementable replication
operator, restricted to input-guarded processes, may be obtained as follows.

\begin{eqnarray}
\bangp{\prefix{u}{v}{P}} 
	:= 
	\binpar{\lift{x}{\prefix{u}{v}{(\binpar{D(x)}{P})}}}{D(x)} \nonumber
\end{eqnarray}

\begin{remark}
  Note that the lazier definition still does not deal with summation
  or mixed summation (i.e. sums over input and output). The reader is
  invited to construct definitions of replication that deal with these
  features. 

  Further, the definitions are parameterized in a name, $x$. Can you,
  gentle reader, make a definition that eliminates this parameter and
  guarantees no accidental interaction between the replication
  machinery and the process being replicated -- i.e. no accidental
  sharing of names used by the process to get its work done and the
  name(s) used by the replication to effect copying. This latter
  revision of the definition of replication is crucial to obtaining
  the expected identity $!!P \sim !P$.
\end{remark}

\begin{remark}\label{rem:paradoxical_combinator}
  The reader familiar with the lambda calculus will have noticed the
  similarity between $D$ and the paradoxical combinator.

  [Ed. note: the existence of this seems to suggest we have to be more
  restrictive on the set of processes and names we admit if we are to
  support no-cloning.]
\end{remark}

\subsubsection{Bisimulation}

The computational dynamics gives rise to another kind of equivalence,
the equivalence of computational behavior. As previously mentioned
this is typically captured \emph{via} some form of bisimulation.

% The notion we use in this paper is weak barbed bisimulation
% \cite{milner91polyadicpi}.

The notion we use in this paper is derived from weak barbed
bisimulation \cite{milner91polyadicpi}. 

\begin{definition}
An \emph{observation relation}, $\downarrow_{\mathcal N}$, over a set
of names, $\mathcal N$, is the smallest relation satisfying the rules
below.

\infrule[Out-barb]{y \in {\mathcal N}, \; x \nameeq y}
		  {\outputp{x}{v} \downarrow_{\mathcal N} x}
\infrule[Par-barb]{\mbox{$P\downarrow_{\mathcal N} x$ or $Q\downarrow_{\mathcal N} x$}}
		  {\binpar{P}{Q} \downarrow_{\mathcal N} x}

We write $P \Downarrow_{\mathcal N} x$ if there is $Q$ such that 
$P \wred Q$ and $Q \downarrow_{\mathcal N} x$.
\end{definition}

\begin{definition}
%\label{def.bbisim}
An  ${\mathcal N}$-\emph{barbed bisimulation} over a set of names, ${\mathcal N}$, is a symmetric binary relation 
${\mathcal S}_{\mathcal N}$ between agents such that $P\rel{S}_{\mathcal N}Q$ implies:
\begin{enumerate}
\item If $P \red P'$ then $Q \wred Q'$ and $P'\rel{S}_{\mathcal N} Q'$.
\item If $P\downarrow_{\mathcal N} x$, then $Q\Downarrow_{\mathcal N} x$.
\end{enumerate}
$P$ is ${\mathcal N}$-barbed bisimilar to $Q$, written
$P \wbbisim_{\mathcal N} Q$, if $P \rel{S}_{\mathcal N} Q$ for some ${\mathcal N}$-barbed bisimulation ${\mathcal S}_{\mathcal N}$.
\end{definition}

$\mathcal{R} \subseteq \pi \times \pi$

$P \mathcal{R} Q => \forall P'. P \red P' \Rightarrow \exists Q'. Q \red Q', P' \mathcal{R} Q'$

$P \vdash x \Rightarrow Q \vdash x$

\begin{mathpar}
  \inferrule*[lab=Out-barb]{x \nameeq y}{{y}!\langle{Q}\rangle \vdash x}
  \and
  \inferrule*[lab=Par-barb]{\mbox{$P\vdash x$ or $Q\vdash x$}}{\binpar{P}{Q} \vdash x}
\end{mathpar}

\subsubsection{Contexts}

One of the principle advantages of computational calculi like the
$\pi$-calculus is a well-defined notion of context,
contextual-equivalence and a correlation between
contextual-equivalence and notions of bisimulation. The notion of
context allows the decomposition of a process into (sub-)process and
its syntactic environment, its context. Thus, a context may be
thought of as a process with a ``hole'' (written $\Box$) in it. The
application of a context $M$ to a process $P$, written $M[P]$, is
tantamount to filling the hole in $M$ with $P$. In this paper we do
not need the full weight of this theory, but do make use of the notion
of context in the proof the main theorem. 

\begin{mathpar}
  \inferrule* [lab=summation] {} {{M_{M},M_{N}} \bc \Box \;|\; x.M_{A} \;|\; M_{M}+M_{N}}
  \and
  \inferrule* [lab=agent] {} {{M_{A}} \bc (\vec{x})M_{P} \;| \; \clift{P_0,\ldots,M_{P},\ldots,P_N}}
  \and \\
  \inferrule* [lab=process] {} {{M_{P}} \bc M_{N} \;| \;P|M_{P} }
\end{mathpar} 

\begin{mathpar}
  \inferrule* [lab=sychronization] {} {M_{N} \bc \Box \;|\; x?M_{F} \;|\; x!M_{C}}
  \and
  \inferrule* [lab=abstraction] {} {{M_{F}} \bc (x)M_{P} }
  \and
  \inferrule* [lab=concretion] {} {{M_{C}} \bc \langle M_{P} \rangle }
  \and \\
  \inferrule* [lab=process] {} {{M_{P}} \bc M_{N} \;| \;P|M_{P} }
\end{mathpar}

\begin{definition}[contextual application] Given a context $M$, and
  process $P$, we define the \emph{contextual application}, $M[P] :=
  M\{P/\Box\}$. That is, the contextual application of M to P is the
  substitution of $P$ for $\Box$ in $M$.
\end{definition}

$\meaningof{-} : L \to \mathcal{P}(\pi)$

\begin{mathpar}
  \inferrule* [lab=collection] {} {\meaningof{true} = \pi, \and \meaningof{~E} = \pi \setminus \meaningof{E}, \and \meaningof{E_{1} \& E_{2}} = \meaningof{E_{1}} \cap \meaningof{E_{2}}}
\end{mathpar}

\begin{mathpar}
  \inferrule* [lab=structure] {} {\meaningof{0} = \{ P \in \pi | P \equiv 0 \}, \and \\ \meaningof{E_1 | E_2} = \{ P \in \pi | P \equiv P_{1} | P_{2}, P_{1} \in \meaningof{E_{1}}, P_{2} \in \meaningof{E_2}\} }
\end{mathpar}

\begin{mathpar}
 \inferrule* [lab=behavior] {} {\meaningof{\langle a?b \rangle E} = \{ P \in \pi | P \equiv Q | u?(y)P', \\ \and \\\\ \and \\ \;\;\; u \in \meaningof{a}, \forall z.P'\{z/y\} \in \meaningof{E\{z/b\}}\}, \and \\ \meaningof{a!E} = \{ P \in \pi | P \equiv Q | x!\langle P' \rangle, x \in \meaningof{a} P' \in \meaningof{E}\} }
\end{mathpar}

\begin{mathpar}
 \inferrule* [lab=nominal] {} {\meaningof{\quotep{E}} = \{ \quotep{P} \in \quotep{\pi} | P \in \meaningof{E} \}, \and \meaningof{\quotep{P}} = \{ \quotep{Q} \in \quotep{\pi} | P \equiv Q \} \and \\ \meaningof{@\quotep{E}} = \{ P \in \pi | P \equiv @x, x \in \meaningof{E} \}}
\end{mathpar}

\begin{eqnarray*}
  \\
  \meaningof{-} : TS \to ST
\end{eqnarray*}

\begin{eqnarray*}
  \\
  L : TS \to ST
\end{eqnarray*}

\begin{eqnarray*}
  \\
  P \models E \iff P \in \meaningof{E}
\end{eqnarray*}

\begin{eqnarray*}
  P \approx_{L} Q \iff \forall E \in L. P \models E \iff Q \models E
\end{eqnarray*}

\begin{eqnarray*}
  P \approx_{K} Q
\end{eqnarray*}

\begin{eqnarray*}
  P \approx Q
\end{eqnarray*}

$\approx_{K} = \approx = \approx_{L}$

\subsubsection{Contextual duality}

Note that contexts extend the quotation operation to a family of
operations from processes to names. Given a context, $M$, we can
define a \emph{nominal context}, $\quotep{M}$ by $\quotep{M}[P] :=
\quotep{M[P]}$. To foreshadow what is to come we observe that these
operations enjoy a duality with processes very much like the duality
between vectors and maps from vectors to scalars.

Further, because the calculus is essentially higher-order, we have a
correspondence between contexts and processes. More specifically,
given a name $x$ and a context $M$ we can construct $M^{*}_{x}$ such
that 

\begin{mathpar}
  M^{*}_{x} | \lift{x}{P} \red M[P]
\end{mathpar}

namely,

\begin{mathpar}
  M^{*}_{x} := x?(u).M[\dropn{u}]
\end{mathpar}

The dependence of $M^{*}_{x}$ on a name makes it an abstraction, 

\begin{mathpar}
  M^{*} := (x)x?(u).M[\dropn{u}]
\end{mathpar}

\subsection{Additional notation}

It will sometimes be convenient to denote the process a name
quotes. We already have the notation $x = \quotep{P}$, but it will be
convenient to introduce an alternate notation, $\procn{x}$, when we
want to emphasize the connection to the use of the name. Note that, by
virtue of name equivalence, $\quotep{\procn{x}} \nameeq x$; so, the
notation is consistent with previous definitions.

Further, because names have structure it is possible to effect
substitutions on the basis of that structure. This means we need to
upgrade our notation for substitutions, which we accomplish by
adapting comprehension notation. Thus,

\begin{mathpar}
  P\{ y / x : x \in S \}
\end{mathpar}

is interpreted to mean the process derived from P by replacing (in a
capture-avoiding manner) each occurrence of $x$ in $S$ by $y$. For example,

\begin{mathpar}
  P\{ \quotep{\procn{x}|\procn{x}} / x : x \in \freenames{P} \}
\end{mathpar}

will replace each (occurrence) of a free name $x$ in $P$ by
$\quotep{\procn{x}|\procn{x}}$.

Also, we will avail ourselves of the notation $x^{L}$ and $x^{R}$ to
denote injections of a name into disjoint copies of the name
space. There are numerous ways to accomplish this. One example can be
found in \cite{MeredithR05}. This notation overloads to vectors of
names: $\vec{x}^{\pi} := (x_{i}^{\pi} \; : \; 0 \leq i < |\vec{x}| )$ where $\pi \in \{L,R\}$.

We also use $P^{\Box} := P|\Box$.

In \cite{MeredithR05} an interpretation of the new operator is
given. It turns out that there are several possible interpretations
all enjoying the requisite algebraic properties of the operator (see
\cite{milner91polyadicpi}). We will therefore make liberal use of
$(\nu\; \vec{x})P$.

% subsection the_syntax_and_semantics_of_the_notation_system (end)   

\input{qm2pi.qmops} 

\input{qm2pi.sterngerlach} 

\input{qm2pi.metric} 

% section concurrent_process_calculi (end)

%\input{qm2pi.proofsketch}

% section proof sketch (end)

%\input{qm2pi.slviaknots} 

% section spatial logic via knots (end)

\input{qm2pi.conclusion}

% section conclusion (end)

%\input{qm2pi.dtcodes} 

% section wiring algorithm (end)

\input{qm2pi.ack} 

% section acknowledgments (end)

\newpage


\bibliographystyle{plain}   
\bibliography{../../biblios/main.bib}

\input{qm2pi.rhodetails}

\end{document}

 

% subsection basic_interpretation (end)

%\input{qm2pi.rho.presentation} 
\subsection{The syntax and semantics of the notation system}\label{sub:the_syntax_and_semantics_of_the_notation_system} % (fold)

We now summarize a technical presentation of the calculus that
embodies our theory of dynamics. The typical presentation of such a
calculus follows the style of giving generators and relations on
them. The grammar, below, describing term constructors, freely
generates the set of processes, $\Proc$. This set is then quotiented
by a relation known as structural congruence and it is over this set
that the notion of dynamics is expressed. This presentation is
essentially that of \cite{MeredithR05} with the addition of
polyadicity and summation. For readability we have relegated some of
the technical subtleties to an appendix.

\subsubsection{Process grammar}\label{subsub:process_grammar}

\begin{mathpar}
  \inferrule* [lab=synchronization] {} {{M} \bc \pzero \;|\; x?F \;|\; x!C }
  \and
  \inferrule* [lab=abstraction] {} {{F} \bc (x)P}
  \and
  \inferrule* [lab=concretion] {} {{C} \bc \langle Q \rangle}
  \and
  \inferrule* [lab=process] {} {{P,Q} \bc M \;| \;P|Q \;|\; @{x}}
  \and
  \inferrule* [lab=name] {} {{x} \bc \quotep{P}}
\end{mathpar} 

Note that $\vec{x}$ (resp. $\vec{P}$) denotes a vector of names
(resp. processes) of length $|\vec{x}|$ (resp. $|\vec{P}|$). We adopt
the following useful abbreviations.

\begin{mathpar}
   x?(\vec{y}).P := x.(\vec{y})P \and  x\clift{\vec{P}} := x.\clift{\vec{P}}
   \and x!(y) := \lift{x}{\dropn{y}}
   \and \Pi_{i=0}^{n-1}P_i := P_0 | \ldots | P_{n-1}
\end{mathpar}

\subsubsection{Structural congruence}

\paragraph{Free and bound names and alpha-equivalence.} At the
core of structural equivalence is alpha-equivalence which identifies
process that are the same up to a change of variable. Formally, we
recognize the distinction between free and bound names. The free names
of a process, $\freenames{P}$, may be calculated recursively as
follows:

\begin{mathpar}
\freenames{\pzero} := \emptyset
  \and \\
  \freenames{x?(y).P} := \{ x \} \cup (\freenames{P} \setminus \{ y \})
  \and 
  \freenames{x!\langle P \rangle} := \{ x \} \cup \{ P \} 
  \and \\
  \freenames{P|Q} := \freenames{P} \cup \freenames{Q}
  \and \\
  \freenames{@{x}} := \{ x \}
\end{mathpar}

$\pi$
$\quotep{\pi}$

$\freenames{-} : \pi \to \mathcal{P}(\quotep{\pi})$

\begin{eqnarray*}
  \freenames{\pzero} & := & \emptyset \\
  \freenames{x?(y).P} & := & \{ x \} \cup (\freenames{P} \setminus \{ y \}) \\
  \freenames{x!\langle P \rangle} & := & \{ x \} \cup \{ P \} \\
  \freenames{P|Q} & := & \freenames{P} \cup \freenames{Q} \\
  \freenames{\dropn{x}} & := & \{ x \}
\end{eqnarray*}

The bound names of a process, $\boundnames{P}$, are those names occurring in $P$
that are not free. For example, in $x?(y).0$, the name $x$ is free, while $y$ is bound.

\begin{mathpar}
  \inferrule* [lab=monoidal-laws] {} { P|Q \equiv Q|P \and P|0 \equiv P \and P|(Q|R) \equiv (P|Q)|R }
\end{mathpar}

\begin{mathpar}
  \inferrule* [lab=alpha-equivalence] {} { (x)P \equiv (y)P\{y/x\} \and y \not\in \freenames{P} }
\end{mathpar}

\begin{definition}
Then two processes, $P,Q$, are alpha-equivalent if $P = Q\{\vec{y}/\vec{x}\}$ for
some $\vec{x} \in \boundnames{Q},\vec{y} \in \boundnames{P}$, where $Q\{\vec{y}/\vec{x}\}$
denotes the capture-avoiding substitution of $\vec{y}$ for $\vec{x}$ in $Q$.
\end{definition}

\begin{definition}
  The {\em structural congruence} \cite{SangiorgiWalker} , $\equiv$,
  between processes is the least congruence containing
  alpha-equivalence, satisfying the abelian monoid laws
  (associativity, commutativity and $\pzero$ as identity) for parallel
  composition $|$ and for summation $+$.
\end{definition}

\subsection{Name equivalence}

We take name equivalence, written $\nameeq$, to be the smallest
equivalence relation generated by the following rules.

\begin{mathpar}
\inferrule*[lab=Quote-drop]
{ }
{ \quotep{@{x}} \nameeq x }

\inferrule*[lab=Struct-equiv]
{ P \scong Q }
{ \quotep{P} \nameeq \quotep{Q} }
\end{mathpar}

The astute reader will have noticed that the mutual recursion of names
and processes imposes a mutual recursion on alpha-equivalence and
structural equivalence via name-equivalence. Fortunately, all of this
works out pleasantly and we may calculate in the natural way, free of
concern. The reader interested in the details is referred to the
appendix \ref{appendix:rho_details}.

\subsection{Substitution}

We use $\Proc$ for the set of processes, $\QProc$ for the set of
names, and $\id{\{}\vec{y} / \vec{x} \id{\}}$ to denote partial maps,
$s : \QProc \rightarrow \QProc$. A map, $s$ lifts, uniquely, to a map
on process terms, $\widehat{s} : \Proc \rightarrow \Proc$ by the
following equations.

\begin{mathpar}
  (0) \psubstp{Q}{P} := 0 \\
  (R \juxtap S) \psubstp{Q}{P}
  :=    
  (R)\psubstp{Q}{P} \juxtap (S) \psubstp{Q}{P} \\
  (x?(y).R) \psubstp{Q}{P}    
  :=    
  (x)\substp{Q}{P} (z)\concat( (R \psubstn{z}{y}) \psubstp{Q}{P} ) \\
  (\lift{x}{R}) \psubstp{Q}{P}  
  :=
  \lift{(x)\substp{Q}{P}}{ R \psubstp{Q}{P} } \\
%   (\dropn{x})  \psubstp{Q}{P}       
%   := 
%   \left\{ 
%     \begin{array}{ccc} 
%       \dropn{\quotep{Q}} & & x \nameeq \quotep{P} \\
%       \dropn{x} & & otherwise \\
%     \end{array}
%   \right. 
  (\dropn{x})  \psubstp{Q}{P}       
  := 
  \left\{ 
    \begin{array}{ccc} 
      Q & & x \nameeq \quotep{P} \\
      \dropn{x} & & otherwise \\
    \end{array}
  \right.
\end{mathpar}
 

where

\begin{eqnarray}
  (x)\id{\{} \lpquote Q \rpquote / \lpquote P \rpquote \id{\}}            = 
  \left\{ 
    \begin{array}{ccc}
      \lpquote Q \rpquote & & x \nameeq \lpquote P \rpquote \\
      x & & otherwise \\
    \end{array}
  \right. \nonumber
\end{eqnarray}

and $z$ is chosen distinct from $\quotep{P}$, $\quotep{Q}$, the free
names in $Q$, and all the names in $R$. Our $\alpha$-equivalence will
be built in the standard way from this substitution.

\begin{remark}\label{rem:no_self_referential_names}
  One consequence of these definitions is that $\forall P. \quotep{P}
  \not\in \freenames{P}$.
\end{remark}

\subsection{ Dynamic quote: an example }

Anticipating something of what's to come, consider applying the
substitution, $\widehat{\id{\{}u / z \id{\}}}$, to the following pair
of processes, $\lift{w}{y!(z)}$ and $w[ \lpquote y!(z) \rpquote ]$.

\begin{eqnarray}
	\lift{w}{y!(z)}\widehat{\id{\{}u / z \id{\}}}
		& = &
		\lift{w}{y!(u)} \nonumber\\
	w[ \lpquote y!(z) \rpquote ] \widehat{ \id{\{}u / z \id{\}} }
		& = &
		w[ \lpquote y!(z) \rpquote ] \nonumber
\end{eqnarray}

Because the body of the process between quotes is impervious to
substitution, we get radically different answers. In fact, by
examining the first process in an input context,
e.g. $x?(z).\lift{w}{y!(z)}$, we see that the process under the lift
operator may be shaped by prefixed inputs binding a name inside it. In
this sense, the lift operator will be seen as a way to dynamically
construct processes before reifying them as names.

Finally equipped with these standard features we can present the
dynamics of the calculus.

\subsubsection{Operational semantics} 

Finally, we introduce the computational dynamics. What marks these
algebras as distinct from other more traditionally studied algebraic
structures, e.g. vector spaces or polynomial rings, is the manner in
which dynamics is captured. In traditional structures, dynamics is typically
expressed through morphisms between such structures, as in linear maps
between vector spaces or morphisms between rings. In algebras
associated with the semantics of computation, the dynamics is
expressed as part of the algebraic structure itself, through a
reduction reduction relation typically denoted by $\red$. Below, we
give a recursive presentation of this relation for the calculus used
in the encoding.

$\red \subseteq \pi \times \pi$
$\red : \pi \to \mathcal{P}(\pi)$

\begin{mathpar}
  \inferrule* [lab=Comm] { \textsf{match}( x_{src}, x_{trgt} ) } { x_{trgt}?(y)P \; | \; x_{src}!\langle {Q} \rangle \red P\{\quotep{Q}/y}\} }
  \and \\
  \inferrule* [lab=Par] {{P} \red {P}'} {{{P} | {Q}} \red {{P}' | {Q}}}
  \and
  \inferrule* [lab=Equiv]{{{P} \scong {P}'} \andalso {{P}' \red {Q}'} \andalso {{Q}' \scong {Q}}}{{P} \red {Q}}
\end{mathpar}

\begin{eqnarray*}
  match_{\equiv} (\quotep{P},\quotep{Q}) & := & P \equiv Q \\
  match_{\dagger}(\quotep{P},\quotep{Q}) & := & \forall R. P|Q \red^{*} R => R \red^{*} 0 \\
  match_{K}(\quotep{P},\quotep{Q}) & := & K \mbox{ for some context } K
\end{eqnarray*}

$u?(x)P | u!\langle Q \rangle \red P\{\quotep{Q}/x\}$

%We write $\wred$ for $\red^*$, and $P\red$ if $\exists Q $ such that $ P \red Q$.
We write $P\red$ if $\exists Q $ such that $ P \red Q$ and $P\not\red$, otherwise.

\section{Replication}

As mentioned before, it is known that replication (and hence
recursion) can be implemented in a higher-order process algebra
\cite{SangiorgiWalker}. As our first example of calculation with the
machinery thus far presented we give the construction explicitly in
the {\rhoc}.

\begin{eqnarray}
	D_{x} & := & \prefix{x}{y}{(\binpar{\outputp{x}{y}}{@{y}})} \nonumber\\
	\bangp_{x}{P} & := & \binpar{{x}!\langle{\binpar{D_{x}}{P}}\rangle}{D_{x}} \nonumber
\end{eqnarray}

\begin{eqnarray}
	\bangp_{x}{P} & & \nonumber\\
	=
	& {x}!\langle{(\prefix{x}{y}{(\outputp{x}{y} | @{y})) | P}}\rangle 
	      | \prefix{x}{y}{(\outputp{x}{y} | @{y})} & \nonumber\\
	\red
	& (\outputp{x}{y} | @{y})\substn{\quotep{(\prefix{x}{y}{(@{y} | \outputp{x}{y})) | P}}}{y} & \nonumber\\
	=
	& \outputp{x}{\quotep{(\prefix{x}{y}{(\outputp{x}{y} | @{y})) | P}}}
	  | {(\prefix{x}{y}{(\outputp{x}{y} | @{y})) | P}} & \nonumber\\
	\red
	& \ldots & \nonumber\\
	\red^*
	& P | P | \ldots & \nonumber
\end{eqnarray}

Of course, this encoding, as an implementation, runs away, unfolding
$\bangp{P}$ eagerly. A lazier and more implementable replication
operator, restricted to input-guarded processes, may be obtained as follows.

\begin{eqnarray}
\bangp{\prefix{u}{v}{P}} 
	:= 
	\binpar{\lift{x}{\prefix{u}{v}{(\binpar{D(x)}{P})}}}{D(x)} \nonumber
\end{eqnarray}

\begin{remark}
  Note that the lazier definition still does not deal with summation
  or mixed summation (i.e. sums over input and output). The reader is
  invited to construct definitions of replication that deal with these
  features. 

  Further, the definitions are parameterized in a name, $x$. Can you,
  gentle reader, make a definition that eliminates this parameter and
  guarantees no accidental interaction between the replication
  machinery and the process being replicated -- i.e. no accidental
  sharing of names used by the process to get its work done and the
  name(s) used by the replication to effect copying. This latter
  revision of the definition of replication is crucial to obtaining
  the expected identity $!!P \sim !P$.
\end{remark}

\begin{remark}\label{rem:paradoxical_combinator}
  The reader familiar with the lambda calculus will have noticed the
  similarity between $D$ and the paradoxical combinator.

  [Ed. note: the existence of this seems to suggest we have to be more
  restrictive on the set of processes and names we admit if we are to
  support no-cloning.]
\end{remark}

\subsubsection{Bisimulation}

The computational dynamics gives rise to another kind of equivalence,
the equivalence of computational behavior. As previously mentioned
this is typically captured \emph{via} some form of bisimulation.

% The notion we use in this paper is weak barbed bisimulation
% \cite{milner91polyadicpi}.

The notion we use in this paper is derived from weak barbed
bisimulation \cite{milner91polyadicpi}. 

\begin{definition}
An \emph{observation relation}, $\downarrow_{\mathcal N}$, over a set
of names, $\mathcal N$, is the smallest relation satisfying the rules
below.

\infrule[Out-barb]{y \in {\mathcal N}, \; x \nameeq y}
		  {\outputp{x}{v} \downarrow_{\mathcal N} x}
\infrule[Par-barb]{\mbox{$P\downarrow_{\mathcal N} x$ or $Q\downarrow_{\mathcal N} x$}}
		  {\binpar{P}{Q} \downarrow_{\mathcal N} x}

We write $P \Downarrow_{\mathcal N} x$ if there is $Q$ such that 
$P \wred Q$ and $Q \downarrow_{\mathcal N} x$.
\end{definition}

\begin{definition}
%\label{def.bbisim}
An  ${\mathcal N}$-\emph{barbed bisimulation} over a set of names, ${\mathcal N}$, is a symmetric binary relation 
${\mathcal S}_{\mathcal N}$ between agents such that $P\rel{S}_{\mathcal N}Q$ implies:
\begin{enumerate}
\item If $P \red P'$ then $Q \wred Q'$ and $P'\rel{S}_{\mathcal N} Q'$.
\item If $P\downarrow_{\mathcal N} x$, then $Q\Downarrow_{\mathcal N} x$.
\end{enumerate}
$P$ is ${\mathcal N}$-barbed bisimilar to $Q$, written
$P \wbbisim_{\mathcal N} Q$, if $P \rel{S}_{\mathcal N} Q$ for some ${\mathcal N}$-barbed bisimulation ${\mathcal S}_{\mathcal N}$.
\end{definition}

$\mathcal{R} \subseteq \pi \times \pi$

$P \mathcal{R} Q => \forall P'. P \red P' \Rightarrow \exists Q'. Q \red Q', P' \mathcal{R} Q'$

$P \vdash x \Rightarrow Q \vdash x$

\begin{mathpar}
  \inferrule*[lab=Out-barb]{x \nameeq y}{{y}!\langle{Q}\rangle \vdash x}
  \and
  \inferrule*[lab=Par-barb]{\mbox{$P\vdash x$ or $Q\vdash x$}}{\binpar{P}{Q} \vdash x}
\end{mathpar}

\subsubsection{Contexts}

One of the principle advantages of computational calculi like the
$\pi$-calculus is a well-defined notion of context,
contextual-equivalence and a correlation between
contextual-equivalence and notions of bisimulation. The notion of
context allows the decomposition of a process into (sub-)process and
its syntactic environment, its context. Thus, a context may be
thought of as a process with a ``hole'' (written $\Box$) in it. The
application of a context $M$ to a process $P$, written $M[P]$, is
tantamount to filling the hole in $M$ with $P$. In this paper we do
not need the full weight of this theory, but do make use of the notion
of context in the proof the main theorem. 

\begin{mathpar}
  \inferrule* [lab=summation] {} {{M_{M},M_{N}} \bc \Box \;|\; x.M_{A} \;|\; M_{M}+M_{N}}
  \and
  \inferrule* [lab=agent] {} {{M_{A}} \bc (\vec{x})M_{P} \;| \; \clift{P_0,\ldots,M_{P},\ldots,P_N}}
  \and \\
  \inferrule* [lab=process] {} {{M_{P}} \bc M_{N} \;| \;P|M_{P} }
\end{mathpar} 

\begin{mathpar}
  \inferrule* [lab=sychronization] {} {M_{N} \bc \Box \;|\; x?M_{F} \;|\; x!M_{C}}
  \and
  \inferrule* [lab=abstraction] {} {{M_{F}} \bc (x)M_{P} }
  \and
  \inferrule* [lab=concretion] {} {{M_{C}} \bc \langle M_{P} \rangle }
  \and \\
  \inferrule* [lab=process] {} {{M_{P}} \bc M_{N} \;| \;P|M_{P} }
\end{mathpar}

\begin{definition}[contextual application] Given a context $M$, and
  process $P$, we define the \emph{contextual application}, $M[P] :=
  M\{P/\Box\}$. That is, the contextual application of M to P is the
  substitution of $P$ for $\Box$ in $M$.
\end{definition}

$\meaningof{-} : L \to \mathcal{P}(\pi)$

\begin{mathpar}
  \inferrule* [lab=collection] {} {\meaningof{true} = \pi, \and \meaningof{~E} = \pi \setminus \meaningof{E}, \and \meaningof{E_{1} \& E_{2}} = \meaningof{E_{1}} \cap \meaningof{E_{2}}}
\end{mathpar}

\begin{mathpar}
  \inferrule* [lab=structure] {} {\meaningof{0} = \{ P \in \pi | P \equiv 0 \}, \and \\ \meaningof{E_1 | E_2} = \{ P \in \pi | P \equiv P_{1} | P_{2}, P_{1} \in \meaningof{E_{1}}, P_{2} \in \meaningof{E_2}\} }
\end{mathpar}

\begin{mathpar}
 \inferrule* [lab=behavior] {} {\meaningof{\langle a?b \rangle E} = \{ P \in \pi | P \equiv Q | u?(y)P', \\ \and \\\\ \and \\ \;\;\; u \in \meaningof{a}, \forall z.P'\{z/y\} \in \meaningof{E\{z/b\}}\}, \and \\ \meaningof{a!E} = \{ P \in \pi | P \equiv Q | x!\langle P' \rangle, x \in \meaningof{a} P' \in \meaningof{E}\} }
\end{mathpar}

\begin{mathpar}
 \inferrule* [lab=nominal] {} {\meaningof{\quotep{E}} = \{ \quotep{P} \in \quotep{\pi} | P \in \meaningof{E} \}, \and \meaningof{\quotep{P}} = \{ \quotep{Q} \in \quotep{\pi} | P \equiv Q \} \and \\ \meaningof{@\quotep{E}} = \{ P \in \pi | P \equiv @x, x \in \meaningof{E} \}}
\end{mathpar}

\begin{eqnarray*}
  \\
  \meaningof{-} : TS \to ST
\end{eqnarray*}

\begin{eqnarray*}
  \\
  L : TS \to ST
\end{eqnarray*}

\begin{eqnarray*}
  \\
  P \models E \iff P \in \meaningof{E}
\end{eqnarray*}

\begin{eqnarray*}
  P \approx_{L} Q \iff \forall E \in L. P \models E \iff Q \models E
\end{eqnarray*}

\begin{eqnarray*}
  P \approx_{K} Q
\end{eqnarray*}

\begin{eqnarray*}
  P \approx Q
\end{eqnarray*}

$\approx_{K} = \approx = \approx_{L}$

\subsubsection{Contextual duality}

Note that contexts extend the quotation operation to a family of
operations from processes to names. Given a context, $M$, we can
define a \emph{nominal context}, $\quotep{M}$ by $\quotep{M}[P] :=
\quotep{M[P]}$. To foreshadow what is to come we observe that these
operations enjoy a duality with processes very much like the duality
between vectors and maps from vectors to scalars.

Further, because the calculus is essentially higher-order, we have a
correspondence between contexts and processes. More specifically,
given a name $x$ and a context $M$ we can construct $M^{*}_{x}$ such
that 

\begin{mathpar}
  M^{*}_{x} | \lift{x}{P} \red M[P]
\end{mathpar}

namely,

\begin{mathpar}
  M^{*}_{x} := x?(u).M[\dropn{u}]
\end{mathpar}

The dependence of $M^{*}_{x}$ on a name makes it an abstraction, 

\begin{mathpar}
  M^{*} := (x)x?(u).M[\dropn{u}]
\end{mathpar}

\subsection{Additional notation}

It will sometimes be convenient to denote the process a name
quotes. We already have the notation $x = \quotep{P}$, but it will be
convenient to introduce an alternate notation, $\procn{x}$, when we
want to emphasize the connection to the use of the name. Note that, by
virtue of name equivalence, $\quotep{\procn{x}} \nameeq x$; so, the
notation is consistent with previous definitions.

Further, because names have structure it is possible to effect
substitutions on the basis of that structure. This means we need to
upgrade our notation for substitutions, which we accomplish by
adapting comprehension notation. Thus,

\begin{mathpar}
  P\{ y / x : x \in S \}
\end{mathpar}

is interpreted to mean the process derived from P by replacing (in a
capture-avoiding manner) each occurrence of $x$ in $S$ by $y$. For example,

\begin{mathpar}
  P\{ \quotep{\procn{x}|\procn{x}} / x : x \in \freenames{P} \}
\end{mathpar}

will replace each (occurrence) of a free name $x$ in $P$ by
$\quotep{\procn{x}|\procn{x}}$.

Also, we will avail ourselves of the notation $x^{L}$ and $x^{R}$ to
denote injections of a name into disjoint copies of the name
space. There are numerous ways to accomplish this. One example can be
found in \cite{MeredithR05}. This notation overloads to vectors of
names: $\vec{x}^{\pi} := (x_{i}^{\pi} \; : \; 0 \leq i < |\vec{x}| )$ where $\pi \in \{L,R\}$.

We also use $P^{\Box} := P|\Box$.

In \cite{MeredithR05} an interpretation of the new operator is
given. It turns out that there are several possible interpretations
all enjoying the requisite algebraic properties of the operator (see
\cite{milner91polyadicpi}). We will therefore make liberal use of
$(\nu\; \vec{x})P$.

% subsection the_syntax_and_semantics_of_the_notation_system (end)   

\section{Interpretation of QM}
\subsection{Supporting definitions}
\subsubsection{Multiplication}
\begin{mathpar}
  \quotep{Q} \cdot \quotep{R} := \quotep{Q|R}
  \and \\
  \quotep{Q} \cdot P := P\{ \quotep{Q|R} / \quotep{R} : \quotep{R} \in \freenames{P} \}
\end{mathpar}

\paragraph{Discussion}
The first line needs little explanation. The second line says that
each free name of the process is replaced with the multiplication of
that name by the scalar. Multiplication of a scalar (name) by a state
(process) results in a process all the names of which have been `moved
over' by parallel composition with the process the scalar
quotes. There is a subtlety that the bound names have to be
manipulated so that multiplied names aren't accidentally
captured. There are many ways to achieve this.

\begin{remark}\label{rem:multiplication_identities}
  The reader is invited to verify that for all $x,y,z \in \QProc$ and $P \in \Proc$
  \begin{mathpar}
    x \cdot \quotep{0} \equiv x 
    \and
    x \cdot y \equiv y \cdot x
    \and
    x \cdot (y \cdot z) \equiv (x \cdot y) \cdot z
    \and \\
    \quotep{0} \cdot P \equiv P
    \and \\
    x \cdot (y \cdot P) \equiv (x \cdot y) \cdot P
    \and \\
    x \cdot (P|Q) \equiv (x \cdot P) | (x \cdot Q)
    \and \\    
  \end{mathpar}
\end{remark}

\subsubsection{Tensor product}

We define a tensor product on processes by structural induction.

\paragraph{Tensor of sums} First note that all summations, including
$\pzero$ and sequence, can be written $\Sigma_{i} x_{i}.A_{i} +
\Sigma_{j} x_{j}.C_{j}$, where we have grouped input-guarded processes
together and output-guarded processes together.

Thus, we can define the tensor product of two summations, $N_{1}\otimes N_{2}$, where

\begin{mathpar}
  N_{1} := \Sigma_{i} x_{i}.A_{i} + \Sigma_{j} x_{j}.C_{j}
  \and
  N_{2} := \Sigma_{i'} y_{i'}.B_{i'} + \Sigma_{j'} y_{j'}.D_{j'} 
\end{mathpar}

as follows.

\begin{mathpar}
  \Sigma_{i} x_{i}.A_{i} + \Sigma_{j} x_{j}.C_{j} \otimes \Sigma_{i'}
  y_{i'}.B_{i'} + \Sigma_{j'} y_{j'}.D_{j'} 
  \and \\
  := \; \Sigma_{i} \Sigma_{i'} \quotep{\stackrel{\vee}{x_{i}}| \stackrel{\vee}{y_{i'}}}.(A_{i}\otimes B_{i'}) \; | \; \Sigma_{i'} \Sigma_{i} \quotep{\stackrel{\vee}{y_{i'}}|\stackrel{\vee}{x_{i}}}.(B_{i'}\otimes A_{i})
  \and
  \;\; | \;\; \Sigma_{j} \Sigma_{j'} \quotep{\stackrel{\vee}{x_{j}}|\stackrel{\vee}{y_{j'}}}.(A_{j}\otimes B_{j'}) \; | \; \Sigma_{j'} \Sigma_{j} \quotep{\stackrel{\vee}{y_{j'}}|\stackrel{\vee}{x_{j}}}.(B_{j'}\otimes A_{j})
\end{mathpar}

\begin{remark}
  Do we need to $x^{L}$ and $y^{R}$ for this construction as well?
\end{remark}

\paragraph{Tensor of parallel compositions} Next, we distribute tensor
over par.

\begin{mathpar}
  P_{1}|P_{2} \otimes Q_{1}|Q_{2} := (P_{1} \otimes Q_{1}) | (P_{1}
  \otimes Q_{2}) | (P_{2} \otimes Q_{1}) | (P_{2} \otimes Q_{2})
\end{mathpar}

\paragraph{Tensor with dropped names} We treat tensor of a
process with a dropped name as parallel composition.

\begin{mathpar}
  P \otimes \dropn{x} := P | \dropn{x}
\end{mathpar}

\paragraph{Tensor of agents}

Finally, we need to define tensor on agents. Note that the definition
of tensor on normal products only tensors inputs with inputs and
outputs with outputs. Thus, we only have to define the operation on
``homogeneous'' pairings.

\begin{mathpar}
  (\vec{x})P \otimes (\vec{y})Q
  \and \\
  := (x_{0}^{L}|y_{0}^{R},\ldots,x_{0}^{L}|y_{n}^{R},\ldots,x_{m}^{L}|y_{0}^{R},\ldots,x_{m}^{L}|y_{n}^R)(P\{ \vec{x}^{L}/\vec{x}\} \otimes Q \{ \vec{y}^{R}/\vec{y}\})
  \and \\
  \clift{\vec{P}} \otimes \clift{\vec{Q}}
  \and \\
  := \clift{P_{0}\otimes Q_{0},\ldots,P_{0}\otimes Q_{n},\ldots,P_{m}\otimes Q_{0},\ldots,P_{m}\otimes Q_{n}}
\end{mathpar}

\begin{remark}
  Observe that arities of tensored abstractions matches arities of
  tensored concretions if the original arities matched. Note also that
  the length of the arities corresponds to the increase in dimension
  we see in ordinary vector space tensor product.
\end{remark}

\begin{remark}
  Operationally, this definition distributes the tensor down to
  components ``linked'' by summation. Tensor over summation is
  intriguing in that it mixes names. Moreover, as a consequence of the
  way it mixes names we have the identities for all $x \in \QProc$ and
  $P,Q \in \Proc$

  \begin{mathpar}
    (x \cdot P) \otimes Q \equiv x \cdot (P \otimes Q) \equiv P \otimes (x \cdot Q)
    \and
    P \otimes \pzero \equiv P
  \end{mathpar}

  that the reader is invited to verify.
\end{remark}

\subsubsection{Annihilation}
\begin{mathpar}
  P^{\perp} := \{ Q | \forall R. P|Q \red^{*} R \Rightarrow R \red^{*} \pzero \}
  \and \\
  P^{\underline{\perp}} := \Sigma_{Q \in P^{\perp}} \quotep{Q}?(y).(\dropn{y}|Q) | \Sigma_{Q \in P^{\perp}} \quotep{Q}\clift{\Box}
\end{mathpar}

\paragraph{Discussion} The reader will note that $P^{\perp}$ is a
\emph{set} of processes, while $P^{\underline{\perp}}$ is a
\emph{context}. We call the set $P^{\perp}$ the \emph{annihilators} of
$P$. The parallel composition of a process in the annihilators of $P$
with $P$ will result in a process, the state space of which has all
paths eventually leading to $\pzero$. Execution may endure loops; but
under reasonable conditions of fairness (naturally guaranteed under
most notions of bisimulation) such a composite process cannot get
stuck in such a loop and will, eventually pop out and terminate.

The context $P^{\underline{\perp}}$ is ready and willing to ``take the
$P$ out of'' the process to which it is applied. It will effectively
transmit the code of the process to which it is applied to one of the
annihilators and run the process against it.

\subsubsection{Evaluation}
We fix $M$ a domain of fully abstract interpretation with an equality
coincident with bisimulation. We take $\meaningof{\cdot} : \Proc \to
M$ to be the map interpreting processes and $\nmeaningof{\cdot} : \M
\to Proc$ to be the map running the other way. Then we define

\begin{mathpar}
  \int P := \nmeaningof{\meaningof{P}}
\end{mathpar}

\paragraph{Discussion}
There are many fully abstract interpretations of Milner's
$\pi$-calculus. Any of them can be used as a basis for interpreting
the reflective calculus here. Equipped with such a domain it is
largely a matter of grinding through to check that the Yoneda
construction for the normalization-by-evaluation program can be
extended to this setting.

\begin{remark}
  The reader is invited to verify that $\int (P^{\underline{\perp}}[P]) = 0$.
\end{remark}

\subsection{Quantum mechanics}

Table \ref{tbl:core_qm_op_defns} gives the core operational definitions

\begin{table}[htp]\label{tbl:core_qm_op_defns}
  \center{
    \fbox{
      \begin{tabular}{c|c}
        quantum mechanics & process calculus \\
        \hline
        scalar & $x := \quotep{P}$ \\
        state vector & $\state{P} := P$ \\
        dual & $\state{P}^{*} := \event{P^{\underline{\perp}}} := \quotep{P^{\underline{\perp}}}[-]$ \\
        matrix & $ \Sigma_{\alpha} \state{P_{\alpha}}x_{\alpha}\event{Q_{\alpha}}$ \\
        vector addition & $\state{P} + \state{Q} := \state{P | Q}$ \\
        tensor product & $\state{P} \otimes \state{Q} := \state{P \otimes Q}$ \\
        inner product & $\innerprod{P}{Q} := \quotep{\int P^{\underline{\perp}}[Q]}$ \\
      \end{tabular}
    }
  }
  \caption{QM - operational definitions}
\end{table}

where

\begin{mathpar}
  \prmatrix{P}{Q} := \fprmatrix{P}{\quotep{\pzero}}{Q}
  \and
  \fprmatrix{P}{x}{Q} := (\state{P},x,\event{Q})
  \and
  (\fprmatrix{P}{x}{Q})(\state{R}) := x \cdot \innerprod{Q}{R} \cdot \state{P}
  \and
  (\fprmatrix{P}{x}{Q})(\event{R}) := x \cdot \innerprod{R}{P} \cdot \event{Q}
\end{mathpar}

\paragraph{Discussion}
As promised: vectors (aka states) are represented as processes; duals
as contextual duals; inner product definition should be compared with
standard inner product definition for ....

\begin{remark}
  Assuming $\int (P^{\underline{\perp}}[P]) = 0$, the reader is
  invited to verify that $(\fprmatrix{P}{x}{P})(\state{P}) = x \cdot \state{P}$.
\end{remark}

\begin{remark}
  The reader is invited to verify that $\innerprod{P}{Q}$ could
  equally well have been written $\quotep{\int \stackrel{\vee}{x}}$
  where $x = \event{P^{\underline{\perp}}}(Q)$.

  One of the motivations for this remark is that there is another way
  to factor these operations. We could package up evaluation in the dual:

  \begin{mathpar}
    \state{P}^{*} := \event{\int P^{\underline{\perp}}} := \quotep{\int P^{\underline{\perp}}}[-]
  \end{mathpar}

  and then have inner product defined by
  
  \begin{mathpar}
    \innerprod{P}{Q} := \event{P}(Q)
  \end{mathpar}

  Hopefully, experience with the calculations will provide guidance on
  the best factoring.
\end{remark}

\begin{remark}
  Assuming $\int (P^{\underline{\perp}}[P]) = 0$, the reader is
  invited to verify that $\forall P,Q. (\prmatrix{0}{Q})(\state{0}) =
  \state{0}$ and dually $(\prmatrix{P}{0})(\event{0}) = \event{0}$.
\end{remark}

\begin{remark}
  i'm a little worried that i don't (yet) have proper support for
  complex conjugacy. But, the observation above may give us a
  clue. According to Abramsky, it must be the case that the scalars
  are iso to the homset of the identity for the tensor -- which the
  observation above characterizes. 

  For now, we will simply bookmark the notion with $\overline{x}$.
\end{remark}

\subsubsection{Adjointness}

We need to give a definition of $(\cdot)^{\dagger}$ for matrices. The
obvious candidate definition is
\begin{mathpar}
(\Sigma_{\alpha}\fprmatrix{P_{\alpha}}{x_{\alpha}}{Q_{\alpha}})^{\dagger}
= \Sigma_{\alpha}\fprmatrix{(Q_{\alpha}^{\underline{\perp}})^{*}}{\overline{x}_{\alpha}}{P_{\alpha}^{\underline{\perp}}} 
\end{mathpar}

But, $(Q_{\alpha}^{\underline{\perp}})^{*}$ requires a name along
which to communicate the process to achieve the context application.

\subsubsection{Basis for a basis}
If processes label states and ``addition'' of states (a.k.a. vector
addition) is interpreted as parallel composition, what corresponds to
notions of linear independence and basis? Here, we recall that Yoshida
has developed a set of \emph{combinators} for an asynchronous verison
of Milner's $\pi$-calculus. These are a finite set of processes such
any process can be expressed as parallel composition of these
combinators together with liberal uses of the new operator and
replication. We can simply give a translation of these into the
present calculus and have reasonable expectation that the property
carries over. That is, that the resultant set allows to express all
processes via parallel composition. Note, however, that there is no
new operator or replication in this calculus. As a result, we expect
that the corresponding set is actually infinite. That is, we expect
that the space is actually infinite dimensional.

\begin{remark}
  The attentive reader may be a bit concerned. Certainly, the
  collection $S$, $K$ and $I$ is a finite set of
  combinators. Shouldn't we expect to see a finite set of combinators
  for an effectively equivalent system? i am very sympathetic to this
  critique and feel it warrants full attention. On the other hand, i
  also have in mind the following analogy. The natural numbers, as a
  monoid under addition, has exactly $1$ generator, while the natural
  numbers, as a monoid under multiplication, has countably many
  generators (the primes). We observe that the application of the
  lambda calculus is much less resource sensitive than the parallel
  composition of the $\pi$-calculus. Could it be the case that we have
  an analogy of the form
  
  \begin{mathpar}
    m + n : MN :: m*n : M|N
  \end{mathpar}

  giving a similar blow up in the set of ``primes''?  This is such a
  wonderful thought that, even if it's not true, i think it's worth
  writing down.
\end{remark}
 

\documentclass[12pt]{llncs}
%\documentclass{jktr}

\usepackage[pdftex]{hyperref}                   
\usepackage {listings}
\usepackage {mathpartir}
\usepackage{bcprules}
%\usepackage{listings}
                       
\usepackage{graphicx} 
%\usepackage[margins=2.5cm,nohead,nofoot]{geometry}
%\usepackage{geometry}
\usepackage{amsfonts}
\usepackage{amstext}
\usepackage{latexsym}
\usepackage{amssymb}
\usepackage{color}


%\include{myPreamble}
\include{qm2pi.local} 

%\ifpdf
%\usepackage[pdftex]{graphicx}
%\else
%\usepackage{graphicx}
%\fi

 % \ifpdf
%  \usepackage{pdfsync}
%  \if


%\title{Brief Article}
%\author{David F. Snyder}
%\author{L.G. Meredith}

%\address{Dept. of Math., Texas State University--San Marcos, San Marcos, TX 78666}
       
\pagestyle{empty}


\begin{document}

\lstset{language=[Objective]Caml,frame=shadowbox}

\input{qm2pi.front}

% section front matter (end)

\input{qm2pi.intro} 
 
% section introduction (end)

% \input{qm2pi.knotations} 

% section notation (end)

\input{qm2pi.process.calculi} 

% section concurrent_process_calculi_and_spatial_logics_ (end)
    
%\input{qm2pi.knots2pi} 

%\input{qm2pi.trefoil} 

%\input{qm2pi.mainthm} 

% subsection basic_interpretation (end)

%\input{qm2pi.rho.presentation} 
\subsection{The syntax and semantics of the notation system}\label{sub:the_syntax_and_semantics_of_the_notation_system} % (fold)

We now summarize a technical presentation of the calculus that
embodies our theory of dynamics. The typical presentation of such a
calculus follows the style of giving generators and relations on
them. The grammar, below, describing term constructors, freely
generates the set of processes, $\Proc$. This set is then quotiented
by a relation known as structural congruence and it is over this set
that the notion of dynamics is expressed. This presentation is
essentially that of \cite{MeredithR05} with the addition of
polyadicity and summation. For readability we have relegated some of
the technical subtleties to an appendix.

\subsubsection{Process grammar}\label{subsub:process_grammar}

\begin{mathpar}
  \inferrule* [lab=synchronization] {} {{M} \bc \pzero \;|\; x?F \;|\; x!C }
  \and
  \inferrule* [lab=abstraction] {} {{F} \bc (x)P}
  \and
  \inferrule* [lab=concretion] {} {{C} \bc \langle Q \rangle}
  \and
  \inferrule* [lab=process] {} {{P,Q} \bc M \;| \;P|Q \;|\; @{x}}
  \and
  \inferrule* [lab=name] {} {{x} \bc \quotep{P}}
\end{mathpar} 

Note that $\vec{x}$ (resp. $\vec{P}$) denotes a vector of names
(resp. processes) of length $|\vec{x}|$ (resp. $|\vec{P}|$). We adopt
the following useful abbreviations.

\begin{mathpar}
   x?(\vec{y}).P := x.(\vec{y})P \and  x\clift{\vec{P}} := x.\clift{\vec{P}}
   \and x!(y) := \lift{x}{\dropn{y}}
   \and \Pi_{i=0}^{n-1}P_i := P_0 | \ldots | P_{n-1}
\end{mathpar}

\subsubsection{Structural congruence}

\paragraph{Free and bound names and alpha-equivalence.} At the
core of structural equivalence is alpha-equivalence which identifies
process that are the same up to a change of variable. Formally, we
recognize the distinction between free and bound names. The free names
of a process, $\freenames{P}$, may be calculated recursively as
follows:

\begin{mathpar}
\freenames{\pzero} := \emptyset
  \and \\
  \freenames{x?(y).P} := \{ x \} \cup (\freenames{P} \setminus \{ y \})
  \and 
  \freenames{x!\langle P \rangle} := \{ x \} \cup \{ P \} 
  \and \\
  \freenames{P|Q} := \freenames{P} \cup \freenames{Q}
  \and \\
  \freenames{@{x}} := \{ x \}
\end{mathpar}

$\pi$
$\quotep{\pi}$

$\freenames{-} : \pi \to \mathcal{P}(\quotep{\pi})$

\begin{eqnarray*}
  \freenames{\pzero} & := & \emptyset \\
  \freenames{x?(y).P} & := & \{ x \} \cup (\freenames{P} \setminus \{ y \}) \\
  \freenames{x!\langle P \rangle} & := & \{ x \} \cup \{ P \} \\
  \freenames{P|Q} & := & \freenames{P} \cup \freenames{Q} \\
  \freenames{\dropn{x}} & := & \{ x \}
\end{eqnarray*}

The bound names of a process, $\boundnames{P}$, are those names occurring in $P$
that are not free. For example, in $x?(y).0$, the name $x$ is free, while $y$ is bound.

\begin{mathpar}
  \inferrule* [lab=monoidal-laws] {} { P|Q \equiv Q|P \and P|0 \equiv P \and P|(Q|R) \equiv (P|Q)|R }
\end{mathpar}

\begin{mathpar}
  \inferrule* [lab=alpha-equivalence] {} { (x)P \equiv (y)P\{y/x\} \and y \not\in \freenames{P} }
\end{mathpar}

\begin{definition}
Then two processes, $P,Q$, are alpha-equivalent if $P = Q\{\vec{y}/\vec{x}\}$ for
some $\vec{x} \in \boundnames{Q},\vec{y} \in \boundnames{P}$, where $Q\{\vec{y}/\vec{x}\}$
denotes the capture-avoiding substitution of $\vec{y}$ for $\vec{x}$ in $Q$.
\end{definition}

\begin{definition}
  The {\em structural congruence} \cite{SangiorgiWalker} , $\equiv$,
  between processes is the least congruence containing
  alpha-equivalence, satisfying the abelian monoid laws
  (associativity, commutativity and $\pzero$ as identity) for parallel
  composition $|$ and for summation $+$.
\end{definition}

\subsection{Name equivalence}

We take name equivalence, written $\nameeq$, to be the smallest
equivalence relation generated by the following rules.

\begin{mathpar}
\inferrule*[lab=Quote-drop]
{ }
{ \quotep{@{x}} \nameeq x }

\inferrule*[lab=Struct-equiv]
{ P \scong Q }
{ \quotep{P} \nameeq \quotep{Q} }
\end{mathpar}

The astute reader will have noticed that the mutual recursion of names
and processes imposes a mutual recursion on alpha-equivalence and
structural equivalence via name-equivalence. Fortunately, all of this
works out pleasantly and we may calculate in the natural way, free of
concern. The reader interested in the details is referred to the
appendix \ref{appendix:rho_details}.

\subsection{Substitution}

We use $\Proc$ for the set of processes, $\QProc$ for the set of
names, and $\id{\{}\vec{y} / \vec{x} \id{\}}$ to denote partial maps,
$s : \QProc \rightarrow \QProc$. A map, $s$ lifts, uniquely, to a map
on process terms, $\widehat{s} : \Proc \rightarrow \Proc$ by the
following equations.

\begin{mathpar}
  (0) \psubstp{Q}{P} := 0 \\
  (R \juxtap S) \psubstp{Q}{P}
  :=    
  (R)\psubstp{Q}{P} \juxtap (S) \psubstp{Q}{P} \\
  (x?(y).R) \psubstp{Q}{P}    
  :=    
  (x)\substp{Q}{P} (z)\concat( (R \psubstn{z}{y}) \psubstp{Q}{P} ) \\
  (\lift{x}{R}) \psubstp{Q}{P}  
  :=
  \lift{(x)\substp{Q}{P}}{ R \psubstp{Q}{P} } \\
%   (\dropn{x})  \psubstp{Q}{P}       
%   := 
%   \left\{ 
%     \begin{array}{ccc} 
%       \dropn{\quotep{Q}} & & x \nameeq \quotep{P} \\
%       \dropn{x} & & otherwise \\
%     \end{array}
%   \right. 
  (\dropn{x})  \psubstp{Q}{P}       
  := 
  \left\{ 
    \begin{array}{ccc} 
      Q & & x \nameeq \quotep{P} \\
      \dropn{x} & & otherwise \\
    \end{array}
  \right.
\end{mathpar}
 

where

\begin{eqnarray}
  (x)\id{\{} \lpquote Q \rpquote / \lpquote P \rpquote \id{\}}            = 
  \left\{ 
    \begin{array}{ccc}
      \lpquote Q \rpquote & & x \nameeq \lpquote P \rpquote \\
      x & & otherwise \\
    \end{array}
  \right. \nonumber
\end{eqnarray}

and $z$ is chosen distinct from $\quotep{P}$, $\quotep{Q}$, the free
names in $Q$, and all the names in $R$. Our $\alpha$-equivalence will
be built in the standard way from this substitution.

\begin{remark}\label{rem:no_self_referential_names}
  One consequence of these definitions is that $\forall P. \quotep{P}
  \not\in \freenames{P}$.
\end{remark}

\subsection{ Dynamic quote: an example }

Anticipating something of what's to come, consider applying the
substitution, $\widehat{\id{\{}u / z \id{\}}}$, to the following pair
of processes, $\lift{w}{y!(z)}$ and $w[ \lpquote y!(z) \rpquote ]$.

\begin{eqnarray}
	\lift{w}{y!(z)}\widehat{\id{\{}u / z \id{\}}}
		& = &
		\lift{w}{y!(u)} \nonumber\\
	w[ \lpquote y!(z) \rpquote ] \widehat{ \id{\{}u / z \id{\}} }
		& = &
		w[ \lpquote y!(z) \rpquote ] \nonumber
\end{eqnarray}

Because the body of the process between quotes is impervious to
substitution, we get radically different answers. In fact, by
examining the first process in an input context,
e.g. $x?(z).\lift{w}{y!(z)}$, we see that the process under the lift
operator may be shaped by prefixed inputs binding a name inside it. In
this sense, the lift operator will be seen as a way to dynamically
construct processes before reifying them as names.

Finally equipped with these standard features we can present the
dynamics of the calculus.

\subsubsection{Operational semantics} 

Finally, we introduce the computational dynamics. What marks these
algebras as distinct from other more traditionally studied algebraic
structures, e.g. vector spaces or polynomial rings, is the manner in
which dynamics is captured. In traditional structures, dynamics is typically
expressed through morphisms between such structures, as in linear maps
between vector spaces or morphisms between rings. In algebras
associated with the semantics of computation, the dynamics is
expressed as part of the algebraic structure itself, through a
reduction reduction relation typically denoted by $\red$. Below, we
give a recursive presentation of this relation for the calculus used
in the encoding.

$\red \subseteq \pi \times \pi$
$\red : \pi \to \mathcal{P}(\pi)$

\begin{mathpar}
  \inferrule* [lab=Comm] { \textsf{match}( x_{src}, x_{trgt} ) } { x_{trgt}?(y)P \; | \; x_{src}!\langle {Q} \rangle \red P\{\quotep{Q}/y}\} }
  \and \\
  \inferrule* [lab=Par] {{P} \red {P}'} {{{P} | {Q}} \red {{P}' | {Q}}}
  \and
  \inferrule* [lab=Equiv]{{{P} \scong {P}'} \andalso {{P}' \red {Q}'} \andalso {{Q}' \scong {Q}}}{{P} \red {Q}}
\end{mathpar}

\begin{eqnarray*}
  match_{\equiv} (\quotep{P},\quotep{Q}) & := & P \equiv Q \\
  match_{\dagger}(\quotep{P},\quotep{Q}) & := & \forall R. P|Q \red^{*} R => R \red^{*} 0 \\
  match_{K}(\quotep{P},\quotep{Q}) & := & K \mbox{ for some context } K
\end{eqnarray*}

$u?(x)P | u!\langle Q \rangle \red P\{\quotep{Q}/x\}$

%We write $\wred$ for $\red^*$, and $P\red$ if $\exists Q $ such that $ P \red Q$.
We write $P\red$ if $\exists Q $ such that $ P \red Q$ and $P\not\red$, otherwise.

\section{Replication}

As mentioned before, it is known that replication (and hence
recursion) can be implemented in a higher-order process algebra
\cite{SangiorgiWalker}. As our first example of calculation with the
machinery thus far presented we give the construction explicitly in
the {\rhoc}.

\begin{eqnarray}
	D_{x} & := & \prefix{x}{y}{(\binpar{\outputp{x}{y}}{@{y}})} \nonumber\\
	\bangp_{x}{P} & := & \binpar{{x}!\langle{\binpar{D_{x}}{P}}\rangle}{D_{x}} \nonumber
\end{eqnarray}

\begin{eqnarray}
	\bangp_{x}{P} & & \nonumber\\
	=
	& {x}!\langle{(\prefix{x}{y}{(\outputp{x}{y} | @{y})) | P}}\rangle 
	      | \prefix{x}{y}{(\outputp{x}{y} | @{y})} & \nonumber\\
	\red
	& (\outputp{x}{y} | @{y})\substn{\quotep{(\prefix{x}{y}{(@{y} | \outputp{x}{y})) | P}}}{y} & \nonumber\\
	=
	& \outputp{x}{\quotep{(\prefix{x}{y}{(\outputp{x}{y} | @{y})) | P}}}
	  | {(\prefix{x}{y}{(\outputp{x}{y} | @{y})) | P}} & \nonumber\\
	\red
	& \ldots & \nonumber\\
	\red^*
	& P | P | \ldots & \nonumber
\end{eqnarray}

Of course, this encoding, as an implementation, runs away, unfolding
$\bangp{P}$ eagerly. A lazier and more implementable replication
operator, restricted to input-guarded processes, may be obtained as follows.

\begin{eqnarray}
\bangp{\prefix{u}{v}{P}} 
	:= 
	\binpar{\lift{x}{\prefix{u}{v}{(\binpar{D(x)}{P})}}}{D(x)} \nonumber
\end{eqnarray}

\begin{remark}
  Note that the lazier definition still does not deal with summation
  or mixed summation (i.e. sums over input and output). The reader is
  invited to construct definitions of replication that deal with these
  features. 

  Further, the definitions are parameterized in a name, $x$. Can you,
  gentle reader, make a definition that eliminates this parameter and
  guarantees no accidental interaction between the replication
  machinery and the process being replicated -- i.e. no accidental
  sharing of names used by the process to get its work done and the
  name(s) used by the replication to effect copying. This latter
  revision of the definition of replication is crucial to obtaining
  the expected identity $!!P \sim !P$.
\end{remark}

\begin{remark}\label{rem:paradoxical_combinator}
  The reader familiar with the lambda calculus will have noticed the
  similarity between $D$ and the paradoxical combinator.

  [Ed. note: the existence of this seems to suggest we have to be more
  restrictive on the set of processes and names we admit if we are to
  support no-cloning.]
\end{remark}

\subsubsection{Bisimulation}

The computational dynamics gives rise to another kind of equivalence,
the equivalence of computational behavior. As previously mentioned
this is typically captured \emph{via} some form of bisimulation.

% The notion we use in this paper is weak barbed bisimulation
% \cite{milner91polyadicpi}.

The notion we use in this paper is derived from weak barbed
bisimulation \cite{milner91polyadicpi}. 

\begin{definition}
An \emph{observation relation}, $\downarrow_{\mathcal N}$, over a set
of names, $\mathcal N$, is the smallest relation satisfying the rules
below.

\infrule[Out-barb]{y \in {\mathcal N}, \; x \nameeq y}
		  {\outputp{x}{v} \downarrow_{\mathcal N} x}
\infrule[Par-barb]{\mbox{$P\downarrow_{\mathcal N} x$ or $Q\downarrow_{\mathcal N} x$}}
		  {\binpar{P}{Q} \downarrow_{\mathcal N} x}

We write $P \Downarrow_{\mathcal N} x$ if there is $Q$ such that 
$P \wred Q$ and $Q \downarrow_{\mathcal N} x$.
\end{definition}

\begin{definition}
%\label{def.bbisim}
An  ${\mathcal N}$-\emph{barbed bisimulation} over a set of names, ${\mathcal N}$, is a symmetric binary relation 
${\mathcal S}_{\mathcal N}$ between agents such that $P\rel{S}_{\mathcal N}Q$ implies:
\begin{enumerate}
\item If $P \red P'$ then $Q \wred Q'$ and $P'\rel{S}_{\mathcal N} Q'$.
\item If $P\downarrow_{\mathcal N} x$, then $Q\Downarrow_{\mathcal N} x$.
\end{enumerate}
$P$ is ${\mathcal N}$-barbed bisimilar to $Q$, written
$P \wbbisim_{\mathcal N} Q$, if $P \rel{S}_{\mathcal N} Q$ for some ${\mathcal N}$-barbed bisimulation ${\mathcal S}_{\mathcal N}$.
\end{definition}

$\mathcal{R} \subseteq \pi \times \pi$

$P \mathcal{R} Q => \forall P'. P \red P' \Rightarrow \exists Q'. Q \red Q', P' \mathcal{R} Q'$

$P \vdash x \Rightarrow Q \vdash x$

\begin{mathpar}
  \inferrule*[lab=Out-barb]{x \nameeq y}{{y}!\langle{Q}\rangle \vdash x}
  \and
  \inferrule*[lab=Par-barb]{\mbox{$P\vdash x$ or $Q\vdash x$}}{\binpar{P}{Q} \vdash x}
\end{mathpar}

\subsubsection{Contexts}

One of the principle advantages of computational calculi like the
$\pi$-calculus is a well-defined notion of context,
contextual-equivalence and a correlation between
contextual-equivalence and notions of bisimulation. The notion of
context allows the decomposition of a process into (sub-)process and
its syntactic environment, its context. Thus, a context may be
thought of as a process with a ``hole'' (written $\Box$) in it. The
application of a context $M$ to a process $P$, written $M[P]$, is
tantamount to filling the hole in $M$ with $P$. In this paper we do
not need the full weight of this theory, but do make use of the notion
of context in the proof the main theorem. 

\begin{mathpar}
  \inferrule* [lab=summation] {} {{M_{M},M_{N}} \bc \Box \;|\; x.M_{A} \;|\; M_{M}+M_{N}}
  \and
  \inferrule* [lab=agent] {} {{M_{A}} \bc (\vec{x})M_{P} \;| \; \clift{P_0,\ldots,M_{P},\ldots,P_N}}
  \and \\
  \inferrule* [lab=process] {} {{M_{P}} \bc M_{N} \;| \;P|M_{P} }
\end{mathpar} 

\begin{mathpar}
  \inferrule* [lab=sychronization] {} {M_{N} \bc \Box \;|\; x?M_{F} \;|\; x!M_{C}}
  \and
  \inferrule* [lab=abstraction] {} {{M_{F}} \bc (x)M_{P} }
  \and
  \inferrule* [lab=concretion] {} {{M_{C}} \bc \langle M_{P} \rangle }
  \and \\
  \inferrule* [lab=process] {} {{M_{P}} \bc M_{N} \;| \;P|M_{P} }
\end{mathpar}

\begin{definition}[contextual application] Given a context $M$, and
  process $P$, we define the \emph{contextual application}, $M[P] :=
  M\{P/\Box\}$. That is, the contextual application of M to P is the
  substitution of $P$ for $\Box$ in $M$.
\end{definition}

$\meaningof{-} : L \to \mathcal{P}(\pi)$

\begin{mathpar}
  \inferrule* [lab=collection] {} {\meaningof{true} = \pi, \and \meaningof{~E} = \pi \setminus \meaningof{E}, \and \meaningof{E_{1} \& E_{2}} = \meaningof{E_{1}} \cap \meaningof{E_{2}}}
\end{mathpar}

\begin{mathpar}
  \inferrule* [lab=structure] {} {\meaningof{0} = \{ P \in \pi | P \equiv 0 \}, \and \\ \meaningof{E_1 | E_2} = \{ P \in \pi | P \equiv P_{1} | P_{2}, P_{1} \in \meaningof{E_{1}}, P_{2} \in \meaningof{E_2}\} }
\end{mathpar}

\begin{mathpar}
 \inferrule* [lab=behavior] {} {\meaningof{\langle a?b \rangle E} = \{ P \in \pi | P \equiv Q | u?(y)P', \\ \and \\\\ \and \\ \;\;\; u \in \meaningof{a}, \forall z.P'\{z/y\} \in \meaningof{E\{z/b\}}\}, \and \\ \meaningof{a!E} = \{ P \in \pi | P \equiv Q | x!\langle P' \rangle, x \in \meaningof{a} P' \in \meaningof{E}\} }
\end{mathpar}

\begin{mathpar}
 \inferrule* [lab=nominal] {} {\meaningof{\quotep{E}} = \{ \quotep{P} \in \quotep{\pi} | P \in \meaningof{E} \}, \and \meaningof{\quotep{P}} = \{ \quotep{Q} \in \quotep{\pi} | P \equiv Q \} \and \\ \meaningof{@\quotep{E}} = \{ P \in \pi | P \equiv @x, x \in \meaningof{E} \}}
\end{mathpar}

\begin{eqnarray*}
  \\
  \meaningof{-} : TS \to ST
\end{eqnarray*}

\begin{eqnarray*}
  \\
  L : TS \to ST
\end{eqnarray*}

\begin{eqnarray*}
  \\
  P \models E \iff P \in \meaningof{E}
\end{eqnarray*}

\begin{eqnarray*}
  P \approx_{L} Q \iff \forall E \in L. P \models E \iff Q \models E
\end{eqnarray*}

\begin{eqnarray*}
  P \approx_{K} Q
\end{eqnarray*}

\begin{eqnarray*}
  P \approx Q
\end{eqnarray*}

$\approx_{K} = \approx = \approx_{L}$

\subsubsection{Contextual duality}

Note that contexts extend the quotation operation to a family of
operations from processes to names. Given a context, $M$, we can
define a \emph{nominal context}, $\quotep{M}$ by $\quotep{M}[P] :=
\quotep{M[P]}$. To foreshadow what is to come we observe that these
operations enjoy a duality with processes very much like the duality
between vectors and maps from vectors to scalars.

Further, because the calculus is essentially higher-order, we have a
correspondence between contexts and processes. More specifically,
given a name $x$ and a context $M$ we can construct $M^{*}_{x}$ such
that 

\begin{mathpar}
  M^{*}_{x} | \lift{x}{P} \red M[P]
\end{mathpar}

namely,

\begin{mathpar}
  M^{*}_{x} := x?(u).M[\dropn{u}]
\end{mathpar}

The dependence of $M^{*}_{x}$ on a name makes it an abstraction, 

\begin{mathpar}
  M^{*} := (x)x?(u).M[\dropn{u}]
\end{mathpar}

\subsection{Additional notation}

It will sometimes be convenient to denote the process a name
quotes. We already have the notation $x = \quotep{P}$, but it will be
convenient to introduce an alternate notation, $\procn{x}$, when we
want to emphasize the connection to the use of the name. Note that, by
virtue of name equivalence, $\quotep{\procn{x}} \nameeq x$; so, the
notation is consistent with previous definitions.

Further, because names have structure it is possible to effect
substitutions on the basis of that structure. This means we need to
upgrade our notation for substitutions, which we accomplish by
adapting comprehension notation. Thus,

\begin{mathpar}
  P\{ y / x : x \in S \}
\end{mathpar}

is interpreted to mean the process derived from P by replacing (in a
capture-avoiding manner) each occurrence of $x$ in $S$ by $y$. For example,

\begin{mathpar}
  P\{ \quotep{\procn{x}|\procn{x}} / x : x \in \freenames{P} \}
\end{mathpar}

will replace each (occurrence) of a free name $x$ in $P$ by
$\quotep{\procn{x}|\procn{x}}$.

Also, we will avail ourselves of the notation $x^{L}$ and $x^{R}$ to
denote injections of a name into disjoint copies of the name
space. There are numerous ways to accomplish this. One example can be
found in \cite{MeredithR05}. This notation overloads to vectors of
names: $\vec{x}^{\pi} := (x_{i}^{\pi} \; : \; 0 \leq i < |\vec{x}| )$ where $\pi \in \{L,R\}$.

We also use $P^{\Box} := P|\Box$.

In \cite{MeredithR05} an interpretation of the new operator is
given. It turns out that there are several possible interpretations
all enjoying the requisite algebraic properties of the operator (see
\cite{milner91polyadicpi}). We will therefore make liberal use of
$(\nu\; \vec{x})P$.

% subsection the_syntax_and_semantics_of_the_notation_system (end)   

\input{qm2pi.qmops} 

\input{qm2pi.sterngerlach} 

\input{qm2pi.metric} 

% section concurrent_process_calculi (end)

%\input{qm2pi.proofsketch}

% section proof sketch (end)

%\input{qm2pi.slviaknots} 

% section spatial logic via knots (end)

\input{qm2pi.conclusion}

% section conclusion (end)

%\input{qm2pi.dtcodes} 

% section wiring algorithm (end)

\input{qm2pi.ack} 

% section acknowledgments (end)

\newpage


\bibliographystyle{plain}   
\bibliography{../../biblios/main.bib}

\input{qm2pi.rhodetails}

\end{document}

 

\documentclass[12pt]{llncs}
%\documentclass{jktr}

\usepackage[pdftex]{hyperref}                   
\usepackage {listings}
\usepackage {mathpartir}
\usepackage{bcprules}
%\usepackage{listings}
                       
\usepackage{graphicx} 
%\usepackage[margins=2.5cm,nohead,nofoot]{geometry}
%\usepackage{geometry}
\usepackage{amsfonts}
\usepackage{amstext}
\usepackage{latexsym}
\usepackage{amssymb}
\usepackage{color}


%\include{myPreamble}
\include{qm2pi.local} 

%\ifpdf
%\usepackage[pdftex]{graphicx}
%\else
%\usepackage{graphicx}
%\fi

 % \ifpdf
%  \usepackage{pdfsync}
%  \if


%\title{Brief Article}
%\author{David F. Snyder}
%\author{L.G. Meredith}

%\address{Dept. of Math., Texas State University--San Marcos, San Marcos, TX 78666}
       
\pagestyle{empty}


\begin{document}

\lstset{language=[Objective]Caml,frame=shadowbox}

\input{qm2pi.front}

% section front matter (end)

\input{qm2pi.intro} 
 
% section introduction (end)

% \input{qm2pi.knotations} 

% section notation (end)

\input{qm2pi.process.calculi} 

% section concurrent_process_calculi_and_spatial_logics_ (end)
    
%\input{qm2pi.knots2pi} 

%\input{qm2pi.trefoil} 

%\input{qm2pi.mainthm} 

% subsection basic_interpretation (end)

%\input{qm2pi.rho.presentation} 
\subsection{The syntax and semantics of the notation system}\label{sub:the_syntax_and_semantics_of_the_notation_system} % (fold)

We now summarize a technical presentation of the calculus that
embodies our theory of dynamics. The typical presentation of such a
calculus follows the style of giving generators and relations on
them. The grammar, below, describing term constructors, freely
generates the set of processes, $\Proc$. This set is then quotiented
by a relation known as structural congruence and it is over this set
that the notion of dynamics is expressed. This presentation is
essentially that of \cite{MeredithR05} with the addition of
polyadicity and summation. For readability we have relegated some of
the technical subtleties to an appendix.

\subsubsection{Process grammar}\label{subsub:process_grammar}

\begin{mathpar}
  \inferrule* [lab=synchronization] {} {{M} \bc \pzero \;|\; x?F \;|\; x!C }
  \and
  \inferrule* [lab=abstraction] {} {{F} \bc (x)P}
  \and
  \inferrule* [lab=concretion] {} {{C} \bc \langle Q \rangle}
  \and
  \inferrule* [lab=process] {} {{P,Q} \bc M \;| \;P|Q \;|\; @{x}}
  \and
  \inferrule* [lab=name] {} {{x} \bc \quotep{P}}
\end{mathpar} 

Note that $\vec{x}$ (resp. $\vec{P}$) denotes a vector of names
(resp. processes) of length $|\vec{x}|$ (resp. $|\vec{P}|$). We adopt
the following useful abbreviations.

\begin{mathpar}
   x?(\vec{y}).P := x.(\vec{y})P \and  x\clift{\vec{P}} := x.\clift{\vec{P}}
   \and x!(y) := \lift{x}{\dropn{y}}
   \and \Pi_{i=0}^{n-1}P_i := P_0 | \ldots | P_{n-1}
\end{mathpar}

\subsubsection{Structural congruence}

\paragraph{Free and bound names and alpha-equivalence.} At the
core of structural equivalence is alpha-equivalence which identifies
process that are the same up to a change of variable. Formally, we
recognize the distinction between free and bound names. The free names
of a process, $\freenames{P}$, may be calculated recursively as
follows:

\begin{mathpar}
\freenames{\pzero} := \emptyset
  \and \\
  \freenames{x?(y).P} := \{ x \} \cup (\freenames{P} \setminus \{ y \})
  \and 
  \freenames{x!\langle P \rangle} := \{ x \} \cup \{ P \} 
  \and \\
  \freenames{P|Q} := \freenames{P} \cup \freenames{Q}
  \and \\
  \freenames{@{x}} := \{ x \}
\end{mathpar}

$\pi$
$\quotep{\pi}$

$\freenames{-} : \pi \to \mathcal{P}(\quotep{\pi})$

\begin{eqnarray*}
  \freenames{\pzero} & := & \emptyset \\
  \freenames{x?(y).P} & := & \{ x \} \cup (\freenames{P} \setminus \{ y \}) \\
  \freenames{x!\langle P \rangle} & := & \{ x \} \cup \{ P \} \\
  \freenames{P|Q} & := & \freenames{P} \cup \freenames{Q} \\
  \freenames{\dropn{x}} & := & \{ x \}
\end{eqnarray*}

The bound names of a process, $\boundnames{P}$, are those names occurring in $P$
that are not free. For example, in $x?(y).0$, the name $x$ is free, while $y$ is bound.

\begin{mathpar}
  \inferrule* [lab=monoidal-laws] {} { P|Q \equiv Q|P \and P|0 \equiv P \and P|(Q|R) \equiv (P|Q)|R }
\end{mathpar}

\begin{mathpar}
  \inferrule* [lab=alpha-equivalence] {} { (x)P \equiv (y)P\{y/x\} \and y \not\in \freenames{P} }
\end{mathpar}

\begin{definition}
Then two processes, $P,Q$, are alpha-equivalent if $P = Q\{\vec{y}/\vec{x}\}$ for
some $\vec{x} \in \boundnames{Q},\vec{y} \in \boundnames{P}$, where $Q\{\vec{y}/\vec{x}\}$
denotes the capture-avoiding substitution of $\vec{y}$ for $\vec{x}$ in $Q$.
\end{definition}

\begin{definition}
  The {\em structural congruence} \cite{SangiorgiWalker} , $\equiv$,
  between processes is the least congruence containing
  alpha-equivalence, satisfying the abelian monoid laws
  (associativity, commutativity and $\pzero$ as identity) for parallel
  composition $|$ and for summation $+$.
\end{definition}

\subsection{Name equivalence}

We take name equivalence, written $\nameeq$, to be the smallest
equivalence relation generated by the following rules.

\begin{mathpar}
\inferrule*[lab=Quote-drop]
{ }
{ \quotep{@{x}} \nameeq x }

\inferrule*[lab=Struct-equiv]
{ P \scong Q }
{ \quotep{P} \nameeq \quotep{Q} }
\end{mathpar}

The astute reader will have noticed that the mutual recursion of names
and processes imposes a mutual recursion on alpha-equivalence and
structural equivalence via name-equivalence. Fortunately, all of this
works out pleasantly and we may calculate in the natural way, free of
concern. The reader interested in the details is referred to the
appendix \ref{appendix:rho_details}.

\subsection{Substitution}

We use $\Proc$ for the set of processes, $\QProc$ for the set of
names, and $\id{\{}\vec{y} / \vec{x} \id{\}}$ to denote partial maps,
$s : \QProc \rightarrow \QProc$. A map, $s$ lifts, uniquely, to a map
on process terms, $\widehat{s} : \Proc \rightarrow \Proc$ by the
following equations.

\begin{mathpar}
  (0) \psubstp{Q}{P} := 0 \\
  (R \juxtap S) \psubstp{Q}{P}
  :=    
  (R)\psubstp{Q}{P} \juxtap (S) \psubstp{Q}{P} \\
  (x?(y).R) \psubstp{Q}{P}    
  :=    
  (x)\substp{Q}{P} (z)\concat( (R \psubstn{z}{y}) \psubstp{Q}{P} ) \\
  (\lift{x}{R}) \psubstp{Q}{P}  
  :=
  \lift{(x)\substp{Q}{P}}{ R \psubstp{Q}{P} } \\
%   (\dropn{x})  \psubstp{Q}{P}       
%   := 
%   \left\{ 
%     \begin{array}{ccc} 
%       \dropn{\quotep{Q}} & & x \nameeq \quotep{P} \\
%       \dropn{x} & & otherwise \\
%     \end{array}
%   \right. 
  (\dropn{x})  \psubstp{Q}{P}       
  := 
  \left\{ 
    \begin{array}{ccc} 
      Q & & x \nameeq \quotep{P} \\
      \dropn{x} & & otherwise \\
    \end{array}
  \right.
\end{mathpar}
 

where

\begin{eqnarray}
  (x)\id{\{} \lpquote Q \rpquote / \lpquote P \rpquote \id{\}}            = 
  \left\{ 
    \begin{array}{ccc}
      \lpquote Q \rpquote & & x \nameeq \lpquote P \rpquote \\
      x & & otherwise \\
    \end{array}
  \right. \nonumber
\end{eqnarray}

and $z$ is chosen distinct from $\quotep{P}$, $\quotep{Q}$, the free
names in $Q$, and all the names in $R$. Our $\alpha$-equivalence will
be built in the standard way from this substitution.

\begin{remark}\label{rem:no_self_referential_names}
  One consequence of these definitions is that $\forall P. \quotep{P}
  \not\in \freenames{P}$.
\end{remark}

\subsection{ Dynamic quote: an example }

Anticipating something of what's to come, consider applying the
substitution, $\widehat{\id{\{}u / z \id{\}}}$, to the following pair
of processes, $\lift{w}{y!(z)}$ and $w[ \lpquote y!(z) \rpquote ]$.

\begin{eqnarray}
	\lift{w}{y!(z)}\widehat{\id{\{}u / z \id{\}}}
		& = &
		\lift{w}{y!(u)} \nonumber\\
	w[ \lpquote y!(z) \rpquote ] \widehat{ \id{\{}u / z \id{\}} }
		& = &
		w[ \lpquote y!(z) \rpquote ] \nonumber
\end{eqnarray}

Because the body of the process between quotes is impervious to
substitution, we get radically different answers. In fact, by
examining the first process in an input context,
e.g. $x?(z).\lift{w}{y!(z)}$, we see that the process under the lift
operator may be shaped by prefixed inputs binding a name inside it. In
this sense, the lift operator will be seen as a way to dynamically
construct processes before reifying them as names.

Finally equipped with these standard features we can present the
dynamics of the calculus.

\subsubsection{Operational semantics} 

Finally, we introduce the computational dynamics. What marks these
algebras as distinct from other more traditionally studied algebraic
structures, e.g. vector spaces or polynomial rings, is the manner in
which dynamics is captured. In traditional structures, dynamics is typically
expressed through morphisms between such structures, as in linear maps
between vector spaces or morphisms between rings. In algebras
associated with the semantics of computation, the dynamics is
expressed as part of the algebraic structure itself, through a
reduction reduction relation typically denoted by $\red$. Below, we
give a recursive presentation of this relation for the calculus used
in the encoding.

$\red \subseteq \pi \times \pi$
$\red : \pi \to \mathcal{P}(\pi)$

\begin{mathpar}
  \inferrule* [lab=Comm] { \textsf{match}( x_{src}, x_{trgt} ) } { x_{trgt}?(y)P \; | \; x_{src}!\langle {Q} \rangle \red P\{\quotep{Q}/y}\} }
  \and \\
  \inferrule* [lab=Par] {{P} \red {P}'} {{{P} | {Q}} \red {{P}' | {Q}}}
  \and
  \inferrule* [lab=Equiv]{{{P} \scong {P}'} \andalso {{P}' \red {Q}'} \andalso {{Q}' \scong {Q}}}{{P} \red {Q}}
\end{mathpar}

\begin{eqnarray*}
  match_{\equiv} (\quotep{P},\quotep{Q}) & := & P \equiv Q \\
  match_{\dagger}(\quotep{P},\quotep{Q}) & := & \forall R. P|Q \red^{*} R => R \red^{*} 0 \\
  match_{K}(\quotep{P},\quotep{Q}) & := & K \mbox{ for some context } K
\end{eqnarray*}

$u?(x)P | u!\langle Q \rangle \red P\{\quotep{Q}/x\}$

%We write $\wred$ for $\red^*$, and $P\red$ if $\exists Q $ such that $ P \red Q$.
We write $P\red$ if $\exists Q $ such that $ P \red Q$ and $P\not\red$, otherwise.

\section{Replication}

As mentioned before, it is known that replication (and hence
recursion) can be implemented in a higher-order process algebra
\cite{SangiorgiWalker}. As our first example of calculation with the
machinery thus far presented we give the construction explicitly in
the {\rhoc}.

\begin{eqnarray}
	D_{x} & := & \prefix{x}{y}{(\binpar{\outputp{x}{y}}{@{y}})} \nonumber\\
	\bangp_{x}{P} & := & \binpar{{x}!\langle{\binpar{D_{x}}{P}}\rangle}{D_{x}} \nonumber
\end{eqnarray}

\begin{eqnarray}
	\bangp_{x}{P} & & \nonumber\\
	=
	& {x}!\langle{(\prefix{x}{y}{(\outputp{x}{y} | @{y})) | P}}\rangle 
	      | \prefix{x}{y}{(\outputp{x}{y} | @{y})} & \nonumber\\
	\red
	& (\outputp{x}{y} | @{y})\substn{\quotep{(\prefix{x}{y}{(@{y} | \outputp{x}{y})) | P}}}{y} & \nonumber\\
	=
	& \outputp{x}{\quotep{(\prefix{x}{y}{(\outputp{x}{y} | @{y})) | P}}}
	  | {(\prefix{x}{y}{(\outputp{x}{y} | @{y})) | P}} & \nonumber\\
	\red
	& \ldots & \nonumber\\
	\red^*
	& P | P | \ldots & \nonumber
\end{eqnarray}

Of course, this encoding, as an implementation, runs away, unfolding
$\bangp{P}$ eagerly. A lazier and more implementable replication
operator, restricted to input-guarded processes, may be obtained as follows.

\begin{eqnarray}
\bangp{\prefix{u}{v}{P}} 
	:= 
	\binpar{\lift{x}{\prefix{u}{v}{(\binpar{D(x)}{P})}}}{D(x)} \nonumber
\end{eqnarray}

\begin{remark}
  Note that the lazier definition still does not deal with summation
  or mixed summation (i.e. sums over input and output). The reader is
  invited to construct definitions of replication that deal with these
  features. 

  Further, the definitions are parameterized in a name, $x$. Can you,
  gentle reader, make a definition that eliminates this parameter and
  guarantees no accidental interaction between the replication
  machinery and the process being replicated -- i.e. no accidental
  sharing of names used by the process to get its work done and the
  name(s) used by the replication to effect copying. This latter
  revision of the definition of replication is crucial to obtaining
  the expected identity $!!P \sim !P$.
\end{remark}

\begin{remark}\label{rem:paradoxical_combinator}
  The reader familiar with the lambda calculus will have noticed the
  similarity between $D$ and the paradoxical combinator.

  [Ed. note: the existence of this seems to suggest we have to be more
  restrictive on the set of processes and names we admit if we are to
  support no-cloning.]
\end{remark}

\subsubsection{Bisimulation}

The computational dynamics gives rise to another kind of equivalence,
the equivalence of computational behavior. As previously mentioned
this is typically captured \emph{via} some form of bisimulation.

% The notion we use in this paper is weak barbed bisimulation
% \cite{milner91polyadicpi}.

The notion we use in this paper is derived from weak barbed
bisimulation \cite{milner91polyadicpi}. 

\begin{definition}
An \emph{observation relation}, $\downarrow_{\mathcal N}$, over a set
of names, $\mathcal N$, is the smallest relation satisfying the rules
below.

\infrule[Out-barb]{y \in {\mathcal N}, \; x \nameeq y}
		  {\outputp{x}{v} \downarrow_{\mathcal N} x}
\infrule[Par-barb]{\mbox{$P\downarrow_{\mathcal N} x$ or $Q\downarrow_{\mathcal N} x$}}
		  {\binpar{P}{Q} \downarrow_{\mathcal N} x}

We write $P \Downarrow_{\mathcal N} x$ if there is $Q$ such that 
$P \wred Q$ and $Q \downarrow_{\mathcal N} x$.
\end{definition}

\begin{definition}
%\label{def.bbisim}
An  ${\mathcal N}$-\emph{barbed bisimulation} over a set of names, ${\mathcal N}$, is a symmetric binary relation 
${\mathcal S}_{\mathcal N}$ between agents such that $P\rel{S}_{\mathcal N}Q$ implies:
\begin{enumerate}
\item If $P \red P'$ then $Q \wred Q'$ and $P'\rel{S}_{\mathcal N} Q'$.
\item If $P\downarrow_{\mathcal N} x$, then $Q\Downarrow_{\mathcal N} x$.
\end{enumerate}
$P$ is ${\mathcal N}$-barbed bisimilar to $Q$, written
$P \wbbisim_{\mathcal N} Q$, if $P \rel{S}_{\mathcal N} Q$ for some ${\mathcal N}$-barbed bisimulation ${\mathcal S}_{\mathcal N}$.
\end{definition}

$\mathcal{R} \subseteq \pi \times \pi$

$P \mathcal{R} Q => \forall P'. P \red P' \Rightarrow \exists Q'. Q \red Q', P' \mathcal{R} Q'$

$P \vdash x \Rightarrow Q \vdash x$

\begin{mathpar}
  \inferrule*[lab=Out-barb]{x \nameeq y}{{y}!\langle{Q}\rangle \vdash x}
  \and
  \inferrule*[lab=Par-barb]{\mbox{$P\vdash x$ or $Q\vdash x$}}{\binpar{P}{Q} \vdash x}
\end{mathpar}

\subsubsection{Contexts}

One of the principle advantages of computational calculi like the
$\pi$-calculus is a well-defined notion of context,
contextual-equivalence and a correlation between
contextual-equivalence and notions of bisimulation. The notion of
context allows the decomposition of a process into (sub-)process and
its syntactic environment, its context. Thus, a context may be
thought of as a process with a ``hole'' (written $\Box$) in it. The
application of a context $M$ to a process $P$, written $M[P]$, is
tantamount to filling the hole in $M$ with $P$. In this paper we do
not need the full weight of this theory, but do make use of the notion
of context in the proof the main theorem. 

\begin{mathpar}
  \inferrule* [lab=summation] {} {{M_{M},M_{N}} \bc \Box \;|\; x.M_{A} \;|\; M_{M}+M_{N}}
  \and
  \inferrule* [lab=agent] {} {{M_{A}} \bc (\vec{x})M_{P} \;| \; \clift{P_0,\ldots,M_{P},\ldots,P_N}}
  \and \\
  \inferrule* [lab=process] {} {{M_{P}} \bc M_{N} \;| \;P|M_{P} }
\end{mathpar} 

\begin{mathpar}
  \inferrule* [lab=sychronization] {} {M_{N} \bc \Box \;|\; x?M_{F} \;|\; x!M_{C}}
  \and
  \inferrule* [lab=abstraction] {} {{M_{F}} \bc (x)M_{P} }
  \and
  \inferrule* [lab=concretion] {} {{M_{C}} \bc \langle M_{P} \rangle }
  \and \\
  \inferrule* [lab=process] {} {{M_{P}} \bc M_{N} \;| \;P|M_{P} }
\end{mathpar}

\begin{definition}[contextual application] Given a context $M$, and
  process $P$, we define the \emph{contextual application}, $M[P] :=
  M\{P/\Box\}$. That is, the contextual application of M to P is the
  substitution of $P$ for $\Box$ in $M$.
\end{definition}

$\meaningof{-} : L \to \mathcal{P}(\pi)$

\begin{mathpar}
  \inferrule* [lab=collection] {} {\meaningof{true} = \pi, \and \meaningof{~E} = \pi \setminus \meaningof{E}, \and \meaningof{E_{1} \& E_{2}} = \meaningof{E_{1}} \cap \meaningof{E_{2}}}
\end{mathpar}

\begin{mathpar}
  \inferrule* [lab=structure] {} {\meaningof{0} = \{ P \in \pi | P \equiv 0 \}, \and \\ \meaningof{E_1 | E_2} = \{ P \in \pi | P \equiv P_{1} | P_{2}, P_{1} \in \meaningof{E_{1}}, P_{2} \in \meaningof{E_2}\} }
\end{mathpar}

\begin{mathpar}
 \inferrule* [lab=behavior] {} {\meaningof{\langle a?b \rangle E} = \{ P \in \pi | P \equiv Q | u?(y)P', \\ \and \\\\ \and \\ \;\;\; u \in \meaningof{a}, \forall z.P'\{z/y\} \in \meaningof{E\{z/b\}}\}, \and \\ \meaningof{a!E} = \{ P \in \pi | P \equiv Q | x!\langle P' \rangle, x \in \meaningof{a} P' \in \meaningof{E}\} }
\end{mathpar}

\begin{mathpar}
 \inferrule* [lab=nominal] {} {\meaningof{\quotep{E}} = \{ \quotep{P} \in \quotep{\pi} | P \in \meaningof{E} \}, \and \meaningof{\quotep{P}} = \{ \quotep{Q} \in \quotep{\pi} | P \equiv Q \} \and \\ \meaningof{@\quotep{E}} = \{ P \in \pi | P \equiv @x, x \in \meaningof{E} \}}
\end{mathpar}

\begin{eqnarray*}
  \\
  \meaningof{-} : TS \to ST
\end{eqnarray*}

\begin{eqnarray*}
  \\
  L : TS \to ST
\end{eqnarray*}

\begin{eqnarray*}
  \\
  P \models E \iff P \in \meaningof{E}
\end{eqnarray*}

\begin{eqnarray*}
  P \approx_{L} Q \iff \forall E \in L. P \models E \iff Q \models E
\end{eqnarray*}

\begin{eqnarray*}
  P \approx_{K} Q
\end{eqnarray*}

\begin{eqnarray*}
  P \approx Q
\end{eqnarray*}

$\approx_{K} = \approx = \approx_{L}$

\subsubsection{Contextual duality}

Note that contexts extend the quotation operation to a family of
operations from processes to names. Given a context, $M$, we can
define a \emph{nominal context}, $\quotep{M}$ by $\quotep{M}[P] :=
\quotep{M[P]}$. To foreshadow what is to come we observe that these
operations enjoy a duality with processes very much like the duality
between vectors and maps from vectors to scalars.

Further, because the calculus is essentially higher-order, we have a
correspondence between contexts and processes. More specifically,
given a name $x$ and a context $M$ we can construct $M^{*}_{x}$ such
that 

\begin{mathpar}
  M^{*}_{x} | \lift{x}{P} \red M[P]
\end{mathpar}

namely,

\begin{mathpar}
  M^{*}_{x} := x?(u).M[\dropn{u}]
\end{mathpar}

The dependence of $M^{*}_{x}$ on a name makes it an abstraction, 

\begin{mathpar}
  M^{*} := (x)x?(u).M[\dropn{u}]
\end{mathpar}

\subsection{Additional notation}

It will sometimes be convenient to denote the process a name
quotes. We already have the notation $x = \quotep{P}$, but it will be
convenient to introduce an alternate notation, $\procn{x}$, when we
want to emphasize the connection to the use of the name. Note that, by
virtue of name equivalence, $\quotep{\procn{x}} \nameeq x$; so, the
notation is consistent with previous definitions.

Further, because names have structure it is possible to effect
substitutions on the basis of that structure. This means we need to
upgrade our notation for substitutions, which we accomplish by
adapting comprehension notation. Thus,

\begin{mathpar}
  P\{ y / x : x \in S \}
\end{mathpar}

is interpreted to mean the process derived from P by replacing (in a
capture-avoiding manner) each occurrence of $x$ in $S$ by $y$. For example,

\begin{mathpar}
  P\{ \quotep{\procn{x}|\procn{x}} / x : x \in \freenames{P} \}
\end{mathpar}

will replace each (occurrence) of a free name $x$ in $P$ by
$\quotep{\procn{x}|\procn{x}}$.

Also, we will avail ourselves of the notation $x^{L}$ and $x^{R}$ to
denote injections of a name into disjoint copies of the name
space. There are numerous ways to accomplish this. One example can be
found in \cite{MeredithR05}. This notation overloads to vectors of
names: $\vec{x}^{\pi} := (x_{i}^{\pi} \; : \; 0 \leq i < |\vec{x}| )$ where $\pi \in \{L,R\}$.

We also use $P^{\Box} := P|\Box$.

In \cite{MeredithR05} an interpretation of the new operator is
given. It turns out that there are several possible interpretations
all enjoying the requisite algebraic properties of the operator (see
\cite{milner91polyadicpi}). We will therefore make liberal use of
$(\nu\; \vec{x})P$.

% subsection the_syntax_and_semantics_of_the_notation_system (end)   

\input{qm2pi.qmops} 

\input{qm2pi.sterngerlach} 

\input{qm2pi.metric} 

% section concurrent_process_calculi (end)

%\input{qm2pi.proofsketch}

% section proof sketch (end)

%\input{qm2pi.slviaknots} 

% section spatial logic via knots (end)

\input{qm2pi.conclusion}

% section conclusion (end)

%\input{qm2pi.dtcodes} 

% section wiring algorithm (end)

\input{qm2pi.ack} 

% section acknowledgments (end)

\newpage


\bibliographystyle{plain}   
\bibliography{../../biblios/main.bib}

\input{qm2pi.rhodetails}

\end{document}

 

% section concurrent_process_calculi (end)

%\documentclass[12pt]{llncs}
%\documentclass{jktr}

\usepackage[pdftex]{hyperref}                   
\usepackage {listings}
\usepackage {mathpartir}
\usepackage{bcprules}
%\usepackage{listings}
                       
\usepackage{graphicx} 
%\usepackage[margins=2.5cm,nohead,nofoot]{geometry}
%\usepackage{geometry}
\usepackage{amsfonts}
\usepackage{amstext}
\usepackage{latexsym}
\usepackage{amssymb}
\usepackage{color}


%\include{myPreamble}
\include{qm2pi.local} 

%\ifpdf
%\usepackage[pdftex]{graphicx}
%\else
%\usepackage{graphicx}
%\fi

 % \ifpdf
%  \usepackage{pdfsync}
%  \if


%\title{Brief Article}
%\author{David F. Snyder}
%\author{L.G. Meredith}

%\address{Dept. of Math., Texas State University--San Marcos, San Marcos, TX 78666}
       
\pagestyle{empty}


\begin{document}

\lstset{language=[Objective]Caml,frame=shadowbox}

\input{qm2pi.front}

% section front matter (end)

\input{qm2pi.intro} 
 
% section introduction (end)

% \input{qm2pi.knotations} 

% section notation (end)

\input{qm2pi.process.calculi} 

% section concurrent_process_calculi_and_spatial_logics_ (end)
    
%\input{qm2pi.knots2pi} 

%\input{qm2pi.trefoil} 

%\input{qm2pi.mainthm} 

% subsection basic_interpretation (end)

%\input{qm2pi.rho.presentation} 
\subsection{The syntax and semantics of the notation system}\label{sub:the_syntax_and_semantics_of_the_notation_system} % (fold)

We now summarize a technical presentation of the calculus that
embodies our theory of dynamics. The typical presentation of such a
calculus follows the style of giving generators and relations on
them. The grammar, below, describing term constructors, freely
generates the set of processes, $\Proc$. This set is then quotiented
by a relation known as structural congruence and it is over this set
that the notion of dynamics is expressed. This presentation is
essentially that of \cite{MeredithR05} with the addition of
polyadicity and summation. For readability we have relegated some of
the technical subtleties to an appendix.

\subsubsection{Process grammar}\label{subsub:process_grammar}

\begin{mathpar}
  \inferrule* [lab=synchronization] {} {{M} \bc \pzero \;|\; x?F \;|\; x!C }
  \and
  \inferrule* [lab=abstraction] {} {{F} \bc (x)P}
  \and
  \inferrule* [lab=concretion] {} {{C} \bc \langle Q \rangle}
  \and
  \inferrule* [lab=process] {} {{P,Q} \bc M \;| \;P|Q \;|\; @{x}}
  \and
  \inferrule* [lab=name] {} {{x} \bc \quotep{P}}
\end{mathpar} 

Note that $\vec{x}$ (resp. $\vec{P}$) denotes a vector of names
(resp. processes) of length $|\vec{x}|$ (resp. $|\vec{P}|$). We adopt
the following useful abbreviations.

\begin{mathpar}
   x?(\vec{y}).P := x.(\vec{y})P \and  x\clift{\vec{P}} := x.\clift{\vec{P}}
   \and x!(y) := \lift{x}{\dropn{y}}
   \and \Pi_{i=0}^{n-1}P_i := P_0 | \ldots | P_{n-1}
\end{mathpar}

\subsubsection{Structural congruence}

\paragraph{Free and bound names and alpha-equivalence.} At the
core of structural equivalence is alpha-equivalence which identifies
process that are the same up to a change of variable. Formally, we
recognize the distinction between free and bound names. The free names
of a process, $\freenames{P}$, may be calculated recursively as
follows:

\begin{mathpar}
\freenames{\pzero} := \emptyset
  \and \\
  \freenames{x?(y).P} := \{ x \} \cup (\freenames{P} \setminus \{ y \})
  \and 
  \freenames{x!\langle P \rangle} := \{ x \} \cup \{ P \} 
  \and \\
  \freenames{P|Q} := \freenames{P} \cup \freenames{Q}
  \and \\
  \freenames{@{x}} := \{ x \}
\end{mathpar}

$\pi$
$\quotep{\pi}$

$\freenames{-} : \pi \to \mathcal{P}(\quotep{\pi})$

\begin{eqnarray*}
  \freenames{\pzero} & := & \emptyset \\
  \freenames{x?(y).P} & := & \{ x \} \cup (\freenames{P} \setminus \{ y \}) \\
  \freenames{x!\langle P \rangle} & := & \{ x \} \cup \{ P \} \\
  \freenames{P|Q} & := & \freenames{P} \cup \freenames{Q} \\
  \freenames{\dropn{x}} & := & \{ x \}
\end{eqnarray*}

The bound names of a process, $\boundnames{P}$, are those names occurring in $P$
that are not free. For example, in $x?(y).0$, the name $x$ is free, while $y$ is bound.

\begin{mathpar}
  \inferrule* [lab=monoidal-laws] {} { P|Q \equiv Q|P \and P|0 \equiv P \and P|(Q|R) \equiv (P|Q)|R }
\end{mathpar}

\begin{mathpar}
  \inferrule* [lab=alpha-equivalence] {} { (x)P \equiv (y)P\{y/x\} \and y \not\in \freenames{P} }
\end{mathpar}

\begin{definition}
Then two processes, $P,Q$, are alpha-equivalent if $P = Q\{\vec{y}/\vec{x}\}$ for
some $\vec{x} \in \boundnames{Q},\vec{y} \in \boundnames{P}$, where $Q\{\vec{y}/\vec{x}\}$
denotes the capture-avoiding substitution of $\vec{y}$ for $\vec{x}$ in $Q$.
\end{definition}

\begin{definition}
  The {\em structural congruence} \cite{SangiorgiWalker} , $\equiv$,
  between processes is the least congruence containing
  alpha-equivalence, satisfying the abelian monoid laws
  (associativity, commutativity and $\pzero$ as identity) for parallel
  composition $|$ and for summation $+$.
\end{definition}

\subsection{Name equivalence}

We take name equivalence, written $\nameeq$, to be the smallest
equivalence relation generated by the following rules.

\begin{mathpar}
\inferrule*[lab=Quote-drop]
{ }
{ \quotep{@{x}} \nameeq x }

\inferrule*[lab=Struct-equiv]
{ P \scong Q }
{ \quotep{P} \nameeq \quotep{Q} }
\end{mathpar}

The astute reader will have noticed that the mutual recursion of names
and processes imposes a mutual recursion on alpha-equivalence and
structural equivalence via name-equivalence. Fortunately, all of this
works out pleasantly and we may calculate in the natural way, free of
concern. The reader interested in the details is referred to the
appendix \ref{appendix:rho_details}.

\subsection{Substitution}

We use $\Proc$ for the set of processes, $\QProc$ for the set of
names, and $\id{\{}\vec{y} / \vec{x} \id{\}}$ to denote partial maps,
$s : \QProc \rightarrow \QProc$. A map, $s$ lifts, uniquely, to a map
on process terms, $\widehat{s} : \Proc \rightarrow \Proc$ by the
following equations.

\begin{mathpar}
  (0) \psubstp{Q}{P} := 0 \\
  (R \juxtap S) \psubstp{Q}{P}
  :=    
  (R)\psubstp{Q}{P} \juxtap (S) \psubstp{Q}{P} \\
  (x?(y).R) \psubstp{Q}{P}    
  :=    
  (x)\substp{Q}{P} (z)\concat( (R \psubstn{z}{y}) \psubstp{Q}{P} ) \\
  (\lift{x}{R}) \psubstp{Q}{P}  
  :=
  \lift{(x)\substp{Q}{P}}{ R \psubstp{Q}{P} } \\
%   (\dropn{x})  \psubstp{Q}{P}       
%   := 
%   \left\{ 
%     \begin{array}{ccc} 
%       \dropn{\quotep{Q}} & & x \nameeq \quotep{P} \\
%       \dropn{x} & & otherwise \\
%     \end{array}
%   \right. 
  (\dropn{x})  \psubstp{Q}{P}       
  := 
  \left\{ 
    \begin{array}{ccc} 
      Q & & x \nameeq \quotep{P} \\
      \dropn{x} & & otherwise \\
    \end{array}
  \right.
\end{mathpar}
 

where

\begin{eqnarray}
  (x)\id{\{} \lpquote Q \rpquote / \lpquote P \rpquote \id{\}}            = 
  \left\{ 
    \begin{array}{ccc}
      \lpquote Q \rpquote & & x \nameeq \lpquote P \rpquote \\
      x & & otherwise \\
    \end{array}
  \right. \nonumber
\end{eqnarray}

and $z$ is chosen distinct from $\quotep{P}$, $\quotep{Q}$, the free
names in $Q$, and all the names in $R$. Our $\alpha$-equivalence will
be built in the standard way from this substitution.

\begin{remark}\label{rem:no_self_referential_names}
  One consequence of these definitions is that $\forall P. \quotep{P}
  \not\in \freenames{P}$.
\end{remark}

\subsection{ Dynamic quote: an example }

Anticipating something of what's to come, consider applying the
substitution, $\widehat{\id{\{}u / z \id{\}}}$, to the following pair
of processes, $\lift{w}{y!(z)}$ and $w[ \lpquote y!(z) \rpquote ]$.

\begin{eqnarray}
	\lift{w}{y!(z)}\widehat{\id{\{}u / z \id{\}}}
		& = &
		\lift{w}{y!(u)} \nonumber\\
	w[ \lpquote y!(z) \rpquote ] \widehat{ \id{\{}u / z \id{\}} }
		& = &
		w[ \lpquote y!(z) \rpquote ] \nonumber
\end{eqnarray}

Because the body of the process between quotes is impervious to
substitution, we get radically different answers. In fact, by
examining the first process in an input context,
e.g. $x?(z).\lift{w}{y!(z)}$, we see that the process under the lift
operator may be shaped by prefixed inputs binding a name inside it. In
this sense, the lift operator will be seen as a way to dynamically
construct processes before reifying them as names.

Finally equipped with these standard features we can present the
dynamics of the calculus.

\subsubsection{Operational semantics} 

Finally, we introduce the computational dynamics. What marks these
algebras as distinct from other more traditionally studied algebraic
structures, e.g. vector spaces or polynomial rings, is the manner in
which dynamics is captured. In traditional structures, dynamics is typically
expressed through morphisms between such structures, as in linear maps
between vector spaces or morphisms between rings. In algebras
associated with the semantics of computation, the dynamics is
expressed as part of the algebraic structure itself, through a
reduction reduction relation typically denoted by $\red$. Below, we
give a recursive presentation of this relation for the calculus used
in the encoding.

$\red \subseteq \pi \times \pi$
$\red : \pi \to \mathcal{P}(\pi)$

\begin{mathpar}
  \inferrule* [lab=Comm] { \textsf{match}( x_{src}, x_{trgt} ) } { x_{trgt}?(y)P \; | \; x_{src}!\langle {Q} \rangle \red P\{\quotep{Q}/y}\} }
  \and \\
  \inferrule* [lab=Par] {{P} \red {P}'} {{{P} | {Q}} \red {{P}' | {Q}}}
  \and
  \inferrule* [lab=Equiv]{{{P} \scong {P}'} \andalso {{P}' \red {Q}'} \andalso {{Q}' \scong {Q}}}{{P} \red {Q}}
\end{mathpar}

\begin{eqnarray*}
  match_{\equiv} (\quotep{P},\quotep{Q}) & := & P \equiv Q \\
  match_{\dagger}(\quotep{P},\quotep{Q}) & := & \forall R. P|Q \red^{*} R => R \red^{*} 0 \\
  match_{K}(\quotep{P},\quotep{Q}) & := & K \mbox{ for some context } K
\end{eqnarray*}

$u?(x)P | u!\langle Q \rangle \red P\{\quotep{Q}/x\}$

%We write $\wred$ for $\red^*$, and $P\red$ if $\exists Q $ such that $ P \red Q$.
We write $P\red$ if $\exists Q $ such that $ P \red Q$ and $P\not\red$, otherwise.

\section{Replication}

As mentioned before, it is known that replication (and hence
recursion) can be implemented in a higher-order process algebra
\cite{SangiorgiWalker}. As our first example of calculation with the
machinery thus far presented we give the construction explicitly in
the {\rhoc}.

\begin{eqnarray}
	D_{x} & := & \prefix{x}{y}{(\binpar{\outputp{x}{y}}{@{y}})} \nonumber\\
	\bangp_{x}{P} & := & \binpar{{x}!\langle{\binpar{D_{x}}{P}}\rangle}{D_{x}} \nonumber
\end{eqnarray}

\begin{eqnarray}
	\bangp_{x}{P} & & \nonumber\\
	=
	& {x}!\langle{(\prefix{x}{y}{(\outputp{x}{y} | @{y})) | P}}\rangle 
	      | \prefix{x}{y}{(\outputp{x}{y} | @{y})} & \nonumber\\
	\red
	& (\outputp{x}{y} | @{y})\substn{\quotep{(\prefix{x}{y}{(@{y} | \outputp{x}{y})) | P}}}{y} & \nonumber\\
	=
	& \outputp{x}{\quotep{(\prefix{x}{y}{(\outputp{x}{y} | @{y})) | P}}}
	  | {(\prefix{x}{y}{(\outputp{x}{y} | @{y})) | P}} & \nonumber\\
	\red
	& \ldots & \nonumber\\
	\red^*
	& P | P | \ldots & \nonumber
\end{eqnarray}

Of course, this encoding, as an implementation, runs away, unfolding
$\bangp{P}$ eagerly. A lazier and more implementable replication
operator, restricted to input-guarded processes, may be obtained as follows.

\begin{eqnarray}
\bangp{\prefix{u}{v}{P}} 
	:= 
	\binpar{\lift{x}{\prefix{u}{v}{(\binpar{D(x)}{P})}}}{D(x)} \nonumber
\end{eqnarray}

\begin{remark}
  Note that the lazier definition still does not deal with summation
  or mixed summation (i.e. sums over input and output). The reader is
  invited to construct definitions of replication that deal with these
  features. 

  Further, the definitions are parameterized in a name, $x$. Can you,
  gentle reader, make a definition that eliminates this parameter and
  guarantees no accidental interaction between the replication
  machinery and the process being replicated -- i.e. no accidental
  sharing of names used by the process to get its work done and the
  name(s) used by the replication to effect copying. This latter
  revision of the definition of replication is crucial to obtaining
  the expected identity $!!P \sim !P$.
\end{remark}

\begin{remark}\label{rem:paradoxical_combinator}
  The reader familiar with the lambda calculus will have noticed the
  similarity between $D$ and the paradoxical combinator.

  [Ed. note: the existence of this seems to suggest we have to be more
  restrictive on the set of processes and names we admit if we are to
  support no-cloning.]
\end{remark}

\subsubsection{Bisimulation}

The computational dynamics gives rise to another kind of equivalence,
the equivalence of computational behavior. As previously mentioned
this is typically captured \emph{via} some form of bisimulation.

% The notion we use in this paper is weak barbed bisimulation
% \cite{milner91polyadicpi}.

The notion we use in this paper is derived from weak barbed
bisimulation \cite{milner91polyadicpi}. 

\begin{definition}
An \emph{observation relation}, $\downarrow_{\mathcal N}$, over a set
of names, $\mathcal N$, is the smallest relation satisfying the rules
below.

\infrule[Out-barb]{y \in {\mathcal N}, \; x \nameeq y}
		  {\outputp{x}{v} \downarrow_{\mathcal N} x}
\infrule[Par-barb]{\mbox{$P\downarrow_{\mathcal N} x$ or $Q\downarrow_{\mathcal N} x$}}
		  {\binpar{P}{Q} \downarrow_{\mathcal N} x}

We write $P \Downarrow_{\mathcal N} x$ if there is $Q$ such that 
$P \wred Q$ and $Q \downarrow_{\mathcal N} x$.
\end{definition}

\begin{definition}
%\label{def.bbisim}
An  ${\mathcal N}$-\emph{barbed bisimulation} over a set of names, ${\mathcal N}$, is a symmetric binary relation 
${\mathcal S}_{\mathcal N}$ between agents such that $P\rel{S}_{\mathcal N}Q$ implies:
\begin{enumerate}
\item If $P \red P'$ then $Q \wred Q'$ and $P'\rel{S}_{\mathcal N} Q'$.
\item If $P\downarrow_{\mathcal N} x$, then $Q\Downarrow_{\mathcal N} x$.
\end{enumerate}
$P$ is ${\mathcal N}$-barbed bisimilar to $Q$, written
$P \wbbisim_{\mathcal N} Q$, if $P \rel{S}_{\mathcal N} Q$ for some ${\mathcal N}$-barbed bisimulation ${\mathcal S}_{\mathcal N}$.
\end{definition}

$\mathcal{R} \subseteq \pi \times \pi$

$P \mathcal{R} Q => \forall P'. P \red P' \Rightarrow \exists Q'. Q \red Q', P' \mathcal{R} Q'$

$P \vdash x \Rightarrow Q \vdash x$

\begin{mathpar}
  \inferrule*[lab=Out-barb]{x \nameeq y}{{y}!\langle{Q}\rangle \vdash x}
  \and
  \inferrule*[lab=Par-barb]{\mbox{$P\vdash x$ or $Q\vdash x$}}{\binpar{P}{Q} \vdash x}
\end{mathpar}

\subsubsection{Contexts}

One of the principle advantages of computational calculi like the
$\pi$-calculus is a well-defined notion of context,
contextual-equivalence and a correlation between
contextual-equivalence and notions of bisimulation. The notion of
context allows the decomposition of a process into (sub-)process and
its syntactic environment, its context. Thus, a context may be
thought of as a process with a ``hole'' (written $\Box$) in it. The
application of a context $M$ to a process $P$, written $M[P]$, is
tantamount to filling the hole in $M$ with $P$. In this paper we do
not need the full weight of this theory, but do make use of the notion
of context in the proof the main theorem. 

\begin{mathpar}
  \inferrule* [lab=summation] {} {{M_{M},M_{N}} \bc \Box \;|\; x.M_{A} \;|\; M_{M}+M_{N}}
  \and
  \inferrule* [lab=agent] {} {{M_{A}} \bc (\vec{x})M_{P} \;| \; \clift{P_0,\ldots,M_{P},\ldots,P_N}}
  \and \\
  \inferrule* [lab=process] {} {{M_{P}} \bc M_{N} \;| \;P|M_{P} }
\end{mathpar} 

\begin{mathpar}
  \inferrule* [lab=sychronization] {} {M_{N} \bc \Box \;|\; x?M_{F} \;|\; x!M_{C}}
  \and
  \inferrule* [lab=abstraction] {} {{M_{F}} \bc (x)M_{P} }
  \and
  \inferrule* [lab=concretion] {} {{M_{C}} \bc \langle M_{P} \rangle }
  \and \\
  \inferrule* [lab=process] {} {{M_{P}} \bc M_{N} \;| \;P|M_{P} }
\end{mathpar}

\begin{definition}[contextual application] Given a context $M$, and
  process $P$, we define the \emph{contextual application}, $M[P] :=
  M\{P/\Box\}$. That is, the contextual application of M to P is the
  substitution of $P$ for $\Box$ in $M$.
\end{definition}

$\meaningof{-} : L \to \mathcal{P}(\pi)$

\begin{mathpar}
  \inferrule* [lab=collection] {} {\meaningof{true} = \pi, \and \meaningof{~E} = \pi \setminus \meaningof{E}, \and \meaningof{E_{1} \& E_{2}} = \meaningof{E_{1}} \cap \meaningof{E_{2}}}
\end{mathpar}

\begin{mathpar}
  \inferrule* [lab=structure] {} {\meaningof{0} = \{ P \in \pi | P \equiv 0 \}, \and \\ \meaningof{E_1 | E_2} = \{ P \in \pi | P \equiv P_{1} | P_{2}, P_{1} \in \meaningof{E_{1}}, P_{2} \in \meaningof{E_2}\} }
\end{mathpar}

\begin{mathpar}
 \inferrule* [lab=behavior] {} {\meaningof{\langle a?b \rangle E} = \{ P \in \pi | P \equiv Q | u?(y)P', \\ \and \\\\ \and \\ \;\;\; u \in \meaningof{a}, \forall z.P'\{z/y\} \in \meaningof{E\{z/b\}}\}, \and \\ \meaningof{a!E} = \{ P \in \pi | P \equiv Q | x!\langle P' \rangle, x \in \meaningof{a} P' \in \meaningof{E}\} }
\end{mathpar}

\begin{mathpar}
 \inferrule* [lab=nominal] {} {\meaningof{\quotep{E}} = \{ \quotep{P} \in \quotep{\pi} | P \in \meaningof{E} \}, \and \meaningof{\quotep{P}} = \{ \quotep{Q} \in \quotep{\pi} | P \equiv Q \} \and \\ \meaningof{@\quotep{E}} = \{ P \in \pi | P \equiv @x, x \in \meaningof{E} \}}
\end{mathpar}

\begin{eqnarray*}
  \\
  \meaningof{-} : TS \to ST
\end{eqnarray*}

\begin{eqnarray*}
  \\
  L : TS \to ST
\end{eqnarray*}

\begin{eqnarray*}
  \\
  P \models E \iff P \in \meaningof{E}
\end{eqnarray*}

\begin{eqnarray*}
  P \approx_{L} Q \iff \forall E \in L. P \models E \iff Q \models E
\end{eqnarray*}

\begin{eqnarray*}
  P \approx_{K} Q
\end{eqnarray*}

\begin{eqnarray*}
  P \approx Q
\end{eqnarray*}

$\approx_{K} = \approx = \approx_{L}$

\subsubsection{Contextual duality}

Note that contexts extend the quotation operation to a family of
operations from processes to names. Given a context, $M$, we can
define a \emph{nominal context}, $\quotep{M}$ by $\quotep{M}[P] :=
\quotep{M[P]}$. To foreshadow what is to come we observe that these
operations enjoy a duality with processes very much like the duality
between vectors and maps from vectors to scalars.

Further, because the calculus is essentially higher-order, we have a
correspondence between contexts and processes. More specifically,
given a name $x$ and a context $M$ we can construct $M^{*}_{x}$ such
that 

\begin{mathpar}
  M^{*}_{x} | \lift{x}{P} \red M[P]
\end{mathpar}

namely,

\begin{mathpar}
  M^{*}_{x} := x?(u).M[\dropn{u}]
\end{mathpar}

The dependence of $M^{*}_{x}$ on a name makes it an abstraction, 

\begin{mathpar}
  M^{*} := (x)x?(u).M[\dropn{u}]
\end{mathpar}

\subsection{Additional notation}

It will sometimes be convenient to denote the process a name
quotes. We already have the notation $x = \quotep{P}$, but it will be
convenient to introduce an alternate notation, $\procn{x}$, when we
want to emphasize the connection to the use of the name. Note that, by
virtue of name equivalence, $\quotep{\procn{x}} \nameeq x$; so, the
notation is consistent with previous definitions.

Further, because names have structure it is possible to effect
substitutions on the basis of that structure. This means we need to
upgrade our notation for substitutions, which we accomplish by
adapting comprehension notation. Thus,

\begin{mathpar}
  P\{ y / x : x \in S \}
\end{mathpar}

is interpreted to mean the process derived from P by replacing (in a
capture-avoiding manner) each occurrence of $x$ in $S$ by $y$. For example,

\begin{mathpar}
  P\{ \quotep{\procn{x}|\procn{x}} / x : x \in \freenames{P} \}
\end{mathpar}

will replace each (occurrence) of a free name $x$ in $P$ by
$\quotep{\procn{x}|\procn{x}}$.

Also, we will avail ourselves of the notation $x^{L}$ and $x^{R}$ to
denote injections of a name into disjoint copies of the name
space. There are numerous ways to accomplish this. One example can be
found in \cite{MeredithR05}. This notation overloads to vectors of
names: $\vec{x}^{\pi} := (x_{i}^{\pi} \; : \; 0 \leq i < |\vec{x}| )$ where $\pi \in \{L,R\}$.

We also use $P^{\Box} := P|\Box$.

In \cite{MeredithR05} an interpretation of the new operator is
given. It turns out that there are several possible interpretations
all enjoying the requisite algebraic properties of the operator (see
\cite{milner91polyadicpi}). We will therefore make liberal use of
$(\nu\; \vec{x})P$.

% subsection the_syntax_and_semantics_of_the_notation_system (end)   

\input{qm2pi.qmops} 

\input{qm2pi.sterngerlach} 

\input{qm2pi.metric} 

% section concurrent_process_calculi (end)

%\input{qm2pi.proofsketch}

% section proof sketch (end)

%\input{qm2pi.slviaknots} 

% section spatial logic via knots (end)

\input{qm2pi.conclusion}

% section conclusion (end)

%\input{qm2pi.dtcodes} 

% section wiring algorithm (end)

\input{qm2pi.ack} 

% section acknowledgments (end)

\newpage


\bibliographystyle{plain}   
\bibliography{../../biblios/main.bib}

\input{qm2pi.rhodetails}

\end{document}



% section proof sketch (end)

%\section{Unlikely characters: spatial logic for
  knots}\label{sub:characteristic_formulae} % (fold)

Associated to the mobile process calculi are a family of logics known
as the Hennessy-Milner logics. These logics typically enjoy a
semantics interpreting formulae as sets of processes that when
factored through the encoding outlined above allows an identification
of classes of knots with logical formulae. In the context of this
encoding the sub-family known as the spatial logics \cite{CairesC03}
\cite{CairesC04} \cite{Caires04} are of particular interest providing
several important features for expressing and reasoning about
properties (i.e. classes) of knots. We hint here at how this may be done.

%\begin{description}
%\item [structural connectives] 
\subsubsection{Structural connectives} The spatial logics enjoy
structural connectives corresponding, at the logical level, to the
parallel composition ($P | Q$) and new name ($(\nu \; x)P$)
connectives for processes. As illustrated in the examples below, these
connectives are extremely expressive given the shape of our encoding.
%\item [decideable satisfaction]

\subsubsection{Decideable satisfaction}
In \cite{Caires04} the satisfaction relation is shown to be decideable
for a rich class of processes. It further turns out that the image of
the our encoding is a proper subset of that class. This result
provides the basis for an algorithm by which to search for knots
enjoying a given property.
%\item [characteristic formulae]

\subsubsection{Characteristic formulae}
In the same paper \cite{Caires04} , Caires presents a means of calculating
characteristic formulae, selecting equivalence classes of processes
up to a pre--specified depth limit on the support set of names. Composed with our
encoding, this characteristic formula can be used to select
characteristic formulae for knots.
%\end{description}

\subsubsection{Spatial logic formulae}

The grammar below (segmented for comprehension) summarizes the syntax
of spatial logic formulae. We employ illustrative examples in the
sequel to provide an intuitive understanding of their meaning
referring the reader to \cite{Caires04} for a more detailed explication
of the semantics.

\begin{mathpar}
  \inferrule* [lab=boolean] {} {{A,B} \bc T \;|\; \neg A \;|\; A \wedge B \;|\; \eta = \eta'}
  \and
  \inferrule* [lab=spatial] {} {|\; \pzero \;|\; A | B \;|\; x \text{\textregistered} A \;|\; \forall x . A \;|\;  H x . A}
  \and
  \inferrule* [lab=behavioral] {} {|\; \alpha . A}
  \and 
  \inferrule* [lab=recursion] {} {|\; X(\vec{u}) \;|\; \mu X(\vec{u}) . A}
  \and
  \inferrule* [lab=action] {} {\alpha \bc \langle x?(\vec{y}) \rangle \;|\; \langle x!(\vec{y}) \rangle \;|\; \langle \tau \rangle}
  \and 
  \inferrule* [lab=name] {} {\eta \bc x \;|\; \tau}
\end{mathpar} 

% subsection characteristic_formulae (end)   	 

\subsection{Example formulae}\label{sub:example_formulae_} % (fold)

\subsubsection{Crossing as formula.}
% 
% \begin{align*}
%   \frac{d}{dx} \sin x &= \cos x 
%   & \frac{d}{dx} e^x &= e^x \\
%   \frac{d}{dx} \cos x &= - \sin x 
%   & \frac{d}{dx} \log x &= \frac{1}{x} \\
% \end{align*} 

\begin{align*}
 \mu C(x_{0},x_{1},y_{0},y_{1},u).&(\langle x_{0}?(z) \rangle(\langle u! \rangle\langle y_{1}!z \rangle C(x_{0},x_{1},y_{0},y_{1},u)) & \\
  & \wedge \langle y_{1}?(z) \rangle (\langle u! \rangle \langle x_{0}!z \rangle C(x_{0},x_{1},y_{0},y_{1},u)) & \\
  & \wedge \langle x_{1}?(z) \rangle (\langle u? \rangle \langle y_{0}!z \rangle C(x_{0},x_{1},y_{0},y_{1},u)) & \\
  & \wedge \langle y_{0}?(z) \rangle (\langle u? \rangle \langle x_{1}!z \rangle C(x_{0},x_{1},y_{0},y_{1},u))) &
\end{align*}

The lexicographical similarity between the shape of this formulae and
the shape of definition of the process representing a crossing reveals
the intuitive meaning of this formulae. It describes the capabilities
of a process that has the right to represent a crossing. For example
it picks out processes that may perform an input on the port $x_0$ in
its initial menu of capabilities. What differentiates the formula
from the process, however, is that the crossing process is the
smallest candidate to satisfy the formula. Infinitely many other
processes -- with internal behavior hidden behind this interface, so
to speak -- also satisfy this formula. Even this simple formula,
then, can be seen to open a new view onto knots, providing a
computational interpretation of \emph{virtual} knots.

Note that this formula is derived by hand. A similar formula can be
derived by employing Caires' calculation of characteristic formula
\cite{Caires04} to the process representing a crossing. In light of
this discussion, we let
$\meaningof{C}_{\phi}(x0,x1,y0,y1,u)$ denote a formula specifying the
dynamics we wish to capture of a crossing. To guarantee we preserve
the shape of the interface and minimal semantics we demand that
$\meaningof{C}_{\phi}(x0,x1,y0,y1,u) \Rightarrow
\textbf{C}(x0,x1,y0,y1,u)$ where $\textbf{C}(x0,x1,y0,y1,u)$ denotes
the formula above.
                            
\subsubsection{Crossing number constraints.}
The moral content of the context lemma (Lemma \ref{context}) is that the notion of
``locality'' in the Reidemeister moves is effectively captured by the
parallel composition operator of the process calculus. This intuition
extends through the logic. Given a formula,
$\meaningof{C}_{\phi}(x0,x1,y0,y1,u)$, we can use the structural
connectives to specify constraints on crossing numbers, such as at
least $n$ crossings, or exactly $n$ crossings.
\begin{mathpar}
  \inferrule* [lab=at-least-n] {} { K^{\geq n}_{\phi}(\vec{xs},\vec{ys}) := \Pi_{i=0}^{n-1} Hu . \meaningof{C}_{\phi}(xs_i,ys_i,u) | T }
  \and 
  \inferrule* [lab=exactly-n] {} { K^{= n}_{\phi}(\vec{xs},\vec{ys}) := \Pi_{i=0}^{n-1} Hu . \meaningof{C}_{\phi}(xs_i,ys_i,u) | \neg (\forall x_0,y_0,x_1,y_1,u . \meaningof{C}_{\phi}(x_0,y_0,x_1,y_1,u) | T) }
\end{mathpar}

To round out this section, recall that the encoding of an $n$-crossing
knot decomposes into a parallel composition of $n$ \emph{copies} of a
crossing process together with a wiring harness. To specify different
knot classes with the same crossing number amounts to specifying
logical constraints on the wiring harness. In the interest of space,
we defer examples to a forthcoming paper. Suffice it to say that both
the conditions ``alternating knot'' and ``contains the tangle
corresponding to 5/3'' are expressible. For example, it is possible to
calculate the characteristic formula of a process corresponding to the
tangle 5/3 and conjoin it into the classifying formula via the
composition connective of the logic.

Finally, we wish to observe that it is entirely within reason to
contemplate a more domain-specific version of spatial logic tailored
to the shape of processes in the image of the encoding. Such a
domain-specific logic would have a better claim to the title formal
language of knot properties.

% subsection example_formulae_ (end)

% section knots_as_processes (end) 

% section spatial logic via knots (end)

\section{Conclusions and future work}

\paragraph{Testing physical space}
You, gentle reader, may wonder why of all the theorems to be proved
given this set up we pick the one above. In some sense it's hardly
central to quantum mechanics. We see it as central in the sense that
it firmly establishes a notion of physical space arising from a notion
of the equivalence of behavior. Relating bisimulation to a metric is a
big step forward, but one is faced with interpreting the relationship
of that metric space to something more physical. Quantum mechanical
notions of ``physical'' space are still far from intuitive, but by
relating this idea of distance as testing to calculations that predict
physical circumstances we are making a not insignificant step forward
toward an understanding of the physical space we inhabit as
essentially dynamic.

\paragraph{Effectivity and simulation}
One of the observations we have yet to make is that the entire program
spelled out here is effective. We have built various interpreters for
the reflective calculus at work in this interpretation. In principle,
then, we can simulate quantum mechanics on a computer. The place where
the simulation may lose fidelity is the infinitely branching summation
for the annihilator.

In this connection i also want to point out that the evaluation style
calculation of the inner product puts the non-determinism of the
summation right at the heart of measurement. This suggests that
Milner's original reduction-based formulation of the dynamics of his
calculi in terms of sums was not just notationally suggestive of a
notion of measure-and-continue but captured some significant part of
the physics.

\paragraph{Quantum continuations}
In light of this last observation i want to point out that the
predominant account of quantum mechanics is missing a key aspect of a
truly compositional story of the physical situation. In a real lab,
when a measurement is made the observation can be made to feed into
another device that then makes another measurement conditioned on the
results of the first. This means that after the superposition was
collapsed the entire experimental set up remained in
superposition. While QM offers a means of writing this down it doesn't
quite line up well with the well-trodden formulation of computation
and continuation that we see so succinctly expressed in Milner's
calculi. This suggests that there might be advantages to this account
of dynamics waiting to be explored.

\paragraph{Quantum logic}
In this connection, we also note that by virtue of having the
Hennessy-Milner construction, we can pull the construction through the
interpretation of QM. This gives us a natural candidate for a quantum
logic that enjoys an extremely tight connection with it's domain of
interpretation, making the construction much less ad hoc (rather it is
the image of functor!).

\paragraph{Quantum probabiity}
i have questions about the basis of the interpretation of inner
product as probability amplitude. In particular, using which
axiomatization of probability theory does the notion of probability
amplitude earn the right to be so dubbed? In other words, where is the
proof that the operation for calculating a probability amplitude (and
then squaring) satisfies the axioms of what it means to calculate a
probability? Even if such a proof exists (i have yet to find it in the
literature), i wonder if it might not be possible to turn things on
their heads. Can we view the calculation of the probability amplitude
as an axiomatization of probability? If so, then the definition we
give for calculating probability amplitude may provide the basis for
an \emph{effective} theory of probability.

\paragraph{Quantum vs ``biological'' information}
Finally, i want to conclude with a more philosophical observation. At
a recent workshop in which QM was a predominant topic i noticed
something about quantum information. The speaker was giving a riveting
discussion of axiomatic QM and showing how properties of ``no
cloning'' and ``no deleting'' emerged as consequences of the
axiomatization. Theorems of this form are necessary to give us a sense
of confidence that our axioms characterize the physical theory. What
struck me, though, was that if quantum information is neither erasable
nor replicable it is markedly different from \emph{life}. Two of the
things we know about life is that

\begin{itemize}
  \item it ends;
  \item to gain some measure of persistence, to transcend it's
    finitude it is imminently copyable.
\end{itemize}

Both of these qualities are summarized succinctly in the aphorism: all
flesh is grass. For me these two kinds of ``information'' -- call them
quantum and biological -- are end points on a spectrum of strategies
for persistence. At one end, we have those curious entities that enjoy
uniqueness and permanence; at the other, we have those who in the face
of a certain end and an uncertain present make a go of passing
something on. To me one of the more remarkable aspects of the latter
strategy is that in the presence of noise (and certain features of
copying) we get a kind of dynamism, a chance for improvement against a
given persistent condition.

% subsection other_calculi_other_bisimulations_and_geometry_as_behavior (end)




% section conclusion (end)

%\documentclass[12pt]{llncs}
%\documentclass{jktr}

\usepackage[pdftex]{hyperref}                   
\usepackage {listings}
\usepackage {mathpartir}
\usepackage{bcprules}
%\usepackage{listings}
                       
\usepackage{graphicx} 
%\usepackage[margins=2.5cm,nohead,nofoot]{geometry}
%\usepackage{geometry}
\usepackage{amsfonts}
\usepackage{amstext}
\usepackage{latexsym}
\usepackage{amssymb}
\usepackage{color}


%\include{myPreamble}
\include{qm2pi.local} 

%\ifpdf
%\usepackage[pdftex]{graphicx}
%\else
%\usepackage{graphicx}
%\fi

 % \ifpdf
%  \usepackage{pdfsync}
%  \if


%\title{Brief Article}
%\author{David F. Snyder}
%\author{L.G. Meredith}

%\address{Dept. of Math., Texas State University--San Marcos, San Marcos, TX 78666}
       
\pagestyle{empty}


\begin{document}

\lstset{language=[Objective]Caml,frame=shadowbox}

\input{qm2pi.front}

% section front matter (end)

\input{qm2pi.intro} 
 
% section introduction (end)

% \input{qm2pi.knotations} 

% section notation (end)

\input{qm2pi.process.calculi} 

% section concurrent_process_calculi_and_spatial_logics_ (end)
    
%\input{qm2pi.knots2pi} 

%\input{qm2pi.trefoil} 

%\input{qm2pi.mainthm} 

% subsection basic_interpretation (end)

%\input{qm2pi.rho.presentation} 
\subsection{The syntax and semantics of the notation system}\label{sub:the_syntax_and_semantics_of_the_notation_system} % (fold)

We now summarize a technical presentation of the calculus that
embodies our theory of dynamics. The typical presentation of such a
calculus follows the style of giving generators and relations on
them. The grammar, below, describing term constructors, freely
generates the set of processes, $\Proc$. This set is then quotiented
by a relation known as structural congruence and it is over this set
that the notion of dynamics is expressed. This presentation is
essentially that of \cite{MeredithR05} with the addition of
polyadicity and summation. For readability we have relegated some of
the technical subtleties to an appendix.

\subsubsection{Process grammar}\label{subsub:process_grammar}

\begin{mathpar}
  \inferrule* [lab=synchronization] {} {{M} \bc \pzero \;|\; x?F \;|\; x!C }
  \and
  \inferrule* [lab=abstraction] {} {{F} \bc (x)P}
  \and
  \inferrule* [lab=concretion] {} {{C} \bc \langle Q \rangle}
  \and
  \inferrule* [lab=process] {} {{P,Q} \bc M \;| \;P|Q \;|\; @{x}}
  \and
  \inferrule* [lab=name] {} {{x} \bc \quotep{P}}
\end{mathpar} 

Note that $\vec{x}$ (resp. $\vec{P}$) denotes a vector of names
(resp. processes) of length $|\vec{x}|$ (resp. $|\vec{P}|$). We adopt
the following useful abbreviations.

\begin{mathpar}
   x?(\vec{y}).P := x.(\vec{y})P \and  x\clift{\vec{P}} := x.\clift{\vec{P}}
   \and x!(y) := \lift{x}{\dropn{y}}
   \and \Pi_{i=0}^{n-1}P_i := P_0 | \ldots | P_{n-1}
\end{mathpar}

\subsubsection{Structural congruence}

\paragraph{Free and bound names and alpha-equivalence.} At the
core of structural equivalence is alpha-equivalence which identifies
process that are the same up to a change of variable. Formally, we
recognize the distinction between free and bound names. The free names
of a process, $\freenames{P}$, may be calculated recursively as
follows:

\begin{mathpar}
\freenames{\pzero} := \emptyset
  \and \\
  \freenames{x?(y).P} := \{ x \} \cup (\freenames{P} \setminus \{ y \})
  \and 
  \freenames{x!\langle P \rangle} := \{ x \} \cup \{ P \} 
  \and \\
  \freenames{P|Q} := \freenames{P} \cup \freenames{Q}
  \and \\
  \freenames{@{x}} := \{ x \}
\end{mathpar}

$\pi$
$\quotep{\pi}$

$\freenames{-} : \pi \to \mathcal{P}(\quotep{\pi})$

\begin{eqnarray*}
  \freenames{\pzero} & := & \emptyset \\
  \freenames{x?(y).P} & := & \{ x \} \cup (\freenames{P} \setminus \{ y \}) \\
  \freenames{x!\langle P \rangle} & := & \{ x \} \cup \{ P \} \\
  \freenames{P|Q} & := & \freenames{P} \cup \freenames{Q} \\
  \freenames{\dropn{x}} & := & \{ x \}
\end{eqnarray*}

The bound names of a process, $\boundnames{P}$, are those names occurring in $P$
that are not free. For example, in $x?(y).0$, the name $x$ is free, while $y$ is bound.

\begin{mathpar}
  \inferrule* [lab=monoidal-laws] {} { P|Q \equiv Q|P \and P|0 \equiv P \and P|(Q|R) \equiv (P|Q)|R }
\end{mathpar}

\begin{mathpar}
  \inferrule* [lab=alpha-equivalence] {} { (x)P \equiv (y)P\{y/x\} \and y \not\in \freenames{P} }
\end{mathpar}

\begin{definition}
Then two processes, $P,Q$, are alpha-equivalent if $P = Q\{\vec{y}/\vec{x}\}$ for
some $\vec{x} \in \boundnames{Q},\vec{y} \in \boundnames{P}$, where $Q\{\vec{y}/\vec{x}\}$
denotes the capture-avoiding substitution of $\vec{y}$ for $\vec{x}$ in $Q$.
\end{definition}

\begin{definition}
  The {\em structural congruence} \cite{SangiorgiWalker} , $\equiv$,
  between processes is the least congruence containing
  alpha-equivalence, satisfying the abelian monoid laws
  (associativity, commutativity and $\pzero$ as identity) for parallel
  composition $|$ and for summation $+$.
\end{definition}

\subsection{Name equivalence}

We take name equivalence, written $\nameeq$, to be the smallest
equivalence relation generated by the following rules.

\begin{mathpar}
\inferrule*[lab=Quote-drop]
{ }
{ \quotep{@{x}} \nameeq x }

\inferrule*[lab=Struct-equiv]
{ P \scong Q }
{ \quotep{P} \nameeq \quotep{Q} }
\end{mathpar}

The astute reader will have noticed that the mutual recursion of names
and processes imposes a mutual recursion on alpha-equivalence and
structural equivalence via name-equivalence. Fortunately, all of this
works out pleasantly and we may calculate in the natural way, free of
concern. The reader interested in the details is referred to the
appendix \ref{appendix:rho_details}.

\subsection{Substitution}

We use $\Proc$ for the set of processes, $\QProc$ for the set of
names, and $\id{\{}\vec{y} / \vec{x} \id{\}}$ to denote partial maps,
$s : \QProc \rightarrow \QProc$. A map, $s$ lifts, uniquely, to a map
on process terms, $\widehat{s} : \Proc \rightarrow \Proc$ by the
following equations.

\begin{mathpar}
  (0) \psubstp{Q}{P} := 0 \\
  (R \juxtap S) \psubstp{Q}{P}
  :=    
  (R)\psubstp{Q}{P} \juxtap (S) \psubstp{Q}{P} \\
  (x?(y).R) \psubstp{Q}{P}    
  :=    
  (x)\substp{Q}{P} (z)\concat( (R \psubstn{z}{y}) \psubstp{Q}{P} ) \\
  (\lift{x}{R}) \psubstp{Q}{P}  
  :=
  \lift{(x)\substp{Q}{P}}{ R \psubstp{Q}{P} } \\
%   (\dropn{x})  \psubstp{Q}{P}       
%   := 
%   \left\{ 
%     \begin{array}{ccc} 
%       \dropn{\quotep{Q}} & & x \nameeq \quotep{P} \\
%       \dropn{x} & & otherwise \\
%     \end{array}
%   \right. 
  (\dropn{x})  \psubstp{Q}{P}       
  := 
  \left\{ 
    \begin{array}{ccc} 
      Q & & x \nameeq \quotep{P} \\
      \dropn{x} & & otherwise \\
    \end{array}
  \right.
\end{mathpar}
 

where

\begin{eqnarray}
  (x)\id{\{} \lpquote Q \rpquote / \lpquote P \rpquote \id{\}}            = 
  \left\{ 
    \begin{array}{ccc}
      \lpquote Q \rpquote & & x \nameeq \lpquote P \rpquote \\
      x & & otherwise \\
    \end{array}
  \right. \nonumber
\end{eqnarray}

and $z$ is chosen distinct from $\quotep{P}$, $\quotep{Q}$, the free
names in $Q$, and all the names in $R$. Our $\alpha$-equivalence will
be built in the standard way from this substitution.

\begin{remark}\label{rem:no_self_referential_names}
  One consequence of these definitions is that $\forall P. \quotep{P}
  \not\in \freenames{P}$.
\end{remark}

\subsection{ Dynamic quote: an example }

Anticipating something of what's to come, consider applying the
substitution, $\widehat{\id{\{}u / z \id{\}}}$, to the following pair
of processes, $\lift{w}{y!(z)}$ and $w[ \lpquote y!(z) \rpquote ]$.

\begin{eqnarray}
	\lift{w}{y!(z)}\widehat{\id{\{}u / z \id{\}}}
		& = &
		\lift{w}{y!(u)} \nonumber\\
	w[ \lpquote y!(z) \rpquote ] \widehat{ \id{\{}u / z \id{\}} }
		& = &
		w[ \lpquote y!(z) \rpquote ] \nonumber
\end{eqnarray}

Because the body of the process between quotes is impervious to
substitution, we get radically different answers. In fact, by
examining the first process in an input context,
e.g. $x?(z).\lift{w}{y!(z)}$, we see that the process under the lift
operator may be shaped by prefixed inputs binding a name inside it. In
this sense, the lift operator will be seen as a way to dynamically
construct processes before reifying them as names.

Finally equipped with these standard features we can present the
dynamics of the calculus.

\subsubsection{Operational semantics} 

Finally, we introduce the computational dynamics. What marks these
algebras as distinct from other more traditionally studied algebraic
structures, e.g. vector spaces or polynomial rings, is the manner in
which dynamics is captured. In traditional structures, dynamics is typically
expressed through morphisms between such structures, as in linear maps
between vector spaces or morphisms between rings. In algebras
associated with the semantics of computation, the dynamics is
expressed as part of the algebraic structure itself, through a
reduction reduction relation typically denoted by $\red$. Below, we
give a recursive presentation of this relation for the calculus used
in the encoding.

$\red \subseteq \pi \times \pi$
$\red : \pi \to \mathcal{P}(\pi)$

\begin{mathpar}
  \inferrule* [lab=Comm] { \textsf{match}( x_{src}, x_{trgt} ) } { x_{trgt}?(y)P \; | \; x_{src}!\langle {Q} \rangle \red P\{\quotep{Q}/y}\} }
  \and \\
  \inferrule* [lab=Par] {{P} \red {P}'} {{{P} | {Q}} \red {{P}' | {Q}}}
  \and
  \inferrule* [lab=Equiv]{{{P} \scong {P}'} \andalso {{P}' \red {Q}'} \andalso {{Q}' \scong {Q}}}{{P} \red {Q}}
\end{mathpar}

\begin{eqnarray*}
  match_{\equiv} (\quotep{P},\quotep{Q}) & := & P \equiv Q \\
  match_{\dagger}(\quotep{P},\quotep{Q}) & := & \forall R. P|Q \red^{*} R => R \red^{*} 0 \\
  match_{K}(\quotep{P},\quotep{Q}) & := & K \mbox{ for some context } K
\end{eqnarray*}

$u?(x)P | u!\langle Q \rangle \red P\{\quotep{Q}/x\}$

%We write $\wred$ for $\red^*$, and $P\red$ if $\exists Q $ such that $ P \red Q$.
We write $P\red$ if $\exists Q $ such that $ P \red Q$ and $P\not\red$, otherwise.

\section{Replication}

As mentioned before, it is known that replication (and hence
recursion) can be implemented in a higher-order process algebra
\cite{SangiorgiWalker}. As our first example of calculation with the
machinery thus far presented we give the construction explicitly in
the {\rhoc}.

\begin{eqnarray}
	D_{x} & := & \prefix{x}{y}{(\binpar{\outputp{x}{y}}{@{y}})} \nonumber\\
	\bangp_{x}{P} & := & \binpar{{x}!\langle{\binpar{D_{x}}{P}}\rangle}{D_{x}} \nonumber
\end{eqnarray}

\begin{eqnarray}
	\bangp_{x}{P} & & \nonumber\\
	=
	& {x}!\langle{(\prefix{x}{y}{(\outputp{x}{y} | @{y})) | P}}\rangle 
	      | \prefix{x}{y}{(\outputp{x}{y} | @{y})} & \nonumber\\
	\red
	& (\outputp{x}{y} | @{y})\substn{\quotep{(\prefix{x}{y}{(@{y} | \outputp{x}{y})) | P}}}{y} & \nonumber\\
	=
	& \outputp{x}{\quotep{(\prefix{x}{y}{(\outputp{x}{y} | @{y})) | P}}}
	  | {(\prefix{x}{y}{(\outputp{x}{y} | @{y})) | P}} & \nonumber\\
	\red
	& \ldots & \nonumber\\
	\red^*
	& P | P | \ldots & \nonumber
\end{eqnarray}

Of course, this encoding, as an implementation, runs away, unfolding
$\bangp{P}$ eagerly. A lazier and more implementable replication
operator, restricted to input-guarded processes, may be obtained as follows.

\begin{eqnarray}
\bangp{\prefix{u}{v}{P}} 
	:= 
	\binpar{\lift{x}{\prefix{u}{v}{(\binpar{D(x)}{P})}}}{D(x)} \nonumber
\end{eqnarray}

\begin{remark}
  Note that the lazier definition still does not deal with summation
  or mixed summation (i.e. sums over input and output). The reader is
  invited to construct definitions of replication that deal with these
  features. 

  Further, the definitions are parameterized in a name, $x$. Can you,
  gentle reader, make a definition that eliminates this parameter and
  guarantees no accidental interaction between the replication
  machinery and the process being replicated -- i.e. no accidental
  sharing of names used by the process to get its work done and the
  name(s) used by the replication to effect copying. This latter
  revision of the definition of replication is crucial to obtaining
  the expected identity $!!P \sim !P$.
\end{remark}

\begin{remark}\label{rem:paradoxical_combinator}
  The reader familiar with the lambda calculus will have noticed the
  similarity between $D$ and the paradoxical combinator.

  [Ed. note: the existence of this seems to suggest we have to be more
  restrictive on the set of processes and names we admit if we are to
  support no-cloning.]
\end{remark}

\subsubsection{Bisimulation}

The computational dynamics gives rise to another kind of equivalence,
the equivalence of computational behavior. As previously mentioned
this is typically captured \emph{via} some form of bisimulation.

% The notion we use in this paper is weak barbed bisimulation
% \cite{milner91polyadicpi}.

The notion we use in this paper is derived from weak barbed
bisimulation \cite{milner91polyadicpi}. 

\begin{definition}
An \emph{observation relation}, $\downarrow_{\mathcal N}$, over a set
of names, $\mathcal N$, is the smallest relation satisfying the rules
below.

\infrule[Out-barb]{y \in {\mathcal N}, \; x \nameeq y}
		  {\outputp{x}{v} \downarrow_{\mathcal N} x}
\infrule[Par-barb]{\mbox{$P\downarrow_{\mathcal N} x$ or $Q\downarrow_{\mathcal N} x$}}
		  {\binpar{P}{Q} \downarrow_{\mathcal N} x}

We write $P \Downarrow_{\mathcal N} x$ if there is $Q$ such that 
$P \wred Q$ and $Q \downarrow_{\mathcal N} x$.
\end{definition}

\begin{definition}
%\label{def.bbisim}
An  ${\mathcal N}$-\emph{barbed bisimulation} over a set of names, ${\mathcal N}$, is a symmetric binary relation 
${\mathcal S}_{\mathcal N}$ between agents such that $P\rel{S}_{\mathcal N}Q$ implies:
\begin{enumerate}
\item If $P \red P'$ then $Q \wred Q'$ and $P'\rel{S}_{\mathcal N} Q'$.
\item If $P\downarrow_{\mathcal N} x$, then $Q\Downarrow_{\mathcal N} x$.
\end{enumerate}
$P$ is ${\mathcal N}$-barbed bisimilar to $Q$, written
$P \wbbisim_{\mathcal N} Q$, if $P \rel{S}_{\mathcal N} Q$ for some ${\mathcal N}$-barbed bisimulation ${\mathcal S}_{\mathcal N}$.
\end{definition}

$\mathcal{R} \subseteq \pi \times \pi$

$P \mathcal{R} Q => \forall P'. P \red P' \Rightarrow \exists Q'. Q \red Q', P' \mathcal{R} Q'$

$P \vdash x \Rightarrow Q \vdash x$

\begin{mathpar}
  \inferrule*[lab=Out-barb]{x \nameeq y}{{y}!\langle{Q}\rangle \vdash x}
  \and
  \inferrule*[lab=Par-barb]{\mbox{$P\vdash x$ or $Q\vdash x$}}{\binpar{P}{Q} \vdash x}
\end{mathpar}

\subsubsection{Contexts}

One of the principle advantages of computational calculi like the
$\pi$-calculus is a well-defined notion of context,
contextual-equivalence and a correlation between
contextual-equivalence and notions of bisimulation. The notion of
context allows the decomposition of a process into (sub-)process and
its syntactic environment, its context. Thus, a context may be
thought of as a process with a ``hole'' (written $\Box$) in it. The
application of a context $M$ to a process $P$, written $M[P]$, is
tantamount to filling the hole in $M$ with $P$. In this paper we do
not need the full weight of this theory, but do make use of the notion
of context in the proof the main theorem. 

\begin{mathpar}
  \inferrule* [lab=summation] {} {{M_{M},M_{N}} \bc \Box \;|\; x.M_{A} \;|\; M_{M}+M_{N}}
  \and
  \inferrule* [lab=agent] {} {{M_{A}} \bc (\vec{x})M_{P} \;| \; \clift{P_0,\ldots,M_{P},\ldots,P_N}}
  \and \\
  \inferrule* [lab=process] {} {{M_{P}} \bc M_{N} \;| \;P|M_{P} }
\end{mathpar} 

\begin{mathpar}
  \inferrule* [lab=sychronization] {} {M_{N} \bc \Box \;|\; x?M_{F} \;|\; x!M_{C}}
  \and
  \inferrule* [lab=abstraction] {} {{M_{F}} \bc (x)M_{P} }
  \and
  \inferrule* [lab=concretion] {} {{M_{C}} \bc \langle M_{P} \rangle }
  \and \\
  \inferrule* [lab=process] {} {{M_{P}} \bc M_{N} \;| \;P|M_{P} }
\end{mathpar}

\begin{definition}[contextual application] Given a context $M$, and
  process $P$, we define the \emph{contextual application}, $M[P] :=
  M\{P/\Box\}$. That is, the contextual application of M to P is the
  substitution of $P$ for $\Box$ in $M$.
\end{definition}

$\meaningof{-} : L \to \mathcal{P}(\pi)$

\begin{mathpar}
  \inferrule* [lab=collection] {} {\meaningof{true} = \pi, \and \meaningof{~E} = \pi \setminus \meaningof{E}, \and \meaningof{E_{1} \& E_{2}} = \meaningof{E_{1}} \cap \meaningof{E_{2}}}
\end{mathpar}

\begin{mathpar}
  \inferrule* [lab=structure] {} {\meaningof{0} = \{ P \in \pi | P \equiv 0 \}, \and \\ \meaningof{E_1 | E_2} = \{ P \in \pi | P \equiv P_{1} | P_{2}, P_{1} \in \meaningof{E_{1}}, P_{2} \in \meaningof{E_2}\} }
\end{mathpar}

\begin{mathpar}
 \inferrule* [lab=behavior] {} {\meaningof{\langle a?b \rangle E} = \{ P \in \pi | P \equiv Q | u?(y)P', \\ \and \\\\ \and \\ \;\;\; u \in \meaningof{a}, \forall z.P'\{z/y\} \in \meaningof{E\{z/b\}}\}, \and \\ \meaningof{a!E} = \{ P \in \pi | P \equiv Q | x!\langle P' \rangle, x \in \meaningof{a} P' \in \meaningof{E}\} }
\end{mathpar}

\begin{mathpar}
 \inferrule* [lab=nominal] {} {\meaningof{\quotep{E}} = \{ \quotep{P} \in \quotep{\pi} | P \in \meaningof{E} \}, \and \meaningof{\quotep{P}} = \{ \quotep{Q} \in \quotep{\pi} | P \equiv Q \} \and \\ \meaningof{@\quotep{E}} = \{ P \in \pi | P \equiv @x, x \in \meaningof{E} \}}
\end{mathpar}

\begin{eqnarray*}
  \\
  \meaningof{-} : TS \to ST
\end{eqnarray*}

\begin{eqnarray*}
  \\
  L : TS \to ST
\end{eqnarray*}

\begin{eqnarray*}
  \\
  P \models E \iff P \in \meaningof{E}
\end{eqnarray*}

\begin{eqnarray*}
  P \approx_{L} Q \iff \forall E \in L. P \models E \iff Q \models E
\end{eqnarray*}

\begin{eqnarray*}
  P \approx_{K} Q
\end{eqnarray*}

\begin{eqnarray*}
  P \approx Q
\end{eqnarray*}

$\approx_{K} = \approx = \approx_{L}$

\subsubsection{Contextual duality}

Note that contexts extend the quotation operation to a family of
operations from processes to names. Given a context, $M$, we can
define a \emph{nominal context}, $\quotep{M}$ by $\quotep{M}[P] :=
\quotep{M[P]}$. To foreshadow what is to come we observe that these
operations enjoy a duality with processes very much like the duality
between vectors and maps from vectors to scalars.

Further, because the calculus is essentially higher-order, we have a
correspondence between contexts and processes. More specifically,
given a name $x$ and a context $M$ we can construct $M^{*}_{x}$ such
that 

\begin{mathpar}
  M^{*}_{x} | \lift{x}{P} \red M[P]
\end{mathpar}

namely,

\begin{mathpar}
  M^{*}_{x} := x?(u).M[\dropn{u}]
\end{mathpar}

The dependence of $M^{*}_{x}$ on a name makes it an abstraction, 

\begin{mathpar}
  M^{*} := (x)x?(u).M[\dropn{u}]
\end{mathpar}

\subsection{Additional notation}

It will sometimes be convenient to denote the process a name
quotes. We already have the notation $x = \quotep{P}$, but it will be
convenient to introduce an alternate notation, $\procn{x}$, when we
want to emphasize the connection to the use of the name. Note that, by
virtue of name equivalence, $\quotep{\procn{x}} \nameeq x$; so, the
notation is consistent with previous definitions.

Further, because names have structure it is possible to effect
substitutions on the basis of that structure. This means we need to
upgrade our notation for substitutions, which we accomplish by
adapting comprehension notation. Thus,

\begin{mathpar}
  P\{ y / x : x \in S \}
\end{mathpar}

is interpreted to mean the process derived from P by replacing (in a
capture-avoiding manner) each occurrence of $x$ in $S$ by $y$. For example,

\begin{mathpar}
  P\{ \quotep{\procn{x}|\procn{x}} / x : x \in \freenames{P} \}
\end{mathpar}

will replace each (occurrence) of a free name $x$ in $P$ by
$\quotep{\procn{x}|\procn{x}}$.

Also, we will avail ourselves of the notation $x^{L}$ and $x^{R}$ to
denote injections of a name into disjoint copies of the name
space. There are numerous ways to accomplish this. One example can be
found in \cite{MeredithR05}. This notation overloads to vectors of
names: $\vec{x}^{\pi} := (x_{i}^{\pi} \; : \; 0 \leq i < |\vec{x}| )$ where $\pi \in \{L,R\}$.

We also use $P^{\Box} := P|\Box$.

In \cite{MeredithR05} an interpretation of the new operator is
given. It turns out that there are several possible interpretations
all enjoying the requisite algebraic properties of the operator (see
\cite{milner91polyadicpi}). We will therefore make liberal use of
$(\nu\; \vec{x})P$.

% subsection the_syntax_and_semantics_of_the_notation_system (end)   

\input{qm2pi.qmops} 

\input{qm2pi.sterngerlach} 

\input{qm2pi.metric} 

% section concurrent_process_calculi (end)

%\input{qm2pi.proofsketch}

% section proof sketch (end)

%\input{qm2pi.slviaknots} 

% section spatial logic via knots (end)

\input{qm2pi.conclusion}

% section conclusion (end)

%\input{qm2pi.dtcodes} 

% section wiring algorithm (end)

\input{qm2pi.ack} 

% section acknowledgments (end)

\newpage


\bibliographystyle{plain}   
\bibliography{../../biblios/main.bib}

\input{qm2pi.rhodetails}

\end{document}

 

% section wiring algorithm (end)

\documentclass[12pt]{llncs}
%\documentclass{jktr}

\usepackage[pdftex]{hyperref}                   
\usepackage {listings}
\usepackage {mathpartir}
\usepackage{bcprules}
%\usepackage{listings}
                       
\usepackage{graphicx} 
%\usepackage[margins=2.5cm,nohead,nofoot]{geometry}
%\usepackage{geometry}
\usepackage{amsfonts}
\usepackage{amstext}
\usepackage{latexsym}
\usepackage{amssymb}
\usepackage{color}


%\include{myPreamble}
\include{qm2pi.local} 

%\ifpdf
%\usepackage[pdftex]{graphicx}
%\else
%\usepackage{graphicx}
%\fi

 % \ifpdf
%  \usepackage{pdfsync}
%  \if


%\title{Brief Article}
%\author{David F. Snyder}
%\author{L.G. Meredith}

%\address{Dept. of Math., Texas State University--San Marcos, San Marcos, TX 78666}
       
\pagestyle{empty}


\begin{document}

\lstset{language=[Objective]Caml,frame=shadowbox}

\input{qm2pi.front}

% section front matter (end)

\input{qm2pi.intro} 
 
% section introduction (end)

% \input{qm2pi.knotations} 

% section notation (end)

\input{qm2pi.process.calculi} 

% section concurrent_process_calculi_and_spatial_logics_ (end)
    
%\input{qm2pi.knots2pi} 

%\input{qm2pi.trefoil} 

%\input{qm2pi.mainthm} 

% subsection basic_interpretation (end)

%\input{qm2pi.rho.presentation} 
\subsection{The syntax and semantics of the notation system}\label{sub:the_syntax_and_semantics_of_the_notation_system} % (fold)

We now summarize a technical presentation of the calculus that
embodies our theory of dynamics. The typical presentation of such a
calculus follows the style of giving generators and relations on
them. The grammar, below, describing term constructors, freely
generates the set of processes, $\Proc$. This set is then quotiented
by a relation known as structural congruence and it is over this set
that the notion of dynamics is expressed. This presentation is
essentially that of \cite{MeredithR05} with the addition of
polyadicity and summation. For readability we have relegated some of
the technical subtleties to an appendix.

\subsubsection{Process grammar}\label{subsub:process_grammar}

\begin{mathpar}
  \inferrule* [lab=synchronization] {} {{M} \bc \pzero \;|\; x?F \;|\; x!C }
  \and
  \inferrule* [lab=abstraction] {} {{F} \bc (x)P}
  \and
  \inferrule* [lab=concretion] {} {{C} \bc \langle Q \rangle}
  \and
  \inferrule* [lab=process] {} {{P,Q} \bc M \;| \;P|Q \;|\; @{x}}
  \and
  \inferrule* [lab=name] {} {{x} \bc \quotep{P}}
\end{mathpar} 

Note that $\vec{x}$ (resp. $\vec{P}$) denotes a vector of names
(resp. processes) of length $|\vec{x}|$ (resp. $|\vec{P}|$). We adopt
the following useful abbreviations.

\begin{mathpar}
   x?(\vec{y}).P := x.(\vec{y})P \and  x\clift{\vec{P}} := x.\clift{\vec{P}}
   \and x!(y) := \lift{x}{\dropn{y}}
   \and \Pi_{i=0}^{n-1}P_i := P_0 | \ldots | P_{n-1}
\end{mathpar}

\subsubsection{Structural congruence}

\paragraph{Free and bound names and alpha-equivalence.} At the
core of structural equivalence is alpha-equivalence which identifies
process that are the same up to a change of variable. Formally, we
recognize the distinction between free and bound names. The free names
of a process, $\freenames{P}$, may be calculated recursively as
follows:

\begin{mathpar}
\freenames{\pzero} := \emptyset
  \and \\
  \freenames{x?(y).P} := \{ x \} \cup (\freenames{P} \setminus \{ y \})
  \and 
  \freenames{x!\langle P \rangle} := \{ x \} \cup \{ P \} 
  \and \\
  \freenames{P|Q} := \freenames{P} \cup \freenames{Q}
  \and \\
  \freenames{@{x}} := \{ x \}
\end{mathpar}

$\pi$
$\quotep{\pi}$

$\freenames{-} : \pi \to \mathcal{P}(\quotep{\pi})$

\begin{eqnarray*}
  \freenames{\pzero} & := & \emptyset \\
  \freenames{x?(y).P} & := & \{ x \} \cup (\freenames{P} \setminus \{ y \}) \\
  \freenames{x!\langle P \rangle} & := & \{ x \} \cup \{ P \} \\
  \freenames{P|Q} & := & \freenames{P} \cup \freenames{Q} \\
  \freenames{\dropn{x}} & := & \{ x \}
\end{eqnarray*}

The bound names of a process, $\boundnames{P}$, are those names occurring in $P$
that are not free. For example, in $x?(y).0$, the name $x$ is free, while $y$ is bound.

\begin{mathpar}
  \inferrule* [lab=monoidal-laws] {} { P|Q \equiv Q|P \and P|0 \equiv P \and P|(Q|R) \equiv (P|Q)|R }
\end{mathpar}

\begin{mathpar}
  \inferrule* [lab=alpha-equivalence] {} { (x)P \equiv (y)P\{y/x\} \and y \not\in \freenames{P} }
\end{mathpar}

\begin{definition}
Then two processes, $P,Q$, are alpha-equivalent if $P = Q\{\vec{y}/\vec{x}\}$ for
some $\vec{x} \in \boundnames{Q},\vec{y} \in \boundnames{P}$, where $Q\{\vec{y}/\vec{x}\}$
denotes the capture-avoiding substitution of $\vec{y}$ for $\vec{x}$ in $Q$.
\end{definition}

\begin{definition}
  The {\em structural congruence} \cite{SangiorgiWalker} , $\equiv$,
  between processes is the least congruence containing
  alpha-equivalence, satisfying the abelian monoid laws
  (associativity, commutativity and $\pzero$ as identity) for parallel
  composition $|$ and for summation $+$.
\end{definition}

\subsection{Name equivalence}

We take name equivalence, written $\nameeq$, to be the smallest
equivalence relation generated by the following rules.

\begin{mathpar}
\inferrule*[lab=Quote-drop]
{ }
{ \quotep{@{x}} \nameeq x }

\inferrule*[lab=Struct-equiv]
{ P \scong Q }
{ \quotep{P} \nameeq \quotep{Q} }
\end{mathpar}

The astute reader will have noticed that the mutual recursion of names
and processes imposes a mutual recursion on alpha-equivalence and
structural equivalence via name-equivalence. Fortunately, all of this
works out pleasantly and we may calculate in the natural way, free of
concern. The reader interested in the details is referred to the
appendix \ref{appendix:rho_details}.

\subsection{Substitution}

We use $\Proc$ for the set of processes, $\QProc$ for the set of
names, and $\id{\{}\vec{y} / \vec{x} \id{\}}$ to denote partial maps,
$s : \QProc \rightarrow \QProc$. A map, $s$ lifts, uniquely, to a map
on process terms, $\widehat{s} : \Proc \rightarrow \Proc$ by the
following equations.

\begin{mathpar}
  (0) \psubstp{Q}{P} := 0 \\
  (R \juxtap S) \psubstp{Q}{P}
  :=    
  (R)\psubstp{Q}{P} \juxtap (S) \psubstp{Q}{P} \\
  (x?(y).R) \psubstp{Q}{P}    
  :=    
  (x)\substp{Q}{P} (z)\concat( (R \psubstn{z}{y}) \psubstp{Q}{P} ) \\
  (\lift{x}{R}) \psubstp{Q}{P}  
  :=
  \lift{(x)\substp{Q}{P}}{ R \psubstp{Q}{P} } \\
%   (\dropn{x})  \psubstp{Q}{P}       
%   := 
%   \left\{ 
%     \begin{array}{ccc} 
%       \dropn{\quotep{Q}} & & x \nameeq \quotep{P} \\
%       \dropn{x} & & otherwise \\
%     \end{array}
%   \right. 
  (\dropn{x})  \psubstp{Q}{P}       
  := 
  \left\{ 
    \begin{array}{ccc} 
      Q & & x \nameeq \quotep{P} \\
      \dropn{x} & & otherwise \\
    \end{array}
  \right.
\end{mathpar}
 

where

\begin{eqnarray}
  (x)\id{\{} \lpquote Q \rpquote / \lpquote P \rpquote \id{\}}            = 
  \left\{ 
    \begin{array}{ccc}
      \lpquote Q \rpquote & & x \nameeq \lpquote P \rpquote \\
      x & & otherwise \\
    \end{array}
  \right. \nonumber
\end{eqnarray}

and $z$ is chosen distinct from $\quotep{P}$, $\quotep{Q}$, the free
names in $Q$, and all the names in $R$. Our $\alpha$-equivalence will
be built in the standard way from this substitution.

\begin{remark}\label{rem:no_self_referential_names}
  One consequence of these definitions is that $\forall P. \quotep{P}
  \not\in \freenames{P}$.
\end{remark}

\subsection{ Dynamic quote: an example }

Anticipating something of what's to come, consider applying the
substitution, $\widehat{\id{\{}u / z \id{\}}}$, to the following pair
of processes, $\lift{w}{y!(z)}$ and $w[ \lpquote y!(z) \rpquote ]$.

\begin{eqnarray}
	\lift{w}{y!(z)}\widehat{\id{\{}u / z \id{\}}}
		& = &
		\lift{w}{y!(u)} \nonumber\\
	w[ \lpquote y!(z) \rpquote ] \widehat{ \id{\{}u / z \id{\}} }
		& = &
		w[ \lpquote y!(z) \rpquote ] \nonumber
\end{eqnarray}

Because the body of the process between quotes is impervious to
substitution, we get radically different answers. In fact, by
examining the first process in an input context,
e.g. $x?(z).\lift{w}{y!(z)}$, we see that the process under the lift
operator may be shaped by prefixed inputs binding a name inside it. In
this sense, the lift operator will be seen as a way to dynamically
construct processes before reifying them as names.

Finally equipped with these standard features we can present the
dynamics of the calculus.

\subsubsection{Operational semantics} 

Finally, we introduce the computational dynamics. What marks these
algebras as distinct from other more traditionally studied algebraic
structures, e.g. vector spaces or polynomial rings, is the manner in
which dynamics is captured. In traditional structures, dynamics is typically
expressed through morphisms between such structures, as in linear maps
between vector spaces or morphisms between rings. In algebras
associated with the semantics of computation, the dynamics is
expressed as part of the algebraic structure itself, through a
reduction reduction relation typically denoted by $\red$. Below, we
give a recursive presentation of this relation for the calculus used
in the encoding.

$\red \subseteq \pi \times \pi$
$\red : \pi \to \mathcal{P}(\pi)$

\begin{mathpar}
  \inferrule* [lab=Comm] { \textsf{match}( x_{src}, x_{trgt} ) } { x_{trgt}?(y)P \; | \; x_{src}!\langle {Q} \rangle \red P\{\quotep{Q}/y}\} }
  \and \\
  \inferrule* [lab=Par] {{P} \red {P}'} {{{P} | {Q}} \red {{P}' | {Q}}}
  \and
  \inferrule* [lab=Equiv]{{{P} \scong {P}'} \andalso {{P}' \red {Q}'} \andalso {{Q}' \scong {Q}}}{{P} \red {Q}}
\end{mathpar}

\begin{eqnarray*}
  match_{\equiv} (\quotep{P},\quotep{Q}) & := & P \equiv Q \\
  match_{\dagger}(\quotep{P},\quotep{Q}) & := & \forall R. P|Q \red^{*} R => R \red^{*} 0 \\
  match_{K}(\quotep{P},\quotep{Q}) & := & K \mbox{ for some context } K
\end{eqnarray*}

$u?(x)P | u!\langle Q \rangle \red P\{\quotep{Q}/x\}$

%We write $\wred$ for $\red^*$, and $P\red$ if $\exists Q $ such that $ P \red Q$.
We write $P\red$ if $\exists Q $ such that $ P \red Q$ and $P\not\red$, otherwise.

\section{Replication}

As mentioned before, it is known that replication (and hence
recursion) can be implemented in a higher-order process algebra
\cite{SangiorgiWalker}. As our first example of calculation with the
machinery thus far presented we give the construction explicitly in
the {\rhoc}.

\begin{eqnarray}
	D_{x} & := & \prefix{x}{y}{(\binpar{\outputp{x}{y}}{@{y}})} \nonumber\\
	\bangp_{x}{P} & := & \binpar{{x}!\langle{\binpar{D_{x}}{P}}\rangle}{D_{x}} \nonumber
\end{eqnarray}

\begin{eqnarray}
	\bangp_{x}{P} & & \nonumber\\
	=
	& {x}!\langle{(\prefix{x}{y}{(\outputp{x}{y} | @{y})) | P}}\rangle 
	      | \prefix{x}{y}{(\outputp{x}{y} | @{y})} & \nonumber\\
	\red
	& (\outputp{x}{y} | @{y})\substn{\quotep{(\prefix{x}{y}{(@{y} | \outputp{x}{y})) | P}}}{y} & \nonumber\\
	=
	& \outputp{x}{\quotep{(\prefix{x}{y}{(\outputp{x}{y} | @{y})) | P}}}
	  | {(\prefix{x}{y}{(\outputp{x}{y} | @{y})) | P}} & \nonumber\\
	\red
	& \ldots & \nonumber\\
	\red^*
	& P | P | \ldots & \nonumber
\end{eqnarray}

Of course, this encoding, as an implementation, runs away, unfolding
$\bangp{P}$ eagerly. A lazier and more implementable replication
operator, restricted to input-guarded processes, may be obtained as follows.

\begin{eqnarray}
\bangp{\prefix{u}{v}{P}} 
	:= 
	\binpar{\lift{x}{\prefix{u}{v}{(\binpar{D(x)}{P})}}}{D(x)} \nonumber
\end{eqnarray}

\begin{remark}
  Note that the lazier definition still does not deal with summation
  or mixed summation (i.e. sums over input and output). The reader is
  invited to construct definitions of replication that deal with these
  features. 

  Further, the definitions are parameterized in a name, $x$. Can you,
  gentle reader, make a definition that eliminates this parameter and
  guarantees no accidental interaction between the replication
  machinery and the process being replicated -- i.e. no accidental
  sharing of names used by the process to get its work done and the
  name(s) used by the replication to effect copying. This latter
  revision of the definition of replication is crucial to obtaining
  the expected identity $!!P \sim !P$.
\end{remark}

\begin{remark}\label{rem:paradoxical_combinator}
  The reader familiar with the lambda calculus will have noticed the
  similarity between $D$ and the paradoxical combinator.

  [Ed. note: the existence of this seems to suggest we have to be more
  restrictive on the set of processes and names we admit if we are to
  support no-cloning.]
\end{remark}

\subsubsection{Bisimulation}

The computational dynamics gives rise to another kind of equivalence,
the equivalence of computational behavior. As previously mentioned
this is typically captured \emph{via} some form of bisimulation.

% The notion we use in this paper is weak barbed bisimulation
% \cite{milner91polyadicpi}.

The notion we use in this paper is derived from weak barbed
bisimulation \cite{milner91polyadicpi}. 

\begin{definition}
An \emph{observation relation}, $\downarrow_{\mathcal N}$, over a set
of names, $\mathcal N$, is the smallest relation satisfying the rules
below.

\infrule[Out-barb]{y \in {\mathcal N}, \; x \nameeq y}
		  {\outputp{x}{v} \downarrow_{\mathcal N} x}
\infrule[Par-barb]{\mbox{$P\downarrow_{\mathcal N} x$ or $Q\downarrow_{\mathcal N} x$}}
		  {\binpar{P}{Q} \downarrow_{\mathcal N} x}

We write $P \Downarrow_{\mathcal N} x$ if there is $Q$ such that 
$P \wred Q$ and $Q \downarrow_{\mathcal N} x$.
\end{definition}

\begin{definition}
%\label{def.bbisim}
An  ${\mathcal N}$-\emph{barbed bisimulation} over a set of names, ${\mathcal N}$, is a symmetric binary relation 
${\mathcal S}_{\mathcal N}$ between agents such that $P\rel{S}_{\mathcal N}Q$ implies:
\begin{enumerate}
\item If $P \red P'$ then $Q \wred Q'$ and $P'\rel{S}_{\mathcal N} Q'$.
\item If $P\downarrow_{\mathcal N} x$, then $Q\Downarrow_{\mathcal N} x$.
\end{enumerate}
$P$ is ${\mathcal N}$-barbed bisimilar to $Q$, written
$P \wbbisim_{\mathcal N} Q$, if $P \rel{S}_{\mathcal N} Q$ for some ${\mathcal N}$-barbed bisimulation ${\mathcal S}_{\mathcal N}$.
\end{definition}

$\mathcal{R} \subseteq \pi \times \pi$

$P \mathcal{R} Q => \forall P'. P \red P' \Rightarrow \exists Q'. Q \red Q', P' \mathcal{R} Q'$

$P \vdash x \Rightarrow Q \vdash x$

\begin{mathpar}
  \inferrule*[lab=Out-barb]{x \nameeq y}{{y}!\langle{Q}\rangle \vdash x}
  \and
  \inferrule*[lab=Par-barb]{\mbox{$P\vdash x$ or $Q\vdash x$}}{\binpar{P}{Q} \vdash x}
\end{mathpar}

\subsubsection{Contexts}

One of the principle advantages of computational calculi like the
$\pi$-calculus is a well-defined notion of context,
contextual-equivalence and a correlation between
contextual-equivalence and notions of bisimulation. The notion of
context allows the decomposition of a process into (sub-)process and
its syntactic environment, its context. Thus, a context may be
thought of as a process with a ``hole'' (written $\Box$) in it. The
application of a context $M$ to a process $P$, written $M[P]$, is
tantamount to filling the hole in $M$ with $P$. In this paper we do
not need the full weight of this theory, but do make use of the notion
of context in the proof the main theorem. 

\begin{mathpar}
  \inferrule* [lab=summation] {} {{M_{M},M_{N}} \bc \Box \;|\; x.M_{A} \;|\; M_{M}+M_{N}}
  \and
  \inferrule* [lab=agent] {} {{M_{A}} \bc (\vec{x})M_{P} \;| \; \clift{P_0,\ldots,M_{P},\ldots,P_N}}
  \and \\
  \inferrule* [lab=process] {} {{M_{P}} \bc M_{N} \;| \;P|M_{P} }
\end{mathpar} 

\begin{mathpar}
  \inferrule* [lab=sychronization] {} {M_{N} \bc \Box \;|\; x?M_{F} \;|\; x!M_{C}}
  \and
  \inferrule* [lab=abstraction] {} {{M_{F}} \bc (x)M_{P} }
  \and
  \inferrule* [lab=concretion] {} {{M_{C}} \bc \langle M_{P} \rangle }
  \and \\
  \inferrule* [lab=process] {} {{M_{P}} \bc M_{N} \;| \;P|M_{P} }
\end{mathpar}

\begin{definition}[contextual application] Given a context $M$, and
  process $P$, we define the \emph{contextual application}, $M[P] :=
  M\{P/\Box\}$. That is, the contextual application of M to P is the
  substitution of $P$ for $\Box$ in $M$.
\end{definition}

$\meaningof{-} : L \to \mathcal{P}(\pi)$

\begin{mathpar}
  \inferrule* [lab=collection] {} {\meaningof{true} = \pi, \and \meaningof{~E} = \pi \setminus \meaningof{E}, \and \meaningof{E_{1} \& E_{2}} = \meaningof{E_{1}} \cap \meaningof{E_{2}}}
\end{mathpar}

\begin{mathpar}
  \inferrule* [lab=structure] {} {\meaningof{0} = \{ P \in \pi | P \equiv 0 \}, \and \\ \meaningof{E_1 | E_2} = \{ P \in \pi | P \equiv P_{1} | P_{2}, P_{1} \in \meaningof{E_{1}}, P_{2} \in \meaningof{E_2}\} }
\end{mathpar}

\begin{mathpar}
 \inferrule* [lab=behavior] {} {\meaningof{\langle a?b \rangle E} = \{ P \in \pi | P \equiv Q | u?(y)P', \\ \and \\\\ \and \\ \;\;\; u \in \meaningof{a}, \forall z.P'\{z/y\} \in \meaningof{E\{z/b\}}\}, \and \\ \meaningof{a!E} = \{ P \in \pi | P \equiv Q | x!\langle P' \rangle, x \in \meaningof{a} P' \in \meaningof{E}\} }
\end{mathpar}

\begin{mathpar}
 \inferrule* [lab=nominal] {} {\meaningof{\quotep{E}} = \{ \quotep{P} \in \quotep{\pi} | P \in \meaningof{E} \}, \and \meaningof{\quotep{P}} = \{ \quotep{Q} \in \quotep{\pi} | P \equiv Q \} \and \\ \meaningof{@\quotep{E}} = \{ P \in \pi | P \equiv @x, x \in \meaningof{E} \}}
\end{mathpar}

\begin{eqnarray*}
  \\
  \meaningof{-} : TS \to ST
\end{eqnarray*}

\begin{eqnarray*}
  \\
  L : TS \to ST
\end{eqnarray*}

\begin{eqnarray*}
  \\
  P \models E \iff P \in \meaningof{E}
\end{eqnarray*}

\begin{eqnarray*}
  P \approx_{L} Q \iff \forall E \in L. P \models E \iff Q \models E
\end{eqnarray*}

\begin{eqnarray*}
  P \approx_{K} Q
\end{eqnarray*}

\begin{eqnarray*}
  P \approx Q
\end{eqnarray*}

$\approx_{K} = \approx = \approx_{L}$

\subsubsection{Contextual duality}

Note that contexts extend the quotation operation to a family of
operations from processes to names. Given a context, $M$, we can
define a \emph{nominal context}, $\quotep{M}$ by $\quotep{M}[P] :=
\quotep{M[P]}$. To foreshadow what is to come we observe that these
operations enjoy a duality with processes very much like the duality
between vectors and maps from vectors to scalars.

Further, because the calculus is essentially higher-order, we have a
correspondence between contexts and processes. More specifically,
given a name $x$ and a context $M$ we can construct $M^{*}_{x}$ such
that 

\begin{mathpar}
  M^{*}_{x} | \lift{x}{P} \red M[P]
\end{mathpar}

namely,

\begin{mathpar}
  M^{*}_{x} := x?(u).M[\dropn{u}]
\end{mathpar}

The dependence of $M^{*}_{x}$ on a name makes it an abstraction, 

\begin{mathpar}
  M^{*} := (x)x?(u).M[\dropn{u}]
\end{mathpar}

\subsection{Additional notation}

It will sometimes be convenient to denote the process a name
quotes. We already have the notation $x = \quotep{P}$, but it will be
convenient to introduce an alternate notation, $\procn{x}$, when we
want to emphasize the connection to the use of the name. Note that, by
virtue of name equivalence, $\quotep{\procn{x}} \nameeq x$; so, the
notation is consistent with previous definitions.

Further, because names have structure it is possible to effect
substitutions on the basis of that structure. This means we need to
upgrade our notation for substitutions, which we accomplish by
adapting comprehension notation. Thus,

\begin{mathpar}
  P\{ y / x : x \in S \}
\end{mathpar}

is interpreted to mean the process derived from P by replacing (in a
capture-avoiding manner) each occurrence of $x$ in $S$ by $y$. For example,

\begin{mathpar}
  P\{ \quotep{\procn{x}|\procn{x}} / x : x \in \freenames{P} \}
\end{mathpar}

will replace each (occurrence) of a free name $x$ in $P$ by
$\quotep{\procn{x}|\procn{x}}$.

Also, we will avail ourselves of the notation $x^{L}$ and $x^{R}$ to
denote injections of a name into disjoint copies of the name
space. There are numerous ways to accomplish this. One example can be
found in \cite{MeredithR05}. This notation overloads to vectors of
names: $\vec{x}^{\pi} := (x_{i}^{\pi} \; : \; 0 \leq i < |\vec{x}| )$ where $\pi \in \{L,R\}$.

We also use $P^{\Box} := P|\Box$.

In \cite{MeredithR05} an interpretation of the new operator is
given. It turns out that there are several possible interpretations
all enjoying the requisite algebraic properties of the operator (see
\cite{milner91polyadicpi}). We will therefore make liberal use of
$(\nu\; \vec{x})P$.

% subsection the_syntax_and_semantics_of_the_notation_system (end)   

\input{qm2pi.qmops} 

\input{qm2pi.sterngerlach} 

\input{qm2pi.metric} 

% section concurrent_process_calculi (end)

%\input{qm2pi.proofsketch}

% section proof sketch (end)

%\input{qm2pi.slviaknots} 

% section spatial logic via knots (end)

\input{qm2pi.conclusion}

% section conclusion (end)

%\input{qm2pi.dtcodes} 

% section wiring algorithm (end)

\input{qm2pi.ack} 

% section acknowledgments (end)

\newpage


\bibliographystyle{plain}   
\bibliography{../../biblios/main.bib}

\input{qm2pi.rhodetails}

\end{document}

 

% section acknowledgments (end)

\newpage


\bibliographystyle{plain}   
\bibliography{../../biblios/main.bib}

\documentclass[12pt]{llncs}
%\documentclass{jktr}

\usepackage[pdftex]{hyperref}                   
\usepackage {listings}
\usepackage {mathpartir}
\usepackage{bcprules}
%\usepackage{listings}
                       
\usepackage{graphicx} 
%\usepackage[margins=2.5cm,nohead,nofoot]{geometry}
%\usepackage{geometry}
\usepackage{amsfonts}
\usepackage{amstext}
\usepackage{latexsym}
\usepackage{amssymb}
\usepackage{color}


%\include{myPreamble}
\include{qm2pi.local} 

%\ifpdf
%\usepackage[pdftex]{graphicx}
%\else
%\usepackage{graphicx}
%\fi

 % \ifpdf
%  \usepackage{pdfsync}
%  \if


%\title{Brief Article}
%\author{David F. Snyder}
%\author{L.G. Meredith}

%\address{Dept. of Math., Texas State University--San Marcos, San Marcos, TX 78666}
       
\pagestyle{empty}


\begin{document}

\lstset{language=[Objective]Caml,frame=shadowbox}

\input{qm2pi.front}

% section front matter (end)

\input{qm2pi.intro} 
 
% section introduction (end)

% \input{qm2pi.knotations} 

% section notation (end)

\input{qm2pi.process.calculi} 

% section concurrent_process_calculi_and_spatial_logics_ (end)
    
%\input{qm2pi.knots2pi} 

%\input{qm2pi.trefoil} 

%\input{qm2pi.mainthm} 

% subsection basic_interpretation (end)

%\input{qm2pi.rho.presentation} 
\subsection{The syntax and semantics of the notation system}\label{sub:the_syntax_and_semantics_of_the_notation_system} % (fold)

We now summarize a technical presentation of the calculus that
embodies our theory of dynamics. The typical presentation of such a
calculus follows the style of giving generators and relations on
them. The grammar, below, describing term constructors, freely
generates the set of processes, $\Proc$. This set is then quotiented
by a relation known as structural congruence and it is over this set
that the notion of dynamics is expressed. This presentation is
essentially that of \cite{MeredithR05} with the addition of
polyadicity and summation. For readability we have relegated some of
the technical subtleties to an appendix.

\subsubsection{Process grammar}\label{subsub:process_grammar}

\begin{mathpar}
  \inferrule* [lab=synchronization] {} {{M} \bc \pzero \;|\; x?F \;|\; x!C }
  \and
  \inferrule* [lab=abstraction] {} {{F} \bc (x)P}
  \and
  \inferrule* [lab=concretion] {} {{C} \bc \langle Q \rangle}
  \and
  \inferrule* [lab=process] {} {{P,Q} \bc M \;| \;P|Q \;|\; @{x}}
  \and
  \inferrule* [lab=name] {} {{x} \bc \quotep{P}}
\end{mathpar} 

Note that $\vec{x}$ (resp. $\vec{P}$) denotes a vector of names
(resp. processes) of length $|\vec{x}|$ (resp. $|\vec{P}|$). We adopt
the following useful abbreviations.

\begin{mathpar}
   x?(\vec{y}).P := x.(\vec{y})P \and  x\clift{\vec{P}} := x.\clift{\vec{P}}
   \and x!(y) := \lift{x}{\dropn{y}}
   \and \Pi_{i=0}^{n-1}P_i := P_0 | \ldots | P_{n-1}
\end{mathpar}

\subsubsection{Structural congruence}

\paragraph{Free and bound names and alpha-equivalence.} At the
core of structural equivalence is alpha-equivalence which identifies
process that are the same up to a change of variable. Formally, we
recognize the distinction between free and bound names. The free names
of a process, $\freenames{P}$, may be calculated recursively as
follows:

\begin{mathpar}
\freenames{\pzero} := \emptyset
  \and \\
  \freenames{x?(y).P} := \{ x \} \cup (\freenames{P} \setminus \{ y \})
  \and 
  \freenames{x!\langle P \rangle} := \{ x \} \cup \{ P \} 
  \and \\
  \freenames{P|Q} := \freenames{P} \cup \freenames{Q}
  \and \\
  \freenames{@{x}} := \{ x \}
\end{mathpar}

$\pi$
$\quotep{\pi}$

$\freenames{-} : \pi \to \mathcal{P}(\quotep{\pi})$

\begin{eqnarray*}
  \freenames{\pzero} & := & \emptyset \\
  \freenames{x?(y).P} & := & \{ x \} \cup (\freenames{P} \setminus \{ y \}) \\
  \freenames{x!\langle P \rangle} & := & \{ x \} \cup \{ P \} \\
  \freenames{P|Q} & := & \freenames{P} \cup \freenames{Q} \\
  \freenames{\dropn{x}} & := & \{ x \}
\end{eqnarray*}

The bound names of a process, $\boundnames{P}$, are those names occurring in $P$
that are not free. For example, in $x?(y).0$, the name $x$ is free, while $y$ is bound.

\begin{mathpar}
  \inferrule* [lab=monoidal-laws] {} { P|Q \equiv Q|P \and P|0 \equiv P \and P|(Q|R) \equiv (P|Q)|R }
\end{mathpar}

\begin{mathpar}
  \inferrule* [lab=alpha-equivalence] {} { (x)P \equiv (y)P\{y/x\} \and y \not\in \freenames{P} }
\end{mathpar}

\begin{definition}
Then two processes, $P,Q$, are alpha-equivalent if $P = Q\{\vec{y}/\vec{x}\}$ for
some $\vec{x} \in \boundnames{Q},\vec{y} \in \boundnames{P}$, where $Q\{\vec{y}/\vec{x}\}$
denotes the capture-avoiding substitution of $\vec{y}$ for $\vec{x}$ in $Q$.
\end{definition}

\begin{definition}
  The {\em structural congruence} \cite{SangiorgiWalker} , $\equiv$,
  between processes is the least congruence containing
  alpha-equivalence, satisfying the abelian monoid laws
  (associativity, commutativity and $\pzero$ as identity) for parallel
  composition $|$ and for summation $+$.
\end{definition}

\subsection{Name equivalence}

We take name equivalence, written $\nameeq$, to be the smallest
equivalence relation generated by the following rules.

\begin{mathpar}
\inferrule*[lab=Quote-drop]
{ }
{ \quotep{@{x}} \nameeq x }

\inferrule*[lab=Struct-equiv]
{ P \scong Q }
{ \quotep{P} \nameeq \quotep{Q} }
\end{mathpar}

The astute reader will have noticed that the mutual recursion of names
and processes imposes a mutual recursion on alpha-equivalence and
structural equivalence via name-equivalence. Fortunately, all of this
works out pleasantly and we may calculate in the natural way, free of
concern. The reader interested in the details is referred to the
appendix \ref{appendix:rho_details}.

\subsection{Substitution}

We use $\Proc$ for the set of processes, $\QProc$ for the set of
names, and $\id{\{}\vec{y} / \vec{x} \id{\}}$ to denote partial maps,
$s : \QProc \rightarrow \QProc$. A map, $s$ lifts, uniquely, to a map
on process terms, $\widehat{s} : \Proc \rightarrow \Proc$ by the
following equations.

\begin{mathpar}
  (0) \psubstp{Q}{P} := 0 \\
  (R \juxtap S) \psubstp{Q}{P}
  :=    
  (R)\psubstp{Q}{P} \juxtap (S) \psubstp{Q}{P} \\
  (x?(y).R) \psubstp{Q}{P}    
  :=    
  (x)\substp{Q}{P} (z)\concat( (R \psubstn{z}{y}) \psubstp{Q}{P} ) \\
  (\lift{x}{R}) \psubstp{Q}{P}  
  :=
  \lift{(x)\substp{Q}{P}}{ R \psubstp{Q}{P} } \\
%   (\dropn{x})  \psubstp{Q}{P}       
%   := 
%   \left\{ 
%     \begin{array}{ccc} 
%       \dropn{\quotep{Q}} & & x \nameeq \quotep{P} \\
%       \dropn{x} & & otherwise \\
%     \end{array}
%   \right. 
  (\dropn{x})  \psubstp{Q}{P}       
  := 
  \left\{ 
    \begin{array}{ccc} 
      Q & & x \nameeq \quotep{P} \\
      \dropn{x} & & otherwise \\
    \end{array}
  \right.
\end{mathpar}
 

where

\begin{eqnarray}
  (x)\id{\{} \lpquote Q \rpquote / \lpquote P \rpquote \id{\}}            = 
  \left\{ 
    \begin{array}{ccc}
      \lpquote Q \rpquote & & x \nameeq \lpquote P \rpquote \\
      x & & otherwise \\
    \end{array}
  \right. \nonumber
\end{eqnarray}

and $z$ is chosen distinct from $\quotep{P}$, $\quotep{Q}$, the free
names in $Q$, and all the names in $R$. Our $\alpha$-equivalence will
be built in the standard way from this substitution.

\begin{remark}\label{rem:no_self_referential_names}
  One consequence of these definitions is that $\forall P. \quotep{P}
  \not\in \freenames{P}$.
\end{remark}

\subsection{ Dynamic quote: an example }

Anticipating something of what's to come, consider applying the
substitution, $\widehat{\id{\{}u / z \id{\}}}$, to the following pair
of processes, $\lift{w}{y!(z)}$ and $w[ \lpquote y!(z) \rpquote ]$.

\begin{eqnarray}
	\lift{w}{y!(z)}\widehat{\id{\{}u / z \id{\}}}
		& = &
		\lift{w}{y!(u)} \nonumber\\
	w[ \lpquote y!(z) \rpquote ] \widehat{ \id{\{}u / z \id{\}} }
		& = &
		w[ \lpquote y!(z) \rpquote ] \nonumber
\end{eqnarray}

Because the body of the process between quotes is impervious to
substitution, we get radically different answers. In fact, by
examining the first process in an input context,
e.g. $x?(z).\lift{w}{y!(z)}$, we see that the process under the lift
operator may be shaped by prefixed inputs binding a name inside it. In
this sense, the lift operator will be seen as a way to dynamically
construct processes before reifying them as names.

Finally equipped with these standard features we can present the
dynamics of the calculus.

\subsubsection{Operational semantics} 

Finally, we introduce the computational dynamics. What marks these
algebras as distinct from other more traditionally studied algebraic
structures, e.g. vector spaces or polynomial rings, is the manner in
which dynamics is captured. In traditional structures, dynamics is typically
expressed through morphisms between such structures, as in linear maps
between vector spaces or morphisms between rings. In algebras
associated with the semantics of computation, the dynamics is
expressed as part of the algebraic structure itself, through a
reduction reduction relation typically denoted by $\red$. Below, we
give a recursive presentation of this relation for the calculus used
in the encoding.

$\red \subseteq \pi \times \pi$
$\red : \pi \to \mathcal{P}(\pi)$

\begin{mathpar}
  \inferrule* [lab=Comm] { \textsf{match}( x_{src}, x_{trgt} ) } { x_{trgt}?(y)P \; | \; x_{src}!\langle {Q} \rangle \red P\{\quotep{Q}/y}\} }
  \and \\
  \inferrule* [lab=Par] {{P} \red {P}'} {{{P} | {Q}} \red {{P}' | {Q}}}
  \and
  \inferrule* [lab=Equiv]{{{P} \scong {P}'} \andalso {{P}' \red {Q}'} \andalso {{Q}' \scong {Q}}}{{P} \red {Q}}
\end{mathpar}

\begin{eqnarray*}
  match_{\equiv} (\quotep{P},\quotep{Q}) & := & P \equiv Q \\
  match_{\dagger}(\quotep{P},\quotep{Q}) & := & \forall R. P|Q \red^{*} R => R \red^{*} 0 \\
  match_{K}(\quotep{P},\quotep{Q}) & := & K \mbox{ for some context } K
\end{eqnarray*}

$u?(x)P | u!\langle Q \rangle \red P\{\quotep{Q}/x\}$

%We write $\wred$ for $\red^*$, and $P\red$ if $\exists Q $ such that $ P \red Q$.
We write $P\red$ if $\exists Q $ such that $ P \red Q$ and $P\not\red$, otherwise.

\section{Replication}

As mentioned before, it is known that replication (and hence
recursion) can be implemented in a higher-order process algebra
\cite{SangiorgiWalker}. As our first example of calculation with the
machinery thus far presented we give the construction explicitly in
the {\rhoc}.

\begin{eqnarray}
	D_{x} & := & \prefix{x}{y}{(\binpar{\outputp{x}{y}}{@{y}})} \nonumber\\
	\bangp_{x}{P} & := & \binpar{{x}!\langle{\binpar{D_{x}}{P}}\rangle}{D_{x}} \nonumber
\end{eqnarray}

\begin{eqnarray}
	\bangp_{x}{P} & & \nonumber\\
	=
	& {x}!\langle{(\prefix{x}{y}{(\outputp{x}{y} | @{y})) | P}}\rangle 
	      | \prefix{x}{y}{(\outputp{x}{y} | @{y})} & \nonumber\\
	\red
	& (\outputp{x}{y} | @{y})\substn{\quotep{(\prefix{x}{y}{(@{y} | \outputp{x}{y})) | P}}}{y} & \nonumber\\
	=
	& \outputp{x}{\quotep{(\prefix{x}{y}{(\outputp{x}{y} | @{y})) | P}}}
	  | {(\prefix{x}{y}{(\outputp{x}{y} | @{y})) | P}} & \nonumber\\
	\red
	& \ldots & \nonumber\\
	\red^*
	& P | P | \ldots & \nonumber
\end{eqnarray}

Of course, this encoding, as an implementation, runs away, unfolding
$\bangp{P}$ eagerly. A lazier and more implementable replication
operator, restricted to input-guarded processes, may be obtained as follows.

\begin{eqnarray}
\bangp{\prefix{u}{v}{P}} 
	:= 
	\binpar{\lift{x}{\prefix{u}{v}{(\binpar{D(x)}{P})}}}{D(x)} \nonumber
\end{eqnarray}

\begin{remark}
  Note that the lazier definition still does not deal with summation
  or mixed summation (i.e. sums over input and output). The reader is
  invited to construct definitions of replication that deal with these
  features. 

  Further, the definitions are parameterized in a name, $x$. Can you,
  gentle reader, make a definition that eliminates this parameter and
  guarantees no accidental interaction between the replication
  machinery and the process being replicated -- i.e. no accidental
  sharing of names used by the process to get its work done and the
  name(s) used by the replication to effect copying. This latter
  revision of the definition of replication is crucial to obtaining
  the expected identity $!!P \sim !P$.
\end{remark}

\begin{remark}\label{rem:paradoxical_combinator}
  The reader familiar with the lambda calculus will have noticed the
  similarity between $D$ and the paradoxical combinator.

  [Ed. note: the existence of this seems to suggest we have to be more
  restrictive on the set of processes and names we admit if we are to
  support no-cloning.]
\end{remark}

\subsubsection{Bisimulation}

The computational dynamics gives rise to another kind of equivalence,
the equivalence of computational behavior. As previously mentioned
this is typically captured \emph{via} some form of bisimulation.

% The notion we use in this paper is weak barbed bisimulation
% \cite{milner91polyadicpi}.

The notion we use in this paper is derived from weak barbed
bisimulation \cite{milner91polyadicpi}. 

\begin{definition}
An \emph{observation relation}, $\downarrow_{\mathcal N}$, over a set
of names, $\mathcal N$, is the smallest relation satisfying the rules
below.

\infrule[Out-barb]{y \in {\mathcal N}, \; x \nameeq y}
		  {\outputp{x}{v} \downarrow_{\mathcal N} x}
\infrule[Par-barb]{\mbox{$P\downarrow_{\mathcal N} x$ or $Q\downarrow_{\mathcal N} x$}}
		  {\binpar{P}{Q} \downarrow_{\mathcal N} x}

We write $P \Downarrow_{\mathcal N} x$ if there is $Q$ such that 
$P \wred Q$ and $Q \downarrow_{\mathcal N} x$.
\end{definition}

\begin{definition}
%\label{def.bbisim}
An  ${\mathcal N}$-\emph{barbed bisimulation} over a set of names, ${\mathcal N}$, is a symmetric binary relation 
${\mathcal S}_{\mathcal N}$ between agents such that $P\rel{S}_{\mathcal N}Q$ implies:
\begin{enumerate}
\item If $P \red P'$ then $Q \wred Q'$ and $P'\rel{S}_{\mathcal N} Q'$.
\item If $P\downarrow_{\mathcal N} x$, then $Q\Downarrow_{\mathcal N} x$.
\end{enumerate}
$P$ is ${\mathcal N}$-barbed bisimilar to $Q$, written
$P \wbbisim_{\mathcal N} Q$, if $P \rel{S}_{\mathcal N} Q$ for some ${\mathcal N}$-barbed bisimulation ${\mathcal S}_{\mathcal N}$.
\end{definition}

$\mathcal{R} \subseteq \pi \times \pi$

$P \mathcal{R} Q => \forall P'. P \red P' \Rightarrow \exists Q'. Q \red Q', P' \mathcal{R} Q'$

$P \vdash x \Rightarrow Q \vdash x$

\begin{mathpar}
  \inferrule*[lab=Out-barb]{x \nameeq y}{{y}!\langle{Q}\rangle \vdash x}
  \and
  \inferrule*[lab=Par-barb]{\mbox{$P\vdash x$ or $Q\vdash x$}}{\binpar{P}{Q} \vdash x}
\end{mathpar}

\subsubsection{Contexts}

One of the principle advantages of computational calculi like the
$\pi$-calculus is a well-defined notion of context,
contextual-equivalence and a correlation between
contextual-equivalence and notions of bisimulation. The notion of
context allows the decomposition of a process into (sub-)process and
its syntactic environment, its context. Thus, a context may be
thought of as a process with a ``hole'' (written $\Box$) in it. The
application of a context $M$ to a process $P$, written $M[P]$, is
tantamount to filling the hole in $M$ with $P$. In this paper we do
not need the full weight of this theory, but do make use of the notion
of context in the proof the main theorem. 

\begin{mathpar}
  \inferrule* [lab=summation] {} {{M_{M},M_{N}} \bc \Box \;|\; x.M_{A} \;|\; M_{M}+M_{N}}
  \and
  \inferrule* [lab=agent] {} {{M_{A}} \bc (\vec{x})M_{P} \;| \; \clift{P_0,\ldots,M_{P},\ldots,P_N}}
  \and \\
  \inferrule* [lab=process] {} {{M_{P}} \bc M_{N} \;| \;P|M_{P} }
\end{mathpar} 

\begin{mathpar}
  \inferrule* [lab=sychronization] {} {M_{N} \bc \Box \;|\; x?M_{F} \;|\; x!M_{C}}
  \and
  \inferrule* [lab=abstraction] {} {{M_{F}} \bc (x)M_{P} }
  \and
  \inferrule* [lab=concretion] {} {{M_{C}} \bc \langle M_{P} \rangle }
  \and \\
  \inferrule* [lab=process] {} {{M_{P}} \bc M_{N} \;| \;P|M_{P} }
\end{mathpar}

\begin{definition}[contextual application] Given a context $M$, and
  process $P$, we define the \emph{contextual application}, $M[P] :=
  M\{P/\Box\}$. That is, the contextual application of M to P is the
  substitution of $P$ for $\Box$ in $M$.
\end{definition}

$\meaningof{-} : L \to \mathcal{P}(\pi)$

\begin{mathpar}
  \inferrule* [lab=collection] {} {\meaningof{true} = \pi, \and \meaningof{~E} = \pi \setminus \meaningof{E}, \and \meaningof{E_{1} \& E_{2}} = \meaningof{E_{1}} \cap \meaningof{E_{2}}}
\end{mathpar}

\begin{mathpar}
  \inferrule* [lab=structure] {} {\meaningof{0} = \{ P \in \pi | P \equiv 0 \}, \and \\ \meaningof{E_1 | E_2} = \{ P \in \pi | P \equiv P_{1} | P_{2}, P_{1} \in \meaningof{E_{1}}, P_{2} \in \meaningof{E_2}\} }
\end{mathpar}

\begin{mathpar}
 \inferrule* [lab=behavior] {} {\meaningof{\langle a?b \rangle E} = \{ P \in \pi | P \equiv Q | u?(y)P', \\ \and \\\\ \and \\ \;\;\; u \in \meaningof{a}, \forall z.P'\{z/y\} \in \meaningof{E\{z/b\}}\}, \and \\ \meaningof{a!E} = \{ P \in \pi | P \equiv Q | x!\langle P' \rangle, x \in \meaningof{a} P' \in \meaningof{E}\} }
\end{mathpar}

\begin{mathpar}
 \inferrule* [lab=nominal] {} {\meaningof{\quotep{E}} = \{ \quotep{P} \in \quotep{\pi} | P \in \meaningof{E} \}, \and \meaningof{\quotep{P}} = \{ \quotep{Q} \in \quotep{\pi} | P \equiv Q \} \and \\ \meaningof{@\quotep{E}} = \{ P \in \pi | P \equiv @x, x \in \meaningof{E} \}}
\end{mathpar}

\begin{eqnarray*}
  \\
  \meaningof{-} : TS \to ST
\end{eqnarray*}

\begin{eqnarray*}
  \\
  L : TS \to ST
\end{eqnarray*}

\begin{eqnarray*}
  \\
  P \models E \iff P \in \meaningof{E}
\end{eqnarray*}

\begin{eqnarray*}
  P \approx_{L} Q \iff \forall E \in L. P \models E \iff Q \models E
\end{eqnarray*}

\begin{eqnarray*}
  P \approx_{K} Q
\end{eqnarray*}

\begin{eqnarray*}
  P \approx Q
\end{eqnarray*}

$\approx_{K} = \approx = \approx_{L}$

\subsubsection{Contextual duality}

Note that contexts extend the quotation operation to a family of
operations from processes to names. Given a context, $M$, we can
define a \emph{nominal context}, $\quotep{M}$ by $\quotep{M}[P] :=
\quotep{M[P]}$. To foreshadow what is to come we observe that these
operations enjoy a duality with processes very much like the duality
between vectors and maps from vectors to scalars.

Further, because the calculus is essentially higher-order, we have a
correspondence between contexts and processes. More specifically,
given a name $x$ and a context $M$ we can construct $M^{*}_{x}$ such
that 

\begin{mathpar}
  M^{*}_{x} | \lift{x}{P} \red M[P]
\end{mathpar}

namely,

\begin{mathpar}
  M^{*}_{x} := x?(u).M[\dropn{u}]
\end{mathpar}

The dependence of $M^{*}_{x}$ on a name makes it an abstraction, 

\begin{mathpar}
  M^{*} := (x)x?(u).M[\dropn{u}]
\end{mathpar}

\subsection{Additional notation}

It will sometimes be convenient to denote the process a name
quotes. We already have the notation $x = \quotep{P}$, but it will be
convenient to introduce an alternate notation, $\procn{x}$, when we
want to emphasize the connection to the use of the name. Note that, by
virtue of name equivalence, $\quotep{\procn{x}} \nameeq x$; so, the
notation is consistent with previous definitions.

Further, because names have structure it is possible to effect
substitutions on the basis of that structure. This means we need to
upgrade our notation for substitutions, which we accomplish by
adapting comprehension notation. Thus,

\begin{mathpar}
  P\{ y / x : x \in S \}
\end{mathpar}

is interpreted to mean the process derived from P by replacing (in a
capture-avoiding manner) each occurrence of $x$ in $S$ by $y$. For example,

\begin{mathpar}
  P\{ \quotep{\procn{x}|\procn{x}} / x : x \in \freenames{P} \}
\end{mathpar}

will replace each (occurrence) of a free name $x$ in $P$ by
$\quotep{\procn{x}|\procn{x}}$.

Also, we will avail ourselves of the notation $x^{L}$ and $x^{R}$ to
denote injections of a name into disjoint copies of the name
space. There are numerous ways to accomplish this. One example can be
found in \cite{MeredithR05}. This notation overloads to vectors of
names: $\vec{x}^{\pi} := (x_{i}^{\pi} \; : \; 0 \leq i < |\vec{x}| )$ where $\pi \in \{L,R\}$.

We also use $P^{\Box} := P|\Box$.

In \cite{MeredithR05} an interpretation of the new operator is
given. It turns out that there are several possible interpretations
all enjoying the requisite algebraic properties of the operator (see
\cite{milner91polyadicpi}). We will therefore make liberal use of
$(\nu\; \vec{x})P$.

% subsection the_syntax_and_semantics_of_the_notation_system (end)   

\input{qm2pi.qmops} 

\input{qm2pi.sterngerlach} 

\input{qm2pi.metric} 

% section concurrent_process_calculi (end)

%\input{qm2pi.proofsketch}

% section proof sketch (end)

%\input{qm2pi.slviaknots} 

% section spatial logic via knots (end)

\input{qm2pi.conclusion}

% section conclusion (end)

%\input{qm2pi.dtcodes} 

% section wiring algorithm (end)

\input{qm2pi.ack} 

% section acknowledgments (end)

\newpage


\bibliographystyle{plain}   
\bibliography{../../biblios/main.bib}

\input{qm2pi.rhodetails}

\end{document}



\end{document}

 

%\documentclass[12pt]{llncs}
%\documentclass{jktr}

\usepackage[pdftex]{hyperref}                   
\usepackage {listings}
\usepackage {mathpartir}
\usepackage{bcprules}
%\usepackage{listings}
                       
\usepackage{graphicx} 
%\usepackage[margins=2.5cm,nohead,nofoot]{geometry}
%\usepackage{geometry}
\usepackage{amsfonts}
\usepackage{amstext}
\usepackage{latexsym}
\usepackage{amssymb}
\usepackage{color}


%\include{myPreamble}
\documentclass[12pt]{llncs}
%\documentclass{jktr}

\usepackage[pdftex]{hyperref}                   
\usepackage {listings}
\usepackage {mathpartir}
\usepackage{bcprules}
%\usepackage{listings}
                       
\usepackage{graphicx} 
%\usepackage[margins=2.5cm,nohead,nofoot]{geometry}
%\usepackage{geometry}
\usepackage{amsfonts}
\usepackage{amstext}
\usepackage{latexsym}
\usepackage{amssymb}
\usepackage{color}


%\include{myPreamble}
\include{qm2pi.local} 

%\ifpdf
%\usepackage[pdftex]{graphicx}
%\else
%\usepackage{graphicx}
%\fi

 % \ifpdf
%  \usepackage{pdfsync}
%  \if


%\title{Brief Article}
%\author{David F. Snyder}
%\author{L.G. Meredith}

%\address{Dept. of Math., Texas State University--San Marcos, San Marcos, TX 78666}
       
\pagestyle{empty}


\begin{document}

\lstset{language=[Objective]Caml,frame=shadowbox}

\input{qm2pi.front}

% section front matter (end)

\input{qm2pi.intro} 
 
% section introduction (end)

% \input{qm2pi.knotations} 

% section notation (end)

\input{qm2pi.process.calculi} 

% section concurrent_process_calculi_and_spatial_logics_ (end)
    
%\input{qm2pi.knots2pi} 

%\input{qm2pi.trefoil} 

%\input{qm2pi.mainthm} 

% subsection basic_interpretation (end)

%\input{qm2pi.rho.presentation} 
\subsection{The syntax and semantics of the notation system}\label{sub:the_syntax_and_semantics_of_the_notation_system} % (fold)

We now summarize a technical presentation of the calculus that
embodies our theory of dynamics. The typical presentation of such a
calculus follows the style of giving generators and relations on
them. The grammar, below, describing term constructors, freely
generates the set of processes, $\Proc$. This set is then quotiented
by a relation known as structural congruence and it is over this set
that the notion of dynamics is expressed. This presentation is
essentially that of \cite{MeredithR05} with the addition of
polyadicity and summation. For readability we have relegated some of
the technical subtleties to an appendix.

\subsubsection{Process grammar}\label{subsub:process_grammar}

\begin{mathpar}
  \inferrule* [lab=synchronization] {} {{M} \bc \pzero \;|\; x?F \;|\; x!C }
  \and
  \inferrule* [lab=abstraction] {} {{F} \bc (x)P}
  \and
  \inferrule* [lab=concretion] {} {{C} \bc \langle Q \rangle}
  \and
  \inferrule* [lab=process] {} {{P,Q} \bc M \;| \;P|Q \;|\; @{x}}
  \and
  \inferrule* [lab=name] {} {{x} \bc \quotep{P}}
\end{mathpar} 

Note that $\vec{x}$ (resp. $\vec{P}$) denotes a vector of names
(resp. processes) of length $|\vec{x}|$ (resp. $|\vec{P}|$). We adopt
the following useful abbreviations.

\begin{mathpar}
   x?(\vec{y}).P := x.(\vec{y})P \and  x\clift{\vec{P}} := x.\clift{\vec{P}}
   \and x!(y) := \lift{x}{\dropn{y}}
   \and \Pi_{i=0}^{n-1}P_i := P_0 | \ldots | P_{n-1}
\end{mathpar}

\subsubsection{Structural congruence}

\paragraph{Free and bound names and alpha-equivalence.} At the
core of structural equivalence is alpha-equivalence which identifies
process that are the same up to a change of variable. Formally, we
recognize the distinction between free and bound names. The free names
of a process, $\freenames{P}$, may be calculated recursively as
follows:

\begin{mathpar}
\freenames{\pzero} := \emptyset
  \and \\
  \freenames{x?(y).P} := \{ x \} \cup (\freenames{P} \setminus \{ y \})
  \and 
  \freenames{x!\langle P \rangle} := \{ x \} \cup \{ P \} 
  \and \\
  \freenames{P|Q} := \freenames{P} \cup \freenames{Q}
  \and \\
  \freenames{@{x}} := \{ x \}
\end{mathpar}

$\pi$
$\quotep{\pi}$

$\freenames{-} : \pi \to \mathcal{P}(\quotep{\pi})$

\begin{eqnarray*}
  \freenames{\pzero} & := & \emptyset \\
  \freenames{x?(y).P} & := & \{ x \} \cup (\freenames{P} \setminus \{ y \}) \\
  \freenames{x!\langle P \rangle} & := & \{ x \} \cup \{ P \} \\
  \freenames{P|Q} & := & \freenames{P} \cup \freenames{Q} \\
  \freenames{\dropn{x}} & := & \{ x \}
\end{eqnarray*}

The bound names of a process, $\boundnames{P}$, are those names occurring in $P$
that are not free. For example, in $x?(y).0$, the name $x$ is free, while $y$ is bound.

\begin{mathpar}
  \inferrule* [lab=monoidal-laws] {} { P|Q \equiv Q|P \and P|0 \equiv P \and P|(Q|R) \equiv (P|Q)|R }
\end{mathpar}

\begin{mathpar}
  \inferrule* [lab=alpha-equivalence] {} { (x)P \equiv (y)P\{y/x\} \and y \not\in \freenames{P} }
\end{mathpar}

\begin{definition}
Then two processes, $P,Q$, are alpha-equivalent if $P = Q\{\vec{y}/\vec{x}\}$ for
some $\vec{x} \in \boundnames{Q},\vec{y} \in \boundnames{P}$, where $Q\{\vec{y}/\vec{x}\}$
denotes the capture-avoiding substitution of $\vec{y}$ for $\vec{x}$ in $Q$.
\end{definition}

\begin{definition}
  The {\em structural congruence} \cite{SangiorgiWalker} , $\equiv$,
  between processes is the least congruence containing
  alpha-equivalence, satisfying the abelian monoid laws
  (associativity, commutativity and $\pzero$ as identity) for parallel
  composition $|$ and for summation $+$.
\end{definition}

\subsection{Name equivalence}

We take name equivalence, written $\nameeq$, to be the smallest
equivalence relation generated by the following rules.

\begin{mathpar}
\inferrule*[lab=Quote-drop]
{ }
{ \quotep{@{x}} \nameeq x }

\inferrule*[lab=Struct-equiv]
{ P \scong Q }
{ \quotep{P} \nameeq \quotep{Q} }
\end{mathpar}

The astute reader will have noticed that the mutual recursion of names
and processes imposes a mutual recursion on alpha-equivalence and
structural equivalence via name-equivalence. Fortunately, all of this
works out pleasantly and we may calculate in the natural way, free of
concern. The reader interested in the details is referred to the
appendix \ref{appendix:rho_details}.

\subsection{Substitution}

We use $\Proc$ for the set of processes, $\QProc$ for the set of
names, and $\id{\{}\vec{y} / \vec{x} \id{\}}$ to denote partial maps,
$s : \QProc \rightarrow \QProc$. A map, $s$ lifts, uniquely, to a map
on process terms, $\widehat{s} : \Proc \rightarrow \Proc$ by the
following equations.

\begin{mathpar}
  (0) \psubstp{Q}{P} := 0 \\
  (R \juxtap S) \psubstp{Q}{P}
  :=    
  (R)\psubstp{Q}{P} \juxtap (S) \psubstp{Q}{P} \\
  (x?(y).R) \psubstp{Q}{P}    
  :=    
  (x)\substp{Q}{P} (z)\concat( (R \psubstn{z}{y}) \psubstp{Q}{P} ) \\
  (\lift{x}{R}) \psubstp{Q}{P}  
  :=
  \lift{(x)\substp{Q}{P}}{ R \psubstp{Q}{P} } \\
%   (\dropn{x})  \psubstp{Q}{P}       
%   := 
%   \left\{ 
%     \begin{array}{ccc} 
%       \dropn{\quotep{Q}} & & x \nameeq \quotep{P} \\
%       \dropn{x} & & otherwise \\
%     \end{array}
%   \right. 
  (\dropn{x})  \psubstp{Q}{P}       
  := 
  \left\{ 
    \begin{array}{ccc} 
      Q & & x \nameeq \quotep{P} \\
      \dropn{x} & & otherwise \\
    \end{array}
  \right.
\end{mathpar}
 

where

\begin{eqnarray}
  (x)\id{\{} \lpquote Q \rpquote / \lpquote P \rpquote \id{\}}            = 
  \left\{ 
    \begin{array}{ccc}
      \lpquote Q \rpquote & & x \nameeq \lpquote P \rpquote \\
      x & & otherwise \\
    \end{array}
  \right. \nonumber
\end{eqnarray}

and $z$ is chosen distinct from $\quotep{P}$, $\quotep{Q}$, the free
names in $Q$, and all the names in $R$. Our $\alpha$-equivalence will
be built in the standard way from this substitution.

\begin{remark}\label{rem:no_self_referential_names}
  One consequence of these definitions is that $\forall P. \quotep{P}
  \not\in \freenames{P}$.
\end{remark}

\subsection{ Dynamic quote: an example }

Anticipating something of what's to come, consider applying the
substitution, $\widehat{\id{\{}u / z \id{\}}}$, to the following pair
of processes, $\lift{w}{y!(z)}$ and $w[ \lpquote y!(z) \rpquote ]$.

\begin{eqnarray}
	\lift{w}{y!(z)}\widehat{\id{\{}u / z \id{\}}}
		& = &
		\lift{w}{y!(u)} \nonumber\\
	w[ \lpquote y!(z) \rpquote ] \widehat{ \id{\{}u / z \id{\}} }
		& = &
		w[ \lpquote y!(z) \rpquote ] \nonumber
\end{eqnarray}

Because the body of the process between quotes is impervious to
substitution, we get radically different answers. In fact, by
examining the first process in an input context,
e.g. $x?(z).\lift{w}{y!(z)}$, we see that the process under the lift
operator may be shaped by prefixed inputs binding a name inside it. In
this sense, the lift operator will be seen as a way to dynamically
construct processes before reifying them as names.

Finally equipped with these standard features we can present the
dynamics of the calculus.

\subsubsection{Operational semantics} 

Finally, we introduce the computational dynamics. What marks these
algebras as distinct from other more traditionally studied algebraic
structures, e.g. vector spaces or polynomial rings, is the manner in
which dynamics is captured. In traditional structures, dynamics is typically
expressed through morphisms between such structures, as in linear maps
between vector spaces or morphisms between rings. In algebras
associated with the semantics of computation, the dynamics is
expressed as part of the algebraic structure itself, through a
reduction reduction relation typically denoted by $\red$. Below, we
give a recursive presentation of this relation for the calculus used
in the encoding.

$\red \subseteq \pi \times \pi$
$\red : \pi \to \mathcal{P}(\pi)$

\begin{mathpar}
  \inferrule* [lab=Comm] { \textsf{match}( x_{src}, x_{trgt} ) } { x_{trgt}?(y)P \; | \; x_{src}!\langle {Q} \rangle \red P\{\quotep{Q}/y}\} }
  \and \\
  \inferrule* [lab=Par] {{P} \red {P}'} {{{P} | {Q}} \red {{P}' | {Q}}}
  \and
  \inferrule* [lab=Equiv]{{{P} \scong {P}'} \andalso {{P}' \red {Q}'} \andalso {{Q}' \scong {Q}}}{{P} \red {Q}}
\end{mathpar}

\begin{eqnarray*}
  match_{\equiv} (\quotep{P},\quotep{Q}) & := & P \equiv Q \\
  match_{\dagger}(\quotep{P},\quotep{Q}) & := & \forall R. P|Q \red^{*} R => R \red^{*} 0 \\
  match_{K}(\quotep{P},\quotep{Q}) & := & K \mbox{ for some context } K
\end{eqnarray*}

$u?(x)P | u!\langle Q \rangle \red P\{\quotep{Q}/x\}$

%We write $\wred$ for $\red^*$, and $P\red$ if $\exists Q $ such that $ P \red Q$.
We write $P\red$ if $\exists Q $ such that $ P \red Q$ and $P\not\red$, otherwise.

\section{Replication}

As mentioned before, it is known that replication (and hence
recursion) can be implemented in a higher-order process algebra
\cite{SangiorgiWalker}. As our first example of calculation with the
machinery thus far presented we give the construction explicitly in
the {\rhoc}.

\begin{eqnarray}
	D_{x} & := & \prefix{x}{y}{(\binpar{\outputp{x}{y}}{@{y}})} \nonumber\\
	\bangp_{x}{P} & := & \binpar{{x}!\langle{\binpar{D_{x}}{P}}\rangle}{D_{x}} \nonumber
\end{eqnarray}

\begin{eqnarray}
	\bangp_{x}{P} & & \nonumber\\
	=
	& {x}!\langle{(\prefix{x}{y}{(\outputp{x}{y} | @{y})) | P}}\rangle 
	      | \prefix{x}{y}{(\outputp{x}{y} | @{y})} & \nonumber\\
	\red
	& (\outputp{x}{y} | @{y})\substn{\quotep{(\prefix{x}{y}{(@{y} | \outputp{x}{y})) | P}}}{y} & \nonumber\\
	=
	& \outputp{x}{\quotep{(\prefix{x}{y}{(\outputp{x}{y} | @{y})) | P}}}
	  | {(\prefix{x}{y}{(\outputp{x}{y} | @{y})) | P}} & \nonumber\\
	\red
	& \ldots & \nonumber\\
	\red^*
	& P | P | \ldots & \nonumber
\end{eqnarray}

Of course, this encoding, as an implementation, runs away, unfolding
$\bangp{P}$ eagerly. A lazier and more implementable replication
operator, restricted to input-guarded processes, may be obtained as follows.

\begin{eqnarray}
\bangp{\prefix{u}{v}{P}} 
	:= 
	\binpar{\lift{x}{\prefix{u}{v}{(\binpar{D(x)}{P})}}}{D(x)} \nonumber
\end{eqnarray}

\begin{remark}
  Note that the lazier definition still does not deal with summation
  or mixed summation (i.e. sums over input and output). The reader is
  invited to construct definitions of replication that deal with these
  features. 

  Further, the definitions are parameterized in a name, $x$. Can you,
  gentle reader, make a definition that eliminates this parameter and
  guarantees no accidental interaction between the replication
  machinery and the process being replicated -- i.e. no accidental
  sharing of names used by the process to get its work done and the
  name(s) used by the replication to effect copying. This latter
  revision of the definition of replication is crucial to obtaining
  the expected identity $!!P \sim !P$.
\end{remark}

\begin{remark}\label{rem:paradoxical_combinator}
  The reader familiar with the lambda calculus will have noticed the
  similarity between $D$ and the paradoxical combinator.

  [Ed. note: the existence of this seems to suggest we have to be more
  restrictive on the set of processes and names we admit if we are to
  support no-cloning.]
\end{remark}

\subsubsection{Bisimulation}

The computational dynamics gives rise to another kind of equivalence,
the equivalence of computational behavior. As previously mentioned
this is typically captured \emph{via} some form of bisimulation.

% The notion we use in this paper is weak barbed bisimulation
% \cite{milner91polyadicpi}.

The notion we use in this paper is derived from weak barbed
bisimulation \cite{milner91polyadicpi}. 

\begin{definition}
An \emph{observation relation}, $\downarrow_{\mathcal N}$, over a set
of names, $\mathcal N$, is the smallest relation satisfying the rules
below.

\infrule[Out-barb]{y \in {\mathcal N}, \; x \nameeq y}
		  {\outputp{x}{v} \downarrow_{\mathcal N} x}
\infrule[Par-barb]{\mbox{$P\downarrow_{\mathcal N} x$ or $Q\downarrow_{\mathcal N} x$}}
		  {\binpar{P}{Q} \downarrow_{\mathcal N} x}

We write $P \Downarrow_{\mathcal N} x$ if there is $Q$ such that 
$P \wred Q$ and $Q \downarrow_{\mathcal N} x$.
\end{definition}

\begin{definition}
%\label{def.bbisim}
An  ${\mathcal N}$-\emph{barbed bisimulation} over a set of names, ${\mathcal N}$, is a symmetric binary relation 
${\mathcal S}_{\mathcal N}$ between agents such that $P\rel{S}_{\mathcal N}Q$ implies:
\begin{enumerate}
\item If $P \red P'$ then $Q \wred Q'$ and $P'\rel{S}_{\mathcal N} Q'$.
\item If $P\downarrow_{\mathcal N} x$, then $Q\Downarrow_{\mathcal N} x$.
\end{enumerate}
$P$ is ${\mathcal N}$-barbed bisimilar to $Q$, written
$P \wbbisim_{\mathcal N} Q$, if $P \rel{S}_{\mathcal N} Q$ for some ${\mathcal N}$-barbed bisimulation ${\mathcal S}_{\mathcal N}$.
\end{definition}

$\mathcal{R} \subseteq \pi \times \pi$

$P \mathcal{R} Q => \forall P'. P \red P' \Rightarrow \exists Q'. Q \red Q', P' \mathcal{R} Q'$

$P \vdash x \Rightarrow Q \vdash x$

\begin{mathpar}
  \inferrule*[lab=Out-barb]{x \nameeq y}{{y}!\langle{Q}\rangle \vdash x}
  \and
  \inferrule*[lab=Par-barb]{\mbox{$P\vdash x$ or $Q\vdash x$}}{\binpar{P}{Q} \vdash x}
\end{mathpar}

\subsubsection{Contexts}

One of the principle advantages of computational calculi like the
$\pi$-calculus is a well-defined notion of context,
contextual-equivalence and a correlation between
contextual-equivalence and notions of bisimulation. The notion of
context allows the decomposition of a process into (sub-)process and
its syntactic environment, its context. Thus, a context may be
thought of as a process with a ``hole'' (written $\Box$) in it. The
application of a context $M$ to a process $P$, written $M[P]$, is
tantamount to filling the hole in $M$ with $P$. In this paper we do
not need the full weight of this theory, but do make use of the notion
of context in the proof the main theorem. 

\begin{mathpar}
  \inferrule* [lab=summation] {} {{M_{M},M_{N}} \bc \Box \;|\; x.M_{A} \;|\; M_{M}+M_{N}}
  \and
  \inferrule* [lab=agent] {} {{M_{A}} \bc (\vec{x})M_{P} \;| \; \clift{P_0,\ldots,M_{P},\ldots,P_N}}
  \and \\
  \inferrule* [lab=process] {} {{M_{P}} \bc M_{N} \;| \;P|M_{P} }
\end{mathpar} 

\begin{mathpar}
  \inferrule* [lab=sychronization] {} {M_{N} \bc \Box \;|\; x?M_{F} \;|\; x!M_{C}}
  \and
  \inferrule* [lab=abstraction] {} {{M_{F}} \bc (x)M_{P} }
  \and
  \inferrule* [lab=concretion] {} {{M_{C}} \bc \langle M_{P} \rangle }
  \and \\
  \inferrule* [lab=process] {} {{M_{P}} \bc M_{N} \;| \;P|M_{P} }
\end{mathpar}

\begin{definition}[contextual application] Given a context $M$, and
  process $P$, we define the \emph{contextual application}, $M[P] :=
  M\{P/\Box\}$. That is, the contextual application of M to P is the
  substitution of $P$ for $\Box$ in $M$.
\end{definition}

$\meaningof{-} : L \to \mathcal{P}(\pi)$

\begin{mathpar}
  \inferrule* [lab=collection] {} {\meaningof{true} = \pi, \and \meaningof{~E} = \pi \setminus \meaningof{E}, \and \meaningof{E_{1} \& E_{2}} = \meaningof{E_{1}} \cap \meaningof{E_{2}}}
\end{mathpar}

\begin{mathpar}
  \inferrule* [lab=structure] {} {\meaningof{0} = \{ P \in \pi | P \equiv 0 \}, \and \\ \meaningof{E_1 | E_2} = \{ P \in \pi | P \equiv P_{1} | P_{2}, P_{1} \in \meaningof{E_{1}}, P_{2} \in \meaningof{E_2}\} }
\end{mathpar}

\begin{mathpar}
 \inferrule* [lab=behavior] {} {\meaningof{\langle a?b \rangle E} = \{ P \in \pi | P \equiv Q | u?(y)P', \\ \and \\\\ \and \\ \;\;\; u \in \meaningof{a}, \forall z.P'\{z/y\} \in \meaningof{E\{z/b\}}\}, \and \\ \meaningof{a!E} = \{ P \in \pi | P \equiv Q | x!\langle P' \rangle, x \in \meaningof{a} P' \in \meaningof{E}\} }
\end{mathpar}

\begin{mathpar}
 \inferrule* [lab=nominal] {} {\meaningof{\quotep{E}} = \{ \quotep{P} \in \quotep{\pi} | P \in \meaningof{E} \}, \and \meaningof{\quotep{P}} = \{ \quotep{Q} \in \quotep{\pi} | P \equiv Q \} \and \\ \meaningof{@\quotep{E}} = \{ P \in \pi | P \equiv @x, x \in \meaningof{E} \}}
\end{mathpar}

\begin{eqnarray*}
  \\
  \meaningof{-} : TS \to ST
\end{eqnarray*}

\begin{eqnarray*}
  \\
  L : TS \to ST
\end{eqnarray*}

\begin{eqnarray*}
  \\
  P \models E \iff P \in \meaningof{E}
\end{eqnarray*}

\begin{eqnarray*}
  P \approx_{L} Q \iff \forall E \in L. P \models E \iff Q \models E
\end{eqnarray*}

\begin{eqnarray*}
  P \approx_{K} Q
\end{eqnarray*}

\begin{eqnarray*}
  P \approx Q
\end{eqnarray*}

$\approx_{K} = \approx = \approx_{L}$

\subsubsection{Contextual duality}

Note that contexts extend the quotation operation to a family of
operations from processes to names. Given a context, $M$, we can
define a \emph{nominal context}, $\quotep{M}$ by $\quotep{M}[P] :=
\quotep{M[P]}$. To foreshadow what is to come we observe that these
operations enjoy a duality with processes very much like the duality
between vectors and maps from vectors to scalars.

Further, because the calculus is essentially higher-order, we have a
correspondence between contexts and processes. More specifically,
given a name $x$ and a context $M$ we can construct $M^{*}_{x}$ such
that 

\begin{mathpar}
  M^{*}_{x} | \lift{x}{P} \red M[P]
\end{mathpar}

namely,

\begin{mathpar}
  M^{*}_{x} := x?(u).M[\dropn{u}]
\end{mathpar}

The dependence of $M^{*}_{x}$ on a name makes it an abstraction, 

\begin{mathpar}
  M^{*} := (x)x?(u).M[\dropn{u}]
\end{mathpar}

\subsection{Additional notation}

It will sometimes be convenient to denote the process a name
quotes. We already have the notation $x = \quotep{P}$, but it will be
convenient to introduce an alternate notation, $\procn{x}$, when we
want to emphasize the connection to the use of the name. Note that, by
virtue of name equivalence, $\quotep{\procn{x}} \nameeq x$; so, the
notation is consistent with previous definitions.

Further, because names have structure it is possible to effect
substitutions on the basis of that structure. This means we need to
upgrade our notation for substitutions, which we accomplish by
adapting comprehension notation. Thus,

\begin{mathpar}
  P\{ y / x : x \in S \}
\end{mathpar}

is interpreted to mean the process derived from P by replacing (in a
capture-avoiding manner) each occurrence of $x$ in $S$ by $y$. For example,

\begin{mathpar}
  P\{ \quotep{\procn{x}|\procn{x}} / x : x \in \freenames{P} \}
\end{mathpar}

will replace each (occurrence) of a free name $x$ in $P$ by
$\quotep{\procn{x}|\procn{x}}$.

Also, we will avail ourselves of the notation $x^{L}$ and $x^{R}$ to
denote injections of a name into disjoint copies of the name
space. There are numerous ways to accomplish this. One example can be
found in \cite{MeredithR05}. This notation overloads to vectors of
names: $\vec{x}^{\pi} := (x_{i}^{\pi} \; : \; 0 \leq i < |\vec{x}| )$ where $\pi \in \{L,R\}$.

We also use $P^{\Box} := P|\Box$.

In \cite{MeredithR05} an interpretation of the new operator is
given. It turns out that there are several possible interpretations
all enjoying the requisite algebraic properties of the operator (see
\cite{milner91polyadicpi}). We will therefore make liberal use of
$(\nu\; \vec{x})P$.

% subsection the_syntax_and_semantics_of_the_notation_system (end)   

\input{qm2pi.qmops} 

\input{qm2pi.sterngerlach} 

\input{qm2pi.metric} 

% section concurrent_process_calculi (end)

%\input{qm2pi.proofsketch}

% section proof sketch (end)

%\input{qm2pi.slviaknots} 

% section spatial logic via knots (end)

\input{qm2pi.conclusion}

% section conclusion (end)

%\input{qm2pi.dtcodes} 

% section wiring algorithm (end)

\input{qm2pi.ack} 

% section acknowledgments (end)

\newpage


\bibliographystyle{plain}   
\bibliography{../../biblios/main.bib}

\input{qm2pi.rhodetails}

\end{document}

 

%\ifpdf
%\usepackage[pdftex]{graphicx}
%\else
%\usepackage{graphicx}
%\fi

 % \ifpdf
%  \usepackage{pdfsync}
%  \if


%\title{Brief Article}
%\author{David F. Snyder}
%\author{L.G. Meredith}

%\address{Dept. of Math., Texas State University--San Marcos, San Marcos, TX 78666}
       
\pagestyle{empty}


\begin{document}

\lstset{language=[Objective]Caml,frame=shadowbox}

\documentclass[12pt]{llncs}
%\documentclass{jktr}

\usepackage[pdftex]{hyperref}                   
\usepackage {listings}
\usepackage {mathpartir}
\usepackage{bcprules}
%\usepackage{listings}
                       
\usepackage{graphicx} 
%\usepackage[margins=2.5cm,nohead,nofoot]{geometry}
%\usepackage{geometry}
\usepackage{amsfonts}
\usepackage{amstext}
\usepackage{latexsym}
\usepackage{amssymb}
\usepackage{color}


%\include{myPreamble}
\include{qm2pi.local} 

%\ifpdf
%\usepackage[pdftex]{graphicx}
%\else
%\usepackage{graphicx}
%\fi

 % \ifpdf
%  \usepackage{pdfsync}
%  \if


%\title{Brief Article}
%\author{David F. Snyder}
%\author{L.G. Meredith}

%\address{Dept. of Math., Texas State University--San Marcos, San Marcos, TX 78666}
       
\pagestyle{empty}


\begin{document}

\lstset{language=[Objective]Caml,frame=shadowbox}

\input{qm2pi.front}

% section front matter (end)

\input{qm2pi.intro} 
 
% section introduction (end)

% \input{qm2pi.knotations} 

% section notation (end)

\input{qm2pi.process.calculi} 

% section concurrent_process_calculi_and_spatial_logics_ (end)
    
%\input{qm2pi.knots2pi} 

%\input{qm2pi.trefoil} 

%\input{qm2pi.mainthm} 

% subsection basic_interpretation (end)

%\input{qm2pi.rho.presentation} 
\subsection{The syntax and semantics of the notation system}\label{sub:the_syntax_and_semantics_of_the_notation_system} % (fold)

We now summarize a technical presentation of the calculus that
embodies our theory of dynamics. The typical presentation of such a
calculus follows the style of giving generators and relations on
them. The grammar, below, describing term constructors, freely
generates the set of processes, $\Proc$. This set is then quotiented
by a relation known as structural congruence and it is over this set
that the notion of dynamics is expressed. This presentation is
essentially that of \cite{MeredithR05} with the addition of
polyadicity and summation. For readability we have relegated some of
the technical subtleties to an appendix.

\subsubsection{Process grammar}\label{subsub:process_grammar}

\begin{mathpar}
  \inferrule* [lab=synchronization] {} {{M} \bc \pzero \;|\; x?F \;|\; x!C }
  \and
  \inferrule* [lab=abstraction] {} {{F} \bc (x)P}
  \and
  \inferrule* [lab=concretion] {} {{C} \bc \langle Q \rangle}
  \and
  \inferrule* [lab=process] {} {{P,Q} \bc M \;| \;P|Q \;|\; @{x}}
  \and
  \inferrule* [lab=name] {} {{x} \bc \quotep{P}}
\end{mathpar} 

Note that $\vec{x}$ (resp. $\vec{P}$) denotes a vector of names
(resp. processes) of length $|\vec{x}|$ (resp. $|\vec{P}|$). We adopt
the following useful abbreviations.

\begin{mathpar}
   x?(\vec{y}).P := x.(\vec{y})P \and  x\clift{\vec{P}} := x.\clift{\vec{P}}
   \and x!(y) := \lift{x}{\dropn{y}}
   \and \Pi_{i=0}^{n-1}P_i := P_0 | \ldots | P_{n-1}
\end{mathpar}

\subsubsection{Structural congruence}

\paragraph{Free and bound names and alpha-equivalence.} At the
core of structural equivalence is alpha-equivalence which identifies
process that are the same up to a change of variable. Formally, we
recognize the distinction between free and bound names. The free names
of a process, $\freenames{P}$, may be calculated recursively as
follows:

\begin{mathpar}
\freenames{\pzero} := \emptyset
  \and \\
  \freenames{x?(y).P} := \{ x \} \cup (\freenames{P} \setminus \{ y \})
  \and 
  \freenames{x!\langle P \rangle} := \{ x \} \cup \{ P \} 
  \and \\
  \freenames{P|Q} := \freenames{P} \cup \freenames{Q}
  \and \\
  \freenames{@{x}} := \{ x \}
\end{mathpar}

$\pi$
$\quotep{\pi}$

$\freenames{-} : \pi \to \mathcal{P}(\quotep{\pi})$

\begin{eqnarray*}
  \freenames{\pzero} & := & \emptyset \\
  \freenames{x?(y).P} & := & \{ x \} \cup (\freenames{P} \setminus \{ y \}) \\
  \freenames{x!\langle P \rangle} & := & \{ x \} \cup \{ P \} \\
  \freenames{P|Q} & := & \freenames{P} \cup \freenames{Q} \\
  \freenames{\dropn{x}} & := & \{ x \}
\end{eqnarray*}

The bound names of a process, $\boundnames{P}$, are those names occurring in $P$
that are not free. For example, in $x?(y).0$, the name $x$ is free, while $y$ is bound.

\begin{mathpar}
  \inferrule* [lab=monoidal-laws] {} { P|Q \equiv Q|P \and P|0 \equiv P \and P|(Q|R) \equiv (P|Q)|R }
\end{mathpar}

\begin{mathpar}
  \inferrule* [lab=alpha-equivalence] {} { (x)P \equiv (y)P\{y/x\} \and y \not\in \freenames{P} }
\end{mathpar}

\begin{definition}
Then two processes, $P,Q$, are alpha-equivalent if $P = Q\{\vec{y}/\vec{x}\}$ for
some $\vec{x} \in \boundnames{Q},\vec{y} \in \boundnames{P}$, where $Q\{\vec{y}/\vec{x}\}$
denotes the capture-avoiding substitution of $\vec{y}$ for $\vec{x}$ in $Q$.
\end{definition}

\begin{definition}
  The {\em structural congruence} \cite{SangiorgiWalker} , $\equiv$,
  between processes is the least congruence containing
  alpha-equivalence, satisfying the abelian monoid laws
  (associativity, commutativity and $\pzero$ as identity) for parallel
  composition $|$ and for summation $+$.
\end{definition}

\subsection{Name equivalence}

We take name equivalence, written $\nameeq$, to be the smallest
equivalence relation generated by the following rules.

\begin{mathpar}
\inferrule*[lab=Quote-drop]
{ }
{ \quotep{@{x}} \nameeq x }

\inferrule*[lab=Struct-equiv]
{ P \scong Q }
{ \quotep{P} \nameeq \quotep{Q} }
\end{mathpar}

The astute reader will have noticed that the mutual recursion of names
and processes imposes a mutual recursion on alpha-equivalence and
structural equivalence via name-equivalence. Fortunately, all of this
works out pleasantly and we may calculate in the natural way, free of
concern. The reader interested in the details is referred to the
appendix \ref{appendix:rho_details}.

\subsection{Substitution}

We use $\Proc$ for the set of processes, $\QProc$ for the set of
names, and $\id{\{}\vec{y} / \vec{x} \id{\}}$ to denote partial maps,
$s : \QProc \rightarrow \QProc$. A map, $s$ lifts, uniquely, to a map
on process terms, $\widehat{s} : \Proc \rightarrow \Proc$ by the
following equations.

\begin{mathpar}
  (0) \psubstp{Q}{P} := 0 \\
  (R \juxtap S) \psubstp{Q}{P}
  :=    
  (R)\psubstp{Q}{P} \juxtap (S) \psubstp{Q}{P} \\
  (x?(y).R) \psubstp{Q}{P}    
  :=    
  (x)\substp{Q}{P} (z)\concat( (R \psubstn{z}{y}) \psubstp{Q}{P} ) \\
  (\lift{x}{R}) \psubstp{Q}{P}  
  :=
  \lift{(x)\substp{Q}{P}}{ R \psubstp{Q}{P} } \\
%   (\dropn{x})  \psubstp{Q}{P}       
%   := 
%   \left\{ 
%     \begin{array}{ccc} 
%       \dropn{\quotep{Q}} & & x \nameeq \quotep{P} \\
%       \dropn{x} & & otherwise \\
%     \end{array}
%   \right. 
  (\dropn{x})  \psubstp{Q}{P}       
  := 
  \left\{ 
    \begin{array}{ccc} 
      Q & & x \nameeq \quotep{P} \\
      \dropn{x} & & otherwise \\
    \end{array}
  \right.
\end{mathpar}
 

where

\begin{eqnarray}
  (x)\id{\{} \lpquote Q \rpquote / \lpquote P \rpquote \id{\}}            = 
  \left\{ 
    \begin{array}{ccc}
      \lpquote Q \rpquote & & x \nameeq \lpquote P \rpquote \\
      x & & otherwise \\
    \end{array}
  \right. \nonumber
\end{eqnarray}

and $z$ is chosen distinct from $\quotep{P}$, $\quotep{Q}$, the free
names in $Q$, and all the names in $R$. Our $\alpha$-equivalence will
be built in the standard way from this substitution.

\begin{remark}\label{rem:no_self_referential_names}
  One consequence of these definitions is that $\forall P. \quotep{P}
  \not\in \freenames{P}$.
\end{remark}

\subsection{ Dynamic quote: an example }

Anticipating something of what's to come, consider applying the
substitution, $\widehat{\id{\{}u / z \id{\}}}$, to the following pair
of processes, $\lift{w}{y!(z)}$ and $w[ \lpquote y!(z) \rpquote ]$.

\begin{eqnarray}
	\lift{w}{y!(z)}\widehat{\id{\{}u / z \id{\}}}
		& = &
		\lift{w}{y!(u)} \nonumber\\
	w[ \lpquote y!(z) \rpquote ] \widehat{ \id{\{}u / z \id{\}} }
		& = &
		w[ \lpquote y!(z) \rpquote ] \nonumber
\end{eqnarray}

Because the body of the process between quotes is impervious to
substitution, we get radically different answers. In fact, by
examining the first process in an input context,
e.g. $x?(z).\lift{w}{y!(z)}$, we see that the process under the lift
operator may be shaped by prefixed inputs binding a name inside it. In
this sense, the lift operator will be seen as a way to dynamically
construct processes before reifying them as names.

Finally equipped with these standard features we can present the
dynamics of the calculus.

\subsubsection{Operational semantics} 

Finally, we introduce the computational dynamics. What marks these
algebras as distinct from other more traditionally studied algebraic
structures, e.g. vector spaces or polynomial rings, is the manner in
which dynamics is captured. In traditional structures, dynamics is typically
expressed through morphisms between such structures, as in linear maps
between vector spaces or morphisms between rings. In algebras
associated with the semantics of computation, the dynamics is
expressed as part of the algebraic structure itself, through a
reduction reduction relation typically denoted by $\red$. Below, we
give a recursive presentation of this relation for the calculus used
in the encoding.

$\red \subseteq \pi \times \pi$
$\red : \pi \to \mathcal{P}(\pi)$

\begin{mathpar}
  \inferrule* [lab=Comm] { \textsf{match}( x_{src}, x_{trgt} ) } { x_{trgt}?(y)P \; | \; x_{src}!\langle {Q} \rangle \red P\{\quotep{Q}/y}\} }
  \and \\
  \inferrule* [lab=Par] {{P} \red {P}'} {{{P} | {Q}} \red {{P}' | {Q}}}
  \and
  \inferrule* [lab=Equiv]{{{P} \scong {P}'} \andalso {{P}' \red {Q}'} \andalso {{Q}' \scong {Q}}}{{P} \red {Q}}
\end{mathpar}

\begin{eqnarray*}
  match_{\equiv} (\quotep{P},\quotep{Q}) & := & P \equiv Q \\
  match_{\dagger}(\quotep{P},\quotep{Q}) & := & \forall R. P|Q \red^{*} R => R \red^{*} 0 \\
  match_{K}(\quotep{P},\quotep{Q}) & := & K \mbox{ for some context } K
\end{eqnarray*}

$u?(x)P | u!\langle Q \rangle \red P\{\quotep{Q}/x\}$

%We write $\wred$ for $\red^*$, and $P\red$ if $\exists Q $ such that $ P \red Q$.
We write $P\red$ if $\exists Q $ such that $ P \red Q$ and $P\not\red$, otherwise.

\section{Replication}

As mentioned before, it is known that replication (and hence
recursion) can be implemented in a higher-order process algebra
\cite{SangiorgiWalker}. As our first example of calculation with the
machinery thus far presented we give the construction explicitly in
the {\rhoc}.

\begin{eqnarray}
	D_{x} & := & \prefix{x}{y}{(\binpar{\outputp{x}{y}}{@{y}})} \nonumber\\
	\bangp_{x}{P} & := & \binpar{{x}!\langle{\binpar{D_{x}}{P}}\rangle}{D_{x}} \nonumber
\end{eqnarray}

\begin{eqnarray}
	\bangp_{x}{P} & & \nonumber\\
	=
	& {x}!\langle{(\prefix{x}{y}{(\outputp{x}{y} | @{y})) | P}}\rangle 
	      | \prefix{x}{y}{(\outputp{x}{y} | @{y})} & \nonumber\\
	\red
	& (\outputp{x}{y} | @{y})\substn{\quotep{(\prefix{x}{y}{(@{y} | \outputp{x}{y})) | P}}}{y} & \nonumber\\
	=
	& \outputp{x}{\quotep{(\prefix{x}{y}{(\outputp{x}{y} | @{y})) | P}}}
	  | {(\prefix{x}{y}{(\outputp{x}{y} | @{y})) | P}} & \nonumber\\
	\red
	& \ldots & \nonumber\\
	\red^*
	& P | P | \ldots & \nonumber
\end{eqnarray}

Of course, this encoding, as an implementation, runs away, unfolding
$\bangp{P}$ eagerly. A lazier and more implementable replication
operator, restricted to input-guarded processes, may be obtained as follows.

\begin{eqnarray}
\bangp{\prefix{u}{v}{P}} 
	:= 
	\binpar{\lift{x}{\prefix{u}{v}{(\binpar{D(x)}{P})}}}{D(x)} \nonumber
\end{eqnarray}

\begin{remark}
  Note that the lazier definition still does not deal with summation
  or mixed summation (i.e. sums over input and output). The reader is
  invited to construct definitions of replication that deal with these
  features. 

  Further, the definitions are parameterized in a name, $x$. Can you,
  gentle reader, make a definition that eliminates this parameter and
  guarantees no accidental interaction between the replication
  machinery and the process being replicated -- i.e. no accidental
  sharing of names used by the process to get its work done and the
  name(s) used by the replication to effect copying. This latter
  revision of the definition of replication is crucial to obtaining
  the expected identity $!!P \sim !P$.
\end{remark}

\begin{remark}\label{rem:paradoxical_combinator}
  The reader familiar with the lambda calculus will have noticed the
  similarity between $D$ and the paradoxical combinator.

  [Ed. note: the existence of this seems to suggest we have to be more
  restrictive on the set of processes and names we admit if we are to
  support no-cloning.]
\end{remark}

\subsubsection{Bisimulation}

The computational dynamics gives rise to another kind of equivalence,
the equivalence of computational behavior. As previously mentioned
this is typically captured \emph{via} some form of bisimulation.

% The notion we use in this paper is weak barbed bisimulation
% \cite{milner91polyadicpi}.

The notion we use in this paper is derived from weak barbed
bisimulation \cite{milner91polyadicpi}. 

\begin{definition}
An \emph{observation relation}, $\downarrow_{\mathcal N}$, over a set
of names, $\mathcal N$, is the smallest relation satisfying the rules
below.

\infrule[Out-barb]{y \in {\mathcal N}, \; x \nameeq y}
		  {\outputp{x}{v} \downarrow_{\mathcal N} x}
\infrule[Par-barb]{\mbox{$P\downarrow_{\mathcal N} x$ or $Q\downarrow_{\mathcal N} x$}}
		  {\binpar{P}{Q} \downarrow_{\mathcal N} x}

We write $P \Downarrow_{\mathcal N} x$ if there is $Q$ such that 
$P \wred Q$ and $Q \downarrow_{\mathcal N} x$.
\end{definition}

\begin{definition}
%\label{def.bbisim}
An  ${\mathcal N}$-\emph{barbed bisimulation} over a set of names, ${\mathcal N}$, is a symmetric binary relation 
${\mathcal S}_{\mathcal N}$ between agents such that $P\rel{S}_{\mathcal N}Q$ implies:
\begin{enumerate}
\item If $P \red P'$ then $Q \wred Q'$ and $P'\rel{S}_{\mathcal N} Q'$.
\item If $P\downarrow_{\mathcal N} x$, then $Q\Downarrow_{\mathcal N} x$.
\end{enumerate}
$P$ is ${\mathcal N}$-barbed bisimilar to $Q$, written
$P \wbbisim_{\mathcal N} Q$, if $P \rel{S}_{\mathcal N} Q$ for some ${\mathcal N}$-barbed bisimulation ${\mathcal S}_{\mathcal N}$.
\end{definition}

$\mathcal{R} \subseteq \pi \times \pi$

$P \mathcal{R} Q => \forall P'. P \red P' \Rightarrow \exists Q'. Q \red Q', P' \mathcal{R} Q'$

$P \vdash x \Rightarrow Q \vdash x$

\begin{mathpar}
  \inferrule*[lab=Out-barb]{x \nameeq y}{{y}!\langle{Q}\rangle \vdash x}
  \and
  \inferrule*[lab=Par-barb]{\mbox{$P\vdash x$ or $Q\vdash x$}}{\binpar{P}{Q} \vdash x}
\end{mathpar}

\subsubsection{Contexts}

One of the principle advantages of computational calculi like the
$\pi$-calculus is a well-defined notion of context,
contextual-equivalence and a correlation between
contextual-equivalence and notions of bisimulation. The notion of
context allows the decomposition of a process into (sub-)process and
its syntactic environment, its context. Thus, a context may be
thought of as a process with a ``hole'' (written $\Box$) in it. The
application of a context $M$ to a process $P$, written $M[P]$, is
tantamount to filling the hole in $M$ with $P$. In this paper we do
not need the full weight of this theory, but do make use of the notion
of context in the proof the main theorem. 

\begin{mathpar}
  \inferrule* [lab=summation] {} {{M_{M},M_{N}} \bc \Box \;|\; x.M_{A} \;|\; M_{M}+M_{N}}
  \and
  \inferrule* [lab=agent] {} {{M_{A}} \bc (\vec{x})M_{P} \;| \; \clift{P_0,\ldots,M_{P},\ldots,P_N}}
  \and \\
  \inferrule* [lab=process] {} {{M_{P}} \bc M_{N} \;| \;P|M_{P} }
\end{mathpar} 

\begin{mathpar}
  \inferrule* [lab=sychronization] {} {M_{N} \bc \Box \;|\; x?M_{F} \;|\; x!M_{C}}
  \and
  \inferrule* [lab=abstraction] {} {{M_{F}} \bc (x)M_{P} }
  \and
  \inferrule* [lab=concretion] {} {{M_{C}} \bc \langle M_{P} \rangle }
  \and \\
  \inferrule* [lab=process] {} {{M_{P}} \bc M_{N} \;| \;P|M_{P} }
\end{mathpar}

\begin{definition}[contextual application] Given a context $M$, and
  process $P$, we define the \emph{contextual application}, $M[P] :=
  M\{P/\Box\}$. That is, the contextual application of M to P is the
  substitution of $P$ for $\Box$ in $M$.
\end{definition}

$\meaningof{-} : L \to \mathcal{P}(\pi)$

\begin{mathpar}
  \inferrule* [lab=collection] {} {\meaningof{true} = \pi, \and \meaningof{~E} = \pi \setminus \meaningof{E}, \and \meaningof{E_{1} \& E_{2}} = \meaningof{E_{1}} \cap \meaningof{E_{2}}}
\end{mathpar}

\begin{mathpar}
  \inferrule* [lab=structure] {} {\meaningof{0} = \{ P \in \pi | P \equiv 0 \}, \and \\ \meaningof{E_1 | E_2} = \{ P \in \pi | P \equiv P_{1} | P_{2}, P_{1} \in \meaningof{E_{1}}, P_{2} \in \meaningof{E_2}\} }
\end{mathpar}

\begin{mathpar}
 \inferrule* [lab=behavior] {} {\meaningof{\langle a?b \rangle E} = \{ P \in \pi | P \equiv Q | u?(y)P', \\ \and \\\\ \and \\ \;\;\; u \in \meaningof{a}, \forall z.P'\{z/y\} \in \meaningof{E\{z/b\}}\}, \and \\ \meaningof{a!E} = \{ P \in \pi | P \equiv Q | x!\langle P' \rangle, x \in \meaningof{a} P' \in \meaningof{E}\} }
\end{mathpar}

\begin{mathpar}
 \inferrule* [lab=nominal] {} {\meaningof{\quotep{E}} = \{ \quotep{P} \in \quotep{\pi} | P \in \meaningof{E} \}, \and \meaningof{\quotep{P}} = \{ \quotep{Q} \in \quotep{\pi} | P \equiv Q \} \and \\ \meaningof{@\quotep{E}} = \{ P \in \pi | P \equiv @x, x \in \meaningof{E} \}}
\end{mathpar}

\begin{eqnarray*}
  \\
  \meaningof{-} : TS \to ST
\end{eqnarray*}

\begin{eqnarray*}
  \\
  L : TS \to ST
\end{eqnarray*}

\begin{eqnarray*}
  \\
  P \models E \iff P \in \meaningof{E}
\end{eqnarray*}

\begin{eqnarray*}
  P \approx_{L} Q \iff \forall E \in L. P \models E \iff Q \models E
\end{eqnarray*}

\begin{eqnarray*}
  P \approx_{K} Q
\end{eqnarray*}

\begin{eqnarray*}
  P \approx Q
\end{eqnarray*}

$\approx_{K} = \approx = \approx_{L}$

\subsubsection{Contextual duality}

Note that contexts extend the quotation operation to a family of
operations from processes to names. Given a context, $M$, we can
define a \emph{nominal context}, $\quotep{M}$ by $\quotep{M}[P] :=
\quotep{M[P]}$. To foreshadow what is to come we observe that these
operations enjoy a duality with processes very much like the duality
between vectors and maps from vectors to scalars.

Further, because the calculus is essentially higher-order, we have a
correspondence between contexts and processes. More specifically,
given a name $x$ and a context $M$ we can construct $M^{*}_{x}$ such
that 

\begin{mathpar}
  M^{*}_{x} | \lift{x}{P} \red M[P]
\end{mathpar}

namely,

\begin{mathpar}
  M^{*}_{x} := x?(u).M[\dropn{u}]
\end{mathpar}

The dependence of $M^{*}_{x}$ on a name makes it an abstraction, 

\begin{mathpar}
  M^{*} := (x)x?(u).M[\dropn{u}]
\end{mathpar}

\subsection{Additional notation}

It will sometimes be convenient to denote the process a name
quotes. We already have the notation $x = \quotep{P}$, but it will be
convenient to introduce an alternate notation, $\procn{x}$, when we
want to emphasize the connection to the use of the name. Note that, by
virtue of name equivalence, $\quotep{\procn{x}} \nameeq x$; so, the
notation is consistent with previous definitions.

Further, because names have structure it is possible to effect
substitutions on the basis of that structure. This means we need to
upgrade our notation for substitutions, which we accomplish by
adapting comprehension notation. Thus,

\begin{mathpar}
  P\{ y / x : x \in S \}
\end{mathpar}

is interpreted to mean the process derived from P by replacing (in a
capture-avoiding manner) each occurrence of $x$ in $S$ by $y$. For example,

\begin{mathpar}
  P\{ \quotep{\procn{x}|\procn{x}} / x : x \in \freenames{P} \}
\end{mathpar}

will replace each (occurrence) of a free name $x$ in $P$ by
$\quotep{\procn{x}|\procn{x}}$.

Also, we will avail ourselves of the notation $x^{L}$ and $x^{R}$ to
denote injections of a name into disjoint copies of the name
space. There are numerous ways to accomplish this. One example can be
found in \cite{MeredithR05}. This notation overloads to vectors of
names: $\vec{x}^{\pi} := (x_{i}^{\pi} \; : \; 0 \leq i < |\vec{x}| )$ where $\pi \in \{L,R\}$.

We also use $P^{\Box} := P|\Box$.

In \cite{MeredithR05} an interpretation of the new operator is
given. It turns out that there are several possible interpretations
all enjoying the requisite algebraic properties of the operator (see
\cite{milner91polyadicpi}). We will therefore make liberal use of
$(\nu\; \vec{x})P$.

% subsection the_syntax_and_semantics_of_the_notation_system (end)   

\input{qm2pi.qmops} 

\input{qm2pi.sterngerlach} 

\input{qm2pi.metric} 

% section concurrent_process_calculi (end)

%\input{qm2pi.proofsketch}

% section proof sketch (end)

%\input{qm2pi.slviaknots} 

% section spatial logic via knots (end)

\input{qm2pi.conclusion}

% section conclusion (end)

%\input{qm2pi.dtcodes} 

% section wiring algorithm (end)

\input{qm2pi.ack} 

% section acknowledgments (end)

\newpage


\bibliographystyle{plain}   
\bibliography{../../biblios/main.bib}

\input{qm2pi.rhodetails}

\end{document}



% section front matter (end)

\section{Introduction}\label{sec:introduction} % (fold)
In this draft of the material i am going to have to dispense with the
usual writing conventions adopted in papers on these topics. i'm going
to have adopt whatever tone i need at the time i'm writing up the
calculations. Sometimes this may be very conversational; others it may
be the barest mathematical grunts; others still it may be that i have
lifted text from one of my other papers because the exposition of some
point was better said there. i hope that my readers are not unduly put
out by this decision. i'm not doing this to flout convention or be
rebellious. i find these calculations very technically challenging. To
keep everything going technically, something has to give; i have to
let go of some cognitive burden. So, the academic writing style --
with all of its trade-offs in terms of facilitating technical
communication -- is what i'm letting go of. Perhaps subsequent drafts
can be tightened and polished, but for now, i'm going to speak as if
we were sitting together in a coffee shop with a laptop, wifi and a
pad of paper and a pencil.

So, here's what i have to say. We -- you and i, comfortably ensconced
in our coffee shop and well-equipped with our tools -- can realize and
carry out the calculations of quantum mechanics over a very different
formal theory of dynamics, a formal theory of dynamics that
corresponds to a theory of concurrent computation with
\emph{reflection}. It has the advantage that the underlying theory is
already `quantized', but supports analogues all of the continuuous
operations. Strikingly, this underlying theory has recently been
connected with a notion of metric that we can show, by calculating
together, coincides with the metric induced by the inner product.

There are a lot of reasons why you might be interested in seeing
calculations of this form. Here's why i'm interested. For the past
several centuries there has been no competitor to the ``Newtonian''
account of dynamics. As a result the predominant share of accounts of
dynamical systems and situations have had to be formulated in terms of
the Newtonian machinery. i view this as an intellectually dangerous
position to occupy. Everything, despite it's intrinsic shape, turns
into a nail to be hit with this hammer. Recently, however, the theory
of computation has matured to the point where we have candidates for
theories of dynamics that offer very different perspective on
reasoning about dynamical systems and situations. Testing these
candidates against very successful accounts of dynamical situations,
like quantum mechanics, is going to give us some sense of how mature
they are and some measure of the quality of these accounts of
dynamics.

\subsection{Summary of contributions and outline of paper}

So, we're going to develop an interpretation of the operations of
quantum mechanics normally interpreted by Hilbert spaces and
operators. We're going to do this over a theory of computation. Note
that this is very different than the usual quantum computation program
which develops notions of computation over quantum mechanics. Rather,
we are developing a story that aligns with Wheeler's slogan: It from
Bit. To do this we will first provide an account of the theory of
computation at play here. Then we will dive into a calculation-driven
interpretation of the operations of quantum mechanics.

The reason we take this approach is that -- until very recently --
there hasn't been an axiomatic account of quantum mechanics. As a
result there has been no sharp delineation of the mathematical theory
supporting interpretation of the physical theory and the physical
theory, itself. So, ambient features of the maths are free to be
exploited (or supressed) without a real accounting of their physical
relevance. There is no sharp statement ``here's the physical theory''
qua \emph{theory} and ``here's the mathematical interpretation''
enabling a judgment of how faithful the interpretation is -- apart
from experimental observation. When there is an axiomatic account we
can judge how well a given mathematical formalism supports an
interpretation of the axioms, independent of
experimentation. Likewise, we can judge how well we have captured our
physical evidence and experience with our axiomatics, independent of
any specific mathematical implementation, with accidental detail that
may or may not have physical significance. 

In lieu of a fully fleshed out and vetted axiomatic account of quantum
mechanics, interpreting the operational notions in service of modeling
physical systems will have to suffice. In other words, we are not in
the business of providing a model of Hilbert spaces and operators. We
are in the business of providing a model of quantum mechanics because
we are motivated by testing our notions of dynamics against physical
theory; and, the predictive calculations of the physical theory must
serve as the best formulation -- shy of a fully fleshed out axiomatic
account -- of the physical theory itself (as they have for scientific
theories since time immemorial). Put another way, despite a
whole-hearted commitment to an It-from-Bit ontology, we are firmly
aligned with the shut-up-and-calculate camp as the best way to obtain
results either from the physical perspective or as a quality assurance
measure of our fledgling theory of dynamics.

In detail, we present a reflective process calculus. Then we develop
intuitive correspondences between the notions available in this
calculus and the usual physical notions supporting quantum mechanical
calculations. Thus, 

\begin{table}[htp]
  \center{
    \fbox{
      \begin{tabular}{c|c}
        quantum mechanics & process calculus \\
        \hline
        scalar & name \\
        state vector & process \\
        dual & contextual duals \\
        matrix & formal sums of process-context-dual pairs \\
        orthogonality & process annihilation \\
        inner product & execution-formula + quoting
      \end{tabular}
    }
  }
  \caption{QM - process calculi correspondences}
\end{table}

Then we tighten up these intuitions to operational definitions. We
employ the Dirac notation as the best proxy we can find for an
abstract syntax of the quantum mechanical notions. The definitions we
develop put us in contact with equational constraints coming from the
theory that we demonstrate the definitions and calculations satisfy.

This puts us in a position to shut up and calculate for the
Stern-Gerlach experimental set up, showing how these predictive
calculations become calculations on processes in our theory of a
reflective process calculus.

Penultimately, we demonstrate that the notion of metric coming from
the inner product coincides with the notion of metric available from
the theory of bisimulation. This demonstration gives us the right to
think of space as arising from behavior. Finally, we consider where we
might go from the new vantage point we have obtained.

% section introduction (end) 
 
% section introduction (end)

% \documentclass[12pt]{llncs}
%\documentclass{jktr}

\usepackage[pdftex]{hyperref}                   
\usepackage {listings}
\usepackage {mathpartir}
\usepackage{bcprules}
%\usepackage{listings}
                       
\usepackage{graphicx} 
%\usepackage[margins=2.5cm,nohead,nofoot]{geometry}
%\usepackage{geometry}
\usepackage{amsfonts}
\usepackage{amstext}
\usepackage{latexsym}
\usepackage{amssymb}
\usepackage{color}


%\include{myPreamble}
\include{qm2pi.local} 

%\ifpdf
%\usepackage[pdftex]{graphicx}
%\else
%\usepackage{graphicx}
%\fi

 % \ifpdf
%  \usepackage{pdfsync}
%  \if


%\title{Brief Article}
%\author{David F. Snyder}
%\author{L.G. Meredith}

%\address{Dept. of Math., Texas State University--San Marcos, San Marcos, TX 78666}
       
\pagestyle{empty}


\begin{document}

\lstset{language=[Objective]Caml,frame=shadowbox}

\input{qm2pi.front}

% section front matter (end)

\input{qm2pi.intro} 
 
% section introduction (end)

% \input{qm2pi.knotations} 

% section notation (end)

\input{qm2pi.process.calculi} 

% section concurrent_process_calculi_and_spatial_logics_ (end)
    
%\input{qm2pi.knots2pi} 

%\input{qm2pi.trefoil} 

%\input{qm2pi.mainthm} 

% subsection basic_interpretation (end)

%\input{qm2pi.rho.presentation} 
\subsection{The syntax and semantics of the notation system}\label{sub:the_syntax_and_semantics_of_the_notation_system} % (fold)

We now summarize a technical presentation of the calculus that
embodies our theory of dynamics. The typical presentation of such a
calculus follows the style of giving generators and relations on
them. The grammar, below, describing term constructors, freely
generates the set of processes, $\Proc$. This set is then quotiented
by a relation known as structural congruence and it is over this set
that the notion of dynamics is expressed. This presentation is
essentially that of \cite{MeredithR05} with the addition of
polyadicity and summation. For readability we have relegated some of
the technical subtleties to an appendix.

\subsubsection{Process grammar}\label{subsub:process_grammar}

\begin{mathpar}
  \inferrule* [lab=synchronization] {} {{M} \bc \pzero \;|\; x?F \;|\; x!C }
  \and
  \inferrule* [lab=abstraction] {} {{F} \bc (x)P}
  \and
  \inferrule* [lab=concretion] {} {{C} \bc \langle Q \rangle}
  \and
  \inferrule* [lab=process] {} {{P,Q} \bc M \;| \;P|Q \;|\; @{x}}
  \and
  \inferrule* [lab=name] {} {{x} \bc \quotep{P}}
\end{mathpar} 

Note that $\vec{x}$ (resp. $\vec{P}$) denotes a vector of names
(resp. processes) of length $|\vec{x}|$ (resp. $|\vec{P}|$). We adopt
the following useful abbreviations.

\begin{mathpar}
   x?(\vec{y}).P := x.(\vec{y})P \and  x\clift{\vec{P}} := x.\clift{\vec{P}}
   \and x!(y) := \lift{x}{\dropn{y}}
   \and \Pi_{i=0}^{n-1}P_i := P_0 | \ldots | P_{n-1}
\end{mathpar}

\subsubsection{Structural congruence}

\paragraph{Free and bound names and alpha-equivalence.} At the
core of structural equivalence is alpha-equivalence which identifies
process that are the same up to a change of variable. Formally, we
recognize the distinction between free and bound names. The free names
of a process, $\freenames{P}$, may be calculated recursively as
follows:

\begin{mathpar}
\freenames{\pzero} := \emptyset
  \and \\
  \freenames{x?(y).P} := \{ x \} \cup (\freenames{P} \setminus \{ y \})
  \and 
  \freenames{x!\langle P \rangle} := \{ x \} \cup \{ P \} 
  \and \\
  \freenames{P|Q} := \freenames{P} \cup \freenames{Q}
  \and \\
  \freenames{@{x}} := \{ x \}
\end{mathpar}

$\pi$
$\quotep{\pi}$

$\freenames{-} : \pi \to \mathcal{P}(\quotep{\pi})$

\begin{eqnarray*}
  \freenames{\pzero} & := & \emptyset \\
  \freenames{x?(y).P} & := & \{ x \} \cup (\freenames{P} \setminus \{ y \}) \\
  \freenames{x!\langle P \rangle} & := & \{ x \} \cup \{ P \} \\
  \freenames{P|Q} & := & \freenames{P} \cup \freenames{Q} \\
  \freenames{\dropn{x}} & := & \{ x \}
\end{eqnarray*}

The bound names of a process, $\boundnames{P}$, are those names occurring in $P$
that are not free. For example, in $x?(y).0$, the name $x$ is free, while $y$ is bound.

\begin{mathpar}
  \inferrule* [lab=monoidal-laws] {} { P|Q \equiv Q|P \and P|0 \equiv P \and P|(Q|R) \equiv (P|Q)|R }
\end{mathpar}

\begin{mathpar}
  \inferrule* [lab=alpha-equivalence] {} { (x)P \equiv (y)P\{y/x\} \and y \not\in \freenames{P} }
\end{mathpar}

\begin{definition}
Then two processes, $P,Q$, are alpha-equivalent if $P = Q\{\vec{y}/\vec{x}\}$ for
some $\vec{x} \in \boundnames{Q},\vec{y} \in \boundnames{P}$, where $Q\{\vec{y}/\vec{x}\}$
denotes the capture-avoiding substitution of $\vec{y}$ for $\vec{x}$ in $Q$.
\end{definition}

\begin{definition}
  The {\em structural congruence} \cite{SangiorgiWalker} , $\equiv$,
  between processes is the least congruence containing
  alpha-equivalence, satisfying the abelian monoid laws
  (associativity, commutativity and $\pzero$ as identity) for parallel
  composition $|$ and for summation $+$.
\end{definition}

\subsection{Name equivalence}

We take name equivalence, written $\nameeq$, to be the smallest
equivalence relation generated by the following rules.

\begin{mathpar}
\inferrule*[lab=Quote-drop]
{ }
{ \quotep{@{x}} \nameeq x }

\inferrule*[lab=Struct-equiv]
{ P \scong Q }
{ \quotep{P} \nameeq \quotep{Q} }
\end{mathpar}

The astute reader will have noticed that the mutual recursion of names
and processes imposes a mutual recursion on alpha-equivalence and
structural equivalence via name-equivalence. Fortunately, all of this
works out pleasantly and we may calculate in the natural way, free of
concern. The reader interested in the details is referred to the
appendix \ref{appendix:rho_details}.

\subsection{Substitution}

We use $\Proc$ for the set of processes, $\QProc$ for the set of
names, and $\id{\{}\vec{y} / \vec{x} \id{\}}$ to denote partial maps,
$s : \QProc \rightarrow \QProc$. A map, $s$ lifts, uniquely, to a map
on process terms, $\widehat{s} : \Proc \rightarrow \Proc$ by the
following equations.

\begin{mathpar}
  (0) \psubstp{Q}{P} := 0 \\
  (R \juxtap S) \psubstp{Q}{P}
  :=    
  (R)\psubstp{Q}{P} \juxtap (S) \psubstp{Q}{P} \\
  (x?(y).R) \psubstp{Q}{P}    
  :=    
  (x)\substp{Q}{P} (z)\concat( (R \psubstn{z}{y}) \psubstp{Q}{P} ) \\
  (\lift{x}{R}) \psubstp{Q}{P}  
  :=
  \lift{(x)\substp{Q}{P}}{ R \psubstp{Q}{P} } \\
%   (\dropn{x})  \psubstp{Q}{P}       
%   := 
%   \left\{ 
%     \begin{array}{ccc} 
%       \dropn{\quotep{Q}} & & x \nameeq \quotep{P} \\
%       \dropn{x} & & otherwise \\
%     \end{array}
%   \right. 
  (\dropn{x})  \psubstp{Q}{P}       
  := 
  \left\{ 
    \begin{array}{ccc} 
      Q & & x \nameeq \quotep{P} \\
      \dropn{x} & & otherwise \\
    \end{array}
  \right.
\end{mathpar}
 

where

\begin{eqnarray}
  (x)\id{\{} \lpquote Q \rpquote / \lpquote P \rpquote \id{\}}            = 
  \left\{ 
    \begin{array}{ccc}
      \lpquote Q \rpquote & & x \nameeq \lpquote P \rpquote \\
      x & & otherwise \\
    \end{array}
  \right. \nonumber
\end{eqnarray}

and $z$ is chosen distinct from $\quotep{P}$, $\quotep{Q}$, the free
names in $Q$, and all the names in $R$. Our $\alpha$-equivalence will
be built in the standard way from this substitution.

\begin{remark}\label{rem:no_self_referential_names}
  One consequence of these definitions is that $\forall P. \quotep{P}
  \not\in \freenames{P}$.
\end{remark}

\subsection{ Dynamic quote: an example }

Anticipating something of what's to come, consider applying the
substitution, $\widehat{\id{\{}u / z \id{\}}}$, to the following pair
of processes, $\lift{w}{y!(z)}$ and $w[ \lpquote y!(z) \rpquote ]$.

\begin{eqnarray}
	\lift{w}{y!(z)}\widehat{\id{\{}u / z \id{\}}}
		& = &
		\lift{w}{y!(u)} \nonumber\\
	w[ \lpquote y!(z) \rpquote ] \widehat{ \id{\{}u / z \id{\}} }
		& = &
		w[ \lpquote y!(z) \rpquote ] \nonumber
\end{eqnarray}

Because the body of the process between quotes is impervious to
substitution, we get radically different answers. In fact, by
examining the first process in an input context,
e.g. $x?(z).\lift{w}{y!(z)}$, we see that the process under the lift
operator may be shaped by prefixed inputs binding a name inside it. In
this sense, the lift operator will be seen as a way to dynamically
construct processes before reifying them as names.

Finally equipped with these standard features we can present the
dynamics of the calculus.

\subsubsection{Operational semantics} 

Finally, we introduce the computational dynamics. What marks these
algebras as distinct from other more traditionally studied algebraic
structures, e.g. vector spaces or polynomial rings, is the manner in
which dynamics is captured. In traditional structures, dynamics is typically
expressed through morphisms between such structures, as in linear maps
between vector spaces or morphisms between rings. In algebras
associated with the semantics of computation, the dynamics is
expressed as part of the algebraic structure itself, through a
reduction reduction relation typically denoted by $\red$. Below, we
give a recursive presentation of this relation for the calculus used
in the encoding.

$\red \subseteq \pi \times \pi$
$\red : \pi \to \mathcal{P}(\pi)$

\begin{mathpar}
  \inferrule* [lab=Comm] { \textsf{match}( x_{src}, x_{trgt} ) } { x_{trgt}?(y)P \; | \; x_{src}!\langle {Q} \rangle \red P\{\quotep{Q}/y}\} }
  \and \\
  \inferrule* [lab=Par] {{P} \red {P}'} {{{P} | {Q}} \red {{P}' | {Q}}}
  \and
  \inferrule* [lab=Equiv]{{{P} \scong {P}'} \andalso {{P}' \red {Q}'} \andalso {{Q}' \scong {Q}}}{{P} \red {Q}}
\end{mathpar}

\begin{eqnarray*}
  match_{\equiv} (\quotep{P},\quotep{Q}) & := & P \equiv Q \\
  match_{\dagger}(\quotep{P},\quotep{Q}) & := & \forall R. P|Q \red^{*} R => R \red^{*} 0 \\
  match_{K}(\quotep{P},\quotep{Q}) & := & K \mbox{ for some context } K
\end{eqnarray*}

$u?(x)P | u!\langle Q \rangle \red P\{\quotep{Q}/x\}$

%We write $\wred$ for $\red^*$, and $P\red$ if $\exists Q $ such that $ P \red Q$.
We write $P\red$ if $\exists Q $ such that $ P \red Q$ and $P\not\red$, otherwise.

\section{Replication}

As mentioned before, it is known that replication (and hence
recursion) can be implemented in a higher-order process algebra
\cite{SangiorgiWalker}. As our first example of calculation with the
machinery thus far presented we give the construction explicitly in
the {\rhoc}.

\begin{eqnarray}
	D_{x} & := & \prefix{x}{y}{(\binpar{\outputp{x}{y}}{@{y}})} \nonumber\\
	\bangp_{x}{P} & := & \binpar{{x}!\langle{\binpar{D_{x}}{P}}\rangle}{D_{x}} \nonumber
\end{eqnarray}

\begin{eqnarray}
	\bangp_{x}{P} & & \nonumber\\
	=
	& {x}!\langle{(\prefix{x}{y}{(\outputp{x}{y} | @{y})) | P}}\rangle 
	      | \prefix{x}{y}{(\outputp{x}{y} | @{y})} & \nonumber\\
	\red
	& (\outputp{x}{y} | @{y})\substn{\quotep{(\prefix{x}{y}{(@{y} | \outputp{x}{y})) | P}}}{y} & \nonumber\\
	=
	& \outputp{x}{\quotep{(\prefix{x}{y}{(\outputp{x}{y} | @{y})) | P}}}
	  | {(\prefix{x}{y}{(\outputp{x}{y} | @{y})) | P}} & \nonumber\\
	\red
	& \ldots & \nonumber\\
	\red^*
	& P | P | \ldots & \nonumber
\end{eqnarray}

Of course, this encoding, as an implementation, runs away, unfolding
$\bangp{P}$ eagerly. A lazier and more implementable replication
operator, restricted to input-guarded processes, may be obtained as follows.

\begin{eqnarray}
\bangp{\prefix{u}{v}{P}} 
	:= 
	\binpar{\lift{x}{\prefix{u}{v}{(\binpar{D(x)}{P})}}}{D(x)} \nonumber
\end{eqnarray}

\begin{remark}
  Note that the lazier definition still does not deal with summation
  or mixed summation (i.e. sums over input and output). The reader is
  invited to construct definitions of replication that deal with these
  features. 

  Further, the definitions are parameterized in a name, $x$. Can you,
  gentle reader, make a definition that eliminates this parameter and
  guarantees no accidental interaction between the replication
  machinery and the process being replicated -- i.e. no accidental
  sharing of names used by the process to get its work done and the
  name(s) used by the replication to effect copying. This latter
  revision of the definition of replication is crucial to obtaining
  the expected identity $!!P \sim !P$.
\end{remark}

\begin{remark}\label{rem:paradoxical_combinator}
  The reader familiar with the lambda calculus will have noticed the
  similarity between $D$ and the paradoxical combinator.

  [Ed. note: the existence of this seems to suggest we have to be more
  restrictive on the set of processes and names we admit if we are to
  support no-cloning.]
\end{remark}

\subsubsection{Bisimulation}

The computational dynamics gives rise to another kind of equivalence,
the equivalence of computational behavior. As previously mentioned
this is typically captured \emph{via} some form of bisimulation.

% The notion we use in this paper is weak barbed bisimulation
% \cite{milner91polyadicpi}.

The notion we use in this paper is derived from weak barbed
bisimulation \cite{milner91polyadicpi}. 

\begin{definition}
An \emph{observation relation}, $\downarrow_{\mathcal N}$, over a set
of names, $\mathcal N$, is the smallest relation satisfying the rules
below.

\infrule[Out-barb]{y \in {\mathcal N}, \; x \nameeq y}
		  {\outputp{x}{v} \downarrow_{\mathcal N} x}
\infrule[Par-barb]{\mbox{$P\downarrow_{\mathcal N} x$ or $Q\downarrow_{\mathcal N} x$}}
		  {\binpar{P}{Q} \downarrow_{\mathcal N} x}

We write $P \Downarrow_{\mathcal N} x$ if there is $Q$ such that 
$P \wred Q$ and $Q \downarrow_{\mathcal N} x$.
\end{definition}

\begin{definition}
%\label{def.bbisim}
An  ${\mathcal N}$-\emph{barbed bisimulation} over a set of names, ${\mathcal N}$, is a symmetric binary relation 
${\mathcal S}_{\mathcal N}$ between agents such that $P\rel{S}_{\mathcal N}Q$ implies:
\begin{enumerate}
\item If $P \red P'$ then $Q \wred Q'$ and $P'\rel{S}_{\mathcal N} Q'$.
\item If $P\downarrow_{\mathcal N} x$, then $Q\Downarrow_{\mathcal N} x$.
\end{enumerate}
$P$ is ${\mathcal N}$-barbed bisimilar to $Q$, written
$P \wbbisim_{\mathcal N} Q$, if $P \rel{S}_{\mathcal N} Q$ for some ${\mathcal N}$-barbed bisimulation ${\mathcal S}_{\mathcal N}$.
\end{definition}

$\mathcal{R} \subseteq \pi \times \pi$

$P \mathcal{R} Q => \forall P'. P \red P' \Rightarrow \exists Q'. Q \red Q', P' \mathcal{R} Q'$

$P \vdash x \Rightarrow Q \vdash x$

\begin{mathpar}
  \inferrule*[lab=Out-barb]{x \nameeq y}{{y}!\langle{Q}\rangle \vdash x}
  \and
  \inferrule*[lab=Par-barb]{\mbox{$P\vdash x$ or $Q\vdash x$}}{\binpar{P}{Q} \vdash x}
\end{mathpar}

\subsubsection{Contexts}

One of the principle advantages of computational calculi like the
$\pi$-calculus is a well-defined notion of context,
contextual-equivalence and a correlation between
contextual-equivalence and notions of bisimulation. The notion of
context allows the decomposition of a process into (sub-)process and
its syntactic environment, its context. Thus, a context may be
thought of as a process with a ``hole'' (written $\Box$) in it. The
application of a context $M$ to a process $P$, written $M[P]$, is
tantamount to filling the hole in $M$ with $P$. In this paper we do
not need the full weight of this theory, but do make use of the notion
of context in the proof the main theorem. 

\begin{mathpar}
  \inferrule* [lab=summation] {} {{M_{M},M_{N}} \bc \Box \;|\; x.M_{A} \;|\; M_{M}+M_{N}}
  \and
  \inferrule* [lab=agent] {} {{M_{A}} \bc (\vec{x})M_{P} \;| \; \clift{P_0,\ldots,M_{P},\ldots,P_N}}
  \and \\
  \inferrule* [lab=process] {} {{M_{P}} \bc M_{N} \;| \;P|M_{P} }
\end{mathpar} 

\begin{mathpar}
  \inferrule* [lab=sychronization] {} {M_{N} \bc \Box \;|\; x?M_{F} \;|\; x!M_{C}}
  \and
  \inferrule* [lab=abstraction] {} {{M_{F}} \bc (x)M_{P} }
  \and
  \inferrule* [lab=concretion] {} {{M_{C}} \bc \langle M_{P} \rangle }
  \and \\
  \inferrule* [lab=process] {} {{M_{P}} \bc M_{N} \;| \;P|M_{P} }
\end{mathpar}

\begin{definition}[contextual application] Given a context $M$, and
  process $P$, we define the \emph{contextual application}, $M[P] :=
  M\{P/\Box\}$. That is, the contextual application of M to P is the
  substitution of $P$ for $\Box$ in $M$.
\end{definition}

$\meaningof{-} : L \to \mathcal{P}(\pi)$

\begin{mathpar}
  \inferrule* [lab=collection] {} {\meaningof{true} = \pi, \and \meaningof{~E} = \pi \setminus \meaningof{E}, \and \meaningof{E_{1} \& E_{2}} = \meaningof{E_{1}} \cap \meaningof{E_{2}}}
\end{mathpar}

\begin{mathpar}
  \inferrule* [lab=structure] {} {\meaningof{0} = \{ P \in \pi | P \equiv 0 \}, \and \\ \meaningof{E_1 | E_2} = \{ P \in \pi | P \equiv P_{1} | P_{2}, P_{1} \in \meaningof{E_{1}}, P_{2} \in \meaningof{E_2}\} }
\end{mathpar}

\begin{mathpar}
 \inferrule* [lab=behavior] {} {\meaningof{\langle a?b \rangle E} = \{ P \in \pi | P \equiv Q | u?(y)P', \\ \and \\\\ \and \\ \;\;\; u \in \meaningof{a}, \forall z.P'\{z/y\} \in \meaningof{E\{z/b\}}\}, \and \\ \meaningof{a!E} = \{ P \in \pi | P \equiv Q | x!\langle P' \rangle, x \in \meaningof{a} P' \in \meaningof{E}\} }
\end{mathpar}

\begin{mathpar}
 \inferrule* [lab=nominal] {} {\meaningof{\quotep{E}} = \{ \quotep{P} \in \quotep{\pi} | P \in \meaningof{E} \}, \and \meaningof{\quotep{P}} = \{ \quotep{Q} \in \quotep{\pi} | P \equiv Q \} \and \\ \meaningof{@\quotep{E}} = \{ P \in \pi | P \equiv @x, x \in \meaningof{E} \}}
\end{mathpar}

\begin{eqnarray*}
  \\
  \meaningof{-} : TS \to ST
\end{eqnarray*}

\begin{eqnarray*}
  \\
  L : TS \to ST
\end{eqnarray*}

\begin{eqnarray*}
  \\
  P \models E \iff P \in \meaningof{E}
\end{eqnarray*}

\begin{eqnarray*}
  P \approx_{L} Q \iff \forall E \in L. P \models E \iff Q \models E
\end{eqnarray*}

\begin{eqnarray*}
  P \approx_{K} Q
\end{eqnarray*}

\begin{eqnarray*}
  P \approx Q
\end{eqnarray*}

$\approx_{K} = \approx = \approx_{L}$

\subsubsection{Contextual duality}

Note that contexts extend the quotation operation to a family of
operations from processes to names. Given a context, $M$, we can
define a \emph{nominal context}, $\quotep{M}$ by $\quotep{M}[P] :=
\quotep{M[P]}$. To foreshadow what is to come we observe that these
operations enjoy a duality with processes very much like the duality
between vectors and maps from vectors to scalars.

Further, because the calculus is essentially higher-order, we have a
correspondence between contexts and processes. More specifically,
given a name $x$ and a context $M$ we can construct $M^{*}_{x}$ such
that 

\begin{mathpar}
  M^{*}_{x} | \lift{x}{P} \red M[P]
\end{mathpar}

namely,

\begin{mathpar}
  M^{*}_{x} := x?(u).M[\dropn{u}]
\end{mathpar}

The dependence of $M^{*}_{x}$ on a name makes it an abstraction, 

\begin{mathpar}
  M^{*} := (x)x?(u).M[\dropn{u}]
\end{mathpar}

\subsection{Additional notation}

It will sometimes be convenient to denote the process a name
quotes. We already have the notation $x = \quotep{P}$, but it will be
convenient to introduce an alternate notation, $\procn{x}$, when we
want to emphasize the connection to the use of the name. Note that, by
virtue of name equivalence, $\quotep{\procn{x}} \nameeq x$; so, the
notation is consistent with previous definitions.

Further, because names have structure it is possible to effect
substitutions on the basis of that structure. This means we need to
upgrade our notation for substitutions, which we accomplish by
adapting comprehension notation. Thus,

\begin{mathpar}
  P\{ y / x : x \in S \}
\end{mathpar}

is interpreted to mean the process derived from P by replacing (in a
capture-avoiding manner) each occurrence of $x$ in $S$ by $y$. For example,

\begin{mathpar}
  P\{ \quotep{\procn{x}|\procn{x}} / x : x \in \freenames{P} \}
\end{mathpar}

will replace each (occurrence) of a free name $x$ in $P$ by
$\quotep{\procn{x}|\procn{x}}$.

Also, we will avail ourselves of the notation $x^{L}$ and $x^{R}$ to
denote injections of a name into disjoint copies of the name
space. There are numerous ways to accomplish this. One example can be
found in \cite{MeredithR05}. This notation overloads to vectors of
names: $\vec{x}^{\pi} := (x_{i}^{\pi} \; : \; 0 \leq i < |\vec{x}| )$ where $\pi \in \{L,R\}$.

We also use $P^{\Box} := P|\Box$.

In \cite{MeredithR05} an interpretation of the new operator is
given. It turns out that there are several possible interpretations
all enjoying the requisite algebraic properties of the operator (see
\cite{milner91polyadicpi}). We will therefore make liberal use of
$(\nu\; \vec{x})P$.

% subsection the_syntax_and_semantics_of_the_notation_system (end)   

\input{qm2pi.qmops} 

\input{qm2pi.sterngerlach} 

\input{qm2pi.metric} 

% section concurrent_process_calculi (end)

%\input{qm2pi.proofsketch}

% section proof sketch (end)

%\input{qm2pi.slviaknots} 

% section spatial logic via knots (end)

\input{qm2pi.conclusion}

% section conclusion (end)

%\input{qm2pi.dtcodes} 

% section wiring algorithm (end)

\input{qm2pi.ack} 

% section acknowledgments (end)

\newpage


\bibliographystyle{plain}   
\bibliography{../../biblios/main.bib}

\input{qm2pi.rhodetails}

\end{document}

 

% section notation (end)

\input{qm2pi.process.calculi} 

% section concurrent_process_calculi_and_spatial_logics_ (end)
    
%\documentclass[12pt]{llncs}
%\documentclass{jktr}

\usepackage[pdftex]{hyperref}                   
\usepackage {listings}
\usepackage {mathpartir}
\usepackage{bcprules}
%\usepackage{listings}
                       
\usepackage{graphicx} 
%\usepackage[margins=2.5cm,nohead,nofoot]{geometry}
%\usepackage{geometry}
\usepackage{amsfonts}
\usepackage{amstext}
\usepackage{latexsym}
\usepackage{amssymb}
\usepackage{color}


%\include{myPreamble}
\include{qm2pi.local} 

%\ifpdf
%\usepackage[pdftex]{graphicx}
%\else
%\usepackage{graphicx}
%\fi

 % \ifpdf
%  \usepackage{pdfsync}
%  \if


%\title{Brief Article}
%\author{David F. Snyder}
%\author{L.G. Meredith}

%\address{Dept. of Math., Texas State University--San Marcos, San Marcos, TX 78666}
       
\pagestyle{empty}


\begin{document}

\lstset{language=[Objective]Caml,frame=shadowbox}

\input{qm2pi.front}

% section front matter (end)

\input{qm2pi.intro} 
 
% section introduction (end)

% \input{qm2pi.knotations} 

% section notation (end)

\input{qm2pi.process.calculi} 

% section concurrent_process_calculi_and_spatial_logics_ (end)
    
%\input{qm2pi.knots2pi} 

%\input{qm2pi.trefoil} 

%\input{qm2pi.mainthm} 

% subsection basic_interpretation (end)

%\input{qm2pi.rho.presentation} 
\subsection{The syntax and semantics of the notation system}\label{sub:the_syntax_and_semantics_of_the_notation_system} % (fold)

We now summarize a technical presentation of the calculus that
embodies our theory of dynamics. The typical presentation of such a
calculus follows the style of giving generators and relations on
them. The grammar, below, describing term constructors, freely
generates the set of processes, $\Proc$. This set is then quotiented
by a relation known as structural congruence and it is over this set
that the notion of dynamics is expressed. This presentation is
essentially that of \cite{MeredithR05} with the addition of
polyadicity and summation. For readability we have relegated some of
the technical subtleties to an appendix.

\subsubsection{Process grammar}\label{subsub:process_grammar}

\begin{mathpar}
  \inferrule* [lab=synchronization] {} {{M} \bc \pzero \;|\; x?F \;|\; x!C }
  \and
  \inferrule* [lab=abstraction] {} {{F} \bc (x)P}
  \and
  \inferrule* [lab=concretion] {} {{C} \bc \langle Q \rangle}
  \and
  \inferrule* [lab=process] {} {{P,Q} \bc M \;| \;P|Q \;|\; @{x}}
  \and
  \inferrule* [lab=name] {} {{x} \bc \quotep{P}}
\end{mathpar} 

Note that $\vec{x}$ (resp. $\vec{P}$) denotes a vector of names
(resp. processes) of length $|\vec{x}|$ (resp. $|\vec{P}|$). We adopt
the following useful abbreviations.

\begin{mathpar}
   x?(\vec{y}).P := x.(\vec{y})P \and  x\clift{\vec{P}} := x.\clift{\vec{P}}
   \and x!(y) := \lift{x}{\dropn{y}}
   \and \Pi_{i=0}^{n-1}P_i := P_0 | \ldots | P_{n-1}
\end{mathpar}

\subsubsection{Structural congruence}

\paragraph{Free and bound names and alpha-equivalence.} At the
core of structural equivalence is alpha-equivalence which identifies
process that are the same up to a change of variable. Formally, we
recognize the distinction between free and bound names. The free names
of a process, $\freenames{P}$, may be calculated recursively as
follows:

\begin{mathpar}
\freenames{\pzero} := \emptyset
  \and \\
  \freenames{x?(y).P} := \{ x \} \cup (\freenames{P} \setminus \{ y \})
  \and 
  \freenames{x!\langle P \rangle} := \{ x \} \cup \{ P \} 
  \and \\
  \freenames{P|Q} := \freenames{P} \cup \freenames{Q}
  \and \\
  \freenames{@{x}} := \{ x \}
\end{mathpar}

$\pi$
$\quotep{\pi}$

$\freenames{-} : \pi \to \mathcal{P}(\quotep{\pi})$

\begin{eqnarray*}
  \freenames{\pzero} & := & \emptyset \\
  \freenames{x?(y).P} & := & \{ x \} \cup (\freenames{P} \setminus \{ y \}) \\
  \freenames{x!\langle P \rangle} & := & \{ x \} \cup \{ P \} \\
  \freenames{P|Q} & := & \freenames{P} \cup \freenames{Q} \\
  \freenames{\dropn{x}} & := & \{ x \}
\end{eqnarray*}

The bound names of a process, $\boundnames{P}$, are those names occurring in $P$
that are not free. For example, in $x?(y).0$, the name $x$ is free, while $y$ is bound.

\begin{mathpar}
  \inferrule* [lab=monoidal-laws] {} { P|Q \equiv Q|P \and P|0 \equiv P \and P|(Q|R) \equiv (P|Q)|R }
\end{mathpar}

\begin{mathpar}
  \inferrule* [lab=alpha-equivalence] {} { (x)P \equiv (y)P\{y/x\} \and y \not\in \freenames{P} }
\end{mathpar}

\begin{definition}
Then two processes, $P,Q$, are alpha-equivalent if $P = Q\{\vec{y}/\vec{x}\}$ for
some $\vec{x} \in \boundnames{Q},\vec{y} \in \boundnames{P}$, where $Q\{\vec{y}/\vec{x}\}$
denotes the capture-avoiding substitution of $\vec{y}$ for $\vec{x}$ in $Q$.
\end{definition}

\begin{definition}
  The {\em structural congruence} \cite{SangiorgiWalker} , $\equiv$,
  between processes is the least congruence containing
  alpha-equivalence, satisfying the abelian monoid laws
  (associativity, commutativity and $\pzero$ as identity) for parallel
  composition $|$ and for summation $+$.
\end{definition}

\subsection{Name equivalence}

We take name equivalence, written $\nameeq$, to be the smallest
equivalence relation generated by the following rules.

\begin{mathpar}
\inferrule*[lab=Quote-drop]
{ }
{ \quotep{@{x}} \nameeq x }

\inferrule*[lab=Struct-equiv]
{ P \scong Q }
{ \quotep{P} \nameeq \quotep{Q} }
\end{mathpar}

The astute reader will have noticed that the mutual recursion of names
and processes imposes a mutual recursion on alpha-equivalence and
structural equivalence via name-equivalence. Fortunately, all of this
works out pleasantly and we may calculate in the natural way, free of
concern. The reader interested in the details is referred to the
appendix \ref{appendix:rho_details}.

\subsection{Substitution}

We use $\Proc$ for the set of processes, $\QProc$ for the set of
names, and $\id{\{}\vec{y} / \vec{x} \id{\}}$ to denote partial maps,
$s : \QProc \rightarrow \QProc$. A map, $s$ lifts, uniquely, to a map
on process terms, $\widehat{s} : \Proc \rightarrow \Proc$ by the
following equations.

\begin{mathpar}
  (0) \psubstp{Q}{P} := 0 \\
  (R \juxtap S) \psubstp{Q}{P}
  :=    
  (R)\psubstp{Q}{P} \juxtap (S) \psubstp{Q}{P} \\
  (x?(y).R) \psubstp{Q}{P}    
  :=    
  (x)\substp{Q}{P} (z)\concat( (R \psubstn{z}{y}) \psubstp{Q}{P} ) \\
  (\lift{x}{R}) \psubstp{Q}{P}  
  :=
  \lift{(x)\substp{Q}{P}}{ R \psubstp{Q}{P} } \\
%   (\dropn{x})  \psubstp{Q}{P}       
%   := 
%   \left\{ 
%     \begin{array}{ccc} 
%       \dropn{\quotep{Q}} & & x \nameeq \quotep{P} \\
%       \dropn{x} & & otherwise \\
%     \end{array}
%   \right. 
  (\dropn{x})  \psubstp{Q}{P}       
  := 
  \left\{ 
    \begin{array}{ccc} 
      Q & & x \nameeq \quotep{P} \\
      \dropn{x} & & otherwise \\
    \end{array}
  \right.
\end{mathpar}
 

where

\begin{eqnarray}
  (x)\id{\{} \lpquote Q \rpquote / \lpquote P \rpquote \id{\}}            = 
  \left\{ 
    \begin{array}{ccc}
      \lpquote Q \rpquote & & x \nameeq \lpquote P \rpquote \\
      x & & otherwise \\
    \end{array}
  \right. \nonumber
\end{eqnarray}

and $z$ is chosen distinct from $\quotep{P}$, $\quotep{Q}$, the free
names in $Q$, and all the names in $R$. Our $\alpha$-equivalence will
be built in the standard way from this substitution.

\begin{remark}\label{rem:no_self_referential_names}
  One consequence of these definitions is that $\forall P. \quotep{P}
  \not\in \freenames{P}$.
\end{remark}

\subsection{ Dynamic quote: an example }

Anticipating something of what's to come, consider applying the
substitution, $\widehat{\id{\{}u / z \id{\}}}$, to the following pair
of processes, $\lift{w}{y!(z)}$ and $w[ \lpquote y!(z) \rpquote ]$.

\begin{eqnarray}
	\lift{w}{y!(z)}\widehat{\id{\{}u / z \id{\}}}
		& = &
		\lift{w}{y!(u)} \nonumber\\
	w[ \lpquote y!(z) \rpquote ] \widehat{ \id{\{}u / z \id{\}} }
		& = &
		w[ \lpquote y!(z) \rpquote ] \nonumber
\end{eqnarray}

Because the body of the process between quotes is impervious to
substitution, we get radically different answers. In fact, by
examining the first process in an input context,
e.g. $x?(z).\lift{w}{y!(z)}$, we see that the process under the lift
operator may be shaped by prefixed inputs binding a name inside it. In
this sense, the lift operator will be seen as a way to dynamically
construct processes before reifying them as names.

Finally equipped with these standard features we can present the
dynamics of the calculus.

\subsubsection{Operational semantics} 

Finally, we introduce the computational dynamics. What marks these
algebras as distinct from other more traditionally studied algebraic
structures, e.g. vector spaces or polynomial rings, is the manner in
which dynamics is captured. In traditional structures, dynamics is typically
expressed through morphisms between such structures, as in linear maps
between vector spaces or morphisms between rings. In algebras
associated with the semantics of computation, the dynamics is
expressed as part of the algebraic structure itself, through a
reduction reduction relation typically denoted by $\red$. Below, we
give a recursive presentation of this relation for the calculus used
in the encoding.

$\red \subseteq \pi \times \pi$
$\red : \pi \to \mathcal{P}(\pi)$

\begin{mathpar}
  \inferrule* [lab=Comm] { \textsf{match}( x_{src}, x_{trgt} ) } { x_{trgt}?(y)P \; | \; x_{src}!\langle {Q} \rangle \red P\{\quotep{Q}/y}\} }
  \and \\
  \inferrule* [lab=Par] {{P} \red {P}'} {{{P} | {Q}} \red {{P}' | {Q}}}
  \and
  \inferrule* [lab=Equiv]{{{P} \scong {P}'} \andalso {{P}' \red {Q}'} \andalso {{Q}' \scong {Q}}}{{P} \red {Q}}
\end{mathpar}

\begin{eqnarray*}
  match_{\equiv} (\quotep{P},\quotep{Q}) & := & P \equiv Q \\
  match_{\dagger}(\quotep{P},\quotep{Q}) & := & \forall R. P|Q \red^{*} R => R \red^{*} 0 \\
  match_{K}(\quotep{P},\quotep{Q}) & := & K \mbox{ for some context } K
\end{eqnarray*}

$u?(x)P | u!\langle Q \rangle \red P\{\quotep{Q}/x\}$

%We write $\wred$ for $\red^*$, and $P\red$ if $\exists Q $ such that $ P \red Q$.
We write $P\red$ if $\exists Q $ such that $ P \red Q$ and $P\not\red$, otherwise.

\section{Replication}

As mentioned before, it is known that replication (and hence
recursion) can be implemented in a higher-order process algebra
\cite{SangiorgiWalker}. As our first example of calculation with the
machinery thus far presented we give the construction explicitly in
the {\rhoc}.

\begin{eqnarray}
	D_{x} & := & \prefix{x}{y}{(\binpar{\outputp{x}{y}}{@{y}})} \nonumber\\
	\bangp_{x}{P} & := & \binpar{{x}!\langle{\binpar{D_{x}}{P}}\rangle}{D_{x}} \nonumber
\end{eqnarray}

\begin{eqnarray}
	\bangp_{x}{P} & & \nonumber\\
	=
	& {x}!\langle{(\prefix{x}{y}{(\outputp{x}{y} | @{y})) | P}}\rangle 
	      | \prefix{x}{y}{(\outputp{x}{y} | @{y})} & \nonumber\\
	\red
	& (\outputp{x}{y} | @{y})\substn{\quotep{(\prefix{x}{y}{(@{y} | \outputp{x}{y})) | P}}}{y} & \nonumber\\
	=
	& \outputp{x}{\quotep{(\prefix{x}{y}{(\outputp{x}{y} | @{y})) | P}}}
	  | {(\prefix{x}{y}{(\outputp{x}{y} | @{y})) | P}} & \nonumber\\
	\red
	& \ldots & \nonumber\\
	\red^*
	& P | P | \ldots & \nonumber
\end{eqnarray}

Of course, this encoding, as an implementation, runs away, unfolding
$\bangp{P}$ eagerly. A lazier and more implementable replication
operator, restricted to input-guarded processes, may be obtained as follows.

\begin{eqnarray}
\bangp{\prefix{u}{v}{P}} 
	:= 
	\binpar{\lift{x}{\prefix{u}{v}{(\binpar{D(x)}{P})}}}{D(x)} \nonumber
\end{eqnarray}

\begin{remark}
  Note that the lazier definition still does not deal with summation
  or mixed summation (i.e. sums over input and output). The reader is
  invited to construct definitions of replication that deal with these
  features. 

  Further, the definitions are parameterized in a name, $x$. Can you,
  gentle reader, make a definition that eliminates this parameter and
  guarantees no accidental interaction between the replication
  machinery and the process being replicated -- i.e. no accidental
  sharing of names used by the process to get its work done and the
  name(s) used by the replication to effect copying. This latter
  revision of the definition of replication is crucial to obtaining
  the expected identity $!!P \sim !P$.
\end{remark}

\begin{remark}\label{rem:paradoxical_combinator}
  The reader familiar with the lambda calculus will have noticed the
  similarity between $D$ and the paradoxical combinator.

  [Ed. note: the existence of this seems to suggest we have to be more
  restrictive on the set of processes and names we admit if we are to
  support no-cloning.]
\end{remark}

\subsubsection{Bisimulation}

The computational dynamics gives rise to another kind of equivalence,
the equivalence of computational behavior. As previously mentioned
this is typically captured \emph{via} some form of bisimulation.

% The notion we use in this paper is weak barbed bisimulation
% \cite{milner91polyadicpi}.

The notion we use in this paper is derived from weak barbed
bisimulation \cite{milner91polyadicpi}. 

\begin{definition}
An \emph{observation relation}, $\downarrow_{\mathcal N}$, over a set
of names, $\mathcal N$, is the smallest relation satisfying the rules
below.

\infrule[Out-barb]{y \in {\mathcal N}, \; x \nameeq y}
		  {\outputp{x}{v} \downarrow_{\mathcal N} x}
\infrule[Par-barb]{\mbox{$P\downarrow_{\mathcal N} x$ or $Q\downarrow_{\mathcal N} x$}}
		  {\binpar{P}{Q} \downarrow_{\mathcal N} x}

We write $P \Downarrow_{\mathcal N} x$ if there is $Q$ such that 
$P \wred Q$ and $Q \downarrow_{\mathcal N} x$.
\end{definition}

\begin{definition}
%\label{def.bbisim}
An  ${\mathcal N}$-\emph{barbed bisimulation} over a set of names, ${\mathcal N}$, is a symmetric binary relation 
${\mathcal S}_{\mathcal N}$ between agents such that $P\rel{S}_{\mathcal N}Q$ implies:
\begin{enumerate}
\item If $P \red P'$ then $Q \wred Q'$ and $P'\rel{S}_{\mathcal N} Q'$.
\item If $P\downarrow_{\mathcal N} x$, then $Q\Downarrow_{\mathcal N} x$.
\end{enumerate}
$P$ is ${\mathcal N}$-barbed bisimilar to $Q$, written
$P \wbbisim_{\mathcal N} Q$, if $P \rel{S}_{\mathcal N} Q$ for some ${\mathcal N}$-barbed bisimulation ${\mathcal S}_{\mathcal N}$.
\end{definition}

$\mathcal{R} \subseteq \pi \times \pi$

$P \mathcal{R} Q => \forall P'. P \red P' \Rightarrow \exists Q'. Q \red Q', P' \mathcal{R} Q'$

$P \vdash x \Rightarrow Q \vdash x$

\begin{mathpar}
  \inferrule*[lab=Out-barb]{x \nameeq y}{{y}!\langle{Q}\rangle \vdash x}
  \and
  \inferrule*[lab=Par-barb]{\mbox{$P\vdash x$ or $Q\vdash x$}}{\binpar{P}{Q} \vdash x}
\end{mathpar}

\subsubsection{Contexts}

One of the principle advantages of computational calculi like the
$\pi$-calculus is a well-defined notion of context,
contextual-equivalence and a correlation between
contextual-equivalence and notions of bisimulation. The notion of
context allows the decomposition of a process into (sub-)process and
its syntactic environment, its context. Thus, a context may be
thought of as a process with a ``hole'' (written $\Box$) in it. The
application of a context $M$ to a process $P$, written $M[P]$, is
tantamount to filling the hole in $M$ with $P$. In this paper we do
not need the full weight of this theory, but do make use of the notion
of context in the proof the main theorem. 

\begin{mathpar}
  \inferrule* [lab=summation] {} {{M_{M},M_{N}} \bc \Box \;|\; x.M_{A} \;|\; M_{M}+M_{N}}
  \and
  \inferrule* [lab=agent] {} {{M_{A}} \bc (\vec{x})M_{P} \;| \; \clift{P_0,\ldots,M_{P},\ldots,P_N}}
  \and \\
  \inferrule* [lab=process] {} {{M_{P}} \bc M_{N} \;| \;P|M_{P} }
\end{mathpar} 

\begin{mathpar}
  \inferrule* [lab=sychronization] {} {M_{N} \bc \Box \;|\; x?M_{F} \;|\; x!M_{C}}
  \and
  \inferrule* [lab=abstraction] {} {{M_{F}} \bc (x)M_{P} }
  \and
  \inferrule* [lab=concretion] {} {{M_{C}} \bc \langle M_{P} \rangle }
  \and \\
  \inferrule* [lab=process] {} {{M_{P}} \bc M_{N} \;| \;P|M_{P} }
\end{mathpar}

\begin{definition}[contextual application] Given a context $M$, and
  process $P$, we define the \emph{contextual application}, $M[P] :=
  M\{P/\Box\}$. That is, the contextual application of M to P is the
  substitution of $P$ for $\Box$ in $M$.
\end{definition}

$\meaningof{-} : L \to \mathcal{P}(\pi)$

\begin{mathpar}
  \inferrule* [lab=collection] {} {\meaningof{true} = \pi, \and \meaningof{~E} = \pi \setminus \meaningof{E}, \and \meaningof{E_{1} \& E_{2}} = \meaningof{E_{1}} \cap \meaningof{E_{2}}}
\end{mathpar}

\begin{mathpar}
  \inferrule* [lab=structure] {} {\meaningof{0} = \{ P \in \pi | P \equiv 0 \}, \and \\ \meaningof{E_1 | E_2} = \{ P \in \pi | P \equiv P_{1} | P_{2}, P_{1} \in \meaningof{E_{1}}, P_{2} \in \meaningof{E_2}\} }
\end{mathpar}

\begin{mathpar}
 \inferrule* [lab=behavior] {} {\meaningof{\langle a?b \rangle E} = \{ P \in \pi | P \equiv Q | u?(y)P', \\ \and \\\\ \and \\ \;\;\; u \in \meaningof{a}, \forall z.P'\{z/y\} \in \meaningof{E\{z/b\}}\}, \and \\ \meaningof{a!E} = \{ P \in \pi | P \equiv Q | x!\langle P' \rangle, x \in \meaningof{a} P' \in \meaningof{E}\} }
\end{mathpar}

\begin{mathpar}
 \inferrule* [lab=nominal] {} {\meaningof{\quotep{E}} = \{ \quotep{P} \in \quotep{\pi} | P \in \meaningof{E} \}, \and \meaningof{\quotep{P}} = \{ \quotep{Q} \in \quotep{\pi} | P \equiv Q \} \and \\ \meaningof{@\quotep{E}} = \{ P \in \pi | P \equiv @x, x \in \meaningof{E} \}}
\end{mathpar}

\begin{eqnarray*}
  \\
  \meaningof{-} : TS \to ST
\end{eqnarray*}

\begin{eqnarray*}
  \\
  L : TS \to ST
\end{eqnarray*}

\begin{eqnarray*}
  \\
  P \models E \iff P \in \meaningof{E}
\end{eqnarray*}

\begin{eqnarray*}
  P \approx_{L} Q \iff \forall E \in L. P \models E \iff Q \models E
\end{eqnarray*}

\begin{eqnarray*}
  P \approx_{K} Q
\end{eqnarray*}

\begin{eqnarray*}
  P \approx Q
\end{eqnarray*}

$\approx_{K} = \approx = \approx_{L}$

\subsubsection{Contextual duality}

Note that contexts extend the quotation operation to a family of
operations from processes to names. Given a context, $M$, we can
define a \emph{nominal context}, $\quotep{M}$ by $\quotep{M}[P] :=
\quotep{M[P]}$. To foreshadow what is to come we observe that these
operations enjoy a duality with processes very much like the duality
between vectors and maps from vectors to scalars.

Further, because the calculus is essentially higher-order, we have a
correspondence between contexts and processes. More specifically,
given a name $x$ and a context $M$ we can construct $M^{*}_{x}$ such
that 

\begin{mathpar}
  M^{*}_{x} | \lift{x}{P} \red M[P]
\end{mathpar}

namely,

\begin{mathpar}
  M^{*}_{x} := x?(u).M[\dropn{u}]
\end{mathpar}

The dependence of $M^{*}_{x}$ on a name makes it an abstraction, 

\begin{mathpar}
  M^{*} := (x)x?(u).M[\dropn{u}]
\end{mathpar}

\subsection{Additional notation}

It will sometimes be convenient to denote the process a name
quotes. We already have the notation $x = \quotep{P}$, but it will be
convenient to introduce an alternate notation, $\procn{x}$, when we
want to emphasize the connection to the use of the name. Note that, by
virtue of name equivalence, $\quotep{\procn{x}} \nameeq x$; so, the
notation is consistent with previous definitions.

Further, because names have structure it is possible to effect
substitutions on the basis of that structure. This means we need to
upgrade our notation for substitutions, which we accomplish by
adapting comprehension notation. Thus,

\begin{mathpar}
  P\{ y / x : x \in S \}
\end{mathpar}

is interpreted to mean the process derived from P by replacing (in a
capture-avoiding manner) each occurrence of $x$ in $S$ by $y$. For example,

\begin{mathpar}
  P\{ \quotep{\procn{x}|\procn{x}} / x : x \in \freenames{P} \}
\end{mathpar}

will replace each (occurrence) of a free name $x$ in $P$ by
$\quotep{\procn{x}|\procn{x}}$.

Also, we will avail ourselves of the notation $x^{L}$ and $x^{R}$ to
denote injections of a name into disjoint copies of the name
space. There are numerous ways to accomplish this. One example can be
found in \cite{MeredithR05}. This notation overloads to vectors of
names: $\vec{x}^{\pi} := (x_{i}^{\pi} \; : \; 0 \leq i < |\vec{x}| )$ where $\pi \in \{L,R\}$.

We also use $P^{\Box} := P|\Box$.

In \cite{MeredithR05} an interpretation of the new operator is
given. It turns out that there are several possible interpretations
all enjoying the requisite algebraic properties of the operator (see
\cite{milner91polyadicpi}). We will therefore make liberal use of
$(\nu\; \vec{x})P$.

% subsection the_syntax_and_semantics_of_the_notation_system (end)   

\input{qm2pi.qmops} 

\input{qm2pi.sterngerlach} 

\input{qm2pi.metric} 

% section concurrent_process_calculi (end)

%\input{qm2pi.proofsketch}

% section proof sketch (end)

%\input{qm2pi.slviaknots} 

% section spatial logic via knots (end)

\input{qm2pi.conclusion}

% section conclusion (end)

%\input{qm2pi.dtcodes} 

% section wiring algorithm (end)

\input{qm2pi.ack} 

% section acknowledgments (end)

\newpage


\bibliographystyle{plain}   
\bibliography{../../biblios/main.bib}

\input{qm2pi.rhodetails}

\end{document}

 

%\documentclass[12pt]{llncs}
%\documentclass{jktr}

\usepackage[pdftex]{hyperref}                   
\usepackage {listings}
\usepackage {mathpartir}
\usepackage{bcprules}
%\usepackage{listings}
                       
\usepackage{graphicx} 
%\usepackage[margins=2.5cm,nohead,nofoot]{geometry}
%\usepackage{geometry}
\usepackage{amsfonts}
\usepackage{amstext}
\usepackage{latexsym}
\usepackage{amssymb}
\usepackage{color}


%\include{myPreamble}
\include{qm2pi.local} 

%\ifpdf
%\usepackage[pdftex]{graphicx}
%\else
%\usepackage{graphicx}
%\fi

 % \ifpdf
%  \usepackage{pdfsync}
%  \if


%\title{Brief Article}
%\author{David F. Snyder}
%\author{L.G. Meredith}

%\address{Dept. of Math., Texas State University--San Marcos, San Marcos, TX 78666}
       
\pagestyle{empty}


\begin{document}

\lstset{language=[Objective]Caml,frame=shadowbox}

\input{qm2pi.front}

% section front matter (end)

\input{qm2pi.intro} 
 
% section introduction (end)

% \input{qm2pi.knotations} 

% section notation (end)

\input{qm2pi.process.calculi} 

% section concurrent_process_calculi_and_spatial_logics_ (end)
    
%\input{qm2pi.knots2pi} 

%\input{qm2pi.trefoil} 

%\input{qm2pi.mainthm} 

% subsection basic_interpretation (end)

%\input{qm2pi.rho.presentation} 
\subsection{The syntax and semantics of the notation system}\label{sub:the_syntax_and_semantics_of_the_notation_system} % (fold)

We now summarize a technical presentation of the calculus that
embodies our theory of dynamics. The typical presentation of such a
calculus follows the style of giving generators and relations on
them. The grammar, below, describing term constructors, freely
generates the set of processes, $\Proc$. This set is then quotiented
by a relation known as structural congruence and it is over this set
that the notion of dynamics is expressed. This presentation is
essentially that of \cite{MeredithR05} with the addition of
polyadicity and summation. For readability we have relegated some of
the technical subtleties to an appendix.

\subsubsection{Process grammar}\label{subsub:process_grammar}

\begin{mathpar}
  \inferrule* [lab=synchronization] {} {{M} \bc \pzero \;|\; x?F \;|\; x!C }
  \and
  \inferrule* [lab=abstraction] {} {{F} \bc (x)P}
  \and
  \inferrule* [lab=concretion] {} {{C} \bc \langle Q \rangle}
  \and
  \inferrule* [lab=process] {} {{P,Q} \bc M \;| \;P|Q \;|\; @{x}}
  \and
  \inferrule* [lab=name] {} {{x} \bc \quotep{P}}
\end{mathpar} 

Note that $\vec{x}$ (resp. $\vec{P}$) denotes a vector of names
(resp. processes) of length $|\vec{x}|$ (resp. $|\vec{P}|$). We adopt
the following useful abbreviations.

\begin{mathpar}
   x?(\vec{y}).P := x.(\vec{y})P \and  x\clift{\vec{P}} := x.\clift{\vec{P}}
   \and x!(y) := \lift{x}{\dropn{y}}
   \and \Pi_{i=0}^{n-1}P_i := P_0 | \ldots | P_{n-1}
\end{mathpar}

\subsubsection{Structural congruence}

\paragraph{Free and bound names and alpha-equivalence.} At the
core of structural equivalence is alpha-equivalence which identifies
process that are the same up to a change of variable. Formally, we
recognize the distinction between free and bound names. The free names
of a process, $\freenames{P}$, may be calculated recursively as
follows:

\begin{mathpar}
\freenames{\pzero} := \emptyset
  \and \\
  \freenames{x?(y).P} := \{ x \} \cup (\freenames{P} \setminus \{ y \})
  \and 
  \freenames{x!\langle P \rangle} := \{ x \} \cup \{ P \} 
  \and \\
  \freenames{P|Q} := \freenames{P} \cup \freenames{Q}
  \and \\
  \freenames{@{x}} := \{ x \}
\end{mathpar}

$\pi$
$\quotep{\pi}$

$\freenames{-} : \pi \to \mathcal{P}(\quotep{\pi})$

\begin{eqnarray*}
  \freenames{\pzero} & := & \emptyset \\
  \freenames{x?(y).P} & := & \{ x \} \cup (\freenames{P} \setminus \{ y \}) \\
  \freenames{x!\langle P \rangle} & := & \{ x \} \cup \{ P \} \\
  \freenames{P|Q} & := & \freenames{P} \cup \freenames{Q} \\
  \freenames{\dropn{x}} & := & \{ x \}
\end{eqnarray*}

The bound names of a process, $\boundnames{P}$, are those names occurring in $P$
that are not free. For example, in $x?(y).0$, the name $x$ is free, while $y$ is bound.

\begin{mathpar}
  \inferrule* [lab=monoidal-laws] {} { P|Q \equiv Q|P \and P|0 \equiv P \and P|(Q|R) \equiv (P|Q)|R }
\end{mathpar}

\begin{mathpar}
  \inferrule* [lab=alpha-equivalence] {} { (x)P \equiv (y)P\{y/x\} \and y \not\in \freenames{P} }
\end{mathpar}

\begin{definition}
Then two processes, $P,Q$, are alpha-equivalent if $P = Q\{\vec{y}/\vec{x}\}$ for
some $\vec{x} \in \boundnames{Q},\vec{y} \in \boundnames{P}$, where $Q\{\vec{y}/\vec{x}\}$
denotes the capture-avoiding substitution of $\vec{y}$ for $\vec{x}$ in $Q$.
\end{definition}

\begin{definition}
  The {\em structural congruence} \cite{SangiorgiWalker} , $\equiv$,
  between processes is the least congruence containing
  alpha-equivalence, satisfying the abelian monoid laws
  (associativity, commutativity and $\pzero$ as identity) for parallel
  composition $|$ and for summation $+$.
\end{definition}

\subsection{Name equivalence}

We take name equivalence, written $\nameeq$, to be the smallest
equivalence relation generated by the following rules.

\begin{mathpar}
\inferrule*[lab=Quote-drop]
{ }
{ \quotep{@{x}} \nameeq x }

\inferrule*[lab=Struct-equiv]
{ P \scong Q }
{ \quotep{P} \nameeq \quotep{Q} }
\end{mathpar}

The astute reader will have noticed that the mutual recursion of names
and processes imposes a mutual recursion on alpha-equivalence and
structural equivalence via name-equivalence. Fortunately, all of this
works out pleasantly and we may calculate in the natural way, free of
concern. The reader interested in the details is referred to the
appendix \ref{appendix:rho_details}.

\subsection{Substitution}

We use $\Proc$ for the set of processes, $\QProc$ for the set of
names, and $\id{\{}\vec{y} / \vec{x} \id{\}}$ to denote partial maps,
$s : \QProc \rightarrow \QProc$. A map, $s$ lifts, uniquely, to a map
on process terms, $\widehat{s} : \Proc \rightarrow \Proc$ by the
following equations.

\begin{mathpar}
  (0) \psubstp{Q}{P} := 0 \\
  (R \juxtap S) \psubstp{Q}{P}
  :=    
  (R)\psubstp{Q}{P} \juxtap (S) \psubstp{Q}{P} \\
  (x?(y).R) \psubstp{Q}{P}    
  :=    
  (x)\substp{Q}{P} (z)\concat( (R \psubstn{z}{y}) \psubstp{Q}{P} ) \\
  (\lift{x}{R}) \psubstp{Q}{P}  
  :=
  \lift{(x)\substp{Q}{P}}{ R \psubstp{Q}{P} } \\
%   (\dropn{x})  \psubstp{Q}{P}       
%   := 
%   \left\{ 
%     \begin{array}{ccc} 
%       \dropn{\quotep{Q}} & & x \nameeq \quotep{P} \\
%       \dropn{x} & & otherwise \\
%     \end{array}
%   \right. 
  (\dropn{x})  \psubstp{Q}{P}       
  := 
  \left\{ 
    \begin{array}{ccc} 
      Q & & x \nameeq \quotep{P} \\
      \dropn{x} & & otherwise \\
    \end{array}
  \right.
\end{mathpar}
 

where

\begin{eqnarray}
  (x)\id{\{} \lpquote Q \rpquote / \lpquote P \rpquote \id{\}}            = 
  \left\{ 
    \begin{array}{ccc}
      \lpquote Q \rpquote & & x \nameeq \lpquote P \rpquote \\
      x & & otherwise \\
    \end{array}
  \right. \nonumber
\end{eqnarray}

and $z$ is chosen distinct from $\quotep{P}$, $\quotep{Q}$, the free
names in $Q$, and all the names in $R$. Our $\alpha$-equivalence will
be built in the standard way from this substitution.

\begin{remark}\label{rem:no_self_referential_names}
  One consequence of these definitions is that $\forall P. \quotep{P}
  \not\in \freenames{P}$.
\end{remark}

\subsection{ Dynamic quote: an example }

Anticipating something of what's to come, consider applying the
substitution, $\widehat{\id{\{}u / z \id{\}}}$, to the following pair
of processes, $\lift{w}{y!(z)}$ and $w[ \lpquote y!(z) \rpquote ]$.

\begin{eqnarray}
	\lift{w}{y!(z)}\widehat{\id{\{}u / z \id{\}}}
		& = &
		\lift{w}{y!(u)} \nonumber\\
	w[ \lpquote y!(z) \rpquote ] \widehat{ \id{\{}u / z \id{\}} }
		& = &
		w[ \lpquote y!(z) \rpquote ] \nonumber
\end{eqnarray}

Because the body of the process between quotes is impervious to
substitution, we get radically different answers. In fact, by
examining the first process in an input context,
e.g. $x?(z).\lift{w}{y!(z)}$, we see that the process under the lift
operator may be shaped by prefixed inputs binding a name inside it. In
this sense, the lift operator will be seen as a way to dynamically
construct processes before reifying them as names.

Finally equipped with these standard features we can present the
dynamics of the calculus.

\subsubsection{Operational semantics} 

Finally, we introduce the computational dynamics. What marks these
algebras as distinct from other more traditionally studied algebraic
structures, e.g. vector spaces or polynomial rings, is the manner in
which dynamics is captured. In traditional structures, dynamics is typically
expressed through morphisms between such structures, as in linear maps
between vector spaces or morphisms between rings. In algebras
associated with the semantics of computation, the dynamics is
expressed as part of the algebraic structure itself, through a
reduction reduction relation typically denoted by $\red$. Below, we
give a recursive presentation of this relation for the calculus used
in the encoding.

$\red \subseteq \pi \times \pi$
$\red : \pi \to \mathcal{P}(\pi)$

\begin{mathpar}
  \inferrule* [lab=Comm] { \textsf{match}( x_{src}, x_{trgt} ) } { x_{trgt}?(y)P \; | \; x_{src}!\langle {Q} \rangle \red P\{\quotep{Q}/y}\} }
  \and \\
  \inferrule* [lab=Par] {{P} \red {P}'} {{{P} | {Q}} \red {{P}' | {Q}}}
  \and
  \inferrule* [lab=Equiv]{{{P} \scong {P}'} \andalso {{P}' \red {Q}'} \andalso {{Q}' \scong {Q}}}{{P} \red {Q}}
\end{mathpar}

\begin{eqnarray*}
  match_{\equiv} (\quotep{P},\quotep{Q}) & := & P \equiv Q \\
  match_{\dagger}(\quotep{P},\quotep{Q}) & := & \forall R. P|Q \red^{*} R => R \red^{*} 0 \\
  match_{K}(\quotep{P},\quotep{Q}) & := & K \mbox{ for some context } K
\end{eqnarray*}

$u?(x)P | u!\langle Q \rangle \red P\{\quotep{Q}/x\}$

%We write $\wred$ for $\red^*$, and $P\red$ if $\exists Q $ such that $ P \red Q$.
We write $P\red$ if $\exists Q $ such that $ P \red Q$ and $P\not\red$, otherwise.

\section{Replication}

As mentioned before, it is known that replication (and hence
recursion) can be implemented in a higher-order process algebra
\cite{SangiorgiWalker}. As our first example of calculation with the
machinery thus far presented we give the construction explicitly in
the {\rhoc}.

\begin{eqnarray}
	D_{x} & := & \prefix{x}{y}{(\binpar{\outputp{x}{y}}{@{y}})} \nonumber\\
	\bangp_{x}{P} & := & \binpar{{x}!\langle{\binpar{D_{x}}{P}}\rangle}{D_{x}} \nonumber
\end{eqnarray}

\begin{eqnarray}
	\bangp_{x}{P} & & \nonumber\\
	=
	& {x}!\langle{(\prefix{x}{y}{(\outputp{x}{y} | @{y})) | P}}\rangle 
	      | \prefix{x}{y}{(\outputp{x}{y} | @{y})} & \nonumber\\
	\red
	& (\outputp{x}{y} | @{y})\substn{\quotep{(\prefix{x}{y}{(@{y} | \outputp{x}{y})) | P}}}{y} & \nonumber\\
	=
	& \outputp{x}{\quotep{(\prefix{x}{y}{(\outputp{x}{y} | @{y})) | P}}}
	  | {(\prefix{x}{y}{(\outputp{x}{y} | @{y})) | P}} & \nonumber\\
	\red
	& \ldots & \nonumber\\
	\red^*
	& P | P | \ldots & \nonumber
\end{eqnarray}

Of course, this encoding, as an implementation, runs away, unfolding
$\bangp{P}$ eagerly. A lazier and more implementable replication
operator, restricted to input-guarded processes, may be obtained as follows.

\begin{eqnarray}
\bangp{\prefix{u}{v}{P}} 
	:= 
	\binpar{\lift{x}{\prefix{u}{v}{(\binpar{D(x)}{P})}}}{D(x)} \nonumber
\end{eqnarray}

\begin{remark}
  Note that the lazier definition still does not deal with summation
  or mixed summation (i.e. sums over input and output). The reader is
  invited to construct definitions of replication that deal with these
  features. 

  Further, the definitions are parameterized in a name, $x$. Can you,
  gentle reader, make a definition that eliminates this parameter and
  guarantees no accidental interaction between the replication
  machinery and the process being replicated -- i.e. no accidental
  sharing of names used by the process to get its work done and the
  name(s) used by the replication to effect copying. This latter
  revision of the definition of replication is crucial to obtaining
  the expected identity $!!P \sim !P$.
\end{remark}

\begin{remark}\label{rem:paradoxical_combinator}
  The reader familiar with the lambda calculus will have noticed the
  similarity between $D$ and the paradoxical combinator.

  [Ed. note: the existence of this seems to suggest we have to be more
  restrictive on the set of processes and names we admit if we are to
  support no-cloning.]
\end{remark}

\subsubsection{Bisimulation}

The computational dynamics gives rise to another kind of equivalence,
the equivalence of computational behavior. As previously mentioned
this is typically captured \emph{via} some form of bisimulation.

% The notion we use in this paper is weak barbed bisimulation
% \cite{milner91polyadicpi}.

The notion we use in this paper is derived from weak barbed
bisimulation \cite{milner91polyadicpi}. 

\begin{definition}
An \emph{observation relation}, $\downarrow_{\mathcal N}$, over a set
of names, $\mathcal N$, is the smallest relation satisfying the rules
below.

\infrule[Out-barb]{y \in {\mathcal N}, \; x \nameeq y}
		  {\outputp{x}{v} \downarrow_{\mathcal N} x}
\infrule[Par-barb]{\mbox{$P\downarrow_{\mathcal N} x$ or $Q\downarrow_{\mathcal N} x$}}
		  {\binpar{P}{Q} \downarrow_{\mathcal N} x}

We write $P \Downarrow_{\mathcal N} x$ if there is $Q$ such that 
$P \wred Q$ and $Q \downarrow_{\mathcal N} x$.
\end{definition}

\begin{definition}
%\label{def.bbisim}
An  ${\mathcal N}$-\emph{barbed bisimulation} over a set of names, ${\mathcal N}$, is a symmetric binary relation 
${\mathcal S}_{\mathcal N}$ between agents such that $P\rel{S}_{\mathcal N}Q$ implies:
\begin{enumerate}
\item If $P \red P'$ then $Q \wred Q'$ and $P'\rel{S}_{\mathcal N} Q'$.
\item If $P\downarrow_{\mathcal N} x$, then $Q\Downarrow_{\mathcal N} x$.
\end{enumerate}
$P$ is ${\mathcal N}$-barbed bisimilar to $Q$, written
$P \wbbisim_{\mathcal N} Q$, if $P \rel{S}_{\mathcal N} Q$ for some ${\mathcal N}$-barbed bisimulation ${\mathcal S}_{\mathcal N}$.
\end{definition}

$\mathcal{R} \subseteq \pi \times \pi$

$P \mathcal{R} Q => \forall P'. P \red P' \Rightarrow \exists Q'. Q \red Q', P' \mathcal{R} Q'$

$P \vdash x \Rightarrow Q \vdash x$

\begin{mathpar}
  \inferrule*[lab=Out-barb]{x \nameeq y}{{y}!\langle{Q}\rangle \vdash x}
  \and
  \inferrule*[lab=Par-barb]{\mbox{$P\vdash x$ or $Q\vdash x$}}{\binpar{P}{Q} \vdash x}
\end{mathpar}

\subsubsection{Contexts}

One of the principle advantages of computational calculi like the
$\pi$-calculus is a well-defined notion of context,
contextual-equivalence and a correlation between
contextual-equivalence and notions of bisimulation. The notion of
context allows the decomposition of a process into (sub-)process and
its syntactic environment, its context. Thus, a context may be
thought of as a process with a ``hole'' (written $\Box$) in it. The
application of a context $M$ to a process $P$, written $M[P]$, is
tantamount to filling the hole in $M$ with $P$. In this paper we do
not need the full weight of this theory, but do make use of the notion
of context in the proof the main theorem. 

\begin{mathpar}
  \inferrule* [lab=summation] {} {{M_{M},M_{N}} \bc \Box \;|\; x.M_{A} \;|\; M_{M}+M_{N}}
  \and
  \inferrule* [lab=agent] {} {{M_{A}} \bc (\vec{x})M_{P} \;| \; \clift{P_0,\ldots,M_{P},\ldots,P_N}}
  \and \\
  \inferrule* [lab=process] {} {{M_{P}} \bc M_{N} \;| \;P|M_{P} }
\end{mathpar} 

\begin{mathpar}
  \inferrule* [lab=sychronization] {} {M_{N} \bc \Box \;|\; x?M_{F} \;|\; x!M_{C}}
  \and
  \inferrule* [lab=abstraction] {} {{M_{F}} \bc (x)M_{P} }
  \and
  \inferrule* [lab=concretion] {} {{M_{C}} \bc \langle M_{P} \rangle }
  \and \\
  \inferrule* [lab=process] {} {{M_{P}} \bc M_{N} \;| \;P|M_{P} }
\end{mathpar}

\begin{definition}[contextual application] Given a context $M$, and
  process $P$, we define the \emph{contextual application}, $M[P] :=
  M\{P/\Box\}$. That is, the contextual application of M to P is the
  substitution of $P$ for $\Box$ in $M$.
\end{definition}

$\meaningof{-} : L \to \mathcal{P}(\pi)$

\begin{mathpar}
  \inferrule* [lab=collection] {} {\meaningof{true} = \pi, \and \meaningof{~E} = \pi \setminus \meaningof{E}, \and \meaningof{E_{1} \& E_{2}} = \meaningof{E_{1}} \cap \meaningof{E_{2}}}
\end{mathpar}

\begin{mathpar}
  \inferrule* [lab=structure] {} {\meaningof{0} = \{ P \in \pi | P \equiv 0 \}, \and \\ \meaningof{E_1 | E_2} = \{ P \in \pi | P \equiv P_{1} | P_{2}, P_{1} \in \meaningof{E_{1}}, P_{2} \in \meaningof{E_2}\} }
\end{mathpar}

\begin{mathpar}
 \inferrule* [lab=behavior] {} {\meaningof{\langle a?b \rangle E} = \{ P \in \pi | P \equiv Q | u?(y)P', \\ \and \\\\ \and \\ \;\;\; u \in \meaningof{a}, \forall z.P'\{z/y\} \in \meaningof{E\{z/b\}}\}, \and \\ \meaningof{a!E} = \{ P \in \pi | P \equiv Q | x!\langle P' \rangle, x \in \meaningof{a} P' \in \meaningof{E}\} }
\end{mathpar}

\begin{mathpar}
 \inferrule* [lab=nominal] {} {\meaningof{\quotep{E}} = \{ \quotep{P} \in \quotep{\pi} | P \in \meaningof{E} \}, \and \meaningof{\quotep{P}} = \{ \quotep{Q} \in \quotep{\pi} | P \equiv Q \} \and \\ \meaningof{@\quotep{E}} = \{ P \in \pi | P \equiv @x, x \in \meaningof{E} \}}
\end{mathpar}

\begin{eqnarray*}
  \\
  \meaningof{-} : TS \to ST
\end{eqnarray*}

\begin{eqnarray*}
  \\
  L : TS \to ST
\end{eqnarray*}

\begin{eqnarray*}
  \\
  P \models E \iff P \in \meaningof{E}
\end{eqnarray*}

\begin{eqnarray*}
  P \approx_{L} Q \iff \forall E \in L. P \models E \iff Q \models E
\end{eqnarray*}

\begin{eqnarray*}
  P \approx_{K} Q
\end{eqnarray*}

\begin{eqnarray*}
  P \approx Q
\end{eqnarray*}

$\approx_{K} = \approx = \approx_{L}$

\subsubsection{Contextual duality}

Note that contexts extend the quotation operation to a family of
operations from processes to names. Given a context, $M$, we can
define a \emph{nominal context}, $\quotep{M}$ by $\quotep{M}[P] :=
\quotep{M[P]}$. To foreshadow what is to come we observe that these
operations enjoy a duality with processes very much like the duality
between vectors and maps from vectors to scalars.

Further, because the calculus is essentially higher-order, we have a
correspondence between contexts and processes. More specifically,
given a name $x$ and a context $M$ we can construct $M^{*}_{x}$ such
that 

\begin{mathpar}
  M^{*}_{x} | \lift{x}{P} \red M[P]
\end{mathpar}

namely,

\begin{mathpar}
  M^{*}_{x} := x?(u).M[\dropn{u}]
\end{mathpar}

The dependence of $M^{*}_{x}$ on a name makes it an abstraction, 

\begin{mathpar}
  M^{*} := (x)x?(u).M[\dropn{u}]
\end{mathpar}

\subsection{Additional notation}

It will sometimes be convenient to denote the process a name
quotes. We already have the notation $x = \quotep{P}$, but it will be
convenient to introduce an alternate notation, $\procn{x}$, when we
want to emphasize the connection to the use of the name. Note that, by
virtue of name equivalence, $\quotep{\procn{x}} \nameeq x$; so, the
notation is consistent with previous definitions.

Further, because names have structure it is possible to effect
substitutions on the basis of that structure. This means we need to
upgrade our notation for substitutions, which we accomplish by
adapting comprehension notation. Thus,

\begin{mathpar}
  P\{ y / x : x \in S \}
\end{mathpar}

is interpreted to mean the process derived from P by replacing (in a
capture-avoiding manner) each occurrence of $x$ in $S$ by $y$. For example,

\begin{mathpar}
  P\{ \quotep{\procn{x}|\procn{x}} / x : x \in \freenames{P} \}
\end{mathpar}

will replace each (occurrence) of a free name $x$ in $P$ by
$\quotep{\procn{x}|\procn{x}}$.

Also, we will avail ourselves of the notation $x^{L}$ and $x^{R}$ to
denote injections of a name into disjoint copies of the name
space. There are numerous ways to accomplish this. One example can be
found in \cite{MeredithR05}. This notation overloads to vectors of
names: $\vec{x}^{\pi} := (x_{i}^{\pi} \; : \; 0 \leq i < |\vec{x}| )$ where $\pi \in \{L,R\}$.

We also use $P^{\Box} := P|\Box$.

In \cite{MeredithR05} an interpretation of the new operator is
given. It turns out that there are several possible interpretations
all enjoying the requisite algebraic properties of the operator (see
\cite{milner91polyadicpi}). We will therefore make liberal use of
$(\nu\; \vec{x})P$.

% subsection the_syntax_and_semantics_of_the_notation_system (end)   

\input{qm2pi.qmops} 

\input{qm2pi.sterngerlach} 

\input{qm2pi.metric} 

% section concurrent_process_calculi (end)

%\input{qm2pi.proofsketch}

% section proof sketch (end)

%\input{qm2pi.slviaknots} 

% section spatial logic via knots (end)

\input{qm2pi.conclusion}

% section conclusion (end)

%\input{qm2pi.dtcodes} 

% section wiring algorithm (end)

\input{qm2pi.ack} 

% section acknowledgments (end)

\newpage


\bibliographystyle{plain}   
\bibliography{../../biblios/main.bib}

\input{qm2pi.rhodetails}

\end{document}

 

%\documentclass[12pt]{llncs}
%\documentclass{jktr}

\usepackage[pdftex]{hyperref}                   
\usepackage {listings}
\usepackage {mathpartir}
\usepackage{bcprules}
%\usepackage{listings}
                       
\usepackage{graphicx} 
%\usepackage[margins=2.5cm,nohead,nofoot]{geometry}
%\usepackage{geometry}
\usepackage{amsfonts}
\usepackage{amstext}
\usepackage{latexsym}
\usepackage{amssymb}
\usepackage{color}


%\include{myPreamble}
\include{qm2pi.local} 

%\ifpdf
%\usepackage[pdftex]{graphicx}
%\else
%\usepackage{graphicx}
%\fi

 % \ifpdf
%  \usepackage{pdfsync}
%  \if


%\title{Brief Article}
%\author{David F. Snyder}
%\author{L.G. Meredith}

%\address{Dept. of Math., Texas State University--San Marcos, San Marcos, TX 78666}
       
\pagestyle{empty}


\begin{document}

\lstset{language=[Objective]Caml,frame=shadowbox}

\input{qm2pi.front}

% section front matter (end)

\input{qm2pi.intro} 
 
% section introduction (end)

% \input{qm2pi.knotations} 

% section notation (end)

\input{qm2pi.process.calculi} 

% section concurrent_process_calculi_and_spatial_logics_ (end)
    
%\input{qm2pi.knots2pi} 

%\input{qm2pi.trefoil} 

%\input{qm2pi.mainthm} 

% subsection basic_interpretation (end)

%\input{qm2pi.rho.presentation} 
\subsection{The syntax and semantics of the notation system}\label{sub:the_syntax_and_semantics_of_the_notation_system} % (fold)

We now summarize a technical presentation of the calculus that
embodies our theory of dynamics. The typical presentation of such a
calculus follows the style of giving generators and relations on
them. The grammar, below, describing term constructors, freely
generates the set of processes, $\Proc$. This set is then quotiented
by a relation known as structural congruence and it is over this set
that the notion of dynamics is expressed. This presentation is
essentially that of \cite{MeredithR05} with the addition of
polyadicity and summation. For readability we have relegated some of
the technical subtleties to an appendix.

\subsubsection{Process grammar}\label{subsub:process_grammar}

\begin{mathpar}
  \inferrule* [lab=synchronization] {} {{M} \bc \pzero \;|\; x?F \;|\; x!C }
  \and
  \inferrule* [lab=abstraction] {} {{F} \bc (x)P}
  \and
  \inferrule* [lab=concretion] {} {{C} \bc \langle Q \rangle}
  \and
  \inferrule* [lab=process] {} {{P,Q} \bc M \;| \;P|Q \;|\; @{x}}
  \and
  \inferrule* [lab=name] {} {{x} \bc \quotep{P}}
\end{mathpar} 

Note that $\vec{x}$ (resp. $\vec{P}$) denotes a vector of names
(resp. processes) of length $|\vec{x}|$ (resp. $|\vec{P}|$). We adopt
the following useful abbreviations.

\begin{mathpar}
   x?(\vec{y}).P := x.(\vec{y})P \and  x\clift{\vec{P}} := x.\clift{\vec{P}}
   \and x!(y) := \lift{x}{\dropn{y}}
   \and \Pi_{i=0}^{n-1}P_i := P_0 | \ldots | P_{n-1}
\end{mathpar}

\subsubsection{Structural congruence}

\paragraph{Free and bound names and alpha-equivalence.} At the
core of structural equivalence is alpha-equivalence which identifies
process that are the same up to a change of variable. Formally, we
recognize the distinction between free and bound names. The free names
of a process, $\freenames{P}$, may be calculated recursively as
follows:

\begin{mathpar}
\freenames{\pzero} := \emptyset
  \and \\
  \freenames{x?(y).P} := \{ x \} \cup (\freenames{P} \setminus \{ y \})
  \and 
  \freenames{x!\langle P \rangle} := \{ x \} \cup \{ P \} 
  \and \\
  \freenames{P|Q} := \freenames{P} \cup \freenames{Q}
  \and \\
  \freenames{@{x}} := \{ x \}
\end{mathpar}

$\pi$
$\quotep{\pi}$

$\freenames{-} : \pi \to \mathcal{P}(\quotep{\pi})$

\begin{eqnarray*}
  \freenames{\pzero} & := & \emptyset \\
  \freenames{x?(y).P} & := & \{ x \} \cup (\freenames{P} \setminus \{ y \}) \\
  \freenames{x!\langle P \rangle} & := & \{ x \} \cup \{ P \} \\
  \freenames{P|Q} & := & \freenames{P} \cup \freenames{Q} \\
  \freenames{\dropn{x}} & := & \{ x \}
\end{eqnarray*}

The bound names of a process, $\boundnames{P}$, are those names occurring in $P$
that are not free. For example, in $x?(y).0$, the name $x$ is free, while $y$ is bound.

\begin{mathpar}
  \inferrule* [lab=monoidal-laws] {} { P|Q \equiv Q|P \and P|0 \equiv P \and P|(Q|R) \equiv (P|Q)|R }
\end{mathpar}

\begin{mathpar}
  \inferrule* [lab=alpha-equivalence] {} { (x)P \equiv (y)P\{y/x\} \and y \not\in \freenames{P} }
\end{mathpar}

\begin{definition}
Then two processes, $P,Q$, are alpha-equivalent if $P = Q\{\vec{y}/\vec{x}\}$ for
some $\vec{x} \in \boundnames{Q},\vec{y} \in \boundnames{P}$, where $Q\{\vec{y}/\vec{x}\}$
denotes the capture-avoiding substitution of $\vec{y}$ for $\vec{x}$ in $Q$.
\end{definition}

\begin{definition}
  The {\em structural congruence} \cite{SangiorgiWalker} , $\equiv$,
  between processes is the least congruence containing
  alpha-equivalence, satisfying the abelian monoid laws
  (associativity, commutativity and $\pzero$ as identity) for parallel
  composition $|$ and for summation $+$.
\end{definition}

\subsection{Name equivalence}

We take name equivalence, written $\nameeq$, to be the smallest
equivalence relation generated by the following rules.

\begin{mathpar}
\inferrule*[lab=Quote-drop]
{ }
{ \quotep{@{x}} \nameeq x }

\inferrule*[lab=Struct-equiv]
{ P \scong Q }
{ \quotep{P} \nameeq \quotep{Q} }
\end{mathpar}

The astute reader will have noticed that the mutual recursion of names
and processes imposes a mutual recursion on alpha-equivalence and
structural equivalence via name-equivalence. Fortunately, all of this
works out pleasantly and we may calculate in the natural way, free of
concern. The reader interested in the details is referred to the
appendix \ref{appendix:rho_details}.

\subsection{Substitution}

We use $\Proc$ for the set of processes, $\QProc$ for the set of
names, and $\id{\{}\vec{y} / \vec{x} \id{\}}$ to denote partial maps,
$s : \QProc \rightarrow \QProc$. A map, $s$ lifts, uniquely, to a map
on process terms, $\widehat{s} : \Proc \rightarrow \Proc$ by the
following equations.

\begin{mathpar}
  (0) \psubstp{Q}{P} := 0 \\
  (R \juxtap S) \psubstp{Q}{P}
  :=    
  (R)\psubstp{Q}{P} \juxtap (S) \psubstp{Q}{P} \\
  (x?(y).R) \psubstp{Q}{P}    
  :=    
  (x)\substp{Q}{P} (z)\concat( (R \psubstn{z}{y}) \psubstp{Q}{P} ) \\
  (\lift{x}{R}) \psubstp{Q}{P}  
  :=
  \lift{(x)\substp{Q}{P}}{ R \psubstp{Q}{P} } \\
%   (\dropn{x})  \psubstp{Q}{P}       
%   := 
%   \left\{ 
%     \begin{array}{ccc} 
%       \dropn{\quotep{Q}} & & x \nameeq \quotep{P} \\
%       \dropn{x} & & otherwise \\
%     \end{array}
%   \right. 
  (\dropn{x})  \psubstp{Q}{P}       
  := 
  \left\{ 
    \begin{array}{ccc} 
      Q & & x \nameeq \quotep{P} \\
      \dropn{x} & & otherwise \\
    \end{array}
  \right.
\end{mathpar}
 

where

\begin{eqnarray}
  (x)\id{\{} \lpquote Q \rpquote / \lpquote P \rpquote \id{\}}            = 
  \left\{ 
    \begin{array}{ccc}
      \lpquote Q \rpquote & & x \nameeq \lpquote P \rpquote \\
      x & & otherwise \\
    \end{array}
  \right. \nonumber
\end{eqnarray}

and $z$ is chosen distinct from $\quotep{P}$, $\quotep{Q}$, the free
names in $Q$, and all the names in $R$. Our $\alpha$-equivalence will
be built in the standard way from this substitution.

\begin{remark}\label{rem:no_self_referential_names}
  One consequence of these definitions is that $\forall P. \quotep{P}
  \not\in \freenames{P}$.
\end{remark}

\subsection{ Dynamic quote: an example }

Anticipating something of what's to come, consider applying the
substitution, $\widehat{\id{\{}u / z \id{\}}}$, to the following pair
of processes, $\lift{w}{y!(z)}$ and $w[ \lpquote y!(z) \rpquote ]$.

\begin{eqnarray}
	\lift{w}{y!(z)}\widehat{\id{\{}u / z \id{\}}}
		& = &
		\lift{w}{y!(u)} \nonumber\\
	w[ \lpquote y!(z) \rpquote ] \widehat{ \id{\{}u / z \id{\}} }
		& = &
		w[ \lpquote y!(z) \rpquote ] \nonumber
\end{eqnarray}

Because the body of the process between quotes is impervious to
substitution, we get radically different answers. In fact, by
examining the first process in an input context,
e.g. $x?(z).\lift{w}{y!(z)}$, we see that the process under the lift
operator may be shaped by prefixed inputs binding a name inside it. In
this sense, the lift operator will be seen as a way to dynamically
construct processes before reifying them as names.

Finally equipped with these standard features we can present the
dynamics of the calculus.

\subsubsection{Operational semantics} 

Finally, we introduce the computational dynamics. What marks these
algebras as distinct from other more traditionally studied algebraic
structures, e.g. vector spaces or polynomial rings, is the manner in
which dynamics is captured. In traditional structures, dynamics is typically
expressed through morphisms between such structures, as in linear maps
between vector spaces or morphisms between rings. In algebras
associated with the semantics of computation, the dynamics is
expressed as part of the algebraic structure itself, through a
reduction reduction relation typically denoted by $\red$. Below, we
give a recursive presentation of this relation for the calculus used
in the encoding.

$\red \subseteq \pi \times \pi$
$\red : \pi \to \mathcal{P}(\pi)$

\begin{mathpar}
  \inferrule* [lab=Comm] { \textsf{match}( x_{src}, x_{trgt} ) } { x_{trgt}?(y)P \; | \; x_{src}!\langle {Q} \rangle \red P\{\quotep{Q}/y}\} }
  \and \\
  \inferrule* [lab=Par] {{P} \red {P}'} {{{P} | {Q}} \red {{P}' | {Q}}}
  \and
  \inferrule* [lab=Equiv]{{{P} \scong {P}'} \andalso {{P}' \red {Q}'} \andalso {{Q}' \scong {Q}}}{{P} \red {Q}}
\end{mathpar}

\begin{eqnarray*}
  match_{\equiv} (\quotep{P},\quotep{Q}) & := & P \equiv Q \\
  match_{\dagger}(\quotep{P},\quotep{Q}) & := & \forall R. P|Q \red^{*} R => R \red^{*} 0 \\
  match_{K}(\quotep{P},\quotep{Q}) & := & K \mbox{ for some context } K
\end{eqnarray*}

$u?(x)P | u!\langle Q \rangle \red P\{\quotep{Q}/x\}$

%We write $\wred$ for $\red^*$, and $P\red$ if $\exists Q $ such that $ P \red Q$.
We write $P\red$ if $\exists Q $ such that $ P \red Q$ and $P\not\red$, otherwise.

\section{Replication}

As mentioned before, it is known that replication (and hence
recursion) can be implemented in a higher-order process algebra
\cite{SangiorgiWalker}. As our first example of calculation with the
machinery thus far presented we give the construction explicitly in
the {\rhoc}.

\begin{eqnarray}
	D_{x} & := & \prefix{x}{y}{(\binpar{\outputp{x}{y}}{@{y}})} \nonumber\\
	\bangp_{x}{P} & := & \binpar{{x}!\langle{\binpar{D_{x}}{P}}\rangle}{D_{x}} \nonumber
\end{eqnarray}

\begin{eqnarray}
	\bangp_{x}{P} & & \nonumber\\
	=
	& {x}!\langle{(\prefix{x}{y}{(\outputp{x}{y} | @{y})) | P}}\rangle 
	      | \prefix{x}{y}{(\outputp{x}{y} | @{y})} & \nonumber\\
	\red
	& (\outputp{x}{y} | @{y})\substn{\quotep{(\prefix{x}{y}{(@{y} | \outputp{x}{y})) | P}}}{y} & \nonumber\\
	=
	& \outputp{x}{\quotep{(\prefix{x}{y}{(\outputp{x}{y} | @{y})) | P}}}
	  | {(\prefix{x}{y}{(\outputp{x}{y} | @{y})) | P}} & \nonumber\\
	\red
	& \ldots & \nonumber\\
	\red^*
	& P | P | \ldots & \nonumber
\end{eqnarray}

Of course, this encoding, as an implementation, runs away, unfolding
$\bangp{P}$ eagerly. A lazier and more implementable replication
operator, restricted to input-guarded processes, may be obtained as follows.

\begin{eqnarray}
\bangp{\prefix{u}{v}{P}} 
	:= 
	\binpar{\lift{x}{\prefix{u}{v}{(\binpar{D(x)}{P})}}}{D(x)} \nonumber
\end{eqnarray}

\begin{remark}
  Note that the lazier definition still does not deal with summation
  or mixed summation (i.e. sums over input and output). The reader is
  invited to construct definitions of replication that deal with these
  features. 

  Further, the definitions are parameterized in a name, $x$. Can you,
  gentle reader, make a definition that eliminates this parameter and
  guarantees no accidental interaction between the replication
  machinery and the process being replicated -- i.e. no accidental
  sharing of names used by the process to get its work done and the
  name(s) used by the replication to effect copying. This latter
  revision of the definition of replication is crucial to obtaining
  the expected identity $!!P \sim !P$.
\end{remark}

\begin{remark}\label{rem:paradoxical_combinator}
  The reader familiar with the lambda calculus will have noticed the
  similarity between $D$ and the paradoxical combinator.

  [Ed. note: the existence of this seems to suggest we have to be more
  restrictive on the set of processes and names we admit if we are to
  support no-cloning.]
\end{remark}

\subsubsection{Bisimulation}

The computational dynamics gives rise to another kind of equivalence,
the equivalence of computational behavior. As previously mentioned
this is typically captured \emph{via} some form of bisimulation.

% The notion we use in this paper is weak barbed bisimulation
% \cite{milner91polyadicpi}.

The notion we use in this paper is derived from weak barbed
bisimulation \cite{milner91polyadicpi}. 

\begin{definition}
An \emph{observation relation}, $\downarrow_{\mathcal N}$, over a set
of names, $\mathcal N$, is the smallest relation satisfying the rules
below.

\infrule[Out-barb]{y \in {\mathcal N}, \; x \nameeq y}
		  {\outputp{x}{v} \downarrow_{\mathcal N} x}
\infrule[Par-barb]{\mbox{$P\downarrow_{\mathcal N} x$ or $Q\downarrow_{\mathcal N} x$}}
		  {\binpar{P}{Q} \downarrow_{\mathcal N} x}

We write $P \Downarrow_{\mathcal N} x$ if there is $Q$ such that 
$P \wred Q$ and $Q \downarrow_{\mathcal N} x$.
\end{definition}

\begin{definition}
%\label{def.bbisim}
An  ${\mathcal N}$-\emph{barbed bisimulation} over a set of names, ${\mathcal N}$, is a symmetric binary relation 
${\mathcal S}_{\mathcal N}$ between agents such that $P\rel{S}_{\mathcal N}Q$ implies:
\begin{enumerate}
\item If $P \red P'$ then $Q \wred Q'$ and $P'\rel{S}_{\mathcal N} Q'$.
\item If $P\downarrow_{\mathcal N} x$, then $Q\Downarrow_{\mathcal N} x$.
\end{enumerate}
$P$ is ${\mathcal N}$-barbed bisimilar to $Q$, written
$P \wbbisim_{\mathcal N} Q$, if $P \rel{S}_{\mathcal N} Q$ for some ${\mathcal N}$-barbed bisimulation ${\mathcal S}_{\mathcal N}$.
\end{definition}

$\mathcal{R} \subseteq \pi \times \pi$

$P \mathcal{R} Q => \forall P'. P \red P' \Rightarrow \exists Q'. Q \red Q', P' \mathcal{R} Q'$

$P \vdash x \Rightarrow Q \vdash x$

\begin{mathpar}
  \inferrule*[lab=Out-barb]{x \nameeq y}{{y}!\langle{Q}\rangle \vdash x}
  \and
  \inferrule*[lab=Par-barb]{\mbox{$P\vdash x$ or $Q\vdash x$}}{\binpar{P}{Q} \vdash x}
\end{mathpar}

\subsubsection{Contexts}

One of the principle advantages of computational calculi like the
$\pi$-calculus is a well-defined notion of context,
contextual-equivalence and a correlation between
contextual-equivalence and notions of bisimulation. The notion of
context allows the decomposition of a process into (sub-)process and
its syntactic environment, its context. Thus, a context may be
thought of as a process with a ``hole'' (written $\Box$) in it. The
application of a context $M$ to a process $P$, written $M[P]$, is
tantamount to filling the hole in $M$ with $P$. In this paper we do
not need the full weight of this theory, but do make use of the notion
of context in the proof the main theorem. 

\begin{mathpar}
  \inferrule* [lab=summation] {} {{M_{M},M_{N}} \bc \Box \;|\; x.M_{A} \;|\; M_{M}+M_{N}}
  \and
  \inferrule* [lab=agent] {} {{M_{A}} \bc (\vec{x})M_{P} \;| \; \clift{P_0,\ldots,M_{P},\ldots,P_N}}
  \and \\
  \inferrule* [lab=process] {} {{M_{P}} \bc M_{N} \;| \;P|M_{P} }
\end{mathpar} 

\begin{mathpar}
  \inferrule* [lab=sychronization] {} {M_{N} \bc \Box \;|\; x?M_{F} \;|\; x!M_{C}}
  \and
  \inferrule* [lab=abstraction] {} {{M_{F}} \bc (x)M_{P} }
  \and
  \inferrule* [lab=concretion] {} {{M_{C}} \bc \langle M_{P} \rangle }
  \and \\
  \inferrule* [lab=process] {} {{M_{P}} \bc M_{N} \;| \;P|M_{P} }
\end{mathpar}

\begin{definition}[contextual application] Given a context $M$, and
  process $P$, we define the \emph{contextual application}, $M[P] :=
  M\{P/\Box\}$. That is, the contextual application of M to P is the
  substitution of $P$ for $\Box$ in $M$.
\end{definition}

$\meaningof{-} : L \to \mathcal{P}(\pi)$

\begin{mathpar}
  \inferrule* [lab=collection] {} {\meaningof{true} = \pi, \and \meaningof{~E} = \pi \setminus \meaningof{E}, \and \meaningof{E_{1} \& E_{2}} = \meaningof{E_{1}} \cap \meaningof{E_{2}}}
\end{mathpar}

\begin{mathpar}
  \inferrule* [lab=structure] {} {\meaningof{0} = \{ P \in \pi | P \equiv 0 \}, \and \\ \meaningof{E_1 | E_2} = \{ P \in \pi | P \equiv P_{1} | P_{2}, P_{1} \in \meaningof{E_{1}}, P_{2} \in \meaningof{E_2}\} }
\end{mathpar}

\begin{mathpar}
 \inferrule* [lab=behavior] {} {\meaningof{\langle a?b \rangle E} = \{ P \in \pi | P \equiv Q | u?(y)P', \\ \and \\\\ \and \\ \;\;\; u \in \meaningof{a}, \forall z.P'\{z/y\} \in \meaningof{E\{z/b\}}\}, \and \\ \meaningof{a!E} = \{ P \in \pi | P \equiv Q | x!\langle P' \rangle, x \in \meaningof{a} P' \in \meaningof{E}\} }
\end{mathpar}

\begin{mathpar}
 \inferrule* [lab=nominal] {} {\meaningof{\quotep{E}} = \{ \quotep{P} \in \quotep{\pi} | P \in \meaningof{E} \}, \and \meaningof{\quotep{P}} = \{ \quotep{Q} \in \quotep{\pi} | P \equiv Q \} \and \\ \meaningof{@\quotep{E}} = \{ P \in \pi | P \equiv @x, x \in \meaningof{E} \}}
\end{mathpar}

\begin{eqnarray*}
  \\
  \meaningof{-} : TS \to ST
\end{eqnarray*}

\begin{eqnarray*}
  \\
  L : TS \to ST
\end{eqnarray*}

\begin{eqnarray*}
  \\
  P \models E \iff P \in \meaningof{E}
\end{eqnarray*}

\begin{eqnarray*}
  P \approx_{L} Q \iff \forall E \in L. P \models E \iff Q \models E
\end{eqnarray*}

\begin{eqnarray*}
  P \approx_{K} Q
\end{eqnarray*}

\begin{eqnarray*}
  P \approx Q
\end{eqnarray*}

$\approx_{K} = \approx = \approx_{L}$

\subsubsection{Contextual duality}

Note that contexts extend the quotation operation to a family of
operations from processes to names. Given a context, $M$, we can
define a \emph{nominal context}, $\quotep{M}$ by $\quotep{M}[P] :=
\quotep{M[P]}$. To foreshadow what is to come we observe that these
operations enjoy a duality with processes very much like the duality
between vectors and maps from vectors to scalars.

Further, because the calculus is essentially higher-order, we have a
correspondence between contexts and processes. More specifically,
given a name $x$ and a context $M$ we can construct $M^{*}_{x}$ such
that 

\begin{mathpar}
  M^{*}_{x} | \lift{x}{P} \red M[P]
\end{mathpar}

namely,

\begin{mathpar}
  M^{*}_{x} := x?(u).M[\dropn{u}]
\end{mathpar}

The dependence of $M^{*}_{x}$ on a name makes it an abstraction, 

\begin{mathpar}
  M^{*} := (x)x?(u).M[\dropn{u}]
\end{mathpar}

\subsection{Additional notation}

It will sometimes be convenient to denote the process a name
quotes. We already have the notation $x = \quotep{P}$, but it will be
convenient to introduce an alternate notation, $\procn{x}$, when we
want to emphasize the connection to the use of the name. Note that, by
virtue of name equivalence, $\quotep{\procn{x}} \nameeq x$; so, the
notation is consistent with previous definitions.

Further, because names have structure it is possible to effect
substitutions on the basis of that structure. This means we need to
upgrade our notation for substitutions, which we accomplish by
adapting comprehension notation. Thus,

\begin{mathpar}
  P\{ y / x : x \in S \}
\end{mathpar}

is interpreted to mean the process derived from P by replacing (in a
capture-avoiding manner) each occurrence of $x$ in $S$ by $y$. For example,

\begin{mathpar}
  P\{ \quotep{\procn{x}|\procn{x}} / x : x \in \freenames{P} \}
\end{mathpar}

will replace each (occurrence) of a free name $x$ in $P$ by
$\quotep{\procn{x}|\procn{x}}$.

Also, we will avail ourselves of the notation $x^{L}$ and $x^{R}$ to
denote injections of a name into disjoint copies of the name
space. There are numerous ways to accomplish this. One example can be
found in \cite{MeredithR05}. This notation overloads to vectors of
names: $\vec{x}^{\pi} := (x_{i}^{\pi} \; : \; 0 \leq i < |\vec{x}| )$ where $\pi \in \{L,R\}$.

We also use $P^{\Box} := P|\Box$.

In \cite{MeredithR05} an interpretation of the new operator is
given. It turns out that there are several possible interpretations
all enjoying the requisite algebraic properties of the operator (see
\cite{milner91polyadicpi}). We will therefore make liberal use of
$(\nu\; \vec{x})P$.

% subsection the_syntax_and_semantics_of_the_notation_system (end)   

\input{qm2pi.qmops} 

\input{qm2pi.sterngerlach} 

\input{qm2pi.metric} 

% section concurrent_process_calculi (end)

%\input{qm2pi.proofsketch}

% section proof sketch (end)

%\input{qm2pi.slviaknots} 

% section spatial logic via knots (end)

\input{qm2pi.conclusion}

% section conclusion (end)

%\input{qm2pi.dtcodes} 

% section wiring algorithm (end)

\input{qm2pi.ack} 

% section acknowledgments (end)

\newpage


\bibliographystyle{plain}   
\bibliography{../../biblios/main.bib}

\input{qm2pi.rhodetails}

\end{document}

 

% subsection basic_interpretation (end)

%\input{qm2pi.rho.presentation} 
\subsection{The syntax and semantics of the notation system}\label{sub:the_syntax_and_semantics_of_the_notation_system} % (fold)

We now summarize a technical presentation of the calculus that
embodies our theory of dynamics. The typical presentation of such a
calculus follows the style of giving generators and relations on
them. The grammar, below, describing term constructors, freely
generates the set of processes, $\Proc$. This set is then quotiented
by a relation known as structural congruence and it is over this set
that the notion of dynamics is expressed. This presentation is
essentially that of \cite{MeredithR05} with the addition of
polyadicity and summation. For readability we have relegated some of
the technical subtleties to an appendix.

\subsubsection{Process grammar}\label{subsub:process_grammar}

\begin{mathpar}
  \inferrule* [lab=synchronization] {} {{M} \bc \pzero \;|\; x?F \;|\; x!C }
  \and
  \inferrule* [lab=abstraction] {} {{F} \bc (x)P}
  \and
  \inferrule* [lab=concretion] {} {{C} \bc \langle Q \rangle}
  \and
  \inferrule* [lab=process] {} {{P,Q} \bc M \;| \;P|Q \;|\; @{x}}
  \and
  \inferrule* [lab=name] {} {{x} \bc \quotep{P}}
\end{mathpar} 

Note that $\vec{x}$ (resp. $\vec{P}$) denotes a vector of names
(resp. processes) of length $|\vec{x}|$ (resp. $|\vec{P}|$). We adopt
the following useful abbreviations.

\begin{mathpar}
   x?(\vec{y}).P := x.(\vec{y})P \and  x\clift{\vec{P}} := x.\clift{\vec{P}}
   \and x!(y) := \lift{x}{\dropn{y}}
   \and \Pi_{i=0}^{n-1}P_i := P_0 | \ldots | P_{n-1}
\end{mathpar}

\subsubsection{Structural congruence}

\paragraph{Free and bound names and alpha-equivalence.} At the
core of structural equivalence is alpha-equivalence which identifies
process that are the same up to a change of variable. Formally, we
recognize the distinction between free and bound names. The free names
of a process, $\freenames{P}$, may be calculated recursively as
follows:

\begin{mathpar}
\freenames{\pzero} := \emptyset
  \and \\
  \freenames{x?(y).P} := \{ x \} \cup (\freenames{P} \setminus \{ y \})
  \and 
  \freenames{x!\langle P \rangle} := \{ x \} \cup \{ P \} 
  \and \\
  \freenames{P|Q} := \freenames{P} \cup \freenames{Q}
  \and \\
  \freenames{@{x}} := \{ x \}
\end{mathpar}

$\pi$
$\quotep{\pi}$

$\freenames{-} : \pi \to \mathcal{P}(\quotep{\pi})$

\begin{eqnarray*}
  \freenames{\pzero} & := & \emptyset \\
  \freenames{x?(y).P} & := & \{ x \} \cup (\freenames{P} \setminus \{ y \}) \\
  \freenames{x!\langle P \rangle} & := & \{ x \} \cup \{ P \} \\
  \freenames{P|Q} & := & \freenames{P} \cup \freenames{Q} \\
  \freenames{\dropn{x}} & := & \{ x \}
\end{eqnarray*}

The bound names of a process, $\boundnames{P}$, are those names occurring in $P$
that are not free. For example, in $x?(y).0$, the name $x$ is free, while $y$ is bound.

\begin{mathpar}
  \inferrule* [lab=monoidal-laws] {} { P|Q \equiv Q|P \and P|0 \equiv P \and P|(Q|R) \equiv (P|Q)|R }
\end{mathpar}

\begin{mathpar}
  \inferrule* [lab=alpha-equivalence] {} { (x)P \equiv (y)P\{y/x\} \and y \not\in \freenames{P} }
\end{mathpar}

\begin{definition}
Then two processes, $P,Q$, are alpha-equivalent if $P = Q\{\vec{y}/\vec{x}\}$ for
some $\vec{x} \in \boundnames{Q},\vec{y} \in \boundnames{P}$, where $Q\{\vec{y}/\vec{x}\}$
denotes the capture-avoiding substitution of $\vec{y}$ for $\vec{x}$ in $Q$.
\end{definition}

\begin{definition}
  The {\em structural congruence} \cite{SangiorgiWalker} , $\equiv$,
  between processes is the least congruence containing
  alpha-equivalence, satisfying the abelian monoid laws
  (associativity, commutativity and $\pzero$ as identity) for parallel
  composition $|$ and for summation $+$.
\end{definition}

\subsection{Name equivalence}

We take name equivalence, written $\nameeq$, to be the smallest
equivalence relation generated by the following rules.

\begin{mathpar}
\inferrule*[lab=Quote-drop]
{ }
{ \quotep{@{x}} \nameeq x }

\inferrule*[lab=Struct-equiv]
{ P \scong Q }
{ \quotep{P} \nameeq \quotep{Q} }
\end{mathpar}

The astute reader will have noticed that the mutual recursion of names
and processes imposes a mutual recursion on alpha-equivalence and
structural equivalence via name-equivalence. Fortunately, all of this
works out pleasantly and we may calculate in the natural way, free of
concern. The reader interested in the details is referred to the
appendix \ref{appendix:rho_details}.

\subsection{Substitution}

We use $\Proc$ for the set of processes, $\QProc$ for the set of
names, and $\id{\{}\vec{y} / \vec{x} \id{\}}$ to denote partial maps,
$s : \QProc \rightarrow \QProc$. A map, $s$ lifts, uniquely, to a map
on process terms, $\widehat{s} : \Proc \rightarrow \Proc$ by the
following equations.

\begin{mathpar}
  (0) \psubstp{Q}{P} := 0 \\
  (R \juxtap S) \psubstp{Q}{P}
  :=    
  (R)\psubstp{Q}{P} \juxtap (S) \psubstp{Q}{P} \\
  (x?(y).R) \psubstp{Q}{P}    
  :=    
  (x)\substp{Q}{P} (z)\concat( (R \psubstn{z}{y}) \psubstp{Q}{P} ) \\
  (\lift{x}{R}) \psubstp{Q}{P}  
  :=
  \lift{(x)\substp{Q}{P}}{ R \psubstp{Q}{P} } \\
%   (\dropn{x})  \psubstp{Q}{P}       
%   := 
%   \left\{ 
%     \begin{array}{ccc} 
%       \dropn{\quotep{Q}} & & x \nameeq \quotep{P} \\
%       \dropn{x} & & otherwise \\
%     \end{array}
%   \right. 
  (\dropn{x})  \psubstp{Q}{P}       
  := 
  \left\{ 
    \begin{array}{ccc} 
      Q & & x \nameeq \quotep{P} \\
      \dropn{x} & & otherwise \\
    \end{array}
  \right.
\end{mathpar}
 

where

\begin{eqnarray}
  (x)\id{\{} \lpquote Q \rpquote / \lpquote P \rpquote \id{\}}            = 
  \left\{ 
    \begin{array}{ccc}
      \lpquote Q \rpquote & & x \nameeq \lpquote P \rpquote \\
      x & & otherwise \\
    \end{array}
  \right. \nonumber
\end{eqnarray}

and $z$ is chosen distinct from $\quotep{P}$, $\quotep{Q}$, the free
names in $Q$, and all the names in $R$. Our $\alpha$-equivalence will
be built in the standard way from this substitution.

\begin{remark}\label{rem:no_self_referential_names}
  One consequence of these definitions is that $\forall P. \quotep{P}
  \not\in \freenames{P}$.
\end{remark}

\subsection{ Dynamic quote: an example }

Anticipating something of what's to come, consider applying the
substitution, $\widehat{\id{\{}u / z \id{\}}}$, to the following pair
of processes, $\lift{w}{y!(z)}$ and $w[ \lpquote y!(z) \rpquote ]$.

\begin{eqnarray}
	\lift{w}{y!(z)}\widehat{\id{\{}u / z \id{\}}}
		& = &
		\lift{w}{y!(u)} \nonumber\\
	w[ \lpquote y!(z) \rpquote ] \widehat{ \id{\{}u / z \id{\}} }
		& = &
		w[ \lpquote y!(z) \rpquote ] \nonumber
\end{eqnarray}

Because the body of the process between quotes is impervious to
substitution, we get radically different answers. In fact, by
examining the first process in an input context,
e.g. $x?(z).\lift{w}{y!(z)}$, we see that the process under the lift
operator may be shaped by prefixed inputs binding a name inside it. In
this sense, the lift operator will be seen as a way to dynamically
construct processes before reifying them as names.

Finally equipped with these standard features we can present the
dynamics of the calculus.

\subsubsection{Operational semantics} 

Finally, we introduce the computational dynamics. What marks these
algebras as distinct from other more traditionally studied algebraic
structures, e.g. vector spaces or polynomial rings, is the manner in
which dynamics is captured. In traditional structures, dynamics is typically
expressed through morphisms between such structures, as in linear maps
between vector spaces or morphisms between rings. In algebras
associated with the semantics of computation, the dynamics is
expressed as part of the algebraic structure itself, through a
reduction reduction relation typically denoted by $\red$. Below, we
give a recursive presentation of this relation for the calculus used
in the encoding.

$\red \subseteq \pi \times \pi$
$\red : \pi \to \mathcal{P}(\pi)$

\begin{mathpar}
  \inferrule* [lab=Comm] { \textsf{match}( x_{src}, x_{trgt} ) } { x_{trgt}?(y)P \; | \; x_{src}!\langle {Q} \rangle \red P\{\quotep{Q}/y}\} }
  \and \\
  \inferrule* [lab=Par] {{P} \red {P}'} {{{P} | {Q}} \red {{P}' | {Q}}}
  \and
  \inferrule* [lab=Equiv]{{{P} \scong {P}'} \andalso {{P}' \red {Q}'} \andalso {{Q}' \scong {Q}}}{{P} \red {Q}}
\end{mathpar}

\begin{eqnarray*}
  match_{\equiv} (\quotep{P},\quotep{Q}) & := & P \equiv Q \\
  match_{\dagger}(\quotep{P},\quotep{Q}) & := & \forall R. P|Q \red^{*} R => R \red^{*} 0 \\
  match_{K}(\quotep{P},\quotep{Q}) & := & K \mbox{ for some context } K
\end{eqnarray*}

$u?(x)P | u!\langle Q \rangle \red P\{\quotep{Q}/x\}$

%We write $\wred$ for $\red^*$, and $P\red$ if $\exists Q $ such that $ P \red Q$.
We write $P\red$ if $\exists Q $ such that $ P \red Q$ and $P\not\red$, otherwise.

\section{Replication}

As mentioned before, it is known that replication (and hence
recursion) can be implemented in a higher-order process algebra
\cite{SangiorgiWalker}. As our first example of calculation with the
machinery thus far presented we give the construction explicitly in
the {\rhoc}.

\begin{eqnarray}
	D_{x} & := & \prefix{x}{y}{(\binpar{\outputp{x}{y}}{@{y}})} \nonumber\\
	\bangp_{x}{P} & := & \binpar{{x}!\langle{\binpar{D_{x}}{P}}\rangle}{D_{x}} \nonumber
\end{eqnarray}

\begin{eqnarray}
	\bangp_{x}{P} & & \nonumber\\
	=
	& {x}!\langle{(\prefix{x}{y}{(\outputp{x}{y} | @{y})) | P}}\rangle 
	      | \prefix{x}{y}{(\outputp{x}{y} | @{y})} & \nonumber\\
	\red
	& (\outputp{x}{y} | @{y})\substn{\quotep{(\prefix{x}{y}{(@{y} | \outputp{x}{y})) | P}}}{y} & \nonumber\\
	=
	& \outputp{x}{\quotep{(\prefix{x}{y}{(\outputp{x}{y} | @{y})) | P}}}
	  | {(\prefix{x}{y}{(\outputp{x}{y} | @{y})) | P}} & \nonumber\\
	\red
	& \ldots & \nonumber\\
	\red^*
	& P | P | \ldots & \nonumber
\end{eqnarray}

Of course, this encoding, as an implementation, runs away, unfolding
$\bangp{P}$ eagerly. A lazier and more implementable replication
operator, restricted to input-guarded processes, may be obtained as follows.

\begin{eqnarray}
\bangp{\prefix{u}{v}{P}} 
	:= 
	\binpar{\lift{x}{\prefix{u}{v}{(\binpar{D(x)}{P})}}}{D(x)} \nonumber
\end{eqnarray}

\begin{remark}
  Note that the lazier definition still does not deal with summation
  or mixed summation (i.e. sums over input and output). The reader is
  invited to construct definitions of replication that deal with these
  features. 

  Further, the definitions are parameterized in a name, $x$. Can you,
  gentle reader, make a definition that eliminates this parameter and
  guarantees no accidental interaction between the replication
  machinery and the process being replicated -- i.e. no accidental
  sharing of names used by the process to get its work done and the
  name(s) used by the replication to effect copying. This latter
  revision of the definition of replication is crucial to obtaining
  the expected identity $!!P \sim !P$.
\end{remark}

\begin{remark}\label{rem:paradoxical_combinator}
  The reader familiar with the lambda calculus will have noticed the
  similarity between $D$ and the paradoxical combinator.

  [Ed. note: the existence of this seems to suggest we have to be more
  restrictive on the set of processes and names we admit if we are to
  support no-cloning.]
\end{remark}

\subsubsection{Bisimulation}

The computational dynamics gives rise to another kind of equivalence,
the equivalence of computational behavior. As previously mentioned
this is typically captured \emph{via} some form of bisimulation.

% The notion we use in this paper is weak barbed bisimulation
% \cite{milner91polyadicpi}.

The notion we use in this paper is derived from weak barbed
bisimulation \cite{milner91polyadicpi}. 

\begin{definition}
An \emph{observation relation}, $\downarrow_{\mathcal N}$, over a set
of names, $\mathcal N$, is the smallest relation satisfying the rules
below.

\infrule[Out-barb]{y \in {\mathcal N}, \; x \nameeq y}
		  {\outputp{x}{v} \downarrow_{\mathcal N} x}
\infrule[Par-barb]{\mbox{$P\downarrow_{\mathcal N} x$ or $Q\downarrow_{\mathcal N} x$}}
		  {\binpar{P}{Q} \downarrow_{\mathcal N} x}

We write $P \Downarrow_{\mathcal N} x$ if there is $Q$ such that 
$P \wred Q$ and $Q \downarrow_{\mathcal N} x$.
\end{definition}

\begin{definition}
%\label{def.bbisim}
An  ${\mathcal N}$-\emph{barbed bisimulation} over a set of names, ${\mathcal N}$, is a symmetric binary relation 
${\mathcal S}_{\mathcal N}$ between agents such that $P\rel{S}_{\mathcal N}Q$ implies:
\begin{enumerate}
\item If $P \red P'$ then $Q \wred Q'$ and $P'\rel{S}_{\mathcal N} Q'$.
\item If $P\downarrow_{\mathcal N} x$, then $Q\Downarrow_{\mathcal N} x$.
\end{enumerate}
$P$ is ${\mathcal N}$-barbed bisimilar to $Q$, written
$P \wbbisim_{\mathcal N} Q$, if $P \rel{S}_{\mathcal N} Q$ for some ${\mathcal N}$-barbed bisimulation ${\mathcal S}_{\mathcal N}$.
\end{definition}

$\mathcal{R} \subseteq \pi \times \pi$

$P \mathcal{R} Q => \forall P'. P \red P' \Rightarrow \exists Q'. Q \red Q', P' \mathcal{R} Q'$

$P \vdash x \Rightarrow Q \vdash x$

\begin{mathpar}
  \inferrule*[lab=Out-barb]{x \nameeq y}{{y}!\langle{Q}\rangle \vdash x}
  \and
  \inferrule*[lab=Par-barb]{\mbox{$P\vdash x$ or $Q\vdash x$}}{\binpar{P}{Q} \vdash x}
\end{mathpar}

\subsubsection{Contexts}

One of the principle advantages of computational calculi like the
$\pi$-calculus is a well-defined notion of context,
contextual-equivalence and a correlation between
contextual-equivalence and notions of bisimulation. The notion of
context allows the decomposition of a process into (sub-)process and
its syntactic environment, its context. Thus, a context may be
thought of as a process with a ``hole'' (written $\Box$) in it. The
application of a context $M$ to a process $P$, written $M[P]$, is
tantamount to filling the hole in $M$ with $P$. In this paper we do
not need the full weight of this theory, but do make use of the notion
of context in the proof the main theorem. 

\begin{mathpar}
  \inferrule* [lab=summation] {} {{M_{M},M_{N}} \bc \Box \;|\; x.M_{A} \;|\; M_{M}+M_{N}}
  \and
  \inferrule* [lab=agent] {} {{M_{A}} \bc (\vec{x})M_{P} \;| \; \clift{P_0,\ldots,M_{P},\ldots,P_N}}
  \and \\
  \inferrule* [lab=process] {} {{M_{P}} \bc M_{N} \;| \;P|M_{P} }
\end{mathpar} 

\begin{mathpar}
  \inferrule* [lab=sychronization] {} {M_{N} \bc \Box \;|\; x?M_{F} \;|\; x!M_{C}}
  \and
  \inferrule* [lab=abstraction] {} {{M_{F}} \bc (x)M_{P} }
  \and
  \inferrule* [lab=concretion] {} {{M_{C}} \bc \langle M_{P} \rangle }
  \and \\
  \inferrule* [lab=process] {} {{M_{P}} \bc M_{N} \;| \;P|M_{P} }
\end{mathpar}

\begin{definition}[contextual application] Given a context $M$, and
  process $P$, we define the \emph{contextual application}, $M[P] :=
  M\{P/\Box\}$. That is, the contextual application of M to P is the
  substitution of $P$ for $\Box$ in $M$.
\end{definition}

$\meaningof{-} : L \to \mathcal{P}(\pi)$

\begin{mathpar}
  \inferrule* [lab=collection] {} {\meaningof{true} = \pi, \and \meaningof{~E} = \pi \setminus \meaningof{E}, \and \meaningof{E_{1} \& E_{2}} = \meaningof{E_{1}} \cap \meaningof{E_{2}}}
\end{mathpar}

\begin{mathpar}
  \inferrule* [lab=structure] {} {\meaningof{0} = \{ P \in \pi | P \equiv 0 \}, \and \\ \meaningof{E_1 | E_2} = \{ P \in \pi | P \equiv P_{1} | P_{2}, P_{1} \in \meaningof{E_{1}}, P_{2} \in \meaningof{E_2}\} }
\end{mathpar}

\begin{mathpar}
 \inferrule* [lab=behavior] {} {\meaningof{\langle a?b \rangle E} = \{ P \in \pi | P \equiv Q | u?(y)P', \\ \and \\\\ \and \\ \;\;\; u \in \meaningof{a}, \forall z.P'\{z/y\} \in \meaningof{E\{z/b\}}\}, \and \\ \meaningof{a!E} = \{ P \in \pi | P \equiv Q | x!\langle P' \rangle, x \in \meaningof{a} P' \in \meaningof{E}\} }
\end{mathpar}

\begin{mathpar}
 \inferrule* [lab=nominal] {} {\meaningof{\quotep{E}} = \{ \quotep{P} \in \quotep{\pi} | P \in \meaningof{E} \}, \and \meaningof{\quotep{P}} = \{ \quotep{Q} \in \quotep{\pi} | P \equiv Q \} \and \\ \meaningof{@\quotep{E}} = \{ P \in \pi | P \equiv @x, x \in \meaningof{E} \}}
\end{mathpar}

\begin{eqnarray*}
  \\
  \meaningof{-} : TS \to ST
\end{eqnarray*}

\begin{eqnarray*}
  \\
  L : TS \to ST
\end{eqnarray*}

\begin{eqnarray*}
  \\
  P \models E \iff P \in \meaningof{E}
\end{eqnarray*}

\begin{eqnarray*}
  P \approx_{L} Q \iff \forall E \in L. P \models E \iff Q \models E
\end{eqnarray*}

\begin{eqnarray*}
  P \approx_{K} Q
\end{eqnarray*}

\begin{eqnarray*}
  P \approx Q
\end{eqnarray*}

$\approx_{K} = \approx = \approx_{L}$

\subsubsection{Contextual duality}

Note that contexts extend the quotation operation to a family of
operations from processes to names. Given a context, $M$, we can
define a \emph{nominal context}, $\quotep{M}$ by $\quotep{M}[P] :=
\quotep{M[P]}$. To foreshadow what is to come we observe that these
operations enjoy a duality with processes very much like the duality
between vectors and maps from vectors to scalars.

Further, because the calculus is essentially higher-order, we have a
correspondence between contexts and processes. More specifically,
given a name $x$ and a context $M$ we can construct $M^{*}_{x}$ such
that 

\begin{mathpar}
  M^{*}_{x} | \lift{x}{P} \red M[P]
\end{mathpar}

namely,

\begin{mathpar}
  M^{*}_{x} := x?(u).M[\dropn{u}]
\end{mathpar}

The dependence of $M^{*}_{x}$ on a name makes it an abstraction, 

\begin{mathpar}
  M^{*} := (x)x?(u).M[\dropn{u}]
\end{mathpar}

\subsection{Additional notation}

It will sometimes be convenient to denote the process a name
quotes. We already have the notation $x = \quotep{P}$, but it will be
convenient to introduce an alternate notation, $\procn{x}$, when we
want to emphasize the connection to the use of the name. Note that, by
virtue of name equivalence, $\quotep{\procn{x}} \nameeq x$; so, the
notation is consistent with previous definitions.

Further, because names have structure it is possible to effect
substitutions on the basis of that structure. This means we need to
upgrade our notation for substitutions, which we accomplish by
adapting comprehension notation. Thus,

\begin{mathpar}
  P\{ y / x : x \in S \}
\end{mathpar}

is interpreted to mean the process derived from P by replacing (in a
capture-avoiding manner) each occurrence of $x$ in $S$ by $y$. For example,

\begin{mathpar}
  P\{ \quotep{\procn{x}|\procn{x}} / x : x \in \freenames{P} \}
\end{mathpar}

will replace each (occurrence) of a free name $x$ in $P$ by
$\quotep{\procn{x}|\procn{x}}$.

Also, we will avail ourselves of the notation $x^{L}$ and $x^{R}$ to
denote injections of a name into disjoint copies of the name
space. There are numerous ways to accomplish this. One example can be
found in \cite{MeredithR05}. This notation overloads to vectors of
names: $\vec{x}^{\pi} := (x_{i}^{\pi} \; : \; 0 \leq i < |\vec{x}| )$ where $\pi \in \{L,R\}$.

We also use $P^{\Box} := P|\Box$.

In \cite{MeredithR05} an interpretation of the new operator is
given. It turns out that there are several possible interpretations
all enjoying the requisite algebraic properties of the operator (see
\cite{milner91polyadicpi}). We will therefore make liberal use of
$(\nu\; \vec{x})P$.

% subsection the_syntax_and_semantics_of_the_notation_system (end)   

\section{Interpretation of QM}
\subsection{Supporting definitions}
\subsubsection{Multiplication}
\begin{mathpar}
  \quotep{Q} \cdot \quotep{R} := \quotep{Q|R}
  \and \\
  \quotep{Q} \cdot P := P\{ \quotep{Q|R} / \quotep{R} : \quotep{R} \in \freenames{P} \}
\end{mathpar}

\paragraph{Discussion}
The first line needs little explanation. The second line says that
each free name of the process is replaced with the multiplication of
that name by the scalar. Multiplication of a scalar (name) by a state
(process) results in a process all the names of which have been `moved
over' by parallel composition with the process the scalar
quotes. There is a subtlety that the bound names have to be
manipulated so that multiplied names aren't accidentally
captured. There are many ways to achieve this.

\begin{remark}\label{rem:multiplication_identities}
  The reader is invited to verify that for all $x,y,z \in \QProc$ and $P \in \Proc$
  \begin{mathpar}
    x \cdot \quotep{0} \equiv x 
    \and
    x \cdot y \equiv y \cdot x
    \and
    x \cdot (y \cdot z) \equiv (x \cdot y) \cdot z
    \and \\
    \quotep{0} \cdot P \equiv P
    \and \\
    x \cdot (y \cdot P) \equiv (x \cdot y) \cdot P
    \and \\
    x \cdot (P|Q) \equiv (x \cdot P) | (x \cdot Q)
    \and \\    
  \end{mathpar}
\end{remark}

\subsubsection{Tensor product}

We define a tensor product on processes by structural induction.

\paragraph{Tensor of sums} First note that all summations, including
$\pzero$ and sequence, can be written $\Sigma_{i} x_{i}.A_{i} +
\Sigma_{j} x_{j}.C_{j}$, where we have grouped input-guarded processes
together and output-guarded processes together.

Thus, we can define the tensor product of two summations, $N_{1}\otimes N_{2}$, where

\begin{mathpar}
  N_{1} := \Sigma_{i} x_{i}.A_{i} + \Sigma_{j} x_{j}.C_{j}
  \and
  N_{2} := \Sigma_{i'} y_{i'}.B_{i'} + \Sigma_{j'} y_{j'}.D_{j'} 
\end{mathpar}

as follows.

\begin{mathpar}
  \Sigma_{i} x_{i}.A_{i} + \Sigma_{j} x_{j}.C_{j} \otimes \Sigma_{i'}
  y_{i'}.B_{i'} + \Sigma_{j'} y_{j'}.D_{j'} 
  \and \\
  := \; \Sigma_{i} \Sigma_{i'} \quotep{\stackrel{\vee}{x_{i}}| \stackrel{\vee}{y_{i'}}}.(A_{i}\otimes B_{i'}) \; | \; \Sigma_{i'} \Sigma_{i} \quotep{\stackrel{\vee}{y_{i'}}|\stackrel{\vee}{x_{i}}}.(B_{i'}\otimes A_{i})
  \and
  \;\; | \;\; \Sigma_{j} \Sigma_{j'} \quotep{\stackrel{\vee}{x_{j}}|\stackrel{\vee}{y_{j'}}}.(A_{j}\otimes B_{j'}) \; | \; \Sigma_{j'} \Sigma_{j} \quotep{\stackrel{\vee}{y_{j'}}|\stackrel{\vee}{x_{j}}}.(B_{j'}\otimes A_{j})
\end{mathpar}

\begin{remark}
  Do we need to $x^{L}$ and $y^{R}$ for this construction as well?
\end{remark}

\paragraph{Tensor of parallel compositions} Next, we distribute tensor
over par.

\begin{mathpar}
  P_{1}|P_{2} \otimes Q_{1}|Q_{2} := (P_{1} \otimes Q_{1}) | (P_{1}
  \otimes Q_{2}) | (P_{2} \otimes Q_{1}) | (P_{2} \otimes Q_{2})
\end{mathpar}

\paragraph{Tensor with dropped names} We treat tensor of a
process with a dropped name as parallel composition.

\begin{mathpar}
  P \otimes \dropn{x} := P | \dropn{x}
\end{mathpar}

\paragraph{Tensor of agents}

Finally, we need to define tensor on agents. Note that the definition
of tensor on normal products only tensors inputs with inputs and
outputs with outputs. Thus, we only have to define the operation on
``homogeneous'' pairings.

\begin{mathpar}
  (\vec{x})P \otimes (\vec{y})Q
  \and \\
  := (x_{0}^{L}|y_{0}^{R},\ldots,x_{0}^{L}|y_{n}^{R},\ldots,x_{m}^{L}|y_{0}^{R},\ldots,x_{m}^{L}|y_{n}^R)(P\{ \vec{x}^{L}/\vec{x}\} \otimes Q \{ \vec{y}^{R}/\vec{y}\})
  \and \\
  \clift{\vec{P}} \otimes \clift{\vec{Q}}
  \and \\
  := \clift{P_{0}\otimes Q_{0},\ldots,P_{0}\otimes Q_{n},\ldots,P_{m}\otimes Q_{0},\ldots,P_{m}\otimes Q_{n}}
\end{mathpar}

\begin{remark}
  Observe that arities of tensored abstractions matches arities of
  tensored concretions if the original arities matched. Note also that
  the length of the arities corresponds to the increase in dimension
  we see in ordinary vector space tensor product.
\end{remark}

\begin{remark}
  Operationally, this definition distributes the tensor down to
  components ``linked'' by summation. Tensor over summation is
  intriguing in that it mixes names. Moreover, as a consequence of the
  way it mixes names we have the identities for all $x \in \QProc$ and
  $P,Q \in \Proc$

  \begin{mathpar}
    (x \cdot P) \otimes Q \equiv x \cdot (P \otimes Q) \equiv P \otimes (x \cdot Q)
    \and
    P \otimes \pzero \equiv P
  \end{mathpar}

  that the reader is invited to verify.
\end{remark}

\subsubsection{Annihilation}
\begin{mathpar}
  P^{\perp} := \{ Q | \forall R. P|Q \red^{*} R \Rightarrow R \red^{*} \pzero \}
  \and \\
  P^{\underline{\perp}} := \Sigma_{Q \in P^{\perp}} \quotep{Q}?(y).(\dropn{y}|Q) | \Sigma_{Q \in P^{\perp}} \quotep{Q}\clift{\Box}
\end{mathpar}

\paragraph{Discussion} The reader will note that $P^{\perp}$ is a
\emph{set} of processes, while $P^{\underline{\perp}}$ is a
\emph{context}. We call the set $P^{\perp}$ the \emph{annihilators} of
$P$. The parallel composition of a process in the annihilators of $P$
with $P$ will result in a process, the state space of which has all
paths eventually leading to $\pzero$. Execution may endure loops; but
under reasonable conditions of fairness (naturally guaranteed under
most notions of bisimulation) such a composite process cannot get
stuck in such a loop and will, eventually pop out and terminate.

The context $P^{\underline{\perp}}$ is ready and willing to ``take the
$P$ out of'' the process to which it is applied. It will effectively
transmit the code of the process to which it is applied to one of the
annihilators and run the process against it.

\subsubsection{Evaluation}
We fix $M$ a domain of fully abstract interpretation with an equality
coincident with bisimulation. We take $\meaningof{\cdot} : \Proc \to
M$ to be the map interpreting processes and $\nmeaningof{\cdot} : \M
\to Proc$ to be the map running the other way. Then we define

\begin{mathpar}
  \int P := \nmeaningof{\meaningof{P}}
\end{mathpar}

\paragraph{Discussion}
There are many fully abstract interpretations of Milner's
$\pi$-calculus. Any of them can be used as a basis for interpreting
the reflective calculus here. Equipped with such a domain it is
largely a matter of grinding through to check that the Yoneda
construction for the normalization-by-evaluation program can be
extended to this setting.

\begin{remark}
  The reader is invited to verify that $\int (P^{\underline{\perp}}[P]) = 0$.
\end{remark}

\subsection{Quantum mechanics}

Table \ref{tbl:core_qm_op_defns} gives the core operational definitions

\begin{table}[htp]\label{tbl:core_qm_op_defns}
  \center{
    \fbox{
      \begin{tabular}{c|c}
        quantum mechanics & process calculus \\
        \hline
        scalar & $x := \quotep{P}$ \\
        state vector & $\state{P} := P$ \\
        dual & $\state{P}^{*} := \event{P^{\underline{\perp}}} := \quotep{P^{\underline{\perp}}}[-]$ \\
        matrix & $ \Sigma_{\alpha} \state{P_{\alpha}}x_{\alpha}\event{Q_{\alpha}}$ \\
        vector addition & $\state{P} + \state{Q} := \state{P | Q}$ \\
        tensor product & $\state{P} \otimes \state{Q} := \state{P \otimes Q}$ \\
        inner product & $\innerprod{P}{Q} := \quotep{\int P^{\underline{\perp}}[Q]}$ \\
      \end{tabular}
    }
  }
  \caption{QM - operational definitions}
\end{table}

where

\begin{mathpar}
  \prmatrix{P}{Q} := \fprmatrix{P}{\quotep{\pzero}}{Q}
  \and
  \fprmatrix{P}{x}{Q} := (\state{P},x,\event{Q})
  \and
  (\fprmatrix{P}{x}{Q})(\state{R}) := x \cdot \innerprod{Q}{R} \cdot \state{P}
  \and
  (\fprmatrix{P}{x}{Q})(\event{R}) := x \cdot \innerprod{R}{P} \cdot \event{Q}
\end{mathpar}

\paragraph{Discussion}
As promised: vectors (aka states) are represented as processes; duals
as contextual duals; inner product definition should be compared with
standard inner product definition for ....

\begin{remark}
  Assuming $\int (P^{\underline{\perp}}[P]) = 0$, the reader is
  invited to verify that $(\fprmatrix{P}{x}{P})(\state{P}) = x \cdot \state{P}$.
\end{remark}

\begin{remark}
  The reader is invited to verify that $\innerprod{P}{Q}$ could
  equally well have been written $\quotep{\int \stackrel{\vee}{x}}$
  where $x = \event{P^{\underline{\perp}}}(Q)$.

  One of the motivations for this remark is that there is another way
  to factor these operations. We could package up evaluation in the dual:

  \begin{mathpar}
    \state{P}^{*} := \event{\int P^{\underline{\perp}}} := \quotep{\int P^{\underline{\perp}}}[-]
  \end{mathpar}

  and then have inner product defined by
  
  \begin{mathpar}
    \innerprod{P}{Q} := \event{P}(Q)
  \end{mathpar}

  Hopefully, experience with the calculations will provide guidance on
  the best factoring.
\end{remark}

\begin{remark}
  Assuming $\int (P^{\underline{\perp}}[P]) = 0$, the reader is
  invited to verify that $\forall P,Q. (\prmatrix{0}{Q})(\state{0}) =
  \state{0}$ and dually $(\prmatrix{P}{0})(\event{0}) = \event{0}$.
\end{remark}

\begin{remark}
  i'm a little worried that i don't (yet) have proper support for
  complex conjugacy. But, the observation above may give us a
  clue. According to Abramsky, it must be the case that the scalars
  are iso to the homset of the identity for the tensor -- which the
  observation above characterizes. 

  For now, we will simply bookmark the notion with $\overline{x}$.
\end{remark}

\subsubsection{Adjointness}

We need to give a definition of $(\cdot)^{\dagger}$ for matrices. The
obvious candidate definition is
\begin{mathpar}
(\Sigma_{\alpha}\fprmatrix{P_{\alpha}}{x_{\alpha}}{Q_{\alpha}})^{\dagger}
= \Sigma_{\alpha}\fprmatrix{(Q_{\alpha}^{\underline{\perp}})^{*}}{\overline{x}_{\alpha}}{P_{\alpha}^{\underline{\perp}}} 
\end{mathpar}

But, $(Q_{\alpha}^{\underline{\perp}})^{*}$ requires a name along
which to communicate the process to achieve the context application.

\subsubsection{Basis for a basis}
If processes label states and ``addition'' of states (a.k.a. vector
addition) is interpreted as parallel composition, what corresponds to
notions of linear independence and basis? Here, we recall that Yoshida
has developed a set of \emph{combinators} for an asynchronous verison
of Milner's $\pi$-calculus. These are a finite set of processes such
any process can be expressed as parallel composition of these
combinators together with liberal uses of the new operator and
replication. We can simply give a translation of these into the
present calculus and have reasonable expectation that the property
carries over. That is, that the resultant set allows to express all
processes via parallel composition. Note, however, that there is no
new operator or replication in this calculus. As a result, we expect
that the corresponding set is actually infinite. That is, we expect
that the space is actually infinite dimensional.

\begin{remark}
  The attentive reader may be a bit concerned. Certainly, the
  collection $S$, $K$ and $I$ is a finite set of
  combinators. Shouldn't we expect to see a finite set of combinators
  for an effectively equivalent system? i am very sympathetic to this
  critique and feel it warrants full attention. On the other hand, i
  also have in mind the following analogy. The natural numbers, as a
  monoid under addition, has exactly $1$ generator, while the natural
  numbers, as a monoid under multiplication, has countably many
  generators (the primes). We observe that the application of the
  lambda calculus is much less resource sensitive than the parallel
  composition of the $\pi$-calculus. Could it be the case that we have
  an analogy of the form
  
  \begin{mathpar}
    m + n : MN :: m*n : M|N
  \end{mathpar}

  giving a similar blow up in the set of ``primes''?  This is such a
  wonderful thought that, even if it's not true, i think it's worth
  writing down.
\end{remark}
 

\documentclass[12pt]{llncs}
%\documentclass{jktr}

\usepackage[pdftex]{hyperref}                   
\usepackage {listings}
\usepackage {mathpartir}
\usepackage{bcprules}
%\usepackage{listings}
                       
\usepackage{graphicx} 
%\usepackage[margins=2.5cm,nohead,nofoot]{geometry}
%\usepackage{geometry}
\usepackage{amsfonts}
\usepackage{amstext}
\usepackage{latexsym}
\usepackage{amssymb}
\usepackage{color}


%\include{myPreamble}
\include{qm2pi.local} 

%\ifpdf
%\usepackage[pdftex]{graphicx}
%\else
%\usepackage{graphicx}
%\fi

 % \ifpdf
%  \usepackage{pdfsync}
%  \if


%\title{Brief Article}
%\author{David F. Snyder}
%\author{L.G. Meredith}

%\address{Dept. of Math., Texas State University--San Marcos, San Marcos, TX 78666}
       
\pagestyle{empty}


\begin{document}

\lstset{language=[Objective]Caml,frame=shadowbox}

\input{qm2pi.front}

% section front matter (end)

\input{qm2pi.intro} 
 
% section introduction (end)

% \input{qm2pi.knotations} 

% section notation (end)

\input{qm2pi.process.calculi} 

% section concurrent_process_calculi_and_spatial_logics_ (end)
    
%\input{qm2pi.knots2pi} 

%\input{qm2pi.trefoil} 

%\input{qm2pi.mainthm} 

% subsection basic_interpretation (end)

%\input{qm2pi.rho.presentation} 
\subsection{The syntax and semantics of the notation system}\label{sub:the_syntax_and_semantics_of_the_notation_system} % (fold)

We now summarize a technical presentation of the calculus that
embodies our theory of dynamics. The typical presentation of such a
calculus follows the style of giving generators and relations on
them. The grammar, below, describing term constructors, freely
generates the set of processes, $\Proc$. This set is then quotiented
by a relation known as structural congruence and it is over this set
that the notion of dynamics is expressed. This presentation is
essentially that of \cite{MeredithR05} with the addition of
polyadicity and summation. For readability we have relegated some of
the technical subtleties to an appendix.

\subsubsection{Process grammar}\label{subsub:process_grammar}

\begin{mathpar}
  \inferrule* [lab=synchronization] {} {{M} \bc \pzero \;|\; x?F \;|\; x!C }
  \and
  \inferrule* [lab=abstraction] {} {{F} \bc (x)P}
  \and
  \inferrule* [lab=concretion] {} {{C} \bc \langle Q \rangle}
  \and
  \inferrule* [lab=process] {} {{P,Q} \bc M \;| \;P|Q \;|\; @{x}}
  \and
  \inferrule* [lab=name] {} {{x} \bc \quotep{P}}
\end{mathpar} 

Note that $\vec{x}$ (resp. $\vec{P}$) denotes a vector of names
(resp. processes) of length $|\vec{x}|$ (resp. $|\vec{P}|$). We adopt
the following useful abbreviations.

\begin{mathpar}
   x?(\vec{y}).P := x.(\vec{y})P \and  x\clift{\vec{P}} := x.\clift{\vec{P}}
   \and x!(y) := \lift{x}{\dropn{y}}
   \and \Pi_{i=0}^{n-1}P_i := P_0 | \ldots | P_{n-1}
\end{mathpar}

\subsubsection{Structural congruence}

\paragraph{Free and bound names and alpha-equivalence.} At the
core of structural equivalence is alpha-equivalence which identifies
process that are the same up to a change of variable. Formally, we
recognize the distinction between free and bound names. The free names
of a process, $\freenames{P}$, may be calculated recursively as
follows:

\begin{mathpar}
\freenames{\pzero} := \emptyset
  \and \\
  \freenames{x?(y).P} := \{ x \} \cup (\freenames{P} \setminus \{ y \})
  \and 
  \freenames{x!\langle P \rangle} := \{ x \} \cup \{ P \} 
  \and \\
  \freenames{P|Q} := \freenames{P} \cup \freenames{Q}
  \and \\
  \freenames{@{x}} := \{ x \}
\end{mathpar}

$\pi$
$\quotep{\pi}$

$\freenames{-} : \pi \to \mathcal{P}(\quotep{\pi})$

\begin{eqnarray*}
  \freenames{\pzero} & := & \emptyset \\
  \freenames{x?(y).P} & := & \{ x \} \cup (\freenames{P} \setminus \{ y \}) \\
  \freenames{x!\langle P \rangle} & := & \{ x \} \cup \{ P \} \\
  \freenames{P|Q} & := & \freenames{P} \cup \freenames{Q} \\
  \freenames{\dropn{x}} & := & \{ x \}
\end{eqnarray*}

The bound names of a process, $\boundnames{P}$, are those names occurring in $P$
that are not free. For example, in $x?(y).0$, the name $x$ is free, while $y$ is bound.

\begin{mathpar}
  \inferrule* [lab=monoidal-laws] {} { P|Q \equiv Q|P \and P|0 \equiv P \and P|(Q|R) \equiv (P|Q)|R }
\end{mathpar}

\begin{mathpar}
  \inferrule* [lab=alpha-equivalence] {} { (x)P \equiv (y)P\{y/x\} \and y \not\in \freenames{P} }
\end{mathpar}

\begin{definition}
Then two processes, $P,Q$, are alpha-equivalent if $P = Q\{\vec{y}/\vec{x}\}$ for
some $\vec{x} \in \boundnames{Q},\vec{y} \in \boundnames{P}$, where $Q\{\vec{y}/\vec{x}\}$
denotes the capture-avoiding substitution of $\vec{y}$ for $\vec{x}$ in $Q$.
\end{definition}

\begin{definition}
  The {\em structural congruence} \cite{SangiorgiWalker} , $\equiv$,
  between processes is the least congruence containing
  alpha-equivalence, satisfying the abelian monoid laws
  (associativity, commutativity and $\pzero$ as identity) for parallel
  composition $|$ and for summation $+$.
\end{definition}

\subsection{Name equivalence}

We take name equivalence, written $\nameeq$, to be the smallest
equivalence relation generated by the following rules.

\begin{mathpar}
\inferrule*[lab=Quote-drop]
{ }
{ \quotep{@{x}} \nameeq x }

\inferrule*[lab=Struct-equiv]
{ P \scong Q }
{ \quotep{P} \nameeq \quotep{Q} }
\end{mathpar}

The astute reader will have noticed that the mutual recursion of names
and processes imposes a mutual recursion on alpha-equivalence and
structural equivalence via name-equivalence. Fortunately, all of this
works out pleasantly and we may calculate in the natural way, free of
concern. The reader interested in the details is referred to the
appendix \ref{appendix:rho_details}.

\subsection{Substitution}

We use $\Proc$ for the set of processes, $\QProc$ for the set of
names, and $\id{\{}\vec{y} / \vec{x} \id{\}}$ to denote partial maps,
$s : \QProc \rightarrow \QProc$. A map, $s$ lifts, uniquely, to a map
on process terms, $\widehat{s} : \Proc \rightarrow \Proc$ by the
following equations.

\begin{mathpar}
  (0) \psubstp{Q}{P} := 0 \\
  (R \juxtap S) \psubstp{Q}{P}
  :=    
  (R)\psubstp{Q}{P} \juxtap (S) \psubstp{Q}{P} \\
  (x?(y).R) \psubstp{Q}{P}    
  :=    
  (x)\substp{Q}{P} (z)\concat( (R \psubstn{z}{y}) \psubstp{Q}{P} ) \\
  (\lift{x}{R}) \psubstp{Q}{P}  
  :=
  \lift{(x)\substp{Q}{P}}{ R \psubstp{Q}{P} } \\
%   (\dropn{x})  \psubstp{Q}{P}       
%   := 
%   \left\{ 
%     \begin{array}{ccc} 
%       \dropn{\quotep{Q}} & & x \nameeq \quotep{P} \\
%       \dropn{x} & & otherwise \\
%     \end{array}
%   \right. 
  (\dropn{x})  \psubstp{Q}{P}       
  := 
  \left\{ 
    \begin{array}{ccc} 
      Q & & x \nameeq \quotep{P} \\
      \dropn{x} & & otherwise \\
    \end{array}
  \right.
\end{mathpar}
 

where

\begin{eqnarray}
  (x)\id{\{} \lpquote Q \rpquote / \lpquote P \rpquote \id{\}}            = 
  \left\{ 
    \begin{array}{ccc}
      \lpquote Q \rpquote & & x \nameeq \lpquote P \rpquote \\
      x & & otherwise \\
    \end{array}
  \right. \nonumber
\end{eqnarray}

and $z$ is chosen distinct from $\quotep{P}$, $\quotep{Q}$, the free
names in $Q$, and all the names in $R$. Our $\alpha$-equivalence will
be built in the standard way from this substitution.

\begin{remark}\label{rem:no_self_referential_names}
  One consequence of these definitions is that $\forall P. \quotep{P}
  \not\in \freenames{P}$.
\end{remark}

\subsection{ Dynamic quote: an example }

Anticipating something of what's to come, consider applying the
substitution, $\widehat{\id{\{}u / z \id{\}}}$, to the following pair
of processes, $\lift{w}{y!(z)}$ and $w[ \lpquote y!(z) \rpquote ]$.

\begin{eqnarray}
	\lift{w}{y!(z)}\widehat{\id{\{}u / z \id{\}}}
		& = &
		\lift{w}{y!(u)} \nonumber\\
	w[ \lpquote y!(z) \rpquote ] \widehat{ \id{\{}u / z \id{\}} }
		& = &
		w[ \lpquote y!(z) \rpquote ] \nonumber
\end{eqnarray}

Because the body of the process between quotes is impervious to
substitution, we get radically different answers. In fact, by
examining the first process in an input context,
e.g. $x?(z).\lift{w}{y!(z)}$, we see that the process under the lift
operator may be shaped by prefixed inputs binding a name inside it. In
this sense, the lift operator will be seen as a way to dynamically
construct processes before reifying them as names.

Finally equipped with these standard features we can present the
dynamics of the calculus.

\subsubsection{Operational semantics} 

Finally, we introduce the computational dynamics. What marks these
algebras as distinct from other more traditionally studied algebraic
structures, e.g. vector spaces or polynomial rings, is the manner in
which dynamics is captured. In traditional structures, dynamics is typically
expressed through morphisms between such structures, as in linear maps
between vector spaces or morphisms between rings. In algebras
associated with the semantics of computation, the dynamics is
expressed as part of the algebraic structure itself, through a
reduction reduction relation typically denoted by $\red$. Below, we
give a recursive presentation of this relation for the calculus used
in the encoding.

$\red \subseteq \pi \times \pi$
$\red : \pi \to \mathcal{P}(\pi)$

\begin{mathpar}
  \inferrule* [lab=Comm] { \textsf{match}( x_{src}, x_{trgt} ) } { x_{trgt}?(y)P \; | \; x_{src}!\langle {Q} \rangle \red P\{\quotep{Q}/y}\} }
  \and \\
  \inferrule* [lab=Par] {{P} \red {P}'} {{{P} | {Q}} \red {{P}' | {Q}}}
  \and
  \inferrule* [lab=Equiv]{{{P} \scong {P}'} \andalso {{P}' \red {Q}'} \andalso {{Q}' \scong {Q}}}{{P} \red {Q}}
\end{mathpar}

\begin{eqnarray*}
  match_{\equiv} (\quotep{P},\quotep{Q}) & := & P \equiv Q \\
  match_{\dagger}(\quotep{P},\quotep{Q}) & := & \forall R. P|Q \red^{*} R => R \red^{*} 0 \\
  match_{K}(\quotep{P},\quotep{Q}) & := & K \mbox{ for some context } K
\end{eqnarray*}

$u?(x)P | u!\langle Q \rangle \red P\{\quotep{Q}/x\}$

%We write $\wred$ for $\red^*$, and $P\red$ if $\exists Q $ such that $ P \red Q$.
We write $P\red$ if $\exists Q $ such that $ P \red Q$ and $P\not\red$, otherwise.

\section{Replication}

As mentioned before, it is known that replication (and hence
recursion) can be implemented in a higher-order process algebra
\cite{SangiorgiWalker}. As our first example of calculation with the
machinery thus far presented we give the construction explicitly in
the {\rhoc}.

\begin{eqnarray}
	D_{x} & := & \prefix{x}{y}{(\binpar{\outputp{x}{y}}{@{y}})} \nonumber\\
	\bangp_{x}{P} & := & \binpar{{x}!\langle{\binpar{D_{x}}{P}}\rangle}{D_{x}} \nonumber
\end{eqnarray}

\begin{eqnarray}
	\bangp_{x}{P} & & \nonumber\\
	=
	& {x}!\langle{(\prefix{x}{y}{(\outputp{x}{y} | @{y})) | P}}\rangle 
	      | \prefix{x}{y}{(\outputp{x}{y} | @{y})} & \nonumber\\
	\red
	& (\outputp{x}{y} | @{y})\substn{\quotep{(\prefix{x}{y}{(@{y} | \outputp{x}{y})) | P}}}{y} & \nonumber\\
	=
	& \outputp{x}{\quotep{(\prefix{x}{y}{(\outputp{x}{y} | @{y})) | P}}}
	  | {(\prefix{x}{y}{(\outputp{x}{y} | @{y})) | P}} & \nonumber\\
	\red
	& \ldots & \nonumber\\
	\red^*
	& P | P | \ldots & \nonumber
\end{eqnarray}

Of course, this encoding, as an implementation, runs away, unfolding
$\bangp{P}$ eagerly. A lazier and more implementable replication
operator, restricted to input-guarded processes, may be obtained as follows.

\begin{eqnarray}
\bangp{\prefix{u}{v}{P}} 
	:= 
	\binpar{\lift{x}{\prefix{u}{v}{(\binpar{D(x)}{P})}}}{D(x)} \nonumber
\end{eqnarray}

\begin{remark}
  Note that the lazier definition still does not deal with summation
  or mixed summation (i.e. sums over input and output). The reader is
  invited to construct definitions of replication that deal with these
  features. 

  Further, the definitions are parameterized in a name, $x$. Can you,
  gentle reader, make a definition that eliminates this parameter and
  guarantees no accidental interaction between the replication
  machinery and the process being replicated -- i.e. no accidental
  sharing of names used by the process to get its work done and the
  name(s) used by the replication to effect copying. This latter
  revision of the definition of replication is crucial to obtaining
  the expected identity $!!P \sim !P$.
\end{remark}

\begin{remark}\label{rem:paradoxical_combinator}
  The reader familiar with the lambda calculus will have noticed the
  similarity between $D$ and the paradoxical combinator.

  [Ed. note: the existence of this seems to suggest we have to be more
  restrictive on the set of processes and names we admit if we are to
  support no-cloning.]
\end{remark}

\subsubsection{Bisimulation}

The computational dynamics gives rise to another kind of equivalence,
the equivalence of computational behavior. As previously mentioned
this is typically captured \emph{via} some form of bisimulation.

% The notion we use in this paper is weak barbed bisimulation
% \cite{milner91polyadicpi}.

The notion we use in this paper is derived from weak barbed
bisimulation \cite{milner91polyadicpi}. 

\begin{definition}
An \emph{observation relation}, $\downarrow_{\mathcal N}$, over a set
of names, $\mathcal N$, is the smallest relation satisfying the rules
below.

\infrule[Out-barb]{y \in {\mathcal N}, \; x \nameeq y}
		  {\outputp{x}{v} \downarrow_{\mathcal N} x}
\infrule[Par-barb]{\mbox{$P\downarrow_{\mathcal N} x$ or $Q\downarrow_{\mathcal N} x$}}
		  {\binpar{P}{Q} \downarrow_{\mathcal N} x}

We write $P \Downarrow_{\mathcal N} x$ if there is $Q$ such that 
$P \wred Q$ and $Q \downarrow_{\mathcal N} x$.
\end{definition}

\begin{definition}
%\label{def.bbisim}
An  ${\mathcal N}$-\emph{barbed bisimulation} over a set of names, ${\mathcal N}$, is a symmetric binary relation 
${\mathcal S}_{\mathcal N}$ between agents such that $P\rel{S}_{\mathcal N}Q$ implies:
\begin{enumerate}
\item If $P \red P'$ then $Q \wred Q'$ and $P'\rel{S}_{\mathcal N} Q'$.
\item If $P\downarrow_{\mathcal N} x$, then $Q\Downarrow_{\mathcal N} x$.
\end{enumerate}
$P$ is ${\mathcal N}$-barbed bisimilar to $Q$, written
$P \wbbisim_{\mathcal N} Q$, if $P \rel{S}_{\mathcal N} Q$ for some ${\mathcal N}$-barbed bisimulation ${\mathcal S}_{\mathcal N}$.
\end{definition}

$\mathcal{R} \subseteq \pi \times \pi$

$P \mathcal{R} Q => \forall P'. P \red P' \Rightarrow \exists Q'. Q \red Q', P' \mathcal{R} Q'$

$P \vdash x \Rightarrow Q \vdash x$

\begin{mathpar}
  \inferrule*[lab=Out-barb]{x \nameeq y}{{y}!\langle{Q}\rangle \vdash x}
  \and
  \inferrule*[lab=Par-barb]{\mbox{$P\vdash x$ or $Q\vdash x$}}{\binpar{P}{Q} \vdash x}
\end{mathpar}

\subsubsection{Contexts}

One of the principle advantages of computational calculi like the
$\pi$-calculus is a well-defined notion of context,
contextual-equivalence and a correlation between
contextual-equivalence and notions of bisimulation. The notion of
context allows the decomposition of a process into (sub-)process and
its syntactic environment, its context. Thus, a context may be
thought of as a process with a ``hole'' (written $\Box$) in it. The
application of a context $M$ to a process $P$, written $M[P]$, is
tantamount to filling the hole in $M$ with $P$. In this paper we do
not need the full weight of this theory, but do make use of the notion
of context in the proof the main theorem. 

\begin{mathpar}
  \inferrule* [lab=summation] {} {{M_{M},M_{N}} \bc \Box \;|\; x.M_{A} \;|\; M_{M}+M_{N}}
  \and
  \inferrule* [lab=agent] {} {{M_{A}} \bc (\vec{x})M_{P} \;| \; \clift{P_0,\ldots,M_{P},\ldots,P_N}}
  \and \\
  \inferrule* [lab=process] {} {{M_{P}} \bc M_{N} \;| \;P|M_{P} }
\end{mathpar} 

\begin{mathpar}
  \inferrule* [lab=sychronization] {} {M_{N} \bc \Box \;|\; x?M_{F} \;|\; x!M_{C}}
  \and
  \inferrule* [lab=abstraction] {} {{M_{F}} \bc (x)M_{P} }
  \and
  \inferrule* [lab=concretion] {} {{M_{C}} \bc \langle M_{P} \rangle }
  \and \\
  \inferrule* [lab=process] {} {{M_{P}} \bc M_{N} \;| \;P|M_{P} }
\end{mathpar}

\begin{definition}[contextual application] Given a context $M$, and
  process $P$, we define the \emph{contextual application}, $M[P] :=
  M\{P/\Box\}$. That is, the contextual application of M to P is the
  substitution of $P$ for $\Box$ in $M$.
\end{definition}

$\meaningof{-} : L \to \mathcal{P}(\pi)$

\begin{mathpar}
  \inferrule* [lab=collection] {} {\meaningof{true} = \pi, \and \meaningof{~E} = \pi \setminus \meaningof{E}, \and \meaningof{E_{1} \& E_{2}} = \meaningof{E_{1}} \cap \meaningof{E_{2}}}
\end{mathpar}

\begin{mathpar}
  \inferrule* [lab=structure] {} {\meaningof{0} = \{ P \in \pi | P \equiv 0 \}, \and \\ \meaningof{E_1 | E_2} = \{ P \in \pi | P \equiv P_{1} | P_{2}, P_{1} \in \meaningof{E_{1}}, P_{2} \in \meaningof{E_2}\} }
\end{mathpar}

\begin{mathpar}
 \inferrule* [lab=behavior] {} {\meaningof{\langle a?b \rangle E} = \{ P \in \pi | P \equiv Q | u?(y)P', \\ \and \\\\ \and \\ \;\;\; u \in \meaningof{a}, \forall z.P'\{z/y\} \in \meaningof{E\{z/b\}}\}, \and \\ \meaningof{a!E} = \{ P \in \pi | P \equiv Q | x!\langle P' \rangle, x \in \meaningof{a} P' \in \meaningof{E}\} }
\end{mathpar}

\begin{mathpar}
 \inferrule* [lab=nominal] {} {\meaningof{\quotep{E}} = \{ \quotep{P} \in \quotep{\pi} | P \in \meaningof{E} \}, \and \meaningof{\quotep{P}} = \{ \quotep{Q} \in \quotep{\pi} | P \equiv Q \} \and \\ \meaningof{@\quotep{E}} = \{ P \in \pi | P \equiv @x, x \in \meaningof{E} \}}
\end{mathpar}

\begin{eqnarray*}
  \\
  \meaningof{-} : TS \to ST
\end{eqnarray*}

\begin{eqnarray*}
  \\
  L : TS \to ST
\end{eqnarray*}

\begin{eqnarray*}
  \\
  P \models E \iff P \in \meaningof{E}
\end{eqnarray*}

\begin{eqnarray*}
  P \approx_{L} Q \iff \forall E \in L. P \models E \iff Q \models E
\end{eqnarray*}

\begin{eqnarray*}
  P \approx_{K} Q
\end{eqnarray*}

\begin{eqnarray*}
  P \approx Q
\end{eqnarray*}

$\approx_{K} = \approx = \approx_{L}$

\subsubsection{Contextual duality}

Note that contexts extend the quotation operation to a family of
operations from processes to names. Given a context, $M$, we can
define a \emph{nominal context}, $\quotep{M}$ by $\quotep{M}[P] :=
\quotep{M[P]}$. To foreshadow what is to come we observe that these
operations enjoy a duality with processes very much like the duality
between vectors and maps from vectors to scalars.

Further, because the calculus is essentially higher-order, we have a
correspondence between contexts and processes. More specifically,
given a name $x$ and a context $M$ we can construct $M^{*}_{x}$ such
that 

\begin{mathpar}
  M^{*}_{x} | \lift{x}{P} \red M[P]
\end{mathpar}

namely,

\begin{mathpar}
  M^{*}_{x} := x?(u).M[\dropn{u}]
\end{mathpar}

The dependence of $M^{*}_{x}$ on a name makes it an abstraction, 

\begin{mathpar}
  M^{*} := (x)x?(u).M[\dropn{u}]
\end{mathpar}

\subsection{Additional notation}

It will sometimes be convenient to denote the process a name
quotes. We already have the notation $x = \quotep{P}$, but it will be
convenient to introduce an alternate notation, $\procn{x}$, when we
want to emphasize the connection to the use of the name. Note that, by
virtue of name equivalence, $\quotep{\procn{x}} \nameeq x$; so, the
notation is consistent with previous definitions.

Further, because names have structure it is possible to effect
substitutions on the basis of that structure. This means we need to
upgrade our notation for substitutions, which we accomplish by
adapting comprehension notation. Thus,

\begin{mathpar}
  P\{ y / x : x \in S \}
\end{mathpar}

is interpreted to mean the process derived from P by replacing (in a
capture-avoiding manner) each occurrence of $x$ in $S$ by $y$. For example,

\begin{mathpar}
  P\{ \quotep{\procn{x}|\procn{x}} / x : x \in \freenames{P} \}
\end{mathpar}

will replace each (occurrence) of a free name $x$ in $P$ by
$\quotep{\procn{x}|\procn{x}}$.

Also, we will avail ourselves of the notation $x^{L}$ and $x^{R}$ to
denote injections of a name into disjoint copies of the name
space. There are numerous ways to accomplish this. One example can be
found in \cite{MeredithR05}. This notation overloads to vectors of
names: $\vec{x}^{\pi} := (x_{i}^{\pi} \; : \; 0 \leq i < |\vec{x}| )$ where $\pi \in \{L,R\}$.

We also use $P^{\Box} := P|\Box$.

In \cite{MeredithR05} an interpretation of the new operator is
given. It turns out that there are several possible interpretations
all enjoying the requisite algebraic properties of the operator (see
\cite{milner91polyadicpi}). We will therefore make liberal use of
$(\nu\; \vec{x})P$.

% subsection the_syntax_and_semantics_of_the_notation_system (end)   

\input{qm2pi.qmops} 

\input{qm2pi.sterngerlach} 

\input{qm2pi.metric} 

% section concurrent_process_calculi (end)

%\input{qm2pi.proofsketch}

% section proof sketch (end)

%\input{qm2pi.slviaknots} 

% section spatial logic via knots (end)

\input{qm2pi.conclusion}

% section conclusion (end)

%\input{qm2pi.dtcodes} 

% section wiring algorithm (end)

\input{qm2pi.ack} 

% section acknowledgments (end)

\newpage


\bibliographystyle{plain}   
\bibliography{../../biblios/main.bib}

\input{qm2pi.rhodetails}

\end{document}

 

\documentclass[12pt]{llncs}
%\documentclass{jktr}

\usepackage[pdftex]{hyperref}                   
\usepackage {listings}
\usepackage {mathpartir}
\usepackage{bcprules}
%\usepackage{listings}
                       
\usepackage{graphicx} 
%\usepackage[margins=2.5cm,nohead,nofoot]{geometry}
%\usepackage{geometry}
\usepackage{amsfonts}
\usepackage{amstext}
\usepackage{latexsym}
\usepackage{amssymb}
\usepackage{color}


%\include{myPreamble}
\include{qm2pi.local} 

%\ifpdf
%\usepackage[pdftex]{graphicx}
%\else
%\usepackage{graphicx}
%\fi

 % \ifpdf
%  \usepackage{pdfsync}
%  \if


%\title{Brief Article}
%\author{David F. Snyder}
%\author{L.G. Meredith}

%\address{Dept. of Math., Texas State University--San Marcos, San Marcos, TX 78666}
       
\pagestyle{empty}


\begin{document}

\lstset{language=[Objective]Caml,frame=shadowbox}

\input{qm2pi.front}

% section front matter (end)

\input{qm2pi.intro} 
 
% section introduction (end)

% \input{qm2pi.knotations} 

% section notation (end)

\input{qm2pi.process.calculi} 

% section concurrent_process_calculi_and_spatial_logics_ (end)
    
%\input{qm2pi.knots2pi} 

%\input{qm2pi.trefoil} 

%\input{qm2pi.mainthm} 

% subsection basic_interpretation (end)

%\input{qm2pi.rho.presentation} 
\subsection{The syntax and semantics of the notation system}\label{sub:the_syntax_and_semantics_of_the_notation_system} % (fold)

We now summarize a technical presentation of the calculus that
embodies our theory of dynamics. The typical presentation of such a
calculus follows the style of giving generators and relations on
them. The grammar, below, describing term constructors, freely
generates the set of processes, $\Proc$. This set is then quotiented
by a relation known as structural congruence and it is over this set
that the notion of dynamics is expressed. This presentation is
essentially that of \cite{MeredithR05} with the addition of
polyadicity and summation. For readability we have relegated some of
the technical subtleties to an appendix.

\subsubsection{Process grammar}\label{subsub:process_grammar}

\begin{mathpar}
  \inferrule* [lab=synchronization] {} {{M} \bc \pzero \;|\; x?F \;|\; x!C }
  \and
  \inferrule* [lab=abstraction] {} {{F} \bc (x)P}
  \and
  \inferrule* [lab=concretion] {} {{C} \bc \langle Q \rangle}
  \and
  \inferrule* [lab=process] {} {{P,Q} \bc M \;| \;P|Q \;|\; @{x}}
  \and
  \inferrule* [lab=name] {} {{x} \bc \quotep{P}}
\end{mathpar} 

Note that $\vec{x}$ (resp. $\vec{P}$) denotes a vector of names
(resp. processes) of length $|\vec{x}|$ (resp. $|\vec{P}|$). We adopt
the following useful abbreviations.

\begin{mathpar}
   x?(\vec{y}).P := x.(\vec{y})P \and  x\clift{\vec{P}} := x.\clift{\vec{P}}
   \and x!(y) := \lift{x}{\dropn{y}}
   \and \Pi_{i=0}^{n-1}P_i := P_0 | \ldots | P_{n-1}
\end{mathpar}

\subsubsection{Structural congruence}

\paragraph{Free and bound names and alpha-equivalence.} At the
core of structural equivalence is alpha-equivalence which identifies
process that are the same up to a change of variable. Formally, we
recognize the distinction between free and bound names. The free names
of a process, $\freenames{P}$, may be calculated recursively as
follows:

\begin{mathpar}
\freenames{\pzero} := \emptyset
  \and \\
  \freenames{x?(y).P} := \{ x \} \cup (\freenames{P} \setminus \{ y \})
  \and 
  \freenames{x!\langle P \rangle} := \{ x \} \cup \{ P \} 
  \and \\
  \freenames{P|Q} := \freenames{P} \cup \freenames{Q}
  \and \\
  \freenames{@{x}} := \{ x \}
\end{mathpar}

$\pi$
$\quotep{\pi}$

$\freenames{-} : \pi \to \mathcal{P}(\quotep{\pi})$

\begin{eqnarray*}
  \freenames{\pzero} & := & \emptyset \\
  \freenames{x?(y).P} & := & \{ x \} \cup (\freenames{P} \setminus \{ y \}) \\
  \freenames{x!\langle P \rangle} & := & \{ x \} \cup \{ P \} \\
  \freenames{P|Q} & := & \freenames{P} \cup \freenames{Q} \\
  \freenames{\dropn{x}} & := & \{ x \}
\end{eqnarray*}

The bound names of a process, $\boundnames{P}$, are those names occurring in $P$
that are not free. For example, in $x?(y).0$, the name $x$ is free, while $y$ is bound.

\begin{mathpar}
  \inferrule* [lab=monoidal-laws] {} { P|Q \equiv Q|P \and P|0 \equiv P \and P|(Q|R) \equiv (P|Q)|R }
\end{mathpar}

\begin{mathpar}
  \inferrule* [lab=alpha-equivalence] {} { (x)P \equiv (y)P\{y/x\} \and y \not\in \freenames{P} }
\end{mathpar}

\begin{definition}
Then two processes, $P,Q$, are alpha-equivalent if $P = Q\{\vec{y}/\vec{x}\}$ for
some $\vec{x} \in \boundnames{Q},\vec{y} \in \boundnames{P}$, where $Q\{\vec{y}/\vec{x}\}$
denotes the capture-avoiding substitution of $\vec{y}$ for $\vec{x}$ in $Q$.
\end{definition}

\begin{definition}
  The {\em structural congruence} \cite{SangiorgiWalker} , $\equiv$,
  between processes is the least congruence containing
  alpha-equivalence, satisfying the abelian monoid laws
  (associativity, commutativity and $\pzero$ as identity) for parallel
  composition $|$ and for summation $+$.
\end{definition}

\subsection{Name equivalence}

We take name equivalence, written $\nameeq$, to be the smallest
equivalence relation generated by the following rules.

\begin{mathpar}
\inferrule*[lab=Quote-drop]
{ }
{ \quotep{@{x}} \nameeq x }

\inferrule*[lab=Struct-equiv]
{ P \scong Q }
{ \quotep{P} \nameeq \quotep{Q} }
\end{mathpar}

The astute reader will have noticed that the mutual recursion of names
and processes imposes a mutual recursion on alpha-equivalence and
structural equivalence via name-equivalence. Fortunately, all of this
works out pleasantly and we may calculate in the natural way, free of
concern. The reader interested in the details is referred to the
appendix \ref{appendix:rho_details}.

\subsection{Substitution}

We use $\Proc$ for the set of processes, $\QProc$ for the set of
names, and $\id{\{}\vec{y} / \vec{x} \id{\}}$ to denote partial maps,
$s : \QProc \rightarrow \QProc$. A map, $s$ lifts, uniquely, to a map
on process terms, $\widehat{s} : \Proc \rightarrow \Proc$ by the
following equations.

\begin{mathpar}
  (0) \psubstp{Q}{P} := 0 \\
  (R \juxtap S) \psubstp{Q}{P}
  :=    
  (R)\psubstp{Q}{P} \juxtap (S) \psubstp{Q}{P} \\
  (x?(y).R) \psubstp{Q}{P}    
  :=    
  (x)\substp{Q}{P} (z)\concat( (R \psubstn{z}{y}) \psubstp{Q}{P} ) \\
  (\lift{x}{R}) \psubstp{Q}{P}  
  :=
  \lift{(x)\substp{Q}{P}}{ R \psubstp{Q}{P} } \\
%   (\dropn{x})  \psubstp{Q}{P}       
%   := 
%   \left\{ 
%     \begin{array}{ccc} 
%       \dropn{\quotep{Q}} & & x \nameeq \quotep{P} \\
%       \dropn{x} & & otherwise \\
%     \end{array}
%   \right. 
  (\dropn{x})  \psubstp{Q}{P}       
  := 
  \left\{ 
    \begin{array}{ccc} 
      Q & & x \nameeq \quotep{P} \\
      \dropn{x} & & otherwise \\
    \end{array}
  \right.
\end{mathpar}
 

where

\begin{eqnarray}
  (x)\id{\{} \lpquote Q \rpquote / \lpquote P \rpquote \id{\}}            = 
  \left\{ 
    \begin{array}{ccc}
      \lpquote Q \rpquote & & x \nameeq \lpquote P \rpquote \\
      x & & otherwise \\
    \end{array}
  \right. \nonumber
\end{eqnarray}

and $z$ is chosen distinct from $\quotep{P}$, $\quotep{Q}$, the free
names in $Q$, and all the names in $R$. Our $\alpha$-equivalence will
be built in the standard way from this substitution.

\begin{remark}\label{rem:no_self_referential_names}
  One consequence of these definitions is that $\forall P. \quotep{P}
  \not\in \freenames{P}$.
\end{remark}

\subsection{ Dynamic quote: an example }

Anticipating something of what's to come, consider applying the
substitution, $\widehat{\id{\{}u / z \id{\}}}$, to the following pair
of processes, $\lift{w}{y!(z)}$ and $w[ \lpquote y!(z) \rpquote ]$.

\begin{eqnarray}
	\lift{w}{y!(z)}\widehat{\id{\{}u / z \id{\}}}
		& = &
		\lift{w}{y!(u)} \nonumber\\
	w[ \lpquote y!(z) \rpquote ] \widehat{ \id{\{}u / z \id{\}} }
		& = &
		w[ \lpquote y!(z) \rpquote ] \nonumber
\end{eqnarray}

Because the body of the process between quotes is impervious to
substitution, we get radically different answers. In fact, by
examining the first process in an input context,
e.g. $x?(z).\lift{w}{y!(z)}$, we see that the process under the lift
operator may be shaped by prefixed inputs binding a name inside it. In
this sense, the lift operator will be seen as a way to dynamically
construct processes before reifying them as names.

Finally equipped with these standard features we can present the
dynamics of the calculus.

\subsubsection{Operational semantics} 

Finally, we introduce the computational dynamics. What marks these
algebras as distinct from other more traditionally studied algebraic
structures, e.g. vector spaces or polynomial rings, is the manner in
which dynamics is captured. In traditional structures, dynamics is typically
expressed through morphisms between such structures, as in linear maps
between vector spaces or morphisms between rings. In algebras
associated with the semantics of computation, the dynamics is
expressed as part of the algebraic structure itself, through a
reduction reduction relation typically denoted by $\red$. Below, we
give a recursive presentation of this relation for the calculus used
in the encoding.

$\red \subseteq \pi \times \pi$
$\red : \pi \to \mathcal{P}(\pi)$

\begin{mathpar}
  \inferrule* [lab=Comm] { \textsf{match}( x_{src}, x_{trgt} ) } { x_{trgt}?(y)P \; | \; x_{src}!\langle {Q} \rangle \red P\{\quotep{Q}/y}\} }
  \and \\
  \inferrule* [lab=Par] {{P} \red {P}'} {{{P} | {Q}} \red {{P}' | {Q}}}
  \and
  \inferrule* [lab=Equiv]{{{P} \scong {P}'} \andalso {{P}' \red {Q}'} \andalso {{Q}' \scong {Q}}}{{P} \red {Q}}
\end{mathpar}

\begin{eqnarray*}
  match_{\equiv} (\quotep{P},\quotep{Q}) & := & P \equiv Q \\
  match_{\dagger}(\quotep{P},\quotep{Q}) & := & \forall R. P|Q \red^{*} R => R \red^{*} 0 \\
  match_{K}(\quotep{P},\quotep{Q}) & := & K \mbox{ for some context } K
\end{eqnarray*}

$u?(x)P | u!\langle Q \rangle \red P\{\quotep{Q}/x\}$

%We write $\wred$ for $\red^*$, and $P\red$ if $\exists Q $ such that $ P \red Q$.
We write $P\red$ if $\exists Q $ such that $ P \red Q$ and $P\not\red$, otherwise.

\section{Replication}

As mentioned before, it is known that replication (and hence
recursion) can be implemented in a higher-order process algebra
\cite{SangiorgiWalker}. As our first example of calculation with the
machinery thus far presented we give the construction explicitly in
the {\rhoc}.

\begin{eqnarray}
	D_{x} & := & \prefix{x}{y}{(\binpar{\outputp{x}{y}}{@{y}})} \nonumber\\
	\bangp_{x}{P} & := & \binpar{{x}!\langle{\binpar{D_{x}}{P}}\rangle}{D_{x}} \nonumber
\end{eqnarray}

\begin{eqnarray}
	\bangp_{x}{P} & & \nonumber\\
	=
	& {x}!\langle{(\prefix{x}{y}{(\outputp{x}{y} | @{y})) | P}}\rangle 
	      | \prefix{x}{y}{(\outputp{x}{y} | @{y})} & \nonumber\\
	\red
	& (\outputp{x}{y} | @{y})\substn{\quotep{(\prefix{x}{y}{(@{y} | \outputp{x}{y})) | P}}}{y} & \nonumber\\
	=
	& \outputp{x}{\quotep{(\prefix{x}{y}{(\outputp{x}{y} | @{y})) | P}}}
	  | {(\prefix{x}{y}{(\outputp{x}{y} | @{y})) | P}} & \nonumber\\
	\red
	& \ldots & \nonumber\\
	\red^*
	& P | P | \ldots & \nonumber
\end{eqnarray}

Of course, this encoding, as an implementation, runs away, unfolding
$\bangp{P}$ eagerly. A lazier and more implementable replication
operator, restricted to input-guarded processes, may be obtained as follows.

\begin{eqnarray}
\bangp{\prefix{u}{v}{P}} 
	:= 
	\binpar{\lift{x}{\prefix{u}{v}{(\binpar{D(x)}{P})}}}{D(x)} \nonumber
\end{eqnarray}

\begin{remark}
  Note that the lazier definition still does not deal with summation
  or mixed summation (i.e. sums over input and output). The reader is
  invited to construct definitions of replication that deal with these
  features. 

  Further, the definitions are parameterized in a name, $x$. Can you,
  gentle reader, make a definition that eliminates this parameter and
  guarantees no accidental interaction between the replication
  machinery and the process being replicated -- i.e. no accidental
  sharing of names used by the process to get its work done and the
  name(s) used by the replication to effect copying. This latter
  revision of the definition of replication is crucial to obtaining
  the expected identity $!!P \sim !P$.
\end{remark}

\begin{remark}\label{rem:paradoxical_combinator}
  The reader familiar with the lambda calculus will have noticed the
  similarity between $D$ and the paradoxical combinator.

  [Ed. note: the existence of this seems to suggest we have to be more
  restrictive on the set of processes and names we admit if we are to
  support no-cloning.]
\end{remark}

\subsubsection{Bisimulation}

The computational dynamics gives rise to another kind of equivalence,
the equivalence of computational behavior. As previously mentioned
this is typically captured \emph{via} some form of bisimulation.

% The notion we use in this paper is weak barbed bisimulation
% \cite{milner91polyadicpi}.

The notion we use in this paper is derived from weak barbed
bisimulation \cite{milner91polyadicpi}. 

\begin{definition}
An \emph{observation relation}, $\downarrow_{\mathcal N}$, over a set
of names, $\mathcal N$, is the smallest relation satisfying the rules
below.

\infrule[Out-barb]{y \in {\mathcal N}, \; x \nameeq y}
		  {\outputp{x}{v} \downarrow_{\mathcal N} x}
\infrule[Par-barb]{\mbox{$P\downarrow_{\mathcal N} x$ or $Q\downarrow_{\mathcal N} x$}}
		  {\binpar{P}{Q} \downarrow_{\mathcal N} x}

We write $P \Downarrow_{\mathcal N} x$ if there is $Q$ such that 
$P \wred Q$ and $Q \downarrow_{\mathcal N} x$.
\end{definition}

\begin{definition}
%\label{def.bbisim}
An  ${\mathcal N}$-\emph{barbed bisimulation} over a set of names, ${\mathcal N}$, is a symmetric binary relation 
${\mathcal S}_{\mathcal N}$ between agents such that $P\rel{S}_{\mathcal N}Q$ implies:
\begin{enumerate}
\item If $P \red P'$ then $Q \wred Q'$ and $P'\rel{S}_{\mathcal N} Q'$.
\item If $P\downarrow_{\mathcal N} x$, then $Q\Downarrow_{\mathcal N} x$.
\end{enumerate}
$P$ is ${\mathcal N}$-barbed bisimilar to $Q$, written
$P \wbbisim_{\mathcal N} Q$, if $P \rel{S}_{\mathcal N} Q$ for some ${\mathcal N}$-barbed bisimulation ${\mathcal S}_{\mathcal N}$.
\end{definition}

$\mathcal{R} \subseteq \pi \times \pi$

$P \mathcal{R} Q => \forall P'. P \red P' \Rightarrow \exists Q'. Q \red Q', P' \mathcal{R} Q'$

$P \vdash x \Rightarrow Q \vdash x$

\begin{mathpar}
  \inferrule*[lab=Out-barb]{x \nameeq y}{{y}!\langle{Q}\rangle \vdash x}
  \and
  \inferrule*[lab=Par-barb]{\mbox{$P\vdash x$ or $Q\vdash x$}}{\binpar{P}{Q} \vdash x}
\end{mathpar}

\subsubsection{Contexts}

One of the principle advantages of computational calculi like the
$\pi$-calculus is a well-defined notion of context,
contextual-equivalence and a correlation between
contextual-equivalence and notions of bisimulation. The notion of
context allows the decomposition of a process into (sub-)process and
its syntactic environment, its context. Thus, a context may be
thought of as a process with a ``hole'' (written $\Box$) in it. The
application of a context $M$ to a process $P$, written $M[P]$, is
tantamount to filling the hole in $M$ with $P$. In this paper we do
not need the full weight of this theory, but do make use of the notion
of context in the proof the main theorem. 

\begin{mathpar}
  \inferrule* [lab=summation] {} {{M_{M},M_{N}} \bc \Box \;|\; x.M_{A} \;|\; M_{M}+M_{N}}
  \and
  \inferrule* [lab=agent] {} {{M_{A}} \bc (\vec{x})M_{P} \;| \; \clift{P_0,\ldots,M_{P},\ldots,P_N}}
  \and \\
  \inferrule* [lab=process] {} {{M_{P}} \bc M_{N} \;| \;P|M_{P} }
\end{mathpar} 

\begin{mathpar}
  \inferrule* [lab=sychronization] {} {M_{N} \bc \Box \;|\; x?M_{F} \;|\; x!M_{C}}
  \and
  \inferrule* [lab=abstraction] {} {{M_{F}} \bc (x)M_{P} }
  \and
  \inferrule* [lab=concretion] {} {{M_{C}} \bc \langle M_{P} \rangle }
  \and \\
  \inferrule* [lab=process] {} {{M_{P}} \bc M_{N} \;| \;P|M_{P} }
\end{mathpar}

\begin{definition}[contextual application] Given a context $M$, and
  process $P$, we define the \emph{contextual application}, $M[P] :=
  M\{P/\Box\}$. That is, the contextual application of M to P is the
  substitution of $P$ for $\Box$ in $M$.
\end{definition}

$\meaningof{-} : L \to \mathcal{P}(\pi)$

\begin{mathpar}
  \inferrule* [lab=collection] {} {\meaningof{true} = \pi, \and \meaningof{~E} = \pi \setminus \meaningof{E}, \and \meaningof{E_{1} \& E_{2}} = \meaningof{E_{1}} \cap \meaningof{E_{2}}}
\end{mathpar}

\begin{mathpar}
  \inferrule* [lab=structure] {} {\meaningof{0} = \{ P \in \pi | P \equiv 0 \}, \and \\ \meaningof{E_1 | E_2} = \{ P \in \pi | P \equiv P_{1} | P_{2}, P_{1} \in \meaningof{E_{1}}, P_{2} \in \meaningof{E_2}\} }
\end{mathpar}

\begin{mathpar}
 \inferrule* [lab=behavior] {} {\meaningof{\langle a?b \rangle E} = \{ P \in \pi | P \equiv Q | u?(y)P', \\ \and \\\\ \and \\ \;\;\; u \in \meaningof{a}, \forall z.P'\{z/y\} \in \meaningof{E\{z/b\}}\}, \and \\ \meaningof{a!E} = \{ P \in \pi | P \equiv Q | x!\langle P' \rangle, x \in \meaningof{a} P' \in \meaningof{E}\} }
\end{mathpar}

\begin{mathpar}
 \inferrule* [lab=nominal] {} {\meaningof{\quotep{E}} = \{ \quotep{P} \in \quotep{\pi} | P \in \meaningof{E} \}, \and \meaningof{\quotep{P}} = \{ \quotep{Q} \in \quotep{\pi} | P \equiv Q \} \and \\ \meaningof{@\quotep{E}} = \{ P \in \pi | P \equiv @x, x \in \meaningof{E} \}}
\end{mathpar}

\begin{eqnarray*}
  \\
  \meaningof{-} : TS \to ST
\end{eqnarray*}

\begin{eqnarray*}
  \\
  L : TS \to ST
\end{eqnarray*}

\begin{eqnarray*}
  \\
  P \models E \iff P \in \meaningof{E}
\end{eqnarray*}

\begin{eqnarray*}
  P \approx_{L} Q \iff \forall E \in L. P \models E \iff Q \models E
\end{eqnarray*}

\begin{eqnarray*}
  P \approx_{K} Q
\end{eqnarray*}

\begin{eqnarray*}
  P \approx Q
\end{eqnarray*}

$\approx_{K} = \approx = \approx_{L}$

\subsubsection{Contextual duality}

Note that contexts extend the quotation operation to a family of
operations from processes to names. Given a context, $M$, we can
define a \emph{nominal context}, $\quotep{M}$ by $\quotep{M}[P] :=
\quotep{M[P]}$. To foreshadow what is to come we observe that these
operations enjoy a duality with processes very much like the duality
between vectors and maps from vectors to scalars.

Further, because the calculus is essentially higher-order, we have a
correspondence between contexts and processes. More specifically,
given a name $x$ and a context $M$ we can construct $M^{*}_{x}$ such
that 

\begin{mathpar}
  M^{*}_{x} | \lift{x}{P} \red M[P]
\end{mathpar}

namely,

\begin{mathpar}
  M^{*}_{x} := x?(u).M[\dropn{u}]
\end{mathpar}

The dependence of $M^{*}_{x}$ on a name makes it an abstraction, 

\begin{mathpar}
  M^{*} := (x)x?(u).M[\dropn{u}]
\end{mathpar}

\subsection{Additional notation}

It will sometimes be convenient to denote the process a name
quotes. We already have the notation $x = \quotep{P}$, but it will be
convenient to introduce an alternate notation, $\procn{x}$, when we
want to emphasize the connection to the use of the name. Note that, by
virtue of name equivalence, $\quotep{\procn{x}} \nameeq x$; so, the
notation is consistent with previous definitions.

Further, because names have structure it is possible to effect
substitutions on the basis of that structure. This means we need to
upgrade our notation for substitutions, which we accomplish by
adapting comprehension notation. Thus,

\begin{mathpar}
  P\{ y / x : x \in S \}
\end{mathpar}

is interpreted to mean the process derived from P by replacing (in a
capture-avoiding manner) each occurrence of $x$ in $S$ by $y$. For example,

\begin{mathpar}
  P\{ \quotep{\procn{x}|\procn{x}} / x : x \in \freenames{P} \}
\end{mathpar}

will replace each (occurrence) of a free name $x$ in $P$ by
$\quotep{\procn{x}|\procn{x}}$.

Also, we will avail ourselves of the notation $x^{L}$ and $x^{R}$ to
denote injections of a name into disjoint copies of the name
space. There are numerous ways to accomplish this. One example can be
found in \cite{MeredithR05}. This notation overloads to vectors of
names: $\vec{x}^{\pi} := (x_{i}^{\pi} \; : \; 0 \leq i < |\vec{x}| )$ where $\pi \in \{L,R\}$.

We also use $P^{\Box} := P|\Box$.

In \cite{MeredithR05} an interpretation of the new operator is
given. It turns out that there are several possible interpretations
all enjoying the requisite algebraic properties of the operator (see
\cite{milner91polyadicpi}). We will therefore make liberal use of
$(\nu\; \vec{x})P$.

% subsection the_syntax_and_semantics_of_the_notation_system (end)   

\input{qm2pi.qmops} 

\input{qm2pi.sterngerlach} 

\input{qm2pi.metric} 

% section concurrent_process_calculi (end)

%\input{qm2pi.proofsketch}

% section proof sketch (end)

%\input{qm2pi.slviaknots} 

% section spatial logic via knots (end)

\input{qm2pi.conclusion}

% section conclusion (end)

%\input{qm2pi.dtcodes} 

% section wiring algorithm (end)

\input{qm2pi.ack} 

% section acknowledgments (end)

\newpage


\bibliographystyle{plain}   
\bibliography{../../biblios/main.bib}

\input{qm2pi.rhodetails}

\end{document}

 

% section concurrent_process_calculi (end)

%\documentclass[12pt]{llncs}
%\documentclass{jktr}

\usepackage[pdftex]{hyperref}                   
\usepackage {listings}
\usepackage {mathpartir}
\usepackage{bcprules}
%\usepackage{listings}
                       
\usepackage{graphicx} 
%\usepackage[margins=2.5cm,nohead,nofoot]{geometry}
%\usepackage{geometry}
\usepackage{amsfonts}
\usepackage{amstext}
\usepackage{latexsym}
\usepackage{amssymb}
\usepackage{color}


%\include{myPreamble}
\include{qm2pi.local} 

%\ifpdf
%\usepackage[pdftex]{graphicx}
%\else
%\usepackage{graphicx}
%\fi

 % \ifpdf
%  \usepackage{pdfsync}
%  \if


%\title{Brief Article}
%\author{David F. Snyder}
%\author{L.G. Meredith}

%\address{Dept. of Math., Texas State University--San Marcos, San Marcos, TX 78666}
       
\pagestyle{empty}


\begin{document}

\lstset{language=[Objective]Caml,frame=shadowbox}

\input{qm2pi.front}

% section front matter (end)

\input{qm2pi.intro} 
 
% section introduction (end)

% \input{qm2pi.knotations} 

% section notation (end)

\input{qm2pi.process.calculi} 

% section concurrent_process_calculi_and_spatial_logics_ (end)
    
%\input{qm2pi.knots2pi} 

%\input{qm2pi.trefoil} 

%\input{qm2pi.mainthm} 

% subsection basic_interpretation (end)

%\input{qm2pi.rho.presentation} 
\subsection{The syntax and semantics of the notation system}\label{sub:the_syntax_and_semantics_of_the_notation_system} % (fold)

We now summarize a technical presentation of the calculus that
embodies our theory of dynamics. The typical presentation of such a
calculus follows the style of giving generators and relations on
them. The grammar, below, describing term constructors, freely
generates the set of processes, $\Proc$. This set is then quotiented
by a relation known as structural congruence and it is over this set
that the notion of dynamics is expressed. This presentation is
essentially that of \cite{MeredithR05} with the addition of
polyadicity and summation. For readability we have relegated some of
the technical subtleties to an appendix.

\subsubsection{Process grammar}\label{subsub:process_grammar}

\begin{mathpar}
  \inferrule* [lab=synchronization] {} {{M} \bc \pzero \;|\; x?F \;|\; x!C }
  \and
  \inferrule* [lab=abstraction] {} {{F} \bc (x)P}
  \and
  \inferrule* [lab=concretion] {} {{C} \bc \langle Q \rangle}
  \and
  \inferrule* [lab=process] {} {{P,Q} \bc M \;| \;P|Q \;|\; @{x}}
  \and
  \inferrule* [lab=name] {} {{x} \bc \quotep{P}}
\end{mathpar} 

Note that $\vec{x}$ (resp. $\vec{P}$) denotes a vector of names
(resp. processes) of length $|\vec{x}|$ (resp. $|\vec{P}|$). We adopt
the following useful abbreviations.

\begin{mathpar}
   x?(\vec{y}).P := x.(\vec{y})P \and  x\clift{\vec{P}} := x.\clift{\vec{P}}
   \and x!(y) := \lift{x}{\dropn{y}}
   \and \Pi_{i=0}^{n-1}P_i := P_0 | \ldots | P_{n-1}
\end{mathpar}

\subsubsection{Structural congruence}

\paragraph{Free and bound names and alpha-equivalence.} At the
core of structural equivalence is alpha-equivalence which identifies
process that are the same up to a change of variable. Formally, we
recognize the distinction between free and bound names. The free names
of a process, $\freenames{P}$, may be calculated recursively as
follows:

\begin{mathpar}
\freenames{\pzero} := \emptyset
  \and \\
  \freenames{x?(y).P} := \{ x \} \cup (\freenames{P} \setminus \{ y \})
  \and 
  \freenames{x!\langle P \rangle} := \{ x \} \cup \{ P \} 
  \and \\
  \freenames{P|Q} := \freenames{P} \cup \freenames{Q}
  \and \\
  \freenames{@{x}} := \{ x \}
\end{mathpar}

$\pi$
$\quotep{\pi}$

$\freenames{-} : \pi \to \mathcal{P}(\quotep{\pi})$

\begin{eqnarray*}
  \freenames{\pzero} & := & \emptyset \\
  \freenames{x?(y).P} & := & \{ x \} \cup (\freenames{P} \setminus \{ y \}) \\
  \freenames{x!\langle P \rangle} & := & \{ x \} \cup \{ P \} \\
  \freenames{P|Q} & := & \freenames{P} \cup \freenames{Q} \\
  \freenames{\dropn{x}} & := & \{ x \}
\end{eqnarray*}

The bound names of a process, $\boundnames{P}$, are those names occurring in $P$
that are not free. For example, in $x?(y).0$, the name $x$ is free, while $y$ is bound.

\begin{mathpar}
  \inferrule* [lab=monoidal-laws] {} { P|Q \equiv Q|P \and P|0 \equiv P \and P|(Q|R) \equiv (P|Q)|R }
\end{mathpar}

\begin{mathpar}
  \inferrule* [lab=alpha-equivalence] {} { (x)P \equiv (y)P\{y/x\} \and y \not\in \freenames{P} }
\end{mathpar}

\begin{definition}
Then two processes, $P,Q$, are alpha-equivalent if $P = Q\{\vec{y}/\vec{x}\}$ for
some $\vec{x} \in \boundnames{Q},\vec{y} \in \boundnames{P}$, where $Q\{\vec{y}/\vec{x}\}$
denotes the capture-avoiding substitution of $\vec{y}$ for $\vec{x}$ in $Q$.
\end{definition}

\begin{definition}
  The {\em structural congruence} \cite{SangiorgiWalker} , $\equiv$,
  between processes is the least congruence containing
  alpha-equivalence, satisfying the abelian monoid laws
  (associativity, commutativity and $\pzero$ as identity) for parallel
  composition $|$ and for summation $+$.
\end{definition}

\subsection{Name equivalence}

We take name equivalence, written $\nameeq$, to be the smallest
equivalence relation generated by the following rules.

\begin{mathpar}
\inferrule*[lab=Quote-drop]
{ }
{ \quotep{@{x}} \nameeq x }

\inferrule*[lab=Struct-equiv]
{ P \scong Q }
{ \quotep{P} \nameeq \quotep{Q} }
\end{mathpar}

The astute reader will have noticed that the mutual recursion of names
and processes imposes a mutual recursion on alpha-equivalence and
structural equivalence via name-equivalence. Fortunately, all of this
works out pleasantly and we may calculate in the natural way, free of
concern. The reader interested in the details is referred to the
appendix \ref{appendix:rho_details}.

\subsection{Substitution}

We use $\Proc$ for the set of processes, $\QProc$ for the set of
names, and $\id{\{}\vec{y} / \vec{x} \id{\}}$ to denote partial maps,
$s : \QProc \rightarrow \QProc$. A map, $s$ lifts, uniquely, to a map
on process terms, $\widehat{s} : \Proc \rightarrow \Proc$ by the
following equations.

\begin{mathpar}
  (0) \psubstp{Q}{P} := 0 \\
  (R \juxtap S) \psubstp{Q}{P}
  :=    
  (R)\psubstp{Q}{P} \juxtap (S) \psubstp{Q}{P} \\
  (x?(y).R) \psubstp{Q}{P}    
  :=    
  (x)\substp{Q}{P} (z)\concat( (R \psubstn{z}{y}) \psubstp{Q}{P} ) \\
  (\lift{x}{R}) \psubstp{Q}{P}  
  :=
  \lift{(x)\substp{Q}{P}}{ R \psubstp{Q}{P} } \\
%   (\dropn{x})  \psubstp{Q}{P}       
%   := 
%   \left\{ 
%     \begin{array}{ccc} 
%       \dropn{\quotep{Q}} & & x \nameeq \quotep{P} \\
%       \dropn{x} & & otherwise \\
%     \end{array}
%   \right. 
  (\dropn{x})  \psubstp{Q}{P}       
  := 
  \left\{ 
    \begin{array}{ccc} 
      Q & & x \nameeq \quotep{P} \\
      \dropn{x} & & otherwise \\
    \end{array}
  \right.
\end{mathpar}
 

where

\begin{eqnarray}
  (x)\id{\{} \lpquote Q \rpquote / \lpquote P \rpquote \id{\}}            = 
  \left\{ 
    \begin{array}{ccc}
      \lpquote Q \rpquote & & x \nameeq \lpquote P \rpquote \\
      x & & otherwise \\
    \end{array}
  \right. \nonumber
\end{eqnarray}

and $z$ is chosen distinct from $\quotep{P}$, $\quotep{Q}$, the free
names in $Q$, and all the names in $R$. Our $\alpha$-equivalence will
be built in the standard way from this substitution.

\begin{remark}\label{rem:no_self_referential_names}
  One consequence of these definitions is that $\forall P. \quotep{P}
  \not\in \freenames{P}$.
\end{remark}

\subsection{ Dynamic quote: an example }

Anticipating something of what's to come, consider applying the
substitution, $\widehat{\id{\{}u / z \id{\}}}$, to the following pair
of processes, $\lift{w}{y!(z)}$ and $w[ \lpquote y!(z) \rpquote ]$.

\begin{eqnarray}
	\lift{w}{y!(z)}\widehat{\id{\{}u / z \id{\}}}
		& = &
		\lift{w}{y!(u)} \nonumber\\
	w[ \lpquote y!(z) \rpquote ] \widehat{ \id{\{}u / z \id{\}} }
		& = &
		w[ \lpquote y!(z) \rpquote ] \nonumber
\end{eqnarray}

Because the body of the process between quotes is impervious to
substitution, we get radically different answers. In fact, by
examining the first process in an input context,
e.g. $x?(z).\lift{w}{y!(z)}$, we see that the process under the lift
operator may be shaped by prefixed inputs binding a name inside it. In
this sense, the lift operator will be seen as a way to dynamically
construct processes before reifying them as names.

Finally equipped with these standard features we can present the
dynamics of the calculus.

\subsubsection{Operational semantics} 

Finally, we introduce the computational dynamics. What marks these
algebras as distinct from other more traditionally studied algebraic
structures, e.g. vector spaces or polynomial rings, is the manner in
which dynamics is captured. In traditional structures, dynamics is typically
expressed through morphisms between such structures, as in linear maps
between vector spaces or morphisms between rings. In algebras
associated with the semantics of computation, the dynamics is
expressed as part of the algebraic structure itself, through a
reduction reduction relation typically denoted by $\red$. Below, we
give a recursive presentation of this relation for the calculus used
in the encoding.

$\red \subseteq \pi \times \pi$
$\red : \pi \to \mathcal{P}(\pi)$

\begin{mathpar}
  \inferrule* [lab=Comm] { \textsf{match}( x_{src}, x_{trgt} ) } { x_{trgt}?(y)P \; | \; x_{src}!\langle {Q} \rangle \red P\{\quotep{Q}/y}\} }
  \and \\
  \inferrule* [lab=Par] {{P} \red {P}'} {{{P} | {Q}} \red {{P}' | {Q}}}
  \and
  \inferrule* [lab=Equiv]{{{P} \scong {P}'} \andalso {{P}' \red {Q}'} \andalso {{Q}' \scong {Q}}}{{P} \red {Q}}
\end{mathpar}

\begin{eqnarray*}
  match_{\equiv} (\quotep{P},\quotep{Q}) & := & P \equiv Q \\
  match_{\dagger}(\quotep{P},\quotep{Q}) & := & \forall R. P|Q \red^{*} R => R \red^{*} 0 \\
  match_{K}(\quotep{P},\quotep{Q}) & := & K \mbox{ for some context } K
\end{eqnarray*}

$u?(x)P | u!\langle Q \rangle \red P\{\quotep{Q}/x\}$

%We write $\wred$ for $\red^*$, and $P\red$ if $\exists Q $ such that $ P \red Q$.
We write $P\red$ if $\exists Q $ such that $ P \red Q$ and $P\not\red$, otherwise.

\section{Replication}

As mentioned before, it is known that replication (and hence
recursion) can be implemented in a higher-order process algebra
\cite{SangiorgiWalker}. As our first example of calculation with the
machinery thus far presented we give the construction explicitly in
the {\rhoc}.

\begin{eqnarray}
	D_{x} & := & \prefix{x}{y}{(\binpar{\outputp{x}{y}}{@{y}})} \nonumber\\
	\bangp_{x}{P} & := & \binpar{{x}!\langle{\binpar{D_{x}}{P}}\rangle}{D_{x}} \nonumber
\end{eqnarray}

\begin{eqnarray}
	\bangp_{x}{P} & & \nonumber\\
	=
	& {x}!\langle{(\prefix{x}{y}{(\outputp{x}{y} | @{y})) | P}}\rangle 
	      | \prefix{x}{y}{(\outputp{x}{y} | @{y})} & \nonumber\\
	\red
	& (\outputp{x}{y} | @{y})\substn{\quotep{(\prefix{x}{y}{(@{y} | \outputp{x}{y})) | P}}}{y} & \nonumber\\
	=
	& \outputp{x}{\quotep{(\prefix{x}{y}{(\outputp{x}{y} | @{y})) | P}}}
	  | {(\prefix{x}{y}{(\outputp{x}{y} | @{y})) | P}} & \nonumber\\
	\red
	& \ldots & \nonumber\\
	\red^*
	& P | P | \ldots & \nonumber
\end{eqnarray}

Of course, this encoding, as an implementation, runs away, unfolding
$\bangp{P}$ eagerly. A lazier and more implementable replication
operator, restricted to input-guarded processes, may be obtained as follows.

\begin{eqnarray}
\bangp{\prefix{u}{v}{P}} 
	:= 
	\binpar{\lift{x}{\prefix{u}{v}{(\binpar{D(x)}{P})}}}{D(x)} \nonumber
\end{eqnarray}

\begin{remark}
  Note that the lazier definition still does not deal with summation
  or mixed summation (i.e. sums over input and output). The reader is
  invited to construct definitions of replication that deal with these
  features. 

  Further, the definitions are parameterized in a name, $x$. Can you,
  gentle reader, make a definition that eliminates this parameter and
  guarantees no accidental interaction between the replication
  machinery and the process being replicated -- i.e. no accidental
  sharing of names used by the process to get its work done and the
  name(s) used by the replication to effect copying. This latter
  revision of the definition of replication is crucial to obtaining
  the expected identity $!!P \sim !P$.
\end{remark}

\begin{remark}\label{rem:paradoxical_combinator}
  The reader familiar with the lambda calculus will have noticed the
  similarity between $D$ and the paradoxical combinator.

  [Ed. note: the existence of this seems to suggest we have to be more
  restrictive on the set of processes and names we admit if we are to
  support no-cloning.]
\end{remark}

\subsubsection{Bisimulation}

The computational dynamics gives rise to another kind of equivalence,
the equivalence of computational behavior. As previously mentioned
this is typically captured \emph{via} some form of bisimulation.

% The notion we use in this paper is weak barbed bisimulation
% \cite{milner91polyadicpi}.

The notion we use in this paper is derived from weak barbed
bisimulation \cite{milner91polyadicpi}. 

\begin{definition}
An \emph{observation relation}, $\downarrow_{\mathcal N}$, over a set
of names, $\mathcal N$, is the smallest relation satisfying the rules
below.

\infrule[Out-barb]{y \in {\mathcal N}, \; x \nameeq y}
		  {\outputp{x}{v} \downarrow_{\mathcal N} x}
\infrule[Par-barb]{\mbox{$P\downarrow_{\mathcal N} x$ or $Q\downarrow_{\mathcal N} x$}}
		  {\binpar{P}{Q} \downarrow_{\mathcal N} x}

We write $P \Downarrow_{\mathcal N} x$ if there is $Q$ such that 
$P \wred Q$ and $Q \downarrow_{\mathcal N} x$.
\end{definition}

\begin{definition}
%\label{def.bbisim}
An  ${\mathcal N}$-\emph{barbed bisimulation} over a set of names, ${\mathcal N}$, is a symmetric binary relation 
${\mathcal S}_{\mathcal N}$ between agents such that $P\rel{S}_{\mathcal N}Q$ implies:
\begin{enumerate}
\item If $P \red P'$ then $Q \wred Q'$ and $P'\rel{S}_{\mathcal N} Q'$.
\item If $P\downarrow_{\mathcal N} x$, then $Q\Downarrow_{\mathcal N} x$.
\end{enumerate}
$P$ is ${\mathcal N}$-barbed bisimilar to $Q$, written
$P \wbbisim_{\mathcal N} Q$, if $P \rel{S}_{\mathcal N} Q$ for some ${\mathcal N}$-barbed bisimulation ${\mathcal S}_{\mathcal N}$.
\end{definition}

$\mathcal{R} \subseteq \pi \times \pi$

$P \mathcal{R} Q => \forall P'. P \red P' \Rightarrow \exists Q'. Q \red Q', P' \mathcal{R} Q'$

$P \vdash x \Rightarrow Q \vdash x$

\begin{mathpar}
  \inferrule*[lab=Out-barb]{x \nameeq y}{{y}!\langle{Q}\rangle \vdash x}
  \and
  \inferrule*[lab=Par-barb]{\mbox{$P\vdash x$ or $Q\vdash x$}}{\binpar{P}{Q} \vdash x}
\end{mathpar}

\subsubsection{Contexts}

One of the principle advantages of computational calculi like the
$\pi$-calculus is a well-defined notion of context,
contextual-equivalence and a correlation between
contextual-equivalence and notions of bisimulation. The notion of
context allows the decomposition of a process into (sub-)process and
its syntactic environment, its context. Thus, a context may be
thought of as a process with a ``hole'' (written $\Box$) in it. The
application of a context $M$ to a process $P$, written $M[P]$, is
tantamount to filling the hole in $M$ with $P$. In this paper we do
not need the full weight of this theory, but do make use of the notion
of context in the proof the main theorem. 

\begin{mathpar}
  \inferrule* [lab=summation] {} {{M_{M},M_{N}} \bc \Box \;|\; x.M_{A} \;|\; M_{M}+M_{N}}
  \and
  \inferrule* [lab=agent] {} {{M_{A}} \bc (\vec{x})M_{P} \;| \; \clift{P_0,\ldots,M_{P},\ldots,P_N}}
  \and \\
  \inferrule* [lab=process] {} {{M_{P}} \bc M_{N} \;| \;P|M_{P} }
\end{mathpar} 

\begin{mathpar}
  \inferrule* [lab=sychronization] {} {M_{N} \bc \Box \;|\; x?M_{F} \;|\; x!M_{C}}
  \and
  \inferrule* [lab=abstraction] {} {{M_{F}} \bc (x)M_{P} }
  \and
  \inferrule* [lab=concretion] {} {{M_{C}} \bc \langle M_{P} \rangle }
  \and \\
  \inferrule* [lab=process] {} {{M_{P}} \bc M_{N} \;| \;P|M_{P} }
\end{mathpar}

\begin{definition}[contextual application] Given a context $M$, and
  process $P$, we define the \emph{contextual application}, $M[P] :=
  M\{P/\Box\}$. That is, the contextual application of M to P is the
  substitution of $P$ for $\Box$ in $M$.
\end{definition}

$\meaningof{-} : L \to \mathcal{P}(\pi)$

\begin{mathpar}
  \inferrule* [lab=collection] {} {\meaningof{true} = \pi, \and \meaningof{~E} = \pi \setminus \meaningof{E}, \and \meaningof{E_{1} \& E_{2}} = \meaningof{E_{1}} \cap \meaningof{E_{2}}}
\end{mathpar}

\begin{mathpar}
  \inferrule* [lab=structure] {} {\meaningof{0} = \{ P \in \pi | P \equiv 0 \}, \and \\ \meaningof{E_1 | E_2} = \{ P \in \pi | P \equiv P_{1} | P_{2}, P_{1} \in \meaningof{E_{1}}, P_{2} \in \meaningof{E_2}\} }
\end{mathpar}

\begin{mathpar}
 \inferrule* [lab=behavior] {} {\meaningof{\langle a?b \rangle E} = \{ P \in \pi | P \equiv Q | u?(y)P', \\ \and \\\\ \and \\ \;\;\; u \in \meaningof{a}, \forall z.P'\{z/y\} \in \meaningof{E\{z/b\}}\}, \and \\ \meaningof{a!E} = \{ P \in \pi | P \equiv Q | x!\langle P' \rangle, x \in \meaningof{a} P' \in \meaningof{E}\} }
\end{mathpar}

\begin{mathpar}
 \inferrule* [lab=nominal] {} {\meaningof{\quotep{E}} = \{ \quotep{P} \in \quotep{\pi} | P \in \meaningof{E} \}, \and \meaningof{\quotep{P}} = \{ \quotep{Q} \in \quotep{\pi} | P \equiv Q \} \and \\ \meaningof{@\quotep{E}} = \{ P \in \pi | P \equiv @x, x \in \meaningof{E} \}}
\end{mathpar}

\begin{eqnarray*}
  \\
  \meaningof{-} : TS \to ST
\end{eqnarray*}

\begin{eqnarray*}
  \\
  L : TS \to ST
\end{eqnarray*}

\begin{eqnarray*}
  \\
  P \models E \iff P \in \meaningof{E}
\end{eqnarray*}

\begin{eqnarray*}
  P \approx_{L} Q \iff \forall E \in L. P \models E \iff Q \models E
\end{eqnarray*}

\begin{eqnarray*}
  P \approx_{K} Q
\end{eqnarray*}

\begin{eqnarray*}
  P \approx Q
\end{eqnarray*}

$\approx_{K} = \approx = \approx_{L}$

\subsubsection{Contextual duality}

Note that contexts extend the quotation operation to a family of
operations from processes to names. Given a context, $M$, we can
define a \emph{nominal context}, $\quotep{M}$ by $\quotep{M}[P] :=
\quotep{M[P]}$. To foreshadow what is to come we observe that these
operations enjoy a duality with processes very much like the duality
between vectors and maps from vectors to scalars.

Further, because the calculus is essentially higher-order, we have a
correspondence between contexts and processes. More specifically,
given a name $x$ and a context $M$ we can construct $M^{*}_{x}$ such
that 

\begin{mathpar}
  M^{*}_{x} | \lift{x}{P} \red M[P]
\end{mathpar}

namely,

\begin{mathpar}
  M^{*}_{x} := x?(u).M[\dropn{u}]
\end{mathpar}

The dependence of $M^{*}_{x}$ on a name makes it an abstraction, 

\begin{mathpar}
  M^{*} := (x)x?(u).M[\dropn{u}]
\end{mathpar}

\subsection{Additional notation}

It will sometimes be convenient to denote the process a name
quotes. We already have the notation $x = \quotep{P}$, but it will be
convenient to introduce an alternate notation, $\procn{x}$, when we
want to emphasize the connection to the use of the name. Note that, by
virtue of name equivalence, $\quotep{\procn{x}} \nameeq x$; so, the
notation is consistent with previous definitions.

Further, because names have structure it is possible to effect
substitutions on the basis of that structure. This means we need to
upgrade our notation for substitutions, which we accomplish by
adapting comprehension notation. Thus,

\begin{mathpar}
  P\{ y / x : x \in S \}
\end{mathpar}

is interpreted to mean the process derived from P by replacing (in a
capture-avoiding manner) each occurrence of $x$ in $S$ by $y$. For example,

\begin{mathpar}
  P\{ \quotep{\procn{x}|\procn{x}} / x : x \in \freenames{P} \}
\end{mathpar}

will replace each (occurrence) of a free name $x$ in $P$ by
$\quotep{\procn{x}|\procn{x}}$.

Also, we will avail ourselves of the notation $x^{L}$ and $x^{R}$ to
denote injections of a name into disjoint copies of the name
space. There are numerous ways to accomplish this. One example can be
found in \cite{MeredithR05}. This notation overloads to vectors of
names: $\vec{x}^{\pi} := (x_{i}^{\pi} \; : \; 0 \leq i < |\vec{x}| )$ where $\pi \in \{L,R\}$.

We also use $P^{\Box} := P|\Box$.

In \cite{MeredithR05} an interpretation of the new operator is
given. It turns out that there are several possible interpretations
all enjoying the requisite algebraic properties of the operator (see
\cite{milner91polyadicpi}). We will therefore make liberal use of
$(\nu\; \vec{x})P$.

% subsection the_syntax_and_semantics_of_the_notation_system (end)   

\input{qm2pi.qmops} 

\input{qm2pi.sterngerlach} 

\input{qm2pi.metric} 

% section concurrent_process_calculi (end)

%\input{qm2pi.proofsketch}

% section proof sketch (end)

%\input{qm2pi.slviaknots} 

% section spatial logic via knots (end)

\input{qm2pi.conclusion}

% section conclusion (end)

%\input{qm2pi.dtcodes} 

% section wiring algorithm (end)

\input{qm2pi.ack} 

% section acknowledgments (end)

\newpage


\bibliographystyle{plain}   
\bibliography{../../biblios/main.bib}

\input{qm2pi.rhodetails}

\end{document}



% section proof sketch (end)

%\section{Unlikely characters: spatial logic for
  knots}\label{sub:characteristic_formulae} % (fold)

Associated to the mobile process calculi are a family of logics known
as the Hennessy-Milner logics. These logics typically enjoy a
semantics interpreting formulae as sets of processes that when
factored through the encoding outlined above allows an identification
of classes of knots with logical formulae. In the context of this
encoding the sub-family known as the spatial logics \cite{CairesC03}
\cite{CairesC04} \cite{Caires04} are of particular interest providing
several important features for expressing and reasoning about
properties (i.e. classes) of knots. We hint here at how this may be done.

%\begin{description}
%\item [structural connectives] 
\subsubsection{Structural connectives} The spatial logics enjoy
structural connectives corresponding, at the logical level, to the
parallel composition ($P | Q$) and new name ($(\nu \; x)P$)
connectives for processes. As illustrated in the examples below, these
connectives are extremely expressive given the shape of our encoding.
%\item [decideable satisfaction]

\subsubsection{Decideable satisfaction}
In \cite{Caires04} the satisfaction relation is shown to be decideable
for a rich class of processes. It further turns out that the image of
the our encoding is a proper subset of that class. This result
provides the basis for an algorithm by which to search for knots
enjoying a given property.
%\item [characteristic formulae]

\subsubsection{Characteristic formulae}
In the same paper \cite{Caires04} , Caires presents a means of calculating
characteristic formulae, selecting equivalence classes of processes
up to a pre--specified depth limit on the support set of names. Composed with our
encoding, this characteristic formula can be used to select
characteristic formulae for knots.
%\end{description}

\subsubsection{Spatial logic formulae}

The grammar below (segmented for comprehension) summarizes the syntax
of spatial logic formulae. We employ illustrative examples in the
sequel to provide an intuitive understanding of their meaning
referring the reader to \cite{Caires04} for a more detailed explication
of the semantics.

\begin{mathpar}
  \inferrule* [lab=boolean] {} {{A,B} \bc T \;|\; \neg A \;|\; A \wedge B \;|\; \eta = \eta'}
  \and
  \inferrule* [lab=spatial] {} {|\; \pzero \;|\; A | B \;|\; x \text{\textregistered} A \;|\; \forall x . A \;|\;  H x . A}
  \and
  \inferrule* [lab=behavioral] {} {|\; \alpha . A}
  \and 
  \inferrule* [lab=recursion] {} {|\; X(\vec{u}) \;|\; \mu X(\vec{u}) . A}
  \and
  \inferrule* [lab=action] {} {\alpha \bc \langle x?(\vec{y}) \rangle \;|\; \langle x!(\vec{y}) \rangle \;|\; \langle \tau \rangle}
  \and 
  \inferrule* [lab=name] {} {\eta \bc x \;|\; \tau}
\end{mathpar} 

% subsection characteristic_formulae (end)   	 

\subsection{Example formulae}\label{sub:example_formulae_} % (fold)

\subsubsection{Crossing as formula.}
% 
% \begin{align*}
%   \frac{d}{dx} \sin x &= \cos x 
%   & \frac{d}{dx} e^x &= e^x \\
%   \frac{d}{dx} \cos x &= - \sin x 
%   & \frac{d}{dx} \log x &= \frac{1}{x} \\
% \end{align*} 

\begin{align*}
 \mu C(x_{0},x_{1},y_{0},y_{1},u).&(\langle x_{0}?(z) \rangle(\langle u! \rangle\langle y_{1}!z \rangle C(x_{0},x_{1},y_{0},y_{1},u)) & \\
  & \wedge \langle y_{1}?(z) \rangle (\langle u! \rangle \langle x_{0}!z \rangle C(x_{0},x_{1},y_{0},y_{1},u)) & \\
  & \wedge \langle x_{1}?(z) \rangle (\langle u? \rangle \langle y_{0}!z \rangle C(x_{0},x_{1},y_{0},y_{1},u)) & \\
  & \wedge \langle y_{0}?(z) \rangle (\langle u? \rangle \langle x_{1}!z \rangle C(x_{0},x_{1},y_{0},y_{1},u))) &
\end{align*}

The lexicographical similarity between the shape of this formulae and
the shape of definition of the process representing a crossing reveals
the intuitive meaning of this formulae. It describes the capabilities
of a process that has the right to represent a crossing. For example
it picks out processes that may perform an input on the port $x_0$ in
its initial menu of capabilities. What differentiates the formula
from the process, however, is that the crossing process is the
smallest candidate to satisfy the formula. Infinitely many other
processes -- with internal behavior hidden behind this interface, so
to speak -- also satisfy this formula. Even this simple formula,
then, can be seen to open a new view onto knots, providing a
computational interpretation of \emph{virtual} knots.

Note that this formula is derived by hand. A similar formula can be
derived by employing Caires' calculation of characteristic formula
\cite{Caires04} to the process representing a crossing. In light of
this discussion, we let
$\meaningof{C}_{\phi}(x0,x1,y0,y1,u)$ denote a formula specifying the
dynamics we wish to capture of a crossing. To guarantee we preserve
the shape of the interface and minimal semantics we demand that
$\meaningof{C}_{\phi}(x0,x1,y0,y1,u) \Rightarrow
\textbf{C}(x0,x1,y0,y1,u)$ where $\textbf{C}(x0,x1,y0,y1,u)$ denotes
the formula above.
                            
\subsubsection{Crossing number constraints.}
The moral content of the context lemma (Lemma \ref{context}) is that the notion of
``locality'' in the Reidemeister moves is effectively captured by the
parallel composition operator of the process calculus. This intuition
extends through the logic. Given a formula,
$\meaningof{C}_{\phi}(x0,x1,y0,y1,u)$, we can use the structural
connectives to specify constraints on crossing numbers, such as at
least $n$ crossings, or exactly $n$ crossings.
\begin{mathpar}
  \inferrule* [lab=at-least-n] {} { K^{\geq n}_{\phi}(\vec{xs},\vec{ys}) := \Pi_{i=0}^{n-1} Hu . \meaningof{C}_{\phi}(xs_i,ys_i,u) | T }
  \and 
  \inferrule* [lab=exactly-n] {} { K^{= n}_{\phi}(\vec{xs},\vec{ys}) := \Pi_{i=0}^{n-1} Hu . \meaningof{C}_{\phi}(xs_i,ys_i,u) | \neg (\forall x_0,y_0,x_1,y_1,u . \meaningof{C}_{\phi}(x_0,y_0,x_1,y_1,u) | T) }
\end{mathpar}

To round out this section, recall that the encoding of an $n$-crossing
knot decomposes into a parallel composition of $n$ \emph{copies} of a
crossing process together with a wiring harness. To specify different
knot classes with the same crossing number amounts to specifying
logical constraints on the wiring harness. In the interest of space,
we defer examples to a forthcoming paper. Suffice it to say that both
the conditions ``alternating knot'' and ``contains the tangle
corresponding to 5/3'' are expressible. For example, it is possible to
calculate the characteristic formula of a process corresponding to the
tangle 5/3 and conjoin it into the classifying formula via the
composition connective of the logic.

Finally, we wish to observe that it is entirely within reason to
contemplate a more domain-specific version of spatial logic tailored
to the shape of processes in the image of the encoding. Such a
domain-specific logic would have a better claim to the title formal
language of knot properties.

% subsection example_formulae_ (end)

% section knots_as_processes (end) 

% section spatial logic via knots (end)

\section{Conclusions and future work}

\paragraph{Testing physical space}
You, gentle reader, may wonder why of all the theorems to be proved
given this set up we pick the one above. In some sense it's hardly
central to quantum mechanics. We see it as central in the sense that
it firmly establishes a notion of physical space arising from a notion
of the equivalence of behavior. Relating bisimulation to a metric is a
big step forward, but one is faced with interpreting the relationship
of that metric space to something more physical. Quantum mechanical
notions of ``physical'' space are still far from intuitive, but by
relating this idea of distance as testing to calculations that predict
physical circumstances we are making a not insignificant step forward
toward an understanding of the physical space we inhabit as
essentially dynamic.

\paragraph{Effectivity and simulation}
One of the observations we have yet to make is that the entire program
spelled out here is effective. We have built various interpreters for
the reflective calculus at work in this interpretation. In principle,
then, we can simulate quantum mechanics on a computer. The place where
the simulation may lose fidelity is the infinitely branching summation
for the annihilator.

In this connection i also want to point out that the evaluation style
calculation of the inner product puts the non-determinism of the
summation right at the heart of measurement. This suggests that
Milner's original reduction-based formulation of the dynamics of his
calculi in terms of sums was not just notationally suggestive of a
notion of measure-and-continue but captured some significant part of
the physics.

\paragraph{Quantum continuations}
In light of this last observation i want to point out that the
predominant account of quantum mechanics is missing a key aspect of a
truly compositional story of the physical situation. In a real lab,
when a measurement is made the observation can be made to feed into
another device that then makes another measurement conditioned on the
results of the first. This means that after the superposition was
collapsed the entire experimental set up remained in
superposition. While QM offers a means of writing this down it doesn't
quite line up well with the well-trodden formulation of computation
and continuation that we see so succinctly expressed in Milner's
calculi. This suggests that there might be advantages to this account
of dynamics waiting to be explored.

\paragraph{Quantum logic}
In this connection, we also note that by virtue of having the
Hennessy-Milner construction, we can pull the construction through the
interpretation of QM. This gives us a natural candidate for a quantum
logic that enjoys an extremely tight connection with it's domain of
interpretation, making the construction much less ad hoc (rather it is
the image of functor!).

\paragraph{Quantum probabiity}
i have questions about the basis of the interpretation of inner
product as probability amplitude. In particular, using which
axiomatization of probability theory does the notion of probability
amplitude earn the right to be so dubbed? In other words, where is the
proof that the operation for calculating a probability amplitude (and
then squaring) satisfies the axioms of what it means to calculate a
probability? Even if such a proof exists (i have yet to find it in the
literature), i wonder if it might not be possible to turn things on
their heads. Can we view the calculation of the probability amplitude
as an axiomatization of probability? If so, then the definition we
give for calculating probability amplitude may provide the basis for
an \emph{effective} theory of probability.

\paragraph{Quantum vs ``biological'' information}
Finally, i want to conclude with a more philosophical observation. At
a recent workshop in which QM was a predominant topic i noticed
something about quantum information. The speaker was giving a riveting
discussion of axiomatic QM and showing how properties of ``no
cloning'' and ``no deleting'' emerged as consequences of the
axiomatization. Theorems of this form are necessary to give us a sense
of confidence that our axioms characterize the physical theory. What
struck me, though, was that if quantum information is neither erasable
nor replicable it is markedly different from \emph{life}. Two of the
things we know about life is that

\begin{itemize}
  \item it ends;
  \item to gain some measure of persistence, to transcend it's
    finitude it is imminently copyable.
\end{itemize}

Both of these qualities are summarized succinctly in the aphorism: all
flesh is grass. For me these two kinds of ``information'' -- call them
quantum and biological -- are end points on a spectrum of strategies
for persistence. At one end, we have those curious entities that enjoy
uniqueness and permanence; at the other, we have those who in the face
of a certain end and an uncertain present make a go of passing
something on. To me one of the more remarkable aspects of the latter
strategy is that in the presence of noise (and certain features of
copying) we get a kind of dynamism, a chance for improvement against a
given persistent condition.

% subsection other_calculi_other_bisimulations_and_geometry_as_behavior (end)




% section conclusion (end)

%\documentclass[12pt]{llncs}
%\documentclass{jktr}

\usepackage[pdftex]{hyperref}                   
\usepackage {listings}
\usepackage {mathpartir}
\usepackage{bcprules}
%\usepackage{listings}
                       
\usepackage{graphicx} 
%\usepackage[margins=2.5cm,nohead,nofoot]{geometry}
%\usepackage{geometry}
\usepackage{amsfonts}
\usepackage{amstext}
\usepackage{latexsym}
\usepackage{amssymb}
\usepackage{color}


%\include{myPreamble}
\include{qm2pi.local} 

%\ifpdf
%\usepackage[pdftex]{graphicx}
%\else
%\usepackage{graphicx}
%\fi

 % \ifpdf
%  \usepackage{pdfsync}
%  \if


%\title{Brief Article}
%\author{David F. Snyder}
%\author{L.G. Meredith}

%\address{Dept. of Math., Texas State University--San Marcos, San Marcos, TX 78666}
       
\pagestyle{empty}


\begin{document}

\lstset{language=[Objective]Caml,frame=shadowbox}

\input{qm2pi.front}

% section front matter (end)

\input{qm2pi.intro} 
 
% section introduction (end)

% \input{qm2pi.knotations} 

% section notation (end)

\input{qm2pi.process.calculi} 

% section concurrent_process_calculi_and_spatial_logics_ (end)
    
%\input{qm2pi.knots2pi} 

%\input{qm2pi.trefoil} 

%\input{qm2pi.mainthm} 

% subsection basic_interpretation (end)

%\input{qm2pi.rho.presentation} 
\subsection{The syntax and semantics of the notation system}\label{sub:the_syntax_and_semantics_of_the_notation_system} % (fold)

We now summarize a technical presentation of the calculus that
embodies our theory of dynamics. The typical presentation of such a
calculus follows the style of giving generators and relations on
them. The grammar, below, describing term constructors, freely
generates the set of processes, $\Proc$. This set is then quotiented
by a relation known as structural congruence and it is over this set
that the notion of dynamics is expressed. This presentation is
essentially that of \cite{MeredithR05} with the addition of
polyadicity and summation. For readability we have relegated some of
the technical subtleties to an appendix.

\subsubsection{Process grammar}\label{subsub:process_grammar}

\begin{mathpar}
  \inferrule* [lab=synchronization] {} {{M} \bc \pzero \;|\; x?F \;|\; x!C }
  \and
  \inferrule* [lab=abstraction] {} {{F} \bc (x)P}
  \and
  \inferrule* [lab=concretion] {} {{C} \bc \langle Q \rangle}
  \and
  \inferrule* [lab=process] {} {{P,Q} \bc M \;| \;P|Q \;|\; @{x}}
  \and
  \inferrule* [lab=name] {} {{x} \bc \quotep{P}}
\end{mathpar} 

Note that $\vec{x}$ (resp. $\vec{P}$) denotes a vector of names
(resp. processes) of length $|\vec{x}|$ (resp. $|\vec{P}|$). We adopt
the following useful abbreviations.

\begin{mathpar}
   x?(\vec{y}).P := x.(\vec{y})P \and  x\clift{\vec{P}} := x.\clift{\vec{P}}
   \and x!(y) := \lift{x}{\dropn{y}}
   \and \Pi_{i=0}^{n-1}P_i := P_0 | \ldots | P_{n-1}
\end{mathpar}

\subsubsection{Structural congruence}

\paragraph{Free and bound names and alpha-equivalence.} At the
core of structural equivalence is alpha-equivalence which identifies
process that are the same up to a change of variable. Formally, we
recognize the distinction between free and bound names. The free names
of a process, $\freenames{P}$, may be calculated recursively as
follows:

\begin{mathpar}
\freenames{\pzero} := \emptyset
  \and \\
  \freenames{x?(y).P} := \{ x \} \cup (\freenames{P} \setminus \{ y \})
  \and 
  \freenames{x!\langle P \rangle} := \{ x \} \cup \{ P \} 
  \and \\
  \freenames{P|Q} := \freenames{P} \cup \freenames{Q}
  \and \\
  \freenames{@{x}} := \{ x \}
\end{mathpar}

$\pi$
$\quotep{\pi}$

$\freenames{-} : \pi \to \mathcal{P}(\quotep{\pi})$

\begin{eqnarray*}
  \freenames{\pzero} & := & \emptyset \\
  \freenames{x?(y).P} & := & \{ x \} \cup (\freenames{P} \setminus \{ y \}) \\
  \freenames{x!\langle P \rangle} & := & \{ x \} \cup \{ P \} \\
  \freenames{P|Q} & := & \freenames{P} \cup \freenames{Q} \\
  \freenames{\dropn{x}} & := & \{ x \}
\end{eqnarray*}

The bound names of a process, $\boundnames{P}$, are those names occurring in $P$
that are not free. For example, in $x?(y).0$, the name $x$ is free, while $y$ is bound.

\begin{mathpar}
  \inferrule* [lab=monoidal-laws] {} { P|Q \equiv Q|P \and P|0 \equiv P \and P|(Q|R) \equiv (P|Q)|R }
\end{mathpar}

\begin{mathpar}
  \inferrule* [lab=alpha-equivalence] {} { (x)P \equiv (y)P\{y/x\} \and y \not\in \freenames{P} }
\end{mathpar}

\begin{definition}
Then two processes, $P,Q$, are alpha-equivalent if $P = Q\{\vec{y}/\vec{x}\}$ for
some $\vec{x} \in \boundnames{Q},\vec{y} \in \boundnames{P}$, where $Q\{\vec{y}/\vec{x}\}$
denotes the capture-avoiding substitution of $\vec{y}$ for $\vec{x}$ in $Q$.
\end{definition}

\begin{definition}
  The {\em structural congruence} \cite{SangiorgiWalker} , $\equiv$,
  between processes is the least congruence containing
  alpha-equivalence, satisfying the abelian monoid laws
  (associativity, commutativity and $\pzero$ as identity) for parallel
  composition $|$ and for summation $+$.
\end{definition}

\subsection{Name equivalence}

We take name equivalence, written $\nameeq$, to be the smallest
equivalence relation generated by the following rules.

\begin{mathpar}
\inferrule*[lab=Quote-drop]
{ }
{ \quotep{@{x}} \nameeq x }

\inferrule*[lab=Struct-equiv]
{ P \scong Q }
{ \quotep{P} \nameeq \quotep{Q} }
\end{mathpar}

The astute reader will have noticed that the mutual recursion of names
and processes imposes a mutual recursion on alpha-equivalence and
structural equivalence via name-equivalence. Fortunately, all of this
works out pleasantly and we may calculate in the natural way, free of
concern. The reader interested in the details is referred to the
appendix \ref{appendix:rho_details}.

\subsection{Substitution}

We use $\Proc$ for the set of processes, $\QProc$ for the set of
names, and $\id{\{}\vec{y} / \vec{x} \id{\}}$ to denote partial maps,
$s : \QProc \rightarrow \QProc$. A map, $s$ lifts, uniquely, to a map
on process terms, $\widehat{s} : \Proc \rightarrow \Proc$ by the
following equations.

\begin{mathpar}
  (0) \psubstp{Q}{P} := 0 \\
  (R \juxtap S) \psubstp{Q}{P}
  :=    
  (R)\psubstp{Q}{P} \juxtap (S) \psubstp{Q}{P} \\
  (x?(y).R) \psubstp{Q}{P}    
  :=    
  (x)\substp{Q}{P} (z)\concat( (R \psubstn{z}{y}) \psubstp{Q}{P} ) \\
  (\lift{x}{R}) \psubstp{Q}{P}  
  :=
  \lift{(x)\substp{Q}{P}}{ R \psubstp{Q}{P} } \\
%   (\dropn{x})  \psubstp{Q}{P}       
%   := 
%   \left\{ 
%     \begin{array}{ccc} 
%       \dropn{\quotep{Q}} & & x \nameeq \quotep{P} \\
%       \dropn{x} & & otherwise \\
%     \end{array}
%   \right. 
  (\dropn{x})  \psubstp{Q}{P}       
  := 
  \left\{ 
    \begin{array}{ccc} 
      Q & & x \nameeq \quotep{P} \\
      \dropn{x} & & otherwise \\
    \end{array}
  \right.
\end{mathpar}
 

where

\begin{eqnarray}
  (x)\id{\{} \lpquote Q \rpquote / \lpquote P \rpquote \id{\}}            = 
  \left\{ 
    \begin{array}{ccc}
      \lpquote Q \rpquote & & x \nameeq \lpquote P \rpquote \\
      x & & otherwise \\
    \end{array}
  \right. \nonumber
\end{eqnarray}

and $z$ is chosen distinct from $\quotep{P}$, $\quotep{Q}$, the free
names in $Q$, and all the names in $R$. Our $\alpha$-equivalence will
be built in the standard way from this substitution.

\begin{remark}\label{rem:no_self_referential_names}
  One consequence of these definitions is that $\forall P. \quotep{P}
  \not\in \freenames{P}$.
\end{remark}

\subsection{ Dynamic quote: an example }

Anticipating something of what's to come, consider applying the
substitution, $\widehat{\id{\{}u / z \id{\}}}$, to the following pair
of processes, $\lift{w}{y!(z)}$ and $w[ \lpquote y!(z) \rpquote ]$.

\begin{eqnarray}
	\lift{w}{y!(z)}\widehat{\id{\{}u / z \id{\}}}
		& = &
		\lift{w}{y!(u)} \nonumber\\
	w[ \lpquote y!(z) \rpquote ] \widehat{ \id{\{}u / z \id{\}} }
		& = &
		w[ \lpquote y!(z) \rpquote ] \nonumber
\end{eqnarray}

Because the body of the process between quotes is impervious to
substitution, we get radically different answers. In fact, by
examining the first process in an input context,
e.g. $x?(z).\lift{w}{y!(z)}$, we see that the process under the lift
operator may be shaped by prefixed inputs binding a name inside it. In
this sense, the lift operator will be seen as a way to dynamically
construct processes before reifying them as names.

Finally equipped with these standard features we can present the
dynamics of the calculus.

\subsubsection{Operational semantics} 

Finally, we introduce the computational dynamics. What marks these
algebras as distinct from other more traditionally studied algebraic
structures, e.g. vector spaces or polynomial rings, is the manner in
which dynamics is captured. In traditional structures, dynamics is typically
expressed through morphisms between such structures, as in linear maps
between vector spaces or morphisms between rings. In algebras
associated with the semantics of computation, the dynamics is
expressed as part of the algebraic structure itself, through a
reduction reduction relation typically denoted by $\red$. Below, we
give a recursive presentation of this relation for the calculus used
in the encoding.

$\red \subseteq \pi \times \pi$
$\red : \pi \to \mathcal{P}(\pi)$

\begin{mathpar}
  \inferrule* [lab=Comm] { \textsf{match}( x_{src}, x_{trgt} ) } { x_{trgt}?(y)P \; | \; x_{src}!\langle {Q} \rangle \red P\{\quotep{Q}/y}\} }
  \and \\
  \inferrule* [lab=Par] {{P} \red {P}'} {{{P} | {Q}} \red {{P}' | {Q}}}
  \and
  \inferrule* [lab=Equiv]{{{P} \scong {P}'} \andalso {{P}' \red {Q}'} \andalso {{Q}' \scong {Q}}}{{P} \red {Q}}
\end{mathpar}

\begin{eqnarray*}
  match_{\equiv} (\quotep{P},\quotep{Q}) & := & P \equiv Q \\
  match_{\dagger}(\quotep{P},\quotep{Q}) & := & \forall R. P|Q \red^{*} R => R \red^{*} 0 \\
  match_{K}(\quotep{P},\quotep{Q}) & := & K \mbox{ for some context } K
\end{eqnarray*}

$u?(x)P | u!\langle Q \rangle \red P\{\quotep{Q}/x\}$

%We write $\wred$ for $\red^*$, and $P\red$ if $\exists Q $ such that $ P \red Q$.
We write $P\red$ if $\exists Q $ such that $ P \red Q$ and $P\not\red$, otherwise.

\section{Replication}

As mentioned before, it is known that replication (and hence
recursion) can be implemented in a higher-order process algebra
\cite{SangiorgiWalker}. As our first example of calculation with the
machinery thus far presented we give the construction explicitly in
the {\rhoc}.

\begin{eqnarray}
	D_{x} & := & \prefix{x}{y}{(\binpar{\outputp{x}{y}}{@{y}})} \nonumber\\
	\bangp_{x}{P} & := & \binpar{{x}!\langle{\binpar{D_{x}}{P}}\rangle}{D_{x}} \nonumber
\end{eqnarray}

\begin{eqnarray}
	\bangp_{x}{P} & & \nonumber\\
	=
	& {x}!\langle{(\prefix{x}{y}{(\outputp{x}{y} | @{y})) | P}}\rangle 
	      | \prefix{x}{y}{(\outputp{x}{y} | @{y})} & \nonumber\\
	\red
	& (\outputp{x}{y} | @{y})\substn{\quotep{(\prefix{x}{y}{(@{y} | \outputp{x}{y})) | P}}}{y} & \nonumber\\
	=
	& \outputp{x}{\quotep{(\prefix{x}{y}{(\outputp{x}{y} | @{y})) | P}}}
	  | {(\prefix{x}{y}{(\outputp{x}{y} | @{y})) | P}} & \nonumber\\
	\red
	& \ldots & \nonumber\\
	\red^*
	& P | P | \ldots & \nonumber
\end{eqnarray}

Of course, this encoding, as an implementation, runs away, unfolding
$\bangp{P}$ eagerly. A lazier and more implementable replication
operator, restricted to input-guarded processes, may be obtained as follows.

\begin{eqnarray}
\bangp{\prefix{u}{v}{P}} 
	:= 
	\binpar{\lift{x}{\prefix{u}{v}{(\binpar{D(x)}{P})}}}{D(x)} \nonumber
\end{eqnarray}

\begin{remark}
  Note that the lazier definition still does not deal with summation
  or mixed summation (i.e. sums over input and output). The reader is
  invited to construct definitions of replication that deal with these
  features. 

  Further, the definitions are parameterized in a name, $x$. Can you,
  gentle reader, make a definition that eliminates this parameter and
  guarantees no accidental interaction between the replication
  machinery and the process being replicated -- i.e. no accidental
  sharing of names used by the process to get its work done and the
  name(s) used by the replication to effect copying. This latter
  revision of the definition of replication is crucial to obtaining
  the expected identity $!!P \sim !P$.
\end{remark}

\begin{remark}\label{rem:paradoxical_combinator}
  The reader familiar with the lambda calculus will have noticed the
  similarity between $D$ and the paradoxical combinator.

  [Ed. note: the existence of this seems to suggest we have to be more
  restrictive on the set of processes and names we admit if we are to
  support no-cloning.]
\end{remark}

\subsubsection{Bisimulation}

The computational dynamics gives rise to another kind of equivalence,
the equivalence of computational behavior. As previously mentioned
this is typically captured \emph{via} some form of bisimulation.

% The notion we use in this paper is weak barbed bisimulation
% \cite{milner91polyadicpi}.

The notion we use in this paper is derived from weak barbed
bisimulation \cite{milner91polyadicpi}. 

\begin{definition}
An \emph{observation relation}, $\downarrow_{\mathcal N}$, over a set
of names, $\mathcal N$, is the smallest relation satisfying the rules
below.

\infrule[Out-barb]{y \in {\mathcal N}, \; x \nameeq y}
		  {\outputp{x}{v} \downarrow_{\mathcal N} x}
\infrule[Par-barb]{\mbox{$P\downarrow_{\mathcal N} x$ or $Q\downarrow_{\mathcal N} x$}}
		  {\binpar{P}{Q} \downarrow_{\mathcal N} x}

We write $P \Downarrow_{\mathcal N} x$ if there is $Q$ such that 
$P \wred Q$ and $Q \downarrow_{\mathcal N} x$.
\end{definition}

\begin{definition}
%\label{def.bbisim}
An  ${\mathcal N}$-\emph{barbed bisimulation} over a set of names, ${\mathcal N}$, is a symmetric binary relation 
${\mathcal S}_{\mathcal N}$ between agents such that $P\rel{S}_{\mathcal N}Q$ implies:
\begin{enumerate}
\item If $P \red P'$ then $Q \wred Q'$ and $P'\rel{S}_{\mathcal N} Q'$.
\item If $P\downarrow_{\mathcal N} x$, then $Q\Downarrow_{\mathcal N} x$.
\end{enumerate}
$P$ is ${\mathcal N}$-barbed bisimilar to $Q$, written
$P \wbbisim_{\mathcal N} Q$, if $P \rel{S}_{\mathcal N} Q$ for some ${\mathcal N}$-barbed bisimulation ${\mathcal S}_{\mathcal N}$.
\end{definition}

$\mathcal{R} \subseteq \pi \times \pi$

$P \mathcal{R} Q => \forall P'. P \red P' \Rightarrow \exists Q'. Q \red Q', P' \mathcal{R} Q'$

$P \vdash x \Rightarrow Q \vdash x$

\begin{mathpar}
  \inferrule*[lab=Out-barb]{x \nameeq y}{{y}!\langle{Q}\rangle \vdash x}
  \and
  \inferrule*[lab=Par-barb]{\mbox{$P\vdash x$ or $Q\vdash x$}}{\binpar{P}{Q} \vdash x}
\end{mathpar}

\subsubsection{Contexts}

One of the principle advantages of computational calculi like the
$\pi$-calculus is a well-defined notion of context,
contextual-equivalence and a correlation between
contextual-equivalence and notions of bisimulation. The notion of
context allows the decomposition of a process into (sub-)process and
its syntactic environment, its context. Thus, a context may be
thought of as a process with a ``hole'' (written $\Box$) in it. The
application of a context $M$ to a process $P$, written $M[P]$, is
tantamount to filling the hole in $M$ with $P$. In this paper we do
not need the full weight of this theory, but do make use of the notion
of context in the proof the main theorem. 

\begin{mathpar}
  \inferrule* [lab=summation] {} {{M_{M},M_{N}} \bc \Box \;|\; x.M_{A} \;|\; M_{M}+M_{N}}
  \and
  \inferrule* [lab=agent] {} {{M_{A}} \bc (\vec{x})M_{P} \;| \; \clift{P_0,\ldots,M_{P},\ldots,P_N}}
  \and \\
  \inferrule* [lab=process] {} {{M_{P}} \bc M_{N} \;| \;P|M_{P} }
\end{mathpar} 

\begin{mathpar}
  \inferrule* [lab=sychronization] {} {M_{N} \bc \Box \;|\; x?M_{F} \;|\; x!M_{C}}
  \and
  \inferrule* [lab=abstraction] {} {{M_{F}} \bc (x)M_{P} }
  \and
  \inferrule* [lab=concretion] {} {{M_{C}} \bc \langle M_{P} \rangle }
  \and \\
  \inferrule* [lab=process] {} {{M_{P}} \bc M_{N} \;| \;P|M_{P} }
\end{mathpar}

\begin{definition}[contextual application] Given a context $M$, and
  process $P$, we define the \emph{contextual application}, $M[P] :=
  M\{P/\Box\}$. That is, the contextual application of M to P is the
  substitution of $P$ for $\Box$ in $M$.
\end{definition}

$\meaningof{-} : L \to \mathcal{P}(\pi)$

\begin{mathpar}
  \inferrule* [lab=collection] {} {\meaningof{true} = \pi, \and \meaningof{~E} = \pi \setminus \meaningof{E}, \and \meaningof{E_{1} \& E_{2}} = \meaningof{E_{1}} \cap \meaningof{E_{2}}}
\end{mathpar}

\begin{mathpar}
  \inferrule* [lab=structure] {} {\meaningof{0} = \{ P \in \pi | P \equiv 0 \}, \and \\ \meaningof{E_1 | E_2} = \{ P \in \pi | P \equiv P_{1} | P_{2}, P_{1} \in \meaningof{E_{1}}, P_{2} \in \meaningof{E_2}\} }
\end{mathpar}

\begin{mathpar}
 \inferrule* [lab=behavior] {} {\meaningof{\langle a?b \rangle E} = \{ P \in \pi | P \equiv Q | u?(y)P', \\ \and \\\\ \and \\ \;\;\; u \in \meaningof{a}, \forall z.P'\{z/y\} \in \meaningof{E\{z/b\}}\}, \and \\ \meaningof{a!E} = \{ P \in \pi | P \equiv Q | x!\langle P' \rangle, x \in \meaningof{a} P' \in \meaningof{E}\} }
\end{mathpar}

\begin{mathpar}
 \inferrule* [lab=nominal] {} {\meaningof{\quotep{E}} = \{ \quotep{P} \in \quotep{\pi} | P \in \meaningof{E} \}, \and \meaningof{\quotep{P}} = \{ \quotep{Q} \in \quotep{\pi} | P \equiv Q \} \and \\ \meaningof{@\quotep{E}} = \{ P \in \pi | P \equiv @x, x \in \meaningof{E} \}}
\end{mathpar}

\begin{eqnarray*}
  \\
  \meaningof{-} : TS \to ST
\end{eqnarray*}

\begin{eqnarray*}
  \\
  L : TS \to ST
\end{eqnarray*}

\begin{eqnarray*}
  \\
  P \models E \iff P \in \meaningof{E}
\end{eqnarray*}

\begin{eqnarray*}
  P \approx_{L} Q \iff \forall E \in L. P \models E \iff Q \models E
\end{eqnarray*}

\begin{eqnarray*}
  P \approx_{K} Q
\end{eqnarray*}

\begin{eqnarray*}
  P \approx Q
\end{eqnarray*}

$\approx_{K} = \approx = \approx_{L}$

\subsubsection{Contextual duality}

Note that contexts extend the quotation operation to a family of
operations from processes to names. Given a context, $M$, we can
define a \emph{nominal context}, $\quotep{M}$ by $\quotep{M}[P] :=
\quotep{M[P]}$. To foreshadow what is to come we observe that these
operations enjoy a duality with processes very much like the duality
between vectors and maps from vectors to scalars.

Further, because the calculus is essentially higher-order, we have a
correspondence between contexts and processes. More specifically,
given a name $x$ and a context $M$ we can construct $M^{*}_{x}$ such
that 

\begin{mathpar}
  M^{*}_{x} | \lift{x}{P} \red M[P]
\end{mathpar}

namely,

\begin{mathpar}
  M^{*}_{x} := x?(u).M[\dropn{u}]
\end{mathpar}

The dependence of $M^{*}_{x}$ on a name makes it an abstraction, 

\begin{mathpar}
  M^{*} := (x)x?(u).M[\dropn{u}]
\end{mathpar}

\subsection{Additional notation}

It will sometimes be convenient to denote the process a name
quotes. We already have the notation $x = \quotep{P}$, but it will be
convenient to introduce an alternate notation, $\procn{x}$, when we
want to emphasize the connection to the use of the name. Note that, by
virtue of name equivalence, $\quotep{\procn{x}} \nameeq x$; so, the
notation is consistent with previous definitions.

Further, because names have structure it is possible to effect
substitutions on the basis of that structure. This means we need to
upgrade our notation for substitutions, which we accomplish by
adapting comprehension notation. Thus,

\begin{mathpar}
  P\{ y / x : x \in S \}
\end{mathpar}

is interpreted to mean the process derived from P by replacing (in a
capture-avoiding manner) each occurrence of $x$ in $S$ by $y$. For example,

\begin{mathpar}
  P\{ \quotep{\procn{x}|\procn{x}} / x : x \in \freenames{P} \}
\end{mathpar}

will replace each (occurrence) of a free name $x$ in $P$ by
$\quotep{\procn{x}|\procn{x}}$.

Also, we will avail ourselves of the notation $x^{L}$ and $x^{R}$ to
denote injections of a name into disjoint copies of the name
space. There are numerous ways to accomplish this. One example can be
found in \cite{MeredithR05}. This notation overloads to vectors of
names: $\vec{x}^{\pi} := (x_{i}^{\pi} \; : \; 0 \leq i < |\vec{x}| )$ where $\pi \in \{L,R\}$.

We also use $P^{\Box} := P|\Box$.

In \cite{MeredithR05} an interpretation of the new operator is
given. It turns out that there are several possible interpretations
all enjoying the requisite algebraic properties of the operator (see
\cite{milner91polyadicpi}). We will therefore make liberal use of
$(\nu\; \vec{x})P$.

% subsection the_syntax_and_semantics_of_the_notation_system (end)   

\input{qm2pi.qmops} 

\input{qm2pi.sterngerlach} 

\input{qm2pi.metric} 

% section concurrent_process_calculi (end)

%\input{qm2pi.proofsketch}

% section proof sketch (end)

%\input{qm2pi.slviaknots} 

% section spatial logic via knots (end)

\input{qm2pi.conclusion}

% section conclusion (end)

%\input{qm2pi.dtcodes} 

% section wiring algorithm (end)

\input{qm2pi.ack} 

% section acknowledgments (end)

\newpage


\bibliographystyle{plain}   
\bibliography{../../biblios/main.bib}

\input{qm2pi.rhodetails}

\end{document}

 

% section wiring algorithm (end)

\documentclass[12pt]{llncs}
%\documentclass{jktr}

\usepackage[pdftex]{hyperref}                   
\usepackage {listings}
\usepackage {mathpartir}
\usepackage{bcprules}
%\usepackage{listings}
                       
\usepackage{graphicx} 
%\usepackage[margins=2.5cm,nohead,nofoot]{geometry}
%\usepackage{geometry}
\usepackage{amsfonts}
\usepackage{amstext}
\usepackage{latexsym}
\usepackage{amssymb}
\usepackage{color}


%\include{myPreamble}
\include{qm2pi.local} 

%\ifpdf
%\usepackage[pdftex]{graphicx}
%\else
%\usepackage{graphicx}
%\fi

 % \ifpdf
%  \usepackage{pdfsync}
%  \if


%\title{Brief Article}
%\author{David F. Snyder}
%\author{L.G. Meredith}

%\address{Dept. of Math., Texas State University--San Marcos, San Marcos, TX 78666}
       
\pagestyle{empty}


\begin{document}

\lstset{language=[Objective]Caml,frame=shadowbox}

\input{qm2pi.front}

% section front matter (end)

\input{qm2pi.intro} 
 
% section introduction (end)

% \input{qm2pi.knotations} 

% section notation (end)

\input{qm2pi.process.calculi} 

% section concurrent_process_calculi_and_spatial_logics_ (end)
    
%\input{qm2pi.knots2pi} 

%\input{qm2pi.trefoil} 

%\input{qm2pi.mainthm} 

% subsection basic_interpretation (end)

%\input{qm2pi.rho.presentation} 
\subsection{The syntax and semantics of the notation system}\label{sub:the_syntax_and_semantics_of_the_notation_system} % (fold)

We now summarize a technical presentation of the calculus that
embodies our theory of dynamics. The typical presentation of such a
calculus follows the style of giving generators and relations on
them. The grammar, below, describing term constructors, freely
generates the set of processes, $\Proc$. This set is then quotiented
by a relation known as structural congruence and it is over this set
that the notion of dynamics is expressed. This presentation is
essentially that of \cite{MeredithR05} with the addition of
polyadicity and summation. For readability we have relegated some of
the technical subtleties to an appendix.

\subsubsection{Process grammar}\label{subsub:process_grammar}

\begin{mathpar}
  \inferrule* [lab=synchronization] {} {{M} \bc \pzero \;|\; x?F \;|\; x!C }
  \and
  \inferrule* [lab=abstraction] {} {{F} \bc (x)P}
  \and
  \inferrule* [lab=concretion] {} {{C} \bc \langle Q \rangle}
  \and
  \inferrule* [lab=process] {} {{P,Q} \bc M \;| \;P|Q \;|\; @{x}}
  \and
  \inferrule* [lab=name] {} {{x} \bc \quotep{P}}
\end{mathpar} 

Note that $\vec{x}$ (resp. $\vec{P}$) denotes a vector of names
(resp. processes) of length $|\vec{x}|$ (resp. $|\vec{P}|$). We adopt
the following useful abbreviations.

\begin{mathpar}
   x?(\vec{y}).P := x.(\vec{y})P \and  x\clift{\vec{P}} := x.\clift{\vec{P}}
   \and x!(y) := \lift{x}{\dropn{y}}
   \and \Pi_{i=0}^{n-1}P_i := P_0 | \ldots | P_{n-1}
\end{mathpar}

\subsubsection{Structural congruence}

\paragraph{Free and bound names and alpha-equivalence.} At the
core of structural equivalence is alpha-equivalence which identifies
process that are the same up to a change of variable. Formally, we
recognize the distinction between free and bound names. The free names
of a process, $\freenames{P}$, may be calculated recursively as
follows:

\begin{mathpar}
\freenames{\pzero} := \emptyset
  \and \\
  \freenames{x?(y).P} := \{ x \} \cup (\freenames{P} \setminus \{ y \})
  \and 
  \freenames{x!\langle P \rangle} := \{ x \} \cup \{ P \} 
  \and \\
  \freenames{P|Q} := \freenames{P} \cup \freenames{Q}
  \and \\
  \freenames{@{x}} := \{ x \}
\end{mathpar}

$\pi$
$\quotep{\pi}$

$\freenames{-} : \pi \to \mathcal{P}(\quotep{\pi})$

\begin{eqnarray*}
  \freenames{\pzero} & := & \emptyset \\
  \freenames{x?(y).P} & := & \{ x \} \cup (\freenames{P} \setminus \{ y \}) \\
  \freenames{x!\langle P \rangle} & := & \{ x \} \cup \{ P \} \\
  \freenames{P|Q} & := & \freenames{P} \cup \freenames{Q} \\
  \freenames{\dropn{x}} & := & \{ x \}
\end{eqnarray*}

The bound names of a process, $\boundnames{P}$, are those names occurring in $P$
that are not free. For example, in $x?(y).0$, the name $x$ is free, while $y$ is bound.

\begin{mathpar}
  \inferrule* [lab=monoidal-laws] {} { P|Q \equiv Q|P \and P|0 \equiv P \and P|(Q|R) \equiv (P|Q)|R }
\end{mathpar}

\begin{mathpar}
  \inferrule* [lab=alpha-equivalence] {} { (x)P \equiv (y)P\{y/x\} \and y \not\in \freenames{P} }
\end{mathpar}

\begin{definition}
Then two processes, $P,Q$, are alpha-equivalent if $P = Q\{\vec{y}/\vec{x}\}$ for
some $\vec{x} \in \boundnames{Q},\vec{y} \in \boundnames{P}$, where $Q\{\vec{y}/\vec{x}\}$
denotes the capture-avoiding substitution of $\vec{y}$ for $\vec{x}$ in $Q$.
\end{definition}

\begin{definition}
  The {\em structural congruence} \cite{SangiorgiWalker} , $\equiv$,
  between processes is the least congruence containing
  alpha-equivalence, satisfying the abelian monoid laws
  (associativity, commutativity and $\pzero$ as identity) for parallel
  composition $|$ and for summation $+$.
\end{definition}

\subsection{Name equivalence}

We take name equivalence, written $\nameeq$, to be the smallest
equivalence relation generated by the following rules.

\begin{mathpar}
\inferrule*[lab=Quote-drop]
{ }
{ \quotep{@{x}} \nameeq x }

\inferrule*[lab=Struct-equiv]
{ P \scong Q }
{ \quotep{P} \nameeq \quotep{Q} }
\end{mathpar}

The astute reader will have noticed that the mutual recursion of names
and processes imposes a mutual recursion on alpha-equivalence and
structural equivalence via name-equivalence. Fortunately, all of this
works out pleasantly and we may calculate in the natural way, free of
concern. The reader interested in the details is referred to the
appendix \ref{appendix:rho_details}.

\subsection{Substitution}

We use $\Proc$ for the set of processes, $\QProc$ for the set of
names, and $\id{\{}\vec{y} / \vec{x} \id{\}}$ to denote partial maps,
$s : \QProc \rightarrow \QProc$. A map, $s$ lifts, uniquely, to a map
on process terms, $\widehat{s} : \Proc \rightarrow \Proc$ by the
following equations.

\begin{mathpar}
  (0) \psubstp{Q}{P} := 0 \\
  (R \juxtap S) \psubstp{Q}{P}
  :=    
  (R)\psubstp{Q}{P} \juxtap (S) \psubstp{Q}{P} \\
  (x?(y).R) \psubstp{Q}{P}    
  :=    
  (x)\substp{Q}{P} (z)\concat( (R \psubstn{z}{y}) \psubstp{Q}{P} ) \\
  (\lift{x}{R}) \psubstp{Q}{P}  
  :=
  \lift{(x)\substp{Q}{P}}{ R \psubstp{Q}{P} } \\
%   (\dropn{x})  \psubstp{Q}{P}       
%   := 
%   \left\{ 
%     \begin{array}{ccc} 
%       \dropn{\quotep{Q}} & & x \nameeq \quotep{P} \\
%       \dropn{x} & & otherwise \\
%     \end{array}
%   \right. 
  (\dropn{x})  \psubstp{Q}{P}       
  := 
  \left\{ 
    \begin{array}{ccc} 
      Q & & x \nameeq \quotep{P} \\
      \dropn{x} & & otherwise \\
    \end{array}
  \right.
\end{mathpar}
 

where

\begin{eqnarray}
  (x)\id{\{} \lpquote Q \rpquote / \lpquote P \rpquote \id{\}}            = 
  \left\{ 
    \begin{array}{ccc}
      \lpquote Q \rpquote & & x \nameeq \lpquote P \rpquote \\
      x & & otherwise \\
    \end{array}
  \right. \nonumber
\end{eqnarray}

and $z$ is chosen distinct from $\quotep{P}$, $\quotep{Q}$, the free
names in $Q$, and all the names in $R$. Our $\alpha$-equivalence will
be built in the standard way from this substitution.

\begin{remark}\label{rem:no_self_referential_names}
  One consequence of these definitions is that $\forall P. \quotep{P}
  \not\in \freenames{P}$.
\end{remark}

\subsection{ Dynamic quote: an example }

Anticipating something of what's to come, consider applying the
substitution, $\widehat{\id{\{}u / z \id{\}}}$, to the following pair
of processes, $\lift{w}{y!(z)}$ and $w[ \lpquote y!(z) \rpquote ]$.

\begin{eqnarray}
	\lift{w}{y!(z)}\widehat{\id{\{}u / z \id{\}}}
		& = &
		\lift{w}{y!(u)} \nonumber\\
	w[ \lpquote y!(z) \rpquote ] \widehat{ \id{\{}u / z \id{\}} }
		& = &
		w[ \lpquote y!(z) \rpquote ] \nonumber
\end{eqnarray}

Because the body of the process between quotes is impervious to
substitution, we get radically different answers. In fact, by
examining the first process in an input context,
e.g. $x?(z).\lift{w}{y!(z)}$, we see that the process under the lift
operator may be shaped by prefixed inputs binding a name inside it. In
this sense, the lift operator will be seen as a way to dynamically
construct processes before reifying them as names.

Finally equipped with these standard features we can present the
dynamics of the calculus.

\subsubsection{Operational semantics} 

Finally, we introduce the computational dynamics. What marks these
algebras as distinct from other more traditionally studied algebraic
structures, e.g. vector spaces or polynomial rings, is the manner in
which dynamics is captured. In traditional structures, dynamics is typically
expressed through morphisms between such structures, as in linear maps
between vector spaces or morphisms between rings. In algebras
associated with the semantics of computation, the dynamics is
expressed as part of the algebraic structure itself, through a
reduction reduction relation typically denoted by $\red$. Below, we
give a recursive presentation of this relation for the calculus used
in the encoding.

$\red \subseteq \pi \times \pi$
$\red : \pi \to \mathcal{P}(\pi)$

\begin{mathpar}
  \inferrule* [lab=Comm] { \textsf{match}( x_{src}, x_{trgt} ) } { x_{trgt}?(y)P \; | \; x_{src}!\langle {Q} \rangle \red P\{\quotep{Q}/y}\} }
  \and \\
  \inferrule* [lab=Par] {{P} \red {P}'} {{{P} | {Q}} \red {{P}' | {Q}}}
  \and
  \inferrule* [lab=Equiv]{{{P} \scong {P}'} \andalso {{P}' \red {Q}'} \andalso {{Q}' \scong {Q}}}{{P} \red {Q}}
\end{mathpar}

\begin{eqnarray*}
  match_{\equiv} (\quotep{P},\quotep{Q}) & := & P \equiv Q \\
  match_{\dagger}(\quotep{P},\quotep{Q}) & := & \forall R. P|Q \red^{*} R => R \red^{*} 0 \\
  match_{K}(\quotep{P},\quotep{Q}) & := & K \mbox{ for some context } K
\end{eqnarray*}

$u?(x)P | u!\langle Q \rangle \red P\{\quotep{Q}/x\}$

%We write $\wred$ for $\red^*$, and $P\red$ if $\exists Q $ such that $ P \red Q$.
We write $P\red$ if $\exists Q $ such that $ P \red Q$ and $P\not\red$, otherwise.

\section{Replication}

As mentioned before, it is known that replication (and hence
recursion) can be implemented in a higher-order process algebra
\cite{SangiorgiWalker}. As our first example of calculation with the
machinery thus far presented we give the construction explicitly in
the {\rhoc}.

\begin{eqnarray}
	D_{x} & := & \prefix{x}{y}{(\binpar{\outputp{x}{y}}{@{y}})} \nonumber\\
	\bangp_{x}{P} & := & \binpar{{x}!\langle{\binpar{D_{x}}{P}}\rangle}{D_{x}} \nonumber
\end{eqnarray}

\begin{eqnarray}
	\bangp_{x}{P} & & \nonumber\\
	=
	& {x}!\langle{(\prefix{x}{y}{(\outputp{x}{y} | @{y})) | P}}\rangle 
	      | \prefix{x}{y}{(\outputp{x}{y} | @{y})} & \nonumber\\
	\red
	& (\outputp{x}{y} | @{y})\substn{\quotep{(\prefix{x}{y}{(@{y} | \outputp{x}{y})) | P}}}{y} & \nonumber\\
	=
	& \outputp{x}{\quotep{(\prefix{x}{y}{(\outputp{x}{y} | @{y})) | P}}}
	  | {(\prefix{x}{y}{(\outputp{x}{y} | @{y})) | P}} & \nonumber\\
	\red
	& \ldots & \nonumber\\
	\red^*
	& P | P | \ldots & \nonumber
\end{eqnarray}

Of course, this encoding, as an implementation, runs away, unfolding
$\bangp{P}$ eagerly. A lazier and more implementable replication
operator, restricted to input-guarded processes, may be obtained as follows.

\begin{eqnarray}
\bangp{\prefix{u}{v}{P}} 
	:= 
	\binpar{\lift{x}{\prefix{u}{v}{(\binpar{D(x)}{P})}}}{D(x)} \nonumber
\end{eqnarray}

\begin{remark}
  Note that the lazier definition still does not deal with summation
  or mixed summation (i.e. sums over input and output). The reader is
  invited to construct definitions of replication that deal with these
  features. 

  Further, the definitions are parameterized in a name, $x$. Can you,
  gentle reader, make a definition that eliminates this parameter and
  guarantees no accidental interaction between the replication
  machinery and the process being replicated -- i.e. no accidental
  sharing of names used by the process to get its work done and the
  name(s) used by the replication to effect copying. This latter
  revision of the definition of replication is crucial to obtaining
  the expected identity $!!P \sim !P$.
\end{remark}

\begin{remark}\label{rem:paradoxical_combinator}
  The reader familiar with the lambda calculus will have noticed the
  similarity between $D$ and the paradoxical combinator.

  [Ed. note: the existence of this seems to suggest we have to be more
  restrictive on the set of processes and names we admit if we are to
  support no-cloning.]
\end{remark}

\subsubsection{Bisimulation}

The computational dynamics gives rise to another kind of equivalence,
the equivalence of computational behavior. As previously mentioned
this is typically captured \emph{via} some form of bisimulation.

% The notion we use in this paper is weak barbed bisimulation
% \cite{milner91polyadicpi}.

The notion we use in this paper is derived from weak barbed
bisimulation \cite{milner91polyadicpi}. 

\begin{definition}
An \emph{observation relation}, $\downarrow_{\mathcal N}$, over a set
of names, $\mathcal N$, is the smallest relation satisfying the rules
below.

\infrule[Out-barb]{y \in {\mathcal N}, \; x \nameeq y}
		  {\outputp{x}{v} \downarrow_{\mathcal N} x}
\infrule[Par-barb]{\mbox{$P\downarrow_{\mathcal N} x$ or $Q\downarrow_{\mathcal N} x$}}
		  {\binpar{P}{Q} \downarrow_{\mathcal N} x}

We write $P \Downarrow_{\mathcal N} x$ if there is $Q$ such that 
$P \wred Q$ and $Q \downarrow_{\mathcal N} x$.
\end{definition}

\begin{definition}
%\label{def.bbisim}
An  ${\mathcal N}$-\emph{barbed bisimulation} over a set of names, ${\mathcal N}$, is a symmetric binary relation 
${\mathcal S}_{\mathcal N}$ between agents such that $P\rel{S}_{\mathcal N}Q$ implies:
\begin{enumerate}
\item If $P \red P'$ then $Q \wred Q'$ and $P'\rel{S}_{\mathcal N} Q'$.
\item If $P\downarrow_{\mathcal N} x$, then $Q\Downarrow_{\mathcal N} x$.
\end{enumerate}
$P$ is ${\mathcal N}$-barbed bisimilar to $Q$, written
$P \wbbisim_{\mathcal N} Q$, if $P \rel{S}_{\mathcal N} Q$ for some ${\mathcal N}$-barbed bisimulation ${\mathcal S}_{\mathcal N}$.
\end{definition}

$\mathcal{R} \subseteq \pi \times \pi$

$P \mathcal{R} Q => \forall P'. P \red P' \Rightarrow \exists Q'. Q \red Q', P' \mathcal{R} Q'$

$P \vdash x \Rightarrow Q \vdash x$

\begin{mathpar}
  \inferrule*[lab=Out-barb]{x \nameeq y}{{y}!\langle{Q}\rangle \vdash x}
  \and
  \inferrule*[lab=Par-barb]{\mbox{$P\vdash x$ or $Q\vdash x$}}{\binpar{P}{Q} \vdash x}
\end{mathpar}

\subsubsection{Contexts}

One of the principle advantages of computational calculi like the
$\pi$-calculus is a well-defined notion of context,
contextual-equivalence and a correlation between
contextual-equivalence and notions of bisimulation. The notion of
context allows the decomposition of a process into (sub-)process and
its syntactic environment, its context. Thus, a context may be
thought of as a process with a ``hole'' (written $\Box$) in it. The
application of a context $M$ to a process $P$, written $M[P]$, is
tantamount to filling the hole in $M$ with $P$. In this paper we do
not need the full weight of this theory, but do make use of the notion
of context in the proof the main theorem. 

\begin{mathpar}
  \inferrule* [lab=summation] {} {{M_{M},M_{N}} \bc \Box \;|\; x.M_{A} \;|\; M_{M}+M_{N}}
  \and
  \inferrule* [lab=agent] {} {{M_{A}} \bc (\vec{x})M_{P} \;| \; \clift{P_0,\ldots,M_{P},\ldots,P_N}}
  \and \\
  \inferrule* [lab=process] {} {{M_{P}} \bc M_{N} \;| \;P|M_{P} }
\end{mathpar} 

\begin{mathpar}
  \inferrule* [lab=sychronization] {} {M_{N} \bc \Box \;|\; x?M_{F} \;|\; x!M_{C}}
  \and
  \inferrule* [lab=abstraction] {} {{M_{F}} \bc (x)M_{P} }
  \and
  \inferrule* [lab=concretion] {} {{M_{C}} \bc \langle M_{P} \rangle }
  \and \\
  \inferrule* [lab=process] {} {{M_{P}} \bc M_{N} \;| \;P|M_{P} }
\end{mathpar}

\begin{definition}[contextual application] Given a context $M$, and
  process $P$, we define the \emph{contextual application}, $M[P] :=
  M\{P/\Box\}$. That is, the contextual application of M to P is the
  substitution of $P$ for $\Box$ in $M$.
\end{definition}

$\meaningof{-} : L \to \mathcal{P}(\pi)$

\begin{mathpar}
  \inferrule* [lab=collection] {} {\meaningof{true} = \pi, \and \meaningof{~E} = \pi \setminus \meaningof{E}, \and \meaningof{E_{1} \& E_{2}} = \meaningof{E_{1}} \cap \meaningof{E_{2}}}
\end{mathpar}

\begin{mathpar}
  \inferrule* [lab=structure] {} {\meaningof{0} = \{ P \in \pi | P \equiv 0 \}, \and \\ \meaningof{E_1 | E_2} = \{ P \in \pi | P \equiv P_{1} | P_{2}, P_{1} \in \meaningof{E_{1}}, P_{2} \in \meaningof{E_2}\} }
\end{mathpar}

\begin{mathpar}
 \inferrule* [lab=behavior] {} {\meaningof{\langle a?b \rangle E} = \{ P \in \pi | P \equiv Q | u?(y)P', \\ \and \\\\ \and \\ \;\;\; u \in \meaningof{a}, \forall z.P'\{z/y\} \in \meaningof{E\{z/b\}}\}, \and \\ \meaningof{a!E} = \{ P \in \pi | P \equiv Q | x!\langle P' \rangle, x \in \meaningof{a} P' \in \meaningof{E}\} }
\end{mathpar}

\begin{mathpar}
 \inferrule* [lab=nominal] {} {\meaningof{\quotep{E}} = \{ \quotep{P} \in \quotep{\pi} | P \in \meaningof{E} \}, \and \meaningof{\quotep{P}} = \{ \quotep{Q} \in \quotep{\pi} | P \equiv Q \} \and \\ \meaningof{@\quotep{E}} = \{ P \in \pi | P \equiv @x, x \in \meaningof{E} \}}
\end{mathpar}

\begin{eqnarray*}
  \\
  \meaningof{-} : TS \to ST
\end{eqnarray*}

\begin{eqnarray*}
  \\
  L : TS \to ST
\end{eqnarray*}

\begin{eqnarray*}
  \\
  P \models E \iff P \in \meaningof{E}
\end{eqnarray*}

\begin{eqnarray*}
  P \approx_{L} Q \iff \forall E \in L. P \models E \iff Q \models E
\end{eqnarray*}

\begin{eqnarray*}
  P \approx_{K} Q
\end{eqnarray*}

\begin{eqnarray*}
  P \approx Q
\end{eqnarray*}

$\approx_{K} = \approx = \approx_{L}$

\subsubsection{Contextual duality}

Note that contexts extend the quotation operation to a family of
operations from processes to names. Given a context, $M$, we can
define a \emph{nominal context}, $\quotep{M}$ by $\quotep{M}[P] :=
\quotep{M[P]}$. To foreshadow what is to come we observe that these
operations enjoy a duality with processes very much like the duality
between vectors and maps from vectors to scalars.

Further, because the calculus is essentially higher-order, we have a
correspondence between contexts and processes. More specifically,
given a name $x$ and a context $M$ we can construct $M^{*}_{x}$ such
that 

\begin{mathpar}
  M^{*}_{x} | \lift{x}{P} \red M[P]
\end{mathpar}

namely,

\begin{mathpar}
  M^{*}_{x} := x?(u).M[\dropn{u}]
\end{mathpar}

The dependence of $M^{*}_{x}$ on a name makes it an abstraction, 

\begin{mathpar}
  M^{*} := (x)x?(u).M[\dropn{u}]
\end{mathpar}

\subsection{Additional notation}

It will sometimes be convenient to denote the process a name
quotes. We already have the notation $x = \quotep{P}$, but it will be
convenient to introduce an alternate notation, $\procn{x}$, when we
want to emphasize the connection to the use of the name. Note that, by
virtue of name equivalence, $\quotep{\procn{x}} \nameeq x$; so, the
notation is consistent with previous definitions.

Further, because names have structure it is possible to effect
substitutions on the basis of that structure. This means we need to
upgrade our notation for substitutions, which we accomplish by
adapting comprehension notation. Thus,

\begin{mathpar}
  P\{ y / x : x \in S \}
\end{mathpar}

is interpreted to mean the process derived from P by replacing (in a
capture-avoiding manner) each occurrence of $x$ in $S$ by $y$. For example,

\begin{mathpar}
  P\{ \quotep{\procn{x}|\procn{x}} / x : x \in \freenames{P} \}
\end{mathpar}

will replace each (occurrence) of a free name $x$ in $P$ by
$\quotep{\procn{x}|\procn{x}}$.

Also, we will avail ourselves of the notation $x^{L}$ and $x^{R}$ to
denote injections of a name into disjoint copies of the name
space. There are numerous ways to accomplish this. One example can be
found in \cite{MeredithR05}. This notation overloads to vectors of
names: $\vec{x}^{\pi} := (x_{i}^{\pi} \; : \; 0 \leq i < |\vec{x}| )$ where $\pi \in \{L,R\}$.

We also use $P^{\Box} := P|\Box$.

In \cite{MeredithR05} an interpretation of the new operator is
given. It turns out that there are several possible interpretations
all enjoying the requisite algebraic properties of the operator (see
\cite{milner91polyadicpi}). We will therefore make liberal use of
$(\nu\; \vec{x})P$.

% subsection the_syntax_and_semantics_of_the_notation_system (end)   

\input{qm2pi.qmops} 

\input{qm2pi.sterngerlach} 

\input{qm2pi.metric} 

% section concurrent_process_calculi (end)

%\input{qm2pi.proofsketch}

% section proof sketch (end)

%\input{qm2pi.slviaknots} 

% section spatial logic via knots (end)

\input{qm2pi.conclusion}

% section conclusion (end)

%\input{qm2pi.dtcodes} 

% section wiring algorithm (end)

\input{qm2pi.ack} 

% section acknowledgments (end)

\newpage


\bibliographystyle{plain}   
\bibliography{../../biblios/main.bib}

\input{qm2pi.rhodetails}

\end{document}

 

% section acknowledgments (end)

\newpage


\bibliographystyle{plain}   
\bibliography{../../biblios/main.bib}

\documentclass[12pt]{llncs}
%\documentclass{jktr}

\usepackage[pdftex]{hyperref}                   
\usepackage {listings}
\usepackage {mathpartir}
\usepackage{bcprules}
%\usepackage{listings}
                       
\usepackage{graphicx} 
%\usepackage[margins=2.5cm,nohead,nofoot]{geometry}
%\usepackage{geometry}
\usepackage{amsfonts}
\usepackage{amstext}
\usepackage{latexsym}
\usepackage{amssymb}
\usepackage{color}


%\include{myPreamble}
\include{qm2pi.local} 

%\ifpdf
%\usepackage[pdftex]{graphicx}
%\else
%\usepackage{graphicx}
%\fi

 % \ifpdf
%  \usepackage{pdfsync}
%  \if


%\title{Brief Article}
%\author{David F. Snyder}
%\author{L.G. Meredith}

%\address{Dept. of Math., Texas State University--San Marcos, San Marcos, TX 78666}
       
\pagestyle{empty}


\begin{document}

\lstset{language=[Objective]Caml,frame=shadowbox}

\input{qm2pi.front}

% section front matter (end)

\input{qm2pi.intro} 
 
% section introduction (end)

% \input{qm2pi.knotations} 

% section notation (end)

\input{qm2pi.process.calculi} 

% section concurrent_process_calculi_and_spatial_logics_ (end)
    
%\input{qm2pi.knots2pi} 

%\input{qm2pi.trefoil} 

%\input{qm2pi.mainthm} 

% subsection basic_interpretation (end)

%\input{qm2pi.rho.presentation} 
\subsection{The syntax and semantics of the notation system}\label{sub:the_syntax_and_semantics_of_the_notation_system} % (fold)

We now summarize a technical presentation of the calculus that
embodies our theory of dynamics. The typical presentation of such a
calculus follows the style of giving generators and relations on
them. The grammar, below, describing term constructors, freely
generates the set of processes, $\Proc$. This set is then quotiented
by a relation known as structural congruence and it is over this set
that the notion of dynamics is expressed. This presentation is
essentially that of \cite{MeredithR05} with the addition of
polyadicity and summation. For readability we have relegated some of
the technical subtleties to an appendix.

\subsubsection{Process grammar}\label{subsub:process_grammar}

\begin{mathpar}
  \inferrule* [lab=synchronization] {} {{M} \bc \pzero \;|\; x?F \;|\; x!C }
  \and
  \inferrule* [lab=abstraction] {} {{F} \bc (x)P}
  \and
  \inferrule* [lab=concretion] {} {{C} \bc \langle Q \rangle}
  \and
  \inferrule* [lab=process] {} {{P,Q} \bc M \;| \;P|Q \;|\; @{x}}
  \and
  \inferrule* [lab=name] {} {{x} \bc \quotep{P}}
\end{mathpar} 

Note that $\vec{x}$ (resp. $\vec{P}$) denotes a vector of names
(resp. processes) of length $|\vec{x}|$ (resp. $|\vec{P}|$). We adopt
the following useful abbreviations.

\begin{mathpar}
   x?(\vec{y}).P := x.(\vec{y})P \and  x\clift{\vec{P}} := x.\clift{\vec{P}}
   \and x!(y) := \lift{x}{\dropn{y}}
   \and \Pi_{i=0}^{n-1}P_i := P_0 | \ldots | P_{n-1}
\end{mathpar}

\subsubsection{Structural congruence}

\paragraph{Free and bound names and alpha-equivalence.} At the
core of structural equivalence is alpha-equivalence which identifies
process that are the same up to a change of variable. Formally, we
recognize the distinction between free and bound names. The free names
of a process, $\freenames{P}$, may be calculated recursively as
follows:

\begin{mathpar}
\freenames{\pzero} := \emptyset
  \and \\
  \freenames{x?(y).P} := \{ x \} \cup (\freenames{P} \setminus \{ y \})
  \and 
  \freenames{x!\langle P \rangle} := \{ x \} \cup \{ P \} 
  \and \\
  \freenames{P|Q} := \freenames{P} \cup \freenames{Q}
  \and \\
  \freenames{@{x}} := \{ x \}
\end{mathpar}

$\pi$
$\quotep{\pi}$

$\freenames{-} : \pi \to \mathcal{P}(\quotep{\pi})$

\begin{eqnarray*}
  \freenames{\pzero} & := & \emptyset \\
  \freenames{x?(y).P} & := & \{ x \} \cup (\freenames{P} \setminus \{ y \}) \\
  \freenames{x!\langle P \rangle} & := & \{ x \} \cup \{ P \} \\
  \freenames{P|Q} & := & \freenames{P} \cup \freenames{Q} \\
  \freenames{\dropn{x}} & := & \{ x \}
\end{eqnarray*}

The bound names of a process, $\boundnames{P}$, are those names occurring in $P$
that are not free. For example, in $x?(y).0$, the name $x$ is free, while $y$ is bound.

\begin{mathpar}
  \inferrule* [lab=monoidal-laws] {} { P|Q \equiv Q|P \and P|0 \equiv P \and P|(Q|R) \equiv (P|Q)|R }
\end{mathpar}

\begin{mathpar}
  \inferrule* [lab=alpha-equivalence] {} { (x)P \equiv (y)P\{y/x\} \and y \not\in \freenames{P} }
\end{mathpar}

\begin{definition}
Then two processes, $P,Q$, are alpha-equivalent if $P = Q\{\vec{y}/\vec{x}\}$ for
some $\vec{x} \in \boundnames{Q},\vec{y} \in \boundnames{P}$, where $Q\{\vec{y}/\vec{x}\}$
denotes the capture-avoiding substitution of $\vec{y}$ for $\vec{x}$ in $Q$.
\end{definition}

\begin{definition}
  The {\em structural congruence} \cite{SangiorgiWalker} , $\equiv$,
  between processes is the least congruence containing
  alpha-equivalence, satisfying the abelian monoid laws
  (associativity, commutativity and $\pzero$ as identity) for parallel
  composition $|$ and for summation $+$.
\end{definition}

\subsection{Name equivalence}

We take name equivalence, written $\nameeq$, to be the smallest
equivalence relation generated by the following rules.

\begin{mathpar}
\inferrule*[lab=Quote-drop]
{ }
{ \quotep{@{x}} \nameeq x }

\inferrule*[lab=Struct-equiv]
{ P \scong Q }
{ \quotep{P} \nameeq \quotep{Q} }
\end{mathpar}

The astute reader will have noticed that the mutual recursion of names
and processes imposes a mutual recursion on alpha-equivalence and
structural equivalence via name-equivalence. Fortunately, all of this
works out pleasantly and we may calculate in the natural way, free of
concern. The reader interested in the details is referred to the
appendix \ref{appendix:rho_details}.

\subsection{Substitution}

We use $\Proc$ for the set of processes, $\QProc$ for the set of
names, and $\id{\{}\vec{y} / \vec{x} \id{\}}$ to denote partial maps,
$s : \QProc \rightarrow \QProc$. A map, $s$ lifts, uniquely, to a map
on process terms, $\widehat{s} : \Proc \rightarrow \Proc$ by the
following equations.

\begin{mathpar}
  (0) \psubstp{Q}{P} := 0 \\
  (R \juxtap S) \psubstp{Q}{P}
  :=    
  (R)\psubstp{Q}{P} \juxtap (S) \psubstp{Q}{P} \\
  (x?(y).R) \psubstp{Q}{P}    
  :=    
  (x)\substp{Q}{P} (z)\concat( (R \psubstn{z}{y}) \psubstp{Q}{P} ) \\
  (\lift{x}{R}) \psubstp{Q}{P}  
  :=
  \lift{(x)\substp{Q}{P}}{ R \psubstp{Q}{P} } \\
%   (\dropn{x})  \psubstp{Q}{P}       
%   := 
%   \left\{ 
%     \begin{array}{ccc} 
%       \dropn{\quotep{Q}} & & x \nameeq \quotep{P} \\
%       \dropn{x} & & otherwise \\
%     \end{array}
%   \right. 
  (\dropn{x})  \psubstp{Q}{P}       
  := 
  \left\{ 
    \begin{array}{ccc} 
      Q & & x \nameeq \quotep{P} \\
      \dropn{x} & & otherwise \\
    \end{array}
  \right.
\end{mathpar}
 

where

\begin{eqnarray}
  (x)\id{\{} \lpquote Q \rpquote / \lpquote P \rpquote \id{\}}            = 
  \left\{ 
    \begin{array}{ccc}
      \lpquote Q \rpquote & & x \nameeq \lpquote P \rpquote \\
      x & & otherwise \\
    \end{array}
  \right. \nonumber
\end{eqnarray}

and $z$ is chosen distinct from $\quotep{P}$, $\quotep{Q}$, the free
names in $Q$, and all the names in $R$. Our $\alpha$-equivalence will
be built in the standard way from this substitution.

\begin{remark}\label{rem:no_self_referential_names}
  One consequence of these definitions is that $\forall P. \quotep{P}
  \not\in \freenames{P}$.
\end{remark}

\subsection{ Dynamic quote: an example }

Anticipating something of what's to come, consider applying the
substitution, $\widehat{\id{\{}u / z \id{\}}}$, to the following pair
of processes, $\lift{w}{y!(z)}$ and $w[ \lpquote y!(z) \rpquote ]$.

\begin{eqnarray}
	\lift{w}{y!(z)}\widehat{\id{\{}u / z \id{\}}}
		& = &
		\lift{w}{y!(u)} \nonumber\\
	w[ \lpquote y!(z) \rpquote ] \widehat{ \id{\{}u / z \id{\}} }
		& = &
		w[ \lpquote y!(z) \rpquote ] \nonumber
\end{eqnarray}

Because the body of the process between quotes is impervious to
substitution, we get radically different answers. In fact, by
examining the first process in an input context,
e.g. $x?(z).\lift{w}{y!(z)}$, we see that the process under the lift
operator may be shaped by prefixed inputs binding a name inside it. In
this sense, the lift operator will be seen as a way to dynamically
construct processes before reifying them as names.

Finally equipped with these standard features we can present the
dynamics of the calculus.

\subsubsection{Operational semantics} 

Finally, we introduce the computational dynamics. What marks these
algebras as distinct from other more traditionally studied algebraic
structures, e.g. vector spaces or polynomial rings, is the manner in
which dynamics is captured. In traditional structures, dynamics is typically
expressed through morphisms between such structures, as in linear maps
between vector spaces or morphisms between rings. In algebras
associated with the semantics of computation, the dynamics is
expressed as part of the algebraic structure itself, through a
reduction reduction relation typically denoted by $\red$. Below, we
give a recursive presentation of this relation for the calculus used
in the encoding.

$\red \subseteq \pi \times \pi$
$\red : \pi \to \mathcal{P}(\pi)$

\begin{mathpar}
  \inferrule* [lab=Comm] { \textsf{match}( x_{src}, x_{trgt} ) } { x_{trgt}?(y)P \; | \; x_{src}!\langle {Q} \rangle \red P\{\quotep{Q}/y}\} }
  \and \\
  \inferrule* [lab=Par] {{P} \red {P}'} {{{P} | {Q}} \red {{P}' | {Q}}}
  \and
  \inferrule* [lab=Equiv]{{{P} \scong {P}'} \andalso {{P}' \red {Q}'} \andalso {{Q}' \scong {Q}}}{{P} \red {Q}}
\end{mathpar}

\begin{eqnarray*}
  match_{\equiv} (\quotep{P},\quotep{Q}) & := & P \equiv Q \\
  match_{\dagger}(\quotep{P},\quotep{Q}) & := & \forall R. P|Q \red^{*} R => R \red^{*} 0 \\
  match_{K}(\quotep{P},\quotep{Q}) & := & K \mbox{ for some context } K
\end{eqnarray*}

$u?(x)P | u!\langle Q \rangle \red P\{\quotep{Q}/x\}$

%We write $\wred$ for $\red^*$, and $P\red$ if $\exists Q $ such that $ P \red Q$.
We write $P\red$ if $\exists Q $ such that $ P \red Q$ and $P\not\red$, otherwise.

\section{Replication}

As mentioned before, it is known that replication (and hence
recursion) can be implemented in a higher-order process algebra
\cite{SangiorgiWalker}. As our first example of calculation with the
machinery thus far presented we give the construction explicitly in
the {\rhoc}.

\begin{eqnarray}
	D_{x} & := & \prefix{x}{y}{(\binpar{\outputp{x}{y}}{@{y}})} \nonumber\\
	\bangp_{x}{P} & := & \binpar{{x}!\langle{\binpar{D_{x}}{P}}\rangle}{D_{x}} \nonumber
\end{eqnarray}

\begin{eqnarray}
	\bangp_{x}{P} & & \nonumber\\
	=
	& {x}!\langle{(\prefix{x}{y}{(\outputp{x}{y} | @{y})) | P}}\rangle 
	      | \prefix{x}{y}{(\outputp{x}{y} | @{y})} & \nonumber\\
	\red
	& (\outputp{x}{y} | @{y})\substn{\quotep{(\prefix{x}{y}{(@{y} | \outputp{x}{y})) | P}}}{y} & \nonumber\\
	=
	& \outputp{x}{\quotep{(\prefix{x}{y}{(\outputp{x}{y} | @{y})) | P}}}
	  | {(\prefix{x}{y}{(\outputp{x}{y} | @{y})) | P}} & \nonumber\\
	\red
	& \ldots & \nonumber\\
	\red^*
	& P | P | \ldots & \nonumber
\end{eqnarray}

Of course, this encoding, as an implementation, runs away, unfolding
$\bangp{P}$ eagerly. A lazier and more implementable replication
operator, restricted to input-guarded processes, may be obtained as follows.

\begin{eqnarray}
\bangp{\prefix{u}{v}{P}} 
	:= 
	\binpar{\lift{x}{\prefix{u}{v}{(\binpar{D(x)}{P})}}}{D(x)} \nonumber
\end{eqnarray}

\begin{remark}
  Note that the lazier definition still does not deal with summation
  or mixed summation (i.e. sums over input and output). The reader is
  invited to construct definitions of replication that deal with these
  features. 

  Further, the definitions are parameterized in a name, $x$. Can you,
  gentle reader, make a definition that eliminates this parameter and
  guarantees no accidental interaction between the replication
  machinery and the process being replicated -- i.e. no accidental
  sharing of names used by the process to get its work done and the
  name(s) used by the replication to effect copying. This latter
  revision of the definition of replication is crucial to obtaining
  the expected identity $!!P \sim !P$.
\end{remark}

\begin{remark}\label{rem:paradoxical_combinator}
  The reader familiar with the lambda calculus will have noticed the
  similarity between $D$ and the paradoxical combinator.

  [Ed. note: the existence of this seems to suggest we have to be more
  restrictive on the set of processes and names we admit if we are to
  support no-cloning.]
\end{remark}

\subsubsection{Bisimulation}

The computational dynamics gives rise to another kind of equivalence,
the equivalence of computational behavior. As previously mentioned
this is typically captured \emph{via} some form of bisimulation.

% The notion we use in this paper is weak barbed bisimulation
% \cite{milner91polyadicpi}.

The notion we use in this paper is derived from weak barbed
bisimulation \cite{milner91polyadicpi}. 

\begin{definition}
An \emph{observation relation}, $\downarrow_{\mathcal N}$, over a set
of names, $\mathcal N$, is the smallest relation satisfying the rules
below.

\infrule[Out-barb]{y \in {\mathcal N}, \; x \nameeq y}
		  {\outputp{x}{v} \downarrow_{\mathcal N} x}
\infrule[Par-barb]{\mbox{$P\downarrow_{\mathcal N} x$ or $Q\downarrow_{\mathcal N} x$}}
		  {\binpar{P}{Q} \downarrow_{\mathcal N} x}

We write $P \Downarrow_{\mathcal N} x$ if there is $Q$ such that 
$P \wred Q$ and $Q \downarrow_{\mathcal N} x$.
\end{definition}

\begin{definition}
%\label{def.bbisim}
An  ${\mathcal N}$-\emph{barbed bisimulation} over a set of names, ${\mathcal N}$, is a symmetric binary relation 
${\mathcal S}_{\mathcal N}$ between agents such that $P\rel{S}_{\mathcal N}Q$ implies:
\begin{enumerate}
\item If $P \red P'$ then $Q \wred Q'$ and $P'\rel{S}_{\mathcal N} Q'$.
\item If $P\downarrow_{\mathcal N} x$, then $Q\Downarrow_{\mathcal N} x$.
\end{enumerate}
$P$ is ${\mathcal N}$-barbed bisimilar to $Q$, written
$P \wbbisim_{\mathcal N} Q$, if $P \rel{S}_{\mathcal N} Q$ for some ${\mathcal N}$-barbed bisimulation ${\mathcal S}_{\mathcal N}$.
\end{definition}

$\mathcal{R} \subseteq \pi \times \pi$

$P \mathcal{R} Q => \forall P'. P \red P' \Rightarrow \exists Q'. Q \red Q', P' \mathcal{R} Q'$

$P \vdash x \Rightarrow Q \vdash x$

\begin{mathpar}
  \inferrule*[lab=Out-barb]{x \nameeq y}{{y}!\langle{Q}\rangle \vdash x}
  \and
  \inferrule*[lab=Par-barb]{\mbox{$P\vdash x$ or $Q\vdash x$}}{\binpar{P}{Q} \vdash x}
\end{mathpar}

\subsubsection{Contexts}

One of the principle advantages of computational calculi like the
$\pi$-calculus is a well-defined notion of context,
contextual-equivalence and a correlation between
contextual-equivalence and notions of bisimulation. The notion of
context allows the decomposition of a process into (sub-)process and
its syntactic environment, its context. Thus, a context may be
thought of as a process with a ``hole'' (written $\Box$) in it. The
application of a context $M$ to a process $P$, written $M[P]$, is
tantamount to filling the hole in $M$ with $P$. In this paper we do
not need the full weight of this theory, but do make use of the notion
of context in the proof the main theorem. 

\begin{mathpar}
  \inferrule* [lab=summation] {} {{M_{M},M_{N}} \bc \Box \;|\; x.M_{A} \;|\; M_{M}+M_{N}}
  \and
  \inferrule* [lab=agent] {} {{M_{A}} \bc (\vec{x})M_{P} \;| \; \clift{P_0,\ldots,M_{P},\ldots,P_N}}
  \and \\
  \inferrule* [lab=process] {} {{M_{P}} \bc M_{N} \;| \;P|M_{P} }
\end{mathpar} 

\begin{mathpar}
  \inferrule* [lab=sychronization] {} {M_{N} \bc \Box \;|\; x?M_{F} \;|\; x!M_{C}}
  \and
  \inferrule* [lab=abstraction] {} {{M_{F}} \bc (x)M_{P} }
  \and
  \inferrule* [lab=concretion] {} {{M_{C}} \bc \langle M_{P} \rangle }
  \and \\
  \inferrule* [lab=process] {} {{M_{P}} \bc M_{N} \;| \;P|M_{P} }
\end{mathpar}

\begin{definition}[contextual application] Given a context $M$, and
  process $P$, we define the \emph{contextual application}, $M[P] :=
  M\{P/\Box\}$. That is, the contextual application of M to P is the
  substitution of $P$ for $\Box$ in $M$.
\end{definition}

$\meaningof{-} : L \to \mathcal{P}(\pi)$

\begin{mathpar}
  \inferrule* [lab=collection] {} {\meaningof{true} = \pi, \and \meaningof{~E} = \pi \setminus \meaningof{E}, \and \meaningof{E_{1} \& E_{2}} = \meaningof{E_{1}} \cap \meaningof{E_{2}}}
\end{mathpar}

\begin{mathpar}
  \inferrule* [lab=structure] {} {\meaningof{0} = \{ P \in \pi | P \equiv 0 \}, \and \\ \meaningof{E_1 | E_2} = \{ P \in \pi | P \equiv P_{1} | P_{2}, P_{1} \in \meaningof{E_{1}}, P_{2} \in \meaningof{E_2}\} }
\end{mathpar}

\begin{mathpar}
 \inferrule* [lab=behavior] {} {\meaningof{\langle a?b \rangle E} = \{ P \in \pi | P \equiv Q | u?(y)P', \\ \and \\\\ \and \\ \;\;\; u \in \meaningof{a}, \forall z.P'\{z/y\} \in \meaningof{E\{z/b\}}\}, \and \\ \meaningof{a!E} = \{ P \in \pi | P \equiv Q | x!\langle P' \rangle, x \in \meaningof{a} P' \in \meaningof{E}\} }
\end{mathpar}

\begin{mathpar}
 \inferrule* [lab=nominal] {} {\meaningof{\quotep{E}} = \{ \quotep{P} \in \quotep{\pi} | P \in \meaningof{E} \}, \and \meaningof{\quotep{P}} = \{ \quotep{Q} \in \quotep{\pi} | P \equiv Q \} \and \\ \meaningof{@\quotep{E}} = \{ P \in \pi | P \equiv @x, x \in \meaningof{E} \}}
\end{mathpar}

\begin{eqnarray*}
  \\
  \meaningof{-} : TS \to ST
\end{eqnarray*}

\begin{eqnarray*}
  \\
  L : TS \to ST
\end{eqnarray*}

\begin{eqnarray*}
  \\
  P \models E \iff P \in \meaningof{E}
\end{eqnarray*}

\begin{eqnarray*}
  P \approx_{L} Q \iff \forall E \in L. P \models E \iff Q \models E
\end{eqnarray*}

\begin{eqnarray*}
  P \approx_{K} Q
\end{eqnarray*}

\begin{eqnarray*}
  P \approx Q
\end{eqnarray*}

$\approx_{K} = \approx = \approx_{L}$

\subsubsection{Contextual duality}

Note that contexts extend the quotation operation to a family of
operations from processes to names. Given a context, $M$, we can
define a \emph{nominal context}, $\quotep{M}$ by $\quotep{M}[P] :=
\quotep{M[P]}$. To foreshadow what is to come we observe that these
operations enjoy a duality with processes very much like the duality
between vectors and maps from vectors to scalars.

Further, because the calculus is essentially higher-order, we have a
correspondence between contexts and processes. More specifically,
given a name $x$ and a context $M$ we can construct $M^{*}_{x}$ such
that 

\begin{mathpar}
  M^{*}_{x} | \lift{x}{P} \red M[P]
\end{mathpar}

namely,

\begin{mathpar}
  M^{*}_{x} := x?(u).M[\dropn{u}]
\end{mathpar}

The dependence of $M^{*}_{x}$ on a name makes it an abstraction, 

\begin{mathpar}
  M^{*} := (x)x?(u).M[\dropn{u}]
\end{mathpar}

\subsection{Additional notation}

It will sometimes be convenient to denote the process a name
quotes. We already have the notation $x = \quotep{P}$, but it will be
convenient to introduce an alternate notation, $\procn{x}$, when we
want to emphasize the connection to the use of the name. Note that, by
virtue of name equivalence, $\quotep{\procn{x}} \nameeq x$; so, the
notation is consistent with previous definitions.

Further, because names have structure it is possible to effect
substitutions on the basis of that structure. This means we need to
upgrade our notation for substitutions, which we accomplish by
adapting comprehension notation. Thus,

\begin{mathpar}
  P\{ y / x : x \in S \}
\end{mathpar}

is interpreted to mean the process derived from P by replacing (in a
capture-avoiding manner) each occurrence of $x$ in $S$ by $y$. For example,

\begin{mathpar}
  P\{ \quotep{\procn{x}|\procn{x}} / x : x \in \freenames{P} \}
\end{mathpar}

will replace each (occurrence) of a free name $x$ in $P$ by
$\quotep{\procn{x}|\procn{x}}$.

Also, we will avail ourselves of the notation $x^{L}$ and $x^{R}$ to
denote injections of a name into disjoint copies of the name
space. There are numerous ways to accomplish this. One example can be
found in \cite{MeredithR05}. This notation overloads to vectors of
names: $\vec{x}^{\pi} := (x_{i}^{\pi} \; : \; 0 \leq i < |\vec{x}| )$ where $\pi \in \{L,R\}$.

We also use $P^{\Box} := P|\Box$.

In \cite{MeredithR05} an interpretation of the new operator is
given. It turns out that there are several possible interpretations
all enjoying the requisite algebraic properties of the operator (see
\cite{milner91polyadicpi}). We will therefore make liberal use of
$(\nu\; \vec{x})P$.

% subsection the_syntax_and_semantics_of_the_notation_system (end)   

\input{qm2pi.qmops} 

\input{qm2pi.sterngerlach} 

\input{qm2pi.metric} 

% section concurrent_process_calculi (end)

%\input{qm2pi.proofsketch}

% section proof sketch (end)

%\input{qm2pi.slviaknots} 

% section spatial logic via knots (end)

\input{qm2pi.conclusion}

% section conclusion (end)

%\input{qm2pi.dtcodes} 

% section wiring algorithm (end)

\input{qm2pi.ack} 

% section acknowledgments (end)

\newpage


\bibliographystyle{plain}   
\bibliography{../../biblios/main.bib}

\input{qm2pi.rhodetails}

\end{document}



\end{document}

 

%\documentclass[12pt]{llncs}
%\documentclass{jktr}

\usepackage[pdftex]{hyperref}                   
\usepackage {listings}
\usepackage {mathpartir}
\usepackage{bcprules}
%\usepackage{listings}
                       
\usepackage{graphicx} 
%\usepackage[margins=2.5cm,nohead,nofoot]{geometry}
%\usepackage{geometry}
\usepackage{amsfonts}
\usepackage{amstext}
\usepackage{latexsym}
\usepackage{amssymb}
\usepackage{color}


%\include{myPreamble}
\documentclass[12pt]{llncs}
%\documentclass{jktr}

\usepackage[pdftex]{hyperref}                   
\usepackage {listings}
\usepackage {mathpartir}
\usepackage{bcprules}
%\usepackage{listings}
                       
\usepackage{graphicx} 
%\usepackage[margins=2.5cm,nohead,nofoot]{geometry}
%\usepackage{geometry}
\usepackage{amsfonts}
\usepackage{amstext}
\usepackage{latexsym}
\usepackage{amssymb}
\usepackage{color}


%\include{myPreamble}
\include{qm2pi.local} 

%\ifpdf
%\usepackage[pdftex]{graphicx}
%\else
%\usepackage{graphicx}
%\fi

 % \ifpdf
%  \usepackage{pdfsync}
%  \if


%\title{Brief Article}
%\author{David F. Snyder}
%\author{L.G. Meredith}

%\address{Dept. of Math., Texas State University--San Marcos, San Marcos, TX 78666}
       
\pagestyle{empty}


\begin{document}

\lstset{language=[Objective]Caml,frame=shadowbox}

\input{qm2pi.front}

% section front matter (end)

\input{qm2pi.intro} 
 
% section introduction (end)

% \input{qm2pi.knotations} 

% section notation (end)

\input{qm2pi.process.calculi} 

% section concurrent_process_calculi_and_spatial_logics_ (end)
    
%\input{qm2pi.knots2pi} 

%\input{qm2pi.trefoil} 

%\input{qm2pi.mainthm} 

% subsection basic_interpretation (end)

%\input{qm2pi.rho.presentation} 
\subsection{The syntax and semantics of the notation system}\label{sub:the_syntax_and_semantics_of_the_notation_system} % (fold)

We now summarize a technical presentation of the calculus that
embodies our theory of dynamics. The typical presentation of such a
calculus follows the style of giving generators and relations on
them. The grammar, below, describing term constructors, freely
generates the set of processes, $\Proc$. This set is then quotiented
by a relation known as structural congruence and it is over this set
that the notion of dynamics is expressed. This presentation is
essentially that of \cite{MeredithR05} with the addition of
polyadicity and summation. For readability we have relegated some of
the technical subtleties to an appendix.

\subsubsection{Process grammar}\label{subsub:process_grammar}

\begin{mathpar}
  \inferrule* [lab=synchronization] {} {{M} \bc \pzero \;|\; x?F \;|\; x!C }
  \and
  \inferrule* [lab=abstraction] {} {{F} \bc (x)P}
  \and
  \inferrule* [lab=concretion] {} {{C} \bc \langle Q \rangle}
  \and
  \inferrule* [lab=process] {} {{P,Q} \bc M \;| \;P|Q \;|\; @{x}}
  \and
  \inferrule* [lab=name] {} {{x} \bc \quotep{P}}
\end{mathpar} 

Note that $\vec{x}$ (resp. $\vec{P}$) denotes a vector of names
(resp. processes) of length $|\vec{x}|$ (resp. $|\vec{P}|$). We adopt
the following useful abbreviations.

\begin{mathpar}
   x?(\vec{y}).P := x.(\vec{y})P \and  x\clift{\vec{P}} := x.\clift{\vec{P}}
   \and x!(y) := \lift{x}{\dropn{y}}
   \and \Pi_{i=0}^{n-1}P_i := P_0 | \ldots | P_{n-1}
\end{mathpar}

\subsubsection{Structural congruence}

\paragraph{Free and bound names and alpha-equivalence.} At the
core of structural equivalence is alpha-equivalence which identifies
process that are the same up to a change of variable. Formally, we
recognize the distinction between free and bound names. The free names
of a process, $\freenames{P}$, may be calculated recursively as
follows:

\begin{mathpar}
\freenames{\pzero} := \emptyset
  \and \\
  \freenames{x?(y).P} := \{ x \} \cup (\freenames{P} \setminus \{ y \})
  \and 
  \freenames{x!\langle P \rangle} := \{ x \} \cup \{ P \} 
  \and \\
  \freenames{P|Q} := \freenames{P} \cup \freenames{Q}
  \and \\
  \freenames{@{x}} := \{ x \}
\end{mathpar}

$\pi$
$\quotep{\pi}$

$\freenames{-} : \pi \to \mathcal{P}(\quotep{\pi})$

\begin{eqnarray*}
  \freenames{\pzero} & := & \emptyset \\
  \freenames{x?(y).P} & := & \{ x \} \cup (\freenames{P} \setminus \{ y \}) \\
  \freenames{x!\langle P \rangle} & := & \{ x \} \cup \{ P \} \\
  \freenames{P|Q} & := & \freenames{P} \cup \freenames{Q} \\
  \freenames{\dropn{x}} & := & \{ x \}
\end{eqnarray*}

The bound names of a process, $\boundnames{P}$, are those names occurring in $P$
that are not free. For example, in $x?(y).0$, the name $x$ is free, while $y$ is bound.

\begin{mathpar}
  \inferrule* [lab=monoidal-laws] {} { P|Q \equiv Q|P \and P|0 \equiv P \and P|(Q|R) \equiv (P|Q)|R }
\end{mathpar}

\begin{mathpar}
  \inferrule* [lab=alpha-equivalence] {} { (x)P \equiv (y)P\{y/x\} \and y \not\in \freenames{P} }
\end{mathpar}

\begin{definition}
Then two processes, $P,Q$, are alpha-equivalent if $P = Q\{\vec{y}/\vec{x}\}$ for
some $\vec{x} \in \boundnames{Q},\vec{y} \in \boundnames{P}$, where $Q\{\vec{y}/\vec{x}\}$
denotes the capture-avoiding substitution of $\vec{y}$ for $\vec{x}$ in $Q$.
\end{definition}

\begin{definition}
  The {\em structural congruence} \cite{SangiorgiWalker} , $\equiv$,
  between processes is the least congruence containing
  alpha-equivalence, satisfying the abelian monoid laws
  (associativity, commutativity and $\pzero$ as identity) for parallel
  composition $|$ and for summation $+$.
\end{definition}

\subsection{Name equivalence}

We take name equivalence, written $\nameeq$, to be the smallest
equivalence relation generated by the following rules.

\begin{mathpar}
\inferrule*[lab=Quote-drop]
{ }
{ \quotep{@{x}} \nameeq x }

\inferrule*[lab=Struct-equiv]
{ P \scong Q }
{ \quotep{P} \nameeq \quotep{Q} }
\end{mathpar}

The astute reader will have noticed that the mutual recursion of names
and processes imposes a mutual recursion on alpha-equivalence and
structural equivalence via name-equivalence. Fortunately, all of this
works out pleasantly and we may calculate in the natural way, free of
concern. The reader interested in the details is referred to the
appendix \ref{appendix:rho_details}.

\subsection{Substitution}

We use $\Proc$ for the set of processes, $\QProc$ for the set of
names, and $\id{\{}\vec{y} / \vec{x} \id{\}}$ to denote partial maps,
$s : \QProc \rightarrow \QProc$. A map, $s$ lifts, uniquely, to a map
on process terms, $\widehat{s} : \Proc \rightarrow \Proc$ by the
following equations.

\begin{mathpar}
  (0) \psubstp{Q}{P} := 0 \\
  (R \juxtap S) \psubstp{Q}{P}
  :=    
  (R)\psubstp{Q}{P} \juxtap (S) \psubstp{Q}{P} \\
  (x?(y).R) \psubstp{Q}{P}    
  :=    
  (x)\substp{Q}{P} (z)\concat( (R \psubstn{z}{y}) \psubstp{Q}{P} ) \\
  (\lift{x}{R}) \psubstp{Q}{P}  
  :=
  \lift{(x)\substp{Q}{P}}{ R \psubstp{Q}{P} } \\
%   (\dropn{x})  \psubstp{Q}{P}       
%   := 
%   \left\{ 
%     \begin{array}{ccc} 
%       \dropn{\quotep{Q}} & & x \nameeq \quotep{P} \\
%       \dropn{x} & & otherwise \\
%     \end{array}
%   \right. 
  (\dropn{x})  \psubstp{Q}{P}       
  := 
  \left\{ 
    \begin{array}{ccc} 
      Q & & x \nameeq \quotep{P} \\
      \dropn{x} & & otherwise \\
    \end{array}
  \right.
\end{mathpar}
 

where

\begin{eqnarray}
  (x)\id{\{} \lpquote Q \rpquote / \lpquote P \rpquote \id{\}}            = 
  \left\{ 
    \begin{array}{ccc}
      \lpquote Q \rpquote & & x \nameeq \lpquote P \rpquote \\
      x & & otherwise \\
    \end{array}
  \right. \nonumber
\end{eqnarray}

and $z$ is chosen distinct from $\quotep{P}$, $\quotep{Q}$, the free
names in $Q$, and all the names in $R$. Our $\alpha$-equivalence will
be built in the standard way from this substitution.

\begin{remark}\label{rem:no_self_referential_names}
  One consequence of these definitions is that $\forall P. \quotep{P}
  \not\in \freenames{P}$.
\end{remark}

\subsection{ Dynamic quote: an example }

Anticipating something of what's to come, consider applying the
substitution, $\widehat{\id{\{}u / z \id{\}}}$, to the following pair
of processes, $\lift{w}{y!(z)}$ and $w[ \lpquote y!(z) \rpquote ]$.

\begin{eqnarray}
	\lift{w}{y!(z)}\widehat{\id{\{}u / z \id{\}}}
		& = &
		\lift{w}{y!(u)} \nonumber\\
	w[ \lpquote y!(z) \rpquote ] \widehat{ \id{\{}u / z \id{\}} }
		& = &
		w[ \lpquote y!(z) \rpquote ] \nonumber
\end{eqnarray}

Because the body of the process between quotes is impervious to
substitution, we get radically different answers. In fact, by
examining the first process in an input context,
e.g. $x?(z).\lift{w}{y!(z)}$, we see that the process under the lift
operator may be shaped by prefixed inputs binding a name inside it. In
this sense, the lift operator will be seen as a way to dynamically
construct processes before reifying them as names.

Finally equipped with these standard features we can present the
dynamics of the calculus.

\subsubsection{Operational semantics} 

Finally, we introduce the computational dynamics. What marks these
algebras as distinct from other more traditionally studied algebraic
structures, e.g. vector spaces or polynomial rings, is the manner in
which dynamics is captured. In traditional structures, dynamics is typically
expressed through morphisms between such structures, as in linear maps
between vector spaces or morphisms between rings. In algebras
associated with the semantics of computation, the dynamics is
expressed as part of the algebraic structure itself, through a
reduction reduction relation typically denoted by $\red$. Below, we
give a recursive presentation of this relation for the calculus used
in the encoding.

$\red \subseteq \pi \times \pi$
$\red : \pi \to \mathcal{P}(\pi)$

\begin{mathpar}
  \inferrule* [lab=Comm] { \textsf{match}( x_{src}, x_{trgt} ) } { x_{trgt}?(y)P \; | \; x_{src}!\langle {Q} \rangle \red P\{\quotep{Q}/y}\} }
  \and \\
  \inferrule* [lab=Par] {{P} \red {P}'} {{{P} | {Q}} \red {{P}' | {Q}}}
  \and
  \inferrule* [lab=Equiv]{{{P} \scong {P}'} \andalso {{P}' \red {Q}'} \andalso {{Q}' \scong {Q}}}{{P} \red {Q}}
\end{mathpar}

\begin{eqnarray*}
  match_{\equiv} (\quotep{P},\quotep{Q}) & := & P \equiv Q \\
  match_{\dagger}(\quotep{P},\quotep{Q}) & := & \forall R. P|Q \red^{*} R => R \red^{*} 0 \\
  match_{K}(\quotep{P},\quotep{Q}) & := & K \mbox{ for some context } K
\end{eqnarray*}

$u?(x)P | u!\langle Q \rangle \red P\{\quotep{Q}/x\}$

%We write $\wred$ for $\red^*$, and $P\red$ if $\exists Q $ such that $ P \red Q$.
We write $P\red$ if $\exists Q $ such that $ P \red Q$ and $P\not\red$, otherwise.

\section{Replication}

As mentioned before, it is known that replication (and hence
recursion) can be implemented in a higher-order process algebra
\cite{SangiorgiWalker}. As our first example of calculation with the
machinery thus far presented we give the construction explicitly in
the {\rhoc}.

\begin{eqnarray}
	D_{x} & := & \prefix{x}{y}{(\binpar{\outputp{x}{y}}{@{y}})} \nonumber\\
	\bangp_{x}{P} & := & \binpar{{x}!\langle{\binpar{D_{x}}{P}}\rangle}{D_{x}} \nonumber
\end{eqnarray}

\begin{eqnarray}
	\bangp_{x}{P} & & \nonumber\\
	=
	& {x}!\langle{(\prefix{x}{y}{(\outputp{x}{y} | @{y})) | P}}\rangle 
	      | \prefix{x}{y}{(\outputp{x}{y} | @{y})} & \nonumber\\
	\red
	& (\outputp{x}{y} | @{y})\substn{\quotep{(\prefix{x}{y}{(@{y} | \outputp{x}{y})) | P}}}{y} & \nonumber\\
	=
	& \outputp{x}{\quotep{(\prefix{x}{y}{(\outputp{x}{y} | @{y})) | P}}}
	  | {(\prefix{x}{y}{(\outputp{x}{y} | @{y})) | P}} & \nonumber\\
	\red
	& \ldots & \nonumber\\
	\red^*
	& P | P | \ldots & \nonumber
\end{eqnarray}

Of course, this encoding, as an implementation, runs away, unfolding
$\bangp{P}$ eagerly. A lazier and more implementable replication
operator, restricted to input-guarded processes, may be obtained as follows.

\begin{eqnarray}
\bangp{\prefix{u}{v}{P}} 
	:= 
	\binpar{\lift{x}{\prefix{u}{v}{(\binpar{D(x)}{P})}}}{D(x)} \nonumber
\end{eqnarray}

\begin{remark}
  Note that the lazier definition still does not deal with summation
  or mixed summation (i.e. sums over input and output). The reader is
  invited to construct definitions of replication that deal with these
  features. 

  Further, the definitions are parameterized in a name, $x$. Can you,
  gentle reader, make a definition that eliminates this parameter and
  guarantees no accidental interaction between the replication
  machinery and the process being replicated -- i.e. no accidental
  sharing of names used by the process to get its work done and the
  name(s) used by the replication to effect copying. This latter
  revision of the definition of replication is crucial to obtaining
  the expected identity $!!P \sim !P$.
\end{remark}

\begin{remark}\label{rem:paradoxical_combinator}
  The reader familiar with the lambda calculus will have noticed the
  similarity between $D$ and the paradoxical combinator.

  [Ed. note: the existence of this seems to suggest we have to be more
  restrictive on the set of processes and names we admit if we are to
  support no-cloning.]
\end{remark}

\subsubsection{Bisimulation}

The computational dynamics gives rise to another kind of equivalence,
the equivalence of computational behavior. As previously mentioned
this is typically captured \emph{via} some form of bisimulation.

% The notion we use in this paper is weak barbed bisimulation
% \cite{milner91polyadicpi}.

The notion we use in this paper is derived from weak barbed
bisimulation \cite{milner91polyadicpi}. 

\begin{definition}
An \emph{observation relation}, $\downarrow_{\mathcal N}$, over a set
of names, $\mathcal N$, is the smallest relation satisfying the rules
below.

\infrule[Out-barb]{y \in {\mathcal N}, \; x \nameeq y}
		  {\outputp{x}{v} \downarrow_{\mathcal N} x}
\infrule[Par-barb]{\mbox{$P\downarrow_{\mathcal N} x$ or $Q\downarrow_{\mathcal N} x$}}
		  {\binpar{P}{Q} \downarrow_{\mathcal N} x}

We write $P \Downarrow_{\mathcal N} x$ if there is $Q$ such that 
$P \wred Q$ and $Q \downarrow_{\mathcal N} x$.
\end{definition}

\begin{definition}
%\label{def.bbisim}
An  ${\mathcal N}$-\emph{barbed bisimulation} over a set of names, ${\mathcal N}$, is a symmetric binary relation 
${\mathcal S}_{\mathcal N}$ between agents such that $P\rel{S}_{\mathcal N}Q$ implies:
\begin{enumerate}
\item If $P \red P'$ then $Q \wred Q'$ and $P'\rel{S}_{\mathcal N} Q'$.
\item If $P\downarrow_{\mathcal N} x$, then $Q\Downarrow_{\mathcal N} x$.
\end{enumerate}
$P$ is ${\mathcal N}$-barbed bisimilar to $Q$, written
$P \wbbisim_{\mathcal N} Q$, if $P \rel{S}_{\mathcal N} Q$ for some ${\mathcal N}$-barbed bisimulation ${\mathcal S}_{\mathcal N}$.
\end{definition}

$\mathcal{R} \subseteq \pi \times \pi$

$P \mathcal{R} Q => \forall P'. P \red P' \Rightarrow \exists Q'. Q \red Q', P' \mathcal{R} Q'$

$P \vdash x \Rightarrow Q \vdash x$

\begin{mathpar}
  \inferrule*[lab=Out-barb]{x \nameeq y}{{y}!\langle{Q}\rangle \vdash x}
  \and
  \inferrule*[lab=Par-barb]{\mbox{$P\vdash x$ or $Q\vdash x$}}{\binpar{P}{Q} \vdash x}
\end{mathpar}

\subsubsection{Contexts}

One of the principle advantages of computational calculi like the
$\pi$-calculus is a well-defined notion of context,
contextual-equivalence and a correlation between
contextual-equivalence and notions of bisimulation. The notion of
context allows the decomposition of a process into (sub-)process and
its syntactic environment, its context. Thus, a context may be
thought of as a process with a ``hole'' (written $\Box$) in it. The
application of a context $M$ to a process $P$, written $M[P]$, is
tantamount to filling the hole in $M$ with $P$. In this paper we do
not need the full weight of this theory, but do make use of the notion
of context in the proof the main theorem. 

\begin{mathpar}
  \inferrule* [lab=summation] {} {{M_{M},M_{N}} \bc \Box \;|\; x.M_{A} \;|\; M_{M}+M_{N}}
  \and
  \inferrule* [lab=agent] {} {{M_{A}} \bc (\vec{x})M_{P} \;| \; \clift{P_0,\ldots,M_{P},\ldots,P_N}}
  \and \\
  \inferrule* [lab=process] {} {{M_{P}} \bc M_{N} \;| \;P|M_{P} }
\end{mathpar} 

\begin{mathpar}
  \inferrule* [lab=sychronization] {} {M_{N} \bc \Box \;|\; x?M_{F} \;|\; x!M_{C}}
  \and
  \inferrule* [lab=abstraction] {} {{M_{F}} \bc (x)M_{P} }
  \and
  \inferrule* [lab=concretion] {} {{M_{C}} \bc \langle M_{P} \rangle }
  \and \\
  \inferrule* [lab=process] {} {{M_{P}} \bc M_{N} \;| \;P|M_{P} }
\end{mathpar}

\begin{definition}[contextual application] Given a context $M$, and
  process $P$, we define the \emph{contextual application}, $M[P] :=
  M\{P/\Box\}$. That is, the contextual application of M to P is the
  substitution of $P$ for $\Box$ in $M$.
\end{definition}

$\meaningof{-} : L \to \mathcal{P}(\pi)$

\begin{mathpar}
  \inferrule* [lab=collection] {} {\meaningof{true} = \pi, \and \meaningof{~E} = \pi \setminus \meaningof{E}, \and \meaningof{E_{1} \& E_{2}} = \meaningof{E_{1}} \cap \meaningof{E_{2}}}
\end{mathpar}

\begin{mathpar}
  \inferrule* [lab=structure] {} {\meaningof{0} = \{ P \in \pi | P \equiv 0 \}, \and \\ \meaningof{E_1 | E_2} = \{ P \in \pi | P \equiv P_{1} | P_{2}, P_{1} \in \meaningof{E_{1}}, P_{2} \in \meaningof{E_2}\} }
\end{mathpar}

\begin{mathpar}
 \inferrule* [lab=behavior] {} {\meaningof{\langle a?b \rangle E} = \{ P \in \pi | P \equiv Q | u?(y)P', \\ \and \\\\ \and \\ \;\;\; u \in \meaningof{a}, \forall z.P'\{z/y\} \in \meaningof{E\{z/b\}}\}, \and \\ \meaningof{a!E} = \{ P \in \pi | P \equiv Q | x!\langle P' \rangle, x \in \meaningof{a} P' \in \meaningof{E}\} }
\end{mathpar}

\begin{mathpar}
 \inferrule* [lab=nominal] {} {\meaningof{\quotep{E}} = \{ \quotep{P} \in \quotep{\pi} | P \in \meaningof{E} \}, \and \meaningof{\quotep{P}} = \{ \quotep{Q} \in \quotep{\pi} | P \equiv Q \} \and \\ \meaningof{@\quotep{E}} = \{ P \in \pi | P \equiv @x, x \in \meaningof{E} \}}
\end{mathpar}

\begin{eqnarray*}
  \\
  \meaningof{-} : TS \to ST
\end{eqnarray*}

\begin{eqnarray*}
  \\
  L : TS \to ST
\end{eqnarray*}

\begin{eqnarray*}
  \\
  P \models E \iff P \in \meaningof{E}
\end{eqnarray*}

\begin{eqnarray*}
  P \approx_{L} Q \iff \forall E \in L. P \models E \iff Q \models E
\end{eqnarray*}

\begin{eqnarray*}
  P \approx_{K} Q
\end{eqnarray*}

\begin{eqnarray*}
  P \approx Q
\end{eqnarray*}

$\approx_{K} = \approx = \approx_{L}$

\subsubsection{Contextual duality}

Note that contexts extend the quotation operation to a family of
operations from processes to names. Given a context, $M$, we can
define a \emph{nominal context}, $\quotep{M}$ by $\quotep{M}[P] :=
\quotep{M[P]}$. To foreshadow what is to come we observe that these
operations enjoy a duality with processes very much like the duality
between vectors and maps from vectors to scalars.

Further, because the calculus is essentially higher-order, we have a
correspondence between contexts and processes. More specifically,
given a name $x$ and a context $M$ we can construct $M^{*}_{x}$ such
that 

\begin{mathpar}
  M^{*}_{x} | \lift{x}{P} \red M[P]
\end{mathpar}

namely,

\begin{mathpar}
  M^{*}_{x} := x?(u).M[\dropn{u}]
\end{mathpar}

The dependence of $M^{*}_{x}$ on a name makes it an abstraction, 

\begin{mathpar}
  M^{*} := (x)x?(u).M[\dropn{u}]
\end{mathpar}

\subsection{Additional notation}

It will sometimes be convenient to denote the process a name
quotes. We already have the notation $x = \quotep{P}$, but it will be
convenient to introduce an alternate notation, $\procn{x}$, when we
want to emphasize the connection to the use of the name. Note that, by
virtue of name equivalence, $\quotep{\procn{x}} \nameeq x$; so, the
notation is consistent with previous definitions.

Further, because names have structure it is possible to effect
substitutions on the basis of that structure. This means we need to
upgrade our notation for substitutions, which we accomplish by
adapting comprehension notation. Thus,

\begin{mathpar}
  P\{ y / x : x \in S \}
\end{mathpar}

is interpreted to mean the process derived from P by replacing (in a
capture-avoiding manner) each occurrence of $x$ in $S$ by $y$. For example,

\begin{mathpar}
  P\{ \quotep{\procn{x}|\procn{x}} / x : x \in \freenames{P} \}
\end{mathpar}

will replace each (occurrence) of a free name $x$ in $P$ by
$\quotep{\procn{x}|\procn{x}}$.

Also, we will avail ourselves of the notation $x^{L}$ and $x^{R}$ to
denote injections of a name into disjoint copies of the name
space. There are numerous ways to accomplish this. One example can be
found in \cite{MeredithR05}. This notation overloads to vectors of
names: $\vec{x}^{\pi} := (x_{i}^{\pi} \; : \; 0 \leq i < |\vec{x}| )$ where $\pi \in \{L,R\}$.

We also use $P^{\Box} := P|\Box$.

In \cite{MeredithR05} an interpretation of the new operator is
given. It turns out that there are several possible interpretations
all enjoying the requisite algebraic properties of the operator (see
\cite{milner91polyadicpi}). We will therefore make liberal use of
$(\nu\; \vec{x})P$.

% subsection the_syntax_and_semantics_of_the_notation_system (end)   

\input{qm2pi.qmops} 

\input{qm2pi.sterngerlach} 

\input{qm2pi.metric} 

% section concurrent_process_calculi (end)

%\input{qm2pi.proofsketch}

% section proof sketch (end)

%\input{qm2pi.slviaknots} 

% section spatial logic via knots (end)

\input{qm2pi.conclusion}

% section conclusion (end)

%\input{qm2pi.dtcodes} 

% section wiring algorithm (end)

\input{qm2pi.ack} 

% section acknowledgments (end)

\newpage


\bibliographystyle{plain}   
\bibliography{../../biblios/main.bib}

\input{qm2pi.rhodetails}

\end{document}

 

%\ifpdf
%\usepackage[pdftex]{graphicx}
%\else
%\usepackage{graphicx}
%\fi

 % \ifpdf
%  \usepackage{pdfsync}
%  \if


%\title{Brief Article}
%\author{David F. Snyder}
%\author{L.G. Meredith}

%\address{Dept. of Math., Texas State University--San Marcos, San Marcos, TX 78666}
       
\pagestyle{empty}


\begin{document}

\lstset{language=[Objective]Caml,frame=shadowbox}

\documentclass[12pt]{llncs}
%\documentclass{jktr}

\usepackage[pdftex]{hyperref}                   
\usepackage {listings}
\usepackage {mathpartir}
\usepackage{bcprules}
%\usepackage{listings}
                       
\usepackage{graphicx} 
%\usepackage[margins=2.5cm,nohead,nofoot]{geometry}
%\usepackage{geometry}
\usepackage{amsfonts}
\usepackage{amstext}
\usepackage{latexsym}
\usepackage{amssymb}
\usepackage{color}


%\include{myPreamble}
\include{qm2pi.local} 

%\ifpdf
%\usepackage[pdftex]{graphicx}
%\else
%\usepackage{graphicx}
%\fi

 % \ifpdf
%  \usepackage{pdfsync}
%  \if


%\title{Brief Article}
%\author{David F. Snyder}
%\author{L.G. Meredith}

%\address{Dept. of Math., Texas State University--San Marcos, San Marcos, TX 78666}
       
\pagestyle{empty}


\begin{document}

\lstset{language=[Objective]Caml,frame=shadowbox}

\input{qm2pi.front}

% section front matter (end)

\input{qm2pi.intro} 
 
% section introduction (end)

% \input{qm2pi.knotations} 

% section notation (end)

\input{qm2pi.process.calculi} 

% section concurrent_process_calculi_and_spatial_logics_ (end)
    
%\input{qm2pi.knots2pi} 

%\input{qm2pi.trefoil} 

%\input{qm2pi.mainthm} 

% subsection basic_interpretation (end)

%\input{qm2pi.rho.presentation} 
\subsection{The syntax and semantics of the notation system}\label{sub:the_syntax_and_semantics_of_the_notation_system} % (fold)

We now summarize a technical presentation of the calculus that
embodies our theory of dynamics. The typical presentation of such a
calculus follows the style of giving generators and relations on
them. The grammar, below, describing term constructors, freely
generates the set of processes, $\Proc$. This set is then quotiented
by a relation known as structural congruence and it is over this set
that the notion of dynamics is expressed. This presentation is
essentially that of \cite{MeredithR05} with the addition of
polyadicity and summation. For readability we have relegated some of
the technical subtleties to an appendix.

\subsubsection{Process grammar}\label{subsub:process_grammar}

\begin{mathpar}
  \inferrule* [lab=synchronization] {} {{M} \bc \pzero \;|\; x?F \;|\; x!C }
  \and
  \inferrule* [lab=abstraction] {} {{F} \bc (x)P}
  \and
  \inferrule* [lab=concretion] {} {{C} \bc \langle Q \rangle}
  \and
  \inferrule* [lab=process] {} {{P,Q} \bc M \;| \;P|Q \;|\; @{x}}
  \and
  \inferrule* [lab=name] {} {{x} \bc \quotep{P}}
\end{mathpar} 

Note that $\vec{x}$ (resp. $\vec{P}$) denotes a vector of names
(resp. processes) of length $|\vec{x}|$ (resp. $|\vec{P}|$). We adopt
the following useful abbreviations.

\begin{mathpar}
   x?(\vec{y}).P := x.(\vec{y})P \and  x\clift{\vec{P}} := x.\clift{\vec{P}}
   \and x!(y) := \lift{x}{\dropn{y}}
   \and \Pi_{i=0}^{n-1}P_i := P_0 | \ldots | P_{n-1}
\end{mathpar}

\subsubsection{Structural congruence}

\paragraph{Free and bound names and alpha-equivalence.} At the
core of structural equivalence is alpha-equivalence which identifies
process that are the same up to a change of variable. Formally, we
recognize the distinction between free and bound names. The free names
of a process, $\freenames{P}$, may be calculated recursively as
follows:

\begin{mathpar}
\freenames{\pzero} := \emptyset
  \and \\
  \freenames{x?(y).P} := \{ x \} \cup (\freenames{P} \setminus \{ y \})
  \and 
  \freenames{x!\langle P \rangle} := \{ x \} \cup \{ P \} 
  \and \\
  \freenames{P|Q} := \freenames{P} \cup \freenames{Q}
  \and \\
  \freenames{@{x}} := \{ x \}
\end{mathpar}

$\pi$
$\quotep{\pi}$

$\freenames{-} : \pi \to \mathcal{P}(\quotep{\pi})$

\begin{eqnarray*}
  \freenames{\pzero} & := & \emptyset \\
  \freenames{x?(y).P} & := & \{ x \} \cup (\freenames{P} \setminus \{ y \}) \\
  \freenames{x!\langle P \rangle} & := & \{ x \} \cup \{ P \} \\
  \freenames{P|Q} & := & \freenames{P} \cup \freenames{Q} \\
  \freenames{\dropn{x}} & := & \{ x \}
\end{eqnarray*}

The bound names of a process, $\boundnames{P}$, are those names occurring in $P$
that are not free. For example, in $x?(y).0$, the name $x$ is free, while $y$ is bound.

\begin{mathpar}
  \inferrule* [lab=monoidal-laws] {} { P|Q \equiv Q|P \and P|0 \equiv P \and P|(Q|R) \equiv (P|Q)|R }
\end{mathpar}

\begin{mathpar}
  \inferrule* [lab=alpha-equivalence] {} { (x)P \equiv (y)P\{y/x\} \and y \not\in \freenames{P} }
\end{mathpar}

\begin{definition}
Then two processes, $P,Q$, are alpha-equivalent if $P = Q\{\vec{y}/\vec{x}\}$ for
some $\vec{x} \in \boundnames{Q},\vec{y} \in \boundnames{P}$, where $Q\{\vec{y}/\vec{x}\}$
denotes the capture-avoiding substitution of $\vec{y}$ for $\vec{x}$ in $Q$.
\end{definition}

\begin{definition}
  The {\em structural congruence} \cite{SangiorgiWalker} , $\equiv$,
  between processes is the least congruence containing
  alpha-equivalence, satisfying the abelian monoid laws
  (associativity, commutativity and $\pzero$ as identity) for parallel
  composition $|$ and for summation $+$.
\end{definition}

\subsection{Name equivalence}

We take name equivalence, written $\nameeq$, to be the smallest
equivalence relation generated by the following rules.

\begin{mathpar}
\inferrule*[lab=Quote-drop]
{ }
{ \quotep{@{x}} \nameeq x }

\inferrule*[lab=Struct-equiv]
{ P \scong Q }
{ \quotep{P} \nameeq \quotep{Q} }
\end{mathpar}

The astute reader will have noticed that the mutual recursion of names
and processes imposes a mutual recursion on alpha-equivalence and
structural equivalence via name-equivalence. Fortunately, all of this
works out pleasantly and we may calculate in the natural way, free of
concern. The reader interested in the details is referred to the
appendix \ref{appendix:rho_details}.

\subsection{Substitution}

We use $\Proc$ for the set of processes, $\QProc$ for the set of
names, and $\id{\{}\vec{y} / \vec{x} \id{\}}$ to denote partial maps,
$s : \QProc \rightarrow \QProc$. A map, $s$ lifts, uniquely, to a map
on process terms, $\widehat{s} : \Proc \rightarrow \Proc$ by the
following equations.

\begin{mathpar}
  (0) \psubstp{Q}{P} := 0 \\
  (R \juxtap S) \psubstp{Q}{P}
  :=    
  (R)\psubstp{Q}{P} \juxtap (S) \psubstp{Q}{P} \\
  (x?(y).R) \psubstp{Q}{P}    
  :=    
  (x)\substp{Q}{P} (z)\concat( (R \psubstn{z}{y}) \psubstp{Q}{P} ) \\
  (\lift{x}{R}) \psubstp{Q}{P}  
  :=
  \lift{(x)\substp{Q}{P}}{ R \psubstp{Q}{P} } \\
%   (\dropn{x})  \psubstp{Q}{P}       
%   := 
%   \left\{ 
%     \begin{array}{ccc} 
%       \dropn{\quotep{Q}} & & x \nameeq \quotep{P} \\
%       \dropn{x} & & otherwise \\
%     \end{array}
%   \right. 
  (\dropn{x})  \psubstp{Q}{P}       
  := 
  \left\{ 
    \begin{array}{ccc} 
      Q & & x \nameeq \quotep{P} \\
      \dropn{x} & & otherwise \\
    \end{array}
  \right.
\end{mathpar}
 

where

\begin{eqnarray}
  (x)\id{\{} \lpquote Q \rpquote / \lpquote P \rpquote \id{\}}            = 
  \left\{ 
    \begin{array}{ccc}
      \lpquote Q \rpquote & & x \nameeq \lpquote P \rpquote \\
      x & & otherwise \\
    \end{array}
  \right. \nonumber
\end{eqnarray}

and $z$ is chosen distinct from $\quotep{P}$, $\quotep{Q}$, the free
names in $Q$, and all the names in $R$. Our $\alpha$-equivalence will
be built in the standard way from this substitution.

\begin{remark}\label{rem:no_self_referential_names}
  One consequence of these definitions is that $\forall P. \quotep{P}
  \not\in \freenames{P}$.
\end{remark}

\subsection{ Dynamic quote: an example }

Anticipating something of what's to come, consider applying the
substitution, $\widehat{\id{\{}u / z \id{\}}}$, to the following pair
of processes, $\lift{w}{y!(z)}$ and $w[ \lpquote y!(z) \rpquote ]$.

\begin{eqnarray}
	\lift{w}{y!(z)}\widehat{\id{\{}u / z \id{\}}}
		& = &
		\lift{w}{y!(u)} \nonumber\\
	w[ \lpquote y!(z) \rpquote ] \widehat{ \id{\{}u / z \id{\}} }
		& = &
		w[ \lpquote y!(z) \rpquote ] \nonumber
\end{eqnarray}

Because the body of the process between quotes is impervious to
substitution, we get radically different answers. In fact, by
examining the first process in an input context,
e.g. $x?(z).\lift{w}{y!(z)}$, we see that the process under the lift
operator may be shaped by prefixed inputs binding a name inside it. In
this sense, the lift operator will be seen as a way to dynamically
construct processes before reifying them as names.

Finally equipped with these standard features we can present the
dynamics of the calculus.

\subsubsection{Operational semantics} 

Finally, we introduce the computational dynamics. What marks these
algebras as distinct from other more traditionally studied algebraic
structures, e.g. vector spaces or polynomial rings, is the manner in
which dynamics is captured. In traditional structures, dynamics is typically
expressed through morphisms between such structures, as in linear maps
between vector spaces or morphisms between rings. In algebras
associated with the semantics of computation, the dynamics is
expressed as part of the algebraic structure itself, through a
reduction reduction relation typically denoted by $\red$. Below, we
give a recursive presentation of this relation for the calculus used
in the encoding.

$\red \subseteq \pi \times \pi$
$\red : \pi \to \mathcal{P}(\pi)$

\begin{mathpar}
  \inferrule* [lab=Comm] { \textsf{match}( x_{src}, x_{trgt} ) } { x_{trgt}?(y)P \; | \; x_{src}!\langle {Q} \rangle \red P\{\quotep{Q}/y}\} }
  \and \\
  \inferrule* [lab=Par] {{P} \red {P}'} {{{P} | {Q}} \red {{P}' | {Q}}}
  \and
  \inferrule* [lab=Equiv]{{{P} \scong {P}'} \andalso {{P}' \red {Q}'} \andalso {{Q}' \scong {Q}}}{{P} \red {Q}}
\end{mathpar}

\begin{eqnarray*}
  match_{\equiv} (\quotep{P},\quotep{Q}) & := & P \equiv Q \\
  match_{\dagger}(\quotep{P},\quotep{Q}) & := & \forall R. P|Q \red^{*} R => R \red^{*} 0 \\
  match_{K}(\quotep{P},\quotep{Q}) & := & K \mbox{ for some context } K
\end{eqnarray*}

$u?(x)P | u!\langle Q \rangle \red P\{\quotep{Q}/x\}$

%We write $\wred$ for $\red^*$, and $P\red$ if $\exists Q $ such that $ P \red Q$.
We write $P\red$ if $\exists Q $ such that $ P \red Q$ and $P\not\red$, otherwise.

\section{Replication}

As mentioned before, it is known that replication (and hence
recursion) can be implemented in a higher-order process algebra
\cite{SangiorgiWalker}. As our first example of calculation with the
machinery thus far presented we give the construction explicitly in
the {\rhoc}.

\begin{eqnarray}
	D_{x} & := & \prefix{x}{y}{(\binpar{\outputp{x}{y}}{@{y}})} \nonumber\\
	\bangp_{x}{P} & := & \binpar{{x}!\langle{\binpar{D_{x}}{P}}\rangle}{D_{x}} \nonumber
\end{eqnarray}

\begin{eqnarray}
	\bangp_{x}{P} & & \nonumber\\
	=
	& {x}!\langle{(\prefix{x}{y}{(\outputp{x}{y} | @{y})) | P}}\rangle 
	      | \prefix{x}{y}{(\outputp{x}{y} | @{y})} & \nonumber\\
	\red
	& (\outputp{x}{y} | @{y})\substn{\quotep{(\prefix{x}{y}{(@{y} | \outputp{x}{y})) | P}}}{y} & \nonumber\\
	=
	& \outputp{x}{\quotep{(\prefix{x}{y}{(\outputp{x}{y} | @{y})) | P}}}
	  | {(\prefix{x}{y}{(\outputp{x}{y} | @{y})) | P}} & \nonumber\\
	\red
	& \ldots & \nonumber\\
	\red^*
	& P | P | \ldots & \nonumber
\end{eqnarray}

Of course, this encoding, as an implementation, runs away, unfolding
$\bangp{P}$ eagerly. A lazier and more implementable replication
operator, restricted to input-guarded processes, may be obtained as follows.

\begin{eqnarray}
\bangp{\prefix{u}{v}{P}} 
	:= 
	\binpar{\lift{x}{\prefix{u}{v}{(\binpar{D(x)}{P})}}}{D(x)} \nonumber
\end{eqnarray}

\begin{remark}
  Note that the lazier definition still does not deal with summation
  or mixed summation (i.e. sums over input and output). The reader is
  invited to construct definitions of replication that deal with these
  features. 

  Further, the definitions are parameterized in a name, $x$. Can you,
  gentle reader, make a definition that eliminates this parameter and
  guarantees no accidental interaction between the replication
  machinery and the process being replicated -- i.e. no accidental
  sharing of names used by the process to get its work done and the
  name(s) used by the replication to effect copying. This latter
  revision of the definition of replication is crucial to obtaining
  the expected identity $!!P \sim !P$.
\end{remark}

\begin{remark}\label{rem:paradoxical_combinator}
  The reader familiar with the lambda calculus will have noticed the
  similarity between $D$ and the paradoxical combinator.

  [Ed. note: the existence of this seems to suggest we have to be more
  restrictive on the set of processes and names we admit if we are to
  support no-cloning.]
\end{remark}

\subsubsection{Bisimulation}

The computational dynamics gives rise to another kind of equivalence,
the equivalence of computational behavior. As previously mentioned
this is typically captured \emph{via} some form of bisimulation.

% The notion we use in this paper is weak barbed bisimulation
% \cite{milner91polyadicpi}.

The notion we use in this paper is derived from weak barbed
bisimulation \cite{milner91polyadicpi}. 

\begin{definition}
An \emph{observation relation}, $\downarrow_{\mathcal N}$, over a set
of names, $\mathcal N$, is the smallest relation satisfying the rules
below.

\infrule[Out-barb]{y \in {\mathcal N}, \; x \nameeq y}
		  {\outputp{x}{v} \downarrow_{\mathcal N} x}
\infrule[Par-barb]{\mbox{$P\downarrow_{\mathcal N} x$ or $Q\downarrow_{\mathcal N} x$}}
		  {\binpar{P}{Q} \downarrow_{\mathcal N} x}

We write $P \Downarrow_{\mathcal N} x$ if there is $Q$ such that 
$P \wred Q$ and $Q \downarrow_{\mathcal N} x$.
\end{definition}

\begin{definition}
%\label{def.bbisim}
An  ${\mathcal N}$-\emph{barbed bisimulation} over a set of names, ${\mathcal N}$, is a symmetric binary relation 
${\mathcal S}_{\mathcal N}$ between agents such that $P\rel{S}_{\mathcal N}Q$ implies:
\begin{enumerate}
\item If $P \red P'$ then $Q \wred Q'$ and $P'\rel{S}_{\mathcal N} Q'$.
\item If $P\downarrow_{\mathcal N} x$, then $Q\Downarrow_{\mathcal N} x$.
\end{enumerate}
$P$ is ${\mathcal N}$-barbed bisimilar to $Q$, written
$P \wbbisim_{\mathcal N} Q$, if $P \rel{S}_{\mathcal N} Q$ for some ${\mathcal N}$-barbed bisimulation ${\mathcal S}_{\mathcal N}$.
\end{definition}

$\mathcal{R} \subseteq \pi \times \pi$

$P \mathcal{R} Q => \forall P'. P \red P' \Rightarrow \exists Q'. Q \red Q', P' \mathcal{R} Q'$

$P \vdash x \Rightarrow Q \vdash x$

\begin{mathpar}
  \inferrule*[lab=Out-barb]{x \nameeq y}{{y}!\langle{Q}\rangle \vdash x}
  \and
  \inferrule*[lab=Par-barb]{\mbox{$P\vdash x$ or $Q\vdash x$}}{\binpar{P}{Q} \vdash x}
\end{mathpar}

\subsubsection{Contexts}

One of the principle advantages of computational calculi like the
$\pi$-calculus is a well-defined notion of context,
contextual-equivalence and a correlation between
contextual-equivalence and notions of bisimulation. The notion of
context allows the decomposition of a process into (sub-)process and
its syntactic environment, its context. Thus, a context may be
thought of as a process with a ``hole'' (written $\Box$) in it. The
application of a context $M$ to a process $P$, written $M[P]$, is
tantamount to filling the hole in $M$ with $P$. In this paper we do
not need the full weight of this theory, but do make use of the notion
of context in the proof the main theorem. 

\begin{mathpar}
  \inferrule* [lab=summation] {} {{M_{M},M_{N}} \bc \Box \;|\; x.M_{A} \;|\; M_{M}+M_{N}}
  \and
  \inferrule* [lab=agent] {} {{M_{A}} \bc (\vec{x})M_{P} \;| \; \clift{P_0,\ldots,M_{P},\ldots,P_N}}
  \and \\
  \inferrule* [lab=process] {} {{M_{P}} \bc M_{N} \;| \;P|M_{P} }
\end{mathpar} 

\begin{mathpar}
  \inferrule* [lab=sychronization] {} {M_{N} \bc \Box \;|\; x?M_{F} \;|\; x!M_{C}}
  \and
  \inferrule* [lab=abstraction] {} {{M_{F}} \bc (x)M_{P} }
  \and
  \inferrule* [lab=concretion] {} {{M_{C}} \bc \langle M_{P} \rangle }
  \and \\
  \inferrule* [lab=process] {} {{M_{P}} \bc M_{N} \;| \;P|M_{P} }
\end{mathpar}

\begin{definition}[contextual application] Given a context $M$, and
  process $P$, we define the \emph{contextual application}, $M[P] :=
  M\{P/\Box\}$. That is, the contextual application of M to P is the
  substitution of $P$ for $\Box$ in $M$.
\end{definition}

$\meaningof{-} : L \to \mathcal{P}(\pi)$

\begin{mathpar}
  \inferrule* [lab=collection] {} {\meaningof{true} = \pi, \and \meaningof{~E} = \pi \setminus \meaningof{E}, \and \meaningof{E_{1} \& E_{2}} = \meaningof{E_{1}} \cap \meaningof{E_{2}}}
\end{mathpar}

\begin{mathpar}
  \inferrule* [lab=structure] {} {\meaningof{0} = \{ P \in \pi | P \equiv 0 \}, \and \\ \meaningof{E_1 | E_2} = \{ P \in \pi | P \equiv P_{1} | P_{2}, P_{1} \in \meaningof{E_{1}}, P_{2} \in \meaningof{E_2}\} }
\end{mathpar}

\begin{mathpar}
 \inferrule* [lab=behavior] {} {\meaningof{\langle a?b \rangle E} = \{ P \in \pi | P \equiv Q | u?(y)P', \\ \and \\\\ \and \\ \;\;\; u \in \meaningof{a}, \forall z.P'\{z/y\} \in \meaningof{E\{z/b\}}\}, \and \\ \meaningof{a!E} = \{ P \in \pi | P \equiv Q | x!\langle P' \rangle, x \in \meaningof{a} P' \in \meaningof{E}\} }
\end{mathpar}

\begin{mathpar}
 \inferrule* [lab=nominal] {} {\meaningof{\quotep{E}} = \{ \quotep{P} \in \quotep{\pi} | P \in \meaningof{E} \}, \and \meaningof{\quotep{P}} = \{ \quotep{Q} \in \quotep{\pi} | P \equiv Q \} \and \\ \meaningof{@\quotep{E}} = \{ P \in \pi | P \equiv @x, x \in \meaningof{E} \}}
\end{mathpar}

\begin{eqnarray*}
  \\
  \meaningof{-} : TS \to ST
\end{eqnarray*}

\begin{eqnarray*}
  \\
  L : TS \to ST
\end{eqnarray*}

\begin{eqnarray*}
  \\
  P \models E \iff P \in \meaningof{E}
\end{eqnarray*}

\begin{eqnarray*}
  P \approx_{L} Q \iff \forall E \in L. P \models E \iff Q \models E
\end{eqnarray*}

\begin{eqnarray*}
  P \approx_{K} Q
\end{eqnarray*}

\begin{eqnarray*}
  P \approx Q
\end{eqnarray*}

$\approx_{K} = \approx = \approx_{L}$

\subsubsection{Contextual duality}

Note that contexts extend the quotation operation to a family of
operations from processes to names. Given a context, $M$, we can
define a \emph{nominal context}, $\quotep{M}$ by $\quotep{M}[P] :=
\quotep{M[P]}$. To foreshadow what is to come we observe that these
operations enjoy a duality with processes very much like the duality
between vectors and maps from vectors to scalars.

Further, because the calculus is essentially higher-order, we have a
correspondence between contexts and processes. More specifically,
given a name $x$ and a context $M$ we can construct $M^{*}_{x}$ such
that 

\begin{mathpar}
  M^{*}_{x} | \lift{x}{P} \red M[P]
\end{mathpar}

namely,

\begin{mathpar}
  M^{*}_{x} := x?(u).M[\dropn{u}]
\end{mathpar}

The dependence of $M^{*}_{x}$ on a name makes it an abstraction, 

\begin{mathpar}
  M^{*} := (x)x?(u).M[\dropn{u}]
\end{mathpar}

\subsection{Additional notation}

It will sometimes be convenient to denote the process a name
quotes. We already have the notation $x = \quotep{P}$, but it will be
convenient to introduce an alternate notation, $\procn{x}$, when we
want to emphasize the connection to the use of the name. Note that, by
virtue of name equivalence, $\quotep{\procn{x}} \nameeq x$; so, the
notation is consistent with previous definitions.

Further, because names have structure it is possible to effect
substitutions on the basis of that structure. This means we need to
upgrade our notation for substitutions, which we accomplish by
adapting comprehension notation. Thus,

\begin{mathpar}
  P\{ y / x : x \in S \}
\end{mathpar}

is interpreted to mean the process derived from P by replacing (in a
capture-avoiding manner) each occurrence of $x$ in $S$ by $y$. For example,

\begin{mathpar}
  P\{ \quotep{\procn{x}|\procn{x}} / x : x \in \freenames{P} \}
\end{mathpar}

will replace each (occurrence) of a free name $x$ in $P$ by
$\quotep{\procn{x}|\procn{x}}$.

Also, we will avail ourselves of the notation $x^{L}$ and $x^{R}$ to
denote injections of a name into disjoint copies of the name
space. There are numerous ways to accomplish this. One example can be
found in \cite{MeredithR05}. This notation overloads to vectors of
names: $\vec{x}^{\pi} := (x_{i}^{\pi} \; : \; 0 \leq i < |\vec{x}| )$ where $\pi \in \{L,R\}$.

We also use $P^{\Box} := P|\Box$.

In \cite{MeredithR05} an interpretation of the new operator is
given. It turns out that there are several possible interpretations
all enjoying the requisite algebraic properties of the operator (see
\cite{milner91polyadicpi}). We will therefore make liberal use of
$(\nu\; \vec{x})P$.

% subsection the_syntax_and_semantics_of_the_notation_system (end)   

\input{qm2pi.qmops} 

\input{qm2pi.sterngerlach} 

\input{qm2pi.metric} 

% section concurrent_process_calculi (end)

%\input{qm2pi.proofsketch}

% section proof sketch (end)

%\input{qm2pi.slviaknots} 

% section spatial logic via knots (end)

\input{qm2pi.conclusion}

% section conclusion (end)

%\input{qm2pi.dtcodes} 

% section wiring algorithm (end)

\input{qm2pi.ack} 

% section acknowledgments (end)

\newpage


\bibliographystyle{plain}   
\bibliography{../../biblios/main.bib}

\input{qm2pi.rhodetails}

\end{document}



% section front matter (end)

\section{Introduction}\label{sec:introduction} % (fold)
In this draft of the material i am going to have to dispense with the
usual writing conventions adopted in papers on these topics. i'm going
to have adopt whatever tone i need at the time i'm writing up the
calculations. Sometimes this may be very conversational; others it may
be the barest mathematical grunts; others still it may be that i have
lifted text from one of my other papers because the exposition of some
point was better said there. i hope that my readers are not unduly put
out by this decision. i'm not doing this to flout convention or be
rebellious. i find these calculations very technically challenging. To
keep everything going technically, something has to give; i have to
let go of some cognitive burden. So, the academic writing style --
with all of its trade-offs in terms of facilitating technical
communication -- is what i'm letting go of. Perhaps subsequent drafts
can be tightened and polished, but for now, i'm going to speak as if
we were sitting together in a coffee shop with a laptop, wifi and a
pad of paper and a pencil.

So, here's what i have to say. We -- you and i, comfortably ensconced
in our coffee shop and well-equipped with our tools -- can realize and
carry out the calculations of quantum mechanics over a very different
formal theory of dynamics, a formal theory of dynamics that
corresponds to a theory of concurrent computation with
\emph{reflection}. It has the advantage that the underlying theory is
already `quantized', but supports analogues all of the continuuous
operations. Strikingly, this underlying theory has recently been
connected with a notion of metric that we can show, by calculating
together, coincides with the metric induced by the inner product.

There are a lot of reasons why you might be interested in seeing
calculations of this form. Here's why i'm interested. For the past
several centuries there has been no competitor to the ``Newtonian''
account of dynamics. As a result the predominant share of accounts of
dynamical systems and situations have had to be formulated in terms of
the Newtonian machinery. i view this as an intellectually dangerous
position to occupy. Everything, despite it's intrinsic shape, turns
into a nail to be hit with this hammer. Recently, however, the theory
of computation has matured to the point where we have candidates for
theories of dynamics that offer very different perspective on
reasoning about dynamical systems and situations. Testing these
candidates against very successful accounts of dynamical situations,
like quantum mechanics, is going to give us some sense of how mature
they are and some measure of the quality of these accounts of
dynamics.

\subsection{Summary of contributions and outline of paper}

So, we're going to develop an interpretation of the operations of
quantum mechanics normally interpreted by Hilbert spaces and
operators. We're going to do this over a theory of computation. Note
that this is very different than the usual quantum computation program
which develops notions of computation over quantum mechanics. Rather,
we are developing a story that aligns with Wheeler's slogan: It from
Bit. To do this we will first provide an account of the theory of
computation at play here. Then we will dive into a calculation-driven
interpretation of the operations of quantum mechanics.

The reason we take this approach is that -- until very recently --
there hasn't been an axiomatic account of quantum mechanics. As a
result there has been no sharp delineation of the mathematical theory
supporting interpretation of the physical theory and the physical
theory, itself. So, ambient features of the maths are free to be
exploited (or supressed) without a real accounting of their physical
relevance. There is no sharp statement ``here's the physical theory''
qua \emph{theory} and ``here's the mathematical interpretation''
enabling a judgment of how faithful the interpretation is -- apart
from experimental observation. When there is an axiomatic account we
can judge how well a given mathematical formalism supports an
interpretation of the axioms, independent of
experimentation. Likewise, we can judge how well we have captured our
physical evidence and experience with our axiomatics, independent of
any specific mathematical implementation, with accidental detail that
may or may not have physical significance. 

In lieu of a fully fleshed out and vetted axiomatic account of quantum
mechanics, interpreting the operational notions in service of modeling
physical systems will have to suffice. In other words, we are not in
the business of providing a model of Hilbert spaces and operators. We
are in the business of providing a model of quantum mechanics because
we are motivated by testing our notions of dynamics against physical
theory; and, the predictive calculations of the physical theory must
serve as the best formulation -- shy of a fully fleshed out axiomatic
account -- of the physical theory itself (as they have for scientific
theories since time immemorial). Put another way, despite a
whole-hearted commitment to an It-from-Bit ontology, we are firmly
aligned with the shut-up-and-calculate camp as the best way to obtain
results either from the physical perspective or as a quality assurance
measure of our fledgling theory of dynamics.

In detail, we present a reflective process calculus. Then we develop
intuitive correspondences between the notions available in this
calculus and the usual physical notions supporting quantum mechanical
calculations. Thus, 

\begin{table}[htp]
  \center{
    \fbox{
      \begin{tabular}{c|c}
        quantum mechanics & process calculus \\
        \hline
        scalar & name \\
        state vector & process \\
        dual & contextual duals \\
        matrix & formal sums of process-context-dual pairs \\
        orthogonality & process annihilation \\
        inner product & execution-formula + quoting
      \end{tabular}
    }
  }
  \caption{QM - process calculi correspondences}
\end{table}

Then we tighten up these intuitions to operational definitions. We
employ the Dirac notation as the best proxy we can find for an
abstract syntax of the quantum mechanical notions. The definitions we
develop put us in contact with equational constraints coming from the
theory that we demonstrate the definitions and calculations satisfy.

This puts us in a position to shut up and calculate for the
Stern-Gerlach experimental set up, showing how these predictive
calculations become calculations on processes in our theory of a
reflective process calculus.

Penultimately, we demonstrate that the notion of metric coming from
the inner product coincides with the notion of metric available from
the theory of bisimulation. This demonstration gives us the right to
think of space as arising from behavior. Finally, we consider where we
might go from the new vantage point we have obtained.

% section introduction (end) 
 
% section introduction (end)

% \documentclass[12pt]{llncs}
%\documentclass{jktr}

\usepackage[pdftex]{hyperref}                   
\usepackage {listings}
\usepackage {mathpartir}
\usepackage{bcprules}
%\usepackage{listings}
                       
\usepackage{graphicx} 
%\usepackage[margins=2.5cm,nohead,nofoot]{geometry}
%\usepackage{geometry}
\usepackage{amsfonts}
\usepackage{amstext}
\usepackage{latexsym}
\usepackage{amssymb}
\usepackage{color}


%\include{myPreamble}
\include{qm2pi.local} 

%\ifpdf
%\usepackage[pdftex]{graphicx}
%\else
%\usepackage{graphicx}
%\fi

 % \ifpdf
%  \usepackage{pdfsync}
%  \if


%\title{Brief Article}
%\author{David F. Snyder}
%\author{L.G. Meredith}

%\address{Dept. of Math., Texas State University--San Marcos, San Marcos, TX 78666}
       
\pagestyle{empty}


\begin{document}

\lstset{language=[Objective]Caml,frame=shadowbox}

\input{qm2pi.front}

% section front matter (end)

\input{qm2pi.intro} 
 
% section introduction (end)

% \input{qm2pi.knotations} 

% section notation (end)

\input{qm2pi.process.calculi} 

% section concurrent_process_calculi_and_spatial_logics_ (end)
    
%\input{qm2pi.knots2pi} 

%\input{qm2pi.trefoil} 

%\input{qm2pi.mainthm} 

% subsection basic_interpretation (end)

%\input{qm2pi.rho.presentation} 
\subsection{The syntax and semantics of the notation system}\label{sub:the_syntax_and_semantics_of_the_notation_system} % (fold)

We now summarize a technical presentation of the calculus that
embodies our theory of dynamics. The typical presentation of such a
calculus follows the style of giving generators and relations on
them. The grammar, below, describing term constructors, freely
generates the set of processes, $\Proc$. This set is then quotiented
by a relation known as structural congruence and it is over this set
that the notion of dynamics is expressed. This presentation is
essentially that of \cite{MeredithR05} with the addition of
polyadicity and summation. For readability we have relegated some of
the technical subtleties to an appendix.

\subsubsection{Process grammar}\label{subsub:process_grammar}

\begin{mathpar}
  \inferrule* [lab=synchronization] {} {{M} \bc \pzero \;|\; x?F \;|\; x!C }
  \and
  \inferrule* [lab=abstraction] {} {{F} \bc (x)P}
  \and
  \inferrule* [lab=concretion] {} {{C} \bc \langle Q \rangle}
  \and
  \inferrule* [lab=process] {} {{P,Q} \bc M \;| \;P|Q \;|\; @{x}}
  \and
  \inferrule* [lab=name] {} {{x} \bc \quotep{P}}
\end{mathpar} 

Note that $\vec{x}$ (resp. $\vec{P}$) denotes a vector of names
(resp. processes) of length $|\vec{x}|$ (resp. $|\vec{P}|$). We adopt
the following useful abbreviations.

\begin{mathpar}
   x?(\vec{y}).P := x.(\vec{y})P \and  x\clift{\vec{P}} := x.\clift{\vec{P}}
   \and x!(y) := \lift{x}{\dropn{y}}
   \and \Pi_{i=0}^{n-1}P_i := P_0 | \ldots | P_{n-1}
\end{mathpar}

\subsubsection{Structural congruence}

\paragraph{Free and bound names and alpha-equivalence.} At the
core of structural equivalence is alpha-equivalence which identifies
process that are the same up to a change of variable. Formally, we
recognize the distinction between free and bound names. The free names
of a process, $\freenames{P}$, may be calculated recursively as
follows:

\begin{mathpar}
\freenames{\pzero} := \emptyset
  \and \\
  \freenames{x?(y).P} := \{ x \} \cup (\freenames{P} \setminus \{ y \})
  \and 
  \freenames{x!\langle P \rangle} := \{ x \} \cup \{ P \} 
  \and \\
  \freenames{P|Q} := \freenames{P} \cup \freenames{Q}
  \and \\
  \freenames{@{x}} := \{ x \}
\end{mathpar}

$\pi$
$\quotep{\pi}$

$\freenames{-} : \pi \to \mathcal{P}(\quotep{\pi})$

\begin{eqnarray*}
  \freenames{\pzero} & := & \emptyset \\
  \freenames{x?(y).P} & := & \{ x \} \cup (\freenames{P} \setminus \{ y \}) \\
  \freenames{x!\langle P \rangle} & := & \{ x \} \cup \{ P \} \\
  \freenames{P|Q} & := & \freenames{P} \cup \freenames{Q} \\
  \freenames{\dropn{x}} & := & \{ x \}
\end{eqnarray*}

The bound names of a process, $\boundnames{P}$, are those names occurring in $P$
that are not free. For example, in $x?(y).0$, the name $x$ is free, while $y$ is bound.

\begin{mathpar}
  \inferrule* [lab=monoidal-laws] {} { P|Q \equiv Q|P \and P|0 \equiv P \and P|(Q|R) \equiv (P|Q)|R }
\end{mathpar}

\begin{mathpar}
  \inferrule* [lab=alpha-equivalence] {} { (x)P \equiv (y)P\{y/x\} \and y \not\in \freenames{P} }
\end{mathpar}

\begin{definition}
Then two processes, $P,Q$, are alpha-equivalent if $P = Q\{\vec{y}/\vec{x}\}$ for
some $\vec{x} \in \boundnames{Q},\vec{y} \in \boundnames{P}$, where $Q\{\vec{y}/\vec{x}\}$
denotes the capture-avoiding substitution of $\vec{y}$ for $\vec{x}$ in $Q$.
\end{definition}

\begin{definition}
  The {\em structural congruence} \cite{SangiorgiWalker} , $\equiv$,
  between processes is the least congruence containing
  alpha-equivalence, satisfying the abelian monoid laws
  (associativity, commutativity and $\pzero$ as identity) for parallel
  composition $|$ and for summation $+$.
\end{definition}

\subsection{Name equivalence}

We take name equivalence, written $\nameeq$, to be the smallest
equivalence relation generated by the following rules.

\begin{mathpar}
\inferrule*[lab=Quote-drop]
{ }
{ \quotep{@{x}} \nameeq x }

\inferrule*[lab=Struct-equiv]
{ P \scong Q }
{ \quotep{P} \nameeq \quotep{Q} }
\end{mathpar}

The astute reader will have noticed that the mutual recursion of names
and processes imposes a mutual recursion on alpha-equivalence and
structural equivalence via name-equivalence. Fortunately, all of this
works out pleasantly and we may calculate in the natural way, free of
concern. The reader interested in the details is referred to the
appendix \ref{appendix:rho_details}.

\subsection{Substitution}

We use $\Proc$ for the set of processes, $\QProc$ for the set of
names, and $\id{\{}\vec{y} / \vec{x} \id{\}}$ to denote partial maps,
$s : \QProc \rightarrow \QProc$. A map, $s$ lifts, uniquely, to a map
on process terms, $\widehat{s} : \Proc \rightarrow \Proc$ by the
following equations.

\begin{mathpar}
  (0) \psubstp{Q}{P} := 0 \\
  (R \juxtap S) \psubstp{Q}{P}
  :=    
  (R)\psubstp{Q}{P} \juxtap (S) \psubstp{Q}{P} \\
  (x?(y).R) \psubstp{Q}{P}    
  :=    
  (x)\substp{Q}{P} (z)\concat( (R \psubstn{z}{y}) \psubstp{Q}{P} ) \\
  (\lift{x}{R}) \psubstp{Q}{P}  
  :=
  \lift{(x)\substp{Q}{P}}{ R \psubstp{Q}{P} } \\
%   (\dropn{x})  \psubstp{Q}{P}       
%   := 
%   \left\{ 
%     \begin{array}{ccc} 
%       \dropn{\quotep{Q}} & & x \nameeq \quotep{P} \\
%       \dropn{x} & & otherwise \\
%     \end{array}
%   \right. 
  (\dropn{x})  \psubstp{Q}{P}       
  := 
  \left\{ 
    \begin{array}{ccc} 
      Q & & x \nameeq \quotep{P} \\
      \dropn{x} & & otherwise \\
    \end{array}
  \right.
\end{mathpar}
 

where

\begin{eqnarray}
  (x)\id{\{} \lpquote Q \rpquote / \lpquote P \rpquote \id{\}}            = 
  \left\{ 
    \begin{array}{ccc}
      \lpquote Q \rpquote & & x \nameeq \lpquote P \rpquote \\
      x & & otherwise \\
    \end{array}
  \right. \nonumber
\end{eqnarray}

and $z$ is chosen distinct from $\quotep{P}$, $\quotep{Q}$, the free
names in $Q$, and all the names in $R$. Our $\alpha$-equivalence will
be built in the standard way from this substitution.

\begin{remark}\label{rem:no_self_referential_names}
  One consequence of these definitions is that $\forall P. \quotep{P}
  \not\in \freenames{P}$.
\end{remark}

\subsection{ Dynamic quote: an example }

Anticipating something of what's to come, consider applying the
substitution, $\widehat{\id{\{}u / z \id{\}}}$, to the following pair
of processes, $\lift{w}{y!(z)}$ and $w[ \lpquote y!(z) \rpquote ]$.

\begin{eqnarray}
	\lift{w}{y!(z)}\widehat{\id{\{}u / z \id{\}}}
		& = &
		\lift{w}{y!(u)} \nonumber\\
	w[ \lpquote y!(z) \rpquote ] \widehat{ \id{\{}u / z \id{\}} }
		& = &
		w[ \lpquote y!(z) \rpquote ] \nonumber
\end{eqnarray}

Because the body of the process between quotes is impervious to
substitution, we get radically different answers. In fact, by
examining the first process in an input context,
e.g. $x?(z).\lift{w}{y!(z)}$, we see that the process under the lift
operator may be shaped by prefixed inputs binding a name inside it. In
this sense, the lift operator will be seen as a way to dynamically
construct processes before reifying them as names.

Finally equipped with these standard features we can present the
dynamics of the calculus.

\subsubsection{Operational semantics} 

Finally, we introduce the computational dynamics. What marks these
algebras as distinct from other more traditionally studied algebraic
structures, e.g. vector spaces or polynomial rings, is the manner in
which dynamics is captured. In traditional structures, dynamics is typically
expressed through morphisms between such structures, as in linear maps
between vector spaces or morphisms between rings. In algebras
associated with the semantics of computation, the dynamics is
expressed as part of the algebraic structure itself, through a
reduction reduction relation typically denoted by $\red$. Below, we
give a recursive presentation of this relation for the calculus used
in the encoding.

$\red \subseteq \pi \times \pi$
$\red : \pi \to \mathcal{P}(\pi)$

\begin{mathpar}
  \inferrule* [lab=Comm] { \textsf{match}( x_{src}, x_{trgt} ) } { x_{trgt}?(y)P \; | \; x_{src}!\langle {Q} \rangle \red P\{\quotep{Q}/y}\} }
  \and \\
  \inferrule* [lab=Par] {{P} \red {P}'} {{{P} | {Q}} \red {{P}' | {Q}}}
  \and
  \inferrule* [lab=Equiv]{{{P} \scong {P}'} \andalso {{P}' \red {Q}'} \andalso {{Q}' \scong {Q}}}{{P} \red {Q}}
\end{mathpar}

\begin{eqnarray*}
  match_{\equiv} (\quotep{P},\quotep{Q}) & := & P \equiv Q \\
  match_{\dagger}(\quotep{P},\quotep{Q}) & := & \forall R. P|Q \red^{*} R => R \red^{*} 0 \\
  match_{K}(\quotep{P},\quotep{Q}) & := & K \mbox{ for some context } K
\end{eqnarray*}

$u?(x)P | u!\langle Q \rangle \red P\{\quotep{Q}/x\}$

%We write $\wred$ for $\red^*$, and $P\red$ if $\exists Q $ such that $ P \red Q$.
We write $P\red$ if $\exists Q $ such that $ P \red Q$ and $P\not\red$, otherwise.

\section{Replication}

As mentioned before, it is known that replication (and hence
recursion) can be implemented in a higher-order process algebra
\cite{SangiorgiWalker}. As our first example of calculation with the
machinery thus far presented we give the construction explicitly in
the {\rhoc}.

\begin{eqnarray}
	D_{x} & := & \prefix{x}{y}{(\binpar{\outputp{x}{y}}{@{y}})} \nonumber\\
	\bangp_{x}{P} & := & \binpar{{x}!\langle{\binpar{D_{x}}{P}}\rangle}{D_{x}} \nonumber
\end{eqnarray}

\begin{eqnarray}
	\bangp_{x}{P} & & \nonumber\\
	=
	& {x}!\langle{(\prefix{x}{y}{(\outputp{x}{y} | @{y})) | P}}\rangle 
	      | \prefix{x}{y}{(\outputp{x}{y} | @{y})} & \nonumber\\
	\red
	& (\outputp{x}{y} | @{y})\substn{\quotep{(\prefix{x}{y}{(@{y} | \outputp{x}{y})) | P}}}{y} & \nonumber\\
	=
	& \outputp{x}{\quotep{(\prefix{x}{y}{(\outputp{x}{y} | @{y})) | P}}}
	  | {(\prefix{x}{y}{(\outputp{x}{y} | @{y})) | P}} & \nonumber\\
	\red
	& \ldots & \nonumber\\
	\red^*
	& P | P | \ldots & \nonumber
\end{eqnarray}

Of course, this encoding, as an implementation, runs away, unfolding
$\bangp{P}$ eagerly. A lazier and more implementable replication
operator, restricted to input-guarded processes, may be obtained as follows.

\begin{eqnarray}
\bangp{\prefix{u}{v}{P}} 
	:= 
	\binpar{\lift{x}{\prefix{u}{v}{(\binpar{D(x)}{P})}}}{D(x)} \nonumber
\end{eqnarray}

\begin{remark}
  Note that the lazier definition still does not deal with summation
  or mixed summation (i.e. sums over input and output). The reader is
  invited to construct definitions of replication that deal with these
  features. 

  Further, the definitions are parameterized in a name, $x$. Can you,
  gentle reader, make a definition that eliminates this parameter and
  guarantees no accidental interaction between the replication
  machinery and the process being replicated -- i.e. no accidental
  sharing of names used by the process to get its work done and the
  name(s) used by the replication to effect copying. This latter
  revision of the definition of replication is crucial to obtaining
  the expected identity $!!P \sim !P$.
\end{remark}

\begin{remark}\label{rem:paradoxical_combinator}
  The reader familiar with the lambda calculus will have noticed the
  similarity between $D$ and the paradoxical combinator.

  [Ed. note: the existence of this seems to suggest we have to be more
  restrictive on the set of processes and names we admit if we are to
  support no-cloning.]
\end{remark}

\subsubsection{Bisimulation}

The computational dynamics gives rise to another kind of equivalence,
the equivalence of computational behavior. As previously mentioned
this is typically captured \emph{via} some form of bisimulation.

% The notion we use in this paper is weak barbed bisimulation
% \cite{milner91polyadicpi}.

The notion we use in this paper is derived from weak barbed
bisimulation \cite{milner91polyadicpi}. 

\begin{definition}
An \emph{observation relation}, $\downarrow_{\mathcal N}$, over a set
of names, $\mathcal N$, is the smallest relation satisfying the rules
below.

\infrule[Out-barb]{y \in {\mathcal N}, \; x \nameeq y}
		  {\outputp{x}{v} \downarrow_{\mathcal N} x}
\infrule[Par-barb]{\mbox{$P\downarrow_{\mathcal N} x$ or $Q\downarrow_{\mathcal N} x$}}
		  {\binpar{P}{Q} \downarrow_{\mathcal N} x}

We write $P \Downarrow_{\mathcal N} x$ if there is $Q$ such that 
$P \wred Q$ and $Q \downarrow_{\mathcal N} x$.
\end{definition}

\begin{definition}
%\label{def.bbisim}
An  ${\mathcal N}$-\emph{barbed bisimulation} over a set of names, ${\mathcal N}$, is a symmetric binary relation 
${\mathcal S}_{\mathcal N}$ between agents such that $P\rel{S}_{\mathcal N}Q$ implies:
\begin{enumerate}
\item If $P \red P'$ then $Q \wred Q'$ and $P'\rel{S}_{\mathcal N} Q'$.
\item If $P\downarrow_{\mathcal N} x$, then $Q\Downarrow_{\mathcal N} x$.
\end{enumerate}
$P$ is ${\mathcal N}$-barbed bisimilar to $Q$, written
$P \wbbisim_{\mathcal N} Q$, if $P \rel{S}_{\mathcal N} Q$ for some ${\mathcal N}$-barbed bisimulation ${\mathcal S}_{\mathcal N}$.
\end{definition}

$\mathcal{R} \subseteq \pi \times \pi$

$P \mathcal{R} Q => \forall P'. P \red P' \Rightarrow \exists Q'. Q \red Q', P' \mathcal{R} Q'$

$P \vdash x \Rightarrow Q \vdash x$

\begin{mathpar}
  \inferrule*[lab=Out-barb]{x \nameeq y}{{y}!\langle{Q}\rangle \vdash x}
  \and
  \inferrule*[lab=Par-barb]{\mbox{$P\vdash x$ or $Q\vdash x$}}{\binpar{P}{Q} \vdash x}
\end{mathpar}

\subsubsection{Contexts}

One of the principle advantages of computational calculi like the
$\pi$-calculus is a well-defined notion of context,
contextual-equivalence and a correlation between
contextual-equivalence and notions of bisimulation. The notion of
context allows the decomposition of a process into (sub-)process and
its syntactic environment, its context. Thus, a context may be
thought of as a process with a ``hole'' (written $\Box$) in it. The
application of a context $M$ to a process $P$, written $M[P]$, is
tantamount to filling the hole in $M$ with $P$. In this paper we do
not need the full weight of this theory, but do make use of the notion
of context in the proof the main theorem. 

\begin{mathpar}
  \inferrule* [lab=summation] {} {{M_{M},M_{N}} \bc \Box \;|\; x.M_{A} \;|\; M_{M}+M_{N}}
  \and
  \inferrule* [lab=agent] {} {{M_{A}} \bc (\vec{x})M_{P} \;| \; \clift{P_0,\ldots,M_{P},\ldots,P_N}}
  \and \\
  \inferrule* [lab=process] {} {{M_{P}} \bc M_{N} \;| \;P|M_{P} }
\end{mathpar} 

\begin{mathpar}
  \inferrule* [lab=sychronization] {} {M_{N} \bc \Box \;|\; x?M_{F} \;|\; x!M_{C}}
  \and
  \inferrule* [lab=abstraction] {} {{M_{F}} \bc (x)M_{P} }
  \and
  \inferrule* [lab=concretion] {} {{M_{C}} \bc \langle M_{P} \rangle }
  \and \\
  \inferrule* [lab=process] {} {{M_{P}} \bc M_{N} \;| \;P|M_{P} }
\end{mathpar}

\begin{definition}[contextual application] Given a context $M$, and
  process $P$, we define the \emph{contextual application}, $M[P] :=
  M\{P/\Box\}$. That is, the contextual application of M to P is the
  substitution of $P$ for $\Box$ in $M$.
\end{definition}

$\meaningof{-} : L \to \mathcal{P}(\pi)$

\begin{mathpar}
  \inferrule* [lab=collection] {} {\meaningof{true} = \pi, \and \meaningof{~E} = \pi \setminus \meaningof{E}, \and \meaningof{E_{1} \& E_{2}} = \meaningof{E_{1}} \cap \meaningof{E_{2}}}
\end{mathpar}

\begin{mathpar}
  \inferrule* [lab=structure] {} {\meaningof{0} = \{ P \in \pi | P \equiv 0 \}, \and \\ \meaningof{E_1 | E_2} = \{ P \in \pi | P \equiv P_{1} | P_{2}, P_{1} \in \meaningof{E_{1}}, P_{2} \in \meaningof{E_2}\} }
\end{mathpar}

\begin{mathpar}
 \inferrule* [lab=behavior] {} {\meaningof{\langle a?b \rangle E} = \{ P \in \pi | P \equiv Q | u?(y)P', \\ \and \\\\ \and \\ \;\;\; u \in \meaningof{a}, \forall z.P'\{z/y\} \in \meaningof{E\{z/b\}}\}, \and \\ \meaningof{a!E} = \{ P \in \pi | P \equiv Q | x!\langle P' \rangle, x \in \meaningof{a} P' \in \meaningof{E}\} }
\end{mathpar}

\begin{mathpar}
 \inferrule* [lab=nominal] {} {\meaningof{\quotep{E}} = \{ \quotep{P} \in \quotep{\pi} | P \in \meaningof{E} \}, \and \meaningof{\quotep{P}} = \{ \quotep{Q} \in \quotep{\pi} | P \equiv Q \} \and \\ \meaningof{@\quotep{E}} = \{ P \in \pi | P \equiv @x, x \in \meaningof{E} \}}
\end{mathpar}

\begin{eqnarray*}
  \\
  \meaningof{-} : TS \to ST
\end{eqnarray*}

\begin{eqnarray*}
  \\
  L : TS \to ST
\end{eqnarray*}

\begin{eqnarray*}
  \\
  P \models E \iff P \in \meaningof{E}
\end{eqnarray*}

\begin{eqnarray*}
  P \approx_{L} Q \iff \forall E \in L. P \models E \iff Q \models E
\end{eqnarray*}

\begin{eqnarray*}
  P \approx_{K} Q
\end{eqnarray*}

\begin{eqnarray*}
  P \approx Q
\end{eqnarray*}

$\approx_{K} = \approx = \approx_{L}$

\subsubsection{Contextual duality}

Note that contexts extend the quotation operation to a family of
operations from processes to names. Given a context, $M$, we can
define a \emph{nominal context}, $\quotep{M}$ by $\quotep{M}[P] :=
\quotep{M[P]}$. To foreshadow what is to come we observe that these
operations enjoy a duality with processes very much like the duality
between vectors and maps from vectors to scalars.

Further, because the calculus is essentially higher-order, we have a
correspondence between contexts and processes. More specifically,
given a name $x$ and a context $M$ we can construct $M^{*}_{x}$ such
that 

\begin{mathpar}
  M^{*}_{x} | \lift{x}{P} \red M[P]
\end{mathpar}

namely,

\begin{mathpar}
  M^{*}_{x} := x?(u).M[\dropn{u}]
\end{mathpar}

The dependence of $M^{*}_{x}$ on a name makes it an abstraction, 

\begin{mathpar}
  M^{*} := (x)x?(u).M[\dropn{u}]
\end{mathpar}

\subsection{Additional notation}

It will sometimes be convenient to denote the process a name
quotes. We already have the notation $x = \quotep{P}$, but it will be
convenient to introduce an alternate notation, $\procn{x}$, when we
want to emphasize the connection to the use of the name. Note that, by
virtue of name equivalence, $\quotep{\procn{x}} \nameeq x$; so, the
notation is consistent with previous definitions.

Further, because names have structure it is possible to effect
substitutions on the basis of that structure. This means we need to
upgrade our notation for substitutions, which we accomplish by
adapting comprehension notation. Thus,

\begin{mathpar}
  P\{ y / x : x \in S \}
\end{mathpar}

is interpreted to mean the process derived from P by replacing (in a
capture-avoiding manner) each occurrence of $x$ in $S$ by $y$. For example,

\begin{mathpar}
  P\{ \quotep{\procn{x}|\procn{x}} / x : x \in \freenames{P} \}
\end{mathpar}

will replace each (occurrence) of a free name $x$ in $P$ by
$\quotep{\procn{x}|\procn{x}}$.

Also, we will avail ourselves of the notation $x^{L}$ and $x^{R}$ to
denote injections of a name into disjoint copies of the name
space. There are numerous ways to accomplish this. One example can be
found in \cite{MeredithR05}. This notation overloads to vectors of
names: $\vec{x}^{\pi} := (x_{i}^{\pi} \; : \; 0 \leq i < |\vec{x}| )$ where $\pi \in \{L,R\}$.

We also use $P^{\Box} := P|\Box$.

In \cite{MeredithR05} an interpretation of the new operator is
given. It turns out that there are several possible interpretations
all enjoying the requisite algebraic properties of the operator (see
\cite{milner91polyadicpi}). We will therefore make liberal use of
$(\nu\; \vec{x})P$.

% subsection the_syntax_and_semantics_of_the_notation_system (end)   

\input{qm2pi.qmops} 

\input{qm2pi.sterngerlach} 

\input{qm2pi.metric} 

% section concurrent_process_calculi (end)

%\input{qm2pi.proofsketch}

% section proof sketch (end)

%\input{qm2pi.slviaknots} 

% section spatial logic via knots (end)

\input{qm2pi.conclusion}

% section conclusion (end)

%\input{qm2pi.dtcodes} 

% section wiring algorithm (end)

\input{qm2pi.ack} 

% section acknowledgments (end)

\newpage


\bibliographystyle{plain}   
\bibliography{../../biblios/main.bib}

\input{qm2pi.rhodetails}

\end{document}

 

% section notation (end)

\input{qm2pi.process.calculi} 

% section concurrent_process_calculi_and_spatial_logics_ (end)
    
%\documentclass[12pt]{llncs}
%\documentclass{jktr}

\usepackage[pdftex]{hyperref}                   
\usepackage {listings}
\usepackage {mathpartir}
\usepackage{bcprules}
%\usepackage{listings}
                       
\usepackage{graphicx} 
%\usepackage[margins=2.5cm,nohead,nofoot]{geometry}
%\usepackage{geometry}
\usepackage{amsfonts}
\usepackage{amstext}
\usepackage{latexsym}
\usepackage{amssymb}
\usepackage{color}


%\include{myPreamble}
\include{qm2pi.local} 

%\ifpdf
%\usepackage[pdftex]{graphicx}
%\else
%\usepackage{graphicx}
%\fi

 % \ifpdf
%  \usepackage{pdfsync}
%  \if


%\title{Brief Article}
%\author{David F. Snyder}
%\author{L.G. Meredith}

%\address{Dept. of Math., Texas State University--San Marcos, San Marcos, TX 78666}
       
\pagestyle{empty}


\begin{document}

\lstset{language=[Objective]Caml,frame=shadowbox}

\input{qm2pi.front}

% section front matter (end)

\input{qm2pi.intro} 
 
% section introduction (end)

% \input{qm2pi.knotations} 

% section notation (end)

\input{qm2pi.process.calculi} 

% section concurrent_process_calculi_and_spatial_logics_ (end)
    
%\input{qm2pi.knots2pi} 

%\input{qm2pi.trefoil} 

%\input{qm2pi.mainthm} 

% subsection basic_interpretation (end)

%\input{qm2pi.rho.presentation} 
\subsection{The syntax and semantics of the notation system}\label{sub:the_syntax_and_semantics_of_the_notation_system} % (fold)

We now summarize a technical presentation of the calculus that
embodies our theory of dynamics. The typical presentation of such a
calculus follows the style of giving generators and relations on
them. The grammar, below, describing term constructors, freely
generates the set of processes, $\Proc$. This set is then quotiented
by a relation known as structural congruence and it is over this set
that the notion of dynamics is expressed. This presentation is
essentially that of \cite{MeredithR05} with the addition of
polyadicity and summation. For readability we have relegated some of
the technical subtleties to an appendix.

\subsubsection{Process grammar}\label{subsub:process_grammar}

\begin{mathpar}
  \inferrule* [lab=synchronization] {} {{M} \bc \pzero \;|\; x?F \;|\; x!C }
  \and
  \inferrule* [lab=abstraction] {} {{F} \bc (x)P}
  \and
  \inferrule* [lab=concretion] {} {{C} \bc \langle Q \rangle}
  \and
  \inferrule* [lab=process] {} {{P,Q} \bc M \;| \;P|Q \;|\; @{x}}
  \and
  \inferrule* [lab=name] {} {{x} \bc \quotep{P}}
\end{mathpar} 

Note that $\vec{x}$ (resp. $\vec{P}$) denotes a vector of names
(resp. processes) of length $|\vec{x}|$ (resp. $|\vec{P}|$). We adopt
the following useful abbreviations.

\begin{mathpar}
   x?(\vec{y}).P := x.(\vec{y})P \and  x\clift{\vec{P}} := x.\clift{\vec{P}}
   \and x!(y) := \lift{x}{\dropn{y}}
   \and \Pi_{i=0}^{n-1}P_i := P_0 | \ldots | P_{n-1}
\end{mathpar}

\subsubsection{Structural congruence}

\paragraph{Free and bound names and alpha-equivalence.} At the
core of structural equivalence is alpha-equivalence which identifies
process that are the same up to a change of variable. Formally, we
recognize the distinction between free and bound names. The free names
of a process, $\freenames{P}$, may be calculated recursively as
follows:

\begin{mathpar}
\freenames{\pzero} := \emptyset
  \and \\
  \freenames{x?(y).P} := \{ x \} \cup (\freenames{P} \setminus \{ y \})
  \and 
  \freenames{x!\langle P \rangle} := \{ x \} \cup \{ P \} 
  \and \\
  \freenames{P|Q} := \freenames{P} \cup \freenames{Q}
  \and \\
  \freenames{@{x}} := \{ x \}
\end{mathpar}

$\pi$
$\quotep{\pi}$

$\freenames{-} : \pi \to \mathcal{P}(\quotep{\pi})$

\begin{eqnarray*}
  \freenames{\pzero} & := & \emptyset \\
  \freenames{x?(y).P} & := & \{ x \} \cup (\freenames{P} \setminus \{ y \}) \\
  \freenames{x!\langle P \rangle} & := & \{ x \} \cup \{ P \} \\
  \freenames{P|Q} & := & \freenames{P} \cup \freenames{Q} \\
  \freenames{\dropn{x}} & := & \{ x \}
\end{eqnarray*}

The bound names of a process, $\boundnames{P}$, are those names occurring in $P$
that are not free. For example, in $x?(y).0$, the name $x$ is free, while $y$ is bound.

\begin{mathpar}
  \inferrule* [lab=monoidal-laws] {} { P|Q \equiv Q|P \and P|0 \equiv P \and P|(Q|R) \equiv (P|Q)|R }
\end{mathpar}

\begin{mathpar}
  \inferrule* [lab=alpha-equivalence] {} { (x)P \equiv (y)P\{y/x\} \and y \not\in \freenames{P} }
\end{mathpar}

\begin{definition}
Then two processes, $P,Q$, are alpha-equivalent if $P = Q\{\vec{y}/\vec{x}\}$ for
some $\vec{x} \in \boundnames{Q},\vec{y} \in \boundnames{P}$, where $Q\{\vec{y}/\vec{x}\}$
denotes the capture-avoiding substitution of $\vec{y}$ for $\vec{x}$ in $Q$.
\end{definition}

\begin{definition}
  The {\em structural congruence} \cite{SangiorgiWalker} , $\equiv$,
  between processes is the least congruence containing
  alpha-equivalence, satisfying the abelian monoid laws
  (associativity, commutativity and $\pzero$ as identity) for parallel
  composition $|$ and for summation $+$.
\end{definition}

\subsection{Name equivalence}

We take name equivalence, written $\nameeq$, to be the smallest
equivalence relation generated by the following rules.

\begin{mathpar}
\inferrule*[lab=Quote-drop]
{ }
{ \quotep{@{x}} \nameeq x }

\inferrule*[lab=Struct-equiv]
{ P \scong Q }
{ \quotep{P} \nameeq \quotep{Q} }
\end{mathpar}

The astute reader will have noticed that the mutual recursion of names
and processes imposes a mutual recursion on alpha-equivalence and
structural equivalence via name-equivalence. Fortunately, all of this
works out pleasantly and we may calculate in the natural way, free of
concern. The reader interested in the details is referred to the
appendix \ref{appendix:rho_details}.

\subsection{Substitution}

We use $\Proc$ for the set of processes, $\QProc$ for the set of
names, and $\id{\{}\vec{y} / \vec{x} \id{\}}$ to denote partial maps,
$s : \QProc \rightarrow \QProc$. A map, $s$ lifts, uniquely, to a map
on process terms, $\widehat{s} : \Proc \rightarrow \Proc$ by the
following equations.

\begin{mathpar}
  (0) \psubstp{Q}{P} := 0 \\
  (R \juxtap S) \psubstp{Q}{P}
  :=    
  (R)\psubstp{Q}{P} \juxtap (S) \psubstp{Q}{P} \\
  (x?(y).R) \psubstp{Q}{P}    
  :=    
  (x)\substp{Q}{P} (z)\concat( (R \psubstn{z}{y}) \psubstp{Q}{P} ) \\
  (\lift{x}{R}) \psubstp{Q}{P}  
  :=
  \lift{(x)\substp{Q}{P}}{ R \psubstp{Q}{P} } \\
%   (\dropn{x})  \psubstp{Q}{P}       
%   := 
%   \left\{ 
%     \begin{array}{ccc} 
%       \dropn{\quotep{Q}} & & x \nameeq \quotep{P} \\
%       \dropn{x} & & otherwise \\
%     \end{array}
%   \right. 
  (\dropn{x})  \psubstp{Q}{P}       
  := 
  \left\{ 
    \begin{array}{ccc} 
      Q & & x \nameeq \quotep{P} \\
      \dropn{x} & & otherwise \\
    \end{array}
  \right.
\end{mathpar}
 

where

\begin{eqnarray}
  (x)\id{\{} \lpquote Q \rpquote / \lpquote P \rpquote \id{\}}            = 
  \left\{ 
    \begin{array}{ccc}
      \lpquote Q \rpquote & & x \nameeq \lpquote P \rpquote \\
      x & & otherwise \\
    \end{array}
  \right. \nonumber
\end{eqnarray}

and $z$ is chosen distinct from $\quotep{P}$, $\quotep{Q}$, the free
names in $Q$, and all the names in $R$. Our $\alpha$-equivalence will
be built in the standard way from this substitution.

\begin{remark}\label{rem:no_self_referential_names}
  One consequence of these definitions is that $\forall P. \quotep{P}
  \not\in \freenames{P}$.
\end{remark}

\subsection{ Dynamic quote: an example }

Anticipating something of what's to come, consider applying the
substitution, $\widehat{\id{\{}u / z \id{\}}}$, to the following pair
of processes, $\lift{w}{y!(z)}$ and $w[ \lpquote y!(z) \rpquote ]$.

\begin{eqnarray}
	\lift{w}{y!(z)}\widehat{\id{\{}u / z \id{\}}}
		& = &
		\lift{w}{y!(u)} \nonumber\\
	w[ \lpquote y!(z) \rpquote ] \widehat{ \id{\{}u / z \id{\}} }
		& = &
		w[ \lpquote y!(z) \rpquote ] \nonumber
\end{eqnarray}

Because the body of the process between quotes is impervious to
substitution, we get radically different answers. In fact, by
examining the first process in an input context,
e.g. $x?(z).\lift{w}{y!(z)}$, we see that the process under the lift
operator may be shaped by prefixed inputs binding a name inside it. In
this sense, the lift operator will be seen as a way to dynamically
construct processes before reifying them as names.

Finally equipped with these standard features we can present the
dynamics of the calculus.

\subsubsection{Operational semantics} 

Finally, we introduce the computational dynamics. What marks these
algebras as distinct from other more traditionally studied algebraic
structures, e.g. vector spaces or polynomial rings, is the manner in
which dynamics is captured. In traditional structures, dynamics is typically
expressed through morphisms between such structures, as in linear maps
between vector spaces or morphisms between rings. In algebras
associated with the semantics of computation, the dynamics is
expressed as part of the algebraic structure itself, through a
reduction reduction relation typically denoted by $\red$. Below, we
give a recursive presentation of this relation for the calculus used
in the encoding.

$\red \subseteq \pi \times \pi$
$\red : \pi \to \mathcal{P}(\pi)$

\begin{mathpar}
  \inferrule* [lab=Comm] { \textsf{match}( x_{src}, x_{trgt} ) } { x_{trgt}?(y)P \; | \; x_{src}!\langle {Q} \rangle \red P\{\quotep{Q}/y}\} }
  \and \\
  \inferrule* [lab=Par] {{P} \red {P}'} {{{P} | {Q}} \red {{P}' | {Q}}}
  \and
  \inferrule* [lab=Equiv]{{{P} \scong {P}'} \andalso {{P}' \red {Q}'} \andalso {{Q}' \scong {Q}}}{{P} \red {Q}}
\end{mathpar}

\begin{eqnarray*}
  match_{\equiv} (\quotep{P},\quotep{Q}) & := & P \equiv Q \\
  match_{\dagger}(\quotep{P},\quotep{Q}) & := & \forall R. P|Q \red^{*} R => R \red^{*} 0 \\
  match_{K}(\quotep{P},\quotep{Q}) & := & K \mbox{ for some context } K
\end{eqnarray*}

$u?(x)P | u!\langle Q \rangle \red P\{\quotep{Q}/x\}$

%We write $\wred$ for $\red^*$, and $P\red$ if $\exists Q $ such that $ P \red Q$.
We write $P\red$ if $\exists Q $ such that $ P \red Q$ and $P\not\red$, otherwise.

\section{Replication}

As mentioned before, it is known that replication (and hence
recursion) can be implemented in a higher-order process algebra
\cite{SangiorgiWalker}. As our first example of calculation with the
machinery thus far presented we give the construction explicitly in
the {\rhoc}.

\begin{eqnarray}
	D_{x} & := & \prefix{x}{y}{(\binpar{\outputp{x}{y}}{@{y}})} \nonumber\\
	\bangp_{x}{P} & := & \binpar{{x}!\langle{\binpar{D_{x}}{P}}\rangle}{D_{x}} \nonumber
\end{eqnarray}

\begin{eqnarray}
	\bangp_{x}{P} & & \nonumber\\
	=
	& {x}!\langle{(\prefix{x}{y}{(\outputp{x}{y} | @{y})) | P}}\rangle 
	      | \prefix{x}{y}{(\outputp{x}{y} | @{y})} & \nonumber\\
	\red
	& (\outputp{x}{y} | @{y})\substn{\quotep{(\prefix{x}{y}{(@{y} | \outputp{x}{y})) | P}}}{y} & \nonumber\\
	=
	& \outputp{x}{\quotep{(\prefix{x}{y}{(\outputp{x}{y} | @{y})) | P}}}
	  | {(\prefix{x}{y}{(\outputp{x}{y} | @{y})) | P}} & \nonumber\\
	\red
	& \ldots & \nonumber\\
	\red^*
	& P | P | \ldots & \nonumber
\end{eqnarray}

Of course, this encoding, as an implementation, runs away, unfolding
$\bangp{P}$ eagerly. A lazier and more implementable replication
operator, restricted to input-guarded processes, may be obtained as follows.

\begin{eqnarray}
\bangp{\prefix{u}{v}{P}} 
	:= 
	\binpar{\lift{x}{\prefix{u}{v}{(\binpar{D(x)}{P})}}}{D(x)} \nonumber
\end{eqnarray}

\begin{remark}
  Note that the lazier definition still does not deal with summation
  or mixed summation (i.e. sums over input and output). The reader is
  invited to construct definitions of replication that deal with these
  features. 

  Further, the definitions are parameterized in a name, $x$. Can you,
  gentle reader, make a definition that eliminates this parameter and
  guarantees no accidental interaction between the replication
  machinery and the process being replicated -- i.e. no accidental
  sharing of names used by the process to get its work done and the
  name(s) used by the replication to effect copying. This latter
  revision of the definition of replication is crucial to obtaining
  the expected identity $!!P \sim !P$.
\end{remark}

\begin{remark}\label{rem:paradoxical_combinator}
  The reader familiar with the lambda calculus will have noticed the
  similarity between $D$ and the paradoxical combinator.

  [Ed. note: the existence of this seems to suggest we have to be more
  restrictive on the set of processes and names we admit if we are to
  support no-cloning.]
\end{remark}

\subsubsection{Bisimulation}

The computational dynamics gives rise to another kind of equivalence,
the equivalence of computational behavior. As previously mentioned
this is typically captured \emph{via} some form of bisimulation.

% The notion we use in this paper is weak barbed bisimulation
% \cite{milner91polyadicpi}.

The notion we use in this paper is derived from weak barbed
bisimulation \cite{milner91polyadicpi}. 

\begin{definition}
An \emph{observation relation}, $\downarrow_{\mathcal N}$, over a set
of names, $\mathcal N$, is the smallest relation satisfying the rules
below.

\infrule[Out-barb]{y \in {\mathcal N}, \; x \nameeq y}
		  {\outputp{x}{v} \downarrow_{\mathcal N} x}
\infrule[Par-barb]{\mbox{$P\downarrow_{\mathcal N} x$ or $Q\downarrow_{\mathcal N} x$}}
		  {\binpar{P}{Q} \downarrow_{\mathcal N} x}

We write $P \Downarrow_{\mathcal N} x$ if there is $Q$ such that 
$P \wred Q$ and $Q \downarrow_{\mathcal N} x$.
\end{definition}

\begin{definition}
%\label{def.bbisim}
An  ${\mathcal N}$-\emph{barbed bisimulation} over a set of names, ${\mathcal N}$, is a symmetric binary relation 
${\mathcal S}_{\mathcal N}$ between agents such that $P\rel{S}_{\mathcal N}Q$ implies:
\begin{enumerate}
\item If $P \red P'$ then $Q \wred Q'$ and $P'\rel{S}_{\mathcal N} Q'$.
\item If $P\downarrow_{\mathcal N} x$, then $Q\Downarrow_{\mathcal N} x$.
\end{enumerate}
$P$ is ${\mathcal N}$-barbed bisimilar to $Q$, written
$P \wbbisim_{\mathcal N} Q$, if $P \rel{S}_{\mathcal N} Q$ for some ${\mathcal N}$-barbed bisimulation ${\mathcal S}_{\mathcal N}$.
\end{definition}

$\mathcal{R} \subseteq \pi \times \pi$

$P \mathcal{R} Q => \forall P'. P \red P' \Rightarrow \exists Q'. Q \red Q', P' \mathcal{R} Q'$

$P \vdash x \Rightarrow Q \vdash x$

\begin{mathpar}
  \inferrule*[lab=Out-barb]{x \nameeq y}{{y}!\langle{Q}\rangle \vdash x}
  \and
  \inferrule*[lab=Par-barb]{\mbox{$P\vdash x$ or $Q\vdash x$}}{\binpar{P}{Q} \vdash x}
\end{mathpar}

\subsubsection{Contexts}

One of the principle advantages of computational calculi like the
$\pi$-calculus is a well-defined notion of context,
contextual-equivalence and a correlation between
contextual-equivalence and notions of bisimulation. The notion of
context allows the decomposition of a process into (sub-)process and
its syntactic environment, its context. Thus, a context may be
thought of as a process with a ``hole'' (written $\Box$) in it. The
application of a context $M$ to a process $P$, written $M[P]$, is
tantamount to filling the hole in $M$ with $P$. In this paper we do
not need the full weight of this theory, but do make use of the notion
of context in the proof the main theorem. 

\begin{mathpar}
  \inferrule* [lab=summation] {} {{M_{M},M_{N}} \bc \Box \;|\; x.M_{A} \;|\; M_{M}+M_{N}}
  \and
  \inferrule* [lab=agent] {} {{M_{A}} \bc (\vec{x})M_{P} \;| \; \clift{P_0,\ldots,M_{P},\ldots,P_N}}
  \and \\
  \inferrule* [lab=process] {} {{M_{P}} \bc M_{N} \;| \;P|M_{P} }
\end{mathpar} 

\begin{mathpar}
  \inferrule* [lab=sychronization] {} {M_{N} \bc \Box \;|\; x?M_{F} \;|\; x!M_{C}}
  \and
  \inferrule* [lab=abstraction] {} {{M_{F}} \bc (x)M_{P} }
  \and
  \inferrule* [lab=concretion] {} {{M_{C}} \bc \langle M_{P} \rangle }
  \and \\
  \inferrule* [lab=process] {} {{M_{P}} \bc M_{N} \;| \;P|M_{P} }
\end{mathpar}

\begin{definition}[contextual application] Given a context $M$, and
  process $P$, we define the \emph{contextual application}, $M[P] :=
  M\{P/\Box\}$. That is, the contextual application of M to P is the
  substitution of $P$ for $\Box$ in $M$.
\end{definition}

$\meaningof{-} : L \to \mathcal{P}(\pi)$

\begin{mathpar}
  \inferrule* [lab=collection] {} {\meaningof{true} = \pi, \and \meaningof{~E} = \pi \setminus \meaningof{E}, \and \meaningof{E_{1} \& E_{2}} = \meaningof{E_{1}} \cap \meaningof{E_{2}}}
\end{mathpar}

\begin{mathpar}
  \inferrule* [lab=structure] {} {\meaningof{0} = \{ P \in \pi | P \equiv 0 \}, \and \\ \meaningof{E_1 | E_2} = \{ P \in \pi | P \equiv P_{1} | P_{2}, P_{1} \in \meaningof{E_{1}}, P_{2} \in \meaningof{E_2}\} }
\end{mathpar}

\begin{mathpar}
 \inferrule* [lab=behavior] {} {\meaningof{\langle a?b \rangle E} = \{ P \in \pi | P \equiv Q | u?(y)P', \\ \and \\\\ \and \\ \;\;\; u \in \meaningof{a}, \forall z.P'\{z/y\} \in \meaningof{E\{z/b\}}\}, \and \\ \meaningof{a!E} = \{ P \in \pi | P \equiv Q | x!\langle P' \rangle, x \in \meaningof{a} P' \in \meaningof{E}\} }
\end{mathpar}

\begin{mathpar}
 \inferrule* [lab=nominal] {} {\meaningof{\quotep{E}} = \{ \quotep{P} \in \quotep{\pi} | P \in \meaningof{E} \}, \and \meaningof{\quotep{P}} = \{ \quotep{Q} \in \quotep{\pi} | P \equiv Q \} \and \\ \meaningof{@\quotep{E}} = \{ P \in \pi | P \equiv @x, x \in \meaningof{E} \}}
\end{mathpar}

\begin{eqnarray*}
  \\
  \meaningof{-} : TS \to ST
\end{eqnarray*}

\begin{eqnarray*}
  \\
  L : TS \to ST
\end{eqnarray*}

\begin{eqnarray*}
  \\
  P \models E \iff P \in \meaningof{E}
\end{eqnarray*}

\begin{eqnarray*}
  P \approx_{L} Q \iff \forall E \in L. P \models E \iff Q \models E
\end{eqnarray*}

\begin{eqnarray*}
  P \approx_{K} Q
\end{eqnarray*}

\begin{eqnarray*}
  P \approx Q
\end{eqnarray*}

$\approx_{K} = \approx = \approx_{L}$

\subsubsection{Contextual duality}

Note that contexts extend the quotation operation to a family of
operations from processes to names. Given a context, $M$, we can
define a \emph{nominal context}, $\quotep{M}$ by $\quotep{M}[P] :=
\quotep{M[P]}$. To foreshadow what is to come we observe that these
operations enjoy a duality with processes very much like the duality
between vectors and maps from vectors to scalars.

Further, because the calculus is essentially higher-order, we have a
correspondence between contexts and processes. More specifically,
given a name $x$ and a context $M$ we can construct $M^{*}_{x}$ such
that 

\begin{mathpar}
  M^{*}_{x} | \lift{x}{P} \red M[P]
\end{mathpar}

namely,

\begin{mathpar}
  M^{*}_{x} := x?(u).M[\dropn{u}]
\end{mathpar}

The dependence of $M^{*}_{x}$ on a name makes it an abstraction, 

\begin{mathpar}
  M^{*} := (x)x?(u).M[\dropn{u}]
\end{mathpar}

\subsection{Additional notation}

It will sometimes be convenient to denote the process a name
quotes. We already have the notation $x = \quotep{P}$, but it will be
convenient to introduce an alternate notation, $\procn{x}$, when we
want to emphasize the connection to the use of the name. Note that, by
virtue of name equivalence, $\quotep{\procn{x}} \nameeq x$; so, the
notation is consistent with previous definitions.

Further, because names have structure it is possible to effect
substitutions on the basis of that structure. This means we need to
upgrade our notation for substitutions, which we accomplish by
adapting comprehension notation. Thus,

\begin{mathpar}
  P\{ y / x : x \in S \}
\end{mathpar}

is interpreted to mean the process derived from P by replacing (in a
capture-avoiding manner) each occurrence of $x$ in $S$ by $y$. For example,

\begin{mathpar}
  P\{ \quotep{\procn{x}|\procn{x}} / x : x \in \freenames{P} \}
\end{mathpar}

will replace each (occurrence) of a free name $x$ in $P$ by
$\quotep{\procn{x}|\procn{x}}$.

Also, we will avail ourselves of the notation $x^{L}$ and $x^{R}$ to
denote injections of a name into disjoint copies of the name
space. There are numerous ways to accomplish this. One example can be
found in \cite{MeredithR05}. This notation overloads to vectors of
names: $\vec{x}^{\pi} := (x_{i}^{\pi} \; : \; 0 \leq i < |\vec{x}| )$ where $\pi \in \{L,R\}$.

We also use $P^{\Box} := P|\Box$.

In \cite{MeredithR05} an interpretation of the new operator is
given. It turns out that there are several possible interpretations
all enjoying the requisite algebraic properties of the operator (see
\cite{milner91polyadicpi}). We will therefore make liberal use of
$(\nu\; \vec{x})P$.

% subsection the_syntax_and_semantics_of_the_notation_system (end)   

\input{qm2pi.qmops} 

\input{qm2pi.sterngerlach} 

\input{qm2pi.metric} 

% section concurrent_process_calculi (end)

%\input{qm2pi.proofsketch}

% section proof sketch (end)

%\input{qm2pi.slviaknots} 

% section spatial logic via knots (end)

\input{qm2pi.conclusion}

% section conclusion (end)

%\input{qm2pi.dtcodes} 

% section wiring algorithm (end)

\input{qm2pi.ack} 

% section acknowledgments (end)

\newpage


\bibliographystyle{plain}   
\bibliography{../../biblios/main.bib}

\input{qm2pi.rhodetails}

\end{document}

 

%\documentclass[12pt]{llncs}
%\documentclass{jktr}

\usepackage[pdftex]{hyperref}                   
\usepackage {listings}
\usepackage {mathpartir}
\usepackage{bcprules}
%\usepackage{listings}
                       
\usepackage{graphicx} 
%\usepackage[margins=2.5cm,nohead,nofoot]{geometry}
%\usepackage{geometry}
\usepackage{amsfonts}
\usepackage{amstext}
\usepackage{latexsym}
\usepackage{amssymb}
\usepackage{color}


%\include{myPreamble}
\include{qm2pi.local} 

%\ifpdf
%\usepackage[pdftex]{graphicx}
%\else
%\usepackage{graphicx}
%\fi

 % \ifpdf
%  \usepackage{pdfsync}
%  \if


%\title{Brief Article}
%\author{David F. Snyder}
%\author{L.G. Meredith}

%\address{Dept. of Math., Texas State University--San Marcos, San Marcos, TX 78666}
       
\pagestyle{empty}


\begin{document}

\lstset{language=[Objective]Caml,frame=shadowbox}

\input{qm2pi.front}

% section front matter (end)

\input{qm2pi.intro} 
 
% section introduction (end)

% \input{qm2pi.knotations} 

% section notation (end)

\input{qm2pi.process.calculi} 

% section concurrent_process_calculi_and_spatial_logics_ (end)
    
%\input{qm2pi.knots2pi} 

%\input{qm2pi.trefoil} 

%\input{qm2pi.mainthm} 

% subsection basic_interpretation (end)

%\input{qm2pi.rho.presentation} 
\subsection{The syntax and semantics of the notation system}\label{sub:the_syntax_and_semantics_of_the_notation_system} % (fold)

We now summarize a technical presentation of the calculus that
embodies our theory of dynamics. The typical presentation of such a
calculus follows the style of giving generators and relations on
them. The grammar, below, describing term constructors, freely
generates the set of processes, $\Proc$. This set is then quotiented
by a relation known as structural congruence and it is over this set
that the notion of dynamics is expressed. This presentation is
essentially that of \cite{MeredithR05} with the addition of
polyadicity and summation. For readability we have relegated some of
the technical subtleties to an appendix.

\subsubsection{Process grammar}\label{subsub:process_grammar}

\begin{mathpar}
  \inferrule* [lab=synchronization] {} {{M} \bc \pzero \;|\; x?F \;|\; x!C }
  \and
  \inferrule* [lab=abstraction] {} {{F} \bc (x)P}
  \and
  \inferrule* [lab=concretion] {} {{C} \bc \langle Q \rangle}
  \and
  \inferrule* [lab=process] {} {{P,Q} \bc M \;| \;P|Q \;|\; @{x}}
  \and
  \inferrule* [lab=name] {} {{x} \bc \quotep{P}}
\end{mathpar} 

Note that $\vec{x}$ (resp. $\vec{P}$) denotes a vector of names
(resp. processes) of length $|\vec{x}|$ (resp. $|\vec{P}|$). We adopt
the following useful abbreviations.

\begin{mathpar}
   x?(\vec{y}).P := x.(\vec{y})P \and  x\clift{\vec{P}} := x.\clift{\vec{P}}
   \and x!(y) := \lift{x}{\dropn{y}}
   \and \Pi_{i=0}^{n-1}P_i := P_0 | \ldots | P_{n-1}
\end{mathpar}

\subsubsection{Structural congruence}

\paragraph{Free and bound names and alpha-equivalence.} At the
core of structural equivalence is alpha-equivalence which identifies
process that are the same up to a change of variable. Formally, we
recognize the distinction between free and bound names. The free names
of a process, $\freenames{P}$, may be calculated recursively as
follows:

\begin{mathpar}
\freenames{\pzero} := \emptyset
  \and \\
  \freenames{x?(y).P} := \{ x \} \cup (\freenames{P} \setminus \{ y \})
  \and 
  \freenames{x!\langle P \rangle} := \{ x \} \cup \{ P \} 
  \and \\
  \freenames{P|Q} := \freenames{P} \cup \freenames{Q}
  \and \\
  \freenames{@{x}} := \{ x \}
\end{mathpar}

$\pi$
$\quotep{\pi}$

$\freenames{-} : \pi \to \mathcal{P}(\quotep{\pi})$

\begin{eqnarray*}
  \freenames{\pzero} & := & \emptyset \\
  \freenames{x?(y).P} & := & \{ x \} \cup (\freenames{P} \setminus \{ y \}) \\
  \freenames{x!\langle P \rangle} & := & \{ x \} \cup \{ P \} \\
  \freenames{P|Q} & := & \freenames{P} \cup \freenames{Q} \\
  \freenames{\dropn{x}} & := & \{ x \}
\end{eqnarray*}

The bound names of a process, $\boundnames{P}$, are those names occurring in $P$
that are not free. For example, in $x?(y).0$, the name $x$ is free, while $y$ is bound.

\begin{mathpar}
  \inferrule* [lab=monoidal-laws] {} { P|Q \equiv Q|P \and P|0 \equiv P \and P|(Q|R) \equiv (P|Q)|R }
\end{mathpar}

\begin{mathpar}
  \inferrule* [lab=alpha-equivalence] {} { (x)P \equiv (y)P\{y/x\} \and y \not\in \freenames{P} }
\end{mathpar}

\begin{definition}
Then two processes, $P,Q$, are alpha-equivalent if $P = Q\{\vec{y}/\vec{x}\}$ for
some $\vec{x} \in \boundnames{Q},\vec{y} \in \boundnames{P}$, where $Q\{\vec{y}/\vec{x}\}$
denotes the capture-avoiding substitution of $\vec{y}$ for $\vec{x}$ in $Q$.
\end{definition}

\begin{definition}
  The {\em structural congruence} \cite{SangiorgiWalker} , $\equiv$,
  between processes is the least congruence containing
  alpha-equivalence, satisfying the abelian monoid laws
  (associativity, commutativity and $\pzero$ as identity) for parallel
  composition $|$ and for summation $+$.
\end{definition}

\subsection{Name equivalence}

We take name equivalence, written $\nameeq$, to be the smallest
equivalence relation generated by the following rules.

\begin{mathpar}
\inferrule*[lab=Quote-drop]
{ }
{ \quotep{@{x}} \nameeq x }

\inferrule*[lab=Struct-equiv]
{ P \scong Q }
{ \quotep{P} \nameeq \quotep{Q} }
\end{mathpar}

The astute reader will have noticed that the mutual recursion of names
and processes imposes a mutual recursion on alpha-equivalence and
structural equivalence via name-equivalence. Fortunately, all of this
works out pleasantly and we may calculate in the natural way, free of
concern. The reader interested in the details is referred to the
appendix \ref{appendix:rho_details}.

\subsection{Substitution}

We use $\Proc$ for the set of processes, $\QProc$ for the set of
names, and $\id{\{}\vec{y} / \vec{x} \id{\}}$ to denote partial maps,
$s : \QProc \rightarrow \QProc$. A map, $s$ lifts, uniquely, to a map
on process terms, $\widehat{s} : \Proc \rightarrow \Proc$ by the
following equations.

\begin{mathpar}
  (0) \psubstp{Q}{P} := 0 \\
  (R \juxtap S) \psubstp{Q}{P}
  :=    
  (R)\psubstp{Q}{P} \juxtap (S) \psubstp{Q}{P} \\
  (x?(y).R) \psubstp{Q}{P}    
  :=    
  (x)\substp{Q}{P} (z)\concat( (R \psubstn{z}{y}) \psubstp{Q}{P} ) \\
  (\lift{x}{R}) \psubstp{Q}{P}  
  :=
  \lift{(x)\substp{Q}{P}}{ R \psubstp{Q}{P} } \\
%   (\dropn{x})  \psubstp{Q}{P}       
%   := 
%   \left\{ 
%     \begin{array}{ccc} 
%       \dropn{\quotep{Q}} & & x \nameeq \quotep{P} \\
%       \dropn{x} & & otherwise \\
%     \end{array}
%   \right. 
  (\dropn{x})  \psubstp{Q}{P}       
  := 
  \left\{ 
    \begin{array}{ccc} 
      Q & & x \nameeq \quotep{P} \\
      \dropn{x} & & otherwise \\
    \end{array}
  \right.
\end{mathpar}
 

where

\begin{eqnarray}
  (x)\id{\{} \lpquote Q \rpquote / \lpquote P \rpquote \id{\}}            = 
  \left\{ 
    \begin{array}{ccc}
      \lpquote Q \rpquote & & x \nameeq \lpquote P \rpquote \\
      x & & otherwise \\
    \end{array}
  \right. \nonumber
\end{eqnarray}

and $z$ is chosen distinct from $\quotep{P}$, $\quotep{Q}$, the free
names in $Q$, and all the names in $R$. Our $\alpha$-equivalence will
be built in the standard way from this substitution.

\begin{remark}\label{rem:no_self_referential_names}
  One consequence of these definitions is that $\forall P. \quotep{P}
  \not\in \freenames{P}$.
\end{remark}

\subsection{ Dynamic quote: an example }

Anticipating something of what's to come, consider applying the
substitution, $\widehat{\id{\{}u / z \id{\}}}$, to the following pair
of processes, $\lift{w}{y!(z)}$ and $w[ \lpquote y!(z) \rpquote ]$.

\begin{eqnarray}
	\lift{w}{y!(z)}\widehat{\id{\{}u / z \id{\}}}
		& = &
		\lift{w}{y!(u)} \nonumber\\
	w[ \lpquote y!(z) \rpquote ] \widehat{ \id{\{}u / z \id{\}} }
		& = &
		w[ \lpquote y!(z) \rpquote ] \nonumber
\end{eqnarray}

Because the body of the process between quotes is impervious to
substitution, we get radically different answers. In fact, by
examining the first process in an input context,
e.g. $x?(z).\lift{w}{y!(z)}$, we see that the process under the lift
operator may be shaped by prefixed inputs binding a name inside it. In
this sense, the lift operator will be seen as a way to dynamically
construct processes before reifying them as names.

Finally equipped with these standard features we can present the
dynamics of the calculus.

\subsubsection{Operational semantics} 

Finally, we introduce the computational dynamics. What marks these
algebras as distinct from other more traditionally studied algebraic
structures, e.g. vector spaces or polynomial rings, is the manner in
which dynamics is captured. In traditional structures, dynamics is typically
expressed through morphisms between such structures, as in linear maps
between vector spaces or morphisms between rings. In algebras
associated with the semantics of computation, the dynamics is
expressed as part of the algebraic structure itself, through a
reduction reduction relation typically denoted by $\red$. Below, we
give a recursive presentation of this relation for the calculus used
in the encoding.

$\red \subseteq \pi \times \pi$
$\red : \pi \to \mathcal{P}(\pi)$

\begin{mathpar}
  \inferrule* [lab=Comm] { \textsf{match}( x_{src}, x_{trgt} ) } { x_{trgt}?(y)P \; | \; x_{src}!\langle {Q} \rangle \red P\{\quotep{Q}/y}\} }
  \and \\
  \inferrule* [lab=Par] {{P} \red {P}'} {{{P} | {Q}} \red {{P}' | {Q}}}
  \and
  \inferrule* [lab=Equiv]{{{P} \scong {P}'} \andalso {{P}' \red {Q}'} \andalso {{Q}' \scong {Q}}}{{P} \red {Q}}
\end{mathpar}

\begin{eqnarray*}
  match_{\equiv} (\quotep{P},\quotep{Q}) & := & P \equiv Q \\
  match_{\dagger}(\quotep{P},\quotep{Q}) & := & \forall R. P|Q \red^{*} R => R \red^{*} 0 \\
  match_{K}(\quotep{P},\quotep{Q}) & := & K \mbox{ for some context } K
\end{eqnarray*}

$u?(x)P | u!\langle Q \rangle \red P\{\quotep{Q}/x\}$

%We write $\wred$ for $\red^*$, and $P\red$ if $\exists Q $ such that $ P \red Q$.
We write $P\red$ if $\exists Q $ such that $ P \red Q$ and $P\not\red$, otherwise.

\section{Replication}

As mentioned before, it is known that replication (and hence
recursion) can be implemented in a higher-order process algebra
\cite{SangiorgiWalker}. As our first example of calculation with the
machinery thus far presented we give the construction explicitly in
the {\rhoc}.

\begin{eqnarray}
	D_{x} & := & \prefix{x}{y}{(\binpar{\outputp{x}{y}}{@{y}})} \nonumber\\
	\bangp_{x}{P} & := & \binpar{{x}!\langle{\binpar{D_{x}}{P}}\rangle}{D_{x}} \nonumber
\end{eqnarray}

\begin{eqnarray}
	\bangp_{x}{P} & & \nonumber\\
	=
	& {x}!\langle{(\prefix{x}{y}{(\outputp{x}{y} | @{y})) | P}}\rangle 
	      | \prefix{x}{y}{(\outputp{x}{y} | @{y})} & \nonumber\\
	\red
	& (\outputp{x}{y} | @{y})\substn{\quotep{(\prefix{x}{y}{(@{y} | \outputp{x}{y})) | P}}}{y} & \nonumber\\
	=
	& \outputp{x}{\quotep{(\prefix{x}{y}{(\outputp{x}{y} | @{y})) | P}}}
	  | {(\prefix{x}{y}{(\outputp{x}{y} | @{y})) | P}} & \nonumber\\
	\red
	& \ldots & \nonumber\\
	\red^*
	& P | P | \ldots & \nonumber
\end{eqnarray}

Of course, this encoding, as an implementation, runs away, unfolding
$\bangp{P}$ eagerly. A lazier and more implementable replication
operator, restricted to input-guarded processes, may be obtained as follows.

\begin{eqnarray}
\bangp{\prefix{u}{v}{P}} 
	:= 
	\binpar{\lift{x}{\prefix{u}{v}{(\binpar{D(x)}{P})}}}{D(x)} \nonumber
\end{eqnarray}

\begin{remark}
  Note that the lazier definition still does not deal with summation
  or mixed summation (i.e. sums over input and output). The reader is
  invited to construct definitions of replication that deal with these
  features. 

  Further, the definitions are parameterized in a name, $x$. Can you,
  gentle reader, make a definition that eliminates this parameter and
  guarantees no accidental interaction between the replication
  machinery and the process being replicated -- i.e. no accidental
  sharing of names used by the process to get its work done and the
  name(s) used by the replication to effect copying. This latter
  revision of the definition of replication is crucial to obtaining
  the expected identity $!!P \sim !P$.
\end{remark}

\begin{remark}\label{rem:paradoxical_combinator}
  The reader familiar with the lambda calculus will have noticed the
  similarity between $D$ and the paradoxical combinator.

  [Ed. note: the existence of this seems to suggest we have to be more
  restrictive on the set of processes and names we admit if we are to
  support no-cloning.]
\end{remark}

\subsubsection{Bisimulation}

The computational dynamics gives rise to another kind of equivalence,
the equivalence of computational behavior. As previously mentioned
this is typically captured \emph{via} some form of bisimulation.

% The notion we use in this paper is weak barbed bisimulation
% \cite{milner91polyadicpi}.

The notion we use in this paper is derived from weak barbed
bisimulation \cite{milner91polyadicpi}. 

\begin{definition}
An \emph{observation relation}, $\downarrow_{\mathcal N}$, over a set
of names, $\mathcal N$, is the smallest relation satisfying the rules
below.

\infrule[Out-barb]{y \in {\mathcal N}, \; x \nameeq y}
		  {\outputp{x}{v} \downarrow_{\mathcal N} x}
\infrule[Par-barb]{\mbox{$P\downarrow_{\mathcal N} x$ or $Q\downarrow_{\mathcal N} x$}}
		  {\binpar{P}{Q} \downarrow_{\mathcal N} x}

We write $P \Downarrow_{\mathcal N} x$ if there is $Q$ such that 
$P \wred Q$ and $Q \downarrow_{\mathcal N} x$.
\end{definition}

\begin{definition}
%\label{def.bbisim}
An  ${\mathcal N}$-\emph{barbed bisimulation} over a set of names, ${\mathcal N}$, is a symmetric binary relation 
${\mathcal S}_{\mathcal N}$ between agents such that $P\rel{S}_{\mathcal N}Q$ implies:
\begin{enumerate}
\item If $P \red P'$ then $Q \wred Q'$ and $P'\rel{S}_{\mathcal N} Q'$.
\item If $P\downarrow_{\mathcal N} x$, then $Q\Downarrow_{\mathcal N} x$.
\end{enumerate}
$P$ is ${\mathcal N}$-barbed bisimilar to $Q$, written
$P \wbbisim_{\mathcal N} Q$, if $P \rel{S}_{\mathcal N} Q$ for some ${\mathcal N}$-barbed bisimulation ${\mathcal S}_{\mathcal N}$.
\end{definition}

$\mathcal{R} \subseteq \pi \times \pi$

$P \mathcal{R} Q => \forall P'. P \red P' \Rightarrow \exists Q'. Q \red Q', P' \mathcal{R} Q'$

$P \vdash x \Rightarrow Q \vdash x$

\begin{mathpar}
  \inferrule*[lab=Out-barb]{x \nameeq y}{{y}!\langle{Q}\rangle \vdash x}
  \and
  \inferrule*[lab=Par-barb]{\mbox{$P\vdash x$ or $Q\vdash x$}}{\binpar{P}{Q} \vdash x}
\end{mathpar}

\subsubsection{Contexts}

One of the principle advantages of computational calculi like the
$\pi$-calculus is a well-defined notion of context,
contextual-equivalence and a correlation between
contextual-equivalence and notions of bisimulation. The notion of
context allows the decomposition of a process into (sub-)process and
its syntactic environment, its context. Thus, a context may be
thought of as a process with a ``hole'' (written $\Box$) in it. The
application of a context $M$ to a process $P$, written $M[P]$, is
tantamount to filling the hole in $M$ with $P$. In this paper we do
not need the full weight of this theory, but do make use of the notion
of context in the proof the main theorem. 

\begin{mathpar}
  \inferrule* [lab=summation] {} {{M_{M},M_{N}} \bc \Box \;|\; x.M_{A} \;|\; M_{M}+M_{N}}
  \and
  \inferrule* [lab=agent] {} {{M_{A}} \bc (\vec{x})M_{P} \;| \; \clift{P_0,\ldots,M_{P},\ldots,P_N}}
  \and \\
  \inferrule* [lab=process] {} {{M_{P}} \bc M_{N} \;| \;P|M_{P} }
\end{mathpar} 

\begin{mathpar}
  \inferrule* [lab=sychronization] {} {M_{N} \bc \Box \;|\; x?M_{F} \;|\; x!M_{C}}
  \and
  \inferrule* [lab=abstraction] {} {{M_{F}} \bc (x)M_{P} }
  \and
  \inferrule* [lab=concretion] {} {{M_{C}} \bc \langle M_{P} \rangle }
  \and \\
  \inferrule* [lab=process] {} {{M_{P}} \bc M_{N} \;| \;P|M_{P} }
\end{mathpar}

\begin{definition}[contextual application] Given a context $M$, and
  process $P$, we define the \emph{contextual application}, $M[P] :=
  M\{P/\Box\}$. That is, the contextual application of M to P is the
  substitution of $P$ for $\Box$ in $M$.
\end{definition}

$\meaningof{-} : L \to \mathcal{P}(\pi)$

\begin{mathpar}
  \inferrule* [lab=collection] {} {\meaningof{true} = \pi, \and \meaningof{~E} = \pi \setminus \meaningof{E}, \and \meaningof{E_{1} \& E_{2}} = \meaningof{E_{1}} \cap \meaningof{E_{2}}}
\end{mathpar}

\begin{mathpar}
  \inferrule* [lab=structure] {} {\meaningof{0} = \{ P \in \pi | P \equiv 0 \}, \and \\ \meaningof{E_1 | E_2} = \{ P \in \pi | P \equiv P_{1} | P_{2}, P_{1} \in \meaningof{E_{1}}, P_{2} \in \meaningof{E_2}\} }
\end{mathpar}

\begin{mathpar}
 \inferrule* [lab=behavior] {} {\meaningof{\langle a?b \rangle E} = \{ P \in \pi | P \equiv Q | u?(y)P', \\ \and \\\\ \and \\ \;\;\; u \in \meaningof{a}, \forall z.P'\{z/y\} \in \meaningof{E\{z/b\}}\}, \and \\ \meaningof{a!E} = \{ P \in \pi | P \equiv Q | x!\langle P' \rangle, x \in \meaningof{a} P' \in \meaningof{E}\} }
\end{mathpar}

\begin{mathpar}
 \inferrule* [lab=nominal] {} {\meaningof{\quotep{E}} = \{ \quotep{P} \in \quotep{\pi} | P \in \meaningof{E} \}, \and \meaningof{\quotep{P}} = \{ \quotep{Q} \in \quotep{\pi} | P \equiv Q \} \and \\ \meaningof{@\quotep{E}} = \{ P \in \pi | P \equiv @x, x \in \meaningof{E} \}}
\end{mathpar}

\begin{eqnarray*}
  \\
  \meaningof{-} : TS \to ST
\end{eqnarray*}

\begin{eqnarray*}
  \\
  L : TS \to ST
\end{eqnarray*}

\begin{eqnarray*}
  \\
  P \models E \iff P \in \meaningof{E}
\end{eqnarray*}

\begin{eqnarray*}
  P \approx_{L} Q \iff \forall E \in L. P \models E \iff Q \models E
\end{eqnarray*}

\begin{eqnarray*}
  P \approx_{K} Q
\end{eqnarray*}

\begin{eqnarray*}
  P \approx Q
\end{eqnarray*}

$\approx_{K} = \approx = \approx_{L}$

\subsubsection{Contextual duality}

Note that contexts extend the quotation operation to a family of
operations from processes to names. Given a context, $M$, we can
define a \emph{nominal context}, $\quotep{M}$ by $\quotep{M}[P] :=
\quotep{M[P]}$. To foreshadow what is to come we observe that these
operations enjoy a duality with processes very much like the duality
between vectors and maps from vectors to scalars.

Further, because the calculus is essentially higher-order, we have a
correspondence between contexts and processes. More specifically,
given a name $x$ and a context $M$ we can construct $M^{*}_{x}$ such
that 

\begin{mathpar}
  M^{*}_{x} | \lift{x}{P} \red M[P]
\end{mathpar}

namely,

\begin{mathpar}
  M^{*}_{x} := x?(u).M[\dropn{u}]
\end{mathpar}

The dependence of $M^{*}_{x}$ on a name makes it an abstraction, 

\begin{mathpar}
  M^{*} := (x)x?(u).M[\dropn{u}]
\end{mathpar}

\subsection{Additional notation}

It will sometimes be convenient to denote the process a name
quotes. We already have the notation $x = \quotep{P}$, but it will be
convenient to introduce an alternate notation, $\procn{x}$, when we
want to emphasize the connection to the use of the name. Note that, by
virtue of name equivalence, $\quotep{\procn{x}} \nameeq x$; so, the
notation is consistent with previous definitions.

Further, because names have structure it is possible to effect
substitutions on the basis of that structure. This means we need to
upgrade our notation for substitutions, which we accomplish by
adapting comprehension notation. Thus,

\begin{mathpar}
  P\{ y / x : x \in S \}
\end{mathpar}

is interpreted to mean the process derived from P by replacing (in a
capture-avoiding manner) each occurrence of $x$ in $S$ by $y$. For example,

\begin{mathpar}
  P\{ \quotep{\procn{x}|\procn{x}} / x : x \in \freenames{P} \}
\end{mathpar}

will replace each (occurrence) of a free name $x$ in $P$ by
$\quotep{\procn{x}|\procn{x}}$.

Also, we will avail ourselves of the notation $x^{L}$ and $x^{R}$ to
denote injections of a name into disjoint copies of the name
space. There are numerous ways to accomplish this. One example can be
found in \cite{MeredithR05}. This notation overloads to vectors of
names: $\vec{x}^{\pi} := (x_{i}^{\pi} \; : \; 0 \leq i < |\vec{x}| )$ where $\pi \in \{L,R\}$.

We also use $P^{\Box} := P|\Box$.

In \cite{MeredithR05} an interpretation of the new operator is
given. It turns out that there are several possible interpretations
all enjoying the requisite algebraic properties of the operator (see
\cite{milner91polyadicpi}). We will therefore make liberal use of
$(\nu\; \vec{x})P$.

% subsection the_syntax_and_semantics_of_the_notation_system (end)   

\input{qm2pi.qmops} 

\input{qm2pi.sterngerlach} 

\input{qm2pi.metric} 

% section concurrent_process_calculi (end)

%\input{qm2pi.proofsketch}

% section proof sketch (end)

%\input{qm2pi.slviaknots} 

% section spatial logic via knots (end)

\input{qm2pi.conclusion}

% section conclusion (end)

%\input{qm2pi.dtcodes} 

% section wiring algorithm (end)

\input{qm2pi.ack} 

% section acknowledgments (end)

\newpage


\bibliographystyle{plain}   
\bibliography{../../biblios/main.bib}

\input{qm2pi.rhodetails}

\end{document}

 

%\documentclass[12pt]{llncs}
%\documentclass{jktr}

\usepackage[pdftex]{hyperref}                   
\usepackage {listings}
\usepackage {mathpartir}
\usepackage{bcprules}
%\usepackage{listings}
                       
\usepackage{graphicx} 
%\usepackage[margins=2.5cm,nohead,nofoot]{geometry}
%\usepackage{geometry}
\usepackage{amsfonts}
\usepackage{amstext}
\usepackage{latexsym}
\usepackage{amssymb}
\usepackage{color}


%\include{myPreamble}
\include{qm2pi.local} 

%\ifpdf
%\usepackage[pdftex]{graphicx}
%\else
%\usepackage{graphicx}
%\fi

 % \ifpdf
%  \usepackage{pdfsync}
%  \if


%\title{Brief Article}
%\author{David F. Snyder}
%\author{L.G. Meredith}

%\address{Dept. of Math., Texas State University--San Marcos, San Marcos, TX 78666}
       
\pagestyle{empty}


\begin{document}

\lstset{language=[Objective]Caml,frame=shadowbox}

\input{qm2pi.front}

% section front matter (end)

\input{qm2pi.intro} 
 
% section introduction (end)

% \input{qm2pi.knotations} 

% section notation (end)

\input{qm2pi.process.calculi} 

% section concurrent_process_calculi_and_spatial_logics_ (end)
    
%\input{qm2pi.knots2pi} 

%\input{qm2pi.trefoil} 

%\input{qm2pi.mainthm} 

% subsection basic_interpretation (end)

%\input{qm2pi.rho.presentation} 
\subsection{The syntax and semantics of the notation system}\label{sub:the_syntax_and_semantics_of_the_notation_system} % (fold)

We now summarize a technical presentation of the calculus that
embodies our theory of dynamics. The typical presentation of such a
calculus follows the style of giving generators and relations on
them. The grammar, below, describing term constructors, freely
generates the set of processes, $\Proc$. This set is then quotiented
by a relation known as structural congruence and it is over this set
that the notion of dynamics is expressed. This presentation is
essentially that of \cite{MeredithR05} with the addition of
polyadicity and summation. For readability we have relegated some of
the technical subtleties to an appendix.

\subsubsection{Process grammar}\label{subsub:process_grammar}

\begin{mathpar}
  \inferrule* [lab=synchronization] {} {{M} \bc \pzero \;|\; x?F \;|\; x!C }
  \and
  \inferrule* [lab=abstraction] {} {{F} \bc (x)P}
  \and
  \inferrule* [lab=concretion] {} {{C} \bc \langle Q \rangle}
  \and
  \inferrule* [lab=process] {} {{P,Q} \bc M \;| \;P|Q \;|\; @{x}}
  \and
  \inferrule* [lab=name] {} {{x} \bc \quotep{P}}
\end{mathpar} 

Note that $\vec{x}$ (resp. $\vec{P}$) denotes a vector of names
(resp. processes) of length $|\vec{x}|$ (resp. $|\vec{P}|$). We adopt
the following useful abbreviations.

\begin{mathpar}
   x?(\vec{y}).P := x.(\vec{y})P \and  x\clift{\vec{P}} := x.\clift{\vec{P}}
   \and x!(y) := \lift{x}{\dropn{y}}
   \and \Pi_{i=0}^{n-1}P_i := P_0 | \ldots | P_{n-1}
\end{mathpar}

\subsubsection{Structural congruence}

\paragraph{Free and bound names and alpha-equivalence.} At the
core of structural equivalence is alpha-equivalence which identifies
process that are the same up to a change of variable. Formally, we
recognize the distinction between free and bound names. The free names
of a process, $\freenames{P}$, may be calculated recursively as
follows:

\begin{mathpar}
\freenames{\pzero} := \emptyset
  \and \\
  \freenames{x?(y).P} := \{ x \} \cup (\freenames{P} \setminus \{ y \})
  \and 
  \freenames{x!\langle P \rangle} := \{ x \} \cup \{ P \} 
  \and \\
  \freenames{P|Q} := \freenames{P} \cup \freenames{Q}
  \and \\
  \freenames{@{x}} := \{ x \}
\end{mathpar}

$\pi$
$\quotep{\pi}$

$\freenames{-} : \pi \to \mathcal{P}(\quotep{\pi})$

\begin{eqnarray*}
  \freenames{\pzero} & := & \emptyset \\
  \freenames{x?(y).P} & := & \{ x \} \cup (\freenames{P} \setminus \{ y \}) \\
  \freenames{x!\langle P \rangle} & := & \{ x \} \cup \{ P \} \\
  \freenames{P|Q} & := & \freenames{P} \cup \freenames{Q} \\
  \freenames{\dropn{x}} & := & \{ x \}
\end{eqnarray*}

The bound names of a process, $\boundnames{P}$, are those names occurring in $P$
that are not free. For example, in $x?(y).0$, the name $x$ is free, while $y$ is bound.

\begin{mathpar}
  \inferrule* [lab=monoidal-laws] {} { P|Q \equiv Q|P \and P|0 \equiv P \and P|(Q|R) \equiv (P|Q)|R }
\end{mathpar}

\begin{mathpar}
  \inferrule* [lab=alpha-equivalence] {} { (x)P \equiv (y)P\{y/x\} \and y \not\in \freenames{P} }
\end{mathpar}

\begin{definition}
Then two processes, $P,Q$, are alpha-equivalent if $P = Q\{\vec{y}/\vec{x}\}$ for
some $\vec{x} \in \boundnames{Q},\vec{y} \in \boundnames{P}$, where $Q\{\vec{y}/\vec{x}\}$
denotes the capture-avoiding substitution of $\vec{y}$ for $\vec{x}$ in $Q$.
\end{definition}

\begin{definition}
  The {\em structural congruence} \cite{SangiorgiWalker} , $\equiv$,
  between processes is the least congruence containing
  alpha-equivalence, satisfying the abelian monoid laws
  (associativity, commutativity and $\pzero$ as identity) for parallel
  composition $|$ and for summation $+$.
\end{definition}

\subsection{Name equivalence}

We take name equivalence, written $\nameeq$, to be the smallest
equivalence relation generated by the following rules.

\begin{mathpar}
\inferrule*[lab=Quote-drop]
{ }
{ \quotep{@{x}} \nameeq x }

\inferrule*[lab=Struct-equiv]
{ P \scong Q }
{ \quotep{P} \nameeq \quotep{Q} }
\end{mathpar}

The astute reader will have noticed that the mutual recursion of names
and processes imposes a mutual recursion on alpha-equivalence and
structural equivalence via name-equivalence. Fortunately, all of this
works out pleasantly and we may calculate in the natural way, free of
concern. The reader interested in the details is referred to the
appendix \ref{appendix:rho_details}.

\subsection{Substitution}

We use $\Proc$ for the set of processes, $\QProc$ for the set of
names, and $\id{\{}\vec{y} / \vec{x} \id{\}}$ to denote partial maps,
$s : \QProc \rightarrow \QProc$. A map, $s$ lifts, uniquely, to a map
on process terms, $\widehat{s} : \Proc \rightarrow \Proc$ by the
following equations.

\begin{mathpar}
  (0) \psubstp{Q}{P} := 0 \\
  (R \juxtap S) \psubstp{Q}{P}
  :=    
  (R)\psubstp{Q}{P} \juxtap (S) \psubstp{Q}{P} \\
  (x?(y).R) \psubstp{Q}{P}    
  :=    
  (x)\substp{Q}{P} (z)\concat( (R \psubstn{z}{y}) \psubstp{Q}{P} ) \\
  (\lift{x}{R}) \psubstp{Q}{P}  
  :=
  \lift{(x)\substp{Q}{P}}{ R \psubstp{Q}{P} } \\
%   (\dropn{x})  \psubstp{Q}{P}       
%   := 
%   \left\{ 
%     \begin{array}{ccc} 
%       \dropn{\quotep{Q}} & & x \nameeq \quotep{P} \\
%       \dropn{x} & & otherwise \\
%     \end{array}
%   \right. 
  (\dropn{x})  \psubstp{Q}{P}       
  := 
  \left\{ 
    \begin{array}{ccc} 
      Q & & x \nameeq \quotep{P} \\
      \dropn{x} & & otherwise \\
    \end{array}
  \right.
\end{mathpar}
 

where

\begin{eqnarray}
  (x)\id{\{} \lpquote Q \rpquote / \lpquote P \rpquote \id{\}}            = 
  \left\{ 
    \begin{array}{ccc}
      \lpquote Q \rpquote & & x \nameeq \lpquote P \rpquote \\
      x & & otherwise \\
    \end{array}
  \right. \nonumber
\end{eqnarray}

and $z$ is chosen distinct from $\quotep{P}$, $\quotep{Q}$, the free
names in $Q$, and all the names in $R$. Our $\alpha$-equivalence will
be built in the standard way from this substitution.

\begin{remark}\label{rem:no_self_referential_names}
  One consequence of these definitions is that $\forall P. \quotep{P}
  \not\in \freenames{P}$.
\end{remark}

\subsection{ Dynamic quote: an example }

Anticipating something of what's to come, consider applying the
substitution, $\widehat{\id{\{}u / z \id{\}}}$, to the following pair
of processes, $\lift{w}{y!(z)}$ and $w[ \lpquote y!(z) \rpquote ]$.

\begin{eqnarray}
	\lift{w}{y!(z)}\widehat{\id{\{}u / z \id{\}}}
		& = &
		\lift{w}{y!(u)} \nonumber\\
	w[ \lpquote y!(z) \rpquote ] \widehat{ \id{\{}u / z \id{\}} }
		& = &
		w[ \lpquote y!(z) \rpquote ] \nonumber
\end{eqnarray}

Because the body of the process between quotes is impervious to
substitution, we get radically different answers. In fact, by
examining the first process in an input context,
e.g. $x?(z).\lift{w}{y!(z)}$, we see that the process under the lift
operator may be shaped by prefixed inputs binding a name inside it. In
this sense, the lift operator will be seen as a way to dynamically
construct processes before reifying them as names.

Finally equipped with these standard features we can present the
dynamics of the calculus.

\subsubsection{Operational semantics} 

Finally, we introduce the computational dynamics. What marks these
algebras as distinct from other more traditionally studied algebraic
structures, e.g. vector spaces or polynomial rings, is the manner in
which dynamics is captured. In traditional structures, dynamics is typically
expressed through morphisms between such structures, as in linear maps
between vector spaces or morphisms between rings. In algebras
associated with the semantics of computation, the dynamics is
expressed as part of the algebraic structure itself, through a
reduction reduction relation typically denoted by $\red$. Below, we
give a recursive presentation of this relation for the calculus used
in the encoding.

$\red \subseteq \pi \times \pi$
$\red : \pi \to \mathcal{P}(\pi)$

\begin{mathpar}
  \inferrule* [lab=Comm] { \textsf{match}( x_{src}, x_{trgt} ) } { x_{trgt}?(y)P \; | \; x_{src}!\langle {Q} \rangle \red P\{\quotep{Q}/y}\} }
  \and \\
  \inferrule* [lab=Par] {{P} \red {P}'} {{{P} | {Q}} \red {{P}' | {Q}}}
  \and
  \inferrule* [lab=Equiv]{{{P} \scong {P}'} \andalso {{P}' \red {Q}'} \andalso {{Q}' \scong {Q}}}{{P} \red {Q}}
\end{mathpar}

\begin{eqnarray*}
  match_{\equiv} (\quotep{P},\quotep{Q}) & := & P \equiv Q \\
  match_{\dagger}(\quotep{P},\quotep{Q}) & := & \forall R. P|Q \red^{*} R => R \red^{*} 0 \\
  match_{K}(\quotep{P},\quotep{Q}) & := & K \mbox{ for some context } K
\end{eqnarray*}

$u?(x)P | u!\langle Q \rangle \red P\{\quotep{Q}/x\}$

%We write $\wred$ for $\red^*$, and $P\red$ if $\exists Q $ such that $ P \red Q$.
We write $P\red$ if $\exists Q $ such that $ P \red Q$ and $P\not\red$, otherwise.

\section{Replication}

As mentioned before, it is known that replication (and hence
recursion) can be implemented in a higher-order process algebra
\cite{SangiorgiWalker}. As our first example of calculation with the
machinery thus far presented we give the construction explicitly in
the {\rhoc}.

\begin{eqnarray}
	D_{x} & := & \prefix{x}{y}{(\binpar{\outputp{x}{y}}{@{y}})} \nonumber\\
	\bangp_{x}{P} & := & \binpar{{x}!\langle{\binpar{D_{x}}{P}}\rangle}{D_{x}} \nonumber
\end{eqnarray}

\begin{eqnarray}
	\bangp_{x}{P} & & \nonumber\\
	=
	& {x}!\langle{(\prefix{x}{y}{(\outputp{x}{y} | @{y})) | P}}\rangle 
	      | \prefix{x}{y}{(\outputp{x}{y} | @{y})} & \nonumber\\
	\red
	& (\outputp{x}{y} | @{y})\substn{\quotep{(\prefix{x}{y}{(@{y} | \outputp{x}{y})) | P}}}{y} & \nonumber\\
	=
	& \outputp{x}{\quotep{(\prefix{x}{y}{(\outputp{x}{y} | @{y})) | P}}}
	  | {(\prefix{x}{y}{(\outputp{x}{y} | @{y})) | P}} & \nonumber\\
	\red
	& \ldots & \nonumber\\
	\red^*
	& P | P | \ldots & \nonumber
\end{eqnarray}

Of course, this encoding, as an implementation, runs away, unfolding
$\bangp{P}$ eagerly. A lazier and more implementable replication
operator, restricted to input-guarded processes, may be obtained as follows.

\begin{eqnarray}
\bangp{\prefix{u}{v}{P}} 
	:= 
	\binpar{\lift{x}{\prefix{u}{v}{(\binpar{D(x)}{P})}}}{D(x)} \nonumber
\end{eqnarray}

\begin{remark}
  Note that the lazier definition still does not deal with summation
  or mixed summation (i.e. sums over input and output). The reader is
  invited to construct definitions of replication that deal with these
  features. 

  Further, the definitions are parameterized in a name, $x$. Can you,
  gentle reader, make a definition that eliminates this parameter and
  guarantees no accidental interaction between the replication
  machinery and the process being replicated -- i.e. no accidental
  sharing of names used by the process to get its work done and the
  name(s) used by the replication to effect copying. This latter
  revision of the definition of replication is crucial to obtaining
  the expected identity $!!P \sim !P$.
\end{remark}

\begin{remark}\label{rem:paradoxical_combinator}
  The reader familiar with the lambda calculus will have noticed the
  similarity between $D$ and the paradoxical combinator.

  [Ed. note: the existence of this seems to suggest we have to be more
  restrictive on the set of processes and names we admit if we are to
  support no-cloning.]
\end{remark}

\subsubsection{Bisimulation}

The computational dynamics gives rise to another kind of equivalence,
the equivalence of computational behavior. As previously mentioned
this is typically captured \emph{via} some form of bisimulation.

% The notion we use in this paper is weak barbed bisimulation
% \cite{milner91polyadicpi}.

The notion we use in this paper is derived from weak barbed
bisimulation \cite{milner91polyadicpi}. 

\begin{definition}
An \emph{observation relation}, $\downarrow_{\mathcal N}$, over a set
of names, $\mathcal N$, is the smallest relation satisfying the rules
below.

\infrule[Out-barb]{y \in {\mathcal N}, \; x \nameeq y}
		  {\outputp{x}{v} \downarrow_{\mathcal N} x}
\infrule[Par-barb]{\mbox{$P\downarrow_{\mathcal N} x$ or $Q\downarrow_{\mathcal N} x$}}
		  {\binpar{P}{Q} \downarrow_{\mathcal N} x}

We write $P \Downarrow_{\mathcal N} x$ if there is $Q$ such that 
$P \wred Q$ and $Q \downarrow_{\mathcal N} x$.
\end{definition}

\begin{definition}
%\label{def.bbisim}
An  ${\mathcal N}$-\emph{barbed bisimulation} over a set of names, ${\mathcal N}$, is a symmetric binary relation 
${\mathcal S}_{\mathcal N}$ between agents such that $P\rel{S}_{\mathcal N}Q$ implies:
\begin{enumerate}
\item If $P \red P'$ then $Q \wred Q'$ and $P'\rel{S}_{\mathcal N} Q'$.
\item If $P\downarrow_{\mathcal N} x$, then $Q\Downarrow_{\mathcal N} x$.
\end{enumerate}
$P$ is ${\mathcal N}$-barbed bisimilar to $Q$, written
$P \wbbisim_{\mathcal N} Q$, if $P \rel{S}_{\mathcal N} Q$ for some ${\mathcal N}$-barbed bisimulation ${\mathcal S}_{\mathcal N}$.
\end{definition}

$\mathcal{R} \subseteq \pi \times \pi$

$P \mathcal{R} Q => \forall P'. P \red P' \Rightarrow \exists Q'. Q \red Q', P' \mathcal{R} Q'$

$P \vdash x \Rightarrow Q \vdash x$

\begin{mathpar}
  \inferrule*[lab=Out-barb]{x \nameeq y}{{y}!\langle{Q}\rangle \vdash x}
  \and
  \inferrule*[lab=Par-barb]{\mbox{$P\vdash x$ or $Q\vdash x$}}{\binpar{P}{Q} \vdash x}
\end{mathpar}

\subsubsection{Contexts}

One of the principle advantages of computational calculi like the
$\pi$-calculus is a well-defined notion of context,
contextual-equivalence and a correlation between
contextual-equivalence and notions of bisimulation. The notion of
context allows the decomposition of a process into (sub-)process and
its syntactic environment, its context. Thus, a context may be
thought of as a process with a ``hole'' (written $\Box$) in it. The
application of a context $M$ to a process $P$, written $M[P]$, is
tantamount to filling the hole in $M$ with $P$. In this paper we do
not need the full weight of this theory, but do make use of the notion
of context in the proof the main theorem. 

\begin{mathpar}
  \inferrule* [lab=summation] {} {{M_{M},M_{N}} \bc \Box \;|\; x.M_{A} \;|\; M_{M}+M_{N}}
  \and
  \inferrule* [lab=agent] {} {{M_{A}} \bc (\vec{x})M_{P} \;| \; \clift{P_0,\ldots,M_{P},\ldots,P_N}}
  \and \\
  \inferrule* [lab=process] {} {{M_{P}} \bc M_{N} \;| \;P|M_{P} }
\end{mathpar} 

\begin{mathpar}
  \inferrule* [lab=sychronization] {} {M_{N} \bc \Box \;|\; x?M_{F} \;|\; x!M_{C}}
  \and
  \inferrule* [lab=abstraction] {} {{M_{F}} \bc (x)M_{P} }
  \and
  \inferrule* [lab=concretion] {} {{M_{C}} \bc \langle M_{P} \rangle }
  \and \\
  \inferrule* [lab=process] {} {{M_{P}} \bc M_{N} \;| \;P|M_{P} }
\end{mathpar}

\begin{definition}[contextual application] Given a context $M$, and
  process $P$, we define the \emph{contextual application}, $M[P] :=
  M\{P/\Box\}$. That is, the contextual application of M to P is the
  substitution of $P$ for $\Box$ in $M$.
\end{definition}

$\meaningof{-} : L \to \mathcal{P}(\pi)$

\begin{mathpar}
  \inferrule* [lab=collection] {} {\meaningof{true} = \pi, \and \meaningof{~E} = \pi \setminus \meaningof{E}, \and \meaningof{E_{1} \& E_{2}} = \meaningof{E_{1}} \cap \meaningof{E_{2}}}
\end{mathpar}

\begin{mathpar}
  \inferrule* [lab=structure] {} {\meaningof{0} = \{ P \in \pi | P \equiv 0 \}, \and \\ \meaningof{E_1 | E_2} = \{ P \in \pi | P \equiv P_{1} | P_{2}, P_{1} \in \meaningof{E_{1}}, P_{2} \in \meaningof{E_2}\} }
\end{mathpar}

\begin{mathpar}
 \inferrule* [lab=behavior] {} {\meaningof{\langle a?b \rangle E} = \{ P \in \pi | P \equiv Q | u?(y)P', \\ \and \\\\ \and \\ \;\;\; u \in \meaningof{a}, \forall z.P'\{z/y\} \in \meaningof{E\{z/b\}}\}, \and \\ \meaningof{a!E} = \{ P \in \pi | P \equiv Q | x!\langle P' \rangle, x \in \meaningof{a} P' \in \meaningof{E}\} }
\end{mathpar}

\begin{mathpar}
 \inferrule* [lab=nominal] {} {\meaningof{\quotep{E}} = \{ \quotep{P} \in \quotep{\pi} | P \in \meaningof{E} \}, \and \meaningof{\quotep{P}} = \{ \quotep{Q} \in \quotep{\pi} | P \equiv Q \} \and \\ \meaningof{@\quotep{E}} = \{ P \in \pi | P \equiv @x, x \in \meaningof{E} \}}
\end{mathpar}

\begin{eqnarray*}
  \\
  \meaningof{-} : TS \to ST
\end{eqnarray*}

\begin{eqnarray*}
  \\
  L : TS \to ST
\end{eqnarray*}

\begin{eqnarray*}
  \\
  P \models E \iff P \in \meaningof{E}
\end{eqnarray*}

\begin{eqnarray*}
  P \approx_{L} Q \iff \forall E \in L. P \models E \iff Q \models E
\end{eqnarray*}

\begin{eqnarray*}
  P \approx_{K} Q
\end{eqnarray*}

\begin{eqnarray*}
  P \approx Q
\end{eqnarray*}

$\approx_{K} = \approx = \approx_{L}$

\subsubsection{Contextual duality}

Note that contexts extend the quotation operation to a family of
operations from processes to names. Given a context, $M$, we can
define a \emph{nominal context}, $\quotep{M}$ by $\quotep{M}[P] :=
\quotep{M[P]}$. To foreshadow what is to come we observe that these
operations enjoy a duality with processes very much like the duality
between vectors and maps from vectors to scalars.

Further, because the calculus is essentially higher-order, we have a
correspondence between contexts and processes. More specifically,
given a name $x$ and a context $M$ we can construct $M^{*}_{x}$ such
that 

\begin{mathpar}
  M^{*}_{x} | \lift{x}{P} \red M[P]
\end{mathpar}

namely,

\begin{mathpar}
  M^{*}_{x} := x?(u).M[\dropn{u}]
\end{mathpar}

The dependence of $M^{*}_{x}$ on a name makes it an abstraction, 

\begin{mathpar}
  M^{*} := (x)x?(u).M[\dropn{u}]
\end{mathpar}

\subsection{Additional notation}

It will sometimes be convenient to denote the process a name
quotes. We already have the notation $x = \quotep{P}$, but it will be
convenient to introduce an alternate notation, $\procn{x}$, when we
want to emphasize the connection to the use of the name. Note that, by
virtue of name equivalence, $\quotep{\procn{x}} \nameeq x$; so, the
notation is consistent with previous definitions.

Further, because names have structure it is possible to effect
substitutions on the basis of that structure. This means we need to
upgrade our notation for substitutions, which we accomplish by
adapting comprehension notation. Thus,

\begin{mathpar}
  P\{ y / x : x \in S \}
\end{mathpar}

is interpreted to mean the process derived from P by replacing (in a
capture-avoiding manner) each occurrence of $x$ in $S$ by $y$. For example,

\begin{mathpar}
  P\{ \quotep{\procn{x}|\procn{x}} / x : x \in \freenames{P} \}
\end{mathpar}

will replace each (occurrence) of a free name $x$ in $P$ by
$\quotep{\procn{x}|\procn{x}}$.

Also, we will avail ourselves of the notation $x^{L}$ and $x^{R}$ to
denote injections of a name into disjoint copies of the name
space. There are numerous ways to accomplish this. One example can be
found in \cite{MeredithR05}. This notation overloads to vectors of
names: $\vec{x}^{\pi} := (x_{i}^{\pi} \; : \; 0 \leq i < |\vec{x}| )$ where $\pi \in \{L,R\}$.

We also use $P^{\Box} := P|\Box$.

In \cite{MeredithR05} an interpretation of the new operator is
given. It turns out that there are several possible interpretations
all enjoying the requisite algebraic properties of the operator (see
\cite{milner91polyadicpi}). We will therefore make liberal use of
$(\nu\; \vec{x})P$.

% subsection the_syntax_and_semantics_of_the_notation_system (end)   

\input{qm2pi.qmops} 

\input{qm2pi.sterngerlach} 

\input{qm2pi.metric} 

% section concurrent_process_calculi (end)

%\input{qm2pi.proofsketch}

% section proof sketch (end)

%\input{qm2pi.slviaknots} 

% section spatial logic via knots (end)

\input{qm2pi.conclusion}

% section conclusion (end)

%\input{qm2pi.dtcodes} 

% section wiring algorithm (end)

\input{qm2pi.ack} 

% section acknowledgments (end)

\newpage


\bibliographystyle{plain}   
\bibliography{../../biblios/main.bib}

\input{qm2pi.rhodetails}

\end{document}

 

% subsection basic_interpretation (end)

%\input{qm2pi.rho.presentation} 
\subsection{The syntax and semantics of the notation system}\label{sub:the_syntax_and_semantics_of_the_notation_system} % (fold)

We now summarize a technical presentation of the calculus that
embodies our theory of dynamics. The typical presentation of such a
calculus follows the style of giving generators and relations on
them. The grammar, below, describing term constructors, freely
generates the set of processes, $\Proc$. This set is then quotiented
by a relation known as structural congruence and it is over this set
that the notion of dynamics is expressed. This presentation is
essentially that of \cite{MeredithR05} with the addition of
polyadicity and summation. For readability we have relegated some of
the technical subtleties to an appendix.

\subsubsection{Process grammar}\label{subsub:process_grammar}

\begin{mathpar}
  \inferrule* [lab=synchronization] {} {{M} \bc \pzero \;|\; x?F \;|\; x!C }
  \and
  \inferrule* [lab=abstraction] {} {{F} \bc (x)P}
  \and
  \inferrule* [lab=concretion] {} {{C} \bc \langle Q \rangle}
  \and
  \inferrule* [lab=process] {} {{P,Q} \bc M \;| \;P|Q \;|\; @{x}}
  \and
  \inferrule* [lab=name] {} {{x} \bc \quotep{P}}
\end{mathpar} 

Note that $\vec{x}$ (resp. $\vec{P}$) denotes a vector of names
(resp. processes) of length $|\vec{x}|$ (resp. $|\vec{P}|$). We adopt
the following useful abbreviations.

\begin{mathpar}
   x?(\vec{y}).P := x.(\vec{y})P \and  x\clift{\vec{P}} := x.\clift{\vec{P}}
   \and x!(y) := \lift{x}{\dropn{y}}
   \and \Pi_{i=0}^{n-1}P_i := P_0 | \ldots | P_{n-1}
\end{mathpar}

\subsubsection{Structural congruence}

\paragraph{Free and bound names and alpha-equivalence.} At the
core of structural equivalence is alpha-equivalence which identifies
process that are the same up to a change of variable. Formally, we
recognize the distinction between free and bound names. The free names
of a process, $\freenames{P}$, may be calculated recursively as
follows:

\begin{mathpar}
\freenames{\pzero} := \emptyset
  \and \\
  \freenames{x?(y).P} := \{ x \} \cup (\freenames{P} \setminus \{ y \})
  \and 
  \freenames{x!\langle P \rangle} := \{ x \} \cup \{ P \} 
  \and \\
  \freenames{P|Q} := \freenames{P} \cup \freenames{Q}
  \and \\
  \freenames{@{x}} := \{ x \}
\end{mathpar}

$\pi$
$\quotep{\pi}$

$\freenames{-} : \pi \to \mathcal{P}(\quotep{\pi})$

\begin{eqnarray*}
  \freenames{\pzero} & := & \emptyset \\
  \freenames{x?(y).P} & := & \{ x \} \cup (\freenames{P} \setminus \{ y \}) \\
  \freenames{x!\langle P \rangle} & := & \{ x \} \cup \{ P \} \\
  \freenames{P|Q} & := & \freenames{P} \cup \freenames{Q} \\
  \freenames{\dropn{x}} & := & \{ x \}
\end{eqnarray*}

The bound names of a process, $\boundnames{P}$, are those names occurring in $P$
that are not free. For example, in $x?(y).0$, the name $x$ is free, while $y$ is bound.

\begin{mathpar}
  \inferrule* [lab=monoidal-laws] {} { P|Q \equiv Q|P \and P|0 \equiv P \and P|(Q|R) \equiv (P|Q)|R }
\end{mathpar}

\begin{mathpar}
  \inferrule* [lab=alpha-equivalence] {} { (x)P \equiv (y)P\{y/x\} \and y \not\in \freenames{P} }
\end{mathpar}

\begin{definition}
Then two processes, $P,Q$, are alpha-equivalent if $P = Q\{\vec{y}/\vec{x}\}$ for
some $\vec{x} \in \boundnames{Q},\vec{y} \in \boundnames{P}$, where $Q\{\vec{y}/\vec{x}\}$
denotes the capture-avoiding substitution of $\vec{y}$ for $\vec{x}$ in $Q$.
\end{definition}

\begin{definition}
  The {\em structural congruence} \cite{SangiorgiWalker} , $\equiv$,
  between processes is the least congruence containing
  alpha-equivalence, satisfying the abelian monoid laws
  (associativity, commutativity and $\pzero$ as identity) for parallel
  composition $|$ and for summation $+$.
\end{definition}

\subsection{Name equivalence}

We take name equivalence, written $\nameeq$, to be the smallest
equivalence relation generated by the following rules.

\begin{mathpar}
\inferrule*[lab=Quote-drop]
{ }
{ \quotep{@{x}} \nameeq x }

\inferrule*[lab=Struct-equiv]
{ P \scong Q }
{ \quotep{P} \nameeq \quotep{Q} }
\end{mathpar}

The astute reader will have noticed that the mutual recursion of names
and processes imposes a mutual recursion on alpha-equivalence and
structural equivalence via name-equivalence. Fortunately, all of this
works out pleasantly and we may calculate in the natural way, free of
concern. The reader interested in the details is referred to the
appendix \ref{appendix:rho_details}.

\subsection{Substitution}

We use $\Proc$ for the set of processes, $\QProc$ for the set of
names, and $\id{\{}\vec{y} / \vec{x} \id{\}}$ to denote partial maps,
$s : \QProc \rightarrow \QProc$. A map, $s$ lifts, uniquely, to a map
on process terms, $\widehat{s} : \Proc \rightarrow \Proc$ by the
following equations.

\begin{mathpar}
  (0) \psubstp{Q}{P} := 0 \\
  (R \juxtap S) \psubstp{Q}{P}
  :=    
  (R)\psubstp{Q}{P} \juxtap (S) \psubstp{Q}{P} \\
  (x?(y).R) \psubstp{Q}{P}    
  :=    
  (x)\substp{Q}{P} (z)\concat( (R \psubstn{z}{y}) \psubstp{Q}{P} ) \\
  (\lift{x}{R}) \psubstp{Q}{P}  
  :=
  \lift{(x)\substp{Q}{P}}{ R \psubstp{Q}{P} } \\
%   (\dropn{x})  \psubstp{Q}{P}       
%   := 
%   \left\{ 
%     \begin{array}{ccc} 
%       \dropn{\quotep{Q}} & & x \nameeq \quotep{P} \\
%       \dropn{x} & & otherwise \\
%     \end{array}
%   \right. 
  (\dropn{x})  \psubstp{Q}{P}       
  := 
  \left\{ 
    \begin{array}{ccc} 
      Q & & x \nameeq \quotep{P} \\
      \dropn{x} & & otherwise \\
    \end{array}
  \right.
\end{mathpar}
 

where

\begin{eqnarray}
  (x)\id{\{} \lpquote Q \rpquote / \lpquote P \rpquote \id{\}}            = 
  \left\{ 
    \begin{array}{ccc}
      \lpquote Q \rpquote & & x \nameeq \lpquote P \rpquote \\
      x & & otherwise \\
    \end{array}
  \right. \nonumber
\end{eqnarray}

and $z$ is chosen distinct from $\quotep{P}$, $\quotep{Q}$, the free
names in $Q$, and all the names in $R$. Our $\alpha$-equivalence will
be built in the standard way from this substitution.

\begin{remark}\label{rem:no_self_referential_names}
  One consequence of these definitions is that $\forall P. \quotep{P}
  \not\in \freenames{P}$.
\end{remark}

\subsection{ Dynamic quote: an example }

Anticipating something of what's to come, consider applying the
substitution, $\widehat{\id{\{}u / z \id{\}}}$, to the following pair
of processes, $\lift{w}{y!(z)}$ and $w[ \lpquote y!(z) \rpquote ]$.

\begin{eqnarray}
	\lift{w}{y!(z)}\widehat{\id{\{}u / z \id{\}}}
		& = &
		\lift{w}{y!(u)} \nonumber\\
	w[ \lpquote y!(z) \rpquote ] \widehat{ \id{\{}u / z \id{\}} }
		& = &
		w[ \lpquote y!(z) \rpquote ] \nonumber
\end{eqnarray}

Because the body of the process between quotes is impervious to
substitution, we get radically different answers. In fact, by
examining the first process in an input context,
e.g. $x?(z).\lift{w}{y!(z)}$, we see that the process under the lift
operator may be shaped by prefixed inputs binding a name inside it. In
this sense, the lift operator will be seen as a way to dynamically
construct processes before reifying them as names.

Finally equipped with these standard features we can present the
dynamics of the calculus.

\subsubsection{Operational semantics} 

Finally, we introduce the computational dynamics. What marks these
algebras as distinct from other more traditionally studied algebraic
structures, e.g. vector spaces or polynomial rings, is the manner in
which dynamics is captured. In traditional structures, dynamics is typically
expressed through morphisms between such structures, as in linear maps
between vector spaces or morphisms between rings. In algebras
associated with the semantics of computation, the dynamics is
expressed as part of the algebraic structure itself, through a
reduction reduction relation typically denoted by $\red$. Below, we
give a recursive presentation of this relation for the calculus used
in the encoding.

$\red \subseteq \pi \times \pi$
$\red : \pi \to \mathcal{P}(\pi)$

\begin{mathpar}
  \inferrule* [lab=Comm] { \textsf{match}( x_{src}, x_{trgt} ) } { x_{trgt}?(y)P \; | \; x_{src}!\langle {Q} \rangle \red P\{\quotep{Q}/y}\} }
  \and \\
  \inferrule* [lab=Par] {{P} \red {P}'} {{{P} | {Q}} \red {{P}' | {Q}}}
  \and
  \inferrule* [lab=Equiv]{{{P} \scong {P}'} \andalso {{P}' \red {Q}'} \andalso {{Q}' \scong {Q}}}{{P} \red {Q}}
\end{mathpar}

\begin{eqnarray*}
  match_{\equiv} (\quotep{P},\quotep{Q}) & := & P \equiv Q \\
  match_{\dagger}(\quotep{P},\quotep{Q}) & := & \forall R. P|Q \red^{*} R => R \red^{*} 0 \\
  match_{K}(\quotep{P},\quotep{Q}) & := & K \mbox{ for some context } K
\end{eqnarray*}

$u?(x)P | u!\langle Q \rangle \red P\{\quotep{Q}/x\}$

%We write $\wred$ for $\red^*$, and $P\red$ if $\exists Q $ such that $ P \red Q$.
We write $P\red$ if $\exists Q $ such that $ P \red Q$ and $P\not\red$, otherwise.

\section{Replication}

As mentioned before, it is known that replication (and hence
recursion) can be implemented in a higher-order process algebra
\cite{SangiorgiWalker}. As our first example of calculation with the
machinery thus far presented we give the construction explicitly in
the {\rhoc}.

\begin{eqnarray}
	D_{x} & := & \prefix{x}{y}{(\binpar{\outputp{x}{y}}{@{y}})} \nonumber\\
	\bangp_{x}{P} & := & \binpar{{x}!\langle{\binpar{D_{x}}{P}}\rangle}{D_{x}} \nonumber
\end{eqnarray}

\begin{eqnarray}
	\bangp_{x}{P} & & \nonumber\\
	=
	& {x}!\langle{(\prefix{x}{y}{(\outputp{x}{y} | @{y})) | P}}\rangle 
	      | \prefix{x}{y}{(\outputp{x}{y} | @{y})} & \nonumber\\
	\red
	& (\outputp{x}{y} | @{y})\substn{\quotep{(\prefix{x}{y}{(@{y} | \outputp{x}{y})) | P}}}{y} & \nonumber\\
	=
	& \outputp{x}{\quotep{(\prefix{x}{y}{(\outputp{x}{y} | @{y})) | P}}}
	  | {(\prefix{x}{y}{(\outputp{x}{y} | @{y})) | P}} & \nonumber\\
	\red
	& \ldots & \nonumber\\
	\red^*
	& P | P | \ldots & \nonumber
\end{eqnarray}

Of course, this encoding, as an implementation, runs away, unfolding
$\bangp{P}$ eagerly. A lazier and more implementable replication
operator, restricted to input-guarded processes, may be obtained as follows.

\begin{eqnarray}
\bangp{\prefix{u}{v}{P}} 
	:= 
	\binpar{\lift{x}{\prefix{u}{v}{(\binpar{D(x)}{P})}}}{D(x)} \nonumber
\end{eqnarray}

\begin{remark}
  Note that the lazier definition still does not deal with summation
  or mixed summation (i.e. sums over input and output). The reader is
  invited to construct definitions of replication that deal with these
  features. 

  Further, the definitions are parameterized in a name, $x$. Can you,
  gentle reader, make a definition that eliminates this parameter and
  guarantees no accidental interaction between the replication
  machinery and the process being replicated -- i.e. no accidental
  sharing of names used by the process to get its work done and the
  name(s) used by the replication to effect copying. This latter
  revision of the definition of replication is crucial to obtaining
  the expected identity $!!P \sim !P$.
\end{remark}

\begin{remark}\label{rem:paradoxical_combinator}
  The reader familiar with the lambda calculus will have noticed the
  similarity between $D$ and the paradoxical combinator.

  [Ed. note: the existence of this seems to suggest we have to be more
  restrictive on the set of processes and names we admit if we are to
  support no-cloning.]
\end{remark}

\subsubsection{Bisimulation}

The computational dynamics gives rise to another kind of equivalence,
the equivalence of computational behavior. As previously mentioned
this is typically captured \emph{via} some form of bisimulation.

% The notion we use in this paper is weak barbed bisimulation
% \cite{milner91polyadicpi}.

The notion we use in this paper is derived from weak barbed
bisimulation \cite{milner91polyadicpi}. 

\begin{definition}
An \emph{observation relation}, $\downarrow_{\mathcal N}$, over a set
of names, $\mathcal N$, is the smallest relation satisfying the rules
below.

\infrule[Out-barb]{y \in {\mathcal N}, \; x \nameeq y}
		  {\outputp{x}{v} \downarrow_{\mathcal N} x}
\infrule[Par-barb]{\mbox{$P\downarrow_{\mathcal N} x$ or $Q\downarrow_{\mathcal N} x$}}
		  {\binpar{P}{Q} \downarrow_{\mathcal N} x}

We write $P \Downarrow_{\mathcal N} x$ if there is $Q$ such that 
$P \wred Q$ and $Q \downarrow_{\mathcal N} x$.
\end{definition}

\begin{definition}
%\label{def.bbisim}
An  ${\mathcal N}$-\emph{barbed bisimulation} over a set of names, ${\mathcal N}$, is a symmetric binary relation 
${\mathcal S}_{\mathcal N}$ between agents such that $P\rel{S}_{\mathcal N}Q$ implies:
\begin{enumerate}
\item If $P \red P'$ then $Q \wred Q'$ and $P'\rel{S}_{\mathcal N} Q'$.
\item If $P\downarrow_{\mathcal N} x$, then $Q\Downarrow_{\mathcal N} x$.
\end{enumerate}
$P$ is ${\mathcal N}$-barbed bisimilar to $Q$, written
$P \wbbisim_{\mathcal N} Q$, if $P \rel{S}_{\mathcal N} Q$ for some ${\mathcal N}$-barbed bisimulation ${\mathcal S}_{\mathcal N}$.
\end{definition}

$\mathcal{R} \subseteq \pi \times \pi$

$P \mathcal{R} Q => \forall P'. P \red P' \Rightarrow \exists Q'. Q \red Q', P' \mathcal{R} Q'$

$P \vdash x \Rightarrow Q \vdash x$

\begin{mathpar}
  \inferrule*[lab=Out-barb]{x \nameeq y}{{y}!\langle{Q}\rangle \vdash x}
  \and
  \inferrule*[lab=Par-barb]{\mbox{$P\vdash x$ or $Q\vdash x$}}{\binpar{P}{Q} \vdash x}
\end{mathpar}

\subsubsection{Contexts}

One of the principle advantages of computational calculi like the
$\pi$-calculus is a well-defined notion of context,
contextual-equivalence and a correlation between
contextual-equivalence and notions of bisimulation. The notion of
context allows the decomposition of a process into (sub-)process and
its syntactic environment, its context. Thus, a context may be
thought of as a process with a ``hole'' (written $\Box$) in it. The
application of a context $M$ to a process $P$, written $M[P]$, is
tantamount to filling the hole in $M$ with $P$. In this paper we do
not need the full weight of this theory, but do make use of the notion
of context in the proof the main theorem. 

\begin{mathpar}
  \inferrule* [lab=summation] {} {{M_{M},M_{N}} \bc \Box \;|\; x.M_{A} \;|\; M_{M}+M_{N}}
  \and
  \inferrule* [lab=agent] {} {{M_{A}} \bc (\vec{x})M_{P} \;| \; \clift{P_0,\ldots,M_{P},\ldots,P_N}}
  \and \\
  \inferrule* [lab=process] {} {{M_{P}} \bc M_{N} \;| \;P|M_{P} }
\end{mathpar} 

\begin{mathpar}
  \inferrule* [lab=sychronization] {} {M_{N} \bc \Box \;|\; x?M_{F} \;|\; x!M_{C}}
  \and
  \inferrule* [lab=abstraction] {} {{M_{F}} \bc (x)M_{P} }
  \and
  \inferrule* [lab=concretion] {} {{M_{C}} \bc \langle M_{P} \rangle }
  \and \\
  \inferrule* [lab=process] {} {{M_{P}} \bc M_{N} \;| \;P|M_{P} }
\end{mathpar}

\begin{definition}[contextual application] Given a context $M$, and
  process $P$, we define the \emph{contextual application}, $M[P] :=
  M\{P/\Box\}$. That is, the contextual application of M to P is the
  substitution of $P$ for $\Box$ in $M$.
\end{definition}

$\meaningof{-} : L \to \mathcal{P}(\pi)$

\begin{mathpar}
  \inferrule* [lab=collection] {} {\meaningof{true} = \pi, \and \meaningof{~E} = \pi \setminus \meaningof{E}, \and \meaningof{E_{1} \& E_{2}} = \meaningof{E_{1}} \cap \meaningof{E_{2}}}
\end{mathpar}

\begin{mathpar}
  \inferrule* [lab=structure] {} {\meaningof{0} = \{ P \in \pi | P \equiv 0 \}, \and \\ \meaningof{E_1 | E_2} = \{ P \in \pi | P \equiv P_{1} | P_{2}, P_{1} \in \meaningof{E_{1}}, P_{2} \in \meaningof{E_2}\} }
\end{mathpar}

\begin{mathpar}
 \inferrule* [lab=behavior] {} {\meaningof{\langle a?b \rangle E} = \{ P \in \pi | P \equiv Q | u?(y)P', \\ \and \\\\ \and \\ \;\;\; u \in \meaningof{a}, \forall z.P'\{z/y\} \in \meaningof{E\{z/b\}}\}, \and \\ \meaningof{a!E} = \{ P \in \pi | P \equiv Q | x!\langle P' \rangle, x \in \meaningof{a} P' \in \meaningof{E}\} }
\end{mathpar}

\begin{mathpar}
 \inferrule* [lab=nominal] {} {\meaningof{\quotep{E}} = \{ \quotep{P} \in \quotep{\pi} | P \in \meaningof{E} \}, \and \meaningof{\quotep{P}} = \{ \quotep{Q} \in \quotep{\pi} | P \equiv Q \} \and \\ \meaningof{@\quotep{E}} = \{ P \in \pi | P \equiv @x, x \in \meaningof{E} \}}
\end{mathpar}

\begin{eqnarray*}
  \\
  \meaningof{-} : TS \to ST
\end{eqnarray*}

\begin{eqnarray*}
  \\
  L : TS \to ST
\end{eqnarray*}

\begin{eqnarray*}
  \\
  P \models E \iff P \in \meaningof{E}
\end{eqnarray*}

\begin{eqnarray*}
  P \approx_{L} Q \iff \forall E \in L. P \models E \iff Q \models E
\end{eqnarray*}

\begin{eqnarray*}
  P \approx_{K} Q
\end{eqnarray*}

\begin{eqnarray*}
  P \approx Q
\end{eqnarray*}

$\approx_{K} = \approx = \approx_{L}$

\subsubsection{Contextual duality}

Note that contexts extend the quotation operation to a family of
operations from processes to names. Given a context, $M$, we can
define a \emph{nominal context}, $\quotep{M}$ by $\quotep{M}[P] :=
\quotep{M[P]}$. To foreshadow what is to come we observe that these
operations enjoy a duality with processes very much like the duality
between vectors and maps from vectors to scalars.

Further, because the calculus is essentially higher-order, we have a
correspondence between contexts and processes. More specifically,
given a name $x$ and a context $M$ we can construct $M^{*}_{x}$ such
that 

\begin{mathpar}
  M^{*}_{x} | \lift{x}{P} \red M[P]
\end{mathpar}

namely,

\begin{mathpar}
  M^{*}_{x} := x?(u).M[\dropn{u}]
\end{mathpar}

The dependence of $M^{*}_{x}$ on a name makes it an abstraction, 

\begin{mathpar}
  M^{*} := (x)x?(u).M[\dropn{u}]
\end{mathpar}

\subsection{Additional notation}

It will sometimes be convenient to denote the process a name
quotes. We already have the notation $x = \quotep{P}$, but it will be
convenient to introduce an alternate notation, $\procn{x}$, when we
want to emphasize the connection to the use of the name. Note that, by
virtue of name equivalence, $\quotep{\procn{x}} \nameeq x$; so, the
notation is consistent with previous definitions.

Further, because names have structure it is possible to effect
substitutions on the basis of that structure. This means we need to
upgrade our notation for substitutions, which we accomplish by
adapting comprehension notation. Thus,

\begin{mathpar}
  P\{ y / x : x \in S \}
\end{mathpar}

is interpreted to mean the process derived from P by replacing (in a
capture-avoiding manner) each occurrence of $x$ in $S$ by $y$. For example,

\begin{mathpar}
  P\{ \quotep{\procn{x}|\procn{x}} / x : x \in \freenames{P} \}
\end{mathpar}

will replace each (occurrence) of a free name $x$ in $P$ by
$\quotep{\procn{x}|\procn{x}}$.

Also, we will avail ourselves of the notation $x^{L}$ and $x^{R}$ to
denote injections of a name into disjoint copies of the name
space. There are numerous ways to accomplish this. One example can be
found in \cite{MeredithR05}. This notation overloads to vectors of
names: $\vec{x}^{\pi} := (x_{i}^{\pi} \; : \; 0 \leq i < |\vec{x}| )$ where $\pi \in \{L,R\}$.

We also use $P^{\Box} := P|\Box$.

In \cite{MeredithR05} an interpretation of the new operator is
given. It turns out that there are several possible interpretations
all enjoying the requisite algebraic properties of the operator (see
\cite{milner91polyadicpi}). We will therefore make liberal use of
$(\nu\; \vec{x})P$.

% subsection the_syntax_and_semantics_of_the_notation_system (end)   

\section{Interpretation of QM}
\subsection{Supporting definitions}
\subsubsection{Multiplication}
\begin{mathpar}
  \quotep{Q} \cdot \quotep{R} := \quotep{Q|R}
  \and \\
  \quotep{Q} \cdot P := P\{ \quotep{Q|R} / \quotep{R} : \quotep{R} \in \freenames{P} \}
\end{mathpar}

\paragraph{Discussion}
The first line needs little explanation. The second line says that
each free name of the process is replaced with the multiplication of
that name by the scalar. Multiplication of a scalar (name) by a state
(process) results in a process all the names of which have been `moved
over' by parallel composition with the process the scalar
quotes. There is a subtlety that the bound names have to be
manipulated so that multiplied names aren't accidentally
captured. There are many ways to achieve this.

\begin{remark}\label{rem:multiplication_identities}
  The reader is invited to verify that for all $x,y,z \in \QProc$ and $P \in \Proc$
  \begin{mathpar}
    x \cdot \quotep{0} \equiv x 
    \and
    x \cdot y \equiv y \cdot x
    \and
    x \cdot (y \cdot z) \equiv (x \cdot y) \cdot z
    \and \\
    \quotep{0} \cdot P \equiv P
    \and \\
    x \cdot (y \cdot P) \equiv (x \cdot y) \cdot P
    \and \\
    x \cdot (P|Q) \equiv (x \cdot P) | (x \cdot Q)
    \and \\    
  \end{mathpar}
\end{remark}

\subsubsection{Tensor product}

We define a tensor product on processes by structural induction.

\paragraph{Tensor of sums} First note that all summations, including
$\pzero$ and sequence, can be written $\Sigma_{i} x_{i}.A_{i} +
\Sigma_{j} x_{j}.C_{j}$, where we have grouped input-guarded processes
together and output-guarded processes together.

Thus, we can define the tensor product of two summations, $N_{1}\otimes N_{2}$, where

\begin{mathpar}
  N_{1} := \Sigma_{i} x_{i}.A_{i} + \Sigma_{j} x_{j}.C_{j}
  \and
  N_{2} := \Sigma_{i'} y_{i'}.B_{i'} + \Sigma_{j'} y_{j'}.D_{j'} 
\end{mathpar}

as follows.

\begin{mathpar}
  \Sigma_{i} x_{i}.A_{i} + \Sigma_{j} x_{j}.C_{j} \otimes \Sigma_{i'}
  y_{i'}.B_{i'} + \Sigma_{j'} y_{j'}.D_{j'} 
  \and \\
  := \; \Sigma_{i} \Sigma_{i'} \quotep{\stackrel{\vee}{x_{i}}| \stackrel{\vee}{y_{i'}}}.(A_{i}\otimes B_{i'}) \; | \; \Sigma_{i'} \Sigma_{i} \quotep{\stackrel{\vee}{y_{i'}}|\stackrel{\vee}{x_{i}}}.(B_{i'}\otimes A_{i})
  \and
  \;\; | \;\; \Sigma_{j} \Sigma_{j'} \quotep{\stackrel{\vee}{x_{j}}|\stackrel{\vee}{y_{j'}}}.(A_{j}\otimes B_{j'}) \; | \; \Sigma_{j'} \Sigma_{j} \quotep{\stackrel{\vee}{y_{j'}}|\stackrel{\vee}{x_{j}}}.(B_{j'}\otimes A_{j})
\end{mathpar}

\begin{remark}
  Do we need to $x^{L}$ and $y^{R}$ for this construction as well?
\end{remark}

\paragraph{Tensor of parallel compositions} Next, we distribute tensor
over par.

\begin{mathpar}
  P_{1}|P_{2} \otimes Q_{1}|Q_{2} := (P_{1} \otimes Q_{1}) | (P_{1}
  \otimes Q_{2}) | (P_{2} \otimes Q_{1}) | (P_{2} \otimes Q_{2})
\end{mathpar}

\paragraph{Tensor with dropped names} We treat tensor of a
process with a dropped name as parallel composition.

\begin{mathpar}
  P \otimes \dropn{x} := P | \dropn{x}
\end{mathpar}

\paragraph{Tensor of agents}

Finally, we need to define tensor on agents. Note that the definition
of tensor on normal products only tensors inputs with inputs and
outputs with outputs. Thus, we only have to define the operation on
``homogeneous'' pairings.

\begin{mathpar}
  (\vec{x})P \otimes (\vec{y})Q
  \and \\
  := (x_{0}^{L}|y_{0}^{R},\ldots,x_{0}^{L}|y_{n}^{R},\ldots,x_{m}^{L}|y_{0}^{R},\ldots,x_{m}^{L}|y_{n}^R)(P\{ \vec{x}^{L}/\vec{x}\} \otimes Q \{ \vec{y}^{R}/\vec{y}\})
  \and \\
  \clift{\vec{P}} \otimes \clift{\vec{Q}}
  \and \\
  := \clift{P_{0}\otimes Q_{0},\ldots,P_{0}\otimes Q_{n},\ldots,P_{m}\otimes Q_{0},\ldots,P_{m}\otimes Q_{n}}
\end{mathpar}

\begin{remark}
  Observe that arities of tensored abstractions matches arities of
  tensored concretions if the original arities matched. Note also that
  the length of the arities corresponds to the increase in dimension
  we see in ordinary vector space tensor product.
\end{remark}

\begin{remark}
  Operationally, this definition distributes the tensor down to
  components ``linked'' by summation. Tensor over summation is
  intriguing in that it mixes names. Moreover, as a consequence of the
  way it mixes names we have the identities for all $x \in \QProc$ and
  $P,Q \in \Proc$

  \begin{mathpar}
    (x \cdot P) \otimes Q \equiv x \cdot (P \otimes Q) \equiv P \otimes (x \cdot Q)
    \and
    P \otimes \pzero \equiv P
  \end{mathpar}

  that the reader is invited to verify.
\end{remark}

\subsubsection{Annihilation}
\begin{mathpar}
  P^{\perp} := \{ Q | \forall R. P|Q \red^{*} R \Rightarrow R \red^{*} \pzero \}
  \and \\
  P^{\underline{\perp}} := \Sigma_{Q \in P^{\perp}} \quotep{Q}?(y).(\dropn{y}|Q) | \Sigma_{Q \in P^{\perp}} \quotep{Q}\clift{\Box}
\end{mathpar}

\paragraph{Discussion} The reader will note that $P^{\perp}$ is a
\emph{set} of processes, while $P^{\underline{\perp}}$ is a
\emph{context}. We call the set $P^{\perp}$ the \emph{annihilators} of
$P$. The parallel composition of a process in the annihilators of $P$
with $P$ will result in a process, the state space of which has all
paths eventually leading to $\pzero$. Execution may endure loops; but
under reasonable conditions of fairness (naturally guaranteed under
most notions of bisimulation) such a composite process cannot get
stuck in such a loop and will, eventually pop out and terminate.

The context $P^{\underline{\perp}}$ is ready and willing to ``take the
$P$ out of'' the process to which it is applied. It will effectively
transmit the code of the process to which it is applied to one of the
annihilators and run the process against it.

\subsubsection{Evaluation}
We fix $M$ a domain of fully abstract interpretation with an equality
coincident with bisimulation. We take $\meaningof{\cdot} : \Proc \to
M$ to be the map interpreting processes and $\nmeaningof{\cdot} : \M
\to Proc$ to be the map running the other way. Then we define

\begin{mathpar}
  \int P := \nmeaningof{\meaningof{P}}
\end{mathpar}

\paragraph{Discussion}
There are many fully abstract interpretations of Milner's
$\pi$-calculus. Any of them can be used as a basis for interpreting
the reflective calculus here. Equipped with such a domain it is
largely a matter of grinding through to check that the Yoneda
construction for the normalization-by-evaluation program can be
extended to this setting.

\begin{remark}
  The reader is invited to verify that $\int (P^{\underline{\perp}}[P]) = 0$.
\end{remark}

\subsection{Quantum mechanics}

Table \ref{tbl:core_qm_op_defns} gives the core operational definitions

\begin{table}[htp]\label{tbl:core_qm_op_defns}
  \center{
    \fbox{
      \begin{tabular}{c|c}
        quantum mechanics & process calculus \\
        \hline
        scalar & $x := \quotep{P}$ \\
        state vector & $\state{P} := P$ \\
        dual & $\state{P}^{*} := \event{P^{\underline{\perp}}} := \quotep{P^{\underline{\perp}}}[-]$ \\
        matrix & $ \Sigma_{\alpha} \state{P_{\alpha}}x_{\alpha}\event{Q_{\alpha}}$ \\
        vector addition & $\state{P} + \state{Q} := \state{P | Q}$ \\
        tensor product & $\state{P} \otimes \state{Q} := \state{P \otimes Q}$ \\
        inner product & $\innerprod{P}{Q} := \quotep{\int P^{\underline{\perp}}[Q]}$ \\
      \end{tabular}
    }
  }
  \caption{QM - operational definitions}
\end{table}

where

\begin{mathpar}
  \prmatrix{P}{Q} := \fprmatrix{P}{\quotep{\pzero}}{Q}
  \and
  \fprmatrix{P}{x}{Q} := (\state{P},x,\event{Q})
  \and
  (\fprmatrix{P}{x}{Q})(\state{R}) := x \cdot \innerprod{Q}{R} \cdot \state{P}
  \and
  (\fprmatrix{P}{x}{Q})(\event{R}) := x \cdot \innerprod{R}{P} \cdot \event{Q}
\end{mathpar}

\paragraph{Discussion}
As promised: vectors (aka states) are represented as processes; duals
as contextual duals; inner product definition should be compared with
standard inner product definition for ....

\begin{remark}
  Assuming $\int (P^{\underline{\perp}}[P]) = 0$, the reader is
  invited to verify that $(\fprmatrix{P}{x}{P})(\state{P}) = x \cdot \state{P}$.
\end{remark}

\begin{remark}
  The reader is invited to verify that $\innerprod{P}{Q}$ could
  equally well have been written $\quotep{\int \stackrel{\vee}{x}}$
  where $x = \event{P^{\underline{\perp}}}(Q)$.

  One of the motivations for this remark is that there is another way
  to factor these operations. We could package up evaluation in the dual:

  \begin{mathpar}
    \state{P}^{*} := \event{\int P^{\underline{\perp}}} := \quotep{\int P^{\underline{\perp}}}[-]
  \end{mathpar}

  and then have inner product defined by
  
  \begin{mathpar}
    \innerprod{P}{Q} := \event{P}(Q)
  \end{mathpar}

  Hopefully, experience with the calculations will provide guidance on
  the best factoring.
\end{remark}

\begin{remark}
  Assuming $\int (P^{\underline{\perp}}[P]) = 0$, the reader is
  invited to verify that $\forall P,Q. (\prmatrix{0}{Q})(\state{0}) =
  \state{0}$ and dually $(\prmatrix{P}{0})(\event{0}) = \event{0}$.
\end{remark}

\begin{remark}
  i'm a little worried that i don't (yet) have proper support for
  complex conjugacy. But, the observation above may give us a
  clue. According to Abramsky, it must be the case that the scalars
  are iso to the homset of the identity for the tensor -- which the
  observation above characterizes. 

  For now, we will simply bookmark the notion with $\overline{x}$.
\end{remark}

\subsubsection{Adjointness}

We need to give a definition of $(\cdot)^{\dagger}$ for matrices. The
obvious candidate definition is
\begin{mathpar}
(\Sigma_{\alpha}\fprmatrix{P_{\alpha}}{x_{\alpha}}{Q_{\alpha}})^{\dagger}
= \Sigma_{\alpha}\fprmatrix{(Q_{\alpha}^{\underline{\perp}})^{*}}{\overline{x}_{\alpha}}{P_{\alpha}^{\underline{\perp}}} 
\end{mathpar}

But, $(Q_{\alpha}^{\underline{\perp}})^{*}$ requires a name along
which to communicate the process to achieve the context application.

\subsubsection{Basis for a basis}
If processes label states and ``addition'' of states (a.k.a. vector
addition) is interpreted as parallel composition, what corresponds to
notions of linear independence and basis? Here, we recall that Yoshida
has developed a set of \emph{combinators} for an asynchronous verison
of Milner's $\pi$-calculus. These are a finite set of processes such
any process can be expressed as parallel composition of these
combinators together with liberal uses of the new operator and
replication. We can simply give a translation of these into the
present calculus and have reasonable expectation that the property
carries over. That is, that the resultant set allows to express all
processes via parallel composition. Note, however, that there is no
new operator or replication in this calculus. As a result, we expect
that the corresponding set is actually infinite. That is, we expect
that the space is actually infinite dimensional.

\begin{remark}
  The attentive reader may be a bit concerned. Certainly, the
  collection $S$, $K$ and $I$ is a finite set of
  combinators. Shouldn't we expect to see a finite set of combinators
  for an effectively equivalent system? i am very sympathetic to this
  critique and feel it warrants full attention. On the other hand, i
  also have in mind the following analogy. The natural numbers, as a
  monoid under addition, has exactly $1$ generator, while the natural
  numbers, as a monoid under multiplication, has countably many
  generators (the primes). We observe that the application of the
  lambda calculus is much less resource sensitive than the parallel
  composition of the $\pi$-calculus. Could it be the case that we have
  an analogy of the form
  
  \begin{mathpar}
    m + n : MN :: m*n : M|N
  \end{mathpar}

  giving a similar blow up in the set of ``primes''?  This is such a
  wonderful thought that, even if it's not true, i think it's worth
  writing down.
\end{remark}
 

\documentclass[12pt]{llncs}
%\documentclass{jktr}

\usepackage[pdftex]{hyperref}                   
\usepackage {listings}
\usepackage {mathpartir}
\usepackage{bcprules}
%\usepackage{listings}
                       
\usepackage{graphicx} 
%\usepackage[margins=2.5cm,nohead,nofoot]{geometry}
%\usepackage{geometry}
\usepackage{amsfonts}
\usepackage{amstext}
\usepackage{latexsym}
\usepackage{amssymb}
\usepackage{color}


%\include{myPreamble}
\include{qm2pi.local} 

%\ifpdf
%\usepackage[pdftex]{graphicx}
%\else
%\usepackage{graphicx}
%\fi

 % \ifpdf
%  \usepackage{pdfsync}
%  \if


%\title{Brief Article}
%\author{David F. Snyder}
%\author{L.G. Meredith}

%\address{Dept. of Math., Texas State University--San Marcos, San Marcos, TX 78666}
       
\pagestyle{empty}


\begin{document}

\lstset{language=[Objective]Caml,frame=shadowbox}

\input{qm2pi.front}

% section front matter (end)

\input{qm2pi.intro} 
 
% section introduction (end)

% \input{qm2pi.knotations} 

% section notation (end)

\input{qm2pi.process.calculi} 

% section concurrent_process_calculi_and_spatial_logics_ (end)
    
%\input{qm2pi.knots2pi} 

%\input{qm2pi.trefoil} 

%\input{qm2pi.mainthm} 

% subsection basic_interpretation (end)

%\input{qm2pi.rho.presentation} 
\subsection{The syntax and semantics of the notation system}\label{sub:the_syntax_and_semantics_of_the_notation_system} % (fold)

We now summarize a technical presentation of the calculus that
embodies our theory of dynamics. The typical presentation of such a
calculus follows the style of giving generators and relations on
them. The grammar, below, describing term constructors, freely
generates the set of processes, $\Proc$. This set is then quotiented
by a relation known as structural congruence and it is over this set
that the notion of dynamics is expressed. This presentation is
essentially that of \cite{MeredithR05} with the addition of
polyadicity and summation. For readability we have relegated some of
the technical subtleties to an appendix.

\subsubsection{Process grammar}\label{subsub:process_grammar}

\begin{mathpar}
  \inferrule* [lab=synchronization] {} {{M} \bc \pzero \;|\; x?F \;|\; x!C }
  \and
  \inferrule* [lab=abstraction] {} {{F} \bc (x)P}
  \and
  \inferrule* [lab=concretion] {} {{C} \bc \langle Q \rangle}
  \and
  \inferrule* [lab=process] {} {{P,Q} \bc M \;| \;P|Q \;|\; @{x}}
  \and
  \inferrule* [lab=name] {} {{x} \bc \quotep{P}}
\end{mathpar} 

Note that $\vec{x}$ (resp. $\vec{P}$) denotes a vector of names
(resp. processes) of length $|\vec{x}|$ (resp. $|\vec{P}|$). We adopt
the following useful abbreviations.

\begin{mathpar}
   x?(\vec{y}).P := x.(\vec{y})P \and  x\clift{\vec{P}} := x.\clift{\vec{P}}
   \and x!(y) := \lift{x}{\dropn{y}}
   \and \Pi_{i=0}^{n-1}P_i := P_0 | \ldots | P_{n-1}
\end{mathpar}

\subsubsection{Structural congruence}

\paragraph{Free and bound names and alpha-equivalence.} At the
core of structural equivalence is alpha-equivalence which identifies
process that are the same up to a change of variable. Formally, we
recognize the distinction between free and bound names. The free names
of a process, $\freenames{P}$, may be calculated recursively as
follows:

\begin{mathpar}
\freenames{\pzero} := \emptyset
  \and \\
  \freenames{x?(y).P} := \{ x \} \cup (\freenames{P} \setminus \{ y \})
  \and 
  \freenames{x!\langle P \rangle} := \{ x \} \cup \{ P \} 
  \and \\
  \freenames{P|Q} := \freenames{P} \cup \freenames{Q}
  \and \\
  \freenames{@{x}} := \{ x \}
\end{mathpar}

$\pi$
$\quotep{\pi}$

$\freenames{-} : \pi \to \mathcal{P}(\quotep{\pi})$

\begin{eqnarray*}
  \freenames{\pzero} & := & \emptyset \\
  \freenames{x?(y).P} & := & \{ x \} \cup (\freenames{P} \setminus \{ y \}) \\
  \freenames{x!\langle P \rangle} & := & \{ x \} \cup \{ P \} \\
  \freenames{P|Q} & := & \freenames{P} \cup \freenames{Q} \\
  \freenames{\dropn{x}} & := & \{ x \}
\end{eqnarray*}

The bound names of a process, $\boundnames{P}$, are those names occurring in $P$
that are not free. For example, in $x?(y).0$, the name $x$ is free, while $y$ is bound.

\begin{mathpar}
  \inferrule* [lab=monoidal-laws] {} { P|Q \equiv Q|P \and P|0 \equiv P \and P|(Q|R) \equiv (P|Q)|R }
\end{mathpar}

\begin{mathpar}
  \inferrule* [lab=alpha-equivalence] {} { (x)P \equiv (y)P\{y/x\} \and y \not\in \freenames{P} }
\end{mathpar}

\begin{definition}
Then two processes, $P,Q$, are alpha-equivalent if $P = Q\{\vec{y}/\vec{x}\}$ for
some $\vec{x} \in \boundnames{Q},\vec{y} \in \boundnames{P}$, where $Q\{\vec{y}/\vec{x}\}$
denotes the capture-avoiding substitution of $\vec{y}$ for $\vec{x}$ in $Q$.
\end{definition}

\begin{definition}
  The {\em structural congruence} \cite{SangiorgiWalker} , $\equiv$,
  between processes is the least congruence containing
  alpha-equivalence, satisfying the abelian monoid laws
  (associativity, commutativity and $\pzero$ as identity) for parallel
  composition $|$ and for summation $+$.
\end{definition}

\subsection{Name equivalence}

We take name equivalence, written $\nameeq$, to be the smallest
equivalence relation generated by the following rules.

\begin{mathpar}
\inferrule*[lab=Quote-drop]
{ }
{ \quotep{@{x}} \nameeq x }

\inferrule*[lab=Struct-equiv]
{ P \scong Q }
{ \quotep{P} \nameeq \quotep{Q} }
\end{mathpar}

The astute reader will have noticed that the mutual recursion of names
and processes imposes a mutual recursion on alpha-equivalence and
structural equivalence via name-equivalence. Fortunately, all of this
works out pleasantly and we may calculate in the natural way, free of
concern. The reader interested in the details is referred to the
appendix \ref{appendix:rho_details}.

\subsection{Substitution}

We use $\Proc$ for the set of processes, $\QProc$ for the set of
names, and $\id{\{}\vec{y} / \vec{x} \id{\}}$ to denote partial maps,
$s : \QProc \rightarrow \QProc$. A map, $s$ lifts, uniquely, to a map
on process terms, $\widehat{s} : \Proc \rightarrow \Proc$ by the
following equations.

\begin{mathpar}
  (0) \psubstp{Q}{P} := 0 \\
  (R \juxtap S) \psubstp{Q}{P}
  :=    
  (R)\psubstp{Q}{P} \juxtap (S) \psubstp{Q}{P} \\
  (x?(y).R) \psubstp{Q}{P}    
  :=    
  (x)\substp{Q}{P} (z)\concat( (R \psubstn{z}{y}) \psubstp{Q}{P} ) \\
  (\lift{x}{R}) \psubstp{Q}{P}  
  :=
  \lift{(x)\substp{Q}{P}}{ R \psubstp{Q}{P} } \\
%   (\dropn{x})  \psubstp{Q}{P}       
%   := 
%   \left\{ 
%     \begin{array}{ccc} 
%       \dropn{\quotep{Q}} & & x \nameeq \quotep{P} \\
%       \dropn{x} & & otherwise \\
%     \end{array}
%   \right. 
  (\dropn{x})  \psubstp{Q}{P}       
  := 
  \left\{ 
    \begin{array}{ccc} 
      Q & & x \nameeq \quotep{P} \\
      \dropn{x} & & otherwise \\
    \end{array}
  \right.
\end{mathpar}
 

where

\begin{eqnarray}
  (x)\id{\{} \lpquote Q \rpquote / \lpquote P \rpquote \id{\}}            = 
  \left\{ 
    \begin{array}{ccc}
      \lpquote Q \rpquote & & x \nameeq \lpquote P \rpquote \\
      x & & otherwise \\
    \end{array}
  \right. \nonumber
\end{eqnarray}

and $z$ is chosen distinct from $\quotep{P}$, $\quotep{Q}$, the free
names in $Q$, and all the names in $R$. Our $\alpha$-equivalence will
be built in the standard way from this substitution.

\begin{remark}\label{rem:no_self_referential_names}
  One consequence of these definitions is that $\forall P. \quotep{P}
  \not\in \freenames{P}$.
\end{remark}

\subsection{ Dynamic quote: an example }

Anticipating something of what's to come, consider applying the
substitution, $\widehat{\id{\{}u / z \id{\}}}$, to the following pair
of processes, $\lift{w}{y!(z)}$ and $w[ \lpquote y!(z) \rpquote ]$.

\begin{eqnarray}
	\lift{w}{y!(z)}\widehat{\id{\{}u / z \id{\}}}
		& = &
		\lift{w}{y!(u)} \nonumber\\
	w[ \lpquote y!(z) \rpquote ] \widehat{ \id{\{}u / z \id{\}} }
		& = &
		w[ \lpquote y!(z) \rpquote ] \nonumber
\end{eqnarray}

Because the body of the process between quotes is impervious to
substitution, we get radically different answers. In fact, by
examining the first process in an input context,
e.g. $x?(z).\lift{w}{y!(z)}$, we see that the process under the lift
operator may be shaped by prefixed inputs binding a name inside it. In
this sense, the lift operator will be seen as a way to dynamically
construct processes before reifying them as names.

Finally equipped with these standard features we can present the
dynamics of the calculus.

\subsubsection{Operational semantics} 

Finally, we introduce the computational dynamics. What marks these
algebras as distinct from other more traditionally studied algebraic
structures, e.g. vector spaces or polynomial rings, is the manner in
which dynamics is captured. In traditional structures, dynamics is typically
expressed through morphisms between such structures, as in linear maps
between vector spaces or morphisms between rings. In algebras
associated with the semantics of computation, the dynamics is
expressed as part of the algebraic structure itself, through a
reduction reduction relation typically denoted by $\red$. Below, we
give a recursive presentation of this relation for the calculus used
in the encoding.

$\red \subseteq \pi \times \pi$
$\red : \pi \to \mathcal{P}(\pi)$

\begin{mathpar}
  \inferrule* [lab=Comm] { \textsf{match}( x_{src}, x_{trgt} ) } { x_{trgt}?(y)P \; | \; x_{src}!\langle {Q} \rangle \red P\{\quotep{Q}/y}\} }
  \and \\
  \inferrule* [lab=Par] {{P} \red {P}'} {{{P} | {Q}} \red {{P}' | {Q}}}
  \and
  \inferrule* [lab=Equiv]{{{P} \scong {P}'} \andalso {{P}' \red {Q}'} \andalso {{Q}' \scong {Q}}}{{P} \red {Q}}
\end{mathpar}

\begin{eqnarray*}
  match_{\equiv} (\quotep{P},\quotep{Q}) & := & P \equiv Q \\
  match_{\dagger}(\quotep{P},\quotep{Q}) & := & \forall R. P|Q \red^{*} R => R \red^{*} 0 \\
  match_{K}(\quotep{P},\quotep{Q}) & := & K \mbox{ for some context } K
\end{eqnarray*}

$u?(x)P | u!\langle Q \rangle \red P\{\quotep{Q}/x\}$

%We write $\wred$ for $\red^*$, and $P\red$ if $\exists Q $ such that $ P \red Q$.
We write $P\red$ if $\exists Q $ such that $ P \red Q$ and $P\not\red$, otherwise.

\section{Replication}

As mentioned before, it is known that replication (and hence
recursion) can be implemented in a higher-order process algebra
\cite{SangiorgiWalker}. As our first example of calculation with the
machinery thus far presented we give the construction explicitly in
the {\rhoc}.

\begin{eqnarray}
	D_{x} & := & \prefix{x}{y}{(\binpar{\outputp{x}{y}}{@{y}})} \nonumber\\
	\bangp_{x}{P} & := & \binpar{{x}!\langle{\binpar{D_{x}}{P}}\rangle}{D_{x}} \nonumber
\end{eqnarray}

\begin{eqnarray}
	\bangp_{x}{P} & & \nonumber\\
	=
	& {x}!\langle{(\prefix{x}{y}{(\outputp{x}{y} | @{y})) | P}}\rangle 
	      | \prefix{x}{y}{(\outputp{x}{y} | @{y})} & \nonumber\\
	\red
	& (\outputp{x}{y} | @{y})\substn{\quotep{(\prefix{x}{y}{(@{y} | \outputp{x}{y})) | P}}}{y} & \nonumber\\
	=
	& \outputp{x}{\quotep{(\prefix{x}{y}{(\outputp{x}{y} | @{y})) | P}}}
	  | {(\prefix{x}{y}{(\outputp{x}{y} | @{y})) | P}} & \nonumber\\
	\red
	& \ldots & \nonumber\\
	\red^*
	& P | P | \ldots & \nonumber
\end{eqnarray}

Of course, this encoding, as an implementation, runs away, unfolding
$\bangp{P}$ eagerly. A lazier and more implementable replication
operator, restricted to input-guarded processes, may be obtained as follows.

\begin{eqnarray}
\bangp{\prefix{u}{v}{P}} 
	:= 
	\binpar{\lift{x}{\prefix{u}{v}{(\binpar{D(x)}{P})}}}{D(x)} \nonumber
\end{eqnarray}

\begin{remark}
  Note that the lazier definition still does not deal with summation
  or mixed summation (i.e. sums over input and output). The reader is
  invited to construct definitions of replication that deal with these
  features. 

  Further, the definitions are parameterized in a name, $x$. Can you,
  gentle reader, make a definition that eliminates this parameter and
  guarantees no accidental interaction between the replication
  machinery and the process being replicated -- i.e. no accidental
  sharing of names used by the process to get its work done and the
  name(s) used by the replication to effect copying. This latter
  revision of the definition of replication is crucial to obtaining
  the expected identity $!!P \sim !P$.
\end{remark}

\begin{remark}\label{rem:paradoxical_combinator}
  The reader familiar with the lambda calculus will have noticed the
  similarity between $D$ and the paradoxical combinator.

  [Ed. note: the existence of this seems to suggest we have to be more
  restrictive on the set of processes and names we admit if we are to
  support no-cloning.]
\end{remark}

\subsubsection{Bisimulation}

The computational dynamics gives rise to another kind of equivalence,
the equivalence of computational behavior. As previously mentioned
this is typically captured \emph{via} some form of bisimulation.

% The notion we use in this paper is weak barbed bisimulation
% \cite{milner91polyadicpi}.

The notion we use in this paper is derived from weak barbed
bisimulation \cite{milner91polyadicpi}. 

\begin{definition}
An \emph{observation relation}, $\downarrow_{\mathcal N}$, over a set
of names, $\mathcal N$, is the smallest relation satisfying the rules
below.

\infrule[Out-barb]{y \in {\mathcal N}, \; x \nameeq y}
		  {\outputp{x}{v} \downarrow_{\mathcal N} x}
\infrule[Par-barb]{\mbox{$P\downarrow_{\mathcal N} x$ or $Q\downarrow_{\mathcal N} x$}}
		  {\binpar{P}{Q} \downarrow_{\mathcal N} x}

We write $P \Downarrow_{\mathcal N} x$ if there is $Q$ such that 
$P \wred Q$ and $Q \downarrow_{\mathcal N} x$.
\end{definition}

\begin{definition}
%\label{def.bbisim}
An  ${\mathcal N}$-\emph{barbed bisimulation} over a set of names, ${\mathcal N}$, is a symmetric binary relation 
${\mathcal S}_{\mathcal N}$ between agents such that $P\rel{S}_{\mathcal N}Q$ implies:
\begin{enumerate}
\item If $P \red P'$ then $Q \wred Q'$ and $P'\rel{S}_{\mathcal N} Q'$.
\item If $P\downarrow_{\mathcal N} x$, then $Q\Downarrow_{\mathcal N} x$.
\end{enumerate}
$P$ is ${\mathcal N}$-barbed bisimilar to $Q$, written
$P \wbbisim_{\mathcal N} Q$, if $P \rel{S}_{\mathcal N} Q$ for some ${\mathcal N}$-barbed bisimulation ${\mathcal S}_{\mathcal N}$.
\end{definition}

$\mathcal{R} \subseteq \pi \times \pi$

$P \mathcal{R} Q => \forall P'. P \red P' \Rightarrow \exists Q'. Q \red Q', P' \mathcal{R} Q'$

$P \vdash x \Rightarrow Q \vdash x$

\begin{mathpar}
  \inferrule*[lab=Out-barb]{x \nameeq y}{{y}!\langle{Q}\rangle \vdash x}
  \and
  \inferrule*[lab=Par-barb]{\mbox{$P\vdash x$ or $Q\vdash x$}}{\binpar{P}{Q} \vdash x}
\end{mathpar}

\subsubsection{Contexts}

One of the principle advantages of computational calculi like the
$\pi$-calculus is a well-defined notion of context,
contextual-equivalence and a correlation between
contextual-equivalence and notions of bisimulation. The notion of
context allows the decomposition of a process into (sub-)process and
its syntactic environment, its context. Thus, a context may be
thought of as a process with a ``hole'' (written $\Box$) in it. The
application of a context $M$ to a process $P$, written $M[P]$, is
tantamount to filling the hole in $M$ with $P$. In this paper we do
not need the full weight of this theory, but do make use of the notion
of context in the proof the main theorem. 

\begin{mathpar}
  \inferrule* [lab=summation] {} {{M_{M},M_{N}} \bc \Box \;|\; x.M_{A} \;|\; M_{M}+M_{N}}
  \and
  \inferrule* [lab=agent] {} {{M_{A}} \bc (\vec{x})M_{P} \;| \; \clift{P_0,\ldots,M_{P},\ldots,P_N}}
  \and \\
  \inferrule* [lab=process] {} {{M_{P}} \bc M_{N} \;| \;P|M_{P} }
\end{mathpar} 

\begin{mathpar}
  \inferrule* [lab=sychronization] {} {M_{N} \bc \Box \;|\; x?M_{F} \;|\; x!M_{C}}
  \and
  \inferrule* [lab=abstraction] {} {{M_{F}} \bc (x)M_{P} }
  \and
  \inferrule* [lab=concretion] {} {{M_{C}} \bc \langle M_{P} \rangle }
  \and \\
  \inferrule* [lab=process] {} {{M_{P}} \bc M_{N} \;| \;P|M_{P} }
\end{mathpar}

\begin{definition}[contextual application] Given a context $M$, and
  process $P$, we define the \emph{contextual application}, $M[P] :=
  M\{P/\Box\}$. That is, the contextual application of M to P is the
  substitution of $P$ for $\Box$ in $M$.
\end{definition}

$\meaningof{-} : L \to \mathcal{P}(\pi)$

\begin{mathpar}
  \inferrule* [lab=collection] {} {\meaningof{true} = \pi, \and \meaningof{~E} = \pi \setminus \meaningof{E}, \and \meaningof{E_{1} \& E_{2}} = \meaningof{E_{1}} \cap \meaningof{E_{2}}}
\end{mathpar}

\begin{mathpar}
  \inferrule* [lab=structure] {} {\meaningof{0} = \{ P \in \pi | P \equiv 0 \}, \and \\ \meaningof{E_1 | E_2} = \{ P \in \pi | P \equiv P_{1} | P_{2}, P_{1} \in \meaningof{E_{1}}, P_{2} \in \meaningof{E_2}\} }
\end{mathpar}

\begin{mathpar}
 \inferrule* [lab=behavior] {} {\meaningof{\langle a?b \rangle E} = \{ P \in \pi | P \equiv Q | u?(y)P', \\ \and \\\\ \and \\ \;\;\; u \in \meaningof{a}, \forall z.P'\{z/y\} \in \meaningof{E\{z/b\}}\}, \and \\ \meaningof{a!E} = \{ P \in \pi | P \equiv Q | x!\langle P' \rangle, x \in \meaningof{a} P' \in \meaningof{E}\} }
\end{mathpar}

\begin{mathpar}
 \inferrule* [lab=nominal] {} {\meaningof{\quotep{E}} = \{ \quotep{P} \in \quotep{\pi} | P \in \meaningof{E} \}, \and \meaningof{\quotep{P}} = \{ \quotep{Q} \in \quotep{\pi} | P \equiv Q \} \and \\ \meaningof{@\quotep{E}} = \{ P \in \pi | P \equiv @x, x \in \meaningof{E} \}}
\end{mathpar}

\begin{eqnarray*}
  \\
  \meaningof{-} : TS \to ST
\end{eqnarray*}

\begin{eqnarray*}
  \\
  L : TS \to ST
\end{eqnarray*}

\begin{eqnarray*}
  \\
  P \models E \iff P \in \meaningof{E}
\end{eqnarray*}

\begin{eqnarray*}
  P \approx_{L} Q \iff \forall E \in L. P \models E \iff Q \models E
\end{eqnarray*}

\begin{eqnarray*}
  P \approx_{K} Q
\end{eqnarray*}

\begin{eqnarray*}
  P \approx Q
\end{eqnarray*}

$\approx_{K} = \approx = \approx_{L}$

\subsubsection{Contextual duality}

Note that contexts extend the quotation operation to a family of
operations from processes to names. Given a context, $M$, we can
define a \emph{nominal context}, $\quotep{M}$ by $\quotep{M}[P] :=
\quotep{M[P]}$. To foreshadow what is to come we observe that these
operations enjoy a duality with processes very much like the duality
between vectors and maps from vectors to scalars.

Further, because the calculus is essentially higher-order, we have a
correspondence between contexts and processes. More specifically,
given a name $x$ and a context $M$ we can construct $M^{*}_{x}$ such
that 

\begin{mathpar}
  M^{*}_{x} | \lift{x}{P} \red M[P]
\end{mathpar}

namely,

\begin{mathpar}
  M^{*}_{x} := x?(u).M[\dropn{u}]
\end{mathpar}

The dependence of $M^{*}_{x}$ on a name makes it an abstraction, 

\begin{mathpar}
  M^{*} := (x)x?(u).M[\dropn{u}]
\end{mathpar}

\subsection{Additional notation}

It will sometimes be convenient to denote the process a name
quotes. We already have the notation $x = \quotep{P}$, but it will be
convenient to introduce an alternate notation, $\procn{x}$, when we
want to emphasize the connection to the use of the name. Note that, by
virtue of name equivalence, $\quotep{\procn{x}} \nameeq x$; so, the
notation is consistent with previous definitions.

Further, because names have structure it is possible to effect
substitutions on the basis of that structure. This means we need to
upgrade our notation for substitutions, which we accomplish by
adapting comprehension notation. Thus,

\begin{mathpar}
  P\{ y / x : x \in S \}
\end{mathpar}

is interpreted to mean the process derived from P by replacing (in a
capture-avoiding manner) each occurrence of $x$ in $S$ by $y$. For example,

\begin{mathpar}
  P\{ \quotep{\procn{x}|\procn{x}} / x : x \in \freenames{P} \}
\end{mathpar}

will replace each (occurrence) of a free name $x$ in $P$ by
$\quotep{\procn{x}|\procn{x}}$.

Also, we will avail ourselves of the notation $x^{L}$ and $x^{R}$ to
denote injections of a name into disjoint copies of the name
space. There are numerous ways to accomplish this. One example can be
found in \cite{MeredithR05}. This notation overloads to vectors of
names: $\vec{x}^{\pi} := (x_{i}^{\pi} \; : \; 0 \leq i < |\vec{x}| )$ where $\pi \in \{L,R\}$.

We also use $P^{\Box} := P|\Box$.

In \cite{MeredithR05} an interpretation of the new operator is
given. It turns out that there are several possible interpretations
all enjoying the requisite algebraic properties of the operator (see
\cite{milner91polyadicpi}). We will therefore make liberal use of
$(\nu\; \vec{x})P$.

% subsection the_syntax_and_semantics_of_the_notation_system (end)   

\input{qm2pi.qmops} 

\input{qm2pi.sterngerlach} 

\input{qm2pi.metric} 

% section concurrent_process_calculi (end)

%\input{qm2pi.proofsketch}

% section proof sketch (end)

%\input{qm2pi.slviaknots} 

% section spatial logic via knots (end)

\input{qm2pi.conclusion}

% section conclusion (end)

%\input{qm2pi.dtcodes} 

% section wiring algorithm (end)

\input{qm2pi.ack} 

% section acknowledgments (end)

\newpage


\bibliographystyle{plain}   
\bibliography{../../biblios/main.bib}

\input{qm2pi.rhodetails}

\end{document}

 

\documentclass[12pt]{llncs}
%\documentclass{jktr}

\usepackage[pdftex]{hyperref}                   
\usepackage {listings}
\usepackage {mathpartir}
\usepackage{bcprules}
%\usepackage{listings}
                       
\usepackage{graphicx} 
%\usepackage[margins=2.5cm,nohead,nofoot]{geometry}
%\usepackage{geometry}
\usepackage{amsfonts}
\usepackage{amstext}
\usepackage{latexsym}
\usepackage{amssymb}
\usepackage{color}


%\include{myPreamble}
\include{qm2pi.local} 

%\ifpdf
%\usepackage[pdftex]{graphicx}
%\else
%\usepackage{graphicx}
%\fi

 % \ifpdf
%  \usepackage{pdfsync}
%  \if


%\title{Brief Article}
%\author{David F. Snyder}
%\author{L.G. Meredith}

%\address{Dept. of Math., Texas State University--San Marcos, San Marcos, TX 78666}
       
\pagestyle{empty}


\begin{document}

\lstset{language=[Objective]Caml,frame=shadowbox}

\input{qm2pi.front}

% section front matter (end)

\input{qm2pi.intro} 
 
% section introduction (end)

% \input{qm2pi.knotations} 

% section notation (end)

\input{qm2pi.process.calculi} 

% section concurrent_process_calculi_and_spatial_logics_ (end)
    
%\input{qm2pi.knots2pi} 

%\input{qm2pi.trefoil} 

%\input{qm2pi.mainthm} 

% subsection basic_interpretation (end)

%\input{qm2pi.rho.presentation} 
\subsection{The syntax and semantics of the notation system}\label{sub:the_syntax_and_semantics_of_the_notation_system} % (fold)

We now summarize a technical presentation of the calculus that
embodies our theory of dynamics. The typical presentation of such a
calculus follows the style of giving generators and relations on
them. The grammar, below, describing term constructors, freely
generates the set of processes, $\Proc$. This set is then quotiented
by a relation known as structural congruence and it is over this set
that the notion of dynamics is expressed. This presentation is
essentially that of \cite{MeredithR05} with the addition of
polyadicity and summation. For readability we have relegated some of
the technical subtleties to an appendix.

\subsubsection{Process grammar}\label{subsub:process_grammar}

\begin{mathpar}
  \inferrule* [lab=synchronization] {} {{M} \bc \pzero \;|\; x?F \;|\; x!C }
  \and
  \inferrule* [lab=abstraction] {} {{F} \bc (x)P}
  \and
  \inferrule* [lab=concretion] {} {{C} \bc \langle Q \rangle}
  \and
  \inferrule* [lab=process] {} {{P,Q} \bc M \;| \;P|Q \;|\; @{x}}
  \and
  \inferrule* [lab=name] {} {{x} \bc \quotep{P}}
\end{mathpar} 

Note that $\vec{x}$ (resp. $\vec{P}$) denotes a vector of names
(resp. processes) of length $|\vec{x}|$ (resp. $|\vec{P}|$). We adopt
the following useful abbreviations.

\begin{mathpar}
   x?(\vec{y}).P := x.(\vec{y})P \and  x\clift{\vec{P}} := x.\clift{\vec{P}}
   \and x!(y) := \lift{x}{\dropn{y}}
   \and \Pi_{i=0}^{n-1}P_i := P_0 | \ldots | P_{n-1}
\end{mathpar}

\subsubsection{Structural congruence}

\paragraph{Free and bound names and alpha-equivalence.} At the
core of structural equivalence is alpha-equivalence which identifies
process that are the same up to a change of variable. Formally, we
recognize the distinction between free and bound names. The free names
of a process, $\freenames{P}$, may be calculated recursively as
follows:

\begin{mathpar}
\freenames{\pzero} := \emptyset
  \and \\
  \freenames{x?(y).P} := \{ x \} \cup (\freenames{P} \setminus \{ y \})
  \and 
  \freenames{x!\langle P \rangle} := \{ x \} \cup \{ P \} 
  \and \\
  \freenames{P|Q} := \freenames{P} \cup \freenames{Q}
  \and \\
  \freenames{@{x}} := \{ x \}
\end{mathpar}

$\pi$
$\quotep{\pi}$

$\freenames{-} : \pi \to \mathcal{P}(\quotep{\pi})$

\begin{eqnarray*}
  \freenames{\pzero} & := & \emptyset \\
  \freenames{x?(y).P} & := & \{ x \} \cup (\freenames{P} \setminus \{ y \}) \\
  \freenames{x!\langle P \rangle} & := & \{ x \} \cup \{ P \} \\
  \freenames{P|Q} & := & \freenames{P} \cup \freenames{Q} \\
  \freenames{\dropn{x}} & := & \{ x \}
\end{eqnarray*}

The bound names of a process, $\boundnames{P}$, are those names occurring in $P$
that are not free. For example, in $x?(y).0$, the name $x$ is free, while $y$ is bound.

\begin{mathpar}
  \inferrule* [lab=monoidal-laws] {} { P|Q \equiv Q|P \and P|0 \equiv P \and P|(Q|R) \equiv (P|Q)|R }
\end{mathpar}

\begin{mathpar}
  \inferrule* [lab=alpha-equivalence] {} { (x)P \equiv (y)P\{y/x\} \and y \not\in \freenames{P} }
\end{mathpar}

\begin{definition}
Then two processes, $P,Q$, are alpha-equivalent if $P = Q\{\vec{y}/\vec{x}\}$ for
some $\vec{x} \in \boundnames{Q},\vec{y} \in \boundnames{P}$, where $Q\{\vec{y}/\vec{x}\}$
denotes the capture-avoiding substitution of $\vec{y}$ for $\vec{x}$ in $Q$.
\end{definition}

\begin{definition}
  The {\em structural congruence} \cite{SangiorgiWalker} , $\equiv$,
  between processes is the least congruence containing
  alpha-equivalence, satisfying the abelian monoid laws
  (associativity, commutativity and $\pzero$ as identity) for parallel
  composition $|$ and for summation $+$.
\end{definition}

\subsection{Name equivalence}

We take name equivalence, written $\nameeq$, to be the smallest
equivalence relation generated by the following rules.

\begin{mathpar}
\inferrule*[lab=Quote-drop]
{ }
{ \quotep{@{x}} \nameeq x }

\inferrule*[lab=Struct-equiv]
{ P \scong Q }
{ \quotep{P} \nameeq \quotep{Q} }
\end{mathpar}

The astute reader will have noticed that the mutual recursion of names
and processes imposes a mutual recursion on alpha-equivalence and
structural equivalence via name-equivalence. Fortunately, all of this
works out pleasantly and we may calculate in the natural way, free of
concern. The reader interested in the details is referred to the
appendix \ref{appendix:rho_details}.

\subsection{Substitution}

We use $\Proc$ for the set of processes, $\QProc$ for the set of
names, and $\id{\{}\vec{y} / \vec{x} \id{\}}$ to denote partial maps,
$s : \QProc \rightarrow \QProc$. A map, $s$ lifts, uniquely, to a map
on process terms, $\widehat{s} : \Proc \rightarrow \Proc$ by the
following equations.

\begin{mathpar}
  (0) \psubstp{Q}{P} := 0 \\
  (R \juxtap S) \psubstp{Q}{P}
  :=    
  (R)\psubstp{Q}{P} \juxtap (S) \psubstp{Q}{P} \\
  (x?(y).R) \psubstp{Q}{P}    
  :=    
  (x)\substp{Q}{P} (z)\concat( (R \psubstn{z}{y}) \psubstp{Q}{P} ) \\
  (\lift{x}{R}) \psubstp{Q}{P}  
  :=
  \lift{(x)\substp{Q}{P}}{ R \psubstp{Q}{P} } \\
%   (\dropn{x})  \psubstp{Q}{P}       
%   := 
%   \left\{ 
%     \begin{array}{ccc} 
%       \dropn{\quotep{Q}} & & x \nameeq \quotep{P} \\
%       \dropn{x} & & otherwise \\
%     \end{array}
%   \right. 
  (\dropn{x})  \psubstp{Q}{P}       
  := 
  \left\{ 
    \begin{array}{ccc} 
      Q & & x \nameeq \quotep{P} \\
      \dropn{x} & & otherwise \\
    \end{array}
  \right.
\end{mathpar}
 

where

\begin{eqnarray}
  (x)\id{\{} \lpquote Q \rpquote / \lpquote P \rpquote \id{\}}            = 
  \left\{ 
    \begin{array}{ccc}
      \lpquote Q \rpquote & & x \nameeq \lpquote P \rpquote \\
      x & & otherwise \\
    \end{array}
  \right. \nonumber
\end{eqnarray}

and $z$ is chosen distinct from $\quotep{P}$, $\quotep{Q}$, the free
names in $Q$, and all the names in $R$. Our $\alpha$-equivalence will
be built in the standard way from this substitution.

\begin{remark}\label{rem:no_self_referential_names}
  One consequence of these definitions is that $\forall P. \quotep{P}
  \not\in \freenames{P}$.
\end{remark}

\subsection{ Dynamic quote: an example }

Anticipating something of what's to come, consider applying the
substitution, $\widehat{\id{\{}u / z \id{\}}}$, to the following pair
of processes, $\lift{w}{y!(z)}$ and $w[ \lpquote y!(z) \rpquote ]$.

\begin{eqnarray}
	\lift{w}{y!(z)}\widehat{\id{\{}u / z \id{\}}}
		& = &
		\lift{w}{y!(u)} \nonumber\\
	w[ \lpquote y!(z) \rpquote ] \widehat{ \id{\{}u / z \id{\}} }
		& = &
		w[ \lpquote y!(z) \rpquote ] \nonumber
\end{eqnarray}

Because the body of the process between quotes is impervious to
substitution, we get radically different answers. In fact, by
examining the first process in an input context,
e.g. $x?(z).\lift{w}{y!(z)}$, we see that the process under the lift
operator may be shaped by prefixed inputs binding a name inside it. In
this sense, the lift operator will be seen as a way to dynamically
construct processes before reifying them as names.

Finally equipped with these standard features we can present the
dynamics of the calculus.

\subsubsection{Operational semantics} 

Finally, we introduce the computational dynamics. What marks these
algebras as distinct from other more traditionally studied algebraic
structures, e.g. vector spaces or polynomial rings, is the manner in
which dynamics is captured. In traditional structures, dynamics is typically
expressed through morphisms between such structures, as in linear maps
between vector spaces or morphisms between rings. In algebras
associated with the semantics of computation, the dynamics is
expressed as part of the algebraic structure itself, through a
reduction reduction relation typically denoted by $\red$. Below, we
give a recursive presentation of this relation for the calculus used
in the encoding.

$\red \subseteq \pi \times \pi$
$\red : \pi \to \mathcal{P}(\pi)$

\begin{mathpar}
  \inferrule* [lab=Comm] { \textsf{match}( x_{src}, x_{trgt} ) } { x_{trgt}?(y)P \; | \; x_{src}!\langle {Q} \rangle \red P\{\quotep{Q}/y}\} }
  \and \\
  \inferrule* [lab=Par] {{P} \red {P}'} {{{P} | {Q}} \red {{P}' | {Q}}}
  \and
  \inferrule* [lab=Equiv]{{{P} \scong {P}'} \andalso {{P}' \red {Q}'} \andalso {{Q}' \scong {Q}}}{{P} \red {Q}}
\end{mathpar}

\begin{eqnarray*}
  match_{\equiv} (\quotep{P},\quotep{Q}) & := & P \equiv Q \\
  match_{\dagger}(\quotep{P},\quotep{Q}) & := & \forall R. P|Q \red^{*} R => R \red^{*} 0 \\
  match_{K}(\quotep{P},\quotep{Q}) & := & K \mbox{ for some context } K
\end{eqnarray*}

$u?(x)P | u!\langle Q \rangle \red P\{\quotep{Q}/x\}$

%We write $\wred$ for $\red^*$, and $P\red$ if $\exists Q $ such that $ P \red Q$.
We write $P\red$ if $\exists Q $ such that $ P \red Q$ and $P\not\red$, otherwise.

\section{Replication}

As mentioned before, it is known that replication (and hence
recursion) can be implemented in a higher-order process algebra
\cite{SangiorgiWalker}. As our first example of calculation with the
machinery thus far presented we give the construction explicitly in
the {\rhoc}.

\begin{eqnarray}
	D_{x} & := & \prefix{x}{y}{(\binpar{\outputp{x}{y}}{@{y}})} \nonumber\\
	\bangp_{x}{P} & := & \binpar{{x}!\langle{\binpar{D_{x}}{P}}\rangle}{D_{x}} \nonumber
\end{eqnarray}

\begin{eqnarray}
	\bangp_{x}{P} & & \nonumber\\
	=
	& {x}!\langle{(\prefix{x}{y}{(\outputp{x}{y} | @{y})) | P}}\rangle 
	      | \prefix{x}{y}{(\outputp{x}{y} | @{y})} & \nonumber\\
	\red
	& (\outputp{x}{y} | @{y})\substn{\quotep{(\prefix{x}{y}{(@{y} | \outputp{x}{y})) | P}}}{y} & \nonumber\\
	=
	& \outputp{x}{\quotep{(\prefix{x}{y}{(\outputp{x}{y} | @{y})) | P}}}
	  | {(\prefix{x}{y}{(\outputp{x}{y} | @{y})) | P}} & \nonumber\\
	\red
	& \ldots & \nonumber\\
	\red^*
	& P | P | \ldots & \nonumber
\end{eqnarray}

Of course, this encoding, as an implementation, runs away, unfolding
$\bangp{P}$ eagerly. A lazier and more implementable replication
operator, restricted to input-guarded processes, may be obtained as follows.

\begin{eqnarray}
\bangp{\prefix{u}{v}{P}} 
	:= 
	\binpar{\lift{x}{\prefix{u}{v}{(\binpar{D(x)}{P})}}}{D(x)} \nonumber
\end{eqnarray}

\begin{remark}
  Note that the lazier definition still does not deal with summation
  or mixed summation (i.e. sums over input and output). The reader is
  invited to construct definitions of replication that deal with these
  features. 

  Further, the definitions are parameterized in a name, $x$. Can you,
  gentle reader, make a definition that eliminates this parameter and
  guarantees no accidental interaction between the replication
  machinery and the process being replicated -- i.e. no accidental
  sharing of names used by the process to get its work done and the
  name(s) used by the replication to effect copying. This latter
  revision of the definition of replication is crucial to obtaining
  the expected identity $!!P \sim !P$.
\end{remark}

\begin{remark}\label{rem:paradoxical_combinator}
  The reader familiar with the lambda calculus will have noticed the
  similarity between $D$ and the paradoxical combinator.

  [Ed. note: the existence of this seems to suggest we have to be more
  restrictive on the set of processes and names we admit if we are to
  support no-cloning.]
\end{remark}

\subsubsection{Bisimulation}

The computational dynamics gives rise to another kind of equivalence,
the equivalence of computational behavior. As previously mentioned
this is typically captured \emph{via} some form of bisimulation.

% The notion we use in this paper is weak barbed bisimulation
% \cite{milner91polyadicpi}.

The notion we use in this paper is derived from weak barbed
bisimulation \cite{milner91polyadicpi}. 

\begin{definition}
An \emph{observation relation}, $\downarrow_{\mathcal N}$, over a set
of names, $\mathcal N$, is the smallest relation satisfying the rules
below.

\infrule[Out-barb]{y \in {\mathcal N}, \; x \nameeq y}
		  {\outputp{x}{v} \downarrow_{\mathcal N} x}
\infrule[Par-barb]{\mbox{$P\downarrow_{\mathcal N} x$ or $Q\downarrow_{\mathcal N} x$}}
		  {\binpar{P}{Q} \downarrow_{\mathcal N} x}

We write $P \Downarrow_{\mathcal N} x$ if there is $Q$ such that 
$P \wred Q$ and $Q \downarrow_{\mathcal N} x$.
\end{definition}

\begin{definition}
%\label{def.bbisim}
An  ${\mathcal N}$-\emph{barbed bisimulation} over a set of names, ${\mathcal N}$, is a symmetric binary relation 
${\mathcal S}_{\mathcal N}$ between agents such that $P\rel{S}_{\mathcal N}Q$ implies:
\begin{enumerate}
\item If $P \red P'$ then $Q \wred Q'$ and $P'\rel{S}_{\mathcal N} Q'$.
\item If $P\downarrow_{\mathcal N} x$, then $Q\Downarrow_{\mathcal N} x$.
\end{enumerate}
$P$ is ${\mathcal N}$-barbed bisimilar to $Q$, written
$P \wbbisim_{\mathcal N} Q$, if $P \rel{S}_{\mathcal N} Q$ for some ${\mathcal N}$-barbed bisimulation ${\mathcal S}_{\mathcal N}$.
\end{definition}

$\mathcal{R} \subseteq \pi \times \pi$

$P \mathcal{R} Q => \forall P'. P \red P' \Rightarrow \exists Q'. Q \red Q', P' \mathcal{R} Q'$

$P \vdash x \Rightarrow Q \vdash x$

\begin{mathpar}
  \inferrule*[lab=Out-barb]{x \nameeq y}{{y}!\langle{Q}\rangle \vdash x}
  \and
  \inferrule*[lab=Par-barb]{\mbox{$P\vdash x$ or $Q\vdash x$}}{\binpar{P}{Q} \vdash x}
\end{mathpar}

\subsubsection{Contexts}

One of the principle advantages of computational calculi like the
$\pi$-calculus is a well-defined notion of context,
contextual-equivalence and a correlation between
contextual-equivalence and notions of bisimulation. The notion of
context allows the decomposition of a process into (sub-)process and
its syntactic environment, its context. Thus, a context may be
thought of as a process with a ``hole'' (written $\Box$) in it. The
application of a context $M$ to a process $P$, written $M[P]$, is
tantamount to filling the hole in $M$ with $P$. In this paper we do
not need the full weight of this theory, but do make use of the notion
of context in the proof the main theorem. 

\begin{mathpar}
  \inferrule* [lab=summation] {} {{M_{M},M_{N}} \bc \Box \;|\; x.M_{A} \;|\; M_{M}+M_{N}}
  \and
  \inferrule* [lab=agent] {} {{M_{A}} \bc (\vec{x})M_{P} \;| \; \clift{P_0,\ldots,M_{P},\ldots,P_N}}
  \and \\
  \inferrule* [lab=process] {} {{M_{P}} \bc M_{N} \;| \;P|M_{P} }
\end{mathpar} 

\begin{mathpar}
  \inferrule* [lab=sychronization] {} {M_{N} \bc \Box \;|\; x?M_{F} \;|\; x!M_{C}}
  \and
  \inferrule* [lab=abstraction] {} {{M_{F}} \bc (x)M_{P} }
  \and
  \inferrule* [lab=concretion] {} {{M_{C}} \bc \langle M_{P} \rangle }
  \and \\
  \inferrule* [lab=process] {} {{M_{P}} \bc M_{N} \;| \;P|M_{P} }
\end{mathpar}

\begin{definition}[contextual application] Given a context $M$, and
  process $P$, we define the \emph{contextual application}, $M[P] :=
  M\{P/\Box\}$. That is, the contextual application of M to P is the
  substitution of $P$ for $\Box$ in $M$.
\end{definition}

$\meaningof{-} : L \to \mathcal{P}(\pi)$

\begin{mathpar}
  \inferrule* [lab=collection] {} {\meaningof{true} = \pi, \and \meaningof{~E} = \pi \setminus \meaningof{E}, \and \meaningof{E_{1} \& E_{2}} = \meaningof{E_{1}} \cap \meaningof{E_{2}}}
\end{mathpar}

\begin{mathpar}
  \inferrule* [lab=structure] {} {\meaningof{0} = \{ P \in \pi | P \equiv 0 \}, \and \\ \meaningof{E_1 | E_2} = \{ P \in \pi | P \equiv P_{1} | P_{2}, P_{1} \in \meaningof{E_{1}}, P_{2} \in \meaningof{E_2}\} }
\end{mathpar}

\begin{mathpar}
 \inferrule* [lab=behavior] {} {\meaningof{\langle a?b \rangle E} = \{ P \in \pi | P \equiv Q | u?(y)P', \\ \and \\\\ \and \\ \;\;\; u \in \meaningof{a}, \forall z.P'\{z/y\} \in \meaningof{E\{z/b\}}\}, \and \\ \meaningof{a!E} = \{ P \in \pi | P \equiv Q | x!\langle P' \rangle, x \in \meaningof{a} P' \in \meaningof{E}\} }
\end{mathpar}

\begin{mathpar}
 \inferrule* [lab=nominal] {} {\meaningof{\quotep{E}} = \{ \quotep{P} \in \quotep{\pi} | P \in \meaningof{E} \}, \and \meaningof{\quotep{P}} = \{ \quotep{Q} \in \quotep{\pi} | P \equiv Q \} \and \\ \meaningof{@\quotep{E}} = \{ P \in \pi | P \equiv @x, x \in \meaningof{E} \}}
\end{mathpar}

\begin{eqnarray*}
  \\
  \meaningof{-} : TS \to ST
\end{eqnarray*}

\begin{eqnarray*}
  \\
  L : TS \to ST
\end{eqnarray*}

\begin{eqnarray*}
  \\
  P \models E \iff P \in \meaningof{E}
\end{eqnarray*}

\begin{eqnarray*}
  P \approx_{L} Q \iff \forall E \in L. P \models E \iff Q \models E
\end{eqnarray*}

\begin{eqnarray*}
  P \approx_{K} Q
\end{eqnarray*}

\begin{eqnarray*}
  P \approx Q
\end{eqnarray*}

$\approx_{K} = \approx = \approx_{L}$

\subsubsection{Contextual duality}

Note that contexts extend the quotation operation to a family of
operations from processes to names. Given a context, $M$, we can
define a \emph{nominal context}, $\quotep{M}$ by $\quotep{M}[P] :=
\quotep{M[P]}$. To foreshadow what is to come we observe that these
operations enjoy a duality with processes very much like the duality
between vectors and maps from vectors to scalars.

Further, because the calculus is essentially higher-order, we have a
correspondence between contexts and processes. More specifically,
given a name $x$ and a context $M$ we can construct $M^{*}_{x}$ such
that 

\begin{mathpar}
  M^{*}_{x} | \lift{x}{P} \red M[P]
\end{mathpar}

namely,

\begin{mathpar}
  M^{*}_{x} := x?(u).M[\dropn{u}]
\end{mathpar}

The dependence of $M^{*}_{x}$ on a name makes it an abstraction, 

\begin{mathpar}
  M^{*} := (x)x?(u).M[\dropn{u}]
\end{mathpar}

\subsection{Additional notation}

It will sometimes be convenient to denote the process a name
quotes. We already have the notation $x = \quotep{P}$, but it will be
convenient to introduce an alternate notation, $\procn{x}$, when we
want to emphasize the connection to the use of the name. Note that, by
virtue of name equivalence, $\quotep{\procn{x}} \nameeq x$; so, the
notation is consistent with previous definitions.

Further, because names have structure it is possible to effect
substitutions on the basis of that structure. This means we need to
upgrade our notation for substitutions, which we accomplish by
adapting comprehension notation. Thus,

\begin{mathpar}
  P\{ y / x : x \in S \}
\end{mathpar}

is interpreted to mean the process derived from P by replacing (in a
capture-avoiding manner) each occurrence of $x$ in $S$ by $y$. For example,

\begin{mathpar}
  P\{ \quotep{\procn{x}|\procn{x}} / x : x \in \freenames{P} \}
\end{mathpar}

will replace each (occurrence) of a free name $x$ in $P$ by
$\quotep{\procn{x}|\procn{x}}$.

Also, we will avail ourselves of the notation $x^{L}$ and $x^{R}$ to
denote injections of a name into disjoint copies of the name
space. There are numerous ways to accomplish this. One example can be
found in \cite{MeredithR05}. This notation overloads to vectors of
names: $\vec{x}^{\pi} := (x_{i}^{\pi} \; : \; 0 \leq i < |\vec{x}| )$ where $\pi \in \{L,R\}$.

We also use $P^{\Box} := P|\Box$.

In \cite{MeredithR05} an interpretation of the new operator is
given. It turns out that there are several possible interpretations
all enjoying the requisite algebraic properties of the operator (see
\cite{milner91polyadicpi}). We will therefore make liberal use of
$(\nu\; \vec{x})P$.

% subsection the_syntax_and_semantics_of_the_notation_system (end)   

\input{qm2pi.qmops} 

\input{qm2pi.sterngerlach} 

\input{qm2pi.metric} 

% section concurrent_process_calculi (end)

%\input{qm2pi.proofsketch}

% section proof sketch (end)

%\input{qm2pi.slviaknots} 

% section spatial logic via knots (end)

\input{qm2pi.conclusion}

% section conclusion (end)

%\input{qm2pi.dtcodes} 

% section wiring algorithm (end)

\input{qm2pi.ack} 

% section acknowledgments (end)

\newpage


\bibliographystyle{plain}   
\bibliography{../../biblios/main.bib}

\input{qm2pi.rhodetails}

\end{document}

 

% section concurrent_process_calculi (end)

%\documentclass[12pt]{llncs}
%\documentclass{jktr}

\usepackage[pdftex]{hyperref}                   
\usepackage {listings}
\usepackage {mathpartir}
\usepackage{bcprules}
%\usepackage{listings}
                       
\usepackage{graphicx} 
%\usepackage[margins=2.5cm,nohead,nofoot]{geometry}
%\usepackage{geometry}
\usepackage{amsfonts}
\usepackage{amstext}
\usepackage{latexsym}
\usepackage{amssymb}
\usepackage{color}


%\include{myPreamble}
\include{qm2pi.local} 

%\ifpdf
%\usepackage[pdftex]{graphicx}
%\else
%\usepackage{graphicx}
%\fi

 % \ifpdf
%  \usepackage{pdfsync}
%  \if


%\title{Brief Article}
%\author{David F. Snyder}
%\author{L.G. Meredith}

%\address{Dept. of Math., Texas State University--San Marcos, San Marcos, TX 78666}
       
\pagestyle{empty}


\begin{document}

\lstset{language=[Objective]Caml,frame=shadowbox}

\input{qm2pi.front}

% section front matter (end)

\input{qm2pi.intro} 
 
% section introduction (end)

% \input{qm2pi.knotations} 

% section notation (end)

\input{qm2pi.process.calculi} 

% section concurrent_process_calculi_and_spatial_logics_ (end)
    
%\input{qm2pi.knots2pi} 

%\input{qm2pi.trefoil} 

%\input{qm2pi.mainthm} 

% subsection basic_interpretation (end)

%\input{qm2pi.rho.presentation} 
\subsection{The syntax and semantics of the notation system}\label{sub:the_syntax_and_semantics_of_the_notation_system} % (fold)

We now summarize a technical presentation of the calculus that
embodies our theory of dynamics. The typical presentation of such a
calculus follows the style of giving generators and relations on
them. The grammar, below, describing term constructors, freely
generates the set of processes, $\Proc$. This set is then quotiented
by a relation known as structural congruence and it is over this set
that the notion of dynamics is expressed. This presentation is
essentially that of \cite{MeredithR05} with the addition of
polyadicity and summation. For readability we have relegated some of
the technical subtleties to an appendix.

\subsubsection{Process grammar}\label{subsub:process_grammar}

\begin{mathpar}
  \inferrule* [lab=synchronization] {} {{M} \bc \pzero \;|\; x?F \;|\; x!C }
  \and
  \inferrule* [lab=abstraction] {} {{F} \bc (x)P}
  \and
  \inferrule* [lab=concretion] {} {{C} \bc \langle Q \rangle}
  \and
  \inferrule* [lab=process] {} {{P,Q} \bc M \;| \;P|Q \;|\; @{x}}
  \and
  \inferrule* [lab=name] {} {{x} \bc \quotep{P}}
\end{mathpar} 

Note that $\vec{x}$ (resp. $\vec{P}$) denotes a vector of names
(resp. processes) of length $|\vec{x}|$ (resp. $|\vec{P}|$). We adopt
the following useful abbreviations.

\begin{mathpar}
   x?(\vec{y}).P := x.(\vec{y})P \and  x\clift{\vec{P}} := x.\clift{\vec{P}}
   \and x!(y) := \lift{x}{\dropn{y}}
   \and \Pi_{i=0}^{n-1}P_i := P_0 | \ldots | P_{n-1}
\end{mathpar}

\subsubsection{Structural congruence}

\paragraph{Free and bound names and alpha-equivalence.} At the
core of structural equivalence is alpha-equivalence which identifies
process that are the same up to a change of variable. Formally, we
recognize the distinction between free and bound names. The free names
of a process, $\freenames{P}$, may be calculated recursively as
follows:

\begin{mathpar}
\freenames{\pzero} := \emptyset
  \and \\
  \freenames{x?(y).P} := \{ x \} \cup (\freenames{P} \setminus \{ y \})
  \and 
  \freenames{x!\langle P \rangle} := \{ x \} \cup \{ P \} 
  \and \\
  \freenames{P|Q} := \freenames{P} \cup \freenames{Q}
  \and \\
  \freenames{@{x}} := \{ x \}
\end{mathpar}

$\pi$
$\quotep{\pi}$

$\freenames{-} : \pi \to \mathcal{P}(\quotep{\pi})$

\begin{eqnarray*}
  \freenames{\pzero} & := & \emptyset \\
  \freenames{x?(y).P} & := & \{ x \} \cup (\freenames{P} \setminus \{ y \}) \\
  \freenames{x!\langle P \rangle} & := & \{ x \} \cup \{ P \} \\
  \freenames{P|Q} & := & \freenames{P} \cup \freenames{Q} \\
  \freenames{\dropn{x}} & := & \{ x \}
\end{eqnarray*}

The bound names of a process, $\boundnames{P}$, are those names occurring in $P$
that are not free. For example, in $x?(y).0$, the name $x$ is free, while $y$ is bound.

\begin{mathpar}
  \inferrule* [lab=monoidal-laws] {} { P|Q \equiv Q|P \and P|0 \equiv P \and P|(Q|R) \equiv (P|Q)|R }
\end{mathpar}

\begin{mathpar}
  \inferrule* [lab=alpha-equivalence] {} { (x)P \equiv (y)P\{y/x\} \and y \not\in \freenames{P} }
\end{mathpar}

\begin{definition}
Then two processes, $P,Q$, are alpha-equivalent if $P = Q\{\vec{y}/\vec{x}\}$ for
some $\vec{x} \in \boundnames{Q},\vec{y} \in \boundnames{P}$, where $Q\{\vec{y}/\vec{x}\}$
denotes the capture-avoiding substitution of $\vec{y}$ for $\vec{x}$ in $Q$.
\end{definition}

\begin{definition}
  The {\em structural congruence} \cite{SangiorgiWalker} , $\equiv$,
  between processes is the least congruence containing
  alpha-equivalence, satisfying the abelian monoid laws
  (associativity, commutativity and $\pzero$ as identity) for parallel
  composition $|$ and for summation $+$.
\end{definition}

\subsection{Name equivalence}

We take name equivalence, written $\nameeq$, to be the smallest
equivalence relation generated by the following rules.

\begin{mathpar}
\inferrule*[lab=Quote-drop]
{ }
{ \quotep{@{x}} \nameeq x }

\inferrule*[lab=Struct-equiv]
{ P \scong Q }
{ \quotep{P} \nameeq \quotep{Q} }
\end{mathpar}

The astute reader will have noticed that the mutual recursion of names
and processes imposes a mutual recursion on alpha-equivalence and
structural equivalence via name-equivalence. Fortunately, all of this
works out pleasantly and we may calculate in the natural way, free of
concern. The reader interested in the details is referred to the
appendix \ref{appendix:rho_details}.

\subsection{Substitution}

We use $\Proc$ for the set of processes, $\QProc$ for the set of
names, and $\id{\{}\vec{y} / \vec{x} \id{\}}$ to denote partial maps,
$s : \QProc \rightarrow \QProc$. A map, $s$ lifts, uniquely, to a map
on process terms, $\widehat{s} : \Proc \rightarrow \Proc$ by the
following equations.

\begin{mathpar}
  (0) \psubstp{Q}{P} := 0 \\
  (R \juxtap S) \psubstp{Q}{P}
  :=    
  (R)\psubstp{Q}{P} \juxtap (S) \psubstp{Q}{P} \\
  (x?(y).R) \psubstp{Q}{P}    
  :=    
  (x)\substp{Q}{P} (z)\concat( (R \psubstn{z}{y}) \psubstp{Q}{P} ) \\
  (\lift{x}{R}) \psubstp{Q}{P}  
  :=
  \lift{(x)\substp{Q}{P}}{ R \psubstp{Q}{P} } \\
%   (\dropn{x})  \psubstp{Q}{P}       
%   := 
%   \left\{ 
%     \begin{array}{ccc} 
%       \dropn{\quotep{Q}} & & x \nameeq \quotep{P} \\
%       \dropn{x} & & otherwise \\
%     \end{array}
%   \right. 
  (\dropn{x})  \psubstp{Q}{P}       
  := 
  \left\{ 
    \begin{array}{ccc} 
      Q & & x \nameeq \quotep{P} \\
      \dropn{x} & & otherwise \\
    \end{array}
  \right.
\end{mathpar}
 

where

\begin{eqnarray}
  (x)\id{\{} \lpquote Q \rpquote / \lpquote P \rpquote \id{\}}            = 
  \left\{ 
    \begin{array}{ccc}
      \lpquote Q \rpquote & & x \nameeq \lpquote P \rpquote \\
      x & & otherwise \\
    \end{array}
  \right. \nonumber
\end{eqnarray}

and $z$ is chosen distinct from $\quotep{P}$, $\quotep{Q}$, the free
names in $Q$, and all the names in $R$. Our $\alpha$-equivalence will
be built in the standard way from this substitution.

\begin{remark}\label{rem:no_self_referential_names}
  One consequence of these definitions is that $\forall P. \quotep{P}
  \not\in \freenames{P}$.
\end{remark}

\subsection{ Dynamic quote: an example }

Anticipating something of what's to come, consider applying the
substitution, $\widehat{\id{\{}u / z \id{\}}}$, to the following pair
of processes, $\lift{w}{y!(z)}$ and $w[ \lpquote y!(z) \rpquote ]$.

\begin{eqnarray}
	\lift{w}{y!(z)}\widehat{\id{\{}u / z \id{\}}}
		& = &
		\lift{w}{y!(u)} \nonumber\\
	w[ \lpquote y!(z) \rpquote ] \widehat{ \id{\{}u / z \id{\}} }
		& = &
		w[ \lpquote y!(z) \rpquote ] \nonumber
\end{eqnarray}

Because the body of the process between quotes is impervious to
substitution, we get radically different answers. In fact, by
examining the first process in an input context,
e.g. $x?(z).\lift{w}{y!(z)}$, we see that the process under the lift
operator may be shaped by prefixed inputs binding a name inside it. In
this sense, the lift operator will be seen as a way to dynamically
construct processes before reifying them as names.

Finally equipped with these standard features we can present the
dynamics of the calculus.

\subsubsection{Operational semantics} 

Finally, we introduce the computational dynamics. What marks these
algebras as distinct from other more traditionally studied algebraic
structures, e.g. vector spaces or polynomial rings, is the manner in
which dynamics is captured. In traditional structures, dynamics is typically
expressed through morphisms between such structures, as in linear maps
between vector spaces or morphisms between rings. In algebras
associated with the semantics of computation, the dynamics is
expressed as part of the algebraic structure itself, through a
reduction reduction relation typically denoted by $\red$. Below, we
give a recursive presentation of this relation for the calculus used
in the encoding.

$\red \subseteq \pi \times \pi$
$\red : \pi \to \mathcal{P}(\pi)$

\begin{mathpar}
  \inferrule* [lab=Comm] { \textsf{match}( x_{src}, x_{trgt} ) } { x_{trgt}?(y)P \; | \; x_{src}!\langle {Q} \rangle \red P\{\quotep{Q}/y}\} }
  \and \\
  \inferrule* [lab=Par] {{P} \red {P}'} {{{P} | {Q}} \red {{P}' | {Q}}}
  \and
  \inferrule* [lab=Equiv]{{{P} \scong {P}'} \andalso {{P}' \red {Q}'} \andalso {{Q}' \scong {Q}}}{{P} \red {Q}}
\end{mathpar}

\begin{eqnarray*}
  match_{\equiv} (\quotep{P},\quotep{Q}) & := & P \equiv Q \\
  match_{\dagger}(\quotep{P},\quotep{Q}) & := & \forall R. P|Q \red^{*} R => R \red^{*} 0 \\
  match_{K}(\quotep{P},\quotep{Q}) & := & K \mbox{ for some context } K
\end{eqnarray*}

$u?(x)P | u!\langle Q \rangle \red P\{\quotep{Q}/x\}$

%We write $\wred$ for $\red^*$, and $P\red$ if $\exists Q $ such that $ P \red Q$.
We write $P\red$ if $\exists Q $ such that $ P \red Q$ and $P\not\red$, otherwise.

\section{Replication}

As mentioned before, it is known that replication (and hence
recursion) can be implemented in a higher-order process algebra
\cite{SangiorgiWalker}. As our first example of calculation with the
machinery thus far presented we give the construction explicitly in
the {\rhoc}.

\begin{eqnarray}
	D_{x} & := & \prefix{x}{y}{(\binpar{\outputp{x}{y}}{@{y}})} \nonumber\\
	\bangp_{x}{P} & := & \binpar{{x}!\langle{\binpar{D_{x}}{P}}\rangle}{D_{x}} \nonumber
\end{eqnarray}

\begin{eqnarray}
	\bangp_{x}{P} & & \nonumber\\
	=
	& {x}!\langle{(\prefix{x}{y}{(\outputp{x}{y} | @{y})) | P}}\rangle 
	      | \prefix{x}{y}{(\outputp{x}{y} | @{y})} & \nonumber\\
	\red
	& (\outputp{x}{y} | @{y})\substn{\quotep{(\prefix{x}{y}{(@{y} | \outputp{x}{y})) | P}}}{y} & \nonumber\\
	=
	& \outputp{x}{\quotep{(\prefix{x}{y}{(\outputp{x}{y} | @{y})) | P}}}
	  | {(\prefix{x}{y}{(\outputp{x}{y} | @{y})) | P}} & \nonumber\\
	\red
	& \ldots & \nonumber\\
	\red^*
	& P | P | \ldots & \nonumber
\end{eqnarray}

Of course, this encoding, as an implementation, runs away, unfolding
$\bangp{P}$ eagerly. A lazier and more implementable replication
operator, restricted to input-guarded processes, may be obtained as follows.

\begin{eqnarray}
\bangp{\prefix{u}{v}{P}} 
	:= 
	\binpar{\lift{x}{\prefix{u}{v}{(\binpar{D(x)}{P})}}}{D(x)} \nonumber
\end{eqnarray}

\begin{remark}
  Note that the lazier definition still does not deal with summation
  or mixed summation (i.e. sums over input and output). The reader is
  invited to construct definitions of replication that deal with these
  features. 

  Further, the definitions are parameterized in a name, $x$. Can you,
  gentle reader, make a definition that eliminates this parameter and
  guarantees no accidental interaction between the replication
  machinery and the process being replicated -- i.e. no accidental
  sharing of names used by the process to get its work done and the
  name(s) used by the replication to effect copying. This latter
  revision of the definition of replication is crucial to obtaining
  the expected identity $!!P \sim !P$.
\end{remark}

\begin{remark}\label{rem:paradoxical_combinator}
  The reader familiar with the lambda calculus will have noticed the
  similarity between $D$ and the paradoxical combinator.

  [Ed. note: the existence of this seems to suggest we have to be more
  restrictive on the set of processes and names we admit if we are to
  support no-cloning.]
\end{remark}

\subsubsection{Bisimulation}

The computational dynamics gives rise to another kind of equivalence,
the equivalence of computational behavior. As previously mentioned
this is typically captured \emph{via} some form of bisimulation.

% The notion we use in this paper is weak barbed bisimulation
% \cite{milner91polyadicpi}.

The notion we use in this paper is derived from weak barbed
bisimulation \cite{milner91polyadicpi}. 

\begin{definition}
An \emph{observation relation}, $\downarrow_{\mathcal N}$, over a set
of names, $\mathcal N$, is the smallest relation satisfying the rules
below.

\infrule[Out-barb]{y \in {\mathcal N}, \; x \nameeq y}
		  {\outputp{x}{v} \downarrow_{\mathcal N} x}
\infrule[Par-barb]{\mbox{$P\downarrow_{\mathcal N} x$ or $Q\downarrow_{\mathcal N} x$}}
		  {\binpar{P}{Q} \downarrow_{\mathcal N} x}

We write $P \Downarrow_{\mathcal N} x$ if there is $Q$ such that 
$P \wred Q$ and $Q \downarrow_{\mathcal N} x$.
\end{definition}

\begin{definition}
%\label{def.bbisim}
An  ${\mathcal N}$-\emph{barbed bisimulation} over a set of names, ${\mathcal N}$, is a symmetric binary relation 
${\mathcal S}_{\mathcal N}$ between agents such that $P\rel{S}_{\mathcal N}Q$ implies:
\begin{enumerate}
\item If $P \red P'$ then $Q \wred Q'$ and $P'\rel{S}_{\mathcal N} Q'$.
\item If $P\downarrow_{\mathcal N} x$, then $Q\Downarrow_{\mathcal N} x$.
\end{enumerate}
$P$ is ${\mathcal N}$-barbed bisimilar to $Q$, written
$P \wbbisim_{\mathcal N} Q$, if $P \rel{S}_{\mathcal N} Q$ for some ${\mathcal N}$-barbed bisimulation ${\mathcal S}_{\mathcal N}$.
\end{definition}

$\mathcal{R} \subseteq \pi \times \pi$

$P \mathcal{R} Q => \forall P'. P \red P' \Rightarrow \exists Q'. Q \red Q', P' \mathcal{R} Q'$

$P \vdash x \Rightarrow Q \vdash x$

\begin{mathpar}
  \inferrule*[lab=Out-barb]{x \nameeq y}{{y}!\langle{Q}\rangle \vdash x}
  \and
  \inferrule*[lab=Par-barb]{\mbox{$P\vdash x$ or $Q\vdash x$}}{\binpar{P}{Q} \vdash x}
\end{mathpar}

\subsubsection{Contexts}

One of the principle advantages of computational calculi like the
$\pi$-calculus is a well-defined notion of context,
contextual-equivalence and a correlation between
contextual-equivalence and notions of bisimulation. The notion of
context allows the decomposition of a process into (sub-)process and
its syntactic environment, its context. Thus, a context may be
thought of as a process with a ``hole'' (written $\Box$) in it. The
application of a context $M$ to a process $P$, written $M[P]$, is
tantamount to filling the hole in $M$ with $P$. In this paper we do
not need the full weight of this theory, but do make use of the notion
of context in the proof the main theorem. 

\begin{mathpar}
  \inferrule* [lab=summation] {} {{M_{M},M_{N}} \bc \Box \;|\; x.M_{A} \;|\; M_{M}+M_{N}}
  \and
  \inferrule* [lab=agent] {} {{M_{A}} \bc (\vec{x})M_{P} \;| \; \clift{P_0,\ldots,M_{P},\ldots,P_N}}
  \and \\
  \inferrule* [lab=process] {} {{M_{P}} \bc M_{N} \;| \;P|M_{P} }
\end{mathpar} 

\begin{mathpar}
  \inferrule* [lab=sychronization] {} {M_{N} \bc \Box \;|\; x?M_{F} \;|\; x!M_{C}}
  \and
  \inferrule* [lab=abstraction] {} {{M_{F}} \bc (x)M_{P} }
  \and
  \inferrule* [lab=concretion] {} {{M_{C}} \bc \langle M_{P} \rangle }
  \and \\
  \inferrule* [lab=process] {} {{M_{P}} \bc M_{N} \;| \;P|M_{P} }
\end{mathpar}

\begin{definition}[contextual application] Given a context $M$, and
  process $P$, we define the \emph{contextual application}, $M[P] :=
  M\{P/\Box\}$. That is, the contextual application of M to P is the
  substitution of $P$ for $\Box$ in $M$.
\end{definition}

$\meaningof{-} : L \to \mathcal{P}(\pi)$

\begin{mathpar}
  \inferrule* [lab=collection] {} {\meaningof{true} = \pi, \and \meaningof{~E} = \pi \setminus \meaningof{E}, \and \meaningof{E_{1} \& E_{2}} = \meaningof{E_{1}} \cap \meaningof{E_{2}}}
\end{mathpar}

\begin{mathpar}
  \inferrule* [lab=structure] {} {\meaningof{0} = \{ P \in \pi | P \equiv 0 \}, \and \\ \meaningof{E_1 | E_2} = \{ P \in \pi | P \equiv P_{1} | P_{2}, P_{1} \in \meaningof{E_{1}}, P_{2} \in \meaningof{E_2}\} }
\end{mathpar}

\begin{mathpar}
 \inferrule* [lab=behavior] {} {\meaningof{\langle a?b \rangle E} = \{ P \in \pi | P \equiv Q | u?(y)P', \\ \and \\\\ \and \\ \;\;\; u \in \meaningof{a}, \forall z.P'\{z/y\} \in \meaningof{E\{z/b\}}\}, \and \\ \meaningof{a!E} = \{ P \in \pi | P \equiv Q | x!\langle P' \rangle, x \in \meaningof{a} P' \in \meaningof{E}\} }
\end{mathpar}

\begin{mathpar}
 \inferrule* [lab=nominal] {} {\meaningof{\quotep{E}} = \{ \quotep{P} \in \quotep{\pi} | P \in \meaningof{E} \}, \and \meaningof{\quotep{P}} = \{ \quotep{Q} \in \quotep{\pi} | P \equiv Q \} \and \\ \meaningof{@\quotep{E}} = \{ P \in \pi | P \equiv @x, x \in \meaningof{E} \}}
\end{mathpar}

\begin{eqnarray*}
  \\
  \meaningof{-} : TS \to ST
\end{eqnarray*}

\begin{eqnarray*}
  \\
  L : TS \to ST
\end{eqnarray*}

\begin{eqnarray*}
  \\
  P \models E \iff P \in \meaningof{E}
\end{eqnarray*}

\begin{eqnarray*}
  P \approx_{L} Q \iff \forall E \in L. P \models E \iff Q \models E
\end{eqnarray*}

\begin{eqnarray*}
  P \approx_{K} Q
\end{eqnarray*}

\begin{eqnarray*}
  P \approx Q
\end{eqnarray*}

$\approx_{K} = \approx = \approx_{L}$

\subsubsection{Contextual duality}

Note that contexts extend the quotation operation to a family of
operations from processes to names. Given a context, $M$, we can
define a \emph{nominal context}, $\quotep{M}$ by $\quotep{M}[P] :=
\quotep{M[P]}$. To foreshadow what is to come we observe that these
operations enjoy a duality with processes very much like the duality
between vectors and maps from vectors to scalars.

Further, because the calculus is essentially higher-order, we have a
correspondence between contexts and processes. More specifically,
given a name $x$ and a context $M$ we can construct $M^{*}_{x}$ such
that 

\begin{mathpar}
  M^{*}_{x} | \lift{x}{P} \red M[P]
\end{mathpar}

namely,

\begin{mathpar}
  M^{*}_{x} := x?(u).M[\dropn{u}]
\end{mathpar}

The dependence of $M^{*}_{x}$ on a name makes it an abstraction, 

\begin{mathpar}
  M^{*} := (x)x?(u).M[\dropn{u}]
\end{mathpar}

\subsection{Additional notation}

It will sometimes be convenient to denote the process a name
quotes. We already have the notation $x = \quotep{P}$, but it will be
convenient to introduce an alternate notation, $\procn{x}$, when we
want to emphasize the connection to the use of the name. Note that, by
virtue of name equivalence, $\quotep{\procn{x}} \nameeq x$; so, the
notation is consistent with previous definitions.

Further, because names have structure it is possible to effect
substitutions on the basis of that structure. This means we need to
upgrade our notation for substitutions, which we accomplish by
adapting comprehension notation. Thus,

\begin{mathpar}
  P\{ y / x : x \in S \}
\end{mathpar}

is interpreted to mean the process derived from P by replacing (in a
capture-avoiding manner) each occurrence of $x$ in $S$ by $y$. For example,

\begin{mathpar}
  P\{ \quotep{\procn{x}|\procn{x}} / x : x \in \freenames{P} \}
\end{mathpar}

will replace each (occurrence) of a free name $x$ in $P$ by
$\quotep{\procn{x}|\procn{x}}$.

Also, we will avail ourselves of the notation $x^{L}$ and $x^{R}$ to
denote injections of a name into disjoint copies of the name
space. There are numerous ways to accomplish this. One example can be
found in \cite{MeredithR05}. This notation overloads to vectors of
names: $\vec{x}^{\pi} := (x_{i}^{\pi} \; : \; 0 \leq i < |\vec{x}| )$ where $\pi \in \{L,R\}$.

We also use $P^{\Box} := P|\Box$.

In \cite{MeredithR05} an interpretation of the new operator is
given. It turns out that there are several possible interpretations
all enjoying the requisite algebraic properties of the operator (see
\cite{milner91polyadicpi}). We will therefore make liberal use of
$(\nu\; \vec{x})P$.

% subsection the_syntax_and_semantics_of_the_notation_system (end)   

\input{qm2pi.qmops} 

\input{qm2pi.sterngerlach} 

\input{qm2pi.metric} 

% section concurrent_process_calculi (end)

%\input{qm2pi.proofsketch}

% section proof sketch (end)

%\input{qm2pi.slviaknots} 

% section spatial logic via knots (end)

\input{qm2pi.conclusion}

% section conclusion (end)

%\input{qm2pi.dtcodes} 

% section wiring algorithm (end)

\input{qm2pi.ack} 

% section acknowledgments (end)

\newpage


\bibliographystyle{plain}   
\bibliography{../../biblios/main.bib}

\input{qm2pi.rhodetails}

\end{document}



% section proof sketch (end)

%\section{Unlikely characters: spatial logic for
  knots}\label{sub:characteristic_formulae} % (fold)

Associated to the mobile process calculi are a family of logics known
as the Hennessy-Milner logics. These logics typically enjoy a
semantics interpreting formulae as sets of processes that when
factored through the encoding outlined above allows an identification
of classes of knots with logical formulae. In the context of this
encoding the sub-family known as the spatial logics \cite{CairesC03}
\cite{CairesC04} \cite{Caires04} are of particular interest providing
several important features for expressing and reasoning about
properties (i.e. classes) of knots. We hint here at how this may be done.

%\begin{description}
%\item [structural connectives] 
\subsubsection{Structural connectives} The spatial logics enjoy
structural connectives corresponding, at the logical level, to the
parallel composition ($P | Q$) and new name ($(\nu \; x)P$)
connectives for processes. As illustrated in the examples below, these
connectives are extremely expressive given the shape of our encoding.
%\item [decideable satisfaction]

\subsubsection{Decideable satisfaction}
In \cite{Caires04} the satisfaction relation is shown to be decideable
for a rich class of processes. It further turns out that the image of
the our encoding is a proper subset of that class. This result
provides the basis for an algorithm by which to search for knots
enjoying a given property.
%\item [characteristic formulae]

\subsubsection{Characteristic formulae}
In the same paper \cite{Caires04} , Caires presents a means of calculating
characteristic formulae, selecting equivalence classes of processes
up to a pre--specified depth limit on the support set of names. Composed with our
encoding, this characteristic formula can be used to select
characteristic formulae for knots.
%\end{description}

\subsubsection{Spatial logic formulae}

The grammar below (segmented for comprehension) summarizes the syntax
of spatial logic formulae. We employ illustrative examples in the
sequel to provide an intuitive understanding of their meaning
referring the reader to \cite{Caires04} for a more detailed explication
of the semantics.

\begin{mathpar}
  \inferrule* [lab=boolean] {} {{A,B} \bc T \;|\; \neg A \;|\; A \wedge B \;|\; \eta = \eta'}
  \and
  \inferrule* [lab=spatial] {} {|\; \pzero \;|\; A | B \;|\; x \text{\textregistered} A \;|\; \forall x . A \;|\;  H x . A}
  \and
  \inferrule* [lab=behavioral] {} {|\; \alpha . A}
  \and 
  \inferrule* [lab=recursion] {} {|\; X(\vec{u}) \;|\; \mu X(\vec{u}) . A}
  \and
  \inferrule* [lab=action] {} {\alpha \bc \langle x?(\vec{y}) \rangle \;|\; \langle x!(\vec{y}) \rangle \;|\; \langle \tau \rangle}
  \and 
  \inferrule* [lab=name] {} {\eta \bc x \;|\; \tau}
\end{mathpar} 

% subsection characteristic_formulae (end)   	 

\subsection{Example formulae}\label{sub:example_formulae_} % (fold)

\subsubsection{Crossing as formula.}
% 
% \begin{align*}
%   \frac{d}{dx} \sin x &= \cos x 
%   & \frac{d}{dx} e^x &= e^x \\
%   \frac{d}{dx} \cos x &= - \sin x 
%   & \frac{d}{dx} \log x &= \frac{1}{x} \\
% \end{align*} 

\begin{align*}
 \mu C(x_{0},x_{1},y_{0},y_{1},u).&(\langle x_{0}?(z) \rangle(\langle u! \rangle\langle y_{1}!z \rangle C(x_{0},x_{1},y_{0},y_{1},u)) & \\
  & \wedge \langle y_{1}?(z) \rangle (\langle u! \rangle \langle x_{0}!z \rangle C(x_{0},x_{1},y_{0},y_{1},u)) & \\
  & \wedge \langle x_{1}?(z) \rangle (\langle u? \rangle \langle y_{0}!z \rangle C(x_{0},x_{1},y_{0},y_{1},u)) & \\
  & \wedge \langle y_{0}?(z) \rangle (\langle u? \rangle \langle x_{1}!z \rangle C(x_{0},x_{1},y_{0},y_{1},u))) &
\end{align*}

The lexicographical similarity between the shape of this formulae and
the shape of definition of the process representing a crossing reveals
the intuitive meaning of this formulae. It describes the capabilities
of a process that has the right to represent a crossing. For example
it picks out processes that may perform an input on the port $x_0$ in
its initial menu of capabilities. What differentiates the formula
from the process, however, is that the crossing process is the
smallest candidate to satisfy the formula. Infinitely many other
processes -- with internal behavior hidden behind this interface, so
to speak -- also satisfy this formula. Even this simple formula,
then, can be seen to open a new view onto knots, providing a
computational interpretation of \emph{virtual} knots.

Note that this formula is derived by hand. A similar formula can be
derived by employing Caires' calculation of characteristic formula
\cite{Caires04} to the process representing a crossing. In light of
this discussion, we let
$\meaningof{C}_{\phi}(x0,x1,y0,y1,u)$ denote a formula specifying the
dynamics we wish to capture of a crossing. To guarantee we preserve
the shape of the interface and minimal semantics we demand that
$\meaningof{C}_{\phi}(x0,x1,y0,y1,u) \Rightarrow
\textbf{C}(x0,x1,y0,y1,u)$ where $\textbf{C}(x0,x1,y0,y1,u)$ denotes
the formula above.
                            
\subsubsection{Crossing number constraints.}
The moral content of the context lemma (Lemma \ref{context}) is that the notion of
``locality'' in the Reidemeister moves is effectively captured by the
parallel composition operator of the process calculus. This intuition
extends through the logic. Given a formula,
$\meaningof{C}_{\phi}(x0,x1,y0,y1,u)$, we can use the structural
connectives to specify constraints on crossing numbers, such as at
least $n$ crossings, or exactly $n$ crossings.
\begin{mathpar}
  \inferrule* [lab=at-least-n] {} { K^{\geq n}_{\phi}(\vec{xs},\vec{ys}) := \Pi_{i=0}^{n-1} Hu . \meaningof{C}_{\phi}(xs_i,ys_i,u) | T }
  \and 
  \inferrule* [lab=exactly-n] {} { K^{= n}_{\phi}(\vec{xs},\vec{ys}) := \Pi_{i=0}^{n-1} Hu . \meaningof{C}_{\phi}(xs_i,ys_i,u) | \neg (\forall x_0,y_0,x_1,y_1,u . \meaningof{C}_{\phi}(x_0,y_0,x_1,y_1,u) | T) }
\end{mathpar}

To round out this section, recall that the encoding of an $n$-crossing
knot decomposes into a parallel composition of $n$ \emph{copies} of a
crossing process together with a wiring harness. To specify different
knot classes with the same crossing number amounts to specifying
logical constraints on the wiring harness. In the interest of space,
we defer examples to a forthcoming paper. Suffice it to say that both
the conditions ``alternating knot'' and ``contains the tangle
corresponding to 5/3'' are expressible. For example, it is possible to
calculate the characteristic formula of a process corresponding to the
tangle 5/3 and conjoin it into the classifying formula via the
composition connective of the logic.

Finally, we wish to observe that it is entirely within reason to
contemplate a more domain-specific version of spatial logic tailored
to the shape of processes in the image of the encoding. Such a
domain-specific logic would have a better claim to the title formal
language of knot properties.

% subsection example_formulae_ (end)

% section knots_as_processes (end) 

% section spatial logic via knots (end)

\section{Conclusions and future work}

\paragraph{Testing physical space}
You, gentle reader, may wonder why of all the theorems to be proved
given this set up we pick the one above. In some sense it's hardly
central to quantum mechanics. We see it as central in the sense that
it firmly establishes a notion of physical space arising from a notion
of the equivalence of behavior. Relating bisimulation to a metric is a
big step forward, but one is faced with interpreting the relationship
of that metric space to something more physical. Quantum mechanical
notions of ``physical'' space are still far from intuitive, but by
relating this idea of distance as testing to calculations that predict
physical circumstances we are making a not insignificant step forward
toward an understanding of the physical space we inhabit as
essentially dynamic.

\paragraph{Effectivity and simulation}
One of the observations we have yet to make is that the entire program
spelled out here is effective. We have built various interpreters for
the reflective calculus at work in this interpretation. In principle,
then, we can simulate quantum mechanics on a computer. The place where
the simulation may lose fidelity is the infinitely branching summation
for the annihilator.

In this connection i also want to point out that the evaluation style
calculation of the inner product puts the non-determinism of the
summation right at the heart of measurement. This suggests that
Milner's original reduction-based formulation of the dynamics of his
calculi in terms of sums was not just notationally suggestive of a
notion of measure-and-continue but captured some significant part of
the physics.

\paragraph{Quantum continuations}
In light of this last observation i want to point out that the
predominant account of quantum mechanics is missing a key aspect of a
truly compositional story of the physical situation. In a real lab,
when a measurement is made the observation can be made to feed into
another device that then makes another measurement conditioned on the
results of the first. This means that after the superposition was
collapsed the entire experimental set up remained in
superposition. While QM offers a means of writing this down it doesn't
quite line up well with the well-trodden formulation of computation
and continuation that we see so succinctly expressed in Milner's
calculi. This suggests that there might be advantages to this account
of dynamics waiting to be explored.

\paragraph{Quantum logic}
In this connection, we also note that by virtue of having the
Hennessy-Milner construction, we can pull the construction through the
interpretation of QM. This gives us a natural candidate for a quantum
logic that enjoys an extremely tight connection with it's domain of
interpretation, making the construction much less ad hoc (rather it is
the image of functor!).

\paragraph{Quantum probabiity}
i have questions about the basis of the interpretation of inner
product as probability amplitude. In particular, using which
axiomatization of probability theory does the notion of probability
amplitude earn the right to be so dubbed? In other words, where is the
proof that the operation for calculating a probability amplitude (and
then squaring) satisfies the axioms of what it means to calculate a
probability? Even if such a proof exists (i have yet to find it in the
literature), i wonder if it might not be possible to turn things on
their heads. Can we view the calculation of the probability amplitude
as an axiomatization of probability? If so, then the definition we
give for calculating probability amplitude may provide the basis for
an \emph{effective} theory of probability.

\paragraph{Quantum vs ``biological'' information}
Finally, i want to conclude with a more philosophical observation. At
a recent workshop in which QM was a predominant topic i noticed
something about quantum information. The speaker was giving a riveting
discussion of axiomatic QM and showing how properties of ``no
cloning'' and ``no deleting'' emerged as consequences of the
axiomatization. Theorems of this form are necessary to give us a sense
of confidence that our axioms characterize the physical theory. What
struck me, though, was that if quantum information is neither erasable
nor replicable it is markedly different from \emph{life}. Two of the
things we know about life is that

\begin{itemize}
  \item it ends;
  \item to gain some measure of persistence, to transcend it's
    finitude it is imminently copyable.
\end{itemize}

Both of these qualities are summarized succinctly in the aphorism: all
flesh is grass. For me these two kinds of ``information'' -- call them
quantum and biological -- are end points on a spectrum of strategies
for persistence. At one end, we have those curious entities that enjoy
uniqueness and permanence; at the other, we have those who in the face
of a certain end and an uncertain present make a go of passing
something on. To me one of the more remarkable aspects of the latter
strategy is that in the presence of noise (and certain features of
copying) we get a kind of dynamism, a chance for improvement against a
given persistent condition.

% subsection other_calculi_other_bisimulations_and_geometry_as_behavior (end)




% section conclusion (end)

%\documentclass[12pt]{llncs}
%\documentclass{jktr}

\usepackage[pdftex]{hyperref}                   
\usepackage {listings}
\usepackage {mathpartir}
\usepackage{bcprules}
%\usepackage{listings}
                       
\usepackage{graphicx} 
%\usepackage[margins=2.5cm,nohead,nofoot]{geometry}
%\usepackage{geometry}
\usepackage{amsfonts}
\usepackage{amstext}
\usepackage{latexsym}
\usepackage{amssymb}
\usepackage{color}


%\include{myPreamble}
\include{qm2pi.local} 

%\ifpdf
%\usepackage[pdftex]{graphicx}
%\else
%\usepackage{graphicx}
%\fi

 % \ifpdf
%  \usepackage{pdfsync}
%  \if


%\title{Brief Article}
%\author{David F. Snyder}
%\author{L.G. Meredith}

%\address{Dept. of Math., Texas State University--San Marcos, San Marcos, TX 78666}
       
\pagestyle{empty}


\begin{document}

\lstset{language=[Objective]Caml,frame=shadowbox}

\input{qm2pi.front}

% section front matter (end)

\input{qm2pi.intro} 
 
% section introduction (end)

% \input{qm2pi.knotations} 

% section notation (end)

\input{qm2pi.process.calculi} 

% section concurrent_process_calculi_and_spatial_logics_ (end)
    
%\input{qm2pi.knots2pi} 

%\input{qm2pi.trefoil} 

%\input{qm2pi.mainthm} 

% subsection basic_interpretation (end)

%\input{qm2pi.rho.presentation} 
\subsection{The syntax and semantics of the notation system}\label{sub:the_syntax_and_semantics_of_the_notation_system} % (fold)

We now summarize a technical presentation of the calculus that
embodies our theory of dynamics. The typical presentation of such a
calculus follows the style of giving generators and relations on
them. The grammar, below, describing term constructors, freely
generates the set of processes, $\Proc$. This set is then quotiented
by a relation known as structural congruence and it is over this set
that the notion of dynamics is expressed. This presentation is
essentially that of \cite{MeredithR05} with the addition of
polyadicity and summation. For readability we have relegated some of
the technical subtleties to an appendix.

\subsubsection{Process grammar}\label{subsub:process_grammar}

\begin{mathpar}
  \inferrule* [lab=synchronization] {} {{M} \bc \pzero \;|\; x?F \;|\; x!C }
  \and
  \inferrule* [lab=abstraction] {} {{F} \bc (x)P}
  \and
  \inferrule* [lab=concretion] {} {{C} \bc \langle Q \rangle}
  \and
  \inferrule* [lab=process] {} {{P,Q} \bc M \;| \;P|Q \;|\; @{x}}
  \and
  \inferrule* [lab=name] {} {{x} \bc \quotep{P}}
\end{mathpar} 

Note that $\vec{x}$ (resp. $\vec{P}$) denotes a vector of names
(resp. processes) of length $|\vec{x}|$ (resp. $|\vec{P}|$). We adopt
the following useful abbreviations.

\begin{mathpar}
   x?(\vec{y}).P := x.(\vec{y})P \and  x\clift{\vec{P}} := x.\clift{\vec{P}}
   \and x!(y) := \lift{x}{\dropn{y}}
   \and \Pi_{i=0}^{n-1}P_i := P_0 | \ldots | P_{n-1}
\end{mathpar}

\subsubsection{Structural congruence}

\paragraph{Free and bound names and alpha-equivalence.} At the
core of structural equivalence is alpha-equivalence which identifies
process that are the same up to a change of variable. Formally, we
recognize the distinction between free and bound names. The free names
of a process, $\freenames{P}$, may be calculated recursively as
follows:

\begin{mathpar}
\freenames{\pzero} := \emptyset
  \and \\
  \freenames{x?(y).P} := \{ x \} \cup (\freenames{P} \setminus \{ y \})
  \and 
  \freenames{x!\langle P \rangle} := \{ x \} \cup \{ P \} 
  \and \\
  \freenames{P|Q} := \freenames{P} \cup \freenames{Q}
  \and \\
  \freenames{@{x}} := \{ x \}
\end{mathpar}

$\pi$
$\quotep{\pi}$

$\freenames{-} : \pi \to \mathcal{P}(\quotep{\pi})$

\begin{eqnarray*}
  \freenames{\pzero} & := & \emptyset \\
  \freenames{x?(y).P} & := & \{ x \} \cup (\freenames{P} \setminus \{ y \}) \\
  \freenames{x!\langle P \rangle} & := & \{ x \} \cup \{ P \} \\
  \freenames{P|Q} & := & \freenames{P} \cup \freenames{Q} \\
  \freenames{\dropn{x}} & := & \{ x \}
\end{eqnarray*}

The bound names of a process, $\boundnames{P}$, are those names occurring in $P$
that are not free. For example, in $x?(y).0$, the name $x$ is free, while $y$ is bound.

\begin{mathpar}
  \inferrule* [lab=monoidal-laws] {} { P|Q \equiv Q|P \and P|0 \equiv P \and P|(Q|R) \equiv (P|Q)|R }
\end{mathpar}

\begin{mathpar}
  \inferrule* [lab=alpha-equivalence] {} { (x)P \equiv (y)P\{y/x\} \and y \not\in \freenames{P} }
\end{mathpar}

\begin{definition}
Then two processes, $P,Q$, are alpha-equivalent if $P = Q\{\vec{y}/\vec{x}\}$ for
some $\vec{x} \in \boundnames{Q},\vec{y} \in \boundnames{P}$, where $Q\{\vec{y}/\vec{x}\}$
denotes the capture-avoiding substitution of $\vec{y}$ for $\vec{x}$ in $Q$.
\end{definition}

\begin{definition}
  The {\em structural congruence} \cite{SangiorgiWalker} , $\equiv$,
  between processes is the least congruence containing
  alpha-equivalence, satisfying the abelian monoid laws
  (associativity, commutativity and $\pzero$ as identity) for parallel
  composition $|$ and for summation $+$.
\end{definition}

\subsection{Name equivalence}

We take name equivalence, written $\nameeq$, to be the smallest
equivalence relation generated by the following rules.

\begin{mathpar}
\inferrule*[lab=Quote-drop]
{ }
{ \quotep{@{x}} \nameeq x }

\inferrule*[lab=Struct-equiv]
{ P \scong Q }
{ \quotep{P} \nameeq \quotep{Q} }
\end{mathpar}

The astute reader will have noticed that the mutual recursion of names
and processes imposes a mutual recursion on alpha-equivalence and
structural equivalence via name-equivalence. Fortunately, all of this
works out pleasantly and we may calculate in the natural way, free of
concern. The reader interested in the details is referred to the
appendix \ref{appendix:rho_details}.

\subsection{Substitution}

We use $\Proc$ for the set of processes, $\QProc$ for the set of
names, and $\id{\{}\vec{y} / \vec{x} \id{\}}$ to denote partial maps,
$s : \QProc \rightarrow \QProc$. A map, $s$ lifts, uniquely, to a map
on process terms, $\widehat{s} : \Proc \rightarrow \Proc$ by the
following equations.

\begin{mathpar}
  (0) \psubstp{Q}{P} := 0 \\
  (R \juxtap S) \psubstp{Q}{P}
  :=    
  (R)\psubstp{Q}{P} \juxtap (S) \psubstp{Q}{P} \\
  (x?(y).R) \psubstp{Q}{P}    
  :=    
  (x)\substp{Q}{P} (z)\concat( (R \psubstn{z}{y}) \psubstp{Q}{P} ) \\
  (\lift{x}{R}) \psubstp{Q}{P}  
  :=
  \lift{(x)\substp{Q}{P}}{ R \psubstp{Q}{P} } \\
%   (\dropn{x})  \psubstp{Q}{P}       
%   := 
%   \left\{ 
%     \begin{array}{ccc} 
%       \dropn{\quotep{Q}} & & x \nameeq \quotep{P} \\
%       \dropn{x} & & otherwise \\
%     \end{array}
%   \right. 
  (\dropn{x})  \psubstp{Q}{P}       
  := 
  \left\{ 
    \begin{array}{ccc} 
      Q & & x \nameeq \quotep{P} \\
      \dropn{x} & & otherwise \\
    \end{array}
  \right.
\end{mathpar}
 

where

\begin{eqnarray}
  (x)\id{\{} \lpquote Q \rpquote / \lpquote P \rpquote \id{\}}            = 
  \left\{ 
    \begin{array}{ccc}
      \lpquote Q \rpquote & & x \nameeq \lpquote P \rpquote \\
      x & & otherwise \\
    \end{array}
  \right. \nonumber
\end{eqnarray}

and $z$ is chosen distinct from $\quotep{P}$, $\quotep{Q}$, the free
names in $Q$, and all the names in $R$. Our $\alpha$-equivalence will
be built in the standard way from this substitution.

\begin{remark}\label{rem:no_self_referential_names}
  One consequence of these definitions is that $\forall P. \quotep{P}
  \not\in \freenames{P}$.
\end{remark}

\subsection{ Dynamic quote: an example }

Anticipating something of what's to come, consider applying the
substitution, $\widehat{\id{\{}u / z \id{\}}}$, to the following pair
of processes, $\lift{w}{y!(z)}$ and $w[ \lpquote y!(z) \rpquote ]$.

\begin{eqnarray}
	\lift{w}{y!(z)}\widehat{\id{\{}u / z \id{\}}}
		& = &
		\lift{w}{y!(u)} \nonumber\\
	w[ \lpquote y!(z) \rpquote ] \widehat{ \id{\{}u / z \id{\}} }
		& = &
		w[ \lpquote y!(z) \rpquote ] \nonumber
\end{eqnarray}

Because the body of the process between quotes is impervious to
substitution, we get radically different answers. In fact, by
examining the first process in an input context,
e.g. $x?(z).\lift{w}{y!(z)}$, we see that the process under the lift
operator may be shaped by prefixed inputs binding a name inside it. In
this sense, the lift operator will be seen as a way to dynamically
construct processes before reifying them as names.

Finally equipped with these standard features we can present the
dynamics of the calculus.

\subsubsection{Operational semantics} 

Finally, we introduce the computational dynamics. What marks these
algebras as distinct from other more traditionally studied algebraic
structures, e.g. vector spaces or polynomial rings, is the manner in
which dynamics is captured. In traditional structures, dynamics is typically
expressed through morphisms between such structures, as in linear maps
between vector spaces or morphisms between rings. In algebras
associated with the semantics of computation, the dynamics is
expressed as part of the algebraic structure itself, through a
reduction reduction relation typically denoted by $\red$. Below, we
give a recursive presentation of this relation for the calculus used
in the encoding.

$\red \subseteq \pi \times \pi$
$\red : \pi \to \mathcal{P}(\pi)$

\begin{mathpar}
  \inferrule* [lab=Comm] { \textsf{match}( x_{src}, x_{trgt} ) } { x_{trgt}?(y)P \; | \; x_{src}!\langle {Q} \rangle \red P\{\quotep{Q}/y}\} }
  \and \\
  \inferrule* [lab=Par] {{P} \red {P}'} {{{P} | {Q}} \red {{P}' | {Q}}}
  \and
  \inferrule* [lab=Equiv]{{{P} \scong {P}'} \andalso {{P}' \red {Q}'} \andalso {{Q}' \scong {Q}}}{{P} \red {Q}}
\end{mathpar}

\begin{eqnarray*}
  match_{\equiv} (\quotep{P},\quotep{Q}) & := & P \equiv Q \\
  match_{\dagger}(\quotep{P},\quotep{Q}) & := & \forall R. P|Q \red^{*} R => R \red^{*} 0 \\
  match_{K}(\quotep{P},\quotep{Q}) & := & K \mbox{ for some context } K
\end{eqnarray*}

$u?(x)P | u!\langle Q \rangle \red P\{\quotep{Q}/x\}$

%We write $\wred$ for $\red^*$, and $P\red$ if $\exists Q $ such that $ P \red Q$.
We write $P\red$ if $\exists Q $ such that $ P \red Q$ and $P\not\red$, otherwise.

\section{Replication}

As mentioned before, it is known that replication (and hence
recursion) can be implemented in a higher-order process algebra
\cite{SangiorgiWalker}. As our first example of calculation with the
machinery thus far presented we give the construction explicitly in
the {\rhoc}.

\begin{eqnarray}
	D_{x} & := & \prefix{x}{y}{(\binpar{\outputp{x}{y}}{@{y}})} \nonumber\\
	\bangp_{x}{P} & := & \binpar{{x}!\langle{\binpar{D_{x}}{P}}\rangle}{D_{x}} \nonumber
\end{eqnarray}

\begin{eqnarray}
	\bangp_{x}{P} & & \nonumber\\
	=
	& {x}!\langle{(\prefix{x}{y}{(\outputp{x}{y} | @{y})) | P}}\rangle 
	      | \prefix{x}{y}{(\outputp{x}{y} | @{y})} & \nonumber\\
	\red
	& (\outputp{x}{y} | @{y})\substn{\quotep{(\prefix{x}{y}{(@{y} | \outputp{x}{y})) | P}}}{y} & \nonumber\\
	=
	& \outputp{x}{\quotep{(\prefix{x}{y}{(\outputp{x}{y} | @{y})) | P}}}
	  | {(\prefix{x}{y}{(\outputp{x}{y} | @{y})) | P}} & \nonumber\\
	\red
	& \ldots & \nonumber\\
	\red^*
	& P | P | \ldots & \nonumber
\end{eqnarray}

Of course, this encoding, as an implementation, runs away, unfolding
$\bangp{P}$ eagerly. A lazier and more implementable replication
operator, restricted to input-guarded processes, may be obtained as follows.

\begin{eqnarray}
\bangp{\prefix{u}{v}{P}} 
	:= 
	\binpar{\lift{x}{\prefix{u}{v}{(\binpar{D(x)}{P})}}}{D(x)} \nonumber
\end{eqnarray}

\begin{remark}
  Note that the lazier definition still does not deal with summation
  or mixed summation (i.e. sums over input and output). The reader is
  invited to construct definitions of replication that deal with these
  features. 

  Further, the definitions are parameterized in a name, $x$. Can you,
  gentle reader, make a definition that eliminates this parameter and
  guarantees no accidental interaction between the replication
  machinery and the process being replicated -- i.e. no accidental
  sharing of names used by the process to get its work done and the
  name(s) used by the replication to effect copying. This latter
  revision of the definition of replication is crucial to obtaining
  the expected identity $!!P \sim !P$.
\end{remark}

\begin{remark}\label{rem:paradoxical_combinator}
  The reader familiar with the lambda calculus will have noticed the
  similarity between $D$ and the paradoxical combinator.

  [Ed. note: the existence of this seems to suggest we have to be more
  restrictive on the set of processes and names we admit if we are to
  support no-cloning.]
\end{remark}

\subsubsection{Bisimulation}

The computational dynamics gives rise to another kind of equivalence,
the equivalence of computational behavior. As previously mentioned
this is typically captured \emph{via} some form of bisimulation.

% The notion we use in this paper is weak barbed bisimulation
% \cite{milner91polyadicpi}.

The notion we use in this paper is derived from weak barbed
bisimulation \cite{milner91polyadicpi}. 

\begin{definition}
An \emph{observation relation}, $\downarrow_{\mathcal N}$, over a set
of names, $\mathcal N$, is the smallest relation satisfying the rules
below.

\infrule[Out-barb]{y \in {\mathcal N}, \; x \nameeq y}
		  {\outputp{x}{v} \downarrow_{\mathcal N} x}
\infrule[Par-barb]{\mbox{$P\downarrow_{\mathcal N} x$ or $Q\downarrow_{\mathcal N} x$}}
		  {\binpar{P}{Q} \downarrow_{\mathcal N} x}

We write $P \Downarrow_{\mathcal N} x$ if there is $Q$ such that 
$P \wred Q$ and $Q \downarrow_{\mathcal N} x$.
\end{definition}

\begin{definition}
%\label{def.bbisim}
An  ${\mathcal N}$-\emph{barbed bisimulation} over a set of names, ${\mathcal N}$, is a symmetric binary relation 
${\mathcal S}_{\mathcal N}$ between agents such that $P\rel{S}_{\mathcal N}Q$ implies:
\begin{enumerate}
\item If $P \red P'$ then $Q \wred Q'$ and $P'\rel{S}_{\mathcal N} Q'$.
\item If $P\downarrow_{\mathcal N} x$, then $Q\Downarrow_{\mathcal N} x$.
\end{enumerate}
$P$ is ${\mathcal N}$-barbed bisimilar to $Q$, written
$P \wbbisim_{\mathcal N} Q$, if $P \rel{S}_{\mathcal N} Q$ for some ${\mathcal N}$-barbed bisimulation ${\mathcal S}_{\mathcal N}$.
\end{definition}

$\mathcal{R} \subseteq \pi \times \pi$

$P \mathcal{R} Q => \forall P'. P \red P' \Rightarrow \exists Q'. Q \red Q', P' \mathcal{R} Q'$

$P \vdash x \Rightarrow Q \vdash x$

\begin{mathpar}
  \inferrule*[lab=Out-barb]{x \nameeq y}{{y}!\langle{Q}\rangle \vdash x}
  \and
  \inferrule*[lab=Par-barb]{\mbox{$P\vdash x$ or $Q\vdash x$}}{\binpar{P}{Q} \vdash x}
\end{mathpar}

\subsubsection{Contexts}

One of the principle advantages of computational calculi like the
$\pi$-calculus is a well-defined notion of context,
contextual-equivalence and a correlation between
contextual-equivalence and notions of bisimulation. The notion of
context allows the decomposition of a process into (sub-)process and
its syntactic environment, its context. Thus, a context may be
thought of as a process with a ``hole'' (written $\Box$) in it. The
application of a context $M$ to a process $P$, written $M[P]$, is
tantamount to filling the hole in $M$ with $P$. In this paper we do
not need the full weight of this theory, but do make use of the notion
of context in the proof the main theorem. 

\begin{mathpar}
  \inferrule* [lab=summation] {} {{M_{M},M_{N}} \bc \Box \;|\; x.M_{A} \;|\; M_{M}+M_{N}}
  \and
  \inferrule* [lab=agent] {} {{M_{A}} \bc (\vec{x})M_{P} \;| \; \clift{P_0,\ldots,M_{P},\ldots,P_N}}
  \and \\
  \inferrule* [lab=process] {} {{M_{P}} \bc M_{N} \;| \;P|M_{P} }
\end{mathpar} 

\begin{mathpar}
  \inferrule* [lab=sychronization] {} {M_{N} \bc \Box \;|\; x?M_{F} \;|\; x!M_{C}}
  \and
  \inferrule* [lab=abstraction] {} {{M_{F}} \bc (x)M_{P} }
  \and
  \inferrule* [lab=concretion] {} {{M_{C}} \bc \langle M_{P} \rangle }
  \and \\
  \inferrule* [lab=process] {} {{M_{P}} \bc M_{N} \;| \;P|M_{P} }
\end{mathpar}

\begin{definition}[contextual application] Given a context $M$, and
  process $P$, we define the \emph{contextual application}, $M[P] :=
  M\{P/\Box\}$. That is, the contextual application of M to P is the
  substitution of $P$ for $\Box$ in $M$.
\end{definition}

$\meaningof{-} : L \to \mathcal{P}(\pi)$

\begin{mathpar}
  \inferrule* [lab=collection] {} {\meaningof{true} = \pi, \and \meaningof{~E} = \pi \setminus \meaningof{E}, \and \meaningof{E_{1} \& E_{2}} = \meaningof{E_{1}} \cap \meaningof{E_{2}}}
\end{mathpar}

\begin{mathpar}
  \inferrule* [lab=structure] {} {\meaningof{0} = \{ P \in \pi | P \equiv 0 \}, \and \\ \meaningof{E_1 | E_2} = \{ P \in \pi | P \equiv P_{1} | P_{2}, P_{1} \in \meaningof{E_{1}}, P_{2} \in \meaningof{E_2}\} }
\end{mathpar}

\begin{mathpar}
 \inferrule* [lab=behavior] {} {\meaningof{\langle a?b \rangle E} = \{ P \in \pi | P \equiv Q | u?(y)P', \\ \and \\\\ \and \\ \;\;\; u \in \meaningof{a}, \forall z.P'\{z/y\} \in \meaningof{E\{z/b\}}\}, \and \\ \meaningof{a!E} = \{ P \in \pi | P \equiv Q | x!\langle P' \rangle, x \in \meaningof{a} P' \in \meaningof{E}\} }
\end{mathpar}

\begin{mathpar}
 \inferrule* [lab=nominal] {} {\meaningof{\quotep{E}} = \{ \quotep{P} \in \quotep{\pi} | P \in \meaningof{E} \}, \and \meaningof{\quotep{P}} = \{ \quotep{Q} \in \quotep{\pi} | P \equiv Q \} \and \\ \meaningof{@\quotep{E}} = \{ P \in \pi | P \equiv @x, x \in \meaningof{E} \}}
\end{mathpar}

\begin{eqnarray*}
  \\
  \meaningof{-} : TS \to ST
\end{eqnarray*}

\begin{eqnarray*}
  \\
  L : TS \to ST
\end{eqnarray*}

\begin{eqnarray*}
  \\
  P \models E \iff P \in \meaningof{E}
\end{eqnarray*}

\begin{eqnarray*}
  P \approx_{L} Q \iff \forall E \in L. P \models E \iff Q \models E
\end{eqnarray*}

\begin{eqnarray*}
  P \approx_{K} Q
\end{eqnarray*}

\begin{eqnarray*}
  P \approx Q
\end{eqnarray*}

$\approx_{K} = \approx = \approx_{L}$

\subsubsection{Contextual duality}

Note that contexts extend the quotation operation to a family of
operations from processes to names. Given a context, $M$, we can
define a \emph{nominal context}, $\quotep{M}$ by $\quotep{M}[P] :=
\quotep{M[P]}$. To foreshadow what is to come we observe that these
operations enjoy a duality with processes very much like the duality
between vectors and maps from vectors to scalars.

Further, because the calculus is essentially higher-order, we have a
correspondence between contexts and processes. More specifically,
given a name $x$ and a context $M$ we can construct $M^{*}_{x}$ such
that 

\begin{mathpar}
  M^{*}_{x} | \lift{x}{P} \red M[P]
\end{mathpar}

namely,

\begin{mathpar}
  M^{*}_{x} := x?(u).M[\dropn{u}]
\end{mathpar}

The dependence of $M^{*}_{x}$ on a name makes it an abstraction, 

\begin{mathpar}
  M^{*} := (x)x?(u).M[\dropn{u}]
\end{mathpar}

\subsection{Additional notation}

It will sometimes be convenient to denote the process a name
quotes. We already have the notation $x = \quotep{P}$, but it will be
convenient to introduce an alternate notation, $\procn{x}$, when we
want to emphasize the connection to the use of the name. Note that, by
virtue of name equivalence, $\quotep{\procn{x}} \nameeq x$; so, the
notation is consistent with previous definitions.

Further, because names have structure it is possible to effect
substitutions on the basis of that structure. This means we need to
upgrade our notation for substitutions, which we accomplish by
adapting comprehension notation. Thus,

\begin{mathpar}
  P\{ y / x : x \in S \}
\end{mathpar}

is interpreted to mean the process derived from P by replacing (in a
capture-avoiding manner) each occurrence of $x$ in $S$ by $y$. For example,

\begin{mathpar}
  P\{ \quotep{\procn{x}|\procn{x}} / x : x \in \freenames{P} \}
\end{mathpar}

will replace each (occurrence) of a free name $x$ in $P$ by
$\quotep{\procn{x}|\procn{x}}$.

Also, we will avail ourselves of the notation $x^{L}$ and $x^{R}$ to
denote injections of a name into disjoint copies of the name
space. There are numerous ways to accomplish this. One example can be
found in \cite{MeredithR05}. This notation overloads to vectors of
names: $\vec{x}^{\pi} := (x_{i}^{\pi} \; : \; 0 \leq i < |\vec{x}| )$ where $\pi \in \{L,R\}$.

We also use $P^{\Box} := P|\Box$.

In \cite{MeredithR05} an interpretation of the new operator is
given. It turns out that there are several possible interpretations
all enjoying the requisite algebraic properties of the operator (see
\cite{milner91polyadicpi}). We will therefore make liberal use of
$(\nu\; \vec{x})P$.

% subsection the_syntax_and_semantics_of_the_notation_system (end)   

\input{qm2pi.qmops} 

\input{qm2pi.sterngerlach} 

\input{qm2pi.metric} 

% section concurrent_process_calculi (end)

%\input{qm2pi.proofsketch}

% section proof sketch (end)

%\input{qm2pi.slviaknots} 

% section spatial logic via knots (end)

\input{qm2pi.conclusion}

% section conclusion (end)

%\input{qm2pi.dtcodes} 

% section wiring algorithm (end)

\input{qm2pi.ack} 

% section acknowledgments (end)

\newpage


\bibliographystyle{plain}   
\bibliography{../../biblios/main.bib}

\input{qm2pi.rhodetails}

\end{document}

 

% section wiring algorithm (end)

\documentclass[12pt]{llncs}
%\documentclass{jktr}

\usepackage[pdftex]{hyperref}                   
\usepackage {listings}
\usepackage {mathpartir}
\usepackage{bcprules}
%\usepackage{listings}
                       
\usepackage{graphicx} 
%\usepackage[margins=2.5cm,nohead,nofoot]{geometry}
%\usepackage{geometry}
\usepackage{amsfonts}
\usepackage{amstext}
\usepackage{latexsym}
\usepackage{amssymb}
\usepackage{color}


%\include{myPreamble}
\include{qm2pi.local} 

%\ifpdf
%\usepackage[pdftex]{graphicx}
%\else
%\usepackage{graphicx}
%\fi

 % \ifpdf
%  \usepackage{pdfsync}
%  \if


%\title{Brief Article}
%\author{David F. Snyder}
%\author{L.G. Meredith}

%\address{Dept. of Math., Texas State University--San Marcos, San Marcos, TX 78666}
       
\pagestyle{empty}


\begin{document}

\lstset{language=[Objective]Caml,frame=shadowbox}

\input{qm2pi.front}

% section front matter (end)

\input{qm2pi.intro} 
 
% section introduction (end)

% \input{qm2pi.knotations} 

% section notation (end)

\input{qm2pi.process.calculi} 

% section concurrent_process_calculi_and_spatial_logics_ (end)
    
%\input{qm2pi.knots2pi} 

%\input{qm2pi.trefoil} 

%\input{qm2pi.mainthm} 

% subsection basic_interpretation (end)

%\input{qm2pi.rho.presentation} 
\subsection{The syntax and semantics of the notation system}\label{sub:the_syntax_and_semantics_of_the_notation_system} % (fold)

We now summarize a technical presentation of the calculus that
embodies our theory of dynamics. The typical presentation of such a
calculus follows the style of giving generators and relations on
them. The grammar, below, describing term constructors, freely
generates the set of processes, $\Proc$. This set is then quotiented
by a relation known as structural congruence and it is over this set
that the notion of dynamics is expressed. This presentation is
essentially that of \cite{MeredithR05} with the addition of
polyadicity and summation. For readability we have relegated some of
the technical subtleties to an appendix.

\subsubsection{Process grammar}\label{subsub:process_grammar}

\begin{mathpar}
  \inferrule* [lab=synchronization] {} {{M} \bc \pzero \;|\; x?F \;|\; x!C }
  \and
  \inferrule* [lab=abstraction] {} {{F} \bc (x)P}
  \and
  \inferrule* [lab=concretion] {} {{C} \bc \langle Q \rangle}
  \and
  \inferrule* [lab=process] {} {{P,Q} \bc M \;| \;P|Q \;|\; @{x}}
  \and
  \inferrule* [lab=name] {} {{x} \bc \quotep{P}}
\end{mathpar} 

Note that $\vec{x}$ (resp. $\vec{P}$) denotes a vector of names
(resp. processes) of length $|\vec{x}|$ (resp. $|\vec{P}|$). We adopt
the following useful abbreviations.

\begin{mathpar}
   x?(\vec{y}).P := x.(\vec{y})P \and  x\clift{\vec{P}} := x.\clift{\vec{P}}
   \and x!(y) := \lift{x}{\dropn{y}}
   \and \Pi_{i=0}^{n-1}P_i := P_0 | \ldots | P_{n-1}
\end{mathpar}

\subsubsection{Structural congruence}

\paragraph{Free and bound names and alpha-equivalence.} At the
core of structural equivalence is alpha-equivalence which identifies
process that are the same up to a change of variable. Formally, we
recognize the distinction between free and bound names. The free names
of a process, $\freenames{P}$, may be calculated recursively as
follows:

\begin{mathpar}
\freenames{\pzero} := \emptyset
  \and \\
  \freenames{x?(y).P} := \{ x \} \cup (\freenames{P} \setminus \{ y \})
  \and 
  \freenames{x!\langle P \rangle} := \{ x \} \cup \{ P \} 
  \and \\
  \freenames{P|Q} := \freenames{P} \cup \freenames{Q}
  \and \\
  \freenames{@{x}} := \{ x \}
\end{mathpar}

$\pi$
$\quotep{\pi}$

$\freenames{-} : \pi \to \mathcal{P}(\quotep{\pi})$

\begin{eqnarray*}
  \freenames{\pzero} & := & \emptyset \\
  \freenames{x?(y).P} & := & \{ x \} \cup (\freenames{P} \setminus \{ y \}) \\
  \freenames{x!\langle P \rangle} & := & \{ x \} \cup \{ P \} \\
  \freenames{P|Q} & := & \freenames{P} \cup \freenames{Q} \\
  \freenames{\dropn{x}} & := & \{ x \}
\end{eqnarray*}

The bound names of a process, $\boundnames{P}$, are those names occurring in $P$
that are not free. For example, in $x?(y).0$, the name $x$ is free, while $y$ is bound.

\begin{mathpar}
  \inferrule* [lab=monoidal-laws] {} { P|Q \equiv Q|P \and P|0 \equiv P \and P|(Q|R) \equiv (P|Q)|R }
\end{mathpar}

\begin{mathpar}
  \inferrule* [lab=alpha-equivalence] {} { (x)P \equiv (y)P\{y/x\} \and y \not\in \freenames{P} }
\end{mathpar}

\begin{definition}
Then two processes, $P,Q$, are alpha-equivalent if $P = Q\{\vec{y}/\vec{x}\}$ for
some $\vec{x} \in \boundnames{Q},\vec{y} \in \boundnames{P}$, where $Q\{\vec{y}/\vec{x}\}$
denotes the capture-avoiding substitution of $\vec{y}$ for $\vec{x}$ in $Q$.
\end{definition}

\begin{definition}
  The {\em structural congruence} \cite{SangiorgiWalker} , $\equiv$,
  between processes is the least congruence containing
  alpha-equivalence, satisfying the abelian monoid laws
  (associativity, commutativity and $\pzero$ as identity) for parallel
  composition $|$ and for summation $+$.
\end{definition}

\subsection{Name equivalence}

We take name equivalence, written $\nameeq$, to be the smallest
equivalence relation generated by the following rules.

\begin{mathpar}
\inferrule*[lab=Quote-drop]
{ }
{ \quotep{@{x}} \nameeq x }

\inferrule*[lab=Struct-equiv]
{ P \scong Q }
{ \quotep{P} \nameeq \quotep{Q} }
\end{mathpar}

The astute reader will have noticed that the mutual recursion of names
and processes imposes a mutual recursion on alpha-equivalence and
structural equivalence via name-equivalence. Fortunately, all of this
works out pleasantly and we may calculate in the natural way, free of
concern. The reader interested in the details is referred to the
appendix \ref{appendix:rho_details}.

\subsection{Substitution}

We use $\Proc$ for the set of processes, $\QProc$ for the set of
names, and $\id{\{}\vec{y} / \vec{x} \id{\}}$ to denote partial maps,
$s : \QProc \rightarrow \QProc$. A map, $s$ lifts, uniquely, to a map
on process terms, $\widehat{s} : \Proc \rightarrow \Proc$ by the
following equations.

\begin{mathpar}
  (0) \psubstp{Q}{P} := 0 \\
  (R \juxtap S) \psubstp{Q}{P}
  :=    
  (R)\psubstp{Q}{P} \juxtap (S) \psubstp{Q}{P} \\
  (x?(y).R) \psubstp{Q}{P}    
  :=    
  (x)\substp{Q}{P} (z)\concat( (R \psubstn{z}{y}) \psubstp{Q}{P} ) \\
  (\lift{x}{R}) \psubstp{Q}{P}  
  :=
  \lift{(x)\substp{Q}{P}}{ R \psubstp{Q}{P} } \\
%   (\dropn{x})  \psubstp{Q}{P}       
%   := 
%   \left\{ 
%     \begin{array}{ccc} 
%       \dropn{\quotep{Q}} & & x \nameeq \quotep{P} \\
%       \dropn{x} & & otherwise \\
%     \end{array}
%   \right. 
  (\dropn{x})  \psubstp{Q}{P}       
  := 
  \left\{ 
    \begin{array}{ccc} 
      Q & & x \nameeq \quotep{P} \\
      \dropn{x} & & otherwise \\
    \end{array}
  \right.
\end{mathpar}
 

where

\begin{eqnarray}
  (x)\id{\{} \lpquote Q \rpquote / \lpquote P \rpquote \id{\}}            = 
  \left\{ 
    \begin{array}{ccc}
      \lpquote Q \rpquote & & x \nameeq \lpquote P \rpquote \\
      x & & otherwise \\
    \end{array}
  \right. \nonumber
\end{eqnarray}

and $z$ is chosen distinct from $\quotep{P}$, $\quotep{Q}$, the free
names in $Q$, and all the names in $R$. Our $\alpha$-equivalence will
be built in the standard way from this substitution.

\begin{remark}\label{rem:no_self_referential_names}
  One consequence of these definitions is that $\forall P. \quotep{P}
  \not\in \freenames{P}$.
\end{remark}

\subsection{ Dynamic quote: an example }

Anticipating something of what's to come, consider applying the
substitution, $\widehat{\id{\{}u / z \id{\}}}$, to the following pair
of processes, $\lift{w}{y!(z)}$ and $w[ \lpquote y!(z) \rpquote ]$.

\begin{eqnarray}
	\lift{w}{y!(z)}\widehat{\id{\{}u / z \id{\}}}
		& = &
		\lift{w}{y!(u)} \nonumber\\
	w[ \lpquote y!(z) \rpquote ] \widehat{ \id{\{}u / z \id{\}} }
		& = &
		w[ \lpquote y!(z) \rpquote ] \nonumber
\end{eqnarray}

Because the body of the process between quotes is impervious to
substitution, we get radically different answers. In fact, by
examining the first process in an input context,
e.g. $x?(z).\lift{w}{y!(z)}$, we see that the process under the lift
operator may be shaped by prefixed inputs binding a name inside it. In
this sense, the lift operator will be seen as a way to dynamically
construct processes before reifying them as names.

Finally equipped with these standard features we can present the
dynamics of the calculus.

\subsubsection{Operational semantics} 

Finally, we introduce the computational dynamics. What marks these
algebras as distinct from other more traditionally studied algebraic
structures, e.g. vector spaces or polynomial rings, is the manner in
which dynamics is captured. In traditional structures, dynamics is typically
expressed through morphisms between such structures, as in linear maps
between vector spaces or morphisms between rings. In algebras
associated with the semantics of computation, the dynamics is
expressed as part of the algebraic structure itself, through a
reduction reduction relation typically denoted by $\red$. Below, we
give a recursive presentation of this relation for the calculus used
in the encoding.

$\red \subseteq \pi \times \pi$
$\red : \pi \to \mathcal{P}(\pi)$

\begin{mathpar}
  \inferrule* [lab=Comm] { \textsf{match}( x_{src}, x_{trgt} ) } { x_{trgt}?(y)P \; | \; x_{src}!\langle {Q} \rangle \red P\{\quotep{Q}/y}\} }
  \and \\
  \inferrule* [lab=Par] {{P} \red {P}'} {{{P} | {Q}} \red {{P}' | {Q}}}
  \and
  \inferrule* [lab=Equiv]{{{P} \scong {P}'} \andalso {{P}' \red {Q}'} \andalso {{Q}' \scong {Q}}}{{P} \red {Q}}
\end{mathpar}

\begin{eqnarray*}
  match_{\equiv} (\quotep{P},\quotep{Q}) & := & P \equiv Q \\
  match_{\dagger}(\quotep{P},\quotep{Q}) & := & \forall R. P|Q \red^{*} R => R \red^{*} 0 \\
  match_{K}(\quotep{P},\quotep{Q}) & := & K \mbox{ for some context } K
\end{eqnarray*}

$u?(x)P | u!\langle Q \rangle \red P\{\quotep{Q}/x\}$

%We write $\wred$ for $\red^*$, and $P\red$ if $\exists Q $ such that $ P \red Q$.
We write $P\red$ if $\exists Q $ such that $ P \red Q$ and $P\not\red$, otherwise.

\section{Replication}

As mentioned before, it is known that replication (and hence
recursion) can be implemented in a higher-order process algebra
\cite{SangiorgiWalker}. As our first example of calculation with the
machinery thus far presented we give the construction explicitly in
the {\rhoc}.

\begin{eqnarray}
	D_{x} & := & \prefix{x}{y}{(\binpar{\outputp{x}{y}}{@{y}})} \nonumber\\
	\bangp_{x}{P} & := & \binpar{{x}!\langle{\binpar{D_{x}}{P}}\rangle}{D_{x}} \nonumber
\end{eqnarray}

\begin{eqnarray}
	\bangp_{x}{P} & & \nonumber\\
	=
	& {x}!\langle{(\prefix{x}{y}{(\outputp{x}{y} | @{y})) | P}}\rangle 
	      | \prefix{x}{y}{(\outputp{x}{y} | @{y})} & \nonumber\\
	\red
	& (\outputp{x}{y} | @{y})\substn{\quotep{(\prefix{x}{y}{(@{y} | \outputp{x}{y})) | P}}}{y} & \nonumber\\
	=
	& \outputp{x}{\quotep{(\prefix{x}{y}{(\outputp{x}{y} | @{y})) | P}}}
	  | {(\prefix{x}{y}{(\outputp{x}{y} | @{y})) | P}} & \nonumber\\
	\red
	& \ldots & \nonumber\\
	\red^*
	& P | P | \ldots & \nonumber
\end{eqnarray}

Of course, this encoding, as an implementation, runs away, unfolding
$\bangp{P}$ eagerly. A lazier and more implementable replication
operator, restricted to input-guarded processes, may be obtained as follows.

\begin{eqnarray}
\bangp{\prefix{u}{v}{P}} 
	:= 
	\binpar{\lift{x}{\prefix{u}{v}{(\binpar{D(x)}{P})}}}{D(x)} \nonumber
\end{eqnarray}

\begin{remark}
  Note that the lazier definition still does not deal with summation
  or mixed summation (i.e. sums over input and output). The reader is
  invited to construct definitions of replication that deal with these
  features. 

  Further, the definitions are parameterized in a name, $x$. Can you,
  gentle reader, make a definition that eliminates this parameter and
  guarantees no accidental interaction between the replication
  machinery and the process being replicated -- i.e. no accidental
  sharing of names used by the process to get its work done and the
  name(s) used by the replication to effect copying. This latter
  revision of the definition of replication is crucial to obtaining
  the expected identity $!!P \sim !P$.
\end{remark}

\begin{remark}\label{rem:paradoxical_combinator}
  The reader familiar with the lambda calculus will have noticed the
  similarity between $D$ and the paradoxical combinator.

  [Ed. note: the existence of this seems to suggest we have to be more
  restrictive on the set of processes and names we admit if we are to
  support no-cloning.]
\end{remark}

\subsubsection{Bisimulation}

The computational dynamics gives rise to another kind of equivalence,
the equivalence of computational behavior. As previously mentioned
this is typically captured \emph{via} some form of bisimulation.

% The notion we use in this paper is weak barbed bisimulation
% \cite{milner91polyadicpi}.

The notion we use in this paper is derived from weak barbed
bisimulation \cite{milner91polyadicpi}. 

\begin{definition}
An \emph{observation relation}, $\downarrow_{\mathcal N}$, over a set
of names, $\mathcal N$, is the smallest relation satisfying the rules
below.

\infrule[Out-barb]{y \in {\mathcal N}, \; x \nameeq y}
		  {\outputp{x}{v} \downarrow_{\mathcal N} x}
\infrule[Par-barb]{\mbox{$P\downarrow_{\mathcal N} x$ or $Q\downarrow_{\mathcal N} x$}}
		  {\binpar{P}{Q} \downarrow_{\mathcal N} x}

We write $P \Downarrow_{\mathcal N} x$ if there is $Q$ such that 
$P \wred Q$ and $Q \downarrow_{\mathcal N} x$.
\end{definition}

\begin{definition}
%\label{def.bbisim}
An  ${\mathcal N}$-\emph{barbed bisimulation} over a set of names, ${\mathcal N}$, is a symmetric binary relation 
${\mathcal S}_{\mathcal N}$ between agents such that $P\rel{S}_{\mathcal N}Q$ implies:
\begin{enumerate}
\item If $P \red P'$ then $Q \wred Q'$ and $P'\rel{S}_{\mathcal N} Q'$.
\item If $P\downarrow_{\mathcal N} x$, then $Q\Downarrow_{\mathcal N} x$.
\end{enumerate}
$P$ is ${\mathcal N}$-barbed bisimilar to $Q$, written
$P \wbbisim_{\mathcal N} Q$, if $P \rel{S}_{\mathcal N} Q$ for some ${\mathcal N}$-barbed bisimulation ${\mathcal S}_{\mathcal N}$.
\end{definition}

$\mathcal{R} \subseteq \pi \times \pi$

$P \mathcal{R} Q => \forall P'. P \red P' \Rightarrow \exists Q'. Q \red Q', P' \mathcal{R} Q'$

$P \vdash x \Rightarrow Q \vdash x$

\begin{mathpar}
  \inferrule*[lab=Out-barb]{x \nameeq y}{{y}!\langle{Q}\rangle \vdash x}
  \and
  \inferrule*[lab=Par-barb]{\mbox{$P\vdash x$ or $Q\vdash x$}}{\binpar{P}{Q} \vdash x}
\end{mathpar}

\subsubsection{Contexts}

One of the principle advantages of computational calculi like the
$\pi$-calculus is a well-defined notion of context,
contextual-equivalence and a correlation between
contextual-equivalence and notions of bisimulation. The notion of
context allows the decomposition of a process into (sub-)process and
its syntactic environment, its context. Thus, a context may be
thought of as a process with a ``hole'' (written $\Box$) in it. The
application of a context $M$ to a process $P$, written $M[P]$, is
tantamount to filling the hole in $M$ with $P$. In this paper we do
not need the full weight of this theory, but do make use of the notion
of context in the proof the main theorem. 

\begin{mathpar}
  \inferrule* [lab=summation] {} {{M_{M},M_{N}} \bc \Box \;|\; x.M_{A} \;|\; M_{M}+M_{N}}
  \and
  \inferrule* [lab=agent] {} {{M_{A}} \bc (\vec{x})M_{P} \;| \; \clift{P_0,\ldots,M_{P},\ldots,P_N}}
  \and \\
  \inferrule* [lab=process] {} {{M_{P}} \bc M_{N} \;| \;P|M_{P} }
\end{mathpar} 

\begin{mathpar}
  \inferrule* [lab=sychronization] {} {M_{N} \bc \Box \;|\; x?M_{F} \;|\; x!M_{C}}
  \and
  \inferrule* [lab=abstraction] {} {{M_{F}} \bc (x)M_{P} }
  \and
  \inferrule* [lab=concretion] {} {{M_{C}} \bc \langle M_{P} \rangle }
  \and \\
  \inferrule* [lab=process] {} {{M_{P}} \bc M_{N} \;| \;P|M_{P} }
\end{mathpar}

\begin{definition}[contextual application] Given a context $M$, and
  process $P$, we define the \emph{contextual application}, $M[P] :=
  M\{P/\Box\}$. That is, the contextual application of M to P is the
  substitution of $P$ for $\Box$ in $M$.
\end{definition}

$\meaningof{-} : L \to \mathcal{P}(\pi)$

\begin{mathpar}
  \inferrule* [lab=collection] {} {\meaningof{true} = \pi, \and \meaningof{~E} = \pi \setminus \meaningof{E}, \and \meaningof{E_{1} \& E_{2}} = \meaningof{E_{1}} \cap \meaningof{E_{2}}}
\end{mathpar}

\begin{mathpar}
  \inferrule* [lab=structure] {} {\meaningof{0} = \{ P \in \pi | P \equiv 0 \}, \and \\ \meaningof{E_1 | E_2} = \{ P \in \pi | P \equiv P_{1} | P_{2}, P_{1} \in \meaningof{E_{1}}, P_{2} \in \meaningof{E_2}\} }
\end{mathpar}

\begin{mathpar}
 \inferrule* [lab=behavior] {} {\meaningof{\langle a?b \rangle E} = \{ P \in \pi | P \equiv Q | u?(y)P', \\ \and \\\\ \and \\ \;\;\; u \in \meaningof{a}, \forall z.P'\{z/y\} \in \meaningof{E\{z/b\}}\}, \and \\ \meaningof{a!E} = \{ P \in \pi | P \equiv Q | x!\langle P' \rangle, x \in \meaningof{a} P' \in \meaningof{E}\} }
\end{mathpar}

\begin{mathpar}
 \inferrule* [lab=nominal] {} {\meaningof{\quotep{E}} = \{ \quotep{P} \in \quotep{\pi} | P \in \meaningof{E} \}, \and \meaningof{\quotep{P}} = \{ \quotep{Q} \in \quotep{\pi} | P \equiv Q \} \and \\ \meaningof{@\quotep{E}} = \{ P \in \pi | P \equiv @x, x \in \meaningof{E} \}}
\end{mathpar}

\begin{eqnarray*}
  \\
  \meaningof{-} : TS \to ST
\end{eqnarray*}

\begin{eqnarray*}
  \\
  L : TS \to ST
\end{eqnarray*}

\begin{eqnarray*}
  \\
  P \models E \iff P \in \meaningof{E}
\end{eqnarray*}

\begin{eqnarray*}
  P \approx_{L} Q \iff \forall E \in L. P \models E \iff Q \models E
\end{eqnarray*}

\begin{eqnarray*}
  P \approx_{K} Q
\end{eqnarray*}

\begin{eqnarray*}
  P \approx Q
\end{eqnarray*}

$\approx_{K} = \approx = \approx_{L}$

\subsubsection{Contextual duality}

Note that contexts extend the quotation operation to a family of
operations from processes to names. Given a context, $M$, we can
define a \emph{nominal context}, $\quotep{M}$ by $\quotep{M}[P] :=
\quotep{M[P]}$. To foreshadow what is to come we observe that these
operations enjoy a duality with processes very much like the duality
between vectors and maps from vectors to scalars.

Further, because the calculus is essentially higher-order, we have a
correspondence between contexts and processes. More specifically,
given a name $x$ and a context $M$ we can construct $M^{*}_{x}$ such
that 

\begin{mathpar}
  M^{*}_{x} | \lift{x}{P} \red M[P]
\end{mathpar}

namely,

\begin{mathpar}
  M^{*}_{x} := x?(u).M[\dropn{u}]
\end{mathpar}

The dependence of $M^{*}_{x}$ on a name makes it an abstraction, 

\begin{mathpar}
  M^{*} := (x)x?(u).M[\dropn{u}]
\end{mathpar}

\subsection{Additional notation}

It will sometimes be convenient to denote the process a name
quotes. We already have the notation $x = \quotep{P}$, but it will be
convenient to introduce an alternate notation, $\procn{x}$, when we
want to emphasize the connection to the use of the name. Note that, by
virtue of name equivalence, $\quotep{\procn{x}} \nameeq x$; so, the
notation is consistent with previous definitions.

Further, because names have structure it is possible to effect
substitutions on the basis of that structure. This means we need to
upgrade our notation for substitutions, which we accomplish by
adapting comprehension notation. Thus,

\begin{mathpar}
  P\{ y / x : x \in S \}
\end{mathpar}

is interpreted to mean the process derived from P by replacing (in a
capture-avoiding manner) each occurrence of $x$ in $S$ by $y$. For example,

\begin{mathpar}
  P\{ \quotep{\procn{x}|\procn{x}} / x : x \in \freenames{P} \}
\end{mathpar}

will replace each (occurrence) of a free name $x$ in $P$ by
$\quotep{\procn{x}|\procn{x}}$.

Also, we will avail ourselves of the notation $x^{L}$ and $x^{R}$ to
denote injections of a name into disjoint copies of the name
space. There are numerous ways to accomplish this. One example can be
found in \cite{MeredithR05}. This notation overloads to vectors of
names: $\vec{x}^{\pi} := (x_{i}^{\pi} \; : \; 0 \leq i < |\vec{x}| )$ where $\pi \in \{L,R\}$.

We also use $P^{\Box} := P|\Box$.

In \cite{MeredithR05} an interpretation of the new operator is
given. It turns out that there are several possible interpretations
all enjoying the requisite algebraic properties of the operator (see
\cite{milner91polyadicpi}). We will therefore make liberal use of
$(\nu\; \vec{x})P$.

% subsection the_syntax_and_semantics_of_the_notation_system (end)   

\input{qm2pi.qmops} 

\input{qm2pi.sterngerlach} 

\input{qm2pi.metric} 

% section concurrent_process_calculi (end)

%\input{qm2pi.proofsketch}

% section proof sketch (end)

%\input{qm2pi.slviaknots} 

% section spatial logic via knots (end)

\input{qm2pi.conclusion}

% section conclusion (end)

%\input{qm2pi.dtcodes} 

% section wiring algorithm (end)

\input{qm2pi.ack} 

% section acknowledgments (end)

\newpage


\bibliographystyle{plain}   
\bibliography{../../biblios/main.bib}

\input{qm2pi.rhodetails}

\end{document}

 

% section acknowledgments (end)

\newpage


\bibliographystyle{plain}   
\bibliography{../../biblios/main.bib}

\documentclass[12pt]{llncs}
%\documentclass{jktr}

\usepackage[pdftex]{hyperref}                   
\usepackage {listings}
\usepackage {mathpartir}
\usepackage{bcprules}
%\usepackage{listings}
                       
\usepackage{graphicx} 
%\usepackage[margins=2.5cm,nohead,nofoot]{geometry}
%\usepackage{geometry}
\usepackage{amsfonts}
\usepackage{amstext}
\usepackage{latexsym}
\usepackage{amssymb}
\usepackage{color}


%\include{myPreamble}
\include{qm2pi.local} 

%\ifpdf
%\usepackage[pdftex]{graphicx}
%\else
%\usepackage{graphicx}
%\fi

 % \ifpdf
%  \usepackage{pdfsync}
%  \if


%\title{Brief Article}
%\author{David F. Snyder}
%\author{L.G. Meredith}

%\address{Dept. of Math., Texas State University--San Marcos, San Marcos, TX 78666}
       
\pagestyle{empty}


\begin{document}

\lstset{language=[Objective]Caml,frame=shadowbox}

\input{qm2pi.front}

% section front matter (end)

\input{qm2pi.intro} 
 
% section introduction (end)

% \input{qm2pi.knotations} 

% section notation (end)

\input{qm2pi.process.calculi} 

% section concurrent_process_calculi_and_spatial_logics_ (end)
    
%\input{qm2pi.knots2pi} 

%\input{qm2pi.trefoil} 

%\input{qm2pi.mainthm} 

% subsection basic_interpretation (end)

%\input{qm2pi.rho.presentation} 
\subsection{The syntax and semantics of the notation system}\label{sub:the_syntax_and_semantics_of_the_notation_system} % (fold)

We now summarize a technical presentation of the calculus that
embodies our theory of dynamics. The typical presentation of such a
calculus follows the style of giving generators and relations on
them. The grammar, below, describing term constructors, freely
generates the set of processes, $\Proc$. This set is then quotiented
by a relation known as structural congruence and it is over this set
that the notion of dynamics is expressed. This presentation is
essentially that of \cite{MeredithR05} with the addition of
polyadicity and summation. For readability we have relegated some of
the technical subtleties to an appendix.

\subsubsection{Process grammar}\label{subsub:process_grammar}

\begin{mathpar}
  \inferrule* [lab=synchronization] {} {{M} \bc \pzero \;|\; x?F \;|\; x!C }
  \and
  \inferrule* [lab=abstraction] {} {{F} \bc (x)P}
  \and
  \inferrule* [lab=concretion] {} {{C} \bc \langle Q \rangle}
  \and
  \inferrule* [lab=process] {} {{P,Q} \bc M \;| \;P|Q \;|\; @{x}}
  \and
  \inferrule* [lab=name] {} {{x} \bc \quotep{P}}
\end{mathpar} 

Note that $\vec{x}$ (resp. $\vec{P}$) denotes a vector of names
(resp. processes) of length $|\vec{x}|$ (resp. $|\vec{P}|$). We adopt
the following useful abbreviations.

\begin{mathpar}
   x?(\vec{y}).P := x.(\vec{y})P \and  x\clift{\vec{P}} := x.\clift{\vec{P}}
   \and x!(y) := \lift{x}{\dropn{y}}
   \and \Pi_{i=0}^{n-1}P_i := P_0 | \ldots | P_{n-1}
\end{mathpar}

\subsubsection{Structural congruence}

\paragraph{Free and bound names and alpha-equivalence.} At the
core of structural equivalence is alpha-equivalence which identifies
process that are the same up to a change of variable. Formally, we
recognize the distinction between free and bound names. The free names
of a process, $\freenames{P}$, may be calculated recursively as
follows:

\begin{mathpar}
\freenames{\pzero} := \emptyset
  \and \\
  \freenames{x?(y).P} := \{ x \} \cup (\freenames{P} \setminus \{ y \})
  \and 
  \freenames{x!\langle P \rangle} := \{ x \} \cup \{ P \} 
  \and \\
  \freenames{P|Q} := \freenames{P} \cup \freenames{Q}
  \and \\
  \freenames{@{x}} := \{ x \}
\end{mathpar}

$\pi$
$\quotep{\pi}$

$\freenames{-} : \pi \to \mathcal{P}(\quotep{\pi})$

\begin{eqnarray*}
  \freenames{\pzero} & := & \emptyset \\
  \freenames{x?(y).P} & := & \{ x \} \cup (\freenames{P} \setminus \{ y \}) \\
  \freenames{x!\langle P \rangle} & := & \{ x \} \cup \{ P \} \\
  \freenames{P|Q} & := & \freenames{P} \cup \freenames{Q} \\
  \freenames{\dropn{x}} & := & \{ x \}
\end{eqnarray*}

The bound names of a process, $\boundnames{P}$, are those names occurring in $P$
that are not free. For example, in $x?(y).0$, the name $x$ is free, while $y$ is bound.

\begin{mathpar}
  \inferrule* [lab=monoidal-laws] {} { P|Q \equiv Q|P \and P|0 \equiv P \and P|(Q|R) \equiv (P|Q)|R }
\end{mathpar}

\begin{mathpar}
  \inferrule* [lab=alpha-equivalence] {} { (x)P \equiv (y)P\{y/x\} \and y \not\in \freenames{P} }
\end{mathpar}

\begin{definition}
Then two processes, $P,Q$, are alpha-equivalent if $P = Q\{\vec{y}/\vec{x}\}$ for
some $\vec{x} \in \boundnames{Q},\vec{y} \in \boundnames{P}$, where $Q\{\vec{y}/\vec{x}\}$
denotes the capture-avoiding substitution of $\vec{y}$ for $\vec{x}$ in $Q$.
\end{definition}

\begin{definition}
  The {\em structural congruence} \cite{SangiorgiWalker} , $\equiv$,
  between processes is the least congruence containing
  alpha-equivalence, satisfying the abelian monoid laws
  (associativity, commutativity and $\pzero$ as identity) for parallel
  composition $|$ and for summation $+$.
\end{definition}

\subsection{Name equivalence}

We take name equivalence, written $\nameeq$, to be the smallest
equivalence relation generated by the following rules.

\begin{mathpar}
\inferrule*[lab=Quote-drop]
{ }
{ \quotep{@{x}} \nameeq x }

\inferrule*[lab=Struct-equiv]
{ P \scong Q }
{ \quotep{P} \nameeq \quotep{Q} }
\end{mathpar}

The astute reader will have noticed that the mutual recursion of names
and processes imposes a mutual recursion on alpha-equivalence and
structural equivalence via name-equivalence. Fortunately, all of this
works out pleasantly and we may calculate in the natural way, free of
concern. The reader interested in the details is referred to the
appendix \ref{appendix:rho_details}.

\subsection{Substitution}

We use $\Proc$ for the set of processes, $\QProc$ for the set of
names, and $\id{\{}\vec{y} / \vec{x} \id{\}}$ to denote partial maps,
$s : \QProc \rightarrow \QProc$. A map, $s$ lifts, uniquely, to a map
on process terms, $\widehat{s} : \Proc \rightarrow \Proc$ by the
following equations.

\begin{mathpar}
  (0) \psubstp{Q}{P} := 0 \\
  (R \juxtap S) \psubstp{Q}{P}
  :=    
  (R)\psubstp{Q}{P} \juxtap (S) \psubstp{Q}{P} \\
  (x?(y).R) \psubstp{Q}{P}    
  :=    
  (x)\substp{Q}{P} (z)\concat( (R \psubstn{z}{y}) \psubstp{Q}{P} ) \\
  (\lift{x}{R}) \psubstp{Q}{P}  
  :=
  \lift{(x)\substp{Q}{P}}{ R \psubstp{Q}{P} } \\
%   (\dropn{x})  \psubstp{Q}{P}       
%   := 
%   \left\{ 
%     \begin{array}{ccc} 
%       \dropn{\quotep{Q}} & & x \nameeq \quotep{P} \\
%       \dropn{x} & & otherwise \\
%     \end{array}
%   \right. 
  (\dropn{x})  \psubstp{Q}{P}       
  := 
  \left\{ 
    \begin{array}{ccc} 
      Q & & x \nameeq \quotep{P} \\
      \dropn{x} & & otherwise \\
    \end{array}
  \right.
\end{mathpar}
 

where

\begin{eqnarray}
  (x)\id{\{} \lpquote Q \rpquote / \lpquote P \rpquote \id{\}}            = 
  \left\{ 
    \begin{array}{ccc}
      \lpquote Q \rpquote & & x \nameeq \lpquote P \rpquote \\
      x & & otherwise \\
    \end{array}
  \right. \nonumber
\end{eqnarray}

and $z$ is chosen distinct from $\quotep{P}$, $\quotep{Q}$, the free
names in $Q$, and all the names in $R$. Our $\alpha$-equivalence will
be built in the standard way from this substitution.

\begin{remark}\label{rem:no_self_referential_names}
  One consequence of these definitions is that $\forall P. \quotep{P}
  \not\in \freenames{P}$.
\end{remark}

\subsection{ Dynamic quote: an example }

Anticipating something of what's to come, consider applying the
substitution, $\widehat{\id{\{}u / z \id{\}}}$, to the following pair
of processes, $\lift{w}{y!(z)}$ and $w[ \lpquote y!(z) \rpquote ]$.

\begin{eqnarray}
	\lift{w}{y!(z)}\widehat{\id{\{}u / z \id{\}}}
		& = &
		\lift{w}{y!(u)} \nonumber\\
	w[ \lpquote y!(z) \rpquote ] \widehat{ \id{\{}u / z \id{\}} }
		& = &
		w[ \lpquote y!(z) \rpquote ] \nonumber
\end{eqnarray}

Because the body of the process between quotes is impervious to
substitution, we get radically different answers. In fact, by
examining the first process in an input context,
e.g. $x?(z).\lift{w}{y!(z)}$, we see that the process under the lift
operator may be shaped by prefixed inputs binding a name inside it. In
this sense, the lift operator will be seen as a way to dynamically
construct processes before reifying them as names.

Finally equipped with these standard features we can present the
dynamics of the calculus.

\subsubsection{Operational semantics} 

Finally, we introduce the computational dynamics. What marks these
algebras as distinct from other more traditionally studied algebraic
structures, e.g. vector spaces or polynomial rings, is the manner in
which dynamics is captured. In traditional structures, dynamics is typically
expressed through morphisms between such structures, as in linear maps
between vector spaces or morphisms between rings. In algebras
associated with the semantics of computation, the dynamics is
expressed as part of the algebraic structure itself, through a
reduction reduction relation typically denoted by $\red$. Below, we
give a recursive presentation of this relation for the calculus used
in the encoding.

$\red \subseteq \pi \times \pi$
$\red : \pi \to \mathcal{P}(\pi)$

\begin{mathpar}
  \inferrule* [lab=Comm] { \textsf{match}( x_{src}, x_{trgt} ) } { x_{trgt}?(y)P \; | \; x_{src}!\langle {Q} \rangle \red P\{\quotep{Q}/y}\} }
  \and \\
  \inferrule* [lab=Par] {{P} \red {P}'} {{{P} | {Q}} \red {{P}' | {Q}}}
  \and
  \inferrule* [lab=Equiv]{{{P} \scong {P}'} \andalso {{P}' \red {Q}'} \andalso {{Q}' \scong {Q}}}{{P} \red {Q}}
\end{mathpar}

\begin{eqnarray*}
  match_{\equiv} (\quotep{P},\quotep{Q}) & := & P \equiv Q \\
  match_{\dagger}(\quotep{P},\quotep{Q}) & := & \forall R. P|Q \red^{*} R => R \red^{*} 0 \\
  match_{K}(\quotep{P},\quotep{Q}) & := & K \mbox{ for some context } K
\end{eqnarray*}

$u?(x)P | u!\langle Q \rangle \red P\{\quotep{Q}/x\}$

%We write $\wred$ for $\red^*$, and $P\red$ if $\exists Q $ such that $ P \red Q$.
We write $P\red$ if $\exists Q $ such that $ P \red Q$ and $P\not\red$, otherwise.

\section{Replication}

As mentioned before, it is known that replication (and hence
recursion) can be implemented in a higher-order process algebra
\cite{SangiorgiWalker}. As our first example of calculation with the
machinery thus far presented we give the construction explicitly in
the {\rhoc}.

\begin{eqnarray}
	D_{x} & := & \prefix{x}{y}{(\binpar{\outputp{x}{y}}{@{y}})} \nonumber\\
	\bangp_{x}{P} & := & \binpar{{x}!\langle{\binpar{D_{x}}{P}}\rangle}{D_{x}} \nonumber
\end{eqnarray}

\begin{eqnarray}
	\bangp_{x}{P} & & \nonumber\\
	=
	& {x}!\langle{(\prefix{x}{y}{(\outputp{x}{y} | @{y})) | P}}\rangle 
	      | \prefix{x}{y}{(\outputp{x}{y} | @{y})} & \nonumber\\
	\red
	& (\outputp{x}{y} | @{y})\substn{\quotep{(\prefix{x}{y}{(@{y} | \outputp{x}{y})) | P}}}{y} & \nonumber\\
	=
	& \outputp{x}{\quotep{(\prefix{x}{y}{(\outputp{x}{y} | @{y})) | P}}}
	  | {(\prefix{x}{y}{(\outputp{x}{y} | @{y})) | P}} & \nonumber\\
	\red
	& \ldots & \nonumber\\
	\red^*
	& P | P | \ldots & \nonumber
\end{eqnarray}

Of course, this encoding, as an implementation, runs away, unfolding
$\bangp{P}$ eagerly. A lazier and more implementable replication
operator, restricted to input-guarded processes, may be obtained as follows.

\begin{eqnarray}
\bangp{\prefix{u}{v}{P}} 
	:= 
	\binpar{\lift{x}{\prefix{u}{v}{(\binpar{D(x)}{P})}}}{D(x)} \nonumber
\end{eqnarray}

\begin{remark}
  Note that the lazier definition still does not deal with summation
  or mixed summation (i.e. sums over input and output). The reader is
  invited to construct definitions of replication that deal with these
  features. 

  Further, the definitions are parameterized in a name, $x$. Can you,
  gentle reader, make a definition that eliminates this parameter and
  guarantees no accidental interaction between the replication
  machinery and the process being replicated -- i.e. no accidental
  sharing of names used by the process to get its work done and the
  name(s) used by the replication to effect copying. This latter
  revision of the definition of replication is crucial to obtaining
  the expected identity $!!P \sim !P$.
\end{remark}

\begin{remark}\label{rem:paradoxical_combinator}
  The reader familiar with the lambda calculus will have noticed the
  similarity between $D$ and the paradoxical combinator.

  [Ed. note: the existence of this seems to suggest we have to be more
  restrictive on the set of processes and names we admit if we are to
  support no-cloning.]
\end{remark}

\subsubsection{Bisimulation}

The computational dynamics gives rise to another kind of equivalence,
the equivalence of computational behavior. As previously mentioned
this is typically captured \emph{via} some form of bisimulation.

% The notion we use in this paper is weak barbed bisimulation
% \cite{milner91polyadicpi}.

The notion we use in this paper is derived from weak barbed
bisimulation \cite{milner91polyadicpi}. 

\begin{definition}
An \emph{observation relation}, $\downarrow_{\mathcal N}$, over a set
of names, $\mathcal N$, is the smallest relation satisfying the rules
below.

\infrule[Out-barb]{y \in {\mathcal N}, \; x \nameeq y}
		  {\outputp{x}{v} \downarrow_{\mathcal N} x}
\infrule[Par-barb]{\mbox{$P\downarrow_{\mathcal N} x$ or $Q\downarrow_{\mathcal N} x$}}
		  {\binpar{P}{Q} \downarrow_{\mathcal N} x}

We write $P \Downarrow_{\mathcal N} x$ if there is $Q$ such that 
$P \wred Q$ and $Q \downarrow_{\mathcal N} x$.
\end{definition}

\begin{definition}
%\label{def.bbisim}
An  ${\mathcal N}$-\emph{barbed bisimulation} over a set of names, ${\mathcal N}$, is a symmetric binary relation 
${\mathcal S}_{\mathcal N}$ between agents such that $P\rel{S}_{\mathcal N}Q$ implies:
\begin{enumerate}
\item If $P \red P'$ then $Q \wred Q'$ and $P'\rel{S}_{\mathcal N} Q'$.
\item If $P\downarrow_{\mathcal N} x$, then $Q\Downarrow_{\mathcal N} x$.
\end{enumerate}
$P$ is ${\mathcal N}$-barbed bisimilar to $Q$, written
$P \wbbisim_{\mathcal N} Q$, if $P \rel{S}_{\mathcal N} Q$ for some ${\mathcal N}$-barbed bisimulation ${\mathcal S}_{\mathcal N}$.
\end{definition}

$\mathcal{R} \subseteq \pi \times \pi$

$P \mathcal{R} Q => \forall P'. P \red P' \Rightarrow \exists Q'. Q \red Q', P' \mathcal{R} Q'$

$P \vdash x \Rightarrow Q \vdash x$

\begin{mathpar}
  \inferrule*[lab=Out-barb]{x \nameeq y}{{y}!\langle{Q}\rangle \vdash x}
  \and
  \inferrule*[lab=Par-barb]{\mbox{$P\vdash x$ or $Q\vdash x$}}{\binpar{P}{Q} \vdash x}
\end{mathpar}

\subsubsection{Contexts}

One of the principle advantages of computational calculi like the
$\pi$-calculus is a well-defined notion of context,
contextual-equivalence and a correlation between
contextual-equivalence and notions of bisimulation. The notion of
context allows the decomposition of a process into (sub-)process and
its syntactic environment, its context. Thus, a context may be
thought of as a process with a ``hole'' (written $\Box$) in it. The
application of a context $M$ to a process $P$, written $M[P]$, is
tantamount to filling the hole in $M$ with $P$. In this paper we do
not need the full weight of this theory, but do make use of the notion
of context in the proof the main theorem. 

\begin{mathpar}
  \inferrule* [lab=summation] {} {{M_{M},M_{N}} \bc \Box \;|\; x.M_{A} \;|\; M_{M}+M_{N}}
  \and
  \inferrule* [lab=agent] {} {{M_{A}} \bc (\vec{x})M_{P} \;| \; \clift{P_0,\ldots,M_{P},\ldots,P_N}}
  \and \\
  \inferrule* [lab=process] {} {{M_{P}} \bc M_{N} \;| \;P|M_{P} }
\end{mathpar} 

\begin{mathpar}
  \inferrule* [lab=sychronization] {} {M_{N} \bc \Box \;|\; x?M_{F} \;|\; x!M_{C}}
  \and
  \inferrule* [lab=abstraction] {} {{M_{F}} \bc (x)M_{P} }
  \and
  \inferrule* [lab=concretion] {} {{M_{C}} \bc \langle M_{P} \rangle }
  \and \\
  \inferrule* [lab=process] {} {{M_{P}} \bc M_{N} \;| \;P|M_{P} }
\end{mathpar}

\begin{definition}[contextual application] Given a context $M$, and
  process $P$, we define the \emph{contextual application}, $M[P] :=
  M\{P/\Box\}$. That is, the contextual application of M to P is the
  substitution of $P$ for $\Box$ in $M$.
\end{definition}

$\meaningof{-} : L \to \mathcal{P}(\pi)$

\begin{mathpar}
  \inferrule* [lab=collection] {} {\meaningof{true} = \pi, \and \meaningof{~E} = \pi \setminus \meaningof{E}, \and \meaningof{E_{1} \& E_{2}} = \meaningof{E_{1}} \cap \meaningof{E_{2}}}
\end{mathpar}

\begin{mathpar}
  \inferrule* [lab=structure] {} {\meaningof{0} = \{ P \in \pi | P \equiv 0 \}, \and \\ \meaningof{E_1 | E_2} = \{ P \in \pi | P \equiv P_{1} | P_{2}, P_{1} \in \meaningof{E_{1}}, P_{2} \in \meaningof{E_2}\} }
\end{mathpar}

\begin{mathpar}
 \inferrule* [lab=behavior] {} {\meaningof{\langle a?b \rangle E} = \{ P \in \pi | P \equiv Q | u?(y)P', \\ \and \\\\ \and \\ \;\;\; u \in \meaningof{a}, \forall z.P'\{z/y\} \in \meaningof{E\{z/b\}}\}, \and \\ \meaningof{a!E} = \{ P \in \pi | P \equiv Q | x!\langle P' \rangle, x \in \meaningof{a} P' \in \meaningof{E}\} }
\end{mathpar}

\begin{mathpar}
 \inferrule* [lab=nominal] {} {\meaningof{\quotep{E}} = \{ \quotep{P} \in \quotep{\pi} | P \in \meaningof{E} \}, \and \meaningof{\quotep{P}} = \{ \quotep{Q} \in \quotep{\pi} | P \equiv Q \} \and \\ \meaningof{@\quotep{E}} = \{ P \in \pi | P \equiv @x, x \in \meaningof{E} \}}
\end{mathpar}

\begin{eqnarray*}
  \\
  \meaningof{-} : TS \to ST
\end{eqnarray*}

\begin{eqnarray*}
  \\
  L : TS \to ST
\end{eqnarray*}

\begin{eqnarray*}
  \\
  P \models E \iff P \in \meaningof{E}
\end{eqnarray*}

\begin{eqnarray*}
  P \approx_{L} Q \iff \forall E \in L. P \models E \iff Q \models E
\end{eqnarray*}

\begin{eqnarray*}
  P \approx_{K} Q
\end{eqnarray*}

\begin{eqnarray*}
  P \approx Q
\end{eqnarray*}

$\approx_{K} = \approx = \approx_{L}$

\subsubsection{Contextual duality}

Note that contexts extend the quotation operation to a family of
operations from processes to names. Given a context, $M$, we can
define a \emph{nominal context}, $\quotep{M}$ by $\quotep{M}[P] :=
\quotep{M[P]}$. To foreshadow what is to come we observe that these
operations enjoy a duality with processes very much like the duality
between vectors and maps from vectors to scalars.

Further, because the calculus is essentially higher-order, we have a
correspondence between contexts and processes. More specifically,
given a name $x$ and a context $M$ we can construct $M^{*}_{x}$ such
that 

\begin{mathpar}
  M^{*}_{x} | \lift{x}{P} \red M[P]
\end{mathpar}

namely,

\begin{mathpar}
  M^{*}_{x} := x?(u).M[\dropn{u}]
\end{mathpar}

The dependence of $M^{*}_{x}$ on a name makes it an abstraction, 

\begin{mathpar}
  M^{*} := (x)x?(u).M[\dropn{u}]
\end{mathpar}

\subsection{Additional notation}

It will sometimes be convenient to denote the process a name
quotes. We already have the notation $x = \quotep{P}$, but it will be
convenient to introduce an alternate notation, $\procn{x}$, when we
want to emphasize the connection to the use of the name. Note that, by
virtue of name equivalence, $\quotep{\procn{x}} \nameeq x$; so, the
notation is consistent with previous definitions.

Further, because names have structure it is possible to effect
substitutions on the basis of that structure. This means we need to
upgrade our notation for substitutions, which we accomplish by
adapting comprehension notation. Thus,

\begin{mathpar}
  P\{ y / x : x \in S \}
\end{mathpar}

is interpreted to mean the process derived from P by replacing (in a
capture-avoiding manner) each occurrence of $x$ in $S$ by $y$. For example,

\begin{mathpar}
  P\{ \quotep{\procn{x}|\procn{x}} / x : x \in \freenames{P} \}
\end{mathpar}

will replace each (occurrence) of a free name $x$ in $P$ by
$\quotep{\procn{x}|\procn{x}}$.

Also, we will avail ourselves of the notation $x^{L}$ and $x^{R}$ to
denote injections of a name into disjoint copies of the name
space. There are numerous ways to accomplish this. One example can be
found in \cite{MeredithR05}. This notation overloads to vectors of
names: $\vec{x}^{\pi} := (x_{i}^{\pi} \; : \; 0 \leq i < |\vec{x}| )$ where $\pi \in \{L,R\}$.

We also use $P^{\Box} := P|\Box$.

In \cite{MeredithR05} an interpretation of the new operator is
given. It turns out that there are several possible interpretations
all enjoying the requisite algebraic properties of the operator (see
\cite{milner91polyadicpi}). We will therefore make liberal use of
$(\nu\; \vec{x})P$.

% subsection the_syntax_and_semantics_of_the_notation_system (end)   

\input{qm2pi.qmops} 

\input{qm2pi.sterngerlach} 

\input{qm2pi.metric} 

% section concurrent_process_calculi (end)

%\input{qm2pi.proofsketch}

% section proof sketch (end)

%\input{qm2pi.slviaknots} 

% section spatial logic via knots (end)

\input{qm2pi.conclusion}

% section conclusion (end)

%\input{qm2pi.dtcodes} 

% section wiring algorithm (end)

\input{qm2pi.ack} 

% section acknowledgments (end)

\newpage


\bibliographystyle{plain}   
\bibliography{../../biblios/main.bib}

\input{qm2pi.rhodetails}

\end{document}



\end{document}

 

% subsection basic_interpretation (end)

%\input{qm2pi.rho.presentation} 
\subsection{The syntax and semantics of the notation system}\label{sub:the_syntax_and_semantics_of_the_notation_system} % (fold)

We now summarize a technical presentation of the calculus that
embodies our theory of dynamics. The typical presentation of such a
calculus follows the style of giving generators and relations on
them. The grammar, below, describing term constructors, freely
generates the set of processes, $\Proc$. This set is then quotiented
by a relation known as structural congruence and it is over this set
that the notion of dynamics is expressed. This presentation is
essentially that of \cite{MeredithR05} with the addition of
polyadicity and summation. For readability we have relegated some of
the technical subtleties to an appendix.

\subsubsection{Process grammar}\label{subsub:process_grammar}

\begin{mathpar}
  \inferrule* [lab=synchronization] {} {{M} \bc \pzero \;|\; x?F \;|\; x!C }
  \and
  \inferrule* [lab=abstraction] {} {{F} \bc (x)P}
  \and
  \inferrule* [lab=concretion] {} {{C} \bc \langle Q \rangle}
  \and
  \inferrule* [lab=process] {} {{P,Q} \bc M \;| \;P|Q \;|\; @{x}}
  \and
  \inferrule* [lab=name] {} {{x} \bc \quotep{P}}
\end{mathpar} 

Note that $\vec{x}$ (resp. $\vec{P}$) denotes a vector of names
(resp. processes) of length $|\vec{x}|$ (resp. $|\vec{P}|$). We adopt
the following useful abbreviations.

\begin{mathpar}
   x?(\vec{y}).P := x.(\vec{y})P \and  x\clift{\vec{P}} := x.\clift{\vec{P}}
   \and x!(y) := \lift{x}{\dropn{y}}
   \and \Pi_{i=0}^{n-1}P_i := P_0 | \ldots | P_{n-1}
\end{mathpar}

\subsubsection{Structural congruence}

\paragraph{Free and bound names and alpha-equivalence.} At the
core of structural equivalence is alpha-equivalence which identifies
process that are the same up to a change of variable. Formally, we
recognize the distinction between free and bound names. The free names
of a process, $\freenames{P}$, may be calculated recursively as
follows:

\begin{mathpar}
\freenames{\pzero} := \emptyset
  \and \\
  \freenames{x?(y).P} := \{ x \} \cup (\freenames{P} \setminus \{ y \})
  \and 
  \freenames{x!\langle P \rangle} := \{ x \} \cup \{ P \} 
  \and \\
  \freenames{P|Q} := \freenames{P} \cup \freenames{Q}
  \and \\
  \freenames{@{x}} := \{ x \}
\end{mathpar}

$\pi$
$\quotep{\pi}$

$\freenames{-} : \pi \to \mathcal{P}(\quotep{\pi})$

\begin{eqnarray*}
  \freenames{\pzero} & := & \emptyset \\
  \freenames{x?(y).P} & := & \{ x \} \cup (\freenames{P} \setminus \{ y \}) \\
  \freenames{x!\langle P \rangle} & := & \{ x \} \cup \{ P \} \\
  \freenames{P|Q} & := & \freenames{P} \cup \freenames{Q} \\
  \freenames{\dropn{x}} & := & \{ x \}
\end{eqnarray*}

The bound names of a process, $\boundnames{P}$, are those names occurring in $P$
that are not free. For example, in $x?(y).0$, the name $x$ is free, while $y$ is bound.

\begin{mathpar}
  \inferrule* [lab=monoidal-laws] {} { P|Q \equiv Q|P \and P|0 \equiv P \and P|(Q|R) \equiv (P|Q)|R }
\end{mathpar}

\begin{mathpar}
  \inferrule* [lab=alpha-equivalence] {} { (x)P \equiv (y)P\{y/x\} \and y \not\in \freenames{P} }
\end{mathpar}

\begin{definition}
Then two processes, $P,Q$, are alpha-equivalent if $P = Q\{\vec{y}/\vec{x}\}$ for
some $\vec{x} \in \boundnames{Q},\vec{y} \in \boundnames{P}$, where $Q\{\vec{y}/\vec{x}\}$
denotes the capture-avoiding substitution of $\vec{y}$ for $\vec{x}$ in $Q$.
\end{definition}

\begin{definition}
  The {\em structural congruence} \cite{SangiorgiWalker} , $\equiv$,
  between processes is the least congruence containing
  alpha-equivalence, satisfying the abelian monoid laws
  (associativity, commutativity and $\pzero$ as identity) for parallel
  composition $|$ and for summation $+$.
\end{definition}

\subsection{Name equivalence}

We take name equivalence, written $\nameeq$, to be the smallest
equivalence relation generated by the following rules.

\begin{mathpar}
\inferrule*[lab=Quote-drop]
{ }
{ \quotep{@{x}} \nameeq x }

\inferrule*[lab=Struct-equiv]
{ P \scong Q }
{ \quotep{P} \nameeq \quotep{Q} }
\end{mathpar}

The astute reader will have noticed that the mutual recursion of names
and processes imposes a mutual recursion on alpha-equivalence and
structural equivalence via name-equivalence. Fortunately, all of this
works out pleasantly and we may calculate in the natural way, free of
concern. The reader interested in the details is referred to the
appendix \ref{appendix:rho_details}.

\subsection{Substitution}

We use $\Proc$ for the set of processes, $\QProc$ for the set of
names, and $\id{\{}\vec{y} / \vec{x} \id{\}}$ to denote partial maps,
$s : \QProc \rightarrow \QProc$. A map, $s$ lifts, uniquely, to a map
on process terms, $\widehat{s} : \Proc \rightarrow \Proc$ by the
following equations.

\begin{mathpar}
  (0) \psubstp{Q}{P} := 0 \\
  (R \juxtap S) \psubstp{Q}{P}
  :=    
  (R)\psubstp{Q}{P} \juxtap (S) \psubstp{Q}{P} \\
  (x?(y).R) \psubstp{Q}{P}    
  :=    
  (x)\substp{Q}{P} (z)\concat( (R \psubstn{z}{y}) \psubstp{Q}{P} ) \\
  (\lift{x}{R}) \psubstp{Q}{P}  
  :=
  \lift{(x)\substp{Q}{P}}{ R \psubstp{Q}{P} } \\
%   (\dropn{x})  \psubstp{Q}{P}       
%   := 
%   \left\{ 
%     \begin{array}{ccc} 
%       \dropn{\quotep{Q}} & & x \nameeq \quotep{P} \\
%       \dropn{x} & & otherwise \\
%     \end{array}
%   \right. 
  (\dropn{x})  \psubstp{Q}{P}       
  := 
  \left\{ 
    \begin{array}{ccc} 
      Q & & x \nameeq \quotep{P} \\
      \dropn{x} & & otherwise \\
    \end{array}
  \right.
\end{mathpar}
 

where

\begin{eqnarray}
  (x)\id{\{} \lpquote Q \rpquote / \lpquote P \rpquote \id{\}}            = 
  \left\{ 
    \begin{array}{ccc}
      \lpquote Q \rpquote & & x \nameeq \lpquote P \rpquote \\
      x & & otherwise \\
    \end{array}
  \right. \nonumber
\end{eqnarray}

and $z$ is chosen distinct from $\quotep{P}$, $\quotep{Q}$, the free
names in $Q$, and all the names in $R$. Our $\alpha$-equivalence will
be built in the standard way from this substitution.

\begin{remark}\label{rem:no_self_referential_names}
  One consequence of these definitions is that $\forall P. \quotep{P}
  \not\in \freenames{P}$.
\end{remark}

\subsection{ Dynamic quote: an example }

Anticipating something of what's to come, consider applying the
substitution, $\widehat{\id{\{}u / z \id{\}}}$, to the following pair
of processes, $\lift{w}{y!(z)}$ and $w[ \lpquote y!(z) \rpquote ]$.

\begin{eqnarray}
	\lift{w}{y!(z)}\widehat{\id{\{}u / z \id{\}}}
		& = &
		\lift{w}{y!(u)} \nonumber\\
	w[ \lpquote y!(z) \rpquote ] \widehat{ \id{\{}u / z \id{\}} }
		& = &
		w[ \lpquote y!(z) \rpquote ] \nonumber
\end{eqnarray}

Because the body of the process between quotes is impervious to
substitution, we get radically different answers. In fact, by
examining the first process in an input context,
e.g. $x?(z).\lift{w}{y!(z)}$, we see that the process under the lift
operator may be shaped by prefixed inputs binding a name inside it. In
this sense, the lift operator will be seen as a way to dynamically
construct processes before reifying them as names.

Finally equipped with these standard features we can present the
dynamics of the calculus.

\subsubsection{Operational semantics} 

Finally, we introduce the computational dynamics. What marks these
algebras as distinct from other more traditionally studied algebraic
structures, e.g. vector spaces or polynomial rings, is the manner in
which dynamics is captured. In traditional structures, dynamics is typically
expressed through morphisms between such structures, as in linear maps
between vector spaces or morphisms between rings. In algebras
associated with the semantics of computation, the dynamics is
expressed as part of the algebraic structure itself, through a
reduction reduction relation typically denoted by $\red$. Below, we
give a recursive presentation of this relation for the calculus used
in the encoding.

$\red \subseteq \pi \times \pi$
$\red : \pi \to \mathcal{P}(\pi)$

\begin{mathpar}
  \inferrule* [lab=Comm] { \textsf{match}( x_{src}, x_{trgt} ) } { x_{trgt}?(y)P \; | \; x_{src}!\langle {Q} \rangle \red P\{\quotep{Q}/y}\} }
  \and \\
  \inferrule* [lab=Par] {{P} \red {P}'} {{{P} | {Q}} \red {{P}' | {Q}}}
  \and
  \inferrule* [lab=Equiv]{{{P} \scong {P}'} \andalso {{P}' \red {Q}'} \andalso {{Q}' \scong {Q}}}{{P} \red {Q}}
\end{mathpar}

\begin{eqnarray*}
  match_{\equiv} (\quotep{P},\quotep{Q}) & := & P \equiv Q \\
  match_{\dagger}(\quotep{P},\quotep{Q}) & := & \forall R. P|Q \red^{*} R => R \red^{*} 0 \\
  match_{K}(\quotep{P},\quotep{Q}) & := & K \mbox{ for some context } K
\end{eqnarray*}

$u?(x)P | u!\langle Q \rangle \red P\{\quotep{Q}/x\}$

%We write $\wred$ for $\red^*$, and $P\red$ if $\exists Q $ such that $ P \red Q$.
We write $P\red$ if $\exists Q $ such that $ P \red Q$ and $P\not\red$, otherwise.

\section{Replication}

As mentioned before, it is known that replication (and hence
recursion) can be implemented in a higher-order process algebra
\cite{SangiorgiWalker}. As our first example of calculation with the
machinery thus far presented we give the construction explicitly in
the {\rhoc}.

\begin{eqnarray}
	D_{x} & := & \prefix{x}{y}{(\binpar{\outputp{x}{y}}{@{y}})} \nonumber\\
	\bangp_{x}{P} & := & \binpar{{x}!\langle{\binpar{D_{x}}{P}}\rangle}{D_{x}} \nonumber
\end{eqnarray}

\begin{eqnarray}
	\bangp_{x}{P} & & \nonumber\\
	=
	& {x}!\langle{(\prefix{x}{y}{(\outputp{x}{y} | @{y})) | P}}\rangle 
	      | \prefix{x}{y}{(\outputp{x}{y} | @{y})} & \nonumber\\
	\red
	& (\outputp{x}{y} | @{y})\substn{\quotep{(\prefix{x}{y}{(@{y} | \outputp{x}{y})) | P}}}{y} & \nonumber\\
	=
	& \outputp{x}{\quotep{(\prefix{x}{y}{(\outputp{x}{y} | @{y})) | P}}}
	  | {(\prefix{x}{y}{(\outputp{x}{y} | @{y})) | P}} & \nonumber\\
	\red
	& \ldots & \nonumber\\
	\red^*
	& P | P | \ldots & \nonumber
\end{eqnarray}

Of course, this encoding, as an implementation, runs away, unfolding
$\bangp{P}$ eagerly. A lazier and more implementable replication
operator, restricted to input-guarded processes, may be obtained as follows.

\begin{eqnarray}
\bangp{\prefix{u}{v}{P}} 
	:= 
	\binpar{\lift{x}{\prefix{u}{v}{(\binpar{D(x)}{P})}}}{D(x)} \nonumber
\end{eqnarray}

\begin{remark}
  Note that the lazier definition still does not deal with summation
  or mixed summation (i.e. sums over input and output). The reader is
  invited to construct definitions of replication that deal with these
  features. 

  Further, the definitions are parameterized in a name, $x$. Can you,
  gentle reader, make a definition that eliminates this parameter and
  guarantees no accidental interaction between the replication
  machinery and the process being replicated -- i.e. no accidental
  sharing of names used by the process to get its work done and the
  name(s) used by the replication to effect copying. This latter
  revision of the definition of replication is crucial to obtaining
  the expected identity $!!P \sim !P$.
\end{remark}

\begin{remark}\label{rem:paradoxical_combinator}
  The reader familiar with the lambda calculus will have noticed the
  similarity between $D$ and the paradoxical combinator.

  [Ed. note: the existence of this seems to suggest we have to be more
  restrictive on the set of processes and names we admit if we are to
  support no-cloning.]
\end{remark}

\subsubsection{Bisimulation}

The computational dynamics gives rise to another kind of equivalence,
the equivalence of computational behavior. As previously mentioned
this is typically captured \emph{via} some form of bisimulation.

% The notion we use in this paper is weak barbed bisimulation
% \cite{milner91polyadicpi}.

The notion we use in this paper is derived from weak barbed
bisimulation \cite{milner91polyadicpi}. 

\begin{definition}
An \emph{observation relation}, $\downarrow_{\mathcal N}$, over a set
of names, $\mathcal N$, is the smallest relation satisfying the rules
below.

\infrule[Out-barb]{y \in {\mathcal N}, \; x \nameeq y}
		  {\outputp{x}{v} \downarrow_{\mathcal N} x}
\infrule[Par-barb]{\mbox{$P\downarrow_{\mathcal N} x$ or $Q\downarrow_{\mathcal N} x$}}
		  {\binpar{P}{Q} \downarrow_{\mathcal N} x}

We write $P \Downarrow_{\mathcal N} x$ if there is $Q$ such that 
$P \wred Q$ and $Q \downarrow_{\mathcal N} x$.
\end{definition}

\begin{definition}
%\label{def.bbisim}
An  ${\mathcal N}$-\emph{barbed bisimulation} over a set of names, ${\mathcal N}$, is a symmetric binary relation 
${\mathcal S}_{\mathcal N}$ between agents such that $P\rel{S}_{\mathcal N}Q$ implies:
\begin{enumerate}
\item If $P \red P'$ then $Q \wred Q'$ and $P'\rel{S}_{\mathcal N} Q'$.
\item If $P\downarrow_{\mathcal N} x$, then $Q\Downarrow_{\mathcal N} x$.
\end{enumerate}
$P$ is ${\mathcal N}$-barbed bisimilar to $Q$, written
$P \wbbisim_{\mathcal N} Q$, if $P \rel{S}_{\mathcal N} Q$ for some ${\mathcal N}$-barbed bisimulation ${\mathcal S}_{\mathcal N}$.
\end{definition}

$\mathcal{R} \subseteq \pi \times \pi$

$P \mathcal{R} Q => \forall P'. P \red P' \Rightarrow \exists Q'. Q \red Q', P' \mathcal{R} Q'$

$P \vdash x \Rightarrow Q \vdash x$

\begin{mathpar}
  \inferrule*[lab=Out-barb]{x \nameeq y}{{y}!\langle{Q}\rangle \vdash x}
  \and
  \inferrule*[lab=Par-barb]{\mbox{$P\vdash x$ or $Q\vdash x$}}{\binpar{P}{Q} \vdash x}
\end{mathpar}

\subsubsection{Contexts}

One of the principle advantages of computational calculi like the
$\pi$-calculus is a well-defined notion of context,
contextual-equivalence and a correlation between
contextual-equivalence and notions of bisimulation. The notion of
context allows the decomposition of a process into (sub-)process and
its syntactic environment, its context. Thus, a context may be
thought of as a process with a ``hole'' (written $\Box$) in it. The
application of a context $M$ to a process $P$, written $M[P]$, is
tantamount to filling the hole in $M$ with $P$. In this paper we do
not need the full weight of this theory, but do make use of the notion
of context in the proof the main theorem. 

\begin{mathpar}
  \inferrule* [lab=summation] {} {{M_{M},M_{N}} \bc \Box \;|\; x.M_{A} \;|\; M_{M}+M_{N}}
  \and
  \inferrule* [lab=agent] {} {{M_{A}} \bc (\vec{x})M_{P} \;| \; \clift{P_0,\ldots,M_{P},\ldots,P_N}}
  \and \\
  \inferrule* [lab=process] {} {{M_{P}} \bc M_{N} \;| \;P|M_{P} }
\end{mathpar} 

\begin{mathpar}
  \inferrule* [lab=sychronization] {} {M_{N} \bc \Box \;|\; x?M_{F} \;|\; x!M_{C}}
  \and
  \inferrule* [lab=abstraction] {} {{M_{F}} \bc (x)M_{P} }
  \and
  \inferrule* [lab=concretion] {} {{M_{C}} \bc \langle M_{P} \rangle }
  \and \\
  \inferrule* [lab=process] {} {{M_{P}} \bc M_{N} \;| \;P|M_{P} }
\end{mathpar}

\begin{definition}[contextual application] Given a context $M$, and
  process $P$, we define the \emph{contextual application}, $M[P] :=
  M\{P/\Box\}$. That is, the contextual application of M to P is the
  substitution of $P$ for $\Box$ in $M$.
\end{definition}

$\meaningof{-} : L \to \mathcal{P}(\pi)$

\begin{mathpar}
  \inferrule* [lab=collection] {} {\meaningof{true} = \pi, \and \meaningof{~E} = \pi \setminus \meaningof{E}, \and \meaningof{E_{1} \& E_{2}} = \meaningof{E_{1}} \cap \meaningof{E_{2}}}
\end{mathpar}

\begin{mathpar}
  \inferrule* [lab=structure] {} {\meaningof{0} = \{ P \in \pi | P \equiv 0 \}, \and \\ \meaningof{E_1 | E_2} = \{ P \in \pi | P \equiv P_{1} | P_{2}, P_{1} \in \meaningof{E_{1}}, P_{2} \in \meaningof{E_2}\} }
\end{mathpar}

\begin{mathpar}
 \inferrule* [lab=behavior] {} {\meaningof{\langle a?b \rangle E} = \{ P \in \pi | P \equiv Q | u?(y)P', \\ \and \\\\ \and \\ \;\;\; u \in \meaningof{a}, \forall z.P'\{z/y\} \in \meaningof{E\{z/b\}}\}, \and \\ \meaningof{a!E} = \{ P \in \pi | P \equiv Q | x!\langle P' \rangle, x \in \meaningof{a} P' \in \meaningof{E}\} }
\end{mathpar}

\begin{mathpar}
 \inferrule* [lab=nominal] {} {\meaningof{\quotep{E}} = \{ \quotep{P} \in \quotep{\pi} | P \in \meaningof{E} \}, \and \meaningof{\quotep{P}} = \{ \quotep{Q} \in \quotep{\pi} | P \equiv Q \} \and \\ \meaningof{@\quotep{E}} = \{ P \in \pi | P \equiv @x, x \in \meaningof{E} \}}
\end{mathpar}

\begin{eqnarray*}
  \\
  \meaningof{-} : TS \to ST
\end{eqnarray*}

\begin{eqnarray*}
  \\
  L : TS \to ST
\end{eqnarray*}

\begin{eqnarray*}
  \\
  P \models E \iff P \in \meaningof{E}
\end{eqnarray*}

\begin{eqnarray*}
  P \approx_{L} Q \iff \forall E \in L. P \models E \iff Q \models E
\end{eqnarray*}

\begin{eqnarray*}
  P \approx_{K} Q
\end{eqnarray*}

\begin{eqnarray*}
  P \approx Q
\end{eqnarray*}

$\approx_{K} = \approx = \approx_{L}$

\subsubsection{Contextual duality}

Note that contexts extend the quotation operation to a family of
operations from processes to names. Given a context, $M$, we can
define a \emph{nominal context}, $\quotep{M}$ by $\quotep{M}[P] :=
\quotep{M[P]}$. To foreshadow what is to come we observe that these
operations enjoy a duality with processes very much like the duality
between vectors and maps from vectors to scalars.

Further, because the calculus is essentially higher-order, we have a
correspondence between contexts and processes. More specifically,
given a name $x$ and a context $M$ we can construct $M^{*}_{x}$ such
that 

\begin{mathpar}
  M^{*}_{x} | \lift{x}{P} \red M[P]
\end{mathpar}

namely,

\begin{mathpar}
  M^{*}_{x} := x?(u).M[\dropn{u}]
\end{mathpar}

The dependence of $M^{*}_{x}$ on a name makes it an abstraction, 

\begin{mathpar}
  M^{*} := (x)x?(u).M[\dropn{u}]
\end{mathpar}

\subsection{Additional notation}

It will sometimes be convenient to denote the process a name
quotes. We already have the notation $x = \quotep{P}$, but it will be
convenient to introduce an alternate notation, $\procn{x}$, when we
want to emphasize the connection to the use of the name. Note that, by
virtue of name equivalence, $\quotep{\procn{x}} \nameeq x$; so, the
notation is consistent with previous definitions.

Further, because names have structure it is possible to effect
substitutions on the basis of that structure. This means we need to
upgrade our notation for substitutions, which we accomplish by
adapting comprehension notation. Thus,

\begin{mathpar}
  P\{ y / x : x \in S \}
\end{mathpar}

is interpreted to mean the process derived from P by replacing (in a
capture-avoiding manner) each occurrence of $x$ in $S$ by $y$. For example,

\begin{mathpar}
  P\{ \quotep{\procn{x}|\procn{x}} / x : x \in \freenames{P} \}
\end{mathpar}

will replace each (occurrence) of a free name $x$ in $P$ by
$\quotep{\procn{x}|\procn{x}}$.

Also, we will avail ourselves of the notation $x^{L}$ and $x^{R}$ to
denote injections of a name into disjoint copies of the name
space. There are numerous ways to accomplish this. One example can be
found in \cite{MeredithR05}. This notation overloads to vectors of
names: $\vec{x}^{\pi} := (x_{i}^{\pi} \; : \; 0 \leq i < |\vec{x}| )$ where $\pi \in \{L,R\}$.

We also use $P^{\Box} := P|\Box$.

In \cite{MeredithR05} an interpretation of the new operator is
given. It turns out that there are several possible interpretations
all enjoying the requisite algebraic properties of the operator (see
\cite{milner91polyadicpi}). We will therefore make liberal use of
$(\nu\; \vec{x})P$.

% subsection the_syntax_and_semantics_of_the_notation_system (end)   

\section{Interpretation of QM}
\subsection{Supporting definitions}
\subsubsection{Multiplication}
\begin{mathpar}
  \quotep{Q} \cdot \quotep{R} := \quotep{Q|R}
  \and \\
  \quotep{Q} \cdot P := P\{ \quotep{Q|R} / \quotep{R} : \quotep{R} \in \freenames{P} \}
\end{mathpar}

\paragraph{Discussion}
The first line needs little explanation. The second line says that
each free name of the process is replaced with the multiplication of
that name by the scalar. Multiplication of a scalar (name) by a state
(process) results in a process all the names of which have been `moved
over' by parallel composition with the process the scalar
quotes. There is a subtlety that the bound names have to be
manipulated so that multiplied names aren't accidentally
captured. There are many ways to achieve this.

\begin{remark}\label{rem:multiplication_identities}
  The reader is invited to verify that for all $x,y,z \in \QProc$ and $P \in \Proc$
  \begin{mathpar}
    x \cdot \quotep{0} \equiv x 
    \and
    x \cdot y \equiv y \cdot x
    \and
    x \cdot (y \cdot z) \equiv (x \cdot y) \cdot z
    \and \\
    \quotep{0} \cdot P \equiv P
    \and \\
    x \cdot (y \cdot P) \equiv (x \cdot y) \cdot P
    \and \\
    x \cdot (P|Q) \equiv (x \cdot P) | (x \cdot Q)
    \and \\    
  \end{mathpar}
\end{remark}

\subsubsection{Tensor product}

We define a tensor product on processes by structural induction.

\paragraph{Tensor of sums} First note that all summations, including
$\pzero$ and sequence, can be written $\Sigma_{i} x_{i}.A_{i} +
\Sigma_{j} x_{j}.C_{j}$, where we have grouped input-guarded processes
together and output-guarded processes together.

Thus, we can define the tensor product of two summations, $N_{1}\otimes N_{2}$, where

\begin{mathpar}
  N_{1} := \Sigma_{i} x_{i}.A_{i} + \Sigma_{j} x_{j}.C_{j}
  \and
  N_{2} := \Sigma_{i'} y_{i'}.B_{i'} + \Sigma_{j'} y_{j'}.D_{j'} 
\end{mathpar}

as follows.

\begin{mathpar}
  \Sigma_{i} x_{i}.A_{i} + \Sigma_{j} x_{j}.C_{j} \otimes \Sigma_{i'}
  y_{i'}.B_{i'} + \Sigma_{j'} y_{j'}.D_{j'} 
  \and \\
  := \; \Sigma_{i} \Sigma_{i'} \quotep{\stackrel{\vee}{x_{i}}| \stackrel{\vee}{y_{i'}}}.(A_{i}\otimes B_{i'}) \; | \; \Sigma_{i'} \Sigma_{i} \quotep{\stackrel{\vee}{y_{i'}}|\stackrel{\vee}{x_{i}}}.(B_{i'}\otimes A_{i})
  \and
  \;\; | \;\; \Sigma_{j} \Sigma_{j'} \quotep{\stackrel{\vee}{x_{j}}|\stackrel{\vee}{y_{j'}}}.(A_{j}\otimes B_{j'}) \; | \; \Sigma_{j'} \Sigma_{j} \quotep{\stackrel{\vee}{y_{j'}}|\stackrel{\vee}{x_{j}}}.(B_{j'}\otimes A_{j})
\end{mathpar}

\begin{remark}
  Do we need to $x^{L}$ and $y^{R}$ for this construction as well?
\end{remark}

\paragraph{Tensor of parallel compositions} Next, we distribute tensor
over par.

\begin{mathpar}
  P_{1}|P_{2} \otimes Q_{1}|Q_{2} := (P_{1} \otimes Q_{1}) | (P_{1}
  \otimes Q_{2}) | (P_{2} \otimes Q_{1}) | (P_{2} \otimes Q_{2})
\end{mathpar}

\paragraph{Tensor with dropped names} We treat tensor of a
process with a dropped name as parallel composition.

\begin{mathpar}
  P \otimes \dropn{x} := P | \dropn{x}
\end{mathpar}

\paragraph{Tensor of agents}

Finally, we need to define tensor on agents. Note that the definition
of tensor on normal products only tensors inputs with inputs and
outputs with outputs. Thus, we only have to define the operation on
``homogeneous'' pairings.

\begin{mathpar}
  (\vec{x})P \otimes (\vec{y})Q
  \and \\
  := (x_{0}^{L}|y_{0}^{R},\ldots,x_{0}^{L}|y_{n}^{R},\ldots,x_{m}^{L}|y_{0}^{R},\ldots,x_{m}^{L}|y_{n}^R)(P\{ \vec{x}^{L}/\vec{x}\} \otimes Q \{ \vec{y}^{R}/\vec{y}\})
  \and \\
  \clift{\vec{P}} \otimes \clift{\vec{Q}}
  \and \\
  := \clift{P_{0}\otimes Q_{0},\ldots,P_{0}\otimes Q_{n},\ldots,P_{m}\otimes Q_{0},\ldots,P_{m}\otimes Q_{n}}
\end{mathpar}

\begin{remark}
  Observe that arities of tensored abstractions matches arities of
  tensored concretions if the original arities matched. Note also that
  the length of the arities corresponds to the increase in dimension
  we see in ordinary vector space tensor product.
\end{remark}

\begin{remark}
  Operationally, this definition distributes the tensor down to
  components ``linked'' by summation. Tensor over summation is
  intriguing in that it mixes names. Moreover, as a consequence of the
  way it mixes names we have the identities for all $x \in \QProc$ and
  $P,Q \in \Proc$

  \begin{mathpar}
    (x \cdot P) \otimes Q \equiv x \cdot (P \otimes Q) \equiv P \otimes (x \cdot Q)
    \and
    P \otimes \pzero \equiv P
  \end{mathpar}

  that the reader is invited to verify.
\end{remark}

\subsubsection{Annihilation}
\begin{mathpar}
  P^{\perp} := \{ Q | \forall R. P|Q \red^{*} R \Rightarrow R \red^{*} \pzero \}
  \and \\
  P^{\underline{\perp}} := \Sigma_{Q \in P^{\perp}} \quotep{Q}?(y).(\dropn{y}|Q) | \Sigma_{Q \in P^{\perp}} \quotep{Q}\clift{\Box}
\end{mathpar}

\paragraph{Discussion} The reader will note that $P^{\perp}$ is a
\emph{set} of processes, while $P^{\underline{\perp}}$ is a
\emph{context}. We call the set $P^{\perp}$ the \emph{annihilators} of
$P$. The parallel composition of a process in the annihilators of $P$
with $P$ will result in a process, the state space of which has all
paths eventually leading to $\pzero$. Execution may endure loops; but
under reasonable conditions of fairness (naturally guaranteed under
most notions of bisimulation) such a composite process cannot get
stuck in such a loop and will, eventually pop out and terminate.

The context $P^{\underline{\perp}}$ is ready and willing to ``take the
$P$ out of'' the process to which it is applied. It will effectively
transmit the code of the process to which it is applied to one of the
annihilators and run the process against it.

\subsubsection{Evaluation}
We fix $M$ a domain of fully abstract interpretation with an equality
coincident with bisimulation. We take $\meaningof{\cdot} : \Proc \to
M$ to be the map interpreting processes and $\nmeaningof{\cdot} : \M
\to Proc$ to be the map running the other way. Then we define

\begin{mathpar}
  \int P := \nmeaningof{\meaningof{P}}
\end{mathpar}

\paragraph{Discussion}
There are many fully abstract interpretations of Milner's
$\pi$-calculus. Any of them can be used as a basis for interpreting
the reflective calculus here. Equipped with such a domain it is
largely a matter of grinding through to check that the Yoneda
construction for the normalization-by-evaluation program can be
extended to this setting.

\begin{remark}
  The reader is invited to verify that $\int (P^{\underline{\perp}}[P]) = 0$.
\end{remark}

\subsection{Quantum mechanics}

Table \ref{tbl:core_qm_op_defns} gives the core operational definitions

\begin{table}[htp]\label{tbl:core_qm_op_defns}
  \center{
    \fbox{
      \begin{tabular}{c|c}
        quantum mechanics & process calculus \\
        \hline
        scalar & $x := \quotep{P}$ \\
        state vector & $\state{P} := P$ \\
        dual & $\state{P}^{*} := \event{P^{\underline{\perp}}} := \quotep{P^{\underline{\perp}}}[-]$ \\
        matrix & $ \Sigma_{\alpha} \state{P_{\alpha}}x_{\alpha}\event{Q_{\alpha}}$ \\
        vector addition & $\state{P} + \state{Q} := \state{P | Q}$ \\
        tensor product & $\state{P} \otimes \state{Q} := \state{P \otimes Q}$ \\
        inner product & $\innerprod{P}{Q} := \quotep{\int P^{\underline{\perp}}[Q]}$ \\
      \end{tabular}
    }
  }
  \caption{QM - operational definitions}
\end{table}

where

\begin{mathpar}
  \prmatrix{P}{Q} := \fprmatrix{P}{\quotep{\pzero}}{Q}
  \and
  \fprmatrix{P}{x}{Q} := (\state{P},x,\event{Q})
  \and
  (\fprmatrix{P}{x}{Q})(\state{R}) := x \cdot \innerprod{Q}{R} \cdot \state{P}
  \and
  (\fprmatrix{P}{x}{Q})(\event{R}) := x \cdot \innerprod{R}{P} \cdot \event{Q}
\end{mathpar}

\paragraph{Discussion}
As promised: vectors (aka states) are represented as processes; duals
as contextual duals; inner product definition should be compared with
standard inner product definition for ....

\begin{remark}
  Assuming $\int (P^{\underline{\perp}}[P]) = 0$, the reader is
  invited to verify that $(\fprmatrix{P}{x}{P})(\state{P}) = x \cdot \state{P}$.
\end{remark}

\begin{remark}
  The reader is invited to verify that $\innerprod{P}{Q}$ could
  equally well have been written $\quotep{\int \stackrel{\vee}{x}}$
  where $x = \event{P^{\underline{\perp}}}(Q)$.

  One of the motivations for this remark is that there is another way
  to factor these operations. We could package up evaluation in the dual:

  \begin{mathpar}
    \state{P}^{*} := \event{\int P^{\underline{\perp}}} := \quotep{\int P^{\underline{\perp}}}[-]
  \end{mathpar}

  and then have inner product defined by
  
  \begin{mathpar}
    \innerprod{P}{Q} := \event{P}(Q)
  \end{mathpar}

  Hopefully, experience with the calculations will provide guidance on
  the best factoring.
\end{remark}

\begin{remark}
  Assuming $\int (P^{\underline{\perp}}[P]) = 0$, the reader is
  invited to verify that $\forall P,Q. (\prmatrix{0}{Q})(\state{0}) =
  \state{0}$ and dually $(\prmatrix{P}{0})(\event{0}) = \event{0}$.
\end{remark}

\begin{remark}
  i'm a little worried that i don't (yet) have proper support for
  complex conjugacy. But, the observation above may give us a
  clue. According to Abramsky, it must be the case that the scalars
  are iso to the homset of the identity for the tensor -- which the
  observation above characterizes. 

  For now, we will simply bookmark the notion with $\overline{x}$.
\end{remark}

\subsubsection{Adjointness}

We need to give a definition of $(\cdot)^{\dagger}$ for matrices. The
obvious candidate definition is
\begin{mathpar}
(\Sigma_{\alpha}\fprmatrix{P_{\alpha}}{x_{\alpha}}{Q_{\alpha}})^{\dagger}
= \Sigma_{\alpha}\fprmatrix{(Q_{\alpha}^{\underline{\perp}})^{*}}{\overline{x}_{\alpha}}{P_{\alpha}^{\underline{\perp}}} 
\end{mathpar}

But, $(Q_{\alpha}^{\underline{\perp}})^{*}$ requires a name along
which to communicate the process to achieve the context application.

\subsubsection{Basis for a basis}
If processes label states and ``addition'' of states (a.k.a. vector
addition) is interpreted as parallel composition, what corresponds to
notions of linear independence and basis? Here, we recall that Yoshida
has developed a set of \emph{combinators} for an asynchronous verison
of Milner's $\pi$-calculus. These are a finite set of processes such
any process can be expressed as parallel composition of these
combinators together with liberal uses of the new operator and
replication. We can simply give a translation of these into the
present calculus and have reasonable expectation that the property
carries over. That is, that the resultant set allows to express all
processes via parallel composition. Note, however, that there is no
new operator or replication in this calculus. As a result, we expect
that the corresponding set is actually infinite. That is, we expect
that the space is actually infinite dimensional.

\begin{remark}
  The attentive reader may be a bit concerned. Certainly, the
  collection $S$, $K$ and $I$ is a finite set of
  combinators. Shouldn't we expect to see a finite set of combinators
  for an effectively equivalent system? i am very sympathetic to this
  critique and feel it warrants full attention. On the other hand, i
  also have in mind the following analogy. The natural numbers, as a
  monoid under addition, has exactly $1$ generator, while the natural
  numbers, as a monoid under multiplication, has countably many
  generators (the primes). We observe that the application of the
  lambda calculus is much less resource sensitive than the parallel
  composition of the $\pi$-calculus. Could it be the case that we have
  an analogy of the form
  
  \begin{mathpar}
    m + n : MN :: m*n : M|N
  \end{mathpar}

  giving a similar blow up in the set of ``primes''?  This is such a
  wonderful thought that, even if it's not true, i think it's worth
  writing down.
\end{remark}
 

\documentclass[12pt]{llncs}
%\documentclass{jktr}

\usepackage[pdftex]{hyperref}                   
\usepackage {listings}
\usepackage {mathpartir}
\usepackage{bcprules}
%\usepackage{listings}
                       
\usepackage{graphicx} 
%\usepackage[margins=2.5cm,nohead,nofoot]{geometry}
%\usepackage{geometry}
\usepackage{amsfonts}
\usepackage{amstext}
\usepackage{latexsym}
\usepackage{amssymb}
\usepackage{color}


%\include{myPreamble}
\documentclass[12pt]{llncs}
%\documentclass{jktr}

\usepackage[pdftex]{hyperref}                   
\usepackage {listings}
\usepackage {mathpartir}
\usepackage{bcprules}
%\usepackage{listings}
                       
\usepackage{graphicx} 
%\usepackage[margins=2.5cm,nohead,nofoot]{geometry}
%\usepackage{geometry}
\usepackage{amsfonts}
\usepackage{amstext}
\usepackage{latexsym}
\usepackage{amssymb}
\usepackage{color}


%\include{myPreamble}
\include{qm2pi.local} 

%\ifpdf
%\usepackage[pdftex]{graphicx}
%\else
%\usepackage{graphicx}
%\fi

 % \ifpdf
%  \usepackage{pdfsync}
%  \if


%\title{Brief Article}
%\author{David F. Snyder}
%\author{L.G. Meredith}

%\address{Dept. of Math., Texas State University--San Marcos, San Marcos, TX 78666}
       
\pagestyle{empty}


\begin{document}

\lstset{language=[Objective]Caml,frame=shadowbox}

\input{qm2pi.front}

% section front matter (end)

\input{qm2pi.intro} 
 
% section introduction (end)

% \input{qm2pi.knotations} 

% section notation (end)

\input{qm2pi.process.calculi} 

% section concurrent_process_calculi_and_spatial_logics_ (end)
    
%\input{qm2pi.knots2pi} 

%\input{qm2pi.trefoil} 

%\input{qm2pi.mainthm} 

% subsection basic_interpretation (end)

%\input{qm2pi.rho.presentation} 
\subsection{The syntax and semantics of the notation system}\label{sub:the_syntax_and_semantics_of_the_notation_system} % (fold)

We now summarize a technical presentation of the calculus that
embodies our theory of dynamics. The typical presentation of such a
calculus follows the style of giving generators and relations on
them. The grammar, below, describing term constructors, freely
generates the set of processes, $\Proc$. This set is then quotiented
by a relation known as structural congruence and it is over this set
that the notion of dynamics is expressed. This presentation is
essentially that of \cite{MeredithR05} with the addition of
polyadicity and summation. For readability we have relegated some of
the technical subtleties to an appendix.

\subsubsection{Process grammar}\label{subsub:process_grammar}

\begin{mathpar}
  \inferrule* [lab=synchronization] {} {{M} \bc \pzero \;|\; x?F \;|\; x!C }
  \and
  \inferrule* [lab=abstraction] {} {{F} \bc (x)P}
  \and
  \inferrule* [lab=concretion] {} {{C} \bc \langle Q \rangle}
  \and
  \inferrule* [lab=process] {} {{P,Q} \bc M \;| \;P|Q \;|\; @{x}}
  \and
  \inferrule* [lab=name] {} {{x} \bc \quotep{P}}
\end{mathpar} 

Note that $\vec{x}$ (resp. $\vec{P}$) denotes a vector of names
(resp. processes) of length $|\vec{x}|$ (resp. $|\vec{P}|$). We adopt
the following useful abbreviations.

\begin{mathpar}
   x?(\vec{y}).P := x.(\vec{y})P \and  x\clift{\vec{P}} := x.\clift{\vec{P}}
   \and x!(y) := \lift{x}{\dropn{y}}
   \and \Pi_{i=0}^{n-1}P_i := P_0 | \ldots | P_{n-1}
\end{mathpar}

\subsubsection{Structural congruence}

\paragraph{Free and bound names and alpha-equivalence.} At the
core of structural equivalence is alpha-equivalence which identifies
process that are the same up to a change of variable. Formally, we
recognize the distinction between free and bound names. The free names
of a process, $\freenames{P}$, may be calculated recursively as
follows:

\begin{mathpar}
\freenames{\pzero} := \emptyset
  \and \\
  \freenames{x?(y).P} := \{ x \} \cup (\freenames{P} \setminus \{ y \})
  \and 
  \freenames{x!\langle P \rangle} := \{ x \} \cup \{ P \} 
  \and \\
  \freenames{P|Q} := \freenames{P} \cup \freenames{Q}
  \and \\
  \freenames{@{x}} := \{ x \}
\end{mathpar}

$\pi$
$\quotep{\pi}$

$\freenames{-} : \pi \to \mathcal{P}(\quotep{\pi})$

\begin{eqnarray*}
  \freenames{\pzero} & := & \emptyset \\
  \freenames{x?(y).P} & := & \{ x \} \cup (\freenames{P} \setminus \{ y \}) \\
  \freenames{x!\langle P \rangle} & := & \{ x \} \cup \{ P \} \\
  \freenames{P|Q} & := & \freenames{P} \cup \freenames{Q} \\
  \freenames{\dropn{x}} & := & \{ x \}
\end{eqnarray*}

The bound names of a process, $\boundnames{P}$, are those names occurring in $P$
that are not free. For example, in $x?(y).0$, the name $x$ is free, while $y$ is bound.

\begin{mathpar}
  \inferrule* [lab=monoidal-laws] {} { P|Q \equiv Q|P \and P|0 \equiv P \and P|(Q|R) \equiv (P|Q)|R }
\end{mathpar}

\begin{mathpar}
  \inferrule* [lab=alpha-equivalence] {} { (x)P \equiv (y)P\{y/x\} \and y \not\in \freenames{P} }
\end{mathpar}

\begin{definition}
Then two processes, $P,Q$, are alpha-equivalent if $P = Q\{\vec{y}/\vec{x}\}$ for
some $\vec{x} \in \boundnames{Q},\vec{y} \in \boundnames{P}$, where $Q\{\vec{y}/\vec{x}\}$
denotes the capture-avoiding substitution of $\vec{y}$ for $\vec{x}$ in $Q$.
\end{definition}

\begin{definition}
  The {\em structural congruence} \cite{SangiorgiWalker} , $\equiv$,
  between processes is the least congruence containing
  alpha-equivalence, satisfying the abelian monoid laws
  (associativity, commutativity and $\pzero$ as identity) for parallel
  composition $|$ and for summation $+$.
\end{definition}

\subsection{Name equivalence}

We take name equivalence, written $\nameeq$, to be the smallest
equivalence relation generated by the following rules.

\begin{mathpar}
\inferrule*[lab=Quote-drop]
{ }
{ \quotep{@{x}} \nameeq x }

\inferrule*[lab=Struct-equiv]
{ P \scong Q }
{ \quotep{P} \nameeq \quotep{Q} }
\end{mathpar}

The astute reader will have noticed that the mutual recursion of names
and processes imposes a mutual recursion on alpha-equivalence and
structural equivalence via name-equivalence. Fortunately, all of this
works out pleasantly and we may calculate in the natural way, free of
concern. The reader interested in the details is referred to the
appendix \ref{appendix:rho_details}.

\subsection{Substitution}

We use $\Proc$ for the set of processes, $\QProc$ for the set of
names, and $\id{\{}\vec{y} / \vec{x} \id{\}}$ to denote partial maps,
$s : \QProc \rightarrow \QProc$. A map, $s$ lifts, uniquely, to a map
on process terms, $\widehat{s} : \Proc \rightarrow \Proc$ by the
following equations.

\begin{mathpar}
  (0) \psubstp{Q}{P} := 0 \\
  (R \juxtap S) \psubstp{Q}{P}
  :=    
  (R)\psubstp{Q}{P} \juxtap (S) \psubstp{Q}{P} \\
  (x?(y).R) \psubstp{Q}{P}    
  :=    
  (x)\substp{Q}{P} (z)\concat( (R \psubstn{z}{y}) \psubstp{Q}{P} ) \\
  (\lift{x}{R}) \psubstp{Q}{P}  
  :=
  \lift{(x)\substp{Q}{P}}{ R \psubstp{Q}{P} } \\
%   (\dropn{x})  \psubstp{Q}{P}       
%   := 
%   \left\{ 
%     \begin{array}{ccc} 
%       \dropn{\quotep{Q}} & & x \nameeq \quotep{P} \\
%       \dropn{x} & & otherwise \\
%     \end{array}
%   \right. 
  (\dropn{x})  \psubstp{Q}{P}       
  := 
  \left\{ 
    \begin{array}{ccc} 
      Q & & x \nameeq \quotep{P} \\
      \dropn{x} & & otherwise \\
    \end{array}
  \right.
\end{mathpar}
 

where

\begin{eqnarray}
  (x)\id{\{} \lpquote Q \rpquote / \lpquote P \rpquote \id{\}}            = 
  \left\{ 
    \begin{array}{ccc}
      \lpquote Q \rpquote & & x \nameeq \lpquote P \rpquote \\
      x & & otherwise \\
    \end{array}
  \right. \nonumber
\end{eqnarray}

and $z$ is chosen distinct from $\quotep{P}$, $\quotep{Q}$, the free
names in $Q$, and all the names in $R$. Our $\alpha$-equivalence will
be built in the standard way from this substitution.

\begin{remark}\label{rem:no_self_referential_names}
  One consequence of these definitions is that $\forall P. \quotep{P}
  \not\in \freenames{P}$.
\end{remark}

\subsection{ Dynamic quote: an example }

Anticipating something of what's to come, consider applying the
substitution, $\widehat{\id{\{}u / z \id{\}}}$, to the following pair
of processes, $\lift{w}{y!(z)}$ and $w[ \lpquote y!(z) \rpquote ]$.

\begin{eqnarray}
	\lift{w}{y!(z)}\widehat{\id{\{}u / z \id{\}}}
		& = &
		\lift{w}{y!(u)} \nonumber\\
	w[ \lpquote y!(z) \rpquote ] \widehat{ \id{\{}u / z \id{\}} }
		& = &
		w[ \lpquote y!(z) \rpquote ] \nonumber
\end{eqnarray}

Because the body of the process between quotes is impervious to
substitution, we get radically different answers. In fact, by
examining the first process in an input context,
e.g. $x?(z).\lift{w}{y!(z)}$, we see that the process under the lift
operator may be shaped by prefixed inputs binding a name inside it. In
this sense, the lift operator will be seen as a way to dynamically
construct processes before reifying them as names.

Finally equipped with these standard features we can present the
dynamics of the calculus.

\subsubsection{Operational semantics} 

Finally, we introduce the computational dynamics. What marks these
algebras as distinct from other more traditionally studied algebraic
structures, e.g. vector spaces or polynomial rings, is the manner in
which dynamics is captured. In traditional structures, dynamics is typically
expressed through morphisms between such structures, as in linear maps
between vector spaces or morphisms between rings. In algebras
associated with the semantics of computation, the dynamics is
expressed as part of the algebraic structure itself, through a
reduction reduction relation typically denoted by $\red$. Below, we
give a recursive presentation of this relation for the calculus used
in the encoding.

$\red \subseteq \pi \times \pi$
$\red : \pi \to \mathcal{P}(\pi)$

\begin{mathpar}
  \inferrule* [lab=Comm] { \textsf{match}( x_{src}, x_{trgt} ) } { x_{trgt}?(y)P \; | \; x_{src}!\langle {Q} \rangle \red P\{\quotep{Q}/y}\} }
  \and \\
  \inferrule* [lab=Par] {{P} \red {P}'} {{{P} | {Q}} \red {{P}' | {Q}}}
  \and
  \inferrule* [lab=Equiv]{{{P} \scong {P}'} \andalso {{P}' \red {Q}'} \andalso {{Q}' \scong {Q}}}{{P} \red {Q}}
\end{mathpar}

\begin{eqnarray*}
  match_{\equiv} (\quotep{P},\quotep{Q}) & := & P \equiv Q \\
  match_{\dagger}(\quotep{P},\quotep{Q}) & := & \forall R. P|Q \red^{*} R => R \red^{*} 0 \\
  match_{K}(\quotep{P},\quotep{Q}) & := & K \mbox{ for some context } K
\end{eqnarray*}

$u?(x)P | u!\langle Q \rangle \red P\{\quotep{Q}/x\}$

%We write $\wred$ for $\red^*$, and $P\red$ if $\exists Q $ such that $ P \red Q$.
We write $P\red$ if $\exists Q $ such that $ P \red Q$ and $P\not\red$, otherwise.

\section{Replication}

As mentioned before, it is known that replication (and hence
recursion) can be implemented in a higher-order process algebra
\cite{SangiorgiWalker}. As our first example of calculation with the
machinery thus far presented we give the construction explicitly in
the {\rhoc}.

\begin{eqnarray}
	D_{x} & := & \prefix{x}{y}{(\binpar{\outputp{x}{y}}{@{y}})} \nonumber\\
	\bangp_{x}{P} & := & \binpar{{x}!\langle{\binpar{D_{x}}{P}}\rangle}{D_{x}} \nonumber
\end{eqnarray}

\begin{eqnarray}
	\bangp_{x}{P} & & \nonumber\\
	=
	& {x}!\langle{(\prefix{x}{y}{(\outputp{x}{y} | @{y})) | P}}\rangle 
	      | \prefix{x}{y}{(\outputp{x}{y} | @{y})} & \nonumber\\
	\red
	& (\outputp{x}{y} | @{y})\substn{\quotep{(\prefix{x}{y}{(@{y} | \outputp{x}{y})) | P}}}{y} & \nonumber\\
	=
	& \outputp{x}{\quotep{(\prefix{x}{y}{(\outputp{x}{y} | @{y})) | P}}}
	  | {(\prefix{x}{y}{(\outputp{x}{y} | @{y})) | P}} & \nonumber\\
	\red
	& \ldots & \nonumber\\
	\red^*
	& P | P | \ldots & \nonumber
\end{eqnarray}

Of course, this encoding, as an implementation, runs away, unfolding
$\bangp{P}$ eagerly. A lazier and more implementable replication
operator, restricted to input-guarded processes, may be obtained as follows.

\begin{eqnarray}
\bangp{\prefix{u}{v}{P}} 
	:= 
	\binpar{\lift{x}{\prefix{u}{v}{(\binpar{D(x)}{P})}}}{D(x)} \nonumber
\end{eqnarray}

\begin{remark}
  Note that the lazier definition still does not deal with summation
  or mixed summation (i.e. sums over input and output). The reader is
  invited to construct definitions of replication that deal with these
  features. 

  Further, the definitions are parameterized in a name, $x$. Can you,
  gentle reader, make a definition that eliminates this parameter and
  guarantees no accidental interaction between the replication
  machinery and the process being replicated -- i.e. no accidental
  sharing of names used by the process to get its work done and the
  name(s) used by the replication to effect copying. This latter
  revision of the definition of replication is crucial to obtaining
  the expected identity $!!P \sim !P$.
\end{remark}

\begin{remark}\label{rem:paradoxical_combinator}
  The reader familiar with the lambda calculus will have noticed the
  similarity between $D$ and the paradoxical combinator.

  [Ed. note: the existence of this seems to suggest we have to be more
  restrictive on the set of processes and names we admit if we are to
  support no-cloning.]
\end{remark}

\subsubsection{Bisimulation}

The computational dynamics gives rise to another kind of equivalence,
the equivalence of computational behavior. As previously mentioned
this is typically captured \emph{via} some form of bisimulation.

% The notion we use in this paper is weak barbed bisimulation
% \cite{milner91polyadicpi}.

The notion we use in this paper is derived from weak barbed
bisimulation \cite{milner91polyadicpi}. 

\begin{definition}
An \emph{observation relation}, $\downarrow_{\mathcal N}$, over a set
of names, $\mathcal N$, is the smallest relation satisfying the rules
below.

\infrule[Out-barb]{y \in {\mathcal N}, \; x \nameeq y}
		  {\outputp{x}{v} \downarrow_{\mathcal N} x}
\infrule[Par-barb]{\mbox{$P\downarrow_{\mathcal N} x$ or $Q\downarrow_{\mathcal N} x$}}
		  {\binpar{P}{Q} \downarrow_{\mathcal N} x}

We write $P \Downarrow_{\mathcal N} x$ if there is $Q$ such that 
$P \wred Q$ and $Q \downarrow_{\mathcal N} x$.
\end{definition}

\begin{definition}
%\label{def.bbisim}
An  ${\mathcal N}$-\emph{barbed bisimulation} over a set of names, ${\mathcal N}$, is a symmetric binary relation 
${\mathcal S}_{\mathcal N}$ between agents such that $P\rel{S}_{\mathcal N}Q$ implies:
\begin{enumerate}
\item If $P \red P'$ then $Q \wred Q'$ and $P'\rel{S}_{\mathcal N} Q'$.
\item If $P\downarrow_{\mathcal N} x$, then $Q\Downarrow_{\mathcal N} x$.
\end{enumerate}
$P$ is ${\mathcal N}$-barbed bisimilar to $Q$, written
$P \wbbisim_{\mathcal N} Q$, if $P \rel{S}_{\mathcal N} Q$ for some ${\mathcal N}$-barbed bisimulation ${\mathcal S}_{\mathcal N}$.
\end{definition}

$\mathcal{R} \subseteq \pi \times \pi$

$P \mathcal{R} Q => \forall P'. P \red P' \Rightarrow \exists Q'. Q \red Q', P' \mathcal{R} Q'$

$P \vdash x \Rightarrow Q \vdash x$

\begin{mathpar}
  \inferrule*[lab=Out-barb]{x \nameeq y}{{y}!\langle{Q}\rangle \vdash x}
  \and
  \inferrule*[lab=Par-barb]{\mbox{$P\vdash x$ or $Q\vdash x$}}{\binpar{P}{Q} \vdash x}
\end{mathpar}

\subsubsection{Contexts}

One of the principle advantages of computational calculi like the
$\pi$-calculus is a well-defined notion of context,
contextual-equivalence and a correlation between
contextual-equivalence and notions of bisimulation. The notion of
context allows the decomposition of a process into (sub-)process and
its syntactic environment, its context. Thus, a context may be
thought of as a process with a ``hole'' (written $\Box$) in it. The
application of a context $M$ to a process $P$, written $M[P]$, is
tantamount to filling the hole in $M$ with $P$. In this paper we do
not need the full weight of this theory, but do make use of the notion
of context in the proof the main theorem. 

\begin{mathpar}
  \inferrule* [lab=summation] {} {{M_{M},M_{N}} \bc \Box \;|\; x.M_{A} \;|\; M_{M}+M_{N}}
  \and
  \inferrule* [lab=agent] {} {{M_{A}} \bc (\vec{x})M_{P} \;| \; \clift{P_0,\ldots,M_{P},\ldots,P_N}}
  \and \\
  \inferrule* [lab=process] {} {{M_{P}} \bc M_{N} \;| \;P|M_{P} }
\end{mathpar} 

\begin{mathpar}
  \inferrule* [lab=sychronization] {} {M_{N} \bc \Box \;|\; x?M_{F} \;|\; x!M_{C}}
  \and
  \inferrule* [lab=abstraction] {} {{M_{F}} \bc (x)M_{P} }
  \and
  \inferrule* [lab=concretion] {} {{M_{C}} \bc \langle M_{P} \rangle }
  \and \\
  \inferrule* [lab=process] {} {{M_{P}} \bc M_{N} \;| \;P|M_{P} }
\end{mathpar}

\begin{definition}[contextual application] Given a context $M$, and
  process $P$, we define the \emph{contextual application}, $M[P] :=
  M\{P/\Box\}$. That is, the contextual application of M to P is the
  substitution of $P$ for $\Box$ in $M$.
\end{definition}

$\meaningof{-} : L \to \mathcal{P}(\pi)$

\begin{mathpar}
  \inferrule* [lab=collection] {} {\meaningof{true} = \pi, \and \meaningof{~E} = \pi \setminus \meaningof{E}, \and \meaningof{E_{1} \& E_{2}} = \meaningof{E_{1}} \cap \meaningof{E_{2}}}
\end{mathpar}

\begin{mathpar}
  \inferrule* [lab=structure] {} {\meaningof{0} = \{ P \in \pi | P \equiv 0 \}, \and \\ \meaningof{E_1 | E_2} = \{ P \in \pi | P \equiv P_{1} | P_{2}, P_{1} \in \meaningof{E_{1}}, P_{2} \in \meaningof{E_2}\} }
\end{mathpar}

\begin{mathpar}
 \inferrule* [lab=behavior] {} {\meaningof{\langle a?b \rangle E} = \{ P \in \pi | P \equiv Q | u?(y)P', \\ \and \\\\ \and \\ \;\;\; u \in \meaningof{a}, \forall z.P'\{z/y\} \in \meaningof{E\{z/b\}}\}, \and \\ \meaningof{a!E} = \{ P \in \pi | P \equiv Q | x!\langle P' \rangle, x \in \meaningof{a} P' \in \meaningof{E}\} }
\end{mathpar}

\begin{mathpar}
 \inferrule* [lab=nominal] {} {\meaningof{\quotep{E}} = \{ \quotep{P} \in \quotep{\pi} | P \in \meaningof{E} \}, \and \meaningof{\quotep{P}} = \{ \quotep{Q} \in \quotep{\pi} | P \equiv Q \} \and \\ \meaningof{@\quotep{E}} = \{ P \in \pi | P \equiv @x, x \in \meaningof{E} \}}
\end{mathpar}

\begin{eqnarray*}
  \\
  \meaningof{-} : TS \to ST
\end{eqnarray*}

\begin{eqnarray*}
  \\
  L : TS \to ST
\end{eqnarray*}

\begin{eqnarray*}
  \\
  P \models E \iff P \in \meaningof{E}
\end{eqnarray*}

\begin{eqnarray*}
  P \approx_{L} Q \iff \forall E \in L. P \models E \iff Q \models E
\end{eqnarray*}

\begin{eqnarray*}
  P \approx_{K} Q
\end{eqnarray*}

\begin{eqnarray*}
  P \approx Q
\end{eqnarray*}

$\approx_{K} = \approx = \approx_{L}$

\subsubsection{Contextual duality}

Note that contexts extend the quotation operation to a family of
operations from processes to names. Given a context, $M$, we can
define a \emph{nominal context}, $\quotep{M}$ by $\quotep{M}[P] :=
\quotep{M[P]}$. To foreshadow what is to come we observe that these
operations enjoy a duality with processes very much like the duality
between vectors and maps from vectors to scalars.

Further, because the calculus is essentially higher-order, we have a
correspondence between contexts and processes. More specifically,
given a name $x$ and a context $M$ we can construct $M^{*}_{x}$ such
that 

\begin{mathpar}
  M^{*}_{x} | \lift{x}{P} \red M[P]
\end{mathpar}

namely,

\begin{mathpar}
  M^{*}_{x} := x?(u).M[\dropn{u}]
\end{mathpar}

The dependence of $M^{*}_{x}$ on a name makes it an abstraction, 

\begin{mathpar}
  M^{*} := (x)x?(u).M[\dropn{u}]
\end{mathpar}

\subsection{Additional notation}

It will sometimes be convenient to denote the process a name
quotes. We already have the notation $x = \quotep{P}$, but it will be
convenient to introduce an alternate notation, $\procn{x}$, when we
want to emphasize the connection to the use of the name. Note that, by
virtue of name equivalence, $\quotep{\procn{x}} \nameeq x$; so, the
notation is consistent with previous definitions.

Further, because names have structure it is possible to effect
substitutions on the basis of that structure. This means we need to
upgrade our notation for substitutions, which we accomplish by
adapting comprehension notation. Thus,

\begin{mathpar}
  P\{ y / x : x \in S \}
\end{mathpar}

is interpreted to mean the process derived from P by replacing (in a
capture-avoiding manner) each occurrence of $x$ in $S$ by $y$. For example,

\begin{mathpar}
  P\{ \quotep{\procn{x}|\procn{x}} / x : x \in \freenames{P} \}
\end{mathpar}

will replace each (occurrence) of a free name $x$ in $P$ by
$\quotep{\procn{x}|\procn{x}}$.

Also, we will avail ourselves of the notation $x^{L}$ and $x^{R}$ to
denote injections of a name into disjoint copies of the name
space. There are numerous ways to accomplish this. One example can be
found in \cite{MeredithR05}. This notation overloads to vectors of
names: $\vec{x}^{\pi} := (x_{i}^{\pi} \; : \; 0 \leq i < |\vec{x}| )$ where $\pi \in \{L,R\}$.

We also use $P^{\Box} := P|\Box$.

In \cite{MeredithR05} an interpretation of the new operator is
given. It turns out that there are several possible interpretations
all enjoying the requisite algebraic properties of the operator (see
\cite{milner91polyadicpi}). We will therefore make liberal use of
$(\nu\; \vec{x})P$.

% subsection the_syntax_and_semantics_of_the_notation_system (end)   

\input{qm2pi.qmops} 

\input{qm2pi.sterngerlach} 

\input{qm2pi.metric} 

% section concurrent_process_calculi (end)

%\input{qm2pi.proofsketch}

% section proof sketch (end)

%\input{qm2pi.slviaknots} 

% section spatial logic via knots (end)

\input{qm2pi.conclusion}

% section conclusion (end)

%\input{qm2pi.dtcodes} 

% section wiring algorithm (end)

\input{qm2pi.ack} 

% section acknowledgments (end)

\newpage


\bibliographystyle{plain}   
\bibliography{../../biblios/main.bib}

\input{qm2pi.rhodetails}

\end{document}

 

%\ifpdf
%\usepackage[pdftex]{graphicx}
%\else
%\usepackage{graphicx}
%\fi

 % \ifpdf
%  \usepackage{pdfsync}
%  \if


%\title{Brief Article}
%\author{David F. Snyder}
%\author{L.G. Meredith}

%\address{Dept. of Math., Texas State University--San Marcos, San Marcos, TX 78666}
       
\pagestyle{empty}


\begin{document}

\lstset{language=[Objective]Caml,frame=shadowbox}

\documentclass[12pt]{llncs}
%\documentclass{jktr}

\usepackage[pdftex]{hyperref}                   
\usepackage {listings}
\usepackage {mathpartir}
\usepackage{bcprules}
%\usepackage{listings}
                       
\usepackage{graphicx} 
%\usepackage[margins=2.5cm,nohead,nofoot]{geometry}
%\usepackage{geometry}
\usepackage{amsfonts}
\usepackage{amstext}
\usepackage{latexsym}
\usepackage{amssymb}
\usepackage{color}


%\include{myPreamble}
\include{qm2pi.local} 

%\ifpdf
%\usepackage[pdftex]{graphicx}
%\else
%\usepackage{graphicx}
%\fi

 % \ifpdf
%  \usepackage{pdfsync}
%  \if


%\title{Brief Article}
%\author{David F. Snyder}
%\author{L.G. Meredith}

%\address{Dept. of Math., Texas State University--San Marcos, San Marcos, TX 78666}
       
\pagestyle{empty}


\begin{document}

\lstset{language=[Objective]Caml,frame=shadowbox}

\input{qm2pi.front}

% section front matter (end)

\input{qm2pi.intro} 
 
% section introduction (end)

% \input{qm2pi.knotations} 

% section notation (end)

\input{qm2pi.process.calculi} 

% section concurrent_process_calculi_and_spatial_logics_ (end)
    
%\input{qm2pi.knots2pi} 

%\input{qm2pi.trefoil} 

%\input{qm2pi.mainthm} 

% subsection basic_interpretation (end)

%\input{qm2pi.rho.presentation} 
\subsection{The syntax and semantics of the notation system}\label{sub:the_syntax_and_semantics_of_the_notation_system} % (fold)

We now summarize a technical presentation of the calculus that
embodies our theory of dynamics. The typical presentation of such a
calculus follows the style of giving generators and relations on
them. The grammar, below, describing term constructors, freely
generates the set of processes, $\Proc$. This set is then quotiented
by a relation known as structural congruence and it is over this set
that the notion of dynamics is expressed. This presentation is
essentially that of \cite{MeredithR05} with the addition of
polyadicity and summation. For readability we have relegated some of
the technical subtleties to an appendix.

\subsubsection{Process grammar}\label{subsub:process_grammar}

\begin{mathpar}
  \inferrule* [lab=synchronization] {} {{M} \bc \pzero \;|\; x?F \;|\; x!C }
  \and
  \inferrule* [lab=abstraction] {} {{F} \bc (x)P}
  \and
  \inferrule* [lab=concretion] {} {{C} \bc \langle Q \rangle}
  \and
  \inferrule* [lab=process] {} {{P,Q} \bc M \;| \;P|Q \;|\; @{x}}
  \and
  \inferrule* [lab=name] {} {{x} \bc \quotep{P}}
\end{mathpar} 

Note that $\vec{x}$ (resp. $\vec{P}$) denotes a vector of names
(resp. processes) of length $|\vec{x}|$ (resp. $|\vec{P}|$). We adopt
the following useful abbreviations.

\begin{mathpar}
   x?(\vec{y}).P := x.(\vec{y})P \and  x\clift{\vec{P}} := x.\clift{\vec{P}}
   \and x!(y) := \lift{x}{\dropn{y}}
   \and \Pi_{i=0}^{n-1}P_i := P_0 | \ldots | P_{n-1}
\end{mathpar}

\subsubsection{Structural congruence}

\paragraph{Free and bound names and alpha-equivalence.} At the
core of structural equivalence is alpha-equivalence which identifies
process that are the same up to a change of variable. Formally, we
recognize the distinction between free and bound names. The free names
of a process, $\freenames{P}$, may be calculated recursively as
follows:

\begin{mathpar}
\freenames{\pzero} := \emptyset
  \and \\
  \freenames{x?(y).P} := \{ x \} \cup (\freenames{P} \setminus \{ y \})
  \and 
  \freenames{x!\langle P \rangle} := \{ x \} \cup \{ P \} 
  \and \\
  \freenames{P|Q} := \freenames{P} \cup \freenames{Q}
  \and \\
  \freenames{@{x}} := \{ x \}
\end{mathpar}

$\pi$
$\quotep{\pi}$

$\freenames{-} : \pi \to \mathcal{P}(\quotep{\pi})$

\begin{eqnarray*}
  \freenames{\pzero} & := & \emptyset \\
  \freenames{x?(y).P} & := & \{ x \} \cup (\freenames{P} \setminus \{ y \}) \\
  \freenames{x!\langle P \rangle} & := & \{ x \} \cup \{ P \} \\
  \freenames{P|Q} & := & \freenames{P} \cup \freenames{Q} \\
  \freenames{\dropn{x}} & := & \{ x \}
\end{eqnarray*}

The bound names of a process, $\boundnames{P}$, are those names occurring in $P$
that are not free. For example, in $x?(y).0$, the name $x$ is free, while $y$ is bound.

\begin{mathpar}
  \inferrule* [lab=monoidal-laws] {} { P|Q \equiv Q|P \and P|0 \equiv P \and P|(Q|R) \equiv (P|Q)|R }
\end{mathpar}

\begin{mathpar}
  \inferrule* [lab=alpha-equivalence] {} { (x)P \equiv (y)P\{y/x\} \and y \not\in \freenames{P} }
\end{mathpar}

\begin{definition}
Then two processes, $P,Q$, are alpha-equivalent if $P = Q\{\vec{y}/\vec{x}\}$ for
some $\vec{x} \in \boundnames{Q},\vec{y} \in \boundnames{P}$, where $Q\{\vec{y}/\vec{x}\}$
denotes the capture-avoiding substitution of $\vec{y}$ for $\vec{x}$ in $Q$.
\end{definition}

\begin{definition}
  The {\em structural congruence} \cite{SangiorgiWalker} , $\equiv$,
  between processes is the least congruence containing
  alpha-equivalence, satisfying the abelian monoid laws
  (associativity, commutativity and $\pzero$ as identity) for parallel
  composition $|$ and for summation $+$.
\end{definition}

\subsection{Name equivalence}

We take name equivalence, written $\nameeq$, to be the smallest
equivalence relation generated by the following rules.

\begin{mathpar}
\inferrule*[lab=Quote-drop]
{ }
{ \quotep{@{x}} \nameeq x }

\inferrule*[lab=Struct-equiv]
{ P \scong Q }
{ \quotep{P} \nameeq \quotep{Q} }
\end{mathpar}

The astute reader will have noticed that the mutual recursion of names
and processes imposes a mutual recursion on alpha-equivalence and
structural equivalence via name-equivalence. Fortunately, all of this
works out pleasantly and we may calculate in the natural way, free of
concern. The reader interested in the details is referred to the
appendix \ref{appendix:rho_details}.

\subsection{Substitution}

We use $\Proc$ for the set of processes, $\QProc$ for the set of
names, and $\id{\{}\vec{y} / \vec{x} \id{\}}$ to denote partial maps,
$s : \QProc \rightarrow \QProc$. A map, $s$ lifts, uniquely, to a map
on process terms, $\widehat{s} : \Proc \rightarrow \Proc$ by the
following equations.

\begin{mathpar}
  (0) \psubstp{Q}{P} := 0 \\
  (R \juxtap S) \psubstp{Q}{P}
  :=    
  (R)\psubstp{Q}{P} \juxtap (S) \psubstp{Q}{P} \\
  (x?(y).R) \psubstp{Q}{P}    
  :=    
  (x)\substp{Q}{P} (z)\concat( (R \psubstn{z}{y}) \psubstp{Q}{P} ) \\
  (\lift{x}{R}) \psubstp{Q}{P}  
  :=
  \lift{(x)\substp{Q}{P}}{ R \psubstp{Q}{P} } \\
%   (\dropn{x})  \psubstp{Q}{P}       
%   := 
%   \left\{ 
%     \begin{array}{ccc} 
%       \dropn{\quotep{Q}} & & x \nameeq \quotep{P} \\
%       \dropn{x} & & otherwise \\
%     \end{array}
%   \right. 
  (\dropn{x})  \psubstp{Q}{P}       
  := 
  \left\{ 
    \begin{array}{ccc} 
      Q & & x \nameeq \quotep{P} \\
      \dropn{x} & & otherwise \\
    \end{array}
  \right.
\end{mathpar}
 

where

\begin{eqnarray}
  (x)\id{\{} \lpquote Q \rpquote / \lpquote P \rpquote \id{\}}            = 
  \left\{ 
    \begin{array}{ccc}
      \lpquote Q \rpquote & & x \nameeq \lpquote P \rpquote \\
      x & & otherwise \\
    \end{array}
  \right. \nonumber
\end{eqnarray}

and $z$ is chosen distinct from $\quotep{P}$, $\quotep{Q}$, the free
names in $Q$, and all the names in $R$. Our $\alpha$-equivalence will
be built in the standard way from this substitution.

\begin{remark}\label{rem:no_self_referential_names}
  One consequence of these definitions is that $\forall P. \quotep{P}
  \not\in \freenames{P}$.
\end{remark}

\subsection{ Dynamic quote: an example }

Anticipating something of what's to come, consider applying the
substitution, $\widehat{\id{\{}u / z \id{\}}}$, to the following pair
of processes, $\lift{w}{y!(z)}$ and $w[ \lpquote y!(z) \rpquote ]$.

\begin{eqnarray}
	\lift{w}{y!(z)}\widehat{\id{\{}u / z \id{\}}}
		& = &
		\lift{w}{y!(u)} \nonumber\\
	w[ \lpquote y!(z) \rpquote ] \widehat{ \id{\{}u / z \id{\}} }
		& = &
		w[ \lpquote y!(z) \rpquote ] \nonumber
\end{eqnarray}

Because the body of the process between quotes is impervious to
substitution, we get radically different answers. In fact, by
examining the first process in an input context,
e.g. $x?(z).\lift{w}{y!(z)}$, we see that the process under the lift
operator may be shaped by prefixed inputs binding a name inside it. In
this sense, the lift operator will be seen as a way to dynamically
construct processes before reifying them as names.

Finally equipped with these standard features we can present the
dynamics of the calculus.

\subsubsection{Operational semantics} 

Finally, we introduce the computational dynamics. What marks these
algebras as distinct from other more traditionally studied algebraic
structures, e.g. vector spaces or polynomial rings, is the manner in
which dynamics is captured. In traditional structures, dynamics is typically
expressed through morphisms between such structures, as in linear maps
between vector spaces or morphisms between rings. In algebras
associated with the semantics of computation, the dynamics is
expressed as part of the algebraic structure itself, through a
reduction reduction relation typically denoted by $\red$. Below, we
give a recursive presentation of this relation for the calculus used
in the encoding.

$\red \subseteq \pi \times \pi$
$\red : \pi \to \mathcal{P}(\pi)$

\begin{mathpar}
  \inferrule* [lab=Comm] { \textsf{match}( x_{src}, x_{trgt} ) } { x_{trgt}?(y)P \; | \; x_{src}!\langle {Q} \rangle \red P\{\quotep{Q}/y}\} }
  \and \\
  \inferrule* [lab=Par] {{P} \red {P}'} {{{P} | {Q}} \red {{P}' | {Q}}}
  \and
  \inferrule* [lab=Equiv]{{{P} \scong {P}'} \andalso {{P}' \red {Q}'} \andalso {{Q}' \scong {Q}}}{{P} \red {Q}}
\end{mathpar}

\begin{eqnarray*}
  match_{\equiv} (\quotep{P},\quotep{Q}) & := & P \equiv Q \\
  match_{\dagger}(\quotep{P},\quotep{Q}) & := & \forall R. P|Q \red^{*} R => R \red^{*} 0 \\
  match_{K}(\quotep{P},\quotep{Q}) & := & K \mbox{ for some context } K
\end{eqnarray*}

$u?(x)P | u!\langle Q \rangle \red P\{\quotep{Q}/x\}$

%We write $\wred$ for $\red^*$, and $P\red$ if $\exists Q $ such that $ P \red Q$.
We write $P\red$ if $\exists Q $ such that $ P \red Q$ and $P\not\red$, otherwise.

\section{Replication}

As mentioned before, it is known that replication (and hence
recursion) can be implemented in a higher-order process algebra
\cite{SangiorgiWalker}. As our first example of calculation with the
machinery thus far presented we give the construction explicitly in
the {\rhoc}.

\begin{eqnarray}
	D_{x} & := & \prefix{x}{y}{(\binpar{\outputp{x}{y}}{@{y}})} \nonumber\\
	\bangp_{x}{P} & := & \binpar{{x}!\langle{\binpar{D_{x}}{P}}\rangle}{D_{x}} \nonumber
\end{eqnarray}

\begin{eqnarray}
	\bangp_{x}{P} & & \nonumber\\
	=
	& {x}!\langle{(\prefix{x}{y}{(\outputp{x}{y} | @{y})) | P}}\rangle 
	      | \prefix{x}{y}{(\outputp{x}{y} | @{y})} & \nonumber\\
	\red
	& (\outputp{x}{y} | @{y})\substn{\quotep{(\prefix{x}{y}{(@{y} | \outputp{x}{y})) | P}}}{y} & \nonumber\\
	=
	& \outputp{x}{\quotep{(\prefix{x}{y}{(\outputp{x}{y} | @{y})) | P}}}
	  | {(\prefix{x}{y}{(\outputp{x}{y} | @{y})) | P}} & \nonumber\\
	\red
	& \ldots & \nonumber\\
	\red^*
	& P | P | \ldots & \nonumber
\end{eqnarray}

Of course, this encoding, as an implementation, runs away, unfolding
$\bangp{P}$ eagerly. A lazier and more implementable replication
operator, restricted to input-guarded processes, may be obtained as follows.

\begin{eqnarray}
\bangp{\prefix{u}{v}{P}} 
	:= 
	\binpar{\lift{x}{\prefix{u}{v}{(\binpar{D(x)}{P})}}}{D(x)} \nonumber
\end{eqnarray}

\begin{remark}
  Note that the lazier definition still does not deal with summation
  or mixed summation (i.e. sums over input and output). The reader is
  invited to construct definitions of replication that deal with these
  features. 

  Further, the definitions are parameterized in a name, $x$. Can you,
  gentle reader, make a definition that eliminates this parameter and
  guarantees no accidental interaction between the replication
  machinery and the process being replicated -- i.e. no accidental
  sharing of names used by the process to get its work done and the
  name(s) used by the replication to effect copying. This latter
  revision of the definition of replication is crucial to obtaining
  the expected identity $!!P \sim !P$.
\end{remark}

\begin{remark}\label{rem:paradoxical_combinator}
  The reader familiar with the lambda calculus will have noticed the
  similarity between $D$ and the paradoxical combinator.

  [Ed. note: the existence of this seems to suggest we have to be more
  restrictive on the set of processes and names we admit if we are to
  support no-cloning.]
\end{remark}

\subsubsection{Bisimulation}

The computational dynamics gives rise to another kind of equivalence,
the equivalence of computational behavior. As previously mentioned
this is typically captured \emph{via} some form of bisimulation.

% The notion we use in this paper is weak barbed bisimulation
% \cite{milner91polyadicpi}.

The notion we use in this paper is derived from weak barbed
bisimulation \cite{milner91polyadicpi}. 

\begin{definition}
An \emph{observation relation}, $\downarrow_{\mathcal N}$, over a set
of names, $\mathcal N$, is the smallest relation satisfying the rules
below.

\infrule[Out-barb]{y \in {\mathcal N}, \; x \nameeq y}
		  {\outputp{x}{v} \downarrow_{\mathcal N} x}
\infrule[Par-barb]{\mbox{$P\downarrow_{\mathcal N} x$ or $Q\downarrow_{\mathcal N} x$}}
		  {\binpar{P}{Q} \downarrow_{\mathcal N} x}

We write $P \Downarrow_{\mathcal N} x$ if there is $Q$ such that 
$P \wred Q$ and $Q \downarrow_{\mathcal N} x$.
\end{definition}

\begin{definition}
%\label{def.bbisim}
An  ${\mathcal N}$-\emph{barbed bisimulation} over a set of names, ${\mathcal N}$, is a symmetric binary relation 
${\mathcal S}_{\mathcal N}$ between agents such that $P\rel{S}_{\mathcal N}Q$ implies:
\begin{enumerate}
\item If $P \red P'$ then $Q \wred Q'$ and $P'\rel{S}_{\mathcal N} Q'$.
\item If $P\downarrow_{\mathcal N} x$, then $Q\Downarrow_{\mathcal N} x$.
\end{enumerate}
$P$ is ${\mathcal N}$-barbed bisimilar to $Q$, written
$P \wbbisim_{\mathcal N} Q$, if $P \rel{S}_{\mathcal N} Q$ for some ${\mathcal N}$-barbed bisimulation ${\mathcal S}_{\mathcal N}$.
\end{definition}

$\mathcal{R} \subseteq \pi \times \pi$

$P \mathcal{R} Q => \forall P'. P \red P' \Rightarrow \exists Q'. Q \red Q', P' \mathcal{R} Q'$

$P \vdash x \Rightarrow Q \vdash x$

\begin{mathpar}
  \inferrule*[lab=Out-barb]{x \nameeq y}{{y}!\langle{Q}\rangle \vdash x}
  \and
  \inferrule*[lab=Par-barb]{\mbox{$P\vdash x$ or $Q\vdash x$}}{\binpar{P}{Q} \vdash x}
\end{mathpar}

\subsubsection{Contexts}

One of the principle advantages of computational calculi like the
$\pi$-calculus is a well-defined notion of context,
contextual-equivalence and a correlation between
contextual-equivalence and notions of bisimulation. The notion of
context allows the decomposition of a process into (sub-)process and
its syntactic environment, its context. Thus, a context may be
thought of as a process with a ``hole'' (written $\Box$) in it. The
application of a context $M$ to a process $P$, written $M[P]$, is
tantamount to filling the hole in $M$ with $P$. In this paper we do
not need the full weight of this theory, but do make use of the notion
of context in the proof the main theorem. 

\begin{mathpar}
  \inferrule* [lab=summation] {} {{M_{M},M_{N}} \bc \Box \;|\; x.M_{A} \;|\; M_{M}+M_{N}}
  \and
  \inferrule* [lab=agent] {} {{M_{A}} \bc (\vec{x})M_{P} \;| \; \clift{P_0,\ldots,M_{P},\ldots,P_N}}
  \and \\
  \inferrule* [lab=process] {} {{M_{P}} \bc M_{N} \;| \;P|M_{P} }
\end{mathpar} 

\begin{mathpar}
  \inferrule* [lab=sychronization] {} {M_{N} \bc \Box \;|\; x?M_{F} \;|\; x!M_{C}}
  \and
  \inferrule* [lab=abstraction] {} {{M_{F}} \bc (x)M_{P} }
  \and
  \inferrule* [lab=concretion] {} {{M_{C}} \bc \langle M_{P} \rangle }
  \and \\
  \inferrule* [lab=process] {} {{M_{P}} \bc M_{N} \;| \;P|M_{P} }
\end{mathpar}

\begin{definition}[contextual application] Given a context $M$, and
  process $P$, we define the \emph{contextual application}, $M[P] :=
  M\{P/\Box\}$. That is, the contextual application of M to P is the
  substitution of $P$ for $\Box$ in $M$.
\end{definition}

$\meaningof{-} : L \to \mathcal{P}(\pi)$

\begin{mathpar}
  \inferrule* [lab=collection] {} {\meaningof{true} = \pi, \and \meaningof{~E} = \pi \setminus \meaningof{E}, \and \meaningof{E_{1} \& E_{2}} = \meaningof{E_{1}} \cap \meaningof{E_{2}}}
\end{mathpar}

\begin{mathpar}
  \inferrule* [lab=structure] {} {\meaningof{0} = \{ P \in \pi | P \equiv 0 \}, \and \\ \meaningof{E_1 | E_2} = \{ P \in \pi | P \equiv P_{1} | P_{2}, P_{1} \in \meaningof{E_{1}}, P_{2} \in \meaningof{E_2}\} }
\end{mathpar}

\begin{mathpar}
 \inferrule* [lab=behavior] {} {\meaningof{\langle a?b \rangle E} = \{ P \in \pi | P \equiv Q | u?(y)P', \\ \and \\\\ \and \\ \;\;\; u \in \meaningof{a}, \forall z.P'\{z/y\} \in \meaningof{E\{z/b\}}\}, \and \\ \meaningof{a!E} = \{ P \in \pi | P \equiv Q | x!\langle P' \rangle, x \in \meaningof{a} P' \in \meaningof{E}\} }
\end{mathpar}

\begin{mathpar}
 \inferrule* [lab=nominal] {} {\meaningof{\quotep{E}} = \{ \quotep{P} \in \quotep{\pi} | P \in \meaningof{E} \}, \and \meaningof{\quotep{P}} = \{ \quotep{Q} \in \quotep{\pi} | P \equiv Q \} \and \\ \meaningof{@\quotep{E}} = \{ P \in \pi | P \equiv @x, x \in \meaningof{E} \}}
\end{mathpar}

\begin{eqnarray*}
  \\
  \meaningof{-} : TS \to ST
\end{eqnarray*}

\begin{eqnarray*}
  \\
  L : TS \to ST
\end{eqnarray*}

\begin{eqnarray*}
  \\
  P \models E \iff P \in \meaningof{E}
\end{eqnarray*}

\begin{eqnarray*}
  P \approx_{L} Q \iff \forall E \in L. P \models E \iff Q \models E
\end{eqnarray*}

\begin{eqnarray*}
  P \approx_{K} Q
\end{eqnarray*}

\begin{eqnarray*}
  P \approx Q
\end{eqnarray*}

$\approx_{K} = \approx = \approx_{L}$

\subsubsection{Contextual duality}

Note that contexts extend the quotation operation to a family of
operations from processes to names. Given a context, $M$, we can
define a \emph{nominal context}, $\quotep{M}$ by $\quotep{M}[P] :=
\quotep{M[P]}$. To foreshadow what is to come we observe that these
operations enjoy a duality with processes very much like the duality
between vectors and maps from vectors to scalars.

Further, because the calculus is essentially higher-order, we have a
correspondence between contexts and processes. More specifically,
given a name $x$ and a context $M$ we can construct $M^{*}_{x}$ such
that 

\begin{mathpar}
  M^{*}_{x} | \lift{x}{P} \red M[P]
\end{mathpar}

namely,

\begin{mathpar}
  M^{*}_{x} := x?(u).M[\dropn{u}]
\end{mathpar}

The dependence of $M^{*}_{x}$ on a name makes it an abstraction, 

\begin{mathpar}
  M^{*} := (x)x?(u).M[\dropn{u}]
\end{mathpar}

\subsection{Additional notation}

It will sometimes be convenient to denote the process a name
quotes. We already have the notation $x = \quotep{P}$, but it will be
convenient to introduce an alternate notation, $\procn{x}$, when we
want to emphasize the connection to the use of the name. Note that, by
virtue of name equivalence, $\quotep{\procn{x}} \nameeq x$; so, the
notation is consistent with previous definitions.

Further, because names have structure it is possible to effect
substitutions on the basis of that structure. This means we need to
upgrade our notation for substitutions, which we accomplish by
adapting comprehension notation. Thus,

\begin{mathpar}
  P\{ y / x : x \in S \}
\end{mathpar}

is interpreted to mean the process derived from P by replacing (in a
capture-avoiding manner) each occurrence of $x$ in $S$ by $y$. For example,

\begin{mathpar}
  P\{ \quotep{\procn{x}|\procn{x}} / x : x \in \freenames{P} \}
\end{mathpar}

will replace each (occurrence) of a free name $x$ in $P$ by
$\quotep{\procn{x}|\procn{x}}$.

Also, we will avail ourselves of the notation $x^{L}$ and $x^{R}$ to
denote injections of a name into disjoint copies of the name
space. There are numerous ways to accomplish this. One example can be
found in \cite{MeredithR05}. This notation overloads to vectors of
names: $\vec{x}^{\pi} := (x_{i}^{\pi} \; : \; 0 \leq i < |\vec{x}| )$ where $\pi \in \{L,R\}$.

We also use $P^{\Box} := P|\Box$.

In \cite{MeredithR05} an interpretation of the new operator is
given. It turns out that there are several possible interpretations
all enjoying the requisite algebraic properties of the operator (see
\cite{milner91polyadicpi}). We will therefore make liberal use of
$(\nu\; \vec{x})P$.

% subsection the_syntax_and_semantics_of_the_notation_system (end)   

\input{qm2pi.qmops} 

\input{qm2pi.sterngerlach} 

\input{qm2pi.metric} 

% section concurrent_process_calculi (end)

%\input{qm2pi.proofsketch}

% section proof sketch (end)

%\input{qm2pi.slviaknots} 

% section spatial logic via knots (end)

\input{qm2pi.conclusion}

% section conclusion (end)

%\input{qm2pi.dtcodes} 

% section wiring algorithm (end)

\input{qm2pi.ack} 

% section acknowledgments (end)

\newpage


\bibliographystyle{plain}   
\bibliography{../../biblios/main.bib}

\input{qm2pi.rhodetails}

\end{document}



% section front matter (end)

\section{Introduction}\label{sec:introduction} % (fold)
In this draft of the material i am going to have to dispense with the
usual writing conventions adopted in papers on these topics. i'm going
to have adopt whatever tone i need at the time i'm writing up the
calculations. Sometimes this may be very conversational; others it may
be the barest mathematical grunts; others still it may be that i have
lifted text from one of my other papers because the exposition of some
point was better said there. i hope that my readers are not unduly put
out by this decision. i'm not doing this to flout convention or be
rebellious. i find these calculations very technically challenging. To
keep everything going technically, something has to give; i have to
let go of some cognitive burden. So, the academic writing style --
with all of its trade-offs in terms of facilitating technical
communication -- is what i'm letting go of. Perhaps subsequent drafts
can be tightened and polished, but for now, i'm going to speak as if
we were sitting together in a coffee shop with a laptop, wifi and a
pad of paper and a pencil.

So, here's what i have to say. We -- you and i, comfortably ensconced
in our coffee shop and well-equipped with our tools -- can realize and
carry out the calculations of quantum mechanics over a very different
formal theory of dynamics, a formal theory of dynamics that
corresponds to a theory of concurrent computation with
\emph{reflection}. It has the advantage that the underlying theory is
already `quantized', but supports analogues all of the continuuous
operations. Strikingly, this underlying theory has recently been
connected with a notion of metric that we can show, by calculating
together, coincides with the metric induced by the inner product.

There are a lot of reasons why you might be interested in seeing
calculations of this form. Here's why i'm interested. For the past
several centuries there has been no competitor to the ``Newtonian''
account of dynamics. As a result the predominant share of accounts of
dynamical systems and situations have had to be formulated in terms of
the Newtonian machinery. i view this as an intellectually dangerous
position to occupy. Everything, despite it's intrinsic shape, turns
into a nail to be hit with this hammer. Recently, however, the theory
of computation has matured to the point where we have candidates for
theories of dynamics that offer very different perspective on
reasoning about dynamical systems and situations. Testing these
candidates against very successful accounts of dynamical situations,
like quantum mechanics, is going to give us some sense of how mature
they are and some measure of the quality of these accounts of
dynamics.

\subsection{Summary of contributions and outline of paper}

So, we're going to develop an interpretation of the operations of
quantum mechanics normally interpreted by Hilbert spaces and
operators. We're going to do this over a theory of computation. Note
that this is very different than the usual quantum computation program
which develops notions of computation over quantum mechanics. Rather,
we are developing a story that aligns with Wheeler's slogan: It from
Bit. To do this we will first provide an account of the theory of
computation at play here. Then we will dive into a calculation-driven
interpretation of the operations of quantum mechanics.

The reason we take this approach is that -- until very recently --
there hasn't been an axiomatic account of quantum mechanics. As a
result there has been no sharp delineation of the mathematical theory
supporting interpretation of the physical theory and the physical
theory, itself. So, ambient features of the maths are free to be
exploited (or supressed) without a real accounting of their physical
relevance. There is no sharp statement ``here's the physical theory''
qua \emph{theory} and ``here's the mathematical interpretation''
enabling a judgment of how faithful the interpretation is -- apart
from experimental observation. When there is an axiomatic account we
can judge how well a given mathematical formalism supports an
interpretation of the axioms, independent of
experimentation. Likewise, we can judge how well we have captured our
physical evidence and experience with our axiomatics, independent of
any specific mathematical implementation, with accidental detail that
may or may not have physical significance. 

In lieu of a fully fleshed out and vetted axiomatic account of quantum
mechanics, interpreting the operational notions in service of modeling
physical systems will have to suffice. In other words, we are not in
the business of providing a model of Hilbert spaces and operators. We
are in the business of providing a model of quantum mechanics because
we are motivated by testing our notions of dynamics against physical
theory; and, the predictive calculations of the physical theory must
serve as the best formulation -- shy of a fully fleshed out axiomatic
account -- of the physical theory itself (as they have for scientific
theories since time immemorial). Put another way, despite a
whole-hearted commitment to an It-from-Bit ontology, we are firmly
aligned with the shut-up-and-calculate camp as the best way to obtain
results either from the physical perspective or as a quality assurance
measure of our fledgling theory of dynamics.

In detail, we present a reflective process calculus. Then we develop
intuitive correspondences between the notions available in this
calculus and the usual physical notions supporting quantum mechanical
calculations. Thus, 

\begin{table}[htp]
  \center{
    \fbox{
      \begin{tabular}{c|c}
        quantum mechanics & process calculus \\
        \hline
        scalar & name \\
        state vector & process \\
        dual & contextual duals \\
        matrix & formal sums of process-context-dual pairs \\
        orthogonality & process annihilation \\
        inner product & execution-formula + quoting
      \end{tabular}
    }
  }
  \caption{QM - process calculi correspondences}
\end{table}

Then we tighten up these intuitions to operational definitions. We
employ the Dirac notation as the best proxy we can find for an
abstract syntax of the quantum mechanical notions. The definitions we
develop put us in contact with equational constraints coming from the
theory that we demonstrate the definitions and calculations satisfy.

This puts us in a position to shut up and calculate for the
Stern-Gerlach experimental set up, showing how these predictive
calculations become calculations on processes in our theory of a
reflective process calculus.

Penultimately, we demonstrate that the notion of metric coming from
the inner product coincides with the notion of metric available from
the theory of bisimulation. This demonstration gives us the right to
think of space as arising from behavior. Finally, we consider where we
might go from the new vantage point we have obtained.

% section introduction (end) 
 
% section introduction (end)

% \documentclass[12pt]{llncs}
%\documentclass{jktr}

\usepackage[pdftex]{hyperref}                   
\usepackage {listings}
\usepackage {mathpartir}
\usepackage{bcprules}
%\usepackage{listings}
                       
\usepackage{graphicx} 
%\usepackage[margins=2.5cm,nohead,nofoot]{geometry}
%\usepackage{geometry}
\usepackage{amsfonts}
\usepackage{amstext}
\usepackage{latexsym}
\usepackage{amssymb}
\usepackage{color}


%\include{myPreamble}
\include{qm2pi.local} 

%\ifpdf
%\usepackage[pdftex]{graphicx}
%\else
%\usepackage{graphicx}
%\fi

 % \ifpdf
%  \usepackage{pdfsync}
%  \if


%\title{Brief Article}
%\author{David F. Snyder}
%\author{L.G. Meredith}

%\address{Dept. of Math., Texas State University--San Marcos, San Marcos, TX 78666}
       
\pagestyle{empty}


\begin{document}

\lstset{language=[Objective]Caml,frame=shadowbox}

\input{qm2pi.front}

% section front matter (end)

\input{qm2pi.intro} 
 
% section introduction (end)

% \input{qm2pi.knotations} 

% section notation (end)

\input{qm2pi.process.calculi} 

% section concurrent_process_calculi_and_spatial_logics_ (end)
    
%\input{qm2pi.knots2pi} 

%\input{qm2pi.trefoil} 

%\input{qm2pi.mainthm} 

% subsection basic_interpretation (end)

%\input{qm2pi.rho.presentation} 
\subsection{The syntax and semantics of the notation system}\label{sub:the_syntax_and_semantics_of_the_notation_system} % (fold)

We now summarize a technical presentation of the calculus that
embodies our theory of dynamics. The typical presentation of such a
calculus follows the style of giving generators and relations on
them. The grammar, below, describing term constructors, freely
generates the set of processes, $\Proc$. This set is then quotiented
by a relation known as structural congruence and it is over this set
that the notion of dynamics is expressed. This presentation is
essentially that of \cite{MeredithR05} with the addition of
polyadicity and summation. For readability we have relegated some of
the technical subtleties to an appendix.

\subsubsection{Process grammar}\label{subsub:process_grammar}

\begin{mathpar}
  \inferrule* [lab=synchronization] {} {{M} \bc \pzero \;|\; x?F \;|\; x!C }
  \and
  \inferrule* [lab=abstraction] {} {{F} \bc (x)P}
  \and
  \inferrule* [lab=concretion] {} {{C} \bc \langle Q \rangle}
  \and
  \inferrule* [lab=process] {} {{P,Q} \bc M \;| \;P|Q \;|\; @{x}}
  \and
  \inferrule* [lab=name] {} {{x} \bc \quotep{P}}
\end{mathpar} 

Note that $\vec{x}$ (resp. $\vec{P}$) denotes a vector of names
(resp. processes) of length $|\vec{x}|$ (resp. $|\vec{P}|$). We adopt
the following useful abbreviations.

\begin{mathpar}
   x?(\vec{y}).P := x.(\vec{y})P \and  x\clift{\vec{P}} := x.\clift{\vec{P}}
   \and x!(y) := \lift{x}{\dropn{y}}
   \and \Pi_{i=0}^{n-1}P_i := P_0 | \ldots | P_{n-1}
\end{mathpar}

\subsubsection{Structural congruence}

\paragraph{Free and bound names and alpha-equivalence.} At the
core of structural equivalence is alpha-equivalence which identifies
process that are the same up to a change of variable. Formally, we
recognize the distinction between free and bound names. The free names
of a process, $\freenames{P}$, may be calculated recursively as
follows:

\begin{mathpar}
\freenames{\pzero} := \emptyset
  \and \\
  \freenames{x?(y).P} := \{ x \} \cup (\freenames{P} \setminus \{ y \})
  \and 
  \freenames{x!\langle P \rangle} := \{ x \} \cup \{ P \} 
  \and \\
  \freenames{P|Q} := \freenames{P} \cup \freenames{Q}
  \and \\
  \freenames{@{x}} := \{ x \}
\end{mathpar}

$\pi$
$\quotep{\pi}$

$\freenames{-} : \pi \to \mathcal{P}(\quotep{\pi})$

\begin{eqnarray*}
  \freenames{\pzero} & := & \emptyset \\
  \freenames{x?(y).P} & := & \{ x \} \cup (\freenames{P} \setminus \{ y \}) \\
  \freenames{x!\langle P \rangle} & := & \{ x \} \cup \{ P \} \\
  \freenames{P|Q} & := & \freenames{P} \cup \freenames{Q} \\
  \freenames{\dropn{x}} & := & \{ x \}
\end{eqnarray*}

The bound names of a process, $\boundnames{P}$, are those names occurring in $P$
that are not free. For example, in $x?(y).0$, the name $x$ is free, while $y$ is bound.

\begin{mathpar}
  \inferrule* [lab=monoidal-laws] {} { P|Q \equiv Q|P \and P|0 \equiv P \and P|(Q|R) \equiv (P|Q)|R }
\end{mathpar}

\begin{mathpar}
  \inferrule* [lab=alpha-equivalence] {} { (x)P \equiv (y)P\{y/x\} \and y \not\in \freenames{P} }
\end{mathpar}

\begin{definition}
Then two processes, $P,Q$, are alpha-equivalent if $P = Q\{\vec{y}/\vec{x}\}$ for
some $\vec{x} \in \boundnames{Q},\vec{y} \in \boundnames{P}$, where $Q\{\vec{y}/\vec{x}\}$
denotes the capture-avoiding substitution of $\vec{y}$ for $\vec{x}$ in $Q$.
\end{definition}

\begin{definition}
  The {\em structural congruence} \cite{SangiorgiWalker} , $\equiv$,
  between processes is the least congruence containing
  alpha-equivalence, satisfying the abelian monoid laws
  (associativity, commutativity and $\pzero$ as identity) for parallel
  composition $|$ and for summation $+$.
\end{definition}

\subsection{Name equivalence}

We take name equivalence, written $\nameeq$, to be the smallest
equivalence relation generated by the following rules.

\begin{mathpar}
\inferrule*[lab=Quote-drop]
{ }
{ \quotep{@{x}} \nameeq x }

\inferrule*[lab=Struct-equiv]
{ P \scong Q }
{ \quotep{P} \nameeq \quotep{Q} }
\end{mathpar}

The astute reader will have noticed that the mutual recursion of names
and processes imposes a mutual recursion on alpha-equivalence and
structural equivalence via name-equivalence. Fortunately, all of this
works out pleasantly and we may calculate in the natural way, free of
concern. The reader interested in the details is referred to the
appendix \ref{appendix:rho_details}.

\subsection{Substitution}

We use $\Proc$ for the set of processes, $\QProc$ for the set of
names, and $\id{\{}\vec{y} / \vec{x} \id{\}}$ to denote partial maps,
$s : \QProc \rightarrow \QProc$. A map, $s$ lifts, uniquely, to a map
on process terms, $\widehat{s} : \Proc \rightarrow \Proc$ by the
following equations.

\begin{mathpar}
  (0) \psubstp{Q}{P} := 0 \\
  (R \juxtap S) \psubstp{Q}{P}
  :=    
  (R)\psubstp{Q}{P} \juxtap (S) \psubstp{Q}{P} \\
  (x?(y).R) \psubstp{Q}{P}    
  :=    
  (x)\substp{Q}{P} (z)\concat( (R \psubstn{z}{y}) \psubstp{Q}{P} ) \\
  (\lift{x}{R}) \psubstp{Q}{P}  
  :=
  \lift{(x)\substp{Q}{P}}{ R \psubstp{Q}{P} } \\
%   (\dropn{x})  \psubstp{Q}{P}       
%   := 
%   \left\{ 
%     \begin{array}{ccc} 
%       \dropn{\quotep{Q}} & & x \nameeq \quotep{P} \\
%       \dropn{x} & & otherwise \\
%     \end{array}
%   \right. 
  (\dropn{x})  \psubstp{Q}{P}       
  := 
  \left\{ 
    \begin{array}{ccc} 
      Q & & x \nameeq \quotep{P} \\
      \dropn{x} & & otherwise \\
    \end{array}
  \right.
\end{mathpar}
 

where

\begin{eqnarray}
  (x)\id{\{} \lpquote Q \rpquote / \lpquote P \rpquote \id{\}}            = 
  \left\{ 
    \begin{array}{ccc}
      \lpquote Q \rpquote & & x \nameeq \lpquote P \rpquote \\
      x & & otherwise \\
    \end{array}
  \right. \nonumber
\end{eqnarray}

and $z$ is chosen distinct from $\quotep{P}$, $\quotep{Q}$, the free
names in $Q$, and all the names in $R$. Our $\alpha$-equivalence will
be built in the standard way from this substitution.

\begin{remark}\label{rem:no_self_referential_names}
  One consequence of these definitions is that $\forall P. \quotep{P}
  \not\in \freenames{P}$.
\end{remark}

\subsection{ Dynamic quote: an example }

Anticipating something of what's to come, consider applying the
substitution, $\widehat{\id{\{}u / z \id{\}}}$, to the following pair
of processes, $\lift{w}{y!(z)}$ and $w[ \lpquote y!(z) \rpquote ]$.

\begin{eqnarray}
	\lift{w}{y!(z)}\widehat{\id{\{}u / z \id{\}}}
		& = &
		\lift{w}{y!(u)} \nonumber\\
	w[ \lpquote y!(z) \rpquote ] \widehat{ \id{\{}u / z \id{\}} }
		& = &
		w[ \lpquote y!(z) \rpquote ] \nonumber
\end{eqnarray}

Because the body of the process between quotes is impervious to
substitution, we get radically different answers. In fact, by
examining the first process in an input context,
e.g. $x?(z).\lift{w}{y!(z)}$, we see that the process under the lift
operator may be shaped by prefixed inputs binding a name inside it. In
this sense, the lift operator will be seen as a way to dynamically
construct processes before reifying them as names.

Finally equipped with these standard features we can present the
dynamics of the calculus.

\subsubsection{Operational semantics} 

Finally, we introduce the computational dynamics. What marks these
algebras as distinct from other more traditionally studied algebraic
structures, e.g. vector spaces or polynomial rings, is the manner in
which dynamics is captured. In traditional structures, dynamics is typically
expressed through morphisms between such structures, as in linear maps
between vector spaces or morphisms between rings. In algebras
associated with the semantics of computation, the dynamics is
expressed as part of the algebraic structure itself, through a
reduction reduction relation typically denoted by $\red$. Below, we
give a recursive presentation of this relation for the calculus used
in the encoding.

$\red \subseteq \pi \times \pi$
$\red : \pi \to \mathcal{P}(\pi)$

\begin{mathpar}
  \inferrule* [lab=Comm] { \textsf{match}( x_{src}, x_{trgt} ) } { x_{trgt}?(y)P \; | \; x_{src}!\langle {Q} \rangle \red P\{\quotep{Q}/y}\} }
  \and \\
  \inferrule* [lab=Par] {{P} \red {P}'} {{{P} | {Q}} \red {{P}' | {Q}}}
  \and
  \inferrule* [lab=Equiv]{{{P} \scong {P}'} \andalso {{P}' \red {Q}'} \andalso {{Q}' \scong {Q}}}{{P} \red {Q}}
\end{mathpar}

\begin{eqnarray*}
  match_{\equiv} (\quotep{P},\quotep{Q}) & := & P \equiv Q \\
  match_{\dagger}(\quotep{P},\quotep{Q}) & := & \forall R. P|Q \red^{*} R => R \red^{*} 0 \\
  match_{K}(\quotep{P},\quotep{Q}) & := & K \mbox{ for some context } K
\end{eqnarray*}

$u?(x)P | u!\langle Q \rangle \red P\{\quotep{Q}/x\}$

%We write $\wred$ for $\red^*$, and $P\red$ if $\exists Q $ such that $ P \red Q$.
We write $P\red$ if $\exists Q $ such that $ P \red Q$ and $P\not\red$, otherwise.

\section{Replication}

As mentioned before, it is known that replication (and hence
recursion) can be implemented in a higher-order process algebra
\cite{SangiorgiWalker}. As our first example of calculation with the
machinery thus far presented we give the construction explicitly in
the {\rhoc}.

\begin{eqnarray}
	D_{x} & := & \prefix{x}{y}{(\binpar{\outputp{x}{y}}{@{y}})} \nonumber\\
	\bangp_{x}{P} & := & \binpar{{x}!\langle{\binpar{D_{x}}{P}}\rangle}{D_{x}} \nonumber
\end{eqnarray}

\begin{eqnarray}
	\bangp_{x}{P} & & \nonumber\\
	=
	& {x}!\langle{(\prefix{x}{y}{(\outputp{x}{y} | @{y})) | P}}\rangle 
	      | \prefix{x}{y}{(\outputp{x}{y} | @{y})} & \nonumber\\
	\red
	& (\outputp{x}{y} | @{y})\substn{\quotep{(\prefix{x}{y}{(@{y} | \outputp{x}{y})) | P}}}{y} & \nonumber\\
	=
	& \outputp{x}{\quotep{(\prefix{x}{y}{(\outputp{x}{y} | @{y})) | P}}}
	  | {(\prefix{x}{y}{(\outputp{x}{y} | @{y})) | P}} & \nonumber\\
	\red
	& \ldots & \nonumber\\
	\red^*
	& P | P | \ldots & \nonumber
\end{eqnarray}

Of course, this encoding, as an implementation, runs away, unfolding
$\bangp{P}$ eagerly. A lazier and more implementable replication
operator, restricted to input-guarded processes, may be obtained as follows.

\begin{eqnarray}
\bangp{\prefix{u}{v}{P}} 
	:= 
	\binpar{\lift{x}{\prefix{u}{v}{(\binpar{D(x)}{P})}}}{D(x)} \nonumber
\end{eqnarray}

\begin{remark}
  Note that the lazier definition still does not deal with summation
  or mixed summation (i.e. sums over input and output). The reader is
  invited to construct definitions of replication that deal with these
  features. 

  Further, the definitions are parameterized in a name, $x$. Can you,
  gentle reader, make a definition that eliminates this parameter and
  guarantees no accidental interaction between the replication
  machinery and the process being replicated -- i.e. no accidental
  sharing of names used by the process to get its work done and the
  name(s) used by the replication to effect copying. This latter
  revision of the definition of replication is crucial to obtaining
  the expected identity $!!P \sim !P$.
\end{remark}

\begin{remark}\label{rem:paradoxical_combinator}
  The reader familiar with the lambda calculus will have noticed the
  similarity between $D$ and the paradoxical combinator.

  [Ed. note: the existence of this seems to suggest we have to be more
  restrictive on the set of processes and names we admit if we are to
  support no-cloning.]
\end{remark}

\subsubsection{Bisimulation}

The computational dynamics gives rise to another kind of equivalence,
the equivalence of computational behavior. As previously mentioned
this is typically captured \emph{via} some form of bisimulation.

% The notion we use in this paper is weak barbed bisimulation
% \cite{milner91polyadicpi}.

The notion we use in this paper is derived from weak barbed
bisimulation \cite{milner91polyadicpi}. 

\begin{definition}
An \emph{observation relation}, $\downarrow_{\mathcal N}$, over a set
of names, $\mathcal N$, is the smallest relation satisfying the rules
below.

\infrule[Out-barb]{y \in {\mathcal N}, \; x \nameeq y}
		  {\outputp{x}{v} \downarrow_{\mathcal N} x}
\infrule[Par-barb]{\mbox{$P\downarrow_{\mathcal N} x$ or $Q\downarrow_{\mathcal N} x$}}
		  {\binpar{P}{Q} \downarrow_{\mathcal N} x}

We write $P \Downarrow_{\mathcal N} x$ if there is $Q$ such that 
$P \wred Q$ and $Q \downarrow_{\mathcal N} x$.
\end{definition}

\begin{definition}
%\label{def.bbisim}
An  ${\mathcal N}$-\emph{barbed bisimulation} over a set of names, ${\mathcal N}$, is a symmetric binary relation 
${\mathcal S}_{\mathcal N}$ between agents such that $P\rel{S}_{\mathcal N}Q$ implies:
\begin{enumerate}
\item If $P \red P'$ then $Q \wred Q'$ and $P'\rel{S}_{\mathcal N} Q'$.
\item If $P\downarrow_{\mathcal N} x$, then $Q\Downarrow_{\mathcal N} x$.
\end{enumerate}
$P$ is ${\mathcal N}$-barbed bisimilar to $Q$, written
$P \wbbisim_{\mathcal N} Q$, if $P \rel{S}_{\mathcal N} Q$ for some ${\mathcal N}$-barbed bisimulation ${\mathcal S}_{\mathcal N}$.
\end{definition}

$\mathcal{R} \subseteq \pi \times \pi$

$P \mathcal{R} Q => \forall P'. P \red P' \Rightarrow \exists Q'. Q \red Q', P' \mathcal{R} Q'$

$P \vdash x \Rightarrow Q \vdash x$

\begin{mathpar}
  \inferrule*[lab=Out-barb]{x \nameeq y}{{y}!\langle{Q}\rangle \vdash x}
  \and
  \inferrule*[lab=Par-barb]{\mbox{$P\vdash x$ or $Q\vdash x$}}{\binpar{P}{Q} \vdash x}
\end{mathpar}

\subsubsection{Contexts}

One of the principle advantages of computational calculi like the
$\pi$-calculus is a well-defined notion of context,
contextual-equivalence and a correlation between
contextual-equivalence and notions of bisimulation. The notion of
context allows the decomposition of a process into (sub-)process and
its syntactic environment, its context. Thus, a context may be
thought of as a process with a ``hole'' (written $\Box$) in it. The
application of a context $M$ to a process $P$, written $M[P]$, is
tantamount to filling the hole in $M$ with $P$. In this paper we do
not need the full weight of this theory, but do make use of the notion
of context in the proof the main theorem. 

\begin{mathpar}
  \inferrule* [lab=summation] {} {{M_{M},M_{N}} \bc \Box \;|\; x.M_{A} \;|\; M_{M}+M_{N}}
  \and
  \inferrule* [lab=agent] {} {{M_{A}} \bc (\vec{x})M_{P} \;| \; \clift{P_0,\ldots,M_{P},\ldots,P_N}}
  \and \\
  \inferrule* [lab=process] {} {{M_{P}} \bc M_{N} \;| \;P|M_{P} }
\end{mathpar} 

\begin{mathpar}
  \inferrule* [lab=sychronization] {} {M_{N} \bc \Box \;|\; x?M_{F} \;|\; x!M_{C}}
  \and
  \inferrule* [lab=abstraction] {} {{M_{F}} \bc (x)M_{P} }
  \and
  \inferrule* [lab=concretion] {} {{M_{C}} \bc \langle M_{P} \rangle }
  \and \\
  \inferrule* [lab=process] {} {{M_{P}} \bc M_{N} \;| \;P|M_{P} }
\end{mathpar}

\begin{definition}[contextual application] Given a context $M$, and
  process $P$, we define the \emph{contextual application}, $M[P] :=
  M\{P/\Box\}$. That is, the contextual application of M to P is the
  substitution of $P$ for $\Box$ in $M$.
\end{definition}

$\meaningof{-} : L \to \mathcal{P}(\pi)$

\begin{mathpar}
  \inferrule* [lab=collection] {} {\meaningof{true} = \pi, \and \meaningof{~E} = \pi \setminus \meaningof{E}, \and \meaningof{E_{1} \& E_{2}} = \meaningof{E_{1}} \cap \meaningof{E_{2}}}
\end{mathpar}

\begin{mathpar}
  \inferrule* [lab=structure] {} {\meaningof{0} = \{ P \in \pi | P \equiv 0 \}, \and \\ \meaningof{E_1 | E_2} = \{ P \in \pi | P \equiv P_{1} | P_{2}, P_{1} \in \meaningof{E_{1}}, P_{2} \in \meaningof{E_2}\} }
\end{mathpar}

\begin{mathpar}
 \inferrule* [lab=behavior] {} {\meaningof{\langle a?b \rangle E} = \{ P \in \pi | P \equiv Q | u?(y)P', \\ \and \\\\ \and \\ \;\;\; u \in \meaningof{a}, \forall z.P'\{z/y\} \in \meaningof{E\{z/b\}}\}, \and \\ \meaningof{a!E} = \{ P \in \pi | P \equiv Q | x!\langle P' \rangle, x \in \meaningof{a} P' \in \meaningof{E}\} }
\end{mathpar}

\begin{mathpar}
 \inferrule* [lab=nominal] {} {\meaningof{\quotep{E}} = \{ \quotep{P} \in \quotep{\pi} | P \in \meaningof{E} \}, \and \meaningof{\quotep{P}} = \{ \quotep{Q} \in \quotep{\pi} | P \equiv Q \} \and \\ \meaningof{@\quotep{E}} = \{ P \in \pi | P \equiv @x, x \in \meaningof{E} \}}
\end{mathpar}

\begin{eqnarray*}
  \\
  \meaningof{-} : TS \to ST
\end{eqnarray*}

\begin{eqnarray*}
  \\
  L : TS \to ST
\end{eqnarray*}

\begin{eqnarray*}
  \\
  P \models E \iff P \in \meaningof{E}
\end{eqnarray*}

\begin{eqnarray*}
  P \approx_{L} Q \iff \forall E \in L. P \models E \iff Q \models E
\end{eqnarray*}

\begin{eqnarray*}
  P \approx_{K} Q
\end{eqnarray*}

\begin{eqnarray*}
  P \approx Q
\end{eqnarray*}

$\approx_{K} = \approx = \approx_{L}$

\subsubsection{Contextual duality}

Note that contexts extend the quotation operation to a family of
operations from processes to names. Given a context, $M$, we can
define a \emph{nominal context}, $\quotep{M}$ by $\quotep{M}[P] :=
\quotep{M[P]}$. To foreshadow what is to come we observe that these
operations enjoy a duality with processes very much like the duality
between vectors and maps from vectors to scalars.

Further, because the calculus is essentially higher-order, we have a
correspondence between contexts and processes. More specifically,
given a name $x$ and a context $M$ we can construct $M^{*}_{x}$ such
that 

\begin{mathpar}
  M^{*}_{x} | \lift{x}{P} \red M[P]
\end{mathpar}

namely,

\begin{mathpar}
  M^{*}_{x} := x?(u).M[\dropn{u}]
\end{mathpar}

The dependence of $M^{*}_{x}$ on a name makes it an abstraction, 

\begin{mathpar}
  M^{*} := (x)x?(u).M[\dropn{u}]
\end{mathpar}

\subsection{Additional notation}

It will sometimes be convenient to denote the process a name
quotes. We already have the notation $x = \quotep{P}$, but it will be
convenient to introduce an alternate notation, $\procn{x}$, when we
want to emphasize the connection to the use of the name. Note that, by
virtue of name equivalence, $\quotep{\procn{x}} \nameeq x$; so, the
notation is consistent with previous definitions.

Further, because names have structure it is possible to effect
substitutions on the basis of that structure. This means we need to
upgrade our notation for substitutions, which we accomplish by
adapting comprehension notation. Thus,

\begin{mathpar}
  P\{ y / x : x \in S \}
\end{mathpar}

is interpreted to mean the process derived from P by replacing (in a
capture-avoiding manner) each occurrence of $x$ in $S$ by $y$. For example,

\begin{mathpar}
  P\{ \quotep{\procn{x}|\procn{x}} / x : x \in \freenames{P} \}
\end{mathpar}

will replace each (occurrence) of a free name $x$ in $P$ by
$\quotep{\procn{x}|\procn{x}}$.

Also, we will avail ourselves of the notation $x^{L}$ and $x^{R}$ to
denote injections of a name into disjoint copies of the name
space. There are numerous ways to accomplish this. One example can be
found in \cite{MeredithR05}. This notation overloads to vectors of
names: $\vec{x}^{\pi} := (x_{i}^{\pi} \; : \; 0 \leq i < |\vec{x}| )$ where $\pi \in \{L,R\}$.

We also use $P^{\Box} := P|\Box$.

In \cite{MeredithR05} an interpretation of the new operator is
given. It turns out that there are several possible interpretations
all enjoying the requisite algebraic properties of the operator (see
\cite{milner91polyadicpi}). We will therefore make liberal use of
$(\nu\; \vec{x})P$.

% subsection the_syntax_and_semantics_of_the_notation_system (end)   

\input{qm2pi.qmops} 

\input{qm2pi.sterngerlach} 

\input{qm2pi.metric} 

% section concurrent_process_calculi (end)

%\input{qm2pi.proofsketch}

% section proof sketch (end)

%\input{qm2pi.slviaknots} 

% section spatial logic via knots (end)

\input{qm2pi.conclusion}

% section conclusion (end)

%\input{qm2pi.dtcodes} 

% section wiring algorithm (end)

\input{qm2pi.ack} 

% section acknowledgments (end)

\newpage


\bibliographystyle{plain}   
\bibliography{../../biblios/main.bib}

\input{qm2pi.rhodetails}

\end{document}

 

% section notation (end)

\input{qm2pi.process.calculi} 

% section concurrent_process_calculi_and_spatial_logics_ (end)
    
%\documentclass[12pt]{llncs}
%\documentclass{jktr}

\usepackage[pdftex]{hyperref}                   
\usepackage {listings}
\usepackage {mathpartir}
\usepackage{bcprules}
%\usepackage{listings}
                       
\usepackage{graphicx} 
%\usepackage[margins=2.5cm,nohead,nofoot]{geometry}
%\usepackage{geometry}
\usepackage{amsfonts}
\usepackage{amstext}
\usepackage{latexsym}
\usepackage{amssymb}
\usepackage{color}


%\include{myPreamble}
\include{qm2pi.local} 

%\ifpdf
%\usepackage[pdftex]{graphicx}
%\else
%\usepackage{graphicx}
%\fi

 % \ifpdf
%  \usepackage{pdfsync}
%  \if


%\title{Brief Article}
%\author{David F. Snyder}
%\author{L.G. Meredith}

%\address{Dept. of Math., Texas State University--San Marcos, San Marcos, TX 78666}
       
\pagestyle{empty}


\begin{document}

\lstset{language=[Objective]Caml,frame=shadowbox}

\input{qm2pi.front}

% section front matter (end)

\input{qm2pi.intro} 
 
% section introduction (end)

% \input{qm2pi.knotations} 

% section notation (end)

\input{qm2pi.process.calculi} 

% section concurrent_process_calculi_and_spatial_logics_ (end)
    
%\input{qm2pi.knots2pi} 

%\input{qm2pi.trefoil} 

%\input{qm2pi.mainthm} 

% subsection basic_interpretation (end)

%\input{qm2pi.rho.presentation} 
\subsection{The syntax and semantics of the notation system}\label{sub:the_syntax_and_semantics_of_the_notation_system} % (fold)

We now summarize a technical presentation of the calculus that
embodies our theory of dynamics. The typical presentation of such a
calculus follows the style of giving generators and relations on
them. The grammar, below, describing term constructors, freely
generates the set of processes, $\Proc$. This set is then quotiented
by a relation known as structural congruence and it is over this set
that the notion of dynamics is expressed. This presentation is
essentially that of \cite{MeredithR05} with the addition of
polyadicity and summation. For readability we have relegated some of
the technical subtleties to an appendix.

\subsubsection{Process grammar}\label{subsub:process_grammar}

\begin{mathpar}
  \inferrule* [lab=synchronization] {} {{M} \bc \pzero \;|\; x?F \;|\; x!C }
  \and
  \inferrule* [lab=abstraction] {} {{F} \bc (x)P}
  \and
  \inferrule* [lab=concretion] {} {{C} \bc \langle Q \rangle}
  \and
  \inferrule* [lab=process] {} {{P,Q} \bc M \;| \;P|Q \;|\; @{x}}
  \and
  \inferrule* [lab=name] {} {{x} \bc \quotep{P}}
\end{mathpar} 

Note that $\vec{x}$ (resp. $\vec{P}$) denotes a vector of names
(resp. processes) of length $|\vec{x}|$ (resp. $|\vec{P}|$). We adopt
the following useful abbreviations.

\begin{mathpar}
   x?(\vec{y}).P := x.(\vec{y})P \and  x\clift{\vec{P}} := x.\clift{\vec{P}}
   \and x!(y) := \lift{x}{\dropn{y}}
   \and \Pi_{i=0}^{n-1}P_i := P_0 | \ldots | P_{n-1}
\end{mathpar}

\subsubsection{Structural congruence}

\paragraph{Free and bound names and alpha-equivalence.} At the
core of structural equivalence is alpha-equivalence which identifies
process that are the same up to a change of variable. Formally, we
recognize the distinction between free and bound names. The free names
of a process, $\freenames{P}$, may be calculated recursively as
follows:

\begin{mathpar}
\freenames{\pzero} := \emptyset
  \and \\
  \freenames{x?(y).P} := \{ x \} \cup (\freenames{P} \setminus \{ y \})
  \and 
  \freenames{x!\langle P \rangle} := \{ x \} \cup \{ P \} 
  \and \\
  \freenames{P|Q} := \freenames{P} \cup \freenames{Q}
  \and \\
  \freenames{@{x}} := \{ x \}
\end{mathpar}

$\pi$
$\quotep{\pi}$

$\freenames{-} : \pi \to \mathcal{P}(\quotep{\pi})$

\begin{eqnarray*}
  \freenames{\pzero} & := & \emptyset \\
  \freenames{x?(y).P} & := & \{ x \} \cup (\freenames{P} \setminus \{ y \}) \\
  \freenames{x!\langle P \rangle} & := & \{ x \} \cup \{ P \} \\
  \freenames{P|Q} & := & \freenames{P} \cup \freenames{Q} \\
  \freenames{\dropn{x}} & := & \{ x \}
\end{eqnarray*}

The bound names of a process, $\boundnames{P}$, are those names occurring in $P$
that are not free. For example, in $x?(y).0$, the name $x$ is free, while $y$ is bound.

\begin{mathpar}
  \inferrule* [lab=monoidal-laws] {} { P|Q \equiv Q|P \and P|0 \equiv P \and P|(Q|R) \equiv (P|Q)|R }
\end{mathpar}

\begin{mathpar}
  \inferrule* [lab=alpha-equivalence] {} { (x)P \equiv (y)P\{y/x\} \and y \not\in \freenames{P} }
\end{mathpar}

\begin{definition}
Then two processes, $P,Q$, are alpha-equivalent if $P = Q\{\vec{y}/\vec{x}\}$ for
some $\vec{x} \in \boundnames{Q},\vec{y} \in \boundnames{P}$, where $Q\{\vec{y}/\vec{x}\}$
denotes the capture-avoiding substitution of $\vec{y}$ for $\vec{x}$ in $Q$.
\end{definition}

\begin{definition}
  The {\em structural congruence} \cite{SangiorgiWalker} , $\equiv$,
  between processes is the least congruence containing
  alpha-equivalence, satisfying the abelian monoid laws
  (associativity, commutativity and $\pzero$ as identity) for parallel
  composition $|$ and for summation $+$.
\end{definition}

\subsection{Name equivalence}

We take name equivalence, written $\nameeq$, to be the smallest
equivalence relation generated by the following rules.

\begin{mathpar}
\inferrule*[lab=Quote-drop]
{ }
{ \quotep{@{x}} \nameeq x }

\inferrule*[lab=Struct-equiv]
{ P \scong Q }
{ \quotep{P} \nameeq \quotep{Q} }
\end{mathpar}

The astute reader will have noticed that the mutual recursion of names
and processes imposes a mutual recursion on alpha-equivalence and
structural equivalence via name-equivalence. Fortunately, all of this
works out pleasantly and we may calculate in the natural way, free of
concern. The reader interested in the details is referred to the
appendix \ref{appendix:rho_details}.

\subsection{Substitution}

We use $\Proc$ for the set of processes, $\QProc$ for the set of
names, and $\id{\{}\vec{y} / \vec{x} \id{\}}$ to denote partial maps,
$s : \QProc \rightarrow \QProc$. A map, $s$ lifts, uniquely, to a map
on process terms, $\widehat{s} : \Proc \rightarrow \Proc$ by the
following equations.

\begin{mathpar}
  (0) \psubstp{Q}{P} := 0 \\
  (R \juxtap S) \psubstp{Q}{P}
  :=    
  (R)\psubstp{Q}{P} \juxtap (S) \psubstp{Q}{P} \\
  (x?(y).R) \psubstp{Q}{P}    
  :=    
  (x)\substp{Q}{P} (z)\concat( (R \psubstn{z}{y}) \psubstp{Q}{P} ) \\
  (\lift{x}{R}) \psubstp{Q}{P}  
  :=
  \lift{(x)\substp{Q}{P}}{ R \psubstp{Q}{P} } \\
%   (\dropn{x})  \psubstp{Q}{P}       
%   := 
%   \left\{ 
%     \begin{array}{ccc} 
%       \dropn{\quotep{Q}} & & x \nameeq \quotep{P} \\
%       \dropn{x} & & otherwise \\
%     \end{array}
%   \right. 
  (\dropn{x})  \psubstp{Q}{P}       
  := 
  \left\{ 
    \begin{array}{ccc} 
      Q & & x \nameeq \quotep{P} \\
      \dropn{x} & & otherwise \\
    \end{array}
  \right.
\end{mathpar}
 

where

\begin{eqnarray}
  (x)\id{\{} \lpquote Q \rpquote / \lpquote P \rpquote \id{\}}            = 
  \left\{ 
    \begin{array}{ccc}
      \lpquote Q \rpquote & & x \nameeq \lpquote P \rpquote \\
      x & & otherwise \\
    \end{array}
  \right. \nonumber
\end{eqnarray}

and $z$ is chosen distinct from $\quotep{P}$, $\quotep{Q}$, the free
names in $Q$, and all the names in $R$. Our $\alpha$-equivalence will
be built in the standard way from this substitution.

\begin{remark}\label{rem:no_self_referential_names}
  One consequence of these definitions is that $\forall P. \quotep{P}
  \not\in \freenames{P}$.
\end{remark}

\subsection{ Dynamic quote: an example }

Anticipating something of what's to come, consider applying the
substitution, $\widehat{\id{\{}u / z \id{\}}}$, to the following pair
of processes, $\lift{w}{y!(z)}$ and $w[ \lpquote y!(z) \rpquote ]$.

\begin{eqnarray}
	\lift{w}{y!(z)}\widehat{\id{\{}u / z \id{\}}}
		& = &
		\lift{w}{y!(u)} \nonumber\\
	w[ \lpquote y!(z) \rpquote ] \widehat{ \id{\{}u / z \id{\}} }
		& = &
		w[ \lpquote y!(z) \rpquote ] \nonumber
\end{eqnarray}

Because the body of the process between quotes is impervious to
substitution, we get radically different answers. In fact, by
examining the first process in an input context,
e.g. $x?(z).\lift{w}{y!(z)}$, we see that the process under the lift
operator may be shaped by prefixed inputs binding a name inside it. In
this sense, the lift operator will be seen as a way to dynamically
construct processes before reifying them as names.

Finally equipped with these standard features we can present the
dynamics of the calculus.

\subsubsection{Operational semantics} 

Finally, we introduce the computational dynamics. What marks these
algebras as distinct from other more traditionally studied algebraic
structures, e.g. vector spaces or polynomial rings, is the manner in
which dynamics is captured. In traditional structures, dynamics is typically
expressed through morphisms between such structures, as in linear maps
between vector spaces or morphisms between rings. In algebras
associated with the semantics of computation, the dynamics is
expressed as part of the algebraic structure itself, through a
reduction reduction relation typically denoted by $\red$. Below, we
give a recursive presentation of this relation for the calculus used
in the encoding.

$\red \subseteq \pi \times \pi$
$\red : \pi \to \mathcal{P}(\pi)$

\begin{mathpar}
  \inferrule* [lab=Comm] { \textsf{match}( x_{src}, x_{trgt} ) } { x_{trgt}?(y)P \; | \; x_{src}!\langle {Q} \rangle \red P\{\quotep{Q}/y}\} }
  \and \\
  \inferrule* [lab=Par] {{P} \red {P}'} {{{P} | {Q}} \red {{P}' | {Q}}}
  \and
  \inferrule* [lab=Equiv]{{{P} \scong {P}'} \andalso {{P}' \red {Q}'} \andalso {{Q}' \scong {Q}}}{{P} \red {Q}}
\end{mathpar}

\begin{eqnarray*}
  match_{\equiv} (\quotep{P},\quotep{Q}) & := & P \equiv Q \\
  match_{\dagger}(\quotep{P},\quotep{Q}) & := & \forall R. P|Q \red^{*} R => R \red^{*} 0 \\
  match_{K}(\quotep{P},\quotep{Q}) & := & K \mbox{ for some context } K
\end{eqnarray*}

$u?(x)P | u!\langle Q \rangle \red P\{\quotep{Q}/x\}$

%We write $\wred$ for $\red^*$, and $P\red$ if $\exists Q $ such that $ P \red Q$.
We write $P\red$ if $\exists Q $ such that $ P \red Q$ and $P\not\red$, otherwise.

\section{Replication}

As mentioned before, it is known that replication (and hence
recursion) can be implemented in a higher-order process algebra
\cite{SangiorgiWalker}. As our first example of calculation with the
machinery thus far presented we give the construction explicitly in
the {\rhoc}.

\begin{eqnarray}
	D_{x} & := & \prefix{x}{y}{(\binpar{\outputp{x}{y}}{@{y}})} \nonumber\\
	\bangp_{x}{P} & := & \binpar{{x}!\langle{\binpar{D_{x}}{P}}\rangle}{D_{x}} \nonumber
\end{eqnarray}

\begin{eqnarray}
	\bangp_{x}{P} & & \nonumber\\
	=
	& {x}!\langle{(\prefix{x}{y}{(\outputp{x}{y} | @{y})) | P}}\rangle 
	      | \prefix{x}{y}{(\outputp{x}{y} | @{y})} & \nonumber\\
	\red
	& (\outputp{x}{y} | @{y})\substn{\quotep{(\prefix{x}{y}{(@{y} | \outputp{x}{y})) | P}}}{y} & \nonumber\\
	=
	& \outputp{x}{\quotep{(\prefix{x}{y}{(\outputp{x}{y} | @{y})) | P}}}
	  | {(\prefix{x}{y}{(\outputp{x}{y} | @{y})) | P}} & \nonumber\\
	\red
	& \ldots & \nonumber\\
	\red^*
	& P | P | \ldots & \nonumber
\end{eqnarray}

Of course, this encoding, as an implementation, runs away, unfolding
$\bangp{P}$ eagerly. A lazier and more implementable replication
operator, restricted to input-guarded processes, may be obtained as follows.

\begin{eqnarray}
\bangp{\prefix{u}{v}{P}} 
	:= 
	\binpar{\lift{x}{\prefix{u}{v}{(\binpar{D(x)}{P})}}}{D(x)} \nonumber
\end{eqnarray}

\begin{remark}
  Note that the lazier definition still does not deal with summation
  or mixed summation (i.e. sums over input and output). The reader is
  invited to construct definitions of replication that deal with these
  features. 

  Further, the definitions are parameterized in a name, $x$. Can you,
  gentle reader, make a definition that eliminates this parameter and
  guarantees no accidental interaction between the replication
  machinery and the process being replicated -- i.e. no accidental
  sharing of names used by the process to get its work done and the
  name(s) used by the replication to effect copying. This latter
  revision of the definition of replication is crucial to obtaining
  the expected identity $!!P \sim !P$.
\end{remark}

\begin{remark}\label{rem:paradoxical_combinator}
  The reader familiar with the lambda calculus will have noticed the
  similarity between $D$ and the paradoxical combinator.

  [Ed. note: the existence of this seems to suggest we have to be more
  restrictive on the set of processes and names we admit if we are to
  support no-cloning.]
\end{remark}

\subsubsection{Bisimulation}

The computational dynamics gives rise to another kind of equivalence,
the equivalence of computational behavior. As previously mentioned
this is typically captured \emph{via} some form of bisimulation.

% The notion we use in this paper is weak barbed bisimulation
% \cite{milner91polyadicpi}.

The notion we use in this paper is derived from weak barbed
bisimulation \cite{milner91polyadicpi}. 

\begin{definition}
An \emph{observation relation}, $\downarrow_{\mathcal N}$, over a set
of names, $\mathcal N$, is the smallest relation satisfying the rules
below.

\infrule[Out-barb]{y \in {\mathcal N}, \; x \nameeq y}
		  {\outputp{x}{v} \downarrow_{\mathcal N} x}
\infrule[Par-barb]{\mbox{$P\downarrow_{\mathcal N} x$ or $Q\downarrow_{\mathcal N} x$}}
		  {\binpar{P}{Q} \downarrow_{\mathcal N} x}

We write $P \Downarrow_{\mathcal N} x$ if there is $Q$ such that 
$P \wred Q$ and $Q \downarrow_{\mathcal N} x$.
\end{definition}

\begin{definition}
%\label{def.bbisim}
An  ${\mathcal N}$-\emph{barbed bisimulation} over a set of names, ${\mathcal N}$, is a symmetric binary relation 
${\mathcal S}_{\mathcal N}$ between agents such that $P\rel{S}_{\mathcal N}Q$ implies:
\begin{enumerate}
\item If $P \red P'$ then $Q \wred Q'$ and $P'\rel{S}_{\mathcal N} Q'$.
\item If $P\downarrow_{\mathcal N} x$, then $Q\Downarrow_{\mathcal N} x$.
\end{enumerate}
$P$ is ${\mathcal N}$-barbed bisimilar to $Q$, written
$P \wbbisim_{\mathcal N} Q$, if $P \rel{S}_{\mathcal N} Q$ for some ${\mathcal N}$-barbed bisimulation ${\mathcal S}_{\mathcal N}$.
\end{definition}

$\mathcal{R} \subseteq \pi \times \pi$

$P \mathcal{R} Q => \forall P'. P \red P' \Rightarrow \exists Q'. Q \red Q', P' \mathcal{R} Q'$

$P \vdash x \Rightarrow Q \vdash x$

\begin{mathpar}
  \inferrule*[lab=Out-barb]{x \nameeq y}{{y}!\langle{Q}\rangle \vdash x}
  \and
  \inferrule*[lab=Par-barb]{\mbox{$P\vdash x$ or $Q\vdash x$}}{\binpar{P}{Q} \vdash x}
\end{mathpar}

\subsubsection{Contexts}

One of the principle advantages of computational calculi like the
$\pi$-calculus is a well-defined notion of context,
contextual-equivalence and a correlation between
contextual-equivalence and notions of bisimulation. The notion of
context allows the decomposition of a process into (sub-)process and
its syntactic environment, its context. Thus, a context may be
thought of as a process with a ``hole'' (written $\Box$) in it. The
application of a context $M$ to a process $P$, written $M[P]$, is
tantamount to filling the hole in $M$ with $P$. In this paper we do
not need the full weight of this theory, but do make use of the notion
of context in the proof the main theorem. 

\begin{mathpar}
  \inferrule* [lab=summation] {} {{M_{M},M_{N}} \bc \Box \;|\; x.M_{A} \;|\; M_{M}+M_{N}}
  \and
  \inferrule* [lab=agent] {} {{M_{A}} \bc (\vec{x})M_{P} \;| \; \clift{P_0,\ldots,M_{P},\ldots,P_N}}
  \and \\
  \inferrule* [lab=process] {} {{M_{P}} \bc M_{N} \;| \;P|M_{P} }
\end{mathpar} 

\begin{mathpar}
  \inferrule* [lab=sychronization] {} {M_{N} \bc \Box \;|\; x?M_{F} \;|\; x!M_{C}}
  \and
  \inferrule* [lab=abstraction] {} {{M_{F}} \bc (x)M_{P} }
  \and
  \inferrule* [lab=concretion] {} {{M_{C}} \bc \langle M_{P} \rangle }
  \and \\
  \inferrule* [lab=process] {} {{M_{P}} \bc M_{N} \;| \;P|M_{P} }
\end{mathpar}

\begin{definition}[contextual application] Given a context $M$, and
  process $P$, we define the \emph{contextual application}, $M[P] :=
  M\{P/\Box\}$. That is, the contextual application of M to P is the
  substitution of $P$ for $\Box$ in $M$.
\end{definition}

$\meaningof{-} : L \to \mathcal{P}(\pi)$

\begin{mathpar}
  \inferrule* [lab=collection] {} {\meaningof{true} = \pi, \and \meaningof{~E} = \pi \setminus \meaningof{E}, \and \meaningof{E_{1} \& E_{2}} = \meaningof{E_{1}} \cap \meaningof{E_{2}}}
\end{mathpar}

\begin{mathpar}
  \inferrule* [lab=structure] {} {\meaningof{0} = \{ P \in \pi | P \equiv 0 \}, \and \\ \meaningof{E_1 | E_2} = \{ P \in \pi | P \equiv P_{1} | P_{2}, P_{1} \in \meaningof{E_{1}}, P_{2} \in \meaningof{E_2}\} }
\end{mathpar}

\begin{mathpar}
 \inferrule* [lab=behavior] {} {\meaningof{\langle a?b \rangle E} = \{ P \in \pi | P \equiv Q | u?(y)P', \\ \and \\\\ \and \\ \;\;\; u \in \meaningof{a}, \forall z.P'\{z/y\} \in \meaningof{E\{z/b\}}\}, \and \\ \meaningof{a!E} = \{ P \in \pi | P \equiv Q | x!\langle P' \rangle, x \in \meaningof{a} P' \in \meaningof{E}\} }
\end{mathpar}

\begin{mathpar}
 \inferrule* [lab=nominal] {} {\meaningof{\quotep{E}} = \{ \quotep{P} \in \quotep{\pi} | P \in \meaningof{E} \}, \and \meaningof{\quotep{P}} = \{ \quotep{Q} \in \quotep{\pi} | P \equiv Q \} \and \\ \meaningof{@\quotep{E}} = \{ P \in \pi | P \equiv @x, x \in \meaningof{E} \}}
\end{mathpar}

\begin{eqnarray*}
  \\
  \meaningof{-} : TS \to ST
\end{eqnarray*}

\begin{eqnarray*}
  \\
  L : TS \to ST
\end{eqnarray*}

\begin{eqnarray*}
  \\
  P \models E \iff P \in \meaningof{E}
\end{eqnarray*}

\begin{eqnarray*}
  P \approx_{L} Q \iff \forall E \in L. P \models E \iff Q \models E
\end{eqnarray*}

\begin{eqnarray*}
  P \approx_{K} Q
\end{eqnarray*}

\begin{eqnarray*}
  P \approx Q
\end{eqnarray*}

$\approx_{K} = \approx = \approx_{L}$

\subsubsection{Contextual duality}

Note that contexts extend the quotation operation to a family of
operations from processes to names. Given a context, $M$, we can
define a \emph{nominal context}, $\quotep{M}$ by $\quotep{M}[P] :=
\quotep{M[P]}$. To foreshadow what is to come we observe that these
operations enjoy a duality with processes very much like the duality
between vectors and maps from vectors to scalars.

Further, because the calculus is essentially higher-order, we have a
correspondence between contexts and processes. More specifically,
given a name $x$ and a context $M$ we can construct $M^{*}_{x}$ such
that 

\begin{mathpar}
  M^{*}_{x} | \lift{x}{P} \red M[P]
\end{mathpar}

namely,

\begin{mathpar}
  M^{*}_{x} := x?(u).M[\dropn{u}]
\end{mathpar}

The dependence of $M^{*}_{x}$ on a name makes it an abstraction, 

\begin{mathpar}
  M^{*} := (x)x?(u).M[\dropn{u}]
\end{mathpar}

\subsection{Additional notation}

It will sometimes be convenient to denote the process a name
quotes. We already have the notation $x = \quotep{P}$, but it will be
convenient to introduce an alternate notation, $\procn{x}$, when we
want to emphasize the connection to the use of the name. Note that, by
virtue of name equivalence, $\quotep{\procn{x}} \nameeq x$; so, the
notation is consistent with previous definitions.

Further, because names have structure it is possible to effect
substitutions on the basis of that structure. This means we need to
upgrade our notation for substitutions, which we accomplish by
adapting comprehension notation. Thus,

\begin{mathpar}
  P\{ y / x : x \in S \}
\end{mathpar}

is interpreted to mean the process derived from P by replacing (in a
capture-avoiding manner) each occurrence of $x$ in $S$ by $y$. For example,

\begin{mathpar}
  P\{ \quotep{\procn{x}|\procn{x}} / x : x \in \freenames{P} \}
\end{mathpar}

will replace each (occurrence) of a free name $x$ in $P$ by
$\quotep{\procn{x}|\procn{x}}$.

Also, we will avail ourselves of the notation $x^{L}$ and $x^{R}$ to
denote injections of a name into disjoint copies of the name
space. There are numerous ways to accomplish this. One example can be
found in \cite{MeredithR05}. This notation overloads to vectors of
names: $\vec{x}^{\pi} := (x_{i}^{\pi} \; : \; 0 \leq i < |\vec{x}| )$ where $\pi \in \{L,R\}$.

We also use $P^{\Box} := P|\Box$.

In \cite{MeredithR05} an interpretation of the new operator is
given. It turns out that there are several possible interpretations
all enjoying the requisite algebraic properties of the operator (see
\cite{milner91polyadicpi}). We will therefore make liberal use of
$(\nu\; \vec{x})P$.

% subsection the_syntax_and_semantics_of_the_notation_system (end)   

\input{qm2pi.qmops} 

\input{qm2pi.sterngerlach} 

\input{qm2pi.metric} 

% section concurrent_process_calculi (end)

%\input{qm2pi.proofsketch}

% section proof sketch (end)

%\input{qm2pi.slviaknots} 

% section spatial logic via knots (end)

\input{qm2pi.conclusion}

% section conclusion (end)

%\input{qm2pi.dtcodes} 

% section wiring algorithm (end)

\input{qm2pi.ack} 

% section acknowledgments (end)

\newpage


\bibliographystyle{plain}   
\bibliography{../../biblios/main.bib}

\input{qm2pi.rhodetails}

\end{document}

 

%\documentclass[12pt]{llncs}
%\documentclass{jktr}

\usepackage[pdftex]{hyperref}                   
\usepackage {listings}
\usepackage {mathpartir}
\usepackage{bcprules}
%\usepackage{listings}
                       
\usepackage{graphicx} 
%\usepackage[margins=2.5cm,nohead,nofoot]{geometry}
%\usepackage{geometry}
\usepackage{amsfonts}
\usepackage{amstext}
\usepackage{latexsym}
\usepackage{amssymb}
\usepackage{color}


%\include{myPreamble}
\include{qm2pi.local} 

%\ifpdf
%\usepackage[pdftex]{graphicx}
%\else
%\usepackage{graphicx}
%\fi

 % \ifpdf
%  \usepackage{pdfsync}
%  \if


%\title{Brief Article}
%\author{David F. Snyder}
%\author{L.G. Meredith}

%\address{Dept. of Math., Texas State University--San Marcos, San Marcos, TX 78666}
       
\pagestyle{empty}


\begin{document}

\lstset{language=[Objective]Caml,frame=shadowbox}

\input{qm2pi.front}

% section front matter (end)

\input{qm2pi.intro} 
 
% section introduction (end)

% \input{qm2pi.knotations} 

% section notation (end)

\input{qm2pi.process.calculi} 

% section concurrent_process_calculi_and_spatial_logics_ (end)
    
%\input{qm2pi.knots2pi} 

%\input{qm2pi.trefoil} 

%\input{qm2pi.mainthm} 

% subsection basic_interpretation (end)

%\input{qm2pi.rho.presentation} 
\subsection{The syntax and semantics of the notation system}\label{sub:the_syntax_and_semantics_of_the_notation_system} % (fold)

We now summarize a technical presentation of the calculus that
embodies our theory of dynamics. The typical presentation of such a
calculus follows the style of giving generators and relations on
them. The grammar, below, describing term constructors, freely
generates the set of processes, $\Proc$. This set is then quotiented
by a relation known as structural congruence and it is over this set
that the notion of dynamics is expressed. This presentation is
essentially that of \cite{MeredithR05} with the addition of
polyadicity and summation. For readability we have relegated some of
the technical subtleties to an appendix.

\subsubsection{Process grammar}\label{subsub:process_grammar}

\begin{mathpar}
  \inferrule* [lab=synchronization] {} {{M} \bc \pzero \;|\; x?F \;|\; x!C }
  \and
  \inferrule* [lab=abstraction] {} {{F} \bc (x)P}
  \and
  \inferrule* [lab=concretion] {} {{C} \bc \langle Q \rangle}
  \and
  \inferrule* [lab=process] {} {{P,Q} \bc M \;| \;P|Q \;|\; @{x}}
  \and
  \inferrule* [lab=name] {} {{x} \bc \quotep{P}}
\end{mathpar} 

Note that $\vec{x}$ (resp. $\vec{P}$) denotes a vector of names
(resp. processes) of length $|\vec{x}|$ (resp. $|\vec{P}|$). We adopt
the following useful abbreviations.

\begin{mathpar}
   x?(\vec{y}).P := x.(\vec{y})P \and  x\clift{\vec{P}} := x.\clift{\vec{P}}
   \and x!(y) := \lift{x}{\dropn{y}}
   \and \Pi_{i=0}^{n-1}P_i := P_0 | \ldots | P_{n-1}
\end{mathpar}

\subsubsection{Structural congruence}

\paragraph{Free and bound names and alpha-equivalence.} At the
core of structural equivalence is alpha-equivalence which identifies
process that are the same up to a change of variable. Formally, we
recognize the distinction between free and bound names. The free names
of a process, $\freenames{P}$, may be calculated recursively as
follows:

\begin{mathpar}
\freenames{\pzero} := \emptyset
  \and \\
  \freenames{x?(y).P} := \{ x \} \cup (\freenames{P} \setminus \{ y \})
  \and 
  \freenames{x!\langle P \rangle} := \{ x \} \cup \{ P \} 
  \and \\
  \freenames{P|Q} := \freenames{P} \cup \freenames{Q}
  \and \\
  \freenames{@{x}} := \{ x \}
\end{mathpar}

$\pi$
$\quotep{\pi}$

$\freenames{-} : \pi \to \mathcal{P}(\quotep{\pi})$

\begin{eqnarray*}
  \freenames{\pzero} & := & \emptyset \\
  \freenames{x?(y).P} & := & \{ x \} \cup (\freenames{P} \setminus \{ y \}) \\
  \freenames{x!\langle P \rangle} & := & \{ x \} \cup \{ P \} \\
  \freenames{P|Q} & := & \freenames{P} \cup \freenames{Q} \\
  \freenames{\dropn{x}} & := & \{ x \}
\end{eqnarray*}

The bound names of a process, $\boundnames{P}$, are those names occurring in $P$
that are not free. For example, in $x?(y).0$, the name $x$ is free, while $y$ is bound.

\begin{mathpar}
  \inferrule* [lab=monoidal-laws] {} { P|Q \equiv Q|P \and P|0 \equiv P \and P|(Q|R) \equiv (P|Q)|R }
\end{mathpar}

\begin{mathpar}
  \inferrule* [lab=alpha-equivalence] {} { (x)P \equiv (y)P\{y/x\} \and y \not\in \freenames{P} }
\end{mathpar}

\begin{definition}
Then two processes, $P,Q$, are alpha-equivalent if $P = Q\{\vec{y}/\vec{x}\}$ for
some $\vec{x} \in \boundnames{Q},\vec{y} \in \boundnames{P}$, where $Q\{\vec{y}/\vec{x}\}$
denotes the capture-avoiding substitution of $\vec{y}$ for $\vec{x}$ in $Q$.
\end{definition}

\begin{definition}
  The {\em structural congruence} \cite{SangiorgiWalker} , $\equiv$,
  between processes is the least congruence containing
  alpha-equivalence, satisfying the abelian monoid laws
  (associativity, commutativity and $\pzero$ as identity) for parallel
  composition $|$ and for summation $+$.
\end{definition}

\subsection{Name equivalence}

We take name equivalence, written $\nameeq$, to be the smallest
equivalence relation generated by the following rules.

\begin{mathpar}
\inferrule*[lab=Quote-drop]
{ }
{ \quotep{@{x}} \nameeq x }

\inferrule*[lab=Struct-equiv]
{ P \scong Q }
{ \quotep{P} \nameeq \quotep{Q} }
\end{mathpar}

The astute reader will have noticed that the mutual recursion of names
and processes imposes a mutual recursion on alpha-equivalence and
structural equivalence via name-equivalence. Fortunately, all of this
works out pleasantly and we may calculate in the natural way, free of
concern. The reader interested in the details is referred to the
appendix \ref{appendix:rho_details}.

\subsection{Substitution}

We use $\Proc$ for the set of processes, $\QProc$ for the set of
names, and $\id{\{}\vec{y} / \vec{x} \id{\}}$ to denote partial maps,
$s : \QProc \rightarrow \QProc$. A map, $s$ lifts, uniquely, to a map
on process terms, $\widehat{s} : \Proc \rightarrow \Proc$ by the
following equations.

\begin{mathpar}
  (0) \psubstp{Q}{P} := 0 \\
  (R \juxtap S) \psubstp{Q}{P}
  :=    
  (R)\psubstp{Q}{P} \juxtap (S) \psubstp{Q}{P} \\
  (x?(y).R) \psubstp{Q}{P}    
  :=    
  (x)\substp{Q}{P} (z)\concat( (R \psubstn{z}{y}) \psubstp{Q}{P} ) \\
  (\lift{x}{R}) \psubstp{Q}{P}  
  :=
  \lift{(x)\substp{Q}{P}}{ R \psubstp{Q}{P} } \\
%   (\dropn{x})  \psubstp{Q}{P}       
%   := 
%   \left\{ 
%     \begin{array}{ccc} 
%       \dropn{\quotep{Q}} & & x \nameeq \quotep{P} \\
%       \dropn{x} & & otherwise \\
%     \end{array}
%   \right. 
  (\dropn{x})  \psubstp{Q}{P}       
  := 
  \left\{ 
    \begin{array}{ccc} 
      Q & & x \nameeq \quotep{P} \\
      \dropn{x} & & otherwise \\
    \end{array}
  \right.
\end{mathpar}
 

where

\begin{eqnarray}
  (x)\id{\{} \lpquote Q \rpquote / \lpquote P \rpquote \id{\}}            = 
  \left\{ 
    \begin{array}{ccc}
      \lpquote Q \rpquote & & x \nameeq \lpquote P \rpquote \\
      x & & otherwise \\
    \end{array}
  \right. \nonumber
\end{eqnarray}

and $z$ is chosen distinct from $\quotep{P}$, $\quotep{Q}$, the free
names in $Q$, and all the names in $R$. Our $\alpha$-equivalence will
be built in the standard way from this substitution.

\begin{remark}\label{rem:no_self_referential_names}
  One consequence of these definitions is that $\forall P. \quotep{P}
  \not\in \freenames{P}$.
\end{remark}

\subsection{ Dynamic quote: an example }

Anticipating something of what's to come, consider applying the
substitution, $\widehat{\id{\{}u / z \id{\}}}$, to the following pair
of processes, $\lift{w}{y!(z)}$ and $w[ \lpquote y!(z) \rpquote ]$.

\begin{eqnarray}
	\lift{w}{y!(z)}\widehat{\id{\{}u / z \id{\}}}
		& = &
		\lift{w}{y!(u)} \nonumber\\
	w[ \lpquote y!(z) \rpquote ] \widehat{ \id{\{}u / z \id{\}} }
		& = &
		w[ \lpquote y!(z) \rpquote ] \nonumber
\end{eqnarray}

Because the body of the process between quotes is impervious to
substitution, we get radically different answers. In fact, by
examining the first process in an input context,
e.g. $x?(z).\lift{w}{y!(z)}$, we see that the process under the lift
operator may be shaped by prefixed inputs binding a name inside it. In
this sense, the lift operator will be seen as a way to dynamically
construct processes before reifying them as names.

Finally equipped with these standard features we can present the
dynamics of the calculus.

\subsubsection{Operational semantics} 

Finally, we introduce the computational dynamics. What marks these
algebras as distinct from other more traditionally studied algebraic
structures, e.g. vector spaces or polynomial rings, is the manner in
which dynamics is captured. In traditional structures, dynamics is typically
expressed through morphisms between such structures, as in linear maps
between vector spaces or morphisms between rings. In algebras
associated with the semantics of computation, the dynamics is
expressed as part of the algebraic structure itself, through a
reduction reduction relation typically denoted by $\red$. Below, we
give a recursive presentation of this relation for the calculus used
in the encoding.

$\red \subseteq \pi \times \pi$
$\red : \pi \to \mathcal{P}(\pi)$

\begin{mathpar}
  \inferrule* [lab=Comm] { \textsf{match}( x_{src}, x_{trgt} ) } { x_{trgt}?(y)P \; | \; x_{src}!\langle {Q} \rangle \red P\{\quotep{Q}/y}\} }
  \and \\
  \inferrule* [lab=Par] {{P} \red {P}'} {{{P} | {Q}} \red {{P}' | {Q}}}
  \and
  \inferrule* [lab=Equiv]{{{P} \scong {P}'} \andalso {{P}' \red {Q}'} \andalso {{Q}' \scong {Q}}}{{P} \red {Q}}
\end{mathpar}

\begin{eqnarray*}
  match_{\equiv} (\quotep{P},\quotep{Q}) & := & P \equiv Q \\
  match_{\dagger}(\quotep{P},\quotep{Q}) & := & \forall R. P|Q \red^{*} R => R \red^{*} 0 \\
  match_{K}(\quotep{P},\quotep{Q}) & := & K \mbox{ for some context } K
\end{eqnarray*}

$u?(x)P | u!\langle Q \rangle \red P\{\quotep{Q}/x\}$

%We write $\wred$ for $\red^*$, and $P\red$ if $\exists Q $ such that $ P \red Q$.
We write $P\red$ if $\exists Q $ such that $ P \red Q$ and $P\not\red$, otherwise.

\section{Replication}

As mentioned before, it is known that replication (and hence
recursion) can be implemented in a higher-order process algebra
\cite{SangiorgiWalker}. As our first example of calculation with the
machinery thus far presented we give the construction explicitly in
the {\rhoc}.

\begin{eqnarray}
	D_{x} & := & \prefix{x}{y}{(\binpar{\outputp{x}{y}}{@{y}})} \nonumber\\
	\bangp_{x}{P} & := & \binpar{{x}!\langle{\binpar{D_{x}}{P}}\rangle}{D_{x}} \nonumber
\end{eqnarray}

\begin{eqnarray}
	\bangp_{x}{P} & & \nonumber\\
	=
	& {x}!\langle{(\prefix{x}{y}{(\outputp{x}{y} | @{y})) | P}}\rangle 
	      | \prefix{x}{y}{(\outputp{x}{y} | @{y})} & \nonumber\\
	\red
	& (\outputp{x}{y} | @{y})\substn{\quotep{(\prefix{x}{y}{(@{y} | \outputp{x}{y})) | P}}}{y} & \nonumber\\
	=
	& \outputp{x}{\quotep{(\prefix{x}{y}{(\outputp{x}{y} | @{y})) | P}}}
	  | {(\prefix{x}{y}{(\outputp{x}{y} | @{y})) | P}} & \nonumber\\
	\red
	& \ldots & \nonumber\\
	\red^*
	& P | P | \ldots & \nonumber
\end{eqnarray}

Of course, this encoding, as an implementation, runs away, unfolding
$\bangp{P}$ eagerly. A lazier and more implementable replication
operator, restricted to input-guarded processes, may be obtained as follows.

\begin{eqnarray}
\bangp{\prefix{u}{v}{P}} 
	:= 
	\binpar{\lift{x}{\prefix{u}{v}{(\binpar{D(x)}{P})}}}{D(x)} \nonumber
\end{eqnarray}

\begin{remark}
  Note that the lazier definition still does not deal with summation
  or mixed summation (i.e. sums over input and output). The reader is
  invited to construct definitions of replication that deal with these
  features. 

  Further, the definitions are parameterized in a name, $x$. Can you,
  gentle reader, make a definition that eliminates this parameter and
  guarantees no accidental interaction between the replication
  machinery and the process being replicated -- i.e. no accidental
  sharing of names used by the process to get its work done and the
  name(s) used by the replication to effect copying. This latter
  revision of the definition of replication is crucial to obtaining
  the expected identity $!!P \sim !P$.
\end{remark}

\begin{remark}\label{rem:paradoxical_combinator}
  The reader familiar with the lambda calculus will have noticed the
  similarity between $D$ and the paradoxical combinator.

  [Ed. note: the existence of this seems to suggest we have to be more
  restrictive on the set of processes and names we admit if we are to
  support no-cloning.]
\end{remark}

\subsubsection{Bisimulation}

The computational dynamics gives rise to another kind of equivalence,
the equivalence of computational behavior. As previously mentioned
this is typically captured \emph{via} some form of bisimulation.

% The notion we use in this paper is weak barbed bisimulation
% \cite{milner91polyadicpi}.

The notion we use in this paper is derived from weak barbed
bisimulation \cite{milner91polyadicpi}. 

\begin{definition}
An \emph{observation relation}, $\downarrow_{\mathcal N}$, over a set
of names, $\mathcal N$, is the smallest relation satisfying the rules
below.

\infrule[Out-barb]{y \in {\mathcal N}, \; x \nameeq y}
		  {\outputp{x}{v} \downarrow_{\mathcal N} x}
\infrule[Par-barb]{\mbox{$P\downarrow_{\mathcal N} x$ or $Q\downarrow_{\mathcal N} x$}}
		  {\binpar{P}{Q} \downarrow_{\mathcal N} x}

We write $P \Downarrow_{\mathcal N} x$ if there is $Q$ such that 
$P \wred Q$ and $Q \downarrow_{\mathcal N} x$.
\end{definition}

\begin{definition}
%\label{def.bbisim}
An  ${\mathcal N}$-\emph{barbed bisimulation} over a set of names, ${\mathcal N}$, is a symmetric binary relation 
${\mathcal S}_{\mathcal N}$ between agents such that $P\rel{S}_{\mathcal N}Q$ implies:
\begin{enumerate}
\item If $P \red P'$ then $Q \wred Q'$ and $P'\rel{S}_{\mathcal N} Q'$.
\item If $P\downarrow_{\mathcal N} x$, then $Q\Downarrow_{\mathcal N} x$.
\end{enumerate}
$P$ is ${\mathcal N}$-barbed bisimilar to $Q$, written
$P \wbbisim_{\mathcal N} Q$, if $P \rel{S}_{\mathcal N} Q$ for some ${\mathcal N}$-barbed bisimulation ${\mathcal S}_{\mathcal N}$.
\end{definition}

$\mathcal{R} \subseteq \pi \times \pi$

$P \mathcal{R} Q => \forall P'. P \red P' \Rightarrow \exists Q'. Q \red Q', P' \mathcal{R} Q'$

$P \vdash x \Rightarrow Q \vdash x$

\begin{mathpar}
  \inferrule*[lab=Out-barb]{x \nameeq y}{{y}!\langle{Q}\rangle \vdash x}
  \and
  \inferrule*[lab=Par-barb]{\mbox{$P\vdash x$ or $Q\vdash x$}}{\binpar{P}{Q} \vdash x}
\end{mathpar}

\subsubsection{Contexts}

One of the principle advantages of computational calculi like the
$\pi$-calculus is a well-defined notion of context,
contextual-equivalence and a correlation between
contextual-equivalence and notions of bisimulation. The notion of
context allows the decomposition of a process into (sub-)process and
its syntactic environment, its context. Thus, a context may be
thought of as a process with a ``hole'' (written $\Box$) in it. The
application of a context $M$ to a process $P$, written $M[P]$, is
tantamount to filling the hole in $M$ with $P$. In this paper we do
not need the full weight of this theory, but do make use of the notion
of context in the proof the main theorem. 

\begin{mathpar}
  \inferrule* [lab=summation] {} {{M_{M},M_{N}} \bc \Box \;|\; x.M_{A} \;|\; M_{M}+M_{N}}
  \and
  \inferrule* [lab=agent] {} {{M_{A}} \bc (\vec{x})M_{P} \;| \; \clift{P_0,\ldots,M_{P},\ldots,P_N}}
  \and \\
  \inferrule* [lab=process] {} {{M_{P}} \bc M_{N} \;| \;P|M_{P} }
\end{mathpar} 

\begin{mathpar}
  \inferrule* [lab=sychronization] {} {M_{N} \bc \Box \;|\; x?M_{F} \;|\; x!M_{C}}
  \and
  \inferrule* [lab=abstraction] {} {{M_{F}} \bc (x)M_{P} }
  \and
  \inferrule* [lab=concretion] {} {{M_{C}} \bc \langle M_{P} \rangle }
  \and \\
  \inferrule* [lab=process] {} {{M_{P}} \bc M_{N} \;| \;P|M_{P} }
\end{mathpar}

\begin{definition}[contextual application] Given a context $M$, and
  process $P$, we define the \emph{contextual application}, $M[P] :=
  M\{P/\Box\}$. That is, the contextual application of M to P is the
  substitution of $P$ for $\Box$ in $M$.
\end{definition}

$\meaningof{-} : L \to \mathcal{P}(\pi)$

\begin{mathpar}
  \inferrule* [lab=collection] {} {\meaningof{true} = \pi, \and \meaningof{~E} = \pi \setminus \meaningof{E}, \and \meaningof{E_{1} \& E_{2}} = \meaningof{E_{1}} \cap \meaningof{E_{2}}}
\end{mathpar}

\begin{mathpar}
  \inferrule* [lab=structure] {} {\meaningof{0} = \{ P \in \pi | P \equiv 0 \}, \and \\ \meaningof{E_1 | E_2} = \{ P \in \pi | P \equiv P_{1} | P_{2}, P_{1} \in \meaningof{E_{1}}, P_{2} \in \meaningof{E_2}\} }
\end{mathpar}

\begin{mathpar}
 \inferrule* [lab=behavior] {} {\meaningof{\langle a?b \rangle E} = \{ P \in \pi | P \equiv Q | u?(y)P', \\ \and \\\\ \and \\ \;\;\; u \in \meaningof{a}, \forall z.P'\{z/y\} \in \meaningof{E\{z/b\}}\}, \and \\ \meaningof{a!E} = \{ P \in \pi | P \equiv Q | x!\langle P' \rangle, x \in \meaningof{a} P' \in \meaningof{E}\} }
\end{mathpar}

\begin{mathpar}
 \inferrule* [lab=nominal] {} {\meaningof{\quotep{E}} = \{ \quotep{P} \in \quotep{\pi} | P \in \meaningof{E} \}, \and \meaningof{\quotep{P}} = \{ \quotep{Q} \in \quotep{\pi} | P \equiv Q \} \and \\ \meaningof{@\quotep{E}} = \{ P \in \pi | P \equiv @x, x \in \meaningof{E} \}}
\end{mathpar}

\begin{eqnarray*}
  \\
  \meaningof{-} : TS \to ST
\end{eqnarray*}

\begin{eqnarray*}
  \\
  L : TS \to ST
\end{eqnarray*}

\begin{eqnarray*}
  \\
  P \models E \iff P \in \meaningof{E}
\end{eqnarray*}

\begin{eqnarray*}
  P \approx_{L} Q \iff \forall E \in L. P \models E \iff Q \models E
\end{eqnarray*}

\begin{eqnarray*}
  P \approx_{K} Q
\end{eqnarray*}

\begin{eqnarray*}
  P \approx Q
\end{eqnarray*}

$\approx_{K} = \approx = \approx_{L}$

\subsubsection{Contextual duality}

Note that contexts extend the quotation operation to a family of
operations from processes to names. Given a context, $M$, we can
define a \emph{nominal context}, $\quotep{M}$ by $\quotep{M}[P] :=
\quotep{M[P]}$. To foreshadow what is to come we observe that these
operations enjoy a duality with processes very much like the duality
between vectors and maps from vectors to scalars.

Further, because the calculus is essentially higher-order, we have a
correspondence between contexts and processes. More specifically,
given a name $x$ and a context $M$ we can construct $M^{*}_{x}$ such
that 

\begin{mathpar}
  M^{*}_{x} | \lift{x}{P} \red M[P]
\end{mathpar}

namely,

\begin{mathpar}
  M^{*}_{x} := x?(u).M[\dropn{u}]
\end{mathpar}

The dependence of $M^{*}_{x}$ on a name makes it an abstraction, 

\begin{mathpar}
  M^{*} := (x)x?(u).M[\dropn{u}]
\end{mathpar}

\subsection{Additional notation}

It will sometimes be convenient to denote the process a name
quotes. We already have the notation $x = \quotep{P}$, but it will be
convenient to introduce an alternate notation, $\procn{x}$, when we
want to emphasize the connection to the use of the name. Note that, by
virtue of name equivalence, $\quotep{\procn{x}} \nameeq x$; so, the
notation is consistent with previous definitions.

Further, because names have structure it is possible to effect
substitutions on the basis of that structure. This means we need to
upgrade our notation for substitutions, which we accomplish by
adapting comprehension notation. Thus,

\begin{mathpar}
  P\{ y / x : x \in S \}
\end{mathpar}

is interpreted to mean the process derived from P by replacing (in a
capture-avoiding manner) each occurrence of $x$ in $S$ by $y$. For example,

\begin{mathpar}
  P\{ \quotep{\procn{x}|\procn{x}} / x : x \in \freenames{P} \}
\end{mathpar}

will replace each (occurrence) of a free name $x$ in $P$ by
$\quotep{\procn{x}|\procn{x}}$.

Also, we will avail ourselves of the notation $x^{L}$ and $x^{R}$ to
denote injections of a name into disjoint copies of the name
space. There are numerous ways to accomplish this. One example can be
found in \cite{MeredithR05}. This notation overloads to vectors of
names: $\vec{x}^{\pi} := (x_{i}^{\pi} \; : \; 0 \leq i < |\vec{x}| )$ where $\pi \in \{L,R\}$.

We also use $P^{\Box} := P|\Box$.

In \cite{MeredithR05} an interpretation of the new operator is
given. It turns out that there are several possible interpretations
all enjoying the requisite algebraic properties of the operator (see
\cite{milner91polyadicpi}). We will therefore make liberal use of
$(\nu\; \vec{x})P$.

% subsection the_syntax_and_semantics_of_the_notation_system (end)   

\input{qm2pi.qmops} 

\input{qm2pi.sterngerlach} 

\input{qm2pi.metric} 

% section concurrent_process_calculi (end)

%\input{qm2pi.proofsketch}

% section proof sketch (end)

%\input{qm2pi.slviaknots} 

% section spatial logic via knots (end)

\input{qm2pi.conclusion}

% section conclusion (end)

%\input{qm2pi.dtcodes} 

% section wiring algorithm (end)

\input{qm2pi.ack} 

% section acknowledgments (end)

\newpage


\bibliographystyle{plain}   
\bibliography{../../biblios/main.bib}

\input{qm2pi.rhodetails}

\end{document}

 

%\documentclass[12pt]{llncs}
%\documentclass{jktr}

\usepackage[pdftex]{hyperref}                   
\usepackage {listings}
\usepackage {mathpartir}
\usepackage{bcprules}
%\usepackage{listings}
                       
\usepackage{graphicx} 
%\usepackage[margins=2.5cm,nohead,nofoot]{geometry}
%\usepackage{geometry}
\usepackage{amsfonts}
\usepackage{amstext}
\usepackage{latexsym}
\usepackage{amssymb}
\usepackage{color}


%\include{myPreamble}
\include{qm2pi.local} 

%\ifpdf
%\usepackage[pdftex]{graphicx}
%\else
%\usepackage{graphicx}
%\fi

 % \ifpdf
%  \usepackage{pdfsync}
%  \if


%\title{Brief Article}
%\author{David F. Snyder}
%\author{L.G. Meredith}

%\address{Dept. of Math., Texas State University--San Marcos, San Marcos, TX 78666}
       
\pagestyle{empty}


\begin{document}

\lstset{language=[Objective]Caml,frame=shadowbox}

\input{qm2pi.front}

% section front matter (end)

\input{qm2pi.intro} 
 
% section introduction (end)

% \input{qm2pi.knotations} 

% section notation (end)

\input{qm2pi.process.calculi} 

% section concurrent_process_calculi_and_spatial_logics_ (end)
    
%\input{qm2pi.knots2pi} 

%\input{qm2pi.trefoil} 

%\input{qm2pi.mainthm} 

% subsection basic_interpretation (end)

%\input{qm2pi.rho.presentation} 
\subsection{The syntax and semantics of the notation system}\label{sub:the_syntax_and_semantics_of_the_notation_system} % (fold)

We now summarize a technical presentation of the calculus that
embodies our theory of dynamics. The typical presentation of such a
calculus follows the style of giving generators and relations on
them. The grammar, below, describing term constructors, freely
generates the set of processes, $\Proc$. This set is then quotiented
by a relation known as structural congruence and it is over this set
that the notion of dynamics is expressed. This presentation is
essentially that of \cite{MeredithR05} with the addition of
polyadicity and summation. For readability we have relegated some of
the technical subtleties to an appendix.

\subsubsection{Process grammar}\label{subsub:process_grammar}

\begin{mathpar}
  \inferrule* [lab=synchronization] {} {{M} \bc \pzero \;|\; x?F \;|\; x!C }
  \and
  \inferrule* [lab=abstraction] {} {{F} \bc (x)P}
  \and
  \inferrule* [lab=concretion] {} {{C} \bc \langle Q \rangle}
  \and
  \inferrule* [lab=process] {} {{P,Q} \bc M \;| \;P|Q \;|\; @{x}}
  \and
  \inferrule* [lab=name] {} {{x} \bc \quotep{P}}
\end{mathpar} 

Note that $\vec{x}$ (resp. $\vec{P}$) denotes a vector of names
(resp. processes) of length $|\vec{x}|$ (resp. $|\vec{P}|$). We adopt
the following useful abbreviations.

\begin{mathpar}
   x?(\vec{y}).P := x.(\vec{y})P \and  x\clift{\vec{P}} := x.\clift{\vec{P}}
   \and x!(y) := \lift{x}{\dropn{y}}
   \and \Pi_{i=0}^{n-1}P_i := P_0 | \ldots | P_{n-1}
\end{mathpar}

\subsubsection{Structural congruence}

\paragraph{Free and bound names and alpha-equivalence.} At the
core of structural equivalence is alpha-equivalence which identifies
process that are the same up to a change of variable. Formally, we
recognize the distinction between free and bound names. The free names
of a process, $\freenames{P}$, may be calculated recursively as
follows:

\begin{mathpar}
\freenames{\pzero} := \emptyset
  \and \\
  \freenames{x?(y).P} := \{ x \} \cup (\freenames{P} \setminus \{ y \})
  \and 
  \freenames{x!\langle P \rangle} := \{ x \} \cup \{ P \} 
  \and \\
  \freenames{P|Q} := \freenames{P} \cup \freenames{Q}
  \and \\
  \freenames{@{x}} := \{ x \}
\end{mathpar}

$\pi$
$\quotep{\pi}$

$\freenames{-} : \pi \to \mathcal{P}(\quotep{\pi})$

\begin{eqnarray*}
  \freenames{\pzero} & := & \emptyset \\
  \freenames{x?(y).P} & := & \{ x \} \cup (\freenames{P} \setminus \{ y \}) \\
  \freenames{x!\langle P \rangle} & := & \{ x \} \cup \{ P \} \\
  \freenames{P|Q} & := & \freenames{P} \cup \freenames{Q} \\
  \freenames{\dropn{x}} & := & \{ x \}
\end{eqnarray*}

The bound names of a process, $\boundnames{P}$, are those names occurring in $P$
that are not free. For example, in $x?(y).0$, the name $x$ is free, while $y$ is bound.

\begin{mathpar}
  \inferrule* [lab=monoidal-laws] {} { P|Q \equiv Q|P \and P|0 \equiv P \and P|(Q|R) \equiv (P|Q)|R }
\end{mathpar}

\begin{mathpar}
  \inferrule* [lab=alpha-equivalence] {} { (x)P \equiv (y)P\{y/x\} \and y \not\in \freenames{P} }
\end{mathpar}

\begin{definition}
Then two processes, $P,Q$, are alpha-equivalent if $P = Q\{\vec{y}/\vec{x}\}$ for
some $\vec{x} \in \boundnames{Q},\vec{y} \in \boundnames{P}$, where $Q\{\vec{y}/\vec{x}\}$
denotes the capture-avoiding substitution of $\vec{y}$ for $\vec{x}$ in $Q$.
\end{definition}

\begin{definition}
  The {\em structural congruence} \cite{SangiorgiWalker} , $\equiv$,
  between processes is the least congruence containing
  alpha-equivalence, satisfying the abelian monoid laws
  (associativity, commutativity and $\pzero$ as identity) for parallel
  composition $|$ and for summation $+$.
\end{definition}

\subsection{Name equivalence}

We take name equivalence, written $\nameeq$, to be the smallest
equivalence relation generated by the following rules.

\begin{mathpar}
\inferrule*[lab=Quote-drop]
{ }
{ \quotep{@{x}} \nameeq x }

\inferrule*[lab=Struct-equiv]
{ P \scong Q }
{ \quotep{P} \nameeq \quotep{Q} }
\end{mathpar}

The astute reader will have noticed that the mutual recursion of names
and processes imposes a mutual recursion on alpha-equivalence and
structural equivalence via name-equivalence. Fortunately, all of this
works out pleasantly and we may calculate in the natural way, free of
concern. The reader interested in the details is referred to the
appendix \ref{appendix:rho_details}.

\subsection{Substitution}

We use $\Proc$ for the set of processes, $\QProc$ for the set of
names, and $\id{\{}\vec{y} / \vec{x} \id{\}}$ to denote partial maps,
$s : \QProc \rightarrow \QProc$. A map, $s$ lifts, uniquely, to a map
on process terms, $\widehat{s} : \Proc \rightarrow \Proc$ by the
following equations.

\begin{mathpar}
  (0) \psubstp{Q}{P} := 0 \\
  (R \juxtap S) \psubstp{Q}{P}
  :=    
  (R)\psubstp{Q}{P} \juxtap (S) \psubstp{Q}{P} \\
  (x?(y).R) \psubstp{Q}{P}    
  :=    
  (x)\substp{Q}{P} (z)\concat( (R \psubstn{z}{y}) \psubstp{Q}{P} ) \\
  (\lift{x}{R}) \psubstp{Q}{P}  
  :=
  \lift{(x)\substp{Q}{P}}{ R \psubstp{Q}{P} } \\
%   (\dropn{x})  \psubstp{Q}{P}       
%   := 
%   \left\{ 
%     \begin{array}{ccc} 
%       \dropn{\quotep{Q}} & & x \nameeq \quotep{P} \\
%       \dropn{x} & & otherwise \\
%     \end{array}
%   \right. 
  (\dropn{x})  \psubstp{Q}{P}       
  := 
  \left\{ 
    \begin{array}{ccc} 
      Q & & x \nameeq \quotep{P} \\
      \dropn{x} & & otherwise \\
    \end{array}
  \right.
\end{mathpar}
 

where

\begin{eqnarray}
  (x)\id{\{} \lpquote Q \rpquote / \lpquote P \rpquote \id{\}}            = 
  \left\{ 
    \begin{array}{ccc}
      \lpquote Q \rpquote & & x \nameeq \lpquote P \rpquote \\
      x & & otherwise \\
    \end{array}
  \right. \nonumber
\end{eqnarray}

and $z$ is chosen distinct from $\quotep{P}$, $\quotep{Q}$, the free
names in $Q$, and all the names in $R$. Our $\alpha$-equivalence will
be built in the standard way from this substitution.

\begin{remark}\label{rem:no_self_referential_names}
  One consequence of these definitions is that $\forall P. \quotep{P}
  \not\in \freenames{P}$.
\end{remark}

\subsection{ Dynamic quote: an example }

Anticipating something of what's to come, consider applying the
substitution, $\widehat{\id{\{}u / z \id{\}}}$, to the following pair
of processes, $\lift{w}{y!(z)}$ and $w[ \lpquote y!(z) \rpquote ]$.

\begin{eqnarray}
	\lift{w}{y!(z)}\widehat{\id{\{}u / z \id{\}}}
		& = &
		\lift{w}{y!(u)} \nonumber\\
	w[ \lpquote y!(z) \rpquote ] \widehat{ \id{\{}u / z \id{\}} }
		& = &
		w[ \lpquote y!(z) \rpquote ] \nonumber
\end{eqnarray}

Because the body of the process between quotes is impervious to
substitution, we get radically different answers. In fact, by
examining the first process in an input context,
e.g. $x?(z).\lift{w}{y!(z)}$, we see that the process under the lift
operator may be shaped by prefixed inputs binding a name inside it. In
this sense, the lift operator will be seen as a way to dynamically
construct processes before reifying them as names.

Finally equipped with these standard features we can present the
dynamics of the calculus.

\subsubsection{Operational semantics} 

Finally, we introduce the computational dynamics. What marks these
algebras as distinct from other more traditionally studied algebraic
structures, e.g. vector spaces or polynomial rings, is the manner in
which dynamics is captured. In traditional structures, dynamics is typically
expressed through morphisms between such structures, as in linear maps
between vector spaces or morphisms between rings. In algebras
associated with the semantics of computation, the dynamics is
expressed as part of the algebraic structure itself, through a
reduction reduction relation typically denoted by $\red$. Below, we
give a recursive presentation of this relation for the calculus used
in the encoding.

$\red \subseteq \pi \times \pi$
$\red : \pi \to \mathcal{P}(\pi)$

\begin{mathpar}
  \inferrule* [lab=Comm] { \textsf{match}( x_{src}, x_{trgt} ) } { x_{trgt}?(y)P \; | \; x_{src}!\langle {Q} \rangle \red P\{\quotep{Q}/y}\} }
  \and \\
  \inferrule* [lab=Par] {{P} \red {P}'} {{{P} | {Q}} \red {{P}' | {Q}}}
  \and
  \inferrule* [lab=Equiv]{{{P} \scong {P}'} \andalso {{P}' \red {Q}'} \andalso {{Q}' \scong {Q}}}{{P} \red {Q}}
\end{mathpar}

\begin{eqnarray*}
  match_{\equiv} (\quotep{P},\quotep{Q}) & := & P \equiv Q \\
  match_{\dagger}(\quotep{P},\quotep{Q}) & := & \forall R. P|Q \red^{*} R => R \red^{*} 0 \\
  match_{K}(\quotep{P},\quotep{Q}) & := & K \mbox{ for some context } K
\end{eqnarray*}

$u?(x)P | u!\langle Q \rangle \red P\{\quotep{Q}/x\}$

%We write $\wred$ for $\red^*$, and $P\red$ if $\exists Q $ such that $ P \red Q$.
We write $P\red$ if $\exists Q $ such that $ P \red Q$ and $P\not\red$, otherwise.

\section{Replication}

As mentioned before, it is known that replication (and hence
recursion) can be implemented in a higher-order process algebra
\cite{SangiorgiWalker}. As our first example of calculation with the
machinery thus far presented we give the construction explicitly in
the {\rhoc}.

\begin{eqnarray}
	D_{x} & := & \prefix{x}{y}{(\binpar{\outputp{x}{y}}{@{y}})} \nonumber\\
	\bangp_{x}{P} & := & \binpar{{x}!\langle{\binpar{D_{x}}{P}}\rangle}{D_{x}} \nonumber
\end{eqnarray}

\begin{eqnarray}
	\bangp_{x}{P} & & \nonumber\\
	=
	& {x}!\langle{(\prefix{x}{y}{(\outputp{x}{y} | @{y})) | P}}\rangle 
	      | \prefix{x}{y}{(\outputp{x}{y} | @{y})} & \nonumber\\
	\red
	& (\outputp{x}{y} | @{y})\substn{\quotep{(\prefix{x}{y}{(@{y} | \outputp{x}{y})) | P}}}{y} & \nonumber\\
	=
	& \outputp{x}{\quotep{(\prefix{x}{y}{(\outputp{x}{y} | @{y})) | P}}}
	  | {(\prefix{x}{y}{(\outputp{x}{y} | @{y})) | P}} & \nonumber\\
	\red
	& \ldots & \nonumber\\
	\red^*
	& P | P | \ldots & \nonumber
\end{eqnarray}

Of course, this encoding, as an implementation, runs away, unfolding
$\bangp{P}$ eagerly. A lazier and more implementable replication
operator, restricted to input-guarded processes, may be obtained as follows.

\begin{eqnarray}
\bangp{\prefix{u}{v}{P}} 
	:= 
	\binpar{\lift{x}{\prefix{u}{v}{(\binpar{D(x)}{P})}}}{D(x)} \nonumber
\end{eqnarray}

\begin{remark}
  Note that the lazier definition still does not deal with summation
  or mixed summation (i.e. sums over input and output). The reader is
  invited to construct definitions of replication that deal with these
  features. 

  Further, the definitions are parameterized in a name, $x$. Can you,
  gentle reader, make a definition that eliminates this parameter and
  guarantees no accidental interaction between the replication
  machinery and the process being replicated -- i.e. no accidental
  sharing of names used by the process to get its work done and the
  name(s) used by the replication to effect copying. This latter
  revision of the definition of replication is crucial to obtaining
  the expected identity $!!P \sim !P$.
\end{remark}

\begin{remark}\label{rem:paradoxical_combinator}
  The reader familiar with the lambda calculus will have noticed the
  similarity between $D$ and the paradoxical combinator.

  [Ed. note: the existence of this seems to suggest we have to be more
  restrictive on the set of processes and names we admit if we are to
  support no-cloning.]
\end{remark}

\subsubsection{Bisimulation}

The computational dynamics gives rise to another kind of equivalence,
the equivalence of computational behavior. As previously mentioned
this is typically captured \emph{via} some form of bisimulation.

% The notion we use in this paper is weak barbed bisimulation
% \cite{milner91polyadicpi}.

The notion we use in this paper is derived from weak barbed
bisimulation \cite{milner91polyadicpi}. 

\begin{definition}
An \emph{observation relation}, $\downarrow_{\mathcal N}$, over a set
of names, $\mathcal N$, is the smallest relation satisfying the rules
below.

\infrule[Out-barb]{y \in {\mathcal N}, \; x \nameeq y}
		  {\outputp{x}{v} \downarrow_{\mathcal N} x}
\infrule[Par-barb]{\mbox{$P\downarrow_{\mathcal N} x$ or $Q\downarrow_{\mathcal N} x$}}
		  {\binpar{P}{Q} \downarrow_{\mathcal N} x}

We write $P \Downarrow_{\mathcal N} x$ if there is $Q$ such that 
$P \wred Q$ and $Q \downarrow_{\mathcal N} x$.
\end{definition}

\begin{definition}
%\label{def.bbisim}
An  ${\mathcal N}$-\emph{barbed bisimulation} over a set of names, ${\mathcal N}$, is a symmetric binary relation 
${\mathcal S}_{\mathcal N}$ between agents such that $P\rel{S}_{\mathcal N}Q$ implies:
\begin{enumerate}
\item If $P \red P'$ then $Q \wred Q'$ and $P'\rel{S}_{\mathcal N} Q'$.
\item If $P\downarrow_{\mathcal N} x$, then $Q\Downarrow_{\mathcal N} x$.
\end{enumerate}
$P$ is ${\mathcal N}$-barbed bisimilar to $Q$, written
$P \wbbisim_{\mathcal N} Q$, if $P \rel{S}_{\mathcal N} Q$ for some ${\mathcal N}$-barbed bisimulation ${\mathcal S}_{\mathcal N}$.
\end{definition}

$\mathcal{R} \subseteq \pi \times \pi$

$P \mathcal{R} Q => \forall P'. P \red P' \Rightarrow \exists Q'. Q \red Q', P' \mathcal{R} Q'$

$P \vdash x \Rightarrow Q \vdash x$

\begin{mathpar}
  \inferrule*[lab=Out-barb]{x \nameeq y}{{y}!\langle{Q}\rangle \vdash x}
  \and
  \inferrule*[lab=Par-barb]{\mbox{$P\vdash x$ or $Q\vdash x$}}{\binpar{P}{Q} \vdash x}
\end{mathpar}

\subsubsection{Contexts}

One of the principle advantages of computational calculi like the
$\pi$-calculus is a well-defined notion of context,
contextual-equivalence and a correlation between
contextual-equivalence and notions of bisimulation. The notion of
context allows the decomposition of a process into (sub-)process and
its syntactic environment, its context. Thus, a context may be
thought of as a process with a ``hole'' (written $\Box$) in it. The
application of a context $M$ to a process $P$, written $M[P]$, is
tantamount to filling the hole in $M$ with $P$. In this paper we do
not need the full weight of this theory, but do make use of the notion
of context in the proof the main theorem. 

\begin{mathpar}
  \inferrule* [lab=summation] {} {{M_{M},M_{N}} \bc \Box \;|\; x.M_{A} \;|\; M_{M}+M_{N}}
  \and
  \inferrule* [lab=agent] {} {{M_{A}} \bc (\vec{x})M_{P} \;| \; \clift{P_0,\ldots,M_{P},\ldots,P_N}}
  \and \\
  \inferrule* [lab=process] {} {{M_{P}} \bc M_{N} \;| \;P|M_{P} }
\end{mathpar} 

\begin{mathpar}
  \inferrule* [lab=sychronization] {} {M_{N} \bc \Box \;|\; x?M_{F} \;|\; x!M_{C}}
  \and
  \inferrule* [lab=abstraction] {} {{M_{F}} \bc (x)M_{P} }
  \and
  \inferrule* [lab=concretion] {} {{M_{C}} \bc \langle M_{P} \rangle }
  \and \\
  \inferrule* [lab=process] {} {{M_{P}} \bc M_{N} \;| \;P|M_{P} }
\end{mathpar}

\begin{definition}[contextual application] Given a context $M$, and
  process $P$, we define the \emph{contextual application}, $M[P] :=
  M\{P/\Box\}$. That is, the contextual application of M to P is the
  substitution of $P$ for $\Box$ in $M$.
\end{definition}

$\meaningof{-} : L \to \mathcal{P}(\pi)$

\begin{mathpar}
  \inferrule* [lab=collection] {} {\meaningof{true} = \pi, \and \meaningof{~E} = \pi \setminus \meaningof{E}, \and \meaningof{E_{1} \& E_{2}} = \meaningof{E_{1}} \cap \meaningof{E_{2}}}
\end{mathpar}

\begin{mathpar}
  \inferrule* [lab=structure] {} {\meaningof{0} = \{ P \in \pi | P \equiv 0 \}, \and \\ \meaningof{E_1 | E_2} = \{ P \in \pi | P \equiv P_{1} | P_{2}, P_{1} \in \meaningof{E_{1}}, P_{2} \in \meaningof{E_2}\} }
\end{mathpar}

\begin{mathpar}
 \inferrule* [lab=behavior] {} {\meaningof{\langle a?b \rangle E} = \{ P \in \pi | P \equiv Q | u?(y)P', \\ \and \\\\ \and \\ \;\;\; u \in \meaningof{a}, \forall z.P'\{z/y\} \in \meaningof{E\{z/b\}}\}, \and \\ \meaningof{a!E} = \{ P \in \pi | P \equiv Q | x!\langle P' \rangle, x \in \meaningof{a} P' \in \meaningof{E}\} }
\end{mathpar}

\begin{mathpar}
 \inferrule* [lab=nominal] {} {\meaningof{\quotep{E}} = \{ \quotep{P} \in \quotep{\pi} | P \in \meaningof{E} \}, \and \meaningof{\quotep{P}} = \{ \quotep{Q} \in \quotep{\pi} | P \equiv Q \} \and \\ \meaningof{@\quotep{E}} = \{ P \in \pi | P \equiv @x, x \in \meaningof{E} \}}
\end{mathpar}

\begin{eqnarray*}
  \\
  \meaningof{-} : TS \to ST
\end{eqnarray*}

\begin{eqnarray*}
  \\
  L : TS \to ST
\end{eqnarray*}

\begin{eqnarray*}
  \\
  P \models E \iff P \in \meaningof{E}
\end{eqnarray*}

\begin{eqnarray*}
  P \approx_{L} Q \iff \forall E \in L. P \models E \iff Q \models E
\end{eqnarray*}

\begin{eqnarray*}
  P \approx_{K} Q
\end{eqnarray*}

\begin{eqnarray*}
  P \approx Q
\end{eqnarray*}

$\approx_{K} = \approx = \approx_{L}$

\subsubsection{Contextual duality}

Note that contexts extend the quotation operation to a family of
operations from processes to names. Given a context, $M$, we can
define a \emph{nominal context}, $\quotep{M}$ by $\quotep{M}[P] :=
\quotep{M[P]}$. To foreshadow what is to come we observe that these
operations enjoy a duality with processes very much like the duality
between vectors and maps from vectors to scalars.

Further, because the calculus is essentially higher-order, we have a
correspondence between contexts and processes. More specifically,
given a name $x$ and a context $M$ we can construct $M^{*}_{x}$ such
that 

\begin{mathpar}
  M^{*}_{x} | \lift{x}{P} \red M[P]
\end{mathpar}

namely,

\begin{mathpar}
  M^{*}_{x} := x?(u).M[\dropn{u}]
\end{mathpar}

The dependence of $M^{*}_{x}$ on a name makes it an abstraction, 

\begin{mathpar}
  M^{*} := (x)x?(u).M[\dropn{u}]
\end{mathpar}

\subsection{Additional notation}

It will sometimes be convenient to denote the process a name
quotes. We already have the notation $x = \quotep{P}$, but it will be
convenient to introduce an alternate notation, $\procn{x}$, when we
want to emphasize the connection to the use of the name. Note that, by
virtue of name equivalence, $\quotep{\procn{x}} \nameeq x$; so, the
notation is consistent with previous definitions.

Further, because names have structure it is possible to effect
substitutions on the basis of that structure. This means we need to
upgrade our notation for substitutions, which we accomplish by
adapting comprehension notation. Thus,

\begin{mathpar}
  P\{ y / x : x \in S \}
\end{mathpar}

is interpreted to mean the process derived from P by replacing (in a
capture-avoiding manner) each occurrence of $x$ in $S$ by $y$. For example,

\begin{mathpar}
  P\{ \quotep{\procn{x}|\procn{x}} / x : x \in \freenames{P} \}
\end{mathpar}

will replace each (occurrence) of a free name $x$ in $P$ by
$\quotep{\procn{x}|\procn{x}}$.

Also, we will avail ourselves of the notation $x^{L}$ and $x^{R}$ to
denote injections of a name into disjoint copies of the name
space. There are numerous ways to accomplish this. One example can be
found in \cite{MeredithR05}. This notation overloads to vectors of
names: $\vec{x}^{\pi} := (x_{i}^{\pi} \; : \; 0 \leq i < |\vec{x}| )$ where $\pi \in \{L,R\}$.

We also use $P^{\Box} := P|\Box$.

In \cite{MeredithR05} an interpretation of the new operator is
given. It turns out that there are several possible interpretations
all enjoying the requisite algebraic properties of the operator (see
\cite{milner91polyadicpi}). We will therefore make liberal use of
$(\nu\; \vec{x})P$.

% subsection the_syntax_and_semantics_of_the_notation_system (end)   

\input{qm2pi.qmops} 

\input{qm2pi.sterngerlach} 

\input{qm2pi.metric} 

% section concurrent_process_calculi (end)

%\input{qm2pi.proofsketch}

% section proof sketch (end)

%\input{qm2pi.slviaknots} 

% section spatial logic via knots (end)

\input{qm2pi.conclusion}

% section conclusion (end)

%\input{qm2pi.dtcodes} 

% section wiring algorithm (end)

\input{qm2pi.ack} 

% section acknowledgments (end)

\newpage


\bibliographystyle{plain}   
\bibliography{../../biblios/main.bib}

\input{qm2pi.rhodetails}

\end{document}

 

% subsection basic_interpretation (end)

%\input{qm2pi.rho.presentation} 
\subsection{The syntax and semantics of the notation system}\label{sub:the_syntax_and_semantics_of_the_notation_system} % (fold)

We now summarize a technical presentation of the calculus that
embodies our theory of dynamics. The typical presentation of such a
calculus follows the style of giving generators and relations on
them. The grammar, below, describing term constructors, freely
generates the set of processes, $\Proc$. This set is then quotiented
by a relation known as structural congruence and it is over this set
that the notion of dynamics is expressed. This presentation is
essentially that of \cite{MeredithR05} with the addition of
polyadicity and summation. For readability we have relegated some of
the technical subtleties to an appendix.

\subsubsection{Process grammar}\label{subsub:process_grammar}

\begin{mathpar}
  \inferrule* [lab=synchronization] {} {{M} \bc \pzero \;|\; x?F \;|\; x!C }
  \and
  \inferrule* [lab=abstraction] {} {{F} \bc (x)P}
  \and
  \inferrule* [lab=concretion] {} {{C} \bc \langle Q \rangle}
  \and
  \inferrule* [lab=process] {} {{P,Q} \bc M \;| \;P|Q \;|\; @{x}}
  \and
  \inferrule* [lab=name] {} {{x} \bc \quotep{P}}
\end{mathpar} 

Note that $\vec{x}$ (resp. $\vec{P}$) denotes a vector of names
(resp. processes) of length $|\vec{x}|$ (resp. $|\vec{P}|$). We adopt
the following useful abbreviations.

\begin{mathpar}
   x?(\vec{y}).P := x.(\vec{y})P \and  x\clift{\vec{P}} := x.\clift{\vec{P}}
   \and x!(y) := \lift{x}{\dropn{y}}
   \and \Pi_{i=0}^{n-1}P_i := P_0 | \ldots | P_{n-1}
\end{mathpar}

\subsubsection{Structural congruence}

\paragraph{Free and bound names and alpha-equivalence.} At the
core of structural equivalence is alpha-equivalence which identifies
process that are the same up to a change of variable. Formally, we
recognize the distinction between free and bound names. The free names
of a process, $\freenames{P}$, may be calculated recursively as
follows:

\begin{mathpar}
\freenames{\pzero} := \emptyset
  \and \\
  \freenames{x?(y).P} := \{ x \} \cup (\freenames{P} \setminus \{ y \})
  \and 
  \freenames{x!\langle P \rangle} := \{ x \} \cup \{ P \} 
  \and \\
  \freenames{P|Q} := \freenames{P} \cup \freenames{Q}
  \and \\
  \freenames{@{x}} := \{ x \}
\end{mathpar}

$\pi$
$\quotep{\pi}$

$\freenames{-} : \pi \to \mathcal{P}(\quotep{\pi})$

\begin{eqnarray*}
  \freenames{\pzero} & := & \emptyset \\
  \freenames{x?(y).P} & := & \{ x \} \cup (\freenames{P} \setminus \{ y \}) \\
  \freenames{x!\langle P \rangle} & := & \{ x \} \cup \{ P \} \\
  \freenames{P|Q} & := & \freenames{P} \cup \freenames{Q} \\
  \freenames{\dropn{x}} & := & \{ x \}
\end{eqnarray*}

The bound names of a process, $\boundnames{P}$, are those names occurring in $P$
that are not free. For example, in $x?(y).0$, the name $x$ is free, while $y$ is bound.

\begin{mathpar}
  \inferrule* [lab=monoidal-laws] {} { P|Q \equiv Q|P \and P|0 \equiv P \and P|(Q|R) \equiv (P|Q)|R }
\end{mathpar}

\begin{mathpar}
  \inferrule* [lab=alpha-equivalence] {} { (x)P \equiv (y)P\{y/x\} \and y \not\in \freenames{P} }
\end{mathpar}

\begin{definition}
Then two processes, $P,Q$, are alpha-equivalent if $P = Q\{\vec{y}/\vec{x}\}$ for
some $\vec{x} \in \boundnames{Q},\vec{y} \in \boundnames{P}$, where $Q\{\vec{y}/\vec{x}\}$
denotes the capture-avoiding substitution of $\vec{y}$ for $\vec{x}$ in $Q$.
\end{definition}

\begin{definition}
  The {\em structural congruence} \cite{SangiorgiWalker} , $\equiv$,
  between processes is the least congruence containing
  alpha-equivalence, satisfying the abelian monoid laws
  (associativity, commutativity and $\pzero$ as identity) for parallel
  composition $|$ and for summation $+$.
\end{definition}

\subsection{Name equivalence}

We take name equivalence, written $\nameeq$, to be the smallest
equivalence relation generated by the following rules.

\begin{mathpar}
\inferrule*[lab=Quote-drop]
{ }
{ \quotep{@{x}} \nameeq x }

\inferrule*[lab=Struct-equiv]
{ P \scong Q }
{ \quotep{P} \nameeq \quotep{Q} }
\end{mathpar}

The astute reader will have noticed that the mutual recursion of names
and processes imposes a mutual recursion on alpha-equivalence and
structural equivalence via name-equivalence. Fortunately, all of this
works out pleasantly and we may calculate in the natural way, free of
concern. The reader interested in the details is referred to the
appendix \ref{appendix:rho_details}.

\subsection{Substitution}

We use $\Proc$ for the set of processes, $\QProc$ for the set of
names, and $\id{\{}\vec{y} / \vec{x} \id{\}}$ to denote partial maps,
$s : \QProc \rightarrow \QProc$. A map, $s$ lifts, uniquely, to a map
on process terms, $\widehat{s} : \Proc \rightarrow \Proc$ by the
following equations.

\begin{mathpar}
  (0) \psubstp{Q}{P} := 0 \\
  (R \juxtap S) \psubstp{Q}{P}
  :=    
  (R)\psubstp{Q}{P} \juxtap (S) \psubstp{Q}{P} \\
  (x?(y).R) \psubstp{Q}{P}    
  :=    
  (x)\substp{Q}{P} (z)\concat( (R \psubstn{z}{y}) \psubstp{Q}{P} ) \\
  (\lift{x}{R}) \psubstp{Q}{P}  
  :=
  \lift{(x)\substp{Q}{P}}{ R \psubstp{Q}{P} } \\
%   (\dropn{x})  \psubstp{Q}{P}       
%   := 
%   \left\{ 
%     \begin{array}{ccc} 
%       \dropn{\quotep{Q}} & & x \nameeq \quotep{P} \\
%       \dropn{x} & & otherwise \\
%     \end{array}
%   \right. 
  (\dropn{x})  \psubstp{Q}{P}       
  := 
  \left\{ 
    \begin{array}{ccc} 
      Q & & x \nameeq \quotep{P} \\
      \dropn{x} & & otherwise \\
    \end{array}
  \right.
\end{mathpar}
 

where

\begin{eqnarray}
  (x)\id{\{} \lpquote Q \rpquote / \lpquote P \rpquote \id{\}}            = 
  \left\{ 
    \begin{array}{ccc}
      \lpquote Q \rpquote & & x \nameeq \lpquote P \rpquote \\
      x & & otherwise \\
    \end{array}
  \right. \nonumber
\end{eqnarray}

and $z$ is chosen distinct from $\quotep{P}$, $\quotep{Q}$, the free
names in $Q$, and all the names in $R$. Our $\alpha$-equivalence will
be built in the standard way from this substitution.

\begin{remark}\label{rem:no_self_referential_names}
  One consequence of these definitions is that $\forall P. \quotep{P}
  \not\in \freenames{P}$.
\end{remark}

\subsection{ Dynamic quote: an example }

Anticipating something of what's to come, consider applying the
substitution, $\widehat{\id{\{}u / z \id{\}}}$, to the following pair
of processes, $\lift{w}{y!(z)}$ and $w[ \lpquote y!(z) \rpquote ]$.

\begin{eqnarray}
	\lift{w}{y!(z)}\widehat{\id{\{}u / z \id{\}}}
		& = &
		\lift{w}{y!(u)} \nonumber\\
	w[ \lpquote y!(z) \rpquote ] \widehat{ \id{\{}u / z \id{\}} }
		& = &
		w[ \lpquote y!(z) \rpquote ] \nonumber
\end{eqnarray}

Because the body of the process between quotes is impervious to
substitution, we get radically different answers. In fact, by
examining the first process in an input context,
e.g. $x?(z).\lift{w}{y!(z)}$, we see that the process under the lift
operator may be shaped by prefixed inputs binding a name inside it. In
this sense, the lift operator will be seen as a way to dynamically
construct processes before reifying them as names.

Finally equipped with these standard features we can present the
dynamics of the calculus.

\subsubsection{Operational semantics} 

Finally, we introduce the computational dynamics. What marks these
algebras as distinct from other more traditionally studied algebraic
structures, e.g. vector spaces or polynomial rings, is the manner in
which dynamics is captured. In traditional structures, dynamics is typically
expressed through morphisms between such structures, as in linear maps
between vector spaces or morphisms between rings. In algebras
associated with the semantics of computation, the dynamics is
expressed as part of the algebraic structure itself, through a
reduction reduction relation typically denoted by $\red$. Below, we
give a recursive presentation of this relation for the calculus used
in the encoding.

$\red \subseteq \pi \times \pi$
$\red : \pi \to \mathcal{P}(\pi)$

\begin{mathpar}
  \inferrule* [lab=Comm] { \textsf{match}( x_{src}, x_{trgt} ) } { x_{trgt}?(y)P \; | \; x_{src}!\langle {Q} \rangle \red P\{\quotep{Q}/y}\} }
  \and \\
  \inferrule* [lab=Par] {{P} \red {P}'} {{{P} | {Q}} \red {{P}' | {Q}}}
  \and
  \inferrule* [lab=Equiv]{{{P} \scong {P}'} \andalso {{P}' \red {Q}'} \andalso {{Q}' \scong {Q}}}{{P} \red {Q}}
\end{mathpar}

\begin{eqnarray*}
  match_{\equiv} (\quotep{P},\quotep{Q}) & := & P \equiv Q \\
  match_{\dagger}(\quotep{P},\quotep{Q}) & := & \forall R. P|Q \red^{*} R => R \red^{*} 0 \\
  match_{K}(\quotep{P},\quotep{Q}) & := & K \mbox{ for some context } K
\end{eqnarray*}

$u?(x)P | u!\langle Q \rangle \red P\{\quotep{Q}/x\}$

%We write $\wred$ for $\red^*$, and $P\red$ if $\exists Q $ such that $ P \red Q$.
We write $P\red$ if $\exists Q $ such that $ P \red Q$ and $P\not\red$, otherwise.

\section{Replication}

As mentioned before, it is known that replication (and hence
recursion) can be implemented in a higher-order process algebra
\cite{SangiorgiWalker}. As our first example of calculation with the
machinery thus far presented we give the construction explicitly in
the {\rhoc}.

\begin{eqnarray}
	D_{x} & := & \prefix{x}{y}{(\binpar{\outputp{x}{y}}{@{y}})} \nonumber\\
	\bangp_{x}{P} & := & \binpar{{x}!\langle{\binpar{D_{x}}{P}}\rangle}{D_{x}} \nonumber
\end{eqnarray}

\begin{eqnarray}
	\bangp_{x}{P} & & \nonumber\\
	=
	& {x}!\langle{(\prefix{x}{y}{(\outputp{x}{y} | @{y})) | P}}\rangle 
	      | \prefix{x}{y}{(\outputp{x}{y} | @{y})} & \nonumber\\
	\red
	& (\outputp{x}{y} | @{y})\substn{\quotep{(\prefix{x}{y}{(@{y} | \outputp{x}{y})) | P}}}{y} & \nonumber\\
	=
	& \outputp{x}{\quotep{(\prefix{x}{y}{(\outputp{x}{y} | @{y})) | P}}}
	  | {(\prefix{x}{y}{(\outputp{x}{y} | @{y})) | P}} & \nonumber\\
	\red
	& \ldots & \nonumber\\
	\red^*
	& P | P | \ldots & \nonumber
\end{eqnarray}

Of course, this encoding, as an implementation, runs away, unfolding
$\bangp{P}$ eagerly. A lazier and more implementable replication
operator, restricted to input-guarded processes, may be obtained as follows.

\begin{eqnarray}
\bangp{\prefix{u}{v}{P}} 
	:= 
	\binpar{\lift{x}{\prefix{u}{v}{(\binpar{D(x)}{P})}}}{D(x)} \nonumber
\end{eqnarray}

\begin{remark}
  Note that the lazier definition still does not deal with summation
  or mixed summation (i.e. sums over input and output). The reader is
  invited to construct definitions of replication that deal with these
  features. 

  Further, the definitions are parameterized in a name, $x$. Can you,
  gentle reader, make a definition that eliminates this parameter and
  guarantees no accidental interaction between the replication
  machinery and the process being replicated -- i.e. no accidental
  sharing of names used by the process to get its work done and the
  name(s) used by the replication to effect copying. This latter
  revision of the definition of replication is crucial to obtaining
  the expected identity $!!P \sim !P$.
\end{remark}

\begin{remark}\label{rem:paradoxical_combinator}
  The reader familiar with the lambda calculus will have noticed the
  similarity between $D$ and the paradoxical combinator.

  [Ed. note: the existence of this seems to suggest we have to be more
  restrictive on the set of processes and names we admit if we are to
  support no-cloning.]
\end{remark}

\subsubsection{Bisimulation}

The computational dynamics gives rise to another kind of equivalence,
the equivalence of computational behavior. As previously mentioned
this is typically captured \emph{via} some form of bisimulation.

% The notion we use in this paper is weak barbed bisimulation
% \cite{milner91polyadicpi}.

The notion we use in this paper is derived from weak barbed
bisimulation \cite{milner91polyadicpi}. 

\begin{definition}
An \emph{observation relation}, $\downarrow_{\mathcal N}$, over a set
of names, $\mathcal N$, is the smallest relation satisfying the rules
below.

\infrule[Out-barb]{y \in {\mathcal N}, \; x \nameeq y}
		  {\outputp{x}{v} \downarrow_{\mathcal N} x}
\infrule[Par-barb]{\mbox{$P\downarrow_{\mathcal N} x$ or $Q\downarrow_{\mathcal N} x$}}
		  {\binpar{P}{Q} \downarrow_{\mathcal N} x}

We write $P \Downarrow_{\mathcal N} x$ if there is $Q$ such that 
$P \wred Q$ and $Q \downarrow_{\mathcal N} x$.
\end{definition}

\begin{definition}
%\label{def.bbisim}
An  ${\mathcal N}$-\emph{barbed bisimulation} over a set of names, ${\mathcal N}$, is a symmetric binary relation 
${\mathcal S}_{\mathcal N}$ between agents such that $P\rel{S}_{\mathcal N}Q$ implies:
\begin{enumerate}
\item If $P \red P'$ then $Q \wred Q'$ and $P'\rel{S}_{\mathcal N} Q'$.
\item If $P\downarrow_{\mathcal N} x$, then $Q\Downarrow_{\mathcal N} x$.
\end{enumerate}
$P$ is ${\mathcal N}$-barbed bisimilar to $Q$, written
$P \wbbisim_{\mathcal N} Q$, if $P \rel{S}_{\mathcal N} Q$ for some ${\mathcal N}$-barbed bisimulation ${\mathcal S}_{\mathcal N}$.
\end{definition}

$\mathcal{R} \subseteq \pi \times \pi$

$P \mathcal{R} Q => \forall P'. P \red P' \Rightarrow \exists Q'. Q \red Q', P' \mathcal{R} Q'$

$P \vdash x \Rightarrow Q \vdash x$

\begin{mathpar}
  \inferrule*[lab=Out-barb]{x \nameeq y}{{y}!\langle{Q}\rangle \vdash x}
  \and
  \inferrule*[lab=Par-barb]{\mbox{$P\vdash x$ or $Q\vdash x$}}{\binpar{P}{Q} \vdash x}
\end{mathpar}

\subsubsection{Contexts}

One of the principle advantages of computational calculi like the
$\pi$-calculus is a well-defined notion of context,
contextual-equivalence and a correlation between
contextual-equivalence and notions of bisimulation. The notion of
context allows the decomposition of a process into (sub-)process and
its syntactic environment, its context. Thus, a context may be
thought of as a process with a ``hole'' (written $\Box$) in it. The
application of a context $M$ to a process $P$, written $M[P]$, is
tantamount to filling the hole in $M$ with $P$. In this paper we do
not need the full weight of this theory, but do make use of the notion
of context in the proof the main theorem. 

\begin{mathpar}
  \inferrule* [lab=summation] {} {{M_{M},M_{N}} \bc \Box \;|\; x.M_{A} \;|\; M_{M}+M_{N}}
  \and
  \inferrule* [lab=agent] {} {{M_{A}} \bc (\vec{x})M_{P} \;| \; \clift{P_0,\ldots,M_{P},\ldots,P_N}}
  \and \\
  \inferrule* [lab=process] {} {{M_{P}} \bc M_{N} \;| \;P|M_{P} }
\end{mathpar} 

\begin{mathpar}
  \inferrule* [lab=sychronization] {} {M_{N} \bc \Box \;|\; x?M_{F} \;|\; x!M_{C}}
  \and
  \inferrule* [lab=abstraction] {} {{M_{F}} \bc (x)M_{P} }
  \and
  \inferrule* [lab=concretion] {} {{M_{C}} \bc \langle M_{P} \rangle }
  \and \\
  \inferrule* [lab=process] {} {{M_{P}} \bc M_{N} \;| \;P|M_{P} }
\end{mathpar}

\begin{definition}[contextual application] Given a context $M$, and
  process $P$, we define the \emph{contextual application}, $M[P] :=
  M\{P/\Box\}$. That is, the contextual application of M to P is the
  substitution of $P$ for $\Box$ in $M$.
\end{definition}

$\meaningof{-} : L \to \mathcal{P}(\pi)$

\begin{mathpar}
  \inferrule* [lab=collection] {} {\meaningof{true} = \pi, \and \meaningof{~E} = \pi \setminus \meaningof{E}, \and \meaningof{E_{1} \& E_{2}} = \meaningof{E_{1}} \cap \meaningof{E_{2}}}
\end{mathpar}

\begin{mathpar}
  \inferrule* [lab=structure] {} {\meaningof{0} = \{ P \in \pi | P \equiv 0 \}, \and \\ \meaningof{E_1 | E_2} = \{ P \in \pi | P \equiv P_{1} | P_{2}, P_{1} \in \meaningof{E_{1}}, P_{2} \in \meaningof{E_2}\} }
\end{mathpar}

\begin{mathpar}
 \inferrule* [lab=behavior] {} {\meaningof{\langle a?b \rangle E} = \{ P \in \pi | P \equiv Q | u?(y)P', \\ \and \\\\ \and \\ \;\;\; u \in \meaningof{a}, \forall z.P'\{z/y\} \in \meaningof{E\{z/b\}}\}, \and \\ \meaningof{a!E} = \{ P \in \pi | P \equiv Q | x!\langle P' \rangle, x \in \meaningof{a} P' \in \meaningof{E}\} }
\end{mathpar}

\begin{mathpar}
 \inferrule* [lab=nominal] {} {\meaningof{\quotep{E}} = \{ \quotep{P} \in \quotep{\pi} | P \in \meaningof{E} \}, \and \meaningof{\quotep{P}} = \{ \quotep{Q} \in \quotep{\pi} | P \equiv Q \} \and \\ \meaningof{@\quotep{E}} = \{ P \in \pi | P \equiv @x, x \in \meaningof{E} \}}
\end{mathpar}

\begin{eqnarray*}
  \\
  \meaningof{-} : TS \to ST
\end{eqnarray*}

\begin{eqnarray*}
  \\
  L : TS \to ST
\end{eqnarray*}

\begin{eqnarray*}
  \\
  P \models E \iff P \in \meaningof{E}
\end{eqnarray*}

\begin{eqnarray*}
  P \approx_{L} Q \iff \forall E \in L. P \models E \iff Q \models E
\end{eqnarray*}

\begin{eqnarray*}
  P \approx_{K} Q
\end{eqnarray*}

\begin{eqnarray*}
  P \approx Q
\end{eqnarray*}

$\approx_{K} = \approx = \approx_{L}$

\subsubsection{Contextual duality}

Note that contexts extend the quotation operation to a family of
operations from processes to names. Given a context, $M$, we can
define a \emph{nominal context}, $\quotep{M}$ by $\quotep{M}[P] :=
\quotep{M[P]}$. To foreshadow what is to come we observe that these
operations enjoy a duality with processes very much like the duality
between vectors and maps from vectors to scalars.

Further, because the calculus is essentially higher-order, we have a
correspondence between contexts and processes. More specifically,
given a name $x$ and a context $M$ we can construct $M^{*}_{x}$ such
that 

\begin{mathpar}
  M^{*}_{x} | \lift{x}{P} \red M[P]
\end{mathpar}

namely,

\begin{mathpar}
  M^{*}_{x} := x?(u).M[\dropn{u}]
\end{mathpar}

The dependence of $M^{*}_{x}$ on a name makes it an abstraction, 

\begin{mathpar}
  M^{*} := (x)x?(u).M[\dropn{u}]
\end{mathpar}

\subsection{Additional notation}

It will sometimes be convenient to denote the process a name
quotes. We already have the notation $x = \quotep{P}$, but it will be
convenient to introduce an alternate notation, $\procn{x}$, when we
want to emphasize the connection to the use of the name. Note that, by
virtue of name equivalence, $\quotep{\procn{x}} \nameeq x$; so, the
notation is consistent with previous definitions.

Further, because names have structure it is possible to effect
substitutions on the basis of that structure. This means we need to
upgrade our notation for substitutions, which we accomplish by
adapting comprehension notation. Thus,

\begin{mathpar}
  P\{ y / x : x \in S \}
\end{mathpar}

is interpreted to mean the process derived from P by replacing (in a
capture-avoiding manner) each occurrence of $x$ in $S$ by $y$. For example,

\begin{mathpar}
  P\{ \quotep{\procn{x}|\procn{x}} / x : x \in \freenames{P} \}
\end{mathpar}

will replace each (occurrence) of a free name $x$ in $P$ by
$\quotep{\procn{x}|\procn{x}}$.

Also, we will avail ourselves of the notation $x^{L}$ and $x^{R}$ to
denote injections of a name into disjoint copies of the name
space. There are numerous ways to accomplish this. One example can be
found in \cite{MeredithR05}. This notation overloads to vectors of
names: $\vec{x}^{\pi} := (x_{i}^{\pi} \; : \; 0 \leq i < |\vec{x}| )$ where $\pi \in \{L,R\}$.

We also use $P^{\Box} := P|\Box$.

In \cite{MeredithR05} an interpretation of the new operator is
given. It turns out that there are several possible interpretations
all enjoying the requisite algebraic properties of the operator (see
\cite{milner91polyadicpi}). We will therefore make liberal use of
$(\nu\; \vec{x})P$.

% subsection the_syntax_and_semantics_of_the_notation_system (end)   

\section{Interpretation of QM}
\subsection{Supporting definitions}
\subsubsection{Multiplication}
\begin{mathpar}
  \quotep{Q} \cdot \quotep{R} := \quotep{Q|R}
  \and \\
  \quotep{Q} \cdot P := P\{ \quotep{Q|R} / \quotep{R} : \quotep{R} \in \freenames{P} \}
\end{mathpar}

\paragraph{Discussion}
The first line needs little explanation. The second line says that
each free name of the process is replaced with the multiplication of
that name by the scalar. Multiplication of a scalar (name) by a state
(process) results in a process all the names of which have been `moved
over' by parallel composition with the process the scalar
quotes. There is a subtlety that the bound names have to be
manipulated so that multiplied names aren't accidentally
captured. There are many ways to achieve this.

\begin{remark}\label{rem:multiplication_identities}
  The reader is invited to verify that for all $x,y,z \in \QProc$ and $P \in \Proc$
  \begin{mathpar}
    x \cdot \quotep{0} \equiv x 
    \and
    x \cdot y \equiv y \cdot x
    \and
    x \cdot (y \cdot z) \equiv (x \cdot y) \cdot z
    \and \\
    \quotep{0} \cdot P \equiv P
    \and \\
    x \cdot (y \cdot P) \equiv (x \cdot y) \cdot P
    \and \\
    x \cdot (P|Q) \equiv (x \cdot P) | (x \cdot Q)
    \and \\    
  \end{mathpar}
\end{remark}

\subsubsection{Tensor product}

We define a tensor product on processes by structural induction.

\paragraph{Tensor of sums} First note that all summations, including
$\pzero$ and sequence, can be written $\Sigma_{i} x_{i}.A_{i} +
\Sigma_{j} x_{j}.C_{j}$, where we have grouped input-guarded processes
together and output-guarded processes together.

Thus, we can define the tensor product of two summations, $N_{1}\otimes N_{2}$, where

\begin{mathpar}
  N_{1} := \Sigma_{i} x_{i}.A_{i} + \Sigma_{j} x_{j}.C_{j}
  \and
  N_{2} := \Sigma_{i'} y_{i'}.B_{i'} + \Sigma_{j'} y_{j'}.D_{j'} 
\end{mathpar}

as follows.

\begin{mathpar}
  \Sigma_{i} x_{i}.A_{i} + \Sigma_{j} x_{j}.C_{j} \otimes \Sigma_{i'}
  y_{i'}.B_{i'} + \Sigma_{j'} y_{j'}.D_{j'} 
  \and \\
  := \; \Sigma_{i} \Sigma_{i'} \quotep{\stackrel{\vee}{x_{i}}| \stackrel{\vee}{y_{i'}}}.(A_{i}\otimes B_{i'}) \; | \; \Sigma_{i'} \Sigma_{i} \quotep{\stackrel{\vee}{y_{i'}}|\stackrel{\vee}{x_{i}}}.(B_{i'}\otimes A_{i})
  \and
  \;\; | \;\; \Sigma_{j} \Sigma_{j'} \quotep{\stackrel{\vee}{x_{j}}|\stackrel{\vee}{y_{j'}}}.(A_{j}\otimes B_{j'}) \; | \; \Sigma_{j'} \Sigma_{j} \quotep{\stackrel{\vee}{y_{j'}}|\stackrel{\vee}{x_{j}}}.(B_{j'}\otimes A_{j})
\end{mathpar}

\begin{remark}
  Do we need to $x^{L}$ and $y^{R}$ for this construction as well?
\end{remark}

\paragraph{Tensor of parallel compositions} Next, we distribute tensor
over par.

\begin{mathpar}
  P_{1}|P_{2} \otimes Q_{1}|Q_{2} := (P_{1} \otimes Q_{1}) | (P_{1}
  \otimes Q_{2}) | (P_{2} \otimes Q_{1}) | (P_{2} \otimes Q_{2})
\end{mathpar}

\paragraph{Tensor with dropped names} We treat tensor of a
process with a dropped name as parallel composition.

\begin{mathpar}
  P \otimes \dropn{x} := P | \dropn{x}
\end{mathpar}

\paragraph{Tensor of agents}

Finally, we need to define tensor on agents. Note that the definition
of tensor on normal products only tensors inputs with inputs and
outputs with outputs. Thus, we only have to define the operation on
``homogeneous'' pairings.

\begin{mathpar}
  (\vec{x})P \otimes (\vec{y})Q
  \and \\
  := (x_{0}^{L}|y_{0}^{R},\ldots,x_{0}^{L}|y_{n}^{R},\ldots,x_{m}^{L}|y_{0}^{R},\ldots,x_{m}^{L}|y_{n}^R)(P\{ \vec{x}^{L}/\vec{x}\} \otimes Q \{ \vec{y}^{R}/\vec{y}\})
  \and \\
  \clift{\vec{P}} \otimes \clift{\vec{Q}}
  \and \\
  := \clift{P_{0}\otimes Q_{0},\ldots,P_{0}\otimes Q_{n},\ldots,P_{m}\otimes Q_{0},\ldots,P_{m}\otimes Q_{n}}
\end{mathpar}

\begin{remark}
  Observe that arities of tensored abstractions matches arities of
  tensored concretions if the original arities matched. Note also that
  the length of the arities corresponds to the increase in dimension
  we see in ordinary vector space tensor product.
\end{remark}

\begin{remark}
  Operationally, this definition distributes the tensor down to
  components ``linked'' by summation. Tensor over summation is
  intriguing in that it mixes names. Moreover, as a consequence of the
  way it mixes names we have the identities for all $x \in \QProc$ and
  $P,Q \in \Proc$

  \begin{mathpar}
    (x \cdot P) \otimes Q \equiv x \cdot (P \otimes Q) \equiv P \otimes (x \cdot Q)
    \and
    P \otimes \pzero \equiv P
  \end{mathpar}

  that the reader is invited to verify.
\end{remark}

\subsubsection{Annihilation}
\begin{mathpar}
  P^{\perp} := \{ Q | \forall R. P|Q \red^{*} R \Rightarrow R \red^{*} \pzero \}
  \and \\
  P^{\underline{\perp}} := \Sigma_{Q \in P^{\perp}} \quotep{Q}?(y).(\dropn{y}|Q) | \Sigma_{Q \in P^{\perp}} \quotep{Q}\clift{\Box}
\end{mathpar}

\paragraph{Discussion} The reader will note that $P^{\perp}$ is a
\emph{set} of processes, while $P^{\underline{\perp}}$ is a
\emph{context}. We call the set $P^{\perp}$ the \emph{annihilators} of
$P$. The parallel composition of a process in the annihilators of $P$
with $P$ will result in a process, the state space of which has all
paths eventually leading to $\pzero$. Execution may endure loops; but
under reasonable conditions of fairness (naturally guaranteed under
most notions of bisimulation) such a composite process cannot get
stuck in such a loop and will, eventually pop out and terminate.

The context $P^{\underline{\perp}}$ is ready and willing to ``take the
$P$ out of'' the process to which it is applied. It will effectively
transmit the code of the process to which it is applied to one of the
annihilators and run the process against it.

\subsubsection{Evaluation}
We fix $M$ a domain of fully abstract interpretation with an equality
coincident with bisimulation. We take $\meaningof{\cdot} : \Proc \to
M$ to be the map interpreting processes and $\nmeaningof{\cdot} : \M
\to Proc$ to be the map running the other way. Then we define

\begin{mathpar}
  \int P := \nmeaningof{\meaningof{P}}
\end{mathpar}

\paragraph{Discussion}
There are many fully abstract interpretations of Milner's
$\pi$-calculus. Any of them can be used as a basis for interpreting
the reflective calculus here. Equipped with such a domain it is
largely a matter of grinding through to check that the Yoneda
construction for the normalization-by-evaluation program can be
extended to this setting.

\begin{remark}
  The reader is invited to verify that $\int (P^{\underline{\perp}}[P]) = 0$.
\end{remark}

\subsection{Quantum mechanics}

Table \ref{tbl:core_qm_op_defns} gives the core operational definitions

\begin{table}[htp]\label{tbl:core_qm_op_defns}
  \center{
    \fbox{
      \begin{tabular}{c|c}
        quantum mechanics & process calculus \\
        \hline
        scalar & $x := \quotep{P}$ \\
        state vector & $\state{P} := P$ \\
        dual & $\state{P}^{*} := \event{P^{\underline{\perp}}} := \quotep{P^{\underline{\perp}}}[-]$ \\
        matrix & $ \Sigma_{\alpha} \state{P_{\alpha}}x_{\alpha}\event{Q_{\alpha}}$ \\
        vector addition & $\state{P} + \state{Q} := \state{P | Q}$ \\
        tensor product & $\state{P} \otimes \state{Q} := \state{P \otimes Q}$ \\
        inner product & $\innerprod{P}{Q} := \quotep{\int P^{\underline{\perp}}[Q]}$ \\
      \end{tabular}
    }
  }
  \caption{QM - operational definitions}
\end{table}

where

\begin{mathpar}
  \prmatrix{P}{Q} := \fprmatrix{P}{\quotep{\pzero}}{Q}
  \and
  \fprmatrix{P}{x}{Q} := (\state{P},x,\event{Q})
  \and
  (\fprmatrix{P}{x}{Q})(\state{R}) := x \cdot \innerprod{Q}{R} \cdot \state{P}
  \and
  (\fprmatrix{P}{x}{Q})(\event{R}) := x \cdot \innerprod{R}{P} \cdot \event{Q}
\end{mathpar}

\paragraph{Discussion}
As promised: vectors (aka states) are represented as processes; duals
as contextual duals; inner product definition should be compared with
standard inner product definition for ....

\begin{remark}
  Assuming $\int (P^{\underline{\perp}}[P]) = 0$, the reader is
  invited to verify that $(\fprmatrix{P}{x}{P})(\state{P}) = x \cdot \state{P}$.
\end{remark}

\begin{remark}
  The reader is invited to verify that $\innerprod{P}{Q}$ could
  equally well have been written $\quotep{\int \stackrel{\vee}{x}}$
  where $x = \event{P^{\underline{\perp}}}(Q)$.

  One of the motivations for this remark is that there is another way
  to factor these operations. We could package up evaluation in the dual:

  \begin{mathpar}
    \state{P}^{*} := \event{\int P^{\underline{\perp}}} := \quotep{\int P^{\underline{\perp}}}[-]
  \end{mathpar}

  and then have inner product defined by
  
  \begin{mathpar}
    \innerprod{P}{Q} := \event{P}(Q)
  \end{mathpar}

  Hopefully, experience with the calculations will provide guidance on
  the best factoring.
\end{remark}

\begin{remark}
  Assuming $\int (P^{\underline{\perp}}[P]) = 0$, the reader is
  invited to verify that $\forall P,Q. (\prmatrix{0}{Q})(\state{0}) =
  \state{0}$ and dually $(\prmatrix{P}{0})(\event{0}) = \event{0}$.
\end{remark}

\begin{remark}
  i'm a little worried that i don't (yet) have proper support for
  complex conjugacy. But, the observation above may give us a
  clue. According to Abramsky, it must be the case that the scalars
  are iso to the homset of the identity for the tensor -- which the
  observation above characterizes. 

  For now, we will simply bookmark the notion with $\overline{x}$.
\end{remark}

\subsubsection{Adjointness}

We need to give a definition of $(\cdot)^{\dagger}$ for matrices. The
obvious candidate definition is
\begin{mathpar}
(\Sigma_{\alpha}\fprmatrix{P_{\alpha}}{x_{\alpha}}{Q_{\alpha}})^{\dagger}
= \Sigma_{\alpha}\fprmatrix{(Q_{\alpha}^{\underline{\perp}})^{*}}{\overline{x}_{\alpha}}{P_{\alpha}^{\underline{\perp}}} 
\end{mathpar}

But, $(Q_{\alpha}^{\underline{\perp}})^{*}$ requires a name along
which to communicate the process to achieve the context application.

\subsubsection{Basis for a basis}
If processes label states and ``addition'' of states (a.k.a. vector
addition) is interpreted as parallel composition, what corresponds to
notions of linear independence and basis? Here, we recall that Yoshida
has developed a set of \emph{combinators} for an asynchronous verison
of Milner's $\pi$-calculus. These are a finite set of processes such
any process can be expressed as parallel composition of these
combinators together with liberal uses of the new operator and
replication. We can simply give a translation of these into the
present calculus and have reasonable expectation that the property
carries over. That is, that the resultant set allows to express all
processes via parallel composition. Note, however, that there is no
new operator or replication in this calculus. As a result, we expect
that the corresponding set is actually infinite. That is, we expect
that the space is actually infinite dimensional.

\begin{remark}
  The attentive reader may be a bit concerned. Certainly, the
  collection $S$, $K$ and $I$ is a finite set of
  combinators. Shouldn't we expect to see a finite set of combinators
  for an effectively equivalent system? i am very sympathetic to this
  critique and feel it warrants full attention. On the other hand, i
  also have in mind the following analogy. The natural numbers, as a
  monoid under addition, has exactly $1$ generator, while the natural
  numbers, as a monoid under multiplication, has countably many
  generators (the primes). We observe that the application of the
  lambda calculus is much less resource sensitive than the parallel
  composition of the $\pi$-calculus. Could it be the case that we have
  an analogy of the form
  
  \begin{mathpar}
    m + n : MN :: m*n : M|N
  \end{mathpar}

  giving a similar blow up in the set of ``primes''?  This is such a
  wonderful thought that, even if it's not true, i think it's worth
  writing down.
\end{remark}
 

\documentclass[12pt]{llncs}
%\documentclass{jktr}

\usepackage[pdftex]{hyperref}                   
\usepackage {listings}
\usepackage {mathpartir}
\usepackage{bcprules}
%\usepackage{listings}
                       
\usepackage{graphicx} 
%\usepackage[margins=2.5cm,nohead,nofoot]{geometry}
%\usepackage{geometry}
\usepackage{amsfonts}
\usepackage{amstext}
\usepackage{latexsym}
\usepackage{amssymb}
\usepackage{color}


%\include{myPreamble}
\include{qm2pi.local} 

%\ifpdf
%\usepackage[pdftex]{graphicx}
%\else
%\usepackage{graphicx}
%\fi

 % \ifpdf
%  \usepackage{pdfsync}
%  \if


%\title{Brief Article}
%\author{David F. Snyder}
%\author{L.G. Meredith}

%\address{Dept. of Math., Texas State University--San Marcos, San Marcos, TX 78666}
       
\pagestyle{empty}


\begin{document}

\lstset{language=[Objective]Caml,frame=shadowbox}

\input{qm2pi.front}

% section front matter (end)

\input{qm2pi.intro} 
 
% section introduction (end)

% \input{qm2pi.knotations} 

% section notation (end)

\input{qm2pi.process.calculi} 

% section concurrent_process_calculi_and_spatial_logics_ (end)
    
%\input{qm2pi.knots2pi} 

%\input{qm2pi.trefoil} 

%\input{qm2pi.mainthm} 

% subsection basic_interpretation (end)

%\input{qm2pi.rho.presentation} 
\subsection{The syntax and semantics of the notation system}\label{sub:the_syntax_and_semantics_of_the_notation_system} % (fold)

We now summarize a technical presentation of the calculus that
embodies our theory of dynamics. The typical presentation of such a
calculus follows the style of giving generators and relations on
them. The grammar, below, describing term constructors, freely
generates the set of processes, $\Proc$. This set is then quotiented
by a relation known as structural congruence and it is over this set
that the notion of dynamics is expressed. This presentation is
essentially that of \cite{MeredithR05} with the addition of
polyadicity and summation. For readability we have relegated some of
the technical subtleties to an appendix.

\subsubsection{Process grammar}\label{subsub:process_grammar}

\begin{mathpar}
  \inferrule* [lab=synchronization] {} {{M} \bc \pzero \;|\; x?F \;|\; x!C }
  \and
  \inferrule* [lab=abstraction] {} {{F} \bc (x)P}
  \and
  \inferrule* [lab=concretion] {} {{C} \bc \langle Q \rangle}
  \and
  \inferrule* [lab=process] {} {{P,Q} \bc M \;| \;P|Q \;|\; @{x}}
  \and
  \inferrule* [lab=name] {} {{x} \bc \quotep{P}}
\end{mathpar} 

Note that $\vec{x}$ (resp. $\vec{P}$) denotes a vector of names
(resp. processes) of length $|\vec{x}|$ (resp. $|\vec{P}|$). We adopt
the following useful abbreviations.

\begin{mathpar}
   x?(\vec{y}).P := x.(\vec{y})P \and  x\clift{\vec{P}} := x.\clift{\vec{P}}
   \and x!(y) := \lift{x}{\dropn{y}}
   \and \Pi_{i=0}^{n-1}P_i := P_0 | \ldots | P_{n-1}
\end{mathpar}

\subsubsection{Structural congruence}

\paragraph{Free and bound names and alpha-equivalence.} At the
core of structural equivalence is alpha-equivalence which identifies
process that are the same up to a change of variable. Formally, we
recognize the distinction between free and bound names. The free names
of a process, $\freenames{P}$, may be calculated recursively as
follows:

\begin{mathpar}
\freenames{\pzero} := \emptyset
  \and \\
  \freenames{x?(y).P} := \{ x \} \cup (\freenames{P} \setminus \{ y \})
  \and 
  \freenames{x!\langle P \rangle} := \{ x \} \cup \{ P \} 
  \and \\
  \freenames{P|Q} := \freenames{P} \cup \freenames{Q}
  \and \\
  \freenames{@{x}} := \{ x \}
\end{mathpar}

$\pi$
$\quotep{\pi}$

$\freenames{-} : \pi \to \mathcal{P}(\quotep{\pi})$

\begin{eqnarray*}
  \freenames{\pzero} & := & \emptyset \\
  \freenames{x?(y).P} & := & \{ x \} \cup (\freenames{P} \setminus \{ y \}) \\
  \freenames{x!\langle P \rangle} & := & \{ x \} \cup \{ P \} \\
  \freenames{P|Q} & := & \freenames{P} \cup \freenames{Q} \\
  \freenames{\dropn{x}} & := & \{ x \}
\end{eqnarray*}

The bound names of a process, $\boundnames{P}$, are those names occurring in $P$
that are not free. For example, in $x?(y).0$, the name $x$ is free, while $y$ is bound.

\begin{mathpar}
  \inferrule* [lab=monoidal-laws] {} { P|Q \equiv Q|P \and P|0 \equiv P \and P|(Q|R) \equiv (P|Q)|R }
\end{mathpar}

\begin{mathpar}
  \inferrule* [lab=alpha-equivalence] {} { (x)P \equiv (y)P\{y/x\} \and y \not\in \freenames{P} }
\end{mathpar}

\begin{definition}
Then two processes, $P,Q$, are alpha-equivalent if $P = Q\{\vec{y}/\vec{x}\}$ for
some $\vec{x} \in \boundnames{Q},\vec{y} \in \boundnames{P}$, where $Q\{\vec{y}/\vec{x}\}$
denotes the capture-avoiding substitution of $\vec{y}$ for $\vec{x}$ in $Q$.
\end{definition}

\begin{definition}
  The {\em structural congruence} \cite{SangiorgiWalker} , $\equiv$,
  between processes is the least congruence containing
  alpha-equivalence, satisfying the abelian monoid laws
  (associativity, commutativity and $\pzero$ as identity) for parallel
  composition $|$ and for summation $+$.
\end{definition}

\subsection{Name equivalence}

We take name equivalence, written $\nameeq$, to be the smallest
equivalence relation generated by the following rules.

\begin{mathpar}
\inferrule*[lab=Quote-drop]
{ }
{ \quotep{@{x}} \nameeq x }

\inferrule*[lab=Struct-equiv]
{ P \scong Q }
{ \quotep{P} \nameeq \quotep{Q} }
\end{mathpar}

The astute reader will have noticed that the mutual recursion of names
and processes imposes a mutual recursion on alpha-equivalence and
structural equivalence via name-equivalence. Fortunately, all of this
works out pleasantly and we may calculate in the natural way, free of
concern. The reader interested in the details is referred to the
appendix \ref{appendix:rho_details}.

\subsection{Substitution}

We use $\Proc$ for the set of processes, $\QProc$ for the set of
names, and $\id{\{}\vec{y} / \vec{x} \id{\}}$ to denote partial maps,
$s : \QProc \rightarrow \QProc$. A map, $s$ lifts, uniquely, to a map
on process terms, $\widehat{s} : \Proc \rightarrow \Proc$ by the
following equations.

\begin{mathpar}
  (0) \psubstp{Q}{P} := 0 \\
  (R \juxtap S) \psubstp{Q}{P}
  :=    
  (R)\psubstp{Q}{P} \juxtap (S) \psubstp{Q}{P} \\
  (x?(y).R) \psubstp{Q}{P}    
  :=    
  (x)\substp{Q}{P} (z)\concat( (R \psubstn{z}{y}) \psubstp{Q}{P} ) \\
  (\lift{x}{R}) \psubstp{Q}{P}  
  :=
  \lift{(x)\substp{Q}{P}}{ R \psubstp{Q}{P} } \\
%   (\dropn{x})  \psubstp{Q}{P}       
%   := 
%   \left\{ 
%     \begin{array}{ccc} 
%       \dropn{\quotep{Q}} & & x \nameeq \quotep{P} \\
%       \dropn{x} & & otherwise \\
%     \end{array}
%   \right. 
  (\dropn{x})  \psubstp{Q}{P}       
  := 
  \left\{ 
    \begin{array}{ccc} 
      Q & & x \nameeq \quotep{P} \\
      \dropn{x} & & otherwise \\
    \end{array}
  \right.
\end{mathpar}
 

where

\begin{eqnarray}
  (x)\id{\{} \lpquote Q \rpquote / \lpquote P \rpquote \id{\}}            = 
  \left\{ 
    \begin{array}{ccc}
      \lpquote Q \rpquote & & x \nameeq \lpquote P \rpquote \\
      x & & otherwise \\
    \end{array}
  \right. \nonumber
\end{eqnarray}

and $z$ is chosen distinct from $\quotep{P}$, $\quotep{Q}$, the free
names in $Q$, and all the names in $R$. Our $\alpha$-equivalence will
be built in the standard way from this substitution.

\begin{remark}\label{rem:no_self_referential_names}
  One consequence of these definitions is that $\forall P. \quotep{P}
  \not\in \freenames{P}$.
\end{remark}

\subsection{ Dynamic quote: an example }

Anticipating something of what's to come, consider applying the
substitution, $\widehat{\id{\{}u / z \id{\}}}$, to the following pair
of processes, $\lift{w}{y!(z)}$ and $w[ \lpquote y!(z) \rpquote ]$.

\begin{eqnarray}
	\lift{w}{y!(z)}\widehat{\id{\{}u / z \id{\}}}
		& = &
		\lift{w}{y!(u)} \nonumber\\
	w[ \lpquote y!(z) \rpquote ] \widehat{ \id{\{}u / z \id{\}} }
		& = &
		w[ \lpquote y!(z) \rpquote ] \nonumber
\end{eqnarray}

Because the body of the process between quotes is impervious to
substitution, we get radically different answers. In fact, by
examining the first process in an input context,
e.g. $x?(z).\lift{w}{y!(z)}$, we see that the process under the lift
operator may be shaped by prefixed inputs binding a name inside it. In
this sense, the lift operator will be seen as a way to dynamically
construct processes before reifying them as names.

Finally equipped with these standard features we can present the
dynamics of the calculus.

\subsubsection{Operational semantics} 

Finally, we introduce the computational dynamics. What marks these
algebras as distinct from other more traditionally studied algebraic
structures, e.g. vector spaces or polynomial rings, is the manner in
which dynamics is captured. In traditional structures, dynamics is typically
expressed through morphisms between such structures, as in linear maps
between vector spaces or morphisms between rings. In algebras
associated with the semantics of computation, the dynamics is
expressed as part of the algebraic structure itself, through a
reduction reduction relation typically denoted by $\red$. Below, we
give a recursive presentation of this relation for the calculus used
in the encoding.

$\red \subseteq \pi \times \pi$
$\red : \pi \to \mathcal{P}(\pi)$

\begin{mathpar}
  \inferrule* [lab=Comm] { \textsf{match}( x_{src}, x_{trgt} ) } { x_{trgt}?(y)P \; | \; x_{src}!\langle {Q} \rangle \red P\{\quotep{Q}/y}\} }
  \and \\
  \inferrule* [lab=Par] {{P} \red {P}'} {{{P} | {Q}} \red {{P}' | {Q}}}
  \and
  \inferrule* [lab=Equiv]{{{P} \scong {P}'} \andalso {{P}' \red {Q}'} \andalso {{Q}' \scong {Q}}}{{P} \red {Q}}
\end{mathpar}

\begin{eqnarray*}
  match_{\equiv} (\quotep{P},\quotep{Q}) & := & P \equiv Q \\
  match_{\dagger}(\quotep{P},\quotep{Q}) & := & \forall R. P|Q \red^{*} R => R \red^{*} 0 \\
  match_{K}(\quotep{P},\quotep{Q}) & := & K \mbox{ for some context } K
\end{eqnarray*}

$u?(x)P | u!\langle Q \rangle \red P\{\quotep{Q}/x\}$

%We write $\wred$ for $\red^*$, and $P\red$ if $\exists Q $ such that $ P \red Q$.
We write $P\red$ if $\exists Q $ such that $ P \red Q$ and $P\not\red$, otherwise.

\section{Replication}

As mentioned before, it is known that replication (and hence
recursion) can be implemented in a higher-order process algebra
\cite{SangiorgiWalker}. As our first example of calculation with the
machinery thus far presented we give the construction explicitly in
the {\rhoc}.

\begin{eqnarray}
	D_{x} & := & \prefix{x}{y}{(\binpar{\outputp{x}{y}}{@{y}})} \nonumber\\
	\bangp_{x}{P} & := & \binpar{{x}!\langle{\binpar{D_{x}}{P}}\rangle}{D_{x}} \nonumber
\end{eqnarray}

\begin{eqnarray}
	\bangp_{x}{P} & & \nonumber\\
	=
	& {x}!\langle{(\prefix{x}{y}{(\outputp{x}{y} | @{y})) | P}}\rangle 
	      | \prefix{x}{y}{(\outputp{x}{y} | @{y})} & \nonumber\\
	\red
	& (\outputp{x}{y} | @{y})\substn{\quotep{(\prefix{x}{y}{(@{y} | \outputp{x}{y})) | P}}}{y} & \nonumber\\
	=
	& \outputp{x}{\quotep{(\prefix{x}{y}{(\outputp{x}{y} | @{y})) | P}}}
	  | {(\prefix{x}{y}{(\outputp{x}{y} | @{y})) | P}} & \nonumber\\
	\red
	& \ldots & \nonumber\\
	\red^*
	& P | P | \ldots & \nonumber
\end{eqnarray}

Of course, this encoding, as an implementation, runs away, unfolding
$\bangp{P}$ eagerly. A lazier and more implementable replication
operator, restricted to input-guarded processes, may be obtained as follows.

\begin{eqnarray}
\bangp{\prefix{u}{v}{P}} 
	:= 
	\binpar{\lift{x}{\prefix{u}{v}{(\binpar{D(x)}{P})}}}{D(x)} \nonumber
\end{eqnarray}

\begin{remark}
  Note that the lazier definition still does not deal with summation
  or mixed summation (i.e. sums over input and output). The reader is
  invited to construct definitions of replication that deal with these
  features. 

  Further, the definitions are parameterized in a name, $x$. Can you,
  gentle reader, make a definition that eliminates this parameter and
  guarantees no accidental interaction between the replication
  machinery and the process being replicated -- i.e. no accidental
  sharing of names used by the process to get its work done and the
  name(s) used by the replication to effect copying. This latter
  revision of the definition of replication is crucial to obtaining
  the expected identity $!!P \sim !P$.
\end{remark}

\begin{remark}\label{rem:paradoxical_combinator}
  The reader familiar with the lambda calculus will have noticed the
  similarity between $D$ and the paradoxical combinator.

  [Ed. note: the existence of this seems to suggest we have to be more
  restrictive on the set of processes and names we admit if we are to
  support no-cloning.]
\end{remark}

\subsubsection{Bisimulation}

The computational dynamics gives rise to another kind of equivalence,
the equivalence of computational behavior. As previously mentioned
this is typically captured \emph{via} some form of bisimulation.

% The notion we use in this paper is weak barbed bisimulation
% \cite{milner91polyadicpi}.

The notion we use in this paper is derived from weak barbed
bisimulation \cite{milner91polyadicpi}. 

\begin{definition}
An \emph{observation relation}, $\downarrow_{\mathcal N}$, over a set
of names, $\mathcal N$, is the smallest relation satisfying the rules
below.

\infrule[Out-barb]{y \in {\mathcal N}, \; x \nameeq y}
		  {\outputp{x}{v} \downarrow_{\mathcal N} x}
\infrule[Par-barb]{\mbox{$P\downarrow_{\mathcal N} x$ or $Q\downarrow_{\mathcal N} x$}}
		  {\binpar{P}{Q} \downarrow_{\mathcal N} x}

We write $P \Downarrow_{\mathcal N} x$ if there is $Q$ such that 
$P \wred Q$ and $Q \downarrow_{\mathcal N} x$.
\end{definition}

\begin{definition}
%\label{def.bbisim}
An  ${\mathcal N}$-\emph{barbed bisimulation} over a set of names, ${\mathcal N}$, is a symmetric binary relation 
${\mathcal S}_{\mathcal N}$ between agents such that $P\rel{S}_{\mathcal N}Q$ implies:
\begin{enumerate}
\item If $P \red P'$ then $Q \wred Q'$ and $P'\rel{S}_{\mathcal N} Q'$.
\item If $P\downarrow_{\mathcal N} x$, then $Q\Downarrow_{\mathcal N} x$.
\end{enumerate}
$P$ is ${\mathcal N}$-barbed bisimilar to $Q$, written
$P \wbbisim_{\mathcal N} Q$, if $P \rel{S}_{\mathcal N} Q$ for some ${\mathcal N}$-barbed bisimulation ${\mathcal S}_{\mathcal N}$.
\end{definition}

$\mathcal{R} \subseteq \pi \times \pi$

$P \mathcal{R} Q => \forall P'. P \red P' \Rightarrow \exists Q'. Q \red Q', P' \mathcal{R} Q'$

$P \vdash x \Rightarrow Q \vdash x$

\begin{mathpar}
  \inferrule*[lab=Out-barb]{x \nameeq y}{{y}!\langle{Q}\rangle \vdash x}
  \and
  \inferrule*[lab=Par-barb]{\mbox{$P\vdash x$ or $Q\vdash x$}}{\binpar{P}{Q} \vdash x}
\end{mathpar}

\subsubsection{Contexts}

One of the principle advantages of computational calculi like the
$\pi$-calculus is a well-defined notion of context,
contextual-equivalence and a correlation between
contextual-equivalence and notions of bisimulation. The notion of
context allows the decomposition of a process into (sub-)process and
its syntactic environment, its context. Thus, a context may be
thought of as a process with a ``hole'' (written $\Box$) in it. The
application of a context $M$ to a process $P$, written $M[P]$, is
tantamount to filling the hole in $M$ with $P$. In this paper we do
not need the full weight of this theory, but do make use of the notion
of context in the proof the main theorem. 

\begin{mathpar}
  \inferrule* [lab=summation] {} {{M_{M},M_{N}} \bc \Box \;|\; x.M_{A} \;|\; M_{M}+M_{N}}
  \and
  \inferrule* [lab=agent] {} {{M_{A}} \bc (\vec{x})M_{P} \;| \; \clift{P_0,\ldots,M_{P},\ldots,P_N}}
  \and \\
  \inferrule* [lab=process] {} {{M_{P}} \bc M_{N} \;| \;P|M_{P} }
\end{mathpar} 

\begin{mathpar}
  \inferrule* [lab=sychronization] {} {M_{N} \bc \Box \;|\; x?M_{F} \;|\; x!M_{C}}
  \and
  \inferrule* [lab=abstraction] {} {{M_{F}} \bc (x)M_{P} }
  \and
  \inferrule* [lab=concretion] {} {{M_{C}} \bc \langle M_{P} \rangle }
  \and \\
  \inferrule* [lab=process] {} {{M_{P}} \bc M_{N} \;| \;P|M_{P} }
\end{mathpar}

\begin{definition}[contextual application] Given a context $M$, and
  process $P$, we define the \emph{contextual application}, $M[P] :=
  M\{P/\Box\}$. That is, the contextual application of M to P is the
  substitution of $P$ for $\Box$ in $M$.
\end{definition}

$\meaningof{-} : L \to \mathcal{P}(\pi)$

\begin{mathpar}
  \inferrule* [lab=collection] {} {\meaningof{true} = \pi, \and \meaningof{~E} = \pi \setminus \meaningof{E}, \and \meaningof{E_{1} \& E_{2}} = \meaningof{E_{1}} \cap \meaningof{E_{2}}}
\end{mathpar}

\begin{mathpar}
  \inferrule* [lab=structure] {} {\meaningof{0} = \{ P \in \pi | P \equiv 0 \}, \and \\ \meaningof{E_1 | E_2} = \{ P \in \pi | P \equiv P_{1} | P_{2}, P_{1} \in \meaningof{E_{1}}, P_{2} \in \meaningof{E_2}\} }
\end{mathpar}

\begin{mathpar}
 \inferrule* [lab=behavior] {} {\meaningof{\langle a?b \rangle E} = \{ P \in \pi | P \equiv Q | u?(y)P', \\ \and \\\\ \and \\ \;\;\; u \in \meaningof{a}, \forall z.P'\{z/y\} \in \meaningof{E\{z/b\}}\}, \and \\ \meaningof{a!E} = \{ P \in \pi | P \equiv Q | x!\langle P' \rangle, x \in \meaningof{a} P' \in \meaningof{E}\} }
\end{mathpar}

\begin{mathpar}
 \inferrule* [lab=nominal] {} {\meaningof{\quotep{E}} = \{ \quotep{P} \in \quotep{\pi} | P \in \meaningof{E} \}, \and \meaningof{\quotep{P}} = \{ \quotep{Q} \in \quotep{\pi} | P \equiv Q \} \and \\ \meaningof{@\quotep{E}} = \{ P \in \pi | P \equiv @x, x \in \meaningof{E} \}}
\end{mathpar}

\begin{eqnarray*}
  \\
  \meaningof{-} : TS \to ST
\end{eqnarray*}

\begin{eqnarray*}
  \\
  L : TS \to ST
\end{eqnarray*}

\begin{eqnarray*}
  \\
  P \models E \iff P \in \meaningof{E}
\end{eqnarray*}

\begin{eqnarray*}
  P \approx_{L} Q \iff \forall E \in L. P \models E \iff Q \models E
\end{eqnarray*}

\begin{eqnarray*}
  P \approx_{K} Q
\end{eqnarray*}

\begin{eqnarray*}
  P \approx Q
\end{eqnarray*}

$\approx_{K} = \approx = \approx_{L}$

\subsubsection{Contextual duality}

Note that contexts extend the quotation operation to a family of
operations from processes to names. Given a context, $M$, we can
define a \emph{nominal context}, $\quotep{M}$ by $\quotep{M}[P] :=
\quotep{M[P]}$. To foreshadow what is to come we observe that these
operations enjoy a duality with processes very much like the duality
between vectors and maps from vectors to scalars.

Further, because the calculus is essentially higher-order, we have a
correspondence between contexts and processes. More specifically,
given a name $x$ and a context $M$ we can construct $M^{*}_{x}$ such
that 

\begin{mathpar}
  M^{*}_{x} | \lift{x}{P} \red M[P]
\end{mathpar}

namely,

\begin{mathpar}
  M^{*}_{x} := x?(u).M[\dropn{u}]
\end{mathpar}

The dependence of $M^{*}_{x}$ on a name makes it an abstraction, 

\begin{mathpar}
  M^{*} := (x)x?(u).M[\dropn{u}]
\end{mathpar}

\subsection{Additional notation}

It will sometimes be convenient to denote the process a name
quotes. We already have the notation $x = \quotep{P}$, but it will be
convenient to introduce an alternate notation, $\procn{x}$, when we
want to emphasize the connection to the use of the name. Note that, by
virtue of name equivalence, $\quotep{\procn{x}} \nameeq x$; so, the
notation is consistent with previous definitions.

Further, because names have structure it is possible to effect
substitutions on the basis of that structure. This means we need to
upgrade our notation for substitutions, which we accomplish by
adapting comprehension notation. Thus,

\begin{mathpar}
  P\{ y / x : x \in S \}
\end{mathpar}

is interpreted to mean the process derived from P by replacing (in a
capture-avoiding manner) each occurrence of $x$ in $S$ by $y$. For example,

\begin{mathpar}
  P\{ \quotep{\procn{x}|\procn{x}} / x : x \in \freenames{P} \}
\end{mathpar}

will replace each (occurrence) of a free name $x$ in $P$ by
$\quotep{\procn{x}|\procn{x}}$.

Also, we will avail ourselves of the notation $x^{L}$ and $x^{R}$ to
denote injections of a name into disjoint copies of the name
space. There are numerous ways to accomplish this. One example can be
found in \cite{MeredithR05}. This notation overloads to vectors of
names: $\vec{x}^{\pi} := (x_{i}^{\pi} \; : \; 0 \leq i < |\vec{x}| )$ where $\pi \in \{L,R\}$.

We also use $P^{\Box} := P|\Box$.

In \cite{MeredithR05} an interpretation of the new operator is
given. It turns out that there are several possible interpretations
all enjoying the requisite algebraic properties of the operator (see
\cite{milner91polyadicpi}). We will therefore make liberal use of
$(\nu\; \vec{x})P$.

% subsection the_syntax_and_semantics_of_the_notation_system (end)   

\input{qm2pi.qmops} 

\input{qm2pi.sterngerlach} 

\input{qm2pi.metric} 

% section concurrent_process_calculi (end)

%\input{qm2pi.proofsketch}

% section proof sketch (end)

%\input{qm2pi.slviaknots} 

% section spatial logic via knots (end)

\input{qm2pi.conclusion}

% section conclusion (end)

%\input{qm2pi.dtcodes} 

% section wiring algorithm (end)

\input{qm2pi.ack} 

% section acknowledgments (end)

\newpage


\bibliographystyle{plain}   
\bibliography{../../biblios/main.bib}

\input{qm2pi.rhodetails}

\end{document}

 

\documentclass[12pt]{llncs}
%\documentclass{jktr}

\usepackage[pdftex]{hyperref}                   
\usepackage {listings}
\usepackage {mathpartir}
\usepackage{bcprules}
%\usepackage{listings}
                       
\usepackage{graphicx} 
%\usepackage[margins=2.5cm,nohead,nofoot]{geometry}
%\usepackage{geometry}
\usepackage{amsfonts}
\usepackage{amstext}
\usepackage{latexsym}
\usepackage{amssymb}
\usepackage{color}


%\include{myPreamble}
\include{qm2pi.local} 

%\ifpdf
%\usepackage[pdftex]{graphicx}
%\else
%\usepackage{graphicx}
%\fi

 % \ifpdf
%  \usepackage{pdfsync}
%  \if


%\title{Brief Article}
%\author{David F. Snyder}
%\author{L.G. Meredith}

%\address{Dept. of Math., Texas State University--San Marcos, San Marcos, TX 78666}
       
\pagestyle{empty}


\begin{document}

\lstset{language=[Objective]Caml,frame=shadowbox}

\input{qm2pi.front}

% section front matter (end)

\input{qm2pi.intro} 
 
% section introduction (end)

% \input{qm2pi.knotations} 

% section notation (end)

\input{qm2pi.process.calculi} 

% section concurrent_process_calculi_and_spatial_logics_ (end)
    
%\input{qm2pi.knots2pi} 

%\input{qm2pi.trefoil} 

%\input{qm2pi.mainthm} 

% subsection basic_interpretation (end)

%\input{qm2pi.rho.presentation} 
\subsection{The syntax and semantics of the notation system}\label{sub:the_syntax_and_semantics_of_the_notation_system} % (fold)

We now summarize a technical presentation of the calculus that
embodies our theory of dynamics. The typical presentation of such a
calculus follows the style of giving generators and relations on
them. The grammar, below, describing term constructors, freely
generates the set of processes, $\Proc$. This set is then quotiented
by a relation known as structural congruence and it is over this set
that the notion of dynamics is expressed. This presentation is
essentially that of \cite{MeredithR05} with the addition of
polyadicity and summation. For readability we have relegated some of
the technical subtleties to an appendix.

\subsubsection{Process grammar}\label{subsub:process_grammar}

\begin{mathpar}
  \inferrule* [lab=synchronization] {} {{M} \bc \pzero \;|\; x?F \;|\; x!C }
  \and
  \inferrule* [lab=abstraction] {} {{F} \bc (x)P}
  \and
  \inferrule* [lab=concretion] {} {{C} \bc \langle Q \rangle}
  \and
  \inferrule* [lab=process] {} {{P,Q} \bc M \;| \;P|Q \;|\; @{x}}
  \and
  \inferrule* [lab=name] {} {{x} \bc \quotep{P}}
\end{mathpar} 

Note that $\vec{x}$ (resp. $\vec{P}$) denotes a vector of names
(resp. processes) of length $|\vec{x}|$ (resp. $|\vec{P}|$). We adopt
the following useful abbreviations.

\begin{mathpar}
   x?(\vec{y}).P := x.(\vec{y})P \and  x\clift{\vec{P}} := x.\clift{\vec{P}}
   \and x!(y) := \lift{x}{\dropn{y}}
   \and \Pi_{i=0}^{n-1}P_i := P_0 | \ldots | P_{n-1}
\end{mathpar}

\subsubsection{Structural congruence}

\paragraph{Free and bound names and alpha-equivalence.} At the
core of structural equivalence is alpha-equivalence which identifies
process that are the same up to a change of variable. Formally, we
recognize the distinction between free and bound names. The free names
of a process, $\freenames{P}$, may be calculated recursively as
follows:

\begin{mathpar}
\freenames{\pzero} := \emptyset
  \and \\
  \freenames{x?(y).P} := \{ x \} \cup (\freenames{P} \setminus \{ y \})
  \and 
  \freenames{x!\langle P \rangle} := \{ x \} \cup \{ P \} 
  \and \\
  \freenames{P|Q} := \freenames{P} \cup \freenames{Q}
  \and \\
  \freenames{@{x}} := \{ x \}
\end{mathpar}

$\pi$
$\quotep{\pi}$

$\freenames{-} : \pi \to \mathcal{P}(\quotep{\pi})$

\begin{eqnarray*}
  \freenames{\pzero} & := & \emptyset \\
  \freenames{x?(y).P} & := & \{ x \} \cup (\freenames{P} \setminus \{ y \}) \\
  \freenames{x!\langle P \rangle} & := & \{ x \} \cup \{ P \} \\
  \freenames{P|Q} & := & \freenames{P} \cup \freenames{Q} \\
  \freenames{\dropn{x}} & := & \{ x \}
\end{eqnarray*}

The bound names of a process, $\boundnames{P}$, are those names occurring in $P$
that are not free. For example, in $x?(y).0$, the name $x$ is free, while $y$ is bound.

\begin{mathpar}
  \inferrule* [lab=monoidal-laws] {} { P|Q \equiv Q|P \and P|0 \equiv P \and P|(Q|R) \equiv (P|Q)|R }
\end{mathpar}

\begin{mathpar}
  \inferrule* [lab=alpha-equivalence] {} { (x)P \equiv (y)P\{y/x\} \and y \not\in \freenames{P} }
\end{mathpar}

\begin{definition}
Then two processes, $P,Q$, are alpha-equivalent if $P = Q\{\vec{y}/\vec{x}\}$ for
some $\vec{x} \in \boundnames{Q},\vec{y} \in \boundnames{P}$, where $Q\{\vec{y}/\vec{x}\}$
denotes the capture-avoiding substitution of $\vec{y}$ for $\vec{x}$ in $Q$.
\end{definition}

\begin{definition}
  The {\em structural congruence} \cite{SangiorgiWalker} , $\equiv$,
  between processes is the least congruence containing
  alpha-equivalence, satisfying the abelian monoid laws
  (associativity, commutativity and $\pzero$ as identity) for parallel
  composition $|$ and for summation $+$.
\end{definition}

\subsection{Name equivalence}

We take name equivalence, written $\nameeq$, to be the smallest
equivalence relation generated by the following rules.

\begin{mathpar}
\inferrule*[lab=Quote-drop]
{ }
{ \quotep{@{x}} \nameeq x }

\inferrule*[lab=Struct-equiv]
{ P \scong Q }
{ \quotep{P} \nameeq \quotep{Q} }
\end{mathpar}

The astute reader will have noticed that the mutual recursion of names
and processes imposes a mutual recursion on alpha-equivalence and
structural equivalence via name-equivalence. Fortunately, all of this
works out pleasantly and we may calculate in the natural way, free of
concern. The reader interested in the details is referred to the
appendix \ref{appendix:rho_details}.

\subsection{Substitution}

We use $\Proc$ for the set of processes, $\QProc$ for the set of
names, and $\id{\{}\vec{y} / \vec{x} \id{\}}$ to denote partial maps,
$s : \QProc \rightarrow \QProc$. A map, $s$ lifts, uniquely, to a map
on process terms, $\widehat{s} : \Proc \rightarrow \Proc$ by the
following equations.

\begin{mathpar}
  (0) \psubstp{Q}{P} := 0 \\
  (R \juxtap S) \psubstp{Q}{P}
  :=    
  (R)\psubstp{Q}{P} \juxtap (S) \psubstp{Q}{P} \\
  (x?(y).R) \psubstp{Q}{P}    
  :=    
  (x)\substp{Q}{P} (z)\concat( (R \psubstn{z}{y}) \psubstp{Q}{P} ) \\
  (\lift{x}{R}) \psubstp{Q}{P}  
  :=
  \lift{(x)\substp{Q}{P}}{ R \psubstp{Q}{P} } \\
%   (\dropn{x})  \psubstp{Q}{P}       
%   := 
%   \left\{ 
%     \begin{array}{ccc} 
%       \dropn{\quotep{Q}} & & x \nameeq \quotep{P} \\
%       \dropn{x} & & otherwise \\
%     \end{array}
%   \right. 
  (\dropn{x})  \psubstp{Q}{P}       
  := 
  \left\{ 
    \begin{array}{ccc} 
      Q & & x \nameeq \quotep{P} \\
      \dropn{x} & & otherwise \\
    \end{array}
  \right.
\end{mathpar}
 

where

\begin{eqnarray}
  (x)\id{\{} \lpquote Q \rpquote / \lpquote P \rpquote \id{\}}            = 
  \left\{ 
    \begin{array}{ccc}
      \lpquote Q \rpquote & & x \nameeq \lpquote P \rpquote \\
      x & & otherwise \\
    \end{array}
  \right. \nonumber
\end{eqnarray}

and $z$ is chosen distinct from $\quotep{P}$, $\quotep{Q}$, the free
names in $Q$, and all the names in $R$. Our $\alpha$-equivalence will
be built in the standard way from this substitution.

\begin{remark}\label{rem:no_self_referential_names}
  One consequence of these definitions is that $\forall P. \quotep{P}
  \not\in \freenames{P}$.
\end{remark}

\subsection{ Dynamic quote: an example }

Anticipating something of what's to come, consider applying the
substitution, $\widehat{\id{\{}u / z \id{\}}}$, to the following pair
of processes, $\lift{w}{y!(z)}$ and $w[ \lpquote y!(z) \rpquote ]$.

\begin{eqnarray}
	\lift{w}{y!(z)}\widehat{\id{\{}u / z \id{\}}}
		& = &
		\lift{w}{y!(u)} \nonumber\\
	w[ \lpquote y!(z) \rpquote ] \widehat{ \id{\{}u / z \id{\}} }
		& = &
		w[ \lpquote y!(z) \rpquote ] \nonumber
\end{eqnarray}

Because the body of the process between quotes is impervious to
substitution, we get radically different answers. In fact, by
examining the first process in an input context,
e.g. $x?(z).\lift{w}{y!(z)}$, we see that the process under the lift
operator may be shaped by prefixed inputs binding a name inside it. In
this sense, the lift operator will be seen as a way to dynamically
construct processes before reifying them as names.

Finally equipped with these standard features we can present the
dynamics of the calculus.

\subsubsection{Operational semantics} 

Finally, we introduce the computational dynamics. What marks these
algebras as distinct from other more traditionally studied algebraic
structures, e.g. vector spaces or polynomial rings, is the manner in
which dynamics is captured. In traditional structures, dynamics is typically
expressed through morphisms between such structures, as in linear maps
between vector spaces or morphisms between rings. In algebras
associated with the semantics of computation, the dynamics is
expressed as part of the algebraic structure itself, through a
reduction reduction relation typically denoted by $\red$. Below, we
give a recursive presentation of this relation for the calculus used
in the encoding.

$\red \subseteq \pi \times \pi$
$\red : \pi \to \mathcal{P}(\pi)$

\begin{mathpar}
  \inferrule* [lab=Comm] { \textsf{match}( x_{src}, x_{trgt} ) } { x_{trgt}?(y)P \; | \; x_{src}!\langle {Q} \rangle \red P\{\quotep{Q}/y}\} }
  \and \\
  \inferrule* [lab=Par] {{P} \red {P}'} {{{P} | {Q}} \red {{P}' | {Q}}}
  \and
  \inferrule* [lab=Equiv]{{{P} \scong {P}'} \andalso {{P}' \red {Q}'} \andalso {{Q}' \scong {Q}}}{{P} \red {Q}}
\end{mathpar}

\begin{eqnarray*}
  match_{\equiv} (\quotep{P},\quotep{Q}) & := & P \equiv Q \\
  match_{\dagger}(\quotep{P},\quotep{Q}) & := & \forall R. P|Q \red^{*} R => R \red^{*} 0 \\
  match_{K}(\quotep{P},\quotep{Q}) & := & K \mbox{ for some context } K
\end{eqnarray*}

$u?(x)P | u!\langle Q \rangle \red P\{\quotep{Q}/x\}$

%We write $\wred$ for $\red^*$, and $P\red$ if $\exists Q $ such that $ P \red Q$.
We write $P\red$ if $\exists Q $ such that $ P \red Q$ and $P\not\red$, otherwise.

\section{Replication}

As mentioned before, it is known that replication (and hence
recursion) can be implemented in a higher-order process algebra
\cite{SangiorgiWalker}. As our first example of calculation with the
machinery thus far presented we give the construction explicitly in
the {\rhoc}.

\begin{eqnarray}
	D_{x} & := & \prefix{x}{y}{(\binpar{\outputp{x}{y}}{@{y}})} \nonumber\\
	\bangp_{x}{P} & := & \binpar{{x}!\langle{\binpar{D_{x}}{P}}\rangle}{D_{x}} \nonumber
\end{eqnarray}

\begin{eqnarray}
	\bangp_{x}{P} & & \nonumber\\
	=
	& {x}!\langle{(\prefix{x}{y}{(\outputp{x}{y} | @{y})) | P}}\rangle 
	      | \prefix{x}{y}{(\outputp{x}{y} | @{y})} & \nonumber\\
	\red
	& (\outputp{x}{y} | @{y})\substn{\quotep{(\prefix{x}{y}{(@{y} | \outputp{x}{y})) | P}}}{y} & \nonumber\\
	=
	& \outputp{x}{\quotep{(\prefix{x}{y}{(\outputp{x}{y} | @{y})) | P}}}
	  | {(\prefix{x}{y}{(\outputp{x}{y} | @{y})) | P}} & \nonumber\\
	\red
	& \ldots & \nonumber\\
	\red^*
	& P | P | \ldots & \nonumber
\end{eqnarray}

Of course, this encoding, as an implementation, runs away, unfolding
$\bangp{P}$ eagerly. A lazier and more implementable replication
operator, restricted to input-guarded processes, may be obtained as follows.

\begin{eqnarray}
\bangp{\prefix{u}{v}{P}} 
	:= 
	\binpar{\lift{x}{\prefix{u}{v}{(\binpar{D(x)}{P})}}}{D(x)} \nonumber
\end{eqnarray}

\begin{remark}
  Note that the lazier definition still does not deal with summation
  or mixed summation (i.e. sums over input and output). The reader is
  invited to construct definitions of replication that deal with these
  features. 

  Further, the definitions are parameterized in a name, $x$. Can you,
  gentle reader, make a definition that eliminates this parameter and
  guarantees no accidental interaction between the replication
  machinery and the process being replicated -- i.e. no accidental
  sharing of names used by the process to get its work done and the
  name(s) used by the replication to effect copying. This latter
  revision of the definition of replication is crucial to obtaining
  the expected identity $!!P \sim !P$.
\end{remark}

\begin{remark}\label{rem:paradoxical_combinator}
  The reader familiar with the lambda calculus will have noticed the
  similarity between $D$ and the paradoxical combinator.

  [Ed. note: the existence of this seems to suggest we have to be more
  restrictive on the set of processes and names we admit if we are to
  support no-cloning.]
\end{remark}

\subsubsection{Bisimulation}

The computational dynamics gives rise to another kind of equivalence,
the equivalence of computational behavior. As previously mentioned
this is typically captured \emph{via} some form of bisimulation.

% The notion we use in this paper is weak barbed bisimulation
% \cite{milner91polyadicpi}.

The notion we use in this paper is derived from weak barbed
bisimulation \cite{milner91polyadicpi}. 

\begin{definition}
An \emph{observation relation}, $\downarrow_{\mathcal N}$, over a set
of names, $\mathcal N$, is the smallest relation satisfying the rules
below.

\infrule[Out-barb]{y \in {\mathcal N}, \; x \nameeq y}
		  {\outputp{x}{v} \downarrow_{\mathcal N} x}
\infrule[Par-barb]{\mbox{$P\downarrow_{\mathcal N} x$ or $Q\downarrow_{\mathcal N} x$}}
		  {\binpar{P}{Q} \downarrow_{\mathcal N} x}

We write $P \Downarrow_{\mathcal N} x$ if there is $Q$ such that 
$P \wred Q$ and $Q \downarrow_{\mathcal N} x$.
\end{definition}

\begin{definition}
%\label{def.bbisim}
An  ${\mathcal N}$-\emph{barbed bisimulation} over a set of names, ${\mathcal N}$, is a symmetric binary relation 
${\mathcal S}_{\mathcal N}$ between agents such that $P\rel{S}_{\mathcal N}Q$ implies:
\begin{enumerate}
\item If $P \red P'$ then $Q \wred Q'$ and $P'\rel{S}_{\mathcal N} Q'$.
\item If $P\downarrow_{\mathcal N} x$, then $Q\Downarrow_{\mathcal N} x$.
\end{enumerate}
$P$ is ${\mathcal N}$-barbed bisimilar to $Q$, written
$P \wbbisim_{\mathcal N} Q$, if $P \rel{S}_{\mathcal N} Q$ for some ${\mathcal N}$-barbed bisimulation ${\mathcal S}_{\mathcal N}$.
\end{definition}

$\mathcal{R} \subseteq \pi \times \pi$

$P \mathcal{R} Q => \forall P'. P \red P' \Rightarrow \exists Q'. Q \red Q', P' \mathcal{R} Q'$

$P \vdash x \Rightarrow Q \vdash x$

\begin{mathpar}
  \inferrule*[lab=Out-barb]{x \nameeq y}{{y}!\langle{Q}\rangle \vdash x}
  \and
  \inferrule*[lab=Par-barb]{\mbox{$P\vdash x$ or $Q\vdash x$}}{\binpar{P}{Q} \vdash x}
\end{mathpar}

\subsubsection{Contexts}

One of the principle advantages of computational calculi like the
$\pi$-calculus is a well-defined notion of context,
contextual-equivalence and a correlation between
contextual-equivalence and notions of bisimulation. The notion of
context allows the decomposition of a process into (sub-)process and
its syntactic environment, its context. Thus, a context may be
thought of as a process with a ``hole'' (written $\Box$) in it. The
application of a context $M$ to a process $P$, written $M[P]$, is
tantamount to filling the hole in $M$ with $P$. In this paper we do
not need the full weight of this theory, but do make use of the notion
of context in the proof the main theorem. 

\begin{mathpar}
  \inferrule* [lab=summation] {} {{M_{M},M_{N}} \bc \Box \;|\; x.M_{A} \;|\; M_{M}+M_{N}}
  \and
  \inferrule* [lab=agent] {} {{M_{A}} \bc (\vec{x})M_{P} \;| \; \clift{P_0,\ldots,M_{P},\ldots,P_N}}
  \and \\
  \inferrule* [lab=process] {} {{M_{P}} \bc M_{N} \;| \;P|M_{P} }
\end{mathpar} 

\begin{mathpar}
  \inferrule* [lab=sychronization] {} {M_{N} \bc \Box \;|\; x?M_{F} \;|\; x!M_{C}}
  \and
  \inferrule* [lab=abstraction] {} {{M_{F}} \bc (x)M_{P} }
  \and
  \inferrule* [lab=concretion] {} {{M_{C}} \bc \langle M_{P} \rangle }
  \and \\
  \inferrule* [lab=process] {} {{M_{P}} \bc M_{N} \;| \;P|M_{P} }
\end{mathpar}

\begin{definition}[contextual application] Given a context $M$, and
  process $P$, we define the \emph{contextual application}, $M[P] :=
  M\{P/\Box\}$. That is, the contextual application of M to P is the
  substitution of $P$ for $\Box$ in $M$.
\end{definition}

$\meaningof{-} : L \to \mathcal{P}(\pi)$

\begin{mathpar}
  \inferrule* [lab=collection] {} {\meaningof{true} = \pi, \and \meaningof{~E} = \pi \setminus \meaningof{E}, \and \meaningof{E_{1} \& E_{2}} = \meaningof{E_{1}} \cap \meaningof{E_{2}}}
\end{mathpar}

\begin{mathpar}
  \inferrule* [lab=structure] {} {\meaningof{0} = \{ P \in \pi | P \equiv 0 \}, \and \\ \meaningof{E_1 | E_2} = \{ P \in \pi | P \equiv P_{1} | P_{2}, P_{1} \in \meaningof{E_{1}}, P_{2} \in \meaningof{E_2}\} }
\end{mathpar}

\begin{mathpar}
 \inferrule* [lab=behavior] {} {\meaningof{\langle a?b \rangle E} = \{ P \in \pi | P \equiv Q | u?(y)P', \\ \and \\\\ \and \\ \;\;\; u \in \meaningof{a}, \forall z.P'\{z/y\} \in \meaningof{E\{z/b\}}\}, \and \\ \meaningof{a!E} = \{ P \in \pi | P \equiv Q | x!\langle P' \rangle, x \in \meaningof{a} P' \in \meaningof{E}\} }
\end{mathpar}

\begin{mathpar}
 \inferrule* [lab=nominal] {} {\meaningof{\quotep{E}} = \{ \quotep{P} \in \quotep{\pi} | P \in \meaningof{E} \}, \and \meaningof{\quotep{P}} = \{ \quotep{Q} \in \quotep{\pi} | P \equiv Q \} \and \\ \meaningof{@\quotep{E}} = \{ P \in \pi | P \equiv @x, x \in \meaningof{E} \}}
\end{mathpar}

\begin{eqnarray*}
  \\
  \meaningof{-} : TS \to ST
\end{eqnarray*}

\begin{eqnarray*}
  \\
  L : TS \to ST
\end{eqnarray*}

\begin{eqnarray*}
  \\
  P \models E \iff P \in \meaningof{E}
\end{eqnarray*}

\begin{eqnarray*}
  P \approx_{L} Q \iff \forall E \in L. P \models E \iff Q \models E
\end{eqnarray*}

\begin{eqnarray*}
  P \approx_{K} Q
\end{eqnarray*}

\begin{eqnarray*}
  P \approx Q
\end{eqnarray*}

$\approx_{K} = \approx = \approx_{L}$

\subsubsection{Contextual duality}

Note that contexts extend the quotation operation to a family of
operations from processes to names. Given a context, $M$, we can
define a \emph{nominal context}, $\quotep{M}$ by $\quotep{M}[P] :=
\quotep{M[P]}$. To foreshadow what is to come we observe that these
operations enjoy a duality with processes very much like the duality
between vectors and maps from vectors to scalars.

Further, because the calculus is essentially higher-order, we have a
correspondence between contexts and processes. More specifically,
given a name $x$ and a context $M$ we can construct $M^{*}_{x}$ such
that 

\begin{mathpar}
  M^{*}_{x} | \lift{x}{P} \red M[P]
\end{mathpar}

namely,

\begin{mathpar}
  M^{*}_{x} := x?(u).M[\dropn{u}]
\end{mathpar}

The dependence of $M^{*}_{x}$ on a name makes it an abstraction, 

\begin{mathpar}
  M^{*} := (x)x?(u).M[\dropn{u}]
\end{mathpar}

\subsection{Additional notation}

It will sometimes be convenient to denote the process a name
quotes. We already have the notation $x = \quotep{P}$, but it will be
convenient to introduce an alternate notation, $\procn{x}$, when we
want to emphasize the connection to the use of the name. Note that, by
virtue of name equivalence, $\quotep{\procn{x}} \nameeq x$; so, the
notation is consistent with previous definitions.

Further, because names have structure it is possible to effect
substitutions on the basis of that structure. This means we need to
upgrade our notation for substitutions, which we accomplish by
adapting comprehension notation. Thus,

\begin{mathpar}
  P\{ y / x : x \in S \}
\end{mathpar}

is interpreted to mean the process derived from P by replacing (in a
capture-avoiding manner) each occurrence of $x$ in $S$ by $y$. For example,

\begin{mathpar}
  P\{ \quotep{\procn{x}|\procn{x}} / x : x \in \freenames{P} \}
\end{mathpar}

will replace each (occurrence) of a free name $x$ in $P$ by
$\quotep{\procn{x}|\procn{x}}$.

Also, we will avail ourselves of the notation $x^{L}$ and $x^{R}$ to
denote injections of a name into disjoint copies of the name
space. There are numerous ways to accomplish this. One example can be
found in \cite{MeredithR05}. This notation overloads to vectors of
names: $\vec{x}^{\pi} := (x_{i}^{\pi} \; : \; 0 \leq i < |\vec{x}| )$ where $\pi \in \{L,R\}$.

We also use $P^{\Box} := P|\Box$.

In \cite{MeredithR05} an interpretation of the new operator is
given. It turns out that there are several possible interpretations
all enjoying the requisite algebraic properties of the operator (see
\cite{milner91polyadicpi}). We will therefore make liberal use of
$(\nu\; \vec{x})P$.

% subsection the_syntax_and_semantics_of_the_notation_system (end)   

\input{qm2pi.qmops} 

\input{qm2pi.sterngerlach} 

\input{qm2pi.metric} 

% section concurrent_process_calculi (end)

%\input{qm2pi.proofsketch}

% section proof sketch (end)

%\input{qm2pi.slviaknots} 

% section spatial logic via knots (end)

\input{qm2pi.conclusion}

% section conclusion (end)

%\input{qm2pi.dtcodes} 

% section wiring algorithm (end)

\input{qm2pi.ack} 

% section acknowledgments (end)

\newpage


\bibliographystyle{plain}   
\bibliography{../../biblios/main.bib}

\input{qm2pi.rhodetails}

\end{document}

 

% section concurrent_process_calculi (end)

%\documentclass[12pt]{llncs}
%\documentclass{jktr}

\usepackage[pdftex]{hyperref}                   
\usepackage {listings}
\usepackage {mathpartir}
\usepackage{bcprules}
%\usepackage{listings}
                       
\usepackage{graphicx} 
%\usepackage[margins=2.5cm,nohead,nofoot]{geometry}
%\usepackage{geometry}
\usepackage{amsfonts}
\usepackage{amstext}
\usepackage{latexsym}
\usepackage{amssymb}
\usepackage{color}


%\include{myPreamble}
\include{qm2pi.local} 

%\ifpdf
%\usepackage[pdftex]{graphicx}
%\else
%\usepackage{graphicx}
%\fi

 % \ifpdf
%  \usepackage{pdfsync}
%  \if


%\title{Brief Article}
%\author{David F. Snyder}
%\author{L.G. Meredith}

%\address{Dept. of Math., Texas State University--San Marcos, San Marcos, TX 78666}
       
\pagestyle{empty}


\begin{document}

\lstset{language=[Objective]Caml,frame=shadowbox}

\input{qm2pi.front}

% section front matter (end)

\input{qm2pi.intro} 
 
% section introduction (end)

% \input{qm2pi.knotations} 

% section notation (end)

\input{qm2pi.process.calculi} 

% section concurrent_process_calculi_and_spatial_logics_ (end)
    
%\input{qm2pi.knots2pi} 

%\input{qm2pi.trefoil} 

%\input{qm2pi.mainthm} 

% subsection basic_interpretation (end)

%\input{qm2pi.rho.presentation} 
\subsection{The syntax and semantics of the notation system}\label{sub:the_syntax_and_semantics_of_the_notation_system} % (fold)

We now summarize a technical presentation of the calculus that
embodies our theory of dynamics. The typical presentation of such a
calculus follows the style of giving generators and relations on
them. The grammar, below, describing term constructors, freely
generates the set of processes, $\Proc$. This set is then quotiented
by a relation known as structural congruence and it is over this set
that the notion of dynamics is expressed. This presentation is
essentially that of \cite{MeredithR05} with the addition of
polyadicity and summation. For readability we have relegated some of
the technical subtleties to an appendix.

\subsubsection{Process grammar}\label{subsub:process_grammar}

\begin{mathpar}
  \inferrule* [lab=synchronization] {} {{M} \bc \pzero \;|\; x?F \;|\; x!C }
  \and
  \inferrule* [lab=abstraction] {} {{F} \bc (x)P}
  \and
  \inferrule* [lab=concretion] {} {{C} \bc \langle Q \rangle}
  \and
  \inferrule* [lab=process] {} {{P,Q} \bc M \;| \;P|Q \;|\; @{x}}
  \and
  \inferrule* [lab=name] {} {{x} \bc \quotep{P}}
\end{mathpar} 

Note that $\vec{x}$ (resp. $\vec{P}$) denotes a vector of names
(resp. processes) of length $|\vec{x}|$ (resp. $|\vec{P}|$). We adopt
the following useful abbreviations.

\begin{mathpar}
   x?(\vec{y}).P := x.(\vec{y})P \and  x\clift{\vec{P}} := x.\clift{\vec{P}}
   \and x!(y) := \lift{x}{\dropn{y}}
   \and \Pi_{i=0}^{n-1}P_i := P_0 | \ldots | P_{n-1}
\end{mathpar}

\subsubsection{Structural congruence}

\paragraph{Free and bound names and alpha-equivalence.} At the
core of structural equivalence is alpha-equivalence which identifies
process that are the same up to a change of variable. Formally, we
recognize the distinction between free and bound names. The free names
of a process, $\freenames{P}$, may be calculated recursively as
follows:

\begin{mathpar}
\freenames{\pzero} := \emptyset
  \and \\
  \freenames{x?(y).P} := \{ x \} \cup (\freenames{P} \setminus \{ y \})
  \and 
  \freenames{x!\langle P \rangle} := \{ x \} \cup \{ P \} 
  \and \\
  \freenames{P|Q} := \freenames{P} \cup \freenames{Q}
  \and \\
  \freenames{@{x}} := \{ x \}
\end{mathpar}

$\pi$
$\quotep{\pi}$

$\freenames{-} : \pi \to \mathcal{P}(\quotep{\pi})$

\begin{eqnarray*}
  \freenames{\pzero} & := & \emptyset \\
  \freenames{x?(y).P} & := & \{ x \} \cup (\freenames{P} \setminus \{ y \}) \\
  \freenames{x!\langle P \rangle} & := & \{ x \} \cup \{ P \} \\
  \freenames{P|Q} & := & \freenames{P} \cup \freenames{Q} \\
  \freenames{\dropn{x}} & := & \{ x \}
\end{eqnarray*}

The bound names of a process, $\boundnames{P}$, are those names occurring in $P$
that are not free. For example, in $x?(y).0$, the name $x$ is free, while $y$ is bound.

\begin{mathpar}
  \inferrule* [lab=monoidal-laws] {} { P|Q \equiv Q|P \and P|0 \equiv P \and P|(Q|R) \equiv (P|Q)|R }
\end{mathpar}

\begin{mathpar}
  \inferrule* [lab=alpha-equivalence] {} { (x)P \equiv (y)P\{y/x\} \and y \not\in \freenames{P} }
\end{mathpar}

\begin{definition}
Then two processes, $P,Q$, are alpha-equivalent if $P = Q\{\vec{y}/\vec{x}\}$ for
some $\vec{x} \in \boundnames{Q},\vec{y} \in \boundnames{P}$, where $Q\{\vec{y}/\vec{x}\}$
denotes the capture-avoiding substitution of $\vec{y}$ for $\vec{x}$ in $Q$.
\end{definition}

\begin{definition}
  The {\em structural congruence} \cite{SangiorgiWalker} , $\equiv$,
  between processes is the least congruence containing
  alpha-equivalence, satisfying the abelian monoid laws
  (associativity, commutativity and $\pzero$ as identity) for parallel
  composition $|$ and for summation $+$.
\end{definition}

\subsection{Name equivalence}

We take name equivalence, written $\nameeq$, to be the smallest
equivalence relation generated by the following rules.

\begin{mathpar}
\inferrule*[lab=Quote-drop]
{ }
{ \quotep{@{x}} \nameeq x }

\inferrule*[lab=Struct-equiv]
{ P \scong Q }
{ \quotep{P} \nameeq \quotep{Q} }
\end{mathpar}

The astute reader will have noticed that the mutual recursion of names
and processes imposes a mutual recursion on alpha-equivalence and
structural equivalence via name-equivalence. Fortunately, all of this
works out pleasantly and we may calculate in the natural way, free of
concern. The reader interested in the details is referred to the
appendix \ref{appendix:rho_details}.

\subsection{Substitution}

We use $\Proc$ for the set of processes, $\QProc$ for the set of
names, and $\id{\{}\vec{y} / \vec{x} \id{\}}$ to denote partial maps,
$s : \QProc \rightarrow \QProc$. A map, $s$ lifts, uniquely, to a map
on process terms, $\widehat{s} : \Proc \rightarrow \Proc$ by the
following equations.

\begin{mathpar}
  (0) \psubstp{Q}{P} := 0 \\
  (R \juxtap S) \psubstp{Q}{P}
  :=    
  (R)\psubstp{Q}{P} \juxtap (S) \psubstp{Q}{P} \\
  (x?(y).R) \psubstp{Q}{P}    
  :=    
  (x)\substp{Q}{P} (z)\concat( (R \psubstn{z}{y}) \psubstp{Q}{P} ) \\
  (\lift{x}{R}) \psubstp{Q}{P}  
  :=
  \lift{(x)\substp{Q}{P}}{ R \psubstp{Q}{P} } \\
%   (\dropn{x})  \psubstp{Q}{P}       
%   := 
%   \left\{ 
%     \begin{array}{ccc} 
%       \dropn{\quotep{Q}} & & x \nameeq \quotep{P} \\
%       \dropn{x} & & otherwise \\
%     \end{array}
%   \right. 
  (\dropn{x})  \psubstp{Q}{P}       
  := 
  \left\{ 
    \begin{array}{ccc} 
      Q & & x \nameeq \quotep{P} \\
      \dropn{x} & & otherwise \\
    \end{array}
  \right.
\end{mathpar}
 

where

\begin{eqnarray}
  (x)\id{\{} \lpquote Q \rpquote / \lpquote P \rpquote \id{\}}            = 
  \left\{ 
    \begin{array}{ccc}
      \lpquote Q \rpquote & & x \nameeq \lpquote P \rpquote \\
      x & & otherwise \\
    \end{array}
  \right. \nonumber
\end{eqnarray}

and $z$ is chosen distinct from $\quotep{P}$, $\quotep{Q}$, the free
names in $Q$, and all the names in $R$. Our $\alpha$-equivalence will
be built in the standard way from this substitution.

\begin{remark}\label{rem:no_self_referential_names}
  One consequence of these definitions is that $\forall P. \quotep{P}
  \not\in \freenames{P}$.
\end{remark}

\subsection{ Dynamic quote: an example }

Anticipating something of what's to come, consider applying the
substitution, $\widehat{\id{\{}u / z \id{\}}}$, to the following pair
of processes, $\lift{w}{y!(z)}$ and $w[ \lpquote y!(z) \rpquote ]$.

\begin{eqnarray}
	\lift{w}{y!(z)}\widehat{\id{\{}u / z \id{\}}}
		& = &
		\lift{w}{y!(u)} \nonumber\\
	w[ \lpquote y!(z) \rpquote ] \widehat{ \id{\{}u / z \id{\}} }
		& = &
		w[ \lpquote y!(z) \rpquote ] \nonumber
\end{eqnarray}

Because the body of the process between quotes is impervious to
substitution, we get radically different answers. In fact, by
examining the first process in an input context,
e.g. $x?(z).\lift{w}{y!(z)}$, we see that the process under the lift
operator may be shaped by prefixed inputs binding a name inside it. In
this sense, the lift operator will be seen as a way to dynamically
construct processes before reifying them as names.

Finally equipped with these standard features we can present the
dynamics of the calculus.

\subsubsection{Operational semantics} 

Finally, we introduce the computational dynamics. What marks these
algebras as distinct from other more traditionally studied algebraic
structures, e.g. vector spaces or polynomial rings, is the manner in
which dynamics is captured. In traditional structures, dynamics is typically
expressed through morphisms between such structures, as in linear maps
between vector spaces or morphisms between rings. In algebras
associated with the semantics of computation, the dynamics is
expressed as part of the algebraic structure itself, through a
reduction reduction relation typically denoted by $\red$. Below, we
give a recursive presentation of this relation for the calculus used
in the encoding.

$\red \subseteq \pi \times \pi$
$\red : \pi \to \mathcal{P}(\pi)$

\begin{mathpar}
  \inferrule* [lab=Comm] { \textsf{match}( x_{src}, x_{trgt} ) } { x_{trgt}?(y)P \; | \; x_{src}!\langle {Q} \rangle \red P\{\quotep{Q}/y}\} }
  \and \\
  \inferrule* [lab=Par] {{P} \red {P}'} {{{P} | {Q}} \red {{P}' | {Q}}}
  \and
  \inferrule* [lab=Equiv]{{{P} \scong {P}'} \andalso {{P}' \red {Q}'} \andalso {{Q}' \scong {Q}}}{{P} \red {Q}}
\end{mathpar}

\begin{eqnarray*}
  match_{\equiv} (\quotep{P},\quotep{Q}) & := & P \equiv Q \\
  match_{\dagger}(\quotep{P},\quotep{Q}) & := & \forall R. P|Q \red^{*} R => R \red^{*} 0 \\
  match_{K}(\quotep{P},\quotep{Q}) & := & K \mbox{ for some context } K
\end{eqnarray*}

$u?(x)P | u!\langle Q \rangle \red P\{\quotep{Q}/x\}$

%We write $\wred$ for $\red^*$, and $P\red$ if $\exists Q $ such that $ P \red Q$.
We write $P\red$ if $\exists Q $ such that $ P \red Q$ and $P\not\red$, otherwise.

\section{Replication}

As mentioned before, it is known that replication (and hence
recursion) can be implemented in a higher-order process algebra
\cite{SangiorgiWalker}. As our first example of calculation with the
machinery thus far presented we give the construction explicitly in
the {\rhoc}.

\begin{eqnarray}
	D_{x} & := & \prefix{x}{y}{(\binpar{\outputp{x}{y}}{@{y}})} \nonumber\\
	\bangp_{x}{P} & := & \binpar{{x}!\langle{\binpar{D_{x}}{P}}\rangle}{D_{x}} \nonumber
\end{eqnarray}

\begin{eqnarray}
	\bangp_{x}{P} & & \nonumber\\
	=
	& {x}!\langle{(\prefix{x}{y}{(\outputp{x}{y} | @{y})) | P}}\rangle 
	      | \prefix{x}{y}{(\outputp{x}{y} | @{y})} & \nonumber\\
	\red
	& (\outputp{x}{y} | @{y})\substn{\quotep{(\prefix{x}{y}{(@{y} | \outputp{x}{y})) | P}}}{y} & \nonumber\\
	=
	& \outputp{x}{\quotep{(\prefix{x}{y}{(\outputp{x}{y} | @{y})) | P}}}
	  | {(\prefix{x}{y}{(\outputp{x}{y} | @{y})) | P}} & \nonumber\\
	\red
	& \ldots & \nonumber\\
	\red^*
	& P | P | \ldots & \nonumber
\end{eqnarray}

Of course, this encoding, as an implementation, runs away, unfolding
$\bangp{P}$ eagerly. A lazier and more implementable replication
operator, restricted to input-guarded processes, may be obtained as follows.

\begin{eqnarray}
\bangp{\prefix{u}{v}{P}} 
	:= 
	\binpar{\lift{x}{\prefix{u}{v}{(\binpar{D(x)}{P})}}}{D(x)} \nonumber
\end{eqnarray}

\begin{remark}
  Note that the lazier definition still does not deal with summation
  or mixed summation (i.e. sums over input and output). The reader is
  invited to construct definitions of replication that deal with these
  features. 

  Further, the definitions are parameterized in a name, $x$. Can you,
  gentle reader, make a definition that eliminates this parameter and
  guarantees no accidental interaction between the replication
  machinery and the process being replicated -- i.e. no accidental
  sharing of names used by the process to get its work done and the
  name(s) used by the replication to effect copying. This latter
  revision of the definition of replication is crucial to obtaining
  the expected identity $!!P \sim !P$.
\end{remark}

\begin{remark}\label{rem:paradoxical_combinator}
  The reader familiar with the lambda calculus will have noticed the
  similarity between $D$ and the paradoxical combinator.

  [Ed. note: the existence of this seems to suggest we have to be more
  restrictive on the set of processes and names we admit if we are to
  support no-cloning.]
\end{remark}

\subsubsection{Bisimulation}

The computational dynamics gives rise to another kind of equivalence,
the equivalence of computational behavior. As previously mentioned
this is typically captured \emph{via} some form of bisimulation.

% The notion we use in this paper is weak barbed bisimulation
% \cite{milner91polyadicpi}.

The notion we use in this paper is derived from weak barbed
bisimulation \cite{milner91polyadicpi}. 

\begin{definition}
An \emph{observation relation}, $\downarrow_{\mathcal N}$, over a set
of names, $\mathcal N$, is the smallest relation satisfying the rules
below.

\infrule[Out-barb]{y \in {\mathcal N}, \; x \nameeq y}
		  {\outputp{x}{v} \downarrow_{\mathcal N} x}
\infrule[Par-barb]{\mbox{$P\downarrow_{\mathcal N} x$ or $Q\downarrow_{\mathcal N} x$}}
		  {\binpar{P}{Q} \downarrow_{\mathcal N} x}

We write $P \Downarrow_{\mathcal N} x$ if there is $Q$ such that 
$P \wred Q$ and $Q \downarrow_{\mathcal N} x$.
\end{definition}

\begin{definition}
%\label{def.bbisim}
An  ${\mathcal N}$-\emph{barbed bisimulation} over a set of names, ${\mathcal N}$, is a symmetric binary relation 
${\mathcal S}_{\mathcal N}$ between agents such that $P\rel{S}_{\mathcal N}Q$ implies:
\begin{enumerate}
\item If $P \red P'$ then $Q \wred Q'$ and $P'\rel{S}_{\mathcal N} Q'$.
\item If $P\downarrow_{\mathcal N} x$, then $Q\Downarrow_{\mathcal N} x$.
\end{enumerate}
$P$ is ${\mathcal N}$-barbed bisimilar to $Q$, written
$P \wbbisim_{\mathcal N} Q$, if $P \rel{S}_{\mathcal N} Q$ for some ${\mathcal N}$-barbed bisimulation ${\mathcal S}_{\mathcal N}$.
\end{definition}

$\mathcal{R} \subseteq \pi \times \pi$

$P \mathcal{R} Q => \forall P'. P \red P' \Rightarrow \exists Q'. Q \red Q', P' \mathcal{R} Q'$

$P \vdash x \Rightarrow Q \vdash x$

\begin{mathpar}
  \inferrule*[lab=Out-barb]{x \nameeq y}{{y}!\langle{Q}\rangle \vdash x}
  \and
  \inferrule*[lab=Par-barb]{\mbox{$P\vdash x$ or $Q\vdash x$}}{\binpar{P}{Q} \vdash x}
\end{mathpar}

\subsubsection{Contexts}

One of the principle advantages of computational calculi like the
$\pi$-calculus is a well-defined notion of context,
contextual-equivalence and a correlation between
contextual-equivalence and notions of bisimulation. The notion of
context allows the decomposition of a process into (sub-)process and
its syntactic environment, its context. Thus, a context may be
thought of as a process with a ``hole'' (written $\Box$) in it. The
application of a context $M$ to a process $P$, written $M[P]$, is
tantamount to filling the hole in $M$ with $P$. In this paper we do
not need the full weight of this theory, but do make use of the notion
of context in the proof the main theorem. 

\begin{mathpar}
  \inferrule* [lab=summation] {} {{M_{M},M_{N}} \bc \Box \;|\; x.M_{A} \;|\; M_{M}+M_{N}}
  \and
  \inferrule* [lab=agent] {} {{M_{A}} \bc (\vec{x})M_{P} \;| \; \clift{P_0,\ldots,M_{P},\ldots,P_N}}
  \and \\
  \inferrule* [lab=process] {} {{M_{P}} \bc M_{N} \;| \;P|M_{P} }
\end{mathpar} 

\begin{mathpar}
  \inferrule* [lab=sychronization] {} {M_{N} \bc \Box \;|\; x?M_{F} \;|\; x!M_{C}}
  \and
  \inferrule* [lab=abstraction] {} {{M_{F}} \bc (x)M_{P} }
  \and
  \inferrule* [lab=concretion] {} {{M_{C}} \bc \langle M_{P} \rangle }
  \and \\
  \inferrule* [lab=process] {} {{M_{P}} \bc M_{N} \;| \;P|M_{P} }
\end{mathpar}

\begin{definition}[contextual application] Given a context $M$, and
  process $P$, we define the \emph{contextual application}, $M[P] :=
  M\{P/\Box\}$. That is, the contextual application of M to P is the
  substitution of $P$ for $\Box$ in $M$.
\end{definition}

$\meaningof{-} : L \to \mathcal{P}(\pi)$

\begin{mathpar}
  \inferrule* [lab=collection] {} {\meaningof{true} = \pi, \and \meaningof{~E} = \pi \setminus \meaningof{E}, \and \meaningof{E_{1} \& E_{2}} = \meaningof{E_{1}} \cap \meaningof{E_{2}}}
\end{mathpar}

\begin{mathpar}
  \inferrule* [lab=structure] {} {\meaningof{0} = \{ P \in \pi | P \equiv 0 \}, \and \\ \meaningof{E_1 | E_2} = \{ P \in \pi | P \equiv P_{1} | P_{2}, P_{1} \in \meaningof{E_{1}}, P_{2} \in \meaningof{E_2}\} }
\end{mathpar}

\begin{mathpar}
 \inferrule* [lab=behavior] {} {\meaningof{\langle a?b \rangle E} = \{ P \in \pi | P \equiv Q | u?(y)P', \\ \and \\\\ \and \\ \;\;\; u \in \meaningof{a}, \forall z.P'\{z/y\} \in \meaningof{E\{z/b\}}\}, \and \\ \meaningof{a!E} = \{ P \in \pi | P \equiv Q | x!\langle P' \rangle, x \in \meaningof{a} P' \in \meaningof{E}\} }
\end{mathpar}

\begin{mathpar}
 \inferrule* [lab=nominal] {} {\meaningof{\quotep{E}} = \{ \quotep{P} \in \quotep{\pi} | P \in \meaningof{E} \}, \and \meaningof{\quotep{P}} = \{ \quotep{Q} \in \quotep{\pi} | P \equiv Q \} \and \\ \meaningof{@\quotep{E}} = \{ P \in \pi | P \equiv @x, x \in \meaningof{E} \}}
\end{mathpar}

\begin{eqnarray*}
  \\
  \meaningof{-} : TS \to ST
\end{eqnarray*}

\begin{eqnarray*}
  \\
  L : TS \to ST
\end{eqnarray*}

\begin{eqnarray*}
  \\
  P \models E \iff P \in \meaningof{E}
\end{eqnarray*}

\begin{eqnarray*}
  P \approx_{L} Q \iff \forall E \in L. P \models E \iff Q \models E
\end{eqnarray*}

\begin{eqnarray*}
  P \approx_{K} Q
\end{eqnarray*}

\begin{eqnarray*}
  P \approx Q
\end{eqnarray*}

$\approx_{K} = \approx = \approx_{L}$

\subsubsection{Contextual duality}

Note that contexts extend the quotation operation to a family of
operations from processes to names. Given a context, $M$, we can
define a \emph{nominal context}, $\quotep{M}$ by $\quotep{M}[P] :=
\quotep{M[P]}$. To foreshadow what is to come we observe that these
operations enjoy a duality with processes very much like the duality
between vectors and maps from vectors to scalars.

Further, because the calculus is essentially higher-order, we have a
correspondence between contexts and processes. More specifically,
given a name $x$ and a context $M$ we can construct $M^{*}_{x}$ such
that 

\begin{mathpar}
  M^{*}_{x} | \lift{x}{P} \red M[P]
\end{mathpar}

namely,

\begin{mathpar}
  M^{*}_{x} := x?(u).M[\dropn{u}]
\end{mathpar}

The dependence of $M^{*}_{x}$ on a name makes it an abstraction, 

\begin{mathpar}
  M^{*} := (x)x?(u).M[\dropn{u}]
\end{mathpar}

\subsection{Additional notation}

It will sometimes be convenient to denote the process a name
quotes. We already have the notation $x = \quotep{P}$, but it will be
convenient to introduce an alternate notation, $\procn{x}$, when we
want to emphasize the connection to the use of the name. Note that, by
virtue of name equivalence, $\quotep{\procn{x}} \nameeq x$; so, the
notation is consistent with previous definitions.

Further, because names have structure it is possible to effect
substitutions on the basis of that structure. This means we need to
upgrade our notation for substitutions, which we accomplish by
adapting comprehension notation. Thus,

\begin{mathpar}
  P\{ y / x : x \in S \}
\end{mathpar}

is interpreted to mean the process derived from P by replacing (in a
capture-avoiding manner) each occurrence of $x$ in $S$ by $y$. For example,

\begin{mathpar}
  P\{ \quotep{\procn{x}|\procn{x}} / x : x \in \freenames{P} \}
\end{mathpar}

will replace each (occurrence) of a free name $x$ in $P$ by
$\quotep{\procn{x}|\procn{x}}$.

Also, we will avail ourselves of the notation $x^{L}$ and $x^{R}$ to
denote injections of a name into disjoint copies of the name
space. There are numerous ways to accomplish this. One example can be
found in \cite{MeredithR05}. This notation overloads to vectors of
names: $\vec{x}^{\pi} := (x_{i}^{\pi} \; : \; 0 \leq i < |\vec{x}| )$ where $\pi \in \{L,R\}$.

We also use $P^{\Box} := P|\Box$.

In \cite{MeredithR05} an interpretation of the new operator is
given. It turns out that there are several possible interpretations
all enjoying the requisite algebraic properties of the operator (see
\cite{milner91polyadicpi}). We will therefore make liberal use of
$(\nu\; \vec{x})P$.

% subsection the_syntax_and_semantics_of_the_notation_system (end)   

\input{qm2pi.qmops} 

\input{qm2pi.sterngerlach} 

\input{qm2pi.metric} 

% section concurrent_process_calculi (end)

%\input{qm2pi.proofsketch}

% section proof sketch (end)

%\input{qm2pi.slviaknots} 

% section spatial logic via knots (end)

\input{qm2pi.conclusion}

% section conclusion (end)

%\input{qm2pi.dtcodes} 

% section wiring algorithm (end)

\input{qm2pi.ack} 

% section acknowledgments (end)

\newpage


\bibliographystyle{plain}   
\bibliography{../../biblios/main.bib}

\input{qm2pi.rhodetails}

\end{document}



% section proof sketch (end)

%\section{Unlikely characters: spatial logic for
  knots}\label{sub:characteristic_formulae} % (fold)

Associated to the mobile process calculi are a family of logics known
as the Hennessy-Milner logics. These logics typically enjoy a
semantics interpreting formulae as sets of processes that when
factored through the encoding outlined above allows an identification
of classes of knots with logical formulae. In the context of this
encoding the sub-family known as the spatial logics \cite{CairesC03}
\cite{CairesC04} \cite{Caires04} are of particular interest providing
several important features for expressing and reasoning about
properties (i.e. classes) of knots. We hint here at how this may be done.

%\begin{description}
%\item [structural connectives] 
\subsubsection{Structural connectives} The spatial logics enjoy
structural connectives corresponding, at the logical level, to the
parallel composition ($P | Q$) and new name ($(\nu \; x)P$)
connectives for processes. As illustrated in the examples below, these
connectives are extremely expressive given the shape of our encoding.
%\item [decideable satisfaction]

\subsubsection{Decideable satisfaction}
In \cite{Caires04} the satisfaction relation is shown to be decideable
for a rich class of processes. It further turns out that the image of
the our encoding is a proper subset of that class. This result
provides the basis for an algorithm by which to search for knots
enjoying a given property.
%\item [characteristic formulae]

\subsubsection{Characteristic formulae}
In the same paper \cite{Caires04} , Caires presents a means of calculating
characteristic formulae, selecting equivalence classes of processes
up to a pre--specified depth limit on the support set of names. Composed with our
encoding, this characteristic formula can be used to select
characteristic formulae for knots.
%\end{description}

\subsubsection{Spatial logic formulae}

The grammar below (segmented for comprehension) summarizes the syntax
of spatial logic formulae. We employ illustrative examples in the
sequel to provide an intuitive understanding of their meaning
referring the reader to \cite{Caires04} for a more detailed explication
of the semantics.

\begin{mathpar}
  \inferrule* [lab=boolean] {} {{A,B} \bc T \;|\; \neg A \;|\; A \wedge B \;|\; \eta = \eta'}
  \and
  \inferrule* [lab=spatial] {} {|\; \pzero \;|\; A | B \;|\; x \text{\textregistered} A \;|\; \forall x . A \;|\;  H x . A}
  \and
  \inferrule* [lab=behavioral] {} {|\; \alpha . A}
  \and 
  \inferrule* [lab=recursion] {} {|\; X(\vec{u}) \;|\; \mu X(\vec{u}) . A}
  \and
  \inferrule* [lab=action] {} {\alpha \bc \langle x?(\vec{y}) \rangle \;|\; \langle x!(\vec{y}) \rangle \;|\; \langle \tau \rangle}
  \and 
  \inferrule* [lab=name] {} {\eta \bc x \;|\; \tau}
\end{mathpar} 

% subsection characteristic_formulae (end)   	 

\subsection{Example formulae}\label{sub:example_formulae_} % (fold)

\subsubsection{Crossing as formula.}
% 
% \begin{align*}
%   \frac{d}{dx} \sin x &= \cos x 
%   & \frac{d}{dx} e^x &= e^x \\
%   \frac{d}{dx} \cos x &= - \sin x 
%   & \frac{d}{dx} \log x &= \frac{1}{x} \\
% \end{align*} 

\begin{align*}
 \mu C(x_{0},x_{1},y_{0},y_{1},u).&(\langle x_{0}?(z) \rangle(\langle u! \rangle\langle y_{1}!z \rangle C(x_{0},x_{1},y_{0},y_{1},u)) & \\
  & \wedge \langle y_{1}?(z) \rangle (\langle u! \rangle \langle x_{0}!z \rangle C(x_{0},x_{1},y_{0},y_{1},u)) & \\
  & \wedge \langle x_{1}?(z) \rangle (\langle u? \rangle \langle y_{0}!z \rangle C(x_{0},x_{1},y_{0},y_{1},u)) & \\
  & \wedge \langle y_{0}?(z) \rangle (\langle u? \rangle \langle x_{1}!z \rangle C(x_{0},x_{1},y_{0},y_{1},u))) &
\end{align*}

The lexicographical similarity between the shape of this formulae and
the shape of definition of the process representing a crossing reveals
the intuitive meaning of this formulae. It describes the capabilities
of a process that has the right to represent a crossing. For example
it picks out processes that may perform an input on the port $x_0$ in
its initial menu of capabilities. What differentiates the formula
from the process, however, is that the crossing process is the
smallest candidate to satisfy the formula. Infinitely many other
processes -- with internal behavior hidden behind this interface, so
to speak -- also satisfy this formula. Even this simple formula,
then, can be seen to open a new view onto knots, providing a
computational interpretation of \emph{virtual} knots.

Note that this formula is derived by hand. A similar formula can be
derived by employing Caires' calculation of characteristic formula
\cite{Caires04} to the process representing a crossing. In light of
this discussion, we let
$\meaningof{C}_{\phi}(x0,x1,y0,y1,u)$ denote a formula specifying the
dynamics we wish to capture of a crossing. To guarantee we preserve
the shape of the interface and minimal semantics we demand that
$\meaningof{C}_{\phi}(x0,x1,y0,y1,u) \Rightarrow
\textbf{C}(x0,x1,y0,y1,u)$ where $\textbf{C}(x0,x1,y0,y1,u)$ denotes
the formula above.
                            
\subsubsection{Crossing number constraints.}
The moral content of the context lemma (Lemma \ref{context}) is that the notion of
``locality'' in the Reidemeister moves is effectively captured by the
parallel composition operator of the process calculus. This intuition
extends through the logic. Given a formula,
$\meaningof{C}_{\phi}(x0,x1,y0,y1,u)$, we can use the structural
connectives to specify constraints on crossing numbers, such as at
least $n$ crossings, or exactly $n$ crossings.
\begin{mathpar}
  \inferrule* [lab=at-least-n] {} { K^{\geq n}_{\phi}(\vec{xs},\vec{ys}) := \Pi_{i=0}^{n-1} Hu . \meaningof{C}_{\phi}(xs_i,ys_i,u) | T }
  \and 
  \inferrule* [lab=exactly-n] {} { K^{= n}_{\phi}(\vec{xs},\vec{ys}) := \Pi_{i=0}^{n-1} Hu . \meaningof{C}_{\phi}(xs_i,ys_i,u) | \neg (\forall x_0,y_0,x_1,y_1,u . \meaningof{C}_{\phi}(x_0,y_0,x_1,y_1,u) | T) }
\end{mathpar}

To round out this section, recall that the encoding of an $n$-crossing
knot decomposes into a parallel composition of $n$ \emph{copies} of a
crossing process together with a wiring harness. To specify different
knot classes with the same crossing number amounts to specifying
logical constraints on the wiring harness. In the interest of space,
we defer examples to a forthcoming paper. Suffice it to say that both
the conditions ``alternating knot'' and ``contains the tangle
corresponding to 5/3'' are expressible. For example, it is possible to
calculate the characteristic formula of a process corresponding to the
tangle 5/3 and conjoin it into the classifying formula via the
composition connective of the logic.

Finally, we wish to observe that it is entirely within reason to
contemplate a more domain-specific version of spatial logic tailored
to the shape of processes in the image of the encoding. Such a
domain-specific logic would have a better claim to the title formal
language of knot properties.

% subsection example_formulae_ (end)

% section knots_as_processes (end) 

% section spatial logic via knots (end)

\section{Conclusions and future work}

\paragraph{Testing physical space}
You, gentle reader, may wonder why of all the theorems to be proved
given this set up we pick the one above. In some sense it's hardly
central to quantum mechanics. We see it as central in the sense that
it firmly establishes a notion of physical space arising from a notion
of the equivalence of behavior. Relating bisimulation to a metric is a
big step forward, but one is faced with interpreting the relationship
of that metric space to something more physical. Quantum mechanical
notions of ``physical'' space are still far from intuitive, but by
relating this idea of distance as testing to calculations that predict
physical circumstances we are making a not insignificant step forward
toward an understanding of the physical space we inhabit as
essentially dynamic.

\paragraph{Effectivity and simulation}
One of the observations we have yet to make is that the entire program
spelled out here is effective. We have built various interpreters for
the reflective calculus at work in this interpretation. In principle,
then, we can simulate quantum mechanics on a computer. The place where
the simulation may lose fidelity is the infinitely branching summation
for the annihilator.

In this connection i also want to point out that the evaluation style
calculation of the inner product puts the non-determinism of the
summation right at the heart of measurement. This suggests that
Milner's original reduction-based formulation of the dynamics of his
calculi in terms of sums was not just notationally suggestive of a
notion of measure-and-continue but captured some significant part of
the physics.

\paragraph{Quantum continuations}
In light of this last observation i want to point out that the
predominant account of quantum mechanics is missing a key aspect of a
truly compositional story of the physical situation. In a real lab,
when a measurement is made the observation can be made to feed into
another device that then makes another measurement conditioned on the
results of the first. This means that after the superposition was
collapsed the entire experimental set up remained in
superposition. While QM offers a means of writing this down it doesn't
quite line up well with the well-trodden formulation of computation
and continuation that we see so succinctly expressed in Milner's
calculi. This suggests that there might be advantages to this account
of dynamics waiting to be explored.

\paragraph{Quantum logic}
In this connection, we also note that by virtue of having the
Hennessy-Milner construction, we can pull the construction through the
interpretation of QM. This gives us a natural candidate for a quantum
logic that enjoys an extremely tight connection with it's domain of
interpretation, making the construction much less ad hoc (rather it is
the image of functor!).

\paragraph{Quantum probabiity}
i have questions about the basis of the interpretation of inner
product as probability amplitude. In particular, using which
axiomatization of probability theory does the notion of probability
amplitude earn the right to be so dubbed? In other words, where is the
proof that the operation for calculating a probability amplitude (and
then squaring) satisfies the axioms of what it means to calculate a
probability? Even if such a proof exists (i have yet to find it in the
literature), i wonder if it might not be possible to turn things on
their heads. Can we view the calculation of the probability amplitude
as an axiomatization of probability? If so, then the definition we
give for calculating probability amplitude may provide the basis for
an \emph{effective} theory of probability.

\paragraph{Quantum vs ``biological'' information}
Finally, i want to conclude with a more philosophical observation. At
a recent workshop in which QM was a predominant topic i noticed
something about quantum information. The speaker was giving a riveting
discussion of axiomatic QM and showing how properties of ``no
cloning'' and ``no deleting'' emerged as consequences of the
axiomatization. Theorems of this form are necessary to give us a sense
of confidence that our axioms characterize the physical theory. What
struck me, though, was that if quantum information is neither erasable
nor replicable it is markedly different from \emph{life}. Two of the
things we know about life is that

\begin{itemize}
  \item it ends;
  \item to gain some measure of persistence, to transcend it's
    finitude it is imminently copyable.
\end{itemize}

Both of these qualities are summarized succinctly in the aphorism: all
flesh is grass. For me these two kinds of ``information'' -- call them
quantum and biological -- are end points on a spectrum of strategies
for persistence. At one end, we have those curious entities that enjoy
uniqueness and permanence; at the other, we have those who in the face
of a certain end and an uncertain present make a go of passing
something on. To me one of the more remarkable aspects of the latter
strategy is that in the presence of noise (and certain features of
copying) we get a kind of dynamism, a chance for improvement against a
given persistent condition.

% subsection other_calculi_other_bisimulations_and_geometry_as_behavior (end)




% section conclusion (end)

%\documentclass[12pt]{llncs}
%\documentclass{jktr}

\usepackage[pdftex]{hyperref}                   
\usepackage {listings}
\usepackage {mathpartir}
\usepackage{bcprules}
%\usepackage{listings}
                       
\usepackage{graphicx} 
%\usepackage[margins=2.5cm,nohead,nofoot]{geometry}
%\usepackage{geometry}
\usepackage{amsfonts}
\usepackage{amstext}
\usepackage{latexsym}
\usepackage{amssymb}
\usepackage{color}


%\include{myPreamble}
\include{qm2pi.local} 

%\ifpdf
%\usepackage[pdftex]{graphicx}
%\else
%\usepackage{graphicx}
%\fi

 % \ifpdf
%  \usepackage{pdfsync}
%  \if


%\title{Brief Article}
%\author{David F. Snyder}
%\author{L.G. Meredith}

%\address{Dept. of Math., Texas State University--San Marcos, San Marcos, TX 78666}
       
\pagestyle{empty}


\begin{document}

\lstset{language=[Objective]Caml,frame=shadowbox}

\input{qm2pi.front}

% section front matter (end)

\input{qm2pi.intro} 
 
% section introduction (end)

% \input{qm2pi.knotations} 

% section notation (end)

\input{qm2pi.process.calculi} 

% section concurrent_process_calculi_and_spatial_logics_ (end)
    
%\input{qm2pi.knots2pi} 

%\input{qm2pi.trefoil} 

%\input{qm2pi.mainthm} 

% subsection basic_interpretation (end)

%\input{qm2pi.rho.presentation} 
\subsection{The syntax and semantics of the notation system}\label{sub:the_syntax_and_semantics_of_the_notation_system} % (fold)

We now summarize a technical presentation of the calculus that
embodies our theory of dynamics. The typical presentation of such a
calculus follows the style of giving generators and relations on
them. The grammar, below, describing term constructors, freely
generates the set of processes, $\Proc$. This set is then quotiented
by a relation known as structural congruence and it is over this set
that the notion of dynamics is expressed. This presentation is
essentially that of \cite{MeredithR05} with the addition of
polyadicity and summation. For readability we have relegated some of
the technical subtleties to an appendix.

\subsubsection{Process grammar}\label{subsub:process_grammar}

\begin{mathpar}
  \inferrule* [lab=synchronization] {} {{M} \bc \pzero \;|\; x?F \;|\; x!C }
  \and
  \inferrule* [lab=abstraction] {} {{F} \bc (x)P}
  \and
  \inferrule* [lab=concretion] {} {{C} \bc \langle Q \rangle}
  \and
  \inferrule* [lab=process] {} {{P,Q} \bc M \;| \;P|Q \;|\; @{x}}
  \and
  \inferrule* [lab=name] {} {{x} \bc \quotep{P}}
\end{mathpar} 

Note that $\vec{x}$ (resp. $\vec{P}$) denotes a vector of names
(resp. processes) of length $|\vec{x}|$ (resp. $|\vec{P}|$). We adopt
the following useful abbreviations.

\begin{mathpar}
   x?(\vec{y}).P := x.(\vec{y})P \and  x\clift{\vec{P}} := x.\clift{\vec{P}}
   \and x!(y) := \lift{x}{\dropn{y}}
   \and \Pi_{i=0}^{n-1}P_i := P_0 | \ldots | P_{n-1}
\end{mathpar}

\subsubsection{Structural congruence}

\paragraph{Free and bound names and alpha-equivalence.} At the
core of structural equivalence is alpha-equivalence which identifies
process that are the same up to a change of variable. Formally, we
recognize the distinction between free and bound names. The free names
of a process, $\freenames{P}$, may be calculated recursively as
follows:

\begin{mathpar}
\freenames{\pzero} := \emptyset
  \and \\
  \freenames{x?(y).P} := \{ x \} \cup (\freenames{P} \setminus \{ y \})
  \and 
  \freenames{x!\langle P \rangle} := \{ x \} \cup \{ P \} 
  \and \\
  \freenames{P|Q} := \freenames{P} \cup \freenames{Q}
  \and \\
  \freenames{@{x}} := \{ x \}
\end{mathpar}

$\pi$
$\quotep{\pi}$

$\freenames{-} : \pi \to \mathcal{P}(\quotep{\pi})$

\begin{eqnarray*}
  \freenames{\pzero} & := & \emptyset \\
  \freenames{x?(y).P} & := & \{ x \} \cup (\freenames{P} \setminus \{ y \}) \\
  \freenames{x!\langle P \rangle} & := & \{ x \} \cup \{ P \} \\
  \freenames{P|Q} & := & \freenames{P} \cup \freenames{Q} \\
  \freenames{\dropn{x}} & := & \{ x \}
\end{eqnarray*}

The bound names of a process, $\boundnames{P}$, are those names occurring in $P$
that are not free. For example, in $x?(y).0$, the name $x$ is free, while $y$ is bound.

\begin{mathpar}
  \inferrule* [lab=monoidal-laws] {} { P|Q \equiv Q|P \and P|0 \equiv P \and P|(Q|R) \equiv (P|Q)|R }
\end{mathpar}

\begin{mathpar}
  \inferrule* [lab=alpha-equivalence] {} { (x)P \equiv (y)P\{y/x\} \and y \not\in \freenames{P} }
\end{mathpar}

\begin{definition}
Then two processes, $P,Q$, are alpha-equivalent if $P = Q\{\vec{y}/\vec{x}\}$ for
some $\vec{x} \in \boundnames{Q},\vec{y} \in \boundnames{P}$, where $Q\{\vec{y}/\vec{x}\}$
denotes the capture-avoiding substitution of $\vec{y}$ for $\vec{x}$ in $Q$.
\end{definition}

\begin{definition}
  The {\em structural congruence} \cite{SangiorgiWalker} , $\equiv$,
  between processes is the least congruence containing
  alpha-equivalence, satisfying the abelian monoid laws
  (associativity, commutativity and $\pzero$ as identity) for parallel
  composition $|$ and for summation $+$.
\end{definition}

\subsection{Name equivalence}

We take name equivalence, written $\nameeq$, to be the smallest
equivalence relation generated by the following rules.

\begin{mathpar}
\inferrule*[lab=Quote-drop]
{ }
{ \quotep{@{x}} \nameeq x }

\inferrule*[lab=Struct-equiv]
{ P \scong Q }
{ \quotep{P} \nameeq \quotep{Q} }
\end{mathpar}

The astute reader will have noticed that the mutual recursion of names
and processes imposes a mutual recursion on alpha-equivalence and
structural equivalence via name-equivalence. Fortunately, all of this
works out pleasantly and we may calculate in the natural way, free of
concern. The reader interested in the details is referred to the
appendix \ref{appendix:rho_details}.

\subsection{Substitution}

We use $\Proc$ for the set of processes, $\QProc$ for the set of
names, and $\id{\{}\vec{y} / \vec{x} \id{\}}$ to denote partial maps,
$s : \QProc \rightarrow \QProc$. A map, $s$ lifts, uniquely, to a map
on process terms, $\widehat{s} : \Proc \rightarrow \Proc$ by the
following equations.

\begin{mathpar}
  (0) \psubstp{Q}{P} := 0 \\
  (R \juxtap S) \psubstp{Q}{P}
  :=    
  (R)\psubstp{Q}{P} \juxtap (S) \psubstp{Q}{P} \\
  (x?(y).R) \psubstp{Q}{P}    
  :=    
  (x)\substp{Q}{P} (z)\concat( (R \psubstn{z}{y}) \psubstp{Q}{P} ) \\
  (\lift{x}{R}) \psubstp{Q}{P}  
  :=
  \lift{(x)\substp{Q}{P}}{ R \psubstp{Q}{P} } \\
%   (\dropn{x})  \psubstp{Q}{P}       
%   := 
%   \left\{ 
%     \begin{array}{ccc} 
%       \dropn{\quotep{Q}} & & x \nameeq \quotep{P} \\
%       \dropn{x} & & otherwise \\
%     \end{array}
%   \right. 
  (\dropn{x})  \psubstp{Q}{P}       
  := 
  \left\{ 
    \begin{array}{ccc} 
      Q & & x \nameeq \quotep{P} \\
      \dropn{x} & & otherwise \\
    \end{array}
  \right.
\end{mathpar}
 

where

\begin{eqnarray}
  (x)\id{\{} \lpquote Q \rpquote / \lpquote P \rpquote \id{\}}            = 
  \left\{ 
    \begin{array}{ccc}
      \lpquote Q \rpquote & & x \nameeq \lpquote P \rpquote \\
      x & & otherwise \\
    \end{array}
  \right. \nonumber
\end{eqnarray}

and $z$ is chosen distinct from $\quotep{P}$, $\quotep{Q}$, the free
names in $Q$, and all the names in $R$. Our $\alpha$-equivalence will
be built in the standard way from this substitution.

\begin{remark}\label{rem:no_self_referential_names}
  One consequence of these definitions is that $\forall P. \quotep{P}
  \not\in \freenames{P}$.
\end{remark}

\subsection{ Dynamic quote: an example }

Anticipating something of what's to come, consider applying the
substitution, $\widehat{\id{\{}u / z \id{\}}}$, to the following pair
of processes, $\lift{w}{y!(z)}$ and $w[ \lpquote y!(z) \rpquote ]$.

\begin{eqnarray}
	\lift{w}{y!(z)}\widehat{\id{\{}u / z \id{\}}}
		& = &
		\lift{w}{y!(u)} \nonumber\\
	w[ \lpquote y!(z) \rpquote ] \widehat{ \id{\{}u / z \id{\}} }
		& = &
		w[ \lpquote y!(z) \rpquote ] \nonumber
\end{eqnarray}

Because the body of the process between quotes is impervious to
substitution, we get radically different answers. In fact, by
examining the first process in an input context,
e.g. $x?(z).\lift{w}{y!(z)}$, we see that the process under the lift
operator may be shaped by prefixed inputs binding a name inside it. In
this sense, the lift operator will be seen as a way to dynamically
construct processes before reifying them as names.

Finally equipped with these standard features we can present the
dynamics of the calculus.

\subsubsection{Operational semantics} 

Finally, we introduce the computational dynamics. What marks these
algebras as distinct from other more traditionally studied algebraic
structures, e.g. vector spaces or polynomial rings, is the manner in
which dynamics is captured. In traditional structures, dynamics is typically
expressed through morphisms between such structures, as in linear maps
between vector spaces or morphisms between rings. In algebras
associated with the semantics of computation, the dynamics is
expressed as part of the algebraic structure itself, through a
reduction reduction relation typically denoted by $\red$. Below, we
give a recursive presentation of this relation for the calculus used
in the encoding.

$\red \subseteq \pi \times \pi$
$\red : \pi \to \mathcal{P}(\pi)$

\begin{mathpar}
  \inferrule* [lab=Comm] { \textsf{match}( x_{src}, x_{trgt} ) } { x_{trgt}?(y)P \; | \; x_{src}!\langle {Q} \rangle \red P\{\quotep{Q}/y}\} }
  \and \\
  \inferrule* [lab=Par] {{P} \red {P}'} {{{P} | {Q}} \red {{P}' | {Q}}}
  \and
  \inferrule* [lab=Equiv]{{{P} \scong {P}'} \andalso {{P}' \red {Q}'} \andalso {{Q}' \scong {Q}}}{{P} \red {Q}}
\end{mathpar}

\begin{eqnarray*}
  match_{\equiv} (\quotep{P},\quotep{Q}) & := & P \equiv Q \\
  match_{\dagger}(\quotep{P},\quotep{Q}) & := & \forall R. P|Q \red^{*} R => R \red^{*} 0 \\
  match_{K}(\quotep{P},\quotep{Q}) & := & K \mbox{ for some context } K
\end{eqnarray*}

$u?(x)P | u!\langle Q \rangle \red P\{\quotep{Q}/x\}$

%We write $\wred$ for $\red^*$, and $P\red$ if $\exists Q $ such that $ P \red Q$.
We write $P\red$ if $\exists Q $ such that $ P \red Q$ and $P\not\red$, otherwise.

\section{Replication}

As mentioned before, it is known that replication (and hence
recursion) can be implemented in a higher-order process algebra
\cite{SangiorgiWalker}. As our first example of calculation with the
machinery thus far presented we give the construction explicitly in
the {\rhoc}.

\begin{eqnarray}
	D_{x} & := & \prefix{x}{y}{(\binpar{\outputp{x}{y}}{@{y}})} \nonumber\\
	\bangp_{x}{P} & := & \binpar{{x}!\langle{\binpar{D_{x}}{P}}\rangle}{D_{x}} \nonumber
\end{eqnarray}

\begin{eqnarray}
	\bangp_{x}{P} & & \nonumber\\
	=
	& {x}!\langle{(\prefix{x}{y}{(\outputp{x}{y} | @{y})) | P}}\rangle 
	      | \prefix{x}{y}{(\outputp{x}{y} | @{y})} & \nonumber\\
	\red
	& (\outputp{x}{y} | @{y})\substn{\quotep{(\prefix{x}{y}{(@{y} | \outputp{x}{y})) | P}}}{y} & \nonumber\\
	=
	& \outputp{x}{\quotep{(\prefix{x}{y}{(\outputp{x}{y} | @{y})) | P}}}
	  | {(\prefix{x}{y}{(\outputp{x}{y} | @{y})) | P}} & \nonumber\\
	\red
	& \ldots & \nonumber\\
	\red^*
	& P | P | \ldots & \nonumber
\end{eqnarray}

Of course, this encoding, as an implementation, runs away, unfolding
$\bangp{P}$ eagerly. A lazier and more implementable replication
operator, restricted to input-guarded processes, may be obtained as follows.

\begin{eqnarray}
\bangp{\prefix{u}{v}{P}} 
	:= 
	\binpar{\lift{x}{\prefix{u}{v}{(\binpar{D(x)}{P})}}}{D(x)} \nonumber
\end{eqnarray}

\begin{remark}
  Note that the lazier definition still does not deal with summation
  or mixed summation (i.e. sums over input and output). The reader is
  invited to construct definitions of replication that deal with these
  features. 

  Further, the definitions are parameterized in a name, $x$. Can you,
  gentle reader, make a definition that eliminates this parameter and
  guarantees no accidental interaction between the replication
  machinery and the process being replicated -- i.e. no accidental
  sharing of names used by the process to get its work done and the
  name(s) used by the replication to effect copying. This latter
  revision of the definition of replication is crucial to obtaining
  the expected identity $!!P \sim !P$.
\end{remark}

\begin{remark}\label{rem:paradoxical_combinator}
  The reader familiar with the lambda calculus will have noticed the
  similarity between $D$ and the paradoxical combinator.

  [Ed. note: the existence of this seems to suggest we have to be more
  restrictive on the set of processes and names we admit if we are to
  support no-cloning.]
\end{remark}

\subsubsection{Bisimulation}

The computational dynamics gives rise to another kind of equivalence,
the equivalence of computational behavior. As previously mentioned
this is typically captured \emph{via} some form of bisimulation.

% The notion we use in this paper is weak barbed bisimulation
% \cite{milner91polyadicpi}.

The notion we use in this paper is derived from weak barbed
bisimulation \cite{milner91polyadicpi}. 

\begin{definition}
An \emph{observation relation}, $\downarrow_{\mathcal N}$, over a set
of names, $\mathcal N$, is the smallest relation satisfying the rules
below.

\infrule[Out-barb]{y \in {\mathcal N}, \; x \nameeq y}
		  {\outputp{x}{v} \downarrow_{\mathcal N} x}
\infrule[Par-barb]{\mbox{$P\downarrow_{\mathcal N} x$ or $Q\downarrow_{\mathcal N} x$}}
		  {\binpar{P}{Q} \downarrow_{\mathcal N} x}

We write $P \Downarrow_{\mathcal N} x$ if there is $Q$ such that 
$P \wred Q$ and $Q \downarrow_{\mathcal N} x$.
\end{definition}

\begin{definition}
%\label{def.bbisim}
An  ${\mathcal N}$-\emph{barbed bisimulation} over a set of names, ${\mathcal N}$, is a symmetric binary relation 
${\mathcal S}_{\mathcal N}$ between agents such that $P\rel{S}_{\mathcal N}Q$ implies:
\begin{enumerate}
\item If $P \red P'$ then $Q \wred Q'$ and $P'\rel{S}_{\mathcal N} Q'$.
\item If $P\downarrow_{\mathcal N} x$, then $Q\Downarrow_{\mathcal N} x$.
\end{enumerate}
$P$ is ${\mathcal N}$-barbed bisimilar to $Q$, written
$P \wbbisim_{\mathcal N} Q$, if $P \rel{S}_{\mathcal N} Q$ for some ${\mathcal N}$-barbed bisimulation ${\mathcal S}_{\mathcal N}$.
\end{definition}

$\mathcal{R} \subseteq \pi \times \pi$

$P \mathcal{R} Q => \forall P'. P \red P' \Rightarrow \exists Q'. Q \red Q', P' \mathcal{R} Q'$

$P \vdash x \Rightarrow Q \vdash x$

\begin{mathpar}
  \inferrule*[lab=Out-barb]{x \nameeq y}{{y}!\langle{Q}\rangle \vdash x}
  \and
  \inferrule*[lab=Par-barb]{\mbox{$P\vdash x$ or $Q\vdash x$}}{\binpar{P}{Q} \vdash x}
\end{mathpar}

\subsubsection{Contexts}

One of the principle advantages of computational calculi like the
$\pi$-calculus is a well-defined notion of context,
contextual-equivalence and a correlation between
contextual-equivalence and notions of bisimulation. The notion of
context allows the decomposition of a process into (sub-)process and
its syntactic environment, its context. Thus, a context may be
thought of as a process with a ``hole'' (written $\Box$) in it. The
application of a context $M$ to a process $P$, written $M[P]$, is
tantamount to filling the hole in $M$ with $P$. In this paper we do
not need the full weight of this theory, but do make use of the notion
of context in the proof the main theorem. 

\begin{mathpar}
  \inferrule* [lab=summation] {} {{M_{M},M_{N}} \bc \Box \;|\; x.M_{A} \;|\; M_{M}+M_{N}}
  \and
  \inferrule* [lab=agent] {} {{M_{A}} \bc (\vec{x})M_{P} \;| \; \clift{P_0,\ldots,M_{P},\ldots,P_N}}
  \and \\
  \inferrule* [lab=process] {} {{M_{P}} \bc M_{N} \;| \;P|M_{P} }
\end{mathpar} 

\begin{mathpar}
  \inferrule* [lab=sychronization] {} {M_{N} \bc \Box \;|\; x?M_{F} \;|\; x!M_{C}}
  \and
  \inferrule* [lab=abstraction] {} {{M_{F}} \bc (x)M_{P} }
  \and
  \inferrule* [lab=concretion] {} {{M_{C}} \bc \langle M_{P} \rangle }
  \and \\
  \inferrule* [lab=process] {} {{M_{P}} \bc M_{N} \;| \;P|M_{P} }
\end{mathpar}

\begin{definition}[contextual application] Given a context $M$, and
  process $P$, we define the \emph{contextual application}, $M[P] :=
  M\{P/\Box\}$. That is, the contextual application of M to P is the
  substitution of $P$ for $\Box$ in $M$.
\end{definition}

$\meaningof{-} : L \to \mathcal{P}(\pi)$

\begin{mathpar}
  \inferrule* [lab=collection] {} {\meaningof{true} = \pi, \and \meaningof{~E} = \pi \setminus \meaningof{E}, \and \meaningof{E_{1} \& E_{2}} = \meaningof{E_{1}} \cap \meaningof{E_{2}}}
\end{mathpar}

\begin{mathpar}
  \inferrule* [lab=structure] {} {\meaningof{0} = \{ P \in \pi | P \equiv 0 \}, \and \\ \meaningof{E_1 | E_2} = \{ P \in \pi | P \equiv P_{1} | P_{2}, P_{1} \in \meaningof{E_{1}}, P_{2} \in \meaningof{E_2}\} }
\end{mathpar}

\begin{mathpar}
 \inferrule* [lab=behavior] {} {\meaningof{\langle a?b \rangle E} = \{ P \in \pi | P \equiv Q | u?(y)P', \\ \and \\\\ \and \\ \;\;\; u \in \meaningof{a}, \forall z.P'\{z/y\} \in \meaningof{E\{z/b\}}\}, \and \\ \meaningof{a!E} = \{ P \in \pi | P \equiv Q | x!\langle P' \rangle, x \in \meaningof{a} P' \in \meaningof{E}\} }
\end{mathpar}

\begin{mathpar}
 \inferrule* [lab=nominal] {} {\meaningof{\quotep{E}} = \{ \quotep{P} \in \quotep{\pi} | P \in \meaningof{E} \}, \and \meaningof{\quotep{P}} = \{ \quotep{Q} \in \quotep{\pi} | P \equiv Q \} \and \\ \meaningof{@\quotep{E}} = \{ P \in \pi | P \equiv @x, x \in \meaningof{E} \}}
\end{mathpar}

\begin{eqnarray*}
  \\
  \meaningof{-} : TS \to ST
\end{eqnarray*}

\begin{eqnarray*}
  \\
  L : TS \to ST
\end{eqnarray*}

\begin{eqnarray*}
  \\
  P \models E \iff P \in \meaningof{E}
\end{eqnarray*}

\begin{eqnarray*}
  P \approx_{L} Q \iff \forall E \in L. P \models E \iff Q \models E
\end{eqnarray*}

\begin{eqnarray*}
  P \approx_{K} Q
\end{eqnarray*}

\begin{eqnarray*}
  P \approx Q
\end{eqnarray*}

$\approx_{K} = \approx = \approx_{L}$

\subsubsection{Contextual duality}

Note that contexts extend the quotation operation to a family of
operations from processes to names. Given a context, $M$, we can
define a \emph{nominal context}, $\quotep{M}$ by $\quotep{M}[P] :=
\quotep{M[P]}$. To foreshadow what is to come we observe that these
operations enjoy a duality with processes very much like the duality
between vectors and maps from vectors to scalars.

Further, because the calculus is essentially higher-order, we have a
correspondence between contexts and processes. More specifically,
given a name $x$ and a context $M$ we can construct $M^{*}_{x}$ such
that 

\begin{mathpar}
  M^{*}_{x} | \lift{x}{P} \red M[P]
\end{mathpar}

namely,

\begin{mathpar}
  M^{*}_{x} := x?(u).M[\dropn{u}]
\end{mathpar}

The dependence of $M^{*}_{x}$ on a name makes it an abstraction, 

\begin{mathpar}
  M^{*} := (x)x?(u).M[\dropn{u}]
\end{mathpar}

\subsection{Additional notation}

It will sometimes be convenient to denote the process a name
quotes. We already have the notation $x = \quotep{P}$, but it will be
convenient to introduce an alternate notation, $\procn{x}$, when we
want to emphasize the connection to the use of the name. Note that, by
virtue of name equivalence, $\quotep{\procn{x}} \nameeq x$; so, the
notation is consistent with previous definitions.

Further, because names have structure it is possible to effect
substitutions on the basis of that structure. This means we need to
upgrade our notation for substitutions, which we accomplish by
adapting comprehension notation. Thus,

\begin{mathpar}
  P\{ y / x : x \in S \}
\end{mathpar}

is interpreted to mean the process derived from P by replacing (in a
capture-avoiding manner) each occurrence of $x$ in $S$ by $y$. For example,

\begin{mathpar}
  P\{ \quotep{\procn{x}|\procn{x}} / x : x \in \freenames{P} \}
\end{mathpar}

will replace each (occurrence) of a free name $x$ in $P$ by
$\quotep{\procn{x}|\procn{x}}$.

Also, we will avail ourselves of the notation $x^{L}$ and $x^{R}$ to
denote injections of a name into disjoint copies of the name
space. There are numerous ways to accomplish this. One example can be
found in \cite{MeredithR05}. This notation overloads to vectors of
names: $\vec{x}^{\pi} := (x_{i}^{\pi} \; : \; 0 \leq i < |\vec{x}| )$ where $\pi \in \{L,R\}$.

We also use $P^{\Box} := P|\Box$.

In \cite{MeredithR05} an interpretation of the new operator is
given. It turns out that there are several possible interpretations
all enjoying the requisite algebraic properties of the operator (see
\cite{milner91polyadicpi}). We will therefore make liberal use of
$(\nu\; \vec{x})P$.

% subsection the_syntax_and_semantics_of_the_notation_system (end)   

\input{qm2pi.qmops} 

\input{qm2pi.sterngerlach} 

\input{qm2pi.metric} 

% section concurrent_process_calculi (end)

%\input{qm2pi.proofsketch}

% section proof sketch (end)

%\input{qm2pi.slviaknots} 

% section spatial logic via knots (end)

\input{qm2pi.conclusion}

% section conclusion (end)

%\input{qm2pi.dtcodes} 

% section wiring algorithm (end)

\input{qm2pi.ack} 

% section acknowledgments (end)

\newpage


\bibliographystyle{plain}   
\bibliography{../../biblios/main.bib}

\input{qm2pi.rhodetails}

\end{document}

 

% section wiring algorithm (end)

\documentclass[12pt]{llncs}
%\documentclass{jktr}

\usepackage[pdftex]{hyperref}                   
\usepackage {listings}
\usepackage {mathpartir}
\usepackage{bcprules}
%\usepackage{listings}
                       
\usepackage{graphicx} 
%\usepackage[margins=2.5cm,nohead,nofoot]{geometry}
%\usepackage{geometry}
\usepackage{amsfonts}
\usepackage{amstext}
\usepackage{latexsym}
\usepackage{amssymb}
\usepackage{color}


%\include{myPreamble}
\include{qm2pi.local} 

%\ifpdf
%\usepackage[pdftex]{graphicx}
%\else
%\usepackage{graphicx}
%\fi

 % \ifpdf
%  \usepackage{pdfsync}
%  \if


%\title{Brief Article}
%\author{David F. Snyder}
%\author{L.G. Meredith}

%\address{Dept. of Math., Texas State University--San Marcos, San Marcos, TX 78666}
       
\pagestyle{empty}


\begin{document}

\lstset{language=[Objective]Caml,frame=shadowbox}

\input{qm2pi.front}

% section front matter (end)

\input{qm2pi.intro} 
 
% section introduction (end)

% \input{qm2pi.knotations} 

% section notation (end)

\input{qm2pi.process.calculi} 

% section concurrent_process_calculi_and_spatial_logics_ (end)
    
%\input{qm2pi.knots2pi} 

%\input{qm2pi.trefoil} 

%\input{qm2pi.mainthm} 

% subsection basic_interpretation (end)

%\input{qm2pi.rho.presentation} 
\subsection{The syntax and semantics of the notation system}\label{sub:the_syntax_and_semantics_of_the_notation_system} % (fold)

We now summarize a technical presentation of the calculus that
embodies our theory of dynamics. The typical presentation of such a
calculus follows the style of giving generators and relations on
them. The grammar, below, describing term constructors, freely
generates the set of processes, $\Proc$. This set is then quotiented
by a relation known as structural congruence and it is over this set
that the notion of dynamics is expressed. This presentation is
essentially that of \cite{MeredithR05} with the addition of
polyadicity and summation. For readability we have relegated some of
the technical subtleties to an appendix.

\subsubsection{Process grammar}\label{subsub:process_grammar}

\begin{mathpar}
  \inferrule* [lab=synchronization] {} {{M} \bc \pzero \;|\; x?F \;|\; x!C }
  \and
  \inferrule* [lab=abstraction] {} {{F} \bc (x)P}
  \and
  \inferrule* [lab=concretion] {} {{C} \bc \langle Q \rangle}
  \and
  \inferrule* [lab=process] {} {{P,Q} \bc M \;| \;P|Q \;|\; @{x}}
  \and
  \inferrule* [lab=name] {} {{x} \bc \quotep{P}}
\end{mathpar} 

Note that $\vec{x}$ (resp. $\vec{P}$) denotes a vector of names
(resp. processes) of length $|\vec{x}|$ (resp. $|\vec{P}|$). We adopt
the following useful abbreviations.

\begin{mathpar}
   x?(\vec{y}).P := x.(\vec{y})P \and  x\clift{\vec{P}} := x.\clift{\vec{P}}
   \and x!(y) := \lift{x}{\dropn{y}}
   \and \Pi_{i=0}^{n-1}P_i := P_0 | \ldots | P_{n-1}
\end{mathpar}

\subsubsection{Structural congruence}

\paragraph{Free and bound names and alpha-equivalence.} At the
core of structural equivalence is alpha-equivalence which identifies
process that are the same up to a change of variable. Formally, we
recognize the distinction between free and bound names. The free names
of a process, $\freenames{P}$, may be calculated recursively as
follows:

\begin{mathpar}
\freenames{\pzero} := \emptyset
  \and \\
  \freenames{x?(y).P} := \{ x \} \cup (\freenames{P} \setminus \{ y \})
  \and 
  \freenames{x!\langle P \rangle} := \{ x \} \cup \{ P \} 
  \and \\
  \freenames{P|Q} := \freenames{P} \cup \freenames{Q}
  \and \\
  \freenames{@{x}} := \{ x \}
\end{mathpar}

$\pi$
$\quotep{\pi}$

$\freenames{-} : \pi \to \mathcal{P}(\quotep{\pi})$

\begin{eqnarray*}
  \freenames{\pzero} & := & \emptyset \\
  \freenames{x?(y).P} & := & \{ x \} \cup (\freenames{P} \setminus \{ y \}) \\
  \freenames{x!\langle P \rangle} & := & \{ x \} \cup \{ P \} \\
  \freenames{P|Q} & := & \freenames{P} \cup \freenames{Q} \\
  \freenames{\dropn{x}} & := & \{ x \}
\end{eqnarray*}

The bound names of a process, $\boundnames{P}$, are those names occurring in $P$
that are not free. For example, in $x?(y).0$, the name $x$ is free, while $y$ is bound.

\begin{mathpar}
  \inferrule* [lab=monoidal-laws] {} { P|Q \equiv Q|P \and P|0 \equiv P \and P|(Q|R) \equiv (P|Q)|R }
\end{mathpar}

\begin{mathpar}
  \inferrule* [lab=alpha-equivalence] {} { (x)P \equiv (y)P\{y/x\} \and y \not\in \freenames{P} }
\end{mathpar}

\begin{definition}
Then two processes, $P,Q$, are alpha-equivalent if $P = Q\{\vec{y}/\vec{x}\}$ for
some $\vec{x} \in \boundnames{Q},\vec{y} \in \boundnames{P}$, where $Q\{\vec{y}/\vec{x}\}$
denotes the capture-avoiding substitution of $\vec{y}$ for $\vec{x}$ in $Q$.
\end{definition}

\begin{definition}
  The {\em structural congruence} \cite{SangiorgiWalker} , $\equiv$,
  between processes is the least congruence containing
  alpha-equivalence, satisfying the abelian monoid laws
  (associativity, commutativity and $\pzero$ as identity) for parallel
  composition $|$ and for summation $+$.
\end{definition}

\subsection{Name equivalence}

We take name equivalence, written $\nameeq$, to be the smallest
equivalence relation generated by the following rules.

\begin{mathpar}
\inferrule*[lab=Quote-drop]
{ }
{ \quotep{@{x}} \nameeq x }

\inferrule*[lab=Struct-equiv]
{ P \scong Q }
{ \quotep{P} \nameeq \quotep{Q} }
\end{mathpar}

The astute reader will have noticed that the mutual recursion of names
and processes imposes a mutual recursion on alpha-equivalence and
structural equivalence via name-equivalence. Fortunately, all of this
works out pleasantly and we may calculate in the natural way, free of
concern. The reader interested in the details is referred to the
appendix \ref{appendix:rho_details}.

\subsection{Substitution}

We use $\Proc$ for the set of processes, $\QProc$ for the set of
names, and $\id{\{}\vec{y} / \vec{x} \id{\}}$ to denote partial maps,
$s : \QProc \rightarrow \QProc$. A map, $s$ lifts, uniquely, to a map
on process terms, $\widehat{s} : \Proc \rightarrow \Proc$ by the
following equations.

\begin{mathpar}
  (0) \psubstp{Q}{P} := 0 \\
  (R \juxtap S) \psubstp{Q}{P}
  :=    
  (R)\psubstp{Q}{P} \juxtap (S) \psubstp{Q}{P} \\
  (x?(y).R) \psubstp{Q}{P}    
  :=    
  (x)\substp{Q}{P} (z)\concat( (R \psubstn{z}{y}) \psubstp{Q}{P} ) \\
  (\lift{x}{R}) \psubstp{Q}{P}  
  :=
  \lift{(x)\substp{Q}{P}}{ R \psubstp{Q}{P} } \\
%   (\dropn{x})  \psubstp{Q}{P}       
%   := 
%   \left\{ 
%     \begin{array}{ccc} 
%       \dropn{\quotep{Q}} & & x \nameeq \quotep{P} \\
%       \dropn{x} & & otherwise \\
%     \end{array}
%   \right. 
  (\dropn{x})  \psubstp{Q}{P}       
  := 
  \left\{ 
    \begin{array}{ccc} 
      Q & & x \nameeq \quotep{P} \\
      \dropn{x} & & otherwise \\
    \end{array}
  \right.
\end{mathpar}
 

where

\begin{eqnarray}
  (x)\id{\{} \lpquote Q \rpquote / \lpquote P \rpquote \id{\}}            = 
  \left\{ 
    \begin{array}{ccc}
      \lpquote Q \rpquote & & x \nameeq \lpquote P \rpquote \\
      x & & otherwise \\
    \end{array}
  \right. \nonumber
\end{eqnarray}

and $z$ is chosen distinct from $\quotep{P}$, $\quotep{Q}$, the free
names in $Q$, and all the names in $R$. Our $\alpha$-equivalence will
be built in the standard way from this substitution.

\begin{remark}\label{rem:no_self_referential_names}
  One consequence of these definitions is that $\forall P. \quotep{P}
  \not\in \freenames{P}$.
\end{remark}

\subsection{ Dynamic quote: an example }

Anticipating something of what's to come, consider applying the
substitution, $\widehat{\id{\{}u / z \id{\}}}$, to the following pair
of processes, $\lift{w}{y!(z)}$ and $w[ \lpquote y!(z) \rpquote ]$.

\begin{eqnarray}
	\lift{w}{y!(z)}\widehat{\id{\{}u / z \id{\}}}
		& = &
		\lift{w}{y!(u)} \nonumber\\
	w[ \lpquote y!(z) \rpquote ] \widehat{ \id{\{}u / z \id{\}} }
		& = &
		w[ \lpquote y!(z) \rpquote ] \nonumber
\end{eqnarray}

Because the body of the process between quotes is impervious to
substitution, we get radically different answers. In fact, by
examining the first process in an input context,
e.g. $x?(z).\lift{w}{y!(z)}$, we see that the process under the lift
operator may be shaped by prefixed inputs binding a name inside it. In
this sense, the lift operator will be seen as a way to dynamically
construct processes before reifying them as names.

Finally equipped with these standard features we can present the
dynamics of the calculus.

\subsubsection{Operational semantics} 

Finally, we introduce the computational dynamics. What marks these
algebras as distinct from other more traditionally studied algebraic
structures, e.g. vector spaces or polynomial rings, is the manner in
which dynamics is captured. In traditional structures, dynamics is typically
expressed through morphisms between such structures, as in linear maps
between vector spaces or morphisms between rings. In algebras
associated with the semantics of computation, the dynamics is
expressed as part of the algebraic structure itself, through a
reduction reduction relation typically denoted by $\red$. Below, we
give a recursive presentation of this relation for the calculus used
in the encoding.

$\red \subseteq \pi \times \pi$
$\red : \pi \to \mathcal{P}(\pi)$

\begin{mathpar}
  \inferrule* [lab=Comm] { \textsf{match}( x_{src}, x_{trgt} ) } { x_{trgt}?(y)P \; | \; x_{src}!\langle {Q} \rangle \red P\{\quotep{Q}/y}\} }
  \and \\
  \inferrule* [lab=Par] {{P} \red {P}'} {{{P} | {Q}} \red {{P}' | {Q}}}
  \and
  \inferrule* [lab=Equiv]{{{P} \scong {P}'} \andalso {{P}' \red {Q}'} \andalso {{Q}' \scong {Q}}}{{P} \red {Q}}
\end{mathpar}

\begin{eqnarray*}
  match_{\equiv} (\quotep{P},\quotep{Q}) & := & P \equiv Q \\
  match_{\dagger}(\quotep{P},\quotep{Q}) & := & \forall R. P|Q \red^{*} R => R \red^{*} 0 \\
  match_{K}(\quotep{P},\quotep{Q}) & := & K \mbox{ for some context } K
\end{eqnarray*}

$u?(x)P | u!\langle Q \rangle \red P\{\quotep{Q}/x\}$

%We write $\wred$ for $\red^*$, and $P\red$ if $\exists Q $ such that $ P \red Q$.
We write $P\red$ if $\exists Q $ such that $ P \red Q$ and $P\not\red$, otherwise.

\section{Replication}

As mentioned before, it is known that replication (and hence
recursion) can be implemented in a higher-order process algebra
\cite{SangiorgiWalker}. As our first example of calculation with the
machinery thus far presented we give the construction explicitly in
the {\rhoc}.

\begin{eqnarray}
	D_{x} & := & \prefix{x}{y}{(\binpar{\outputp{x}{y}}{@{y}})} \nonumber\\
	\bangp_{x}{P} & := & \binpar{{x}!\langle{\binpar{D_{x}}{P}}\rangle}{D_{x}} \nonumber
\end{eqnarray}

\begin{eqnarray}
	\bangp_{x}{P} & & \nonumber\\
	=
	& {x}!\langle{(\prefix{x}{y}{(\outputp{x}{y} | @{y})) | P}}\rangle 
	      | \prefix{x}{y}{(\outputp{x}{y} | @{y})} & \nonumber\\
	\red
	& (\outputp{x}{y} | @{y})\substn{\quotep{(\prefix{x}{y}{(@{y} | \outputp{x}{y})) | P}}}{y} & \nonumber\\
	=
	& \outputp{x}{\quotep{(\prefix{x}{y}{(\outputp{x}{y} | @{y})) | P}}}
	  | {(\prefix{x}{y}{(\outputp{x}{y} | @{y})) | P}} & \nonumber\\
	\red
	& \ldots & \nonumber\\
	\red^*
	& P | P | \ldots & \nonumber
\end{eqnarray}

Of course, this encoding, as an implementation, runs away, unfolding
$\bangp{P}$ eagerly. A lazier and more implementable replication
operator, restricted to input-guarded processes, may be obtained as follows.

\begin{eqnarray}
\bangp{\prefix{u}{v}{P}} 
	:= 
	\binpar{\lift{x}{\prefix{u}{v}{(\binpar{D(x)}{P})}}}{D(x)} \nonumber
\end{eqnarray}

\begin{remark}
  Note that the lazier definition still does not deal with summation
  or mixed summation (i.e. sums over input and output). The reader is
  invited to construct definitions of replication that deal with these
  features. 

  Further, the definitions are parameterized in a name, $x$. Can you,
  gentle reader, make a definition that eliminates this parameter and
  guarantees no accidental interaction between the replication
  machinery and the process being replicated -- i.e. no accidental
  sharing of names used by the process to get its work done and the
  name(s) used by the replication to effect copying. This latter
  revision of the definition of replication is crucial to obtaining
  the expected identity $!!P \sim !P$.
\end{remark}

\begin{remark}\label{rem:paradoxical_combinator}
  The reader familiar with the lambda calculus will have noticed the
  similarity between $D$ and the paradoxical combinator.

  [Ed. note: the existence of this seems to suggest we have to be more
  restrictive on the set of processes and names we admit if we are to
  support no-cloning.]
\end{remark}

\subsubsection{Bisimulation}

The computational dynamics gives rise to another kind of equivalence,
the equivalence of computational behavior. As previously mentioned
this is typically captured \emph{via} some form of bisimulation.

% The notion we use in this paper is weak barbed bisimulation
% \cite{milner91polyadicpi}.

The notion we use in this paper is derived from weak barbed
bisimulation \cite{milner91polyadicpi}. 

\begin{definition}
An \emph{observation relation}, $\downarrow_{\mathcal N}$, over a set
of names, $\mathcal N$, is the smallest relation satisfying the rules
below.

\infrule[Out-barb]{y \in {\mathcal N}, \; x \nameeq y}
		  {\outputp{x}{v} \downarrow_{\mathcal N} x}
\infrule[Par-barb]{\mbox{$P\downarrow_{\mathcal N} x$ or $Q\downarrow_{\mathcal N} x$}}
		  {\binpar{P}{Q} \downarrow_{\mathcal N} x}

We write $P \Downarrow_{\mathcal N} x$ if there is $Q$ such that 
$P \wred Q$ and $Q \downarrow_{\mathcal N} x$.
\end{definition}

\begin{definition}
%\label{def.bbisim}
An  ${\mathcal N}$-\emph{barbed bisimulation} over a set of names, ${\mathcal N}$, is a symmetric binary relation 
${\mathcal S}_{\mathcal N}$ between agents such that $P\rel{S}_{\mathcal N}Q$ implies:
\begin{enumerate}
\item If $P \red P'$ then $Q \wred Q'$ and $P'\rel{S}_{\mathcal N} Q'$.
\item If $P\downarrow_{\mathcal N} x$, then $Q\Downarrow_{\mathcal N} x$.
\end{enumerate}
$P$ is ${\mathcal N}$-barbed bisimilar to $Q$, written
$P \wbbisim_{\mathcal N} Q$, if $P \rel{S}_{\mathcal N} Q$ for some ${\mathcal N}$-barbed bisimulation ${\mathcal S}_{\mathcal N}$.
\end{definition}

$\mathcal{R} \subseteq \pi \times \pi$

$P \mathcal{R} Q => \forall P'. P \red P' \Rightarrow \exists Q'. Q \red Q', P' \mathcal{R} Q'$

$P \vdash x \Rightarrow Q \vdash x$

\begin{mathpar}
  \inferrule*[lab=Out-barb]{x \nameeq y}{{y}!\langle{Q}\rangle \vdash x}
  \and
  \inferrule*[lab=Par-barb]{\mbox{$P\vdash x$ or $Q\vdash x$}}{\binpar{P}{Q} \vdash x}
\end{mathpar}

\subsubsection{Contexts}

One of the principle advantages of computational calculi like the
$\pi$-calculus is a well-defined notion of context,
contextual-equivalence and a correlation between
contextual-equivalence and notions of bisimulation. The notion of
context allows the decomposition of a process into (sub-)process and
its syntactic environment, its context. Thus, a context may be
thought of as a process with a ``hole'' (written $\Box$) in it. The
application of a context $M$ to a process $P$, written $M[P]$, is
tantamount to filling the hole in $M$ with $P$. In this paper we do
not need the full weight of this theory, but do make use of the notion
of context in the proof the main theorem. 

\begin{mathpar}
  \inferrule* [lab=summation] {} {{M_{M},M_{N}} \bc \Box \;|\; x.M_{A} \;|\; M_{M}+M_{N}}
  \and
  \inferrule* [lab=agent] {} {{M_{A}} \bc (\vec{x})M_{P} \;| \; \clift{P_0,\ldots,M_{P},\ldots,P_N}}
  \and \\
  \inferrule* [lab=process] {} {{M_{P}} \bc M_{N} \;| \;P|M_{P} }
\end{mathpar} 

\begin{mathpar}
  \inferrule* [lab=sychronization] {} {M_{N} \bc \Box \;|\; x?M_{F} \;|\; x!M_{C}}
  \and
  \inferrule* [lab=abstraction] {} {{M_{F}} \bc (x)M_{P} }
  \and
  \inferrule* [lab=concretion] {} {{M_{C}} \bc \langle M_{P} \rangle }
  \and \\
  \inferrule* [lab=process] {} {{M_{P}} \bc M_{N} \;| \;P|M_{P} }
\end{mathpar}

\begin{definition}[contextual application] Given a context $M$, and
  process $P$, we define the \emph{contextual application}, $M[P] :=
  M\{P/\Box\}$. That is, the contextual application of M to P is the
  substitution of $P$ for $\Box$ in $M$.
\end{definition}

$\meaningof{-} : L \to \mathcal{P}(\pi)$

\begin{mathpar}
  \inferrule* [lab=collection] {} {\meaningof{true} = \pi, \and \meaningof{~E} = \pi \setminus \meaningof{E}, \and \meaningof{E_{1} \& E_{2}} = \meaningof{E_{1}} \cap \meaningof{E_{2}}}
\end{mathpar}

\begin{mathpar}
  \inferrule* [lab=structure] {} {\meaningof{0} = \{ P \in \pi | P \equiv 0 \}, \and \\ \meaningof{E_1 | E_2} = \{ P \in \pi | P \equiv P_{1} | P_{2}, P_{1} \in \meaningof{E_{1}}, P_{2} \in \meaningof{E_2}\} }
\end{mathpar}

\begin{mathpar}
 \inferrule* [lab=behavior] {} {\meaningof{\langle a?b \rangle E} = \{ P \in \pi | P \equiv Q | u?(y)P', \\ \and \\\\ \and \\ \;\;\; u \in \meaningof{a}, \forall z.P'\{z/y\} \in \meaningof{E\{z/b\}}\}, \and \\ \meaningof{a!E} = \{ P \in \pi | P \equiv Q | x!\langle P' \rangle, x \in \meaningof{a} P' \in \meaningof{E}\} }
\end{mathpar}

\begin{mathpar}
 \inferrule* [lab=nominal] {} {\meaningof{\quotep{E}} = \{ \quotep{P} \in \quotep{\pi} | P \in \meaningof{E} \}, \and \meaningof{\quotep{P}} = \{ \quotep{Q} \in \quotep{\pi} | P \equiv Q \} \and \\ \meaningof{@\quotep{E}} = \{ P \in \pi | P \equiv @x, x \in \meaningof{E} \}}
\end{mathpar}

\begin{eqnarray*}
  \\
  \meaningof{-} : TS \to ST
\end{eqnarray*}

\begin{eqnarray*}
  \\
  L : TS \to ST
\end{eqnarray*}

\begin{eqnarray*}
  \\
  P \models E \iff P \in \meaningof{E}
\end{eqnarray*}

\begin{eqnarray*}
  P \approx_{L} Q \iff \forall E \in L. P \models E \iff Q \models E
\end{eqnarray*}

\begin{eqnarray*}
  P \approx_{K} Q
\end{eqnarray*}

\begin{eqnarray*}
  P \approx Q
\end{eqnarray*}

$\approx_{K} = \approx = \approx_{L}$

\subsubsection{Contextual duality}

Note that contexts extend the quotation operation to a family of
operations from processes to names. Given a context, $M$, we can
define a \emph{nominal context}, $\quotep{M}$ by $\quotep{M}[P] :=
\quotep{M[P]}$. To foreshadow what is to come we observe that these
operations enjoy a duality with processes very much like the duality
between vectors and maps from vectors to scalars.

Further, because the calculus is essentially higher-order, we have a
correspondence between contexts and processes. More specifically,
given a name $x$ and a context $M$ we can construct $M^{*}_{x}$ such
that 

\begin{mathpar}
  M^{*}_{x} | \lift{x}{P} \red M[P]
\end{mathpar}

namely,

\begin{mathpar}
  M^{*}_{x} := x?(u).M[\dropn{u}]
\end{mathpar}

The dependence of $M^{*}_{x}$ on a name makes it an abstraction, 

\begin{mathpar}
  M^{*} := (x)x?(u).M[\dropn{u}]
\end{mathpar}

\subsection{Additional notation}

It will sometimes be convenient to denote the process a name
quotes. We already have the notation $x = \quotep{P}$, but it will be
convenient to introduce an alternate notation, $\procn{x}$, when we
want to emphasize the connection to the use of the name. Note that, by
virtue of name equivalence, $\quotep{\procn{x}} \nameeq x$; so, the
notation is consistent with previous definitions.

Further, because names have structure it is possible to effect
substitutions on the basis of that structure. This means we need to
upgrade our notation for substitutions, which we accomplish by
adapting comprehension notation. Thus,

\begin{mathpar}
  P\{ y / x : x \in S \}
\end{mathpar}

is interpreted to mean the process derived from P by replacing (in a
capture-avoiding manner) each occurrence of $x$ in $S$ by $y$. For example,

\begin{mathpar}
  P\{ \quotep{\procn{x}|\procn{x}} / x : x \in \freenames{P} \}
\end{mathpar}

will replace each (occurrence) of a free name $x$ in $P$ by
$\quotep{\procn{x}|\procn{x}}$.

Also, we will avail ourselves of the notation $x^{L}$ and $x^{R}$ to
denote injections of a name into disjoint copies of the name
space. There are numerous ways to accomplish this. One example can be
found in \cite{MeredithR05}. This notation overloads to vectors of
names: $\vec{x}^{\pi} := (x_{i}^{\pi} \; : \; 0 \leq i < |\vec{x}| )$ where $\pi \in \{L,R\}$.

We also use $P^{\Box} := P|\Box$.

In \cite{MeredithR05} an interpretation of the new operator is
given. It turns out that there are several possible interpretations
all enjoying the requisite algebraic properties of the operator (see
\cite{milner91polyadicpi}). We will therefore make liberal use of
$(\nu\; \vec{x})P$.

% subsection the_syntax_and_semantics_of_the_notation_system (end)   

\input{qm2pi.qmops} 

\input{qm2pi.sterngerlach} 

\input{qm2pi.metric} 

% section concurrent_process_calculi (end)

%\input{qm2pi.proofsketch}

% section proof sketch (end)

%\input{qm2pi.slviaknots} 

% section spatial logic via knots (end)

\input{qm2pi.conclusion}

% section conclusion (end)

%\input{qm2pi.dtcodes} 

% section wiring algorithm (end)

\input{qm2pi.ack} 

% section acknowledgments (end)

\newpage


\bibliographystyle{plain}   
\bibliography{../../biblios/main.bib}

\input{qm2pi.rhodetails}

\end{document}

 

% section acknowledgments (end)

\newpage


\bibliographystyle{plain}   
\bibliography{../../biblios/main.bib}

\documentclass[12pt]{llncs}
%\documentclass{jktr}

\usepackage[pdftex]{hyperref}                   
\usepackage {listings}
\usepackage {mathpartir}
\usepackage{bcprules}
%\usepackage{listings}
                       
\usepackage{graphicx} 
%\usepackage[margins=2.5cm,nohead,nofoot]{geometry}
%\usepackage{geometry}
\usepackage{amsfonts}
\usepackage{amstext}
\usepackage{latexsym}
\usepackage{amssymb}
\usepackage{color}


%\include{myPreamble}
\include{qm2pi.local} 

%\ifpdf
%\usepackage[pdftex]{graphicx}
%\else
%\usepackage{graphicx}
%\fi

 % \ifpdf
%  \usepackage{pdfsync}
%  \if


%\title{Brief Article}
%\author{David F. Snyder}
%\author{L.G. Meredith}

%\address{Dept. of Math., Texas State University--San Marcos, San Marcos, TX 78666}
       
\pagestyle{empty}


\begin{document}

\lstset{language=[Objective]Caml,frame=shadowbox}

\input{qm2pi.front}

% section front matter (end)

\input{qm2pi.intro} 
 
% section introduction (end)

% \input{qm2pi.knotations} 

% section notation (end)

\input{qm2pi.process.calculi} 

% section concurrent_process_calculi_and_spatial_logics_ (end)
    
%\input{qm2pi.knots2pi} 

%\input{qm2pi.trefoil} 

%\input{qm2pi.mainthm} 

% subsection basic_interpretation (end)

%\input{qm2pi.rho.presentation} 
\subsection{The syntax and semantics of the notation system}\label{sub:the_syntax_and_semantics_of_the_notation_system} % (fold)

We now summarize a technical presentation of the calculus that
embodies our theory of dynamics. The typical presentation of such a
calculus follows the style of giving generators and relations on
them. The grammar, below, describing term constructors, freely
generates the set of processes, $\Proc$. This set is then quotiented
by a relation known as structural congruence and it is over this set
that the notion of dynamics is expressed. This presentation is
essentially that of \cite{MeredithR05} with the addition of
polyadicity and summation. For readability we have relegated some of
the technical subtleties to an appendix.

\subsubsection{Process grammar}\label{subsub:process_grammar}

\begin{mathpar}
  \inferrule* [lab=synchronization] {} {{M} \bc \pzero \;|\; x?F \;|\; x!C }
  \and
  \inferrule* [lab=abstraction] {} {{F} \bc (x)P}
  \and
  \inferrule* [lab=concretion] {} {{C} \bc \langle Q \rangle}
  \and
  \inferrule* [lab=process] {} {{P,Q} \bc M \;| \;P|Q \;|\; @{x}}
  \and
  \inferrule* [lab=name] {} {{x} \bc \quotep{P}}
\end{mathpar} 

Note that $\vec{x}$ (resp. $\vec{P}$) denotes a vector of names
(resp. processes) of length $|\vec{x}|$ (resp. $|\vec{P}|$). We adopt
the following useful abbreviations.

\begin{mathpar}
   x?(\vec{y}).P := x.(\vec{y})P \and  x\clift{\vec{P}} := x.\clift{\vec{P}}
   \and x!(y) := \lift{x}{\dropn{y}}
   \and \Pi_{i=0}^{n-1}P_i := P_0 | \ldots | P_{n-1}
\end{mathpar}

\subsubsection{Structural congruence}

\paragraph{Free and bound names and alpha-equivalence.} At the
core of structural equivalence is alpha-equivalence which identifies
process that are the same up to a change of variable. Formally, we
recognize the distinction between free and bound names. The free names
of a process, $\freenames{P}$, may be calculated recursively as
follows:

\begin{mathpar}
\freenames{\pzero} := \emptyset
  \and \\
  \freenames{x?(y).P} := \{ x \} \cup (\freenames{P} \setminus \{ y \})
  \and 
  \freenames{x!\langle P \rangle} := \{ x \} \cup \{ P \} 
  \and \\
  \freenames{P|Q} := \freenames{P} \cup \freenames{Q}
  \and \\
  \freenames{@{x}} := \{ x \}
\end{mathpar}

$\pi$
$\quotep{\pi}$

$\freenames{-} : \pi \to \mathcal{P}(\quotep{\pi})$

\begin{eqnarray*}
  \freenames{\pzero} & := & \emptyset \\
  \freenames{x?(y).P} & := & \{ x \} \cup (\freenames{P} \setminus \{ y \}) \\
  \freenames{x!\langle P \rangle} & := & \{ x \} \cup \{ P \} \\
  \freenames{P|Q} & := & \freenames{P} \cup \freenames{Q} \\
  \freenames{\dropn{x}} & := & \{ x \}
\end{eqnarray*}

The bound names of a process, $\boundnames{P}$, are those names occurring in $P$
that are not free. For example, in $x?(y).0$, the name $x$ is free, while $y$ is bound.

\begin{mathpar}
  \inferrule* [lab=monoidal-laws] {} { P|Q \equiv Q|P \and P|0 \equiv P \and P|(Q|R) \equiv (P|Q)|R }
\end{mathpar}

\begin{mathpar}
  \inferrule* [lab=alpha-equivalence] {} { (x)P \equiv (y)P\{y/x\} \and y \not\in \freenames{P} }
\end{mathpar}

\begin{definition}
Then two processes, $P,Q$, are alpha-equivalent if $P = Q\{\vec{y}/\vec{x}\}$ for
some $\vec{x} \in \boundnames{Q},\vec{y} \in \boundnames{P}$, where $Q\{\vec{y}/\vec{x}\}$
denotes the capture-avoiding substitution of $\vec{y}$ for $\vec{x}$ in $Q$.
\end{definition}

\begin{definition}
  The {\em structural congruence} \cite{SangiorgiWalker} , $\equiv$,
  between processes is the least congruence containing
  alpha-equivalence, satisfying the abelian monoid laws
  (associativity, commutativity and $\pzero$ as identity) for parallel
  composition $|$ and for summation $+$.
\end{definition}

\subsection{Name equivalence}

We take name equivalence, written $\nameeq$, to be the smallest
equivalence relation generated by the following rules.

\begin{mathpar}
\inferrule*[lab=Quote-drop]
{ }
{ \quotep{@{x}} \nameeq x }

\inferrule*[lab=Struct-equiv]
{ P \scong Q }
{ \quotep{P} \nameeq \quotep{Q} }
\end{mathpar}

The astute reader will have noticed that the mutual recursion of names
and processes imposes a mutual recursion on alpha-equivalence and
structural equivalence via name-equivalence. Fortunately, all of this
works out pleasantly and we may calculate in the natural way, free of
concern. The reader interested in the details is referred to the
appendix \ref{appendix:rho_details}.

\subsection{Substitution}

We use $\Proc$ for the set of processes, $\QProc$ for the set of
names, and $\id{\{}\vec{y} / \vec{x} \id{\}}$ to denote partial maps,
$s : \QProc \rightarrow \QProc$. A map, $s$ lifts, uniquely, to a map
on process terms, $\widehat{s} : \Proc \rightarrow \Proc$ by the
following equations.

\begin{mathpar}
  (0) \psubstp{Q}{P} := 0 \\
  (R \juxtap S) \psubstp{Q}{P}
  :=    
  (R)\psubstp{Q}{P} \juxtap (S) \psubstp{Q}{P} \\
  (x?(y).R) \psubstp{Q}{P}    
  :=    
  (x)\substp{Q}{P} (z)\concat( (R \psubstn{z}{y}) \psubstp{Q}{P} ) \\
  (\lift{x}{R}) \psubstp{Q}{P}  
  :=
  \lift{(x)\substp{Q}{P}}{ R \psubstp{Q}{P} } \\
%   (\dropn{x})  \psubstp{Q}{P}       
%   := 
%   \left\{ 
%     \begin{array}{ccc} 
%       \dropn{\quotep{Q}} & & x \nameeq \quotep{P} \\
%       \dropn{x} & & otherwise \\
%     \end{array}
%   \right. 
  (\dropn{x})  \psubstp{Q}{P}       
  := 
  \left\{ 
    \begin{array}{ccc} 
      Q & & x \nameeq \quotep{P} \\
      \dropn{x} & & otherwise \\
    \end{array}
  \right.
\end{mathpar}
 

where

\begin{eqnarray}
  (x)\id{\{} \lpquote Q \rpquote / \lpquote P \rpquote \id{\}}            = 
  \left\{ 
    \begin{array}{ccc}
      \lpquote Q \rpquote & & x \nameeq \lpquote P \rpquote \\
      x & & otherwise \\
    \end{array}
  \right. \nonumber
\end{eqnarray}

and $z$ is chosen distinct from $\quotep{P}$, $\quotep{Q}$, the free
names in $Q$, and all the names in $R$. Our $\alpha$-equivalence will
be built in the standard way from this substitution.

\begin{remark}\label{rem:no_self_referential_names}
  One consequence of these definitions is that $\forall P. \quotep{P}
  \not\in \freenames{P}$.
\end{remark}

\subsection{ Dynamic quote: an example }

Anticipating something of what's to come, consider applying the
substitution, $\widehat{\id{\{}u / z \id{\}}}$, to the following pair
of processes, $\lift{w}{y!(z)}$ and $w[ \lpquote y!(z) \rpquote ]$.

\begin{eqnarray}
	\lift{w}{y!(z)}\widehat{\id{\{}u / z \id{\}}}
		& = &
		\lift{w}{y!(u)} \nonumber\\
	w[ \lpquote y!(z) \rpquote ] \widehat{ \id{\{}u / z \id{\}} }
		& = &
		w[ \lpquote y!(z) \rpquote ] \nonumber
\end{eqnarray}

Because the body of the process between quotes is impervious to
substitution, we get radically different answers. In fact, by
examining the first process in an input context,
e.g. $x?(z).\lift{w}{y!(z)}$, we see that the process under the lift
operator may be shaped by prefixed inputs binding a name inside it. In
this sense, the lift operator will be seen as a way to dynamically
construct processes before reifying them as names.

Finally equipped with these standard features we can present the
dynamics of the calculus.

\subsubsection{Operational semantics} 

Finally, we introduce the computational dynamics. What marks these
algebras as distinct from other more traditionally studied algebraic
structures, e.g. vector spaces or polynomial rings, is the manner in
which dynamics is captured. In traditional structures, dynamics is typically
expressed through morphisms between such structures, as in linear maps
between vector spaces or morphisms between rings. In algebras
associated with the semantics of computation, the dynamics is
expressed as part of the algebraic structure itself, through a
reduction reduction relation typically denoted by $\red$. Below, we
give a recursive presentation of this relation for the calculus used
in the encoding.

$\red \subseteq \pi \times \pi$
$\red : \pi \to \mathcal{P}(\pi)$

\begin{mathpar}
  \inferrule* [lab=Comm] { \textsf{match}( x_{src}, x_{trgt} ) } { x_{trgt}?(y)P \; | \; x_{src}!\langle {Q} \rangle \red P\{\quotep{Q}/y}\} }
  \and \\
  \inferrule* [lab=Par] {{P} \red {P}'} {{{P} | {Q}} \red {{P}' | {Q}}}
  \and
  \inferrule* [lab=Equiv]{{{P} \scong {P}'} \andalso {{P}' \red {Q}'} \andalso {{Q}' \scong {Q}}}{{P} \red {Q}}
\end{mathpar}

\begin{eqnarray*}
  match_{\equiv} (\quotep{P},\quotep{Q}) & := & P \equiv Q \\
  match_{\dagger}(\quotep{P},\quotep{Q}) & := & \forall R. P|Q \red^{*} R => R \red^{*} 0 \\
  match_{K}(\quotep{P},\quotep{Q}) & := & K \mbox{ for some context } K
\end{eqnarray*}

$u?(x)P | u!\langle Q \rangle \red P\{\quotep{Q}/x\}$

%We write $\wred$ for $\red^*$, and $P\red$ if $\exists Q $ such that $ P \red Q$.
We write $P\red$ if $\exists Q $ such that $ P \red Q$ and $P\not\red$, otherwise.

\section{Replication}

As mentioned before, it is known that replication (and hence
recursion) can be implemented in a higher-order process algebra
\cite{SangiorgiWalker}. As our first example of calculation with the
machinery thus far presented we give the construction explicitly in
the {\rhoc}.

\begin{eqnarray}
	D_{x} & := & \prefix{x}{y}{(\binpar{\outputp{x}{y}}{@{y}})} \nonumber\\
	\bangp_{x}{P} & := & \binpar{{x}!\langle{\binpar{D_{x}}{P}}\rangle}{D_{x}} \nonumber
\end{eqnarray}

\begin{eqnarray}
	\bangp_{x}{P} & & \nonumber\\
	=
	& {x}!\langle{(\prefix{x}{y}{(\outputp{x}{y} | @{y})) | P}}\rangle 
	      | \prefix{x}{y}{(\outputp{x}{y} | @{y})} & \nonumber\\
	\red
	& (\outputp{x}{y} | @{y})\substn{\quotep{(\prefix{x}{y}{(@{y} | \outputp{x}{y})) | P}}}{y} & \nonumber\\
	=
	& \outputp{x}{\quotep{(\prefix{x}{y}{(\outputp{x}{y} | @{y})) | P}}}
	  | {(\prefix{x}{y}{(\outputp{x}{y} | @{y})) | P}} & \nonumber\\
	\red
	& \ldots & \nonumber\\
	\red^*
	& P | P | \ldots & \nonumber
\end{eqnarray}

Of course, this encoding, as an implementation, runs away, unfolding
$\bangp{P}$ eagerly. A lazier and more implementable replication
operator, restricted to input-guarded processes, may be obtained as follows.

\begin{eqnarray}
\bangp{\prefix{u}{v}{P}} 
	:= 
	\binpar{\lift{x}{\prefix{u}{v}{(\binpar{D(x)}{P})}}}{D(x)} \nonumber
\end{eqnarray}

\begin{remark}
  Note that the lazier definition still does not deal with summation
  or mixed summation (i.e. sums over input and output). The reader is
  invited to construct definitions of replication that deal with these
  features. 

  Further, the definitions are parameterized in a name, $x$. Can you,
  gentle reader, make a definition that eliminates this parameter and
  guarantees no accidental interaction between the replication
  machinery and the process being replicated -- i.e. no accidental
  sharing of names used by the process to get its work done and the
  name(s) used by the replication to effect copying. This latter
  revision of the definition of replication is crucial to obtaining
  the expected identity $!!P \sim !P$.
\end{remark}

\begin{remark}\label{rem:paradoxical_combinator}
  The reader familiar with the lambda calculus will have noticed the
  similarity between $D$ and the paradoxical combinator.

  [Ed. note: the existence of this seems to suggest we have to be more
  restrictive on the set of processes and names we admit if we are to
  support no-cloning.]
\end{remark}

\subsubsection{Bisimulation}

The computational dynamics gives rise to another kind of equivalence,
the equivalence of computational behavior. As previously mentioned
this is typically captured \emph{via} some form of bisimulation.

% The notion we use in this paper is weak barbed bisimulation
% \cite{milner91polyadicpi}.

The notion we use in this paper is derived from weak barbed
bisimulation \cite{milner91polyadicpi}. 

\begin{definition}
An \emph{observation relation}, $\downarrow_{\mathcal N}$, over a set
of names, $\mathcal N$, is the smallest relation satisfying the rules
below.

\infrule[Out-barb]{y \in {\mathcal N}, \; x \nameeq y}
		  {\outputp{x}{v} \downarrow_{\mathcal N} x}
\infrule[Par-barb]{\mbox{$P\downarrow_{\mathcal N} x$ or $Q\downarrow_{\mathcal N} x$}}
		  {\binpar{P}{Q} \downarrow_{\mathcal N} x}

We write $P \Downarrow_{\mathcal N} x$ if there is $Q$ such that 
$P \wred Q$ and $Q \downarrow_{\mathcal N} x$.
\end{definition}

\begin{definition}
%\label{def.bbisim}
An  ${\mathcal N}$-\emph{barbed bisimulation} over a set of names, ${\mathcal N}$, is a symmetric binary relation 
${\mathcal S}_{\mathcal N}$ between agents such that $P\rel{S}_{\mathcal N}Q$ implies:
\begin{enumerate}
\item If $P \red P'$ then $Q \wred Q'$ and $P'\rel{S}_{\mathcal N} Q'$.
\item If $P\downarrow_{\mathcal N} x$, then $Q\Downarrow_{\mathcal N} x$.
\end{enumerate}
$P$ is ${\mathcal N}$-barbed bisimilar to $Q$, written
$P \wbbisim_{\mathcal N} Q$, if $P \rel{S}_{\mathcal N} Q$ for some ${\mathcal N}$-barbed bisimulation ${\mathcal S}_{\mathcal N}$.
\end{definition}

$\mathcal{R} \subseteq \pi \times \pi$

$P \mathcal{R} Q => \forall P'. P \red P' \Rightarrow \exists Q'. Q \red Q', P' \mathcal{R} Q'$

$P \vdash x \Rightarrow Q \vdash x$

\begin{mathpar}
  \inferrule*[lab=Out-barb]{x \nameeq y}{{y}!\langle{Q}\rangle \vdash x}
  \and
  \inferrule*[lab=Par-barb]{\mbox{$P\vdash x$ or $Q\vdash x$}}{\binpar{P}{Q} \vdash x}
\end{mathpar}

\subsubsection{Contexts}

One of the principle advantages of computational calculi like the
$\pi$-calculus is a well-defined notion of context,
contextual-equivalence and a correlation between
contextual-equivalence and notions of bisimulation. The notion of
context allows the decomposition of a process into (sub-)process and
its syntactic environment, its context. Thus, a context may be
thought of as a process with a ``hole'' (written $\Box$) in it. The
application of a context $M$ to a process $P$, written $M[P]$, is
tantamount to filling the hole in $M$ with $P$. In this paper we do
not need the full weight of this theory, but do make use of the notion
of context in the proof the main theorem. 

\begin{mathpar}
  \inferrule* [lab=summation] {} {{M_{M},M_{N}} \bc \Box \;|\; x.M_{A} \;|\; M_{M}+M_{N}}
  \and
  \inferrule* [lab=agent] {} {{M_{A}} \bc (\vec{x})M_{P} \;| \; \clift{P_0,\ldots,M_{P},\ldots,P_N}}
  \and \\
  \inferrule* [lab=process] {} {{M_{P}} \bc M_{N} \;| \;P|M_{P} }
\end{mathpar} 

\begin{mathpar}
  \inferrule* [lab=sychronization] {} {M_{N} \bc \Box \;|\; x?M_{F} \;|\; x!M_{C}}
  \and
  \inferrule* [lab=abstraction] {} {{M_{F}} \bc (x)M_{P} }
  \and
  \inferrule* [lab=concretion] {} {{M_{C}} \bc \langle M_{P} \rangle }
  \and \\
  \inferrule* [lab=process] {} {{M_{P}} \bc M_{N} \;| \;P|M_{P} }
\end{mathpar}

\begin{definition}[contextual application] Given a context $M$, and
  process $P$, we define the \emph{contextual application}, $M[P] :=
  M\{P/\Box\}$. That is, the contextual application of M to P is the
  substitution of $P$ for $\Box$ in $M$.
\end{definition}

$\meaningof{-} : L \to \mathcal{P}(\pi)$

\begin{mathpar}
  \inferrule* [lab=collection] {} {\meaningof{true} = \pi, \and \meaningof{~E} = \pi \setminus \meaningof{E}, \and \meaningof{E_{1} \& E_{2}} = \meaningof{E_{1}} \cap \meaningof{E_{2}}}
\end{mathpar}

\begin{mathpar}
  \inferrule* [lab=structure] {} {\meaningof{0} = \{ P \in \pi | P \equiv 0 \}, \and \\ \meaningof{E_1 | E_2} = \{ P \in \pi | P \equiv P_{1} | P_{2}, P_{1} \in \meaningof{E_{1}}, P_{2} \in \meaningof{E_2}\} }
\end{mathpar}

\begin{mathpar}
 \inferrule* [lab=behavior] {} {\meaningof{\langle a?b \rangle E} = \{ P \in \pi | P \equiv Q | u?(y)P', \\ \and \\\\ \and \\ \;\;\; u \in \meaningof{a}, \forall z.P'\{z/y\} \in \meaningof{E\{z/b\}}\}, \and \\ \meaningof{a!E} = \{ P \in \pi | P \equiv Q | x!\langle P' \rangle, x \in \meaningof{a} P' \in \meaningof{E}\} }
\end{mathpar}

\begin{mathpar}
 \inferrule* [lab=nominal] {} {\meaningof{\quotep{E}} = \{ \quotep{P} \in \quotep{\pi} | P \in \meaningof{E} \}, \and \meaningof{\quotep{P}} = \{ \quotep{Q} \in \quotep{\pi} | P \equiv Q \} \and \\ \meaningof{@\quotep{E}} = \{ P \in \pi | P \equiv @x, x \in \meaningof{E} \}}
\end{mathpar}

\begin{eqnarray*}
  \\
  \meaningof{-} : TS \to ST
\end{eqnarray*}

\begin{eqnarray*}
  \\
  L : TS \to ST
\end{eqnarray*}

\begin{eqnarray*}
  \\
  P \models E \iff P \in \meaningof{E}
\end{eqnarray*}

\begin{eqnarray*}
  P \approx_{L} Q \iff \forall E \in L. P \models E \iff Q \models E
\end{eqnarray*}

\begin{eqnarray*}
  P \approx_{K} Q
\end{eqnarray*}

\begin{eqnarray*}
  P \approx Q
\end{eqnarray*}

$\approx_{K} = \approx = \approx_{L}$

\subsubsection{Contextual duality}

Note that contexts extend the quotation operation to a family of
operations from processes to names. Given a context, $M$, we can
define a \emph{nominal context}, $\quotep{M}$ by $\quotep{M}[P] :=
\quotep{M[P]}$. To foreshadow what is to come we observe that these
operations enjoy a duality with processes very much like the duality
between vectors and maps from vectors to scalars.

Further, because the calculus is essentially higher-order, we have a
correspondence between contexts and processes. More specifically,
given a name $x$ and a context $M$ we can construct $M^{*}_{x}$ such
that 

\begin{mathpar}
  M^{*}_{x} | \lift{x}{P} \red M[P]
\end{mathpar}

namely,

\begin{mathpar}
  M^{*}_{x} := x?(u).M[\dropn{u}]
\end{mathpar}

The dependence of $M^{*}_{x}$ on a name makes it an abstraction, 

\begin{mathpar}
  M^{*} := (x)x?(u).M[\dropn{u}]
\end{mathpar}

\subsection{Additional notation}

It will sometimes be convenient to denote the process a name
quotes. We already have the notation $x = \quotep{P}$, but it will be
convenient to introduce an alternate notation, $\procn{x}$, when we
want to emphasize the connection to the use of the name. Note that, by
virtue of name equivalence, $\quotep{\procn{x}} \nameeq x$; so, the
notation is consistent with previous definitions.

Further, because names have structure it is possible to effect
substitutions on the basis of that structure. This means we need to
upgrade our notation for substitutions, which we accomplish by
adapting comprehension notation. Thus,

\begin{mathpar}
  P\{ y / x : x \in S \}
\end{mathpar}

is interpreted to mean the process derived from P by replacing (in a
capture-avoiding manner) each occurrence of $x$ in $S$ by $y$. For example,

\begin{mathpar}
  P\{ \quotep{\procn{x}|\procn{x}} / x : x \in \freenames{P} \}
\end{mathpar}

will replace each (occurrence) of a free name $x$ in $P$ by
$\quotep{\procn{x}|\procn{x}}$.

Also, we will avail ourselves of the notation $x^{L}$ and $x^{R}$ to
denote injections of a name into disjoint copies of the name
space. There are numerous ways to accomplish this. One example can be
found in \cite{MeredithR05}. This notation overloads to vectors of
names: $\vec{x}^{\pi} := (x_{i}^{\pi} \; : \; 0 \leq i < |\vec{x}| )$ where $\pi \in \{L,R\}$.

We also use $P^{\Box} := P|\Box$.

In \cite{MeredithR05} an interpretation of the new operator is
given. It turns out that there are several possible interpretations
all enjoying the requisite algebraic properties of the operator (see
\cite{milner91polyadicpi}). We will therefore make liberal use of
$(\nu\; \vec{x})P$.

% subsection the_syntax_and_semantics_of_the_notation_system (end)   

\input{qm2pi.qmops} 

\input{qm2pi.sterngerlach} 

\input{qm2pi.metric} 

% section concurrent_process_calculi (end)

%\input{qm2pi.proofsketch}

% section proof sketch (end)

%\input{qm2pi.slviaknots} 

% section spatial logic via knots (end)

\input{qm2pi.conclusion}

% section conclusion (end)

%\input{qm2pi.dtcodes} 

% section wiring algorithm (end)

\input{qm2pi.ack} 

% section acknowledgments (end)

\newpage


\bibliographystyle{plain}   
\bibliography{../../biblios/main.bib}

\input{qm2pi.rhodetails}

\end{document}



\end{document}

 

\documentclass[12pt]{llncs}
%\documentclass{jktr}

\usepackage[pdftex]{hyperref}                   
\usepackage {listings}
\usepackage {mathpartir}
\usepackage{bcprules}
%\usepackage{listings}
                       
\usepackage{graphicx} 
%\usepackage[margins=2.5cm,nohead,nofoot]{geometry}
%\usepackage{geometry}
\usepackage{amsfonts}
\usepackage{amstext}
\usepackage{latexsym}
\usepackage{amssymb}
\usepackage{color}


%\include{myPreamble}
\documentclass[12pt]{llncs}
%\documentclass{jktr}

\usepackage[pdftex]{hyperref}                   
\usepackage {listings}
\usepackage {mathpartir}
\usepackage{bcprules}
%\usepackage{listings}
                       
\usepackage{graphicx} 
%\usepackage[margins=2.5cm,nohead,nofoot]{geometry}
%\usepackage{geometry}
\usepackage{amsfonts}
\usepackage{amstext}
\usepackage{latexsym}
\usepackage{amssymb}
\usepackage{color}


%\include{myPreamble}
\include{qm2pi.local} 

%\ifpdf
%\usepackage[pdftex]{graphicx}
%\else
%\usepackage{graphicx}
%\fi

 % \ifpdf
%  \usepackage{pdfsync}
%  \if


%\title{Brief Article}
%\author{David F. Snyder}
%\author{L.G. Meredith}

%\address{Dept. of Math., Texas State University--San Marcos, San Marcos, TX 78666}
       
\pagestyle{empty}


\begin{document}

\lstset{language=[Objective]Caml,frame=shadowbox}

\input{qm2pi.front}

% section front matter (end)

\input{qm2pi.intro} 
 
% section introduction (end)

% \input{qm2pi.knotations} 

% section notation (end)

\input{qm2pi.process.calculi} 

% section concurrent_process_calculi_and_spatial_logics_ (end)
    
%\input{qm2pi.knots2pi} 

%\input{qm2pi.trefoil} 

%\input{qm2pi.mainthm} 

% subsection basic_interpretation (end)

%\input{qm2pi.rho.presentation} 
\subsection{The syntax and semantics of the notation system}\label{sub:the_syntax_and_semantics_of_the_notation_system} % (fold)

We now summarize a technical presentation of the calculus that
embodies our theory of dynamics. The typical presentation of such a
calculus follows the style of giving generators and relations on
them. The grammar, below, describing term constructors, freely
generates the set of processes, $\Proc$. This set is then quotiented
by a relation known as structural congruence and it is over this set
that the notion of dynamics is expressed. This presentation is
essentially that of \cite{MeredithR05} with the addition of
polyadicity and summation. For readability we have relegated some of
the technical subtleties to an appendix.

\subsubsection{Process grammar}\label{subsub:process_grammar}

\begin{mathpar}
  \inferrule* [lab=synchronization] {} {{M} \bc \pzero \;|\; x?F \;|\; x!C }
  \and
  \inferrule* [lab=abstraction] {} {{F} \bc (x)P}
  \and
  \inferrule* [lab=concretion] {} {{C} \bc \langle Q \rangle}
  \and
  \inferrule* [lab=process] {} {{P,Q} \bc M \;| \;P|Q \;|\; @{x}}
  \and
  \inferrule* [lab=name] {} {{x} \bc \quotep{P}}
\end{mathpar} 

Note that $\vec{x}$ (resp. $\vec{P}$) denotes a vector of names
(resp. processes) of length $|\vec{x}|$ (resp. $|\vec{P}|$). We adopt
the following useful abbreviations.

\begin{mathpar}
   x?(\vec{y}).P := x.(\vec{y})P \and  x\clift{\vec{P}} := x.\clift{\vec{P}}
   \and x!(y) := \lift{x}{\dropn{y}}
   \and \Pi_{i=0}^{n-1}P_i := P_0 | \ldots | P_{n-1}
\end{mathpar}

\subsubsection{Structural congruence}

\paragraph{Free and bound names and alpha-equivalence.} At the
core of structural equivalence is alpha-equivalence which identifies
process that are the same up to a change of variable. Formally, we
recognize the distinction between free and bound names. The free names
of a process, $\freenames{P}$, may be calculated recursively as
follows:

\begin{mathpar}
\freenames{\pzero} := \emptyset
  \and \\
  \freenames{x?(y).P} := \{ x \} \cup (\freenames{P} \setminus \{ y \})
  \and 
  \freenames{x!\langle P \rangle} := \{ x \} \cup \{ P \} 
  \and \\
  \freenames{P|Q} := \freenames{P} \cup \freenames{Q}
  \and \\
  \freenames{@{x}} := \{ x \}
\end{mathpar}

$\pi$
$\quotep{\pi}$

$\freenames{-} : \pi \to \mathcal{P}(\quotep{\pi})$

\begin{eqnarray*}
  \freenames{\pzero} & := & \emptyset \\
  \freenames{x?(y).P} & := & \{ x \} \cup (\freenames{P} \setminus \{ y \}) \\
  \freenames{x!\langle P \rangle} & := & \{ x \} \cup \{ P \} \\
  \freenames{P|Q} & := & \freenames{P} \cup \freenames{Q} \\
  \freenames{\dropn{x}} & := & \{ x \}
\end{eqnarray*}

The bound names of a process, $\boundnames{P}$, are those names occurring in $P$
that are not free. For example, in $x?(y).0$, the name $x$ is free, while $y$ is bound.

\begin{mathpar}
  \inferrule* [lab=monoidal-laws] {} { P|Q \equiv Q|P \and P|0 \equiv P \and P|(Q|R) \equiv (P|Q)|R }
\end{mathpar}

\begin{mathpar}
  \inferrule* [lab=alpha-equivalence] {} { (x)P \equiv (y)P\{y/x\} \and y \not\in \freenames{P} }
\end{mathpar}

\begin{definition}
Then two processes, $P,Q$, are alpha-equivalent if $P = Q\{\vec{y}/\vec{x}\}$ for
some $\vec{x} \in \boundnames{Q},\vec{y} \in \boundnames{P}$, where $Q\{\vec{y}/\vec{x}\}$
denotes the capture-avoiding substitution of $\vec{y}$ for $\vec{x}$ in $Q$.
\end{definition}

\begin{definition}
  The {\em structural congruence} \cite{SangiorgiWalker} , $\equiv$,
  between processes is the least congruence containing
  alpha-equivalence, satisfying the abelian monoid laws
  (associativity, commutativity and $\pzero$ as identity) for parallel
  composition $|$ and for summation $+$.
\end{definition}

\subsection{Name equivalence}

We take name equivalence, written $\nameeq$, to be the smallest
equivalence relation generated by the following rules.

\begin{mathpar}
\inferrule*[lab=Quote-drop]
{ }
{ \quotep{@{x}} \nameeq x }

\inferrule*[lab=Struct-equiv]
{ P \scong Q }
{ \quotep{P} \nameeq \quotep{Q} }
\end{mathpar}

The astute reader will have noticed that the mutual recursion of names
and processes imposes a mutual recursion on alpha-equivalence and
structural equivalence via name-equivalence. Fortunately, all of this
works out pleasantly and we may calculate in the natural way, free of
concern. The reader interested in the details is referred to the
appendix \ref{appendix:rho_details}.

\subsection{Substitution}

We use $\Proc$ for the set of processes, $\QProc$ for the set of
names, and $\id{\{}\vec{y} / \vec{x} \id{\}}$ to denote partial maps,
$s : \QProc \rightarrow \QProc$. A map, $s$ lifts, uniquely, to a map
on process terms, $\widehat{s} : \Proc \rightarrow \Proc$ by the
following equations.

\begin{mathpar}
  (0) \psubstp{Q}{P} := 0 \\
  (R \juxtap S) \psubstp{Q}{P}
  :=    
  (R)\psubstp{Q}{P} \juxtap (S) \psubstp{Q}{P} \\
  (x?(y).R) \psubstp{Q}{P}    
  :=    
  (x)\substp{Q}{P} (z)\concat( (R \psubstn{z}{y}) \psubstp{Q}{P} ) \\
  (\lift{x}{R}) \psubstp{Q}{P}  
  :=
  \lift{(x)\substp{Q}{P}}{ R \psubstp{Q}{P} } \\
%   (\dropn{x})  \psubstp{Q}{P}       
%   := 
%   \left\{ 
%     \begin{array}{ccc} 
%       \dropn{\quotep{Q}} & & x \nameeq \quotep{P} \\
%       \dropn{x} & & otherwise \\
%     \end{array}
%   \right. 
  (\dropn{x})  \psubstp{Q}{P}       
  := 
  \left\{ 
    \begin{array}{ccc} 
      Q & & x \nameeq \quotep{P} \\
      \dropn{x} & & otherwise \\
    \end{array}
  \right.
\end{mathpar}
 

where

\begin{eqnarray}
  (x)\id{\{} \lpquote Q \rpquote / \lpquote P \rpquote \id{\}}            = 
  \left\{ 
    \begin{array}{ccc}
      \lpquote Q \rpquote & & x \nameeq \lpquote P \rpquote \\
      x & & otherwise \\
    \end{array}
  \right. \nonumber
\end{eqnarray}

and $z$ is chosen distinct from $\quotep{P}$, $\quotep{Q}$, the free
names in $Q$, and all the names in $R$. Our $\alpha$-equivalence will
be built in the standard way from this substitution.

\begin{remark}\label{rem:no_self_referential_names}
  One consequence of these definitions is that $\forall P. \quotep{P}
  \not\in \freenames{P}$.
\end{remark}

\subsection{ Dynamic quote: an example }

Anticipating something of what's to come, consider applying the
substitution, $\widehat{\id{\{}u / z \id{\}}}$, to the following pair
of processes, $\lift{w}{y!(z)}$ and $w[ \lpquote y!(z) \rpquote ]$.

\begin{eqnarray}
	\lift{w}{y!(z)}\widehat{\id{\{}u / z \id{\}}}
		& = &
		\lift{w}{y!(u)} \nonumber\\
	w[ \lpquote y!(z) \rpquote ] \widehat{ \id{\{}u / z \id{\}} }
		& = &
		w[ \lpquote y!(z) \rpquote ] \nonumber
\end{eqnarray}

Because the body of the process between quotes is impervious to
substitution, we get radically different answers. In fact, by
examining the first process in an input context,
e.g. $x?(z).\lift{w}{y!(z)}$, we see that the process under the lift
operator may be shaped by prefixed inputs binding a name inside it. In
this sense, the lift operator will be seen as a way to dynamically
construct processes before reifying them as names.

Finally equipped with these standard features we can present the
dynamics of the calculus.

\subsubsection{Operational semantics} 

Finally, we introduce the computational dynamics. What marks these
algebras as distinct from other more traditionally studied algebraic
structures, e.g. vector spaces or polynomial rings, is the manner in
which dynamics is captured. In traditional structures, dynamics is typically
expressed through morphisms between such structures, as in linear maps
between vector spaces or morphisms between rings. In algebras
associated with the semantics of computation, the dynamics is
expressed as part of the algebraic structure itself, through a
reduction reduction relation typically denoted by $\red$. Below, we
give a recursive presentation of this relation for the calculus used
in the encoding.

$\red \subseteq \pi \times \pi$
$\red : \pi \to \mathcal{P}(\pi)$

\begin{mathpar}
  \inferrule* [lab=Comm] { \textsf{match}( x_{src}, x_{trgt} ) } { x_{trgt}?(y)P \; | \; x_{src}!\langle {Q} \rangle \red P\{\quotep{Q}/y}\} }
  \and \\
  \inferrule* [lab=Par] {{P} \red {P}'} {{{P} | {Q}} \red {{P}' | {Q}}}
  \and
  \inferrule* [lab=Equiv]{{{P} \scong {P}'} \andalso {{P}' \red {Q}'} \andalso {{Q}' \scong {Q}}}{{P} \red {Q}}
\end{mathpar}

\begin{eqnarray*}
  match_{\equiv} (\quotep{P},\quotep{Q}) & := & P \equiv Q \\
  match_{\dagger}(\quotep{P},\quotep{Q}) & := & \forall R. P|Q \red^{*} R => R \red^{*} 0 \\
  match_{K}(\quotep{P},\quotep{Q}) & := & K \mbox{ for some context } K
\end{eqnarray*}

$u?(x)P | u!\langle Q \rangle \red P\{\quotep{Q}/x\}$

%We write $\wred$ for $\red^*$, and $P\red$ if $\exists Q $ such that $ P \red Q$.
We write $P\red$ if $\exists Q $ such that $ P \red Q$ and $P\not\red$, otherwise.

\section{Replication}

As mentioned before, it is known that replication (and hence
recursion) can be implemented in a higher-order process algebra
\cite{SangiorgiWalker}. As our first example of calculation with the
machinery thus far presented we give the construction explicitly in
the {\rhoc}.

\begin{eqnarray}
	D_{x} & := & \prefix{x}{y}{(\binpar{\outputp{x}{y}}{@{y}})} \nonumber\\
	\bangp_{x}{P} & := & \binpar{{x}!\langle{\binpar{D_{x}}{P}}\rangle}{D_{x}} \nonumber
\end{eqnarray}

\begin{eqnarray}
	\bangp_{x}{P} & & \nonumber\\
	=
	& {x}!\langle{(\prefix{x}{y}{(\outputp{x}{y} | @{y})) | P}}\rangle 
	      | \prefix{x}{y}{(\outputp{x}{y} | @{y})} & \nonumber\\
	\red
	& (\outputp{x}{y} | @{y})\substn{\quotep{(\prefix{x}{y}{(@{y} | \outputp{x}{y})) | P}}}{y} & \nonumber\\
	=
	& \outputp{x}{\quotep{(\prefix{x}{y}{(\outputp{x}{y} | @{y})) | P}}}
	  | {(\prefix{x}{y}{(\outputp{x}{y} | @{y})) | P}} & \nonumber\\
	\red
	& \ldots & \nonumber\\
	\red^*
	& P | P | \ldots & \nonumber
\end{eqnarray}

Of course, this encoding, as an implementation, runs away, unfolding
$\bangp{P}$ eagerly. A lazier and more implementable replication
operator, restricted to input-guarded processes, may be obtained as follows.

\begin{eqnarray}
\bangp{\prefix{u}{v}{P}} 
	:= 
	\binpar{\lift{x}{\prefix{u}{v}{(\binpar{D(x)}{P})}}}{D(x)} \nonumber
\end{eqnarray}

\begin{remark}
  Note that the lazier definition still does not deal with summation
  or mixed summation (i.e. sums over input and output). The reader is
  invited to construct definitions of replication that deal with these
  features. 

  Further, the definitions are parameterized in a name, $x$. Can you,
  gentle reader, make a definition that eliminates this parameter and
  guarantees no accidental interaction between the replication
  machinery and the process being replicated -- i.e. no accidental
  sharing of names used by the process to get its work done and the
  name(s) used by the replication to effect copying. This latter
  revision of the definition of replication is crucial to obtaining
  the expected identity $!!P \sim !P$.
\end{remark}

\begin{remark}\label{rem:paradoxical_combinator}
  The reader familiar with the lambda calculus will have noticed the
  similarity between $D$ and the paradoxical combinator.

  [Ed. note: the existence of this seems to suggest we have to be more
  restrictive on the set of processes and names we admit if we are to
  support no-cloning.]
\end{remark}

\subsubsection{Bisimulation}

The computational dynamics gives rise to another kind of equivalence,
the equivalence of computational behavior. As previously mentioned
this is typically captured \emph{via} some form of bisimulation.

% The notion we use in this paper is weak barbed bisimulation
% \cite{milner91polyadicpi}.

The notion we use in this paper is derived from weak barbed
bisimulation \cite{milner91polyadicpi}. 

\begin{definition}
An \emph{observation relation}, $\downarrow_{\mathcal N}$, over a set
of names, $\mathcal N$, is the smallest relation satisfying the rules
below.

\infrule[Out-barb]{y \in {\mathcal N}, \; x \nameeq y}
		  {\outputp{x}{v} \downarrow_{\mathcal N} x}
\infrule[Par-barb]{\mbox{$P\downarrow_{\mathcal N} x$ or $Q\downarrow_{\mathcal N} x$}}
		  {\binpar{P}{Q} \downarrow_{\mathcal N} x}

We write $P \Downarrow_{\mathcal N} x$ if there is $Q$ such that 
$P \wred Q$ and $Q \downarrow_{\mathcal N} x$.
\end{definition}

\begin{definition}
%\label{def.bbisim}
An  ${\mathcal N}$-\emph{barbed bisimulation} over a set of names, ${\mathcal N}$, is a symmetric binary relation 
${\mathcal S}_{\mathcal N}$ between agents such that $P\rel{S}_{\mathcal N}Q$ implies:
\begin{enumerate}
\item If $P \red P'$ then $Q \wred Q'$ and $P'\rel{S}_{\mathcal N} Q'$.
\item If $P\downarrow_{\mathcal N} x$, then $Q\Downarrow_{\mathcal N} x$.
\end{enumerate}
$P$ is ${\mathcal N}$-barbed bisimilar to $Q$, written
$P \wbbisim_{\mathcal N} Q$, if $P \rel{S}_{\mathcal N} Q$ for some ${\mathcal N}$-barbed bisimulation ${\mathcal S}_{\mathcal N}$.
\end{definition}

$\mathcal{R} \subseteq \pi \times \pi$

$P \mathcal{R} Q => \forall P'. P \red P' \Rightarrow \exists Q'. Q \red Q', P' \mathcal{R} Q'$

$P \vdash x \Rightarrow Q \vdash x$

\begin{mathpar}
  \inferrule*[lab=Out-barb]{x \nameeq y}{{y}!\langle{Q}\rangle \vdash x}
  \and
  \inferrule*[lab=Par-barb]{\mbox{$P\vdash x$ or $Q\vdash x$}}{\binpar{P}{Q} \vdash x}
\end{mathpar}

\subsubsection{Contexts}

One of the principle advantages of computational calculi like the
$\pi$-calculus is a well-defined notion of context,
contextual-equivalence and a correlation between
contextual-equivalence and notions of bisimulation. The notion of
context allows the decomposition of a process into (sub-)process and
its syntactic environment, its context. Thus, a context may be
thought of as a process with a ``hole'' (written $\Box$) in it. The
application of a context $M$ to a process $P$, written $M[P]$, is
tantamount to filling the hole in $M$ with $P$. In this paper we do
not need the full weight of this theory, but do make use of the notion
of context in the proof the main theorem. 

\begin{mathpar}
  \inferrule* [lab=summation] {} {{M_{M},M_{N}} \bc \Box \;|\; x.M_{A} \;|\; M_{M}+M_{N}}
  \and
  \inferrule* [lab=agent] {} {{M_{A}} \bc (\vec{x})M_{P} \;| \; \clift{P_0,\ldots,M_{P},\ldots,P_N}}
  \and \\
  \inferrule* [lab=process] {} {{M_{P}} \bc M_{N} \;| \;P|M_{P} }
\end{mathpar} 

\begin{mathpar}
  \inferrule* [lab=sychronization] {} {M_{N} \bc \Box \;|\; x?M_{F} \;|\; x!M_{C}}
  \and
  \inferrule* [lab=abstraction] {} {{M_{F}} \bc (x)M_{P} }
  \and
  \inferrule* [lab=concretion] {} {{M_{C}} \bc \langle M_{P} \rangle }
  \and \\
  \inferrule* [lab=process] {} {{M_{P}} \bc M_{N} \;| \;P|M_{P} }
\end{mathpar}

\begin{definition}[contextual application] Given a context $M$, and
  process $P$, we define the \emph{contextual application}, $M[P] :=
  M\{P/\Box\}$. That is, the contextual application of M to P is the
  substitution of $P$ for $\Box$ in $M$.
\end{definition}

$\meaningof{-} : L \to \mathcal{P}(\pi)$

\begin{mathpar}
  \inferrule* [lab=collection] {} {\meaningof{true} = \pi, \and \meaningof{~E} = \pi \setminus \meaningof{E}, \and \meaningof{E_{1} \& E_{2}} = \meaningof{E_{1}} \cap \meaningof{E_{2}}}
\end{mathpar}

\begin{mathpar}
  \inferrule* [lab=structure] {} {\meaningof{0} = \{ P \in \pi | P \equiv 0 \}, \and \\ \meaningof{E_1 | E_2} = \{ P \in \pi | P \equiv P_{1} | P_{2}, P_{1} \in \meaningof{E_{1}}, P_{2} \in \meaningof{E_2}\} }
\end{mathpar}

\begin{mathpar}
 \inferrule* [lab=behavior] {} {\meaningof{\langle a?b \rangle E} = \{ P \in \pi | P \equiv Q | u?(y)P', \\ \and \\\\ \and \\ \;\;\; u \in \meaningof{a}, \forall z.P'\{z/y\} \in \meaningof{E\{z/b\}}\}, \and \\ \meaningof{a!E} = \{ P \in \pi | P \equiv Q | x!\langle P' \rangle, x \in \meaningof{a} P' \in \meaningof{E}\} }
\end{mathpar}

\begin{mathpar}
 \inferrule* [lab=nominal] {} {\meaningof{\quotep{E}} = \{ \quotep{P} \in \quotep{\pi} | P \in \meaningof{E} \}, \and \meaningof{\quotep{P}} = \{ \quotep{Q} \in \quotep{\pi} | P \equiv Q \} \and \\ \meaningof{@\quotep{E}} = \{ P \in \pi | P \equiv @x, x \in \meaningof{E} \}}
\end{mathpar}

\begin{eqnarray*}
  \\
  \meaningof{-} : TS \to ST
\end{eqnarray*}

\begin{eqnarray*}
  \\
  L : TS \to ST
\end{eqnarray*}

\begin{eqnarray*}
  \\
  P \models E \iff P \in \meaningof{E}
\end{eqnarray*}

\begin{eqnarray*}
  P \approx_{L} Q \iff \forall E \in L. P \models E \iff Q \models E
\end{eqnarray*}

\begin{eqnarray*}
  P \approx_{K} Q
\end{eqnarray*}

\begin{eqnarray*}
  P \approx Q
\end{eqnarray*}

$\approx_{K} = \approx = \approx_{L}$

\subsubsection{Contextual duality}

Note that contexts extend the quotation operation to a family of
operations from processes to names. Given a context, $M$, we can
define a \emph{nominal context}, $\quotep{M}$ by $\quotep{M}[P] :=
\quotep{M[P]}$. To foreshadow what is to come we observe that these
operations enjoy a duality with processes very much like the duality
between vectors and maps from vectors to scalars.

Further, because the calculus is essentially higher-order, we have a
correspondence between contexts and processes. More specifically,
given a name $x$ and a context $M$ we can construct $M^{*}_{x}$ such
that 

\begin{mathpar}
  M^{*}_{x} | \lift{x}{P} \red M[P]
\end{mathpar}

namely,

\begin{mathpar}
  M^{*}_{x} := x?(u).M[\dropn{u}]
\end{mathpar}

The dependence of $M^{*}_{x}$ on a name makes it an abstraction, 

\begin{mathpar}
  M^{*} := (x)x?(u).M[\dropn{u}]
\end{mathpar}

\subsection{Additional notation}

It will sometimes be convenient to denote the process a name
quotes. We already have the notation $x = \quotep{P}$, but it will be
convenient to introduce an alternate notation, $\procn{x}$, when we
want to emphasize the connection to the use of the name. Note that, by
virtue of name equivalence, $\quotep{\procn{x}} \nameeq x$; so, the
notation is consistent with previous definitions.

Further, because names have structure it is possible to effect
substitutions on the basis of that structure. This means we need to
upgrade our notation for substitutions, which we accomplish by
adapting comprehension notation. Thus,

\begin{mathpar}
  P\{ y / x : x \in S \}
\end{mathpar}

is interpreted to mean the process derived from P by replacing (in a
capture-avoiding manner) each occurrence of $x$ in $S$ by $y$. For example,

\begin{mathpar}
  P\{ \quotep{\procn{x}|\procn{x}} / x : x \in \freenames{P} \}
\end{mathpar}

will replace each (occurrence) of a free name $x$ in $P$ by
$\quotep{\procn{x}|\procn{x}}$.

Also, we will avail ourselves of the notation $x^{L}$ and $x^{R}$ to
denote injections of a name into disjoint copies of the name
space. There are numerous ways to accomplish this. One example can be
found in \cite{MeredithR05}. This notation overloads to vectors of
names: $\vec{x}^{\pi} := (x_{i}^{\pi} \; : \; 0 \leq i < |\vec{x}| )$ where $\pi \in \{L,R\}$.

We also use $P^{\Box} := P|\Box$.

In \cite{MeredithR05} an interpretation of the new operator is
given. It turns out that there are several possible interpretations
all enjoying the requisite algebraic properties of the operator (see
\cite{milner91polyadicpi}). We will therefore make liberal use of
$(\nu\; \vec{x})P$.

% subsection the_syntax_and_semantics_of_the_notation_system (end)   

\input{qm2pi.qmops} 

\input{qm2pi.sterngerlach} 

\input{qm2pi.metric} 

% section concurrent_process_calculi (end)

%\input{qm2pi.proofsketch}

% section proof sketch (end)

%\input{qm2pi.slviaknots} 

% section spatial logic via knots (end)

\input{qm2pi.conclusion}

% section conclusion (end)

%\input{qm2pi.dtcodes} 

% section wiring algorithm (end)

\input{qm2pi.ack} 

% section acknowledgments (end)

\newpage


\bibliographystyle{plain}   
\bibliography{../../biblios/main.bib}

\input{qm2pi.rhodetails}

\end{document}

 

%\ifpdf
%\usepackage[pdftex]{graphicx}
%\else
%\usepackage{graphicx}
%\fi

 % \ifpdf
%  \usepackage{pdfsync}
%  \if


%\title{Brief Article}
%\author{David F. Snyder}
%\author{L.G. Meredith}

%\address{Dept. of Math., Texas State University--San Marcos, San Marcos, TX 78666}
       
\pagestyle{empty}


\begin{document}

\lstset{language=[Objective]Caml,frame=shadowbox}

\documentclass[12pt]{llncs}
%\documentclass{jktr}

\usepackage[pdftex]{hyperref}                   
\usepackage {listings}
\usepackage {mathpartir}
\usepackage{bcprules}
%\usepackage{listings}
                       
\usepackage{graphicx} 
%\usepackage[margins=2.5cm,nohead,nofoot]{geometry}
%\usepackage{geometry}
\usepackage{amsfonts}
\usepackage{amstext}
\usepackage{latexsym}
\usepackage{amssymb}
\usepackage{color}


%\include{myPreamble}
\include{qm2pi.local} 

%\ifpdf
%\usepackage[pdftex]{graphicx}
%\else
%\usepackage{graphicx}
%\fi

 % \ifpdf
%  \usepackage{pdfsync}
%  \if


%\title{Brief Article}
%\author{David F. Snyder}
%\author{L.G. Meredith}

%\address{Dept. of Math., Texas State University--San Marcos, San Marcos, TX 78666}
       
\pagestyle{empty}


\begin{document}

\lstset{language=[Objective]Caml,frame=shadowbox}

\input{qm2pi.front}

% section front matter (end)

\input{qm2pi.intro} 
 
% section introduction (end)

% \input{qm2pi.knotations} 

% section notation (end)

\input{qm2pi.process.calculi} 

% section concurrent_process_calculi_and_spatial_logics_ (end)
    
%\input{qm2pi.knots2pi} 

%\input{qm2pi.trefoil} 

%\input{qm2pi.mainthm} 

% subsection basic_interpretation (end)

%\input{qm2pi.rho.presentation} 
\subsection{The syntax and semantics of the notation system}\label{sub:the_syntax_and_semantics_of_the_notation_system} % (fold)

We now summarize a technical presentation of the calculus that
embodies our theory of dynamics. The typical presentation of such a
calculus follows the style of giving generators and relations on
them. The grammar, below, describing term constructors, freely
generates the set of processes, $\Proc$. This set is then quotiented
by a relation known as structural congruence and it is over this set
that the notion of dynamics is expressed. This presentation is
essentially that of \cite{MeredithR05} with the addition of
polyadicity and summation. For readability we have relegated some of
the technical subtleties to an appendix.

\subsubsection{Process grammar}\label{subsub:process_grammar}

\begin{mathpar}
  \inferrule* [lab=synchronization] {} {{M} \bc \pzero \;|\; x?F \;|\; x!C }
  \and
  \inferrule* [lab=abstraction] {} {{F} \bc (x)P}
  \and
  \inferrule* [lab=concretion] {} {{C} \bc \langle Q \rangle}
  \and
  \inferrule* [lab=process] {} {{P,Q} \bc M \;| \;P|Q \;|\; @{x}}
  \and
  \inferrule* [lab=name] {} {{x} \bc \quotep{P}}
\end{mathpar} 

Note that $\vec{x}$ (resp. $\vec{P}$) denotes a vector of names
(resp. processes) of length $|\vec{x}|$ (resp. $|\vec{P}|$). We adopt
the following useful abbreviations.

\begin{mathpar}
   x?(\vec{y}).P := x.(\vec{y})P \and  x\clift{\vec{P}} := x.\clift{\vec{P}}
   \and x!(y) := \lift{x}{\dropn{y}}
   \and \Pi_{i=0}^{n-1}P_i := P_0 | \ldots | P_{n-1}
\end{mathpar}

\subsubsection{Structural congruence}

\paragraph{Free and bound names and alpha-equivalence.} At the
core of structural equivalence is alpha-equivalence which identifies
process that are the same up to a change of variable. Formally, we
recognize the distinction between free and bound names. The free names
of a process, $\freenames{P}$, may be calculated recursively as
follows:

\begin{mathpar}
\freenames{\pzero} := \emptyset
  \and \\
  \freenames{x?(y).P} := \{ x \} \cup (\freenames{P} \setminus \{ y \})
  \and 
  \freenames{x!\langle P \rangle} := \{ x \} \cup \{ P \} 
  \and \\
  \freenames{P|Q} := \freenames{P} \cup \freenames{Q}
  \and \\
  \freenames{@{x}} := \{ x \}
\end{mathpar}

$\pi$
$\quotep{\pi}$

$\freenames{-} : \pi \to \mathcal{P}(\quotep{\pi})$

\begin{eqnarray*}
  \freenames{\pzero} & := & \emptyset \\
  \freenames{x?(y).P} & := & \{ x \} \cup (\freenames{P} \setminus \{ y \}) \\
  \freenames{x!\langle P \rangle} & := & \{ x \} \cup \{ P \} \\
  \freenames{P|Q} & := & \freenames{P} \cup \freenames{Q} \\
  \freenames{\dropn{x}} & := & \{ x \}
\end{eqnarray*}

The bound names of a process, $\boundnames{P}$, are those names occurring in $P$
that are not free. For example, in $x?(y).0$, the name $x$ is free, while $y$ is bound.

\begin{mathpar}
  \inferrule* [lab=monoidal-laws] {} { P|Q \equiv Q|P \and P|0 \equiv P \and P|(Q|R) \equiv (P|Q)|R }
\end{mathpar}

\begin{mathpar}
  \inferrule* [lab=alpha-equivalence] {} { (x)P \equiv (y)P\{y/x\} \and y \not\in \freenames{P} }
\end{mathpar}

\begin{definition}
Then two processes, $P,Q$, are alpha-equivalent if $P = Q\{\vec{y}/\vec{x}\}$ for
some $\vec{x} \in \boundnames{Q},\vec{y} \in \boundnames{P}$, where $Q\{\vec{y}/\vec{x}\}$
denotes the capture-avoiding substitution of $\vec{y}$ for $\vec{x}$ in $Q$.
\end{definition}

\begin{definition}
  The {\em structural congruence} \cite{SangiorgiWalker} , $\equiv$,
  between processes is the least congruence containing
  alpha-equivalence, satisfying the abelian monoid laws
  (associativity, commutativity and $\pzero$ as identity) for parallel
  composition $|$ and for summation $+$.
\end{definition}

\subsection{Name equivalence}

We take name equivalence, written $\nameeq$, to be the smallest
equivalence relation generated by the following rules.

\begin{mathpar}
\inferrule*[lab=Quote-drop]
{ }
{ \quotep{@{x}} \nameeq x }

\inferrule*[lab=Struct-equiv]
{ P \scong Q }
{ \quotep{P} \nameeq \quotep{Q} }
\end{mathpar}

The astute reader will have noticed that the mutual recursion of names
and processes imposes a mutual recursion on alpha-equivalence and
structural equivalence via name-equivalence. Fortunately, all of this
works out pleasantly and we may calculate in the natural way, free of
concern. The reader interested in the details is referred to the
appendix \ref{appendix:rho_details}.

\subsection{Substitution}

We use $\Proc$ for the set of processes, $\QProc$ for the set of
names, and $\id{\{}\vec{y} / \vec{x} \id{\}}$ to denote partial maps,
$s : \QProc \rightarrow \QProc$. A map, $s$ lifts, uniquely, to a map
on process terms, $\widehat{s} : \Proc \rightarrow \Proc$ by the
following equations.

\begin{mathpar}
  (0) \psubstp{Q}{P} := 0 \\
  (R \juxtap S) \psubstp{Q}{P}
  :=    
  (R)\psubstp{Q}{P} \juxtap (S) \psubstp{Q}{P} \\
  (x?(y).R) \psubstp{Q}{P}    
  :=    
  (x)\substp{Q}{P} (z)\concat( (R \psubstn{z}{y}) \psubstp{Q}{P} ) \\
  (\lift{x}{R}) \psubstp{Q}{P}  
  :=
  \lift{(x)\substp{Q}{P}}{ R \psubstp{Q}{P} } \\
%   (\dropn{x})  \psubstp{Q}{P}       
%   := 
%   \left\{ 
%     \begin{array}{ccc} 
%       \dropn{\quotep{Q}} & & x \nameeq \quotep{P} \\
%       \dropn{x} & & otherwise \\
%     \end{array}
%   \right. 
  (\dropn{x})  \psubstp{Q}{P}       
  := 
  \left\{ 
    \begin{array}{ccc} 
      Q & & x \nameeq \quotep{P} \\
      \dropn{x} & & otherwise \\
    \end{array}
  \right.
\end{mathpar}
 

where

\begin{eqnarray}
  (x)\id{\{} \lpquote Q \rpquote / \lpquote P \rpquote \id{\}}            = 
  \left\{ 
    \begin{array}{ccc}
      \lpquote Q \rpquote & & x \nameeq \lpquote P \rpquote \\
      x & & otherwise \\
    \end{array}
  \right. \nonumber
\end{eqnarray}

and $z$ is chosen distinct from $\quotep{P}$, $\quotep{Q}$, the free
names in $Q$, and all the names in $R$. Our $\alpha$-equivalence will
be built in the standard way from this substitution.

\begin{remark}\label{rem:no_self_referential_names}
  One consequence of these definitions is that $\forall P. \quotep{P}
  \not\in \freenames{P}$.
\end{remark}

\subsection{ Dynamic quote: an example }

Anticipating something of what's to come, consider applying the
substitution, $\widehat{\id{\{}u / z \id{\}}}$, to the following pair
of processes, $\lift{w}{y!(z)}$ and $w[ \lpquote y!(z) \rpquote ]$.

\begin{eqnarray}
	\lift{w}{y!(z)}\widehat{\id{\{}u / z \id{\}}}
		& = &
		\lift{w}{y!(u)} \nonumber\\
	w[ \lpquote y!(z) \rpquote ] \widehat{ \id{\{}u / z \id{\}} }
		& = &
		w[ \lpquote y!(z) \rpquote ] \nonumber
\end{eqnarray}

Because the body of the process between quotes is impervious to
substitution, we get radically different answers. In fact, by
examining the first process in an input context,
e.g. $x?(z).\lift{w}{y!(z)}$, we see that the process under the lift
operator may be shaped by prefixed inputs binding a name inside it. In
this sense, the lift operator will be seen as a way to dynamically
construct processes before reifying them as names.

Finally equipped with these standard features we can present the
dynamics of the calculus.

\subsubsection{Operational semantics} 

Finally, we introduce the computational dynamics. What marks these
algebras as distinct from other more traditionally studied algebraic
structures, e.g. vector spaces or polynomial rings, is the manner in
which dynamics is captured. In traditional structures, dynamics is typically
expressed through morphisms between such structures, as in linear maps
between vector spaces or morphisms between rings. In algebras
associated with the semantics of computation, the dynamics is
expressed as part of the algebraic structure itself, through a
reduction reduction relation typically denoted by $\red$. Below, we
give a recursive presentation of this relation for the calculus used
in the encoding.

$\red \subseteq \pi \times \pi$
$\red : \pi \to \mathcal{P}(\pi)$

\begin{mathpar}
  \inferrule* [lab=Comm] { \textsf{match}( x_{src}, x_{trgt} ) } { x_{trgt}?(y)P \; | \; x_{src}!\langle {Q} \rangle \red P\{\quotep{Q}/y}\} }
  \and \\
  \inferrule* [lab=Par] {{P} \red {P}'} {{{P} | {Q}} \red {{P}' | {Q}}}
  \and
  \inferrule* [lab=Equiv]{{{P} \scong {P}'} \andalso {{P}' \red {Q}'} \andalso {{Q}' \scong {Q}}}{{P} \red {Q}}
\end{mathpar}

\begin{eqnarray*}
  match_{\equiv} (\quotep{P},\quotep{Q}) & := & P \equiv Q \\
  match_{\dagger}(\quotep{P},\quotep{Q}) & := & \forall R. P|Q \red^{*} R => R \red^{*} 0 \\
  match_{K}(\quotep{P},\quotep{Q}) & := & K \mbox{ for some context } K
\end{eqnarray*}

$u?(x)P | u!\langle Q \rangle \red P\{\quotep{Q}/x\}$

%We write $\wred$ for $\red^*$, and $P\red$ if $\exists Q $ such that $ P \red Q$.
We write $P\red$ if $\exists Q $ such that $ P \red Q$ and $P\not\red$, otherwise.

\section{Replication}

As mentioned before, it is known that replication (and hence
recursion) can be implemented in a higher-order process algebra
\cite{SangiorgiWalker}. As our first example of calculation with the
machinery thus far presented we give the construction explicitly in
the {\rhoc}.

\begin{eqnarray}
	D_{x} & := & \prefix{x}{y}{(\binpar{\outputp{x}{y}}{@{y}})} \nonumber\\
	\bangp_{x}{P} & := & \binpar{{x}!\langle{\binpar{D_{x}}{P}}\rangle}{D_{x}} \nonumber
\end{eqnarray}

\begin{eqnarray}
	\bangp_{x}{P} & & \nonumber\\
	=
	& {x}!\langle{(\prefix{x}{y}{(\outputp{x}{y} | @{y})) | P}}\rangle 
	      | \prefix{x}{y}{(\outputp{x}{y} | @{y})} & \nonumber\\
	\red
	& (\outputp{x}{y} | @{y})\substn{\quotep{(\prefix{x}{y}{(@{y} | \outputp{x}{y})) | P}}}{y} & \nonumber\\
	=
	& \outputp{x}{\quotep{(\prefix{x}{y}{(\outputp{x}{y} | @{y})) | P}}}
	  | {(\prefix{x}{y}{(\outputp{x}{y} | @{y})) | P}} & \nonumber\\
	\red
	& \ldots & \nonumber\\
	\red^*
	& P | P | \ldots & \nonumber
\end{eqnarray}

Of course, this encoding, as an implementation, runs away, unfolding
$\bangp{P}$ eagerly. A lazier and more implementable replication
operator, restricted to input-guarded processes, may be obtained as follows.

\begin{eqnarray}
\bangp{\prefix{u}{v}{P}} 
	:= 
	\binpar{\lift{x}{\prefix{u}{v}{(\binpar{D(x)}{P})}}}{D(x)} \nonumber
\end{eqnarray}

\begin{remark}
  Note that the lazier definition still does not deal with summation
  or mixed summation (i.e. sums over input and output). The reader is
  invited to construct definitions of replication that deal with these
  features. 

  Further, the definitions are parameterized in a name, $x$. Can you,
  gentle reader, make a definition that eliminates this parameter and
  guarantees no accidental interaction between the replication
  machinery and the process being replicated -- i.e. no accidental
  sharing of names used by the process to get its work done and the
  name(s) used by the replication to effect copying. This latter
  revision of the definition of replication is crucial to obtaining
  the expected identity $!!P \sim !P$.
\end{remark}

\begin{remark}\label{rem:paradoxical_combinator}
  The reader familiar with the lambda calculus will have noticed the
  similarity between $D$ and the paradoxical combinator.

  [Ed. note: the existence of this seems to suggest we have to be more
  restrictive on the set of processes and names we admit if we are to
  support no-cloning.]
\end{remark}

\subsubsection{Bisimulation}

The computational dynamics gives rise to another kind of equivalence,
the equivalence of computational behavior. As previously mentioned
this is typically captured \emph{via} some form of bisimulation.

% The notion we use in this paper is weak barbed bisimulation
% \cite{milner91polyadicpi}.

The notion we use in this paper is derived from weak barbed
bisimulation \cite{milner91polyadicpi}. 

\begin{definition}
An \emph{observation relation}, $\downarrow_{\mathcal N}$, over a set
of names, $\mathcal N$, is the smallest relation satisfying the rules
below.

\infrule[Out-barb]{y \in {\mathcal N}, \; x \nameeq y}
		  {\outputp{x}{v} \downarrow_{\mathcal N} x}
\infrule[Par-barb]{\mbox{$P\downarrow_{\mathcal N} x$ or $Q\downarrow_{\mathcal N} x$}}
		  {\binpar{P}{Q} \downarrow_{\mathcal N} x}

We write $P \Downarrow_{\mathcal N} x$ if there is $Q$ such that 
$P \wred Q$ and $Q \downarrow_{\mathcal N} x$.
\end{definition}

\begin{definition}
%\label{def.bbisim}
An  ${\mathcal N}$-\emph{barbed bisimulation} over a set of names, ${\mathcal N}$, is a symmetric binary relation 
${\mathcal S}_{\mathcal N}$ between agents such that $P\rel{S}_{\mathcal N}Q$ implies:
\begin{enumerate}
\item If $P \red P'$ then $Q \wred Q'$ and $P'\rel{S}_{\mathcal N} Q'$.
\item If $P\downarrow_{\mathcal N} x$, then $Q\Downarrow_{\mathcal N} x$.
\end{enumerate}
$P$ is ${\mathcal N}$-barbed bisimilar to $Q$, written
$P \wbbisim_{\mathcal N} Q$, if $P \rel{S}_{\mathcal N} Q$ for some ${\mathcal N}$-barbed bisimulation ${\mathcal S}_{\mathcal N}$.
\end{definition}

$\mathcal{R} \subseteq \pi \times \pi$

$P \mathcal{R} Q => \forall P'. P \red P' \Rightarrow \exists Q'. Q \red Q', P' \mathcal{R} Q'$

$P \vdash x \Rightarrow Q \vdash x$

\begin{mathpar}
  \inferrule*[lab=Out-barb]{x \nameeq y}{{y}!\langle{Q}\rangle \vdash x}
  \and
  \inferrule*[lab=Par-barb]{\mbox{$P\vdash x$ or $Q\vdash x$}}{\binpar{P}{Q} \vdash x}
\end{mathpar}

\subsubsection{Contexts}

One of the principle advantages of computational calculi like the
$\pi$-calculus is a well-defined notion of context,
contextual-equivalence and a correlation between
contextual-equivalence and notions of bisimulation. The notion of
context allows the decomposition of a process into (sub-)process and
its syntactic environment, its context. Thus, a context may be
thought of as a process with a ``hole'' (written $\Box$) in it. The
application of a context $M$ to a process $P$, written $M[P]$, is
tantamount to filling the hole in $M$ with $P$. In this paper we do
not need the full weight of this theory, but do make use of the notion
of context in the proof the main theorem. 

\begin{mathpar}
  \inferrule* [lab=summation] {} {{M_{M},M_{N}} \bc \Box \;|\; x.M_{A} \;|\; M_{M}+M_{N}}
  \and
  \inferrule* [lab=agent] {} {{M_{A}} \bc (\vec{x})M_{P} \;| \; \clift{P_0,\ldots,M_{P},\ldots,P_N}}
  \and \\
  \inferrule* [lab=process] {} {{M_{P}} \bc M_{N} \;| \;P|M_{P} }
\end{mathpar} 

\begin{mathpar}
  \inferrule* [lab=sychronization] {} {M_{N} \bc \Box \;|\; x?M_{F} \;|\; x!M_{C}}
  \and
  \inferrule* [lab=abstraction] {} {{M_{F}} \bc (x)M_{P} }
  \and
  \inferrule* [lab=concretion] {} {{M_{C}} \bc \langle M_{P} \rangle }
  \and \\
  \inferrule* [lab=process] {} {{M_{P}} \bc M_{N} \;| \;P|M_{P} }
\end{mathpar}

\begin{definition}[contextual application] Given a context $M$, and
  process $P$, we define the \emph{contextual application}, $M[P] :=
  M\{P/\Box\}$. That is, the contextual application of M to P is the
  substitution of $P$ for $\Box$ in $M$.
\end{definition}

$\meaningof{-} : L \to \mathcal{P}(\pi)$

\begin{mathpar}
  \inferrule* [lab=collection] {} {\meaningof{true} = \pi, \and \meaningof{~E} = \pi \setminus \meaningof{E}, \and \meaningof{E_{1} \& E_{2}} = \meaningof{E_{1}} \cap \meaningof{E_{2}}}
\end{mathpar}

\begin{mathpar}
  \inferrule* [lab=structure] {} {\meaningof{0} = \{ P \in \pi | P \equiv 0 \}, \and \\ \meaningof{E_1 | E_2} = \{ P \in \pi | P \equiv P_{1} | P_{2}, P_{1} \in \meaningof{E_{1}}, P_{2} \in \meaningof{E_2}\} }
\end{mathpar}

\begin{mathpar}
 \inferrule* [lab=behavior] {} {\meaningof{\langle a?b \rangle E} = \{ P \in \pi | P \equiv Q | u?(y)P', \\ \and \\\\ \and \\ \;\;\; u \in \meaningof{a}, \forall z.P'\{z/y\} \in \meaningof{E\{z/b\}}\}, \and \\ \meaningof{a!E} = \{ P \in \pi | P \equiv Q | x!\langle P' \rangle, x \in \meaningof{a} P' \in \meaningof{E}\} }
\end{mathpar}

\begin{mathpar}
 \inferrule* [lab=nominal] {} {\meaningof{\quotep{E}} = \{ \quotep{P} \in \quotep{\pi} | P \in \meaningof{E} \}, \and \meaningof{\quotep{P}} = \{ \quotep{Q} \in \quotep{\pi} | P \equiv Q \} \and \\ \meaningof{@\quotep{E}} = \{ P \in \pi | P \equiv @x, x \in \meaningof{E} \}}
\end{mathpar}

\begin{eqnarray*}
  \\
  \meaningof{-} : TS \to ST
\end{eqnarray*}

\begin{eqnarray*}
  \\
  L : TS \to ST
\end{eqnarray*}

\begin{eqnarray*}
  \\
  P \models E \iff P \in \meaningof{E}
\end{eqnarray*}

\begin{eqnarray*}
  P \approx_{L} Q \iff \forall E \in L. P \models E \iff Q \models E
\end{eqnarray*}

\begin{eqnarray*}
  P \approx_{K} Q
\end{eqnarray*}

\begin{eqnarray*}
  P \approx Q
\end{eqnarray*}

$\approx_{K} = \approx = \approx_{L}$

\subsubsection{Contextual duality}

Note that contexts extend the quotation operation to a family of
operations from processes to names. Given a context, $M$, we can
define a \emph{nominal context}, $\quotep{M}$ by $\quotep{M}[P] :=
\quotep{M[P]}$. To foreshadow what is to come we observe that these
operations enjoy a duality with processes very much like the duality
between vectors and maps from vectors to scalars.

Further, because the calculus is essentially higher-order, we have a
correspondence between contexts and processes. More specifically,
given a name $x$ and a context $M$ we can construct $M^{*}_{x}$ such
that 

\begin{mathpar}
  M^{*}_{x} | \lift{x}{P} \red M[P]
\end{mathpar}

namely,

\begin{mathpar}
  M^{*}_{x} := x?(u).M[\dropn{u}]
\end{mathpar}

The dependence of $M^{*}_{x}$ on a name makes it an abstraction, 

\begin{mathpar}
  M^{*} := (x)x?(u).M[\dropn{u}]
\end{mathpar}

\subsection{Additional notation}

It will sometimes be convenient to denote the process a name
quotes. We already have the notation $x = \quotep{P}$, but it will be
convenient to introduce an alternate notation, $\procn{x}$, when we
want to emphasize the connection to the use of the name. Note that, by
virtue of name equivalence, $\quotep{\procn{x}} \nameeq x$; so, the
notation is consistent with previous definitions.

Further, because names have structure it is possible to effect
substitutions on the basis of that structure. This means we need to
upgrade our notation for substitutions, which we accomplish by
adapting comprehension notation. Thus,

\begin{mathpar}
  P\{ y / x : x \in S \}
\end{mathpar}

is interpreted to mean the process derived from P by replacing (in a
capture-avoiding manner) each occurrence of $x$ in $S$ by $y$. For example,

\begin{mathpar}
  P\{ \quotep{\procn{x}|\procn{x}} / x : x \in \freenames{P} \}
\end{mathpar}

will replace each (occurrence) of a free name $x$ in $P$ by
$\quotep{\procn{x}|\procn{x}}$.

Also, we will avail ourselves of the notation $x^{L}$ and $x^{R}$ to
denote injections of a name into disjoint copies of the name
space. There are numerous ways to accomplish this. One example can be
found in \cite{MeredithR05}. This notation overloads to vectors of
names: $\vec{x}^{\pi} := (x_{i}^{\pi} \; : \; 0 \leq i < |\vec{x}| )$ where $\pi \in \{L,R\}$.

We also use $P^{\Box} := P|\Box$.

In \cite{MeredithR05} an interpretation of the new operator is
given. It turns out that there are several possible interpretations
all enjoying the requisite algebraic properties of the operator (see
\cite{milner91polyadicpi}). We will therefore make liberal use of
$(\nu\; \vec{x})P$.

% subsection the_syntax_and_semantics_of_the_notation_system (end)   

\input{qm2pi.qmops} 

\input{qm2pi.sterngerlach} 

\input{qm2pi.metric} 

% section concurrent_process_calculi (end)

%\input{qm2pi.proofsketch}

% section proof sketch (end)

%\input{qm2pi.slviaknots} 

% section spatial logic via knots (end)

\input{qm2pi.conclusion}

% section conclusion (end)

%\input{qm2pi.dtcodes} 

% section wiring algorithm (end)

\input{qm2pi.ack} 

% section acknowledgments (end)

\newpage


\bibliographystyle{plain}   
\bibliography{../../biblios/main.bib}

\input{qm2pi.rhodetails}

\end{document}



% section front matter (end)

\section{Introduction}\label{sec:introduction} % (fold)
In this draft of the material i am going to have to dispense with the
usual writing conventions adopted in papers on these topics. i'm going
to have adopt whatever tone i need at the time i'm writing up the
calculations. Sometimes this may be very conversational; others it may
be the barest mathematical grunts; others still it may be that i have
lifted text from one of my other papers because the exposition of some
point was better said there. i hope that my readers are not unduly put
out by this decision. i'm not doing this to flout convention or be
rebellious. i find these calculations very technically challenging. To
keep everything going technically, something has to give; i have to
let go of some cognitive burden. So, the academic writing style --
with all of its trade-offs in terms of facilitating technical
communication -- is what i'm letting go of. Perhaps subsequent drafts
can be tightened and polished, but for now, i'm going to speak as if
we were sitting together in a coffee shop with a laptop, wifi and a
pad of paper and a pencil.

So, here's what i have to say. We -- you and i, comfortably ensconced
in our coffee shop and well-equipped with our tools -- can realize and
carry out the calculations of quantum mechanics over a very different
formal theory of dynamics, a formal theory of dynamics that
corresponds to a theory of concurrent computation with
\emph{reflection}. It has the advantage that the underlying theory is
already `quantized', but supports analogues all of the continuuous
operations. Strikingly, this underlying theory has recently been
connected with a notion of metric that we can show, by calculating
together, coincides with the metric induced by the inner product.

There are a lot of reasons why you might be interested in seeing
calculations of this form. Here's why i'm interested. For the past
several centuries there has been no competitor to the ``Newtonian''
account of dynamics. As a result the predominant share of accounts of
dynamical systems and situations have had to be formulated in terms of
the Newtonian machinery. i view this as an intellectually dangerous
position to occupy. Everything, despite it's intrinsic shape, turns
into a nail to be hit with this hammer. Recently, however, the theory
of computation has matured to the point where we have candidates for
theories of dynamics that offer very different perspective on
reasoning about dynamical systems and situations. Testing these
candidates against very successful accounts of dynamical situations,
like quantum mechanics, is going to give us some sense of how mature
they are and some measure of the quality of these accounts of
dynamics.

\subsection{Summary of contributions and outline of paper}

So, we're going to develop an interpretation of the operations of
quantum mechanics normally interpreted by Hilbert spaces and
operators. We're going to do this over a theory of computation. Note
that this is very different than the usual quantum computation program
which develops notions of computation over quantum mechanics. Rather,
we are developing a story that aligns with Wheeler's slogan: It from
Bit. To do this we will first provide an account of the theory of
computation at play here. Then we will dive into a calculation-driven
interpretation of the operations of quantum mechanics.

The reason we take this approach is that -- until very recently --
there hasn't been an axiomatic account of quantum mechanics. As a
result there has been no sharp delineation of the mathematical theory
supporting interpretation of the physical theory and the physical
theory, itself. So, ambient features of the maths are free to be
exploited (or supressed) without a real accounting of their physical
relevance. There is no sharp statement ``here's the physical theory''
qua \emph{theory} and ``here's the mathematical interpretation''
enabling a judgment of how faithful the interpretation is -- apart
from experimental observation. When there is an axiomatic account we
can judge how well a given mathematical formalism supports an
interpretation of the axioms, independent of
experimentation. Likewise, we can judge how well we have captured our
physical evidence and experience with our axiomatics, independent of
any specific mathematical implementation, with accidental detail that
may or may not have physical significance. 

In lieu of a fully fleshed out and vetted axiomatic account of quantum
mechanics, interpreting the operational notions in service of modeling
physical systems will have to suffice. In other words, we are not in
the business of providing a model of Hilbert spaces and operators. We
are in the business of providing a model of quantum mechanics because
we are motivated by testing our notions of dynamics against physical
theory; and, the predictive calculations of the physical theory must
serve as the best formulation -- shy of a fully fleshed out axiomatic
account -- of the physical theory itself (as they have for scientific
theories since time immemorial). Put another way, despite a
whole-hearted commitment to an It-from-Bit ontology, we are firmly
aligned with the shut-up-and-calculate camp as the best way to obtain
results either from the physical perspective or as a quality assurance
measure of our fledgling theory of dynamics.

In detail, we present a reflective process calculus. Then we develop
intuitive correspondences between the notions available in this
calculus and the usual physical notions supporting quantum mechanical
calculations. Thus, 

\begin{table}[htp]
  \center{
    \fbox{
      \begin{tabular}{c|c}
        quantum mechanics & process calculus \\
        \hline
        scalar & name \\
        state vector & process \\
        dual & contextual duals \\
        matrix & formal sums of process-context-dual pairs \\
        orthogonality & process annihilation \\
        inner product & execution-formula + quoting
      \end{tabular}
    }
  }
  \caption{QM - process calculi correspondences}
\end{table}

Then we tighten up these intuitions to operational definitions. We
employ the Dirac notation as the best proxy we can find for an
abstract syntax of the quantum mechanical notions. The definitions we
develop put us in contact with equational constraints coming from the
theory that we demonstrate the definitions and calculations satisfy.

This puts us in a position to shut up and calculate for the
Stern-Gerlach experimental set up, showing how these predictive
calculations become calculations on processes in our theory of a
reflective process calculus.

Penultimately, we demonstrate that the notion of metric coming from
the inner product coincides with the notion of metric available from
the theory of bisimulation. This demonstration gives us the right to
think of space as arising from behavior. Finally, we consider where we
might go from the new vantage point we have obtained.

% section introduction (end) 
 
% section introduction (end)

% \documentclass[12pt]{llncs}
%\documentclass{jktr}

\usepackage[pdftex]{hyperref}                   
\usepackage {listings}
\usepackage {mathpartir}
\usepackage{bcprules}
%\usepackage{listings}
                       
\usepackage{graphicx} 
%\usepackage[margins=2.5cm,nohead,nofoot]{geometry}
%\usepackage{geometry}
\usepackage{amsfonts}
\usepackage{amstext}
\usepackage{latexsym}
\usepackage{amssymb}
\usepackage{color}


%\include{myPreamble}
\include{qm2pi.local} 

%\ifpdf
%\usepackage[pdftex]{graphicx}
%\else
%\usepackage{graphicx}
%\fi

 % \ifpdf
%  \usepackage{pdfsync}
%  \if


%\title{Brief Article}
%\author{David F. Snyder}
%\author{L.G. Meredith}

%\address{Dept. of Math., Texas State University--San Marcos, San Marcos, TX 78666}
       
\pagestyle{empty}


\begin{document}

\lstset{language=[Objective]Caml,frame=shadowbox}

\input{qm2pi.front}

% section front matter (end)

\input{qm2pi.intro} 
 
% section introduction (end)

% \input{qm2pi.knotations} 

% section notation (end)

\input{qm2pi.process.calculi} 

% section concurrent_process_calculi_and_spatial_logics_ (end)
    
%\input{qm2pi.knots2pi} 

%\input{qm2pi.trefoil} 

%\input{qm2pi.mainthm} 

% subsection basic_interpretation (end)

%\input{qm2pi.rho.presentation} 
\subsection{The syntax and semantics of the notation system}\label{sub:the_syntax_and_semantics_of_the_notation_system} % (fold)

We now summarize a technical presentation of the calculus that
embodies our theory of dynamics. The typical presentation of such a
calculus follows the style of giving generators and relations on
them. The grammar, below, describing term constructors, freely
generates the set of processes, $\Proc$. This set is then quotiented
by a relation known as structural congruence and it is over this set
that the notion of dynamics is expressed. This presentation is
essentially that of \cite{MeredithR05} with the addition of
polyadicity and summation. For readability we have relegated some of
the technical subtleties to an appendix.

\subsubsection{Process grammar}\label{subsub:process_grammar}

\begin{mathpar}
  \inferrule* [lab=synchronization] {} {{M} \bc \pzero \;|\; x?F \;|\; x!C }
  \and
  \inferrule* [lab=abstraction] {} {{F} \bc (x)P}
  \and
  \inferrule* [lab=concretion] {} {{C} \bc \langle Q \rangle}
  \and
  \inferrule* [lab=process] {} {{P,Q} \bc M \;| \;P|Q \;|\; @{x}}
  \and
  \inferrule* [lab=name] {} {{x} \bc \quotep{P}}
\end{mathpar} 

Note that $\vec{x}$ (resp. $\vec{P}$) denotes a vector of names
(resp. processes) of length $|\vec{x}|$ (resp. $|\vec{P}|$). We adopt
the following useful abbreviations.

\begin{mathpar}
   x?(\vec{y}).P := x.(\vec{y})P \and  x\clift{\vec{P}} := x.\clift{\vec{P}}
   \and x!(y) := \lift{x}{\dropn{y}}
   \and \Pi_{i=0}^{n-1}P_i := P_0 | \ldots | P_{n-1}
\end{mathpar}

\subsubsection{Structural congruence}

\paragraph{Free and bound names and alpha-equivalence.} At the
core of structural equivalence is alpha-equivalence which identifies
process that are the same up to a change of variable. Formally, we
recognize the distinction between free and bound names. The free names
of a process, $\freenames{P}$, may be calculated recursively as
follows:

\begin{mathpar}
\freenames{\pzero} := \emptyset
  \and \\
  \freenames{x?(y).P} := \{ x \} \cup (\freenames{P} \setminus \{ y \})
  \and 
  \freenames{x!\langle P \rangle} := \{ x \} \cup \{ P \} 
  \and \\
  \freenames{P|Q} := \freenames{P} \cup \freenames{Q}
  \and \\
  \freenames{@{x}} := \{ x \}
\end{mathpar}

$\pi$
$\quotep{\pi}$

$\freenames{-} : \pi \to \mathcal{P}(\quotep{\pi})$

\begin{eqnarray*}
  \freenames{\pzero} & := & \emptyset \\
  \freenames{x?(y).P} & := & \{ x \} \cup (\freenames{P} \setminus \{ y \}) \\
  \freenames{x!\langle P \rangle} & := & \{ x \} \cup \{ P \} \\
  \freenames{P|Q} & := & \freenames{P} \cup \freenames{Q} \\
  \freenames{\dropn{x}} & := & \{ x \}
\end{eqnarray*}

The bound names of a process, $\boundnames{P}$, are those names occurring in $P$
that are not free. For example, in $x?(y).0$, the name $x$ is free, while $y$ is bound.

\begin{mathpar}
  \inferrule* [lab=monoidal-laws] {} { P|Q \equiv Q|P \and P|0 \equiv P \and P|(Q|R) \equiv (P|Q)|R }
\end{mathpar}

\begin{mathpar}
  \inferrule* [lab=alpha-equivalence] {} { (x)P \equiv (y)P\{y/x\} \and y \not\in \freenames{P} }
\end{mathpar}

\begin{definition}
Then two processes, $P,Q$, are alpha-equivalent if $P = Q\{\vec{y}/\vec{x}\}$ for
some $\vec{x} \in \boundnames{Q},\vec{y} \in \boundnames{P}$, where $Q\{\vec{y}/\vec{x}\}$
denotes the capture-avoiding substitution of $\vec{y}$ for $\vec{x}$ in $Q$.
\end{definition}

\begin{definition}
  The {\em structural congruence} \cite{SangiorgiWalker} , $\equiv$,
  between processes is the least congruence containing
  alpha-equivalence, satisfying the abelian monoid laws
  (associativity, commutativity and $\pzero$ as identity) for parallel
  composition $|$ and for summation $+$.
\end{definition}

\subsection{Name equivalence}

We take name equivalence, written $\nameeq$, to be the smallest
equivalence relation generated by the following rules.

\begin{mathpar}
\inferrule*[lab=Quote-drop]
{ }
{ \quotep{@{x}} \nameeq x }

\inferrule*[lab=Struct-equiv]
{ P \scong Q }
{ \quotep{P} \nameeq \quotep{Q} }
\end{mathpar}

The astute reader will have noticed that the mutual recursion of names
and processes imposes a mutual recursion on alpha-equivalence and
structural equivalence via name-equivalence. Fortunately, all of this
works out pleasantly and we may calculate in the natural way, free of
concern. The reader interested in the details is referred to the
appendix \ref{appendix:rho_details}.

\subsection{Substitution}

We use $\Proc$ for the set of processes, $\QProc$ for the set of
names, and $\id{\{}\vec{y} / \vec{x} \id{\}}$ to denote partial maps,
$s : \QProc \rightarrow \QProc$. A map, $s$ lifts, uniquely, to a map
on process terms, $\widehat{s} : \Proc \rightarrow \Proc$ by the
following equations.

\begin{mathpar}
  (0) \psubstp{Q}{P} := 0 \\
  (R \juxtap S) \psubstp{Q}{P}
  :=    
  (R)\psubstp{Q}{P} \juxtap (S) \psubstp{Q}{P} \\
  (x?(y).R) \psubstp{Q}{P}    
  :=    
  (x)\substp{Q}{P} (z)\concat( (R \psubstn{z}{y}) \psubstp{Q}{P} ) \\
  (\lift{x}{R}) \psubstp{Q}{P}  
  :=
  \lift{(x)\substp{Q}{P}}{ R \psubstp{Q}{P} } \\
%   (\dropn{x})  \psubstp{Q}{P}       
%   := 
%   \left\{ 
%     \begin{array}{ccc} 
%       \dropn{\quotep{Q}} & & x \nameeq \quotep{P} \\
%       \dropn{x} & & otherwise \\
%     \end{array}
%   \right. 
  (\dropn{x})  \psubstp{Q}{P}       
  := 
  \left\{ 
    \begin{array}{ccc} 
      Q & & x \nameeq \quotep{P} \\
      \dropn{x} & & otherwise \\
    \end{array}
  \right.
\end{mathpar}
 

where

\begin{eqnarray}
  (x)\id{\{} \lpquote Q \rpquote / \lpquote P \rpquote \id{\}}            = 
  \left\{ 
    \begin{array}{ccc}
      \lpquote Q \rpquote & & x \nameeq \lpquote P \rpquote \\
      x & & otherwise \\
    \end{array}
  \right. \nonumber
\end{eqnarray}

and $z$ is chosen distinct from $\quotep{P}$, $\quotep{Q}$, the free
names in $Q$, and all the names in $R$. Our $\alpha$-equivalence will
be built in the standard way from this substitution.

\begin{remark}\label{rem:no_self_referential_names}
  One consequence of these definitions is that $\forall P. \quotep{P}
  \not\in \freenames{P}$.
\end{remark}

\subsection{ Dynamic quote: an example }

Anticipating something of what's to come, consider applying the
substitution, $\widehat{\id{\{}u / z \id{\}}}$, to the following pair
of processes, $\lift{w}{y!(z)}$ and $w[ \lpquote y!(z) \rpquote ]$.

\begin{eqnarray}
	\lift{w}{y!(z)}\widehat{\id{\{}u / z \id{\}}}
		& = &
		\lift{w}{y!(u)} \nonumber\\
	w[ \lpquote y!(z) \rpquote ] \widehat{ \id{\{}u / z \id{\}} }
		& = &
		w[ \lpquote y!(z) \rpquote ] \nonumber
\end{eqnarray}

Because the body of the process between quotes is impervious to
substitution, we get radically different answers. In fact, by
examining the first process in an input context,
e.g. $x?(z).\lift{w}{y!(z)}$, we see that the process under the lift
operator may be shaped by prefixed inputs binding a name inside it. In
this sense, the lift operator will be seen as a way to dynamically
construct processes before reifying them as names.

Finally equipped with these standard features we can present the
dynamics of the calculus.

\subsubsection{Operational semantics} 

Finally, we introduce the computational dynamics. What marks these
algebras as distinct from other more traditionally studied algebraic
structures, e.g. vector spaces or polynomial rings, is the manner in
which dynamics is captured. In traditional structures, dynamics is typically
expressed through morphisms between such structures, as in linear maps
between vector spaces or morphisms between rings. In algebras
associated with the semantics of computation, the dynamics is
expressed as part of the algebraic structure itself, through a
reduction reduction relation typically denoted by $\red$. Below, we
give a recursive presentation of this relation for the calculus used
in the encoding.

$\red \subseteq \pi \times \pi$
$\red : \pi \to \mathcal{P}(\pi)$

\begin{mathpar}
  \inferrule* [lab=Comm] { \textsf{match}( x_{src}, x_{trgt} ) } { x_{trgt}?(y)P \; | \; x_{src}!\langle {Q} \rangle \red P\{\quotep{Q}/y}\} }
  \and \\
  \inferrule* [lab=Par] {{P} \red {P}'} {{{P} | {Q}} \red {{P}' | {Q}}}
  \and
  \inferrule* [lab=Equiv]{{{P} \scong {P}'} \andalso {{P}' \red {Q}'} \andalso {{Q}' \scong {Q}}}{{P} \red {Q}}
\end{mathpar}

\begin{eqnarray*}
  match_{\equiv} (\quotep{P},\quotep{Q}) & := & P \equiv Q \\
  match_{\dagger}(\quotep{P},\quotep{Q}) & := & \forall R. P|Q \red^{*} R => R \red^{*} 0 \\
  match_{K}(\quotep{P},\quotep{Q}) & := & K \mbox{ for some context } K
\end{eqnarray*}

$u?(x)P | u!\langle Q \rangle \red P\{\quotep{Q}/x\}$

%We write $\wred$ for $\red^*$, and $P\red$ if $\exists Q $ such that $ P \red Q$.
We write $P\red$ if $\exists Q $ such that $ P \red Q$ and $P\not\red$, otherwise.

\section{Replication}

As mentioned before, it is known that replication (and hence
recursion) can be implemented in a higher-order process algebra
\cite{SangiorgiWalker}. As our first example of calculation with the
machinery thus far presented we give the construction explicitly in
the {\rhoc}.

\begin{eqnarray}
	D_{x} & := & \prefix{x}{y}{(\binpar{\outputp{x}{y}}{@{y}})} \nonumber\\
	\bangp_{x}{P} & := & \binpar{{x}!\langle{\binpar{D_{x}}{P}}\rangle}{D_{x}} \nonumber
\end{eqnarray}

\begin{eqnarray}
	\bangp_{x}{P} & & \nonumber\\
	=
	& {x}!\langle{(\prefix{x}{y}{(\outputp{x}{y} | @{y})) | P}}\rangle 
	      | \prefix{x}{y}{(\outputp{x}{y} | @{y})} & \nonumber\\
	\red
	& (\outputp{x}{y} | @{y})\substn{\quotep{(\prefix{x}{y}{(@{y} | \outputp{x}{y})) | P}}}{y} & \nonumber\\
	=
	& \outputp{x}{\quotep{(\prefix{x}{y}{(\outputp{x}{y} | @{y})) | P}}}
	  | {(\prefix{x}{y}{(\outputp{x}{y} | @{y})) | P}} & \nonumber\\
	\red
	& \ldots & \nonumber\\
	\red^*
	& P | P | \ldots & \nonumber
\end{eqnarray}

Of course, this encoding, as an implementation, runs away, unfolding
$\bangp{P}$ eagerly. A lazier and more implementable replication
operator, restricted to input-guarded processes, may be obtained as follows.

\begin{eqnarray}
\bangp{\prefix{u}{v}{P}} 
	:= 
	\binpar{\lift{x}{\prefix{u}{v}{(\binpar{D(x)}{P})}}}{D(x)} \nonumber
\end{eqnarray}

\begin{remark}
  Note that the lazier definition still does not deal with summation
  or mixed summation (i.e. sums over input and output). The reader is
  invited to construct definitions of replication that deal with these
  features. 

  Further, the definitions are parameterized in a name, $x$. Can you,
  gentle reader, make a definition that eliminates this parameter and
  guarantees no accidental interaction between the replication
  machinery and the process being replicated -- i.e. no accidental
  sharing of names used by the process to get its work done and the
  name(s) used by the replication to effect copying. This latter
  revision of the definition of replication is crucial to obtaining
  the expected identity $!!P \sim !P$.
\end{remark}

\begin{remark}\label{rem:paradoxical_combinator}
  The reader familiar with the lambda calculus will have noticed the
  similarity between $D$ and the paradoxical combinator.

  [Ed. note: the existence of this seems to suggest we have to be more
  restrictive on the set of processes and names we admit if we are to
  support no-cloning.]
\end{remark}

\subsubsection{Bisimulation}

The computational dynamics gives rise to another kind of equivalence,
the equivalence of computational behavior. As previously mentioned
this is typically captured \emph{via} some form of bisimulation.

% The notion we use in this paper is weak barbed bisimulation
% \cite{milner91polyadicpi}.

The notion we use in this paper is derived from weak barbed
bisimulation \cite{milner91polyadicpi}. 

\begin{definition}
An \emph{observation relation}, $\downarrow_{\mathcal N}$, over a set
of names, $\mathcal N$, is the smallest relation satisfying the rules
below.

\infrule[Out-barb]{y \in {\mathcal N}, \; x \nameeq y}
		  {\outputp{x}{v} \downarrow_{\mathcal N} x}
\infrule[Par-barb]{\mbox{$P\downarrow_{\mathcal N} x$ or $Q\downarrow_{\mathcal N} x$}}
		  {\binpar{P}{Q} \downarrow_{\mathcal N} x}

We write $P \Downarrow_{\mathcal N} x$ if there is $Q$ such that 
$P \wred Q$ and $Q \downarrow_{\mathcal N} x$.
\end{definition}

\begin{definition}
%\label{def.bbisim}
An  ${\mathcal N}$-\emph{barbed bisimulation} over a set of names, ${\mathcal N}$, is a symmetric binary relation 
${\mathcal S}_{\mathcal N}$ between agents such that $P\rel{S}_{\mathcal N}Q$ implies:
\begin{enumerate}
\item If $P \red P'$ then $Q \wred Q'$ and $P'\rel{S}_{\mathcal N} Q'$.
\item If $P\downarrow_{\mathcal N} x$, then $Q\Downarrow_{\mathcal N} x$.
\end{enumerate}
$P$ is ${\mathcal N}$-barbed bisimilar to $Q$, written
$P \wbbisim_{\mathcal N} Q$, if $P \rel{S}_{\mathcal N} Q$ for some ${\mathcal N}$-barbed bisimulation ${\mathcal S}_{\mathcal N}$.
\end{definition}

$\mathcal{R} \subseteq \pi \times \pi$

$P \mathcal{R} Q => \forall P'. P \red P' \Rightarrow \exists Q'. Q \red Q', P' \mathcal{R} Q'$

$P \vdash x \Rightarrow Q \vdash x$

\begin{mathpar}
  \inferrule*[lab=Out-barb]{x \nameeq y}{{y}!\langle{Q}\rangle \vdash x}
  \and
  \inferrule*[lab=Par-barb]{\mbox{$P\vdash x$ or $Q\vdash x$}}{\binpar{P}{Q} \vdash x}
\end{mathpar}

\subsubsection{Contexts}

One of the principle advantages of computational calculi like the
$\pi$-calculus is a well-defined notion of context,
contextual-equivalence and a correlation between
contextual-equivalence and notions of bisimulation. The notion of
context allows the decomposition of a process into (sub-)process and
its syntactic environment, its context. Thus, a context may be
thought of as a process with a ``hole'' (written $\Box$) in it. The
application of a context $M$ to a process $P$, written $M[P]$, is
tantamount to filling the hole in $M$ with $P$. In this paper we do
not need the full weight of this theory, but do make use of the notion
of context in the proof the main theorem. 

\begin{mathpar}
  \inferrule* [lab=summation] {} {{M_{M},M_{N}} \bc \Box \;|\; x.M_{A} \;|\; M_{M}+M_{N}}
  \and
  \inferrule* [lab=agent] {} {{M_{A}} \bc (\vec{x})M_{P} \;| \; \clift{P_0,\ldots,M_{P},\ldots,P_N}}
  \and \\
  \inferrule* [lab=process] {} {{M_{P}} \bc M_{N} \;| \;P|M_{P} }
\end{mathpar} 

\begin{mathpar}
  \inferrule* [lab=sychronization] {} {M_{N} \bc \Box \;|\; x?M_{F} \;|\; x!M_{C}}
  \and
  \inferrule* [lab=abstraction] {} {{M_{F}} \bc (x)M_{P} }
  \and
  \inferrule* [lab=concretion] {} {{M_{C}} \bc \langle M_{P} \rangle }
  \and \\
  \inferrule* [lab=process] {} {{M_{P}} \bc M_{N} \;| \;P|M_{P} }
\end{mathpar}

\begin{definition}[contextual application] Given a context $M$, and
  process $P$, we define the \emph{contextual application}, $M[P] :=
  M\{P/\Box\}$. That is, the contextual application of M to P is the
  substitution of $P$ for $\Box$ in $M$.
\end{definition}

$\meaningof{-} : L \to \mathcal{P}(\pi)$

\begin{mathpar}
  \inferrule* [lab=collection] {} {\meaningof{true} = \pi, \and \meaningof{~E} = \pi \setminus \meaningof{E}, \and \meaningof{E_{1} \& E_{2}} = \meaningof{E_{1}} \cap \meaningof{E_{2}}}
\end{mathpar}

\begin{mathpar}
  \inferrule* [lab=structure] {} {\meaningof{0} = \{ P \in \pi | P \equiv 0 \}, \and \\ \meaningof{E_1 | E_2} = \{ P \in \pi | P \equiv P_{1} | P_{2}, P_{1} \in \meaningof{E_{1}}, P_{2} \in \meaningof{E_2}\} }
\end{mathpar}

\begin{mathpar}
 \inferrule* [lab=behavior] {} {\meaningof{\langle a?b \rangle E} = \{ P \in \pi | P \equiv Q | u?(y)P', \\ \and \\\\ \and \\ \;\;\; u \in \meaningof{a}, \forall z.P'\{z/y\} \in \meaningof{E\{z/b\}}\}, \and \\ \meaningof{a!E} = \{ P \in \pi | P \equiv Q | x!\langle P' \rangle, x \in \meaningof{a} P' \in \meaningof{E}\} }
\end{mathpar}

\begin{mathpar}
 \inferrule* [lab=nominal] {} {\meaningof{\quotep{E}} = \{ \quotep{P} \in \quotep{\pi} | P \in \meaningof{E} \}, \and \meaningof{\quotep{P}} = \{ \quotep{Q} \in \quotep{\pi} | P \equiv Q \} \and \\ \meaningof{@\quotep{E}} = \{ P \in \pi | P \equiv @x, x \in \meaningof{E} \}}
\end{mathpar}

\begin{eqnarray*}
  \\
  \meaningof{-} : TS \to ST
\end{eqnarray*}

\begin{eqnarray*}
  \\
  L : TS \to ST
\end{eqnarray*}

\begin{eqnarray*}
  \\
  P \models E \iff P \in \meaningof{E}
\end{eqnarray*}

\begin{eqnarray*}
  P \approx_{L} Q \iff \forall E \in L. P \models E \iff Q \models E
\end{eqnarray*}

\begin{eqnarray*}
  P \approx_{K} Q
\end{eqnarray*}

\begin{eqnarray*}
  P \approx Q
\end{eqnarray*}

$\approx_{K} = \approx = \approx_{L}$

\subsubsection{Contextual duality}

Note that contexts extend the quotation operation to a family of
operations from processes to names. Given a context, $M$, we can
define a \emph{nominal context}, $\quotep{M}$ by $\quotep{M}[P] :=
\quotep{M[P]}$. To foreshadow what is to come we observe that these
operations enjoy a duality with processes very much like the duality
between vectors and maps from vectors to scalars.

Further, because the calculus is essentially higher-order, we have a
correspondence between contexts and processes. More specifically,
given a name $x$ and a context $M$ we can construct $M^{*}_{x}$ such
that 

\begin{mathpar}
  M^{*}_{x} | \lift{x}{P} \red M[P]
\end{mathpar}

namely,

\begin{mathpar}
  M^{*}_{x} := x?(u).M[\dropn{u}]
\end{mathpar}

The dependence of $M^{*}_{x}$ on a name makes it an abstraction, 

\begin{mathpar}
  M^{*} := (x)x?(u).M[\dropn{u}]
\end{mathpar}

\subsection{Additional notation}

It will sometimes be convenient to denote the process a name
quotes. We already have the notation $x = \quotep{P}$, but it will be
convenient to introduce an alternate notation, $\procn{x}$, when we
want to emphasize the connection to the use of the name. Note that, by
virtue of name equivalence, $\quotep{\procn{x}} \nameeq x$; so, the
notation is consistent with previous definitions.

Further, because names have structure it is possible to effect
substitutions on the basis of that structure. This means we need to
upgrade our notation for substitutions, which we accomplish by
adapting comprehension notation. Thus,

\begin{mathpar}
  P\{ y / x : x \in S \}
\end{mathpar}

is interpreted to mean the process derived from P by replacing (in a
capture-avoiding manner) each occurrence of $x$ in $S$ by $y$. For example,

\begin{mathpar}
  P\{ \quotep{\procn{x}|\procn{x}} / x : x \in \freenames{P} \}
\end{mathpar}

will replace each (occurrence) of a free name $x$ in $P$ by
$\quotep{\procn{x}|\procn{x}}$.

Also, we will avail ourselves of the notation $x^{L}$ and $x^{R}$ to
denote injections of a name into disjoint copies of the name
space. There are numerous ways to accomplish this. One example can be
found in \cite{MeredithR05}. This notation overloads to vectors of
names: $\vec{x}^{\pi} := (x_{i}^{\pi} \; : \; 0 \leq i < |\vec{x}| )$ where $\pi \in \{L,R\}$.

We also use $P^{\Box} := P|\Box$.

In \cite{MeredithR05} an interpretation of the new operator is
given. It turns out that there are several possible interpretations
all enjoying the requisite algebraic properties of the operator (see
\cite{milner91polyadicpi}). We will therefore make liberal use of
$(\nu\; \vec{x})P$.

% subsection the_syntax_and_semantics_of_the_notation_system (end)   

\input{qm2pi.qmops} 

\input{qm2pi.sterngerlach} 

\input{qm2pi.metric} 

% section concurrent_process_calculi (end)

%\input{qm2pi.proofsketch}

% section proof sketch (end)

%\input{qm2pi.slviaknots} 

% section spatial logic via knots (end)

\input{qm2pi.conclusion}

% section conclusion (end)

%\input{qm2pi.dtcodes} 

% section wiring algorithm (end)

\input{qm2pi.ack} 

% section acknowledgments (end)

\newpage


\bibliographystyle{plain}   
\bibliography{../../biblios/main.bib}

\input{qm2pi.rhodetails}

\end{document}

 

% section notation (end)

\input{qm2pi.process.calculi} 

% section concurrent_process_calculi_and_spatial_logics_ (end)
    
%\documentclass[12pt]{llncs}
%\documentclass{jktr}

\usepackage[pdftex]{hyperref}                   
\usepackage {listings}
\usepackage {mathpartir}
\usepackage{bcprules}
%\usepackage{listings}
                       
\usepackage{graphicx} 
%\usepackage[margins=2.5cm,nohead,nofoot]{geometry}
%\usepackage{geometry}
\usepackage{amsfonts}
\usepackage{amstext}
\usepackage{latexsym}
\usepackage{amssymb}
\usepackage{color}


%\include{myPreamble}
\include{qm2pi.local} 

%\ifpdf
%\usepackage[pdftex]{graphicx}
%\else
%\usepackage{graphicx}
%\fi

 % \ifpdf
%  \usepackage{pdfsync}
%  \if


%\title{Brief Article}
%\author{David F. Snyder}
%\author{L.G. Meredith}

%\address{Dept. of Math., Texas State University--San Marcos, San Marcos, TX 78666}
       
\pagestyle{empty}


\begin{document}

\lstset{language=[Objective]Caml,frame=shadowbox}

\input{qm2pi.front}

% section front matter (end)

\input{qm2pi.intro} 
 
% section introduction (end)

% \input{qm2pi.knotations} 

% section notation (end)

\input{qm2pi.process.calculi} 

% section concurrent_process_calculi_and_spatial_logics_ (end)
    
%\input{qm2pi.knots2pi} 

%\input{qm2pi.trefoil} 

%\input{qm2pi.mainthm} 

% subsection basic_interpretation (end)

%\input{qm2pi.rho.presentation} 
\subsection{The syntax and semantics of the notation system}\label{sub:the_syntax_and_semantics_of_the_notation_system} % (fold)

We now summarize a technical presentation of the calculus that
embodies our theory of dynamics. The typical presentation of such a
calculus follows the style of giving generators and relations on
them. The grammar, below, describing term constructors, freely
generates the set of processes, $\Proc$. This set is then quotiented
by a relation known as structural congruence and it is over this set
that the notion of dynamics is expressed. This presentation is
essentially that of \cite{MeredithR05} with the addition of
polyadicity and summation. For readability we have relegated some of
the technical subtleties to an appendix.

\subsubsection{Process grammar}\label{subsub:process_grammar}

\begin{mathpar}
  \inferrule* [lab=synchronization] {} {{M} \bc \pzero \;|\; x?F \;|\; x!C }
  \and
  \inferrule* [lab=abstraction] {} {{F} \bc (x)P}
  \and
  \inferrule* [lab=concretion] {} {{C} \bc \langle Q \rangle}
  \and
  \inferrule* [lab=process] {} {{P,Q} \bc M \;| \;P|Q \;|\; @{x}}
  \and
  \inferrule* [lab=name] {} {{x} \bc \quotep{P}}
\end{mathpar} 

Note that $\vec{x}$ (resp. $\vec{P}$) denotes a vector of names
(resp. processes) of length $|\vec{x}|$ (resp. $|\vec{P}|$). We adopt
the following useful abbreviations.

\begin{mathpar}
   x?(\vec{y}).P := x.(\vec{y})P \and  x\clift{\vec{P}} := x.\clift{\vec{P}}
   \and x!(y) := \lift{x}{\dropn{y}}
   \and \Pi_{i=0}^{n-1}P_i := P_0 | \ldots | P_{n-1}
\end{mathpar}

\subsubsection{Structural congruence}

\paragraph{Free and bound names and alpha-equivalence.} At the
core of structural equivalence is alpha-equivalence which identifies
process that are the same up to a change of variable. Formally, we
recognize the distinction between free and bound names. The free names
of a process, $\freenames{P}$, may be calculated recursively as
follows:

\begin{mathpar}
\freenames{\pzero} := \emptyset
  \and \\
  \freenames{x?(y).P} := \{ x \} \cup (\freenames{P} \setminus \{ y \})
  \and 
  \freenames{x!\langle P \rangle} := \{ x \} \cup \{ P \} 
  \and \\
  \freenames{P|Q} := \freenames{P} \cup \freenames{Q}
  \and \\
  \freenames{@{x}} := \{ x \}
\end{mathpar}

$\pi$
$\quotep{\pi}$

$\freenames{-} : \pi \to \mathcal{P}(\quotep{\pi})$

\begin{eqnarray*}
  \freenames{\pzero} & := & \emptyset \\
  \freenames{x?(y).P} & := & \{ x \} \cup (\freenames{P} \setminus \{ y \}) \\
  \freenames{x!\langle P \rangle} & := & \{ x \} \cup \{ P \} \\
  \freenames{P|Q} & := & \freenames{P} \cup \freenames{Q} \\
  \freenames{\dropn{x}} & := & \{ x \}
\end{eqnarray*}

The bound names of a process, $\boundnames{P}$, are those names occurring in $P$
that are not free. For example, in $x?(y).0$, the name $x$ is free, while $y$ is bound.

\begin{mathpar}
  \inferrule* [lab=monoidal-laws] {} { P|Q \equiv Q|P \and P|0 \equiv P \and P|(Q|R) \equiv (P|Q)|R }
\end{mathpar}

\begin{mathpar}
  \inferrule* [lab=alpha-equivalence] {} { (x)P \equiv (y)P\{y/x\} \and y \not\in \freenames{P} }
\end{mathpar}

\begin{definition}
Then two processes, $P,Q$, are alpha-equivalent if $P = Q\{\vec{y}/\vec{x}\}$ for
some $\vec{x} \in \boundnames{Q},\vec{y} \in \boundnames{P}$, where $Q\{\vec{y}/\vec{x}\}$
denotes the capture-avoiding substitution of $\vec{y}$ for $\vec{x}$ in $Q$.
\end{definition}

\begin{definition}
  The {\em structural congruence} \cite{SangiorgiWalker} , $\equiv$,
  between processes is the least congruence containing
  alpha-equivalence, satisfying the abelian monoid laws
  (associativity, commutativity and $\pzero$ as identity) for parallel
  composition $|$ and for summation $+$.
\end{definition}

\subsection{Name equivalence}

We take name equivalence, written $\nameeq$, to be the smallest
equivalence relation generated by the following rules.

\begin{mathpar}
\inferrule*[lab=Quote-drop]
{ }
{ \quotep{@{x}} \nameeq x }

\inferrule*[lab=Struct-equiv]
{ P \scong Q }
{ \quotep{P} \nameeq \quotep{Q} }
\end{mathpar}

The astute reader will have noticed that the mutual recursion of names
and processes imposes a mutual recursion on alpha-equivalence and
structural equivalence via name-equivalence. Fortunately, all of this
works out pleasantly and we may calculate in the natural way, free of
concern. The reader interested in the details is referred to the
appendix \ref{appendix:rho_details}.

\subsection{Substitution}

We use $\Proc$ for the set of processes, $\QProc$ for the set of
names, and $\id{\{}\vec{y} / \vec{x} \id{\}}$ to denote partial maps,
$s : \QProc \rightarrow \QProc$. A map, $s$ lifts, uniquely, to a map
on process terms, $\widehat{s} : \Proc \rightarrow \Proc$ by the
following equations.

\begin{mathpar}
  (0) \psubstp{Q}{P} := 0 \\
  (R \juxtap S) \psubstp{Q}{P}
  :=    
  (R)\psubstp{Q}{P} \juxtap (S) \psubstp{Q}{P} \\
  (x?(y).R) \psubstp{Q}{P}    
  :=    
  (x)\substp{Q}{P} (z)\concat( (R \psubstn{z}{y}) \psubstp{Q}{P} ) \\
  (\lift{x}{R}) \psubstp{Q}{P}  
  :=
  \lift{(x)\substp{Q}{P}}{ R \psubstp{Q}{P} } \\
%   (\dropn{x})  \psubstp{Q}{P}       
%   := 
%   \left\{ 
%     \begin{array}{ccc} 
%       \dropn{\quotep{Q}} & & x \nameeq \quotep{P} \\
%       \dropn{x} & & otherwise \\
%     \end{array}
%   \right. 
  (\dropn{x})  \psubstp{Q}{P}       
  := 
  \left\{ 
    \begin{array}{ccc} 
      Q & & x \nameeq \quotep{P} \\
      \dropn{x} & & otherwise \\
    \end{array}
  \right.
\end{mathpar}
 

where

\begin{eqnarray}
  (x)\id{\{} \lpquote Q \rpquote / \lpquote P \rpquote \id{\}}            = 
  \left\{ 
    \begin{array}{ccc}
      \lpquote Q \rpquote & & x \nameeq \lpquote P \rpquote \\
      x & & otherwise \\
    \end{array}
  \right. \nonumber
\end{eqnarray}

and $z$ is chosen distinct from $\quotep{P}$, $\quotep{Q}$, the free
names in $Q$, and all the names in $R$. Our $\alpha$-equivalence will
be built in the standard way from this substitution.

\begin{remark}\label{rem:no_self_referential_names}
  One consequence of these definitions is that $\forall P. \quotep{P}
  \not\in \freenames{P}$.
\end{remark}

\subsection{ Dynamic quote: an example }

Anticipating something of what's to come, consider applying the
substitution, $\widehat{\id{\{}u / z \id{\}}}$, to the following pair
of processes, $\lift{w}{y!(z)}$ and $w[ \lpquote y!(z) \rpquote ]$.

\begin{eqnarray}
	\lift{w}{y!(z)}\widehat{\id{\{}u / z \id{\}}}
		& = &
		\lift{w}{y!(u)} \nonumber\\
	w[ \lpquote y!(z) \rpquote ] \widehat{ \id{\{}u / z \id{\}} }
		& = &
		w[ \lpquote y!(z) \rpquote ] \nonumber
\end{eqnarray}

Because the body of the process between quotes is impervious to
substitution, we get radically different answers. In fact, by
examining the first process in an input context,
e.g. $x?(z).\lift{w}{y!(z)}$, we see that the process under the lift
operator may be shaped by prefixed inputs binding a name inside it. In
this sense, the lift operator will be seen as a way to dynamically
construct processes before reifying them as names.

Finally equipped with these standard features we can present the
dynamics of the calculus.

\subsubsection{Operational semantics} 

Finally, we introduce the computational dynamics. What marks these
algebras as distinct from other more traditionally studied algebraic
structures, e.g. vector spaces or polynomial rings, is the manner in
which dynamics is captured. In traditional structures, dynamics is typically
expressed through morphisms between such structures, as in linear maps
between vector spaces or morphisms between rings. In algebras
associated with the semantics of computation, the dynamics is
expressed as part of the algebraic structure itself, through a
reduction reduction relation typically denoted by $\red$. Below, we
give a recursive presentation of this relation for the calculus used
in the encoding.

$\red \subseteq \pi \times \pi$
$\red : \pi \to \mathcal{P}(\pi)$

\begin{mathpar}
  \inferrule* [lab=Comm] { \textsf{match}( x_{src}, x_{trgt} ) } { x_{trgt}?(y)P \; | \; x_{src}!\langle {Q} \rangle \red P\{\quotep{Q}/y}\} }
  \and \\
  \inferrule* [lab=Par] {{P} \red {P}'} {{{P} | {Q}} \red {{P}' | {Q}}}
  \and
  \inferrule* [lab=Equiv]{{{P} \scong {P}'} \andalso {{P}' \red {Q}'} \andalso {{Q}' \scong {Q}}}{{P} \red {Q}}
\end{mathpar}

\begin{eqnarray*}
  match_{\equiv} (\quotep{P},\quotep{Q}) & := & P \equiv Q \\
  match_{\dagger}(\quotep{P},\quotep{Q}) & := & \forall R. P|Q \red^{*} R => R \red^{*} 0 \\
  match_{K}(\quotep{P},\quotep{Q}) & := & K \mbox{ for some context } K
\end{eqnarray*}

$u?(x)P | u!\langle Q \rangle \red P\{\quotep{Q}/x\}$

%We write $\wred$ for $\red^*$, and $P\red$ if $\exists Q $ such that $ P \red Q$.
We write $P\red$ if $\exists Q $ such that $ P \red Q$ and $P\not\red$, otherwise.

\section{Replication}

As mentioned before, it is known that replication (and hence
recursion) can be implemented in a higher-order process algebra
\cite{SangiorgiWalker}. As our first example of calculation with the
machinery thus far presented we give the construction explicitly in
the {\rhoc}.

\begin{eqnarray}
	D_{x} & := & \prefix{x}{y}{(\binpar{\outputp{x}{y}}{@{y}})} \nonumber\\
	\bangp_{x}{P} & := & \binpar{{x}!\langle{\binpar{D_{x}}{P}}\rangle}{D_{x}} \nonumber
\end{eqnarray}

\begin{eqnarray}
	\bangp_{x}{P} & & \nonumber\\
	=
	& {x}!\langle{(\prefix{x}{y}{(\outputp{x}{y} | @{y})) | P}}\rangle 
	      | \prefix{x}{y}{(\outputp{x}{y} | @{y})} & \nonumber\\
	\red
	& (\outputp{x}{y} | @{y})\substn{\quotep{(\prefix{x}{y}{(@{y} | \outputp{x}{y})) | P}}}{y} & \nonumber\\
	=
	& \outputp{x}{\quotep{(\prefix{x}{y}{(\outputp{x}{y} | @{y})) | P}}}
	  | {(\prefix{x}{y}{(\outputp{x}{y} | @{y})) | P}} & \nonumber\\
	\red
	& \ldots & \nonumber\\
	\red^*
	& P | P | \ldots & \nonumber
\end{eqnarray}

Of course, this encoding, as an implementation, runs away, unfolding
$\bangp{P}$ eagerly. A lazier and more implementable replication
operator, restricted to input-guarded processes, may be obtained as follows.

\begin{eqnarray}
\bangp{\prefix{u}{v}{P}} 
	:= 
	\binpar{\lift{x}{\prefix{u}{v}{(\binpar{D(x)}{P})}}}{D(x)} \nonumber
\end{eqnarray}

\begin{remark}
  Note that the lazier definition still does not deal with summation
  or mixed summation (i.e. sums over input and output). The reader is
  invited to construct definitions of replication that deal with these
  features. 

  Further, the definitions are parameterized in a name, $x$. Can you,
  gentle reader, make a definition that eliminates this parameter and
  guarantees no accidental interaction between the replication
  machinery and the process being replicated -- i.e. no accidental
  sharing of names used by the process to get its work done and the
  name(s) used by the replication to effect copying. This latter
  revision of the definition of replication is crucial to obtaining
  the expected identity $!!P \sim !P$.
\end{remark}

\begin{remark}\label{rem:paradoxical_combinator}
  The reader familiar with the lambda calculus will have noticed the
  similarity between $D$ and the paradoxical combinator.

  [Ed. note: the existence of this seems to suggest we have to be more
  restrictive on the set of processes and names we admit if we are to
  support no-cloning.]
\end{remark}

\subsubsection{Bisimulation}

The computational dynamics gives rise to another kind of equivalence,
the equivalence of computational behavior. As previously mentioned
this is typically captured \emph{via} some form of bisimulation.

% The notion we use in this paper is weak barbed bisimulation
% \cite{milner91polyadicpi}.

The notion we use in this paper is derived from weak barbed
bisimulation \cite{milner91polyadicpi}. 

\begin{definition}
An \emph{observation relation}, $\downarrow_{\mathcal N}$, over a set
of names, $\mathcal N$, is the smallest relation satisfying the rules
below.

\infrule[Out-barb]{y \in {\mathcal N}, \; x \nameeq y}
		  {\outputp{x}{v} \downarrow_{\mathcal N} x}
\infrule[Par-barb]{\mbox{$P\downarrow_{\mathcal N} x$ or $Q\downarrow_{\mathcal N} x$}}
		  {\binpar{P}{Q} \downarrow_{\mathcal N} x}

We write $P \Downarrow_{\mathcal N} x$ if there is $Q$ such that 
$P \wred Q$ and $Q \downarrow_{\mathcal N} x$.
\end{definition}

\begin{definition}
%\label{def.bbisim}
An  ${\mathcal N}$-\emph{barbed bisimulation} over a set of names, ${\mathcal N}$, is a symmetric binary relation 
${\mathcal S}_{\mathcal N}$ between agents such that $P\rel{S}_{\mathcal N}Q$ implies:
\begin{enumerate}
\item If $P \red P'$ then $Q \wred Q'$ and $P'\rel{S}_{\mathcal N} Q'$.
\item If $P\downarrow_{\mathcal N} x$, then $Q\Downarrow_{\mathcal N} x$.
\end{enumerate}
$P$ is ${\mathcal N}$-barbed bisimilar to $Q$, written
$P \wbbisim_{\mathcal N} Q$, if $P \rel{S}_{\mathcal N} Q$ for some ${\mathcal N}$-barbed bisimulation ${\mathcal S}_{\mathcal N}$.
\end{definition}

$\mathcal{R} \subseteq \pi \times \pi$

$P \mathcal{R} Q => \forall P'. P \red P' \Rightarrow \exists Q'. Q \red Q', P' \mathcal{R} Q'$

$P \vdash x \Rightarrow Q \vdash x$

\begin{mathpar}
  \inferrule*[lab=Out-barb]{x \nameeq y}{{y}!\langle{Q}\rangle \vdash x}
  \and
  \inferrule*[lab=Par-barb]{\mbox{$P\vdash x$ or $Q\vdash x$}}{\binpar{P}{Q} \vdash x}
\end{mathpar}

\subsubsection{Contexts}

One of the principle advantages of computational calculi like the
$\pi$-calculus is a well-defined notion of context,
contextual-equivalence and a correlation between
contextual-equivalence and notions of bisimulation. The notion of
context allows the decomposition of a process into (sub-)process and
its syntactic environment, its context. Thus, a context may be
thought of as a process with a ``hole'' (written $\Box$) in it. The
application of a context $M$ to a process $P$, written $M[P]$, is
tantamount to filling the hole in $M$ with $P$. In this paper we do
not need the full weight of this theory, but do make use of the notion
of context in the proof the main theorem. 

\begin{mathpar}
  \inferrule* [lab=summation] {} {{M_{M},M_{N}} \bc \Box \;|\; x.M_{A} \;|\; M_{M}+M_{N}}
  \and
  \inferrule* [lab=agent] {} {{M_{A}} \bc (\vec{x})M_{P} \;| \; \clift{P_0,\ldots,M_{P},\ldots,P_N}}
  \and \\
  \inferrule* [lab=process] {} {{M_{P}} \bc M_{N} \;| \;P|M_{P} }
\end{mathpar} 

\begin{mathpar}
  \inferrule* [lab=sychronization] {} {M_{N} \bc \Box \;|\; x?M_{F} \;|\; x!M_{C}}
  \and
  \inferrule* [lab=abstraction] {} {{M_{F}} \bc (x)M_{P} }
  \and
  \inferrule* [lab=concretion] {} {{M_{C}} \bc \langle M_{P} \rangle }
  \and \\
  \inferrule* [lab=process] {} {{M_{P}} \bc M_{N} \;| \;P|M_{P} }
\end{mathpar}

\begin{definition}[contextual application] Given a context $M$, and
  process $P$, we define the \emph{contextual application}, $M[P] :=
  M\{P/\Box\}$. That is, the contextual application of M to P is the
  substitution of $P$ for $\Box$ in $M$.
\end{definition}

$\meaningof{-} : L \to \mathcal{P}(\pi)$

\begin{mathpar}
  \inferrule* [lab=collection] {} {\meaningof{true} = \pi, \and \meaningof{~E} = \pi \setminus \meaningof{E}, \and \meaningof{E_{1} \& E_{2}} = \meaningof{E_{1}} \cap \meaningof{E_{2}}}
\end{mathpar}

\begin{mathpar}
  \inferrule* [lab=structure] {} {\meaningof{0} = \{ P \in \pi | P \equiv 0 \}, \and \\ \meaningof{E_1 | E_2} = \{ P \in \pi | P \equiv P_{1} | P_{2}, P_{1} \in \meaningof{E_{1}}, P_{2} \in \meaningof{E_2}\} }
\end{mathpar}

\begin{mathpar}
 \inferrule* [lab=behavior] {} {\meaningof{\langle a?b \rangle E} = \{ P \in \pi | P \equiv Q | u?(y)P', \\ \and \\\\ \and \\ \;\;\; u \in \meaningof{a}, \forall z.P'\{z/y\} \in \meaningof{E\{z/b\}}\}, \and \\ \meaningof{a!E} = \{ P \in \pi | P \equiv Q | x!\langle P' \rangle, x \in \meaningof{a} P' \in \meaningof{E}\} }
\end{mathpar}

\begin{mathpar}
 \inferrule* [lab=nominal] {} {\meaningof{\quotep{E}} = \{ \quotep{P} \in \quotep{\pi} | P \in \meaningof{E} \}, \and \meaningof{\quotep{P}} = \{ \quotep{Q} \in \quotep{\pi} | P \equiv Q \} \and \\ \meaningof{@\quotep{E}} = \{ P \in \pi | P \equiv @x, x \in \meaningof{E} \}}
\end{mathpar}

\begin{eqnarray*}
  \\
  \meaningof{-} : TS \to ST
\end{eqnarray*}

\begin{eqnarray*}
  \\
  L : TS \to ST
\end{eqnarray*}

\begin{eqnarray*}
  \\
  P \models E \iff P \in \meaningof{E}
\end{eqnarray*}

\begin{eqnarray*}
  P \approx_{L} Q \iff \forall E \in L. P \models E \iff Q \models E
\end{eqnarray*}

\begin{eqnarray*}
  P \approx_{K} Q
\end{eqnarray*}

\begin{eqnarray*}
  P \approx Q
\end{eqnarray*}

$\approx_{K} = \approx = \approx_{L}$

\subsubsection{Contextual duality}

Note that contexts extend the quotation operation to a family of
operations from processes to names. Given a context, $M$, we can
define a \emph{nominal context}, $\quotep{M}$ by $\quotep{M}[P] :=
\quotep{M[P]}$. To foreshadow what is to come we observe that these
operations enjoy a duality with processes very much like the duality
between vectors and maps from vectors to scalars.

Further, because the calculus is essentially higher-order, we have a
correspondence between contexts and processes. More specifically,
given a name $x$ and a context $M$ we can construct $M^{*}_{x}$ such
that 

\begin{mathpar}
  M^{*}_{x} | \lift{x}{P} \red M[P]
\end{mathpar}

namely,

\begin{mathpar}
  M^{*}_{x} := x?(u).M[\dropn{u}]
\end{mathpar}

The dependence of $M^{*}_{x}$ on a name makes it an abstraction, 

\begin{mathpar}
  M^{*} := (x)x?(u).M[\dropn{u}]
\end{mathpar}

\subsection{Additional notation}

It will sometimes be convenient to denote the process a name
quotes. We already have the notation $x = \quotep{P}$, but it will be
convenient to introduce an alternate notation, $\procn{x}$, when we
want to emphasize the connection to the use of the name. Note that, by
virtue of name equivalence, $\quotep{\procn{x}} \nameeq x$; so, the
notation is consistent with previous definitions.

Further, because names have structure it is possible to effect
substitutions on the basis of that structure. This means we need to
upgrade our notation for substitutions, which we accomplish by
adapting comprehension notation. Thus,

\begin{mathpar}
  P\{ y / x : x \in S \}
\end{mathpar}

is interpreted to mean the process derived from P by replacing (in a
capture-avoiding manner) each occurrence of $x$ in $S$ by $y$. For example,

\begin{mathpar}
  P\{ \quotep{\procn{x}|\procn{x}} / x : x \in \freenames{P} \}
\end{mathpar}

will replace each (occurrence) of a free name $x$ in $P$ by
$\quotep{\procn{x}|\procn{x}}$.

Also, we will avail ourselves of the notation $x^{L}$ and $x^{R}$ to
denote injections of a name into disjoint copies of the name
space. There are numerous ways to accomplish this. One example can be
found in \cite{MeredithR05}. This notation overloads to vectors of
names: $\vec{x}^{\pi} := (x_{i}^{\pi} \; : \; 0 \leq i < |\vec{x}| )$ where $\pi \in \{L,R\}$.

We also use $P^{\Box} := P|\Box$.

In \cite{MeredithR05} an interpretation of the new operator is
given. It turns out that there are several possible interpretations
all enjoying the requisite algebraic properties of the operator (see
\cite{milner91polyadicpi}). We will therefore make liberal use of
$(\nu\; \vec{x})P$.

% subsection the_syntax_and_semantics_of_the_notation_system (end)   

\input{qm2pi.qmops} 

\input{qm2pi.sterngerlach} 

\input{qm2pi.metric} 

% section concurrent_process_calculi (end)

%\input{qm2pi.proofsketch}

% section proof sketch (end)

%\input{qm2pi.slviaknots} 

% section spatial logic via knots (end)

\input{qm2pi.conclusion}

% section conclusion (end)

%\input{qm2pi.dtcodes} 

% section wiring algorithm (end)

\input{qm2pi.ack} 

% section acknowledgments (end)

\newpage


\bibliographystyle{plain}   
\bibliography{../../biblios/main.bib}

\input{qm2pi.rhodetails}

\end{document}

 

%\documentclass[12pt]{llncs}
%\documentclass{jktr}

\usepackage[pdftex]{hyperref}                   
\usepackage {listings}
\usepackage {mathpartir}
\usepackage{bcprules}
%\usepackage{listings}
                       
\usepackage{graphicx} 
%\usepackage[margins=2.5cm,nohead,nofoot]{geometry}
%\usepackage{geometry}
\usepackage{amsfonts}
\usepackage{amstext}
\usepackage{latexsym}
\usepackage{amssymb}
\usepackage{color}


%\include{myPreamble}
\include{qm2pi.local} 

%\ifpdf
%\usepackage[pdftex]{graphicx}
%\else
%\usepackage{graphicx}
%\fi

 % \ifpdf
%  \usepackage{pdfsync}
%  \if


%\title{Brief Article}
%\author{David F. Snyder}
%\author{L.G. Meredith}

%\address{Dept. of Math., Texas State University--San Marcos, San Marcos, TX 78666}
       
\pagestyle{empty}


\begin{document}

\lstset{language=[Objective]Caml,frame=shadowbox}

\input{qm2pi.front}

% section front matter (end)

\input{qm2pi.intro} 
 
% section introduction (end)

% \input{qm2pi.knotations} 

% section notation (end)

\input{qm2pi.process.calculi} 

% section concurrent_process_calculi_and_spatial_logics_ (end)
    
%\input{qm2pi.knots2pi} 

%\input{qm2pi.trefoil} 

%\input{qm2pi.mainthm} 

% subsection basic_interpretation (end)

%\input{qm2pi.rho.presentation} 
\subsection{The syntax and semantics of the notation system}\label{sub:the_syntax_and_semantics_of_the_notation_system} % (fold)

We now summarize a technical presentation of the calculus that
embodies our theory of dynamics. The typical presentation of such a
calculus follows the style of giving generators and relations on
them. The grammar, below, describing term constructors, freely
generates the set of processes, $\Proc$. This set is then quotiented
by a relation known as structural congruence and it is over this set
that the notion of dynamics is expressed. This presentation is
essentially that of \cite{MeredithR05} with the addition of
polyadicity and summation. For readability we have relegated some of
the technical subtleties to an appendix.

\subsubsection{Process grammar}\label{subsub:process_grammar}

\begin{mathpar}
  \inferrule* [lab=synchronization] {} {{M} \bc \pzero \;|\; x?F \;|\; x!C }
  \and
  \inferrule* [lab=abstraction] {} {{F} \bc (x)P}
  \and
  \inferrule* [lab=concretion] {} {{C} \bc \langle Q \rangle}
  \and
  \inferrule* [lab=process] {} {{P,Q} \bc M \;| \;P|Q \;|\; @{x}}
  \and
  \inferrule* [lab=name] {} {{x} \bc \quotep{P}}
\end{mathpar} 

Note that $\vec{x}$ (resp. $\vec{P}$) denotes a vector of names
(resp. processes) of length $|\vec{x}|$ (resp. $|\vec{P}|$). We adopt
the following useful abbreviations.

\begin{mathpar}
   x?(\vec{y}).P := x.(\vec{y})P \and  x\clift{\vec{P}} := x.\clift{\vec{P}}
   \and x!(y) := \lift{x}{\dropn{y}}
   \and \Pi_{i=0}^{n-1}P_i := P_0 | \ldots | P_{n-1}
\end{mathpar}

\subsubsection{Structural congruence}

\paragraph{Free and bound names and alpha-equivalence.} At the
core of structural equivalence is alpha-equivalence which identifies
process that are the same up to a change of variable. Formally, we
recognize the distinction between free and bound names. The free names
of a process, $\freenames{P}$, may be calculated recursively as
follows:

\begin{mathpar}
\freenames{\pzero} := \emptyset
  \and \\
  \freenames{x?(y).P} := \{ x \} \cup (\freenames{P} \setminus \{ y \})
  \and 
  \freenames{x!\langle P \rangle} := \{ x \} \cup \{ P \} 
  \and \\
  \freenames{P|Q} := \freenames{P} \cup \freenames{Q}
  \and \\
  \freenames{@{x}} := \{ x \}
\end{mathpar}

$\pi$
$\quotep{\pi}$

$\freenames{-} : \pi \to \mathcal{P}(\quotep{\pi})$

\begin{eqnarray*}
  \freenames{\pzero} & := & \emptyset \\
  \freenames{x?(y).P} & := & \{ x \} \cup (\freenames{P} \setminus \{ y \}) \\
  \freenames{x!\langle P \rangle} & := & \{ x \} \cup \{ P \} \\
  \freenames{P|Q} & := & \freenames{P} \cup \freenames{Q} \\
  \freenames{\dropn{x}} & := & \{ x \}
\end{eqnarray*}

The bound names of a process, $\boundnames{P}$, are those names occurring in $P$
that are not free. For example, in $x?(y).0$, the name $x$ is free, while $y$ is bound.

\begin{mathpar}
  \inferrule* [lab=monoidal-laws] {} { P|Q \equiv Q|P \and P|0 \equiv P \and P|(Q|R) \equiv (P|Q)|R }
\end{mathpar}

\begin{mathpar}
  \inferrule* [lab=alpha-equivalence] {} { (x)P \equiv (y)P\{y/x\} \and y \not\in \freenames{P} }
\end{mathpar}

\begin{definition}
Then two processes, $P,Q$, are alpha-equivalent if $P = Q\{\vec{y}/\vec{x}\}$ for
some $\vec{x} \in \boundnames{Q},\vec{y} \in \boundnames{P}$, where $Q\{\vec{y}/\vec{x}\}$
denotes the capture-avoiding substitution of $\vec{y}$ for $\vec{x}$ in $Q$.
\end{definition}

\begin{definition}
  The {\em structural congruence} \cite{SangiorgiWalker} , $\equiv$,
  between processes is the least congruence containing
  alpha-equivalence, satisfying the abelian monoid laws
  (associativity, commutativity and $\pzero$ as identity) for parallel
  composition $|$ and for summation $+$.
\end{definition}

\subsection{Name equivalence}

We take name equivalence, written $\nameeq$, to be the smallest
equivalence relation generated by the following rules.

\begin{mathpar}
\inferrule*[lab=Quote-drop]
{ }
{ \quotep{@{x}} \nameeq x }

\inferrule*[lab=Struct-equiv]
{ P \scong Q }
{ \quotep{P} \nameeq \quotep{Q} }
\end{mathpar}

The astute reader will have noticed that the mutual recursion of names
and processes imposes a mutual recursion on alpha-equivalence and
structural equivalence via name-equivalence. Fortunately, all of this
works out pleasantly and we may calculate in the natural way, free of
concern. The reader interested in the details is referred to the
appendix \ref{appendix:rho_details}.

\subsection{Substitution}

We use $\Proc$ for the set of processes, $\QProc$ for the set of
names, and $\id{\{}\vec{y} / \vec{x} \id{\}}$ to denote partial maps,
$s : \QProc \rightarrow \QProc$. A map, $s$ lifts, uniquely, to a map
on process terms, $\widehat{s} : \Proc \rightarrow \Proc$ by the
following equations.

\begin{mathpar}
  (0) \psubstp{Q}{P} := 0 \\
  (R \juxtap S) \psubstp{Q}{P}
  :=    
  (R)\psubstp{Q}{P} \juxtap (S) \psubstp{Q}{P} \\
  (x?(y).R) \psubstp{Q}{P}    
  :=    
  (x)\substp{Q}{P} (z)\concat( (R \psubstn{z}{y}) \psubstp{Q}{P} ) \\
  (\lift{x}{R}) \psubstp{Q}{P}  
  :=
  \lift{(x)\substp{Q}{P}}{ R \psubstp{Q}{P} } \\
%   (\dropn{x})  \psubstp{Q}{P}       
%   := 
%   \left\{ 
%     \begin{array}{ccc} 
%       \dropn{\quotep{Q}} & & x \nameeq \quotep{P} \\
%       \dropn{x} & & otherwise \\
%     \end{array}
%   \right. 
  (\dropn{x})  \psubstp{Q}{P}       
  := 
  \left\{ 
    \begin{array}{ccc} 
      Q & & x \nameeq \quotep{P} \\
      \dropn{x} & & otherwise \\
    \end{array}
  \right.
\end{mathpar}
 

where

\begin{eqnarray}
  (x)\id{\{} \lpquote Q \rpquote / \lpquote P \rpquote \id{\}}            = 
  \left\{ 
    \begin{array}{ccc}
      \lpquote Q \rpquote & & x \nameeq \lpquote P \rpquote \\
      x & & otherwise \\
    \end{array}
  \right. \nonumber
\end{eqnarray}

and $z$ is chosen distinct from $\quotep{P}$, $\quotep{Q}$, the free
names in $Q$, and all the names in $R$. Our $\alpha$-equivalence will
be built in the standard way from this substitution.

\begin{remark}\label{rem:no_self_referential_names}
  One consequence of these definitions is that $\forall P. \quotep{P}
  \not\in \freenames{P}$.
\end{remark}

\subsection{ Dynamic quote: an example }

Anticipating something of what's to come, consider applying the
substitution, $\widehat{\id{\{}u / z \id{\}}}$, to the following pair
of processes, $\lift{w}{y!(z)}$ and $w[ \lpquote y!(z) \rpquote ]$.

\begin{eqnarray}
	\lift{w}{y!(z)}\widehat{\id{\{}u / z \id{\}}}
		& = &
		\lift{w}{y!(u)} \nonumber\\
	w[ \lpquote y!(z) \rpquote ] \widehat{ \id{\{}u / z \id{\}} }
		& = &
		w[ \lpquote y!(z) \rpquote ] \nonumber
\end{eqnarray}

Because the body of the process between quotes is impervious to
substitution, we get radically different answers. In fact, by
examining the first process in an input context,
e.g. $x?(z).\lift{w}{y!(z)}$, we see that the process under the lift
operator may be shaped by prefixed inputs binding a name inside it. In
this sense, the lift operator will be seen as a way to dynamically
construct processes before reifying them as names.

Finally equipped with these standard features we can present the
dynamics of the calculus.

\subsubsection{Operational semantics} 

Finally, we introduce the computational dynamics. What marks these
algebras as distinct from other more traditionally studied algebraic
structures, e.g. vector spaces or polynomial rings, is the manner in
which dynamics is captured. In traditional structures, dynamics is typically
expressed through morphisms between such structures, as in linear maps
between vector spaces or morphisms between rings. In algebras
associated with the semantics of computation, the dynamics is
expressed as part of the algebraic structure itself, through a
reduction reduction relation typically denoted by $\red$. Below, we
give a recursive presentation of this relation for the calculus used
in the encoding.

$\red \subseteq \pi \times \pi$
$\red : \pi \to \mathcal{P}(\pi)$

\begin{mathpar}
  \inferrule* [lab=Comm] { \textsf{match}( x_{src}, x_{trgt} ) } { x_{trgt}?(y)P \; | \; x_{src}!\langle {Q} \rangle \red P\{\quotep{Q}/y}\} }
  \and \\
  \inferrule* [lab=Par] {{P} \red {P}'} {{{P} | {Q}} \red {{P}' | {Q}}}
  \and
  \inferrule* [lab=Equiv]{{{P} \scong {P}'} \andalso {{P}' \red {Q}'} \andalso {{Q}' \scong {Q}}}{{P} \red {Q}}
\end{mathpar}

\begin{eqnarray*}
  match_{\equiv} (\quotep{P},\quotep{Q}) & := & P \equiv Q \\
  match_{\dagger}(\quotep{P},\quotep{Q}) & := & \forall R. P|Q \red^{*} R => R \red^{*} 0 \\
  match_{K}(\quotep{P},\quotep{Q}) & := & K \mbox{ for some context } K
\end{eqnarray*}

$u?(x)P | u!\langle Q \rangle \red P\{\quotep{Q}/x\}$

%We write $\wred$ for $\red^*$, and $P\red$ if $\exists Q $ such that $ P \red Q$.
We write $P\red$ if $\exists Q $ such that $ P \red Q$ and $P\not\red$, otherwise.

\section{Replication}

As mentioned before, it is known that replication (and hence
recursion) can be implemented in a higher-order process algebra
\cite{SangiorgiWalker}. As our first example of calculation with the
machinery thus far presented we give the construction explicitly in
the {\rhoc}.

\begin{eqnarray}
	D_{x} & := & \prefix{x}{y}{(\binpar{\outputp{x}{y}}{@{y}})} \nonumber\\
	\bangp_{x}{P} & := & \binpar{{x}!\langle{\binpar{D_{x}}{P}}\rangle}{D_{x}} \nonumber
\end{eqnarray}

\begin{eqnarray}
	\bangp_{x}{P} & & \nonumber\\
	=
	& {x}!\langle{(\prefix{x}{y}{(\outputp{x}{y} | @{y})) | P}}\rangle 
	      | \prefix{x}{y}{(\outputp{x}{y} | @{y})} & \nonumber\\
	\red
	& (\outputp{x}{y} | @{y})\substn{\quotep{(\prefix{x}{y}{(@{y} | \outputp{x}{y})) | P}}}{y} & \nonumber\\
	=
	& \outputp{x}{\quotep{(\prefix{x}{y}{(\outputp{x}{y} | @{y})) | P}}}
	  | {(\prefix{x}{y}{(\outputp{x}{y} | @{y})) | P}} & \nonumber\\
	\red
	& \ldots & \nonumber\\
	\red^*
	& P | P | \ldots & \nonumber
\end{eqnarray}

Of course, this encoding, as an implementation, runs away, unfolding
$\bangp{P}$ eagerly. A lazier and more implementable replication
operator, restricted to input-guarded processes, may be obtained as follows.

\begin{eqnarray}
\bangp{\prefix{u}{v}{P}} 
	:= 
	\binpar{\lift{x}{\prefix{u}{v}{(\binpar{D(x)}{P})}}}{D(x)} \nonumber
\end{eqnarray}

\begin{remark}
  Note that the lazier definition still does not deal with summation
  or mixed summation (i.e. sums over input and output). The reader is
  invited to construct definitions of replication that deal with these
  features. 

  Further, the definitions are parameterized in a name, $x$. Can you,
  gentle reader, make a definition that eliminates this parameter and
  guarantees no accidental interaction between the replication
  machinery and the process being replicated -- i.e. no accidental
  sharing of names used by the process to get its work done and the
  name(s) used by the replication to effect copying. This latter
  revision of the definition of replication is crucial to obtaining
  the expected identity $!!P \sim !P$.
\end{remark}

\begin{remark}\label{rem:paradoxical_combinator}
  The reader familiar with the lambda calculus will have noticed the
  similarity between $D$ and the paradoxical combinator.

  [Ed. note: the existence of this seems to suggest we have to be more
  restrictive on the set of processes and names we admit if we are to
  support no-cloning.]
\end{remark}

\subsubsection{Bisimulation}

The computational dynamics gives rise to another kind of equivalence,
the equivalence of computational behavior. As previously mentioned
this is typically captured \emph{via} some form of bisimulation.

% The notion we use in this paper is weak barbed bisimulation
% \cite{milner91polyadicpi}.

The notion we use in this paper is derived from weak barbed
bisimulation \cite{milner91polyadicpi}. 

\begin{definition}
An \emph{observation relation}, $\downarrow_{\mathcal N}$, over a set
of names, $\mathcal N$, is the smallest relation satisfying the rules
below.

\infrule[Out-barb]{y \in {\mathcal N}, \; x \nameeq y}
		  {\outputp{x}{v} \downarrow_{\mathcal N} x}
\infrule[Par-barb]{\mbox{$P\downarrow_{\mathcal N} x$ or $Q\downarrow_{\mathcal N} x$}}
		  {\binpar{P}{Q} \downarrow_{\mathcal N} x}

We write $P \Downarrow_{\mathcal N} x$ if there is $Q$ such that 
$P \wred Q$ and $Q \downarrow_{\mathcal N} x$.
\end{definition}

\begin{definition}
%\label{def.bbisim}
An  ${\mathcal N}$-\emph{barbed bisimulation} over a set of names, ${\mathcal N}$, is a symmetric binary relation 
${\mathcal S}_{\mathcal N}$ between agents such that $P\rel{S}_{\mathcal N}Q$ implies:
\begin{enumerate}
\item If $P \red P'$ then $Q \wred Q'$ and $P'\rel{S}_{\mathcal N} Q'$.
\item If $P\downarrow_{\mathcal N} x$, then $Q\Downarrow_{\mathcal N} x$.
\end{enumerate}
$P$ is ${\mathcal N}$-barbed bisimilar to $Q$, written
$P \wbbisim_{\mathcal N} Q$, if $P \rel{S}_{\mathcal N} Q$ for some ${\mathcal N}$-barbed bisimulation ${\mathcal S}_{\mathcal N}$.
\end{definition}

$\mathcal{R} \subseteq \pi \times \pi$

$P \mathcal{R} Q => \forall P'. P \red P' \Rightarrow \exists Q'. Q \red Q', P' \mathcal{R} Q'$

$P \vdash x \Rightarrow Q \vdash x$

\begin{mathpar}
  \inferrule*[lab=Out-barb]{x \nameeq y}{{y}!\langle{Q}\rangle \vdash x}
  \and
  \inferrule*[lab=Par-barb]{\mbox{$P\vdash x$ or $Q\vdash x$}}{\binpar{P}{Q} \vdash x}
\end{mathpar}

\subsubsection{Contexts}

One of the principle advantages of computational calculi like the
$\pi$-calculus is a well-defined notion of context,
contextual-equivalence and a correlation between
contextual-equivalence and notions of bisimulation. The notion of
context allows the decomposition of a process into (sub-)process and
its syntactic environment, its context. Thus, a context may be
thought of as a process with a ``hole'' (written $\Box$) in it. The
application of a context $M$ to a process $P$, written $M[P]$, is
tantamount to filling the hole in $M$ with $P$. In this paper we do
not need the full weight of this theory, but do make use of the notion
of context in the proof the main theorem. 

\begin{mathpar}
  \inferrule* [lab=summation] {} {{M_{M},M_{N}} \bc \Box \;|\; x.M_{A} \;|\; M_{M}+M_{N}}
  \and
  \inferrule* [lab=agent] {} {{M_{A}} \bc (\vec{x})M_{P} \;| \; \clift{P_0,\ldots,M_{P},\ldots,P_N}}
  \and \\
  \inferrule* [lab=process] {} {{M_{P}} \bc M_{N} \;| \;P|M_{P} }
\end{mathpar} 

\begin{mathpar}
  \inferrule* [lab=sychronization] {} {M_{N} \bc \Box \;|\; x?M_{F} \;|\; x!M_{C}}
  \and
  \inferrule* [lab=abstraction] {} {{M_{F}} \bc (x)M_{P} }
  \and
  \inferrule* [lab=concretion] {} {{M_{C}} \bc \langle M_{P} \rangle }
  \and \\
  \inferrule* [lab=process] {} {{M_{P}} \bc M_{N} \;| \;P|M_{P} }
\end{mathpar}

\begin{definition}[contextual application] Given a context $M$, and
  process $P$, we define the \emph{contextual application}, $M[P] :=
  M\{P/\Box\}$. That is, the contextual application of M to P is the
  substitution of $P$ for $\Box$ in $M$.
\end{definition}

$\meaningof{-} : L \to \mathcal{P}(\pi)$

\begin{mathpar}
  \inferrule* [lab=collection] {} {\meaningof{true} = \pi, \and \meaningof{~E} = \pi \setminus \meaningof{E}, \and \meaningof{E_{1} \& E_{2}} = \meaningof{E_{1}} \cap \meaningof{E_{2}}}
\end{mathpar}

\begin{mathpar}
  \inferrule* [lab=structure] {} {\meaningof{0} = \{ P \in \pi | P \equiv 0 \}, \and \\ \meaningof{E_1 | E_2} = \{ P \in \pi | P \equiv P_{1} | P_{2}, P_{1} \in \meaningof{E_{1}}, P_{2} \in \meaningof{E_2}\} }
\end{mathpar}

\begin{mathpar}
 \inferrule* [lab=behavior] {} {\meaningof{\langle a?b \rangle E} = \{ P \in \pi | P \equiv Q | u?(y)P', \\ \and \\\\ \and \\ \;\;\; u \in \meaningof{a}, \forall z.P'\{z/y\} \in \meaningof{E\{z/b\}}\}, \and \\ \meaningof{a!E} = \{ P \in \pi | P \equiv Q | x!\langle P' \rangle, x \in \meaningof{a} P' \in \meaningof{E}\} }
\end{mathpar}

\begin{mathpar}
 \inferrule* [lab=nominal] {} {\meaningof{\quotep{E}} = \{ \quotep{P} \in \quotep{\pi} | P \in \meaningof{E} \}, \and \meaningof{\quotep{P}} = \{ \quotep{Q} \in \quotep{\pi} | P \equiv Q \} \and \\ \meaningof{@\quotep{E}} = \{ P \in \pi | P \equiv @x, x \in \meaningof{E} \}}
\end{mathpar}

\begin{eqnarray*}
  \\
  \meaningof{-} : TS \to ST
\end{eqnarray*}

\begin{eqnarray*}
  \\
  L : TS \to ST
\end{eqnarray*}

\begin{eqnarray*}
  \\
  P \models E \iff P \in \meaningof{E}
\end{eqnarray*}

\begin{eqnarray*}
  P \approx_{L} Q \iff \forall E \in L. P \models E \iff Q \models E
\end{eqnarray*}

\begin{eqnarray*}
  P \approx_{K} Q
\end{eqnarray*}

\begin{eqnarray*}
  P \approx Q
\end{eqnarray*}

$\approx_{K} = \approx = \approx_{L}$

\subsubsection{Contextual duality}

Note that contexts extend the quotation operation to a family of
operations from processes to names. Given a context, $M$, we can
define a \emph{nominal context}, $\quotep{M}$ by $\quotep{M}[P] :=
\quotep{M[P]}$. To foreshadow what is to come we observe that these
operations enjoy a duality with processes very much like the duality
between vectors and maps from vectors to scalars.

Further, because the calculus is essentially higher-order, we have a
correspondence between contexts and processes. More specifically,
given a name $x$ and a context $M$ we can construct $M^{*}_{x}$ such
that 

\begin{mathpar}
  M^{*}_{x} | \lift{x}{P} \red M[P]
\end{mathpar}

namely,

\begin{mathpar}
  M^{*}_{x} := x?(u).M[\dropn{u}]
\end{mathpar}

The dependence of $M^{*}_{x}$ on a name makes it an abstraction, 

\begin{mathpar}
  M^{*} := (x)x?(u).M[\dropn{u}]
\end{mathpar}

\subsection{Additional notation}

It will sometimes be convenient to denote the process a name
quotes. We already have the notation $x = \quotep{P}$, but it will be
convenient to introduce an alternate notation, $\procn{x}$, when we
want to emphasize the connection to the use of the name. Note that, by
virtue of name equivalence, $\quotep{\procn{x}} \nameeq x$; so, the
notation is consistent with previous definitions.

Further, because names have structure it is possible to effect
substitutions on the basis of that structure. This means we need to
upgrade our notation for substitutions, which we accomplish by
adapting comprehension notation. Thus,

\begin{mathpar}
  P\{ y / x : x \in S \}
\end{mathpar}

is interpreted to mean the process derived from P by replacing (in a
capture-avoiding manner) each occurrence of $x$ in $S$ by $y$. For example,

\begin{mathpar}
  P\{ \quotep{\procn{x}|\procn{x}} / x : x \in \freenames{P} \}
\end{mathpar}

will replace each (occurrence) of a free name $x$ in $P$ by
$\quotep{\procn{x}|\procn{x}}$.

Also, we will avail ourselves of the notation $x^{L}$ and $x^{R}$ to
denote injections of a name into disjoint copies of the name
space. There are numerous ways to accomplish this. One example can be
found in \cite{MeredithR05}. This notation overloads to vectors of
names: $\vec{x}^{\pi} := (x_{i}^{\pi} \; : \; 0 \leq i < |\vec{x}| )$ where $\pi \in \{L,R\}$.

We also use $P^{\Box} := P|\Box$.

In \cite{MeredithR05} an interpretation of the new operator is
given. It turns out that there are several possible interpretations
all enjoying the requisite algebraic properties of the operator (see
\cite{milner91polyadicpi}). We will therefore make liberal use of
$(\nu\; \vec{x})P$.

% subsection the_syntax_and_semantics_of_the_notation_system (end)   

\input{qm2pi.qmops} 

\input{qm2pi.sterngerlach} 

\input{qm2pi.metric} 

% section concurrent_process_calculi (end)

%\input{qm2pi.proofsketch}

% section proof sketch (end)

%\input{qm2pi.slviaknots} 

% section spatial logic via knots (end)

\input{qm2pi.conclusion}

% section conclusion (end)

%\input{qm2pi.dtcodes} 

% section wiring algorithm (end)

\input{qm2pi.ack} 

% section acknowledgments (end)

\newpage


\bibliographystyle{plain}   
\bibliography{../../biblios/main.bib}

\input{qm2pi.rhodetails}

\end{document}

 

%\documentclass[12pt]{llncs}
%\documentclass{jktr}

\usepackage[pdftex]{hyperref}                   
\usepackage {listings}
\usepackage {mathpartir}
\usepackage{bcprules}
%\usepackage{listings}
                       
\usepackage{graphicx} 
%\usepackage[margins=2.5cm,nohead,nofoot]{geometry}
%\usepackage{geometry}
\usepackage{amsfonts}
\usepackage{amstext}
\usepackage{latexsym}
\usepackage{amssymb}
\usepackage{color}


%\include{myPreamble}
\include{qm2pi.local} 

%\ifpdf
%\usepackage[pdftex]{graphicx}
%\else
%\usepackage{graphicx}
%\fi

 % \ifpdf
%  \usepackage{pdfsync}
%  \if


%\title{Brief Article}
%\author{David F. Snyder}
%\author{L.G. Meredith}

%\address{Dept. of Math., Texas State University--San Marcos, San Marcos, TX 78666}
       
\pagestyle{empty}


\begin{document}

\lstset{language=[Objective]Caml,frame=shadowbox}

\input{qm2pi.front}

% section front matter (end)

\input{qm2pi.intro} 
 
% section introduction (end)

% \input{qm2pi.knotations} 

% section notation (end)

\input{qm2pi.process.calculi} 

% section concurrent_process_calculi_and_spatial_logics_ (end)
    
%\input{qm2pi.knots2pi} 

%\input{qm2pi.trefoil} 

%\input{qm2pi.mainthm} 

% subsection basic_interpretation (end)

%\input{qm2pi.rho.presentation} 
\subsection{The syntax and semantics of the notation system}\label{sub:the_syntax_and_semantics_of_the_notation_system} % (fold)

We now summarize a technical presentation of the calculus that
embodies our theory of dynamics. The typical presentation of such a
calculus follows the style of giving generators and relations on
them. The grammar, below, describing term constructors, freely
generates the set of processes, $\Proc$. This set is then quotiented
by a relation known as structural congruence and it is over this set
that the notion of dynamics is expressed. This presentation is
essentially that of \cite{MeredithR05} with the addition of
polyadicity and summation. For readability we have relegated some of
the technical subtleties to an appendix.

\subsubsection{Process grammar}\label{subsub:process_grammar}

\begin{mathpar}
  \inferrule* [lab=synchronization] {} {{M} \bc \pzero \;|\; x?F \;|\; x!C }
  \and
  \inferrule* [lab=abstraction] {} {{F} \bc (x)P}
  \and
  \inferrule* [lab=concretion] {} {{C} \bc \langle Q \rangle}
  \and
  \inferrule* [lab=process] {} {{P,Q} \bc M \;| \;P|Q \;|\; @{x}}
  \and
  \inferrule* [lab=name] {} {{x} \bc \quotep{P}}
\end{mathpar} 

Note that $\vec{x}$ (resp. $\vec{P}$) denotes a vector of names
(resp. processes) of length $|\vec{x}|$ (resp. $|\vec{P}|$). We adopt
the following useful abbreviations.

\begin{mathpar}
   x?(\vec{y}).P := x.(\vec{y})P \and  x\clift{\vec{P}} := x.\clift{\vec{P}}
   \and x!(y) := \lift{x}{\dropn{y}}
   \and \Pi_{i=0}^{n-1}P_i := P_0 | \ldots | P_{n-1}
\end{mathpar}

\subsubsection{Structural congruence}

\paragraph{Free and bound names and alpha-equivalence.} At the
core of structural equivalence is alpha-equivalence which identifies
process that are the same up to a change of variable. Formally, we
recognize the distinction between free and bound names. The free names
of a process, $\freenames{P}$, may be calculated recursively as
follows:

\begin{mathpar}
\freenames{\pzero} := \emptyset
  \and \\
  \freenames{x?(y).P} := \{ x \} \cup (\freenames{P} \setminus \{ y \})
  \and 
  \freenames{x!\langle P \rangle} := \{ x \} \cup \{ P \} 
  \and \\
  \freenames{P|Q} := \freenames{P} \cup \freenames{Q}
  \and \\
  \freenames{@{x}} := \{ x \}
\end{mathpar}

$\pi$
$\quotep{\pi}$

$\freenames{-} : \pi \to \mathcal{P}(\quotep{\pi})$

\begin{eqnarray*}
  \freenames{\pzero} & := & \emptyset \\
  \freenames{x?(y).P} & := & \{ x \} \cup (\freenames{P} \setminus \{ y \}) \\
  \freenames{x!\langle P \rangle} & := & \{ x \} \cup \{ P \} \\
  \freenames{P|Q} & := & \freenames{P} \cup \freenames{Q} \\
  \freenames{\dropn{x}} & := & \{ x \}
\end{eqnarray*}

The bound names of a process, $\boundnames{P}$, are those names occurring in $P$
that are not free. For example, in $x?(y).0$, the name $x$ is free, while $y$ is bound.

\begin{mathpar}
  \inferrule* [lab=monoidal-laws] {} { P|Q \equiv Q|P \and P|0 \equiv P \and P|(Q|R) \equiv (P|Q)|R }
\end{mathpar}

\begin{mathpar}
  \inferrule* [lab=alpha-equivalence] {} { (x)P \equiv (y)P\{y/x\} \and y \not\in \freenames{P} }
\end{mathpar}

\begin{definition}
Then two processes, $P,Q$, are alpha-equivalent if $P = Q\{\vec{y}/\vec{x}\}$ for
some $\vec{x} \in \boundnames{Q},\vec{y} \in \boundnames{P}$, where $Q\{\vec{y}/\vec{x}\}$
denotes the capture-avoiding substitution of $\vec{y}$ for $\vec{x}$ in $Q$.
\end{definition}

\begin{definition}
  The {\em structural congruence} \cite{SangiorgiWalker} , $\equiv$,
  between processes is the least congruence containing
  alpha-equivalence, satisfying the abelian monoid laws
  (associativity, commutativity and $\pzero$ as identity) for parallel
  composition $|$ and for summation $+$.
\end{definition}

\subsection{Name equivalence}

We take name equivalence, written $\nameeq$, to be the smallest
equivalence relation generated by the following rules.

\begin{mathpar}
\inferrule*[lab=Quote-drop]
{ }
{ \quotep{@{x}} \nameeq x }

\inferrule*[lab=Struct-equiv]
{ P \scong Q }
{ \quotep{P} \nameeq \quotep{Q} }
\end{mathpar}

The astute reader will have noticed that the mutual recursion of names
and processes imposes a mutual recursion on alpha-equivalence and
structural equivalence via name-equivalence. Fortunately, all of this
works out pleasantly and we may calculate in the natural way, free of
concern. The reader interested in the details is referred to the
appendix \ref{appendix:rho_details}.

\subsection{Substitution}

We use $\Proc$ for the set of processes, $\QProc$ for the set of
names, and $\id{\{}\vec{y} / \vec{x} \id{\}}$ to denote partial maps,
$s : \QProc \rightarrow \QProc$. A map, $s$ lifts, uniquely, to a map
on process terms, $\widehat{s} : \Proc \rightarrow \Proc$ by the
following equations.

\begin{mathpar}
  (0) \psubstp{Q}{P} := 0 \\
  (R \juxtap S) \psubstp{Q}{P}
  :=    
  (R)\psubstp{Q}{P} \juxtap (S) \psubstp{Q}{P} \\
  (x?(y).R) \psubstp{Q}{P}    
  :=    
  (x)\substp{Q}{P} (z)\concat( (R \psubstn{z}{y}) \psubstp{Q}{P} ) \\
  (\lift{x}{R}) \psubstp{Q}{P}  
  :=
  \lift{(x)\substp{Q}{P}}{ R \psubstp{Q}{P} } \\
%   (\dropn{x})  \psubstp{Q}{P}       
%   := 
%   \left\{ 
%     \begin{array}{ccc} 
%       \dropn{\quotep{Q}} & & x \nameeq \quotep{P} \\
%       \dropn{x} & & otherwise \\
%     \end{array}
%   \right. 
  (\dropn{x})  \psubstp{Q}{P}       
  := 
  \left\{ 
    \begin{array}{ccc} 
      Q & & x \nameeq \quotep{P} \\
      \dropn{x} & & otherwise \\
    \end{array}
  \right.
\end{mathpar}
 

where

\begin{eqnarray}
  (x)\id{\{} \lpquote Q \rpquote / \lpquote P \rpquote \id{\}}            = 
  \left\{ 
    \begin{array}{ccc}
      \lpquote Q \rpquote & & x \nameeq \lpquote P \rpquote \\
      x & & otherwise \\
    \end{array}
  \right. \nonumber
\end{eqnarray}

and $z$ is chosen distinct from $\quotep{P}$, $\quotep{Q}$, the free
names in $Q$, and all the names in $R$. Our $\alpha$-equivalence will
be built in the standard way from this substitution.

\begin{remark}\label{rem:no_self_referential_names}
  One consequence of these definitions is that $\forall P. \quotep{P}
  \not\in \freenames{P}$.
\end{remark}

\subsection{ Dynamic quote: an example }

Anticipating something of what's to come, consider applying the
substitution, $\widehat{\id{\{}u / z \id{\}}}$, to the following pair
of processes, $\lift{w}{y!(z)}$ and $w[ \lpquote y!(z) \rpquote ]$.

\begin{eqnarray}
	\lift{w}{y!(z)}\widehat{\id{\{}u / z \id{\}}}
		& = &
		\lift{w}{y!(u)} \nonumber\\
	w[ \lpquote y!(z) \rpquote ] \widehat{ \id{\{}u / z \id{\}} }
		& = &
		w[ \lpquote y!(z) \rpquote ] \nonumber
\end{eqnarray}

Because the body of the process between quotes is impervious to
substitution, we get radically different answers. In fact, by
examining the first process in an input context,
e.g. $x?(z).\lift{w}{y!(z)}$, we see that the process under the lift
operator may be shaped by prefixed inputs binding a name inside it. In
this sense, the lift operator will be seen as a way to dynamically
construct processes before reifying them as names.

Finally equipped with these standard features we can present the
dynamics of the calculus.

\subsubsection{Operational semantics} 

Finally, we introduce the computational dynamics. What marks these
algebras as distinct from other more traditionally studied algebraic
structures, e.g. vector spaces or polynomial rings, is the manner in
which dynamics is captured. In traditional structures, dynamics is typically
expressed through morphisms between such structures, as in linear maps
between vector spaces or morphisms between rings. In algebras
associated with the semantics of computation, the dynamics is
expressed as part of the algebraic structure itself, through a
reduction reduction relation typically denoted by $\red$. Below, we
give a recursive presentation of this relation for the calculus used
in the encoding.

$\red \subseteq \pi \times \pi$
$\red : \pi \to \mathcal{P}(\pi)$

\begin{mathpar}
  \inferrule* [lab=Comm] { \textsf{match}( x_{src}, x_{trgt} ) } { x_{trgt}?(y)P \; | \; x_{src}!\langle {Q} \rangle \red P\{\quotep{Q}/y}\} }
  \and \\
  \inferrule* [lab=Par] {{P} \red {P}'} {{{P} | {Q}} \red {{P}' | {Q}}}
  \and
  \inferrule* [lab=Equiv]{{{P} \scong {P}'} \andalso {{P}' \red {Q}'} \andalso {{Q}' \scong {Q}}}{{P} \red {Q}}
\end{mathpar}

\begin{eqnarray*}
  match_{\equiv} (\quotep{P},\quotep{Q}) & := & P \equiv Q \\
  match_{\dagger}(\quotep{P},\quotep{Q}) & := & \forall R. P|Q \red^{*} R => R \red^{*} 0 \\
  match_{K}(\quotep{P},\quotep{Q}) & := & K \mbox{ for some context } K
\end{eqnarray*}

$u?(x)P | u!\langle Q \rangle \red P\{\quotep{Q}/x\}$

%We write $\wred$ for $\red^*$, and $P\red$ if $\exists Q $ such that $ P \red Q$.
We write $P\red$ if $\exists Q $ such that $ P \red Q$ and $P\not\red$, otherwise.

\section{Replication}

As mentioned before, it is known that replication (and hence
recursion) can be implemented in a higher-order process algebra
\cite{SangiorgiWalker}. As our first example of calculation with the
machinery thus far presented we give the construction explicitly in
the {\rhoc}.

\begin{eqnarray}
	D_{x} & := & \prefix{x}{y}{(\binpar{\outputp{x}{y}}{@{y}})} \nonumber\\
	\bangp_{x}{P} & := & \binpar{{x}!\langle{\binpar{D_{x}}{P}}\rangle}{D_{x}} \nonumber
\end{eqnarray}

\begin{eqnarray}
	\bangp_{x}{P} & & \nonumber\\
	=
	& {x}!\langle{(\prefix{x}{y}{(\outputp{x}{y} | @{y})) | P}}\rangle 
	      | \prefix{x}{y}{(\outputp{x}{y} | @{y})} & \nonumber\\
	\red
	& (\outputp{x}{y} | @{y})\substn{\quotep{(\prefix{x}{y}{(@{y} | \outputp{x}{y})) | P}}}{y} & \nonumber\\
	=
	& \outputp{x}{\quotep{(\prefix{x}{y}{(\outputp{x}{y} | @{y})) | P}}}
	  | {(\prefix{x}{y}{(\outputp{x}{y} | @{y})) | P}} & \nonumber\\
	\red
	& \ldots & \nonumber\\
	\red^*
	& P | P | \ldots & \nonumber
\end{eqnarray}

Of course, this encoding, as an implementation, runs away, unfolding
$\bangp{P}$ eagerly. A lazier and more implementable replication
operator, restricted to input-guarded processes, may be obtained as follows.

\begin{eqnarray}
\bangp{\prefix{u}{v}{P}} 
	:= 
	\binpar{\lift{x}{\prefix{u}{v}{(\binpar{D(x)}{P})}}}{D(x)} \nonumber
\end{eqnarray}

\begin{remark}
  Note that the lazier definition still does not deal with summation
  or mixed summation (i.e. sums over input and output). The reader is
  invited to construct definitions of replication that deal with these
  features. 

  Further, the definitions are parameterized in a name, $x$. Can you,
  gentle reader, make a definition that eliminates this parameter and
  guarantees no accidental interaction between the replication
  machinery and the process being replicated -- i.e. no accidental
  sharing of names used by the process to get its work done and the
  name(s) used by the replication to effect copying. This latter
  revision of the definition of replication is crucial to obtaining
  the expected identity $!!P \sim !P$.
\end{remark}

\begin{remark}\label{rem:paradoxical_combinator}
  The reader familiar with the lambda calculus will have noticed the
  similarity between $D$ and the paradoxical combinator.

  [Ed. note: the existence of this seems to suggest we have to be more
  restrictive on the set of processes and names we admit if we are to
  support no-cloning.]
\end{remark}

\subsubsection{Bisimulation}

The computational dynamics gives rise to another kind of equivalence,
the equivalence of computational behavior. As previously mentioned
this is typically captured \emph{via} some form of bisimulation.

% The notion we use in this paper is weak barbed bisimulation
% \cite{milner91polyadicpi}.

The notion we use in this paper is derived from weak barbed
bisimulation \cite{milner91polyadicpi}. 

\begin{definition}
An \emph{observation relation}, $\downarrow_{\mathcal N}$, over a set
of names, $\mathcal N$, is the smallest relation satisfying the rules
below.

\infrule[Out-barb]{y \in {\mathcal N}, \; x \nameeq y}
		  {\outputp{x}{v} \downarrow_{\mathcal N} x}
\infrule[Par-barb]{\mbox{$P\downarrow_{\mathcal N} x$ or $Q\downarrow_{\mathcal N} x$}}
		  {\binpar{P}{Q} \downarrow_{\mathcal N} x}

We write $P \Downarrow_{\mathcal N} x$ if there is $Q$ such that 
$P \wred Q$ and $Q \downarrow_{\mathcal N} x$.
\end{definition}

\begin{definition}
%\label{def.bbisim}
An  ${\mathcal N}$-\emph{barbed bisimulation} over a set of names, ${\mathcal N}$, is a symmetric binary relation 
${\mathcal S}_{\mathcal N}$ between agents such that $P\rel{S}_{\mathcal N}Q$ implies:
\begin{enumerate}
\item If $P \red P'$ then $Q \wred Q'$ and $P'\rel{S}_{\mathcal N} Q'$.
\item If $P\downarrow_{\mathcal N} x$, then $Q\Downarrow_{\mathcal N} x$.
\end{enumerate}
$P$ is ${\mathcal N}$-barbed bisimilar to $Q$, written
$P \wbbisim_{\mathcal N} Q$, if $P \rel{S}_{\mathcal N} Q$ for some ${\mathcal N}$-barbed bisimulation ${\mathcal S}_{\mathcal N}$.
\end{definition}

$\mathcal{R} \subseteq \pi \times \pi$

$P \mathcal{R} Q => \forall P'. P \red P' \Rightarrow \exists Q'. Q \red Q', P' \mathcal{R} Q'$

$P \vdash x \Rightarrow Q \vdash x$

\begin{mathpar}
  \inferrule*[lab=Out-barb]{x \nameeq y}{{y}!\langle{Q}\rangle \vdash x}
  \and
  \inferrule*[lab=Par-barb]{\mbox{$P\vdash x$ or $Q\vdash x$}}{\binpar{P}{Q} \vdash x}
\end{mathpar}

\subsubsection{Contexts}

One of the principle advantages of computational calculi like the
$\pi$-calculus is a well-defined notion of context,
contextual-equivalence and a correlation between
contextual-equivalence and notions of bisimulation. The notion of
context allows the decomposition of a process into (sub-)process and
its syntactic environment, its context. Thus, a context may be
thought of as a process with a ``hole'' (written $\Box$) in it. The
application of a context $M$ to a process $P$, written $M[P]$, is
tantamount to filling the hole in $M$ with $P$. In this paper we do
not need the full weight of this theory, but do make use of the notion
of context in the proof the main theorem. 

\begin{mathpar}
  \inferrule* [lab=summation] {} {{M_{M},M_{N}} \bc \Box \;|\; x.M_{A} \;|\; M_{M}+M_{N}}
  \and
  \inferrule* [lab=agent] {} {{M_{A}} \bc (\vec{x})M_{P} \;| \; \clift{P_0,\ldots,M_{P},\ldots,P_N}}
  \and \\
  \inferrule* [lab=process] {} {{M_{P}} \bc M_{N} \;| \;P|M_{P} }
\end{mathpar} 

\begin{mathpar}
  \inferrule* [lab=sychronization] {} {M_{N} \bc \Box \;|\; x?M_{F} \;|\; x!M_{C}}
  \and
  \inferrule* [lab=abstraction] {} {{M_{F}} \bc (x)M_{P} }
  \and
  \inferrule* [lab=concretion] {} {{M_{C}} \bc \langle M_{P} \rangle }
  \and \\
  \inferrule* [lab=process] {} {{M_{P}} \bc M_{N} \;| \;P|M_{P} }
\end{mathpar}

\begin{definition}[contextual application] Given a context $M$, and
  process $P$, we define the \emph{contextual application}, $M[P] :=
  M\{P/\Box\}$. That is, the contextual application of M to P is the
  substitution of $P$ for $\Box$ in $M$.
\end{definition}

$\meaningof{-} : L \to \mathcal{P}(\pi)$

\begin{mathpar}
  \inferrule* [lab=collection] {} {\meaningof{true} = \pi, \and \meaningof{~E} = \pi \setminus \meaningof{E}, \and \meaningof{E_{1} \& E_{2}} = \meaningof{E_{1}} \cap \meaningof{E_{2}}}
\end{mathpar}

\begin{mathpar}
  \inferrule* [lab=structure] {} {\meaningof{0} = \{ P \in \pi | P \equiv 0 \}, \and \\ \meaningof{E_1 | E_2} = \{ P \in \pi | P \equiv P_{1} | P_{2}, P_{1} \in \meaningof{E_{1}}, P_{2} \in \meaningof{E_2}\} }
\end{mathpar}

\begin{mathpar}
 \inferrule* [lab=behavior] {} {\meaningof{\langle a?b \rangle E} = \{ P \in \pi | P \equiv Q | u?(y)P', \\ \and \\\\ \and \\ \;\;\; u \in \meaningof{a}, \forall z.P'\{z/y\} \in \meaningof{E\{z/b\}}\}, \and \\ \meaningof{a!E} = \{ P \in \pi | P \equiv Q | x!\langle P' \rangle, x \in \meaningof{a} P' \in \meaningof{E}\} }
\end{mathpar}

\begin{mathpar}
 \inferrule* [lab=nominal] {} {\meaningof{\quotep{E}} = \{ \quotep{P} \in \quotep{\pi} | P \in \meaningof{E} \}, \and \meaningof{\quotep{P}} = \{ \quotep{Q} \in \quotep{\pi} | P \equiv Q \} \and \\ \meaningof{@\quotep{E}} = \{ P \in \pi | P \equiv @x, x \in \meaningof{E} \}}
\end{mathpar}

\begin{eqnarray*}
  \\
  \meaningof{-} : TS \to ST
\end{eqnarray*}

\begin{eqnarray*}
  \\
  L : TS \to ST
\end{eqnarray*}

\begin{eqnarray*}
  \\
  P \models E \iff P \in \meaningof{E}
\end{eqnarray*}

\begin{eqnarray*}
  P \approx_{L} Q \iff \forall E \in L. P \models E \iff Q \models E
\end{eqnarray*}

\begin{eqnarray*}
  P \approx_{K} Q
\end{eqnarray*}

\begin{eqnarray*}
  P \approx Q
\end{eqnarray*}

$\approx_{K} = \approx = \approx_{L}$

\subsubsection{Contextual duality}

Note that contexts extend the quotation operation to a family of
operations from processes to names. Given a context, $M$, we can
define a \emph{nominal context}, $\quotep{M}$ by $\quotep{M}[P] :=
\quotep{M[P]}$. To foreshadow what is to come we observe that these
operations enjoy a duality with processes very much like the duality
between vectors and maps from vectors to scalars.

Further, because the calculus is essentially higher-order, we have a
correspondence between contexts and processes. More specifically,
given a name $x$ and a context $M$ we can construct $M^{*}_{x}$ such
that 

\begin{mathpar}
  M^{*}_{x} | \lift{x}{P} \red M[P]
\end{mathpar}

namely,

\begin{mathpar}
  M^{*}_{x} := x?(u).M[\dropn{u}]
\end{mathpar}

The dependence of $M^{*}_{x}$ on a name makes it an abstraction, 

\begin{mathpar}
  M^{*} := (x)x?(u).M[\dropn{u}]
\end{mathpar}

\subsection{Additional notation}

It will sometimes be convenient to denote the process a name
quotes. We already have the notation $x = \quotep{P}$, but it will be
convenient to introduce an alternate notation, $\procn{x}$, when we
want to emphasize the connection to the use of the name. Note that, by
virtue of name equivalence, $\quotep{\procn{x}} \nameeq x$; so, the
notation is consistent with previous definitions.

Further, because names have structure it is possible to effect
substitutions on the basis of that structure. This means we need to
upgrade our notation for substitutions, which we accomplish by
adapting comprehension notation. Thus,

\begin{mathpar}
  P\{ y / x : x \in S \}
\end{mathpar}

is interpreted to mean the process derived from P by replacing (in a
capture-avoiding manner) each occurrence of $x$ in $S$ by $y$. For example,

\begin{mathpar}
  P\{ \quotep{\procn{x}|\procn{x}} / x : x \in \freenames{P} \}
\end{mathpar}

will replace each (occurrence) of a free name $x$ in $P$ by
$\quotep{\procn{x}|\procn{x}}$.

Also, we will avail ourselves of the notation $x^{L}$ and $x^{R}$ to
denote injections of a name into disjoint copies of the name
space. There are numerous ways to accomplish this. One example can be
found in \cite{MeredithR05}. This notation overloads to vectors of
names: $\vec{x}^{\pi} := (x_{i}^{\pi} \; : \; 0 \leq i < |\vec{x}| )$ where $\pi \in \{L,R\}$.

We also use $P^{\Box} := P|\Box$.

In \cite{MeredithR05} an interpretation of the new operator is
given. It turns out that there are several possible interpretations
all enjoying the requisite algebraic properties of the operator (see
\cite{milner91polyadicpi}). We will therefore make liberal use of
$(\nu\; \vec{x})P$.

% subsection the_syntax_and_semantics_of_the_notation_system (end)   

\input{qm2pi.qmops} 

\input{qm2pi.sterngerlach} 

\input{qm2pi.metric} 

% section concurrent_process_calculi (end)

%\input{qm2pi.proofsketch}

% section proof sketch (end)

%\input{qm2pi.slviaknots} 

% section spatial logic via knots (end)

\input{qm2pi.conclusion}

% section conclusion (end)

%\input{qm2pi.dtcodes} 

% section wiring algorithm (end)

\input{qm2pi.ack} 

% section acknowledgments (end)

\newpage


\bibliographystyle{plain}   
\bibliography{../../biblios/main.bib}

\input{qm2pi.rhodetails}

\end{document}

 

% subsection basic_interpretation (end)

%\input{qm2pi.rho.presentation} 
\subsection{The syntax and semantics of the notation system}\label{sub:the_syntax_and_semantics_of_the_notation_system} % (fold)

We now summarize a technical presentation of the calculus that
embodies our theory of dynamics. The typical presentation of such a
calculus follows the style of giving generators and relations on
them. The grammar, below, describing term constructors, freely
generates the set of processes, $\Proc$. This set is then quotiented
by a relation known as structural congruence and it is over this set
that the notion of dynamics is expressed. This presentation is
essentially that of \cite{MeredithR05} with the addition of
polyadicity and summation. For readability we have relegated some of
the technical subtleties to an appendix.

\subsubsection{Process grammar}\label{subsub:process_grammar}

\begin{mathpar}
  \inferrule* [lab=synchronization] {} {{M} \bc \pzero \;|\; x?F \;|\; x!C }
  \and
  \inferrule* [lab=abstraction] {} {{F} \bc (x)P}
  \and
  \inferrule* [lab=concretion] {} {{C} \bc \langle Q \rangle}
  \and
  \inferrule* [lab=process] {} {{P,Q} \bc M \;| \;P|Q \;|\; @{x}}
  \and
  \inferrule* [lab=name] {} {{x} \bc \quotep{P}}
\end{mathpar} 

Note that $\vec{x}$ (resp. $\vec{P}$) denotes a vector of names
(resp. processes) of length $|\vec{x}|$ (resp. $|\vec{P}|$). We adopt
the following useful abbreviations.

\begin{mathpar}
   x?(\vec{y}).P := x.(\vec{y})P \and  x\clift{\vec{P}} := x.\clift{\vec{P}}
   \and x!(y) := \lift{x}{\dropn{y}}
   \and \Pi_{i=0}^{n-1}P_i := P_0 | \ldots | P_{n-1}
\end{mathpar}

\subsubsection{Structural congruence}

\paragraph{Free and bound names and alpha-equivalence.} At the
core of structural equivalence is alpha-equivalence which identifies
process that are the same up to a change of variable. Formally, we
recognize the distinction between free and bound names. The free names
of a process, $\freenames{P}$, may be calculated recursively as
follows:

\begin{mathpar}
\freenames{\pzero} := \emptyset
  \and \\
  \freenames{x?(y).P} := \{ x \} \cup (\freenames{P} \setminus \{ y \})
  \and 
  \freenames{x!\langle P \rangle} := \{ x \} \cup \{ P \} 
  \and \\
  \freenames{P|Q} := \freenames{P} \cup \freenames{Q}
  \and \\
  \freenames{@{x}} := \{ x \}
\end{mathpar}

$\pi$
$\quotep{\pi}$

$\freenames{-} : \pi \to \mathcal{P}(\quotep{\pi})$

\begin{eqnarray*}
  \freenames{\pzero} & := & \emptyset \\
  \freenames{x?(y).P} & := & \{ x \} \cup (\freenames{P} \setminus \{ y \}) \\
  \freenames{x!\langle P \rangle} & := & \{ x \} \cup \{ P \} \\
  \freenames{P|Q} & := & \freenames{P} \cup \freenames{Q} \\
  \freenames{\dropn{x}} & := & \{ x \}
\end{eqnarray*}

The bound names of a process, $\boundnames{P}$, are those names occurring in $P$
that are not free. For example, in $x?(y).0$, the name $x$ is free, while $y$ is bound.

\begin{mathpar}
  \inferrule* [lab=monoidal-laws] {} { P|Q \equiv Q|P \and P|0 \equiv P \and P|(Q|R) \equiv (P|Q)|R }
\end{mathpar}

\begin{mathpar}
  \inferrule* [lab=alpha-equivalence] {} { (x)P \equiv (y)P\{y/x\} \and y \not\in \freenames{P} }
\end{mathpar}

\begin{definition}
Then two processes, $P,Q$, are alpha-equivalent if $P = Q\{\vec{y}/\vec{x}\}$ for
some $\vec{x} \in \boundnames{Q},\vec{y} \in \boundnames{P}$, where $Q\{\vec{y}/\vec{x}\}$
denotes the capture-avoiding substitution of $\vec{y}$ for $\vec{x}$ in $Q$.
\end{definition}

\begin{definition}
  The {\em structural congruence} \cite{SangiorgiWalker} , $\equiv$,
  between processes is the least congruence containing
  alpha-equivalence, satisfying the abelian monoid laws
  (associativity, commutativity and $\pzero$ as identity) for parallel
  composition $|$ and for summation $+$.
\end{definition}

\subsection{Name equivalence}

We take name equivalence, written $\nameeq$, to be the smallest
equivalence relation generated by the following rules.

\begin{mathpar}
\inferrule*[lab=Quote-drop]
{ }
{ \quotep{@{x}} \nameeq x }

\inferrule*[lab=Struct-equiv]
{ P \scong Q }
{ \quotep{P} \nameeq \quotep{Q} }
\end{mathpar}

The astute reader will have noticed that the mutual recursion of names
and processes imposes a mutual recursion on alpha-equivalence and
structural equivalence via name-equivalence. Fortunately, all of this
works out pleasantly and we may calculate in the natural way, free of
concern. The reader interested in the details is referred to the
appendix \ref{appendix:rho_details}.

\subsection{Substitution}

We use $\Proc$ for the set of processes, $\QProc$ for the set of
names, and $\id{\{}\vec{y} / \vec{x} \id{\}}$ to denote partial maps,
$s : \QProc \rightarrow \QProc$. A map, $s$ lifts, uniquely, to a map
on process terms, $\widehat{s} : \Proc \rightarrow \Proc$ by the
following equations.

\begin{mathpar}
  (0) \psubstp{Q}{P} := 0 \\
  (R \juxtap S) \psubstp{Q}{P}
  :=    
  (R)\psubstp{Q}{P} \juxtap (S) \psubstp{Q}{P} \\
  (x?(y).R) \psubstp{Q}{P}    
  :=    
  (x)\substp{Q}{P} (z)\concat( (R \psubstn{z}{y}) \psubstp{Q}{P} ) \\
  (\lift{x}{R}) \psubstp{Q}{P}  
  :=
  \lift{(x)\substp{Q}{P}}{ R \psubstp{Q}{P} } \\
%   (\dropn{x})  \psubstp{Q}{P}       
%   := 
%   \left\{ 
%     \begin{array}{ccc} 
%       \dropn{\quotep{Q}} & & x \nameeq \quotep{P} \\
%       \dropn{x} & & otherwise \\
%     \end{array}
%   \right. 
  (\dropn{x})  \psubstp{Q}{P}       
  := 
  \left\{ 
    \begin{array}{ccc} 
      Q & & x \nameeq \quotep{P} \\
      \dropn{x} & & otherwise \\
    \end{array}
  \right.
\end{mathpar}
 

where

\begin{eqnarray}
  (x)\id{\{} \lpquote Q \rpquote / \lpquote P \rpquote \id{\}}            = 
  \left\{ 
    \begin{array}{ccc}
      \lpquote Q \rpquote & & x \nameeq \lpquote P \rpquote \\
      x & & otherwise \\
    \end{array}
  \right. \nonumber
\end{eqnarray}

and $z$ is chosen distinct from $\quotep{P}$, $\quotep{Q}$, the free
names in $Q$, and all the names in $R$. Our $\alpha$-equivalence will
be built in the standard way from this substitution.

\begin{remark}\label{rem:no_self_referential_names}
  One consequence of these definitions is that $\forall P. \quotep{P}
  \not\in \freenames{P}$.
\end{remark}

\subsection{ Dynamic quote: an example }

Anticipating something of what's to come, consider applying the
substitution, $\widehat{\id{\{}u / z \id{\}}}$, to the following pair
of processes, $\lift{w}{y!(z)}$ and $w[ \lpquote y!(z) \rpquote ]$.

\begin{eqnarray}
	\lift{w}{y!(z)}\widehat{\id{\{}u / z \id{\}}}
		& = &
		\lift{w}{y!(u)} \nonumber\\
	w[ \lpquote y!(z) \rpquote ] \widehat{ \id{\{}u / z \id{\}} }
		& = &
		w[ \lpquote y!(z) \rpquote ] \nonumber
\end{eqnarray}

Because the body of the process between quotes is impervious to
substitution, we get radically different answers. In fact, by
examining the first process in an input context,
e.g. $x?(z).\lift{w}{y!(z)}$, we see that the process under the lift
operator may be shaped by prefixed inputs binding a name inside it. In
this sense, the lift operator will be seen as a way to dynamically
construct processes before reifying them as names.

Finally equipped with these standard features we can present the
dynamics of the calculus.

\subsubsection{Operational semantics} 

Finally, we introduce the computational dynamics. What marks these
algebras as distinct from other more traditionally studied algebraic
structures, e.g. vector spaces or polynomial rings, is the manner in
which dynamics is captured. In traditional structures, dynamics is typically
expressed through morphisms between such structures, as in linear maps
between vector spaces or morphisms between rings. In algebras
associated with the semantics of computation, the dynamics is
expressed as part of the algebraic structure itself, through a
reduction reduction relation typically denoted by $\red$. Below, we
give a recursive presentation of this relation for the calculus used
in the encoding.

$\red \subseteq \pi \times \pi$
$\red : \pi \to \mathcal{P}(\pi)$

\begin{mathpar}
  \inferrule* [lab=Comm] { \textsf{match}( x_{src}, x_{trgt} ) } { x_{trgt}?(y)P \; | \; x_{src}!\langle {Q} \rangle \red P\{\quotep{Q}/y}\} }
  \and \\
  \inferrule* [lab=Par] {{P} \red {P}'} {{{P} | {Q}} \red {{P}' | {Q}}}
  \and
  \inferrule* [lab=Equiv]{{{P} \scong {P}'} \andalso {{P}' \red {Q}'} \andalso {{Q}' \scong {Q}}}{{P} \red {Q}}
\end{mathpar}

\begin{eqnarray*}
  match_{\equiv} (\quotep{P},\quotep{Q}) & := & P \equiv Q \\
  match_{\dagger}(\quotep{P},\quotep{Q}) & := & \forall R. P|Q \red^{*} R => R \red^{*} 0 \\
  match_{K}(\quotep{P},\quotep{Q}) & := & K \mbox{ for some context } K
\end{eqnarray*}

$u?(x)P | u!\langle Q \rangle \red P\{\quotep{Q}/x\}$

%We write $\wred$ for $\red^*$, and $P\red$ if $\exists Q $ such that $ P \red Q$.
We write $P\red$ if $\exists Q $ such that $ P \red Q$ and $P\not\red$, otherwise.

\section{Replication}

As mentioned before, it is known that replication (and hence
recursion) can be implemented in a higher-order process algebra
\cite{SangiorgiWalker}. As our first example of calculation with the
machinery thus far presented we give the construction explicitly in
the {\rhoc}.

\begin{eqnarray}
	D_{x} & := & \prefix{x}{y}{(\binpar{\outputp{x}{y}}{@{y}})} \nonumber\\
	\bangp_{x}{P} & := & \binpar{{x}!\langle{\binpar{D_{x}}{P}}\rangle}{D_{x}} \nonumber
\end{eqnarray}

\begin{eqnarray}
	\bangp_{x}{P} & & \nonumber\\
	=
	& {x}!\langle{(\prefix{x}{y}{(\outputp{x}{y} | @{y})) | P}}\rangle 
	      | \prefix{x}{y}{(\outputp{x}{y} | @{y})} & \nonumber\\
	\red
	& (\outputp{x}{y} | @{y})\substn{\quotep{(\prefix{x}{y}{(@{y} | \outputp{x}{y})) | P}}}{y} & \nonumber\\
	=
	& \outputp{x}{\quotep{(\prefix{x}{y}{(\outputp{x}{y} | @{y})) | P}}}
	  | {(\prefix{x}{y}{(\outputp{x}{y} | @{y})) | P}} & \nonumber\\
	\red
	& \ldots & \nonumber\\
	\red^*
	& P | P | \ldots & \nonumber
\end{eqnarray}

Of course, this encoding, as an implementation, runs away, unfolding
$\bangp{P}$ eagerly. A lazier and more implementable replication
operator, restricted to input-guarded processes, may be obtained as follows.

\begin{eqnarray}
\bangp{\prefix{u}{v}{P}} 
	:= 
	\binpar{\lift{x}{\prefix{u}{v}{(\binpar{D(x)}{P})}}}{D(x)} \nonumber
\end{eqnarray}

\begin{remark}
  Note that the lazier definition still does not deal with summation
  or mixed summation (i.e. sums over input and output). The reader is
  invited to construct definitions of replication that deal with these
  features. 

  Further, the definitions are parameterized in a name, $x$. Can you,
  gentle reader, make a definition that eliminates this parameter and
  guarantees no accidental interaction between the replication
  machinery and the process being replicated -- i.e. no accidental
  sharing of names used by the process to get its work done and the
  name(s) used by the replication to effect copying. This latter
  revision of the definition of replication is crucial to obtaining
  the expected identity $!!P \sim !P$.
\end{remark}

\begin{remark}\label{rem:paradoxical_combinator}
  The reader familiar with the lambda calculus will have noticed the
  similarity between $D$ and the paradoxical combinator.

  [Ed. note: the existence of this seems to suggest we have to be more
  restrictive on the set of processes and names we admit if we are to
  support no-cloning.]
\end{remark}

\subsubsection{Bisimulation}

The computational dynamics gives rise to another kind of equivalence,
the equivalence of computational behavior. As previously mentioned
this is typically captured \emph{via} some form of bisimulation.

% The notion we use in this paper is weak barbed bisimulation
% \cite{milner91polyadicpi}.

The notion we use in this paper is derived from weak barbed
bisimulation \cite{milner91polyadicpi}. 

\begin{definition}
An \emph{observation relation}, $\downarrow_{\mathcal N}$, over a set
of names, $\mathcal N$, is the smallest relation satisfying the rules
below.

\infrule[Out-barb]{y \in {\mathcal N}, \; x \nameeq y}
		  {\outputp{x}{v} \downarrow_{\mathcal N} x}
\infrule[Par-barb]{\mbox{$P\downarrow_{\mathcal N} x$ or $Q\downarrow_{\mathcal N} x$}}
		  {\binpar{P}{Q} \downarrow_{\mathcal N} x}

We write $P \Downarrow_{\mathcal N} x$ if there is $Q$ such that 
$P \wred Q$ and $Q \downarrow_{\mathcal N} x$.
\end{definition}

\begin{definition}
%\label{def.bbisim}
An  ${\mathcal N}$-\emph{barbed bisimulation} over a set of names, ${\mathcal N}$, is a symmetric binary relation 
${\mathcal S}_{\mathcal N}$ between agents such that $P\rel{S}_{\mathcal N}Q$ implies:
\begin{enumerate}
\item If $P \red P'$ then $Q \wred Q'$ and $P'\rel{S}_{\mathcal N} Q'$.
\item If $P\downarrow_{\mathcal N} x$, then $Q\Downarrow_{\mathcal N} x$.
\end{enumerate}
$P$ is ${\mathcal N}$-barbed bisimilar to $Q$, written
$P \wbbisim_{\mathcal N} Q$, if $P \rel{S}_{\mathcal N} Q$ for some ${\mathcal N}$-barbed bisimulation ${\mathcal S}_{\mathcal N}$.
\end{definition}

$\mathcal{R} \subseteq \pi \times \pi$

$P \mathcal{R} Q => \forall P'. P \red P' \Rightarrow \exists Q'. Q \red Q', P' \mathcal{R} Q'$

$P \vdash x \Rightarrow Q \vdash x$

\begin{mathpar}
  \inferrule*[lab=Out-barb]{x \nameeq y}{{y}!\langle{Q}\rangle \vdash x}
  \and
  \inferrule*[lab=Par-barb]{\mbox{$P\vdash x$ or $Q\vdash x$}}{\binpar{P}{Q} \vdash x}
\end{mathpar}

\subsubsection{Contexts}

One of the principle advantages of computational calculi like the
$\pi$-calculus is a well-defined notion of context,
contextual-equivalence and a correlation between
contextual-equivalence and notions of bisimulation. The notion of
context allows the decomposition of a process into (sub-)process and
its syntactic environment, its context. Thus, a context may be
thought of as a process with a ``hole'' (written $\Box$) in it. The
application of a context $M$ to a process $P$, written $M[P]$, is
tantamount to filling the hole in $M$ with $P$. In this paper we do
not need the full weight of this theory, but do make use of the notion
of context in the proof the main theorem. 

\begin{mathpar}
  \inferrule* [lab=summation] {} {{M_{M},M_{N}} \bc \Box \;|\; x.M_{A} \;|\; M_{M}+M_{N}}
  \and
  \inferrule* [lab=agent] {} {{M_{A}} \bc (\vec{x})M_{P} \;| \; \clift{P_0,\ldots,M_{P},\ldots,P_N}}
  \and \\
  \inferrule* [lab=process] {} {{M_{P}} \bc M_{N} \;| \;P|M_{P} }
\end{mathpar} 

\begin{mathpar}
  \inferrule* [lab=sychronization] {} {M_{N} \bc \Box \;|\; x?M_{F} \;|\; x!M_{C}}
  \and
  \inferrule* [lab=abstraction] {} {{M_{F}} \bc (x)M_{P} }
  \and
  \inferrule* [lab=concretion] {} {{M_{C}} \bc \langle M_{P} \rangle }
  \and \\
  \inferrule* [lab=process] {} {{M_{P}} \bc M_{N} \;| \;P|M_{P} }
\end{mathpar}

\begin{definition}[contextual application] Given a context $M$, and
  process $P$, we define the \emph{contextual application}, $M[P] :=
  M\{P/\Box\}$. That is, the contextual application of M to P is the
  substitution of $P$ for $\Box$ in $M$.
\end{definition}

$\meaningof{-} : L \to \mathcal{P}(\pi)$

\begin{mathpar}
  \inferrule* [lab=collection] {} {\meaningof{true} = \pi, \and \meaningof{~E} = \pi \setminus \meaningof{E}, \and \meaningof{E_{1} \& E_{2}} = \meaningof{E_{1}} \cap \meaningof{E_{2}}}
\end{mathpar}

\begin{mathpar}
  \inferrule* [lab=structure] {} {\meaningof{0} = \{ P \in \pi | P \equiv 0 \}, \and \\ \meaningof{E_1 | E_2} = \{ P \in \pi | P \equiv P_{1} | P_{2}, P_{1} \in \meaningof{E_{1}}, P_{2} \in \meaningof{E_2}\} }
\end{mathpar}

\begin{mathpar}
 \inferrule* [lab=behavior] {} {\meaningof{\langle a?b \rangle E} = \{ P \in \pi | P \equiv Q | u?(y)P', \\ \and \\\\ \and \\ \;\;\; u \in \meaningof{a}, \forall z.P'\{z/y\} \in \meaningof{E\{z/b\}}\}, \and \\ \meaningof{a!E} = \{ P \in \pi | P \equiv Q | x!\langle P' \rangle, x \in \meaningof{a} P' \in \meaningof{E}\} }
\end{mathpar}

\begin{mathpar}
 \inferrule* [lab=nominal] {} {\meaningof{\quotep{E}} = \{ \quotep{P} \in \quotep{\pi} | P \in \meaningof{E} \}, \and \meaningof{\quotep{P}} = \{ \quotep{Q} \in \quotep{\pi} | P \equiv Q \} \and \\ \meaningof{@\quotep{E}} = \{ P \in \pi | P \equiv @x, x \in \meaningof{E} \}}
\end{mathpar}

\begin{eqnarray*}
  \\
  \meaningof{-} : TS \to ST
\end{eqnarray*}

\begin{eqnarray*}
  \\
  L : TS \to ST
\end{eqnarray*}

\begin{eqnarray*}
  \\
  P \models E \iff P \in \meaningof{E}
\end{eqnarray*}

\begin{eqnarray*}
  P \approx_{L} Q \iff \forall E \in L. P \models E \iff Q \models E
\end{eqnarray*}

\begin{eqnarray*}
  P \approx_{K} Q
\end{eqnarray*}

\begin{eqnarray*}
  P \approx Q
\end{eqnarray*}

$\approx_{K} = \approx = \approx_{L}$

\subsubsection{Contextual duality}

Note that contexts extend the quotation operation to a family of
operations from processes to names. Given a context, $M$, we can
define a \emph{nominal context}, $\quotep{M}$ by $\quotep{M}[P] :=
\quotep{M[P]}$. To foreshadow what is to come we observe that these
operations enjoy a duality with processes very much like the duality
between vectors and maps from vectors to scalars.

Further, because the calculus is essentially higher-order, we have a
correspondence between contexts and processes. More specifically,
given a name $x$ and a context $M$ we can construct $M^{*}_{x}$ such
that 

\begin{mathpar}
  M^{*}_{x} | \lift{x}{P} \red M[P]
\end{mathpar}

namely,

\begin{mathpar}
  M^{*}_{x} := x?(u).M[\dropn{u}]
\end{mathpar}

The dependence of $M^{*}_{x}$ on a name makes it an abstraction, 

\begin{mathpar}
  M^{*} := (x)x?(u).M[\dropn{u}]
\end{mathpar}

\subsection{Additional notation}

It will sometimes be convenient to denote the process a name
quotes. We already have the notation $x = \quotep{P}$, but it will be
convenient to introduce an alternate notation, $\procn{x}$, when we
want to emphasize the connection to the use of the name. Note that, by
virtue of name equivalence, $\quotep{\procn{x}} \nameeq x$; so, the
notation is consistent with previous definitions.

Further, because names have structure it is possible to effect
substitutions on the basis of that structure. This means we need to
upgrade our notation for substitutions, which we accomplish by
adapting comprehension notation. Thus,

\begin{mathpar}
  P\{ y / x : x \in S \}
\end{mathpar}

is interpreted to mean the process derived from P by replacing (in a
capture-avoiding manner) each occurrence of $x$ in $S$ by $y$. For example,

\begin{mathpar}
  P\{ \quotep{\procn{x}|\procn{x}} / x : x \in \freenames{P} \}
\end{mathpar}

will replace each (occurrence) of a free name $x$ in $P$ by
$\quotep{\procn{x}|\procn{x}}$.

Also, we will avail ourselves of the notation $x^{L}$ and $x^{R}$ to
denote injections of a name into disjoint copies of the name
space. There are numerous ways to accomplish this. One example can be
found in \cite{MeredithR05}. This notation overloads to vectors of
names: $\vec{x}^{\pi} := (x_{i}^{\pi} \; : \; 0 \leq i < |\vec{x}| )$ where $\pi \in \{L,R\}$.

We also use $P^{\Box} := P|\Box$.

In \cite{MeredithR05} an interpretation of the new operator is
given. It turns out that there are several possible interpretations
all enjoying the requisite algebraic properties of the operator (see
\cite{milner91polyadicpi}). We will therefore make liberal use of
$(\nu\; \vec{x})P$.

% subsection the_syntax_and_semantics_of_the_notation_system (end)   

\section{Interpretation of QM}
\subsection{Supporting definitions}
\subsubsection{Multiplication}
\begin{mathpar}
  \quotep{Q} \cdot \quotep{R} := \quotep{Q|R}
  \and \\
  \quotep{Q} \cdot P := P\{ \quotep{Q|R} / \quotep{R} : \quotep{R} \in \freenames{P} \}
\end{mathpar}

\paragraph{Discussion}
The first line needs little explanation. The second line says that
each free name of the process is replaced with the multiplication of
that name by the scalar. Multiplication of a scalar (name) by a state
(process) results in a process all the names of which have been `moved
over' by parallel composition with the process the scalar
quotes. There is a subtlety that the bound names have to be
manipulated so that multiplied names aren't accidentally
captured. There are many ways to achieve this.

\begin{remark}\label{rem:multiplication_identities}
  The reader is invited to verify that for all $x,y,z \in \QProc$ and $P \in \Proc$
  \begin{mathpar}
    x \cdot \quotep{0} \equiv x 
    \and
    x \cdot y \equiv y \cdot x
    \and
    x \cdot (y \cdot z) \equiv (x \cdot y) \cdot z
    \and \\
    \quotep{0} \cdot P \equiv P
    \and \\
    x \cdot (y \cdot P) \equiv (x \cdot y) \cdot P
    \and \\
    x \cdot (P|Q) \equiv (x \cdot P) | (x \cdot Q)
    \and \\    
  \end{mathpar}
\end{remark}

\subsubsection{Tensor product}

We define a tensor product on processes by structural induction.

\paragraph{Tensor of sums} First note that all summations, including
$\pzero$ and sequence, can be written $\Sigma_{i} x_{i}.A_{i} +
\Sigma_{j} x_{j}.C_{j}$, where we have grouped input-guarded processes
together and output-guarded processes together.

Thus, we can define the tensor product of two summations, $N_{1}\otimes N_{2}$, where

\begin{mathpar}
  N_{1} := \Sigma_{i} x_{i}.A_{i} + \Sigma_{j} x_{j}.C_{j}
  \and
  N_{2} := \Sigma_{i'} y_{i'}.B_{i'} + \Sigma_{j'} y_{j'}.D_{j'} 
\end{mathpar}

as follows.

\begin{mathpar}
  \Sigma_{i} x_{i}.A_{i} + \Sigma_{j} x_{j}.C_{j} \otimes \Sigma_{i'}
  y_{i'}.B_{i'} + \Sigma_{j'} y_{j'}.D_{j'} 
  \and \\
  := \; \Sigma_{i} \Sigma_{i'} \quotep{\stackrel{\vee}{x_{i}}| \stackrel{\vee}{y_{i'}}}.(A_{i}\otimes B_{i'}) \; | \; \Sigma_{i'} \Sigma_{i} \quotep{\stackrel{\vee}{y_{i'}}|\stackrel{\vee}{x_{i}}}.(B_{i'}\otimes A_{i})
  \and
  \;\; | \;\; \Sigma_{j} \Sigma_{j'} \quotep{\stackrel{\vee}{x_{j}}|\stackrel{\vee}{y_{j'}}}.(A_{j}\otimes B_{j'}) \; | \; \Sigma_{j'} \Sigma_{j} \quotep{\stackrel{\vee}{y_{j'}}|\stackrel{\vee}{x_{j}}}.(B_{j'}\otimes A_{j})
\end{mathpar}

\begin{remark}
  Do we need to $x^{L}$ and $y^{R}$ for this construction as well?
\end{remark}

\paragraph{Tensor of parallel compositions} Next, we distribute tensor
over par.

\begin{mathpar}
  P_{1}|P_{2} \otimes Q_{1}|Q_{2} := (P_{1} \otimes Q_{1}) | (P_{1}
  \otimes Q_{2}) | (P_{2} \otimes Q_{1}) | (P_{2} \otimes Q_{2})
\end{mathpar}

\paragraph{Tensor with dropped names} We treat tensor of a
process with a dropped name as parallel composition.

\begin{mathpar}
  P \otimes \dropn{x} := P | \dropn{x}
\end{mathpar}

\paragraph{Tensor of agents}

Finally, we need to define tensor on agents. Note that the definition
of tensor on normal products only tensors inputs with inputs and
outputs with outputs. Thus, we only have to define the operation on
``homogeneous'' pairings.

\begin{mathpar}
  (\vec{x})P \otimes (\vec{y})Q
  \and \\
  := (x_{0}^{L}|y_{0}^{R},\ldots,x_{0}^{L}|y_{n}^{R},\ldots,x_{m}^{L}|y_{0}^{R},\ldots,x_{m}^{L}|y_{n}^R)(P\{ \vec{x}^{L}/\vec{x}\} \otimes Q \{ \vec{y}^{R}/\vec{y}\})
  \and \\
  \clift{\vec{P}} \otimes \clift{\vec{Q}}
  \and \\
  := \clift{P_{0}\otimes Q_{0},\ldots,P_{0}\otimes Q_{n},\ldots,P_{m}\otimes Q_{0},\ldots,P_{m}\otimes Q_{n}}
\end{mathpar}

\begin{remark}
  Observe that arities of tensored abstractions matches arities of
  tensored concretions if the original arities matched. Note also that
  the length of the arities corresponds to the increase in dimension
  we see in ordinary vector space tensor product.
\end{remark}

\begin{remark}
  Operationally, this definition distributes the tensor down to
  components ``linked'' by summation. Tensor over summation is
  intriguing in that it mixes names. Moreover, as a consequence of the
  way it mixes names we have the identities for all $x \in \QProc$ and
  $P,Q \in \Proc$

  \begin{mathpar}
    (x \cdot P) \otimes Q \equiv x \cdot (P \otimes Q) \equiv P \otimes (x \cdot Q)
    \and
    P \otimes \pzero \equiv P
  \end{mathpar}

  that the reader is invited to verify.
\end{remark}

\subsubsection{Annihilation}
\begin{mathpar}
  P^{\perp} := \{ Q | \forall R. P|Q \red^{*} R \Rightarrow R \red^{*} \pzero \}
  \and \\
  P^{\underline{\perp}} := \Sigma_{Q \in P^{\perp}} \quotep{Q}?(y).(\dropn{y}|Q) | \Sigma_{Q \in P^{\perp}} \quotep{Q}\clift{\Box}
\end{mathpar}

\paragraph{Discussion} The reader will note that $P^{\perp}$ is a
\emph{set} of processes, while $P^{\underline{\perp}}$ is a
\emph{context}. We call the set $P^{\perp}$ the \emph{annihilators} of
$P$. The parallel composition of a process in the annihilators of $P$
with $P$ will result in a process, the state space of which has all
paths eventually leading to $\pzero$. Execution may endure loops; but
under reasonable conditions of fairness (naturally guaranteed under
most notions of bisimulation) such a composite process cannot get
stuck in such a loop and will, eventually pop out and terminate.

The context $P^{\underline{\perp}}$ is ready and willing to ``take the
$P$ out of'' the process to which it is applied. It will effectively
transmit the code of the process to which it is applied to one of the
annihilators and run the process against it.

\subsubsection{Evaluation}
We fix $M$ a domain of fully abstract interpretation with an equality
coincident with bisimulation. We take $\meaningof{\cdot} : \Proc \to
M$ to be the map interpreting processes and $\nmeaningof{\cdot} : \M
\to Proc$ to be the map running the other way. Then we define

\begin{mathpar}
  \int P := \nmeaningof{\meaningof{P}}
\end{mathpar}

\paragraph{Discussion}
There are many fully abstract interpretations of Milner's
$\pi$-calculus. Any of them can be used as a basis for interpreting
the reflective calculus here. Equipped with such a domain it is
largely a matter of grinding through to check that the Yoneda
construction for the normalization-by-evaluation program can be
extended to this setting.

\begin{remark}
  The reader is invited to verify that $\int (P^{\underline{\perp}}[P]) = 0$.
\end{remark}

\subsection{Quantum mechanics}

Table \ref{tbl:core_qm_op_defns} gives the core operational definitions

\begin{table}[htp]\label{tbl:core_qm_op_defns}
  \center{
    \fbox{
      \begin{tabular}{c|c}
        quantum mechanics & process calculus \\
        \hline
        scalar & $x := \quotep{P}$ \\
        state vector & $\state{P} := P$ \\
        dual & $\state{P}^{*} := \event{P^{\underline{\perp}}} := \quotep{P^{\underline{\perp}}}[-]$ \\
        matrix & $ \Sigma_{\alpha} \state{P_{\alpha}}x_{\alpha}\event{Q_{\alpha}}$ \\
        vector addition & $\state{P} + \state{Q} := \state{P | Q}$ \\
        tensor product & $\state{P} \otimes \state{Q} := \state{P \otimes Q}$ \\
        inner product & $\innerprod{P}{Q} := \quotep{\int P^{\underline{\perp}}[Q]}$ \\
      \end{tabular}
    }
  }
  \caption{QM - operational definitions}
\end{table}

where

\begin{mathpar}
  \prmatrix{P}{Q} := \fprmatrix{P}{\quotep{\pzero}}{Q}
  \and
  \fprmatrix{P}{x}{Q} := (\state{P},x,\event{Q})
  \and
  (\fprmatrix{P}{x}{Q})(\state{R}) := x \cdot \innerprod{Q}{R} \cdot \state{P}
  \and
  (\fprmatrix{P}{x}{Q})(\event{R}) := x \cdot \innerprod{R}{P} \cdot \event{Q}
\end{mathpar}

\paragraph{Discussion}
As promised: vectors (aka states) are represented as processes; duals
as contextual duals; inner product definition should be compared with
standard inner product definition for ....

\begin{remark}
  Assuming $\int (P^{\underline{\perp}}[P]) = 0$, the reader is
  invited to verify that $(\fprmatrix{P}{x}{P})(\state{P}) = x \cdot \state{P}$.
\end{remark}

\begin{remark}
  The reader is invited to verify that $\innerprod{P}{Q}$ could
  equally well have been written $\quotep{\int \stackrel{\vee}{x}}$
  where $x = \event{P^{\underline{\perp}}}(Q)$.

  One of the motivations for this remark is that there is another way
  to factor these operations. We could package up evaluation in the dual:

  \begin{mathpar}
    \state{P}^{*} := \event{\int P^{\underline{\perp}}} := \quotep{\int P^{\underline{\perp}}}[-]
  \end{mathpar}

  and then have inner product defined by
  
  \begin{mathpar}
    \innerprod{P}{Q} := \event{P}(Q)
  \end{mathpar}

  Hopefully, experience with the calculations will provide guidance on
  the best factoring.
\end{remark}

\begin{remark}
  Assuming $\int (P^{\underline{\perp}}[P]) = 0$, the reader is
  invited to verify that $\forall P,Q. (\prmatrix{0}{Q})(\state{0}) =
  \state{0}$ and dually $(\prmatrix{P}{0})(\event{0}) = \event{0}$.
\end{remark}

\begin{remark}
  i'm a little worried that i don't (yet) have proper support for
  complex conjugacy. But, the observation above may give us a
  clue. According to Abramsky, it must be the case that the scalars
  are iso to the homset of the identity for the tensor -- which the
  observation above characterizes. 

  For now, we will simply bookmark the notion with $\overline{x}$.
\end{remark}

\subsubsection{Adjointness}

We need to give a definition of $(\cdot)^{\dagger}$ for matrices. The
obvious candidate definition is
\begin{mathpar}
(\Sigma_{\alpha}\fprmatrix{P_{\alpha}}{x_{\alpha}}{Q_{\alpha}})^{\dagger}
= \Sigma_{\alpha}\fprmatrix{(Q_{\alpha}^{\underline{\perp}})^{*}}{\overline{x}_{\alpha}}{P_{\alpha}^{\underline{\perp}}} 
\end{mathpar}

But, $(Q_{\alpha}^{\underline{\perp}})^{*}$ requires a name along
which to communicate the process to achieve the context application.

\subsubsection{Basis for a basis}
If processes label states and ``addition'' of states (a.k.a. vector
addition) is interpreted as parallel composition, what corresponds to
notions of linear independence and basis? Here, we recall that Yoshida
has developed a set of \emph{combinators} for an asynchronous verison
of Milner's $\pi$-calculus. These are a finite set of processes such
any process can be expressed as parallel composition of these
combinators together with liberal uses of the new operator and
replication. We can simply give a translation of these into the
present calculus and have reasonable expectation that the property
carries over. That is, that the resultant set allows to express all
processes via parallel composition. Note, however, that there is no
new operator or replication in this calculus. As a result, we expect
that the corresponding set is actually infinite. That is, we expect
that the space is actually infinite dimensional.

\begin{remark}
  The attentive reader may be a bit concerned. Certainly, the
  collection $S$, $K$ and $I$ is a finite set of
  combinators. Shouldn't we expect to see a finite set of combinators
  for an effectively equivalent system? i am very sympathetic to this
  critique and feel it warrants full attention. On the other hand, i
  also have in mind the following analogy. The natural numbers, as a
  monoid under addition, has exactly $1$ generator, while the natural
  numbers, as a monoid under multiplication, has countably many
  generators (the primes). We observe that the application of the
  lambda calculus is much less resource sensitive than the parallel
  composition of the $\pi$-calculus. Could it be the case that we have
  an analogy of the form
  
  \begin{mathpar}
    m + n : MN :: m*n : M|N
  \end{mathpar}

  giving a similar blow up in the set of ``primes''?  This is such a
  wonderful thought that, even if it's not true, i think it's worth
  writing down.
\end{remark}
 

\documentclass[12pt]{llncs}
%\documentclass{jktr}

\usepackage[pdftex]{hyperref}                   
\usepackage {listings}
\usepackage {mathpartir}
\usepackage{bcprules}
%\usepackage{listings}
                       
\usepackage{graphicx} 
%\usepackage[margins=2.5cm,nohead,nofoot]{geometry}
%\usepackage{geometry}
\usepackage{amsfonts}
\usepackage{amstext}
\usepackage{latexsym}
\usepackage{amssymb}
\usepackage{color}


%\include{myPreamble}
\include{qm2pi.local} 

%\ifpdf
%\usepackage[pdftex]{graphicx}
%\else
%\usepackage{graphicx}
%\fi

 % \ifpdf
%  \usepackage{pdfsync}
%  \if


%\title{Brief Article}
%\author{David F. Snyder}
%\author{L.G. Meredith}

%\address{Dept. of Math., Texas State University--San Marcos, San Marcos, TX 78666}
       
\pagestyle{empty}


\begin{document}

\lstset{language=[Objective]Caml,frame=shadowbox}

\input{qm2pi.front}

% section front matter (end)

\input{qm2pi.intro} 
 
% section introduction (end)

% \input{qm2pi.knotations} 

% section notation (end)

\input{qm2pi.process.calculi} 

% section concurrent_process_calculi_and_spatial_logics_ (end)
    
%\input{qm2pi.knots2pi} 

%\input{qm2pi.trefoil} 

%\input{qm2pi.mainthm} 

% subsection basic_interpretation (end)

%\input{qm2pi.rho.presentation} 
\subsection{The syntax and semantics of the notation system}\label{sub:the_syntax_and_semantics_of_the_notation_system} % (fold)

We now summarize a technical presentation of the calculus that
embodies our theory of dynamics. The typical presentation of such a
calculus follows the style of giving generators and relations on
them. The grammar, below, describing term constructors, freely
generates the set of processes, $\Proc$. This set is then quotiented
by a relation known as structural congruence and it is over this set
that the notion of dynamics is expressed. This presentation is
essentially that of \cite{MeredithR05} with the addition of
polyadicity and summation. For readability we have relegated some of
the technical subtleties to an appendix.

\subsubsection{Process grammar}\label{subsub:process_grammar}

\begin{mathpar}
  \inferrule* [lab=synchronization] {} {{M} \bc \pzero \;|\; x?F \;|\; x!C }
  \and
  \inferrule* [lab=abstraction] {} {{F} \bc (x)P}
  \and
  \inferrule* [lab=concretion] {} {{C} \bc \langle Q \rangle}
  \and
  \inferrule* [lab=process] {} {{P,Q} \bc M \;| \;P|Q \;|\; @{x}}
  \and
  \inferrule* [lab=name] {} {{x} \bc \quotep{P}}
\end{mathpar} 

Note that $\vec{x}$ (resp. $\vec{P}$) denotes a vector of names
(resp. processes) of length $|\vec{x}|$ (resp. $|\vec{P}|$). We adopt
the following useful abbreviations.

\begin{mathpar}
   x?(\vec{y}).P := x.(\vec{y})P \and  x\clift{\vec{P}} := x.\clift{\vec{P}}
   \and x!(y) := \lift{x}{\dropn{y}}
   \and \Pi_{i=0}^{n-1}P_i := P_0 | \ldots | P_{n-1}
\end{mathpar}

\subsubsection{Structural congruence}

\paragraph{Free and bound names and alpha-equivalence.} At the
core of structural equivalence is alpha-equivalence which identifies
process that are the same up to a change of variable. Formally, we
recognize the distinction between free and bound names. The free names
of a process, $\freenames{P}$, may be calculated recursively as
follows:

\begin{mathpar}
\freenames{\pzero} := \emptyset
  \and \\
  \freenames{x?(y).P} := \{ x \} \cup (\freenames{P} \setminus \{ y \})
  \and 
  \freenames{x!\langle P \rangle} := \{ x \} \cup \{ P \} 
  \and \\
  \freenames{P|Q} := \freenames{P} \cup \freenames{Q}
  \and \\
  \freenames{@{x}} := \{ x \}
\end{mathpar}

$\pi$
$\quotep{\pi}$

$\freenames{-} : \pi \to \mathcal{P}(\quotep{\pi})$

\begin{eqnarray*}
  \freenames{\pzero} & := & \emptyset \\
  \freenames{x?(y).P} & := & \{ x \} \cup (\freenames{P} \setminus \{ y \}) \\
  \freenames{x!\langle P \rangle} & := & \{ x \} \cup \{ P \} \\
  \freenames{P|Q} & := & \freenames{P} \cup \freenames{Q} \\
  \freenames{\dropn{x}} & := & \{ x \}
\end{eqnarray*}

The bound names of a process, $\boundnames{P}$, are those names occurring in $P$
that are not free. For example, in $x?(y).0$, the name $x$ is free, while $y$ is bound.

\begin{mathpar}
  \inferrule* [lab=monoidal-laws] {} { P|Q \equiv Q|P \and P|0 \equiv P \and P|(Q|R) \equiv (P|Q)|R }
\end{mathpar}

\begin{mathpar}
  \inferrule* [lab=alpha-equivalence] {} { (x)P \equiv (y)P\{y/x\} \and y \not\in \freenames{P} }
\end{mathpar}

\begin{definition}
Then two processes, $P,Q$, are alpha-equivalent if $P = Q\{\vec{y}/\vec{x}\}$ for
some $\vec{x} \in \boundnames{Q},\vec{y} \in \boundnames{P}$, where $Q\{\vec{y}/\vec{x}\}$
denotes the capture-avoiding substitution of $\vec{y}$ for $\vec{x}$ in $Q$.
\end{definition}

\begin{definition}
  The {\em structural congruence} \cite{SangiorgiWalker} , $\equiv$,
  between processes is the least congruence containing
  alpha-equivalence, satisfying the abelian monoid laws
  (associativity, commutativity and $\pzero$ as identity) for parallel
  composition $|$ and for summation $+$.
\end{definition}

\subsection{Name equivalence}

We take name equivalence, written $\nameeq$, to be the smallest
equivalence relation generated by the following rules.

\begin{mathpar}
\inferrule*[lab=Quote-drop]
{ }
{ \quotep{@{x}} \nameeq x }

\inferrule*[lab=Struct-equiv]
{ P \scong Q }
{ \quotep{P} \nameeq \quotep{Q} }
\end{mathpar}

The astute reader will have noticed that the mutual recursion of names
and processes imposes a mutual recursion on alpha-equivalence and
structural equivalence via name-equivalence. Fortunately, all of this
works out pleasantly and we may calculate in the natural way, free of
concern. The reader interested in the details is referred to the
appendix \ref{appendix:rho_details}.

\subsection{Substitution}

We use $\Proc$ for the set of processes, $\QProc$ for the set of
names, and $\id{\{}\vec{y} / \vec{x} \id{\}}$ to denote partial maps,
$s : \QProc \rightarrow \QProc$. A map, $s$ lifts, uniquely, to a map
on process terms, $\widehat{s} : \Proc \rightarrow \Proc$ by the
following equations.

\begin{mathpar}
  (0) \psubstp{Q}{P} := 0 \\
  (R \juxtap S) \psubstp{Q}{P}
  :=    
  (R)\psubstp{Q}{P} \juxtap (S) \psubstp{Q}{P} \\
  (x?(y).R) \psubstp{Q}{P}    
  :=    
  (x)\substp{Q}{P} (z)\concat( (R \psubstn{z}{y}) \psubstp{Q}{P} ) \\
  (\lift{x}{R}) \psubstp{Q}{P}  
  :=
  \lift{(x)\substp{Q}{P}}{ R \psubstp{Q}{P} } \\
%   (\dropn{x})  \psubstp{Q}{P}       
%   := 
%   \left\{ 
%     \begin{array}{ccc} 
%       \dropn{\quotep{Q}} & & x \nameeq \quotep{P} \\
%       \dropn{x} & & otherwise \\
%     \end{array}
%   \right. 
  (\dropn{x})  \psubstp{Q}{P}       
  := 
  \left\{ 
    \begin{array}{ccc} 
      Q & & x \nameeq \quotep{P} \\
      \dropn{x} & & otherwise \\
    \end{array}
  \right.
\end{mathpar}
 

where

\begin{eqnarray}
  (x)\id{\{} \lpquote Q \rpquote / \lpquote P \rpquote \id{\}}            = 
  \left\{ 
    \begin{array}{ccc}
      \lpquote Q \rpquote & & x \nameeq \lpquote P \rpquote \\
      x & & otherwise \\
    \end{array}
  \right. \nonumber
\end{eqnarray}

and $z$ is chosen distinct from $\quotep{P}$, $\quotep{Q}$, the free
names in $Q$, and all the names in $R$. Our $\alpha$-equivalence will
be built in the standard way from this substitution.

\begin{remark}\label{rem:no_self_referential_names}
  One consequence of these definitions is that $\forall P. \quotep{P}
  \not\in \freenames{P}$.
\end{remark}

\subsection{ Dynamic quote: an example }

Anticipating something of what's to come, consider applying the
substitution, $\widehat{\id{\{}u / z \id{\}}}$, to the following pair
of processes, $\lift{w}{y!(z)}$ and $w[ \lpquote y!(z) \rpquote ]$.

\begin{eqnarray}
	\lift{w}{y!(z)}\widehat{\id{\{}u / z \id{\}}}
		& = &
		\lift{w}{y!(u)} \nonumber\\
	w[ \lpquote y!(z) \rpquote ] \widehat{ \id{\{}u / z \id{\}} }
		& = &
		w[ \lpquote y!(z) \rpquote ] \nonumber
\end{eqnarray}

Because the body of the process between quotes is impervious to
substitution, we get radically different answers. In fact, by
examining the first process in an input context,
e.g. $x?(z).\lift{w}{y!(z)}$, we see that the process under the lift
operator may be shaped by prefixed inputs binding a name inside it. In
this sense, the lift operator will be seen as a way to dynamically
construct processes before reifying them as names.

Finally equipped with these standard features we can present the
dynamics of the calculus.

\subsubsection{Operational semantics} 

Finally, we introduce the computational dynamics. What marks these
algebras as distinct from other more traditionally studied algebraic
structures, e.g. vector spaces or polynomial rings, is the manner in
which dynamics is captured. In traditional structures, dynamics is typically
expressed through morphisms between such structures, as in linear maps
between vector spaces or morphisms between rings. In algebras
associated with the semantics of computation, the dynamics is
expressed as part of the algebraic structure itself, through a
reduction reduction relation typically denoted by $\red$. Below, we
give a recursive presentation of this relation for the calculus used
in the encoding.

$\red \subseteq \pi \times \pi$
$\red : \pi \to \mathcal{P}(\pi)$

\begin{mathpar}
  \inferrule* [lab=Comm] { \textsf{match}( x_{src}, x_{trgt} ) } { x_{trgt}?(y)P \; | \; x_{src}!\langle {Q} \rangle \red P\{\quotep{Q}/y}\} }
  \and \\
  \inferrule* [lab=Par] {{P} \red {P}'} {{{P} | {Q}} \red {{P}' | {Q}}}
  \and
  \inferrule* [lab=Equiv]{{{P} \scong {P}'} \andalso {{P}' \red {Q}'} \andalso {{Q}' \scong {Q}}}{{P} \red {Q}}
\end{mathpar}

\begin{eqnarray*}
  match_{\equiv} (\quotep{P},\quotep{Q}) & := & P \equiv Q \\
  match_{\dagger}(\quotep{P},\quotep{Q}) & := & \forall R. P|Q \red^{*} R => R \red^{*} 0 \\
  match_{K}(\quotep{P},\quotep{Q}) & := & K \mbox{ for some context } K
\end{eqnarray*}

$u?(x)P | u!\langle Q \rangle \red P\{\quotep{Q}/x\}$

%We write $\wred$ for $\red^*$, and $P\red$ if $\exists Q $ such that $ P \red Q$.
We write $P\red$ if $\exists Q $ such that $ P \red Q$ and $P\not\red$, otherwise.

\section{Replication}

As mentioned before, it is known that replication (and hence
recursion) can be implemented in a higher-order process algebra
\cite{SangiorgiWalker}. As our first example of calculation with the
machinery thus far presented we give the construction explicitly in
the {\rhoc}.

\begin{eqnarray}
	D_{x} & := & \prefix{x}{y}{(\binpar{\outputp{x}{y}}{@{y}})} \nonumber\\
	\bangp_{x}{P} & := & \binpar{{x}!\langle{\binpar{D_{x}}{P}}\rangle}{D_{x}} \nonumber
\end{eqnarray}

\begin{eqnarray}
	\bangp_{x}{P} & & \nonumber\\
	=
	& {x}!\langle{(\prefix{x}{y}{(\outputp{x}{y} | @{y})) | P}}\rangle 
	      | \prefix{x}{y}{(\outputp{x}{y} | @{y})} & \nonumber\\
	\red
	& (\outputp{x}{y} | @{y})\substn{\quotep{(\prefix{x}{y}{(@{y} | \outputp{x}{y})) | P}}}{y} & \nonumber\\
	=
	& \outputp{x}{\quotep{(\prefix{x}{y}{(\outputp{x}{y} | @{y})) | P}}}
	  | {(\prefix{x}{y}{(\outputp{x}{y} | @{y})) | P}} & \nonumber\\
	\red
	& \ldots & \nonumber\\
	\red^*
	& P | P | \ldots & \nonumber
\end{eqnarray}

Of course, this encoding, as an implementation, runs away, unfolding
$\bangp{P}$ eagerly. A lazier and more implementable replication
operator, restricted to input-guarded processes, may be obtained as follows.

\begin{eqnarray}
\bangp{\prefix{u}{v}{P}} 
	:= 
	\binpar{\lift{x}{\prefix{u}{v}{(\binpar{D(x)}{P})}}}{D(x)} \nonumber
\end{eqnarray}

\begin{remark}
  Note that the lazier definition still does not deal with summation
  or mixed summation (i.e. sums over input and output). The reader is
  invited to construct definitions of replication that deal with these
  features. 

  Further, the definitions are parameterized in a name, $x$. Can you,
  gentle reader, make a definition that eliminates this parameter and
  guarantees no accidental interaction between the replication
  machinery and the process being replicated -- i.e. no accidental
  sharing of names used by the process to get its work done and the
  name(s) used by the replication to effect copying. This latter
  revision of the definition of replication is crucial to obtaining
  the expected identity $!!P \sim !P$.
\end{remark}

\begin{remark}\label{rem:paradoxical_combinator}
  The reader familiar with the lambda calculus will have noticed the
  similarity between $D$ and the paradoxical combinator.

  [Ed. note: the existence of this seems to suggest we have to be more
  restrictive on the set of processes and names we admit if we are to
  support no-cloning.]
\end{remark}

\subsubsection{Bisimulation}

The computational dynamics gives rise to another kind of equivalence,
the equivalence of computational behavior. As previously mentioned
this is typically captured \emph{via} some form of bisimulation.

% The notion we use in this paper is weak barbed bisimulation
% \cite{milner91polyadicpi}.

The notion we use in this paper is derived from weak barbed
bisimulation \cite{milner91polyadicpi}. 

\begin{definition}
An \emph{observation relation}, $\downarrow_{\mathcal N}$, over a set
of names, $\mathcal N$, is the smallest relation satisfying the rules
below.

\infrule[Out-barb]{y \in {\mathcal N}, \; x \nameeq y}
		  {\outputp{x}{v} \downarrow_{\mathcal N} x}
\infrule[Par-barb]{\mbox{$P\downarrow_{\mathcal N} x$ or $Q\downarrow_{\mathcal N} x$}}
		  {\binpar{P}{Q} \downarrow_{\mathcal N} x}

We write $P \Downarrow_{\mathcal N} x$ if there is $Q$ such that 
$P \wred Q$ and $Q \downarrow_{\mathcal N} x$.
\end{definition}

\begin{definition}
%\label{def.bbisim}
An  ${\mathcal N}$-\emph{barbed bisimulation} over a set of names, ${\mathcal N}$, is a symmetric binary relation 
${\mathcal S}_{\mathcal N}$ between agents such that $P\rel{S}_{\mathcal N}Q$ implies:
\begin{enumerate}
\item If $P \red P'$ then $Q \wred Q'$ and $P'\rel{S}_{\mathcal N} Q'$.
\item If $P\downarrow_{\mathcal N} x$, then $Q\Downarrow_{\mathcal N} x$.
\end{enumerate}
$P$ is ${\mathcal N}$-barbed bisimilar to $Q$, written
$P \wbbisim_{\mathcal N} Q$, if $P \rel{S}_{\mathcal N} Q$ for some ${\mathcal N}$-barbed bisimulation ${\mathcal S}_{\mathcal N}$.
\end{definition}

$\mathcal{R} \subseteq \pi \times \pi$

$P \mathcal{R} Q => \forall P'. P \red P' \Rightarrow \exists Q'. Q \red Q', P' \mathcal{R} Q'$

$P \vdash x \Rightarrow Q \vdash x$

\begin{mathpar}
  \inferrule*[lab=Out-barb]{x \nameeq y}{{y}!\langle{Q}\rangle \vdash x}
  \and
  \inferrule*[lab=Par-barb]{\mbox{$P\vdash x$ or $Q\vdash x$}}{\binpar{P}{Q} \vdash x}
\end{mathpar}

\subsubsection{Contexts}

One of the principle advantages of computational calculi like the
$\pi$-calculus is a well-defined notion of context,
contextual-equivalence and a correlation between
contextual-equivalence and notions of bisimulation. The notion of
context allows the decomposition of a process into (sub-)process and
its syntactic environment, its context. Thus, a context may be
thought of as a process with a ``hole'' (written $\Box$) in it. The
application of a context $M$ to a process $P$, written $M[P]$, is
tantamount to filling the hole in $M$ with $P$. In this paper we do
not need the full weight of this theory, but do make use of the notion
of context in the proof the main theorem. 

\begin{mathpar}
  \inferrule* [lab=summation] {} {{M_{M},M_{N}} \bc \Box \;|\; x.M_{A} \;|\; M_{M}+M_{N}}
  \and
  \inferrule* [lab=agent] {} {{M_{A}} \bc (\vec{x})M_{P} \;| \; \clift{P_0,\ldots,M_{P},\ldots,P_N}}
  \and \\
  \inferrule* [lab=process] {} {{M_{P}} \bc M_{N} \;| \;P|M_{P} }
\end{mathpar} 

\begin{mathpar}
  \inferrule* [lab=sychronization] {} {M_{N} \bc \Box \;|\; x?M_{F} \;|\; x!M_{C}}
  \and
  \inferrule* [lab=abstraction] {} {{M_{F}} \bc (x)M_{P} }
  \and
  \inferrule* [lab=concretion] {} {{M_{C}} \bc \langle M_{P} \rangle }
  \and \\
  \inferrule* [lab=process] {} {{M_{P}} \bc M_{N} \;| \;P|M_{P} }
\end{mathpar}

\begin{definition}[contextual application] Given a context $M$, and
  process $P$, we define the \emph{contextual application}, $M[P] :=
  M\{P/\Box\}$. That is, the contextual application of M to P is the
  substitution of $P$ for $\Box$ in $M$.
\end{definition}

$\meaningof{-} : L \to \mathcal{P}(\pi)$

\begin{mathpar}
  \inferrule* [lab=collection] {} {\meaningof{true} = \pi, \and \meaningof{~E} = \pi \setminus \meaningof{E}, \and \meaningof{E_{1} \& E_{2}} = \meaningof{E_{1}} \cap \meaningof{E_{2}}}
\end{mathpar}

\begin{mathpar}
  \inferrule* [lab=structure] {} {\meaningof{0} = \{ P \in \pi | P \equiv 0 \}, \and \\ \meaningof{E_1 | E_2} = \{ P \in \pi | P \equiv P_{1} | P_{2}, P_{1} \in \meaningof{E_{1}}, P_{2} \in \meaningof{E_2}\} }
\end{mathpar}

\begin{mathpar}
 \inferrule* [lab=behavior] {} {\meaningof{\langle a?b \rangle E} = \{ P \in \pi | P \equiv Q | u?(y)P', \\ \and \\\\ \and \\ \;\;\; u \in \meaningof{a}, \forall z.P'\{z/y\} \in \meaningof{E\{z/b\}}\}, \and \\ \meaningof{a!E} = \{ P \in \pi | P \equiv Q | x!\langle P' \rangle, x \in \meaningof{a} P' \in \meaningof{E}\} }
\end{mathpar}

\begin{mathpar}
 \inferrule* [lab=nominal] {} {\meaningof{\quotep{E}} = \{ \quotep{P} \in \quotep{\pi} | P \in \meaningof{E} \}, \and \meaningof{\quotep{P}} = \{ \quotep{Q} \in \quotep{\pi} | P \equiv Q \} \and \\ \meaningof{@\quotep{E}} = \{ P \in \pi | P \equiv @x, x \in \meaningof{E} \}}
\end{mathpar}

\begin{eqnarray*}
  \\
  \meaningof{-} : TS \to ST
\end{eqnarray*}

\begin{eqnarray*}
  \\
  L : TS \to ST
\end{eqnarray*}

\begin{eqnarray*}
  \\
  P \models E \iff P \in \meaningof{E}
\end{eqnarray*}

\begin{eqnarray*}
  P \approx_{L} Q \iff \forall E \in L. P \models E \iff Q \models E
\end{eqnarray*}

\begin{eqnarray*}
  P \approx_{K} Q
\end{eqnarray*}

\begin{eqnarray*}
  P \approx Q
\end{eqnarray*}

$\approx_{K} = \approx = \approx_{L}$

\subsubsection{Contextual duality}

Note that contexts extend the quotation operation to a family of
operations from processes to names. Given a context, $M$, we can
define a \emph{nominal context}, $\quotep{M}$ by $\quotep{M}[P] :=
\quotep{M[P]}$. To foreshadow what is to come we observe that these
operations enjoy a duality with processes very much like the duality
between vectors and maps from vectors to scalars.

Further, because the calculus is essentially higher-order, we have a
correspondence between contexts and processes. More specifically,
given a name $x$ and a context $M$ we can construct $M^{*}_{x}$ such
that 

\begin{mathpar}
  M^{*}_{x} | \lift{x}{P} \red M[P]
\end{mathpar}

namely,

\begin{mathpar}
  M^{*}_{x} := x?(u).M[\dropn{u}]
\end{mathpar}

The dependence of $M^{*}_{x}$ on a name makes it an abstraction, 

\begin{mathpar}
  M^{*} := (x)x?(u).M[\dropn{u}]
\end{mathpar}

\subsection{Additional notation}

It will sometimes be convenient to denote the process a name
quotes. We already have the notation $x = \quotep{P}$, but it will be
convenient to introduce an alternate notation, $\procn{x}$, when we
want to emphasize the connection to the use of the name. Note that, by
virtue of name equivalence, $\quotep{\procn{x}} \nameeq x$; so, the
notation is consistent with previous definitions.

Further, because names have structure it is possible to effect
substitutions on the basis of that structure. This means we need to
upgrade our notation for substitutions, which we accomplish by
adapting comprehension notation. Thus,

\begin{mathpar}
  P\{ y / x : x \in S \}
\end{mathpar}

is interpreted to mean the process derived from P by replacing (in a
capture-avoiding manner) each occurrence of $x$ in $S$ by $y$. For example,

\begin{mathpar}
  P\{ \quotep{\procn{x}|\procn{x}} / x : x \in \freenames{P} \}
\end{mathpar}

will replace each (occurrence) of a free name $x$ in $P$ by
$\quotep{\procn{x}|\procn{x}}$.

Also, we will avail ourselves of the notation $x^{L}$ and $x^{R}$ to
denote injections of a name into disjoint copies of the name
space. There are numerous ways to accomplish this. One example can be
found in \cite{MeredithR05}. This notation overloads to vectors of
names: $\vec{x}^{\pi} := (x_{i}^{\pi} \; : \; 0 \leq i < |\vec{x}| )$ where $\pi \in \{L,R\}$.

We also use $P^{\Box} := P|\Box$.

In \cite{MeredithR05} an interpretation of the new operator is
given. It turns out that there are several possible interpretations
all enjoying the requisite algebraic properties of the operator (see
\cite{milner91polyadicpi}). We will therefore make liberal use of
$(\nu\; \vec{x})P$.

% subsection the_syntax_and_semantics_of_the_notation_system (end)   

\input{qm2pi.qmops} 

\input{qm2pi.sterngerlach} 

\input{qm2pi.metric} 

% section concurrent_process_calculi (end)

%\input{qm2pi.proofsketch}

% section proof sketch (end)

%\input{qm2pi.slviaknots} 

% section spatial logic via knots (end)

\input{qm2pi.conclusion}

% section conclusion (end)

%\input{qm2pi.dtcodes} 

% section wiring algorithm (end)

\input{qm2pi.ack} 

% section acknowledgments (end)

\newpage


\bibliographystyle{plain}   
\bibliography{../../biblios/main.bib}

\input{qm2pi.rhodetails}

\end{document}

 

\documentclass[12pt]{llncs}
%\documentclass{jktr}

\usepackage[pdftex]{hyperref}                   
\usepackage {listings}
\usepackage {mathpartir}
\usepackage{bcprules}
%\usepackage{listings}
                       
\usepackage{graphicx} 
%\usepackage[margins=2.5cm,nohead,nofoot]{geometry}
%\usepackage{geometry}
\usepackage{amsfonts}
\usepackage{amstext}
\usepackage{latexsym}
\usepackage{amssymb}
\usepackage{color}


%\include{myPreamble}
\include{qm2pi.local} 

%\ifpdf
%\usepackage[pdftex]{graphicx}
%\else
%\usepackage{graphicx}
%\fi

 % \ifpdf
%  \usepackage{pdfsync}
%  \if


%\title{Brief Article}
%\author{David F. Snyder}
%\author{L.G. Meredith}

%\address{Dept. of Math., Texas State University--San Marcos, San Marcos, TX 78666}
       
\pagestyle{empty}


\begin{document}

\lstset{language=[Objective]Caml,frame=shadowbox}

\input{qm2pi.front}

% section front matter (end)

\input{qm2pi.intro} 
 
% section introduction (end)

% \input{qm2pi.knotations} 

% section notation (end)

\input{qm2pi.process.calculi} 

% section concurrent_process_calculi_and_spatial_logics_ (end)
    
%\input{qm2pi.knots2pi} 

%\input{qm2pi.trefoil} 

%\input{qm2pi.mainthm} 

% subsection basic_interpretation (end)

%\input{qm2pi.rho.presentation} 
\subsection{The syntax and semantics of the notation system}\label{sub:the_syntax_and_semantics_of_the_notation_system} % (fold)

We now summarize a technical presentation of the calculus that
embodies our theory of dynamics. The typical presentation of such a
calculus follows the style of giving generators and relations on
them. The grammar, below, describing term constructors, freely
generates the set of processes, $\Proc$. This set is then quotiented
by a relation known as structural congruence and it is over this set
that the notion of dynamics is expressed. This presentation is
essentially that of \cite{MeredithR05} with the addition of
polyadicity and summation. For readability we have relegated some of
the technical subtleties to an appendix.

\subsubsection{Process grammar}\label{subsub:process_grammar}

\begin{mathpar}
  \inferrule* [lab=synchronization] {} {{M} \bc \pzero \;|\; x?F \;|\; x!C }
  \and
  \inferrule* [lab=abstraction] {} {{F} \bc (x)P}
  \and
  \inferrule* [lab=concretion] {} {{C} \bc \langle Q \rangle}
  \and
  \inferrule* [lab=process] {} {{P,Q} \bc M \;| \;P|Q \;|\; @{x}}
  \and
  \inferrule* [lab=name] {} {{x} \bc \quotep{P}}
\end{mathpar} 

Note that $\vec{x}$ (resp. $\vec{P}$) denotes a vector of names
(resp. processes) of length $|\vec{x}|$ (resp. $|\vec{P}|$). We adopt
the following useful abbreviations.

\begin{mathpar}
   x?(\vec{y}).P := x.(\vec{y})P \and  x\clift{\vec{P}} := x.\clift{\vec{P}}
   \and x!(y) := \lift{x}{\dropn{y}}
   \and \Pi_{i=0}^{n-1}P_i := P_0 | \ldots | P_{n-1}
\end{mathpar}

\subsubsection{Structural congruence}

\paragraph{Free and bound names and alpha-equivalence.} At the
core of structural equivalence is alpha-equivalence which identifies
process that are the same up to a change of variable. Formally, we
recognize the distinction between free and bound names. The free names
of a process, $\freenames{P}$, may be calculated recursively as
follows:

\begin{mathpar}
\freenames{\pzero} := \emptyset
  \and \\
  \freenames{x?(y).P} := \{ x \} \cup (\freenames{P} \setminus \{ y \})
  \and 
  \freenames{x!\langle P \rangle} := \{ x \} \cup \{ P \} 
  \and \\
  \freenames{P|Q} := \freenames{P} \cup \freenames{Q}
  \and \\
  \freenames{@{x}} := \{ x \}
\end{mathpar}

$\pi$
$\quotep{\pi}$

$\freenames{-} : \pi \to \mathcal{P}(\quotep{\pi})$

\begin{eqnarray*}
  \freenames{\pzero} & := & \emptyset \\
  \freenames{x?(y).P} & := & \{ x \} \cup (\freenames{P} \setminus \{ y \}) \\
  \freenames{x!\langle P \rangle} & := & \{ x \} \cup \{ P \} \\
  \freenames{P|Q} & := & \freenames{P} \cup \freenames{Q} \\
  \freenames{\dropn{x}} & := & \{ x \}
\end{eqnarray*}

The bound names of a process, $\boundnames{P}$, are those names occurring in $P$
that are not free. For example, in $x?(y).0$, the name $x$ is free, while $y$ is bound.

\begin{mathpar}
  \inferrule* [lab=monoidal-laws] {} { P|Q \equiv Q|P \and P|0 \equiv P \and P|(Q|R) \equiv (P|Q)|R }
\end{mathpar}

\begin{mathpar}
  \inferrule* [lab=alpha-equivalence] {} { (x)P \equiv (y)P\{y/x\} \and y \not\in \freenames{P} }
\end{mathpar}

\begin{definition}
Then two processes, $P,Q$, are alpha-equivalent if $P = Q\{\vec{y}/\vec{x}\}$ for
some $\vec{x} \in \boundnames{Q},\vec{y} \in \boundnames{P}$, where $Q\{\vec{y}/\vec{x}\}$
denotes the capture-avoiding substitution of $\vec{y}$ for $\vec{x}$ in $Q$.
\end{definition}

\begin{definition}
  The {\em structural congruence} \cite{SangiorgiWalker} , $\equiv$,
  between processes is the least congruence containing
  alpha-equivalence, satisfying the abelian monoid laws
  (associativity, commutativity and $\pzero$ as identity) for parallel
  composition $|$ and for summation $+$.
\end{definition}

\subsection{Name equivalence}

We take name equivalence, written $\nameeq$, to be the smallest
equivalence relation generated by the following rules.

\begin{mathpar}
\inferrule*[lab=Quote-drop]
{ }
{ \quotep{@{x}} \nameeq x }

\inferrule*[lab=Struct-equiv]
{ P \scong Q }
{ \quotep{P} \nameeq \quotep{Q} }
\end{mathpar}

The astute reader will have noticed that the mutual recursion of names
and processes imposes a mutual recursion on alpha-equivalence and
structural equivalence via name-equivalence. Fortunately, all of this
works out pleasantly and we may calculate in the natural way, free of
concern. The reader interested in the details is referred to the
appendix \ref{appendix:rho_details}.

\subsection{Substitution}

We use $\Proc$ for the set of processes, $\QProc$ for the set of
names, and $\id{\{}\vec{y} / \vec{x} \id{\}}$ to denote partial maps,
$s : \QProc \rightarrow \QProc$. A map, $s$ lifts, uniquely, to a map
on process terms, $\widehat{s} : \Proc \rightarrow \Proc$ by the
following equations.

\begin{mathpar}
  (0) \psubstp{Q}{P} := 0 \\
  (R \juxtap S) \psubstp{Q}{P}
  :=    
  (R)\psubstp{Q}{P} \juxtap (S) \psubstp{Q}{P} \\
  (x?(y).R) \psubstp{Q}{P}    
  :=    
  (x)\substp{Q}{P} (z)\concat( (R \psubstn{z}{y}) \psubstp{Q}{P} ) \\
  (\lift{x}{R}) \psubstp{Q}{P}  
  :=
  \lift{(x)\substp{Q}{P}}{ R \psubstp{Q}{P} } \\
%   (\dropn{x})  \psubstp{Q}{P}       
%   := 
%   \left\{ 
%     \begin{array}{ccc} 
%       \dropn{\quotep{Q}} & & x \nameeq \quotep{P} \\
%       \dropn{x} & & otherwise \\
%     \end{array}
%   \right. 
  (\dropn{x})  \psubstp{Q}{P}       
  := 
  \left\{ 
    \begin{array}{ccc} 
      Q & & x \nameeq \quotep{P} \\
      \dropn{x} & & otherwise \\
    \end{array}
  \right.
\end{mathpar}
 

where

\begin{eqnarray}
  (x)\id{\{} \lpquote Q \rpquote / \lpquote P \rpquote \id{\}}            = 
  \left\{ 
    \begin{array}{ccc}
      \lpquote Q \rpquote & & x \nameeq \lpquote P \rpquote \\
      x & & otherwise \\
    \end{array}
  \right. \nonumber
\end{eqnarray}

and $z$ is chosen distinct from $\quotep{P}$, $\quotep{Q}$, the free
names in $Q$, and all the names in $R$. Our $\alpha$-equivalence will
be built in the standard way from this substitution.

\begin{remark}\label{rem:no_self_referential_names}
  One consequence of these definitions is that $\forall P. \quotep{P}
  \not\in \freenames{P}$.
\end{remark}

\subsection{ Dynamic quote: an example }

Anticipating something of what's to come, consider applying the
substitution, $\widehat{\id{\{}u / z \id{\}}}$, to the following pair
of processes, $\lift{w}{y!(z)}$ and $w[ \lpquote y!(z) \rpquote ]$.

\begin{eqnarray}
	\lift{w}{y!(z)}\widehat{\id{\{}u / z \id{\}}}
		& = &
		\lift{w}{y!(u)} \nonumber\\
	w[ \lpquote y!(z) \rpquote ] \widehat{ \id{\{}u / z \id{\}} }
		& = &
		w[ \lpquote y!(z) \rpquote ] \nonumber
\end{eqnarray}

Because the body of the process between quotes is impervious to
substitution, we get radically different answers. In fact, by
examining the first process in an input context,
e.g. $x?(z).\lift{w}{y!(z)}$, we see that the process under the lift
operator may be shaped by prefixed inputs binding a name inside it. In
this sense, the lift operator will be seen as a way to dynamically
construct processes before reifying them as names.

Finally equipped with these standard features we can present the
dynamics of the calculus.

\subsubsection{Operational semantics} 

Finally, we introduce the computational dynamics. What marks these
algebras as distinct from other more traditionally studied algebraic
structures, e.g. vector spaces or polynomial rings, is the manner in
which dynamics is captured. In traditional structures, dynamics is typically
expressed through morphisms between such structures, as in linear maps
between vector spaces or morphisms between rings. In algebras
associated with the semantics of computation, the dynamics is
expressed as part of the algebraic structure itself, through a
reduction reduction relation typically denoted by $\red$. Below, we
give a recursive presentation of this relation for the calculus used
in the encoding.

$\red \subseteq \pi \times \pi$
$\red : \pi \to \mathcal{P}(\pi)$

\begin{mathpar}
  \inferrule* [lab=Comm] { \textsf{match}( x_{src}, x_{trgt} ) } { x_{trgt}?(y)P \; | \; x_{src}!\langle {Q} \rangle \red P\{\quotep{Q}/y}\} }
  \and \\
  \inferrule* [lab=Par] {{P} \red {P}'} {{{P} | {Q}} \red {{P}' | {Q}}}
  \and
  \inferrule* [lab=Equiv]{{{P} \scong {P}'} \andalso {{P}' \red {Q}'} \andalso {{Q}' \scong {Q}}}{{P} \red {Q}}
\end{mathpar}

\begin{eqnarray*}
  match_{\equiv} (\quotep{P},\quotep{Q}) & := & P \equiv Q \\
  match_{\dagger}(\quotep{P},\quotep{Q}) & := & \forall R. P|Q \red^{*} R => R \red^{*} 0 \\
  match_{K}(\quotep{P},\quotep{Q}) & := & K \mbox{ for some context } K
\end{eqnarray*}

$u?(x)P | u!\langle Q \rangle \red P\{\quotep{Q}/x\}$

%We write $\wred$ for $\red^*$, and $P\red$ if $\exists Q $ such that $ P \red Q$.
We write $P\red$ if $\exists Q $ such that $ P \red Q$ and $P\not\red$, otherwise.

\section{Replication}

As mentioned before, it is known that replication (and hence
recursion) can be implemented in a higher-order process algebra
\cite{SangiorgiWalker}. As our first example of calculation with the
machinery thus far presented we give the construction explicitly in
the {\rhoc}.

\begin{eqnarray}
	D_{x} & := & \prefix{x}{y}{(\binpar{\outputp{x}{y}}{@{y}})} \nonumber\\
	\bangp_{x}{P} & := & \binpar{{x}!\langle{\binpar{D_{x}}{P}}\rangle}{D_{x}} \nonumber
\end{eqnarray}

\begin{eqnarray}
	\bangp_{x}{P} & & \nonumber\\
	=
	& {x}!\langle{(\prefix{x}{y}{(\outputp{x}{y} | @{y})) | P}}\rangle 
	      | \prefix{x}{y}{(\outputp{x}{y} | @{y})} & \nonumber\\
	\red
	& (\outputp{x}{y} | @{y})\substn{\quotep{(\prefix{x}{y}{(@{y} | \outputp{x}{y})) | P}}}{y} & \nonumber\\
	=
	& \outputp{x}{\quotep{(\prefix{x}{y}{(\outputp{x}{y} | @{y})) | P}}}
	  | {(\prefix{x}{y}{(\outputp{x}{y} | @{y})) | P}} & \nonumber\\
	\red
	& \ldots & \nonumber\\
	\red^*
	& P | P | \ldots & \nonumber
\end{eqnarray}

Of course, this encoding, as an implementation, runs away, unfolding
$\bangp{P}$ eagerly. A lazier and more implementable replication
operator, restricted to input-guarded processes, may be obtained as follows.

\begin{eqnarray}
\bangp{\prefix{u}{v}{P}} 
	:= 
	\binpar{\lift{x}{\prefix{u}{v}{(\binpar{D(x)}{P})}}}{D(x)} \nonumber
\end{eqnarray}

\begin{remark}
  Note that the lazier definition still does not deal with summation
  or mixed summation (i.e. sums over input and output). The reader is
  invited to construct definitions of replication that deal with these
  features. 

  Further, the definitions are parameterized in a name, $x$. Can you,
  gentle reader, make a definition that eliminates this parameter and
  guarantees no accidental interaction between the replication
  machinery and the process being replicated -- i.e. no accidental
  sharing of names used by the process to get its work done and the
  name(s) used by the replication to effect copying. This latter
  revision of the definition of replication is crucial to obtaining
  the expected identity $!!P \sim !P$.
\end{remark}

\begin{remark}\label{rem:paradoxical_combinator}
  The reader familiar with the lambda calculus will have noticed the
  similarity between $D$ and the paradoxical combinator.

  [Ed. note: the existence of this seems to suggest we have to be more
  restrictive on the set of processes and names we admit if we are to
  support no-cloning.]
\end{remark}

\subsubsection{Bisimulation}

The computational dynamics gives rise to another kind of equivalence,
the equivalence of computational behavior. As previously mentioned
this is typically captured \emph{via} some form of bisimulation.

% The notion we use in this paper is weak barbed bisimulation
% \cite{milner91polyadicpi}.

The notion we use in this paper is derived from weak barbed
bisimulation \cite{milner91polyadicpi}. 

\begin{definition}
An \emph{observation relation}, $\downarrow_{\mathcal N}$, over a set
of names, $\mathcal N$, is the smallest relation satisfying the rules
below.

\infrule[Out-barb]{y \in {\mathcal N}, \; x \nameeq y}
		  {\outputp{x}{v} \downarrow_{\mathcal N} x}
\infrule[Par-barb]{\mbox{$P\downarrow_{\mathcal N} x$ or $Q\downarrow_{\mathcal N} x$}}
		  {\binpar{P}{Q} \downarrow_{\mathcal N} x}

We write $P \Downarrow_{\mathcal N} x$ if there is $Q$ such that 
$P \wred Q$ and $Q \downarrow_{\mathcal N} x$.
\end{definition}

\begin{definition}
%\label{def.bbisim}
An  ${\mathcal N}$-\emph{barbed bisimulation} over a set of names, ${\mathcal N}$, is a symmetric binary relation 
${\mathcal S}_{\mathcal N}$ between agents such that $P\rel{S}_{\mathcal N}Q$ implies:
\begin{enumerate}
\item If $P \red P'$ then $Q \wred Q'$ and $P'\rel{S}_{\mathcal N} Q'$.
\item If $P\downarrow_{\mathcal N} x$, then $Q\Downarrow_{\mathcal N} x$.
\end{enumerate}
$P$ is ${\mathcal N}$-barbed bisimilar to $Q$, written
$P \wbbisim_{\mathcal N} Q$, if $P \rel{S}_{\mathcal N} Q$ for some ${\mathcal N}$-barbed bisimulation ${\mathcal S}_{\mathcal N}$.
\end{definition}

$\mathcal{R} \subseteq \pi \times \pi$

$P \mathcal{R} Q => \forall P'. P \red P' \Rightarrow \exists Q'. Q \red Q', P' \mathcal{R} Q'$

$P \vdash x \Rightarrow Q \vdash x$

\begin{mathpar}
  \inferrule*[lab=Out-barb]{x \nameeq y}{{y}!\langle{Q}\rangle \vdash x}
  \and
  \inferrule*[lab=Par-barb]{\mbox{$P\vdash x$ or $Q\vdash x$}}{\binpar{P}{Q} \vdash x}
\end{mathpar}

\subsubsection{Contexts}

One of the principle advantages of computational calculi like the
$\pi$-calculus is a well-defined notion of context,
contextual-equivalence and a correlation between
contextual-equivalence and notions of bisimulation. The notion of
context allows the decomposition of a process into (sub-)process and
its syntactic environment, its context. Thus, a context may be
thought of as a process with a ``hole'' (written $\Box$) in it. The
application of a context $M$ to a process $P$, written $M[P]$, is
tantamount to filling the hole in $M$ with $P$. In this paper we do
not need the full weight of this theory, but do make use of the notion
of context in the proof the main theorem. 

\begin{mathpar}
  \inferrule* [lab=summation] {} {{M_{M},M_{N}} \bc \Box \;|\; x.M_{A} \;|\; M_{M}+M_{N}}
  \and
  \inferrule* [lab=agent] {} {{M_{A}} \bc (\vec{x})M_{P} \;| \; \clift{P_0,\ldots,M_{P},\ldots,P_N}}
  \and \\
  \inferrule* [lab=process] {} {{M_{P}} \bc M_{N} \;| \;P|M_{P} }
\end{mathpar} 

\begin{mathpar}
  \inferrule* [lab=sychronization] {} {M_{N} \bc \Box \;|\; x?M_{F} \;|\; x!M_{C}}
  \and
  \inferrule* [lab=abstraction] {} {{M_{F}} \bc (x)M_{P} }
  \and
  \inferrule* [lab=concretion] {} {{M_{C}} \bc \langle M_{P} \rangle }
  \and \\
  \inferrule* [lab=process] {} {{M_{P}} \bc M_{N} \;| \;P|M_{P} }
\end{mathpar}

\begin{definition}[contextual application] Given a context $M$, and
  process $P$, we define the \emph{contextual application}, $M[P] :=
  M\{P/\Box\}$. That is, the contextual application of M to P is the
  substitution of $P$ for $\Box$ in $M$.
\end{definition}

$\meaningof{-} : L \to \mathcal{P}(\pi)$

\begin{mathpar}
  \inferrule* [lab=collection] {} {\meaningof{true} = \pi, \and \meaningof{~E} = \pi \setminus \meaningof{E}, \and \meaningof{E_{1} \& E_{2}} = \meaningof{E_{1}} \cap \meaningof{E_{2}}}
\end{mathpar}

\begin{mathpar}
  \inferrule* [lab=structure] {} {\meaningof{0} = \{ P \in \pi | P \equiv 0 \}, \and \\ \meaningof{E_1 | E_2} = \{ P \in \pi | P \equiv P_{1} | P_{2}, P_{1} \in \meaningof{E_{1}}, P_{2} \in \meaningof{E_2}\} }
\end{mathpar}

\begin{mathpar}
 \inferrule* [lab=behavior] {} {\meaningof{\langle a?b \rangle E} = \{ P \in \pi | P \equiv Q | u?(y)P', \\ \and \\\\ \and \\ \;\;\; u \in \meaningof{a}, \forall z.P'\{z/y\} \in \meaningof{E\{z/b\}}\}, \and \\ \meaningof{a!E} = \{ P \in \pi | P \equiv Q | x!\langle P' \rangle, x \in \meaningof{a} P' \in \meaningof{E}\} }
\end{mathpar}

\begin{mathpar}
 \inferrule* [lab=nominal] {} {\meaningof{\quotep{E}} = \{ \quotep{P} \in \quotep{\pi} | P \in \meaningof{E} \}, \and \meaningof{\quotep{P}} = \{ \quotep{Q} \in \quotep{\pi} | P \equiv Q \} \and \\ \meaningof{@\quotep{E}} = \{ P \in \pi | P \equiv @x, x \in \meaningof{E} \}}
\end{mathpar}

\begin{eqnarray*}
  \\
  \meaningof{-} : TS \to ST
\end{eqnarray*}

\begin{eqnarray*}
  \\
  L : TS \to ST
\end{eqnarray*}

\begin{eqnarray*}
  \\
  P \models E \iff P \in \meaningof{E}
\end{eqnarray*}

\begin{eqnarray*}
  P \approx_{L} Q \iff \forall E \in L. P \models E \iff Q \models E
\end{eqnarray*}

\begin{eqnarray*}
  P \approx_{K} Q
\end{eqnarray*}

\begin{eqnarray*}
  P \approx Q
\end{eqnarray*}

$\approx_{K} = \approx = \approx_{L}$

\subsubsection{Contextual duality}

Note that contexts extend the quotation operation to a family of
operations from processes to names. Given a context, $M$, we can
define a \emph{nominal context}, $\quotep{M}$ by $\quotep{M}[P] :=
\quotep{M[P]}$. To foreshadow what is to come we observe that these
operations enjoy a duality with processes very much like the duality
between vectors and maps from vectors to scalars.

Further, because the calculus is essentially higher-order, we have a
correspondence between contexts and processes. More specifically,
given a name $x$ and a context $M$ we can construct $M^{*}_{x}$ such
that 

\begin{mathpar}
  M^{*}_{x} | \lift{x}{P} \red M[P]
\end{mathpar}

namely,

\begin{mathpar}
  M^{*}_{x} := x?(u).M[\dropn{u}]
\end{mathpar}

The dependence of $M^{*}_{x}$ on a name makes it an abstraction, 

\begin{mathpar}
  M^{*} := (x)x?(u).M[\dropn{u}]
\end{mathpar}

\subsection{Additional notation}

It will sometimes be convenient to denote the process a name
quotes. We already have the notation $x = \quotep{P}$, but it will be
convenient to introduce an alternate notation, $\procn{x}$, when we
want to emphasize the connection to the use of the name. Note that, by
virtue of name equivalence, $\quotep{\procn{x}} \nameeq x$; so, the
notation is consistent with previous definitions.

Further, because names have structure it is possible to effect
substitutions on the basis of that structure. This means we need to
upgrade our notation for substitutions, which we accomplish by
adapting comprehension notation. Thus,

\begin{mathpar}
  P\{ y / x : x \in S \}
\end{mathpar}

is interpreted to mean the process derived from P by replacing (in a
capture-avoiding manner) each occurrence of $x$ in $S$ by $y$. For example,

\begin{mathpar}
  P\{ \quotep{\procn{x}|\procn{x}} / x : x \in \freenames{P} \}
\end{mathpar}

will replace each (occurrence) of a free name $x$ in $P$ by
$\quotep{\procn{x}|\procn{x}}$.

Also, we will avail ourselves of the notation $x^{L}$ and $x^{R}$ to
denote injections of a name into disjoint copies of the name
space. There are numerous ways to accomplish this. One example can be
found in \cite{MeredithR05}. This notation overloads to vectors of
names: $\vec{x}^{\pi} := (x_{i}^{\pi} \; : \; 0 \leq i < |\vec{x}| )$ where $\pi \in \{L,R\}$.

We also use $P^{\Box} := P|\Box$.

In \cite{MeredithR05} an interpretation of the new operator is
given. It turns out that there are several possible interpretations
all enjoying the requisite algebraic properties of the operator (see
\cite{milner91polyadicpi}). We will therefore make liberal use of
$(\nu\; \vec{x})P$.

% subsection the_syntax_and_semantics_of_the_notation_system (end)   

\input{qm2pi.qmops} 

\input{qm2pi.sterngerlach} 

\input{qm2pi.metric} 

% section concurrent_process_calculi (end)

%\input{qm2pi.proofsketch}

% section proof sketch (end)

%\input{qm2pi.slviaknots} 

% section spatial logic via knots (end)

\input{qm2pi.conclusion}

% section conclusion (end)

%\input{qm2pi.dtcodes} 

% section wiring algorithm (end)

\input{qm2pi.ack} 

% section acknowledgments (end)

\newpage


\bibliographystyle{plain}   
\bibliography{../../biblios/main.bib}

\input{qm2pi.rhodetails}

\end{document}

 

% section concurrent_process_calculi (end)

%\documentclass[12pt]{llncs}
%\documentclass{jktr}

\usepackage[pdftex]{hyperref}                   
\usepackage {listings}
\usepackage {mathpartir}
\usepackage{bcprules}
%\usepackage{listings}
                       
\usepackage{graphicx} 
%\usepackage[margins=2.5cm,nohead,nofoot]{geometry}
%\usepackage{geometry}
\usepackage{amsfonts}
\usepackage{amstext}
\usepackage{latexsym}
\usepackage{amssymb}
\usepackage{color}


%\include{myPreamble}
\include{qm2pi.local} 

%\ifpdf
%\usepackage[pdftex]{graphicx}
%\else
%\usepackage{graphicx}
%\fi

 % \ifpdf
%  \usepackage{pdfsync}
%  \if


%\title{Brief Article}
%\author{David F. Snyder}
%\author{L.G. Meredith}

%\address{Dept. of Math., Texas State University--San Marcos, San Marcos, TX 78666}
       
\pagestyle{empty}


\begin{document}

\lstset{language=[Objective]Caml,frame=shadowbox}

\input{qm2pi.front}

% section front matter (end)

\input{qm2pi.intro} 
 
% section introduction (end)

% \input{qm2pi.knotations} 

% section notation (end)

\input{qm2pi.process.calculi} 

% section concurrent_process_calculi_and_spatial_logics_ (end)
    
%\input{qm2pi.knots2pi} 

%\input{qm2pi.trefoil} 

%\input{qm2pi.mainthm} 

% subsection basic_interpretation (end)

%\input{qm2pi.rho.presentation} 
\subsection{The syntax and semantics of the notation system}\label{sub:the_syntax_and_semantics_of_the_notation_system} % (fold)

We now summarize a technical presentation of the calculus that
embodies our theory of dynamics. The typical presentation of such a
calculus follows the style of giving generators and relations on
them. The grammar, below, describing term constructors, freely
generates the set of processes, $\Proc$. This set is then quotiented
by a relation known as structural congruence and it is over this set
that the notion of dynamics is expressed. This presentation is
essentially that of \cite{MeredithR05} with the addition of
polyadicity and summation. For readability we have relegated some of
the technical subtleties to an appendix.

\subsubsection{Process grammar}\label{subsub:process_grammar}

\begin{mathpar}
  \inferrule* [lab=synchronization] {} {{M} \bc \pzero \;|\; x?F \;|\; x!C }
  \and
  \inferrule* [lab=abstraction] {} {{F} \bc (x)P}
  \and
  \inferrule* [lab=concretion] {} {{C} \bc \langle Q \rangle}
  \and
  \inferrule* [lab=process] {} {{P,Q} \bc M \;| \;P|Q \;|\; @{x}}
  \and
  \inferrule* [lab=name] {} {{x} \bc \quotep{P}}
\end{mathpar} 

Note that $\vec{x}$ (resp. $\vec{P}$) denotes a vector of names
(resp. processes) of length $|\vec{x}|$ (resp. $|\vec{P}|$). We adopt
the following useful abbreviations.

\begin{mathpar}
   x?(\vec{y}).P := x.(\vec{y})P \and  x\clift{\vec{P}} := x.\clift{\vec{P}}
   \and x!(y) := \lift{x}{\dropn{y}}
   \and \Pi_{i=0}^{n-1}P_i := P_0 | \ldots | P_{n-1}
\end{mathpar}

\subsubsection{Structural congruence}

\paragraph{Free and bound names and alpha-equivalence.} At the
core of structural equivalence is alpha-equivalence which identifies
process that are the same up to a change of variable. Formally, we
recognize the distinction between free and bound names. The free names
of a process, $\freenames{P}$, may be calculated recursively as
follows:

\begin{mathpar}
\freenames{\pzero} := \emptyset
  \and \\
  \freenames{x?(y).P} := \{ x \} \cup (\freenames{P} \setminus \{ y \})
  \and 
  \freenames{x!\langle P \rangle} := \{ x \} \cup \{ P \} 
  \and \\
  \freenames{P|Q} := \freenames{P} \cup \freenames{Q}
  \and \\
  \freenames{@{x}} := \{ x \}
\end{mathpar}

$\pi$
$\quotep{\pi}$

$\freenames{-} : \pi \to \mathcal{P}(\quotep{\pi})$

\begin{eqnarray*}
  \freenames{\pzero} & := & \emptyset \\
  \freenames{x?(y).P} & := & \{ x \} \cup (\freenames{P} \setminus \{ y \}) \\
  \freenames{x!\langle P \rangle} & := & \{ x \} \cup \{ P \} \\
  \freenames{P|Q} & := & \freenames{P} \cup \freenames{Q} \\
  \freenames{\dropn{x}} & := & \{ x \}
\end{eqnarray*}

The bound names of a process, $\boundnames{P}$, are those names occurring in $P$
that are not free. For example, in $x?(y).0$, the name $x$ is free, while $y$ is bound.

\begin{mathpar}
  \inferrule* [lab=monoidal-laws] {} { P|Q \equiv Q|P \and P|0 \equiv P \and P|(Q|R) \equiv (P|Q)|R }
\end{mathpar}

\begin{mathpar}
  \inferrule* [lab=alpha-equivalence] {} { (x)P \equiv (y)P\{y/x\} \and y \not\in \freenames{P} }
\end{mathpar}

\begin{definition}
Then two processes, $P,Q$, are alpha-equivalent if $P = Q\{\vec{y}/\vec{x}\}$ for
some $\vec{x} \in \boundnames{Q},\vec{y} \in \boundnames{P}$, where $Q\{\vec{y}/\vec{x}\}$
denotes the capture-avoiding substitution of $\vec{y}$ for $\vec{x}$ in $Q$.
\end{definition}

\begin{definition}
  The {\em structural congruence} \cite{SangiorgiWalker} , $\equiv$,
  between processes is the least congruence containing
  alpha-equivalence, satisfying the abelian monoid laws
  (associativity, commutativity and $\pzero$ as identity) for parallel
  composition $|$ and for summation $+$.
\end{definition}

\subsection{Name equivalence}

We take name equivalence, written $\nameeq$, to be the smallest
equivalence relation generated by the following rules.

\begin{mathpar}
\inferrule*[lab=Quote-drop]
{ }
{ \quotep{@{x}} \nameeq x }

\inferrule*[lab=Struct-equiv]
{ P \scong Q }
{ \quotep{P} \nameeq \quotep{Q} }
\end{mathpar}

The astute reader will have noticed that the mutual recursion of names
and processes imposes a mutual recursion on alpha-equivalence and
structural equivalence via name-equivalence. Fortunately, all of this
works out pleasantly and we may calculate in the natural way, free of
concern. The reader interested in the details is referred to the
appendix \ref{appendix:rho_details}.

\subsection{Substitution}

We use $\Proc$ for the set of processes, $\QProc$ for the set of
names, and $\id{\{}\vec{y} / \vec{x} \id{\}}$ to denote partial maps,
$s : \QProc \rightarrow \QProc$. A map, $s$ lifts, uniquely, to a map
on process terms, $\widehat{s} : \Proc \rightarrow \Proc$ by the
following equations.

\begin{mathpar}
  (0) \psubstp{Q}{P} := 0 \\
  (R \juxtap S) \psubstp{Q}{P}
  :=    
  (R)\psubstp{Q}{P} \juxtap (S) \psubstp{Q}{P} \\
  (x?(y).R) \psubstp{Q}{P}    
  :=    
  (x)\substp{Q}{P} (z)\concat( (R \psubstn{z}{y}) \psubstp{Q}{P} ) \\
  (\lift{x}{R}) \psubstp{Q}{P}  
  :=
  \lift{(x)\substp{Q}{P}}{ R \psubstp{Q}{P} } \\
%   (\dropn{x})  \psubstp{Q}{P}       
%   := 
%   \left\{ 
%     \begin{array}{ccc} 
%       \dropn{\quotep{Q}} & & x \nameeq \quotep{P} \\
%       \dropn{x} & & otherwise \\
%     \end{array}
%   \right. 
  (\dropn{x})  \psubstp{Q}{P}       
  := 
  \left\{ 
    \begin{array}{ccc} 
      Q & & x \nameeq \quotep{P} \\
      \dropn{x} & & otherwise \\
    \end{array}
  \right.
\end{mathpar}
 

where

\begin{eqnarray}
  (x)\id{\{} \lpquote Q \rpquote / \lpquote P \rpquote \id{\}}            = 
  \left\{ 
    \begin{array}{ccc}
      \lpquote Q \rpquote & & x \nameeq \lpquote P \rpquote \\
      x & & otherwise \\
    \end{array}
  \right. \nonumber
\end{eqnarray}

and $z$ is chosen distinct from $\quotep{P}$, $\quotep{Q}$, the free
names in $Q$, and all the names in $R$. Our $\alpha$-equivalence will
be built in the standard way from this substitution.

\begin{remark}\label{rem:no_self_referential_names}
  One consequence of these definitions is that $\forall P. \quotep{P}
  \not\in \freenames{P}$.
\end{remark}

\subsection{ Dynamic quote: an example }

Anticipating something of what's to come, consider applying the
substitution, $\widehat{\id{\{}u / z \id{\}}}$, to the following pair
of processes, $\lift{w}{y!(z)}$ and $w[ \lpquote y!(z) \rpquote ]$.

\begin{eqnarray}
	\lift{w}{y!(z)}\widehat{\id{\{}u / z \id{\}}}
		& = &
		\lift{w}{y!(u)} \nonumber\\
	w[ \lpquote y!(z) \rpquote ] \widehat{ \id{\{}u / z \id{\}} }
		& = &
		w[ \lpquote y!(z) \rpquote ] \nonumber
\end{eqnarray}

Because the body of the process between quotes is impervious to
substitution, we get radically different answers. In fact, by
examining the first process in an input context,
e.g. $x?(z).\lift{w}{y!(z)}$, we see that the process under the lift
operator may be shaped by prefixed inputs binding a name inside it. In
this sense, the lift operator will be seen as a way to dynamically
construct processes before reifying them as names.

Finally equipped with these standard features we can present the
dynamics of the calculus.

\subsubsection{Operational semantics} 

Finally, we introduce the computational dynamics. What marks these
algebras as distinct from other more traditionally studied algebraic
structures, e.g. vector spaces or polynomial rings, is the manner in
which dynamics is captured. In traditional structures, dynamics is typically
expressed through morphisms between such structures, as in linear maps
between vector spaces or morphisms between rings. In algebras
associated with the semantics of computation, the dynamics is
expressed as part of the algebraic structure itself, through a
reduction reduction relation typically denoted by $\red$. Below, we
give a recursive presentation of this relation for the calculus used
in the encoding.

$\red \subseteq \pi \times \pi$
$\red : \pi \to \mathcal{P}(\pi)$

\begin{mathpar}
  \inferrule* [lab=Comm] { \textsf{match}( x_{src}, x_{trgt} ) } { x_{trgt}?(y)P \; | \; x_{src}!\langle {Q} \rangle \red P\{\quotep{Q}/y}\} }
  \and \\
  \inferrule* [lab=Par] {{P} \red {P}'} {{{P} | {Q}} \red {{P}' | {Q}}}
  \and
  \inferrule* [lab=Equiv]{{{P} \scong {P}'} \andalso {{P}' \red {Q}'} \andalso {{Q}' \scong {Q}}}{{P} \red {Q}}
\end{mathpar}

\begin{eqnarray*}
  match_{\equiv} (\quotep{P},\quotep{Q}) & := & P \equiv Q \\
  match_{\dagger}(\quotep{P},\quotep{Q}) & := & \forall R. P|Q \red^{*} R => R \red^{*} 0 \\
  match_{K}(\quotep{P},\quotep{Q}) & := & K \mbox{ for some context } K
\end{eqnarray*}

$u?(x)P | u!\langle Q \rangle \red P\{\quotep{Q}/x\}$

%We write $\wred$ for $\red^*$, and $P\red$ if $\exists Q $ such that $ P \red Q$.
We write $P\red$ if $\exists Q $ such that $ P \red Q$ and $P\not\red$, otherwise.

\section{Replication}

As mentioned before, it is known that replication (and hence
recursion) can be implemented in a higher-order process algebra
\cite{SangiorgiWalker}. As our first example of calculation with the
machinery thus far presented we give the construction explicitly in
the {\rhoc}.

\begin{eqnarray}
	D_{x} & := & \prefix{x}{y}{(\binpar{\outputp{x}{y}}{@{y}})} \nonumber\\
	\bangp_{x}{P} & := & \binpar{{x}!\langle{\binpar{D_{x}}{P}}\rangle}{D_{x}} \nonumber
\end{eqnarray}

\begin{eqnarray}
	\bangp_{x}{P} & & \nonumber\\
	=
	& {x}!\langle{(\prefix{x}{y}{(\outputp{x}{y} | @{y})) | P}}\rangle 
	      | \prefix{x}{y}{(\outputp{x}{y} | @{y})} & \nonumber\\
	\red
	& (\outputp{x}{y} | @{y})\substn{\quotep{(\prefix{x}{y}{(@{y} | \outputp{x}{y})) | P}}}{y} & \nonumber\\
	=
	& \outputp{x}{\quotep{(\prefix{x}{y}{(\outputp{x}{y} | @{y})) | P}}}
	  | {(\prefix{x}{y}{(\outputp{x}{y} | @{y})) | P}} & \nonumber\\
	\red
	& \ldots & \nonumber\\
	\red^*
	& P | P | \ldots & \nonumber
\end{eqnarray}

Of course, this encoding, as an implementation, runs away, unfolding
$\bangp{P}$ eagerly. A lazier and more implementable replication
operator, restricted to input-guarded processes, may be obtained as follows.

\begin{eqnarray}
\bangp{\prefix{u}{v}{P}} 
	:= 
	\binpar{\lift{x}{\prefix{u}{v}{(\binpar{D(x)}{P})}}}{D(x)} \nonumber
\end{eqnarray}

\begin{remark}
  Note that the lazier definition still does not deal with summation
  or mixed summation (i.e. sums over input and output). The reader is
  invited to construct definitions of replication that deal with these
  features. 

  Further, the definitions are parameterized in a name, $x$. Can you,
  gentle reader, make a definition that eliminates this parameter and
  guarantees no accidental interaction between the replication
  machinery and the process being replicated -- i.e. no accidental
  sharing of names used by the process to get its work done and the
  name(s) used by the replication to effect copying. This latter
  revision of the definition of replication is crucial to obtaining
  the expected identity $!!P \sim !P$.
\end{remark}

\begin{remark}\label{rem:paradoxical_combinator}
  The reader familiar with the lambda calculus will have noticed the
  similarity between $D$ and the paradoxical combinator.

  [Ed. note: the existence of this seems to suggest we have to be more
  restrictive on the set of processes and names we admit if we are to
  support no-cloning.]
\end{remark}

\subsubsection{Bisimulation}

The computational dynamics gives rise to another kind of equivalence,
the equivalence of computational behavior. As previously mentioned
this is typically captured \emph{via} some form of bisimulation.

% The notion we use in this paper is weak barbed bisimulation
% \cite{milner91polyadicpi}.

The notion we use in this paper is derived from weak barbed
bisimulation \cite{milner91polyadicpi}. 

\begin{definition}
An \emph{observation relation}, $\downarrow_{\mathcal N}$, over a set
of names, $\mathcal N$, is the smallest relation satisfying the rules
below.

\infrule[Out-barb]{y \in {\mathcal N}, \; x \nameeq y}
		  {\outputp{x}{v} \downarrow_{\mathcal N} x}
\infrule[Par-barb]{\mbox{$P\downarrow_{\mathcal N} x$ or $Q\downarrow_{\mathcal N} x$}}
		  {\binpar{P}{Q} \downarrow_{\mathcal N} x}

We write $P \Downarrow_{\mathcal N} x$ if there is $Q$ such that 
$P \wred Q$ and $Q \downarrow_{\mathcal N} x$.
\end{definition}

\begin{definition}
%\label{def.bbisim}
An  ${\mathcal N}$-\emph{barbed bisimulation} over a set of names, ${\mathcal N}$, is a symmetric binary relation 
${\mathcal S}_{\mathcal N}$ between agents such that $P\rel{S}_{\mathcal N}Q$ implies:
\begin{enumerate}
\item If $P \red P'$ then $Q \wred Q'$ and $P'\rel{S}_{\mathcal N} Q'$.
\item If $P\downarrow_{\mathcal N} x$, then $Q\Downarrow_{\mathcal N} x$.
\end{enumerate}
$P$ is ${\mathcal N}$-barbed bisimilar to $Q$, written
$P \wbbisim_{\mathcal N} Q$, if $P \rel{S}_{\mathcal N} Q$ for some ${\mathcal N}$-barbed bisimulation ${\mathcal S}_{\mathcal N}$.
\end{definition}

$\mathcal{R} \subseteq \pi \times \pi$

$P \mathcal{R} Q => \forall P'. P \red P' \Rightarrow \exists Q'. Q \red Q', P' \mathcal{R} Q'$

$P \vdash x \Rightarrow Q \vdash x$

\begin{mathpar}
  \inferrule*[lab=Out-barb]{x \nameeq y}{{y}!\langle{Q}\rangle \vdash x}
  \and
  \inferrule*[lab=Par-barb]{\mbox{$P\vdash x$ or $Q\vdash x$}}{\binpar{P}{Q} \vdash x}
\end{mathpar}

\subsubsection{Contexts}

One of the principle advantages of computational calculi like the
$\pi$-calculus is a well-defined notion of context,
contextual-equivalence and a correlation between
contextual-equivalence and notions of bisimulation. The notion of
context allows the decomposition of a process into (sub-)process and
its syntactic environment, its context. Thus, a context may be
thought of as a process with a ``hole'' (written $\Box$) in it. The
application of a context $M$ to a process $P$, written $M[P]$, is
tantamount to filling the hole in $M$ with $P$. In this paper we do
not need the full weight of this theory, but do make use of the notion
of context in the proof the main theorem. 

\begin{mathpar}
  \inferrule* [lab=summation] {} {{M_{M},M_{N}} \bc \Box \;|\; x.M_{A} \;|\; M_{M}+M_{N}}
  \and
  \inferrule* [lab=agent] {} {{M_{A}} \bc (\vec{x})M_{P} \;| \; \clift{P_0,\ldots,M_{P},\ldots,P_N}}
  \and \\
  \inferrule* [lab=process] {} {{M_{P}} \bc M_{N} \;| \;P|M_{P} }
\end{mathpar} 

\begin{mathpar}
  \inferrule* [lab=sychronization] {} {M_{N} \bc \Box \;|\; x?M_{F} \;|\; x!M_{C}}
  \and
  \inferrule* [lab=abstraction] {} {{M_{F}} \bc (x)M_{P} }
  \and
  \inferrule* [lab=concretion] {} {{M_{C}} \bc \langle M_{P} \rangle }
  \and \\
  \inferrule* [lab=process] {} {{M_{P}} \bc M_{N} \;| \;P|M_{P} }
\end{mathpar}

\begin{definition}[contextual application] Given a context $M$, and
  process $P$, we define the \emph{contextual application}, $M[P] :=
  M\{P/\Box\}$. That is, the contextual application of M to P is the
  substitution of $P$ for $\Box$ in $M$.
\end{definition}

$\meaningof{-} : L \to \mathcal{P}(\pi)$

\begin{mathpar}
  \inferrule* [lab=collection] {} {\meaningof{true} = \pi, \and \meaningof{~E} = \pi \setminus \meaningof{E}, \and \meaningof{E_{1} \& E_{2}} = \meaningof{E_{1}} \cap \meaningof{E_{2}}}
\end{mathpar}

\begin{mathpar}
  \inferrule* [lab=structure] {} {\meaningof{0} = \{ P \in \pi | P \equiv 0 \}, \and \\ \meaningof{E_1 | E_2} = \{ P \in \pi | P \equiv P_{1} | P_{2}, P_{1} \in \meaningof{E_{1}}, P_{2} \in \meaningof{E_2}\} }
\end{mathpar}

\begin{mathpar}
 \inferrule* [lab=behavior] {} {\meaningof{\langle a?b \rangle E} = \{ P \in \pi | P \equiv Q | u?(y)P', \\ \and \\\\ \and \\ \;\;\; u \in \meaningof{a}, \forall z.P'\{z/y\} \in \meaningof{E\{z/b\}}\}, \and \\ \meaningof{a!E} = \{ P \in \pi | P \equiv Q | x!\langle P' \rangle, x \in \meaningof{a} P' \in \meaningof{E}\} }
\end{mathpar}

\begin{mathpar}
 \inferrule* [lab=nominal] {} {\meaningof{\quotep{E}} = \{ \quotep{P} \in \quotep{\pi} | P \in \meaningof{E} \}, \and \meaningof{\quotep{P}} = \{ \quotep{Q} \in \quotep{\pi} | P \equiv Q \} \and \\ \meaningof{@\quotep{E}} = \{ P \in \pi | P \equiv @x, x \in \meaningof{E} \}}
\end{mathpar}

\begin{eqnarray*}
  \\
  \meaningof{-} : TS \to ST
\end{eqnarray*}

\begin{eqnarray*}
  \\
  L : TS \to ST
\end{eqnarray*}

\begin{eqnarray*}
  \\
  P \models E \iff P \in \meaningof{E}
\end{eqnarray*}

\begin{eqnarray*}
  P \approx_{L} Q \iff \forall E \in L. P \models E \iff Q \models E
\end{eqnarray*}

\begin{eqnarray*}
  P \approx_{K} Q
\end{eqnarray*}

\begin{eqnarray*}
  P \approx Q
\end{eqnarray*}

$\approx_{K} = \approx = \approx_{L}$

\subsubsection{Contextual duality}

Note that contexts extend the quotation operation to a family of
operations from processes to names. Given a context, $M$, we can
define a \emph{nominal context}, $\quotep{M}$ by $\quotep{M}[P] :=
\quotep{M[P]}$. To foreshadow what is to come we observe that these
operations enjoy a duality with processes very much like the duality
between vectors and maps from vectors to scalars.

Further, because the calculus is essentially higher-order, we have a
correspondence between contexts and processes. More specifically,
given a name $x$ and a context $M$ we can construct $M^{*}_{x}$ such
that 

\begin{mathpar}
  M^{*}_{x} | \lift{x}{P} \red M[P]
\end{mathpar}

namely,

\begin{mathpar}
  M^{*}_{x} := x?(u).M[\dropn{u}]
\end{mathpar}

The dependence of $M^{*}_{x}$ on a name makes it an abstraction, 

\begin{mathpar}
  M^{*} := (x)x?(u).M[\dropn{u}]
\end{mathpar}

\subsection{Additional notation}

It will sometimes be convenient to denote the process a name
quotes. We already have the notation $x = \quotep{P}$, but it will be
convenient to introduce an alternate notation, $\procn{x}$, when we
want to emphasize the connection to the use of the name. Note that, by
virtue of name equivalence, $\quotep{\procn{x}} \nameeq x$; so, the
notation is consistent with previous definitions.

Further, because names have structure it is possible to effect
substitutions on the basis of that structure. This means we need to
upgrade our notation for substitutions, which we accomplish by
adapting comprehension notation. Thus,

\begin{mathpar}
  P\{ y / x : x \in S \}
\end{mathpar}

is interpreted to mean the process derived from P by replacing (in a
capture-avoiding manner) each occurrence of $x$ in $S$ by $y$. For example,

\begin{mathpar}
  P\{ \quotep{\procn{x}|\procn{x}} / x : x \in \freenames{P} \}
\end{mathpar}

will replace each (occurrence) of a free name $x$ in $P$ by
$\quotep{\procn{x}|\procn{x}}$.

Also, we will avail ourselves of the notation $x^{L}$ and $x^{R}$ to
denote injections of a name into disjoint copies of the name
space. There are numerous ways to accomplish this. One example can be
found in \cite{MeredithR05}. This notation overloads to vectors of
names: $\vec{x}^{\pi} := (x_{i}^{\pi} \; : \; 0 \leq i < |\vec{x}| )$ where $\pi \in \{L,R\}$.

We also use $P^{\Box} := P|\Box$.

In \cite{MeredithR05} an interpretation of the new operator is
given. It turns out that there are several possible interpretations
all enjoying the requisite algebraic properties of the operator (see
\cite{milner91polyadicpi}). We will therefore make liberal use of
$(\nu\; \vec{x})P$.

% subsection the_syntax_and_semantics_of_the_notation_system (end)   

\input{qm2pi.qmops} 

\input{qm2pi.sterngerlach} 

\input{qm2pi.metric} 

% section concurrent_process_calculi (end)

%\input{qm2pi.proofsketch}

% section proof sketch (end)

%\input{qm2pi.slviaknots} 

% section spatial logic via knots (end)

\input{qm2pi.conclusion}

% section conclusion (end)

%\input{qm2pi.dtcodes} 

% section wiring algorithm (end)

\input{qm2pi.ack} 

% section acknowledgments (end)

\newpage


\bibliographystyle{plain}   
\bibliography{../../biblios/main.bib}

\input{qm2pi.rhodetails}

\end{document}



% section proof sketch (end)

%\section{Unlikely characters: spatial logic for
  knots}\label{sub:characteristic_formulae} % (fold)

Associated to the mobile process calculi are a family of logics known
as the Hennessy-Milner logics. These logics typically enjoy a
semantics interpreting formulae as sets of processes that when
factored through the encoding outlined above allows an identification
of classes of knots with logical formulae. In the context of this
encoding the sub-family known as the spatial logics \cite{CairesC03}
\cite{CairesC04} \cite{Caires04} are of particular interest providing
several important features for expressing and reasoning about
properties (i.e. classes) of knots. We hint here at how this may be done.

%\begin{description}
%\item [structural connectives] 
\subsubsection{Structural connectives} The spatial logics enjoy
structural connectives corresponding, at the logical level, to the
parallel composition ($P | Q$) and new name ($(\nu \; x)P$)
connectives for processes. As illustrated in the examples below, these
connectives are extremely expressive given the shape of our encoding.
%\item [decideable satisfaction]

\subsubsection{Decideable satisfaction}
In \cite{Caires04} the satisfaction relation is shown to be decideable
for a rich class of processes. It further turns out that the image of
the our encoding is a proper subset of that class. This result
provides the basis for an algorithm by which to search for knots
enjoying a given property.
%\item [characteristic formulae]

\subsubsection{Characteristic formulae}
In the same paper \cite{Caires04} , Caires presents a means of calculating
characteristic formulae, selecting equivalence classes of processes
up to a pre--specified depth limit on the support set of names. Composed with our
encoding, this characteristic formula can be used to select
characteristic formulae for knots.
%\end{description}

\subsubsection{Spatial logic formulae}

The grammar below (segmented for comprehension) summarizes the syntax
of spatial logic formulae. We employ illustrative examples in the
sequel to provide an intuitive understanding of their meaning
referring the reader to \cite{Caires04} for a more detailed explication
of the semantics.

\begin{mathpar}
  \inferrule* [lab=boolean] {} {{A,B} \bc T \;|\; \neg A \;|\; A \wedge B \;|\; \eta = \eta'}
  \and
  \inferrule* [lab=spatial] {} {|\; \pzero \;|\; A | B \;|\; x \text{\textregistered} A \;|\; \forall x . A \;|\;  H x . A}
  \and
  \inferrule* [lab=behavioral] {} {|\; \alpha . A}
  \and 
  \inferrule* [lab=recursion] {} {|\; X(\vec{u}) \;|\; \mu X(\vec{u}) . A}
  \and
  \inferrule* [lab=action] {} {\alpha \bc \langle x?(\vec{y}) \rangle \;|\; \langle x!(\vec{y}) \rangle \;|\; \langle \tau \rangle}
  \and 
  \inferrule* [lab=name] {} {\eta \bc x \;|\; \tau}
\end{mathpar} 

% subsection characteristic_formulae (end)   	 

\subsection{Example formulae}\label{sub:example_formulae_} % (fold)

\subsubsection{Crossing as formula.}
% 
% \begin{align*}
%   \frac{d}{dx} \sin x &= \cos x 
%   & \frac{d}{dx} e^x &= e^x \\
%   \frac{d}{dx} \cos x &= - \sin x 
%   & \frac{d}{dx} \log x &= \frac{1}{x} \\
% \end{align*} 

\begin{align*}
 \mu C(x_{0},x_{1},y_{0},y_{1},u).&(\langle x_{0}?(z) \rangle(\langle u! \rangle\langle y_{1}!z \rangle C(x_{0},x_{1},y_{0},y_{1},u)) & \\
  & \wedge \langle y_{1}?(z) \rangle (\langle u! \rangle \langle x_{0}!z \rangle C(x_{0},x_{1},y_{0},y_{1},u)) & \\
  & \wedge \langle x_{1}?(z) \rangle (\langle u? \rangle \langle y_{0}!z \rangle C(x_{0},x_{1},y_{0},y_{1},u)) & \\
  & \wedge \langle y_{0}?(z) \rangle (\langle u? \rangle \langle x_{1}!z \rangle C(x_{0},x_{1},y_{0},y_{1},u))) &
\end{align*}

The lexicographical similarity between the shape of this formulae and
the shape of definition of the process representing a crossing reveals
the intuitive meaning of this formulae. It describes the capabilities
of a process that has the right to represent a crossing. For example
it picks out processes that may perform an input on the port $x_0$ in
its initial menu of capabilities. What differentiates the formula
from the process, however, is that the crossing process is the
smallest candidate to satisfy the formula. Infinitely many other
processes -- with internal behavior hidden behind this interface, so
to speak -- also satisfy this formula. Even this simple formula,
then, can be seen to open a new view onto knots, providing a
computational interpretation of \emph{virtual} knots.

Note that this formula is derived by hand. A similar formula can be
derived by employing Caires' calculation of characteristic formula
\cite{Caires04} to the process representing a crossing. In light of
this discussion, we let
$\meaningof{C}_{\phi}(x0,x1,y0,y1,u)$ denote a formula specifying the
dynamics we wish to capture of a crossing. To guarantee we preserve
the shape of the interface and minimal semantics we demand that
$\meaningof{C}_{\phi}(x0,x1,y0,y1,u) \Rightarrow
\textbf{C}(x0,x1,y0,y1,u)$ where $\textbf{C}(x0,x1,y0,y1,u)$ denotes
the formula above.
                            
\subsubsection{Crossing number constraints.}
The moral content of the context lemma (Lemma \ref{context}) is that the notion of
``locality'' in the Reidemeister moves is effectively captured by the
parallel composition operator of the process calculus. This intuition
extends through the logic. Given a formula,
$\meaningof{C}_{\phi}(x0,x1,y0,y1,u)$, we can use the structural
connectives to specify constraints on crossing numbers, such as at
least $n$ crossings, or exactly $n$ crossings.
\begin{mathpar}
  \inferrule* [lab=at-least-n] {} { K^{\geq n}_{\phi}(\vec{xs},\vec{ys}) := \Pi_{i=0}^{n-1} Hu . \meaningof{C}_{\phi}(xs_i,ys_i,u) | T }
  \and 
  \inferrule* [lab=exactly-n] {} { K^{= n}_{\phi}(\vec{xs},\vec{ys}) := \Pi_{i=0}^{n-1} Hu . \meaningof{C}_{\phi}(xs_i,ys_i,u) | \neg (\forall x_0,y_0,x_1,y_1,u . \meaningof{C}_{\phi}(x_0,y_0,x_1,y_1,u) | T) }
\end{mathpar}

To round out this section, recall that the encoding of an $n$-crossing
knot decomposes into a parallel composition of $n$ \emph{copies} of a
crossing process together with a wiring harness. To specify different
knot classes with the same crossing number amounts to specifying
logical constraints on the wiring harness. In the interest of space,
we defer examples to a forthcoming paper. Suffice it to say that both
the conditions ``alternating knot'' and ``contains the tangle
corresponding to 5/3'' are expressible. For example, it is possible to
calculate the characteristic formula of a process corresponding to the
tangle 5/3 and conjoin it into the classifying formula via the
composition connective of the logic.

Finally, we wish to observe that it is entirely within reason to
contemplate a more domain-specific version of spatial logic tailored
to the shape of processes in the image of the encoding. Such a
domain-specific logic would have a better claim to the title formal
language of knot properties.

% subsection example_formulae_ (end)

% section knots_as_processes (end) 

% section spatial logic via knots (end)

\section{Conclusions and future work}

\paragraph{Testing physical space}
You, gentle reader, may wonder why of all the theorems to be proved
given this set up we pick the one above. In some sense it's hardly
central to quantum mechanics. We see it as central in the sense that
it firmly establishes a notion of physical space arising from a notion
of the equivalence of behavior. Relating bisimulation to a metric is a
big step forward, but one is faced with interpreting the relationship
of that metric space to something more physical. Quantum mechanical
notions of ``physical'' space are still far from intuitive, but by
relating this idea of distance as testing to calculations that predict
physical circumstances we are making a not insignificant step forward
toward an understanding of the physical space we inhabit as
essentially dynamic.

\paragraph{Effectivity and simulation}
One of the observations we have yet to make is that the entire program
spelled out here is effective. We have built various interpreters for
the reflective calculus at work in this interpretation. In principle,
then, we can simulate quantum mechanics on a computer. The place where
the simulation may lose fidelity is the infinitely branching summation
for the annihilator.

In this connection i also want to point out that the evaluation style
calculation of the inner product puts the non-determinism of the
summation right at the heart of measurement. This suggests that
Milner's original reduction-based formulation of the dynamics of his
calculi in terms of sums was not just notationally suggestive of a
notion of measure-and-continue but captured some significant part of
the physics.

\paragraph{Quantum continuations}
In light of this last observation i want to point out that the
predominant account of quantum mechanics is missing a key aspect of a
truly compositional story of the physical situation. In a real lab,
when a measurement is made the observation can be made to feed into
another device that then makes another measurement conditioned on the
results of the first. This means that after the superposition was
collapsed the entire experimental set up remained in
superposition. While QM offers a means of writing this down it doesn't
quite line up well with the well-trodden formulation of computation
and continuation that we see so succinctly expressed in Milner's
calculi. This suggests that there might be advantages to this account
of dynamics waiting to be explored.

\paragraph{Quantum logic}
In this connection, we also note that by virtue of having the
Hennessy-Milner construction, we can pull the construction through the
interpretation of QM. This gives us a natural candidate for a quantum
logic that enjoys an extremely tight connection with it's domain of
interpretation, making the construction much less ad hoc (rather it is
the image of functor!).

\paragraph{Quantum probabiity}
i have questions about the basis of the interpretation of inner
product as probability amplitude. In particular, using which
axiomatization of probability theory does the notion of probability
amplitude earn the right to be so dubbed? In other words, where is the
proof that the operation for calculating a probability amplitude (and
then squaring) satisfies the axioms of what it means to calculate a
probability? Even if such a proof exists (i have yet to find it in the
literature), i wonder if it might not be possible to turn things on
their heads. Can we view the calculation of the probability amplitude
as an axiomatization of probability? If so, then the definition we
give for calculating probability amplitude may provide the basis for
an \emph{effective} theory of probability.

\paragraph{Quantum vs ``biological'' information}
Finally, i want to conclude with a more philosophical observation. At
a recent workshop in which QM was a predominant topic i noticed
something about quantum information. The speaker was giving a riveting
discussion of axiomatic QM and showing how properties of ``no
cloning'' and ``no deleting'' emerged as consequences of the
axiomatization. Theorems of this form are necessary to give us a sense
of confidence that our axioms characterize the physical theory. What
struck me, though, was that if quantum information is neither erasable
nor replicable it is markedly different from \emph{life}. Two of the
things we know about life is that

\begin{itemize}
  \item it ends;
  \item to gain some measure of persistence, to transcend it's
    finitude it is imminently copyable.
\end{itemize}

Both of these qualities are summarized succinctly in the aphorism: all
flesh is grass. For me these two kinds of ``information'' -- call them
quantum and biological -- are end points on a spectrum of strategies
for persistence. At one end, we have those curious entities that enjoy
uniqueness and permanence; at the other, we have those who in the face
of a certain end and an uncertain present make a go of passing
something on. To me one of the more remarkable aspects of the latter
strategy is that in the presence of noise (and certain features of
copying) we get a kind of dynamism, a chance for improvement against a
given persistent condition.

% subsection other_calculi_other_bisimulations_and_geometry_as_behavior (end)




% section conclusion (end)

%\documentclass[12pt]{llncs}
%\documentclass{jktr}

\usepackage[pdftex]{hyperref}                   
\usepackage {listings}
\usepackage {mathpartir}
\usepackage{bcprules}
%\usepackage{listings}
                       
\usepackage{graphicx} 
%\usepackage[margins=2.5cm,nohead,nofoot]{geometry}
%\usepackage{geometry}
\usepackage{amsfonts}
\usepackage{amstext}
\usepackage{latexsym}
\usepackage{amssymb}
\usepackage{color}


%\include{myPreamble}
\include{qm2pi.local} 

%\ifpdf
%\usepackage[pdftex]{graphicx}
%\else
%\usepackage{graphicx}
%\fi

 % \ifpdf
%  \usepackage{pdfsync}
%  \if


%\title{Brief Article}
%\author{David F. Snyder}
%\author{L.G. Meredith}

%\address{Dept. of Math., Texas State University--San Marcos, San Marcos, TX 78666}
       
\pagestyle{empty}


\begin{document}

\lstset{language=[Objective]Caml,frame=shadowbox}

\input{qm2pi.front}

% section front matter (end)

\input{qm2pi.intro} 
 
% section introduction (end)

% \input{qm2pi.knotations} 

% section notation (end)

\input{qm2pi.process.calculi} 

% section concurrent_process_calculi_and_spatial_logics_ (end)
    
%\input{qm2pi.knots2pi} 

%\input{qm2pi.trefoil} 

%\input{qm2pi.mainthm} 

% subsection basic_interpretation (end)

%\input{qm2pi.rho.presentation} 
\subsection{The syntax and semantics of the notation system}\label{sub:the_syntax_and_semantics_of_the_notation_system} % (fold)

We now summarize a technical presentation of the calculus that
embodies our theory of dynamics. The typical presentation of such a
calculus follows the style of giving generators and relations on
them. The grammar, below, describing term constructors, freely
generates the set of processes, $\Proc$. This set is then quotiented
by a relation known as structural congruence and it is over this set
that the notion of dynamics is expressed. This presentation is
essentially that of \cite{MeredithR05} with the addition of
polyadicity and summation. For readability we have relegated some of
the technical subtleties to an appendix.

\subsubsection{Process grammar}\label{subsub:process_grammar}

\begin{mathpar}
  \inferrule* [lab=synchronization] {} {{M} \bc \pzero \;|\; x?F \;|\; x!C }
  \and
  \inferrule* [lab=abstraction] {} {{F} \bc (x)P}
  \and
  \inferrule* [lab=concretion] {} {{C} \bc \langle Q \rangle}
  \and
  \inferrule* [lab=process] {} {{P,Q} \bc M \;| \;P|Q \;|\; @{x}}
  \and
  \inferrule* [lab=name] {} {{x} \bc \quotep{P}}
\end{mathpar} 

Note that $\vec{x}$ (resp. $\vec{P}$) denotes a vector of names
(resp. processes) of length $|\vec{x}|$ (resp. $|\vec{P}|$). We adopt
the following useful abbreviations.

\begin{mathpar}
   x?(\vec{y}).P := x.(\vec{y})P \and  x\clift{\vec{P}} := x.\clift{\vec{P}}
   \and x!(y) := \lift{x}{\dropn{y}}
   \and \Pi_{i=0}^{n-1}P_i := P_0 | \ldots | P_{n-1}
\end{mathpar}

\subsubsection{Structural congruence}

\paragraph{Free and bound names and alpha-equivalence.} At the
core of structural equivalence is alpha-equivalence which identifies
process that are the same up to a change of variable. Formally, we
recognize the distinction between free and bound names. The free names
of a process, $\freenames{P}$, may be calculated recursively as
follows:

\begin{mathpar}
\freenames{\pzero} := \emptyset
  \and \\
  \freenames{x?(y).P} := \{ x \} \cup (\freenames{P} \setminus \{ y \})
  \and 
  \freenames{x!\langle P \rangle} := \{ x \} \cup \{ P \} 
  \and \\
  \freenames{P|Q} := \freenames{P} \cup \freenames{Q}
  \and \\
  \freenames{@{x}} := \{ x \}
\end{mathpar}

$\pi$
$\quotep{\pi}$

$\freenames{-} : \pi \to \mathcal{P}(\quotep{\pi})$

\begin{eqnarray*}
  \freenames{\pzero} & := & \emptyset \\
  \freenames{x?(y).P} & := & \{ x \} \cup (\freenames{P} \setminus \{ y \}) \\
  \freenames{x!\langle P \rangle} & := & \{ x \} \cup \{ P \} \\
  \freenames{P|Q} & := & \freenames{P} \cup \freenames{Q} \\
  \freenames{\dropn{x}} & := & \{ x \}
\end{eqnarray*}

The bound names of a process, $\boundnames{P}$, are those names occurring in $P$
that are not free. For example, in $x?(y).0$, the name $x$ is free, while $y$ is bound.

\begin{mathpar}
  \inferrule* [lab=monoidal-laws] {} { P|Q \equiv Q|P \and P|0 \equiv P \and P|(Q|R) \equiv (P|Q)|R }
\end{mathpar}

\begin{mathpar}
  \inferrule* [lab=alpha-equivalence] {} { (x)P \equiv (y)P\{y/x\} \and y \not\in \freenames{P} }
\end{mathpar}

\begin{definition}
Then two processes, $P,Q$, are alpha-equivalent if $P = Q\{\vec{y}/\vec{x}\}$ for
some $\vec{x} \in \boundnames{Q},\vec{y} \in \boundnames{P}$, where $Q\{\vec{y}/\vec{x}\}$
denotes the capture-avoiding substitution of $\vec{y}$ for $\vec{x}$ in $Q$.
\end{definition}

\begin{definition}
  The {\em structural congruence} \cite{SangiorgiWalker} , $\equiv$,
  between processes is the least congruence containing
  alpha-equivalence, satisfying the abelian monoid laws
  (associativity, commutativity and $\pzero$ as identity) for parallel
  composition $|$ and for summation $+$.
\end{definition}

\subsection{Name equivalence}

We take name equivalence, written $\nameeq$, to be the smallest
equivalence relation generated by the following rules.

\begin{mathpar}
\inferrule*[lab=Quote-drop]
{ }
{ \quotep{@{x}} \nameeq x }

\inferrule*[lab=Struct-equiv]
{ P \scong Q }
{ \quotep{P} \nameeq \quotep{Q} }
\end{mathpar}

The astute reader will have noticed that the mutual recursion of names
and processes imposes a mutual recursion on alpha-equivalence and
structural equivalence via name-equivalence. Fortunately, all of this
works out pleasantly and we may calculate in the natural way, free of
concern. The reader interested in the details is referred to the
appendix \ref{appendix:rho_details}.

\subsection{Substitution}

We use $\Proc$ for the set of processes, $\QProc$ for the set of
names, and $\id{\{}\vec{y} / \vec{x} \id{\}}$ to denote partial maps,
$s : \QProc \rightarrow \QProc$. A map, $s$ lifts, uniquely, to a map
on process terms, $\widehat{s} : \Proc \rightarrow \Proc$ by the
following equations.

\begin{mathpar}
  (0) \psubstp{Q}{P} := 0 \\
  (R \juxtap S) \psubstp{Q}{P}
  :=    
  (R)\psubstp{Q}{P} \juxtap (S) \psubstp{Q}{P} \\
  (x?(y).R) \psubstp{Q}{P}    
  :=    
  (x)\substp{Q}{P} (z)\concat( (R \psubstn{z}{y}) \psubstp{Q}{P} ) \\
  (\lift{x}{R}) \psubstp{Q}{P}  
  :=
  \lift{(x)\substp{Q}{P}}{ R \psubstp{Q}{P} } \\
%   (\dropn{x})  \psubstp{Q}{P}       
%   := 
%   \left\{ 
%     \begin{array}{ccc} 
%       \dropn{\quotep{Q}} & & x \nameeq \quotep{P} \\
%       \dropn{x} & & otherwise \\
%     \end{array}
%   \right. 
  (\dropn{x})  \psubstp{Q}{P}       
  := 
  \left\{ 
    \begin{array}{ccc} 
      Q & & x \nameeq \quotep{P} \\
      \dropn{x} & & otherwise \\
    \end{array}
  \right.
\end{mathpar}
 

where

\begin{eqnarray}
  (x)\id{\{} \lpquote Q \rpquote / \lpquote P \rpquote \id{\}}            = 
  \left\{ 
    \begin{array}{ccc}
      \lpquote Q \rpquote & & x \nameeq \lpquote P \rpquote \\
      x & & otherwise \\
    \end{array}
  \right. \nonumber
\end{eqnarray}

and $z$ is chosen distinct from $\quotep{P}$, $\quotep{Q}$, the free
names in $Q$, and all the names in $R$. Our $\alpha$-equivalence will
be built in the standard way from this substitution.

\begin{remark}\label{rem:no_self_referential_names}
  One consequence of these definitions is that $\forall P. \quotep{P}
  \not\in \freenames{P}$.
\end{remark}

\subsection{ Dynamic quote: an example }

Anticipating something of what's to come, consider applying the
substitution, $\widehat{\id{\{}u / z \id{\}}}$, to the following pair
of processes, $\lift{w}{y!(z)}$ and $w[ \lpquote y!(z) \rpquote ]$.

\begin{eqnarray}
	\lift{w}{y!(z)}\widehat{\id{\{}u / z \id{\}}}
		& = &
		\lift{w}{y!(u)} \nonumber\\
	w[ \lpquote y!(z) \rpquote ] \widehat{ \id{\{}u / z \id{\}} }
		& = &
		w[ \lpquote y!(z) \rpquote ] \nonumber
\end{eqnarray}

Because the body of the process between quotes is impervious to
substitution, we get radically different answers. In fact, by
examining the first process in an input context,
e.g. $x?(z).\lift{w}{y!(z)}$, we see that the process under the lift
operator may be shaped by prefixed inputs binding a name inside it. In
this sense, the lift operator will be seen as a way to dynamically
construct processes before reifying them as names.

Finally equipped with these standard features we can present the
dynamics of the calculus.

\subsubsection{Operational semantics} 

Finally, we introduce the computational dynamics. What marks these
algebras as distinct from other more traditionally studied algebraic
structures, e.g. vector spaces or polynomial rings, is the manner in
which dynamics is captured. In traditional structures, dynamics is typically
expressed through morphisms between such structures, as in linear maps
between vector spaces or morphisms between rings. In algebras
associated with the semantics of computation, the dynamics is
expressed as part of the algebraic structure itself, through a
reduction reduction relation typically denoted by $\red$. Below, we
give a recursive presentation of this relation for the calculus used
in the encoding.

$\red \subseteq \pi \times \pi$
$\red : \pi \to \mathcal{P}(\pi)$

\begin{mathpar}
  \inferrule* [lab=Comm] { \textsf{match}( x_{src}, x_{trgt} ) } { x_{trgt}?(y)P \; | \; x_{src}!\langle {Q} \rangle \red P\{\quotep{Q}/y}\} }
  \and \\
  \inferrule* [lab=Par] {{P} \red {P}'} {{{P} | {Q}} \red {{P}' | {Q}}}
  \and
  \inferrule* [lab=Equiv]{{{P} \scong {P}'} \andalso {{P}' \red {Q}'} \andalso {{Q}' \scong {Q}}}{{P} \red {Q}}
\end{mathpar}

\begin{eqnarray*}
  match_{\equiv} (\quotep{P},\quotep{Q}) & := & P \equiv Q \\
  match_{\dagger}(\quotep{P},\quotep{Q}) & := & \forall R. P|Q \red^{*} R => R \red^{*} 0 \\
  match_{K}(\quotep{P},\quotep{Q}) & := & K \mbox{ for some context } K
\end{eqnarray*}

$u?(x)P | u!\langle Q \rangle \red P\{\quotep{Q}/x\}$

%We write $\wred$ for $\red^*$, and $P\red$ if $\exists Q $ such that $ P \red Q$.
We write $P\red$ if $\exists Q $ such that $ P \red Q$ and $P\not\red$, otherwise.

\section{Replication}

As mentioned before, it is known that replication (and hence
recursion) can be implemented in a higher-order process algebra
\cite{SangiorgiWalker}. As our first example of calculation with the
machinery thus far presented we give the construction explicitly in
the {\rhoc}.

\begin{eqnarray}
	D_{x} & := & \prefix{x}{y}{(\binpar{\outputp{x}{y}}{@{y}})} \nonumber\\
	\bangp_{x}{P} & := & \binpar{{x}!\langle{\binpar{D_{x}}{P}}\rangle}{D_{x}} \nonumber
\end{eqnarray}

\begin{eqnarray}
	\bangp_{x}{P} & & \nonumber\\
	=
	& {x}!\langle{(\prefix{x}{y}{(\outputp{x}{y} | @{y})) | P}}\rangle 
	      | \prefix{x}{y}{(\outputp{x}{y} | @{y})} & \nonumber\\
	\red
	& (\outputp{x}{y} | @{y})\substn{\quotep{(\prefix{x}{y}{(@{y} | \outputp{x}{y})) | P}}}{y} & \nonumber\\
	=
	& \outputp{x}{\quotep{(\prefix{x}{y}{(\outputp{x}{y} | @{y})) | P}}}
	  | {(\prefix{x}{y}{(\outputp{x}{y} | @{y})) | P}} & \nonumber\\
	\red
	& \ldots & \nonumber\\
	\red^*
	& P | P | \ldots & \nonumber
\end{eqnarray}

Of course, this encoding, as an implementation, runs away, unfolding
$\bangp{P}$ eagerly. A lazier and more implementable replication
operator, restricted to input-guarded processes, may be obtained as follows.

\begin{eqnarray}
\bangp{\prefix{u}{v}{P}} 
	:= 
	\binpar{\lift{x}{\prefix{u}{v}{(\binpar{D(x)}{P})}}}{D(x)} \nonumber
\end{eqnarray}

\begin{remark}
  Note that the lazier definition still does not deal with summation
  or mixed summation (i.e. sums over input and output). The reader is
  invited to construct definitions of replication that deal with these
  features. 

  Further, the definitions are parameterized in a name, $x$. Can you,
  gentle reader, make a definition that eliminates this parameter and
  guarantees no accidental interaction between the replication
  machinery and the process being replicated -- i.e. no accidental
  sharing of names used by the process to get its work done and the
  name(s) used by the replication to effect copying. This latter
  revision of the definition of replication is crucial to obtaining
  the expected identity $!!P \sim !P$.
\end{remark}

\begin{remark}\label{rem:paradoxical_combinator}
  The reader familiar with the lambda calculus will have noticed the
  similarity between $D$ and the paradoxical combinator.

  [Ed. note: the existence of this seems to suggest we have to be more
  restrictive on the set of processes and names we admit if we are to
  support no-cloning.]
\end{remark}

\subsubsection{Bisimulation}

The computational dynamics gives rise to another kind of equivalence,
the equivalence of computational behavior. As previously mentioned
this is typically captured \emph{via} some form of bisimulation.

% The notion we use in this paper is weak barbed bisimulation
% \cite{milner91polyadicpi}.

The notion we use in this paper is derived from weak barbed
bisimulation \cite{milner91polyadicpi}. 

\begin{definition}
An \emph{observation relation}, $\downarrow_{\mathcal N}$, over a set
of names, $\mathcal N$, is the smallest relation satisfying the rules
below.

\infrule[Out-barb]{y \in {\mathcal N}, \; x \nameeq y}
		  {\outputp{x}{v} \downarrow_{\mathcal N} x}
\infrule[Par-barb]{\mbox{$P\downarrow_{\mathcal N} x$ or $Q\downarrow_{\mathcal N} x$}}
		  {\binpar{P}{Q} \downarrow_{\mathcal N} x}

We write $P \Downarrow_{\mathcal N} x$ if there is $Q$ such that 
$P \wred Q$ and $Q \downarrow_{\mathcal N} x$.
\end{definition}

\begin{definition}
%\label{def.bbisim}
An  ${\mathcal N}$-\emph{barbed bisimulation} over a set of names, ${\mathcal N}$, is a symmetric binary relation 
${\mathcal S}_{\mathcal N}$ between agents such that $P\rel{S}_{\mathcal N}Q$ implies:
\begin{enumerate}
\item If $P \red P'$ then $Q \wred Q'$ and $P'\rel{S}_{\mathcal N} Q'$.
\item If $P\downarrow_{\mathcal N} x$, then $Q\Downarrow_{\mathcal N} x$.
\end{enumerate}
$P$ is ${\mathcal N}$-barbed bisimilar to $Q$, written
$P \wbbisim_{\mathcal N} Q$, if $P \rel{S}_{\mathcal N} Q$ for some ${\mathcal N}$-barbed bisimulation ${\mathcal S}_{\mathcal N}$.
\end{definition}

$\mathcal{R} \subseteq \pi \times \pi$

$P \mathcal{R} Q => \forall P'. P \red P' \Rightarrow \exists Q'. Q \red Q', P' \mathcal{R} Q'$

$P \vdash x \Rightarrow Q \vdash x$

\begin{mathpar}
  \inferrule*[lab=Out-barb]{x \nameeq y}{{y}!\langle{Q}\rangle \vdash x}
  \and
  \inferrule*[lab=Par-barb]{\mbox{$P\vdash x$ or $Q\vdash x$}}{\binpar{P}{Q} \vdash x}
\end{mathpar}

\subsubsection{Contexts}

One of the principle advantages of computational calculi like the
$\pi$-calculus is a well-defined notion of context,
contextual-equivalence and a correlation between
contextual-equivalence and notions of bisimulation. The notion of
context allows the decomposition of a process into (sub-)process and
its syntactic environment, its context. Thus, a context may be
thought of as a process with a ``hole'' (written $\Box$) in it. The
application of a context $M$ to a process $P$, written $M[P]$, is
tantamount to filling the hole in $M$ with $P$. In this paper we do
not need the full weight of this theory, but do make use of the notion
of context in the proof the main theorem. 

\begin{mathpar}
  \inferrule* [lab=summation] {} {{M_{M},M_{N}} \bc \Box \;|\; x.M_{A} \;|\; M_{M}+M_{N}}
  \and
  \inferrule* [lab=agent] {} {{M_{A}} \bc (\vec{x})M_{P} \;| \; \clift{P_0,\ldots,M_{P},\ldots,P_N}}
  \and \\
  \inferrule* [lab=process] {} {{M_{P}} \bc M_{N} \;| \;P|M_{P} }
\end{mathpar} 

\begin{mathpar}
  \inferrule* [lab=sychronization] {} {M_{N} \bc \Box \;|\; x?M_{F} \;|\; x!M_{C}}
  \and
  \inferrule* [lab=abstraction] {} {{M_{F}} \bc (x)M_{P} }
  \and
  \inferrule* [lab=concretion] {} {{M_{C}} \bc \langle M_{P} \rangle }
  \and \\
  \inferrule* [lab=process] {} {{M_{P}} \bc M_{N} \;| \;P|M_{P} }
\end{mathpar}

\begin{definition}[contextual application] Given a context $M$, and
  process $P$, we define the \emph{contextual application}, $M[P] :=
  M\{P/\Box\}$. That is, the contextual application of M to P is the
  substitution of $P$ for $\Box$ in $M$.
\end{definition}

$\meaningof{-} : L \to \mathcal{P}(\pi)$

\begin{mathpar}
  \inferrule* [lab=collection] {} {\meaningof{true} = \pi, \and \meaningof{~E} = \pi \setminus \meaningof{E}, \and \meaningof{E_{1} \& E_{2}} = \meaningof{E_{1}} \cap \meaningof{E_{2}}}
\end{mathpar}

\begin{mathpar}
  \inferrule* [lab=structure] {} {\meaningof{0} = \{ P \in \pi | P \equiv 0 \}, \and \\ \meaningof{E_1 | E_2} = \{ P \in \pi | P \equiv P_{1} | P_{2}, P_{1} \in \meaningof{E_{1}}, P_{2} \in \meaningof{E_2}\} }
\end{mathpar}

\begin{mathpar}
 \inferrule* [lab=behavior] {} {\meaningof{\langle a?b \rangle E} = \{ P \in \pi | P \equiv Q | u?(y)P', \\ \and \\\\ \and \\ \;\;\; u \in \meaningof{a}, \forall z.P'\{z/y\} \in \meaningof{E\{z/b\}}\}, \and \\ \meaningof{a!E} = \{ P \in \pi | P \equiv Q | x!\langle P' \rangle, x \in \meaningof{a} P' \in \meaningof{E}\} }
\end{mathpar}

\begin{mathpar}
 \inferrule* [lab=nominal] {} {\meaningof{\quotep{E}} = \{ \quotep{P} \in \quotep{\pi} | P \in \meaningof{E} \}, \and \meaningof{\quotep{P}} = \{ \quotep{Q} \in \quotep{\pi} | P \equiv Q \} \and \\ \meaningof{@\quotep{E}} = \{ P \in \pi | P \equiv @x, x \in \meaningof{E} \}}
\end{mathpar}

\begin{eqnarray*}
  \\
  \meaningof{-} : TS \to ST
\end{eqnarray*}

\begin{eqnarray*}
  \\
  L : TS \to ST
\end{eqnarray*}

\begin{eqnarray*}
  \\
  P \models E \iff P \in \meaningof{E}
\end{eqnarray*}

\begin{eqnarray*}
  P \approx_{L} Q \iff \forall E \in L. P \models E \iff Q \models E
\end{eqnarray*}

\begin{eqnarray*}
  P \approx_{K} Q
\end{eqnarray*}

\begin{eqnarray*}
  P \approx Q
\end{eqnarray*}

$\approx_{K} = \approx = \approx_{L}$

\subsubsection{Contextual duality}

Note that contexts extend the quotation operation to a family of
operations from processes to names. Given a context, $M$, we can
define a \emph{nominal context}, $\quotep{M}$ by $\quotep{M}[P] :=
\quotep{M[P]}$. To foreshadow what is to come we observe that these
operations enjoy a duality with processes very much like the duality
between vectors and maps from vectors to scalars.

Further, because the calculus is essentially higher-order, we have a
correspondence between contexts and processes. More specifically,
given a name $x$ and a context $M$ we can construct $M^{*}_{x}$ such
that 

\begin{mathpar}
  M^{*}_{x} | \lift{x}{P} \red M[P]
\end{mathpar}

namely,

\begin{mathpar}
  M^{*}_{x} := x?(u).M[\dropn{u}]
\end{mathpar}

The dependence of $M^{*}_{x}$ on a name makes it an abstraction, 

\begin{mathpar}
  M^{*} := (x)x?(u).M[\dropn{u}]
\end{mathpar}

\subsection{Additional notation}

It will sometimes be convenient to denote the process a name
quotes. We already have the notation $x = \quotep{P}$, but it will be
convenient to introduce an alternate notation, $\procn{x}$, when we
want to emphasize the connection to the use of the name. Note that, by
virtue of name equivalence, $\quotep{\procn{x}} \nameeq x$; so, the
notation is consistent with previous definitions.

Further, because names have structure it is possible to effect
substitutions on the basis of that structure. This means we need to
upgrade our notation for substitutions, which we accomplish by
adapting comprehension notation. Thus,

\begin{mathpar}
  P\{ y / x : x \in S \}
\end{mathpar}

is interpreted to mean the process derived from P by replacing (in a
capture-avoiding manner) each occurrence of $x$ in $S$ by $y$. For example,

\begin{mathpar}
  P\{ \quotep{\procn{x}|\procn{x}} / x : x \in \freenames{P} \}
\end{mathpar}

will replace each (occurrence) of a free name $x$ in $P$ by
$\quotep{\procn{x}|\procn{x}}$.

Also, we will avail ourselves of the notation $x^{L}$ and $x^{R}$ to
denote injections of a name into disjoint copies of the name
space. There are numerous ways to accomplish this. One example can be
found in \cite{MeredithR05}. This notation overloads to vectors of
names: $\vec{x}^{\pi} := (x_{i}^{\pi} \; : \; 0 \leq i < |\vec{x}| )$ where $\pi \in \{L,R\}$.

We also use $P^{\Box} := P|\Box$.

In \cite{MeredithR05} an interpretation of the new operator is
given. It turns out that there are several possible interpretations
all enjoying the requisite algebraic properties of the operator (see
\cite{milner91polyadicpi}). We will therefore make liberal use of
$(\nu\; \vec{x})P$.

% subsection the_syntax_and_semantics_of_the_notation_system (end)   

\input{qm2pi.qmops} 

\input{qm2pi.sterngerlach} 

\input{qm2pi.metric} 

% section concurrent_process_calculi (end)

%\input{qm2pi.proofsketch}

% section proof sketch (end)

%\input{qm2pi.slviaknots} 

% section spatial logic via knots (end)

\input{qm2pi.conclusion}

% section conclusion (end)

%\input{qm2pi.dtcodes} 

% section wiring algorithm (end)

\input{qm2pi.ack} 

% section acknowledgments (end)

\newpage


\bibliographystyle{plain}   
\bibliography{../../biblios/main.bib}

\input{qm2pi.rhodetails}

\end{document}

 

% section wiring algorithm (end)

\documentclass[12pt]{llncs}
%\documentclass{jktr}

\usepackage[pdftex]{hyperref}                   
\usepackage {listings}
\usepackage {mathpartir}
\usepackage{bcprules}
%\usepackage{listings}
                       
\usepackage{graphicx} 
%\usepackage[margins=2.5cm,nohead,nofoot]{geometry}
%\usepackage{geometry}
\usepackage{amsfonts}
\usepackage{amstext}
\usepackage{latexsym}
\usepackage{amssymb}
\usepackage{color}


%\include{myPreamble}
\include{qm2pi.local} 

%\ifpdf
%\usepackage[pdftex]{graphicx}
%\else
%\usepackage{graphicx}
%\fi

 % \ifpdf
%  \usepackage{pdfsync}
%  \if


%\title{Brief Article}
%\author{David F. Snyder}
%\author{L.G. Meredith}

%\address{Dept. of Math., Texas State University--San Marcos, San Marcos, TX 78666}
       
\pagestyle{empty}


\begin{document}

\lstset{language=[Objective]Caml,frame=shadowbox}

\input{qm2pi.front}

% section front matter (end)

\input{qm2pi.intro} 
 
% section introduction (end)

% \input{qm2pi.knotations} 

% section notation (end)

\input{qm2pi.process.calculi} 

% section concurrent_process_calculi_and_spatial_logics_ (end)
    
%\input{qm2pi.knots2pi} 

%\input{qm2pi.trefoil} 

%\input{qm2pi.mainthm} 

% subsection basic_interpretation (end)

%\input{qm2pi.rho.presentation} 
\subsection{The syntax and semantics of the notation system}\label{sub:the_syntax_and_semantics_of_the_notation_system} % (fold)

We now summarize a technical presentation of the calculus that
embodies our theory of dynamics. The typical presentation of such a
calculus follows the style of giving generators and relations on
them. The grammar, below, describing term constructors, freely
generates the set of processes, $\Proc$. This set is then quotiented
by a relation known as structural congruence and it is over this set
that the notion of dynamics is expressed. This presentation is
essentially that of \cite{MeredithR05} with the addition of
polyadicity and summation. For readability we have relegated some of
the technical subtleties to an appendix.

\subsubsection{Process grammar}\label{subsub:process_grammar}

\begin{mathpar}
  \inferrule* [lab=synchronization] {} {{M} \bc \pzero \;|\; x?F \;|\; x!C }
  \and
  \inferrule* [lab=abstraction] {} {{F} \bc (x)P}
  \and
  \inferrule* [lab=concretion] {} {{C} \bc \langle Q \rangle}
  \and
  \inferrule* [lab=process] {} {{P,Q} \bc M \;| \;P|Q \;|\; @{x}}
  \and
  \inferrule* [lab=name] {} {{x} \bc \quotep{P}}
\end{mathpar} 

Note that $\vec{x}$ (resp. $\vec{P}$) denotes a vector of names
(resp. processes) of length $|\vec{x}|$ (resp. $|\vec{P}|$). We adopt
the following useful abbreviations.

\begin{mathpar}
   x?(\vec{y}).P := x.(\vec{y})P \and  x\clift{\vec{P}} := x.\clift{\vec{P}}
   \and x!(y) := \lift{x}{\dropn{y}}
   \and \Pi_{i=0}^{n-1}P_i := P_0 | \ldots | P_{n-1}
\end{mathpar}

\subsubsection{Structural congruence}

\paragraph{Free and bound names and alpha-equivalence.} At the
core of structural equivalence is alpha-equivalence which identifies
process that are the same up to a change of variable. Formally, we
recognize the distinction between free and bound names. The free names
of a process, $\freenames{P}$, may be calculated recursively as
follows:

\begin{mathpar}
\freenames{\pzero} := \emptyset
  \and \\
  \freenames{x?(y).P} := \{ x \} \cup (\freenames{P} \setminus \{ y \})
  \and 
  \freenames{x!\langle P \rangle} := \{ x \} \cup \{ P \} 
  \and \\
  \freenames{P|Q} := \freenames{P} \cup \freenames{Q}
  \and \\
  \freenames{@{x}} := \{ x \}
\end{mathpar}

$\pi$
$\quotep{\pi}$

$\freenames{-} : \pi \to \mathcal{P}(\quotep{\pi})$

\begin{eqnarray*}
  \freenames{\pzero} & := & \emptyset \\
  \freenames{x?(y).P} & := & \{ x \} \cup (\freenames{P} \setminus \{ y \}) \\
  \freenames{x!\langle P \rangle} & := & \{ x \} \cup \{ P \} \\
  \freenames{P|Q} & := & \freenames{P} \cup \freenames{Q} \\
  \freenames{\dropn{x}} & := & \{ x \}
\end{eqnarray*}

The bound names of a process, $\boundnames{P}$, are those names occurring in $P$
that are not free. For example, in $x?(y).0$, the name $x$ is free, while $y$ is bound.

\begin{mathpar}
  \inferrule* [lab=monoidal-laws] {} { P|Q \equiv Q|P \and P|0 \equiv P \and P|(Q|R) \equiv (P|Q)|R }
\end{mathpar}

\begin{mathpar}
  \inferrule* [lab=alpha-equivalence] {} { (x)P \equiv (y)P\{y/x\} \and y \not\in \freenames{P} }
\end{mathpar}

\begin{definition}
Then two processes, $P,Q$, are alpha-equivalent if $P = Q\{\vec{y}/\vec{x}\}$ for
some $\vec{x} \in \boundnames{Q},\vec{y} \in \boundnames{P}$, where $Q\{\vec{y}/\vec{x}\}$
denotes the capture-avoiding substitution of $\vec{y}$ for $\vec{x}$ in $Q$.
\end{definition}

\begin{definition}
  The {\em structural congruence} \cite{SangiorgiWalker} , $\equiv$,
  between processes is the least congruence containing
  alpha-equivalence, satisfying the abelian monoid laws
  (associativity, commutativity and $\pzero$ as identity) for parallel
  composition $|$ and for summation $+$.
\end{definition}

\subsection{Name equivalence}

We take name equivalence, written $\nameeq$, to be the smallest
equivalence relation generated by the following rules.

\begin{mathpar}
\inferrule*[lab=Quote-drop]
{ }
{ \quotep{@{x}} \nameeq x }

\inferrule*[lab=Struct-equiv]
{ P \scong Q }
{ \quotep{P} \nameeq \quotep{Q} }
\end{mathpar}

The astute reader will have noticed that the mutual recursion of names
and processes imposes a mutual recursion on alpha-equivalence and
structural equivalence via name-equivalence. Fortunately, all of this
works out pleasantly and we may calculate in the natural way, free of
concern. The reader interested in the details is referred to the
appendix \ref{appendix:rho_details}.

\subsection{Substitution}

We use $\Proc$ for the set of processes, $\QProc$ for the set of
names, and $\id{\{}\vec{y} / \vec{x} \id{\}}$ to denote partial maps,
$s : \QProc \rightarrow \QProc$. A map, $s$ lifts, uniquely, to a map
on process terms, $\widehat{s} : \Proc \rightarrow \Proc$ by the
following equations.

\begin{mathpar}
  (0) \psubstp{Q}{P} := 0 \\
  (R \juxtap S) \psubstp{Q}{P}
  :=    
  (R)\psubstp{Q}{P} \juxtap (S) \psubstp{Q}{P} \\
  (x?(y).R) \psubstp{Q}{P}    
  :=    
  (x)\substp{Q}{P} (z)\concat( (R \psubstn{z}{y}) \psubstp{Q}{P} ) \\
  (\lift{x}{R}) \psubstp{Q}{P}  
  :=
  \lift{(x)\substp{Q}{P}}{ R \psubstp{Q}{P} } \\
%   (\dropn{x})  \psubstp{Q}{P}       
%   := 
%   \left\{ 
%     \begin{array}{ccc} 
%       \dropn{\quotep{Q}} & & x \nameeq \quotep{P} \\
%       \dropn{x} & & otherwise \\
%     \end{array}
%   \right. 
  (\dropn{x})  \psubstp{Q}{P}       
  := 
  \left\{ 
    \begin{array}{ccc} 
      Q & & x \nameeq \quotep{P} \\
      \dropn{x} & & otherwise \\
    \end{array}
  \right.
\end{mathpar}
 

where

\begin{eqnarray}
  (x)\id{\{} \lpquote Q \rpquote / \lpquote P \rpquote \id{\}}            = 
  \left\{ 
    \begin{array}{ccc}
      \lpquote Q \rpquote & & x \nameeq \lpquote P \rpquote \\
      x & & otherwise \\
    \end{array}
  \right. \nonumber
\end{eqnarray}

and $z$ is chosen distinct from $\quotep{P}$, $\quotep{Q}$, the free
names in $Q$, and all the names in $R$. Our $\alpha$-equivalence will
be built in the standard way from this substitution.

\begin{remark}\label{rem:no_self_referential_names}
  One consequence of these definitions is that $\forall P. \quotep{P}
  \not\in \freenames{P}$.
\end{remark}

\subsection{ Dynamic quote: an example }

Anticipating something of what's to come, consider applying the
substitution, $\widehat{\id{\{}u / z \id{\}}}$, to the following pair
of processes, $\lift{w}{y!(z)}$ and $w[ \lpquote y!(z) \rpquote ]$.

\begin{eqnarray}
	\lift{w}{y!(z)}\widehat{\id{\{}u / z \id{\}}}
		& = &
		\lift{w}{y!(u)} \nonumber\\
	w[ \lpquote y!(z) \rpquote ] \widehat{ \id{\{}u / z \id{\}} }
		& = &
		w[ \lpquote y!(z) \rpquote ] \nonumber
\end{eqnarray}

Because the body of the process between quotes is impervious to
substitution, we get radically different answers. In fact, by
examining the first process in an input context,
e.g. $x?(z).\lift{w}{y!(z)}$, we see that the process under the lift
operator may be shaped by prefixed inputs binding a name inside it. In
this sense, the lift operator will be seen as a way to dynamically
construct processes before reifying them as names.

Finally equipped with these standard features we can present the
dynamics of the calculus.

\subsubsection{Operational semantics} 

Finally, we introduce the computational dynamics. What marks these
algebras as distinct from other more traditionally studied algebraic
structures, e.g. vector spaces or polynomial rings, is the manner in
which dynamics is captured. In traditional structures, dynamics is typically
expressed through morphisms between such structures, as in linear maps
between vector spaces or morphisms between rings. In algebras
associated with the semantics of computation, the dynamics is
expressed as part of the algebraic structure itself, through a
reduction reduction relation typically denoted by $\red$. Below, we
give a recursive presentation of this relation for the calculus used
in the encoding.

$\red \subseteq \pi \times \pi$
$\red : \pi \to \mathcal{P}(\pi)$

\begin{mathpar}
  \inferrule* [lab=Comm] { \textsf{match}( x_{src}, x_{trgt} ) } { x_{trgt}?(y)P \; | \; x_{src}!\langle {Q} \rangle \red P\{\quotep{Q}/y}\} }
  \and \\
  \inferrule* [lab=Par] {{P} \red {P}'} {{{P} | {Q}} \red {{P}' | {Q}}}
  \and
  \inferrule* [lab=Equiv]{{{P} \scong {P}'} \andalso {{P}' \red {Q}'} \andalso {{Q}' \scong {Q}}}{{P} \red {Q}}
\end{mathpar}

\begin{eqnarray*}
  match_{\equiv} (\quotep{P},\quotep{Q}) & := & P \equiv Q \\
  match_{\dagger}(\quotep{P},\quotep{Q}) & := & \forall R. P|Q \red^{*} R => R \red^{*} 0 \\
  match_{K}(\quotep{P},\quotep{Q}) & := & K \mbox{ for some context } K
\end{eqnarray*}

$u?(x)P | u!\langle Q \rangle \red P\{\quotep{Q}/x\}$

%We write $\wred$ for $\red^*$, and $P\red$ if $\exists Q $ such that $ P \red Q$.
We write $P\red$ if $\exists Q $ such that $ P \red Q$ and $P\not\red$, otherwise.

\section{Replication}

As mentioned before, it is known that replication (and hence
recursion) can be implemented in a higher-order process algebra
\cite{SangiorgiWalker}. As our first example of calculation with the
machinery thus far presented we give the construction explicitly in
the {\rhoc}.

\begin{eqnarray}
	D_{x} & := & \prefix{x}{y}{(\binpar{\outputp{x}{y}}{@{y}})} \nonumber\\
	\bangp_{x}{P} & := & \binpar{{x}!\langle{\binpar{D_{x}}{P}}\rangle}{D_{x}} \nonumber
\end{eqnarray}

\begin{eqnarray}
	\bangp_{x}{P} & & \nonumber\\
	=
	& {x}!\langle{(\prefix{x}{y}{(\outputp{x}{y} | @{y})) | P}}\rangle 
	      | \prefix{x}{y}{(\outputp{x}{y} | @{y})} & \nonumber\\
	\red
	& (\outputp{x}{y} | @{y})\substn{\quotep{(\prefix{x}{y}{(@{y} | \outputp{x}{y})) | P}}}{y} & \nonumber\\
	=
	& \outputp{x}{\quotep{(\prefix{x}{y}{(\outputp{x}{y} | @{y})) | P}}}
	  | {(\prefix{x}{y}{(\outputp{x}{y} | @{y})) | P}} & \nonumber\\
	\red
	& \ldots & \nonumber\\
	\red^*
	& P | P | \ldots & \nonumber
\end{eqnarray}

Of course, this encoding, as an implementation, runs away, unfolding
$\bangp{P}$ eagerly. A lazier and more implementable replication
operator, restricted to input-guarded processes, may be obtained as follows.

\begin{eqnarray}
\bangp{\prefix{u}{v}{P}} 
	:= 
	\binpar{\lift{x}{\prefix{u}{v}{(\binpar{D(x)}{P})}}}{D(x)} \nonumber
\end{eqnarray}

\begin{remark}
  Note that the lazier definition still does not deal with summation
  or mixed summation (i.e. sums over input and output). The reader is
  invited to construct definitions of replication that deal with these
  features. 

  Further, the definitions are parameterized in a name, $x$. Can you,
  gentle reader, make a definition that eliminates this parameter and
  guarantees no accidental interaction between the replication
  machinery and the process being replicated -- i.e. no accidental
  sharing of names used by the process to get its work done and the
  name(s) used by the replication to effect copying. This latter
  revision of the definition of replication is crucial to obtaining
  the expected identity $!!P \sim !P$.
\end{remark}

\begin{remark}\label{rem:paradoxical_combinator}
  The reader familiar with the lambda calculus will have noticed the
  similarity between $D$ and the paradoxical combinator.

  [Ed. note: the existence of this seems to suggest we have to be more
  restrictive on the set of processes and names we admit if we are to
  support no-cloning.]
\end{remark}

\subsubsection{Bisimulation}

The computational dynamics gives rise to another kind of equivalence,
the equivalence of computational behavior. As previously mentioned
this is typically captured \emph{via} some form of bisimulation.

% The notion we use in this paper is weak barbed bisimulation
% \cite{milner91polyadicpi}.

The notion we use in this paper is derived from weak barbed
bisimulation \cite{milner91polyadicpi}. 

\begin{definition}
An \emph{observation relation}, $\downarrow_{\mathcal N}$, over a set
of names, $\mathcal N$, is the smallest relation satisfying the rules
below.

\infrule[Out-barb]{y \in {\mathcal N}, \; x \nameeq y}
		  {\outputp{x}{v} \downarrow_{\mathcal N} x}
\infrule[Par-barb]{\mbox{$P\downarrow_{\mathcal N} x$ or $Q\downarrow_{\mathcal N} x$}}
		  {\binpar{P}{Q} \downarrow_{\mathcal N} x}

We write $P \Downarrow_{\mathcal N} x$ if there is $Q$ such that 
$P \wred Q$ and $Q \downarrow_{\mathcal N} x$.
\end{definition}

\begin{definition}
%\label{def.bbisim}
An  ${\mathcal N}$-\emph{barbed bisimulation} over a set of names, ${\mathcal N}$, is a symmetric binary relation 
${\mathcal S}_{\mathcal N}$ between agents such that $P\rel{S}_{\mathcal N}Q$ implies:
\begin{enumerate}
\item If $P \red P'$ then $Q \wred Q'$ and $P'\rel{S}_{\mathcal N} Q'$.
\item If $P\downarrow_{\mathcal N} x$, then $Q\Downarrow_{\mathcal N} x$.
\end{enumerate}
$P$ is ${\mathcal N}$-barbed bisimilar to $Q$, written
$P \wbbisim_{\mathcal N} Q$, if $P \rel{S}_{\mathcal N} Q$ for some ${\mathcal N}$-barbed bisimulation ${\mathcal S}_{\mathcal N}$.
\end{definition}

$\mathcal{R} \subseteq \pi \times \pi$

$P \mathcal{R} Q => \forall P'. P \red P' \Rightarrow \exists Q'. Q \red Q', P' \mathcal{R} Q'$

$P \vdash x \Rightarrow Q \vdash x$

\begin{mathpar}
  \inferrule*[lab=Out-barb]{x \nameeq y}{{y}!\langle{Q}\rangle \vdash x}
  \and
  \inferrule*[lab=Par-barb]{\mbox{$P\vdash x$ or $Q\vdash x$}}{\binpar{P}{Q} \vdash x}
\end{mathpar}

\subsubsection{Contexts}

One of the principle advantages of computational calculi like the
$\pi$-calculus is a well-defined notion of context,
contextual-equivalence and a correlation between
contextual-equivalence and notions of bisimulation. The notion of
context allows the decomposition of a process into (sub-)process and
its syntactic environment, its context. Thus, a context may be
thought of as a process with a ``hole'' (written $\Box$) in it. The
application of a context $M$ to a process $P$, written $M[P]$, is
tantamount to filling the hole in $M$ with $P$. In this paper we do
not need the full weight of this theory, but do make use of the notion
of context in the proof the main theorem. 

\begin{mathpar}
  \inferrule* [lab=summation] {} {{M_{M},M_{N}} \bc \Box \;|\; x.M_{A} \;|\; M_{M}+M_{N}}
  \and
  \inferrule* [lab=agent] {} {{M_{A}} \bc (\vec{x})M_{P} \;| \; \clift{P_0,\ldots,M_{P},\ldots,P_N}}
  \and \\
  \inferrule* [lab=process] {} {{M_{P}} \bc M_{N} \;| \;P|M_{P} }
\end{mathpar} 

\begin{mathpar}
  \inferrule* [lab=sychronization] {} {M_{N} \bc \Box \;|\; x?M_{F} \;|\; x!M_{C}}
  \and
  \inferrule* [lab=abstraction] {} {{M_{F}} \bc (x)M_{P} }
  \and
  \inferrule* [lab=concretion] {} {{M_{C}} \bc \langle M_{P} \rangle }
  \and \\
  \inferrule* [lab=process] {} {{M_{P}} \bc M_{N} \;| \;P|M_{P} }
\end{mathpar}

\begin{definition}[contextual application] Given a context $M$, and
  process $P$, we define the \emph{contextual application}, $M[P] :=
  M\{P/\Box\}$. That is, the contextual application of M to P is the
  substitution of $P$ for $\Box$ in $M$.
\end{definition}

$\meaningof{-} : L \to \mathcal{P}(\pi)$

\begin{mathpar}
  \inferrule* [lab=collection] {} {\meaningof{true} = \pi, \and \meaningof{~E} = \pi \setminus \meaningof{E}, \and \meaningof{E_{1} \& E_{2}} = \meaningof{E_{1}} \cap \meaningof{E_{2}}}
\end{mathpar}

\begin{mathpar}
  \inferrule* [lab=structure] {} {\meaningof{0} = \{ P \in \pi | P \equiv 0 \}, \and \\ \meaningof{E_1 | E_2} = \{ P \in \pi | P \equiv P_{1} | P_{2}, P_{1} \in \meaningof{E_{1}}, P_{2} \in \meaningof{E_2}\} }
\end{mathpar}

\begin{mathpar}
 \inferrule* [lab=behavior] {} {\meaningof{\langle a?b \rangle E} = \{ P \in \pi | P \equiv Q | u?(y)P', \\ \and \\\\ \and \\ \;\;\; u \in \meaningof{a}, \forall z.P'\{z/y\} \in \meaningof{E\{z/b\}}\}, \and \\ \meaningof{a!E} = \{ P \in \pi | P \equiv Q | x!\langle P' \rangle, x \in \meaningof{a} P' \in \meaningof{E}\} }
\end{mathpar}

\begin{mathpar}
 \inferrule* [lab=nominal] {} {\meaningof{\quotep{E}} = \{ \quotep{P} \in \quotep{\pi} | P \in \meaningof{E} \}, \and \meaningof{\quotep{P}} = \{ \quotep{Q} \in \quotep{\pi} | P \equiv Q \} \and \\ \meaningof{@\quotep{E}} = \{ P \in \pi | P \equiv @x, x \in \meaningof{E} \}}
\end{mathpar}

\begin{eqnarray*}
  \\
  \meaningof{-} : TS \to ST
\end{eqnarray*}

\begin{eqnarray*}
  \\
  L : TS \to ST
\end{eqnarray*}

\begin{eqnarray*}
  \\
  P \models E \iff P \in \meaningof{E}
\end{eqnarray*}

\begin{eqnarray*}
  P \approx_{L} Q \iff \forall E \in L. P \models E \iff Q \models E
\end{eqnarray*}

\begin{eqnarray*}
  P \approx_{K} Q
\end{eqnarray*}

\begin{eqnarray*}
  P \approx Q
\end{eqnarray*}

$\approx_{K} = \approx = \approx_{L}$

\subsubsection{Contextual duality}

Note that contexts extend the quotation operation to a family of
operations from processes to names. Given a context, $M$, we can
define a \emph{nominal context}, $\quotep{M}$ by $\quotep{M}[P] :=
\quotep{M[P]}$. To foreshadow what is to come we observe that these
operations enjoy a duality with processes very much like the duality
between vectors and maps from vectors to scalars.

Further, because the calculus is essentially higher-order, we have a
correspondence between contexts and processes. More specifically,
given a name $x$ and a context $M$ we can construct $M^{*}_{x}$ such
that 

\begin{mathpar}
  M^{*}_{x} | \lift{x}{P} \red M[P]
\end{mathpar}

namely,

\begin{mathpar}
  M^{*}_{x} := x?(u).M[\dropn{u}]
\end{mathpar}

The dependence of $M^{*}_{x}$ on a name makes it an abstraction, 

\begin{mathpar}
  M^{*} := (x)x?(u).M[\dropn{u}]
\end{mathpar}

\subsection{Additional notation}

It will sometimes be convenient to denote the process a name
quotes. We already have the notation $x = \quotep{P}$, but it will be
convenient to introduce an alternate notation, $\procn{x}$, when we
want to emphasize the connection to the use of the name. Note that, by
virtue of name equivalence, $\quotep{\procn{x}} \nameeq x$; so, the
notation is consistent with previous definitions.

Further, because names have structure it is possible to effect
substitutions on the basis of that structure. This means we need to
upgrade our notation for substitutions, which we accomplish by
adapting comprehension notation. Thus,

\begin{mathpar}
  P\{ y / x : x \in S \}
\end{mathpar}

is interpreted to mean the process derived from P by replacing (in a
capture-avoiding manner) each occurrence of $x$ in $S$ by $y$. For example,

\begin{mathpar}
  P\{ \quotep{\procn{x}|\procn{x}} / x : x \in \freenames{P} \}
\end{mathpar}

will replace each (occurrence) of a free name $x$ in $P$ by
$\quotep{\procn{x}|\procn{x}}$.

Also, we will avail ourselves of the notation $x^{L}$ and $x^{R}$ to
denote injections of a name into disjoint copies of the name
space. There are numerous ways to accomplish this. One example can be
found in \cite{MeredithR05}. This notation overloads to vectors of
names: $\vec{x}^{\pi} := (x_{i}^{\pi} \; : \; 0 \leq i < |\vec{x}| )$ where $\pi \in \{L,R\}$.

We also use $P^{\Box} := P|\Box$.

In \cite{MeredithR05} an interpretation of the new operator is
given. It turns out that there are several possible interpretations
all enjoying the requisite algebraic properties of the operator (see
\cite{milner91polyadicpi}). We will therefore make liberal use of
$(\nu\; \vec{x})P$.

% subsection the_syntax_and_semantics_of_the_notation_system (end)   

\input{qm2pi.qmops} 

\input{qm2pi.sterngerlach} 

\input{qm2pi.metric} 

% section concurrent_process_calculi (end)

%\input{qm2pi.proofsketch}

% section proof sketch (end)

%\input{qm2pi.slviaknots} 

% section spatial logic via knots (end)

\input{qm2pi.conclusion}

% section conclusion (end)

%\input{qm2pi.dtcodes} 

% section wiring algorithm (end)

\input{qm2pi.ack} 

% section acknowledgments (end)

\newpage


\bibliographystyle{plain}   
\bibliography{../../biblios/main.bib}

\input{qm2pi.rhodetails}

\end{document}

 

% section acknowledgments (end)

\newpage


\bibliographystyle{plain}   
\bibliography{../../biblios/main.bib}

\documentclass[12pt]{llncs}
%\documentclass{jktr}

\usepackage[pdftex]{hyperref}                   
\usepackage {listings}
\usepackage {mathpartir}
\usepackage{bcprules}
%\usepackage{listings}
                       
\usepackage{graphicx} 
%\usepackage[margins=2.5cm,nohead,nofoot]{geometry}
%\usepackage{geometry}
\usepackage{amsfonts}
\usepackage{amstext}
\usepackage{latexsym}
\usepackage{amssymb}
\usepackage{color}


%\include{myPreamble}
\include{qm2pi.local} 

%\ifpdf
%\usepackage[pdftex]{graphicx}
%\else
%\usepackage{graphicx}
%\fi

 % \ifpdf
%  \usepackage{pdfsync}
%  \if


%\title{Brief Article}
%\author{David F. Snyder}
%\author{L.G. Meredith}

%\address{Dept. of Math., Texas State University--San Marcos, San Marcos, TX 78666}
       
\pagestyle{empty}


\begin{document}

\lstset{language=[Objective]Caml,frame=shadowbox}

\input{qm2pi.front}

% section front matter (end)

\input{qm2pi.intro} 
 
% section introduction (end)

% \input{qm2pi.knotations} 

% section notation (end)

\input{qm2pi.process.calculi} 

% section concurrent_process_calculi_and_spatial_logics_ (end)
    
%\input{qm2pi.knots2pi} 

%\input{qm2pi.trefoil} 

%\input{qm2pi.mainthm} 

% subsection basic_interpretation (end)

%\input{qm2pi.rho.presentation} 
\subsection{The syntax and semantics of the notation system}\label{sub:the_syntax_and_semantics_of_the_notation_system} % (fold)

We now summarize a technical presentation of the calculus that
embodies our theory of dynamics. The typical presentation of such a
calculus follows the style of giving generators and relations on
them. The grammar, below, describing term constructors, freely
generates the set of processes, $\Proc$. This set is then quotiented
by a relation known as structural congruence and it is over this set
that the notion of dynamics is expressed. This presentation is
essentially that of \cite{MeredithR05} with the addition of
polyadicity and summation. For readability we have relegated some of
the technical subtleties to an appendix.

\subsubsection{Process grammar}\label{subsub:process_grammar}

\begin{mathpar}
  \inferrule* [lab=synchronization] {} {{M} \bc \pzero \;|\; x?F \;|\; x!C }
  \and
  \inferrule* [lab=abstraction] {} {{F} \bc (x)P}
  \and
  \inferrule* [lab=concretion] {} {{C} \bc \langle Q \rangle}
  \and
  \inferrule* [lab=process] {} {{P,Q} \bc M \;| \;P|Q \;|\; @{x}}
  \and
  \inferrule* [lab=name] {} {{x} \bc \quotep{P}}
\end{mathpar} 

Note that $\vec{x}$ (resp. $\vec{P}$) denotes a vector of names
(resp. processes) of length $|\vec{x}|$ (resp. $|\vec{P}|$). We adopt
the following useful abbreviations.

\begin{mathpar}
   x?(\vec{y}).P := x.(\vec{y})P \and  x\clift{\vec{P}} := x.\clift{\vec{P}}
   \and x!(y) := \lift{x}{\dropn{y}}
   \and \Pi_{i=0}^{n-1}P_i := P_0 | \ldots | P_{n-1}
\end{mathpar}

\subsubsection{Structural congruence}

\paragraph{Free and bound names and alpha-equivalence.} At the
core of structural equivalence is alpha-equivalence which identifies
process that are the same up to a change of variable. Formally, we
recognize the distinction between free and bound names. The free names
of a process, $\freenames{P}$, may be calculated recursively as
follows:

\begin{mathpar}
\freenames{\pzero} := \emptyset
  \and \\
  \freenames{x?(y).P} := \{ x \} \cup (\freenames{P} \setminus \{ y \})
  \and 
  \freenames{x!\langle P \rangle} := \{ x \} \cup \{ P \} 
  \and \\
  \freenames{P|Q} := \freenames{P} \cup \freenames{Q}
  \and \\
  \freenames{@{x}} := \{ x \}
\end{mathpar}

$\pi$
$\quotep{\pi}$

$\freenames{-} : \pi \to \mathcal{P}(\quotep{\pi})$

\begin{eqnarray*}
  \freenames{\pzero} & := & \emptyset \\
  \freenames{x?(y).P} & := & \{ x \} \cup (\freenames{P} \setminus \{ y \}) \\
  \freenames{x!\langle P \rangle} & := & \{ x \} \cup \{ P \} \\
  \freenames{P|Q} & := & \freenames{P} \cup \freenames{Q} \\
  \freenames{\dropn{x}} & := & \{ x \}
\end{eqnarray*}

The bound names of a process, $\boundnames{P}$, are those names occurring in $P$
that are not free. For example, in $x?(y).0$, the name $x$ is free, while $y$ is bound.

\begin{mathpar}
  \inferrule* [lab=monoidal-laws] {} { P|Q \equiv Q|P \and P|0 \equiv P \and P|(Q|R) \equiv (P|Q)|R }
\end{mathpar}

\begin{mathpar}
  \inferrule* [lab=alpha-equivalence] {} { (x)P \equiv (y)P\{y/x\} \and y \not\in \freenames{P} }
\end{mathpar}

\begin{definition}
Then two processes, $P,Q$, are alpha-equivalent if $P = Q\{\vec{y}/\vec{x}\}$ for
some $\vec{x} \in \boundnames{Q},\vec{y} \in \boundnames{P}$, where $Q\{\vec{y}/\vec{x}\}$
denotes the capture-avoiding substitution of $\vec{y}$ for $\vec{x}$ in $Q$.
\end{definition}

\begin{definition}
  The {\em structural congruence} \cite{SangiorgiWalker} , $\equiv$,
  between processes is the least congruence containing
  alpha-equivalence, satisfying the abelian monoid laws
  (associativity, commutativity and $\pzero$ as identity) for parallel
  composition $|$ and for summation $+$.
\end{definition}

\subsection{Name equivalence}

We take name equivalence, written $\nameeq$, to be the smallest
equivalence relation generated by the following rules.

\begin{mathpar}
\inferrule*[lab=Quote-drop]
{ }
{ \quotep{@{x}} \nameeq x }

\inferrule*[lab=Struct-equiv]
{ P \scong Q }
{ \quotep{P} \nameeq \quotep{Q} }
\end{mathpar}

The astute reader will have noticed that the mutual recursion of names
and processes imposes a mutual recursion on alpha-equivalence and
structural equivalence via name-equivalence. Fortunately, all of this
works out pleasantly and we may calculate in the natural way, free of
concern. The reader interested in the details is referred to the
appendix \ref{appendix:rho_details}.

\subsection{Substitution}

We use $\Proc$ for the set of processes, $\QProc$ for the set of
names, and $\id{\{}\vec{y} / \vec{x} \id{\}}$ to denote partial maps,
$s : \QProc \rightarrow \QProc$. A map, $s$ lifts, uniquely, to a map
on process terms, $\widehat{s} : \Proc \rightarrow \Proc$ by the
following equations.

\begin{mathpar}
  (0) \psubstp{Q}{P} := 0 \\
  (R \juxtap S) \psubstp{Q}{P}
  :=    
  (R)\psubstp{Q}{P} \juxtap (S) \psubstp{Q}{P} \\
  (x?(y).R) \psubstp{Q}{P}    
  :=    
  (x)\substp{Q}{P} (z)\concat( (R \psubstn{z}{y}) \psubstp{Q}{P} ) \\
  (\lift{x}{R}) \psubstp{Q}{P}  
  :=
  \lift{(x)\substp{Q}{P}}{ R \psubstp{Q}{P} } \\
%   (\dropn{x})  \psubstp{Q}{P}       
%   := 
%   \left\{ 
%     \begin{array}{ccc} 
%       \dropn{\quotep{Q}} & & x \nameeq \quotep{P} \\
%       \dropn{x} & & otherwise \\
%     \end{array}
%   \right. 
  (\dropn{x})  \psubstp{Q}{P}       
  := 
  \left\{ 
    \begin{array}{ccc} 
      Q & & x \nameeq \quotep{P} \\
      \dropn{x} & & otherwise \\
    \end{array}
  \right.
\end{mathpar}
 

where

\begin{eqnarray}
  (x)\id{\{} \lpquote Q \rpquote / \lpquote P \rpquote \id{\}}            = 
  \left\{ 
    \begin{array}{ccc}
      \lpquote Q \rpquote & & x \nameeq \lpquote P \rpquote \\
      x & & otherwise \\
    \end{array}
  \right. \nonumber
\end{eqnarray}

and $z$ is chosen distinct from $\quotep{P}$, $\quotep{Q}$, the free
names in $Q$, and all the names in $R$. Our $\alpha$-equivalence will
be built in the standard way from this substitution.

\begin{remark}\label{rem:no_self_referential_names}
  One consequence of these definitions is that $\forall P. \quotep{P}
  \not\in \freenames{P}$.
\end{remark}

\subsection{ Dynamic quote: an example }

Anticipating something of what's to come, consider applying the
substitution, $\widehat{\id{\{}u / z \id{\}}}$, to the following pair
of processes, $\lift{w}{y!(z)}$ and $w[ \lpquote y!(z) \rpquote ]$.

\begin{eqnarray}
	\lift{w}{y!(z)}\widehat{\id{\{}u / z \id{\}}}
		& = &
		\lift{w}{y!(u)} \nonumber\\
	w[ \lpquote y!(z) \rpquote ] \widehat{ \id{\{}u / z \id{\}} }
		& = &
		w[ \lpquote y!(z) \rpquote ] \nonumber
\end{eqnarray}

Because the body of the process between quotes is impervious to
substitution, we get radically different answers. In fact, by
examining the first process in an input context,
e.g. $x?(z).\lift{w}{y!(z)}$, we see that the process under the lift
operator may be shaped by prefixed inputs binding a name inside it. In
this sense, the lift operator will be seen as a way to dynamically
construct processes before reifying them as names.

Finally equipped with these standard features we can present the
dynamics of the calculus.

\subsubsection{Operational semantics} 

Finally, we introduce the computational dynamics. What marks these
algebras as distinct from other more traditionally studied algebraic
structures, e.g. vector spaces or polynomial rings, is the manner in
which dynamics is captured. In traditional structures, dynamics is typically
expressed through morphisms between such structures, as in linear maps
between vector spaces or morphisms between rings. In algebras
associated with the semantics of computation, the dynamics is
expressed as part of the algebraic structure itself, through a
reduction reduction relation typically denoted by $\red$. Below, we
give a recursive presentation of this relation for the calculus used
in the encoding.

$\red \subseteq \pi \times \pi$
$\red : \pi \to \mathcal{P}(\pi)$

\begin{mathpar}
  \inferrule* [lab=Comm] { \textsf{match}( x_{src}, x_{trgt} ) } { x_{trgt}?(y)P \; | \; x_{src}!\langle {Q} \rangle \red P\{\quotep{Q}/y}\} }
  \and \\
  \inferrule* [lab=Par] {{P} \red {P}'} {{{P} | {Q}} \red {{P}' | {Q}}}
  \and
  \inferrule* [lab=Equiv]{{{P} \scong {P}'} \andalso {{P}' \red {Q}'} \andalso {{Q}' \scong {Q}}}{{P} \red {Q}}
\end{mathpar}

\begin{eqnarray*}
  match_{\equiv} (\quotep{P},\quotep{Q}) & := & P \equiv Q \\
  match_{\dagger}(\quotep{P},\quotep{Q}) & := & \forall R. P|Q \red^{*} R => R \red^{*} 0 \\
  match_{K}(\quotep{P},\quotep{Q}) & := & K \mbox{ for some context } K
\end{eqnarray*}

$u?(x)P | u!\langle Q \rangle \red P\{\quotep{Q}/x\}$

%We write $\wred$ for $\red^*$, and $P\red$ if $\exists Q $ such that $ P \red Q$.
We write $P\red$ if $\exists Q $ such that $ P \red Q$ and $P\not\red$, otherwise.

\section{Replication}

As mentioned before, it is known that replication (and hence
recursion) can be implemented in a higher-order process algebra
\cite{SangiorgiWalker}. As our first example of calculation with the
machinery thus far presented we give the construction explicitly in
the {\rhoc}.

\begin{eqnarray}
	D_{x} & := & \prefix{x}{y}{(\binpar{\outputp{x}{y}}{@{y}})} \nonumber\\
	\bangp_{x}{P} & := & \binpar{{x}!\langle{\binpar{D_{x}}{P}}\rangle}{D_{x}} \nonumber
\end{eqnarray}

\begin{eqnarray}
	\bangp_{x}{P} & & \nonumber\\
	=
	& {x}!\langle{(\prefix{x}{y}{(\outputp{x}{y} | @{y})) | P}}\rangle 
	      | \prefix{x}{y}{(\outputp{x}{y} | @{y})} & \nonumber\\
	\red
	& (\outputp{x}{y} | @{y})\substn{\quotep{(\prefix{x}{y}{(@{y} | \outputp{x}{y})) | P}}}{y} & \nonumber\\
	=
	& \outputp{x}{\quotep{(\prefix{x}{y}{(\outputp{x}{y} | @{y})) | P}}}
	  | {(\prefix{x}{y}{(\outputp{x}{y} | @{y})) | P}} & \nonumber\\
	\red
	& \ldots & \nonumber\\
	\red^*
	& P | P | \ldots & \nonumber
\end{eqnarray}

Of course, this encoding, as an implementation, runs away, unfolding
$\bangp{P}$ eagerly. A lazier and more implementable replication
operator, restricted to input-guarded processes, may be obtained as follows.

\begin{eqnarray}
\bangp{\prefix{u}{v}{P}} 
	:= 
	\binpar{\lift{x}{\prefix{u}{v}{(\binpar{D(x)}{P})}}}{D(x)} \nonumber
\end{eqnarray}

\begin{remark}
  Note that the lazier definition still does not deal with summation
  or mixed summation (i.e. sums over input and output). The reader is
  invited to construct definitions of replication that deal with these
  features. 

  Further, the definitions are parameterized in a name, $x$. Can you,
  gentle reader, make a definition that eliminates this parameter and
  guarantees no accidental interaction between the replication
  machinery and the process being replicated -- i.e. no accidental
  sharing of names used by the process to get its work done and the
  name(s) used by the replication to effect copying. This latter
  revision of the definition of replication is crucial to obtaining
  the expected identity $!!P \sim !P$.
\end{remark}

\begin{remark}\label{rem:paradoxical_combinator}
  The reader familiar with the lambda calculus will have noticed the
  similarity between $D$ and the paradoxical combinator.

  [Ed. note: the existence of this seems to suggest we have to be more
  restrictive on the set of processes and names we admit if we are to
  support no-cloning.]
\end{remark}

\subsubsection{Bisimulation}

The computational dynamics gives rise to another kind of equivalence,
the equivalence of computational behavior. As previously mentioned
this is typically captured \emph{via} some form of bisimulation.

% The notion we use in this paper is weak barbed bisimulation
% \cite{milner91polyadicpi}.

The notion we use in this paper is derived from weak barbed
bisimulation \cite{milner91polyadicpi}. 

\begin{definition}
An \emph{observation relation}, $\downarrow_{\mathcal N}$, over a set
of names, $\mathcal N$, is the smallest relation satisfying the rules
below.

\infrule[Out-barb]{y \in {\mathcal N}, \; x \nameeq y}
		  {\outputp{x}{v} \downarrow_{\mathcal N} x}
\infrule[Par-barb]{\mbox{$P\downarrow_{\mathcal N} x$ or $Q\downarrow_{\mathcal N} x$}}
		  {\binpar{P}{Q} \downarrow_{\mathcal N} x}

We write $P \Downarrow_{\mathcal N} x$ if there is $Q$ such that 
$P \wred Q$ and $Q \downarrow_{\mathcal N} x$.
\end{definition}

\begin{definition}
%\label{def.bbisim}
An  ${\mathcal N}$-\emph{barbed bisimulation} over a set of names, ${\mathcal N}$, is a symmetric binary relation 
${\mathcal S}_{\mathcal N}$ between agents such that $P\rel{S}_{\mathcal N}Q$ implies:
\begin{enumerate}
\item If $P \red P'$ then $Q \wred Q'$ and $P'\rel{S}_{\mathcal N} Q'$.
\item If $P\downarrow_{\mathcal N} x$, then $Q\Downarrow_{\mathcal N} x$.
\end{enumerate}
$P$ is ${\mathcal N}$-barbed bisimilar to $Q$, written
$P \wbbisim_{\mathcal N} Q$, if $P \rel{S}_{\mathcal N} Q$ for some ${\mathcal N}$-barbed bisimulation ${\mathcal S}_{\mathcal N}$.
\end{definition}

$\mathcal{R} \subseteq \pi \times \pi$

$P \mathcal{R} Q => \forall P'. P \red P' \Rightarrow \exists Q'. Q \red Q', P' \mathcal{R} Q'$

$P \vdash x \Rightarrow Q \vdash x$

\begin{mathpar}
  \inferrule*[lab=Out-barb]{x \nameeq y}{{y}!\langle{Q}\rangle \vdash x}
  \and
  \inferrule*[lab=Par-barb]{\mbox{$P\vdash x$ or $Q\vdash x$}}{\binpar{P}{Q} \vdash x}
\end{mathpar}

\subsubsection{Contexts}

One of the principle advantages of computational calculi like the
$\pi$-calculus is a well-defined notion of context,
contextual-equivalence and a correlation between
contextual-equivalence and notions of bisimulation. The notion of
context allows the decomposition of a process into (sub-)process and
its syntactic environment, its context. Thus, a context may be
thought of as a process with a ``hole'' (written $\Box$) in it. The
application of a context $M$ to a process $P$, written $M[P]$, is
tantamount to filling the hole in $M$ with $P$. In this paper we do
not need the full weight of this theory, but do make use of the notion
of context in the proof the main theorem. 

\begin{mathpar}
  \inferrule* [lab=summation] {} {{M_{M},M_{N}} \bc \Box \;|\; x.M_{A} \;|\; M_{M}+M_{N}}
  \and
  \inferrule* [lab=agent] {} {{M_{A}} \bc (\vec{x})M_{P} \;| \; \clift{P_0,\ldots,M_{P},\ldots,P_N}}
  \and \\
  \inferrule* [lab=process] {} {{M_{P}} \bc M_{N} \;| \;P|M_{P} }
\end{mathpar} 

\begin{mathpar}
  \inferrule* [lab=sychronization] {} {M_{N} \bc \Box \;|\; x?M_{F} \;|\; x!M_{C}}
  \and
  \inferrule* [lab=abstraction] {} {{M_{F}} \bc (x)M_{P} }
  \and
  \inferrule* [lab=concretion] {} {{M_{C}} \bc \langle M_{P} \rangle }
  \and \\
  \inferrule* [lab=process] {} {{M_{P}} \bc M_{N} \;| \;P|M_{P} }
\end{mathpar}

\begin{definition}[contextual application] Given a context $M$, and
  process $P$, we define the \emph{contextual application}, $M[P] :=
  M\{P/\Box\}$. That is, the contextual application of M to P is the
  substitution of $P$ for $\Box$ in $M$.
\end{definition}

$\meaningof{-} : L \to \mathcal{P}(\pi)$

\begin{mathpar}
  \inferrule* [lab=collection] {} {\meaningof{true} = \pi, \and \meaningof{~E} = \pi \setminus \meaningof{E}, \and \meaningof{E_{1} \& E_{2}} = \meaningof{E_{1}} \cap \meaningof{E_{2}}}
\end{mathpar}

\begin{mathpar}
  \inferrule* [lab=structure] {} {\meaningof{0} = \{ P \in \pi | P \equiv 0 \}, \and \\ \meaningof{E_1 | E_2} = \{ P \in \pi | P \equiv P_{1} | P_{2}, P_{1} \in \meaningof{E_{1}}, P_{2} \in \meaningof{E_2}\} }
\end{mathpar}

\begin{mathpar}
 \inferrule* [lab=behavior] {} {\meaningof{\langle a?b \rangle E} = \{ P \in \pi | P \equiv Q | u?(y)P', \\ \and \\\\ \and \\ \;\;\; u \in \meaningof{a}, \forall z.P'\{z/y\} \in \meaningof{E\{z/b\}}\}, \and \\ \meaningof{a!E} = \{ P \in \pi | P \equiv Q | x!\langle P' \rangle, x \in \meaningof{a} P' \in \meaningof{E}\} }
\end{mathpar}

\begin{mathpar}
 \inferrule* [lab=nominal] {} {\meaningof{\quotep{E}} = \{ \quotep{P} \in \quotep{\pi} | P \in \meaningof{E} \}, \and \meaningof{\quotep{P}} = \{ \quotep{Q} \in \quotep{\pi} | P \equiv Q \} \and \\ \meaningof{@\quotep{E}} = \{ P \in \pi | P \equiv @x, x \in \meaningof{E} \}}
\end{mathpar}

\begin{eqnarray*}
  \\
  \meaningof{-} : TS \to ST
\end{eqnarray*}

\begin{eqnarray*}
  \\
  L : TS \to ST
\end{eqnarray*}

\begin{eqnarray*}
  \\
  P \models E \iff P \in \meaningof{E}
\end{eqnarray*}

\begin{eqnarray*}
  P \approx_{L} Q \iff \forall E \in L. P \models E \iff Q \models E
\end{eqnarray*}

\begin{eqnarray*}
  P \approx_{K} Q
\end{eqnarray*}

\begin{eqnarray*}
  P \approx Q
\end{eqnarray*}

$\approx_{K} = \approx = \approx_{L}$

\subsubsection{Contextual duality}

Note that contexts extend the quotation operation to a family of
operations from processes to names. Given a context, $M$, we can
define a \emph{nominal context}, $\quotep{M}$ by $\quotep{M}[P] :=
\quotep{M[P]}$. To foreshadow what is to come we observe that these
operations enjoy a duality with processes very much like the duality
between vectors and maps from vectors to scalars.

Further, because the calculus is essentially higher-order, we have a
correspondence between contexts and processes. More specifically,
given a name $x$ and a context $M$ we can construct $M^{*}_{x}$ such
that 

\begin{mathpar}
  M^{*}_{x} | \lift{x}{P} \red M[P]
\end{mathpar}

namely,

\begin{mathpar}
  M^{*}_{x} := x?(u).M[\dropn{u}]
\end{mathpar}

The dependence of $M^{*}_{x}$ on a name makes it an abstraction, 

\begin{mathpar}
  M^{*} := (x)x?(u).M[\dropn{u}]
\end{mathpar}

\subsection{Additional notation}

It will sometimes be convenient to denote the process a name
quotes. We already have the notation $x = \quotep{P}$, but it will be
convenient to introduce an alternate notation, $\procn{x}$, when we
want to emphasize the connection to the use of the name. Note that, by
virtue of name equivalence, $\quotep{\procn{x}} \nameeq x$; so, the
notation is consistent with previous definitions.

Further, because names have structure it is possible to effect
substitutions on the basis of that structure. This means we need to
upgrade our notation for substitutions, which we accomplish by
adapting comprehension notation. Thus,

\begin{mathpar}
  P\{ y / x : x \in S \}
\end{mathpar}

is interpreted to mean the process derived from P by replacing (in a
capture-avoiding manner) each occurrence of $x$ in $S$ by $y$. For example,

\begin{mathpar}
  P\{ \quotep{\procn{x}|\procn{x}} / x : x \in \freenames{P} \}
\end{mathpar}

will replace each (occurrence) of a free name $x$ in $P$ by
$\quotep{\procn{x}|\procn{x}}$.

Also, we will avail ourselves of the notation $x^{L}$ and $x^{R}$ to
denote injections of a name into disjoint copies of the name
space. There are numerous ways to accomplish this. One example can be
found in \cite{MeredithR05}. This notation overloads to vectors of
names: $\vec{x}^{\pi} := (x_{i}^{\pi} \; : \; 0 \leq i < |\vec{x}| )$ where $\pi \in \{L,R\}$.

We also use $P^{\Box} := P|\Box$.

In \cite{MeredithR05} an interpretation of the new operator is
given. It turns out that there are several possible interpretations
all enjoying the requisite algebraic properties of the operator (see
\cite{milner91polyadicpi}). We will therefore make liberal use of
$(\nu\; \vec{x})P$.

% subsection the_syntax_and_semantics_of_the_notation_system (end)   

\input{qm2pi.qmops} 

\input{qm2pi.sterngerlach} 

\input{qm2pi.metric} 

% section concurrent_process_calculi (end)

%\input{qm2pi.proofsketch}

% section proof sketch (end)

%\input{qm2pi.slviaknots} 

% section spatial logic via knots (end)

\input{qm2pi.conclusion}

% section conclusion (end)

%\input{qm2pi.dtcodes} 

% section wiring algorithm (end)

\input{qm2pi.ack} 

% section acknowledgments (end)

\newpage


\bibliographystyle{plain}   
\bibliography{../../biblios/main.bib}

\input{qm2pi.rhodetails}

\end{document}



\end{document}

 

% section concurrent_process_calculi (end)

%\documentclass[12pt]{llncs}
%\documentclass{jktr}

\usepackage[pdftex]{hyperref}                   
\usepackage {listings}
\usepackage {mathpartir}
\usepackage{bcprules}
%\usepackage{listings}
                       
\usepackage{graphicx} 
%\usepackage[margins=2.5cm,nohead,nofoot]{geometry}
%\usepackage{geometry}
\usepackage{amsfonts}
\usepackage{amstext}
\usepackage{latexsym}
\usepackage{amssymb}
\usepackage{color}


%\include{myPreamble}
\documentclass[12pt]{llncs}
%\documentclass{jktr}

\usepackage[pdftex]{hyperref}                   
\usepackage {listings}
\usepackage {mathpartir}
\usepackage{bcprules}
%\usepackage{listings}
                       
\usepackage{graphicx} 
%\usepackage[margins=2.5cm,nohead,nofoot]{geometry}
%\usepackage{geometry}
\usepackage{amsfonts}
\usepackage{amstext}
\usepackage{latexsym}
\usepackage{amssymb}
\usepackage{color}


%\include{myPreamble}
\include{qm2pi.local} 

%\ifpdf
%\usepackage[pdftex]{graphicx}
%\else
%\usepackage{graphicx}
%\fi

 % \ifpdf
%  \usepackage{pdfsync}
%  \if


%\title{Brief Article}
%\author{David F. Snyder}
%\author{L.G. Meredith}

%\address{Dept. of Math., Texas State University--San Marcos, San Marcos, TX 78666}
       
\pagestyle{empty}


\begin{document}

\lstset{language=[Objective]Caml,frame=shadowbox}

\input{qm2pi.front}

% section front matter (end)

\input{qm2pi.intro} 
 
% section introduction (end)

% \input{qm2pi.knotations} 

% section notation (end)

\input{qm2pi.process.calculi} 

% section concurrent_process_calculi_and_spatial_logics_ (end)
    
%\input{qm2pi.knots2pi} 

%\input{qm2pi.trefoil} 

%\input{qm2pi.mainthm} 

% subsection basic_interpretation (end)

%\input{qm2pi.rho.presentation} 
\subsection{The syntax and semantics of the notation system}\label{sub:the_syntax_and_semantics_of_the_notation_system} % (fold)

We now summarize a technical presentation of the calculus that
embodies our theory of dynamics. The typical presentation of such a
calculus follows the style of giving generators and relations on
them. The grammar, below, describing term constructors, freely
generates the set of processes, $\Proc$. This set is then quotiented
by a relation known as structural congruence and it is over this set
that the notion of dynamics is expressed. This presentation is
essentially that of \cite{MeredithR05} with the addition of
polyadicity and summation. For readability we have relegated some of
the technical subtleties to an appendix.

\subsubsection{Process grammar}\label{subsub:process_grammar}

\begin{mathpar}
  \inferrule* [lab=synchronization] {} {{M} \bc \pzero \;|\; x?F \;|\; x!C }
  \and
  \inferrule* [lab=abstraction] {} {{F} \bc (x)P}
  \and
  \inferrule* [lab=concretion] {} {{C} \bc \langle Q \rangle}
  \and
  \inferrule* [lab=process] {} {{P,Q} \bc M \;| \;P|Q \;|\; @{x}}
  \and
  \inferrule* [lab=name] {} {{x} \bc \quotep{P}}
\end{mathpar} 

Note that $\vec{x}$ (resp. $\vec{P}$) denotes a vector of names
(resp. processes) of length $|\vec{x}|$ (resp. $|\vec{P}|$). We adopt
the following useful abbreviations.

\begin{mathpar}
   x?(\vec{y}).P := x.(\vec{y})P \and  x\clift{\vec{P}} := x.\clift{\vec{P}}
   \and x!(y) := \lift{x}{\dropn{y}}
   \and \Pi_{i=0}^{n-1}P_i := P_0 | \ldots | P_{n-1}
\end{mathpar}

\subsubsection{Structural congruence}

\paragraph{Free and bound names and alpha-equivalence.} At the
core of structural equivalence is alpha-equivalence which identifies
process that are the same up to a change of variable. Formally, we
recognize the distinction between free and bound names. The free names
of a process, $\freenames{P}$, may be calculated recursively as
follows:

\begin{mathpar}
\freenames{\pzero} := \emptyset
  \and \\
  \freenames{x?(y).P} := \{ x \} \cup (\freenames{P} \setminus \{ y \})
  \and 
  \freenames{x!\langle P \rangle} := \{ x \} \cup \{ P \} 
  \and \\
  \freenames{P|Q} := \freenames{P} \cup \freenames{Q}
  \and \\
  \freenames{@{x}} := \{ x \}
\end{mathpar}

$\pi$
$\quotep{\pi}$

$\freenames{-} : \pi \to \mathcal{P}(\quotep{\pi})$

\begin{eqnarray*}
  \freenames{\pzero} & := & \emptyset \\
  \freenames{x?(y).P} & := & \{ x \} \cup (\freenames{P} \setminus \{ y \}) \\
  \freenames{x!\langle P \rangle} & := & \{ x \} \cup \{ P \} \\
  \freenames{P|Q} & := & \freenames{P} \cup \freenames{Q} \\
  \freenames{\dropn{x}} & := & \{ x \}
\end{eqnarray*}

The bound names of a process, $\boundnames{P}$, are those names occurring in $P$
that are not free. For example, in $x?(y).0$, the name $x$ is free, while $y$ is bound.

\begin{mathpar}
  \inferrule* [lab=monoidal-laws] {} { P|Q \equiv Q|P \and P|0 \equiv P \and P|(Q|R) \equiv (P|Q)|R }
\end{mathpar}

\begin{mathpar}
  \inferrule* [lab=alpha-equivalence] {} { (x)P \equiv (y)P\{y/x\} \and y \not\in \freenames{P} }
\end{mathpar}

\begin{definition}
Then two processes, $P,Q$, are alpha-equivalent if $P = Q\{\vec{y}/\vec{x}\}$ for
some $\vec{x} \in \boundnames{Q},\vec{y} \in \boundnames{P}$, where $Q\{\vec{y}/\vec{x}\}$
denotes the capture-avoiding substitution of $\vec{y}$ for $\vec{x}$ in $Q$.
\end{definition}

\begin{definition}
  The {\em structural congruence} \cite{SangiorgiWalker} , $\equiv$,
  between processes is the least congruence containing
  alpha-equivalence, satisfying the abelian monoid laws
  (associativity, commutativity and $\pzero$ as identity) for parallel
  composition $|$ and for summation $+$.
\end{definition}

\subsection{Name equivalence}

We take name equivalence, written $\nameeq$, to be the smallest
equivalence relation generated by the following rules.

\begin{mathpar}
\inferrule*[lab=Quote-drop]
{ }
{ \quotep{@{x}} \nameeq x }

\inferrule*[lab=Struct-equiv]
{ P \scong Q }
{ \quotep{P} \nameeq \quotep{Q} }
\end{mathpar}

The astute reader will have noticed that the mutual recursion of names
and processes imposes a mutual recursion on alpha-equivalence and
structural equivalence via name-equivalence. Fortunately, all of this
works out pleasantly and we may calculate in the natural way, free of
concern. The reader interested in the details is referred to the
appendix \ref{appendix:rho_details}.

\subsection{Substitution}

We use $\Proc$ for the set of processes, $\QProc$ for the set of
names, and $\id{\{}\vec{y} / \vec{x} \id{\}}$ to denote partial maps,
$s : \QProc \rightarrow \QProc$. A map, $s$ lifts, uniquely, to a map
on process terms, $\widehat{s} : \Proc \rightarrow \Proc$ by the
following equations.

\begin{mathpar}
  (0) \psubstp{Q}{P} := 0 \\
  (R \juxtap S) \psubstp{Q}{P}
  :=    
  (R)\psubstp{Q}{P} \juxtap (S) \psubstp{Q}{P} \\
  (x?(y).R) \psubstp{Q}{P}    
  :=    
  (x)\substp{Q}{P} (z)\concat( (R \psubstn{z}{y}) \psubstp{Q}{P} ) \\
  (\lift{x}{R}) \psubstp{Q}{P}  
  :=
  \lift{(x)\substp{Q}{P}}{ R \psubstp{Q}{P} } \\
%   (\dropn{x})  \psubstp{Q}{P}       
%   := 
%   \left\{ 
%     \begin{array}{ccc} 
%       \dropn{\quotep{Q}} & & x \nameeq \quotep{P} \\
%       \dropn{x} & & otherwise \\
%     \end{array}
%   \right. 
  (\dropn{x})  \psubstp{Q}{P}       
  := 
  \left\{ 
    \begin{array}{ccc} 
      Q & & x \nameeq \quotep{P} \\
      \dropn{x} & & otherwise \\
    \end{array}
  \right.
\end{mathpar}
 

where

\begin{eqnarray}
  (x)\id{\{} \lpquote Q \rpquote / \lpquote P \rpquote \id{\}}            = 
  \left\{ 
    \begin{array}{ccc}
      \lpquote Q \rpquote & & x \nameeq \lpquote P \rpquote \\
      x & & otherwise \\
    \end{array}
  \right. \nonumber
\end{eqnarray}

and $z$ is chosen distinct from $\quotep{P}$, $\quotep{Q}$, the free
names in $Q$, and all the names in $R$. Our $\alpha$-equivalence will
be built in the standard way from this substitution.

\begin{remark}\label{rem:no_self_referential_names}
  One consequence of these definitions is that $\forall P. \quotep{P}
  \not\in \freenames{P}$.
\end{remark}

\subsection{ Dynamic quote: an example }

Anticipating something of what's to come, consider applying the
substitution, $\widehat{\id{\{}u / z \id{\}}}$, to the following pair
of processes, $\lift{w}{y!(z)}$ and $w[ \lpquote y!(z) \rpquote ]$.

\begin{eqnarray}
	\lift{w}{y!(z)}\widehat{\id{\{}u / z \id{\}}}
		& = &
		\lift{w}{y!(u)} \nonumber\\
	w[ \lpquote y!(z) \rpquote ] \widehat{ \id{\{}u / z \id{\}} }
		& = &
		w[ \lpquote y!(z) \rpquote ] \nonumber
\end{eqnarray}

Because the body of the process between quotes is impervious to
substitution, we get radically different answers. In fact, by
examining the first process in an input context,
e.g. $x?(z).\lift{w}{y!(z)}$, we see that the process under the lift
operator may be shaped by prefixed inputs binding a name inside it. In
this sense, the lift operator will be seen as a way to dynamically
construct processes before reifying them as names.

Finally equipped with these standard features we can present the
dynamics of the calculus.

\subsubsection{Operational semantics} 

Finally, we introduce the computational dynamics. What marks these
algebras as distinct from other more traditionally studied algebraic
structures, e.g. vector spaces or polynomial rings, is the manner in
which dynamics is captured. In traditional structures, dynamics is typically
expressed through morphisms between such structures, as in linear maps
between vector spaces or morphisms between rings. In algebras
associated with the semantics of computation, the dynamics is
expressed as part of the algebraic structure itself, through a
reduction reduction relation typically denoted by $\red$. Below, we
give a recursive presentation of this relation for the calculus used
in the encoding.

$\red \subseteq \pi \times \pi$
$\red : \pi \to \mathcal{P}(\pi)$

\begin{mathpar}
  \inferrule* [lab=Comm] { \textsf{match}( x_{src}, x_{trgt} ) } { x_{trgt}?(y)P \; | \; x_{src}!\langle {Q} \rangle \red P\{\quotep{Q}/y}\} }
  \and \\
  \inferrule* [lab=Par] {{P} \red {P}'} {{{P} | {Q}} \red {{P}' | {Q}}}
  \and
  \inferrule* [lab=Equiv]{{{P} \scong {P}'} \andalso {{P}' \red {Q}'} \andalso {{Q}' \scong {Q}}}{{P} \red {Q}}
\end{mathpar}

\begin{eqnarray*}
  match_{\equiv} (\quotep{P},\quotep{Q}) & := & P \equiv Q \\
  match_{\dagger}(\quotep{P},\quotep{Q}) & := & \forall R. P|Q \red^{*} R => R \red^{*} 0 \\
  match_{K}(\quotep{P},\quotep{Q}) & := & K \mbox{ for some context } K
\end{eqnarray*}

$u?(x)P | u!\langle Q \rangle \red P\{\quotep{Q}/x\}$

%We write $\wred$ for $\red^*$, and $P\red$ if $\exists Q $ such that $ P \red Q$.
We write $P\red$ if $\exists Q $ such that $ P \red Q$ and $P\not\red$, otherwise.

\section{Replication}

As mentioned before, it is known that replication (and hence
recursion) can be implemented in a higher-order process algebra
\cite{SangiorgiWalker}. As our first example of calculation with the
machinery thus far presented we give the construction explicitly in
the {\rhoc}.

\begin{eqnarray}
	D_{x} & := & \prefix{x}{y}{(\binpar{\outputp{x}{y}}{@{y}})} \nonumber\\
	\bangp_{x}{P} & := & \binpar{{x}!\langle{\binpar{D_{x}}{P}}\rangle}{D_{x}} \nonumber
\end{eqnarray}

\begin{eqnarray}
	\bangp_{x}{P} & & \nonumber\\
	=
	& {x}!\langle{(\prefix{x}{y}{(\outputp{x}{y} | @{y})) | P}}\rangle 
	      | \prefix{x}{y}{(\outputp{x}{y} | @{y})} & \nonumber\\
	\red
	& (\outputp{x}{y} | @{y})\substn{\quotep{(\prefix{x}{y}{(@{y} | \outputp{x}{y})) | P}}}{y} & \nonumber\\
	=
	& \outputp{x}{\quotep{(\prefix{x}{y}{(\outputp{x}{y} | @{y})) | P}}}
	  | {(\prefix{x}{y}{(\outputp{x}{y} | @{y})) | P}} & \nonumber\\
	\red
	& \ldots & \nonumber\\
	\red^*
	& P | P | \ldots & \nonumber
\end{eqnarray}

Of course, this encoding, as an implementation, runs away, unfolding
$\bangp{P}$ eagerly. A lazier and more implementable replication
operator, restricted to input-guarded processes, may be obtained as follows.

\begin{eqnarray}
\bangp{\prefix{u}{v}{P}} 
	:= 
	\binpar{\lift{x}{\prefix{u}{v}{(\binpar{D(x)}{P})}}}{D(x)} \nonumber
\end{eqnarray}

\begin{remark}
  Note that the lazier definition still does not deal with summation
  or mixed summation (i.e. sums over input and output). The reader is
  invited to construct definitions of replication that deal with these
  features. 

  Further, the definitions are parameterized in a name, $x$. Can you,
  gentle reader, make a definition that eliminates this parameter and
  guarantees no accidental interaction between the replication
  machinery and the process being replicated -- i.e. no accidental
  sharing of names used by the process to get its work done and the
  name(s) used by the replication to effect copying. This latter
  revision of the definition of replication is crucial to obtaining
  the expected identity $!!P \sim !P$.
\end{remark}

\begin{remark}\label{rem:paradoxical_combinator}
  The reader familiar with the lambda calculus will have noticed the
  similarity between $D$ and the paradoxical combinator.

  [Ed. note: the existence of this seems to suggest we have to be more
  restrictive on the set of processes and names we admit if we are to
  support no-cloning.]
\end{remark}

\subsubsection{Bisimulation}

The computational dynamics gives rise to another kind of equivalence,
the equivalence of computational behavior. As previously mentioned
this is typically captured \emph{via} some form of bisimulation.

% The notion we use in this paper is weak barbed bisimulation
% \cite{milner91polyadicpi}.

The notion we use in this paper is derived from weak barbed
bisimulation \cite{milner91polyadicpi}. 

\begin{definition}
An \emph{observation relation}, $\downarrow_{\mathcal N}$, over a set
of names, $\mathcal N$, is the smallest relation satisfying the rules
below.

\infrule[Out-barb]{y \in {\mathcal N}, \; x \nameeq y}
		  {\outputp{x}{v} \downarrow_{\mathcal N} x}
\infrule[Par-barb]{\mbox{$P\downarrow_{\mathcal N} x$ or $Q\downarrow_{\mathcal N} x$}}
		  {\binpar{P}{Q} \downarrow_{\mathcal N} x}

We write $P \Downarrow_{\mathcal N} x$ if there is $Q$ such that 
$P \wred Q$ and $Q \downarrow_{\mathcal N} x$.
\end{definition}

\begin{definition}
%\label{def.bbisim}
An  ${\mathcal N}$-\emph{barbed bisimulation} over a set of names, ${\mathcal N}$, is a symmetric binary relation 
${\mathcal S}_{\mathcal N}$ between agents such that $P\rel{S}_{\mathcal N}Q$ implies:
\begin{enumerate}
\item If $P \red P'$ then $Q \wred Q'$ and $P'\rel{S}_{\mathcal N} Q'$.
\item If $P\downarrow_{\mathcal N} x$, then $Q\Downarrow_{\mathcal N} x$.
\end{enumerate}
$P$ is ${\mathcal N}$-barbed bisimilar to $Q$, written
$P \wbbisim_{\mathcal N} Q$, if $P \rel{S}_{\mathcal N} Q$ for some ${\mathcal N}$-barbed bisimulation ${\mathcal S}_{\mathcal N}$.
\end{definition}

$\mathcal{R} \subseteq \pi \times \pi$

$P \mathcal{R} Q => \forall P'. P \red P' \Rightarrow \exists Q'. Q \red Q', P' \mathcal{R} Q'$

$P \vdash x \Rightarrow Q \vdash x$

\begin{mathpar}
  \inferrule*[lab=Out-barb]{x \nameeq y}{{y}!\langle{Q}\rangle \vdash x}
  \and
  \inferrule*[lab=Par-barb]{\mbox{$P\vdash x$ or $Q\vdash x$}}{\binpar{P}{Q} \vdash x}
\end{mathpar}

\subsubsection{Contexts}

One of the principle advantages of computational calculi like the
$\pi$-calculus is a well-defined notion of context,
contextual-equivalence and a correlation between
contextual-equivalence and notions of bisimulation. The notion of
context allows the decomposition of a process into (sub-)process and
its syntactic environment, its context. Thus, a context may be
thought of as a process with a ``hole'' (written $\Box$) in it. The
application of a context $M$ to a process $P$, written $M[P]$, is
tantamount to filling the hole in $M$ with $P$. In this paper we do
not need the full weight of this theory, but do make use of the notion
of context in the proof the main theorem. 

\begin{mathpar}
  \inferrule* [lab=summation] {} {{M_{M},M_{N}} \bc \Box \;|\; x.M_{A} \;|\; M_{M}+M_{N}}
  \and
  \inferrule* [lab=agent] {} {{M_{A}} \bc (\vec{x})M_{P} \;| \; \clift{P_0,\ldots,M_{P},\ldots,P_N}}
  \and \\
  \inferrule* [lab=process] {} {{M_{P}} \bc M_{N} \;| \;P|M_{P} }
\end{mathpar} 

\begin{mathpar}
  \inferrule* [lab=sychronization] {} {M_{N} \bc \Box \;|\; x?M_{F} \;|\; x!M_{C}}
  \and
  \inferrule* [lab=abstraction] {} {{M_{F}} \bc (x)M_{P} }
  \and
  \inferrule* [lab=concretion] {} {{M_{C}} \bc \langle M_{P} \rangle }
  \and \\
  \inferrule* [lab=process] {} {{M_{P}} \bc M_{N} \;| \;P|M_{P} }
\end{mathpar}

\begin{definition}[contextual application] Given a context $M$, and
  process $P$, we define the \emph{contextual application}, $M[P] :=
  M\{P/\Box\}$. That is, the contextual application of M to P is the
  substitution of $P$ for $\Box$ in $M$.
\end{definition}

$\meaningof{-} : L \to \mathcal{P}(\pi)$

\begin{mathpar}
  \inferrule* [lab=collection] {} {\meaningof{true} = \pi, \and \meaningof{~E} = \pi \setminus \meaningof{E}, \and \meaningof{E_{1} \& E_{2}} = \meaningof{E_{1}} \cap \meaningof{E_{2}}}
\end{mathpar}

\begin{mathpar}
  \inferrule* [lab=structure] {} {\meaningof{0} = \{ P \in \pi | P \equiv 0 \}, \and \\ \meaningof{E_1 | E_2} = \{ P \in \pi | P \equiv P_{1} | P_{2}, P_{1} \in \meaningof{E_{1}}, P_{2} \in \meaningof{E_2}\} }
\end{mathpar}

\begin{mathpar}
 \inferrule* [lab=behavior] {} {\meaningof{\langle a?b \rangle E} = \{ P \in \pi | P \equiv Q | u?(y)P', \\ \and \\\\ \and \\ \;\;\; u \in \meaningof{a}, \forall z.P'\{z/y\} \in \meaningof{E\{z/b\}}\}, \and \\ \meaningof{a!E} = \{ P \in \pi | P \equiv Q | x!\langle P' \rangle, x \in \meaningof{a} P' \in \meaningof{E}\} }
\end{mathpar}

\begin{mathpar}
 \inferrule* [lab=nominal] {} {\meaningof{\quotep{E}} = \{ \quotep{P} \in \quotep{\pi} | P \in \meaningof{E} \}, \and \meaningof{\quotep{P}} = \{ \quotep{Q} \in \quotep{\pi} | P \equiv Q \} \and \\ \meaningof{@\quotep{E}} = \{ P \in \pi | P \equiv @x, x \in \meaningof{E} \}}
\end{mathpar}

\begin{eqnarray*}
  \\
  \meaningof{-} : TS \to ST
\end{eqnarray*}

\begin{eqnarray*}
  \\
  L : TS \to ST
\end{eqnarray*}

\begin{eqnarray*}
  \\
  P \models E \iff P \in \meaningof{E}
\end{eqnarray*}

\begin{eqnarray*}
  P \approx_{L} Q \iff \forall E \in L. P \models E \iff Q \models E
\end{eqnarray*}

\begin{eqnarray*}
  P \approx_{K} Q
\end{eqnarray*}

\begin{eqnarray*}
  P \approx Q
\end{eqnarray*}

$\approx_{K} = \approx = \approx_{L}$

\subsubsection{Contextual duality}

Note that contexts extend the quotation operation to a family of
operations from processes to names. Given a context, $M$, we can
define a \emph{nominal context}, $\quotep{M}$ by $\quotep{M}[P] :=
\quotep{M[P]}$. To foreshadow what is to come we observe that these
operations enjoy a duality with processes very much like the duality
between vectors and maps from vectors to scalars.

Further, because the calculus is essentially higher-order, we have a
correspondence between contexts and processes. More specifically,
given a name $x$ and a context $M$ we can construct $M^{*}_{x}$ such
that 

\begin{mathpar}
  M^{*}_{x} | \lift{x}{P} \red M[P]
\end{mathpar}

namely,

\begin{mathpar}
  M^{*}_{x} := x?(u).M[\dropn{u}]
\end{mathpar}

The dependence of $M^{*}_{x}$ on a name makes it an abstraction, 

\begin{mathpar}
  M^{*} := (x)x?(u).M[\dropn{u}]
\end{mathpar}

\subsection{Additional notation}

It will sometimes be convenient to denote the process a name
quotes. We already have the notation $x = \quotep{P}$, but it will be
convenient to introduce an alternate notation, $\procn{x}$, when we
want to emphasize the connection to the use of the name. Note that, by
virtue of name equivalence, $\quotep{\procn{x}} \nameeq x$; so, the
notation is consistent with previous definitions.

Further, because names have structure it is possible to effect
substitutions on the basis of that structure. This means we need to
upgrade our notation for substitutions, which we accomplish by
adapting comprehension notation. Thus,

\begin{mathpar}
  P\{ y / x : x \in S \}
\end{mathpar}

is interpreted to mean the process derived from P by replacing (in a
capture-avoiding manner) each occurrence of $x$ in $S$ by $y$. For example,

\begin{mathpar}
  P\{ \quotep{\procn{x}|\procn{x}} / x : x \in \freenames{P} \}
\end{mathpar}

will replace each (occurrence) of a free name $x$ in $P$ by
$\quotep{\procn{x}|\procn{x}}$.

Also, we will avail ourselves of the notation $x^{L}$ and $x^{R}$ to
denote injections of a name into disjoint copies of the name
space. There are numerous ways to accomplish this. One example can be
found in \cite{MeredithR05}. This notation overloads to vectors of
names: $\vec{x}^{\pi} := (x_{i}^{\pi} \; : \; 0 \leq i < |\vec{x}| )$ where $\pi \in \{L,R\}$.

We also use $P^{\Box} := P|\Box$.

In \cite{MeredithR05} an interpretation of the new operator is
given. It turns out that there are several possible interpretations
all enjoying the requisite algebraic properties of the operator (see
\cite{milner91polyadicpi}). We will therefore make liberal use of
$(\nu\; \vec{x})P$.

% subsection the_syntax_and_semantics_of_the_notation_system (end)   

\input{qm2pi.qmops} 

\input{qm2pi.sterngerlach} 

\input{qm2pi.metric} 

% section concurrent_process_calculi (end)

%\input{qm2pi.proofsketch}

% section proof sketch (end)

%\input{qm2pi.slviaknots} 

% section spatial logic via knots (end)

\input{qm2pi.conclusion}

% section conclusion (end)

%\input{qm2pi.dtcodes} 

% section wiring algorithm (end)

\input{qm2pi.ack} 

% section acknowledgments (end)

\newpage


\bibliographystyle{plain}   
\bibliography{../../biblios/main.bib}

\input{qm2pi.rhodetails}

\end{document}

 

%\ifpdf
%\usepackage[pdftex]{graphicx}
%\else
%\usepackage{graphicx}
%\fi

 % \ifpdf
%  \usepackage{pdfsync}
%  \if


%\title{Brief Article}
%\author{David F. Snyder}
%\author{L.G. Meredith}

%\address{Dept. of Math., Texas State University--San Marcos, San Marcos, TX 78666}
       
\pagestyle{empty}


\begin{document}

\lstset{language=[Objective]Caml,frame=shadowbox}

\documentclass[12pt]{llncs}
%\documentclass{jktr}

\usepackage[pdftex]{hyperref}                   
\usepackage {listings}
\usepackage {mathpartir}
\usepackage{bcprules}
%\usepackage{listings}
                       
\usepackage{graphicx} 
%\usepackage[margins=2.5cm,nohead,nofoot]{geometry}
%\usepackage{geometry}
\usepackage{amsfonts}
\usepackage{amstext}
\usepackage{latexsym}
\usepackage{amssymb}
\usepackage{color}


%\include{myPreamble}
\include{qm2pi.local} 

%\ifpdf
%\usepackage[pdftex]{graphicx}
%\else
%\usepackage{graphicx}
%\fi

 % \ifpdf
%  \usepackage{pdfsync}
%  \if


%\title{Brief Article}
%\author{David F. Snyder}
%\author{L.G. Meredith}

%\address{Dept. of Math., Texas State University--San Marcos, San Marcos, TX 78666}
       
\pagestyle{empty}


\begin{document}

\lstset{language=[Objective]Caml,frame=shadowbox}

\input{qm2pi.front}

% section front matter (end)

\input{qm2pi.intro} 
 
% section introduction (end)

% \input{qm2pi.knotations} 

% section notation (end)

\input{qm2pi.process.calculi} 

% section concurrent_process_calculi_and_spatial_logics_ (end)
    
%\input{qm2pi.knots2pi} 

%\input{qm2pi.trefoil} 

%\input{qm2pi.mainthm} 

% subsection basic_interpretation (end)

%\input{qm2pi.rho.presentation} 
\subsection{The syntax and semantics of the notation system}\label{sub:the_syntax_and_semantics_of_the_notation_system} % (fold)

We now summarize a technical presentation of the calculus that
embodies our theory of dynamics. The typical presentation of such a
calculus follows the style of giving generators and relations on
them. The grammar, below, describing term constructors, freely
generates the set of processes, $\Proc$. This set is then quotiented
by a relation known as structural congruence and it is over this set
that the notion of dynamics is expressed. This presentation is
essentially that of \cite{MeredithR05} with the addition of
polyadicity and summation. For readability we have relegated some of
the technical subtleties to an appendix.

\subsubsection{Process grammar}\label{subsub:process_grammar}

\begin{mathpar}
  \inferrule* [lab=synchronization] {} {{M} \bc \pzero \;|\; x?F \;|\; x!C }
  \and
  \inferrule* [lab=abstraction] {} {{F} \bc (x)P}
  \and
  \inferrule* [lab=concretion] {} {{C} \bc \langle Q \rangle}
  \and
  \inferrule* [lab=process] {} {{P,Q} \bc M \;| \;P|Q \;|\; @{x}}
  \and
  \inferrule* [lab=name] {} {{x} \bc \quotep{P}}
\end{mathpar} 

Note that $\vec{x}$ (resp. $\vec{P}$) denotes a vector of names
(resp. processes) of length $|\vec{x}|$ (resp. $|\vec{P}|$). We adopt
the following useful abbreviations.

\begin{mathpar}
   x?(\vec{y}).P := x.(\vec{y})P \and  x\clift{\vec{P}} := x.\clift{\vec{P}}
   \and x!(y) := \lift{x}{\dropn{y}}
   \and \Pi_{i=0}^{n-1}P_i := P_0 | \ldots | P_{n-1}
\end{mathpar}

\subsubsection{Structural congruence}

\paragraph{Free and bound names and alpha-equivalence.} At the
core of structural equivalence is alpha-equivalence which identifies
process that are the same up to a change of variable. Formally, we
recognize the distinction between free and bound names. The free names
of a process, $\freenames{P}$, may be calculated recursively as
follows:

\begin{mathpar}
\freenames{\pzero} := \emptyset
  \and \\
  \freenames{x?(y).P} := \{ x \} \cup (\freenames{P} \setminus \{ y \})
  \and 
  \freenames{x!\langle P \rangle} := \{ x \} \cup \{ P \} 
  \and \\
  \freenames{P|Q} := \freenames{P} \cup \freenames{Q}
  \and \\
  \freenames{@{x}} := \{ x \}
\end{mathpar}

$\pi$
$\quotep{\pi}$

$\freenames{-} : \pi \to \mathcal{P}(\quotep{\pi})$

\begin{eqnarray*}
  \freenames{\pzero} & := & \emptyset \\
  \freenames{x?(y).P} & := & \{ x \} \cup (\freenames{P} \setminus \{ y \}) \\
  \freenames{x!\langle P \rangle} & := & \{ x \} \cup \{ P \} \\
  \freenames{P|Q} & := & \freenames{P} \cup \freenames{Q} \\
  \freenames{\dropn{x}} & := & \{ x \}
\end{eqnarray*}

The bound names of a process, $\boundnames{P}$, are those names occurring in $P$
that are not free. For example, in $x?(y).0$, the name $x$ is free, while $y$ is bound.

\begin{mathpar}
  \inferrule* [lab=monoidal-laws] {} { P|Q \equiv Q|P \and P|0 \equiv P \and P|(Q|R) \equiv (P|Q)|R }
\end{mathpar}

\begin{mathpar}
  \inferrule* [lab=alpha-equivalence] {} { (x)P \equiv (y)P\{y/x\} \and y \not\in \freenames{P} }
\end{mathpar}

\begin{definition}
Then two processes, $P,Q$, are alpha-equivalent if $P = Q\{\vec{y}/\vec{x}\}$ for
some $\vec{x} \in \boundnames{Q},\vec{y} \in \boundnames{P}$, where $Q\{\vec{y}/\vec{x}\}$
denotes the capture-avoiding substitution of $\vec{y}$ for $\vec{x}$ in $Q$.
\end{definition}

\begin{definition}
  The {\em structural congruence} \cite{SangiorgiWalker} , $\equiv$,
  between processes is the least congruence containing
  alpha-equivalence, satisfying the abelian monoid laws
  (associativity, commutativity and $\pzero$ as identity) for parallel
  composition $|$ and for summation $+$.
\end{definition}

\subsection{Name equivalence}

We take name equivalence, written $\nameeq$, to be the smallest
equivalence relation generated by the following rules.

\begin{mathpar}
\inferrule*[lab=Quote-drop]
{ }
{ \quotep{@{x}} \nameeq x }

\inferrule*[lab=Struct-equiv]
{ P \scong Q }
{ \quotep{P} \nameeq \quotep{Q} }
\end{mathpar}

The astute reader will have noticed that the mutual recursion of names
and processes imposes a mutual recursion on alpha-equivalence and
structural equivalence via name-equivalence. Fortunately, all of this
works out pleasantly and we may calculate in the natural way, free of
concern. The reader interested in the details is referred to the
appendix \ref{appendix:rho_details}.

\subsection{Substitution}

We use $\Proc$ for the set of processes, $\QProc$ for the set of
names, and $\id{\{}\vec{y} / \vec{x} \id{\}}$ to denote partial maps,
$s : \QProc \rightarrow \QProc$. A map, $s$ lifts, uniquely, to a map
on process terms, $\widehat{s} : \Proc \rightarrow \Proc$ by the
following equations.

\begin{mathpar}
  (0) \psubstp{Q}{P} := 0 \\
  (R \juxtap S) \psubstp{Q}{P}
  :=    
  (R)\psubstp{Q}{P} \juxtap (S) \psubstp{Q}{P} \\
  (x?(y).R) \psubstp{Q}{P}    
  :=    
  (x)\substp{Q}{P} (z)\concat( (R \psubstn{z}{y}) \psubstp{Q}{P} ) \\
  (\lift{x}{R}) \psubstp{Q}{P}  
  :=
  \lift{(x)\substp{Q}{P}}{ R \psubstp{Q}{P} } \\
%   (\dropn{x})  \psubstp{Q}{P}       
%   := 
%   \left\{ 
%     \begin{array}{ccc} 
%       \dropn{\quotep{Q}} & & x \nameeq \quotep{P} \\
%       \dropn{x} & & otherwise \\
%     \end{array}
%   \right. 
  (\dropn{x})  \psubstp{Q}{P}       
  := 
  \left\{ 
    \begin{array}{ccc} 
      Q & & x \nameeq \quotep{P} \\
      \dropn{x} & & otherwise \\
    \end{array}
  \right.
\end{mathpar}
 

where

\begin{eqnarray}
  (x)\id{\{} \lpquote Q \rpquote / \lpquote P \rpquote \id{\}}            = 
  \left\{ 
    \begin{array}{ccc}
      \lpquote Q \rpquote & & x \nameeq \lpquote P \rpquote \\
      x & & otherwise \\
    \end{array}
  \right. \nonumber
\end{eqnarray}

and $z$ is chosen distinct from $\quotep{P}$, $\quotep{Q}$, the free
names in $Q$, and all the names in $R$. Our $\alpha$-equivalence will
be built in the standard way from this substitution.

\begin{remark}\label{rem:no_self_referential_names}
  One consequence of these definitions is that $\forall P. \quotep{P}
  \not\in \freenames{P}$.
\end{remark}

\subsection{ Dynamic quote: an example }

Anticipating something of what's to come, consider applying the
substitution, $\widehat{\id{\{}u / z \id{\}}}$, to the following pair
of processes, $\lift{w}{y!(z)}$ and $w[ \lpquote y!(z) \rpquote ]$.

\begin{eqnarray}
	\lift{w}{y!(z)}\widehat{\id{\{}u / z \id{\}}}
		& = &
		\lift{w}{y!(u)} \nonumber\\
	w[ \lpquote y!(z) \rpquote ] \widehat{ \id{\{}u / z \id{\}} }
		& = &
		w[ \lpquote y!(z) \rpquote ] \nonumber
\end{eqnarray}

Because the body of the process between quotes is impervious to
substitution, we get radically different answers. In fact, by
examining the first process in an input context,
e.g. $x?(z).\lift{w}{y!(z)}$, we see that the process under the lift
operator may be shaped by prefixed inputs binding a name inside it. In
this sense, the lift operator will be seen as a way to dynamically
construct processes before reifying them as names.

Finally equipped with these standard features we can present the
dynamics of the calculus.

\subsubsection{Operational semantics} 

Finally, we introduce the computational dynamics. What marks these
algebras as distinct from other more traditionally studied algebraic
structures, e.g. vector spaces or polynomial rings, is the manner in
which dynamics is captured. In traditional structures, dynamics is typically
expressed through morphisms between such structures, as in linear maps
between vector spaces or morphisms between rings. In algebras
associated with the semantics of computation, the dynamics is
expressed as part of the algebraic structure itself, through a
reduction reduction relation typically denoted by $\red$. Below, we
give a recursive presentation of this relation for the calculus used
in the encoding.

$\red \subseteq \pi \times \pi$
$\red : \pi \to \mathcal{P}(\pi)$

\begin{mathpar}
  \inferrule* [lab=Comm] { \textsf{match}( x_{src}, x_{trgt} ) } { x_{trgt}?(y)P \; | \; x_{src}!\langle {Q} \rangle \red P\{\quotep{Q}/y}\} }
  \and \\
  \inferrule* [lab=Par] {{P} \red {P}'} {{{P} | {Q}} \red {{P}' | {Q}}}
  \and
  \inferrule* [lab=Equiv]{{{P} \scong {P}'} \andalso {{P}' \red {Q}'} \andalso {{Q}' \scong {Q}}}{{P} \red {Q}}
\end{mathpar}

\begin{eqnarray*}
  match_{\equiv} (\quotep{P},\quotep{Q}) & := & P \equiv Q \\
  match_{\dagger}(\quotep{P},\quotep{Q}) & := & \forall R. P|Q \red^{*} R => R \red^{*} 0 \\
  match_{K}(\quotep{P},\quotep{Q}) & := & K \mbox{ for some context } K
\end{eqnarray*}

$u?(x)P | u!\langle Q \rangle \red P\{\quotep{Q}/x\}$

%We write $\wred$ for $\red^*$, and $P\red$ if $\exists Q $ such that $ P \red Q$.
We write $P\red$ if $\exists Q $ such that $ P \red Q$ and $P\not\red$, otherwise.

\section{Replication}

As mentioned before, it is known that replication (and hence
recursion) can be implemented in a higher-order process algebra
\cite{SangiorgiWalker}. As our first example of calculation with the
machinery thus far presented we give the construction explicitly in
the {\rhoc}.

\begin{eqnarray}
	D_{x} & := & \prefix{x}{y}{(\binpar{\outputp{x}{y}}{@{y}})} \nonumber\\
	\bangp_{x}{P} & := & \binpar{{x}!\langle{\binpar{D_{x}}{P}}\rangle}{D_{x}} \nonumber
\end{eqnarray}

\begin{eqnarray}
	\bangp_{x}{P} & & \nonumber\\
	=
	& {x}!\langle{(\prefix{x}{y}{(\outputp{x}{y} | @{y})) | P}}\rangle 
	      | \prefix{x}{y}{(\outputp{x}{y} | @{y})} & \nonumber\\
	\red
	& (\outputp{x}{y} | @{y})\substn{\quotep{(\prefix{x}{y}{(@{y} | \outputp{x}{y})) | P}}}{y} & \nonumber\\
	=
	& \outputp{x}{\quotep{(\prefix{x}{y}{(\outputp{x}{y} | @{y})) | P}}}
	  | {(\prefix{x}{y}{(\outputp{x}{y} | @{y})) | P}} & \nonumber\\
	\red
	& \ldots & \nonumber\\
	\red^*
	& P | P | \ldots & \nonumber
\end{eqnarray}

Of course, this encoding, as an implementation, runs away, unfolding
$\bangp{P}$ eagerly. A lazier and more implementable replication
operator, restricted to input-guarded processes, may be obtained as follows.

\begin{eqnarray}
\bangp{\prefix{u}{v}{P}} 
	:= 
	\binpar{\lift{x}{\prefix{u}{v}{(\binpar{D(x)}{P})}}}{D(x)} \nonumber
\end{eqnarray}

\begin{remark}
  Note that the lazier definition still does not deal with summation
  or mixed summation (i.e. sums over input and output). The reader is
  invited to construct definitions of replication that deal with these
  features. 

  Further, the definitions are parameterized in a name, $x$. Can you,
  gentle reader, make a definition that eliminates this parameter and
  guarantees no accidental interaction between the replication
  machinery and the process being replicated -- i.e. no accidental
  sharing of names used by the process to get its work done and the
  name(s) used by the replication to effect copying. This latter
  revision of the definition of replication is crucial to obtaining
  the expected identity $!!P \sim !P$.
\end{remark}

\begin{remark}\label{rem:paradoxical_combinator}
  The reader familiar with the lambda calculus will have noticed the
  similarity between $D$ and the paradoxical combinator.

  [Ed. note: the existence of this seems to suggest we have to be more
  restrictive on the set of processes and names we admit if we are to
  support no-cloning.]
\end{remark}

\subsubsection{Bisimulation}

The computational dynamics gives rise to another kind of equivalence,
the equivalence of computational behavior. As previously mentioned
this is typically captured \emph{via} some form of bisimulation.

% The notion we use in this paper is weak barbed bisimulation
% \cite{milner91polyadicpi}.

The notion we use in this paper is derived from weak barbed
bisimulation \cite{milner91polyadicpi}. 

\begin{definition}
An \emph{observation relation}, $\downarrow_{\mathcal N}$, over a set
of names, $\mathcal N$, is the smallest relation satisfying the rules
below.

\infrule[Out-barb]{y \in {\mathcal N}, \; x \nameeq y}
		  {\outputp{x}{v} \downarrow_{\mathcal N} x}
\infrule[Par-barb]{\mbox{$P\downarrow_{\mathcal N} x$ or $Q\downarrow_{\mathcal N} x$}}
		  {\binpar{P}{Q} \downarrow_{\mathcal N} x}

We write $P \Downarrow_{\mathcal N} x$ if there is $Q$ such that 
$P \wred Q$ and $Q \downarrow_{\mathcal N} x$.
\end{definition}

\begin{definition}
%\label{def.bbisim}
An  ${\mathcal N}$-\emph{barbed bisimulation} over a set of names, ${\mathcal N}$, is a symmetric binary relation 
${\mathcal S}_{\mathcal N}$ between agents such that $P\rel{S}_{\mathcal N}Q$ implies:
\begin{enumerate}
\item If $P \red P'$ then $Q \wred Q'$ and $P'\rel{S}_{\mathcal N} Q'$.
\item If $P\downarrow_{\mathcal N} x$, then $Q\Downarrow_{\mathcal N} x$.
\end{enumerate}
$P$ is ${\mathcal N}$-barbed bisimilar to $Q$, written
$P \wbbisim_{\mathcal N} Q$, if $P \rel{S}_{\mathcal N} Q$ for some ${\mathcal N}$-barbed bisimulation ${\mathcal S}_{\mathcal N}$.
\end{definition}

$\mathcal{R} \subseteq \pi \times \pi$

$P \mathcal{R} Q => \forall P'. P \red P' \Rightarrow \exists Q'. Q \red Q', P' \mathcal{R} Q'$

$P \vdash x \Rightarrow Q \vdash x$

\begin{mathpar}
  \inferrule*[lab=Out-barb]{x \nameeq y}{{y}!\langle{Q}\rangle \vdash x}
  \and
  \inferrule*[lab=Par-barb]{\mbox{$P\vdash x$ or $Q\vdash x$}}{\binpar{P}{Q} \vdash x}
\end{mathpar}

\subsubsection{Contexts}

One of the principle advantages of computational calculi like the
$\pi$-calculus is a well-defined notion of context,
contextual-equivalence and a correlation between
contextual-equivalence and notions of bisimulation. The notion of
context allows the decomposition of a process into (sub-)process and
its syntactic environment, its context. Thus, a context may be
thought of as a process with a ``hole'' (written $\Box$) in it. The
application of a context $M$ to a process $P$, written $M[P]$, is
tantamount to filling the hole in $M$ with $P$. In this paper we do
not need the full weight of this theory, but do make use of the notion
of context in the proof the main theorem. 

\begin{mathpar}
  \inferrule* [lab=summation] {} {{M_{M},M_{N}} \bc \Box \;|\; x.M_{A} \;|\; M_{M}+M_{N}}
  \and
  \inferrule* [lab=agent] {} {{M_{A}} \bc (\vec{x})M_{P} \;| \; \clift{P_0,\ldots,M_{P},\ldots,P_N}}
  \and \\
  \inferrule* [lab=process] {} {{M_{P}} \bc M_{N} \;| \;P|M_{P} }
\end{mathpar} 

\begin{mathpar}
  \inferrule* [lab=sychronization] {} {M_{N} \bc \Box \;|\; x?M_{F} \;|\; x!M_{C}}
  \and
  \inferrule* [lab=abstraction] {} {{M_{F}} \bc (x)M_{P} }
  \and
  \inferrule* [lab=concretion] {} {{M_{C}} \bc \langle M_{P} \rangle }
  \and \\
  \inferrule* [lab=process] {} {{M_{P}} \bc M_{N} \;| \;P|M_{P} }
\end{mathpar}

\begin{definition}[contextual application] Given a context $M$, and
  process $P$, we define the \emph{contextual application}, $M[P] :=
  M\{P/\Box\}$. That is, the contextual application of M to P is the
  substitution of $P$ for $\Box$ in $M$.
\end{definition}

$\meaningof{-} : L \to \mathcal{P}(\pi)$

\begin{mathpar}
  \inferrule* [lab=collection] {} {\meaningof{true} = \pi, \and \meaningof{~E} = \pi \setminus \meaningof{E}, \and \meaningof{E_{1} \& E_{2}} = \meaningof{E_{1}} \cap \meaningof{E_{2}}}
\end{mathpar}

\begin{mathpar}
  \inferrule* [lab=structure] {} {\meaningof{0} = \{ P \in \pi | P \equiv 0 \}, \and \\ \meaningof{E_1 | E_2} = \{ P \in \pi | P \equiv P_{1} | P_{2}, P_{1} \in \meaningof{E_{1}}, P_{2} \in \meaningof{E_2}\} }
\end{mathpar}

\begin{mathpar}
 \inferrule* [lab=behavior] {} {\meaningof{\langle a?b \rangle E} = \{ P \in \pi | P \equiv Q | u?(y)P', \\ \and \\\\ \and \\ \;\;\; u \in \meaningof{a}, \forall z.P'\{z/y\} \in \meaningof{E\{z/b\}}\}, \and \\ \meaningof{a!E} = \{ P \in \pi | P \equiv Q | x!\langle P' \rangle, x \in \meaningof{a} P' \in \meaningof{E}\} }
\end{mathpar}

\begin{mathpar}
 \inferrule* [lab=nominal] {} {\meaningof{\quotep{E}} = \{ \quotep{P} \in \quotep{\pi} | P \in \meaningof{E} \}, \and \meaningof{\quotep{P}} = \{ \quotep{Q} \in \quotep{\pi} | P \equiv Q \} \and \\ \meaningof{@\quotep{E}} = \{ P \in \pi | P \equiv @x, x \in \meaningof{E} \}}
\end{mathpar}

\begin{eqnarray*}
  \\
  \meaningof{-} : TS \to ST
\end{eqnarray*}

\begin{eqnarray*}
  \\
  L : TS \to ST
\end{eqnarray*}

\begin{eqnarray*}
  \\
  P \models E \iff P \in \meaningof{E}
\end{eqnarray*}

\begin{eqnarray*}
  P \approx_{L} Q \iff \forall E \in L. P \models E \iff Q \models E
\end{eqnarray*}

\begin{eqnarray*}
  P \approx_{K} Q
\end{eqnarray*}

\begin{eqnarray*}
  P \approx Q
\end{eqnarray*}

$\approx_{K} = \approx = \approx_{L}$

\subsubsection{Contextual duality}

Note that contexts extend the quotation operation to a family of
operations from processes to names. Given a context, $M$, we can
define a \emph{nominal context}, $\quotep{M}$ by $\quotep{M}[P] :=
\quotep{M[P]}$. To foreshadow what is to come we observe that these
operations enjoy a duality with processes very much like the duality
between vectors and maps from vectors to scalars.

Further, because the calculus is essentially higher-order, we have a
correspondence between contexts and processes. More specifically,
given a name $x$ and a context $M$ we can construct $M^{*}_{x}$ such
that 

\begin{mathpar}
  M^{*}_{x} | \lift{x}{P} \red M[P]
\end{mathpar}

namely,

\begin{mathpar}
  M^{*}_{x} := x?(u).M[\dropn{u}]
\end{mathpar}

The dependence of $M^{*}_{x}$ on a name makes it an abstraction, 

\begin{mathpar}
  M^{*} := (x)x?(u).M[\dropn{u}]
\end{mathpar}

\subsection{Additional notation}

It will sometimes be convenient to denote the process a name
quotes. We already have the notation $x = \quotep{P}$, but it will be
convenient to introduce an alternate notation, $\procn{x}$, when we
want to emphasize the connection to the use of the name. Note that, by
virtue of name equivalence, $\quotep{\procn{x}} \nameeq x$; so, the
notation is consistent with previous definitions.

Further, because names have structure it is possible to effect
substitutions on the basis of that structure. This means we need to
upgrade our notation for substitutions, which we accomplish by
adapting comprehension notation. Thus,

\begin{mathpar}
  P\{ y / x : x \in S \}
\end{mathpar}

is interpreted to mean the process derived from P by replacing (in a
capture-avoiding manner) each occurrence of $x$ in $S$ by $y$. For example,

\begin{mathpar}
  P\{ \quotep{\procn{x}|\procn{x}} / x : x \in \freenames{P} \}
\end{mathpar}

will replace each (occurrence) of a free name $x$ in $P$ by
$\quotep{\procn{x}|\procn{x}}$.

Also, we will avail ourselves of the notation $x^{L}$ and $x^{R}$ to
denote injections of a name into disjoint copies of the name
space. There are numerous ways to accomplish this. One example can be
found in \cite{MeredithR05}. This notation overloads to vectors of
names: $\vec{x}^{\pi} := (x_{i}^{\pi} \; : \; 0 \leq i < |\vec{x}| )$ where $\pi \in \{L,R\}$.

We also use $P^{\Box} := P|\Box$.

In \cite{MeredithR05} an interpretation of the new operator is
given. It turns out that there are several possible interpretations
all enjoying the requisite algebraic properties of the operator (see
\cite{milner91polyadicpi}). We will therefore make liberal use of
$(\nu\; \vec{x})P$.

% subsection the_syntax_and_semantics_of_the_notation_system (end)   

\input{qm2pi.qmops} 

\input{qm2pi.sterngerlach} 

\input{qm2pi.metric} 

% section concurrent_process_calculi (end)

%\input{qm2pi.proofsketch}

% section proof sketch (end)

%\input{qm2pi.slviaknots} 

% section spatial logic via knots (end)

\input{qm2pi.conclusion}

% section conclusion (end)

%\input{qm2pi.dtcodes} 

% section wiring algorithm (end)

\input{qm2pi.ack} 

% section acknowledgments (end)

\newpage


\bibliographystyle{plain}   
\bibliography{../../biblios/main.bib}

\input{qm2pi.rhodetails}

\end{document}



% section front matter (end)

\section{Introduction}\label{sec:introduction} % (fold)
In this draft of the material i am going to have to dispense with the
usual writing conventions adopted in papers on these topics. i'm going
to have adopt whatever tone i need at the time i'm writing up the
calculations. Sometimes this may be very conversational; others it may
be the barest mathematical grunts; others still it may be that i have
lifted text from one of my other papers because the exposition of some
point was better said there. i hope that my readers are not unduly put
out by this decision. i'm not doing this to flout convention or be
rebellious. i find these calculations very technically challenging. To
keep everything going technically, something has to give; i have to
let go of some cognitive burden. So, the academic writing style --
with all of its trade-offs in terms of facilitating technical
communication -- is what i'm letting go of. Perhaps subsequent drafts
can be tightened and polished, but for now, i'm going to speak as if
we were sitting together in a coffee shop with a laptop, wifi and a
pad of paper and a pencil.

So, here's what i have to say. We -- you and i, comfortably ensconced
in our coffee shop and well-equipped with our tools -- can realize and
carry out the calculations of quantum mechanics over a very different
formal theory of dynamics, a formal theory of dynamics that
corresponds to a theory of concurrent computation with
\emph{reflection}. It has the advantage that the underlying theory is
already `quantized', but supports analogues all of the continuuous
operations. Strikingly, this underlying theory has recently been
connected with a notion of metric that we can show, by calculating
together, coincides with the metric induced by the inner product.

There are a lot of reasons why you might be interested in seeing
calculations of this form. Here's why i'm interested. For the past
several centuries there has been no competitor to the ``Newtonian''
account of dynamics. As a result the predominant share of accounts of
dynamical systems and situations have had to be formulated in terms of
the Newtonian machinery. i view this as an intellectually dangerous
position to occupy. Everything, despite it's intrinsic shape, turns
into a nail to be hit with this hammer. Recently, however, the theory
of computation has matured to the point where we have candidates for
theories of dynamics that offer very different perspective on
reasoning about dynamical systems and situations. Testing these
candidates against very successful accounts of dynamical situations,
like quantum mechanics, is going to give us some sense of how mature
they are and some measure of the quality of these accounts of
dynamics.

\subsection{Summary of contributions and outline of paper}

So, we're going to develop an interpretation of the operations of
quantum mechanics normally interpreted by Hilbert spaces and
operators. We're going to do this over a theory of computation. Note
that this is very different than the usual quantum computation program
which develops notions of computation over quantum mechanics. Rather,
we are developing a story that aligns with Wheeler's slogan: It from
Bit. To do this we will first provide an account of the theory of
computation at play here. Then we will dive into a calculation-driven
interpretation of the operations of quantum mechanics.

The reason we take this approach is that -- until very recently --
there hasn't been an axiomatic account of quantum mechanics. As a
result there has been no sharp delineation of the mathematical theory
supporting interpretation of the physical theory and the physical
theory, itself. So, ambient features of the maths are free to be
exploited (or supressed) without a real accounting of their physical
relevance. There is no sharp statement ``here's the physical theory''
qua \emph{theory} and ``here's the mathematical interpretation''
enabling a judgment of how faithful the interpretation is -- apart
from experimental observation. When there is an axiomatic account we
can judge how well a given mathematical formalism supports an
interpretation of the axioms, independent of
experimentation. Likewise, we can judge how well we have captured our
physical evidence and experience with our axiomatics, independent of
any specific mathematical implementation, with accidental detail that
may or may not have physical significance. 

In lieu of a fully fleshed out and vetted axiomatic account of quantum
mechanics, interpreting the operational notions in service of modeling
physical systems will have to suffice. In other words, we are not in
the business of providing a model of Hilbert spaces and operators. We
are in the business of providing a model of quantum mechanics because
we are motivated by testing our notions of dynamics against physical
theory; and, the predictive calculations of the physical theory must
serve as the best formulation -- shy of a fully fleshed out axiomatic
account -- of the physical theory itself (as they have for scientific
theories since time immemorial). Put another way, despite a
whole-hearted commitment to an It-from-Bit ontology, we are firmly
aligned with the shut-up-and-calculate camp as the best way to obtain
results either from the physical perspective or as a quality assurance
measure of our fledgling theory of dynamics.

In detail, we present a reflective process calculus. Then we develop
intuitive correspondences between the notions available in this
calculus and the usual physical notions supporting quantum mechanical
calculations. Thus, 

\begin{table}[htp]
  \center{
    \fbox{
      \begin{tabular}{c|c}
        quantum mechanics & process calculus \\
        \hline
        scalar & name \\
        state vector & process \\
        dual & contextual duals \\
        matrix & formal sums of process-context-dual pairs \\
        orthogonality & process annihilation \\
        inner product & execution-formula + quoting
      \end{tabular}
    }
  }
  \caption{QM - process calculi correspondences}
\end{table}

Then we tighten up these intuitions to operational definitions. We
employ the Dirac notation as the best proxy we can find for an
abstract syntax of the quantum mechanical notions. The definitions we
develop put us in contact with equational constraints coming from the
theory that we demonstrate the definitions and calculations satisfy.

This puts us in a position to shut up and calculate for the
Stern-Gerlach experimental set up, showing how these predictive
calculations become calculations on processes in our theory of a
reflective process calculus.

Penultimately, we demonstrate that the notion of metric coming from
the inner product coincides with the notion of metric available from
the theory of bisimulation. This demonstration gives us the right to
think of space as arising from behavior. Finally, we consider where we
might go from the new vantage point we have obtained.

% section introduction (end) 
 
% section introduction (end)

% \documentclass[12pt]{llncs}
%\documentclass{jktr}

\usepackage[pdftex]{hyperref}                   
\usepackage {listings}
\usepackage {mathpartir}
\usepackage{bcprules}
%\usepackage{listings}
                       
\usepackage{graphicx} 
%\usepackage[margins=2.5cm,nohead,nofoot]{geometry}
%\usepackage{geometry}
\usepackage{amsfonts}
\usepackage{amstext}
\usepackage{latexsym}
\usepackage{amssymb}
\usepackage{color}


%\include{myPreamble}
\include{qm2pi.local} 

%\ifpdf
%\usepackage[pdftex]{graphicx}
%\else
%\usepackage{graphicx}
%\fi

 % \ifpdf
%  \usepackage{pdfsync}
%  \if


%\title{Brief Article}
%\author{David F. Snyder}
%\author{L.G. Meredith}

%\address{Dept. of Math., Texas State University--San Marcos, San Marcos, TX 78666}
       
\pagestyle{empty}


\begin{document}

\lstset{language=[Objective]Caml,frame=shadowbox}

\input{qm2pi.front}

% section front matter (end)

\input{qm2pi.intro} 
 
% section introduction (end)

% \input{qm2pi.knotations} 

% section notation (end)

\input{qm2pi.process.calculi} 

% section concurrent_process_calculi_and_spatial_logics_ (end)
    
%\input{qm2pi.knots2pi} 

%\input{qm2pi.trefoil} 

%\input{qm2pi.mainthm} 

% subsection basic_interpretation (end)

%\input{qm2pi.rho.presentation} 
\subsection{The syntax and semantics of the notation system}\label{sub:the_syntax_and_semantics_of_the_notation_system} % (fold)

We now summarize a technical presentation of the calculus that
embodies our theory of dynamics. The typical presentation of such a
calculus follows the style of giving generators and relations on
them. The grammar, below, describing term constructors, freely
generates the set of processes, $\Proc$. This set is then quotiented
by a relation known as structural congruence and it is over this set
that the notion of dynamics is expressed. This presentation is
essentially that of \cite{MeredithR05} with the addition of
polyadicity and summation. For readability we have relegated some of
the technical subtleties to an appendix.

\subsubsection{Process grammar}\label{subsub:process_grammar}

\begin{mathpar}
  \inferrule* [lab=synchronization] {} {{M} \bc \pzero \;|\; x?F \;|\; x!C }
  \and
  \inferrule* [lab=abstraction] {} {{F} \bc (x)P}
  \and
  \inferrule* [lab=concretion] {} {{C} \bc \langle Q \rangle}
  \and
  \inferrule* [lab=process] {} {{P,Q} \bc M \;| \;P|Q \;|\; @{x}}
  \and
  \inferrule* [lab=name] {} {{x} \bc \quotep{P}}
\end{mathpar} 

Note that $\vec{x}$ (resp. $\vec{P}$) denotes a vector of names
(resp. processes) of length $|\vec{x}|$ (resp. $|\vec{P}|$). We adopt
the following useful abbreviations.

\begin{mathpar}
   x?(\vec{y}).P := x.(\vec{y})P \and  x\clift{\vec{P}} := x.\clift{\vec{P}}
   \and x!(y) := \lift{x}{\dropn{y}}
   \and \Pi_{i=0}^{n-1}P_i := P_0 | \ldots | P_{n-1}
\end{mathpar}

\subsubsection{Structural congruence}

\paragraph{Free and bound names and alpha-equivalence.} At the
core of structural equivalence is alpha-equivalence which identifies
process that are the same up to a change of variable. Formally, we
recognize the distinction between free and bound names. The free names
of a process, $\freenames{P}$, may be calculated recursively as
follows:

\begin{mathpar}
\freenames{\pzero} := \emptyset
  \and \\
  \freenames{x?(y).P} := \{ x \} \cup (\freenames{P} \setminus \{ y \})
  \and 
  \freenames{x!\langle P \rangle} := \{ x \} \cup \{ P \} 
  \and \\
  \freenames{P|Q} := \freenames{P} \cup \freenames{Q}
  \and \\
  \freenames{@{x}} := \{ x \}
\end{mathpar}

$\pi$
$\quotep{\pi}$

$\freenames{-} : \pi \to \mathcal{P}(\quotep{\pi})$

\begin{eqnarray*}
  \freenames{\pzero} & := & \emptyset \\
  \freenames{x?(y).P} & := & \{ x \} \cup (\freenames{P} \setminus \{ y \}) \\
  \freenames{x!\langle P \rangle} & := & \{ x \} \cup \{ P \} \\
  \freenames{P|Q} & := & \freenames{P} \cup \freenames{Q} \\
  \freenames{\dropn{x}} & := & \{ x \}
\end{eqnarray*}

The bound names of a process, $\boundnames{P}$, are those names occurring in $P$
that are not free. For example, in $x?(y).0$, the name $x$ is free, while $y$ is bound.

\begin{mathpar}
  \inferrule* [lab=monoidal-laws] {} { P|Q \equiv Q|P \and P|0 \equiv P \and P|(Q|R) \equiv (P|Q)|R }
\end{mathpar}

\begin{mathpar}
  \inferrule* [lab=alpha-equivalence] {} { (x)P \equiv (y)P\{y/x\} \and y \not\in \freenames{P} }
\end{mathpar}

\begin{definition}
Then two processes, $P,Q$, are alpha-equivalent if $P = Q\{\vec{y}/\vec{x}\}$ for
some $\vec{x} \in \boundnames{Q},\vec{y} \in \boundnames{P}$, where $Q\{\vec{y}/\vec{x}\}$
denotes the capture-avoiding substitution of $\vec{y}$ for $\vec{x}$ in $Q$.
\end{definition}

\begin{definition}
  The {\em structural congruence} \cite{SangiorgiWalker} , $\equiv$,
  between processes is the least congruence containing
  alpha-equivalence, satisfying the abelian monoid laws
  (associativity, commutativity and $\pzero$ as identity) for parallel
  composition $|$ and for summation $+$.
\end{definition}

\subsection{Name equivalence}

We take name equivalence, written $\nameeq$, to be the smallest
equivalence relation generated by the following rules.

\begin{mathpar}
\inferrule*[lab=Quote-drop]
{ }
{ \quotep{@{x}} \nameeq x }

\inferrule*[lab=Struct-equiv]
{ P \scong Q }
{ \quotep{P} \nameeq \quotep{Q} }
\end{mathpar}

The astute reader will have noticed that the mutual recursion of names
and processes imposes a mutual recursion on alpha-equivalence and
structural equivalence via name-equivalence. Fortunately, all of this
works out pleasantly and we may calculate in the natural way, free of
concern. The reader interested in the details is referred to the
appendix \ref{appendix:rho_details}.

\subsection{Substitution}

We use $\Proc$ for the set of processes, $\QProc$ for the set of
names, and $\id{\{}\vec{y} / \vec{x} \id{\}}$ to denote partial maps,
$s : \QProc \rightarrow \QProc$. A map, $s$ lifts, uniquely, to a map
on process terms, $\widehat{s} : \Proc \rightarrow \Proc$ by the
following equations.

\begin{mathpar}
  (0) \psubstp{Q}{P} := 0 \\
  (R \juxtap S) \psubstp{Q}{P}
  :=    
  (R)\psubstp{Q}{P} \juxtap (S) \psubstp{Q}{P} \\
  (x?(y).R) \psubstp{Q}{P}    
  :=    
  (x)\substp{Q}{P} (z)\concat( (R \psubstn{z}{y}) \psubstp{Q}{P} ) \\
  (\lift{x}{R}) \psubstp{Q}{P}  
  :=
  \lift{(x)\substp{Q}{P}}{ R \psubstp{Q}{P} } \\
%   (\dropn{x})  \psubstp{Q}{P}       
%   := 
%   \left\{ 
%     \begin{array}{ccc} 
%       \dropn{\quotep{Q}} & & x \nameeq \quotep{P} \\
%       \dropn{x} & & otherwise \\
%     \end{array}
%   \right. 
  (\dropn{x})  \psubstp{Q}{P}       
  := 
  \left\{ 
    \begin{array}{ccc} 
      Q & & x \nameeq \quotep{P} \\
      \dropn{x} & & otherwise \\
    \end{array}
  \right.
\end{mathpar}
 

where

\begin{eqnarray}
  (x)\id{\{} \lpquote Q \rpquote / \lpquote P \rpquote \id{\}}            = 
  \left\{ 
    \begin{array}{ccc}
      \lpquote Q \rpquote & & x \nameeq \lpquote P \rpquote \\
      x & & otherwise \\
    \end{array}
  \right. \nonumber
\end{eqnarray}

and $z$ is chosen distinct from $\quotep{P}$, $\quotep{Q}$, the free
names in $Q$, and all the names in $R$. Our $\alpha$-equivalence will
be built in the standard way from this substitution.

\begin{remark}\label{rem:no_self_referential_names}
  One consequence of these definitions is that $\forall P. \quotep{P}
  \not\in \freenames{P}$.
\end{remark}

\subsection{ Dynamic quote: an example }

Anticipating something of what's to come, consider applying the
substitution, $\widehat{\id{\{}u / z \id{\}}}$, to the following pair
of processes, $\lift{w}{y!(z)}$ and $w[ \lpquote y!(z) \rpquote ]$.

\begin{eqnarray}
	\lift{w}{y!(z)}\widehat{\id{\{}u / z \id{\}}}
		& = &
		\lift{w}{y!(u)} \nonumber\\
	w[ \lpquote y!(z) \rpquote ] \widehat{ \id{\{}u / z \id{\}} }
		& = &
		w[ \lpquote y!(z) \rpquote ] \nonumber
\end{eqnarray}

Because the body of the process between quotes is impervious to
substitution, we get radically different answers. In fact, by
examining the first process in an input context,
e.g. $x?(z).\lift{w}{y!(z)}$, we see that the process under the lift
operator may be shaped by prefixed inputs binding a name inside it. In
this sense, the lift operator will be seen as a way to dynamically
construct processes before reifying them as names.

Finally equipped with these standard features we can present the
dynamics of the calculus.

\subsubsection{Operational semantics} 

Finally, we introduce the computational dynamics. What marks these
algebras as distinct from other more traditionally studied algebraic
structures, e.g. vector spaces or polynomial rings, is the manner in
which dynamics is captured. In traditional structures, dynamics is typically
expressed through morphisms between such structures, as in linear maps
between vector spaces or morphisms between rings. In algebras
associated with the semantics of computation, the dynamics is
expressed as part of the algebraic structure itself, through a
reduction reduction relation typically denoted by $\red$. Below, we
give a recursive presentation of this relation for the calculus used
in the encoding.

$\red \subseteq \pi \times \pi$
$\red : \pi \to \mathcal{P}(\pi)$

\begin{mathpar}
  \inferrule* [lab=Comm] { \textsf{match}( x_{src}, x_{trgt} ) } { x_{trgt}?(y)P \; | \; x_{src}!\langle {Q} \rangle \red P\{\quotep{Q}/y}\} }
  \and \\
  \inferrule* [lab=Par] {{P} \red {P}'} {{{P} | {Q}} \red {{P}' | {Q}}}
  \and
  \inferrule* [lab=Equiv]{{{P} \scong {P}'} \andalso {{P}' \red {Q}'} \andalso {{Q}' \scong {Q}}}{{P} \red {Q}}
\end{mathpar}

\begin{eqnarray*}
  match_{\equiv} (\quotep{P},\quotep{Q}) & := & P \equiv Q \\
  match_{\dagger}(\quotep{P},\quotep{Q}) & := & \forall R. P|Q \red^{*} R => R \red^{*} 0 \\
  match_{K}(\quotep{P},\quotep{Q}) & := & K \mbox{ for some context } K
\end{eqnarray*}

$u?(x)P | u!\langle Q \rangle \red P\{\quotep{Q}/x\}$

%We write $\wred$ for $\red^*$, and $P\red$ if $\exists Q $ such that $ P \red Q$.
We write $P\red$ if $\exists Q $ such that $ P \red Q$ and $P\not\red$, otherwise.

\section{Replication}

As mentioned before, it is known that replication (and hence
recursion) can be implemented in a higher-order process algebra
\cite{SangiorgiWalker}. As our first example of calculation with the
machinery thus far presented we give the construction explicitly in
the {\rhoc}.

\begin{eqnarray}
	D_{x} & := & \prefix{x}{y}{(\binpar{\outputp{x}{y}}{@{y}})} \nonumber\\
	\bangp_{x}{P} & := & \binpar{{x}!\langle{\binpar{D_{x}}{P}}\rangle}{D_{x}} \nonumber
\end{eqnarray}

\begin{eqnarray}
	\bangp_{x}{P} & & \nonumber\\
	=
	& {x}!\langle{(\prefix{x}{y}{(\outputp{x}{y} | @{y})) | P}}\rangle 
	      | \prefix{x}{y}{(\outputp{x}{y} | @{y})} & \nonumber\\
	\red
	& (\outputp{x}{y} | @{y})\substn{\quotep{(\prefix{x}{y}{(@{y} | \outputp{x}{y})) | P}}}{y} & \nonumber\\
	=
	& \outputp{x}{\quotep{(\prefix{x}{y}{(\outputp{x}{y} | @{y})) | P}}}
	  | {(\prefix{x}{y}{(\outputp{x}{y} | @{y})) | P}} & \nonumber\\
	\red
	& \ldots & \nonumber\\
	\red^*
	& P | P | \ldots & \nonumber
\end{eqnarray}

Of course, this encoding, as an implementation, runs away, unfolding
$\bangp{P}$ eagerly. A lazier and more implementable replication
operator, restricted to input-guarded processes, may be obtained as follows.

\begin{eqnarray}
\bangp{\prefix{u}{v}{P}} 
	:= 
	\binpar{\lift{x}{\prefix{u}{v}{(\binpar{D(x)}{P})}}}{D(x)} \nonumber
\end{eqnarray}

\begin{remark}
  Note that the lazier definition still does not deal with summation
  or mixed summation (i.e. sums over input and output). The reader is
  invited to construct definitions of replication that deal with these
  features. 

  Further, the definitions are parameterized in a name, $x$. Can you,
  gentle reader, make a definition that eliminates this parameter and
  guarantees no accidental interaction between the replication
  machinery and the process being replicated -- i.e. no accidental
  sharing of names used by the process to get its work done and the
  name(s) used by the replication to effect copying. This latter
  revision of the definition of replication is crucial to obtaining
  the expected identity $!!P \sim !P$.
\end{remark}

\begin{remark}\label{rem:paradoxical_combinator}
  The reader familiar with the lambda calculus will have noticed the
  similarity between $D$ and the paradoxical combinator.

  [Ed. note: the existence of this seems to suggest we have to be more
  restrictive on the set of processes and names we admit if we are to
  support no-cloning.]
\end{remark}

\subsubsection{Bisimulation}

The computational dynamics gives rise to another kind of equivalence,
the equivalence of computational behavior. As previously mentioned
this is typically captured \emph{via} some form of bisimulation.

% The notion we use in this paper is weak barbed bisimulation
% \cite{milner91polyadicpi}.

The notion we use in this paper is derived from weak barbed
bisimulation \cite{milner91polyadicpi}. 

\begin{definition}
An \emph{observation relation}, $\downarrow_{\mathcal N}$, over a set
of names, $\mathcal N$, is the smallest relation satisfying the rules
below.

\infrule[Out-barb]{y \in {\mathcal N}, \; x \nameeq y}
		  {\outputp{x}{v} \downarrow_{\mathcal N} x}
\infrule[Par-barb]{\mbox{$P\downarrow_{\mathcal N} x$ or $Q\downarrow_{\mathcal N} x$}}
		  {\binpar{P}{Q} \downarrow_{\mathcal N} x}

We write $P \Downarrow_{\mathcal N} x$ if there is $Q$ such that 
$P \wred Q$ and $Q \downarrow_{\mathcal N} x$.
\end{definition}

\begin{definition}
%\label{def.bbisim}
An  ${\mathcal N}$-\emph{barbed bisimulation} over a set of names, ${\mathcal N}$, is a symmetric binary relation 
${\mathcal S}_{\mathcal N}$ between agents such that $P\rel{S}_{\mathcal N}Q$ implies:
\begin{enumerate}
\item If $P \red P'$ then $Q \wred Q'$ and $P'\rel{S}_{\mathcal N} Q'$.
\item If $P\downarrow_{\mathcal N} x$, then $Q\Downarrow_{\mathcal N} x$.
\end{enumerate}
$P$ is ${\mathcal N}$-barbed bisimilar to $Q$, written
$P \wbbisim_{\mathcal N} Q$, if $P \rel{S}_{\mathcal N} Q$ for some ${\mathcal N}$-barbed bisimulation ${\mathcal S}_{\mathcal N}$.
\end{definition}

$\mathcal{R} \subseteq \pi \times \pi$

$P \mathcal{R} Q => \forall P'. P \red P' \Rightarrow \exists Q'. Q \red Q', P' \mathcal{R} Q'$

$P \vdash x \Rightarrow Q \vdash x$

\begin{mathpar}
  \inferrule*[lab=Out-barb]{x \nameeq y}{{y}!\langle{Q}\rangle \vdash x}
  \and
  \inferrule*[lab=Par-barb]{\mbox{$P\vdash x$ or $Q\vdash x$}}{\binpar{P}{Q} \vdash x}
\end{mathpar}

\subsubsection{Contexts}

One of the principle advantages of computational calculi like the
$\pi$-calculus is a well-defined notion of context,
contextual-equivalence and a correlation between
contextual-equivalence and notions of bisimulation. The notion of
context allows the decomposition of a process into (sub-)process and
its syntactic environment, its context. Thus, a context may be
thought of as a process with a ``hole'' (written $\Box$) in it. The
application of a context $M$ to a process $P$, written $M[P]$, is
tantamount to filling the hole in $M$ with $P$. In this paper we do
not need the full weight of this theory, but do make use of the notion
of context in the proof the main theorem. 

\begin{mathpar}
  \inferrule* [lab=summation] {} {{M_{M},M_{N}} \bc \Box \;|\; x.M_{A} \;|\; M_{M}+M_{N}}
  \and
  \inferrule* [lab=agent] {} {{M_{A}} \bc (\vec{x})M_{P} \;| \; \clift{P_0,\ldots,M_{P},\ldots,P_N}}
  \and \\
  \inferrule* [lab=process] {} {{M_{P}} \bc M_{N} \;| \;P|M_{P} }
\end{mathpar} 

\begin{mathpar}
  \inferrule* [lab=sychronization] {} {M_{N} \bc \Box \;|\; x?M_{F} \;|\; x!M_{C}}
  \and
  \inferrule* [lab=abstraction] {} {{M_{F}} \bc (x)M_{P} }
  \and
  \inferrule* [lab=concretion] {} {{M_{C}} \bc \langle M_{P} \rangle }
  \and \\
  \inferrule* [lab=process] {} {{M_{P}} \bc M_{N} \;| \;P|M_{P} }
\end{mathpar}

\begin{definition}[contextual application] Given a context $M$, and
  process $P$, we define the \emph{contextual application}, $M[P] :=
  M\{P/\Box\}$. That is, the contextual application of M to P is the
  substitution of $P$ for $\Box$ in $M$.
\end{definition}

$\meaningof{-} : L \to \mathcal{P}(\pi)$

\begin{mathpar}
  \inferrule* [lab=collection] {} {\meaningof{true} = \pi, \and \meaningof{~E} = \pi \setminus \meaningof{E}, \and \meaningof{E_{1} \& E_{2}} = \meaningof{E_{1}} \cap \meaningof{E_{2}}}
\end{mathpar}

\begin{mathpar}
  \inferrule* [lab=structure] {} {\meaningof{0} = \{ P \in \pi | P \equiv 0 \}, \and \\ \meaningof{E_1 | E_2} = \{ P \in \pi | P \equiv P_{1} | P_{2}, P_{1} \in \meaningof{E_{1}}, P_{2} \in \meaningof{E_2}\} }
\end{mathpar}

\begin{mathpar}
 \inferrule* [lab=behavior] {} {\meaningof{\langle a?b \rangle E} = \{ P \in \pi | P \equiv Q | u?(y)P', \\ \and \\\\ \and \\ \;\;\; u \in \meaningof{a}, \forall z.P'\{z/y\} \in \meaningof{E\{z/b\}}\}, \and \\ \meaningof{a!E} = \{ P \in \pi | P \equiv Q | x!\langle P' \rangle, x \in \meaningof{a} P' \in \meaningof{E}\} }
\end{mathpar}

\begin{mathpar}
 \inferrule* [lab=nominal] {} {\meaningof{\quotep{E}} = \{ \quotep{P} \in \quotep{\pi} | P \in \meaningof{E} \}, \and \meaningof{\quotep{P}} = \{ \quotep{Q} \in \quotep{\pi} | P \equiv Q \} \and \\ \meaningof{@\quotep{E}} = \{ P \in \pi | P \equiv @x, x \in \meaningof{E} \}}
\end{mathpar}

\begin{eqnarray*}
  \\
  \meaningof{-} : TS \to ST
\end{eqnarray*}

\begin{eqnarray*}
  \\
  L : TS \to ST
\end{eqnarray*}

\begin{eqnarray*}
  \\
  P \models E \iff P \in \meaningof{E}
\end{eqnarray*}

\begin{eqnarray*}
  P \approx_{L} Q \iff \forall E \in L. P \models E \iff Q \models E
\end{eqnarray*}

\begin{eqnarray*}
  P \approx_{K} Q
\end{eqnarray*}

\begin{eqnarray*}
  P \approx Q
\end{eqnarray*}

$\approx_{K} = \approx = \approx_{L}$

\subsubsection{Contextual duality}

Note that contexts extend the quotation operation to a family of
operations from processes to names. Given a context, $M$, we can
define a \emph{nominal context}, $\quotep{M}$ by $\quotep{M}[P] :=
\quotep{M[P]}$. To foreshadow what is to come we observe that these
operations enjoy a duality with processes very much like the duality
between vectors and maps from vectors to scalars.

Further, because the calculus is essentially higher-order, we have a
correspondence between contexts and processes. More specifically,
given a name $x$ and a context $M$ we can construct $M^{*}_{x}$ such
that 

\begin{mathpar}
  M^{*}_{x} | \lift{x}{P} \red M[P]
\end{mathpar}

namely,

\begin{mathpar}
  M^{*}_{x} := x?(u).M[\dropn{u}]
\end{mathpar}

The dependence of $M^{*}_{x}$ on a name makes it an abstraction, 

\begin{mathpar}
  M^{*} := (x)x?(u).M[\dropn{u}]
\end{mathpar}

\subsection{Additional notation}

It will sometimes be convenient to denote the process a name
quotes. We already have the notation $x = \quotep{P}$, but it will be
convenient to introduce an alternate notation, $\procn{x}$, when we
want to emphasize the connection to the use of the name. Note that, by
virtue of name equivalence, $\quotep{\procn{x}} \nameeq x$; so, the
notation is consistent with previous definitions.

Further, because names have structure it is possible to effect
substitutions on the basis of that structure. This means we need to
upgrade our notation for substitutions, which we accomplish by
adapting comprehension notation. Thus,

\begin{mathpar}
  P\{ y / x : x \in S \}
\end{mathpar}

is interpreted to mean the process derived from P by replacing (in a
capture-avoiding manner) each occurrence of $x$ in $S$ by $y$. For example,

\begin{mathpar}
  P\{ \quotep{\procn{x}|\procn{x}} / x : x \in \freenames{P} \}
\end{mathpar}

will replace each (occurrence) of a free name $x$ in $P$ by
$\quotep{\procn{x}|\procn{x}}$.

Also, we will avail ourselves of the notation $x^{L}$ and $x^{R}$ to
denote injections of a name into disjoint copies of the name
space. There are numerous ways to accomplish this. One example can be
found in \cite{MeredithR05}. This notation overloads to vectors of
names: $\vec{x}^{\pi} := (x_{i}^{\pi} \; : \; 0 \leq i < |\vec{x}| )$ where $\pi \in \{L,R\}$.

We also use $P^{\Box} := P|\Box$.

In \cite{MeredithR05} an interpretation of the new operator is
given. It turns out that there are several possible interpretations
all enjoying the requisite algebraic properties of the operator (see
\cite{milner91polyadicpi}). We will therefore make liberal use of
$(\nu\; \vec{x})P$.

% subsection the_syntax_and_semantics_of_the_notation_system (end)   

\input{qm2pi.qmops} 

\input{qm2pi.sterngerlach} 

\input{qm2pi.metric} 

% section concurrent_process_calculi (end)

%\input{qm2pi.proofsketch}

% section proof sketch (end)

%\input{qm2pi.slviaknots} 

% section spatial logic via knots (end)

\input{qm2pi.conclusion}

% section conclusion (end)

%\input{qm2pi.dtcodes} 

% section wiring algorithm (end)

\input{qm2pi.ack} 

% section acknowledgments (end)

\newpage


\bibliographystyle{plain}   
\bibliography{../../biblios/main.bib}

\input{qm2pi.rhodetails}

\end{document}

 

% section notation (end)

\input{qm2pi.process.calculi} 

% section concurrent_process_calculi_and_spatial_logics_ (end)
    
%\documentclass[12pt]{llncs}
%\documentclass{jktr}

\usepackage[pdftex]{hyperref}                   
\usepackage {listings}
\usepackage {mathpartir}
\usepackage{bcprules}
%\usepackage{listings}
                       
\usepackage{graphicx} 
%\usepackage[margins=2.5cm,nohead,nofoot]{geometry}
%\usepackage{geometry}
\usepackage{amsfonts}
\usepackage{amstext}
\usepackage{latexsym}
\usepackage{amssymb}
\usepackage{color}


%\include{myPreamble}
\include{qm2pi.local} 

%\ifpdf
%\usepackage[pdftex]{graphicx}
%\else
%\usepackage{graphicx}
%\fi

 % \ifpdf
%  \usepackage{pdfsync}
%  \if


%\title{Brief Article}
%\author{David F. Snyder}
%\author{L.G. Meredith}

%\address{Dept. of Math., Texas State University--San Marcos, San Marcos, TX 78666}
       
\pagestyle{empty}


\begin{document}

\lstset{language=[Objective]Caml,frame=shadowbox}

\input{qm2pi.front}

% section front matter (end)

\input{qm2pi.intro} 
 
% section introduction (end)

% \input{qm2pi.knotations} 

% section notation (end)

\input{qm2pi.process.calculi} 

% section concurrent_process_calculi_and_spatial_logics_ (end)
    
%\input{qm2pi.knots2pi} 

%\input{qm2pi.trefoil} 

%\input{qm2pi.mainthm} 

% subsection basic_interpretation (end)

%\input{qm2pi.rho.presentation} 
\subsection{The syntax and semantics of the notation system}\label{sub:the_syntax_and_semantics_of_the_notation_system} % (fold)

We now summarize a technical presentation of the calculus that
embodies our theory of dynamics. The typical presentation of such a
calculus follows the style of giving generators and relations on
them. The grammar, below, describing term constructors, freely
generates the set of processes, $\Proc$. This set is then quotiented
by a relation known as structural congruence and it is over this set
that the notion of dynamics is expressed. This presentation is
essentially that of \cite{MeredithR05} with the addition of
polyadicity and summation. For readability we have relegated some of
the technical subtleties to an appendix.

\subsubsection{Process grammar}\label{subsub:process_grammar}

\begin{mathpar}
  \inferrule* [lab=synchronization] {} {{M} \bc \pzero \;|\; x?F \;|\; x!C }
  \and
  \inferrule* [lab=abstraction] {} {{F} \bc (x)P}
  \and
  \inferrule* [lab=concretion] {} {{C} \bc \langle Q \rangle}
  \and
  \inferrule* [lab=process] {} {{P,Q} \bc M \;| \;P|Q \;|\; @{x}}
  \and
  \inferrule* [lab=name] {} {{x} \bc \quotep{P}}
\end{mathpar} 

Note that $\vec{x}$ (resp. $\vec{P}$) denotes a vector of names
(resp. processes) of length $|\vec{x}|$ (resp. $|\vec{P}|$). We adopt
the following useful abbreviations.

\begin{mathpar}
   x?(\vec{y}).P := x.(\vec{y})P \and  x\clift{\vec{P}} := x.\clift{\vec{P}}
   \and x!(y) := \lift{x}{\dropn{y}}
   \and \Pi_{i=0}^{n-1}P_i := P_0 | \ldots | P_{n-1}
\end{mathpar}

\subsubsection{Structural congruence}

\paragraph{Free and bound names and alpha-equivalence.} At the
core of structural equivalence is alpha-equivalence which identifies
process that are the same up to a change of variable. Formally, we
recognize the distinction between free and bound names. The free names
of a process, $\freenames{P}$, may be calculated recursively as
follows:

\begin{mathpar}
\freenames{\pzero} := \emptyset
  \and \\
  \freenames{x?(y).P} := \{ x \} \cup (\freenames{P} \setminus \{ y \})
  \and 
  \freenames{x!\langle P \rangle} := \{ x \} \cup \{ P \} 
  \and \\
  \freenames{P|Q} := \freenames{P} \cup \freenames{Q}
  \and \\
  \freenames{@{x}} := \{ x \}
\end{mathpar}

$\pi$
$\quotep{\pi}$

$\freenames{-} : \pi \to \mathcal{P}(\quotep{\pi})$

\begin{eqnarray*}
  \freenames{\pzero} & := & \emptyset \\
  \freenames{x?(y).P} & := & \{ x \} \cup (\freenames{P} \setminus \{ y \}) \\
  \freenames{x!\langle P \rangle} & := & \{ x \} \cup \{ P \} \\
  \freenames{P|Q} & := & \freenames{P} \cup \freenames{Q} \\
  \freenames{\dropn{x}} & := & \{ x \}
\end{eqnarray*}

The bound names of a process, $\boundnames{P}$, are those names occurring in $P$
that are not free. For example, in $x?(y).0$, the name $x$ is free, while $y$ is bound.

\begin{mathpar}
  \inferrule* [lab=monoidal-laws] {} { P|Q \equiv Q|P \and P|0 \equiv P \and P|(Q|R) \equiv (P|Q)|R }
\end{mathpar}

\begin{mathpar}
  \inferrule* [lab=alpha-equivalence] {} { (x)P \equiv (y)P\{y/x\} \and y \not\in \freenames{P} }
\end{mathpar}

\begin{definition}
Then two processes, $P,Q$, are alpha-equivalent if $P = Q\{\vec{y}/\vec{x}\}$ for
some $\vec{x} \in \boundnames{Q},\vec{y} \in \boundnames{P}$, where $Q\{\vec{y}/\vec{x}\}$
denotes the capture-avoiding substitution of $\vec{y}$ for $\vec{x}$ in $Q$.
\end{definition}

\begin{definition}
  The {\em structural congruence} \cite{SangiorgiWalker} , $\equiv$,
  between processes is the least congruence containing
  alpha-equivalence, satisfying the abelian monoid laws
  (associativity, commutativity and $\pzero$ as identity) for parallel
  composition $|$ and for summation $+$.
\end{definition}

\subsection{Name equivalence}

We take name equivalence, written $\nameeq$, to be the smallest
equivalence relation generated by the following rules.

\begin{mathpar}
\inferrule*[lab=Quote-drop]
{ }
{ \quotep{@{x}} \nameeq x }

\inferrule*[lab=Struct-equiv]
{ P \scong Q }
{ \quotep{P} \nameeq \quotep{Q} }
\end{mathpar}

The astute reader will have noticed that the mutual recursion of names
and processes imposes a mutual recursion on alpha-equivalence and
structural equivalence via name-equivalence. Fortunately, all of this
works out pleasantly and we may calculate in the natural way, free of
concern. The reader interested in the details is referred to the
appendix \ref{appendix:rho_details}.

\subsection{Substitution}

We use $\Proc$ for the set of processes, $\QProc$ for the set of
names, and $\id{\{}\vec{y} / \vec{x} \id{\}}$ to denote partial maps,
$s : \QProc \rightarrow \QProc$. A map, $s$ lifts, uniquely, to a map
on process terms, $\widehat{s} : \Proc \rightarrow \Proc$ by the
following equations.

\begin{mathpar}
  (0) \psubstp{Q}{P} := 0 \\
  (R \juxtap S) \psubstp{Q}{P}
  :=    
  (R)\psubstp{Q}{P} \juxtap (S) \psubstp{Q}{P} \\
  (x?(y).R) \psubstp{Q}{P}    
  :=    
  (x)\substp{Q}{P} (z)\concat( (R \psubstn{z}{y}) \psubstp{Q}{P} ) \\
  (\lift{x}{R}) \psubstp{Q}{P}  
  :=
  \lift{(x)\substp{Q}{P}}{ R \psubstp{Q}{P} } \\
%   (\dropn{x})  \psubstp{Q}{P}       
%   := 
%   \left\{ 
%     \begin{array}{ccc} 
%       \dropn{\quotep{Q}} & & x \nameeq \quotep{P} \\
%       \dropn{x} & & otherwise \\
%     \end{array}
%   \right. 
  (\dropn{x})  \psubstp{Q}{P}       
  := 
  \left\{ 
    \begin{array}{ccc} 
      Q & & x \nameeq \quotep{P} \\
      \dropn{x} & & otherwise \\
    \end{array}
  \right.
\end{mathpar}
 

where

\begin{eqnarray}
  (x)\id{\{} \lpquote Q \rpquote / \lpquote P \rpquote \id{\}}            = 
  \left\{ 
    \begin{array}{ccc}
      \lpquote Q \rpquote & & x \nameeq \lpquote P \rpquote \\
      x & & otherwise \\
    \end{array}
  \right. \nonumber
\end{eqnarray}

and $z$ is chosen distinct from $\quotep{P}$, $\quotep{Q}$, the free
names in $Q$, and all the names in $R$. Our $\alpha$-equivalence will
be built in the standard way from this substitution.

\begin{remark}\label{rem:no_self_referential_names}
  One consequence of these definitions is that $\forall P. \quotep{P}
  \not\in \freenames{P}$.
\end{remark}

\subsection{ Dynamic quote: an example }

Anticipating something of what's to come, consider applying the
substitution, $\widehat{\id{\{}u / z \id{\}}}$, to the following pair
of processes, $\lift{w}{y!(z)}$ and $w[ \lpquote y!(z) \rpquote ]$.

\begin{eqnarray}
	\lift{w}{y!(z)}\widehat{\id{\{}u / z \id{\}}}
		& = &
		\lift{w}{y!(u)} \nonumber\\
	w[ \lpquote y!(z) \rpquote ] \widehat{ \id{\{}u / z \id{\}} }
		& = &
		w[ \lpquote y!(z) \rpquote ] \nonumber
\end{eqnarray}

Because the body of the process between quotes is impervious to
substitution, we get radically different answers. In fact, by
examining the first process in an input context,
e.g. $x?(z).\lift{w}{y!(z)}$, we see that the process under the lift
operator may be shaped by prefixed inputs binding a name inside it. In
this sense, the lift operator will be seen as a way to dynamically
construct processes before reifying them as names.

Finally equipped with these standard features we can present the
dynamics of the calculus.

\subsubsection{Operational semantics} 

Finally, we introduce the computational dynamics. What marks these
algebras as distinct from other more traditionally studied algebraic
structures, e.g. vector spaces or polynomial rings, is the manner in
which dynamics is captured. In traditional structures, dynamics is typically
expressed through morphisms between such structures, as in linear maps
between vector spaces or morphisms between rings. In algebras
associated with the semantics of computation, the dynamics is
expressed as part of the algebraic structure itself, through a
reduction reduction relation typically denoted by $\red$. Below, we
give a recursive presentation of this relation for the calculus used
in the encoding.

$\red \subseteq \pi \times \pi$
$\red : \pi \to \mathcal{P}(\pi)$

\begin{mathpar}
  \inferrule* [lab=Comm] { \textsf{match}( x_{src}, x_{trgt} ) } { x_{trgt}?(y)P \; | \; x_{src}!\langle {Q} \rangle \red P\{\quotep{Q}/y}\} }
  \and \\
  \inferrule* [lab=Par] {{P} \red {P}'} {{{P} | {Q}} \red {{P}' | {Q}}}
  \and
  \inferrule* [lab=Equiv]{{{P} \scong {P}'} \andalso {{P}' \red {Q}'} \andalso {{Q}' \scong {Q}}}{{P} \red {Q}}
\end{mathpar}

\begin{eqnarray*}
  match_{\equiv} (\quotep{P},\quotep{Q}) & := & P \equiv Q \\
  match_{\dagger}(\quotep{P},\quotep{Q}) & := & \forall R. P|Q \red^{*} R => R \red^{*} 0 \\
  match_{K}(\quotep{P},\quotep{Q}) & := & K \mbox{ for some context } K
\end{eqnarray*}

$u?(x)P | u!\langle Q \rangle \red P\{\quotep{Q}/x\}$

%We write $\wred$ for $\red^*$, and $P\red$ if $\exists Q $ such that $ P \red Q$.
We write $P\red$ if $\exists Q $ such that $ P \red Q$ and $P\not\red$, otherwise.

\section{Replication}

As mentioned before, it is known that replication (and hence
recursion) can be implemented in a higher-order process algebra
\cite{SangiorgiWalker}. As our first example of calculation with the
machinery thus far presented we give the construction explicitly in
the {\rhoc}.

\begin{eqnarray}
	D_{x} & := & \prefix{x}{y}{(\binpar{\outputp{x}{y}}{@{y}})} \nonumber\\
	\bangp_{x}{P} & := & \binpar{{x}!\langle{\binpar{D_{x}}{P}}\rangle}{D_{x}} \nonumber
\end{eqnarray}

\begin{eqnarray}
	\bangp_{x}{P} & & \nonumber\\
	=
	& {x}!\langle{(\prefix{x}{y}{(\outputp{x}{y} | @{y})) | P}}\rangle 
	      | \prefix{x}{y}{(\outputp{x}{y} | @{y})} & \nonumber\\
	\red
	& (\outputp{x}{y} | @{y})\substn{\quotep{(\prefix{x}{y}{(@{y} | \outputp{x}{y})) | P}}}{y} & \nonumber\\
	=
	& \outputp{x}{\quotep{(\prefix{x}{y}{(\outputp{x}{y} | @{y})) | P}}}
	  | {(\prefix{x}{y}{(\outputp{x}{y} | @{y})) | P}} & \nonumber\\
	\red
	& \ldots & \nonumber\\
	\red^*
	& P | P | \ldots & \nonumber
\end{eqnarray}

Of course, this encoding, as an implementation, runs away, unfolding
$\bangp{P}$ eagerly. A lazier and more implementable replication
operator, restricted to input-guarded processes, may be obtained as follows.

\begin{eqnarray}
\bangp{\prefix{u}{v}{P}} 
	:= 
	\binpar{\lift{x}{\prefix{u}{v}{(\binpar{D(x)}{P})}}}{D(x)} \nonumber
\end{eqnarray}

\begin{remark}
  Note that the lazier definition still does not deal with summation
  or mixed summation (i.e. sums over input and output). The reader is
  invited to construct definitions of replication that deal with these
  features. 

  Further, the definitions are parameterized in a name, $x$. Can you,
  gentle reader, make a definition that eliminates this parameter and
  guarantees no accidental interaction between the replication
  machinery and the process being replicated -- i.e. no accidental
  sharing of names used by the process to get its work done and the
  name(s) used by the replication to effect copying. This latter
  revision of the definition of replication is crucial to obtaining
  the expected identity $!!P \sim !P$.
\end{remark}

\begin{remark}\label{rem:paradoxical_combinator}
  The reader familiar with the lambda calculus will have noticed the
  similarity between $D$ and the paradoxical combinator.

  [Ed. note: the existence of this seems to suggest we have to be more
  restrictive on the set of processes and names we admit if we are to
  support no-cloning.]
\end{remark}

\subsubsection{Bisimulation}

The computational dynamics gives rise to another kind of equivalence,
the equivalence of computational behavior. As previously mentioned
this is typically captured \emph{via} some form of bisimulation.

% The notion we use in this paper is weak barbed bisimulation
% \cite{milner91polyadicpi}.

The notion we use in this paper is derived from weak barbed
bisimulation \cite{milner91polyadicpi}. 

\begin{definition}
An \emph{observation relation}, $\downarrow_{\mathcal N}$, over a set
of names, $\mathcal N$, is the smallest relation satisfying the rules
below.

\infrule[Out-barb]{y \in {\mathcal N}, \; x \nameeq y}
		  {\outputp{x}{v} \downarrow_{\mathcal N} x}
\infrule[Par-barb]{\mbox{$P\downarrow_{\mathcal N} x$ or $Q\downarrow_{\mathcal N} x$}}
		  {\binpar{P}{Q} \downarrow_{\mathcal N} x}

We write $P \Downarrow_{\mathcal N} x$ if there is $Q$ such that 
$P \wred Q$ and $Q \downarrow_{\mathcal N} x$.
\end{definition}

\begin{definition}
%\label{def.bbisim}
An  ${\mathcal N}$-\emph{barbed bisimulation} over a set of names, ${\mathcal N}$, is a symmetric binary relation 
${\mathcal S}_{\mathcal N}$ between agents such that $P\rel{S}_{\mathcal N}Q$ implies:
\begin{enumerate}
\item If $P \red P'$ then $Q \wred Q'$ and $P'\rel{S}_{\mathcal N} Q'$.
\item If $P\downarrow_{\mathcal N} x$, then $Q\Downarrow_{\mathcal N} x$.
\end{enumerate}
$P$ is ${\mathcal N}$-barbed bisimilar to $Q$, written
$P \wbbisim_{\mathcal N} Q$, if $P \rel{S}_{\mathcal N} Q$ for some ${\mathcal N}$-barbed bisimulation ${\mathcal S}_{\mathcal N}$.
\end{definition}

$\mathcal{R} \subseteq \pi \times \pi$

$P \mathcal{R} Q => \forall P'. P \red P' \Rightarrow \exists Q'. Q \red Q', P' \mathcal{R} Q'$

$P \vdash x \Rightarrow Q \vdash x$

\begin{mathpar}
  \inferrule*[lab=Out-barb]{x \nameeq y}{{y}!\langle{Q}\rangle \vdash x}
  \and
  \inferrule*[lab=Par-barb]{\mbox{$P\vdash x$ or $Q\vdash x$}}{\binpar{P}{Q} \vdash x}
\end{mathpar}

\subsubsection{Contexts}

One of the principle advantages of computational calculi like the
$\pi$-calculus is a well-defined notion of context,
contextual-equivalence and a correlation between
contextual-equivalence and notions of bisimulation. The notion of
context allows the decomposition of a process into (sub-)process and
its syntactic environment, its context. Thus, a context may be
thought of as a process with a ``hole'' (written $\Box$) in it. The
application of a context $M$ to a process $P$, written $M[P]$, is
tantamount to filling the hole in $M$ with $P$. In this paper we do
not need the full weight of this theory, but do make use of the notion
of context in the proof the main theorem. 

\begin{mathpar}
  \inferrule* [lab=summation] {} {{M_{M},M_{N}} \bc \Box \;|\; x.M_{A} \;|\; M_{M}+M_{N}}
  \and
  \inferrule* [lab=agent] {} {{M_{A}} \bc (\vec{x})M_{P} \;| \; \clift{P_0,\ldots,M_{P},\ldots,P_N}}
  \and \\
  \inferrule* [lab=process] {} {{M_{P}} \bc M_{N} \;| \;P|M_{P} }
\end{mathpar} 

\begin{mathpar}
  \inferrule* [lab=sychronization] {} {M_{N} \bc \Box \;|\; x?M_{F} \;|\; x!M_{C}}
  \and
  \inferrule* [lab=abstraction] {} {{M_{F}} \bc (x)M_{P} }
  \and
  \inferrule* [lab=concretion] {} {{M_{C}} \bc \langle M_{P} \rangle }
  \and \\
  \inferrule* [lab=process] {} {{M_{P}} \bc M_{N} \;| \;P|M_{P} }
\end{mathpar}

\begin{definition}[contextual application] Given a context $M$, and
  process $P$, we define the \emph{contextual application}, $M[P] :=
  M\{P/\Box\}$. That is, the contextual application of M to P is the
  substitution of $P$ for $\Box$ in $M$.
\end{definition}

$\meaningof{-} : L \to \mathcal{P}(\pi)$

\begin{mathpar}
  \inferrule* [lab=collection] {} {\meaningof{true} = \pi, \and \meaningof{~E} = \pi \setminus \meaningof{E}, \and \meaningof{E_{1} \& E_{2}} = \meaningof{E_{1}} \cap \meaningof{E_{2}}}
\end{mathpar}

\begin{mathpar}
  \inferrule* [lab=structure] {} {\meaningof{0} = \{ P \in \pi | P \equiv 0 \}, \and \\ \meaningof{E_1 | E_2} = \{ P \in \pi | P \equiv P_{1} | P_{2}, P_{1} \in \meaningof{E_{1}}, P_{2} \in \meaningof{E_2}\} }
\end{mathpar}

\begin{mathpar}
 \inferrule* [lab=behavior] {} {\meaningof{\langle a?b \rangle E} = \{ P \in \pi | P \equiv Q | u?(y)P', \\ \and \\\\ \and \\ \;\;\; u \in \meaningof{a}, \forall z.P'\{z/y\} \in \meaningof{E\{z/b\}}\}, \and \\ \meaningof{a!E} = \{ P \in \pi | P \equiv Q | x!\langle P' \rangle, x \in \meaningof{a} P' \in \meaningof{E}\} }
\end{mathpar}

\begin{mathpar}
 \inferrule* [lab=nominal] {} {\meaningof{\quotep{E}} = \{ \quotep{P} \in \quotep{\pi} | P \in \meaningof{E} \}, \and \meaningof{\quotep{P}} = \{ \quotep{Q} \in \quotep{\pi} | P \equiv Q \} \and \\ \meaningof{@\quotep{E}} = \{ P \in \pi | P \equiv @x, x \in \meaningof{E} \}}
\end{mathpar}

\begin{eqnarray*}
  \\
  \meaningof{-} : TS \to ST
\end{eqnarray*}

\begin{eqnarray*}
  \\
  L : TS \to ST
\end{eqnarray*}

\begin{eqnarray*}
  \\
  P \models E \iff P \in \meaningof{E}
\end{eqnarray*}

\begin{eqnarray*}
  P \approx_{L} Q \iff \forall E \in L. P \models E \iff Q \models E
\end{eqnarray*}

\begin{eqnarray*}
  P \approx_{K} Q
\end{eqnarray*}

\begin{eqnarray*}
  P \approx Q
\end{eqnarray*}

$\approx_{K} = \approx = \approx_{L}$

\subsubsection{Contextual duality}

Note that contexts extend the quotation operation to a family of
operations from processes to names. Given a context, $M$, we can
define a \emph{nominal context}, $\quotep{M}$ by $\quotep{M}[P] :=
\quotep{M[P]}$. To foreshadow what is to come we observe that these
operations enjoy a duality with processes very much like the duality
between vectors and maps from vectors to scalars.

Further, because the calculus is essentially higher-order, we have a
correspondence between contexts and processes. More specifically,
given a name $x$ and a context $M$ we can construct $M^{*}_{x}$ such
that 

\begin{mathpar}
  M^{*}_{x} | \lift{x}{P} \red M[P]
\end{mathpar}

namely,

\begin{mathpar}
  M^{*}_{x} := x?(u).M[\dropn{u}]
\end{mathpar}

The dependence of $M^{*}_{x}$ on a name makes it an abstraction, 

\begin{mathpar}
  M^{*} := (x)x?(u).M[\dropn{u}]
\end{mathpar}

\subsection{Additional notation}

It will sometimes be convenient to denote the process a name
quotes. We already have the notation $x = \quotep{P}$, but it will be
convenient to introduce an alternate notation, $\procn{x}$, when we
want to emphasize the connection to the use of the name. Note that, by
virtue of name equivalence, $\quotep{\procn{x}} \nameeq x$; so, the
notation is consistent with previous definitions.

Further, because names have structure it is possible to effect
substitutions on the basis of that structure. This means we need to
upgrade our notation for substitutions, which we accomplish by
adapting comprehension notation. Thus,

\begin{mathpar}
  P\{ y / x : x \in S \}
\end{mathpar}

is interpreted to mean the process derived from P by replacing (in a
capture-avoiding manner) each occurrence of $x$ in $S$ by $y$. For example,

\begin{mathpar}
  P\{ \quotep{\procn{x}|\procn{x}} / x : x \in \freenames{P} \}
\end{mathpar}

will replace each (occurrence) of a free name $x$ in $P$ by
$\quotep{\procn{x}|\procn{x}}$.

Also, we will avail ourselves of the notation $x^{L}$ and $x^{R}$ to
denote injections of a name into disjoint copies of the name
space. There are numerous ways to accomplish this. One example can be
found in \cite{MeredithR05}. This notation overloads to vectors of
names: $\vec{x}^{\pi} := (x_{i}^{\pi} \; : \; 0 \leq i < |\vec{x}| )$ where $\pi \in \{L,R\}$.

We also use $P^{\Box} := P|\Box$.

In \cite{MeredithR05} an interpretation of the new operator is
given. It turns out that there are several possible interpretations
all enjoying the requisite algebraic properties of the operator (see
\cite{milner91polyadicpi}). We will therefore make liberal use of
$(\nu\; \vec{x})P$.

% subsection the_syntax_and_semantics_of_the_notation_system (end)   

\input{qm2pi.qmops} 

\input{qm2pi.sterngerlach} 

\input{qm2pi.metric} 

% section concurrent_process_calculi (end)

%\input{qm2pi.proofsketch}

% section proof sketch (end)

%\input{qm2pi.slviaknots} 

% section spatial logic via knots (end)

\input{qm2pi.conclusion}

% section conclusion (end)

%\input{qm2pi.dtcodes} 

% section wiring algorithm (end)

\input{qm2pi.ack} 

% section acknowledgments (end)

\newpage


\bibliographystyle{plain}   
\bibliography{../../biblios/main.bib}

\input{qm2pi.rhodetails}

\end{document}

 

%\documentclass[12pt]{llncs}
%\documentclass{jktr}

\usepackage[pdftex]{hyperref}                   
\usepackage {listings}
\usepackage {mathpartir}
\usepackage{bcprules}
%\usepackage{listings}
                       
\usepackage{graphicx} 
%\usepackage[margins=2.5cm,nohead,nofoot]{geometry}
%\usepackage{geometry}
\usepackage{amsfonts}
\usepackage{amstext}
\usepackage{latexsym}
\usepackage{amssymb}
\usepackage{color}


%\include{myPreamble}
\include{qm2pi.local} 

%\ifpdf
%\usepackage[pdftex]{graphicx}
%\else
%\usepackage{graphicx}
%\fi

 % \ifpdf
%  \usepackage{pdfsync}
%  \if


%\title{Brief Article}
%\author{David F. Snyder}
%\author{L.G. Meredith}

%\address{Dept. of Math., Texas State University--San Marcos, San Marcos, TX 78666}
       
\pagestyle{empty}


\begin{document}

\lstset{language=[Objective]Caml,frame=shadowbox}

\input{qm2pi.front}

% section front matter (end)

\input{qm2pi.intro} 
 
% section introduction (end)

% \input{qm2pi.knotations} 

% section notation (end)

\input{qm2pi.process.calculi} 

% section concurrent_process_calculi_and_spatial_logics_ (end)
    
%\input{qm2pi.knots2pi} 

%\input{qm2pi.trefoil} 

%\input{qm2pi.mainthm} 

% subsection basic_interpretation (end)

%\input{qm2pi.rho.presentation} 
\subsection{The syntax and semantics of the notation system}\label{sub:the_syntax_and_semantics_of_the_notation_system} % (fold)

We now summarize a technical presentation of the calculus that
embodies our theory of dynamics. The typical presentation of such a
calculus follows the style of giving generators and relations on
them. The grammar, below, describing term constructors, freely
generates the set of processes, $\Proc$. This set is then quotiented
by a relation known as structural congruence and it is over this set
that the notion of dynamics is expressed. This presentation is
essentially that of \cite{MeredithR05} with the addition of
polyadicity and summation. For readability we have relegated some of
the technical subtleties to an appendix.

\subsubsection{Process grammar}\label{subsub:process_grammar}

\begin{mathpar}
  \inferrule* [lab=synchronization] {} {{M} \bc \pzero \;|\; x?F \;|\; x!C }
  \and
  \inferrule* [lab=abstraction] {} {{F} \bc (x)P}
  \and
  \inferrule* [lab=concretion] {} {{C} \bc \langle Q \rangle}
  \and
  \inferrule* [lab=process] {} {{P,Q} \bc M \;| \;P|Q \;|\; @{x}}
  \and
  \inferrule* [lab=name] {} {{x} \bc \quotep{P}}
\end{mathpar} 

Note that $\vec{x}$ (resp. $\vec{P}$) denotes a vector of names
(resp. processes) of length $|\vec{x}|$ (resp. $|\vec{P}|$). We adopt
the following useful abbreviations.

\begin{mathpar}
   x?(\vec{y}).P := x.(\vec{y})P \and  x\clift{\vec{P}} := x.\clift{\vec{P}}
   \and x!(y) := \lift{x}{\dropn{y}}
   \and \Pi_{i=0}^{n-1}P_i := P_0 | \ldots | P_{n-1}
\end{mathpar}

\subsubsection{Structural congruence}

\paragraph{Free and bound names and alpha-equivalence.} At the
core of structural equivalence is alpha-equivalence which identifies
process that are the same up to a change of variable. Formally, we
recognize the distinction between free and bound names. The free names
of a process, $\freenames{P}$, may be calculated recursively as
follows:

\begin{mathpar}
\freenames{\pzero} := \emptyset
  \and \\
  \freenames{x?(y).P} := \{ x \} \cup (\freenames{P} \setminus \{ y \})
  \and 
  \freenames{x!\langle P \rangle} := \{ x \} \cup \{ P \} 
  \and \\
  \freenames{P|Q} := \freenames{P} \cup \freenames{Q}
  \and \\
  \freenames{@{x}} := \{ x \}
\end{mathpar}

$\pi$
$\quotep{\pi}$

$\freenames{-} : \pi \to \mathcal{P}(\quotep{\pi})$

\begin{eqnarray*}
  \freenames{\pzero} & := & \emptyset \\
  \freenames{x?(y).P} & := & \{ x \} \cup (\freenames{P} \setminus \{ y \}) \\
  \freenames{x!\langle P \rangle} & := & \{ x \} \cup \{ P \} \\
  \freenames{P|Q} & := & \freenames{P} \cup \freenames{Q} \\
  \freenames{\dropn{x}} & := & \{ x \}
\end{eqnarray*}

The bound names of a process, $\boundnames{P}$, are those names occurring in $P$
that are not free. For example, in $x?(y).0$, the name $x$ is free, while $y$ is bound.

\begin{mathpar}
  \inferrule* [lab=monoidal-laws] {} { P|Q \equiv Q|P \and P|0 \equiv P \and P|(Q|R) \equiv (P|Q)|R }
\end{mathpar}

\begin{mathpar}
  \inferrule* [lab=alpha-equivalence] {} { (x)P \equiv (y)P\{y/x\} \and y \not\in \freenames{P} }
\end{mathpar}

\begin{definition}
Then two processes, $P,Q$, are alpha-equivalent if $P = Q\{\vec{y}/\vec{x}\}$ for
some $\vec{x} \in \boundnames{Q},\vec{y} \in \boundnames{P}$, where $Q\{\vec{y}/\vec{x}\}$
denotes the capture-avoiding substitution of $\vec{y}$ for $\vec{x}$ in $Q$.
\end{definition}

\begin{definition}
  The {\em structural congruence} \cite{SangiorgiWalker} , $\equiv$,
  between processes is the least congruence containing
  alpha-equivalence, satisfying the abelian monoid laws
  (associativity, commutativity and $\pzero$ as identity) for parallel
  composition $|$ and for summation $+$.
\end{definition}

\subsection{Name equivalence}

We take name equivalence, written $\nameeq$, to be the smallest
equivalence relation generated by the following rules.

\begin{mathpar}
\inferrule*[lab=Quote-drop]
{ }
{ \quotep{@{x}} \nameeq x }

\inferrule*[lab=Struct-equiv]
{ P \scong Q }
{ \quotep{P} \nameeq \quotep{Q} }
\end{mathpar}

The astute reader will have noticed that the mutual recursion of names
and processes imposes a mutual recursion on alpha-equivalence and
structural equivalence via name-equivalence. Fortunately, all of this
works out pleasantly and we may calculate in the natural way, free of
concern. The reader interested in the details is referred to the
appendix \ref{appendix:rho_details}.

\subsection{Substitution}

We use $\Proc$ for the set of processes, $\QProc$ for the set of
names, and $\id{\{}\vec{y} / \vec{x} \id{\}}$ to denote partial maps,
$s : \QProc \rightarrow \QProc$. A map, $s$ lifts, uniquely, to a map
on process terms, $\widehat{s} : \Proc \rightarrow \Proc$ by the
following equations.

\begin{mathpar}
  (0) \psubstp{Q}{P} := 0 \\
  (R \juxtap S) \psubstp{Q}{P}
  :=    
  (R)\psubstp{Q}{P} \juxtap (S) \psubstp{Q}{P} \\
  (x?(y).R) \psubstp{Q}{P}    
  :=    
  (x)\substp{Q}{P} (z)\concat( (R \psubstn{z}{y}) \psubstp{Q}{P} ) \\
  (\lift{x}{R}) \psubstp{Q}{P}  
  :=
  \lift{(x)\substp{Q}{P}}{ R \psubstp{Q}{P} } \\
%   (\dropn{x})  \psubstp{Q}{P}       
%   := 
%   \left\{ 
%     \begin{array}{ccc} 
%       \dropn{\quotep{Q}} & & x \nameeq \quotep{P} \\
%       \dropn{x} & & otherwise \\
%     \end{array}
%   \right. 
  (\dropn{x})  \psubstp{Q}{P}       
  := 
  \left\{ 
    \begin{array}{ccc} 
      Q & & x \nameeq \quotep{P} \\
      \dropn{x} & & otherwise \\
    \end{array}
  \right.
\end{mathpar}
 

where

\begin{eqnarray}
  (x)\id{\{} \lpquote Q \rpquote / \lpquote P \rpquote \id{\}}            = 
  \left\{ 
    \begin{array}{ccc}
      \lpquote Q \rpquote & & x \nameeq \lpquote P \rpquote \\
      x & & otherwise \\
    \end{array}
  \right. \nonumber
\end{eqnarray}

and $z$ is chosen distinct from $\quotep{P}$, $\quotep{Q}$, the free
names in $Q$, and all the names in $R$. Our $\alpha$-equivalence will
be built in the standard way from this substitution.

\begin{remark}\label{rem:no_self_referential_names}
  One consequence of these definitions is that $\forall P. \quotep{P}
  \not\in \freenames{P}$.
\end{remark}

\subsection{ Dynamic quote: an example }

Anticipating something of what's to come, consider applying the
substitution, $\widehat{\id{\{}u / z \id{\}}}$, to the following pair
of processes, $\lift{w}{y!(z)}$ and $w[ \lpquote y!(z) \rpquote ]$.

\begin{eqnarray}
	\lift{w}{y!(z)}\widehat{\id{\{}u / z \id{\}}}
		& = &
		\lift{w}{y!(u)} \nonumber\\
	w[ \lpquote y!(z) \rpquote ] \widehat{ \id{\{}u / z \id{\}} }
		& = &
		w[ \lpquote y!(z) \rpquote ] \nonumber
\end{eqnarray}

Because the body of the process between quotes is impervious to
substitution, we get radically different answers. In fact, by
examining the first process in an input context,
e.g. $x?(z).\lift{w}{y!(z)}$, we see that the process under the lift
operator may be shaped by prefixed inputs binding a name inside it. In
this sense, the lift operator will be seen as a way to dynamically
construct processes before reifying them as names.

Finally equipped with these standard features we can present the
dynamics of the calculus.

\subsubsection{Operational semantics} 

Finally, we introduce the computational dynamics. What marks these
algebras as distinct from other more traditionally studied algebraic
structures, e.g. vector spaces or polynomial rings, is the manner in
which dynamics is captured. In traditional structures, dynamics is typically
expressed through morphisms between such structures, as in linear maps
between vector spaces or morphisms between rings. In algebras
associated with the semantics of computation, the dynamics is
expressed as part of the algebraic structure itself, through a
reduction reduction relation typically denoted by $\red$. Below, we
give a recursive presentation of this relation for the calculus used
in the encoding.

$\red \subseteq \pi \times \pi$
$\red : \pi \to \mathcal{P}(\pi)$

\begin{mathpar}
  \inferrule* [lab=Comm] { \textsf{match}( x_{src}, x_{trgt} ) } { x_{trgt}?(y)P \; | \; x_{src}!\langle {Q} \rangle \red P\{\quotep{Q}/y}\} }
  \and \\
  \inferrule* [lab=Par] {{P} \red {P}'} {{{P} | {Q}} \red {{P}' | {Q}}}
  \and
  \inferrule* [lab=Equiv]{{{P} \scong {P}'} \andalso {{P}' \red {Q}'} \andalso {{Q}' \scong {Q}}}{{P} \red {Q}}
\end{mathpar}

\begin{eqnarray*}
  match_{\equiv} (\quotep{P},\quotep{Q}) & := & P \equiv Q \\
  match_{\dagger}(\quotep{P},\quotep{Q}) & := & \forall R. P|Q \red^{*} R => R \red^{*} 0 \\
  match_{K}(\quotep{P},\quotep{Q}) & := & K \mbox{ for some context } K
\end{eqnarray*}

$u?(x)P | u!\langle Q \rangle \red P\{\quotep{Q}/x\}$

%We write $\wred$ for $\red^*$, and $P\red$ if $\exists Q $ such that $ P \red Q$.
We write $P\red$ if $\exists Q $ such that $ P \red Q$ and $P\not\red$, otherwise.

\section{Replication}

As mentioned before, it is known that replication (and hence
recursion) can be implemented in a higher-order process algebra
\cite{SangiorgiWalker}. As our first example of calculation with the
machinery thus far presented we give the construction explicitly in
the {\rhoc}.

\begin{eqnarray}
	D_{x} & := & \prefix{x}{y}{(\binpar{\outputp{x}{y}}{@{y}})} \nonumber\\
	\bangp_{x}{P} & := & \binpar{{x}!\langle{\binpar{D_{x}}{P}}\rangle}{D_{x}} \nonumber
\end{eqnarray}

\begin{eqnarray}
	\bangp_{x}{P} & & \nonumber\\
	=
	& {x}!\langle{(\prefix{x}{y}{(\outputp{x}{y} | @{y})) | P}}\rangle 
	      | \prefix{x}{y}{(\outputp{x}{y} | @{y})} & \nonumber\\
	\red
	& (\outputp{x}{y} | @{y})\substn{\quotep{(\prefix{x}{y}{(@{y} | \outputp{x}{y})) | P}}}{y} & \nonumber\\
	=
	& \outputp{x}{\quotep{(\prefix{x}{y}{(\outputp{x}{y} | @{y})) | P}}}
	  | {(\prefix{x}{y}{(\outputp{x}{y} | @{y})) | P}} & \nonumber\\
	\red
	& \ldots & \nonumber\\
	\red^*
	& P | P | \ldots & \nonumber
\end{eqnarray}

Of course, this encoding, as an implementation, runs away, unfolding
$\bangp{P}$ eagerly. A lazier and more implementable replication
operator, restricted to input-guarded processes, may be obtained as follows.

\begin{eqnarray}
\bangp{\prefix{u}{v}{P}} 
	:= 
	\binpar{\lift{x}{\prefix{u}{v}{(\binpar{D(x)}{P})}}}{D(x)} \nonumber
\end{eqnarray}

\begin{remark}
  Note that the lazier definition still does not deal with summation
  or mixed summation (i.e. sums over input and output). The reader is
  invited to construct definitions of replication that deal with these
  features. 

  Further, the definitions are parameterized in a name, $x$. Can you,
  gentle reader, make a definition that eliminates this parameter and
  guarantees no accidental interaction between the replication
  machinery and the process being replicated -- i.e. no accidental
  sharing of names used by the process to get its work done and the
  name(s) used by the replication to effect copying. This latter
  revision of the definition of replication is crucial to obtaining
  the expected identity $!!P \sim !P$.
\end{remark}

\begin{remark}\label{rem:paradoxical_combinator}
  The reader familiar with the lambda calculus will have noticed the
  similarity between $D$ and the paradoxical combinator.

  [Ed. note: the existence of this seems to suggest we have to be more
  restrictive on the set of processes and names we admit if we are to
  support no-cloning.]
\end{remark}

\subsubsection{Bisimulation}

The computational dynamics gives rise to another kind of equivalence,
the equivalence of computational behavior. As previously mentioned
this is typically captured \emph{via} some form of bisimulation.

% The notion we use in this paper is weak barbed bisimulation
% \cite{milner91polyadicpi}.

The notion we use in this paper is derived from weak barbed
bisimulation \cite{milner91polyadicpi}. 

\begin{definition}
An \emph{observation relation}, $\downarrow_{\mathcal N}$, over a set
of names, $\mathcal N$, is the smallest relation satisfying the rules
below.

\infrule[Out-barb]{y \in {\mathcal N}, \; x \nameeq y}
		  {\outputp{x}{v} \downarrow_{\mathcal N} x}
\infrule[Par-barb]{\mbox{$P\downarrow_{\mathcal N} x$ or $Q\downarrow_{\mathcal N} x$}}
		  {\binpar{P}{Q} \downarrow_{\mathcal N} x}

We write $P \Downarrow_{\mathcal N} x$ if there is $Q$ such that 
$P \wred Q$ and $Q \downarrow_{\mathcal N} x$.
\end{definition}

\begin{definition}
%\label{def.bbisim}
An  ${\mathcal N}$-\emph{barbed bisimulation} over a set of names, ${\mathcal N}$, is a symmetric binary relation 
${\mathcal S}_{\mathcal N}$ between agents such that $P\rel{S}_{\mathcal N}Q$ implies:
\begin{enumerate}
\item If $P \red P'$ then $Q \wred Q'$ and $P'\rel{S}_{\mathcal N} Q'$.
\item If $P\downarrow_{\mathcal N} x$, then $Q\Downarrow_{\mathcal N} x$.
\end{enumerate}
$P$ is ${\mathcal N}$-barbed bisimilar to $Q$, written
$P \wbbisim_{\mathcal N} Q$, if $P \rel{S}_{\mathcal N} Q$ for some ${\mathcal N}$-barbed bisimulation ${\mathcal S}_{\mathcal N}$.
\end{definition}

$\mathcal{R} \subseteq \pi \times \pi$

$P \mathcal{R} Q => \forall P'. P \red P' \Rightarrow \exists Q'. Q \red Q', P' \mathcal{R} Q'$

$P \vdash x \Rightarrow Q \vdash x$

\begin{mathpar}
  \inferrule*[lab=Out-barb]{x \nameeq y}{{y}!\langle{Q}\rangle \vdash x}
  \and
  \inferrule*[lab=Par-barb]{\mbox{$P\vdash x$ or $Q\vdash x$}}{\binpar{P}{Q} \vdash x}
\end{mathpar}

\subsubsection{Contexts}

One of the principle advantages of computational calculi like the
$\pi$-calculus is a well-defined notion of context,
contextual-equivalence and a correlation between
contextual-equivalence and notions of bisimulation. The notion of
context allows the decomposition of a process into (sub-)process and
its syntactic environment, its context. Thus, a context may be
thought of as a process with a ``hole'' (written $\Box$) in it. The
application of a context $M$ to a process $P$, written $M[P]$, is
tantamount to filling the hole in $M$ with $P$. In this paper we do
not need the full weight of this theory, but do make use of the notion
of context in the proof the main theorem. 

\begin{mathpar}
  \inferrule* [lab=summation] {} {{M_{M},M_{N}} \bc \Box \;|\; x.M_{A} \;|\; M_{M}+M_{N}}
  \and
  \inferrule* [lab=agent] {} {{M_{A}} \bc (\vec{x})M_{P} \;| \; \clift{P_0,\ldots,M_{P},\ldots,P_N}}
  \and \\
  \inferrule* [lab=process] {} {{M_{P}} \bc M_{N} \;| \;P|M_{P} }
\end{mathpar} 

\begin{mathpar}
  \inferrule* [lab=sychronization] {} {M_{N} \bc \Box \;|\; x?M_{F} \;|\; x!M_{C}}
  \and
  \inferrule* [lab=abstraction] {} {{M_{F}} \bc (x)M_{P} }
  \and
  \inferrule* [lab=concretion] {} {{M_{C}} \bc \langle M_{P} \rangle }
  \and \\
  \inferrule* [lab=process] {} {{M_{P}} \bc M_{N} \;| \;P|M_{P} }
\end{mathpar}

\begin{definition}[contextual application] Given a context $M$, and
  process $P$, we define the \emph{contextual application}, $M[P] :=
  M\{P/\Box\}$. That is, the contextual application of M to P is the
  substitution of $P$ for $\Box$ in $M$.
\end{definition}

$\meaningof{-} : L \to \mathcal{P}(\pi)$

\begin{mathpar}
  \inferrule* [lab=collection] {} {\meaningof{true} = \pi, \and \meaningof{~E} = \pi \setminus \meaningof{E}, \and \meaningof{E_{1} \& E_{2}} = \meaningof{E_{1}} \cap \meaningof{E_{2}}}
\end{mathpar}

\begin{mathpar}
  \inferrule* [lab=structure] {} {\meaningof{0} = \{ P \in \pi | P \equiv 0 \}, \and \\ \meaningof{E_1 | E_2} = \{ P \in \pi | P \equiv P_{1} | P_{2}, P_{1} \in \meaningof{E_{1}}, P_{2} \in \meaningof{E_2}\} }
\end{mathpar}

\begin{mathpar}
 \inferrule* [lab=behavior] {} {\meaningof{\langle a?b \rangle E} = \{ P \in \pi | P \equiv Q | u?(y)P', \\ \and \\\\ \and \\ \;\;\; u \in \meaningof{a}, \forall z.P'\{z/y\} \in \meaningof{E\{z/b\}}\}, \and \\ \meaningof{a!E} = \{ P \in \pi | P \equiv Q | x!\langle P' \rangle, x \in \meaningof{a} P' \in \meaningof{E}\} }
\end{mathpar}

\begin{mathpar}
 \inferrule* [lab=nominal] {} {\meaningof{\quotep{E}} = \{ \quotep{P} \in \quotep{\pi} | P \in \meaningof{E} \}, \and \meaningof{\quotep{P}} = \{ \quotep{Q} \in \quotep{\pi} | P \equiv Q \} \and \\ \meaningof{@\quotep{E}} = \{ P \in \pi | P \equiv @x, x \in \meaningof{E} \}}
\end{mathpar}

\begin{eqnarray*}
  \\
  \meaningof{-} : TS \to ST
\end{eqnarray*}

\begin{eqnarray*}
  \\
  L : TS \to ST
\end{eqnarray*}

\begin{eqnarray*}
  \\
  P \models E \iff P \in \meaningof{E}
\end{eqnarray*}

\begin{eqnarray*}
  P \approx_{L} Q \iff \forall E \in L. P \models E \iff Q \models E
\end{eqnarray*}

\begin{eqnarray*}
  P \approx_{K} Q
\end{eqnarray*}

\begin{eqnarray*}
  P \approx Q
\end{eqnarray*}

$\approx_{K} = \approx = \approx_{L}$

\subsubsection{Contextual duality}

Note that contexts extend the quotation operation to a family of
operations from processes to names. Given a context, $M$, we can
define a \emph{nominal context}, $\quotep{M}$ by $\quotep{M}[P] :=
\quotep{M[P]}$. To foreshadow what is to come we observe that these
operations enjoy a duality with processes very much like the duality
between vectors and maps from vectors to scalars.

Further, because the calculus is essentially higher-order, we have a
correspondence between contexts and processes. More specifically,
given a name $x$ and a context $M$ we can construct $M^{*}_{x}$ such
that 

\begin{mathpar}
  M^{*}_{x} | \lift{x}{P} \red M[P]
\end{mathpar}

namely,

\begin{mathpar}
  M^{*}_{x} := x?(u).M[\dropn{u}]
\end{mathpar}

The dependence of $M^{*}_{x}$ on a name makes it an abstraction, 

\begin{mathpar}
  M^{*} := (x)x?(u).M[\dropn{u}]
\end{mathpar}

\subsection{Additional notation}

It will sometimes be convenient to denote the process a name
quotes. We already have the notation $x = \quotep{P}$, but it will be
convenient to introduce an alternate notation, $\procn{x}$, when we
want to emphasize the connection to the use of the name. Note that, by
virtue of name equivalence, $\quotep{\procn{x}} \nameeq x$; so, the
notation is consistent with previous definitions.

Further, because names have structure it is possible to effect
substitutions on the basis of that structure. This means we need to
upgrade our notation for substitutions, which we accomplish by
adapting comprehension notation. Thus,

\begin{mathpar}
  P\{ y / x : x \in S \}
\end{mathpar}

is interpreted to mean the process derived from P by replacing (in a
capture-avoiding manner) each occurrence of $x$ in $S$ by $y$. For example,

\begin{mathpar}
  P\{ \quotep{\procn{x}|\procn{x}} / x : x \in \freenames{P} \}
\end{mathpar}

will replace each (occurrence) of a free name $x$ in $P$ by
$\quotep{\procn{x}|\procn{x}}$.

Also, we will avail ourselves of the notation $x^{L}$ and $x^{R}$ to
denote injections of a name into disjoint copies of the name
space. There are numerous ways to accomplish this. One example can be
found in \cite{MeredithR05}. This notation overloads to vectors of
names: $\vec{x}^{\pi} := (x_{i}^{\pi} \; : \; 0 \leq i < |\vec{x}| )$ where $\pi \in \{L,R\}$.

We also use $P^{\Box} := P|\Box$.

In \cite{MeredithR05} an interpretation of the new operator is
given. It turns out that there are several possible interpretations
all enjoying the requisite algebraic properties of the operator (see
\cite{milner91polyadicpi}). We will therefore make liberal use of
$(\nu\; \vec{x})P$.

% subsection the_syntax_and_semantics_of_the_notation_system (end)   

\input{qm2pi.qmops} 

\input{qm2pi.sterngerlach} 

\input{qm2pi.metric} 

% section concurrent_process_calculi (end)

%\input{qm2pi.proofsketch}

% section proof sketch (end)

%\input{qm2pi.slviaknots} 

% section spatial logic via knots (end)

\input{qm2pi.conclusion}

% section conclusion (end)

%\input{qm2pi.dtcodes} 

% section wiring algorithm (end)

\input{qm2pi.ack} 

% section acknowledgments (end)

\newpage


\bibliographystyle{plain}   
\bibliography{../../biblios/main.bib}

\input{qm2pi.rhodetails}

\end{document}

 

%\documentclass[12pt]{llncs}
%\documentclass{jktr}

\usepackage[pdftex]{hyperref}                   
\usepackage {listings}
\usepackage {mathpartir}
\usepackage{bcprules}
%\usepackage{listings}
                       
\usepackage{graphicx} 
%\usepackage[margins=2.5cm,nohead,nofoot]{geometry}
%\usepackage{geometry}
\usepackage{amsfonts}
\usepackage{amstext}
\usepackage{latexsym}
\usepackage{amssymb}
\usepackage{color}


%\include{myPreamble}
\include{qm2pi.local} 

%\ifpdf
%\usepackage[pdftex]{graphicx}
%\else
%\usepackage{graphicx}
%\fi

 % \ifpdf
%  \usepackage{pdfsync}
%  \if


%\title{Brief Article}
%\author{David F. Snyder}
%\author{L.G. Meredith}

%\address{Dept. of Math., Texas State University--San Marcos, San Marcos, TX 78666}
       
\pagestyle{empty}


\begin{document}

\lstset{language=[Objective]Caml,frame=shadowbox}

\input{qm2pi.front}

% section front matter (end)

\input{qm2pi.intro} 
 
% section introduction (end)

% \input{qm2pi.knotations} 

% section notation (end)

\input{qm2pi.process.calculi} 

% section concurrent_process_calculi_and_spatial_logics_ (end)
    
%\input{qm2pi.knots2pi} 

%\input{qm2pi.trefoil} 

%\input{qm2pi.mainthm} 

% subsection basic_interpretation (end)

%\input{qm2pi.rho.presentation} 
\subsection{The syntax and semantics of the notation system}\label{sub:the_syntax_and_semantics_of_the_notation_system} % (fold)

We now summarize a technical presentation of the calculus that
embodies our theory of dynamics. The typical presentation of such a
calculus follows the style of giving generators and relations on
them. The grammar, below, describing term constructors, freely
generates the set of processes, $\Proc$. This set is then quotiented
by a relation known as structural congruence and it is over this set
that the notion of dynamics is expressed. This presentation is
essentially that of \cite{MeredithR05} with the addition of
polyadicity and summation. For readability we have relegated some of
the technical subtleties to an appendix.

\subsubsection{Process grammar}\label{subsub:process_grammar}

\begin{mathpar}
  \inferrule* [lab=synchronization] {} {{M} \bc \pzero \;|\; x?F \;|\; x!C }
  \and
  \inferrule* [lab=abstraction] {} {{F} \bc (x)P}
  \and
  \inferrule* [lab=concretion] {} {{C} \bc \langle Q \rangle}
  \and
  \inferrule* [lab=process] {} {{P,Q} \bc M \;| \;P|Q \;|\; @{x}}
  \and
  \inferrule* [lab=name] {} {{x} \bc \quotep{P}}
\end{mathpar} 

Note that $\vec{x}$ (resp. $\vec{P}$) denotes a vector of names
(resp. processes) of length $|\vec{x}|$ (resp. $|\vec{P}|$). We adopt
the following useful abbreviations.

\begin{mathpar}
   x?(\vec{y}).P := x.(\vec{y})P \and  x\clift{\vec{P}} := x.\clift{\vec{P}}
   \and x!(y) := \lift{x}{\dropn{y}}
   \and \Pi_{i=0}^{n-1}P_i := P_0 | \ldots | P_{n-1}
\end{mathpar}

\subsubsection{Structural congruence}

\paragraph{Free and bound names and alpha-equivalence.} At the
core of structural equivalence is alpha-equivalence which identifies
process that are the same up to a change of variable. Formally, we
recognize the distinction between free and bound names. The free names
of a process, $\freenames{P}$, may be calculated recursively as
follows:

\begin{mathpar}
\freenames{\pzero} := \emptyset
  \and \\
  \freenames{x?(y).P} := \{ x \} \cup (\freenames{P} \setminus \{ y \})
  \and 
  \freenames{x!\langle P \rangle} := \{ x \} \cup \{ P \} 
  \and \\
  \freenames{P|Q} := \freenames{P} \cup \freenames{Q}
  \and \\
  \freenames{@{x}} := \{ x \}
\end{mathpar}

$\pi$
$\quotep{\pi}$

$\freenames{-} : \pi \to \mathcal{P}(\quotep{\pi})$

\begin{eqnarray*}
  \freenames{\pzero} & := & \emptyset \\
  \freenames{x?(y).P} & := & \{ x \} \cup (\freenames{P} \setminus \{ y \}) \\
  \freenames{x!\langle P \rangle} & := & \{ x \} \cup \{ P \} \\
  \freenames{P|Q} & := & \freenames{P} \cup \freenames{Q} \\
  \freenames{\dropn{x}} & := & \{ x \}
\end{eqnarray*}

The bound names of a process, $\boundnames{P}$, are those names occurring in $P$
that are not free. For example, in $x?(y).0$, the name $x$ is free, while $y$ is bound.

\begin{mathpar}
  \inferrule* [lab=monoidal-laws] {} { P|Q \equiv Q|P \and P|0 \equiv P \and P|(Q|R) \equiv (P|Q)|R }
\end{mathpar}

\begin{mathpar}
  \inferrule* [lab=alpha-equivalence] {} { (x)P \equiv (y)P\{y/x\} \and y \not\in \freenames{P} }
\end{mathpar}

\begin{definition}
Then two processes, $P,Q$, are alpha-equivalent if $P = Q\{\vec{y}/\vec{x}\}$ for
some $\vec{x} \in \boundnames{Q},\vec{y} \in \boundnames{P}$, where $Q\{\vec{y}/\vec{x}\}$
denotes the capture-avoiding substitution of $\vec{y}$ for $\vec{x}$ in $Q$.
\end{definition}

\begin{definition}
  The {\em structural congruence} \cite{SangiorgiWalker} , $\equiv$,
  between processes is the least congruence containing
  alpha-equivalence, satisfying the abelian monoid laws
  (associativity, commutativity and $\pzero$ as identity) for parallel
  composition $|$ and for summation $+$.
\end{definition}

\subsection{Name equivalence}

We take name equivalence, written $\nameeq$, to be the smallest
equivalence relation generated by the following rules.

\begin{mathpar}
\inferrule*[lab=Quote-drop]
{ }
{ \quotep{@{x}} \nameeq x }

\inferrule*[lab=Struct-equiv]
{ P \scong Q }
{ \quotep{P} \nameeq \quotep{Q} }
\end{mathpar}

The astute reader will have noticed that the mutual recursion of names
and processes imposes a mutual recursion on alpha-equivalence and
structural equivalence via name-equivalence. Fortunately, all of this
works out pleasantly and we may calculate in the natural way, free of
concern. The reader interested in the details is referred to the
appendix \ref{appendix:rho_details}.

\subsection{Substitution}

We use $\Proc$ for the set of processes, $\QProc$ for the set of
names, and $\id{\{}\vec{y} / \vec{x} \id{\}}$ to denote partial maps,
$s : \QProc \rightarrow \QProc$. A map, $s$ lifts, uniquely, to a map
on process terms, $\widehat{s} : \Proc \rightarrow \Proc$ by the
following equations.

\begin{mathpar}
  (0) \psubstp{Q}{P} := 0 \\
  (R \juxtap S) \psubstp{Q}{P}
  :=    
  (R)\psubstp{Q}{P} \juxtap (S) \psubstp{Q}{P} \\
  (x?(y).R) \psubstp{Q}{P}    
  :=    
  (x)\substp{Q}{P} (z)\concat( (R \psubstn{z}{y}) \psubstp{Q}{P} ) \\
  (\lift{x}{R}) \psubstp{Q}{P}  
  :=
  \lift{(x)\substp{Q}{P}}{ R \psubstp{Q}{P} } \\
%   (\dropn{x})  \psubstp{Q}{P}       
%   := 
%   \left\{ 
%     \begin{array}{ccc} 
%       \dropn{\quotep{Q}} & & x \nameeq \quotep{P} \\
%       \dropn{x} & & otherwise \\
%     \end{array}
%   \right. 
  (\dropn{x})  \psubstp{Q}{P}       
  := 
  \left\{ 
    \begin{array}{ccc} 
      Q & & x \nameeq \quotep{P} \\
      \dropn{x} & & otherwise \\
    \end{array}
  \right.
\end{mathpar}
 

where

\begin{eqnarray}
  (x)\id{\{} \lpquote Q \rpquote / \lpquote P \rpquote \id{\}}            = 
  \left\{ 
    \begin{array}{ccc}
      \lpquote Q \rpquote & & x \nameeq \lpquote P \rpquote \\
      x & & otherwise \\
    \end{array}
  \right. \nonumber
\end{eqnarray}

and $z$ is chosen distinct from $\quotep{P}$, $\quotep{Q}$, the free
names in $Q$, and all the names in $R$. Our $\alpha$-equivalence will
be built in the standard way from this substitution.

\begin{remark}\label{rem:no_self_referential_names}
  One consequence of these definitions is that $\forall P. \quotep{P}
  \not\in \freenames{P}$.
\end{remark}

\subsection{ Dynamic quote: an example }

Anticipating something of what's to come, consider applying the
substitution, $\widehat{\id{\{}u / z \id{\}}}$, to the following pair
of processes, $\lift{w}{y!(z)}$ and $w[ \lpquote y!(z) \rpquote ]$.

\begin{eqnarray}
	\lift{w}{y!(z)}\widehat{\id{\{}u / z \id{\}}}
		& = &
		\lift{w}{y!(u)} \nonumber\\
	w[ \lpquote y!(z) \rpquote ] \widehat{ \id{\{}u / z \id{\}} }
		& = &
		w[ \lpquote y!(z) \rpquote ] \nonumber
\end{eqnarray}

Because the body of the process between quotes is impervious to
substitution, we get radically different answers. In fact, by
examining the first process in an input context,
e.g. $x?(z).\lift{w}{y!(z)}$, we see that the process under the lift
operator may be shaped by prefixed inputs binding a name inside it. In
this sense, the lift operator will be seen as a way to dynamically
construct processes before reifying them as names.

Finally equipped with these standard features we can present the
dynamics of the calculus.

\subsubsection{Operational semantics} 

Finally, we introduce the computational dynamics. What marks these
algebras as distinct from other more traditionally studied algebraic
structures, e.g. vector spaces or polynomial rings, is the manner in
which dynamics is captured. In traditional structures, dynamics is typically
expressed through morphisms between such structures, as in linear maps
between vector spaces or morphisms between rings. In algebras
associated with the semantics of computation, the dynamics is
expressed as part of the algebraic structure itself, through a
reduction reduction relation typically denoted by $\red$. Below, we
give a recursive presentation of this relation for the calculus used
in the encoding.

$\red \subseteq \pi \times \pi$
$\red : \pi \to \mathcal{P}(\pi)$

\begin{mathpar}
  \inferrule* [lab=Comm] { \textsf{match}( x_{src}, x_{trgt} ) } { x_{trgt}?(y)P \; | \; x_{src}!\langle {Q} \rangle \red P\{\quotep{Q}/y}\} }
  \and \\
  \inferrule* [lab=Par] {{P} \red {P}'} {{{P} | {Q}} \red {{P}' | {Q}}}
  \and
  \inferrule* [lab=Equiv]{{{P} \scong {P}'} \andalso {{P}' \red {Q}'} \andalso {{Q}' \scong {Q}}}{{P} \red {Q}}
\end{mathpar}

\begin{eqnarray*}
  match_{\equiv} (\quotep{P},\quotep{Q}) & := & P \equiv Q \\
  match_{\dagger}(\quotep{P},\quotep{Q}) & := & \forall R. P|Q \red^{*} R => R \red^{*} 0 \\
  match_{K}(\quotep{P},\quotep{Q}) & := & K \mbox{ for some context } K
\end{eqnarray*}

$u?(x)P | u!\langle Q \rangle \red P\{\quotep{Q}/x\}$

%We write $\wred$ for $\red^*$, and $P\red$ if $\exists Q $ such that $ P \red Q$.
We write $P\red$ if $\exists Q $ such that $ P \red Q$ and $P\not\red$, otherwise.

\section{Replication}

As mentioned before, it is known that replication (and hence
recursion) can be implemented in a higher-order process algebra
\cite{SangiorgiWalker}. As our first example of calculation with the
machinery thus far presented we give the construction explicitly in
the {\rhoc}.

\begin{eqnarray}
	D_{x} & := & \prefix{x}{y}{(\binpar{\outputp{x}{y}}{@{y}})} \nonumber\\
	\bangp_{x}{P} & := & \binpar{{x}!\langle{\binpar{D_{x}}{P}}\rangle}{D_{x}} \nonumber
\end{eqnarray}

\begin{eqnarray}
	\bangp_{x}{P} & & \nonumber\\
	=
	& {x}!\langle{(\prefix{x}{y}{(\outputp{x}{y} | @{y})) | P}}\rangle 
	      | \prefix{x}{y}{(\outputp{x}{y} | @{y})} & \nonumber\\
	\red
	& (\outputp{x}{y} | @{y})\substn{\quotep{(\prefix{x}{y}{(@{y} | \outputp{x}{y})) | P}}}{y} & \nonumber\\
	=
	& \outputp{x}{\quotep{(\prefix{x}{y}{(\outputp{x}{y} | @{y})) | P}}}
	  | {(\prefix{x}{y}{(\outputp{x}{y} | @{y})) | P}} & \nonumber\\
	\red
	& \ldots & \nonumber\\
	\red^*
	& P | P | \ldots & \nonumber
\end{eqnarray}

Of course, this encoding, as an implementation, runs away, unfolding
$\bangp{P}$ eagerly. A lazier and more implementable replication
operator, restricted to input-guarded processes, may be obtained as follows.

\begin{eqnarray}
\bangp{\prefix{u}{v}{P}} 
	:= 
	\binpar{\lift{x}{\prefix{u}{v}{(\binpar{D(x)}{P})}}}{D(x)} \nonumber
\end{eqnarray}

\begin{remark}
  Note that the lazier definition still does not deal with summation
  or mixed summation (i.e. sums over input and output). The reader is
  invited to construct definitions of replication that deal with these
  features. 

  Further, the definitions are parameterized in a name, $x$. Can you,
  gentle reader, make a definition that eliminates this parameter and
  guarantees no accidental interaction between the replication
  machinery and the process being replicated -- i.e. no accidental
  sharing of names used by the process to get its work done and the
  name(s) used by the replication to effect copying. This latter
  revision of the definition of replication is crucial to obtaining
  the expected identity $!!P \sim !P$.
\end{remark}

\begin{remark}\label{rem:paradoxical_combinator}
  The reader familiar with the lambda calculus will have noticed the
  similarity between $D$ and the paradoxical combinator.

  [Ed. note: the existence of this seems to suggest we have to be more
  restrictive on the set of processes and names we admit if we are to
  support no-cloning.]
\end{remark}

\subsubsection{Bisimulation}

The computational dynamics gives rise to another kind of equivalence,
the equivalence of computational behavior. As previously mentioned
this is typically captured \emph{via} some form of bisimulation.

% The notion we use in this paper is weak barbed bisimulation
% \cite{milner91polyadicpi}.

The notion we use in this paper is derived from weak barbed
bisimulation \cite{milner91polyadicpi}. 

\begin{definition}
An \emph{observation relation}, $\downarrow_{\mathcal N}$, over a set
of names, $\mathcal N$, is the smallest relation satisfying the rules
below.

\infrule[Out-barb]{y \in {\mathcal N}, \; x \nameeq y}
		  {\outputp{x}{v} \downarrow_{\mathcal N} x}
\infrule[Par-barb]{\mbox{$P\downarrow_{\mathcal N} x$ or $Q\downarrow_{\mathcal N} x$}}
		  {\binpar{P}{Q} \downarrow_{\mathcal N} x}

We write $P \Downarrow_{\mathcal N} x$ if there is $Q$ such that 
$P \wred Q$ and $Q \downarrow_{\mathcal N} x$.
\end{definition}

\begin{definition}
%\label{def.bbisim}
An  ${\mathcal N}$-\emph{barbed bisimulation} over a set of names, ${\mathcal N}$, is a symmetric binary relation 
${\mathcal S}_{\mathcal N}$ between agents such that $P\rel{S}_{\mathcal N}Q$ implies:
\begin{enumerate}
\item If $P \red P'$ then $Q \wred Q'$ and $P'\rel{S}_{\mathcal N} Q'$.
\item If $P\downarrow_{\mathcal N} x$, then $Q\Downarrow_{\mathcal N} x$.
\end{enumerate}
$P$ is ${\mathcal N}$-barbed bisimilar to $Q$, written
$P \wbbisim_{\mathcal N} Q$, if $P \rel{S}_{\mathcal N} Q$ for some ${\mathcal N}$-barbed bisimulation ${\mathcal S}_{\mathcal N}$.
\end{definition}

$\mathcal{R} \subseteq \pi \times \pi$

$P \mathcal{R} Q => \forall P'. P \red P' \Rightarrow \exists Q'. Q \red Q', P' \mathcal{R} Q'$

$P \vdash x \Rightarrow Q \vdash x$

\begin{mathpar}
  \inferrule*[lab=Out-barb]{x \nameeq y}{{y}!\langle{Q}\rangle \vdash x}
  \and
  \inferrule*[lab=Par-barb]{\mbox{$P\vdash x$ or $Q\vdash x$}}{\binpar{P}{Q} \vdash x}
\end{mathpar}

\subsubsection{Contexts}

One of the principle advantages of computational calculi like the
$\pi$-calculus is a well-defined notion of context,
contextual-equivalence and a correlation between
contextual-equivalence and notions of bisimulation. The notion of
context allows the decomposition of a process into (sub-)process and
its syntactic environment, its context. Thus, a context may be
thought of as a process with a ``hole'' (written $\Box$) in it. The
application of a context $M$ to a process $P$, written $M[P]$, is
tantamount to filling the hole in $M$ with $P$. In this paper we do
not need the full weight of this theory, but do make use of the notion
of context in the proof the main theorem. 

\begin{mathpar}
  \inferrule* [lab=summation] {} {{M_{M},M_{N}} \bc \Box \;|\; x.M_{A} \;|\; M_{M}+M_{N}}
  \and
  \inferrule* [lab=agent] {} {{M_{A}} \bc (\vec{x})M_{P} \;| \; \clift{P_0,\ldots,M_{P},\ldots,P_N}}
  \and \\
  \inferrule* [lab=process] {} {{M_{P}} \bc M_{N} \;| \;P|M_{P} }
\end{mathpar} 

\begin{mathpar}
  \inferrule* [lab=sychronization] {} {M_{N} \bc \Box \;|\; x?M_{F} \;|\; x!M_{C}}
  \and
  \inferrule* [lab=abstraction] {} {{M_{F}} \bc (x)M_{P} }
  \and
  \inferrule* [lab=concretion] {} {{M_{C}} \bc \langle M_{P} \rangle }
  \and \\
  \inferrule* [lab=process] {} {{M_{P}} \bc M_{N} \;| \;P|M_{P} }
\end{mathpar}

\begin{definition}[contextual application] Given a context $M$, and
  process $P$, we define the \emph{contextual application}, $M[P] :=
  M\{P/\Box\}$. That is, the contextual application of M to P is the
  substitution of $P$ for $\Box$ in $M$.
\end{definition}

$\meaningof{-} : L \to \mathcal{P}(\pi)$

\begin{mathpar}
  \inferrule* [lab=collection] {} {\meaningof{true} = \pi, \and \meaningof{~E} = \pi \setminus \meaningof{E}, \and \meaningof{E_{1} \& E_{2}} = \meaningof{E_{1}} \cap \meaningof{E_{2}}}
\end{mathpar}

\begin{mathpar}
  \inferrule* [lab=structure] {} {\meaningof{0} = \{ P \in \pi | P \equiv 0 \}, \and \\ \meaningof{E_1 | E_2} = \{ P \in \pi | P \equiv P_{1} | P_{2}, P_{1} \in \meaningof{E_{1}}, P_{2} \in \meaningof{E_2}\} }
\end{mathpar}

\begin{mathpar}
 \inferrule* [lab=behavior] {} {\meaningof{\langle a?b \rangle E} = \{ P \in \pi | P \equiv Q | u?(y)P', \\ \and \\\\ \and \\ \;\;\; u \in \meaningof{a}, \forall z.P'\{z/y\} \in \meaningof{E\{z/b\}}\}, \and \\ \meaningof{a!E} = \{ P \in \pi | P \equiv Q | x!\langle P' \rangle, x \in \meaningof{a} P' \in \meaningof{E}\} }
\end{mathpar}

\begin{mathpar}
 \inferrule* [lab=nominal] {} {\meaningof{\quotep{E}} = \{ \quotep{P} \in \quotep{\pi} | P \in \meaningof{E} \}, \and \meaningof{\quotep{P}} = \{ \quotep{Q} \in \quotep{\pi} | P \equiv Q \} \and \\ \meaningof{@\quotep{E}} = \{ P \in \pi | P \equiv @x, x \in \meaningof{E} \}}
\end{mathpar}

\begin{eqnarray*}
  \\
  \meaningof{-} : TS \to ST
\end{eqnarray*}

\begin{eqnarray*}
  \\
  L : TS \to ST
\end{eqnarray*}

\begin{eqnarray*}
  \\
  P \models E \iff P \in \meaningof{E}
\end{eqnarray*}

\begin{eqnarray*}
  P \approx_{L} Q \iff \forall E \in L. P \models E \iff Q \models E
\end{eqnarray*}

\begin{eqnarray*}
  P \approx_{K} Q
\end{eqnarray*}

\begin{eqnarray*}
  P \approx Q
\end{eqnarray*}

$\approx_{K} = \approx = \approx_{L}$

\subsubsection{Contextual duality}

Note that contexts extend the quotation operation to a family of
operations from processes to names. Given a context, $M$, we can
define a \emph{nominal context}, $\quotep{M}$ by $\quotep{M}[P] :=
\quotep{M[P]}$. To foreshadow what is to come we observe that these
operations enjoy a duality with processes very much like the duality
between vectors and maps from vectors to scalars.

Further, because the calculus is essentially higher-order, we have a
correspondence between contexts and processes. More specifically,
given a name $x$ and a context $M$ we can construct $M^{*}_{x}$ such
that 

\begin{mathpar}
  M^{*}_{x} | \lift{x}{P} \red M[P]
\end{mathpar}

namely,

\begin{mathpar}
  M^{*}_{x} := x?(u).M[\dropn{u}]
\end{mathpar}

The dependence of $M^{*}_{x}$ on a name makes it an abstraction, 

\begin{mathpar}
  M^{*} := (x)x?(u).M[\dropn{u}]
\end{mathpar}

\subsection{Additional notation}

It will sometimes be convenient to denote the process a name
quotes. We already have the notation $x = \quotep{P}$, but it will be
convenient to introduce an alternate notation, $\procn{x}$, when we
want to emphasize the connection to the use of the name. Note that, by
virtue of name equivalence, $\quotep{\procn{x}} \nameeq x$; so, the
notation is consistent with previous definitions.

Further, because names have structure it is possible to effect
substitutions on the basis of that structure. This means we need to
upgrade our notation for substitutions, which we accomplish by
adapting comprehension notation. Thus,

\begin{mathpar}
  P\{ y / x : x \in S \}
\end{mathpar}

is interpreted to mean the process derived from P by replacing (in a
capture-avoiding manner) each occurrence of $x$ in $S$ by $y$. For example,

\begin{mathpar}
  P\{ \quotep{\procn{x}|\procn{x}} / x : x \in \freenames{P} \}
\end{mathpar}

will replace each (occurrence) of a free name $x$ in $P$ by
$\quotep{\procn{x}|\procn{x}}$.

Also, we will avail ourselves of the notation $x^{L}$ and $x^{R}$ to
denote injections of a name into disjoint copies of the name
space. There are numerous ways to accomplish this. One example can be
found in \cite{MeredithR05}. This notation overloads to vectors of
names: $\vec{x}^{\pi} := (x_{i}^{\pi} \; : \; 0 \leq i < |\vec{x}| )$ where $\pi \in \{L,R\}$.

We also use $P^{\Box} := P|\Box$.

In \cite{MeredithR05} an interpretation of the new operator is
given. It turns out that there are several possible interpretations
all enjoying the requisite algebraic properties of the operator (see
\cite{milner91polyadicpi}). We will therefore make liberal use of
$(\nu\; \vec{x})P$.

% subsection the_syntax_and_semantics_of_the_notation_system (end)   

\input{qm2pi.qmops} 

\input{qm2pi.sterngerlach} 

\input{qm2pi.metric} 

% section concurrent_process_calculi (end)

%\input{qm2pi.proofsketch}

% section proof sketch (end)

%\input{qm2pi.slviaknots} 

% section spatial logic via knots (end)

\input{qm2pi.conclusion}

% section conclusion (end)

%\input{qm2pi.dtcodes} 

% section wiring algorithm (end)

\input{qm2pi.ack} 

% section acknowledgments (end)

\newpage


\bibliographystyle{plain}   
\bibliography{../../biblios/main.bib}

\input{qm2pi.rhodetails}

\end{document}

 

% subsection basic_interpretation (end)

%\input{qm2pi.rho.presentation} 
\subsection{The syntax and semantics of the notation system}\label{sub:the_syntax_and_semantics_of_the_notation_system} % (fold)

We now summarize a technical presentation of the calculus that
embodies our theory of dynamics. The typical presentation of such a
calculus follows the style of giving generators and relations on
them. The grammar, below, describing term constructors, freely
generates the set of processes, $\Proc$. This set is then quotiented
by a relation known as structural congruence and it is over this set
that the notion of dynamics is expressed. This presentation is
essentially that of \cite{MeredithR05} with the addition of
polyadicity and summation. For readability we have relegated some of
the technical subtleties to an appendix.

\subsubsection{Process grammar}\label{subsub:process_grammar}

\begin{mathpar}
  \inferrule* [lab=synchronization] {} {{M} \bc \pzero \;|\; x?F \;|\; x!C }
  \and
  \inferrule* [lab=abstraction] {} {{F} \bc (x)P}
  \and
  \inferrule* [lab=concretion] {} {{C} \bc \langle Q \rangle}
  \and
  \inferrule* [lab=process] {} {{P,Q} \bc M \;| \;P|Q \;|\; @{x}}
  \and
  \inferrule* [lab=name] {} {{x} \bc \quotep{P}}
\end{mathpar} 

Note that $\vec{x}$ (resp. $\vec{P}$) denotes a vector of names
(resp. processes) of length $|\vec{x}|$ (resp. $|\vec{P}|$). We adopt
the following useful abbreviations.

\begin{mathpar}
   x?(\vec{y}).P := x.(\vec{y})P \and  x\clift{\vec{P}} := x.\clift{\vec{P}}
   \and x!(y) := \lift{x}{\dropn{y}}
   \and \Pi_{i=0}^{n-1}P_i := P_0 | \ldots | P_{n-1}
\end{mathpar}

\subsubsection{Structural congruence}

\paragraph{Free and bound names and alpha-equivalence.} At the
core of structural equivalence is alpha-equivalence which identifies
process that are the same up to a change of variable. Formally, we
recognize the distinction between free and bound names. The free names
of a process, $\freenames{P}$, may be calculated recursively as
follows:

\begin{mathpar}
\freenames{\pzero} := \emptyset
  \and \\
  \freenames{x?(y).P} := \{ x \} \cup (\freenames{P} \setminus \{ y \})
  \and 
  \freenames{x!\langle P \rangle} := \{ x \} \cup \{ P \} 
  \and \\
  \freenames{P|Q} := \freenames{P} \cup \freenames{Q}
  \and \\
  \freenames{@{x}} := \{ x \}
\end{mathpar}

$\pi$
$\quotep{\pi}$

$\freenames{-} : \pi \to \mathcal{P}(\quotep{\pi})$

\begin{eqnarray*}
  \freenames{\pzero} & := & \emptyset \\
  \freenames{x?(y).P} & := & \{ x \} \cup (\freenames{P} \setminus \{ y \}) \\
  \freenames{x!\langle P \rangle} & := & \{ x \} \cup \{ P \} \\
  \freenames{P|Q} & := & \freenames{P} \cup \freenames{Q} \\
  \freenames{\dropn{x}} & := & \{ x \}
\end{eqnarray*}

The bound names of a process, $\boundnames{P}$, are those names occurring in $P$
that are not free. For example, in $x?(y).0$, the name $x$ is free, while $y$ is bound.

\begin{mathpar}
  \inferrule* [lab=monoidal-laws] {} { P|Q \equiv Q|P \and P|0 \equiv P \and P|(Q|R) \equiv (P|Q)|R }
\end{mathpar}

\begin{mathpar}
  \inferrule* [lab=alpha-equivalence] {} { (x)P \equiv (y)P\{y/x\} \and y \not\in \freenames{P} }
\end{mathpar}

\begin{definition}
Then two processes, $P,Q$, are alpha-equivalent if $P = Q\{\vec{y}/\vec{x}\}$ for
some $\vec{x} \in \boundnames{Q},\vec{y} \in \boundnames{P}$, where $Q\{\vec{y}/\vec{x}\}$
denotes the capture-avoiding substitution of $\vec{y}$ for $\vec{x}$ in $Q$.
\end{definition}

\begin{definition}
  The {\em structural congruence} \cite{SangiorgiWalker} , $\equiv$,
  between processes is the least congruence containing
  alpha-equivalence, satisfying the abelian monoid laws
  (associativity, commutativity and $\pzero$ as identity) for parallel
  composition $|$ and for summation $+$.
\end{definition}

\subsection{Name equivalence}

We take name equivalence, written $\nameeq$, to be the smallest
equivalence relation generated by the following rules.

\begin{mathpar}
\inferrule*[lab=Quote-drop]
{ }
{ \quotep{@{x}} \nameeq x }

\inferrule*[lab=Struct-equiv]
{ P \scong Q }
{ \quotep{P} \nameeq \quotep{Q} }
\end{mathpar}

The astute reader will have noticed that the mutual recursion of names
and processes imposes a mutual recursion on alpha-equivalence and
structural equivalence via name-equivalence. Fortunately, all of this
works out pleasantly and we may calculate in the natural way, free of
concern. The reader interested in the details is referred to the
appendix \ref{appendix:rho_details}.

\subsection{Substitution}

We use $\Proc$ for the set of processes, $\QProc$ for the set of
names, and $\id{\{}\vec{y} / \vec{x} \id{\}}$ to denote partial maps,
$s : \QProc \rightarrow \QProc$. A map, $s$ lifts, uniquely, to a map
on process terms, $\widehat{s} : \Proc \rightarrow \Proc$ by the
following equations.

\begin{mathpar}
  (0) \psubstp{Q}{P} := 0 \\
  (R \juxtap S) \psubstp{Q}{P}
  :=    
  (R)\psubstp{Q}{P} \juxtap (S) \psubstp{Q}{P} \\
  (x?(y).R) \psubstp{Q}{P}    
  :=    
  (x)\substp{Q}{P} (z)\concat( (R \psubstn{z}{y}) \psubstp{Q}{P} ) \\
  (\lift{x}{R}) \psubstp{Q}{P}  
  :=
  \lift{(x)\substp{Q}{P}}{ R \psubstp{Q}{P} } \\
%   (\dropn{x})  \psubstp{Q}{P}       
%   := 
%   \left\{ 
%     \begin{array}{ccc} 
%       \dropn{\quotep{Q}} & & x \nameeq \quotep{P} \\
%       \dropn{x} & & otherwise \\
%     \end{array}
%   \right. 
  (\dropn{x})  \psubstp{Q}{P}       
  := 
  \left\{ 
    \begin{array}{ccc} 
      Q & & x \nameeq \quotep{P} \\
      \dropn{x} & & otherwise \\
    \end{array}
  \right.
\end{mathpar}
 

where

\begin{eqnarray}
  (x)\id{\{} \lpquote Q \rpquote / \lpquote P \rpquote \id{\}}            = 
  \left\{ 
    \begin{array}{ccc}
      \lpquote Q \rpquote & & x \nameeq \lpquote P \rpquote \\
      x & & otherwise \\
    \end{array}
  \right. \nonumber
\end{eqnarray}

and $z$ is chosen distinct from $\quotep{P}$, $\quotep{Q}$, the free
names in $Q$, and all the names in $R$. Our $\alpha$-equivalence will
be built in the standard way from this substitution.

\begin{remark}\label{rem:no_self_referential_names}
  One consequence of these definitions is that $\forall P. \quotep{P}
  \not\in \freenames{P}$.
\end{remark}

\subsection{ Dynamic quote: an example }

Anticipating something of what's to come, consider applying the
substitution, $\widehat{\id{\{}u / z \id{\}}}$, to the following pair
of processes, $\lift{w}{y!(z)}$ and $w[ \lpquote y!(z) \rpquote ]$.

\begin{eqnarray}
	\lift{w}{y!(z)}\widehat{\id{\{}u / z \id{\}}}
		& = &
		\lift{w}{y!(u)} \nonumber\\
	w[ \lpquote y!(z) \rpquote ] \widehat{ \id{\{}u / z \id{\}} }
		& = &
		w[ \lpquote y!(z) \rpquote ] \nonumber
\end{eqnarray}

Because the body of the process between quotes is impervious to
substitution, we get radically different answers. In fact, by
examining the first process in an input context,
e.g. $x?(z).\lift{w}{y!(z)}$, we see that the process under the lift
operator may be shaped by prefixed inputs binding a name inside it. In
this sense, the lift operator will be seen as a way to dynamically
construct processes before reifying them as names.

Finally equipped with these standard features we can present the
dynamics of the calculus.

\subsubsection{Operational semantics} 

Finally, we introduce the computational dynamics. What marks these
algebras as distinct from other more traditionally studied algebraic
structures, e.g. vector spaces or polynomial rings, is the manner in
which dynamics is captured. In traditional structures, dynamics is typically
expressed through morphisms between such structures, as in linear maps
between vector spaces or morphisms between rings. In algebras
associated with the semantics of computation, the dynamics is
expressed as part of the algebraic structure itself, through a
reduction reduction relation typically denoted by $\red$. Below, we
give a recursive presentation of this relation for the calculus used
in the encoding.

$\red \subseteq \pi \times \pi$
$\red : \pi \to \mathcal{P}(\pi)$

\begin{mathpar}
  \inferrule* [lab=Comm] { \textsf{match}( x_{src}, x_{trgt} ) } { x_{trgt}?(y)P \; | \; x_{src}!\langle {Q} \rangle \red P\{\quotep{Q}/y}\} }
  \and \\
  \inferrule* [lab=Par] {{P} \red {P}'} {{{P} | {Q}} \red {{P}' | {Q}}}
  \and
  \inferrule* [lab=Equiv]{{{P} \scong {P}'} \andalso {{P}' \red {Q}'} \andalso {{Q}' \scong {Q}}}{{P} \red {Q}}
\end{mathpar}

\begin{eqnarray*}
  match_{\equiv} (\quotep{P},\quotep{Q}) & := & P \equiv Q \\
  match_{\dagger}(\quotep{P},\quotep{Q}) & := & \forall R. P|Q \red^{*} R => R \red^{*} 0 \\
  match_{K}(\quotep{P},\quotep{Q}) & := & K \mbox{ for some context } K
\end{eqnarray*}

$u?(x)P | u!\langle Q \rangle \red P\{\quotep{Q}/x\}$

%We write $\wred$ for $\red^*$, and $P\red$ if $\exists Q $ such that $ P \red Q$.
We write $P\red$ if $\exists Q $ such that $ P \red Q$ and $P\not\red$, otherwise.

\section{Replication}

As mentioned before, it is known that replication (and hence
recursion) can be implemented in a higher-order process algebra
\cite{SangiorgiWalker}. As our first example of calculation with the
machinery thus far presented we give the construction explicitly in
the {\rhoc}.

\begin{eqnarray}
	D_{x} & := & \prefix{x}{y}{(\binpar{\outputp{x}{y}}{@{y}})} \nonumber\\
	\bangp_{x}{P} & := & \binpar{{x}!\langle{\binpar{D_{x}}{P}}\rangle}{D_{x}} \nonumber
\end{eqnarray}

\begin{eqnarray}
	\bangp_{x}{P} & & \nonumber\\
	=
	& {x}!\langle{(\prefix{x}{y}{(\outputp{x}{y} | @{y})) | P}}\rangle 
	      | \prefix{x}{y}{(\outputp{x}{y} | @{y})} & \nonumber\\
	\red
	& (\outputp{x}{y} | @{y})\substn{\quotep{(\prefix{x}{y}{(@{y} | \outputp{x}{y})) | P}}}{y} & \nonumber\\
	=
	& \outputp{x}{\quotep{(\prefix{x}{y}{(\outputp{x}{y} | @{y})) | P}}}
	  | {(\prefix{x}{y}{(\outputp{x}{y} | @{y})) | P}} & \nonumber\\
	\red
	& \ldots & \nonumber\\
	\red^*
	& P | P | \ldots & \nonumber
\end{eqnarray}

Of course, this encoding, as an implementation, runs away, unfolding
$\bangp{P}$ eagerly. A lazier and more implementable replication
operator, restricted to input-guarded processes, may be obtained as follows.

\begin{eqnarray}
\bangp{\prefix{u}{v}{P}} 
	:= 
	\binpar{\lift{x}{\prefix{u}{v}{(\binpar{D(x)}{P})}}}{D(x)} \nonumber
\end{eqnarray}

\begin{remark}
  Note that the lazier definition still does not deal with summation
  or mixed summation (i.e. sums over input and output). The reader is
  invited to construct definitions of replication that deal with these
  features. 

  Further, the definitions are parameterized in a name, $x$. Can you,
  gentle reader, make a definition that eliminates this parameter and
  guarantees no accidental interaction between the replication
  machinery and the process being replicated -- i.e. no accidental
  sharing of names used by the process to get its work done and the
  name(s) used by the replication to effect copying. This latter
  revision of the definition of replication is crucial to obtaining
  the expected identity $!!P \sim !P$.
\end{remark}

\begin{remark}\label{rem:paradoxical_combinator}
  The reader familiar with the lambda calculus will have noticed the
  similarity between $D$ and the paradoxical combinator.

  [Ed. note: the existence of this seems to suggest we have to be more
  restrictive on the set of processes and names we admit if we are to
  support no-cloning.]
\end{remark}

\subsubsection{Bisimulation}

The computational dynamics gives rise to another kind of equivalence,
the equivalence of computational behavior. As previously mentioned
this is typically captured \emph{via} some form of bisimulation.

% The notion we use in this paper is weak barbed bisimulation
% \cite{milner91polyadicpi}.

The notion we use in this paper is derived from weak barbed
bisimulation \cite{milner91polyadicpi}. 

\begin{definition}
An \emph{observation relation}, $\downarrow_{\mathcal N}$, over a set
of names, $\mathcal N$, is the smallest relation satisfying the rules
below.

\infrule[Out-barb]{y \in {\mathcal N}, \; x \nameeq y}
		  {\outputp{x}{v} \downarrow_{\mathcal N} x}
\infrule[Par-barb]{\mbox{$P\downarrow_{\mathcal N} x$ or $Q\downarrow_{\mathcal N} x$}}
		  {\binpar{P}{Q} \downarrow_{\mathcal N} x}

We write $P \Downarrow_{\mathcal N} x$ if there is $Q$ such that 
$P \wred Q$ and $Q \downarrow_{\mathcal N} x$.
\end{definition}

\begin{definition}
%\label{def.bbisim}
An  ${\mathcal N}$-\emph{barbed bisimulation} over a set of names, ${\mathcal N}$, is a symmetric binary relation 
${\mathcal S}_{\mathcal N}$ between agents such that $P\rel{S}_{\mathcal N}Q$ implies:
\begin{enumerate}
\item If $P \red P'$ then $Q \wred Q'$ and $P'\rel{S}_{\mathcal N} Q'$.
\item If $P\downarrow_{\mathcal N} x$, then $Q\Downarrow_{\mathcal N} x$.
\end{enumerate}
$P$ is ${\mathcal N}$-barbed bisimilar to $Q$, written
$P \wbbisim_{\mathcal N} Q$, if $P \rel{S}_{\mathcal N} Q$ for some ${\mathcal N}$-barbed bisimulation ${\mathcal S}_{\mathcal N}$.
\end{definition}

$\mathcal{R} \subseteq \pi \times \pi$

$P \mathcal{R} Q => \forall P'. P \red P' \Rightarrow \exists Q'. Q \red Q', P' \mathcal{R} Q'$

$P \vdash x \Rightarrow Q \vdash x$

\begin{mathpar}
  \inferrule*[lab=Out-barb]{x \nameeq y}{{y}!\langle{Q}\rangle \vdash x}
  \and
  \inferrule*[lab=Par-barb]{\mbox{$P\vdash x$ or $Q\vdash x$}}{\binpar{P}{Q} \vdash x}
\end{mathpar}

\subsubsection{Contexts}

One of the principle advantages of computational calculi like the
$\pi$-calculus is a well-defined notion of context,
contextual-equivalence and a correlation between
contextual-equivalence and notions of bisimulation. The notion of
context allows the decomposition of a process into (sub-)process and
its syntactic environment, its context. Thus, a context may be
thought of as a process with a ``hole'' (written $\Box$) in it. The
application of a context $M$ to a process $P$, written $M[P]$, is
tantamount to filling the hole in $M$ with $P$. In this paper we do
not need the full weight of this theory, but do make use of the notion
of context in the proof the main theorem. 

\begin{mathpar}
  \inferrule* [lab=summation] {} {{M_{M},M_{N}} \bc \Box \;|\; x.M_{A} \;|\; M_{M}+M_{N}}
  \and
  \inferrule* [lab=agent] {} {{M_{A}} \bc (\vec{x})M_{P} \;| \; \clift{P_0,\ldots,M_{P},\ldots,P_N}}
  \and \\
  \inferrule* [lab=process] {} {{M_{P}} \bc M_{N} \;| \;P|M_{P} }
\end{mathpar} 

\begin{mathpar}
  \inferrule* [lab=sychronization] {} {M_{N} \bc \Box \;|\; x?M_{F} \;|\; x!M_{C}}
  \and
  \inferrule* [lab=abstraction] {} {{M_{F}} \bc (x)M_{P} }
  \and
  \inferrule* [lab=concretion] {} {{M_{C}} \bc \langle M_{P} \rangle }
  \and \\
  \inferrule* [lab=process] {} {{M_{P}} \bc M_{N} \;| \;P|M_{P} }
\end{mathpar}

\begin{definition}[contextual application] Given a context $M$, and
  process $P$, we define the \emph{contextual application}, $M[P] :=
  M\{P/\Box\}$. That is, the contextual application of M to P is the
  substitution of $P$ for $\Box$ in $M$.
\end{definition}

$\meaningof{-} : L \to \mathcal{P}(\pi)$

\begin{mathpar}
  \inferrule* [lab=collection] {} {\meaningof{true} = \pi, \and \meaningof{~E} = \pi \setminus \meaningof{E}, \and \meaningof{E_{1} \& E_{2}} = \meaningof{E_{1}} \cap \meaningof{E_{2}}}
\end{mathpar}

\begin{mathpar}
  \inferrule* [lab=structure] {} {\meaningof{0} = \{ P \in \pi | P \equiv 0 \}, \and \\ \meaningof{E_1 | E_2} = \{ P \in \pi | P \equiv P_{1} | P_{2}, P_{1} \in \meaningof{E_{1}}, P_{2} \in \meaningof{E_2}\} }
\end{mathpar}

\begin{mathpar}
 \inferrule* [lab=behavior] {} {\meaningof{\langle a?b \rangle E} = \{ P \in \pi | P \equiv Q | u?(y)P', \\ \and \\\\ \and \\ \;\;\; u \in \meaningof{a}, \forall z.P'\{z/y\} \in \meaningof{E\{z/b\}}\}, \and \\ \meaningof{a!E} = \{ P \in \pi | P \equiv Q | x!\langle P' \rangle, x \in \meaningof{a} P' \in \meaningof{E}\} }
\end{mathpar}

\begin{mathpar}
 \inferrule* [lab=nominal] {} {\meaningof{\quotep{E}} = \{ \quotep{P} \in \quotep{\pi} | P \in \meaningof{E} \}, \and \meaningof{\quotep{P}} = \{ \quotep{Q} \in \quotep{\pi} | P \equiv Q \} \and \\ \meaningof{@\quotep{E}} = \{ P \in \pi | P \equiv @x, x \in \meaningof{E} \}}
\end{mathpar}

\begin{eqnarray*}
  \\
  \meaningof{-} : TS \to ST
\end{eqnarray*}

\begin{eqnarray*}
  \\
  L : TS \to ST
\end{eqnarray*}

\begin{eqnarray*}
  \\
  P \models E \iff P \in \meaningof{E}
\end{eqnarray*}

\begin{eqnarray*}
  P \approx_{L} Q \iff \forall E \in L. P \models E \iff Q \models E
\end{eqnarray*}

\begin{eqnarray*}
  P \approx_{K} Q
\end{eqnarray*}

\begin{eqnarray*}
  P \approx Q
\end{eqnarray*}

$\approx_{K} = \approx = \approx_{L}$

\subsubsection{Contextual duality}

Note that contexts extend the quotation operation to a family of
operations from processes to names. Given a context, $M$, we can
define a \emph{nominal context}, $\quotep{M}$ by $\quotep{M}[P] :=
\quotep{M[P]}$. To foreshadow what is to come we observe that these
operations enjoy a duality with processes very much like the duality
between vectors and maps from vectors to scalars.

Further, because the calculus is essentially higher-order, we have a
correspondence between contexts and processes. More specifically,
given a name $x$ and a context $M$ we can construct $M^{*}_{x}$ such
that 

\begin{mathpar}
  M^{*}_{x} | \lift{x}{P} \red M[P]
\end{mathpar}

namely,

\begin{mathpar}
  M^{*}_{x} := x?(u).M[\dropn{u}]
\end{mathpar}

The dependence of $M^{*}_{x}$ on a name makes it an abstraction, 

\begin{mathpar}
  M^{*} := (x)x?(u).M[\dropn{u}]
\end{mathpar}

\subsection{Additional notation}

It will sometimes be convenient to denote the process a name
quotes. We already have the notation $x = \quotep{P}$, but it will be
convenient to introduce an alternate notation, $\procn{x}$, when we
want to emphasize the connection to the use of the name. Note that, by
virtue of name equivalence, $\quotep{\procn{x}} \nameeq x$; so, the
notation is consistent with previous definitions.

Further, because names have structure it is possible to effect
substitutions on the basis of that structure. This means we need to
upgrade our notation for substitutions, which we accomplish by
adapting comprehension notation. Thus,

\begin{mathpar}
  P\{ y / x : x \in S \}
\end{mathpar}

is interpreted to mean the process derived from P by replacing (in a
capture-avoiding manner) each occurrence of $x$ in $S$ by $y$. For example,

\begin{mathpar}
  P\{ \quotep{\procn{x}|\procn{x}} / x : x \in \freenames{P} \}
\end{mathpar}

will replace each (occurrence) of a free name $x$ in $P$ by
$\quotep{\procn{x}|\procn{x}}$.

Also, we will avail ourselves of the notation $x^{L}$ and $x^{R}$ to
denote injections of a name into disjoint copies of the name
space. There are numerous ways to accomplish this. One example can be
found in \cite{MeredithR05}. This notation overloads to vectors of
names: $\vec{x}^{\pi} := (x_{i}^{\pi} \; : \; 0 \leq i < |\vec{x}| )$ where $\pi \in \{L,R\}$.

We also use $P^{\Box} := P|\Box$.

In \cite{MeredithR05} an interpretation of the new operator is
given. It turns out that there are several possible interpretations
all enjoying the requisite algebraic properties of the operator (see
\cite{milner91polyadicpi}). We will therefore make liberal use of
$(\nu\; \vec{x})P$.

% subsection the_syntax_and_semantics_of_the_notation_system (end)   

\section{Interpretation of QM}
\subsection{Supporting definitions}
\subsubsection{Multiplication}
\begin{mathpar}
  \quotep{Q} \cdot \quotep{R} := \quotep{Q|R}
  \and \\
  \quotep{Q} \cdot P := P\{ \quotep{Q|R} / \quotep{R} : \quotep{R} \in \freenames{P} \}
\end{mathpar}

\paragraph{Discussion}
The first line needs little explanation. The second line says that
each free name of the process is replaced with the multiplication of
that name by the scalar. Multiplication of a scalar (name) by a state
(process) results in a process all the names of which have been `moved
over' by parallel composition with the process the scalar
quotes. There is a subtlety that the bound names have to be
manipulated so that multiplied names aren't accidentally
captured. There are many ways to achieve this.

\begin{remark}\label{rem:multiplication_identities}
  The reader is invited to verify that for all $x,y,z \in \QProc$ and $P \in \Proc$
  \begin{mathpar}
    x \cdot \quotep{0} \equiv x 
    \and
    x \cdot y \equiv y \cdot x
    \and
    x \cdot (y \cdot z) \equiv (x \cdot y) \cdot z
    \and \\
    \quotep{0} \cdot P \equiv P
    \and \\
    x \cdot (y \cdot P) \equiv (x \cdot y) \cdot P
    \and \\
    x \cdot (P|Q) \equiv (x \cdot P) | (x \cdot Q)
    \and \\    
  \end{mathpar}
\end{remark}

\subsubsection{Tensor product}

We define a tensor product on processes by structural induction.

\paragraph{Tensor of sums} First note that all summations, including
$\pzero$ and sequence, can be written $\Sigma_{i} x_{i}.A_{i} +
\Sigma_{j} x_{j}.C_{j}$, where we have grouped input-guarded processes
together and output-guarded processes together.

Thus, we can define the tensor product of two summations, $N_{1}\otimes N_{2}$, where

\begin{mathpar}
  N_{1} := \Sigma_{i} x_{i}.A_{i} + \Sigma_{j} x_{j}.C_{j}
  \and
  N_{2} := \Sigma_{i'} y_{i'}.B_{i'} + \Sigma_{j'} y_{j'}.D_{j'} 
\end{mathpar}

as follows.

\begin{mathpar}
  \Sigma_{i} x_{i}.A_{i} + \Sigma_{j} x_{j}.C_{j} \otimes \Sigma_{i'}
  y_{i'}.B_{i'} + \Sigma_{j'} y_{j'}.D_{j'} 
  \and \\
  := \; \Sigma_{i} \Sigma_{i'} \quotep{\stackrel{\vee}{x_{i}}| \stackrel{\vee}{y_{i'}}}.(A_{i}\otimes B_{i'}) \; | \; \Sigma_{i'} \Sigma_{i} \quotep{\stackrel{\vee}{y_{i'}}|\stackrel{\vee}{x_{i}}}.(B_{i'}\otimes A_{i})
  \and
  \;\; | \;\; \Sigma_{j} \Sigma_{j'} \quotep{\stackrel{\vee}{x_{j}}|\stackrel{\vee}{y_{j'}}}.(A_{j}\otimes B_{j'}) \; | \; \Sigma_{j'} \Sigma_{j} \quotep{\stackrel{\vee}{y_{j'}}|\stackrel{\vee}{x_{j}}}.(B_{j'}\otimes A_{j})
\end{mathpar}

\begin{remark}
  Do we need to $x^{L}$ and $y^{R}$ for this construction as well?
\end{remark}

\paragraph{Tensor of parallel compositions} Next, we distribute tensor
over par.

\begin{mathpar}
  P_{1}|P_{2} \otimes Q_{1}|Q_{2} := (P_{1} \otimes Q_{1}) | (P_{1}
  \otimes Q_{2}) | (P_{2} \otimes Q_{1}) | (P_{2} \otimes Q_{2})
\end{mathpar}

\paragraph{Tensor with dropped names} We treat tensor of a
process with a dropped name as parallel composition.

\begin{mathpar}
  P \otimes \dropn{x} := P | \dropn{x}
\end{mathpar}

\paragraph{Tensor of agents}

Finally, we need to define tensor on agents. Note that the definition
of tensor on normal products only tensors inputs with inputs and
outputs with outputs. Thus, we only have to define the operation on
``homogeneous'' pairings.

\begin{mathpar}
  (\vec{x})P \otimes (\vec{y})Q
  \and \\
  := (x_{0}^{L}|y_{0}^{R},\ldots,x_{0}^{L}|y_{n}^{R},\ldots,x_{m}^{L}|y_{0}^{R},\ldots,x_{m}^{L}|y_{n}^R)(P\{ \vec{x}^{L}/\vec{x}\} \otimes Q \{ \vec{y}^{R}/\vec{y}\})
  \and \\
  \clift{\vec{P}} \otimes \clift{\vec{Q}}
  \and \\
  := \clift{P_{0}\otimes Q_{0},\ldots,P_{0}\otimes Q_{n},\ldots,P_{m}\otimes Q_{0},\ldots,P_{m}\otimes Q_{n}}
\end{mathpar}

\begin{remark}
  Observe that arities of tensored abstractions matches arities of
  tensored concretions if the original arities matched. Note also that
  the length of the arities corresponds to the increase in dimension
  we see in ordinary vector space tensor product.
\end{remark}

\begin{remark}
  Operationally, this definition distributes the tensor down to
  components ``linked'' by summation. Tensor over summation is
  intriguing in that it mixes names. Moreover, as a consequence of the
  way it mixes names we have the identities for all $x \in \QProc$ and
  $P,Q \in \Proc$

  \begin{mathpar}
    (x \cdot P) \otimes Q \equiv x \cdot (P \otimes Q) \equiv P \otimes (x \cdot Q)
    \and
    P \otimes \pzero \equiv P
  \end{mathpar}

  that the reader is invited to verify.
\end{remark}

\subsubsection{Annihilation}
\begin{mathpar}
  P^{\perp} := \{ Q | \forall R. P|Q \red^{*} R \Rightarrow R \red^{*} \pzero \}
  \and \\
  P^{\underline{\perp}} := \Sigma_{Q \in P^{\perp}} \quotep{Q}?(y).(\dropn{y}|Q) | \Sigma_{Q \in P^{\perp}} \quotep{Q}\clift{\Box}
\end{mathpar}

\paragraph{Discussion} The reader will note that $P^{\perp}$ is a
\emph{set} of processes, while $P^{\underline{\perp}}$ is a
\emph{context}. We call the set $P^{\perp}$ the \emph{annihilators} of
$P$. The parallel composition of a process in the annihilators of $P$
with $P$ will result in a process, the state space of which has all
paths eventually leading to $\pzero$. Execution may endure loops; but
under reasonable conditions of fairness (naturally guaranteed under
most notions of bisimulation) such a composite process cannot get
stuck in such a loop and will, eventually pop out and terminate.

The context $P^{\underline{\perp}}$ is ready and willing to ``take the
$P$ out of'' the process to which it is applied. It will effectively
transmit the code of the process to which it is applied to one of the
annihilators and run the process against it.

\subsubsection{Evaluation}
We fix $M$ a domain of fully abstract interpretation with an equality
coincident with bisimulation. We take $\meaningof{\cdot} : \Proc \to
M$ to be the map interpreting processes and $\nmeaningof{\cdot} : \M
\to Proc$ to be the map running the other way. Then we define

\begin{mathpar}
  \int P := \nmeaningof{\meaningof{P}}
\end{mathpar}

\paragraph{Discussion}
There are many fully abstract interpretations of Milner's
$\pi$-calculus. Any of them can be used as a basis for interpreting
the reflective calculus here. Equipped with such a domain it is
largely a matter of grinding through to check that the Yoneda
construction for the normalization-by-evaluation program can be
extended to this setting.

\begin{remark}
  The reader is invited to verify that $\int (P^{\underline{\perp}}[P]) = 0$.
\end{remark}

\subsection{Quantum mechanics}

Table \ref{tbl:core_qm_op_defns} gives the core operational definitions

\begin{table}[htp]\label{tbl:core_qm_op_defns}
  \center{
    \fbox{
      \begin{tabular}{c|c}
        quantum mechanics & process calculus \\
        \hline
        scalar & $x := \quotep{P}$ \\
        state vector & $\state{P} := P$ \\
        dual & $\state{P}^{*} := \event{P^{\underline{\perp}}} := \quotep{P^{\underline{\perp}}}[-]$ \\
        matrix & $ \Sigma_{\alpha} \state{P_{\alpha}}x_{\alpha}\event{Q_{\alpha}}$ \\
        vector addition & $\state{P} + \state{Q} := \state{P | Q}$ \\
        tensor product & $\state{P} \otimes \state{Q} := \state{P \otimes Q}$ \\
        inner product & $\innerprod{P}{Q} := \quotep{\int P^{\underline{\perp}}[Q]}$ \\
      \end{tabular}
    }
  }
  \caption{QM - operational definitions}
\end{table}

where

\begin{mathpar}
  \prmatrix{P}{Q} := \fprmatrix{P}{\quotep{\pzero}}{Q}
  \and
  \fprmatrix{P}{x}{Q} := (\state{P},x,\event{Q})
  \and
  (\fprmatrix{P}{x}{Q})(\state{R}) := x \cdot \innerprod{Q}{R} \cdot \state{P}
  \and
  (\fprmatrix{P}{x}{Q})(\event{R}) := x \cdot \innerprod{R}{P} \cdot \event{Q}
\end{mathpar}

\paragraph{Discussion}
As promised: vectors (aka states) are represented as processes; duals
as contextual duals; inner product definition should be compared with
standard inner product definition for ....

\begin{remark}
  Assuming $\int (P^{\underline{\perp}}[P]) = 0$, the reader is
  invited to verify that $(\fprmatrix{P}{x}{P})(\state{P}) = x \cdot \state{P}$.
\end{remark}

\begin{remark}
  The reader is invited to verify that $\innerprod{P}{Q}$ could
  equally well have been written $\quotep{\int \stackrel{\vee}{x}}$
  where $x = \event{P^{\underline{\perp}}}(Q)$.

  One of the motivations for this remark is that there is another way
  to factor these operations. We could package up evaluation in the dual:

  \begin{mathpar}
    \state{P}^{*} := \event{\int P^{\underline{\perp}}} := \quotep{\int P^{\underline{\perp}}}[-]
  \end{mathpar}

  and then have inner product defined by
  
  \begin{mathpar}
    \innerprod{P}{Q} := \event{P}(Q)
  \end{mathpar}

  Hopefully, experience with the calculations will provide guidance on
  the best factoring.
\end{remark}

\begin{remark}
  Assuming $\int (P^{\underline{\perp}}[P]) = 0$, the reader is
  invited to verify that $\forall P,Q. (\prmatrix{0}{Q})(\state{0}) =
  \state{0}$ and dually $(\prmatrix{P}{0})(\event{0}) = \event{0}$.
\end{remark}

\begin{remark}
  i'm a little worried that i don't (yet) have proper support for
  complex conjugacy. But, the observation above may give us a
  clue. According to Abramsky, it must be the case that the scalars
  are iso to the homset of the identity for the tensor -- which the
  observation above characterizes. 

  For now, we will simply bookmark the notion with $\overline{x}$.
\end{remark}

\subsubsection{Adjointness}

We need to give a definition of $(\cdot)^{\dagger}$ for matrices. The
obvious candidate definition is
\begin{mathpar}
(\Sigma_{\alpha}\fprmatrix{P_{\alpha}}{x_{\alpha}}{Q_{\alpha}})^{\dagger}
= \Sigma_{\alpha}\fprmatrix{(Q_{\alpha}^{\underline{\perp}})^{*}}{\overline{x}_{\alpha}}{P_{\alpha}^{\underline{\perp}}} 
\end{mathpar}

But, $(Q_{\alpha}^{\underline{\perp}})^{*}$ requires a name along
which to communicate the process to achieve the context application.

\subsubsection{Basis for a basis}
If processes label states and ``addition'' of states (a.k.a. vector
addition) is interpreted as parallel composition, what corresponds to
notions of linear independence and basis? Here, we recall that Yoshida
has developed a set of \emph{combinators} for an asynchronous verison
of Milner's $\pi$-calculus. These are a finite set of processes such
any process can be expressed as parallel composition of these
combinators together with liberal uses of the new operator and
replication. We can simply give a translation of these into the
present calculus and have reasonable expectation that the property
carries over. That is, that the resultant set allows to express all
processes via parallel composition. Note, however, that there is no
new operator or replication in this calculus. As a result, we expect
that the corresponding set is actually infinite. That is, we expect
that the space is actually infinite dimensional.

\begin{remark}
  The attentive reader may be a bit concerned. Certainly, the
  collection $S$, $K$ and $I$ is a finite set of
  combinators. Shouldn't we expect to see a finite set of combinators
  for an effectively equivalent system? i am very sympathetic to this
  critique and feel it warrants full attention. On the other hand, i
  also have in mind the following analogy. The natural numbers, as a
  monoid under addition, has exactly $1$ generator, while the natural
  numbers, as a monoid under multiplication, has countably many
  generators (the primes). We observe that the application of the
  lambda calculus is much less resource sensitive than the parallel
  composition of the $\pi$-calculus. Could it be the case that we have
  an analogy of the form
  
  \begin{mathpar}
    m + n : MN :: m*n : M|N
  \end{mathpar}

  giving a similar blow up in the set of ``primes''?  This is such a
  wonderful thought that, even if it's not true, i think it's worth
  writing down.
\end{remark}
 

\documentclass[12pt]{llncs}
%\documentclass{jktr}

\usepackage[pdftex]{hyperref}                   
\usepackage {listings}
\usepackage {mathpartir}
\usepackage{bcprules}
%\usepackage{listings}
                       
\usepackage{graphicx} 
%\usepackage[margins=2.5cm,nohead,nofoot]{geometry}
%\usepackage{geometry}
\usepackage{amsfonts}
\usepackage{amstext}
\usepackage{latexsym}
\usepackage{amssymb}
\usepackage{color}


%\include{myPreamble}
\include{qm2pi.local} 

%\ifpdf
%\usepackage[pdftex]{graphicx}
%\else
%\usepackage{graphicx}
%\fi

 % \ifpdf
%  \usepackage{pdfsync}
%  \if


%\title{Brief Article}
%\author{David F. Snyder}
%\author{L.G. Meredith}

%\address{Dept. of Math., Texas State University--San Marcos, San Marcos, TX 78666}
       
\pagestyle{empty}


\begin{document}

\lstset{language=[Objective]Caml,frame=shadowbox}

\input{qm2pi.front}

% section front matter (end)

\input{qm2pi.intro} 
 
% section introduction (end)

% \input{qm2pi.knotations} 

% section notation (end)

\input{qm2pi.process.calculi} 

% section concurrent_process_calculi_and_spatial_logics_ (end)
    
%\input{qm2pi.knots2pi} 

%\input{qm2pi.trefoil} 

%\input{qm2pi.mainthm} 

% subsection basic_interpretation (end)

%\input{qm2pi.rho.presentation} 
\subsection{The syntax and semantics of the notation system}\label{sub:the_syntax_and_semantics_of_the_notation_system} % (fold)

We now summarize a technical presentation of the calculus that
embodies our theory of dynamics. The typical presentation of such a
calculus follows the style of giving generators and relations on
them. The grammar, below, describing term constructors, freely
generates the set of processes, $\Proc$. This set is then quotiented
by a relation known as structural congruence and it is over this set
that the notion of dynamics is expressed. This presentation is
essentially that of \cite{MeredithR05} with the addition of
polyadicity and summation. For readability we have relegated some of
the technical subtleties to an appendix.

\subsubsection{Process grammar}\label{subsub:process_grammar}

\begin{mathpar}
  \inferrule* [lab=synchronization] {} {{M} \bc \pzero \;|\; x?F \;|\; x!C }
  \and
  \inferrule* [lab=abstraction] {} {{F} \bc (x)P}
  \and
  \inferrule* [lab=concretion] {} {{C} \bc \langle Q \rangle}
  \and
  \inferrule* [lab=process] {} {{P,Q} \bc M \;| \;P|Q \;|\; @{x}}
  \and
  \inferrule* [lab=name] {} {{x} \bc \quotep{P}}
\end{mathpar} 

Note that $\vec{x}$ (resp. $\vec{P}$) denotes a vector of names
(resp. processes) of length $|\vec{x}|$ (resp. $|\vec{P}|$). We adopt
the following useful abbreviations.

\begin{mathpar}
   x?(\vec{y}).P := x.(\vec{y})P \and  x\clift{\vec{P}} := x.\clift{\vec{P}}
   \and x!(y) := \lift{x}{\dropn{y}}
   \and \Pi_{i=0}^{n-1}P_i := P_0 | \ldots | P_{n-1}
\end{mathpar}

\subsubsection{Structural congruence}

\paragraph{Free and bound names and alpha-equivalence.} At the
core of structural equivalence is alpha-equivalence which identifies
process that are the same up to a change of variable. Formally, we
recognize the distinction between free and bound names. The free names
of a process, $\freenames{P}$, may be calculated recursively as
follows:

\begin{mathpar}
\freenames{\pzero} := \emptyset
  \and \\
  \freenames{x?(y).P} := \{ x \} \cup (\freenames{P} \setminus \{ y \})
  \and 
  \freenames{x!\langle P \rangle} := \{ x \} \cup \{ P \} 
  \and \\
  \freenames{P|Q} := \freenames{P} \cup \freenames{Q}
  \and \\
  \freenames{@{x}} := \{ x \}
\end{mathpar}

$\pi$
$\quotep{\pi}$

$\freenames{-} : \pi \to \mathcal{P}(\quotep{\pi})$

\begin{eqnarray*}
  \freenames{\pzero} & := & \emptyset \\
  \freenames{x?(y).P} & := & \{ x \} \cup (\freenames{P} \setminus \{ y \}) \\
  \freenames{x!\langle P \rangle} & := & \{ x \} \cup \{ P \} \\
  \freenames{P|Q} & := & \freenames{P} \cup \freenames{Q} \\
  \freenames{\dropn{x}} & := & \{ x \}
\end{eqnarray*}

The bound names of a process, $\boundnames{P}$, are those names occurring in $P$
that are not free. For example, in $x?(y).0$, the name $x$ is free, while $y$ is bound.

\begin{mathpar}
  \inferrule* [lab=monoidal-laws] {} { P|Q \equiv Q|P \and P|0 \equiv P \and P|(Q|R) \equiv (P|Q)|R }
\end{mathpar}

\begin{mathpar}
  \inferrule* [lab=alpha-equivalence] {} { (x)P \equiv (y)P\{y/x\} \and y \not\in \freenames{P} }
\end{mathpar}

\begin{definition}
Then two processes, $P,Q$, are alpha-equivalent if $P = Q\{\vec{y}/\vec{x}\}$ for
some $\vec{x} \in \boundnames{Q},\vec{y} \in \boundnames{P}$, where $Q\{\vec{y}/\vec{x}\}$
denotes the capture-avoiding substitution of $\vec{y}$ for $\vec{x}$ in $Q$.
\end{definition}

\begin{definition}
  The {\em structural congruence} \cite{SangiorgiWalker} , $\equiv$,
  between processes is the least congruence containing
  alpha-equivalence, satisfying the abelian monoid laws
  (associativity, commutativity and $\pzero$ as identity) for parallel
  composition $|$ and for summation $+$.
\end{definition}

\subsection{Name equivalence}

We take name equivalence, written $\nameeq$, to be the smallest
equivalence relation generated by the following rules.

\begin{mathpar}
\inferrule*[lab=Quote-drop]
{ }
{ \quotep{@{x}} \nameeq x }

\inferrule*[lab=Struct-equiv]
{ P \scong Q }
{ \quotep{P} \nameeq \quotep{Q} }
\end{mathpar}

The astute reader will have noticed that the mutual recursion of names
and processes imposes a mutual recursion on alpha-equivalence and
structural equivalence via name-equivalence. Fortunately, all of this
works out pleasantly and we may calculate in the natural way, free of
concern. The reader interested in the details is referred to the
appendix \ref{appendix:rho_details}.

\subsection{Substitution}

We use $\Proc$ for the set of processes, $\QProc$ for the set of
names, and $\id{\{}\vec{y} / \vec{x} \id{\}}$ to denote partial maps,
$s : \QProc \rightarrow \QProc$. A map, $s$ lifts, uniquely, to a map
on process terms, $\widehat{s} : \Proc \rightarrow \Proc$ by the
following equations.

\begin{mathpar}
  (0) \psubstp{Q}{P} := 0 \\
  (R \juxtap S) \psubstp{Q}{P}
  :=    
  (R)\psubstp{Q}{P} \juxtap (S) \psubstp{Q}{P} \\
  (x?(y).R) \psubstp{Q}{P}    
  :=    
  (x)\substp{Q}{P} (z)\concat( (R \psubstn{z}{y}) \psubstp{Q}{P} ) \\
  (\lift{x}{R}) \psubstp{Q}{P}  
  :=
  \lift{(x)\substp{Q}{P}}{ R \psubstp{Q}{P} } \\
%   (\dropn{x})  \psubstp{Q}{P}       
%   := 
%   \left\{ 
%     \begin{array}{ccc} 
%       \dropn{\quotep{Q}} & & x \nameeq \quotep{P} \\
%       \dropn{x} & & otherwise \\
%     \end{array}
%   \right. 
  (\dropn{x})  \psubstp{Q}{P}       
  := 
  \left\{ 
    \begin{array}{ccc} 
      Q & & x \nameeq \quotep{P} \\
      \dropn{x} & & otherwise \\
    \end{array}
  \right.
\end{mathpar}
 

where

\begin{eqnarray}
  (x)\id{\{} \lpquote Q \rpquote / \lpquote P \rpquote \id{\}}            = 
  \left\{ 
    \begin{array}{ccc}
      \lpquote Q \rpquote & & x \nameeq \lpquote P \rpquote \\
      x & & otherwise \\
    \end{array}
  \right. \nonumber
\end{eqnarray}

and $z$ is chosen distinct from $\quotep{P}$, $\quotep{Q}$, the free
names in $Q$, and all the names in $R$. Our $\alpha$-equivalence will
be built in the standard way from this substitution.

\begin{remark}\label{rem:no_self_referential_names}
  One consequence of these definitions is that $\forall P. \quotep{P}
  \not\in \freenames{P}$.
\end{remark}

\subsection{ Dynamic quote: an example }

Anticipating something of what's to come, consider applying the
substitution, $\widehat{\id{\{}u / z \id{\}}}$, to the following pair
of processes, $\lift{w}{y!(z)}$ and $w[ \lpquote y!(z) \rpquote ]$.

\begin{eqnarray}
	\lift{w}{y!(z)}\widehat{\id{\{}u / z \id{\}}}
		& = &
		\lift{w}{y!(u)} \nonumber\\
	w[ \lpquote y!(z) \rpquote ] \widehat{ \id{\{}u / z \id{\}} }
		& = &
		w[ \lpquote y!(z) \rpquote ] \nonumber
\end{eqnarray}

Because the body of the process between quotes is impervious to
substitution, we get radically different answers. In fact, by
examining the first process in an input context,
e.g. $x?(z).\lift{w}{y!(z)}$, we see that the process under the lift
operator may be shaped by prefixed inputs binding a name inside it. In
this sense, the lift operator will be seen as a way to dynamically
construct processes before reifying them as names.

Finally equipped with these standard features we can present the
dynamics of the calculus.

\subsubsection{Operational semantics} 

Finally, we introduce the computational dynamics. What marks these
algebras as distinct from other more traditionally studied algebraic
structures, e.g. vector spaces or polynomial rings, is the manner in
which dynamics is captured. In traditional structures, dynamics is typically
expressed through morphisms between such structures, as in linear maps
between vector spaces or morphisms between rings. In algebras
associated with the semantics of computation, the dynamics is
expressed as part of the algebraic structure itself, through a
reduction reduction relation typically denoted by $\red$. Below, we
give a recursive presentation of this relation for the calculus used
in the encoding.

$\red \subseteq \pi \times \pi$
$\red : \pi \to \mathcal{P}(\pi)$

\begin{mathpar}
  \inferrule* [lab=Comm] { \textsf{match}( x_{src}, x_{trgt} ) } { x_{trgt}?(y)P \; | \; x_{src}!\langle {Q} \rangle \red P\{\quotep{Q}/y}\} }
  \and \\
  \inferrule* [lab=Par] {{P} \red {P}'} {{{P} | {Q}} \red {{P}' | {Q}}}
  \and
  \inferrule* [lab=Equiv]{{{P} \scong {P}'} \andalso {{P}' \red {Q}'} \andalso {{Q}' \scong {Q}}}{{P} \red {Q}}
\end{mathpar}

\begin{eqnarray*}
  match_{\equiv} (\quotep{P},\quotep{Q}) & := & P \equiv Q \\
  match_{\dagger}(\quotep{P},\quotep{Q}) & := & \forall R. P|Q \red^{*} R => R \red^{*} 0 \\
  match_{K}(\quotep{P},\quotep{Q}) & := & K \mbox{ for some context } K
\end{eqnarray*}

$u?(x)P | u!\langle Q \rangle \red P\{\quotep{Q}/x\}$

%We write $\wred$ for $\red^*$, and $P\red$ if $\exists Q $ such that $ P \red Q$.
We write $P\red$ if $\exists Q $ such that $ P \red Q$ and $P\not\red$, otherwise.

\section{Replication}

As mentioned before, it is known that replication (and hence
recursion) can be implemented in a higher-order process algebra
\cite{SangiorgiWalker}. As our first example of calculation with the
machinery thus far presented we give the construction explicitly in
the {\rhoc}.

\begin{eqnarray}
	D_{x} & := & \prefix{x}{y}{(\binpar{\outputp{x}{y}}{@{y}})} \nonumber\\
	\bangp_{x}{P} & := & \binpar{{x}!\langle{\binpar{D_{x}}{P}}\rangle}{D_{x}} \nonumber
\end{eqnarray}

\begin{eqnarray}
	\bangp_{x}{P} & & \nonumber\\
	=
	& {x}!\langle{(\prefix{x}{y}{(\outputp{x}{y} | @{y})) | P}}\rangle 
	      | \prefix{x}{y}{(\outputp{x}{y} | @{y})} & \nonumber\\
	\red
	& (\outputp{x}{y} | @{y})\substn{\quotep{(\prefix{x}{y}{(@{y} | \outputp{x}{y})) | P}}}{y} & \nonumber\\
	=
	& \outputp{x}{\quotep{(\prefix{x}{y}{(\outputp{x}{y} | @{y})) | P}}}
	  | {(\prefix{x}{y}{(\outputp{x}{y} | @{y})) | P}} & \nonumber\\
	\red
	& \ldots & \nonumber\\
	\red^*
	& P | P | \ldots & \nonumber
\end{eqnarray}

Of course, this encoding, as an implementation, runs away, unfolding
$\bangp{P}$ eagerly. A lazier and more implementable replication
operator, restricted to input-guarded processes, may be obtained as follows.

\begin{eqnarray}
\bangp{\prefix{u}{v}{P}} 
	:= 
	\binpar{\lift{x}{\prefix{u}{v}{(\binpar{D(x)}{P})}}}{D(x)} \nonumber
\end{eqnarray}

\begin{remark}
  Note that the lazier definition still does not deal with summation
  or mixed summation (i.e. sums over input and output). The reader is
  invited to construct definitions of replication that deal with these
  features. 

  Further, the definitions are parameterized in a name, $x$. Can you,
  gentle reader, make a definition that eliminates this parameter and
  guarantees no accidental interaction between the replication
  machinery and the process being replicated -- i.e. no accidental
  sharing of names used by the process to get its work done and the
  name(s) used by the replication to effect copying. This latter
  revision of the definition of replication is crucial to obtaining
  the expected identity $!!P \sim !P$.
\end{remark}

\begin{remark}\label{rem:paradoxical_combinator}
  The reader familiar with the lambda calculus will have noticed the
  similarity between $D$ and the paradoxical combinator.

  [Ed. note: the existence of this seems to suggest we have to be more
  restrictive on the set of processes and names we admit if we are to
  support no-cloning.]
\end{remark}

\subsubsection{Bisimulation}

The computational dynamics gives rise to another kind of equivalence,
the equivalence of computational behavior. As previously mentioned
this is typically captured \emph{via} some form of bisimulation.

% The notion we use in this paper is weak barbed bisimulation
% \cite{milner91polyadicpi}.

The notion we use in this paper is derived from weak barbed
bisimulation \cite{milner91polyadicpi}. 

\begin{definition}
An \emph{observation relation}, $\downarrow_{\mathcal N}$, over a set
of names, $\mathcal N$, is the smallest relation satisfying the rules
below.

\infrule[Out-barb]{y \in {\mathcal N}, \; x \nameeq y}
		  {\outputp{x}{v} \downarrow_{\mathcal N} x}
\infrule[Par-barb]{\mbox{$P\downarrow_{\mathcal N} x$ or $Q\downarrow_{\mathcal N} x$}}
		  {\binpar{P}{Q} \downarrow_{\mathcal N} x}

We write $P \Downarrow_{\mathcal N} x$ if there is $Q$ such that 
$P \wred Q$ and $Q \downarrow_{\mathcal N} x$.
\end{definition}

\begin{definition}
%\label{def.bbisim}
An  ${\mathcal N}$-\emph{barbed bisimulation} over a set of names, ${\mathcal N}$, is a symmetric binary relation 
${\mathcal S}_{\mathcal N}$ between agents such that $P\rel{S}_{\mathcal N}Q$ implies:
\begin{enumerate}
\item If $P \red P'$ then $Q \wred Q'$ and $P'\rel{S}_{\mathcal N} Q'$.
\item If $P\downarrow_{\mathcal N} x$, then $Q\Downarrow_{\mathcal N} x$.
\end{enumerate}
$P$ is ${\mathcal N}$-barbed bisimilar to $Q$, written
$P \wbbisim_{\mathcal N} Q$, if $P \rel{S}_{\mathcal N} Q$ for some ${\mathcal N}$-barbed bisimulation ${\mathcal S}_{\mathcal N}$.
\end{definition}

$\mathcal{R} \subseteq \pi \times \pi$

$P \mathcal{R} Q => \forall P'. P \red P' \Rightarrow \exists Q'. Q \red Q', P' \mathcal{R} Q'$

$P \vdash x \Rightarrow Q \vdash x$

\begin{mathpar}
  \inferrule*[lab=Out-barb]{x \nameeq y}{{y}!\langle{Q}\rangle \vdash x}
  \and
  \inferrule*[lab=Par-barb]{\mbox{$P\vdash x$ or $Q\vdash x$}}{\binpar{P}{Q} \vdash x}
\end{mathpar}

\subsubsection{Contexts}

One of the principle advantages of computational calculi like the
$\pi$-calculus is a well-defined notion of context,
contextual-equivalence and a correlation between
contextual-equivalence and notions of bisimulation. The notion of
context allows the decomposition of a process into (sub-)process and
its syntactic environment, its context. Thus, a context may be
thought of as a process with a ``hole'' (written $\Box$) in it. The
application of a context $M$ to a process $P$, written $M[P]$, is
tantamount to filling the hole in $M$ with $P$. In this paper we do
not need the full weight of this theory, but do make use of the notion
of context in the proof the main theorem. 

\begin{mathpar}
  \inferrule* [lab=summation] {} {{M_{M},M_{N}} \bc \Box \;|\; x.M_{A} \;|\; M_{M}+M_{N}}
  \and
  \inferrule* [lab=agent] {} {{M_{A}} \bc (\vec{x})M_{P} \;| \; \clift{P_0,\ldots,M_{P},\ldots,P_N}}
  \and \\
  \inferrule* [lab=process] {} {{M_{P}} \bc M_{N} \;| \;P|M_{P} }
\end{mathpar} 

\begin{mathpar}
  \inferrule* [lab=sychronization] {} {M_{N} \bc \Box \;|\; x?M_{F} \;|\; x!M_{C}}
  \and
  \inferrule* [lab=abstraction] {} {{M_{F}} \bc (x)M_{P} }
  \and
  \inferrule* [lab=concretion] {} {{M_{C}} \bc \langle M_{P} \rangle }
  \and \\
  \inferrule* [lab=process] {} {{M_{P}} \bc M_{N} \;| \;P|M_{P} }
\end{mathpar}

\begin{definition}[contextual application] Given a context $M$, and
  process $P$, we define the \emph{contextual application}, $M[P] :=
  M\{P/\Box\}$. That is, the contextual application of M to P is the
  substitution of $P$ for $\Box$ in $M$.
\end{definition}

$\meaningof{-} : L \to \mathcal{P}(\pi)$

\begin{mathpar}
  \inferrule* [lab=collection] {} {\meaningof{true} = \pi, \and \meaningof{~E} = \pi \setminus \meaningof{E}, \and \meaningof{E_{1} \& E_{2}} = \meaningof{E_{1}} \cap \meaningof{E_{2}}}
\end{mathpar}

\begin{mathpar}
  \inferrule* [lab=structure] {} {\meaningof{0} = \{ P \in \pi | P \equiv 0 \}, \and \\ \meaningof{E_1 | E_2} = \{ P \in \pi | P \equiv P_{1} | P_{2}, P_{1} \in \meaningof{E_{1}}, P_{2} \in \meaningof{E_2}\} }
\end{mathpar}

\begin{mathpar}
 \inferrule* [lab=behavior] {} {\meaningof{\langle a?b \rangle E} = \{ P \in \pi | P \equiv Q | u?(y)P', \\ \and \\\\ \and \\ \;\;\; u \in \meaningof{a}, \forall z.P'\{z/y\} \in \meaningof{E\{z/b\}}\}, \and \\ \meaningof{a!E} = \{ P \in \pi | P \equiv Q | x!\langle P' \rangle, x \in \meaningof{a} P' \in \meaningof{E}\} }
\end{mathpar}

\begin{mathpar}
 \inferrule* [lab=nominal] {} {\meaningof{\quotep{E}} = \{ \quotep{P} \in \quotep{\pi} | P \in \meaningof{E} \}, \and \meaningof{\quotep{P}} = \{ \quotep{Q} \in \quotep{\pi} | P \equiv Q \} \and \\ \meaningof{@\quotep{E}} = \{ P \in \pi | P \equiv @x, x \in \meaningof{E} \}}
\end{mathpar}

\begin{eqnarray*}
  \\
  \meaningof{-} : TS \to ST
\end{eqnarray*}

\begin{eqnarray*}
  \\
  L : TS \to ST
\end{eqnarray*}

\begin{eqnarray*}
  \\
  P \models E \iff P \in \meaningof{E}
\end{eqnarray*}

\begin{eqnarray*}
  P \approx_{L} Q \iff \forall E \in L. P \models E \iff Q \models E
\end{eqnarray*}

\begin{eqnarray*}
  P \approx_{K} Q
\end{eqnarray*}

\begin{eqnarray*}
  P \approx Q
\end{eqnarray*}

$\approx_{K} = \approx = \approx_{L}$

\subsubsection{Contextual duality}

Note that contexts extend the quotation operation to a family of
operations from processes to names. Given a context, $M$, we can
define a \emph{nominal context}, $\quotep{M}$ by $\quotep{M}[P] :=
\quotep{M[P]}$. To foreshadow what is to come we observe that these
operations enjoy a duality with processes very much like the duality
between vectors and maps from vectors to scalars.

Further, because the calculus is essentially higher-order, we have a
correspondence between contexts and processes. More specifically,
given a name $x$ and a context $M$ we can construct $M^{*}_{x}$ such
that 

\begin{mathpar}
  M^{*}_{x} | \lift{x}{P} \red M[P]
\end{mathpar}

namely,

\begin{mathpar}
  M^{*}_{x} := x?(u).M[\dropn{u}]
\end{mathpar}

The dependence of $M^{*}_{x}$ on a name makes it an abstraction, 

\begin{mathpar}
  M^{*} := (x)x?(u).M[\dropn{u}]
\end{mathpar}

\subsection{Additional notation}

It will sometimes be convenient to denote the process a name
quotes. We already have the notation $x = \quotep{P}$, but it will be
convenient to introduce an alternate notation, $\procn{x}$, when we
want to emphasize the connection to the use of the name. Note that, by
virtue of name equivalence, $\quotep{\procn{x}} \nameeq x$; so, the
notation is consistent with previous definitions.

Further, because names have structure it is possible to effect
substitutions on the basis of that structure. This means we need to
upgrade our notation for substitutions, which we accomplish by
adapting comprehension notation. Thus,

\begin{mathpar}
  P\{ y / x : x \in S \}
\end{mathpar}

is interpreted to mean the process derived from P by replacing (in a
capture-avoiding manner) each occurrence of $x$ in $S$ by $y$. For example,

\begin{mathpar}
  P\{ \quotep{\procn{x}|\procn{x}} / x : x \in \freenames{P} \}
\end{mathpar}

will replace each (occurrence) of a free name $x$ in $P$ by
$\quotep{\procn{x}|\procn{x}}$.

Also, we will avail ourselves of the notation $x^{L}$ and $x^{R}$ to
denote injections of a name into disjoint copies of the name
space. There are numerous ways to accomplish this. One example can be
found in \cite{MeredithR05}. This notation overloads to vectors of
names: $\vec{x}^{\pi} := (x_{i}^{\pi} \; : \; 0 \leq i < |\vec{x}| )$ where $\pi \in \{L,R\}$.

We also use $P^{\Box} := P|\Box$.

In \cite{MeredithR05} an interpretation of the new operator is
given. It turns out that there are several possible interpretations
all enjoying the requisite algebraic properties of the operator (see
\cite{milner91polyadicpi}). We will therefore make liberal use of
$(\nu\; \vec{x})P$.

% subsection the_syntax_and_semantics_of_the_notation_system (end)   

\input{qm2pi.qmops} 

\input{qm2pi.sterngerlach} 

\input{qm2pi.metric} 

% section concurrent_process_calculi (end)

%\input{qm2pi.proofsketch}

% section proof sketch (end)

%\input{qm2pi.slviaknots} 

% section spatial logic via knots (end)

\input{qm2pi.conclusion}

% section conclusion (end)

%\input{qm2pi.dtcodes} 

% section wiring algorithm (end)

\input{qm2pi.ack} 

% section acknowledgments (end)

\newpage


\bibliographystyle{plain}   
\bibliography{../../biblios/main.bib}

\input{qm2pi.rhodetails}

\end{document}

 

\documentclass[12pt]{llncs}
%\documentclass{jktr}

\usepackage[pdftex]{hyperref}                   
\usepackage {listings}
\usepackage {mathpartir}
\usepackage{bcprules}
%\usepackage{listings}
                       
\usepackage{graphicx} 
%\usepackage[margins=2.5cm,nohead,nofoot]{geometry}
%\usepackage{geometry}
\usepackage{amsfonts}
\usepackage{amstext}
\usepackage{latexsym}
\usepackage{amssymb}
\usepackage{color}


%\include{myPreamble}
\include{qm2pi.local} 

%\ifpdf
%\usepackage[pdftex]{graphicx}
%\else
%\usepackage{graphicx}
%\fi

 % \ifpdf
%  \usepackage{pdfsync}
%  \if


%\title{Brief Article}
%\author{David F. Snyder}
%\author{L.G. Meredith}

%\address{Dept. of Math., Texas State University--San Marcos, San Marcos, TX 78666}
       
\pagestyle{empty}


\begin{document}

\lstset{language=[Objective]Caml,frame=shadowbox}

\input{qm2pi.front}

% section front matter (end)

\input{qm2pi.intro} 
 
% section introduction (end)

% \input{qm2pi.knotations} 

% section notation (end)

\input{qm2pi.process.calculi} 

% section concurrent_process_calculi_and_spatial_logics_ (end)
    
%\input{qm2pi.knots2pi} 

%\input{qm2pi.trefoil} 

%\input{qm2pi.mainthm} 

% subsection basic_interpretation (end)

%\input{qm2pi.rho.presentation} 
\subsection{The syntax and semantics of the notation system}\label{sub:the_syntax_and_semantics_of_the_notation_system} % (fold)

We now summarize a technical presentation of the calculus that
embodies our theory of dynamics. The typical presentation of such a
calculus follows the style of giving generators and relations on
them. The grammar, below, describing term constructors, freely
generates the set of processes, $\Proc$. This set is then quotiented
by a relation known as structural congruence and it is over this set
that the notion of dynamics is expressed. This presentation is
essentially that of \cite{MeredithR05} with the addition of
polyadicity and summation. For readability we have relegated some of
the technical subtleties to an appendix.

\subsubsection{Process grammar}\label{subsub:process_grammar}

\begin{mathpar}
  \inferrule* [lab=synchronization] {} {{M} \bc \pzero \;|\; x?F \;|\; x!C }
  \and
  \inferrule* [lab=abstraction] {} {{F} \bc (x)P}
  \and
  \inferrule* [lab=concretion] {} {{C} \bc \langle Q \rangle}
  \and
  \inferrule* [lab=process] {} {{P,Q} \bc M \;| \;P|Q \;|\; @{x}}
  \and
  \inferrule* [lab=name] {} {{x} \bc \quotep{P}}
\end{mathpar} 

Note that $\vec{x}$ (resp. $\vec{P}$) denotes a vector of names
(resp. processes) of length $|\vec{x}|$ (resp. $|\vec{P}|$). We adopt
the following useful abbreviations.

\begin{mathpar}
   x?(\vec{y}).P := x.(\vec{y})P \and  x\clift{\vec{P}} := x.\clift{\vec{P}}
   \and x!(y) := \lift{x}{\dropn{y}}
   \and \Pi_{i=0}^{n-1}P_i := P_0 | \ldots | P_{n-1}
\end{mathpar}

\subsubsection{Structural congruence}

\paragraph{Free and bound names and alpha-equivalence.} At the
core of structural equivalence is alpha-equivalence which identifies
process that are the same up to a change of variable. Formally, we
recognize the distinction between free and bound names. The free names
of a process, $\freenames{P}$, may be calculated recursively as
follows:

\begin{mathpar}
\freenames{\pzero} := \emptyset
  \and \\
  \freenames{x?(y).P} := \{ x \} \cup (\freenames{P} \setminus \{ y \})
  \and 
  \freenames{x!\langle P \rangle} := \{ x \} \cup \{ P \} 
  \and \\
  \freenames{P|Q} := \freenames{P} \cup \freenames{Q}
  \and \\
  \freenames{@{x}} := \{ x \}
\end{mathpar}

$\pi$
$\quotep{\pi}$

$\freenames{-} : \pi \to \mathcal{P}(\quotep{\pi})$

\begin{eqnarray*}
  \freenames{\pzero} & := & \emptyset \\
  \freenames{x?(y).P} & := & \{ x \} \cup (\freenames{P} \setminus \{ y \}) \\
  \freenames{x!\langle P \rangle} & := & \{ x \} \cup \{ P \} \\
  \freenames{P|Q} & := & \freenames{P} \cup \freenames{Q} \\
  \freenames{\dropn{x}} & := & \{ x \}
\end{eqnarray*}

The bound names of a process, $\boundnames{P}$, are those names occurring in $P$
that are not free. For example, in $x?(y).0$, the name $x$ is free, while $y$ is bound.

\begin{mathpar}
  \inferrule* [lab=monoidal-laws] {} { P|Q \equiv Q|P \and P|0 \equiv P \and P|(Q|R) \equiv (P|Q)|R }
\end{mathpar}

\begin{mathpar}
  \inferrule* [lab=alpha-equivalence] {} { (x)P \equiv (y)P\{y/x\} \and y \not\in \freenames{P} }
\end{mathpar}

\begin{definition}
Then two processes, $P,Q$, are alpha-equivalent if $P = Q\{\vec{y}/\vec{x}\}$ for
some $\vec{x} \in \boundnames{Q},\vec{y} \in \boundnames{P}$, where $Q\{\vec{y}/\vec{x}\}$
denotes the capture-avoiding substitution of $\vec{y}$ for $\vec{x}$ in $Q$.
\end{definition}

\begin{definition}
  The {\em structural congruence} \cite{SangiorgiWalker} , $\equiv$,
  between processes is the least congruence containing
  alpha-equivalence, satisfying the abelian monoid laws
  (associativity, commutativity and $\pzero$ as identity) for parallel
  composition $|$ and for summation $+$.
\end{definition}

\subsection{Name equivalence}

We take name equivalence, written $\nameeq$, to be the smallest
equivalence relation generated by the following rules.

\begin{mathpar}
\inferrule*[lab=Quote-drop]
{ }
{ \quotep{@{x}} \nameeq x }

\inferrule*[lab=Struct-equiv]
{ P \scong Q }
{ \quotep{P} \nameeq \quotep{Q} }
\end{mathpar}

The astute reader will have noticed that the mutual recursion of names
and processes imposes a mutual recursion on alpha-equivalence and
structural equivalence via name-equivalence. Fortunately, all of this
works out pleasantly and we may calculate in the natural way, free of
concern. The reader interested in the details is referred to the
appendix \ref{appendix:rho_details}.

\subsection{Substitution}

We use $\Proc$ for the set of processes, $\QProc$ for the set of
names, and $\id{\{}\vec{y} / \vec{x} \id{\}}$ to denote partial maps,
$s : \QProc \rightarrow \QProc$. A map, $s$ lifts, uniquely, to a map
on process terms, $\widehat{s} : \Proc \rightarrow \Proc$ by the
following equations.

\begin{mathpar}
  (0) \psubstp{Q}{P} := 0 \\
  (R \juxtap S) \psubstp{Q}{P}
  :=    
  (R)\psubstp{Q}{P} \juxtap (S) \psubstp{Q}{P} \\
  (x?(y).R) \psubstp{Q}{P}    
  :=    
  (x)\substp{Q}{P} (z)\concat( (R \psubstn{z}{y}) \psubstp{Q}{P} ) \\
  (\lift{x}{R}) \psubstp{Q}{P}  
  :=
  \lift{(x)\substp{Q}{P}}{ R \psubstp{Q}{P} } \\
%   (\dropn{x})  \psubstp{Q}{P}       
%   := 
%   \left\{ 
%     \begin{array}{ccc} 
%       \dropn{\quotep{Q}} & & x \nameeq \quotep{P} \\
%       \dropn{x} & & otherwise \\
%     \end{array}
%   \right. 
  (\dropn{x})  \psubstp{Q}{P}       
  := 
  \left\{ 
    \begin{array}{ccc} 
      Q & & x \nameeq \quotep{P} \\
      \dropn{x} & & otherwise \\
    \end{array}
  \right.
\end{mathpar}
 

where

\begin{eqnarray}
  (x)\id{\{} \lpquote Q \rpquote / \lpquote P \rpquote \id{\}}            = 
  \left\{ 
    \begin{array}{ccc}
      \lpquote Q \rpquote & & x \nameeq \lpquote P \rpquote \\
      x & & otherwise \\
    \end{array}
  \right. \nonumber
\end{eqnarray}

and $z$ is chosen distinct from $\quotep{P}$, $\quotep{Q}$, the free
names in $Q$, and all the names in $R$. Our $\alpha$-equivalence will
be built in the standard way from this substitution.

\begin{remark}\label{rem:no_self_referential_names}
  One consequence of these definitions is that $\forall P. \quotep{P}
  \not\in \freenames{P}$.
\end{remark}

\subsection{ Dynamic quote: an example }

Anticipating something of what's to come, consider applying the
substitution, $\widehat{\id{\{}u / z \id{\}}}$, to the following pair
of processes, $\lift{w}{y!(z)}$ and $w[ \lpquote y!(z) \rpquote ]$.

\begin{eqnarray}
	\lift{w}{y!(z)}\widehat{\id{\{}u / z \id{\}}}
		& = &
		\lift{w}{y!(u)} \nonumber\\
	w[ \lpquote y!(z) \rpquote ] \widehat{ \id{\{}u / z \id{\}} }
		& = &
		w[ \lpquote y!(z) \rpquote ] \nonumber
\end{eqnarray}

Because the body of the process between quotes is impervious to
substitution, we get radically different answers. In fact, by
examining the first process in an input context,
e.g. $x?(z).\lift{w}{y!(z)}$, we see that the process under the lift
operator may be shaped by prefixed inputs binding a name inside it. In
this sense, the lift operator will be seen as a way to dynamically
construct processes before reifying them as names.

Finally equipped with these standard features we can present the
dynamics of the calculus.

\subsubsection{Operational semantics} 

Finally, we introduce the computational dynamics. What marks these
algebras as distinct from other more traditionally studied algebraic
structures, e.g. vector spaces or polynomial rings, is the manner in
which dynamics is captured. In traditional structures, dynamics is typically
expressed through morphisms between such structures, as in linear maps
between vector spaces or morphisms between rings. In algebras
associated with the semantics of computation, the dynamics is
expressed as part of the algebraic structure itself, through a
reduction reduction relation typically denoted by $\red$. Below, we
give a recursive presentation of this relation for the calculus used
in the encoding.

$\red \subseteq \pi \times \pi$
$\red : \pi \to \mathcal{P}(\pi)$

\begin{mathpar}
  \inferrule* [lab=Comm] { \textsf{match}( x_{src}, x_{trgt} ) } { x_{trgt}?(y)P \; | \; x_{src}!\langle {Q} \rangle \red P\{\quotep{Q}/y}\} }
  \and \\
  \inferrule* [lab=Par] {{P} \red {P}'} {{{P} | {Q}} \red {{P}' | {Q}}}
  \and
  \inferrule* [lab=Equiv]{{{P} \scong {P}'} \andalso {{P}' \red {Q}'} \andalso {{Q}' \scong {Q}}}{{P} \red {Q}}
\end{mathpar}

\begin{eqnarray*}
  match_{\equiv} (\quotep{P},\quotep{Q}) & := & P \equiv Q \\
  match_{\dagger}(\quotep{P},\quotep{Q}) & := & \forall R. P|Q \red^{*} R => R \red^{*} 0 \\
  match_{K}(\quotep{P},\quotep{Q}) & := & K \mbox{ for some context } K
\end{eqnarray*}

$u?(x)P | u!\langle Q \rangle \red P\{\quotep{Q}/x\}$

%We write $\wred$ for $\red^*$, and $P\red$ if $\exists Q $ such that $ P \red Q$.
We write $P\red$ if $\exists Q $ such that $ P \red Q$ and $P\not\red$, otherwise.

\section{Replication}

As mentioned before, it is known that replication (and hence
recursion) can be implemented in a higher-order process algebra
\cite{SangiorgiWalker}. As our first example of calculation with the
machinery thus far presented we give the construction explicitly in
the {\rhoc}.

\begin{eqnarray}
	D_{x} & := & \prefix{x}{y}{(\binpar{\outputp{x}{y}}{@{y}})} \nonumber\\
	\bangp_{x}{P} & := & \binpar{{x}!\langle{\binpar{D_{x}}{P}}\rangle}{D_{x}} \nonumber
\end{eqnarray}

\begin{eqnarray}
	\bangp_{x}{P} & & \nonumber\\
	=
	& {x}!\langle{(\prefix{x}{y}{(\outputp{x}{y} | @{y})) | P}}\rangle 
	      | \prefix{x}{y}{(\outputp{x}{y} | @{y})} & \nonumber\\
	\red
	& (\outputp{x}{y} | @{y})\substn{\quotep{(\prefix{x}{y}{(@{y} | \outputp{x}{y})) | P}}}{y} & \nonumber\\
	=
	& \outputp{x}{\quotep{(\prefix{x}{y}{(\outputp{x}{y} | @{y})) | P}}}
	  | {(\prefix{x}{y}{(\outputp{x}{y} | @{y})) | P}} & \nonumber\\
	\red
	& \ldots & \nonumber\\
	\red^*
	& P | P | \ldots & \nonumber
\end{eqnarray}

Of course, this encoding, as an implementation, runs away, unfolding
$\bangp{P}$ eagerly. A lazier and more implementable replication
operator, restricted to input-guarded processes, may be obtained as follows.

\begin{eqnarray}
\bangp{\prefix{u}{v}{P}} 
	:= 
	\binpar{\lift{x}{\prefix{u}{v}{(\binpar{D(x)}{P})}}}{D(x)} \nonumber
\end{eqnarray}

\begin{remark}
  Note that the lazier definition still does not deal with summation
  or mixed summation (i.e. sums over input and output). The reader is
  invited to construct definitions of replication that deal with these
  features. 

  Further, the definitions are parameterized in a name, $x$. Can you,
  gentle reader, make a definition that eliminates this parameter and
  guarantees no accidental interaction between the replication
  machinery and the process being replicated -- i.e. no accidental
  sharing of names used by the process to get its work done and the
  name(s) used by the replication to effect copying. This latter
  revision of the definition of replication is crucial to obtaining
  the expected identity $!!P \sim !P$.
\end{remark}

\begin{remark}\label{rem:paradoxical_combinator}
  The reader familiar with the lambda calculus will have noticed the
  similarity between $D$ and the paradoxical combinator.

  [Ed. note: the existence of this seems to suggest we have to be more
  restrictive on the set of processes and names we admit if we are to
  support no-cloning.]
\end{remark}

\subsubsection{Bisimulation}

The computational dynamics gives rise to another kind of equivalence,
the equivalence of computational behavior. As previously mentioned
this is typically captured \emph{via} some form of bisimulation.

% The notion we use in this paper is weak barbed bisimulation
% \cite{milner91polyadicpi}.

The notion we use in this paper is derived from weak barbed
bisimulation \cite{milner91polyadicpi}. 

\begin{definition}
An \emph{observation relation}, $\downarrow_{\mathcal N}$, over a set
of names, $\mathcal N$, is the smallest relation satisfying the rules
below.

\infrule[Out-barb]{y \in {\mathcal N}, \; x \nameeq y}
		  {\outputp{x}{v} \downarrow_{\mathcal N} x}
\infrule[Par-barb]{\mbox{$P\downarrow_{\mathcal N} x$ or $Q\downarrow_{\mathcal N} x$}}
		  {\binpar{P}{Q} \downarrow_{\mathcal N} x}

We write $P \Downarrow_{\mathcal N} x$ if there is $Q$ such that 
$P \wred Q$ and $Q \downarrow_{\mathcal N} x$.
\end{definition}

\begin{definition}
%\label{def.bbisim}
An  ${\mathcal N}$-\emph{barbed bisimulation} over a set of names, ${\mathcal N}$, is a symmetric binary relation 
${\mathcal S}_{\mathcal N}$ between agents such that $P\rel{S}_{\mathcal N}Q$ implies:
\begin{enumerate}
\item If $P \red P'$ then $Q \wred Q'$ and $P'\rel{S}_{\mathcal N} Q'$.
\item If $P\downarrow_{\mathcal N} x$, then $Q\Downarrow_{\mathcal N} x$.
\end{enumerate}
$P$ is ${\mathcal N}$-barbed bisimilar to $Q$, written
$P \wbbisim_{\mathcal N} Q$, if $P \rel{S}_{\mathcal N} Q$ for some ${\mathcal N}$-barbed bisimulation ${\mathcal S}_{\mathcal N}$.
\end{definition}

$\mathcal{R} \subseteq \pi \times \pi$

$P \mathcal{R} Q => \forall P'. P \red P' \Rightarrow \exists Q'. Q \red Q', P' \mathcal{R} Q'$

$P \vdash x \Rightarrow Q \vdash x$

\begin{mathpar}
  \inferrule*[lab=Out-barb]{x \nameeq y}{{y}!\langle{Q}\rangle \vdash x}
  \and
  \inferrule*[lab=Par-barb]{\mbox{$P\vdash x$ or $Q\vdash x$}}{\binpar{P}{Q} \vdash x}
\end{mathpar}

\subsubsection{Contexts}

One of the principle advantages of computational calculi like the
$\pi$-calculus is a well-defined notion of context,
contextual-equivalence and a correlation between
contextual-equivalence and notions of bisimulation. The notion of
context allows the decomposition of a process into (sub-)process and
its syntactic environment, its context. Thus, a context may be
thought of as a process with a ``hole'' (written $\Box$) in it. The
application of a context $M$ to a process $P$, written $M[P]$, is
tantamount to filling the hole in $M$ with $P$. In this paper we do
not need the full weight of this theory, but do make use of the notion
of context in the proof the main theorem. 

\begin{mathpar}
  \inferrule* [lab=summation] {} {{M_{M},M_{N}} \bc \Box \;|\; x.M_{A} \;|\; M_{M}+M_{N}}
  \and
  \inferrule* [lab=agent] {} {{M_{A}} \bc (\vec{x})M_{P} \;| \; \clift{P_0,\ldots,M_{P},\ldots,P_N}}
  \and \\
  \inferrule* [lab=process] {} {{M_{P}} \bc M_{N} \;| \;P|M_{P} }
\end{mathpar} 

\begin{mathpar}
  \inferrule* [lab=sychronization] {} {M_{N} \bc \Box \;|\; x?M_{F} \;|\; x!M_{C}}
  \and
  \inferrule* [lab=abstraction] {} {{M_{F}} \bc (x)M_{P} }
  \and
  \inferrule* [lab=concretion] {} {{M_{C}} \bc \langle M_{P} \rangle }
  \and \\
  \inferrule* [lab=process] {} {{M_{P}} \bc M_{N} \;| \;P|M_{P} }
\end{mathpar}

\begin{definition}[contextual application] Given a context $M$, and
  process $P$, we define the \emph{contextual application}, $M[P] :=
  M\{P/\Box\}$. That is, the contextual application of M to P is the
  substitution of $P$ for $\Box$ in $M$.
\end{definition}

$\meaningof{-} : L \to \mathcal{P}(\pi)$

\begin{mathpar}
  \inferrule* [lab=collection] {} {\meaningof{true} = \pi, \and \meaningof{~E} = \pi \setminus \meaningof{E}, \and \meaningof{E_{1} \& E_{2}} = \meaningof{E_{1}} \cap \meaningof{E_{2}}}
\end{mathpar}

\begin{mathpar}
  \inferrule* [lab=structure] {} {\meaningof{0} = \{ P \in \pi | P \equiv 0 \}, \and \\ \meaningof{E_1 | E_2} = \{ P \in \pi | P \equiv P_{1} | P_{2}, P_{1} \in \meaningof{E_{1}}, P_{2} \in \meaningof{E_2}\} }
\end{mathpar}

\begin{mathpar}
 \inferrule* [lab=behavior] {} {\meaningof{\langle a?b \rangle E} = \{ P \in \pi | P \equiv Q | u?(y)P', \\ \and \\\\ \and \\ \;\;\; u \in \meaningof{a}, \forall z.P'\{z/y\} \in \meaningof{E\{z/b\}}\}, \and \\ \meaningof{a!E} = \{ P \in \pi | P \equiv Q | x!\langle P' \rangle, x \in \meaningof{a} P' \in \meaningof{E}\} }
\end{mathpar}

\begin{mathpar}
 \inferrule* [lab=nominal] {} {\meaningof{\quotep{E}} = \{ \quotep{P} \in \quotep{\pi} | P \in \meaningof{E} \}, \and \meaningof{\quotep{P}} = \{ \quotep{Q} \in \quotep{\pi} | P \equiv Q \} \and \\ \meaningof{@\quotep{E}} = \{ P \in \pi | P \equiv @x, x \in \meaningof{E} \}}
\end{mathpar}

\begin{eqnarray*}
  \\
  \meaningof{-} : TS \to ST
\end{eqnarray*}

\begin{eqnarray*}
  \\
  L : TS \to ST
\end{eqnarray*}

\begin{eqnarray*}
  \\
  P \models E \iff P \in \meaningof{E}
\end{eqnarray*}

\begin{eqnarray*}
  P \approx_{L} Q \iff \forall E \in L. P \models E \iff Q \models E
\end{eqnarray*}

\begin{eqnarray*}
  P \approx_{K} Q
\end{eqnarray*}

\begin{eqnarray*}
  P \approx Q
\end{eqnarray*}

$\approx_{K} = \approx = \approx_{L}$

\subsubsection{Contextual duality}

Note that contexts extend the quotation operation to a family of
operations from processes to names. Given a context, $M$, we can
define a \emph{nominal context}, $\quotep{M}$ by $\quotep{M}[P] :=
\quotep{M[P]}$. To foreshadow what is to come we observe that these
operations enjoy a duality with processes very much like the duality
between vectors and maps from vectors to scalars.

Further, because the calculus is essentially higher-order, we have a
correspondence between contexts and processes. More specifically,
given a name $x$ and a context $M$ we can construct $M^{*}_{x}$ such
that 

\begin{mathpar}
  M^{*}_{x} | \lift{x}{P} \red M[P]
\end{mathpar}

namely,

\begin{mathpar}
  M^{*}_{x} := x?(u).M[\dropn{u}]
\end{mathpar}

The dependence of $M^{*}_{x}$ on a name makes it an abstraction, 

\begin{mathpar}
  M^{*} := (x)x?(u).M[\dropn{u}]
\end{mathpar}

\subsection{Additional notation}

It will sometimes be convenient to denote the process a name
quotes. We already have the notation $x = \quotep{P}$, but it will be
convenient to introduce an alternate notation, $\procn{x}$, when we
want to emphasize the connection to the use of the name. Note that, by
virtue of name equivalence, $\quotep{\procn{x}} \nameeq x$; so, the
notation is consistent with previous definitions.

Further, because names have structure it is possible to effect
substitutions on the basis of that structure. This means we need to
upgrade our notation for substitutions, which we accomplish by
adapting comprehension notation. Thus,

\begin{mathpar}
  P\{ y / x : x \in S \}
\end{mathpar}

is interpreted to mean the process derived from P by replacing (in a
capture-avoiding manner) each occurrence of $x$ in $S$ by $y$. For example,

\begin{mathpar}
  P\{ \quotep{\procn{x}|\procn{x}} / x : x \in \freenames{P} \}
\end{mathpar}

will replace each (occurrence) of a free name $x$ in $P$ by
$\quotep{\procn{x}|\procn{x}}$.

Also, we will avail ourselves of the notation $x^{L}$ and $x^{R}$ to
denote injections of a name into disjoint copies of the name
space. There are numerous ways to accomplish this. One example can be
found in \cite{MeredithR05}. This notation overloads to vectors of
names: $\vec{x}^{\pi} := (x_{i}^{\pi} \; : \; 0 \leq i < |\vec{x}| )$ where $\pi \in \{L,R\}$.

We also use $P^{\Box} := P|\Box$.

In \cite{MeredithR05} an interpretation of the new operator is
given. It turns out that there are several possible interpretations
all enjoying the requisite algebraic properties of the operator (see
\cite{milner91polyadicpi}). We will therefore make liberal use of
$(\nu\; \vec{x})P$.

% subsection the_syntax_and_semantics_of_the_notation_system (end)   

\input{qm2pi.qmops} 

\input{qm2pi.sterngerlach} 

\input{qm2pi.metric} 

% section concurrent_process_calculi (end)

%\input{qm2pi.proofsketch}

% section proof sketch (end)

%\input{qm2pi.slviaknots} 

% section spatial logic via knots (end)

\input{qm2pi.conclusion}

% section conclusion (end)

%\input{qm2pi.dtcodes} 

% section wiring algorithm (end)

\input{qm2pi.ack} 

% section acknowledgments (end)

\newpage


\bibliographystyle{plain}   
\bibliography{../../biblios/main.bib}

\input{qm2pi.rhodetails}

\end{document}

 

% section concurrent_process_calculi (end)

%\documentclass[12pt]{llncs}
%\documentclass{jktr}

\usepackage[pdftex]{hyperref}                   
\usepackage {listings}
\usepackage {mathpartir}
\usepackage{bcprules}
%\usepackage{listings}
                       
\usepackage{graphicx} 
%\usepackage[margins=2.5cm,nohead,nofoot]{geometry}
%\usepackage{geometry}
\usepackage{amsfonts}
\usepackage{amstext}
\usepackage{latexsym}
\usepackage{amssymb}
\usepackage{color}


%\include{myPreamble}
\include{qm2pi.local} 

%\ifpdf
%\usepackage[pdftex]{graphicx}
%\else
%\usepackage{graphicx}
%\fi

 % \ifpdf
%  \usepackage{pdfsync}
%  \if


%\title{Brief Article}
%\author{David F. Snyder}
%\author{L.G. Meredith}

%\address{Dept. of Math., Texas State University--San Marcos, San Marcos, TX 78666}
       
\pagestyle{empty}


\begin{document}

\lstset{language=[Objective]Caml,frame=shadowbox}

\input{qm2pi.front}

% section front matter (end)

\input{qm2pi.intro} 
 
% section introduction (end)

% \input{qm2pi.knotations} 

% section notation (end)

\input{qm2pi.process.calculi} 

% section concurrent_process_calculi_and_spatial_logics_ (end)
    
%\input{qm2pi.knots2pi} 

%\input{qm2pi.trefoil} 

%\input{qm2pi.mainthm} 

% subsection basic_interpretation (end)

%\input{qm2pi.rho.presentation} 
\subsection{The syntax and semantics of the notation system}\label{sub:the_syntax_and_semantics_of_the_notation_system} % (fold)

We now summarize a technical presentation of the calculus that
embodies our theory of dynamics. The typical presentation of such a
calculus follows the style of giving generators and relations on
them. The grammar, below, describing term constructors, freely
generates the set of processes, $\Proc$. This set is then quotiented
by a relation known as structural congruence and it is over this set
that the notion of dynamics is expressed. This presentation is
essentially that of \cite{MeredithR05} with the addition of
polyadicity and summation. For readability we have relegated some of
the technical subtleties to an appendix.

\subsubsection{Process grammar}\label{subsub:process_grammar}

\begin{mathpar}
  \inferrule* [lab=synchronization] {} {{M} \bc \pzero \;|\; x?F \;|\; x!C }
  \and
  \inferrule* [lab=abstraction] {} {{F} \bc (x)P}
  \and
  \inferrule* [lab=concretion] {} {{C} \bc \langle Q \rangle}
  \and
  \inferrule* [lab=process] {} {{P,Q} \bc M \;| \;P|Q \;|\; @{x}}
  \and
  \inferrule* [lab=name] {} {{x} \bc \quotep{P}}
\end{mathpar} 

Note that $\vec{x}$ (resp. $\vec{P}$) denotes a vector of names
(resp. processes) of length $|\vec{x}|$ (resp. $|\vec{P}|$). We adopt
the following useful abbreviations.

\begin{mathpar}
   x?(\vec{y}).P := x.(\vec{y})P \and  x\clift{\vec{P}} := x.\clift{\vec{P}}
   \and x!(y) := \lift{x}{\dropn{y}}
   \and \Pi_{i=0}^{n-1}P_i := P_0 | \ldots | P_{n-1}
\end{mathpar}

\subsubsection{Structural congruence}

\paragraph{Free and bound names and alpha-equivalence.} At the
core of structural equivalence is alpha-equivalence which identifies
process that are the same up to a change of variable. Formally, we
recognize the distinction between free and bound names. The free names
of a process, $\freenames{P}$, may be calculated recursively as
follows:

\begin{mathpar}
\freenames{\pzero} := \emptyset
  \and \\
  \freenames{x?(y).P} := \{ x \} \cup (\freenames{P} \setminus \{ y \})
  \and 
  \freenames{x!\langle P \rangle} := \{ x \} \cup \{ P \} 
  \and \\
  \freenames{P|Q} := \freenames{P} \cup \freenames{Q}
  \and \\
  \freenames{@{x}} := \{ x \}
\end{mathpar}

$\pi$
$\quotep{\pi}$

$\freenames{-} : \pi \to \mathcal{P}(\quotep{\pi})$

\begin{eqnarray*}
  \freenames{\pzero} & := & \emptyset \\
  \freenames{x?(y).P} & := & \{ x \} \cup (\freenames{P} \setminus \{ y \}) \\
  \freenames{x!\langle P \rangle} & := & \{ x \} \cup \{ P \} \\
  \freenames{P|Q} & := & \freenames{P} \cup \freenames{Q} \\
  \freenames{\dropn{x}} & := & \{ x \}
\end{eqnarray*}

The bound names of a process, $\boundnames{P}$, are those names occurring in $P$
that are not free. For example, in $x?(y).0$, the name $x$ is free, while $y$ is bound.

\begin{mathpar}
  \inferrule* [lab=monoidal-laws] {} { P|Q \equiv Q|P \and P|0 \equiv P \and P|(Q|R) \equiv (P|Q)|R }
\end{mathpar}

\begin{mathpar}
  \inferrule* [lab=alpha-equivalence] {} { (x)P \equiv (y)P\{y/x\} \and y \not\in \freenames{P} }
\end{mathpar}

\begin{definition}
Then two processes, $P,Q$, are alpha-equivalent if $P = Q\{\vec{y}/\vec{x}\}$ for
some $\vec{x} \in \boundnames{Q},\vec{y} \in \boundnames{P}$, where $Q\{\vec{y}/\vec{x}\}$
denotes the capture-avoiding substitution of $\vec{y}$ for $\vec{x}$ in $Q$.
\end{definition}

\begin{definition}
  The {\em structural congruence} \cite{SangiorgiWalker} , $\equiv$,
  between processes is the least congruence containing
  alpha-equivalence, satisfying the abelian monoid laws
  (associativity, commutativity and $\pzero$ as identity) for parallel
  composition $|$ and for summation $+$.
\end{definition}

\subsection{Name equivalence}

We take name equivalence, written $\nameeq$, to be the smallest
equivalence relation generated by the following rules.

\begin{mathpar}
\inferrule*[lab=Quote-drop]
{ }
{ \quotep{@{x}} \nameeq x }

\inferrule*[lab=Struct-equiv]
{ P \scong Q }
{ \quotep{P} \nameeq \quotep{Q} }
\end{mathpar}

The astute reader will have noticed that the mutual recursion of names
and processes imposes a mutual recursion on alpha-equivalence and
structural equivalence via name-equivalence. Fortunately, all of this
works out pleasantly and we may calculate in the natural way, free of
concern. The reader interested in the details is referred to the
appendix \ref{appendix:rho_details}.

\subsection{Substitution}

We use $\Proc$ for the set of processes, $\QProc$ for the set of
names, and $\id{\{}\vec{y} / \vec{x} \id{\}}$ to denote partial maps,
$s : \QProc \rightarrow \QProc$. A map, $s$ lifts, uniquely, to a map
on process terms, $\widehat{s} : \Proc \rightarrow \Proc$ by the
following equations.

\begin{mathpar}
  (0) \psubstp{Q}{P} := 0 \\
  (R \juxtap S) \psubstp{Q}{P}
  :=    
  (R)\psubstp{Q}{P} \juxtap (S) \psubstp{Q}{P} \\
  (x?(y).R) \psubstp{Q}{P}    
  :=    
  (x)\substp{Q}{P} (z)\concat( (R \psubstn{z}{y}) \psubstp{Q}{P} ) \\
  (\lift{x}{R}) \psubstp{Q}{P}  
  :=
  \lift{(x)\substp{Q}{P}}{ R \psubstp{Q}{P} } \\
%   (\dropn{x})  \psubstp{Q}{P}       
%   := 
%   \left\{ 
%     \begin{array}{ccc} 
%       \dropn{\quotep{Q}} & & x \nameeq \quotep{P} \\
%       \dropn{x} & & otherwise \\
%     \end{array}
%   \right. 
  (\dropn{x})  \psubstp{Q}{P}       
  := 
  \left\{ 
    \begin{array}{ccc} 
      Q & & x \nameeq \quotep{P} \\
      \dropn{x} & & otherwise \\
    \end{array}
  \right.
\end{mathpar}
 

where

\begin{eqnarray}
  (x)\id{\{} \lpquote Q \rpquote / \lpquote P \rpquote \id{\}}            = 
  \left\{ 
    \begin{array}{ccc}
      \lpquote Q \rpquote & & x \nameeq \lpquote P \rpquote \\
      x & & otherwise \\
    \end{array}
  \right. \nonumber
\end{eqnarray}

and $z$ is chosen distinct from $\quotep{P}$, $\quotep{Q}$, the free
names in $Q$, and all the names in $R$. Our $\alpha$-equivalence will
be built in the standard way from this substitution.

\begin{remark}\label{rem:no_self_referential_names}
  One consequence of these definitions is that $\forall P. \quotep{P}
  \not\in \freenames{P}$.
\end{remark}

\subsection{ Dynamic quote: an example }

Anticipating something of what's to come, consider applying the
substitution, $\widehat{\id{\{}u / z \id{\}}}$, to the following pair
of processes, $\lift{w}{y!(z)}$ and $w[ \lpquote y!(z) \rpquote ]$.

\begin{eqnarray}
	\lift{w}{y!(z)}\widehat{\id{\{}u / z \id{\}}}
		& = &
		\lift{w}{y!(u)} \nonumber\\
	w[ \lpquote y!(z) \rpquote ] \widehat{ \id{\{}u / z \id{\}} }
		& = &
		w[ \lpquote y!(z) \rpquote ] \nonumber
\end{eqnarray}

Because the body of the process between quotes is impervious to
substitution, we get radically different answers. In fact, by
examining the first process in an input context,
e.g. $x?(z).\lift{w}{y!(z)}$, we see that the process under the lift
operator may be shaped by prefixed inputs binding a name inside it. In
this sense, the lift operator will be seen as a way to dynamically
construct processes before reifying them as names.

Finally equipped with these standard features we can present the
dynamics of the calculus.

\subsubsection{Operational semantics} 

Finally, we introduce the computational dynamics. What marks these
algebras as distinct from other more traditionally studied algebraic
structures, e.g. vector spaces or polynomial rings, is the manner in
which dynamics is captured. In traditional structures, dynamics is typically
expressed through morphisms between such structures, as in linear maps
between vector spaces or morphisms between rings. In algebras
associated with the semantics of computation, the dynamics is
expressed as part of the algebraic structure itself, through a
reduction reduction relation typically denoted by $\red$. Below, we
give a recursive presentation of this relation for the calculus used
in the encoding.

$\red \subseteq \pi \times \pi$
$\red : \pi \to \mathcal{P}(\pi)$

\begin{mathpar}
  \inferrule* [lab=Comm] { \textsf{match}( x_{src}, x_{trgt} ) } { x_{trgt}?(y)P \; | \; x_{src}!\langle {Q} \rangle \red P\{\quotep{Q}/y}\} }
  \and \\
  \inferrule* [lab=Par] {{P} \red {P}'} {{{P} | {Q}} \red {{P}' | {Q}}}
  \and
  \inferrule* [lab=Equiv]{{{P} \scong {P}'} \andalso {{P}' \red {Q}'} \andalso {{Q}' \scong {Q}}}{{P} \red {Q}}
\end{mathpar}

\begin{eqnarray*}
  match_{\equiv} (\quotep{P},\quotep{Q}) & := & P \equiv Q \\
  match_{\dagger}(\quotep{P},\quotep{Q}) & := & \forall R. P|Q \red^{*} R => R \red^{*} 0 \\
  match_{K}(\quotep{P},\quotep{Q}) & := & K \mbox{ for some context } K
\end{eqnarray*}

$u?(x)P | u!\langle Q \rangle \red P\{\quotep{Q}/x\}$

%We write $\wred$ for $\red^*$, and $P\red$ if $\exists Q $ such that $ P \red Q$.
We write $P\red$ if $\exists Q $ such that $ P \red Q$ and $P\not\red$, otherwise.

\section{Replication}

As mentioned before, it is known that replication (and hence
recursion) can be implemented in a higher-order process algebra
\cite{SangiorgiWalker}. As our first example of calculation with the
machinery thus far presented we give the construction explicitly in
the {\rhoc}.

\begin{eqnarray}
	D_{x} & := & \prefix{x}{y}{(\binpar{\outputp{x}{y}}{@{y}})} \nonumber\\
	\bangp_{x}{P} & := & \binpar{{x}!\langle{\binpar{D_{x}}{P}}\rangle}{D_{x}} \nonumber
\end{eqnarray}

\begin{eqnarray}
	\bangp_{x}{P} & & \nonumber\\
	=
	& {x}!\langle{(\prefix{x}{y}{(\outputp{x}{y} | @{y})) | P}}\rangle 
	      | \prefix{x}{y}{(\outputp{x}{y} | @{y})} & \nonumber\\
	\red
	& (\outputp{x}{y} | @{y})\substn{\quotep{(\prefix{x}{y}{(@{y} | \outputp{x}{y})) | P}}}{y} & \nonumber\\
	=
	& \outputp{x}{\quotep{(\prefix{x}{y}{(\outputp{x}{y} | @{y})) | P}}}
	  | {(\prefix{x}{y}{(\outputp{x}{y} | @{y})) | P}} & \nonumber\\
	\red
	& \ldots & \nonumber\\
	\red^*
	& P | P | \ldots & \nonumber
\end{eqnarray}

Of course, this encoding, as an implementation, runs away, unfolding
$\bangp{P}$ eagerly. A lazier and more implementable replication
operator, restricted to input-guarded processes, may be obtained as follows.

\begin{eqnarray}
\bangp{\prefix{u}{v}{P}} 
	:= 
	\binpar{\lift{x}{\prefix{u}{v}{(\binpar{D(x)}{P})}}}{D(x)} \nonumber
\end{eqnarray}

\begin{remark}
  Note that the lazier definition still does not deal with summation
  or mixed summation (i.e. sums over input and output). The reader is
  invited to construct definitions of replication that deal with these
  features. 

  Further, the definitions are parameterized in a name, $x$. Can you,
  gentle reader, make a definition that eliminates this parameter and
  guarantees no accidental interaction between the replication
  machinery and the process being replicated -- i.e. no accidental
  sharing of names used by the process to get its work done and the
  name(s) used by the replication to effect copying. This latter
  revision of the definition of replication is crucial to obtaining
  the expected identity $!!P \sim !P$.
\end{remark}

\begin{remark}\label{rem:paradoxical_combinator}
  The reader familiar with the lambda calculus will have noticed the
  similarity between $D$ and the paradoxical combinator.

  [Ed. note: the existence of this seems to suggest we have to be more
  restrictive on the set of processes and names we admit if we are to
  support no-cloning.]
\end{remark}

\subsubsection{Bisimulation}

The computational dynamics gives rise to another kind of equivalence,
the equivalence of computational behavior. As previously mentioned
this is typically captured \emph{via} some form of bisimulation.

% The notion we use in this paper is weak barbed bisimulation
% \cite{milner91polyadicpi}.

The notion we use in this paper is derived from weak barbed
bisimulation \cite{milner91polyadicpi}. 

\begin{definition}
An \emph{observation relation}, $\downarrow_{\mathcal N}$, over a set
of names, $\mathcal N$, is the smallest relation satisfying the rules
below.

\infrule[Out-barb]{y \in {\mathcal N}, \; x \nameeq y}
		  {\outputp{x}{v} \downarrow_{\mathcal N} x}
\infrule[Par-barb]{\mbox{$P\downarrow_{\mathcal N} x$ or $Q\downarrow_{\mathcal N} x$}}
		  {\binpar{P}{Q} \downarrow_{\mathcal N} x}

We write $P \Downarrow_{\mathcal N} x$ if there is $Q$ such that 
$P \wred Q$ and $Q \downarrow_{\mathcal N} x$.
\end{definition}

\begin{definition}
%\label{def.bbisim}
An  ${\mathcal N}$-\emph{barbed bisimulation} over a set of names, ${\mathcal N}$, is a symmetric binary relation 
${\mathcal S}_{\mathcal N}$ between agents such that $P\rel{S}_{\mathcal N}Q$ implies:
\begin{enumerate}
\item If $P \red P'$ then $Q \wred Q'$ and $P'\rel{S}_{\mathcal N} Q'$.
\item If $P\downarrow_{\mathcal N} x$, then $Q\Downarrow_{\mathcal N} x$.
\end{enumerate}
$P$ is ${\mathcal N}$-barbed bisimilar to $Q$, written
$P \wbbisim_{\mathcal N} Q$, if $P \rel{S}_{\mathcal N} Q$ for some ${\mathcal N}$-barbed bisimulation ${\mathcal S}_{\mathcal N}$.
\end{definition}

$\mathcal{R} \subseteq \pi \times \pi$

$P \mathcal{R} Q => \forall P'. P \red P' \Rightarrow \exists Q'. Q \red Q', P' \mathcal{R} Q'$

$P \vdash x \Rightarrow Q \vdash x$

\begin{mathpar}
  \inferrule*[lab=Out-barb]{x \nameeq y}{{y}!\langle{Q}\rangle \vdash x}
  \and
  \inferrule*[lab=Par-barb]{\mbox{$P\vdash x$ or $Q\vdash x$}}{\binpar{P}{Q} \vdash x}
\end{mathpar}

\subsubsection{Contexts}

One of the principle advantages of computational calculi like the
$\pi$-calculus is a well-defined notion of context,
contextual-equivalence and a correlation between
contextual-equivalence and notions of bisimulation. The notion of
context allows the decomposition of a process into (sub-)process and
its syntactic environment, its context. Thus, a context may be
thought of as a process with a ``hole'' (written $\Box$) in it. The
application of a context $M$ to a process $P$, written $M[P]$, is
tantamount to filling the hole in $M$ with $P$. In this paper we do
not need the full weight of this theory, but do make use of the notion
of context in the proof the main theorem. 

\begin{mathpar}
  \inferrule* [lab=summation] {} {{M_{M},M_{N}} \bc \Box \;|\; x.M_{A} \;|\; M_{M}+M_{N}}
  \and
  \inferrule* [lab=agent] {} {{M_{A}} \bc (\vec{x})M_{P} \;| \; \clift{P_0,\ldots,M_{P},\ldots,P_N}}
  \and \\
  \inferrule* [lab=process] {} {{M_{P}} \bc M_{N} \;| \;P|M_{P} }
\end{mathpar} 

\begin{mathpar}
  \inferrule* [lab=sychronization] {} {M_{N} \bc \Box \;|\; x?M_{F} \;|\; x!M_{C}}
  \and
  \inferrule* [lab=abstraction] {} {{M_{F}} \bc (x)M_{P} }
  \and
  \inferrule* [lab=concretion] {} {{M_{C}} \bc \langle M_{P} \rangle }
  \and \\
  \inferrule* [lab=process] {} {{M_{P}} \bc M_{N} \;| \;P|M_{P} }
\end{mathpar}

\begin{definition}[contextual application] Given a context $M$, and
  process $P$, we define the \emph{contextual application}, $M[P] :=
  M\{P/\Box\}$. That is, the contextual application of M to P is the
  substitution of $P$ for $\Box$ in $M$.
\end{definition}

$\meaningof{-} : L \to \mathcal{P}(\pi)$

\begin{mathpar}
  \inferrule* [lab=collection] {} {\meaningof{true} = \pi, \and \meaningof{~E} = \pi \setminus \meaningof{E}, \and \meaningof{E_{1} \& E_{2}} = \meaningof{E_{1}} \cap \meaningof{E_{2}}}
\end{mathpar}

\begin{mathpar}
  \inferrule* [lab=structure] {} {\meaningof{0} = \{ P \in \pi | P \equiv 0 \}, \and \\ \meaningof{E_1 | E_2} = \{ P \in \pi | P \equiv P_{1} | P_{2}, P_{1} \in \meaningof{E_{1}}, P_{2} \in \meaningof{E_2}\} }
\end{mathpar}

\begin{mathpar}
 \inferrule* [lab=behavior] {} {\meaningof{\langle a?b \rangle E} = \{ P \in \pi | P \equiv Q | u?(y)P', \\ \and \\\\ \and \\ \;\;\; u \in \meaningof{a}, \forall z.P'\{z/y\} \in \meaningof{E\{z/b\}}\}, \and \\ \meaningof{a!E} = \{ P \in \pi | P \equiv Q | x!\langle P' \rangle, x \in \meaningof{a} P' \in \meaningof{E}\} }
\end{mathpar}

\begin{mathpar}
 \inferrule* [lab=nominal] {} {\meaningof{\quotep{E}} = \{ \quotep{P} \in \quotep{\pi} | P \in \meaningof{E} \}, \and \meaningof{\quotep{P}} = \{ \quotep{Q} \in \quotep{\pi} | P \equiv Q \} \and \\ \meaningof{@\quotep{E}} = \{ P \in \pi | P \equiv @x, x \in \meaningof{E} \}}
\end{mathpar}

\begin{eqnarray*}
  \\
  \meaningof{-} : TS \to ST
\end{eqnarray*}

\begin{eqnarray*}
  \\
  L : TS \to ST
\end{eqnarray*}

\begin{eqnarray*}
  \\
  P \models E \iff P \in \meaningof{E}
\end{eqnarray*}

\begin{eqnarray*}
  P \approx_{L} Q \iff \forall E \in L. P \models E \iff Q \models E
\end{eqnarray*}

\begin{eqnarray*}
  P \approx_{K} Q
\end{eqnarray*}

\begin{eqnarray*}
  P \approx Q
\end{eqnarray*}

$\approx_{K} = \approx = \approx_{L}$

\subsubsection{Contextual duality}

Note that contexts extend the quotation operation to a family of
operations from processes to names. Given a context, $M$, we can
define a \emph{nominal context}, $\quotep{M}$ by $\quotep{M}[P] :=
\quotep{M[P]}$. To foreshadow what is to come we observe that these
operations enjoy a duality with processes very much like the duality
between vectors and maps from vectors to scalars.

Further, because the calculus is essentially higher-order, we have a
correspondence between contexts and processes. More specifically,
given a name $x$ and a context $M$ we can construct $M^{*}_{x}$ such
that 

\begin{mathpar}
  M^{*}_{x} | \lift{x}{P} \red M[P]
\end{mathpar}

namely,

\begin{mathpar}
  M^{*}_{x} := x?(u).M[\dropn{u}]
\end{mathpar}

The dependence of $M^{*}_{x}$ on a name makes it an abstraction, 

\begin{mathpar}
  M^{*} := (x)x?(u).M[\dropn{u}]
\end{mathpar}

\subsection{Additional notation}

It will sometimes be convenient to denote the process a name
quotes. We already have the notation $x = \quotep{P}$, but it will be
convenient to introduce an alternate notation, $\procn{x}$, when we
want to emphasize the connection to the use of the name. Note that, by
virtue of name equivalence, $\quotep{\procn{x}} \nameeq x$; so, the
notation is consistent with previous definitions.

Further, because names have structure it is possible to effect
substitutions on the basis of that structure. This means we need to
upgrade our notation for substitutions, which we accomplish by
adapting comprehension notation. Thus,

\begin{mathpar}
  P\{ y / x : x \in S \}
\end{mathpar}

is interpreted to mean the process derived from P by replacing (in a
capture-avoiding manner) each occurrence of $x$ in $S$ by $y$. For example,

\begin{mathpar}
  P\{ \quotep{\procn{x}|\procn{x}} / x : x \in \freenames{P} \}
\end{mathpar}

will replace each (occurrence) of a free name $x$ in $P$ by
$\quotep{\procn{x}|\procn{x}}$.

Also, we will avail ourselves of the notation $x^{L}$ and $x^{R}$ to
denote injections of a name into disjoint copies of the name
space. There are numerous ways to accomplish this. One example can be
found in \cite{MeredithR05}. This notation overloads to vectors of
names: $\vec{x}^{\pi} := (x_{i}^{\pi} \; : \; 0 \leq i < |\vec{x}| )$ where $\pi \in \{L,R\}$.

We also use $P^{\Box} := P|\Box$.

In \cite{MeredithR05} an interpretation of the new operator is
given. It turns out that there are several possible interpretations
all enjoying the requisite algebraic properties of the operator (see
\cite{milner91polyadicpi}). We will therefore make liberal use of
$(\nu\; \vec{x})P$.

% subsection the_syntax_and_semantics_of_the_notation_system (end)   

\input{qm2pi.qmops} 

\input{qm2pi.sterngerlach} 

\input{qm2pi.metric} 

% section concurrent_process_calculi (end)

%\input{qm2pi.proofsketch}

% section proof sketch (end)

%\input{qm2pi.slviaknots} 

% section spatial logic via knots (end)

\input{qm2pi.conclusion}

% section conclusion (end)

%\input{qm2pi.dtcodes} 

% section wiring algorithm (end)

\input{qm2pi.ack} 

% section acknowledgments (end)

\newpage


\bibliographystyle{plain}   
\bibliography{../../biblios/main.bib}

\input{qm2pi.rhodetails}

\end{document}



% section proof sketch (end)

%\section{Unlikely characters: spatial logic for
  knots}\label{sub:characteristic_formulae} % (fold)

Associated to the mobile process calculi are a family of logics known
as the Hennessy-Milner logics. These logics typically enjoy a
semantics interpreting formulae as sets of processes that when
factored through the encoding outlined above allows an identification
of classes of knots with logical formulae. In the context of this
encoding the sub-family known as the spatial logics \cite{CairesC03}
\cite{CairesC04} \cite{Caires04} are of particular interest providing
several important features for expressing and reasoning about
properties (i.e. classes) of knots. We hint here at how this may be done.

%\begin{description}
%\item [structural connectives] 
\subsubsection{Structural connectives} The spatial logics enjoy
structural connectives corresponding, at the logical level, to the
parallel composition ($P | Q$) and new name ($(\nu \; x)P$)
connectives for processes. As illustrated in the examples below, these
connectives are extremely expressive given the shape of our encoding.
%\item [decideable satisfaction]

\subsubsection{Decideable satisfaction}
In \cite{Caires04} the satisfaction relation is shown to be decideable
for a rich class of processes. It further turns out that the image of
the our encoding is a proper subset of that class. This result
provides the basis for an algorithm by which to search for knots
enjoying a given property.
%\item [characteristic formulae]

\subsubsection{Characteristic formulae}
In the same paper \cite{Caires04} , Caires presents a means of calculating
characteristic formulae, selecting equivalence classes of processes
up to a pre--specified depth limit on the support set of names. Composed with our
encoding, this characteristic formula can be used to select
characteristic formulae for knots.
%\end{description}

\subsubsection{Spatial logic formulae}

The grammar below (segmented for comprehension) summarizes the syntax
of spatial logic formulae. We employ illustrative examples in the
sequel to provide an intuitive understanding of their meaning
referring the reader to \cite{Caires04} for a more detailed explication
of the semantics.

\begin{mathpar}
  \inferrule* [lab=boolean] {} {{A,B} \bc T \;|\; \neg A \;|\; A \wedge B \;|\; \eta = \eta'}
  \and
  \inferrule* [lab=spatial] {} {|\; \pzero \;|\; A | B \;|\; x \text{\textregistered} A \;|\; \forall x . A \;|\;  H x . A}
  \and
  \inferrule* [lab=behavioral] {} {|\; \alpha . A}
  \and 
  \inferrule* [lab=recursion] {} {|\; X(\vec{u}) \;|\; \mu X(\vec{u}) . A}
  \and
  \inferrule* [lab=action] {} {\alpha \bc \langle x?(\vec{y}) \rangle \;|\; \langle x!(\vec{y}) \rangle \;|\; \langle \tau \rangle}
  \and 
  \inferrule* [lab=name] {} {\eta \bc x \;|\; \tau}
\end{mathpar} 

% subsection characteristic_formulae (end)   	 

\subsection{Example formulae}\label{sub:example_formulae_} % (fold)

\subsubsection{Crossing as formula.}
% 
% \begin{align*}
%   \frac{d}{dx} \sin x &= \cos x 
%   & \frac{d}{dx} e^x &= e^x \\
%   \frac{d}{dx} \cos x &= - \sin x 
%   & \frac{d}{dx} \log x &= \frac{1}{x} \\
% \end{align*} 

\begin{align*}
 \mu C(x_{0},x_{1},y_{0},y_{1},u).&(\langle x_{0}?(z) \rangle(\langle u! \rangle\langle y_{1}!z \rangle C(x_{0},x_{1},y_{0},y_{1},u)) & \\
  & \wedge \langle y_{1}?(z) \rangle (\langle u! \rangle \langle x_{0}!z \rangle C(x_{0},x_{1},y_{0},y_{1},u)) & \\
  & \wedge \langle x_{1}?(z) \rangle (\langle u? \rangle \langle y_{0}!z \rangle C(x_{0},x_{1},y_{0},y_{1},u)) & \\
  & \wedge \langle y_{0}?(z) \rangle (\langle u? \rangle \langle x_{1}!z \rangle C(x_{0},x_{1},y_{0},y_{1},u))) &
\end{align*}

The lexicographical similarity between the shape of this formulae and
the shape of definition of the process representing a crossing reveals
the intuitive meaning of this formulae. It describes the capabilities
of a process that has the right to represent a crossing. For example
it picks out processes that may perform an input on the port $x_0$ in
its initial menu of capabilities. What differentiates the formula
from the process, however, is that the crossing process is the
smallest candidate to satisfy the formula. Infinitely many other
processes -- with internal behavior hidden behind this interface, so
to speak -- also satisfy this formula. Even this simple formula,
then, can be seen to open a new view onto knots, providing a
computational interpretation of \emph{virtual} knots.

Note that this formula is derived by hand. A similar formula can be
derived by employing Caires' calculation of characteristic formula
\cite{Caires04} to the process representing a crossing. In light of
this discussion, we let
$\meaningof{C}_{\phi}(x0,x1,y0,y1,u)$ denote a formula specifying the
dynamics we wish to capture of a crossing. To guarantee we preserve
the shape of the interface and minimal semantics we demand that
$\meaningof{C}_{\phi}(x0,x1,y0,y1,u) \Rightarrow
\textbf{C}(x0,x1,y0,y1,u)$ where $\textbf{C}(x0,x1,y0,y1,u)$ denotes
the formula above.
                            
\subsubsection{Crossing number constraints.}
The moral content of the context lemma (Lemma \ref{context}) is that the notion of
``locality'' in the Reidemeister moves is effectively captured by the
parallel composition operator of the process calculus. This intuition
extends through the logic. Given a formula,
$\meaningof{C}_{\phi}(x0,x1,y0,y1,u)$, we can use the structural
connectives to specify constraints on crossing numbers, such as at
least $n$ crossings, or exactly $n$ crossings.
\begin{mathpar}
  \inferrule* [lab=at-least-n] {} { K^{\geq n}_{\phi}(\vec{xs},\vec{ys}) := \Pi_{i=0}^{n-1} Hu . \meaningof{C}_{\phi}(xs_i,ys_i,u) | T }
  \and 
  \inferrule* [lab=exactly-n] {} { K^{= n}_{\phi}(\vec{xs},\vec{ys}) := \Pi_{i=0}^{n-1} Hu . \meaningof{C}_{\phi}(xs_i,ys_i,u) | \neg (\forall x_0,y_0,x_1,y_1,u . \meaningof{C}_{\phi}(x_0,y_0,x_1,y_1,u) | T) }
\end{mathpar}

To round out this section, recall that the encoding of an $n$-crossing
knot decomposes into a parallel composition of $n$ \emph{copies} of a
crossing process together with a wiring harness. To specify different
knot classes with the same crossing number amounts to specifying
logical constraints on the wiring harness. In the interest of space,
we defer examples to a forthcoming paper. Suffice it to say that both
the conditions ``alternating knot'' and ``contains the tangle
corresponding to 5/3'' are expressible. For example, it is possible to
calculate the characteristic formula of a process corresponding to the
tangle 5/3 and conjoin it into the classifying formula via the
composition connective of the logic.

Finally, we wish to observe that it is entirely within reason to
contemplate a more domain-specific version of spatial logic tailored
to the shape of processes in the image of the encoding. Such a
domain-specific logic would have a better claim to the title formal
language of knot properties.

% subsection example_formulae_ (end)

% section knots_as_processes (end) 

% section spatial logic via knots (end)

\section{Conclusions and future work}

\paragraph{Testing physical space}
You, gentle reader, may wonder why of all the theorems to be proved
given this set up we pick the one above. In some sense it's hardly
central to quantum mechanics. We see it as central in the sense that
it firmly establishes a notion of physical space arising from a notion
of the equivalence of behavior. Relating bisimulation to a metric is a
big step forward, but one is faced with interpreting the relationship
of that metric space to something more physical. Quantum mechanical
notions of ``physical'' space are still far from intuitive, but by
relating this idea of distance as testing to calculations that predict
physical circumstances we are making a not insignificant step forward
toward an understanding of the physical space we inhabit as
essentially dynamic.

\paragraph{Effectivity and simulation}
One of the observations we have yet to make is that the entire program
spelled out here is effective. We have built various interpreters for
the reflective calculus at work in this interpretation. In principle,
then, we can simulate quantum mechanics on a computer. The place where
the simulation may lose fidelity is the infinitely branching summation
for the annihilator.

In this connection i also want to point out that the evaluation style
calculation of the inner product puts the non-determinism of the
summation right at the heart of measurement. This suggests that
Milner's original reduction-based formulation of the dynamics of his
calculi in terms of sums was not just notationally suggestive of a
notion of measure-and-continue but captured some significant part of
the physics.

\paragraph{Quantum continuations}
In light of this last observation i want to point out that the
predominant account of quantum mechanics is missing a key aspect of a
truly compositional story of the physical situation. In a real lab,
when a measurement is made the observation can be made to feed into
another device that then makes another measurement conditioned on the
results of the first. This means that after the superposition was
collapsed the entire experimental set up remained in
superposition. While QM offers a means of writing this down it doesn't
quite line up well with the well-trodden formulation of computation
and continuation that we see so succinctly expressed in Milner's
calculi. This suggests that there might be advantages to this account
of dynamics waiting to be explored.

\paragraph{Quantum logic}
In this connection, we also note that by virtue of having the
Hennessy-Milner construction, we can pull the construction through the
interpretation of QM. This gives us a natural candidate for a quantum
logic that enjoys an extremely tight connection with it's domain of
interpretation, making the construction much less ad hoc (rather it is
the image of functor!).

\paragraph{Quantum probabiity}
i have questions about the basis of the interpretation of inner
product as probability amplitude. In particular, using which
axiomatization of probability theory does the notion of probability
amplitude earn the right to be so dubbed? In other words, where is the
proof that the operation for calculating a probability amplitude (and
then squaring) satisfies the axioms of what it means to calculate a
probability? Even if such a proof exists (i have yet to find it in the
literature), i wonder if it might not be possible to turn things on
their heads. Can we view the calculation of the probability amplitude
as an axiomatization of probability? If so, then the definition we
give for calculating probability amplitude may provide the basis for
an \emph{effective} theory of probability.

\paragraph{Quantum vs ``biological'' information}
Finally, i want to conclude with a more philosophical observation. At
a recent workshop in which QM was a predominant topic i noticed
something about quantum information. The speaker was giving a riveting
discussion of axiomatic QM and showing how properties of ``no
cloning'' and ``no deleting'' emerged as consequences of the
axiomatization. Theorems of this form are necessary to give us a sense
of confidence that our axioms characterize the physical theory. What
struck me, though, was that if quantum information is neither erasable
nor replicable it is markedly different from \emph{life}. Two of the
things we know about life is that

\begin{itemize}
  \item it ends;
  \item to gain some measure of persistence, to transcend it's
    finitude it is imminently copyable.
\end{itemize}

Both of these qualities are summarized succinctly in the aphorism: all
flesh is grass. For me these two kinds of ``information'' -- call them
quantum and biological -- are end points on a spectrum of strategies
for persistence. At one end, we have those curious entities that enjoy
uniqueness and permanence; at the other, we have those who in the face
of a certain end and an uncertain present make a go of passing
something on. To me one of the more remarkable aspects of the latter
strategy is that in the presence of noise (and certain features of
copying) we get a kind of dynamism, a chance for improvement against a
given persistent condition.

% subsection other_calculi_other_bisimulations_and_geometry_as_behavior (end)




% section conclusion (end)

%\documentclass[12pt]{llncs}
%\documentclass{jktr}

\usepackage[pdftex]{hyperref}                   
\usepackage {listings}
\usepackage {mathpartir}
\usepackage{bcprules}
%\usepackage{listings}
                       
\usepackage{graphicx} 
%\usepackage[margins=2.5cm,nohead,nofoot]{geometry}
%\usepackage{geometry}
\usepackage{amsfonts}
\usepackage{amstext}
\usepackage{latexsym}
\usepackage{amssymb}
\usepackage{color}


%\include{myPreamble}
\include{qm2pi.local} 

%\ifpdf
%\usepackage[pdftex]{graphicx}
%\else
%\usepackage{graphicx}
%\fi

 % \ifpdf
%  \usepackage{pdfsync}
%  \if


%\title{Brief Article}
%\author{David F. Snyder}
%\author{L.G. Meredith}

%\address{Dept. of Math., Texas State University--San Marcos, San Marcos, TX 78666}
       
\pagestyle{empty}


\begin{document}

\lstset{language=[Objective]Caml,frame=shadowbox}

\input{qm2pi.front}

% section front matter (end)

\input{qm2pi.intro} 
 
% section introduction (end)

% \input{qm2pi.knotations} 

% section notation (end)

\input{qm2pi.process.calculi} 

% section concurrent_process_calculi_and_spatial_logics_ (end)
    
%\input{qm2pi.knots2pi} 

%\input{qm2pi.trefoil} 

%\input{qm2pi.mainthm} 

% subsection basic_interpretation (end)

%\input{qm2pi.rho.presentation} 
\subsection{The syntax and semantics of the notation system}\label{sub:the_syntax_and_semantics_of_the_notation_system} % (fold)

We now summarize a technical presentation of the calculus that
embodies our theory of dynamics. The typical presentation of such a
calculus follows the style of giving generators and relations on
them. The grammar, below, describing term constructors, freely
generates the set of processes, $\Proc$. This set is then quotiented
by a relation known as structural congruence and it is over this set
that the notion of dynamics is expressed. This presentation is
essentially that of \cite{MeredithR05} with the addition of
polyadicity and summation. For readability we have relegated some of
the technical subtleties to an appendix.

\subsubsection{Process grammar}\label{subsub:process_grammar}

\begin{mathpar}
  \inferrule* [lab=synchronization] {} {{M} \bc \pzero \;|\; x?F \;|\; x!C }
  \and
  \inferrule* [lab=abstraction] {} {{F} \bc (x)P}
  \and
  \inferrule* [lab=concretion] {} {{C} \bc \langle Q \rangle}
  \and
  \inferrule* [lab=process] {} {{P,Q} \bc M \;| \;P|Q \;|\; @{x}}
  \and
  \inferrule* [lab=name] {} {{x} \bc \quotep{P}}
\end{mathpar} 

Note that $\vec{x}$ (resp. $\vec{P}$) denotes a vector of names
(resp. processes) of length $|\vec{x}|$ (resp. $|\vec{P}|$). We adopt
the following useful abbreviations.

\begin{mathpar}
   x?(\vec{y}).P := x.(\vec{y})P \and  x\clift{\vec{P}} := x.\clift{\vec{P}}
   \and x!(y) := \lift{x}{\dropn{y}}
   \and \Pi_{i=0}^{n-1}P_i := P_0 | \ldots | P_{n-1}
\end{mathpar}

\subsubsection{Structural congruence}

\paragraph{Free and bound names and alpha-equivalence.} At the
core of structural equivalence is alpha-equivalence which identifies
process that are the same up to a change of variable. Formally, we
recognize the distinction between free and bound names. The free names
of a process, $\freenames{P}$, may be calculated recursively as
follows:

\begin{mathpar}
\freenames{\pzero} := \emptyset
  \and \\
  \freenames{x?(y).P} := \{ x \} \cup (\freenames{P} \setminus \{ y \})
  \and 
  \freenames{x!\langle P \rangle} := \{ x \} \cup \{ P \} 
  \and \\
  \freenames{P|Q} := \freenames{P} \cup \freenames{Q}
  \and \\
  \freenames{@{x}} := \{ x \}
\end{mathpar}

$\pi$
$\quotep{\pi}$

$\freenames{-} : \pi \to \mathcal{P}(\quotep{\pi})$

\begin{eqnarray*}
  \freenames{\pzero} & := & \emptyset \\
  \freenames{x?(y).P} & := & \{ x \} \cup (\freenames{P} \setminus \{ y \}) \\
  \freenames{x!\langle P \rangle} & := & \{ x \} \cup \{ P \} \\
  \freenames{P|Q} & := & \freenames{P} \cup \freenames{Q} \\
  \freenames{\dropn{x}} & := & \{ x \}
\end{eqnarray*}

The bound names of a process, $\boundnames{P}$, are those names occurring in $P$
that are not free. For example, in $x?(y).0$, the name $x$ is free, while $y$ is bound.

\begin{mathpar}
  \inferrule* [lab=monoidal-laws] {} { P|Q \equiv Q|P \and P|0 \equiv P \and P|(Q|R) \equiv (P|Q)|R }
\end{mathpar}

\begin{mathpar}
  \inferrule* [lab=alpha-equivalence] {} { (x)P \equiv (y)P\{y/x\} \and y \not\in \freenames{P} }
\end{mathpar}

\begin{definition}
Then two processes, $P,Q$, are alpha-equivalent if $P = Q\{\vec{y}/\vec{x}\}$ for
some $\vec{x} \in \boundnames{Q},\vec{y} \in \boundnames{P}$, where $Q\{\vec{y}/\vec{x}\}$
denotes the capture-avoiding substitution of $\vec{y}$ for $\vec{x}$ in $Q$.
\end{definition}

\begin{definition}
  The {\em structural congruence} \cite{SangiorgiWalker} , $\equiv$,
  between processes is the least congruence containing
  alpha-equivalence, satisfying the abelian monoid laws
  (associativity, commutativity and $\pzero$ as identity) for parallel
  composition $|$ and for summation $+$.
\end{definition}

\subsection{Name equivalence}

We take name equivalence, written $\nameeq$, to be the smallest
equivalence relation generated by the following rules.

\begin{mathpar}
\inferrule*[lab=Quote-drop]
{ }
{ \quotep{@{x}} \nameeq x }

\inferrule*[lab=Struct-equiv]
{ P \scong Q }
{ \quotep{P} \nameeq \quotep{Q} }
\end{mathpar}

The astute reader will have noticed that the mutual recursion of names
and processes imposes a mutual recursion on alpha-equivalence and
structural equivalence via name-equivalence. Fortunately, all of this
works out pleasantly and we may calculate in the natural way, free of
concern. The reader interested in the details is referred to the
appendix \ref{appendix:rho_details}.

\subsection{Substitution}

We use $\Proc$ for the set of processes, $\QProc$ for the set of
names, and $\id{\{}\vec{y} / \vec{x} \id{\}}$ to denote partial maps,
$s : \QProc \rightarrow \QProc$. A map, $s$ lifts, uniquely, to a map
on process terms, $\widehat{s} : \Proc \rightarrow \Proc$ by the
following equations.

\begin{mathpar}
  (0) \psubstp{Q}{P} := 0 \\
  (R \juxtap S) \psubstp{Q}{P}
  :=    
  (R)\psubstp{Q}{P} \juxtap (S) \psubstp{Q}{P} \\
  (x?(y).R) \psubstp{Q}{P}    
  :=    
  (x)\substp{Q}{P} (z)\concat( (R \psubstn{z}{y}) \psubstp{Q}{P} ) \\
  (\lift{x}{R}) \psubstp{Q}{P}  
  :=
  \lift{(x)\substp{Q}{P}}{ R \psubstp{Q}{P} } \\
%   (\dropn{x})  \psubstp{Q}{P}       
%   := 
%   \left\{ 
%     \begin{array}{ccc} 
%       \dropn{\quotep{Q}} & & x \nameeq \quotep{P} \\
%       \dropn{x} & & otherwise \\
%     \end{array}
%   \right. 
  (\dropn{x})  \psubstp{Q}{P}       
  := 
  \left\{ 
    \begin{array}{ccc} 
      Q & & x \nameeq \quotep{P} \\
      \dropn{x} & & otherwise \\
    \end{array}
  \right.
\end{mathpar}
 

where

\begin{eqnarray}
  (x)\id{\{} \lpquote Q \rpquote / \lpquote P \rpquote \id{\}}            = 
  \left\{ 
    \begin{array}{ccc}
      \lpquote Q \rpquote & & x \nameeq \lpquote P \rpquote \\
      x & & otherwise \\
    \end{array}
  \right. \nonumber
\end{eqnarray}

and $z$ is chosen distinct from $\quotep{P}$, $\quotep{Q}$, the free
names in $Q$, and all the names in $R$. Our $\alpha$-equivalence will
be built in the standard way from this substitution.

\begin{remark}\label{rem:no_self_referential_names}
  One consequence of these definitions is that $\forall P. \quotep{P}
  \not\in \freenames{P}$.
\end{remark}

\subsection{ Dynamic quote: an example }

Anticipating something of what's to come, consider applying the
substitution, $\widehat{\id{\{}u / z \id{\}}}$, to the following pair
of processes, $\lift{w}{y!(z)}$ and $w[ \lpquote y!(z) \rpquote ]$.

\begin{eqnarray}
	\lift{w}{y!(z)}\widehat{\id{\{}u / z \id{\}}}
		& = &
		\lift{w}{y!(u)} \nonumber\\
	w[ \lpquote y!(z) \rpquote ] \widehat{ \id{\{}u / z \id{\}} }
		& = &
		w[ \lpquote y!(z) \rpquote ] \nonumber
\end{eqnarray}

Because the body of the process between quotes is impervious to
substitution, we get radically different answers. In fact, by
examining the first process in an input context,
e.g. $x?(z).\lift{w}{y!(z)}$, we see that the process under the lift
operator may be shaped by prefixed inputs binding a name inside it. In
this sense, the lift operator will be seen as a way to dynamically
construct processes before reifying them as names.

Finally equipped with these standard features we can present the
dynamics of the calculus.

\subsubsection{Operational semantics} 

Finally, we introduce the computational dynamics. What marks these
algebras as distinct from other more traditionally studied algebraic
structures, e.g. vector spaces or polynomial rings, is the manner in
which dynamics is captured. In traditional structures, dynamics is typically
expressed through morphisms between such structures, as in linear maps
between vector spaces or morphisms between rings. In algebras
associated with the semantics of computation, the dynamics is
expressed as part of the algebraic structure itself, through a
reduction reduction relation typically denoted by $\red$. Below, we
give a recursive presentation of this relation for the calculus used
in the encoding.

$\red \subseteq \pi \times \pi$
$\red : \pi \to \mathcal{P}(\pi)$

\begin{mathpar}
  \inferrule* [lab=Comm] { \textsf{match}( x_{src}, x_{trgt} ) } { x_{trgt}?(y)P \; | \; x_{src}!\langle {Q} \rangle \red P\{\quotep{Q}/y}\} }
  \and \\
  \inferrule* [lab=Par] {{P} \red {P}'} {{{P} | {Q}} \red {{P}' | {Q}}}
  \and
  \inferrule* [lab=Equiv]{{{P} \scong {P}'} \andalso {{P}' \red {Q}'} \andalso {{Q}' \scong {Q}}}{{P} \red {Q}}
\end{mathpar}

\begin{eqnarray*}
  match_{\equiv} (\quotep{P},\quotep{Q}) & := & P \equiv Q \\
  match_{\dagger}(\quotep{P},\quotep{Q}) & := & \forall R. P|Q \red^{*} R => R \red^{*} 0 \\
  match_{K}(\quotep{P},\quotep{Q}) & := & K \mbox{ for some context } K
\end{eqnarray*}

$u?(x)P | u!\langle Q \rangle \red P\{\quotep{Q}/x\}$

%We write $\wred$ for $\red^*$, and $P\red$ if $\exists Q $ such that $ P \red Q$.
We write $P\red$ if $\exists Q $ such that $ P \red Q$ and $P\not\red$, otherwise.

\section{Replication}

As mentioned before, it is known that replication (and hence
recursion) can be implemented in a higher-order process algebra
\cite{SangiorgiWalker}. As our first example of calculation with the
machinery thus far presented we give the construction explicitly in
the {\rhoc}.

\begin{eqnarray}
	D_{x} & := & \prefix{x}{y}{(\binpar{\outputp{x}{y}}{@{y}})} \nonumber\\
	\bangp_{x}{P} & := & \binpar{{x}!\langle{\binpar{D_{x}}{P}}\rangle}{D_{x}} \nonumber
\end{eqnarray}

\begin{eqnarray}
	\bangp_{x}{P} & & \nonumber\\
	=
	& {x}!\langle{(\prefix{x}{y}{(\outputp{x}{y} | @{y})) | P}}\rangle 
	      | \prefix{x}{y}{(\outputp{x}{y} | @{y})} & \nonumber\\
	\red
	& (\outputp{x}{y} | @{y})\substn{\quotep{(\prefix{x}{y}{(@{y} | \outputp{x}{y})) | P}}}{y} & \nonumber\\
	=
	& \outputp{x}{\quotep{(\prefix{x}{y}{(\outputp{x}{y} | @{y})) | P}}}
	  | {(\prefix{x}{y}{(\outputp{x}{y} | @{y})) | P}} & \nonumber\\
	\red
	& \ldots & \nonumber\\
	\red^*
	& P | P | \ldots & \nonumber
\end{eqnarray}

Of course, this encoding, as an implementation, runs away, unfolding
$\bangp{P}$ eagerly. A lazier and more implementable replication
operator, restricted to input-guarded processes, may be obtained as follows.

\begin{eqnarray}
\bangp{\prefix{u}{v}{P}} 
	:= 
	\binpar{\lift{x}{\prefix{u}{v}{(\binpar{D(x)}{P})}}}{D(x)} \nonumber
\end{eqnarray}

\begin{remark}
  Note that the lazier definition still does not deal with summation
  or mixed summation (i.e. sums over input and output). The reader is
  invited to construct definitions of replication that deal with these
  features. 

  Further, the definitions are parameterized in a name, $x$. Can you,
  gentle reader, make a definition that eliminates this parameter and
  guarantees no accidental interaction between the replication
  machinery and the process being replicated -- i.e. no accidental
  sharing of names used by the process to get its work done and the
  name(s) used by the replication to effect copying. This latter
  revision of the definition of replication is crucial to obtaining
  the expected identity $!!P \sim !P$.
\end{remark}

\begin{remark}\label{rem:paradoxical_combinator}
  The reader familiar with the lambda calculus will have noticed the
  similarity between $D$ and the paradoxical combinator.

  [Ed. note: the existence of this seems to suggest we have to be more
  restrictive on the set of processes and names we admit if we are to
  support no-cloning.]
\end{remark}

\subsubsection{Bisimulation}

The computational dynamics gives rise to another kind of equivalence,
the equivalence of computational behavior. As previously mentioned
this is typically captured \emph{via} some form of bisimulation.

% The notion we use in this paper is weak barbed bisimulation
% \cite{milner91polyadicpi}.

The notion we use in this paper is derived from weak barbed
bisimulation \cite{milner91polyadicpi}. 

\begin{definition}
An \emph{observation relation}, $\downarrow_{\mathcal N}$, over a set
of names, $\mathcal N$, is the smallest relation satisfying the rules
below.

\infrule[Out-barb]{y \in {\mathcal N}, \; x \nameeq y}
		  {\outputp{x}{v} \downarrow_{\mathcal N} x}
\infrule[Par-barb]{\mbox{$P\downarrow_{\mathcal N} x$ or $Q\downarrow_{\mathcal N} x$}}
		  {\binpar{P}{Q} \downarrow_{\mathcal N} x}

We write $P \Downarrow_{\mathcal N} x$ if there is $Q$ such that 
$P \wred Q$ and $Q \downarrow_{\mathcal N} x$.
\end{definition}

\begin{definition}
%\label{def.bbisim}
An  ${\mathcal N}$-\emph{barbed bisimulation} over a set of names, ${\mathcal N}$, is a symmetric binary relation 
${\mathcal S}_{\mathcal N}$ between agents such that $P\rel{S}_{\mathcal N}Q$ implies:
\begin{enumerate}
\item If $P \red P'$ then $Q \wred Q'$ and $P'\rel{S}_{\mathcal N} Q'$.
\item If $P\downarrow_{\mathcal N} x$, then $Q\Downarrow_{\mathcal N} x$.
\end{enumerate}
$P$ is ${\mathcal N}$-barbed bisimilar to $Q$, written
$P \wbbisim_{\mathcal N} Q$, if $P \rel{S}_{\mathcal N} Q$ for some ${\mathcal N}$-barbed bisimulation ${\mathcal S}_{\mathcal N}$.
\end{definition}

$\mathcal{R} \subseteq \pi \times \pi$

$P \mathcal{R} Q => \forall P'. P \red P' \Rightarrow \exists Q'. Q \red Q', P' \mathcal{R} Q'$

$P \vdash x \Rightarrow Q \vdash x$

\begin{mathpar}
  \inferrule*[lab=Out-barb]{x \nameeq y}{{y}!\langle{Q}\rangle \vdash x}
  \and
  \inferrule*[lab=Par-barb]{\mbox{$P\vdash x$ or $Q\vdash x$}}{\binpar{P}{Q} \vdash x}
\end{mathpar}

\subsubsection{Contexts}

One of the principle advantages of computational calculi like the
$\pi$-calculus is a well-defined notion of context,
contextual-equivalence and a correlation between
contextual-equivalence and notions of bisimulation. The notion of
context allows the decomposition of a process into (sub-)process and
its syntactic environment, its context. Thus, a context may be
thought of as a process with a ``hole'' (written $\Box$) in it. The
application of a context $M$ to a process $P$, written $M[P]$, is
tantamount to filling the hole in $M$ with $P$. In this paper we do
not need the full weight of this theory, but do make use of the notion
of context in the proof the main theorem. 

\begin{mathpar}
  \inferrule* [lab=summation] {} {{M_{M},M_{N}} \bc \Box \;|\; x.M_{A} \;|\; M_{M}+M_{N}}
  \and
  \inferrule* [lab=agent] {} {{M_{A}} \bc (\vec{x})M_{P} \;| \; \clift{P_0,\ldots,M_{P},\ldots,P_N}}
  \and \\
  \inferrule* [lab=process] {} {{M_{P}} \bc M_{N} \;| \;P|M_{P} }
\end{mathpar} 

\begin{mathpar}
  \inferrule* [lab=sychronization] {} {M_{N} \bc \Box \;|\; x?M_{F} \;|\; x!M_{C}}
  \and
  \inferrule* [lab=abstraction] {} {{M_{F}} \bc (x)M_{P} }
  \and
  \inferrule* [lab=concretion] {} {{M_{C}} \bc \langle M_{P} \rangle }
  \and \\
  \inferrule* [lab=process] {} {{M_{P}} \bc M_{N} \;| \;P|M_{P} }
\end{mathpar}

\begin{definition}[contextual application] Given a context $M$, and
  process $P$, we define the \emph{contextual application}, $M[P] :=
  M\{P/\Box\}$. That is, the contextual application of M to P is the
  substitution of $P$ for $\Box$ in $M$.
\end{definition}

$\meaningof{-} : L \to \mathcal{P}(\pi)$

\begin{mathpar}
  \inferrule* [lab=collection] {} {\meaningof{true} = \pi, \and \meaningof{~E} = \pi \setminus \meaningof{E}, \and \meaningof{E_{1} \& E_{2}} = \meaningof{E_{1}} \cap \meaningof{E_{2}}}
\end{mathpar}

\begin{mathpar}
  \inferrule* [lab=structure] {} {\meaningof{0} = \{ P \in \pi | P \equiv 0 \}, \and \\ \meaningof{E_1 | E_2} = \{ P \in \pi | P \equiv P_{1} | P_{2}, P_{1} \in \meaningof{E_{1}}, P_{2} \in \meaningof{E_2}\} }
\end{mathpar}

\begin{mathpar}
 \inferrule* [lab=behavior] {} {\meaningof{\langle a?b \rangle E} = \{ P \in \pi | P \equiv Q | u?(y)P', \\ \and \\\\ \and \\ \;\;\; u \in \meaningof{a}, \forall z.P'\{z/y\} \in \meaningof{E\{z/b\}}\}, \and \\ \meaningof{a!E} = \{ P \in \pi | P \equiv Q | x!\langle P' \rangle, x \in \meaningof{a} P' \in \meaningof{E}\} }
\end{mathpar}

\begin{mathpar}
 \inferrule* [lab=nominal] {} {\meaningof{\quotep{E}} = \{ \quotep{P} \in \quotep{\pi} | P \in \meaningof{E} \}, \and \meaningof{\quotep{P}} = \{ \quotep{Q} \in \quotep{\pi} | P \equiv Q \} \and \\ \meaningof{@\quotep{E}} = \{ P \in \pi | P \equiv @x, x \in \meaningof{E} \}}
\end{mathpar}

\begin{eqnarray*}
  \\
  \meaningof{-} : TS \to ST
\end{eqnarray*}

\begin{eqnarray*}
  \\
  L : TS \to ST
\end{eqnarray*}

\begin{eqnarray*}
  \\
  P \models E \iff P \in \meaningof{E}
\end{eqnarray*}

\begin{eqnarray*}
  P \approx_{L} Q \iff \forall E \in L. P \models E \iff Q \models E
\end{eqnarray*}

\begin{eqnarray*}
  P \approx_{K} Q
\end{eqnarray*}

\begin{eqnarray*}
  P \approx Q
\end{eqnarray*}

$\approx_{K} = \approx = \approx_{L}$

\subsubsection{Contextual duality}

Note that contexts extend the quotation operation to a family of
operations from processes to names. Given a context, $M$, we can
define a \emph{nominal context}, $\quotep{M}$ by $\quotep{M}[P] :=
\quotep{M[P]}$. To foreshadow what is to come we observe that these
operations enjoy a duality with processes very much like the duality
between vectors and maps from vectors to scalars.

Further, because the calculus is essentially higher-order, we have a
correspondence between contexts and processes. More specifically,
given a name $x$ and a context $M$ we can construct $M^{*}_{x}$ such
that 

\begin{mathpar}
  M^{*}_{x} | \lift{x}{P} \red M[P]
\end{mathpar}

namely,

\begin{mathpar}
  M^{*}_{x} := x?(u).M[\dropn{u}]
\end{mathpar}

The dependence of $M^{*}_{x}$ on a name makes it an abstraction, 

\begin{mathpar}
  M^{*} := (x)x?(u).M[\dropn{u}]
\end{mathpar}

\subsection{Additional notation}

It will sometimes be convenient to denote the process a name
quotes. We already have the notation $x = \quotep{P}$, but it will be
convenient to introduce an alternate notation, $\procn{x}$, when we
want to emphasize the connection to the use of the name. Note that, by
virtue of name equivalence, $\quotep{\procn{x}} \nameeq x$; so, the
notation is consistent with previous definitions.

Further, because names have structure it is possible to effect
substitutions on the basis of that structure. This means we need to
upgrade our notation for substitutions, which we accomplish by
adapting comprehension notation. Thus,

\begin{mathpar}
  P\{ y / x : x \in S \}
\end{mathpar}

is interpreted to mean the process derived from P by replacing (in a
capture-avoiding manner) each occurrence of $x$ in $S$ by $y$. For example,

\begin{mathpar}
  P\{ \quotep{\procn{x}|\procn{x}} / x : x \in \freenames{P} \}
\end{mathpar}

will replace each (occurrence) of a free name $x$ in $P$ by
$\quotep{\procn{x}|\procn{x}}$.

Also, we will avail ourselves of the notation $x^{L}$ and $x^{R}$ to
denote injections of a name into disjoint copies of the name
space. There are numerous ways to accomplish this. One example can be
found in \cite{MeredithR05}. This notation overloads to vectors of
names: $\vec{x}^{\pi} := (x_{i}^{\pi} \; : \; 0 \leq i < |\vec{x}| )$ where $\pi \in \{L,R\}$.

We also use $P^{\Box} := P|\Box$.

In \cite{MeredithR05} an interpretation of the new operator is
given. It turns out that there are several possible interpretations
all enjoying the requisite algebraic properties of the operator (see
\cite{milner91polyadicpi}). We will therefore make liberal use of
$(\nu\; \vec{x})P$.

% subsection the_syntax_and_semantics_of_the_notation_system (end)   

\input{qm2pi.qmops} 

\input{qm2pi.sterngerlach} 

\input{qm2pi.metric} 

% section concurrent_process_calculi (end)

%\input{qm2pi.proofsketch}

% section proof sketch (end)

%\input{qm2pi.slviaknots} 

% section spatial logic via knots (end)

\input{qm2pi.conclusion}

% section conclusion (end)

%\input{qm2pi.dtcodes} 

% section wiring algorithm (end)

\input{qm2pi.ack} 

% section acknowledgments (end)

\newpage


\bibliographystyle{plain}   
\bibliography{../../biblios/main.bib}

\input{qm2pi.rhodetails}

\end{document}

 

% section wiring algorithm (end)

\documentclass[12pt]{llncs}
%\documentclass{jktr}

\usepackage[pdftex]{hyperref}                   
\usepackage {listings}
\usepackage {mathpartir}
\usepackage{bcprules}
%\usepackage{listings}
                       
\usepackage{graphicx} 
%\usepackage[margins=2.5cm,nohead,nofoot]{geometry}
%\usepackage{geometry}
\usepackage{amsfonts}
\usepackage{amstext}
\usepackage{latexsym}
\usepackage{amssymb}
\usepackage{color}


%\include{myPreamble}
\include{qm2pi.local} 

%\ifpdf
%\usepackage[pdftex]{graphicx}
%\else
%\usepackage{graphicx}
%\fi

 % \ifpdf
%  \usepackage{pdfsync}
%  \if


%\title{Brief Article}
%\author{David F. Snyder}
%\author{L.G. Meredith}

%\address{Dept. of Math., Texas State University--San Marcos, San Marcos, TX 78666}
       
\pagestyle{empty}


\begin{document}

\lstset{language=[Objective]Caml,frame=shadowbox}

\input{qm2pi.front}

% section front matter (end)

\input{qm2pi.intro} 
 
% section introduction (end)

% \input{qm2pi.knotations} 

% section notation (end)

\input{qm2pi.process.calculi} 

% section concurrent_process_calculi_and_spatial_logics_ (end)
    
%\input{qm2pi.knots2pi} 

%\input{qm2pi.trefoil} 

%\input{qm2pi.mainthm} 

% subsection basic_interpretation (end)

%\input{qm2pi.rho.presentation} 
\subsection{The syntax and semantics of the notation system}\label{sub:the_syntax_and_semantics_of_the_notation_system} % (fold)

We now summarize a technical presentation of the calculus that
embodies our theory of dynamics. The typical presentation of such a
calculus follows the style of giving generators and relations on
them. The grammar, below, describing term constructors, freely
generates the set of processes, $\Proc$. This set is then quotiented
by a relation known as structural congruence and it is over this set
that the notion of dynamics is expressed. This presentation is
essentially that of \cite{MeredithR05} with the addition of
polyadicity and summation. For readability we have relegated some of
the technical subtleties to an appendix.

\subsubsection{Process grammar}\label{subsub:process_grammar}

\begin{mathpar}
  \inferrule* [lab=synchronization] {} {{M} \bc \pzero \;|\; x?F \;|\; x!C }
  \and
  \inferrule* [lab=abstraction] {} {{F} \bc (x)P}
  \and
  \inferrule* [lab=concretion] {} {{C} \bc \langle Q \rangle}
  \and
  \inferrule* [lab=process] {} {{P,Q} \bc M \;| \;P|Q \;|\; @{x}}
  \and
  \inferrule* [lab=name] {} {{x} \bc \quotep{P}}
\end{mathpar} 

Note that $\vec{x}$ (resp. $\vec{P}$) denotes a vector of names
(resp. processes) of length $|\vec{x}|$ (resp. $|\vec{P}|$). We adopt
the following useful abbreviations.

\begin{mathpar}
   x?(\vec{y}).P := x.(\vec{y})P \and  x\clift{\vec{P}} := x.\clift{\vec{P}}
   \and x!(y) := \lift{x}{\dropn{y}}
   \and \Pi_{i=0}^{n-1}P_i := P_0 | \ldots | P_{n-1}
\end{mathpar}

\subsubsection{Structural congruence}

\paragraph{Free and bound names and alpha-equivalence.} At the
core of structural equivalence is alpha-equivalence which identifies
process that are the same up to a change of variable. Formally, we
recognize the distinction between free and bound names. The free names
of a process, $\freenames{P}$, may be calculated recursively as
follows:

\begin{mathpar}
\freenames{\pzero} := \emptyset
  \and \\
  \freenames{x?(y).P} := \{ x \} \cup (\freenames{P} \setminus \{ y \})
  \and 
  \freenames{x!\langle P \rangle} := \{ x \} \cup \{ P \} 
  \and \\
  \freenames{P|Q} := \freenames{P} \cup \freenames{Q}
  \and \\
  \freenames{@{x}} := \{ x \}
\end{mathpar}

$\pi$
$\quotep{\pi}$

$\freenames{-} : \pi \to \mathcal{P}(\quotep{\pi})$

\begin{eqnarray*}
  \freenames{\pzero} & := & \emptyset \\
  \freenames{x?(y).P} & := & \{ x \} \cup (\freenames{P} \setminus \{ y \}) \\
  \freenames{x!\langle P \rangle} & := & \{ x \} \cup \{ P \} \\
  \freenames{P|Q} & := & \freenames{P} \cup \freenames{Q} \\
  \freenames{\dropn{x}} & := & \{ x \}
\end{eqnarray*}

The bound names of a process, $\boundnames{P}$, are those names occurring in $P$
that are not free. For example, in $x?(y).0$, the name $x$ is free, while $y$ is bound.

\begin{mathpar}
  \inferrule* [lab=monoidal-laws] {} { P|Q \equiv Q|P \and P|0 \equiv P \and P|(Q|R) \equiv (P|Q)|R }
\end{mathpar}

\begin{mathpar}
  \inferrule* [lab=alpha-equivalence] {} { (x)P \equiv (y)P\{y/x\} \and y \not\in \freenames{P} }
\end{mathpar}

\begin{definition}
Then two processes, $P,Q$, are alpha-equivalent if $P = Q\{\vec{y}/\vec{x}\}$ for
some $\vec{x} \in \boundnames{Q},\vec{y} \in \boundnames{P}$, where $Q\{\vec{y}/\vec{x}\}$
denotes the capture-avoiding substitution of $\vec{y}$ for $\vec{x}$ in $Q$.
\end{definition}

\begin{definition}
  The {\em structural congruence} \cite{SangiorgiWalker} , $\equiv$,
  between processes is the least congruence containing
  alpha-equivalence, satisfying the abelian monoid laws
  (associativity, commutativity and $\pzero$ as identity) for parallel
  composition $|$ and for summation $+$.
\end{definition}

\subsection{Name equivalence}

We take name equivalence, written $\nameeq$, to be the smallest
equivalence relation generated by the following rules.

\begin{mathpar}
\inferrule*[lab=Quote-drop]
{ }
{ \quotep{@{x}} \nameeq x }

\inferrule*[lab=Struct-equiv]
{ P \scong Q }
{ \quotep{P} \nameeq \quotep{Q} }
\end{mathpar}

The astute reader will have noticed that the mutual recursion of names
and processes imposes a mutual recursion on alpha-equivalence and
structural equivalence via name-equivalence. Fortunately, all of this
works out pleasantly and we may calculate in the natural way, free of
concern. The reader interested in the details is referred to the
appendix \ref{appendix:rho_details}.

\subsection{Substitution}

We use $\Proc$ for the set of processes, $\QProc$ for the set of
names, and $\id{\{}\vec{y} / \vec{x} \id{\}}$ to denote partial maps,
$s : \QProc \rightarrow \QProc$. A map, $s$ lifts, uniquely, to a map
on process terms, $\widehat{s} : \Proc \rightarrow \Proc$ by the
following equations.

\begin{mathpar}
  (0) \psubstp{Q}{P} := 0 \\
  (R \juxtap S) \psubstp{Q}{P}
  :=    
  (R)\psubstp{Q}{P} \juxtap (S) \psubstp{Q}{P} \\
  (x?(y).R) \psubstp{Q}{P}    
  :=    
  (x)\substp{Q}{P} (z)\concat( (R \psubstn{z}{y}) \psubstp{Q}{P} ) \\
  (\lift{x}{R}) \psubstp{Q}{P}  
  :=
  \lift{(x)\substp{Q}{P}}{ R \psubstp{Q}{P} } \\
%   (\dropn{x})  \psubstp{Q}{P}       
%   := 
%   \left\{ 
%     \begin{array}{ccc} 
%       \dropn{\quotep{Q}} & & x \nameeq \quotep{P} \\
%       \dropn{x} & & otherwise \\
%     \end{array}
%   \right. 
  (\dropn{x})  \psubstp{Q}{P}       
  := 
  \left\{ 
    \begin{array}{ccc} 
      Q & & x \nameeq \quotep{P} \\
      \dropn{x} & & otherwise \\
    \end{array}
  \right.
\end{mathpar}
 

where

\begin{eqnarray}
  (x)\id{\{} \lpquote Q \rpquote / \lpquote P \rpquote \id{\}}            = 
  \left\{ 
    \begin{array}{ccc}
      \lpquote Q \rpquote & & x \nameeq \lpquote P \rpquote \\
      x & & otherwise \\
    \end{array}
  \right. \nonumber
\end{eqnarray}

and $z$ is chosen distinct from $\quotep{P}$, $\quotep{Q}$, the free
names in $Q$, and all the names in $R$. Our $\alpha$-equivalence will
be built in the standard way from this substitution.

\begin{remark}\label{rem:no_self_referential_names}
  One consequence of these definitions is that $\forall P. \quotep{P}
  \not\in \freenames{P}$.
\end{remark}

\subsection{ Dynamic quote: an example }

Anticipating something of what's to come, consider applying the
substitution, $\widehat{\id{\{}u / z \id{\}}}$, to the following pair
of processes, $\lift{w}{y!(z)}$ and $w[ \lpquote y!(z) \rpquote ]$.

\begin{eqnarray}
	\lift{w}{y!(z)}\widehat{\id{\{}u / z \id{\}}}
		& = &
		\lift{w}{y!(u)} \nonumber\\
	w[ \lpquote y!(z) \rpquote ] \widehat{ \id{\{}u / z \id{\}} }
		& = &
		w[ \lpquote y!(z) \rpquote ] \nonumber
\end{eqnarray}

Because the body of the process between quotes is impervious to
substitution, we get radically different answers. In fact, by
examining the first process in an input context,
e.g. $x?(z).\lift{w}{y!(z)}$, we see that the process under the lift
operator may be shaped by prefixed inputs binding a name inside it. In
this sense, the lift operator will be seen as a way to dynamically
construct processes before reifying them as names.

Finally equipped with these standard features we can present the
dynamics of the calculus.

\subsubsection{Operational semantics} 

Finally, we introduce the computational dynamics. What marks these
algebras as distinct from other more traditionally studied algebraic
structures, e.g. vector spaces or polynomial rings, is the manner in
which dynamics is captured. In traditional structures, dynamics is typically
expressed through morphisms between such structures, as in linear maps
between vector spaces or morphisms between rings. In algebras
associated with the semantics of computation, the dynamics is
expressed as part of the algebraic structure itself, through a
reduction reduction relation typically denoted by $\red$. Below, we
give a recursive presentation of this relation for the calculus used
in the encoding.

$\red \subseteq \pi \times \pi$
$\red : \pi \to \mathcal{P}(\pi)$

\begin{mathpar}
  \inferrule* [lab=Comm] { \textsf{match}( x_{src}, x_{trgt} ) } { x_{trgt}?(y)P \; | \; x_{src}!\langle {Q} \rangle \red P\{\quotep{Q}/y}\} }
  \and \\
  \inferrule* [lab=Par] {{P} \red {P}'} {{{P} | {Q}} \red {{P}' | {Q}}}
  \and
  \inferrule* [lab=Equiv]{{{P} \scong {P}'} \andalso {{P}' \red {Q}'} \andalso {{Q}' \scong {Q}}}{{P} \red {Q}}
\end{mathpar}

\begin{eqnarray*}
  match_{\equiv} (\quotep{P},\quotep{Q}) & := & P \equiv Q \\
  match_{\dagger}(\quotep{P},\quotep{Q}) & := & \forall R. P|Q \red^{*} R => R \red^{*} 0 \\
  match_{K}(\quotep{P},\quotep{Q}) & := & K \mbox{ for some context } K
\end{eqnarray*}

$u?(x)P | u!\langle Q \rangle \red P\{\quotep{Q}/x\}$

%We write $\wred$ for $\red^*$, and $P\red$ if $\exists Q $ such that $ P \red Q$.
We write $P\red$ if $\exists Q $ such that $ P \red Q$ and $P\not\red$, otherwise.

\section{Replication}

As mentioned before, it is known that replication (and hence
recursion) can be implemented in a higher-order process algebra
\cite{SangiorgiWalker}. As our first example of calculation with the
machinery thus far presented we give the construction explicitly in
the {\rhoc}.

\begin{eqnarray}
	D_{x} & := & \prefix{x}{y}{(\binpar{\outputp{x}{y}}{@{y}})} \nonumber\\
	\bangp_{x}{P} & := & \binpar{{x}!\langle{\binpar{D_{x}}{P}}\rangle}{D_{x}} \nonumber
\end{eqnarray}

\begin{eqnarray}
	\bangp_{x}{P} & & \nonumber\\
	=
	& {x}!\langle{(\prefix{x}{y}{(\outputp{x}{y} | @{y})) | P}}\rangle 
	      | \prefix{x}{y}{(\outputp{x}{y} | @{y})} & \nonumber\\
	\red
	& (\outputp{x}{y} | @{y})\substn{\quotep{(\prefix{x}{y}{(@{y} | \outputp{x}{y})) | P}}}{y} & \nonumber\\
	=
	& \outputp{x}{\quotep{(\prefix{x}{y}{(\outputp{x}{y} | @{y})) | P}}}
	  | {(\prefix{x}{y}{(\outputp{x}{y} | @{y})) | P}} & \nonumber\\
	\red
	& \ldots & \nonumber\\
	\red^*
	& P | P | \ldots & \nonumber
\end{eqnarray}

Of course, this encoding, as an implementation, runs away, unfolding
$\bangp{P}$ eagerly. A lazier and more implementable replication
operator, restricted to input-guarded processes, may be obtained as follows.

\begin{eqnarray}
\bangp{\prefix{u}{v}{P}} 
	:= 
	\binpar{\lift{x}{\prefix{u}{v}{(\binpar{D(x)}{P})}}}{D(x)} \nonumber
\end{eqnarray}

\begin{remark}
  Note that the lazier definition still does not deal with summation
  or mixed summation (i.e. sums over input and output). The reader is
  invited to construct definitions of replication that deal with these
  features. 

  Further, the definitions are parameterized in a name, $x$. Can you,
  gentle reader, make a definition that eliminates this parameter and
  guarantees no accidental interaction between the replication
  machinery and the process being replicated -- i.e. no accidental
  sharing of names used by the process to get its work done and the
  name(s) used by the replication to effect copying. This latter
  revision of the definition of replication is crucial to obtaining
  the expected identity $!!P \sim !P$.
\end{remark}

\begin{remark}\label{rem:paradoxical_combinator}
  The reader familiar with the lambda calculus will have noticed the
  similarity between $D$ and the paradoxical combinator.

  [Ed. note: the existence of this seems to suggest we have to be more
  restrictive on the set of processes and names we admit if we are to
  support no-cloning.]
\end{remark}

\subsubsection{Bisimulation}

The computational dynamics gives rise to another kind of equivalence,
the equivalence of computational behavior. As previously mentioned
this is typically captured \emph{via} some form of bisimulation.

% The notion we use in this paper is weak barbed bisimulation
% \cite{milner91polyadicpi}.

The notion we use in this paper is derived from weak barbed
bisimulation \cite{milner91polyadicpi}. 

\begin{definition}
An \emph{observation relation}, $\downarrow_{\mathcal N}$, over a set
of names, $\mathcal N$, is the smallest relation satisfying the rules
below.

\infrule[Out-barb]{y \in {\mathcal N}, \; x \nameeq y}
		  {\outputp{x}{v} \downarrow_{\mathcal N} x}
\infrule[Par-barb]{\mbox{$P\downarrow_{\mathcal N} x$ or $Q\downarrow_{\mathcal N} x$}}
		  {\binpar{P}{Q} \downarrow_{\mathcal N} x}

We write $P \Downarrow_{\mathcal N} x$ if there is $Q$ such that 
$P \wred Q$ and $Q \downarrow_{\mathcal N} x$.
\end{definition}

\begin{definition}
%\label{def.bbisim}
An  ${\mathcal N}$-\emph{barbed bisimulation} over a set of names, ${\mathcal N}$, is a symmetric binary relation 
${\mathcal S}_{\mathcal N}$ between agents such that $P\rel{S}_{\mathcal N}Q$ implies:
\begin{enumerate}
\item If $P \red P'$ then $Q \wred Q'$ and $P'\rel{S}_{\mathcal N} Q'$.
\item If $P\downarrow_{\mathcal N} x$, then $Q\Downarrow_{\mathcal N} x$.
\end{enumerate}
$P$ is ${\mathcal N}$-barbed bisimilar to $Q$, written
$P \wbbisim_{\mathcal N} Q$, if $P \rel{S}_{\mathcal N} Q$ for some ${\mathcal N}$-barbed bisimulation ${\mathcal S}_{\mathcal N}$.
\end{definition}

$\mathcal{R} \subseteq \pi \times \pi$

$P \mathcal{R} Q => \forall P'. P \red P' \Rightarrow \exists Q'. Q \red Q', P' \mathcal{R} Q'$

$P \vdash x \Rightarrow Q \vdash x$

\begin{mathpar}
  \inferrule*[lab=Out-barb]{x \nameeq y}{{y}!\langle{Q}\rangle \vdash x}
  \and
  \inferrule*[lab=Par-barb]{\mbox{$P\vdash x$ or $Q\vdash x$}}{\binpar{P}{Q} \vdash x}
\end{mathpar}

\subsubsection{Contexts}

One of the principle advantages of computational calculi like the
$\pi$-calculus is a well-defined notion of context,
contextual-equivalence and a correlation between
contextual-equivalence and notions of bisimulation. The notion of
context allows the decomposition of a process into (sub-)process and
its syntactic environment, its context. Thus, a context may be
thought of as a process with a ``hole'' (written $\Box$) in it. The
application of a context $M$ to a process $P$, written $M[P]$, is
tantamount to filling the hole in $M$ with $P$. In this paper we do
not need the full weight of this theory, but do make use of the notion
of context in the proof the main theorem. 

\begin{mathpar}
  \inferrule* [lab=summation] {} {{M_{M},M_{N}} \bc \Box \;|\; x.M_{A} \;|\; M_{M}+M_{N}}
  \and
  \inferrule* [lab=agent] {} {{M_{A}} \bc (\vec{x})M_{P} \;| \; \clift{P_0,\ldots,M_{P},\ldots,P_N}}
  \and \\
  \inferrule* [lab=process] {} {{M_{P}} \bc M_{N} \;| \;P|M_{P} }
\end{mathpar} 

\begin{mathpar}
  \inferrule* [lab=sychronization] {} {M_{N} \bc \Box \;|\; x?M_{F} \;|\; x!M_{C}}
  \and
  \inferrule* [lab=abstraction] {} {{M_{F}} \bc (x)M_{P} }
  \and
  \inferrule* [lab=concretion] {} {{M_{C}} \bc \langle M_{P} \rangle }
  \and \\
  \inferrule* [lab=process] {} {{M_{P}} \bc M_{N} \;| \;P|M_{P} }
\end{mathpar}

\begin{definition}[contextual application] Given a context $M$, and
  process $P$, we define the \emph{contextual application}, $M[P] :=
  M\{P/\Box\}$. That is, the contextual application of M to P is the
  substitution of $P$ for $\Box$ in $M$.
\end{definition}

$\meaningof{-} : L \to \mathcal{P}(\pi)$

\begin{mathpar}
  \inferrule* [lab=collection] {} {\meaningof{true} = \pi, \and \meaningof{~E} = \pi \setminus \meaningof{E}, \and \meaningof{E_{1} \& E_{2}} = \meaningof{E_{1}} \cap \meaningof{E_{2}}}
\end{mathpar}

\begin{mathpar}
  \inferrule* [lab=structure] {} {\meaningof{0} = \{ P \in \pi | P \equiv 0 \}, \and \\ \meaningof{E_1 | E_2} = \{ P \in \pi | P \equiv P_{1} | P_{2}, P_{1} \in \meaningof{E_{1}}, P_{2} \in \meaningof{E_2}\} }
\end{mathpar}

\begin{mathpar}
 \inferrule* [lab=behavior] {} {\meaningof{\langle a?b \rangle E} = \{ P \in \pi | P \equiv Q | u?(y)P', \\ \and \\\\ \and \\ \;\;\; u \in \meaningof{a}, \forall z.P'\{z/y\} \in \meaningof{E\{z/b\}}\}, \and \\ \meaningof{a!E} = \{ P \in \pi | P \equiv Q | x!\langle P' \rangle, x \in \meaningof{a} P' \in \meaningof{E}\} }
\end{mathpar}

\begin{mathpar}
 \inferrule* [lab=nominal] {} {\meaningof{\quotep{E}} = \{ \quotep{P} \in \quotep{\pi} | P \in \meaningof{E} \}, \and \meaningof{\quotep{P}} = \{ \quotep{Q} \in \quotep{\pi} | P \equiv Q \} \and \\ \meaningof{@\quotep{E}} = \{ P \in \pi | P \equiv @x, x \in \meaningof{E} \}}
\end{mathpar}

\begin{eqnarray*}
  \\
  \meaningof{-} : TS \to ST
\end{eqnarray*}

\begin{eqnarray*}
  \\
  L : TS \to ST
\end{eqnarray*}

\begin{eqnarray*}
  \\
  P \models E \iff P \in \meaningof{E}
\end{eqnarray*}

\begin{eqnarray*}
  P \approx_{L} Q \iff \forall E \in L. P \models E \iff Q \models E
\end{eqnarray*}

\begin{eqnarray*}
  P \approx_{K} Q
\end{eqnarray*}

\begin{eqnarray*}
  P \approx Q
\end{eqnarray*}

$\approx_{K} = \approx = \approx_{L}$

\subsubsection{Contextual duality}

Note that contexts extend the quotation operation to a family of
operations from processes to names. Given a context, $M$, we can
define a \emph{nominal context}, $\quotep{M}$ by $\quotep{M}[P] :=
\quotep{M[P]}$. To foreshadow what is to come we observe that these
operations enjoy a duality with processes very much like the duality
between vectors and maps from vectors to scalars.

Further, because the calculus is essentially higher-order, we have a
correspondence between contexts and processes. More specifically,
given a name $x$ and a context $M$ we can construct $M^{*}_{x}$ such
that 

\begin{mathpar}
  M^{*}_{x} | \lift{x}{P} \red M[P]
\end{mathpar}

namely,

\begin{mathpar}
  M^{*}_{x} := x?(u).M[\dropn{u}]
\end{mathpar}

The dependence of $M^{*}_{x}$ on a name makes it an abstraction, 

\begin{mathpar}
  M^{*} := (x)x?(u).M[\dropn{u}]
\end{mathpar}

\subsection{Additional notation}

It will sometimes be convenient to denote the process a name
quotes. We already have the notation $x = \quotep{P}$, but it will be
convenient to introduce an alternate notation, $\procn{x}$, when we
want to emphasize the connection to the use of the name. Note that, by
virtue of name equivalence, $\quotep{\procn{x}} \nameeq x$; so, the
notation is consistent with previous definitions.

Further, because names have structure it is possible to effect
substitutions on the basis of that structure. This means we need to
upgrade our notation for substitutions, which we accomplish by
adapting comprehension notation. Thus,

\begin{mathpar}
  P\{ y / x : x \in S \}
\end{mathpar}

is interpreted to mean the process derived from P by replacing (in a
capture-avoiding manner) each occurrence of $x$ in $S$ by $y$. For example,

\begin{mathpar}
  P\{ \quotep{\procn{x}|\procn{x}} / x : x \in \freenames{P} \}
\end{mathpar}

will replace each (occurrence) of a free name $x$ in $P$ by
$\quotep{\procn{x}|\procn{x}}$.

Also, we will avail ourselves of the notation $x^{L}$ and $x^{R}$ to
denote injections of a name into disjoint copies of the name
space. There are numerous ways to accomplish this. One example can be
found in \cite{MeredithR05}. This notation overloads to vectors of
names: $\vec{x}^{\pi} := (x_{i}^{\pi} \; : \; 0 \leq i < |\vec{x}| )$ where $\pi \in \{L,R\}$.

We also use $P^{\Box} := P|\Box$.

In \cite{MeredithR05} an interpretation of the new operator is
given. It turns out that there are several possible interpretations
all enjoying the requisite algebraic properties of the operator (see
\cite{milner91polyadicpi}). We will therefore make liberal use of
$(\nu\; \vec{x})P$.

% subsection the_syntax_and_semantics_of_the_notation_system (end)   

\input{qm2pi.qmops} 

\input{qm2pi.sterngerlach} 

\input{qm2pi.metric} 

% section concurrent_process_calculi (end)

%\input{qm2pi.proofsketch}

% section proof sketch (end)

%\input{qm2pi.slviaknots} 

% section spatial logic via knots (end)

\input{qm2pi.conclusion}

% section conclusion (end)

%\input{qm2pi.dtcodes} 

% section wiring algorithm (end)

\input{qm2pi.ack} 

% section acknowledgments (end)

\newpage


\bibliographystyle{plain}   
\bibliography{../../biblios/main.bib}

\input{qm2pi.rhodetails}

\end{document}

 

% section acknowledgments (end)

\newpage


\bibliographystyle{plain}   
\bibliography{../../biblios/main.bib}

\documentclass[12pt]{llncs}
%\documentclass{jktr}

\usepackage[pdftex]{hyperref}                   
\usepackage {listings}
\usepackage {mathpartir}
\usepackage{bcprules}
%\usepackage{listings}
                       
\usepackage{graphicx} 
%\usepackage[margins=2.5cm,nohead,nofoot]{geometry}
%\usepackage{geometry}
\usepackage{amsfonts}
\usepackage{amstext}
\usepackage{latexsym}
\usepackage{amssymb}
\usepackage{color}


%\include{myPreamble}
\include{qm2pi.local} 

%\ifpdf
%\usepackage[pdftex]{graphicx}
%\else
%\usepackage{graphicx}
%\fi

 % \ifpdf
%  \usepackage{pdfsync}
%  \if


%\title{Brief Article}
%\author{David F. Snyder}
%\author{L.G. Meredith}

%\address{Dept. of Math., Texas State University--San Marcos, San Marcos, TX 78666}
       
\pagestyle{empty}


\begin{document}

\lstset{language=[Objective]Caml,frame=shadowbox}

\input{qm2pi.front}

% section front matter (end)

\input{qm2pi.intro} 
 
% section introduction (end)

% \input{qm2pi.knotations} 

% section notation (end)

\input{qm2pi.process.calculi} 

% section concurrent_process_calculi_and_spatial_logics_ (end)
    
%\input{qm2pi.knots2pi} 

%\input{qm2pi.trefoil} 

%\input{qm2pi.mainthm} 

% subsection basic_interpretation (end)

%\input{qm2pi.rho.presentation} 
\subsection{The syntax and semantics of the notation system}\label{sub:the_syntax_and_semantics_of_the_notation_system} % (fold)

We now summarize a technical presentation of the calculus that
embodies our theory of dynamics. The typical presentation of such a
calculus follows the style of giving generators and relations on
them. The grammar, below, describing term constructors, freely
generates the set of processes, $\Proc$. This set is then quotiented
by a relation known as structural congruence and it is over this set
that the notion of dynamics is expressed. This presentation is
essentially that of \cite{MeredithR05} with the addition of
polyadicity and summation. For readability we have relegated some of
the technical subtleties to an appendix.

\subsubsection{Process grammar}\label{subsub:process_grammar}

\begin{mathpar}
  \inferrule* [lab=synchronization] {} {{M} \bc \pzero \;|\; x?F \;|\; x!C }
  \and
  \inferrule* [lab=abstraction] {} {{F} \bc (x)P}
  \and
  \inferrule* [lab=concretion] {} {{C} \bc \langle Q \rangle}
  \and
  \inferrule* [lab=process] {} {{P,Q} \bc M \;| \;P|Q \;|\; @{x}}
  \and
  \inferrule* [lab=name] {} {{x} \bc \quotep{P}}
\end{mathpar} 

Note that $\vec{x}$ (resp. $\vec{P}$) denotes a vector of names
(resp. processes) of length $|\vec{x}|$ (resp. $|\vec{P}|$). We adopt
the following useful abbreviations.

\begin{mathpar}
   x?(\vec{y}).P := x.(\vec{y})P \and  x\clift{\vec{P}} := x.\clift{\vec{P}}
   \and x!(y) := \lift{x}{\dropn{y}}
   \and \Pi_{i=0}^{n-1}P_i := P_0 | \ldots | P_{n-1}
\end{mathpar}

\subsubsection{Structural congruence}

\paragraph{Free and bound names and alpha-equivalence.} At the
core of structural equivalence is alpha-equivalence which identifies
process that are the same up to a change of variable. Formally, we
recognize the distinction between free and bound names. The free names
of a process, $\freenames{P}$, may be calculated recursively as
follows:

\begin{mathpar}
\freenames{\pzero} := \emptyset
  \and \\
  \freenames{x?(y).P} := \{ x \} \cup (\freenames{P} \setminus \{ y \})
  \and 
  \freenames{x!\langle P \rangle} := \{ x \} \cup \{ P \} 
  \and \\
  \freenames{P|Q} := \freenames{P} \cup \freenames{Q}
  \and \\
  \freenames{@{x}} := \{ x \}
\end{mathpar}

$\pi$
$\quotep{\pi}$

$\freenames{-} : \pi \to \mathcal{P}(\quotep{\pi})$

\begin{eqnarray*}
  \freenames{\pzero} & := & \emptyset \\
  \freenames{x?(y).P} & := & \{ x \} \cup (\freenames{P} \setminus \{ y \}) \\
  \freenames{x!\langle P \rangle} & := & \{ x \} \cup \{ P \} \\
  \freenames{P|Q} & := & \freenames{P} \cup \freenames{Q} \\
  \freenames{\dropn{x}} & := & \{ x \}
\end{eqnarray*}

The bound names of a process, $\boundnames{P}$, are those names occurring in $P$
that are not free. For example, in $x?(y).0$, the name $x$ is free, while $y$ is bound.

\begin{mathpar}
  \inferrule* [lab=monoidal-laws] {} { P|Q \equiv Q|P \and P|0 \equiv P \and P|(Q|R) \equiv (P|Q)|R }
\end{mathpar}

\begin{mathpar}
  \inferrule* [lab=alpha-equivalence] {} { (x)P \equiv (y)P\{y/x\} \and y \not\in \freenames{P} }
\end{mathpar}

\begin{definition}
Then two processes, $P,Q$, are alpha-equivalent if $P = Q\{\vec{y}/\vec{x}\}$ for
some $\vec{x} \in \boundnames{Q},\vec{y} \in \boundnames{P}$, where $Q\{\vec{y}/\vec{x}\}$
denotes the capture-avoiding substitution of $\vec{y}$ for $\vec{x}$ in $Q$.
\end{definition}

\begin{definition}
  The {\em structural congruence} \cite{SangiorgiWalker} , $\equiv$,
  between processes is the least congruence containing
  alpha-equivalence, satisfying the abelian monoid laws
  (associativity, commutativity and $\pzero$ as identity) for parallel
  composition $|$ and for summation $+$.
\end{definition}

\subsection{Name equivalence}

We take name equivalence, written $\nameeq$, to be the smallest
equivalence relation generated by the following rules.

\begin{mathpar}
\inferrule*[lab=Quote-drop]
{ }
{ \quotep{@{x}} \nameeq x }

\inferrule*[lab=Struct-equiv]
{ P \scong Q }
{ \quotep{P} \nameeq \quotep{Q} }
\end{mathpar}

The astute reader will have noticed that the mutual recursion of names
and processes imposes a mutual recursion on alpha-equivalence and
structural equivalence via name-equivalence. Fortunately, all of this
works out pleasantly and we may calculate in the natural way, free of
concern. The reader interested in the details is referred to the
appendix \ref{appendix:rho_details}.

\subsection{Substitution}

We use $\Proc$ for the set of processes, $\QProc$ for the set of
names, and $\id{\{}\vec{y} / \vec{x} \id{\}}$ to denote partial maps,
$s : \QProc \rightarrow \QProc$. A map, $s$ lifts, uniquely, to a map
on process terms, $\widehat{s} : \Proc \rightarrow \Proc$ by the
following equations.

\begin{mathpar}
  (0) \psubstp{Q}{P} := 0 \\
  (R \juxtap S) \psubstp{Q}{P}
  :=    
  (R)\psubstp{Q}{P} \juxtap (S) \psubstp{Q}{P} \\
  (x?(y).R) \psubstp{Q}{P}    
  :=    
  (x)\substp{Q}{P} (z)\concat( (R \psubstn{z}{y}) \psubstp{Q}{P} ) \\
  (\lift{x}{R}) \psubstp{Q}{P}  
  :=
  \lift{(x)\substp{Q}{P}}{ R \psubstp{Q}{P} } \\
%   (\dropn{x})  \psubstp{Q}{P}       
%   := 
%   \left\{ 
%     \begin{array}{ccc} 
%       \dropn{\quotep{Q}} & & x \nameeq \quotep{P} \\
%       \dropn{x} & & otherwise \\
%     \end{array}
%   \right. 
  (\dropn{x})  \psubstp{Q}{P}       
  := 
  \left\{ 
    \begin{array}{ccc} 
      Q & & x \nameeq \quotep{P} \\
      \dropn{x} & & otherwise \\
    \end{array}
  \right.
\end{mathpar}
 

where

\begin{eqnarray}
  (x)\id{\{} \lpquote Q \rpquote / \lpquote P \rpquote \id{\}}            = 
  \left\{ 
    \begin{array}{ccc}
      \lpquote Q \rpquote & & x \nameeq \lpquote P \rpquote \\
      x & & otherwise \\
    \end{array}
  \right. \nonumber
\end{eqnarray}

and $z$ is chosen distinct from $\quotep{P}$, $\quotep{Q}$, the free
names in $Q$, and all the names in $R$. Our $\alpha$-equivalence will
be built in the standard way from this substitution.

\begin{remark}\label{rem:no_self_referential_names}
  One consequence of these definitions is that $\forall P. \quotep{P}
  \not\in \freenames{P}$.
\end{remark}

\subsection{ Dynamic quote: an example }

Anticipating something of what's to come, consider applying the
substitution, $\widehat{\id{\{}u / z \id{\}}}$, to the following pair
of processes, $\lift{w}{y!(z)}$ and $w[ \lpquote y!(z) \rpquote ]$.

\begin{eqnarray}
	\lift{w}{y!(z)}\widehat{\id{\{}u / z \id{\}}}
		& = &
		\lift{w}{y!(u)} \nonumber\\
	w[ \lpquote y!(z) \rpquote ] \widehat{ \id{\{}u / z \id{\}} }
		& = &
		w[ \lpquote y!(z) \rpquote ] \nonumber
\end{eqnarray}

Because the body of the process between quotes is impervious to
substitution, we get radically different answers. In fact, by
examining the first process in an input context,
e.g. $x?(z).\lift{w}{y!(z)}$, we see that the process under the lift
operator may be shaped by prefixed inputs binding a name inside it. In
this sense, the lift operator will be seen as a way to dynamically
construct processes before reifying them as names.

Finally equipped with these standard features we can present the
dynamics of the calculus.

\subsubsection{Operational semantics} 

Finally, we introduce the computational dynamics. What marks these
algebras as distinct from other more traditionally studied algebraic
structures, e.g. vector spaces or polynomial rings, is the manner in
which dynamics is captured. In traditional structures, dynamics is typically
expressed through morphisms between such structures, as in linear maps
between vector spaces or morphisms between rings. In algebras
associated with the semantics of computation, the dynamics is
expressed as part of the algebraic structure itself, through a
reduction reduction relation typically denoted by $\red$. Below, we
give a recursive presentation of this relation for the calculus used
in the encoding.

$\red \subseteq \pi \times \pi$
$\red : \pi \to \mathcal{P}(\pi)$

\begin{mathpar}
  \inferrule* [lab=Comm] { \textsf{match}( x_{src}, x_{trgt} ) } { x_{trgt}?(y)P \; | \; x_{src}!\langle {Q} \rangle \red P\{\quotep{Q}/y}\} }
  \and \\
  \inferrule* [lab=Par] {{P} \red {P}'} {{{P} | {Q}} \red {{P}' | {Q}}}
  \and
  \inferrule* [lab=Equiv]{{{P} \scong {P}'} \andalso {{P}' \red {Q}'} \andalso {{Q}' \scong {Q}}}{{P} \red {Q}}
\end{mathpar}

\begin{eqnarray*}
  match_{\equiv} (\quotep{P},\quotep{Q}) & := & P \equiv Q \\
  match_{\dagger}(\quotep{P},\quotep{Q}) & := & \forall R. P|Q \red^{*} R => R \red^{*} 0 \\
  match_{K}(\quotep{P},\quotep{Q}) & := & K \mbox{ for some context } K
\end{eqnarray*}

$u?(x)P | u!\langle Q \rangle \red P\{\quotep{Q}/x\}$

%We write $\wred$ for $\red^*$, and $P\red$ if $\exists Q $ such that $ P \red Q$.
We write $P\red$ if $\exists Q $ such that $ P \red Q$ and $P\not\red$, otherwise.

\section{Replication}

As mentioned before, it is known that replication (and hence
recursion) can be implemented in a higher-order process algebra
\cite{SangiorgiWalker}. As our first example of calculation with the
machinery thus far presented we give the construction explicitly in
the {\rhoc}.

\begin{eqnarray}
	D_{x} & := & \prefix{x}{y}{(\binpar{\outputp{x}{y}}{@{y}})} \nonumber\\
	\bangp_{x}{P} & := & \binpar{{x}!\langle{\binpar{D_{x}}{P}}\rangle}{D_{x}} \nonumber
\end{eqnarray}

\begin{eqnarray}
	\bangp_{x}{P} & & \nonumber\\
	=
	& {x}!\langle{(\prefix{x}{y}{(\outputp{x}{y} | @{y})) | P}}\rangle 
	      | \prefix{x}{y}{(\outputp{x}{y} | @{y})} & \nonumber\\
	\red
	& (\outputp{x}{y} | @{y})\substn{\quotep{(\prefix{x}{y}{(@{y} | \outputp{x}{y})) | P}}}{y} & \nonumber\\
	=
	& \outputp{x}{\quotep{(\prefix{x}{y}{(\outputp{x}{y} | @{y})) | P}}}
	  | {(\prefix{x}{y}{(\outputp{x}{y} | @{y})) | P}} & \nonumber\\
	\red
	& \ldots & \nonumber\\
	\red^*
	& P | P | \ldots & \nonumber
\end{eqnarray}

Of course, this encoding, as an implementation, runs away, unfolding
$\bangp{P}$ eagerly. A lazier and more implementable replication
operator, restricted to input-guarded processes, may be obtained as follows.

\begin{eqnarray}
\bangp{\prefix{u}{v}{P}} 
	:= 
	\binpar{\lift{x}{\prefix{u}{v}{(\binpar{D(x)}{P})}}}{D(x)} \nonumber
\end{eqnarray}

\begin{remark}
  Note that the lazier definition still does not deal with summation
  or mixed summation (i.e. sums over input and output). The reader is
  invited to construct definitions of replication that deal with these
  features. 

  Further, the definitions are parameterized in a name, $x$. Can you,
  gentle reader, make a definition that eliminates this parameter and
  guarantees no accidental interaction between the replication
  machinery and the process being replicated -- i.e. no accidental
  sharing of names used by the process to get its work done and the
  name(s) used by the replication to effect copying. This latter
  revision of the definition of replication is crucial to obtaining
  the expected identity $!!P \sim !P$.
\end{remark}

\begin{remark}\label{rem:paradoxical_combinator}
  The reader familiar with the lambda calculus will have noticed the
  similarity between $D$ and the paradoxical combinator.

  [Ed. note: the existence of this seems to suggest we have to be more
  restrictive on the set of processes and names we admit if we are to
  support no-cloning.]
\end{remark}

\subsubsection{Bisimulation}

The computational dynamics gives rise to another kind of equivalence,
the equivalence of computational behavior. As previously mentioned
this is typically captured \emph{via} some form of bisimulation.

% The notion we use in this paper is weak barbed bisimulation
% \cite{milner91polyadicpi}.

The notion we use in this paper is derived from weak barbed
bisimulation \cite{milner91polyadicpi}. 

\begin{definition}
An \emph{observation relation}, $\downarrow_{\mathcal N}$, over a set
of names, $\mathcal N$, is the smallest relation satisfying the rules
below.

\infrule[Out-barb]{y \in {\mathcal N}, \; x \nameeq y}
		  {\outputp{x}{v} \downarrow_{\mathcal N} x}
\infrule[Par-barb]{\mbox{$P\downarrow_{\mathcal N} x$ or $Q\downarrow_{\mathcal N} x$}}
		  {\binpar{P}{Q} \downarrow_{\mathcal N} x}

We write $P \Downarrow_{\mathcal N} x$ if there is $Q$ such that 
$P \wred Q$ and $Q \downarrow_{\mathcal N} x$.
\end{definition}

\begin{definition}
%\label{def.bbisim}
An  ${\mathcal N}$-\emph{barbed bisimulation} over a set of names, ${\mathcal N}$, is a symmetric binary relation 
${\mathcal S}_{\mathcal N}$ between agents such that $P\rel{S}_{\mathcal N}Q$ implies:
\begin{enumerate}
\item If $P \red P'$ then $Q \wred Q'$ and $P'\rel{S}_{\mathcal N} Q'$.
\item If $P\downarrow_{\mathcal N} x$, then $Q\Downarrow_{\mathcal N} x$.
\end{enumerate}
$P$ is ${\mathcal N}$-barbed bisimilar to $Q$, written
$P \wbbisim_{\mathcal N} Q$, if $P \rel{S}_{\mathcal N} Q$ for some ${\mathcal N}$-barbed bisimulation ${\mathcal S}_{\mathcal N}$.
\end{definition}

$\mathcal{R} \subseteq \pi \times \pi$

$P \mathcal{R} Q => \forall P'. P \red P' \Rightarrow \exists Q'. Q \red Q', P' \mathcal{R} Q'$

$P \vdash x \Rightarrow Q \vdash x$

\begin{mathpar}
  \inferrule*[lab=Out-barb]{x \nameeq y}{{y}!\langle{Q}\rangle \vdash x}
  \and
  \inferrule*[lab=Par-barb]{\mbox{$P\vdash x$ or $Q\vdash x$}}{\binpar{P}{Q} \vdash x}
\end{mathpar}

\subsubsection{Contexts}

One of the principle advantages of computational calculi like the
$\pi$-calculus is a well-defined notion of context,
contextual-equivalence and a correlation between
contextual-equivalence and notions of bisimulation. The notion of
context allows the decomposition of a process into (sub-)process and
its syntactic environment, its context. Thus, a context may be
thought of as a process with a ``hole'' (written $\Box$) in it. The
application of a context $M$ to a process $P$, written $M[P]$, is
tantamount to filling the hole in $M$ with $P$. In this paper we do
not need the full weight of this theory, but do make use of the notion
of context in the proof the main theorem. 

\begin{mathpar}
  \inferrule* [lab=summation] {} {{M_{M},M_{N}} \bc \Box \;|\; x.M_{A} \;|\; M_{M}+M_{N}}
  \and
  \inferrule* [lab=agent] {} {{M_{A}} \bc (\vec{x})M_{P} \;| \; \clift{P_0,\ldots,M_{P},\ldots,P_N}}
  \and \\
  \inferrule* [lab=process] {} {{M_{P}} \bc M_{N} \;| \;P|M_{P} }
\end{mathpar} 

\begin{mathpar}
  \inferrule* [lab=sychronization] {} {M_{N} \bc \Box \;|\; x?M_{F} \;|\; x!M_{C}}
  \and
  \inferrule* [lab=abstraction] {} {{M_{F}} \bc (x)M_{P} }
  \and
  \inferrule* [lab=concretion] {} {{M_{C}} \bc \langle M_{P} \rangle }
  \and \\
  \inferrule* [lab=process] {} {{M_{P}} \bc M_{N} \;| \;P|M_{P} }
\end{mathpar}

\begin{definition}[contextual application] Given a context $M$, and
  process $P$, we define the \emph{contextual application}, $M[P] :=
  M\{P/\Box\}$. That is, the contextual application of M to P is the
  substitution of $P$ for $\Box$ in $M$.
\end{definition}

$\meaningof{-} : L \to \mathcal{P}(\pi)$

\begin{mathpar}
  \inferrule* [lab=collection] {} {\meaningof{true} = \pi, \and \meaningof{~E} = \pi \setminus \meaningof{E}, \and \meaningof{E_{1} \& E_{2}} = \meaningof{E_{1}} \cap \meaningof{E_{2}}}
\end{mathpar}

\begin{mathpar}
  \inferrule* [lab=structure] {} {\meaningof{0} = \{ P \in \pi | P \equiv 0 \}, \and \\ \meaningof{E_1 | E_2} = \{ P \in \pi | P \equiv P_{1} | P_{2}, P_{1} \in \meaningof{E_{1}}, P_{2} \in \meaningof{E_2}\} }
\end{mathpar}

\begin{mathpar}
 \inferrule* [lab=behavior] {} {\meaningof{\langle a?b \rangle E} = \{ P \in \pi | P \equiv Q | u?(y)P', \\ \and \\\\ \and \\ \;\;\; u \in \meaningof{a}, \forall z.P'\{z/y\} \in \meaningof{E\{z/b\}}\}, \and \\ \meaningof{a!E} = \{ P \in \pi | P \equiv Q | x!\langle P' \rangle, x \in \meaningof{a} P' \in \meaningof{E}\} }
\end{mathpar}

\begin{mathpar}
 \inferrule* [lab=nominal] {} {\meaningof{\quotep{E}} = \{ \quotep{P} \in \quotep{\pi} | P \in \meaningof{E} \}, \and \meaningof{\quotep{P}} = \{ \quotep{Q} \in \quotep{\pi} | P \equiv Q \} \and \\ \meaningof{@\quotep{E}} = \{ P \in \pi | P \equiv @x, x \in \meaningof{E} \}}
\end{mathpar}

\begin{eqnarray*}
  \\
  \meaningof{-} : TS \to ST
\end{eqnarray*}

\begin{eqnarray*}
  \\
  L : TS \to ST
\end{eqnarray*}

\begin{eqnarray*}
  \\
  P \models E \iff P \in \meaningof{E}
\end{eqnarray*}

\begin{eqnarray*}
  P \approx_{L} Q \iff \forall E \in L. P \models E \iff Q \models E
\end{eqnarray*}

\begin{eqnarray*}
  P \approx_{K} Q
\end{eqnarray*}

\begin{eqnarray*}
  P \approx Q
\end{eqnarray*}

$\approx_{K} = \approx = \approx_{L}$

\subsubsection{Contextual duality}

Note that contexts extend the quotation operation to a family of
operations from processes to names. Given a context, $M$, we can
define a \emph{nominal context}, $\quotep{M}$ by $\quotep{M}[P] :=
\quotep{M[P]}$. To foreshadow what is to come we observe that these
operations enjoy a duality with processes very much like the duality
between vectors and maps from vectors to scalars.

Further, because the calculus is essentially higher-order, we have a
correspondence between contexts and processes. More specifically,
given a name $x$ and a context $M$ we can construct $M^{*}_{x}$ such
that 

\begin{mathpar}
  M^{*}_{x} | \lift{x}{P} \red M[P]
\end{mathpar}

namely,

\begin{mathpar}
  M^{*}_{x} := x?(u).M[\dropn{u}]
\end{mathpar}

The dependence of $M^{*}_{x}$ on a name makes it an abstraction, 

\begin{mathpar}
  M^{*} := (x)x?(u).M[\dropn{u}]
\end{mathpar}

\subsection{Additional notation}

It will sometimes be convenient to denote the process a name
quotes. We already have the notation $x = \quotep{P}$, but it will be
convenient to introduce an alternate notation, $\procn{x}$, when we
want to emphasize the connection to the use of the name. Note that, by
virtue of name equivalence, $\quotep{\procn{x}} \nameeq x$; so, the
notation is consistent with previous definitions.

Further, because names have structure it is possible to effect
substitutions on the basis of that structure. This means we need to
upgrade our notation for substitutions, which we accomplish by
adapting comprehension notation. Thus,

\begin{mathpar}
  P\{ y / x : x \in S \}
\end{mathpar}

is interpreted to mean the process derived from P by replacing (in a
capture-avoiding manner) each occurrence of $x$ in $S$ by $y$. For example,

\begin{mathpar}
  P\{ \quotep{\procn{x}|\procn{x}} / x : x \in \freenames{P} \}
\end{mathpar}

will replace each (occurrence) of a free name $x$ in $P$ by
$\quotep{\procn{x}|\procn{x}}$.

Also, we will avail ourselves of the notation $x^{L}$ and $x^{R}$ to
denote injections of a name into disjoint copies of the name
space. There are numerous ways to accomplish this. One example can be
found in \cite{MeredithR05}. This notation overloads to vectors of
names: $\vec{x}^{\pi} := (x_{i}^{\pi} \; : \; 0 \leq i < |\vec{x}| )$ where $\pi \in \{L,R\}$.

We also use $P^{\Box} := P|\Box$.

In \cite{MeredithR05} an interpretation of the new operator is
given. It turns out that there are several possible interpretations
all enjoying the requisite algebraic properties of the operator (see
\cite{milner91polyadicpi}). We will therefore make liberal use of
$(\nu\; \vec{x})P$.

% subsection the_syntax_and_semantics_of_the_notation_system (end)   

\input{qm2pi.qmops} 

\input{qm2pi.sterngerlach} 

\input{qm2pi.metric} 

% section concurrent_process_calculi (end)

%\input{qm2pi.proofsketch}

% section proof sketch (end)

%\input{qm2pi.slviaknots} 

% section spatial logic via knots (end)

\input{qm2pi.conclusion}

% section conclusion (end)

%\input{qm2pi.dtcodes} 

% section wiring algorithm (end)

\input{qm2pi.ack} 

% section acknowledgments (end)

\newpage


\bibliographystyle{plain}   
\bibliography{../../biblios/main.bib}

\input{qm2pi.rhodetails}

\end{document}



\end{document}



% section proof sketch (end)

%\section{Unlikely characters: spatial logic for
  knots}\label{sub:characteristic_formulae} % (fold)

Associated to the mobile process calculi are a family of logics known
as the Hennessy-Milner logics. These logics typically enjoy a
semantics interpreting formulae as sets of processes that when
factored through the encoding outlined above allows an identification
of classes of knots with logical formulae. In the context of this
encoding the sub-family known as the spatial logics \cite{CairesC03}
\cite{CairesC04} \cite{Caires04} are of particular interest providing
several important features for expressing and reasoning about
properties (i.e. classes) of knots. We hint here at how this may be done.

%\begin{description}
%\item [structural connectives] 
\subsubsection{Structural connectives} The spatial logics enjoy
structural connectives corresponding, at the logical level, to the
parallel composition ($P | Q$) and new name ($(\nu \; x)P$)
connectives for processes. As illustrated in the examples below, these
connectives are extremely expressive given the shape of our encoding.
%\item [decideable satisfaction]

\subsubsection{Decideable satisfaction}
In \cite{Caires04} the satisfaction relation is shown to be decideable
for a rich class of processes. It further turns out that the image of
the our encoding is a proper subset of that class. This result
provides the basis for an algorithm by which to search for knots
enjoying a given property.
%\item [characteristic formulae]

\subsubsection{Characteristic formulae}
In the same paper \cite{Caires04} , Caires presents a means of calculating
characteristic formulae, selecting equivalence classes of processes
up to a pre--specified depth limit on the support set of names. Composed with our
encoding, this characteristic formula can be used to select
characteristic formulae for knots.
%\end{description}

\subsubsection{Spatial logic formulae}

The grammar below (segmented for comprehension) summarizes the syntax
of spatial logic formulae. We employ illustrative examples in the
sequel to provide an intuitive understanding of their meaning
referring the reader to \cite{Caires04} for a more detailed explication
of the semantics.

\begin{mathpar}
  \inferrule* [lab=boolean] {} {{A,B} \bc T \;|\; \neg A \;|\; A \wedge B \;|\; \eta = \eta'}
  \and
  \inferrule* [lab=spatial] {} {|\; \pzero \;|\; A | B \;|\; x \text{\textregistered} A \;|\; \forall x . A \;|\;  H x . A}
  \and
  \inferrule* [lab=behavioral] {} {|\; \alpha . A}
  \and 
  \inferrule* [lab=recursion] {} {|\; X(\vec{u}) \;|\; \mu X(\vec{u}) . A}
  \and
  \inferrule* [lab=action] {} {\alpha \bc \langle x?(\vec{y}) \rangle \;|\; \langle x!(\vec{y}) \rangle \;|\; \langle \tau \rangle}
  \and 
  \inferrule* [lab=name] {} {\eta \bc x \;|\; \tau}
\end{mathpar} 

% subsection characteristic_formulae (end)   	 

\subsection{Example formulae}\label{sub:example_formulae_} % (fold)

\subsubsection{Crossing as formula.}
% 
% \begin{align*}
%   \frac{d}{dx} \sin x &= \cos x 
%   & \frac{d}{dx} e^x &= e^x \\
%   \frac{d}{dx} \cos x &= - \sin x 
%   & \frac{d}{dx} \log x &= \frac{1}{x} \\
% \end{align*} 

\begin{align*}
 \mu C(x_{0},x_{1},y_{0},y_{1},u).&(\langle x_{0}?(z) \rangle(\langle u! \rangle\langle y_{1}!z \rangle C(x_{0},x_{1},y_{0},y_{1},u)) & \\
  & \wedge \langle y_{1}?(z) \rangle (\langle u! \rangle \langle x_{0}!z \rangle C(x_{0},x_{1},y_{0},y_{1},u)) & \\
  & \wedge \langle x_{1}?(z) \rangle (\langle u? \rangle \langle y_{0}!z \rangle C(x_{0},x_{1},y_{0},y_{1},u)) & \\
  & \wedge \langle y_{0}?(z) \rangle (\langle u? \rangle \langle x_{1}!z \rangle C(x_{0},x_{1},y_{0},y_{1},u))) &
\end{align*}

The lexicographical similarity between the shape of this formulae and
the shape of definition of the process representing a crossing reveals
the intuitive meaning of this formulae. It describes the capabilities
of a process that has the right to represent a crossing. For example
it picks out processes that may perform an input on the port $x_0$ in
its initial menu of capabilities. What differentiates the formula
from the process, however, is that the crossing process is the
smallest candidate to satisfy the formula. Infinitely many other
processes -- with internal behavior hidden behind this interface, so
to speak -- also satisfy this formula. Even this simple formula,
then, can be seen to open a new view onto knots, providing a
computational interpretation of \emph{virtual} knots.

Note that this formula is derived by hand. A similar formula can be
derived by employing Caires' calculation of characteristic formula
\cite{Caires04} to the process representing a crossing. In light of
this discussion, we let
$\meaningof{C}_{\phi}(x0,x1,y0,y1,u)$ denote a formula specifying the
dynamics we wish to capture of a crossing. To guarantee we preserve
the shape of the interface and minimal semantics we demand that
$\meaningof{C}_{\phi}(x0,x1,y0,y1,u) \Rightarrow
\textbf{C}(x0,x1,y0,y1,u)$ where $\textbf{C}(x0,x1,y0,y1,u)$ denotes
the formula above.
                            
\subsubsection{Crossing number constraints.}
The moral content of the context lemma (Lemma \ref{context}) is that the notion of
``locality'' in the Reidemeister moves is effectively captured by the
parallel composition operator of the process calculus. This intuition
extends through the logic. Given a formula,
$\meaningof{C}_{\phi}(x0,x1,y0,y1,u)$, we can use the structural
connectives to specify constraints on crossing numbers, such as at
least $n$ crossings, or exactly $n$ crossings.
\begin{mathpar}
  \inferrule* [lab=at-least-n] {} { K^{\geq n}_{\phi}(\vec{xs},\vec{ys}) := \Pi_{i=0}^{n-1} Hu . \meaningof{C}_{\phi}(xs_i,ys_i,u) | T }
  \and 
  \inferrule* [lab=exactly-n] {} { K^{= n}_{\phi}(\vec{xs},\vec{ys}) := \Pi_{i=0}^{n-1} Hu . \meaningof{C}_{\phi}(xs_i,ys_i,u) | \neg (\forall x_0,y_0,x_1,y_1,u . \meaningof{C}_{\phi}(x_0,y_0,x_1,y_1,u) | T) }
\end{mathpar}

To round out this section, recall that the encoding of an $n$-crossing
knot decomposes into a parallel composition of $n$ \emph{copies} of a
crossing process together with a wiring harness. To specify different
knot classes with the same crossing number amounts to specifying
logical constraints on the wiring harness. In the interest of space,
we defer examples to a forthcoming paper. Suffice it to say that both
the conditions ``alternating knot'' and ``contains the tangle
corresponding to 5/3'' are expressible. For example, it is possible to
calculate the characteristic formula of a process corresponding to the
tangle 5/3 and conjoin it into the classifying formula via the
composition connective of the logic.

Finally, we wish to observe that it is entirely within reason to
contemplate a more domain-specific version of spatial logic tailored
to the shape of processes in the image of the encoding. Such a
domain-specific logic would have a better claim to the title formal
language of knot properties.

% subsection example_formulae_ (end)

% section knots_as_processes (end) 

% section spatial logic via knots (end)

\section{Conclusions and future work}

\paragraph{Testing physical space}
You, gentle reader, may wonder why of all the theorems to be proved
given this set up we pick the one above. In some sense it's hardly
central to quantum mechanics. We see it as central in the sense that
it firmly establishes a notion of physical space arising from a notion
of the equivalence of behavior. Relating bisimulation to a metric is a
big step forward, but one is faced with interpreting the relationship
of that metric space to something more physical. Quantum mechanical
notions of ``physical'' space are still far from intuitive, but by
relating this idea of distance as testing to calculations that predict
physical circumstances we are making a not insignificant step forward
toward an understanding of the physical space we inhabit as
essentially dynamic.

\paragraph{Effectivity and simulation}
One of the observations we have yet to make is that the entire program
spelled out here is effective. We have built various interpreters for
the reflective calculus at work in this interpretation. In principle,
then, we can simulate quantum mechanics on a computer. The place where
the simulation may lose fidelity is the infinitely branching summation
for the annihilator.

In this connection i also want to point out that the evaluation style
calculation of the inner product puts the non-determinism of the
summation right at the heart of measurement. This suggests that
Milner's original reduction-based formulation of the dynamics of his
calculi in terms of sums was not just notationally suggestive of a
notion of measure-and-continue but captured some significant part of
the physics.

\paragraph{Quantum continuations}
In light of this last observation i want to point out that the
predominant account of quantum mechanics is missing a key aspect of a
truly compositional story of the physical situation. In a real lab,
when a measurement is made the observation can be made to feed into
another device that then makes another measurement conditioned on the
results of the first. This means that after the superposition was
collapsed the entire experimental set up remained in
superposition. While QM offers a means of writing this down it doesn't
quite line up well with the well-trodden formulation of computation
and continuation that we see so succinctly expressed in Milner's
calculi. This suggests that there might be advantages to this account
of dynamics waiting to be explored.

\paragraph{Quantum logic}
In this connection, we also note that by virtue of having the
Hennessy-Milner construction, we can pull the construction through the
interpretation of QM. This gives us a natural candidate for a quantum
logic that enjoys an extremely tight connection with it's domain of
interpretation, making the construction much less ad hoc (rather it is
the image of functor!).

\paragraph{Quantum probabiity}
i have questions about the basis of the interpretation of inner
product as probability amplitude. In particular, using which
axiomatization of probability theory does the notion of probability
amplitude earn the right to be so dubbed? In other words, where is the
proof that the operation for calculating a probability amplitude (and
then squaring) satisfies the axioms of what it means to calculate a
probability? Even if such a proof exists (i have yet to find it in the
literature), i wonder if it might not be possible to turn things on
their heads. Can we view the calculation of the probability amplitude
as an axiomatization of probability? If so, then the definition we
give for calculating probability amplitude may provide the basis for
an \emph{effective} theory of probability.

\paragraph{Quantum vs ``biological'' information}
Finally, i want to conclude with a more philosophical observation. At
a recent workshop in which QM was a predominant topic i noticed
something about quantum information. The speaker was giving a riveting
discussion of axiomatic QM and showing how properties of ``no
cloning'' and ``no deleting'' emerged as consequences of the
axiomatization. Theorems of this form are necessary to give us a sense
of confidence that our axioms characterize the physical theory. What
struck me, though, was that if quantum information is neither erasable
nor replicable it is markedly different from \emph{life}. Two of the
things we know about life is that

\begin{itemize}
  \item it ends;
  \item to gain some measure of persistence, to transcend it's
    finitude it is imminently copyable.
\end{itemize}

Both of these qualities are summarized succinctly in the aphorism: all
flesh is grass. For me these two kinds of ``information'' -- call them
quantum and biological -- are end points on a spectrum of strategies
for persistence. At one end, we have those curious entities that enjoy
uniqueness and permanence; at the other, we have those who in the face
of a certain end and an uncertain present make a go of passing
something on. To me one of the more remarkable aspects of the latter
strategy is that in the presence of noise (and certain features of
copying) we get a kind of dynamism, a chance for improvement against a
given persistent condition.

% subsection other_calculi_other_bisimulations_and_geometry_as_behavior (end)




% section conclusion (end)

%\documentclass[12pt]{llncs}
%\documentclass{jktr}

\usepackage[pdftex]{hyperref}                   
\usepackage {listings}
\usepackage {mathpartir}
\usepackage{bcprules}
%\usepackage{listings}
                       
\usepackage{graphicx} 
%\usepackage[margins=2.5cm,nohead,nofoot]{geometry}
%\usepackage{geometry}
\usepackage{amsfonts}
\usepackage{amstext}
\usepackage{latexsym}
\usepackage{amssymb}
\usepackage{color}


%\include{myPreamble}
\documentclass[12pt]{llncs}
%\documentclass{jktr}

\usepackage[pdftex]{hyperref}                   
\usepackage {listings}
\usepackage {mathpartir}
\usepackage{bcprules}
%\usepackage{listings}
                       
\usepackage{graphicx} 
%\usepackage[margins=2.5cm,nohead,nofoot]{geometry}
%\usepackage{geometry}
\usepackage{amsfonts}
\usepackage{amstext}
\usepackage{latexsym}
\usepackage{amssymb}
\usepackage{color}


%\include{myPreamble}
\include{qm2pi.local} 

%\ifpdf
%\usepackage[pdftex]{graphicx}
%\else
%\usepackage{graphicx}
%\fi

 % \ifpdf
%  \usepackage{pdfsync}
%  \if


%\title{Brief Article}
%\author{David F. Snyder}
%\author{L.G. Meredith}

%\address{Dept. of Math., Texas State University--San Marcos, San Marcos, TX 78666}
       
\pagestyle{empty}


\begin{document}

\lstset{language=[Objective]Caml,frame=shadowbox}

\input{qm2pi.front}

% section front matter (end)

\input{qm2pi.intro} 
 
% section introduction (end)

% \input{qm2pi.knotations} 

% section notation (end)

\input{qm2pi.process.calculi} 

% section concurrent_process_calculi_and_spatial_logics_ (end)
    
%\input{qm2pi.knots2pi} 

%\input{qm2pi.trefoil} 

%\input{qm2pi.mainthm} 

% subsection basic_interpretation (end)

%\input{qm2pi.rho.presentation} 
\subsection{The syntax and semantics of the notation system}\label{sub:the_syntax_and_semantics_of_the_notation_system} % (fold)

We now summarize a technical presentation of the calculus that
embodies our theory of dynamics. The typical presentation of such a
calculus follows the style of giving generators and relations on
them. The grammar, below, describing term constructors, freely
generates the set of processes, $\Proc$. This set is then quotiented
by a relation known as structural congruence and it is over this set
that the notion of dynamics is expressed. This presentation is
essentially that of \cite{MeredithR05} with the addition of
polyadicity and summation. For readability we have relegated some of
the technical subtleties to an appendix.

\subsubsection{Process grammar}\label{subsub:process_grammar}

\begin{mathpar}
  \inferrule* [lab=synchronization] {} {{M} \bc \pzero \;|\; x?F \;|\; x!C }
  \and
  \inferrule* [lab=abstraction] {} {{F} \bc (x)P}
  \and
  \inferrule* [lab=concretion] {} {{C} \bc \langle Q \rangle}
  \and
  \inferrule* [lab=process] {} {{P,Q} \bc M \;| \;P|Q \;|\; @{x}}
  \and
  \inferrule* [lab=name] {} {{x} \bc \quotep{P}}
\end{mathpar} 

Note that $\vec{x}$ (resp. $\vec{P}$) denotes a vector of names
(resp. processes) of length $|\vec{x}|$ (resp. $|\vec{P}|$). We adopt
the following useful abbreviations.

\begin{mathpar}
   x?(\vec{y}).P := x.(\vec{y})P \and  x\clift{\vec{P}} := x.\clift{\vec{P}}
   \and x!(y) := \lift{x}{\dropn{y}}
   \and \Pi_{i=0}^{n-1}P_i := P_0 | \ldots | P_{n-1}
\end{mathpar}

\subsubsection{Structural congruence}

\paragraph{Free and bound names and alpha-equivalence.} At the
core of structural equivalence is alpha-equivalence which identifies
process that are the same up to a change of variable. Formally, we
recognize the distinction between free and bound names. The free names
of a process, $\freenames{P}$, may be calculated recursively as
follows:

\begin{mathpar}
\freenames{\pzero} := \emptyset
  \and \\
  \freenames{x?(y).P} := \{ x \} \cup (\freenames{P} \setminus \{ y \})
  \and 
  \freenames{x!\langle P \rangle} := \{ x \} \cup \{ P \} 
  \and \\
  \freenames{P|Q} := \freenames{P} \cup \freenames{Q}
  \and \\
  \freenames{@{x}} := \{ x \}
\end{mathpar}

$\pi$
$\quotep{\pi}$

$\freenames{-} : \pi \to \mathcal{P}(\quotep{\pi})$

\begin{eqnarray*}
  \freenames{\pzero} & := & \emptyset \\
  \freenames{x?(y).P} & := & \{ x \} \cup (\freenames{P} \setminus \{ y \}) \\
  \freenames{x!\langle P \rangle} & := & \{ x \} \cup \{ P \} \\
  \freenames{P|Q} & := & \freenames{P} \cup \freenames{Q} \\
  \freenames{\dropn{x}} & := & \{ x \}
\end{eqnarray*}

The bound names of a process, $\boundnames{P}$, are those names occurring in $P$
that are not free. For example, in $x?(y).0$, the name $x$ is free, while $y$ is bound.

\begin{mathpar}
  \inferrule* [lab=monoidal-laws] {} { P|Q \equiv Q|P \and P|0 \equiv P \and P|(Q|R) \equiv (P|Q)|R }
\end{mathpar}

\begin{mathpar}
  \inferrule* [lab=alpha-equivalence] {} { (x)P \equiv (y)P\{y/x\} \and y \not\in \freenames{P} }
\end{mathpar}

\begin{definition}
Then two processes, $P,Q$, are alpha-equivalent if $P = Q\{\vec{y}/\vec{x}\}$ for
some $\vec{x} \in \boundnames{Q},\vec{y} \in \boundnames{P}$, where $Q\{\vec{y}/\vec{x}\}$
denotes the capture-avoiding substitution of $\vec{y}$ for $\vec{x}$ in $Q$.
\end{definition}

\begin{definition}
  The {\em structural congruence} \cite{SangiorgiWalker} , $\equiv$,
  between processes is the least congruence containing
  alpha-equivalence, satisfying the abelian monoid laws
  (associativity, commutativity and $\pzero$ as identity) for parallel
  composition $|$ and for summation $+$.
\end{definition}

\subsection{Name equivalence}

We take name equivalence, written $\nameeq$, to be the smallest
equivalence relation generated by the following rules.

\begin{mathpar}
\inferrule*[lab=Quote-drop]
{ }
{ \quotep{@{x}} \nameeq x }

\inferrule*[lab=Struct-equiv]
{ P \scong Q }
{ \quotep{P} \nameeq \quotep{Q} }
\end{mathpar}

The astute reader will have noticed that the mutual recursion of names
and processes imposes a mutual recursion on alpha-equivalence and
structural equivalence via name-equivalence. Fortunately, all of this
works out pleasantly and we may calculate in the natural way, free of
concern. The reader interested in the details is referred to the
appendix \ref{appendix:rho_details}.

\subsection{Substitution}

We use $\Proc$ for the set of processes, $\QProc$ for the set of
names, and $\id{\{}\vec{y} / \vec{x} \id{\}}$ to denote partial maps,
$s : \QProc \rightarrow \QProc$. A map, $s$ lifts, uniquely, to a map
on process terms, $\widehat{s} : \Proc \rightarrow \Proc$ by the
following equations.

\begin{mathpar}
  (0) \psubstp{Q}{P} := 0 \\
  (R \juxtap S) \psubstp{Q}{P}
  :=    
  (R)\psubstp{Q}{P} \juxtap (S) \psubstp{Q}{P} \\
  (x?(y).R) \psubstp{Q}{P}    
  :=    
  (x)\substp{Q}{P} (z)\concat( (R \psubstn{z}{y}) \psubstp{Q}{P} ) \\
  (\lift{x}{R}) \psubstp{Q}{P}  
  :=
  \lift{(x)\substp{Q}{P}}{ R \psubstp{Q}{P} } \\
%   (\dropn{x})  \psubstp{Q}{P}       
%   := 
%   \left\{ 
%     \begin{array}{ccc} 
%       \dropn{\quotep{Q}} & & x \nameeq \quotep{P} \\
%       \dropn{x} & & otherwise \\
%     \end{array}
%   \right. 
  (\dropn{x})  \psubstp{Q}{P}       
  := 
  \left\{ 
    \begin{array}{ccc} 
      Q & & x \nameeq \quotep{P} \\
      \dropn{x} & & otherwise \\
    \end{array}
  \right.
\end{mathpar}
 

where

\begin{eqnarray}
  (x)\id{\{} \lpquote Q \rpquote / \lpquote P \rpquote \id{\}}            = 
  \left\{ 
    \begin{array}{ccc}
      \lpquote Q \rpquote & & x \nameeq \lpquote P \rpquote \\
      x & & otherwise \\
    \end{array}
  \right. \nonumber
\end{eqnarray}

and $z$ is chosen distinct from $\quotep{P}$, $\quotep{Q}$, the free
names in $Q$, and all the names in $R$. Our $\alpha$-equivalence will
be built in the standard way from this substitution.

\begin{remark}\label{rem:no_self_referential_names}
  One consequence of these definitions is that $\forall P. \quotep{P}
  \not\in \freenames{P}$.
\end{remark}

\subsection{ Dynamic quote: an example }

Anticipating something of what's to come, consider applying the
substitution, $\widehat{\id{\{}u / z \id{\}}}$, to the following pair
of processes, $\lift{w}{y!(z)}$ and $w[ \lpquote y!(z) \rpquote ]$.

\begin{eqnarray}
	\lift{w}{y!(z)}\widehat{\id{\{}u / z \id{\}}}
		& = &
		\lift{w}{y!(u)} \nonumber\\
	w[ \lpquote y!(z) \rpquote ] \widehat{ \id{\{}u / z \id{\}} }
		& = &
		w[ \lpquote y!(z) \rpquote ] \nonumber
\end{eqnarray}

Because the body of the process between quotes is impervious to
substitution, we get radically different answers. In fact, by
examining the first process in an input context,
e.g. $x?(z).\lift{w}{y!(z)}$, we see that the process under the lift
operator may be shaped by prefixed inputs binding a name inside it. In
this sense, the lift operator will be seen as a way to dynamically
construct processes before reifying them as names.

Finally equipped with these standard features we can present the
dynamics of the calculus.

\subsubsection{Operational semantics} 

Finally, we introduce the computational dynamics. What marks these
algebras as distinct from other more traditionally studied algebraic
structures, e.g. vector spaces or polynomial rings, is the manner in
which dynamics is captured. In traditional structures, dynamics is typically
expressed through morphisms between such structures, as in linear maps
between vector spaces or morphisms between rings. In algebras
associated with the semantics of computation, the dynamics is
expressed as part of the algebraic structure itself, through a
reduction reduction relation typically denoted by $\red$. Below, we
give a recursive presentation of this relation for the calculus used
in the encoding.

$\red \subseteq \pi \times \pi$
$\red : \pi \to \mathcal{P}(\pi)$

\begin{mathpar}
  \inferrule* [lab=Comm] { \textsf{match}( x_{src}, x_{trgt} ) } { x_{trgt}?(y)P \; | \; x_{src}!\langle {Q} \rangle \red P\{\quotep{Q}/y}\} }
  \and \\
  \inferrule* [lab=Par] {{P} \red {P}'} {{{P} | {Q}} \red {{P}' | {Q}}}
  \and
  \inferrule* [lab=Equiv]{{{P} \scong {P}'} \andalso {{P}' \red {Q}'} \andalso {{Q}' \scong {Q}}}{{P} \red {Q}}
\end{mathpar}

\begin{eqnarray*}
  match_{\equiv} (\quotep{P},\quotep{Q}) & := & P \equiv Q \\
  match_{\dagger}(\quotep{P},\quotep{Q}) & := & \forall R. P|Q \red^{*} R => R \red^{*} 0 \\
  match_{K}(\quotep{P},\quotep{Q}) & := & K \mbox{ for some context } K
\end{eqnarray*}

$u?(x)P | u!\langle Q \rangle \red P\{\quotep{Q}/x\}$

%We write $\wred$ for $\red^*$, and $P\red$ if $\exists Q $ such that $ P \red Q$.
We write $P\red$ if $\exists Q $ such that $ P \red Q$ and $P\not\red$, otherwise.

\section{Replication}

As mentioned before, it is known that replication (and hence
recursion) can be implemented in a higher-order process algebra
\cite{SangiorgiWalker}. As our first example of calculation with the
machinery thus far presented we give the construction explicitly in
the {\rhoc}.

\begin{eqnarray}
	D_{x} & := & \prefix{x}{y}{(\binpar{\outputp{x}{y}}{@{y}})} \nonumber\\
	\bangp_{x}{P} & := & \binpar{{x}!\langle{\binpar{D_{x}}{P}}\rangle}{D_{x}} \nonumber
\end{eqnarray}

\begin{eqnarray}
	\bangp_{x}{P} & & \nonumber\\
	=
	& {x}!\langle{(\prefix{x}{y}{(\outputp{x}{y} | @{y})) | P}}\rangle 
	      | \prefix{x}{y}{(\outputp{x}{y} | @{y})} & \nonumber\\
	\red
	& (\outputp{x}{y} | @{y})\substn{\quotep{(\prefix{x}{y}{(@{y} | \outputp{x}{y})) | P}}}{y} & \nonumber\\
	=
	& \outputp{x}{\quotep{(\prefix{x}{y}{(\outputp{x}{y} | @{y})) | P}}}
	  | {(\prefix{x}{y}{(\outputp{x}{y} | @{y})) | P}} & \nonumber\\
	\red
	& \ldots & \nonumber\\
	\red^*
	& P | P | \ldots & \nonumber
\end{eqnarray}

Of course, this encoding, as an implementation, runs away, unfolding
$\bangp{P}$ eagerly. A lazier and more implementable replication
operator, restricted to input-guarded processes, may be obtained as follows.

\begin{eqnarray}
\bangp{\prefix{u}{v}{P}} 
	:= 
	\binpar{\lift{x}{\prefix{u}{v}{(\binpar{D(x)}{P})}}}{D(x)} \nonumber
\end{eqnarray}

\begin{remark}
  Note that the lazier definition still does not deal with summation
  or mixed summation (i.e. sums over input and output). The reader is
  invited to construct definitions of replication that deal with these
  features. 

  Further, the definitions are parameterized in a name, $x$. Can you,
  gentle reader, make a definition that eliminates this parameter and
  guarantees no accidental interaction between the replication
  machinery and the process being replicated -- i.e. no accidental
  sharing of names used by the process to get its work done and the
  name(s) used by the replication to effect copying. This latter
  revision of the definition of replication is crucial to obtaining
  the expected identity $!!P \sim !P$.
\end{remark}

\begin{remark}\label{rem:paradoxical_combinator}
  The reader familiar with the lambda calculus will have noticed the
  similarity between $D$ and the paradoxical combinator.

  [Ed. note: the existence of this seems to suggest we have to be more
  restrictive on the set of processes and names we admit if we are to
  support no-cloning.]
\end{remark}

\subsubsection{Bisimulation}

The computational dynamics gives rise to another kind of equivalence,
the equivalence of computational behavior. As previously mentioned
this is typically captured \emph{via} some form of bisimulation.

% The notion we use in this paper is weak barbed bisimulation
% \cite{milner91polyadicpi}.

The notion we use in this paper is derived from weak barbed
bisimulation \cite{milner91polyadicpi}. 

\begin{definition}
An \emph{observation relation}, $\downarrow_{\mathcal N}$, over a set
of names, $\mathcal N$, is the smallest relation satisfying the rules
below.

\infrule[Out-barb]{y \in {\mathcal N}, \; x \nameeq y}
		  {\outputp{x}{v} \downarrow_{\mathcal N} x}
\infrule[Par-barb]{\mbox{$P\downarrow_{\mathcal N} x$ or $Q\downarrow_{\mathcal N} x$}}
		  {\binpar{P}{Q} \downarrow_{\mathcal N} x}

We write $P \Downarrow_{\mathcal N} x$ if there is $Q$ such that 
$P \wred Q$ and $Q \downarrow_{\mathcal N} x$.
\end{definition}

\begin{definition}
%\label{def.bbisim}
An  ${\mathcal N}$-\emph{barbed bisimulation} over a set of names, ${\mathcal N}$, is a symmetric binary relation 
${\mathcal S}_{\mathcal N}$ between agents such that $P\rel{S}_{\mathcal N}Q$ implies:
\begin{enumerate}
\item If $P \red P'$ then $Q \wred Q'$ and $P'\rel{S}_{\mathcal N} Q'$.
\item If $P\downarrow_{\mathcal N} x$, then $Q\Downarrow_{\mathcal N} x$.
\end{enumerate}
$P$ is ${\mathcal N}$-barbed bisimilar to $Q$, written
$P \wbbisim_{\mathcal N} Q$, if $P \rel{S}_{\mathcal N} Q$ for some ${\mathcal N}$-barbed bisimulation ${\mathcal S}_{\mathcal N}$.
\end{definition}

$\mathcal{R} \subseteq \pi \times \pi$

$P \mathcal{R} Q => \forall P'. P \red P' \Rightarrow \exists Q'. Q \red Q', P' \mathcal{R} Q'$

$P \vdash x \Rightarrow Q \vdash x$

\begin{mathpar}
  \inferrule*[lab=Out-barb]{x \nameeq y}{{y}!\langle{Q}\rangle \vdash x}
  \and
  \inferrule*[lab=Par-barb]{\mbox{$P\vdash x$ or $Q\vdash x$}}{\binpar{P}{Q} \vdash x}
\end{mathpar}

\subsubsection{Contexts}

One of the principle advantages of computational calculi like the
$\pi$-calculus is a well-defined notion of context,
contextual-equivalence and a correlation between
contextual-equivalence and notions of bisimulation. The notion of
context allows the decomposition of a process into (sub-)process and
its syntactic environment, its context. Thus, a context may be
thought of as a process with a ``hole'' (written $\Box$) in it. The
application of a context $M$ to a process $P$, written $M[P]$, is
tantamount to filling the hole in $M$ with $P$. In this paper we do
not need the full weight of this theory, but do make use of the notion
of context in the proof the main theorem. 

\begin{mathpar}
  \inferrule* [lab=summation] {} {{M_{M},M_{N}} \bc \Box \;|\; x.M_{A} \;|\; M_{M}+M_{N}}
  \and
  \inferrule* [lab=agent] {} {{M_{A}} \bc (\vec{x})M_{P} \;| \; \clift{P_0,\ldots,M_{P},\ldots,P_N}}
  \and \\
  \inferrule* [lab=process] {} {{M_{P}} \bc M_{N} \;| \;P|M_{P} }
\end{mathpar} 

\begin{mathpar}
  \inferrule* [lab=sychronization] {} {M_{N} \bc \Box \;|\; x?M_{F} \;|\; x!M_{C}}
  \and
  \inferrule* [lab=abstraction] {} {{M_{F}} \bc (x)M_{P} }
  \and
  \inferrule* [lab=concretion] {} {{M_{C}} \bc \langle M_{P} \rangle }
  \and \\
  \inferrule* [lab=process] {} {{M_{P}} \bc M_{N} \;| \;P|M_{P} }
\end{mathpar}

\begin{definition}[contextual application] Given a context $M$, and
  process $P$, we define the \emph{contextual application}, $M[P] :=
  M\{P/\Box\}$. That is, the contextual application of M to P is the
  substitution of $P$ for $\Box$ in $M$.
\end{definition}

$\meaningof{-} : L \to \mathcal{P}(\pi)$

\begin{mathpar}
  \inferrule* [lab=collection] {} {\meaningof{true} = \pi, \and \meaningof{~E} = \pi \setminus \meaningof{E}, \and \meaningof{E_{1} \& E_{2}} = \meaningof{E_{1}} \cap \meaningof{E_{2}}}
\end{mathpar}

\begin{mathpar}
  \inferrule* [lab=structure] {} {\meaningof{0} = \{ P \in \pi | P \equiv 0 \}, \and \\ \meaningof{E_1 | E_2} = \{ P \in \pi | P \equiv P_{1} | P_{2}, P_{1} \in \meaningof{E_{1}}, P_{2} \in \meaningof{E_2}\} }
\end{mathpar}

\begin{mathpar}
 \inferrule* [lab=behavior] {} {\meaningof{\langle a?b \rangle E} = \{ P \in \pi | P \equiv Q | u?(y)P', \\ \and \\\\ \and \\ \;\;\; u \in \meaningof{a}, \forall z.P'\{z/y\} \in \meaningof{E\{z/b\}}\}, \and \\ \meaningof{a!E} = \{ P \in \pi | P \equiv Q | x!\langle P' \rangle, x \in \meaningof{a} P' \in \meaningof{E}\} }
\end{mathpar}

\begin{mathpar}
 \inferrule* [lab=nominal] {} {\meaningof{\quotep{E}} = \{ \quotep{P} \in \quotep{\pi} | P \in \meaningof{E} \}, \and \meaningof{\quotep{P}} = \{ \quotep{Q} \in \quotep{\pi} | P \equiv Q \} \and \\ \meaningof{@\quotep{E}} = \{ P \in \pi | P \equiv @x, x \in \meaningof{E} \}}
\end{mathpar}

\begin{eqnarray*}
  \\
  \meaningof{-} : TS \to ST
\end{eqnarray*}

\begin{eqnarray*}
  \\
  L : TS \to ST
\end{eqnarray*}

\begin{eqnarray*}
  \\
  P \models E \iff P \in \meaningof{E}
\end{eqnarray*}

\begin{eqnarray*}
  P \approx_{L} Q \iff \forall E \in L. P \models E \iff Q \models E
\end{eqnarray*}

\begin{eqnarray*}
  P \approx_{K} Q
\end{eqnarray*}

\begin{eqnarray*}
  P \approx Q
\end{eqnarray*}

$\approx_{K} = \approx = \approx_{L}$

\subsubsection{Contextual duality}

Note that contexts extend the quotation operation to a family of
operations from processes to names. Given a context, $M$, we can
define a \emph{nominal context}, $\quotep{M}$ by $\quotep{M}[P] :=
\quotep{M[P]}$. To foreshadow what is to come we observe that these
operations enjoy a duality with processes very much like the duality
between vectors and maps from vectors to scalars.

Further, because the calculus is essentially higher-order, we have a
correspondence between contexts and processes. More specifically,
given a name $x$ and a context $M$ we can construct $M^{*}_{x}$ such
that 

\begin{mathpar}
  M^{*}_{x} | \lift{x}{P} \red M[P]
\end{mathpar}

namely,

\begin{mathpar}
  M^{*}_{x} := x?(u).M[\dropn{u}]
\end{mathpar}

The dependence of $M^{*}_{x}$ on a name makes it an abstraction, 

\begin{mathpar}
  M^{*} := (x)x?(u).M[\dropn{u}]
\end{mathpar}

\subsection{Additional notation}

It will sometimes be convenient to denote the process a name
quotes. We already have the notation $x = \quotep{P}$, but it will be
convenient to introduce an alternate notation, $\procn{x}$, when we
want to emphasize the connection to the use of the name. Note that, by
virtue of name equivalence, $\quotep{\procn{x}} \nameeq x$; so, the
notation is consistent with previous definitions.

Further, because names have structure it is possible to effect
substitutions on the basis of that structure. This means we need to
upgrade our notation for substitutions, which we accomplish by
adapting comprehension notation. Thus,

\begin{mathpar}
  P\{ y / x : x \in S \}
\end{mathpar}

is interpreted to mean the process derived from P by replacing (in a
capture-avoiding manner) each occurrence of $x$ in $S$ by $y$. For example,

\begin{mathpar}
  P\{ \quotep{\procn{x}|\procn{x}} / x : x \in \freenames{P} \}
\end{mathpar}

will replace each (occurrence) of a free name $x$ in $P$ by
$\quotep{\procn{x}|\procn{x}}$.

Also, we will avail ourselves of the notation $x^{L}$ and $x^{R}$ to
denote injections of a name into disjoint copies of the name
space. There are numerous ways to accomplish this. One example can be
found in \cite{MeredithR05}. This notation overloads to vectors of
names: $\vec{x}^{\pi} := (x_{i}^{\pi} \; : \; 0 \leq i < |\vec{x}| )$ where $\pi \in \{L,R\}$.

We also use $P^{\Box} := P|\Box$.

In \cite{MeredithR05} an interpretation of the new operator is
given. It turns out that there are several possible interpretations
all enjoying the requisite algebraic properties of the operator (see
\cite{milner91polyadicpi}). We will therefore make liberal use of
$(\nu\; \vec{x})P$.

% subsection the_syntax_and_semantics_of_the_notation_system (end)   

\input{qm2pi.qmops} 

\input{qm2pi.sterngerlach} 

\input{qm2pi.metric} 

% section concurrent_process_calculi (end)

%\input{qm2pi.proofsketch}

% section proof sketch (end)

%\input{qm2pi.slviaknots} 

% section spatial logic via knots (end)

\input{qm2pi.conclusion}

% section conclusion (end)

%\input{qm2pi.dtcodes} 

% section wiring algorithm (end)

\input{qm2pi.ack} 

% section acknowledgments (end)

\newpage


\bibliographystyle{plain}   
\bibliography{../../biblios/main.bib}

\input{qm2pi.rhodetails}

\end{document}

 

%\ifpdf
%\usepackage[pdftex]{graphicx}
%\else
%\usepackage{graphicx}
%\fi

 % \ifpdf
%  \usepackage{pdfsync}
%  \if


%\title{Brief Article}
%\author{David F. Snyder}
%\author{L.G. Meredith}

%\address{Dept. of Math., Texas State University--San Marcos, San Marcos, TX 78666}
       
\pagestyle{empty}


\begin{document}

\lstset{language=[Objective]Caml,frame=shadowbox}

\documentclass[12pt]{llncs}
%\documentclass{jktr}

\usepackage[pdftex]{hyperref}                   
\usepackage {listings}
\usepackage {mathpartir}
\usepackage{bcprules}
%\usepackage{listings}
                       
\usepackage{graphicx} 
%\usepackage[margins=2.5cm,nohead,nofoot]{geometry}
%\usepackage{geometry}
\usepackage{amsfonts}
\usepackage{amstext}
\usepackage{latexsym}
\usepackage{amssymb}
\usepackage{color}


%\include{myPreamble}
\include{qm2pi.local} 

%\ifpdf
%\usepackage[pdftex]{graphicx}
%\else
%\usepackage{graphicx}
%\fi

 % \ifpdf
%  \usepackage{pdfsync}
%  \if


%\title{Brief Article}
%\author{David F. Snyder}
%\author{L.G. Meredith}

%\address{Dept. of Math., Texas State University--San Marcos, San Marcos, TX 78666}
       
\pagestyle{empty}


\begin{document}

\lstset{language=[Objective]Caml,frame=shadowbox}

\input{qm2pi.front}

% section front matter (end)

\input{qm2pi.intro} 
 
% section introduction (end)

% \input{qm2pi.knotations} 

% section notation (end)

\input{qm2pi.process.calculi} 

% section concurrent_process_calculi_and_spatial_logics_ (end)
    
%\input{qm2pi.knots2pi} 

%\input{qm2pi.trefoil} 

%\input{qm2pi.mainthm} 

% subsection basic_interpretation (end)

%\input{qm2pi.rho.presentation} 
\subsection{The syntax and semantics of the notation system}\label{sub:the_syntax_and_semantics_of_the_notation_system} % (fold)

We now summarize a technical presentation of the calculus that
embodies our theory of dynamics. The typical presentation of such a
calculus follows the style of giving generators and relations on
them. The grammar, below, describing term constructors, freely
generates the set of processes, $\Proc$. This set is then quotiented
by a relation known as structural congruence and it is over this set
that the notion of dynamics is expressed. This presentation is
essentially that of \cite{MeredithR05} with the addition of
polyadicity and summation. For readability we have relegated some of
the technical subtleties to an appendix.

\subsubsection{Process grammar}\label{subsub:process_grammar}

\begin{mathpar}
  \inferrule* [lab=synchronization] {} {{M} \bc \pzero \;|\; x?F \;|\; x!C }
  \and
  \inferrule* [lab=abstraction] {} {{F} \bc (x)P}
  \and
  \inferrule* [lab=concretion] {} {{C} \bc \langle Q \rangle}
  \and
  \inferrule* [lab=process] {} {{P,Q} \bc M \;| \;P|Q \;|\; @{x}}
  \and
  \inferrule* [lab=name] {} {{x} \bc \quotep{P}}
\end{mathpar} 

Note that $\vec{x}$ (resp. $\vec{P}$) denotes a vector of names
(resp. processes) of length $|\vec{x}|$ (resp. $|\vec{P}|$). We adopt
the following useful abbreviations.

\begin{mathpar}
   x?(\vec{y}).P := x.(\vec{y})P \and  x\clift{\vec{P}} := x.\clift{\vec{P}}
   \and x!(y) := \lift{x}{\dropn{y}}
   \and \Pi_{i=0}^{n-1}P_i := P_0 | \ldots | P_{n-1}
\end{mathpar}

\subsubsection{Structural congruence}

\paragraph{Free and bound names and alpha-equivalence.} At the
core of structural equivalence is alpha-equivalence which identifies
process that are the same up to a change of variable. Formally, we
recognize the distinction between free and bound names. The free names
of a process, $\freenames{P}$, may be calculated recursively as
follows:

\begin{mathpar}
\freenames{\pzero} := \emptyset
  \and \\
  \freenames{x?(y).P} := \{ x \} \cup (\freenames{P} \setminus \{ y \})
  \and 
  \freenames{x!\langle P \rangle} := \{ x \} \cup \{ P \} 
  \and \\
  \freenames{P|Q} := \freenames{P} \cup \freenames{Q}
  \and \\
  \freenames{@{x}} := \{ x \}
\end{mathpar}

$\pi$
$\quotep{\pi}$

$\freenames{-} : \pi \to \mathcal{P}(\quotep{\pi})$

\begin{eqnarray*}
  \freenames{\pzero} & := & \emptyset \\
  \freenames{x?(y).P} & := & \{ x \} \cup (\freenames{P} \setminus \{ y \}) \\
  \freenames{x!\langle P \rangle} & := & \{ x \} \cup \{ P \} \\
  \freenames{P|Q} & := & \freenames{P} \cup \freenames{Q} \\
  \freenames{\dropn{x}} & := & \{ x \}
\end{eqnarray*}

The bound names of a process, $\boundnames{P}$, are those names occurring in $P$
that are not free. For example, in $x?(y).0$, the name $x$ is free, while $y$ is bound.

\begin{mathpar}
  \inferrule* [lab=monoidal-laws] {} { P|Q \equiv Q|P \and P|0 \equiv P \and P|(Q|R) \equiv (P|Q)|R }
\end{mathpar}

\begin{mathpar}
  \inferrule* [lab=alpha-equivalence] {} { (x)P \equiv (y)P\{y/x\} \and y \not\in \freenames{P} }
\end{mathpar}

\begin{definition}
Then two processes, $P,Q$, are alpha-equivalent if $P = Q\{\vec{y}/\vec{x}\}$ for
some $\vec{x} \in \boundnames{Q},\vec{y} \in \boundnames{P}$, where $Q\{\vec{y}/\vec{x}\}$
denotes the capture-avoiding substitution of $\vec{y}$ for $\vec{x}$ in $Q$.
\end{definition}

\begin{definition}
  The {\em structural congruence} \cite{SangiorgiWalker} , $\equiv$,
  between processes is the least congruence containing
  alpha-equivalence, satisfying the abelian monoid laws
  (associativity, commutativity and $\pzero$ as identity) for parallel
  composition $|$ and for summation $+$.
\end{definition}

\subsection{Name equivalence}

We take name equivalence, written $\nameeq$, to be the smallest
equivalence relation generated by the following rules.

\begin{mathpar}
\inferrule*[lab=Quote-drop]
{ }
{ \quotep{@{x}} \nameeq x }

\inferrule*[lab=Struct-equiv]
{ P \scong Q }
{ \quotep{P} \nameeq \quotep{Q} }
\end{mathpar}

The astute reader will have noticed that the mutual recursion of names
and processes imposes a mutual recursion on alpha-equivalence and
structural equivalence via name-equivalence. Fortunately, all of this
works out pleasantly and we may calculate in the natural way, free of
concern. The reader interested in the details is referred to the
appendix \ref{appendix:rho_details}.

\subsection{Substitution}

We use $\Proc$ for the set of processes, $\QProc$ for the set of
names, and $\id{\{}\vec{y} / \vec{x} \id{\}}$ to denote partial maps,
$s : \QProc \rightarrow \QProc$. A map, $s$ lifts, uniquely, to a map
on process terms, $\widehat{s} : \Proc \rightarrow \Proc$ by the
following equations.

\begin{mathpar}
  (0) \psubstp{Q}{P} := 0 \\
  (R \juxtap S) \psubstp{Q}{P}
  :=    
  (R)\psubstp{Q}{P} \juxtap (S) \psubstp{Q}{P} \\
  (x?(y).R) \psubstp{Q}{P}    
  :=    
  (x)\substp{Q}{P} (z)\concat( (R \psubstn{z}{y}) \psubstp{Q}{P} ) \\
  (\lift{x}{R}) \psubstp{Q}{P}  
  :=
  \lift{(x)\substp{Q}{P}}{ R \psubstp{Q}{P} } \\
%   (\dropn{x})  \psubstp{Q}{P}       
%   := 
%   \left\{ 
%     \begin{array}{ccc} 
%       \dropn{\quotep{Q}} & & x \nameeq \quotep{P} \\
%       \dropn{x} & & otherwise \\
%     \end{array}
%   \right. 
  (\dropn{x})  \psubstp{Q}{P}       
  := 
  \left\{ 
    \begin{array}{ccc} 
      Q & & x \nameeq \quotep{P} \\
      \dropn{x} & & otherwise \\
    \end{array}
  \right.
\end{mathpar}
 

where

\begin{eqnarray}
  (x)\id{\{} \lpquote Q \rpquote / \lpquote P \rpquote \id{\}}            = 
  \left\{ 
    \begin{array}{ccc}
      \lpquote Q \rpquote & & x \nameeq \lpquote P \rpquote \\
      x & & otherwise \\
    \end{array}
  \right. \nonumber
\end{eqnarray}

and $z$ is chosen distinct from $\quotep{P}$, $\quotep{Q}$, the free
names in $Q$, and all the names in $R$. Our $\alpha$-equivalence will
be built in the standard way from this substitution.

\begin{remark}\label{rem:no_self_referential_names}
  One consequence of these definitions is that $\forall P. \quotep{P}
  \not\in \freenames{P}$.
\end{remark}

\subsection{ Dynamic quote: an example }

Anticipating something of what's to come, consider applying the
substitution, $\widehat{\id{\{}u / z \id{\}}}$, to the following pair
of processes, $\lift{w}{y!(z)}$ and $w[ \lpquote y!(z) \rpquote ]$.

\begin{eqnarray}
	\lift{w}{y!(z)}\widehat{\id{\{}u / z \id{\}}}
		& = &
		\lift{w}{y!(u)} \nonumber\\
	w[ \lpquote y!(z) \rpquote ] \widehat{ \id{\{}u / z \id{\}} }
		& = &
		w[ \lpquote y!(z) \rpquote ] \nonumber
\end{eqnarray}

Because the body of the process between quotes is impervious to
substitution, we get radically different answers. In fact, by
examining the first process in an input context,
e.g. $x?(z).\lift{w}{y!(z)}$, we see that the process under the lift
operator may be shaped by prefixed inputs binding a name inside it. In
this sense, the lift operator will be seen as a way to dynamically
construct processes before reifying them as names.

Finally equipped with these standard features we can present the
dynamics of the calculus.

\subsubsection{Operational semantics} 

Finally, we introduce the computational dynamics. What marks these
algebras as distinct from other more traditionally studied algebraic
structures, e.g. vector spaces or polynomial rings, is the manner in
which dynamics is captured. In traditional structures, dynamics is typically
expressed through morphisms between such structures, as in linear maps
between vector spaces or morphisms between rings. In algebras
associated with the semantics of computation, the dynamics is
expressed as part of the algebraic structure itself, through a
reduction reduction relation typically denoted by $\red$. Below, we
give a recursive presentation of this relation for the calculus used
in the encoding.

$\red \subseteq \pi \times \pi$
$\red : \pi \to \mathcal{P}(\pi)$

\begin{mathpar}
  \inferrule* [lab=Comm] { \textsf{match}( x_{src}, x_{trgt} ) } { x_{trgt}?(y)P \; | \; x_{src}!\langle {Q} \rangle \red P\{\quotep{Q}/y}\} }
  \and \\
  \inferrule* [lab=Par] {{P} \red {P}'} {{{P} | {Q}} \red {{P}' | {Q}}}
  \and
  \inferrule* [lab=Equiv]{{{P} \scong {P}'} \andalso {{P}' \red {Q}'} \andalso {{Q}' \scong {Q}}}{{P} \red {Q}}
\end{mathpar}

\begin{eqnarray*}
  match_{\equiv} (\quotep{P},\quotep{Q}) & := & P \equiv Q \\
  match_{\dagger}(\quotep{P},\quotep{Q}) & := & \forall R. P|Q \red^{*} R => R \red^{*} 0 \\
  match_{K}(\quotep{P},\quotep{Q}) & := & K \mbox{ for some context } K
\end{eqnarray*}

$u?(x)P | u!\langle Q \rangle \red P\{\quotep{Q}/x\}$

%We write $\wred$ for $\red^*$, and $P\red$ if $\exists Q $ such that $ P \red Q$.
We write $P\red$ if $\exists Q $ such that $ P \red Q$ and $P\not\red$, otherwise.

\section{Replication}

As mentioned before, it is known that replication (and hence
recursion) can be implemented in a higher-order process algebra
\cite{SangiorgiWalker}. As our first example of calculation with the
machinery thus far presented we give the construction explicitly in
the {\rhoc}.

\begin{eqnarray}
	D_{x} & := & \prefix{x}{y}{(\binpar{\outputp{x}{y}}{@{y}})} \nonumber\\
	\bangp_{x}{P} & := & \binpar{{x}!\langle{\binpar{D_{x}}{P}}\rangle}{D_{x}} \nonumber
\end{eqnarray}

\begin{eqnarray}
	\bangp_{x}{P} & & \nonumber\\
	=
	& {x}!\langle{(\prefix{x}{y}{(\outputp{x}{y} | @{y})) | P}}\rangle 
	      | \prefix{x}{y}{(\outputp{x}{y} | @{y})} & \nonumber\\
	\red
	& (\outputp{x}{y} | @{y})\substn{\quotep{(\prefix{x}{y}{(@{y} | \outputp{x}{y})) | P}}}{y} & \nonumber\\
	=
	& \outputp{x}{\quotep{(\prefix{x}{y}{(\outputp{x}{y} | @{y})) | P}}}
	  | {(\prefix{x}{y}{(\outputp{x}{y} | @{y})) | P}} & \nonumber\\
	\red
	& \ldots & \nonumber\\
	\red^*
	& P | P | \ldots & \nonumber
\end{eqnarray}

Of course, this encoding, as an implementation, runs away, unfolding
$\bangp{P}$ eagerly. A lazier and more implementable replication
operator, restricted to input-guarded processes, may be obtained as follows.

\begin{eqnarray}
\bangp{\prefix{u}{v}{P}} 
	:= 
	\binpar{\lift{x}{\prefix{u}{v}{(\binpar{D(x)}{P})}}}{D(x)} \nonumber
\end{eqnarray}

\begin{remark}
  Note that the lazier definition still does not deal with summation
  or mixed summation (i.e. sums over input and output). The reader is
  invited to construct definitions of replication that deal with these
  features. 

  Further, the definitions are parameterized in a name, $x$. Can you,
  gentle reader, make a definition that eliminates this parameter and
  guarantees no accidental interaction between the replication
  machinery and the process being replicated -- i.e. no accidental
  sharing of names used by the process to get its work done and the
  name(s) used by the replication to effect copying. This latter
  revision of the definition of replication is crucial to obtaining
  the expected identity $!!P \sim !P$.
\end{remark}

\begin{remark}\label{rem:paradoxical_combinator}
  The reader familiar with the lambda calculus will have noticed the
  similarity between $D$ and the paradoxical combinator.

  [Ed. note: the existence of this seems to suggest we have to be more
  restrictive on the set of processes and names we admit if we are to
  support no-cloning.]
\end{remark}

\subsubsection{Bisimulation}

The computational dynamics gives rise to another kind of equivalence,
the equivalence of computational behavior. As previously mentioned
this is typically captured \emph{via} some form of bisimulation.

% The notion we use in this paper is weak barbed bisimulation
% \cite{milner91polyadicpi}.

The notion we use in this paper is derived from weak barbed
bisimulation \cite{milner91polyadicpi}. 

\begin{definition}
An \emph{observation relation}, $\downarrow_{\mathcal N}$, over a set
of names, $\mathcal N$, is the smallest relation satisfying the rules
below.

\infrule[Out-barb]{y \in {\mathcal N}, \; x \nameeq y}
		  {\outputp{x}{v} \downarrow_{\mathcal N} x}
\infrule[Par-barb]{\mbox{$P\downarrow_{\mathcal N} x$ or $Q\downarrow_{\mathcal N} x$}}
		  {\binpar{P}{Q} \downarrow_{\mathcal N} x}

We write $P \Downarrow_{\mathcal N} x$ if there is $Q$ such that 
$P \wred Q$ and $Q \downarrow_{\mathcal N} x$.
\end{definition}

\begin{definition}
%\label{def.bbisim}
An  ${\mathcal N}$-\emph{barbed bisimulation} over a set of names, ${\mathcal N}$, is a symmetric binary relation 
${\mathcal S}_{\mathcal N}$ between agents such that $P\rel{S}_{\mathcal N}Q$ implies:
\begin{enumerate}
\item If $P \red P'$ then $Q \wred Q'$ and $P'\rel{S}_{\mathcal N} Q'$.
\item If $P\downarrow_{\mathcal N} x$, then $Q\Downarrow_{\mathcal N} x$.
\end{enumerate}
$P$ is ${\mathcal N}$-barbed bisimilar to $Q$, written
$P \wbbisim_{\mathcal N} Q$, if $P \rel{S}_{\mathcal N} Q$ for some ${\mathcal N}$-barbed bisimulation ${\mathcal S}_{\mathcal N}$.
\end{definition}

$\mathcal{R} \subseteq \pi \times \pi$

$P \mathcal{R} Q => \forall P'. P \red P' \Rightarrow \exists Q'. Q \red Q', P' \mathcal{R} Q'$

$P \vdash x \Rightarrow Q \vdash x$

\begin{mathpar}
  \inferrule*[lab=Out-barb]{x \nameeq y}{{y}!\langle{Q}\rangle \vdash x}
  \and
  \inferrule*[lab=Par-barb]{\mbox{$P\vdash x$ or $Q\vdash x$}}{\binpar{P}{Q} \vdash x}
\end{mathpar}

\subsubsection{Contexts}

One of the principle advantages of computational calculi like the
$\pi$-calculus is a well-defined notion of context,
contextual-equivalence and a correlation between
contextual-equivalence and notions of bisimulation. The notion of
context allows the decomposition of a process into (sub-)process and
its syntactic environment, its context. Thus, a context may be
thought of as a process with a ``hole'' (written $\Box$) in it. The
application of a context $M$ to a process $P$, written $M[P]$, is
tantamount to filling the hole in $M$ with $P$. In this paper we do
not need the full weight of this theory, but do make use of the notion
of context in the proof the main theorem. 

\begin{mathpar}
  \inferrule* [lab=summation] {} {{M_{M},M_{N}} \bc \Box \;|\; x.M_{A} \;|\; M_{M}+M_{N}}
  \and
  \inferrule* [lab=agent] {} {{M_{A}} \bc (\vec{x})M_{P} \;| \; \clift{P_0,\ldots,M_{P},\ldots,P_N}}
  \and \\
  \inferrule* [lab=process] {} {{M_{P}} \bc M_{N} \;| \;P|M_{P} }
\end{mathpar} 

\begin{mathpar}
  \inferrule* [lab=sychronization] {} {M_{N} \bc \Box \;|\; x?M_{F} \;|\; x!M_{C}}
  \and
  \inferrule* [lab=abstraction] {} {{M_{F}} \bc (x)M_{P} }
  \and
  \inferrule* [lab=concretion] {} {{M_{C}} \bc \langle M_{P} \rangle }
  \and \\
  \inferrule* [lab=process] {} {{M_{P}} \bc M_{N} \;| \;P|M_{P} }
\end{mathpar}

\begin{definition}[contextual application] Given a context $M$, and
  process $P$, we define the \emph{contextual application}, $M[P] :=
  M\{P/\Box\}$. That is, the contextual application of M to P is the
  substitution of $P$ for $\Box$ in $M$.
\end{definition}

$\meaningof{-} : L \to \mathcal{P}(\pi)$

\begin{mathpar}
  \inferrule* [lab=collection] {} {\meaningof{true} = \pi, \and \meaningof{~E} = \pi \setminus \meaningof{E}, \and \meaningof{E_{1} \& E_{2}} = \meaningof{E_{1}} \cap \meaningof{E_{2}}}
\end{mathpar}

\begin{mathpar}
  \inferrule* [lab=structure] {} {\meaningof{0} = \{ P \in \pi | P \equiv 0 \}, \and \\ \meaningof{E_1 | E_2} = \{ P \in \pi | P \equiv P_{1} | P_{2}, P_{1} \in \meaningof{E_{1}}, P_{2} \in \meaningof{E_2}\} }
\end{mathpar}

\begin{mathpar}
 \inferrule* [lab=behavior] {} {\meaningof{\langle a?b \rangle E} = \{ P \in \pi | P \equiv Q | u?(y)P', \\ \and \\\\ \and \\ \;\;\; u \in \meaningof{a}, \forall z.P'\{z/y\} \in \meaningof{E\{z/b\}}\}, \and \\ \meaningof{a!E} = \{ P \in \pi | P \equiv Q | x!\langle P' \rangle, x \in \meaningof{a} P' \in \meaningof{E}\} }
\end{mathpar}

\begin{mathpar}
 \inferrule* [lab=nominal] {} {\meaningof{\quotep{E}} = \{ \quotep{P} \in \quotep{\pi} | P \in \meaningof{E} \}, \and \meaningof{\quotep{P}} = \{ \quotep{Q} \in \quotep{\pi} | P \equiv Q \} \and \\ \meaningof{@\quotep{E}} = \{ P \in \pi | P \equiv @x, x \in \meaningof{E} \}}
\end{mathpar}

\begin{eqnarray*}
  \\
  \meaningof{-} : TS \to ST
\end{eqnarray*}

\begin{eqnarray*}
  \\
  L : TS \to ST
\end{eqnarray*}

\begin{eqnarray*}
  \\
  P \models E \iff P \in \meaningof{E}
\end{eqnarray*}

\begin{eqnarray*}
  P \approx_{L} Q \iff \forall E \in L. P \models E \iff Q \models E
\end{eqnarray*}

\begin{eqnarray*}
  P \approx_{K} Q
\end{eqnarray*}

\begin{eqnarray*}
  P \approx Q
\end{eqnarray*}

$\approx_{K} = \approx = \approx_{L}$

\subsubsection{Contextual duality}

Note that contexts extend the quotation operation to a family of
operations from processes to names. Given a context, $M$, we can
define a \emph{nominal context}, $\quotep{M}$ by $\quotep{M}[P] :=
\quotep{M[P]}$. To foreshadow what is to come we observe that these
operations enjoy a duality with processes very much like the duality
between vectors and maps from vectors to scalars.

Further, because the calculus is essentially higher-order, we have a
correspondence between contexts and processes. More specifically,
given a name $x$ and a context $M$ we can construct $M^{*}_{x}$ such
that 

\begin{mathpar}
  M^{*}_{x} | \lift{x}{P} \red M[P]
\end{mathpar}

namely,

\begin{mathpar}
  M^{*}_{x} := x?(u).M[\dropn{u}]
\end{mathpar}

The dependence of $M^{*}_{x}$ on a name makes it an abstraction, 

\begin{mathpar}
  M^{*} := (x)x?(u).M[\dropn{u}]
\end{mathpar}

\subsection{Additional notation}

It will sometimes be convenient to denote the process a name
quotes. We already have the notation $x = \quotep{P}$, but it will be
convenient to introduce an alternate notation, $\procn{x}$, when we
want to emphasize the connection to the use of the name. Note that, by
virtue of name equivalence, $\quotep{\procn{x}} \nameeq x$; so, the
notation is consistent with previous definitions.

Further, because names have structure it is possible to effect
substitutions on the basis of that structure. This means we need to
upgrade our notation for substitutions, which we accomplish by
adapting comprehension notation. Thus,

\begin{mathpar}
  P\{ y / x : x \in S \}
\end{mathpar}

is interpreted to mean the process derived from P by replacing (in a
capture-avoiding manner) each occurrence of $x$ in $S$ by $y$. For example,

\begin{mathpar}
  P\{ \quotep{\procn{x}|\procn{x}} / x : x \in \freenames{P} \}
\end{mathpar}

will replace each (occurrence) of a free name $x$ in $P$ by
$\quotep{\procn{x}|\procn{x}}$.

Also, we will avail ourselves of the notation $x^{L}$ and $x^{R}$ to
denote injections of a name into disjoint copies of the name
space. There are numerous ways to accomplish this. One example can be
found in \cite{MeredithR05}. This notation overloads to vectors of
names: $\vec{x}^{\pi} := (x_{i}^{\pi} \; : \; 0 \leq i < |\vec{x}| )$ where $\pi \in \{L,R\}$.

We also use $P^{\Box} := P|\Box$.

In \cite{MeredithR05} an interpretation of the new operator is
given. It turns out that there are several possible interpretations
all enjoying the requisite algebraic properties of the operator (see
\cite{milner91polyadicpi}). We will therefore make liberal use of
$(\nu\; \vec{x})P$.

% subsection the_syntax_and_semantics_of_the_notation_system (end)   

\input{qm2pi.qmops} 

\input{qm2pi.sterngerlach} 

\input{qm2pi.metric} 

% section concurrent_process_calculi (end)

%\input{qm2pi.proofsketch}

% section proof sketch (end)

%\input{qm2pi.slviaknots} 

% section spatial logic via knots (end)

\input{qm2pi.conclusion}

% section conclusion (end)

%\input{qm2pi.dtcodes} 

% section wiring algorithm (end)

\input{qm2pi.ack} 

% section acknowledgments (end)

\newpage


\bibliographystyle{plain}   
\bibliography{../../biblios/main.bib}

\input{qm2pi.rhodetails}

\end{document}



% section front matter (end)

\section{Introduction}\label{sec:introduction} % (fold)
In this draft of the material i am going to have to dispense with the
usual writing conventions adopted in papers on these topics. i'm going
to have adopt whatever tone i need at the time i'm writing up the
calculations. Sometimes this may be very conversational; others it may
be the barest mathematical grunts; others still it may be that i have
lifted text from one of my other papers because the exposition of some
point was better said there. i hope that my readers are not unduly put
out by this decision. i'm not doing this to flout convention or be
rebellious. i find these calculations very technically challenging. To
keep everything going technically, something has to give; i have to
let go of some cognitive burden. So, the academic writing style --
with all of its trade-offs in terms of facilitating technical
communication -- is what i'm letting go of. Perhaps subsequent drafts
can be tightened and polished, but for now, i'm going to speak as if
we were sitting together in a coffee shop with a laptop, wifi and a
pad of paper and a pencil.

So, here's what i have to say. We -- you and i, comfortably ensconced
in our coffee shop and well-equipped with our tools -- can realize and
carry out the calculations of quantum mechanics over a very different
formal theory of dynamics, a formal theory of dynamics that
corresponds to a theory of concurrent computation with
\emph{reflection}. It has the advantage that the underlying theory is
already `quantized', but supports analogues all of the continuuous
operations. Strikingly, this underlying theory has recently been
connected with a notion of metric that we can show, by calculating
together, coincides with the metric induced by the inner product.

There are a lot of reasons why you might be interested in seeing
calculations of this form. Here's why i'm interested. For the past
several centuries there has been no competitor to the ``Newtonian''
account of dynamics. As a result the predominant share of accounts of
dynamical systems and situations have had to be formulated in terms of
the Newtonian machinery. i view this as an intellectually dangerous
position to occupy. Everything, despite it's intrinsic shape, turns
into a nail to be hit with this hammer. Recently, however, the theory
of computation has matured to the point where we have candidates for
theories of dynamics that offer very different perspective on
reasoning about dynamical systems and situations. Testing these
candidates against very successful accounts of dynamical situations,
like quantum mechanics, is going to give us some sense of how mature
they are and some measure of the quality of these accounts of
dynamics.

\subsection{Summary of contributions and outline of paper}

So, we're going to develop an interpretation of the operations of
quantum mechanics normally interpreted by Hilbert spaces and
operators. We're going to do this over a theory of computation. Note
that this is very different than the usual quantum computation program
which develops notions of computation over quantum mechanics. Rather,
we are developing a story that aligns with Wheeler's slogan: It from
Bit. To do this we will first provide an account of the theory of
computation at play here. Then we will dive into a calculation-driven
interpretation of the operations of quantum mechanics.

The reason we take this approach is that -- until very recently --
there hasn't been an axiomatic account of quantum mechanics. As a
result there has been no sharp delineation of the mathematical theory
supporting interpretation of the physical theory and the physical
theory, itself. So, ambient features of the maths are free to be
exploited (or supressed) without a real accounting of their physical
relevance. There is no sharp statement ``here's the physical theory''
qua \emph{theory} and ``here's the mathematical interpretation''
enabling a judgment of how faithful the interpretation is -- apart
from experimental observation. When there is an axiomatic account we
can judge how well a given mathematical formalism supports an
interpretation of the axioms, independent of
experimentation. Likewise, we can judge how well we have captured our
physical evidence and experience with our axiomatics, independent of
any specific mathematical implementation, with accidental detail that
may or may not have physical significance. 

In lieu of a fully fleshed out and vetted axiomatic account of quantum
mechanics, interpreting the operational notions in service of modeling
physical systems will have to suffice. In other words, we are not in
the business of providing a model of Hilbert spaces and operators. We
are in the business of providing a model of quantum mechanics because
we are motivated by testing our notions of dynamics against physical
theory; and, the predictive calculations of the physical theory must
serve as the best formulation -- shy of a fully fleshed out axiomatic
account -- of the physical theory itself (as they have for scientific
theories since time immemorial). Put another way, despite a
whole-hearted commitment to an It-from-Bit ontology, we are firmly
aligned with the shut-up-and-calculate camp as the best way to obtain
results either from the physical perspective or as a quality assurance
measure of our fledgling theory of dynamics.

In detail, we present a reflective process calculus. Then we develop
intuitive correspondences between the notions available in this
calculus and the usual physical notions supporting quantum mechanical
calculations. Thus, 

\begin{table}[htp]
  \center{
    \fbox{
      \begin{tabular}{c|c}
        quantum mechanics & process calculus \\
        \hline
        scalar & name \\
        state vector & process \\
        dual & contextual duals \\
        matrix & formal sums of process-context-dual pairs \\
        orthogonality & process annihilation \\
        inner product & execution-formula + quoting
      \end{tabular}
    }
  }
  \caption{QM - process calculi correspondences}
\end{table}

Then we tighten up these intuitions to operational definitions. We
employ the Dirac notation as the best proxy we can find for an
abstract syntax of the quantum mechanical notions. The definitions we
develop put us in contact with equational constraints coming from the
theory that we demonstrate the definitions and calculations satisfy.

This puts us in a position to shut up and calculate for the
Stern-Gerlach experimental set up, showing how these predictive
calculations become calculations on processes in our theory of a
reflective process calculus.

Penultimately, we demonstrate that the notion of metric coming from
the inner product coincides with the notion of metric available from
the theory of bisimulation. This demonstration gives us the right to
think of space as arising from behavior. Finally, we consider where we
might go from the new vantage point we have obtained.

% section introduction (end) 
 
% section introduction (end)

% \documentclass[12pt]{llncs}
%\documentclass{jktr}

\usepackage[pdftex]{hyperref}                   
\usepackage {listings}
\usepackage {mathpartir}
\usepackage{bcprules}
%\usepackage{listings}
                       
\usepackage{graphicx} 
%\usepackage[margins=2.5cm,nohead,nofoot]{geometry}
%\usepackage{geometry}
\usepackage{amsfonts}
\usepackage{amstext}
\usepackage{latexsym}
\usepackage{amssymb}
\usepackage{color}


%\include{myPreamble}
\include{qm2pi.local} 

%\ifpdf
%\usepackage[pdftex]{graphicx}
%\else
%\usepackage{graphicx}
%\fi

 % \ifpdf
%  \usepackage{pdfsync}
%  \if


%\title{Brief Article}
%\author{David F. Snyder}
%\author{L.G. Meredith}

%\address{Dept. of Math., Texas State University--San Marcos, San Marcos, TX 78666}
       
\pagestyle{empty}


\begin{document}

\lstset{language=[Objective]Caml,frame=shadowbox}

\input{qm2pi.front}

% section front matter (end)

\input{qm2pi.intro} 
 
% section introduction (end)

% \input{qm2pi.knotations} 

% section notation (end)

\input{qm2pi.process.calculi} 

% section concurrent_process_calculi_and_spatial_logics_ (end)
    
%\input{qm2pi.knots2pi} 

%\input{qm2pi.trefoil} 

%\input{qm2pi.mainthm} 

% subsection basic_interpretation (end)

%\input{qm2pi.rho.presentation} 
\subsection{The syntax and semantics of the notation system}\label{sub:the_syntax_and_semantics_of_the_notation_system} % (fold)

We now summarize a technical presentation of the calculus that
embodies our theory of dynamics. The typical presentation of such a
calculus follows the style of giving generators and relations on
them. The grammar, below, describing term constructors, freely
generates the set of processes, $\Proc$. This set is then quotiented
by a relation known as structural congruence and it is over this set
that the notion of dynamics is expressed. This presentation is
essentially that of \cite{MeredithR05} with the addition of
polyadicity and summation. For readability we have relegated some of
the technical subtleties to an appendix.

\subsubsection{Process grammar}\label{subsub:process_grammar}

\begin{mathpar}
  \inferrule* [lab=synchronization] {} {{M} \bc \pzero \;|\; x?F \;|\; x!C }
  \and
  \inferrule* [lab=abstraction] {} {{F} \bc (x)P}
  \and
  \inferrule* [lab=concretion] {} {{C} \bc \langle Q \rangle}
  \and
  \inferrule* [lab=process] {} {{P,Q} \bc M \;| \;P|Q \;|\; @{x}}
  \and
  \inferrule* [lab=name] {} {{x} \bc \quotep{P}}
\end{mathpar} 

Note that $\vec{x}$ (resp. $\vec{P}$) denotes a vector of names
(resp. processes) of length $|\vec{x}|$ (resp. $|\vec{P}|$). We adopt
the following useful abbreviations.

\begin{mathpar}
   x?(\vec{y}).P := x.(\vec{y})P \and  x\clift{\vec{P}} := x.\clift{\vec{P}}
   \and x!(y) := \lift{x}{\dropn{y}}
   \and \Pi_{i=0}^{n-1}P_i := P_0 | \ldots | P_{n-1}
\end{mathpar}

\subsubsection{Structural congruence}

\paragraph{Free and bound names and alpha-equivalence.} At the
core of structural equivalence is alpha-equivalence which identifies
process that are the same up to a change of variable. Formally, we
recognize the distinction between free and bound names. The free names
of a process, $\freenames{P}$, may be calculated recursively as
follows:

\begin{mathpar}
\freenames{\pzero} := \emptyset
  \and \\
  \freenames{x?(y).P} := \{ x \} \cup (\freenames{P} \setminus \{ y \})
  \and 
  \freenames{x!\langle P \rangle} := \{ x \} \cup \{ P \} 
  \and \\
  \freenames{P|Q} := \freenames{P} \cup \freenames{Q}
  \and \\
  \freenames{@{x}} := \{ x \}
\end{mathpar}

$\pi$
$\quotep{\pi}$

$\freenames{-} : \pi \to \mathcal{P}(\quotep{\pi})$

\begin{eqnarray*}
  \freenames{\pzero} & := & \emptyset \\
  \freenames{x?(y).P} & := & \{ x \} \cup (\freenames{P} \setminus \{ y \}) \\
  \freenames{x!\langle P \rangle} & := & \{ x \} \cup \{ P \} \\
  \freenames{P|Q} & := & \freenames{P} \cup \freenames{Q} \\
  \freenames{\dropn{x}} & := & \{ x \}
\end{eqnarray*}

The bound names of a process, $\boundnames{P}$, are those names occurring in $P$
that are not free. For example, in $x?(y).0$, the name $x$ is free, while $y$ is bound.

\begin{mathpar}
  \inferrule* [lab=monoidal-laws] {} { P|Q \equiv Q|P \and P|0 \equiv P \and P|(Q|R) \equiv (P|Q)|R }
\end{mathpar}

\begin{mathpar}
  \inferrule* [lab=alpha-equivalence] {} { (x)P \equiv (y)P\{y/x\} \and y \not\in \freenames{P} }
\end{mathpar}

\begin{definition}
Then two processes, $P,Q$, are alpha-equivalent if $P = Q\{\vec{y}/\vec{x}\}$ for
some $\vec{x} \in \boundnames{Q},\vec{y} \in \boundnames{P}$, where $Q\{\vec{y}/\vec{x}\}$
denotes the capture-avoiding substitution of $\vec{y}$ for $\vec{x}$ in $Q$.
\end{definition}

\begin{definition}
  The {\em structural congruence} \cite{SangiorgiWalker} , $\equiv$,
  between processes is the least congruence containing
  alpha-equivalence, satisfying the abelian monoid laws
  (associativity, commutativity and $\pzero$ as identity) for parallel
  composition $|$ and for summation $+$.
\end{definition}

\subsection{Name equivalence}

We take name equivalence, written $\nameeq$, to be the smallest
equivalence relation generated by the following rules.

\begin{mathpar}
\inferrule*[lab=Quote-drop]
{ }
{ \quotep{@{x}} \nameeq x }

\inferrule*[lab=Struct-equiv]
{ P \scong Q }
{ \quotep{P} \nameeq \quotep{Q} }
\end{mathpar}

The astute reader will have noticed that the mutual recursion of names
and processes imposes a mutual recursion on alpha-equivalence and
structural equivalence via name-equivalence. Fortunately, all of this
works out pleasantly and we may calculate in the natural way, free of
concern. The reader interested in the details is referred to the
appendix \ref{appendix:rho_details}.

\subsection{Substitution}

We use $\Proc$ for the set of processes, $\QProc$ for the set of
names, and $\id{\{}\vec{y} / \vec{x} \id{\}}$ to denote partial maps,
$s : \QProc \rightarrow \QProc$. A map, $s$ lifts, uniquely, to a map
on process terms, $\widehat{s} : \Proc \rightarrow \Proc$ by the
following equations.

\begin{mathpar}
  (0) \psubstp{Q}{P} := 0 \\
  (R \juxtap S) \psubstp{Q}{P}
  :=    
  (R)\psubstp{Q}{P} \juxtap (S) \psubstp{Q}{P} \\
  (x?(y).R) \psubstp{Q}{P}    
  :=    
  (x)\substp{Q}{P} (z)\concat( (R \psubstn{z}{y}) \psubstp{Q}{P} ) \\
  (\lift{x}{R}) \psubstp{Q}{P}  
  :=
  \lift{(x)\substp{Q}{P}}{ R \psubstp{Q}{P} } \\
%   (\dropn{x})  \psubstp{Q}{P}       
%   := 
%   \left\{ 
%     \begin{array}{ccc} 
%       \dropn{\quotep{Q}} & & x \nameeq \quotep{P} \\
%       \dropn{x} & & otherwise \\
%     \end{array}
%   \right. 
  (\dropn{x})  \psubstp{Q}{P}       
  := 
  \left\{ 
    \begin{array}{ccc} 
      Q & & x \nameeq \quotep{P} \\
      \dropn{x} & & otherwise \\
    \end{array}
  \right.
\end{mathpar}
 

where

\begin{eqnarray}
  (x)\id{\{} \lpquote Q \rpquote / \lpquote P \rpquote \id{\}}            = 
  \left\{ 
    \begin{array}{ccc}
      \lpquote Q \rpquote & & x \nameeq \lpquote P \rpquote \\
      x & & otherwise \\
    \end{array}
  \right. \nonumber
\end{eqnarray}

and $z$ is chosen distinct from $\quotep{P}$, $\quotep{Q}$, the free
names in $Q$, and all the names in $R$. Our $\alpha$-equivalence will
be built in the standard way from this substitution.

\begin{remark}\label{rem:no_self_referential_names}
  One consequence of these definitions is that $\forall P. \quotep{P}
  \not\in \freenames{P}$.
\end{remark}

\subsection{ Dynamic quote: an example }

Anticipating something of what's to come, consider applying the
substitution, $\widehat{\id{\{}u / z \id{\}}}$, to the following pair
of processes, $\lift{w}{y!(z)}$ and $w[ \lpquote y!(z) \rpquote ]$.

\begin{eqnarray}
	\lift{w}{y!(z)}\widehat{\id{\{}u / z \id{\}}}
		& = &
		\lift{w}{y!(u)} \nonumber\\
	w[ \lpquote y!(z) \rpquote ] \widehat{ \id{\{}u / z \id{\}} }
		& = &
		w[ \lpquote y!(z) \rpquote ] \nonumber
\end{eqnarray}

Because the body of the process between quotes is impervious to
substitution, we get radically different answers. In fact, by
examining the first process in an input context,
e.g. $x?(z).\lift{w}{y!(z)}$, we see that the process under the lift
operator may be shaped by prefixed inputs binding a name inside it. In
this sense, the lift operator will be seen as a way to dynamically
construct processes before reifying them as names.

Finally equipped with these standard features we can present the
dynamics of the calculus.

\subsubsection{Operational semantics} 

Finally, we introduce the computational dynamics. What marks these
algebras as distinct from other more traditionally studied algebraic
structures, e.g. vector spaces or polynomial rings, is the manner in
which dynamics is captured. In traditional structures, dynamics is typically
expressed through morphisms between such structures, as in linear maps
between vector spaces or morphisms between rings. In algebras
associated with the semantics of computation, the dynamics is
expressed as part of the algebraic structure itself, through a
reduction reduction relation typically denoted by $\red$. Below, we
give a recursive presentation of this relation for the calculus used
in the encoding.

$\red \subseteq \pi \times \pi$
$\red : \pi \to \mathcal{P}(\pi)$

\begin{mathpar}
  \inferrule* [lab=Comm] { \textsf{match}( x_{src}, x_{trgt} ) } { x_{trgt}?(y)P \; | \; x_{src}!\langle {Q} \rangle \red P\{\quotep{Q}/y}\} }
  \and \\
  \inferrule* [lab=Par] {{P} \red {P}'} {{{P} | {Q}} \red {{P}' | {Q}}}
  \and
  \inferrule* [lab=Equiv]{{{P} \scong {P}'} \andalso {{P}' \red {Q}'} \andalso {{Q}' \scong {Q}}}{{P} \red {Q}}
\end{mathpar}

\begin{eqnarray*}
  match_{\equiv} (\quotep{P},\quotep{Q}) & := & P \equiv Q \\
  match_{\dagger}(\quotep{P},\quotep{Q}) & := & \forall R. P|Q \red^{*} R => R \red^{*} 0 \\
  match_{K}(\quotep{P},\quotep{Q}) & := & K \mbox{ for some context } K
\end{eqnarray*}

$u?(x)P | u!\langle Q \rangle \red P\{\quotep{Q}/x\}$

%We write $\wred$ for $\red^*$, and $P\red$ if $\exists Q $ such that $ P \red Q$.
We write $P\red$ if $\exists Q $ such that $ P \red Q$ and $P\not\red$, otherwise.

\section{Replication}

As mentioned before, it is known that replication (and hence
recursion) can be implemented in a higher-order process algebra
\cite{SangiorgiWalker}. As our first example of calculation with the
machinery thus far presented we give the construction explicitly in
the {\rhoc}.

\begin{eqnarray}
	D_{x} & := & \prefix{x}{y}{(\binpar{\outputp{x}{y}}{@{y}})} \nonumber\\
	\bangp_{x}{P} & := & \binpar{{x}!\langle{\binpar{D_{x}}{P}}\rangle}{D_{x}} \nonumber
\end{eqnarray}

\begin{eqnarray}
	\bangp_{x}{P} & & \nonumber\\
	=
	& {x}!\langle{(\prefix{x}{y}{(\outputp{x}{y} | @{y})) | P}}\rangle 
	      | \prefix{x}{y}{(\outputp{x}{y} | @{y})} & \nonumber\\
	\red
	& (\outputp{x}{y} | @{y})\substn{\quotep{(\prefix{x}{y}{(@{y} | \outputp{x}{y})) | P}}}{y} & \nonumber\\
	=
	& \outputp{x}{\quotep{(\prefix{x}{y}{(\outputp{x}{y} | @{y})) | P}}}
	  | {(\prefix{x}{y}{(\outputp{x}{y} | @{y})) | P}} & \nonumber\\
	\red
	& \ldots & \nonumber\\
	\red^*
	& P | P | \ldots & \nonumber
\end{eqnarray}

Of course, this encoding, as an implementation, runs away, unfolding
$\bangp{P}$ eagerly. A lazier and more implementable replication
operator, restricted to input-guarded processes, may be obtained as follows.

\begin{eqnarray}
\bangp{\prefix{u}{v}{P}} 
	:= 
	\binpar{\lift{x}{\prefix{u}{v}{(\binpar{D(x)}{P})}}}{D(x)} \nonumber
\end{eqnarray}

\begin{remark}
  Note that the lazier definition still does not deal with summation
  or mixed summation (i.e. sums over input and output). The reader is
  invited to construct definitions of replication that deal with these
  features. 

  Further, the definitions are parameterized in a name, $x$. Can you,
  gentle reader, make a definition that eliminates this parameter and
  guarantees no accidental interaction between the replication
  machinery and the process being replicated -- i.e. no accidental
  sharing of names used by the process to get its work done and the
  name(s) used by the replication to effect copying. This latter
  revision of the definition of replication is crucial to obtaining
  the expected identity $!!P \sim !P$.
\end{remark}

\begin{remark}\label{rem:paradoxical_combinator}
  The reader familiar with the lambda calculus will have noticed the
  similarity between $D$ and the paradoxical combinator.

  [Ed. note: the existence of this seems to suggest we have to be more
  restrictive on the set of processes and names we admit if we are to
  support no-cloning.]
\end{remark}

\subsubsection{Bisimulation}

The computational dynamics gives rise to another kind of equivalence,
the equivalence of computational behavior. As previously mentioned
this is typically captured \emph{via} some form of bisimulation.

% The notion we use in this paper is weak barbed bisimulation
% \cite{milner91polyadicpi}.

The notion we use in this paper is derived from weak barbed
bisimulation \cite{milner91polyadicpi}. 

\begin{definition}
An \emph{observation relation}, $\downarrow_{\mathcal N}$, over a set
of names, $\mathcal N$, is the smallest relation satisfying the rules
below.

\infrule[Out-barb]{y \in {\mathcal N}, \; x \nameeq y}
		  {\outputp{x}{v} \downarrow_{\mathcal N} x}
\infrule[Par-barb]{\mbox{$P\downarrow_{\mathcal N} x$ or $Q\downarrow_{\mathcal N} x$}}
		  {\binpar{P}{Q} \downarrow_{\mathcal N} x}

We write $P \Downarrow_{\mathcal N} x$ if there is $Q$ such that 
$P \wred Q$ and $Q \downarrow_{\mathcal N} x$.
\end{definition}

\begin{definition}
%\label{def.bbisim}
An  ${\mathcal N}$-\emph{barbed bisimulation} over a set of names, ${\mathcal N}$, is a symmetric binary relation 
${\mathcal S}_{\mathcal N}$ between agents such that $P\rel{S}_{\mathcal N}Q$ implies:
\begin{enumerate}
\item If $P \red P'$ then $Q \wred Q'$ and $P'\rel{S}_{\mathcal N} Q'$.
\item If $P\downarrow_{\mathcal N} x$, then $Q\Downarrow_{\mathcal N} x$.
\end{enumerate}
$P$ is ${\mathcal N}$-barbed bisimilar to $Q$, written
$P \wbbisim_{\mathcal N} Q$, if $P \rel{S}_{\mathcal N} Q$ for some ${\mathcal N}$-barbed bisimulation ${\mathcal S}_{\mathcal N}$.
\end{definition}

$\mathcal{R} \subseteq \pi \times \pi$

$P \mathcal{R} Q => \forall P'. P \red P' \Rightarrow \exists Q'. Q \red Q', P' \mathcal{R} Q'$

$P \vdash x \Rightarrow Q \vdash x$

\begin{mathpar}
  \inferrule*[lab=Out-barb]{x \nameeq y}{{y}!\langle{Q}\rangle \vdash x}
  \and
  \inferrule*[lab=Par-barb]{\mbox{$P\vdash x$ or $Q\vdash x$}}{\binpar{P}{Q} \vdash x}
\end{mathpar}

\subsubsection{Contexts}

One of the principle advantages of computational calculi like the
$\pi$-calculus is a well-defined notion of context,
contextual-equivalence and a correlation between
contextual-equivalence and notions of bisimulation. The notion of
context allows the decomposition of a process into (sub-)process and
its syntactic environment, its context. Thus, a context may be
thought of as a process with a ``hole'' (written $\Box$) in it. The
application of a context $M$ to a process $P$, written $M[P]$, is
tantamount to filling the hole in $M$ with $P$. In this paper we do
not need the full weight of this theory, but do make use of the notion
of context in the proof the main theorem. 

\begin{mathpar}
  \inferrule* [lab=summation] {} {{M_{M},M_{N}} \bc \Box \;|\; x.M_{A} \;|\; M_{M}+M_{N}}
  \and
  \inferrule* [lab=agent] {} {{M_{A}} \bc (\vec{x})M_{P} \;| \; \clift{P_0,\ldots,M_{P},\ldots,P_N}}
  \and \\
  \inferrule* [lab=process] {} {{M_{P}} \bc M_{N} \;| \;P|M_{P} }
\end{mathpar} 

\begin{mathpar}
  \inferrule* [lab=sychronization] {} {M_{N} \bc \Box \;|\; x?M_{F} \;|\; x!M_{C}}
  \and
  \inferrule* [lab=abstraction] {} {{M_{F}} \bc (x)M_{P} }
  \and
  \inferrule* [lab=concretion] {} {{M_{C}} \bc \langle M_{P} \rangle }
  \and \\
  \inferrule* [lab=process] {} {{M_{P}} \bc M_{N} \;| \;P|M_{P} }
\end{mathpar}

\begin{definition}[contextual application] Given a context $M$, and
  process $P$, we define the \emph{contextual application}, $M[P] :=
  M\{P/\Box\}$. That is, the contextual application of M to P is the
  substitution of $P$ for $\Box$ in $M$.
\end{definition}

$\meaningof{-} : L \to \mathcal{P}(\pi)$

\begin{mathpar}
  \inferrule* [lab=collection] {} {\meaningof{true} = \pi, \and \meaningof{~E} = \pi \setminus \meaningof{E}, \and \meaningof{E_{1} \& E_{2}} = \meaningof{E_{1}} \cap \meaningof{E_{2}}}
\end{mathpar}

\begin{mathpar}
  \inferrule* [lab=structure] {} {\meaningof{0} = \{ P \in \pi | P \equiv 0 \}, \and \\ \meaningof{E_1 | E_2} = \{ P \in \pi | P \equiv P_{1} | P_{2}, P_{1} \in \meaningof{E_{1}}, P_{2} \in \meaningof{E_2}\} }
\end{mathpar}

\begin{mathpar}
 \inferrule* [lab=behavior] {} {\meaningof{\langle a?b \rangle E} = \{ P \in \pi | P \equiv Q | u?(y)P', \\ \and \\\\ \and \\ \;\;\; u \in \meaningof{a}, \forall z.P'\{z/y\} \in \meaningof{E\{z/b\}}\}, \and \\ \meaningof{a!E} = \{ P \in \pi | P \equiv Q | x!\langle P' \rangle, x \in \meaningof{a} P' \in \meaningof{E}\} }
\end{mathpar}

\begin{mathpar}
 \inferrule* [lab=nominal] {} {\meaningof{\quotep{E}} = \{ \quotep{P} \in \quotep{\pi} | P \in \meaningof{E} \}, \and \meaningof{\quotep{P}} = \{ \quotep{Q} \in \quotep{\pi} | P \equiv Q \} \and \\ \meaningof{@\quotep{E}} = \{ P \in \pi | P \equiv @x, x \in \meaningof{E} \}}
\end{mathpar}

\begin{eqnarray*}
  \\
  \meaningof{-} : TS \to ST
\end{eqnarray*}

\begin{eqnarray*}
  \\
  L : TS \to ST
\end{eqnarray*}

\begin{eqnarray*}
  \\
  P \models E \iff P \in \meaningof{E}
\end{eqnarray*}

\begin{eqnarray*}
  P \approx_{L} Q \iff \forall E \in L. P \models E \iff Q \models E
\end{eqnarray*}

\begin{eqnarray*}
  P \approx_{K} Q
\end{eqnarray*}

\begin{eqnarray*}
  P \approx Q
\end{eqnarray*}

$\approx_{K} = \approx = \approx_{L}$

\subsubsection{Contextual duality}

Note that contexts extend the quotation operation to a family of
operations from processes to names. Given a context, $M$, we can
define a \emph{nominal context}, $\quotep{M}$ by $\quotep{M}[P] :=
\quotep{M[P]}$. To foreshadow what is to come we observe that these
operations enjoy a duality with processes very much like the duality
between vectors and maps from vectors to scalars.

Further, because the calculus is essentially higher-order, we have a
correspondence between contexts and processes. More specifically,
given a name $x$ and a context $M$ we can construct $M^{*}_{x}$ such
that 

\begin{mathpar}
  M^{*}_{x} | \lift{x}{P} \red M[P]
\end{mathpar}

namely,

\begin{mathpar}
  M^{*}_{x} := x?(u).M[\dropn{u}]
\end{mathpar}

The dependence of $M^{*}_{x}$ on a name makes it an abstraction, 

\begin{mathpar}
  M^{*} := (x)x?(u).M[\dropn{u}]
\end{mathpar}

\subsection{Additional notation}

It will sometimes be convenient to denote the process a name
quotes. We already have the notation $x = \quotep{P}$, but it will be
convenient to introduce an alternate notation, $\procn{x}$, when we
want to emphasize the connection to the use of the name. Note that, by
virtue of name equivalence, $\quotep{\procn{x}} \nameeq x$; so, the
notation is consistent with previous definitions.

Further, because names have structure it is possible to effect
substitutions on the basis of that structure. This means we need to
upgrade our notation for substitutions, which we accomplish by
adapting comprehension notation. Thus,

\begin{mathpar}
  P\{ y / x : x \in S \}
\end{mathpar}

is interpreted to mean the process derived from P by replacing (in a
capture-avoiding manner) each occurrence of $x$ in $S$ by $y$. For example,

\begin{mathpar}
  P\{ \quotep{\procn{x}|\procn{x}} / x : x \in \freenames{P} \}
\end{mathpar}

will replace each (occurrence) of a free name $x$ in $P$ by
$\quotep{\procn{x}|\procn{x}}$.

Also, we will avail ourselves of the notation $x^{L}$ and $x^{R}$ to
denote injections of a name into disjoint copies of the name
space. There are numerous ways to accomplish this. One example can be
found in \cite{MeredithR05}. This notation overloads to vectors of
names: $\vec{x}^{\pi} := (x_{i}^{\pi} \; : \; 0 \leq i < |\vec{x}| )$ where $\pi \in \{L,R\}$.

We also use $P^{\Box} := P|\Box$.

In \cite{MeredithR05} an interpretation of the new operator is
given. It turns out that there are several possible interpretations
all enjoying the requisite algebraic properties of the operator (see
\cite{milner91polyadicpi}). We will therefore make liberal use of
$(\nu\; \vec{x})P$.

% subsection the_syntax_and_semantics_of_the_notation_system (end)   

\input{qm2pi.qmops} 

\input{qm2pi.sterngerlach} 

\input{qm2pi.metric} 

% section concurrent_process_calculi (end)

%\input{qm2pi.proofsketch}

% section proof sketch (end)

%\input{qm2pi.slviaknots} 

% section spatial logic via knots (end)

\input{qm2pi.conclusion}

% section conclusion (end)

%\input{qm2pi.dtcodes} 

% section wiring algorithm (end)

\input{qm2pi.ack} 

% section acknowledgments (end)

\newpage


\bibliographystyle{plain}   
\bibliography{../../biblios/main.bib}

\input{qm2pi.rhodetails}

\end{document}

 

% section notation (end)

\input{qm2pi.process.calculi} 

% section concurrent_process_calculi_and_spatial_logics_ (end)
    
%\documentclass[12pt]{llncs}
%\documentclass{jktr}

\usepackage[pdftex]{hyperref}                   
\usepackage {listings}
\usepackage {mathpartir}
\usepackage{bcprules}
%\usepackage{listings}
                       
\usepackage{graphicx} 
%\usepackage[margins=2.5cm,nohead,nofoot]{geometry}
%\usepackage{geometry}
\usepackage{amsfonts}
\usepackage{amstext}
\usepackage{latexsym}
\usepackage{amssymb}
\usepackage{color}


%\include{myPreamble}
\include{qm2pi.local} 

%\ifpdf
%\usepackage[pdftex]{graphicx}
%\else
%\usepackage{graphicx}
%\fi

 % \ifpdf
%  \usepackage{pdfsync}
%  \if


%\title{Brief Article}
%\author{David F. Snyder}
%\author{L.G. Meredith}

%\address{Dept. of Math., Texas State University--San Marcos, San Marcos, TX 78666}
       
\pagestyle{empty}


\begin{document}

\lstset{language=[Objective]Caml,frame=shadowbox}

\input{qm2pi.front}

% section front matter (end)

\input{qm2pi.intro} 
 
% section introduction (end)

% \input{qm2pi.knotations} 

% section notation (end)

\input{qm2pi.process.calculi} 

% section concurrent_process_calculi_and_spatial_logics_ (end)
    
%\input{qm2pi.knots2pi} 

%\input{qm2pi.trefoil} 

%\input{qm2pi.mainthm} 

% subsection basic_interpretation (end)

%\input{qm2pi.rho.presentation} 
\subsection{The syntax and semantics of the notation system}\label{sub:the_syntax_and_semantics_of_the_notation_system} % (fold)

We now summarize a technical presentation of the calculus that
embodies our theory of dynamics. The typical presentation of such a
calculus follows the style of giving generators and relations on
them. The grammar, below, describing term constructors, freely
generates the set of processes, $\Proc$. This set is then quotiented
by a relation known as structural congruence and it is over this set
that the notion of dynamics is expressed. This presentation is
essentially that of \cite{MeredithR05} with the addition of
polyadicity and summation. For readability we have relegated some of
the technical subtleties to an appendix.

\subsubsection{Process grammar}\label{subsub:process_grammar}

\begin{mathpar}
  \inferrule* [lab=synchronization] {} {{M} \bc \pzero \;|\; x?F \;|\; x!C }
  \and
  \inferrule* [lab=abstraction] {} {{F} \bc (x)P}
  \and
  \inferrule* [lab=concretion] {} {{C} \bc \langle Q \rangle}
  \and
  \inferrule* [lab=process] {} {{P,Q} \bc M \;| \;P|Q \;|\; @{x}}
  \and
  \inferrule* [lab=name] {} {{x} \bc \quotep{P}}
\end{mathpar} 

Note that $\vec{x}$ (resp. $\vec{P}$) denotes a vector of names
(resp. processes) of length $|\vec{x}|$ (resp. $|\vec{P}|$). We adopt
the following useful abbreviations.

\begin{mathpar}
   x?(\vec{y}).P := x.(\vec{y})P \and  x\clift{\vec{P}} := x.\clift{\vec{P}}
   \and x!(y) := \lift{x}{\dropn{y}}
   \and \Pi_{i=0}^{n-1}P_i := P_0 | \ldots | P_{n-1}
\end{mathpar}

\subsubsection{Structural congruence}

\paragraph{Free and bound names and alpha-equivalence.} At the
core of structural equivalence is alpha-equivalence which identifies
process that are the same up to a change of variable. Formally, we
recognize the distinction between free and bound names. The free names
of a process, $\freenames{P}$, may be calculated recursively as
follows:

\begin{mathpar}
\freenames{\pzero} := \emptyset
  \and \\
  \freenames{x?(y).P} := \{ x \} \cup (\freenames{P} \setminus \{ y \})
  \and 
  \freenames{x!\langle P \rangle} := \{ x \} \cup \{ P \} 
  \and \\
  \freenames{P|Q} := \freenames{P} \cup \freenames{Q}
  \and \\
  \freenames{@{x}} := \{ x \}
\end{mathpar}

$\pi$
$\quotep{\pi}$

$\freenames{-} : \pi \to \mathcal{P}(\quotep{\pi})$

\begin{eqnarray*}
  \freenames{\pzero} & := & \emptyset \\
  \freenames{x?(y).P} & := & \{ x \} \cup (\freenames{P} \setminus \{ y \}) \\
  \freenames{x!\langle P \rangle} & := & \{ x \} \cup \{ P \} \\
  \freenames{P|Q} & := & \freenames{P} \cup \freenames{Q} \\
  \freenames{\dropn{x}} & := & \{ x \}
\end{eqnarray*}

The bound names of a process, $\boundnames{P}$, are those names occurring in $P$
that are not free. For example, in $x?(y).0$, the name $x$ is free, while $y$ is bound.

\begin{mathpar}
  \inferrule* [lab=monoidal-laws] {} { P|Q \equiv Q|P \and P|0 \equiv P \and P|(Q|R) \equiv (P|Q)|R }
\end{mathpar}

\begin{mathpar}
  \inferrule* [lab=alpha-equivalence] {} { (x)P \equiv (y)P\{y/x\} \and y \not\in \freenames{P} }
\end{mathpar}

\begin{definition}
Then two processes, $P,Q$, are alpha-equivalent if $P = Q\{\vec{y}/\vec{x}\}$ for
some $\vec{x} \in \boundnames{Q},\vec{y} \in \boundnames{P}$, where $Q\{\vec{y}/\vec{x}\}$
denotes the capture-avoiding substitution of $\vec{y}$ for $\vec{x}$ in $Q$.
\end{definition}

\begin{definition}
  The {\em structural congruence} \cite{SangiorgiWalker} , $\equiv$,
  between processes is the least congruence containing
  alpha-equivalence, satisfying the abelian monoid laws
  (associativity, commutativity and $\pzero$ as identity) for parallel
  composition $|$ and for summation $+$.
\end{definition}

\subsection{Name equivalence}

We take name equivalence, written $\nameeq$, to be the smallest
equivalence relation generated by the following rules.

\begin{mathpar}
\inferrule*[lab=Quote-drop]
{ }
{ \quotep{@{x}} \nameeq x }

\inferrule*[lab=Struct-equiv]
{ P \scong Q }
{ \quotep{P} \nameeq \quotep{Q} }
\end{mathpar}

The astute reader will have noticed that the mutual recursion of names
and processes imposes a mutual recursion on alpha-equivalence and
structural equivalence via name-equivalence. Fortunately, all of this
works out pleasantly and we may calculate in the natural way, free of
concern. The reader interested in the details is referred to the
appendix \ref{appendix:rho_details}.

\subsection{Substitution}

We use $\Proc$ for the set of processes, $\QProc$ for the set of
names, and $\id{\{}\vec{y} / \vec{x} \id{\}}$ to denote partial maps,
$s : \QProc \rightarrow \QProc$. A map, $s$ lifts, uniquely, to a map
on process terms, $\widehat{s} : \Proc \rightarrow \Proc$ by the
following equations.

\begin{mathpar}
  (0) \psubstp{Q}{P} := 0 \\
  (R \juxtap S) \psubstp{Q}{P}
  :=    
  (R)\psubstp{Q}{P} \juxtap (S) \psubstp{Q}{P} \\
  (x?(y).R) \psubstp{Q}{P}    
  :=    
  (x)\substp{Q}{P} (z)\concat( (R \psubstn{z}{y}) \psubstp{Q}{P} ) \\
  (\lift{x}{R}) \psubstp{Q}{P}  
  :=
  \lift{(x)\substp{Q}{P}}{ R \psubstp{Q}{P} } \\
%   (\dropn{x})  \psubstp{Q}{P}       
%   := 
%   \left\{ 
%     \begin{array}{ccc} 
%       \dropn{\quotep{Q}} & & x \nameeq \quotep{P} \\
%       \dropn{x} & & otherwise \\
%     \end{array}
%   \right. 
  (\dropn{x})  \psubstp{Q}{P}       
  := 
  \left\{ 
    \begin{array}{ccc} 
      Q & & x \nameeq \quotep{P} \\
      \dropn{x} & & otherwise \\
    \end{array}
  \right.
\end{mathpar}
 

where

\begin{eqnarray}
  (x)\id{\{} \lpquote Q \rpquote / \lpquote P \rpquote \id{\}}            = 
  \left\{ 
    \begin{array}{ccc}
      \lpquote Q \rpquote & & x \nameeq \lpquote P \rpquote \\
      x & & otherwise \\
    \end{array}
  \right. \nonumber
\end{eqnarray}

and $z$ is chosen distinct from $\quotep{P}$, $\quotep{Q}$, the free
names in $Q$, and all the names in $R$. Our $\alpha$-equivalence will
be built in the standard way from this substitution.

\begin{remark}\label{rem:no_self_referential_names}
  One consequence of these definitions is that $\forall P. \quotep{P}
  \not\in \freenames{P}$.
\end{remark}

\subsection{ Dynamic quote: an example }

Anticipating something of what's to come, consider applying the
substitution, $\widehat{\id{\{}u / z \id{\}}}$, to the following pair
of processes, $\lift{w}{y!(z)}$ and $w[ \lpquote y!(z) \rpquote ]$.

\begin{eqnarray}
	\lift{w}{y!(z)}\widehat{\id{\{}u / z \id{\}}}
		& = &
		\lift{w}{y!(u)} \nonumber\\
	w[ \lpquote y!(z) \rpquote ] \widehat{ \id{\{}u / z \id{\}} }
		& = &
		w[ \lpquote y!(z) \rpquote ] \nonumber
\end{eqnarray}

Because the body of the process between quotes is impervious to
substitution, we get radically different answers. In fact, by
examining the first process in an input context,
e.g. $x?(z).\lift{w}{y!(z)}$, we see that the process under the lift
operator may be shaped by prefixed inputs binding a name inside it. In
this sense, the lift operator will be seen as a way to dynamically
construct processes before reifying them as names.

Finally equipped with these standard features we can present the
dynamics of the calculus.

\subsubsection{Operational semantics} 

Finally, we introduce the computational dynamics. What marks these
algebras as distinct from other more traditionally studied algebraic
structures, e.g. vector spaces or polynomial rings, is the manner in
which dynamics is captured. In traditional structures, dynamics is typically
expressed through morphisms between such structures, as in linear maps
between vector spaces or morphisms between rings. In algebras
associated with the semantics of computation, the dynamics is
expressed as part of the algebraic structure itself, through a
reduction reduction relation typically denoted by $\red$. Below, we
give a recursive presentation of this relation for the calculus used
in the encoding.

$\red \subseteq \pi \times \pi$
$\red : \pi \to \mathcal{P}(\pi)$

\begin{mathpar}
  \inferrule* [lab=Comm] { \textsf{match}( x_{src}, x_{trgt} ) } { x_{trgt}?(y)P \; | \; x_{src}!\langle {Q} \rangle \red P\{\quotep{Q}/y}\} }
  \and \\
  \inferrule* [lab=Par] {{P} \red {P}'} {{{P} | {Q}} \red {{P}' | {Q}}}
  \and
  \inferrule* [lab=Equiv]{{{P} \scong {P}'} \andalso {{P}' \red {Q}'} \andalso {{Q}' \scong {Q}}}{{P} \red {Q}}
\end{mathpar}

\begin{eqnarray*}
  match_{\equiv} (\quotep{P},\quotep{Q}) & := & P \equiv Q \\
  match_{\dagger}(\quotep{P},\quotep{Q}) & := & \forall R. P|Q \red^{*} R => R \red^{*} 0 \\
  match_{K}(\quotep{P},\quotep{Q}) & := & K \mbox{ for some context } K
\end{eqnarray*}

$u?(x)P | u!\langle Q \rangle \red P\{\quotep{Q}/x\}$

%We write $\wred$ for $\red^*$, and $P\red$ if $\exists Q $ such that $ P \red Q$.
We write $P\red$ if $\exists Q $ such that $ P \red Q$ and $P\not\red$, otherwise.

\section{Replication}

As mentioned before, it is known that replication (and hence
recursion) can be implemented in a higher-order process algebra
\cite{SangiorgiWalker}. As our first example of calculation with the
machinery thus far presented we give the construction explicitly in
the {\rhoc}.

\begin{eqnarray}
	D_{x} & := & \prefix{x}{y}{(\binpar{\outputp{x}{y}}{@{y}})} \nonumber\\
	\bangp_{x}{P} & := & \binpar{{x}!\langle{\binpar{D_{x}}{P}}\rangle}{D_{x}} \nonumber
\end{eqnarray}

\begin{eqnarray}
	\bangp_{x}{P} & & \nonumber\\
	=
	& {x}!\langle{(\prefix{x}{y}{(\outputp{x}{y} | @{y})) | P}}\rangle 
	      | \prefix{x}{y}{(\outputp{x}{y} | @{y})} & \nonumber\\
	\red
	& (\outputp{x}{y} | @{y})\substn{\quotep{(\prefix{x}{y}{(@{y} | \outputp{x}{y})) | P}}}{y} & \nonumber\\
	=
	& \outputp{x}{\quotep{(\prefix{x}{y}{(\outputp{x}{y} | @{y})) | P}}}
	  | {(\prefix{x}{y}{(\outputp{x}{y} | @{y})) | P}} & \nonumber\\
	\red
	& \ldots & \nonumber\\
	\red^*
	& P | P | \ldots & \nonumber
\end{eqnarray}

Of course, this encoding, as an implementation, runs away, unfolding
$\bangp{P}$ eagerly. A lazier and more implementable replication
operator, restricted to input-guarded processes, may be obtained as follows.

\begin{eqnarray}
\bangp{\prefix{u}{v}{P}} 
	:= 
	\binpar{\lift{x}{\prefix{u}{v}{(\binpar{D(x)}{P})}}}{D(x)} \nonumber
\end{eqnarray}

\begin{remark}
  Note that the lazier definition still does not deal with summation
  or mixed summation (i.e. sums over input and output). The reader is
  invited to construct definitions of replication that deal with these
  features. 

  Further, the definitions are parameterized in a name, $x$. Can you,
  gentle reader, make a definition that eliminates this parameter and
  guarantees no accidental interaction between the replication
  machinery and the process being replicated -- i.e. no accidental
  sharing of names used by the process to get its work done and the
  name(s) used by the replication to effect copying. This latter
  revision of the definition of replication is crucial to obtaining
  the expected identity $!!P \sim !P$.
\end{remark}

\begin{remark}\label{rem:paradoxical_combinator}
  The reader familiar with the lambda calculus will have noticed the
  similarity between $D$ and the paradoxical combinator.

  [Ed. note: the existence of this seems to suggest we have to be more
  restrictive on the set of processes and names we admit if we are to
  support no-cloning.]
\end{remark}

\subsubsection{Bisimulation}

The computational dynamics gives rise to another kind of equivalence,
the equivalence of computational behavior. As previously mentioned
this is typically captured \emph{via} some form of bisimulation.

% The notion we use in this paper is weak barbed bisimulation
% \cite{milner91polyadicpi}.

The notion we use in this paper is derived from weak barbed
bisimulation \cite{milner91polyadicpi}. 

\begin{definition}
An \emph{observation relation}, $\downarrow_{\mathcal N}$, over a set
of names, $\mathcal N$, is the smallest relation satisfying the rules
below.

\infrule[Out-barb]{y \in {\mathcal N}, \; x \nameeq y}
		  {\outputp{x}{v} \downarrow_{\mathcal N} x}
\infrule[Par-barb]{\mbox{$P\downarrow_{\mathcal N} x$ or $Q\downarrow_{\mathcal N} x$}}
		  {\binpar{P}{Q} \downarrow_{\mathcal N} x}

We write $P \Downarrow_{\mathcal N} x$ if there is $Q$ such that 
$P \wred Q$ and $Q \downarrow_{\mathcal N} x$.
\end{definition}

\begin{definition}
%\label{def.bbisim}
An  ${\mathcal N}$-\emph{barbed bisimulation} over a set of names, ${\mathcal N}$, is a symmetric binary relation 
${\mathcal S}_{\mathcal N}$ between agents such that $P\rel{S}_{\mathcal N}Q$ implies:
\begin{enumerate}
\item If $P \red P'$ then $Q \wred Q'$ and $P'\rel{S}_{\mathcal N} Q'$.
\item If $P\downarrow_{\mathcal N} x$, then $Q\Downarrow_{\mathcal N} x$.
\end{enumerate}
$P$ is ${\mathcal N}$-barbed bisimilar to $Q$, written
$P \wbbisim_{\mathcal N} Q$, if $P \rel{S}_{\mathcal N} Q$ for some ${\mathcal N}$-barbed bisimulation ${\mathcal S}_{\mathcal N}$.
\end{definition}

$\mathcal{R} \subseteq \pi \times \pi$

$P \mathcal{R} Q => \forall P'. P \red P' \Rightarrow \exists Q'. Q \red Q', P' \mathcal{R} Q'$

$P \vdash x \Rightarrow Q \vdash x$

\begin{mathpar}
  \inferrule*[lab=Out-barb]{x \nameeq y}{{y}!\langle{Q}\rangle \vdash x}
  \and
  \inferrule*[lab=Par-barb]{\mbox{$P\vdash x$ or $Q\vdash x$}}{\binpar{P}{Q} \vdash x}
\end{mathpar}

\subsubsection{Contexts}

One of the principle advantages of computational calculi like the
$\pi$-calculus is a well-defined notion of context,
contextual-equivalence and a correlation between
contextual-equivalence and notions of bisimulation. The notion of
context allows the decomposition of a process into (sub-)process and
its syntactic environment, its context. Thus, a context may be
thought of as a process with a ``hole'' (written $\Box$) in it. The
application of a context $M$ to a process $P$, written $M[P]$, is
tantamount to filling the hole in $M$ with $P$. In this paper we do
not need the full weight of this theory, but do make use of the notion
of context in the proof the main theorem. 

\begin{mathpar}
  \inferrule* [lab=summation] {} {{M_{M},M_{N}} \bc \Box \;|\; x.M_{A} \;|\; M_{M}+M_{N}}
  \and
  \inferrule* [lab=agent] {} {{M_{A}} \bc (\vec{x})M_{P} \;| \; \clift{P_0,\ldots,M_{P},\ldots,P_N}}
  \and \\
  \inferrule* [lab=process] {} {{M_{P}} \bc M_{N} \;| \;P|M_{P} }
\end{mathpar} 

\begin{mathpar}
  \inferrule* [lab=sychronization] {} {M_{N} \bc \Box \;|\; x?M_{F} \;|\; x!M_{C}}
  \and
  \inferrule* [lab=abstraction] {} {{M_{F}} \bc (x)M_{P} }
  \and
  \inferrule* [lab=concretion] {} {{M_{C}} \bc \langle M_{P} \rangle }
  \and \\
  \inferrule* [lab=process] {} {{M_{P}} \bc M_{N} \;| \;P|M_{P} }
\end{mathpar}

\begin{definition}[contextual application] Given a context $M$, and
  process $P$, we define the \emph{contextual application}, $M[P] :=
  M\{P/\Box\}$. That is, the contextual application of M to P is the
  substitution of $P$ for $\Box$ in $M$.
\end{definition}

$\meaningof{-} : L \to \mathcal{P}(\pi)$

\begin{mathpar}
  \inferrule* [lab=collection] {} {\meaningof{true} = \pi, \and \meaningof{~E} = \pi \setminus \meaningof{E}, \and \meaningof{E_{1} \& E_{2}} = \meaningof{E_{1}} \cap \meaningof{E_{2}}}
\end{mathpar}

\begin{mathpar}
  \inferrule* [lab=structure] {} {\meaningof{0} = \{ P \in \pi | P \equiv 0 \}, \and \\ \meaningof{E_1 | E_2} = \{ P \in \pi | P \equiv P_{1} | P_{2}, P_{1} \in \meaningof{E_{1}}, P_{2} \in \meaningof{E_2}\} }
\end{mathpar}

\begin{mathpar}
 \inferrule* [lab=behavior] {} {\meaningof{\langle a?b \rangle E} = \{ P \in \pi | P \equiv Q | u?(y)P', \\ \and \\\\ \and \\ \;\;\; u \in \meaningof{a}, \forall z.P'\{z/y\} \in \meaningof{E\{z/b\}}\}, \and \\ \meaningof{a!E} = \{ P \in \pi | P \equiv Q | x!\langle P' \rangle, x \in \meaningof{a} P' \in \meaningof{E}\} }
\end{mathpar}

\begin{mathpar}
 \inferrule* [lab=nominal] {} {\meaningof{\quotep{E}} = \{ \quotep{P} \in \quotep{\pi} | P \in \meaningof{E} \}, \and \meaningof{\quotep{P}} = \{ \quotep{Q} \in \quotep{\pi} | P \equiv Q \} \and \\ \meaningof{@\quotep{E}} = \{ P \in \pi | P \equiv @x, x \in \meaningof{E} \}}
\end{mathpar}

\begin{eqnarray*}
  \\
  \meaningof{-} : TS \to ST
\end{eqnarray*}

\begin{eqnarray*}
  \\
  L : TS \to ST
\end{eqnarray*}

\begin{eqnarray*}
  \\
  P \models E \iff P \in \meaningof{E}
\end{eqnarray*}

\begin{eqnarray*}
  P \approx_{L} Q \iff \forall E \in L. P \models E \iff Q \models E
\end{eqnarray*}

\begin{eqnarray*}
  P \approx_{K} Q
\end{eqnarray*}

\begin{eqnarray*}
  P \approx Q
\end{eqnarray*}

$\approx_{K} = \approx = \approx_{L}$

\subsubsection{Contextual duality}

Note that contexts extend the quotation operation to a family of
operations from processes to names. Given a context, $M$, we can
define a \emph{nominal context}, $\quotep{M}$ by $\quotep{M}[P] :=
\quotep{M[P]}$. To foreshadow what is to come we observe that these
operations enjoy a duality with processes very much like the duality
between vectors and maps from vectors to scalars.

Further, because the calculus is essentially higher-order, we have a
correspondence between contexts and processes. More specifically,
given a name $x$ and a context $M$ we can construct $M^{*}_{x}$ such
that 

\begin{mathpar}
  M^{*}_{x} | \lift{x}{P} \red M[P]
\end{mathpar}

namely,

\begin{mathpar}
  M^{*}_{x} := x?(u).M[\dropn{u}]
\end{mathpar}

The dependence of $M^{*}_{x}$ on a name makes it an abstraction, 

\begin{mathpar}
  M^{*} := (x)x?(u).M[\dropn{u}]
\end{mathpar}

\subsection{Additional notation}

It will sometimes be convenient to denote the process a name
quotes. We already have the notation $x = \quotep{P}$, but it will be
convenient to introduce an alternate notation, $\procn{x}$, when we
want to emphasize the connection to the use of the name. Note that, by
virtue of name equivalence, $\quotep{\procn{x}} \nameeq x$; so, the
notation is consistent with previous definitions.

Further, because names have structure it is possible to effect
substitutions on the basis of that structure. This means we need to
upgrade our notation for substitutions, which we accomplish by
adapting comprehension notation. Thus,

\begin{mathpar}
  P\{ y / x : x \in S \}
\end{mathpar}

is interpreted to mean the process derived from P by replacing (in a
capture-avoiding manner) each occurrence of $x$ in $S$ by $y$. For example,

\begin{mathpar}
  P\{ \quotep{\procn{x}|\procn{x}} / x : x \in \freenames{P} \}
\end{mathpar}

will replace each (occurrence) of a free name $x$ in $P$ by
$\quotep{\procn{x}|\procn{x}}$.

Also, we will avail ourselves of the notation $x^{L}$ and $x^{R}$ to
denote injections of a name into disjoint copies of the name
space. There are numerous ways to accomplish this. One example can be
found in \cite{MeredithR05}. This notation overloads to vectors of
names: $\vec{x}^{\pi} := (x_{i}^{\pi} \; : \; 0 \leq i < |\vec{x}| )$ where $\pi \in \{L,R\}$.

We also use $P^{\Box} := P|\Box$.

In \cite{MeredithR05} an interpretation of the new operator is
given. It turns out that there are several possible interpretations
all enjoying the requisite algebraic properties of the operator (see
\cite{milner91polyadicpi}). We will therefore make liberal use of
$(\nu\; \vec{x})P$.

% subsection the_syntax_and_semantics_of_the_notation_system (end)   

\input{qm2pi.qmops} 

\input{qm2pi.sterngerlach} 

\input{qm2pi.metric} 

% section concurrent_process_calculi (end)

%\input{qm2pi.proofsketch}

% section proof sketch (end)

%\input{qm2pi.slviaknots} 

% section spatial logic via knots (end)

\input{qm2pi.conclusion}

% section conclusion (end)

%\input{qm2pi.dtcodes} 

% section wiring algorithm (end)

\input{qm2pi.ack} 

% section acknowledgments (end)

\newpage


\bibliographystyle{plain}   
\bibliography{../../biblios/main.bib}

\input{qm2pi.rhodetails}

\end{document}

 

%\documentclass[12pt]{llncs}
%\documentclass{jktr}

\usepackage[pdftex]{hyperref}                   
\usepackage {listings}
\usepackage {mathpartir}
\usepackage{bcprules}
%\usepackage{listings}
                       
\usepackage{graphicx} 
%\usepackage[margins=2.5cm,nohead,nofoot]{geometry}
%\usepackage{geometry}
\usepackage{amsfonts}
\usepackage{amstext}
\usepackage{latexsym}
\usepackage{amssymb}
\usepackage{color}


%\include{myPreamble}
\include{qm2pi.local} 

%\ifpdf
%\usepackage[pdftex]{graphicx}
%\else
%\usepackage{graphicx}
%\fi

 % \ifpdf
%  \usepackage{pdfsync}
%  \if


%\title{Brief Article}
%\author{David F. Snyder}
%\author{L.G. Meredith}

%\address{Dept. of Math., Texas State University--San Marcos, San Marcos, TX 78666}
       
\pagestyle{empty}


\begin{document}

\lstset{language=[Objective]Caml,frame=shadowbox}

\input{qm2pi.front}

% section front matter (end)

\input{qm2pi.intro} 
 
% section introduction (end)

% \input{qm2pi.knotations} 

% section notation (end)

\input{qm2pi.process.calculi} 

% section concurrent_process_calculi_and_spatial_logics_ (end)
    
%\input{qm2pi.knots2pi} 

%\input{qm2pi.trefoil} 

%\input{qm2pi.mainthm} 

% subsection basic_interpretation (end)

%\input{qm2pi.rho.presentation} 
\subsection{The syntax and semantics of the notation system}\label{sub:the_syntax_and_semantics_of_the_notation_system} % (fold)

We now summarize a technical presentation of the calculus that
embodies our theory of dynamics. The typical presentation of such a
calculus follows the style of giving generators and relations on
them. The grammar, below, describing term constructors, freely
generates the set of processes, $\Proc$. This set is then quotiented
by a relation known as structural congruence and it is over this set
that the notion of dynamics is expressed. This presentation is
essentially that of \cite{MeredithR05} with the addition of
polyadicity and summation. For readability we have relegated some of
the technical subtleties to an appendix.

\subsubsection{Process grammar}\label{subsub:process_grammar}

\begin{mathpar}
  \inferrule* [lab=synchronization] {} {{M} \bc \pzero \;|\; x?F \;|\; x!C }
  \and
  \inferrule* [lab=abstraction] {} {{F} \bc (x)P}
  \and
  \inferrule* [lab=concretion] {} {{C} \bc \langle Q \rangle}
  \and
  \inferrule* [lab=process] {} {{P,Q} \bc M \;| \;P|Q \;|\; @{x}}
  \and
  \inferrule* [lab=name] {} {{x} \bc \quotep{P}}
\end{mathpar} 

Note that $\vec{x}$ (resp. $\vec{P}$) denotes a vector of names
(resp. processes) of length $|\vec{x}|$ (resp. $|\vec{P}|$). We adopt
the following useful abbreviations.

\begin{mathpar}
   x?(\vec{y}).P := x.(\vec{y})P \and  x\clift{\vec{P}} := x.\clift{\vec{P}}
   \and x!(y) := \lift{x}{\dropn{y}}
   \and \Pi_{i=0}^{n-1}P_i := P_0 | \ldots | P_{n-1}
\end{mathpar}

\subsubsection{Structural congruence}

\paragraph{Free and bound names and alpha-equivalence.} At the
core of structural equivalence is alpha-equivalence which identifies
process that are the same up to a change of variable. Formally, we
recognize the distinction between free and bound names. The free names
of a process, $\freenames{P}$, may be calculated recursively as
follows:

\begin{mathpar}
\freenames{\pzero} := \emptyset
  \and \\
  \freenames{x?(y).P} := \{ x \} \cup (\freenames{P} \setminus \{ y \})
  \and 
  \freenames{x!\langle P \rangle} := \{ x \} \cup \{ P \} 
  \and \\
  \freenames{P|Q} := \freenames{P} \cup \freenames{Q}
  \and \\
  \freenames{@{x}} := \{ x \}
\end{mathpar}

$\pi$
$\quotep{\pi}$

$\freenames{-} : \pi \to \mathcal{P}(\quotep{\pi})$

\begin{eqnarray*}
  \freenames{\pzero} & := & \emptyset \\
  \freenames{x?(y).P} & := & \{ x \} \cup (\freenames{P} \setminus \{ y \}) \\
  \freenames{x!\langle P \rangle} & := & \{ x \} \cup \{ P \} \\
  \freenames{P|Q} & := & \freenames{P} \cup \freenames{Q} \\
  \freenames{\dropn{x}} & := & \{ x \}
\end{eqnarray*}

The bound names of a process, $\boundnames{P}$, are those names occurring in $P$
that are not free. For example, in $x?(y).0$, the name $x$ is free, while $y$ is bound.

\begin{mathpar}
  \inferrule* [lab=monoidal-laws] {} { P|Q \equiv Q|P \and P|0 \equiv P \and P|(Q|R) \equiv (P|Q)|R }
\end{mathpar}

\begin{mathpar}
  \inferrule* [lab=alpha-equivalence] {} { (x)P \equiv (y)P\{y/x\} \and y \not\in \freenames{P} }
\end{mathpar}

\begin{definition}
Then two processes, $P,Q$, are alpha-equivalent if $P = Q\{\vec{y}/\vec{x}\}$ for
some $\vec{x} \in \boundnames{Q},\vec{y} \in \boundnames{P}$, where $Q\{\vec{y}/\vec{x}\}$
denotes the capture-avoiding substitution of $\vec{y}$ for $\vec{x}$ in $Q$.
\end{definition}

\begin{definition}
  The {\em structural congruence} \cite{SangiorgiWalker} , $\equiv$,
  between processes is the least congruence containing
  alpha-equivalence, satisfying the abelian monoid laws
  (associativity, commutativity and $\pzero$ as identity) for parallel
  composition $|$ and for summation $+$.
\end{definition}

\subsection{Name equivalence}

We take name equivalence, written $\nameeq$, to be the smallest
equivalence relation generated by the following rules.

\begin{mathpar}
\inferrule*[lab=Quote-drop]
{ }
{ \quotep{@{x}} \nameeq x }

\inferrule*[lab=Struct-equiv]
{ P \scong Q }
{ \quotep{P} \nameeq \quotep{Q} }
\end{mathpar}

The astute reader will have noticed that the mutual recursion of names
and processes imposes a mutual recursion on alpha-equivalence and
structural equivalence via name-equivalence. Fortunately, all of this
works out pleasantly and we may calculate in the natural way, free of
concern. The reader interested in the details is referred to the
appendix \ref{appendix:rho_details}.

\subsection{Substitution}

We use $\Proc$ for the set of processes, $\QProc$ for the set of
names, and $\id{\{}\vec{y} / \vec{x} \id{\}}$ to denote partial maps,
$s : \QProc \rightarrow \QProc$. A map, $s$ lifts, uniquely, to a map
on process terms, $\widehat{s} : \Proc \rightarrow \Proc$ by the
following equations.

\begin{mathpar}
  (0) \psubstp{Q}{P} := 0 \\
  (R \juxtap S) \psubstp{Q}{P}
  :=    
  (R)\psubstp{Q}{P} \juxtap (S) \psubstp{Q}{P} \\
  (x?(y).R) \psubstp{Q}{P}    
  :=    
  (x)\substp{Q}{P} (z)\concat( (R \psubstn{z}{y}) \psubstp{Q}{P} ) \\
  (\lift{x}{R}) \psubstp{Q}{P}  
  :=
  \lift{(x)\substp{Q}{P}}{ R \psubstp{Q}{P} } \\
%   (\dropn{x})  \psubstp{Q}{P}       
%   := 
%   \left\{ 
%     \begin{array}{ccc} 
%       \dropn{\quotep{Q}} & & x \nameeq \quotep{P} \\
%       \dropn{x} & & otherwise \\
%     \end{array}
%   \right. 
  (\dropn{x})  \psubstp{Q}{P}       
  := 
  \left\{ 
    \begin{array}{ccc} 
      Q & & x \nameeq \quotep{P} \\
      \dropn{x} & & otherwise \\
    \end{array}
  \right.
\end{mathpar}
 

where

\begin{eqnarray}
  (x)\id{\{} \lpquote Q \rpquote / \lpquote P \rpquote \id{\}}            = 
  \left\{ 
    \begin{array}{ccc}
      \lpquote Q \rpquote & & x \nameeq \lpquote P \rpquote \\
      x & & otherwise \\
    \end{array}
  \right. \nonumber
\end{eqnarray}

and $z$ is chosen distinct from $\quotep{P}$, $\quotep{Q}$, the free
names in $Q$, and all the names in $R$. Our $\alpha$-equivalence will
be built in the standard way from this substitution.

\begin{remark}\label{rem:no_self_referential_names}
  One consequence of these definitions is that $\forall P. \quotep{P}
  \not\in \freenames{P}$.
\end{remark}

\subsection{ Dynamic quote: an example }

Anticipating something of what's to come, consider applying the
substitution, $\widehat{\id{\{}u / z \id{\}}}$, to the following pair
of processes, $\lift{w}{y!(z)}$ and $w[ \lpquote y!(z) \rpquote ]$.

\begin{eqnarray}
	\lift{w}{y!(z)}\widehat{\id{\{}u / z \id{\}}}
		& = &
		\lift{w}{y!(u)} \nonumber\\
	w[ \lpquote y!(z) \rpquote ] \widehat{ \id{\{}u / z \id{\}} }
		& = &
		w[ \lpquote y!(z) \rpquote ] \nonumber
\end{eqnarray}

Because the body of the process between quotes is impervious to
substitution, we get radically different answers. In fact, by
examining the first process in an input context,
e.g. $x?(z).\lift{w}{y!(z)}$, we see that the process under the lift
operator may be shaped by prefixed inputs binding a name inside it. In
this sense, the lift operator will be seen as a way to dynamically
construct processes before reifying them as names.

Finally equipped with these standard features we can present the
dynamics of the calculus.

\subsubsection{Operational semantics} 

Finally, we introduce the computational dynamics. What marks these
algebras as distinct from other more traditionally studied algebraic
structures, e.g. vector spaces or polynomial rings, is the manner in
which dynamics is captured. In traditional structures, dynamics is typically
expressed through morphisms between such structures, as in linear maps
between vector spaces or morphisms between rings. In algebras
associated with the semantics of computation, the dynamics is
expressed as part of the algebraic structure itself, through a
reduction reduction relation typically denoted by $\red$. Below, we
give a recursive presentation of this relation for the calculus used
in the encoding.

$\red \subseteq \pi \times \pi$
$\red : \pi \to \mathcal{P}(\pi)$

\begin{mathpar}
  \inferrule* [lab=Comm] { \textsf{match}( x_{src}, x_{trgt} ) } { x_{trgt}?(y)P \; | \; x_{src}!\langle {Q} \rangle \red P\{\quotep{Q}/y}\} }
  \and \\
  \inferrule* [lab=Par] {{P} \red {P}'} {{{P} | {Q}} \red {{P}' | {Q}}}
  \and
  \inferrule* [lab=Equiv]{{{P} \scong {P}'} \andalso {{P}' \red {Q}'} \andalso {{Q}' \scong {Q}}}{{P} \red {Q}}
\end{mathpar}

\begin{eqnarray*}
  match_{\equiv} (\quotep{P},\quotep{Q}) & := & P \equiv Q \\
  match_{\dagger}(\quotep{P},\quotep{Q}) & := & \forall R. P|Q \red^{*} R => R \red^{*} 0 \\
  match_{K}(\quotep{P},\quotep{Q}) & := & K \mbox{ for some context } K
\end{eqnarray*}

$u?(x)P | u!\langle Q \rangle \red P\{\quotep{Q}/x\}$

%We write $\wred$ for $\red^*$, and $P\red$ if $\exists Q $ such that $ P \red Q$.
We write $P\red$ if $\exists Q $ such that $ P \red Q$ and $P\not\red$, otherwise.

\section{Replication}

As mentioned before, it is known that replication (and hence
recursion) can be implemented in a higher-order process algebra
\cite{SangiorgiWalker}. As our first example of calculation with the
machinery thus far presented we give the construction explicitly in
the {\rhoc}.

\begin{eqnarray}
	D_{x} & := & \prefix{x}{y}{(\binpar{\outputp{x}{y}}{@{y}})} \nonumber\\
	\bangp_{x}{P} & := & \binpar{{x}!\langle{\binpar{D_{x}}{P}}\rangle}{D_{x}} \nonumber
\end{eqnarray}

\begin{eqnarray}
	\bangp_{x}{P} & & \nonumber\\
	=
	& {x}!\langle{(\prefix{x}{y}{(\outputp{x}{y} | @{y})) | P}}\rangle 
	      | \prefix{x}{y}{(\outputp{x}{y} | @{y})} & \nonumber\\
	\red
	& (\outputp{x}{y} | @{y})\substn{\quotep{(\prefix{x}{y}{(@{y} | \outputp{x}{y})) | P}}}{y} & \nonumber\\
	=
	& \outputp{x}{\quotep{(\prefix{x}{y}{(\outputp{x}{y} | @{y})) | P}}}
	  | {(\prefix{x}{y}{(\outputp{x}{y} | @{y})) | P}} & \nonumber\\
	\red
	& \ldots & \nonumber\\
	\red^*
	& P | P | \ldots & \nonumber
\end{eqnarray}

Of course, this encoding, as an implementation, runs away, unfolding
$\bangp{P}$ eagerly. A lazier and more implementable replication
operator, restricted to input-guarded processes, may be obtained as follows.

\begin{eqnarray}
\bangp{\prefix{u}{v}{P}} 
	:= 
	\binpar{\lift{x}{\prefix{u}{v}{(\binpar{D(x)}{P})}}}{D(x)} \nonumber
\end{eqnarray}

\begin{remark}
  Note that the lazier definition still does not deal with summation
  or mixed summation (i.e. sums over input and output). The reader is
  invited to construct definitions of replication that deal with these
  features. 

  Further, the definitions are parameterized in a name, $x$. Can you,
  gentle reader, make a definition that eliminates this parameter and
  guarantees no accidental interaction between the replication
  machinery and the process being replicated -- i.e. no accidental
  sharing of names used by the process to get its work done and the
  name(s) used by the replication to effect copying. This latter
  revision of the definition of replication is crucial to obtaining
  the expected identity $!!P \sim !P$.
\end{remark}

\begin{remark}\label{rem:paradoxical_combinator}
  The reader familiar with the lambda calculus will have noticed the
  similarity between $D$ and the paradoxical combinator.

  [Ed. note: the existence of this seems to suggest we have to be more
  restrictive on the set of processes and names we admit if we are to
  support no-cloning.]
\end{remark}

\subsubsection{Bisimulation}

The computational dynamics gives rise to another kind of equivalence,
the equivalence of computational behavior. As previously mentioned
this is typically captured \emph{via} some form of bisimulation.

% The notion we use in this paper is weak barbed bisimulation
% \cite{milner91polyadicpi}.

The notion we use in this paper is derived from weak barbed
bisimulation \cite{milner91polyadicpi}. 

\begin{definition}
An \emph{observation relation}, $\downarrow_{\mathcal N}$, over a set
of names, $\mathcal N$, is the smallest relation satisfying the rules
below.

\infrule[Out-barb]{y \in {\mathcal N}, \; x \nameeq y}
		  {\outputp{x}{v} \downarrow_{\mathcal N} x}
\infrule[Par-barb]{\mbox{$P\downarrow_{\mathcal N} x$ or $Q\downarrow_{\mathcal N} x$}}
		  {\binpar{P}{Q} \downarrow_{\mathcal N} x}

We write $P \Downarrow_{\mathcal N} x$ if there is $Q$ such that 
$P \wred Q$ and $Q \downarrow_{\mathcal N} x$.
\end{definition}

\begin{definition}
%\label{def.bbisim}
An  ${\mathcal N}$-\emph{barbed bisimulation} over a set of names, ${\mathcal N}$, is a symmetric binary relation 
${\mathcal S}_{\mathcal N}$ between agents such that $P\rel{S}_{\mathcal N}Q$ implies:
\begin{enumerate}
\item If $P \red P'$ then $Q \wred Q'$ and $P'\rel{S}_{\mathcal N} Q'$.
\item If $P\downarrow_{\mathcal N} x$, then $Q\Downarrow_{\mathcal N} x$.
\end{enumerate}
$P$ is ${\mathcal N}$-barbed bisimilar to $Q$, written
$P \wbbisim_{\mathcal N} Q$, if $P \rel{S}_{\mathcal N} Q$ for some ${\mathcal N}$-barbed bisimulation ${\mathcal S}_{\mathcal N}$.
\end{definition}

$\mathcal{R} \subseteq \pi \times \pi$

$P \mathcal{R} Q => \forall P'. P \red P' \Rightarrow \exists Q'. Q \red Q', P' \mathcal{R} Q'$

$P \vdash x \Rightarrow Q \vdash x$

\begin{mathpar}
  \inferrule*[lab=Out-barb]{x \nameeq y}{{y}!\langle{Q}\rangle \vdash x}
  \and
  \inferrule*[lab=Par-barb]{\mbox{$P\vdash x$ or $Q\vdash x$}}{\binpar{P}{Q} \vdash x}
\end{mathpar}

\subsubsection{Contexts}

One of the principle advantages of computational calculi like the
$\pi$-calculus is a well-defined notion of context,
contextual-equivalence and a correlation between
contextual-equivalence and notions of bisimulation. The notion of
context allows the decomposition of a process into (sub-)process and
its syntactic environment, its context. Thus, a context may be
thought of as a process with a ``hole'' (written $\Box$) in it. The
application of a context $M$ to a process $P$, written $M[P]$, is
tantamount to filling the hole in $M$ with $P$. In this paper we do
not need the full weight of this theory, but do make use of the notion
of context in the proof the main theorem. 

\begin{mathpar}
  \inferrule* [lab=summation] {} {{M_{M},M_{N}} \bc \Box \;|\; x.M_{A} \;|\; M_{M}+M_{N}}
  \and
  \inferrule* [lab=agent] {} {{M_{A}} \bc (\vec{x})M_{P} \;| \; \clift{P_0,\ldots,M_{P},\ldots,P_N}}
  \and \\
  \inferrule* [lab=process] {} {{M_{P}} \bc M_{N} \;| \;P|M_{P} }
\end{mathpar} 

\begin{mathpar}
  \inferrule* [lab=sychronization] {} {M_{N} \bc \Box \;|\; x?M_{F} \;|\; x!M_{C}}
  \and
  \inferrule* [lab=abstraction] {} {{M_{F}} \bc (x)M_{P} }
  \and
  \inferrule* [lab=concretion] {} {{M_{C}} \bc \langle M_{P} \rangle }
  \and \\
  \inferrule* [lab=process] {} {{M_{P}} \bc M_{N} \;| \;P|M_{P} }
\end{mathpar}

\begin{definition}[contextual application] Given a context $M$, and
  process $P$, we define the \emph{contextual application}, $M[P] :=
  M\{P/\Box\}$. That is, the contextual application of M to P is the
  substitution of $P$ for $\Box$ in $M$.
\end{definition}

$\meaningof{-} : L \to \mathcal{P}(\pi)$

\begin{mathpar}
  \inferrule* [lab=collection] {} {\meaningof{true} = \pi, \and \meaningof{~E} = \pi \setminus \meaningof{E}, \and \meaningof{E_{1} \& E_{2}} = \meaningof{E_{1}} \cap \meaningof{E_{2}}}
\end{mathpar}

\begin{mathpar}
  \inferrule* [lab=structure] {} {\meaningof{0} = \{ P \in \pi | P \equiv 0 \}, \and \\ \meaningof{E_1 | E_2} = \{ P \in \pi | P \equiv P_{1} | P_{2}, P_{1} \in \meaningof{E_{1}}, P_{2} \in \meaningof{E_2}\} }
\end{mathpar}

\begin{mathpar}
 \inferrule* [lab=behavior] {} {\meaningof{\langle a?b \rangle E} = \{ P \in \pi | P \equiv Q | u?(y)P', \\ \and \\\\ \and \\ \;\;\; u \in \meaningof{a}, \forall z.P'\{z/y\} \in \meaningof{E\{z/b\}}\}, \and \\ \meaningof{a!E} = \{ P \in \pi | P \equiv Q | x!\langle P' \rangle, x \in \meaningof{a} P' \in \meaningof{E}\} }
\end{mathpar}

\begin{mathpar}
 \inferrule* [lab=nominal] {} {\meaningof{\quotep{E}} = \{ \quotep{P} \in \quotep{\pi} | P \in \meaningof{E} \}, \and \meaningof{\quotep{P}} = \{ \quotep{Q} \in \quotep{\pi} | P \equiv Q \} \and \\ \meaningof{@\quotep{E}} = \{ P \in \pi | P \equiv @x, x \in \meaningof{E} \}}
\end{mathpar}

\begin{eqnarray*}
  \\
  \meaningof{-} : TS \to ST
\end{eqnarray*}

\begin{eqnarray*}
  \\
  L : TS \to ST
\end{eqnarray*}

\begin{eqnarray*}
  \\
  P \models E \iff P \in \meaningof{E}
\end{eqnarray*}

\begin{eqnarray*}
  P \approx_{L} Q \iff \forall E \in L. P \models E \iff Q \models E
\end{eqnarray*}

\begin{eqnarray*}
  P \approx_{K} Q
\end{eqnarray*}

\begin{eqnarray*}
  P \approx Q
\end{eqnarray*}

$\approx_{K} = \approx = \approx_{L}$

\subsubsection{Contextual duality}

Note that contexts extend the quotation operation to a family of
operations from processes to names. Given a context, $M$, we can
define a \emph{nominal context}, $\quotep{M}$ by $\quotep{M}[P] :=
\quotep{M[P]}$. To foreshadow what is to come we observe that these
operations enjoy a duality with processes very much like the duality
between vectors and maps from vectors to scalars.

Further, because the calculus is essentially higher-order, we have a
correspondence between contexts and processes. More specifically,
given a name $x$ and a context $M$ we can construct $M^{*}_{x}$ such
that 

\begin{mathpar}
  M^{*}_{x} | \lift{x}{P} \red M[P]
\end{mathpar}

namely,

\begin{mathpar}
  M^{*}_{x} := x?(u).M[\dropn{u}]
\end{mathpar}

The dependence of $M^{*}_{x}$ on a name makes it an abstraction, 

\begin{mathpar}
  M^{*} := (x)x?(u).M[\dropn{u}]
\end{mathpar}

\subsection{Additional notation}

It will sometimes be convenient to denote the process a name
quotes. We already have the notation $x = \quotep{P}$, but it will be
convenient to introduce an alternate notation, $\procn{x}$, when we
want to emphasize the connection to the use of the name. Note that, by
virtue of name equivalence, $\quotep{\procn{x}} \nameeq x$; so, the
notation is consistent with previous definitions.

Further, because names have structure it is possible to effect
substitutions on the basis of that structure. This means we need to
upgrade our notation for substitutions, which we accomplish by
adapting comprehension notation. Thus,

\begin{mathpar}
  P\{ y / x : x \in S \}
\end{mathpar}

is interpreted to mean the process derived from P by replacing (in a
capture-avoiding manner) each occurrence of $x$ in $S$ by $y$. For example,

\begin{mathpar}
  P\{ \quotep{\procn{x}|\procn{x}} / x : x \in \freenames{P} \}
\end{mathpar}

will replace each (occurrence) of a free name $x$ in $P$ by
$\quotep{\procn{x}|\procn{x}}$.

Also, we will avail ourselves of the notation $x^{L}$ and $x^{R}$ to
denote injections of a name into disjoint copies of the name
space. There are numerous ways to accomplish this. One example can be
found in \cite{MeredithR05}. This notation overloads to vectors of
names: $\vec{x}^{\pi} := (x_{i}^{\pi} \; : \; 0 \leq i < |\vec{x}| )$ where $\pi \in \{L,R\}$.

We also use $P^{\Box} := P|\Box$.

In \cite{MeredithR05} an interpretation of the new operator is
given. It turns out that there are several possible interpretations
all enjoying the requisite algebraic properties of the operator (see
\cite{milner91polyadicpi}). We will therefore make liberal use of
$(\nu\; \vec{x})P$.

% subsection the_syntax_and_semantics_of_the_notation_system (end)   

\input{qm2pi.qmops} 

\input{qm2pi.sterngerlach} 

\input{qm2pi.metric} 

% section concurrent_process_calculi (end)

%\input{qm2pi.proofsketch}

% section proof sketch (end)

%\input{qm2pi.slviaknots} 

% section spatial logic via knots (end)

\input{qm2pi.conclusion}

% section conclusion (end)

%\input{qm2pi.dtcodes} 

% section wiring algorithm (end)

\input{qm2pi.ack} 

% section acknowledgments (end)

\newpage


\bibliographystyle{plain}   
\bibliography{../../biblios/main.bib}

\input{qm2pi.rhodetails}

\end{document}

 

%\documentclass[12pt]{llncs}
%\documentclass{jktr}

\usepackage[pdftex]{hyperref}                   
\usepackage {listings}
\usepackage {mathpartir}
\usepackage{bcprules}
%\usepackage{listings}
                       
\usepackage{graphicx} 
%\usepackage[margins=2.5cm,nohead,nofoot]{geometry}
%\usepackage{geometry}
\usepackage{amsfonts}
\usepackage{amstext}
\usepackage{latexsym}
\usepackage{amssymb}
\usepackage{color}


%\include{myPreamble}
\include{qm2pi.local} 

%\ifpdf
%\usepackage[pdftex]{graphicx}
%\else
%\usepackage{graphicx}
%\fi

 % \ifpdf
%  \usepackage{pdfsync}
%  \if


%\title{Brief Article}
%\author{David F. Snyder}
%\author{L.G. Meredith}

%\address{Dept. of Math., Texas State University--San Marcos, San Marcos, TX 78666}
       
\pagestyle{empty}


\begin{document}

\lstset{language=[Objective]Caml,frame=shadowbox}

\input{qm2pi.front}

% section front matter (end)

\input{qm2pi.intro} 
 
% section introduction (end)

% \input{qm2pi.knotations} 

% section notation (end)

\input{qm2pi.process.calculi} 

% section concurrent_process_calculi_and_spatial_logics_ (end)
    
%\input{qm2pi.knots2pi} 

%\input{qm2pi.trefoil} 

%\input{qm2pi.mainthm} 

% subsection basic_interpretation (end)

%\input{qm2pi.rho.presentation} 
\subsection{The syntax and semantics of the notation system}\label{sub:the_syntax_and_semantics_of_the_notation_system} % (fold)

We now summarize a technical presentation of the calculus that
embodies our theory of dynamics. The typical presentation of such a
calculus follows the style of giving generators and relations on
them. The grammar, below, describing term constructors, freely
generates the set of processes, $\Proc$. This set is then quotiented
by a relation known as structural congruence and it is over this set
that the notion of dynamics is expressed. This presentation is
essentially that of \cite{MeredithR05} with the addition of
polyadicity and summation. For readability we have relegated some of
the technical subtleties to an appendix.

\subsubsection{Process grammar}\label{subsub:process_grammar}

\begin{mathpar}
  \inferrule* [lab=synchronization] {} {{M} \bc \pzero \;|\; x?F \;|\; x!C }
  \and
  \inferrule* [lab=abstraction] {} {{F} \bc (x)P}
  \and
  \inferrule* [lab=concretion] {} {{C} \bc \langle Q \rangle}
  \and
  \inferrule* [lab=process] {} {{P,Q} \bc M \;| \;P|Q \;|\; @{x}}
  \and
  \inferrule* [lab=name] {} {{x} \bc \quotep{P}}
\end{mathpar} 

Note that $\vec{x}$ (resp. $\vec{P}$) denotes a vector of names
(resp. processes) of length $|\vec{x}|$ (resp. $|\vec{P}|$). We adopt
the following useful abbreviations.

\begin{mathpar}
   x?(\vec{y}).P := x.(\vec{y})P \and  x\clift{\vec{P}} := x.\clift{\vec{P}}
   \and x!(y) := \lift{x}{\dropn{y}}
   \and \Pi_{i=0}^{n-1}P_i := P_0 | \ldots | P_{n-1}
\end{mathpar}

\subsubsection{Structural congruence}

\paragraph{Free and bound names and alpha-equivalence.} At the
core of structural equivalence is alpha-equivalence which identifies
process that are the same up to a change of variable. Formally, we
recognize the distinction between free and bound names. The free names
of a process, $\freenames{P}$, may be calculated recursively as
follows:

\begin{mathpar}
\freenames{\pzero} := \emptyset
  \and \\
  \freenames{x?(y).P} := \{ x \} \cup (\freenames{P} \setminus \{ y \})
  \and 
  \freenames{x!\langle P \rangle} := \{ x \} \cup \{ P \} 
  \and \\
  \freenames{P|Q} := \freenames{P} \cup \freenames{Q}
  \and \\
  \freenames{@{x}} := \{ x \}
\end{mathpar}

$\pi$
$\quotep{\pi}$

$\freenames{-} : \pi \to \mathcal{P}(\quotep{\pi})$

\begin{eqnarray*}
  \freenames{\pzero} & := & \emptyset \\
  \freenames{x?(y).P} & := & \{ x \} \cup (\freenames{P} \setminus \{ y \}) \\
  \freenames{x!\langle P \rangle} & := & \{ x \} \cup \{ P \} \\
  \freenames{P|Q} & := & \freenames{P} \cup \freenames{Q} \\
  \freenames{\dropn{x}} & := & \{ x \}
\end{eqnarray*}

The bound names of a process, $\boundnames{P}$, are those names occurring in $P$
that are not free. For example, in $x?(y).0$, the name $x$ is free, while $y$ is bound.

\begin{mathpar}
  \inferrule* [lab=monoidal-laws] {} { P|Q \equiv Q|P \and P|0 \equiv P \and P|(Q|R) \equiv (P|Q)|R }
\end{mathpar}

\begin{mathpar}
  \inferrule* [lab=alpha-equivalence] {} { (x)P \equiv (y)P\{y/x\} \and y \not\in \freenames{P} }
\end{mathpar}

\begin{definition}
Then two processes, $P,Q$, are alpha-equivalent if $P = Q\{\vec{y}/\vec{x}\}$ for
some $\vec{x} \in \boundnames{Q},\vec{y} \in \boundnames{P}$, where $Q\{\vec{y}/\vec{x}\}$
denotes the capture-avoiding substitution of $\vec{y}$ for $\vec{x}$ in $Q$.
\end{definition}

\begin{definition}
  The {\em structural congruence} \cite{SangiorgiWalker} , $\equiv$,
  between processes is the least congruence containing
  alpha-equivalence, satisfying the abelian monoid laws
  (associativity, commutativity and $\pzero$ as identity) for parallel
  composition $|$ and for summation $+$.
\end{definition}

\subsection{Name equivalence}

We take name equivalence, written $\nameeq$, to be the smallest
equivalence relation generated by the following rules.

\begin{mathpar}
\inferrule*[lab=Quote-drop]
{ }
{ \quotep{@{x}} \nameeq x }

\inferrule*[lab=Struct-equiv]
{ P \scong Q }
{ \quotep{P} \nameeq \quotep{Q} }
\end{mathpar}

The astute reader will have noticed that the mutual recursion of names
and processes imposes a mutual recursion on alpha-equivalence and
structural equivalence via name-equivalence. Fortunately, all of this
works out pleasantly and we may calculate in the natural way, free of
concern. The reader interested in the details is referred to the
appendix \ref{appendix:rho_details}.

\subsection{Substitution}

We use $\Proc$ for the set of processes, $\QProc$ for the set of
names, and $\id{\{}\vec{y} / \vec{x} \id{\}}$ to denote partial maps,
$s : \QProc \rightarrow \QProc$. A map, $s$ lifts, uniquely, to a map
on process terms, $\widehat{s} : \Proc \rightarrow \Proc$ by the
following equations.

\begin{mathpar}
  (0) \psubstp{Q}{P} := 0 \\
  (R \juxtap S) \psubstp{Q}{P}
  :=    
  (R)\psubstp{Q}{P} \juxtap (S) \psubstp{Q}{P} \\
  (x?(y).R) \psubstp{Q}{P}    
  :=    
  (x)\substp{Q}{P} (z)\concat( (R \psubstn{z}{y}) \psubstp{Q}{P} ) \\
  (\lift{x}{R}) \psubstp{Q}{P}  
  :=
  \lift{(x)\substp{Q}{P}}{ R \psubstp{Q}{P} } \\
%   (\dropn{x})  \psubstp{Q}{P}       
%   := 
%   \left\{ 
%     \begin{array}{ccc} 
%       \dropn{\quotep{Q}} & & x \nameeq \quotep{P} \\
%       \dropn{x} & & otherwise \\
%     \end{array}
%   \right. 
  (\dropn{x})  \psubstp{Q}{P}       
  := 
  \left\{ 
    \begin{array}{ccc} 
      Q & & x \nameeq \quotep{P} \\
      \dropn{x} & & otherwise \\
    \end{array}
  \right.
\end{mathpar}
 

where

\begin{eqnarray}
  (x)\id{\{} \lpquote Q \rpquote / \lpquote P \rpquote \id{\}}            = 
  \left\{ 
    \begin{array}{ccc}
      \lpquote Q \rpquote & & x \nameeq \lpquote P \rpquote \\
      x & & otherwise \\
    \end{array}
  \right. \nonumber
\end{eqnarray}

and $z$ is chosen distinct from $\quotep{P}$, $\quotep{Q}$, the free
names in $Q$, and all the names in $R$. Our $\alpha$-equivalence will
be built in the standard way from this substitution.

\begin{remark}\label{rem:no_self_referential_names}
  One consequence of these definitions is that $\forall P. \quotep{P}
  \not\in \freenames{P}$.
\end{remark}

\subsection{ Dynamic quote: an example }

Anticipating something of what's to come, consider applying the
substitution, $\widehat{\id{\{}u / z \id{\}}}$, to the following pair
of processes, $\lift{w}{y!(z)}$ and $w[ \lpquote y!(z) \rpquote ]$.

\begin{eqnarray}
	\lift{w}{y!(z)}\widehat{\id{\{}u / z \id{\}}}
		& = &
		\lift{w}{y!(u)} \nonumber\\
	w[ \lpquote y!(z) \rpquote ] \widehat{ \id{\{}u / z \id{\}} }
		& = &
		w[ \lpquote y!(z) \rpquote ] \nonumber
\end{eqnarray}

Because the body of the process between quotes is impervious to
substitution, we get radically different answers. In fact, by
examining the first process in an input context,
e.g. $x?(z).\lift{w}{y!(z)}$, we see that the process under the lift
operator may be shaped by prefixed inputs binding a name inside it. In
this sense, the lift operator will be seen as a way to dynamically
construct processes before reifying them as names.

Finally equipped with these standard features we can present the
dynamics of the calculus.

\subsubsection{Operational semantics} 

Finally, we introduce the computational dynamics. What marks these
algebras as distinct from other more traditionally studied algebraic
structures, e.g. vector spaces or polynomial rings, is the manner in
which dynamics is captured. In traditional structures, dynamics is typically
expressed through morphisms between such structures, as in linear maps
between vector spaces or morphisms between rings. In algebras
associated with the semantics of computation, the dynamics is
expressed as part of the algebraic structure itself, through a
reduction reduction relation typically denoted by $\red$. Below, we
give a recursive presentation of this relation for the calculus used
in the encoding.

$\red \subseteq \pi \times \pi$
$\red : \pi \to \mathcal{P}(\pi)$

\begin{mathpar}
  \inferrule* [lab=Comm] { \textsf{match}( x_{src}, x_{trgt} ) } { x_{trgt}?(y)P \; | \; x_{src}!\langle {Q} \rangle \red P\{\quotep{Q}/y}\} }
  \and \\
  \inferrule* [lab=Par] {{P} \red {P}'} {{{P} | {Q}} \red {{P}' | {Q}}}
  \and
  \inferrule* [lab=Equiv]{{{P} \scong {P}'} \andalso {{P}' \red {Q}'} \andalso {{Q}' \scong {Q}}}{{P} \red {Q}}
\end{mathpar}

\begin{eqnarray*}
  match_{\equiv} (\quotep{P},\quotep{Q}) & := & P \equiv Q \\
  match_{\dagger}(\quotep{P},\quotep{Q}) & := & \forall R. P|Q \red^{*} R => R \red^{*} 0 \\
  match_{K}(\quotep{P},\quotep{Q}) & := & K \mbox{ for some context } K
\end{eqnarray*}

$u?(x)P | u!\langle Q \rangle \red P\{\quotep{Q}/x\}$

%We write $\wred$ for $\red^*$, and $P\red$ if $\exists Q $ such that $ P \red Q$.
We write $P\red$ if $\exists Q $ such that $ P \red Q$ and $P\not\red$, otherwise.

\section{Replication}

As mentioned before, it is known that replication (and hence
recursion) can be implemented in a higher-order process algebra
\cite{SangiorgiWalker}. As our first example of calculation with the
machinery thus far presented we give the construction explicitly in
the {\rhoc}.

\begin{eqnarray}
	D_{x} & := & \prefix{x}{y}{(\binpar{\outputp{x}{y}}{@{y}})} \nonumber\\
	\bangp_{x}{P} & := & \binpar{{x}!\langle{\binpar{D_{x}}{P}}\rangle}{D_{x}} \nonumber
\end{eqnarray}

\begin{eqnarray}
	\bangp_{x}{P} & & \nonumber\\
	=
	& {x}!\langle{(\prefix{x}{y}{(\outputp{x}{y} | @{y})) | P}}\rangle 
	      | \prefix{x}{y}{(\outputp{x}{y} | @{y})} & \nonumber\\
	\red
	& (\outputp{x}{y} | @{y})\substn{\quotep{(\prefix{x}{y}{(@{y} | \outputp{x}{y})) | P}}}{y} & \nonumber\\
	=
	& \outputp{x}{\quotep{(\prefix{x}{y}{(\outputp{x}{y} | @{y})) | P}}}
	  | {(\prefix{x}{y}{(\outputp{x}{y} | @{y})) | P}} & \nonumber\\
	\red
	& \ldots & \nonumber\\
	\red^*
	& P | P | \ldots & \nonumber
\end{eqnarray}

Of course, this encoding, as an implementation, runs away, unfolding
$\bangp{P}$ eagerly. A lazier and more implementable replication
operator, restricted to input-guarded processes, may be obtained as follows.

\begin{eqnarray}
\bangp{\prefix{u}{v}{P}} 
	:= 
	\binpar{\lift{x}{\prefix{u}{v}{(\binpar{D(x)}{P})}}}{D(x)} \nonumber
\end{eqnarray}

\begin{remark}
  Note that the lazier definition still does not deal with summation
  or mixed summation (i.e. sums over input and output). The reader is
  invited to construct definitions of replication that deal with these
  features. 

  Further, the definitions are parameterized in a name, $x$. Can you,
  gentle reader, make a definition that eliminates this parameter and
  guarantees no accidental interaction between the replication
  machinery and the process being replicated -- i.e. no accidental
  sharing of names used by the process to get its work done and the
  name(s) used by the replication to effect copying. This latter
  revision of the definition of replication is crucial to obtaining
  the expected identity $!!P \sim !P$.
\end{remark}

\begin{remark}\label{rem:paradoxical_combinator}
  The reader familiar with the lambda calculus will have noticed the
  similarity between $D$ and the paradoxical combinator.

  [Ed. note: the existence of this seems to suggest we have to be more
  restrictive on the set of processes and names we admit if we are to
  support no-cloning.]
\end{remark}

\subsubsection{Bisimulation}

The computational dynamics gives rise to another kind of equivalence,
the equivalence of computational behavior. As previously mentioned
this is typically captured \emph{via} some form of bisimulation.

% The notion we use in this paper is weak barbed bisimulation
% \cite{milner91polyadicpi}.

The notion we use in this paper is derived from weak barbed
bisimulation \cite{milner91polyadicpi}. 

\begin{definition}
An \emph{observation relation}, $\downarrow_{\mathcal N}$, over a set
of names, $\mathcal N$, is the smallest relation satisfying the rules
below.

\infrule[Out-barb]{y \in {\mathcal N}, \; x \nameeq y}
		  {\outputp{x}{v} \downarrow_{\mathcal N} x}
\infrule[Par-barb]{\mbox{$P\downarrow_{\mathcal N} x$ or $Q\downarrow_{\mathcal N} x$}}
		  {\binpar{P}{Q} \downarrow_{\mathcal N} x}

We write $P \Downarrow_{\mathcal N} x$ if there is $Q$ such that 
$P \wred Q$ and $Q \downarrow_{\mathcal N} x$.
\end{definition}

\begin{definition}
%\label{def.bbisim}
An  ${\mathcal N}$-\emph{barbed bisimulation} over a set of names, ${\mathcal N}$, is a symmetric binary relation 
${\mathcal S}_{\mathcal N}$ between agents such that $P\rel{S}_{\mathcal N}Q$ implies:
\begin{enumerate}
\item If $P \red P'$ then $Q \wred Q'$ and $P'\rel{S}_{\mathcal N} Q'$.
\item If $P\downarrow_{\mathcal N} x$, then $Q\Downarrow_{\mathcal N} x$.
\end{enumerate}
$P$ is ${\mathcal N}$-barbed bisimilar to $Q$, written
$P \wbbisim_{\mathcal N} Q$, if $P \rel{S}_{\mathcal N} Q$ for some ${\mathcal N}$-barbed bisimulation ${\mathcal S}_{\mathcal N}$.
\end{definition}

$\mathcal{R} \subseteq \pi \times \pi$

$P \mathcal{R} Q => \forall P'. P \red P' \Rightarrow \exists Q'. Q \red Q', P' \mathcal{R} Q'$

$P \vdash x \Rightarrow Q \vdash x$

\begin{mathpar}
  \inferrule*[lab=Out-barb]{x \nameeq y}{{y}!\langle{Q}\rangle \vdash x}
  \and
  \inferrule*[lab=Par-barb]{\mbox{$P\vdash x$ or $Q\vdash x$}}{\binpar{P}{Q} \vdash x}
\end{mathpar}

\subsubsection{Contexts}

One of the principle advantages of computational calculi like the
$\pi$-calculus is a well-defined notion of context,
contextual-equivalence and a correlation between
contextual-equivalence and notions of bisimulation. The notion of
context allows the decomposition of a process into (sub-)process and
its syntactic environment, its context. Thus, a context may be
thought of as a process with a ``hole'' (written $\Box$) in it. The
application of a context $M$ to a process $P$, written $M[P]$, is
tantamount to filling the hole in $M$ with $P$. In this paper we do
not need the full weight of this theory, but do make use of the notion
of context in the proof the main theorem. 

\begin{mathpar}
  \inferrule* [lab=summation] {} {{M_{M},M_{N}} \bc \Box \;|\; x.M_{A} \;|\; M_{M}+M_{N}}
  \and
  \inferrule* [lab=agent] {} {{M_{A}} \bc (\vec{x})M_{P} \;| \; \clift{P_0,\ldots,M_{P},\ldots,P_N}}
  \and \\
  \inferrule* [lab=process] {} {{M_{P}} \bc M_{N} \;| \;P|M_{P} }
\end{mathpar} 

\begin{mathpar}
  \inferrule* [lab=sychronization] {} {M_{N} \bc \Box \;|\; x?M_{F} \;|\; x!M_{C}}
  \and
  \inferrule* [lab=abstraction] {} {{M_{F}} \bc (x)M_{P} }
  \and
  \inferrule* [lab=concretion] {} {{M_{C}} \bc \langle M_{P} \rangle }
  \and \\
  \inferrule* [lab=process] {} {{M_{P}} \bc M_{N} \;| \;P|M_{P} }
\end{mathpar}

\begin{definition}[contextual application] Given a context $M$, and
  process $P$, we define the \emph{contextual application}, $M[P] :=
  M\{P/\Box\}$. That is, the contextual application of M to P is the
  substitution of $P$ for $\Box$ in $M$.
\end{definition}

$\meaningof{-} : L \to \mathcal{P}(\pi)$

\begin{mathpar}
  \inferrule* [lab=collection] {} {\meaningof{true} = \pi, \and \meaningof{~E} = \pi \setminus \meaningof{E}, \and \meaningof{E_{1} \& E_{2}} = \meaningof{E_{1}} \cap \meaningof{E_{2}}}
\end{mathpar}

\begin{mathpar}
  \inferrule* [lab=structure] {} {\meaningof{0} = \{ P \in \pi | P \equiv 0 \}, \and \\ \meaningof{E_1 | E_2} = \{ P \in \pi | P \equiv P_{1} | P_{2}, P_{1} \in \meaningof{E_{1}}, P_{2} \in \meaningof{E_2}\} }
\end{mathpar}

\begin{mathpar}
 \inferrule* [lab=behavior] {} {\meaningof{\langle a?b \rangle E} = \{ P \in \pi | P \equiv Q | u?(y)P', \\ \and \\\\ \and \\ \;\;\; u \in \meaningof{a}, \forall z.P'\{z/y\} \in \meaningof{E\{z/b\}}\}, \and \\ \meaningof{a!E} = \{ P \in \pi | P \equiv Q | x!\langle P' \rangle, x \in \meaningof{a} P' \in \meaningof{E}\} }
\end{mathpar}

\begin{mathpar}
 \inferrule* [lab=nominal] {} {\meaningof{\quotep{E}} = \{ \quotep{P} \in \quotep{\pi} | P \in \meaningof{E} \}, \and \meaningof{\quotep{P}} = \{ \quotep{Q} \in \quotep{\pi} | P \equiv Q \} \and \\ \meaningof{@\quotep{E}} = \{ P \in \pi | P \equiv @x, x \in \meaningof{E} \}}
\end{mathpar}

\begin{eqnarray*}
  \\
  \meaningof{-} : TS \to ST
\end{eqnarray*}

\begin{eqnarray*}
  \\
  L : TS \to ST
\end{eqnarray*}

\begin{eqnarray*}
  \\
  P \models E \iff P \in \meaningof{E}
\end{eqnarray*}

\begin{eqnarray*}
  P \approx_{L} Q \iff \forall E \in L. P \models E \iff Q \models E
\end{eqnarray*}

\begin{eqnarray*}
  P \approx_{K} Q
\end{eqnarray*}

\begin{eqnarray*}
  P \approx Q
\end{eqnarray*}

$\approx_{K} = \approx = \approx_{L}$

\subsubsection{Contextual duality}

Note that contexts extend the quotation operation to a family of
operations from processes to names. Given a context, $M$, we can
define a \emph{nominal context}, $\quotep{M}$ by $\quotep{M}[P] :=
\quotep{M[P]}$. To foreshadow what is to come we observe that these
operations enjoy a duality with processes very much like the duality
between vectors and maps from vectors to scalars.

Further, because the calculus is essentially higher-order, we have a
correspondence between contexts and processes. More specifically,
given a name $x$ and a context $M$ we can construct $M^{*}_{x}$ such
that 

\begin{mathpar}
  M^{*}_{x} | \lift{x}{P} \red M[P]
\end{mathpar}

namely,

\begin{mathpar}
  M^{*}_{x} := x?(u).M[\dropn{u}]
\end{mathpar}

The dependence of $M^{*}_{x}$ on a name makes it an abstraction, 

\begin{mathpar}
  M^{*} := (x)x?(u).M[\dropn{u}]
\end{mathpar}

\subsection{Additional notation}

It will sometimes be convenient to denote the process a name
quotes. We already have the notation $x = \quotep{P}$, but it will be
convenient to introduce an alternate notation, $\procn{x}$, when we
want to emphasize the connection to the use of the name. Note that, by
virtue of name equivalence, $\quotep{\procn{x}} \nameeq x$; so, the
notation is consistent with previous definitions.

Further, because names have structure it is possible to effect
substitutions on the basis of that structure. This means we need to
upgrade our notation for substitutions, which we accomplish by
adapting comprehension notation. Thus,

\begin{mathpar}
  P\{ y / x : x \in S \}
\end{mathpar}

is interpreted to mean the process derived from P by replacing (in a
capture-avoiding manner) each occurrence of $x$ in $S$ by $y$. For example,

\begin{mathpar}
  P\{ \quotep{\procn{x}|\procn{x}} / x : x \in \freenames{P} \}
\end{mathpar}

will replace each (occurrence) of a free name $x$ in $P$ by
$\quotep{\procn{x}|\procn{x}}$.

Also, we will avail ourselves of the notation $x^{L}$ and $x^{R}$ to
denote injections of a name into disjoint copies of the name
space. There are numerous ways to accomplish this. One example can be
found in \cite{MeredithR05}. This notation overloads to vectors of
names: $\vec{x}^{\pi} := (x_{i}^{\pi} \; : \; 0 \leq i < |\vec{x}| )$ where $\pi \in \{L,R\}$.

We also use $P^{\Box} := P|\Box$.

In \cite{MeredithR05} an interpretation of the new operator is
given. It turns out that there are several possible interpretations
all enjoying the requisite algebraic properties of the operator (see
\cite{milner91polyadicpi}). We will therefore make liberal use of
$(\nu\; \vec{x})P$.

% subsection the_syntax_and_semantics_of_the_notation_system (end)   

\input{qm2pi.qmops} 

\input{qm2pi.sterngerlach} 

\input{qm2pi.metric} 

% section concurrent_process_calculi (end)

%\input{qm2pi.proofsketch}

% section proof sketch (end)

%\input{qm2pi.slviaknots} 

% section spatial logic via knots (end)

\input{qm2pi.conclusion}

% section conclusion (end)

%\input{qm2pi.dtcodes} 

% section wiring algorithm (end)

\input{qm2pi.ack} 

% section acknowledgments (end)

\newpage


\bibliographystyle{plain}   
\bibliography{../../biblios/main.bib}

\input{qm2pi.rhodetails}

\end{document}

 

% subsection basic_interpretation (end)

%\input{qm2pi.rho.presentation} 
\subsection{The syntax and semantics of the notation system}\label{sub:the_syntax_and_semantics_of_the_notation_system} % (fold)

We now summarize a technical presentation of the calculus that
embodies our theory of dynamics. The typical presentation of such a
calculus follows the style of giving generators and relations on
them. The grammar, below, describing term constructors, freely
generates the set of processes, $\Proc$. This set is then quotiented
by a relation known as structural congruence and it is over this set
that the notion of dynamics is expressed. This presentation is
essentially that of \cite{MeredithR05} with the addition of
polyadicity and summation. For readability we have relegated some of
the technical subtleties to an appendix.

\subsubsection{Process grammar}\label{subsub:process_grammar}

\begin{mathpar}
  \inferrule* [lab=synchronization] {} {{M} \bc \pzero \;|\; x?F \;|\; x!C }
  \and
  \inferrule* [lab=abstraction] {} {{F} \bc (x)P}
  \and
  \inferrule* [lab=concretion] {} {{C} \bc \langle Q \rangle}
  \and
  \inferrule* [lab=process] {} {{P,Q} \bc M \;| \;P|Q \;|\; @{x}}
  \and
  \inferrule* [lab=name] {} {{x} \bc \quotep{P}}
\end{mathpar} 

Note that $\vec{x}$ (resp. $\vec{P}$) denotes a vector of names
(resp. processes) of length $|\vec{x}|$ (resp. $|\vec{P}|$). We adopt
the following useful abbreviations.

\begin{mathpar}
   x?(\vec{y}).P := x.(\vec{y})P \and  x\clift{\vec{P}} := x.\clift{\vec{P}}
   \and x!(y) := \lift{x}{\dropn{y}}
   \and \Pi_{i=0}^{n-1}P_i := P_0 | \ldots | P_{n-1}
\end{mathpar}

\subsubsection{Structural congruence}

\paragraph{Free and bound names and alpha-equivalence.} At the
core of structural equivalence is alpha-equivalence which identifies
process that are the same up to a change of variable. Formally, we
recognize the distinction between free and bound names. The free names
of a process, $\freenames{P}$, may be calculated recursively as
follows:

\begin{mathpar}
\freenames{\pzero} := \emptyset
  \and \\
  \freenames{x?(y).P} := \{ x \} \cup (\freenames{P} \setminus \{ y \})
  \and 
  \freenames{x!\langle P \rangle} := \{ x \} \cup \{ P \} 
  \and \\
  \freenames{P|Q} := \freenames{P} \cup \freenames{Q}
  \and \\
  \freenames{@{x}} := \{ x \}
\end{mathpar}

$\pi$
$\quotep{\pi}$

$\freenames{-} : \pi \to \mathcal{P}(\quotep{\pi})$

\begin{eqnarray*}
  \freenames{\pzero} & := & \emptyset \\
  \freenames{x?(y).P} & := & \{ x \} \cup (\freenames{P} \setminus \{ y \}) \\
  \freenames{x!\langle P \rangle} & := & \{ x \} \cup \{ P \} \\
  \freenames{P|Q} & := & \freenames{P} \cup \freenames{Q} \\
  \freenames{\dropn{x}} & := & \{ x \}
\end{eqnarray*}

The bound names of a process, $\boundnames{P}$, are those names occurring in $P$
that are not free. For example, in $x?(y).0$, the name $x$ is free, while $y$ is bound.

\begin{mathpar}
  \inferrule* [lab=monoidal-laws] {} { P|Q \equiv Q|P \and P|0 \equiv P \and P|(Q|R) \equiv (P|Q)|R }
\end{mathpar}

\begin{mathpar}
  \inferrule* [lab=alpha-equivalence] {} { (x)P \equiv (y)P\{y/x\} \and y \not\in \freenames{P} }
\end{mathpar}

\begin{definition}
Then two processes, $P,Q$, are alpha-equivalent if $P = Q\{\vec{y}/\vec{x}\}$ for
some $\vec{x} \in \boundnames{Q},\vec{y} \in \boundnames{P}$, where $Q\{\vec{y}/\vec{x}\}$
denotes the capture-avoiding substitution of $\vec{y}$ for $\vec{x}$ in $Q$.
\end{definition}

\begin{definition}
  The {\em structural congruence} \cite{SangiorgiWalker} , $\equiv$,
  between processes is the least congruence containing
  alpha-equivalence, satisfying the abelian monoid laws
  (associativity, commutativity and $\pzero$ as identity) for parallel
  composition $|$ and for summation $+$.
\end{definition}

\subsection{Name equivalence}

We take name equivalence, written $\nameeq$, to be the smallest
equivalence relation generated by the following rules.

\begin{mathpar}
\inferrule*[lab=Quote-drop]
{ }
{ \quotep{@{x}} \nameeq x }

\inferrule*[lab=Struct-equiv]
{ P \scong Q }
{ \quotep{P} \nameeq \quotep{Q} }
\end{mathpar}

The astute reader will have noticed that the mutual recursion of names
and processes imposes a mutual recursion on alpha-equivalence and
structural equivalence via name-equivalence. Fortunately, all of this
works out pleasantly and we may calculate in the natural way, free of
concern. The reader interested in the details is referred to the
appendix \ref{appendix:rho_details}.

\subsection{Substitution}

We use $\Proc$ for the set of processes, $\QProc$ for the set of
names, and $\id{\{}\vec{y} / \vec{x} \id{\}}$ to denote partial maps,
$s : \QProc \rightarrow \QProc$. A map, $s$ lifts, uniquely, to a map
on process terms, $\widehat{s} : \Proc \rightarrow \Proc$ by the
following equations.

\begin{mathpar}
  (0) \psubstp{Q}{P} := 0 \\
  (R \juxtap S) \psubstp{Q}{P}
  :=    
  (R)\psubstp{Q}{P} \juxtap (S) \psubstp{Q}{P} \\
  (x?(y).R) \psubstp{Q}{P}    
  :=    
  (x)\substp{Q}{P} (z)\concat( (R \psubstn{z}{y}) \psubstp{Q}{P} ) \\
  (\lift{x}{R}) \psubstp{Q}{P}  
  :=
  \lift{(x)\substp{Q}{P}}{ R \psubstp{Q}{P} } \\
%   (\dropn{x})  \psubstp{Q}{P}       
%   := 
%   \left\{ 
%     \begin{array}{ccc} 
%       \dropn{\quotep{Q}} & & x \nameeq \quotep{P} \\
%       \dropn{x} & & otherwise \\
%     \end{array}
%   \right. 
  (\dropn{x})  \psubstp{Q}{P}       
  := 
  \left\{ 
    \begin{array}{ccc} 
      Q & & x \nameeq \quotep{P} \\
      \dropn{x} & & otherwise \\
    \end{array}
  \right.
\end{mathpar}
 

where

\begin{eqnarray}
  (x)\id{\{} \lpquote Q \rpquote / \lpquote P \rpquote \id{\}}            = 
  \left\{ 
    \begin{array}{ccc}
      \lpquote Q \rpquote & & x \nameeq \lpquote P \rpquote \\
      x & & otherwise \\
    \end{array}
  \right. \nonumber
\end{eqnarray}

and $z$ is chosen distinct from $\quotep{P}$, $\quotep{Q}$, the free
names in $Q$, and all the names in $R$. Our $\alpha$-equivalence will
be built in the standard way from this substitution.

\begin{remark}\label{rem:no_self_referential_names}
  One consequence of these definitions is that $\forall P. \quotep{P}
  \not\in \freenames{P}$.
\end{remark}

\subsection{ Dynamic quote: an example }

Anticipating something of what's to come, consider applying the
substitution, $\widehat{\id{\{}u / z \id{\}}}$, to the following pair
of processes, $\lift{w}{y!(z)}$ and $w[ \lpquote y!(z) \rpquote ]$.

\begin{eqnarray}
	\lift{w}{y!(z)}\widehat{\id{\{}u / z \id{\}}}
		& = &
		\lift{w}{y!(u)} \nonumber\\
	w[ \lpquote y!(z) \rpquote ] \widehat{ \id{\{}u / z \id{\}} }
		& = &
		w[ \lpquote y!(z) \rpquote ] \nonumber
\end{eqnarray}

Because the body of the process between quotes is impervious to
substitution, we get radically different answers. In fact, by
examining the first process in an input context,
e.g. $x?(z).\lift{w}{y!(z)}$, we see that the process under the lift
operator may be shaped by prefixed inputs binding a name inside it. In
this sense, the lift operator will be seen as a way to dynamically
construct processes before reifying them as names.

Finally equipped with these standard features we can present the
dynamics of the calculus.

\subsubsection{Operational semantics} 

Finally, we introduce the computational dynamics. What marks these
algebras as distinct from other more traditionally studied algebraic
structures, e.g. vector spaces or polynomial rings, is the manner in
which dynamics is captured. In traditional structures, dynamics is typically
expressed through morphisms between such structures, as in linear maps
between vector spaces or morphisms between rings. In algebras
associated with the semantics of computation, the dynamics is
expressed as part of the algebraic structure itself, through a
reduction reduction relation typically denoted by $\red$. Below, we
give a recursive presentation of this relation for the calculus used
in the encoding.

$\red \subseteq \pi \times \pi$
$\red : \pi \to \mathcal{P}(\pi)$

\begin{mathpar}
  \inferrule* [lab=Comm] { \textsf{match}( x_{src}, x_{trgt} ) } { x_{trgt}?(y)P \; | \; x_{src}!\langle {Q} \rangle \red P\{\quotep{Q}/y}\} }
  \and \\
  \inferrule* [lab=Par] {{P} \red {P}'} {{{P} | {Q}} \red {{P}' | {Q}}}
  \and
  \inferrule* [lab=Equiv]{{{P} \scong {P}'} \andalso {{P}' \red {Q}'} \andalso {{Q}' \scong {Q}}}{{P} \red {Q}}
\end{mathpar}

\begin{eqnarray*}
  match_{\equiv} (\quotep{P},\quotep{Q}) & := & P \equiv Q \\
  match_{\dagger}(\quotep{P},\quotep{Q}) & := & \forall R. P|Q \red^{*} R => R \red^{*} 0 \\
  match_{K}(\quotep{P},\quotep{Q}) & := & K \mbox{ for some context } K
\end{eqnarray*}

$u?(x)P | u!\langle Q \rangle \red P\{\quotep{Q}/x\}$

%We write $\wred$ for $\red^*$, and $P\red$ if $\exists Q $ such that $ P \red Q$.
We write $P\red$ if $\exists Q $ such that $ P \red Q$ and $P\not\red$, otherwise.

\section{Replication}

As mentioned before, it is known that replication (and hence
recursion) can be implemented in a higher-order process algebra
\cite{SangiorgiWalker}. As our first example of calculation with the
machinery thus far presented we give the construction explicitly in
the {\rhoc}.

\begin{eqnarray}
	D_{x} & := & \prefix{x}{y}{(\binpar{\outputp{x}{y}}{@{y}})} \nonumber\\
	\bangp_{x}{P} & := & \binpar{{x}!\langle{\binpar{D_{x}}{P}}\rangle}{D_{x}} \nonumber
\end{eqnarray}

\begin{eqnarray}
	\bangp_{x}{P} & & \nonumber\\
	=
	& {x}!\langle{(\prefix{x}{y}{(\outputp{x}{y} | @{y})) | P}}\rangle 
	      | \prefix{x}{y}{(\outputp{x}{y} | @{y})} & \nonumber\\
	\red
	& (\outputp{x}{y} | @{y})\substn{\quotep{(\prefix{x}{y}{(@{y} | \outputp{x}{y})) | P}}}{y} & \nonumber\\
	=
	& \outputp{x}{\quotep{(\prefix{x}{y}{(\outputp{x}{y} | @{y})) | P}}}
	  | {(\prefix{x}{y}{(\outputp{x}{y} | @{y})) | P}} & \nonumber\\
	\red
	& \ldots & \nonumber\\
	\red^*
	& P | P | \ldots & \nonumber
\end{eqnarray}

Of course, this encoding, as an implementation, runs away, unfolding
$\bangp{P}$ eagerly. A lazier and more implementable replication
operator, restricted to input-guarded processes, may be obtained as follows.

\begin{eqnarray}
\bangp{\prefix{u}{v}{P}} 
	:= 
	\binpar{\lift{x}{\prefix{u}{v}{(\binpar{D(x)}{P})}}}{D(x)} \nonumber
\end{eqnarray}

\begin{remark}
  Note that the lazier definition still does not deal with summation
  or mixed summation (i.e. sums over input and output). The reader is
  invited to construct definitions of replication that deal with these
  features. 

  Further, the definitions are parameterized in a name, $x$. Can you,
  gentle reader, make a definition that eliminates this parameter and
  guarantees no accidental interaction between the replication
  machinery and the process being replicated -- i.e. no accidental
  sharing of names used by the process to get its work done and the
  name(s) used by the replication to effect copying. This latter
  revision of the definition of replication is crucial to obtaining
  the expected identity $!!P \sim !P$.
\end{remark}

\begin{remark}\label{rem:paradoxical_combinator}
  The reader familiar with the lambda calculus will have noticed the
  similarity between $D$ and the paradoxical combinator.

  [Ed. note: the existence of this seems to suggest we have to be more
  restrictive on the set of processes and names we admit if we are to
  support no-cloning.]
\end{remark}

\subsubsection{Bisimulation}

The computational dynamics gives rise to another kind of equivalence,
the equivalence of computational behavior. As previously mentioned
this is typically captured \emph{via} some form of bisimulation.

% The notion we use in this paper is weak barbed bisimulation
% \cite{milner91polyadicpi}.

The notion we use in this paper is derived from weak barbed
bisimulation \cite{milner91polyadicpi}. 

\begin{definition}
An \emph{observation relation}, $\downarrow_{\mathcal N}$, over a set
of names, $\mathcal N$, is the smallest relation satisfying the rules
below.

\infrule[Out-barb]{y \in {\mathcal N}, \; x \nameeq y}
		  {\outputp{x}{v} \downarrow_{\mathcal N} x}
\infrule[Par-barb]{\mbox{$P\downarrow_{\mathcal N} x$ or $Q\downarrow_{\mathcal N} x$}}
		  {\binpar{P}{Q} \downarrow_{\mathcal N} x}

We write $P \Downarrow_{\mathcal N} x$ if there is $Q$ such that 
$P \wred Q$ and $Q \downarrow_{\mathcal N} x$.
\end{definition}

\begin{definition}
%\label{def.bbisim}
An  ${\mathcal N}$-\emph{barbed bisimulation} over a set of names, ${\mathcal N}$, is a symmetric binary relation 
${\mathcal S}_{\mathcal N}$ between agents such that $P\rel{S}_{\mathcal N}Q$ implies:
\begin{enumerate}
\item If $P \red P'$ then $Q \wred Q'$ and $P'\rel{S}_{\mathcal N} Q'$.
\item If $P\downarrow_{\mathcal N} x$, then $Q\Downarrow_{\mathcal N} x$.
\end{enumerate}
$P$ is ${\mathcal N}$-barbed bisimilar to $Q$, written
$P \wbbisim_{\mathcal N} Q$, if $P \rel{S}_{\mathcal N} Q$ for some ${\mathcal N}$-barbed bisimulation ${\mathcal S}_{\mathcal N}$.
\end{definition}

$\mathcal{R} \subseteq \pi \times \pi$

$P \mathcal{R} Q => \forall P'. P \red P' \Rightarrow \exists Q'. Q \red Q', P' \mathcal{R} Q'$

$P \vdash x \Rightarrow Q \vdash x$

\begin{mathpar}
  \inferrule*[lab=Out-barb]{x \nameeq y}{{y}!\langle{Q}\rangle \vdash x}
  \and
  \inferrule*[lab=Par-barb]{\mbox{$P\vdash x$ or $Q\vdash x$}}{\binpar{P}{Q} \vdash x}
\end{mathpar}

\subsubsection{Contexts}

One of the principle advantages of computational calculi like the
$\pi$-calculus is a well-defined notion of context,
contextual-equivalence and a correlation between
contextual-equivalence and notions of bisimulation. The notion of
context allows the decomposition of a process into (sub-)process and
its syntactic environment, its context. Thus, a context may be
thought of as a process with a ``hole'' (written $\Box$) in it. The
application of a context $M$ to a process $P$, written $M[P]$, is
tantamount to filling the hole in $M$ with $P$. In this paper we do
not need the full weight of this theory, but do make use of the notion
of context in the proof the main theorem. 

\begin{mathpar}
  \inferrule* [lab=summation] {} {{M_{M},M_{N}} \bc \Box \;|\; x.M_{A} \;|\; M_{M}+M_{N}}
  \and
  \inferrule* [lab=agent] {} {{M_{A}} \bc (\vec{x})M_{P} \;| \; \clift{P_0,\ldots,M_{P},\ldots,P_N}}
  \and \\
  \inferrule* [lab=process] {} {{M_{P}} \bc M_{N} \;| \;P|M_{P} }
\end{mathpar} 

\begin{mathpar}
  \inferrule* [lab=sychronization] {} {M_{N} \bc \Box \;|\; x?M_{F} \;|\; x!M_{C}}
  \and
  \inferrule* [lab=abstraction] {} {{M_{F}} \bc (x)M_{P} }
  \and
  \inferrule* [lab=concretion] {} {{M_{C}} \bc \langle M_{P} \rangle }
  \and \\
  \inferrule* [lab=process] {} {{M_{P}} \bc M_{N} \;| \;P|M_{P} }
\end{mathpar}

\begin{definition}[contextual application] Given a context $M$, and
  process $P$, we define the \emph{contextual application}, $M[P] :=
  M\{P/\Box\}$. That is, the contextual application of M to P is the
  substitution of $P$ for $\Box$ in $M$.
\end{definition}

$\meaningof{-} : L \to \mathcal{P}(\pi)$

\begin{mathpar}
  \inferrule* [lab=collection] {} {\meaningof{true} = \pi, \and \meaningof{~E} = \pi \setminus \meaningof{E}, \and \meaningof{E_{1} \& E_{2}} = \meaningof{E_{1}} \cap \meaningof{E_{2}}}
\end{mathpar}

\begin{mathpar}
  \inferrule* [lab=structure] {} {\meaningof{0} = \{ P \in \pi | P \equiv 0 \}, \and \\ \meaningof{E_1 | E_2} = \{ P \in \pi | P \equiv P_{1} | P_{2}, P_{1} \in \meaningof{E_{1}}, P_{2} \in \meaningof{E_2}\} }
\end{mathpar}

\begin{mathpar}
 \inferrule* [lab=behavior] {} {\meaningof{\langle a?b \rangle E} = \{ P \in \pi | P \equiv Q | u?(y)P', \\ \and \\\\ \and \\ \;\;\; u \in \meaningof{a}, \forall z.P'\{z/y\} \in \meaningof{E\{z/b\}}\}, \and \\ \meaningof{a!E} = \{ P \in \pi | P \equiv Q | x!\langle P' \rangle, x \in \meaningof{a} P' \in \meaningof{E}\} }
\end{mathpar}

\begin{mathpar}
 \inferrule* [lab=nominal] {} {\meaningof{\quotep{E}} = \{ \quotep{P} \in \quotep{\pi} | P \in \meaningof{E} \}, \and \meaningof{\quotep{P}} = \{ \quotep{Q} \in \quotep{\pi} | P \equiv Q \} \and \\ \meaningof{@\quotep{E}} = \{ P \in \pi | P \equiv @x, x \in \meaningof{E} \}}
\end{mathpar}

\begin{eqnarray*}
  \\
  \meaningof{-} : TS \to ST
\end{eqnarray*}

\begin{eqnarray*}
  \\
  L : TS \to ST
\end{eqnarray*}

\begin{eqnarray*}
  \\
  P \models E \iff P \in \meaningof{E}
\end{eqnarray*}

\begin{eqnarray*}
  P \approx_{L} Q \iff \forall E \in L. P \models E \iff Q \models E
\end{eqnarray*}

\begin{eqnarray*}
  P \approx_{K} Q
\end{eqnarray*}

\begin{eqnarray*}
  P \approx Q
\end{eqnarray*}

$\approx_{K} = \approx = \approx_{L}$

\subsubsection{Contextual duality}

Note that contexts extend the quotation operation to a family of
operations from processes to names. Given a context, $M$, we can
define a \emph{nominal context}, $\quotep{M}$ by $\quotep{M}[P] :=
\quotep{M[P]}$. To foreshadow what is to come we observe that these
operations enjoy a duality with processes very much like the duality
between vectors and maps from vectors to scalars.

Further, because the calculus is essentially higher-order, we have a
correspondence between contexts and processes. More specifically,
given a name $x$ and a context $M$ we can construct $M^{*}_{x}$ such
that 

\begin{mathpar}
  M^{*}_{x} | \lift{x}{P} \red M[P]
\end{mathpar}

namely,

\begin{mathpar}
  M^{*}_{x} := x?(u).M[\dropn{u}]
\end{mathpar}

The dependence of $M^{*}_{x}$ on a name makes it an abstraction, 

\begin{mathpar}
  M^{*} := (x)x?(u).M[\dropn{u}]
\end{mathpar}

\subsection{Additional notation}

It will sometimes be convenient to denote the process a name
quotes. We already have the notation $x = \quotep{P}$, but it will be
convenient to introduce an alternate notation, $\procn{x}$, when we
want to emphasize the connection to the use of the name. Note that, by
virtue of name equivalence, $\quotep{\procn{x}} \nameeq x$; so, the
notation is consistent with previous definitions.

Further, because names have structure it is possible to effect
substitutions on the basis of that structure. This means we need to
upgrade our notation for substitutions, which we accomplish by
adapting comprehension notation. Thus,

\begin{mathpar}
  P\{ y / x : x \in S \}
\end{mathpar}

is interpreted to mean the process derived from P by replacing (in a
capture-avoiding manner) each occurrence of $x$ in $S$ by $y$. For example,

\begin{mathpar}
  P\{ \quotep{\procn{x}|\procn{x}} / x : x \in \freenames{P} \}
\end{mathpar}

will replace each (occurrence) of a free name $x$ in $P$ by
$\quotep{\procn{x}|\procn{x}}$.

Also, we will avail ourselves of the notation $x^{L}$ and $x^{R}$ to
denote injections of a name into disjoint copies of the name
space. There are numerous ways to accomplish this. One example can be
found in \cite{MeredithR05}. This notation overloads to vectors of
names: $\vec{x}^{\pi} := (x_{i}^{\pi} \; : \; 0 \leq i < |\vec{x}| )$ where $\pi \in \{L,R\}$.

We also use $P^{\Box} := P|\Box$.

In \cite{MeredithR05} an interpretation of the new operator is
given. It turns out that there are several possible interpretations
all enjoying the requisite algebraic properties of the operator (see
\cite{milner91polyadicpi}). We will therefore make liberal use of
$(\nu\; \vec{x})P$.

% subsection the_syntax_and_semantics_of_the_notation_system (end)   

\section{Interpretation of QM}
\subsection{Supporting definitions}
\subsubsection{Multiplication}
\begin{mathpar}
  \quotep{Q} \cdot \quotep{R} := \quotep{Q|R}
  \and \\
  \quotep{Q} \cdot P := P\{ \quotep{Q|R} / \quotep{R} : \quotep{R} \in \freenames{P} \}
\end{mathpar}

\paragraph{Discussion}
The first line needs little explanation. The second line says that
each free name of the process is replaced with the multiplication of
that name by the scalar. Multiplication of a scalar (name) by a state
(process) results in a process all the names of which have been `moved
over' by parallel composition with the process the scalar
quotes. There is a subtlety that the bound names have to be
manipulated so that multiplied names aren't accidentally
captured. There are many ways to achieve this.

\begin{remark}\label{rem:multiplication_identities}
  The reader is invited to verify that for all $x,y,z \in \QProc$ and $P \in \Proc$
  \begin{mathpar}
    x \cdot \quotep{0} \equiv x 
    \and
    x \cdot y \equiv y \cdot x
    \and
    x \cdot (y \cdot z) \equiv (x \cdot y) \cdot z
    \and \\
    \quotep{0} \cdot P \equiv P
    \and \\
    x \cdot (y \cdot P) \equiv (x \cdot y) \cdot P
    \and \\
    x \cdot (P|Q) \equiv (x \cdot P) | (x \cdot Q)
    \and \\    
  \end{mathpar}
\end{remark}

\subsubsection{Tensor product}

We define a tensor product on processes by structural induction.

\paragraph{Tensor of sums} First note that all summations, including
$\pzero$ and sequence, can be written $\Sigma_{i} x_{i}.A_{i} +
\Sigma_{j} x_{j}.C_{j}$, where we have grouped input-guarded processes
together and output-guarded processes together.

Thus, we can define the tensor product of two summations, $N_{1}\otimes N_{2}$, where

\begin{mathpar}
  N_{1} := \Sigma_{i} x_{i}.A_{i} + \Sigma_{j} x_{j}.C_{j}
  \and
  N_{2} := \Sigma_{i'} y_{i'}.B_{i'} + \Sigma_{j'} y_{j'}.D_{j'} 
\end{mathpar}

as follows.

\begin{mathpar}
  \Sigma_{i} x_{i}.A_{i} + \Sigma_{j} x_{j}.C_{j} \otimes \Sigma_{i'}
  y_{i'}.B_{i'} + \Sigma_{j'} y_{j'}.D_{j'} 
  \and \\
  := \; \Sigma_{i} \Sigma_{i'} \quotep{\stackrel{\vee}{x_{i}}| \stackrel{\vee}{y_{i'}}}.(A_{i}\otimes B_{i'}) \; | \; \Sigma_{i'} \Sigma_{i} \quotep{\stackrel{\vee}{y_{i'}}|\stackrel{\vee}{x_{i}}}.(B_{i'}\otimes A_{i})
  \and
  \;\; | \;\; \Sigma_{j} \Sigma_{j'} \quotep{\stackrel{\vee}{x_{j}}|\stackrel{\vee}{y_{j'}}}.(A_{j}\otimes B_{j'}) \; | \; \Sigma_{j'} \Sigma_{j} \quotep{\stackrel{\vee}{y_{j'}}|\stackrel{\vee}{x_{j}}}.(B_{j'}\otimes A_{j})
\end{mathpar}

\begin{remark}
  Do we need to $x^{L}$ and $y^{R}$ for this construction as well?
\end{remark}

\paragraph{Tensor of parallel compositions} Next, we distribute tensor
over par.

\begin{mathpar}
  P_{1}|P_{2} \otimes Q_{1}|Q_{2} := (P_{1} \otimes Q_{1}) | (P_{1}
  \otimes Q_{2}) | (P_{2} \otimes Q_{1}) | (P_{2} \otimes Q_{2})
\end{mathpar}

\paragraph{Tensor with dropped names} We treat tensor of a
process with a dropped name as parallel composition.

\begin{mathpar}
  P \otimes \dropn{x} := P | \dropn{x}
\end{mathpar}

\paragraph{Tensor of agents}

Finally, we need to define tensor on agents. Note that the definition
of tensor on normal products only tensors inputs with inputs and
outputs with outputs. Thus, we only have to define the operation on
``homogeneous'' pairings.

\begin{mathpar}
  (\vec{x})P \otimes (\vec{y})Q
  \and \\
  := (x_{0}^{L}|y_{0}^{R},\ldots,x_{0}^{L}|y_{n}^{R},\ldots,x_{m}^{L}|y_{0}^{R},\ldots,x_{m}^{L}|y_{n}^R)(P\{ \vec{x}^{L}/\vec{x}\} \otimes Q \{ \vec{y}^{R}/\vec{y}\})
  \and \\
  \clift{\vec{P}} \otimes \clift{\vec{Q}}
  \and \\
  := \clift{P_{0}\otimes Q_{0},\ldots,P_{0}\otimes Q_{n},\ldots,P_{m}\otimes Q_{0},\ldots,P_{m}\otimes Q_{n}}
\end{mathpar}

\begin{remark}
  Observe that arities of tensored abstractions matches arities of
  tensored concretions if the original arities matched. Note also that
  the length of the arities corresponds to the increase in dimension
  we see in ordinary vector space tensor product.
\end{remark}

\begin{remark}
  Operationally, this definition distributes the tensor down to
  components ``linked'' by summation. Tensor over summation is
  intriguing in that it mixes names. Moreover, as a consequence of the
  way it mixes names we have the identities for all $x \in \QProc$ and
  $P,Q \in \Proc$

  \begin{mathpar}
    (x \cdot P) \otimes Q \equiv x \cdot (P \otimes Q) \equiv P \otimes (x \cdot Q)
    \and
    P \otimes \pzero \equiv P
  \end{mathpar}

  that the reader is invited to verify.
\end{remark}

\subsubsection{Annihilation}
\begin{mathpar}
  P^{\perp} := \{ Q | \forall R. P|Q \red^{*} R \Rightarrow R \red^{*} \pzero \}
  \and \\
  P^{\underline{\perp}} := \Sigma_{Q \in P^{\perp}} \quotep{Q}?(y).(\dropn{y}|Q) | \Sigma_{Q \in P^{\perp}} \quotep{Q}\clift{\Box}
\end{mathpar}

\paragraph{Discussion} The reader will note that $P^{\perp}$ is a
\emph{set} of processes, while $P^{\underline{\perp}}$ is a
\emph{context}. We call the set $P^{\perp}$ the \emph{annihilators} of
$P$. The parallel composition of a process in the annihilators of $P$
with $P$ will result in a process, the state space of which has all
paths eventually leading to $\pzero$. Execution may endure loops; but
under reasonable conditions of fairness (naturally guaranteed under
most notions of bisimulation) such a composite process cannot get
stuck in such a loop and will, eventually pop out and terminate.

The context $P^{\underline{\perp}}$ is ready and willing to ``take the
$P$ out of'' the process to which it is applied. It will effectively
transmit the code of the process to which it is applied to one of the
annihilators and run the process against it.

\subsubsection{Evaluation}
We fix $M$ a domain of fully abstract interpretation with an equality
coincident with bisimulation. We take $\meaningof{\cdot} : \Proc \to
M$ to be the map interpreting processes and $\nmeaningof{\cdot} : \M
\to Proc$ to be the map running the other way. Then we define

\begin{mathpar}
  \int P := \nmeaningof{\meaningof{P}}
\end{mathpar}

\paragraph{Discussion}
There are many fully abstract interpretations of Milner's
$\pi$-calculus. Any of them can be used as a basis for interpreting
the reflective calculus here. Equipped with such a domain it is
largely a matter of grinding through to check that the Yoneda
construction for the normalization-by-evaluation program can be
extended to this setting.

\begin{remark}
  The reader is invited to verify that $\int (P^{\underline{\perp}}[P]) = 0$.
\end{remark}

\subsection{Quantum mechanics}

Table \ref{tbl:core_qm_op_defns} gives the core operational definitions

\begin{table}[htp]\label{tbl:core_qm_op_defns}
  \center{
    \fbox{
      \begin{tabular}{c|c}
        quantum mechanics & process calculus \\
        \hline
        scalar & $x := \quotep{P}$ \\
        state vector & $\state{P} := P$ \\
        dual & $\state{P}^{*} := \event{P^{\underline{\perp}}} := \quotep{P^{\underline{\perp}}}[-]$ \\
        matrix & $ \Sigma_{\alpha} \state{P_{\alpha}}x_{\alpha}\event{Q_{\alpha}}$ \\
        vector addition & $\state{P} + \state{Q} := \state{P | Q}$ \\
        tensor product & $\state{P} \otimes \state{Q} := \state{P \otimes Q}$ \\
        inner product & $\innerprod{P}{Q} := \quotep{\int P^{\underline{\perp}}[Q]}$ \\
      \end{tabular}
    }
  }
  \caption{QM - operational definitions}
\end{table}

where

\begin{mathpar}
  \prmatrix{P}{Q} := \fprmatrix{P}{\quotep{\pzero}}{Q}
  \and
  \fprmatrix{P}{x}{Q} := (\state{P},x,\event{Q})
  \and
  (\fprmatrix{P}{x}{Q})(\state{R}) := x \cdot \innerprod{Q}{R} \cdot \state{P}
  \and
  (\fprmatrix{P}{x}{Q})(\event{R}) := x \cdot \innerprod{R}{P} \cdot \event{Q}
\end{mathpar}

\paragraph{Discussion}
As promised: vectors (aka states) are represented as processes; duals
as contextual duals; inner product definition should be compared with
standard inner product definition for ....

\begin{remark}
  Assuming $\int (P^{\underline{\perp}}[P]) = 0$, the reader is
  invited to verify that $(\fprmatrix{P}{x}{P})(\state{P}) = x \cdot \state{P}$.
\end{remark}

\begin{remark}
  The reader is invited to verify that $\innerprod{P}{Q}$ could
  equally well have been written $\quotep{\int \stackrel{\vee}{x}}$
  where $x = \event{P^{\underline{\perp}}}(Q)$.

  One of the motivations for this remark is that there is another way
  to factor these operations. We could package up evaluation in the dual:

  \begin{mathpar}
    \state{P}^{*} := \event{\int P^{\underline{\perp}}} := \quotep{\int P^{\underline{\perp}}}[-]
  \end{mathpar}

  and then have inner product defined by
  
  \begin{mathpar}
    \innerprod{P}{Q} := \event{P}(Q)
  \end{mathpar}

  Hopefully, experience with the calculations will provide guidance on
  the best factoring.
\end{remark}

\begin{remark}
  Assuming $\int (P^{\underline{\perp}}[P]) = 0$, the reader is
  invited to verify that $\forall P,Q. (\prmatrix{0}{Q})(\state{0}) =
  \state{0}$ and dually $(\prmatrix{P}{0})(\event{0}) = \event{0}$.
\end{remark}

\begin{remark}
  i'm a little worried that i don't (yet) have proper support for
  complex conjugacy. But, the observation above may give us a
  clue. According to Abramsky, it must be the case that the scalars
  are iso to the homset of the identity for the tensor -- which the
  observation above characterizes. 

  For now, we will simply bookmark the notion with $\overline{x}$.
\end{remark}

\subsubsection{Adjointness}

We need to give a definition of $(\cdot)^{\dagger}$ for matrices. The
obvious candidate definition is
\begin{mathpar}
(\Sigma_{\alpha}\fprmatrix{P_{\alpha}}{x_{\alpha}}{Q_{\alpha}})^{\dagger}
= \Sigma_{\alpha}\fprmatrix{(Q_{\alpha}^{\underline{\perp}})^{*}}{\overline{x}_{\alpha}}{P_{\alpha}^{\underline{\perp}}} 
\end{mathpar}

But, $(Q_{\alpha}^{\underline{\perp}})^{*}$ requires a name along
which to communicate the process to achieve the context application.

\subsubsection{Basis for a basis}
If processes label states and ``addition'' of states (a.k.a. vector
addition) is interpreted as parallel composition, what corresponds to
notions of linear independence and basis? Here, we recall that Yoshida
has developed a set of \emph{combinators} for an asynchronous verison
of Milner's $\pi$-calculus. These are a finite set of processes such
any process can be expressed as parallel composition of these
combinators together with liberal uses of the new operator and
replication. We can simply give a translation of these into the
present calculus and have reasonable expectation that the property
carries over. That is, that the resultant set allows to express all
processes via parallel composition. Note, however, that there is no
new operator or replication in this calculus. As a result, we expect
that the corresponding set is actually infinite. That is, we expect
that the space is actually infinite dimensional.

\begin{remark}
  The attentive reader may be a bit concerned. Certainly, the
  collection $S$, $K$ and $I$ is a finite set of
  combinators. Shouldn't we expect to see a finite set of combinators
  for an effectively equivalent system? i am very sympathetic to this
  critique and feel it warrants full attention. On the other hand, i
  also have in mind the following analogy. The natural numbers, as a
  monoid under addition, has exactly $1$ generator, while the natural
  numbers, as a monoid under multiplication, has countably many
  generators (the primes). We observe that the application of the
  lambda calculus is much less resource sensitive than the parallel
  composition of the $\pi$-calculus. Could it be the case that we have
  an analogy of the form
  
  \begin{mathpar}
    m + n : MN :: m*n : M|N
  \end{mathpar}

  giving a similar blow up in the set of ``primes''?  This is such a
  wonderful thought that, even if it's not true, i think it's worth
  writing down.
\end{remark}
 

\documentclass[12pt]{llncs}
%\documentclass{jktr}

\usepackage[pdftex]{hyperref}                   
\usepackage {listings}
\usepackage {mathpartir}
\usepackage{bcprules}
%\usepackage{listings}
                       
\usepackage{graphicx} 
%\usepackage[margins=2.5cm,nohead,nofoot]{geometry}
%\usepackage{geometry}
\usepackage{amsfonts}
\usepackage{amstext}
\usepackage{latexsym}
\usepackage{amssymb}
\usepackage{color}


%\include{myPreamble}
\include{qm2pi.local} 

%\ifpdf
%\usepackage[pdftex]{graphicx}
%\else
%\usepackage{graphicx}
%\fi

 % \ifpdf
%  \usepackage{pdfsync}
%  \if


%\title{Brief Article}
%\author{David F. Snyder}
%\author{L.G. Meredith}

%\address{Dept. of Math., Texas State University--San Marcos, San Marcos, TX 78666}
       
\pagestyle{empty}


\begin{document}

\lstset{language=[Objective]Caml,frame=shadowbox}

\input{qm2pi.front}

% section front matter (end)

\input{qm2pi.intro} 
 
% section introduction (end)

% \input{qm2pi.knotations} 

% section notation (end)

\input{qm2pi.process.calculi} 

% section concurrent_process_calculi_and_spatial_logics_ (end)
    
%\input{qm2pi.knots2pi} 

%\input{qm2pi.trefoil} 

%\input{qm2pi.mainthm} 

% subsection basic_interpretation (end)

%\input{qm2pi.rho.presentation} 
\subsection{The syntax and semantics of the notation system}\label{sub:the_syntax_and_semantics_of_the_notation_system} % (fold)

We now summarize a technical presentation of the calculus that
embodies our theory of dynamics. The typical presentation of such a
calculus follows the style of giving generators and relations on
them. The grammar, below, describing term constructors, freely
generates the set of processes, $\Proc$. This set is then quotiented
by a relation known as structural congruence and it is over this set
that the notion of dynamics is expressed. This presentation is
essentially that of \cite{MeredithR05} with the addition of
polyadicity and summation. For readability we have relegated some of
the technical subtleties to an appendix.

\subsubsection{Process grammar}\label{subsub:process_grammar}

\begin{mathpar}
  \inferrule* [lab=synchronization] {} {{M} \bc \pzero \;|\; x?F \;|\; x!C }
  \and
  \inferrule* [lab=abstraction] {} {{F} \bc (x)P}
  \and
  \inferrule* [lab=concretion] {} {{C} \bc \langle Q \rangle}
  \and
  \inferrule* [lab=process] {} {{P,Q} \bc M \;| \;P|Q \;|\; @{x}}
  \and
  \inferrule* [lab=name] {} {{x} \bc \quotep{P}}
\end{mathpar} 

Note that $\vec{x}$ (resp. $\vec{P}$) denotes a vector of names
(resp. processes) of length $|\vec{x}|$ (resp. $|\vec{P}|$). We adopt
the following useful abbreviations.

\begin{mathpar}
   x?(\vec{y}).P := x.(\vec{y})P \and  x\clift{\vec{P}} := x.\clift{\vec{P}}
   \and x!(y) := \lift{x}{\dropn{y}}
   \and \Pi_{i=0}^{n-1}P_i := P_0 | \ldots | P_{n-1}
\end{mathpar}

\subsubsection{Structural congruence}

\paragraph{Free and bound names and alpha-equivalence.} At the
core of structural equivalence is alpha-equivalence which identifies
process that are the same up to a change of variable. Formally, we
recognize the distinction between free and bound names. The free names
of a process, $\freenames{P}$, may be calculated recursively as
follows:

\begin{mathpar}
\freenames{\pzero} := \emptyset
  \and \\
  \freenames{x?(y).P} := \{ x \} \cup (\freenames{P} \setminus \{ y \})
  \and 
  \freenames{x!\langle P \rangle} := \{ x \} \cup \{ P \} 
  \and \\
  \freenames{P|Q} := \freenames{P} \cup \freenames{Q}
  \and \\
  \freenames{@{x}} := \{ x \}
\end{mathpar}

$\pi$
$\quotep{\pi}$

$\freenames{-} : \pi \to \mathcal{P}(\quotep{\pi})$

\begin{eqnarray*}
  \freenames{\pzero} & := & \emptyset \\
  \freenames{x?(y).P} & := & \{ x \} \cup (\freenames{P} \setminus \{ y \}) \\
  \freenames{x!\langle P \rangle} & := & \{ x \} \cup \{ P \} \\
  \freenames{P|Q} & := & \freenames{P} \cup \freenames{Q} \\
  \freenames{\dropn{x}} & := & \{ x \}
\end{eqnarray*}

The bound names of a process, $\boundnames{P}$, are those names occurring in $P$
that are not free. For example, in $x?(y).0$, the name $x$ is free, while $y$ is bound.

\begin{mathpar}
  \inferrule* [lab=monoidal-laws] {} { P|Q \equiv Q|P \and P|0 \equiv P \and P|(Q|R) \equiv (P|Q)|R }
\end{mathpar}

\begin{mathpar}
  \inferrule* [lab=alpha-equivalence] {} { (x)P \equiv (y)P\{y/x\} \and y \not\in \freenames{P} }
\end{mathpar}

\begin{definition}
Then two processes, $P,Q$, are alpha-equivalent if $P = Q\{\vec{y}/\vec{x}\}$ for
some $\vec{x} \in \boundnames{Q},\vec{y} \in \boundnames{P}$, where $Q\{\vec{y}/\vec{x}\}$
denotes the capture-avoiding substitution of $\vec{y}$ for $\vec{x}$ in $Q$.
\end{definition}

\begin{definition}
  The {\em structural congruence} \cite{SangiorgiWalker} , $\equiv$,
  between processes is the least congruence containing
  alpha-equivalence, satisfying the abelian monoid laws
  (associativity, commutativity and $\pzero$ as identity) for parallel
  composition $|$ and for summation $+$.
\end{definition}

\subsection{Name equivalence}

We take name equivalence, written $\nameeq$, to be the smallest
equivalence relation generated by the following rules.

\begin{mathpar}
\inferrule*[lab=Quote-drop]
{ }
{ \quotep{@{x}} \nameeq x }

\inferrule*[lab=Struct-equiv]
{ P \scong Q }
{ \quotep{P} \nameeq \quotep{Q} }
\end{mathpar}

The astute reader will have noticed that the mutual recursion of names
and processes imposes a mutual recursion on alpha-equivalence and
structural equivalence via name-equivalence. Fortunately, all of this
works out pleasantly and we may calculate in the natural way, free of
concern. The reader interested in the details is referred to the
appendix \ref{appendix:rho_details}.

\subsection{Substitution}

We use $\Proc$ for the set of processes, $\QProc$ for the set of
names, and $\id{\{}\vec{y} / \vec{x} \id{\}}$ to denote partial maps,
$s : \QProc \rightarrow \QProc$. A map, $s$ lifts, uniquely, to a map
on process terms, $\widehat{s} : \Proc \rightarrow \Proc$ by the
following equations.

\begin{mathpar}
  (0) \psubstp{Q}{P} := 0 \\
  (R \juxtap S) \psubstp{Q}{P}
  :=    
  (R)\psubstp{Q}{P} \juxtap (S) \psubstp{Q}{P} \\
  (x?(y).R) \psubstp{Q}{P}    
  :=    
  (x)\substp{Q}{P} (z)\concat( (R \psubstn{z}{y}) \psubstp{Q}{P} ) \\
  (\lift{x}{R}) \psubstp{Q}{P}  
  :=
  \lift{(x)\substp{Q}{P}}{ R \psubstp{Q}{P} } \\
%   (\dropn{x})  \psubstp{Q}{P}       
%   := 
%   \left\{ 
%     \begin{array}{ccc} 
%       \dropn{\quotep{Q}} & & x \nameeq \quotep{P} \\
%       \dropn{x} & & otherwise \\
%     \end{array}
%   \right. 
  (\dropn{x})  \psubstp{Q}{P}       
  := 
  \left\{ 
    \begin{array}{ccc} 
      Q & & x \nameeq \quotep{P} \\
      \dropn{x} & & otherwise \\
    \end{array}
  \right.
\end{mathpar}
 

where

\begin{eqnarray}
  (x)\id{\{} \lpquote Q \rpquote / \lpquote P \rpquote \id{\}}            = 
  \left\{ 
    \begin{array}{ccc}
      \lpquote Q \rpquote & & x \nameeq \lpquote P \rpquote \\
      x & & otherwise \\
    \end{array}
  \right. \nonumber
\end{eqnarray}

and $z$ is chosen distinct from $\quotep{P}$, $\quotep{Q}$, the free
names in $Q$, and all the names in $R$. Our $\alpha$-equivalence will
be built in the standard way from this substitution.

\begin{remark}\label{rem:no_self_referential_names}
  One consequence of these definitions is that $\forall P. \quotep{P}
  \not\in \freenames{P}$.
\end{remark}

\subsection{ Dynamic quote: an example }

Anticipating something of what's to come, consider applying the
substitution, $\widehat{\id{\{}u / z \id{\}}}$, to the following pair
of processes, $\lift{w}{y!(z)}$ and $w[ \lpquote y!(z) \rpquote ]$.

\begin{eqnarray}
	\lift{w}{y!(z)}\widehat{\id{\{}u / z \id{\}}}
		& = &
		\lift{w}{y!(u)} \nonumber\\
	w[ \lpquote y!(z) \rpquote ] \widehat{ \id{\{}u / z \id{\}} }
		& = &
		w[ \lpquote y!(z) \rpquote ] \nonumber
\end{eqnarray}

Because the body of the process between quotes is impervious to
substitution, we get radically different answers. In fact, by
examining the first process in an input context,
e.g. $x?(z).\lift{w}{y!(z)}$, we see that the process under the lift
operator may be shaped by prefixed inputs binding a name inside it. In
this sense, the lift operator will be seen as a way to dynamically
construct processes before reifying them as names.

Finally equipped with these standard features we can present the
dynamics of the calculus.

\subsubsection{Operational semantics} 

Finally, we introduce the computational dynamics. What marks these
algebras as distinct from other more traditionally studied algebraic
structures, e.g. vector spaces or polynomial rings, is the manner in
which dynamics is captured. In traditional structures, dynamics is typically
expressed through morphisms between such structures, as in linear maps
between vector spaces or morphisms between rings. In algebras
associated with the semantics of computation, the dynamics is
expressed as part of the algebraic structure itself, through a
reduction reduction relation typically denoted by $\red$. Below, we
give a recursive presentation of this relation for the calculus used
in the encoding.

$\red \subseteq \pi \times \pi$
$\red : \pi \to \mathcal{P}(\pi)$

\begin{mathpar}
  \inferrule* [lab=Comm] { \textsf{match}( x_{src}, x_{trgt} ) } { x_{trgt}?(y)P \; | \; x_{src}!\langle {Q} \rangle \red P\{\quotep{Q}/y}\} }
  \and \\
  \inferrule* [lab=Par] {{P} \red {P}'} {{{P} | {Q}} \red {{P}' | {Q}}}
  \and
  \inferrule* [lab=Equiv]{{{P} \scong {P}'} \andalso {{P}' \red {Q}'} \andalso {{Q}' \scong {Q}}}{{P} \red {Q}}
\end{mathpar}

\begin{eqnarray*}
  match_{\equiv} (\quotep{P},\quotep{Q}) & := & P \equiv Q \\
  match_{\dagger}(\quotep{P},\quotep{Q}) & := & \forall R. P|Q \red^{*} R => R \red^{*} 0 \\
  match_{K}(\quotep{P},\quotep{Q}) & := & K \mbox{ for some context } K
\end{eqnarray*}

$u?(x)P | u!\langle Q \rangle \red P\{\quotep{Q}/x\}$

%We write $\wred$ for $\red^*$, and $P\red$ if $\exists Q $ such that $ P \red Q$.
We write $P\red$ if $\exists Q $ such that $ P \red Q$ and $P\not\red$, otherwise.

\section{Replication}

As mentioned before, it is known that replication (and hence
recursion) can be implemented in a higher-order process algebra
\cite{SangiorgiWalker}. As our first example of calculation with the
machinery thus far presented we give the construction explicitly in
the {\rhoc}.

\begin{eqnarray}
	D_{x} & := & \prefix{x}{y}{(\binpar{\outputp{x}{y}}{@{y}})} \nonumber\\
	\bangp_{x}{P} & := & \binpar{{x}!\langle{\binpar{D_{x}}{P}}\rangle}{D_{x}} \nonumber
\end{eqnarray}

\begin{eqnarray}
	\bangp_{x}{P} & & \nonumber\\
	=
	& {x}!\langle{(\prefix{x}{y}{(\outputp{x}{y} | @{y})) | P}}\rangle 
	      | \prefix{x}{y}{(\outputp{x}{y} | @{y})} & \nonumber\\
	\red
	& (\outputp{x}{y} | @{y})\substn{\quotep{(\prefix{x}{y}{(@{y} | \outputp{x}{y})) | P}}}{y} & \nonumber\\
	=
	& \outputp{x}{\quotep{(\prefix{x}{y}{(\outputp{x}{y} | @{y})) | P}}}
	  | {(\prefix{x}{y}{(\outputp{x}{y} | @{y})) | P}} & \nonumber\\
	\red
	& \ldots & \nonumber\\
	\red^*
	& P | P | \ldots & \nonumber
\end{eqnarray}

Of course, this encoding, as an implementation, runs away, unfolding
$\bangp{P}$ eagerly. A lazier and more implementable replication
operator, restricted to input-guarded processes, may be obtained as follows.

\begin{eqnarray}
\bangp{\prefix{u}{v}{P}} 
	:= 
	\binpar{\lift{x}{\prefix{u}{v}{(\binpar{D(x)}{P})}}}{D(x)} \nonumber
\end{eqnarray}

\begin{remark}
  Note that the lazier definition still does not deal with summation
  or mixed summation (i.e. sums over input and output). The reader is
  invited to construct definitions of replication that deal with these
  features. 

  Further, the definitions are parameterized in a name, $x$. Can you,
  gentle reader, make a definition that eliminates this parameter and
  guarantees no accidental interaction between the replication
  machinery and the process being replicated -- i.e. no accidental
  sharing of names used by the process to get its work done and the
  name(s) used by the replication to effect copying. This latter
  revision of the definition of replication is crucial to obtaining
  the expected identity $!!P \sim !P$.
\end{remark}

\begin{remark}\label{rem:paradoxical_combinator}
  The reader familiar with the lambda calculus will have noticed the
  similarity between $D$ and the paradoxical combinator.

  [Ed. note: the existence of this seems to suggest we have to be more
  restrictive on the set of processes and names we admit if we are to
  support no-cloning.]
\end{remark}

\subsubsection{Bisimulation}

The computational dynamics gives rise to another kind of equivalence,
the equivalence of computational behavior. As previously mentioned
this is typically captured \emph{via} some form of bisimulation.

% The notion we use in this paper is weak barbed bisimulation
% \cite{milner91polyadicpi}.

The notion we use in this paper is derived from weak barbed
bisimulation \cite{milner91polyadicpi}. 

\begin{definition}
An \emph{observation relation}, $\downarrow_{\mathcal N}$, over a set
of names, $\mathcal N$, is the smallest relation satisfying the rules
below.

\infrule[Out-barb]{y \in {\mathcal N}, \; x \nameeq y}
		  {\outputp{x}{v} \downarrow_{\mathcal N} x}
\infrule[Par-barb]{\mbox{$P\downarrow_{\mathcal N} x$ or $Q\downarrow_{\mathcal N} x$}}
		  {\binpar{P}{Q} \downarrow_{\mathcal N} x}

We write $P \Downarrow_{\mathcal N} x$ if there is $Q$ such that 
$P \wred Q$ and $Q \downarrow_{\mathcal N} x$.
\end{definition}

\begin{definition}
%\label{def.bbisim}
An  ${\mathcal N}$-\emph{barbed bisimulation} over a set of names, ${\mathcal N}$, is a symmetric binary relation 
${\mathcal S}_{\mathcal N}$ between agents such that $P\rel{S}_{\mathcal N}Q$ implies:
\begin{enumerate}
\item If $P \red P'$ then $Q \wred Q'$ and $P'\rel{S}_{\mathcal N} Q'$.
\item If $P\downarrow_{\mathcal N} x$, then $Q\Downarrow_{\mathcal N} x$.
\end{enumerate}
$P$ is ${\mathcal N}$-barbed bisimilar to $Q$, written
$P \wbbisim_{\mathcal N} Q$, if $P \rel{S}_{\mathcal N} Q$ for some ${\mathcal N}$-barbed bisimulation ${\mathcal S}_{\mathcal N}$.
\end{definition}

$\mathcal{R} \subseteq \pi \times \pi$

$P \mathcal{R} Q => \forall P'. P \red P' \Rightarrow \exists Q'. Q \red Q', P' \mathcal{R} Q'$

$P \vdash x \Rightarrow Q \vdash x$

\begin{mathpar}
  \inferrule*[lab=Out-barb]{x \nameeq y}{{y}!\langle{Q}\rangle \vdash x}
  \and
  \inferrule*[lab=Par-barb]{\mbox{$P\vdash x$ or $Q\vdash x$}}{\binpar{P}{Q} \vdash x}
\end{mathpar}

\subsubsection{Contexts}

One of the principle advantages of computational calculi like the
$\pi$-calculus is a well-defined notion of context,
contextual-equivalence and a correlation between
contextual-equivalence and notions of bisimulation. The notion of
context allows the decomposition of a process into (sub-)process and
its syntactic environment, its context. Thus, a context may be
thought of as a process with a ``hole'' (written $\Box$) in it. The
application of a context $M$ to a process $P$, written $M[P]$, is
tantamount to filling the hole in $M$ with $P$. In this paper we do
not need the full weight of this theory, but do make use of the notion
of context in the proof the main theorem. 

\begin{mathpar}
  \inferrule* [lab=summation] {} {{M_{M},M_{N}} \bc \Box \;|\; x.M_{A} \;|\; M_{M}+M_{N}}
  \and
  \inferrule* [lab=agent] {} {{M_{A}} \bc (\vec{x})M_{P} \;| \; \clift{P_0,\ldots,M_{P},\ldots,P_N}}
  \and \\
  \inferrule* [lab=process] {} {{M_{P}} \bc M_{N} \;| \;P|M_{P} }
\end{mathpar} 

\begin{mathpar}
  \inferrule* [lab=sychronization] {} {M_{N} \bc \Box \;|\; x?M_{F} \;|\; x!M_{C}}
  \and
  \inferrule* [lab=abstraction] {} {{M_{F}} \bc (x)M_{P} }
  \and
  \inferrule* [lab=concretion] {} {{M_{C}} \bc \langle M_{P} \rangle }
  \and \\
  \inferrule* [lab=process] {} {{M_{P}} \bc M_{N} \;| \;P|M_{P} }
\end{mathpar}

\begin{definition}[contextual application] Given a context $M$, and
  process $P$, we define the \emph{contextual application}, $M[P] :=
  M\{P/\Box\}$. That is, the contextual application of M to P is the
  substitution of $P$ for $\Box$ in $M$.
\end{definition}

$\meaningof{-} : L \to \mathcal{P}(\pi)$

\begin{mathpar}
  \inferrule* [lab=collection] {} {\meaningof{true} = \pi, \and \meaningof{~E} = \pi \setminus \meaningof{E}, \and \meaningof{E_{1} \& E_{2}} = \meaningof{E_{1}} \cap \meaningof{E_{2}}}
\end{mathpar}

\begin{mathpar}
  \inferrule* [lab=structure] {} {\meaningof{0} = \{ P \in \pi | P \equiv 0 \}, \and \\ \meaningof{E_1 | E_2} = \{ P \in \pi | P \equiv P_{1} | P_{2}, P_{1} \in \meaningof{E_{1}}, P_{2} \in \meaningof{E_2}\} }
\end{mathpar}

\begin{mathpar}
 \inferrule* [lab=behavior] {} {\meaningof{\langle a?b \rangle E} = \{ P \in \pi | P \equiv Q | u?(y)P', \\ \and \\\\ \and \\ \;\;\; u \in \meaningof{a}, \forall z.P'\{z/y\} \in \meaningof{E\{z/b\}}\}, \and \\ \meaningof{a!E} = \{ P \in \pi | P \equiv Q | x!\langle P' \rangle, x \in \meaningof{a} P' \in \meaningof{E}\} }
\end{mathpar}

\begin{mathpar}
 \inferrule* [lab=nominal] {} {\meaningof{\quotep{E}} = \{ \quotep{P} \in \quotep{\pi} | P \in \meaningof{E} \}, \and \meaningof{\quotep{P}} = \{ \quotep{Q} \in \quotep{\pi} | P \equiv Q \} \and \\ \meaningof{@\quotep{E}} = \{ P \in \pi | P \equiv @x, x \in \meaningof{E} \}}
\end{mathpar}

\begin{eqnarray*}
  \\
  \meaningof{-} : TS \to ST
\end{eqnarray*}

\begin{eqnarray*}
  \\
  L : TS \to ST
\end{eqnarray*}

\begin{eqnarray*}
  \\
  P \models E \iff P \in \meaningof{E}
\end{eqnarray*}

\begin{eqnarray*}
  P \approx_{L} Q \iff \forall E \in L. P \models E \iff Q \models E
\end{eqnarray*}

\begin{eqnarray*}
  P \approx_{K} Q
\end{eqnarray*}

\begin{eqnarray*}
  P \approx Q
\end{eqnarray*}

$\approx_{K} = \approx = \approx_{L}$

\subsubsection{Contextual duality}

Note that contexts extend the quotation operation to a family of
operations from processes to names. Given a context, $M$, we can
define a \emph{nominal context}, $\quotep{M}$ by $\quotep{M}[P] :=
\quotep{M[P]}$. To foreshadow what is to come we observe that these
operations enjoy a duality with processes very much like the duality
between vectors and maps from vectors to scalars.

Further, because the calculus is essentially higher-order, we have a
correspondence between contexts and processes. More specifically,
given a name $x$ and a context $M$ we can construct $M^{*}_{x}$ such
that 

\begin{mathpar}
  M^{*}_{x} | \lift{x}{P} \red M[P]
\end{mathpar}

namely,

\begin{mathpar}
  M^{*}_{x} := x?(u).M[\dropn{u}]
\end{mathpar}

The dependence of $M^{*}_{x}$ on a name makes it an abstraction, 

\begin{mathpar}
  M^{*} := (x)x?(u).M[\dropn{u}]
\end{mathpar}

\subsection{Additional notation}

It will sometimes be convenient to denote the process a name
quotes. We already have the notation $x = \quotep{P}$, but it will be
convenient to introduce an alternate notation, $\procn{x}$, when we
want to emphasize the connection to the use of the name. Note that, by
virtue of name equivalence, $\quotep{\procn{x}} \nameeq x$; so, the
notation is consistent with previous definitions.

Further, because names have structure it is possible to effect
substitutions on the basis of that structure. This means we need to
upgrade our notation for substitutions, which we accomplish by
adapting comprehension notation. Thus,

\begin{mathpar}
  P\{ y / x : x \in S \}
\end{mathpar}

is interpreted to mean the process derived from P by replacing (in a
capture-avoiding manner) each occurrence of $x$ in $S$ by $y$. For example,

\begin{mathpar}
  P\{ \quotep{\procn{x}|\procn{x}} / x : x \in \freenames{P} \}
\end{mathpar}

will replace each (occurrence) of a free name $x$ in $P$ by
$\quotep{\procn{x}|\procn{x}}$.

Also, we will avail ourselves of the notation $x^{L}$ and $x^{R}$ to
denote injections of a name into disjoint copies of the name
space. There are numerous ways to accomplish this. One example can be
found in \cite{MeredithR05}. This notation overloads to vectors of
names: $\vec{x}^{\pi} := (x_{i}^{\pi} \; : \; 0 \leq i < |\vec{x}| )$ where $\pi \in \{L,R\}$.

We also use $P^{\Box} := P|\Box$.

In \cite{MeredithR05} an interpretation of the new operator is
given. It turns out that there are several possible interpretations
all enjoying the requisite algebraic properties of the operator (see
\cite{milner91polyadicpi}). We will therefore make liberal use of
$(\nu\; \vec{x})P$.

% subsection the_syntax_and_semantics_of_the_notation_system (end)   

\input{qm2pi.qmops} 

\input{qm2pi.sterngerlach} 

\input{qm2pi.metric} 

% section concurrent_process_calculi (end)

%\input{qm2pi.proofsketch}

% section proof sketch (end)

%\input{qm2pi.slviaknots} 

% section spatial logic via knots (end)

\input{qm2pi.conclusion}

% section conclusion (end)

%\input{qm2pi.dtcodes} 

% section wiring algorithm (end)

\input{qm2pi.ack} 

% section acknowledgments (end)

\newpage


\bibliographystyle{plain}   
\bibliography{../../biblios/main.bib}

\input{qm2pi.rhodetails}

\end{document}

 

\documentclass[12pt]{llncs}
%\documentclass{jktr}

\usepackage[pdftex]{hyperref}                   
\usepackage {listings}
\usepackage {mathpartir}
\usepackage{bcprules}
%\usepackage{listings}
                       
\usepackage{graphicx} 
%\usepackage[margins=2.5cm,nohead,nofoot]{geometry}
%\usepackage{geometry}
\usepackage{amsfonts}
\usepackage{amstext}
\usepackage{latexsym}
\usepackage{amssymb}
\usepackage{color}


%\include{myPreamble}
\include{qm2pi.local} 

%\ifpdf
%\usepackage[pdftex]{graphicx}
%\else
%\usepackage{graphicx}
%\fi

 % \ifpdf
%  \usepackage{pdfsync}
%  \if


%\title{Brief Article}
%\author{David F. Snyder}
%\author{L.G. Meredith}

%\address{Dept. of Math., Texas State University--San Marcos, San Marcos, TX 78666}
       
\pagestyle{empty}


\begin{document}

\lstset{language=[Objective]Caml,frame=shadowbox}

\input{qm2pi.front}

% section front matter (end)

\input{qm2pi.intro} 
 
% section introduction (end)

% \input{qm2pi.knotations} 

% section notation (end)

\input{qm2pi.process.calculi} 

% section concurrent_process_calculi_and_spatial_logics_ (end)
    
%\input{qm2pi.knots2pi} 

%\input{qm2pi.trefoil} 

%\input{qm2pi.mainthm} 

% subsection basic_interpretation (end)

%\input{qm2pi.rho.presentation} 
\subsection{The syntax and semantics of the notation system}\label{sub:the_syntax_and_semantics_of_the_notation_system} % (fold)

We now summarize a technical presentation of the calculus that
embodies our theory of dynamics. The typical presentation of such a
calculus follows the style of giving generators and relations on
them. The grammar, below, describing term constructors, freely
generates the set of processes, $\Proc$. This set is then quotiented
by a relation known as structural congruence and it is over this set
that the notion of dynamics is expressed. This presentation is
essentially that of \cite{MeredithR05} with the addition of
polyadicity and summation. For readability we have relegated some of
the technical subtleties to an appendix.

\subsubsection{Process grammar}\label{subsub:process_grammar}

\begin{mathpar}
  \inferrule* [lab=synchronization] {} {{M} \bc \pzero \;|\; x?F \;|\; x!C }
  \and
  \inferrule* [lab=abstraction] {} {{F} \bc (x)P}
  \and
  \inferrule* [lab=concretion] {} {{C} \bc \langle Q \rangle}
  \and
  \inferrule* [lab=process] {} {{P,Q} \bc M \;| \;P|Q \;|\; @{x}}
  \and
  \inferrule* [lab=name] {} {{x} \bc \quotep{P}}
\end{mathpar} 

Note that $\vec{x}$ (resp. $\vec{P}$) denotes a vector of names
(resp. processes) of length $|\vec{x}|$ (resp. $|\vec{P}|$). We adopt
the following useful abbreviations.

\begin{mathpar}
   x?(\vec{y}).P := x.(\vec{y})P \and  x\clift{\vec{P}} := x.\clift{\vec{P}}
   \and x!(y) := \lift{x}{\dropn{y}}
   \and \Pi_{i=0}^{n-1}P_i := P_0 | \ldots | P_{n-1}
\end{mathpar}

\subsubsection{Structural congruence}

\paragraph{Free and bound names and alpha-equivalence.} At the
core of structural equivalence is alpha-equivalence which identifies
process that are the same up to a change of variable. Formally, we
recognize the distinction between free and bound names. The free names
of a process, $\freenames{P}$, may be calculated recursively as
follows:

\begin{mathpar}
\freenames{\pzero} := \emptyset
  \and \\
  \freenames{x?(y).P} := \{ x \} \cup (\freenames{P} \setminus \{ y \})
  \and 
  \freenames{x!\langle P \rangle} := \{ x \} \cup \{ P \} 
  \and \\
  \freenames{P|Q} := \freenames{P} \cup \freenames{Q}
  \and \\
  \freenames{@{x}} := \{ x \}
\end{mathpar}

$\pi$
$\quotep{\pi}$

$\freenames{-} : \pi \to \mathcal{P}(\quotep{\pi})$

\begin{eqnarray*}
  \freenames{\pzero} & := & \emptyset \\
  \freenames{x?(y).P} & := & \{ x \} \cup (\freenames{P} \setminus \{ y \}) \\
  \freenames{x!\langle P \rangle} & := & \{ x \} \cup \{ P \} \\
  \freenames{P|Q} & := & \freenames{P} \cup \freenames{Q} \\
  \freenames{\dropn{x}} & := & \{ x \}
\end{eqnarray*}

The bound names of a process, $\boundnames{P}$, are those names occurring in $P$
that are not free. For example, in $x?(y).0$, the name $x$ is free, while $y$ is bound.

\begin{mathpar}
  \inferrule* [lab=monoidal-laws] {} { P|Q \equiv Q|P \and P|0 \equiv P \and P|(Q|R) \equiv (P|Q)|R }
\end{mathpar}

\begin{mathpar}
  \inferrule* [lab=alpha-equivalence] {} { (x)P \equiv (y)P\{y/x\} \and y \not\in \freenames{P} }
\end{mathpar}

\begin{definition}
Then two processes, $P,Q$, are alpha-equivalent if $P = Q\{\vec{y}/\vec{x}\}$ for
some $\vec{x} \in \boundnames{Q},\vec{y} \in \boundnames{P}$, where $Q\{\vec{y}/\vec{x}\}$
denotes the capture-avoiding substitution of $\vec{y}$ for $\vec{x}$ in $Q$.
\end{definition}

\begin{definition}
  The {\em structural congruence} \cite{SangiorgiWalker} , $\equiv$,
  between processes is the least congruence containing
  alpha-equivalence, satisfying the abelian monoid laws
  (associativity, commutativity and $\pzero$ as identity) for parallel
  composition $|$ and for summation $+$.
\end{definition}

\subsection{Name equivalence}

We take name equivalence, written $\nameeq$, to be the smallest
equivalence relation generated by the following rules.

\begin{mathpar}
\inferrule*[lab=Quote-drop]
{ }
{ \quotep{@{x}} \nameeq x }

\inferrule*[lab=Struct-equiv]
{ P \scong Q }
{ \quotep{P} \nameeq \quotep{Q} }
\end{mathpar}

The astute reader will have noticed that the mutual recursion of names
and processes imposes a mutual recursion on alpha-equivalence and
structural equivalence via name-equivalence. Fortunately, all of this
works out pleasantly and we may calculate in the natural way, free of
concern. The reader interested in the details is referred to the
appendix \ref{appendix:rho_details}.

\subsection{Substitution}

We use $\Proc$ for the set of processes, $\QProc$ for the set of
names, and $\id{\{}\vec{y} / \vec{x} \id{\}}$ to denote partial maps,
$s : \QProc \rightarrow \QProc$. A map, $s$ lifts, uniquely, to a map
on process terms, $\widehat{s} : \Proc \rightarrow \Proc$ by the
following equations.

\begin{mathpar}
  (0) \psubstp{Q}{P} := 0 \\
  (R \juxtap S) \psubstp{Q}{P}
  :=    
  (R)\psubstp{Q}{P} \juxtap (S) \psubstp{Q}{P} \\
  (x?(y).R) \psubstp{Q}{P}    
  :=    
  (x)\substp{Q}{P} (z)\concat( (R \psubstn{z}{y}) \psubstp{Q}{P} ) \\
  (\lift{x}{R}) \psubstp{Q}{P}  
  :=
  \lift{(x)\substp{Q}{P}}{ R \psubstp{Q}{P} } \\
%   (\dropn{x})  \psubstp{Q}{P}       
%   := 
%   \left\{ 
%     \begin{array}{ccc} 
%       \dropn{\quotep{Q}} & & x \nameeq \quotep{P} \\
%       \dropn{x} & & otherwise \\
%     \end{array}
%   \right. 
  (\dropn{x})  \psubstp{Q}{P}       
  := 
  \left\{ 
    \begin{array}{ccc} 
      Q & & x \nameeq \quotep{P} \\
      \dropn{x} & & otherwise \\
    \end{array}
  \right.
\end{mathpar}
 

where

\begin{eqnarray}
  (x)\id{\{} \lpquote Q \rpquote / \lpquote P \rpquote \id{\}}            = 
  \left\{ 
    \begin{array}{ccc}
      \lpquote Q \rpquote & & x \nameeq \lpquote P \rpquote \\
      x & & otherwise \\
    \end{array}
  \right. \nonumber
\end{eqnarray}

and $z$ is chosen distinct from $\quotep{P}$, $\quotep{Q}$, the free
names in $Q$, and all the names in $R$. Our $\alpha$-equivalence will
be built in the standard way from this substitution.

\begin{remark}\label{rem:no_self_referential_names}
  One consequence of these definitions is that $\forall P. \quotep{P}
  \not\in \freenames{P}$.
\end{remark}

\subsection{ Dynamic quote: an example }

Anticipating something of what's to come, consider applying the
substitution, $\widehat{\id{\{}u / z \id{\}}}$, to the following pair
of processes, $\lift{w}{y!(z)}$ and $w[ \lpquote y!(z) \rpquote ]$.

\begin{eqnarray}
	\lift{w}{y!(z)}\widehat{\id{\{}u / z \id{\}}}
		& = &
		\lift{w}{y!(u)} \nonumber\\
	w[ \lpquote y!(z) \rpquote ] \widehat{ \id{\{}u / z \id{\}} }
		& = &
		w[ \lpquote y!(z) \rpquote ] \nonumber
\end{eqnarray}

Because the body of the process between quotes is impervious to
substitution, we get radically different answers. In fact, by
examining the first process in an input context,
e.g. $x?(z).\lift{w}{y!(z)}$, we see that the process under the lift
operator may be shaped by prefixed inputs binding a name inside it. In
this sense, the lift operator will be seen as a way to dynamically
construct processes before reifying them as names.

Finally equipped with these standard features we can present the
dynamics of the calculus.

\subsubsection{Operational semantics} 

Finally, we introduce the computational dynamics. What marks these
algebras as distinct from other more traditionally studied algebraic
structures, e.g. vector spaces or polynomial rings, is the manner in
which dynamics is captured. In traditional structures, dynamics is typically
expressed through morphisms between such structures, as in linear maps
between vector spaces or morphisms between rings. In algebras
associated with the semantics of computation, the dynamics is
expressed as part of the algebraic structure itself, through a
reduction reduction relation typically denoted by $\red$. Below, we
give a recursive presentation of this relation for the calculus used
in the encoding.

$\red \subseteq \pi \times \pi$
$\red : \pi \to \mathcal{P}(\pi)$

\begin{mathpar}
  \inferrule* [lab=Comm] { \textsf{match}( x_{src}, x_{trgt} ) } { x_{trgt}?(y)P \; | \; x_{src}!\langle {Q} \rangle \red P\{\quotep{Q}/y}\} }
  \and \\
  \inferrule* [lab=Par] {{P} \red {P}'} {{{P} | {Q}} \red {{P}' | {Q}}}
  \and
  \inferrule* [lab=Equiv]{{{P} \scong {P}'} \andalso {{P}' \red {Q}'} \andalso {{Q}' \scong {Q}}}{{P} \red {Q}}
\end{mathpar}

\begin{eqnarray*}
  match_{\equiv} (\quotep{P},\quotep{Q}) & := & P \equiv Q \\
  match_{\dagger}(\quotep{P},\quotep{Q}) & := & \forall R. P|Q \red^{*} R => R \red^{*} 0 \\
  match_{K}(\quotep{P},\quotep{Q}) & := & K \mbox{ for some context } K
\end{eqnarray*}

$u?(x)P | u!\langle Q \rangle \red P\{\quotep{Q}/x\}$

%We write $\wred$ for $\red^*$, and $P\red$ if $\exists Q $ such that $ P \red Q$.
We write $P\red$ if $\exists Q $ such that $ P \red Q$ and $P\not\red$, otherwise.

\section{Replication}

As mentioned before, it is known that replication (and hence
recursion) can be implemented in a higher-order process algebra
\cite{SangiorgiWalker}. As our first example of calculation with the
machinery thus far presented we give the construction explicitly in
the {\rhoc}.

\begin{eqnarray}
	D_{x} & := & \prefix{x}{y}{(\binpar{\outputp{x}{y}}{@{y}})} \nonumber\\
	\bangp_{x}{P} & := & \binpar{{x}!\langle{\binpar{D_{x}}{P}}\rangle}{D_{x}} \nonumber
\end{eqnarray}

\begin{eqnarray}
	\bangp_{x}{P} & & \nonumber\\
	=
	& {x}!\langle{(\prefix{x}{y}{(\outputp{x}{y} | @{y})) | P}}\rangle 
	      | \prefix{x}{y}{(\outputp{x}{y} | @{y})} & \nonumber\\
	\red
	& (\outputp{x}{y} | @{y})\substn{\quotep{(\prefix{x}{y}{(@{y} | \outputp{x}{y})) | P}}}{y} & \nonumber\\
	=
	& \outputp{x}{\quotep{(\prefix{x}{y}{(\outputp{x}{y} | @{y})) | P}}}
	  | {(\prefix{x}{y}{(\outputp{x}{y} | @{y})) | P}} & \nonumber\\
	\red
	& \ldots & \nonumber\\
	\red^*
	& P | P | \ldots & \nonumber
\end{eqnarray}

Of course, this encoding, as an implementation, runs away, unfolding
$\bangp{P}$ eagerly. A lazier and more implementable replication
operator, restricted to input-guarded processes, may be obtained as follows.

\begin{eqnarray}
\bangp{\prefix{u}{v}{P}} 
	:= 
	\binpar{\lift{x}{\prefix{u}{v}{(\binpar{D(x)}{P})}}}{D(x)} \nonumber
\end{eqnarray}

\begin{remark}
  Note that the lazier definition still does not deal with summation
  or mixed summation (i.e. sums over input and output). The reader is
  invited to construct definitions of replication that deal with these
  features. 

  Further, the definitions are parameterized in a name, $x$. Can you,
  gentle reader, make a definition that eliminates this parameter and
  guarantees no accidental interaction between the replication
  machinery and the process being replicated -- i.e. no accidental
  sharing of names used by the process to get its work done and the
  name(s) used by the replication to effect copying. This latter
  revision of the definition of replication is crucial to obtaining
  the expected identity $!!P \sim !P$.
\end{remark}

\begin{remark}\label{rem:paradoxical_combinator}
  The reader familiar with the lambda calculus will have noticed the
  similarity between $D$ and the paradoxical combinator.

  [Ed. note: the existence of this seems to suggest we have to be more
  restrictive on the set of processes and names we admit if we are to
  support no-cloning.]
\end{remark}

\subsubsection{Bisimulation}

The computational dynamics gives rise to another kind of equivalence,
the equivalence of computational behavior. As previously mentioned
this is typically captured \emph{via} some form of bisimulation.

% The notion we use in this paper is weak barbed bisimulation
% \cite{milner91polyadicpi}.

The notion we use in this paper is derived from weak barbed
bisimulation \cite{milner91polyadicpi}. 

\begin{definition}
An \emph{observation relation}, $\downarrow_{\mathcal N}$, over a set
of names, $\mathcal N$, is the smallest relation satisfying the rules
below.

\infrule[Out-barb]{y \in {\mathcal N}, \; x \nameeq y}
		  {\outputp{x}{v} \downarrow_{\mathcal N} x}
\infrule[Par-barb]{\mbox{$P\downarrow_{\mathcal N} x$ or $Q\downarrow_{\mathcal N} x$}}
		  {\binpar{P}{Q} \downarrow_{\mathcal N} x}

We write $P \Downarrow_{\mathcal N} x$ if there is $Q$ such that 
$P \wred Q$ and $Q \downarrow_{\mathcal N} x$.
\end{definition}

\begin{definition}
%\label{def.bbisim}
An  ${\mathcal N}$-\emph{barbed bisimulation} over a set of names, ${\mathcal N}$, is a symmetric binary relation 
${\mathcal S}_{\mathcal N}$ between agents such that $P\rel{S}_{\mathcal N}Q$ implies:
\begin{enumerate}
\item If $P \red P'$ then $Q \wred Q'$ and $P'\rel{S}_{\mathcal N} Q'$.
\item If $P\downarrow_{\mathcal N} x$, then $Q\Downarrow_{\mathcal N} x$.
\end{enumerate}
$P$ is ${\mathcal N}$-barbed bisimilar to $Q$, written
$P \wbbisim_{\mathcal N} Q$, if $P \rel{S}_{\mathcal N} Q$ for some ${\mathcal N}$-barbed bisimulation ${\mathcal S}_{\mathcal N}$.
\end{definition}

$\mathcal{R} \subseteq \pi \times \pi$

$P \mathcal{R} Q => \forall P'. P \red P' \Rightarrow \exists Q'. Q \red Q', P' \mathcal{R} Q'$

$P \vdash x \Rightarrow Q \vdash x$

\begin{mathpar}
  \inferrule*[lab=Out-barb]{x \nameeq y}{{y}!\langle{Q}\rangle \vdash x}
  \and
  \inferrule*[lab=Par-barb]{\mbox{$P\vdash x$ or $Q\vdash x$}}{\binpar{P}{Q} \vdash x}
\end{mathpar}

\subsubsection{Contexts}

One of the principle advantages of computational calculi like the
$\pi$-calculus is a well-defined notion of context,
contextual-equivalence and a correlation between
contextual-equivalence and notions of bisimulation. The notion of
context allows the decomposition of a process into (sub-)process and
its syntactic environment, its context. Thus, a context may be
thought of as a process with a ``hole'' (written $\Box$) in it. The
application of a context $M$ to a process $P$, written $M[P]$, is
tantamount to filling the hole in $M$ with $P$. In this paper we do
not need the full weight of this theory, but do make use of the notion
of context in the proof the main theorem. 

\begin{mathpar}
  \inferrule* [lab=summation] {} {{M_{M},M_{N}} \bc \Box \;|\; x.M_{A} \;|\; M_{M}+M_{N}}
  \and
  \inferrule* [lab=agent] {} {{M_{A}} \bc (\vec{x})M_{P} \;| \; \clift{P_0,\ldots,M_{P},\ldots,P_N}}
  \and \\
  \inferrule* [lab=process] {} {{M_{P}} \bc M_{N} \;| \;P|M_{P} }
\end{mathpar} 

\begin{mathpar}
  \inferrule* [lab=sychronization] {} {M_{N} \bc \Box \;|\; x?M_{F} \;|\; x!M_{C}}
  \and
  \inferrule* [lab=abstraction] {} {{M_{F}} \bc (x)M_{P} }
  \and
  \inferrule* [lab=concretion] {} {{M_{C}} \bc \langle M_{P} \rangle }
  \and \\
  \inferrule* [lab=process] {} {{M_{P}} \bc M_{N} \;| \;P|M_{P} }
\end{mathpar}

\begin{definition}[contextual application] Given a context $M$, and
  process $P$, we define the \emph{contextual application}, $M[P] :=
  M\{P/\Box\}$. That is, the contextual application of M to P is the
  substitution of $P$ for $\Box$ in $M$.
\end{definition}

$\meaningof{-} : L \to \mathcal{P}(\pi)$

\begin{mathpar}
  \inferrule* [lab=collection] {} {\meaningof{true} = \pi, \and \meaningof{~E} = \pi \setminus \meaningof{E}, \and \meaningof{E_{1} \& E_{2}} = \meaningof{E_{1}} \cap \meaningof{E_{2}}}
\end{mathpar}

\begin{mathpar}
  \inferrule* [lab=structure] {} {\meaningof{0} = \{ P \in \pi | P \equiv 0 \}, \and \\ \meaningof{E_1 | E_2} = \{ P \in \pi | P \equiv P_{1} | P_{2}, P_{1} \in \meaningof{E_{1}}, P_{2} \in \meaningof{E_2}\} }
\end{mathpar}

\begin{mathpar}
 \inferrule* [lab=behavior] {} {\meaningof{\langle a?b \rangle E} = \{ P \in \pi | P \equiv Q | u?(y)P', \\ \and \\\\ \and \\ \;\;\; u \in \meaningof{a}, \forall z.P'\{z/y\} \in \meaningof{E\{z/b\}}\}, \and \\ \meaningof{a!E} = \{ P \in \pi | P \equiv Q | x!\langle P' \rangle, x \in \meaningof{a} P' \in \meaningof{E}\} }
\end{mathpar}

\begin{mathpar}
 \inferrule* [lab=nominal] {} {\meaningof{\quotep{E}} = \{ \quotep{P} \in \quotep{\pi} | P \in \meaningof{E} \}, \and \meaningof{\quotep{P}} = \{ \quotep{Q} \in \quotep{\pi} | P \equiv Q \} \and \\ \meaningof{@\quotep{E}} = \{ P \in \pi | P \equiv @x, x \in \meaningof{E} \}}
\end{mathpar}

\begin{eqnarray*}
  \\
  \meaningof{-} : TS \to ST
\end{eqnarray*}

\begin{eqnarray*}
  \\
  L : TS \to ST
\end{eqnarray*}

\begin{eqnarray*}
  \\
  P \models E \iff P \in \meaningof{E}
\end{eqnarray*}

\begin{eqnarray*}
  P \approx_{L} Q \iff \forall E \in L. P \models E \iff Q \models E
\end{eqnarray*}

\begin{eqnarray*}
  P \approx_{K} Q
\end{eqnarray*}

\begin{eqnarray*}
  P \approx Q
\end{eqnarray*}

$\approx_{K} = \approx = \approx_{L}$

\subsubsection{Contextual duality}

Note that contexts extend the quotation operation to a family of
operations from processes to names. Given a context, $M$, we can
define a \emph{nominal context}, $\quotep{M}$ by $\quotep{M}[P] :=
\quotep{M[P]}$. To foreshadow what is to come we observe that these
operations enjoy a duality with processes very much like the duality
between vectors and maps from vectors to scalars.

Further, because the calculus is essentially higher-order, we have a
correspondence between contexts and processes. More specifically,
given a name $x$ and a context $M$ we can construct $M^{*}_{x}$ such
that 

\begin{mathpar}
  M^{*}_{x} | \lift{x}{P} \red M[P]
\end{mathpar}

namely,

\begin{mathpar}
  M^{*}_{x} := x?(u).M[\dropn{u}]
\end{mathpar}

The dependence of $M^{*}_{x}$ on a name makes it an abstraction, 

\begin{mathpar}
  M^{*} := (x)x?(u).M[\dropn{u}]
\end{mathpar}

\subsection{Additional notation}

It will sometimes be convenient to denote the process a name
quotes. We already have the notation $x = \quotep{P}$, but it will be
convenient to introduce an alternate notation, $\procn{x}$, when we
want to emphasize the connection to the use of the name. Note that, by
virtue of name equivalence, $\quotep{\procn{x}} \nameeq x$; so, the
notation is consistent with previous definitions.

Further, because names have structure it is possible to effect
substitutions on the basis of that structure. This means we need to
upgrade our notation for substitutions, which we accomplish by
adapting comprehension notation. Thus,

\begin{mathpar}
  P\{ y / x : x \in S \}
\end{mathpar}

is interpreted to mean the process derived from P by replacing (in a
capture-avoiding manner) each occurrence of $x$ in $S$ by $y$. For example,

\begin{mathpar}
  P\{ \quotep{\procn{x}|\procn{x}} / x : x \in \freenames{P} \}
\end{mathpar}

will replace each (occurrence) of a free name $x$ in $P$ by
$\quotep{\procn{x}|\procn{x}}$.

Also, we will avail ourselves of the notation $x^{L}$ and $x^{R}$ to
denote injections of a name into disjoint copies of the name
space. There are numerous ways to accomplish this. One example can be
found in \cite{MeredithR05}. This notation overloads to vectors of
names: $\vec{x}^{\pi} := (x_{i}^{\pi} \; : \; 0 \leq i < |\vec{x}| )$ where $\pi \in \{L,R\}$.

We also use $P^{\Box} := P|\Box$.

In \cite{MeredithR05} an interpretation of the new operator is
given. It turns out that there are several possible interpretations
all enjoying the requisite algebraic properties of the operator (see
\cite{milner91polyadicpi}). We will therefore make liberal use of
$(\nu\; \vec{x})P$.

% subsection the_syntax_and_semantics_of_the_notation_system (end)   

\input{qm2pi.qmops} 

\input{qm2pi.sterngerlach} 

\input{qm2pi.metric} 

% section concurrent_process_calculi (end)

%\input{qm2pi.proofsketch}

% section proof sketch (end)

%\input{qm2pi.slviaknots} 

% section spatial logic via knots (end)

\input{qm2pi.conclusion}

% section conclusion (end)

%\input{qm2pi.dtcodes} 

% section wiring algorithm (end)

\input{qm2pi.ack} 

% section acknowledgments (end)

\newpage


\bibliographystyle{plain}   
\bibliography{../../biblios/main.bib}

\input{qm2pi.rhodetails}

\end{document}

 

% section concurrent_process_calculi (end)

%\documentclass[12pt]{llncs}
%\documentclass{jktr}

\usepackage[pdftex]{hyperref}                   
\usepackage {listings}
\usepackage {mathpartir}
\usepackage{bcprules}
%\usepackage{listings}
                       
\usepackage{graphicx} 
%\usepackage[margins=2.5cm,nohead,nofoot]{geometry}
%\usepackage{geometry}
\usepackage{amsfonts}
\usepackage{amstext}
\usepackage{latexsym}
\usepackage{amssymb}
\usepackage{color}


%\include{myPreamble}
\include{qm2pi.local} 

%\ifpdf
%\usepackage[pdftex]{graphicx}
%\else
%\usepackage{graphicx}
%\fi

 % \ifpdf
%  \usepackage{pdfsync}
%  \if


%\title{Brief Article}
%\author{David F. Snyder}
%\author{L.G. Meredith}

%\address{Dept. of Math., Texas State University--San Marcos, San Marcos, TX 78666}
       
\pagestyle{empty}


\begin{document}

\lstset{language=[Objective]Caml,frame=shadowbox}

\input{qm2pi.front}

% section front matter (end)

\input{qm2pi.intro} 
 
% section introduction (end)

% \input{qm2pi.knotations} 

% section notation (end)

\input{qm2pi.process.calculi} 

% section concurrent_process_calculi_and_spatial_logics_ (end)
    
%\input{qm2pi.knots2pi} 

%\input{qm2pi.trefoil} 

%\input{qm2pi.mainthm} 

% subsection basic_interpretation (end)

%\input{qm2pi.rho.presentation} 
\subsection{The syntax and semantics of the notation system}\label{sub:the_syntax_and_semantics_of_the_notation_system} % (fold)

We now summarize a technical presentation of the calculus that
embodies our theory of dynamics. The typical presentation of such a
calculus follows the style of giving generators and relations on
them. The grammar, below, describing term constructors, freely
generates the set of processes, $\Proc$. This set is then quotiented
by a relation known as structural congruence and it is over this set
that the notion of dynamics is expressed. This presentation is
essentially that of \cite{MeredithR05} with the addition of
polyadicity and summation. For readability we have relegated some of
the technical subtleties to an appendix.

\subsubsection{Process grammar}\label{subsub:process_grammar}

\begin{mathpar}
  \inferrule* [lab=synchronization] {} {{M} \bc \pzero \;|\; x?F \;|\; x!C }
  \and
  \inferrule* [lab=abstraction] {} {{F} \bc (x)P}
  \and
  \inferrule* [lab=concretion] {} {{C} \bc \langle Q \rangle}
  \and
  \inferrule* [lab=process] {} {{P,Q} \bc M \;| \;P|Q \;|\; @{x}}
  \and
  \inferrule* [lab=name] {} {{x} \bc \quotep{P}}
\end{mathpar} 

Note that $\vec{x}$ (resp. $\vec{P}$) denotes a vector of names
(resp. processes) of length $|\vec{x}|$ (resp. $|\vec{P}|$). We adopt
the following useful abbreviations.

\begin{mathpar}
   x?(\vec{y}).P := x.(\vec{y})P \and  x\clift{\vec{P}} := x.\clift{\vec{P}}
   \and x!(y) := \lift{x}{\dropn{y}}
   \and \Pi_{i=0}^{n-1}P_i := P_0 | \ldots | P_{n-1}
\end{mathpar}

\subsubsection{Structural congruence}

\paragraph{Free and bound names and alpha-equivalence.} At the
core of structural equivalence is alpha-equivalence which identifies
process that are the same up to a change of variable. Formally, we
recognize the distinction between free and bound names. The free names
of a process, $\freenames{P}$, may be calculated recursively as
follows:

\begin{mathpar}
\freenames{\pzero} := \emptyset
  \and \\
  \freenames{x?(y).P} := \{ x \} \cup (\freenames{P} \setminus \{ y \})
  \and 
  \freenames{x!\langle P \rangle} := \{ x \} \cup \{ P \} 
  \and \\
  \freenames{P|Q} := \freenames{P} \cup \freenames{Q}
  \and \\
  \freenames{@{x}} := \{ x \}
\end{mathpar}

$\pi$
$\quotep{\pi}$

$\freenames{-} : \pi \to \mathcal{P}(\quotep{\pi})$

\begin{eqnarray*}
  \freenames{\pzero} & := & \emptyset \\
  \freenames{x?(y).P} & := & \{ x \} \cup (\freenames{P} \setminus \{ y \}) \\
  \freenames{x!\langle P \rangle} & := & \{ x \} \cup \{ P \} \\
  \freenames{P|Q} & := & \freenames{P} \cup \freenames{Q} \\
  \freenames{\dropn{x}} & := & \{ x \}
\end{eqnarray*}

The bound names of a process, $\boundnames{P}$, are those names occurring in $P$
that are not free. For example, in $x?(y).0$, the name $x$ is free, while $y$ is bound.

\begin{mathpar}
  \inferrule* [lab=monoidal-laws] {} { P|Q \equiv Q|P \and P|0 \equiv P \and P|(Q|R) \equiv (P|Q)|R }
\end{mathpar}

\begin{mathpar}
  \inferrule* [lab=alpha-equivalence] {} { (x)P \equiv (y)P\{y/x\} \and y \not\in \freenames{P} }
\end{mathpar}

\begin{definition}
Then two processes, $P,Q$, are alpha-equivalent if $P = Q\{\vec{y}/\vec{x}\}$ for
some $\vec{x} \in \boundnames{Q},\vec{y} \in \boundnames{P}$, where $Q\{\vec{y}/\vec{x}\}$
denotes the capture-avoiding substitution of $\vec{y}$ for $\vec{x}$ in $Q$.
\end{definition}

\begin{definition}
  The {\em structural congruence} \cite{SangiorgiWalker} , $\equiv$,
  between processes is the least congruence containing
  alpha-equivalence, satisfying the abelian monoid laws
  (associativity, commutativity and $\pzero$ as identity) for parallel
  composition $|$ and for summation $+$.
\end{definition}

\subsection{Name equivalence}

We take name equivalence, written $\nameeq$, to be the smallest
equivalence relation generated by the following rules.

\begin{mathpar}
\inferrule*[lab=Quote-drop]
{ }
{ \quotep{@{x}} \nameeq x }

\inferrule*[lab=Struct-equiv]
{ P \scong Q }
{ \quotep{P} \nameeq \quotep{Q} }
\end{mathpar}

The astute reader will have noticed that the mutual recursion of names
and processes imposes a mutual recursion on alpha-equivalence and
structural equivalence via name-equivalence. Fortunately, all of this
works out pleasantly and we may calculate in the natural way, free of
concern. The reader interested in the details is referred to the
appendix \ref{appendix:rho_details}.

\subsection{Substitution}

We use $\Proc$ for the set of processes, $\QProc$ for the set of
names, and $\id{\{}\vec{y} / \vec{x} \id{\}}$ to denote partial maps,
$s : \QProc \rightarrow \QProc$. A map, $s$ lifts, uniquely, to a map
on process terms, $\widehat{s} : \Proc \rightarrow \Proc$ by the
following equations.

\begin{mathpar}
  (0) \psubstp{Q}{P} := 0 \\
  (R \juxtap S) \psubstp{Q}{P}
  :=    
  (R)\psubstp{Q}{P} \juxtap (S) \psubstp{Q}{P} \\
  (x?(y).R) \psubstp{Q}{P}    
  :=    
  (x)\substp{Q}{P} (z)\concat( (R \psubstn{z}{y}) \psubstp{Q}{P} ) \\
  (\lift{x}{R}) \psubstp{Q}{P}  
  :=
  \lift{(x)\substp{Q}{P}}{ R \psubstp{Q}{P} } \\
%   (\dropn{x})  \psubstp{Q}{P}       
%   := 
%   \left\{ 
%     \begin{array}{ccc} 
%       \dropn{\quotep{Q}} & & x \nameeq \quotep{P} \\
%       \dropn{x} & & otherwise \\
%     \end{array}
%   \right. 
  (\dropn{x})  \psubstp{Q}{P}       
  := 
  \left\{ 
    \begin{array}{ccc} 
      Q & & x \nameeq \quotep{P} \\
      \dropn{x} & & otherwise \\
    \end{array}
  \right.
\end{mathpar}
 

where

\begin{eqnarray}
  (x)\id{\{} \lpquote Q \rpquote / \lpquote P \rpquote \id{\}}            = 
  \left\{ 
    \begin{array}{ccc}
      \lpquote Q \rpquote & & x \nameeq \lpquote P \rpquote \\
      x & & otherwise \\
    \end{array}
  \right. \nonumber
\end{eqnarray}

and $z$ is chosen distinct from $\quotep{P}$, $\quotep{Q}$, the free
names in $Q$, and all the names in $R$. Our $\alpha$-equivalence will
be built in the standard way from this substitution.

\begin{remark}\label{rem:no_self_referential_names}
  One consequence of these definitions is that $\forall P. \quotep{P}
  \not\in \freenames{P}$.
\end{remark}

\subsection{ Dynamic quote: an example }

Anticipating something of what's to come, consider applying the
substitution, $\widehat{\id{\{}u / z \id{\}}}$, to the following pair
of processes, $\lift{w}{y!(z)}$ and $w[ \lpquote y!(z) \rpquote ]$.

\begin{eqnarray}
	\lift{w}{y!(z)}\widehat{\id{\{}u / z \id{\}}}
		& = &
		\lift{w}{y!(u)} \nonumber\\
	w[ \lpquote y!(z) \rpquote ] \widehat{ \id{\{}u / z \id{\}} }
		& = &
		w[ \lpquote y!(z) \rpquote ] \nonumber
\end{eqnarray}

Because the body of the process between quotes is impervious to
substitution, we get radically different answers. In fact, by
examining the first process in an input context,
e.g. $x?(z).\lift{w}{y!(z)}$, we see that the process under the lift
operator may be shaped by prefixed inputs binding a name inside it. In
this sense, the lift operator will be seen as a way to dynamically
construct processes before reifying them as names.

Finally equipped with these standard features we can present the
dynamics of the calculus.

\subsubsection{Operational semantics} 

Finally, we introduce the computational dynamics. What marks these
algebras as distinct from other more traditionally studied algebraic
structures, e.g. vector spaces or polynomial rings, is the manner in
which dynamics is captured. In traditional structures, dynamics is typically
expressed through morphisms between such structures, as in linear maps
between vector spaces or morphisms between rings. In algebras
associated with the semantics of computation, the dynamics is
expressed as part of the algebraic structure itself, through a
reduction reduction relation typically denoted by $\red$. Below, we
give a recursive presentation of this relation for the calculus used
in the encoding.

$\red \subseteq \pi \times \pi$
$\red : \pi \to \mathcal{P}(\pi)$

\begin{mathpar}
  \inferrule* [lab=Comm] { \textsf{match}( x_{src}, x_{trgt} ) } { x_{trgt}?(y)P \; | \; x_{src}!\langle {Q} \rangle \red P\{\quotep{Q}/y}\} }
  \and \\
  \inferrule* [lab=Par] {{P} \red {P}'} {{{P} | {Q}} \red {{P}' | {Q}}}
  \and
  \inferrule* [lab=Equiv]{{{P} \scong {P}'} \andalso {{P}' \red {Q}'} \andalso {{Q}' \scong {Q}}}{{P} \red {Q}}
\end{mathpar}

\begin{eqnarray*}
  match_{\equiv} (\quotep{P},\quotep{Q}) & := & P \equiv Q \\
  match_{\dagger}(\quotep{P},\quotep{Q}) & := & \forall R. P|Q \red^{*} R => R \red^{*} 0 \\
  match_{K}(\quotep{P},\quotep{Q}) & := & K \mbox{ for some context } K
\end{eqnarray*}

$u?(x)P | u!\langle Q \rangle \red P\{\quotep{Q}/x\}$

%We write $\wred$ for $\red^*$, and $P\red$ if $\exists Q $ such that $ P \red Q$.
We write $P\red$ if $\exists Q $ such that $ P \red Q$ and $P\not\red$, otherwise.

\section{Replication}

As mentioned before, it is known that replication (and hence
recursion) can be implemented in a higher-order process algebra
\cite{SangiorgiWalker}. As our first example of calculation with the
machinery thus far presented we give the construction explicitly in
the {\rhoc}.

\begin{eqnarray}
	D_{x} & := & \prefix{x}{y}{(\binpar{\outputp{x}{y}}{@{y}})} \nonumber\\
	\bangp_{x}{P} & := & \binpar{{x}!\langle{\binpar{D_{x}}{P}}\rangle}{D_{x}} \nonumber
\end{eqnarray}

\begin{eqnarray}
	\bangp_{x}{P} & & \nonumber\\
	=
	& {x}!\langle{(\prefix{x}{y}{(\outputp{x}{y} | @{y})) | P}}\rangle 
	      | \prefix{x}{y}{(\outputp{x}{y} | @{y})} & \nonumber\\
	\red
	& (\outputp{x}{y} | @{y})\substn{\quotep{(\prefix{x}{y}{(@{y} | \outputp{x}{y})) | P}}}{y} & \nonumber\\
	=
	& \outputp{x}{\quotep{(\prefix{x}{y}{(\outputp{x}{y} | @{y})) | P}}}
	  | {(\prefix{x}{y}{(\outputp{x}{y} | @{y})) | P}} & \nonumber\\
	\red
	& \ldots & \nonumber\\
	\red^*
	& P | P | \ldots & \nonumber
\end{eqnarray}

Of course, this encoding, as an implementation, runs away, unfolding
$\bangp{P}$ eagerly. A lazier and more implementable replication
operator, restricted to input-guarded processes, may be obtained as follows.

\begin{eqnarray}
\bangp{\prefix{u}{v}{P}} 
	:= 
	\binpar{\lift{x}{\prefix{u}{v}{(\binpar{D(x)}{P})}}}{D(x)} \nonumber
\end{eqnarray}

\begin{remark}
  Note that the lazier definition still does not deal with summation
  or mixed summation (i.e. sums over input and output). The reader is
  invited to construct definitions of replication that deal with these
  features. 

  Further, the definitions are parameterized in a name, $x$. Can you,
  gentle reader, make a definition that eliminates this parameter and
  guarantees no accidental interaction between the replication
  machinery and the process being replicated -- i.e. no accidental
  sharing of names used by the process to get its work done and the
  name(s) used by the replication to effect copying. This latter
  revision of the definition of replication is crucial to obtaining
  the expected identity $!!P \sim !P$.
\end{remark}

\begin{remark}\label{rem:paradoxical_combinator}
  The reader familiar with the lambda calculus will have noticed the
  similarity between $D$ and the paradoxical combinator.

  [Ed. note: the existence of this seems to suggest we have to be more
  restrictive on the set of processes and names we admit if we are to
  support no-cloning.]
\end{remark}

\subsubsection{Bisimulation}

The computational dynamics gives rise to another kind of equivalence,
the equivalence of computational behavior. As previously mentioned
this is typically captured \emph{via} some form of bisimulation.

% The notion we use in this paper is weak barbed bisimulation
% \cite{milner91polyadicpi}.

The notion we use in this paper is derived from weak barbed
bisimulation \cite{milner91polyadicpi}. 

\begin{definition}
An \emph{observation relation}, $\downarrow_{\mathcal N}$, over a set
of names, $\mathcal N$, is the smallest relation satisfying the rules
below.

\infrule[Out-barb]{y \in {\mathcal N}, \; x \nameeq y}
		  {\outputp{x}{v} \downarrow_{\mathcal N} x}
\infrule[Par-barb]{\mbox{$P\downarrow_{\mathcal N} x$ or $Q\downarrow_{\mathcal N} x$}}
		  {\binpar{P}{Q} \downarrow_{\mathcal N} x}

We write $P \Downarrow_{\mathcal N} x$ if there is $Q$ such that 
$P \wred Q$ and $Q \downarrow_{\mathcal N} x$.
\end{definition}

\begin{definition}
%\label{def.bbisim}
An  ${\mathcal N}$-\emph{barbed bisimulation} over a set of names, ${\mathcal N}$, is a symmetric binary relation 
${\mathcal S}_{\mathcal N}$ between agents such that $P\rel{S}_{\mathcal N}Q$ implies:
\begin{enumerate}
\item If $P \red P'$ then $Q \wred Q'$ and $P'\rel{S}_{\mathcal N} Q'$.
\item If $P\downarrow_{\mathcal N} x$, then $Q\Downarrow_{\mathcal N} x$.
\end{enumerate}
$P$ is ${\mathcal N}$-barbed bisimilar to $Q$, written
$P \wbbisim_{\mathcal N} Q$, if $P \rel{S}_{\mathcal N} Q$ for some ${\mathcal N}$-barbed bisimulation ${\mathcal S}_{\mathcal N}$.
\end{definition}

$\mathcal{R} \subseteq \pi \times \pi$

$P \mathcal{R} Q => \forall P'. P \red P' \Rightarrow \exists Q'. Q \red Q', P' \mathcal{R} Q'$

$P \vdash x \Rightarrow Q \vdash x$

\begin{mathpar}
  \inferrule*[lab=Out-barb]{x \nameeq y}{{y}!\langle{Q}\rangle \vdash x}
  \and
  \inferrule*[lab=Par-barb]{\mbox{$P\vdash x$ or $Q\vdash x$}}{\binpar{P}{Q} \vdash x}
\end{mathpar}

\subsubsection{Contexts}

One of the principle advantages of computational calculi like the
$\pi$-calculus is a well-defined notion of context,
contextual-equivalence and a correlation between
contextual-equivalence and notions of bisimulation. The notion of
context allows the decomposition of a process into (sub-)process and
its syntactic environment, its context. Thus, a context may be
thought of as a process with a ``hole'' (written $\Box$) in it. The
application of a context $M$ to a process $P$, written $M[P]$, is
tantamount to filling the hole in $M$ with $P$. In this paper we do
not need the full weight of this theory, but do make use of the notion
of context in the proof the main theorem. 

\begin{mathpar}
  \inferrule* [lab=summation] {} {{M_{M},M_{N}} \bc \Box \;|\; x.M_{A} \;|\; M_{M}+M_{N}}
  \and
  \inferrule* [lab=agent] {} {{M_{A}} \bc (\vec{x})M_{P} \;| \; \clift{P_0,\ldots,M_{P},\ldots,P_N}}
  \and \\
  \inferrule* [lab=process] {} {{M_{P}} \bc M_{N} \;| \;P|M_{P} }
\end{mathpar} 

\begin{mathpar}
  \inferrule* [lab=sychronization] {} {M_{N} \bc \Box \;|\; x?M_{F} \;|\; x!M_{C}}
  \and
  \inferrule* [lab=abstraction] {} {{M_{F}} \bc (x)M_{P} }
  \and
  \inferrule* [lab=concretion] {} {{M_{C}} \bc \langle M_{P} \rangle }
  \and \\
  \inferrule* [lab=process] {} {{M_{P}} \bc M_{N} \;| \;P|M_{P} }
\end{mathpar}

\begin{definition}[contextual application] Given a context $M$, and
  process $P$, we define the \emph{contextual application}, $M[P] :=
  M\{P/\Box\}$. That is, the contextual application of M to P is the
  substitution of $P$ for $\Box$ in $M$.
\end{definition}

$\meaningof{-} : L \to \mathcal{P}(\pi)$

\begin{mathpar}
  \inferrule* [lab=collection] {} {\meaningof{true} = \pi, \and \meaningof{~E} = \pi \setminus \meaningof{E}, \and \meaningof{E_{1} \& E_{2}} = \meaningof{E_{1}} \cap \meaningof{E_{2}}}
\end{mathpar}

\begin{mathpar}
  \inferrule* [lab=structure] {} {\meaningof{0} = \{ P \in \pi | P \equiv 0 \}, \and \\ \meaningof{E_1 | E_2} = \{ P \in \pi | P \equiv P_{1} | P_{2}, P_{1} \in \meaningof{E_{1}}, P_{2} \in \meaningof{E_2}\} }
\end{mathpar}

\begin{mathpar}
 \inferrule* [lab=behavior] {} {\meaningof{\langle a?b \rangle E} = \{ P \in \pi | P \equiv Q | u?(y)P', \\ \and \\\\ \and \\ \;\;\; u \in \meaningof{a}, \forall z.P'\{z/y\} \in \meaningof{E\{z/b\}}\}, \and \\ \meaningof{a!E} = \{ P \in \pi | P \equiv Q | x!\langle P' \rangle, x \in \meaningof{a} P' \in \meaningof{E}\} }
\end{mathpar}

\begin{mathpar}
 \inferrule* [lab=nominal] {} {\meaningof{\quotep{E}} = \{ \quotep{P} \in \quotep{\pi} | P \in \meaningof{E} \}, \and \meaningof{\quotep{P}} = \{ \quotep{Q} \in \quotep{\pi} | P \equiv Q \} \and \\ \meaningof{@\quotep{E}} = \{ P \in \pi | P \equiv @x, x \in \meaningof{E} \}}
\end{mathpar}

\begin{eqnarray*}
  \\
  \meaningof{-} : TS \to ST
\end{eqnarray*}

\begin{eqnarray*}
  \\
  L : TS \to ST
\end{eqnarray*}

\begin{eqnarray*}
  \\
  P \models E \iff P \in \meaningof{E}
\end{eqnarray*}

\begin{eqnarray*}
  P \approx_{L} Q \iff \forall E \in L. P \models E \iff Q \models E
\end{eqnarray*}

\begin{eqnarray*}
  P \approx_{K} Q
\end{eqnarray*}

\begin{eqnarray*}
  P \approx Q
\end{eqnarray*}

$\approx_{K} = \approx = \approx_{L}$

\subsubsection{Contextual duality}

Note that contexts extend the quotation operation to a family of
operations from processes to names. Given a context, $M$, we can
define a \emph{nominal context}, $\quotep{M}$ by $\quotep{M}[P] :=
\quotep{M[P]}$. To foreshadow what is to come we observe that these
operations enjoy a duality with processes very much like the duality
between vectors and maps from vectors to scalars.

Further, because the calculus is essentially higher-order, we have a
correspondence between contexts and processes. More specifically,
given a name $x$ and a context $M$ we can construct $M^{*}_{x}$ such
that 

\begin{mathpar}
  M^{*}_{x} | \lift{x}{P} \red M[P]
\end{mathpar}

namely,

\begin{mathpar}
  M^{*}_{x} := x?(u).M[\dropn{u}]
\end{mathpar}

The dependence of $M^{*}_{x}$ on a name makes it an abstraction, 

\begin{mathpar}
  M^{*} := (x)x?(u).M[\dropn{u}]
\end{mathpar}

\subsection{Additional notation}

It will sometimes be convenient to denote the process a name
quotes. We already have the notation $x = \quotep{P}$, but it will be
convenient to introduce an alternate notation, $\procn{x}$, when we
want to emphasize the connection to the use of the name. Note that, by
virtue of name equivalence, $\quotep{\procn{x}} \nameeq x$; so, the
notation is consistent with previous definitions.

Further, because names have structure it is possible to effect
substitutions on the basis of that structure. This means we need to
upgrade our notation for substitutions, which we accomplish by
adapting comprehension notation. Thus,

\begin{mathpar}
  P\{ y / x : x \in S \}
\end{mathpar}

is interpreted to mean the process derived from P by replacing (in a
capture-avoiding manner) each occurrence of $x$ in $S$ by $y$. For example,

\begin{mathpar}
  P\{ \quotep{\procn{x}|\procn{x}} / x : x \in \freenames{P} \}
\end{mathpar}

will replace each (occurrence) of a free name $x$ in $P$ by
$\quotep{\procn{x}|\procn{x}}$.

Also, we will avail ourselves of the notation $x^{L}$ and $x^{R}$ to
denote injections of a name into disjoint copies of the name
space. There are numerous ways to accomplish this. One example can be
found in \cite{MeredithR05}. This notation overloads to vectors of
names: $\vec{x}^{\pi} := (x_{i}^{\pi} \; : \; 0 \leq i < |\vec{x}| )$ where $\pi \in \{L,R\}$.

We also use $P^{\Box} := P|\Box$.

In \cite{MeredithR05} an interpretation of the new operator is
given. It turns out that there are several possible interpretations
all enjoying the requisite algebraic properties of the operator (see
\cite{milner91polyadicpi}). We will therefore make liberal use of
$(\nu\; \vec{x})P$.

% subsection the_syntax_and_semantics_of_the_notation_system (end)   

\input{qm2pi.qmops} 

\input{qm2pi.sterngerlach} 

\input{qm2pi.metric} 

% section concurrent_process_calculi (end)

%\input{qm2pi.proofsketch}

% section proof sketch (end)

%\input{qm2pi.slviaknots} 

% section spatial logic via knots (end)

\input{qm2pi.conclusion}

% section conclusion (end)

%\input{qm2pi.dtcodes} 

% section wiring algorithm (end)

\input{qm2pi.ack} 

% section acknowledgments (end)

\newpage


\bibliographystyle{plain}   
\bibliography{../../biblios/main.bib}

\input{qm2pi.rhodetails}

\end{document}



% section proof sketch (end)

%\section{Unlikely characters: spatial logic for
  knots}\label{sub:characteristic_formulae} % (fold)

Associated to the mobile process calculi are a family of logics known
as the Hennessy-Milner logics. These logics typically enjoy a
semantics interpreting formulae as sets of processes that when
factored through the encoding outlined above allows an identification
of classes of knots with logical formulae. In the context of this
encoding the sub-family known as the spatial logics \cite{CairesC03}
\cite{CairesC04} \cite{Caires04} are of particular interest providing
several important features for expressing and reasoning about
properties (i.e. classes) of knots. We hint here at how this may be done.

%\begin{description}
%\item [structural connectives] 
\subsubsection{Structural connectives} The spatial logics enjoy
structural connectives corresponding, at the logical level, to the
parallel composition ($P | Q$) and new name ($(\nu \; x)P$)
connectives for processes. As illustrated in the examples below, these
connectives are extremely expressive given the shape of our encoding.
%\item [decideable satisfaction]

\subsubsection{Decideable satisfaction}
In \cite{Caires04} the satisfaction relation is shown to be decideable
for a rich class of processes. It further turns out that the image of
the our encoding is a proper subset of that class. This result
provides the basis for an algorithm by which to search for knots
enjoying a given property.
%\item [characteristic formulae]

\subsubsection{Characteristic formulae}
In the same paper \cite{Caires04} , Caires presents a means of calculating
characteristic formulae, selecting equivalence classes of processes
up to a pre--specified depth limit on the support set of names. Composed with our
encoding, this characteristic formula can be used to select
characteristic formulae for knots.
%\end{description}

\subsubsection{Spatial logic formulae}

The grammar below (segmented for comprehension) summarizes the syntax
of spatial logic formulae. We employ illustrative examples in the
sequel to provide an intuitive understanding of their meaning
referring the reader to \cite{Caires04} for a more detailed explication
of the semantics.

\begin{mathpar}
  \inferrule* [lab=boolean] {} {{A,B} \bc T \;|\; \neg A \;|\; A \wedge B \;|\; \eta = \eta'}
  \and
  \inferrule* [lab=spatial] {} {|\; \pzero \;|\; A | B \;|\; x \text{\textregistered} A \;|\; \forall x . A \;|\;  H x . A}
  \and
  \inferrule* [lab=behavioral] {} {|\; \alpha . A}
  \and 
  \inferrule* [lab=recursion] {} {|\; X(\vec{u}) \;|\; \mu X(\vec{u}) . A}
  \and
  \inferrule* [lab=action] {} {\alpha \bc \langle x?(\vec{y}) \rangle \;|\; \langle x!(\vec{y}) \rangle \;|\; \langle \tau \rangle}
  \and 
  \inferrule* [lab=name] {} {\eta \bc x \;|\; \tau}
\end{mathpar} 

% subsection characteristic_formulae (end)   	 

\subsection{Example formulae}\label{sub:example_formulae_} % (fold)

\subsubsection{Crossing as formula.}
% 
% \begin{align*}
%   \frac{d}{dx} \sin x &= \cos x 
%   & \frac{d}{dx} e^x &= e^x \\
%   \frac{d}{dx} \cos x &= - \sin x 
%   & \frac{d}{dx} \log x &= \frac{1}{x} \\
% \end{align*} 

\begin{align*}
 \mu C(x_{0},x_{1},y_{0},y_{1},u).&(\langle x_{0}?(z) \rangle(\langle u! \rangle\langle y_{1}!z \rangle C(x_{0},x_{1},y_{0},y_{1},u)) & \\
  & \wedge \langle y_{1}?(z) \rangle (\langle u! \rangle \langle x_{0}!z \rangle C(x_{0},x_{1},y_{0},y_{1},u)) & \\
  & \wedge \langle x_{1}?(z) \rangle (\langle u? \rangle \langle y_{0}!z \rangle C(x_{0},x_{1},y_{0},y_{1},u)) & \\
  & \wedge \langle y_{0}?(z) \rangle (\langle u? \rangle \langle x_{1}!z \rangle C(x_{0},x_{1},y_{0},y_{1},u))) &
\end{align*}

The lexicographical similarity between the shape of this formulae and
the shape of definition of the process representing a crossing reveals
the intuitive meaning of this formulae. It describes the capabilities
of a process that has the right to represent a crossing. For example
it picks out processes that may perform an input on the port $x_0$ in
its initial menu of capabilities. What differentiates the formula
from the process, however, is that the crossing process is the
smallest candidate to satisfy the formula. Infinitely many other
processes -- with internal behavior hidden behind this interface, so
to speak -- also satisfy this formula. Even this simple formula,
then, can be seen to open a new view onto knots, providing a
computational interpretation of \emph{virtual} knots.

Note that this formula is derived by hand. A similar formula can be
derived by employing Caires' calculation of characteristic formula
\cite{Caires04} to the process representing a crossing. In light of
this discussion, we let
$\meaningof{C}_{\phi}(x0,x1,y0,y1,u)$ denote a formula specifying the
dynamics we wish to capture of a crossing. To guarantee we preserve
the shape of the interface and minimal semantics we demand that
$\meaningof{C}_{\phi}(x0,x1,y0,y1,u) \Rightarrow
\textbf{C}(x0,x1,y0,y1,u)$ where $\textbf{C}(x0,x1,y0,y1,u)$ denotes
the formula above.
                            
\subsubsection{Crossing number constraints.}
The moral content of the context lemma (Lemma \ref{context}) is that the notion of
``locality'' in the Reidemeister moves is effectively captured by the
parallel composition operator of the process calculus. This intuition
extends through the logic. Given a formula,
$\meaningof{C}_{\phi}(x0,x1,y0,y1,u)$, we can use the structural
connectives to specify constraints on crossing numbers, such as at
least $n$ crossings, or exactly $n$ crossings.
\begin{mathpar}
  \inferrule* [lab=at-least-n] {} { K^{\geq n}_{\phi}(\vec{xs},\vec{ys}) := \Pi_{i=0}^{n-1} Hu . \meaningof{C}_{\phi}(xs_i,ys_i,u) | T }
  \and 
  \inferrule* [lab=exactly-n] {} { K^{= n}_{\phi}(\vec{xs},\vec{ys}) := \Pi_{i=0}^{n-1} Hu . \meaningof{C}_{\phi}(xs_i,ys_i,u) | \neg (\forall x_0,y_0,x_1,y_1,u . \meaningof{C}_{\phi}(x_0,y_0,x_1,y_1,u) | T) }
\end{mathpar}

To round out this section, recall that the encoding of an $n$-crossing
knot decomposes into a parallel composition of $n$ \emph{copies} of a
crossing process together with a wiring harness. To specify different
knot classes with the same crossing number amounts to specifying
logical constraints on the wiring harness. In the interest of space,
we defer examples to a forthcoming paper. Suffice it to say that both
the conditions ``alternating knot'' and ``contains the tangle
corresponding to 5/3'' are expressible. For example, it is possible to
calculate the characteristic formula of a process corresponding to the
tangle 5/3 and conjoin it into the classifying formula via the
composition connective of the logic.

Finally, we wish to observe that it is entirely within reason to
contemplate a more domain-specific version of spatial logic tailored
to the shape of processes in the image of the encoding. Such a
domain-specific logic would have a better claim to the title formal
language of knot properties.

% subsection example_formulae_ (end)

% section knots_as_processes (end) 

% section spatial logic via knots (end)

\section{Conclusions and future work}

\paragraph{Testing physical space}
You, gentle reader, may wonder why of all the theorems to be proved
given this set up we pick the one above. In some sense it's hardly
central to quantum mechanics. We see it as central in the sense that
it firmly establishes a notion of physical space arising from a notion
of the equivalence of behavior. Relating bisimulation to a metric is a
big step forward, but one is faced with interpreting the relationship
of that metric space to something more physical. Quantum mechanical
notions of ``physical'' space are still far from intuitive, but by
relating this idea of distance as testing to calculations that predict
physical circumstances we are making a not insignificant step forward
toward an understanding of the physical space we inhabit as
essentially dynamic.

\paragraph{Effectivity and simulation}
One of the observations we have yet to make is that the entire program
spelled out here is effective. We have built various interpreters for
the reflective calculus at work in this interpretation. In principle,
then, we can simulate quantum mechanics on a computer. The place where
the simulation may lose fidelity is the infinitely branching summation
for the annihilator.

In this connection i also want to point out that the evaluation style
calculation of the inner product puts the non-determinism of the
summation right at the heart of measurement. This suggests that
Milner's original reduction-based formulation of the dynamics of his
calculi in terms of sums was not just notationally suggestive of a
notion of measure-and-continue but captured some significant part of
the physics.

\paragraph{Quantum continuations}
In light of this last observation i want to point out that the
predominant account of quantum mechanics is missing a key aspect of a
truly compositional story of the physical situation. In a real lab,
when a measurement is made the observation can be made to feed into
another device that then makes another measurement conditioned on the
results of the first. This means that after the superposition was
collapsed the entire experimental set up remained in
superposition. While QM offers a means of writing this down it doesn't
quite line up well with the well-trodden formulation of computation
and continuation that we see so succinctly expressed in Milner's
calculi. This suggests that there might be advantages to this account
of dynamics waiting to be explored.

\paragraph{Quantum logic}
In this connection, we also note that by virtue of having the
Hennessy-Milner construction, we can pull the construction through the
interpretation of QM. This gives us a natural candidate for a quantum
logic that enjoys an extremely tight connection with it's domain of
interpretation, making the construction much less ad hoc (rather it is
the image of functor!).

\paragraph{Quantum probabiity}
i have questions about the basis of the interpretation of inner
product as probability amplitude. In particular, using which
axiomatization of probability theory does the notion of probability
amplitude earn the right to be so dubbed? In other words, where is the
proof that the operation for calculating a probability amplitude (and
then squaring) satisfies the axioms of what it means to calculate a
probability? Even if such a proof exists (i have yet to find it in the
literature), i wonder if it might not be possible to turn things on
their heads. Can we view the calculation of the probability amplitude
as an axiomatization of probability? If so, then the definition we
give for calculating probability amplitude may provide the basis for
an \emph{effective} theory of probability.

\paragraph{Quantum vs ``biological'' information}
Finally, i want to conclude with a more philosophical observation. At
a recent workshop in which QM was a predominant topic i noticed
something about quantum information. The speaker was giving a riveting
discussion of axiomatic QM and showing how properties of ``no
cloning'' and ``no deleting'' emerged as consequences of the
axiomatization. Theorems of this form are necessary to give us a sense
of confidence that our axioms characterize the physical theory. What
struck me, though, was that if quantum information is neither erasable
nor replicable it is markedly different from \emph{life}. Two of the
things we know about life is that

\begin{itemize}
  \item it ends;
  \item to gain some measure of persistence, to transcend it's
    finitude it is imminently copyable.
\end{itemize}

Both of these qualities are summarized succinctly in the aphorism: all
flesh is grass. For me these two kinds of ``information'' -- call them
quantum and biological -- are end points on a spectrum of strategies
for persistence. At one end, we have those curious entities that enjoy
uniqueness and permanence; at the other, we have those who in the face
of a certain end and an uncertain present make a go of passing
something on. To me one of the more remarkable aspects of the latter
strategy is that in the presence of noise (and certain features of
copying) we get a kind of dynamism, a chance for improvement against a
given persistent condition.

% subsection other_calculi_other_bisimulations_and_geometry_as_behavior (end)




% section conclusion (end)

%\documentclass[12pt]{llncs}
%\documentclass{jktr}

\usepackage[pdftex]{hyperref}                   
\usepackage {listings}
\usepackage {mathpartir}
\usepackage{bcprules}
%\usepackage{listings}
                       
\usepackage{graphicx} 
%\usepackage[margins=2.5cm,nohead,nofoot]{geometry}
%\usepackage{geometry}
\usepackage{amsfonts}
\usepackage{amstext}
\usepackage{latexsym}
\usepackage{amssymb}
\usepackage{color}


%\include{myPreamble}
\include{qm2pi.local} 

%\ifpdf
%\usepackage[pdftex]{graphicx}
%\else
%\usepackage{graphicx}
%\fi

 % \ifpdf
%  \usepackage{pdfsync}
%  \if


%\title{Brief Article}
%\author{David F. Snyder}
%\author{L.G. Meredith}

%\address{Dept. of Math., Texas State University--San Marcos, San Marcos, TX 78666}
       
\pagestyle{empty}


\begin{document}

\lstset{language=[Objective]Caml,frame=shadowbox}

\input{qm2pi.front}

% section front matter (end)

\input{qm2pi.intro} 
 
% section introduction (end)

% \input{qm2pi.knotations} 

% section notation (end)

\input{qm2pi.process.calculi} 

% section concurrent_process_calculi_and_spatial_logics_ (end)
    
%\input{qm2pi.knots2pi} 

%\input{qm2pi.trefoil} 

%\input{qm2pi.mainthm} 

% subsection basic_interpretation (end)

%\input{qm2pi.rho.presentation} 
\subsection{The syntax and semantics of the notation system}\label{sub:the_syntax_and_semantics_of_the_notation_system} % (fold)

We now summarize a technical presentation of the calculus that
embodies our theory of dynamics. The typical presentation of such a
calculus follows the style of giving generators and relations on
them. The grammar, below, describing term constructors, freely
generates the set of processes, $\Proc$. This set is then quotiented
by a relation known as structural congruence and it is over this set
that the notion of dynamics is expressed. This presentation is
essentially that of \cite{MeredithR05} with the addition of
polyadicity and summation. For readability we have relegated some of
the technical subtleties to an appendix.

\subsubsection{Process grammar}\label{subsub:process_grammar}

\begin{mathpar}
  \inferrule* [lab=synchronization] {} {{M} \bc \pzero \;|\; x?F \;|\; x!C }
  \and
  \inferrule* [lab=abstraction] {} {{F} \bc (x)P}
  \and
  \inferrule* [lab=concretion] {} {{C} \bc \langle Q \rangle}
  \and
  \inferrule* [lab=process] {} {{P,Q} \bc M \;| \;P|Q \;|\; @{x}}
  \and
  \inferrule* [lab=name] {} {{x} \bc \quotep{P}}
\end{mathpar} 

Note that $\vec{x}$ (resp. $\vec{P}$) denotes a vector of names
(resp. processes) of length $|\vec{x}|$ (resp. $|\vec{P}|$). We adopt
the following useful abbreviations.

\begin{mathpar}
   x?(\vec{y}).P := x.(\vec{y})P \and  x\clift{\vec{P}} := x.\clift{\vec{P}}
   \and x!(y) := \lift{x}{\dropn{y}}
   \and \Pi_{i=0}^{n-1}P_i := P_0 | \ldots | P_{n-1}
\end{mathpar}

\subsubsection{Structural congruence}

\paragraph{Free and bound names and alpha-equivalence.} At the
core of structural equivalence is alpha-equivalence which identifies
process that are the same up to a change of variable. Formally, we
recognize the distinction between free and bound names. The free names
of a process, $\freenames{P}$, may be calculated recursively as
follows:

\begin{mathpar}
\freenames{\pzero} := \emptyset
  \and \\
  \freenames{x?(y).P} := \{ x \} \cup (\freenames{P} \setminus \{ y \})
  \and 
  \freenames{x!\langle P \rangle} := \{ x \} \cup \{ P \} 
  \and \\
  \freenames{P|Q} := \freenames{P} \cup \freenames{Q}
  \and \\
  \freenames{@{x}} := \{ x \}
\end{mathpar}

$\pi$
$\quotep{\pi}$

$\freenames{-} : \pi \to \mathcal{P}(\quotep{\pi})$

\begin{eqnarray*}
  \freenames{\pzero} & := & \emptyset \\
  \freenames{x?(y).P} & := & \{ x \} \cup (\freenames{P} \setminus \{ y \}) \\
  \freenames{x!\langle P \rangle} & := & \{ x \} \cup \{ P \} \\
  \freenames{P|Q} & := & \freenames{P} \cup \freenames{Q} \\
  \freenames{\dropn{x}} & := & \{ x \}
\end{eqnarray*}

The bound names of a process, $\boundnames{P}$, are those names occurring in $P$
that are not free. For example, in $x?(y).0$, the name $x$ is free, while $y$ is bound.

\begin{mathpar}
  \inferrule* [lab=monoidal-laws] {} { P|Q \equiv Q|P \and P|0 \equiv P \and P|(Q|R) \equiv (P|Q)|R }
\end{mathpar}

\begin{mathpar}
  \inferrule* [lab=alpha-equivalence] {} { (x)P \equiv (y)P\{y/x\} \and y \not\in \freenames{P} }
\end{mathpar}

\begin{definition}
Then two processes, $P,Q$, are alpha-equivalent if $P = Q\{\vec{y}/\vec{x}\}$ for
some $\vec{x} \in \boundnames{Q},\vec{y} \in \boundnames{P}$, where $Q\{\vec{y}/\vec{x}\}$
denotes the capture-avoiding substitution of $\vec{y}$ for $\vec{x}$ in $Q$.
\end{definition}

\begin{definition}
  The {\em structural congruence} \cite{SangiorgiWalker} , $\equiv$,
  between processes is the least congruence containing
  alpha-equivalence, satisfying the abelian monoid laws
  (associativity, commutativity and $\pzero$ as identity) for parallel
  composition $|$ and for summation $+$.
\end{definition}

\subsection{Name equivalence}

We take name equivalence, written $\nameeq$, to be the smallest
equivalence relation generated by the following rules.

\begin{mathpar}
\inferrule*[lab=Quote-drop]
{ }
{ \quotep{@{x}} \nameeq x }

\inferrule*[lab=Struct-equiv]
{ P \scong Q }
{ \quotep{P} \nameeq \quotep{Q} }
\end{mathpar}

The astute reader will have noticed that the mutual recursion of names
and processes imposes a mutual recursion on alpha-equivalence and
structural equivalence via name-equivalence. Fortunately, all of this
works out pleasantly and we may calculate in the natural way, free of
concern. The reader interested in the details is referred to the
appendix \ref{appendix:rho_details}.

\subsection{Substitution}

We use $\Proc$ for the set of processes, $\QProc$ for the set of
names, and $\id{\{}\vec{y} / \vec{x} \id{\}}$ to denote partial maps,
$s : \QProc \rightarrow \QProc$. A map, $s$ lifts, uniquely, to a map
on process terms, $\widehat{s} : \Proc \rightarrow \Proc$ by the
following equations.

\begin{mathpar}
  (0) \psubstp{Q}{P} := 0 \\
  (R \juxtap S) \psubstp{Q}{P}
  :=    
  (R)\psubstp{Q}{P} \juxtap (S) \psubstp{Q}{P} \\
  (x?(y).R) \psubstp{Q}{P}    
  :=    
  (x)\substp{Q}{P} (z)\concat( (R \psubstn{z}{y}) \psubstp{Q}{P} ) \\
  (\lift{x}{R}) \psubstp{Q}{P}  
  :=
  \lift{(x)\substp{Q}{P}}{ R \psubstp{Q}{P} } \\
%   (\dropn{x})  \psubstp{Q}{P}       
%   := 
%   \left\{ 
%     \begin{array}{ccc} 
%       \dropn{\quotep{Q}} & & x \nameeq \quotep{P} \\
%       \dropn{x} & & otherwise \\
%     \end{array}
%   \right. 
  (\dropn{x})  \psubstp{Q}{P}       
  := 
  \left\{ 
    \begin{array}{ccc} 
      Q & & x \nameeq \quotep{P} \\
      \dropn{x} & & otherwise \\
    \end{array}
  \right.
\end{mathpar}
 

where

\begin{eqnarray}
  (x)\id{\{} \lpquote Q \rpquote / \lpquote P \rpquote \id{\}}            = 
  \left\{ 
    \begin{array}{ccc}
      \lpquote Q \rpquote & & x \nameeq \lpquote P \rpquote \\
      x & & otherwise \\
    \end{array}
  \right. \nonumber
\end{eqnarray}

and $z$ is chosen distinct from $\quotep{P}$, $\quotep{Q}$, the free
names in $Q$, and all the names in $R$. Our $\alpha$-equivalence will
be built in the standard way from this substitution.

\begin{remark}\label{rem:no_self_referential_names}
  One consequence of these definitions is that $\forall P. \quotep{P}
  \not\in \freenames{P}$.
\end{remark}

\subsection{ Dynamic quote: an example }

Anticipating something of what's to come, consider applying the
substitution, $\widehat{\id{\{}u / z \id{\}}}$, to the following pair
of processes, $\lift{w}{y!(z)}$ and $w[ \lpquote y!(z) \rpquote ]$.

\begin{eqnarray}
	\lift{w}{y!(z)}\widehat{\id{\{}u / z \id{\}}}
		& = &
		\lift{w}{y!(u)} \nonumber\\
	w[ \lpquote y!(z) \rpquote ] \widehat{ \id{\{}u / z \id{\}} }
		& = &
		w[ \lpquote y!(z) \rpquote ] \nonumber
\end{eqnarray}

Because the body of the process between quotes is impervious to
substitution, we get radically different answers. In fact, by
examining the first process in an input context,
e.g. $x?(z).\lift{w}{y!(z)}$, we see that the process under the lift
operator may be shaped by prefixed inputs binding a name inside it. In
this sense, the lift operator will be seen as a way to dynamically
construct processes before reifying them as names.

Finally equipped with these standard features we can present the
dynamics of the calculus.

\subsubsection{Operational semantics} 

Finally, we introduce the computational dynamics. What marks these
algebras as distinct from other more traditionally studied algebraic
structures, e.g. vector spaces or polynomial rings, is the manner in
which dynamics is captured. In traditional structures, dynamics is typically
expressed through morphisms between such structures, as in linear maps
between vector spaces or morphisms between rings. In algebras
associated with the semantics of computation, the dynamics is
expressed as part of the algebraic structure itself, through a
reduction reduction relation typically denoted by $\red$. Below, we
give a recursive presentation of this relation for the calculus used
in the encoding.

$\red \subseteq \pi \times \pi$
$\red : \pi \to \mathcal{P}(\pi)$

\begin{mathpar}
  \inferrule* [lab=Comm] { \textsf{match}( x_{src}, x_{trgt} ) } { x_{trgt}?(y)P \; | \; x_{src}!\langle {Q} \rangle \red P\{\quotep{Q}/y}\} }
  \and \\
  \inferrule* [lab=Par] {{P} \red {P}'} {{{P} | {Q}} \red {{P}' | {Q}}}
  \and
  \inferrule* [lab=Equiv]{{{P} \scong {P}'} \andalso {{P}' \red {Q}'} \andalso {{Q}' \scong {Q}}}{{P} \red {Q}}
\end{mathpar}

\begin{eqnarray*}
  match_{\equiv} (\quotep{P},\quotep{Q}) & := & P \equiv Q \\
  match_{\dagger}(\quotep{P},\quotep{Q}) & := & \forall R. P|Q \red^{*} R => R \red^{*} 0 \\
  match_{K}(\quotep{P},\quotep{Q}) & := & K \mbox{ for some context } K
\end{eqnarray*}

$u?(x)P | u!\langle Q \rangle \red P\{\quotep{Q}/x\}$

%We write $\wred$ for $\red^*$, and $P\red$ if $\exists Q $ such that $ P \red Q$.
We write $P\red$ if $\exists Q $ such that $ P \red Q$ and $P\not\red$, otherwise.

\section{Replication}

As mentioned before, it is known that replication (and hence
recursion) can be implemented in a higher-order process algebra
\cite{SangiorgiWalker}. As our first example of calculation with the
machinery thus far presented we give the construction explicitly in
the {\rhoc}.

\begin{eqnarray}
	D_{x} & := & \prefix{x}{y}{(\binpar{\outputp{x}{y}}{@{y}})} \nonumber\\
	\bangp_{x}{P} & := & \binpar{{x}!\langle{\binpar{D_{x}}{P}}\rangle}{D_{x}} \nonumber
\end{eqnarray}

\begin{eqnarray}
	\bangp_{x}{P} & & \nonumber\\
	=
	& {x}!\langle{(\prefix{x}{y}{(\outputp{x}{y} | @{y})) | P}}\rangle 
	      | \prefix{x}{y}{(\outputp{x}{y} | @{y})} & \nonumber\\
	\red
	& (\outputp{x}{y} | @{y})\substn{\quotep{(\prefix{x}{y}{(@{y} | \outputp{x}{y})) | P}}}{y} & \nonumber\\
	=
	& \outputp{x}{\quotep{(\prefix{x}{y}{(\outputp{x}{y} | @{y})) | P}}}
	  | {(\prefix{x}{y}{(\outputp{x}{y} | @{y})) | P}} & \nonumber\\
	\red
	& \ldots & \nonumber\\
	\red^*
	& P | P | \ldots & \nonumber
\end{eqnarray}

Of course, this encoding, as an implementation, runs away, unfolding
$\bangp{P}$ eagerly. A lazier and more implementable replication
operator, restricted to input-guarded processes, may be obtained as follows.

\begin{eqnarray}
\bangp{\prefix{u}{v}{P}} 
	:= 
	\binpar{\lift{x}{\prefix{u}{v}{(\binpar{D(x)}{P})}}}{D(x)} \nonumber
\end{eqnarray}

\begin{remark}
  Note that the lazier definition still does not deal with summation
  or mixed summation (i.e. sums over input and output). The reader is
  invited to construct definitions of replication that deal with these
  features. 

  Further, the definitions are parameterized in a name, $x$. Can you,
  gentle reader, make a definition that eliminates this parameter and
  guarantees no accidental interaction between the replication
  machinery and the process being replicated -- i.e. no accidental
  sharing of names used by the process to get its work done and the
  name(s) used by the replication to effect copying. This latter
  revision of the definition of replication is crucial to obtaining
  the expected identity $!!P \sim !P$.
\end{remark}

\begin{remark}\label{rem:paradoxical_combinator}
  The reader familiar with the lambda calculus will have noticed the
  similarity between $D$ and the paradoxical combinator.

  [Ed. note: the existence of this seems to suggest we have to be more
  restrictive on the set of processes and names we admit if we are to
  support no-cloning.]
\end{remark}

\subsubsection{Bisimulation}

The computational dynamics gives rise to another kind of equivalence,
the equivalence of computational behavior. As previously mentioned
this is typically captured \emph{via} some form of bisimulation.

% The notion we use in this paper is weak barbed bisimulation
% \cite{milner91polyadicpi}.

The notion we use in this paper is derived from weak barbed
bisimulation \cite{milner91polyadicpi}. 

\begin{definition}
An \emph{observation relation}, $\downarrow_{\mathcal N}$, over a set
of names, $\mathcal N$, is the smallest relation satisfying the rules
below.

\infrule[Out-barb]{y \in {\mathcal N}, \; x \nameeq y}
		  {\outputp{x}{v} \downarrow_{\mathcal N} x}
\infrule[Par-barb]{\mbox{$P\downarrow_{\mathcal N} x$ or $Q\downarrow_{\mathcal N} x$}}
		  {\binpar{P}{Q} \downarrow_{\mathcal N} x}

We write $P \Downarrow_{\mathcal N} x$ if there is $Q$ such that 
$P \wred Q$ and $Q \downarrow_{\mathcal N} x$.
\end{definition}

\begin{definition}
%\label{def.bbisim}
An  ${\mathcal N}$-\emph{barbed bisimulation} over a set of names, ${\mathcal N}$, is a symmetric binary relation 
${\mathcal S}_{\mathcal N}$ between agents such that $P\rel{S}_{\mathcal N}Q$ implies:
\begin{enumerate}
\item If $P \red P'$ then $Q \wred Q'$ and $P'\rel{S}_{\mathcal N} Q'$.
\item If $P\downarrow_{\mathcal N} x$, then $Q\Downarrow_{\mathcal N} x$.
\end{enumerate}
$P$ is ${\mathcal N}$-barbed bisimilar to $Q$, written
$P \wbbisim_{\mathcal N} Q$, if $P \rel{S}_{\mathcal N} Q$ for some ${\mathcal N}$-barbed bisimulation ${\mathcal S}_{\mathcal N}$.
\end{definition}

$\mathcal{R} \subseteq \pi \times \pi$

$P \mathcal{R} Q => \forall P'. P \red P' \Rightarrow \exists Q'. Q \red Q', P' \mathcal{R} Q'$

$P \vdash x \Rightarrow Q \vdash x$

\begin{mathpar}
  \inferrule*[lab=Out-barb]{x \nameeq y}{{y}!\langle{Q}\rangle \vdash x}
  \and
  \inferrule*[lab=Par-barb]{\mbox{$P\vdash x$ or $Q\vdash x$}}{\binpar{P}{Q} \vdash x}
\end{mathpar}

\subsubsection{Contexts}

One of the principle advantages of computational calculi like the
$\pi$-calculus is a well-defined notion of context,
contextual-equivalence and a correlation between
contextual-equivalence and notions of bisimulation. The notion of
context allows the decomposition of a process into (sub-)process and
its syntactic environment, its context. Thus, a context may be
thought of as a process with a ``hole'' (written $\Box$) in it. The
application of a context $M$ to a process $P$, written $M[P]$, is
tantamount to filling the hole in $M$ with $P$. In this paper we do
not need the full weight of this theory, but do make use of the notion
of context in the proof the main theorem. 

\begin{mathpar}
  \inferrule* [lab=summation] {} {{M_{M},M_{N}} \bc \Box \;|\; x.M_{A} \;|\; M_{M}+M_{N}}
  \and
  \inferrule* [lab=agent] {} {{M_{A}} \bc (\vec{x})M_{P} \;| \; \clift{P_0,\ldots,M_{P},\ldots,P_N}}
  \and \\
  \inferrule* [lab=process] {} {{M_{P}} \bc M_{N} \;| \;P|M_{P} }
\end{mathpar} 

\begin{mathpar}
  \inferrule* [lab=sychronization] {} {M_{N} \bc \Box \;|\; x?M_{F} \;|\; x!M_{C}}
  \and
  \inferrule* [lab=abstraction] {} {{M_{F}} \bc (x)M_{P} }
  \and
  \inferrule* [lab=concretion] {} {{M_{C}} \bc \langle M_{P} \rangle }
  \and \\
  \inferrule* [lab=process] {} {{M_{P}} \bc M_{N} \;| \;P|M_{P} }
\end{mathpar}

\begin{definition}[contextual application] Given a context $M$, and
  process $P$, we define the \emph{contextual application}, $M[P] :=
  M\{P/\Box\}$. That is, the contextual application of M to P is the
  substitution of $P$ for $\Box$ in $M$.
\end{definition}

$\meaningof{-} : L \to \mathcal{P}(\pi)$

\begin{mathpar}
  \inferrule* [lab=collection] {} {\meaningof{true} = \pi, \and \meaningof{~E} = \pi \setminus \meaningof{E}, \and \meaningof{E_{1} \& E_{2}} = \meaningof{E_{1}} \cap \meaningof{E_{2}}}
\end{mathpar}

\begin{mathpar}
  \inferrule* [lab=structure] {} {\meaningof{0} = \{ P \in \pi | P \equiv 0 \}, \and \\ \meaningof{E_1 | E_2} = \{ P \in \pi | P \equiv P_{1} | P_{2}, P_{1} \in \meaningof{E_{1}}, P_{2} \in \meaningof{E_2}\} }
\end{mathpar}

\begin{mathpar}
 \inferrule* [lab=behavior] {} {\meaningof{\langle a?b \rangle E} = \{ P \in \pi | P \equiv Q | u?(y)P', \\ \and \\\\ \and \\ \;\;\; u \in \meaningof{a}, \forall z.P'\{z/y\} \in \meaningof{E\{z/b\}}\}, \and \\ \meaningof{a!E} = \{ P \in \pi | P \equiv Q | x!\langle P' \rangle, x \in \meaningof{a} P' \in \meaningof{E}\} }
\end{mathpar}

\begin{mathpar}
 \inferrule* [lab=nominal] {} {\meaningof{\quotep{E}} = \{ \quotep{P} \in \quotep{\pi} | P \in \meaningof{E} \}, \and \meaningof{\quotep{P}} = \{ \quotep{Q} \in \quotep{\pi} | P \equiv Q \} \and \\ \meaningof{@\quotep{E}} = \{ P \in \pi | P \equiv @x, x \in \meaningof{E} \}}
\end{mathpar}

\begin{eqnarray*}
  \\
  \meaningof{-} : TS \to ST
\end{eqnarray*}

\begin{eqnarray*}
  \\
  L : TS \to ST
\end{eqnarray*}

\begin{eqnarray*}
  \\
  P \models E \iff P \in \meaningof{E}
\end{eqnarray*}

\begin{eqnarray*}
  P \approx_{L} Q \iff \forall E \in L. P \models E \iff Q \models E
\end{eqnarray*}

\begin{eqnarray*}
  P \approx_{K} Q
\end{eqnarray*}

\begin{eqnarray*}
  P \approx Q
\end{eqnarray*}

$\approx_{K} = \approx = \approx_{L}$

\subsubsection{Contextual duality}

Note that contexts extend the quotation operation to a family of
operations from processes to names. Given a context, $M$, we can
define a \emph{nominal context}, $\quotep{M}$ by $\quotep{M}[P] :=
\quotep{M[P]}$. To foreshadow what is to come we observe that these
operations enjoy a duality with processes very much like the duality
between vectors and maps from vectors to scalars.

Further, because the calculus is essentially higher-order, we have a
correspondence between contexts and processes. More specifically,
given a name $x$ and a context $M$ we can construct $M^{*}_{x}$ such
that 

\begin{mathpar}
  M^{*}_{x} | \lift{x}{P} \red M[P]
\end{mathpar}

namely,

\begin{mathpar}
  M^{*}_{x} := x?(u).M[\dropn{u}]
\end{mathpar}

The dependence of $M^{*}_{x}$ on a name makes it an abstraction, 

\begin{mathpar}
  M^{*} := (x)x?(u).M[\dropn{u}]
\end{mathpar}

\subsection{Additional notation}

It will sometimes be convenient to denote the process a name
quotes. We already have the notation $x = \quotep{P}$, but it will be
convenient to introduce an alternate notation, $\procn{x}$, when we
want to emphasize the connection to the use of the name. Note that, by
virtue of name equivalence, $\quotep{\procn{x}} \nameeq x$; so, the
notation is consistent with previous definitions.

Further, because names have structure it is possible to effect
substitutions on the basis of that structure. This means we need to
upgrade our notation for substitutions, which we accomplish by
adapting comprehension notation. Thus,

\begin{mathpar}
  P\{ y / x : x \in S \}
\end{mathpar}

is interpreted to mean the process derived from P by replacing (in a
capture-avoiding manner) each occurrence of $x$ in $S$ by $y$. For example,

\begin{mathpar}
  P\{ \quotep{\procn{x}|\procn{x}} / x : x \in \freenames{P} \}
\end{mathpar}

will replace each (occurrence) of a free name $x$ in $P$ by
$\quotep{\procn{x}|\procn{x}}$.

Also, we will avail ourselves of the notation $x^{L}$ and $x^{R}$ to
denote injections of a name into disjoint copies of the name
space. There are numerous ways to accomplish this. One example can be
found in \cite{MeredithR05}. This notation overloads to vectors of
names: $\vec{x}^{\pi} := (x_{i}^{\pi} \; : \; 0 \leq i < |\vec{x}| )$ where $\pi \in \{L,R\}$.

We also use $P^{\Box} := P|\Box$.

In \cite{MeredithR05} an interpretation of the new operator is
given. It turns out that there are several possible interpretations
all enjoying the requisite algebraic properties of the operator (see
\cite{milner91polyadicpi}). We will therefore make liberal use of
$(\nu\; \vec{x})P$.

% subsection the_syntax_and_semantics_of_the_notation_system (end)   

\input{qm2pi.qmops} 

\input{qm2pi.sterngerlach} 

\input{qm2pi.metric} 

% section concurrent_process_calculi (end)

%\input{qm2pi.proofsketch}

% section proof sketch (end)

%\input{qm2pi.slviaknots} 

% section spatial logic via knots (end)

\input{qm2pi.conclusion}

% section conclusion (end)

%\input{qm2pi.dtcodes} 

% section wiring algorithm (end)

\input{qm2pi.ack} 

% section acknowledgments (end)

\newpage


\bibliographystyle{plain}   
\bibliography{../../biblios/main.bib}

\input{qm2pi.rhodetails}

\end{document}

 

% section wiring algorithm (end)

\documentclass[12pt]{llncs}
%\documentclass{jktr}

\usepackage[pdftex]{hyperref}                   
\usepackage {listings}
\usepackage {mathpartir}
\usepackage{bcprules}
%\usepackage{listings}
                       
\usepackage{graphicx} 
%\usepackage[margins=2.5cm,nohead,nofoot]{geometry}
%\usepackage{geometry}
\usepackage{amsfonts}
\usepackage{amstext}
\usepackage{latexsym}
\usepackage{amssymb}
\usepackage{color}


%\include{myPreamble}
\include{qm2pi.local} 

%\ifpdf
%\usepackage[pdftex]{graphicx}
%\else
%\usepackage{graphicx}
%\fi

 % \ifpdf
%  \usepackage{pdfsync}
%  \if


%\title{Brief Article}
%\author{David F. Snyder}
%\author{L.G. Meredith}

%\address{Dept. of Math., Texas State University--San Marcos, San Marcos, TX 78666}
       
\pagestyle{empty}


\begin{document}

\lstset{language=[Objective]Caml,frame=shadowbox}

\input{qm2pi.front}

% section front matter (end)

\input{qm2pi.intro} 
 
% section introduction (end)

% \input{qm2pi.knotations} 

% section notation (end)

\input{qm2pi.process.calculi} 

% section concurrent_process_calculi_and_spatial_logics_ (end)
    
%\input{qm2pi.knots2pi} 

%\input{qm2pi.trefoil} 

%\input{qm2pi.mainthm} 

% subsection basic_interpretation (end)

%\input{qm2pi.rho.presentation} 
\subsection{The syntax and semantics of the notation system}\label{sub:the_syntax_and_semantics_of_the_notation_system} % (fold)

We now summarize a technical presentation of the calculus that
embodies our theory of dynamics. The typical presentation of such a
calculus follows the style of giving generators and relations on
them. The grammar, below, describing term constructors, freely
generates the set of processes, $\Proc$. This set is then quotiented
by a relation known as structural congruence and it is over this set
that the notion of dynamics is expressed. This presentation is
essentially that of \cite{MeredithR05} with the addition of
polyadicity and summation. For readability we have relegated some of
the technical subtleties to an appendix.

\subsubsection{Process grammar}\label{subsub:process_grammar}

\begin{mathpar}
  \inferrule* [lab=synchronization] {} {{M} \bc \pzero \;|\; x?F \;|\; x!C }
  \and
  \inferrule* [lab=abstraction] {} {{F} \bc (x)P}
  \and
  \inferrule* [lab=concretion] {} {{C} \bc \langle Q \rangle}
  \and
  \inferrule* [lab=process] {} {{P,Q} \bc M \;| \;P|Q \;|\; @{x}}
  \and
  \inferrule* [lab=name] {} {{x} \bc \quotep{P}}
\end{mathpar} 

Note that $\vec{x}$ (resp. $\vec{P}$) denotes a vector of names
(resp. processes) of length $|\vec{x}|$ (resp. $|\vec{P}|$). We adopt
the following useful abbreviations.

\begin{mathpar}
   x?(\vec{y}).P := x.(\vec{y})P \and  x\clift{\vec{P}} := x.\clift{\vec{P}}
   \and x!(y) := \lift{x}{\dropn{y}}
   \and \Pi_{i=0}^{n-1}P_i := P_0 | \ldots | P_{n-1}
\end{mathpar}

\subsubsection{Structural congruence}

\paragraph{Free and bound names and alpha-equivalence.} At the
core of structural equivalence is alpha-equivalence which identifies
process that are the same up to a change of variable. Formally, we
recognize the distinction between free and bound names. The free names
of a process, $\freenames{P}$, may be calculated recursively as
follows:

\begin{mathpar}
\freenames{\pzero} := \emptyset
  \and \\
  \freenames{x?(y).P} := \{ x \} \cup (\freenames{P} \setminus \{ y \})
  \and 
  \freenames{x!\langle P \rangle} := \{ x \} \cup \{ P \} 
  \and \\
  \freenames{P|Q} := \freenames{P} \cup \freenames{Q}
  \and \\
  \freenames{@{x}} := \{ x \}
\end{mathpar}

$\pi$
$\quotep{\pi}$

$\freenames{-} : \pi \to \mathcal{P}(\quotep{\pi})$

\begin{eqnarray*}
  \freenames{\pzero} & := & \emptyset \\
  \freenames{x?(y).P} & := & \{ x \} \cup (\freenames{P} \setminus \{ y \}) \\
  \freenames{x!\langle P \rangle} & := & \{ x \} \cup \{ P \} \\
  \freenames{P|Q} & := & \freenames{P} \cup \freenames{Q} \\
  \freenames{\dropn{x}} & := & \{ x \}
\end{eqnarray*}

The bound names of a process, $\boundnames{P}$, are those names occurring in $P$
that are not free. For example, in $x?(y).0$, the name $x$ is free, while $y$ is bound.

\begin{mathpar}
  \inferrule* [lab=monoidal-laws] {} { P|Q \equiv Q|P \and P|0 \equiv P \and P|(Q|R) \equiv (P|Q)|R }
\end{mathpar}

\begin{mathpar}
  \inferrule* [lab=alpha-equivalence] {} { (x)P \equiv (y)P\{y/x\} \and y \not\in \freenames{P} }
\end{mathpar}

\begin{definition}
Then two processes, $P,Q$, are alpha-equivalent if $P = Q\{\vec{y}/\vec{x}\}$ for
some $\vec{x} \in \boundnames{Q},\vec{y} \in \boundnames{P}$, where $Q\{\vec{y}/\vec{x}\}$
denotes the capture-avoiding substitution of $\vec{y}$ for $\vec{x}$ in $Q$.
\end{definition}

\begin{definition}
  The {\em structural congruence} \cite{SangiorgiWalker} , $\equiv$,
  between processes is the least congruence containing
  alpha-equivalence, satisfying the abelian monoid laws
  (associativity, commutativity and $\pzero$ as identity) for parallel
  composition $|$ and for summation $+$.
\end{definition}

\subsection{Name equivalence}

We take name equivalence, written $\nameeq$, to be the smallest
equivalence relation generated by the following rules.

\begin{mathpar}
\inferrule*[lab=Quote-drop]
{ }
{ \quotep{@{x}} \nameeq x }

\inferrule*[lab=Struct-equiv]
{ P \scong Q }
{ \quotep{P} \nameeq \quotep{Q} }
\end{mathpar}

The astute reader will have noticed that the mutual recursion of names
and processes imposes a mutual recursion on alpha-equivalence and
structural equivalence via name-equivalence. Fortunately, all of this
works out pleasantly and we may calculate in the natural way, free of
concern. The reader interested in the details is referred to the
appendix \ref{appendix:rho_details}.

\subsection{Substitution}

We use $\Proc$ for the set of processes, $\QProc$ for the set of
names, and $\id{\{}\vec{y} / \vec{x} \id{\}}$ to denote partial maps,
$s : \QProc \rightarrow \QProc$. A map, $s$ lifts, uniquely, to a map
on process terms, $\widehat{s} : \Proc \rightarrow \Proc$ by the
following equations.

\begin{mathpar}
  (0) \psubstp{Q}{P} := 0 \\
  (R \juxtap S) \psubstp{Q}{P}
  :=    
  (R)\psubstp{Q}{P} \juxtap (S) \psubstp{Q}{P} \\
  (x?(y).R) \psubstp{Q}{P}    
  :=    
  (x)\substp{Q}{P} (z)\concat( (R \psubstn{z}{y}) \psubstp{Q}{P} ) \\
  (\lift{x}{R}) \psubstp{Q}{P}  
  :=
  \lift{(x)\substp{Q}{P}}{ R \psubstp{Q}{P} } \\
%   (\dropn{x})  \psubstp{Q}{P}       
%   := 
%   \left\{ 
%     \begin{array}{ccc} 
%       \dropn{\quotep{Q}} & & x \nameeq \quotep{P} \\
%       \dropn{x} & & otherwise \\
%     \end{array}
%   \right. 
  (\dropn{x})  \psubstp{Q}{P}       
  := 
  \left\{ 
    \begin{array}{ccc} 
      Q & & x \nameeq \quotep{P} \\
      \dropn{x} & & otherwise \\
    \end{array}
  \right.
\end{mathpar}
 

where

\begin{eqnarray}
  (x)\id{\{} \lpquote Q \rpquote / \lpquote P \rpquote \id{\}}            = 
  \left\{ 
    \begin{array}{ccc}
      \lpquote Q \rpquote & & x \nameeq \lpquote P \rpquote \\
      x & & otherwise \\
    \end{array}
  \right. \nonumber
\end{eqnarray}

and $z$ is chosen distinct from $\quotep{P}$, $\quotep{Q}$, the free
names in $Q$, and all the names in $R$. Our $\alpha$-equivalence will
be built in the standard way from this substitution.

\begin{remark}\label{rem:no_self_referential_names}
  One consequence of these definitions is that $\forall P. \quotep{P}
  \not\in \freenames{P}$.
\end{remark}

\subsection{ Dynamic quote: an example }

Anticipating something of what's to come, consider applying the
substitution, $\widehat{\id{\{}u / z \id{\}}}$, to the following pair
of processes, $\lift{w}{y!(z)}$ and $w[ \lpquote y!(z) \rpquote ]$.

\begin{eqnarray}
	\lift{w}{y!(z)}\widehat{\id{\{}u / z \id{\}}}
		& = &
		\lift{w}{y!(u)} \nonumber\\
	w[ \lpquote y!(z) \rpquote ] \widehat{ \id{\{}u / z \id{\}} }
		& = &
		w[ \lpquote y!(z) \rpquote ] \nonumber
\end{eqnarray}

Because the body of the process between quotes is impervious to
substitution, we get radically different answers. In fact, by
examining the first process in an input context,
e.g. $x?(z).\lift{w}{y!(z)}$, we see that the process under the lift
operator may be shaped by prefixed inputs binding a name inside it. In
this sense, the lift operator will be seen as a way to dynamically
construct processes before reifying them as names.

Finally equipped with these standard features we can present the
dynamics of the calculus.

\subsubsection{Operational semantics} 

Finally, we introduce the computational dynamics. What marks these
algebras as distinct from other more traditionally studied algebraic
structures, e.g. vector spaces or polynomial rings, is the manner in
which dynamics is captured. In traditional structures, dynamics is typically
expressed through morphisms between such structures, as in linear maps
between vector spaces or morphisms between rings. In algebras
associated with the semantics of computation, the dynamics is
expressed as part of the algebraic structure itself, through a
reduction reduction relation typically denoted by $\red$. Below, we
give a recursive presentation of this relation for the calculus used
in the encoding.

$\red \subseteq \pi \times \pi$
$\red : \pi \to \mathcal{P}(\pi)$

\begin{mathpar}
  \inferrule* [lab=Comm] { \textsf{match}( x_{src}, x_{trgt} ) } { x_{trgt}?(y)P \; | \; x_{src}!\langle {Q} \rangle \red P\{\quotep{Q}/y}\} }
  \and \\
  \inferrule* [lab=Par] {{P} \red {P}'} {{{P} | {Q}} \red {{P}' | {Q}}}
  \and
  \inferrule* [lab=Equiv]{{{P} \scong {P}'} \andalso {{P}' \red {Q}'} \andalso {{Q}' \scong {Q}}}{{P} \red {Q}}
\end{mathpar}

\begin{eqnarray*}
  match_{\equiv} (\quotep{P},\quotep{Q}) & := & P \equiv Q \\
  match_{\dagger}(\quotep{P},\quotep{Q}) & := & \forall R. P|Q \red^{*} R => R \red^{*} 0 \\
  match_{K}(\quotep{P},\quotep{Q}) & := & K \mbox{ for some context } K
\end{eqnarray*}

$u?(x)P | u!\langle Q \rangle \red P\{\quotep{Q}/x\}$

%We write $\wred$ for $\red^*$, and $P\red$ if $\exists Q $ such that $ P \red Q$.
We write $P\red$ if $\exists Q $ such that $ P \red Q$ and $P\not\red$, otherwise.

\section{Replication}

As mentioned before, it is known that replication (and hence
recursion) can be implemented in a higher-order process algebra
\cite{SangiorgiWalker}. As our first example of calculation with the
machinery thus far presented we give the construction explicitly in
the {\rhoc}.

\begin{eqnarray}
	D_{x} & := & \prefix{x}{y}{(\binpar{\outputp{x}{y}}{@{y}})} \nonumber\\
	\bangp_{x}{P} & := & \binpar{{x}!\langle{\binpar{D_{x}}{P}}\rangle}{D_{x}} \nonumber
\end{eqnarray}

\begin{eqnarray}
	\bangp_{x}{P} & & \nonumber\\
	=
	& {x}!\langle{(\prefix{x}{y}{(\outputp{x}{y} | @{y})) | P}}\rangle 
	      | \prefix{x}{y}{(\outputp{x}{y} | @{y})} & \nonumber\\
	\red
	& (\outputp{x}{y} | @{y})\substn{\quotep{(\prefix{x}{y}{(@{y} | \outputp{x}{y})) | P}}}{y} & \nonumber\\
	=
	& \outputp{x}{\quotep{(\prefix{x}{y}{(\outputp{x}{y} | @{y})) | P}}}
	  | {(\prefix{x}{y}{(\outputp{x}{y} | @{y})) | P}} & \nonumber\\
	\red
	& \ldots & \nonumber\\
	\red^*
	& P | P | \ldots & \nonumber
\end{eqnarray}

Of course, this encoding, as an implementation, runs away, unfolding
$\bangp{P}$ eagerly. A lazier and more implementable replication
operator, restricted to input-guarded processes, may be obtained as follows.

\begin{eqnarray}
\bangp{\prefix{u}{v}{P}} 
	:= 
	\binpar{\lift{x}{\prefix{u}{v}{(\binpar{D(x)}{P})}}}{D(x)} \nonumber
\end{eqnarray}

\begin{remark}
  Note that the lazier definition still does not deal with summation
  or mixed summation (i.e. sums over input and output). The reader is
  invited to construct definitions of replication that deal with these
  features. 

  Further, the definitions are parameterized in a name, $x$. Can you,
  gentle reader, make a definition that eliminates this parameter and
  guarantees no accidental interaction between the replication
  machinery and the process being replicated -- i.e. no accidental
  sharing of names used by the process to get its work done and the
  name(s) used by the replication to effect copying. This latter
  revision of the definition of replication is crucial to obtaining
  the expected identity $!!P \sim !P$.
\end{remark}

\begin{remark}\label{rem:paradoxical_combinator}
  The reader familiar with the lambda calculus will have noticed the
  similarity between $D$ and the paradoxical combinator.

  [Ed. note: the existence of this seems to suggest we have to be more
  restrictive on the set of processes and names we admit if we are to
  support no-cloning.]
\end{remark}

\subsubsection{Bisimulation}

The computational dynamics gives rise to another kind of equivalence,
the equivalence of computational behavior. As previously mentioned
this is typically captured \emph{via} some form of bisimulation.

% The notion we use in this paper is weak barbed bisimulation
% \cite{milner91polyadicpi}.

The notion we use in this paper is derived from weak barbed
bisimulation \cite{milner91polyadicpi}. 

\begin{definition}
An \emph{observation relation}, $\downarrow_{\mathcal N}$, over a set
of names, $\mathcal N$, is the smallest relation satisfying the rules
below.

\infrule[Out-barb]{y \in {\mathcal N}, \; x \nameeq y}
		  {\outputp{x}{v} \downarrow_{\mathcal N} x}
\infrule[Par-barb]{\mbox{$P\downarrow_{\mathcal N} x$ or $Q\downarrow_{\mathcal N} x$}}
		  {\binpar{P}{Q} \downarrow_{\mathcal N} x}

We write $P \Downarrow_{\mathcal N} x$ if there is $Q$ such that 
$P \wred Q$ and $Q \downarrow_{\mathcal N} x$.
\end{definition}

\begin{definition}
%\label{def.bbisim}
An  ${\mathcal N}$-\emph{barbed bisimulation} over a set of names, ${\mathcal N}$, is a symmetric binary relation 
${\mathcal S}_{\mathcal N}$ between agents such that $P\rel{S}_{\mathcal N}Q$ implies:
\begin{enumerate}
\item If $P \red P'$ then $Q \wred Q'$ and $P'\rel{S}_{\mathcal N} Q'$.
\item If $P\downarrow_{\mathcal N} x$, then $Q\Downarrow_{\mathcal N} x$.
\end{enumerate}
$P$ is ${\mathcal N}$-barbed bisimilar to $Q$, written
$P \wbbisim_{\mathcal N} Q$, if $P \rel{S}_{\mathcal N} Q$ for some ${\mathcal N}$-barbed bisimulation ${\mathcal S}_{\mathcal N}$.
\end{definition}

$\mathcal{R} \subseteq \pi \times \pi$

$P \mathcal{R} Q => \forall P'. P \red P' \Rightarrow \exists Q'. Q \red Q', P' \mathcal{R} Q'$

$P \vdash x \Rightarrow Q \vdash x$

\begin{mathpar}
  \inferrule*[lab=Out-barb]{x \nameeq y}{{y}!\langle{Q}\rangle \vdash x}
  \and
  \inferrule*[lab=Par-barb]{\mbox{$P\vdash x$ or $Q\vdash x$}}{\binpar{P}{Q} \vdash x}
\end{mathpar}

\subsubsection{Contexts}

One of the principle advantages of computational calculi like the
$\pi$-calculus is a well-defined notion of context,
contextual-equivalence and a correlation between
contextual-equivalence and notions of bisimulation. The notion of
context allows the decomposition of a process into (sub-)process and
its syntactic environment, its context. Thus, a context may be
thought of as a process with a ``hole'' (written $\Box$) in it. The
application of a context $M$ to a process $P$, written $M[P]$, is
tantamount to filling the hole in $M$ with $P$. In this paper we do
not need the full weight of this theory, but do make use of the notion
of context in the proof the main theorem. 

\begin{mathpar}
  \inferrule* [lab=summation] {} {{M_{M},M_{N}} \bc \Box \;|\; x.M_{A} \;|\; M_{M}+M_{N}}
  \and
  \inferrule* [lab=agent] {} {{M_{A}} \bc (\vec{x})M_{P} \;| \; \clift{P_0,\ldots,M_{P},\ldots,P_N}}
  \and \\
  \inferrule* [lab=process] {} {{M_{P}} \bc M_{N} \;| \;P|M_{P} }
\end{mathpar} 

\begin{mathpar}
  \inferrule* [lab=sychronization] {} {M_{N} \bc \Box \;|\; x?M_{F} \;|\; x!M_{C}}
  \and
  \inferrule* [lab=abstraction] {} {{M_{F}} \bc (x)M_{P} }
  \and
  \inferrule* [lab=concretion] {} {{M_{C}} \bc \langle M_{P} \rangle }
  \and \\
  \inferrule* [lab=process] {} {{M_{P}} \bc M_{N} \;| \;P|M_{P} }
\end{mathpar}

\begin{definition}[contextual application] Given a context $M$, and
  process $P$, we define the \emph{contextual application}, $M[P] :=
  M\{P/\Box\}$. That is, the contextual application of M to P is the
  substitution of $P$ for $\Box$ in $M$.
\end{definition}

$\meaningof{-} : L \to \mathcal{P}(\pi)$

\begin{mathpar}
  \inferrule* [lab=collection] {} {\meaningof{true} = \pi, \and \meaningof{~E} = \pi \setminus \meaningof{E}, \and \meaningof{E_{1} \& E_{2}} = \meaningof{E_{1}} \cap \meaningof{E_{2}}}
\end{mathpar}

\begin{mathpar}
  \inferrule* [lab=structure] {} {\meaningof{0} = \{ P \in \pi | P \equiv 0 \}, \and \\ \meaningof{E_1 | E_2} = \{ P \in \pi | P \equiv P_{1} | P_{2}, P_{1} \in \meaningof{E_{1}}, P_{2} \in \meaningof{E_2}\} }
\end{mathpar}

\begin{mathpar}
 \inferrule* [lab=behavior] {} {\meaningof{\langle a?b \rangle E} = \{ P \in \pi | P \equiv Q | u?(y)P', \\ \and \\\\ \and \\ \;\;\; u \in \meaningof{a}, \forall z.P'\{z/y\} \in \meaningof{E\{z/b\}}\}, \and \\ \meaningof{a!E} = \{ P \in \pi | P \equiv Q | x!\langle P' \rangle, x \in \meaningof{a} P' \in \meaningof{E}\} }
\end{mathpar}

\begin{mathpar}
 \inferrule* [lab=nominal] {} {\meaningof{\quotep{E}} = \{ \quotep{P} \in \quotep{\pi} | P \in \meaningof{E} \}, \and \meaningof{\quotep{P}} = \{ \quotep{Q} \in \quotep{\pi} | P \equiv Q \} \and \\ \meaningof{@\quotep{E}} = \{ P \in \pi | P \equiv @x, x \in \meaningof{E} \}}
\end{mathpar}

\begin{eqnarray*}
  \\
  \meaningof{-} : TS \to ST
\end{eqnarray*}

\begin{eqnarray*}
  \\
  L : TS \to ST
\end{eqnarray*}

\begin{eqnarray*}
  \\
  P \models E \iff P \in \meaningof{E}
\end{eqnarray*}

\begin{eqnarray*}
  P \approx_{L} Q \iff \forall E \in L. P \models E \iff Q \models E
\end{eqnarray*}

\begin{eqnarray*}
  P \approx_{K} Q
\end{eqnarray*}

\begin{eqnarray*}
  P \approx Q
\end{eqnarray*}

$\approx_{K} = \approx = \approx_{L}$

\subsubsection{Contextual duality}

Note that contexts extend the quotation operation to a family of
operations from processes to names. Given a context, $M$, we can
define a \emph{nominal context}, $\quotep{M}$ by $\quotep{M}[P] :=
\quotep{M[P]}$. To foreshadow what is to come we observe that these
operations enjoy a duality with processes very much like the duality
between vectors and maps from vectors to scalars.

Further, because the calculus is essentially higher-order, we have a
correspondence between contexts and processes. More specifically,
given a name $x$ and a context $M$ we can construct $M^{*}_{x}$ such
that 

\begin{mathpar}
  M^{*}_{x} | \lift{x}{P} \red M[P]
\end{mathpar}

namely,

\begin{mathpar}
  M^{*}_{x} := x?(u).M[\dropn{u}]
\end{mathpar}

The dependence of $M^{*}_{x}$ on a name makes it an abstraction, 

\begin{mathpar}
  M^{*} := (x)x?(u).M[\dropn{u}]
\end{mathpar}

\subsection{Additional notation}

It will sometimes be convenient to denote the process a name
quotes. We already have the notation $x = \quotep{P}$, but it will be
convenient to introduce an alternate notation, $\procn{x}$, when we
want to emphasize the connection to the use of the name. Note that, by
virtue of name equivalence, $\quotep{\procn{x}} \nameeq x$; so, the
notation is consistent with previous definitions.

Further, because names have structure it is possible to effect
substitutions on the basis of that structure. This means we need to
upgrade our notation for substitutions, which we accomplish by
adapting comprehension notation. Thus,

\begin{mathpar}
  P\{ y / x : x \in S \}
\end{mathpar}

is interpreted to mean the process derived from P by replacing (in a
capture-avoiding manner) each occurrence of $x$ in $S$ by $y$. For example,

\begin{mathpar}
  P\{ \quotep{\procn{x}|\procn{x}} / x : x \in \freenames{P} \}
\end{mathpar}

will replace each (occurrence) of a free name $x$ in $P$ by
$\quotep{\procn{x}|\procn{x}}$.

Also, we will avail ourselves of the notation $x^{L}$ and $x^{R}$ to
denote injections of a name into disjoint copies of the name
space. There are numerous ways to accomplish this. One example can be
found in \cite{MeredithR05}. This notation overloads to vectors of
names: $\vec{x}^{\pi} := (x_{i}^{\pi} \; : \; 0 \leq i < |\vec{x}| )$ where $\pi \in \{L,R\}$.

We also use $P^{\Box} := P|\Box$.

In \cite{MeredithR05} an interpretation of the new operator is
given. It turns out that there are several possible interpretations
all enjoying the requisite algebraic properties of the operator (see
\cite{milner91polyadicpi}). We will therefore make liberal use of
$(\nu\; \vec{x})P$.

% subsection the_syntax_and_semantics_of_the_notation_system (end)   

\input{qm2pi.qmops} 

\input{qm2pi.sterngerlach} 

\input{qm2pi.metric} 

% section concurrent_process_calculi (end)

%\input{qm2pi.proofsketch}

% section proof sketch (end)

%\input{qm2pi.slviaknots} 

% section spatial logic via knots (end)

\input{qm2pi.conclusion}

% section conclusion (end)

%\input{qm2pi.dtcodes} 

% section wiring algorithm (end)

\input{qm2pi.ack} 

% section acknowledgments (end)

\newpage


\bibliographystyle{plain}   
\bibliography{../../biblios/main.bib}

\input{qm2pi.rhodetails}

\end{document}

 

% section acknowledgments (end)

\newpage


\bibliographystyle{plain}   
\bibliography{../../biblios/main.bib}

\documentclass[12pt]{llncs}
%\documentclass{jktr}

\usepackage[pdftex]{hyperref}                   
\usepackage {listings}
\usepackage {mathpartir}
\usepackage{bcprules}
%\usepackage{listings}
                       
\usepackage{graphicx} 
%\usepackage[margins=2.5cm,nohead,nofoot]{geometry}
%\usepackage{geometry}
\usepackage{amsfonts}
\usepackage{amstext}
\usepackage{latexsym}
\usepackage{amssymb}
\usepackage{color}


%\include{myPreamble}
\include{qm2pi.local} 

%\ifpdf
%\usepackage[pdftex]{graphicx}
%\else
%\usepackage{graphicx}
%\fi

 % \ifpdf
%  \usepackage{pdfsync}
%  \if


%\title{Brief Article}
%\author{David F. Snyder}
%\author{L.G. Meredith}

%\address{Dept. of Math., Texas State University--San Marcos, San Marcos, TX 78666}
       
\pagestyle{empty}


\begin{document}

\lstset{language=[Objective]Caml,frame=shadowbox}

\input{qm2pi.front}

% section front matter (end)

\input{qm2pi.intro} 
 
% section introduction (end)

% \input{qm2pi.knotations} 

% section notation (end)

\input{qm2pi.process.calculi} 

% section concurrent_process_calculi_and_spatial_logics_ (end)
    
%\input{qm2pi.knots2pi} 

%\input{qm2pi.trefoil} 

%\input{qm2pi.mainthm} 

% subsection basic_interpretation (end)

%\input{qm2pi.rho.presentation} 
\subsection{The syntax and semantics of the notation system}\label{sub:the_syntax_and_semantics_of_the_notation_system} % (fold)

We now summarize a technical presentation of the calculus that
embodies our theory of dynamics. The typical presentation of such a
calculus follows the style of giving generators and relations on
them. The grammar, below, describing term constructors, freely
generates the set of processes, $\Proc$. This set is then quotiented
by a relation known as structural congruence and it is over this set
that the notion of dynamics is expressed. This presentation is
essentially that of \cite{MeredithR05} with the addition of
polyadicity and summation. For readability we have relegated some of
the technical subtleties to an appendix.

\subsubsection{Process grammar}\label{subsub:process_grammar}

\begin{mathpar}
  \inferrule* [lab=synchronization] {} {{M} \bc \pzero \;|\; x?F \;|\; x!C }
  \and
  \inferrule* [lab=abstraction] {} {{F} \bc (x)P}
  \and
  \inferrule* [lab=concretion] {} {{C} \bc \langle Q \rangle}
  \and
  \inferrule* [lab=process] {} {{P,Q} \bc M \;| \;P|Q \;|\; @{x}}
  \and
  \inferrule* [lab=name] {} {{x} \bc \quotep{P}}
\end{mathpar} 

Note that $\vec{x}$ (resp. $\vec{P}$) denotes a vector of names
(resp. processes) of length $|\vec{x}|$ (resp. $|\vec{P}|$). We adopt
the following useful abbreviations.

\begin{mathpar}
   x?(\vec{y}).P := x.(\vec{y})P \and  x\clift{\vec{P}} := x.\clift{\vec{P}}
   \and x!(y) := \lift{x}{\dropn{y}}
   \and \Pi_{i=0}^{n-1}P_i := P_0 | \ldots | P_{n-1}
\end{mathpar}

\subsubsection{Structural congruence}

\paragraph{Free and bound names and alpha-equivalence.} At the
core of structural equivalence is alpha-equivalence which identifies
process that are the same up to a change of variable. Formally, we
recognize the distinction between free and bound names. The free names
of a process, $\freenames{P}$, may be calculated recursively as
follows:

\begin{mathpar}
\freenames{\pzero} := \emptyset
  \and \\
  \freenames{x?(y).P} := \{ x \} \cup (\freenames{P} \setminus \{ y \})
  \and 
  \freenames{x!\langle P \rangle} := \{ x \} \cup \{ P \} 
  \and \\
  \freenames{P|Q} := \freenames{P} \cup \freenames{Q}
  \and \\
  \freenames{@{x}} := \{ x \}
\end{mathpar}

$\pi$
$\quotep{\pi}$

$\freenames{-} : \pi \to \mathcal{P}(\quotep{\pi})$

\begin{eqnarray*}
  \freenames{\pzero} & := & \emptyset \\
  \freenames{x?(y).P} & := & \{ x \} \cup (\freenames{P} \setminus \{ y \}) \\
  \freenames{x!\langle P \rangle} & := & \{ x \} \cup \{ P \} \\
  \freenames{P|Q} & := & \freenames{P} \cup \freenames{Q} \\
  \freenames{\dropn{x}} & := & \{ x \}
\end{eqnarray*}

The bound names of a process, $\boundnames{P}$, are those names occurring in $P$
that are not free. For example, in $x?(y).0$, the name $x$ is free, while $y$ is bound.

\begin{mathpar}
  \inferrule* [lab=monoidal-laws] {} { P|Q \equiv Q|P \and P|0 \equiv P \and P|(Q|R) \equiv (P|Q)|R }
\end{mathpar}

\begin{mathpar}
  \inferrule* [lab=alpha-equivalence] {} { (x)P \equiv (y)P\{y/x\} \and y \not\in \freenames{P} }
\end{mathpar}

\begin{definition}
Then two processes, $P,Q$, are alpha-equivalent if $P = Q\{\vec{y}/\vec{x}\}$ for
some $\vec{x} \in \boundnames{Q},\vec{y} \in \boundnames{P}$, where $Q\{\vec{y}/\vec{x}\}$
denotes the capture-avoiding substitution of $\vec{y}$ for $\vec{x}$ in $Q$.
\end{definition}

\begin{definition}
  The {\em structural congruence} \cite{SangiorgiWalker} , $\equiv$,
  between processes is the least congruence containing
  alpha-equivalence, satisfying the abelian monoid laws
  (associativity, commutativity and $\pzero$ as identity) for parallel
  composition $|$ and for summation $+$.
\end{definition}

\subsection{Name equivalence}

We take name equivalence, written $\nameeq$, to be the smallest
equivalence relation generated by the following rules.

\begin{mathpar}
\inferrule*[lab=Quote-drop]
{ }
{ \quotep{@{x}} \nameeq x }

\inferrule*[lab=Struct-equiv]
{ P \scong Q }
{ \quotep{P} \nameeq \quotep{Q} }
\end{mathpar}

The astute reader will have noticed that the mutual recursion of names
and processes imposes a mutual recursion on alpha-equivalence and
structural equivalence via name-equivalence. Fortunately, all of this
works out pleasantly and we may calculate in the natural way, free of
concern. The reader interested in the details is referred to the
appendix \ref{appendix:rho_details}.

\subsection{Substitution}

We use $\Proc$ for the set of processes, $\QProc$ for the set of
names, and $\id{\{}\vec{y} / \vec{x} \id{\}}$ to denote partial maps,
$s : \QProc \rightarrow \QProc$. A map, $s$ lifts, uniquely, to a map
on process terms, $\widehat{s} : \Proc \rightarrow \Proc$ by the
following equations.

\begin{mathpar}
  (0) \psubstp{Q}{P} := 0 \\
  (R \juxtap S) \psubstp{Q}{P}
  :=    
  (R)\psubstp{Q}{P} \juxtap (S) \psubstp{Q}{P} \\
  (x?(y).R) \psubstp{Q}{P}    
  :=    
  (x)\substp{Q}{P} (z)\concat( (R \psubstn{z}{y}) \psubstp{Q}{P} ) \\
  (\lift{x}{R}) \psubstp{Q}{P}  
  :=
  \lift{(x)\substp{Q}{P}}{ R \psubstp{Q}{P} } \\
%   (\dropn{x})  \psubstp{Q}{P}       
%   := 
%   \left\{ 
%     \begin{array}{ccc} 
%       \dropn{\quotep{Q}} & & x \nameeq \quotep{P} \\
%       \dropn{x} & & otherwise \\
%     \end{array}
%   \right. 
  (\dropn{x})  \psubstp{Q}{P}       
  := 
  \left\{ 
    \begin{array}{ccc} 
      Q & & x \nameeq \quotep{P} \\
      \dropn{x} & & otherwise \\
    \end{array}
  \right.
\end{mathpar}
 

where

\begin{eqnarray}
  (x)\id{\{} \lpquote Q \rpquote / \lpquote P \rpquote \id{\}}            = 
  \left\{ 
    \begin{array}{ccc}
      \lpquote Q \rpquote & & x \nameeq \lpquote P \rpquote \\
      x & & otherwise \\
    \end{array}
  \right. \nonumber
\end{eqnarray}

and $z$ is chosen distinct from $\quotep{P}$, $\quotep{Q}$, the free
names in $Q$, and all the names in $R$. Our $\alpha$-equivalence will
be built in the standard way from this substitution.

\begin{remark}\label{rem:no_self_referential_names}
  One consequence of these definitions is that $\forall P. \quotep{P}
  \not\in \freenames{P}$.
\end{remark}

\subsection{ Dynamic quote: an example }

Anticipating something of what's to come, consider applying the
substitution, $\widehat{\id{\{}u / z \id{\}}}$, to the following pair
of processes, $\lift{w}{y!(z)}$ and $w[ \lpquote y!(z) \rpquote ]$.

\begin{eqnarray}
	\lift{w}{y!(z)}\widehat{\id{\{}u / z \id{\}}}
		& = &
		\lift{w}{y!(u)} \nonumber\\
	w[ \lpquote y!(z) \rpquote ] \widehat{ \id{\{}u / z \id{\}} }
		& = &
		w[ \lpquote y!(z) \rpquote ] \nonumber
\end{eqnarray}

Because the body of the process between quotes is impervious to
substitution, we get radically different answers. In fact, by
examining the first process in an input context,
e.g. $x?(z).\lift{w}{y!(z)}$, we see that the process under the lift
operator may be shaped by prefixed inputs binding a name inside it. In
this sense, the lift operator will be seen as a way to dynamically
construct processes before reifying them as names.

Finally equipped with these standard features we can present the
dynamics of the calculus.

\subsubsection{Operational semantics} 

Finally, we introduce the computational dynamics. What marks these
algebras as distinct from other more traditionally studied algebraic
structures, e.g. vector spaces or polynomial rings, is the manner in
which dynamics is captured. In traditional structures, dynamics is typically
expressed through morphisms between such structures, as in linear maps
between vector spaces or morphisms between rings. In algebras
associated with the semantics of computation, the dynamics is
expressed as part of the algebraic structure itself, through a
reduction reduction relation typically denoted by $\red$. Below, we
give a recursive presentation of this relation for the calculus used
in the encoding.

$\red \subseteq \pi \times \pi$
$\red : \pi \to \mathcal{P}(\pi)$

\begin{mathpar}
  \inferrule* [lab=Comm] { \textsf{match}( x_{src}, x_{trgt} ) } { x_{trgt}?(y)P \; | \; x_{src}!\langle {Q} \rangle \red P\{\quotep{Q}/y}\} }
  \and \\
  \inferrule* [lab=Par] {{P} \red {P}'} {{{P} | {Q}} \red {{P}' | {Q}}}
  \and
  \inferrule* [lab=Equiv]{{{P} \scong {P}'} \andalso {{P}' \red {Q}'} \andalso {{Q}' \scong {Q}}}{{P} \red {Q}}
\end{mathpar}

\begin{eqnarray*}
  match_{\equiv} (\quotep{P},\quotep{Q}) & := & P \equiv Q \\
  match_{\dagger}(\quotep{P},\quotep{Q}) & := & \forall R. P|Q \red^{*} R => R \red^{*} 0 \\
  match_{K}(\quotep{P},\quotep{Q}) & := & K \mbox{ for some context } K
\end{eqnarray*}

$u?(x)P | u!\langle Q \rangle \red P\{\quotep{Q}/x\}$

%We write $\wred$ for $\red^*$, and $P\red$ if $\exists Q $ such that $ P \red Q$.
We write $P\red$ if $\exists Q $ such that $ P \red Q$ and $P\not\red$, otherwise.

\section{Replication}

As mentioned before, it is known that replication (and hence
recursion) can be implemented in a higher-order process algebra
\cite{SangiorgiWalker}. As our first example of calculation with the
machinery thus far presented we give the construction explicitly in
the {\rhoc}.

\begin{eqnarray}
	D_{x} & := & \prefix{x}{y}{(\binpar{\outputp{x}{y}}{@{y}})} \nonumber\\
	\bangp_{x}{P} & := & \binpar{{x}!\langle{\binpar{D_{x}}{P}}\rangle}{D_{x}} \nonumber
\end{eqnarray}

\begin{eqnarray}
	\bangp_{x}{P} & & \nonumber\\
	=
	& {x}!\langle{(\prefix{x}{y}{(\outputp{x}{y} | @{y})) | P}}\rangle 
	      | \prefix{x}{y}{(\outputp{x}{y} | @{y})} & \nonumber\\
	\red
	& (\outputp{x}{y} | @{y})\substn{\quotep{(\prefix{x}{y}{(@{y} | \outputp{x}{y})) | P}}}{y} & \nonumber\\
	=
	& \outputp{x}{\quotep{(\prefix{x}{y}{(\outputp{x}{y} | @{y})) | P}}}
	  | {(\prefix{x}{y}{(\outputp{x}{y} | @{y})) | P}} & \nonumber\\
	\red
	& \ldots & \nonumber\\
	\red^*
	& P | P | \ldots & \nonumber
\end{eqnarray}

Of course, this encoding, as an implementation, runs away, unfolding
$\bangp{P}$ eagerly. A lazier and more implementable replication
operator, restricted to input-guarded processes, may be obtained as follows.

\begin{eqnarray}
\bangp{\prefix{u}{v}{P}} 
	:= 
	\binpar{\lift{x}{\prefix{u}{v}{(\binpar{D(x)}{P})}}}{D(x)} \nonumber
\end{eqnarray}

\begin{remark}
  Note that the lazier definition still does not deal with summation
  or mixed summation (i.e. sums over input and output). The reader is
  invited to construct definitions of replication that deal with these
  features. 

  Further, the definitions are parameterized in a name, $x$. Can you,
  gentle reader, make a definition that eliminates this parameter and
  guarantees no accidental interaction between the replication
  machinery and the process being replicated -- i.e. no accidental
  sharing of names used by the process to get its work done and the
  name(s) used by the replication to effect copying. This latter
  revision of the definition of replication is crucial to obtaining
  the expected identity $!!P \sim !P$.
\end{remark}

\begin{remark}\label{rem:paradoxical_combinator}
  The reader familiar with the lambda calculus will have noticed the
  similarity between $D$ and the paradoxical combinator.

  [Ed. note: the existence of this seems to suggest we have to be more
  restrictive on the set of processes and names we admit if we are to
  support no-cloning.]
\end{remark}

\subsubsection{Bisimulation}

The computational dynamics gives rise to another kind of equivalence,
the equivalence of computational behavior. As previously mentioned
this is typically captured \emph{via} some form of bisimulation.

% The notion we use in this paper is weak barbed bisimulation
% \cite{milner91polyadicpi}.

The notion we use in this paper is derived from weak barbed
bisimulation \cite{milner91polyadicpi}. 

\begin{definition}
An \emph{observation relation}, $\downarrow_{\mathcal N}$, over a set
of names, $\mathcal N$, is the smallest relation satisfying the rules
below.

\infrule[Out-barb]{y \in {\mathcal N}, \; x \nameeq y}
		  {\outputp{x}{v} \downarrow_{\mathcal N} x}
\infrule[Par-barb]{\mbox{$P\downarrow_{\mathcal N} x$ or $Q\downarrow_{\mathcal N} x$}}
		  {\binpar{P}{Q} \downarrow_{\mathcal N} x}

We write $P \Downarrow_{\mathcal N} x$ if there is $Q$ such that 
$P \wred Q$ and $Q \downarrow_{\mathcal N} x$.
\end{definition}

\begin{definition}
%\label{def.bbisim}
An  ${\mathcal N}$-\emph{barbed bisimulation} over a set of names, ${\mathcal N}$, is a symmetric binary relation 
${\mathcal S}_{\mathcal N}$ between agents such that $P\rel{S}_{\mathcal N}Q$ implies:
\begin{enumerate}
\item If $P \red P'$ then $Q \wred Q'$ and $P'\rel{S}_{\mathcal N} Q'$.
\item If $P\downarrow_{\mathcal N} x$, then $Q\Downarrow_{\mathcal N} x$.
\end{enumerate}
$P$ is ${\mathcal N}$-barbed bisimilar to $Q$, written
$P \wbbisim_{\mathcal N} Q$, if $P \rel{S}_{\mathcal N} Q$ for some ${\mathcal N}$-barbed bisimulation ${\mathcal S}_{\mathcal N}$.
\end{definition}

$\mathcal{R} \subseteq \pi \times \pi$

$P \mathcal{R} Q => \forall P'. P \red P' \Rightarrow \exists Q'. Q \red Q', P' \mathcal{R} Q'$

$P \vdash x \Rightarrow Q \vdash x$

\begin{mathpar}
  \inferrule*[lab=Out-barb]{x \nameeq y}{{y}!\langle{Q}\rangle \vdash x}
  \and
  \inferrule*[lab=Par-barb]{\mbox{$P\vdash x$ or $Q\vdash x$}}{\binpar{P}{Q} \vdash x}
\end{mathpar}

\subsubsection{Contexts}

One of the principle advantages of computational calculi like the
$\pi$-calculus is a well-defined notion of context,
contextual-equivalence and a correlation between
contextual-equivalence and notions of bisimulation. The notion of
context allows the decomposition of a process into (sub-)process and
its syntactic environment, its context. Thus, a context may be
thought of as a process with a ``hole'' (written $\Box$) in it. The
application of a context $M$ to a process $P$, written $M[P]$, is
tantamount to filling the hole in $M$ with $P$. In this paper we do
not need the full weight of this theory, but do make use of the notion
of context in the proof the main theorem. 

\begin{mathpar}
  \inferrule* [lab=summation] {} {{M_{M},M_{N}} \bc \Box \;|\; x.M_{A} \;|\; M_{M}+M_{N}}
  \and
  \inferrule* [lab=agent] {} {{M_{A}} \bc (\vec{x})M_{P} \;| \; \clift{P_0,\ldots,M_{P},\ldots,P_N}}
  \and \\
  \inferrule* [lab=process] {} {{M_{P}} \bc M_{N} \;| \;P|M_{P} }
\end{mathpar} 

\begin{mathpar}
  \inferrule* [lab=sychronization] {} {M_{N} \bc \Box \;|\; x?M_{F} \;|\; x!M_{C}}
  \and
  \inferrule* [lab=abstraction] {} {{M_{F}} \bc (x)M_{P} }
  \and
  \inferrule* [lab=concretion] {} {{M_{C}} \bc \langle M_{P} \rangle }
  \and \\
  \inferrule* [lab=process] {} {{M_{P}} \bc M_{N} \;| \;P|M_{P} }
\end{mathpar}

\begin{definition}[contextual application] Given a context $M$, and
  process $P$, we define the \emph{contextual application}, $M[P] :=
  M\{P/\Box\}$. That is, the contextual application of M to P is the
  substitution of $P$ for $\Box$ in $M$.
\end{definition}

$\meaningof{-} : L \to \mathcal{P}(\pi)$

\begin{mathpar}
  \inferrule* [lab=collection] {} {\meaningof{true} = \pi, \and \meaningof{~E} = \pi \setminus \meaningof{E}, \and \meaningof{E_{1} \& E_{2}} = \meaningof{E_{1}} \cap \meaningof{E_{2}}}
\end{mathpar}

\begin{mathpar}
  \inferrule* [lab=structure] {} {\meaningof{0} = \{ P \in \pi | P \equiv 0 \}, \and \\ \meaningof{E_1 | E_2} = \{ P \in \pi | P \equiv P_{1} | P_{2}, P_{1} \in \meaningof{E_{1}}, P_{2} \in \meaningof{E_2}\} }
\end{mathpar}

\begin{mathpar}
 \inferrule* [lab=behavior] {} {\meaningof{\langle a?b \rangle E} = \{ P \in \pi | P \equiv Q | u?(y)P', \\ \and \\\\ \and \\ \;\;\; u \in \meaningof{a}, \forall z.P'\{z/y\} \in \meaningof{E\{z/b\}}\}, \and \\ \meaningof{a!E} = \{ P \in \pi | P \equiv Q | x!\langle P' \rangle, x \in \meaningof{a} P' \in \meaningof{E}\} }
\end{mathpar}

\begin{mathpar}
 \inferrule* [lab=nominal] {} {\meaningof{\quotep{E}} = \{ \quotep{P} \in \quotep{\pi} | P \in \meaningof{E} \}, \and \meaningof{\quotep{P}} = \{ \quotep{Q} \in \quotep{\pi} | P \equiv Q \} \and \\ \meaningof{@\quotep{E}} = \{ P \in \pi | P \equiv @x, x \in \meaningof{E} \}}
\end{mathpar}

\begin{eqnarray*}
  \\
  \meaningof{-} : TS \to ST
\end{eqnarray*}

\begin{eqnarray*}
  \\
  L : TS \to ST
\end{eqnarray*}

\begin{eqnarray*}
  \\
  P \models E \iff P \in \meaningof{E}
\end{eqnarray*}

\begin{eqnarray*}
  P \approx_{L} Q \iff \forall E \in L. P \models E \iff Q \models E
\end{eqnarray*}

\begin{eqnarray*}
  P \approx_{K} Q
\end{eqnarray*}

\begin{eqnarray*}
  P \approx Q
\end{eqnarray*}

$\approx_{K} = \approx = \approx_{L}$

\subsubsection{Contextual duality}

Note that contexts extend the quotation operation to a family of
operations from processes to names. Given a context, $M$, we can
define a \emph{nominal context}, $\quotep{M}$ by $\quotep{M}[P] :=
\quotep{M[P]}$. To foreshadow what is to come we observe that these
operations enjoy a duality with processes very much like the duality
between vectors and maps from vectors to scalars.

Further, because the calculus is essentially higher-order, we have a
correspondence between contexts and processes. More specifically,
given a name $x$ and a context $M$ we can construct $M^{*}_{x}$ such
that 

\begin{mathpar}
  M^{*}_{x} | \lift{x}{P} \red M[P]
\end{mathpar}

namely,

\begin{mathpar}
  M^{*}_{x} := x?(u).M[\dropn{u}]
\end{mathpar}

The dependence of $M^{*}_{x}$ on a name makes it an abstraction, 

\begin{mathpar}
  M^{*} := (x)x?(u).M[\dropn{u}]
\end{mathpar}

\subsection{Additional notation}

It will sometimes be convenient to denote the process a name
quotes. We already have the notation $x = \quotep{P}$, but it will be
convenient to introduce an alternate notation, $\procn{x}$, when we
want to emphasize the connection to the use of the name. Note that, by
virtue of name equivalence, $\quotep{\procn{x}} \nameeq x$; so, the
notation is consistent with previous definitions.

Further, because names have structure it is possible to effect
substitutions on the basis of that structure. This means we need to
upgrade our notation for substitutions, which we accomplish by
adapting comprehension notation. Thus,

\begin{mathpar}
  P\{ y / x : x \in S \}
\end{mathpar}

is interpreted to mean the process derived from P by replacing (in a
capture-avoiding manner) each occurrence of $x$ in $S$ by $y$. For example,

\begin{mathpar}
  P\{ \quotep{\procn{x}|\procn{x}} / x : x \in \freenames{P} \}
\end{mathpar}

will replace each (occurrence) of a free name $x$ in $P$ by
$\quotep{\procn{x}|\procn{x}}$.

Also, we will avail ourselves of the notation $x^{L}$ and $x^{R}$ to
denote injections of a name into disjoint copies of the name
space. There are numerous ways to accomplish this. One example can be
found in \cite{MeredithR05}. This notation overloads to vectors of
names: $\vec{x}^{\pi} := (x_{i}^{\pi} \; : \; 0 \leq i < |\vec{x}| )$ where $\pi \in \{L,R\}$.

We also use $P^{\Box} := P|\Box$.

In \cite{MeredithR05} an interpretation of the new operator is
given. It turns out that there are several possible interpretations
all enjoying the requisite algebraic properties of the operator (see
\cite{milner91polyadicpi}). We will therefore make liberal use of
$(\nu\; \vec{x})P$.

% subsection the_syntax_and_semantics_of_the_notation_system (end)   

\input{qm2pi.qmops} 

\input{qm2pi.sterngerlach} 

\input{qm2pi.metric} 

% section concurrent_process_calculi (end)

%\input{qm2pi.proofsketch}

% section proof sketch (end)

%\input{qm2pi.slviaknots} 

% section spatial logic via knots (end)

\input{qm2pi.conclusion}

% section conclusion (end)

%\input{qm2pi.dtcodes} 

% section wiring algorithm (end)

\input{qm2pi.ack} 

% section acknowledgments (end)

\newpage


\bibliographystyle{plain}   
\bibliography{../../biblios/main.bib}

\input{qm2pi.rhodetails}

\end{document}



\end{document}

 

% section wiring algorithm (end)

\documentclass[12pt]{llncs}
%\documentclass{jktr}

\usepackage[pdftex]{hyperref}                   
\usepackage {listings}
\usepackage {mathpartir}
\usepackage{bcprules}
%\usepackage{listings}
                       
\usepackage{graphicx} 
%\usepackage[margins=2.5cm,nohead,nofoot]{geometry}
%\usepackage{geometry}
\usepackage{amsfonts}
\usepackage{amstext}
\usepackage{latexsym}
\usepackage{amssymb}
\usepackage{color}


%\include{myPreamble}
\documentclass[12pt]{llncs}
%\documentclass{jktr}

\usepackage[pdftex]{hyperref}                   
\usepackage {listings}
\usepackage {mathpartir}
\usepackage{bcprules}
%\usepackage{listings}
                       
\usepackage{graphicx} 
%\usepackage[margins=2.5cm,nohead,nofoot]{geometry}
%\usepackage{geometry}
\usepackage{amsfonts}
\usepackage{amstext}
\usepackage{latexsym}
\usepackage{amssymb}
\usepackage{color}


%\include{myPreamble}
\include{qm2pi.local} 

%\ifpdf
%\usepackage[pdftex]{graphicx}
%\else
%\usepackage{graphicx}
%\fi

 % \ifpdf
%  \usepackage{pdfsync}
%  \if


%\title{Brief Article}
%\author{David F. Snyder}
%\author{L.G. Meredith}

%\address{Dept. of Math., Texas State University--San Marcos, San Marcos, TX 78666}
       
\pagestyle{empty}


\begin{document}

\lstset{language=[Objective]Caml,frame=shadowbox}

\input{qm2pi.front}

% section front matter (end)

\input{qm2pi.intro} 
 
% section introduction (end)

% \input{qm2pi.knotations} 

% section notation (end)

\input{qm2pi.process.calculi} 

% section concurrent_process_calculi_and_spatial_logics_ (end)
    
%\input{qm2pi.knots2pi} 

%\input{qm2pi.trefoil} 

%\input{qm2pi.mainthm} 

% subsection basic_interpretation (end)

%\input{qm2pi.rho.presentation} 
\subsection{The syntax and semantics of the notation system}\label{sub:the_syntax_and_semantics_of_the_notation_system} % (fold)

We now summarize a technical presentation of the calculus that
embodies our theory of dynamics. The typical presentation of such a
calculus follows the style of giving generators and relations on
them. The grammar, below, describing term constructors, freely
generates the set of processes, $\Proc$. This set is then quotiented
by a relation known as structural congruence and it is over this set
that the notion of dynamics is expressed. This presentation is
essentially that of \cite{MeredithR05} with the addition of
polyadicity and summation. For readability we have relegated some of
the technical subtleties to an appendix.

\subsubsection{Process grammar}\label{subsub:process_grammar}

\begin{mathpar}
  \inferrule* [lab=synchronization] {} {{M} \bc \pzero \;|\; x?F \;|\; x!C }
  \and
  \inferrule* [lab=abstraction] {} {{F} \bc (x)P}
  \and
  \inferrule* [lab=concretion] {} {{C} \bc \langle Q \rangle}
  \and
  \inferrule* [lab=process] {} {{P,Q} \bc M \;| \;P|Q \;|\; @{x}}
  \and
  \inferrule* [lab=name] {} {{x} \bc \quotep{P}}
\end{mathpar} 

Note that $\vec{x}$ (resp. $\vec{P}$) denotes a vector of names
(resp. processes) of length $|\vec{x}|$ (resp. $|\vec{P}|$). We adopt
the following useful abbreviations.

\begin{mathpar}
   x?(\vec{y}).P := x.(\vec{y})P \and  x\clift{\vec{P}} := x.\clift{\vec{P}}
   \and x!(y) := \lift{x}{\dropn{y}}
   \and \Pi_{i=0}^{n-1}P_i := P_0 | \ldots | P_{n-1}
\end{mathpar}

\subsubsection{Structural congruence}

\paragraph{Free and bound names and alpha-equivalence.} At the
core of structural equivalence is alpha-equivalence which identifies
process that are the same up to a change of variable. Formally, we
recognize the distinction between free and bound names. The free names
of a process, $\freenames{P}$, may be calculated recursively as
follows:

\begin{mathpar}
\freenames{\pzero} := \emptyset
  \and \\
  \freenames{x?(y).P} := \{ x \} \cup (\freenames{P} \setminus \{ y \})
  \and 
  \freenames{x!\langle P \rangle} := \{ x \} \cup \{ P \} 
  \and \\
  \freenames{P|Q} := \freenames{P} \cup \freenames{Q}
  \and \\
  \freenames{@{x}} := \{ x \}
\end{mathpar}

$\pi$
$\quotep{\pi}$

$\freenames{-} : \pi \to \mathcal{P}(\quotep{\pi})$

\begin{eqnarray*}
  \freenames{\pzero} & := & \emptyset \\
  \freenames{x?(y).P} & := & \{ x \} \cup (\freenames{P} \setminus \{ y \}) \\
  \freenames{x!\langle P \rangle} & := & \{ x \} \cup \{ P \} \\
  \freenames{P|Q} & := & \freenames{P} \cup \freenames{Q} \\
  \freenames{\dropn{x}} & := & \{ x \}
\end{eqnarray*}

The bound names of a process, $\boundnames{P}$, are those names occurring in $P$
that are not free. For example, in $x?(y).0$, the name $x$ is free, while $y$ is bound.

\begin{mathpar}
  \inferrule* [lab=monoidal-laws] {} { P|Q \equiv Q|P \and P|0 \equiv P \and P|(Q|R) \equiv (P|Q)|R }
\end{mathpar}

\begin{mathpar}
  \inferrule* [lab=alpha-equivalence] {} { (x)P \equiv (y)P\{y/x\} \and y \not\in \freenames{P} }
\end{mathpar}

\begin{definition}
Then two processes, $P,Q$, are alpha-equivalent if $P = Q\{\vec{y}/\vec{x}\}$ for
some $\vec{x} \in \boundnames{Q},\vec{y} \in \boundnames{P}$, where $Q\{\vec{y}/\vec{x}\}$
denotes the capture-avoiding substitution of $\vec{y}$ for $\vec{x}$ in $Q$.
\end{definition}

\begin{definition}
  The {\em structural congruence} \cite{SangiorgiWalker} , $\equiv$,
  between processes is the least congruence containing
  alpha-equivalence, satisfying the abelian monoid laws
  (associativity, commutativity and $\pzero$ as identity) for parallel
  composition $|$ and for summation $+$.
\end{definition}

\subsection{Name equivalence}

We take name equivalence, written $\nameeq$, to be the smallest
equivalence relation generated by the following rules.

\begin{mathpar}
\inferrule*[lab=Quote-drop]
{ }
{ \quotep{@{x}} \nameeq x }

\inferrule*[lab=Struct-equiv]
{ P \scong Q }
{ \quotep{P} \nameeq \quotep{Q} }
\end{mathpar}

The astute reader will have noticed that the mutual recursion of names
and processes imposes a mutual recursion on alpha-equivalence and
structural equivalence via name-equivalence. Fortunately, all of this
works out pleasantly and we may calculate in the natural way, free of
concern. The reader interested in the details is referred to the
appendix \ref{appendix:rho_details}.

\subsection{Substitution}

We use $\Proc$ for the set of processes, $\QProc$ for the set of
names, and $\id{\{}\vec{y} / \vec{x} \id{\}}$ to denote partial maps,
$s : \QProc \rightarrow \QProc$. A map, $s$ lifts, uniquely, to a map
on process terms, $\widehat{s} : \Proc \rightarrow \Proc$ by the
following equations.

\begin{mathpar}
  (0) \psubstp{Q}{P} := 0 \\
  (R \juxtap S) \psubstp{Q}{P}
  :=    
  (R)\psubstp{Q}{P} \juxtap (S) \psubstp{Q}{P} \\
  (x?(y).R) \psubstp{Q}{P}    
  :=    
  (x)\substp{Q}{P} (z)\concat( (R \psubstn{z}{y}) \psubstp{Q}{P} ) \\
  (\lift{x}{R}) \psubstp{Q}{P}  
  :=
  \lift{(x)\substp{Q}{P}}{ R \psubstp{Q}{P} } \\
%   (\dropn{x})  \psubstp{Q}{P}       
%   := 
%   \left\{ 
%     \begin{array}{ccc} 
%       \dropn{\quotep{Q}} & & x \nameeq \quotep{P} \\
%       \dropn{x} & & otherwise \\
%     \end{array}
%   \right. 
  (\dropn{x})  \psubstp{Q}{P}       
  := 
  \left\{ 
    \begin{array}{ccc} 
      Q & & x \nameeq \quotep{P} \\
      \dropn{x} & & otherwise \\
    \end{array}
  \right.
\end{mathpar}
 

where

\begin{eqnarray}
  (x)\id{\{} \lpquote Q \rpquote / \lpquote P \rpquote \id{\}}            = 
  \left\{ 
    \begin{array}{ccc}
      \lpquote Q \rpquote & & x \nameeq \lpquote P \rpquote \\
      x & & otherwise \\
    \end{array}
  \right. \nonumber
\end{eqnarray}

and $z$ is chosen distinct from $\quotep{P}$, $\quotep{Q}$, the free
names in $Q$, and all the names in $R$. Our $\alpha$-equivalence will
be built in the standard way from this substitution.

\begin{remark}\label{rem:no_self_referential_names}
  One consequence of these definitions is that $\forall P. \quotep{P}
  \not\in \freenames{P}$.
\end{remark}

\subsection{ Dynamic quote: an example }

Anticipating something of what's to come, consider applying the
substitution, $\widehat{\id{\{}u / z \id{\}}}$, to the following pair
of processes, $\lift{w}{y!(z)}$ and $w[ \lpquote y!(z) \rpquote ]$.

\begin{eqnarray}
	\lift{w}{y!(z)}\widehat{\id{\{}u / z \id{\}}}
		& = &
		\lift{w}{y!(u)} \nonumber\\
	w[ \lpquote y!(z) \rpquote ] \widehat{ \id{\{}u / z \id{\}} }
		& = &
		w[ \lpquote y!(z) \rpquote ] \nonumber
\end{eqnarray}

Because the body of the process between quotes is impervious to
substitution, we get radically different answers. In fact, by
examining the first process in an input context,
e.g. $x?(z).\lift{w}{y!(z)}$, we see that the process under the lift
operator may be shaped by prefixed inputs binding a name inside it. In
this sense, the lift operator will be seen as a way to dynamically
construct processes before reifying them as names.

Finally equipped with these standard features we can present the
dynamics of the calculus.

\subsubsection{Operational semantics} 

Finally, we introduce the computational dynamics. What marks these
algebras as distinct from other more traditionally studied algebraic
structures, e.g. vector spaces or polynomial rings, is the manner in
which dynamics is captured. In traditional structures, dynamics is typically
expressed through morphisms between such structures, as in linear maps
between vector spaces or morphisms between rings. In algebras
associated with the semantics of computation, the dynamics is
expressed as part of the algebraic structure itself, through a
reduction reduction relation typically denoted by $\red$. Below, we
give a recursive presentation of this relation for the calculus used
in the encoding.

$\red \subseteq \pi \times \pi$
$\red : \pi \to \mathcal{P}(\pi)$

\begin{mathpar}
  \inferrule* [lab=Comm] { \textsf{match}( x_{src}, x_{trgt} ) } { x_{trgt}?(y)P \; | \; x_{src}!\langle {Q} \rangle \red P\{\quotep{Q}/y}\} }
  \and \\
  \inferrule* [lab=Par] {{P} \red {P}'} {{{P} | {Q}} \red {{P}' | {Q}}}
  \and
  \inferrule* [lab=Equiv]{{{P} \scong {P}'} \andalso {{P}' \red {Q}'} \andalso {{Q}' \scong {Q}}}{{P} \red {Q}}
\end{mathpar}

\begin{eqnarray*}
  match_{\equiv} (\quotep{P},\quotep{Q}) & := & P \equiv Q \\
  match_{\dagger}(\quotep{P},\quotep{Q}) & := & \forall R. P|Q \red^{*} R => R \red^{*} 0 \\
  match_{K}(\quotep{P},\quotep{Q}) & := & K \mbox{ for some context } K
\end{eqnarray*}

$u?(x)P | u!\langle Q \rangle \red P\{\quotep{Q}/x\}$

%We write $\wred$ for $\red^*$, and $P\red$ if $\exists Q $ such that $ P \red Q$.
We write $P\red$ if $\exists Q $ such that $ P \red Q$ and $P\not\red$, otherwise.

\section{Replication}

As mentioned before, it is known that replication (and hence
recursion) can be implemented in a higher-order process algebra
\cite{SangiorgiWalker}. As our first example of calculation with the
machinery thus far presented we give the construction explicitly in
the {\rhoc}.

\begin{eqnarray}
	D_{x} & := & \prefix{x}{y}{(\binpar{\outputp{x}{y}}{@{y}})} \nonumber\\
	\bangp_{x}{P} & := & \binpar{{x}!\langle{\binpar{D_{x}}{P}}\rangle}{D_{x}} \nonumber
\end{eqnarray}

\begin{eqnarray}
	\bangp_{x}{P} & & \nonumber\\
	=
	& {x}!\langle{(\prefix{x}{y}{(\outputp{x}{y} | @{y})) | P}}\rangle 
	      | \prefix{x}{y}{(\outputp{x}{y} | @{y})} & \nonumber\\
	\red
	& (\outputp{x}{y} | @{y})\substn{\quotep{(\prefix{x}{y}{(@{y} | \outputp{x}{y})) | P}}}{y} & \nonumber\\
	=
	& \outputp{x}{\quotep{(\prefix{x}{y}{(\outputp{x}{y} | @{y})) | P}}}
	  | {(\prefix{x}{y}{(\outputp{x}{y} | @{y})) | P}} & \nonumber\\
	\red
	& \ldots & \nonumber\\
	\red^*
	& P | P | \ldots & \nonumber
\end{eqnarray}

Of course, this encoding, as an implementation, runs away, unfolding
$\bangp{P}$ eagerly. A lazier and more implementable replication
operator, restricted to input-guarded processes, may be obtained as follows.

\begin{eqnarray}
\bangp{\prefix{u}{v}{P}} 
	:= 
	\binpar{\lift{x}{\prefix{u}{v}{(\binpar{D(x)}{P})}}}{D(x)} \nonumber
\end{eqnarray}

\begin{remark}
  Note that the lazier definition still does not deal with summation
  or mixed summation (i.e. sums over input and output). The reader is
  invited to construct definitions of replication that deal with these
  features. 

  Further, the definitions are parameterized in a name, $x$. Can you,
  gentle reader, make a definition that eliminates this parameter and
  guarantees no accidental interaction between the replication
  machinery and the process being replicated -- i.e. no accidental
  sharing of names used by the process to get its work done and the
  name(s) used by the replication to effect copying. This latter
  revision of the definition of replication is crucial to obtaining
  the expected identity $!!P \sim !P$.
\end{remark}

\begin{remark}\label{rem:paradoxical_combinator}
  The reader familiar with the lambda calculus will have noticed the
  similarity between $D$ and the paradoxical combinator.

  [Ed. note: the existence of this seems to suggest we have to be more
  restrictive on the set of processes and names we admit if we are to
  support no-cloning.]
\end{remark}

\subsubsection{Bisimulation}

The computational dynamics gives rise to another kind of equivalence,
the equivalence of computational behavior. As previously mentioned
this is typically captured \emph{via} some form of bisimulation.

% The notion we use in this paper is weak barbed bisimulation
% \cite{milner91polyadicpi}.

The notion we use in this paper is derived from weak barbed
bisimulation \cite{milner91polyadicpi}. 

\begin{definition}
An \emph{observation relation}, $\downarrow_{\mathcal N}$, over a set
of names, $\mathcal N$, is the smallest relation satisfying the rules
below.

\infrule[Out-barb]{y \in {\mathcal N}, \; x \nameeq y}
		  {\outputp{x}{v} \downarrow_{\mathcal N} x}
\infrule[Par-barb]{\mbox{$P\downarrow_{\mathcal N} x$ or $Q\downarrow_{\mathcal N} x$}}
		  {\binpar{P}{Q} \downarrow_{\mathcal N} x}

We write $P \Downarrow_{\mathcal N} x$ if there is $Q$ such that 
$P \wred Q$ and $Q \downarrow_{\mathcal N} x$.
\end{definition}

\begin{definition}
%\label{def.bbisim}
An  ${\mathcal N}$-\emph{barbed bisimulation} over a set of names, ${\mathcal N}$, is a symmetric binary relation 
${\mathcal S}_{\mathcal N}$ between agents such that $P\rel{S}_{\mathcal N}Q$ implies:
\begin{enumerate}
\item If $P \red P'$ then $Q \wred Q'$ and $P'\rel{S}_{\mathcal N} Q'$.
\item If $P\downarrow_{\mathcal N} x$, then $Q\Downarrow_{\mathcal N} x$.
\end{enumerate}
$P$ is ${\mathcal N}$-barbed bisimilar to $Q$, written
$P \wbbisim_{\mathcal N} Q$, if $P \rel{S}_{\mathcal N} Q$ for some ${\mathcal N}$-barbed bisimulation ${\mathcal S}_{\mathcal N}$.
\end{definition}

$\mathcal{R} \subseteq \pi \times \pi$

$P \mathcal{R} Q => \forall P'. P \red P' \Rightarrow \exists Q'. Q \red Q', P' \mathcal{R} Q'$

$P \vdash x \Rightarrow Q \vdash x$

\begin{mathpar}
  \inferrule*[lab=Out-barb]{x \nameeq y}{{y}!\langle{Q}\rangle \vdash x}
  \and
  \inferrule*[lab=Par-barb]{\mbox{$P\vdash x$ or $Q\vdash x$}}{\binpar{P}{Q} \vdash x}
\end{mathpar}

\subsubsection{Contexts}

One of the principle advantages of computational calculi like the
$\pi$-calculus is a well-defined notion of context,
contextual-equivalence and a correlation between
contextual-equivalence and notions of bisimulation. The notion of
context allows the decomposition of a process into (sub-)process and
its syntactic environment, its context. Thus, a context may be
thought of as a process with a ``hole'' (written $\Box$) in it. The
application of a context $M$ to a process $P$, written $M[P]$, is
tantamount to filling the hole in $M$ with $P$. In this paper we do
not need the full weight of this theory, but do make use of the notion
of context in the proof the main theorem. 

\begin{mathpar}
  \inferrule* [lab=summation] {} {{M_{M},M_{N}} \bc \Box \;|\; x.M_{A} \;|\; M_{M}+M_{N}}
  \and
  \inferrule* [lab=agent] {} {{M_{A}} \bc (\vec{x})M_{P} \;| \; \clift{P_0,\ldots,M_{P},\ldots,P_N}}
  \and \\
  \inferrule* [lab=process] {} {{M_{P}} \bc M_{N} \;| \;P|M_{P} }
\end{mathpar} 

\begin{mathpar}
  \inferrule* [lab=sychronization] {} {M_{N} \bc \Box \;|\; x?M_{F} \;|\; x!M_{C}}
  \and
  \inferrule* [lab=abstraction] {} {{M_{F}} \bc (x)M_{P} }
  \and
  \inferrule* [lab=concretion] {} {{M_{C}} \bc \langle M_{P} \rangle }
  \and \\
  \inferrule* [lab=process] {} {{M_{P}} \bc M_{N} \;| \;P|M_{P} }
\end{mathpar}

\begin{definition}[contextual application] Given a context $M$, and
  process $P$, we define the \emph{contextual application}, $M[P] :=
  M\{P/\Box\}$. That is, the contextual application of M to P is the
  substitution of $P$ for $\Box$ in $M$.
\end{definition}

$\meaningof{-} : L \to \mathcal{P}(\pi)$

\begin{mathpar}
  \inferrule* [lab=collection] {} {\meaningof{true} = \pi, \and \meaningof{~E} = \pi \setminus \meaningof{E}, \and \meaningof{E_{1} \& E_{2}} = \meaningof{E_{1}} \cap \meaningof{E_{2}}}
\end{mathpar}

\begin{mathpar}
  \inferrule* [lab=structure] {} {\meaningof{0} = \{ P \in \pi | P \equiv 0 \}, \and \\ \meaningof{E_1 | E_2} = \{ P \in \pi | P \equiv P_{1} | P_{2}, P_{1} \in \meaningof{E_{1}}, P_{2} \in \meaningof{E_2}\} }
\end{mathpar}

\begin{mathpar}
 \inferrule* [lab=behavior] {} {\meaningof{\langle a?b \rangle E} = \{ P \in \pi | P \equiv Q | u?(y)P', \\ \and \\\\ \and \\ \;\;\; u \in \meaningof{a}, \forall z.P'\{z/y\} \in \meaningof{E\{z/b\}}\}, \and \\ \meaningof{a!E} = \{ P \in \pi | P \equiv Q | x!\langle P' \rangle, x \in \meaningof{a} P' \in \meaningof{E}\} }
\end{mathpar}

\begin{mathpar}
 \inferrule* [lab=nominal] {} {\meaningof{\quotep{E}} = \{ \quotep{P} \in \quotep{\pi} | P \in \meaningof{E} \}, \and \meaningof{\quotep{P}} = \{ \quotep{Q} \in \quotep{\pi} | P \equiv Q \} \and \\ \meaningof{@\quotep{E}} = \{ P \in \pi | P \equiv @x, x \in \meaningof{E} \}}
\end{mathpar}

\begin{eqnarray*}
  \\
  \meaningof{-} : TS \to ST
\end{eqnarray*}

\begin{eqnarray*}
  \\
  L : TS \to ST
\end{eqnarray*}

\begin{eqnarray*}
  \\
  P \models E \iff P \in \meaningof{E}
\end{eqnarray*}

\begin{eqnarray*}
  P \approx_{L} Q \iff \forall E \in L. P \models E \iff Q \models E
\end{eqnarray*}

\begin{eqnarray*}
  P \approx_{K} Q
\end{eqnarray*}

\begin{eqnarray*}
  P \approx Q
\end{eqnarray*}

$\approx_{K} = \approx = \approx_{L}$

\subsubsection{Contextual duality}

Note that contexts extend the quotation operation to a family of
operations from processes to names. Given a context, $M$, we can
define a \emph{nominal context}, $\quotep{M}$ by $\quotep{M}[P] :=
\quotep{M[P]}$. To foreshadow what is to come we observe that these
operations enjoy a duality with processes very much like the duality
between vectors and maps from vectors to scalars.

Further, because the calculus is essentially higher-order, we have a
correspondence between contexts and processes. More specifically,
given a name $x$ and a context $M$ we can construct $M^{*}_{x}$ such
that 

\begin{mathpar}
  M^{*}_{x} | \lift{x}{P} \red M[P]
\end{mathpar}

namely,

\begin{mathpar}
  M^{*}_{x} := x?(u).M[\dropn{u}]
\end{mathpar}

The dependence of $M^{*}_{x}$ on a name makes it an abstraction, 

\begin{mathpar}
  M^{*} := (x)x?(u).M[\dropn{u}]
\end{mathpar}

\subsection{Additional notation}

It will sometimes be convenient to denote the process a name
quotes. We already have the notation $x = \quotep{P}$, but it will be
convenient to introduce an alternate notation, $\procn{x}$, when we
want to emphasize the connection to the use of the name. Note that, by
virtue of name equivalence, $\quotep{\procn{x}} \nameeq x$; so, the
notation is consistent with previous definitions.

Further, because names have structure it is possible to effect
substitutions on the basis of that structure. This means we need to
upgrade our notation for substitutions, which we accomplish by
adapting comprehension notation. Thus,

\begin{mathpar}
  P\{ y / x : x \in S \}
\end{mathpar}

is interpreted to mean the process derived from P by replacing (in a
capture-avoiding manner) each occurrence of $x$ in $S$ by $y$. For example,

\begin{mathpar}
  P\{ \quotep{\procn{x}|\procn{x}} / x : x \in \freenames{P} \}
\end{mathpar}

will replace each (occurrence) of a free name $x$ in $P$ by
$\quotep{\procn{x}|\procn{x}}$.

Also, we will avail ourselves of the notation $x^{L}$ and $x^{R}$ to
denote injections of a name into disjoint copies of the name
space. There are numerous ways to accomplish this. One example can be
found in \cite{MeredithR05}. This notation overloads to vectors of
names: $\vec{x}^{\pi} := (x_{i}^{\pi} \; : \; 0 \leq i < |\vec{x}| )$ where $\pi \in \{L,R\}$.

We also use $P^{\Box} := P|\Box$.

In \cite{MeredithR05} an interpretation of the new operator is
given. It turns out that there are several possible interpretations
all enjoying the requisite algebraic properties of the operator (see
\cite{milner91polyadicpi}). We will therefore make liberal use of
$(\nu\; \vec{x})P$.

% subsection the_syntax_and_semantics_of_the_notation_system (end)   

\input{qm2pi.qmops} 

\input{qm2pi.sterngerlach} 

\input{qm2pi.metric} 

% section concurrent_process_calculi (end)

%\input{qm2pi.proofsketch}

% section proof sketch (end)

%\input{qm2pi.slviaknots} 

% section spatial logic via knots (end)

\input{qm2pi.conclusion}

% section conclusion (end)

%\input{qm2pi.dtcodes} 

% section wiring algorithm (end)

\input{qm2pi.ack} 

% section acknowledgments (end)

\newpage


\bibliographystyle{plain}   
\bibliography{../../biblios/main.bib}

\input{qm2pi.rhodetails}

\end{document}

 

%\ifpdf
%\usepackage[pdftex]{graphicx}
%\else
%\usepackage{graphicx}
%\fi

 % \ifpdf
%  \usepackage{pdfsync}
%  \if


%\title{Brief Article}
%\author{David F. Snyder}
%\author{L.G. Meredith}

%\address{Dept. of Math., Texas State University--San Marcos, San Marcos, TX 78666}
       
\pagestyle{empty}


\begin{document}

\lstset{language=[Objective]Caml,frame=shadowbox}

\documentclass[12pt]{llncs}
%\documentclass{jktr}

\usepackage[pdftex]{hyperref}                   
\usepackage {listings}
\usepackage {mathpartir}
\usepackage{bcprules}
%\usepackage{listings}
                       
\usepackage{graphicx} 
%\usepackage[margins=2.5cm,nohead,nofoot]{geometry}
%\usepackage{geometry}
\usepackage{amsfonts}
\usepackage{amstext}
\usepackage{latexsym}
\usepackage{amssymb}
\usepackage{color}


%\include{myPreamble}
\include{qm2pi.local} 

%\ifpdf
%\usepackage[pdftex]{graphicx}
%\else
%\usepackage{graphicx}
%\fi

 % \ifpdf
%  \usepackage{pdfsync}
%  \if


%\title{Brief Article}
%\author{David F. Snyder}
%\author{L.G. Meredith}

%\address{Dept. of Math., Texas State University--San Marcos, San Marcos, TX 78666}
       
\pagestyle{empty}


\begin{document}

\lstset{language=[Objective]Caml,frame=shadowbox}

\input{qm2pi.front}

% section front matter (end)

\input{qm2pi.intro} 
 
% section introduction (end)

% \input{qm2pi.knotations} 

% section notation (end)

\input{qm2pi.process.calculi} 

% section concurrent_process_calculi_and_spatial_logics_ (end)
    
%\input{qm2pi.knots2pi} 

%\input{qm2pi.trefoil} 

%\input{qm2pi.mainthm} 

% subsection basic_interpretation (end)

%\input{qm2pi.rho.presentation} 
\subsection{The syntax and semantics of the notation system}\label{sub:the_syntax_and_semantics_of_the_notation_system} % (fold)

We now summarize a technical presentation of the calculus that
embodies our theory of dynamics. The typical presentation of such a
calculus follows the style of giving generators and relations on
them. The grammar, below, describing term constructors, freely
generates the set of processes, $\Proc$. This set is then quotiented
by a relation known as structural congruence and it is over this set
that the notion of dynamics is expressed. This presentation is
essentially that of \cite{MeredithR05} with the addition of
polyadicity and summation. For readability we have relegated some of
the technical subtleties to an appendix.

\subsubsection{Process grammar}\label{subsub:process_grammar}

\begin{mathpar}
  \inferrule* [lab=synchronization] {} {{M} \bc \pzero \;|\; x?F \;|\; x!C }
  \and
  \inferrule* [lab=abstraction] {} {{F} \bc (x)P}
  \and
  \inferrule* [lab=concretion] {} {{C} \bc \langle Q \rangle}
  \and
  \inferrule* [lab=process] {} {{P,Q} \bc M \;| \;P|Q \;|\; @{x}}
  \and
  \inferrule* [lab=name] {} {{x} \bc \quotep{P}}
\end{mathpar} 

Note that $\vec{x}$ (resp. $\vec{P}$) denotes a vector of names
(resp. processes) of length $|\vec{x}|$ (resp. $|\vec{P}|$). We adopt
the following useful abbreviations.

\begin{mathpar}
   x?(\vec{y}).P := x.(\vec{y})P \and  x\clift{\vec{P}} := x.\clift{\vec{P}}
   \and x!(y) := \lift{x}{\dropn{y}}
   \and \Pi_{i=0}^{n-1}P_i := P_0 | \ldots | P_{n-1}
\end{mathpar}

\subsubsection{Structural congruence}

\paragraph{Free and bound names and alpha-equivalence.} At the
core of structural equivalence is alpha-equivalence which identifies
process that are the same up to a change of variable. Formally, we
recognize the distinction between free and bound names. The free names
of a process, $\freenames{P}$, may be calculated recursively as
follows:

\begin{mathpar}
\freenames{\pzero} := \emptyset
  \and \\
  \freenames{x?(y).P} := \{ x \} \cup (\freenames{P} \setminus \{ y \})
  \and 
  \freenames{x!\langle P \rangle} := \{ x \} \cup \{ P \} 
  \and \\
  \freenames{P|Q} := \freenames{P} \cup \freenames{Q}
  \and \\
  \freenames{@{x}} := \{ x \}
\end{mathpar}

$\pi$
$\quotep{\pi}$

$\freenames{-} : \pi \to \mathcal{P}(\quotep{\pi})$

\begin{eqnarray*}
  \freenames{\pzero} & := & \emptyset \\
  \freenames{x?(y).P} & := & \{ x \} \cup (\freenames{P} \setminus \{ y \}) \\
  \freenames{x!\langle P \rangle} & := & \{ x \} \cup \{ P \} \\
  \freenames{P|Q} & := & \freenames{P} \cup \freenames{Q} \\
  \freenames{\dropn{x}} & := & \{ x \}
\end{eqnarray*}

The bound names of a process, $\boundnames{P}$, are those names occurring in $P$
that are not free. For example, in $x?(y).0$, the name $x$ is free, while $y$ is bound.

\begin{mathpar}
  \inferrule* [lab=monoidal-laws] {} { P|Q \equiv Q|P \and P|0 \equiv P \and P|(Q|R) \equiv (P|Q)|R }
\end{mathpar}

\begin{mathpar}
  \inferrule* [lab=alpha-equivalence] {} { (x)P \equiv (y)P\{y/x\} \and y \not\in \freenames{P} }
\end{mathpar}

\begin{definition}
Then two processes, $P,Q$, are alpha-equivalent if $P = Q\{\vec{y}/\vec{x}\}$ for
some $\vec{x} \in \boundnames{Q},\vec{y} \in \boundnames{P}$, where $Q\{\vec{y}/\vec{x}\}$
denotes the capture-avoiding substitution of $\vec{y}$ for $\vec{x}$ in $Q$.
\end{definition}

\begin{definition}
  The {\em structural congruence} \cite{SangiorgiWalker} , $\equiv$,
  between processes is the least congruence containing
  alpha-equivalence, satisfying the abelian monoid laws
  (associativity, commutativity and $\pzero$ as identity) for parallel
  composition $|$ and for summation $+$.
\end{definition}

\subsection{Name equivalence}

We take name equivalence, written $\nameeq$, to be the smallest
equivalence relation generated by the following rules.

\begin{mathpar}
\inferrule*[lab=Quote-drop]
{ }
{ \quotep{@{x}} \nameeq x }

\inferrule*[lab=Struct-equiv]
{ P \scong Q }
{ \quotep{P} \nameeq \quotep{Q} }
\end{mathpar}

The astute reader will have noticed that the mutual recursion of names
and processes imposes a mutual recursion on alpha-equivalence and
structural equivalence via name-equivalence. Fortunately, all of this
works out pleasantly and we may calculate in the natural way, free of
concern. The reader interested in the details is referred to the
appendix \ref{appendix:rho_details}.

\subsection{Substitution}

We use $\Proc$ for the set of processes, $\QProc$ for the set of
names, and $\id{\{}\vec{y} / \vec{x} \id{\}}$ to denote partial maps,
$s : \QProc \rightarrow \QProc$. A map, $s$ lifts, uniquely, to a map
on process terms, $\widehat{s} : \Proc \rightarrow \Proc$ by the
following equations.

\begin{mathpar}
  (0) \psubstp{Q}{P} := 0 \\
  (R \juxtap S) \psubstp{Q}{P}
  :=    
  (R)\psubstp{Q}{P} \juxtap (S) \psubstp{Q}{P} \\
  (x?(y).R) \psubstp{Q}{P}    
  :=    
  (x)\substp{Q}{P} (z)\concat( (R \psubstn{z}{y}) \psubstp{Q}{P} ) \\
  (\lift{x}{R}) \psubstp{Q}{P}  
  :=
  \lift{(x)\substp{Q}{P}}{ R \psubstp{Q}{P} } \\
%   (\dropn{x})  \psubstp{Q}{P}       
%   := 
%   \left\{ 
%     \begin{array}{ccc} 
%       \dropn{\quotep{Q}} & & x \nameeq \quotep{P} \\
%       \dropn{x} & & otherwise \\
%     \end{array}
%   \right. 
  (\dropn{x})  \psubstp{Q}{P}       
  := 
  \left\{ 
    \begin{array}{ccc} 
      Q & & x \nameeq \quotep{P} \\
      \dropn{x} & & otherwise \\
    \end{array}
  \right.
\end{mathpar}
 

where

\begin{eqnarray}
  (x)\id{\{} \lpquote Q \rpquote / \lpquote P \rpquote \id{\}}            = 
  \left\{ 
    \begin{array}{ccc}
      \lpquote Q \rpquote & & x \nameeq \lpquote P \rpquote \\
      x & & otherwise \\
    \end{array}
  \right. \nonumber
\end{eqnarray}

and $z$ is chosen distinct from $\quotep{P}$, $\quotep{Q}$, the free
names in $Q$, and all the names in $R$. Our $\alpha$-equivalence will
be built in the standard way from this substitution.

\begin{remark}\label{rem:no_self_referential_names}
  One consequence of these definitions is that $\forall P. \quotep{P}
  \not\in \freenames{P}$.
\end{remark}

\subsection{ Dynamic quote: an example }

Anticipating something of what's to come, consider applying the
substitution, $\widehat{\id{\{}u / z \id{\}}}$, to the following pair
of processes, $\lift{w}{y!(z)}$ and $w[ \lpquote y!(z) \rpquote ]$.

\begin{eqnarray}
	\lift{w}{y!(z)}\widehat{\id{\{}u / z \id{\}}}
		& = &
		\lift{w}{y!(u)} \nonumber\\
	w[ \lpquote y!(z) \rpquote ] \widehat{ \id{\{}u / z \id{\}} }
		& = &
		w[ \lpquote y!(z) \rpquote ] \nonumber
\end{eqnarray}

Because the body of the process between quotes is impervious to
substitution, we get radically different answers. In fact, by
examining the first process in an input context,
e.g. $x?(z).\lift{w}{y!(z)}$, we see that the process under the lift
operator may be shaped by prefixed inputs binding a name inside it. In
this sense, the lift operator will be seen as a way to dynamically
construct processes before reifying them as names.

Finally equipped with these standard features we can present the
dynamics of the calculus.

\subsubsection{Operational semantics} 

Finally, we introduce the computational dynamics. What marks these
algebras as distinct from other more traditionally studied algebraic
structures, e.g. vector spaces or polynomial rings, is the manner in
which dynamics is captured. In traditional structures, dynamics is typically
expressed through morphisms between such structures, as in linear maps
between vector spaces or morphisms between rings. In algebras
associated with the semantics of computation, the dynamics is
expressed as part of the algebraic structure itself, through a
reduction reduction relation typically denoted by $\red$. Below, we
give a recursive presentation of this relation for the calculus used
in the encoding.

$\red \subseteq \pi \times \pi$
$\red : \pi \to \mathcal{P}(\pi)$

\begin{mathpar}
  \inferrule* [lab=Comm] { \textsf{match}( x_{src}, x_{trgt} ) } { x_{trgt}?(y)P \; | \; x_{src}!\langle {Q} \rangle \red P\{\quotep{Q}/y}\} }
  \and \\
  \inferrule* [lab=Par] {{P} \red {P}'} {{{P} | {Q}} \red {{P}' | {Q}}}
  \and
  \inferrule* [lab=Equiv]{{{P} \scong {P}'} \andalso {{P}' \red {Q}'} \andalso {{Q}' \scong {Q}}}{{P} \red {Q}}
\end{mathpar}

\begin{eqnarray*}
  match_{\equiv} (\quotep{P},\quotep{Q}) & := & P \equiv Q \\
  match_{\dagger}(\quotep{P},\quotep{Q}) & := & \forall R. P|Q \red^{*} R => R \red^{*} 0 \\
  match_{K}(\quotep{P},\quotep{Q}) & := & K \mbox{ for some context } K
\end{eqnarray*}

$u?(x)P | u!\langle Q \rangle \red P\{\quotep{Q}/x\}$

%We write $\wred$ for $\red^*$, and $P\red$ if $\exists Q $ such that $ P \red Q$.
We write $P\red$ if $\exists Q $ such that $ P \red Q$ and $P\not\red$, otherwise.

\section{Replication}

As mentioned before, it is known that replication (and hence
recursion) can be implemented in a higher-order process algebra
\cite{SangiorgiWalker}. As our first example of calculation with the
machinery thus far presented we give the construction explicitly in
the {\rhoc}.

\begin{eqnarray}
	D_{x} & := & \prefix{x}{y}{(\binpar{\outputp{x}{y}}{@{y}})} \nonumber\\
	\bangp_{x}{P} & := & \binpar{{x}!\langle{\binpar{D_{x}}{P}}\rangle}{D_{x}} \nonumber
\end{eqnarray}

\begin{eqnarray}
	\bangp_{x}{P} & & \nonumber\\
	=
	& {x}!\langle{(\prefix{x}{y}{(\outputp{x}{y} | @{y})) | P}}\rangle 
	      | \prefix{x}{y}{(\outputp{x}{y} | @{y})} & \nonumber\\
	\red
	& (\outputp{x}{y} | @{y})\substn{\quotep{(\prefix{x}{y}{(@{y} | \outputp{x}{y})) | P}}}{y} & \nonumber\\
	=
	& \outputp{x}{\quotep{(\prefix{x}{y}{(\outputp{x}{y} | @{y})) | P}}}
	  | {(\prefix{x}{y}{(\outputp{x}{y} | @{y})) | P}} & \nonumber\\
	\red
	& \ldots & \nonumber\\
	\red^*
	& P | P | \ldots & \nonumber
\end{eqnarray}

Of course, this encoding, as an implementation, runs away, unfolding
$\bangp{P}$ eagerly. A lazier and more implementable replication
operator, restricted to input-guarded processes, may be obtained as follows.

\begin{eqnarray}
\bangp{\prefix{u}{v}{P}} 
	:= 
	\binpar{\lift{x}{\prefix{u}{v}{(\binpar{D(x)}{P})}}}{D(x)} \nonumber
\end{eqnarray}

\begin{remark}
  Note that the lazier definition still does not deal with summation
  or mixed summation (i.e. sums over input and output). The reader is
  invited to construct definitions of replication that deal with these
  features. 

  Further, the definitions are parameterized in a name, $x$. Can you,
  gentle reader, make a definition that eliminates this parameter and
  guarantees no accidental interaction between the replication
  machinery and the process being replicated -- i.e. no accidental
  sharing of names used by the process to get its work done and the
  name(s) used by the replication to effect copying. This latter
  revision of the definition of replication is crucial to obtaining
  the expected identity $!!P \sim !P$.
\end{remark}

\begin{remark}\label{rem:paradoxical_combinator}
  The reader familiar with the lambda calculus will have noticed the
  similarity between $D$ and the paradoxical combinator.

  [Ed. note: the existence of this seems to suggest we have to be more
  restrictive on the set of processes and names we admit if we are to
  support no-cloning.]
\end{remark}

\subsubsection{Bisimulation}

The computational dynamics gives rise to another kind of equivalence,
the equivalence of computational behavior. As previously mentioned
this is typically captured \emph{via} some form of bisimulation.

% The notion we use in this paper is weak barbed bisimulation
% \cite{milner91polyadicpi}.

The notion we use in this paper is derived from weak barbed
bisimulation \cite{milner91polyadicpi}. 

\begin{definition}
An \emph{observation relation}, $\downarrow_{\mathcal N}$, over a set
of names, $\mathcal N$, is the smallest relation satisfying the rules
below.

\infrule[Out-barb]{y \in {\mathcal N}, \; x \nameeq y}
		  {\outputp{x}{v} \downarrow_{\mathcal N} x}
\infrule[Par-barb]{\mbox{$P\downarrow_{\mathcal N} x$ or $Q\downarrow_{\mathcal N} x$}}
		  {\binpar{P}{Q} \downarrow_{\mathcal N} x}

We write $P \Downarrow_{\mathcal N} x$ if there is $Q$ such that 
$P \wred Q$ and $Q \downarrow_{\mathcal N} x$.
\end{definition}

\begin{definition}
%\label{def.bbisim}
An  ${\mathcal N}$-\emph{barbed bisimulation} over a set of names, ${\mathcal N}$, is a symmetric binary relation 
${\mathcal S}_{\mathcal N}$ between agents such that $P\rel{S}_{\mathcal N}Q$ implies:
\begin{enumerate}
\item If $P \red P'$ then $Q \wred Q'$ and $P'\rel{S}_{\mathcal N} Q'$.
\item If $P\downarrow_{\mathcal N} x$, then $Q\Downarrow_{\mathcal N} x$.
\end{enumerate}
$P$ is ${\mathcal N}$-barbed bisimilar to $Q$, written
$P \wbbisim_{\mathcal N} Q$, if $P \rel{S}_{\mathcal N} Q$ for some ${\mathcal N}$-barbed bisimulation ${\mathcal S}_{\mathcal N}$.
\end{definition}

$\mathcal{R} \subseteq \pi \times \pi$

$P \mathcal{R} Q => \forall P'. P \red P' \Rightarrow \exists Q'. Q \red Q', P' \mathcal{R} Q'$

$P \vdash x \Rightarrow Q \vdash x$

\begin{mathpar}
  \inferrule*[lab=Out-barb]{x \nameeq y}{{y}!\langle{Q}\rangle \vdash x}
  \and
  \inferrule*[lab=Par-barb]{\mbox{$P\vdash x$ or $Q\vdash x$}}{\binpar{P}{Q} \vdash x}
\end{mathpar}

\subsubsection{Contexts}

One of the principle advantages of computational calculi like the
$\pi$-calculus is a well-defined notion of context,
contextual-equivalence and a correlation between
contextual-equivalence and notions of bisimulation. The notion of
context allows the decomposition of a process into (sub-)process and
its syntactic environment, its context. Thus, a context may be
thought of as a process with a ``hole'' (written $\Box$) in it. The
application of a context $M$ to a process $P$, written $M[P]$, is
tantamount to filling the hole in $M$ with $P$. In this paper we do
not need the full weight of this theory, but do make use of the notion
of context in the proof the main theorem. 

\begin{mathpar}
  \inferrule* [lab=summation] {} {{M_{M},M_{N}} \bc \Box \;|\; x.M_{A} \;|\; M_{M}+M_{N}}
  \and
  \inferrule* [lab=agent] {} {{M_{A}} \bc (\vec{x})M_{P} \;| \; \clift{P_0,\ldots,M_{P},\ldots,P_N}}
  \and \\
  \inferrule* [lab=process] {} {{M_{P}} \bc M_{N} \;| \;P|M_{P} }
\end{mathpar} 

\begin{mathpar}
  \inferrule* [lab=sychronization] {} {M_{N} \bc \Box \;|\; x?M_{F} \;|\; x!M_{C}}
  \and
  \inferrule* [lab=abstraction] {} {{M_{F}} \bc (x)M_{P} }
  \and
  \inferrule* [lab=concretion] {} {{M_{C}} \bc \langle M_{P} \rangle }
  \and \\
  \inferrule* [lab=process] {} {{M_{P}} \bc M_{N} \;| \;P|M_{P} }
\end{mathpar}

\begin{definition}[contextual application] Given a context $M$, and
  process $P$, we define the \emph{contextual application}, $M[P] :=
  M\{P/\Box\}$. That is, the contextual application of M to P is the
  substitution of $P$ for $\Box$ in $M$.
\end{definition}

$\meaningof{-} : L \to \mathcal{P}(\pi)$

\begin{mathpar}
  \inferrule* [lab=collection] {} {\meaningof{true} = \pi, \and \meaningof{~E} = \pi \setminus \meaningof{E}, \and \meaningof{E_{1} \& E_{2}} = \meaningof{E_{1}} \cap \meaningof{E_{2}}}
\end{mathpar}

\begin{mathpar}
  \inferrule* [lab=structure] {} {\meaningof{0} = \{ P \in \pi | P \equiv 0 \}, \and \\ \meaningof{E_1 | E_2} = \{ P \in \pi | P \equiv P_{1} | P_{2}, P_{1} \in \meaningof{E_{1}}, P_{2} \in \meaningof{E_2}\} }
\end{mathpar}

\begin{mathpar}
 \inferrule* [lab=behavior] {} {\meaningof{\langle a?b \rangle E} = \{ P \in \pi | P \equiv Q | u?(y)P', \\ \and \\\\ \and \\ \;\;\; u \in \meaningof{a}, \forall z.P'\{z/y\} \in \meaningof{E\{z/b\}}\}, \and \\ \meaningof{a!E} = \{ P \in \pi | P \equiv Q | x!\langle P' \rangle, x \in \meaningof{a} P' \in \meaningof{E}\} }
\end{mathpar}

\begin{mathpar}
 \inferrule* [lab=nominal] {} {\meaningof{\quotep{E}} = \{ \quotep{P} \in \quotep{\pi} | P \in \meaningof{E} \}, \and \meaningof{\quotep{P}} = \{ \quotep{Q} \in \quotep{\pi} | P \equiv Q \} \and \\ \meaningof{@\quotep{E}} = \{ P \in \pi | P \equiv @x, x \in \meaningof{E} \}}
\end{mathpar}

\begin{eqnarray*}
  \\
  \meaningof{-} : TS \to ST
\end{eqnarray*}

\begin{eqnarray*}
  \\
  L : TS \to ST
\end{eqnarray*}

\begin{eqnarray*}
  \\
  P \models E \iff P \in \meaningof{E}
\end{eqnarray*}

\begin{eqnarray*}
  P \approx_{L} Q \iff \forall E \in L. P \models E \iff Q \models E
\end{eqnarray*}

\begin{eqnarray*}
  P \approx_{K} Q
\end{eqnarray*}

\begin{eqnarray*}
  P \approx Q
\end{eqnarray*}

$\approx_{K} = \approx = \approx_{L}$

\subsubsection{Contextual duality}

Note that contexts extend the quotation operation to a family of
operations from processes to names. Given a context, $M$, we can
define a \emph{nominal context}, $\quotep{M}$ by $\quotep{M}[P] :=
\quotep{M[P]}$. To foreshadow what is to come we observe that these
operations enjoy a duality with processes very much like the duality
between vectors and maps from vectors to scalars.

Further, because the calculus is essentially higher-order, we have a
correspondence between contexts and processes. More specifically,
given a name $x$ and a context $M$ we can construct $M^{*}_{x}$ such
that 

\begin{mathpar}
  M^{*}_{x} | \lift{x}{P} \red M[P]
\end{mathpar}

namely,

\begin{mathpar}
  M^{*}_{x} := x?(u).M[\dropn{u}]
\end{mathpar}

The dependence of $M^{*}_{x}$ on a name makes it an abstraction, 

\begin{mathpar}
  M^{*} := (x)x?(u).M[\dropn{u}]
\end{mathpar}

\subsection{Additional notation}

It will sometimes be convenient to denote the process a name
quotes. We already have the notation $x = \quotep{P}$, but it will be
convenient to introduce an alternate notation, $\procn{x}$, when we
want to emphasize the connection to the use of the name. Note that, by
virtue of name equivalence, $\quotep{\procn{x}} \nameeq x$; so, the
notation is consistent with previous definitions.

Further, because names have structure it is possible to effect
substitutions on the basis of that structure. This means we need to
upgrade our notation for substitutions, which we accomplish by
adapting comprehension notation. Thus,

\begin{mathpar}
  P\{ y / x : x \in S \}
\end{mathpar}

is interpreted to mean the process derived from P by replacing (in a
capture-avoiding manner) each occurrence of $x$ in $S$ by $y$. For example,

\begin{mathpar}
  P\{ \quotep{\procn{x}|\procn{x}} / x : x \in \freenames{P} \}
\end{mathpar}

will replace each (occurrence) of a free name $x$ in $P$ by
$\quotep{\procn{x}|\procn{x}}$.

Also, we will avail ourselves of the notation $x^{L}$ and $x^{R}$ to
denote injections of a name into disjoint copies of the name
space. There are numerous ways to accomplish this. One example can be
found in \cite{MeredithR05}. This notation overloads to vectors of
names: $\vec{x}^{\pi} := (x_{i}^{\pi} \; : \; 0 \leq i < |\vec{x}| )$ where $\pi \in \{L,R\}$.

We also use $P^{\Box} := P|\Box$.

In \cite{MeredithR05} an interpretation of the new operator is
given. It turns out that there are several possible interpretations
all enjoying the requisite algebraic properties of the operator (see
\cite{milner91polyadicpi}). We will therefore make liberal use of
$(\nu\; \vec{x})P$.

% subsection the_syntax_and_semantics_of_the_notation_system (end)   

\input{qm2pi.qmops} 

\input{qm2pi.sterngerlach} 

\input{qm2pi.metric} 

% section concurrent_process_calculi (end)

%\input{qm2pi.proofsketch}

% section proof sketch (end)

%\input{qm2pi.slviaknots} 

% section spatial logic via knots (end)

\input{qm2pi.conclusion}

% section conclusion (end)

%\input{qm2pi.dtcodes} 

% section wiring algorithm (end)

\input{qm2pi.ack} 

% section acknowledgments (end)

\newpage


\bibliographystyle{plain}   
\bibliography{../../biblios/main.bib}

\input{qm2pi.rhodetails}

\end{document}



% section front matter (end)

\section{Introduction}\label{sec:introduction} % (fold)
In this draft of the material i am going to have to dispense with the
usual writing conventions adopted in papers on these topics. i'm going
to have adopt whatever tone i need at the time i'm writing up the
calculations. Sometimes this may be very conversational; others it may
be the barest mathematical grunts; others still it may be that i have
lifted text from one of my other papers because the exposition of some
point was better said there. i hope that my readers are not unduly put
out by this decision. i'm not doing this to flout convention or be
rebellious. i find these calculations very technically challenging. To
keep everything going technically, something has to give; i have to
let go of some cognitive burden. So, the academic writing style --
with all of its trade-offs in terms of facilitating technical
communication -- is what i'm letting go of. Perhaps subsequent drafts
can be tightened and polished, but for now, i'm going to speak as if
we were sitting together in a coffee shop with a laptop, wifi and a
pad of paper and a pencil.

So, here's what i have to say. We -- you and i, comfortably ensconced
in our coffee shop and well-equipped with our tools -- can realize and
carry out the calculations of quantum mechanics over a very different
formal theory of dynamics, a formal theory of dynamics that
corresponds to a theory of concurrent computation with
\emph{reflection}. It has the advantage that the underlying theory is
already `quantized', but supports analogues all of the continuuous
operations. Strikingly, this underlying theory has recently been
connected with a notion of metric that we can show, by calculating
together, coincides with the metric induced by the inner product.

There are a lot of reasons why you might be interested in seeing
calculations of this form. Here's why i'm interested. For the past
several centuries there has been no competitor to the ``Newtonian''
account of dynamics. As a result the predominant share of accounts of
dynamical systems and situations have had to be formulated in terms of
the Newtonian machinery. i view this as an intellectually dangerous
position to occupy. Everything, despite it's intrinsic shape, turns
into a nail to be hit with this hammer. Recently, however, the theory
of computation has matured to the point where we have candidates for
theories of dynamics that offer very different perspective on
reasoning about dynamical systems and situations. Testing these
candidates against very successful accounts of dynamical situations,
like quantum mechanics, is going to give us some sense of how mature
they are and some measure of the quality of these accounts of
dynamics.

\subsection{Summary of contributions and outline of paper}

So, we're going to develop an interpretation of the operations of
quantum mechanics normally interpreted by Hilbert spaces and
operators. We're going to do this over a theory of computation. Note
that this is very different than the usual quantum computation program
which develops notions of computation over quantum mechanics. Rather,
we are developing a story that aligns with Wheeler's slogan: It from
Bit. To do this we will first provide an account of the theory of
computation at play here. Then we will dive into a calculation-driven
interpretation of the operations of quantum mechanics.

The reason we take this approach is that -- until very recently --
there hasn't been an axiomatic account of quantum mechanics. As a
result there has been no sharp delineation of the mathematical theory
supporting interpretation of the physical theory and the physical
theory, itself. So, ambient features of the maths are free to be
exploited (or supressed) without a real accounting of their physical
relevance. There is no sharp statement ``here's the physical theory''
qua \emph{theory} and ``here's the mathematical interpretation''
enabling a judgment of how faithful the interpretation is -- apart
from experimental observation. When there is an axiomatic account we
can judge how well a given mathematical formalism supports an
interpretation of the axioms, independent of
experimentation. Likewise, we can judge how well we have captured our
physical evidence and experience with our axiomatics, independent of
any specific mathematical implementation, with accidental detail that
may or may not have physical significance. 

In lieu of a fully fleshed out and vetted axiomatic account of quantum
mechanics, interpreting the operational notions in service of modeling
physical systems will have to suffice. In other words, we are not in
the business of providing a model of Hilbert spaces and operators. We
are in the business of providing a model of quantum mechanics because
we are motivated by testing our notions of dynamics against physical
theory; and, the predictive calculations of the physical theory must
serve as the best formulation -- shy of a fully fleshed out axiomatic
account -- of the physical theory itself (as they have for scientific
theories since time immemorial). Put another way, despite a
whole-hearted commitment to an It-from-Bit ontology, we are firmly
aligned with the shut-up-and-calculate camp as the best way to obtain
results either from the physical perspective or as a quality assurance
measure of our fledgling theory of dynamics.

In detail, we present a reflective process calculus. Then we develop
intuitive correspondences between the notions available in this
calculus and the usual physical notions supporting quantum mechanical
calculations. Thus, 

\begin{table}[htp]
  \center{
    \fbox{
      \begin{tabular}{c|c}
        quantum mechanics & process calculus \\
        \hline
        scalar & name \\
        state vector & process \\
        dual & contextual duals \\
        matrix & formal sums of process-context-dual pairs \\
        orthogonality & process annihilation \\
        inner product & execution-formula + quoting
      \end{tabular}
    }
  }
  \caption{QM - process calculi correspondences}
\end{table}

Then we tighten up these intuitions to operational definitions. We
employ the Dirac notation as the best proxy we can find for an
abstract syntax of the quantum mechanical notions. The definitions we
develop put us in contact with equational constraints coming from the
theory that we demonstrate the definitions and calculations satisfy.

This puts us in a position to shut up and calculate for the
Stern-Gerlach experimental set up, showing how these predictive
calculations become calculations on processes in our theory of a
reflective process calculus.

Penultimately, we demonstrate that the notion of metric coming from
the inner product coincides with the notion of metric available from
the theory of bisimulation. This demonstration gives us the right to
think of space as arising from behavior. Finally, we consider where we
might go from the new vantage point we have obtained.

% section introduction (end) 
 
% section introduction (end)

% \documentclass[12pt]{llncs}
%\documentclass{jktr}

\usepackage[pdftex]{hyperref}                   
\usepackage {listings}
\usepackage {mathpartir}
\usepackage{bcprules}
%\usepackage{listings}
                       
\usepackage{graphicx} 
%\usepackage[margins=2.5cm,nohead,nofoot]{geometry}
%\usepackage{geometry}
\usepackage{amsfonts}
\usepackage{amstext}
\usepackage{latexsym}
\usepackage{amssymb}
\usepackage{color}


%\include{myPreamble}
\include{qm2pi.local} 

%\ifpdf
%\usepackage[pdftex]{graphicx}
%\else
%\usepackage{graphicx}
%\fi

 % \ifpdf
%  \usepackage{pdfsync}
%  \if


%\title{Brief Article}
%\author{David F. Snyder}
%\author{L.G. Meredith}

%\address{Dept. of Math., Texas State University--San Marcos, San Marcos, TX 78666}
       
\pagestyle{empty}


\begin{document}

\lstset{language=[Objective]Caml,frame=shadowbox}

\input{qm2pi.front}

% section front matter (end)

\input{qm2pi.intro} 
 
% section introduction (end)

% \input{qm2pi.knotations} 

% section notation (end)

\input{qm2pi.process.calculi} 

% section concurrent_process_calculi_and_spatial_logics_ (end)
    
%\input{qm2pi.knots2pi} 

%\input{qm2pi.trefoil} 

%\input{qm2pi.mainthm} 

% subsection basic_interpretation (end)

%\input{qm2pi.rho.presentation} 
\subsection{The syntax and semantics of the notation system}\label{sub:the_syntax_and_semantics_of_the_notation_system} % (fold)

We now summarize a technical presentation of the calculus that
embodies our theory of dynamics. The typical presentation of such a
calculus follows the style of giving generators and relations on
them. The grammar, below, describing term constructors, freely
generates the set of processes, $\Proc$. This set is then quotiented
by a relation known as structural congruence and it is over this set
that the notion of dynamics is expressed. This presentation is
essentially that of \cite{MeredithR05} with the addition of
polyadicity and summation. For readability we have relegated some of
the technical subtleties to an appendix.

\subsubsection{Process grammar}\label{subsub:process_grammar}

\begin{mathpar}
  \inferrule* [lab=synchronization] {} {{M} \bc \pzero \;|\; x?F \;|\; x!C }
  \and
  \inferrule* [lab=abstraction] {} {{F} \bc (x)P}
  \and
  \inferrule* [lab=concretion] {} {{C} \bc \langle Q \rangle}
  \and
  \inferrule* [lab=process] {} {{P,Q} \bc M \;| \;P|Q \;|\; @{x}}
  \and
  \inferrule* [lab=name] {} {{x} \bc \quotep{P}}
\end{mathpar} 

Note that $\vec{x}$ (resp. $\vec{P}$) denotes a vector of names
(resp. processes) of length $|\vec{x}|$ (resp. $|\vec{P}|$). We adopt
the following useful abbreviations.

\begin{mathpar}
   x?(\vec{y}).P := x.(\vec{y})P \and  x\clift{\vec{P}} := x.\clift{\vec{P}}
   \and x!(y) := \lift{x}{\dropn{y}}
   \and \Pi_{i=0}^{n-1}P_i := P_0 | \ldots | P_{n-1}
\end{mathpar}

\subsubsection{Structural congruence}

\paragraph{Free and bound names and alpha-equivalence.} At the
core of structural equivalence is alpha-equivalence which identifies
process that are the same up to a change of variable. Formally, we
recognize the distinction between free and bound names. The free names
of a process, $\freenames{P}$, may be calculated recursively as
follows:

\begin{mathpar}
\freenames{\pzero} := \emptyset
  \and \\
  \freenames{x?(y).P} := \{ x \} \cup (\freenames{P} \setminus \{ y \})
  \and 
  \freenames{x!\langle P \rangle} := \{ x \} \cup \{ P \} 
  \and \\
  \freenames{P|Q} := \freenames{P} \cup \freenames{Q}
  \and \\
  \freenames{@{x}} := \{ x \}
\end{mathpar}

$\pi$
$\quotep{\pi}$

$\freenames{-} : \pi \to \mathcal{P}(\quotep{\pi})$

\begin{eqnarray*}
  \freenames{\pzero} & := & \emptyset \\
  \freenames{x?(y).P} & := & \{ x \} \cup (\freenames{P} \setminus \{ y \}) \\
  \freenames{x!\langle P \rangle} & := & \{ x \} \cup \{ P \} \\
  \freenames{P|Q} & := & \freenames{P} \cup \freenames{Q} \\
  \freenames{\dropn{x}} & := & \{ x \}
\end{eqnarray*}

The bound names of a process, $\boundnames{P}$, are those names occurring in $P$
that are not free. For example, in $x?(y).0$, the name $x$ is free, while $y$ is bound.

\begin{mathpar}
  \inferrule* [lab=monoidal-laws] {} { P|Q \equiv Q|P \and P|0 \equiv P \and P|(Q|R) \equiv (P|Q)|R }
\end{mathpar}

\begin{mathpar}
  \inferrule* [lab=alpha-equivalence] {} { (x)P \equiv (y)P\{y/x\} \and y \not\in \freenames{P} }
\end{mathpar}

\begin{definition}
Then two processes, $P,Q$, are alpha-equivalent if $P = Q\{\vec{y}/\vec{x}\}$ for
some $\vec{x} \in \boundnames{Q},\vec{y} \in \boundnames{P}$, where $Q\{\vec{y}/\vec{x}\}$
denotes the capture-avoiding substitution of $\vec{y}$ for $\vec{x}$ in $Q$.
\end{definition}

\begin{definition}
  The {\em structural congruence} \cite{SangiorgiWalker} , $\equiv$,
  between processes is the least congruence containing
  alpha-equivalence, satisfying the abelian monoid laws
  (associativity, commutativity and $\pzero$ as identity) for parallel
  composition $|$ and for summation $+$.
\end{definition}

\subsection{Name equivalence}

We take name equivalence, written $\nameeq$, to be the smallest
equivalence relation generated by the following rules.

\begin{mathpar}
\inferrule*[lab=Quote-drop]
{ }
{ \quotep{@{x}} \nameeq x }

\inferrule*[lab=Struct-equiv]
{ P \scong Q }
{ \quotep{P} \nameeq \quotep{Q} }
\end{mathpar}

The astute reader will have noticed that the mutual recursion of names
and processes imposes a mutual recursion on alpha-equivalence and
structural equivalence via name-equivalence. Fortunately, all of this
works out pleasantly and we may calculate in the natural way, free of
concern. The reader interested in the details is referred to the
appendix \ref{appendix:rho_details}.

\subsection{Substitution}

We use $\Proc$ for the set of processes, $\QProc$ for the set of
names, and $\id{\{}\vec{y} / \vec{x} \id{\}}$ to denote partial maps,
$s : \QProc \rightarrow \QProc$. A map, $s$ lifts, uniquely, to a map
on process terms, $\widehat{s} : \Proc \rightarrow \Proc$ by the
following equations.

\begin{mathpar}
  (0) \psubstp{Q}{P} := 0 \\
  (R \juxtap S) \psubstp{Q}{P}
  :=    
  (R)\psubstp{Q}{P} \juxtap (S) \psubstp{Q}{P} \\
  (x?(y).R) \psubstp{Q}{P}    
  :=    
  (x)\substp{Q}{P} (z)\concat( (R \psubstn{z}{y}) \psubstp{Q}{P} ) \\
  (\lift{x}{R}) \psubstp{Q}{P}  
  :=
  \lift{(x)\substp{Q}{P}}{ R \psubstp{Q}{P} } \\
%   (\dropn{x})  \psubstp{Q}{P}       
%   := 
%   \left\{ 
%     \begin{array}{ccc} 
%       \dropn{\quotep{Q}} & & x \nameeq \quotep{P} \\
%       \dropn{x} & & otherwise \\
%     \end{array}
%   \right. 
  (\dropn{x})  \psubstp{Q}{P}       
  := 
  \left\{ 
    \begin{array}{ccc} 
      Q & & x \nameeq \quotep{P} \\
      \dropn{x} & & otherwise \\
    \end{array}
  \right.
\end{mathpar}
 

where

\begin{eqnarray}
  (x)\id{\{} \lpquote Q \rpquote / \lpquote P \rpquote \id{\}}            = 
  \left\{ 
    \begin{array}{ccc}
      \lpquote Q \rpquote & & x \nameeq \lpquote P \rpquote \\
      x & & otherwise \\
    \end{array}
  \right. \nonumber
\end{eqnarray}

and $z$ is chosen distinct from $\quotep{P}$, $\quotep{Q}$, the free
names in $Q$, and all the names in $R$. Our $\alpha$-equivalence will
be built in the standard way from this substitution.

\begin{remark}\label{rem:no_self_referential_names}
  One consequence of these definitions is that $\forall P. \quotep{P}
  \not\in \freenames{P}$.
\end{remark}

\subsection{ Dynamic quote: an example }

Anticipating something of what's to come, consider applying the
substitution, $\widehat{\id{\{}u / z \id{\}}}$, to the following pair
of processes, $\lift{w}{y!(z)}$ and $w[ \lpquote y!(z) \rpquote ]$.

\begin{eqnarray}
	\lift{w}{y!(z)}\widehat{\id{\{}u / z \id{\}}}
		& = &
		\lift{w}{y!(u)} \nonumber\\
	w[ \lpquote y!(z) \rpquote ] \widehat{ \id{\{}u / z \id{\}} }
		& = &
		w[ \lpquote y!(z) \rpquote ] \nonumber
\end{eqnarray}

Because the body of the process between quotes is impervious to
substitution, we get radically different answers. In fact, by
examining the first process in an input context,
e.g. $x?(z).\lift{w}{y!(z)}$, we see that the process under the lift
operator may be shaped by prefixed inputs binding a name inside it. In
this sense, the lift operator will be seen as a way to dynamically
construct processes before reifying them as names.

Finally equipped with these standard features we can present the
dynamics of the calculus.

\subsubsection{Operational semantics} 

Finally, we introduce the computational dynamics. What marks these
algebras as distinct from other more traditionally studied algebraic
structures, e.g. vector spaces or polynomial rings, is the manner in
which dynamics is captured. In traditional structures, dynamics is typically
expressed through morphisms between such structures, as in linear maps
between vector spaces or morphisms between rings. In algebras
associated with the semantics of computation, the dynamics is
expressed as part of the algebraic structure itself, through a
reduction reduction relation typically denoted by $\red$. Below, we
give a recursive presentation of this relation for the calculus used
in the encoding.

$\red \subseteq \pi \times \pi$
$\red : \pi \to \mathcal{P}(\pi)$

\begin{mathpar}
  \inferrule* [lab=Comm] { \textsf{match}( x_{src}, x_{trgt} ) } { x_{trgt}?(y)P \; | \; x_{src}!\langle {Q} \rangle \red P\{\quotep{Q}/y}\} }
  \and \\
  \inferrule* [lab=Par] {{P} \red {P}'} {{{P} | {Q}} \red {{P}' | {Q}}}
  \and
  \inferrule* [lab=Equiv]{{{P} \scong {P}'} \andalso {{P}' \red {Q}'} \andalso {{Q}' \scong {Q}}}{{P} \red {Q}}
\end{mathpar}

\begin{eqnarray*}
  match_{\equiv} (\quotep{P},\quotep{Q}) & := & P \equiv Q \\
  match_{\dagger}(\quotep{P},\quotep{Q}) & := & \forall R. P|Q \red^{*} R => R \red^{*} 0 \\
  match_{K}(\quotep{P},\quotep{Q}) & := & K \mbox{ for some context } K
\end{eqnarray*}

$u?(x)P | u!\langle Q \rangle \red P\{\quotep{Q}/x\}$

%We write $\wred$ for $\red^*$, and $P\red$ if $\exists Q $ such that $ P \red Q$.
We write $P\red$ if $\exists Q $ such that $ P \red Q$ and $P\not\red$, otherwise.

\section{Replication}

As mentioned before, it is known that replication (and hence
recursion) can be implemented in a higher-order process algebra
\cite{SangiorgiWalker}. As our first example of calculation with the
machinery thus far presented we give the construction explicitly in
the {\rhoc}.

\begin{eqnarray}
	D_{x} & := & \prefix{x}{y}{(\binpar{\outputp{x}{y}}{@{y}})} \nonumber\\
	\bangp_{x}{P} & := & \binpar{{x}!\langle{\binpar{D_{x}}{P}}\rangle}{D_{x}} \nonumber
\end{eqnarray}

\begin{eqnarray}
	\bangp_{x}{P} & & \nonumber\\
	=
	& {x}!\langle{(\prefix{x}{y}{(\outputp{x}{y} | @{y})) | P}}\rangle 
	      | \prefix{x}{y}{(\outputp{x}{y} | @{y})} & \nonumber\\
	\red
	& (\outputp{x}{y} | @{y})\substn{\quotep{(\prefix{x}{y}{(@{y} | \outputp{x}{y})) | P}}}{y} & \nonumber\\
	=
	& \outputp{x}{\quotep{(\prefix{x}{y}{(\outputp{x}{y} | @{y})) | P}}}
	  | {(\prefix{x}{y}{(\outputp{x}{y} | @{y})) | P}} & \nonumber\\
	\red
	& \ldots & \nonumber\\
	\red^*
	& P | P | \ldots & \nonumber
\end{eqnarray}

Of course, this encoding, as an implementation, runs away, unfolding
$\bangp{P}$ eagerly. A lazier and more implementable replication
operator, restricted to input-guarded processes, may be obtained as follows.

\begin{eqnarray}
\bangp{\prefix{u}{v}{P}} 
	:= 
	\binpar{\lift{x}{\prefix{u}{v}{(\binpar{D(x)}{P})}}}{D(x)} \nonumber
\end{eqnarray}

\begin{remark}
  Note that the lazier definition still does not deal with summation
  or mixed summation (i.e. sums over input and output). The reader is
  invited to construct definitions of replication that deal with these
  features. 

  Further, the definitions are parameterized in a name, $x$. Can you,
  gentle reader, make a definition that eliminates this parameter and
  guarantees no accidental interaction between the replication
  machinery and the process being replicated -- i.e. no accidental
  sharing of names used by the process to get its work done and the
  name(s) used by the replication to effect copying. This latter
  revision of the definition of replication is crucial to obtaining
  the expected identity $!!P \sim !P$.
\end{remark}

\begin{remark}\label{rem:paradoxical_combinator}
  The reader familiar with the lambda calculus will have noticed the
  similarity between $D$ and the paradoxical combinator.

  [Ed. note: the existence of this seems to suggest we have to be more
  restrictive on the set of processes and names we admit if we are to
  support no-cloning.]
\end{remark}

\subsubsection{Bisimulation}

The computational dynamics gives rise to another kind of equivalence,
the equivalence of computational behavior. As previously mentioned
this is typically captured \emph{via} some form of bisimulation.

% The notion we use in this paper is weak barbed bisimulation
% \cite{milner91polyadicpi}.

The notion we use in this paper is derived from weak barbed
bisimulation \cite{milner91polyadicpi}. 

\begin{definition}
An \emph{observation relation}, $\downarrow_{\mathcal N}$, over a set
of names, $\mathcal N$, is the smallest relation satisfying the rules
below.

\infrule[Out-barb]{y \in {\mathcal N}, \; x \nameeq y}
		  {\outputp{x}{v} \downarrow_{\mathcal N} x}
\infrule[Par-barb]{\mbox{$P\downarrow_{\mathcal N} x$ or $Q\downarrow_{\mathcal N} x$}}
		  {\binpar{P}{Q} \downarrow_{\mathcal N} x}

We write $P \Downarrow_{\mathcal N} x$ if there is $Q$ such that 
$P \wred Q$ and $Q \downarrow_{\mathcal N} x$.
\end{definition}

\begin{definition}
%\label{def.bbisim}
An  ${\mathcal N}$-\emph{barbed bisimulation} over a set of names, ${\mathcal N}$, is a symmetric binary relation 
${\mathcal S}_{\mathcal N}$ between agents such that $P\rel{S}_{\mathcal N}Q$ implies:
\begin{enumerate}
\item If $P \red P'$ then $Q \wred Q'$ and $P'\rel{S}_{\mathcal N} Q'$.
\item If $P\downarrow_{\mathcal N} x$, then $Q\Downarrow_{\mathcal N} x$.
\end{enumerate}
$P$ is ${\mathcal N}$-barbed bisimilar to $Q$, written
$P \wbbisim_{\mathcal N} Q$, if $P \rel{S}_{\mathcal N} Q$ for some ${\mathcal N}$-barbed bisimulation ${\mathcal S}_{\mathcal N}$.
\end{definition}

$\mathcal{R} \subseteq \pi \times \pi$

$P \mathcal{R} Q => \forall P'. P \red P' \Rightarrow \exists Q'. Q \red Q', P' \mathcal{R} Q'$

$P \vdash x \Rightarrow Q \vdash x$

\begin{mathpar}
  \inferrule*[lab=Out-barb]{x \nameeq y}{{y}!\langle{Q}\rangle \vdash x}
  \and
  \inferrule*[lab=Par-barb]{\mbox{$P\vdash x$ or $Q\vdash x$}}{\binpar{P}{Q} \vdash x}
\end{mathpar}

\subsubsection{Contexts}

One of the principle advantages of computational calculi like the
$\pi$-calculus is a well-defined notion of context,
contextual-equivalence and a correlation between
contextual-equivalence and notions of bisimulation. The notion of
context allows the decomposition of a process into (sub-)process and
its syntactic environment, its context. Thus, a context may be
thought of as a process with a ``hole'' (written $\Box$) in it. The
application of a context $M$ to a process $P$, written $M[P]$, is
tantamount to filling the hole in $M$ with $P$. In this paper we do
not need the full weight of this theory, but do make use of the notion
of context in the proof the main theorem. 

\begin{mathpar}
  \inferrule* [lab=summation] {} {{M_{M},M_{N}} \bc \Box \;|\; x.M_{A} \;|\; M_{M}+M_{N}}
  \and
  \inferrule* [lab=agent] {} {{M_{A}} \bc (\vec{x})M_{P} \;| \; \clift{P_0,\ldots,M_{P},\ldots,P_N}}
  \and \\
  \inferrule* [lab=process] {} {{M_{P}} \bc M_{N} \;| \;P|M_{P} }
\end{mathpar} 

\begin{mathpar}
  \inferrule* [lab=sychronization] {} {M_{N} \bc \Box \;|\; x?M_{F} \;|\; x!M_{C}}
  \and
  \inferrule* [lab=abstraction] {} {{M_{F}} \bc (x)M_{P} }
  \and
  \inferrule* [lab=concretion] {} {{M_{C}} \bc \langle M_{P} \rangle }
  \and \\
  \inferrule* [lab=process] {} {{M_{P}} \bc M_{N} \;| \;P|M_{P} }
\end{mathpar}

\begin{definition}[contextual application] Given a context $M$, and
  process $P$, we define the \emph{contextual application}, $M[P] :=
  M\{P/\Box\}$. That is, the contextual application of M to P is the
  substitution of $P$ for $\Box$ in $M$.
\end{definition}

$\meaningof{-} : L \to \mathcal{P}(\pi)$

\begin{mathpar}
  \inferrule* [lab=collection] {} {\meaningof{true} = \pi, \and \meaningof{~E} = \pi \setminus \meaningof{E}, \and \meaningof{E_{1} \& E_{2}} = \meaningof{E_{1}} \cap \meaningof{E_{2}}}
\end{mathpar}

\begin{mathpar}
  \inferrule* [lab=structure] {} {\meaningof{0} = \{ P \in \pi | P \equiv 0 \}, \and \\ \meaningof{E_1 | E_2} = \{ P \in \pi | P \equiv P_{1} | P_{2}, P_{1} \in \meaningof{E_{1}}, P_{2} \in \meaningof{E_2}\} }
\end{mathpar}

\begin{mathpar}
 \inferrule* [lab=behavior] {} {\meaningof{\langle a?b \rangle E} = \{ P \in \pi | P \equiv Q | u?(y)P', \\ \and \\\\ \and \\ \;\;\; u \in \meaningof{a}, \forall z.P'\{z/y\} \in \meaningof{E\{z/b\}}\}, \and \\ \meaningof{a!E} = \{ P \in \pi | P \equiv Q | x!\langle P' \rangle, x \in \meaningof{a} P' \in \meaningof{E}\} }
\end{mathpar}

\begin{mathpar}
 \inferrule* [lab=nominal] {} {\meaningof{\quotep{E}} = \{ \quotep{P} \in \quotep{\pi} | P \in \meaningof{E} \}, \and \meaningof{\quotep{P}} = \{ \quotep{Q} \in \quotep{\pi} | P \equiv Q \} \and \\ \meaningof{@\quotep{E}} = \{ P \in \pi | P \equiv @x, x \in \meaningof{E} \}}
\end{mathpar}

\begin{eqnarray*}
  \\
  \meaningof{-} : TS \to ST
\end{eqnarray*}

\begin{eqnarray*}
  \\
  L : TS \to ST
\end{eqnarray*}

\begin{eqnarray*}
  \\
  P \models E \iff P \in \meaningof{E}
\end{eqnarray*}

\begin{eqnarray*}
  P \approx_{L} Q \iff \forall E \in L. P \models E \iff Q \models E
\end{eqnarray*}

\begin{eqnarray*}
  P \approx_{K} Q
\end{eqnarray*}

\begin{eqnarray*}
  P \approx Q
\end{eqnarray*}

$\approx_{K} = \approx = \approx_{L}$

\subsubsection{Contextual duality}

Note that contexts extend the quotation operation to a family of
operations from processes to names. Given a context, $M$, we can
define a \emph{nominal context}, $\quotep{M}$ by $\quotep{M}[P] :=
\quotep{M[P]}$. To foreshadow what is to come we observe that these
operations enjoy a duality with processes very much like the duality
between vectors and maps from vectors to scalars.

Further, because the calculus is essentially higher-order, we have a
correspondence between contexts and processes. More specifically,
given a name $x$ and a context $M$ we can construct $M^{*}_{x}$ such
that 

\begin{mathpar}
  M^{*}_{x} | \lift{x}{P} \red M[P]
\end{mathpar}

namely,

\begin{mathpar}
  M^{*}_{x} := x?(u).M[\dropn{u}]
\end{mathpar}

The dependence of $M^{*}_{x}$ on a name makes it an abstraction, 

\begin{mathpar}
  M^{*} := (x)x?(u).M[\dropn{u}]
\end{mathpar}

\subsection{Additional notation}

It will sometimes be convenient to denote the process a name
quotes. We already have the notation $x = \quotep{P}$, but it will be
convenient to introduce an alternate notation, $\procn{x}$, when we
want to emphasize the connection to the use of the name. Note that, by
virtue of name equivalence, $\quotep{\procn{x}} \nameeq x$; so, the
notation is consistent with previous definitions.

Further, because names have structure it is possible to effect
substitutions on the basis of that structure. This means we need to
upgrade our notation for substitutions, which we accomplish by
adapting comprehension notation. Thus,

\begin{mathpar}
  P\{ y / x : x \in S \}
\end{mathpar}

is interpreted to mean the process derived from P by replacing (in a
capture-avoiding manner) each occurrence of $x$ in $S$ by $y$. For example,

\begin{mathpar}
  P\{ \quotep{\procn{x}|\procn{x}} / x : x \in \freenames{P} \}
\end{mathpar}

will replace each (occurrence) of a free name $x$ in $P$ by
$\quotep{\procn{x}|\procn{x}}$.

Also, we will avail ourselves of the notation $x^{L}$ and $x^{R}$ to
denote injections of a name into disjoint copies of the name
space. There are numerous ways to accomplish this. One example can be
found in \cite{MeredithR05}. This notation overloads to vectors of
names: $\vec{x}^{\pi} := (x_{i}^{\pi} \; : \; 0 \leq i < |\vec{x}| )$ where $\pi \in \{L,R\}$.

We also use $P^{\Box} := P|\Box$.

In \cite{MeredithR05} an interpretation of the new operator is
given. It turns out that there are several possible interpretations
all enjoying the requisite algebraic properties of the operator (see
\cite{milner91polyadicpi}). We will therefore make liberal use of
$(\nu\; \vec{x})P$.

% subsection the_syntax_and_semantics_of_the_notation_system (end)   

\input{qm2pi.qmops} 

\input{qm2pi.sterngerlach} 

\input{qm2pi.metric} 

% section concurrent_process_calculi (end)

%\input{qm2pi.proofsketch}

% section proof sketch (end)

%\input{qm2pi.slviaknots} 

% section spatial logic via knots (end)

\input{qm2pi.conclusion}

% section conclusion (end)

%\input{qm2pi.dtcodes} 

% section wiring algorithm (end)

\input{qm2pi.ack} 

% section acknowledgments (end)

\newpage


\bibliographystyle{plain}   
\bibliography{../../biblios/main.bib}

\input{qm2pi.rhodetails}

\end{document}

 

% section notation (end)

\input{qm2pi.process.calculi} 

% section concurrent_process_calculi_and_spatial_logics_ (end)
    
%\documentclass[12pt]{llncs}
%\documentclass{jktr}

\usepackage[pdftex]{hyperref}                   
\usepackage {listings}
\usepackage {mathpartir}
\usepackage{bcprules}
%\usepackage{listings}
                       
\usepackage{graphicx} 
%\usepackage[margins=2.5cm,nohead,nofoot]{geometry}
%\usepackage{geometry}
\usepackage{amsfonts}
\usepackage{amstext}
\usepackage{latexsym}
\usepackage{amssymb}
\usepackage{color}


%\include{myPreamble}
\include{qm2pi.local} 

%\ifpdf
%\usepackage[pdftex]{graphicx}
%\else
%\usepackage{graphicx}
%\fi

 % \ifpdf
%  \usepackage{pdfsync}
%  \if


%\title{Brief Article}
%\author{David F. Snyder}
%\author{L.G. Meredith}

%\address{Dept. of Math., Texas State University--San Marcos, San Marcos, TX 78666}
       
\pagestyle{empty}


\begin{document}

\lstset{language=[Objective]Caml,frame=shadowbox}

\input{qm2pi.front}

% section front matter (end)

\input{qm2pi.intro} 
 
% section introduction (end)

% \input{qm2pi.knotations} 

% section notation (end)

\input{qm2pi.process.calculi} 

% section concurrent_process_calculi_and_spatial_logics_ (end)
    
%\input{qm2pi.knots2pi} 

%\input{qm2pi.trefoil} 

%\input{qm2pi.mainthm} 

% subsection basic_interpretation (end)

%\input{qm2pi.rho.presentation} 
\subsection{The syntax and semantics of the notation system}\label{sub:the_syntax_and_semantics_of_the_notation_system} % (fold)

We now summarize a technical presentation of the calculus that
embodies our theory of dynamics. The typical presentation of such a
calculus follows the style of giving generators and relations on
them. The grammar, below, describing term constructors, freely
generates the set of processes, $\Proc$. This set is then quotiented
by a relation known as structural congruence and it is over this set
that the notion of dynamics is expressed. This presentation is
essentially that of \cite{MeredithR05} with the addition of
polyadicity and summation. For readability we have relegated some of
the technical subtleties to an appendix.

\subsubsection{Process grammar}\label{subsub:process_grammar}

\begin{mathpar}
  \inferrule* [lab=synchronization] {} {{M} \bc \pzero \;|\; x?F \;|\; x!C }
  \and
  \inferrule* [lab=abstraction] {} {{F} \bc (x)P}
  \and
  \inferrule* [lab=concretion] {} {{C} \bc \langle Q \rangle}
  \and
  \inferrule* [lab=process] {} {{P,Q} \bc M \;| \;P|Q \;|\; @{x}}
  \and
  \inferrule* [lab=name] {} {{x} \bc \quotep{P}}
\end{mathpar} 

Note that $\vec{x}$ (resp. $\vec{P}$) denotes a vector of names
(resp. processes) of length $|\vec{x}|$ (resp. $|\vec{P}|$). We adopt
the following useful abbreviations.

\begin{mathpar}
   x?(\vec{y}).P := x.(\vec{y})P \and  x\clift{\vec{P}} := x.\clift{\vec{P}}
   \and x!(y) := \lift{x}{\dropn{y}}
   \and \Pi_{i=0}^{n-1}P_i := P_0 | \ldots | P_{n-1}
\end{mathpar}

\subsubsection{Structural congruence}

\paragraph{Free and bound names and alpha-equivalence.} At the
core of structural equivalence is alpha-equivalence which identifies
process that are the same up to a change of variable. Formally, we
recognize the distinction between free and bound names. The free names
of a process, $\freenames{P}$, may be calculated recursively as
follows:

\begin{mathpar}
\freenames{\pzero} := \emptyset
  \and \\
  \freenames{x?(y).P} := \{ x \} \cup (\freenames{P} \setminus \{ y \})
  \and 
  \freenames{x!\langle P \rangle} := \{ x \} \cup \{ P \} 
  \and \\
  \freenames{P|Q} := \freenames{P} \cup \freenames{Q}
  \and \\
  \freenames{@{x}} := \{ x \}
\end{mathpar}

$\pi$
$\quotep{\pi}$

$\freenames{-} : \pi \to \mathcal{P}(\quotep{\pi})$

\begin{eqnarray*}
  \freenames{\pzero} & := & \emptyset \\
  \freenames{x?(y).P} & := & \{ x \} \cup (\freenames{P} \setminus \{ y \}) \\
  \freenames{x!\langle P \rangle} & := & \{ x \} \cup \{ P \} \\
  \freenames{P|Q} & := & \freenames{P} \cup \freenames{Q} \\
  \freenames{\dropn{x}} & := & \{ x \}
\end{eqnarray*}

The bound names of a process, $\boundnames{P}$, are those names occurring in $P$
that are not free. For example, in $x?(y).0$, the name $x$ is free, while $y$ is bound.

\begin{mathpar}
  \inferrule* [lab=monoidal-laws] {} { P|Q \equiv Q|P \and P|0 \equiv P \and P|(Q|R) \equiv (P|Q)|R }
\end{mathpar}

\begin{mathpar}
  \inferrule* [lab=alpha-equivalence] {} { (x)P \equiv (y)P\{y/x\} \and y \not\in \freenames{P} }
\end{mathpar}

\begin{definition}
Then two processes, $P,Q$, are alpha-equivalent if $P = Q\{\vec{y}/\vec{x}\}$ for
some $\vec{x} \in \boundnames{Q},\vec{y} \in \boundnames{P}$, where $Q\{\vec{y}/\vec{x}\}$
denotes the capture-avoiding substitution of $\vec{y}$ for $\vec{x}$ in $Q$.
\end{definition}

\begin{definition}
  The {\em structural congruence} \cite{SangiorgiWalker} , $\equiv$,
  between processes is the least congruence containing
  alpha-equivalence, satisfying the abelian monoid laws
  (associativity, commutativity and $\pzero$ as identity) for parallel
  composition $|$ and for summation $+$.
\end{definition}

\subsection{Name equivalence}

We take name equivalence, written $\nameeq$, to be the smallest
equivalence relation generated by the following rules.

\begin{mathpar}
\inferrule*[lab=Quote-drop]
{ }
{ \quotep{@{x}} \nameeq x }

\inferrule*[lab=Struct-equiv]
{ P \scong Q }
{ \quotep{P} \nameeq \quotep{Q} }
\end{mathpar}

The astute reader will have noticed that the mutual recursion of names
and processes imposes a mutual recursion on alpha-equivalence and
structural equivalence via name-equivalence. Fortunately, all of this
works out pleasantly and we may calculate in the natural way, free of
concern. The reader interested in the details is referred to the
appendix \ref{appendix:rho_details}.

\subsection{Substitution}

We use $\Proc$ for the set of processes, $\QProc$ for the set of
names, and $\id{\{}\vec{y} / \vec{x} \id{\}}$ to denote partial maps,
$s : \QProc \rightarrow \QProc$. A map, $s$ lifts, uniquely, to a map
on process terms, $\widehat{s} : \Proc \rightarrow \Proc$ by the
following equations.

\begin{mathpar}
  (0) \psubstp{Q}{P} := 0 \\
  (R \juxtap S) \psubstp{Q}{P}
  :=    
  (R)\psubstp{Q}{P} \juxtap (S) \psubstp{Q}{P} \\
  (x?(y).R) \psubstp{Q}{P}    
  :=    
  (x)\substp{Q}{P} (z)\concat( (R \psubstn{z}{y}) \psubstp{Q}{P} ) \\
  (\lift{x}{R}) \psubstp{Q}{P}  
  :=
  \lift{(x)\substp{Q}{P}}{ R \psubstp{Q}{P} } \\
%   (\dropn{x})  \psubstp{Q}{P}       
%   := 
%   \left\{ 
%     \begin{array}{ccc} 
%       \dropn{\quotep{Q}} & & x \nameeq \quotep{P} \\
%       \dropn{x} & & otherwise \\
%     \end{array}
%   \right. 
  (\dropn{x})  \psubstp{Q}{P}       
  := 
  \left\{ 
    \begin{array}{ccc} 
      Q & & x \nameeq \quotep{P} \\
      \dropn{x} & & otherwise \\
    \end{array}
  \right.
\end{mathpar}
 

where

\begin{eqnarray}
  (x)\id{\{} \lpquote Q \rpquote / \lpquote P \rpquote \id{\}}            = 
  \left\{ 
    \begin{array}{ccc}
      \lpquote Q \rpquote & & x \nameeq \lpquote P \rpquote \\
      x & & otherwise \\
    \end{array}
  \right. \nonumber
\end{eqnarray}

and $z$ is chosen distinct from $\quotep{P}$, $\quotep{Q}$, the free
names in $Q$, and all the names in $R$. Our $\alpha$-equivalence will
be built in the standard way from this substitution.

\begin{remark}\label{rem:no_self_referential_names}
  One consequence of these definitions is that $\forall P. \quotep{P}
  \not\in \freenames{P}$.
\end{remark}

\subsection{ Dynamic quote: an example }

Anticipating something of what's to come, consider applying the
substitution, $\widehat{\id{\{}u / z \id{\}}}$, to the following pair
of processes, $\lift{w}{y!(z)}$ and $w[ \lpquote y!(z) \rpquote ]$.

\begin{eqnarray}
	\lift{w}{y!(z)}\widehat{\id{\{}u / z \id{\}}}
		& = &
		\lift{w}{y!(u)} \nonumber\\
	w[ \lpquote y!(z) \rpquote ] \widehat{ \id{\{}u / z \id{\}} }
		& = &
		w[ \lpquote y!(z) \rpquote ] \nonumber
\end{eqnarray}

Because the body of the process between quotes is impervious to
substitution, we get radically different answers. In fact, by
examining the first process in an input context,
e.g. $x?(z).\lift{w}{y!(z)}$, we see that the process under the lift
operator may be shaped by prefixed inputs binding a name inside it. In
this sense, the lift operator will be seen as a way to dynamically
construct processes before reifying them as names.

Finally equipped with these standard features we can present the
dynamics of the calculus.

\subsubsection{Operational semantics} 

Finally, we introduce the computational dynamics. What marks these
algebras as distinct from other more traditionally studied algebraic
structures, e.g. vector spaces or polynomial rings, is the manner in
which dynamics is captured. In traditional structures, dynamics is typically
expressed through morphisms between such structures, as in linear maps
between vector spaces or morphisms between rings. In algebras
associated with the semantics of computation, the dynamics is
expressed as part of the algebraic structure itself, through a
reduction reduction relation typically denoted by $\red$. Below, we
give a recursive presentation of this relation for the calculus used
in the encoding.

$\red \subseteq \pi \times \pi$
$\red : \pi \to \mathcal{P}(\pi)$

\begin{mathpar}
  \inferrule* [lab=Comm] { \textsf{match}( x_{src}, x_{trgt} ) } { x_{trgt}?(y)P \; | \; x_{src}!\langle {Q} \rangle \red P\{\quotep{Q}/y}\} }
  \and \\
  \inferrule* [lab=Par] {{P} \red {P}'} {{{P} | {Q}} \red {{P}' | {Q}}}
  \and
  \inferrule* [lab=Equiv]{{{P} \scong {P}'} \andalso {{P}' \red {Q}'} \andalso {{Q}' \scong {Q}}}{{P} \red {Q}}
\end{mathpar}

\begin{eqnarray*}
  match_{\equiv} (\quotep{P},\quotep{Q}) & := & P \equiv Q \\
  match_{\dagger}(\quotep{P},\quotep{Q}) & := & \forall R. P|Q \red^{*} R => R \red^{*} 0 \\
  match_{K}(\quotep{P},\quotep{Q}) & := & K \mbox{ for some context } K
\end{eqnarray*}

$u?(x)P | u!\langle Q \rangle \red P\{\quotep{Q}/x\}$

%We write $\wred$ for $\red^*$, and $P\red$ if $\exists Q $ such that $ P \red Q$.
We write $P\red$ if $\exists Q $ such that $ P \red Q$ and $P\not\red$, otherwise.

\section{Replication}

As mentioned before, it is known that replication (and hence
recursion) can be implemented in a higher-order process algebra
\cite{SangiorgiWalker}. As our first example of calculation with the
machinery thus far presented we give the construction explicitly in
the {\rhoc}.

\begin{eqnarray}
	D_{x} & := & \prefix{x}{y}{(\binpar{\outputp{x}{y}}{@{y}})} \nonumber\\
	\bangp_{x}{P} & := & \binpar{{x}!\langle{\binpar{D_{x}}{P}}\rangle}{D_{x}} \nonumber
\end{eqnarray}

\begin{eqnarray}
	\bangp_{x}{P} & & \nonumber\\
	=
	& {x}!\langle{(\prefix{x}{y}{(\outputp{x}{y} | @{y})) | P}}\rangle 
	      | \prefix{x}{y}{(\outputp{x}{y} | @{y})} & \nonumber\\
	\red
	& (\outputp{x}{y} | @{y})\substn{\quotep{(\prefix{x}{y}{(@{y} | \outputp{x}{y})) | P}}}{y} & \nonumber\\
	=
	& \outputp{x}{\quotep{(\prefix{x}{y}{(\outputp{x}{y} | @{y})) | P}}}
	  | {(\prefix{x}{y}{(\outputp{x}{y} | @{y})) | P}} & \nonumber\\
	\red
	& \ldots & \nonumber\\
	\red^*
	& P | P | \ldots & \nonumber
\end{eqnarray}

Of course, this encoding, as an implementation, runs away, unfolding
$\bangp{P}$ eagerly. A lazier and more implementable replication
operator, restricted to input-guarded processes, may be obtained as follows.

\begin{eqnarray}
\bangp{\prefix{u}{v}{P}} 
	:= 
	\binpar{\lift{x}{\prefix{u}{v}{(\binpar{D(x)}{P})}}}{D(x)} \nonumber
\end{eqnarray}

\begin{remark}
  Note that the lazier definition still does not deal with summation
  or mixed summation (i.e. sums over input and output). The reader is
  invited to construct definitions of replication that deal with these
  features. 

  Further, the definitions are parameterized in a name, $x$. Can you,
  gentle reader, make a definition that eliminates this parameter and
  guarantees no accidental interaction between the replication
  machinery and the process being replicated -- i.e. no accidental
  sharing of names used by the process to get its work done and the
  name(s) used by the replication to effect copying. This latter
  revision of the definition of replication is crucial to obtaining
  the expected identity $!!P \sim !P$.
\end{remark}

\begin{remark}\label{rem:paradoxical_combinator}
  The reader familiar with the lambda calculus will have noticed the
  similarity between $D$ and the paradoxical combinator.

  [Ed. note: the existence of this seems to suggest we have to be more
  restrictive on the set of processes and names we admit if we are to
  support no-cloning.]
\end{remark}

\subsubsection{Bisimulation}

The computational dynamics gives rise to another kind of equivalence,
the equivalence of computational behavior. As previously mentioned
this is typically captured \emph{via} some form of bisimulation.

% The notion we use in this paper is weak barbed bisimulation
% \cite{milner91polyadicpi}.

The notion we use in this paper is derived from weak barbed
bisimulation \cite{milner91polyadicpi}. 

\begin{definition}
An \emph{observation relation}, $\downarrow_{\mathcal N}$, over a set
of names, $\mathcal N$, is the smallest relation satisfying the rules
below.

\infrule[Out-barb]{y \in {\mathcal N}, \; x \nameeq y}
		  {\outputp{x}{v} \downarrow_{\mathcal N} x}
\infrule[Par-barb]{\mbox{$P\downarrow_{\mathcal N} x$ or $Q\downarrow_{\mathcal N} x$}}
		  {\binpar{P}{Q} \downarrow_{\mathcal N} x}

We write $P \Downarrow_{\mathcal N} x$ if there is $Q$ such that 
$P \wred Q$ and $Q \downarrow_{\mathcal N} x$.
\end{definition}

\begin{definition}
%\label{def.bbisim}
An  ${\mathcal N}$-\emph{barbed bisimulation} over a set of names, ${\mathcal N}$, is a symmetric binary relation 
${\mathcal S}_{\mathcal N}$ between agents such that $P\rel{S}_{\mathcal N}Q$ implies:
\begin{enumerate}
\item If $P \red P'$ then $Q \wred Q'$ and $P'\rel{S}_{\mathcal N} Q'$.
\item If $P\downarrow_{\mathcal N} x$, then $Q\Downarrow_{\mathcal N} x$.
\end{enumerate}
$P$ is ${\mathcal N}$-barbed bisimilar to $Q$, written
$P \wbbisim_{\mathcal N} Q$, if $P \rel{S}_{\mathcal N} Q$ for some ${\mathcal N}$-barbed bisimulation ${\mathcal S}_{\mathcal N}$.
\end{definition}

$\mathcal{R} \subseteq \pi \times \pi$

$P \mathcal{R} Q => \forall P'. P \red P' \Rightarrow \exists Q'. Q \red Q', P' \mathcal{R} Q'$

$P \vdash x \Rightarrow Q \vdash x$

\begin{mathpar}
  \inferrule*[lab=Out-barb]{x \nameeq y}{{y}!\langle{Q}\rangle \vdash x}
  \and
  \inferrule*[lab=Par-barb]{\mbox{$P\vdash x$ or $Q\vdash x$}}{\binpar{P}{Q} \vdash x}
\end{mathpar}

\subsubsection{Contexts}

One of the principle advantages of computational calculi like the
$\pi$-calculus is a well-defined notion of context,
contextual-equivalence and a correlation between
contextual-equivalence and notions of bisimulation. The notion of
context allows the decomposition of a process into (sub-)process and
its syntactic environment, its context. Thus, a context may be
thought of as a process with a ``hole'' (written $\Box$) in it. The
application of a context $M$ to a process $P$, written $M[P]$, is
tantamount to filling the hole in $M$ with $P$. In this paper we do
not need the full weight of this theory, but do make use of the notion
of context in the proof the main theorem. 

\begin{mathpar}
  \inferrule* [lab=summation] {} {{M_{M},M_{N}} \bc \Box \;|\; x.M_{A} \;|\; M_{M}+M_{N}}
  \and
  \inferrule* [lab=agent] {} {{M_{A}} \bc (\vec{x})M_{P} \;| \; \clift{P_0,\ldots,M_{P},\ldots,P_N}}
  \and \\
  \inferrule* [lab=process] {} {{M_{P}} \bc M_{N} \;| \;P|M_{P} }
\end{mathpar} 

\begin{mathpar}
  \inferrule* [lab=sychronization] {} {M_{N} \bc \Box \;|\; x?M_{F} \;|\; x!M_{C}}
  \and
  \inferrule* [lab=abstraction] {} {{M_{F}} \bc (x)M_{P} }
  \and
  \inferrule* [lab=concretion] {} {{M_{C}} \bc \langle M_{P} \rangle }
  \and \\
  \inferrule* [lab=process] {} {{M_{P}} \bc M_{N} \;| \;P|M_{P} }
\end{mathpar}

\begin{definition}[contextual application] Given a context $M$, and
  process $P$, we define the \emph{contextual application}, $M[P] :=
  M\{P/\Box\}$. That is, the contextual application of M to P is the
  substitution of $P$ for $\Box$ in $M$.
\end{definition}

$\meaningof{-} : L \to \mathcal{P}(\pi)$

\begin{mathpar}
  \inferrule* [lab=collection] {} {\meaningof{true} = \pi, \and \meaningof{~E} = \pi \setminus \meaningof{E}, \and \meaningof{E_{1} \& E_{2}} = \meaningof{E_{1}} \cap \meaningof{E_{2}}}
\end{mathpar}

\begin{mathpar}
  \inferrule* [lab=structure] {} {\meaningof{0} = \{ P \in \pi | P \equiv 0 \}, \and \\ \meaningof{E_1 | E_2} = \{ P \in \pi | P \equiv P_{1} | P_{2}, P_{1} \in \meaningof{E_{1}}, P_{2} \in \meaningof{E_2}\} }
\end{mathpar}

\begin{mathpar}
 \inferrule* [lab=behavior] {} {\meaningof{\langle a?b \rangle E} = \{ P \in \pi | P \equiv Q | u?(y)P', \\ \and \\\\ \and \\ \;\;\; u \in \meaningof{a}, \forall z.P'\{z/y\} \in \meaningof{E\{z/b\}}\}, \and \\ \meaningof{a!E} = \{ P \in \pi | P \equiv Q | x!\langle P' \rangle, x \in \meaningof{a} P' \in \meaningof{E}\} }
\end{mathpar}

\begin{mathpar}
 \inferrule* [lab=nominal] {} {\meaningof{\quotep{E}} = \{ \quotep{P} \in \quotep{\pi} | P \in \meaningof{E} \}, \and \meaningof{\quotep{P}} = \{ \quotep{Q} \in \quotep{\pi} | P \equiv Q \} \and \\ \meaningof{@\quotep{E}} = \{ P \in \pi | P \equiv @x, x \in \meaningof{E} \}}
\end{mathpar}

\begin{eqnarray*}
  \\
  \meaningof{-} : TS \to ST
\end{eqnarray*}

\begin{eqnarray*}
  \\
  L : TS \to ST
\end{eqnarray*}

\begin{eqnarray*}
  \\
  P \models E \iff P \in \meaningof{E}
\end{eqnarray*}

\begin{eqnarray*}
  P \approx_{L} Q \iff \forall E \in L. P \models E \iff Q \models E
\end{eqnarray*}

\begin{eqnarray*}
  P \approx_{K} Q
\end{eqnarray*}

\begin{eqnarray*}
  P \approx Q
\end{eqnarray*}

$\approx_{K} = \approx = \approx_{L}$

\subsubsection{Contextual duality}

Note that contexts extend the quotation operation to a family of
operations from processes to names. Given a context, $M$, we can
define a \emph{nominal context}, $\quotep{M}$ by $\quotep{M}[P] :=
\quotep{M[P]}$. To foreshadow what is to come we observe that these
operations enjoy a duality with processes very much like the duality
between vectors and maps from vectors to scalars.

Further, because the calculus is essentially higher-order, we have a
correspondence between contexts and processes. More specifically,
given a name $x$ and a context $M$ we can construct $M^{*}_{x}$ such
that 

\begin{mathpar}
  M^{*}_{x} | \lift{x}{P} \red M[P]
\end{mathpar}

namely,

\begin{mathpar}
  M^{*}_{x} := x?(u).M[\dropn{u}]
\end{mathpar}

The dependence of $M^{*}_{x}$ on a name makes it an abstraction, 

\begin{mathpar}
  M^{*} := (x)x?(u).M[\dropn{u}]
\end{mathpar}

\subsection{Additional notation}

It will sometimes be convenient to denote the process a name
quotes. We already have the notation $x = \quotep{P}$, but it will be
convenient to introduce an alternate notation, $\procn{x}$, when we
want to emphasize the connection to the use of the name. Note that, by
virtue of name equivalence, $\quotep{\procn{x}} \nameeq x$; so, the
notation is consistent with previous definitions.

Further, because names have structure it is possible to effect
substitutions on the basis of that structure. This means we need to
upgrade our notation for substitutions, which we accomplish by
adapting comprehension notation. Thus,

\begin{mathpar}
  P\{ y / x : x \in S \}
\end{mathpar}

is interpreted to mean the process derived from P by replacing (in a
capture-avoiding manner) each occurrence of $x$ in $S$ by $y$. For example,

\begin{mathpar}
  P\{ \quotep{\procn{x}|\procn{x}} / x : x \in \freenames{P} \}
\end{mathpar}

will replace each (occurrence) of a free name $x$ in $P$ by
$\quotep{\procn{x}|\procn{x}}$.

Also, we will avail ourselves of the notation $x^{L}$ and $x^{R}$ to
denote injections of a name into disjoint copies of the name
space. There are numerous ways to accomplish this. One example can be
found in \cite{MeredithR05}. This notation overloads to vectors of
names: $\vec{x}^{\pi} := (x_{i}^{\pi} \; : \; 0 \leq i < |\vec{x}| )$ where $\pi \in \{L,R\}$.

We also use $P^{\Box} := P|\Box$.

In \cite{MeredithR05} an interpretation of the new operator is
given. It turns out that there are several possible interpretations
all enjoying the requisite algebraic properties of the operator (see
\cite{milner91polyadicpi}). We will therefore make liberal use of
$(\nu\; \vec{x})P$.

% subsection the_syntax_and_semantics_of_the_notation_system (end)   

\input{qm2pi.qmops} 

\input{qm2pi.sterngerlach} 

\input{qm2pi.metric} 

% section concurrent_process_calculi (end)

%\input{qm2pi.proofsketch}

% section proof sketch (end)

%\input{qm2pi.slviaknots} 

% section spatial logic via knots (end)

\input{qm2pi.conclusion}

% section conclusion (end)

%\input{qm2pi.dtcodes} 

% section wiring algorithm (end)

\input{qm2pi.ack} 

% section acknowledgments (end)

\newpage


\bibliographystyle{plain}   
\bibliography{../../biblios/main.bib}

\input{qm2pi.rhodetails}

\end{document}

 

%\documentclass[12pt]{llncs}
%\documentclass{jktr}

\usepackage[pdftex]{hyperref}                   
\usepackage {listings}
\usepackage {mathpartir}
\usepackage{bcprules}
%\usepackage{listings}
                       
\usepackage{graphicx} 
%\usepackage[margins=2.5cm,nohead,nofoot]{geometry}
%\usepackage{geometry}
\usepackage{amsfonts}
\usepackage{amstext}
\usepackage{latexsym}
\usepackage{amssymb}
\usepackage{color}


%\include{myPreamble}
\include{qm2pi.local} 

%\ifpdf
%\usepackage[pdftex]{graphicx}
%\else
%\usepackage{graphicx}
%\fi

 % \ifpdf
%  \usepackage{pdfsync}
%  \if


%\title{Brief Article}
%\author{David F. Snyder}
%\author{L.G. Meredith}

%\address{Dept. of Math., Texas State University--San Marcos, San Marcos, TX 78666}
       
\pagestyle{empty}


\begin{document}

\lstset{language=[Objective]Caml,frame=shadowbox}

\input{qm2pi.front}

% section front matter (end)

\input{qm2pi.intro} 
 
% section introduction (end)

% \input{qm2pi.knotations} 

% section notation (end)

\input{qm2pi.process.calculi} 

% section concurrent_process_calculi_and_spatial_logics_ (end)
    
%\input{qm2pi.knots2pi} 

%\input{qm2pi.trefoil} 

%\input{qm2pi.mainthm} 

% subsection basic_interpretation (end)

%\input{qm2pi.rho.presentation} 
\subsection{The syntax and semantics of the notation system}\label{sub:the_syntax_and_semantics_of_the_notation_system} % (fold)

We now summarize a technical presentation of the calculus that
embodies our theory of dynamics. The typical presentation of such a
calculus follows the style of giving generators and relations on
them. The grammar, below, describing term constructors, freely
generates the set of processes, $\Proc$. This set is then quotiented
by a relation known as structural congruence and it is over this set
that the notion of dynamics is expressed. This presentation is
essentially that of \cite{MeredithR05} with the addition of
polyadicity and summation. For readability we have relegated some of
the technical subtleties to an appendix.

\subsubsection{Process grammar}\label{subsub:process_grammar}

\begin{mathpar}
  \inferrule* [lab=synchronization] {} {{M} \bc \pzero \;|\; x?F \;|\; x!C }
  \and
  \inferrule* [lab=abstraction] {} {{F} \bc (x)P}
  \and
  \inferrule* [lab=concretion] {} {{C} \bc \langle Q \rangle}
  \and
  \inferrule* [lab=process] {} {{P,Q} \bc M \;| \;P|Q \;|\; @{x}}
  \and
  \inferrule* [lab=name] {} {{x} \bc \quotep{P}}
\end{mathpar} 

Note that $\vec{x}$ (resp. $\vec{P}$) denotes a vector of names
(resp. processes) of length $|\vec{x}|$ (resp. $|\vec{P}|$). We adopt
the following useful abbreviations.

\begin{mathpar}
   x?(\vec{y}).P := x.(\vec{y})P \and  x\clift{\vec{P}} := x.\clift{\vec{P}}
   \and x!(y) := \lift{x}{\dropn{y}}
   \and \Pi_{i=0}^{n-1}P_i := P_0 | \ldots | P_{n-1}
\end{mathpar}

\subsubsection{Structural congruence}

\paragraph{Free and bound names and alpha-equivalence.} At the
core of structural equivalence is alpha-equivalence which identifies
process that are the same up to a change of variable. Formally, we
recognize the distinction between free and bound names. The free names
of a process, $\freenames{P}$, may be calculated recursively as
follows:

\begin{mathpar}
\freenames{\pzero} := \emptyset
  \and \\
  \freenames{x?(y).P} := \{ x \} \cup (\freenames{P} \setminus \{ y \})
  \and 
  \freenames{x!\langle P \rangle} := \{ x \} \cup \{ P \} 
  \and \\
  \freenames{P|Q} := \freenames{P} \cup \freenames{Q}
  \and \\
  \freenames{@{x}} := \{ x \}
\end{mathpar}

$\pi$
$\quotep{\pi}$

$\freenames{-} : \pi \to \mathcal{P}(\quotep{\pi})$

\begin{eqnarray*}
  \freenames{\pzero} & := & \emptyset \\
  \freenames{x?(y).P} & := & \{ x \} \cup (\freenames{P} \setminus \{ y \}) \\
  \freenames{x!\langle P \rangle} & := & \{ x \} \cup \{ P \} \\
  \freenames{P|Q} & := & \freenames{P} \cup \freenames{Q} \\
  \freenames{\dropn{x}} & := & \{ x \}
\end{eqnarray*}

The bound names of a process, $\boundnames{P}$, are those names occurring in $P$
that are not free. For example, in $x?(y).0$, the name $x$ is free, while $y$ is bound.

\begin{mathpar}
  \inferrule* [lab=monoidal-laws] {} { P|Q \equiv Q|P \and P|0 \equiv P \and P|(Q|R) \equiv (P|Q)|R }
\end{mathpar}

\begin{mathpar}
  \inferrule* [lab=alpha-equivalence] {} { (x)P \equiv (y)P\{y/x\} \and y \not\in \freenames{P} }
\end{mathpar}

\begin{definition}
Then two processes, $P,Q$, are alpha-equivalent if $P = Q\{\vec{y}/\vec{x}\}$ for
some $\vec{x} \in \boundnames{Q},\vec{y} \in \boundnames{P}$, where $Q\{\vec{y}/\vec{x}\}$
denotes the capture-avoiding substitution of $\vec{y}$ for $\vec{x}$ in $Q$.
\end{definition}

\begin{definition}
  The {\em structural congruence} \cite{SangiorgiWalker} , $\equiv$,
  between processes is the least congruence containing
  alpha-equivalence, satisfying the abelian monoid laws
  (associativity, commutativity and $\pzero$ as identity) for parallel
  composition $|$ and for summation $+$.
\end{definition}

\subsection{Name equivalence}

We take name equivalence, written $\nameeq$, to be the smallest
equivalence relation generated by the following rules.

\begin{mathpar}
\inferrule*[lab=Quote-drop]
{ }
{ \quotep{@{x}} \nameeq x }

\inferrule*[lab=Struct-equiv]
{ P \scong Q }
{ \quotep{P} \nameeq \quotep{Q} }
\end{mathpar}

The astute reader will have noticed that the mutual recursion of names
and processes imposes a mutual recursion on alpha-equivalence and
structural equivalence via name-equivalence. Fortunately, all of this
works out pleasantly and we may calculate in the natural way, free of
concern. The reader interested in the details is referred to the
appendix \ref{appendix:rho_details}.

\subsection{Substitution}

We use $\Proc$ for the set of processes, $\QProc$ for the set of
names, and $\id{\{}\vec{y} / \vec{x} \id{\}}$ to denote partial maps,
$s : \QProc \rightarrow \QProc$. A map, $s$ lifts, uniquely, to a map
on process terms, $\widehat{s} : \Proc \rightarrow \Proc$ by the
following equations.

\begin{mathpar}
  (0) \psubstp{Q}{P} := 0 \\
  (R \juxtap S) \psubstp{Q}{P}
  :=    
  (R)\psubstp{Q}{P} \juxtap (S) \psubstp{Q}{P} \\
  (x?(y).R) \psubstp{Q}{P}    
  :=    
  (x)\substp{Q}{P} (z)\concat( (R \psubstn{z}{y}) \psubstp{Q}{P} ) \\
  (\lift{x}{R}) \psubstp{Q}{P}  
  :=
  \lift{(x)\substp{Q}{P}}{ R \psubstp{Q}{P} } \\
%   (\dropn{x})  \psubstp{Q}{P}       
%   := 
%   \left\{ 
%     \begin{array}{ccc} 
%       \dropn{\quotep{Q}} & & x \nameeq \quotep{P} \\
%       \dropn{x} & & otherwise \\
%     \end{array}
%   \right. 
  (\dropn{x})  \psubstp{Q}{P}       
  := 
  \left\{ 
    \begin{array}{ccc} 
      Q & & x \nameeq \quotep{P} \\
      \dropn{x} & & otherwise \\
    \end{array}
  \right.
\end{mathpar}
 

where

\begin{eqnarray}
  (x)\id{\{} \lpquote Q \rpquote / \lpquote P \rpquote \id{\}}            = 
  \left\{ 
    \begin{array}{ccc}
      \lpquote Q \rpquote & & x \nameeq \lpquote P \rpquote \\
      x & & otherwise \\
    \end{array}
  \right. \nonumber
\end{eqnarray}

and $z$ is chosen distinct from $\quotep{P}$, $\quotep{Q}$, the free
names in $Q$, and all the names in $R$. Our $\alpha$-equivalence will
be built in the standard way from this substitution.

\begin{remark}\label{rem:no_self_referential_names}
  One consequence of these definitions is that $\forall P. \quotep{P}
  \not\in \freenames{P}$.
\end{remark}

\subsection{ Dynamic quote: an example }

Anticipating something of what's to come, consider applying the
substitution, $\widehat{\id{\{}u / z \id{\}}}$, to the following pair
of processes, $\lift{w}{y!(z)}$ and $w[ \lpquote y!(z) \rpquote ]$.

\begin{eqnarray}
	\lift{w}{y!(z)}\widehat{\id{\{}u / z \id{\}}}
		& = &
		\lift{w}{y!(u)} \nonumber\\
	w[ \lpquote y!(z) \rpquote ] \widehat{ \id{\{}u / z \id{\}} }
		& = &
		w[ \lpquote y!(z) \rpquote ] \nonumber
\end{eqnarray}

Because the body of the process between quotes is impervious to
substitution, we get radically different answers. In fact, by
examining the first process in an input context,
e.g. $x?(z).\lift{w}{y!(z)}$, we see that the process under the lift
operator may be shaped by prefixed inputs binding a name inside it. In
this sense, the lift operator will be seen as a way to dynamically
construct processes before reifying them as names.

Finally equipped with these standard features we can present the
dynamics of the calculus.

\subsubsection{Operational semantics} 

Finally, we introduce the computational dynamics. What marks these
algebras as distinct from other more traditionally studied algebraic
structures, e.g. vector spaces or polynomial rings, is the manner in
which dynamics is captured. In traditional structures, dynamics is typically
expressed through morphisms between such structures, as in linear maps
between vector spaces or morphisms between rings. In algebras
associated with the semantics of computation, the dynamics is
expressed as part of the algebraic structure itself, through a
reduction reduction relation typically denoted by $\red$. Below, we
give a recursive presentation of this relation for the calculus used
in the encoding.

$\red \subseteq \pi \times \pi$
$\red : \pi \to \mathcal{P}(\pi)$

\begin{mathpar}
  \inferrule* [lab=Comm] { \textsf{match}( x_{src}, x_{trgt} ) } { x_{trgt}?(y)P \; | \; x_{src}!\langle {Q} \rangle \red P\{\quotep{Q}/y}\} }
  \and \\
  \inferrule* [lab=Par] {{P} \red {P}'} {{{P} | {Q}} \red {{P}' | {Q}}}
  \and
  \inferrule* [lab=Equiv]{{{P} \scong {P}'} \andalso {{P}' \red {Q}'} \andalso {{Q}' \scong {Q}}}{{P} \red {Q}}
\end{mathpar}

\begin{eqnarray*}
  match_{\equiv} (\quotep{P},\quotep{Q}) & := & P \equiv Q \\
  match_{\dagger}(\quotep{P},\quotep{Q}) & := & \forall R. P|Q \red^{*} R => R \red^{*} 0 \\
  match_{K}(\quotep{P},\quotep{Q}) & := & K \mbox{ for some context } K
\end{eqnarray*}

$u?(x)P | u!\langle Q \rangle \red P\{\quotep{Q}/x\}$

%We write $\wred$ for $\red^*$, and $P\red$ if $\exists Q $ such that $ P \red Q$.
We write $P\red$ if $\exists Q $ such that $ P \red Q$ and $P\not\red$, otherwise.

\section{Replication}

As mentioned before, it is known that replication (and hence
recursion) can be implemented in a higher-order process algebra
\cite{SangiorgiWalker}. As our first example of calculation with the
machinery thus far presented we give the construction explicitly in
the {\rhoc}.

\begin{eqnarray}
	D_{x} & := & \prefix{x}{y}{(\binpar{\outputp{x}{y}}{@{y}})} \nonumber\\
	\bangp_{x}{P} & := & \binpar{{x}!\langle{\binpar{D_{x}}{P}}\rangle}{D_{x}} \nonumber
\end{eqnarray}

\begin{eqnarray}
	\bangp_{x}{P} & & \nonumber\\
	=
	& {x}!\langle{(\prefix{x}{y}{(\outputp{x}{y} | @{y})) | P}}\rangle 
	      | \prefix{x}{y}{(\outputp{x}{y} | @{y})} & \nonumber\\
	\red
	& (\outputp{x}{y} | @{y})\substn{\quotep{(\prefix{x}{y}{(@{y} | \outputp{x}{y})) | P}}}{y} & \nonumber\\
	=
	& \outputp{x}{\quotep{(\prefix{x}{y}{(\outputp{x}{y} | @{y})) | P}}}
	  | {(\prefix{x}{y}{(\outputp{x}{y} | @{y})) | P}} & \nonumber\\
	\red
	& \ldots & \nonumber\\
	\red^*
	& P | P | \ldots & \nonumber
\end{eqnarray}

Of course, this encoding, as an implementation, runs away, unfolding
$\bangp{P}$ eagerly. A lazier and more implementable replication
operator, restricted to input-guarded processes, may be obtained as follows.

\begin{eqnarray}
\bangp{\prefix{u}{v}{P}} 
	:= 
	\binpar{\lift{x}{\prefix{u}{v}{(\binpar{D(x)}{P})}}}{D(x)} \nonumber
\end{eqnarray}

\begin{remark}
  Note that the lazier definition still does not deal with summation
  or mixed summation (i.e. sums over input and output). The reader is
  invited to construct definitions of replication that deal with these
  features. 

  Further, the definitions are parameterized in a name, $x$. Can you,
  gentle reader, make a definition that eliminates this parameter and
  guarantees no accidental interaction between the replication
  machinery and the process being replicated -- i.e. no accidental
  sharing of names used by the process to get its work done and the
  name(s) used by the replication to effect copying. This latter
  revision of the definition of replication is crucial to obtaining
  the expected identity $!!P \sim !P$.
\end{remark}

\begin{remark}\label{rem:paradoxical_combinator}
  The reader familiar with the lambda calculus will have noticed the
  similarity between $D$ and the paradoxical combinator.

  [Ed. note: the existence of this seems to suggest we have to be more
  restrictive on the set of processes and names we admit if we are to
  support no-cloning.]
\end{remark}

\subsubsection{Bisimulation}

The computational dynamics gives rise to another kind of equivalence,
the equivalence of computational behavior. As previously mentioned
this is typically captured \emph{via} some form of bisimulation.

% The notion we use in this paper is weak barbed bisimulation
% \cite{milner91polyadicpi}.

The notion we use in this paper is derived from weak barbed
bisimulation \cite{milner91polyadicpi}. 

\begin{definition}
An \emph{observation relation}, $\downarrow_{\mathcal N}$, over a set
of names, $\mathcal N$, is the smallest relation satisfying the rules
below.

\infrule[Out-barb]{y \in {\mathcal N}, \; x \nameeq y}
		  {\outputp{x}{v} \downarrow_{\mathcal N} x}
\infrule[Par-barb]{\mbox{$P\downarrow_{\mathcal N} x$ or $Q\downarrow_{\mathcal N} x$}}
		  {\binpar{P}{Q} \downarrow_{\mathcal N} x}

We write $P \Downarrow_{\mathcal N} x$ if there is $Q$ such that 
$P \wred Q$ and $Q \downarrow_{\mathcal N} x$.
\end{definition}

\begin{definition}
%\label{def.bbisim}
An  ${\mathcal N}$-\emph{barbed bisimulation} over a set of names, ${\mathcal N}$, is a symmetric binary relation 
${\mathcal S}_{\mathcal N}$ between agents such that $P\rel{S}_{\mathcal N}Q$ implies:
\begin{enumerate}
\item If $P \red P'$ then $Q \wred Q'$ and $P'\rel{S}_{\mathcal N} Q'$.
\item If $P\downarrow_{\mathcal N} x$, then $Q\Downarrow_{\mathcal N} x$.
\end{enumerate}
$P$ is ${\mathcal N}$-barbed bisimilar to $Q$, written
$P \wbbisim_{\mathcal N} Q$, if $P \rel{S}_{\mathcal N} Q$ for some ${\mathcal N}$-barbed bisimulation ${\mathcal S}_{\mathcal N}$.
\end{definition}

$\mathcal{R} \subseteq \pi \times \pi$

$P \mathcal{R} Q => \forall P'. P \red P' \Rightarrow \exists Q'. Q \red Q', P' \mathcal{R} Q'$

$P \vdash x \Rightarrow Q \vdash x$

\begin{mathpar}
  \inferrule*[lab=Out-barb]{x \nameeq y}{{y}!\langle{Q}\rangle \vdash x}
  \and
  \inferrule*[lab=Par-barb]{\mbox{$P\vdash x$ or $Q\vdash x$}}{\binpar{P}{Q} \vdash x}
\end{mathpar}

\subsubsection{Contexts}

One of the principle advantages of computational calculi like the
$\pi$-calculus is a well-defined notion of context,
contextual-equivalence and a correlation between
contextual-equivalence and notions of bisimulation. The notion of
context allows the decomposition of a process into (sub-)process and
its syntactic environment, its context. Thus, a context may be
thought of as a process with a ``hole'' (written $\Box$) in it. The
application of a context $M$ to a process $P$, written $M[P]$, is
tantamount to filling the hole in $M$ with $P$. In this paper we do
not need the full weight of this theory, but do make use of the notion
of context in the proof the main theorem. 

\begin{mathpar}
  \inferrule* [lab=summation] {} {{M_{M},M_{N}} \bc \Box \;|\; x.M_{A} \;|\; M_{M}+M_{N}}
  \and
  \inferrule* [lab=agent] {} {{M_{A}} \bc (\vec{x})M_{P} \;| \; \clift{P_0,\ldots,M_{P},\ldots,P_N}}
  \and \\
  \inferrule* [lab=process] {} {{M_{P}} \bc M_{N} \;| \;P|M_{P} }
\end{mathpar} 

\begin{mathpar}
  \inferrule* [lab=sychronization] {} {M_{N} \bc \Box \;|\; x?M_{F} \;|\; x!M_{C}}
  \and
  \inferrule* [lab=abstraction] {} {{M_{F}} \bc (x)M_{P} }
  \and
  \inferrule* [lab=concretion] {} {{M_{C}} \bc \langle M_{P} \rangle }
  \and \\
  \inferrule* [lab=process] {} {{M_{P}} \bc M_{N} \;| \;P|M_{P} }
\end{mathpar}

\begin{definition}[contextual application] Given a context $M$, and
  process $P$, we define the \emph{contextual application}, $M[P] :=
  M\{P/\Box\}$. That is, the contextual application of M to P is the
  substitution of $P$ for $\Box$ in $M$.
\end{definition}

$\meaningof{-} : L \to \mathcal{P}(\pi)$

\begin{mathpar}
  \inferrule* [lab=collection] {} {\meaningof{true} = \pi, \and \meaningof{~E} = \pi \setminus \meaningof{E}, \and \meaningof{E_{1} \& E_{2}} = \meaningof{E_{1}} \cap \meaningof{E_{2}}}
\end{mathpar}

\begin{mathpar}
  \inferrule* [lab=structure] {} {\meaningof{0} = \{ P \in \pi | P \equiv 0 \}, \and \\ \meaningof{E_1 | E_2} = \{ P \in \pi | P \equiv P_{1} | P_{2}, P_{1} \in \meaningof{E_{1}}, P_{2} \in \meaningof{E_2}\} }
\end{mathpar}

\begin{mathpar}
 \inferrule* [lab=behavior] {} {\meaningof{\langle a?b \rangle E} = \{ P \in \pi | P \equiv Q | u?(y)P', \\ \and \\\\ \and \\ \;\;\; u \in \meaningof{a}, \forall z.P'\{z/y\} \in \meaningof{E\{z/b\}}\}, \and \\ \meaningof{a!E} = \{ P \in \pi | P \equiv Q | x!\langle P' \rangle, x \in \meaningof{a} P' \in \meaningof{E}\} }
\end{mathpar}

\begin{mathpar}
 \inferrule* [lab=nominal] {} {\meaningof{\quotep{E}} = \{ \quotep{P} \in \quotep{\pi} | P \in \meaningof{E} \}, \and \meaningof{\quotep{P}} = \{ \quotep{Q} \in \quotep{\pi} | P \equiv Q \} \and \\ \meaningof{@\quotep{E}} = \{ P \in \pi | P \equiv @x, x \in \meaningof{E} \}}
\end{mathpar}

\begin{eqnarray*}
  \\
  \meaningof{-} : TS \to ST
\end{eqnarray*}

\begin{eqnarray*}
  \\
  L : TS \to ST
\end{eqnarray*}

\begin{eqnarray*}
  \\
  P \models E \iff P \in \meaningof{E}
\end{eqnarray*}

\begin{eqnarray*}
  P \approx_{L} Q \iff \forall E \in L. P \models E \iff Q \models E
\end{eqnarray*}

\begin{eqnarray*}
  P \approx_{K} Q
\end{eqnarray*}

\begin{eqnarray*}
  P \approx Q
\end{eqnarray*}

$\approx_{K} = \approx = \approx_{L}$

\subsubsection{Contextual duality}

Note that contexts extend the quotation operation to a family of
operations from processes to names. Given a context, $M$, we can
define a \emph{nominal context}, $\quotep{M}$ by $\quotep{M}[P] :=
\quotep{M[P]}$. To foreshadow what is to come we observe that these
operations enjoy a duality with processes very much like the duality
between vectors and maps from vectors to scalars.

Further, because the calculus is essentially higher-order, we have a
correspondence between contexts and processes. More specifically,
given a name $x$ and a context $M$ we can construct $M^{*}_{x}$ such
that 

\begin{mathpar}
  M^{*}_{x} | \lift{x}{P} \red M[P]
\end{mathpar}

namely,

\begin{mathpar}
  M^{*}_{x} := x?(u).M[\dropn{u}]
\end{mathpar}

The dependence of $M^{*}_{x}$ on a name makes it an abstraction, 

\begin{mathpar}
  M^{*} := (x)x?(u).M[\dropn{u}]
\end{mathpar}

\subsection{Additional notation}

It will sometimes be convenient to denote the process a name
quotes. We already have the notation $x = \quotep{P}$, but it will be
convenient to introduce an alternate notation, $\procn{x}$, when we
want to emphasize the connection to the use of the name. Note that, by
virtue of name equivalence, $\quotep{\procn{x}} \nameeq x$; so, the
notation is consistent with previous definitions.

Further, because names have structure it is possible to effect
substitutions on the basis of that structure. This means we need to
upgrade our notation for substitutions, which we accomplish by
adapting comprehension notation. Thus,

\begin{mathpar}
  P\{ y / x : x \in S \}
\end{mathpar}

is interpreted to mean the process derived from P by replacing (in a
capture-avoiding manner) each occurrence of $x$ in $S$ by $y$. For example,

\begin{mathpar}
  P\{ \quotep{\procn{x}|\procn{x}} / x : x \in \freenames{P} \}
\end{mathpar}

will replace each (occurrence) of a free name $x$ in $P$ by
$\quotep{\procn{x}|\procn{x}}$.

Also, we will avail ourselves of the notation $x^{L}$ and $x^{R}$ to
denote injections of a name into disjoint copies of the name
space. There are numerous ways to accomplish this. One example can be
found in \cite{MeredithR05}. This notation overloads to vectors of
names: $\vec{x}^{\pi} := (x_{i}^{\pi} \; : \; 0 \leq i < |\vec{x}| )$ where $\pi \in \{L,R\}$.

We also use $P^{\Box} := P|\Box$.

In \cite{MeredithR05} an interpretation of the new operator is
given. It turns out that there are several possible interpretations
all enjoying the requisite algebraic properties of the operator (see
\cite{milner91polyadicpi}). We will therefore make liberal use of
$(\nu\; \vec{x})P$.

% subsection the_syntax_and_semantics_of_the_notation_system (end)   

\input{qm2pi.qmops} 

\input{qm2pi.sterngerlach} 

\input{qm2pi.metric} 

% section concurrent_process_calculi (end)

%\input{qm2pi.proofsketch}

% section proof sketch (end)

%\input{qm2pi.slviaknots} 

% section spatial logic via knots (end)

\input{qm2pi.conclusion}

% section conclusion (end)

%\input{qm2pi.dtcodes} 

% section wiring algorithm (end)

\input{qm2pi.ack} 

% section acknowledgments (end)

\newpage


\bibliographystyle{plain}   
\bibliography{../../biblios/main.bib}

\input{qm2pi.rhodetails}

\end{document}

 

%\documentclass[12pt]{llncs}
%\documentclass{jktr}

\usepackage[pdftex]{hyperref}                   
\usepackage {listings}
\usepackage {mathpartir}
\usepackage{bcprules}
%\usepackage{listings}
                       
\usepackage{graphicx} 
%\usepackage[margins=2.5cm,nohead,nofoot]{geometry}
%\usepackage{geometry}
\usepackage{amsfonts}
\usepackage{amstext}
\usepackage{latexsym}
\usepackage{amssymb}
\usepackage{color}


%\include{myPreamble}
\include{qm2pi.local} 

%\ifpdf
%\usepackage[pdftex]{graphicx}
%\else
%\usepackage{graphicx}
%\fi

 % \ifpdf
%  \usepackage{pdfsync}
%  \if


%\title{Brief Article}
%\author{David F. Snyder}
%\author{L.G. Meredith}

%\address{Dept. of Math., Texas State University--San Marcos, San Marcos, TX 78666}
       
\pagestyle{empty}


\begin{document}

\lstset{language=[Objective]Caml,frame=shadowbox}

\input{qm2pi.front}

% section front matter (end)

\input{qm2pi.intro} 
 
% section introduction (end)

% \input{qm2pi.knotations} 

% section notation (end)

\input{qm2pi.process.calculi} 

% section concurrent_process_calculi_and_spatial_logics_ (end)
    
%\input{qm2pi.knots2pi} 

%\input{qm2pi.trefoil} 

%\input{qm2pi.mainthm} 

% subsection basic_interpretation (end)

%\input{qm2pi.rho.presentation} 
\subsection{The syntax and semantics of the notation system}\label{sub:the_syntax_and_semantics_of_the_notation_system} % (fold)

We now summarize a technical presentation of the calculus that
embodies our theory of dynamics. The typical presentation of such a
calculus follows the style of giving generators and relations on
them. The grammar, below, describing term constructors, freely
generates the set of processes, $\Proc$. This set is then quotiented
by a relation known as structural congruence and it is over this set
that the notion of dynamics is expressed. This presentation is
essentially that of \cite{MeredithR05} with the addition of
polyadicity and summation. For readability we have relegated some of
the technical subtleties to an appendix.

\subsubsection{Process grammar}\label{subsub:process_grammar}

\begin{mathpar}
  \inferrule* [lab=synchronization] {} {{M} \bc \pzero \;|\; x?F \;|\; x!C }
  \and
  \inferrule* [lab=abstraction] {} {{F} \bc (x)P}
  \and
  \inferrule* [lab=concretion] {} {{C} \bc \langle Q \rangle}
  \and
  \inferrule* [lab=process] {} {{P,Q} \bc M \;| \;P|Q \;|\; @{x}}
  \and
  \inferrule* [lab=name] {} {{x} \bc \quotep{P}}
\end{mathpar} 

Note that $\vec{x}$ (resp. $\vec{P}$) denotes a vector of names
(resp. processes) of length $|\vec{x}|$ (resp. $|\vec{P}|$). We adopt
the following useful abbreviations.

\begin{mathpar}
   x?(\vec{y}).P := x.(\vec{y})P \and  x\clift{\vec{P}} := x.\clift{\vec{P}}
   \and x!(y) := \lift{x}{\dropn{y}}
   \and \Pi_{i=0}^{n-1}P_i := P_0 | \ldots | P_{n-1}
\end{mathpar}

\subsubsection{Structural congruence}

\paragraph{Free and bound names and alpha-equivalence.} At the
core of structural equivalence is alpha-equivalence which identifies
process that are the same up to a change of variable. Formally, we
recognize the distinction between free and bound names. The free names
of a process, $\freenames{P}$, may be calculated recursively as
follows:

\begin{mathpar}
\freenames{\pzero} := \emptyset
  \and \\
  \freenames{x?(y).P} := \{ x \} \cup (\freenames{P} \setminus \{ y \})
  \and 
  \freenames{x!\langle P \rangle} := \{ x \} \cup \{ P \} 
  \and \\
  \freenames{P|Q} := \freenames{P} \cup \freenames{Q}
  \and \\
  \freenames{@{x}} := \{ x \}
\end{mathpar}

$\pi$
$\quotep{\pi}$

$\freenames{-} : \pi \to \mathcal{P}(\quotep{\pi})$

\begin{eqnarray*}
  \freenames{\pzero} & := & \emptyset \\
  \freenames{x?(y).P} & := & \{ x \} \cup (\freenames{P} \setminus \{ y \}) \\
  \freenames{x!\langle P \rangle} & := & \{ x \} \cup \{ P \} \\
  \freenames{P|Q} & := & \freenames{P} \cup \freenames{Q} \\
  \freenames{\dropn{x}} & := & \{ x \}
\end{eqnarray*}

The bound names of a process, $\boundnames{P}$, are those names occurring in $P$
that are not free. For example, in $x?(y).0$, the name $x$ is free, while $y$ is bound.

\begin{mathpar}
  \inferrule* [lab=monoidal-laws] {} { P|Q \equiv Q|P \and P|0 \equiv P \and P|(Q|R) \equiv (P|Q)|R }
\end{mathpar}

\begin{mathpar}
  \inferrule* [lab=alpha-equivalence] {} { (x)P \equiv (y)P\{y/x\} \and y \not\in \freenames{P} }
\end{mathpar}

\begin{definition}
Then two processes, $P,Q$, are alpha-equivalent if $P = Q\{\vec{y}/\vec{x}\}$ for
some $\vec{x} \in \boundnames{Q},\vec{y} \in \boundnames{P}$, where $Q\{\vec{y}/\vec{x}\}$
denotes the capture-avoiding substitution of $\vec{y}$ for $\vec{x}$ in $Q$.
\end{definition}

\begin{definition}
  The {\em structural congruence} \cite{SangiorgiWalker} , $\equiv$,
  between processes is the least congruence containing
  alpha-equivalence, satisfying the abelian monoid laws
  (associativity, commutativity and $\pzero$ as identity) for parallel
  composition $|$ and for summation $+$.
\end{definition}

\subsection{Name equivalence}

We take name equivalence, written $\nameeq$, to be the smallest
equivalence relation generated by the following rules.

\begin{mathpar}
\inferrule*[lab=Quote-drop]
{ }
{ \quotep{@{x}} \nameeq x }

\inferrule*[lab=Struct-equiv]
{ P \scong Q }
{ \quotep{P} \nameeq \quotep{Q} }
\end{mathpar}

The astute reader will have noticed that the mutual recursion of names
and processes imposes a mutual recursion on alpha-equivalence and
structural equivalence via name-equivalence. Fortunately, all of this
works out pleasantly and we may calculate in the natural way, free of
concern. The reader interested in the details is referred to the
appendix \ref{appendix:rho_details}.

\subsection{Substitution}

We use $\Proc$ for the set of processes, $\QProc$ for the set of
names, and $\id{\{}\vec{y} / \vec{x} \id{\}}$ to denote partial maps,
$s : \QProc \rightarrow \QProc$. A map, $s$ lifts, uniquely, to a map
on process terms, $\widehat{s} : \Proc \rightarrow \Proc$ by the
following equations.

\begin{mathpar}
  (0) \psubstp{Q}{P} := 0 \\
  (R \juxtap S) \psubstp{Q}{P}
  :=    
  (R)\psubstp{Q}{P} \juxtap (S) \psubstp{Q}{P} \\
  (x?(y).R) \psubstp{Q}{P}    
  :=    
  (x)\substp{Q}{P} (z)\concat( (R \psubstn{z}{y}) \psubstp{Q}{P} ) \\
  (\lift{x}{R}) \psubstp{Q}{P}  
  :=
  \lift{(x)\substp{Q}{P}}{ R \psubstp{Q}{P} } \\
%   (\dropn{x})  \psubstp{Q}{P}       
%   := 
%   \left\{ 
%     \begin{array}{ccc} 
%       \dropn{\quotep{Q}} & & x \nameeq \quotep{P} \\
%       \dropn{x} & & otherwise \\
%     \end{array}
%   \right. 
  (\dropn{x})  \psubstp{Q}{P}       
  := 
  \left\{ 
    \begin{array}{ccc} 
      Q & & x \nameeq \quotep{P} \\
      \dropn{x} & & otherwise \\
    \end{array}
  \right.
\end{mathpar}
 

where

\begin{eqnarray}
  (x)\id{\{} \lpquote Q \rpquote / \lpquote P \rpquote \id{\}}            = 
  \left\{ 
    \begin{array}{ccc}
      \lpquote Q \rpquote & & x \nameeq \lpquote P \rpquote \\
      x & & otherwise \\
    \end{array}
  \right. \nonumber
\end{eqnarray}

and $z$ is chosen distinct from $\quotep{P}$, $\quotep{Q}$, the free
names in $Q$, and all the names in $R$. Our $\alpha$-equivalence will
be built in the standard way from this substitution.

\begin{remark}\label{rem:no_self_referential_names}
  One consequence of these definitions is that $\forall P. \quotep{P}
  \not\in \freenames{P}$.
\end{remark}

\subsection{ Dynamic quote: an example }

Anticipating something of what's to come, consider applying the
substitution, $\widehat{\id{\{}u / z \id{\}}}$, to the following pair
of processes, $\lift{w}{y!(z)}$ and $w[ \lpquote y!(z) \rpquote ]$.

\begin{eqnarray}
	\lift{w}{y!(z)}\widehat{\id{\{}u / z \id{\}}}
		& = &
		\lift{w}{y!(u)} \nonumber\\
	w[ \lpquote y!(z) \rpquote ] \widehat{ \id{\{}u / z \id{\}} }
		& = &
		w[ \lpquote y!(z) \rpquote ] \nonumber
\end{eqnarray}

Because the body of the process between quotes is impervious to
substitution, we get radically different answers. In fact, by
examining the first process in an input context,
e.g. $x?(z).\lift{w}{y!(z)}$, we see that the process under the lift
operator may be shaped by prefixed inputs binding a name inside it. In
this sense, the lift operator will be seen as a way to dynamically
construct processes before reifying them as names.

Finally equipped with these standard features we can present the
dynamics of the calculus.

\subsubsection{Operational semantics} 

Finally, we introduce the computational dynamics. What marks these
algebras as distinct from other more traditionally studied algebraic
structures, e.g. vector spaces or polynomial rings, is the manner in
which dynamics is captured. In traditional structures, dynamics is typically
expressed through morphisms between such structures, as in linear maps
between vector spaces or morphisms between rings. In algebras
associated with the semantics of computation, the dynamics is
expressed as part of the algebraic structure itself, through a
reduction reduction relation typically denoted by $\red$. Below, we
give a recursive presentation of this relation for the calculus used
in the encoding.

$\red \subseteq \pi \times \pi$
$\red : \pi \to \mathcal{P}(\pi)$

\begin{mathpar}
  \inferrule* [lab=Comm] { \textsf{match}( x_{src}, x_{trgt} ) } { x_{trgt}?(y)P \; | \; x_{src}!\langle {Q} \rangle \red P\{\quotep{Q}/y}\} }
  \and \\
  \inferrule* [lab=Par] {{P} \red {P}'} {{{P} | {Q}} \red {{P}' | {Q}}}
  \and
  \inferrule* [lab=Equiv]{{{P} \scong {P}'} \andalso {{P}' \red {Q}'} \andalso {{Q}' \scong {Q}}}{{P} \red {Q}}
\end{mathpar}

\begin{eqnarray*}
  match_{\equiv} (\quotep{P},\quotep{Q}) & := & P \equiv Q \\
  match_{\dagger}(\quotep{P},\quotep{Q}) & := & \forall R. P|Q \red^{*} R => R \red^{*} 0 \\
  match_{K}(\quotep{P},\quotep{Q}) & := & K \mbox{ for some context } K
\end{eqnarray*}

$u?(x)P | u!\langle Q \rangle \red P\{\quotep{Q}/x\}$

%We write $\wred$ for $\red^*$, and $P\red$ if $\exists Q $ such that $ P \red Q$.
We write $P\red$ if $\exists Q $ such that $ P \red Q$ and $P\not\red$, otherwise.

\section{Replication}

As mentioned before, it is known that replication (and hence
recursion) can be implemented in a higher-order process algebra
\cite{SangiorgiWalker}. As our first example of calculation with the
machinery thus far presented we give the construction explicitly in
the {\rhoc}.

\begin{eqnarray}
	D_{x} & := & \prefix{x}{y}{(\binpar{\outputp{x}{y}}{@{y}})} \nonumber\\
	\bangp_{x}{P} & := & \binpar{{x}!\langle{\binpar{D_{x}}{P}}\rangle}{D_{x}} \nonumber
\end{eqnarray}

\begin{eqnarray}
	\bangp_{x}{P} & & \nonumber\\
	=
	& {x}!\langle{(\prefix{x}{y}{(\outputp{x}{y} | @{y})) | P}}\rangle 
	      | \prefix{x}{y}{(\outputp{x}{y} | @{y})} & \nonumber\\
	\red
	& (\outputp{x}{y} | @{y})\substn{\quotep{(\prefix{x}{y}{(@{y} | \outputp{x}{y})) | P}}}{y} & \nonumber\\
	=
	& \outputp{x}{\quotep{(\prefix{x}{y}{(\outputp{x}{y} | @{y})) | P}}}
	  | {(\prefix{x}{y}{(\outputp{x}{y} | @{y})) | P}} & \nonumber\\
	\red
	& \ldots & \nonumber\\
	\red^*
	& P | P | \ldots & \nonumber
\end{eqnarray}

Of course, this encoding, as an implementation, runs away, unfolding
$\bangp{P}$ eagerly. A lazier and more implementable replication
operator, restricted to input-guarded processes, may be obtained as follows.

\begin{eqnarray}
\bangp{\prefix{u}{v}{P}} 
	:= 
	\binpar{\lift{x}{\prefix{u}{v}{(\binpar{D(x)}{P})}}}{D(x)} \nonumber
\end{eqnarray}

\begin{remark}
  Note that the lazier definition still does not deal with summation
  or mixed summation (i.e. sums over input and output). The reader is
  invited to construct definitions of replication that deal with these
  features. 

  Further, the definitions are parameterized in a name, $x$. Can you,
  gentle reader, make a definition that eliminates this parameter and
  guarantees no accidental interaction between the replication
  machinery and the process being replicated -- i.e. no accidental
  sharing of names used by the process to get its work done and the
  name(s) used by the replication to effect copying. This latter
  revision of the definition of replication is crucial to obtaining
  the expected identity $!!P \sim !P$.
\end{remark}

\begin{remark}\label{rem:paradoxical_combinator}
  The reader familiar with the lambda calculus will have noticed the
  similarity between $D$ and the paradoxical combinator.

  [Ed. note: the existence of this seems to suggest we have to be more
  restrictive on the set of processes and names we admit if we are to
  support no-cloning.]
\end{remark}

\subsubsection{Bisimulation}

The computational dynamics gives rise to another kind of equivalence,
the equivalence of computational behavior. As previously mentioned
this is typically captured \emph{via} some form of bisimulation.

% The notion we use in this paper is weak barbed bisimulation
% \cite{milner91polyadicpi}.

The notion we use in this paper is derived from weak barbed
bisimulation \cite{milner91polyadicpi}. 

\begin{definition}
An \emph{observation relation}, $\downarrow_{\mathcal N}$, over a set
of names, $\mathcal N$, is the smallest relation satisfying the rules
below.

\infrule[Out-barb]{y \in {\mathcal N}, \; x \nameeq y}
		  {\outputp{x}{v} \downarrow_{\mathcal N} x}
\infrule[Par-barb]{\mbox{$P\downarrow_{\mathcal N} x$ or $Q\downarrow_{\mathcal N} x$}}
		  {\binpar{P}{Q} \downarrow_{\mathcal N} x}

We write $P \Downarrow_{\mathcal N} x$ if there is $Q$ such that 
$P \wred Q$ and $Q \downarrow_{\mathcal N} x$.
\end{definition}

\begin{definition}
%\label{def.bbisim}
An  ${\mathcal N}$-\emph{barbed bisimulation} over a set of names, ${\mathcal N}$, is a symmetric binary relation 
${\mathcal S}_{\mathcal N}$ between agents such that $P\rel{S}_{\mathcal N}Q$ implies:
\begin{enumerate}
\item If $P \red P'$ then $Q \wred Q'$ and $P'\rel{S}_{\mathcal N} Q'$.
\item If $P\downarrow_{\mathcal N} x$, then $Q\Downarrow_{\mathcal N} x$.
\end{enumerate}
$P$ is ${\mathcal N}$-barbed bisimilar to $Q$, written
$P \wbbisim_{\mathcal N} Q$, if $P \rel{S}_{\mathcal N} Q$ for some ${\mathcal N}$-barbed bisimulation ${\mathcal S}_{\mathcal N}$.
\end{definition}

$\mathcal{R} \subseteq \pi \times \pi$

$P \mathcal{R} Q => \forall P'. P \red P' \Rightarrow \exists Q'. Q \red Q', P' \mathcal{R} Q'$

$P \vdash x \Rightarrow Q \vdash x$

\begin{mathpar}
  \inferrule*[lab=Out-barb]{x \nameeq y}{{y}!\langle{Q}\rangle \vdash x}
  \and
  \inferrule*[lab=Par-barb]{\mbox{$P\vdash x$ or $Q\vdash x$}}{\binpar{P}{Q} \vdash x}
\end{mathpar}

\subsubsection{Contexts}

One of the principle advantages of computational calculi like the
$\pi$-calculus is a well-defined notion of context,
contextual-equivalence and a correlation between
contextual-equivalence and notions of bisimulation. The notion of
context allows the decomposition of a process into (sub-)process and
its syntactic environment, its context. Thus, a context may be
thought of as a process with a ``hole'' (written $\Box$) in it. The
application of a context $M$ to a process $P$, written $M[P]$, is
tantamount to filling the hole in $M$ with $P$. In this paper we do
not need the full weight of this theory, but do make use of the notion
of context in the proof the main theorem. 

\begin{mathpar}
  \inferrule* [lab=summation] {} {{M_{M},M_{N}} \bc \Box \;|\; x.M_{A} \;|\; M_{M}+M_{N}}
  \and
  \inferrule* [lab=agent] {} {{M_{A}} \bc (\vec{x})M_{P} \;| \; \clift{P_0,\ldots,M_{P},\ldots,P_N}}
  \and \\
  \inferrule* [lab=process] {} {{M_{P}} \bc M_{N} \;| \;P|M_{P} }
\end{mathpar} 

\begin{mathpar}
  \inferrule* [lab=sychronization] {} {M_{N} \bc \Box \;|\; x?M_{F} \;|\; x!M_{C}}
  \and
  \inferrule* [lab=abstraction] {} {{M_{F}} \bc (x)M_{P} }
  \and
  \inferrule* [lab=concretion] {} {{M_{C}} \bc \langle M_{P} \rangle }
  \and \\
  \inferrule* [lab=process] {} {{M_{P}} \bc M_{N} \;| \;P|M_{P} }
\end{mathpar}

\begin{definition}[contextual application] Given a context $M$, and
  process $P$, we define the \emph{contextual application}, $M[P] :=
  M\{P/\Box\}$. That is, the contextual application of M to P is the
  substitution of $P$ for $\Box$ in $M$.
\end{definition}

$\meaningof{-} : L \to \mathcal{P}(\pi)$

\begin{mathpar}
  \inferrule* [lab=collection] {} {\meaningof{true} = \pi, \and \meaningof{~E} = \pi \setminus \meaningof{E}, \and \meaningof{E_{1} \& E_{2}} = \meaningof{E_{1}} \cap \meaningof{E_{2}}}
\end{mathpar}

\begin{mathpar}
  \inferrule* [lab=structure] {} {\meaningof{0} = \{ P \in \pi | P \equiv 0 \}, \and \\ \meaningof{E_1 | E_2} = \{ P \in \pi | P \equiv P_{1} | P_{2}, P_{1} \in \meaningof{E_{1}}, P_{2} \in \meaningof{E_2}\} }
\end{mathpar}

\begin{mathpar}
 \inferrule* [lab=behavior] {} {\meaningof{\langle a?b \rangle E} = \{ P \in \pi | P \equiv Q | u?(y)P', \\ \and \\\\ \and \\ \;\;\; u \in \meaningof{a}, \forall z.P'\{z/y\} \in \meaningof{E\{z/b\}}\}, \and \\ \meaningof{a!E} = \{ P \in \pi | P \equiv Q | x!\langle P' \rangle, x \in \meaningof{a} P' \in \meaningof{E}\} }
\end{mathpar}

\begin{mathpar}
 \inferrule* [lab=nominal] {} {\meaningof{\quotep{E}} = \{ \quotep{P} \in \quotep{\pi} | P \in \meaningof{E} \}, \and \meaningof{\quotep{P}} = \{ \quotep{Q} \in \quotep{\pi} | P \equiv Q \} \and \\ \meaningof{@\quotep{E}} = \{ P \in \pi | P \equiv @x, x \in \meaningof{E} \}}
\end{mathpar}

\begin{eqnarray*}
  \\
  \meaningof{-} : TS \to ST
\end{eqnarray*}

\begin{eqnarray*}
  \\
  L : TS \to ST
\end{eqnarray*}

\begin{eqnarray*}
  \\
  P \models E \iff P \in \meaningof{E}
\end{eqnarray*}

\begin{eqnarray*}
  P \approx_{L} Q \iff \forall E \in L. P \models E \iff Q \models E
\end{eqnarray*}

\begin{eqnarray*}
  P \approx_{K} Q
\end{eqnarray*}

\begin{eqnarray*}
  P \approx Q
\end{eqnarray*}

$\approx_{K} = \approx = \approx_{L}$

\subsubsection{Contextual duality}

Note that contexts extend the quotation operation to a family of
operations from processes to names. Given a context, $M$, we can
define a \emph{nominal context}, $\quotep{M}$ by $\quotep{M}[P] :=
\quotep{M[P]}$. To foreshadow what is to come we observe that these
operations enjoy a duality with processes very much like the duality
between vectors and maps from vectors to scalars.

Further, because the calculus is essentially higher-order, we have a
correspondence between contexts and processes. More specifically,
given a name $x$ and a context $M$ we can construct $M^{*}_{x}$ such
that 

\begin{mathpar}
  M^{*}_{x} | \lift{x}{P} \red M[P]
\end{mathpar}

namely,

\begin{mathpar}
  M^{*}_{x} := x?(u).M[\dropn{u}]
\end{mathpar}

The dependence of $M^{*}_{x}$ on a name makes it an abstraction, 

\begin{mathpar}
  M^{*} := (x)x?(u).M[\dropn{u}]
\end{mathpar}

\subsection{Additional notation}

It will sometimes be convenient to denote the process a name
quotes. We already have the notation $x = \quotep{P}$, but it will be
convenient to introduce an alternate notation, $\procn{x}$, when we
want to emphasize the connection to the use of the name. Note that, by
virtue of name equivalence, $\quotep{\procn{x}} \nameeq x$; so, the
notation is consistent with previous definitions.

Further, because names have structure it is possible to effect
substitutions on the basis of that structure. This means we need to
upgrade our notation for substitutions, which we accomplish by
adapting comprehension notation. Thus,

\begin{mathpar}
  P\{ y / x : x \in S \}
\end{mathpar}

is interpreted to mean the process derived from P by replacing (in a
capture-avoiding manner) each occurrence of $x$ in $S$ by $y$. For example,

\begin{mathpar}
  P\{ \quotep{\procn{x}|\procn{x}} / x : x \in \freenames{P} \}
\end{mathpar}

will replace each (occurrence) of a free name $x$ in $P$ by
$\quotep{\procn{x}|\procn{x}}$.

Also, we will avail ourselves of the notation $x^{L}$ and $x^{R}$ to
denote injections of a name into disjoint copies of the name
space. There are numerous ways to accomplish this. One example can be
found in \cite{MeredithR05}. This notation overloads to vectors of
names: $\vec{x}^{\pi} := (x_{i}^{\pi} \; : \; 0 \leq i < |\vec{x}| )$ where $\pi \in \{L,R\}$.

We also use $P^{\Box} := P|\Box$.

In \cite{MeredithR05} an interpretation of the new operator is
given. It turns out that there are several possible interpretations
all enjoying the requisite algebraic properties of the operator (see
\cite{milner91polyadicpi}). We will therefore make liberal use of
$(\nu\; \vec{x})P$.

% subsection the_syntax_and_semantics_of_the_notation_system (end)   

\input{qm2pi.qmops} 

\input{qm2pi.sterngerlach} 

\input{qm2pi.metric} 

% section concurrent_process_calculi (end)

%\input{qm2pi.proofsketch}

% section proof sketch (end)

%\input{qm2pi.slviaknots} 

% section spatial logic via knots (end)

\input{qm2pi.conclusion}

% section conclusion (end)

%\input{qm2pi.dtcodes} 

% section wiring algorithm (end)

\input{qm2pi.ack} 

% section acknowledgments (end)

\newpage


\bibliographystyle{plain}   
\bibliography{../../biblios/main.bib}

\input{qm2pi.rhodetails}

\end{document}

 

% subsection basic_interpretation (end)

%\input{qm2pi.rho.presentation} 
\subsection{The syntax and semantics of the notation system}\label{sub:the_syntax_and_semantics_of_the_notation_system} % (fold)

We now summarize a technical presentation of the calculus that
embodies our theory of dynamics. The typical presentation of such a
calculus follows the style of giving generators and relations on
them. The grammar, below, describing term constructors, freely
generates the set of processes, $\Proc$. This set is then quotiented
by a relation known as structural congruence and it is over this set
that the notion of dynamics is expressed. This presentation is
essentially that of \cite{MeredithR05} with the addition of
polyadicity and summation. For readability we have relegated some of
the technical subtleties to an appendix.

\subsubsection{Process grammar}\label{subsub:process_grammar}

\begin{mathpar}
  \inferrule* [lab=synchronization] {} {{M} \bc \pzero \;|\; x?F \;|\; x!C }
  \and
  \inferrule* [lab=abstraction] {} {{F} \bc (x)P}
  \and
  \inferrule* [lab=concretion] {} {{C} \bc \langle Q \rangle}
  \and
  \inferrule* [lab=process] {} {{P,Q} \bc M \;| \;P|Q \;|\; @{x}}
  \and
  \inferrule* [lab=name] {} {{x} \bc \quotep{P}}
\end{mathpar} 

Note that $\vec{x}$ (resp. $\vec{P}$) denotes a vector of names
(resp. processes) of length $|\vec{x}|$ (resp. $|\vec{P}|$). We adopt
the following useful abbreviations.

\begin{mathpar}
   x?(\vec{y}).P := x.(\vec{y})P \and  x\clift{\vec{P}} := x.\clift{\vec{P}}
   \and x!(y) := \lift{x}{\dropn{y}}
   \and \Pi_{i=0}^{n-1}P_i := P_0 | \ldots | P_{n-1}
\end{mathpar}

\subsubsection{Structural congruence}

\paragraph{Free and bound names and alpha-equivalence.} At the
core of structural equivalence is alpha-equivalence which identifies
process that are the same up to a change of variable. Formally, we
recognize the distinction between free and bound names. The free names
of a process, $\freenames{P}$, may be calculated recursively as
follows:

\begin{mathpar}
\freenames{\pzero} := \emptyset
  \and \\
  \freenames{x?(y).P} := \{ x \} \cup (\freenames{P} \setminus \{ y \})
  \and 
  \freenames{x!\langle P \rangle} := \{ x \} \cup \{ P \} 
  \and \\
  \freenames{P|Q} := \freenames{P} \cup \freenames{Q}
  \and \\
  \freenames{@{x}} := \{ x \}
\end{mathpar}

$\pi$
$\quotep{\pi}$

$\freenames{-} : \pi \to \mathcal{P}(\quotep{\pi})$

\begin{eqnarray*}
  \freenames{\pzero} & := & \emptyset \\
  \freenames{x?(y).P} & := & \{ x \} \cup (\freenames{P} \setminus \{ y \}) \\
  \freenames{x!\langle P \rangle} & := & \{ x \} \cup \{ P \} \\
  \freenames{P|Q} & := & \freenames{P} \cup \freenames{Q} \\
  \freenames{\dropn{x}} & := & \{ x \}
\end{eqnarray*}

The bound names of a process, $\boundnames{P}$, are those names occurring in $P$
that are not free. For example, in $x?(y).0$, the name $x$ is free, while $y$ is bound.

\begin{mathpar}
  \inferrule* [lab=monoidal-laws] {} { P|Q \equiv Q|P \and P|0 \equiv P \and P|(Q|R) \equiv (P|Q)|R }
\end{mathpar}

\begin{mathpar}
  \inferrule* [lab=alpha-equivalence] {} { (x)P \equiv (y)P\{y/x\} \and y \not\in \freenames{P} }
\end{mathpar}

\begin{definition}
Then two processes, $P,Q$, are alpha-equivalent if $P = Q\{\vec{y}/\vec{x}\}$ for
some $\vec{x} \in \boundnames{Q},\vec{y} \in \boundnames{P}$, where $Q\{\vec{y}/\vec{x}\}$
denotes the capture-avoiding substitution of $\vec{y}$ for $\vec{x}$ in $Q$.
\end{definition}

\begin{definition}
  The {\em structural congruence} \cite{SangiorgiWalker} , $\equiv$,
  between processes is the least congruence containing
  alpha-equivalence, satisfying the abelian monoid laws
  (associativity, commutativity and $\pzero$ as identity) for parallel
  composition $|$ and for summation $+$.
\end{definition}

\subsection{Name equivalence}

We take name equivalence, written $\nameeq$, to be the smallest
equivalence relation generated by the following rules.

\begin{mathpar}
\inferrule*[lab=Quote-drop]
{ }
{ \quotep{@{x}} \nameeq x }

\inferrule*[lab=Struct-equiv]
{ P \scong Q }
{ \quotep{P} \nameeq \quotep{Q} }
\end{mathpar}

The astute reader will have noticed that the mutual recursion of names
and processes imposes a mutual recursion on alpha-equivalence and
structural equivalence via name-equivalence. Fortunately, all of this
works out pleasantly and we may calculate in the natural way, free of
concern. The reader interested in the details is referred to the
appendix \ref{appendix:rho_details}.

\subsection{Substitution}

We use $\Proc$ for the set of processes, $\QProc$ for the set of
names, and $\id{\{}\vec{y} / \vec{x} \id{\}}$ to denote partial maps,
$s : \QProc \rightarrow \QProc$. A map, $s$ lifts, uniquely, to a map
on process terms, $\widehat{s} : \Proc \rightarrow \Proc$ by the
following equations.

\begin{mathpar}
  (0) \psubstp{Q}{P} := 0 \\
  (R \juxtap S) \psubstp{Q}{P}
  :=    
  (R)\psubstp{Q}{P} \juxtap (S) \psubstp{Q}{P} \\
  (x?(y).R) \psubstp{Q}{P}    
  :=    
  (x)\substp{Q}{P} (z)\concat( (R \psubstn{z}{y}) \psubstp{Q}{P} ) \\
  (\lift{x}{R}) \psubstp{Q}{P}  
  :=
  \lift{(x)\substp{Q}{P}}{ R \psubstp{Q}{P} } \\
%   (\dropn{x})  \psubstp{Q}{P}       
%   := 
%   \left\{ 
%     \begin{array}{ccc} 
%       \dropn{\quotep{Q}} & & x \nameeq \quotep{P} \\
%       \dropn{x} & & otherwise \\
%     \end{array}
%   \right. 
  (\dropn{x})  \psubstp{Q}{P}       
  := 
  \left\{ 
    \begin{array}{ccc} 
      Q & & x \nameeq \quotep{P} \\
      \dropn{x} & & otherwise \\
    \end{array}
  \right.
\end{mathpar}
 

where

\begin{eqnarray}
  (x)\id{\{} \lpquote Q \rpquote / \lpquote P \rpquote \id{\}}            = 
  \left\{ 
    \begin{array}{ccc}
      \lpquote Q \rpquote & & x \nameeq \lpquote P \rpquote \\
      x & & otherwise \\
    \end{array}
  \right. \nonumber
\end{eqnarray}

and $z$ is chosen distinct from $\quotep{P}$, $\quotep{Q}$, the free
names in $Q$, and all the names in $R$. Our $\alpha$-equivalence will
be built in the standard way from this substitution.

\begin{remark}\label{rem:no_self_referential_names}
  One consequence of these definitions is that $\forall P. \quotep{P}
  \not\in \freenames{P}$.
\end{remark}

\subsection{ Dynamic quote: an example }

Anticipating something of what's to come, consider applying the
substitution, $\widehat{\id{\{}u / z \id{\}}}$, to the following pair
of processes, $\lift{w}{y!(z)}$ and $w[ \lpquote y!(z) \rpquote ]$.

\begin{eqnarray}
	\lift{w}{y!(z)}\widehat{\id{\{}u / z \id{\}}}
		& = &
		\lift{w}{y!(u)} \nonumber\\
	w[ \lpquote y!(z) \rpquote ] \widehat{ \id{\{}u / z \id{\}} }
		& = &
		w[ \lpquote y!(z) \rpquote ] \nonumber
\end{eqnarray}

Because the body of the process between quotes is impervious to
substitution, we get radically different answers. In fact, by
examining the first process in an input context,
e.g. $x?(z).\lift{w}{y!(z)}$, we see that the process under the lift
operator may be shaped by prefixed inputs binding a name inside it. In
this sense, the lift operator will be seen as a way to dynamically
construct processes before reifying them as names.

Finally equipped with these standard features we can present the
dynamics of the calculus.

\subsubsection{Operational semantics} 

Finally, we introduce the computational dynamics. What marks these
algebras as distinct from other more traditionally studied algebraic
structures, e.g. vector spaces or polynomial rings, is the manner in
which dynamics is captured. In traditional structures, dynamics is typically
expressed through morphisms between such structures, as in linear maps
between vector spaces or morphisms between rings. In algebras
associated with the semantics of computation, the dynamics is
expressed as part of the algebraic structure itself, through a
reduction reduction relation typically denoted by $\red$. Below, we
give a recursive presentation of this relation for the calculus used
in the encoding.

$\red \subseteq \pi \times \pi$
$\red : \pi \to \mathcal{P}(\pi)$

\begin{mathpar}
  \inferrule* [lab=Comm] { \textsf{match}( x_{src}, x_{trgt} ) } { x_{trgt}?(y)P \; | \; x_{src}!\langle {Q} \rangle \red P\{\quotep{Q}/y}\} }
  \and \\
  \inferrule* [lab=Par] {{P} \red {P}'} {{{P} | {Q}} \red {{P}' | {Q}}}
  \and
  \inferrule* [lab=Equiv]{{{P} \scong {P}'} \andalso {{P}' \red {Q}'} \andalso {{Q}' \scong {Q}}}{{P} \red {Q}}
\end{mathpar}

\begin{eqnarray*}
  match_{\equiv} (\quotep{P},\quotep{Q}) & := & P \equiv Q \\
  match_{\dagger}(\quotep{P},\quotep{Q}) & := & \forall R. P|Q \red^{*} R => R \red^{*} 0 \\
  match_{K}(\quotep{P},\quotep{Q}) & := & K \mbox{ for some context } K
\end{eqnarray*}

$u?(x)P | u!\langle Q \rangle \red P\{\quotep{Q}/x\}$

%We write $\wred$ for $\red^*$, and $P\red$ if $\exists Q $ such that $ P \red Q$.
We write $P\red$ if $\exists Q $ such that $ P \red Q$ and $P\not\red$, otherwise.

\section{Replication}

As mentioned before, it is known that replication (and hence
recursion) can be implemented in a higher-order process algebra
\cite{SangiorgiWalker}. As our first example of calculation with the
machinery thus far presented we give the construction explicitly in
the {\rhoc}.

\begin{eqnarray}
	D_{x} & := & \prefix{x}{y}{(\binpar{\outputp{x}{y}}{@{y}})} \nonumber\\
	\bangp_{x}{P} & := & \binpar{{x}!\langle{\binpar{D_{x}}{P}}\rangle}{D_{x}} \nonumber
\end{eqnarray}

\begin{eqnarray}
	\bangp_{x}{P} & & \nonumber\\
	=
	& {x}!\langle{(\prefix{x}{y}{(\outputp{x}{y} | @{y})) | P}}\rangle 
	      | \prefix{x}{y}{(\outputp{x}{y} | @{y})} & \nonumber\\
	\red
	& (\outputp{x}{y} | @{y})\substn{\quotep{(\prefix{x}{y}{(@{y} | \outputp{x}{y})) | P}}}{y} & \nonumber\\
	=
	& \outputp{x}{\quotep{(\prefix{x}{y}{(\outputp{x}{y} | @{y})) | P}}}
	  | {(\prefix{x}{y}{(\outputp{x}{y} | @{y})) | P}} & \nonumber\\
	\red
	& \ldots & \nonumber\\
	\red^*
	& P | P | \ldots & \nonumber
\end{eqnarray}

Of course, this encoding, as an implementation, runs away, unfolding
$\bangp{P}$ eagerly. A lazier and more implementable replication
operator, restricted to input-guarded processes, may be obtained as follows.

\begin{eqnarray}
\bangp{\prefix{u}{v}{P}} 
	:= 
	\binpar{\lift{x}{\prefix{u}{v}{(\binpar{D(x)}{P})}}}{D(x)} \nonumber
\end{eqnarray}

\begin{remark}
  Note that the lazier definition still does not deal with summation
  or mixed summation (i.e. sums over input and output). The reader is
  invited to construct definitions of replication that deal with these
  features. 

  Further, the definitions are parameterized in a name, $x$. Can you,
  gentle reader, make a definition that eliminates this parameter and
  guarantees no accidental interaction between the replication
  machinery and the process being replicated -- i.e. no accidental
  sharing of names used by the process to get its work done and the
  name(s) used by the replication to effect copying. This latter
  revision of the definition of replication is crucial to obtaining
  the expected identity $!!P \sim !P$.
\end{remark}

\begin{remark}\label{rem:paradoxical_combinator}
  The reader familiar with the lambda calculus will have noticed the
  similarity between $D$ and the paradoxical combinator.

  [Ed. note: the existence of this seems to suggest we have to be more
  restrictive on the set of processes and names we admit if we are to
  support no-cloning.]
\end{remark}

\subsubsection{Bisimulation}

The computational dynamics gives rise to another kind of equivalence,
the equivalence of computational behavior. As previously mentioned
this is typically captured \emph{via} some form of bisimulation.

% The notion we use in this paper is weak barbed bisimulation
% \cite{milner91polyadicpi}.

The notion we use in this paper is derived from weak barbed
bisimulation \cite{milner91polyadicpi}. 

\begin{definition}
An \emph{observation relation}, $\downarrow_{\mathcal N}$, over a set
of names, $\mathcal N$, is the smallest relation satisfying the rules
below.

\infrule[Out-barb]{y \in {\mathcal N}, \; x \nameeq y}
		  {\outputp{x}{v} \downarrow_{\mathcal N} x}
\infrule[Par-barb]{\mbox{$P\downarrow_{\mathcal N} x$ or $Q\downarrow_{\mathcal N} x$}}
		  {\binpar{P}{Q} \downarrow_{\mathcal N} x}

We write $P \Downarrow_{\mathcal N} x$ if there is $Q$ such that 
$P \wred Q$ and $Q \downarrow_{\mathcal N} x$.
\end{definition}

\begin{definition}
%\label{def.bbisim}
An  ${\mathcal N}$-\emph{barbed bisimulation} over a set of names, ${\mathcal N}$, is a symmetric binary relation 
${\mathcal S}_{\mathcal N}$ between agents such that $P\rel{S}_{\mathcal N}Q$ implies:
\begin{enumerate}
\item If $P \red P'$ then $Q \wred Q'$ and $P'\rel{S}_{\mathcal N} Q'$.
\item If $P\downarrow_{\mathcal N} x$, then $Q\Downarrow_{\mathcal N} x$.
\end{enumerate}
$P$ is ${\mathcal N}$-barbed bisimilar to $Q$, written
$P \wbbisim_{\mathcal N} Q$, if $P \rel{S}_{\mathcal N} Q$ for some ${\mathcal N}$-barbed bisimulation ${\mathcal S}_{\mathcal N}$.
\end{definition}

$\mathcal{R} \subseteq \pi \times \pi$

$P \mathcal{R} Q => \forall P'. P \red P' \Rightarrow \exists Q'. Q \red Q', P' \mathcal{R} Q'$

$P \vdash x \Rightarrow Q \vdash x$

\begin{mathpar}
  \inferrule*[lab=Out-barb]{x \nameeq y}{{y}!\langle{Q}\rangle \vdash x}
  \and
  \inferrule*[lab=Par-barb]{\mbox{$P\vdash x$ or $Q\vdash x$}}{\binpar{P}{Q} \vdash x}
\end{mathpar}

\subsubsection{Contexts}

One of the principle advantages of computational calculi like the
$\pi$-calculus is a well-defined notion of context,
contextual-equivalence and a correlation between
contextual-equivalence and notions of bisimulation. The notion of
context allows the decomposition of a process into (sub-)process and
its syntactic environment, its context. Thus, a context may be
thought of as a process with a ``hole'' (written $\Box$) in it. The
application of a context $M$ to a process $P$, written $M[P]$, is
tantamount to filling the hole in $M$ with $P$. In this paper we do
not need the full weight of this theory, but do make use of the notion
of context in the proof the main theorem. 

\begin{mathpar}
  \inferrule* [lab=summation] {} {{M_{M},M_{N}} \bc \Box \;|\; x.M_{A} \;|\; M_{M}+M_{N}}
  \and
  \inferrule* [lab=agent] {} {{M_{A}} \bc (\vec{x})M_{P} \;| \; \clift{P_0,\ldots,M_{P},\ldots,P_N}}
  \and \\
  \inferrule* [lab=process] {} {{M_{P}} \bc M_{N} \;| \;P|M_{P} }
\end{mathpar} 

\begin{mathpar}
  \inferrule* [lab=sychronization] {} {M_{N} \bc \Box \;|\; x?M_{F} \;|\; x!M_{C}}
  \and
  \inferrule* [lab=abstraction] {} {{M_{F}} \bc (x)M_{P} }
  \and
  \inferrule* [lab=concretion] {} {{M_{C}} \bc \langle M_{P} \rangle }
  \and \\
  \inferrule* [lab=process] {} {{M_{P}} \bc M_{N} \;| \;P|M_{P} }
\end{mathpar}

\begin{definition}[contextual application] Given a context $M$, and
  process $P$, we define the \emph{contextual application}, $M[P] :=
  M\{P/\Box\}$. That is, the contextual application of M to P is the
  substitution of $P$ for $\Box$ in $M$.
\end{definition}

$\meaningof{-} : L \to \mathcal{P}(\pi)$

\begin{mathpar}
  \inferrule* [lab=collection] {} {\meaningof{true} = \pi, \and \meaningof{~E} = \pi \setminus \meaningof{E}, \and \meaningof{E_{1} \& E_{2}} = \meaningof{E_{1}} \cap \meaningof{E_{2}}}
\end{mathpar}

\begin{mathpar}
  \inferrule* [lab=structure] {} {\meaningof{0} = \{ P \in \pi | P \equiv 0 \}, \and \\ \meaningof{E_1 | E_2} = \{ P \in \pi | P \equiv P_{1} | P_{2}, P_{1} \in \meaningof{E_{1}}, P_{2} \in \meaningof{E_2}\} }
\end{mathpar}

\begin{mathpar}
 \inferrule* [lab=behavior] {} {\meaningof{\langle a?b \rangle E} = \{ P \in \pi | P \equiv Q | u?(y)P', \\ \and \\\\ \and \\ \;\;\; u \in \meaningof{a}, \forall z.P'\{z/y\} \in \meaningof{E\{z/b\}}\}, \and \\ \meaningof{a!E} = \{ P \in \pi | P \equiv Q | x!\langle P' \rangle, x \in \meaningof{a} P' \in \meaningof{E}\} }
\end{mathpar}

\begin{mathpar}
 \inferrule* [lab=nominal] {} {\meaningof{\quotep{E}} = \{ \quotep{P} \in \quotep{\pi} | P \in \meaningof{E} \}, \and \meaningof{\quotep{P}} = \{ \quotep{Q} \in \quotep{\pi} | P \equiv Q \} \and \\ \meaningof{@\quotep{E}} = \{ P \in \pi | P \equiv @x, x \in \meaningof{E} \}}
\end{mathpar}

\begin{eqnarray*}
  \\
  \meaningof{-} : TS \to ST
\end{eqnarray*}

\begin{eqnarray*}
  \\
  L : TS \to ST
\end{eqnarray*}

\begin{eqnarray*}
  \\
  P \models E \iff P \in \meaningof{E}
\end{eqnarray*}

\begin{eqnarray*}
  P \approx_{L} Q \iff \forall E \in L. P \models E \iff Q \models E
\end{eqnarray*}

\begin{eqnarray*}
  P \approx_{K} Q
\end{eqnarray*}

\begin{eqnarray*}
  P \approx Q
\end{eqnarray*}

$\approx_{K} = \approx = \approx_{L}$

\subsubsection{Contextual duality}

Note that contexts extend the quotation operation to a family of
operations from processes to names. Given a context, $M$, we can
define a \emph{nominal context}, $\quotep{M}$ by $\quotep{M}[P] :=
\quotep{M[P]}$. To foreshadow what is to come we observe that these
operations enjoy a duality with processes very much like the duality
between vectors and maps from vectors to scalars.

Further, because the calculus is essentially higher-order, we have a
correspondence between contexts and processes. More specifically,
given a name $x$ and a context $M$ we can construct $M^{*}_{x}$ such
that 

\begin{mathpar}
  M^{*}_{x} | \lift{x}{P} \red M[P]
\end{mathpar}

namely,

\begin{mathpar}
  M^{*}_{x} := x?(u).M[\dropn{u}]
\end{mathpar}

The dependence of $M^{*}_{x}$ on a name makes it an abstraction, 

\begin{mathpar}
  M^{*} := (x)x?(u).M[\dropn{u}]
\end{mathpar}

\subsection{Additional notation}

It will sometimes be convenient to denote the process a name
quotes. We already have the notation $x = \quotep{P}$, but it will be
convenient to introduce an alternate notation, $\procn{x}$, when we
want to emphasize the connection to the use of the name. Note that, by
virtue of name equivalence, $\quotep{\procn{x}} \nameeq x$; so, the
notation is consistent with previous definitions.

Further, because names have structure it is possible to effect
substitutions on the basis of that structure. This means we need to
upgrade our notation for substitutions, which we accomplish by
adapting comprehension notation. Thus,

\begin{mathpar}
  P\{ y / x : x \in S \}
\end{mathpar}

is interpreted to mean the process derived from P by replacing (in a
capture-avoiding manner) each occurrence of $x$ in $S$ by $y$. For example,

\begin{mathpar}
  P\{ \quotep{\procn{x}|\procn{x}} / x : x \in \freenames{P} \}
\end{mathpar}

will replace each (occurrence) of a free name $x$ in $P$ by
$\quotep{\procn{x}|\procn{x}}$.

Also, we will avail ourselves of the notation $x^{L}$ and $x^{R}$ to
denote injections of a name into disjoint copies of the name
space. There are numerous ways to accomplish this. One example can be
found in \cite{MeredithR05}. This notation overloads to vectors of
names: $\vec{x}^{\pi} := (x_{i}^{\pi} \; : \; 0 \leq i < |\vec{x}| )$ where $\pi \in \{L,R\}$.

We also use $P^{\Box} := P|\Box$.

In \cite{MeredithR05} an interpretation of the new operator is
given. It turns out that there are several possible interpretations
all enjoying the requisite algebraic properties of the operator (see
\cite{milner91polyadicpi}). We will therefore make liberal use of
$(\nu\; \vec{x})P$.

% subsection the_syntax_and_semantics_of_the_notation_system (end)   

\section{Interpretation of QM}
\subsection{Supporting definitions}
\subsubsection{Multiplication}
\begin{mathpar}
  \quotep{Q} \cdot \quotep{R} := \quotep{Q|R}
  \and \\
  \quotep{Q} \cdot P := P\{ \quotep{Q|R} / \quotep{R} : \quotep{R} \in \freenames{P} \}
\end{mathpar}

\paragraph{Discussion}
The first line needs little explanation. The second line says that
each free name of the process is replaced with the multiplication of
that name by the scalar. Multiplication of a scalar (name) by a state
(process) results in a process all the names of which have been `moved
over' by parallel composition with the process the scalar
quotes. There is a subtlety that the bound names have to be
manipulated so that multiplied names aren't accidentally
captured. There are many ways to achieve this.

\begin{remark}\label{rem:multiplication_identities}
  The reader is invited to verify that for all $x,y,z \in \QProc$ and $P \in \Proc$
  \begin{mathpar}
    x \cdot \quotep{0} \equiv x 
    \and
    x \cdot y \equiv y \cdot x
    \and
    x \cdot (y \cdot z) \equiv (x \cdot y) \cdot z
    \and \\
    \quotep{0} \cdot P \equiv P
    \and \\
    x \cdot (y \cdot P) \equiv (x \cdot y) \cdot P
    \and \\
    x \cdot (P|Q) \equiv (x \cdot P) | (x \cdot Q)
    \and \\    
  \end{mathpar}
\end{remark}

\subsubsection{Tensor product}

We define a tensor product on processes by structural induction.

\paragraph{Tensor of sums} First note that all summations, including
$\pzero$ and sequence, can be written $\Sigma_{i} x_{i}.A_{i} +
\Sigma_{j} x_{j}.C_{j}$, where we have grouped input-guarded processes
together and output-guarded processes together.

Thus, we can define the tensor product of two summations, $N_{1}\otimes N_{2}$, where

\begin{mathpar}
  N_{1} := \Sigma_{i} x_{i}.A_{i} + \Sigma_{j} x_{j}.C_{j}
  \and
  N_{2} := \Sigma_{i'} y_{i'}.B_{i'} + \Sigma_{j'} y_{j'}.D_{j'} 
\end{mathpar}

as follows.

\begin{mathpar}
  \Sigma_{i} x_{i}.A_{i} + \Sigma_{j} x_{j}.C_{j} \otimes \Sigma_{i'}
  y_{i'}.B_{i'} + \Sigma_{j'} y_{j'}.D_{j'} 
  \and \\
  := \; \Sigma_{i} \Sigma_{i'} \quotep{\stackrel{\vee}{x_{i}}| \stackrel{\vee}{y_{i'}}}.(A_{i}\otimes B_{i'}) \; | \; \Sigma_{i'} \Sigma_{i} \quotep{\stackrel{\vee}{y_{i'}}|\stackrel{\vee}{x_{i}}}.(B_{i'}\otimes A_{i})
  \and
  \;\; | \;\; \Sigma_{j} \Sigma_{j'} \quotep{\stackrel{\vee}{x_{j}}|\stackrel{\vee}{y_{j'}}}.(A_{j}\otimes B_{j'}) \; | \; \Sigma_{j'} \Sigma_{j} \quotep{\stackrel{\vee}{y_{j'}}|\stackrel{\vee}{x_{j}}}.(B_{j'}\otimes A_{j})
\end{mathpar}

\begin{remark}
  Do we need to $x^{L}$ and $y^{R}$ for this construction as well?
\end{remark}

\paragraph{Tensor of parallel compositions} Next, we distribute tensor
over par.

\begin{mathpar}
  P_{1}|P_{2} \otimes Q_{1}|Q_{2} := (P_{1} \otimes Q_{1}) | (P_{1}
  \otimes Q_{2}) | (P_{2} \otimes Q_{1}) | (P_{2} \otimes Q_{2})
\end{mathpar}

\paragraph{Tensor with dropped names} We treat tensor of a
process with a dropped name as parallel composition.

\begin{mathpar}
  P \otimes \dropn{x} := P | \dropn{x}
\end{mathpar}

\paragraph{Tensor of agents}

Finally, we need to define tensor on agents. Note that the definition
of tensor on normal products only tensors inputs with inputs and
outputs with outputs. Thus, we only have to define the operation on
``homogeneous'' pairings.

\begin{mathpar}
  (\vec{x})P \otimes (\vec{y})Q
  \and \\
  := (x_{0}^{L}|y_{0}^{R},\ldots,x_{0}^{L}|y_{n}^{R},\ldots,x_{m}^{L}|y_{0}^{R},\ldots,x_{m}^{L}|y_{n}^R)(P\{ \vec{x}^{L}/\vec{x}\} \otimes Q \{ \vec{y}^{R}/\vec{y}\})
  \and \\
  \clift{\vec{P}} \otimes \clift{\vec{Q}}
  \and \\
  := \clift{P_{0}\otimes Q_{0},\ldots,P_{0}\otimes Q_{n},\ldots,P_{m}\otimes Q_{0},\ldots,P_{m}\otimes Q_{n}}
\end{mathpar}

\begin{remark}
  Observe that arities of tensored abstractions matches arities of
  tensored concretions if the original arities matched. Note also that
  the length of the arities corresponds to the increase in dimension
  we see in ordinary vector space tensor product.
\end{remark}

\begin{remark}
  Operationally, this definition distributes the tensor down to
  components ``linked'' by summation. Tensor over summation is
  intriguing in that it mixes names. Moreover, as a consequence of the
  way it mixes names we have the identities for all $x \in \QProc$ and
  $P,Q \in \Proc$

  \begin{mathpar}
    (x \cdot P) \otimes Q \equiv x \cdot (P \otimes Q) \equiv P \otimes (x \cdot Q)
    \and
    P \otimes \pzero \equiv P
  \end{mathpar}

  that the reader is invited to verify.
\end{remark}

\subsubsection{Annihilation}
\begin{mathpar}
  P^{\perp} := \{ Q | \forall R. P|Q \red^{*} R \Rightarrow R \red^{*} \pzero \}
  \and \\
  P^{\underline{\perp}} := \Sigma_{Q \in P^{\perp}} \quotep{Q}?(y).(\dropn{y}|Q) | \Sigma_{Q \in P^{\perp}} \quotep{Q}\clift{\Box}
\end{mathpar}

\paragraph{Discussion} The reader will note that $P^{\perp}$ is a
\emph{set} of processes, while $P^{\underline{\perp}}$ is a
\emph{context}. We call the set $P^{\perp}$ the \emph{annihilators} of
$P$. The parallel composition of a process in the annihilators of $P$
with $P$ will result in a process, the state space of which has all
paths eventually leading to $\pzero$. Execution may endure loops; but
under reasonable conditions of fairness (naturally guaranteed under
most notions of bisimulation) such a composite process cannot get
stuck in such a loop and will, eventually pop out and terminate.

The context $P^{\underline{\perp}}$ is ready and willing to ``take the
$P$ out of'' the process to which it is applied. It will effectively
transmit the code of the process to which it is applied to one of the
annihilators and run the process against it.

\subsubsection{Evaluation}
We fix $M$ a domain of fully abstract interpretation with an equality
coincident with bisimulation. We take $\meaningof{\cdot} : \Proc \to
M$ to be the map interpreting processes and $\nmeaningof{\cdot} : \M
\to Proc$ to be the map running the other way. Then we define

\begin{mathpar}
  \int P := \nmeaningof{\meaningof{P}}
\end{mathpar}

\paragraph{Discussion}
There are many fully abstract interpretations of Milner's
$\pi$-calculus. Any of them can be used as a basis for interpreting
the reflective calculus here. Equipped with such a domain it is
largely a matter of grinding through to check that the Yoneda
construction for the normalization-by-evaluation program can be
extended to this setting.

\begin{remark}
  The reader is invited to verify that $\int (P^{\underline{\perp}}[P]) = 0$.
\end{remark}

\subsection{Quantum mechanics}

Table \ref{tbl:core_qm_op_defns} gives the core operational definitions

\begin{table}[htp]\label{tbl:core_qm_op_defns}
  \center{
    \fbox{
      \begin{tabular}{c|c}
        quantum mechanics & process calculus \\
        \hline
        scalar & $x := \quotep{P}$ \\
        state vector & $\state{P} := P$ \\
        dual & $\state{P}^{*} := \event{P^{\underline{\perp}}} := \quotep{P^{\underline{\perp}}}[-]$ \\
        matrix & $ \Sigma_{\alpha} \state{P_{\alpha}}x_{\alpha}\event{Q_{\alpha}}$ \\
        vector addition & $\state{P} + \state{Q} := \state{P | Q}$ \\
        tensor product & $\state{P} \otimes \state{Q} := \state{P \otimes Q}$ \\
        inner product & $\innerprod{P}{Q} := \quotep{\int P^{\underline{\perp}}[Q]}$ \\
      \end{tabular}
    }
  }
  \caption{QM - operational definitions}
\end{table}

where

\begin{mathpar}
  \prmatrix{P}{Q} := \fprmatrix{P}{\quotep{\pzero}}{Q}
  \and
  \fprmatrix{P}{x}{Q} := (\state{P},x,\event{Q})
  \and
  (\fprmatrix{P}{x}{Q})(\state{R}) := x \cdot \innerprod{Q}{R} \cdot \state{P}
  \and
  (\fprmatrix{P}{x}{Q})(\event{R}) := x \cdot \innerprod{R}{P} \cdot \event{Q}
\end{mathpar}

\paragraph{Discussion}
As promised: vectors (aka states) are represented as processes; duals
as contextual duals; inner product definition should be compared with
standard inner product definition for ....

\begin{remark}
  Assuming $\int (P^{\underline{\perp}}[P]) = 0$, the reader is
  invited to verify that $(\fprmatrix{P}{x}{P})(\state{P}) = x \cdot \state{P}$.
\end{remark}

\begin{remark}
  The reader is invited to verify that $\innerprod{P}{Q}$ could
  equally well have been written $\quotep{\int \stackrel{\vee}{x}}$
  where $x = \event{P^{\underline{\perp}}}(Q)$.

  One of the motivations for this remark is that there is another way
  to factor these operations. We could package up evaluation in the dual:

  \begin{mathpar}
    \state{P}^{*} := \event{\int P^{\underline{\perp}}} := \quotep{\int P^{\underline{\perp}}}[-]
  \end{mathpar}

  and then have inner product defined by
  
  \begin{mathpar}
    \innerprod{P}{Q} := \event{P}(Q)
  \end{mathpar}

  Hopefully, experience with the calculations will provide guidance on
  the best factoring.
\end{remark}

\begin{remark}
  Assuming $\int (P^{\underline{\perp}}[P]) = 0$, the reader is
  invited to verify that $\forall P,Q. (\prmatrix{0}{Q})(\state{0}) =
  \state{0}$ and dually $(\prmatrix{P}{0})(\event{0}) = \event{0}$.
\end{remark}

\begin{remark}
  i'm a little worried that i don't (yet) have proper support for
  complex conjugacy. But, the observation above may give us a
  clue. According to Abramsky, it must be the case that the scalars
  are iso to the homset of the identity for the tensor -- which the
  observation above characterizes. 

  For now, we will simply bookmark the notion with $\overline{x}$.
\end{remark}

\subsubsection{Adjointness}

We need to give a definition of $(\cdot)^{\dagger}$ for matrices. The
obvious candidate definition is
\begin{mathpar}
(\Sigma_{\alpha}\fprmatrix{P_{\alpha}}{x_{\alpha}}{Q_{\alpha}})^{\dagger}
= \Sigma_{\alpha}\fprmatrix{(Q_{\alpha}^{\underline{\perp}})^{*}}{\overline{x}_{\alpha}}{P_{\alpha}^{\underline{\perp}}} 
\end{mathpar}

But, $(Q_{\alpha}^{\underline{\perp}})^{*}$ requires a name along
which to communicate the process to achieve the context application.

\subsubsection{Basis for a basis}
If processes label states and ``addition'' of states (a.k.a. vector
addition) is interpreted as parallel composition, what corresponds to
notions of linear independence and basis? Here, we recall that Yoshida
has developed a set of \emph{combinators} for an asynchronous verison
of Milner's $\pi$-calculus. These are a finite set of processes such
any process can be expressed as parallel composition of these
combinators together with liberal uses of the new operator and
replication. We can simply give a translation of these into the
present calculus and have reasonable expectation that the property
carries over. That is, that the resultant set allows to express all
processes via parallel composition. Note, however, that there is no
new operator or replication in this calculus. As a result, we expect
that the corresponding set is actually infinite. That is, we expect
that the space is actually infinite dimensional.

\begin{remark}
  The attentive reader may be a bit concerned. Certainly, the
  collection $S$, $K$ and $I$ is a finite set of
  combinators. Shouldn't we expect to see a finite set of combinators
  for an effectively equivalent system? i am very sympathetic to this
  critique and feel it warrants full attention. On the other hand, i
  also have in mind the following analogy. The natural numbers, as a
  monoid under addition, has exactly $1$ generator, while the natural
  numbers, as a monoid under multiplication, has countably many
  generators (the primes). We observe that the application of the
  lambda calculus is much less resource sensitive than the parallel
  composition of the $\pi$-calculus. Could it be the case that we have
  an analogy of the form
  
  \begin{mathpar}
    m + n : MN :: m*n : M|N
  \end{mathpar}

  giving a similar blow up in the set of ``primes''?  This is such a
  wonderful thought that, even if it's not true, i think it's worth
  writing down.
\end{remark}
 

\documentclass[12pt]{llncs}
%\documentclass{jktr}

\usepackage[pdftex]{hyperref}                   
\usepackage {listings}
\usepackage {mathpartir}
\usepackage{bcprules}
%\usepackage{listings}
                       
\usepackage{graphicx} 
%\usepackage[margins=2.5cm,nohead,nofoot]{geometry}
%\usepackage{geometry}
\usepackage{amsfonts}
\usepackage{amstext}
\usepackage{latexsym}
\usepackage{amssymb}
\usepackage{color}


%\include{myPreamble}
\include{qm2pi.local} 

%\ifpdf
%\usepackage[pdftex]{graphicx}
%\else
%\usepackage{graphicx}
%\fi

 % \ifpdf
%  \usepackage{pdfsync}
%  \if


%\title{Brief Article}
%\author{David F. Snyder}
%\author{L.G. Meredith}

%\address{Dept. of Math., Texas State University--San Marcos, San Marcos, TX 78666}
       
\pagestyle{empty}


\begin{document}

\lstset{language=[Objective]Caml,frame=shadowbox}

\input{qm2pi.front}

% section front matter (end)

\input{qm2pi.intro} 
 
% section introduction (end)

% \input{qm2pi.knotations} 

% section notation (end)

\input{qm2pi.process.calculi} 

% section concurrent_process_calculi_and_spatial_logics_ (end)
    
%\input{qm2pi.knots2pi} 

%\input{qm2pi.trefoil} 

%\input{qm2pi.mainthm} 

% subsection basic_interpretation (end)

%\input{qm2pi.rho.presentation} 
\subsection{The syntax and semantics of the notation system}\label{sub:the_syntax_and_semantics_of_the_notation_system} % (fold)

We now summarize a technical presentation of the calculus that
embodies our theory of dynamics. The typical presentation of such a
calculus follows the style of giving generators and relations on
them. The grammar, below, describing term constructors, freely
generates the set of processes, $\Proc$. This set is then quotiented
by a relation known as structural congruence and it is over this set
that the notion of dynamics is expressed. This presentation is
essentially that of \cite{MeredithR05} with the addition of
polyadicity and summation. For readability we have relegated some of
the technical subtleties to an appendix.

\subsubsection{Process grammar}\label{subsub:process_grammar}

\begin{mathpar}
  \inferrule* [lab=synchronization] {} {{M} \bc \pzero \;|\; x?F \;|\; x!C }
  \and
  \inferrule* [lab=abstraction] {} {{F} \bc (x)P}
  \and
  \inferrule* [lab=concretion] {} {{C} \bc \langle Q \rangle}
  \and
  \inferrule* [lab=process] {} {{P,Q} \bc M \;| \;P|Q \;|\; @{x}}
  \and
  \inferrule* [lab=name] {} {{x} \bc \quotep{P}}
\end{mathpar} 

Note that $\vec{x}$ (resp. $\vec{P}$) denotes a vector of names
(resp. processes) of length $|\vec{x}|$ (resp. $|\vec{P}|$). We adopt
the following useful abbreviations.

\begin{mathpar}
   x?(\vec{y}).P := x.(\vec{y})P \and  x\clift{\vec{P}} := x.\clift{\vec{P}}
   \and x!(y) := \lift{x}{\dropn{y}}
   \and \Pi_{i=0}^{n-1}P_i := P_0 | \ldots | P_{n-1}
\end{mathpar}

\subsubsection{Structural congruence}

\paragraph{Free and bound names and alpha-equivalence.} At the
core of structural equivalence is alpha-equivalence which identifies
process that are the same up to a change of variable. Formally, we
recognize the distinction between free and bound names. The free names
of a process, $\freenames{P}$, may be calculated recursively as
follows:

\begin{mathpar}
\freenames{\pzero} := \emptyset
  \and \\
  \freenames{x?(y).P} := \{ x \} \cup (\freenames{P} \setminus \{ y \})
  \and 
  \freenames{x!\langle P \rangle} := \{ x \} \cup \{ P \} 
  \and \\
  \freenames{P|Q} := \freenames{P} \cup \freenames{Q}
  \and \\
  \freenames{@{x}} := \{ x \}
\end{mathpar}

$\pi$
$\quotep{\pi}$

$\freenames{-} : \pi \to \mathcal{P}(\quotep{\pi})$

\begin{eqnarray*}
  \freenames{\pzero} & := & \emptyset \\
  \freenames{x?(y).P} & := & \{ x \} \cup (\freenames{P} \setminus \{ y \}) \\
  \freenames{x!\langle P \rangle} & := & \{ x \} \cup \{ P \} \\
  \freenames{P|Q} & := & \freenames{P} \cup \freenames{Q} \\
  \freenames{\dropn{x}} & := & \{ x \}
\end{eqnarray*}

The bound names of a process, $\boundnames{P}$, are those names occurring in $P$
that are not free. For example, in $x?(y).0$, the name $x$ is free, while $y$ is bound.

\begin{mathpar}
  \inferrule* [lab=monoidal-laws] {} { P|Q \equiv Q|P \and P|0 \equiv P \and P|(Q|R) \equiv (P|Q)|R }
\end{mathpar}

\begin{mathpar}
  \inferrule* [lab=alpha-equivalence] {} { (x)P \equiv (y)P\{y/x\} \and y \not\in \freenames{P} }
\end{mathpar}

\begin{definition}
Then two processes, $P,Q$, are alpha-equivalent if $P = Q\{\vec{y}/\vec{x}\}$ for
some $\vec{x} \in \boundnames{Q},\vec{y} \in \boundnames{P}$, where $Q\{\vec{y}/\vec{x}\}$
denotes the capture-avoiding substitution of $\vec{y}$ for $\vec{x}$ in $Q$.
\end{definition}

\begin{definition}
  The {\em structural congruence} \cite{SangiorgiWalker} , $\equiv$,
  between processes is the least congruence containing
  alpha-equivalence, satisfying the abelian monoid laws
  (associativity, commutativity and $\pzero$ as identity) for parallel
  composition $|$ and for summation $+$.
\end{definition}

\subsection{Name equivalence}

We take name equivalence, written $\nameeq$, to be the smallest
equivalence relation generated by the following rules.

\begin{mathpar}
\inferrule*[lab=Quote-drop]
{ }
{ \quotep{@{x}} \nameeq x }

\inferrule*[lab=Struct-equiv]
{ P \scong Q }
{ \quotep{P} \nameeq \quotep{Q} }
\end{mathpar}

The astute reader will have noticed that the mutual recursion of names
and processes imposes a mutual recursion on alpha-equivalence and
structural equivalence via name-equivalence. Fortunately, all of this
works out pleasantly and we may calculate in the natural way, free of
concern. The reader interested in the details is referred to the
appendix \ref{appendix:rho_details}.

\subsection{Substitution}

We use $\Proc$ for the set of processes, $\QProc$ for the set of
names, and $\id{\{}\vec{y} / \vec{x} \id{\}}$ to denote partial maps,
$s : \QProc \rightarrow \QProc$. A map, $s$ lifts, uniquely, to a map
on process terms, $\widehat{s} : \Proc \rightarrow \Proc$ by the
following equations.

\begin{mathpar}
  (0) \psubstp{Q}{P} := 0 \\
  (R \juxtap S) \psubstp{Q}{P}
  :=    
  (R)\psubstp{Q}{P} \juxtap (S) \psubstp{Q}{P} \\
  (x?(y).R) \psubstp{Q}{P}    
  :=    
  (x)\substp{Q}{P} (z)\concat( (R \psubstn{z}{y}) \psubstp{Q}{P} ) \\
  (\lift{x}{R}) \psubstp{Q}{P}  
  :=
  \lift{(x)\substp{Q}{P}}{ R \psubstp{Q}{P} } \\
%   (\dropn{x})  \psubstp{Q}{P}       
%   := 
%   \left\{ 
%     \begin{array}{ccc} 
%       \dropn{\quotep{Q}} & & x \nameeq \quotep{P} \\
%       \dropn{x} & & otherwise \\
%     \end{array}
%   \right. 
  (\dropn{x})  \psubstp{Q}{P}       
  := 
  \left\{ 
    \begin{array}{ccc} 
      Q & & x \nameeq \quotep{P} \\
      \dropn{x} & & otherwise \\
    \end{array}
  \right.
\end{mathpar}
 

where

\begin{eqnarray}
  (x)\id{\{} \lpquote Q \rpquote / \lpquote P \rpquote \id{\}}            = 
  \left\{ 
    \begin{array}{ccc}
      \lpquote Q \rpquote & & x \nameeq \lpquote P \rpquote \\
      x & & otherwise \\
    \end{array}
  \right. \nonumber
\end{eqnarray}

and $z$ is chosen distinct from $\quotep{P}$, $\quotep{Q}$, the free
names in $Q$, and all the names in $R$. Our $\alpha$-equivalence will
be built in the standard way from this substitution.

\begin{remark}\label{rem:no_self_referential_names}
  One consequence of these definitions is that $\forall P. \quotep{P}
  \not\in \freenames{P}$.
\end{remark}

\subsection{ Dynamic quote: an example }

Anticipating something of what's to come, consider applying the
substitution, $\widehat{\id{\{}u / z \id{\}}}$, to the following pair
of processes, $\lift{w}{y!(z)}$ and $w[ \lpquote y!(z) \rpquote ]$.

\begin{eqnarray}
	\lift{w}{y!(z)}\widehat{\id{\{}u / z \id{\}}}
		& = &
		\lift{w}{y!(u)} \nonumber\\
	w[ \lpquote y!(z) \rpquote ] \widehat{ \id{\{}u / z \id{\}} }
		& = &
		w[ \lpquote y!(z) \rpquote ] \nonumber
\end{eqnarray}

Because the body of the process between quotes is impervious to
substitution, we get radically different answers. In fact, by
examining the first process in an input context,
e.g. $x?(z).\lift{w}{y!(z)}$, we see that the process under the lift
operator may be shaped by prefixed inputs binding a name inside it. In
this sense, the lift operator will be seen as a way to dynamically
construct processes before reifying them as names.

Finally equipped with these standard features we can present the
dynamics of the calculus.

\subsubsection{Operational semantics} 

Finally, we introduce the computational dynamics. What marks these
algebras as distinct from other more traditionally studied algebraic
structures, e.g. vector spaces or polynomial rings, is the manner in
which dynamics is captured. In traditional structures, dynamics is typically
expressed through morphisms between such structures, as in linear maps
between vector spaces or morphisms between rings. In algebras
associated with the semantics of computation, the dynamics is
expressed as part of the algebraic structure itself, through a
reduction reduction relation typically denoted by $\red$. Below, we
give a recursive presentation of this relation for the calculus used
in the encoding.

$\red \subseteq \pi \times \pi$
$\red : \pi \to \mathcal{P}(\pi)$

\begin{mathpar}
  \inferrule* [lab=Comm] { \textsf{match}( x_{src}, x_{trgt} ) } { x_{trgt}?(y)P \; | \; x_{src}!\langle {Q} \rangle \red P\{\quotep{Q}/y}\} }
  \and \\
  \inferrule* [lab=Par] {{P} \red {P}'} {{{P} | {Q}} \red {{P}' | {Q}}}
  \and
  \inferrule* [lab=Equiv]{{{P} \scong {P}'} \andalso {{P}' \red {Q}'} \andalso {{Q}' \scong {Q}}}{{P} \red {Q}}
\end{mathpar}

\begin{eqnarray*}
  match_{\equiv} (\quotep{P},\quotep{Q}) & := & P \equiv Q \\
  match_{\dagger}(\quotep{P},\quotep{Q}) & := & \forall R. P|Q \red^{*} R => R \red^{*} 0 \\
  match_{K}(\quotep{P},\quotep{Q}) & := & K \mbox{ for some context } K
\end{eqnarray*}

$u?(x)P | u!\langle Q \rangle \red P\{\quotep{Q}/x\}$

%We write $\wred$ for $\red^*$, and $P\red$ if $\exists Q $ such that $ P \red Q$.
We write $P\red$ if $\exists Q $ such that $ P \red Q$ and $P\not\red$, otherwise.

\section{Replication}

As mentioned before, it is known that replication (and hence
recursion) can be implemented in a higher-order process algebra
\cite{SangiorgiWalker}. As our first example of calculation with the
machinery thus far presented we give the construction explicitly in
the {\rhoc}.

\begin{eqnarray}
	D_{x} & := & \prefix{x}{y}{(\binpar{\outputp{x}{y}}{@{y}})} \nonumber\\
	\bangp_{x}{P} & := & \binpar{{x}!\langle{\binpar{D_{x}}{P}}\rangle}{D_{x}} \nonumber
\end{eqnarray}

\begin{eqnarray}
	\bangp_{x}{P} & & \nonumber\\
	=
	& {x}!\langle{(\prefix{x}{y}{(\outputp{x}{y} | @{y})) | P}}\rangle 
	      | \prefix{x}{y}{(\outputp{x}{y} | @{y})} & \nonumber\\
	\red
	& (\outputp{x}{y} | @{y})\substn{\quotep{(\prefix{x}{y}{(@{y} | \outputp{x}{y})) | P}}}{y} & \nonumber\\
	=
	& \outputp{x}{\quotep{(\prefix{x}{y}{(\outputp{x}{y} | @{y})) | P}}}
	  | {(\prefix{x}{y}{(\outputp{x}{y} | @{y})) | P}} & \nonumber\\
	\red
	& \ldots & \nonumber\\
	\red^*
	& P | P | \ldots & \nonumber
\end{eqnarray}

Of course, this encoding, as an implementation, runs away, unfolding
$\bangp{P}$ eagerly. A lazier and more implementable replication
operator, restricted to input-guarded processes, may be obtained as follows.

\begin{eqnarray}
\bangp{\prefix{u}{v}{P}} 
	:= 
	\binpar{\lift{x}{\prefix{u}{v}{(\binpar{D(x)}{P})}}}{D(x)} \nonumber
\end{eqnarray}

\begin{remark}
  Note that the lazier definition still does not deal with summation
  or mixed summation (i.e. sums over input and output). The reader is
  invited to construct definitions of replication that deal with these
  features. 

  Further, the definitions are parameterized in a name, $x$. Can you,
  gentle reader, make a definition that eliminates this parameter and
  guarantees no accidental interaction between the replication
  machinery and the process being replicated -- i.e. no accidental
  sharing of names used by the process to get its work done and the
  name(s) used by the replication to effect copying. This latter
  revision of the definition of replication is crucial to obtaining
  the expected identity $!!P \sim !P$.
\end{remark}

\begin{remark}\label{rem:paradoxical_combinator}
  The reader familiar with the lambda calculus will have noticed the
  similarity between $D$ and the paradoxical combinator.

  [Ed. note: the existence of this seems to suggest we have to be more
  restrictive on the set of processes and names we admit if we are to
  support no-cloning.]
\end{remark}

\subsubsection{Bisimulation}

The computational dynamics gives rise to another kind of equivalence,
the equivalence of computational behavior. As previously mentioned
this is typically captured \emph{via} some form of bisimulation.

% The notion we use in this paper is weak barbed bisimulation
% \cite{milner91polyadicpi}.

The notion we use in this paper is derived from weak barbed
bisimulation \cite{milner91polyadicpi}. 

\begin{definition}
An \emph{observation relation}, $\downarrow_{\mathcal N}$, over a set
of names, $\mathcal N$, is the smallest relation satisfying the rules
below.

\infrule[Out-barb]{y \in {\mathcal N}, \; x \nameeq y}
		  {\outputp{x}{v} \downarrow_{\mathcal N} x}
\infrule[Par-barb]{\mbox{$P\downarrow_{\mathcal N} x$ or $Q\downarrow_{\mathcal N} x$}}
		  {\binpar{P}{Q} \downarrow_{\mathcal N} x}

We write $P \Downarrow_{\mathcal N} x$ if there is $Q$ such that 
$P \wred Q$ and $Q \downarrow_{\mathcal N} x$.
\end{definition}

\begin{definition}
%\label{def.bbisim}
An  ${\mathcal N}$-\emph{barbed bisimulation} over a set of names, ${\mathcal N}$, is a symmetric binary relation 
${\mathcal S}_{\mathcal N}$ between agents such that $P\rel{S}_{\mathcal N}Q$ implies:
\begin{enumerate}
\item If $P \red P'$ then $Q \wred Q'$ and $P'\rel{S}_{\mathcal N} Q'$.
\item If $P\downarrow_{\mathcal N} x$, then $Q\Downarrow_{\mathcal N} x$.
\end{enumerate}
$P$ is ${\mathcal N}$-barbed bisimilar to $Q$, written
$P \wbbisim_{\mathcal N} Q$, if $P \rel{S}_{\mathcal N} Q$ for some ${\mathcal N}$-barbed bisimulation ${\mathcal S}_{\mathcal N}$.
\end{definition}

$\mathcal{R} \subseteq \pi \times \pi$

$P \mathcal{R} Q => \forall P'. P \red P' \Rightarrow \exists Q'. Q \red Q', P' \mathcal{R} Q'$

$P \vdash x \Rightarrow Q \vdash x$

\begin{mathpar}
  \inferrule*[lab=Out-barb]{x \nameeq y}{{y}!\langle{Q}\rangle \vdash x}
  \and
  \inferrule*[lab=Par-barb]{\mbox{$P\vdash x$ or $Q\vdash x$}}{\binpar{P}{Q} \vdash x}
\end{mathpar}

\subsubsection{Contexts}

One of the principle advantages of computational calculi like the
$\pi$-calculus is a well-defined notion of context,
contextual-equivalence and a correlation between
contextual-equivalence and notions of bisimulation. The notion of
context allows the decomposition of a process into (sub-)process and
its syntactic environment, its context. Thus, a context may be
thought of as a process with a ``hole'' (written $\Box$) in it. The
application of a context $M$ to a process $P$, written $M[P]$, is
tantamount to filling the hole in $M$ with $P$. In this paper we do
not need the full weight of this theory, but do make use of the notion
of context in the proof the main theorem. 

\begin{mathpar}
  \inferrule* [lab=summation] {} {{M_{M},M_{N}} \bc \Box \;|\; x.M_{A} \;|\; M_{M}+M_{N}}
  \and
  \inferrule* [lab=agent] {} {{M_{A}} \bc (\vec{x})M_{P} \;| \; \clift{P_0,\ldots,M_{P},\ldots,P_N}}
  \and \\
  \inferrule* [lab=process] {} {{M_{P}} \bc M_{N} \;| \;P|M_{P} }
\end{mathpar} 

\begin{mathpar}
  \inferrule* [lab=sychronization] {} {M_{N} \bc \Box \;|\; x?M_{F} \;|\; x!M_{C}}
  \and
  \inferrule* [lab=abstraction] {} {{M_{F}} \bc (x)M_{P} }
  \and
  \inferrule* [lab=concretion] {} {{M_{C}} \bc \langle M_{P} \rangle }
  \and \\
  \inferrule* [lab=process] {} {{M_{P}} \bc M_{N} \;| \;P|M_{P} }
\end{mathpar}

\begin{definition}[contextual application] Given a context $M$, and
  process $P$, we define the \emph{contextual application}, $M[P] :=
  M\{P/\Box\}$. That is, the contextual application of M to P is the
  substitution of $P$ for $\Box$ in $M$.
\end{definition}

$\meaningof{-} : L \to \mathcal{P}(\pi)$

\begin{mathpar}
  \inferrule* [lab=collection] {} {\meaningof{true} = \pi, \and \meaningof{~E} = \pi \setminus \meaningof{E}, \and \meaningof{E_{1} \& E_{2}} = \meaningof{E_{1}} \cap \meaningof{E_{2}}}
\end{mathpar}

\begin{mathpar}
  \inferrule* [lab=structure] {} {\meaningof{0} = \{ P \in \pi | P \equiv 0 \}, \and \\ \meaningof{E_1 | E_2} = \{ P \in \pi | P \equiv P_{1} | P_{2}, P_{1} \in \meaningof{E_{1}}, P_{2} \in \meaningof{E_2}\} }
\end{mathpar}

\begin{mathpar}
 \inferrule* [lab=behavior] {} {\meaningof{\langle a?b \rangle E} = \{ P \in \pi | P \equiv Q | u?(y)P', \\ \and \\\\ \and \\ \;\;\; u \in \meaningof{a}, \forall z.P'\{z/y\} \in \meaningof{E\{z/b\}}\}, \and \\ \meaningof{a!E} = \{ P \in \pi | P \equiv Q | x!\langle P' \rangle, x \in \meaningof{a} P' \in \meaningof{E}\} }
\end{mathpar}

\begin{mathpar}
 \inferrule* [lab=nominal] {} {\meaningof{\quotep{E}} = \{ \quotep{P} \in \quotep{\pi} | P \in \meaningof{E} \}, \and \meaningof{\quotep{P}} = \{ \quotep{Q} \in \quotep{\pi} | P \equiv Q \} \and \\ \meaningof{@\quotep{E}} = \{ P \in \pi | P \equiv @x, x \in \meaningof{E} \}}
\end{mathpar}

\begin{eqnarray*}
  \\
  \meaningof{-} : TS \to ST
\end{eqnarray*}

\begin{eqnarray*}
  \\
  L : TS \to ST
\end{eqnarray*}

\begin{eqnarray*}
  \\
  P \models E \iff P \in \meaningof{E}
\end{eqnarray*}

\begin{eqnarray*}
  P \approx_{L} Q \iff \forall E \in L. P \models E \iff Q \models E
\end{eqnarray*}

\begin{eqnarray*}
  P \approx_{K} Q
\end{eqnarray*}

\begin{eqnarray*}
  P \approx Q
\end{eqnarray*}

$\approx_{K} = \approx = \approx_{L}$

\subsubsection{Contextual duality}

Note that contexts extend the quotation operation to a family of
operations from processes to names. Given a context, $M$, we can
define a \emph{nominal context}, $\quotep{M}$ by $\quotep{M}[P] :=
\quotep{M[P]}$. To foreshadow what is to come we observe that these
operations enjoy a duality with processes very much like the duality
between vectors and maps from vectors to scalars.

Further, because the calculus is essentially higher-order, we have a
correspondence between contexts and processes. More specifically,
given a name $x$ and a context $M$ we can construct $M^{*}_{x}$ such
that 

\begin{mathpar}
  M^{*}_{x} | \lift{x}{P} \red M[P]
\end{mathpar}

namely,

\begin{mathpar}
  M^{*}_{x} := x?(u).M[\dropn{u}]
\end{mathpar}

The dependence of $M^{*}_{x}$ on a name makes it an abstraction, 

\begin{mathpar}
  M^{*} := (x)x?(u).M[\dropn{u}]
\end{mathpar}

\subsection{Additional notation}

It will sometimes be convenient to denote the process a name
quotes. We already have the notation $x = \quotep{P}$, but it will be
convenient to introduce an alternate notation, $\procn{x}$, when we
want to emphasize the connection to the use of the name. Note that, by
virtue of name equivalence, $\quotep{\procn{x}} \nameeq x$; so, the
notation is consistent with previous definitions.

Further, because names have structure it is possible to effect
substitutions on the basis of that structure. This means we need to
upgrade our notation for substitutions, which we accomplish by
adapting comprehension notation. Thus,

\begin{mathpar}
  P\{ y / x : x \in S \}
\end{mathpar}

is interpreted to mean the process derived from P by replacing (in a
capture-avoiding manner) each occurrence of $x$ in $S$ by $y$. For example,

\begin{mathpar}
  P\{ \quotep{\procn{x}|\procn{x}} / x : x \in \freenames{P} \}
\end{mathpar}

will replace each (occurrence) of a free name $x$ in $P$ by
$\quotep{\procn{x}|\procn{x}}$.

Also, we will avail ourselves of the notation $x^{L}$ and $x^{R}$ to
denote injections of a name into disjoint copies of the name
space. There are numerous ways to accomplish this. One example can be
found in \cite{MeredithR05}. This notation overloads to vectors of
names: $\vec{x}^{\pi} := (x_{i}^{\pi} \; : \; 0 \leq i < |\vec{x}| )$ where $\pi \in \{L,R\}$.

We also use $P^{\Box} := P|\Box$.

In \cite{MeredithR05} an interpretation of the new operator is
given. It turns out that there are several possible interpretations
all enjoying the requisite algebraic properties of the operator (see
\cite{milner91polyadicpi}). We will therefore make liberal use of
$(\nu\; \vec{x})P$.

% subsection the_syntax_and_semantics_of_the_notation_system (end)   

\input{qm2pi.qmops} 

\input{qm2pi.sterngerlach} 

\input{qm2pi.metric} 

% section concurrent_process_calculi (end)

%\input{qm2pi.proofsketch}

% section proof sketch (end)

%\input{qm2pi.slviaknots} 

% section spatial logic via knots (end)

\input{qm2pi.conclusion}

% section conclusion (end)

%\input{qm2pi.dtcodes} 

% section wiring algorithm (end)

\input{qm2pi.ack} 

% section acknowledgments (end)

\newpage


\bibliographystyle{plain}   
\bibliography{../../biblios/main.bib}

\input{qm2pi.rhodetails}

\end{document}

 

\documentclass[12pt]{llncs}
%\documentclass{jktr}

\usepackage[pdftex]{hyperref}                   
\usepackage {listings}
\usepackage {mathpartir}
\usepackage{bcprules}
%\usepackage{listings}
                       
\usepackage{graphicx} 
%\usepackage[margins=2.5cm,nohead,nofoot]{geometry}
%\usepackage{geometry}
\usepackage{amsfonts}
\usepackage{amstext}
\usepackage{latexsym}
\usepackage{amssymb}
\usepackage{color}


%\include{myPreamble}
\include{qm2pi.local} 

%\ifpdf
%\usepackage[pdftex]{graphicx}
%\else
%\usepackage{graphicx}
%\fi

 % \ifpdf
%  \usepackage{pdfsync}
%  \if


%\title{Brief Article}
%\author{David F. Snyder}
%\author{L.G. Meredith}

%\address{Dept. of Math., Texas State University--San Marcos, San Marcos, TX 78666}
       
\pagestyle{empty}


\begin{document}

\lstset{language=[Objective]Caml,frame=shadowbox}

\input{qm2pi.front}

% section front matter (end)

\input{qm2pi.intro} 
 
% section introduction (end)

% \input{qm2pi.knotations} 

% section notation (end)

\input{qm2pi.process.calculi} 

% section concurrent_process_calculi_and_spatial_logics_ (end)
    
%\input{qm2pi.knots2pi} 

%\input{qm2pi.trefoil} 

%\input{qm2pi.mainthm} 

% subsection basic_interpretation (end)

%\input{qm2pi.rho.presentation} 
\subsection{The syntax and semantics of the notation system}\label{sub:the_syntax_and_semantics_of_the_notation_system} % (fold)

We now summarize a technical presentation of the calculus that
embodies our theory of dynamics. The typical presentation of such a
calculus follows the style of giving generators and relations on
them. The grammar, below, describing term constructors, freely
generates the set of processes, $\Proc$. This set is then quotiented
by a relation known as structural congruence and it is over this set
that the notion of dynamics is expressed. This presentation is
essentially that of \cite{MeredithR05} with the addition of
polyadicity and summation. For readability we have relegated some of
the technical subtleties to an appendix.

\subsubsection{Process grammar}\label{subsub:process_grammar}

\begin{mathpar}
  \inferrule* [lab=synchronization] {} {{M} \bc \pzero \;|\; x?F \;|\; x!C }
  \and
  \inferrule* [lab=abstraction] {} {{F} \bc (x)P}
  \and
  \inferrule* [lab=concretion] {} {{C} \bc \langle Q \rangle}
  \and
  \inferrule* [lab=process] {} {{P,Q} \bc M \;| \;P|Q \;|\; @{x}}
  \and
  \inferrule* [lab=name] {} {{x} \bc \quotep{P}}
\end{mathpar} 

Note that $\vec{x}$ (resp. $\vec{P}$) denotes a vector of names
(resp. processes) of length $|\vec{x}|$ (resp. $|\vec{P}|$). We adopt
the following useful abbreviations.

\begin{mathpar}
   x?(\vec{y}).P := x.(\vec{y})P \and  x\clift{\vec{P}} := x.\clift{\vec{P}}
   \and x!(y) := \lift{x}{\dropn{y}}
   \and \Pi_{i=0}^{n-1}P_i := P_0 | \ldots | P_{n-1}
\end{mathpar}

\subsubsection{Structural congruence}

\paragraph{Free and bound names and alpha-equivalence.} At the
core of structural equivalence is alpha-equivalence which identifies
process that are the same up to a change of variable. Formally, we
recognize the distinction between free and bound names. The free names
of a process, $\freenames{P}$, may be calculated recursively as
follows:

\begin{mathpar}
\freenames{\pzero} := \emptyset
  \and \\
  \freenames{x?(y).P} := \{ x \} \cup (\freenames{P} \setminus \{ y \})
  \and 
  \freenames{x!\langle P \rangle} := \{ x \} \cup \{ P \} 
  \and \\
  \freenames{P|Q} := \freenames{P} \cup \freenames{Q}
  \and \\
  \freenames{@{x}} := \{ x \}
\end{mathpar}

$\pi$
$\quotep{\pi}$

$\freenames{-} : \pi \to \mathcal{P}(\quotep{\pi})$

\begin{eqnarray*}
  \freenames{\pzero} & := & \emptyset \\
  \freenames{x?(y).P} & := & \{ x \} \cup (\freenames{P} \setminus \{ y \}) \\
  \freenames{x!\langle P \rangle} & := & \{ x \} \cup \{ P \} \\
  \freenames{P|Q} & := & \freenames{P} \cup \freenames{Q} \\
  \freenames{\dropn{x}} & := & \{ x \}
\end{eqnarray*}

The bound names of a process, $\boundnames{P}$, are those names occurring in $P$
that are not free. For example, in $x?(y).0$, the name $x$ is free, while $y$ is bound.

\begin{mathpar}
  \inferrule* [lab=monoidal-laws] {} { P|Q \equiv Q|P \and P|0 \equiv P \and P|(Q|R) \equiv (P|Q)|R }
\end{mathpar}

\begin{mathpar}
  \inferrule* [lab=alpha-equivalence] {} { (x)P \equiv (y)P\{y/x\} \and y \not\in \freenames{P} }
\end{mathpar}

\begin{definition}
Then two processes, $P,Q$, are alpha-equivalent if $P = Q\{\vec{y}/\vec{x}\}$ for
some $\vec{x} \in \boundnames{Q},\vec{y} \in \boundnames{P}$, where $Q\{\vec{y}/\vec{x}\}$
denotes the capture-avoiding substitution of $\vec{y}$ for $\vec{x}$ in $Q$.
\end{definition}

\begin{definition}
  The {\em structural congruence} \cite{SangiorgiWalker} , $\equiv$,
  between processes is the least congruence containing
  alpha-equivalence, satisfying the abelian monoid laws
  (associativity, commutativity and $\pzero$ as identity) for parallel
  composition $|$ and for summation $+$.
\end{definition}

\subsection{Name equivalence}

We take name equivalence, written $\nameeq$, to be the smallest
equivalence relation generated by the following rules.

\begin{mathpar}
\inferrule*[lab=Quote-drop]
{ }
{ \quotep{@{x}} \nameeq x }

\inferrule*[lab=Struct-equiv]
{ P \scong Q }
{ \quotep{P} \nameeq \quotep{Q} }
\end{mathpar}

The astute reader will have noticed that the mutual recursion of names
and processes imposes a mutual recursion on alpha-equivalence and
structural equivalence via name-equivalence. Fortunately, all of this
works out pleasantly and we may calculate in the natural way, free of
concern. The reader interested in the details is referred to the
appendix \ref{appendix:rho_details}.

\subsection{Substitution}

We use $\Proc$ for the set of processes, $\QProc$ for the set of
names, and $\id{\{}\vec{y} / \vec{x} \id{\}}$ to denote partial maps,
$s : \QProc \rightarrow \QProc$. A map, $s$ lifts, uniquely, to a map
on process terms, $\widehat{s} : \Proc \rightarrow \Proc$ by the
following equations.

\begin{mathpar}
  (0) \psubstp{Q}{P} := 0 \\
  (R \juxtap S) \psubstp{Q}{P}
  :=    
  (R)\psubstp{Q}{P} \juxtap (S) \psubstp{Q}{P} \\
  (x?(y).R) \psubstp{Q}{P}    
  :=    
  (x)\substp{Q}{P} (z)\concat( (R \psubstn{z}{y}) \psubstp{Q}{P} ) \\
  (\lift{x}{R}) \psubstp{Q}{P}  
  :=
  \lift{(x)\substp{Q}{P}}{ R \psubstp{Q}{P} } \\
%   (\dropn{x})  \psubstp{Q}{P}       
%   := 
%   \left\{ 
%     \begin{array}{ccc} 
%       \dropn{\quotep{Q}} & & x \nameeq \quotep{P} \\
%       \dropn{x} & & otherwise \\
%     \end{array}
%   \right. 
  (\dropn{x})  \psubstp{Q}{P}       
  := 
  \left\{ 
    \begin{array}{ccc} 
      Q & & x \nameeq \quotep{P} \\
      \dropn{x} & & otherwise \\
    \end{array}
  \right.
\end{mathpar}
 

where

\begin{eqnarray}
  (x)\id{\{} \lpquote Q \rpquote / \lpquote P \rpquote \id{\}}            = 
  \left\{ 
    \begin{array}{ccc}
      \lpquote Q \rpquote & & x \nameeq \lpquote P \rpquote \\
      x & & otherwise \\
    \end{array}
  \right. \nonumber
\end{eqnarray}

and $z$ is chosen distinct from $\quotep{P}$, $\quotep{Q}$, the free
names in $Q$, and all the names in $R$. Our $\alpha$-equivalence will
be built in the standard way from this substitution.

\begin{remark}\label{rem:no_self_referential_names}
  One consequence of these definitions is that $\forall P. \quotep{P}
  \not\in \freenames{P}$.
\end{remark}

\subsection{ Dynamic quote: an example }

Anticipating something of what's to come, consider applying the
substitution, $\widehat{\id{\{}u / z \id{\}}}$, to the following pair
of processes, $\lift{w}{y!(z)}$ and $w[ \lpquote y!(z) \rpquote ]$.

\begin{eqnarray}
	\lift{w}{y!(z)}\widehat{\id{\{}u / z \id{\}}}
		& = &
		\lift{w}{y!(u)} \nonumber\\
	w[ \lpquote y!(z) \rpquote ] \widehat{ \id{\{}u / z \id{\}} }
		& = &
		w[ \lpquote y!(z) \rpquote ] \nonumber
\end{eqnarray}

Because the body of the process between quotes is impervious to
substitution, we get radically different answers. In fact, by
examining the first process in an input context,
e.g. $x?(z).\lift{w}{y!(z)}$, we see that the process under the lift
operator may be shaped by prefixed inputs binding a name inside it. In
this sense, the lift operator will be seen as a way to dynamically
construct processes before reifying them as names.

Finally equipped with these standard features we can present the
dynamics of the calculus.

\subsubsection{Operational semantics} 

Finally, we introduce the computational dynamics. What marks these
algebras as distinct from other more traditionally studied algebraic
structures, e.g. vector spaces or polynomial rings, is the manner in
which dynamics is captured. In traditional structures, dynamics is typically
expressed through morphisms between such structures, as in linear maps
between vector spaces or morphisms between rings. In algebras
associated with the semantics of computation, the dynamics is
expressed as part of the algebraic structure itself, through a
reduction reduction relation typically denoted by $\red$. Below, we
give a recursive presentation of this relation for the calculus used
in the encoding.

$\red \subseteq \pi \times \pi$
$\red : \pi \to \mathcal{P}(\pi)$

\begin{mathpar}
  \inferrule* [lab=Comm] { \textsf{match}( x_{src}, x_{trgt} ) } { x_{trgt}?(y)P \; | \; x_{src}!\langle {Q} \rangle \red P\{\quotep{Q}/y}\} }
  \and \\
  \inferrule* [lab=Par] {{P} \red {P}'} {{{P} | {Q}} \red {{P}' | {Q}}}
  \and
  \inferrule* [lab=Equiv]{{{P} \scong {P}'} \andalso {{P}' \red {Q}'} \andalso {{Q}' \scong {Q}}}{{P} \red {Q}}
\end{mathpar}

\begin{eqnarray*}
  match_{\equiv} (\quotep{P},\quotep{Q}) & := & P \equiv Q \\
  match_{\dagger}(\quotep{P},\quotep{Q}) & := & \forall R. P|Q \red^{*} R => R \red^{*} 0 \\
  match_{K}(\quotep{P},\quotep{Q}) & := & K \mbox{ for some context } K
\end{eqnarray*}

$u?(x)P | u!\langle Q \rangle \red P\{\quotep{Q}/x\}$

%We write $\wred$ for $\red^*$, and $P\red$ if $\exists Q $ such that $ P \red Q$.
We write $P\red$ if $\exists Q $ such that $ P \red Q$ and $P\not\red$, otherwise.

\section{Replication}

As mentioned before, it is known that replication (and hence
recursion) can be implemented in a higher-order process algebra
\cite{SangiorgiWalker}. As our first example of calculation with the
machinery thus far presented we give the construction explicitly in
the {\rhoc}.

\begin{eqnarray}
	D_{x} & := & \prefix{x}{y}{(\binpar{\outputp{x}{y}}{@{y}})} \nonumber\\
	\bangp_{x}{P} & := & \binpar{{x}!\langle{\binpar{D_{x}}{P}}\rangle}{D_{x}} \nonumber
\end{eqnarray}

\begin{eqnarray}
	\bangp_{x}{P} & & \nonumber\\
	=
	& {x}!\langle{(\prefix{x}{y}{(\outputp{x}{y} | @{y})) | P}}\rangle 
	      | \prefix{x}{y}{(\outputp{x}{y} | @{y})} & \nonumber\\
	\red
	& (\outputp{x}{y} | @{y})\substn{\quotep{(\prefix{x}{y}{(@{y} | \outputp{x}{y})) | P}}}{y} & \nonumber\\
	=
	& \outputp{x}{\quotep{(\prefix{x}{y}{(\outputp{x}{y} | @{y})) | P}}}
	  | {(\prefix{x}{y}{(\outputp{x}{y} | @{y})) | P}} & \nonumber\\
	\red
	& \ldots & \nonumber\\
	\red^*
	& P | P | \ldots & \nonumber
\end{eqnarray}

Of course, this encoding, as an implementation, runs away, unfolding
$\bangp{P}$ eagerly. A lazier and more implementable replication
operator, restricted to input-guarded processes, may be obtained as follows.

\begin{eqnarray}
\bangp{\prefix{u}{v}{P}} 
	:= 
	\binpar{\lift{x}{\prefix{u}{v}{(\binpar{D(x)}{P})}}}{D(x)} \nonumber
\end{eqnarray}

\begin{remark}
  Note that the lazier definition still does not deal with summation
  or mixed summation (i.e. sums over input and output). The reader is
  invited to construct definitions of replication that deal with these
  features. 

  Further, the definitions are parameterized in a name, $x$. Can you,
  gentle reader, make a definition that eliminates this parameter and
  guarantees no accidental interaction between the replication
  machinery and the process being replicated -- i.e. no accidental
  sharing of names used by the process to get its work done and the
  name(s) used by the replication to effect copying. This latter
  revision of the definition of replication is crucial to obtaining
  the expected identity $!!P \sim !P$.
\end{remark}

\begin{remark}\label{rem:paradoxical_combinator}
  The reader familiar with the lambda calculus will have noticed the
  similarity between $D$ and the paradoxical combinator.

  [Ed. note: the existence of this seems to suggest we have to be more
  restrictive on the set of processes and names we admit if we are to
  support no-cloning.]
\end{remark}

\subsubsection{Bisimulation}

The computational dynamics gives rise to another kind of equivalence,
the equivalence of computational behavior. As previously mentioned
this is typically captured \emph{via} some form of bisimulation.

% The notion we use in this paper is weak barbed bisimulation
% \cite{milner91polyadicpi}.

The notion we use in this paper is derived from weak barbed
bisimulation \cite{milner91polyadicpi}. 

\begin{definition}
An \emph{observation relation}, $\downarrow_{\mathcal N}$, over a set
of names, $\mathcal N$, is the smallest relation satisfying the rules
below.

\infrule[Out-barb]{y \in {\mathcal N}, \; x \nameeq y}
		  {\outputp{x}{v} \downarrow_{\mathcal N} x}
\infrule[Par-barb]{\mbox{$P\downarrow_{\mathcal N} x$ or $Q\downarrow_{\mathcal N} x$}}
		  {\binpar{P}{Q} \downarrow_{\mathcal N} x}

We write $P \Downarrow_{\mathcal N} x$ if there is $Q$ such that 
$P \wred Q$ and $Q \downarrow_{\mathcal N} x$.
\end{definition}

\begin{definition}
%\label{def.bbisim}
An  ${\mathcal N}$-\emph{barbed bisimulation} over a set of names, ${\mathcal N}$, is a symmetric binary relation 
${\mathcal S}_{\mathcal N}$ between agents such that $P\rel{S}_{\mathcal N}Q$ implies:
\begin{enumerate}
\item If $P \red P'$ then $Q \wred Q'$ and $P'\rel{S}_{\mathcal N} Q'$.
\item If $P\downarrow_{\mathcal N} x$, then $Q\Downarrow_{\mathcal N} x$.
\end{enumerate}
$P$ is ${\mathcal N}$-barbed bisimilar to $Q$, written
$P \wbbisim_{\mathcal N} Q$, if $P \rel{S}_{\mathcal N} Q$ for some ${\mathcal N}$-barbed bisimulation ${\mathcal S}_{\mathcal N}$.
\end{definition}

$\mathcal{R} \subseteq \pi \times \pi$

$P \mathcal{R} Q => \forall P'. P \red P' \Rightarrow \exists Q'. Q \red Q', P' \mathcal{R} Q'$

$P \vdash x \Rightarrow Q \vdash x$

\begin{mathpar}
  \inferrule*[lab=Out-barb]{x \nameeq y}{{y}!\langle{Q}\rangle \vdash x}
  \and
  \inferrule*[lab=Par-barb]{\mbox{$P\vdash x$ or $Q\vdash x$}}{\binpar{P}{Q} \vdash x}
\end{mathpar}

\subsubsection{Contexts}

One of the principle advantages of computational calculi like the
$\pi$-calculus is a well-defined notion of context,
contextual-equivalence and a correlation between
contextual-equivalence and notions of bisimulation. The notion of
context allows the decomposition of a process into (sub-)process and
its syntactic environment, its context. Thus, a context may be
thought of as a process with a ``hole'' (written $\Box$) in it. The
application of a context $M$ to a process $P$, written $M[P]$, is
tantamount to filling the hole in $M$ with $P$. In this paper we do
not need the full weight of this theory, but do make use of the notion
of context in the proof the main theorem. 

\begin{mathpar}
  \inferrule* [lab=summation] {} {{M_{M},M_{N}} \bc \Box \;|\; x.M_{A} \;|\; M_{M}+M_{N}}
  \and
  \inferrule* [lab=agent] {} {{M_{A}} \bc (\vec{x})M_{P} \;| \; \clift{P_0,\ldots,M_{P},\ldots,P_N}}
  \and \\
  \inferrule* [lab=process] {} {{M_{P}} \bc M_{N} \;| \;P|M_{P} }
\end{mathpar} 

\begin{mathpar}
  \inferrule* [lab=sychronization] {} {M_{N} \bc \Box \;|\; x?M_{F} \;|\; x!M_{C}}
  \and
  \inferrule* [lab=abstraction] {} {{M_{F}} \bc (x)M_{P} }
  \and
  \inferrule* [lab=concretion] {} {{M_{C}} \bc \langle M_{P} \rangle }
  \and \\
  \inferrule* [lab=process] {} {{M_{P}} \bc M_{N} \;| \;P|M_{P} }
\end{mathpar}

\begin{definition}[contextual application] Given a context $M$, and
  process $P$, we define the \emph{contextual application}, $M[P] :=
  M\{P/\Box\}$. That is, the contextual application of M to P is the
  substitution of $P$ for $\Box$ in $M$.
\end{definition}

$\meaningof{-} : L \to \mathcal{P}(\pi)$

\begin{mathpar}
  \inferrule* [lab=collection] {} {\meaningof{true} = \pi, \and \meaningof{~E} = \pi \setminus \meaningof{E}, \and \meaningof{E_{1} \& E_{2}} = \meaningof{E_{1}} \cap \meaningof{E_{2}}}
\end{mathpar}

\begin{mathpar}
  \inferrule* [lab=structure] {} {\meaningof{0} = \{ P \in \pi | P \equiv 0 \}, \and \\ \meaningof{E_1 | E_2} = \{ P \in \pi | P \equiv P_{1} | P_{2}, P_{1} \in \meaningof{E_{1}}, P_{2} \in \meaningof{E_2}\} }
\end{mathpar}

\begin{mathpar}
 \inferrule* [lab=behavior] {} {\meaningof{\langle a?b \rangle E} = \{ P \in \pi | P \equiv Q | u?(y)P', \\ \and \\\\ \and \\ \;\;\; u \in \meaningof{a}, \forall z.P'\{z/y\} \in \meaningof{E\{z/b\}}\}, \and \\ \meaningof{a!E} = \{ P \in \pi | P \equiv Q | x!\langle P' \rangle, x \in \meaningof{a} P' \in \meaningof{E}\} }
\end{mathpar}

\begin{mathpar}
 \inferrule* [lab=nominal] {} {\meaningof{\quotep{E}} = \{ \quotep{P} \in \quotep{\pi} | P \in \meaningof{E} \}, \and \meaningof{\quotep{P}} = \{ \quotep{Q} \in \quotep{\pi} | P \equiv Q \} \and \\ \meaningof{@\quotep{E}} = \{ P \in \pi | P \equiv @x, x \in \meaningof{E} \}}
\end{mathpar}

\begin{eqnarray*}
  \\
  \meaningof{-} : TS \to ST
\end{eqnarray*}

\begin{eqnarray*}
  \\
  L : TS \to ST
\end{eqnarray*}

\begin{eqnarray*}
  \\
  P \models E \iff P \in \meaningof{E}
\end{eqnarray*}

\begin{eqnarray*}
  P \approx_{L} Q \iff \forall E \in L. P \models E \iff Q \models E
\end{eqnarray*}

\begin{eqnarray*}
  P \approx_{K} Q
\end{eqnarray*}

\begin{eqnarray*}
  P \approx Q
\end{eqnarray*}

$\approx_{K} = \approx = \approx_{L}$

\subsubsection{Contextual duality}

Note that contexts extend the quotation operation to a family of
operations from processes to names. Given a context, $M$, we can
define a \emph{nominal context}, $\quotep{M}$ by $\quotep{M}[P] :=
\quotep{M[P]}$. To foreshadow what is to come we observe that these
operations enjoy a duality with processes very much like the duality
between vectors and maps from vectors to scalars.

Further, because the calculus is essentially higher-order, we have a
correspondence between contexts and processes. More specifically,
given a name $x$ and a context $M$ we can construct $M^{*}_{x}$ such
that 

\begin{mathpar}
  M^{*}_{x} | \lift{x}{P} \red M[P]
\end{mathpar}

namely,

\begin{mathpar}
  M^{*}_{x} := x?(u).M[\dropn{u}]
\end{mathpar}

The dependence of $M^{*}_{x}$ on a name makes it an abstraction, 

\begin{mathpar}
  M^{*} := (x)x?(u).M[\dropn{u}]
\end{mathpar}

\subsection{Additional notation}

It will sometimes be convenient to denote the process a name
quotes. We already have the notation $x = \quotep{P}$, but it will be
convenient to introduce an alternate notation, $\procn{x}$, when we
want to emphasize the connection to the use of the name. Note that, by
virtue of name equivalence, $\quotep{\procn{x}} \nameeq x$; so, the
notation is consistent with previous definitions.

Further, because names have structure it is possible to effect
substitutions on the basis of that structure. This means we need to
upgrade our notation for substitutions, which we accomplish by
adapting comprehension notation. Thus,

\begin{mathpar}
  P\{ y / x : x \in S \}
\end{mathpar}

is interpreted to mean the process derived from P by replacing (in a
capture-avoiding manner) each occurrence of $x$ in $S$ by $y$. For example,

\begin{mathpar}
  P\{ \quotep{\procn{x}|\procn{x}} / x : x \in \freenames{P} \}
\end{mathpar}

will replace each (occurrence) of a free name $x$ in $P$ by
$\quotep{\procn{x}|\procn{x}}$.

Also, we will avail ourselves of the notation $x^{L}$ and $x^{R}$ to
denote injections of a name into disjoint copies of the name
space. There are numerous ways to accomplish this. One example can be
found in \cite{MeredithR05}. This notation overloads to vectors of
names: $\vec{x}^{\pi} := (x_{i}^{\pi} \; : \; 0 \leq i < |\vec{x}| )$ where $\pi \in \{L,R\}$.

We also use $P^{\Box} := P|\Box$.

In \cite{MeredithR05} an interpretation of the new operator is
given. It turns out that there are several possible interpretations
all enjoying the requisite algebraic properties of the operator (see
\cite{milner91polyadicpi}). We will therefore make liberal use of
$(\nu\; \vec{x})P$.

% subsection the_syntax_and_semantics_of_the_notation_system (end)   

\input{qm2pi.qmops} 

\input{qm2pi.sterngerlach} 

\input{qm2pi.metric} 

% section concurrent_process_calculi (end)

%\input{qm2pi.proofsketch}

% section proof sketch (end)

%\input{qm2pi.slviaknots} 

% section spatial logic via knots (end)

\input{qm2pi.conclusion}

% section conclusion (end)

%\input{qm2pi.dtcodes} 

% section wiring algorithm (end)

\input{qm2pi.ack} 

% section acknowledgments (end)

\newpage


\bibliographystyle{plain}   
\bibliography{../../biblios/main.bib}

\input{qm2pi.rhodetails}

\end{document}

 

% section concurrent_process_calculi (end)

%\documentclass[12pt]{llncs}
%\documentclass{jktr}

\usepackage[pdftex]{hyperref}                   
\usepackage {listings}
\usepackage {mathpartir}
\usepackage{bcprules}
%\usepackage{listings}
                       
\usepackage{graphicx} 
%\usepackage[margins=2.5cm,nohead,nofoot]{geometry}
%\usepackage{geometry}
\usepackage{amsfonts}
\usepackage{amstext}
\usepackage{latexsym}
\usepackage{amssymb}
\usepackage{color}


%\include{myPreamble}
\include{qm2pi.local} 

%\ifpdf
%\usepackage[pdftex]{graphicx}
%\else
%\usepackage{graphicx}
%\fi

 % \ifpdf
%  \usepackage{pdfsync}
%  \if


%\title{Brief Article}
%\author{David F. Snyder}
%\author{L.G. Meredith}

%\address{Dept. of Math., Texas State University--San Marcos, San Marcos, TX 78666}
       
\pagestyle{empty}


\begin{document}

\lstset{language=[Objective]Caml,frame=shadowbox}

\input{qm2pi.front}

% section front matter (end)

\input{qm2pi.intro} 
 
% section introduction (end)

% \input{qm2pi.knotations} 

% section notation (end)

\input{qm2pi.process.calculi} 

% section concurrent_process_calculi_and_spatial_logics_ (end)
    
%\input{qm2pi.knots2pi} 

%\input{qm2pi.trefoil} 

%\input{qm2pi.mainthm} 

% subsection basic_interpretation (end)

%\input{qm2pi.rho.presentation} 
\subsection{The syntax and semantics of the notation system}\label{sub:the_syntax_and_semantics_of_the_notation_system} % (fold)

We now summarize a technical presentation of the calculus that
embodies our theory of dynamics. The typical presentation of such a
calculus follows the style of giving generators and relations on
them. The grammar, below, describing term constructors, freely
generates the set of processes, $\Proc$. This set is then quotiented
by a relation known as structural congruence and it is over this set
that the notion of dynamics is expressed. This presentation is
essentially that of \cite{MeredithR05} with the addition of
polyadicity and summation. For readability we have relegated some of
the technical subtleties to an appendix.

\subsubsection{Process grammar}\label{subsub:process_grammar}

\begin{mathpar}
  \inferrule* [lab=synchronization] {} {{M} \bc \pzero \;|\; x?F \;|\; x!C }
  \and
  \inferrule* [lab=abstraction] {} {{F} \bc (x)P}
  \and
  \inferrule* [lab=concretion] {} {{C} \bc \langle Q \rangle}
  \and
  \inferrule* [lab=process] {} {{P,Q} \bc M \;| \;P|Q \;|\; @{x}}
  \and
  \inferrule* [lab=name] {} {{x} \bc \quotep{P}}
\end{mathpar} 

Note that $\vec{x}$ (resp. $\vec{P}$) denotes a vector of names
(resp. processes) of length $|\vec{x}|$ (resp. $|\vec{P}|$). We adopt
the following useful abbreviations.

\begin{mathpar}
   x?(\vec{y}).P := x.(\vec{y})P \and  x\clift{\vec{P}} := x.\clift{\vec{P}}
   \and x!(y) := \lift{x}{\dropn{y}}
   \and \Pi_{i=0}^{n-1}P_i := P_0 | \ldots | P_{n-1}
\end{mathpar}

\subsubsection{Structural congruence}

\paragraph{Free and bound names and alpha-equivalence.} At the
core of structural equivalence is alpha-equivalence which identifies
process that are the same up to a change of variable. Formally, we
recognize the distinction between free and bound names. The free names
of a process, $\freenames{P}$, may be calculated recursively as
follows:

\begin{mathpar}
\freenames{\pzero} := \emptyset
  \and \\
  \freenames{x?(y).P} := \{ x \} \cup (\freenames{P} \setminus \{ y \})
  \and 
  \freenames{x!\langle P \rangle} := \{ x \} \cup \{ P \} 
  \and \\
  \freenames{P|Q} := \freenames{P} \cup \freenames{Q}
  \and \\
  \freenames{@{x}} := \{ x \}
\end{mathpar}

$\pi$
$\quotep{\pi}$

$\freenames{-} : \pi \to \mathcal{P}(\quotep{\pi})$

\begin{eqnarray*}
  \freenames{\pzero} & := & \emptyset \\
  \freenames{x?(y).P} & := & \{ x \} \cup (\freenames{P} \setminus \{ y \}) \\
  \freenames{x!\langle P \rangle} & := & \{ x \} \cup \{ P \} \\
  \freenames{P|Q} & := & \freenames{P} \cup \freenames{Q} \\
  \freenames{\dropn{x}} & := & \{ x \}
\end{eqnarray*}

The bound names of a process, $\boundnames{P}$, are those names occurring in $P$
that are not free. For example, in $x?(y).0$, the name $x$ is free, while $y$ is bound.

\begin{mathpar}
  \inferrule* [lab=monoidal-laws] {} { P|Q \equiv Q|P \and P|0 \equiv P \and P|(Q|R) \equiv (P|Q)|R }
\end{mathpar}

\begin{mathpar}
  \inferrule* [lab=alpha-equivalence] {} { (x)P \equiv (y)P\{y/x\} \and y \not\in \freenames{P} }
\end{mathpar}

\begin{definition}
Then two processes, $P,Q$, are alpha-equivalent if $P = Q\{\vec{y}/\vec{x}\}$ for
some $\vec{x} \in \boundnames{Q},\vec{y} \in \boundnames{P}$, where $Q\{\vec{y}/\vec{x}\}$
denotes the capture-avoiding substitution of $\vec{y}$ for $\vec{x}$ in $Q$.
\end{definition}

\begin{definition}
  The {\em structural congruence} \cite{SangiorgiWalker} , $\equiv$,
  between processes is the least congruence containing
  alpha-equivalence, satisfying the abelian monoid laws
  (associativity, commutativity and $\pzero$ as identity) for parallel
  composition $|$ and for summation $+$.
\end{definition}

\subsection{Name equivalence}

We take name equivalence, written $\nameeq$, to be the smallest
equivalence relation generated by the following rules.

\begin{mathpar}
\inferrule*[lab=Quote-drop]
{ }
{ \quotep{@{x}} \nameeq x }

\inferrule*[lab=Struct-equiv]
{ P \scong Q }
{ \quotep{P} \nameeq \quotep{Q} }
\end{mathpar}

The astute reader will have noticed that the mutual recursion of names
and processes imposes a mutual recursion on alpha-equivalence and
structural equivalence via name-equivalence. Fortunately, all of this
works out pleasantly and we may calculate in the natural way, free of
concern. The reader interested in the details is referred to the
appendix \ref{appendix:rho_details}.

\subsection{Substitution}

We use $\Proc$ for the set of processes, $\QProc$ for the set of
names, and $\id{\{}\vec{y} / \vec{x} \id{\}}$ to denote partial maps,
$s : \QProc \rightarrow \QProc$. A map, $s$ lifts, uniquely, to a map
on process terms, $\widehat{s} : \Proc \rightarrow \Proc$ by the
following equations.

\begin{mathpar}
  (0) \psubstp{Q}{P} := 0 \\
  (R \juxtap S) \psubstp{Q}{P}
  :=    
  (R)\psubstp{Q}{P} \juxtap (S) \psubstp{Q}{P} \\
  (x?(y).R) \psubstp{Q}{P}    
  :=    
  (x)\substp{Q}{P} (z)\concat( (R \psubstn{z}{y}) \psubstp{Q}{P} ) \\
  (\lift{x}{R}) \psubstp{Q}{P}  
  :=
  \lift{(x)\substp{Q}{P}}{ R \psubstp{Q}{P} } \\
%   (\dropn{x})  \psubstp{Q}{P}       
%   := 
%   \left\{ 
%     \begin{array}{ccc} 
%       \dropn{\quotep{Q}} & & x \nameeq \quotep{P} \\
%       \dropn{x} & & otherwise \\
%     \end{array}
%   \right. 
  (\dropn{x})  \psubstp{Q}{P}       
  := 
  \left\{ 
    \begin{array}{ccc} 
      Q & & x \nameeq \quotep{P} \\
      \dropn{x} & & otherwise \\
    \end{array}
  \right.
\end{mathpar}
 

where

\begin{eqnarray}
  (x)\id{\{} \lpquote Q \rpquote / \lpquote P \rpquote \id{\}}            = 
  \left\{ 
    \begin{array}{ccc}
      \lpquote Q \rpquote & & x \nameeq \lpquote P \rpquote \\
      x & & otherwise \\
    \end{array}
  \right. \nonumber
\end{eqnarray}

and $z$ is chosen distinct from $\quotep{P}$, $\quotep{Q}$, the free
names in $Q$, and all the names in $R$. Our $\alpha$-equivalence will
be built in the standard way from this substitution.

\begin{remark}\label{rem:no_self_referential_names}
  One consequence of these definitions is that $\forall P. \quotep{P}
  \not\in \freenames{P}$.
\end{remark}

\subsection{ Dynamic quote: an example }

Anticipating something of what's to come, consider applying the
substitution, $\widehat{\id{\{}u / z \id{\}}}$, to the following pair
of processes, $\lift{w}{y!(z)}$ and $w[ \lpquote y!(z) \rpquote ]$.

\begin{eqnarray}
	\lift{w}{y!(z)}\widehat{\id{\{}u / z \id{\}}}
		& = &
		\lift{w}{y!(u)} \nonumber\\
	w[ \lpquote y!(z) \rpquote ] \widehat{ \id{\{}u / z \id{\}} }
		& = &
		w[ \lpquote y!(z) \rpquote ] \nonumber
\end{eqnarray}

Because the body of the process between quotes is impervious to
substitution, we get radically different answers. In fact, by
examining the first process in an input context,
e.g. $x?(z).\lift{w}{y!(z)}$, we see that the process under the lift
operator may be shaped by prefixed inputs binding a name inside it. In
this sense, the lift operator will be seen as a way to dynamically
construct processes before reifying them as names.

Finally equipped with these standard features we can present the
dynamics of the calculus.

\subsubsection{Operational semantics} 

Finally, we introduce the computational dynamics. What marks these
algebras as distinct from other more traditionally studied algebraic
structures, e.g. vector spaces or polynomial rings, is the manner in
which dynamics is captured. In traditional structures, dynamics is typically
expressed through morphisms between such structures, as in linear maps
between vector spaces or morphisms between rings. In algebras
associated with the semantics of computation, the dynamics is
expressed as part of the algebraic structure itself, through a
reduction reduction relation typically denoted by $\red$. Below, we
give a recursive presentation of this relation for the calculus used
in the encoding.

$\red \subseteq \pi \times \pi$
$\red : \pi \to \mathcal{P}(\pi)$

\begin{mathpar}
  \inferrule* [lab=Comm] { \textsf{match}( x_{src}, x_{trgt} ) } { x_{trgt}?(y)P \; | \; x_{src}!\langle {Q} \rangle \red P\{\quotep{Q}/y}\} }
  \and \\
  \inferrule* [lab=Par] {{P} \red {P}'} {{{P} | {Q}} \red {{P}' | {Q}}}
  \and
  \inferrule* [lab=Equiv]{{{P} \scong {P}'} \andalso {{P}' \red {Q}'} \andalso {{Q}' \scong {Q}}}{{P} \red {Q}}
\end{mathpar}

\begin{eqnarray*}
  match_{\equiv} (\quotep{P},\quotep{Q}) & := & P \equiv Q \\
  match_{\dagger}(\quotep{P},\quotep{Q}) & := & \forall R. P|Q \red^{*} R => R \red^{*} 0 \\
  match_{K}(\quotep{P},\quotep{Q}) & := & K \mbox{ for some context } K
\end{eqnarray*}

$u?(x)P | u!\langle Q \rangle \red P\{\quotep{Q}/x\}$

%We write $\wred$ for $\red^*$, and $P\red$ if $\exists Q $ such that $ P \red Q$.
We write $P\red$ if $\exists Q $ such that $ P \red Q$ and $P\not\red$, otherwise.

\section{Replication}

As mentioned before, it is known that replication (and hence
recursion) can be implemented in a higher-order process algebra
\cite{SangiorgiWalker}. As our first example of calculation with the
machinery thus far presented we give the construction explicitly in
the {\rhoc}.

\begin{eqnarray}
	D_{x} & := & \prefix{x}{y}{(\binpar{\outputp{x}{y}}{@{y}})} \nonumber\\
	\bangp_{x}{P} & := & \binpar{{x}!\langle{\binpar{D_{x}}{P}}\rangle}{D_{x}} \nonumber
\end{eqnarray}

\begin{eqnarray}
	\bangp_{x}{P} & & \nonumber\\
	=
	& {x}!\langle{(\prefix{x}{y}{(\outputp{x}{y} | @{y})) | P}}\rangle 
	      | \prefix{x}{y}{(\outputp{x}{y} | @{y})} & \nonumber\\
	\red
	& (\outputp{x}{y} | @{y})\substn{\quotep{(\prefix{x}{y}{(@{y} | \outputp{x}{y})) | P}}}{y} & \nonumber\\
	=
	& \outputp{x}{\quotep{(\prefix{x}{y}{(\outputp{x}{y} | @{y})) | P}}}
	  | {(\prefix{x}{y}{(\outputp{x}{y} | @{y})) | P}} & \nonumber\\
	\red
	& \ldots & \nonumber\\
	\red^*
	& P | P | \ldots & \nonumber
\end{eqnarray}

Of course, this encoding, as an implementation, runs away, unfolding
$\bangp{P}$ eagerly. A lazier and more implementable replication
operator, restricted to input-guarded processes, may be obtained as follows.

\begin{eqnarray}
\bangp{\prefix{u}{v}{P}} 
	:= 
	\binpar{\lift{x}{\prefix{u}{v}{(\binpar{D(x)}{P})}}}{D(x)} \nonumber
\end{eqnarray}

\begin{remark}
  Note that the lazier definition still does not deal with summation
  or mixed summation (i.e. sums over input and output). The reader is
  invited to construct definitions of replication that deal with these
  features. 

  Further, the definitions are parameterized in a name, $x$. Can you,
  gentle reader, make a definition that eliminates this parameter and
  guarantees no accidental interaction between the replication
  machinery and the process being replicated -- i.e. no accidental
  sharing of names used by the process to get its work done and the
  name(s) used by the replication to effect copying. This latter
  revision of the definition of replication is crucial to obtaining
  the expected identity $!!P \sim !P$.
\end{remark}

\begin{remark}\label{rem:paradoxical_combinator}
  The reader familiar with the lambda calculus will have noticed the
  similarity between $D$ and the paradoxical combinator.

  [Ed. note: the existence of this seems to suggest we have to be more
  restrictive on the set of processes and names we admit if we are to
  support no-cloning.]
\end{remark}

\subsubsection{Bisimulation}

The computational dynamics gives rise to another kind of equivalence,
the equivalence of computational behavior. As previously mentioned
this is typically captured \emph{via} some form of bisimulation.

% The notion we use in this paper is weak barbed bisimulation
% \cite{milner91polyadicpi}.

The notion we use in this paper is derived from weak barbed
bisimulation \cite{milner91polyadicpi}. 

\begin{definition}
An \emph{observation relation}, $\downarrow_{\mathcal N}$, over a set
of names, $\mathcal N$, is the smallest relation satisfying the rules
below.

\infrule[Out-barb]{y \in {\mathcal N}, \; x \nameeq y}
		  {\outputp{x}{v} \downarrow_{\mathcal N} x}
\infrule[Par-barb]{\mbox{$P\downarrow_{\mathcal N} x$ or $Q\downarrow_{\mathcal N} x$}}
		  {\binpar{P}{Q} \downarrow_{\mathcal N} x}

We write $P \Downarrow_{\mathcal N} x$ if there is $Q$ such that 
$P \wred Q$ and $Q \downarrow_{\mathcal N} x$.
\end{definition}

\begin{definition}
%\label{def.bbisim}
An  ${\mathcal N}$-\emph{barbed bisimulation} over a set of names, ${\mathcal N}$, is a symmetric binary relation 
${\mathcal S}_{\mathcal N}$ between agents such that $P\rel{S}_{\mathcal N}Q$ implies:
\begin{enumerate}
\item If $P \red P'$ then $Q \wred Q'$ and $P'\rel{S}_{\mathcal N} Q'$.
\item If $P\downarrow_{\mathcal N} x$, then $Q\Downarrow_{\mathcal N} x$.
\end{enumerate}
$P$ is ${\mathcal N}$-barbed bisimilar to $Q$, written
$P \wbbisim_{\mathcal N} Q$, if $P \rel{S}_{\mathcal N} Q$ for some ${\mathcal N}$-barbed bisimulation ${\mathcal S}_{\mathcal N}$.
\end{definition}

$\mathcal{R} \subseteq \pi \times \pi$

$P \mathcal{R} Q => \forall P'. P \red P' \Rightarrow \exists Q'. Q \red Q', P' \mathcal{R} Q'$

$P \vdash x \Rightarrow Q \vdash x$

\begin{mathpar}
  \inferrule*[lab=Out-barb]{x \nameeq y}{{y}!\langle{Q}\rangle \vdash x}
  \and
  \inferrule*[lab=Par-barb]{\mbox{$P\vdash x$ or $Q\vdash x$}}{\binpar{P}{Q} \vdash x}
\end{mathpar}

\subsubsection{Contexts}

One of the principle advantages of computational calculi like the
$\pi$-calculus is a well-defined notion of context,
contextual-equivalence and a correlation between
contextual-equivalence and notions of bisimulation. The notion of
context allows the decomposition of a process into (sub-)process and
its syntactic environment, its context. Thus, a context may be
thought of as a process with a ``hole'' (written $\Box$) in it. The
application of a context $M$ to a process $P$, written $M[P]$, is
tantamount to filling the hole in $M$ with $P$. In this paper we do
not need the full weight of this theory, but do make use of the notion
of context in the proof the main theorem. 

\begin{mathpar}
  \inferrule* [lab=summation] {} {{M_{M},M_{N}} \bc \Box \;|\; x.M_{A} \;|\; M_{M}+M_{N}}
  \and
  \inferrule* [lab=agent] {} {{M_{A}} \bc (\vec{x})M_{P} \;| \; \clift{P_0,\ldots,M_{P},\ldots,P_N}}
  \and \\
  \inferrule* [lab=process] {} {{M_{P}} \bc M_{N} \;| \;P|M_{P} }
\end{mathpar} 

\begin{mathpar}
  \inferrule* [lab=sychronization] {} {M_{N} \bc \Box \;|\; x?M_{F} \;|\; x!M_{C}}
  \and
  \inferrule* [lab=abstraction] {} {{M_{F}} \bc (x)M_{P} }
  \and
  \inferrule* [lab=concretion] {} {{M_{C}} \bc \langle M_{P} \rangle }
  \and \\
  \inferrule* [lab=process] {} {{M_{P}} \bc M_{N} \;| \;P|M_{P} }
\end{mathpar}

\begin{definition}[contextual application] Given a context $M$, and
  process $P$, we define the \emph{contextual application}, $M[P] :=
  M\{P/\Box\}$. That is, the contextual application of M to P is the
  substitution of $P$ for $\Box$ in $M$.
\end{definition}

$\meaningof{-} : L \to \mathcal{P}(\pi)$

\begin{mathpar}
  \inferrule* [lab=collection] {} {\meaningof{true} = \pi, \and \meaningof{~E} = \pi \setminus \meaningof{E}, \and \meaningof{E_{1} \& E_{2}} = \meaningof{E_{1}} \cap \meaningof{E_{2}}}
\end{mathpar}

\begin{mathpar}
  \inferrule* [lab=structure] {} {\meaningof{0} = \{ P \in \pi | P \equiv 0 \}, \and \\ \meaningof{E_1 | E_2} = \{ P \in \pi | P \equiv P_{1} | P_{2}, P_{1} \in \meaningof{E_{1}}, P_{2} \in \meaningof{E_2}\} }
\end{mathpar}

\begin{mathpar}
 \inferrule* [lab=behavior] {} {\meaningof{\langle a?b \rangle E} = \{ P \in \pi | P \equiv Q | u?(y)P', \\ \and \\\\ \and \\ \;\;\; u \in \meaningof{a}, \forall z.P'\{z/y\} \in \meaningof{E\{z/b\}}\}, \and \\ \meaningof{a!E} = \{ P \in \pi | P \equiv Q | x!\langle P' \rangle, x \in \meaningof{a} P' \in \meaningof{E}\} }
\end{mathpar}

\begin{mathpar}
 \inferrule* [lab=nominal] {} {\meaningof{\quotep{E}} = \{ \quotep{P} \in \quotep{\pi} | P \in \meaningof{E} \}, \and \meaningof{\quotep{P}} = \{ \quotep{Q} \in \quotep{\pi} | P \equiv Q \} \and \\ \meaningof{@\quotep{E}} = \{ P \in \pi | P \equiv @x, x \in \meaningof{E} \}}
\end{mathpar}

\begin{eqnarray*}
  \\
  \meaningof{-} : TS \to ST
\end{eqnarray*}

\begin{eqnarray*}
  \\
  L : TS \to ST
\end{eqnarray*}

\begin{eqnarray*}
  \\
  P \models E \iff P \in \meaningof{E}
\end{eqnarray*}

\begin{eqnarray*}
  P \approx_{L} Q \iff \forall E \in L. P \models E \iff Q \models E
\end{eqnarray*}

\begin{eqnarray*}
  P \approx_{K} Q
\end{eqnarray*}

\begin{eqnarray*}
  P \approx Q
\end{eqnarray*}

$\approx_{K} = \approx = \approx_{L}$

\subsubsection{Contextual duality}

Note that contexts extend the quotation operation to a family of
operations from processes to names. Given a context, $M$, we can
define a \emph{nominal context}, $\quotep{M}$ by $\quotep{M}[P] :=
\quotep{M[P]}$. To foreshadow what is to come we observe that these
operations enjoy a duality with processes very much like the duality
between vectors and maps from vectors to scalars.

Further, because the calculus is essentially higher-order, we have a
correspondence between contexts and processes. More specifically,
given a name $x$ and a context $M$ we can construct $M^{*}_{x}$ such
that 

\begin{mathpar}
  M^{*}_{x} | \lift{x}{P} \red M[P]
\end{mathpar}

namely,

\begin{mathpar}
  M^{*}_{x} := x?(u).M[\dropn{u}]
\end{mathpar}

The dependence of $M^{*}_{x}$ on a name makes it an abstraction, 

\begin{mathpar}
  M^{*} := (x)x?(u).M[\dropn{u}]
\end{mathpar}

\subsection{Additional notation}

It will sometimes be convenient to denote the process a name
quotes. We already have the notation $x = \quotep{P}$, but it will be
convenient to introduce an alternate notation, $\procn{x}$, when we
want to emphasize the connection to the use of the name. Note that, by
virtue of name equivalence, $\quotep{\procn{x}} \nameeq x$; so, the
notation is consistent with previous definitions.

Further, because names have structure it is possible to effect
substitutions on the basis of that structure. This means we need to
upgrade our notation for substitutions, which we accomplish by
adapting comprehension notation. Thus,

\begin{mathpar}
  P\{ y / x : x \in S \}
\end{mathpar}

is interpreted to mean the process derived from P by replacing (in a
capture-avoiding manner) each occurrence of $x$ in $S$ by $y$. For example,

\begin{mathpar}
  P\{ \quotep{\procn{x}|\procn{x}} / x : x \in \freenames{P} \}
\end{mathpar}

will replace each (occurrence) of a free name $x$ in $P$ by
$\quotep{\procn{x}|\procn{x}}$.

Also, we will avail ourselves of the notation $x^{L}$ and $x^{R}$ to
denote injections of a name into disjoint copies of the name
space. There are numerous ways to accomplish this. One example can be
found in \cite{MeredithR05}. This notation overloads to vectors of
names: $\vec{x}^{\pi} := (x_{i}^{\pi} \; : \; 0 \leq i < |\vec{x}| )$ where $\pi \in \{L,R\}$.

We also use $P^{\Box} := P|\Box$.

In \cite{MeredithR05} an interpretation of the new operator is
given. It turns out that there are several possible interpretations
all enjoying the requisite algebraic properties of the operator (see
\cite{milner91polyadicpi}). We will therefore make liberal use of
$(\nu\; \vec{x})P$.

% subsection the_syntax_and_semantics_of_the_notation_system (end)   

\input{qm2pi.qmops} 

\input{qm2pi.sterngerlach} 

\input{qm2pi.metric} 

% section concurrent_process_calculi (end)

%\input{qm2pi.proofsketch}

% section proof sketch (end)

%\input{qm2pi.slviaknots} 

% section spatial logic via knots (end)

\input{qm2pi.conclusion}

% section conclusion (end)

%\input{qm2pi.dtcodes} 

% section wiring algorithm (end)

\input{qm2pi.ack} 

% section acknowledgments (end)

\newpage


\bibliographystyle{plain}   
\bibliography{../../biblios/main.bib}

\input{qm2pi.rhodetails}

\end{document}



% section proof sketch (end)

%\section{Unlikely characters: spatial logic for
  knots}\label{sub:characteristic_formulae} % (fold)

Associated to the mobile process calculi are a family of logics known
as the Hennessy-Milner logics. These logics typically enjoy a
semantics interpreting formulae as sets of processes that when
factored through the encoding outlined above allows an identification
of classes of knots with logical formulae. In the context of this
encoding the sub-family known as the spatial logics \cite{CairesC03}
\cite{CairesC04} \cite{Caires04} are of particular interest providing
several important features for expressing and reasoning about
properties (i.e. classes) of knots. We hint here at how this may be done.

%\begin{description}
%\item [structural connectives] 
\subsubsection{Structural connectives} The spatial logics enjoy
structural connectives corresponding, at the logical level, to the
parallel composition ($P | Q$) and new name ($(\nu \; x)P$)
connectives for processes. As illustrated in the examples below, these
connectives are extremely expressive given the shape of our encoding.
%\item [decideable satisfaction]

\subsubsection{Decideable satisfaction}
In \cite{Caires04} the satisfaction relation is shown to be decideable
for a rich class of processes. It further turns out that the image of
the our encoding is a proper subset of that class. This result
provides the basis for an algorithm by which to search for knots
enjoying a given property.
%\item [characteristic formulae]

\subsubsection{Characteristic formulae}
In the same paper \cite{Caires04} , Caires presents a means of calculating
characteristic formulae, selecting equivalence classes of processes
up to a pre--specified depth limit on the support set of names. Composed with our
encoding, this characteristic formula can be used to select
characteristic formulae for knots.
%\end{description}

\subsubsection{Spatial logic formulae}

The grammar below (segmented for comprehension) summarizes the syntax
of spatial logic formulae. We employ illustrative examples in the
sequel to provide an intuitive understanding of their meaning
referring the reader to \cite{Caires04} for a more detailed explication
of the semantics.

\begin{mathpar}
  \inferrule* [lab=boolean] {} {{A,B} \bc T \;|\; \neg A \;|\; A \wedge B \;|\; \eta = \eta'}
  \and
  \inferrule* [lab=spatial] {} {|\; \pzero \;|\; A | B \;|\; x \text{\textregistered} A \;|\; \forall x . A \;|\;  H x . A}
  \and
  \inferrule* [lab=behavioral] {} {|\; \alpha . A}
  \and 
  \inferrule* [lab=recursion] {} {|\; X(\vec{u}) \;|\; \mu X(\vec{u}) . A}
  \and
  \inferrule* [lab=action] {} {\alpha \bc \langle x?(\vec{y}) \rangle \;|\; \langle x!(\vec{y}) \rangle \;|\; \langle \tau \rangle}
  \and 
  \inferrule* [lab=name] {} {\eta \bc x \;|\; \tau}
\end{mathpar} 

% subsection characteristic_formulae (end)   	 

\subsection{Example formulae}\label{sub:example_formulae_} % (fold)

\subsubsection{Crossing as formula.}
% 
% \begin{align*}
%   \frac{d}{dx} \sin x &= \cos x 
%   & \frac{d}{dx} e^x &= e^x \\
%   \frac{d}{dx} \cos x &= - \sin x 
%   & \frac{d}{dx} \log x &= \frac{1}{x} \\
% \end{align*} 

\begin{align*}
 \mu C(x_{0},x_{1},y_{0},y_{1},u).&(\langle x_{0}?(z) \rangle(\langle u! \rangle\langle y_{1}!z \rangle C(x_{0},x_{1},y_{0},y_{1},u)) & \\
  & \wedge \langle y_{1}?(z) \rangle (\langle u! \rangle \langle x_{0}!z \rangle C(x_{0},x_{1},y_{0},y_{1},u)) & \\
  & \wedge \langle x_{1}?(z) \rangle (\langle u? \rangle \langle y_{0}!z \rangle C(x_{0},x_{1},y_{0},y_{1},u)) & \\
  & \wedge \langle y_{0}?(z) \rangle (\langle u? \rangle \langle x_{1}!z \rangle C(x_{0},x_{1},y_{0},y_{1},u))) &
\end{align*}

The lexicographical similarity between the shape of this formulae and
the shape of definition of the process representing a crossing reveals
the intuitive meaning of this formulae. It describes the capabilities
of a process that has the right to represent a crossing. For example
it picks out processes that may perform an input on the port $x_0$ in
its initial menu of capabilities. What differentiates the formula
from the process, however, is that the crossing process is the
smallest candidate to satisfy the formula. Infinitely many other
processes -- with internal behavior hidden behind this interface, so
to speak -- also satisfy this formula. Even this simple formula,
then, can be seen to open a new view onto knots, providing a
computational interpretation of \emph{virtual} knots.

Note that this formula is derived by hand. A similar formula can be
derived by employing Caires' calculation of characteristic formula
\cite{Caires04} to the process representing a crossing. In light of
this discussion, we let
$\meaningof{C}_{\phi}(x0,x1,y0,y1,u)$ denote a formula specifying the
dynamics we wish to capture of a crossing. To guarantee we preserve
the shape of the interface and minimal semantics we demand that
$\meaningof{C}_{\phi}(x0,x1,y0,y1,u) \Rightarrow
\textbf{C}(x0,x1,y0,y1,u)$ where $\textbf{C}(x0,x1,y0,y1,u)$ denotes
the formula above.
                            
\subsubsection{Crossing number constraints.}
The moral content of the context lemma (Lemma \ref{context}) is that the notion of
``locality'' in the Reidemeister moves is effectively captured by the
parallel composition operator of the process calculus. This intuition
extends through the logic. Given a formula,
$\meaningof{C}_{\phi}(x0,x1,y0,y1,u)$, we can use the structural
connectives to specify constraints on crossing numbers, such as at
least $n$ crossings, or exactly $n$ crossings.
\begin{mathpar}
  \inferrule* [lab=at-least-n] {} { K^{\geq n}_{\phi}(\vec{xs},\vec{ys}) := \Pi_{i=0}^{n-1} Hu . \meaningof{C}_{\phi}(xs_i,ys_i,u) | T }
  \and 
  \inferrule* [lab=exactly-n] {} { K^{= n}_{\phi}(\vec{xs},\vec{ys}) := \Pi_{i=0}^{n-1} Hu . \meaningof{C}_{\phi}(xs_i,ys_i,u) | \neg (\forall x_0,y_0,x_1,y_1,u . \meaningof{C}_{\phi}(x_0,y_0,x_1,y_1,u) | T) }
\end{mathpar}

To round out this section, recall that the encoding of an $n$-crossing
knot decomposes into a parallel composition of $n$ \emph{copies} of a
crossing process together with a wiring harness. To specify different
knot classes with the same crossing number amounts to specifying
logical constraints on the wiring harness. In the interest of space,
we defer examples to a forthcoming paper. Suffice it to say that both
the conditions ``alternating knot'' and ``contains the tangle
corresponding to 5/3'' are expressible. For example, it is possible to
calculate the characteristic formula of a process corresponding to the
tangle 5/3 and conjoin it into the classifying formula via the
composition connective of the logic.

Finally, we wish to observe that it is entirely within reason to
contemplate a more domain-specific version of spatial logic tailored
to the shape of processes in the image of the encoding. Such a
domain-specific logic would have a better claim to the title formal
language of knot properties.

% subsection example_formulae_ (end)

% section knots_as_processes (end) 

% section spatial logic via knots (end)

\section{Conclusions and future work}

\paragraph{Testing physical space}
You, gentle reader, may wonder why of all the theorems to be proved
given this set up we pick the one above. In some sense it's hardly
central to quantum mechanics. We see it as central in the sense that
it firmly establishes a notion of physical space arising from a notion
of the equivalence of behavior. Relating bisimulation to a metric is a
big step forward, but one is faced with interpreting the relationship
of that metric space to something more physical. Quantum mechanical
notions of ``physical'' space are still far from intuitive, but by
relating this idea of distance as testing to calculations that predict
physical circumstances we are making a not insignificant step forward
toward an understanding of the physical space we inhabit as
essentially dynamic.

\paragraph{Effectivity and simulation}
One of the observations we have yet to make is that the entire program
spelled out here is effective. We have built various interpreters for
the reflective calculus at work in this interpretation. In principle,
then, we can simulate quantum mechanics on a computer. The place where
the simulation may lose fidelity is the infinitely branching summation
for the annihilator.

In this connection i also want to point out that the evaluation style
calculation of the inner product puts the non-determinism of the
summation right at the heart of measurement. This suggests that
Milner's original reduction-based formulation of the dynamics of his
calculi in terms of sums was not just notationally suggestive of a
notion of measure-and-continue but captured some significant part of
the physics.

\paragraph{Quantum continuations}
In light of this last observation i want to point out that the
predominant account of quantum mechanics is missing a key aspect of a
truly compositional story of the physical situation. In a real lab,
when a measurement is made the observation can be made to feed into
another device that then makes another measurement conditioned on the
results of the first. This means that after the superposition was
collapsed the entire experimental set up remained in
superposition. While QM offers a means of writing this down it doesn't
quite line up well with the well-trodden formulation of computation
and continuation that we see so succinctly expressed in Milner's
calculi. This suggests that there might be advantages to this account
of dynamics waiting to be explored.

\paragraph{Quantum logic}
In this connection, we also note that by virtue of having the
Hennessy-Milner construction, we can pull the construction through the
interpretation of QM. This gives us a natural candidate for a quantum
logic that enjoys an extremely tight connection with it's domain of
interpretation, making the construction much less ad hoc (rather it is
the image of functor!).

\paragraph{Quantum probabiity}
i have questions about the basis of the interpretation of inner
product as probability amplitude. In particular, using which
axiomatization of probability theory does the notion of probability
amplitude earn the right to be so dubbed? In other words, where is the
proof that the operation for calculating a probability amplitude (and
then squaring) satisfies the axioms of what it means to calculate a
probability? Even if such a proof exists (i have yet to find it in the
literature), i wonder if it might not be possible to turn things on
their heads. Can we view the calculation of the probability amplitude
as an axiomatization of probability? If so, then the definition we
give for calculating probability amplitude may provide the basis for
an \emph{effective} theory of probability.

\paragraph{Quantum vs ``biological'' information}
Finally, i want to conclude with a more philosophical observation. At
a recent workshop in which QM was a predominant topic i noticed
something about quantum information. The speaker was giving a riveting
discussion of axiomatic QM and showing how properties of ``no
cloning'' and ``no deleting'' emerged as consequences of the
axiomatization. Theorems of this form are necessary to give us a sense
of confidence that our axioms characterize the physical theory. What
struck me, though, was that if quantum information is neither erasable
nor replicable it is markedly different from \emph{life}. Two of the
things we know about life is that

\begin{itemize}
  \item it ends;
  \item to gain some measure of persistence, to transcend it's
    finitude it is imminently copyable.
\end{itemize}

Both of these qualities are summarized succinctly in the aphorism: all
flesh is grass. For me these two kinds of ``information'' -- call them
quantum and biological -- are end points on a spectrum of strategies
for persistence. At one end, we have those curious entities that enjoy
uniqueness and permanence; at the other, we have those who in the face
of a certain end and an uncertain present make a go of passing
something on. To me one of the more remarkable aspects of the latter
strategy is that in the presence of noise (and certain features of
copying) we get a kind of dynamism, a chance for improvement against a
given persistent condition.

% subsection other_calculi_other_bisimulations_and_geometry_as_behavior (end)




% section conclusion (end)

%\documentclass[12pt]{llncs}
%\documentclass{jktr}

\usepackage[pdftex]{hyperref}                   
\usepackage {listings}
\usepackage {mathpartir}
\usepackage{bcprules}
%\usepackage{listings}
                       
\usepackage{graphicx} 
%\usepackage[margins=2.5cm,nohead,nofoot]{geometry}
%\usepackage{geometry}
\usepackage{amsfonts}
\usepackage{amstext}
\usepackage{latexsym}
\usepackage{amssymb}
\usepackage{color}


%\include{myPreamble}
\include{qm2pi.local} 

%\ifpdf
%\usepackage[pdftex]{graphicx}
%\else
%\usepackage{graphicx}
%\fi

 % \ifpdf
%  \usepackage{pdfsync}
%  \if


%\title{Brief Article}
%\author{David F. Snyder}
%\author{L.G. Meredith}

%\address{Dept. of Math., Texas State University--San Marcos, San Marcos, TX 78666}
       
\pagestyle{empty}


\begin{document}

\lstset{language=[Objective]Caml,frame=shadowbox}

\input{qm2pi.front}

% section front matter (end)

\input{qm2pi.intro} 
 
% section introduction (end)

% \input{qm2pi.knotations} 

% section notation (end)

\input{qm2pi.process.calculi} 

% section concurrent_process_calculi_and_spatial_logics_ (end)
    
%\input{qm2pi.knots2pi} 

%\input{qm2pi.trefoil} 

%\input{qm2pi.mainthm} 

% subsection basic_interpretation (end)

%\input{qm2pi.rho.presentation} 
\subsection{The syntax and semantics of the notation system}\label{sub:the_syntax_and_semantics_of_the_notation_system} % (fold)

We now summarize a technical presentation of the calculus that
embodies our theory of dynamics. The typical presentation of such a
calculus follows the style of giving generators and relations on
them. The grammar, below, describing term constructors, freely
generates the set of processes, $\Proc$. This set is then quotiented
by a relation known as structural congruence and it is over this set
that the notion of dynamics is expressed. This presentation is
essentially that of \cite{MeredithR05} with the addition of
polyadicity and summation. For readability we have relegated some of
the technical subtleties to an appendix.

\subsubsection{Process grammar}\label{subsub:process_grammar}

\begin{mathpar}
  \inferrule* [lab=synchronization] {} {{M} \bc \pzero \;|\; x?F \;|\; x!C }
  \and
  \inferrule* [lab=abstraction] {} {{F} \bc (x)P}
  \and
  \inferrule* [lab=concretion] {} {{C} \bc \langle Q \rangle}
  \and
  \inferrule* [lab=process] {} {{P,Q} \bc M \;| \;P|Q \;|\; @{x}}
  \and
  \inferrule* [lab=name] {} {{x} \bc \quotep{P}}
\end{mathpar} 

Note that $\vec{x}$ (resp. $\vec{P}$) denotes a vector of names
(resp. processes) of length $|\vec{x}|$ (resp. $|\vec{P}|$). We adopt
the following useful abbreviations.

\begin{mathpar}
   x?(\vec{y}).P := x.(\vec{y})P \and  x\clift{\vec{P}} := x.\clift{\vec{P}}
   \and x!(y) := \lift{x}{\dropn{y}}
   \and \Pi_{i=0}^{n-1}P_i := P_0 | \ldots | P_{n-1}
\end{mathpar}

\subsubsection{Structural congruence}

\paragraph{Free and bound names and alpha-equivalence.} At the
core of structural equivalence is alpha-equivalence which identifies
process that are the same up to a change of variable. Formally, we
recognize the distinction between free and bound names. The free names
of a process, $\freenames{P}$, may be calculated recursively as
follows:

\begin{mathpar}
\freenames{\pzero} := \emptyset
  \and \\
  \freenames{x?(y).P} := \{ x \} \cup (\freenames{P} \setminus \{ y \})
  \and 
  \freenames{x!\langle P \rangle} := \{ x \} \cup \{ P \} 
  \and \\
  \freenames{P|Q} := \freenames{P} \cup \freenames{Q}
  \and \\
  \freenames{@{x}} := \{ x \}
\end{mathpar}

$\pi$
$\quotep{\pi}$

$\freenames{-} : \pi \to \mathcal{P}(\quotep{\pi})$

\begin{eqnarray*}
  \freenames{\pzero} & := & \emptyset \\
  \freenames{x?(y).P} & := & \{ x \} \cup (\freenames{P} \setminus \{ y \}) \\
  \freenames{x!\langle P \rangle} & := & \{ x \} \cup \{ P \} \\
  \freenames{P|Q} & := & \freenames{P} \cup \freenames{Q} \\
  \freenames{\dropn{x}} & := & \{ x \}
\end{eqnarray*}

The bound names of a process, $\boundnames{P}$, are those names occurring in $P$
that are not free. For example, in $x?(y).0$, the name $x$ is free, while $y$ is bound.

\begin{mathpar}
  \inferrule* [lab=monoidal-laws] {} { P|Q \equiv Q|P \and P|0 \equiv P \and P|(Q|R) \equiv (P|Q)|R }
\end{mathpar}

\begin{mathpar}
  \inferrule* [lab=alpha-equivalence] {} { (x)P \equiv (y)P\{y/x\} \and y \not\in \freenames{P} }
\end{mathpar}

\begin{definition}
Then two processes, $P,Q$, are alpha-equivalent if $P = Q\{\vec{y}/\vec{x}\}$ for
some $\vec{x} \in \boundnames{Q},\vec{y} \in \boundnames{P}$, where $Q\{\vec{y}/\vec{x}\}$
denotes the capture-avoiding substitution of $\vec{y}$ for $\vec{x}$ in $Q$.
\end{definition}

\begin{definition}
  The {\em structural congruence} \cite{SangiorgiWalker} , $\equiv$,
  between processes is the least congruence containing
  alpha-equivalence, satisfying the abelian monoid laws
  (associativity, commutativity and $\pzero$ as identity) for parallel
  composition $|$ and for summation $+$.
\end{definition}

\subsection{Name equivalence}

We take name equivalence, written $\nameeq$, to be the smallest
equivalence relation generated by the following rules.

\begin{mathpar}
\inferrule*[lab=Quote-drop]
{ }
{ \quotep{@{x}} \nameeq x }

\inferrule*[lab=Struct-equiv]
{ P \scong Q }
{ \quotep{P} \nameeq \quotep{Q} }
\end{mathpar}

The astute reader will have noticed that the mutual recursion of names
and processes imposes a mutual recursion on alpha-equivalence and
structural equivalence via name-equivalence. Fortunately, all of this
works out pleasantly and we may calculate in the natural way, free of
concern. The reader interested in the details is referred to the
appendix \ref{appendix:rho_details}.

\subsection{Substitution}

We use $\Proc$ for the set of processes, $\QProc$ for the set of
names, and $\id{\{}\vec{y} / \vec{x} \id{\}}$ to denote partial maps,
$s : \QProc \rightarrow \QProc$. A map, $s$ lifts, uniquely, to a map
on process terms, $\widehat{s} : \Proc \rightarrow \Proc$ by the
following equations.

\begin{mathpar}
  (0) \psubstp{Q}{P} := 0 \\
  (R \juxtap S) \psubstp{Q}{P}
  :=    
  (R)\psubstp{Q}{P} \juxtap (S) \psubstp{Q}{P} \\
  (x?(y).R) \psubstp{Q}{P}    
  :=    
  (x)\substp{Q}{P} (z)\concat( (R \psubstn{z}{y}) \psubstp{Q}{P} ) \\
  (\lift{x}{R}) \psubstp{Q}{P}  
  :=
  \lift{(x)\substp{Q}{P}}{ R \psubstp{Q}{P} } \\
%   (\dropn{x})  \psubstp{Q}{P}       
%   := 
%   \left\{ 
%     \begin{array}{ccc} 
%       \dropn{\quotep{Q}} & & x \nameeq \quotep{P} \\
%       \dropn{x} & & otherwise \\
%     \end{array}
%   \right. 
  (\dropn{x})  \psubstp{Q}{P}       
  := 
  \left\{ 
    \begin{array}{ccc} 
      Q & & x \nameeq \quotep{P} \\
      \dropn{x} & & otherwise \\
    \end{array}
  \right.
\end{mathpar}
 

where

\begin{eqnarray}
  (x)\id{\{} \lpquote Q \rpquote / \lpquote P \rpquote \id{\}}            = 
  \left\{ 
    \begin{array}{ccc}
      \lpquote Q \rpquote & & x \nameeq \lpquote P \rpquote \\
      x & & otherwise \\
    \end{array}
  \right. \nonumber
\end{eqnarray}

and $z$ is chosen distinct from $\quotep{P}$, $\quotep{Q}$, the free
names in $Q$, and all the names in $R$. Our $\alpha$-equivalence will
be built in the standard way from this substitution.

\begin{remark}\label{rem:no_self_referential_names}
  One consequence of these definitions is that $\forall P. \quotep{P}
  \not\in \freenames{P}$.
\end{remark}

\subsection{ Dynamic quote: an example }

Anticipating something of what's to come, consider applying the
substitution, $\widehat{\id{\{}u / z \id{\}}}$, to the following pair
of processes, $\lift{w}{y!(z)}$ and $w[ \lpquote y!(z) \rpquote ]$.

\begin{eqnarray}
	\lift{w}{y!(z)}\widehat{\id{\{}u / z \id{\}}}
		& = &
		\lift{w}{y!(u)} \nonumber\\
	w[ \lpquote y!(z) \rpquote ] \widehat{ \id{\{}u / z \id{\}} }
		& = &
		w[ \lpquote y!(z) \rpquote ] \nonumber
\end{eqnarray}

Because the body of the process between quotes is impervious to
substitution, we get radically different answers. In fact, by
examining the first process in an input context,
e.g. $x?(z).\lift{w}{y!(z)}$, we see that the process under the lift
operator may be shaped by prefixed inputs binding a name inside it. In
this sense, the lift operator will be seen as a way to dynamically
construct processes before reifying them as names.

Finally equipped with these standard features we can present the
dynamics of the calculus.

\subsubsection{Operational semantics} 

Finally, we introduce the computational dynamics. What marks these
algebras as distinct from other more traditionally studied algebraic
structures, e.g. vector spaces or polynomial rings, is the manner in
which dynamics is captured. In traditional structures, dynamics is typically
expressed through morphisms between such structures, as in linear maps
between vector spaces or morphisms between rings. In algebras
associated with the semantics of computation, the dynamics is
expressed as part of the algebraic structure itself, through a
reduction reduction relation typically denoted by $\red$. Below, we
give a recursive presentation of this relation for the calculus used
in the encoding.

$\red \subseteq \pi \times \pi$
$\red : \pi \to \mathcal{P}(\pi)$

\begin{mathpar}
  \inferrule* [lab=Comm] { \textsf{match}( x_{src}, x_{trgt} ) } { x_{trgt}?(y)P \; | \; x_{src}!\langle {Q} \rangle \red P\{\quotep{Q}/y}\} }
  \and \\
  \inferrule* [lab=Par] {{P} \red {P}'} {{{P} | {Q}} \red {{P}' | {Q}}}
  \and
  \inferrule* [lab=Equiv]{{{P} \scong {P}'} \andalso {{P}' \red {Q}'} \andalso {{Q}' \scong {Q}}}{{P} \red {Q}}
\end{mathpar}

\begin{eqnarray*}
  match_{\equiv} (\quotep{P},\quotep{Q}) & := & P \equiv Q \\
  match_{\dagger}(\quotep{P},\quotep{Q}) & := & \forall R. P|Q \red^{*} R => R \red^{*} 0 \\
  match_{K}(\quotep{P},\quotep{Q}) & := & K \mbox{ for some context } K
\end{eqnarray*}

$u?(x)P | u!\langle Q \rangle \red P\{\quotep{Q}/x\}$

%We write $\wred$ for $\red^*$, and $P\red$ if $\exists Q $ such that $ P \red Q$.
We write $P\red$ if $\exists Q $ such that $ P \red Q$ and $P\not\red$, otherwise.

\section{Replication}

As mentioned before, it is known that replication (and hence
recursion) can be implemented in a higher-order process algebra
\cite{SangiorgiWalker}. As our first example of calculation with the
machinery thus far presented we give the construction explicitly in
the {\rhoc}.

\begin{eqnarray}
	D_{x} & := & \prefix{x}{y}{(\binpar{\outputp{x}{y}}{@{y}})} \nonumber\\
	\bangp_{x}{P} & := & \binpar{{x}!\langle{\binpar{D_{x}}{P}}\rangle}{D_{x}} \nonumber
\end{eqnarray}

\begin{eqnarray}
	\bangp_{x}{P} & & \nonumber\\
	=
	& {x}!\langle{(\prefix{x}{y}{(\outputp{x}{y} | @{y})) | P}}\rangle 
	      | \prefix{x}{y}{(\outputp{x}{y} | @{y})} & \nonumber\\
	\red
	& (\outputp{x}{y} | @{y})\substn{\quotep{(\prefix{x}{y}{(@{y} | \outputp{x}{y})) | P}}}{y} & \nonumber\\
	=
	& \outputp{x}{\quotep{(\prefix{x}{y}{(\outputp{x}{y} | @{y})) | P}}}
	  | {(\prefix{x}{y}{(\outputp{x}{y} | @{y})) | P}} & \nonumber\\
	\red
	& \ldots & \nonumber\\
	\red^*
	& P | P | \ldots & \nonumber
\end{eqnarray}

Of course, this encoding, as an implementation, runs away, unfolding
$\bangp{P}$ eagerly. A lazier and more implementable replication
operator, restricted to input-guarded processes, may be obtained as follows.

\begin{eqnarray}
\bangp{\prefix{u}{v}{P}} 
	:= 
	\binpar{\lift{x}{\prefix{u}{v}{(\binpar{D(x)}{P})}}}{D(x)} \nonumber
\end{eqnarray}

\begin{remark}
  Note that the lazier definition still does not deal with summation
  or mixed summation (i.e. sums over input and output). The reader is
  invited to construct definitions of replication that deal with these
  features. 

  Further, the definitions are parameterized in a name, $x$. Can you,
  gentle reader, make a definition that eliminates this parameter and
  guarantees no accidental interaction between the replication
  machinery and the process being replicated -- i.e. no accidental
  sharing of names used by the process to get its work done and the
  name(s) used by the replication to effect copying. This latter
  revision of the definition of replication is crucial to obtaining
  the expected identity $!!P \sim !P$.
\end{remark}

\begin{remark}\label{rem:paradoxical_combinator}
  The reader familiar with the lambda calculus will have noticed the
  similarity between $D$ and the paradoxical combinator.

  [Ed. note: the existence of this seems to suggest we have to be more
  restrictive on the set of processes and names we admit if we are to
  support no-cloning.]
\end{remark}

\subsubsection{Bisimulation}

The computational dynamics gives rise to another kind of equivalence,
the equivalence of computational behavior. As previously mentioned
this is typically captured \emph{via} some form of bisimulation.

% The notion we use in this paper is weak barbed bisimulation
% \cite{milner91polyadicpi}.

The notion we use in this paper is derived from weak barbed
bisimulation \cite{milner91polyadicpi}. 

\begin{definition}
An \emph{observation relation}, $\downarrow_{\mathcal N}$, over a set
of names, $\mathcal N$, is the smallest relation satisfying the rules
below.

\infrule[Out-barb]{y \in {\mathcal N}, \; x \nameeq y}
		  {\outputp{x}{v} \downarrow_{\mathcal N} x}
\infrule[Par-barb]{\mbox{$P\downarrow_{\mathcal N} x$ or $Q\downarrow_{\mathcal N} x$}}
		  {\binpar{P}{Q} \downarrow_{\mathcal N} x}

We write $P \Downarrow_{\mathcal N} x$ if there is $Q$ such that 
$P \wred Q$ and $Q \downarrow_{\mathcal N} x$.
\end{definition}

\begin{definition}
%\label{def.bbisim}
An  ${\mathcal N}$-\emph{barbed bisimulation} over a set of names, ${\mathcal N}$, is a symmetric binary relation 
${\mathcal S}_{\mathcal N}$ between agents such that $P\rel{S}_{\mathcal N}Q$ implies:
\begin{enumerate}
\item If $P \red P'$ then $Q \wred Q'$ and $P'\rel{S}_{\mathcal N} Q'$.
\item If $P\downarrow_{\mathcal N} x$, then $Q\Downarrow_{\mathcal N} x$.
\end{enumerate}
$P$ is ${\mathcal N}$-barbed bisimilar to $Q$, written
$P \wbbisim_{\mathcal N} Q$, if $P \rel{S}_{\mathcal N} Q$ for some ${\mathcal N}$-barbed bisimulation ${\mathcal S}_{\mathcal N}$.
\end{definition}

$\mathcal{R} \subseteq \pi \times \pi$

$P \mathcal{R} Q => \forall P'. P \red P' \Rightarrow \exists Q'. Q \red Q', P' \mathcal{R} Q'$

$P \vdash x \Rightarrow Q \vdash x$

\begin{mathpar}
  \inferrule*[lab=Out-barb]{x \nameeq y}{{y}!\langle{Q}\rangle \vdash x}
  \and
  \inferrule*[lab=Par-barb]{\mbox{$P\vdash x$ or $Q\vdash x$}}{\binpar{P}{Q} \vdash x}
\end{mathpar}

\subsubsection{Contexts}

One of the principle advantages of computational calculi like the
$\pi$-calculus is a well-defined notion of context,
contextual-equivalence and a correlation between
contextual-equivalence and notions of bisimulation. The notion of
context allows the decomposition of a process into (sub-)process and
its syntactic environment, its context. Thus, a context may be
thought of as a process with a ``hole'' (written $\Box$) in it. The
application of a context $M$ to a process $P$, written $M[P]$, is
tantamount to filling the hole in $M$ with $P$. In this paper we do
not need the full weight of this theory, but do make use of the notion
of context in the proof the main theorem. 

\begin{mathpar}
  \inferrule* [lab=summation] {} {{M_{M},M_{N}} \bc \Box \;|\; x.M_{A} \;|\; M_{M}+M_{N}}
  \and
  \inferrule* [lab=agent] {} {{M_{A}} \bc (\vec{x})M_{P} \;| \; \clift{P_0,\ldots,M_{P},\ldots,P_N}}
  \and \\
  \inferrule* [lab=process] {} {{M_{P}} \bc M_{N} \;| \;P|M_{P} }
\end{mathpar} 

\begin{mathpar}
  \inferrule* [lab=sychronization] {} {M_{N} \bc \Box \;|\; x?M_{F} \;|\; x!M_{C}}
  \and
  \inferrule* [lab=abstraction] {} {{M_{F}} \bc (x)M_{P} }
  \and
  \inferrule* [lab=concretion] {} {{M_{C}} \bc \langle M_{P} \rangle }
  \and \\
  \inferrule* [lab=process] {} {{M_{P}} \bc M_{N} \;| \;P|M_{P} }
\end{mathpar}

\begin{definition}[contextual application] Given a context $M$, and
  process $P$, we define the \emph{contextual application}, $M[P] :=
  M\{P/\Box\}$. That is, the contextual application of M to P is the
  substitution of $P$ for $\Box$ in $M$.
\end{definition}

$\meaningof{-} : L \to \mathcal{P}(\pi)$

\begin{mathpar}
  \inferrule* [lab=collection] {} {\meaningof{true} = \pi, \and \meaningof{~E} = \pi \setminus \meaningof{E}, \and \meaningof{E_{1} \& E_{2}} = \meaningof{E_{1}} \cap \meaningof{E_{2}}}
\end{mathpar}

\begin{mathpar}
  \inferrule* [lab=structure] {} {\meaningof{0} = \{ P \in \pi | P \equiv 0 \}, \and \\ \meaningof{E_1 | E_2} = \{ P \in \pi | P \equiv P_{1} | P_{2}, P_{1} \in \meaningof{E_{1}}, P_{2} \in \meaningof{E_2}\} }
\end{mathpar}

\begin{mathpar}
 \inferrule* [lab=behavior] {} {\meaningof{\langle a?b \rangle E} = \{ P \in \pi | P \equiv Q | u?(y)P', \\ \and \\\\ \and \\ \;\;\; u \in \meaningof{a}, \forall z.P'\{z/y\} \in \meaningof{E\{z/b\}}\}, \and \\ \meaningof{a!E} = \{ P \in \pi | P \equiv Q | x!\langle P' \rangle, x \in \meaningof{a} P' \in \meaningof{E}\} }
\end{mathpar}

\begin{mathpar}
 \inferrule* [lab=nominal] {} {\meaningof{\quotep{E}} = \{ \quotep{P} \in \quotep{\pi} | P \in \meaningof{E} \}, \and \meaningof{\quotep{P}} = \{ \quotep{Q} \in \quotep{\pi} | P \equiv Q \} \and \\ \meaningof{@\quotep{E}} = \{ P \in \pi | P \equiv @x, x \in \meaningof{E} \}}
\end{mathpar}

\begin{eqnarray*}
  \\
  \meaningof{-} : TS \to ST
\end{eqnarray*}

\begin{eqnarray*}
  \\
  L : TS \to ST
\end{eqnarray*}

\begin{eqnarray*}
  \\
  P \models E \iff P \in \meaningof{E}
\end{eqnarray*}

\begin{eqnarray*}
  P \approx_{L} Q \iff \forall E \in L. P \models E \iff Q \models E
\end{eqnarray*}

\begin{eqnarray*}
  P \approx_{K} Q
\end{eqnarray*}

\begin{eqnarray*}
  P \approx Q
\end{eqnarray*}

$\approx_{K} = \approx = \approx_{L}$

\subsubsection{Contextual duality}

Note that contexts extend the quotation operation to a family of
operations from processes to names. Given a context, $M$, we can
define a \emph{nominal context}, $\quotep{M}$ by $\quotep{M}[P] :=
\quotep{M[P]}$. To foreshadow what is to come we observe that these
operations enjoy a duality with processes very much like the duality
between vectors and maps from vectors to scalars.

Further, because the calculus is essentially higher-order, we have a
correspondence between contexts and processes. More specifically,
given a name $x$ and a context $M$ we can construct $M^{*}_{x}$ such
that 

\begin{mathpar}
  M^{*}_{x} | \lift{x}{P} \red M[P]
\end{mathpar}

namely,

\begin{mathpar}
  M^{*}_{x} := x?(u).M[\dropn{u}]
\end{mathpar}

The dependence of $M^{*}_{x}$ on a name makes it an abstraction, 

\begin{mathpar}
  M^{*} := (x)x?(u).M[\dropn{u}]
\end{mathpar}

\subsection{Additional notation}

It will sometimes be convenient to denote the process a name
quotes. We already have the notation $x = \quotep{P}$, but it will be
convenient to introduce an alternate notation, $\procn{x}$, when we
want to emphasize the connection to the use of the name. Note that, by
virtue of name equivalence, $\quotep{\procn{x}} \nameeq x$; so, the
notation is consistent with previous definitions.

Further, because names have structure it is possible to effect
substitutions on the basis of that structure. This means we need to
upgrade our notation for substitutions, which we accomplish by
adapting comprehension notation. Thus,

\begin{mathpar}
  P\{ y / x : x \in S \}
\end{mathpar}

is interpreted to mean the process derived from P by replacing (in a
capture-avoiding manner) each occurrence of $x$ in $S$ by $y$. For example,

\begin{mathpar}
  P\{ \quotep{\procn{x}|\procn{x}} / x : x \in \freenames{P} \}
\end{mathpar}

will replace each (occurrence) of a free name $x$ in $P$ by
$\quotep{\procn{x}|\procn{x}}$.

Also, we will avail ourselves of the notation $x^{L}$ and $x^{R}$ to
denote injections of a name into disjoint copies of the name
space. There are numerous ways to accomplish this. One example can be
found in \cite{MeredithR05}. This notation overloads to vectors of
names: $\vec{x}^{\pi} := (x_{i}^{\pi} \; : \; 0 \leq i < |\vec{x}| )$ where $\pi \in \{L,R\}$.

We also use $P^{\Box} := P|\Box$.

In \cite{MeredithR05} an interpretation of the new operator is
given. It turns out that there are several possible interpretations
all enjoying the requisite algebraic properties of the operator (see
\cite{milner91polyadicpi}). We will therefore make liberal use of
$(\nu\; \vec{x})P$.

% subsection the_syntax_and_semantics_of_the_notation_system (end)   

\input{qm2pi.qmops} 

\input{qm2pi.sterngerlach} 

\input{qm2pi.metric} 

% section concurrent_process_calculi (end)

%\input{qm2pi.proofsketch}

% section proof sketch (end)

%\input{qm2pi.slviaknots} 

% section spatial logic via knots (end)

\input{qm2pi.conclusion}

% section conclusion (end)

%\input{qm2pi.dtcodes} 

% section wiring algorithm (end)

\input{qm2pi.ack} 

% section acknowledgments (end)

\newpage


\bibliographystyle{plain}   
\bibliography{../../biblios/main.bib}

\input{qm2pi.rhodetails}

\end{document}

 

% section wiring algorithm (end)

\documentclass[12pt]{llncs}
%\documentclass{jktr}

\usepackage[pdftex]{hyperref}                   
\usepackage {listings}
\usepackage {mathpartir}
\usepackage{bcprules}
%\usepackage{listings}
                       
\usepackage{graphicx} 
%\usepackage[margins=2.5cm,nohead,nofoot]{geometry}
%\usepackage{geometry}
\usepackage{amsfonts}
\usepackage{amstext}
\usepackage{latexsym}
\usepackage{amssymb}
\usepackage{color}


%\include{myPreamble}
\include{qm2pi.local} 

%\ifpdf
%\usepackage[pdftex]{graphicx}
%\else
%\usepackage{graphicx}
%\fi

 % \ifpdf
%  \usepackage{pdfsync}
%  \if


%\title{Brief Article}
%\author{David F. Snyder}
%\author{L.G. Meredith}

%\address{Dept. of Math., Texas State University--San Marcos, San Marcos, TX 78666}
       
\pagestyle{empty}


\begin{document}

\lstset{language=[Objective]Caml,frame=shadowbox}

\input{qm2pi.front}

% section front matter (end)

\input{qm2pi.intro} 
 
% section introduction (end)

% \input{qm2pi.knotations} 

% section notation (end)

\input{qm2pi.process.calculi} 

% section concurrent_process_calculi_and_spatial_logics_ (end)
    
%\input{qm2pi.knots2pi} 

%\input{qm2pi.trefoil} 

%\input{qm2pi.mainthm} 

% subsection basic_interpretation (end)

%\input{qm2pi.rho.presentation} 
\subsection{The syntax and semantics of the notation system}\label{sub:the_syntax_and_semantics_of_the_notation_system} % (fold)

We now summarize a technical presentation of the calculus that
embodies our theory of dynamics. The typical presentation of such a
calculus follows the style of giving generators and relations on
them. The grammar, below, describing term constructors, freely
generates the set of processes, $\Proc$. This set is then quotiented
by a relation known as structural congruence and it is over this set
that the notion of dynamics is expressed. This presentation is
essentially that of \cite{MeredithR05} with the addition of
polyadicity and summation. For readability we have relegated some of
the technical subtleties to an appendix.

\subsubsection{Process grammar}\label{subsub:process_grammar}

\begin{mathpar}
  \inferrule* [lab=synchronization] {} {{M} \bc \pzero \;|\; x?F \;|\; x!C }
  \and
  \inferrule* [lab=abstraction] {} {{F} \bc (x)P}
  \and
  \inferrule* [lab=concretion] {} {{C} \bc \langle Q \rangle}
  \and
  \inferrule* [lab=process] {} {{P,Q} \bc M \;| \;P|Q \;|\; @{x}}
  \and
  \inferrule* [lab=name] {} {{x} \bc \quotep{P}}
\end{mathpar} 

Note that $\vec{x}$ (resp. $\vec{P}$) denotes a vector of names
(resp. processes) of length $|\vec{x}|$ (resp. $|\vec{P}|$). We adopt
the following useful abbreviations.

\begin{mathpar}
   x?(\vec{y}).P := x.(\vec{y})P \and  x\clift{\vec{P}} := x.\clift{\vec{P}}
   \and x!(y) := \lift{x}{\dropn{y}}
   \and \Pi_{i=0}^{n-1}P_i := P_0 | \ldots | P_{n-1}
\end{mathpar}

\subsubsection{Structural congruence}

\paragraph{Free and bound names and alpha-equivalence.} At the
core of structural equivalence is alpha-equivalence which identifies
process that are the same up to a change of variable. Formally, we
recognize the distinction between free and bound names. The free names
of a process, $\freenames{P}$, may be calculated recursively as
follows:

\begin{mathpar}
\freenames{\pzero} := \emptyset
  \and \\
  \freenames{x?(y).P} := \{ x \} \cup (\freenames{P} \setminus \{ y \})
  \and 
  \freenames{x!\langle P \rangle} := \{ x \} \cup \{ P \} 
  \and \\
  \freenames{P|Q} := \freenames{P} \cup \freenames{Q}
  \and \\
  \freenames{@{x}} := \{ x \}
\end{mathpar}

$\pi$
$\quotep{\pi}$

$\freenames{-} : \pi \to \mathcal{P}(\quotep{\pi})$

\begin{eqnarray*}
  \freenames{\pzero} & := & \emptyset \\
  \freenames{x?(y).P} & := & \{ x \} \cup (\freenames{P} \setminus \{ y \}) \\
  \freenames{x!\langle P \rangle} & := & \{ x \} \cup \{ P \} \\
  \freenames{P|Q} & := & \freenames{P} \cup \freenames{Q} \\
  \freenames{\dropn{x}} & := & \{ x \}
\end{eqnarray*}

The bound names of a process, $\boundnames{P}$, are those names occurring in $P$
that are not free. For example, in $x?(y).0$, the name $x$ is free, while $y$ is bound.

\begin{mathpar}
  \inferrule* [lab=monoidal-laws] {} { P|Q \equiv Q|P \and P|0 \equiv P \and P|(Q|R) \equiv (P|Q)|R }
\end{mathpar}

\begin{mathpar}
  \inferrule* [lab=alpha-equivalence] {} { (x)P \equiv (y)P\{y/x\} \and y \not\in \freenames{P} }
\end{mathpar}

\begin{definition}
Then two processes, $P,Q$, are alpha-equivalent if $P = Q\{\vec{y}/\vec{x}\}$ for
some $\vec{x} \in \boundnames{Q},\vec{y} \in \boundnames{P}$, where $Q\{\vec{y}/\vec{x}\}$
denotes the capture-avoiding substitution of $\vec{y}$ for $\vec{x}$ in $Q$.
\end{definition}

\begin{definition}
  The {\em structural congruence} \cite{SangiorgiWalker} , $\equiv$,
  between processes is the least congruence containing
  alpha-equivalence, satisfying the abelian monoid laws
  (associativity, commutativity and $\pzero$ as identity) for parallel
  composition $|$ and for summation $+$.
\end{definition}

\subsection{Name equivalence}

We take name equivalence, written $\nameeq$, to be the smallest
equivalence relation generated by the following rules.

\begin{mathpar}
\inferrule*[lab=Quote-drop]
{ }
{ \quotep{@{x}} \nameeq x }

\inferrule*[lab=Struct-equiv]
{ P \scong Q }
{ \quotep{P} \nameeq \quotep{Q} }
\end{mathpar}

The astute reader will have noticed that the mutual recursion of names
and processes imposes a mutual recursion on alpha-equivalence and
structural equivalence via name-equivalence. Fortunately, all of this
works out pleasantly and we may calculate in the natural way, free of
concern. The reader interested in the details is referred to the
appendix \ref{appendix:rho_details}.

\subsection{Substitution}

We use $\Proc$ for the set of processes, $\QProc$ for the set of
names, and $\id{\{}\vec{y} / \vec{x} \id{\}}$ to denote partial maps,
$s : \QProc \rightarrow \QProc$. A map, $s$ lifts, uniquely, to a map
on process terms, $\widehat{s} : \Proc \rightarrow \Proc$ by the
following equations.

\begin{mathpar}
  (0) \psubstp{Q}{P} := 0 \\
  (R \juxtap S) \psubstp{Q}{P}
  :=    
  (R)\psubstp{Q}{P} \juxtap (S) \psubstp{Q}{P} \\
  (x?(y).R) \psubstp{Q}{P}    
  :=    
  (x)\substp{Q}{P} (z)\concat( (R \psubstn{z}{y}) \psubstp{Q}{P} ) \\
  (\lift{x}{R}) \psubstp{Q}{P}  
  :=
  \lift{(x)\substp{Q}{P}}{ R \psubstp{Q}{P} } \\
%   (\dropn{x})  \psubstp{Q}{P}       
%   := 
%   \left\{ 
%     \begin{array}{ccc} 
%       \dropn{\quotep{Q}} & & x \nameeq \quotep{P} \\
%       \dropn{x} & & otherwise \\
%     \end{array}
%   \right. 
  (\dropn{x})  \psubstp{Q}{P}       
  := 
  \left\{ 
    \begin{array}{ccc} 
      Q & & x \nameeq \quotep{P} \\
      \dropn{x} & & otherwise \\
    \end{array}
  \right.
\end{mathpar}
 

where

\begin{eqnarray}
  (x)\id{\{} \lpquote Q \rpquote / \lpquote P \rpquote \id{\}}            = 
  \left\{ 
    \begin{array}{ccc}
      \lpquote Q \rpquote & & x \nameeq \lpquote P \rpquote \\
      x & & otherwise \\
    \end{array}
  \right. \nonumber
\end{eqnarray}

and $z$ is chosen distinct from $\quotep{P}$, $\quotep{Q}$, the free
names in $Q$, and all the names in $R$. Our $\alpha$-equivalence will
be built in the standard way from this substitution.

\begin{remark}\label{rem:no_self_referential_names}
  One consequence of these definitions is that $\forall P. \quotep{P}
  \not\in \freenames{P}$.
\end{remark}

\subsection{ Dynamic quote: an example }

Anticipating something of what's to come, consider applying the
substitution, $\widehat{\id{\{}u / z \id{\}}}$, to the following pair
of processes, $\lift{w}{y!(z)}$ and $w[ \lpquote y!(z) \rpquote ]$.

\begin{eqnarray}
	\lift{w}{y!(z)}\widehat{\id{\{}u / z \id{\}}}
		& = &
		\lift{w}{y!(u)} \nonumber\\
	w[ \lpquote y!(z) \rpquote ] \widehat{ \id{\{}u / z \id{\}} }
		& = &
		w[ \lpquote y!(z) \rpquote ] \nonumber
\end{eqnarray}

Because the body of the process between quotes is impervious to
substitution, we get radically different answers. In fact, by
examining the first process in an input context,
e.g. $x?(z).\lift{w}{y!(z)}$, we see that the process under the lift
operator may be shaped by prefixed inputs binding a name inside it. In
this sense, the lift operator will be seen as a way to dynamically
construct processes before reifying them as names.

Finally equipped with these standard features we can present the
dynamics of the calculus.

\subsubsection{Operational semantics} 

Finally, we introduce the computational dynamics. What marks these
algebras as distinct from other more traditionally studied algebraic
structures, e.g. vector spaces or polynomial rings, is the manner in
which dynamics is captured. In traditional structures, dynamics is typically
expressed through morphisms between such structures, as in linear maps
between vector spaces or morphisms between rings. In algebras
associated with the semantics of computation, the dynamics is
expressed as part of the algebraic structure itself, through a
reduction reduction relation typically denoted by $\red$. Below, we
give a recursive presentation of this relation for the calculus used
in the encoding.

$\red \subseteq \pi \times \pi$
$\red : \pi \to \mathcal{P}(\pi)$

\begin{mathpar}
  \inferrule* [lab=Comm] { \textsf{match}( x_{src}, x_{trgt} ) } { x_{trgt}?(y)P \; | \; x_{src}!\langle {Q} \rangle \red P\{\quotep{Q}/y}\} }
  \and \\
  \inferrule* [lab=Par] {{P} \red {P}'} {{{P} | {Q}} \red {{P}' | {Q}}}
  \and
  \inferrule* [lab=Equiv]{{{P} \scong {P}'} \andalso {{P}' \red {Q}'} \andalso {{Q}' \scong {Q}}}{{P} \red {Q}}
\end{mathpar}

\begin{eqnarray*}
  match_{\equiv} (\quotep{P},\quotep{Q}) & := & P \equiv Q \\
  match_{\dagger}(\quotep{P},\quotep{Q}) & := & \forall R. P|Q \red^{*} R => R \red^{*} 0 \\
  match_{K}(\quotep{P},\quotep{Q}) & := & K \mbox{ for some context } K
\end{eqnarray*}

$u?(x)P | u!\langle Q \rangle \red P\{\quotep{Q}/x\}$

%We write $\wred$ for $\red^*$, and $P\red$ if $\exists Q $ such that $ P \red Q$.
We write $P\red$ if $\exists Q $ such that $ P \red Q$ and $P\not\red$, otherwise.

\section{Replication}

As mentioned before, it is known that replication (and hence
recursion) can be implemented in a higher-order process algebra
\cite{SangiorgiWalker}. As our first example of calculation with the
machinery thus far presented we give the construction explicitly in
the {\rhoc}.

\begin{eqnarray}
	D_{x} & := & \prefix{x}{y}{(\binpar{\outputp{x}{y}}{@{y}})} \nonumber\\
	\bangp_{x}{P} & := & \binpar{{x}!\langle{\binpar{D_{x}}{P}}\rangle}{D_{x}} \nonumber
\end{eqnarray}

\begin{eqnarray}
	\bangp_{x}{P} & & \nonumber\\
	=
	& {x}!\langle{(\prefix{x}{y}{(\outputp{x}{y} | @{y})) | P}}\rangle 
	      | \prefix{x}{y}{(\outputp{x}{y} | @{y})} & \nonumber\\
	\red
	& (\outputp{x}{y} | @{y})\substn{\quotep{(\prefix{x}{y}{(@{y} | \outputp{x}{y})) | P}}}{y} & \nonumber\\
	=
	& \outputp{x}{\quotep{(\prefix{x}{y}{(\outputp{x}{y} | @{y})) | P}}}
	  | {(\prefix{x}{y}{(\outputp{x}{y} | @{y})) | P}} & \nonumber\\
	\red
	& \ldots & \nonumber\\
	\red^*
	& P | P | \ldots & \nonumber
\end{eqnarray}

Of course, this encoding, as an implementation, runs away, unfolding
$\bangp{P}$ eagerly. A lazier and more implementable replication
operator, restricted to input-guarded processes, may be obtained as follows.

\begin{eqnarray}
\bangp{\prefix{u}{v}{P}} 
	:= 
	\binpar{\lift{x}{\prefix{u}{v}{(\binpar{D(x)}{P})}}}{D(x)} \nonumber
\end{eqnarray}

\begin{remark}
  Note that the lazier definition still does not deal with summation
  or mixed summation (i.e. sums over input and output). The reader is
  invited to construct definitions of replication that deal with these
  features. 

  Further, the definitions are parameterized in a name, $x$. Can you,
  gentle reader, make a definition that eliminates this parameter and
  guarantees no accidental interaction between the replication
  machinery and the process being replicated -- i.e. no accidental
  sharing of names used by the process to get its work done and the
  name(s) used by the replication to effect copying. This latter
  revision of the definition of replication is crucial to obtaining
  the expected identity $!!P \sim !P$.
\end{remark}

\begin{remark}\label{rem:paradoxical_combinator}
  The reader familiar with the lambda calculus will have noticed the
  similarity between $D$ and the paradoxical combinator.

  [Ed. note: the existence of this seems to suggest we have to be more
  restrictive on the set of processes and names we admit if we are to
  support no-cloning.]
\end{remark}

\subsubsection{Bisimulation}

The computational dynamics gives rise to another kind of equivalence,
the equivalence of computational behavior. As previously mentioned
this is typically captured \emph{via} some form of bisimulation.

% The notion we use in this paper is weak barbed bisimulation
% \cite{milner91polyadicpi}.

The notion we use in this paper is derived from weak barbed
bisimulation \cite{milner91polyadicpi}. 

\begin{definition}
An \emph{observation relation}, $\downarrow_{\mathcal N}$, over a set
of names, $\mathcal N$, is the smallest relation satisfying the rules
below.

\infrule[Out-barb]{y \in {\mathcal N}, \; x \nameeq y}
		  {\outputp{x}{v} \downarrow_{\mathcal N} x}
\infrule[Par-barb]{\mbox{$P\downarrow_{\mathcal N} x$ or $Q\downarrow_{\mathcal N} x$}}
		  {\binpar{P}{Q} \downarrow_{\mathcal N} x}

We write $P \Downarrow_{\mathcal N} x$ if there is $Q$ such that 
$P \wred Q$ and $Q \downarrow_{\mathcal N} x$.
\end{definition}

\begin{definition}
%\label{def.bbisim}
An  ${\mathcal N}$-\emph{barbed bisimulation} over a set of names, ${\mathcal N}$, is a symmetric binary relation 
${\mathcal S}_{\mathcal N}$ between agents such that $P\rel{S}_{\mathcal N}Q$ implies:
\begin{enumerate}
\item If $P \red P'$ then $Q \wred Q'$ and $P'\rel{S}_{\mathcal N} Q'$.
\item If $P\downarrow_{\mathcal N} x$, then $Q\Downarrow_{\mathcal N} x$.
\end{enumerate}
$P$ is ${\mathcal N}$-barbed bisimilar to $Q$, written
$P \wbbisim_{\mathcal N} Q$, if $P \rel{S}_{\mathcal N} Q$ for some ${\mathcal N}$-barbed bisimulation ${\mathcal S}_{\mathcal N}$.
\end{definition}

$\mathcal{R} \subseteq \pi \times \pi$

$P \mathcal{R} Q => \forall P'. P \red P' \Rightarrow \exists Q'. Q \red Q', P' \mathcal{R} Q'$

$P \vdash x \Rightarrow Q \vdash x$

\begin{mathpar}
  \inferrule*[lab=Out-barb]{x \nameeq y}{{y}!\langle{Q}\rangle \vdash x}
  \and
  \inferrule*[lab=Par-barb]{\mbox{$P\vdash x$ or $Q\vdash x$}}{\binpar{P}{Q} \vdash x}
\end{mathpar}

\subsubsection{Contexts}

One of the principle advantages of computational calculi like the
$\pi$-calculus is a well-defined notion of context,
contextual-equivalence and a correlation between
contextual-equivalence and notions of bisimulation. The notion of
context allows the decomposition of a process into (sub-)process and
its syntactic environment, its context. Thus, a context may be
thought of as a process with a ``hole'' (written $\Box$) in it. The
application of a context $M$ to a process $P$, written $M[P]$, is
tantamount to filling the hole in $M$ with $P$. In this paper we do
not need the full weight of this theory, but do make use of the notion
of context in the proof the main theorem. 

\begin{mathpar}
  \inferrule* [lab=summation] {} {{M_{M},M_{N}} \bc \Box \;|\; x.M_{A} \;|\; M_{M}+M_{N}}
  \and
  \inferrule* [lab=agent] {} {{M_{A}} \bc (\vec{x})M_{P} \;| \; \clift{P_0,\ldots,M_{P},\ldots,P_N}}
  \and \\
  \inferrule* [lab=process] {} {{M_{P}} \bc M_{N} \;| \;P|M_{P} }
\end{mathpar} 

\begin{mathpar}
  \inferrule* [lab=sychronization] {} {M_{N} \bc \Box \;|\; x?M_{F} \;|\; x!M_{C}}
  \and
  \inferrule* [lab=abstraction] {} {{M_{F}} \bc (x)M_{P} }
  \and
  \inferrule* [lab=concretion] {} {{M_{C}} \bc \langle M_{P} \rangle }
  \and \\
  \inferrule* [lab=process] {} {{M_{P}} \bc M_{N} \;| \;P|M_{P} }
\end{mathpar}

\begin{definition}[contextual application] Given a context $M$, and
  process $P$, we define the \emph{contextual application}, $M[P] :=
  M\{P/\Box\}$. That is, the contextual application of M to P is the
  substitution of $P$ for $\Box$ in $M$.
\end{definition}

$\meaningof{-} : L \to \mathcal{P}(\pi)$

\begin{mathpar}
  \inferrule* [lab=collection] {} {\meaningof{true} = \pi, \and \meaningof{~E} = \pi \setminus \meaningof{E}, \and \meaningof{E_{1} \& E_{2}} = \meaningof{E_{1}} \cap \meaningof{E_{2}}}
\end{mathpar}

\begin{mathpar}
  \inferrule* [lab=structure] {} {\meaningof{0} = \{ P \in \pi | P \equiv 0 \}, \and \\ \meaningof{E_1 | E_2} = \{ P \in \pi | P \equiv P_{1} | P_{2}, P_{1} \in \meaningof{E_{1}}, P_{2} \in \meaningof{E_2}\} }
\end{mathpar}

\begin{mathpar}
 \inferrule* [lab=behavior] {} {\meaningof{\langle a?b \rangle E} = \{ P \in \pi | P \equiv Q | u?(y)P', \\ \and \\\\ \and \\ \;\;\; u \in \meaningof{a}, \forall z.P'\{z/y\} \in \meaningof{E\{z/b\}}\}, \and \\ \meaningof{a!E} = \{ P \in \pi | P \equiv Q | x!\langle P' \rangle, x \in \meaningof{a} P' \in \meaningof{E}\} }
\end{mathpar}

\begin{mathpar}
 \inferrule* [lab=nominal] {} {\meaningof{\quotep{E}} = \{ \quotep{P} \in \quotep{\pi} | P \in \meaningof{E} \}, \and \meaningof{\quotep{P}} = \{ \quotep{Q} \in \quotep{\pi} | P \equiv Q \} \and \\ \meaningof{@\quotep{E}} = \{ P \in \pi | P \equiv @x, x \in \meaningof{E} \}}
\end{mathpar}

\begin{eqnarray*}
  \\
  \meaningof{-} : TS \to ST
\end{eqnarray*}

\begin{eqnarray*}
  \\
  L : TS \to ST
\end{eqnarray*}

\begin{eqnarray*}
  \\
  P \models E \iff P \in \meaningof{E}
\end{eqnarray*}

\begin{eqnarray*}
  P \approx_{L} Q \iff \forall E \in L. P \models E \iff Q \models E
\end{eqnarray*}

\begin{eqnarray*}
  P \approx_{K} Q
\end{eqnarray*}

\begin{eqnarray*}
  P \approx Q
\end{eqnarray*}

$\approx_{K} = \approx = \approx_{L}$

\subsubsection{Contextual duality}

Note that contexts extend the quotation operation to a family of
operations from processes to names. Given a context, $M$, we can
define a \emph{nominal context}, $\quotep{M}$ by $\quotep{M}[P] :=
\quotep{M[P]}$. To foreshadow what is to come we observe that these
operations enjoy a duality with processes very much like the duality
between vectors and maps from vectors to scalars.

Further, because the calculus is essentially higher-order, we have a
correspondence between contexts and processes. More specifically,
given a name $x$ and a context $M$ we can construct $M^{*}_{x}$ such
that 

\begin{mathpar}
  M^{*}_{x} | \lift{x}{P} \red M[P]
\end{mathpar}

namely,

\begin{mathpar}
  M^{*}_{x} := x?(u).M[\dropn{u}]
\end{mathpar}

The dependence of $M^{*}_{x}$ on a name makes it an abstraction, 

\begin{mathpar}
  M^{*} := (x)x?(u).M[\dropn{u}]
\end{mathpar}

\subsection{Additional notation}

It will sometimes be convenient to denote the process a name
quotes. We already have the notation $x = \quotep{P}$, but it will be
convenient to introduce an alternate notation, $\procn{x}$, when we
want to emphasize the connection to the use of the name. Note that, by
virtue of name equivalence, $\quotep{\procn{x}} \nameeq x$; so, the
notation is consistent with previous definitions.

Further, because names have structure it is possible to effect
substitutions on the basis of that structure. This means we need to
upgrade our notation for substitutions, which we accomplish by
adapting comprehension notation. Thus,

\begin{mathpar}
  P\{ y / x : x \in S \}
\end{mathpar}

is interpreted to mean the process derived from P by replacing (in a
capture-avoiding manner) each occurrence of $x$ in $S$ by $y$. For example,

\begin{mathpar}
  P\{ \quotep{\procn{x}|\procn{x}} / x : x \in \freenames{P} \}
\end{mathpar}

will replace each (occurrence) of a free name $x$ in $P$ by
$\quotep{\procn{x}|\procn{x}}$.

Also, we will avail ourselves of the notation $x^{L}$ and $x^{R}$ to
denote injections of a name into disjoint copies of the name
space. There are numerous ways to accomplish this. One example can be
found in \cite{MeredithR05}. This notation overloads to vectors of
names: $\vec{x}^{\pi} := (x_{i}^{\pi} \; : \; 0 \leq i < |\vec{x}| )$ where $\pi \in \{L,R\}$.

We also use $P^{\Box} := P|\Box$.

In \cite{MeredithR05} an interpretation of the new operator is
given. It turns out that there are several possible interpretations
all enjoying the requisite algebraic properties of the operator (see
\cite{milner91polyadicpi}). We will therefore make liberal use of
$(\nu\; \vec{x})P$.

% subsection the_syntax_and_semantics_of_the_notation_system (end)   

\input{qm2pi.qmops} 

\input{qm2pi.sterngerlach} 

\input{qm2pi.metric} 

% section concurrent_process_calculi (end)

%\input{qm2pi.proofsketch}

% section proof sketch (end)

%\input{qm2pi.slviaknots} 

% section spatial logic via knots (end)

\input{qm2pi.conclusion}

% section conclusion (end)

%\input{qm2pi.dtcodes} 

% section wiring algorithm (end)

\input{qm2pi.ack} 

% section acknowledgments (end)

\newpage


\bibliographystyle{plain}   
\bibliography{../../biblios/main.bib}

\input{qm2pi.rhodetails}

\end{document}

 

% section acknowledgments (end)

\newpage


\bibliographystyle{plain}   
\bibliography{../../biblios/main.bib}

\documentclass[12pt]{llncs}
%\documentclass{jktr}

\usepackage[pdftex]{hyperref}                   
\usepackage {listings}
\usepackage {mathpartir}
\usepackage{bcprules}
%\usepackage{listings}
                       
\usepackage{graphicx} 
%\usepackage[margins=2.5cm,nohead,nofoot]{geometry}
%\usepackage{geometry}
\usepackage{amsfonts}
\usepackage{amstext}
\usepackage{latexsym}
\usepackage{amssymb}
\usepackage{color}


%\include{myPreamble}
\include{qm2pi.local} 

%\ifpdf
%\usepackage[pdftex]{graphicx}
%\else
%\usepackage{graphicx}
%\fi

 % \ifpdf
%  \usepackage{pdfsync}
%  \if


%\title{Brief Article}
%\author{David F. Snyder}
%\author{L.G. Meredith}

%\address{Dept. of Math., Texas State University--San Marcos, San Marcos, TX 78666}
       
\pagestyle{empty}


\begin{document}

\lstset{language=[Objective]Caml,frame=shadowbox}

\input{qm2pi.front}

% section front matter (end)

\input{qm2pi.intro} 
 
% section introduction (end)

% \input{qm2pi.knotations} 

% section notation (end)

\input{qm2pi.process.calculi} 

% section concurrent_process_calculi_and_spatial_logics_ (end)
    
%\input{qm2pi.knots2pi} 

%\input{qm2pi.trefoil} 

%\input{qm2pi.mainthm} 

% subsection basic_interpretation (end)

%\input{qm2pi.rho.presentation} 
\subsection{The syntax and semantics of the notation system}\label{sub:the_syntax_and_semantics_of_the_notation_system} % (fold)

We now summarize a technical presentation of the calculus that
embodies our theory of dynamics. The typical presentation of such a
calculus follows the style of giving generators and relations on
them. The grammar, below, describing term constructors, freely
generates the set of processes, $\Proc$. This set is then quotiented
by a relation known as structural congruence and it is over this set
that the notion of dynamics is expressed. This presentation is
essentially that of \cite{MeredithR05} with the addition of
polyadicity and summation. For readability we have relegated some of
the technical subtleties to an appendix.

\subsubsection{Process grammar}\label{subsub:process_grammar}

\begin{mathpar}
  \inferrule* [lab=synchronization] {} {{M} \bc \pzero \;|\; x?F \;|\; x!C }
  \and
  \inferrule* [lab=abstraction] {} {{F} \bc (x)P}
  \and
  \inferrule* [lab=concretion] {} {{C} \bc \langle Q \rangle}
  \and
  \inferrule* [lab=process] {} {{P,Q} \bc M \;| \;P|Q \;|\; @{x}}
  \and
  \inferrule* [lab=name] {} {{x} \bc \quotep{P}}
\end{mathpar} 

Note that $\vec{x}$ (resp. $\vec{P}$) denotes a vector of names
(resp. processes) of length $|\vec{x}|$ (resp. $|\vec{P}|$). We adopt
the following useful abbreviations.

\begin{mathpar}
   x?(\vec{y}).P := x.(\vec{y})P \and  x\clift{\vec{P}} := x.\clift{\vec{P}}
   \and x!(y) := \lift{x}{\dropn{y}}
   \and \Pi_{i=0}^{n-1}P_i := P_0 | \ldots | P_{n-1}
\end{mathpar}

\subsubsection{Structural congruence}

\paragraph{Free and bound names and alpha-equivalence.} At the
core of structural equivalence is alpha-equivalence which identifies
process that are the same up to a change of variable. Formally, we
recognize the distinction between free and bound names. The free names
of a process, $\freenames{P}$, may be calculated recursively as
follows:

\begin{mathpar}
\freenames{\pzero} := \emptyset
  \and \\
  \freenames{x?(y).P} := \{ x \} \cup (\freenames{P} \setminus \{ y \})
  \and 
  \freenames{x!\langle P \rangle} := \{ x \} \cup \{ P \} 
  \and \\
  \freenames{P|Q} := \freenames{P} \cup \freenames{Q}
  \and \\
  \freenames{@{x}} := \{ x \}
\end{mathpar}

$\pi$
$\quotep{\pi}$

$\freenames{-} : \pi \to \mathcal{P}(\quotep{\pi})$

\begin{eqnarray*}
  \freenames{\pzero} & := & \emptyset \\
  \freenames{x?(y).P} & := & \{ x \} \cup (\freenames{P} \setminus \{ y \}) \\
  \freenames{x!\langle P \rangle} & := & \{ x \} \cup \{ P \} \\
  \freenames{P|Q} & := & \freenames{P} \cup \freenames{Q} \\
  \freenames{\dropn{x}} & := & \{ x \}
\end{eqnarray*}

The bound names of a process, $\boundnames{P}$, are those names occurring in $P$
that are not free. For example, in $x?(y).0$, the name $x$ is free, while $y$ is bound.

\begin{mathpar}
  \inferrule* [lab=monoidal-laws] {} { P|Q \equiv Q|P \and P|0 \equiv P \and P|(Q|R) \equiv (P|Q)|R }
\end{mathpar}

\begin{mathpar}
  \inferrule* [lab=alpha-equivalence] {} { (x)P \equiv (y)P\{y/x\} \and y \not\in \freenames{P} }
\end{mathpar}

\begin{definition}
Then two processes, $P,Q$, are alpha-equivalent if $P = Q\{\vec{y}/\vec{x}\}$ for
some $\vec{x} \in \boundnames{Q},\vec{y} \in \boundnames{P}$, where $Q\{\vec{y}/\vec{x}\}$
denotes the capture-avoiding substitution of $\vec{y}$ for $\vec{x}$ in $Q$.
\end{definition}

\begin{definition}
  The {\em structural congruence} \cite{SangiorgiWalker} , $\equiv$,
  between processes is the least congruence containing
  alpha-equivalence, satisfying the abelian monoid laws
  (associativity, commutativity and $\pzero$ as identity) for parallel
  composition $|$ and for summation $+$.
\end{definition}

\subsection{Name equivalence}

We take name equivalence, written $\nameeq$, to be the smallest
equivalence relation generated by the following rules.

\begin{mathpar}
\inferrule*[lab=Quote-drop]
{ }
{ \quotep{@{x}} \nameeq x }

\inferrule*[lab=Struct-equiv]
{ P \scong Q }
{ \quotep{P} \nameeq \quotep{Q} }
\end{mathpar}

The astute reader will have noticed that the mutual recursion of names
and processes imposes a mutual recursion on alpha-equivalence and
structural equivalence via name-equivalence. Fortunately, all of this
works out pleasantly and we may calculate in the natural way, free of
concern. The reader interested in the details is referred to the
appendix \ref{appendix:rho_details}.

\subsection{Substitution}

We use $\Proc$ for the set of processes, $\QProc$ for the set of
names, and $\id{\{}\vec{y} / \vec{x} \id{\}}$ to denote partial maps,
$s : \QProc \rightarrow \QProc$. A map, $s$ lifts, uniquely, to a map
on process terms, $\widehat{s} : \Proc \rightarrow \Proc$ by the
following equations.

\begin{mathpar}
  (0) \psubstp{Q}{P} := 0 \\
  (R \juxtap S) \psubstp{Q}{P}
  :=    
  (R)\psubstp{Q}{P} \juxtap (S) \psubstp{Q}{P} \\
  (x?(y).R) \psubstp{Q}{P}    
  :=    
  (x)\substp{Q}{P} (z)\concat( (R \psubstn{z}{y}) \psubstp{Q}{P} ) \\
  (\lift{x}{R}) \psubstp{Q}{P}  
  :=
  \lift{(x)\substp{Q}{P}}{ R \psubstp{Q}{P} } \\
%   (\dropn{x})  \psubstp{Q}{P}       
%   := 
%   \left\{ 
%     \begin{array}{ccc} 
%       \dropn{\quotep{Q}} & & x \nameeq \quotep{P} \\
%       \dropn{x} & & otherwise \\
%     \end{array}
%   \right. 
  (\dropn{x})  \psubstp{Q}{P}       
  := 
  \left\{ 
    \begin{array}{ccc} 
      Q & & x \nameeq \quotep{P} \\
      \dropn{x} & & otherwise \\
    \end{array}
  \right.
\end{mathpar}
 

where

\begin{eqnarray}
  (x)\id{\{} \lpquote Q \rpquote / \lpquote P \rpquote \id{\}}            = 
  \left\{ 
    \begin{array}{ccc}
      \lpquote Q \rpquote & & x \nameeq \lpquote P \rpquote \\
      x & & otherwise \\
    \end{array}
  \right. \nonumber
\end{eqnarray}

and $z$ is chosen distinct from $\quotep{P}$, $\quotep{Q}$, the free
names in $Q$, and all the names in $R$. Our $\alpha$-equivalence will
be built in the standard way from this substitution.

\begin{remark}\label{rem:no_self_referential_names}
  One consequence of these definitions is that $\forall P. \quotep{P}
  \not\in \freenames{P}$.
\end{remark}

\subsection{ Dynamic quote: an example }

Anticipating something of what's to come, consider applying the
substitution, $\widehat{\id{\{}u / z \id{\}}}$, to the following pair
of processes, $\lift{w}{y!(z)}$ and $w[ \lpquote y!(z) \rpquote ]$.

\begin{eqnarray}
	\lift{w}{y!(z)}\widehat{\id{\{}u / z \id{\}}}
		& = &
		\lift{w}{y!(u)} \nonumber\\
	w[ \lpquote y!(z) \rpquote ] \widehat{ \id{\{}u / z \id{\}} }
		& = &
		w[ \lpquote y!(z) \rpquote ] \nonumber
\end{eqnarray}

Because the body of the process between quotes is impervious to
substitution, we get radically different answers. In fact, by
examining the first process in an input context,
e.g. $x?(z).\lift{w}{y!(z)}$, we see that the process under the lift
operator may be shaped by prefixed inputs binding a name inside it. In
this sense, the lift operator will be seen as a way to dynamically
construct processes before reifying them as names.

Finally equipped with these standard features we can present the
dynamics of the calculus.

\subsubsection{Operational semantics} 

Finally, we introduce the computational dynamics. What marks these
algebras as distinct from other more traditionally studied algebraic
structures, e.g. vector spaces or polynomial rings, is the manner in
which dynamics is captured. In traditional structures, dynamics is typically
expressed through morphisms between such structures, as in linear maps
between vector spaces or morphisms between rings. In algebras
associated with the semantics of computation, the dynamics is
expressed as part of the algebraic structure itself, through a
reduction reduction relation typically denoted by $\red$. Below, we
give a recursive presentation of this relation for the calculus used
in the encoding.

$\red \subseteq \pi \times \pi$
$\red : \pi \to \mathcal{P}(\pi)$

\begin{mathpar}
  \inferrule* [lab=Comm] { \textsf{match}( x_{src}, x_{trgt} ) } { x_{trgt}?(y)P \; | \; x_{src}!\langle {Q} \rangle \red P\{\quotep{Q}/y}\} }
  \and \\
  \inferrule* [lab=Par] {{P} \red {P}'} {{{P} | {Q}} \red {{P}' | {Q}}}
  \and
  \inferrule* [lab=Equiv]{{{P} \scong {P}'} \andalso {{P}' \red {Q}'} \andalso {{Q}' \scong {Q}}}{{P} \red {Q}}
\end{mathpar}

\begin{eqnarray*}
  match_{\equiv} (\quotep{P},\quotep{Q}) & := & P \equiv Q \\
  match_{\dagger}(\quotep{P},\quotep{Q}) & := & \forall R. P|Q \red^{*} R => R \red^{*} 0 \\
  match_{K}(\quotep{P},\quotep{Q}) & := & K \mbox{ for some context } K
\end{eqnarray*}

$u?(x)P | u!\langle Q \rangle \red P\{\quotep{Q}/x\}$

%We write $\wred$ for $\red^*$, and $P\red$ if $\exists Q $ such that $ P \red Q$.
We write $P\red$ if $\exists Q $ such that $ P \red Q$ and $P\not\red$, otherwise.

\section{Replication}

As mentioned before, it is known that replication (and hence
recursion) can be implemented in a higher-order process algebra
\cite{SangiorgiWalker}. As our first example of calculation with the
machinery thus far presented we give the construction explicitly in
the {\rhoc}.

\begin{eqnarray}
	D_{x} & := & \prefix{x}{y}{(\binpar{\outputp{x}{y}}{@{y}})} \nonumber\\
	\bangp_{x}{P} & := & \binpar{{x}!\langle{\binpar{D_{x}}{P}}\rangle}{D_{x}} \nonumber
\end{eqnarray}

\begin{eqnarray}
	\bangp_{x}{P} & & \nonumber\\
	=
	& {x}!\langle{(\prefix{x}{y}{(\outputp{x}{y} | @{y})) | P}}\rangle 
	      | \prefix{x}{y}{(\outputp{x}{y} | @{y})} & \nonumber\\
	\red
	& (\outputp{x}{y} | @{y})\substn{\quotep{(\prefix{x}{y}{(@{y} | \outputp{x}{y})) | P}}}{y} & \nonumber\\
	=
	& \outputp{x}{\quotep{(\prefix{x}{y}{(\outputp{x}{y} | @{y})) | P}}}
	  | {(\prefix{x}{y}{(\outputp{x}{y} | @{y})) | P}} & \nonumber\\
	\red
	& \ldots & \nonumber\\
	\red^*
	& P | P | \ldots & \nonumber
\end{eqnarray}

Of course, this encoding, as an implementation, runs away, unfolding
$\bangp{P}$ eagerly. A lazier and more implementable replication
operator, restricted to input-guarded processes, may be obtained as follows.

\begin{eqnarray}
\bangp{\prefix{u}{v}{P}} 
	:= 
	\binpar{\lift{x}{\prefix{u}{v}{(\binpar{D(x)}{P})}}}{D(x)} \nonumber
\end{eqnarray}

\begin{remark}
  Note that the lazier definition still does not deal with summation
  or mixed summation (i.e. sums over input and output). The reader is
  invited to construct definitions of replication that deal with these
  features. 

  Further, the definitions are parameterized in a name, $x$. Can you,
  gentle reader, make a definition that eliminates this parameter and
  guarantees no accidental interaction between the replication
  machinery and the process being replicated -- i.e. no accidental
  sharing of names used by the process to get its work done and the
  name(s) used by the replication to effect copying. This latter
  revision of the definition of replication is crucial to obtaining
  the expected identity $!!P \sim !P$.
\end{remark}

\begin{remark}\label{rem:paradoxical_combinator}
  The reader familiar with the lambda calculus will have noticed the
  similarity between $D$ and the paradoxical combinator.

  [Ed. note: the existence of this seems to suggest we have to be more
  restrictive on the set of processes and names we admit if we are to
  support no-cloning.]
\end{remark}

\subsubsection{Bisimulation}

The computational dynamics gives rise to another kind of equivalence,
the equivalence of computational behavior. As previously mentioned
this is typically captured \emph{via} some form of bisimulation.

% The notion we use in this paper is weak barbed bisimulation
% \cite{milner91polyadicpi}.

The notion we use in this paper is derived from weak barbed
bisimulation \cite{milner91polyadicpi}. 

\begin{definition}
An \emph{observation relation}, $\downarrow_{\mathcal N}$, over a set
of names, $\mathcal N$, is the smallest relation satisfying the rules
below.

\infrule[Out-barb]{y \in {\mathcal N}, \; x \nameeq y}
		  {\outputp{x}{v} \downarrow_{\mathcal N} x}
\infrule[Par-barb]{\mbox{$P\downarrow_{\mathcal N} x$ or $Q\downarrow_{\mathcal N} x$}}
		  {\binpar{P}{Q} \downarrow_{\mathcal N} x}

We write $P \Downarrow_{\mathcal N} x$ if there is $Q$ such that 
$P \wred Q$ and $Q \downarrow_{\mathcal N} x$.
\end{definition}

\begin{definition}
%\label{def.bbisim}
An  ${\mathcal N}$-\emph{barbed bisimulation} over a set of names, ${\mathcal N}$, is a symmetric binary relation 
${\mathcal S}_{\mathcal N}$ between agents such that $P\rel{S}_{\mathcal N}Q$ implies:
\begin{enumerate}
\item If $P \red P'$ then $Q \wred Q'$ and $P'\rel{S}_{\mathcal N} Q'$.
\item If $P\downarrow_{\mathcal N} x$, then $Q\Downarrow_{\mathcal N} x$.
\end{enumerate}
$P$ is ${\mathcal N}$-barbed bisimilar to $Q$, written
$P \wbbisim_{\mathcal N} Q$, if $P \rel{S}_{\mathcal N} Q$ for some ${\mathcal N}$-barbed bisimulation ${\mathcal S}_{\mathcal N}$.
\end{definition}

$\mathcal{R} \subseteq \pi \times \pi$

$P \mathcal{R} Q => \forall P'. P \red P' \Rightarrow \exists Q'. Q \red Q', P' \mathcal{R} Q'$

$P \vdash x \Rightarrow Q \vdash x$

\begin{mathpar}
  \inferrule*[lab=Out-barb]{x \nameeq y}{{y}!\langle{Q}\rangle \vdash x}
  \and
  \inferrule*[lab=Par-barb]{\mbox{$P\vdash x$ or $Q\vdash x$}}{\binpar{P}{Q} \vdash x}
\end{mathpar}

\subsubsection{Contexts}

One of the principle advantages of computational calculi like the
$\pi$-calculus is a well-defined notion of context,
contextual-equivalence and a correlation between
contextual-equivalence and notions of bisimulation. The notion of
context allows the decomposition of a process into (sub-)process and
its syntactic environment, its context. Thus, a context may be
thought of as a process with a ``hole'' (written $\Box$) in it. The
application of a context $M$ to a process $P$, written $M[P]$, is
tantamount to filling the hole in $M$ with $P$. In this paper we do
not need the full weight of this theory, but do make use of the notion
of context in the proof the main theorem. 

\begin{mathpar}
  \inferrule* [lab=summation] {} {{M_{M},M_{N}} \bc \Box \;|\; x.M_{A} \;|\; M_{M}+M_{N}}
  \and
  \inferrule* [lab=agent] {} {{M_{A}} \bc (\vec{x})M_{P} \;| \; \clift{P_0,\ldots,M_{P},\ldots,P_N}}
  \and \\
  \inferrule* [lab=process] {} {{M_{P}} \bc M_{N} \;| \;P|M_{P} }
\end{mathpar} 

\begin{mathpar}
  \inferrule* [lab=sychronization] {} {M_{N} \bc \Box \;|\; x?M_{F} \;|\; x!M_{C}}
  \and
  \inferrule* [lab=abstraction] {} {{M_{F}} \bc (x)M_{P} }
  \and
  \inferrule* [lab=concretion] {} {{M_{C}} \bc \langle M_{P} \rangle }
  \and \\
  \inferrule* [lab=process] {} {{M_{P}} \bc M_{N} \;| \;P|M_{P} }
\end{mathpar}

\begin{definition}[contextual application] Given a context $M$, and
  process $P$, we define the \emph{contextual application}, $M[P] :=
  M\{P/\Box\}$. That is, the contextual application of M to P is the
  substitution of $P$ for $\Box$ in $M$.
\end{definition}

$\meaningof{-} : L \to \mathcal{P}(\pi)$

\begin{mathpar}
  \inferrule* [lab=collection] {} {\meaningof{true} = \pi, \and \meaningof{~E} = \pi \setminus \meaningof{E}, \and \meaningof{E_{1} \& E_{2}} = \meaningof{E_{1}} \cap \meaningof{E_{2}}}
\end{mathpar}

\begin{mathpar}
  \inferrule* [lab=structure] {} {\meaningof{0} = \{ P \in \pi | P \equiv 0 \}, \and \\ \meaningof{E_1 | E_2} = \{ P \in \pi | P \equiv P_{1} | P_{2}, P_{1} \in \meaningof{E_{1}}, P_{2} \in \meaningof{E_2}\} }
\end{mathpar}

\begin{mathpar}
 \inferrule* [lab=behavior] {} {\meaningof{\langle a?b \rangle E} = \{ P \in \pi | P \equiv Q | u?(y)P', \\ \and \\\\ \and \\ \;\;\; u \in \meaningof{a}, \forall z.P'\{z/y\} \in \meaningof{E\{z/b\}}\}, \and \\ \meaningof{a!E} = \{ P \in \pi | P \equiv Q | x!\langle P' \rangle, x \in \meaningof{a} P' \in \meaningof{E}\} }
\end{mathpar}

\begin{mathpar}
 \inferrule* [lab=nominal] {} {\meaningof{\quotep{E}} = \{ \quotep{P} \in \quotep{\pi} | P \in \meaningof{E} \}, \and \meaningof{\quotep{P}} = \{ \quotep{Q} \in \quotep{\pi} | P \equiv Q \} \and \\ \meaningof{@\quotep{E}} = \{ P \in \pi | P \equiv @x, x \in \meaningof{E} \}}
\end{mathpar}

\begin{eqnarray*}
  \\
  \meaningof{-} : TS \to ST
\end{eqnarray*}

\begin{eqnarray*}
  \\
  L : TS \to ST
\end{eqnarray*}

\begin{eqnarray*}
  \\
  P \models E \iff P \in \meaningof{E}
\end{eqnarray*}

\begin{eqnarray*}
  P \approx_{L} Q \iff \forall E \in L. P \models E \iff Q \models E
\end{eqnarray*}

\begin{eqnarray*}
  P \approx_{K} Q
\end{eqnarray*}

\begin{eqnarray*}
  P \approx Q
\end{eqnarray*}

$\approx_{K} = \approx = \approx_{L}$

\subsubsection{Contextual duality}

Note that contexts extend the quotation operation to a family of
operations from processes to names. Given a context, $M$, we can
define a \emph{nominal context}, $\quotep{M}$ by $\quotep{M}[P] :=
\quotep{M[P]}$. To foreshadow what is to come we observe that these
operations enjoy a duality with processes very much like the duality
between vectors and maps from vectors to scalars.

Further, because the calculus is essentially higher-order, we have a
correspondence between contexts and processes. More specifically,
given a name $x$ and a context $M$ we can construct $M^{*}_{x}$ such
that 

\begin{mathpar}
  M^{*}_{x} | \lift{x}{P} \red M[P]
\end{mathpar}

namely,

\begin{mathpar}
  M^{*}_{x} := x?(u).M[\dropn{u}]
\end{mathpar}

The dependence of $M^{*}_{x}$ on a name makes it an abstraction, 

\begin{mathpar}
  M^{*} := (x)x?(u).M[\dropn{u}]
\end{mathpar}

\subsection{Additional notation}

It will sometimes be convenient to denote the process a name
quotes. We already have the notation $x = \quotep{P}$, but it will be
convenient to introduce an alternate notation, $\procn{x}$, when we
want to emphasize the connection to the use of the name. Note that, by
virtue of name equivalence, $\quotep{\procn{x}} \nameeq x$; so, the
notation is consistent with previous definitions.

Further, because names have structure it is possible to effect
substitutions on the basis of that structure. This means we need to
upgrade our notation for substitutions, which we accomplish by
adapting comprehension notation. Thus,

\begin{mathpar}
  P\{ y / x : x \in S \}
\end{mathpar}

is interpreted to mean the process derived from P by replacing (in a
capture-avoiding manner) each occurrence of $x$ in $S$ by $y$. For example,

\begin{mathpar}
  P\{ \quotep{\procn{x}|\procn{x}} / x : x \in \freenames{P} \}
\end{mathpar}

will replace each (occurrence) of a free name $x$ in $P$ by
$\quotep{\procn{x}|\procn{x}}$.

Also, we will avail ourselves of the notation $x^{L}$ and $x^{R}$ to
denote injections of a name into disjoint copies of the name
space. There are numerous ways to accomplish this. One example can be
found in \cite{MeredithR05}. This notation overloads to vectors of
names: $\vec{x}^{\pi} := (x_{i}^{\pi} \; : \; 0 \leq i < |\vec{x}| )$ where $\pi \in \{L,R\}$.

We also use $P^{\Box} := P|\Box$.

In \cite{MeredithR05} an interpretation of the new operator is
given. It turns out that there are several possible interpretations
all enjoying the requisite algebraic properties of the operator (see
\cite{milner91polyadicpi}). We will therefore make liberal use of
$(\nu\; \vec{x})P$.

% subsection the_syntax_and_semantics_of_the_notation_system (end)   

\input{qm2pi.qmops} 

\input{qm2pi.sterngerlach} 

\input{qm2pi.metric} 

% section concurrent_process_calculi (end)

%\input{qm2pi.proofsketch}

% section proof sketch (end)

%\input{qm2pi.slviaknots} 

% section spatial logic via knots (end)

\input{qm2pi.conclusion}

% section conclusion (end)

%\input{qm2pi.dtcodes} 

% section wiring algorithm (end)

\input{qm2pi.ack} 

% section acknowledgments (end)

\newpage


\bibliographystyle{plain}   
\bibliography{../../biblios/main.bib}

\input{qm2pi.rhodetails}

\end{document}



\end{document}

 

% section acknowledgments (end)

\newpage


\bibliographystyle{plain}   
\bibliography{../../biblios/main.bib}

\documentclass[12pt]{llncs}
%\documentclass{jktr}

\usepackage[pdftex]{hyperref}                   
\usepackage {listings}
\usepackage {mathpartir}
\usepackage{bcprules}
%\usepackage{listings}
                       
\usepackage{graphicx} 
%\usepackage[margins=2.5cm,nohead,nofoot]{geometry}
%\usepackage{geometry}
\usepackage{amsfonts}
\usepackage{amstext}
\usepackage{latexsym}
\usepackage{amssymb}
\usepackage{color}


%\include{myPreamble}
\documentclass[12pt]{llncs}
%\documentclass{jktr}

\usepackage[pdftex]{hyperref}                   
\usepackage {listings}
\usepackage {mathpartir}
\usepackage{bcprules}
%\usepackage{listings}
                       
\usepackage{graphicx} 
%\usepackage[margins=2.5cm,nohead,nofoot]{geometry}
%\usepackage{geometry}
\usepackage{amsfonts}
\usepackage{amstext}
\usepackage{latexsym}
\usepackage{amssymb}
\usepackage{color}


%\include{myPreamble}
\include{qm2pi.local} 

%\ifpdf
%\usepackage[pdftex]{graphicx}
%\else
%\usepackage{graphicx}
%\fi

 % \ifpdf
%  \usepackage{pdfsync}
%  \if


%\title{Brief Article}
%\author{David F. Snyder}
%\author{L.G. Meredith}

%\address{Dept. of Math., Texas State University--San Marcos, San Marcos, TX 78666}
       
\pagestyle{empty}


\begin{document}

\lstset{language=[Objective]Caml,frame=shadowbox}

\input{qm2pi.front}

% section front matter (end)

\input{qm2pi.intro} 
 
% section introduction (end)

% \input{qm2pi.knotations} 

% section notation (end)

\input{qm2pi.process.calculi} 

% section concurrent_process_calculi_and_spatial_logics_ (end)
    
%\input{qm2pi.knots2pi} 

%\input{qm2pi.trefoil} 

%\input{qm2pi.mainthm} 

% subsection basic_interpretation (end)

%\input{qm2pi.rho.presentation} 
\subsection{The syntax and semantics of the notation system}\label{sub:the_syntax_and_semantics_of_the_notation_system} % (fold)

We now summarize a technical presentation of the calculus that
embodies our theory of dynamics. The typical presentation of such a
calculus follows the style of giving generators and relations on
them. The grammar, below, describing term constructors, freely
generates the set of processes, $\Proc$. This set is then quotiented
by a relation known as structural congruence and it is over this set
that the notion of dynamics is expressed. This presentation is
essentially that of \cite{MeredithR05} with the addition of
polyadicity and summation. For readability we have relegated some of
the technical subtleties to an appendix.

\subsubsection{Process grammar}\label{subsub:process_grammar}

\begin{mathpar}
  \inferrule* [lab=synchronization] {} {{M} \bc \pzero \;|\; x?F \;|\; x!C }
  \and
  \inferrule* [lab=abstraction] {} {{F} \bc (x)P}
  \and
  \inferrule* [lab=concretion] {} {{C} \bc \langle Q \rangle}
  \and
  \inferrule* [lab=process] {} {{P,Q} \bc M \;| \;P|Q \;|\; @{x}}
  \and
  \inferrule* [lab=name] {} {{x} \bc \quotep{P}}
\end{mathpar} 

Note that $\vec{x}$ (resp. $\vec{P}$) denotes a vector of names
(resp. processes) of length $|\vec{x}|$ (resp. $|\vec{P}|$). We adopt
the following useful abbreviations.

\begin{mathpar}
   x?(\vec{y}).P := x.(\vec{y})P \and  x\clift{\vec{P}} := x.\clift{\vec{P}}
   \and x!(y) := \lift{x}{\dropn{y}}
   \and \Pi_{i=0}^{n-1}P_i := P_0 | \ldots | P_{n-1}
\end{mathpar}

\subsubsection{Structural congruence}

\paragraph{Free and bound names and alpha-equivalence.} At the
core of structural equivalence is alpha-equivalence which identifies
process that are the same up to a change of variable. Formally, we
recognize the distinction between free and bound names. The free names
of a process, $\freenames{P}$, may be calculated recursively as
follows:

\begin{mathpar}
\freenames{\pzero} := \emptyset
  \and \\
  \freenames{x?(y).P} := \{ x \} \cup (\freenames{P} \setminus \{ y \})
  \and 
  \freenames{x!\langle P \rangle} := \{ x \} \cup \{ P \} 
  \and \\
  \freenames{P|Q} := \freenames{P} \cup \freenames{Q}
  \and \\
  \freenames{@{x}} := \{ x \}
\end{mathpar}

$\pi$
$\quotep{\pi}$

$\freenames{-} : \pi \to \mathcal{P}(\quotep{\pi})$

\begin{eqnarray*}
  \freenames{\pzero} & := & \emptyset \\
  \freenames{x?(y).P} & := & \{ x \} \cup (\freenames{P} \setminus \{ y \}) \\
  \freenames{x!\langle P \rangle} & := & \{ x \} \cup \{ P \} \\
  \freenames{P|Q} & := & \freenames{P} \cup \freenames{Q} \\
  \freenames{\dropn{x}} & := & \{ x \}
\end{eqnarray*}

The bound names of a process, $\boundnames{P}$, are those names occurring in $P$
that are not free. For example, in $x?(y).0$, the name $x$ is free, while $y$ is bound.

\begin{mathpar}
  \inferrule* [lab=monoidal-laws] {} { P|Q \equiv Q|P \and P|0 \equiv P \and P|(Q|R) \equiv (P|Q)|R }
\end{mathpar}

\begin{mathpar}
  \inferrule* [lab=alpha-equivalence] {} { (x)P \equiv (y)P\{y/x\} \and y \not\in \freenames{P} }
\end{mathpar}

\begin{definition}
Then two processes, $P,Q$, are alpha-equivalent if $P = Q\{\vec{y}/\vec{x}\}$ for
some $\vec{x} \in \boundnames{Q},\vec{y} \in \boundnames{P}$, where $Q\{\vec{y}/\vec{x}\}$
denotes the capture-avoiding substitution of $\vec{y}$ for $\vec{x}$ in $Q$.
\end{definition}

\begin{definition}
  The {\em structural congruence} \cite{SangiorgiWalker} , $\equiv$,
  between processes is the least congruence containing
  alpha-equivalence, satisfying the abelian monoid laws
  (associativity, commutativity and $\pzero$ as identity) for parallel
  composition $|$ and for summation $+$.
\end{definition}

\subsection{Name equivalence}

We take name equivalence, written $\nameeq$, to be the smallest
equivalence relation generated by the following rules.

\begin{mathpar}
\inferrule*[lab=Quote-drop]
{ }
{ \quotep{@{x}} \nameeq x }

\inferrule*[lab=Struct-equiv]
{ P \scong Q }
{ \quotep{P} \nameeq \quotep{Q} }
\end{mathpar}

The astute reader will have noticed that the mutual recursion of names
and processes imposes a mutual recursion on alpha-equivalence and
structural equivalence via name-equivalence. Fortunately, all of this
works out pleasantly and we may calculate in the natural way, free of
concern. The reader interested in the details is referred to the
appendix \ref{appendix:rho_details}.

\subsection{Substitution}

We use $\Proc$ for the set of processes, $\QProc$ for the set of
names, and $\id{\{}\vec{y} / \vec{x} \id{\}}$ to denote partial maps,
$s : \QProc \rightarrow \QProc$. A map, $s$ lifts, uniquely, to a map
on process terms, $\widehat{s} : \Proc \rightarrow \Proc$ by the
following equations.

\begin{mathpar}
  (0) \psubstp{Q}{P} := 0 \\
  (R \juxtap S) \psubstp{Q}{P}
  :=    
  (R)\psubstp{Q}{P} \juxtap (S) \psubstp{Q}{P} \\
  (x?(y).R) \psubstp{Q}{P}    
  :=    
  (x)\substp{Q}{P} (z)\concat( (R \psubstn{z}{y}) \psubstp{Q}{P} ) \\
  (\lift{x}{R}) \psubstp{Q}{P}  
  :=
  \lift{(x)\substp{Q}{P}}{ R \psubstp{Q}{P} } \\
%   (\dropn{x})  \psubstp{Q}{P}       
%   := 
%   \left\{ 
%     \begin{array}{ccc} 
%       \dropn{\quotep{Q}} & & x \nameeq \quotep{P} \\
%       \dropn{x} & & otherwise \\
%     \end{array}
%   \right. 
  (\dropn{x})  \psubstp{Q}{P}       
  := 
  \left\{ 
    \begin{array}{ccc} 
      Q & & x \nameeq \quotep{P} \\
      \dropn{x} & & otherwise \\
    \end{array}
  \right.
\end{mathpar}
 

where

\begin{eqnarray}
  (x)\id{\{} \lpquote Q \rpquote / \lpquote P \rpquote \id{\}}            = 
  \left\{ 
    \begin{array}{ccc}
      \lpquote Q \rpquote & & x \nameeq \lpquote P \rpquote \\
      x & & otherwise \\
    \end{array}
  \right. \nonumber
\end{eqnarray}

and $z$ is chosen distinct from $\quotep{P}$, $\quotep{Q}$, the free
names in $Q$, and all the names in $R$. Our $\alpha$-equivalence will
be built in the standard way from this substitution.

\begin{remark}\label{rem:no_self_referential_names}
  One consequence of these definitions is that $\forall P. \quotep{P}
  \not\in \freenames{P}$.
\end{remark}

\subsection{ Dynamic quote: an example }

Anticipating something of what's to come, consider applying the
substitution, $\widehat{\id{\{}u / z \id{\}}}$, to the following pair
of processes, $\lift{w}{y!(z)}$ and $w[ \lpquote y!(z) \rpquote ]$.

\begin{eqnarray}
	\lift{w}{y!(z)}\widehat{\id{\{}u / z \id{\}}}
		& = &
		\lift{w}{y!(u)} \nonumber\\
	w[ \lpquote y!(z) \rpquote ] \widehat{ \id{\{}u / z \id{\}} }
		& = &
		w[ \lpquote y!(z) \rpquote ] \nonumber
\end{eqnarray}

Because the body of the process between quotes is impervious to
substitution, we get radically different answers. In fact, by
examining the first process in an input context,
e.g. $x?(z).\lift{w}{y!(z)}$, we see that the process under the lift
operator may be shaped by prefixed inputs binding a name inside it. In
this sense, the lift operator will be seen as a way to dynamically
construct processes before reifying them as names.

Finally equipped with these standard features we can present the
dynamics of the calculus.

\subsubsection{Operational semantics} 

Finally, we introduce the computational dynamics. What marks these
algebras as distinct from other more traditionally studied algebraic
structures, e.g. vector spaces or polynomial rings, is the manner in
which dynamics is captured. In traditional structures, dynamics is typically
expressed through morphisms between such structures, as in linear maps
between vector spaces or morphisms between rings. In algebras
associated with the semantics of computation, the dynamics is
expressed as part of the algebraic structure itself, through a
reduction reduction relation typically denoted by $\red$. Below, we
give a recursive presentation of this relation for the calculus used
in the encoding.

$\red \subseteq \pi \times \pi$
$\red : \pi \to \mathcal{P}(\pi)$

\begin{mathpar}
  \inferrule* [lab=Comm] { \textsf{match}( x_{src}, x_{trgt} ) } { x_{trgt}?(y)P \; | \; x_{src}!\langle {Q} \rangle \red P\{\quotep{Q}/y}\} }
  \and \\
  \inferrule* [lab=Par] {{P} \red {P}'} {{{P} | {Q}} \red {{P}' | {Q}}}
  \and
  \inferrule* [lab=Equiv]{{{P} \scong {P}'} \andalso {{P}' \red {Q}'} \andalso {{Q}' \scong {Q}}}{{P} \red {Q}}
\end{mathpar}

\begin{eqnarray*}
  match_{\equiv} (\quotep{P},\quotep{Q}) & := & P \equiv Q \\
  match_{\dagger}(\quotep{P},\quotep{Q}) & := & \forall R. P|Q \red^{*} R => R \red^{*} 0 \\
  match_{K}(\quotep{P},\quotep{Q}) & := & K \mbox{ for some context } K
\end{eqnarray*}

$u?(x)P | u!\langle Q \rangle \red P\{\quotep{Q}/x\}$

%We write $\wred$ for $\red^*$, and $P\red$ if $\exists Q $ such that $ P \red Q$.
We write $P\red$ if $\exists Q $ such that $ P \red Q$ and $P\not\red$, otherwise.

\section{Replication}

As mentioned before, it is known that replication (and hence
recursion) can be implemented in a higher-order process algebra
\cite{SangiorgiWalker}. As our first example of calculation with the
machinery thus far presented we give the construction explicitly in
the {\rhoc}.

\begin{eqnarray}
	D_{x} & := & \prefix{x}{y}{(\binpar{\outputp{x}{y}}{@{y}})} \nonumber\\
	\bangp_{x}{P} & := & \binpar{{x}!\langle{\binpar{D_{x}}{P}}\rangle}{D_{x}} \nonumber
\end{eqnarray}

\begin{eqnarray}
	\bangp_{x}{P} & & \nonumber\\
	=
	& {x}!\langle{(\prefix{x}{y}{(\outputp{x}{y} | @{y})) | P}}\rangle 
	      | \prefix{x}{y}{(\outputp{x}{y} | @{y})} & \nonumber\\
	\red
	& (\outputp{x}{y} | @{y})\substn{\quotep{(\prefix{x}{y}{(@{y} | \outputp{x}{y})) | P}}}{y} & \nonumber\\
	=
	& \outputp{x}{\quotep{(\prefix{x}{y}{(\outputp{x}{y} | @{y})) | P}}}
	  | {(\prefix{x}{y}{(\outputp{x}{y} | @{y})) | P}} & \nonumber\\
	\red
	& \ldots & \nonumber\\
	\red^*
	& P | P | \ldots & \nonumber
\end{eqnarray}

Of course, this encoding, as an implementation, runs away, unfolding
$\bangp{P}$ eagerly. A lazier and more implementable replication
operator, restricted to input-guarded processes, may be obtained as follows.

\begin{eqnarray}
\bangp{\prefix{u}{v}{P}} 
	:= 
	\binpar{\lift{x}{\prefix{u}{v}{(\binpar{D(x)}{P})}}}{D(x)} \nonumber
\end{eqnarray}

\begin{remark}
  Note that the lazier definition still does not deal with summation
  or mixed summation (i.e. sums over input and output). The reader is
  invited to construct definitions of replication that deal with these
  features. 

  Further, the definitions are parameterized in a name, $x$. Can you,
  gentle reader, make a definition that eliminates this parameter and
  guarantees no accidental interaction between the replication
  machinery and the process being replicated -- i.e. no accidental
  sharing of names used by the process to get its work done and the
  name(s) used by the replication to effect copying. This latter
  revision of the definition of replication is crucial to obtaining
  the expected identity $!!P \sim !P$.
\end{remark}

\begin{remark}\label{rem:paradoxical_combinator}
  The reader familiar with the lambda calculus will have noticed the
  similarity between $D$ and the paradoxical combinator.

  [Ed. note: the existence of this seems to suggest we have to be more
  restrictive on the set of processes and names we admit if we are to
  support no-cloning.]
\end{remark}

\subsubsection{Bisimulation}

The computational dynamics gives rise to another kind of equivalence,
the equivalence of computational behavior. As previously mentioned
this is typically captured \emph{via} some form of bisimulation.

% The notion we use in this paper is weak barbed bisimulation
% \cite{milner91polyadicpi}.

The notion we use in this paper is derived from weak barbed
bisimulation \cite{milner91polyadicpi}. 

\begin{definition}
An \emph{observation relation}, $\downarrow_{\mathcal N}$, over a set
of names, $\mathcal N$, is the smallest relation satisfying the rules
below.

\infrule[Out-barb]{y \in {\mathcal N}, \; x \nameeq y}
		  {\outputp{x}{v} \downarrow_{\mathcal N} x}
\infrule[Par-barb]{\mbox{$P\downarrow_{\mathcal N} x$ or $Q\downarrow_{\mathcal N} x$}}
		  {\binpar{P}{Q} \downarrow_{\mathcal N} x}

We write $P \Downarrow_{\mathcal N} x$ if there is $Q$ such that 
$P \wred Q$ and $Q \downarrow_{\mathcal N} x$.
\end{definition}

\begin{definition}
%\label{def.bbisim}
An  ${\mathcal N}$-\emph{barbed bisimulation} over a set of names, ${\mathcal N}$, is a symmetric binary relation 
${\mathcal S}_{\mathcal N}$ between agents such that $P\rel{S}_{\mathcal N}Q$ implies:
\begin{enumerate}
\item If $P \red P'$ then $Q \wred Q'$ and $P'\rel{S}_{\mathcal N} Q'$.
\item If $P\downarrow_{\mathcal N} x$, then $Q\Downarrow_{\mathcal N} x$.
\end{enumerate}
$P$ is ${\mathcal N}$-barbed bisimilar to $Q$, written
$P \wbbisim_{\mathcal N} Q$, if $P \rel{S}_{\mathcal N} Q$ for some ${\mathcal N}$-barbed bisimulation ${\mathcal S}_{\mathcal N}$.
\end{definition}

$\mathcal{R} \subseteq \pi \times \pi$

$P \mathcal{R} Q => \forall P'. P \red P' \Rightarrow \exists Q'. Q \red Q', P' \mathcal{R} Q'$

$P \vdash x \Rightarrow Q \vdash x$

\begin{mathpar}
  \inferrule*[lab=Out-barb]{x \nameeq y}{{y}!\langle{Q}\rangle \vdash x}
  \and
  \inferrule*[lab=Par-barb]{\mbox{$P\vdash x$ or $Q\vdash x$}}{\binpar{P}{Q} \vdash x}
\end{mathpar}

\subsubsection{Contexts}

One of the principle advantages of computational calculi like the
$\pi$-calculus is a well-defined notion of context,
contextual-equivalence and a correlation between
contextual-equivalence and notions of bisimulation. The notion of
context allows the decomposition of a process into (sub-)process and
its syntactic environment, its context. Thus, a context may be
thought of as a process with a ``hole'' (written $\Box$) in it. The
application of a context $M$ to a process $P$, written $M[P]$, is
tantamount to filling the hole in $M$ with $P$. In this paper we do
not need the full weight of this theory, but do make use of the notion
of context in the proof the main theorem. 

\begin{mathpar}
  \inferrule* [lab=summation] {} {{M_{M},M_{N}} \bc \Box \;|\; x.M_{A} \;|\; M_{M}+M_{N}}
  \and
  \inferrule* [lab=agent] {} {{M_{A}} \bc (\vec{x})M_{P} \;| \; \clift{P_0,\ldots,M_{P},\ldots,P_N}}
  \and \\
  \inferrule* [lab=process] {} {{M_{P}} \bc M_{N} \;| \;P|M_{P} }
\end{mathpar} 

\begin{mathpar}
  \inferrule* [lab=sychronization] {} {M_{N} \bc \Box \;|\; x?M_{F} \;|\; x!M_{C}}
  \and
  \inferrule* [lab=abstraction] {} {{M_{F}} \bc (x)M_{P} }
  \and
  \inferrule* [lab=concretion] {} {{M_{C}} \bc \langle M_{P} \rangle }
  \and \\
  \inferrule* [lab=process] {} {{M_{P}} \bc M_{N} \;| \;P|M_{P} }
\end{mathpar}

\begin{definition}[contextual application] Given a context $M$, and
  process $P$, we define the \emph{contextual application}, $M[P] :=
  M\{P/\Box\}$. That is, the contextual application of M to P is the
  substitution of $P$ for $\Box$ in $M$.
\end{definition}

$\meaningof{-} : L \to \mathcal{P}(\pi)$

\begin{mathpar}
  \inferrule* [lab=collection] {} {\meaningof{true} = \pi, \and \meaningof{~E} = \pi \setminus \meaningof{E}, \and \meaningof{E_{1} \& E_{2}} = \meaningof{E_{1}} \cap \meaningof{E_{2}}}
\end{mathpar}

\begin{mathpar}
  \inferrule* [lab=structure] {} {\meaningof{0} = \{ P \in \pi | P \equiv 0 \}, \and \\ \meaningof{E_1 | E_2} = \{ P \in \pi | P \equiv P_{1} | P_{2}, P_{1} \in \meaningof{E_{1}}, P_{2} \in \meaningof{E_2}\} }
\end{mathpar}

\begin{mathpar}
 \inferrule* [lab=behavior] {} {\meaningof{\langle a?b \rangle E} = \{ P \in \pi | P \equiv Q | u?(y)P', \\ \and \\\\ \and \\ \;\;\; u \in \meaningof{a}, \forall z.P'\{z/y\} \in \meaningof{E\{z/b\}}\}, \and \\ \meaningof{a!E} = \{ P \in \pi | P \equiv Q | x!\langle P' \rangle, x \in \meaningof{a} P' \in \meaningof{E}\} }
\end{mathpar}

\begin{mathpar}
 \inferrule* [lab=nominal] {} {\meaningof{\quotep{E}} = \{ \quotep{P} \in \quotep{\pi} | P \in \meaningof{E} \}, \and \meaningof{\quotep{P}} = \{ \quotep{Q} \in \quotep{\pi} | P \equiv Q \} \and \\ \meaningof{@\quotep{E}} = \{ P \in \pi | P \equiv @x, x \in \meaningof{E} \}}
\end{mathpar}

\begin{eqnarray*}
  \\
  \meaningof{-} : TS \to ST
\end{eqnarray*}

\begin{eqnarray*}
  \\
  L : TS \to ST
\end{eqnarray*}

\begin{eqnarray*}
  \\
  P \models E \iff P \in \meaningof{E}
\end{eqnarray*}

\begin{eqnarray*}
  P \approx_{L} Q \iff \forall E \in L. P \models E \iff Q \models E
\end{eqnarray*}

\begin{eqnarray*}
  P \approx_{K} Q
\end{eqnarray*}

\begin{eqnarray*}
  P \approx Q
\end{eqnarray*}

$\approx_{K} = \approx = \approx_{L}$

\subsubsection{Contextual duality}

Note that contexts extend the quotation operation to a family of
operations from processes to names. Given a context, $M$, we can
define a \emph{nominal context}, $\quotep{M}$ by $\quotep{M}[P] :=
\quotep{M[P]}$. To foreshadow what is to come we observe that these
operations enjoy a duality with processes very much like the duality
between vectors and maps from vectors to scalars.

Further, because the calculus is essentially higher-order, we have a
correspondence between contexts and processes. More specifically,
given a name $x$ and a context $M$ we can construct $M^{*}_{x}$ such
that 

\begin{mathpar}
  M^{*}_{x} | \lift{x}{P} \red M[P]
\end{mathpar}

namely,

\begin{mathpar}
  M^{*}_{x} := x?(u).M[\dropn{u}]
\end{mathpar}

The dependence of $M^{*}_{x}$ on a name makes it an abstraction, 

\begin{mathpar}
  M^{*} := (x)x?(u).M[\dropn{u}]
\end{mathpar}

\subsection{Additional notation}

It will sometimes be convenient to denote the process a name
quotes. We already have the notation $x = \quotep{P}$, but it will be
convenient to introduce an alternate notation, $\procn{x}$, when we
want to emphasize the connection to the use of the name. Note that, by
virtue of name equivalence, $\quotep{\procn{x}} \nameeq x$; so, the
notation is consistent with previous definitions.

Further, because names have structure it is possible to effect
substitutions on the basis of that structure. This means we need to
upgrade our notation for substitutions, which we accomplish by
adapting comprehension notation. Thus,

\begin{mathpar}
  P\{ y / x : x \in S \}
\end{mathpar}

is interpreted to mean the process derived from P by replacing (in a
capture-avoiding manner) each occurrence of $x$ in $S$ by $y$. For example,

\begin{mathpar}
  P\{ \quotep{\procn{x}|\procn{x}} / x : x \in \freenames{P} \}
\end{mathpar}

will replace each (occurrence) of a free name $x$ in $P$ by
$\quotep{\procn{x}|\procn{x}}$.

Also, we will avail ourselves of the notation $x^{L}$ and $x^{R}$ to
denote injections of a name into disjoint copies of the name
space. There are numerous ways to accomplish this. One example can be
found in \cite{MeredithR05}. This notation overloads to vectors of
names: $\vec{x}^{\pi} := (x_{i}^{\pi} \; : \; 0 \leq i < |\vec{x}| )$ where $\pi \in \{L,R\}$.

We also use $P^{\Box} := P|\Box$.

In \cite{MeredithR05} an interpretation of the new operator is
given. It turns out that there are several possible interpretations
all enjoying the requisite algebraic properties of the operator (see
\cite{milner91polyadicpi}). We will therefore make liberal use of
$(\nu\; \vec{x})P$.

% subsection the_syntax_and_semantics_of_the_notation_system (end)   

\input{qm2pi.qmops} 

\input{qm2pi.sterngerlach} 

\input{qm2pi.metric} 

% section concurrent_process_calculi (end)

%\input{qm2pi.proofsketch}

% section proof sketch (end)

%\input{qm2pi.slviaknots} 

% section spatial logic via knots (end)

\input{qm2pi.conclusion}

% section conclusion (end)

%\input{qm2pi.dtcodes} 

% section wiring algorithm (end)

\input{qm2pi.ack} 

% section acknowledgments (end)

\newpage


\bibliographystyle{plain}   
\bibliography{../../biblios/main.bib}

\input{qm2pi.rhodetails}

\end{document}

 

%\ifpdf
%\usepackage[pdftex]{graphicx}
%\else
%\usepackage{graphicx}
%\fi

 % \ifpdf
%  \usepackage{pdfsync}
%  \if


%\title{Brief Article}
%\author{David F. Snyder}
%\author{L.G. Meredith}

%\address{Dept. of Math., Texas State University--San Marcos, San Marcos, TX 78666}
       
\pagestyle{empty}


\begin{document}

\lstset{language=[Objective]Caml,frame=shadowbox}

\documentclass[12pt]{llncs}
%\documentclass{jktr}

\usepackage[pdftex]{hyperref}                   
\usepackage {listings}
\usepackage {mathpartir}
\usepackage{bcprules}
%\usepackage{listings}
                       
\usepackage{graphicx} 
%\usepackage[margins=2.5cm,nohead,nofoot]{geometry}
%\usepackage{geometry}
\usepackage{amsfonts}
\usepackage{amstext}
\usepackage{latexsym}
\usepackage{amssymb}
\usepackage{color}


%\include{myPreamble}
\include{qm2pi.local} 

%\ifpdf
%\usepackage[pdftex]{graphicx}
%\else
%\usepackage{graphicx}
%\fi

 % \ifpdf
%  \usepackage{pdfsync}
%  \if


%\title{Brief Article}
%\author{David F. Snyder}
%\author{L.G. Meredith}

%\address{Dept. of Math., Texas State University--San Marcos, San Marcos, TX 78666}
       
\pagestyle{empty}


\begin{document}

\lstset{language=[Objective]Caml,frame=shadowbox}

\input{qm2pi.front}

% section front matter (end)

\input{qm2pi.intro} 
 
% section introduction (end)

% \input{qm2pi.knotations} 

% section notation (end)

\input{qm2pi.process.calculi} 

% section concurrent_process_calculi_and_spatial_logics_ (end)
    
%\input{qm2pi.knots2pi} 

%\input{qm2pi.trefoil} 

%\input{qm2pi.mainthm} 

% subsection basic_interpretation (end)

%\input{qm2pi.rho.presentation} 
\subsection{The syntax and semantics of the notation system}\label{sub:the_syntax_and_semantics_of_the_notation_system} % (fold)

We now summarize a technical presentation of the calculus that
embodies our theory of dynamics. The typical presentation of such a
calculus follows the style of giving generators and relations on
them. The grammar, below, describing term constructors, freely
generates the set of processes, $\Proc$. This set is then quotiented
by a relation known as structural congruence and it is over this set
that the notion of dynamics is expressed. This presentation is
essentially that of \cite{MeredithR05} with the addition of
polyadicity and summation. For readability we have relegated some of
the technical subtleties to an appendix.

\subsubsection{Process grammar}\label{subsub:process_grammar}

\begin{mathpar}
  \inferrule* [lab=synchronization] {} {{M} \bc \pzero \;|\; x?F \;|\; x!C }
  \and
  \inferrule* [lab=abstraction] {} {{F} \bc (x)P}
  \and
  \inferrule* [lab=concretion] {} {{C} \bc \langle Q \rangle}
  \and
  \inferrule* [lab=process] {} {{P,Q} \bc M \;| \;P|Q \;|\; @{x}}
  \and
  \inferrule* [lab=name] {} {{x} \bc \quotep{P}}
\end{mathpar} 

Note that $\vec{x}$ (resp. $\vec{P}$) denotes a vector of names
(resp. processes) of length $|\vec{x}|$ (resp. $|\vec{P}|$). We adopt
the following useful abbreviations.

\begin{mathpar}
   x?(\vec{y}).P := x.(\vec{y})P \and  x\clift{\vec{P}} := x.\clift{\vec{P}}
   \and x!(y) := \lift{x}{\dropn{y}}
   \and \Pi_{i=0}^{n-1}P_i := P_0 | \ldots | P_{n-1}
\end{mathpar}

\subsubsection{Structural congruence}

\paragraph{Free and bound names and alpha-equivalence.} At the
core of structural equivalence is alpha-equivalence which identifies
process that are the same up to a change of variable. Formally, we
recognize the distinction between free and bound names. The free names
of a process, $\freenames{P}$, may be calculated recursively as
follows:

\begin{mathpar}
\freenames{\pzero} := \emptyset
  \and \\
  \freenames{x?(y).P} := \{ x \} \cup (\freenames{P} \setminus \{ y \})
  \and 
  \freenames{x!\langle P \rangle} := \{ x \} \cup \{ P \} 
  \and \\
  \freenames{P|Q} := \freenames{P} \cup \freenames{Q}
  \and \\
  \freenames{@{x}} := \{ x \}
\end{mathpar}

$\pi$
$\quotep{\pi}$

$\freenames{-} : \pi \to \mathcal{P}(\quotep{\pi})$

\begin{eqnarray*}
  \freenames{\pzero} & := & \emptyset \\
  \freenames{x?(y).P} & := & \{ x \} \cup (\freenames{P} \setminus \{ y \}) \\
  \freenames{x!\langle P \rangle} & := & \{ x \} \cup \{ P \} \\
  \freenames{P|Q} & := & \freenames{P} \cup \freenames{Q} \\
  \freenames{\dropn{x}} & := & \{ x \}
\end{eqnarray*}

The bound names of a process, $\boundnames{P}$, are those names occurring in $P$
that are not free. For example, in $x?(y).0$, the name $x$ is free, while $y$ is bound.

\begin{mathpar}
  \inferrule* [lab=monoidal-laws] {} { P|Q \equiv Q|P \and P|0 \equiv P \and P|(Q|R) \equiv (P|Q)|R }
\end{mathpar}

\begin{mathpar}
  \inferrule* [lab=alpha-equivalence] {} { (x)P \equiv (y)P\{y/x\} \and y \not\in \freenames{P} }
\end{mathpar}

\begin{definition}
Then two processes, $P,Q$, are alpha-equivalent if $P = Q\{\vec{y}/\vec{x}\}$ for
some $\vec{x} \in \boundnames{Q},\vec{y} \in \boundnames{P}$, where $Q\{\vec{y}/\vec{x}\}$
denotes the capture-avoiding substitution of $\vec{y}$ for $\vec{x}$ in $Q$.
\end{definition}

\begin{definition}
  The {\em structural congruence} \cite{SangiorgiWalker} , $\equiv$,
  between processes is the least congruence containing
  alpha-equivalence, satisfying the abelian monoid laws
  (associativity, commutativity and $\pzero$ as identity) for parallel
  composition $|$ and for summation $+$.
\end{definition}

\subsection{Name equivalence}

We take name equivalence, written $\nameeq$, to be the smallest
equivalence relation generated by the following rules.

\begin{mathpar}
\inferrule*[lab=Quote-drop]
{ }
{ \quotep{@{x}} \nameeq x }

\inferrule*[lab=Struct-equiv]
{ P \scong Q }
{ \quotep{P} \nameeq \quotep{Q} }
\end{mathpar}

The astute reader will have noticed that the mutual recursion of names
and processes imposes a mutual recursion on alpha-equivalence and
structural equivalence via name-equivalence. Fortunately, all of this
works out pleasantly and we may calculate in the natural way, free of
concern. The reader interested in the details is referred to the
appendix \ref{appendix:rho_details}.

\subsection{Substitution}

We use $\Proc$ for the set of processes, $\QProc$ for the set of
names, and $\id{\{}\vec{y} / \vec{x} \id{\}}$ to denote partial maps,
$s : \QProc \rightarrow \QProc$. A map, $s$ lifts, uniquely, to a map
on process terms, $\widehat{s} : \Proc \rightarrow \Proc$ by the
following equations.

\begin{mathpar}
  (0) \psubstp{Q}{P} := 0 \\
  (R \juxtap S) \psubstp{Q}{P}
  :=    
  (R)\psubstp{Q}{P} \juxtap (S) \psubstp{Q}{P} \\
  (x?(y).R) \psubstp{Q}{P}    
  :=    
  (x)\substp{Q}{P} (z)\concat( (R \psubstn{z}{y}) \psubstp{Q}{P} ) \\
  (\lift{x}{R}) \psubstp{Q}{P}  
  :=
  \lift{(x)\substp{Q}{P}}{ R \psubstp{Q}{P} } \\
%   (\dropn{x})  \psubstp{Q}{P}       
%   := 
%   \left\{ 
%     \begin{array}{ccc} 
%       \dropn{\quotep{Q}} & & x \nameeq \quotep{P} \\
%       \dropn{x} & & otherwise \\
%     \end{array}
%   \right. 
  (\dropn{x})  \psubstp{Q}{P}       
  := 
  \left\{ 
    \begin{array}{ccc} 
      Q & & x \nameeq \quotep{P} \\
      \dropn{x} & & otherwise \\
    \end{array}
  \right.
\end{mathpar}
 

where

\begin{eqnarray}
  (x)\id{\{} \lpquote Q \rpquote / \lpquote P \rpquote \id{\}}            = 
  \left\{ 
    \begin{array}{ccc}
      \lpquote Q \rpquote & & x \nameeq \lpquote P \rpquote \\
      x & & otherwise \\
    \end{array}
  \right. \nonumber
\end{eqnarray}

and $z$ is chosen distinct from $\quotep{P}$, $\quotep{Q}$, the free
names in $Q$, and all the names in $R$. Our $\alpha$-equivalence will
be built in the standard way from this substitution.

\begin{remark}\label{rem:no_self_referential_names}
  One consequence of these definitions is that $\forall P. \quotep{P}
  \not\in \freenames{P}$.
\end{remark}

\subsection{ Dynamic quote: an example }

Anticipating something of what's to come, consider applying the
substitution, $\widehat{\id{\{}u / z \id{\}}}$, to the following pair
of processes, $\lift{w}{y!(z)}$ and $w[ \lpquote y!(z) \rpquote ]$.

\begin{eqnarray}
	\lift{w}{y!(z)}\widehat{\id{\{}u / z \id{\}}}
		& = &
		\lift{w}{y!(u)} \nonumber\\
	w[ \lpquote y!(z) \rpquote ] \widehat{ \id{\{}u / z \id{\}} }
		& = &
		w[ \lpquote y!(z) \rpquote ] \nonumber
\end{eqnarray}

Because the body of the process between quotes is impervious to
substitution, we get radically different answers. In fact, by
examining the first process in an input context,
e.g. $x?(z).\lift{w}{y!(z)}$, we see that the process under the lift
operator may be shaped by prefixed inputs binding a name inside it. In
this sense, the lift operator will be seen as a way to dynamically
construct processes before reifying them as names.

Finally equipped with these standard features we can present the
dynamics of the calculus.

\subsubsection{Operational semantics} 

Finally, we introduce the computational dynamics. What marks these
algebras as distinct from other more traditionally studied algebraic
structures, e.g. vector spaces or polynomial rings, is the manner in
which dynamics is captured. In traditional structures, dynamics is typically
expressed through morphisms between such structures, as in linear maps
between vector spaces or morphisms between rings. In algebras
associated with the semantics of computation, the dynamics is
expressed as part of the algebraic structure itself, through a
reduction reduction relation typically denoted by $\red$. Below, we
give a recursive presentation of this relation for the calculus used
in the encoding.

$\red \subseteq \pi \times \pi$
$\red : \pi \to \mathcal{P}(\pi)$

\begin{mathpar}
  \inferrule* [lab=Comm] { \textsf{match}( x_{src}, x_{trgt} ) } { x_{trgt}?(y)P \; | \; x_{src}!\langle {Q} \rangle \red P\{\quotep{Q}/y}\} }
  \and \\
  \inferrule* [lab=Par] {{P} \red {P}'} {{{P} | {Q}} \red {{P}' | {Q}}}
  \and
  \inferrule* [lab=Equiv]{{{P} \scong {P}'} \andalso {{P}' \red {Q}'} \andalso {{Q}' \scong {Q}}}{{P} \red {Q}}
\end{mathpar}

\begin{eqnarray*}
  match_{\equiv} (\quotep{P},\quotep{Q}) & := & P \equiv Q \\
  match_{\dagger}(\quotep{P},\quotep{Q}) & := & \forall R. P|Q \red^{*} R => R \red^{*} 0 \\
  match_{K}(\quotep{P},\quotep{Q}) & := & K \mbox{ for some context } K
\end{eqnarray*}

$u?(x)P | u!\langle Q \rangle \red P\{\quotep{Q}/x\}$

%We write $\wred$ for $\red^*$, and $P\red$ if $\exists Q $ such that $ P \red Q$.
We write $P\red$ if $\exists Q $ such that $ P \red Q$ and $P\not\red$, otherwise.

\section{Replication}

As mentioned before, it is known that replication (and hence
recursion) can be implemented in a higher-order process algebra
\cite{SangiorgiWalker}. As our first example of calculation with the
machinery thus far presented we give the construction explicitly in
the {\rhoc}.

\begin{eqnarray}
	D_{x} & := & \prefix{x}{y}{(\binpar{\outputp{x}{y}}{@{y}})} \nonumber\\
	\bangp_{x}{P} & := & \binpar{{x}!\langle{\binpar{D_{x}}{P}}\rangle}{D_{x}} \nonumber
\end{eqnarray}

\begin{eqnarray}
	\bangp_{x}{P} & & \nonumber\\
	=
	& {x}!\langle{(\prefix{x}{y}{(\outputp{x}{y} | @{y})) | P}}\rangle 
	      | \prefix{x}{y}{(\outputp{x}{y} | @{y})} & \nonumber\\
	\red
	& (\outputp{x}{y} | @{y})\substn{\quotep{(\prefix{x}{y}{(@{y} | \outputp{x}{y})) | P}}}{y} & \nonumber\\
	=
	& \outputp{x}{\quotep{(\prefix{x}{y}{(\outputp{x}{y} | @{y})) | P}}}
	  | {(\prefix{x}{y}{(\outputp{x}{y} | @{y})) | P}} & \nonumber\\
	\red
	& \ldots & \nonumber\\
	\red^*
	& P | P | \ldots & \nonumber
\end{eqnarray}

Of course, this encoding, as an implementation, runs away, unfolding
$\bangp{P}$ eagerly. A lazier and more implementable replication
operator, restricted to input-guarded processes, may be obtained as follows.

\begin{eqnarray}
\bangp{\prefix{u}{v}{P}} 
	:= 
	\binpar{\lift{x}{\prefix{u}{v}{(\binpar{D(x)}{P})}}}{D(x)} \nonumber
\end{eqnarray}

\begin{remark}
  Note that the lazier definition still does not deal with summation
  or mixed summation (i.e. sums over input and output). The reader is
  invited to construct definitions of replication that deal with these
  features. 

  Further, the definitions are parameterized in a name, $x$. Can you,
  gentle reader, make a definition that eliminates this parameter and
  guarantees no accidental interaction between the replication
  machinery and the process being replicated -- i.e. no accidental
  sharing of names used by the process to get its work done and the
  name(s) used by the replication to effect copying. This latter
  revision of the definition of replication is crucial to obtaining
  the expected identity $!!P \sim !P$.
\end{remark}

\begin{remark}\label{rem:paradoxical_combinator}
  The reader familiar with the lambda calculus will have noticed the
  similarity between $D$ and the paradoxical combinator.

  [Ed. note: the existence of this seems to suggest we have to be more
  restrictive on the set of processes and names we admit if we are to
  support no-cloning.]
\end{remark}

\subsubsection{Bisimulation}

The computational dynamics gives rise to another kind of equivalence,
the equivalence of computational behavior. As previously mentioned
this is typically captured \emph{via} some form of bisimulation.

% The notion we use in this paper is weak barbed bisimulation
% \cite{milner91polyadicpi}.

The notion we use in this paper is derived from weak barbed
bisimulation \cite{milner91polyadicpi}. 

\begin{definition}
An \emph{observation relation}, $\downarrow_{\mathcal N}$, over a set
of names, $\mathcal N$, is the smallest relation satisfying the rules
below.

\infrule[Out-barb]{y \in {\mathcal N}, \; x \nameeq y}
		  {\outputp{x}{v} \downarrow_{\mathcal N} x}
\infrule[Par-barb]{\mbox{$P\downarrow_{\mathcal N} x$ or $Q\downarrow_{\mathcal N} x$}}
		  {\binpar{P}{Q} \downarrow_{\mathcal N} x}

We write $P \Downarrow_{\mathcal N} x$ if there is $Q$ such that 
$P \wred Q$ and $Q \downarrow_{\mathcal N} x$.
\end{definition}

\begin{definition}
%\label{def.bbisim}
An  ${\mathcal N}$-\emph{barbed bisimulation} over a set of names, ${\mathcal N}$, is a symmetric binary relation 
${\mathcal S}_{\mathcal N}$ between agents such that $P\rel{S}_{\mathcal N}Q$ implies:
\begin{enumerate}
\item If $P \red P'$ then $Q \wred Q'$ and $P'\rel{S}_{\mathcal N} Q'$.
\item If $P\downarrow_{\mathcal N} x$, then $Q\Downarrow_{\mathcal N} x$.
\end{enumerate}
$P$ is ${\mathcal N}$-barbed bisimilar to $Q$, written
$P \wbbisim_{\mathcal N} Q$, if $P \rel{S}_{\mathcal N} Q$ for some ${\mathcal N}$-barbed bisimulation ${\mathcal S}_{\mathcal N}$.
\end{definition}

$\mathcal{R} \subseteq \pi \times \pi$

$P \mathcal{R} Q => \forall P'. P \red P' \Rightarrow \exists Q'. Q \red Q', P' \mathcal{R} Q'$

$P \vdash x \Rightarrow Q \vdash x$

\begin{mathpar}
  \inferrule*[lab=Out-barb]{x \nameeq y}{{y}!\langle{Q}\rangle \vdash x}
  \and
  \inferrule*[lab=Par-barb]{\mbox{$P\vdash x$ or $Q\vdash x$}}{\binpar{P}{Q} \vdash x}
\end{mathpar}

\subsubsection{Contexts}

One of the principle advantages of computational calculi like the
$\pi$-calculus is a well-defined notion of context,
contextual-equivalence and a correlation between
contextual-equivalence and notions of bisimulation. The notion of
context allows the decomposition of a process into (sub-)process and
its syntactic environment, its context. Thus, a context may be
thought of as a process with a ``hole'' (written $\Box$) in it. The
application of a context $M$ to a process $P$, written $M[P]$, is
tantamount to filling the hole in $M$ with $P$. In this paper we do
not need the full weight of this theory, but do make use of the notion
of context in the proof the main theorem. 

\begin{mathpar}
  \inferrule* [lab=summation] {} {{M_{M},M_{N}} \bc \Box \;|\; x.M_{A} \;|\; M_{M}+M_{N}}
  \and
  \inferrule* [lab=agent] {} {{M_{A}} \bc (\vec{x})M_{P} \;| \; \clift{P_0,\ldots,M_{P},\ldots,P_N}}
  \and \\
  \inferrule* [lab=process] {} {{M_{P}} \bc M_{N} \;| \;P|M_{P} }
\end{mathpar} 

\begin{mathpar}
  \inferrule* [lab=sychronization] {} {M_{N} \bc \Box \;|\; x?M_{F} \;|\; x!M_{C}}
  \and
  \inferrule* [lab=abstraction] {} {{M_{F}} \bc (x)M_{P} }
  \and
  \inferrule* [lab=concretion] {} {{M_{C}} \bc \langle M_{P} \rangle }
  \and \\
  \inferrule* [lab=process] {} {{M_{P}} \bc M_{N} \;| \;P|M_{P} }
\end{mathpar}

\begin{definition}[contextual application] Given a context $M$, and
  process $P$, we define the \emph{contextual application}, $M[P] :=
  M\{P/\Box\}$. That is, the contextual application of M to P is the
  substitution of $P$ for $\Box$ in $M$.
\end{definition}

$\meaningof{-} : L \to \mathcal{P}(\pi)$

\begin{mathpar}
  \inferrule* [lab=collection] {} {\meaningof{true} = \pi, \and \meaningof{~E} = \pi \setminus \meaningof{E}, \and \meaningof{E_{1} \& E_{2}} = \meaningof{E_{1}} \cap \meaningof{E_{2}}}
\end{mathpar}

\begin{mathpar}
  \inferrule* [lab=structure] {} {\meaningof{0} = \{ P \in \pi | P \equiv 0 \}, \and \\ \meaningof{E_1 | E_2} = \{ P \in \pi | P \equiv P_{1} | P_{2}, P_{1} \in \meaningof{E_{1}}, P_{2} \in \meaningof{E_2}\} }
\end{mathpar}

\begin{mathpar}
 \inferrule* [lab=behavior] {} {\meaningof{\langle a?b \rangle E} = \{ P \in \pi | P \equiv Q | u?(y)P', \\ \and \\\\ \and \\ \;\;\; u \in \meaningof{a}, \forall z.P'\{z/y\} \in \meaningof{E\{z/b\}}\}, \and \\ \meaningof{a!E} = \{ P \in \pi | P \equiv Q | x!\langle P' \rangle, x \in \meaningof{a} P' \in \meaningof{E}\} }
\end{mathpar}

\begin{mathpar}
 \inferrule* [lab=nominal] {} {\meaningof{\quotep{E}} = \{ \quotep{P} \in \quotep{\pi} | P \in \meaningof{E} \}, \and \meaningof{\quotep{P}} = \{ \quotep{Q} \in \quotep{\pi} | P \equiv Q \} \and \\ \meaningof{@\quotep{E}} = \{ P \in \pi | P \equiv @x, x \in \meaningof{E} \}}
\end{mathpar}

\begin{eqnarray*}
  \\
  \meaningof{-} : TS \to ST
\end{eqnarray*}

\begin{eqnarray*}
  \\
  L : TS \to ST
\end{eqnarray*}

\begin{eqnarray*}
  \\
  P \models E \iff P \in \meaningof{E}
\end{eqnarray*}

\begin{eqnarray*}
  P \approx_{L} Q \iff \forall E \in L. P \models E \iff Q \models E
\end{eqnarray*}

\begin{eqnarray*}
  P \approx_{K} Q
\end{eqnarray*}

\begin{eqnarray*}
  P \approx Q
\end{eqnarray*}

$\approx_{K} = \approx = \approx_{L}$

\subsubsection{Contextual duality}

Note that contexts extend the quotation operation to a family of
operations from processes to names. Given a context, $M$, we can
define a \emph{nominal context}, $\quotep{M}$ by $\quotep{M}[P] :=
\quotep{M[P]}$. To foreshadow what is to come we observe that these
operations enjoy a duality with processes very much like the duality
between vectors and maps from vectors to scalars.

Further, because the calculus is essentially higher-order, we have a
correspondence between contexts and processes. More specifically,
given a name $x$ and a context $M$ we can construct $M^{*}_{x}$ such
that 

\begin{mathpar}
  M^{*}_{x} | \lift{x}{P} \red M[P]
\end{mathpar}

namely,

\begin{mathpar}
  M^{*}_{x} := x?(u).M[\dropn{u}]
\end{mathpar}

The dependence of $M^{*}_{x}$ on a name makes it an abstraction, 

\begin{mathpar}
  M^{*} := (x)x?(u).M[\dropn{u}]
\end{mathpar}

\subsection{Additional notation}

It will sometimes be convenient to denote the process a name
quotes. We already have the notation $x = \quotep{P}$, but it will be
convenient to introduce an alternate notation, $\procn{x}$, when we
want to emphasize the connection to the use of the name. Note that, by
virtue of name equivalence, $\quotep{\procn{x}} \nameeq x$; so, the
notation is consistent with previous definitions.

Further, because names have structure it is possible to effect
substitutions on the basis of that structure. This means we need to
upgrade our notation for substitutions, which we accomplish by
adapting comprehension notation. Thus,

\begin{mathpar}
  P\{ y / x : x \in S \}
\end{mathpar}

is interpreted to mean the process derived from P by replacing (in a
capture-avoiding manner) each occurrence of $x$ in $S$ by $y$. For example,

\begin{mathpar}
  P\{ \quotep{\procn{x}|\procn{x}} / x : x \in \freenames{P} \}
\end{mathpar}

will replace each (occurrence) of a free name $x$ in $P$ by
$\quotep{\procn{x}|\procn{x}}$.

Also, we will avail ourselves of the notation $x^{L}$ and $x^{R}$ to
denote injections of a name into disjoint copies of the name
space. There are numerous ways to accomplish this. One example can be
found in \cite{MeredithR05}. This notation overloads to vectors of
names: $\vec{x}^{\pi} := (x_{i}^{\pi} \; : \; 0 \leq i < |\vec{x}| )$ where $\pi \in \{L,R\}$.

We also use $P^{\Box} := P|\Box$.

In \cite{MeredithR05} an interpretation of the new operator is
given. It turns out that there are several possible interpretations
all enjoying the requisite algebraic properties of the operator (see
\cite{milner91polyadicpi}). We will therefore make liberal use of
$(\nu\; \vec{x})P$.

% subsection the_syntax_and_semantics_of_the_notation_system (end)   

\input{qm2pi.qmops} 

\input{qm2pi.sterngerlach} 

\input{qm2pi.metric} 

% section concurrent_process_calculi (end)

%\input{qm2pi.proofsketch}

% section proof sketch (end)

%\input{qm2pi.slviaknots} 

% section spatial logic via knots (end)

\input{qm2pi.conclusion}

% section conclusion (end)

%\input{qm2pi.dtcodes} 

% section wiring algorithm (end)

\input{qm2pi.ack} 

% section acknowledgments (end)

\newpage


\bibliographystyle{plain}   
\bibliography{../../biblios/main.bib}

\input{qm2pi.rhodetails}

\end{document}



% section front matter (end)

\section{Introduction}\label{sec:introduction} % (fold)
In this draft of the material i am going to have to dispense with the
usual writing conventions adopted in papers on these topics. i'm going
to have adopt whatever tone i need at the time i'm writing up the
calculations. Sometimes this may be very conversational; others it may
be the barest mathematical grunts; others still it may be that i have
lifted text from one of my other papers because the exposition of some
point was better said there. i hope that my readers are not unduly put
out by this decision. i'm not doing this to flout convention or be
rebellious. i find these calculations very technically challenging. To
keep everything going technically, something has to give; i have to
let go of some cognitive burden. So, the academic writing style --
with all of its trade-offs in terms of facilitating technical
communication -- is what i'm letting go of. Perhaps subsequent drafts
can be tightened and polished, but for now, i'm going to speak as if
we were sitting together in a coffee shop with a laptop, wifi and a
pad of paper and a pencil.

So, here's what i have to say. We -- you and i, comfortably ensconced
in our coffee shop and well-equipped with our tools -- can realize and
carry out the calculations of quantum mechanics over a very different
formal theory of dynamics, a formal theory of dynamics that
corresponds to a theory of concurrent computation with
\emph{reflection}. It has the advantage that the underlying theory is
already `quantized', but supports analogues all of the continuuous
operations. Strikingly, this underlying theory has recently been
connected with a notion of metric that we can show, by calculating
together, coincides with the metric induced by the inner product.

There are a lot of reasons why you might be interested in seeing
calculations of this form. Here's why i'm interested. For the past
several centuries there has been no competitor to the ``Newtonian''
account of dynamics. As a result the predominant share of accounts of
dynamical systems and situations have had to be formulated in terms of
the Newtonian machinery. i view this as an intellectually dangerous
position to occupy. Everything, despite it's intrinsic shape, turns
into a nail to be hit with this hammer. Recently, however, the theory
of computation has matured to the point where we have candidates for
theories of dynamics that offer very different perspective on
reasoning about dynamical systems and situations. Testing these
candidates against very successful accounts of dynamical situations,
like quantum mechanics, is going to give us some sense of how mature
they are and some measure of the quality of these accounts of
dynamics.

\subsection{Summary of contributions and outline of paper}

So, we're going to develop an interpretation of the operations of
quantum mechanics normally interpreted by Hilbert spaces and
operators. We're going to do this over a theory of computation. Note
that this is very different than the usual quantum computation program
which develops notions of computation over quantum mechanics. Rather,
we are developing a story that aligns with Wheeler's slogan: It from
Bit. To do this we will first provide an account of the theory of
computation at play here. Then we will dive into a calculation-driven
interpretation of the operations of quantum mechanics.

The reason we take this approach is that -- until very recently --
there hasn't been an axiomatic account of quantum mechanics. As a
result there has been no sharp delineation of the mathematical theory
supporting interpretation of the physical theory and the physical
theory, itself. So, ambient features of the maths are free to be
exploited (or supressed) without a real accounting of their physical
relevance. There is no sharp statement ``here's the physical theory''
qua \emph{theory} and ``here's the mathematical interpretation''
enabling a judgment of how faithful the interpretation is -- apart
from experimental observation. When there is an axiomatic account we
can judge how well a given mathematical formalism supports an
interpretation of the axioms, independent of
experimentation. Likewise, we can judge how well we have captured our
physical evidence and experience with our axiomatics, independent of
any specific mathematical implementation, with accidental detail that
may or may not have physical significance. 

In lieu of a fully fleshed out and vetted axiomatic account of quantum
mechanics, interpreting the operational notions in service of modeling
physical systems will have to suffice. In other words, we are not in
the business of providing a model of Hilbert spaces and operators. We
are in the business of providing a model of quantum mechanics because
we are motivated by testing our notions of dynamics against physical
theory; and, the predictive calculations of the physical theory must
serve as the best formulation -- shy of a fully fleshed out axiomatic
account -- of the physical theory itself (as they have for scientific
theories since time immemorial). Put another way, despite a
whole-hearted commitment to an It-from-Bit ontology, we are firmly
aligned with the shut-up-and-calculate camp as the best way to obtain
results either from the physical perspective or as a quality assurance
measure of our fledgling theory of dynamics.

In detail, we present a reflective process calculus. Then we develop
intuitive correspondences between the notions available in this
calculus and the usual physical notions supporting quantum mechanical
calculations. Thus, 

\begin{table}[htp]
  \center{
    \fbox{
      \begin{tabular}{c|c}
        quantum mechanics & process calculus \\
        \hline
        scalar & name \\
        state vector & process \\
        dual & contextual duals \\
        matrix & formal sums of process-context-dual pairs \\
        orthogonality & process annihilation \\
        inner product & execution-formula + quoting
      \end{tabular}
    }
  }
  \caption{QM - process calculi correspondences}
\end{table}

Then we tighten up these intuitions to operational definitions. We
employ the Dirac notation as the best proxy we can find for an
abstract syntax of the quantum mechanical notions. The definitions we
develop put us in contact with equational constraints coming from the
theory that we demonstrate the definitions and calculations satisfy.

This puts us in a position to shut up and calculate for the
Stern-Gerlach experimental set up, showing how these predictive
calculations become calculations on processes in our theory of a
reflective process calculus.

Penultimately, we demonstrate that the notion of metric coming from
the inner product coincides with the notion of metric available from
the theory of bisimulation. This demonstration gives us the right to
think of space as arising from behavior. Finally, we consider where we
might go from the new vantage point we have obtained.

% section introduction (end) 
 
% section introduction (end)

% \documentclass[12pt]{llncs}
%\documentclass{jktr}

\usepackage[pdftex]{hyperref}                   
\usepackage {listings}
\usepackage {mathpartir}
\usepackage{bcprules}
%\usepackage{listings}
                       
\usepackage{graphicx} 
%\usepackage[margins=2.5cm,nohead,nofoot]{geometry}
%\usepackage{geometry}
\usepackage{amsfonts}
\usepackage{amstext}
\usepackage{latexsym}
\usepackage{amssymb}
\usepackage{color}


%\include{myPreamble}
\include{qm2pi.local} 

%\ifpdf
%\usepackage[pdftex]{graphicx}
%\else
%\usepackage{graphicx}
%\fi

 % \ifpdf
%  \usepackage{pdfsync}
%  \if


%\title{Brief Article}
%\author{David F. Snyder}
%\author{L.G. Meredith}

%\address{Dept. of Math., Texas State University--San Marcos, San Marcos, TX 78666}
       
\pagestyle{empty}


\begin{document}

\lstset{language=[Objective]Caml,frame=shadowbox}

\input{qm2pi.front}

% section front matter (end)

\input{qm2pi.intro} 
 
% section introduction (end)

% \input{qm2pi.knotations} 

% section notation (end)

\input{qm2pi.process.calculi} 

% section concurrent_process_calculi_and_spatial_logics_ (end)
    
%\input{qm2pi.knots2pi} 

%\input{qm2pi.trefoil} 

%\input{qm2pi.mainthm} 

% subsection basic_interpretation (end)

%\input{qm2pi.rho.presentation} 
\subsection{The syntax and semantics of the notation system}\label{sub:the_syntax_and_semantics_of_the_notation_system} % (fold)

We now summarize a technical presentation of the calculus that
embodies our theory of dynamics. The typical presentation of such a
calculus follows the style of giving generators and relations on
them. The grammar, below, describing term constructors, freely
generates the set of processes, $\Proc$. This set is then quotiented
by a relation known as structural congruence and it is over this set
that the notion of dynamics is expressed. This presentation is
essentially that of \cite{MeredithR05} with the addition of
polyadicity and summation. For readability we have relegated some of
the technical subtleties to an appendix.

\subsubsection{Process grammar}\label{subsub:process_grammar}

\begin{mathpar}
  \inferrule* [lab=synchronization] {} {{M} \bc \pzero \;|\; x?F \;|\; x!C }
  \and
  \inferrule* [lab=abstraction] {} {{F} \bc (x)P}
  \and
  \inferrule* [lab=concretion] {} {{C} \bc \langle Q \rangle}
  \and
  \inferrule* [lab=process] {} {{P,Q} \bc M \;| \;P|Q \;|\; @{x}}
  \and
  \inferrule* [lab=name] {} {{x} \bc \quotep{P}}
\end{mathpar} 

Note that $\vec{x}$ (resp. $\vec{P}$) denotes a vector of names
(resp. processes) of length $|\vec{x}|$ (resp. $|\vec{P}|$). We adopt
the following useful abbreviations.

\begin{mathpar}
   x?(\vec{y}).P := x.(\vec{y})P \and  x\clift{\vec{P}} := x.\clift{\vec{P}}
   \and x!(y) := \lift{x}{\dropn{y}}
   \and \Pi_{i=0}^{n-1}P_i := P_0 | \ldots | P_{n-1}
\end{mathpar}

\subsubsection{Structural congruence}

\paragraph{Free and bound names and alpha-equivalence.} At the
core of structural equivalence is alpha-equivalence which identifies
process that are the same up to a change of variable. Formally, we
recognize the distinction between free and bound names. The free names
of a process, $\freenames{P}$, may be calculated recursively as
follows:

\begin{mathpar}
\freenames{\pzero} := \emptyset
  \and \\
  \freenames{x?(y).P} := \{ x \} \cup (\freenames{P} \setminus \{ y \})
  \and 
  \freenames{x!\langle P \rangle} := \{ x \} \cup \{ P \} 
  \and \\
  \freenames{P|Q} := \freenames{P} \cup \freenames{Q}
  \and \\
  \freenames{@{x}} := \{ x \}
\end{mathpar}

$\pi$
$\quotep{\pi}$

$\freenames{-} : \pi \to \mathcal{P}(\quotep{\pi})$

\begin{eqnarray*}
  \freenames{\pzero} & := & \emptyset \\
  \freenames{x?(y).P} & := & \{ x \} \cup (\freenames{P} \setminus \{ y \}) \\
  \freenames{x!\langle P \rangle} & := & \{ x \} \cup \{ P \} \\
  \freenames{P|Q} & := & \freenames{P} \cup \freenames{Q} \\
  \freenames{\dropn{x}} & := & \{ x \}
\end{eqnarray*}

The bound names of a process, $\boundnames{P}$, are those names occurring in $P$
that are not free. For example, in $x?(y).0$, the name $x$ is free, while $y$ is bound.

\begin{mathpar}
  \inferrule* [lab=monoidal-laws] {} { P|Q \equiv Q|P \and P|0 \equiv P \and P|(Q|R) \equiv (P|Q)|R }
\end{mathpar}

\begin{mathpar}
  \inferrule* [lab=alpha-equivalence] {} { (x)P \equiv (y)P\{y/x\} \and y \not\in \freenames{P} }
\end{mathpar}

\begin{definition}
Then two processes, $P,Q$, are alpha-equivalent if $P = Q\{\vec{y}/\vec{x}\}$ for
some $\vec{x} \in \boundnames{Q},\vec{y} \in \boundnames{P}$, where $Q\{\vec{y}/\vec{x}\}$
denotes the capture-avoiding substitution of $\vec{y}$ for $\vec{x}$ in $Q$.
\end{definition}

\begin{definition}
  The {\em structural congruence} \cite{SangiorgiWalker} , $\equiv$,
  between processes is the least congruence containing
  alpha-equivalence, satisfying the abelian monoid laws
  (associativity, commutativity and $\pzero$ as identity) for parallel
  composition $|$ and for summation $+$.
\end{definition}

\subsection{Name equivalence}

We take name equivalence, written $\nameeq$, to be the smallest
equivalence relation generated by the following rules.

\begin{mathpar}
\inferrule*[lab=Quote-drop]
{ }
{ \quotep{@{x}} \nameeq x }

\inferrule*[lab=Struct-equiv]
{ P \scong Q }
{ \quotep{P} \nameeq \quotep{Q} }
\end{mathpar}

The astute reader will have noticed that the mutual recursion of names
and processes imposes a mutual recursion on alpha-equivalence and
structural equivalence via name-equivalence. Fortunately, all of this
works out pleasantly and we may calculate in the natural way, free of
concern. The reader interested in the details is referred to the
appendix \ref{appendix:rho_details}.

\subsection{Substitution}

We use $\Proc$ for the set of processes, $\QProc$ for the set of
names, and $\id{\{}\vec{y} / \vec{x} \id{\}}$ to denote partial maps,
$s : \QProc \rightarrow \QProc$. A map, $s$ lifts, uniquely, to a map
on process terms, $\widehat{s} : \Proc \rightarrow \Proc$ by the
following equations.

\begin{mathpar}
  (0) \psubstp{Q}{P} := 0 \\
  (R \juxtap S) \psubstp{Q}{P}
  :=    
  (R)\psubstp{Q}{P} \juxtap (S) \psubstp{Q}{P} \\
  (x?(y).R) \psubstp{Q}{P}    
  :=    
  (x)\substp{Q}{P} (z)\concat( (R \psubstn{z}{y}) \psubstp{Q}{P} ) \\
  (\lift{x}{R}) \psubstp{Q}{P}  
  :=
  \lift{(x)\substp{Q}{P}}{ R \psubstp{Q}{P} } \\
%   (\dropn{x})  \psubstp{Q}{P}       
%   := 
%   \left\{ 
%     \begin{array}{ccc} 
%       \dropn{\quotep{Q}} & & x \nameeq \quotep{P} \\
%       \dropn{x} & & otherwise \\
%     \end{array}
%   \right. 
  (\dropn{x})  \psubstp{Q}{P}       
  := 
  \left\{ 
    \begin{array}{ccc} 
      Q & & x \nameeq \quotep{P} \\
      \dropn{x} & & otherwise \\
    \end{array}
  \right.
\end{mathpar}
 

where

\begin{eqnarray}
  (x)\id{\{} \lpquote Q \rpquote / \lpquote P \rpquote \id{\}}            = 
  \left\{ 
    \begin{array}{ccc}
      \lpquote Q \rpquote & & x \nameeq \lpquote P \rpquote \\
      x & & otherwise \\
    \end{array}
  \right. \nonumber
\end{eqnarray}

and $z$ is chosen distinct from $\quotep{P}$, $\quotep{Q}$, the free
names in $Q$, and all the names in $R$. Our $\alpha$-equivalence will
be built in the standard way from this substitution.

\begin{remark}\label{rem:no_self_referential_names}
  One consequence of these definitions is that $\forall P. \quotep{P}
  \not\in \freenames{P}$.
\end{remark}

\subsection{ Dynamic quote: an example }

Anticipating something of what's to come, consider applying the
substitution, $\widehat{\id{\{}u / z \id{\}}}$, to the following pair
of processes, $\lift{w}{y!(z)}$ and $w[ \lpquote y!(z) \rpquote ]$.

\begin{eqnarray}
	\lift{w}{y!(z)}\widehat{\id{\{}u / z \id{\}}}
		& = &
		\lift{w}{y!(u)} \nonumber\\
	w[ \lpquote y!(z) \rpquote ] \widehat{ \id{\{}u / z \id{\}} }
		& = &
		w[ \lpquote y!(z) \rpquote ] \nonumber
\end{eqnarray}

Because the body of the process between quotes is impervious to
substitution, we get radically different answers. In fact, by
examining the first process in an input context,
e.g. $x?(z).\lift{w}{y!(z)}$, we see that the process under the lift
operator may be shaped by prefixed inputs binding a name inside it. In
this sense, the lift operator will be seen as a way to dynamically
construct processes before reifying them as names.

Finally equipped with these standard features we can present the
dynamics of the calculus.

\subsubsection{Operational semantics} 

Finally, we introduce the computational dynamics. What marks these
algebras as distinct from other more traditionally studied algebraic
structures, e.g. vector spaces or polynomial rings, is the manner in
which dynamics is captured. In traditional structures, dynamics is typically
expressed through morphisms between such structures, as in linear maps
between vector spaces or morphisms between rings. In algebras
associated with the semantics of computation, the dynamics is
expressed as part of the algebraic structure itself, through a
reduction reduction relation typically denoted by $\red$. Below, we
give a recursive presentation of this relation for the calculus used
in the encoding.

$\red \subseteq \pi \times \pi$
$\red : \pi \to \mathcal{P}(\pi)$

\begin{mathpar}
  \inferrule* [lab=Comm] { \textsf{match}( x_{src}, x_{trgt} ) } { x_{trgt}?(y)P \; | \; x_{src}!\langle {Q} \rangle \red P\{\quotep{Q}/y}\} }
  \and \\
  \inferrule* [lab=Par] {{P} \red {P}'} {{{P} | {Q}} \red {{P}' | {Q}}}
  \and
  \inferrule* [lab=Equiv]{{{P} \scong {P}'} \andalso {{P}' \red {Q}'} \andalso {{Q}' \scong {Q}}}{{P} \red {Q}}
\end{mathpar}

\begin{eqnarray*}
  match_{\equiv} (\quotep{P},\quotep{Q}) & := & P \equiv Q \\
  match_{\dagger}(\quotep{P},\quotep{Q}) & := & \forall R. P|Q \red^{*} R => R \red^{*} 0 \\
  match_{K}(\quotep{P},\quotep{Q}) & := & K \mbox{ for some context } K
\end{eqnarray*}

$u?(x)P | u!\langle Q \rangle \red P\{\quotep{Q}/x\}$

%We write $\wred$ for $\red^*$, and $P\red$ if $\exists Q $ such that $ P \red Q$.
We write $P\red$ if $\exists Q $ such that $ P \red Q$ and $P\not\red$, otherwise.

\section{Replication}

As mentioned before, it is known that replication (and hence
recursion) can be implemented in a higher-order process algebra
\cite{SangiorgiWalker}. As our first example of calculation with the
machinery thus far presented we give the construction explicitly in
the {\rhoc}.

\begin{eqnarray}
	D_{x} & := & \prefix{x}{y}{(\binpar{\outputp{x}{y}}{@{y}})} \nonumber\\
	\bangp_{x}{P} & := & \binpar{{x}!\langle{\binpar{D_{x}}{P}}\rangle}{D_{x}} \nonumber
\end{eqnarray}

\begin{eqnarray}
	\bangp_{x}{P} & & \nonumber\\
	=
	& {x}!\langle{(\prefix{x}{y}{(\outputp{x}{y} | @{y})) | P}}\rangle 
	      | \prefix{x}{y}{(\outputp{x}{y} | @{y})} & \nonumber\\
	\red
	& (\outputp{x}{y} | @{y})\substn{\quotep{(\prefix{x}{y}{(@{y} | \outputp{x}{y})) | P}}}{y} & \nonumber\\
	=
	& \outputp{x}{\quotep{(\prefix{x}{y}{(\outputp{x}{y} | @{y})) | P}}}
	  | {(\prefix{x}{y}{(\outputp{x}{y} | @{y})) | P}} & \nonumber\\
	\red
	& \ldots & \nonumber\\
	\red^*
	& P | P | \ldots & \nonumber
\end{eqnarray}

Of course, this encoding, as an implementation, runs away, unfolding
$\bangp{P}$ eagerly. A lazier and more implementable replication
operator, restricted to input-guarded processes, may be obtained as follows.

\begin{eqnarray}
\bangp{\prefix{u}{v}{P}} 
	:= 
	\binpar{\lift{x}{\prefix{u}{v}{(\binpar{D(x)}{P})}}}{D(x)} \nonumber
\end{eqnarray}

\begin{remark}
  Note that the lazier definition still does not deal with summation
  or mixed summation (i.e. sums over input and output). The reader is
  invited to construct definitions of replication that deal with these
  features. 

  Further, the definitions are parameterized in a name, $x$. Can you,
  gentle reader, make a definition that eliminates this parameter and
  guarantees no accidental interaction between the replication
  machinery and the process being replicated -- i.e. no accidental
  sharing of names used by the process to get its work done and the
  name(s) used by the replication to effect copying. This latter
  revision of the definition of replication is crucial to obtaining
  the expected identity $!!P \sim !P$.
\end{remark}

\begin{remark}\label{rem:paradoxical_combinator}
  The reader familiar with the lambda calculus will have noticed the
  similarity between $D$ and the paradoxical combinator.

  [Ed. note: the existence of this seems to suggest we have to be more
  restrictive on the set of processes and names we admit if we are to
  support no-cloning.]
\end{remark}

\subsubsection{Bisimulation}

The computational dynamics gives rise to another kind of equivalence,
the equivalence of computational behavior. As previously mentioned
this is typically captured \emph{via} some form of bisimulation.

% The notion we use in this paper is weak barbed bisimulation
% \cite{milner91polyadicpi}.

The notion we use in this paper is derived from weak barbed
bisimulation \cite{milner91polyadicpi}. 

\begin{definition}
An \emph{observation relation}, $\downarrow_{\mathcal N}$, over a set
of names, $\mathcal N$, is the smallest relation satisfying the rules
below.

\infrule[Out-barb]{y \in {\mathcal N}, \; x \nameeq y}
		  {\outputp{x}{v} \downarrow_{\mathcal N} x}
\infrule[Par-barb]{\mbox{$P\downarrow_{\mathcal N} x$ or $Q\downarrow_{\mathcal N} x$}}
		  {\binpar{P}{Q} \downarrow_{\mathcal N} x}

We write $P \Downarrow_{\mathcal N} x$ if there is $Q$ such that 
$P \wred Q$ and $Q \downarrow_{\mathcal N} x$.
\end{definition}

\begin{definition}
%\label{def.bbisim}
An  ${\mathcal N}$-\emph{barbed bisimulation} over a set of names, ${\mathcal N}$, is a symmetric binary relation 
${\mathcal S}_{\mathcal N}$ between agents such that $P\rel{S}_{\mathcal N}Q$ implies:
\begin{enumerate}
\item If $P \red P'$ then $Q \wred Q'$ and $P'\rel{S}_{\mathcal N} Q'$.
\item If $P\downarrow_{\mathcal N} x$, then $Q\Downarrow_{\mathcal N} x$.
\end{enumerate}
$P$ is ${\mathcal N}$-barbed bisimilar to $Q$, written
$P \wbbisim_{\mathcal N} Q$, if $P \rel{S}_{\mathcal N} Q$ for some ${\mathcal N}$-barbed bisimulation ${\mathcal S}_{\mathcal N}$.
\end{definition}

$\mathcal{R} \subseteq \pi \times \pi$

$P \mathcal{R} Q => \forall P'. P \red P' \Rightarrow \exists Q'. Q \red Q', P' \mathcal{R} Q'$

$P \vdash x \Rightarrow Q \vdash x$

\begin{mathpar}
  \inferrule*[lab=Out-barb]{x \nameeq y}{{y}!\langle{Q}\rangle \vdash x}
  \and
  \inferrule*[lab=Par-barb]{\mbox{$P\vdash x$ or $Q\vdash x$}}{\binpar{P}{Q} \vdash x}
\end{mathpar}

\subsubsection{Contexts}

One of the principle advantages of computational calculi like the
$\pi$-calculus is a well-defined notion of context,
contextual-equivalence and a correlation between
contextual-equivalence and notions of bisimulation. The notion of
context allows the decomposition of a process into (sub-)process and
its syntactic environment, its context. Thus, a context may be
thought of as a process with a ``hole'' (written $\Box$) in it. The
application of a context $M$ to a process $P$, written $M[P]$, is
tantamount to filling the hole in $M$ with $P$. In this paper we do
not need the full weight of this theory, but do make use of the notion
of context in the proof the main theorem. 

\begin{mathpar}
  \inferrule* [lab=summation] {} {{M_{M},M_{N}} \bc \Box \;|\; x.M_{A} \;|\; M_{M}+M_{N}}
  \and
  \inferrule* [lab=agent] {} {{M_{A}} \bc (\vec{x})M_{P} \;| \; \clift{P_0,\ldots,M_{P},\ldots,P_N}}
  \and \\
  \inferrule* [lab=process] {} {{M_{P}} \bc M_{N} \;| \;P|M_{P} }
\end{mathpar} 

\begin{mathpar}
  \inferrule* [lab=sychronization] {} {M_{N} \bc \Box \;|\; x?M_{F} \;|\; x!M_{C}}
  \and
  \inferrule* [lab=abstraction] {} {{M_{F}} \bc (x)M_{P} }
  \and
  \inferrule* [lab=concretion] {} {{M_{C}} \bc \langle M_{P} \rangle }
  \and \\
  \inferrule* [lab=process] {} {{M_{P}} \bc M_{N} \;| \;P|M_{P} }
\end{mathpar}

\begin{definition}[contextual application] Given a context $M$, and
  process $P$, we define the \emph{contextual application}, $M[P] :=
  M\{P/\Box\}$. That is, the contextual application of M to P is the
  substitution of $P$ for $\Box$ in $M$.
\end{definition}

$\meaningof{-} : L \to \mathcal{P}(\pi)$

\begin{mathpar}
  \inferrule* [lab=collection] {} {\meaningof{true} = \pi, \and \meaningof{~E} = \pi \setminus \meaningof{E}, \and \meaningof{E_{1} \& E_{2}} = \meaningof{E_{1}} \cap \meaningof{E_{2}}}
\end{mathpar}

\begin{mathpar}
  \inferrule* [lab=structure] {} {\meaningof{0} = \{ P \in \pi | P \equiv 0 \}, \and \\ \meaningof{E_1 | E_2} = \{ P \in \pi | P \equiv P_{1} | P_{2}, P_{1} \in \meaningof{E_{1}}, P_{2} \in \meaningof{E_2}\} }
\end{mathpar}

\begin{mathpar}
 \inferrule* [lab=behavior] {} {\meaningof{\langle a?b \rangle E} = \{ P \in \pi | P \equiv Q | u?(y)P', \\ \and \\\\ \and \\ \;\;\; u \in \meaningof{a}, \forall z.P'\{z/y\} \in \meaningof{E\{z/b\}}\}, \and \\ \meaningof{a!E} = \{ P \in \pi | P \equiv Q | x!\langle P' \rangle, x \in \meaningof{a} P' \in \meaningof{E}\} }
\end{mathpar}

\begin{mathpar}
 \inferrule* [lab=nominal] {} {\meaningof{\quotep{E}} = \{ \quotep{P} \in \quotep{\pi} | P \in \meaningof{E} \}, \and \meaningof{\quotep{P}} = \{ \quotep{Q} \in \quotep{\pi} | P \equiv Q \} \and \\ \meaningof{@\quotep{E}} = \{ P \in \pi | P \equiv @x, x \in \meaningof{E} \}}
\end{mathpar}

\begin{eqnarray*}
  \\
  \meaningof{-} : TS \to ST
\end{eqnarray*}

\begin{eqnarray*}
  \\
  L : TS \to ST
\end{eqnarray*}

\begin{eqnarray*}
  \\
  P \models E \iff P \in \meaningof{E}
\end{eqnarray*}

\begin{eqnarray*}
  P \approx_{L} Q \iff \forall E \in L. P \models E \iff Q \models E
\end{eqnarray*}

\begin{eqnarray*}
  P \approx_{K} Q
\end{eqnarray*}

\begin{eqnarray*}
  P \approx Q
\end{eqnarray*}

$\approx_{K} = \approx = \approx_{L}$

\subsubsection{Contextual duality}

Note that contexts extend the quotation operation to a family of
operations from processes to names. Given a context, $M$, we can
define a \emph{nominal context}, $\quotep{M}$ by $\quotep{M}[P] :=
\quotep{M[P]}$. To foreshadow what is to come we observe that these
operations enjoy a duality with processes very much like the duality
between vectors and maps from vectors to scalars.

Further, because the calculus is essentially higher-order, we have a
correspondence between contexts and processes. More specifically,
given a name $x$ and a context $M$ we can construct $M^{*}_{x}$ such
that 

\begin{mathpar}
  M^{*}_{x} | \lift{x}{P} \red M[P]
\end{mathpar}

namely,

\begin{mathpar}
  M^{*}_{x} := x?(u).M[\dropn{u}]
\end{mathpar}

The dependence of $M^{*}_{x}$ on a name makes it an abstraction, 

\begin{mathpar}
  M^{*} := (x)x?(u).M[\dropn{u}]
\end{mathpar}

\subsection{Additional notation}

It will sometimes be convenient to denote the process a name
quotes. We already have the notation $x = \quotep{P}$, but it will be
convenient to introduce an alternate notation, $\procn{x}$, when we
want to emphasize the connection to the use of the name. Note that, by
virtue of name equivalence, $\quotep{\procn{x}} \nameeq x$; so, the
notation is consistent with previous definitions.

Further, because names have structure it is possible to effect
substitutions on the basis of that structure. This means we need to
upgrade our notation for substitutions, which we accomplish by
adapting comprehension notation. Thus,

\begin{mathpar}
  P\{ y / x : x \in S \}
\end{mathpar}

is interpreted to mean the process derived from P by replacing (in a
capture-avoiding manner) each occurrence of $x$ in $S$ by $y$. For example,

\begin{mathpar}
  P\{ \quotep{\procn{x}|\procn{x}} / x : x \in \freenames{P} \}
\end{mathpar}

will replace each (occurrence) of a free name $x$ in $P$ by
$\quotep{\procn{x}|\procn{x}}$.

Also, we will avail ourselves of the notation $x^{L}$ and $x^{R}$ to
denote injections of a name into disjoint copies of the name
space. There are numerous ways to accomplish this. One example can be
found in \cite{MeredithR05}. This notation overloads to vectors of
names: $\vec{x}^{\pi} := (x_{i}^{\pi} \; : \; 0 \leq i < |\vec{x}| )$ where $\pi \in \{L,R\}$.

We also use $P^{\Box} := P|\Box$.

In \cite{MeredithR05} an interpretation of the new operator is
given. It turns out that there are several possible interpretations
all enjoying the requisite algebraic properties of the operator (see
\cite{milner91polyadicpi}). We will therefore make liberal use of
$(\nu\; \vec{x})P$.

% subsection the_syntax_and_semantics_of_the_notation_system (end)   

\input{qm2pi.qmops} 

\input{qm2pi.sterngerlach} 

\input{qm2pi.metric} 

% section concurrent_process_calculi (end)

%\input{qm2pi.proofsketch}

% section proof sketch (end)

%\input{qm2pi.slviaknots} 

% section spatial logic via knots (end)

\input{qm2pi.conclusion}

% section conclusion (end)

%\input{qm2pi.dtcodes} 

% section wiring algorithm (end)

\input{qm2pi.ack} 

% section acknowledgments (end)

\newpage


\bibliographystyle{plain}   
\bibliography{../../biblios/main.bib}

\input{qm2pi.rhodetails}

\end{document}

 

% section notation (end)

\input{qm2pi.process.calculi} 

% section concurrent_process_calculi_and_spatial_logics_ (end)
    
%\documentclass[12pt]{llncs}
%\documentclass{jktr}

\usepackage[pdftex]{hyperref}                   
\usepackage {listings}
\usepackage {mathpartir}
\usepackage{bcprules}
%\usepackage{listings}
                       
\usepackage{graphicx} 
%\usepackage[margins=2.5cm,nohead,nofoot]{geometry}
%\usepackage{geometry}
\usepackage{amsfonts}
\usepackage{amstext}
\usepackage{latexsym}
\usepackage{amssymb}
\usepackage{color}


%\include{myPreamble}
\include{qm2pi.local} 

%\ifpdf
%\usepackage[pdftex]{graphicx}
%\else
%\usepackage{graphicx}
%\fi

 % \ifpdf
%  \usepackage{pdfsync}
%  \if


%\title{Brief Article}
%\author{David F. Snyder}
%\author{L.G. Meredith}

%\address{Dept. of Math., Texas State University--San Marcos, San Marcos, TX 78666}
       
\pagestyle{empty}


\begin{document}

\lstset{language=[Objective]Caml,frame=shadowbox}

\input{qm2pi.front}

% section front matter (end)

\input{qm2pi.intro} 
 
% section introduction (end)

% \input{qm2pi.knotations} 

% section notation (end)

\input{qm2pi.process.calculi} 

% section concurrent_process_calculi_and_spatial_logics_ (end)
    
%\input{qm2pi.knots2pi} 

%\input{qm2pi.trefoil} 

%\input{qm2pi.mainthm} 

% subsection basic_interpretation (end)

%\input{qm2pi.rho.presentation} 
\subsection{The syntax and semantics of the notation system}\label{sub:the_syntax_and_semantics_of_the_notation_system} % (fold)

We now summarize a technical presentation of the calculus that
embodies our theory of dynamics. The typical presentation of such a
calculus follows the style of giving generators and relations on
them. The grammar, below, describing term constructors, freely
generates the set of processes, $\Proc$. This set is then quotiented
by a relation known as structural congruence and it is over this set
that the notion of dynamics is expressed. This presentation is
essentially that of \cite{MeredithR05} with the addition of
polyadicity and summation. For readability we have relegated some of
the technical subtleties to an appendix.

\subsubsection{Process grammar}\label{subsub:process_grammar}

\begin{mathpar}
  \inferrule* [lab=synchronization] {} {{M} \bc \pzero \;|\; x?F \;|\; x!C }
  \and
  \inferrule* [lab=abstraction] {} {{F} \bc (x)P}
  \and
  \inferrule* [lab=concretion] {} {{C} \bc \langle Q \rangle}
  \and
  \inferrule* [lab=process] {} {{P,Q} \bc M \;| \;P|Q \;|\; @{x}}
  \and
  \inferrule* [lab=name] {} {{x} \bc \quotep{P}}
\end{mathpar} 

Note that $\vec{x}$ (resp. $\vec{P}$) denotes a vector of names
(resp. processes) of length $|\vec{x}|$ (resp. $|\vec{P}|$). We adopt
the following useful abbreviations.

\begin{mathpar}
   x?(\vec{y}).P := x.(\vec{y})P \and  x\clift{\vec{P}} := x.\clift{\vec{P}}
   \and x!(y) := \lift{x}{\dropn{y}}
   \and \Pi_{i=0}^{n-1}P_i := P_0 | \ldots | P_{n-1}
\end{mathpar}

\subsubsection{Structural congruence}

\paragraph{Free and bound names and alpha-equivalence.} At the
core of structural equivalence is alpha-equivalence which identifies
process that are the same up to a change of variable. Formally, we
recognize the distinction between free and bound names. The free names
of a process, $\freenames{P}$, may be calculated recursively as
follows:

\begin{mathpar}
\freenames{\pzero} := \emptyset
  \and \\
  \freenames{x?(y).P} := \{ x \} \cup (\freenames{P} \setminus \{ y \})
  \and 
  \freenames{x!\langle P \rangle} := \{ x \} \cup \{ P \} 
  \and \\
  \freenames{P|Q} := \freenames{P} \cup \freenames{Q}
  \and \\
  \freenames{@{x}} := \{ x \}
\end{mathpar}

$\pi$
$\quotep{\pi}$

$\freenames{-} : \pi \to \mathcal{P}(\quotep{\pi})$

\begin{eqnarray*}
  \freenames{\pzero} & := & \emptyset \\
  \freenames{x?(y).P} & := & \{ x \} \cup (\freenames{P} \setminus \{ y \}) \\
  \freenames{x!\langle P \rangle} & := & \{ x \} \cup \{ P \} \\
  \freenames{P|Q} & := & \freenames{P} \cup \freenames{Q} \\
  \freenames{\dropn{x}} & := & \{ x \}
\end{eqnarray*}

The bound names of a process, $\boundnames{P}$, are those names occurring in $P$
that are not free. For example, in $x?(y).0$, the name $x$ is free, while $y$ is bound.

\begin{mathpar}
  \inferrule* [lab=monoidal-laws] {} { P|Q \equiv Q|P \and P|0 \equiv P \and P|(Q|R) \equiv (P|Q)|R }
\end{mathpar}

\begin{mathpar}
  \inferrule* [lab=alpha-equivalence] {} { (x)P \equiv (y)P\{y/x\} \and y \not\in \freenames{P} }
\end{mathpar}

\begin{definition}
Then two processes, $P,Q$, are alpha-equivalent if $P = Q\{\vec{y}/\vec{x}\}$ for
some $\vec{x} \in \boundnames{Q},\vec{y} \in \boundnames{P}$, where $Q\{\vec{y}/\vec{x}\}$
denotes the capture-avoiding substitution of $\vec{y}$ for $\vec{x}$ in $Q$.
\end{definition}

\begin{definition}
  The {\em structural congruence} \cite{SangiorgiWalker} , $\equiv$,
  between processes is the least congruence containing
  alpha-equivalence, satisfying the abelian monoid laws
  (associativity, commutativity and $\pzero$ as identity) for parallel
  composition $|$ and for summation $+$.
\end{definition}

\subsection{Name equivalence}

We take name equivalence, written $\nameeq$, to be the smallest
equivalence relation generated by the following rules.

\begin{mathpar}
\inferrule*[lab=Quote-drop]
{ }
{ \quotep{@{x}} \nameeq x }

\inferrule*[lab=Struct-equiv]
{ P \scong Q }
{ \quotep{P} \nameeq \quotep{Q} }
\end{mathpar}

The astute reader will have noticed that the mutual recursion of names
and processes imposes a mutual recursion on alpha-equivalence and
structural equivalence via name-equivalence. Fortunately, all of this
works out pleasantly and we may calculate in the natural way, free of
concern. The reader interested in the details is referred to the
appendix \ref{appendix:rho_details}.

\subsection{Substitution}

We use $\Proc$ for the set of processes, $\QProc$ for the set of
names, and $\id{\{}\vec{y} / \vec{x} \id{\}}$ to denote partial maps,
$s : \QProc \rightarrow \QProc$. A map, $s$ lifts, uniquely, to a map
on process terms, $\widehat{s} : \Proc \rightarrow \Proc$ by the
following equations.

\begin{mathpar}
  (0) \psubstp{Q}{P} := 0 \\
  (R \juxtap S) \psubstp{Q}{P}
  :=    
  (R)\psubstp{Q}{P} \juxtap (S) \psubstp{Q}{P} \\
  (x?(y).R) \psubstp{Q}{P}    
  :=    
  (x)\substp{Q}{P} (z)\concat( (R \psubstn{z}{y}) \psubstp{Q}{P} ) \\
  (\lift{x}{R}) \psubstp{Q}{P}  
  :=
  \lift{(x)\substp{Q}{P}}{ R \psubstp{Q}{P} } \\
%   (\dropn{x})  \psubstp{Q}{P}       
%   := 
%   \left\{ 
%     \begin{array}{ccc} 
%       \dropn{\quotep{Q}} & & x \nameeq \quotep{P} \\
%       \dropn{x} & & otherwise \\
%     \end{array}
%   \right. 
  (\dropn{x})  \psubstp{Q}{P}       
  := 
  \left\{ 
    \begin{array}{ccc} 
      Q & & x \nameeq \quotep{P} \\
      \dropn{x} & & otherwise \\
    \end{array}
  \right.
\end{mathpar}
 

where

\begin{eqnarray}
  (x)\id{\{} \lpquote Q \rpquote / \lpquote P \rpquote \id{\}}            = 
  \left\{ 
    \begin{array}{ccc}
      \lpquote Q \rpquote & & x \nameeq \lpquote P \rpquote \\
      x & & otherwise \\
    \end{array}
  \right. \nonumber
\end{eqnarray}

and $z$ is chosen distinct from $\quotep{P}$, $\quotep{Q}$, the free
names in $Q$, and all the names in $R$. Our $\alpha$-equivalence will
be built in the standard way from this substitution.

\begin{remark}\label{rem:no_self_referential_names}
  One consequence of these definitions is that $\forall P. \quotep{P}
  \not\in \freenames{P}$.
\end{remark}

\subsection{ Dynamic quote: an example }

Anticipating something of what's to come, consider applying the
substitution, $\widehat{\id{\{}u / z \id{\}}}$, to the following pair
of processes, $\lift{w}{y!(z)}$ and $w[ \lpquote y!(z) \rpquote ]$.

\begin{eqnarray}
	\lift{w}{y!(z)}\widehat{\id{\{}u / z \id{\}}}
		& = &
		\lift{w}{y!(u)} \nonumber\\
	w[ \lpquote y!(z) \rpquote ] \widehat{ \id{\{}u / z \id{\}} }
		& = &
		w[ \lpquote y!(z) \rpquote ] \nonumber
\end{eqnarray}

Because the body of the process between quotes is impervious to
substitution, we get radically different answers. In fact, by
examining the first process in an input context,
e.g. $x?(z).\lift{w}{y!(z)}$, we see that the process under the lift
operator may be shaped by prefixed inputs binding a name inside it. In
this sense, the lift operator will be seen as a way to dynamically
construct processes before reifying them as names.

Finally equipped with these standard features we can present the
dynamics of the calculus.

\subsubsection{Operational semantics} 

Finally, we introduce the computational dynamics. What marks these
algebras as distinct from other more traditionally studied algebraic
structures, e.g. vector spaces or polynomial rings, is the manner in
which dynamics is captured. In traditional structures, dynamics is typically
expressed through morphisms between such structures, as in linear maps
between vector spaces or morphisms between rings. In algebras
associated with the semantics of computation, the dynamics is
expressed as part of the algebraic structure itself, through a
reduction reduction relation typically denoted by $\red$. Below, we
give a recursive presentation of this relation for the calculus used
in the encoding.

$\red \subseteq \pi \times \pi$
$\red : \pi \to \mathcal{P}(\pi)$

\begin{mathpar}
  \inferrule* [lab=Comm] { \textsf{match}( x_{src}, x_{trgt} ) } { x_{trgt}?(y)P \; | \; x_{src}!\langle {Q} \rangle \red P\{\quotep{Q}/y}\} }
  \and \\
  \inferrule* [lab=Par] {{P} \red {P}'} {{{P} | {Q}} \red {{P}' | {Q}}}
  \and
  \inferrule* [lab=Equiv]{{{P} \scong {P}'} \andalso {{P}' \red {Q}'} \andalso {{Q}' \scong {Q}}}{{P} \red {Q}}
\end{mathpar}

\begin{eqnarray*}
  match_{\equiv} (\quotep{P},\quotep{Q}) & := & P \equiv Q \\
  match_{\dagger}(\quotep{P},\quotep{Q}) & := & \forall R. P|Q \red^{*} R => R \red^{*} 0 \\
  match_{K}(\quotep{P},\quotep{Q}) & := & K \mbox{ for some context } K
\end{eqnarray*}

$u?(x)P | u!\langle Q \rangle \red P\{\quotep{Q}/x\}$

%We write $\wred$ for $\red^*$, and $P\red$ if $\exists Q $ such that $ P \red Q$.
We write $P\red$ if $\exists Q $ such that $ P \red Q$ and $P\not\red$, otherwise.

\section{Replication}

As mentioned before, it is known that replication (and hence
recursion) can be implemented in a higher-order process algebra
\cite{SangiorgiWalker}. As our first example of calculation with the
machinery thus far presented we give the construction explicitly in
the {\rhoc}.

\begin{eqnarray}
	D_{x} & := & \prefix{x}{y}{(\binpar{\outputp{x}{y}}{@{y}})} \nonumber\\
	\bangp_{x}{P} & := & \binpar{{x}!\langle{\binpar{D_{x}}{P}}\rangle}{D_{x}} \nonumber
\end{eqnarray}

\begin{eqnarray}
	\bangp_{x}{P} & & \nonumber\\
	=
	& {x}!\langle{(\prefix{x}{y}{(\outputp{x}{y} | @{y})) | P}}\rangle 
	      | \prefix{x}{y}{(\outputp{x}{y} | @{y})} & \nonumber\\
	\red
	& (\outputp{x}{y} | @{y})\substn{\quotep{(\prefix{x}{y}{(@{y} | \outputp{x}{y})) | P}}}{y} & \nonumber\\
	=
	& \outputp{x}{\quotep{(\prefix{x}{y}{(\outputp{x}{y} | @{y})) | P}}}
	  | {(\prefix{x}{y}{(\outputp{x}{y} | @{y})) | P}} & \nonumber\\
	\red
	& \ldots & \nonumber\\
	\red^*
	& P | P | \ldots & \nonumber
\end{eqnarray}

Of course, this encoding, as an implementation, runs away, unfolding
$\bangp{P}$ eagerly. A lazier and more implementable replication
operator, restricted to input-guarded processes, may be obtained as follows.

\begin{eqnarray}
\bangp{\prefix{u}{v}{P}} 
	:= 
	\binpar{\lift{x}{\prefix{u}{v}{(\binpar{D(x)}{P})}}}{D(x)} \nonumber
\end{eqnarray}

\begin{remark}
  Note that the lazier definition still does not deal with summation
  or mixed summation (i.e. sums over input and output). The reader is
  invited to construct definitions of replication that deal with these
  features. 

  Further, the definitions are parameterized in a name, $x$. Can you,
  gentle reader, make a definition that eliminates this parameter and
  guarantees no accidental interaction between the replication
  machinery and the process being replicated -- i.e. no accidental
  sharing of names used by the process to get its work done and the
  name(s) used by the replication to effect copying. This latter
  revision of the definition of replication is crucial to obtaining
  the expected identity $!!P \sim !P$.
\end{remark}

\begin{remark}\label{rem:paradoxical_combinator}
  The reader familiar with the lambda calculus will have noticed the
  similarity between $D$ and the paradoxical combinator.

  [Ed. note: the existence of this seems to suggest we have to be more
  restrictive on the set of processes and names we admit if we are to
  support no-cloning.]
\end{remark}

\subsubsection{Bisimulation}

The computational dynamics gives rise to another kind of equivalence,
the equivalence of computational behavior. As previously mentioned
this is typically captured \emph{via} some form of bisimulation.

% The notion we use in this paper is weak barbed bisimulation
% \cite{milner91polyadicpi}.

The notion we use in this paper is derived from weak barbed
bisimulation \cite{milner91polyadicpi}. 

\begin{definition}
An \emph{observation relation}, $\downarrow_{\mathcal N}$, over a set
of names, $\mathcal N$, is the smallest relation satisfying the rules
below.

\infrule[Out-barb]{y \in {\mathcal N}, \; x \nameeq y}
		  {\outputp{x}{v} \downarrow_{\mathcal N} x}
\infrule[Par-barb]{\mbox{$P\downarrow_{\mathcal N} x$ or $Q\downarrow_{\mathcal N} x$}}
		  {\binpar{P}{Q} \downarrow_{\mathcal N} x}

We write $P \Downarrow_{\mathcal N} x$ if there is $Q$ such that 
$P \wred Q$ and $Q \downarrow_{\mathcal N} x$.
\end{definition}

\begin{definition}
%\label{def.bbisim}
An  ${\mathcal N}$-\emph{barbed bisimulation} over a set of names, ${\mathcal N}$, is a symmetric binary relation 
${\mathcal S}_{\mathcal N}$ between agents such that $P\rel{S}_{\mathcal N}Q$ implies:
\begin{enumerate}
\item If $P \red P'$ then $Q \wred Q'$ and $P'\rel{S}_{\mathcal N} Q'$.
\item If $P\downarrow_{\mathcal N} x$, then $Q\Downarrow_{\mathcal N} x$.
\end{enumerate}
$P$ is ${\mathcal N}$-barbed bisimilar to $Q$, written
$P \wbbisim_{\mathcal N} Q$, if $P \rel{S}_{\mathcal N} Q$ for some ${\mathcal N}$-barbed bisimulation ${\mathcal S}_{\mathcal N}$.
\end{definition}

$\mathcal{R} \subseteq \pi \times \pi$

$P \mathcal{R} Q => \forall P'. P \red P' \Rightarrow \exists Q'. Q \red Q', P' \mathcal{R} Q'$

$P \vdash x \Rightarrow Q \vdash x$

\begin{mathpar}
  \inferrule*[lab=Out-barb]{x \nameeq y}{{y}!\langle{Q}\rangle \vdash x}
  \and
  \inferrule*[lab=Par-barb]{\mbox{$P\vdash x$ or $Q\vdash x$}}{\binpar{P}{Q} \vdash x}
\end{mathpar}

\subsubsection{Contexts}

One of the principle advantages of computational calculi like the
$\pi$-calculus is a well-defined notion of context,
contextual-equivalence and a correlation between
contextual-equivalence and notions of bisimulation. The notion of
context allows the decomposition of a process into (sub-)process and
its syntactic environment, its context. Thus, a context may be
thought of as a process with a ``hole'' (written $\Box$) in it. The
application of a context $M$ to a process $P$, written $M[P]$, is
tantamount to filling the hole in $M$ with $P$. In this paper we do
not need the full weight of this theory, but do make use of the notion
of context in the proof the main theorem. 

\begin{mathpar}
  \inferrule* [lab=summation] {} {{M_{M},M_{N}} \bc \Box \;|\; x.M_{A} \;|\; M_{M}+M_{N}}
  \and
  \inferrule* [lab=agent] {} {{M_{A}} \bc (\vec{x})M_{P} \;| \; \clift{P_0,\ldots,M_{P},\ldots,P_N}}
  \and \\
  \inferrule* [lab=process] {} {{M_{P}} \bc M_{N} \;| \;P|M_{P} }
\end{mathpar} 

\begin{mathpar}
  \inferrule* [lab=sychronization] {} {M_{N} \bc \Box \;|\; x?M_{F} \;|\; x!M_{C}}
  \and
  \inferrule* [lab=abstraction] {} {{M_{F}} \bc (x)M_{P} }
  \and
  \inferrule* [lab=concretion] {} {{M_{C}} \bc \langle M_{P} \rangle }
  \and \\
  \inferrule* [lab=process] {} {{M_{P}} \bc M_{N} \;| \;P|M_{P} }
\end{mathpar}

\begin{definition}[contextual application] Given a context $M$, and
  process $P$, we define the \emph{contextual application}, $M[P] :=
  M\{P/\Box\}$. That is, the contextual application of M to P is the
  substitution of $P$ for $\Box$ in $M$.
\end{definition}

$\meaningof{-} : L \to \mathcal{P}(\pi)$

\begin{mathpar}
  \inferrule* [lab=collection] {} {\meaningof{true} = \pi, \and \meaningof{~E} = \pi \setminus \meaningof{E}, \and \meaningof{E_{1} \& E_{2}} = \meaningof{E_{1}} \cap \meaningof{E_{2}}}
\end{mathpar}

\begin{mathpar}
  \inferrule* [lab=structure] {} {\meaningof{0} = \{ P \in \pi | P \equiv 0 \}, \and \\ \meaningof{E_1 | E_2} = \{ P \in \pi | P \equiv P_{1} | P_{2}, P_{1} \in \meaningof{E_{1}}, P_{2} \in \meaningof{E_2}\} }
\end{mathpar}

\begin{mathpar}
 \inferrule* [lab=behavior] {} {\meaningof{\langle a?b \rangle E} = \{ P \in \pi | P \equiv Q | u?(y)P', \\ \and \\\\ \and \\ \;\;\; u \in \meaningof{a}, \forall z.P'\{z/y\} \in \meaningof{E\{z/b\}}\}, \and \\ \meaningof{a!E} = \{ P \in \pi | P \equiv Q | x!\langle P' \rangle, x \in \meaningof{a} P' \in \meaningof{E}\} }
\end{mathpar}

\begin{mathpar}
 \inferrule* [lab=nominal] {} {\meaningof{\quotep{E}} = \{ \quotep{P} \in \quotep{\pi} | P \in \meaningof{E} \}, \and \meaningof{\quotep{P}} = \{ \quotep{Q} \in \quotep{\pi} | P \equiv Q \} \and \\ \meaningof{@\quotep{E}} = \{ P \in \pi | P \equiv @x, x \in \meaningof{E} \}}
\end{mathpar}

\begin{eqnarray*}
  \\
  \meaningof{-} : TS \to ST
\end{eqnarray*}

\begin{eqnarray*}
  \\
  L : TS \to ST
\end{eqnarray*}

\begin{eqnarray*}
  \\
  P \models E \iff P \in \meaningof{E}
\end{eqnarray*}

\begin{eqnarray*}
  P \approx_{L} Q \iff \forall E \in L. P \models E \iff Q \models E
\end{eqnarray*}

\begin{eqnarray*}
  P \approx_{K} Q
\end{eqnarray*}

\begin{eqnarray*}
  P \approx Q
\end{eqnarray*}

$\approx_{K} = \approx = \approx_{L}$

\subsubsection{Contextual duality}

Note that contexts extend the quotation operation to a family of
operations from processes to names. Given a context, $M$, we can
define a \emph{nominal context}, $\quotep{M}$ by $\quotep{M}[P] :=
\quotep{M[P]}$. To foreshadow what is to come we observe that these
operations enjoy a duality with processes very much like the duality
between vectors and maps from vectors to scalars.

Further, because the calculus is essentially higher-order, we have a
correspondence between contexts and processes. More specifically,
given a name $x$ and a context $M$ we can construct $M^{*}_{x}$ such
that 

\begin{mathpar}
  M^{*}_{x} | \lift{x}{P} \red M[P]
\end{mathpar}

namely,

\begin{mathpar}
  M^{*}_{x} := x?(u).M[\dropn{u}]
\end{mathpar}

The dependence of $M^{*}_{x}$ on a name makes it an abstraction, 

\begin{mathpar}
  M^{*} := (x)x?(u).M[\dropn{u}]
\end{mathpar}

\subsection{Additional notation}

It will sometimes be convenient to denote the process a name
quotes. We already have the notation $x = \quotep{P}$, but it will be
convenient to introduce an alternate notation, $\procn{x}$, when we
want to emphasize the connection to the use of the name. Note that, by
virtue of name equivalence, $\quotep{\procn{x}} \nameeq x$; so, the
notation is consistent with previous definitions.

Further, because names have structure it is possible to effect
substitutions on the basis of that structure. This means we need to
upgrade our notation for substitutions, which we accomplish by
adapting comprehension notation. Thus,

\begin{mathpar}
  P\{ y / x : x \in S \}
\end{mathpar}

is interpreted to mean the process derived from P by replacing (in a
capture-avoiding manner) each occurrence of $x$ in $S$ by $y$. For example,

\begin{mathpar}
  P\{ \quotep{\procn{x}|\procn{x}} / x : x \in \freenames{P} \}
\end{mathpar}

will replace each (occurrence) of a free name $x$ in $P$ by
$\quotep{\procn{x}|\procn{x}}$.

Also, we will avail ourselves of the notation $x^{L}$ and $x^{R}$ to
denote injections of a name into disjoint copies of the name
space. There are numerous ways to accomplish this. One example can be
found in \cite{MeredithR05}. This notation overloads to vectors of
names: $\vec{x}^{\pi} := (x_{i}^{\pi} \; : \; 0 \leq i < |\vec{x}| )$ where $\pi \in \{L,R\}$.

We also use $P^{\Box} := P|\Box$.

In \cite{MeredithR05} an interpretation of the new operator is
given. It turns out that there are several possible interpretations
all enjoying the requisite algebraic properties of the operator (see
\cite{milner91polyadicpi}). We will therefore make liberal use of
$(\nu\; \vec{x})P$.

% subsection the_syntax_and_semantics_of_the_notation_system (end)   

\input{qm2pi.qmops} 

\input{qm2pi.sterngerlach} 

\input{qm2pi.metric} 

% section concurrent_process_calculi (end)

%\input{qm2pi.proofsketch}

% section proof sketch (end)

%\input{qm2pi.slviaknots} 

% section spatial logic via knots (end)

\input{qm2pi.conclusion}

% section conclusion (end)

%\input{qm2pi.dtcodes} 

% section wiring algorithm (end)

\input{qm2pi.ack} 

% section acknowledgments (end)

\newpage


\bibliographystyle{plain}   
\bibliography{../../biblios/main.bib}

\input{qm2pi.rhodetails}

\end{document}

 

%\documentclass[12pt]{llncs}
%\documentclass{jktr}

\usepackage[pdftex]{hyperref}                   
\usepackage {listings}
\usepackage {mathpartir}
\usepackage{bcprules}
%\usepackage{listings}
                       
\usepackage{graphicx} 
%\usepackage[margins=2.5cm,nohead,nofoot]{geometry}
%\usepackage{geometry}
\usepackage{amsfonts}
\usepackage{amstext}
\usepackage{latexsym}
\usepackage{amssymb}
\usepackage{color}


%\include{myPreamble}
\include{qm2pi.local} 

%\ifpdf
%\usepackage[pdftex]{graphicx}
%\else
%\usepackage{graphicx}
%\fi

 % \ifpdf
%  \usepackage{pdfsync}
%  \if


%\title{Brief Article}
%\author{David F. Snyder}
%\author{L.G. Meredith}

%\address{Dept. of Math., Texas State University--San Marcos, San Marcos, TX 78666}
       
\pagestyle{empty}


\begin{document}

\lstset{language=[Objective]Caml,frame=shadowbox}

\input{qm2pi.front}

% section front matter (end)

\input{qm2pi.intro} 
 
% section introduction (end)

% \input{qm2pi.knotations} 

% section notation (end)

\input{qm2pi.process.calculi} 

% section concurrent_process_calculi_and_spatial_logics_ (end)
    
%\input{qm2pi.knots2pi} 

%\input{qm2pi.trefoil} 

%\input{qm2pi.mainthm} 

% subsection basic_interpretation (end)

%\input{qm2pi.rho.presentation} 
\subsection{The syntax and semantics of the notation system}\label{sub:the_syntax_and_semantics_of_the_notation_system} % (fold)

We now summarize a technical presentation of the calculus that
embodies our theory of dynamics. The typical presentation of such a
calculus follows the style of giving generators and relations on
them. The grammar, below, describing term constructors, freely
generates the set of processes, $\Proc$. This set is then quotiented
by a relation known as structural congruence and it is over this set
that the notion of dynamics is expressed. This presentation is
essentially that of \cite{MeredithR05} with the addition of
polyadicity and summation. For readability we have relegated some of
the technical subtleties to an appendix.

\subsubsection{Process grammar}\label{subsub:process_grammar}

\begin{mathpar}
  \inferrule* [lab=synchronization] {} {{M} \bc \pzero \;|\; x?F \;|\; x!C }
  \and
  \inferrule* [lab=abstraction] {} {{F} \bc (x)P}
  \and
  \inferrule* [lab=concretion] {} {{C} \bc \langle Q \rangle}
  \and
  \inferrule* [lab=process] {} {{P,Q} \bc M \;| \;P|Q \;|\; @{x}}
  \and
  \inferrule* [lab=name] {} {{x} \bc \quotep{P}}
\end{mathpar} 

Note that $\vec{x}$ (resp. $\vec{P}$) denotes a vector of names
(resp. processes) of length $|\vec{x}|$ (resp. $|\vec{P}|$). We adopt
the following useful abbreviations.

\begin{mathpar}
   x?(\vec{y}).P := x.(\vec{y})P \and  x\clift{\vec{P}} := x.\clift{\vec{P}}
   \and x!(y) := \lift{x}{\dropn{y}}
   \and \Pi_{i=0}^{n-1}P_i := P_0 | \ldots | P_{n-1}
\end{mathpar}

\subsubsection{Structural congruence}

\paragraph{Free and bound names and alpha-equivalence.} At the
core of structural equivalence is alpha-equivalence which identifies
process that are the same up to a change of variable. Formally, we
recognize the distinction between free and bound names. The free names
of a process, $\freenames{P}$, may be calculated recursively as
follows:

\begin{mathpar}
\freenames{\pzero} := \emptyset
  \and \\
  \freenames{x?(y).P} := \{ x \} \cup (\freenames{P} \setminus \{ y \})
  \and 
  \freenames{x!\langle P \rangle} := \{ x \} \cup \{ P \} 
  \and \\
  \freenames{P|Q} := \freenames{P} \cup \freenames{Q}
  \and \\
  \freenames{@{x}} := \{ x \}
\end{mathpar}

$\pi$
$\quotep{\pi}$

$\freenames{-} : \pi \to \mathcal{P}(\quotep{\pi})$

\begin{eqnarray*}
  \freenames{\pzero} & := & \emptyset \\
  \freenames{x?(y).P} & := & \{ x \} \cup (\freenames{P} \setminus \{ y \}) \\
  \freenames{x!\langle P \rangle} & := & \{ x \} \cup \{ P \} \\
  \freenames{P|Q} & := & \freenames{P} \cup \freenames{Q} \\
  \freenames{\dropn{x}} & := & \{ x \}
\end{eqnarray*}

The bound names of a process, $\boundnames{P}$, are those names occurring in $P$
that are not free. For example, in $x?(y).0$, the name $x$ is free, while $y$ is bound.

\begin{mathpar}
  \inferrule* [lab=monoidal-laws] {} { P|Q \equiv Q|P \and P|0 \equiv P \and P|(Q|R) \equiv (P|Q)|R }
\end{mathpar}

\begin{mathpar}
  \inferrule* [lab=alpha-equivalence] {} { (x)P \equiv (y)P\{y/x\} \and y \not\in \freenames{P} }
\end{mathpar}

\begin{definition}
Then two processes, $P,Q$, are alpha-equivalent if $P = Q\{\vec{y}/\vec{x}\}$ for
some $\vec{x} \in \boundnames{Q},\vec{y} \in \boundnames{P}$, where $Q\{\vec{y}/\vec{x}\}$
denotes the capture-avoiding substitution of $\vec{y}$ for $\vec{x}$ in $Q$.
\end{definition}

\begin{definition}
  The {\em structural congruence} \cite{SangiorgiWalker} , $\equiv$,
  between processes is the least congruence containing
  alpha-equivalence, satisfying the abelian monoid laws
  (associativity, commutativity and $\pzero$ as identity) for parallel
  composition $|$ and for summation $+$.
\end{definition}

\subsection{Name equivalence}

We take name equivalence, written $\nameeq$, to be the smallest
equivalence relation generated by the following rules.

\begin{mathpar}
\inferrule*[lab=Quote-drop]
{ }
{ \quotep{@{x}} \nameeq x }

\inferrule*[lab=Struct-equiv]
{ P \scong Q }
{ \quotep{P} \nameeq \quotep{Q} }
\end{mathpar}

The astute reader will have noticed that the mutual recursion of names
and processes imposes a mutual recursion on alpha-equivalence and
structural equivalence via name-equivalence. Fortunately, all of this
works out pleasantly and we may calculate in the natural way, free of
concern. The reader interested in the details is referred to the
appendix \ref{appendix:rho_details}.

\subsection{Substitution}

We use $\Proc$ for the set of processes, $\QProc$ for the set of
names, and $\id{\{}\vec{y} / \vec{x} \id{\}}$ to denote partial maps,
$s : \QProc \rightarrow \QProc$. A map, $s$ lifts, uniquely, to a map
on process terms, $\widehat{s} : \Proc \rightarrow \Proc$ by the
following equations.

\begin{mathpar}
  (0) \psubstp{Q}{P} := 0 \\
  (R \juxtap S) \psubstp{Q}{P}
  :=    
  (R)\psubstp{Q}{P} \juxtap (S) \psubstp{Q}{P} \\
  (x?(y).R) \psubstp{Q}{P}    
  :=    
  (x)\substp{Q}{P} (z)\concat( (R \psubstn{z}{y}) \psubstp{Q}{P} ) \\
  (\lift{x}{R}) \psubstp{Q}{P}  
  :=
  \lift{(x)\substp{Q}{P}}{ R \psubstp{Q}{P} } \\
%   (\dropn{x})  \psubstp{Q}{P}       
%   := 
%   \left\{ 
%     \begin{array}{ccc} 
%       \dropn{\quotep{Q}} & & x \nameeq \quotep{P} \\
%       \dropn{x} & & otherwise \\
%     \end{array}
%   \right. 
  (\dropn{x})  \psubstp{Q}{P}       
  := 
  \left\{ 
    \begin{array}{ccc} 
      Q & & x \nameeq \quotep{P} \\
      \dropn{x} & & otherwise \\
    \end{array}
  \right.
\end{mathpar}
 

where

\begin{eqnarray}
  (x)\id{\{} \lpquote Q \rpquote / \lpquote P \rpquote \id{\}}            = 
  \left\{ 
    \begin{array}{ccc}
      \lpquote Q \rpquote & & x \nameeq \lpquote P \rpquote \\
      x & & otherwise \\
    \end{array}
  \right. \nonumber
\end{eqnarray}

and $z$ is chosen distinct from $\quotep{P}$, $\quotep{Q}$, the free
names in $Q$, and all the names in $R$. Our $\alpha$-equivalence will
be built in the standard way from this substitution.

\begin{remark}\label{rem:no_self_referential_names}
  One consequence of these definitions is that $\forall P. \quotep{P}
  \not\in \freenames{P}$.
\end{remark}

\subsection{ Dynamic quote: an example }

Anticipating something of what's to come, consider applying the
substitution, $\widehat{\id{\{}u / z \id{\}}}$, to the following pair
of processes, $\lift{w}{y!(z)}$ and $w[ \lpquote y!(z) \rpquote ]$.

\begin{eqnarray}
	\lift{w}{y!(z)}\widehat{\id{\{}u / z \id{\}}}
		& = &
		\lift{w}{y!(u)} \nonumber\\
	w[ \lpquote y!(z) \rpquote ] \widehat{ \id{\{}u / z \id{\}} }
		& = &
		w[ \lpquote y!(z) \rpquote ] \nonumber
\end{eqnarray}

Because the body of the process between quotes is impervious to
substitution, we get radically different answers. In fact, by
examining the first process in an input context,
e.g. $x?(z).\lift{w}{y!(z)}$, we see that the process under the lift
operator may be shaped by prefixed inputs binding a name inside it. In
this sense, the lift operator will be seen as a way to dynamically
construct processes before reifying them as names.

Finally equipped with these standard features we can present the
dynamics of the calculus.

\subsubsection{Operational semantics} 

Finally, we introduce the computational dynamics. What marks these
algebras as distinct from other more traditionally studied algebraic
structures, e.g. vector spaces or polynomial rings, is the manner in
which dynamics is captured. In traditional structures, dynamics is typically
expressed through morphisms between such structures, as in linear maps
between vector spaces or morphisms between rings. In algebras
associated with the semantics of computation, the dynamics is
expressed as part of the algebraic structure itself, through a
reduction reduction relation typically denoted by $\red$. Below, we
give a recursive presentation of this relation for the calculus used
in the encoding.

$\red \subseteq \pi \times \pi$
$\red : \pi \to \mathcal{P}(\pi)$

\begin{mathpar}
  \inferrule* [lab=Comm] { \textsf{match}( x_{src}, x_{trgt} ) } { x_{trgt}?(y)P \; | \; x_{src}!\langle {Q} \rangle \red P\{\quotep{Q}/y}\} }
  \and \\
  \inferrule* [lab=Par] {{P} \red {P}'} {{{P} | {Q}} \red {{P}' | {Q}}}
  \and
  \inferrule* [lab=Equiv]{{{P} \scong {P}'} \andalso {{P}' \red {Q}'} \andalso {{Q}' \scong {Q}}}{{P} \red {Q}}
\end{mathpar}

\begin{eqnarray*}
  match_{\equiv} (\quotep{P},\quotep{Q}) & := & P \equiv Q \\
  match_{\dagger}(\quotep{P},\quotep{Q}) & := & \forall R. P|Q \red^{*} R => R \red^{*} 0 \\
  match_{K}(\quotep{P},\quotep{Q}) & := & K \mbox{ for some context } K
\end{eqnarray*}

$u?(x)P | u!\langle Q \rangle \red P\{\quotep{Q}/x\}$

%We write $\wred$ for $\red^*$, and $P\red$ if $\exists Q $ such that $ P \red Q$.
We write $P\red$ if $\exists Q $ such that $ P \red Q$ and $P\not\red$, otherwise.

\section{Replication}

As mentioned before, it is known that replication (and hence
recursion) can be implemented in a higher-order process algebra
\cite{SangiorgiWalker}. As our first example of calculation with the
machinery thus far presented we give the construction explicitly in
the {\rhoc}.

\begin{eqnarray}
	D_{x} & := & \prefix{x}{y}{(\binpar{\outputp{x}{y}}{@{y}})} \nonumber\\
	\bangp_{x}{P} & := & \binpar{{x}!\langle{\binpar{D_{x}}{P}}\rangle}{D_{x}} \nonumber
\end{eqnarray}

\begin{eqnarray}
	\bangp_{x}{P} & & \nonumber\\
	=
	& {x}!\langle{(\prefix{x}{y}{(\outputp{x}{y} | @{y})) | P}}\rangle 
	      | \prefix{x}{y}{(\outputp{x}{y} | @{y})} & \nonumber\\
	\red
	& (\outputp{x}{y} | @{y})\substn{\quotep{(\prefix{x}{y}{(@{y} | \outputp{x}{y})) | P}}}{y} & \nonumber\\
	=
	& \outputp{x}{\quotep{(\prefix{x}{y}{(\outputp{x}{y} | @{y})) | P}}}
	  | {(\prefix{x}{y}{(\outputp{x}{y} | @{y})) | P}} & \nonumber\\
	\red
	& \ldots & \nonumber\\
	\red^*
	& P | P | \ldots & \nonumber
\end{eqnarray}

Of course, this encoding, as an implementation, runs away, unfolding
$\bangp{P}$ eagerly. A lazier and more implementable replication
operator, restricted to input-guarded processes, may be obtained as follows.

\begin{eqnarray}
\bangp{\prefix{u}{v}{P}} 
	:= 
	\binpar{\lift{x}{\prefix{u}{v}{(\binpar{D(x)}{P})}}}{D(x)} \nonumber
\end{eqnarray}

\begin{remark}
  Note that the lazier definition still does not deal with summation
  or mixed summation (i.e. sums over input and output). The reader is
  invited to construct definitions of replication that deal with these
  features. 

  Further, the definitions are parameterized in a name, $x$. Can you,
  gentle reader, make a definition that eliminates this parameter and
  guarantees no accidental interaction between the replication
  machinery and the process being replicated -- i.e. no accidental
  sharing of names used by the process to get its work done and the
  name(s) used by the replication to effect copying. This latter
  revision of the definition of replication is crucial to obtaining
  the expected identity $!!P \sim !P$.
\end{remark}

\begin{remark}\label{rem:paradoxical_combinator}
  The reader familiar with the lambda calculus will have noticed the
  similarity between $D$ and the paradoxical combinator.

  [Ed. note: the existence of this seems to suggest we have to be more
  restrictive on the set of processes and names we admit if we are to
  support no-cloning.]
\end{remark}

\subsubsection{Bisimulation}

The computational dynamics gives rise to another kind of equivalence,
the equivalence of computational behavior. As previously mentioned
this is typically captured \emph{via} some form of bisimulation.

% The notion we use in this paper is weak barbed bisimulation
% \cite{milner91polyadicpi}.

The notion we use in this paper is derived from weak barbed
bisimulation \cite{milner91polyadicpi}. 

\begin{definition}
An \emph{observation relation}, $\downarrow_{\mathcal N}$, over a set
of names, $\mathcal N$, is the smallest relation satisfying the rules
below.

\infrule[Out-barb]{y \in {\mathcal N}, \; x \nameeq y}
		  {\outputp{x}{v} \downarrow_{\mathcal N} x}
\infrule[Par-barb]{\mbox{$P\downarrow_{\mathcal N} x$ or $Q\downarrow_{\mathcal N} x$}}
		  {\binpar{P}{Q} \downarrow_{\mathcal N} x}

We write $P \Downarrow_{\mathcal N} x$ if there is $Q$ such that 
$P \wred Q$ and $Q \downarrow_{\mathcal N} x$.
\end{definition}

\begin{definition}
%\label{def.bbisim}
An  ${\mathcal N}$-\emph{barbed bisimulation} over a set of names, ${\mathcal N}$, is a symmetric binary relation 
${\mathcal S}_{\mathcal N}$ between agents such that $P\rel{S}_{\mathcal N}Q$ implies:
\begin{enumerate}
\item If $P \red P'$ then $Q \wred Q'$ and $P'\rel{S}_{\mathcal N} Q'$.
\item If $P\downarrow_{\mathcal N} x$, then $Q\Downarrow_{\mathcal N} x$.
\end{enumerate}
$P$ is ${\mathcal N}$-barbed bisimilar to $Q$, written
$P \wbbisim_{\mathcal N} Q$, if $P \rel{S}_{\mathcal N} Q$ for some ${\mathcal N}$-barbed bisimulation ${\mathcal S}_{\mathcal N}$.
\end{definition}

$\mathcal{R} \subseteq \pi \times \pi$

$P \mathcal{R} Q => \forall P'. P \red P' \Rightarrow \exists Q'. Q \red Q', P' \mathcal{R} Q'$

$P \vdash x \Rightarrow Q \vdash x$

\begin{mathpar}
  \inferrule*[lab=Out-barb]{x \nameeq y}{{y}!\langle{Q}\rangle \vdash x}
  \and
  \inferrule*[lab=Par-barb]{\mbox{$P\vdash x$ or $Q\vdash x$}}{\binpar{P}{Q} \vdash x}
\end{mathpar}

\subsubsection{Contexts}

One of the principle advantages of computational calculi like the
$\pi$-calculus is a well-defined notion of context,
contextual-equivalence and a correlation between
contextual-equivalence and notions of bisimulation. The notion of
context allows the decomposition of a process into (sub-)process and
its syntactic environment, its context. Thus, a context may be
thought of as a process with a ``hole'' (written $\Box$) in it. The
application of a context $M$ to a process $P$, written $M[P]$, is
tantamount to filling the hole in $M$ with $P$. In this paper we do
not need the full weight of this theory, but do make use of the notion
of context in the proof the main theorem. 

\begin{mathpar}
  \inferrule* [lab=summation] {} {{M_{M},M_{N}} \bc \Box \;|\; x.M_{A} \;|\; M_{M}+M_{N}}
  \and
  \inferrule* [lab=agent] {} {{M_{A}} \bc (\vec{x})M_{P} \;| \; \clift{P_0,\ldots,M_{P},\ldots,P_N}}
  \and \\
  \inferrule* [lab=process] {} {{M_{P}} \bc M_{N} \;| \;P|M_{P} }
\end{mathpar} 

\begin{mathpar}
  \inferrule* [lab=sychronization] {} {M_{N} \bc \Box \;|\; x?M_{F} \;|\; x!M_{C}}
  \and
  \inferrule* [lab=abstraction] {} {{M_{F}} \bc (x)M_{P} }
  \and
  \inferrule* [lab=concretion] {} {{M_{C}} \bc \langle M_{P} \rangle }
  \and \\
  \inferrule* [lab=process] {} {{M_{P}} \bc M_{N} \;| \;P|M_{P} }
\end{mathpar}

\begin{definition}[contextual application] Given a context $M$, and
  process $P$, we define the \emph{contextual application}, $M[P] :=
  M\{P/\Box\}$. That is, the contextual application of M to P is the
  substitution of $P$ for $\Box$ in $M$.
\end{definition}

$\meaningof{-} : L \to \mathcal{P}(\pi)$

\begin{mathpar}
  \inferrule* [lab=collection] {} {\meaningof{true} = \pi, \and \meaningof{~E} = \pi \setminus \meaningof{E}, \and \meaningof{E_{1} \& E_{2}} = \meaningof{E_{1}} \cap \meaningof{E_{2}}}
\end{mathpar}

\begin{mathpar}
  \inferrule* [lab=structure] {} {\meaningof{0} = \{ P \in \pi | P \equiv 0 \}, \and \\ \meaningof{E_1 | E_2} = \{ P \in \pi | P \equiv P_{1} | P_{2}, P_{1} \in \meaningof{E_{1}}, P_{2} \in \meaningof{E_2}\} }
\end{mathpar}

\begin{mathpar}
 \inferrule* [lab=behavior] {} {\meaningof{\langle a?b \rangle E} = \{ P \in \pi | P \equiv Q | u?(y)P', \\ \and \\\\ \and \\ \;\;\; u \in \meaningof{a}, \forall z.P'\{z/y\} \in \meaningof{E\{z/b\}}\}, \and \\ \meaningof{a!E} = \{ P \in \pi | P \equiv Q | x!\langle P' \rangle, x \in \meaningof{a} P' \in \meaningof{E}\} }
\end{mathpar}

\begin{mathpar}
 \inferrule* [lab=nominal] {} {\meaningof{\quotep{E}} = \{ \quotep{P} \in \quotep{\pi} | P \in \meaningof{E} \}, \and \meaningof{\quotep{P}} = \{ \quotep{Q} \in \quotep{\pi} | P \equiv Q \} \and \\ \meaningof{@\quotep{E}} = \{ P \in \pi | P \equiv @x, x \in \meaningof{E} \}}
\end{mathpar}

\begin{eqnarray*}
  \\
  \meaningof{-} : TS \to ST
\end{eqnarray*}

\begin{eqnarray*}
  \\
  L : TS \to ST
\end{eqnarray*}

\begin{eqnarray*}
  \\
  P \models E \iff P \in \meaningof{E}
\end{eqnarray*}

\begin{eqnarray*}
  P \approx_{L} Q \iff \forall E \in L. P \models E \iff Q \models E
\end{eqnarray*}

\begin{eqnarray*}
  P \approx_{K} Q
\end{eqnarray*}

\begin{eqnarray*}
  P \approx Q
\end{eqnarray*}

$\approx_{K} = \approx = \approx_{L}$

\subsubsection{Contextual duality}

Note that contexts extend the quotation operation to a family of
operations from processes to names. Given a context, $M$, we can
define a \emph{nominal context}, $\quotep{M}$ by $\quotep{M}[P] :=
\quotep{M[P]}$. To foreshadow what is to come we observe that these
operations enjoy a duality with processes very much like the duality
between vectors and maps from vectors to scalars.

Further, because the calculus is essentially higher-order, we have a
correspondence between contexts and processes. More specifically,
given a name $x$ and a context $M$ we can construct $M^{*}_{x}$ such
that 

\begin{mathpar}
  M^{*}_{x} | \lift{x}{P} \red M[P]
\end{mathpar}

namely,

\begin{mathpar}
  M^{*}_{x} := x?(u).M[\dropn{u}]
\end{mathpar}

The dependence of $M^{*}_{x}$ on a name makes it an abstraction, 

\begin{mathpar}
  M^{*} := (x)x?(u).M[\dropn{u}]
\end{mathpar}

\subsection{Additional notation}

It will sometimes be convenient to denote the process a name
quotes. We already have the notation $x = \quotep{P}$, but it will be
convenient to introduce an alternate notation, $\procn{x}$, when we
want to emphasize the connection to the use of the name. Note that, by
virtue of name equivalence, $\quotep{\procn{x}} \nameeq x$; so, the
notation is consistent with previous definitions.

Further, because names have structure it is possible to effect
substitutions on the basis of that structure. This means we need to
upgrade our notation for substitutions, which we accomplish by
adapting comprehension notation. Thus,

\begin{mathpar}
  P\{ y / x : x \in S \}
\end{mathpar}

is interpreted to mean the process derived from P by replacing (in a
capture-avoiding manner) each occurrence of $x$ in $S$ by $y$. For example,

\begin{mathpar}
  P\{ \quotep{\procn{x}|\procn{x}} / x : x \in \freenames{P} \}
\end{mathpar}

will replace each (occurrence) of a free name $x$ in $P$ by
$\quotep{\procn{x}|\procn{x}}$.

Also, we will avail ourselves of the notation $x^{L}$ and $x^{R}$ to
denote injections of a name into disjoint copies of the name
space. There are numerous ways to accomplish this. One example can be
found in \cite{MeredithR05}. This notation overloads to vectors of
names: $\vec{x}^{\pi} := (x_{i}^{\pi} \; : \; 0 \leq i < |\vec{x}| )$ where $\pi \in \{L,R\}$.

We also use $P^{\Box} := P|\Box$.

In \cite{MeredithR05} an interpretation of the new operator is
given. It turns out that there are several possible interpretations
all enjoying the requisite algebraic properties of the operator (see
\cite{milner91polyadicpi}). We will therefore make liberal use of
$(\nu\; \vec{x})P$.

% subsection the_syntax_and_semantics_of_the_notation_system (end)   

\input{qm2pi.qmops} 

\input{qm2pi.sterngerlach} 

\input{qm2pi.metric} 

% section concurrent_process_calculi (end)

%\input{qm2pi.proofsketch}

% section proof sketch (end)

%\input{qm2pi.slviaknots} 

% section spatial logic via knots (end)

\input{qm2pi.conclusion}

% section conclusion (end)

%\input{qm2pi.dtcodes} 

% section wiring algorithm (end)

\input{qm2pi.ack} 

% section acknowledgments (end)

\newpage


\bibliographystyle{plain}   
\bibliography{../../biblios/main.bib}

\input{qm2pi.rhodetails}

\end{document}

 

%\documentclass[12pt]{llncs}
%\documentclass{jktr}

\usepackage[pdftex]{hyperref}                   
\usepackage {listings}
\usepackage {mathpartir}
\usepackage{bcprules}
%\usepackage{listings}
                       
\usepackage{graphicx} 
%\usepackage[margins=2.5cm,nohead,nofoot]{geometry}
%\usepackage{geometry}
\usepackage{amsfonts}
\usepackage{amstext}
\usepackage{latexsym}
\usepackage{amssymb}
\usepackage{color}


%\include{myPreamble}
\include{qm2pi.local} 

%\ifpdf
%\usepackage[pdftex]{graphicx}
%\else
%\usepackage{graphicx}
%\fi

 % \ifpdf
%  \usepackage{pdfsync}
%  \if


%\title{Brief Article}
%\author{David F. Snyder}
%\author{L.G. Meredith}

%\address{Dept. of Math., Texas State University--San Marcos, San Marcos, TX 78666}
       
\pagestyle{empty}


\begin{document}

\lstset{language=[Objective]Caml,frame=shadowbox}

\input{qm2pi.front}

% section front matter (end)

\input{qm2pi.intro} 
 
% section introduction (end)

% \input{qm2pi.knotations} 

% section notation (end)

\input{qm2pi.process.calculi} 

% section concurrent_process_calculi_and_spatial_logics_ (end)
    
%\input{qm2pi.knots2pi} 

%\input{qm2pi.trefoil} 

%\input{qm2pi.mainthm} 

% subsection basic_interpretation (end)

%\input{qm2pi.rho.presentation} 
\subsection{The syntax and semantics of the notation system}\label{sub:the_syntax_and_semantics_of_the_notation_system} % (fold)

We now summarize a technical presentation of the calculus that
embodies our theory of dynamics. The typical presentation of such a
calculus follows the style of giving generators and relations on
them. The grammar, below, describing term constructors, freely
generates the set of processes, $\Proc$. This set is then quotiented
by a relation known as structural congruence and it is over this set
that the notion of dynamics is expressed. This presentation is
essentially that of \cite{MeredithR05} with the addition of
polyadicity and summation. For readability we have relegated some of
the technical subtleties to an appendix.

\subsubsection{Process grammar}\label{subsub:process_grammar}

\begin{mathpar}
  \inferrule* [lab=synchronization] {} {{M} \bc \pzero \;|\; x?F \;|\; x!C }
  \and
  \inferrule* [lab=abstraction] {} {{F} \bc (x)P}
  \and
  \inferrule* [lab=concretion] {} {{C} \bc \langle Q \rangle}
  \and
  \inferrule* [lab=process] {} {{P,Q} \bc M \;| \;P|Q \;|\; @{x}}
  \and
  \inferrule* [lab=name] {} {{x} \bc \quotep{P}}
\end{mathpar} 

Note that $\vec{x}$ (resp. $\vec{P}$) denotes a vector of names
(resp. processes) of length $|\vec{x}|$ (resp. $|\vec{P}|$). We adopt
the following useful abbreviations.

\begin{mathpar}
   x?(\vec{y}).P := x.(\vec{y})P \and  x\clift{\vec{P}} := x.\clift{\vec{P}}
   \and x!(y) := \lift{x}{\dropn{y}}
   \and \Pi_{i=0}^{n-1}P_i := P_0 | \ldots | P_{n-1}
\end{mathpar}

\subsubsection{Structural congruence}

\paragraph{Free and bound names and alpha-equivalence.} At the
core of structural equivalence is alpha-equivalence which identifies
process that are the same up to a change of variable. Formally, we
recognize the distinction between free and bound names. The free names
of a process, $\freenames{P}$, may be calculated recursively as
follows:

\begin{mathpar}
\freenames{\pzero} := \emptyset
  \and \\
  \freenames{x?(y).P} := \{ x \} \cup (\freenames{P} \setminus \{ y \})
  \and 
  \freenames{x!\langle P \rangle} := \{ x \} \cup \{ P \} 
  \and \\
  \freenames{P|Q} := \freenames{P} \cup \freenames{Q}
  \and \\
  \freenames{@{x}} := \{ x \}
\end{mathpar}

$\pi$
$\quotep{\pi}$

$\freenames{-} : \pi \to \mathcal{P}(\quotep{\pi})$

\begin{eqnarray*}
  \freenames{\pzero} & := & \emptyset \\
  \freenames{x?(y).P} & := & \{ x \} \cup (\freenames{P} \setminus \{ y \}) \\
  \freenames{x!\langle P \rangle} & := & \{ x \} \cup \{ P \} \\
  \freenames{P|Q} & := & \freenames{P} \cup \freenames{Q} \\
  \freenames{\dropn{x}} & := & \{ x \}
\end{eqnarray*}

The bound names of a process, $\boundnames{P}$, are those names occurring in $P$
that are not free. For example, in $x?(y).0$, the name $x$ is free, while $y$ is bound.

\begin{mathpar}
  \inferrule* [lab=monoidal-laws] {} { P|Q \equiv Q|P \and P|0 \equiv P \and P|(Q|R) \equiv (P|Q)|R }
\end{mathpar}

\begin{mathpar}
  \inferrule* [lab=alpha-equivalence] {} { (x)P \equiv (y)P\{y/x\} \and y \not\in \freenames{P} }
\end{mathpar}

\begin{definition}
Then two processes, $P,Q$, are alpha-equivalent if $P = Q\{\vec{y}/\vec{x}\}$ for
some $\vec{x} \in \boundnames{Q},\vec{y} \in \boundnames{P}$, where $Q\{\vec{y}/\vec{x}\}$
denotes the capture-avoiding substitution of $\vec{y}$ for $\vec{x}$ in $Q$.
\end{definition}

\begin{definition}
  The {\em structural congruence} \cite{SangiorgiWalker} , $\equiv$,
  between processes is the least congruence containing
  alpha-equivalence, satisfying the abelian monoid laws
  (associativity, commutativity and $\pzero$ as identity) for parallel
  composition $|$ and for summation $+$.
\end{definition}

\subsection{Name equivalence}

We take name equivalence, written $\nameeq$, to be the smallest
equivalence relation generated by the following rules.

\begin{mathpar}
\inferrule*[lab=Quote-drop]
{ }
{ \quotep{@{x}} \nameeq x }

\inferrule*[lab=Struct-equiv]
{ P \scong Q }
{ \quotep{P} \nameeq \quotep{Q} }
\end{mathpar}

The astute reader will have noticed that the mutual recursion of names
and processes imposes a mutual recursion on alpha-equivalence and
structural equivalence via name-equivalence. Fortunately, all of this
works out pleasantly and we may calculate in the natural way, free of
concern. The reader interested in the details is referred to the
appendix \ref{appendix:rho_details}.

\subsection{Substitution}

We use $\Proc$ for the set of processes, $\QProc$ for the set of
names, and $\id{\{}\vec{y} / \vec{x} \id{\}}$ to denote partial maps,
$s : \QProc \rightarrow \QProc$. A map, $s$ lifts, uniquely, to a map
on process terms, $\widehat{s} : \Proc \rightarrow \Proc$ by the
following equations.

\begin{mathpar}
  (0) \psubstp{Q}{P} := 0 \\
  (R \juxtap S) \psubstp{Q}{P}
  :=    
  (R)\psubstp{Q}{P} \juxtap (S) \psubstp{Q}{P} \\
  (x?(y).R) \psubstp{Q}{P}    
  :=    
  (x)\substp{Q}{P} (z)\concat( (R \psubstn{z}{y}) \psubstp{Q}{P} ) \\
  (\lift{x}{R}) \psubstp{Q}{P}  
  :=
  \lift{(x)\substp{Q}{P}}{ R \psubstp{Q}{P} } \\
%   (\dropn{x})  \psubstp{Q}{P}       
%   := 
%   \left\{ 
%     \begin{array}{ccc} 
%       \dropn{\quotep{Q}} & & x \nameeq \quotep{P} \\
%       \dropn{x} & & otherwise \\
%     \end{array}
%   \right. 
  (\dropn{x})  \psubstp{Q}{P}       
  := 
  \left\{ 
    \begin{array}{ccc} 
      Q & & x \nameeq \quotep{P} \\
      \dropn{x} & & otherwise \\
    \end{array}
  \right.
\end{mathpar}
 

where

\begin{eqnarray}
  (x)\id{\{} \lpquote Q \rpquote / \lpquote P \rpquote \id{\}}            = 
  \left\{ 
    \begin{array}{ccc}
      \lpquote Q \rpquote & & x \nameeq \lpquote P \rpquote \\
      x & & otherwise \\
    \end{array}
  \right. \nonumber
\end{eqnarray}

and $z$ is chosen distinct from $\quotep{P}$, $\quotep{Q}$, the free
names in $Q$, and all the names in $R$. Our $\alpha$-equivalence will
be built in the standard way from this substitution.

\begin{remark}\label{rem:no_self_referential_names}
  One consequence of these definitions is that $\forall P. \quotep{P}
  \not\in \freenames{P}$.
\end{remark}

\subsection{ Dynamic quote: an example }

Anticipating something of what's to come, consider applying the
substitution, $\widehat{\id{\{}u / z \id{\}}}$, to the following pair
of processes, $\lift{w}{y!(z)}$ and $w[ \lpquote y!(z) \rpquote ]$.

\begin{eqnarray}
	\lift{w}{y!(z)}\widehat{\id{\{}u / z \id{\}}}
		& = &
		\lift{w}{y!(u)} \nonumber\\
	w[ \lpquote y!(z) \rpquote ] \widehat{ \id{\{}u / z \id{\}} }
		& = &
		w[ \lpquote y!(z) \rpquote ] \nonumber
\end{eqnarray}

Because the body of the process between quotes is impervious to
substitution, we get radically different answers. In fact, by
examining the first process in an input context,
e.g. $x?(z).\lift{w}{y!(z)}$, we see that the process under the lift
operator may be shaped by prefixed inputs binding a name inside it. In
this sense, the lift operator will be seen as a way to dynamically
construct processes before reifying them as names.

Finally equipped with these standard features we can present the
dynamics of the calculus.

\subsubsection{Operational semantics} 

Finally, we introduce the computational dynamics. What marks these
algebras as distinct from other more traditionally studied algebraic
structures, e.g. vector spaces or polynomial rings, is the manner in
which dynamics is captured. In traditional structures, dynamics is typically
expressed through morphisms between such structures, as in linear maps
between vector spaces or morphisms between rings. In algebras
associated with the semantics of computation, the dynamics is
expressed as part of the algebraic structure itself, through a
reduction reduction relation typically denoted by $\red$. Below, we
give a recursive presentation of this relation for the calculus used
in the encoding.

$\red \subseteq \pi \times \pi$
$\red : \pi \to \mathcal{P}(\pi)$

\begin{mathpar}
  \inferrule* [lab=Comm] { \textsf{match}( x_{src}, x_{trgt} ) } { x_{trgt}?(y)P \; | \; x_{src}!\langle {Q} \rangle \red P\{\quotep{Q}/y}\} }
  \and \\
  \inferrule* [lab=Par] {{P} \red {P}'} {{{P} | {Q}} \red {{P}' | {Q}}}
  \and
  \inferrule* [lab=Equiv]{{{P} \scong {P}'} \andalso {{P}' \red {Q}'} \andalso {{Q}' \scong {Q}}}{{P} \red {Q}}
\end{mathpar}

\begin{eqnarray*}
  match_{\equiv} (\quotep{P},\quotep{Q}) & := & P \equiv Q \\
  match_{\dagger}(\quotep{P},\quotep{Q}) & := & \forall R. P|Q \red^{*} R => R \red^{*} 0 \\
  match_{K}(\quotep{P},\quotep{Q}) & := & K \mbox{ for some context } K
\end{eqnarray*}

$u?(x)P | u!\langle Q \rangle \red P\{\quotep{Q}/x\}$

%We write $\wred$ for $\red^*$, and $P\red$ if $\exists Q $ such that $ P \red Q$.
We write $P\red$ if $\exists Q $ such that $ P \red Q$ and $P\not\red$, otherwise.

\section{Replication}

As mentioned before, it is known that replication (and hence
recursion) can be implemented in a higher-order process algebra
\cite{SangiorgiWalker}. As our first example of calculation with the
machinery thus far presented we give the construction explicitly in
the {\rhoc}.

\begin{eqnarray}
	D_{x} & := & \prefix{x}{y}{(\binpar{\outputp{x}{y}}{@{y}})} \nonumber\\
	\bangp_{x}{P} & := & \binpar{{x}!\langle{\binpar{D_{x}}{P}}\rangle}{D_{x}} \nonumber
\end{eqnarray}

\begin{eqnarray}
	\bangp_{x}{P} & & \nonumber\\
	=
	& {x}!\langle{(\prefix{x}{y}{(\outputp{x}{y} | @{y})) | P}}\rangle 
	      | \prefix{x}{y}{(\outputp{x}{y} | @{y})} & \nonumber\\
	\red
	& (\outputp{x}{y} | @{y})\substn{\quotep{(\prefix{x}{y}{(@{y} | \outputp{x}{y})) | P}}}{y} & \nonumber\\
	=
	& \outputp{x}{\quotep{(\prefix{x}{y}{(\outputp{x}{y} | @{y})) | P}}}
	  | {(\prefix{x}{y}{(\outputp{x}{y} | @{y})) | P}} & \nonumber\\
	\red
	& \ldots & \nonumber\\
	\red^*
	& P | P | \ldots & \nonumber
\end{eqnarray}

Of course, this encoding, as an implementation, runs away, unfolding
$\bangp{P}$ eagerly. A lazier and more implementable replication
operator, restricted to input-guarded processes, may be obtained as follows.

\begin{eqnarray}
\bangp{\prefix{u}{v}{P}} 
	:= 
	\binpar{\lift{x}{\prefix{u}{v}{(\binpar{D(x)}{P})}}}{D(x)} \nonumber
\end{eqnarray}

\begin{remark}
  Note that the lazier definition still does not deal with summation
  or mixed summation (i.e. sums over input and output). The reader is
  invited to construct definitions of replication that deal with these
  features. 

  Further, the definitions are parameterized in a name, $x$. Can you,
  gentle reader, make a definition that eliminates this parameter and
  guarantees no accidental interaction between the replication
  machinery and the process being replicated -- i.e. no accidental
  sharing of names used by the process to get its work done and the
  name(s) used by the replication to effect copying. This latter
  revision of the definition of replication is crucial to obtaining
  the expected identity $!!P \sim !P$.
\end{remark}

\begin{remark}\label{rem:paradoxical_combinator}
  The reader familiar with the lambda calculus will have noticed the
  similarity between $D$ and the paradoxical combinator.

  [Ed. note: the existence of this seems to suggest we have to be more
  restrictive on the set of processes and names we admit if we are to
  support no-cloning.]
\end{remark}

\subsubsection{Bisimulation}

The computational dynamics gives rise to another kind of equivalence,
the equivalence of computational behavior. As previously mentioned
this is typically captured \emph{via} some form of bisimulation.

% The notion we use in this paper is weak barbed bisimulation
% \cite{milner91polyadicpi}.

The notion we use in this paper is derived from weak barbed
bisimulation \cite{milner91polyadicpi}. 

\begin{definition}
An \emph{observation relation}, $\downarrow_{\mathcal N}$, over a set
of names, $\mathcal N$, is the smallest relation satisfying the rules
below.

\infrule[Out-barb]{y \in {\mathcal N}, \; x \nameeq y}
		  {\outputp{x}{v} \downarrow_{\mathcal N} x}
\infrule[Par-barb]{\mbox{$P\downarrow_{\mathcal N} x$ or $Q\downarrow_{\mathcal N} x$}}
		  {\binpar{P}{Q} \downarrow_{\mathcal N} x}

We write $P \Downarrow_{\mathcal N} x$ if there is $Q$ such that 
$P \wred Q$ and $Q \downarrow_{\mathcal N} x$.
\end{definition}

\begin{definition}
%\label{def.bbisim}
An  ${\mathcal N}$-\emph{barbed bisimulation} over a set of names, ${\mathcal N}$, is a symmetric binary relation 
${\mathcal S}_{\mathcal N}$ between agents such that $P\rel{S}_{\mathcal N}Q$ implies:
\begin{enumerate}
\item If $P \red P'$ then $Q \wred Q'$ and $P'\rel{S}_{\mathcal N} Q'$.
\item If $P\downarrow_{\mathcal N} x$, then $Q\Downarrow_{\mathcal N} x$.
\end{enumerate}
$P$ is ${\mathcal N}$-barbed bisimilar to $Q$, written
$P \wbbisim_{\mathcal N} Q$, if $P \rel{S}_{\mathcal N} Q$ for some ${\mathcal N}$-barbed bisimulation ${\mathcal S}_{\mathcal N}$.
\end{definition}

$\mathcal{R} \subseteq \pi \times \pi$

$P \mathcal{R} Q => \forall P'. P \red P' \Rightarrow \exists Q'. Q \red Q', P' \mathcal{R} Q'$

$P \vdash x \Rightarrow Q \vdash x$

\begin{mathpar}
  \inferrule*[lab=Out-barb]{x \nameeq y}{{y}!\langle{Q}\rangle \vdash x}
  \and
  \inferrule*[lab=Par-barb]{\mbox{$P\vdash x$ or $Q\vdash x$}}{\binpar{P}{Q} \vdash x}
\end{mathpar}

\subsubsection{Contexts}

One of the principle advantages of computational calculi like the
$\pi$-calculus is a well-defined notion of context,
contextual-equivalence and a correlation between
contextual-equivalence and notions of bisimulation. The notion of
context allows the decomposition of a process into (sub-)process and
its syntactic environment, its context. Thus, a context may be
thought of as a process with a ``hole'' (written $\Box$) in it. The
application of a context $M$ to a process $P$, written $M[P]$, is
tantamount to filling the hole in $M$ with $P$. In this paper we do
not need the full weight of this theory, but do make use of the notion
of context in the proof the main theorem. 

\begin{mathpar}
  \inferrule* [lab=summation] {} {{M_{M},M_{N}} \bc \Box \;|\; x.M_{A} \;|\; M_{M}+M_{N}}
  \and
  \inferrule* [lab=agent] {} {{M_{A}} \bc (\vec{x})M_{P} \;| \; \clift{P_0,\ldots,M_{P},\ldots,P_N}}
  \and \\
  \inferrule* [lab=process] {} {{M_{P}} \bc M_{N} \;| \;P|M_{P} }
\end{mathpar} 

\begin{mathpar}
  \inferrule* [lab=sychronization] {} {M_{N} \bc \Box \;|\; x?M_{F} \;|\; x!M_{C}}
  \and
  \inferrule* [lab=abstraction] {} {{M_{F}} \bc (x)M_{P} }
  \and
  \inferrule* [lab=concretion] {} {{M_{C}} \bc \langle M_{P} \rangle }
  \and \\
  \inferrule* [lab=process] {} {{M_{P}} \bc M_{N} \;| \;P|M_{P} }
\end{mathpar}

\begin{definition}[contextual application] Given a context $M$, and
  process $P$, we define the \emph{contextual application}, $M[P] :=
  M\{P/\Box\}$. That is, the contextual application of M to P is the
  substitution of $P$ for $\Box$ in $M$.
\end{definition}

$\meaningof{-} : L \to \mathcal{P}(\pi)$

\begin{mathpar}
  \inferrule* [lab=collection] {} {\meaningof{true} = \pi, \and \meaningof{~E} = \pi \setminus \meaningof{E}, \and \meaningof{E_{1} \& E_{2}} = \meaningof{E_{1}} \cap \meaningof{E_{2}}}
\end{mathpar}

\begin{mathpar}
  \inferrule* [lab=structure] {} {\meaningof{0} = \{ P \in \pi | P \equiv 0 \}, \and \\ \meaningof{E_1 | E_2} = \{ P \in \pi | P \equiv P_{1} | P_{2}, P_{1} \in \meaningof{E_{1}}, P_{2} \in \meaningof{E_2}\} }
\end{mathpar}

\begin{mathpar}
 \inferrule* [lab=behavior] {} {\meaningof{\langle a?b \rangle E} = \{ P \in \pi | P \equiv Q | u?(y)P', \\ \and \\\\ \and \\ \;\;\; u \in \meaningof{a}, \forall z.P'\{z/y\} \in \meaningof{E\{z/b\}}\}, \and \\ \meaningof{a!E} = \{ P \in \pi | P \equiv Q | x!\langle P' \rangle, x \in \meaningof{a} P' \in \meaningof{E}\} }
\end{mathpar}

\begin{mathpar}
 \inferrule* [lab=nominal] {} {\meaningof{\quotep{E}} = \{ \quotep{P} \in \quotep{\pi} | P \in \meaningof{E} \}, \and \meaningof{\quotep{P}} = \{ \quotep{Q} \in \quotep{\pi} | P \equiv Q \} \and \\ \meaningof{@\quotep{E}} = \{ P \in \pi | P \equiv @x, x \in \meaningof{E} \}}
\end{mathpar}

\begin{eqnarray*}
  \\
  \meaningof{-} : TS \to ST
\end{eqnarray*}

\begin{eqnarray*}
  \\
  L : TS \to ST
\end{eqnarray*}

\begin{eqnarray*}
  \\
  P \models E \iff P \in \meaningof{E}
\end{eqnarray*}

\begin{eqnarray*}
  P \approx_{L} Q \iff \forall E \in L. P \models E \iff Q \models E
\end{eqnarray*}

\begin{eqnarray*}
  P \approx_{K} Q
\end{eqnarray*}

\begin{eqnarray*}
  P \approx Q
\end{eqnarray*}

$\approx_{K} = \approx = \approx_{L}$

\subsubsection{Contextual duality}

Note that contexts extend the quotation operation to a family of
operations from processes to names. Given a context, $M$, we can
define a \emph{nominal context}, $\quotep{M}$ by $\quotep{M}[P] :=
\quotep{M[P]}$. To foreshadow what is to come we observe that these
operations enjoy a duality with processes very much like the duality
between vectors and maps from vectors to scalars.

Further, because the calculus is essentially higher-order, we have a
correspondence between contexts and processes. More specifically,
given a name $x$ and a context $M$ we can construct $M^{*}_{x}$ such
that 

\begin{mathpar}
  M^{*}_{x} | \lift{x}{P} \red M[P]
\end{mathpar}

namely,

\begin{mathpar}
  M^{*}_{x} := x?(u).M[\dropn{u}]
\end{mathpar}

The dependence of $M^{*}_{x}$ on a name makes it an abstraction, 

\begin{mathpar}
  M^{*} := (x)x?(u).M[\dropn{u}]
\end{mathpar}

\subsection{Additional notation}

It will sometimes be convenient to denote the process a name
quotes. We already have the notation $x = \quotep{P}$, but it will be
convenient to introduce an alternate notation, $\procn{x}$, when we
want to emphasize the connection to the use of the name. Note that, by
virtue of name equivalence, $\quotep{\procn{x}} \nameeq x$; so, the
notation is consistent with previous definitions.

Further, because names have structure it is possible to effect
substitutions on the basis of that structure. This means we need to
upgrade our notation for substitutions, which we accomplish by
adapting comprehension notation. Thus,

\begin{mathpar}
  P\{ y / x : x \in S \}
\end{mathpar}

is interpreted to mean the process derived from P by replacing (in a
capture-avoiding manner) each occurrence of $x$ in $S$ by $y$. For example,

\begin{mathpar}
  P\{ \quotep{\procn{x}|\procn{x}} / x : x \in \freenames{P} \}
\end{mathpar}

will replace each (occurrence) of a free name $x$ in $P$ by
$\quotep{\procn{x}|\procn{x}}$.

Also, we will avail ourselves of the notation $x^{L}$ and $x^{R}$ to
denote injections of a name into disjoint copies of the name
space. There are numerous ways to accomplish this. One example can be
found in \cite{MeredithR05}. This notation overloads to vectors of
names: $\vec{x}^{\pi} := (x_{i}^{\pi} \; : \; 0 \leq i < |\vec{x}| )$ where $\pi \in \{L,R\}$.

We also use $P^{\Box} := P|\Box$.

In \cite{MeredithR05} an interpretation of the new operator is
given. It turns out that there are several possible interpretations
all enjoying the requisite algebraic properties of the operator (see
\cite{milner91polyadicpi}). We will therefore make liberal use of
$(\nu\; \vec{x})P$.

% subsection the_syntax_and_semantics_of_the_notation_system (end)   

\input{qm2pi.qmops} 

\input{qm2pi.sterngerlach} 

\input{qm2pi.metric} 

% section concurrent_process_calculi (end)

%\input{qm2pi.proofsketch}

% section proof sketch (end)

%\input{qm2pi.slviaknots} 

% section spatial logic via knots (end)

\input{qm2pi.conclusion}

% section conclusion (end)

%\input{qm2pi.dtcodes} 

% section wiring algorithm (end)

\input{qm2pi.ack} 

% section acknowledgments (end)

\newpage


\bibliographystyle{plain}   
\bibliography{../../biblios/main.bib}

\input{qm2pi.rhodetails}

\end{document}

 

% subsection basic_interpretation (end)

%\input{qm2pi.rho.presentation} 
\subsection{The syntax and semantics of the notation system}\label{sub:the_syntax_and_semantics_of_the_notation_system} % (fold)

We now summarize a technical presentation of the calculus that
embodies our theory of dynamics. The typical presentation of such a
calculus follows the style of giving generators and relations on
them. The grammar, below, describing term constructors, freely
generates the set of processes, $\Proc$. This set is then quotiented
by a relation known as structural congruence and it is over this set
that the notion of dynamics is expressed. This presentation is
essentially that of \cite{MeredithR05} with the addition of
polyadicity and summation. For readability we have relegated some of
the technical subtleties to an appendix.

\subsubsection{Process grammar}\label{subsub:process_grammar}

\begin{mathpar}
  \inferrule* [lab=synchronization] {} {{M} \bc \pzero \;|\; x?F \;|\; x!C }
  \and
  \inferrule* [lab=abstraction] {} {{F} \bc (x)P}
  \and
  \inferrule* [lab=concretion] {} {{C} \bc \langle Q \rangle}
  \and
  \inferrule* [lab=process] {} {{P,Q} \bc M \;| \;P|Q \;|\; @{x}}
  \and
  \inferrule* [lab=name] {} {{x} \bc \quotep{P}}
\end{mathpar} 

Note that $\vec{x}$ (resp. $\vec{P}$) denotes a vector of names
(resp. processes) of length $|\vec{x}|$ (resp. $|\vec{P}|$). We adopt
the following useful abbreviations.

\begin{mathpar}
   x?(\vec{y}).P := x.(\vec{y})P \and  x\clift{\vec{P}} := x.\clift{\vec{P}}
   \and x!(y) := \lift{x}{\dropn{y}}
   \and \Pi_{i=0}^{n-1}P_i := P_0 | \ldots | P_{n-1}
\end{mathpar}

\subsubsection{Structural congruence}

\paragraph{Free and bound names and alpha-equivalence.} At the
core of structural equivalence is alpha-equivalence which identifies
process that are the same up to a change of variable. Formally, we
recognize the distinction between free and bound names. The free names
of a process, $\freenames{P}$, may be calculated recursively as
follows:

\begin{mathpar}
\freenames{\pzero} := \emptyset
  \and \\
  \freenames{x?(y).P} := \{ x \} \cup (\freenames{P} \setminus \{ y \})
  \and 
  \freenames{x!\langle P \rangle} := \{ x \} \cup \{ P \} 
  \and \\
  \freenames{P|Q} := \freenames{P} \cup \freenames{Q}
  \and \\
  \freenames{@{x}} := \{ x \}
\end{mathpar}

$\pi$
$\quotep{\pi}$

$\freenames{-} : \pi \to \mathcal{P}(\quotep{\pi})$

\begin{eqnarray*}
  \freenames{\pzero} & := & \emptyset \\
  \freenames{x?(y).P} & := & \{ x \} \cup (\freenames{P} \setminus \{ y \}) \\
  \freenames{x!\langle P \rangle} & := & \{ x \} \cup \{ P \} \\
  \freenames{P|Q} & := & \freenames{P} \cup \freenames{Q} \\
  \freenames{\dropn{x}} & := & \{ x \}
\end{eqnarray*}

The bound names of a process, $\boundnames{P}$, are those names occurring in $P$
that are not free. For example, in $x?(y).0$, the name $x$ is free, while $y$ is bound.

\begin{mathpar}
  \inferrule* [lab=monoidal-laws] {} { P|Q \equiv Q|P \and P|0 \equiv P \and P|(Q|R) \equiv (P|Q)|R }
\end{mathpar}

\begin{mathpar}
  \inferrule* [lab=alpha-equivalence] {} { (x)P \equiv (y)P\{y/x\} \and y \not\in \freenames{P} }
\end{mathpar}

\begin{definition}
Then two processes, $P,Q$, are alpha-equivalent if $P = Q\{\vec{y}/\vec{x}\}$ for
some $\vec{x} \in \boundnames{Q},\vec{y} \in \boundnames{P}$, where $Q\{\vec{y}/\vec{x}\}$
denotes the capture-avoiding substitution of $\vec{y}$ for $\vec{x}$ in $Q$.
\end{definition}

\begin{definition}
  The {\em structural congruence} \cite{SangiorgiWalker} , $\equiv$,
  between processes is the least congruence containing
  alpha-equivalence, satisfying the abelian monoid laws
  (associativity, commutativity and $\pzero$ as identity) for parallel
  composition $|$ and for summation $+$.
\end{definition}

\subsection{Name equivalence}

We take name equivalence, written $\nameeq$, to be the smallest
equivalence relation generated by the following rules.

\begin{mathpar}
\inferrule*[lab=Quote-drop]
{ }
{ \quotep{@{x}} \nameeq x }

\inferrule*[lab=Struct-equiv]
{ P \scong Q }
{ \quotep{P} \nameeq \quotep{Q} }
\end{mathpar}

The astute reader will have noticed that the mutual recursion of names
and processes imposes a mutual recursion on alpha-equivalence and
structural equivalence via name-equivalence. Fortunately, all of this
works out pleasantly and we may calculate in the natural way, free of
concern. The reader interested in the details is referred to the
appendix \ref{appendix:rho_details}.

\subsection{Substitution}

We use $\Proc$ for the set of processes, $\QProc$ for the set of
names, and $\id{\{}\vec{y} / \vec{x} \id{\}}$ to denote partial maps,
$s : \QProc \rightarrow \QProc$. A map, $s$ lifts, uniquely, to a map
on process terms, $\widehat{s} : \Proc \rightarrow \Proc$ by the
following equations.

\begin{mathpar}
  (0) \psubstp{Q}{P} := 0 \\
  (R \juxtap S) \psubstp{Q}{P}
  :=    
  (R)\psubstp{Q}{P} \juxtap (S) \psubstp{Q}{P} \\
  (x?(y).R) \psubstp{Q}{P}    
  :=    
  (x)\substp{Q}{P} (z)\concat( (R \psubstn{z}{y}) \psubstp{Q}{P} ) \\
  (\lift{x}{R}) \psubstp{Q}{P}  
  :=
  \lift{(x)\substp{Q}{P}}{ R \psubstp{Q}{P} } \\
%   (\dropn{x})  \psubstp{Q}{P}       
%   := 
%   \left\{ 
%     \begin{array}{ccc} 
%       \dropn{\quotep{Q}} & & x \nameeq \quotep{P} \\
%       \dropn{x} & & otherwise \\
%     \end{array}
%   \right. 
  (\dropn{x})  \psubstp{Q}{P}       
  := 
  \left\{ 
    \begin{array}{ccc} 
      Q & & x \nameeq \quotep{P} \\
      \dropn{x} & & otherwise \\
    \end{array}
  \right.
\end{mathpar}
 

where

\begin{eqnarray}
  (x)\id{\{} \lpquote Q \rpquote / \lpquote P \rpquote \id{\}}            = 
  \left\{ 
    \begin{array}{ccc}
      \lpquote Q \rpquote & & x \nameeq \lpquote P \rpquote \\
      x & & otherwise \\
    \end{array}
  \right. \nonumber
\end{eqnarray}

and $z$ is chosen distinct from $\quotep{P}$, $\quotep{Q}$, the free
names in $Q$, and all the names in $R$. Our $\alpha$-equivalence will
be built in the standard way from this substitution.

\begin{remark}\label{rem:no_self_referential_names}
  One consequence of these definitions is that $\forall P. \quotep{P}
  \not\in \freenames{P}$.
\end{remark}

\subsection{ Dynamic quote: an example }

Anticipating something of what's to come, consider applying the
substitution, $\widehat{\id{\{}u / z \id{\}}}$, to the following pair
of processes, $\lift{w}{y!(z)}$ and $w[ \lpquote y!(z) \rpquote ]$.

\begin{eqnarray}
	\lift{w}{y!(z)}\widehat{\id{\{}u / z \id{\}}}
		& = &
		\lift{w}{y!(u)} \nonumber\\
	w[ \lpquote y!(z) \rpquote ] \widehat{ \id{\{}u / z \id{\}} }
		& = &
		w[ \lpquote y!(z) \rpquote ] \nonumber
\end{eqnarray}

Because the body of the process between quotes is impervious to
substitution, we get radically different answers. In fact, by
examining the first process in an input context,
e.g. $x?(z).\lift{w}{y!(z)}$, we see that the process under the lift
operator may be shaped by prefixed inputs binding a name inside it. In
this sense, the lift operator will be seen as a way to dynamically
construct processes before reifying them as names.

Finally equipped with these standard features we can present the
dynamics of the calculus.

\subsubsection{Operational semantics} 

Finally, we introduce the computational dynamics. What marks these
algebras as distinct from other more traditionally studied algebraic
structures, e.g. vector spaces or polynomial rings, is the manner in
which dynamics is captured. In traditional structures, dynamics is typically
expressed through morphisms between such structures, as in linear maps
between vector spaces or morphisms between rings. In algebras
associated with the semantics of computation, the dynamics is
expressed as part of the algebraic structure itself, through a
reduction reduction relation typically denoted by $\red$. Below, we
give a recursive presentation of this relation for the calculus used
in the encoding.

$\red \subseteq \pi \times \pi$
$\red : \pi \to \mathcal{P}(\pi)$

\begin{mathpar}
  \inferrule* [lab=Comm] { \textsf{match}( x_{src}, x_{trgt} ) } { x_{trgt}?(y)P \; | \; x_{src}!\langle {Q} \rangle \red P\{\quotep{Q}/y}\} }
  \and \\
  \inferrule* [lab=Par] {{P} \red {P}'} {{{P} | {Q}} \red {{P}' | {Q}}}
  \and
  \inferrule* [lab=Equiv]{{{P} \scong {P}'} \andalso {{P}' \red {Q}'} \andalso {{Q}' \scong {Q}}}{{P} \red {Q}}
\end{mathpar}

\begin{eqnarray*}
  match_{\equiv} (\quotep{P},\quotep{Q}) & := & P \equiv Q \\
  match_{\dagger}(\quotep{P},\quotep{Q}) & := & \forall R. P|Q \red^{*} R => R \red^{*} 0 \\
  match_{K}(\quotep{P},\quotep{Q}) & := & K \mbox{ for some context } K
\end{eqnarray*}

$u?(x)P | u!\langle Q \rangle \red P\{\quotep{Q}/x\}$

%We write $\wred$ for $\red^*$, and $P\red$ if $\exists Q $ such that $ P \red Q$.
We write $P\red$ if $\exists Q $ such that $ P \red Q$ and $P\not\red$, otherwise.

\section{Replication}

As mentioned before, it is known that replication (and hence
recursion) can be implemented in a higher-order process algebra
\cite{SangiorgiWalker}. As our first example of calculation with the
machinery thus far presented we give the construction explicitly in
the {\rhoc}.

\begin{eqnarray}
	D_{x} & := & \prefix{x}{y}{(\binpar{\outputp{x}{y}}{@{y}})} \nonumber\\
	\bangp_{x}{P} & := & \binpar{{x}!\langle{\binpar{D_{x}}{P}}\rangle}{D_{x}} \nonumber
\end{eqnarray}

\begin{eqnarray}
	\bangp_{x}{P} & & \nonumber\\
	=
	& {x}!\langle{(\prefix{x}{y}{(\outputp{x}{y} | @{y})) | P}}\rangle 
	      | \prefix{x}{y}{(\outputp{x}{y} | @{y})} & \nonumber\\
	\red
	& (\outputp{x}{y} | @{y})\substn{\quotep{(\prefix{x}{y}{(@{y} | \outputp{x}{y})) | P}}}{y} & \nonumber\\
	=
	& \outputp{x}{\quotep{(\prefix{x}{y}{(\outputp{x}{y} | @{y})) | P}}}
	  | {(\prefix{x}{y}{(\outputp{x}{y} | @{y})) | P}} & \nonumber\\
	\red
	& \ldots & \nonumber\\
	\red^*
	& P | P | \ldots & \nonumber
\end{eqnarray}

Of course, this encoding, as an implementation, runs away, unfolding
$\bangp{P}$ eagerly. A lazier and more implementable replication
operator, restricted to input-guarded processes, may be obtained as follows.

\begin{eqnarray}
\bangp{\prefix{u}{v}{P}} 
	:= 
	\binpar{\lift{x}{\prefix{u}{v}{(\binpar{D(x)}{P})}}}{D(x)} \nonumber
\end{eqnarray}

\begin{remark}
  Note that the lazier definition still does not deal with summation
  or mixed summation (i.e. sums over input and output). The reader is
  invited to construct definitions of replication that deal with these
  features. 

  Further, the definitions are parameterized in a name, $x$. Can you,
  gentle reader, make a definition that eliminates this parameter and
  guarantees no accidental interaction between the replication
  machinery and the process being replicated -- i.e. no accidental
  sharing of names used by the process to get its work done and the
  name(s) used by the replication to effect copying. This latter
  revision of the definition of replication is crucial to obtaining
  the expected identity $!!P \sim !P$.
\end{remark}

\begin{remark}\label{rem:paradoxical_combinator}
  The reader familiar with the lambda calculus will have noticed the
  similarity between $D$ and the paradoxical combinator.

  [Ed. note: the existence of this seems to suggest we have to be more
  restrictive on the set of processes and names we admit if we are to
  support no-cloning.]
\end{remark}

\subsubsection{Bisimulation}

The computational dynamics gives rise to another kind of equivalence,
the equivalence of computational behavior. As previously mentioned
this is typically captured \emph{via} some form of bisimulation.

% The notion we use in this paper is weak barbed bisimulation
% \cite{milner91polyadicpi}.

The notion we use in this paper is derived from weak barbed
bisimulation \cite{milner91polyadicpi}. 

\begin{definition}
An \emph{observation relation}, $\downarrow_{\mathcal N}$, over a set
of names, $\mathcal N$, is the smallest relation satisfying the rules
below.

\infrule[Out-barb]{y \in {\mathcal N}, \; x \nameeq y}
		  {\outputp{x}{v} \downarrow_{\mathcal N} x}
\infrule[Par-barb]{\mbox{$P\downarrow_{\mathcal N} x$ or $Q\downarrow_{\mathcal N} x$}}
		  {\binpar{P}{Q} \downarrow_{\mathcal N} x}

We write $P \Downarrow_{\mathcal N} x$ if there is $Q$ such that 
$P \wred Q$ and $Q \downarrow_{\mathcal N} x$.
\end{definition}

\begin{definition}
%\label{def.bbisim}
An  ${\mathcal N}$-\emph{barbed bisimulation} over a set of names, ${\mathcal N}$, is a symmetric binary relation 
${\mathcal S}_{\mathcal N}$ between agents such that $P\rel{S}_{\mathcal N}Q$ implies:
\begin{enumerate}
\item If $P \red P'$ then $Q \wred Q'$ and $P'\rel{S}_{\mathcal N} Q'$.
\item If $P\downarrow_{\mathcal N} x$, then $Q\Downarrow_{\mathcal N} x$.
\end{enumerate}
$P$ is ${\mathcal N}$-barbed bisimilar to $Q$, written
$P \wbbisim_{\mathcal N} Q$, if $P \rel{S}_{\mathcal N} Q$ for some ${\mathcal N}$-barbed bisimulation ${\mathcal S}_{\mathcal N}$.
\end{definition}

$\mathcal{R} \subseteq \pi \times \pi$

$P \mathcal{R} Q => \forall P'. P \red P' \Rightarrow \exists Q'. Q \red Q', P' \mathcal{R} Q'$

$P \vdash x \Rightarrow Q \vdash x$

\begin{mathpar}
  \inferrule*[lab=Out-barb]{x \nameeq y}{{y}!\langle{Q}\rangle \vdash x}
  \and
  \inferrule*[lab=Par-barb]{\mbox{$P\vdash x$ or $Q\vdash x$}}{\binpar{P}{Q} \vdash x}
\end{mathpar}

\subsubsection{Contexts}

One of the principle advantages of computational calculi like the
$\pi$-calculus is a well-defined notion of context,
contextual-equivalence and a correlation between
contextual-equivalence and notions of bisimulation. The notion of
context allows the decomposition of a process into (sub-)process and
its syntactic environment, its context. Thus, a context may be
thought of as a process with a ``hole'' (written $\Box$) in it. The
application of a context $M$ to a process $P$, written $M[P]$, is
tantamount to filling the hole in $M$ with $P$. In this paper we do
not need the full weight of this theory, but do make use of the notion
of context in the proof the main theorem. 

\begin{mathpar}
  \inferrule* [lab=summation] {} {{M_{M},M_{N}} \bc \Box \;|\; x.M_{A} \;|\; M_{M}+M_{N}}
  \and
  \inferrule* [lab=agent] {} {{M_{A}} \bc (\vec{x})M_{P} \;| \; \clift{P_0,\ldots,M_{P},\ldots,P_N}}
  \and \\
  \inferrule* [lab=process] {} {{M_{P}} \bc M_{N} \;| \;P|M_{P} }
\end{mathpar} 

\begin{mathpar}
  \inferrule* [lab=sychronization] {} {M_{N} \bc \Box \;|\; x?M_{F} \;|\; x!M_{C}}
  \and
  \inferrule* [lab=abstraction] {} {{M_{F}} \bc (x)M_{P} }
  \and
  \inferrule* [lab=concretion] {} {{M_{C}} \bc \langle M_{P} \rangle }
  \and \\
  \inferrule* [lab=process] {} {{M_{P}} \bc M_{N} \;| \;P|M_{P} }
\end{mathpar}

\begin{definition}[contextual application] Given a context $M$, and
  process $P$, we define the \emph{contextual application}, $M[P] :=
  M\{P/\Box\}$. That is, the contextual application of M to P is the
  substitution of $P$ for $\Box$ in $M$.
\end{definition}

$\meaningof{-} : L \to \mathcal{P}(\pi)$

\begin{mathpar}
  \inferrule* [lab=collection] {} {\meaningof{true} = \pi, \and \meaningof{~E} = \pi \setminus \meaningof{E}, \and \meaningof{E_{1} \& E_{2}} = \meaningof{E_{1}} \cap \meaningof{E_{2}}}
\end{mathpar}

\begin{mathpar}
  \inferrule* [lab=structure] {} {\meaningof{0} = \{ P \in \pi | P \equiv 0 \}, \and \\ \meaningof{E_1 | E_2} = \{ P \in \pi | P \equiv P_{1} | P_{2}, P_{1} \in \meaningof{E_{1}}, P_{2} \in \meaningof{E_2}\} }
\end{mathpar}

\begin{mathpar}
 \inferrule* [lab=behavior] {} {\meaningof{\langle a?b \rangle E} = \{ P \in \pi | P \equiv Q | u?(y)P', \\ \and \\\\ \and \\ \;\;\; u \in \meaningof{a}, \forall z.P'\{z/y\} \in \meaningof{E\{z/b\}}\}, \and \\ \meaningof{a!E} = \{ P \in \pi | P \equiv Q | x!\langle P' \rangle, x \in \meaningof{a} P' \in \meaningof{E}\} }
\end{mathpar}

\begin{mathpar}
 \inferrule* [lab=nominal] {} {\meaningof{\quotep{E}} = \{ \quotep{P} \in \quotep{\pi} | P \in \meaningof{E} \}, \and \meaningof{\quotep{P}} = \{ \quotep{Q} \in \quotep{\pi} | P \equiv Q \} \and \\ \meaningof{@\quotep{E}} = \{ P \in \pi | P \equiv @x, x \in \meaningof{E} \}}
\end{mathpar}

\begin{eqnarray*}
  \\
  \meaningof{-} : TS \to ST
\end{eqnarray*}

\begin{eqnarray*}
  \\
  L : TS \to ST
\end{eqnarray*}

\begin{eqnarray*}
  \\
  P \models E \iff P \in \meaningof{E}
\end{eqnarray*}

\begin{eqnarray*}
  P \approx_{L} Q \iff \forall E \in L. P \models E \iff Q \models E
\end{eqnarray*}

\begin{eqnarray*}
  P \approx_{K} Q
\end{eqnarray*}

\begin{eqnarray*}
  P \approx Q
\end{eqnarray*}

$\approx_{K} = \approx = \approx_{L}$

\subsubsection{Contextual duality}

Note that contexts extend the quotation operation to a family of
operations from processes to names. Given a context, $M$, we can
define a \emph{nominal context}, $\quotep{M}$ by $\quotep{M}[P] :=
\quotep{M[P]}$. To foreshadow what is to come we observe that these
operations enjoy a duality with processes very much like the duality
between vectors and maps from vectors to scalars.

Further, because the calculus is essentially higher-order, we have a
correspondence between contexts and processes. More specifically,
given a name $x$ and a context $M$ we can construct $M^{*}_{x}$ such
that 

\begin{mathpar}
  M^{*}_{x} | \lift{x}{P} \red M[P]
\end{mathpar}

namely,

\begin{mathpar}
  M^{*}_{x} := x?(u).M[\dropn{u}]
\end{mathpar}

The dependence of $M^{*}_{x}$ on a name makes it an abstraction, 

\begin{mathpar}
  M^{*} := (x)x?(u).M[\dropn{u}]
\end{mathpar}

\subsection{Additional notation}

It will sometimes be convenient to denote the process a name
quotes. We already have the notation $x = \quotep{P}$, but it will be
convenient to introduce an alternate notation, $\procn{x}$, when we
want to emphasize the connection to the use of the name. Note that, by
virtue of name equivalence, $\quotep{\procn{x}} \nameeq x$; so, the
notation is consistent with previous definitions.

Further, because names have structure it is possible to effect
substitutions on the basis of that structure. This means we need to
upgrade our notation for substitutions, which we accomplish by
adapting comprehension notation. Thus,

\begin{mathpar}
  P\{ y / x : x \in S \}
\end{mathpar}

is interpreted to mean the process derived from P by replacing (in a
capture-avoiding manner) each occurrence of $x$ in $S$ by $y$. For example,

\begin{mathpar}
  P\{ \quotep{\procn{x}|\procn{x}} / x : x \in \freenames{P} \}
\end{mathpar}

will replace each (occurrence) of a free name $x$ in $P$ by
$\quotep{\procn{x}|\procn{x}}$.

Also, we will avail ourselves of the notation $x^{L}$ and $x^{R}$ to
denote injections of a name into disjoint copies of the name
space. There are numerous ways to accomplish this. One example can be
found in \cite{MeredithR05}. This notation overloads to vectors of
names: $\vec{x}^{\pi} := (x_{i}^{\pi} \; : \; 0 \leq i < |\vec{x}| )$ where $\pi \in \{L,R\}$.

We also use $P^{\Box} := P|\Box$.

In \cite{MeredithR05} an interpretation of the new operator is
given. It turns out that there are several possible interpretations
all enjoying the requisite algebraic properties of the operator (see
\cite{milner91polyadicpi}). We will therefore make liberal use of
$(\nu\; \vec{x})P$.

% subsection the_syntax_and_semantics_of_the_notation_system (end)   

\section{Interpretation of QM}
\subsection{Supporting definitions}
\subsubsection{Multiplication}
\begin{mathpar}
  \quotep{Q} \cdot \quotep{R} := \quotep{Q|R}
  \and \\
  \quotep{Q} \cdot P := P\{ \quotep{Q|R} / \quotep{R} : \quotep{R} \in \freenames{P} \}
\end{mathpar}

\paragraph{Discussion}
The first line needs little explanation. The second line says that
each free name of the process is replaced with the multiplication of
that name by the scalar. Multiplication of a scalar (name) by a state
(process) results in a process all the names of which have been `moved
over' by parallel composition with the process the scalar
quotes. There is a subtlety that the bound names have to be
manipulated so that multiplied names aren't accidentally
captured. There are many ways to achieve this.

\begin{remark}\label{rem:multiplication_identities}
  The reader is invited to verify that for all $x,y,z \in \QProc$ and $P \in \Proc$
  \begin{mathpar}
    x \cdot \quotep{0} \equiv x 
    \and
    x \cdot y \equiv y \cdot x
    \and
    x \cdot (y \cdot z) \equiv (x \cdot y) \cdot z
    \and \\
    \quotep{0} \cdot P \equiv P
    \and \\
    x \cdot (y \cdot P) \equiv (x \cdot y) \cdot P
    \and \\
    x \cdot (P|Q) \equiv (x \cdot P) | (x \cdot Q)
    \and \\    
  \end{mathpar}
\end{remark}

\subsubsection{Tensor product}

We define a tensor product on processes by structural induction.

\paragraph{Tensor of sums} First note that all summations, including
$\pzero$ and sequence, can be written $\Sigma_{i} x_{i}.A_{i} +
\Sigma_{j} x_{j}.C_{j}$, where we have grouped input-guarded processes
together and output-guarded processes together.

Thus, we can define the tensor product of two summations, $N_{1}\otimes N_{2}$, where

\begin{mathpar}
  N_{1} := \Sigma_{i} x_{i}.A_{i} + \Sigma_{j} x_{j}.C_{j}
  \and
  N_{2} := \Sigma_{i'} y_{i'}.B_{i'} + \Sigma_{j'} y_{j'}.D_{j'} 
\end{mathpar}

as follows.

\begin{mathpar}
  \Sigma_{i} x_{i}.A_{i} + \Sigma_{j} x_{j}.C_{j} \otimes \Sigma_{i'}
  y_{i'}.B_{i'} + \Sigma_{j'} y_{j'}.D_{j'} 
  \and \\
  := \; \Sigma_{i} \Sigma_{i'} \quotep{\stackrel{\vee}{x_{i}}| \stackrel{\vee}{y_{i'}}}.(A_{i}\otimes B_{i'}) \; | \; \Sigma_{i'} \Sigma_{i} \quotep{\stackrel{\vee}{y_{i'}}|\stackrel{\vee}{x_{i}}}.(B_{i'}\otimes A_{i})
  \and
  \;\; | \;\; \Sigma_{j} \Sigma_{j'} \quotep{\stackrel{\vee}{x_{j}}|\stackrel{\vee}{y_{j'}}}.(A_{j}\otimes B_{j'}) \; | \; \Sigma_{j'} \Sigma_{j} \quotep{\stackrel{\vee}{y_{j'}}|\stackrel{\vee}{x_{j}}}.(B_{j'}\otimes A_{j})
\end{mathpar}

\begin{remark}
  Do we need to $x^{L}$ and $y^{R}$ for this construction as well?
\end{remark}

\paragraph{Tensor of parallel compositions} Next, we distribute tensor
over par.

\begin{mathpar}
  P_{1}|P_{2} \otimes Q_{1}|Q_{2} := (P_{1} \otimes Q_{1}) | (P_{1}
  \otimes Q_{2}) | (P_{2} \otimes Q_{1}) | (P_{2} \otimes Q_{2})
\end{mathpar}

\paragraph{Tensor with dropped names} We treat tensor of a
process with a dropped name as parallel composition.

\begin{mathpar}
  P \otimes \dropn{x} := P | \dropn{x}
\end{mathpar}

\paragraph{Tensor of agents}

Finally, we need to define tensor on agents. Note that the definition
of tensor on normal products only tensors inputs with inputs and
outputs with outputs. Thus, we only have to define the operation on
``homogeneous'' pairings.

\begin{mathpar}
  (\vec{x})P \otimes (\vec{y})Q
  \and \\
  := (x_{0}^{L}|y_{0}^{R},\ldots,x_{0}^{L}|y_{n}^{R},\ldots,x_{m}^{L}|y_{0}^{R},\ldots,x_{m}^{L}|y_{n}^R)(P\{ \vec{x}^{L}/\vec{x}\} \otimes Q \{ \vec{y}^{R}/\vec{y}\})
  \and \\
  \clift{\vec{P}} \otimes \clift{\vec{Q}}
  \and \\
  := \clift{P_{0}\otimes Q_{0},\ldots,P_{0}\otimes Q_{n},\ldots,P_{m}\otimes Q_{0},\ldots,P_{m}\otimes Q_{n}}
\end{mathpar}

\begin{remark}
  Observe that arities of tensored abstractions matches arities of
  tensored concretions if the original arities matched. Note also that
  the length of the arities corresponds to the increase in dimension
  we see in ordinary vector space tensor product.
\end{remark}

\begin{remark}
  Operationally, this definition distributes the tensor down to
  components ``linked'' by summation. Tensor over summation is
  intriguing in that it mixes names. Moreover, as a consequence of the
  way it mixes names we have the identities for all $x \in \QProc$ and
  $P,Q \in \Proc$

  \begin{mathpar}
    (x \cdot P) \otimes Q \equiv x \cdot (P \otimes Q) \equiv P \otimes (x \cdot Q)
    \and
    P \otimes \pzero \equiv P
  \end{mathpar}

  that the reader is invited to verify.
\end{remark}

\subsubsection{Annihilation}
\begin{mathpar}
  P^{\perp} := \{ Q | \forall R. P|Q \red^{*} R \Rightarrow R \red^{*} \pzero \}
  \and \\
  P^{\underline{\perp}} := \Sigma_{Q \in P^{\perp}} \quotep{Q}?(y).(\dropn{y}|Q) | \Sigma_{Q \in P^{\perp}} \quotep{Q}\clift{\Box}
\end{mathpar}

\paragraph{Discussion} The reader will note that $P^{\perp}$ is a
\emph{set} of processes, while $P^{\underline{\perp}}$ is a
\emph{context}. We call the set $P^{\perp}$ the \emph{annihilators} of
$P$. The parallel composition of a process in the annihilators of $P$
with $P$ will result in a process, the state space of which has all
paths eventually leading to $\pzero$. Execution may endure loops; but
under reasonable conditions of fairness (naturally guaranteed under
most notions of bisimulation) such a composite process cannot get
stuck in such a loop and will, eventually pop out and terminate.

The context $P^{\underline{\perp}}$ is ready and willing to ``take the
$P$ out of'' the process to which it is applied. It will effectively
transmit the code of the process to which it is applied to one of the
annihilators and run the process against it.

\subsubsection{Evaluation}
We fix $M$ a domain of fully abstract interpretation with an equality
coincident with bisimulation. We take $\meaningof{\cdot} : \Proc \to
M$ to be the map interpreting processes and $\nmeaningof{\cdot} : \M
\to Proc$ to be the map running the other way. Then we define

\begin{mathpar}
  \int P := \nmeaningof{\meaningof{P}}
\end{mathpar}

\paragraph{Discussion}
There are many fully abstract interpretations of Milner's
$\pi$-calculus. Any of them can be used as a basis for interpreting
the reflective calculus here. Equipped with such a domain it is
largely a matter of grinding through to check that the Yoneda
construction for the normalization-by-evaluation program can be
extended to this setting.

\begin{remark}
  The reader is invited to verify that $\int (P^{\underline{\perp}}[P]) = 0$.
\end{remark}

\subsection{Quantum mechanics}

Table \ref{tbl:core_qm_op_defns} gives the core operational definitions

\begin{table}[htp]\label{tbl:core_qm_op_defns}
  \center{
    \fbox{
      \begin{tabular}{c|c}
        quantum mechanics & process calculus \\
        \hline
        scalar & $x := \quotep{P}$ \\
        state vector & $\state{P} := P$ \\
        dual & $\state{P}^{*} := \event{P^{\underline{\perp}}} := \quotep{P^{\underline{\perp}}}[-]$ \\
        matrix & $ \Sigma_{\alpha} \state{P_{\alpha}}x_{\alpha}\event{Q_{\alpha}}$ \\
        vector addition & $\state{P} + \state{Q} := \state{P | Q}$ \\
        tensor product & $\state{P} \otimes \state{Q} := \state{P \otimes Q}$ \\
        inner product & $\innerprod{P}{Q} := \quotep{\int P^{\underline{\perp}}[Q]}$ \\
      \end{tabular}
    }
  }
  \caption{QM - operational definitions}
\end{table}

where

\begin{mathpar}
  \prmatrix{P}{Q} := \fprmatrix{P}{\quotep{\pzero}}{Q}
  \and
  \fprmatrix{P}{x}{Q} := (\state{P},x,\event{Q})
  \and
  (\fprmatrix{P}{x}{Q})(\state{R}) := x \cdot \innerprod{Q}{R} \cdot \state{P}
  \and
  (\fprmatrix{P}{x}{Q})(\event{R}) := x \cdot \innerprod{R}{P} \cdot \event{Q}
\end{mathpar}

\paragraph{Discussion}
As promised: vectors (aka states) are represented as processes; duals
as contextual duals; inner product definition should be compared with
standard inner product definition for ....

\begin{remark}
  Assuming $\int (P^{\underline{\perp}}[P]) = 0$, the reader is
  invited to verify that $(\fprmatrix{P}{x}{P})(\state{P}) = x \cdot \state{P}$.
\end{remark}

\begin{remark}
  The reader is invited to verify that $\innerprod{P}{Q}$ could
  equally well have been written $\quotep{\int \stackrel{\vee}{x}}$
  where $x = \event{P^{\underline{\perp}}}(Q)$.

  One of the motivations for this remark is that there is another way
  to factor these operations. We could package up evaluation in the dual:

  \begin{mathpar}
    \state{P}^{*} := \event{\int P^{\underline{\perp}}} := \quotep{\int P^{\underline{\perp}}}[-]
  \end{mathpar}

  and then have inner product defined by
  
  \begin{mathpar}
    \innerprod{P}{Q} := \event{P}(Q)
  \end{mathpar}

  Hopefully, experience with the calculations will provide guidance on
  the best factoring.
\end{remark}

\begin{remark}
  Assuming $\int (P^{\underline{\perp}}[P]) = 0$, the reader is
  invited to verify that $\forall P,Q. (\prmatrix{0}{Q})(\state{0}) =
  \state{0}$ and dually $(\prmatrix{P}{0})(\event{0}) = \event{0}$.
\end{remark}

\begin{remark}
  i'm a little worried that i don't (yet) have proper support for
  complex conjugacy. But, the observation above may give us a
  clue. According to Abramsky, it must be the case that the scalars
  are iso to the homset of the identity for the tensor -- which the
  observation above characterizes. 

  For now, we will simply bookmark the notion with $\overline{x}$.
\end{remark}

\subsubsection{Adjointness}

We need to give a definition of $(\cdot)^{\dagger}$ for matrices. The
obvious candidate definition is
\begin{mathpar}
(\Sigma_{\alpha}\fprmatrix{P_{\alpha}}{x_{\alpha}}{Q_{\alpha}})^{\dagger}
= \Sigma_{\alpha}\fprmatrix{(Q_{\alpha}^{\underline{\perp}})^{*}}{\overline{x}_{\alpha}}{P_{\alpha}^{\underline{\perp}}} 
\end{mathpar}

But, $(Q_{\alpha}^{\underline{\perp}})^{*}$ requires a name along
which to communicate the process to achieve the context application.

\subsubsection{Basis for a basis}
If processes label states and ``addition'' of states (a.k.a. vector
addition) is interpreted as parallel composition, what corresponds to
notions of linear independence and basis? Here, we recall that Yoshida
has developed a set of \emph{combinators} for an asynchronous verison
of Milner's $\pi$-calculus. These are a finite set of processes such
any process can be expressed as parallel composition of these
combinators together with liberal uses of the new operator and
replication. We can simply give a translation of these into the
present calculus and have reasonable expectation that the property
carries over. That is, that the resultant set allows to express all
processes via parallel composition. Note, however, that there is no
new operator or replication in this calculus. As a result, we expect
that the corresponding set is actually infinite. That is, we expect
that the space is actually infinite dimensional.

\begin{remark}
  The attentive reader may be a bit concerned. Certainly, the
  collection $S$, $K$ and $I$ is a finite set of
  combinators. Shouldn't we expect to see a finite set of combinators
  for an effectively equivalent system? i am very sympathetic to this
  critique and feel it warrants full attention. On the other hand, i
  also have in mind the following analogy. The natural numbers, as a
  monoid under addition, has exactly $1$ generator, while the natural
  numbers, as a monoid under multiplication, has countably many
  generators (the primes). We observe that the application of the
  lambda calculus is much less resource sensitive than the parallel
  composition of the $\pi$-calculus. Could it be the case that we have
  an analogy of the form
  
  \begin{mathpar}
    m + n : MN :: m*n : M|N
  \end{mathpar}

  giving a similar blow up in the set of ``primes''?  This is such a
  wonderful thought that, even if it's not true, i think it's worth
  writing down.
\end{remark}
 

\documentclass[12pt]{llncs}
%\documentclass{jktr}

\usepackage[pdftex]{hyperref}                   
\usepackage {listings}
\usepackage {mathpartir}
\usepackage{bcprules}
%\usepackage{listings}
                       
\usepackage{graphicx} 
%\usepackage[margins=2.5cm,nohead,nofoot]{geometry}
%\usepackage{geometry}
\usepackage{amsfonts}
\usepackage{amstext}
\usepackage{latexsym}
\usepackage{amssymb}
\usepackage{color}


%\include{myPreamble}
\include{qm2pi.local} 

%\ifpdf
%\usepackage[pdftex]{graphicx}
%\else
%\usepackage{graphicx}
%\fi

 % \ifpdf
%  \usepackage{pdfsync}
%  \if


%\title{Brief Article}
%\author{David F. Snyder}
%\author{L.G. Meredith}

%\address{Dept. of Math., Texas State University--San Marcos, San Marcos, TX 78666}
       
\pagestyle{empty}


\begin{document}

\lstset{language=[Objective]Caml,frame=shadowbox}

\input{qm2pi.front}

% section front matter (end)

\input{qm2pi.intro} 
 
% section introduction (end)

% \input{qm2pi.knotations} 

% section notation (end)

\input{qm2pi.process.calculi} 

% section concurrent_process_calculi_and_spatial_logics_ (end)
    
%\input{qm2pi.knots2pi} 

%\input{qm2pi.trefoil} 

%\input{qm2pi.mainthm} 

% subsection basic_interpretation (end)

%\input{qm2pi.rho.presentation} 
\subsection{The syntax and semantics of the notation system}\label{sub:the_syntax_and_semantics_of_the_notation_system} % (fold)

We now summarize a technical presentation of the calculus that
embodies our theory of dynamics. The typical presentation of such a
calculus follows the style of giving generators and relations on
them. The grammar, below, describing term constructors, freely
generates the set of processes, $\Proc$. This set is then quotiented
by a relation known as structural congruence and it is over this set
that the notion of dynamics is expressed. This presentation is
essentially that of \cite{MeredithR05} with the addition of
polyadicity and summation. For readability we have relegated some of
the technical subtleties to an appendix.

\subsubsection{Process grammar}\label{subsub:process_grammar}

\begin{mathpar}
  \inferrule* [lab=synchronization] {} {{M} \bc \pzero \;|\; x?F \;|\; x!C }
  \and
  \inferrule* [lab=abstraction] {} {{F} \bc (x)P}
  \and
  \inferrule* [lab=concretion] {} {{C} \bc \langle Q \rangle}
  \and
  \inferrule* [lab=process] {} {{P,Q} \bc M \;| \;P|Q \;|\; @{x}}
  \and
  \inferrule* [lab=name] {} {{x} \bc \quotep{P}}
\end{mathpar} 

Note that $\vec{x}$ (resp. $\vec{P}$) denotes a vector of names
(resp. processes) of length $|\vec{x}|$ (resp. $|\vec{P}|$). We adopt
the following useful abbreviations.

\begin{mathpar}
   x?(\vec{y}).P := x.(\vec{y})P \and  x\clift{\vec{P}} := x.\clift{\vec{P}}
   \and x!(y) := \lift{x}{\dropn{y}}
   \and \Pi_{i=0}^{n-1}P_i := P_0 | \ldots | P_{n-1}
\end{mathpar}

\subsubsection{Structural congruence}

\paragraph{Free and bound names and alpha-equivalence.} At the
core of structural equivalence is alpha-equivalence which identifies
process that are the same up to a change of variable. Formally, we
recognize the distinction between free and bound names. The free names
of a process, $\freenames{P}$, may be calculated recursively as
follows:

\begin{mathpar}
\freenames{\pzero} := \emptyset
  \and \\
  \freenames{x?(y).P} := \{ x \} \cup (\freenames{P} \setminus \{ y \})
  \and 
  \freenames{x!\langle P \rangle} := \{ x \} \cup \{ P \} 
  \and \\
  \freenames{P|Q} := \freenames{P} \cup \freenames{Q}
  \and \\
  \freenames{@{x}} := \{ x \}
\end{mathpar}

$\pi$
$\quotep{\pi}$

$\freenames{-} : \pi \to \mathcal{P}(\quotep{\pi})$

\begin{eqnarray*}
  \freenames{\pzero} & := & \emptyset \\
  \freenames{x?(y).P} & := & \{ x \} \cup (\freenames{P} \setminus \{ y \}) \\
  \freenames{x!\langle P \rangle} & := & \{ x \} \cup \{ P \} \\
  \freenames{P|Q} & := & \freenames{P} \cup \freenames{Q} \\
  \freenames{\dropn{x}} & := & \{ x \}
\end{eqnarray*}

The bound names of a process, $\boundnames{P}$, are those names occurring in $P$
that are not free. For example, in $x?(y).0$, the name $x$ is free, while $y$ is bound.

\begin{mathpar}
  \inferrule* [lab=monoidal-laws] {} { P|Q \equiv Q|P \and P|0 \equiv P \and P|(Q|R) \equiv (P|Q)|R }
\end{mathpar}

\begin{mathpar}
  \inferrule* [lab=alpha-equivalence] {} { (x)P \equiv (y)P\{y/x\} \and y \not\in \freenames{P} }
\end{mathpar}

\begin{definition}
Then two processes, $P,Q$, are alpha-equivalent if $P = Q\{\vec{y}/\vec{x}\}$ for
some $\vec{x} \in \boundnames{Q},\vec{y} \in \boundnames{P}$, where $Q\{\vec{y}/\vec{x}\}$
denotes the capture-avoiding substitution of $\vec{y}$ for $\vec{x}$ in $Q$.
\end{definition}

\begin{definition}
  The {\em structural congruence} \cite{SangiorgiWalker} , $\equiv$,
  between processes is the least congruence containing
  alpha-equivalence, satisfying the abelian monoid laws
  (associativity, commutativity and $\pzero$ as identity) for parallel
  composition $|$ and for summation $+$.
\end{definition}

\subsection{Name equivalence}

We take name equivalence, written $\nameeq$, to be the smallest
equivalence relation generated by the following rules.

\begin{mathpar}
\inferrule*[lab=Quote-drop]
{ }
{ \quotep{@{x}} \nameeq x }

\inferrule*[lab=Struct-equiv]
{ P \scong Q }
{ \quotep{P} \nameeq \quotep{Q} }
\end{mathpar}

The astute reader will have noticed that the mutual recursion of names
and processes imposes a mutual recursion on alpha-equivalence and
structural equivalence via name-equivalence. Fortunately, all of this
works out pleasantly and we may calculate in the natural way, free of
concern. The reader interested in the details is referred to the
appendix \ref{appendix:rho_details}.

\subsection{Substitution}

We use $\Proc$ for the set of processes, $\QProc$ for the set of
names, and $\id{\{}\vec{y} / \vec{x} \id{\}}$ to denote partial maps,
$s : \QProc \rightarrow \QProc$. A map, $s$ lifts, uniquely, to a map
on process terms, $\widehat{s} : \Proc \rightarrow \Proc$ by the
following equations.

\begin{mathpar}
  (0) \psubstp{Q}{P} := 0 \\
  (R \juxtap S) \psubstp{Q}{P}
  :=    
  (R)\psubstp{Q}{P} \juxtap (S) \psubstp{Q}{P} \\
  (x?(y).R) \psubstp{Q}{P}    
  :=    
  (x)\substp{Q}{P} (z)\concat( (R \psubstn{z}{y}) \psubstp{Q}{P} ) \\
  (\lift{x}{R}) \psubstp{Q}{P}  
  :=
  \lift{(x)\substp{Q}{P}}{ R \psubstp{Q}{P} } \\
%   (\dropn{x})  \psubstp{Q}{P}       
%   := 
%   \left\{ 
%     \begin{array}{ccc} 
%       \dropn{\quotep{Q}} & & x \nameeq \quotep{P} \\
%       \dropn{x} & & otherwise \\
%     \end{array}
%   \right. 
  (\dropn{x})  \psubstp{Q}{P}       
  := 
  \left\{ 
    \begin{array}{ccc} 
      Q & & x \nameeq \quotep{P} \\
      \dropn{x} & & otherwise \\
    \end{array}
  \right.
\end{mathpar}
 

where

\begin{eqnarray}
  (x)\id{\{} \lpquote Q \rpquote / \lpquote P \rpquote \id{\}}            = 
  \left\{ 
    \begin{array}{ccc}
      \lpquote Q \rpquote & & x \nameeq \lpquote P \rpquote \\
      x & & otherwise \\
    \end{array}
  \right. \nonumber
\end{eqnarray}

and $z$ is chosen distinct from $\quotep{P}$, $\quotep{Q}$, the free
names in $Q$, and all the names in $R$. Our $\alpha$-equivalence will
be built in the standard way from this substitution.

\begin{remark}\label{rem:no_self_referential_names}
  One consequence of these definitions is that $\forall P. \quotep{P}
  \not\in \freenames{P}$.
\end{remark}

\subsection{ Dynamic quote: an example }

Anticipating something of what's to come, consider applying the
substitution, $\widehat{\id{\{}u / z \id{\}}}$, to the following pair
of processes, $\lift{w}{y!(z)}$ and $w[ \lpquote y!(z) \rpquote ]$.

\begin{eqnarray}
	\lift{w}{y!(z)}\widehat{\id{\{}u / z \id{\}}}
		& = &
		\lift{w}{y!(u)} \nonumber\\
	w[ \lpquote y!(z) \rpquote ] \widehat{ \id{\{}u / z \id{\}} }
		& = &
		w[ \lpquote y!(z) \rpquote ] \nonumber
\end{eqnarray}

Because the body of the process between quotes is impervious to
substitution, we get radically different answers. In fact, by
examining the first process in an input context,
e.g. $x?(z).\lift{w}{y!(z)}$, we see that the process under the lift
operator may be shaped by prefixed inputs binding a name inside it. In
this sense, the lift operator will be seen as a way to dynamically
construct processes before reifying them as names.

Finally equipped with these standard features we can present the
dynamics of the calculus.

\subsubsection{Operational semantics} 

Finally, we introduce the computational dynamics. What marks these
algebras as distinct from other more traditionally studied algebraic
structures, e.g. vector spaces or polynomial rings, is the manner in
which dynamics is captured. In traditional structures, dynamics is typically
expressed through morphisms between such structures, as in linear maps
between vector spaces or morphisms between rings. In algebras
associated with the semantics of computation, the dynamics is
expressed as part of the algebraic structure itself, through a
reduction reduction relation typically denoted by $\red$. Below, we
give a recursive presentation of this relation for the calculus used
in the encoding.

$\red \subseteq \pi \times \pi$
$\red : \pi \to \mathcal{P}(\pi)$

\begin{mathpar}
  \inferrule* [lab=Comm] { \textsf{match}( x_{src}, x_{trgt} ) } { x_{trgt}?(y)P \; | \; x_{src}!\langle {Q} \rangle \red P\{\quotep{Q}/y}\} }
  \and \\
  \inferrule* [lab=Par] {{P} \red {P}'} {{{P} | {Q}} \red {{P}' | {Q}}}
  \and
  \inferrule* [lab=Equiv]{{{P} \scong {P}'} \andalso {{P}' \red {Q}'} \andalso {{Q}' \scong {Q}}}{{P} \red {Q}}
\end{mathpar}

\begin{eqnarray*}
  match_{\equiv} (\quotep{P},\quotep{Q}) & := & P \equiv Q \\
  match_{\dagger}(\quotep{P},\quotep{Q}) & := & \forall R. P|Q \red^{*} R => R \red^{*} 0 \\
  match_{K}(\quotep{P},\quotep{Q}) & := & K \mbox{ for some context } K
\end{eqnarray*}

$u?(x)P | u!\langle Q \rangle \red P\{\quotep{Q}/x\}$

%We write $\wred$ for $\red^*$, and $P\red$ if $\exists Q $ such that $ P \red Q$.
We write $P\red$ if $\exists Q $ such that $ P \red Q$ and $P\not\red$, otherwise.

\section{Replication}

As mentioned before, it is known that replication (and hence
recursion) can be implemented in a higher-order process algebra
\cite{SangiorgiWalker}. As our first example of calculation with the
machinery thus far presented we give the construction explicitly in
the {\rhoc}.

\begin{eqnarray}
	D_{x} & := & \prefix{x}{y}{(\binpar{\outputp{x}{y}}{@{y}})} \nonumber\\
	\bangp_{x}{P} & := & \binpar{{x}!\langle{\binpar{D_{x}}{P}}\rangle}{D_{x}} \nonumber
\end{eqnarray}

\begin{eqnarray}
	\bangp_{x}{P} & & \nonumber\\
	=
	& {x}!\langle{(\prefix{x}{y}{(\outputp{x}{y} | @{y})) | P}}\rangle 
	      | \prefix{x}{y}{(\outputp{x}{y} | @{y})} & \nonumber\\
	\red
	& (\outputp{x}{y} | @{y})\substn{\quotep{(\prefix{x}{y}{(@{y} | \outputp{x}{y})) | P}}}{y} & \nonumber\\
	=
	& \outputp{x}{\quotep{(\prefix{x}{y}{(\outputp{x}{y} | @{y})) | P}}}
	  | {(\prefix{x}{y}{(\outputp{x}{y} | @{y})) | P}} & \nonumber\\
	\red
	& \ldots & \nonumber\\
	\red^*
	& P | P | \ldots & \nonumber
\end{eqnarray}

Of course, this encoding, as an implementation, runs away, unfolding
$\bangp{P}$ eagerly. A lazier and more implementable replication
operator, restricted to input-guarded processes, may be obtained as follows.

\begin{eqnarray}
\bangp{\prefix{u}{v}{P}} 
	:= 
	\binpar{\lift{x}{\prefix{u}{v}{(\binpar{D(x)}{P})}}}{D(x)} \nonumber
\end{eqnarray}

\begin{remark}
  Note that the lazier definition still does not deal with summation
  or mixed summation (i.e. sums over input and output). The reader is
  invited to construct definitions of replication that deal with these
  features. 

  Further, the definitions are parameterized in a name, $x$. Can you,
  gentle reader, make a definition that eliminates this parameter and
  guarantees no accidental interaction between the replication
  machinery and the process being replicated -- i.e. no accidental
  sharing of names used by the process to get its work done and the
  name(s) used by the replication to effect copying. This latter
  revision of the definition of replication is crucial to obtaining
  the expected identity $!!P \sim !P$.
\end{remark}

\begin{remark}\label{rem:paradoxical_combinator}
  The reader familiar with the lambda calculus will have noticed the
  similarity between $D$ and the paradoxical combinator.

  [Ed. note: the existence of this seems to suggest we have to be more
  restrictive on the set of processes and names we admit if we are to
  support no-cloning.]
\end{remark}

\subsubsection{Bisimulation}

The computational dynamics gives rise to another kind of equivalence,
the equivalence of computational behavior. As previously mentioned
this is typically captured \emph{via} some form of bisimulation.

% The notion we use in this paper is weak barbed bisimulation
% \cite{milner91polyadicpi}.

The notion we use in this paper is derived from weak barbed
bisimulation \cite{milner91polyadicpi}. 

\begin{definition}
An \emph{observation relation}, $\downarrow_{\mathcal N}$, over a set
of names, $\mathcal N$, is the smallest relation satisfying the rules
below.

\infrule[Out-barb]{y \in {\mathcal N}, \; x \nameeq y}
		  {\outputp{x}{v} \downarrow_{\mathcal N} x}
\infrule[Par-barb]{\mbox{$P\downarrow_{\mathcal N} x$ or $Q\downarrow_{\mathcal N} x$}}
		  {\binpar{P}{Q} \downarrow_{\mathcal N} x}

We write $P \Downarrow_{\mathcal N} x$ if there is $Q$ such that 
$P \wred Q$ and $Q \downarrow_{\mathcal N} x$.
\end{definition}

\begin{definition}
%\label{def.bbisim}
An  ${\mathcal N}$-\emph{barbed bisimulation} over a set of names, ${\mathcal N}$, is a symmetric binary relation 
${\mathcal S}_{\mathcal N}$ between agents such that $P\rel{S}_{\mathcal N}Q$ implies:
\begin{enumerate}
\item If $P \red P'$ then $Q \wred Q'$ and $P'\rel{S}_{\mathcal N} Q'$.
\item If $P\downarrow_{\mathcal N} x$, then $Q\Downarrow_{\mathcal N} x$.
\end{enumerate}
$P$ is ${\mathcal N}$-barbed bisimilar to $Q$, written
$P \wbbisim_{\mathcal N} Q$, if $P \rel{S}_{\mathcal N} Q$ for some ${\mathcal N}$-barbed bisimulation ${\mathcal S}_{\mathcal N}$.
\end{definition}

$\mathcal{R} \subseteq \pi \times \pi$

$P \mathcal{R} Q => \forall P'. P \red P' \Rightarrow \exists Q'. Q \red Q', P' \mathcal{R} Q'$

$P \vdash x \Rightarrow Q \vdash x$

\begin{mathpar}
  \inferrule*[lab=Out-barb]{x \nameeq y}{{y}!\langle{Q}\rangle \vdash x}
  \and
  \inferrule*[lab=Par-barb]{\mbox{$P\vdash x$ or $Q\vdash x$}}{\binpar{P}{Q} \vdash x}
\end{mathpar}

\subsubsection{Contexts}

One of the principle advantages of computational calculi like the
$\pi$-calculus is a well-defined notion of context,
contextual-equivalence and a correlation between
contextual-equivalence and notions of bisimulation. The notion of
context allows the decomposition of a process into (sub-)process and
its syntactic environment, its context. Thus, a context may be
thought of as a process with a ``hole'' (written $\Box$) in it. The
application of a context $M$ to a process $P$, written $M[P]$, is
tantamount to filling the hole in $M$ with $P$. In this paper we do
not need the full weight of this theory, but do make use of the notion
of context in the proof the main theorem. 

\begin{mathpar}
  \inferrule* [lab=summation] {} {{M_{M},M_{N}} \bc \Box \;|\; x.M_{A} \;|\; M_{M}+M_{N}}
  \and
  \inferrule* [lab=agent] {} {{M_{A}} \bc (\vec{x})M_{P} \;| \; \clift{P_0,\ldots,M_{P},\ldots,P_N}}
  \and \\
  \inferrule* [lab=process] {} {{M_{P}} \bc M_{N} \;| \;P|M_{P} }
\end{mathpar} 

\begin{mathpar}
  \inferrule* [lab=sychronization] {} {M_{N} \bc \Box \;|\; x?M_{F} \;|\; x!M_{C}}
  \and
  \inferrule* [lab=abstraction] {} {{M_{F}} \bc (x)M_{P} }
  \and
  \inferrule* [lab=concretion] {} {{M_{C}} \bc \langle M_{P} \rangle }
  \and \\
  \inferrule* [lab=process] {} {{M_{P}} \bc M_{N} \;| \;P|M_{P} }
\end{mathpar}

\begin{definition}[contextual application] Given a context $M$, and
  process $P$, we define the \emph{contextual application}, $M[P] :=
  M\{P/\Box\}$. That is, the contextual application of M to P is the
  substitution of $P$ for $\Box$ in $M$.
\end{definition}

$\meaningof{-} : L \to \mathcal{P}(\pi)$

\begin{mathpar}
  \inferrule* [lab=collection] {} {\meaningof{true} = \pi, \and \meaningof{~E} = \pi \setminus \meaningof{E}, \and \meaningof{E_{1} \& E_{2}} = \meaningof{E_{1}} \cap \meaningof{E_{2}}}
\end{mathpar}

\begin{mathpar}
  \inferrule* [lab=structure] {} {\meaningof{0} = \{ P \in \pi | P \equiv 0 \}, \and \\ \meaningof{E_1 | E_2} = \{ P \in \pi | P \equiv P_{1} | P_{2}, P_{1} \in \meaningof{E_{1}}, P_{2} \in \meaningof{E_2}\} }
\end{mathpar}

\begin{mathpar}
 \inferrule* [lab=behavior] {} {\meaningof{\langle a?b \rangle E} = \{ P \in \pi | P \equiv Q | u?(y)P', \\ \and \\\\ \and \\ \;\;\; u \in \meaningof{a}, \forall z.P'\{z/y\} \in \meaningof{E\{z/b\}}\}, \and \\ \meaningof{a!E} = \{ P \in \pi | P \equiv Q | x!\langle P' \rangle, x \in \meaningof{a} P' \in \meaningof{E}\} }
\end{mathpar}

\begin{mathpar}
 \inferrule* [lab=nominal] {} {\meaningof{\quotep{E}} = \{ \quotep{P} \in \quotep{\pi} | P \in \meaningof{E} \}, \and \meaningof{\quotep{P}} = \{ \quotep{Q} \in \quotep{\pi} | P \equiv Q \} \and \\ \meaningof{@\quotep{E}} = \{ P \in \pi | P \equiv @x, x \in \meaningof{E} \}}
\end{mathpar}

\begin{eqnarray*}
  \\
  \meaningof{-} : TS \to ST
\end{eqnarray*}

\begin{eqnarray*}
  \\
  L : TS \to ST
\end{eqnarray*}

\begin{eqnarray*}
  \\
  P \models E \iff P \in \meaningof{E}
\end{eqnarray*}

\begin{eqnarray*}
  P \approx_{L} Q \iff \forall E \in L. P \models E \iff Q \models E
\end{eqnarray*}

\begin{eqnarray*}
  P \approx_{K} Q
\end{eqnarray*}

\begin{eqnarray*}
  P \approx Q
\end{eqnarray*}

$\approx_{K} = \approx = \approx_{L}$

\subsubsection{Contextual duality}

Note that contexts extend the quotation operation to a family of
operations from processes to names. Given a context, $M$, we can
define a \emph{nominal context}, $\quotep{M}$ by $\quotep{M}[P] :=
\quotep{M[P]}$. To foreshadow what is to come we observe that these
operations enjoy a duality with processes very much like the duality
between vectors and maps from vectors to scalars.

Further, because the calculus is essentially higher-order, we have a
correspondence between contexts and processes. More specifically,
given a name $x$ and a context $M$ we can construct $M^{*}_{x}$ such
that 

\begin{mathpar}
  M^{*}_{x} | \lift{x}{P} \red M[P]
\end{mathpar}

namely,

\begin{mathpar}
  M^{*}_{x} := x?(u).M[\dropn{u}]
\end{mathpar}

The dependence of $M^{*}_{x}$ on a name makes it an abstraction, 

\begin{mathpar}
  M^{*} := (x)x?(u).M[\dropn{u}]
\end{mathpar}

\subsection{Additional notation}

It will sometimes be convenient to denote the process a name
quotes. We already have the notation $x = \quotep{P}$, but it will be
convenient to introduce an alternate notation, $\procn{x}$, when we
want to emphasize the connection to the use of the name. Note that, by
virtue of name equivalence, $\quotep{\procn{x}} \nameeq x$; so, the
notation is consistent with previous definitions.

Further, because names have structure it is possible to effect
substitutions on the basis of that structure. This means we need to
upgrade our notation for substitutions, which we accomplish by
adapting comprehension notation. Thus,

\begin{mathpar}
  P\{ y / x : x \in S \}
\end{mathpar}

is interpreted to mean the process derived from P by replacing (in a
capture-avoiding manner) each occurrence of $x$ in $S$ by $y$. For example,

\begin{mathpar}
  P\{ \quotep{\procn{x}|\procn{x}} / x : x \in \freenames{P} \}
\end{mathpar}

will replace each (occurrence) of a free name $x$ in $P$ by
$\quotep{\procn{x}|\procn{x}}$.

Also, we will avail ourselves of the notation $x^{L}$ and $x^{R}$ to
denote injections of a name into disjoint copies of the name
space. There are numerous ways to accomplish this. One example can be
found in \cite{MeredithR05}. This notation overloads to vectors of
names: $\vec{x}^{\pi} := (x_{i}^{\pi} \; : \; 0 \leq i < |\vec{x}| )$ where $\pi \in \{L,R\}$.

We also use $P^{\Box} := P|\Box$.

In \cite{MeredithR05} an interpretation of the new operator is
given. It turns out that there are several possible interpretations
all enjoying the requisite algebraic properties of the operator (see
\cite{milner91polyadicpi}). We will therefore make liberal use of
$(\nu\; \vec{x})P$.

% subsection the_syntax_and_semantics_of_the_notation_system (end)   

\input{qm2pi.qmops} 

\input{qm2pi.sterngerlach} 

\input{qm2pi.metric} 

% section concurrent_process_calculi (end)

%\input{qm2pi.proofsketch}

% section proof sketch (end)

%\input{qm2pi.slviaknots} 

% section spatial logic via knots (end)

\input{qm2pi.conclusion}

% section conclusion (end)

%\input{qm2pi.dtcodes} 

% section wiring algorithm (end)

\input{qm2pi.ack} 

% section acknowledgments (end)

\newpage


\bibliographystyle{plain}   
\bibliography{../../biblios/main.bib}

\input{qm2pi.rhodetails}

\end{document}

 

\documentclass[12pt]{llncs}
%\documentclass{jktr}

\usepackage[pdftex]{hyperref}                   
\usepackage {listings}
\usepackage {mathpartir}
\usepackage{bcprules}
%\usepackage{listings}
                       
\usepackage{graphicx} 
%\usepackage[margins=2.5cm,nohead,nofoot]{geometry}
%\usepackage{geometry}
\usepackage{amsfonts}
\usepackage{amstext}
\usepackage{latexsym}
\usepackage{amssymb}
\usepackage{color}


%\include{myPreamble}
\include{qm2pi.local} 

%\ifpdf
%\usepackage[pdftex]{graphicx}
%\else
%\usepackage{graphicx}
%\fi

 % \ifpdf
%  \usepackage{pdfsync}
%  \if


%\title{Brief Article}
%\author{David F. Snyder}
%\author{L.G. Meredith}

%\address{Dept. of Math., Texas State University--San Marcos, San Marcos, TX 78666}
       
\pagestyle{empty}


\begin{document}

\lstset{language=[Objective]Caml,frame=shadowbox}

\input{qm2pi.front}

% section front matter (end)

\input{qm2pi.intro} 
 
% section introduction (end)

% \input{qm2pi.knotations} 

% section notation (end)

\input{qm2pi.process.calculi} 

% section concurrent_process_calculi_and_spatial_logics_ (end)
    
%\input{qm2pi.knots2pi} 

%\input{qm2pi.trefoil} 

%\input{qm2pi.mainthm} 

% subsection basic_interpretation (end)

%\input{qm2pi.rho.presentation} 
\subsection{The syntax and semantics of the notation system}\label{sub:the_syntax_and_semantics_of_the_notation_system} % (fold)

We now summarize a technical presentation of the calculus that
embodies our theory of dynamics. The typical presentation of such a
calculus follows the style of giving generators and relations on
them. The grammar, below, describing term constructors, freely
generates the set of processes, $\Proc$. This set is then quotiented
by a relation known as structural congruence and it is over this set
that the notion of dynamics is expressed. This presentation is
essentially that of \cite{MeredithR05} with the addition of
polyadicity and summation. For readability we have relegated some of
the technical subtleties to an appendix.

\subsubsection{Process grammar}\label{subsub:process_grammar}

\begin{mathpar}
  \inferrule* [lab=synchronization] {} {{M} \bc \pzero \;|\; x?F \;|\; x!C }
  \and
  \inferrule* [lab=abstraction] {} {{F} \bc (x)P}
  \and
  \inferrule* [lab=concretion] {} {{C} \bc \langle Q \rangle}
  \and
  \inferrule* [lab=process] {} {{P,Q} \bc M \;| \;P|Q \;|\; @{x}}
  \and
  \inferrule* [lab=name] {} {{x} \bc \quotep{P}}
\end{mathpar} 

Note that $\vec{x}$ (resp. $\vec{P}$) denotes a vector of names
(resp. processes) of length $|\vec{x}|$ (resp. $|\vec{P}|$). We adopt
the following useful abbreviations.

\begin{mathpar}
   x?(\vec{y}).P := x.(\vec{y})P \and  x\clift{\vec{P}} := x.\clift{\vec{P}}
   \and x!(y) := \lift{x}{\dropn{y}}
   \and \Pi_{i=0}^{n-1}P_i := P_0 | \ldots | P_{n-1}
\end{mathpar}

\subsubsection{Structural congruence}

\paragraph{Free and bound names and alpha-equivalence.} At the
core of structural equivalence is alpha-equivalence which identifies
process that are the same up to a change of variable. Formally, we
recognize the distinction between free and bound names. The free names
of a process, $\freenames{P}$, may be calculated recursively as
follows:

\begin{mathpar}
\freenames{\pzero} := \emptyset
  \and \\
  \freenames{x?(y).P} := \{ x \} \cup (\freenames{P} \setminus \{ y \})
  \and 
  \freenames{x!\langle P \rangle} := \{ x \} \cup \{ P \} 
  \and \\
  \freenames{P|Q} := \freenames{P} \cup \freenames{Q}
  \and \\
  \freenames{@{x}} := \{ x \}
\end{mathpar}

$\pi$
$\quotep{\pi}$

$\freenames{-} : \pi \to \mathcal{P}(\quotep{\pi})$

\begin{eqnarray*}
  \freenames{\pzero} & := & \emptyset \\
  \freenames{x?(y).P} & := & \{ x \} \cup (\freenames{P} \setminus \{ y \}) \\
  \freenames{x!\langle P \rangle} & := & \{ x \} \cup \{ P \} \\
  \freenames{P|Q} & := & \freenames{P} \cup \freenames{Q} \\
  \freenames{\dropn{x}} & := & \{ x \}
\end{eqnarray*}

The bound names of a process, $\boundnames{P}$, are those names occurring in $P$
that are not free. For example, in $x?(y).0$, the name $x$ is free, while $y$ is bound.

\begin{mathpar}
  \inferrule* [lab=monoidal-laws] {} { P|Q \equiv Q|P \and P|0 \equiv P \and P|(Q|R) \equiv (P|Q)|R }
\end{mathpar}

\begin{mathpar}
  \inferrule* [lab=alpha-equivalence] {} { (x)P \equiv (y)P\{y/x\} \and y \not\in \freenames{P} }
\end{mathpar}

\begin{definition}
Then two processes, $P,Q$, are alpha-equivalent if $P = Q\{\vec{y}/\vec{x}\}$ for
some $\vec{x} \in \boundnames{Q},\vec{y} \in \boundnames{P}$, where $Q\{\vec{y}/\vec{x}\}$
denotes the capture-avoiding substitution of $\vec{y}$ for $\vec{x}$ in $Q$.
\end{definition}

\begin{definition}
  The {\em structural congruence} \cite{SangiorgiWalker} , $\equiv$,
  between processes is the least congruence containing
  alpha-equivalence, satisfying the abelian monoid laws
  (associativity, commutativity and $\pzero$ as identity) for parallel
  composition $|$ and for summation $+$.
\end{definition}

\subsection{Name equivalence}

We take name equivalence, written $\nameeq$, to be the smallest
equivalence relation generated by the following rules.

\begin{mathpar}
\inferrule*[lab=Quote-drop]
{ }
{ \quotep{@{x}} \nameeq x }

\inferrule*[lab=Struct-equiv]
{ P \scong Q }
{ \quotep{P} \nameeq \quotep{Q} }
\end{mathpar}

The astute reader will have noticed that the mutual recursion of names
and processes imposes a mutual recursion on alpha-equivalence and
structural equivalence via name-equivalence. Fortunately, all of this
works out pleasantly and we may calculate in the natural way, free of
concern. The reader interested in the details is referred to the
appendix \ref{appendix:rho_details}.

\subsection{Substitution}

We use $\Proc$ for the set of processes, $\QProc$ for the set of
names, and $\id{\{}\vec{y} / \vec{x} \id{\}}$ to denote partial maps,
$s : \QProc \rightarrow \QProc$. A map, $s$ lifts, uniquely, to a map
on process terms, $\widehat{s} : \Proc \rightarrow \Proc$ by the
following equations.

\begin{mathpar}
  (0) \psubstp{Q}{P} := 0 \\
  (R \juxtap S) \psubstp{Q}{P}
  :=    
  (R)\psubstp{Q}{P} \juxtap (S) \psubstp{Q}{P} \\
  (x?(y).R) \psubstp{Q}{P}    
  :=    
  (x)\substp{Q}{P} (z)\concat( (R \psubstn{z}{y}) \psubstp{Q}{P} ) \\
  (\lift{x}{R}) \psubstp{Q}{P}  
  :=
  \lift{(x)\substp{Q}{P}}{ R \psubstp{Q}{P} } \\
%   (\dropn{x})  \psubstp{Q}{P}       
%   := 
%   \left\{ 
%     \begin{array}{ccc} 
%       \dropn{\quotep{Q}} & & x \nameeq \quotep{P} \\
%       \dropn{x} & & otherwise \\
%     \end{array}
%   \right. 
  (\dropn{x})  \psubstp{Q}{P}       
  := 
  \left\{ 
    \begin{array}{ccc} 
      Q & & x \nameeq \quotep{P} \\
      \dropn{x} & & otherwise \\
    \end{array}
  \right.
\end{mathpar}
 

where

\begin{eqnarray}
  (x)\id{\{} \lpquote Q \rpquote / \lpquote P \rpquote \id{\}}            = 
  \left\{ 
    \begin{array}{ccc}
      \lpquote Q \rpquote & & x \nameeq \lpquote P \rpquote \\
      x & & otherwise \\
    \end{array}
  \right. \nonumber
\end{eqnarray}

and $z$ is chosen distinct from $\quotep{P}$, $\quotep{Q}$, the free
names in $Q$, and all the names in $R$. Our $\alpha$-equivalence will
be built in the standard way from this substitution.

\begin{remark}\label{rem:no_self_referential_names}
  One consequence of these definitions is that $\forall P. \quotep{P}
  \not\in \freenames{P}$.
\end{remark}

\subsection{ Dynamic quote: an example }

Anticipating something of what's to come, consider applying the
substitution, $\widehat{\id{\{}u / z \id{\}}}$, to the following pair
of processes, $\lift{w}{y!(z)}$ and $w[ \lpquote y!(z) \rpquote ]$.

\begin{eqnarray}
	\lift{w}{y!(z)}\widehat{\id{\{}u / z \id{\}}}
		& = &
		\lift{w}{y!(u)} \nonumber\\
	w[ \lpquote y!(z) \rpquote ] \widehat{ \id{\{}u / z \id{\}} }
		& = &
		w[ \lpquote y!(z) \rpquote ] \nonumber
\end{eqnarray}

Because the body of the process between quotes is impervious to
substitution, we get radically different answers. In fact, by
examining the first process in an input context,
e.g. $x?(z).\lift{w}{y!(z)}$, we see that the process under the lift
operator may be shaped by prefixed inputs binding a name inside it. In
this sense, the lift operator will be seen as a way to dynamically
construct processes before reifying them as names.

Finally equipped with these standard features we can present the
dynamics of the calculus.

\subsubsection{Operational semantics} 

Finally, we introduce the computational dynamics. What marks these
algebras as distinct from other more traditionally studied algebraic
structures, e.g. vector spaces or polynomial rings, is the manner in
which dynamics is captured. In traditional structures, dynamics is typically
expressed through morphisms between such structures, as in linear maps
between vector spaces or morphisms between rings. In algebras
associated with the semantics of computation, the dynamics is
expressed as part of the algebraic structure itself, through a
reduction reduction relation typically denoted by $\red$. Below, we
give a recursive presentation of this relation for the calculus used
in the encoding.

$\red \subseteq \pi \times \pi$
$\red : \pi \to \mathcal{P}(\pi)$

\begin{mathpar}
  \inferrule* [lab=Comm] { \textsf{match}( x_{src}, x_{trgt} ) } { x_{trgt}?(y)P \; | \; x_{src}!\langle {Q} \rangle \red P\{\quotep{Q}/y}\} }
  \and \\
  \inferrule* [lab=Par] {{P} \red {P}'} {{{P} | {Q}} \red {{P}' | {Q}}}
  \and
  \inferrule* [lab=Equiv]{{{P} \scong {P}'} \andalso {{P}' \red {Q}'} \andalso {{Q}' \scong {Q}}}{{P} \red {Q}}
\end{mathpar}

\begin{eqnarray*}
  match_{\equiv} (\quotep{P},\quotep{Q}) & := & P \equiv Q \\
  match_{\dagger}(\quotep{P},\quotep{Q}) & := & \forall R. P|Q \red^{*} R => R \red^{*} 0 \\
  match_{K}(\quotep{P},\quotep{Q}) & := & K \mbox{ for some context } K
\end{eqnarray*}

$u?(x)P | u!\langle Q \rangle \red P\{\quotep{Q}/x\}$

%We write $\wred$ for $\red^*$, and $P\red$ if $\exists Q $ such that $ P \red Q$.
We write $P\red$ if $\exists Q $ such that $ P \red Q$ and $P\not\red$, otherwise.

\section{Replication}

As mentioned before, it is known that replication (and hence
recursion) can be implemented in a higher-order process algebra
\cite{SangiorgiWalker}. As our first example of calculation with the
machinery thus far presented we give the construction explicitly in
the {\rhoc}.

\begin{eqnarray}
	D_{x} & := & \prefix{x}{y}{(\binpar{\outputp{x}{y}}{@{y}})} \nonumber\\
	\bangp_{x}{P} & := & \binpar{{x}!\langle{\binpar{D_{x}}{P}}\rangle}{D_{x}} \nonumber
\end{eqnarray}

\begin{eqnarray}
	\bangp_{x}{P} & & \nonumber\\
	=
	& {x}!\langle{(\prefix{x}{y}{(\outputp{x}{y} | @{y})) | P}}\rangle 
	      | \prefix{x}{y}{(\outputp{x}{y} | @{y})} & \nonumber\\
	\red
	& (\outputp{x}{y} | @{y})\substn{\quotep{(\prefix{x}{y}{(@{y} | \outputp{x}{y})) | P}}}{y} & \nonumber\\
	=
	& \outputp{x}{\quotep{(\prefix{x}{y}{(\outputp{x}{y} | @{y})) | P}}}
	  | {(\prefix{x}{y}{(\outputp{x}{y} | @{y})) | P}} & \nonumber\\
	\red
	& \ldots & \nonumber\\
	\red^*
	& P | P | \ldots & \nonumber
\end{eqnarray}

Of course, this encoding, as an implementation, runs away, unfolding
$\bangp{P}$ eagerly. A lazier and more implementable replication
operator, restricted to input-guarded processes, may be obtained as follows.

\begin{eqnarray}
\bangp{\prefix{u}{v}{P}} 
	:= 
	\binpar{\lift{x}{\prefix{u}{v}{(\binpar{D(x)}{P})}}}{D(x)} \nonumber
\end{eqnarray}

\begin{remark}
  Note that the lazier definition still does not deal with summation
  or mixed summation (i.e. sums over input and output). The reader is
  invited to construct definitions of replication that deal with these
  features. 

  Further, the definitions are parameterized in a name, $x$. Can you,
  gentle reader, make a definition that eliminates this parameter and
  guarantees no accidental interaction between the replication
  machinery and the process being replicated -- i.e. no accidental
  sharing of names used by the process to get its work done and the
  name(s) used by the replication to effect copying. This latter
  revision of the definition of replication is crucial to obtaining
  the expected identity $!!P \sim !P$.
\end{remark}

\begin{remark}\label{rem:paradoxical_combinator}
  The reader familiar with the lambda calculus will have noticed the
  similarity between $D$ and the paradoxical combinator.

  [Ed. note: the existence of this seems to suggest we have to be more
  restrictive on the set of processes and names we admit if we are to
  support no-cloning.]
\end{remark}

\subsubsection{Bisimulation}

The computational dynamics gives rise to another kind of equivalence,
the equivalence of computational behavior. As previously mentioned
this is typically captured \emph{via} some form of bisimulation.

% The notion we use in this paper is weak barbed bisimulation
% \cite{milner91polyadicpi}.

The notion we use in this paper is derived from weak barbed
bisimulation \cite{milner91polyadicpi}. 

\begin{definition}
An \emph{observation relation}, $\downarrow_{\mathcal N}$, over a set
of names, $\mathcal N$, is the smallest relation satisfying the rules
below.

\infrule[Out-barb]{y \in {\mathcal N}, \; x \nameeq y}
		  {\outputp{x}{v} \downarrow_{\mathcal N} x}
\infrule[Par-barb]{\mbox{$P\downarrow_{\mathcal N} x$ or $Q\downarrow_{\mathcal N} x$}}
		  {\binpar{P}{Q} \downarrow_{\mathcal N} x}

We write $P \Downarrow_{\mathcal N} x$ if there is $Q$ such that 
$P \wred Q$ and $Q \downarrow_{\mathcal N} x$.
\end{definition}

\begin{definition}
%\label{def.bbisim}
An  ${\mathcal N}$-\emph{barbed bisimulation} over a set of names, ${\mathcal N}$, is a symmetric binary relation 
${\mathcal S}_{\mathcal N}$ between agents such that $P\rel{S}_{\mathcal N}Q$ implies:
\begin{enumerate}
\item If $P \red P'$ then $Q \wred Q'$ and $P'\rel{S}_{\mathcal N} Q'$.
\item If $P\downarrow_{\mathcal N} x$, then $Q\Downarrow_{\mathcal N} x$.
\end{enumerate}
$P$ is ${\mathcal N}$-barbed bisimilar to $Q$, written
$P \wbbisim_{\mathcal N} Q$, if $P \rel{S}_{\mathcal N} Q$ for some ${\mathcal N}$-barbed bisimulation ${\mathcal S}_{\mathcal N}$.
\end{definition}

$\mathcal{R} \subseteq \pi \times \pi$

$P \mathcal{R} Q => \forall P'. P \red P' \Rightarrow \exists Q'. Q \red Q', P' \mathcal{R} Q'$

$P \vdash x \Rightarrow Q \vdash x$

\begin{mathpar}
  \inferrule*[lab=Out-barb]{x \nameeq y}{{y}!\langle{Q}\rangle \vdash x}
  \and
  \inferrule*[lab=Par-barb]{\mbox{$P\vdash x$ or $Q\vdash x$}}{\binpar{P}{Q} \vdash x}
\end{mathpar}

\subsubsection{Contexts}

One of the principle advantages of computational calculi like the
$\pi$-calculus is a well-defined notion of context,
contextual-equivalence and a correlation between
contextual-equivalence and notions of bisimulation. The notion of
context allows the decomposition of a process into (sub-)process and
its syntactic environment, its context. Thus, a context may be
thought of as a process with a ``hole'' (written $\Box$) in it. The
application of a context $M$ to a process $P$, written $M[P]$, is
tantamount to filling the hole in $M$ with $P$. In this paper we do
not need the full weight of this theory, but do make use of the notion
of context in the proof the main theorem. 

\begin{mathpar}
  \inferrule* [lab=summation] {} {{M_{M},M_{N}} \bc \Box \;|\; x.M_{A} \;|\; M_{M}+M_{N}}
  \and
  \inferrule* [lab=agent] {} {{M_{A}} \bc (\vec{x})M_{P} \;| \; \clift{P_0,\ldots,M_{P},\ldots,P_N}}
  \and \\
  \inferrule* [lab=process] {} {{M_{P}} \bc M_{N} \;| \;P|M_{P} }
\end{mathpar} 

\begin{mathpar}
  \inferrule* [lab=sychronization] {} {M_{N} \bc \Box \;|\; x?M_{F} \;|\; x!M_{C}}
  \and
  \inferrule* [lab=abstraction] {} {{M_{F}} \bc (x)M_{P} }
  \and
  \inferrule* [lab=concretion] {} {{M_{C}} \bc \langle M_{P} \rangle }
  \and \\
  \inferrule* [lab=process] {} {{M_{P}} \bc M_{N} \;| \;P|M_{P} }
\end{mathpar}

\begin{definition}[contextual application] Given a context $M$, and
  process $P$, we define the \emph{contextual application}, $M[P] :=
  M\{P/\Box\}$. That is, the contextual application of M to P is the
  substitution of $P$ for $\Box$ in $M$.
\end{definition}

$\meaningof{-} : L \to \mathcal{P}(\pi)$

\begin{mathpar}
  \inferrule* [lab=collection] {} {\meaningof{true} = \pi, \and \meaningof{~E} = \pi \setminus \meaningof{E}, \and \meaningof{E_{1} \& E_{2}} = \meaningof{E_{1}} \cap \meaningof{E_{2}}}
\end{mathpar}

\begin{mathpar}
  \inferrule* [lab=structure] {} {\meaningof{0} = \{ P \in \pi | P \equiv 0 \}, \and \\ \meaningof{E_1 | E_2} = \{ P \in \pi | P \equiv P_{1} | P_{2}, P_{1} \in \meaningof{E_{1}}, P_{2} \in \meaningof{E_2}\} }
\end{mathpar}

\begin{mathpar}
 \inferrule* [lab=behavior] {} {\meaningof{\langle a?b \rangle E} = \{ P \in \pi | P \equiv Q | u?(y)P', \\ \and \\\\ \and \\ \;\;\; u \in \meaningof{a}, \forall z.P'\{z/y\} \in \meaningof{E\{z/b\}}\}, \and \\ \meaningof{a!E} = \{ P \in \pi | P \equiv Q | x!\langle P' \rangle, x \in \meaningof{a} P' \in \meaningof{E}\} }
\end{mathpar}

\begin{mathpar}
 \inferrule* [lab=nominal] {} {\meaningof{\quotep{E}} = \{ \quotep{P} \in \quotep{\pi} | P \in \meaningof{E} \}, \and \meaningof{\quotep{P}} = \{ \quotep{Q} \in \quotep{\pi} | P \equiv Q \} \and \\ \meaningof{@\quotep{E}} = \{ P \in \pi | P \equiv @x, x \in \meaningof{E} \}}
\end{mathpar}

\begin{eqnarray*}
  \\
  \meaningof{-} : TS \to ST
\end{eqnarray*}

\begin{eqnarray*}
  \\
  L : TS \to ST
\end{eqnarray*}

\begin{eqnarray*}
  \\
  P \models E \iff P \in \meaningof{E}
\end{eqnarray*}

\begin{eqnarray*}
  P \approx_{L} Q \iff \forall E \in L. P \models E \iff Q \models E
\end{eqnarray*}

\begin{eqnarray*}
  P \approx_{K} Q
\end{eqnarray*}

\begin{eqnarray*}
  P \approx Q
\end{eqnarray*}

$\approx_{K} = \approx = \approx_{L}$

\subsubsection{Contextual duality}

Note that contexts extend the quotation operation to a family of
operations from processes to names. Given a context, $M$, we can
define a \emph{nominal context}, $\quotep{M}$ by $\quotep{M}[P] :=
\quotep{M[P]}$. To foreshadow what is to come we observe that these
operations enjoy a duality with processes very much like the duality
between vectors and maps from vectors to scalars.

Further, because the calculus is essentially higher-order, we have a
correspondence between contexts and processes. More specifically,
given a name $x$ and a context $M$ we can construct $M^{*}_{x}$ such
that 

\begin{mathpar}
  M^{*}_{x} | \lift{x}{P} \red M[P]
\end{mathpar}

namely,

\begin{mathpar}
  M^{*}_{x} := x?(u).M[\dropn{u}]
\end{mathpar}

The dependence of $M^{*}_{x}$ on a name makes it an abstraction, 

\begin{mathpar}
  M^{*} := (x)x?(u).M[\dropn{u}]
\end{mathpar}

\subsection{Additional notation}

It will sometimes be convenient to denote the process a name
quotes. We already have the notation $x = \quotep{P}$, but it will be
convenient to introduce an alternate notation, $\procn{x}$, when we
want to emphasize the connection to the use of the name. Note that, by
virtue of name equivalence, $\quotep{\procn{x}} \nameeq x$; so, the
notation is consistent with previous definitions.

Further, because names have structure it is possible to effect
substitutions on the basis of that structure. This means we need to
upgrade our notation for substitutions, which we accomplish by
adapting comprehension notation. Thus,

\begin{mathpar}
  P\{ y / x : x \in S \}
\end{mathpar}

is interpreted to mean the process derived from P by replacing (in a
capture-avoiding manner) each occurrence of $x$ in $S$ by $y$. For example,

\begin{mathpar}
  P\{ \quotep{\procn{x}|\procn{x}} / x : x \in \freenames{P} \}
\end{mathpar}

will replace each (occurrence) of a free name $x$ in $P$ by
$\quotep{\procn{x}|\procn{x}}$.

Also, we will avail ourselves of the notation $x^{L}$ and $x^{R}$ to
denote injections of a name into disjoint copies of the name
space. There are numerous ways to accomplish this. One example can be
found in \cite{MeredithR05}. This notation overloads to vectors of
names: $\vec{x}^{\pi} := (x_{i}^{\pi} \; : \; 0 \leq i < |\vec{x}| )$ where $\pi \in \{L,R\}$.

We also use $P^{\Box} := P|\Box$.

In \cite{MeredithR05} an interpretation of the new operator is
given. It turns out that there are several possible interpretations
all enjoying the requisite algebraic properties of the operator (see
\cite{milner91polyadicpi}). We will therefore make liberal use of
$(\nu\; \vec{x})P$.

% subsection the_syntax_and_semantics_of_the_notation_system (end)   

\input{qm2pi.qmops} 

\input{qm2pi.sterngerlach} 

\input{qm2pi.metric} 

% section concurrent_process_calculi (end)

%\input{qm2pi.proofsketch}

% section proof sketch (end)

%\input{qm2pi.slviaknots} 

% section spatial logic via knots (end)

\input{qm2pi.conclusion}

% section conclusion (end)

%\input{qm2pi.dtcodes} 

% section wiring algorithm (end)

\input{qm2pi.ack} 

% section acknowledgments (end)

\newpage


\bibliographystyle{plain}   
\bibliography{../../biblios/main.bib}

\input{qm2pi.rhodetails}

\end{document}

 

% section concurrent_process_calculi (end)

%\documentclass[12pt]{llncs}
%\documentclass{jktr}

\usepackage[pdftex]{hyperref}                   
\usepackage {listings}
\usepackage {mathpartir}
\usepackage{bcprules}
%\usepackage{listings}
                       
\usepackage{graphicx} 
%\usepackage[margins=2.5cm,nohead,nofoot]{geometry}
%\usepackage{geometry}
\usepackage{amsfonts}
\usepackage{amstext}
\usepackage{latexsym}
\usepackage{amssymb}
\usepackage{color}


%\include{myPreamble}
\include{qm2pi.local} 

%\ifpdf
%\usepackage[pdftex]{graphicx}
%\else
%\usepackage{graphicx}
%\fi

 % \ifpdf
%  \usepackage{pdfsync}
%  \if


%\title{Brief Article}
%\author{David F. Snyder}
%\author{L.G. Meredith}

%\address{Dept. of Math., Texas State University--San Marcos, San Marcos, TX 78666}
       
\pagestyle{empty}


\begin{document}

\lstset{language=[Objective]Caml,frame=shadowbox}

\input{qm2pi.front}

% section front matter (end)

\input{qm2pi.intro} 
 
% section introduction (end)

% \input{qm2pi.knotations} 

% section notation (end)

\input{qm2pi.process.calculi} 

% section concurrent_process_calculi_and_spatial_logics_ (end)
    
%\input{qm2pi.knots2pi} 

%\input{qm2pi.trefoil} 

%\input{qm2pi.mainthm} 

% subsection basic_interpretation (end)

%\input{qm2pi.rho.presentation} 
\subsection{The syntax and semantics of the notation system}\label{sub:the_syntax_and_semantics_of_the_notation_system} % (fold)

We now summarize a technical presentation of the calculus that
embodies our theory of dynamics. The typical presentation of such a
calculus follows the style of giving generators and relations on
them. The grammar, below, describing term constructors, freely
generates the set of processes, $\Proc$. This set is then quotiented
by a relation known as structural congruence and it is over this set
that the notion of dynamics is expressed. This presentation is
essentially that of \cite{MeredithR05} with the addition of
polyadicity and summation. For readability we have relegated some of
the technical subtleties to an appendix.

\subsubsection{Process grammar}\label{subsub:process_grammar}

\begin{mathpar}
  \inferrule* [lab=synchronization] {} {{M} \bc \pzero \;|\; x?F \;|\; x!C }
  \and
  \inferrule* [lab=abstraction] {} {{F} \bc (x)P}
  \and
  \inferrule* [lab=concretion] {} {{C} \bc \langle Q \rangle}
  \and
  \inferrule* [lab=process] {} {{P,Q} \bc M \;| \;P|Q \;|\; @{x}}
  \and
  \inferrule* [lab=name] {} {{x} \bc \quotep{P}}
\end{mathpar} 

Note that $\vec{x}$ (resp. $\vec{P}$) denotes a vector of names
(resp. processes) of length $|\vec{x}|$ (resp. $|\vec{P}|$). We adopt
the following useful abbreviations.

\begin{mathpar}
   x?(\vec{y}).P := x.(\vec{y})P \and  x\clift{\vec{P}} := x.\clift{\vec{P}}
   \and x!(y) := \lift{x}{\dropn{y}}
   \and \Pi_{i=0}^{n-1}P_i := P_0 | \ldots | P_{n-1}
\end{mathpar}

\subsubsection{Structural congruence}

\paragraph{Free and bound names and alpha-equivalence.} At the
core of structural equivalence is alpha-equivalence which identifies
process that are the same up to a change of variable. Formally, we
recognize the distinction between free and bound names. The free names
of a process, $\freenames{P}$, may be calculated recursively as
follows:

\begin{mathpar}
\freenames{\pzero} := \emptyset
  \and \\
  \freenames{x?(y).P} := \{ x \} \cup (\freenames{P} \setminus \{ y \})
  \and 
  \freenames{x!\langle P \rangle} := \{ x \} \cup \{ P \} 
  \and \\
  \freenames{P|Q} := \freenames{P} \cup \freenames{Q}
  \and \\
  \freenames{@{x}} := \{ x \}
\end{mathpar}

$\pi$
$\quotep{\pi}$

$\freenames{-} : \pi \to \mathcal{P}(\quotep{\pi})$

\begin{eqnarray*}
  \freenames{\pzero} & := & \emptyset \\
  \freenames{x?(y).P} & := & \{ x \} \cup (\freenames{P} \setminus \{ y \}) \\
  \freenames{x!\langle P \rangle} & := & \{ x \} \cup \{ P \} \\
  \freenames{P|Q} & := & \freenames{P} \cup \freenames{Q} \\
  \freenames{\dropn{x}} & := & \{ x \}
\end{eqnarray*}

The bound names of a process, $\boundnames{P}$, are those names occurring in $P$
that are not free. For example, in $x?(y).0$, the name $x$ is free, while $y$ is bound.

\begin{mathpar}
  \inferrule* [lab=monoidal-laws] {} { P|Q \equiv Q|P \and P|0 \equiv P \and P|(Q|R) \equiv (P|Q)|R }
\end{mathpar}

\begin{mathpar}
  \inferrule* [lab=alpha-equivalence] {} { (x)P \equiv (y)P\{y/x\} \and y \not\in \freenames{P} }
\end{mathpar}

\begin{definition}
Then two processes, $P,Q$, are alpha-equivalent if $P = Q\{\vec{y}/\vec{x}\}$ for
some $\vec{x} \in \boundnames{Q},\vec{y} \in \boundnames{P}$, where $Q\{\vec{y}/\vec{x}\}$
denotes the capture-avoiding substitution of $\vec{y}$ for $\vec{x}$ in $Q$.
\end{definition}

\begin{definition}
  The {\em structural congruence} \cite{SangiorgiWalker} , $\equiv$,
  between processes is the least congruence containing
  alpha-equivalence, satisfying the abelian monoid laws
  (associativity, commutativity and $\pzero$ as identity) for parallel
  composition $|$ and for summation $+$.
\end{definition}

\subsection{Name equivalence}

We take name equivalence, written $\nameeq$, to be the smallest
equivalence relation generated by the following rules.

\begin{mathpar}
\inferrule*[lab=Quote-drop]
{ }
{ \quotep{@{x}} \nameeq x }

\inferrule*[lab=Struct-equiv]
{ P \scong Q }
{ \quotep{P} \nameeq \quotep{Q} }
\end{mathpar}

The astute reader will have noticed that the mutual recursion of names
and processes imposes a mutual recursion on alpha-equivalence and
structural equivalence via name-equivalence. Fortunately, all of this
works out pleasantly and we may calculate in the natural way, free of
concern. The reader interested in the details is referred to the
appendix \ref{appendix:rho_details}.

\subsection{Substitution}

We use $\Proc$ for the set of processes, $\QProc$ for the set of
names, and $\id{\{}\vec{y} / \vec{x} \id{\}}$ to denote partial maps,
$s : \QProc \rightarrow \QProc$. A map, $s$ lifts, uniquely, to a map
on process terms, $\widehat{s} : \Proc \rightarrow \Proc$ by the
following equations.

\begin{mathpar}
  (0) \psubstp{Q}{P} := 0 \\
  (R \juxtap S) \psubstp{Q}{P}
  :=    
  (R)\psubstp{Q}{P} \juxtap (S) \psubstp{Q}{P} \\
  (x?(y).R) \psubstp{Q}{P}    
  :=    
  (x)\substp{Q}{P} (z)\concat( (R \psubstn{z}{y}) \psubstp{Q}{P} ) \\
  (\lift{x}{R}) \psubstp{Q}{P}  
  :=
  \lift{(x)\substp{Q}{P}}{ R \psubstp{Q}{P} } \\
%   (\dropn{x})  \psubstp{Q}{P}       
%   := 
%   \left\{ 
%     \begin{array}{ccc} 
%       \dropn{\quotep{Q}} & & x \nameeq \quotep{P} \\
%       \dropn{x} & & otherwise \\
%     \end{array}
%   \right. 
  (\dropn{x})  \psubstp{Q}{P}       
  := 
  \left\{ 
    \begin{array}{ccc} 
      Q & & x \nameeq \quotep{P} \\
      \dropn{x} & & otherwise \\
    \end{array}
  \right.
\end{mathpar}
 

where

\begin{eqnarray}
  (x)\id{\{} \lpquote Q \rpquote / \lpquote P \rpquote \id{\}}            = 
  \left\{ 
    \begin{array}{ccc}
      \lpquote Q \rpquote & & x \nameeq \lpquote P \rpquote \\
      x & & otherwise \\
    \end{array}
  \right. \nonumber
\end{eqnarray}

and $z$ is chosen distinct from $\quotep{P}$, $\quotep{Q}$, the free
names in $Q$, and all the names in $R$. Our $\alpha$-equivalence will
be built in the standard way from this substitution.

\begin{remark}\label{rem:no_self_referential_names}
  One consequence of these definitions is that $\forall P. \quotep{P}
  \not\in \freenames{P}$.
\end{remark}

\subsection{ Dynamic quote: an example }

Anticipating something of what's to come, consider applying the
substitution, $\widehat{\id{\{}u / z \id{\}}}$, to the following pair
of processes, $\lift{w}{y!(z)}$ and $w[ \lpquote y!(z) \rpquote ]$.

\begin{eqnarray}
	\lift{w}{y!(z)}\widehat{\id{\{}u / z \id{\}}}
		& = &
		\lift{w}{y!(u)} \nonumber\\
	w[ \lpquote y!(z) \rpquote ] \widehat{ \id{\{}u / z \id{\}} }
		& = &
		w[ \lpquote y!(z) \rpquote ] \nonumber
\end{eqnarray}

Because the body of the process between quotes is impervious to
substitution, we get radically different answers. In fact, by
examining the first process in an input context,
e.g. $x?(z).\lift{w}{y!(z)}$, we see that the process under the lift
operator may be shaped by prefixed inputs binding a name inside it. In
this sense, the lift operator will be seen as a way to dynamically
construct processes before reifying them as names.

Finally equipped with these standard features we can present the
dynamics of the calculus.

\subsubsection{Operational semantics} 

Finally, we introduce the computational dynamics. What marks these
algebras as distinct from other more traditionally studied algebraic
structures, e.g. vector spaces or polynomial rings, is the manner in
which dynamics is captured. In traditional structures, dynamics is typically
expressed through morphisms between such structures, as in linear maps
between vector spaces or morphisms between rings. In algebras
associated with the semantics of computation, the dynamics is
expressed as part of the algebraic structure itself, through a
reduction reduction relation typically denoted by $\red$. Below, we
give a recursive presentation of this relation for the calculus used
in the encoding.

$\red \subseteq \pi \times \pi$
$\red : \pi \to \mathcal{P}(\pi)$

\begin{mathpar}
  \inferrule* [lab=Comm] { \textsf{match}( x_{src}, x_{trgt} ) } { x_{trgt}?(y)P \; | \; x_{src}!\langle {Q} \rangle \red P\{\quotep{Q}/y}\} }
  \and \\
  \inferrule* [lab=Par] {{P} \red {P}'} {{{P} | {Q}} \red {{P}' | {Q}}}
  \and
  \inferrule* [lab=Equiv]{{{P} \scong {P}'} \andalso {{P}' \red {Q}'} \andalso {{Q}' \scong {Q}}}{{P} \red {Q}}
\end{mathpar}

\begin{eqnarray*}
  match_{\equiv} (\quotep{P},\quotep{Q}) & := & P \equiv Q \\
  match_{\dagger}(\quotep{P},\quotep{Q}) & := & \forall R. P|Q \red^{*} R => R \red^{*} 0 \\
  match_{K}(\quotep{P},\quotep{Q}) & := & K \mbox{ for some context } K
\end{eqnarray*}

$u?(x)P | u!\langle Q \rangle \red P\{\quotep{Q}/x\}$

%We write $\wred$ for $\red^*$, and $P\red$ if $\exists Q $ such that $ P \red Q$.
We write $P\red$ if $\exists Q $ such that $ P \red Q$ and $P\not\red$, otherwise.

\section{Replication}

As mentioned before, it is known that replication (and hence
recursion) can be implemented in a higher-order process algebra
\cite{SangiorgiWalker}. As our first example of calculation with the
machinery thus far presented we give the construction explicitly in
the {\rhoc}.

\begin{eqnarray}
	D_{x} & := & \prefix{x}{y}{(\binpar{\outputp{x}{y}}{@{y}})} \nonumber\\
	\bangp_{x}{P} & := & \binpar{{x}!\langle{\binpar{D_{x}}{P}}\rangle}{D_{x}} \nonumber
\end{eqnarray}

\begin{eqnarray}
	\bangp_{x}{P} & & \nonumber\\
	=
	& {x}!\langle{(\prefix{x}{y}{(\outputp{x}{y} | @{y})) | P}}\rangle 
	      | \prefix{x}{y}{(\outputp{x}{y} | @{y})} & \nonumber\\
	\red
	& (\outputp{x}{y} | @{y})\substn{\quotep{(\prefix{x}{y}{(@{y} | \outputp{x}{y})) | P}}}{y} & \nonumber\\
	=
	& \outputp{x}{\quotep{(\prefix{x}{y}{(\outputp{x}{y} | @{y})) | P}}}
	  | {(\prefix{x}{y}{(\outputp{x}{y} | @{y})) | P}} & \nonumber\\
	\red
	& \ldots & \nonumber\\
	\red^*
	& P | P | \ldots & \nonumber
\end{eqnarray}

Of course, this encoding, as an implementation, runs away, unfolding
$\bangp{P}$ eagerly. A lazier and more implementable replication
operator, restricted to input-guarded processes, may be obtained as follows.

\begin{eqnarray}
\bangp{\prefix{u}{v}{P}} 
	:= 
	\binpar{\lift{x}{\prefix{u}{v}{(\binpar{D(x)}{P})}}}{D(x)} \nonumber
\end{eqnarray}

\begin{remark}
  Note that the lazier definition still does not deal with summation
  or mixed summation (i.e. sums over input and output). The reader is
  invited to construct definitions of replication that deal with these
  features. 

  Further, the definitions are parameterized in a name, $x$. Can you,
  gentle reader, make a definition that eliminates this parameter and
  guarantees no accidental interaction between the replication
  machinery and the process being replicated -- i.e. no accidental
  sharing of names used by the process to get its work done and the
  name(s) used by the replication to effect copying. This latter
  revision of the definition of replication is crucial to obtaining
  the expected identity $!!P \sim !P$.
\end{remark}

\begin{remark}\label{rem:paradoxical_combinator}
  The reader familiar with the lambda calculus will have noticed the
  similarity between $D$ and the paradoxical combinator.

  [Ed. note: the existence of this seems to suggest we have to be more
  restrictive on the set of processes and names we admit if we are to
  support no-cloning.]
\end{remark}

\subsubsection{Bisimulation}

The computational dynamics gives rise to another kind of equivalence,
the equivalence of computational behavior. As previously mentioned
this is typically captured \emph{via} some form of bisimulation.

% The notion we use in this paper is weak barbed bisimulation
% \cite{milner91polyadicpi}.

The notion we use in this paper is derived from weak barbed
bisimulation \cite{milner91polyadicpi}. 

\begin{definition}
An \emph{observation relation}, $\downarrow_{\mathcal N}$, over a set
of names, $\mathcal N$, is the smallest relation satisfying the rules
below.

\infrule[Out-barb]{y \in {\mathcal N}, \; x \nameeq y}
		  {\outputp{x}{v} \downarrow_{\mathcal N} x}
\infrule[Par-barb]{\mbox{$P\downarrow_{\mathcal N} x$ or $Q\downarrow_{\mathcal N} x$}}
		  {\binpar{P}{Q} \downarrow_{\mathcal N} x}

We write $P \Downarrow_{\mathcal N} x$ if there is $Q$ such that 
$P \wred Q$ and $Q \downarrow_{\mathcal N} x$.
\end{definition}

\begin{definition}
%\label{def.bbisim}
An  ${\mathcal N}$-\emph{barbed bisimulation} over a set of names, ${\mathcal N}$, is a symmetric binary relation 
${\mathcal S}_{\mathcal N}$ between agents such that $P\rel{S}_{\mathcal N}Q$ implies:
\begin{enumerate}
\item If $P \red P'$ then $Q \wred Q'$ and $P'\rel{S}_{\mathcal N} Q'$.
\item If $P\downarrow_{\mathcal N} x$, then $Q\Downarrow_{\mathcal N} x$.
\end{enumerate}
$P$ is ${\mathcal N}$-barbed bisimilar to $Q$, written
$P \wbbisim_{\mathcal N} Q$, if $P \rel{S}_{\mathcal N} Q$ for some ${\mathcal N}$-barbed bisimulation ${\mathcal S}_{\mathcal N}$.
\end{definition}

$\mathcal{R} \subseteq \pi \times \pi$

$P \mathcal{R} Q => \forall P'. P \red P' \Rightarrow \exists Q'. Q \red Q', P' \mathcal{R} Q'$

$P \vdash x \Rightarrow Q \vdash x$

\begin{mathpar}
  \inferrule*[lab=Out-barb]{x \nameeq y}{{y}!\langle{Q}\rangle \vdash x}
  \and
  \inferrule*[lab=Par-barb]{\mbox{$P\vdash x$ or $Q\vdash x$}}{\binpar{P}{Q} \vdash x}
\end{mathpar}

\subsubsection{Contexts}

One of the principle advantages of computational calculi like the
$\pi$-calculus is a well-defined notion of context,
contextual-equivalence and a correlation between
contextual-equivalence and notions of bisimulation. The notion of
context allows the decomposition of a process into (sub-)process and
its syntactic environment, its context. Thus, a context may be
thought of as a process with a ``hole'' (written $\Box$) in it. The
application of a context $M$ to a process $P$, written $M[P]$, is
tantamount to filling the hole in $M$ with $P$. In this paper we do
not need the full weight of this theory, but do make use of the notion
of context in the proof the main theorem. 

\begin{mathpar}
  \inferrule* [lab=summation] {} {{M_{M},M_{N}} \bc \Box \;|\; x.M_{A} \;|\; M_{M}+M_{N}}
  \and
  \inferrule* [lab=agent] {} {{M_{A}} \bc (\vec{x})M_{P} \;| \; \clift{P_0,\ldots,M_{P},\ldots,P_N}}
  \and \\
  \inferrule* [lab=process] {} {{M_{P}} \bc M_{N} \;| \;P|M_{P} }
\end{mathpar} 

\begin{mathpar}
  \inferrule* [lab=sychronization] {} {M_{N} \bc \Box \;|\; x?M_{F} \;|\; x!M_{C}}
  \and
  \inferrule* [lab=abstraction] {} {{M_{F}} \bc (x)M_{P} }
  \and
  \inferrule* [lab=concretion] {} {{M_{C}} \bc \langle M_{P} \rangle }
  \and \\
  \inferrule* [lab=process] {} {{M_{P}} \bc M_{N} \;| \;P|M_{P} }
\end{mathpar}

\begin{definition}[contextual application] Given a context $M$, and
  process $P$, we define the \emph{contextual application}, $M[P] :=
  M\{P/\Box\}$. That is, the contextual application of M to P is the
  substitution of $P$ for $\Box$ in $M$.
\end{definition}

$\meaningof{-} : L \to \mathcal{P}(\pi)$

\begin{mathpar}
  \inferrule* [lab=collection] {} {\meaningof{true} = \pi, \and \meaningof{~E} = \pi \setminus \meaningof{E}, \and \meaningof{E_{1} \& E_{2}} = \meaningof{E_{1}} \cap \meaningof{E_{2}}}
\end{mathpar}

\begin{mathpar}
  \inferrule* [lab=structure] {} {\meaningof{0} = \{ P \in \pi | P \equiv 0 \}, \and \\ \meaningof{E_1 | E_2} = \{ P \in \pi | P \equiv P_{1} | P_{2}, P_{1} \in \meaningof{E_{1}}, P_{2} \in \meaningof{E_2}\} }
\end{mathpar}

\begin{mathpar}
 \inferrule* [lab=behavior] {} {\meaningof{\langle a?b \rangle E} = \{ P \in \pi | P \equiv Q | u?(y)P', \\ \and \\\\ \and \\ \;\;\; u \in \meaningof{a}, \forall z.P'\{z/y\} \in \meaningof{E\{z/b\}}\}, \and \\ \meaningof{a!E} = \{ P \in \pi | P \equiv Q | x!\langle P' \rangle, x \in \meaningof{a} P' \in \meaningof{E}\} }
\end{mathpar}

\begin{mathpar}
 \inferrule* [lab=nominal] {} {\meaningof{\quotep{E}} = \{ \quotep{P} \in \quotep{\pi} | P \in \meaningof{E} \}, \and \meaningof{\quotep{P}} = \{ \quotep{Q} \in \quotep{\pi} | P \equiv Q \} \and \\ \meaningof{@\quotep{E}} = \{ P \in \pi | P \equiv @x, x \in \meaningof{E} \}}
\end{mathpar}

\begin{eqnarray*}
  \\
  \meaningof{-} : TS \to ST
\end{eqnarray*}

\begin{eqnarray*}
  \\
  L : TS \to ST
\end{eqnarray*}

\begin{eqnarray*}
  \\
  P \models E \iff P \in \meaningof{E}
\end{eqnarray*}

\begin{eqnarray*}
  P \approx_{L} Q \iff \forall E \in L. P \models E \iff Q \models E
\end{eqnarray*}

\begin{eqnarray*}
  P \approx_{K} Q
\end{eqnarray*}

\begin{eqnarray*}
  P \approx Q
\end{eqnarray*}

$\approx_{K} = \approx = \approx_{L}$

\subsubsection{Contextual duality}

Note that contexts extend the quotation operation to a family of
operations from processes to names. Given a context, $M$, we can
define a \emph{nominal context}, $\quotep{M}$ by $\quotep{M}[P] :=
\quotep{M[P]}$. To foreshadow what is to come we observe that these
operations enjoy a duality with processes very much like the duality
between vectors and maps from vectors to scalars.

Further, because the calculus is essentially higher-order, we have a
correspondence between contexts and processes. More specifically,
given a name $x$ and a context $M$ we can construct $M^{*}_{x}$ such
that 

\begin{mathpar}
  M^{*}_{x} | \lift{x}{P} \red M[P]
\end{mathpar}

namely,

\begin{mathpar}
  M^{*}_{x} := x?(u).M[\dropn{u}]
\end{mathpar}

The dependence of $M^{*}_{x}$ on a name makes it an abstraction, 

\begin{mathpar}
  M^{*} := (x)x?(u).M[\dropn{u}]
\end{mathpar}

\subsection{Additional notation}

It will sometimes be convenient to denote the process a name
quotes. We already have the notation $x = \quotep{P}$, but it will be
convenient to introduce an alternate notation, $\procn{x}$, when we
want to emphasize the connection to the use of the name. Note that, by
virtue of name equivalence, $\quotep{\procn{x}} \nameeq x$; so, the
notation is consistent with previous definitions.

Further, because names have structure it is possible to effect
substitutions on the basis of that structure. This means we need to
upgrade our notation for substitutions, which we accomplish by
adapting comprehension notation. Thus,

\begin{mathpar}
  P\{ y / x : x \in S \}
\end{mathpar}

is interpreted to mean the process derived from P by replacing (in a
capture-avoiding manner) each occurrence of $x$ in $S$ by $y$. For example,

\begin{mathpar}
  P\{ \quotep{\procn{x}|\procn{x}} / x : x \in \freenames{P} \}
\end{mathpar}

will replace each (occurrence) of a free name $x$ in $P$ by
$\quotep{\procn{x}|\procn{x}}$.

Also, we will avail ourselves of the notation $x^{L}$ and $x^{R}$ to
denote injections of a name into disjoint copies of the name
space. There are numerous ways to accomplish this. One example can be
found in \cite{MeredithR05}. This notation overloads to vectors of
names: $\vec{x}^{\pi} := (x_{i}^{\pi} \; : \; 0 \leq i < |\vec{x}| )$ where $\pi \in \{L,R\}$.

We also use $P^{\Box} := P|\Box$.

In \cite{MeredithR05} an interpretation of the new operator is
given. It turns out that there are several possible interpretations
all enjoying the requisite algebraic properties of the operator (see
\cite{milner91polyadicpi}). We will therefore make liberal use of
$(\nu\; \vec{x})P$.

% subsection the_syntax_and_semantics_of_the_notation_system (end)   

\input{qm2pi.qmops} 

\input{qm2pi.sterngerlach} 

\input{qm2pi.metric} 

% section concurrent_process_calculi (end)

%\input{qm2pi.proofsketch}

% section proof sketch (end)

%\input{qm2pi.slviaknots} 

% section spatial logic via knots (end)

\input{qm2pi.conclusion}

% section conclusion (end)

%\input{qm2pi.dtcodes} 

% section wiring algorithm (end)

\input{qm2pi.ack} 

% section acknowledgments (end)

\newpage


\bibliographystyle{plain}   
\bibliography{../../biblios/main.bib}

\input{qm2pi.rhodetails}

\end{document}



% section proof sketch (end)

%\section{Unlikely characters: spatial logic for
  knots}\label{sub:characteristic_formulae} % (fold)

Associated to the mobile process calculi are a family of logics known
as the Hennessy-Milner logics. These logics typically enjoy a
semantics interpreting formulae as sets of processes that when
factored through the encoding outlined above allows an identification
of classes of knots with logical formulae. In the context of this
encoding the sub-family known as the spatial logics \cite{CairesC03}
\cite{CairesC04} \cite{Caires04} are of particular interest providing
several important features for expressing and reasoning about
properties (i.e. classes) of knots. We hint here at how this may be done.

%\begin{description}
%\item [structural connectives] 
\subsubsection{Structural connectives} The spatial logics enjoy
structural connectives corresponding, at the logical level, to the
parallel composition ($P | Q$) and new name ($(\nu \; x)P$)
connectives for processes. As illustrated in the examples below, these
connectives are extremely expressive given the shape of our encoding.
%\item [decideable satisfaction]

\subsubsection{Decideable satisfaction}
In \cite{Caires04} the satisfaction relation is shown to be decideable
for a rich class of processes. It further turns out that the image of
the our encoding is a proper subset of that class. This result
provides the basis for an algorithm by which to search for knots
enjoying a given property.
%\item [characteristic formulae]

\subsubsection{Characteristic formulae}
In the same paper \cite{Caires04} , Caires presents a means of calculating
characteristic formulae, selecting equivalence classes of processes
up to a pre--specified depth limit on the support set of names. Composed with our
encoding, this characteristic formula can be used to select
characteristic formulae for knots.
%\end{description}

\subsubsection{Spatial logic formulae}

The grammar below (segmented for comprehension) summarizes the syntax
of spatial logic formulae. We employ illustrative examples in the
sequel to provide an intuitive understanding of their meaning
referring the reader to \cite{Caires04} for a more detailed explication
of the semantics.

\begin{mathpar}
  \inferrule* [lab=boolean] {} {{A,B} \bc T \;|\; \neg A \;|\; A \wedge B \;|\; \eta = \eta'}
  \and
  \inferrule* [lab=spatial] {} {|\; \pzero \;|\; A | B \;|\; x \text{\textregistered} A \;|\; \forall x . A \;|\;  H x . A}
  \and
  \inferrule* [lab=behavioral] {} {|\; \alpha . A}
  \and 
  \inferrule* [lab=recursion] {} {|\; X(\vec{u}) \;|\; \mu X(\vec{u}) . A}
  \and
  \inferrule* [lab=action] {} {\alpha \bc \langle x?(\vec{y}) \rangle \;|\; \langle x!(\vec{y}) \rangle \;|\; \langle \tau \rangle}
  \and 
  \inferrule* [lab=name] {} {\eta \bc x \;|\; \tau}
\end{mathpar} 

% subsection characteristic_formulae (end)   	 

\subsection{Example formulae}\label{sub:example_formulae_} % (fold)

\subsubsection{Crossing as formula.}
% 
% \begin{align*}
%   \frac{d}{dx} \sin x &= \cos x 
%   & \frac{d}{dx} e^x &= e^x \\
%   \frac{d}{dx} \cos x &= - \sin x 
%   & \frac{d}{dx} \log x &= \frac{1}{x} \\
% \end{align*} 

\begin{align*}
 \mu C(x_{0},x_{1},y_{0},y_{1},u).&(\langle x_{0}?(z) \rangle(\langle u! \rangle\langle y_{1}!z \rangle C(x_{0},x_{1},y_{0},y_{1},u)) & \\
  & \wedge \langle y_{1}?(z) \rangle (\langle u! \rangle \langle x_{0}!z \rangle C(x_{0},x_{1},y_{0},y_{1},u)) & \\
  & \wedge \langle x_{1}?(z) \rangle (\langle u? \rangle \langle y_{0}!z \rangle C(x_{0},x_{1},y_{0},y_{1},u)) & \\
  & \wedge \langle y_{0}?(z) \rangle (\langle u? \rangle \langle x_{1}!z \rangle C(x_{0},x_{1},y_{0},y_{1},u))) &
\end{align*}

The lexicographical similarity between the shape of this formulae and
the shape of definition of the process representing a crossing reveals
the intuitive meaning of this formulae. It describes the capabilities
of a process that has the right to represent a crossing. For example
it picks out processes that may perform an input on the port $x_0$ in
its initial menu of capabilities. What differentiates the formula
from the process, however, is that the crossing process is the
smallest candidate to satisfy the formula. Infinitely many other
processes -- with internal behavior hidden behind this interface, so
to speak -- also satisfy this formula. Even this simple formula,
then, can be seen to open a new view onto knots, providing a
computational interpretation of \emph{virtual} knots.

Note that this formula is derived by hand. A similar formula can be
derived by employing Caires' calculation of characteristic formula
\cite{Caires04} to the process representing a crossing. In light of
this discussion, we let
$\meaningof{C}_{\phi}(x0,x1,y0,y1,u)$ denote a formula specifying the
dynamics we wish to capture of a crossing. To guarantee we preserve
the shape of the interface and minimal semantics we demand that
$\meaningof{C}_{\phi}(x0,x1,y0,y1,u) \Rightarrow
\textbf{C}(x0,x1,y0,y1,u)$ where $\textbf{C}(x0,x1,y0,y1,u)$ denotes
the formula above.
                            
\subsubsection{Crossing number constraints.}
The moral content of the context lemma (Lemma \ref{context}) is that the notion of
``locality'' in the Reidemeister moves is effectively captured by the
parallel composition operator of the process calculus. This intuition
extends through the logic. Given a formula,
$\meaningof{C}_{\phi}(x0,x1,y0,y1,u)$, we can use the structural
connectives to specify constraints on crossing numbers, such as at
least $n$ crossings, or exactly $n$ crossings.
\begin{mathpar}
  \inferrule* [lab=at-least-n] {} { K^{\geq n}_{\phi}(\vec{xs},\vec{ys}) := \Pi_{i=0}^{n-1} Hu . \meaningof{C}_{\phi}(xs_i,ys_i,u) | T }
  \and 
  \inferrule* [lab=exactly-n] {} { K^{= n}_{\phi}(\vec{xs},\vec{ys}) := \Pi_{i=0}^{n-1} Hu . \meaningof{C}_{\phi}(xs_i,ys_i,u) | \neg (\forall x_0,y_0,x_1,y_1,u . \meaningof{C}_{\phi}(x_0,y_0,x_1,y_1,u) | T) }
\end{mathpar}

To round out this section, recall that the encoding of an $n$-crossing
knot decomposes into a parallel composition of $n$ \emph{copies} of a
crossing process together with a wiring harness. To specify different
knot classes with the same crossing number amounts to specifying
logical constraints on the wiring harness. In the interest of space,
we defer examples to a forthcoming paper. Suffice it to say that both
the conditions ``alternating knot'' and ``contains the tangle
corresponding to 5/3'' are expressible. For example, it is possible to
calculate the characteristic formula of a process corresponding to the
tangle 5/3 and conjoin it into the classifying formula via the
composition connective of the logic.

Finally, we wish to observe that it is entirely within reason to
contemplate a more domain-specific version of spatial logic tailored
to the shape of processes in the image of the encoding. Such a
domain-specific logic would have a better claim to the title formal
language of knot properties.

% subsection example_formulae_ (end)

% section knots_as_processes (end) 

% section spatial logic via knots (end)

\section{Conclusions and future work}

\paragraph{Testing physical space}
You, gentle reader, may wonder why of all the theorems to be proved
given this set up we pick the one above. In some sense it's hardly
central to quantum mechanics. We see it as central in the sense that
it firmly establishes a notion of physical space arising from a notion
of the equivalence of behavior. Relating bisimulation to a metric is a
big step forward, but one is faced with interpreting the relationship
of that metric space to something more physical. Quantum mechanical
notions of ``physical'' space are still far from intuitive, but by
relating this idea of distance as testing to calculations that predict
physical circumstances we are making a not insignificant step forward
toward an understanding of the physical space we inhabit as
essentially dynamic.

\paragraph{Effectivity and simulation}
One of the observations we have yet to make is that the entire program
spelled out here is effective. We have built various interpreters for
the reflective calculus at work in this interpretation. In principle,
then, we can simulate quantum mechanics on a computer. The place where
the simulation may lose fidelity is the infinitely branching summation
for the annihilator.

In this connection i also want to point out that the evaluation style
calculation of the inner product puts the non-determinism of the
summation right at the heart of measurement. This suggests that
Milner's original reduction-based formulation of the dynamics of his
calculi in terms of sums was not just notationally suggestive of a
notion of measure-and-continue but captured some significant part of
the physics.

\paragraph{Quantum continuations}
In light of this last observation i want to point out that the
predominant account of quantum mechanics is missing a key aspect of a
truly compositional story of the physical situation. In a real lab,
when a measurement is made the observation can be made to feed into
another device that then makes another measurement conditioned on the
results of the first. This means that after the superposition was
collapsed the entire experimental set up remained in
superposition. While QM offers a means of writing this down it doesn't
quite line up well with the well-trodden formulation of computation
and continuation that we see so succinctly expressed in Milner's
calculi. This suggests that there might be advantages to this account
of dynamics waiting to be explored.

\paragraph{Quantum logic}
In this connection, we also note that by virtue of having the
Hennessy-Milner construction, we can pull the construction through the
interpretation of QM. This gives us a natural candidate for a quantum
logic that enjoys an extremely tight connection with it's domain of
interpretation, making the construction much less ad hoc (rather it is
the image of functor!).

\paragraph{Quantum probabiity}
i have questions about the basis of the interpretation of inner
product as probability amplitude. In particular, using which
axiomatization of probability theory does the notion of probability
amplitude earn the right to be so dubbed? In other words, where is the
proof that the operation for calculating a probability amplitude (and
then squaring) satisfies the axioms of what it means to calculate a
probability? Even if such a proof exists (i have yet to find it in the
literature), i wonder if it might not be possible to turn things on
their heads. Can we view the calculation of the probability amplitude
as an axiomatization of probability? If so, then the definition we
give for calculating probability amplitude may provide the basis for
an \emph{effective} theory of probability.

\paragraph{Quantum vs ``biological'' information}
Finally, i want to conclude with a more philosophical observation. At
a recent workshop in which QM was a predominant topic i noticed
something about quantum information. The speaker was giving a riveting
discussion of axiomatic QM and showing how properties of ``no
cloning'' and ``no deleting'' emerged as consequences of the
axiomatization. Theorems of this form are necessary to give us a sense
of confidence that our axioms characterize the physical theory. What
struck me, though, was that if quantum information is neither erasable
nor replicable it is markedly different from \emph{life}. Two of the
things we know about life is that

\begin{itemize}
  \item it ends;
  \item to gain some measure of persistence, to transcend it's
    finitude it is imminently copyable.
\end{itemize}

Both of these qualities are summarized succinctly in the aphorism: all
flesh is grass. For me these two kinds of ``information'' -- call them
quantum and biological -- are end points on a spectrum of strategies
for persistence. At one end, we have those curious entities that enjoy
uniqueness and permanence; at the other, we have those who in the face
of a certain end and an uncertain present make a go of passing
something on. To me one of the more remarkable aspects of the latter
strategy is that in the presence of noise (and certain features of
copying) we get a kind of dynamism, a chance for improvement against a
given persistent condition.

% subsection other_calculi_other_bisimulations_and_geometry_as_behavior (end)




% section conclusion (end)

%\documentclass[12pt]{llncs}
%\documentclass{jktr}

\usepackage[pdftex]{hyperref}                   
\usepackage {listings}
\usepackage {mathpartir}
\usepackage{bcprules}
%\usepackage{listings}
                       
\usepackage{graphicx} 
%\usepackage[margins=2.5cm,nohead,nofoot]{geometry}
%\usepackage{geometry}
\usepackage{amsfonts}
\usepackage{amstext}
\usepackage{latexsym}
\usepackage{amssymb}
\usepackage{color}


%\include{myPreamble}
\include{qm2pi.local} 

%\ifpdf
%\usepackage[pdftex]{graphicx}
%\else
%\usepackage{graphicx}
%\fi

 % \ifpdf
%  \usepackage{pdfsync}
%  \if


%\title{Brief Article}
%\author{David F. Snyder}
%\author{L.G. Meredith}

%\address{Dept. of Math., Texas State University--San Marcos, San Marcos, TX 78666}
       
\pagestyle{empty}


\begin{document}

\lstset{language=[Objective]Caml,frame=shadowbox}

\input{qm2pi.front}

% section front matter (end)

\input{qm2pi.intro} 
 
% section introduction (end)

% \input{qm2pi.knotations} 

% section notation (end)

\input{qm2pi.process.calculi} 

% section concurrent_process_calculi_and_spatial_logics_ (end)
    
%\input{qm2pi.knots2pi} 

%\input{qm2pi.trefoil} 

%\input{qm2pi.mainthm} 

% subsection basic_interpretation (end)

%\input{qm2pi.rho.presentation} 
\subsection{The syntax and semantics of the notation system}\label{sub:the_syntax_and_semantics_of_the_notation_system} % (fold)

We now summarize a technical presentation of the calculus that
embodies our theory of dynamics. The typical presentation of such a
calculus follows the style of giving generators and relations on
them. The grammar, below, describing term constructors, freely
generates the set of processes, $\Proc$. This set is then quotiented
by a relation known as structural congruence and it is over this set
that the notion of dynamics is expressed. This presentation is
essentially that of \cite{MeredithR05} with the addition of
polyadicity and summation. For readability we have relegated some of
the technical subtleties to an appendix.

\subsubsection{Process grammar}\label{subsub:process_grammar}

\begin{mathpar}
  \inferrule* [lab=synchronization] {} {{M} \bc \pzero \;|\; x?F \;|\; x!C }
  \and
  \inferrule* [lab=abstraction] {} {{F} \bc (x)P}
  \and
  \inferrule* [lab=concretion] {} {{C} \bc \langle Q \rangle}
  \and
  \inferrule* [lab=process] {} {{P,Q} \bc M \;| \;P|Q \;|\; @{x}}
  \and
  \inferrule* [lab=name] {} {{x} \bc \quotep{P}}
\end{mathpar} 

Note that $\vec{x}$ (resp. $\vec{P}$) denotes a vector of names
(resp. processes) of length $|\vec{x}|$ (resp. $|\vec{P}|$). We adopt
the following useful abbreviations.

\begin{mathpar}
   x?(\vec{y}).P := x.(\vec{y})P \and  x\clift{\vec{P}} := x.\clift{\vec{P}}
   \and x!(y) := \lift{x}{\dropn{y}}
   \and \Pi_{i=0}^{n-1}P_i := P_0 | \ldots | P_{n-1}
\end{mathpar}

\subsubsection{Structural congruence}

\paragraph{Free and bound names and alpha-equivalence.} At the
core of structural equivalence is alpha-equivalence which identifies
process that are the same up to a change of variable. Formally, we
recognize the distinction between free and bound names. The free names
of a process, $\freenames{P}$, may be calculated recursively as
follows:

\begin{mathpar}
\freenames{\pzero} := \emptyset
  \and \\
  \freenames{x?(y).P} := \{ x \} \cup (\freenames{P} \setminus \{ y \})
  \and 
  \freenames{x!\langle P \rangle} := \{ x \} \cup \{ P \} 
  \and \\
  \freenames{P|Q} := \freenames{P} \cup \freenames{Q}
  \and \\
  \freenames{@{x}} := \{ x \}
\end{mathpar}

$\pi$
$\quotep{\pi}$

$\freenames{-} : \pi \to \mathcal{P}(\quotep{\pi})$

\begin{eqnarray*}
  \freenames{\pzero} & := & \emptyset \\
  \freenames{x?(y).P} & := & \{ x \} \cup (\freenames{P} \setminus \{ y \}) \\
  \freenames{x!\langle P \rangle} & := & \{ x \} \cup \{ P \} \\
  \freenames{P|Q} & := & \freenames{P} \cup \freenames{Q} \\
  \freenames{\dropn{x}} & := & \{ x \}
\end{eqnarray*}

The bound names of a process, $\boundnames{P}$, are those names occurring in $P$
that are not free. For example, in $x?(y).0$, the name $x$ is free, while $y$ is bound.

\begin{mathpar}
  \inferrule* [lab=monoidal-laws] {} { P|Q \equiv Q|P \and P|0 \equiv P \and P|(Q|R) \equiv (P|Q)|R }
\end{mathpar}

\begin{mathpar}
  \inferrule* [lab=alpha-equivalence] {} { (x)P \equiv (y)P\{y/x\} \and y \not\in \freenames{P} }
\end{mathpar}

\begin{definition}
Then two processes, $P,Q$, are alpha-equivalent if $P = Q\{\vec{y}/\vec{x}\}$ for
some $\vec{x} \in \boundnames{Q},\vec{y} \in \boundnames{P}$, where $Q\{\vec{y}/\vec{x}\}$
denotes the capture-avoiding substitution of $\vec{y}$ for $\vec{x}$ in $Q$.
\end{definition}

\begin{definition}
  The {\em structural congruence} \cite{SangiorgiWalker} , $\equiv$,
  between processes is the least congruence containing
  alpha-equivalence, satisfying the abelian monoid laws
  (associativity, commutativity and $\pzero$ as identity) for parallel
  composition $|$ and for summation $+$.
\end{definition}

\subsection{Name equivalence}

We take name equivalence, written $\nameeq$, to be the smallest
equivalence relation generated by the following rules.

\begin{mathpar}
\inferrule*[lab=Quote-drop]
{ }
{ \quotep{@{x}} \nameeq x }

\inferrule*[lab=Struct-equiv]
{ P \scong Q }
{ \quotep{P} \nameeq \quotep{Q} }
\end{mathpar}

The astute reader will have noticed that the mutual recursion of names
and processes imposes a mutual recursion on alpha-equivalence and
structural equivalence via name-equivalence. Fortunately, all of this
works out pleasantly and we may calculate in the natural way, free of
concern. The reader interested in the details is referred to the
appendix \ref{appendix:rho_details}.

\subsection{Substitution}

We use $\Proc$ for the set of processes, $\QProc$ for the set of
names, and $\id{\{}\vec{y} / \vec{x} \id{\}}$ to denote partial maps,
$s : \QProc \rightarrow \QProc$. A map, $s$ lifts, uniquely, to a map
on process terms, $\widehat{s} : \Proc \rightarrow \Proc$ by the
following equations.

\begin{mathpar}
  (0) \psubstp{Q}{P} := 0 \\
  (R \juxtap S) \psubstp{Q}{P}
  :=    
  (R)\psubstp{Q}{P} \juxtap (S) \psubstp{Q}{P} \\
  (x?(y).R) \psubstp{Q}{P}    
  :=    
  (x)\substp{Q}{P} (z)\concat( (R \psubstn{z}{y}) \psubstp{Q}{P} ) \\
  (\lift{x}{R}) \psubstp{Q}{P}  
  :=
  \lift{(x)\substp{Q}{P}}{ R \psubstp{Q}{P} } \\
%   (\dropn{x})  \psubstp{Q}{P}       
%   := 
%   \left\{ 
%     \begin{array}{ccc} 
%       \dropn{\quotep{Q}} & & x \nameeq \quotep{P} \\
%       \dropn{x} & & otherwise \\
%     \end{array}
%   \right. 
  (\dropn{x})  \psubstp{Q}{P}       
  := 
  \left\{ 
    \begin{array}{ccc} 
      Q & & x \nameeq \quotep{P} \\
      \dropn{x} & & otherwise \\
    \end{array}
  \right.
\end{mathpar}
 

where

\begin{eqnarray}
  (x)\id{\{} \lpquote Q \rpquote / \lpquote P \rpquote \id{\}}            = 
  \left\{ 
    \begin{array}{ccc}
      \lpquote Q \rpquote & & x \nameeq \lpquote P \rpquote \\
      x & & otherwise \\
    \end{array}
  \right. \nonumber
\end{eqnarray}

and $z$ is chosen distinct from $\quotep{P}$, $\quotep{Q}$, the free
names in $Q$, and all the names in $R$. Our $\alpha$-equivalence will
be built in the standard way from this substitution.

\begin{remark}\label{rem:no_self_referential_names}
  One consequence of these definitions is that $\forall P. \quotep{P}
  \not\in \freenames{P}$.
\end{remark}

\subsection{ Dynamic quote: an example }

Anticipating something of what's to come, consider applying the
substitution, $\widehat{\id{\{}u / z \id{\}}}$, to the following pair
of processes, $\lift{w}{y!(z)}$ and $w[ \lpquote y!(z) \rpquote ]$.

\begin{eqnarray}
	\lift{w}{y!(z)}\widehat{\id{\{}u / z \id{\}}}
		& = &
		\lift{w}{y!(u)} \nonumber\\
	w[ \lpquote y!(z) \rpquote ] \widehat{ \id{\{}u / z \id{\}} }
		& = &
		w[ \lpquote y!(z) \rpquote ] \nonumber
\end{eqnarray}

Because the body of the process between quotes is impervious to
substitution, we get radically different answers. In fact, by
examining the first process in an input context,
e.g. $x?(z).\lift{w}{y!(z)}$, we see that the process under the lift
operator may be shaped by prefixed inputs binding a name inside it. In
this sense, the lift operator will be seen as a way to dynamically
construct processes before reifying them as names.

Finally equipped with these standard features we can present the
dynamics of the calculus.

\subsubsection{Operational semantics} 

Finally, we introduce the computational dynamics. What marks these
algebras as distinct from other more traditionally studied algebraic
structures, e.g. vector spaces or polynomial rings, is the manner in
which dynamics is captured. In traditional structures, dynamics is typically
expressed through morphisms between such structures, as in linear maps
between vector spaces or morphisms between rings. In algebras
associated with the semantics of computation, the dynamics is
expressed as part of the algebraic structure itself, through a
reduction reduction relation typically denoted by $\red$. Below, we
give a recursive presentation of this relation for the calculus used
in the encoding.

$\red \subseteq \pi \times \pi$
$\red : \pi \to \mathcal{P}(\pi)$

\begin{mathpar}
  \inferrule* [lab=Comm] { \textsf{match}( x_{src}, x_{trgt} ) } { x_{trgt}?(y)P \; | \; x_{src}!\langle {Q} \rangle \red P\{\quotep{Q}/y}\} }
  \and \\
  \inferrule* [lab=Par] {{P} \red {P}'} {{{P} | {Q}} \red {{P}' | {Q}}}
  \and
  \inferrule* [lab=Equiv]{{{P} \scong {P}'} \andalso {{P}' \red {Q}'} \andalso {{Q}' \scong {Q}}}{{P} \red {Q}}
\end{mathpar}

\begin{eqnarray*}
  match_{\equiv} (\quotep{P},\quotep{Q}) & := & P \equiv Q \\
  match_{\dagger}(\quotep{P},\quotep{Q}) & := & \forall R. P|Q \red^{*} R => R \red^{*} 0 \\
  match_{K}(\quotep{P},\quotep{Q}) & := & K \mbox{ for some context } K
\end{eqnarray*}

$u?(x)P | u!\langle Q \rangle \red P\{\quotep{Q}/x\}$

%We write $\wred$ for $\red^*$, and $P\red$ if $\exists Q $ such that $ P \red Q$.
We write $P\red$ if $\exists Q $ such that $ P \red Q$ and $P\not\red$, otherwise.

\section{Replication}

As mentioned before, it is known that replication (and hence
recursion) can be implemented in a higher-order process algebra
\cite{SangiorgiWalker}. As our first example of calculation with the
machinery thus far presented we give the construction explicitly in
the {\rhoc}.

\begin{eqnarray}
	D_{x} & := & \prefix{x}{y}{(\binpar{\outputp{x}{y}}{@{y}})} \nonumber\\
	\bangp_{x}{P} & := & \binpar{{x}!\langle{\binpar{D_{x}}{P}}\rangle}{D_{x}} \nonumber
\end{eqnarray}

\begin{eqnarray}
	\bangp_{x}{P} & & \nonumber\\
	=
	& {x}!\langle{(\prefix{x}{y}{(\outputp{x}{y} | @{y})) | P}}\rangle 
	      | \prefix{x}{y}{(\outputp{x}{y} | @{y})} & \nonumber\\
	\red
	& (\outputp{x}{y} | @{y})\substn{\quotep{(\prefix{x}{y}{(@{y} | \outputp{x}{y})) | P}}}{y} & \nonumber\\
	=
	& \outputp{x}{\quotep{(\prefix{x}{y}{(\outputp{x}{y} | @{y})) | P}}}
	  | {(\prefix{x}{y}{(\outputp{x}{y} | @{y})) | P}} & \nonumber\\
	\red
	& \ldots & \nonumber\\
	\red^*
	& P | P | \ldots & \nonumber
\end{eqnarray}

Of course, this encoding, as an implementation, runs away, unfolding
$\bangp{P}$ eagerly. A lazier and more implementable replication
operator, restricted to input-guarded processes, may be obtained as follows.

\begin{eqnarray}
\bangp{\prefix{u}{v}{P}} 
	:= 
	\binpar{\lift{x}{\prefix{u}{v}{(\binpar{D(x)}{P})}}}{D(x)} \nonumber
\end{eqnarray}

\begin{remark}
  Note that the lazier definition still does not deal with summation
  or mixed summation (i.e. sums over input and output). The reader is
  invited to construct definitions of replication that deal with these
  features. 

  Further, the definitions are parameterized in a name, $x$. Can you,
  gentle reader, make a definition that eliminates this parameter and
  guarantees no accidental interaction between the replication
  machinery and the process being replicated -- i.e. no accidental
  sharing of names used by the process to get its work done and the
  name(s) used by the replication to effect copying. This latter
  revision of the definition of replication is crucial to obtaining
  the expected identity $!!P \sim !P$.
\end{remark}

\begin{remark}\label{rem:paradoxical_combinator}
  The reader familiar with the lambda calculus will have noticed the
  similarity between $D$ and the paradoxical combinator.

  [Ed. note: the existence of this seems to suggest we have to be more
  restrictive on the set of processes and names we admit if we are to
  support no-cloning.]
\end{remark}

\subsubsection{Bisimulation}

The computational dynamics gives rise to another kind of equivalence,
the equivalence of computational behavior. As previously mentioned
this is typically captured \emph{via} some form of bisimulation.

% The notion we use in this paper is weak barbed bisimulation
% \cite{milner91polyadicpi}.

The notion we use in this paper is derived from weak barbed
bisimulation \cite{milner91polyadicpi}. 

\begin{definition}
An \emph{observation relation}, $\downarrow_{\mathcal N}$, over a set
of names, $\mathcal N$, is the smallest relation satisfying the rules
below.

\infrule[Out-barb]{y \in {\mathcal N}, \; x \nameeq y}
		  {\outputp{x}{v} \downarrow_{\mathcal N} x}
\infrule[Par-barb]{\mbox{$P\downarrow_{\mathcal N} x$ or $Q\downarrow_{\mathcal N} x$}}
		  {\binpar{P}{Q} \downarrow_{\mathcal N} x}

We write $P \Downarrow_{\mathcal N} x$ if there is $Q$ such that 
$P \wred Q$ and $Q \downarrow_{\mathcal N} x$.
\end{definition}

\begin{definition}
%\label{def.bbisim}
An  ${\mathcal N}$-\emph{barbed bisimulation} over a set of names, ${\mathcal N}$, is a symmetric binary relation 
${\mathcal S}_{\mathcal N}$ between agents such that $P\rel{S}_{\mathcal N}Q$ implies:
\begin{enumerate}
\item If $P \red P'$ then $Q \wred Q'$ and $P'\rel{S}_{\mathcal N} Q'$.
\item If $P\downarrow_{\mathcal N} x$, then $Q\Downarrow_{\mathcal N} x$.
\end{enumerate}
$P$ is ${\mathcal N}$-barbed bisimilar to $Q$, written
$P \wbbisim_{\mathcal N} Q$, if $P \rel{S}_{\mathcal N} Q$ for some ${\mathcal N}$-barbed bisimulation ${\mathcal S}_{\mathcal N}$.
\end{definition}

$\mathcal{R} \subseteq \pi \times \pi$

$P \mathcal{R} Q => \forall P'. P \red P' \Rightarrow \exists Q'. Q \red Q', P' \mathcal{R} Q'$

$P \vdash x \Rightarrow Q \vdash x$

\begin{mathpar}
  \inferrule*[lab=Out-barb]{x \nameeq y}{{y}!\langle{Q}\rangle \vdash x}
  \and
  \inferrule*[lab=Par-barb]{\mbox{$P\vdash x$ or $Q\vdash x$}}{\binpar{P}{Q} \vdash x}
\end{mathpar}

\subsubsection{Contexts}

One of the principle advantages of computational calculi like the
$\pi$-calculus is a well-defined notion of context,
contextual-equivalence and a correlation between
contextual-equivalence and notions of bisimulation. The notion of
context allows the decomposition of a process into (sub-)process and
its syntactic environment, its context. Thus, a context may be
thought of as a process with a ``hole'' (written $\Box$) in it. The
application of a context $M$ to a process $P$, written $M[P]$, is
tantamount to filling the hole in $M$ with $P$. In this paper we do
not need the full weight of this theory, but do make use of the notion
of context in the proof the main theorem. 

\begin{mathpar}
  \inferrule* [lab=summation] {} {{M_{M},M_{N}} \bc \Box \;|\; x.M_{A} \;|\; M_{M}+M_{N}}
  \and
  \inferrule* [lab=agent] {} {{M_{A}} \bc (\vec{x})M_{P} \;| \; \clift{P_0,\ldots,M_{P},\ldots,P_N}}
  \and \\
  \inferrule* [lab=process] {} {{M_{P}} \bc M_{N} \;| \;P|M_{P} }
\end{mathpar} 

\begin{mathpar}
  \inferrule* [lab=sychronization] {} {M_{N} \bc \Box \;|\; x?M_{F} \;|\; x!M_{C}}
  \and
  \inferrule* [lab=abstraction] {} {{M_{F}} \bc (x)M_{P} }
  \and
  \inferrule* [lab=concretion] {} {{M_{C}} \bc \langle M_{P} \rangle }
  \and \\
  \inferrule* [lab=process] {} {{M_{P}} \bc M_{N} \;| \;P|M_{P} }
\end{mathpar}

\begin{definition}[contextual application] Given a context $M$, and
  process $P$, we define the \emph{contextual application}, $M[P] :=
  M\{P/\Box\}$. That is, the contextual application of M to P is the
  substitution of $P$ for $\Box$ in $M$.
\end{definition}

$\meaningof{-} : L \to \mathcal{P}(\pi)$

\begin{mathpar}
  \inferrule* [lab=collection] {} {\meaningof{true} = \pi, \and \meaningof{~E} = \pi \setminus \meaningof{E}, \and \meaningof{E_{1} \& E_{2}} = \meaningof{E_{1}} \cap \meaningof{E_{2}}}
\end{mathpar}

\begin{mathpar}
  \inferrule* [lab=structure] {} {\meaningof{0} = \{ P \in \pi | P \equiv 0 \}, \and \\ \meaningof{E_1 | E_2} = \{ P \in \pi | P \equiv P_{1} | P_{2}, P_{1} \in \meaningof{E_{1}}, P_{2} \in \meaningof{E_2}\} }
\end{mathpar}

\begin{mathpar}
 \inferrule* [lab=behavior] {} {\meaningof{\langle a?b \rangle E} = \{ P \in \pi | P \equiv Q | u?(y)P', \\ \and \\\\ \and \\ \;\;\; u \in \meaningof{a}, \forall z.P'\{z/y\} \in \meaningof{E\{z/b\}}\}, \and \\ \meaningof{a!E} = \{ P \in \pi | P \equiv Q | x!\langle P' \rangle, x \in \meaningof{a} P' \in \meaningof{E}\} }
\end{mathpar}

\begin{mathpar}
 \inferrule* [lab=nominal] {} {\meaningof{\quotep{E}} = \{ \quotep{P} \in \quotep{\pi} | P \in \meaningof{E} \}, \and \meaningof{\quotep{P}} = \{ \quotep{Q} \in \quotep{\pi} | P \equiv Q \} \and \\ \meaningof{@\quotep{E}} = \{ P \in \pi | P \equiv @x, x \in \meaningof{E} \}}
\end{mathpar}

\begin{eqnarray*}
  \\
  \meaningof{-} : TS \to ST
\end{eqnarray*}

\begin{eqnarray*}
  \\
  L : TS \to ST
\end{eqnarray*}

\begin{eqnarray*}
  \\
  P \models E \iff P \in \meaningof{E}
\end{eqnarray*}

\begin{eqnarray*}
  P \approx_{L} Q \iff \forall E \in L. P \models E \iff Q \models E
\end{eqnarray*}

\begin{eqnarray*}
  P \approx_{K} Q
\end{eqnarray*}

\begin{eqnarray*}
  P \approx Q
\end{eqnarray*}

$\approx_{K} = \approx = \approx_{L}$

\subsubsection{Contextual duality}

Note that contexts extend the quotation operation to a family of
operations from processes to names. Given a context, $M$, we can
define a \emph{nominal context}, $\quotep{M}$ by $\quotep{M}[P] :=
\quotep{M[P]}$. To foreshadow what is to come we observe that these
operations enjoy a duality with processes very much like the duality
between vectors and maps from vectors to scalars.

Further, because the calculus is essentially higher-order, we have a
correspondence between contexts and processes. More specifically,
given a name $x$ and a context $M$ we can construct $M^{*}_{x}$ such
that 

\begin{mathpar}
  M^{*}_{x} | \lift{x}{P} \red M[P]
\end{mathpar}

namely,

\begin{mathpar}
  M^{*}_{x} := x?(u).M[\dropn{u}]
\end{mathpar}

The dependence of $M^{*}_{x}$ on a name makes it an abstraction, 

\begin{mathpar}
  M^{*} := (x)x?(u).M[\dropn{u}]
\end{mathpar}

\subsection{Additional notation}

It will sometimes be convenient to denote the process a name
quotes. We already have the notation $x = \quotep{P}$, but it will be
convenient to introduce an alternate notation, $\procn{x}$, when we
want to emphasize the connection to the use of the name. Note that, by
virtue of name equivalence, $\quotep{\procn{x}} \nameeq x$; so, the
notation is consistent with previous definitions.

Further, because names have structure it is possible to effect
substitutions on the basis of that structure. This means we need to
upgrade our notation for substitutions, which we accomplish by
adapting comprehension notation. Thus,

\begin{mathpar}
  P\{ y / x : x \in S \}
\end{mathpar}

is interpreted to mean the process derived from P by replacing (in a
capture-avoiding manner) each occurrence of $x$ in $S$ by $y$. For example,

\begin{mathpar}
  P\{ \quotep{\procn{x}|\procn{x}} / x : x \in \freenames{P} \}
\end{mathpar}

will replace each (occurrence) of a free name $x$ in $P$ by
$\quotep{\procn{x}|\procn{x}}$.

Also, we will avail ourselves of the notation $x^{L}$ and $x^{R}$ to
denote injections of a name into disjoint copies of the name
space. There are numerous ways to accomplish this. One example can be
found in \cite{MeredithR05}. This notation overloads to vectors of
names: $\vec{x}^{\pi} := (x_{i}^{\pi} \; : \; 0 \leq i < |\vec{x}| )$ where $\pi \in \{L,R\}$.

We also use $P^{\Box} := P|\Box$.

In \cite{MeredithR05} an interpretation of the new operator is
given. It turns out that there are several possible interpretations
all enjoying the requisite algebraic properties of the operator (see
\cite{milner91polyadicpi}). We will therefore make liberal use of
$(\nu\; \vec{x})P$.

% subsection the_syntax_and_semantics_of_the_notation_system (end)   

\input{qm2pi.qmops} 

\input{qm2pi.sterngerlach} 

\input{qm2pi.metric} 

% section concurrent_process_calculi (end)

%\input{qm2pi.proofsketch}

% section proof sketch (end)

%\input{qm2pi.slviaknots} 

% section spatial logic via knots (end)

\input{qm2pi.conclusion}

% section conclusion (end)

%\input{qm2pi.dtcodes} 

% section wiring algorithm (end)

\input{qm2pi.ack} 

% section acknowledgments (end)

\newpage


\bibliographystyle{plain}   
\bibliography{../../biblios/main.bib}

\input{qm2pi.rhodetails}

\end{document}

 

% section wiring algorithm (end)

\documentclass[12pt]{llncs}
%\documentclass{jktr}

\usepackage[pdftex]{hyperref}                   
\usepackage {listings}
\usepackage {mathpartir}
\usepackage{bcprules}
%\usepackage{listings}
                       
\usepackage{graphicx} 
%\usepackage[margins=2.5cm,nohead,nofoot]{geometry}
%\usepackage{geometry}
\usepackage{amsfonts}
\usepackage{amstext}
\usepackage{latexsym}
\usepackage{amssymb}
\usepackage{color}


%\include{myPreamble}
\include{qm2pi.local} 

%\ifpdf
%\usepackage[pdftex]{graphicx}
%\else
%\usepackage{graphicx}
%\fi

 % \ifpdf
%  \usepackage{pdfsync}
%  \if


%\title{Brief Article}
%\author{David F. Snyder}
%\author{L.G. Meredith}

%\address{Dept. of Math., Texas State University--San Marcos, San Marcos, TX 78666}
       
\pagestyle{empty}


\begin{document}

\lstset{language=[Objective]Caml,frame=shadowbox}

\input{qm2pi.front}

% section front matter (end)

\input{qm2pi.intro} 
 
% section introduction (end)

% \input{qm2pi.knotations} 

% section notation (end)

\input{qm2pi.process.calculi} 

% section concurrent_process_calculi_and_spatial_logics_ (end)
    
%\input{qm2pi.knots2pi} 

%\input{qm2pi.trefoil} 

%\input{qm2pi.mainthm} 

% subsection basic_interpretation (end)

%\input{qm2pi.rho.presentation} 
\subsection{The syntax and semantics of the notation system}\label{sub:the_syntax_and_semantics_of_the_notation_system} % (fold)

We now summarize a technical presentation of the calculus that
embodies our theory of dynamics. The typical presentation of such a
calculus follows the style of giving generators and relations on
them. The grammar, below, describing term constructors, freely
generates the set of processes, $\Proc$. This set is then quotiented
by a relation known as structural congruence and it is over this set
that the notion of dynamics is expressed. This presentation is
essentially that of \cite{MeredithR05} with the addition of
polyadicity and summation. For readability we have relegated some of
the technical subtleties to an appendix.

\subsubsection{Process grammar}\label{subsub:process_grammar}

\begin{mathpar}
  \inferrule* [lab=synchronization] {} {{M} \bc \pzero \;|\; x?F \;|\; x!C }
  \and
  \inferrule* [lab=abstraction] {} {{F} \bc (x)P}
  \and
  \inferrule* [lab=concretion] {} {{C} \bc \langle Q \rangle}
  \and
  \inferrule* [lab=process] {} {{P,Q} \bc M \;| \;P|Q \;|\; @{x}}
  \and
  \inferrule* [lab=name] {} {{x} \bc \quotep{P}}
\end{mathpar} 

Note that $\vec{x}$ (resp. $\vec{P}$) denotes a vector of names
(resp. processes) of length $|\vec{x}|$ (resp. $|\vec{P}|$). We adopt
the following useful abbreviations.

\begin{mathpar}
   x?(\vec{y}).P := x.(\vec{y})P \and  x\clift{\vec{P}} := x.\clift{\vec{P}}
   \and x!(y) := \lift{x}{\dropn{y}}
   \and \Pi_{i=0}^{n-1}P_i := P_0 | \ldots | P_{n-1}
\end{mathpar}

\subsubsection{Structural congruence}

\paragraph{Free and bound names and alpha-equivalence.} At the
core of structural equivalence is alpha-equivalence which identifies
process that are the same up to a change of variable. Formally, we
recognize the distinction between free and bound names. The free names
of a process, $\freenames{P}$, may be calculated recursively as
follows:

\begin{mathpar}
\freenames{\pzero} := \emptyset
  \and \\
  \freenames{x?(y).P} := \{ x \} \cup (\freenames{P} \setminus \{ y \})
  \and 
  \freenames{x!\langle P \rangle} := \{ x \} \cup \{ P \} 
  \and \\
  \freenames{P|Q} := \freenames{P} \cup \freenames{Q}
  \and \\
  \freenames{@{x}} := \{ x \}
\end{mathpar}

$\pi$
$\quotep{\pi}$

$\freenames{-} : \pi \to \mathcal{P}(\quotep{\pi})$

\begin{eqnarray*}
  \freenames{\pzero} & := & \emptyset \\
  \freenames{x?(y).P} & := & \{ x \} \cup (\freenames{P} \setminus \{ y \}) \\
  \freenames{x!\langle P \rangle} & := & \{ x \} \cup \{ P \} \\
  \freenames{P|Q} & := & \freenames{P} \cup \freenames{Q} \\
  \freenames{\dropn{x}} & := & \{ x \}
\end{eqnarray*}

The bound names of a process, $\boundnames{P}$, are those names occurring in $P$
that are not free. For example, in $x?(y).0$, the name $x$ is free, while $y$ is bound.

\begin{mathpar}
  \inferrule* [lab=monoidal-laws] {} { P|Q \equiv Q|P \and P|0 \equiv P \and P|(Q|R) \equiv (P|Q)|R }
\end{mathpar}

\begin{mathpar}
  \inferrule* [lab=alpha-equivalence] {} { (x)P \equiv (y)P\{y/x\} \and y \not\in \freenames{P} }
\end{mathpar}

\begin{definition}
Then two processes, $P,Q$, are alpha-equivalent if $P = Q\{\vec{y}/\vec{x}\}$ for
some $\vec{x} \in \boundnames{Q},\vec{y} \in \boundnames{P}$, where $Q\{\vec{y}/\vec{x}\}$
denotes the capture-avoiding substitution of $\vec{y}$ for $\vec{x}$ in $Q$.
\end{definition}

\begin{definition}
  The {\em structural congruence} \cite{SangiorgiWalker} , $\equiv$,
  between processes is the least congruence containing
  alpha-equivalence, satisfying the abelian monoid laws
  (associativity, commutativity and $\pzero$ as identity) for parallel
  composition $|$ and for summation $+$.
\end{definition}

\subsection{Name equivalence}

We take name equivalence, written $\nameeq$, to be the smallest
equivalence relation generated by the following rules.

\begin{mathpar}
\inferrule*[lab=Quote-drop]
{ }
{ \quotep{@{x}} \nameeq x }

\inferrule*[lab=Struct-equiv]
{ P \scong Q }
{ \quotep{P} \nameeq \quotep{Q} }
\end{mathpar}

The astute reader will have noticed that the mutual recursion of names
and processes imposes a mutual recursion on alpha-equivalence and
structural equivalence via name-equivalence. Fortunately, all of this
works out pleasantly and we may calculate in the natural way, free of
concern. The reader interested in the details is referred to the
appendix \ref{appendix:rho_details}.

\subsection{Substitution}

We use $\Proc$ for the set of processes, $\QProc$ for the set of
names, and $\id{\{}\vec{y} / \vec{x} \id{\}}$ to denote partial maps,
$s : \QProc \rightarrow \QProc$. A map, $s$ lifts, uniquely, to a map
on process terms, $\widehat{s} : \Proc \rightarrow \Proc$ by the
following equations.

\begin{mathpar}
  (0) \psubstp{Q}{P} := 0 \\
  (R \juxtap S) \psubstp{Q}{P}
  :=    
  (R)\psubstp{Q}{P} \juxtap (S) \psubstp{Q}{P} \\
  (x?(y).R) \psubstp{Q}{P}    
  :=    
  (x)\substp{Q}{P} (z)\concat( (R \psubstn{z}{y}) \psubstp{Q}{P} ) \\
  (\lift{x}{R}) \psubstp{Q}{P}  
  :=
  \lift{(x)\substp{Q}{P}}{ R \psubstp{Q}{P} } \\
%   (\dropn{x})  \psubstp{Q}{P}       
%   := 
%   \left\{ 
%     \begin{array}{ccc} 
%       \dropn{\quotep{Q}} & & x \nameeq \quotep{P} \\
%       \dropn{x} & & otherwise \\
%     \end{array}
%   \right. 
  (\dropn{x})  \psubstp{Q}{P}       
  := 
  \left\{ 
    \begin{array}{ccc} 
      Q & & x \nameeq \quotep{P} \\
      \dropn{x} & & otherwise \\
    \end{array}
  \right.
\end{mathpar}
 

where

\begin{eqnarray}
  (x)\id{\{} \lpquote Q \rpquote / \lpquote P \rpquote \id{\}}            = 
  \left\{ 
    \begin{array}{ccc}
      \lpquote Q \rpquote & & x \nameeq \lpquote P \rpquote \\
      x & & otherwise \\
    \end{array}
  \right. \nonumber
\end{eqnarray}

and $z$ is chosen distinct from $\quotep{P}$, $\quotep{Q}$, the free
names in $Q$, and all the names in $R$. Our $\alpha$-equivalence will
be built in the standard way from this substitution.

\begin{remark}\label{rem:no_self_referential_names}
  One consequence of these definitions is that $\forall P. \quotep{P}
  \not\in \freenames{P}$.
\end{remark}

\subsection{ Dynamic quote: an example }

Anticipating something of what's to come, consider applying the
substitution, $\widehat{\id{\{}u / z \id{\}}}$, to the following pair
of processes, $\lift{w}{y!(z)}$ and $w[ \lpquote y!(z) \rpquote ]$.

\begin{eqnarray}
	\lift{w}{y!(z)}\widehat{\id{\{}u / z \id{\}}}
		& = &
		\lift{w}{y!(u)} \nonumber\\
	w[ \lpquote y!(z) \rpquote ] \widehat{ \id{\{}u / z \id{\}} }
		& = &
		w[ \lpquote y!(z) \rpquote ] \nonumber
\end{eqnarray}

Because the body of the process between quotes is impervious to
substitution, we get radically different answers. In fact, by
examining the first process in an input context,
e.g. $x?(z).\lift{w}{y!(z)}$, we see that the process under the lift
operator may be shaped by prefixed inputs binding a name inside it. In
this sense, the lift operator will be seen as a way to dynamically
construct processes before reifying them as names.

Finally equipped with these standard features we can present the
dynamics of the calculus.

\subsubsection{Operational semantics} 

Finally, we introduce the computational dynamics. What marks these
algebras as distinct from other more traditionally studied algebraic
structures, e.g. vector spaces or polynomial rings, is the manner in
which dynamics is captured. In traditional structures, dynamics is typically
expressed through morphisms between such structures, as in linear maps
between vector spaces or morphisms between rings. In algebras
associated with the semantics of computation, the dynamics is
expressed as part of the algebraic structure itself, through a
reduction reduction relation typically denoted by $\red$. Below, we
give a recursive presentation of this relation for the calculus used
in the encoding.

$\red \subseteq \pi \times \pi$
$\red : \pi \to \mathcal{P}(\pi)$

\begin{mathpar}
  \inferrule* [lab=Comm] { \textsf{match}( x_{src}, x_{trgt} ) } { x_{trgt}?(y)P \; | \; x_{src}!\langle {Q} \rangle \red P\{\quotep{Q}/y}\} }
  \and \\
  \inferrule* [lab=Par] {{P} \red {P}'} {{{P} | {Q}} \red {{P}' | {Q}}}
  \and
  \inferrule* [lab=Equiv]{{{P} \scong {P}'} \andalso {{P}' \red {Q}'} \andalso {{Q}' \scong {Q}}}{{P} \red {Q}}
\end{mathpar}

\begin{eqnarray*}
  match_{\equiv} (\quotep{P},\quotep{Q}) & := & P \equiv Q \\
  match_{\dagger}(\quotep{P},\quotep{Q}) & := & \forall R. P|Q \red^{*} R => R \red^{*} 0 \\
  match_{K}(\quotep{P},\quotep{Q}) & := & K \mbox{ for some context } K
\end{eqnarray*}

$u?(x)P | u!\langle Q \rangle \red P\{\quotep{Q}/x\}$

%We write $\wred$ for $\red^*$, and $P\red$ if $\exists Q $ such that $ P \red Q$.
We write $P\red$ if $\exists Q $ such that $ P \red Q$ and $P\not\red$, otherwise.

\section{Replication}

As mentioned before, it is known that replication (and hence
recursion) can be implemented in a higher-order process algebra
\cite{SangiorgiWalker}. As our first example of calculation with the
machinery thus far presented we give the construction explicitly in
the {\rhoc}.

\begin{eqnarray}
	D_{x} & := & \prefix{x}{y}{(\binpar{\outputp{x}{y}}{@{y}})} \nonumber\\
	\bangp_{x}{P} & := & \binpar{{x}!\langle{\binpar{D_{x}}{P}}\rangle}{D_{x}} \nonumber
\end{eqnarray}

\begin{eqnarray}
	\bangp_{x}{P} & & \nonumber\\
	=
	& {x}!\langle{(\prefix{x}{y}{(\outputp{x}{y} | @{y})) | P}}\rangle 
	      | \prefix{x}{y}{(\outputp{x}{y} | @{y})} & \nonumber\\
	\red
	& (\outputp{x}{y} | @{y})\substn{\quotep{(\prefix{x}{y}{(@{y} | \outputp{x}{y})) | P}}}{y} & \nonumber\\
	=
	& \outputp{x}{\quotep{(\prefix{x}{y}{(\outputp{x}{y} | @{y})) | P}}}
	  | {(\prefix{x}{y}{(\outputp{x}{y} | @{y})) | P}} & \nonumber\\
	\red
	& \ldots & \nonumber\\
	\red^*
	& P | P | \ldots & \nonumber
\end{eqnarray}

Of course, this encoding, as an implementation, runs away, unfolding
$\bangp{P}$ eagerly. A lazier and more implementable replication
operator, restricted to input-guarded processes, may be obtained as follows.

\begin{eqnarray}
\bangp{\prefix{u}{v}{P}} 
	:= 
	\binpar{\lift{x}{\prefix{u}{v}{(\binpar{D(x)}{P})}}}{D(x)} \nonumber
\end{eqnarray}

\begin{remark}
  Note that the lazier definition still does not deal with summation
  or mixed summation (i.e. sums over input and output). The reader is
  invited to construct definitions of replication that deal with these
  features. 

  Further, the definitions are parameterized in a name, $x$. Can you,
  gentle reader, make a definition that eliminates this parameter and
  guarantees no accidental interaction between the replication
  machinery and the process being replicated -- i.e. no accidental
  sharing of names used by the process to get its work done and the
  name(s) used by the replication to effect copying. This latter
  revision of the definition of replication is crucial to obtaining
  the expected identity $!!P \sim !P$.
\end{remark}

\begin{remark}\label{rem:paradoxical_combinator}
  The reader familiar with the lambda calculus will have noticed the
  similarity between $D$ and the paradoxical combinator.

  [Ed. note: the existence of this seems to suggest we have to be more
  restrictive on the set of processes and names we admit if we are to
  support no-cloning.]
\end{remark}

\subsubsection{Bisimulation}

The computational dynamics gives rise to another kind of equivalence,
the equivalence of computational behavior. As previously mentioned
this is typically captured \emph{via} some form of bisimulation.

% The notion we use in this paper is weak barbed bisimulation
% \cite{milner91polyadicpi}.

The notion we use in this paper is derived from weak barbed
bisimulation \cite{milner91polyadicpi}. 

\begin{definition}
An \emph{observation relation}, $\downarrow_{\mathcal N}$, over a set
of names, $\mathcal N$, is the smallest relation satisfying the rules
below.

\infrule[Out-barb]{y \in {\mathcal N}, \; x \nameeq y}
		  {\outputp{x}{v} \downarrow_{\mathcal N} x}
\infrule[Par-barb]{\mbox{$P\downarrow_{\mathcal N} x$ or $Q\downarrow_{\mathcal N} x$}}
		  {\binpar{P}{Q} \downarrow_{\mathcal N} x}

We write $P \Downarrow_{\mathcal N} x$ if there is $Q$ such that 
$P \wred Q$ and $Q \downarrow_{\mathcal N} x$.
\end{definition}

\begin{definition}
%\label{def.bbisim}
An  ${\mathcal N}$-\emph{barbed bisimulation} over a set of names, ${\mathcal N}$, is a symmetric binary relation 
${\mathcal S}_{\mathcal N}$ between agents such that $P\rel{S}_{\mathcal N}Q$ implies:
\begin{enumerate}
\item If $P \red P'$ then $Q \wred Q'$ and $P'\rel{S}_{\mathcal N} Q'$.
\item If $P\downarrow_{\mathcal N} x$, then $Q\Downarrow_{\mathcal N} x$.
\end{enumerate}
$P$ is ${\mathcal N}$-barbed bisimilar to $Q$, written
$P \wbbisim_{\mathcal N} Q$, if $P \rel{S}_{\mathcal N} Q$ for some ${\mathcal N}$-barbed bisimulation ${\mathcal S}_{\mathcal N}$.
\end{definition}

$\mathcal{R} \subseteq \pi \times \pi$

$P \mathcal{R} Q => \forall P'. P \red P' \Rightarrow \exists Q'. Q \red Q', P' \mathcal{R} Q'$

$P \vdash x \Rightarrow Q \vdash x$

\begin{mathpar}
  \inferrule*[lab=Out-barb]{x \nameeq y}{{y}!\langle{Q}\rangle \vdash x}
  \and
  \inferrule*[lab=Par-barb]{\mbox{$P\vdash x$ or $Q\vdash x$}}{\binpar{P}{Q} \vdash x}
\end{mathpar}

\subsubsection{Contexts}

One of the principle advantages of computational calculi like the
$\pi$-calculus is a well-defined notion of context,
contextual-equivalence and a correlation between
contextual-equivalence and notions of bisimulation. The notion of
context allows the decomposition of a process into (sub-)process and
its syntactic environment, its context. Thus, a context may be
thought of as a process with a ``hole'' (written $\Box$) in it. The
application of a context $M$ to a process $P$, written $M[P]$, is
tantamount to filling the hole in $M$ with $P$. In this paper we do
not need the full weight of this theory, but do make use of the notion
of context in the proof the main theorem. 

\begin{mathpar}
  \inferrule* [lab=summation] {} {{M_{M},M_{N}} \bc \Box \;|\; x.M_{A} \;|\; M_{M}+M_{N}}
  \and
  \inferrule* [lab=agent] {} {{M_{A}} \bc (\vec{x})M_{P} \;| \; \clift{P_0,\ldots,M_{P},\ldots,P_N}}
  \and \\
  \inferrule* [lab=process] {} {{M_{P}} \bc M_{N} \;| \;P|M_{P} }
\end{mathpar} 

\begin{mathpar}
  \inferrule* [lab=sychronization] {} {M_{N} \bc \Box \;|\; x?M_{F} \;|\; x!M_{C}}
  \and
  \inferrule* [lab=abstraction] {} {{M_{F}} \bc (x)M_{P} }
  \and
  \inferrule* [lab=concretion] {} {{M_{C}} \bc \langle M_{P} \rangle }
  \and \\
  \inferrule* [lab=process] {} {{M_{P}} \bc M_{N} \;| \;P|M_{P} }
\end{mathpar}

\begin{definition}[contextual application] Given a context $M$, and
  process $P$, we define the \emph{contextual application}, $M[P] :=
  M\{P/\Box\}$. That is, the contextual application of M to P is the
  substitution of $P$ for $\Box$ in $M$.
\end{definition}

$\meaningof{-} : L \to \mathcal{P}(\pi)$

\begin{mathpar}
  \inferrule* [lab=collection] {} {\meaningof{true} = \pi, \and \meaningof{~E} = \pi \setminus \meaningof{E}, \and \meaningof{E_{1} \& E_{2}} = \meaningof{E_{1}} \cap \meaningof{E_{2}}}
\end{mathpar}

\begin{mathpar}
  \inferrule* [lab=structure] {} {\meaningof{0} = \{ P \in \pi | P \equiv 0 \}, \and \\ \meaningof{E_1 | E_2} = \{ P \in \pi | P \equiv P_{1} | P_{2}, P_{1} \in \meaningof{E_{1}}, P_{2} \in \meaningof{E_2}\} }
\end{mathpar}

\begin{mathpar}
 \inferrule* [lab=behavior] {} {\meaningof{\langle a?b \rangle E} = \{ P \in \pi | P \equiv Q | u?(y)P', \\ \and \\\\ \and \\ \;\;\; u \in \meaningof{a}, \forall z.P'\{z/y\} \in \meaningof{E\{z/b\}}\}, \and \\ \meaningof{a!E} = \{ P \in \pi | P \equiv Q | x!\langle P' \rangle, x \in \meaningof{a} P' \in \meaningof{E}\} }
\end{mathpar}

\begin{mathpar}
 \inferrule* [lab=nominal] {} {\meaningof{\quotep{E}} = \{ \quotep{P} \in \quotep{\pi} | P \in \meaningof{E} \}, \and \meaningof{\quotep{P}} = \{ \quotep{Q} \in \quotep{\pi} | P \equiv Q \} \and \\ \meaningof{@\quotep{E}} = \{ P \in \pi | P \equiv @x, x \in \meaningof{E} \}}
\end{mathpar}

\begin{eqnarray*}
  \\
  \meaningof{-} : TS \to ST
\end{eqnarray*}

\begin{eqnarray*}
  \\
  L : TS \to ST
\end{eqnarray*}

\begin{eqnarray*}
  \\
  P \models E \iff P \in \meaningof{E}
\end{eqnarray*}

\begin{eqnarray*}
  P \approx_{L} Q \iff \forall E \in L. P \models E \iff Q \models E
\end{eqnarray*}

\begin{eqnarray*}
  P \approx_{K} Q
\end{eqnarray*}

\begin{eqnarray*}
  P \approx Q
\end{eqnarray*}

$\approx_{K} = \approx = \approx_{L}$

\subsubsection{Contextual duality}

Note that contexts extend the quotation operation to a family of
operations from processes to names. Given a context, $M$, we can
define a \emph{nominal context}, $\quotep{M}$ by $\quotep{M}[P] :=
\quotep{M[P]}$. To foreshadow what is to come we observe that these
operations enjoy a duality with processes very much like the duality
between vectors and maps from vectors to scalars.

Further, because the calculus is essentially higher-order, we have a
correspondence between contexts and processes. More specifically,
given a name $x$ and a context $M$ we can construct $M^{*}_{x}$ such
that 

\begin{mathpar}
  M^{*}_{x} | \lift{x}{P} \red M[P]
\end{mathpar}

namely,

\begin{mathpar}
  M^{*}_{x} := x?(u).M[\dropn{u}]
\end{mathpar}

The dependence of $M^{*}_{x}$ on a name makes it an abstraction, 

\begin{mathpar}
  M^{*} := (x)x?(u).M[\dropn{u}]
\end{mathpar}

\subsection{Additional notation}

It will sometimes be convenient to denote the process a name
quotes. We already have the notation $x = \quotep{P}$, but it will be
convenient to introduce an alternate notation, $\procn{x}$, when we
want to emphasize the connection to the use of the name. Note that, by
virtue of name equivalence, $\quotep{\procn{x}} \nameeq x$; so, the
notation is consistent with previous definitions.

Further, because names have structure it is possible to effect
substitutions on the basis of that structure. This means we need to
upgrade our notation for substitutions, which we accomplish by
adapting comprehension notation. Thus,

\begin{mathpar}
  P\{ y / x : x \in S \}
\end{mathpar}

is interpreted to mean the process derived from P by replacing (in a
capture-avoiding manner) each occurrence of $x$ in $S$ by $y$. For example,

\begin{mathpar}
  P\{ \quotep{\procn{x}|\procn{x}} / x : x \in \freenames{P} \}
\end{mathpar}

will replace each (occurrence) of a free name $x$ in $P$ by
$\quotep{\procn{x}|\procn{x}}$.

Also, we will avail ourselves of the notation $x^{L}$ and $x^{R}$ to
denote injections of a name into disjoint copies of the name
space. There are numerous ways to accomplish this. One example can be
found in \cite{MeredithR05}. This notation overloads to vectors of
names: $\vec{x}^{\pi} := (x_{i}^{\pi} \; : \; 0 \leq i < |\vec{x}| )$ where $\pi \in \{L,R\}$.

We also use $P^{\Box} := P|\Box$.

In \cite{MeredithR05} an interpretation of the new operator is
given. It turns out that there are several possible interpretations
all enjoying the requisite algebraic properties of the operator (see
\cite{milner91polyadicpi}). We will therefore make liberal use of
$(\nu\; \vec{x})P$.

% subsection the_syntax_and_semantics_of_the_notation_system (end)   

\input{qm2pi.qmops} 

\input{qm2pi.sterngerlach} 

\input{qm2pi.metric} 

% section concurrent_process_calculi (end)

%\input{qm2pi.proofsketch}

% section proof sketch (end)

%\input{qm2pi.slviaknots} 

% section spatial logic via knots (end)

\input{qm2pi.conclusion}

% section conclusion (end)

%\input{qm2pi.dtcodes} 

% section wiring algorithm (end)

\input{qm2pi.ack} 

% section acknowledgments (end)

\newpage


\bibliographystyle{plain}   
\bibliography{../../biblios/main.bib}

\input{qm2pi.rhodetails}

\end{document}

 

% section acknowledgments (end)

\newpage


\bibliographystyle{plain}   
\bibliography{../../biblios/main.bib}

\documentclass[12pt]{llncs}
%\documentclass{jktr}

\usepackage[pdftex]{hyperref}                   
\usepackage {listings}
\usepackage {mathpartir}
\usepackage{bcprules}
%\usepackage{listings}
                       
\usepackage{graphicx} 
%\usepackage[margins=2.5cm,nohead,nofoot]{geometry}
%\usepackage{geometry}
\usepackage{amsfonts}
\usepackage{amstext}
\usepackage{latexsym}
\usepackage{amssymb}
\usepackage{color}


%\include{myPreamble}
\include{qm2pi.local} 

%\ifpdf
%\usepackage[pdftex]{graphicx}
%\else
%\usepackage{graphicx}
%\fi

 % \ifpdf
%  \usepackage{pdfsync}
%  \if


%\title{Brief Article}
%\author{David F. Snyder}
%\author{L.G. Meredith}

%\address{Dept. of Math., Texas State University--San Marcos, San Marcos, TX 78666}
       
\pagestyle{empty}


\begin{document}

\lstset{language=[Objective]Caml,frame=shadowbox}

\input{qm2pi.front}

% section front matter (end)

\input{qm2pi.intro} 
 
% section introduction (end)

% \input{qm2pi.knotations} 

% section notation (end)

\input{qm2pi.process.calculi} 

% section concurrent_process_calculi_and_spatial_logics_ (end)
    
%\input{qm2pi.knots2pi} 

%\input{qm2pi.trefoil} 

%\input{qm2pi.mainthm} 

% subsection basic_interpretation (end)

%\input{qm2pi.rho.presentation} 
\subsection{The syntax and semantics of the notation system}\label{sub:the_syntax_and_semantics_of_the_notation_system} % (fold)

We now summarize a technical presentation of the calculus that
embodies our theory of dynamics. The typical presentation of such a
calculus follows the style of giving generators and relations on
them. The grammar, below, describing term constructors, freely
generates the set of processes, $\Proc$. This set is then quotiented
by a relation known as structural congruence and it is over this set
that the notion of dynamics is expressed. This presentation is
essentially that of \cite{MeredithR05} with the addition of
polyadicity and summation. For readability we have relegated some of
the technical subtleties to an appendix.

\subsubsection{Process grammar}\label{subsub:process_grammar}

\begin{mathpar}
  \inferrule* [lab=synchronization] {} {{M} \bc \pzero \;|\; x?F \;|\; x!C }
  \and
  \inferrule* [lab=abstraction] {} {{F} \bc (x)P}
  \and
  \inferrule* [lab=concretion] {} {{C} \bc \langle Q \rangle}
  \and
  \inferrule* [lab=process] {} {{P,Q} \bc M \;| \;P|Q \;|\; @{x}}
  \and
  \inferrule* [lab=name] {} {{x} \bc \quotep{P}}
\end{mathpar} 

Note that $\vec{x}$ (resp. $\vec{P}$) denotes a vector of names
(resp. processes) of length $|\vec{x}|$ (resp. $|\vec{P}|$). We adopt
the following useful abbreviations.

\begin{mathpar}
   x?(\vec{y}).P := x.(\vec{y})P \and  x\clift{\vec{P}} := x.\clift{\vec{P}}
   \and x!(y) := \lift{x}{\dropn{y}}
   \and \Pi_{i=0}^{n-1}P_i := P_0 | \ldots | P_{n-1}
\end{mathpar}

\subsubsection{Structural congruence}

\paragraph{Free and bound names and alpha-equivalence.} At the
core of structural equivalence is alpha-equivalence which identifies
process that are the same up to a change of variable. Formally, we
recognize the distinction between free and bound names. The free names
of a process, $\freenames{P}$, may be calculated recursively as
follows:

\begin{mathpar}
\freenames{\pzero} := \emptyset
  \and \\
  \freenames{x?(y).P} := \{ x \} \cup (\freenames{P} \setminus \{ y \})
  \and 
  \freenames{x!\langle P \rangle} := \{ x \} \cup \{ P \} 
  \and \\
  \freenames{P|Q} := \freenames{P} \cup \freenames{Q}
  \and \\
  \freenames{@{x}} := \{ x \}
\end{mathpar}

$\pi$
$\quotep{\pi}$

$\freenames{-} : \pi \to \mathcal{P}(\quotep{\pi})$

\begin{eqnarray*}
  \freenames{\pzero} & := & \emptyset \\
  \freenames{x?(y).P} & := & \{ x \} \cup (\freenames{P} \setminus \{ y \}) \\
  \freenames{x!\langle P \rangle} & := & \{ x \} \cup \{ P \} \\
  \freenames{P|Q} & := & \freenames{P} \cup \freenames{Q} \\
  \freenames{\dropn{x}} & := & \{ x \}
\end{eqnarray*}

The bound names of a process, $\boundnames{P}$, are those names occurring in $P$
that are not free. For example, in $x?(y).0$, the name $x$ is free, while $y$ is bound.

\begin{mathpar}
  \inferrule* [lab=monoidal-laws] {} { P|Q \equiv Q|P \and P|0 \equiv P \and P|(Q|R) \equiv (P|Q)|R }
\end{mathpar}

\begin{mathpar}
  \inferrule* [lab=alpha-equivalence] {} { (x)P \equiv (y)P\{y/x\} \and y \not\in \freenames{P} }
\end{mathpar}

\begin{definition}
Then two processes, $P,Q$, are alpha-equivalent if $P = Q\{\vec{y}/\vec{x}\}$ for
some $\vec{x} \in \boundnames{Q},\vec{y} \in \boundnames{P}$, where $Q\{\vec{y}/\vec{x}\}$
denotes the capture-avoiding substitution of $\vec{y}$ for $\vec{x}$ in $Q$.
\end{definition}

\begin{definition}
  The {\em structural congruence} \cite{SangiorgiWalker} , $\equiv$,
  between processes is the least congruence containing
  alpha-equivalence, satisfying the abelian monoid laws
  (associativity, commutativity and $\pzero$ as identity) for parallel
  composition $|$ and for summation $+$.
\end{definition}

\subsection{Name equivalence}

We take name equivalence, written $\nameeq$, to be the smallest
equivalence relation generated by the following rules.

\begin{mathpar}
\inferrule*[lab=Quote-drop]
{ }
{ \quotep{@{x}} \nameeq x }

\inferrule*[lab=Struct-equiv]
{ P \scong Q }
{ \quotep{P} \nameeq \quotep{Q} }
\end{mathpar}

The astute reader will have noticed that the mutual recursion of names
and processes imposes a mutual recursion on alpha-equivalence and
structural equivalence via name-equivalence. Fortunately, all of this
works out pleasantly and we may calculate in the natural way, free of
concern. The reader interested in the details is referred to the
appendix \ref{appendix:rho_details}.

\subsection{Substitution}

We use $\Proc$ for the set of processes, $\QProc$ for the set of
names, and $\id{\{}\vec{y} / \vec{x} \id{\}}$ to denote partial maps,
$s : \QProc \rightarrow \QProc$. A map, $s$ lifts, uniquely, to a map
on process terms, $\widehat{s} : \Proc \rightarrow \Proc$ by the
following equations.

\begin{mathpar}
  (0) \psubstp{Q}{P} := 0 \\
  (R \juxtap S) \psubstp{Q}{P}
  :=    
  (R)\psubstp{Q}{P} \juxtap (S) \psubstp{Q}{P} \\
  (x?(y).R) \psubstp{Q}{P}    
  :=    
  (x)\substp{Q}{P} (z)\concat( (R \psubstn{z}{y}) \psubstp{Q}{P} ) \\
  (\lift{x}{R}) \psubstp{Q}{P}  
  :=
  \lift{(x)\substp{Q}{P}}{ R \psubstp{Q}{P} } \\
%   (\dropn{x})  \psubstp{Q}{P}       
%   := 
%   \left\{ 
%     \begin{array}{ccc} 
%       \dropn{\quotep{Q}} & & x \nameeq \quotep{P} \\
%       \dropn{x} & & otherwise \\
%     \end{array}
%   \right. 
  (\dropn{x})  \psubstp{Q}{P}       
  := 
  \left\{ 
    \begin{array}{ccc} 
      Q & & x \nameeq \quotep{P} \\
      \dropn{x} & & otherwise \\
    \end{array}
  \right.
\end{mathpar}
 

where

\begin{eqnarray}
  (x)\id{\{} \lpquote Q \rpquote / \lpquote P \rpquote \id{\}}            = 
  \left\{ 
    \begin{array}{ccc}
      \lpquote Q \rpquote & & x \nameeq \lpquote P \rpquote \\
      x & & otherwise \\
    \end{array}
  \right. \nonumber
\end{eqnarray}

and $z$ is chosen distinct from $\quotep{P}$, $\quotep{Q}$, the free
names in $Q$, and all the names in $R$. Our $\alpha$-equivalence will
be built in the standard way from this substitution.

\begin{remark}\label{rem:no_self_referential_names}
  One consequence of these definitions is that $\forall P. \quotep{P}
  \not\in \freenames{P}$.
\end{remark}

\subsection{ Dynamic quote: an example }

Anticipating something of what's to come, consider applying the
substitution, $\widehat{\id{\{}u / z \id{\}}}$, to the following pair
of processes, $\lift{w}{y!(z)}$ and $w[ \lpquote y!(z) \rpquote ]$.

\begin{eqnarray}
	\lift{w}{y!(z)}\widehat{\id{\{}u / z \id{\}}}
		& = &
		\lift{w}{y!(u)} \nonumber\\
	w[ \lpquote y!(z) \rpquote ] \widehat{ \id{\{}u / z \id{\}} }
		& = &
		w[ \lpquote y!(z) \rpquote ] \nonumber
\end{eqnarray}

Because the body of the process between quotes is impervious to
substitution, we get radically different answers. In fact, by
examining the first process in an input context,
e.g. $x?(z).\lift{w}{y!(z)}$, we see that the process under the lift
operator may be shaped by prefixed inputs binding a name inside it. In
this sense, the lift operator will be seen as a way to dynamically
construct processes before reifying them as names.

Finally equipped with these standard features we can present the
dynamics of the calculus.

\subsubsection{Operational semantics} 

Finally, we introduce the computational dynamics. What marks these
algebras as distinct from other more traditionally studied algebraic
structures, e.g. vector spaces or polynomial rings, is the manner in
which dynamics is captured. In traditional structures, dynamics is typically
expressed through morphisms between such structures, as in linear maps
between vector spaces or morphisms between rings. In algebras
associated with the semantics of computation, the dynamics is
expressed as part of the algebraic structure itself, through a
reduction reduction relation typically denoted by $\red$. Below, we
give a recursive presentation of this relation for the calculus used
in the encoding.

$\red \subseteq \pi \times \pi$
$\red : \pi \to \mathcal{P}(\pi)$

\begin{mathpar}
  \inferrule* [lab=Comm] { \textsf{match}( x_{src}, x_{trgt} ) } { x_{trgt}?(y)P \; | \; x_{src}!\langle {Q} \rangle \red P\{\quotep{Q}/y}\} }
  \and \\
  \inferrule* [lab=Par] {{P} \red {P}'} {{{P} | {Q}} \red {{P}' | {Q}}}
  \and
  \inferrule* [lab=Equiv]{{{P} \scong {P}'} \andalso {{P}' \red {Q}'} \andalso {{Q}' \scong {Q}}}{{P} \red {Q}}
\end{mathpar}

\begin{eqnarray*}
  match_{\equiv} (\quotep{P},\quotep{Q}) & := & P \equiv Q \\
  match_{\dagger}(\quotep{P},\quotep{Q}) & := & \forall R. P|Q \red^{*} R => R \red^{*} 0 \\
  match_{K}(\quotep{P},\quotep{Q}) & := & K \mbox{ for some context } K
\end{eqnarray*}

$u?(x)P | u!\langle Q \rangle \red P\{\quotep{Q}/x\}$

%We write $\wred$ for $\red^*$, and $P\red$ if $\exists Q $ such that $ P \red Q$.
We write $P\red$ if $\exists Q $ such that $ P \red Q$ and $P\not\red$, otherwise.

\section{Replication}

As mentioned before, it is known that replication (and hence
recursion) can be implemented in a higher-order process algebra
\cite{SangiorgiWalker}. As our first example of calculation with the
machinery thus far presented we give the construction explicitly in
the {\rhoc}.

\begin{eqnarray}
	D_{x} & := & \prefix{x}{y}{(\binpar{\outputp{x}{y}}{@{y}})} \nonumber\\
	\bangp_{x}{P} & := & \binpar{{x}!\langle{\binpar{D_{x}}{P}}\rangle}{D_{x}} \nonumber
\end{eqnarray}

\begin{eqnarray}
	\bangp_{x}{P} & & \nonumber\\
	=
	& {x}!\langle{(\prefix{x}{y}{(\outputp{x}{y} | @{y})) | P}}\rangle 
	      | \prefix{x}{y}{(\outputp{x}{y} | @{y})} & \nonumber\\
	\red
	& (\outputp{x}{y} | @{y})\substn{\quotep{(\prefix{x}{y}{(@{y} | \outputp{x}{y})) | P}}}{y} & \nonumber\\
	=
	& \outputp{x}{\quotep{(\prefix{x}{y}{(\outputp{x}{y} | @{y})) | P}}}
	  | {(\prefix{x}{y}{(\outputp{x}{y} | @{y})) | P}} & \nonumber\\
	\red
	& \ldots & \nonumber\\
	\red^*
	& P | P | \ldots & \nonumber
\end{eqnarray}

Of course, this encoding, as an implementation, runs away, unfolding
$\bangp{P}$ eagerly. A lazier and more implementable replication
operator, restricted to input-guarded processes, may be obtained as follows.

\begin{eqnarray}
\bangp{\prefix{u}{v}{P}} 
	:= 
	\binpar{\lift{x}{\prefix{u}{v}{(\binpar{D(x)}{P})}}}{D(x)} \nonumber
\end{eqnarray}

\begin{remark}
  Note that the lazier definition still does not deal with summation
  or mixed summation (i.e. sums over input and output). The reader is
  invited to construct definitions of replication that deal with these
  features. 

  Further, the definitions are parameterized in a name, $x$. Can you,
  gentle reader, make a definition that eliminates this parameter and
  guarantees no accidental interaction between the replication
  machinery and the process being replicated -- i.e. no accidental
  sharing of names used by the process to get its work done and the
  name(s) used by the replication to effect copying. This latter
  revision of the definition of replication is crucial to obtaining
  the expected identity $!!P \sim !P$.
\end{remark}

\begin{remark}\label{rem:paradoxical_combinator}
  The reader familiar with the lambda calculus will have noticed the
  similarity between $D$ and the paradoxical combinator.

  [Ed. note: the existence of this seems to suggest we have to be more
  restrictive on the set of processes and names we admit if we are to
  support no-cloning.]
\end{remark}

\subsubsection{Bisimulation}

The computational dynamics gives rise to another kind of equivalence,
the equivalence of computational behavior. As previously mentioned
this is typically captured \emph{via} some form of bisimulation.

% The notion we use in this paper is weak barbed bisimulation
% \cite{milner91polyadicpi}.

The notion we use in this paper is derived from weak barbed
bisimulation \cite{milner91polyadicpi}. 

\begin{definition}
An \emph{observation relation}, $\downarrow_{\mathcal N}$, over a set
of names, $\mathcal N$, is the smallest relation satisfying the rules
below.

\infrule[Out-barb]{y \in {\mathcal N}, \; x \nameeq y}
		  {\outputp{x}{v} \downarrow_{\mathcal N} x}
\infrule[Par-barb]{\mbox{$P\downarrow_{\mathcal N} x$ or $Q\downarrow_{\mathcal N} x$}}
		  {\binpar{P}{Q} \downarrow_{\mathcal N} x}

We write $P \Downarrow_{\mathcal N} x$ if there is $Q$ such that 
$P \wred Q$ and $Q \downarrow_{\mathcal N} x$.
\end{definition}

\begin{definition}
%\label{def.bbisim}
An  ${\mathcal N}$-\emph{barbed bisimulation} over a set of names, ${\mathcal N}$, is a symmetric binary relation 
${\mathcal S}_{\mathcal N}$ between agents such that $P\rel{S}_{\mathcal N}Q$ implies:
\begin{enumerate}
\item If $P \red P'$ then $Q \wred Q'$ and $P'\rel{S}_{\mathcal N} Q'$.
\item If $P\downarrow_{\mathcal N} x$, then $Q\Downarrow_{\mathcal N} x$.
\end{enumerate}
$P$ is ${\mathcal N}$-barbed bisimilar to $Q$, written
$P \wbbisim_{\mathcal N} Q$, if $P \rel{S}_{\mathcal N} Q$ for some ${\mathcal N}$-barbed bisimulation ${\mathcal S}_{\mathcal N}$.
\end{definition}

$\mathcal{R} \subseteq \pi \times \pi$

$P \mathcal{R} Q => \forall P'. P \red P' \Rightarrow \exists Q'. Q \red Q', P' \mathcal{R} Q'$

$P \vdash x \Rightarrow Q \vdash x$

\begin{mathpar}
  \inferrule*[lab=Out-barb]{x \nameeq y}{{y}!\langle{Q}\rangle \vdash x}
  \and
  \inferrule*[lab=Par-barb]{\mbox{$P\vdash x$ or $Q\vdash x$}}{\binpar{P}{Q} \vdash x}
\end{mathpar}

\subsubsection{Contexts}

One of the principle advantages of computational calculi like the
$\pi$-calculus is a well-defined notion of context,
contextual-equivalence and a correlation between
contextual-equivalence and notions of bisimulation. The notion of
context allows the decomposition of a process into (sub-)process and
its syntactic environment, its context. Thus, a context may be
thought of as a process with a ``hole'' (written $\Box$) in it. The
application of a context $M$ to a process $P$, written $M[P]$, is
tantamount to filling the hole in $M$ with $P$. In this paper we do
not need the full weight of this theory, but do make use of the notion
of context in the proof the main theorem. 

\begin{mathpar}
  \inferrule* [lab=summation] {} {{M_{M},M_{N}} \bc \Box \;|\; x.M_{A} \;|\; M_{M}+M_{N}}
  \and
  \inferrule* [lab=agent] {} {{M_{A}} \bc (\vec{x})M_{P} \;| \; \clift{P_0,\ldots,M_{P},\ldots,P_N}}
  \and \\
  \inferrule* [lab=process] {} {{M_{P}} \bc M_{N} \;| \;P|M_{P} }
\end{mathpar} 

\begin{mathpar}
  \inferrule* [lab=sychronization] {} {M_{N} \bc \Box \;|\; x?M_{F} \;|\; x!M_{C}}
  \and
  \inferrule* [lab=abstraction] {} {{M_{F}} \bc (x)M_{P} }
  \and
  \inferrule* [lab=concretion] {} {{M_{C}} \bc \langle M_{P} \rangle }
  \and \\
  \inferrule* [lab=process] {} {{M_{P}} \bc M_{N} \;| \;P|M_{P} }
\end{mathpar}

\begin{definition}[contextual application] Given a context $M$, and
  process $P$, we define the \emph{contextual application}, $M[P] :=
  M\{P/\Box\}$. That is, the contextual application of M to P is the
  substitution of $P$ for $\Box$ in $M$.
\end{definition}

$\meaningof{-} : L \to \mathcal{P}(\pi)$

\begin{mathpar}
  \inferrule* [lab=collection] {} {\meaningof{true} = \pi, \and \meaningof{~E} = \pi \setminus \meaningof{E}, \and \meaningof{E_{1} \& E_{2}} = \meaningof{E_{1}} \cap \meaningof{E_{2}}}
\end{mathpar}

\begin{mathpar}
  \inferrule* [lab=structure] {} {\meaningof{0} = \{ P \in \pi | P \equiv 0 \}, \and \\ \meaningof{E_1 | E_2} = \{ P \in \pi | P \equiv P_{1} | P_{2}, P_{1} \in \meaningof{E_{1}}, P_{2} \in \meaningof{E_2}\} }
\end{mathpar}

\begin{mathpar}
 \inferrule* [lab=behavior] {} {\meaningof{\langle a?b \rangle E} = \{ P \in \pi | P \equiv Q | u?(y)P', \\ \and \\\\ \and \\ \;\;\; u \in \meaningof{a}, \forall z.P'\{z/y\} \in \meaningof{E\{z/b\}}\}, \and \\ \meaningof{a!E} = \{ P \in \pi | P \equiv Q | x!\langle P' \rangle, x \in \meaningof{a} P' \in \meaningof{E}\} }
\end{mathpar}

\begin{mathpar}
 \inferrule* [lab=nominal] {} {\meaningof{\quotep{E}} = \{ \quotep{P} \in \quotep{\pi} | P \in \meaningof{E} \}, \and \meaningof{\quotep{P}} = \{ \quotep{Q} \in \quotep{\pi} | P \equiv Q \} \and \\ \meaningof{@\quotep{E}} = \{ P \in \pi | P \equiv @x, x \in \meaningof{E} \}}
\end{mathpar}

\begin{eqnarray*}
  \\
  \meaningof{-} : TS \to ST
\end{eqnarray*}

\begin{eqnarray*}
  \\
  L : TS \to ST
\end{eqnarray*}

\begin{eqnarray*}
  \\
  P \models E \iff P \in \meaningof{E}
\end{eqnarray*}

\begin{eqnarray*}
  P \approx_{L} Q \iff \forall E \in L. P \models E \iff Q \models E
\end{eqnarray*}

\begin{eqnarray*}
  P \approx_{K} Q
\end{eqnarray*}

\begin{eqnarray*}
  P \approx Q
\end{eqnarray*}

$\approx_{K} = \approx = \approx_{L}$

\subsubsection{Contextual duality}

Note that contexts extend the quotation operation to a family of
operations from processes to names. Given a context, $M$, we can
define a \emph{nominal context}, $\quotep{M}$ by $\quotep{M}[P] :=
\quotep{M[P]}$. To foreshadow what is to come we observe that these
operations enjoy a duality with processes very much like the duality
between vectors and maps from vectors to scalars.

Further, because the calculus is essentially higher-order, we have a
correspondence between contexts and processes. More specifically,
given a name $x$ and a context $M$ we can construct $M^{*}_{x}$ such
that 

\begin{mathpar}
  M^{*}_{x} | \lift{x}{P} \red M[P]
\end{mathpar}

namely,

\begin{mathpar}
  M^{*}_{x} := x?(u).M[\dropn{u}]
\end{mathpar}

The dependence of $M^{*}_{x}$ on a name makes it an abstraction, 

\begin{mathpar}
  M^{*} := (x)x?(u).M[\dropn{u}]
\end{mathpar}

\subsection{Additional notation}

It will sometimes be convenient to denote the process a name
quotes. We already have the notation $x = \quotep{P}$, but it will be
convenient to introduce an alternate notation, $\procn{x}$, when we
want to emphasize the connection to the use of the name. Note that, by
virtue of name equivalence, $\quotep{\procn{x}} \nameeq x$; so, the
notation is consistent with previous definitions.

Further, because names have structure it is possible to effect
substitutions on the basis of that structure. This means we need to
upgrade our notation for substitutions, which we accomplish by
adapting comprehension notation. Thus,

\begin{mathpar}
  P\{ y / x : x \in S \}
\end{mathpar}

is interpreted to mean the process derived from P by replacing (in a
capture-avoiding manner) each occurrence of $x$ in $S$ by $y$. For example,

\begin{mathpar}
  P\{ \quotep{\procn{x}|\procn{x}} / x : x \in \freenames{P} \}
\end{mathpar}

will replace each (occurrence) of a free name $x$ in $P$ by
$\quotep{\procn{x}|\procn{x}}$.

Also, we will avail ourselves of the notation $x^{L}$ and $x^{R}$ to
denote injections of a name into disjoint copies of the name
space. There are numerous ways to accomplish this. One example can be
found in \cite{MeredithR05}. This notation overloads to vectors of
names: $\vec{x}^{\pi} := (x_{i}^{\pi} \; : \; 0 \leq i < |\vec{x}| )$ where $\pi \in \{L,R\}$.

We also use $P^{\Box} := P|\Box$.

In \cite{MeredithR05} an interpretation of the new operator is
given. It turns out that there are several possible interpretations
all enjoying the requisite algebraic properties of the operator (see
\cite{milner91polyadicpi}). We will therefore make liberal use of
$(\nu\; \vec{x})P$.

% subsection the_syntax_and_semantics_of_the_notation_system (end)   

\input{qm2pi.qmops} 

\input{qm2pi.sterngerlach} 

\input{qm2pi.metric} 

% section concurrent_process_calculi (end)

%\input{qm2pi.proofsketch}

% section proof sketch (end)

%\input{qm2pi.slviaknots} 

% section spatial logic via knots (end)

\input{qm2pi.conclusion}

% section conclusion (end)

%\input{qm2pi.dtcodes} 

% section wiring algorithm (end)

\input{qm2pi.ack} 

% section acknowledgments (end)

\newpage


\bibliographystyle{plain}   
\bibliography{../../biblios/main.bib}

\input{qm2pi.rhodetails}

\end{document}



\end{document}



\end{document}

 

\documentclass[12pt]{llncs}
%\documentclass{jktr}

\usepackage[pdftex]{hyperref}                   
\usepackage {listings}
\usepackage {mathpartir}
\usepackage{bcprules}
%\usepackage{listings}
                       
\usepackage{graphicx} 
%\usepackage[margins=2.5cm,nohead,nofoot]{geometry}
%\usepackage{geometry}
\usepackage{amsfonts}
\usepackage{amstext}
\usepackage{latexsym}
\usepackage{amssymb}
\usepackage{color}


%\include{myPreamble}
\documentclass[12pt]{llncs}
%\documentclass{jktr}

\usepackage[pdftex]{hyperref}                   
\usepackage {listings}
\usepackage {mathpartir}
\usepackage{bcprules}
%\usepackage{listings}
                       
\usepackage{graphicx} 
%\usepackage[margins=2.5cm,nohead,nofoot]{geometry}
%\usepackage{geometry}
\usepackage{amsfonts}
\usepackage{amstext}
\usepackage{latexsym}
\usepackage{amssymb}
\usepackage{color}


%\include{myPreamble}
\documentclass[12pt]{llncs}
%\documentclass{jktr}

\usepackage[pdftex]{hyperref}                   
\usepackage {listings}
\usepackage {mathpartir}
\usepackage{bcprules}
%\usepackage{listings}
                       
\usepackage{graphicx} 
%\usepackage[margins=2.5cm,nohead,nofoot]{geometry}
%\usepackage{geometry}
\usepackage{amsfonts}
\usepackage{amstext}
\usepackage{latexsym}
\usepackage{amssymb}
\usepackage{color}


%\include{myPreamble}
\include{qm2pi.local} 

%\ifpdf
%\usepackage[pdftex]{graphicx}
%\else
%\usepackage{graphicx}
%\fi

 % \ifpdf
%  \usepackage{pdfsync}
%  \if


%\title{Brief Article}
%\author{David F. Snyder}
%\author{L.G. Meredith}

%\address{Dept. of Math., Texas State University--San Marcos, San Marcos, TX 78666}
       
\pagestyle{empty}


\begin{document}

\lstset{language=[Objective]Caml,frame=shadowbox}

\input{qm2pi.front}

% section front matter (end)

\input{qm2pi.intro} 
 
% section introduction (end)

% \input{qm2pi.knotations} 

% section notation (end)

\input{qm2pi.process.calculi} 

% section concurrent_process_calculi_and_spatial_logics_ (end)
    
%\input{qm2pi.knots2pi} 

%\input{qm2pi.trefoil} 

%\input{qm2pi.mainthm} 

% subsection basic_interpretation (end)

%\input{qm2pi.rho.presentation} 
\subsection{The syntax and semantics of the notation system}\label{sub:the_syntax_and_semantics_of_the_notation_system} % (fold)

We now summarize a technical presentation of the calculus that
embodies our theory of dynamics. The typical presentation of such a
calculus follows the style of giving generators and relations on
them. The grammar, below, describing term constructors, freely
generates the set of processes, $\Proc$. This set is then quotiented
by a relation known as structural congruence and it is over this set
that the notion of dynamics is expressed. This presentation is
essentially that of \cite{MeredithR05} with the addition of
polyadicity and summation. For readability we have relegated some of
the technical subtleties to an appendix.

\subsubsection{Process grammar}\label{subsub:process_grammar}

\begin{mathpar}
  \inferrule* [lab=synchronization] {} {{M} \bc \pzero \;|\; x?F \;|\; x!C }
  \and
  \inferrule* [lab=abstraction] {} {{F} \bc (x)P}
  \and
  \inferrule* [lab=concretion] {} {{C} \bc \langle Q \rangle}
  \and
  \inferrule* [lab=process] {} {{P,Q} \bc M \;| \;P|Q \;|\; @{x}}
  \and
  \inferrule* [lab=name] {} {{x} \bc \quotep{P}}
\end{mathpar} 

Note that $\vec{x}$ (resp. $\vec{P}$) denotes a vector of names
(resp. processes) of length $|\vec{x}|$ (resp. $|\vec{P}|$). We adopt
the following useful abbreviations.

\begin{mathpar}
   x?(\vec{y}).P := x.(\vec{y})P \and  x\clift{\vec{P}} := x.\clift{\vec{P}}
   \and x!(y) := \lift{x}{\dropn{y}}
   \and \Pi_{i=0}^{n-1}P_i := P_0 | \ldots | P_{n-1}
\end{mathpar}

\subsubsection{Structural congruence}

\paragraph{Free and bound names and alpha-equivalence.} At the
core of structural equivalence is alpha-equivalence which identifies
process that are the same up to a change of variable. Formally, we
recognize the distinction between free and bound names. The free names
of a process, $\freenames{P}$, may be calculated recursively as
follows:

\begin{mathpar}
\freenames{\pzero} := \emptyset
  \and \\
  \freenames{x?(y).P} := \{ x \} \cup (\freenames{P} \setminus \{ y \})
  \and 
  \freenames{x!\langle P \rangle} := \{ x \} \cup \{ P \} 
  \and \\
  \freenames{P|Q} := \freenames{P} \cup \freenames{Q}
  \and \\
  \freenames{@{x}} := \{ x \}
\end{mathpar}

$\pi$
$\quotep{\pi}$

$\freenames{-} : \pi \to \mathcal{P}(\quotep{\pi})$

\begin{eqnarray*}
  \freenames{\pzero} & := & \emptyset \\
  \freenames{x?(y).P} & := & \{ x \} \cup (\freenames{P} \setminus \{ y \}) \\
  \freenames{x!\langle P \rangle} & := & \{ x \} \cup \{ P \} \\
  \freenames{P|Q} & := & \freenames{P} \cup \freenames{Q} \\
  \freenames{\dropn{x}} & := & \{ x \}
\end{eqnarray*}

The bound names of a process, $\boundnames{P}$, are those names occurring in $P$
that are not free. For example, in $x?(y).0$, the name $x$ is free, while $y$ is bound.

\begin{mathpar}
  \inferrule* [lab=monoidal-laws] {} { P|Q \equiv Q|P \and P|0 \equiv P \and P|(Q|R) \equiv (P|Q)|R }
\end{mathpar}

\begin{mathpar}
  \inferrule* [lab=alpha-equivalence] {} { (x)P \equiv (y)P\{y/x\} \and y \not\in \freenames{P} }
\end{mathpar}

\begin{definition}
Then two processes, $P,Q$, are alpha-equivalent if $P = Q\{\vec{y}/\vec{x}\}$ for
some $\vec{x} \in \boundnames{Q},\vec{y} \in \boundnames{P}$, where $Q\{\vec{y}/\vec{x}\}$
denotes the capture-avoiding substitution of $\vec{y}$ for $\vec{x}$ in $Q$.
\end{definition}

\begin{definition}
  The {\em structural congruence} \cite{SangiorgiWalker} , $\equiv$,
  between processes is the least congruence containing
  alpha-equivalence, satisfying the abelian monoid laws
  (associativity, commutativity and $\pzero$ as identity) for parallel
  composition $|$ and for summation $+$.
\end{definition}

\subsection{Name equivalence}

We take name equivalence, written $\nameeq$, to be the smallest
equivalence relation generated by the following rules.

\begin{mathpar}
\inferrule*[lab=Quote-drop]
{ }
{ \quotep{@{x}} \nameeq x }

\inferrule*[lab=Struct-equiv]
{ P \scong Q }
{ \quotep{P} \nameeq \quotep{Q} }
\end{mathpar}

The astute reader will have noticed that the mutual recursion of names
and processes imposes a mutual recursion on alpha-equivalence and
structural equivalence via name-equivalence. Fortunately, all of this
works out pleasantly and we may calculate in the natural way, free of
concern. The reader interested in the details is referred to the
appendix \ref{appendix:rho_details}.

\subsection{Substitution}

We use $\Proc$ for the set of processes, $\QProc$ for the set of
names, and $\id{\{}\vec{y} / \vec{x} \id{\}}$ to denote partial maps,
$s : \QProc \rightarrow \QProc$. A map, $s$ lifts, uniquely, to a map
on process terms, $\widehat{s} : \Proc \rightarrow \Proc$ by the
following equations.

\begin{mathpar}
  (0) \psubstp{Q}{P} := 0 \\
  (R \juxtap S) \psubstp{Q}{P}
  :=    
  (R)\psubstp{Q}{P} \juxtap (S) \psubstp{Q}{P} \\
  (x?(y).R) \psubstp{Q}{P}    
  :=    
  (x)\substp{Q}{P} (z)\concat( (R \psubstn{z}{y}) \psubstp{Q}{P} ) \\
  (\lift{x}{R}) \psubstp{Q}{P}  
  :=
  \lift{(x)\substp{Q}{P}}{ R \psubstp{Q}{P} } \\
%   (\dropn{x})  \psubstp{Q}{P}       
%   := 
%   \left\{ 
%     \begin{array}{ccc} 
%       \dropn{\quotep{Q}} & & x \nameeq \quotep{P} \\
%       \dropn{x} & & otherwise \\
%     \end{array}
%   \right. 
  (\dropn{x})  \psubstp{Q}{P}       
  := 
  \left\{ 
    \begin{array}{ccc} 
      Q & & x \nameeq \quotep{P} \\
      \dropn{x} & & otherwise \\
    \end{array}
  \right.
\end{mathpar}
 

where

\begin{eqnarray}
  (x)\id{\{} \lpquote Q \rpquote / \lpquote P \rpquote \id{\}}            = 
  \left\{ 
    \begin{array}{ccc}
      \lpquote Q \rpquote & & x \nameeq \lpquote P \rpquote \\
      x & & otherwise \\
    \end{array}
  \right. \nonumber
\end{eqnarray}

and $z$ is chosen distinct from $\quotep{P}$, $\quotep{Q}$, the free
names in $Q$, and all the names in $R$. Our $\alpha$-equivalence will
be built in the standard way from this substitution.

\begin{remark}\label{rem:no_self_referential_names}
  One consequence of these definitions is that $\forall P. \quotep{P}
  \not\in \freenames{P}$.
\end{remark}

\subsection{ Dynamic quote: an example }

Anticipating something of what's to come, consider applying the
substitution, $\widehat{\id{\{}u / z \id{\}}}$, to the following pair
of processes, $\lift{w}{y!(z)}$ and $w[ \lpquote y!(z) \rpquote ]$.

\begin{eqnarray}
	\lift{w}{y!(z)}\widehat{\id{\{}u / z \id{\}}}
		& = &
		\lift{w}{y!(u)} \nonumber\\
	w[ \lpquote y!(z) \rpquote ] \widehat{ \id{\{}u / z \id{\}} }
		& = &
		w[ \lpquote y!(z) \rpquote ] \nonumber
\end{eqnarray}

Because the body of the process between quotes is impervious to
substitution, we get radically different answers. In fact, by
examining the first process in an input context,
e.g. $x?(z).\lift{w}{y!(z)}$, we see that the process under the lift
operator may be shaped by prefixed inputs binding a name inside it. In
this sense, the lift operator will be seen as a way to dynamically
construct processes before reifying them as names.

Finally equipped with these standard features we can present the
dynamics of the calculus.

\subsubsection{Operational semantics} 

Finally, we introduce the computational dynamics. What marks these
algebras as distinct from other more traditionally studied algebraic
structures, e.g. vector spaces or polynomial rings, is the manner in
which dynamics is captured. In traditional structures, dynamics is typically
expressed through morphisms between such structures, as in linear maps
between vector spaces or morphisms between rings. In algebras
associated with the semantics of computation, the dynamics is
expressed as part of the algebraic structure itself, through a
reduction reduction relation typically denoted by $\red$. Below, we
give a recursive presentation of this relation for the calculus used
in the encoding.

$\red \subseteq \pi \times \pi$
$\red : \pi \to \mathcal{P}(\pi)$

\begin{mathpar}
  \inferrule* [lab=Comm] { \textsf{match}( x_{src}, x_{trgt} ) } { x_{trgt}?(y)P \; | \; x_{src}!\langle {Q} \rangle \red P\{\quotep{Q}/y}\} }
  \and \\
  \inferrule* [lab=Par] {{P} \red {P}'} {{{P} | {Q}} \red {{P}' | {Q}}}
  \and
  \inferrule* [lab=Equiv]{{{P} \scong {P}'} \andalso {{P}' \red {Q}'} \andalso {{Q}' \scong {Q}}}{{P} \red {Q}}
\end{mathpar}

\begin{eqnarray*}
  match_{\equiv} (\quotep{P},\quotep{Q}) & := & P \equiv Q \\
  match_{\dagger}(\quotep{P},\quotep{Q}) & := & \forall R. P|Q \red^{*} R => R \red^{*} 0 \\
  match_{K}(\quotep{P},\quotep{Q}) & := & K \mbox{ for some context } K
\end{eqnarray*}

$u?(x)P | u!\langle Q \rangle \red P\{\quotep{Q}/x\}$

%We write $\wred$ for $\red^*$, and $P\red$ if $\exists Q $ such that $ P \red Q$.
We write $P\red$ if $\exists Q $ such that $ P \red Q$ and $P\not\red$, otherwise.

\section{Replication}

As mentioned before, it is known that replication (and hence
recursion) can be implemented in a higher-order process algebra
\cite{SangiorgiWalker}. As our first example of calculation with the
machinery thus far presented we give the construction explicitly in
the {\rhoc}.

\begin{eqnarray}
	D_{x} & := & \prefix{x}{y}{(\binpar{\outputp{x}{y}}{@{y}})} \nonumber\\
	\bangp_{x}{P} & := & \binpar{{x}!\langle{\binpar{D_{x}}{P}}\rangle}{D_{x}} \nonumber
\end{eqnarray}

\begin{eqnarray}
	\bangp_{x}{P} & & \nonumber\\
	=
	& {x}!\langle{(\prefix{x}{y}{(\outputp{x}{y} | @{y})) | P}}\rangle 
	      | \prefix{x}{y}{(\outputp{x}{y} | @{y})} & \nonumber\\
	\red
	& (\outputp{x}{y} | @{y})\substn{\quotep{(\prefix{x}{y}{(@{y} | \outputp{x}{y})) | P}}}{y} & \nonumber\\
	=
	& \outputp{x}{\quotep{(\prefix{x}{y}{(\outputp{x}{y} | @{y})) | P}}}
	  | {(\prefix{x}{y}{(\outputp{x}{y} | @{y})) | P}} & \nonumber\\
	\red
	& \ldots & \nonumber\\
	\red^*
	& P | P | \ldots & \nonumber
\end{eqnarray}

Of course, this encoding, as an implementation, runs away, unfolding
$\bangp{P}$ eagerly. A lazier and more implementable replication
operator, restricted to input-guarded processes, may be obtained as follows.

\begin{eqnarray}
\bangp{\prefix{u}{v}{P}} 
	:= 
	\binpar{\lift{x}{\prefix{u}{v}{(\binpar{D(x)}{P})}}}{D(x)} \nonumber
\end{eqnarray}

\begin{remark}
  Note that the lazier definition still does not deal with summation
  or mixed summation (i.e. sums over input and output). The reader is
  invited to construct definitions of replication that deal with these
  features. 

  Further, the definitions are parameterized in a name, $x$. Can you,
  gentle reader, make a definition that eliminates this parameter and
  guarantees no accidental interaction between the replication
  machinery and the process being replicated -- i.e. no accidental
  sharing of names used by the process to get its work done and the
  name(s) used by the replication to effect copying. This latter
  revision of the definition of replication is crucial to obtaining
  the expected identity $!!P \sim !P$.
\end{remark}

\begin{remark}\label{rem:paradoxical_combinator}
  The reader familiar with the lambda calculus will have noticed the
  similarity between $D$ and the paradoxical combinator.

  [Ed. note: the existence of this seems to suggest we have to be more
  restrictive on the set of processes and names we admit if we are to
  support no-cloning.]
\end{remark}

\subsubsection{Bisimulation}

The computational dynamics gives rise to another kind of equivalence,
the equivalence of computational behavior. As previously mentioned
this is typically captured \emph{via} some form of bisimulation.

% The notion we use in this paper is weak barbed bisimulation
% \cite{milner91polyadicpi}.

The notion we use in this paper is derived from weak barbed
bisimulation \cite{milner91polyadicpi}. 

\begin{definition}
An \emph{observation relation}, $\downarrow_{\mathcal N}$, over a set
of names, $\mathcal N$, is the smallest relation satisfying the rules
below.

\infrule[Out-barb]{y \in {\mathcal N}, \; x \nameeq y}
		  {\outputp{x}{v} \downarrow_{\mathcal N} x}
\infrule[Par-barb]{\mbox{$P\downarrow_{\mathcal N} x$ or $Q\downarrow_{\mathcal N} x$}}
		  {\binpar{P}{Q} \downarrow_{\mathcal N} x}

We write $P \Downarrow_{\mathcal N} x$ if there is $Q$ such that 
$P \wred Q$ and $Q \downarrow_{\mathcal N} x$.
\end{definition}

\begin{definition}
%\label{def.bbisim}
An  ${\mathcal N}$-\emph{barbed bisimulation} over a set of names, ${\mathcal N}$, is a symmetric binary relation 
${\mathcal S}_{\mathcal N}$ between agents such that $P\rel{S}_{\mathcal N}Q$ implies:
\begin{enumerate}
\item If $P \red P'$ then $Q \wred Q'$ and $P'\rel{S}_{\mathcal N} Q'$.
\item If $P\downarrow_{\mathcal N} x$, then $Q\Downarrow_{\mathcal N} x$.
\end{enumerate}
$P$ is ${\mathcal N}$-barbed bisimilar to $Q$, written
$P \wbbisim_{\mathcal N} Q$, if $P \rel{S}_{\mathcal N} Q$ for some ${\mathcal N}$-barbed bisimulation ${\mathcal S}_{\mathcal N}$.
\end{definition}

$\mathcal{R} \subseteq \pi \times \pi$

$P \mathcal{R} Q => \forall P'. P \red P' \Rightarrow \exists Q'. Q \red Q', P' \mathcal{R} Q'$

$P \vdash x \Rightarrow Q \vdash x$

\begin{mathpar}
  \inferrule*[lab=Out-barb]{x \nameeq y}{{y}!\langle{Q}\rangle \vdash x}
  \and
  \inferrule*[lab=Par-barb]{\mbox{$P\vdash x$ or $Q\vdash x$}}{\binpar{P}{Q} \vdash x}
\end{mathpar}

\subsubsection{Contexts}

One of the principle advantages of computational calculi like the
$\pi$-calculus is a well-defined notion of context,
contextual-equivalence and a correlation between
contextual-equivalence and notions of bisimulation. The notion of
context allows the decomposition of a process into (sub-)process and
its syntactic environment, its context. Thus, a context may be
thought of as a process with a ``hole'' (written $\Box$) in it. The
application of a context $M$ to a process $P$, written $M[P]$, is
tantamount to filling the hole in $M$ with $P$. In this paper we do
not need the full weight of this theory, but do make use of the notion
of context in the proof the main theorem. 

\begin{mathpar}
  \inferrule* [lab=summation] {} {{M_{M},M_{N}} \bc \Box \;|\; x.M_{A} \;|\; M_{M}+M_{N}}
  \and
  \inferrule* [lab=agent] {} {{M_{A}} \bc (\vec{x})M_{P} \;| \; \clift{P_0,\ldots,M_{P},\ldots,P_N}}
  \and \\
  \inferrule* [lab=process] {} {{M_{P}} \bc M_{N} \;| \;P|M_{P} }
\end{mathpar} 

\begin{mathpar}
  \inferrule* [lab=sychronization] {} {M_{N} \bc \Box \;|\; x?M_{F} \;|\; x!M_{C}}
  \and
  \inferrule* [lab=abstraction] {} {{M_{F}} \bc (x)M_{P} }
  \and
  \inferrule* [lab=concretion] {} {{M_{C}} \bc \langle M_{P} \rangle }
  \and \\
  \inferrule* [lab=process] {} {{M_{P}} \bc M_{N} \;| \;P|M_{P} }
\end{mathpar}

\begin{definition}[contextual application] Given a context $M$, and
  process $P$, we define the \emph{contextual application}, $M[P] :=
  M\{P/\Box\}$. That is, the contextual application of M to P is the
  substitution of $P$ for $\Box$ in $M$.
\end{definition}

$\meaningof{-} : L \to \mathcal{P}(\pi)$

\begin{mathpar}
  \inferrule* [lab=collection] {} {\meaningof{true} = \pi, \and \meaningof{~E} = \pi \setminus \meaningof{E}, \and \meaningof{E_{1} \& E_{2}} = \meaningof{E_{1}} \cap \meaningof{E_{2}}}
\end{mathpar}

\begin{mathpar}
  \inferrule* [lab=structure] {} {\meaningof{0} = \{ P \in \pi | P \equiv 0 \}, \and \\ \meaningof{E_1 | E_2} = \{ P \in \pi | P \equiv P_{1} | P_{2}, P_{1} \in \meaningof{E_{1}}, P_{2} \in \meaningof{E_2}\} }
\end{mathpar}

\begin{mathpar}
 \inferrule* [lab=behavior] {} {\meaningof{\langle a?b \rangle E} = \{ P \in \pi | P \equiv Q | u?(y)P', \\ \and \\\\ \and \\ \;\;\; u \in \meaningof{a}, \forall z.P'\{z/y\} \in \meaningof{E\{z/b\}}\}, \and \\ \meaningof{a!E} = \{ P \in \pi | P \equiv Q | x!\langle P' \rangle, x \in \meaningof{a} P' \in \meaningof{E}\} }
\end{mathpar}

\begin{mathpar}
 \inferrule* [lab=nominal] {} {\meaningof{\quotep{E}} = \{ \quotep{P} \in \quotep{\pi} | P \in \meaningof{E} \}, \and \meaningof{\quotep{P}} = \{ \quotep{Q} \in \quotep{\pi} | P \equiv Q \} \and \\ \meaningof{@\quotep{E}} = \{ P \in \pi | P \equiv @x, x \in \meaningof{E} \}}
\end{mathpar}

\begin{eqnarray*}
  \\
  \meaningof{-} : TS \to ST
\end{eqnarray*}

\begin{eqnarray*}
  \\
  L : TS \to ST
\end{eqnarray*}

\begin{eqnarray*}
  \\
  P \models E \iff P \in \meaningof{E}
\end{eqnarray*}

\begin{eqnarray*}
  P \approx_{L} Q \iff \forall E \in L. P \models E \iff Q \models E
\end{eqnarray*}

\begin{eqnarray*}
  P \approx_{K} Q
\end{eqnarray*}

\begin{eqnarray*}
  P \approx Q
\end{eqnarray*}

$\approx_{K} = \approx = \approx_{L}$

\subsubsection{Contextual duality}

Note that contexts extend the quotation operation to a family of
operations from processes to names. Given a context, $M$, we can
define a \emph{nominal context}, $\quotep{M}$ by $\quotep{M}[P] :=
\quotep{M[P]}$. To foreshadow what is to come we observe that these
operations enjoy a duality with processes very much like the duality
between vectors and maps from vectors to scalars.

Further, because the calculus is essentially higher-order, we have a
correspondence between contexts and processes. More specifically,
given a name $x$ and a context $M$ we can construct $M^{*}_{x}$ such
that 

\begin{mathpar}
  M^{*}_{x} | \lift{x}{P} \red M[P]
\end{mathpar}

namely,

\begin{mathpar}
  M^{*}_{x} := x?(u).M[\dropn{u}]
\end{mathpar}

The dependence of $M^{*}_{x}$ on a name makes it an abstraction, 

\begin{mathpar}
  M^{*} := (x)x?(u).M[\dropn{u}]
\end{mathpar}

\subsection{Additional notation}

It will sometimes be convenient to denote the process a name
quotes. We already have the notation $x = \quotep{P}$, but it will be
convenient to introduce an alternate notation, $\procn{x}$, when we
want to emphasize the connection to the use of the name. Note that, by
virtue of name equivalence, $\quotep{\procn{x}} \nameeq x$; so, the
notation is consistent with previous definitions.

Further, because names have structure it is possible to effect
substitutions on the basis of that structure. This means we need to
upgrade our notation for substitutions, which we accomplish by
adapting comprehension notation. Thus,

\begin{mathpar}
  P\{ y / x : x \in S \}
\end{mathpar}

is interpreted to mean the process derived from P by replacing (in a
capture-avoiding manner) each occurrence of $x$ in $S$ by $y$. For example,

\begin{mathpar}
  P\{ \quotep{\procn{x}|\procn{x}} / x : x \in \freenames{P} \}
\end{mathpar}

will replace each (occurrence) of a free name $x$ in $P$ by
$\quotep{\procn{x}|\procn{x}}$.

Also, we will avail ourselves of the notation $x^{L}$ and $x^{R}$ to
denote injections of a name into disjoint copies of the name
space. There are numerous ways to accomplish this. One example can be
found in \cite{MeredithR05}. This notation overloads to vectors of
names: $\vec{x}^{\pi} := (x_{i}^{\pi} \; : \; 0 \leq i < |\vec{x}| )$ where $\pi \in \{L,R\}$.

We also use $P^{\Box} := P|\Box$.

In \cite{MeredithR05} an interpretation of the new operator is
given. It turns out that there are several possible interpretations
all enjoying the requisite algebraic properties of the operator (see
\cite{milner91polyadicpi}). We will therefore make liberal use of
$(\nu\; \vec{x})P$.

% subsection the_syntax_and_semantics_of_the_notation_system (end)   

\input{qm2pi.qmops} 

\input{qm2pi.sterngerlach} 

\input{qm2pi.metric} 

% section concurrent_process_calculi (end)

%\input{qm2pi.proofsketch}

% section proof sketch (end)

%\input{qm2pi.slviaknots} 

% section spatial logic via knots (end)

\input{qm2pi.conclusion}

% section conclusion (end)

%\input{qm2pi.dtcodes} 

% section wiring algorithm (end)

\input{qm2pi.ack} 

% section acknowledgments (end)

\newpage


\bibliographystyle{plain}   
\bibliography{../../biblios/main.bib}

\input{qm2pi.rhodetails}

\end{document}

 

%\ifpdf
%\usepackage[pdftex]{graphicx}
%\else
%\usepackage{graphicx}
%\fi

 % \ifpdf
%  \usepackage{pdfsync}
%  \if


%\title{Brief Article}
%\author{David F. Snyder}
%\author{L.G. Meredith}

%\address{Dept. of Math., Texas State University--San Marcos, San Marcos, TX 78666}
       
\pagestyle{empty}


\begin{document}

\lstset{language=[Objective]Caml,frame=shadowbox}

\documentclass[12pt]{llncs}
%\documentclass{jktr}

\usepackage[pdftex]{hyperref}                   
\usepackage {listings}
\usepackage {mathpartir}
\usepackage{bcprules}
%\usepackage{listings}
                       
\usepackage{graphicx} 
%\usepackage[margins=2.5cm,nohead,nofoot]{geometry}
%\usepackage{geometry}
\usepackage{amsfonts}
\usepackage{amstext}
\usepackage{latexsym}
\usepackage{amssymb}
\usepackage{color}


%\include{myPreamble}
\include{qm2pi.local} 

%\ifpdf
%\usepackage[pdftex]{graphicx}
%\else
%\usepackage{graphicx}
%\fi

 % \ifpdf
%  \usepackage{pdfsync}
%  \if


%\title{Brief Article}
%\author{David F. Snyder}
%\author{L.G. Meredith}

%\address{Dept. of Math., Texas State University--San Marcos, San Marcos, TX 78666}
       
\pagestyle{empty}


\begin{document}

\lstset{language=[Objective]Caml,frame=shadowbox}

\input{qm2pi.front}

% section front matter (end)

\input{qm2pi.intro} 
 
% section introduction (end)

% \input{qm2pi.knotations} 

% section notation (end)

\input{qm2pi.process.calculi} 

% section concurrent_process_calculi_and_spatial_logics_ (end)
    
%\input{qm2pi.knots2pi} 

%\input{qm2pi.trefoil} 

%\input{qm2pi.mainthm} 

% subsection basic_interpretation (end)

%\input{qm2pi.rho.presentation} 
\subsection{The syntax and semantics of the notation system}\label{sub:the_syntax_and_semantics_of_the_notation_system} % (fold)

We now summarize a technical presentation of the calculus that
embodies our theory of dynamics. The typical presentation of such a
calculus follows the style of giving generators and relations on
them. The grammar, below, describing term constructors, freely
generates the set of processes, $\Proc$. This set is then quotiented
by a relation known as structural congruence and it is over this set
that the notion of dynamics is expressed. This presentation is
essentially that of \cite{MeredithR05} with the addition of
polyadicity and summation. For readability we have relegated some of
the technical subtleties to an appendix.

\subsubsection{Process grammar}\label{subsub:process_grammar}

\begin{mathpar}
  \inferrule* [lab=synchronization] {} {{M} \bc \pzero \;|\; x?F \;|\; x!C }
  \and
  \inferrule* [lab=abstraction] {} {{F} \bc (x)P}
  \and
  \inferrule* [lab=concretion] {} {{C} \bc \langle Q \rangle}
  \and
  \inferrule* [lab=process] {} {{P,Q} \bc M \;| \;P|Q \;|\; @{x}}
  \and
  \inferrule* [lab=name] {} {{x} \bc \quotep{P}}
\end{mathpar} 

Note that $\vec{x}$ (resp. $\vec{P}$) denotes a vector of names
(resp. processes) of length $|\vec{x}|$ (resp. $|\vec{P}|$). We adopt
the following useful abbreviations.

\begin{mathpar}
   x?(\vec{y}).P := x.(\vec{y})P \and  x\clift{\vec{P}} := x.\clift{\vec{P}}
   \and x!(y) := \lift{x}{\dropn{y}}
   \and \Pi_{i=0}^{n-1}P_i := P_0 | \ldots | P_{n-1}
\end{mathpar}

\subsubsection{Structural congruence}

\paragraph{Free and bound names and alpha-equivalence.} At the
core of structural equivalence is alpha-equivalence which identifies
process that are the same up to a change of variable. Formally, we
recognize the distinction between free and bound names. The free names
of a process, $\freenames{P}$, may be calculated recursively as
follows:

\begin{mathpar}
\freenames{\pzero} := \emptyset
  \and \\
  \freenames{x?(y).P} := \{ x \} \cup (\freenames{P} \setminus \{ y \})
  \and 
  \freenames{x!\langle P \rangle} := \{ x \} \cup \{ P \} 
  \and \\
  \freenames{P|Q} := \freenames{P} \cup \freenames{Q}
  \and \\
  \freenames{@{x}} := \{ x \}
\end{mathpar}

$\pi$
$\quotep{\pi}$

$\freenames{-} : \pi \to \mathcal{P}(\quotep{\pi})$

\begin{eqnarray*}
  \freenames{\pzero} & := & \emptyset \\
  \freenames{x?(y).P} & := & \{ x \} \cup (\freenames{P} \setminus \{ y \}) \\
  \freenames{x!\langle P \rangle} & := & \{ x \} \cup \{ P \} \\
  \freenames{P|Q} & := & \freenames{P} \cup \freenames{Q} \\
  \freenames{\dropn{x}} & := & \{ x \}
\end{eqnarray*}

The bound names of a process, $\boundnames{P}$, are those names occurring in $P$
that are not free. For example, in $x?(y).0$, the name $x$ is free, while $y$ is bound.

\begin{mathpar}
  \inferrule* [lab=monoidal-laws] {} { P|Q \equiv Q|P \and P|0 \equiv P \and P|(Q|R) \equiv (P|Q)|R }
\end{mathpar}

\begin{mathpar}
  \inferrule* [lab=alpha-equivalence] {} { (x)P \equiv (y)P\{y/x\} \and y \not\in \freenames{P} }
\end{mathpar}

\begin{definition}
Then two processes, $P,Q$, are alpha-equivalent if $P = Q\{\vec{y}/\vec{x}\}$ for
some $\vec{x} \in \boundnames{Q},\vec{y} \in \boundnames{P}$, where $Q\{\vec{y}/\vec{x}\}$
denotes the capture-avoiding substitution of $\vec{y}$ for $\vec{x}$ in $Q$.
\end{definition}

\begin{definition}
  The {\em structural congruence} \cite{SangiorgiWalker} , $\equiv$,
  between processes is the least congruence containing
  alpha-equivalence, satisfying the abelian monoid laws
  (associativity, commutativity and $\pzero$ as identity) for parallel
  composition $|$ and for summation $+$.
\end{definition}

\subsection{Name equivalence}

We take name equivalence, written $\nameeq$, to be the smallest
equivalence relation generated by the following rules.

\begin{mathpar}
\inferrule*[lab=Quote-drop]
{ }
{ \quotep{@{x}} \nameeq x }

\inferrule*[lab=Struct-equiv]
{ P \scong Q }
{ \quotep{P} \nameeq \quotep{Q} }
\end{mathpar}

The astute reader will have noticed that the mutual recursion of names
and processes imposes a mutual recursion on alpha-equivalence and
structural equivalence via name-equivalence. Fortunately, all of this
works out pleasantly and we may calculate in the natural way, free of
concern. The reader interested in the details is referred to the
appendix \ref{appendix:rho_details}.

\subsection{Substitution}

We use $\Proc$ for the set of processes, $\QProc$ for the set of
names, and $\id{\{}\vec{y} / \vec{x} \id{\}}$ to denote partial maps,
$s : \QProc \rightarrow \QProc$. A map, $s$ lifts, uniquely, to a map
on process terms, $\widehat{s} : \Proc \rightarrow \Proc$ by the
following equations.

\begin{mathpar}
  (0) \psubstp{Q}{P} := 0 \\
  (R \juxtap S) \psubstp{Q}{P}
  :=    
  (R)\psubstp{Q}{P} \juxtap (S) \psubstp{Q}{P} \\
  (x?(y).R) \psubstp{Q}{P}    
  :=    
  (x)\substp{Q}{P} (z)\concat( (R \psubstn{z}{y}) \psubstp{Q}{P} ) \\
  (\lift{x}{R}) \psubstp{Q}{P}  
  :=
  \lift{(x)\substp{Q}{P}}{ R \psubstp{Q}{P} } \\
%   (\dropn{x})  \psubstp{Q}{P}       
%   := 
%   \left\{ 
%     \begin{array}{ccc} 
%       \dropn{\quotep{Q}} & & x \nameeq \quotep{P} \\
%       \dropn{x} & & otherwise \\
%     \end{array}
%   \right. 
  (\dropn{x})  \psubstp{Q}{P}       
  := 
  \left\{ 
    \begin{array}{ccc} 
      Q & & x \nameeq \quotep{P} \\
      \dropn{x} & & otherwise \\
    \end{array}
  \right.
\end{mathpar}
 

where

\begin{eqnarray}
  (x)\id{\{} \lpquote Q \rpquote / \lpquote P \rpquote \id{\}}            = 
  \left\{ 
    \begin{array}{ccc}
      \lpquote Q \rpquote & & x \nameeq \lpquote P \rpquote \\
      x & & otherwise \\
    \end{array}
  \right. \nonumber
\end{eqnarray}

and $z$ is chosen distinct from $\quotep{P}$, $\quotep{Q}$, the free
names in $Q$, and all the names in $R$. Our $\alpha$-equivalence will
be built in the standard way from this substitution.

\begin{remark}\label{rem:no_self_referential_names}
  One consequence of these definitions is that $\forall P. \quotep{P}
  \not\in \freenames{P}$.
\end{remark}

\subsection{ Dynamic quote: an example }

Anticipating something of what's to come, consider applying the
substitution, $\widehat{\id{\{}u / z \id{\}}}$, to the following pair
of processes, $\lift{w}{y!(z)}$ and $w[ \lpquote y!(z) \rpquote ]$.

\begin{eqnarray}
	\lift{w}{y!(z)}\widehat{\id{\{}u / z \id{\}}}
		& = &
		\lift{w}{y!(u)} \nonumber\\
	w[ \lpquote y!(z) \rpquote ] \widehat{ \id{\{}u / z \id{\}} }
		& = &
		w[ \lpquote y!(z) \rpquote ] \nonumber
\end{eqnarray}

Because the body of the process between quotes is impervious to
substitution, we get radically different answers. In fact, by
examining the first process in an input context,
e.g. $x?(z).\lift{w}{y!(z)}$, we see that the process under the lift
operator may be shaped by prefixed inputs binding a name inside it. In
this sense, the lift operator will be seen as a way to dynamically
construct processes before reifying them as names.

Finally equipped with these standard features we can present the
dynamics of the calculus.

\subsubsection{Operational semantics} 

Finally, we introduce the computational dynamics. What marks these
algebras as distinct from other more traditionally studied algebraic
structures, e.g. vector spaces or polynomial rings, is the manner in
which dynamics is captured. In traditional structures, dynamics is typically
expressed through morphisms between such structures, as in linear maps
between vector spaces or morphisms between rings. In algebras
associated with the semantics of computation, the dynamics is
expressed as part of the algebraic structure itself, through a
reduction reduction relation typically denoted by $\red$. Below, we
give a recursive presentation of this relation for the calculus used
in the encoding.

$\red \subseteq \pi \times \pi$
$\red : \pi \to \mathcal{P}(\pi)$

\begin{mathpar}
  \inferrule* [lab=Comm] { \textsf{match}( x_{src}, x_{trgt} ) } { x_{trgt}?(y)P \; | \; x_{src}!\langle {Q} \rangle \red P\{\quotep{Q}/y}\} }
  \and \\
  \inferrule* [lab=Par] {{P} \red {P}'} {{{P} | {Q}} \red {{P}' | {Q}}}
  \and
  \inferrule* [lab=Equiv]{{{P} \scong {P}'} \andalso {{P}' \red {Q}'} \andalso {{Q}' \scong {Q}}}{{P} \red {Q}}
\end{mathpar}

\begin{eqnarray*}
  match_{\equiv} (\quotep{P},\quotep{Q}) & := & P \equiv Q \\
  match_{\dagger}(\quotep{P},\quotep{Q}) & := & \forall R. P|Q \red^{*} R => R \red^{*} 0 \\
  match_{K}(\quotep{P},\quotep{Q}) & := & K \mbox{ for some context } K
\end{eqnarray*}

$u?(x)P | u!\langle Q \rangle \red P\{\quotep{Q}/x\}$

%We write $\wred$ for $\red^*$, and $P\red$ if $\exists Q $ such that $ P \red Q$.
We write $P\red$ if $\exists Q $ such that $ P \red Q$ and $P\not\red$, otherwise.

\section{Replication}

As mentioned before, it is known that replication (and hence
recursion) can be implemented in a higher-order process algebra
\cite{SangiorgiWalker}. As our first example of calculation with the
machinery thus far presented we give the construction explicitly in
the {\rhoc}.

\begin{eqnarray}
	D_{x} & := & \prefix{x}{y}{(\binpar{\outputp{x}{y}}{@{y}})} \nonumber\\
	\bangp_{x}{P} & := & \binpar{{x}!\langle{\binpar{D_{x}}{P}}\rangle}{D_{x}} \nonumber
\end{eqnarray}

\begin{eqnarray}
	\bangp_{x}{P} & & \nonumber\\
	=
	& {x}!\langle{(\prefix{x}{y}{(\outputp{x}{y} | @{y})) | P}}\rangle 
	      | \prefix{x}{y}{(\outputp{x}{y} | @{y})} & \nonumber\\
	\red
	& (\outputp{x}{y} | @{y})\substn{\quotep{(\prefix{x}{y}{(@{y} | \outputp{x}{y})) | P}}}{y} & \nonumber\\
	=
	& \outputp{x}{\quotep{(\prefix{x}{y}{(\outputp{x}{y} | @{y})) | P}}}
	  | {(\prefix{x}{y}{(\outputp{x}{y} | @{y})) | P}} & \nonumber\\
	\red
	& \ldots & \nonumber\\
	\red^*
	& P | P | \ldots & \nonumber
\end{eqnarray}

Of course, this encoding, as an implementation, runs away, unfolding
$\bangp{P}$ eagerly. A lazier and more implementable replication
operator, restricted to input-guarded processes, may be obtained as follows.

\begin{eqnarray}
\bangp{\prefix{u}{v}{P}} 
	:= 
	\binpar{\lift{x}{\prefix{u}{v}{(\binpar{D(x)}{P})}}}{D(x)} \nonumber
\end{eqnarray}

\begin{remark}
  Note that the lazier definition still does not deal with summation
  or mixed summation (i.e. sums over input and output). The reader is
  invited to construct definitions of replication that deal with these
  features. 

  Further, the definitions are parameterized in a name, $x$. Can you,
  gentle reader, make a definition that eliminates this parameter and
  guarantees no accidental interaction between the replication
  machinery and the process being replicated -- i.e. no accidental
  sharing of names used by the process to get its work done and the
  name(s) used by the replication to effect copying. This latter
  revision of the definition of replication is crucial to obtaining
  the expected identity $!!P \sim !P$.
\end{remark}

\begin{remark}\label{rem:paradoxical_combinator}
  The reader familiar with the lambda calculus will have noticed the
  similarity between $D$ and the paradoxical combinator.

  [Ed. note: the existence of this seems to suggest we have to be more
  restrictive on the set of processes and names we admit if we are to
  support no-cloning.]
\end{remark}

\subsubsection{Bisimulation}

The computational dynamics gives rise to another kind of equivalence,
the equivalence of computational behavior. As previously mentioned
this is typically captured \emph{via} some form of bisimulation.

% The notion we use in this paper is weak barbed bisimulation
% \cite{milner91polyadicpi}.

The notion we use in this paper is derived from weak barbed
bisimulation \cite{milner91polyadicpi}. 

\begin{definition}
An \emph{observation relation}, $\downarrow_{\mathcal N}$, over a set
of names, $\mathcal N$, is the smallest relation satisfying the rules
below.

\infrule[Out-barb]{y \in {\mathcal N}, \; x \nameeq y}
		  {\outputp{x}{v} \downarrow_{\mathcal N} x}
\infrule[Par-barb]{\mbox{$P\downarrow_{\mathcal N} x$ or $Q\downarrow_{\mathcal N} x$}}
		  {\binpar{P}{Q} \downarrow_{\mathcal N} x}

We write $P \Downarrow_{\mathcal N} x$ if there is $Q$ such that 
$P \wred Q$ and $Q \downarrow_{\mathcal N} x$.
\end{definition}

\begin{definition}
%\label{def.bbisim}
An  ${\mathcal N}$-\emph{barbed bisimulation} over a set of names, ${\mathcal N}$, is a symmetric binary relation 
${\mathcal S}_{\mathcal N}$ between agents such that $P\rel{S}_{\mathcal N}Q$ implies:
\begin{enumerate}
\item If $P \red P'$ then $Q \wred Q'$ and $P'\rel{S}_{\mathcal N} Q'$.
\item If $P\downarrow_{\mathcal N} x$, then $Q\Downarrow_{\mathcal N} x$.
\end{enumerate}
$P$ is ${\mathcal N}$-barbed bisimilar to $Q$, written
$P \wbbisim_{\mathcal N} Q$, if $P \rel{S}_{\mathcal N} Q$ for some ${\mathcal N}$-barbed bisimulation ${\mathcal S}_{\mathcal N}$.
\end{definition}

$\mathcal{R} \subseteq \pi \times \pi$

$P \mathcal{R} Q => \forall P'. P \red P' \Rightarrow \exists Q'. Q \red Q', P' \mathcal{R} Q'$

$P \vdash x \Rightarrow Q \vdash x$

\begin{mathpar}
  \inferrule*[lab=Out-barb]{x \nameeq y}{{y}!\langle{Q}\rangle \vdash x}
  \and
  \inferrule*[lab=Par-barb]{\mbox{$P\vdash x$ or $Q\vdash x$}}{\binpar{P}{Q} \vdash x}
\end{mathpar}

\subsubsection{Contexts}

One of the principle advantages of computational calculi like the
$\pi$-calculus is a well-defined notion of context,
contextual-equivalence and a correlation between
contextual-equivalence and notions of bisimulation. The notion of
context allows the decomposition of a process into (sub-)process and
its syntactic environment, its context. Thus, a context may be
thought of as a process with a ``hole'' (written $\Box$) in it. The
application of a context $M$ to a process $P$, written $M[P]$, is
tantamount to filling the hole in $M$ with $P$. In this paper we do
not need the full weight of this theory, but do make use of the notion
of context in the proof the main theorem. 

\begin{mathpar}
  \inferrule* [lab=summation] {} {{M_{M},M_{N}} \bc \Box \;|\; x.M_{A} \;|\; M_{M}+M_{N}}
  \and
  \inferrule* [lab=agent] {} {{M_{A}} \bc (\vec{x})M_{P} \;| \; \clift{P_0,\ldots,M_{P},\ldots,P_N}}
  \and \\
  \inferrule* [lab=process] {} {{M_{P}} \bc M_{N} \;| \;P|M_{P} }
\end{mathpar} 

\begin{mathpar}
  \inferrule* [lab=sychronization] {} {M_{N} \bc \Box \;|\; x?M_{F} \;|\; x!M_{C}}
  \and
  \inferrule* [lab=abstraction] {} {{M_{F}} \bc (x)M_{P} }
  \and
  \inferrule* [lab=concretion] {} {{M_{C}} \bc \langle M_{P} \rangle }
  \and \\
  \inferrule* [lab=process] {} {{M_{P}} \bc M_{N} \;| \;P|M_{P} }
\end{mathpar}

\begin{definition}[contextual application] Given a context $M$, and
  process $P$, we define the \emph{contextual application}, $M[P] :=
  M\{P/\Box\}$. That is, the contextual application of M to P is the
  substitution of $P$ for $\Box$ in $M$.
\end{definition}

$\meaningof{-} : L \to \mathcal{P}(\pi)$

\begin{mathpar}
  \inferrule* [lab=collection] {} {\meaningof{true} = \pi, \and \meaningof{~E} = \pi \setminus \meaningof{E}, \and \meaningof{E_{1} \& E_{2}} = \meaningof{E_{1}} \cap \meaningof{E_{2}}}
\end{mathpar}

\begin{mathpar}
  \inferrule* [lab=structure] {} {\meaningof{0} = \{ P \in \pi | P \equiv 0 \}, \and \\ \meaningof{E_1 | E_2} = \{ P \in \pi | P \equiv P_{1} | P_{2}, P_{1} \in \meaningof{E_{1}}, P_{2} \in \meaningof{E_2}\} }
\end{mathpar}

\begin{mathpar}
 \inferrule* [lab=behavior] {} {\meaningof{\langle a?b \rangle E} = \{ P \in \pi | P \equiv Q | u?(y)P', \\ \and \\\\ \and \\ \;\;\; u \in \meaningof{a}, \forall z.P'\{z/y\} \in \meaningof{E\{z/b\}}\}, \and \\ \meaningof{a!E} = \{ P \in \pi | P \equiv Q | x!\langle P' \rangle, x \in \meaningof{a} P' \in \meaningof{E}\} }
\end{mathpar}

\begin{mathpar}
 \inferrule* [lab=nominal] {} {\meaningof{\quotep{E}} = \{ \quotep{P} \in \quotep{\pi} | P \in \meaningof{E} \}, \and \meaningof{\quotep{P}} = \{ \quotep{Q} \in \quotep{\pi} | P \equiv Q \} \and \\ \meaningof{@\quotep{E}} = \{ P \in \pi | P \equiv @x, x \in \meaningof{E} \}}
\end{mathpar}

\begin{eqnarray*}
  \\
  \meaningof{-} : TS \to ST
\end{eqnarray*}

\begin{eqnarray*}
  \\
  L : TS \to ST
\end{eqnarray*}

\begin{eqnarray*}
  \\
  P \models E \iff P \in \meaningof{E}
\end{eqnarray*}

\begin{eqnarray*}
  P \approx_{L} Q \iff \forall E \in L. P \models E \iff Q \models E
\end{eqnarray*}

\begin{eqnarray*}
  P \approx_{K} Q
\end{eqnarray*}

\begin{eqnarray*}
  P \approx Q
\end{eqnarray*}

$\approx_{K} = \approx = \approx_{L}$

\subsubsection{Contextual duality}

Note that contexts extend the quotation operation to a family of
operations from processes to names. Given a context, $M$, we can
define a \emph{nominal context}, $\quotep{M}$ by $\quotep{M}[P] :=
\quotep{M[P]}$. To foreshadow what is to come we observe that these
operations enjoy a duality with processes very much like the duality
between vectors and maps from vectors to scalars.

Further, because the calculus is essentially higher-order, we have a
correspondence between contexts and processes. More specifically,
given a name $x$ and a context $M$ we can construct $M^{*}_{x}$ such
that 

\begin{mathpar}
  M^{*}_{x} | \lift{x}{P} \red M[P]
\end{mathpar}

namely,

\begin{mathpar}
  M^{*}_{x} := x?(u).M[\dropn{u}]
\end{mathpar}

The dependence of $M^{*}_{x}$ on a name makes it an abstraction, 

\begin{mathpar}
  M^{*} := (x)x?(u).M[\dropn{u}]
\end{mathpar}

\subsection{Additional notation}

It will sometimes be convenient to denote the process a name
quotes. We already have the notation $x = \quotep{P}$, but it will be
convenient to introduce an alternate notation, $\procn{x}$, when we
want to emphasize the connection to the use of the name. Note that, by
virtue of name equivalence, $\quotep{\procn{x}} \nameeq x$; so, the
notation is consistent with previous definitions.

Further, because names have structure it is possible to effect
substitutions on the basis of that structure. This means we need to
upgrade our notation for substitutions, which we accomplish by
adapting comprehension notation. Thus,

\begin{mathpar}
  P\{ y / x : x \in S \}
\end{mathpar}

is interpreted to mean the process derived from P by replacing (in a
capture-avoiding manner) each occurrence of $x$ in $S$ by $y$. For example,

\begin{mathpar}
  P\{ \quotep{\procn{x}|\procn{x}} / x : x \in \freenames{P} \}
\end{mathpar}

will replace each (occurrence) of a free name $x$ in $P$ by
$\quotep{\procn{x}|\procn{x}}$.

Also, we will avail ourselves of the notation $x^{L}$ and $x^{R}$ to
denote injections of a name into disjoint copies of the name
space. There are numerous ways to accomplish this. One example can be
found in \cite{MeredithR05}. This notation overloads to vectors of
names: $\vec{x}^{\pi} := (x_{i}^{\pi} \; : \; 0 \leq i < |\vec{x}| )$ where $\pi \in \{L,R\}$.

We also use $P^{\Box} := P|\Box$.

In \cite{MeredithR05} an interpretation of the new operator is
given. It turns out that there are several possible interpretations
all enjoying the requisite algebraic properties of the operator (see
\cite{milner91polyadicpi}). We will therefore make liberal use of
$(\nu\; \vec{x})P$.

% subsection the_syntax_and_semantics_of_the_notation_system (end)   

\input{qm2pi.qmops} 

\input{qm2pi.sterngerlach} 

\input{qm2pi.metric} 

% section concurrent_process_calculi (end)

%\input{qm2pi.proofsketch}

% section proof sketch (end)

%\input{qm2pi.slviaknots} 

% section spatial logic via knots (end)

\input{qm2pi.conclusion}

% section conclusion (end)

%\input{qm2pi.dtcodes} 

% section wiring algorithm (end)

\input{qm2pi.ack} 

% section acknowledgments (end)

\newpage


\bibliographystyle{plain}   
\bibliography{../../biblios/main.bib}

\input{qm2pi.rhodetails}

\end{document}



% section front matter (end)

\section{Introduction}\label{sec:introduction} % (fold)
In this draft of the material i am going to have to dispense with the
usual writing conventions adopted in papers on these topics. i'm going
to have adopt whatever tone i need at the time i'm writing up the
calculations. Sometimes this may be very conversational; others it may
be the barest mathematical grunts; others still it may be that i have
lifted text from one of my other papers because the exposition of some
point was better said there. i hope that my readers are not unduly put
out by this decision. i'm not doing this to flout convention or be
rebellious. i find these calculations very technically challenging. To
keep everything going technically, something has to give; i have to
let go of some cognitive burden. So, the academic writing style --
with all of its trade-offs in terms of facilitating technical
communication -- is what i'm letting go of. Perhaps subsequent drafts
can be tightened and polished, but for now, i'm going to speak as if
we were sitting together in a coffee shop with a laptop, wifi and a
pad of paper and a pencil.

So, here's what i have to say. We -- you and i, comfortably ensconced
in our coffee shop and well-equipped with our tools -- can realize and
carry out the calculations of quantum mechanics over a very different
formal theory of dynamics, a formal theory of dynamics that
corresponds to a theory of concurrent computation with
\emph{reflection}. It has the advantage that the underlying theory is
already `quantized', but supports analogues all of the continuuous
operations. Strikingly, this underlying theory has recently been
connected with a notion of metric that we can show, by calculating
together, coincides with the metric induced by the inner product.

There are a lot of reasons why you might be interested in seeing
calculations of this form. Here's why i'm interested. For the past
several centuries there has been no competitor to the ``Newtonian''
account of dynamics. As a result the predominant share of accounts of
dynamical systems and situations have had to be formulated in terms of
the Newtonian machinery. i view this as an intellectually dangerous
position to occupy. Everything, despite it's intrinsic shape, turns
into a nail to be hit with this hammer. Recently, however, the theory
of computation has matured to the point where we have candidates for
theories of dynamics that offer very different perspective on
reasoning about dynamical systems and situations. Testing these
candidates against very successful accounts of dynamical situations,
like quantum mechanics, is going to give us some sense of how mature
they are and some measure of the quality of these accounts of
dynamics.

\subsection{Summary of contributions and outline of paper}

So, we're going to develop an interpretation of the operations of
quantum mechanics normally interpreted by Hilbert spaces and
operators. We're going to do this over a theory of computation. Note
that this is very different than the usual quantum computation program
which develops notions of computation over quantum mechanics. Rather,
we are developing a story that aligns with Wheeler's slogan: It from
Bit. To do this we will first provide an account of the theory of
computation at play here. Then we will dive into a calculation-driven
interpretation of the operations of quantum mechanics.

The reason we take this approach is that -- until very recently --
there hasn't been an axiomatic account of quantum mechanics. As a
result there has been no sharp delineation of the mathematical theory
supporting interpretation of the physical theory and the physical
theory, itself. So, ambient features of the maths are free to be
exploited (or supressed) without a real accounting of their physical
relevance. There is no sharp statement ``here's the physical theory''
qua \emph{theory} and ``here's the mathematical interpretation''
enabling a judgment of how faithful the interpretation is -- apart
from experimental observation. When there is an axiomatic account we
can judge how well a given mathematical formalism supports an
interpretation of the axioms, independent of
experimentation. Likewise, we can judge how well we have captured our
physical evidence and experience with our axiomatics, independent of
any specific mathematical implementation, with accidental detail that
may or may not have physical significance. 

In lieu of a fully fleshed out and vetted axiomatic account of quantum
mechanics, interpreting the operational notions in service of modeling
physical systems will have to suffice. In other words, we are not in
the business of providing a model of Hilbert spaces and operators. We
are in the business of providing a model of quantum mechanics because
we are motivated by testing our notions of dynamics against physical
theory; and, the predictive calculations of the physical theory must
serve as the best formulation -- shy of a fully fleshed out axiomatic
account -- of the physical theory itself (as they have for scientific
theories since time immemorial). Put another way, despite a
whole-hearted commitment to an It-from-Bit ontology, we are firmly
aligned with the shut-up-and-calculate camp as the best way to obtain
results either from the physical perspective or as a quality assurance
measure of our fledgling theory of dynamics.

In detail, we present a reflective process calculus. Then we develop
intuitive correspondences between the notions available in this
calculus and the usual physical notions supporting quantum mechanical
calculations. Thus, 

\begin{table}[htp]
  \center{
    \fbox{
      \begin{tabular}{c|c}
        quantum mechanics & process calculus \\
        \hline
        scalar & name \\
        state vector & process \\
        dual & contextual duals \\
        matrix & formal sums of process-context-dual pairs \\
        orthogonality & process annihilation \\
        inner product & execution-formula + quoting
      \end{tabular}
    }
  }
  \caption{QM - process calculi correspondences}
\end{table}

Then we tighten up these intuitions to operational definitions. We
employ the Dirac notation as the best proxy we can find for an
abstract syntax of the quantum mechanical notions. The definitions we
develop put us in contact with equational constraints coming from the
theory that we demonstrate the definitions and calculations satisfy.

This puts us in a position to shut up and calculate for the
Stern-Gerlach experimental set up, showing how these predictive
calculations become calculations on processes in our theory of a
reflective process calculus.

Penultimately, we demonstrate that the notion of metric coming from
the inner product coincides with the notion of metric available from
the theory of bisimulation. This demonstration gives us the right to
think of space as arising from behavior. Finally, we consider where we
might go from the new vantage point we have obtained.

% section introduction (end) 
 
% section introduction (end)

% \documentclass[12pt]{llncs}
%\documentclass{jktr}

\usepackage[pdftex]{hyperref}                   
\usepackage {listings}
\usepackage {mathpartir}
\usepackage{bcprules}
%\usepackage{listings}
                       
\usepackage{graphicx} 
%\usepackage[margins=2.5cm,nohead,nofoot]{geometry}
%\usepackage{geometry}
\usepackage{amsfonts}
\usepackage{amstext}
\usepackage{latexsym}
\usepackage{amssymb}
\usepackage{color}


%\include{myPreamble}
\include{qm2pi.local} 

%\ifpdf
%\usepackage[pdftex]{graphicx}
%\else
%\usepackage{graphicx}
%\fi

 % \ifpdf
%  \usepackage{pdfsync}
%  \if


%\title{Brief Article}
%\author{David F. Snyder}
%\author{L.G. Meredith}

%\address{Dept. of Math., Texas State University--San Marcos, San Marcos, TX 78666}
       
\pagestyle{empty}


\begin{document}

\lstset{language=[Objective]Caml,frame=shadowbox}

\input{qm2pi.front}

% section front matter (end)

\input{qm2pi.intro} 
 
% section introduction (end)

% \input{qm2pi.knotations} 

% section notation (end)

\input{qm2pi.process.calculi} 

% section concurrent_process_calculi_and_spatial_logics_ (end)
    
%\input{qm2pi.knots2pi} 

%\input{qm2pi.trefoil} 

%\input{qm2pi.mainthm} 

% subsection basic_interpretation (end)

%\input{qm2pi.rho.presentation} 
\subsection{The syntax and semantics of the notation system}\label{sub:the_syntax_and_semantics_of_the_notation_system} % (fold)

We now summarize a technical presentation of the calculus that
embodies our theory of dynamics. The typical presentation of such a
calculus follows the style of giving generators and relations on
them. The grammar, below, describing term constructors, freely
generates the set of processes, $\Proc$. This set is then quotiented
by a relation known as structural congruence and it is over this set
that the notion of dynamics is expressed. This presentation is
essentially that of \cite{MeredithR05} with the addition of
polyadicity and summation. For readability we have relegated some of
the technical subtleties to an appendix.

\subsubsection{Process grammar}\label{subsub:process_grammar}

\begin{mathpar}
  \inferrule* [lab=synchronization] {} {{M} \bc \pzero \;|\; x?F \;|\; x!C }
  \and
  \inferrule* [lab=abstraction] {} {{F} \bc (x)P}
  \and
  \inferrule* [lab=concretion] {} {{C} \bc \langle Q \rangle}
  \and
  \inferrule* [lab=process] {} {{P,Q} \bc M \;| \;P|Q \;|\; @{x}}
  \and
  \inferrule* [lab=name] {} {{x} \bc \quotep{P}}
\end{mathpar} 

Note that $\vec{x}$ (resp. $\vec{P}$) denotes a vector of names
(resp. processes) of length $|\vec{x}|$ (resp. $|\vec{P}|$). We adopt
the following useful abbreviations.

\begin{mathpar}
   x?(\vec{y}).P := x.(\vec{y})P \and  x\clift{\vec{P}} := x.\clift{\vec{P}}
   \and x!(y) := \lift{x}{\dropn{y}}
   \and \Pi_{i=0}^{n-1}P_i := P_0 | \ldots | P_{n-1}
\end{mathpar}

\subsubsection{Structural congruence}

\paragraph{Free and bound names and alpha-equivalence.} At the
core of structural equivalence is alpha-equivalence which identifies
process that are the same up to a change of variable. Formally, we
recognize the distinction between free and bound names. The free names
of a process, $\freenames{P}$, may be calculated recursively as
follows:

\begin{mathpar}
\freenames{\pzero} := \emptyset
  \and \\
  \freenames{x?(y).P} := \{ x \} \cup (\freenames{P} \setminus \{ y \})
  \and 
  \freenames{x!\langle P \rangle} := \{ x \} \cup \{ P \} 
  \and \\
  \freenames{P|Q} := \freenames{P} \cup \freenames{Q}
  \and \\
  \freenames{@{x}} := \{ x \}
\end{mathpar}

$\pi$
$\quotep{\pi}$

$\freenames{-} : \pi \to \mathcal{P}(\quotep{\pi})$

\begin{eqnarray*}
  \freenames{\pzero} & := & \emptyset \\
  \freenames{x?(y).P} & := & \{ x \} \cup (\freenames{P} \setminus \{ y \}) \\
  \freenames{x!\langle P \rangle} & := & \{ x \} \cup \{ P \} \\
  \freenames{P|Q} & := & \freenames{P} \cup \freenames{Q} \\
  \freenames{\dropn{x}} & := & \{ x \}
\end{eqnarray*}

The bound names of a process, $\boundnames{P}$, are those names occurring in $P$
that are not free. For example, in $x?(y).0$, the name $x$ is free, while $y$ is bound.

\begin{mathpar}
  \inferrule* [lab=monoidal-laws] {} { P|Q \equiv Q|P \and P|0 \equiv P \and P|(Q|R) \equiv (P|Q)|R }
\end{mathpar}

\begin{mathpar}
  \inferrule* [lab=alpha-equivalence] {} { (x)P \equiv (y)P\{y/x\} \and y \not\in \freenames{P} }
\end{mathpar}

\begin{definition}
Then two processes, $P,Q$, are alpha-equivalent if $P = Q\{\vec{y}/\vec{x}\}$ for
some $\vec{x} \in \boundnames{Q},\vec{y} \in \boundnames{P}$, where $Q\{\vec{y}/\vec{x}\}$
denotes the capture-avoiding substitution of $\vec{y}$ for $\vec{x}$ in $Q$.
\end{definition}

\begin{definition}
  The {\em structural congruence} \cite{SangiorgiWalker} , $\equiv$,
  between processes is the least congruence containing
  alpha-equivalence, satisfying the abelian monoid laws
  (associativity, commutativity and $\pzero$ as identity) for parallel
  composition $|$ and for summation $+$.
\end{definition}

\subsection{Name equivalence}

We take name equivalence, written $\nameeq$, to be the smallest
equivalence relation generated by the following rules.

\begin{mathpar}
\inferrule*[lab=Quote-drop]
{ }
{ \quotep{@{x}} \nameeq x }

\inferrule*[lab=Struct-equiv]
{ P \scong Q }
{ \quotep{P} \nameeq \quotep{Q} }
\end{mathpar}

The astute reader will have noticed that the mutual recursion of names
and processes imposes a mutual recursion on alpha-equivalence and
structural equivalence via name-equivalence. Fortunately, all of this
works out pleasantly and we may calculate in the natural way, free of
concern. The reader interested in the details is referred to the
appendix \ref{appendix:rho_details}.

\subsection{Substitution}

We use $\Proc$ for the set of processes, $\QProc$ for the set of
names, and $\id{\{}\vec{y} / \vec{x} \id{\}}$ to denote partial maps,
$s : \QProc \rightarrow \QProc$. A map, $s$ lifts, uniquely, to a map
on process terms, $\widehat{s} : \Proc \rightarrow \Proc$ by the
following equations.

\begin{mathpar}
  (0) \psubstp{Q}{P} := 0 \\
  (R \juxtap S) \psubstp{Q}{P}
  :=    
  (R)\psubstp{Q}{P} \juxtap (S) \psubstp{Q}{P} \\
  (x?(y).R) \psubstp{Q}{P}    
  :=    
  (x)\substp{Q}{P} (z)\concat( (R \psubstn{z}{y}) \psubstp{Q}{P} ) \\
  (\lift{x}{R}) \psubstp{Q}{P}  
  :=
  \lift{(x)\substp{Q}{P}}{ R \psubstp{Q}{P} } \\
%   (\dropn{x})  \psubstp{Q}{P}       
%   := 
%   \left\{ 
%     \begin{array}{ccc} 
%       \dropn{\quotep{Q}} & & x \nameeq \quotep{P} \\
%       \dropn{x} & & otherwise \\
%     \end{array}
%   \right. 
  (\dropn{x})  \psubstp{Q}{P}       
  := 
  \left\{ 
    \begin{array}{ccc} 
      Q & & x \nameeq \quotep{P} \\
      \dropn{x} & & otherwise \\
    \end{array}
  \right.
\end{mathpar}
 

where

\begin{eqnarray}
  (x)\id{\{} \lpquote Q \rpquote / \lpquote P \rpquote \id{\}}            = 
  \left\{ 
    \begin{array}{ccc}
      \lpquote Q \rpquote & & x \nameeq \lpquote P \rpquote \\
      x & & otherwise \\
    \end{array}
  \right. \nonumber
\end{eqnarray}

and $z$ is chosen distinct from $\quotep{P}$, $\quotep{Q}$, the free
names in $Q$, and all the names in $R$. Our $\alpha$-equivalence will
be built in the standard way from this substitution.

\begin{remark}\label{rem:no_self_referential_names}
  One consequence of these definitions is that $\forall P. \quotep{P}
  \not\in \freenames{P}$.
\end{remark}

\subsection{ Dynamic quote: an example }

Anticipating something of what's to come, consider applying the
substitution, $\widehat{\id{\{}u / z \id{\}}}$, to the following pair
of processes, $\lift{w}{y!(z)}$ and $w[ \lpquote y!(z) \rpquote ]$.

\begin{eqnarray}
	\lift{w}{y!(z)}\widehat{\id{\{}u / z \id{\}}}
		& = &
		\lift{w}{y!(u)} \nonumber\\
	w[ \lpquote y!(z) \rpquote ] \widehat{ \id{\{}u / z \id{\}} }
		& = &
		w[ \lpquote y!(z) \rpquote ] \nonumber
\end{eqnarray}

Because the body of the process between quotes is impervious to
substitution, we get radically different answers. In fact, by
examining the first process in an input context,
e.g. $x?(z).\lift{w}{y!(z)}$, we see that the process under the lift
operator may be shaped by prefixed inputs binding a name inside it. In
this sense, the lift operator will be seen as a way to dynamically
construct processes before reifying them as names.

Finally equipped with these standard features we can present the
dynamics of the calculus.

\subsubsection{Operational semantics} 

Finally, we introduce the computational dynamics. What marks these
algebras as distinct from other more traditionally studied algebraic
structures, e.g. vector spaces or polynomial rings, is the manner in
which dynamics is captured. In traditional structures, dynamics is typically
expressed through morphisms between such structures, as in linear maps
between vector spaces or morphisms between rings. In algebras
associated with the semantics of computation, the dynamics is
expressed as part of the algebraic structure itself, through a
reduction reduction relation typically denoted by $\red$. Below, we
give a recursive presentation of this relation for the calculus used
in the encoding.

$\red \subseteq \pi \times \pi$
$\red : \pi \to \mathcal{P}(\pi)$

\begin{mathpar}
  \inferrule* [lab=Comm] { \textsf{match}( x_{src}, x_{trgt} ) } { x_{trgt}?(y)P \; | \; x_{src}!\langle {Q} \rangle \red P\{\quotep{Q}/y}\} }
  \and \\
  \inferrule* [lab=Par] {{P} \red {P}'} {{{P} | {Q}} \red {{P}' | {Q}}}
  \and
  \inferrule* [lab=Equiv]{{{P} \scong {P}'} \andalso {{P}' \red {Q}'} \andalso {{Q}' \scong {Q}}}{{P} \red {Q}}
\end{mathpar}

\begin{eqnarray*}
  match_{\equiv} (\quotep{P},\quotep{Q}) & := & P \equiv Q \\
  match_{\dagger}(\quotep{P},\quotep{Q}) & := & \forall R. P|Q \red^{*} R => R \red^{*} 0 \\
  match_{K}(\quotep{P},\quotep{Q}) & := & K \mbox{ for some context } K
\end{eqnarray*}

$u?(x)P | u!\langle Q \rangle \red P\{\quotep{Q}/x\}$

%We write $\wred$ for $\red^*$, and $P\red$ if $\exists Q $ such that $ P \red Q$.
We write $P\red$ if $\exists Q $ such that $ P \red Q$ and $P\not\red$, otherwise.

\section{Replication}

As mentioned before, it is known that replication (and hence
recursion) can be implemented in a higher-order process algebra
\cite{SangiorgiWalker}. As our first example of calculation with the
machinery thus far presented we give the construction explicitly in
the {\rhoc}.

\begin{eqnarray}
	D_{x} & := & \prefix{x}{y}{(\binpar{\outputp{x}{y}}{@{y}})} \nonumber\\
	\bangp_{x}{P} & := & \binpar{{x}!\langle{\binpar{D_{x}}{P}}\rangle}{D_{x}} \nonumber
\end{eqnarray}

\begin{eqnarray}
	\bangp_{x}{P} & & \nonumber\\
	=
	& {x}!\langle{(\prefix{x}{y}{(\outputp{x}{y} | @{y})) | P}}\rangle 
	      | \prefix{x}{y}{(\outputp{x}{y} | @{y})} & \nonumber\\
	\red
	& (\outputp{x}{y} | @{y})\substn{\quotep{(\prefix{x}{y}{(@{y} | \outputp{x}{y})) | P}}}{y} & \nonumber\\
	=
	& \outputp{x}{\quotep{(\prefix{x}{y}{(\outputp{x}{y} | @{y})) | P}}}
	  | {(\prefix{x}{y}{(\outputp{x}{y} | @{y})) | P}} & \nonumber\\
	\red
	& \ldots & \nonumber\\
	\red^*
	& P | P | \ldots & \nonumber
\end{eqnarray}

Of course, this encoding, as an implementation, runs away, unfolding
$\bangp{P}$ eagerly. A lazier and more implementable replication
operator, restricted to input-guarded processes, may be obtained as follows.

\begin{eqnarray}
\bangp{\prefix{u}{v}{P}} 
	:= 
	\binpar{\lift{x}{\prefix{u}{v}{(\binpar{D(x)}{P})}}}{D(x)} \nonumber
\end{eqnarray}

\begin{remark}
  Note that the lazier definition still does not deal with summation
  or mixed summation (i.e. sums over input and output). The reader is
  invited to construct definitions of replication that deal with these
  features. 

  Further, the definitions are parameterized in a name, $x$. Can you,
  gentle reader, make a definition that eliminates this parameter and
  guarantees no accidental interaction between the replication
  machinery and the process being replicated -- i.e. no accidental
  sharing of names used by the process to get its work done and the
  name(s) used by the replication to effect copying. This latter
  revision of the definition of replication is crucial to obtaining
  the expected identity $!!P \sim !P$.
\end{remark}

\begin{remark}\label{rem:paradoxical_combinator}
  The reader familiar with the lambda calculus will have noticed the
  similarity between $D$ and the paradoxical combinator.

  [Ed. note: the existence of this seems to suggest we have to be more
  restrictive on the set of processes and names we admit if we are to
  support no-cloning.]
\end{remark}

\subsubsection{Bisimulation}

The computational dynamics gives rise to another kind of equivalence,
the equivalence of computational behavior. As previously mentioned
this is typically captured \emph{via} some form of bisimulation.

% The notion we use in this paper is weak barbed bisimulation
% \cite{milner91polyadicpi}.

The notion we use in this paper is derived from weak barbed
bisimulation \cite{milner91polyadicpi}. 

\begin{definition}
An \emph{observation relation}, $\downarrow_{\mathcal N}$, over a set
of names, $\mathcal N$, is the smallest relation satisfying the rules
below.

\infrule[Out-barb]{y \in {\mathcal N}, \; x \nameeq y}
		  {\outputp{x}{v} \downarrow_{\mathcal N} x}
\infrule[Par-barb]{\mbox{$P\downarrow_{\mathcal N} x$ or $Q\downarrow_{\mathcal N} x$}}
		  {\binpar{P}{Q} \downarrow_{\mathcal N} x}

We write $P \Downarrow_{\mathcal N} x$ if there is $Q$ such that 
$P \wred Q$ and $Q \downarrow_{\mathcal N} x$.
\end{definition}

\begin{definition}
%\label{def.bbisim}
An  ${\mathcal N}$-\emph{barbed bisimulation} over a set of names, ${\mathcal N}$, is a symmetric binary relation 
${\mathcal S}_{\mathcal N}$ between agents such that $P\rel{S}_{\mathcal N}Q$ implies:
\begin{enumerate}
\item If $P \red P'$ then $Q \wred Q'$ and $P'\rel{S}_{\mathcal N} Q'$.
\item If $P\downarrow_{\mathcal N} x$, then $Q\Downarrow_{\mathcal N} x$.
\end{enumerate}
$P$ is ${\mathcal N}$-barbed bisimilar to $Q$, written
$P \wbbisim_{\mathcal N} Q$, if $P \rel{S}_{\mathcal N} Q$ for some ${\mathcal N}$-barbed bisimulation ${\mathcal S}_{\mathcal N}$.
\end{definition}

$\mathcal{R} \subseteq \pi \times \pi$

$P \mathcal{R} Q => \forall P'. P \red P' \Rightarrow \exists Q'. Q \red Q', P' \mathcal{R} Q'$

$P \vdash x \Rightarrow Q \vdash x$

\begin{mathpar}
  \inferrule*[lab=Out-barb]{x \nameeq y}{{y}!\langle{Q}\rangle \vdash x}
  \and
  \inferrule*[lab=Par-barb]{\mbox{$P\vdash x$ or $Q\vdash x$}}{\binpar{P}{Q} \vdash x}
\end{mathpar}

\subsubsection{Contexts}

One of the principle advantages of computational calculi like the
$\pi$-calculus is a well-defined notion of context,
contextual-equivalence and a correlation between
contextual-equivalence and notions of bisimulation. The notion of
context allows the decomposition of a process into (sub-)process and
its syntactic environment, its context. Thus, a context may be
thought of as a process with a ``hole'' (written $\Box$) in it. The
application of a context $M$ to a process $P$, written $M[P]$, is
tantamount to filling the hole in $M$ with $P$. In this paper we do
not need the full weight of this theory, but do make use of the notion
of context in the proof the main theorem. 

\begin{mathpar}
  \inferrule* [lab=summation] {} {{M_{M},M_{N}} \bc \Box \;|\; x.M_{A} \;|\; M_{M}+M_{N}}
  \and
  \inferrule* [lab=agent] {} {{M_{A}} \bc (\vec{x})M_{P} \;| \; \clift{P_0,\ldots,M_{P},\ldots,P_N}}
  \and \\
  \inferrule* [lab=process] {} {{M_{P}} \bc M_{N} \;| \;P|M_{P} }
\end{mathpar} 

\begin{mathpar}
  \inferrule* [lab=sychronization] {} {M_{N} \bc \Box \;|\; x?M_{F} \;|\; x!M_{C}}
  \and
  \inferrule* [lab=abstraction] {} {{M_{F}} \bc (x)M_{P} }
  \and
  \inferrule* [lab=concretion] {} {{M_{C}} \bc \langle M_{P} \rangle }
  \and \\
  \inferrule* [lab=process] {} {{M_{P}} \bc M_{N} \;| \;P|M_{P} }
\end{mathpar}

\begin{definition}[contextual application] Given a context $M$, and
  process $P$, we define the \emph{contextual application}, $M[P] :=
  M\{P/\Box\}$. That is, the contextual application of M to P is the
  substitution of $P$ for $\Box$ in $M$.
\end{definition}

$\meaningof{-} : L \to \mathcal{P}(\pi)$

\begin{mathpar}
  \inferrule* [lab=collection] {} {\meaningof{true} = \pi, \and \meaningof{~E} = \pi \setminus \meaningof{E}, \and \meaningof{E_{1} \& E_{2}} = \meaningof{E_{1}} \cap \meaningof{E_{2}}}
\end{mathpar}

\begin{mathpar}
  \inferrule* [lab=structure] {} {\meaningof{0} = \{ P \in \pi | P \equiv 0 \}, \and \\ \meaningof{E_1 | E_2} = \{ P \in \pi | P \equiv P_{1} | P_{2}, P_{1} \in \meaningof{E_{1}}, P_{2} \in \meaningof{E_2}\} }
\end{mathpar}

\begin{mathpar}
 \inferrule* [lab=behavior] {} {\meaningof{\langle a?b \rangle E} = \{ P \in \pi | P \equiv Q | u?(y)P', \\ \and \\\\ \and \\ \;\;\; u \in \meaningof{a}, \forall z.P'\{z/y\} \in \meaningof{E\{z/b\}}\}, \and \\ \meaningof{a!E} = \{ P \in \pi | P \equiv Q | x!\langle P' \rangle, x \in \meaningof{a} P' \in \meaningof{E}\} }
\end{mathpar}

\begin{mathpar}
 \inferrule* [lab=nominal] {} {\meaningof{\quotep{E}} = \{ \quotep{P} \in \quotep{\pi} | P \in \meaningof{E} \}, \and \meaningof{\quotep{P}} = \{ \quotep{Q} \in \quotep{\pi} | P \equiv Q \} \and \\ \meaningof{@\quotep{E}} = \{ P \in \pi | P \equiv @x, x \in \meaningof{E} \}}
\end{mathpar}

\begin{eqnarray*}
  \\
  \meaningof{-} : TS \to ST
\end{eqnarray*}

\begin{eqnarray*}
  \\
  L : TS \to ST
\end{eqnarray*}

\begin{eqnarray*}
  \\
  P \models E \iff P \in \meaningof{E}
\end{eqnarray*}

\begin{eqnarray*}
  P \approx_{L} Q \iff \forall E \in L. P \models E \iff Q \models E
\end{eqnarray*}

\begin{eqnarray*}
  P \approx_{K} Q
\end{eqnarray*}

\begin{eqnarray*}
  P \approx Q
\end{eqnarray*}

$\approx_{K} = \approx = \approx_{L}$

\subsubsection{Contextual duality}

Note that contexts extend the quotation operation to a family of
operations from processes to names. Given a context, $M$, we can
define a \emph{nominal context}, $\quotep{M}$ by $\quotep{M}[P] :=
\quotep{M[P]}$. To foreshadow what is to come we observe that these
operations enjoy a duality with processes very much like the duality
between vectors and maps from vectors to scalars.

Further, because the calculus is essentially higher-order, we have a
correspondence between contexts and processes. More specifically,
given a name $x$ and a context $M$ we can construct $M^{*}_{x}$ such
that 

\begin{mathpar}
  M^{*}_{x} | \lift{x}{P} \red M[P]
\end{mathpar}

namely,

\begin{mathpar}
  M^{*}_{x} := x?(u).M[\dropn{u}]
\end{mathpar}

The dependence of $M^{*}_{x}$ on a name makes it an abstraction, 

\begin{mathpar}
  M^{*} := (x)x?(u).M[\dropn{u}]
\end{mathpar}

\subsection{Additional notation}

It will sometimes be convenient to denote the process a name
quotes. We already have the notation $x = \quotep{P}$, but it will be
convenient to introduce an alternate notation, $\procn{x}$, when we
want to emphasize the connection to the use of the name. Note that, by
virtue of name equivalence, $\quotep{\procn{x}} \nameeq x$; so, the
notation is consistent with previous definitions.

Further, because names have structure it is possible to effect
substitutions on the basis of that structure. This means we need to
upgrade our notation for substitutions, which we accomplish by
adapting comprehension notation. Thus,

\begin{mathpar}
  P\{ y / x : x \in S \}
\end{mathpar}

is interpreted to mean the process derived from P by replacing (in a
capture-avoiding manner) each occurrence of $x$ in $S$ by $y$. For example,

\begin{mathpar}
  P\{ \quotep{\procn{x}|\procn{x}} / x : x \in \freenames{P} \}
\end{mathpar}

will replace each (occurrence) of a free name $x$ in $P$ by
$\quotep{\procn{x}|\procn{x}}$.

Also, we will avail ourselves of the notation $x^{L}$ and $x^{R}$ to
denote injections of a name into disjoint copies of the name
space. There are numerous ways to accomplish this. One example can be
found in \cite{MeredithR05}. This notation overloads to vectors of
names: $\vec{x}^{\pi} := (x_{i}^{\pi} \; : \; 0 \leq i < |\vec{x}| )$ where $\pi \in \{L,R\}$.

We also use $P^{\Box} := P|\Box$.

In \cite{MeredithR05} an interpretation of the new operator is
given. It turns out that there are several possible interpretations
all enjoying the requisite algebraic properties of the operator (see
\cite{milner91polyadicpi}). We will therefore make liberal use of
$(\nu\; \vec{x})P$.

% subsection the_syntax_and_semantics_of_the_notation_system (end)   

\input{qm2pi.qmops} 

\input{qm2pi.sterngerlach} 

\input{qm2pi.metric} 

% section concurrent_process_calculi (end)

%\input{qm2pi.proofsketch}

% section proof sketch (end)

%\input{qm2pi.slviaknots} 

% section spatial logic via knots (end)

\input{qm2pi.conclusion}

% section conclusion (end)

%\input{qm2pi.dtcodes} 

% section wiring algorithm (end)

\input{qm2pi.ack} 

% section acknowledgments (end)

\newpage


\bibliographystyle{plain}   
\bibliography{../../biblios/main.bib}

\input{qm2pi.rhodetails}

\end{document}

 

% section notation (end)

\input{qm2pi.process.calculi} 

% section concurrent_process_calculi_and_spatial_logics_ (end)
    
%\documentclass[12pt]{llncs}
%\documentclass{jktr}

\usepackage[pdftex]{hyperref}                   
\usepackage {listings}
\usepackage {mathpartir}
\usepackage{bcprules}
%\usepackage{listings}
                       
\usepackage{graphicx} 
%\usepackage[margins=2.5cm,nohead,nofoot]{geometry}
%\usepackage{geometry}
\usepackage{amsfonts}
\usepackage{amstext}
\usepackage{latexsym}
\usepackage{amssymb}
\usepackage{color}


%\include{myPreamble}
\include{qm2pi.local} 

%\ifpdf
%\usepackage[pdftex]{graphicx}
%\else
%\usepackage{graphicx}
%\fi

 % \ifpdf
%  \usepackage{pdfsync}
%  \if


%\title{Brief Article}
%\author{David F. Snyder}
%\author{L.G. Meredith}

%\address{Dept. of Math., Texas State University--San Marcos, San Marcos, TX 78666}
       
\pagestyle{empty}


\begin{document}

\lstset{language=[Objective]Caml,frame=shadowbox}

\input{qm2pi.front}

% section front matter (end)

\input{qm2pi.intro} 
 
% section introduction (end)

% \input{qm2pi.knotations} 

% section notation (end)

\input{qm2pi.process.calculi} 

% section concurrent_process_calculi_and_spatial_logics_ (end)
    
%\input{qm2pi.knots2pi} 

%\input{qm2pi.trefoil} 

%\input{qm2pi.mainthm} 

% subsection basic_interpretation (end)

%\input{qm2pi.rho.presentation} 
\subsection{The syntax and semantics of the notation system}\label{sub:the_syntax_and_semantics_of_the_notation_system} % (fold)

We now summarize a technical presentation of the calculus that
embodies our theory of dynamics. The typical presentation of such a
calculus follows the style of giving generators and relations on
them. The grammar, below, describing term constructors, freely
generates the set of processes, $\Proc$. This set is then quotiented
by a relation known as structural congruence and it is over this set
that the notion of dynamics is expressed. This presentation is
essentially that of \cite{MeredithR05} with the addition of
polyadicity and summation. For readability we have relegated some of
the technical subtleties to an appendix.

\subsubsection{Process grammar}\label{subsub:process_grammar}

\begin{mathpar}
  \inferrule* [lab=synchronization] {} {{M} \bc \pzero \;|\; x?F \;|\; x!C }
  \and
  \inferrule* [lab=abstraction] {} {{F} \bc (x)P}
  \and
  \inferrule* [lab=concretion] {} {{C} \bc \langle Q \rangle}
  \and
  \inferrule* [lab=process] {} {{P,Q} \bc M \;| \;P|Q \;|\; @{x}}
  \and
  \inferrule* [lab=name] {} {{x} \bc \quotep{P}}
\end{mathpar} 

Note that $\vec{x}$ (resp. $\vec{P}$) denotes a vector of names
(resp. processes) of length $|\vec{x}|$ (resp. $|\vec{P}|$). We adopt
the following useful abbreviations.

\begin{mathpar}
   x?(\vec{y}).P := x.(\vec{y})P \and  x\clift{\vec{P}} := x.\clift{\vec{P}}
   \and x!(y) := \lift{x}{\dropn{y}}
   \and \Pi_{i=0}^{n-1}P_i := P_0 | \ldots | P_{n-1}
\end{mathpar}

\subsubsection{Structural congruence}

\paragraph{Free and bound names and alpha-equivalence.} At the
core of structural equivalence is alpha-equivalence which identifies
process that are the same up to a change of variable. Formally, we
recognize the distinction between free and bound names. The free names
of a process, $\freenames{P}$, may be calculated recursively as
follows:

\begin{mathpar}
\freenames{\pzero} := \emptyset
  \and \\
  \freenames{x?(y).P} := \{ x \} \cup (\freenames{P} \setminus \{ y \})
  \and 
  \freenames{x!\langle P \rangle} := \{ x \} \cup \{ P \} 
  \and \\
  \freenames{P|Q} := \freenames{P} \cup \freenames{Q}
  \and \\
  \freenames{@{x}} := \{ x \}
\end{mathpar}

$\pi$
$\quotep{\pi}$

$\freenames{-} : \pi \to \mathcal{P}(\quotep{\pi})$

\begin{eqnarray*}
  \freenames{\pzero} & := & \emptyset \\
  \freenames{x?(y).P} & := & \{ x \} \cup (\freenames{P} \setminus \{ y \}) \\
  \freenames{x!\langle P \rangle} & := & \{ x \} \cup \{ P \} \\
  \freenames{P|Q} & := & \freenames{P} \cup \freenames{Q} \\
  \freenames{\dropn{x}} & := & \{ x \}
\end{eqnarray*}

The bound names of a process, $\boundnames{P}$, are those names occurring in $P$
that are not free. For example, in $x?(y).0$, the name $x$ is free, while $y$ is bound.

\begin{mathpar}
  \inferrule* [lab=monoidal-laws] {} { P|Q \equiv Q|P \and P|0 \equiv P \and P|(Q|R) \equiv (P|Q)|R }
\end{mathpar}

\begin{mathpar}
  \inferrule* [lab=alpha-equivalence] {} { (x)P \equiv (y)P\{y/x\} \and y \not\in \freenames{P} }
\end{mathpar}

\begin{definition}
Then two processes, $P,Q$, are alpha-equivalent if $P = Q\{\vec{y}/\vec{x}\}$ for
some $\vec{x} \in \boundnames{Q},\vec{y} \in \boundnames{P}$, where $Q\{\vec{y}/\vec{x}\}$
denotes the capture-avoiding substitution of $\vec{y}$ for $\vec{x}$ in $Q$.
\end{definition}

\begin{definition}
  The {\em structural congruence} \cite{SangiorgiWalker} , $\equiv$,
  between processes is the least congruence containing
  alpha-equivalence, satisfying the abelian monoid laws
  (associativity, commutativity and $\pzero$ as identity) for parallel
  composition $|$ and for summation $+$.
\end{definition}

\subsection{Name equivalence}

We take name equivalence, written $\nameeq$, to be the smallest
equivalence relation generated by the following rules.

\begin{mathpar}
\inferrule*[lab=Quote-drop]
{ }
{ \quotep{@{x}} \nameeq x }

\inferrule*[lab=Struct-equiv]
{ P \scong Q }
{ \quotep{P} \nameeq \quotep{Q} }
\end{mathpar}

The astute reader will have noticed that the mutual recursion of names
and processes imposes a mutual recursion on alpha-equivalence and
structural equivalence via name-equivalence. Fortunately, all of this
works out pleasantly and we may calculate in the natural way, free of
concern. The reader interested in the details is referred to the
appendix \ref{appendix:rho_details}.

\subsection{Substitution}

We use $\Proc$ for the set of processes, $\QProc$ for the set of
names, and $\id{\{}\vec{y} / \vec{x} \id{\}}$ to denote partial maps,
$s : \QProc \rightarrow \QProc$. A map, $s$ lifts, uniquely, to a map
on process terms, $\widehat{s} : \Proc \rightarrow \Proc$ by the
following equations.

\begin{mathpar}
  (0) \psubstp{Q}{P} := 0 \\
  (R \juxtap S) \psubstp{Q}{P}
  :=    
  (R)\psubstp{Q}{P} \juxtap (S) \psubstp{Q}{P} \\
  (x?(y).R) \psubstp{Q}{P}    
  :=    
  (x)\substp{Q}{P} (z)\concat( (R \psubstn{z}{y}) \psubstp{Q}{P} ) \\
  (\lift{x}{R}) \psubstp{Q}{P}  
  :=
  \lift{(x)\substp{Q}{P}}{ R \psubstp{Q}{P} } \\
%   (\dropn{x})  \psubstp{Q}{P}       
%   := 
%   \left\{ 
%     \begin{array}{ccc} 
%       \dropn{\quotep{Q}} & & x \nameeq \quotep{P} \\
%       \dropn{x} & & otherwise \\
%     \end{array}
%   \right. 
  (\dropn{x})  \psubstp{Q}{P}       
  := 
  \left\{ 
    \begin{array}{ccc} 
      Q & & x \nameeq \quotep{P} \\
      \dropn{x} & & otherwise \\
    \end{array}
  \right.
\end{mathpar}
 

where

\begin{eqnarray}
  (x)\id{\{} \lpquote Q \rpquote / \lpquote P \rpquote \id{\}}            = 
  \left\{ 
    \begin{array}{ccc}
      \lpquote Q \rpquote & & x \nameeq \lpquote P \rpquote \\
      x & & otherwise \\
    \end{array}
  \right. \nonumber
\end{eqnarray}

and $z$ is chosen distinct from $\quotep{P}$, $\quotep{Q}$, the free
names in $Q$, and all the names in $R$. Our $\alpha$-equivalence will
be built in the standard way from this substitution.

\begin{remark}\label{rem:no_self_referential_names}
  One consequence of these definitions is that $\forall P. \quotep{P}
  \not\in \freenames{P}$.
\end{remark}

\subsection{ Dynamic quote: an example }

Anticipating something of what's to come, consider applying the
substitution, $\widehat{\id{\{}u / z \id{\}}}$, to the following pair
of processes, $\lift{w}{y!(z)}$ and $w[ \lpquote y!(z) \rpquote ]$.

\begin{eqnarray}
	\lift{w}{y!(z)}\widehat{\id{\{}u / z \id{\}}}
		& = &
		\lift{w}{y!(u)} \nonumber\\
	w[ \lpquote y!(z) \rpquote ] \widehat{ \id{\{}u / z \id{\}} }
		& = &
		w[ \lpquote y!(z) \rpquote ] \nonumber
\end{eqnarray}

Because the body of the process between quotes is impervious to
substitution, we get radically different answers. In fact, by
examining the first process in an input context,
e.g. $x?(z).\lift{w}{y!(z)}$, we see that the process under the lift
operator may be shaped by prefixed inputs binding a name inside it. In
this sense, the lift operator will be seen as a way to dynamically
construct processes before reifying them as names.

Finally equipped with these standard features we can present the
dynamics of the calculus.

\subsubsection{Operational semantics} 

Finally, we introduce the computational dynamics. What marks these
algebras as distinct from other more traditionally studied algebraic
structures, e.g. vector spaces or polynomial rings, is the manner in
which dynamics is captured. In traditional structures, dynamics is typically
expressed through morphisms between such structures, as in linear maps
between vector spaces or morphisms between rings. In algebras
associated with the semantics of computation, the dynamics is
expressed as part of the algebraic structure itself, through a
reduction reduction relation typically denoted by $\red$. Below, we
give a recursive presentation of this relation for the calculus used
in the encoding.

$\red \subseteq \pi \times \pi$
$\red : \pi \to \mathcal{P}(\pi)$

\begin{mathpar}
  \inferrule* [lab=Comm] { \textsf{match}( x_{src}, x_{trgt} ) } { x_{trgt}?(y)P \; | \; x_{src}!\langle {Q} \rangle \red P\{\quotep{Q}/y}\} }
  \and \\
  \inferrule* [lab=Par] {{P} \red {P}'} {{{P} | {Q}} \red {{P}' | {Q}}}
  \and
  \inferrule* [lab=Equiv]{{{P} \scong {P}'} \andalso {{P}' \red {Q}'} \andalso {{Q}' \scong {Q}}}{{P} \red {Q}}
\end{mathpar}

\begin{eqnarray*}
  match_{\equiv} (\quotep{P},\quotep{Q}) & := & P \equiv Q \\
  match_{\dagger}(\quotep{P},\quotep{Q}) & := & \forall R. P|Q \red^{*} R => R \red^{*} 0 \\
  match_{K}(\quotep{P},\quotep{Q}) & := & K \mbox{ for some context } K
\end{eqnarray*}

$u?(x)P | u!\langle Q \rangle \red P\{\quotep{Q}/x\}$

%We write $\wred$ for $\red^*$, and $P\red$ if $\exists Q $ such that $ P \red Q$.
We write $P\red$ if $\exists Q $ such that $ P \red Q$ and $P\not\red$, otherwise.

\section{Replication}

As mentioned before, it is known that replication (and hence
recursion) can be implemented in a higher-order process algebra
\cite{SangiorgiWalker}. As our first example of calculation with the
machinery thus far presented we give the construction explicitly in
the {\rhoc}.

\begin{eqnarray}
	D_{x} & := & \prefix{x}{y}{(\binpar{\outputp{x}{y}}{@{y}})} \nonumber\\
	\bangp_{x}{P} & := & \binpar{{x}!\langle{\binpar{D_{x}}{P}}\rangle}{D_{x}} \nonumber
\end{eqnarray}

\begin{eqnarray}
	\bangp_{x}{P} & & \nonumber\\
	=
	& {x}!\langle{(\prefix{x}{y}{(\outputp{x}{y} | @{y})) | P}}\rangle 
	      | \prefix{x}{y}{(\outputp{x}{y} | @{y})} & \nonumber\\
	\red
	& (\outputp{x}{y} | @{y})\substn{\quotep{(\prefix{x}{y}{(@{y} | \outputp{x}{y})) | P}}}{y} & \nonumber\\
	=
	& \outputp{x}{\quotep{(\prefix{x}{y}{(\outputp{x}{y} | @{y})) | P}}}
	  | {(\prefix{x}{y}{(\outputp{x}{y} | @{y})) | P}} & \nonumber\\
	\red
	& \ldots & \nonumber\\
	\red^*
	& P | P | \ldots & \nonumber
\end{eqnarray}

Of course, this encoding, as an implementation, runs away, unfolding
$\bangp{P}$ eagerly. A lazier and more implementable replication
operator, restricted to input-guarded processes, may be obtained as follows.

\begin{eqnarray}
\bangp{\prefix{u}{v}{P}} 
	:= 
	\binpar{\lift{x}{\prefix{u}{v}{(\binpar{D(x)}{P})}}}{D(x)} \nonumber
\end{eqnarray}

\begin{remark}
  Note that the lazier definition still does not deal with summation
  or mixed summation (i.e. sums over input and output). The reader is
  invited to construct definitions of replication that deal with these
  features. 

  Further, the definitions are parameterized in a name, $x$. Can you,
  gentle reader, make a definition that eliminates this parameter and
  guarantees no accidental interaction between the replication
  machinery and the process being replicated -- i.e. no accidental
  sharing of names used by the process to get its work done and the
  name(s) used by the replication to effect copying. This latter
  revision of the definition of replication is crucial to obtaining
  the expected identity $!!P \sim !P$.
\end{remark}

\begin{remark}\label{rem:paradoxical_combinator}
  The reader familiar with the lambda calculus will have noticed the
  similarity between $D$ and the paradoxical combinator.

  [Ed. note: the existence of this seems to suggest we have to be more
  restrictive on the set of processes and names we admit if we are to
  support no-cloning.]
\end{remark}

\subsubsection{Bisimulation}

The computational dynamics gives rise to another kind of equivalence,
the equivalence of computational behavior. As previously mentioned
this is typically captured \emph{via} some form of bisimulation.

% The notion we use in this paper is weak barbed bisimulation
% \cite{milner91polyadicpi}.

The notion we use in this paper is derived from weak barbed
bisimulation \cite{milner91polyadicpi}. 

\begin{definition}
An \emph{observation relation}, $\downarrow_{\mathcal N}$, over a set
of names, $\mathcal N$, is the smallest relation satisfying the rules
below.

\infrule[Out-barb]{y \in {\mathcal N}, \; x \nameeq y}
		  {\outputp{x}{v} \downarrow_{\mathcal N} x}
\infrule[Par-barb]{\mbox{$P\downarrow_{\mathcal N} x$ or $Q\downarrow_{\mathcal N} x$}}
		  {\binpar{P}{Q} \downarrow_{\mathcal N} x}

We write $P \Downarrow_{\mathcal N} x$ if there is $Q$ such that 
$P \wred Q$ and $Q \downarrow_{\mathcal N} x$.
\end{definition}

\begin{definition}
%\label{def.bbisim}
An  ${\mathcal N}$-\emph{barbed bisimulation} over a set of names, ${\mathcal N}$, is a symmetric binary relation 
${\mathcal S}_{\mathcal N}$ between agents such that $P\rel{S}_{\mathcal N}Q$ implies:
\begin{enumerate}
\item If $P \red P'$ then $Q \wred Q'$ and $P'\rel{S}_{\mathcal N} Q'$.
\item If $P\downarrow_{\mathcal N} x$, then $Q\Downarrow_{\mathcal N} x$.
\end{enumerate}
$P$ is ${\mathcal N}$-barbed bisimilar to $Q$, written
$P \wbbisim_{\mathcal N} Q$, if $P \rel{S}_{\mathcal N} Q$ for some ${\mathcal N}$-barbed bisimulation ${\mathcal S}_{\mathcal N}$.
\end{definition}

$\mathcal{R} \subseteq \pi \times \pi$

$P \mathcal{R} Q => \forall P'. P \red P' \Rightarrow \exists Q'. Q \red Q', P' \mathcal{R} Q'$

$P \vdash x \Rightarrow Q \vdash x$

\begin{mathpar}
  \inferrule*[lab=Out-barb]{x \nameeq y}{{y}!\langle{Q}\rangle \vdash x}
  \and
  \inferrule*[lab=Par-barb]{\mbox{$P\vdash x$ or $Q\vdash x$}}{\binpar{P}{Q} \vdash x}
\end{mathpar}

\subsubsection{Contexts}

One of the principle advantages of computational calculi like the
$\pi$-calculus is a well-defined notion of context,
contextual-equivalence and a correlation between
contextual-equivalence and notions of bisimulation. The notion of
context allows the decomposition of a process into (sub-)process and
its syntactic environment, its context. Thus, a context may be
thought of as a process with a ``hole'' (written $\Box$) in it. The
application of a context $M$ to a process $P$, written $M[P]$, is
tantamount to filling the hole in $M$ with $P$. In this paper we do
not need the full weight of this theory, but do make use of the notion
of context in the proof the main theorem. 

\begin{mathpar}
  \inferrule* [lab=summation] {} {{M_{M},M_{N}} \bc \Box \;|\; x.M_{A} \;|\; M_{M}+M_{N}}
  \and
  \inferrule* [lab=agent] {} {{M_{A}} \bc (\vec{x})M_{P} \;| \; \clift{P_0,\ldots,M_{P},\ldots,P_N}}
  \and \\
  \inferrule* [lab=process] {} {{M_{P}} \bc M_{N} \;| \;P|M_{P} }
\end{mathpar} 

\begin{mathpar}
  \inferrule* [lab=sychronization] {} {M_{N} \bc \Box \;|\; x?M_{F} \;|\; x!M_{C}}
  \and
  \inferrule* [lab=abstraction] {} {{M_{F}} \bc (x)M_{P} }
  \and
  \inferrule* [lab=concretion] {} {{M_{C}} \bc \langle M_{P} \rangle }
  \and \\
  \inferrule* [lab=process] {} {{M_{P}} \bc M_{N} \;| \;P|M_{P} }
\end{mathpar}

\begin{definition}[contextual application] Given a context $M$, and
  process $P$, we define the \emph{contextual application}, $M[P] :=
  M\{P/\Box\}$. That is, the contextual application of M to P is the
  substitution of $P$ for $\Box$ in $M$.
\end{definition}

$\meaningof{-} : L \to \mathcal{P}(\pi)$

\begin{mathpar}
  \inferrule* [lab=collection] {} {\meaningof{true} = \pi, \and \meaningof{~E} = \pi \setminus \meaningof{E}, \and \meaningof{E_{1} \& E_{2}} = \meaningof{E_{1}} \cap \meaningof{E_{2}}}
\end{mathpar}

\begin{mathpar}
  \inferrule* [lab=structure] {} {\meaningof{0} = \{ P \in \pi | P \equiv 0 \}, \and \\ \meaningof{E_1 | E_2} = \{ P \in \pi | P \equiv P_{1} | P_{2}, P_{1} \in \meaningof{E_{1}}, P_{2} \in \meaningof{E_2}\} }
\end{mathpar}

\begin{mathpar}
 \inferrule* [lab=behavior] {} {\meaningof{\langle a?b \rangle E} = \{ P \in \pi | P \equiv Q | u?(y)P', \\ \and \\\\ \and \\ \;\;\; u \in \meaningof{a}, \forall z.P'\{z/y\} \in \meaningof{E\{z/b\}}\}, \and \\ \meaningof{a!E} = \{ P \in \pi | P \equiv Q | x!\langle P' \rangle, x \in \meaningof{a} P' \in \meaningof{E}\} }
\end{mathpar}

\begin{mathpar}
 \inferrule* [lab=nominal] {} {\meaningof{\quotep{E}} = \{ \quotep{P} \in \quotep{\pi} | P \in \meaningof{E} \}, \and \meaningof{\quotep{P}} = \{ \quotep{Q} \in \quotep{\pi} | P \equiv Q \} \and \\ \meaningof{@\quotep{E}} = \{ P \in \pi | P \equiv @x, x \in \meaningof{E} \}}
\end{mathpar}

\begin{eqnarray*}
  \\
  \meaningof{-} : TS \to ST
\end{eqnarray*}

\begin{eqnarray*}
  \\
  L : TS \to ST
\end{eqnarray*}

\begin{eqnarray*}
  \\
  P \models E \iff P \in \meaningof{E}
\end{eqnarray*}

\begin{eqnarray*}
  P \approx_{L} Q \iff \forall E \in L. P \models E \iff Q \models E
\end{eqnarray*}

\begin{eqnarray*}
  P \approx_{K} Q
\end{eqnarray*}

\begin{eqnarray*}
  P \approx Q
\end{eqnarray*}

$\approx_{K} = \approx = \approx_{L}$

\subsubsection{Contextual duality}

Note that contexts extend the quotation operation to a family of
operations from processes to names. Given a context, $M$, we can
define a \emph{nominal context}, $\quotep{M}$ by $\quotep{M}[P] :=
\quotep{M[P]}$. To foreshadow what is to come we observe that these
operations enjoy a duality with processes very much like the duality
between vectors and maps from vectors to scalars.

Further, because the calculus is essentially higher-order, we have a
correspondence between contexts and processes. More specifically,
given a name $x$ and a context $M$ we can construct $M^{*}_{x}$ such
that 

\begin{mathpar}
  M^{*}_{x} | \lift{x}{P} \red M[P]
\end{mathpar}

namely,

\begin{mathpar}
  M^{*}_{x} := x?(u).M[\dropn{u}]
\end{mathpar}

The dependence of $M^{*}_{x}$ on a name makes it an abstraction, 

\begin{mathpar}
  M^{*} := (x)x?(u).M[\dropn{u}]
\end{mathpar}

\subsection{Additional notation}

It will sometimes be convenient to denote the process a name
quotes. We already have the notation $x = \quotep{P}$, but it will be
convenient to introduce an alternate notation, $\procn{x}$, when we
want to emphasize the connection to the use of the name. Note that, by
virtue of name equivalence, $\quotep{\procn{x}} \nameeq x$; so, the
notation is consistent with previous definitions.

Further, because names have structure it is possible to effect
substitutions on the basis of that structure. This means we need to
upgrade our notation for substitutions, which we accomplish by
adapting comprehension notation. Thus,

\begin{mathpar}
  P\{ y / x : x \in S \}
\end{mathpar}

is interpreted to mean the process derived from P by replacing (in a
capture-avoiding manner) each occurrence of $x$ in $S$ by $y$. For example,

\begin{mathpar}
  P\{ \quotep{\procn{x}|\procn{x}} / x : x \in \freenames{P} \}
\end{mathpar}

will replace each (occurrence) of a free name $x$ in $P$ by
$\quotep{\procn{x}|\procn{x}}$.

Also, we will avail ourselves of the notation $x^{L}$ and $x^{R}$ to
denote injections of a name into disjoint copies of the name
space. There are numerous ways to accomplish this. One example can be
found in \cite{MeredithR05}. This notation overloads to vectors of
names: $\vec{x}^{\pi} := (x_{i}^{\pi} \; : \; 0 \leq i < |\vec{x}| )$ where $\pi \in \{L,R\}$.

We also use $P^{\Box} := P|\Box$.

In \cite{MeredithR05} an interpretation of the new operator is
given. It turns out that there are several possible interpretations
all enjoying the requisite algebraic properties of the operator (see
\cite{milner91polyadicpi}). We will therefore make liberal use of
$(\nu\; \vec{x})P$.

% subsection the_syntax_and_semantics_of_the_notation_system (end)   

\input{qm2pi.qmops} 

\input{qm2pi.sterngerlach} 

\input{qm2pi.metric} 

% section concurrent_process_calculi (end)

%\input{qm2pi.proofsketch}

% section proof sketch (end)

%\input{qm2pi.slviaknots} 

% section spatial logic via knots (end)

\input{qm2pi.conclusion}

% section conclusion (end)

%\input{qm2pi.dtcodes} 

% section wiring algorithm (end)

\input{qm2pi.ack} 

% section acknowledgments (end)

\newpage


\bibliographystyle{plain}   
\bibliography{../../biblios/main.bib}

\input{qm2pi.rhodetails}

\end{document}

 

%\documentclass[12pt]{llncs}
%\documentclass{jktr}

\usepackage[pdftex]{hyperref}                   
\usepackage {listings}
\usepackage {mathpartir}
\usepackage{bcprules}
%\usepackage{listings}
                       
\usepackage{graphicx} 
%\usepackage[margins=2.5cm,nohead,nofoot]{geometry}
%\usepackage{geometry}
\usepackage{amsfonts}
\usepackage{amstext}
\usepackage{latexsym}
\usepackage{amssymb}
\usepackage{color}


%\include{myPreamble}
\include{qm2pi.local} 

%\ifpdf
%\usepackage[pdftex]{graphicx}
%\else
%\usepackage{graphicx}
%\fi

 % \ifpdf
%  \usepackage{pdfsync}
%  \if


%\title{Brief Article}
%\author{David F. Snyder}
%\author{L.G. Meredith}

%\address{Dept. of Math., Texas State University--San Marcos, San Marcos, TX 78666}
       
\pagestyle{empty}


\begin{document}

\lstset{language=[Objective]Caml,frame=shadowbox}

\input{qm2pi.front}

% section front matter (end)

\input{qm2pi.intro} 
 
% section introduction (end)

% \input{qm2pi.knotations} 

% section notation (end)

\input{qm2pi.process.calculi} 

% section concurrent_process_calculi_and_spatial_logics_ (end)
    
%\input{qm2pi.knots2pi} 

%\input{qm2pi.trefoil} 

%\input{qm2pi.mainthm} 

% subsection basic_interpretation (end)

%\input{qm2pi.rho.presentation} 
\subsection{The syntax and semantics of the notation system}\label{sub:the_syntax_and_semantics_of_the_notation_system} % (fold)

We now summarize a technical presentation of the calculus that
embodies our theory of dynamics. The typical presentation of such a
calculus follows the style of giving generators and relations on
them. The grammar, below, describing term constructors, freely
generates the set of processes, $\Proc$. This set is then quotiented
by a relation known as structural congruence and it is over this set
that the notion of dynamics is expressed. This presentation is
essentially that of \cite{MeredithR05} with the addition of
polyadicity and summation. For readability we have relegated some of
the technical subtleties to an appendix.

\subsubsection{Process grammar}\label{subsub:process_grammar}

\begin{mathpar}
  \inferrule* [lab=synchronization] {} {{M} \bc \pzero \;|\; x?F \;|\; x!C }
  \and
  \inferrule* [lab=abstraction] {} {{F} \bc (x)P}
  \and
  \inferrule* [lab=concretion] {} {{C} \bc \langle Q \rangle}
  \and
  \inferrule* [lab=process] {} {{P,Q} \bc M \;| \;P|Q \;|\; @{x}}
  \and
  \inferrule* [lab=name] {} {{x} \bc \quotep{P}}
\end{mathpar} 

Note that $\vec{x}$ (resp. $\vec{P}$) denotes a vector of names
(resp. processes) of length $|\vec{x}|$ (resp. $|\vec{P}|$). We adopt
the following useful abbreviations.

\begin{mathpar}
   x?(\vec{y}).P := x.(\vec{y})P \and  x\clift{\vec{P}} := x.\clift{\vec{P}}
   \and x!(y) := \lift{x}{\dropn{y}}
   \and \Pi_{i=0}^{n-1}P_i := P_0 | \ldots | P_{n-1}
\end{mathpar}

\subsubsection{Structural congruence}

\paragraph{Free and bound names and alpha-equivalence.} At the
core of structural equivalence is alpha-equivalence which identifies
process that are the same up to a change of variable. Formally, we
recognize the distinction between free and bound names. The free names
of a process, $\freenames{P}$, may be calculated recursively as
follows:

\begin{mathpar}
\freenames{\pzero} := \emptyset
  \and \\
  \freenames{x?(y).P} := \{ x \} \cup (\freenames{P} \setminus \{ y \})
  \and 
  \freenames{x!\langle P \rangle} := \{ x \} \cup \{ P \} 
  \and \\
  \freenames{P|Q} := \freenames{P} \cup \freenames{Q}
  \and \\
  \freenames{@{x}} := \{ x \}
\end{mathpar}

$\pi$
$\quotep{\pi}$

$\freenames{-} : \pi \to \mathcal{P}(\quotep{\pi})$

\begin{eqnarray*}
  \freenames{\pzero} & := & \emptyset \\
  \freenames{x?(y).P} & := & \{ x \} \cup (\freenames{P} \setminus \{ y \}) \\
  \freenames{x!\langle P \rangle} & := & \{ x \} \cup \{ P \} \\
  \freenames{P|Q} & := & \freenames{P} \cup \freenames{Q} \\
  \freenames{\dropn{x}} & := & \{ x \}
\end{eqnarray*}

The bound names of a process, $\boundnames{P}$, are those names occurring in $P$
that are not free. For example, in $x?(y).0$, the name $x$ is free, while $y$ is bound.

\begin{mathpar}
  \inferrule* [lab=monoidal-laws] {} { P|Q \equiv Q|P \and P|0 \equiv P \and P|(Q|R) \equiv (P|Q)|R }
\end{mathpar}

\begin{mathpar}
  \inferrule* [lab=alpha-equivalence] {} { (x)P \equiv (y)P\{y/x\} \and y \not\in \freenames{P} }
\end{mathpar}

\begin{definition}
Then two processes, $P,Q$, are alpha-equivalent if $P = Q\{\vec{y}/\vec{x}\}$ for
some $\vec{x} \in \boundnames{Q},\vec{y} \in \boundnames{P}$, where $Q\{\vec{y}/\vec{x}\}$
denotes the capture-avoiding substitution of $\vec{y}$ for $\vec{x}$ in $Q$.
\end{definition}

\begin{definition}
  The {\em structural congruence} \cite{SangiorgiWalker} , $\equiv$,
  between processes is the least congruence containing
  alpha-equivalence, satisfying the abelian monoid laws
  (associativity, commutativity and $\pzero$ as identity) for parallel
  composition $|$ and for summation $+$.
\end{definition}

\subsection{Name equivalence}

We take name equivalence, written $\nameeq$, to be the smallest
equivalence relation generated by the following rules.

\begin{mathpar}
\inferrule*[lab=Quote-drop]
{ }
{ \quotep{@{x}} \nameeq x }

\inferrule*[lab=Struct-equiv]
{ P \scong Q }
{ \quotep{P} \nameeq \quotep{Q} }
\end{mathpar}

The astute reader will have noticed that the mutual recursion of names
and processes imposes a mutual recursion on alpha-equivalence and
structural equivalence via name-equivalence. Fortunately, all of this
works out pleasantly and we may calculate in the natural way, free of
concern. The reader interested in the details is referred to the
appendix \ref{appendix:rho_details}.

\subsection{Substitution}

We use $\Proc$ for the set of processes, $\QProc$ for the set of
names, and $\id{\{}\vec{y} / \vec{x} \id{\}}$ to denote partial maps,
$s : \QProc \rightarrow \QProc$. A map, $s$ lifts, uniquely, to a map
on process terms, $\widehat{s} : \Proc \rightarrow \Proc$ by the
following equations.

\begin{mathpar}
  (0) \psubstp{Q}{P} := 0 \\
  (R \juxtap S) \psubstp{Q}{P}
  :=    
  (R)\psubstp{Q}{P} \juxtap (S) \psubstp{Q}{P} \\
  (x?(y).R) \psubstp{Q}{P}    
  :=    
  (x)\substp{Q}{P} (z)\concat( (R \psubstn{z}{y}) \psubstp{Q}{P} ) \\
  (\lift{x}{R}) \psubstp{Q}{P}  
  :=
  \lift{(x)\substp{Q}{P}}{ R \psubstp{Q}{P} } \\
%   (\dropn{x})  \psubstp{Q}{P}       
%   := 
%   \left\{ 
%     \begin{array}{ccc} 
%       \dropn{\quotep{Q}} & & x \nameeq \quotep{P} \\
%       \dropn{x} & & otherwise \\
%     \end{array}
%   \right. 
  (\dropn{x})  \psubstp{Q}{P}       
  := 
  \left\{ 
    \begin{array}{ccc} 
      Q & & x \nameeq \quotep{P} \\
      \dropn{x} & & otherwise \\
    \end{array}
  \right.
\end{mathpar}
 

where

\begin{eqnarray}
  (x)\id{\{} \lpquote Q \rpquote / \lpquote P \rpquote \id{\}}            = 
  \left\{ 
    \begin{array}{ccc}
      \lpquote Q \rpquote & & x \nameeq \lpquote P \rpquote \\
      x & & otherwise \\
    \end{array}
  \right. \nonumber
\end{eqnarray}

and $z$ is chosen distinct from $\quotep{P}$, $\quotep{Q}$, the free
names in $Q$, and all the names in $R$. Our $\alpha$-equivalence will
be built in the standard way from this substitution.

\begin{remark}\label{rem:no_self_referential_names}
  One consequence of these definitions is that $\forall P. \quotep{P}
  \not\in \freenames{P}$.
\end{remark}

\subsection{ Dynamic quote: an example }

Anticipating something of what's to come, consider applying the
substitution, $\widehat{\id{\{}u / z \id{\}}}$, to the following pair
of processes, $\lift{w}{y!(z)}$ and $w[ \lpquote y!(z) \rpquote ]$.

\begin{eqnarray}
	\lift{w}{y!(z)}\widehat{\id{\{}u / z \id{\}}}
		& = &
		\lift{w}{y!(u)} \nonumber\\
	w[ \lpquote y!(z) \rpquote ] \widehat{ \id{\{}u / z \id{\}} }
		& = &
		w[ \lpquote y!(z) \rpquote ] \nonumber
\end{eqnarray}

Because the body of the process between quotes is impervious to
substitution, we get radically different answers. In fact, by
examining the first process in an input context,
e.g. $x?(z).\lift{w}{y!(z)}$, we see that the process under the lift
operator may be shaped by prefixed inputs binding a name inside it. In
this sense, the lift operator will be seen as a way to dynamically
construct processes before reifying them as names.

Finally equipped with these standard features we can present the
dynamics of the calculus.

\subsubsection{Operational semantics} 

Finally, we introduce the computational dynamics. What marks these
algebras as distinct from other more traditionally studied algebraic
structures, e.g. vector spaces or polynomial rings, is the manner in
which dynamics is captured. In traditional structures, dynamics is typically
expressed through morphisms between such structures, as in linear maps
between vector spaces or morphisms between rings. In algebras
associated with the semantics of computation, the dynamics is
expressed as part of the algebraic structure itself, through a
reduction reduction relation typically denoted by $\red$. Below, we
give a recursive presentation of this relation for the calculus used
in the encoding.

$\red \subseteq \pi \times \pi$
$\red : \pi \to \mathcal{P}(\pi)$

\begin{mathpar}
  \inferrule* [lab=Comm] { \textsf{match}( x_{src}, x_{trgt} ) } { x_{trgt}?(y)P \; | \; x_{src}!\langle {Q} \rangle \red P\{\quotep{Q}/y}\} }
  \and \\
  \inferrule* [lab=Par] {{P} \red {P}'} {{{P} | {Q}} \red {{P}' | {Q}}}
  \and
  \inferrule* [lab=Equiv]{{{P} \scong {P}'} \andalso {{P}' \red {Q}'} \andalso {{Q}' \scong {Q}}}{{P} \red {Q}}
\end{mathpar}

\begin{eqnarray*}
  match_{\equiv} (\quotep{P},\quotep{Q}) & := & P \equiv Q \\
  match_{\dagger}(\quotep{P},\quotep{Q}) & := & \forall R. P|Q \red^{*} R => R \red^{*} 0 \\
  match_{K}(\quotep{P},\quotep{Q}) & := & K \mbox{ for some context } K
\end{eqnarray*}

$u?(x)P | u!\langle Q \rangle \red P\{\quotep{Q}/x\}$

%We write $\wred$ for $\red^*$, and $P\red$ if $\exists Q $ such that $ P \red Q$.
We write $P\red$ if $\exists Q $ such that $ P \red Q$ and $P\not\red$, otherwise.

\section{Replication}

As mentioned before, it is known that replication (and hence
recursion) can be implemented in a higher-order process algebra
\cite{SangiorgiWalker}. As our first example of calculation with the
machinery thus far presented we give the construction explicitly in
the {\rhoc}.

\begin{eqnarray}
	D_{x} & := & \prefix{x}{y}{(\binpar{\outputp{x}{y}}{@{y}})} \nonumber\\
	\bangp_{x}{P} & := & \binpar{{x}!\langle{\binpar{D_{x}}{P}}\rangle}{D_{x}} \nonumber
\end{eqnarray}

\begin{eqnarray}
	\bangp_{x}{P} & & \nonumber\\
	=
	& {x}!\langle{(\prefix{x}{y}{(\outputp{x}{y} | @{y})) | P}}\rangle 
	      | \prefix{x}{y}{(\outputp{x}{y} | @{y})} & \nonumber\\
	\red
	& (\outputp{x}{y} | @{y})\substn{\quotep{(\prefix{x}{y}{(@{y} | \outputp{x}{y})) | P}}}{y} & \nonumber\\
	=
	& \outputp{x}{\quotep{(\prefix{x}{y}{(\outputp{x}{y} | @{y})) | P}}}
	  | {(\prefix{x}{y}{(\outputp{x}{y} | @{y})) | P}} & \nonumber\\
	\red
	& \ldots & \nonumber\\
	\red^*
	& P | P | \ldots & \nonumber
\end{eqnarray}

Of course, this encoding, as an implementation, runs away, unfolding
$\bangp{P}$ eagerly. A lazier and more implementable replication
operator, restricted to input-guarded processes, may be obtained as follows.

\begin{eqnarray}
\bangp{\prefix{u}{v}{P}} 
	:= 
	\binpar{\lift{x}{\prefix{u}{v}{(\binpar{D(x)}{P})}}}{D(x)} \nonumber
\end{eqnarray}

\begin{remark}
  Note that the lazier definition still does not deal with summation
  or mixed summation (i.e. sums over input and output). The reader is
  invited to construct definitions of replication that deal with these
  features. 

  Further, the definitions are parameterized in a name, $x$. Can you,
  gentle reader, make a definition that eliminates this parameter and
  guarantees no accidental interaction between the replication
  machinery and the process being replicated -- i.e. no accidental
  sharing of names used by the process to get its work done and the
  name(s) used by the replication to effect copying. This latter
  revision of the definition of replication is crucial to obtaining
  the expected identity $!!P \sim !P$.
\end{remark}

\begin{remark}\label{rem:paradoxical_combinator}
  The reader familiar with the lambda calculus will have noticed the
  similarity between $D$ and the paradoxical combinator.

  [Ed. note: the existence of this seems to suggest we have to be more
  restrictive on the set of processes and names we admit if we are to
  support no-cloning.]
\end{remark}

\subsubsection{Bisimulation}

The computational dynamics gives rise to another kind of equivalence,
the equivalence of computational behavior. As previously mentioned
this is typically captured \emph{via} some form of bisimulation.

% The notion we use in this paper is weak barbed bisimulation
% \cite{milner91polyadicpi}.

The notion we use in this paper is derived from weak barbed
bisimulation \cite{milner91polyadicpi}. 

\begin{definition}
An \emph{observation relation}, $\downarrow_{\mathcal N}$, over a set
of names, $\mathcal N$, is the smallest relation satisfying the rules
below.

\infrule[Out-barb]{y \in {\mathcal N}, \; x \nameeq y}
		  {\outputp{x}{v} \downarrow_{\mathcal N} x}
\infrule[Par-barb]{\mbox{$P\downarrow_{\mathcal N} x$ or $Q\downarrow_{\mathcal N} x$}}
		  {\binpar{P}{Q} \downarrow_{\mathcal N} x}

We write $P \Downarrow_{\mathcal N} x$ if there is $Q$ such that 
$P \wred Q$ and $Q \downarrow_{\mathcal N} x$.
\end{definition}

\begin{definition}
%\label{def.bbisim}
An  ${\mathcal N}$-\emph{barbed bisimulation} over a set of names, ${\mathcal N}$, is a symmetric binary relation 
${\mathcal S}_{\mathcal N}$ between agents such that $P\rel{S}_{\mathcal N}Q$ implies:
\begin{enumerate}
\item If $P \red P'$ then $Q \wred Q'$ and $P'\rel{S}_{\mathcal N} Q'$.
\item If $P\downarrow_{\mathcal N} x$, then $Q\Downarrow_{\mathcal N} x$.
\end{enumerate}
$P$ is ${\mathcal N}$-barbed bisimilar to $Q$, written
$P \wbbisim_{\mathcal N} Q$, if $P \rel{S}_{\mathcal N} Q$ for some ${\mathcal N}$-barbed bisimulation ${\mathcal S}_{\mathcal N}$.
\end{definition}

$\mathcal{R} \subseteq \pi \times \pi$

$P \mathcal{R} Q => \forall P'. P \red P' \Rightarrow \exists Q'. Q \red Q', P' \mathcal{R} Q'$

$P \vdash x \Rightarrow Q \vdash x$

\begin{mathpar}
  \inferrule*[lab=Out-barb]{x \nameeq y}{{y}!\langle{Q}\rangle \vdash x}
  \and
  \inferrule*[lab=Par-barb]{\mbox{$P\vdash x$ or $Q\vdash x$}}{\binpar{P}{Q} \vdash x}
\end{mathpar}

\subsubsection{Contexts}

One of the principle advantages of computational calculi like the
$\pi$-calculus is a well-defined notion of context,
contextual-equivalence and a correlation between
contextual-equivalence and notions of bisimulation. The notion of
context allows the decomposition of a process into (sub-)process and
its syntactic environment, its context. Thus, a context may be
thought of as a process with a ``hole'' (written $\Box$) in it. The
application of a context $M$ to a process $P$, written $M[P]$, is
tantamount to filling the hole in $M$ with $P$. In this paper we do
not need the full weight of this theory, but do make use of the notion
of context in the proof the main theorem. 

\begin{mathpar}
  \inferrule* [lab=summation] {} {{M_{M},M_{N}} \bc \Box \;|\; x.M_{A} \;|\; M_{M}+M_{N}}
  \and
  \inferrule* [lab=agent] {} {{M_{A}} \bc (\vec{x})M_{P} \;| \; \clift{P_0,\ldots,M_{P},\ldots,P_N}}
  \and \\
  \inferrule* [lab=process] {} {{M_{P}} \bc M_{N} \;| \;P|M_{P} }
\end{mathpar} 

\begin{mathpar}
  \inferrule* [lab=sychronization] {} {M_{N} \bc \Box \;|\; x?M_{F} \;|\; x!M_{C}}
  \and
  \inferrule* [lab=abstraction] {} {{M_{F}} \bc (x)M_{P} }
  \and
  \inferrule* [lab=concretion] {} {{M_{C}} \bc \langle M_{P} \rangle }
  \and \\
  \inferrule* [lab=process] {} {{M_{P}} \bc M_{N} \;| \;P|M_{P} }
\end{mathpar}

\begin{definition}[contextual application] Given a context $M$, and
  process $P$, we define the \emph{contextual application}, $M[P] :=
  M\{P/\Box\}$. That is, the contextual application of M to P is the
  substitution of $P$ for $\Box$ in $M$.
\end{definition}

$\meaningof{-} : L \to \mathcal{P}(\pi)$

\begin{mathpar}
  \inferrule* [lab=collection] {} {\meaningof{true} = \pi, \and \meaningof{~E} = \pi \setminus \meaningof{E}, \and \meaningof{E_{1} \& E_{2}} = \meaningof{E_{1}} \cap \meaningof{E_{2}}}
\end{mathpar}

\begin{mathpar}
  \inferrule* [lab=structure] {} {\meaningof{0} = \{ P \in \pi | P \equiv 0 \}, \and \\ \meaningof{E_1 | E_2} = \{ P \in \pi | P \equiv P_{1} | P_{2}, P_{1} \in \meaningof{E_{1}}, P_{2} \in \meaningof{E_2}\} }
\end{mathpar}

\begin{mathpar}
 \inferrule* [lab=behavior] {} {\meaningof{\langle a?b \rangle E} = \{ P \in \pi | P \equiv Q | u?(y)P', \\ \and \\\\ \and \\ \;\;\; u \in \meaningof{a}, \forall z.P'\{z/y\} \in \meaningof{E\{z/b\}}\}, \and \\ \meaningof{a!E} = \{ P \in \pi | P \equiv Q | x!\langle P' \rangle, x \in \meaningof{a} P' \in \meaningof{E}\} }
\end{mathpar}

\begin{mathpar}
 \inferrule* [lab=nominal] {} {\meaningof{\quotep{E}} = \{ \quotep{P} \in \quotep{\pi} | P \in \meaningof{E} \}, \and \meaningof{\quotep{P}} = \{ \quotep{Q} \in \quotep{\pi} | P \equiv Q \} \and \\ \meaningof{@\quotep{E}} = \{ P \in \pi | P \equiv @x, x \in \meaningof{E} \}}
\end{mathpar}

\begin{eqnarray*}
  \\
  \meaningof{-} : TS \to ST
\end{eqnarray*}

\begin{eqnarray*}
  \\
  L : TS \to ST
\end{eqnarray*}

\begin{eqnarray*}
  \\
  P \models E \iff P \in \meaningof{E}
\end{eqnarray*}

\begin{eqnarray*}
  P \approx_{L} Q \iff \forall E \in L. P \models E \iff Q \models E
\end{eqnarray*}

\begin{eqnarray*}
  P \approx_{K} Q
\end{eqnarray*}

\begin{eqnarray*}
  P \approx Q
\end{eqnarray*}

$\approx_{K} = \approx = \approx_{L}$

\subsubsection{Contextual duality}

Note that contexts extend the quotation operation to a family of
operations from processes to names. Given a context, $M$, we can
define a \emph{nominal context}, $\quotep{M}$ by $\quotep{M}[P] :=
\quotep{M[P]}$. To foreshadow what is to come we observe that these
operations enjoy a duality with processes very much like the duality
between vectors and maps from vectors to scalars.

Further, because the calculus is essentially higher-order, we have a
correspondence between contexts and processes. More specifically,
given a name $x$ and a context $M$ we can construct $M^{*}_{x}$ such
that 

\begin{mathpar}
  M^{*}_{x} | \lift{x}{P} \red M[P]
\end{mathpar}

namely,

\begin{mathpar}
  M^{*}_{x} := x?(u).M[\dropn{u}]
\end{mathpar}

The dependence of $M^{*}_{x}$ on a name makes it an abstraction, 

\begin{mathpar}
  M^{*} := (x)x?(u).M[\dropn{u}]
\end{mathpar}

\subsection{Additional notation}

It will sometimes be convenient to denote the process a name
quotes. We already have the notation $x = \quotep{P}$, but it will be
convenient to introduce an alternate notation, $\procn{x}$, when we
want to emphasize the connection to the use of the name. Note that, by
virtue of name equivalence, $\quotep{\procn{x}} \nameeq x$; so, the
notation is consistent with previous definitions.

Further, because names have structure it is possible to effect
substitutions on the basis of that structure. This means we need to
upgrade our notation for substitutions, which we accomplish by
adapting comprehension notation. Thus,

\begin{mathpar}
  P\{ y / x : x \in S \}
\end{mathpar}

is interpreted to mean the process derived from P by replacing (in a
capture-avoiding manner) each occurrence of $x$ in $S$ by $y$. For example,

\begin{mathpar}
  P\{ \quotep{\procn{x}|\procn{x}} / x : x \in \freenames{P} \}
\end{mathpar}

will replace each (occurrence) of a free name $x$ in $P$ by
$\quotep{\procn{x}|\procn{x}}$.

Also, we will avail ourselves of the notation $x^{L}$ and $x^{R}$ to
denote injections of a name into disjoint copies of the name
space. There are numerous ways to accomplish this. One example can be
found in \cite{MeredithR05}. This notation overloads to vectors of
names: $\vec{x}^{\pi} := (x_{i}^{\pi} \; : \; 0 \leq i < |\vec{x}| )$ where $\pi \in \{L,R\}$.

We also use $P^{\Box} := P|\Box$.

In \cite{MeredithR05} an interpretation of the new operator is
given. It turns out that there are several possible interpretations
all enjoying the requisite algebraic properties of the operator (see
\cite{milner91polyadicpi}). We will therefore make liberal use of
$(\nu\; \vec{x})P$.

% subsection the_syntax_and_semantics_of_the_notation_system (end)   

\input{qm2pi.qmops} 

\input{qm2pi.sterngerlach} 

\input{qm2pi.metric} 

% section concurrent_process_calculi (end)

%\input{qm2pi.proofsketch}

% section proof sketch (end)

%\input{qm2pi.slviaknots} 

% section spatial logic via knots (end)

\input{qm2pi.conclusion}

% section conclusion (end)

%\input{qm2pi.dtcodes} 

% section wiring algorithm (end)

\input{qm2pi.ack} 

% section acknowledgments (end)

\newpage


\bibliographystyle{plain}   
\bibliography{../../biblios/main.bib}

\input{qm2pi.rhodetails}

\end{document}

 

%\documentclass[12pt]{llncs}
%\documentclass{jktr}

\usepackage[pdftex]{hyperref}                   
\usepackage {listings}
\usepackage {mathpartir}
\usepackage{bcprules}
%\usepackage{listings}
                       
\usepackage{graphicx} 
%\usepackage[margins=2.5cm,nohead,nofoot]{geometry}
%\usepackage{geometry}
\usepackage{amsfonts}
\usepackage{amstext}
\usepackage{latexsym}
\usepackage{amssymb}
\usepackage{color}


%\include{myPreamble}
\include{qm2pi.local} 

%\ifpdf
%\usepackage[pdftex]{graphicx}
%\else
%\usepackage{graphicx}
%\fi

 % \ifpdf
%  \usepackage{pdfsync}
%  \if


%\title{Brief Article}
%\author{David F. Snyder}
%\author{L.G. Meredith}

%\address{Dept. of Math., Texas State University--San Marcos, San Marcos, TX 78666}
       
\pagestyle{empty}


\begin{document}

\lstset{language=[Objective]Caml,frame=shadowbox}

\input{qm2pi.front}

% section front matter (end)

\input{qm2pi.intro} 
 
% section introduction (end)

% \input{qm2pi.knotations} 

% section notation (end)

\input{qm2pi.process.calculi} 

% section concurrent_process_calculi_and_spatial_logics_ (end)
    
%\input{qm2pi.knots2pi} 

%\input{qm2pi.trefoil} 

%\input{qm2pi.mainthm} 

% subsection basic_interpretation (end)

%\input{qm2pi.rho.presentation} 
\subsection{The syntax and semantics of the notation system}\label{sub:the_syntax_and_semantics_of_the_notation_system} % (fold)

We now summarize a technical presentation of the calculus that
embodies our theory of dynamics. The typical presentation of such a
calculus follows the style of giving generators and relations on
them. The grammar, below, describing term constructors, freely
generates the set of processes, $\Proc$. This set is then quotiented
by a relation known as structural congruence and it is over this set
that the notion of dynamics is expressed. This presentation is
essentially that of \cite{MeredithR05} with the addition of
polyadicity and summation. For readability we have relegated some of
the technical subtleties to an appendix.

\subsubsection{Process grammar}\label{subsub:process_grammar}

\begin{mathpar}
  \inferrule* [lab=synchronization] {} {{M} \bc \pzero \;|\; x?F \;|\; x!C }
  \and
  \inferrule* [lab=abstraction] {} {{F} \bc (x)P}
  \and
  \inferrule* [lab=concretion] {} {{C} \bc \langle Q \rangle}
  \and
  \inferrule* [lab=process] {} {{P,Q} \bc M \;| \;P|Q \;|\; @{x}}
  \and
  \inferrule* [lab=name] {} {{x} \bc \quotep{P}}
\end{mathpar} 

Note that $\vec{x}$ (resp. $\vec{P}$) denotes a vector of names
(resp. processes) of length $|\vec{x}|$ (resp. $|\vec{P}|$). We adopt
the following useful abbreviations.

\begin{mathpar}
   x?(\vec{y}).P := x.(\vec{y})P \and  x\clift{\vec{P}} := x.\clift{\vec{P}}
   \and x!(y) := \lift{x}{\dropn{y}}
   \and \Pi_{i=0}^{n-1}P_i := P_0 | \ldots | P_{n-1}
\end{mathpar}

\subsubsection{Structural congruence}

\paragraph{Free and bound names and alpha-equivalence.} At the
core of structural equivalence is alpha-equivalence which identifies
process that are the same up to a change of variable. Formally, we
recognize the distinction between free and bound names. The free names
of a process, $\freenames{P}$, may be calculated recursively as
follows:

\begin{mathpar}
\freenames{\pzero} := \emptyset
  \and \\
  \freenames{x?(y).P} := \{ x \} \cup (\freenames{P} \setminus \{ y \})
  \and 
  \freenames{x!\langle P \rangle} := \{ x \} \cup \{ P \} 
  \and \\
  \freenames{P|Q} := \freenames{P} \cup \freenames{Q}
  \and \\
  \freenames{@{x}} := \{ x \}
\end{mathpar}

$\pi$
$\quotep{\pi}$

$\freenames{-} : \pi \to \mathcal{P}(\quotep{\pi})$

\begin{eqnarray*}
  \freenames{\pzero} & := & \emptyset \\
  \freenames{x?(y).P} & := & \{ x \} \cup (\freenames{P} \setminus \{ y \}) \\
  \freenames{x!\langle P \rangle} & := & \{ x \} \cup \{ P \} \\
  \freenames{P|Q} & := & \freenames{P} \cup \freenames{Q} \\
  \freenames{\dropn{x}} & := & \{ x \}
\end{eqnarray*}

The bound names of a process, $\boundnames{P}$, are those names occurring in $P$
that are not free. For example, in $x?(y).0$, the name $x$ is free, while $y$ is bound.

\begin{mathpar}
  \inferrule* [lab=monoidal-laws] {} { P|Q \equiv Q|P \and P|0 \equiv P \and P|(Q|R) \equiv (P|Q)|R }
\end{mathpar}

\begin{mathpar}
  \inferrule* [lab=alpha-equivalence] {} { (x)P \equiv (y)P\{y/x\} \and y \not\in \freenames{P} }
\end{mathpar}

\begin{definition}
Then two processes, $P,Q$, are alpha-equivalent if $P = Q\{\vec{y}/\vec{x}\}$ for
some $\vec{x} \in \boundnames{Q},\vec{y} \in \boundnames{P}$, where $Q\{\vec{y}/\vec{x}\}$
denotes the capture-avoiding substitution of $\vec{y}$ for $\vec{x}$ in $Q$.
\end{definition}

\begin{definition}
  The {\em structural congruence} \cite{SangiorgiWalker} , $\equiv$,
  between processes is the least congruence containing
  alpha-equivalence, satisfying the abelian monoid laws
  (associativity, commutativity and $\pzero$ as identity) for parallel
  composition $|$ and for summation $+$.
\end{definition}

\subsection{Name equivalence}

We take name equivalence, written $\nameeq$, to be the smallest
equivalence relation generated by the following rules.

\begin{mathpar}
\inferrule*[lab=Quote-drop]
{ }
{ \quotep{@{x}} \nameeq x }

\inferrule*[lab=Struct-equiv]
{ P \scong Q }
{ \quotep{P} \nameeq \quotep{Q} }
\end{mathpar}

The astute reader will have noticed that the mutual recursion of names
and processes imposes a mutual recursion on alpha-equivalence and
structural equivalence via name-equivalence. Fortunately, all of this
works out pleasantly and we may calculate in the natural way, free of
concern. The reader interested in the details is referred to the
appendix \ref{appendix:rho_details}.

\subsection{Substitution}

We use $\Proc$ for the set of processes, $\QProc$ for the set of
names, and $\id{\{}\vec{y} / \vec{x} \id{\}}$ to denote partial maps,
$s : \QProc \rightarrow \QProc$. A map, $s$ lifts, uniquely, to a map
on process terms, $\widehat{s} : \Proc \rightarrow \Proc$ by the
following equations.

\begin{mathpar}
  (0) \psubstp{Q}{P} := 0 \\
  (R \juxtap S) \psubstp{Q}{P}
  :=    
  (R)\psubstp{Q}{P} \juxtap (S) \psubstp{Q}{P} \\
  (x?(y).R) \psubstp{Q}{P}    
  :=    
  (x)\substp{Q}{P} (z)\concat( (R \psubstn{z}{y}) \psubstp{Q}{P} ) \\
  (\lift{x}{R}) \psubstp{Q}{P}  
  :=
  \lift{(x)\substp{Q}{P}}{ R \psubstp{Q}{P} } \\
%   (\dropn{x})  \psubstp{Q}{P}       
%   := 
%   \left\{ 
%     \begin{array}{ccc} 
%       \dropn{\quotep{Q}} & & x \nameeq \quotep{P} \\
%       \dropn{x} & & otherwise \\
%     \end{array}
%   \right. 
  (\dropn{x})  \psubstp{Q}{P}       
  := 
  \left\{ 
    \begin{array}{ccc} 
      Q & & x \nameeq \quotep{P} \\
      \dropn{x} & & otherwise \\
    \end{array}
  \right.
\end{mathpar}
 

where

\begin{eqnarray}
  (x)\id{\{} \lpquote Q \rpquote / \lpquote P \rpquote \id{\}}            = 
  \left\{ 
    \begin{array}{ccc}
      \lpquote Q \rpquote & & x \nameeq \lpquote P \rpquote \\
      x & & otherwise \\
    \end{array}
  \right. \nonumber
\end{eqnarray}

and $z$ is chosen distinct from $\quotep{P}$, $\quotep{Q}$, the free
names in $Q$, and all the names in $R$. Our $\alpha$-equivalence will
be built in the standard way from this substitution.

\begin{remark}\label{rem:no_self_referential_names}
  One consequence of these definitions is that $\forall P. \quotep{P}
  \not\in \freenames{P}$.
\end{remark}

\subsection{ Dynamic quote: an example }

Anticipating something of what's to come, consider applying the
substitution, $\widehat{\id{\{}u / z \id{\}}}$, to the following pair
of processes, $\lift{w}{y!(z)}$ and $w[ \lpquote y!(z) \rpquote ]$.

\begin{eqnarray}
	\lift{w}{y!(z)}\widehat{\id{\{}u / z \id{\}}}
		& = &
		\lift{w}{y!(u)} \nonumber\\
	w[ \lpquote y!(z) \rpquote ] \widehat{ \id{\{}u / z \id{\}} }
		& = &
		w[ \lpquote y!(z) \rpquote ] \nonumber
\end{eqnarray}

Because the body of the process between quotes is impervious to
substitution, we get radically different answers. In fact, by
examining the first process in an input context,
e.g. $x?(z).\lift{w}{y!(z)}$, we see that the process under the lift
operator may be shaped by prefixed inputs binding a name inside it. In
this sense, the lift operator will be seen as a way to dynamically
construct processes before reifying them as names.

Finally equipped with these standard features we can present the
dynamics of the calculus.

\subsubsection{Operational semantics} 

Finally, we introduce the computational dynamics. What marks these
algebras as distinct from other more traditionally studied algebraic
structures, e.g. vector spaces or polynomial rings, is the manner in
which dynamics is captured. In traditional structures, dynamics is typically
expressed through morphisms between such structures, as in linear maps
between vector spaces or morphisms between rings. In algebras
associated with the semantics of computation, the dynamics is
expressed as part of the algebraic structure itself, through a
reduction reduction relation typically denoted by $\red$. Below, we
give a recursive presentation of this relation for the calculus used
in the encoding.

$\red \subseteq \pi \times \pi$
$\red : \pi \to \mathcal{P}(\pi)$

\begin{mathpar}
  \inferrule* [lab=Comm] { \textsf{match}( x_{src}, x_{trgt} ) } { x_{trgt}?(y)P \; | \; x_{src}!\langle {Q} \rangle \red P\{\quotep{Q}/y}\} }
  \and \\
  \inferrule* [lab=Par] {{P} \red {P}'} {{{P} | {Q}} \red {{P}' | {Q}}}
  \and
  \inferrule* [lab=Equiv]{{{P} \scong {P}'} \andalso {{P}' \red {Q}'} \andalso {{Q}' \scong {Q}}}{{P} \red {Q}}
\end{mathpar}

\begin{eqnarray*}
  match_{\equiv} (\quotep{P},\quotep{Q}) & := & P \equiv Q \\
  match_{\dagger}(\quotep{P},\quotep{Q}) & := & \forall R. P|Q \red^{*} R => R \red^{*} 0 \\
  match_{K}(\quotep{P},\quotep{Q}) & := & K \mbox{ for some context } K
\end{eqnarray*}

$u?(x)P | u!\langle Q \rangle \red P\{\quotep{Q}/x\}$

%We write $\wred$ for $\red^*$, and $P\red$ if $\exists Q $ such that $ P \red Q$.
We write $P\red$ if $\exists Q $ such that $ P \red Q$ and $P\not\red$, otherwise.

\section{Replication}

As mentioned before, it is known that replication (and hence
recursion) can be implemented in a higher-order process algebra
\cite{SangiorgiWalker}. As our first example of calculation with the
machinery thus far presented we give the construction explicitly in
the {\rhoc}.

\begin{eqnarray}
	D_{x} & := & \prefix{x}{y}{(\binpar{\outputp{x}{y}}{@{y}})} \nonumber\\
	\bangp_{x}{P} & := & \binpar{{x}!\langle{\binpar{D_{x}}{P}}\rangle}{D_{x}} \nonumber
\end{eqnarray}

\begin{eqnarray}
	\bangp_{x}{P} & & \nonumber\\
	=
	& {x}!\langle{(\prefix{x}{y}{(\outputp{x}{y} | @{y})) | P}}\rangle 
	      | \prefix{x}{y}{(\outputp{x}{y} | @{y})} & \nonumber\\
	\red
	& (\outputp{x}{y} | @{y})\substn{\quotep{(\prefix{x}{y}{(@{y} | \outputp{x}{y})) | P}}}{y} & \nonumber\\
	=
	& \outputp{x}{\quotep{(\prefix{x}{y}{(\outputp{x}{y} | @{y})) | P}}}
	  | {(\prefix{x}{y}{(\outputp{x}{y} | @{y})) | P}} & \nonumber\\
	\red
	& \ldots & \nonumber\\
	\red^*
	& P | P | \ldots & \nonumber
\end{eqnarray}

Of course, this encoding, as an implementation, runs away, unfolding
$\bangp{P}$ eagerly. A lazier and more implementable replication
operator, restricted to input-guarded processes, may be obtained as follows.

\begin{eqnarray}
\bangp{\prefix{u}{v}{P}} 
	:= 
	\binpar{\lift{x}{\prefix{u}{v}{(\binpar{D(x)}{P})}}}{D(x)} \nonumber
\end{eqnarray}

\begin{remark}
  Note that the lazier definition still does not deal with summation
  or mixed summation (i.e. sums over input and output). The reader is
  invited to construct definitions of replication that deal with these
  features. 

  Further, the definitions are parameterized in a name, $x$. Can you,
  gentle reader, make a definition that eliminates this parameter and
  guarantees no accidental interaction between the replication
  machinery and the process being replicated -- i.e. no accidental
  sharing of names used by the process to get its work done and the
  name(s) used by the replication to effect copying. This latter
  revision of the definition of replication is crucial to obtaining
  the expected identity $!!P \sim !P$.
\end{remark}

\begin{remark}\label{rem:paradoxical_combinator}
  The reader familiar with the lambda calculus will have noticed the
  similarity between $D$ and the paradoxical combinator.

  [Ed. note: the existence of this seems to suggest we have to be more
  restrictive on the set of processes and names we admit if we are to
  support no-cloning.]
\end{remark}

\subsubsection{Bisimulation}

The computational dynamics gives rise to another kind of equivalence,
the equivalence of computational behavior. As previously mentioned
this is typically captured \emph{via} some form of bisimulation.

% The notion we use in this paper is weak barbed bisimulation
% \cite{milner91polyadicpi}.

The notion we use in this paper is derived from weak barbed
bisimulation \cite{milner91polyadicpi}. 

\begin{definition}
An \emph{observation relation}, $\downarrow_{\mathcal N}$, over a set
of names, $\mathcal N$, is the smallest relation satisfying the rules
below.

\infrule[Out-barb]{y \in {\mathcal N}, \; x \nameeq y}
		  {\outputp{x}{v} \downarrow_{\mathcal N} x}
\infrule[Par-barb]{\mbox{$P\downarrow_{\mathcal N} x$ or $Q\downarrow_{\mathcal N} x$}}
		  {\binpar{P}{Q} \downarrow_{\mathcal N} x}

We write $P \Downarrow_{\mathcal N} x$ if there is $Q$ such that 
$P \wred Q$ and $Q \downarrow_{\mathcal N} x$.
\end{definition}

\begin{definition}
%\label{def.bbisim}
An  ${\mathcal N}$-\emph{barbed bisimulation} over a set of names, ${\mathcal N}$, is a symmetric binary relation 
${\mathcal S}_{\mathcal N}$ between agents such that $P\rel{S}_{\mathcal N}Q$ implies:
\begin{enumerate}
\item If $P \red P'$ then $Q \wred Q'$ and $P'\rel{S}_{\mathcal N} Q'$.
\item If $P\downarrow_{\mathcal N} x$, then $Q\Downarrow_{\mathcal N} x$.
\end{enumerate}
$P$ is ${\mathcal N}$-barbed bisimilar to $Q$, written
$P \wbbisim_{\mathcal N} Q$, if $P \rel{S}_{\mathcal N} Q$ for some ${\mathcal N}$-barbed bisimulation ${\mathcal S}_{\mathcal N}$.
\end{definition}

$\mathcal{R} \subseteq \pi \times \pi$

$P \mathcal{R} Q => \forall P'. P \red P' \Rightarrow \exists Q'. Q \red Q', P' \mathcal{R} Q'$

$P \vdash x \Rightarrow Q \vdash x$

\begin{mathpar}
  \inferrule*[lab=Out-barb]{x \nameeq y}{{y}!\langle{Q}\rangle \vdash x}
  \and
  \inferrule*[lab=Par-barb]{\mbox{$P\vdash x$ or $Q\vdash x$}}{\binpar{P}{Q} \vdash x}
\end{mathpar}

\subsubsection{Contexts}

One of the principle advantages of computational calculi like the
$\pi$-calculus is a well-defined notion of context,
contextual-equivalence and a correlation between
contextual-equivalence and notions of bisimulation. The notion of
context allows the decomposition of a process into (sub-)process and
its syntactic environment, its context. Thus, a context may be
thought of as a process with a ``hole'' (written $\Box$) in it. The
application of a context $M$ to a process $P$, written $M[P]$, is
tantamount to filling the hole in $M$ with $P$. In this paper we do
not need the full weight of this theory, but do make use of the notion
of context in the proof the main theorem. 

\begin{mathpar}
  \inferrule* [lab=summation] {} {{M_{M},M_{N}} \bc \Box \;|\; x.M_{A} \;|\; M_{M}+M_{N}}
  \and
  \inferrule* [lab=agent] {} {{M_{A}} \bc (\vec{x})M_{P} \;| \; \clift{P_0,\ldots,M_{P},\ldots,P_N}}
  \and \\
  \inferrule* [lab=process] {} {{M_{P}} \bc M_{N} \;| \;P|M_{P} }
\end{mathpar} 

\begin{mathpar}
  \inferrule* [lab=sychronization] {} {M_{N} \bc \Box \;|\; x?M_{F} \;|\; x!M_{C}}
  \and
  \inferrule* [lab=abstraction] {} {{M_{F}} \bc (x)M_{P} }
  \and
  \inferrule* [lab=concretion] {} {{M_{C}} \bc \langle M_{P} \rangle }
  \and \\
  \inferrule* [lab=process] {} {{M_{P}} \bc M_{N} \;| \;P|M_{P} }
\end{mathpar}

\begin{definition}[contextual application] Given a context $M$, and
  process $P$, we define the \emph{contextual application}, $M[P] :=
  M\{P/\Box\}$. That is, the contextual application of M to P is the
  substitution of $P$ for $\Box$ in $M$.
\end{definition}

$\meaningof{-} : L \to \mathcal{P}(\pi)$

\begin{mathpar}
  \inferrule* [lab=collection] {} {\meaningof{true} = \pi, \and \meaningof{~E} = \pi \setminus \meaningof{E}, \and \meaningof{E_{1} \& E_{2}} = \meaningof{E_{1}} \cap \meaningof{E_{2}}}
\end{mathpar}

\begin{mathpar}
  \inferrule* [lab=structure] {} {\meaningof{0} = \{ P \in \pi | P \equiv 0 \}, \and \\ \meaningof{E_1 | E_2} = \{ P \in \pi | P \equiv P_{1} | P_{2}, P_{1} \in \meaningof{E_{1}}, P_{2} \in \meaningof{E_2}\} }
\end{mathpar}

\begin{mathpar}
 \inferrule* [lab=behavior] {} {\meaningof{\langle a?b \rangle E} = \{ P \in \pi | P \equiv Q | u?(y)P', \\ \and \\\\ \and \\ \;\;\; u \in \meaningof{a}, \forall z.P'\{z/y\} \in \meaningof{E\{z/b\}}\}, \and \\ \meaningof{a!E} = \{ P \in \pi | P \equiv Q | x!\langle P' \rangle, x \in \meaningof{a} P' \in \meaningof{E}\} }
\end{mathpar}

\begin{mathpar}
 \inferrule* [lab=nominal] {} {\meaningof{\quotep{E}} = \{ \quotep{P} \in \quotep{\pi} | P \in \meaningof{E} \}, \and \meaningof{\quotep{P}} = \{ \quotep{Q} \in \quotep{\pi} | P \equiv Q \} \and \\ \meaningof{@\quotep{E}} = \{ P \in \pi | P \equiv @x, x \in \meaningof{E} \}}
\end{mathpar}

\begin{eqnarray*}
  \\
  \meaningof{-} : TS \to ST
\end{eqnarray*}

\begin{eqnarray*}
  \\
  L : TS \to ST
\end{eqnarray*}

\begin{eqnarray*}
  \\
  P \models E \iff P \in \meaningof{E}
\end{eqnarray*}

\begin{eqnarray*}
  P \approx_{L} Q \iff \forall E \in L. P \models E \iff Q \models E
\end{eqnarray*}

\begin{eqnarray*}
  P \approx_{K} Q
\end{eqnarray*}

\begin{eqnarray*}
  P \approx Q
\end{eqnarray*}

$\approx_{K} = \approx = \approx_{L}$

\subsubsection{Contextual duality}

Note that contexts extend the quotation operation to a family of
operations from processes to names. Given a context, $M$, we can
define a \emph{nominal context}, $\quotep{M}$ by $\quotep{M}[P] :=
\quotep{M[P]}$. To foreshadow what is to come we observe that these
operations enjoy a duality with processes very much like the duality
between vectors and maps from vectors to scalars.

Further, because the calculus is essentially higher-order, we have a
correspondence between contexts and processes. More specifically,
given a name $x$ and a context $M$ we can construct $M^{*}_{x}$ such
that 

\begin{mathpar}
  M^{*}_{x} | \lift{x}{P} \red M[P]
\end{mathpar}

namely,

\begin{mathpar}
  M^{*}_{x} := x?(u).M[\dropn{u}]
\end{mathpar}

The dependence of $M^{*}_{x}$ on a name makes it an abstraction, 

\begin{mathpar}
  M^{*} := (x)x?(u).M[\dropn{u}]
\end{mathpar}

\subsection{Additional notation}

It will sometimes be convenient to denote the process a name
quotes. We already have the notation $x = \quotep{P}$, but it will be
convenient to introduce an alternate notation, $\procn{x}$, when we
want to emphasize the connection to the use of the name. Note that, by
virtue of name equivalence, $\quotep{\procn{x}} \nameeq x$; so, the
notation is consistent with previous definitions.

Further, because names have structure it is possible to effect
substitutions on the basis of that structure. This means we need to
upgrade our notation for substitutions, which we accomplish by
adapting comprehension notation. Thus,

\begin{mathpar}
  P\{ y / x : x \in S \}
\end{mathpar}

is interpreted to mean the process derived from P by replacing (in a
capture-avoiding manner) each occurrence of $x$ in $S$ by $y$. For example,

\begin{mathpar}
  P\{ \quotep{\procn{x}|\procn{x}} / x : x \in \freenames{P} \}
\end{mathpar}

will replace each (occurrence) of a free name $x$ in $P$ by
$\quotep{\procn{x}|\procn{x}}$.

Also, we will avail ourselves of the notation $x^{L}$ and $x^{R}$ to
denote injections of a name into disjoint copies of the name
space. There are numerous ways to accomplish this. One example can be
found in \cite{MeredithR05}. This notation overloads to vectors of
names: $\vec{x}^{\pi} := (x_{i}^{\pi} \; : \; 0 \leq i < |\vec{x}| )$ where $\pi \in \{L,R\}$.

We also use $P^{\Box} := P|\Box$.

In \cite{MeredithR05} an interpretation of the new operator is
given. It turns out that there are several possible interpretations
all enjoying the requisite algebraic properties of the operator (see
\cite{milner91polyadicpi}). We will therefore make liberal use of
$(\nu\; \vec{x})P$.

% subsection the_syntax_and_semantics_of_the_notation_system (end)   

\input{qm2pi.qmops} 

\input{qm2pi.sterngerlach} 

\input{qm2pi.metric} 

% section concurrent_process_calculi (end)

%\input{qm2pi.proofsketch}

% section proof sketch (end)

%\input{qm2pi.slviaknots} 

% section spatial logic via knots (end)

\input{qm2pi.conclusion}

% section conclusion (end)

%\input{qm2pi.dtcodes} 

% section wiring algorithm (end)

\input{qm2pi.ack} 

% section acknowledgments (end)

\newpage


\bibliographystyle{plain}   
\bibliography{../../biblios/main.bib}

\input{qm2pi.rhodetails}

\end{document}

 

% subsection basic_interpretation (end)

%\input{qm2pi.rho.presentation} 
\subsection{The syntax and semantics of the notation system}\label{sub:the_syntax_and_semantics_of_the_notation_system} % (fold)

We now summarize a technical presentation of the calculus that
embodies our theory of dynamics. The typical presentation of such a
calculus follows the style of giving generators and relations on
them. The grammar, below, describing term constructors, freely
generates the set of processes, $\Proc$. This set is then quotiented
by a relation known as structural congruence and it is over this set
that the notion of dynamics is expressed. This presentation is
essentially that of \cite{MeredithR05} with the addition of
polyadicity and summation. For readability we have relegated some of
the technical subtleties to an appendix.

\subsubsection{Process grammar}\label{subsub:process_grammar}

\begin{mathpar}
  \inferrule* [lab=synchronization] {} {{M} \bc \pzero \;|\; x?F \;|\; x!C }
  \and
  \inferrule* [lab=abstraction] {} {{F} \bc (x)P}
  \and
  \inferrule* [lab=concretion] {} {{C} \bc \langle Q \rangle}
  \and
  \inferrule* [lab=process] {} {{P,Q} \bc M \;| \;P|Q \;|\; @{x}}
  \and
  \inferrule* [lab=name] {} {{x} \bc \quotep{P}}
\end{mathpar} 

Note that $\vec{x}$ (resp. $\vec{P}$) denotes a vector of names
(resp. processes) of length $|\vec{x}|$ (resp. $|\vec{P}|$). We adopt
the following useful abbreviations.

\begin{mathpar}
   x?(\vec{y}).P := x.(\vec{y})P \and  x\clift{\vec{P}} := x.\clift{\vec{P}}
   \and x!(y) := \lift{x}{\dropn{y}}
   \and \Pi_{i=0}^{n-1}P_i := P_0 | \ldots | P_{n-1}
\end{mathpar}

\subsubsection{Structural congruence}

\paragraph{Free and bound names and alpha-equivalence.} At the
core of structural equivalence is alpha-equivalence which identifies
process that are the same up to a change of variable. Formally, we
recognize the distinction between free and bound names. The free names
of a process, $\freenames{P}$, may be calculated recursively as
follows:

\begin{mathpar}
\freenames{\pzero} := \emptyset
  \and \\
  \freenames{x?(y).P} := \{ x \} \cup (\freenames{P} \setminus \{ y \})
  \and 
  \freenames{x!\langle P \rangle} := \{ x \} \cup \{ P \} 
  \and \\
  \freenames{P|Q} := \freenames{P} \cup \freenames{Q}
  \and \\
  \freenames{@{x}} := \{ x \}
\end{mathpar}

$\pi$
$\quotep{\pi}$

$\freenames{-} : \pi \to \mathcal{P}(\quotep{\pi})$

\begin{eqnarray*}
  \freenames{\pzero} & := & \emptyset \\
  \freenames{x?(y).P} & := & \{ x \} \cup (\freenames{P} \setminus \{ y \}) \\
  \freenames{x!\langle P \rangle} & := & \{ x \} \cup \{ P \} \\
  \freenames{P|Q} & := & \freenames{P} \cup \freenames{Q} \\
  \freenames{\dropn{x}} & := & \{ x \}
\end{eqnarray*}

The bound names of a process, $\boundnames{P}$, are those names occurring in $P$
that are not free. For example, in $x?(y).0$, the name $x$ is free, while $y$ is bound.

\begin{mathpar}
  \inferrule* [lab=monoidal-laws] {} { P|Q \equiv Q|P \and P|0 \equiv P \and P|(Q|R) \equiv (P|Q)|R }
\end{mathpar}

\begin{mathpar}
  \inferrule* [lab=alpha-equivalence] {} { (x)P \equiv (y)P\{y/x\} \and y \not\in \freenames{P} }
\end{mathpar}

\begin{definition}
Then two processes, $P,Q$, are alpha-equivalent if $P = Q\{\vec{y}/\vec{x}\}$ for
some $\vec{x} \in \boundnames{Q},\vec{y} \in \boundnames{P}$, where $Q\{\vec{y}/\vec{x}\}$
denotes the capture-avoiding substitution of $\vec{y}$ for $\vec{x}$ in $Q$.
\end{definition}

\begin{definition}
  The {\em structural congruence} \cite{SangiorgiWalker} , $\equiv$,
  between processes is the least congruence containing
  alpha-equivalence, satisfying the abelian monoid laws
  (associativity, commutativity and $\pzero$ as identity) for parallel
  composition $|$ and for summation $+$.
\end{definition}

\subsection{Name equivalence}

We take name equivalence, written $\nameeq$, to be the smallest
equivalence relation generated by the following rules.

\begin{mathpar}
\inferrule*[lab=Quote-drop]
{ }
{ \quotep{@{x}} \nameeq x }

\inferrule*[lab=Struct-equiv]
{ P \scong Q }
{ \quotep{P} \nameeq \quotep{Q} }
\end{mathpar}

The astute reader will have noticed that the mutual recursion of names
and processes imposes a mutual recursion on alpha-equivalence and
structural equivalence via name-equivalence. Fortunately, all of this
works out pleasantly and we may calculate in the natural way, free of
concern. The reader interested in the details is referred to the
appendix \ref{appendix:rho_details}.

\subsection{Substitution}

We use $\Proc$ for the set of processes, $\QProc$ for the set of
names, and $\id{\{}\vec{y} / \vec{x} \id{\}}$ to denote partial maps,
$s : \QProc \rightarrow \QProc$. A map, $s$ lifts, uniquely, to a map
on process terms, $\widehat{s} : \Proc \rightarrow \Proc$ by the
following equations.

\begin{mathpar}
  (0) \psubstp{Q}{P} := 0 \\
  (R \juxtap S) \psubstp{Q}{P}
  :=    
  (R)\psubstp{Q}{P} \juxtap (S) \psubstp{Q}{P} \\
  (x?(y).R) \psubstp{Q}{P}    
  :=    
  (x)\substp{Q}{P} (z)\concat( (R \psubstn{z}{y}) \psubstp{Q}{P} ) \\
  (\lift{x}{R}) \psubstp{Q}{P}  
  :=
  \lift{(x)\substp{Q}{P}}{ R \psubstp{Q}{P} } \\
%   (\dropn{x})  \psubstp{Q}{P}       
%   := 
%   \left\{ 
%     \begin{array}{ccc} 
%       \dropn{\quotep{Q}} & & x \nameeq \quotep{P} \\
%       \dropn{x} & & otherwise \\
%     \end{array}
%   \right. 
  (\dropn{x})  \psubstp{Q}{P}       
  := 
  \left\{ 
    \begin{array}{ccc} 
      Q & & x \nameeq \quotep{P} \\
      \dropn{x} & & otherwise \\
    \end{array}
  \right.
\end{mathpar}
 

where

\begin{eqnarray}
  (x)\id{\{} \lpquote Q \rpquote / \lpquote P \rpquote \id{\}}            = 
  \left\{ 
    \begin{array}{ccc}
      \lpquote Q \rpquote & & x \nameeq \lpquote P \rpquote \\
      x & & otherwise \\
    \end{array}
  \right. \nonumber
\end{eqnarray}

and $z$ is chosen distinct from $\quotep{P}$, $\quotep{Q}$, the free
names in $Q$, and all the names in $R$. Our $\alpha$-equivalence will
be built in the standard way from this substitution.

\begin{remark}\label{rem:no_self_referential_names}
  One consequence of these definitions is that $\forall P. \quotep{P}
  \not\in \freenames{P}$.
\end{remark}

\subsection{ Dynamic quote: an example }

Anticipating something of what's to come, consider applying the
substitution, $\widehat{\id{\{}u / z \id{\}}}$, to the following pair
of processes, $\lift{w}{y!(z)}$ and $w[ \lpquote y!(z) \rpquote ]$.

\begin{eqnarray}
	\lift{w}{y!(z)}\widehat{\id{\{}u / z \id{\}}}
		& = &
		\lift{w}{y!(u)} \nonumber\\
	w[ \lpquote y!(z) \rpquote ] \widehat{ \id{\{}u / z \id{\}} }
		& = &
		w[ \lpquote y!(z) \rpquote ] \nonumber
\end{eqnarray}

Because the body of the process between quotes is impervious to
substitution, we get radically different answers. In fact, by
examining the first process in an input context,
e.g. $x?(z).\lift{w}{y!(z)}$, we see that the process under the lift
operator may be shaped by prefixed inputs binding a name inside it. In
this sense, the lift operator will be seen as a way to dynamically
construct processes before reifying them as names.

Finally equipped with these standard features we can present the
dynamics of the calculus.

\subsubsection{Operational semantics} 

Finally, we introduce the computational dynamics. What marks these
algebras as distinct from other more traditionally studied algebraic
structures, e.g. vector spaces or polynomial rings, is the manner in
which dynamics is captured. In traditional structures, dynamics is typically
expressed through morphisms between such structures, as in linear maps
between vector spaces or morphisms between rings. In algebras
associated with the semantics of computation, the dynamics is
expressed as part of the algebraic structure itself, through a
reduction reduction relation typically denoted by $\red$. Below, we
give a recursive presentation of this relation for the calculus used
in the encoding.

$\red \subseteq \pi \times \pi$
$\red : \pi \to \mathcal{P}(\pi)$

\begin{mathpar}
  \inferrule* [lab=Comm] { \textsf{match}( x_{src}, x_{trgt} ) } { x_{trgt}?(y)P \; | \; x_{src}!\langle {Q} \rangle \red P\{\quotep{Q}/y}\} }
  \and \\
  \inferrule* [lab=Par] {{P} \red {P}'} {{{P} | {Q}} \red {{P}' | {Q}}}
  \and
  \inferrule* [lab=Equiv]{{{P} \scong {P}'} \andalso {{P}' \red {Q}'} \andalso {{Q}' \scong {Q}}}{{P} \red {Q}}
\end{mathpar}

\begin{eqnarray*}
  match_{\equiv} (\quotep{P},\quotep{Q}) & := & P \equiv Q \\
  match_{\dagger}(\quotep{P},\quotep{Q}) & := & \forall R. P|Q \red^{*} R => R \red^{*} 0 \\
  match_{K}(\quotep{P},\quotep{Q}) & := & K \mbox{ for some context } K
\end{eqnarray*}

$u?(x)P | u!\langle Q \rangle \red P\{\quotep{Q}/x\}$

%We write $\wred$ for $\red^*$, and $P\red$ if $\exists Q $ such that $ P \red Q$.
We write $P\red$ if $\exists Q $ such that $ P \red Q$ and $P\not\red$, otherwise.

\section{Replication}

As mentioned before, it is known that replication (and hence
recursion) can be implemented in a higher-order process algebra
\cite{SangiorgiWalker}. As our first example of calculation with the
machinery thus far presented we give the construction explicitly in
the {\rhoc}.

\begin{eqnarray}
	D_{x} & := & \prefix{x}{y}{(\binpar{\outputp{x}{y}}{@{y}})} \nonumber\\
	\bangp_{x}{P} & := & \binpar{{x}!\langle{\binpar{D_{x}}{P}}\rangle}{D_{x}} \nonumber
\end{eqnarray}

\begin{eqnarray}
	\bangp_{x}{P} & & \nonumber\\
	=
	& {x}!\langle{(\prefix{x}{y}{(\outputp{x}{y} | @{y})) | P}}\rangle 
	      | \prefix{x}{y}{(\outputp{x}{y} | @{y})} & \nonumber\\
	\red
	& (\outputp{x}{y} | @{y})\substn{\quotep{(\prefix{x}{y}{(@{y} | \outputp{x}{y})) | P}}}{y} & \nonumber\\
	=
	& \outputp{x}{\quotep{(\prefix{x}{y}{(\outputp{x}{y} | @{y})) | P}}}
	  | {(\prefix{x}{y}{(\outputp{x}{y} | @{y})) | P}} & \nonumber\\
	\red
	& \ldots & \nonumber\\
	\red^*
	& P | P | \ldots & \nonumber
\end{eqnarray}

Of course, this encoding, as an implementation, runs away, unfolding
$\bangp{P}$ eagerly. A lazier and more implementable replication
operator, restricted to input-guarded processes, may be obtained as follows.

\begin{eqnarray}
\bangp{\prefix{u}{v}{P}} 
	:= 
	\binpar{\lift{x}{\prefix{u}{v}{(\binpar{D(x)}{P})}}}{D(x)} \nonumber
\end{eqnarray}

\begin{remark}
  Note that the lazier definition still does not deal with summation
  or mixed summation (i.e. sums over input and output). The reader is
  invited to construct definitions of replication that deal with these
  features. 

  Further, the definitions are parameterized in a name, $x$. Can you,
  gentle reader, make a definition that eliminates this parameter and
  guarantees no accidental interaction between the replication
  machinery and the process being replicated -- i.e. no accidental
  sharing of names used by the process to get its work done and the
  name(s) used by the replication to effect copying. This latter
  revision of the definition of replication is crucial to obtaining
  the expected identity $!!P \sim !P$.
\end{remark}

\begin{remark}\label{rem:paradoxical_combinator}
  The reader familiar with the lambda calculus will have noticed the
  similarity between $D$ and the paradoxical combinator.

  [Ed. note: the existence of this seems to suggest we have to be more
  restrictive on the set of processes and names we admit if we are to
  support no-cloning.]
\end{remark}

\subsubsection{Bisimulation}

The computational dynamics gives rise to another kind of equivalence,
the equivalence of computational behavior. As previously mentioned
this is typically captured \emph{via} some form of bisimulation.

% The notion we use in this paper is weak barbed bisimulation
% \cite{milner91polyadicpi}.

The notion we use in this paper is derived from weak barbed
bisimulation \cite{milner91polyadicpi}. 

\begin{definition}
An \emph{observation relation}, $\downarrow_{\mathcal N}$, over a set
of names, $\mathcal N$, is the smallest relation satisfying the rules
below.

\infrule[Out-barb]{y \in {\mathcal N}, \; x \nameeq y}
		  {\outputp{x}{v} \downarrow_{\mathcal N} x}
\infrule[Par-barb]{\mbox{$P\downarrow_{\mathcal N} x$ or $Q\downarrow_{\mathcal N} x$}}
		  {\binpar{P}{Q} \downarrow_{\mathcal N} x}

We write $P \Downarrow_{\mathcal N} x$ if there is $Q$ such that 
$P \wred Q$ and $Q \downarrow_{\mathcal N} x$.
\end{definition}

\begin{definition}
%\label{def.bbisim}
An  ${\mathcal N}$-\emph{barbed bisimulation} over a set of names, ${\mathcal N}$, is a symmetric binary relation 
${\mathcal S}_{\mathcal N}$ between agents such that $P\rel{S}_{\mathcal N}Q$ implies:
\begin{enumerate}
\item If $P \red P'$ then $Q \wred Q'$ and $P'\rel{S}_{\mathcal N} Q'$.
\item If $P\downarrow_{\mathcal N} x$, then $Q\Downarrow_{\mathcal N} x$.
\end{enumerate}
$P$ is ${\mathcal N}$-barbed bisimilar to $Q$, written
$P \wbbisim_{\mathcal N} Q$, if $P \rel{S}_{\mathcal N} Q$ for some ${\mathcal N}$-barbed bisimulation ${\mathcal S}_{\mathcal N}$.
\end{definition}

$\mathcal{R} \subseteq \pi \times \pi$

$P \mathcal{R} Q => \forall P'. P \red P' \Rightarrow \exists Q'. Q \red Q', P' \mathcal{R} Q'$

$P \vdash x \Rightarrow Q \vdash x$

\begin{mathpar}
  \inferrule*[lab=Out-barb]{x \nameeq y}{{y}!\langle{Q}\rangle \vdash x}
  \and
  \inferrule*[lab=Par-barb]{\mbox{$P\vdash x$ or $Q\vdash x$}}{\binpar{P}{Q} \vdash x}
\end{mathpar}

\subsubsection{Contexts}

One of the principle advantages of computational calculi like the
$\pi$-calculus is a well-defined notion of context,
contextual-equivalence and a correlation between
contextual-equivalence and notions of bisimulation. The notion of
context allows the decomposition of a process into (sub-)process and
its syntactic environment, its context. Thus, a context may be
thought of as a process with a ``hole'' (written $\Box$) in it. The
application of a context $M$ to a process $P$, written $M[P]$, is
tantamount to filling the hole in $M$ with $P$. In this paper we do
not need the full weight of this theory, but do make use of the notion
of context in the proof the main theorem. 

\begin{mathpar}
  \inferrule* [lab=summation] {} {{M_{M},M_{N}} \bc \Box \;|\; x.M_{A} \;|\; M_{M}+M_{N}}
  \and
  \inferrule* [lab=agent] {} {{M_{A}} \bc (\vec{x})M_{P} \;| \; \clift{P_0,\ldots,M_{P},\ldots,P_N}}
  \and \\
  \inferrule* [lab=process] {} {{M_{P}} \bc M_{N} \;| \;P|M_{P} }
\end{mathpar} 

\begin{mathpar}
  \inferrule* [lab=sychronization] {} {M_{N} \bc \Box \;|\; x?M_{F} \;|\; x!M_{C}}
  \and
  \inferrule* [lab=abstraction] {} {{M_{F}} \bc (x)M_{P} }
  \and
  \inferrule* [lab=concretion] {} {{M_{C}} \bc \langle M_{P} \rangle }
  \and \\
  \inferrule* [lab=process] {} {{M_{P}} \bc M_{N} \;| \;P|M_{P} }
\end{mathpar}

\begin{definition}[contextual application] Given a context $M$, and
  process $P$, we define the \emph{contextual application}, $M[P] :=
  M\{P/\Box\}$. That is, the contextual application of M to P is the
  substitution of $P$ for $\Box$ in $M$.
\end{definition}

$\meaningof{-} : L \to \mathcal{P}(\pi)$

\begin{mathpar}
  \inferrule* [lab=collection] {} {\meaningof{true} = \pi, \and \meaningof{~E} = \pi \setminus \meaningof{E}, \and \meaningof{E_{1} \& E_{2}} = \meaningof{E_{1}} \cap \meaningof{E_{2}}}
\end{mathpar}

\begin{mathpar}
  \inferrule* [lab=structure] {} {\meaningof{0} = \{ P \in \pi | P \equiv 0 \}, \and \\ \meaningof{E_1 | E_2} = \{ P \in \pi | P \equiv P_{1} | P_{2}, P_{1} \in \meaningof{E_{1}}, P_{2} \in \meaningof{E_2}\} }
\end{mathpar}

\begin{mathpar}
 \inferrule* [lab=behavior] {} {\meaningof{\langle a?b \rangle E} = \{ P \in \pi | P \equiv Q | u?(y)P', \\ \and \\\\ \and \\ \;\;\; u \in \meaningof{a}, \forall z.P'\{z/y\} \in \meaningof{E\{z/b\}}\}, \and \\ \meaningof{a!E} = \{ P \in \pi | P \equiv Q | x!\langle P' \rangle, x \in \meaningof{a} P' \in \meaningof{E}\} }
\end{mathpar}

\begin{mathpar}
 \inferrule* [lab=nominal] {} {\meaningof{\quotep{E}} = \{ \quotep{P} \in \quotep{\pi} | P \in \meaningof{E} \}, \and \meaningof{\quotep{P}} = \{ \quotep{Q} \in \quotep{\pi} | P \equiv Q \} \and \\ \meaningof{@\quotep{E}} = \{ P \in \pi | P \equiv @x, x \in \meaningof{E} \}}
\end{mathpar}

\begin{eqnarray*}
  \\
  \meaningof{-} : TS \to ST
\end{eqnarray*}

\begin{eqnarray*}
  \\
  L : TS \to ST
\end{eqnarray*}

\begin{eqnarray*}
  \\
  P \models E \iff P \in \meaningof{E}
\end{eqnarray*}

\begin{eqnarray*}
  P \approx_{L} Q \iff \forall E \in L. P \models E \iff Q \models E
\end{eqnarray*}

\begin{eqnarray*}
  P \approx_{K} Q
\end{eqnarray*}

\begin{eqnarray*}
  P \approx Q
\end{eqnarray*}

$\approx_{K} = \approx = \approx_{L}$

\subsubsection{Contextual duality}

Note that contexts extend the quotation operation to a family of
operations from processes to names. Given a context, $M$, we can
define a \emph{nominal context}, $\quotep{M}$ by $\quotep{M}[P] :=
\quotep{M[P]}$. To foreshadow what is to come we observe that these
operations enjoy a duality with processes very much like the duality
between vectors and maps from vectors to scalars.

Further, because the calculus is essentially higher-order, we have a
correspondence between contexts and processes. More specifically,
given a name $x$ and a context $M$ we can construct $M^{*}_{x}$ such
that 

\begin{mathpar}
  M^{*}_{x} | \lift{x}{P} \red M[P]
\end{mathpar}

namely,

\begin{mathpar}
  M^{*}_{x} := x?(u).M[\dropn{u}]
\end{mathpar}

The dependence of $M^{*}_{x}$ on a name makes it an abstraction, 

\begin{mathpar}
  M^{*} := (x)x?(u).M[\dropn{u}]
\end{mathpar}

\subsection{Additional notation}

It will sometimes be convenient to denote the process a name
quotes. We already have the notation $x = \quotep{P}$, but it will be
convenient to introduce an alternate notation, $\procn{x}$, when we
want to emphasize the connection to the use of the name. Note that, by
virtue of name equivalence, $\quotep{\procn{x}} \nameeq x$; so, the
notation is consistent with previous definitions.

Further, because names have structure it is possible to effect
substitutions on the basis of that structure. This means we need to
upgrade our notation for substitutions, which we accomplish by
adapting comprehension notation. Thus,

\begin{mathpar}
  P\{ y / x : x \in S \}
\end{mathpar}

is interpreted to mean the process derived from P by replacing (in a
capture-avoiding manner) each occurrence of $x$ in $S$ by $y$. For example,

\begin{mathpar}
  P\{ \quotep{\procn{x}|\procn{x}} / x : x \in \freenames{P} \}
\end{mathpar}

will replace each (occurrence) of a free name $x$ in $P$ by
$\quotep{\procn{x}|\procn{x}}$.

Also, we will avail ourselves of the notation $x^{L}$ and $x^{R}$ to
denote injections of a name into disjoint copies of the name
space. There are numerous ways to accomplish this. One example can be
found in \cite{MeredithR05}. This notation overloads to vectors of
names: $\vec{x}^{\pi} := (x_{i}^{\pi} \; : \; 0 \leq i < |\vec{x}| )$ where $\pi \in \{L,R\}$.

We also use $P^{\Box} := P|\Box$.

In \cite{MeredithR05} an interpretation of the new operator is
given. It turns out that there are several possible interpretations
all enjoying the requisite algebraic properties of the operator (see
\cite{milner91polyadicpi}). We will therefore make liberal use of
$(\nu\; \vec{x})P$.

% subsection the_syntax_and_semantics_of_the_notation_system (end)   

\section{Interpretation of QM}
\subsection{Supporting definitions}
\subsubsection{Multiplication}
\begin{mathpar}
  \quotep{Q} \cdot \quotep{R} := \quotep{Q|R}
  \and \\
  \quotep{Q} \cdot P := P\{ \quotep{Q|R} / \quotep{R} : \quotep{R} \in \freenames{P} \}
\end{mathpar}

\paragraph{Discussion}
The first line needs little explanation. The second line says that
each free name of the process is replaced with the multiplication of
that name by the scalar. Multiplication of a scalar (name) by a state
(process) results in a process all the names of which have been `moved
over' by parallel composition with the process the scalar
quotes. There is a subtlety that the bound names have to be
manipulated so that multiplied names aren't accidentally
captured. There are many ways to achieve this.

\begin{remark}\label{rem:multiplication_identities}
  The reader is invited to verify that for all $x,y,z \in \QProc$ and $P \in \Proc$
  \begin{mathpar}
    x \cdot \quotep{0} \equiv x 
    \and
    x \cdot y \equiv y \cdot x
    \and
    x \cdot (y \cdot z) \equiv (x \cdot y) \cdot z
    \and \\
    \quotep{0} \cdot P \equiv P
    \and \\
    x \cdot (y \cdot P) \equiv (x \cdot y) \cdot P
    \and \\
    x \cdot (P|Q) \equiv (x \cdot P) | (x \cdot Q)
    \and \\    
  \end{mathpar}
\end{remark}

\subsubsection{Tensor product}

We define a tensor product on processes by structural induction.

\paragraph{Tensor of sums} First note that all summations, including
$\pzero$ and sequence, can be written $\Sigma_{i} x_{i}.A_{i} +
\Sigma_{j} x_{j}.C_{j}$, where we have grouped input-guarded processes
together and output-guarded processes together.

Thus, we can define the tensor product of two summations, $N_{1}\otimes N_{2}$, where

\begin{mathpar}
  N_{1} := \Sigma_{i} x_{i}.A_{i} + \Sigma_{j} x_{j}.C_{j}
  \and
  N_{2} := \Sigma_{i'} y_{i'}.B_{i'} + \Sigma_{j'} y_{j'}.D_{j'} 
\end{mathpar}

as follows.

\begin{mathpar}
  \Sigma_{i} x_{i}.A_{i} + \Sigma_{j} x_{j}.C_{j} \otimes \Sigma_{i'}
  y_{i'}.B_{i'} + \Sigma_{j'} y_{j'}.D_{j'} 
  \and \\
  := \; \Sigma_{i} \Sigma_{i'} \quotep{\stackrel{\vee}{x_{i}}| \stackrel{\vee}{y_{i'}}}.(A_{i}\otimes B_{i'}) \; | \; \Sigma_{i'} \Sigma_{i} \quotep{\stackrel{\vee}{y_{i'}}|\stackrel{\vee}{x_{i}}}.(B_{i'}\otimes A_{i})
  \and
  \;\; | \;\; \Sigma_{j} \Sigma_{j'} \quotep{\stackrel{\vee}{x_{j}}|\stackrel{\vee}{y_{j'}}}.(A_{j}\otimes B_{j'}) \; | \; \Sigma_{j'} \Sigma_{j} \quotep{\stackrel{\vee}{y_{j'}}|\stackrel{\vee}{x_{j}}}.(B_{j'}\otimes A_{j})
\end{mathpar}

\begin{remark}
  Do we need to $x^{L}$ and $y^{R}$ for this construction as well?
\end{remark}

\paragraph{Tensor of parallel compositions} Next, we distribute tensor
over par.

\begin{mathpar}
  P_{1}|P_{2} \otimes Q_{1}|Q_{2} := (P_{1} \otimes Q_{1}) | (P_{1}
  \otimes Q_{2}) | (P_{2} \otimes Q_{1}) | (P_{2} \otimes Q_{2})
\end{mathpar}

\paragraph{Tensor with dropped names} We treat tensor of a
process with a dropped name as parallel composition.

\begin{mathpar}
  P \otimes \dropn{x} := P | \dropn{x}
\end{mathpar}

\paragraph{Tensor of agents}

Finally, we need to define tensor on agents. Note that the definition
of tensor on normal products only tensors inputs with inputs and
outputs with outputs. Thus, we only have to define the operation on
``homogeneous'' pairings.

\begin{mathpar}
  (\vec{x})P \otimes (\vec{y})Q
  \and \\
  := (x_{0}^{L}|y_{0}^{R},\ldots,x_{0}^{L}|y_{n}^{R},\ldots,x_{m}^{L}|y_{0}^{R},\ldots,x_{m}^{L}|y_{n}^R)(P\{ \vec{x}^{L}/\vec{x}\} \otimes Q \{ \vec{y}^{R}/\vec{y}\})
  \and \\
  \clift{\vec{P}} \otimes \clift{\vec{Q}}
  \and \\
  := \clift{P_{0}\otimes Q_{0},\ldots,P_{0}\otimes Q_{n},\ldots,P_{m}\otimes Q_{0},\ldots,P_{m}\otimes Q_{n}}
\end{mathpar}

\begin{remark}
  Observe that arities of tensored abstractions matches arities of
  tensored concretions if the original arities matched. Note also that
  the length of the arities corresponds to the increase in dimension
  we see in ordinary vector space tensor product.
\end{remark}

\begin{remark}
  Operationally, this definition distributes the tensor down to
  components ``linked'' by summation. Tensor over summation is
  intriguing in that it mixes names. Moreover, as a consequence of the
  way it mixes names we have the identities for all $x \in \QProc$ and
  $P,Q \in \Proc$

  \begin{mathpar}
    (x \cdot P) \otimes Q \equiv x \cdot (P \otimes Q) \equiv P \otimes (x \cdot Q)
    \and
    P \otimes \pzero \equiv P
  \end{mathpar}

  that the reader is invited to verify.
\end{remark}

\subsubsection{Annihilation}
\begin{mathpar}
  P^{\perp} := \{ Q | \forall R. P|Q \red^{*} R \Rightarrow R \red^{*} \pzero \}
  \and \\
  P^{\underline{\perp}} := \Sigma_{Q \in P^{\perp}} \quotep{Q}?(y).(\dropn{y}|Q) | \Sigma_{Q \in P^{\perp}} \quotep{Q}\clift{\Box}
\end{mathpar}

\paragraph{Discussion} The reader will note that $P^{\perp}$ is a
\emph{set} of processes, while $P^{\underline{\perp}}$ is a
\emph{context}. We call the set $P^{\perp}$ the \emph{annihilators} of
$P$. The parallel composition of a process in the annihilators of $P$
with $P$ will result in a process, the state space of which has all
paths eventually leading to $\pzero$. Execution may endure loops; but
under reasonable conditions of fairness (naturally guaranteed under
most notions of bisimulation) such a composite process cannot get
stuck in such a loop and will, eventually pop out and terminate.

The context $P^{\underline{\perp}}$ is ready and willing to ``take the
$P$ out of'' the process to which it is applied. It will effectively
transmit the code of the process to which it is applied to one of the
annihilators and run the process against it.

\subsubsection{Evaluation}
We fix $M$ a domain of fully abstract interpretation with an equality
coincident with bisimulation. We take $\meaningof{\cdot} : \Proc \to
M$ to be the map interpreting processes and $\nmeaningof{\cdot} : \M
\to Proc$ to be the map running the other way. Then we define

\begin{mathpar}
  \int P := \nmeaningof{\meaningof{P}}
\end{mathpar}

\paragraph{Discussion}
There are many fully abstract interpretations of Milner's
$\pi$-calculus. Any of them can be used as a basis for interpreting
the reflective calculus here. Equipped with such a domain it is
largely a matter of grinding through to check that the Yoneda
construction for the normalization-by-evaluation program can be
extended to this setting.

\begin{remark}
  The reader is invited to verify that $\int (P^{\underline{\perp}}[P]) = 0$.
\end{remark}

\subsection{Quantum mechanics}

Table \ref{tbl:core_qm_op_defns} gives the core operational definitions

\begin{table}[htp]\label{tbl:core_qm_op_defns}
  \center{
    \fbox{
      \begin{tabular}{c|c}
        quantum mechanics & process calculus \\
        \hline
        scalar & $x := \quotep{P}$ \\
        state vector & $\state{P} := P$ \\
        dual & $\state{P}^{*} := \event{P^{\underline{\perp}}} := \quotep{P^{\underline{\perp}}}[-]$ \\
        matrix & $ \Sigma_{\alpha} \state{P_{\alpha}}x_{\alpha}\event{Q_{\alpha}}$ \\
        vector addition & $\state{P} + \state{Q} := \state{P | Q}$ \\
        tensor product & $\state{P} \otimes \state{Q} := \state{P \otimes Q}$ \\
        inner product & $\innerprod{P}{Q} := \quotep{\int P^{\underline{\perp}}[Q]}$ \\
      \end{tabular}
    }
  }
  \caption{QM - operational definitions}
\end{table}

where

\begin{mathpar}
  \prmatrix{P}{Q} := \fprmatrix{P}{\quotep{\pzero}}{Q}
  \and
  \fprmatrix{P}{x}{Q} := (\state{P},x,\event{Q})
  \and
  (\fprmatrix{P}{x}{Q})(\state{R}) := x \cdot \innerprod{Q}{R} \cdot \state{P}
  \and
  (\fprmatrix{P}{x}{Q})(\event{R}) := x \cdot \innerprod{R}{P} \cdot \event{Q}
\end{mathpar}

\paragraph{Discussion}
As promised: vectors (aka states) are represented as processes; duals
as contextual duals; inner product definition should be compared with
standard inner product definition for ....

\begin{remark}
  Assuming $\int (P^{\underline{\perp}}[P]) = 0$, the reader is
  invited to verify that $(\fprmatrix{P}{x}{P})(\state{P}) = x \cdot \state{P}$.
\end{remark}

\begin{remark}
  The reader is invited to verify that $\innerprod{P}{Q}$ could
  equally well have been written $\quotep{\int \stackrel{\vee}{x}}$
  where $x = \event{P^{\underline{\perp}}}(Q)$.

  One of the motivations for this remark is that there is another way
  to factor these operations. We could package up evaluation in the dual:

  \begin{mathpar}
    \state{P}^{*} := \event{\int P^{\underline{\perp}}} := \quotep{\int P^{\underline{\perp}}}[-]
  \end{mathpar}

  and then have inner product defined by
  
  \begin{mathpar}
    \innerprod{P}{Q} := \event{P}(Q)
  \end{mathpar}

  Hopefully, experience with the calculations will provide guidance on
  the best factoring.
\end{remark}

\begin{remark}
  Assuming $\int (P^{\underline{\perp}}[P]) = 0$, the reader is
  invited to verify that $\forall P,Q. (\prmatrix{0}{Q})(\state{0}) =
  \state{0}$ and dually $(\prmatrix{P}{0})(\event{0}) = \event{0}$.
\end{remark}

\begin{remark}
  i'm a little worried that i don't (yet) have proper support for
  complex conjugacy. But, the observation above may give us a
  clue. According to Abramsky, it must be the case that the scalars
  are iso to the homset of the identity for the tensor -- which the
  observation above characterizes. 

  For now, we will simply bookmark the notion with $\overline{x}$.
\end{remark}

\subsubsection{Adjointness}

We need to give a definition of $(\cdot)^{\dagger}$ for matrices. The
obvious candidate definition is
\begin{mathpar}
(\Sigma_{\alpha}\fprmatrix{P_{\alpha}}{x_{\alpha}}{Q_{\alpha}})^{\dagger}
= \Sigma_{\alpha}\fprmatrix{(Q_{\alpha}^{\underline{\perp}})^{*}}{\overline{x}_{\alpha}}{P_{\alpha}^{\underline{\perp}}} 
\end{mathpar}

But, $(Q_{\alpha}^{\underline{\perp}})^{*}$ requires a name along
which to communicate the process to achieve the context application.

\subsubsection{Basis for a basis}
If processes label states and ``addition'' of states (a.k.a. vector
addition) is interpreted as parallel composition, what corresponds to
notions of linear independence and basis? Here, we recall that Yoshida
has developed a set of \emph{combinators} for an asynchronous verison
of Milner's $\pi$-calculus. These are a finite set of processes such
any process can be expressed as parallel composition of these
combinators together with liberal uses of the new operator and
replication. We can simply give a translation of these into the
present calculus and have reasonable expectation that the property
carries over. That is, that the resultant set allows to express all
processes via parallel composition. Note, however, that there is no
new operator or replication in this calculus. As a result, we expect
that the corresponding set is actually infinite. That is, we expect
that the space is actually infinite dimensional.

\begin{remark}
  The attentive reader may be a bit concerned. Certainly, the
  collection $S$, $K$ and $I$ is a finite set of
  combinators. Shouldn't we expect to see a finite set of combinators
  for an effectively equivalent system? i am very sympathetic to this
  critique and feel it warrants full attention. On the other hand, i
  also have in mind the following analogy. The natural numbers, as a
  monoid under addition, has exactly $1$ generator, while the natural
  numbers, as a monoid under multiplication, has countably many
  generators (the primes). We observe that the application of the
  lambda calculus is much less resource sensitive than the parallel
  composition of the $\pi$-calculus. Could it be the case that we have
  an analogy of the form
  
  \begin{mathpar}
    m + n : MN :: m*n : M|N
  \end{mathpar}

  giving a similar blow up in the set of ``primes''?  This is such a
  wonderful thought that, even if it's not true, i think it's worth
  writing down.
\end{remark}
 

\documentclass[12pt]{llncs}
%\documentclass{jktr}

\usepackage[pdftex]{hyperref}                   
\usepackage {listings}
\usepackage {mathpartir}
\usepackage{bcprules}
%\usepackage{listings}
                       
\usepackage{graphicx} 
%\usepackage[margins=2.5cm,nohead,nofoot]{geometry}
%\usepackage{geometry}
\usepackage{amsfonts}
\usepackage{amstext}
\usepackage{latexsym}
\usepackage{amssymb}
\usepackage{color}


%\include{myPreamble}
\include{qm2pi.local} 

%\ifpdf
%\usepackage[pdftex]{graphicx}
%\else
%\usepackage{graphicx}
%\fi

 % \ifpdf
%  \usepackage{pdfsync}
%  \if


%\title{Brief Article}
%\author{David F. Snyder}
%\author{L.G. Meredith}

%\address{Dept. of Math., Texas State University--San Marcos, San Marcos, TX 78666}
       
\pagestyle{empty}


\begin{document}

\lstset{language=[Objective]Caml,frame=shadowbox}

\input{qm2pi.front}

% section front matter (end)

\input{qm2pi.intro} 
 
% section introduction (end)

% \input{qm2pi.knotations} 

% section notation (end)

\input{qm2pi.process.calculi} 

% section concurrent_process_calculi_and_spatial_logics_ (end)
    
%\input{qm2pi.knots2pi} 

%\input{qm2pi.trefoil} 

%\input{qm2pi.mainthm} 

% subsection basic_interpretation (end)

%\input{qm2pi.rho.presentation} 
\subsection{The syntax and semantics of the notation system}\label{sub:the_syntax_and_semantics_of_the_notation_system} % (fold)

We now summarize a technical presentation of the calculus that
embodies our theory of dynamics. The typical presentation of such a
calculus follows the style of giving generators and relations on
them. The grammar, below, describing term constructors, freely
generates the set of processes, $\Proc$. This set is then quotiented
by a relation known as structural congruence and it is over this set
that the notion of dynamics is expressed. This presentation is
essentially that of \cite{MeredithR05} with the addition of
polyadicity and summation. For readability we have relegated some of
the technical subtleties to an appendix.

\subsubsection{Process grammar}\label{subsub:process_grammar}

\begin{mathpar}
  \inferrule* [lab=synchronization] {} {{M} \bc \pzero \;|\; x?F \;|\; x!C }
  \and
  \inferrule* [lab=abstraction] {} {{F} \bc (x)P}
  \and
  \inferrule* [lab=concretion] {} {{C} \bc \langle Q \rangle}
  \and
  \inferrule* [lab=process] {} {{P,Q} \bc M \;| \;P|Q \;|\; @{x}}
  \and
  \inferrule* [lab=name] {} {{x} \bc \quotep{P}}
\end{mathpar} 

Note that $\vec{x}$ (resp. $\vec{P}$) denotes a vector of names
(resp. processes) of length $|\vec{x}|$ (resp. $|\vec{P}|$). We adopt
the following useful abbreviations.

\begin{mathpar}
   x?(\vec{y}).P := x.(\vec{y})P \and  x\clift{\vec{P}} := x.\clift{\vec{P}}
   \and x!(y) := \lift{x}{\dropn{y}}
   \and \Pi_{i=0}^{n-1}P_i := P_0 | \ldots | P_{n-1}
\end{mathpar}

\subsubsection{Structural congruence}

\paragraph{Free and bound names and alpha-equivalence.} At the
core of structural equivalence is alpha-equivalence which identifies
process that are the same up to a change of variable. Formally, we
recognize the distinction between free and bound names. The free names
of a process, $\freenames{P}$, may be calculated recursively as
follows:

\begin{mathpar}
\freenames{\pzero} := \emptyset
  \and \\
  \freenames{x?(y).P} := \{ x \} \cup (\freenames{P} \setminus \{ y \})
  \and 
  \freenames{x!\langle P \rangle} := \{ x \} \cup \{ P \} 
  \and \\
  \freenames{P|Q} := \freenames{P} \cup \freenames{Q}
  \and \\
  \freenames{@{x}} := \{ x \}
\end{mathpar}

$\pi$
$\quotep{\pi}$

$\freenames{-} : \pi \to \mathcal{P}(\quotep{\pi})$

\begin{eqnarray*}
  \freenames{\pzero} & := & \emptyset \\
  \freenames{x?(y).P} & := & \{ x \} \cup (\freenames{P} \setminus \{ y \}) \\
  \freenames{x!\langle P \rangle} & := & \{ x \} \cup \{ P \} \\
  \freenames{P|Q} & := & \freenames{P} \cup \freenames{Q} \\
  \freenames{\dropn{x}} & := & \{ x \}
\end{eqnarray*}

The bound names of a process, $\boundnames{P}$, are those names occurring in $P$
that are not free. For example, in $x?(y).0$, the name $x$ is free, while $y$ is bound.

\begin{mathpar}
  \inferrule* [lab=monoidal-laws] {} { P|Q \equiv Q|P \and P|0 \equiv P \and P|(Q|R) \equiv (P|Q)|R }
\end{mathpar}

\begin{mathpar}
  \inferrule* [lab=alpha-equivalence] {} { (x)P \equiv (y)P\{y/x\} \and y \not\in \freenames{P} }
\end{mathpar}

\begin{definition}
Then two processes, $P,Q$, are alpha-equivalent if $P = Q\{\vec{y}/\vec{x}\}$ for
some $\vec{x} \in \boundnames{Q},\vec{y} \in \boundnames{P}$, where $Q\{\vec{y}/\vec{x}\}$
denotes the capture-avoiding substitution of $\vec{y}$ for $\vec{x}$ in $Q$.
\end{definition}

\begin{definition}
  The {\em structural congruence} \cite{SangiorgiWalker} , $\equiv$,
  between processes is the least congruence containing
  alpha-equivalence, satisfying the abelian monoid laws
  (associativity, commutativity and $\pzero$ as identity) for parallel
  composition $|$ and for summation $+$.
\end{definition}

\subsection{Name equivalence}

We take name equivalence, written $\nameeq$, to be the smallest
equivalence relation generated by the following rules.

\begin{mathpar}
\inferrule*[lab=Quote-drop]
{ }
{ \quotep{@{x}} \nameeq x }

\inferrule*[lab=Struct-equiv]
{ P \scong Q }
{ \quotep{P} \nameeq \quotep{Q} }
\end{mathpar}

The astute reader will have noticed that the mutual recursion of names
and processes imposes a mutual recursion on alpha-equivalence and
structural equivalence via name-equivalence. Fortunately, all of this
works out pleasantly and we may calculate in the natural way, free of
concern. The reader interested in the details is referred to the
appendix \ref{appendix:rho_details}.

\subsection{Substitution}

We use $\Proc$ for the set of processes, $\QProc$ for the set of
names, and $\id{\{}\vec{y} / \vec{x} \id{\}}$ to denote partial maps,
$s : \QProc \rightarrow \QProc$. A map, $s$ lifts, uniquely, to a map
on process terms, $\widehat{s} : \Proc \rightarrow \Proc$ by the
following equations.

\begin{mathpar}
  (0) \psubstp{Q}{P} := 0 \\
  (R \juxtap S) \psubstp{Q}{P}
  :=    
  (R)\psubstp{Q}{P} \juxtap (S) \psubstp{Q}{P} \\
  (x?(y).R) \psubstp{Q}{P}    
  :=    
  (x)\substp{Q}{P} (z)\concat( (R \psubstn{z}{y}) \psubstp{Q}{P} ) \\
  (\lift{x}{R}) \psubstp{Q}{P}  
  :=
  \lift{(x)\substp{Q}{P}}{ R \psubstp{Q}{P} } \\
%   (\dropn{x})  \psubstp{Q}{P}       
%   := 
%   \left\{ 
%     \begin{array}{ccc} 
%       \dropn{\quotep{Q}} & & x \nameeq \quotep{P} \\
%       \dropn{x} & & otherwise \\
%     \end{array}
%   \right. 
  (\dropn{x})  \psubstp{Q}{P}       
  := 
  \left\{ 
    \begin{array}{ccc} 
      Q & & x \nameeq \quotep{P} \\
      \dropn{x} & & otherwise \\
    \end{array}
  \right.
\end{mathpar}
 

where

\begin{eqnarray}
  (x)\id{\{} \lpquote Q \rpquote / \lpquote P \rpquote \id{\}}            = 
  \left\{ 
    \begin{array}{ccc}
      \lpquote Q \rpquote & & x \nameeq \lpquote P \rpquote \\
      x & & otherwise \\
    \end{array}
  \right. \nonumber
\end{eqnarray}

and $z$ is chosen distinct from $\quotep{P}$, $\quotep{Q}$, the free
names in $Q$, and all the names in $R$. Our $\alpha$-equivalence will
be built in the standard way from this substitution.

\begin{remark}\label{rem:no_self_referential_names}
  One consequence of these definitions is that $\forall P. \quotep{P}
  \not\in \freenames{P}$.
\end{remark}

\subsection{ Dynamic quote: an example }

Anticipating something of what's to come, consider applying the
substitution, $\widehat{\id{\{}u / z \id{\}}}$, to the following pair
of processes, $\lift{w}{y!(z)}$ and $w[ \lpquote y!(z) \rpquote ]$.

\begin{eqnarray}
	\lift{w}{y!(z)}\widehat{\id{\{}u / z \id{\}}}
		& = &
		\lift{w}{y!(u)} \nonumber\\
	w[ \lpquote y!(z) \rpquote ] \widehat{ \id{\{}u / z \id{\}} }
		& = &
		w[ \lpquote y!(z) \rpquote ] \nonumber
\end{eqnarray}

Because the body of the process between quotes is impervious to
substitution, we get radically different answers. In fact, by
examining the first process in an input context,
e.g. $x?(z).\lift{w}{y!(z)}$, we see that the process under the lift
operator may be shaped by prefixed inputs binding a name inside it. In
this sense, the lift operator will be seen as a way to dynamically
construct processes before reifying them as names.

Finally equipped with these standard features we can present the
dynamics of the calculus.

\subsubsection{Operational semantics} 

Finally, we introduce the computational dynamics. What marks these
algebras as distinct from other more traditionally studied algebraic
structures, e.g. vector spaces or polynomial rings, is the manner in
which dynamics is captured. In traditional structures, dynamics is typically
expressed through morphisms between such structures, as in linear maps
between vector spaces or morphisms between rings. In algebras
associated with the semantics of computation, the dynamics is
expressed as part of the algebraic structure itself, through a
reduction reduction relation typically denoted by $\red$. Below, we
give a recursive presentation of this relation for the calculus used
in the encoding.

$\red \subseteq \pi \times \pi$
$\red : \pi \to \mathcal{P}(\pi)$

\begin{mathpar}
  \inferrule* [lab=Comm] { \textsf{match}( x_{src}, x_{trgt} ) } { x_{trgt}?(y)P \; | \; x_{src}!\langle {Q} \rangle \red P\{\quotep{Q}/y}\} }
  \and \\
  \inferrule* [lab=Par] {{P} \red {P}'} {{{P} | {Q}} \red {{P}' | {Q}}}
  \and
  \inferrule* [lab=Equiv]{{{P} \scong {P}'} \andalso {{P}' \red {Q}'} \andalso {{Q}' \scong {Q}}}{{P} \red {Q}}
\end{mathpar}

\begin{eqnarray*}
  match_{\equiv} (\quotep{P},\quotep{Q}) & := & P \equiv Q \\
  match_{\dagger}(\quotep{P},\quotep{Q}) & := & \forall R. P|Q \red^{*} R => R \red^{*} 0 \\
  match_{K}(\quotep{P},\quotep{Q}) & := & K \mbox{ for some context } K
\end{eqnarray*}

$u?(x)P | u!\langle Q \rangle \red P\{\quotep{Q}/x\}$

%We write $\wred$ for $\red^*$, and $P\red$ if $\exists Q $ such that $ P \red Q$.
We write $P\red$ if $\exists Q $ such that $ P \red Q$ and $P\not\red$, otherwise.

\section{Replication}

As mentioned before, it is known that replication (and hence
recursion) can be implemented in a higher-order process algebra
\cite{SangiorgiWalker}. As our first example of calculation with the
machinery thus far presented we give the construction explicitly in
the {\rhoc}.

\begin{eqnarray}
	D_{x} & := & \prefix{x}{y}{(\binpar{\outputp{x}{y}}{@{y}})} \nonumber\\
	\bangp_{x}{P} & := & \binpar{{x}!\langle{\binpar{D_{x}}{P}}\rangle}{D_{x}} \nonumber
\end{eqnarray}

\begin{eqnarray}
	\bangp_{x}{P} & & \nonumber\\
	=
	& {x}!\langle{(\prefix{x}{y}{(\outputp{x}{y} | @{y})) | P}}\rangle 
	      | \prefix{x}{y}{(\outputp{x}{y} | @{y})} & \nonumber\\
	\red
	& (\outputp{x}{y} | @{y})\substn{\quotep{(\prefix{x}{y}{(@{y} | \outputp{x}{y})) | P}}}{y} & \nonumber\\
	=
	& \outputp{x}{\quotep{(\prefix{x}{y}{(\outputp{x}{y} | @{y})) | P}}}
	  | {(\prefix{x}{y}{(\outputp{x}{y} | @{y})) | P}} & \nonumber\\
	\red
	& \ldots & \nonumber\\
	\red^*
	& P | P | \ldots & \nonumber
\end{eqnarray}

Of course, this encoding, as an implementation, runs away, unfolding
$\bangp{P}$ eagerly. A lazier and more implementable replication
operator, restricted to input-guarded processes, may be obtained as follows.

\begin{eqnarray}
\bangp{\prefix{u}{v}{P}} 
	:= 
	\binpar{\lift{x}{\prefix{u}{v}{(\binpar{D(x)}{P})}}}{D(x)} \nonumber
\end{eqnarray}

\begin{remark}
  Note that the lazier definition still does not deal with summation
  or mixed summation (i.e. sums over input and output). The reader is
  invited to construct definitions of replication that deal with these
  features. 

  Further, the definitions are parameterized in a name, $x$. Can you,
  gentle reader, make a definition that eliminates this parameter and
  guarantees no accidental interaction between the replication
  machinery and the process being replicated -- i.e. no accidental
  sharing of names used by the process to get its work done and the
  name(s) used by the replication to effect copying. This latter
  revision of the definition of replication is crucial to obtaining
  the expected identity $!!P \sim !P$.
\end{remark}

\begin{remark}\label{rem:paradoxical_combinator}
  The reader familiar with the lambda calculus will have noticed the
  similarity between $D$ and the paradoxical combinator.

  [Ed. note: the existence of this seems to suggest we have to be more
  restrictive on the set of processes and names we admit if we are to
  support no-cloning.]
\end{remark}

\subsubsection{Bisimulation}

The computational dynamics gives rise to another kind of equivalence,
the equivalence of computational behavior. As previously mentioned
this is typically captured \emph{via} some form of bisimulation.

% The notion we use in this paper is weak barbed bisimulation
% \cite{milner91polyadicpi}.

The notion we use in this paper is derived from weak barbed
bisimulation \cite{milner91polyadicpi}. 

\begin{definition}
An \emph{observation relation}, $\downarrow_{\mathcal N}$, over a set
of names, $\mathcal N$, is the smallest relation satisfying the rules
below.

\infrule[Out-barb]{y \in {\mathcal N}, \; x \nameeq y}
		  {\outputp{x}{v} \downarrow_{\mathcal N} x}
\infrule[Par-barb]{\mbox{$P\downarrow_{\mathcal N} x$ or $Q\downarrow_{\mathcal N} x$}}
		  {\binpar{P}{Q} \downarrow_{\mathcal N} x}

We write $P \Downarrow_{\mathcal N} x$ if there is $Q$ such that 
$P \wred Q$ and $Q \downarrow_{\mathcal N} x$.
\end{definition}

\begin{definition}
%\label{def.bbisim}
An  ${\mathcal N}$-\emph{barbed bisimulation} over a set of names, ${\mathcal N}$, is a symmetric binary relation 
${\mathcal S}_{\mathcal N}$ between agents such that $P\rel{S}_{\mathcal N}Q$ implies:
\begin{enumerate}
\item If $P \red P'$ then $Q \wred Q'$ and $P'\rel{S}_{\mathcal N} Q'$.
\item If $P\downarrow_{\mathcal N} x$, then $Q\Downarrow_{\mathcal N} x$.
\end{enumerate}
$P$ is ${\mathcal N}$-barbed bisimilar to $Q$, written
$P \wbbisim_{\mathcal N} Q$, if $P \rel{S}_{\mathcal N} Q$ for some ${\mathcal N}$-barbed bisimulation ${\mathcal S}_{\mathcal N}$.
\end{definition}

$\mathcal{R} \subseteq \pi \times \pi$

$P \mathcal{R} Q => \forall P'. P \red P' \Rightarrow \exists Q'. Q \red Q', P' \mathcal{R} Q'$

$P \vdash x \Rightarrow Q \vdash x$

\begin{mathpar}
  \inferrule*[lab=Out-barb]{x \nameeq y}{{y}!\langle{Q}\rangle \vdash x}
  \and
  \inferrule*[lab=Par-barb]{\mbox{$P\vdash x$ or $Q\vdash x$}}{\binpar{P}{Q} \vdash x}
\end{mathpar}

\subsubsection{Contexts}

One of the principle advantages of computational calculi like the
$\pi$-calculus is a well-defined notion of context,
contextual-equivalence and a correlation between
contextual-equivalence and notions of bisimulation. The notion of
context allows the decomposition of a process into (sub-)process and
its syntactic environment, its context. Thus, a context may be
thought of as a process with a ``hole'' (written $\Box$) in it. The
application of a context $M$ to a process $P$, written $M[P]$, is
tantamount to filling the hole in $M$ with $P$. In this paper we do
not need the full weight of this theory, but do make use of the notion
of context in the proof the main theorem. 

\begin{mathpar}
  \inferrule* [lab=summation] {} {{M_{M},M_{N}} \bc \Box \;|\; x.M_{A} \;|\; M_{M}+M_{N}}
  \and
  \inferrule* [lab=agent] {} {{M_{A}} \bc (\vec{x})M_{P} \;| \; \clift{P_0,\ldots,M_{P},\ldots,P_N}}
  \and \\
  \inferrule* [lab=process] {} {{M_{P}} \bc M_{N} \;| \;P|M_{P} }
\end{mathpar} 

\begin{mathpar}
  \inferrule* [lab=sychronization] {} {M_{N} \bc \Box \;|\; x?M_{F} \;|\; x!M_{C}}
  \and
  \inferrule* [lab=abstraction] {} {{M_{F}} \bc (x)M_{P} }
  \and
  \inferrule* [lab=concretion] {} {{M_{C}} \bc \langle M_{P} \rangle }
  \and \\
  \inferrule* [lab=process] {} {{M_{P}} \bc M_{N} \;| \;P|M_{P} }
\end{mathpar}

\begin{definition}[contextual application] Given a context $M$, and
  process $P$, we define the \emph{contextual application}, $M[P] :=
  M\{P/\Box\}$. That is, the contextual application of M to P is the
  substitution of $P$ for $\Box$ in $M$.
\end{definition}

$\meaningof{-} : L \to \mathcal{P}(\pi)$

\begin{mathpar}
  \inferrule* [lab=collection] {} {\meaningof{true} = \pi, \and \meaningof{~E} = \pi \setminus \meaningof{E}, \and \meaningof{E_{1} \& E_{2}} = \meaningof{E_{1}} \cap \meaningof{E_{2}}}
\end{mathpar}

\begin{mathpar}
  \inferrule* [lab=structure] {} {\meaningof{0} = \{ P \in \pi | P \equiv 0 \}, \and \\ \meaningof{E_1 | E_2} = \{ P \in \pi | P \equiv P_{1} | P_{2}, P_{1} \in \meaningof{E_{1}}, P_{2} \in \meaningof{E_2}\} }
\end{mathpar}

\begin{mathpar}
 \inferrule* [lab=behavior] {} {\meaningof{\langle a?b \rangle E} = \{ P \in \pi | P \equiv Q | u?(y)P', \\ \and \\\\ \and \\ \;\;\; u \in \meaningof{a}, \forall z.P'\{z/y\} \in \meaningof{E\{z/b\}}\}, \and \\ \meaningof{a!E} = \{ P \in \pi | P \equiv Q | x!\langle P' \rangle, x \in \meaningof{a} P' \in \meaningof{E}\} }
\end{mathpar}

\begin{mathpar}
 \inferrule* [lab=nominal] {} {\meaningof{\quotep{E}} = \{ \quotep{P} \in \quotep{\pi} | P \in \meaningof{E} \}, \and \meaningof{\quotep{P}} = \{ \quotep{Q} \in \quotep{\pi} | P \equiv Q \} \and \\ \meaningof{@\quotep{E}} = \{ P \in \pi | P \equiv @x, x \in \meaningof{E} \}}
\end{mathpar}

\begin{eqnarray*}
  \\
  \meaningof{-} : TS \to ST
\end{eqnarray*}

\begin{eqnarray*}
  \\
  L : TS \to ST
\end{eqnarray*}

\begin{eqnarray*}
  \\
  P \models E \iff P \in \meaningof{E}
\end{eqnarray*}

\begin{eqnarray*}
  P \approx_{L} Q \iff \forall E \in L. P \models E \iff Q \models E
\end{eqnarray*}

\begin{eqnarray*}
  P \approx_{K} Q
\end{eqnarray*}

\begin{eqnarray*}
  P \approx Q
\end{eqnarray*}

$\approx_{K} = \approx = \approx_{L}$

\subsubsection{Contextual duality}

Note that contexts extend the quotation operation to a family of
operations from processes to names. Given a context, $M$, we can
define a \emph{nominal context}, $\quotep{M}$ by $\quotep{M}[P] :=
\quotep{M[P]}$. To foreshadow what is to come we observe that these
operations enjoy a duality with processes very much like the duality
between vectors and maps from vectors to scalars.

Further, because the calculus is essentially higher-order, we have a
correspondence between contexts and processes. More specifically,
given a name $x$ and a context $M$ we can construct $M^{*}_{x}$ such
that 

\begin{mathpar}
  M^{*}_{x} | \lift{x}{P} \red M[P]
\end{mathpar}

namely,

\begin{mathpar}
  M^{*}_{x} := x?(u).M[\dropn{u}]
\end{mathpar}

The dependence of $M^{*}_{x}$ on a name makes it an abstraction, 

\begin{mathpar}
  M^{*} := (x)x?(u).M[\dropn{u}]
\end{mathpar}

\subsection{Additional notation}

It will sometimes be convenient to denote the process a name
quotes. We already have the notation $x = \quotep{P}$, but it will be
convenient to introduce an alternate notation, $\procn{x}$, when we
want to emphasize the connection to the use of the name. Note that, by
virtue of name equivalence, $\quotep{\procn{x}} \nameeq x$; so, the
notation is consistent with previous definitions.

Further, because names have structure it is possible to effect
substitutions on the basis of that structure. This means we need to
upgrade our notation for substitutions, which we accomplish by
adapting comprehension notation. Thus,

\begin{mathpar}
  P\{ y / x : x \in S \}
\end{mathpar}

is interpreted to mean the process derived from P by replacing (in a
capture-avoiding manner) each occurrence of $x$ in $S$ by $y$. For example,

\begin{mathpar}
  P\{ \quotep{\procn{x}|\procn{x}} / x : x \in \freenames{P} \}
\end{mathpar}

will replace each (occurrence) of a free name $x$ in $P$ by
$\quotep{\procn{x}|\procn{x}}$.

Also, we will avail ourselves of the notation $x^{L}$ and $x^{R}$ to
denote injections of a name into disjoint copies of the name
space. There are numerous ways to accomplish this. One example can be
found in \cite{MeredithR05}. This notation overloads to vectors of
names: $\vec{x}^{\pi} := (x_{i}^{\pi} \; : \; 0 \leq i < |\vec{x}| )$ where $\pi \in \{L,R\}$.

We also use $P^{\Box} := P|\Box$.

In \cite{MeredithR05} an interpretation of the new operator is
given. It turns out that there are several possible interpretations
all enjoying the requisite algebraic properties of the operator (see
\cite{milner91polyadicpi}). We will therefore make liberal use of
$(\nu\; \vec{x})P$.

% subsection the_syntax_and_semantics_of_the_notation_system (end)   

\input{qm2pi.qmops} 

\input{qm2pi.sterngerlach} 

\input{qm2pi.metric} 

% section concurrent_process_calculi (end)

%\input{qm2pi.proofsketch}

% section proof sketch (end)

%\input{qm2pi.slviaknots} 

% section spatial logic via knots (end)

\input{qm2pi.conclusion}

% section conclusion (end)

%\input{qm2pi.dtcodes} 

% section wiring algorithm (end)

\input{qm2pi.ack} 

% section acknowledgments (end)

\newpage


\bibliographystyle{plain}   
\bibliography{../../biblios/main.bib}

\input{qm2pi.rhodetails}

\end{document}

 

\documentclass[12pt]{llncs}
%\documentclass{jktr}

\usepackage[pdftex]{hyperref}                   
\usepackage {listings}
\usepackage {mathpartir}
\usepackage{bcprules}
%\usepackage{listings}
                       
\usepackage{graphicx} 
%\usepackage[margins=2.5cm,nohead,nofoot]{geometry}
%\usepackage{geometry}
\usepackage{amsfonts}
\usepackage{amstext}
\usepackage{latexsym}
\usepackage{amssymb}
\usepackage{color}


%\include{myPreamble}
\include{qm2pi.local} 

%\ifpdf
%\usepackage[pdftex]{graphicx}
%\else
%\usepackage{graphicx}
%\fi

 % \ifpdf
%  \usepackage{pdfsync}
%  \if


%\title{Brief Article}
%\author{David F. Snyder}
%\author{L.G. Meredith}

%\address{Dept. of Math., Texas State University--San Marcos, San Marcos, TX 78666}
       
\pagestyle{empty}


\begin{document}

\lstset{language=[Objective]Caml,frame=shadowbox}

\input{qm2pi.front}

% section front matter (end)

\input{qm2pi.intro} 
 
% section introduction (end)

% \input{qm2pi.knotations} 

% section notation (end)

\input{qm2pi.process.calculi} 

% section concurrent_process_calculi_and_spatial_logics_ (end)
    
%\input{qm2pi.knots2pi} 

%\input{qm2pi.trefoil} 

%\input{qm2pi.mainthm} 

% subsection basic_interpretation (end)

%\input{qm2pi.rho.presentation} 
\subsection{The syntax and semantics of the notation system}\label{sub:the_syntax_and_semantics_of_the_notation_system} % (fold)

We now summarize a technical presentation of the calculus that
embodies our theory of dynamics. The typical presentation of such a
calculus follows the style of giving generators and relations on
them. The grammar, below, describing term constructors, freely
generates the set of processes, $\Proc$. This set is then quotiented
by a relation known as structural congruence and it is over this set
that the notion of dynamics is expressed. This presentation is
essentially that of \cite{MeredithR05} with the addition of
polyadicity and summation. For readability we have relegated some of
the technical subtleties to an appendix.

\subsubsection{Process grammar}\label{subsub:process_grammar}

\begin{mathpar}
  \inferrule* [lab=synchronization] {} {{M} \bc \pzero \;|\; x?F \;|\; x!C }
  \and
  \inferrule* [lab=abstraction] {} {{F} \bc (x)P}
  \and
  \inferrule* [lab=concretion] {} {{C} \bc \langle Q \rangle}
  \and
  \inferrule* [lab=process] {} {{P,Q} \bc M \;| \;P|Q \;|\; @{x}}
  \and
  \inferrule* [lab=name] {} {{x} \bc \quotep{P}}
\end{mathpar} 

Note that $\vec{x}$ (resp. $\vec{P}$) denotes a vector of names
(resp. processes) of length $|\vec{x}|$ (resp. $|\vec{P}|$). We adopt
the following useful abbreviations.

\begin{mathpar}
   x?(\vec{y}).P := x.(\vec{y})P \and  x\clift{\vec{P}} := x.\clift{\vec{P}}
   \and x!(y) := \lift{x}{\dropn{y}}
   \and \Pi_{i=0}^{n-1}P_i := P_0 | \ldots | P_{n-1}
\end{mathpar}

\subsubsection{Structural congruence}

\paragraph{Free and bound names and alpha-equivalence.} At the
core of structural equivalence is alpha-equivalence which identifies
process that are the same up to a change of variable. Formally, we
recognize the distinction between free and bound names. The free names
of a process, $\freenames{P}$, may be calculated recursively as
follows:

\begin{mathpar}
\freenames{\pzero} := \emptyset
  \and \\
  \freenames{x?(y).P} := \{ x \} \cup (\freenames{P} \setminus \{ y \})
  \and 
  \freenames{x!\langle P \rangle} := \{ x \} \cup \{ P \} 
  \and \\
  \freenames{P|Q} := \freenames{P} \cup \freenames{Q}
  \and \\
  \freenames{@{x}} := \{ x \}
\end{mathpar}

$\pi$
$\quotep{\pi}$

$\freenames{-} : \pi \to \mathcal{P}(\quotep{\pi})$

\begin{eqnarray*}
  \freenames{\pzero} & := & \emptyset \\
  \freenames{x?(y).P} & := & \{ x \} \cup (\freenames{P} \setminus \{ y \}) \\
  \freenames{x!\langle P \rangle} & := & \{ x \} \cup \{ P \} \\
  \freenames{P|Q} & := & \freenames{P} \cup \freenames{Q} \\
  \freenames{\dropn{x}} & := & \{ x \}
\end{eqnarray*}

The bound names of a process, $\boundnames{P}$, are those names occurring in $P$
that are not free. For example, in $x?(y).0$, the name $x$ is free, while $y$ is bound.

\begin{mathpar}
  \inferrule* [lab=monoidal-laws] {} { P|Q \equiv Q|P \and P|0 \equiv P \and P|(Q|R) \equiv (P|Q)|R }
\end{mathpar}

\begin{mathpar}
  \inferrule* [lab=alpha-equivalence] {} { (x)P \equiv (y)P\{y/x\} \and y \not\in \freenames{P} }
\end{mathpar}

\begin{definition}
Then two processes, $P,Q$, are alpha-equivalent if $P = Q\{\vec{y}/\vec{x}\}$ for
some $\vec{x} \in \boundnames{Q},\vec{y} \in \boundnames{P}$, where $Q\{\vec{y}/\vec{x}\}$
denotes the capture-avoiding substitution of $\vec{y}$ for $\vec{x}$ in $Q$.
\end{definition}

\begin{definition}
  The {\em structural congruence} \cite{SangiorgiWalker} , $\equiv$,
  between processes is the least congruence containing
  alpha-equivalence, satisfying the abelian monoid laws
  (associativity, commutativity and $\pzero$ as identity) for parallel
  composition $|$ and for summation $+$.
\end{definition}

\subsection{Name equivalence}

We take name equivalence, written $\nameeq$, to be the smallest
equivalence relation generated by the following rules.

\begin{mathpar}
\inferrule*[lab=Quote-drop]
{ }
{ \quotep{@{x}} \nameeq x }

\inferrule*[lab=Struct-equiv]
{ P \scong Q }
{ \quotep{P} \nameeq \quotep{Q} }
\end{mathpar}

The astute reader will have noticed that the mutual recursion of names
and processes imposes a mutual recursion on alpha-equivalence and
structural equivalence via name-equivalence. Fortunately, all of this
works out pleasantly and we may calculate in the natural way, free of
concern. The reader interested in the details is referred to the
appendix \ref{appendix:rho_details}.

\subsection{Substitution}

We use $\Proc$ for the set of processes, $\QProc$ for the set of
names, and $\id{\{}\vec{y} / \vec{x} \id{\}}$ to denote partial maps,
$s : \QProc \rightarrow \QProc$. A map, $s$ lifts, uniquely, to a map
on process terms, $\widehat{s} : \Proc \rightarrow \Proc$ by the
following equations.

\begin{mathpar}
  (0) \psubstp{Q}{P} := 0 \\
  (R \juxtap S) \psubstp{Q}{P}
  :=    
  (R)\psubstp{Q}{P} \juxtap (S) \psubstp{Q}{P} \\
  (x?(y).R) \psubstp{Q}{P}    
  :=    
  (x)\substp{Q}{P} (z)\concat( (R \psubstn{z}{y}) \psubstp{Q}{P} ) \\
  (\lift{x}{R}) \psubstp{Q}{P}  
  :=
  \lift{(x)\substp{Q}{P}}{ R \psubstp{Q}{P} } \\
%   (\dropn{x})  \psubstp{Q}{P}       
%   := 
%   \left\{ 
%     \begin{array}{ccc} 
%       \dropn{\quotep{Q}} & & x \nameeq \quotep{P} \\
%       \dropn{x} & & otherwise \\
%     \end{array}
%   \right. 
  (\dropn{x})  \psubstp{Q}{P}       
  := 
  \left\{ 
    \begin{array}{ccc} 
      Q & & x \nameeq \quotep{P} \\
      \dropn{x} & & otherwise \\
    \end{array}
  \right.
\end{mathpar}
 

where

\begin{eqnarray}
  (x)\id{\{} \lpquote Q \rpquote / \lpquote P \rpquote \id{\}}            = 
  \left\{ 
    \begin{array}{ccc}
      \lpquote Q \rpquote & & x \nameeq \lpquote P \rpquote \\
      x & & otherwise \\
    \end{array}
  \right. \nonumber
\end{eqnarray}

and $z$ is chosen distinct from $\quotep{P}$, $\quotep{Q}$, the free
names in $Q$, and all the names in $R$. Our $\alpha$-equivalence will
be built in the standard way from this substitution.

\begin{remark}\label{rem:no_self_referential_names}
  One consequence of these definitions is that $\forall P. \quotep{P}
  \not\in \freenames{P}$.
\end{remark}

\subsection{ Dynamic quote: an example }

Anticipating something of what's to come, consider applying the
substitution, $\widehat{\id{\{}u / z \id{\}}}$, to the following pair
of processes, $\lift{w}{y!(z)}$ and $w[ \lpquote y!(z) \rpquote ]$.

\begin{eqnarray}
	\lift{w}{y!(z)}\widehat{\id{\{}u / z \id{\}}}
		& = &
		\lift{w}{y!(u)} \nonumber\\
	w[ \lpquote y!(z) \rpquote ] \widehat{ \id{\{}u / z \id{\}} }
		& = &
		w[ \lpquote y!(z) \rpquote ] \nonumber
\end{eqnarray}

Because the body of the process between quotes is impervious to
substitution, we get radically different answers. In fact, by
examining the first process in an input context,
e.g. $x?(z).\lift{w}{y!(z)}$, we see that the process under the lift
operator may be shaped by prefixed inputs binding a name inside it. In
this sense, the lift operator will be seen as a way to dynamically
construct processes before reifying them as names.

Finally equipped with these standard features we can present the
dynamics of the calculus.

\subsubsection{Operational semantics} 

Finally, we introduce the computational dynamics. What marks these
algebras as distinct from other more traditionally studied algebraic
structures, e.g. vector spaces or polynomial rings, is the manner in
which dynamics is captured. In traditional structures, dynamics is typically
expressed through morphisms between such structures, as in linear maps
between vector spaces or morphisms between rings. In algebras
associated with the semantics of computation, the dynamics is
expressed as part of the algebraic structure itself, through a
reduction reduction relation typically denoted by $\red$. Below, we
give a recursive presentation of this relation for the calculus used
in the encoding.

$\red \subseteq \pi \times \pi$
$\red : \pi \to \mathcal{P}(\pi)$

\begin{mathpar}
  \inferrule* [lab=Comm] { \textsf{match}( x_{src}, x_{trgt} ) } { x_{trgt}?(y)P \; | \; x_{src}!\langle {Q} \rangle \red P\{\quotep{Q}/y}\} }
  \and \\
  \inferrule* [lab=Par] {{P} \red {P}'} {{{P} | {Q}} \red {{P}' | {Q}}}
  \and
  \inferrule* [lab=Equiv]{{{P} \scong {P}'} \andalso {{P}' \red {Q}'} \andalso {{Q}' \scong {Q}}}{{P} \red {Q}}
\end{mathpar}

\begin{eqnarray*}
  match_{\equiv} (\quotep{P},\quotep{Q}) & := & P \equiv Q \\
  match_{\dagger}(\quotep{P},\quotep{Q}) & := & \forall R. P|Q \red^{*} R => R \red^{*} 0 \\
  match_{K}(\quotep{P},\quotep{Q}) & := & K \mbox{ for some context } K
\end{eqnarray*}

$u?(x)P | u!\langle Q \rangle \red P\{\quotep{Q}/x\}$

%We write $\wred$ for $\red^*$, and $P\red$ if $\exists Q $ such that $ P \red Q$.
We write $P\red$ if $\exists Q $ such that $ P \red Q$ and $P\not\red$, otherwise.

\section{Replication}

As mentioned before, it is known that replication (and hence
recursion) can be implemented in a higher-order process algebra
\cite{SangiorgiWalker}. As our first example of calculation with the
machinery thus far presented we give the construction explicitly in
the {\rhoc}.

\begin{eqnarray}
	D_{x} & := & \prefix{x}{y}{(\binpar{\outputp{x}{y}}{@{y}})} \nonumber\\
	\bangp_{x}{P} & := & \binpar{{x}!\langle{\binpar{D_{x}}{P}}\rangle}{D_{x}} \nonumber
\end{eqnarray}

\begin{eqnarray}
	\bangp_{x}{P} & & \nonumber\\
	=
	& {x}!\langle{(\prefix{x}{y}{(\outputp{x}{y} | @{y})) | P}}\rangle 
	      | \prefix{x}{y}{(\outputp{x}{y} | @{y})} & \nonumber\\
	\red
	& (\outputp{x}{y} | @{y})\substn{\quotep{(\prefix{x}{y}{(@{y} | \outputp{x}{y})) | P}}}{y} & \nonumber\\
	=
	& \outputp{x}{\quotep{(\prefix{x}{y}{(\outputp{x}{y} | @{y})) | P}}}
	  | {(\prefix{x}{y}{(\outputp{x}{y} | @{y})) | P}} & \nonumber\\
	\red
	& \ldots & \nonumber\\
	\red^*
	& P | P | \ldots & \nonumber
\end{eqnarray}

Of course, this encoding, as an implementation, runs away, unfolding
$\bangp{P}$ eagerly. A lazier and more implementable replication
operator, restricted to input-guarded processes, may be obtained as follows.

\begin{eqnarray}
\bangp{\prefix{u}{v}{P}} 
	:= 
	\binpar{\lift{x}{\prefix{u}{v}{(\binpar{D(x)}{P})}}}{D(x)} \nonumber
\end{eqnarray}

\begin{remark}
  Note that the lazier definition still does not deal with summation
  or mixed summation (i.e. sums over input and output). The reader is
  invited to construct definitions of replication that deal with these
  features. 

  Further, the definitions are parameterized in a name, $x$. Can you,
  gentle reader, make a definition that eliminates this parameter and
  guarantees no accidental interaction between the replication
  machinery and the process being replicated -- i.e. no accidental
  sharing of names used by the process to get its work done and the
  name(s) used by the replication to effect copying. This latter
  revision of the definition of replication is crucial to obtaining
  the expected identity $!!P \sim !P$.
\end{remark}

\begin{remark}\label{rem:paradoxical_combinator}
  The reader familiar with the lambda calculus will have noticed the
  similarity between $D$ and the paradoxical combinator.

  [Ed. note: the existence of this seems to suggest we have to be more
  restrictive on the set of processes and names we admit if we are to
  support no-cloning.]
\end{remark}

\subsubsection{Bisimulation}

The computational dynamics gives rise to another kind of equivalence,
the equivalence of computational behavior. As previously mentioned
this is typically captured \emph{via} some form of bisimulation.

% The notion we use in this paper is weak barbed bisimulation
% \cite{milner91polyadicpi}.

The notion we use in this paper is derived from weak barbed
bisimulation \cite{milner91polyadicpi}. 

\begin{definition}
An \emph{observation relation}, $\downarrow_{\mathcal N}$, over a set
of names, $\mathcal N$, is the smallest relation satisfying the rules
below.

\infrule[Out-barb]{y \in {\mathcal N}, \; x \nameeq y}
		  {\outputp{x}{v} \downarrow_{\mathcal N} x}
\infrule[Par-barb]{\mbox{$P\downarrow_{\mathcal N} x$ or $Q\downarrow_{\mathcal N} x$}}
		  {\binpar{P}{Q} \downarrow_{\mathcal N} x}

We write $P \Downarrow_{\mathcal N} x$ if there is $Q$ such that 
$P \wred Q$ and $Q \downarrow_{\mathcal N} x$.
\end{definition}

\begin{definition}
%\label{def.bbisim}
An  ${\mathcal N}$-\emph{barbed bisimulation} over a set of names, ${\mathcal N}$, is a symmetric binary relation 
${\mathcal S}_{\mathcal N}$ between agents such that $P\rel{S}_{\mathcal N}Q$ implies:
\begin{enumerate}
\item If $P \red P'$ then $Q \wred Q'$ and $P'\rel{S}_{\mathcal N} Q'$.
\item If $P\downarrow_{\mathcal N} x$, then $Q\Downarrow_{\mathcal N} x$.
\end{enumerate}
$P$ is ${\mathcal N}$-barbed bisimilar to $Q$, written
$P \wbbisim_{\mathcal N} Q$, if $P \rel{S}_{\mathcal N} Q$ for some ${\mathcal N}$-barbed bisimulation ${\mathcal S}_{\mathcal N}$.
\end{definition}

$\mathcal{R} \subseteq \pi \times \pi$

$P \mathcal{R} Q => \forall P'. P \red P' \Rightarrow \exists Q'. Q \red Q', P' \mathcal{R} Q'$

$P \vdash x \Rightarrow Q \vdash x$

\begin{mathpar}
  \inferrule*[lab=Out-barb]{x \nameeq y}{{y}!\langle{Q}\rangle \vdash x}
  \and
  \inferrule*[lab=Par-barb]{\mbox{$P\vdash x$ or $Q\vdash x$}}{\binpar{P}{Q} \vdash x}
\end{mathpar}

\subsubsection{Contexts}

One of the principle advantages of computational calculi like the
$\pi$-calculus is a well-defined notion of context,
contextual-equivalence and a correlation between
contextual-equivalence and notions of bisimulation. The notion of
context allows the decomposition of a process into (sub-)process and
its syntactic environment, its context. Thus, a context may be
thought of as a process with a ``hole'' (written $\Box$) in it. The
application of a context $M$ to a process $P$, written $M[P]$, is
tantamount to filling the hole in $M$ with $P$. In this paper we do
not need the full weight of this theory, but do make use of the notion
of context in the proof the main theorem. 

\begin{mathpar}
  \inferrule* [lab=summation] {} {{M_{M},M_{N}} \bc \Box \;|\; x.M_{A} \;|\; M_{M}+M_{N}}
  \and
  \inferrule* [lab=agent] {} {{M_{A}} \bc (\vec{x})M_{P} \;| \; \clift{P_0,\ldots,M_{P},\ldots,P_N}}
  \and \\
  \inferrule* [lab=process] {} {{M_{P}} \bc M_{N} \;| \;P|M_{P} }
\end{mathpar} 

\begin{mathpar}
  \inferrule* [lab=sychronization] {} {M_{N} \bc \Box \;|\; x?M_{F} \;|\; x!M_{C}}
  \and
  \inferrule* [lab=abstraction] {} {{M_{F}} \bc (x)M_{P} }
  \and
  \inferrule* [lab=concretion] {} {{M_{C}} \bc \langle M_{P} \rangle }
  \and \\
  \inferrule* [lab=process] {} {{M_{P}} \bc M_{N} \;| \;P|M_{P} }
\end{mathpar}

\begin{definition}[contextual application] Given a context $M$, and
  process $P$, we define the \emph{contextual application}, $M[P] :=
  M\{P/\Box\}$. That is, the contextual application of M to P is the
  substitution of $P$ for $\Box$ in $M$.
\end{definition}

$\meaningof{-} : L \to \mathcal{P}(\pi)$

\begin{mathpar}
  \inferrule* [lab=collection] {} {\meaningof{true} = \pi, \and \meaningof{~E} = \pi \setminus \meaningof{E}, \and \meaningof{E_{1} \& E_{2}} = \meaningof{E_{1}} \cap \meaningof{E_{2}}}
\end{mathpar}

\begin{mathpar}
  \inferrule* [lab=structure] {} {\meaningof{0} = \{ P \in \pi | P \equiv 0 \}, \and \\ \meaningof{E_1 | E_2} = \{ P \in \pi | P \equiv P_{1} | P_{2}, P_{1} \in \meaningof{E_{1}}, P_{2} \in \meaningof{E_2}\} }
\end{mathpar}

\begin{mathpar}
 \inferrule* [lab=behavior] {} {\meaningof{\langle a?b \rangle E} = \{ P \in \pi | P \equiv Q | u?(y)P', \\ \and \\\\ \and \\ \;\;\; u \in \meaningof{a}, \forall z.P'\{z/y\} \in \meaningof{E\{z/b\}}\}, \and \\ \meaningof{a!E} = \{ P \in \pi | P \equiv Q | x!\langle P' \rangle, x \in \meaningof{a} P' \in \meaningof{E}\} }
\end{mathpar}

\begin{mathpar}
 \inferrule* [lab=nominal] {} {\meaningof{\quotep{E}} = \{ \quotep{P} \in \quotep{\pi} | P \in \meaningof{E} \}, \and \meaningof{\quotep{P}} = \{ \quotep{Q} \in \quotep{\pi} | P \equiv Q \} \and \\ \meaningof{@\quotep{E}} = \{ P \in \pi | P \equiv @x, x \in \meaningof{E} \}}
\end{mathpar}

\begin{eqnarray*}
  \\
  \meaningof{-} : TS \to ST
\end{eqnarray*}

\begin{eqnarray*}
  \\
  L : TS \to ST
\end{eqnarray*}

\begin{eqnarray*}
  \\
  P \models E \iff P \in \meaningof{E}
\end{eqnarray*}

\begin{eqnarray*}
  P \approx_{L} Q \iff \forall E \in L. P \models E \iff Q \models E
\end{eqnarray*}

\begin{eqnarray*}
  P \approx_{K} Q
\end{eqnarray*}

\begin{eqnarray*}
  P \approx Q
\end{eqnarray*}

$\approx_{K} = \approx = \approx_{L}$

\subsubsection{Contextual duality}

Note that contexts extend the quotation operation to a family of
operations from processes to names. Given a context, $M$, we can
define a \emph{nominal context}, $\quotep{M}$ by $\quotep{M}[P] :=
\quotep{M[P]}$. To foreshadow what is to come we observe that these
operations enjoy a duality with processes very much like the duality
between vectors and maps from vectors to scalars.

Further, because the calculus is essentially higher-order, we have a
correspondence between contexts and processes. More specifically,
given a name $x$ and a context $M$ we can construct $M^{*}_{x}$ such
that 

\begin{mathpar}
  M^{*}_{x} | \lift{x}{P} \red M[P]
\end{mathpar}

namely,

\begin{mathpar}
  M^{*}_{x} := x?(u).M[\dropn{u}]
\end{mathpar}

The dependence of $M^{*}_{x}$ on a name makes it an abstraction, 

\begin{mathpar}
  M^{*} := (x)x?(u).M[\dropn{u}]
\end{mathpar}

\subsection{Additional notation}

It will sometimes be convenient to denote the process a name
quotes. We already have the notation $x = \quotep{P}$, but it will be
convenient to introduce an alternate notation, $\procn{x}$, when we
want to emphasize the connection to the use of the name. Note that, by
virtue of name equivalence, $\quotep{\procn{x}} \nameeq x$; so, the
notation is consistent with previous definitions.

Further, because names have structure it is possible to effect
substitutions on the basis of that structure. This means we need to
upgrade our notation for substitutions, which we accomplish by
adapting comprehension notation. Thus,

\begin{mathpar}
  P\{ y / x : x \in S \}
\end{mathpar}

is interpreted to mean the process derived from P by replacing (in a
capture-avoiding manner) each occurrence of $x$ in $S$ by $y$. For example,

\begin{mathpar}
  P\{ \quotep{\procn{x}|\procn{x}} / x : x \in \freenames{P} \}
\end{mathpar}

will replace each (occurrence) of a free name $x$ in $P$ by
$\quotep{\procn{x}|\procn{x}}$.

Also, we will avail ourselves of the notation $x^{L}$ and $x^{R}$ to
denote injections of a name into disjoint copies of the name
space. There are numerous ways to accomplish this. One example can be
found in \cite{MeredithR05}. This notation overloads to vectors of
names: $\vec{x}^{\pi} := (x_{i}^{\pi} \; : \; 0 \leq i < |\vec{x}| )$ where $\pi \in \{L,R\}$.

We also use $P^{\Box} := P|\Box$.

In \cite{MeredithR05} an interpretation of the new operator is
given. It turns out that there are several possible interpretations
all enjoying the requisite algebraic properties of the operator (see
\cite{milner91polyadicpi}). We will therefore make liberal use of
$(\nu\; \vec{x})P$.

% subsection the_syntax_and_semantics_of_the_notation_system (end)   

\input{qm2pi.qmops} 

\input{qm2pi.sterngerlach} 

\input{qm2pi.metric} 

% section concurrent_process_calculi (end)

%\input{qm2pi.proofsketch}

% section proof sketch (end)

%\input{qm2pi.slviaknots} 

% section spatial logic via knots (end)

\input{qm2pi.conclusion}

% section conclusion (end)

%\input{qm2pi.dtcodes} 

% section wiring algorithm (end)

\input{qm2pi.ack} 

% section acknowledgments (end)

\newpage


\bibliographystyle{plain}   
\bibliography{../../biblios/main.bib}

\input{qm2pi.rhodetails}

\end{document}

 

% section concurrent_process_calculi (end)

%\documentclass[12pt]{llncs}
%\documentclass{jktr}

\usepackage[pdftex]{hyperref}                   
\usepackage {listings}
\usepackage {mathpartir}
\usepackage{bcprules}
%\usepackage{listings}
                       
\usepackage{graphicx} 
%\usepackage[margins=2.5cm,nohead,nofoot]{geometry}
%\usepackage{geometry}
\usepackage{amsfonts}
\usepackage{amstext}
\usepackage{latexsym}
\usepackage{amssymb}
\usepackage{color}


%\include{myPreamble}
\include{qm2pi.local} 

%\ifpdf
%\usepackage[pdftex]{graphicx}
%\else
%\usepackage{graphicx}
%\fi

 % \ifpdf
%  \usepackage{pdfsync}
%  \if


%\title{Brief Article}
%\author{David F. Snyder}
%\author{L.G. Meredith}

%\address{Dept. of Math., Texas State University--San Marcos, San Marcos, TX 78666}
       
\pagestyle{empty}


\begin{document}

\lstset{language=[Objective]Caml,frame=shadowbox}

\input{qm2pi.front}

% section front matter (end)

\input{qm2pi.intro} 
 
% section introduction (end)

% \input{qm2pi.knotations} 

% section notation (end)

\input{qm2pi.process.calculi} 

% section concurrent_process_calculi_and_spatial_logics_ (end)
    
%\input{qm2pi.knots2pi} 

%\input{qm2pi.trefoil} 

%\input{qm2pi.mainthm} 

% subsection basic_interpretation (end)

%\input{qm2pi.rho.presentation} 
\subsection{The syntax and semantics of the notation system}\label{sub:the_syntax_and_semantics_of_the_notation_system} % (fold)

We now summarize a technical presentation of the calculus that
embodies our theory of dynamics. The typical presentation of such a
calculus follows the style of giving generators and relations on
them. The grammar, below, describing term constructors, freely
generates the set of processes, $\Proc$. This set is then quotiented
by a relation known as structural congruence and it is over this set
that the notion of dynamics is expressed. This presentation is
essentially that of \cite{MeredithR05} with the addition of
polyadicity and summation. For readability we have relegated some of
the technical subtleties to an appendix.

\subsubsection{Process grammar}\label{subsub:process_grammar}

\begin{mathpar}
  \inferrule* [lab=synchronization] {} {{M} \bc \pzero \;|\; x?F \;|\; x!C }
  \and
  \inferrule* [lab=abstraction] {} {{F} \bc (x)P}
  \and
  \inferrule* [lab=concretion] {} {{C} \bc \langle Q \rangle}
  \and
  \inferrule* [lab=process] {} {{P,Q} \bc M \;| \;P|Q \;|\; @{x}}
  \and
  \inferrule* [lab=name] {} {{x} \bc \quotep{P}}
\end{mathpar} 

Note that $\vec{x}$ (resp. $\vec{P}$) denotes a vector of names
(resp. processes) of length $|\vec{x}|$ (resp. $|\vec{P}|$). We adopt
the following useful abbreviations.

\begin{mathpar}
   x?(\vec{y}).P := x.(\vec{y})P \and  x\clift{\vec{P}} := x.\clift{\vec{P}}
   \and x!(y) := \lift{x}{\dropn{y}}
   \and \Pi_{i=0}^{n-1}P_i := P_0 | \ldots | P_{n-1}
\end{mathpar}

\subsubsection{Structural congruence}

\paragraph{Free and bound names and alpha-equivalence.} At the
core of structural equivalence is alpha-equivalence which identifies
process that are the same up to a change of variable. Formally, we
recognize the distinction between free and bound names. The free names
of a process, $\freenames{P}$, may be calculated recursively as
follows:

\begin{mathpar}
\freenames{\pzero} := \emptyset
  \and \\
  \freenames{x?(y).P} := \{ x \} \cup (\freenames{P} \setminus \{ y \})
  \and 
  \freenames{x!\langle P \rangle} := \{ x \} \cup \{ P \} 
  \and \\
  \freenames{P|Q} := \freenames{P} \cup \freenames{Q}
  \and \\
  \freenames{@{x}} := \{ x \}
\end{mathpar}

$\pi$
$\quotep{\pi}$

$\freenames{-} : \pi \to \mathcal{P}(\quotep{\pi})$

\begin{eqnarray*}
  \freenames{\pzero} & := & \emptyset \\
  \freenames{x?(y).P} & := & \{ x \} \cup (\freenames{P} \setminus \{ y \}) \\
  \freenames{x!\langle P \rangle} & := & \{ x \} \cup \{ P \} \\
  \freenames{P|Q} & := & \freenames{P} \cup \freenames{Q} \\
  \freenames{\dropn{x}} & := & \{ x \}
\end{eqnarray*}

The bound names of a process, $\boundnames{P}$, are those names occurring in $P$
that are not free. For example, in $x?(y).0$, the name $x$ is free, while $y$ is bound.

\begin{mathpar}
  \inferrule* [lab=monoidal-laws] {} { P|Q \equiv Q|P \and P|0 \equiv P \and P|(Q|R) \equiv (P|Q)|R }
\end{mathpar}

\begin{mathpar}
  \inferrule* [lab=alpha-equivalence] {} { (x)P \equiv (y)P\{y/x\} \and y \not\in \freenames{P} }
\end{mathpar}

\begin{definition}
Then two processes, $P,Q$, are alpha-equivalent if $P = Q\{\vec{y}/\vec{x}\}$ for
some $\vec{x} \in \boundnames{Q},\vec{y} \in \boundnames{P}$, where $Q\{\vec{y}/\vec{x}\}$
denotes the capture-avoiding substitution of $\vec{y}$ for $\vec{x}$ in $Q$.
\end{definition}

\begin{definition}
  The {\em structural congruence} \cite{SangiorgiWalker} , $\equiv$,
  between processes is the least congruence containing
  alpha-equivalence, satisfying the abelian monoid laws
  (associativity, commutativity and $\pzero$ as identity) for parallel
  composition $|$ and for summation $+$.
\end{definition}

\subsection{Name equivalence}

We take name equivalence, written $\nameeq$, to be the smallest
equivalence relation generated by the following rules.

\begin{mathpar}
\inferrule*[lab=Quote-drop]
{ }
{ \quotep{@{x}} \nameeq x }

\inferrule*[lab=Struct-equiv]
{ P \scong Q }
{ \quotep{P} \nameeq \quotep{Q} }
\end{mathpar}

The astute reader will have noticed that the mutual recursion of names
and processes imposes a mutual recursion on alpha-equivalence and
structural equivalence via name-equivalence. Fortunately, all of this
works out pleasantly and we may calculate in the natural way, free of
concern. The reader interested in the details is referred to the
appendix \ref{appendix:rho_details}.

\subsection{Substitution}

We use $\Proc$ for the set of processes, $\QProc$ for the set of
names, and $\id{\{}\vec{y} / \vec{x} \id{\}}$ to denote partial maps,
$s : \QProc \rightarrow \QProc$. A map, $s$ lifts, uniquely, to a map
on process terms, $\widehat{s} : \Proc \rightarrow \Proc$ by the
following equations.

\begin{mathpar}
  (0) \psubstp{Q}{P} := 0 \\
  (R \juxtap S) \psubstp{Q}{P}
  :=    
  (R)\psubstp{Q}{P} \juxtap (S) \psubstp{Q}{P} \\
  (x?(y).R) \psubstp{Q}{P}    
  :=    
  (x)\substp{Q}{P} (z)\concat( (R \psubstn{z}{y}) \psubstp{Q}{P} ) \\
  (\lift{x}{R}) \psubstp{Q}{P}  
  :=
  \lift{(x)\substp{Q}{P}}{ R \psubstp{Q}{P} } \\
%   (\dropn{x})  \psubstp{Q}{P}       
%   := 
%   \left\{ 
%     \begin{array}{ccc} 
%       \dropn{\quotep{Q}} & & x \nameeq \quotep{P} \\
%       \dropn{x} & & otherwise \\
%     \end{array}
%   \right. 
  (\dropn{x})  \psubstp{Q}{P}       
  := 
  \left\{ 
    \begin{array}{ccc} 
      Q & & x \nameeq \quotep{P} \\
      \dropn{x} & & otherwise \\
    \end{array}
  \right.
\end{mathpar}
 

where

\begin{eqnarray}
  (x)\id{\{} \lpquote Q \rpquote / \lpquote P \rpquote \id{\}}            = 
  \left\{ 
    \begin{array}{ccc}
      \lpquote Q \rpquote & & x \nameeq \lpquote P \rpquote \\
      x & & otherwise \\
    \end{array}
  \right. \nonumber
\end{eqnarray}

and $z$ is chosen distinct from $\quotep{P}$, $\quotep{Q}$, the free
names in $Q$, and all the names in $R$. Our $\alpha$-equivalence will
be built in the standard way from this substitution.

\begin{remark}\label{rem:no_self_referential_names}
  One consequence of these definitions is that $\forall P. \quotep{P}
  \not\in \freenames{P}$.
\end{remark}

\subsection{ Dynamic quote: an example }

Anticipating something of what's to come, consider applying the
substitution, $\widehat{\id{\{}u / z \id{\}}}$, to the following pair
of processes, $\lift{w}{y!(z)}$ and $w[ \lpquote y!(z) \rpquote ]$.

\begin{eqnarray}
	\lift{w}{y!(z)}\widehat{\id{\{}u / z \id{\}}}
		& = &
		\lift{w}{y!(u)} \nonumber\\
	w[ \lpquote y!(z) \rpquote ] \widehat{ \id{\{}u / z \id{\}} }
		& = &
		w[ \lpquote y!(z) \rpquote ] \nonumber
\end{eqnarray}

Because the body of the process between quotes is impervious to
substitution, we get radically different answers. In fact, by
examining the first process in an input context,
e.g. $x?(z).\lift{w}{y!(z)}$, we see that the process under the lift
operator may be shaped by prefixed inputs binding a name inside it. In
this sense, the lift operator will be seen as a way to dynamically
construct processes before reifying them as names.

Finally equipped with these standard features we can present the
dynamics of the calculus.

\subsubsection{Operational semantics} 

Finally, we introduce the computational dynamics. What marks these
algebras as distinct from other more traditionally studied algebraic
structures, e.g. vector spaces or polynomial rings, is the manner in
which dynamics is captured. In traditional structures, dynamics is typically
expressed through morphisms between such structures, as in linear maps
between vector spaces or morphisms between rings. In algebras
associated with the semantics of computation, the dynamics is
expressed as part of the algebraic structure itself, through a
reduction reduction relation typically denoted by $\red$. Below, we
give a recursive presentation of this relation for the calculus used
in the encoding.

$\red \subseteq \pi \times \pi$
$\red : \pi \to \mathcal{P}(\pi)$

\begin{mathpar}
  \inferrule* [lab=Comm] { \textsf{match}( x_{src}, x_{trgt} ) } { x_{trgt}?(y)P \; | \; x_{src}!\langle {Q} \rangle \red P\{\quotep{Q}/y}\} }
  \and \\
  \inferrule* [lab=Par] {{P} \red {P}'} {{{P} | {Q}} \red {{P}' | {Q}}}
  \and
  \inferrule* [lab=Equiv]{{{P} \scong {P}'} \andalso {{P}' \red {Q}'} \andalso {{Q}' \scong {Q}}}{{P} \red {Q}}
\end{mathpar}

\begin{eqnarray*}
  match_{\equiv} (\quotep{P},\quotep{Q}) & := & P \equiv Q \\
  match_{\dagger}(\quotep{P},\quotep{Q}) & := & \forall R. P|Q \red^{*} R => R \red^{*} 0 \\
  match_{K}(\quotep{P},\quotep{Q}) & := & K \mbox{ for some context } K
\end{eqnarray*}

$u?(x)P | u!\langle Q \rangle \red P\{\quotep{Q}/x\}$

%We write $\wred$ for $\red^*$, and $P\red$ if $\exists Q $ such that $ P \red Q$.
We write $P\red$ if $\exists Q $ such that $ P \red Q$ and $P\not\red$, otherwise.

\section{Replication}

As mentioned before, it is known that replication (and hence
recursion) can be implemented in a higher-order process algebra
\cite{SangiorgiWalker}. As our first example of calculation with the
machinery thus far presented we give the construction explicitly in
the {\rhoc}.

\begin{eqnarray}
	D_{x} & := & \prefix{x}{y}{(\binpar{\outputp{x}{y}}{@{y}})} \nonumber\\
	\bangp_{x}{P} & := & \binpar{{x}!\langle{\binpar{D_{x}}{P}}\rangle}{D_{x}} \nonumber
\end{eqnarray}

\begin{eqnarray}
	\bangp_{x}{P} & & \nonumber\\
	=
	& {x}!\langle{(\prefix{x}{y}{(\outputp{x}{y} | @{y})) | P}}\rangle 
	      | \prefix{x}{y}{(\outputp{x}{y} | @{y})} & \nonumber\\
	\red
	& (\outputp{x}{y} | @{y})\substn{\quotep{(\prefix{x}{y}{(@{y} | \outputp{x}{y})) | P}}}{y} & \nonumber\\
	=
	& \outputp{x}{\quotep{(\prefix{x}{y}{(\outputp{x}{y} | @{y})) | P}}}
	  | {(\prefix{x}{y}{(\outputp{x}{y} | @{y})) | P}} & \nonumber\\
	\red
	& \ldots & \nonumber\\
	\red^*
	& P | P | \ldots & \nonumber
\end{eqnarray}

Of course, this encoding, as an implementation, runs away, unfolding
$\bangp{P}$ eagerly. A lazier and more implementable replication
operator, restricted to input-guarded processes, may be obtained as follows.

\begin{eqnarray}
\bangp{\prefix{u}{v}{P}} 
	:= 
	\binpar{\lift{x}{\prefix{u}{v}{(\binpar{D(x)}{P})}}}{D(x)} \nonumber
\end{eqnarray}

\begin{remark}
  Note that the lazier definition still does not deal with summation
  or mixed summation (i.e. sums over input and output). The reader is
  invited to construct definitions of replication that deal with these
  features. 

  Further, the definitions are parameterized in a name, $x$. Can you,
  gentle reader, make a definition that eliminates this parameter and
  guarantees no accidental interaction between the replication
  machinery and the process being replicated -- i.e. no accidental
  sharing of names used by the process to get its work done and the
  name(s) used by the replication to effect copying. This latter
  revision of the definition of replication is crucial to obtaining
  the expected identity $!!P \sim !P$.
\end{remark}

\begin{remark}\label{rem:paradoxical_combinator}
  The reader familiar with the lambda calculus will have noticed the
  similarity between $D$ and the paradoxical combinator.

  [Ed. note: the existence of this seems to suggest we have to be more
  restrictive on the set of processes and names we admit if we are to
  support no-cloning.]
\end{remark}

\subsubsection{Bisimulation}

The computational dynamics gives rise to another kind of equivalence,
the equivalence of computational behavior. As previously mentioned
this is typically captured \emph{via} some form of bisimulation.

% The notion we use in this paper is weak barbed bisimulation
% \cite{milner91polyadicpi}.

The notion we use in this paper is derived from weak barbed
bisimulation \cite{milner91polyadicpi}. 

\begin{definition}
An \emph{observation relation}, $\downarrow_{\mathcal N}$, over a set
of names, $\mathcal N$, is the smallest relation satisfying the rules
below.

\infrule[Out-barb]{y \in {\mathcal N}, \; x \nameeq y}
		  {\outputp{x}{v} \downarrow_{\mathcal N} x}
\infrule[Par-barb]{\mbox{$P\downarrow_{\mathcal N} x$ or $Q\downarrow_{\mathcal N} x$}}
		  {\binpar{P}{Q} \downarrow_{\mathcal N} x}

We write $P \Downarrow_{\mathcal N} x$ if there is $Q$ such that 
$P \wred Q$ and $Q \downarrow_{\mathcal N} x$.
\end{definition}

\begin{definition}
%\label{def.bbisim}
An  ${\mathcal N}$-\emph{barbed bisimulation} over a set of names, ${\mathcal N}$, is a symmetric binary relation 
${\mathcal S}_{\mathcal N}$ between agents such that $P\rel{S}_{\mathcal N}Q$ implies:
\begin{enumerate}
\item If $P \red P'$ then $Q \wred Q'$ and $P'\rel{S}_{\mathcal N} Q'$.
\item If $P\downarrow_{\mathcal N} x$, then $Q\Downarrow_{\mathcal N} x$.
\end{enumerate}
$P$ is ${\mathcal N}$-barbed bisimilar to $Q$, written
$P \wbbisim_{\mathcal N} Q$, if $P \rel{S}_{\mathcal N} Q$ for some ${\mathcal N}$-barbed bisimulation ${\mathcal S}_{\mathcal N}$.
\end{definition}

$\mathcal{R} \subseteq \pi \times \pi$

$P \mathcal{R} Q => \forall P'. P \red P' \Rightarrow \exists Q'. Q \red Q', P' \mathcal{R} Q'$

$P \vdash x \Rightarrow Q \vdash x$

\begin{mathpar}
  \inferrule*[lab=Out-barb]{x \nameeq y}{{y}!\langle{Q}\rangle \vdash x}
  \and
  \inferrule*[lab=Par-barb]{\mbox{$P\vdash x$ or $Q\vdash x$}}{\binpar{P}{Q} \vdash x}
\end{mathpar}

\subsubsection{Contexts}

One of the principle advantages of computational calculi like the
$\pi$-calculus is a well-defined notion of context,
contextual-equivalence and a correlation between
contextual-equivalence and notions of bisimulation. The notion of
context allows the decomposition of a process into (sub-)process and
its syntactic environment, its context. Thus, a context may be
thought of as a process with a ``hole'' (written $\Box$) in it. The
application of a context $M$ to a process $P$, written $M[P]$, is
tantamount to filling the hole in $M$ with $P$. In this paper we do
not need the full weight of this theory, but do make use of the notion
of context in the proof the main theorem. 

\begin{mathpar}
  \inferrule* [lab=summation] {} {{M_{M},M_{N}} \bc \Box \;|\; x.M_{A} \;|\; M_{M}+M_{N}}
  \and
  \inferrule* [lab=agent] {} {{M_{A}} \bc (\vec{x})M_{P} \;| \; \clift{P_0,\ldots,M_{P},\ldots,P_N}}
  \and \\
  \inferrule* [lab=process] {} {{M_{P}} \bc M_{N} \;| \;P|M_{P} }
\end{mathpar} 

\begin{mathpar}
  \inferrule* [lab=sychronization] {} {M_{N} \bc \Box \;|\; x?M_{F} \;|\; x!M_{C}}
  \and
  \inferrule* [lab=abstraction] {} {{M_{F}} \bc (x)M_{P} }
  \and
  \inferrule* [lab=concretion] {} {{M_{C}} \bc \langle M_{P} \rangle }
  \and \\
  \inferrule* [lab=process] {} {{M_{P}} \bc M_{N} \;| \;P|M_{P} }
\end{mathpar}

\begin{definition}[contextual application] Given a context $M$, and
  process $P$, we define the \emph{contextual application}, $M[P] :=
  M\{P/\Box\}$. That is, the contextual application of M to P is the
  substitution of $P$ for $\Box$ in $M$.
\end{definition}

$\meaningof{-} : L \to \mathcal{P}(\pi)$

\begin{mathpar}
  \inferrule* [lab=collection] {} {\meaningof{true} = \pi, \and \meaningof{~E} = \pi \setminus \meaningof{E}, \and \meaningof{E_{1} \& E_{2}} = \meaningof{E_{1}} \cap \meaningof{E_{2}}}
\end{mathpar}

\begin{mathpar}
  \inferrule* [lab=structure] {} {\meaningof{0} = \{ P \in \pi | P \equiv 0 \}, \and \\ \meaningof{E_1 | E_2} = \{ P \in \pi | P \equiv P_{1} | P_{2}, P_{1} \in \meaningof{E_{1}}, P_{2} \in \meaningof{E_2}\} }
\end{mathpar}

\begin{mathpar}
 \inferrule* [lab=behavior] {} {\meaningof{\langle a?b \rangle E} = \{ P \in \pi | P \equiv Q | u?(y)P', \\ \and \\\\ \and \\ \;\;\; u \in \meaningof{a}, \forall z.P'\{z/y\} \in \meaningof{E\{z/b\}}\}, \and \\ \meaningof{a!E} = \{ P \in \pi | P \equiv Q | x!\langle P' \rangle, x \in \meaningof{a} P' \in \meaningof{E}\} }
\end{mathpar}

\begin{mathpar}
 \inferrule* [lab=nominal] {} {\meaningof{\quotep{E}} = \{ \quotep{P} \in \quotep{\pi} | P \in \meaningof{E} \}, \and \meaningof{\quotep{P}} = \{ \quotep{Q} \in \quotep{\pi} | P \equiv Q \} \and \\ \meaningof{@\quotep{E}} = \{ P \in \pi | P \equiv @x, x \in \meaningof{E} \}}
\end{mathpar}

\begin{eqnarray*}
  \\
  \meaningof{-} : TS \to ST
\end{eqnarray*}

\begin{eqnarray*}
  \\
  L : TS \to ST
\end{eqnarray*}

\begin{eqnarray*}
  \\
  P \models E \iff P \in \meaningof{E}
\end{eqnarray*}

\begin{eqnarray*}
  P \approx_{L} Q \iff \forall E \in L. P \models E \iff Q \models E
\end{eqnarray*}

\begin{eqnarray*}
  P \approx_{K} Q
\end{eqnarray*}

\begin{eqnarray*}
  P \approx Q
\end{eqnarray*}

$\approx_{K} = \approx = \approx_{L}$

\subsubsection{Contextual duality}

Note that contexts extend the quotation operation to a family of
operations from processes to names. Given a context, $M$, we can
define a \emph{nominal context}, $\quotep{M}$ by $\quotep{M}[P] :=
\quotep{M[P]}$. To foreshadow what is to come we observe that these
operations enjoy a duality with processes very much like the duality
between vectors and maps from vectors to scalars.

Further, because the calculus is essentially higher-order, we have a
correspondence between contexts and processes. More specifically,
given a name $x$ and a context $M$ we can construct $M^{*}_{x}$ such
that 

\begin{mathpar}
  M^{*}_{x} | \lift{x}{P} \red M[P]
\end{mathpar}

namely,

\begin{mathpar}
  M^{*}_{x} := x?(u).M[\dropn{u}]
\end{mathpar}

The dependence of $M^{*}_{x}$ on a name makes it an abstraction, 

\begin{mathpar}
  M^{*} := (x)x?(u).M[\dropn{u}]
\end{mathpar}

\subsection{Additional notation}

It will sometimes be convenient to denote the process a name
quotes. We already have the notation $x = \quotep{P}$, but it will be
convenient to introduce an alternate notation, $\procn{x}$, when we
want to emphasize the connection to the use of the name. Note that, by
virtue of name equivalence, $\quotep{\procn{x}} \nameeq x$; so, the
notation is consistent with previous definitions.

Further, because names have structure it is possible to effect
substitutions on the basis of that structure. This means we need to
upgrade our notation for substitutions, which we accomplish by
adapting comprehension notation. Thus,

\begin{mathpar}
  P\{ y / x : x \in S \}
\end{mathpar}

is interpreted to mean the process derived from P by replacing (in a
capture-avoiding manner) each occurrence of $x$ in $S$ by $y$. For example,

\begin{mathpar}
  P\{ \quotep{\procn{x}|\procn{x}} / x : x \in \freenames{P} \}
\end{mathpar}

will replace each (occurrence) of a free name $x$ in $P$ by
$\quotep{\procn{x}|\procn{x}}$.

Also, we will avail ourselves of the notation $x^{L}$ and $x^{R}$ to
denote injections of a name into disjoint copies of the name
space. There are numerous ways to accomplish this. One example can be
found in \cite{MeredithR05}. This notation overloads to vectors of
names: $\vec{x}^{\pi} := (x_{i}^{\pi} \; : \; 0 \leq i < |\vec{x}| )$ where $\pi \in \{L,R\}$.

We also use $P^{\Box} := P|\Box$.

In \cite{MeredithR05} an interpretation of the new operator is
given. It turns out that there are several possible interpretations
all enjoying the requisite algebraic properties of the operator (see
\cite{milner91polyadicpi}). We will therefore make liberal use of
$(\nu\; \vec{x})P$.

% subsection the_syntax_and_semantics_of_the_notation_system (end)   

\input{qm2pi.qmops} 

\input{qm2pi.sterngerlach} 

\input{qm2pi.metric} 

% section concurrent_process_calculi (end)

%\input{qm2pi.proofsketch}

% section proof sketch (end)

%\input{qm2pi.slviaknots} 

% section spatial logic via knots (end)

\input{qm2pi.conclusion}

% section conclusion (end)

%\input{qm2pi.dtcodes} 

% section wiring algorithm (end)

\input{qm2pi.ack} 

% section acknowledgments (end)

\newpage


\bibliographystyle{plain}   
\bibliography{../../biblios/main.bib}

\input{qm2pi.rhodetails}

\end{document}



% section proof sketch (end)

%\section{Unlikely characters: spatial logic for
  knots}\label{sub:characteristic_formulae} % (fold)

Associated to the mobile process calculi are a family of logics known
as the Hennessy-Milner logics. These logics typically enjoy a
semantics interpreting formulae as sets of processes that when
factored through the encoding outlined above allows an identification
of classes of knots with logical formulae. In the context of this
encoding the sub-family known as the spatial logics \cite{CairesC03}
\cite{CairesC04} \cite{Caires04} are of particular interest providing
several important features for expressing and reasoning about
properties (i.e. classes) of knots. We hint here at how this may be done.

%\begin{description}
%\item [structural connectives] 
\subsubsection{Structural connectives} The spatial logics enjoy
structural connectives corresponding, at the logical level, to the
parallel composition ($P | Q$) and new name ($(\nu \; x)P$)
connectives for processes. As illustrated in the examples below, these
connectives are extremely expressive given the shape of our encoding.
%\item [decideable satisfaction]

\subsubsection{Decideable satisfaction}
In \cite{Caires04} the satisfaction relation is shown to be decideable
for a rich class of processes. It further turns out that the image of
the our encoding is a proper subset of that class. This result
provides the basis for an algorithm by which to search for knots
enjoying a given property.
%\item [characteristic formulae]

\subsubsection{Characteristic formulae}
In the same paper \cite{Caires04} , Caires presents a means of calculating
characteristic formulae, selecting equivalence classes of processes
up to a pre--specified depth limit on the support set of names. Composed with our
encoding, this characteristic formula can be used to select
characteristic formulae for knots.
%\end{description}

\subsubsection{Spatial logic formulae}

The grammar below (segmented for comprehension) summarizes the syntax
of spatial logic formulae. We employ illustrative examples in the
sequel to provide an intuitive understanding of their meaning
referring the reader to \cite{Caires04} for a more detailed explication
of the semantics.

\begin{mathpar}
  \inferrule* [lab=boolean] {} {{A,B} \bc T \;|\; \neg A \;|\; A \wedge B \;|\; \eta = \eta'}
  \and
  \inferrule* [lab=spatial] {} {|\; \pzero \;|\; A | B \;|\; x \text{\textregistered} A \;|\; \forall x . A \;|\;  H x . A}
  \and
  \inferrule* [lab=behavioral] {} {|\; \alpha . A}
  \and 
  \inferrule* [lab=recursion] {} {|\; X(\vec{u}) \;|\; \mu X(\vec{u}) . A}
  \and
  \inferrule* [lab=action] {} {\alpha \bc \langle x?(\vec{y}) \rangle \;|\; \langle x!(\vec{y}) \rangle \;|\; \langle \tau \rangle}
  \and 
  \inferrule* [lab=name] {} {\eta \bc x \;|\; \tau}
\end{mathpar} 

% subsection characteristic_formulae (end)   	 

\subsection{Example formulae}\label{sub:example_formulae_} % (fold)

\subsubsection{Crossing as formula.}
% 
% \begin{align*}
%   \frac{d}{dx} \sin x &= \cos x 
%   & \frac{d}{dx} e^x &= e^x \\
%   \frac{d}{dx} \cos x &= - \sin x 
%   & \frac{d}{dx} \log x &= \frac{1}{x} \\
% \end{align*} 

\begin{align*}
 \mu C(x_{0},x_{1},y_{0},y_{1},u).&(\langle x_{0}?(z) \rangle(\langle u! \rangle\langle y_{1}!z \rangle C(x_{0},x_{1},y_{0},y_{1},u)) & \\
  & \wedge \langle y_{1}?(z) \rangle (\langle u! \rangle \langle x_{0}!z \rangle C(x_{0},x_{1},y_{0},y_{1},u)) & \\
  & \wedge \langle x_{1}?(z) \rangle (\langle u? \rangle \langle y_{0}!z \rangle C(x_{0},x_{1},y_{0},y_{1},u)) & \\
  & \wedge \langle y_{0}?(z) \rangle (\langle u? \rangle \langle x_{1}!z \rangle C(x_{0},x_{1},y_{0},y_{1},u))) &
\end{align*}

The lexicographical similarity between the shape of this formulae and
the shape of definition of the process representing a crossing reveals
the intuitive meaning of this formulae. It describes the capabilities
of a process that has the right to represent a crossing. For example
it picks out processes that may perform an input on the port $x_0$ in
its initial menu of capabilities. What differentiates the formula
from the process, however, is that the crossing process is the
smallest candidate to satisfy the formula. Infinitely many other
processes -- with internal behavior hidden behind this interface, so
to speak -- also satisfy this formula. Even this simple formula,
then, can be seen to open a new view onto knots, providing a
computational interpretation of \emph{virtual} knots.

Note that this formula is derived by hand. A similar formula can be
derived by employing Caires' calculation of characteristic formula
\cite{Caires04} to the process representing a crossing. In light of
this discussion, we let
$\meaningof{C}_{\phi}(x0,x1,y0,y1,u)$ denote a formula specifying the
dynamics we wish to capture of a crossing. To guarantee we preserve
the shape of the interface and minimal semantics we demand that
$\meaningof{C}_{\phi}(x0,x1,y0,y1,u) \Rightarrow
\textbf{C}(x0,x1,y0,y1,u)$ where $\textbf{C}(x0,x1,y0,y1,u)$ denotes
the formula above.
                            
\subsubsection{Crossing number constraints.}
The moral content of the context lemma (Lemma \ref{context}) is that the notion of
``locality'' in the Reidemeister moves is effectively captured by the
parallel composition operator of the process calculus. This intuition
extends through the logic. Given a formula,
$\meaningof{C}_{\phi}(x0,x1,y0,y1,u)$, we can use the structural
connectives to specify constraints on crossing numbers, such as at
least $n$ crossings, or exactly $n$ crossings.
\begin{mathpar}
  \inferrule* [lab=at-least-n] {} { K^{\geq n}_{\phi}(\vec{xs},\vec{ys}) := \Pi_{i=0}^{n-1} Hu . \meaningof{C}_{\phi}(xs_i,ys_i,u) | T }
  \and 
  \inferrule* [lab=exactly-n] {} { K^{= n}_{\phi}(\vec{xs},\vec{ys}) := \Pi_{i=0}^{n-1} Hu . \meaningof{C}_{\phi}(xs_i,ys_i,u) | \neg (\forall x_0,y_0,x_1,y_1,u . \meaningof{C}_{\phi}(x_0,y_0,x_1,y_1,u) | T) }
\end{mathpar}

To round out this section, recall that the encoding of an $n$-crossing
knot decomposes into a parallel composition of $n$ \emph{copies} of a
crossing process together with a wiring harness. To specify different
knot classes with the same crossing number amounts to specifying
logical constraints on the wiring harness. In the interest of space,
we defer examples to a forthcoming paper. Suffice it to say that both
the conditions ``alternating knot'' and ``contains the tangle
corresponding to 5/3'' are expressible. For example, it is possible to
calculate the characteristic formula of a process corresponding to the
tangle 5/3 and conjoin it into the classifying formula via the
composition connective of the logic.

Finally, we wish to observe that it is entirely within reason to
contemplate a more domain-specific version of spatial logic tailored
to the shape of processes in the image of the encoding. Such a
domain-specific logic would have a better claim to the title formal
language of knot properties.

% subsection example_formulae_ (end)

% section knots_as_processes (end) 

% section spatial logic via knots (end)

\section{Conclusions and future work}

\paragraph{Testing physical space}
You, gentle reader, may wonder why of all the theorems to be proved
given this set up we pick the one above. In some sense it's hardly
central to quantum mechanics. We see it as central in the sense that
it firmly establishes a notion of physical space arising from a notion
of the equivalence of behavior. Relating bisimulation to a metric is a
big step forward, but one is faced with interpreting the relationship
of that metric space to something more physical. Quantum mechanical
notions of ``physical'' space are still far from intuitive, but by
relating this idea of distance as testing to calculations that predict
physical circumstances we are making a not insignificant step forward
toward an understanding of the physical space we inhabit as
essentially dynamic.

\paragraph{Effectivity and simulation}
One of the observations we have yet to make is that the entire program
spelled out here is effective. We have built various interpreters for
the reflective calculus at work in this interpretation. In principle,
then, we can simulate quantum mechanics on a computer. The place where
the simulation may lose fidelity is the infinitely branching summation
for the annihilator.

In this connection i also want to point out that the evaluation style
calculation of the inner product puts the non-determinism of the
summation right at the heart of measurement. This suggests that
Milner's original reduction-based formulation of the dynamics of his
calculi in terms of sums was not just notationally suggestive of a
notion of measure-and-continue but captured some significant part of
the physics.

\paragraph{Quantum continuations}
In light of this last observation i want to point out that the
predominant account of quantum mechanics is missing a key aspect of a
truly compositional story of the physical situation. In a real lab,
when a measurement is made the observation can be made to feed into
another device that then makes another measurement conditioned on the
results of the first. This means that after the superposition was
collapsed the entire experimental set up remained in
superposition. While QM offers a means of writing this down it doesn't
quite line up well with the well-trodden formulation of computation
and continuation that we see so succinctly expressed in Milner's
calculi. This suggests that there might be advantages to this account
of dynamics waiting to be explored.

\paragraph{Quantum logic}
In this connection, we also note that by virtue of having the
Hennessy-Milner construction, we can pull the construction through the
interpretation of QM. This gives us a natural candidate for a quantum
logic that enjoys an extremely tight connection with it's domain of
interpretation, making the construction much less ad hoc (rather it is
the image of functor!).

\paragraph{Quantum probabiity}
i have questions about the basis of the interpretation of inner
product as probability amplitude. In particular, using which
axiomatization of probability theory does the notion of probability
amplitude earn the right to be so dubbed? In other words, where is the
proof that the operation for calculating a probability amplitude (and
then squaring) satisfies the axioms of what it means to calculate a
probability? Even if such a proof exists (i have yet to find it in the
literature), i wonder if it might not be possible to turn things on
their heads. Can we view the calculation of the probability amplitude
as an axiomatization of probability? If so, then the definition we
give for calculating probability amplitude may provide the basis for
an \emph{effective} theory of probability.

\paragraph{Quantum vs ``biological'' information}
Finally, i want to conclude with a more philosophical observation. At
a recent workshop in which QM was a predominant topic i noticed
something about quantum information. The speaker was giving a riveting
discussion of axiomatic QM and showing how properties of ``no
cloning'' and ``no deleting'' emerged as consequences of the
axiomatization. Theorems of this form are necessary to give us a sense
of confidence that our axioms characterize the physical theory. What
struck me, though, was that if quantum information is neither erasable
nor replicable it is markedly different from \emph{life}. Two of the
things we know about life is that

\begin{itemize}
  \item it ends;
  \item to gain some measure of persistence, to transcend it's
    finitude it is imminently copyable.
\end{itemize}

Both of these qualities are summarized succinctly in the aphorism: all
flesh is grass. For me these two kinds of ``information'' -- call them
quantum and biological -- are end points on a spectrum of strategies
for persistence. At one end, we have those curious entities that enjoy
uniqueness and permanence; at the other, we have those who in the face
of a certain end and an uncertain present make a go of passing
something on. To me one of the more remarkable aspects of the latter
strategy is that in the presence of noise (and certain features of
copying) we get a kind of dynamism, a chance for improvement against a
given persistent condition.

% subsection other_calculi_other_bisimulations_and_geometry_as_behavior (end)




% section conclusion (end)

%\documentclass[12pt]{llncs}
%\documentclass{jktr}

\usepackage[pdftex]{hyperref}                   
\usepackage {listings}
\usepackage {mathpartir}
\usepackage{bcprules}
%\usepackage{listings}
                       
\usepackage{graphicx} 
%\usepackage[margins=2.5cm,nohead,nofoot]{geometry}
%\usepackage{geometry}
\usepackage{amsfonts}
\usepackage{amstext}
\usepackage{latexsym}
\usepackage{amssymb}
\usepackage{color}


%\include{myPreamble}
\include{qm2pi.local} 

%\ifpdf
%\usepackage[pdftex]{graphicx}
%\else
%\usepackage{graphicx}
%\fi

 % \ifpdf
%  \usepackage{pdfsync}
%  \if


%\title{Brief Article}
%\author{David F. Snyder}
%\author{L.G. Meredith}

%\address{Dept. of Math., Texas State University--San Marcos, San Marcos, TX 78666}
       
\pagestyle{empty}


\begin{document}

\lstset{language=[Objective]Caml,frame=shadowbox}

\input{qm2pi.front}

% section front matter (end)

\input{qm2pi.intro} 
 
% section introduction (end)

% \input{qm2pi.knotations} 

% section notation (end)

\input{qm2pi.process.calculi} 

% section concurrent_process_calculi_and_spatial_logics_ (end)
    
%\input{qm2pi.knots2pi} 

%\input{qm2pi.trefoil} 

%\input{qm2pi.mainthm} 

% subsection basic_interpretation (end)

%\input{qm2pi.rho.presentation} 
\subsection{The syntax and semantics of the notation system}\label{sub:the_syntax_and_semantics_of_the_notation_system} % (fold)

We now summarize a technical presentation of the calculus that
embodies our theory of dynamics. The typical presentation of such a
calculus follows the style of giving generators and relations on
them. The grammar, below, describing term constructors, freely
generates the set of processes, $\Proc$. This set is then quotiented
by a relation known as structural congruence and it is over this set
that the notion of dynamics is expressed. This presentation is
essentially that of \cite{MeredithR05} with the addition of
polyadicity and summation. For readability we have relegated some of
the technical subtleties to an appendix.

\subsubsection{Process grammar}\label{subsub:process_grammar}

\begin{mathpar}
  \inferrule* [lab=synchronization] {} {{M} \bc \pzero \;|\; x?F \;|\; x!C }
  \and
  \inferrule* [lab=abstraction] {} {{F} \bc (x)P}
  \and
  \inferrule* [lab=concretion] {} {{C} \bc \langle Q \rangle}
  \and
  \inferrule* [lab=process] {} {{P,Q} \bc M \;| \;P|Q \;|\; @{x}}
  \and
  \inferrule* [lab=name] {} {{x} \bc \quotep{P}}
\end{mathpar} 

Note that $\vec{x}$ (resp. $\vec{P}$) denotes a vector of names
(resp. processes) of length $|\vec{x}|$ (resp. $|\vec{P}|$). We adopt
the following useful abbreviations.

\begin{mathpar}
   x?(\vec{y}).P := x.(\vec{y})P \and  x\clift{\vec{P}} := x.\clift{\vec{P}}
   \and x!(y) := \lift{x}{\dropn{y}}
   \and \Pi_{i=0}^{n-1}P_i := P_0 | \ldots | P_{n-1}
\end{mathpar}

\subsubsection{Structural congruence}

\paragraph{Free and bound names and alpha-equivalence.} At the
core of structural equivalence is alpha-equivalence which identifies
process that are the same up to a change of variable. Formally, we
recognize the distinction between free and bound names. The free names
of a process, $\freenames{P}$, may be calculated recursively as
follows:

\begin{mathpar}
\freenames{\pzero} := \emptyset
  \and \\
  \freenames{x?(y).P} := \{ x \} \cup (\freenames{P} \setminus \{ y \})
  \and 
  \freenames{x!\langle P \rangle} := \{ x \} \cup \{ P \} 
  \and \\
  \freenames{P|Q} := \freenames{P} \cup \freenames{Q}
  \and \\
  \freenames{@{x}} := \{ x \}
\end{mathpar}

$\pi$
$\quotep{\pi}$

$\freenames{-} : \pi \to \mathcal{P}(\quotep{\pi})$

\begin{eqnarray*}
  \freenames{\pzero} & := & \emptyset \\
  \freenames{x?(y).P} & := & \{ x \} \cup (\freenames{P} \setminus \{ y \}) \\
  \freenames{x!\langle P \rangle} & := & \{ x \} \cup \{ P \} \\
  \freenames{P|Q} & := & \freenames{P} \cup \freenames{Q} \\
  \freenames{\dropn{x}} & := & \{ x \}
\end{eqnarray*}

The bound names of a process, $\boundnames{P}$, are those names occurring in $P$
that are not free. For example, in $x?(y).0$, the name $x$ is free, while $y$ is bound.

\begin{mathpar}
  \inferrule* [lab=monoidal-laws] {} { P|Q \equiv Q|P \and P|0 \equiv P \and P|(Q|R) \equiv (P|Q)|R }
\end{mathpar}

\begin{mathpar}
  \inferrule* [lab=alpha-equivalence] {} { (x)P \equiv (y)P\{y/x\} \and y \not\in \freenames{P} }
\end{mathpar}

\begin{definition}
Then two processes, $P,Q$, are alpha-equivalent if $P = Q\{\vec{y}/\vec{x}\}$ for
some $\vec{x} \in \boundnames{Q},\vec{y} \in \boundnames{P}$, where $Q\{\vec{y}/\vec{x}\}$
denotes the capture-avoiding substitution of $\vec{y}$ for $\vec{x}$ in $Q$.
\end{definition}

\begin{definition}
  The {\em structural congruence} \cite{SangiorgiWalker} , $\equiv$,
  between processes is the least congruence containing
  alpha-equivalence, satisfying the abelian monoid laws
  (associativity, commutativity and $\pzero$ as identity) for parallel
  composition $|$ and for summation $+$.
\end{definition}

\subsection{Name equivalence}

We take name equivalence, written $\nameeq$, to be the smallest
equivalence relation generated by the following rules.

\begin{mathpar}
\inferrule*[lab=Quote-drop]
{ }
{ \quotep{@{x}} \nameeq x }

\inferrule*[lab=Struct-equiv]
{ P \scong Q }
{ \quotep{P} \nameeq \quotep{Q} }
\end{mathpar}

The astute reader will have noticed that the mutual recursion of names
and processes imposes a mutual recursion on alpha-equivalence and
structural equivalence via name-equivalence. Fortunately, all of this
works out pleasantly and we may calculate in the natural way, free of
concern. The reader interested in the details is referred to the
appendix \ref{appendix:rho_details}.

\subsection{Substitution}

We use $\Proc$ for the set of processes, $\QProc$ for the set of
names, and $\id{\{}\vec{y} / \vec{x} \id{\}}$ to denote partial maps,
$s : \QProc \rightarrow \QProc$. A map, $s$ lifts, uniquely, to a map
on process terms, $\widehat{s} : \Proc \rightarrow \Proc$ by the
following equations.

\begin{mathpar}
  (0) \psubstp{Q}{P} := 0 \\
  (R \juxtap S) \psubstp{Q}{P}
  :=    
  (R)\psubstp{Q}{P} \juxtap (S) \psubstp{Q}{P} \\
  (x?(y).R) \psubstp{Q}{P}    
  :=    
  (x)\substp{Q}{P} (z)\concat( (R \psubstn{z}{y}) \psubstp{Q}{P} ) \\
  (\lift{x}{R}) \psubstp{Q}{P}  
  :=
  \lift{(x)\substp{Q}{P}}{ R \psubstp{Q}{P} } \\
%   (\dropn{x})  \psubstp{Q}{P}       
%   := 
%   \left\{ 
%     \begin{array}{ccc} 
%       \dropn{\quotep{Q}} & & x \nameeq \quotep{P} \\
%       \dropn{x} & & otherwise \\
%     \end{array}
%   \right. 
  (\dropn{x})  \psubstp{Q}{P}       
  := 
  \left\{ 
    \begin{array}{ccc} 
      Q & & x \nameeq \quotep{P} \\
      \dropn{x} & & otherwise \\
    \end{array}
  \right.
\end{mathpar}
 

where

\begin{eqnarray}
  (x)\id{\{} \lpquote Q \rpquote / \lpquote P \rpquote \id{\}}            = 
  \left\{ 
    \begin{array}{ccc}
      \lpquote Q \rpquote & & x \nameeq \lpquote P \rpquote \\
      x & & otherwise \\
    \end{array}
  \right. \nonumber
\end{eqnarray}

and $z$ is chosen distinct from $\quotep{P}$, $\quotep{Q}$, the free
names in $Q$, and all the names in $R$. Our $\alpha$-equivalence will
be built in the standard way from this substitution.

\begin{remark}\label{rem:no_self_referential_names}
  One consequence of these definitions is that $\forall P. \quotep{P}
  \not\in \freenames{P}$.
\end{remark}

\subsection{ Dynamic quote: an example }

Anticipating something of what's to come, consider applying the
substitution, $\widehat{\id{\{}u / z \id{\}}}$, to the following pair
of processes, $\lift{w}{y!(z)}$ and $w[ \lpquote y!(z) \rpquote ]$.

\begin{eqnarray}
	\lift{w}{y!(z)}\widehat{\id{\{}u / z \id{\}}}
		& = &
		\lift{w}{y!(u)} \nonumber\\
	w[ \lpquote y!(z) \rpquote ] \widehat{ \id{\{}u / z \id{\}} }
		& = &
		w[ \lpquote y!(z) \rpquote ] \nonumber
\end{eqnarray}

Because the body of the process between quotes is impervious to
substitution, we get radically different answers. In fact, by
examining the first process in an input context,
e.g. $x?(z).\lift{w}{y!(z)}$, we see that the process under the lift
operator may be shaped by prefixed inputs binding a name inside it. In
this sense, the lift operator will be seen as a way to dynamically
construct processes before reifying them as names.

Finally equipped with these standard features we can present the
dynamics of the calculus.

\subsubsection{Operational semantics} 

Finally, we introduce the computational dynamics. What marks these
algebras as distinct from other more traditionally studied algebraic
structures, e.g. vector spaces or polynomial rings, is the manner in
which dynamics is captured. In traditional structures, dynamics is typically
expressed through morphisms between such structures, as in linear maps
between vector spaces or morphisms between rings. In algebras
associated with the semantics of computation, the dynamics is
expressed as part of the algebraic structure itself, through a
reduction reduction relation typically denoted by $\red$. Below, we
give a recursive presentation of this relation for the calculus used
in the encoding.

$\red \subseteq \pi \times \pi$
$\red : \pi \to \mathcal{P}(\pi)$

\begin{mathpar}
  \inferrule* [lab=Comm] { \textsf{match}( x_{src}, x_{trgt} ) } { x_{trgt}?(y)P \; | \; x_{src}!\langle {Q} \rangle \red P\{\quotep{Q}/y}\} }
  \and \\
  \inferrule* [lab=Par] {{P} \red {P}'} {{{P} | {Q}} \red {{P}' | {Q}}}
  \and
  \inferrule* [lab=Equiv]{{{P} \scong {P}'} \andalso {{P}' \red {Q}'} \andalso {{Q}' \scong {Q}}}{{P} \red {Q}}
\end{mathpar}

\begin{eqnarray*}
  match_{\equiv} (\quotep{P},\quotep{Q}) & := & P \equiv Q \\
  match_{\dagger}(\quotep{P},\quotep{Q}) & := & \forall R. P|Q \red^{*} R => R \red^{*} 0 \\
  match_{K}(\quotep{P},\quotep{Q}) & := & K \mbox{ for some context } K
\end{eqnarray*}

$u?(x)P | u!\langle Q \rangle \red P\{\quotep{Q}/x\}$

%We write $\wred$ for $\red^*$, and $P\red$ if $\exists Q $ such that $ P \red Q$.
We write $P\red$ if $\exists Q $ such that $ P \red Q$ and $P\not\red$, otherwise.

\section{Replication}

As mentioned before, it is known that replication (and hence
recursion) can be implemented in a higher-order process algebra
\cite{SangiorgiWalker}. As our first example of calculation with the
machinery thus far presented we give the construction explicitly in
the {\rhoc}.

\begin{eqnarray}
	D_{x} & := & \prefix{x}{y}{(\binpar{\outputp{x}{y}}{@{y}})} \nonumber\\
	\bangp_{x}{P} & := & \binpar{{x}!\langle{\binpar{D_{x}}{P}}\rangle}{D_{x}} \nonumber
\end{eqnarray}

\begin{eqnarray}
	\bangp_{x}{P} & & \nonumber\\
	=
	& {x}!\langle{(\prefix{x}{y}{(\outputp{x}{y} | @{y})) | P}}\rangle 
	      | \prefix{x}{y}{(\outputp{x}{y} | @{y})} & \nonumber\\
	\red
	& (\outputp{x}{y} | @{y})\substn{\quotep{(\prefix{x}{y}{(@{y} | \outputp{x}{y})) | P}}}{y} & \nonumber\\
	=
	& \outputp{x}{\quotep{(\prefix{x}{y}{(\outputp{x}{y} | @{y})) | P}}}
	  | {(\prefix{x}{y}{(\outputp{x}{y} | @{y})) | P}} & \nonumber\\
	\red
	& \ldots & \nonumber\\
	\red^*
	& P | P | \ldots & \nonumber
\end{eqnarray}

Of course, this encoding, as an implementation, runs away, unfolding
$\bangp{P}$ eagerly. A lazier and more implementable replication
operator, restricted to input-guarded processes, may be obtained as follows.

\begin{eqnarray}
\bangp{\prefix{u}{v}{P}} 
	:= 
	\binpar{\lift{x}{\prefix{u}{v}{(\binpar{D(x)}{P})}}}{D(x)} \nonumber
\end{eqnarray}

\begin{remark}
  Note that the lazier definition still does not deal with summation
  or mixed summation (i.e. sums over input and output). The reader is
  invited to construct definitions of replication that deal with these
  features. 

  Further, the definitions are parameterized in a name, $x$. Can you,
  gentle reader, make a definition that eliminates this parameter and
  guarantees no accidental interaction between the replication
  machinery and the process being replicated -- i.e. no accidental
  sharing of names used by the process to get its work done and the
  name(s) used by the replication to effect copying. This latter
  revision of the definition of replication is crucial to obtaining
  the expected identity $!!P \sim !P$.
\end{remark}

\begin{remark}\label{rem:paradoxical_combinator}
  The reader familiar with the lambda calculus will have noticed the
  similarity between $D$ and the paradoxical combinator.

  [Ed. note: the existence of this seems to suggest we have to be more
  restrictive on the set of processes and names we admit if we are to
  support no-cloning.]
\end{remark}

\subsubsection{Bisimulation}

The computational dynamics gives rise to another kind of equivalence,
the equivalence of computational behavior. As previously mentioned
this is typically captured \emph{via} some form of bisimulation.

% The notion we use in this paper is weak barbed bisimulation
% \cite{milner91polyadicpi}.

The notion we use in this paper is derived from weak barbed
bisimulation \cite{milner91polyadicpi}. 

\begin{definition}
An \emph{observation relation}, $\downarrow_{\mathcal N}$, over a set
of names, $\mathcal N$, is the smallest relation satisfying the rules
below.

\infrule[Out-barb]{y \in {\mathcal N}, \; x \nameeq y}
		  {\outputp{x}{v} \downarrow_{\mathcal N} x}
\infrule[Par-barb]{\mbox{$P\downarrow_{\mathcal N} x$ or $Q\downarrow_{\mathcal N} x$}}
		  {\binpar{P}{Q} \downarrow_{\mathcal N} x}

We write $P \Downarrow_{\mathcal N} x$ if there is $Q$ such that 
$P \wred Q$ and $Q \downarrow_{\mathcal N} x$.
\end{definition}

\begin{definition}
%\label{def.bbisim}
An  ${\mathcal N}$-\emph{barbed bisimulation} over a set of names, ${\mathcal N}$, is a symmetric binary relation 
${\mathcal S}_{\mathcal N}$ between agents such that $P\rel{S}_{\mathcal N}Q$ implies:
\begin{enumerate}
\item If $P \red P'$ then $Q \wred Q'$ and $P'\rel{S}_{\mathcal N} Q'$.
\item If $P\downarrow_{\mathcal N} x$, then $Q\Downarrow_{\mathcal N} x$.
\end{enumerate}
$P$ is ${\mathcal N}$-barbed bisimilar to $Q$, written
$P \wbbisim_{\mathcal N} Q$, if $P \rel{S}_{\mathcal N} Q$ for some ${\mathcal N}$-barbed bisimulation ${\mathcal S}_{\mathcal N}$.
\end{definition}

$\mathcal{R} \subseteq \pi \times \pi$

$P \mathcal{R} Q => \forall P'. P \red P' \Rightarrow \exists Q'. Q \red Q', P' \mathcal{R} Q'$

$P \vdash x \Rightarrow Q \vdash x$

\begin{mathpar}
  \inferrule*[lab=Out-barb]{x \nameeq y}{{y}!\langle{Q}\rangle \vdash x}
  \and
  \inferrule*[lab=Par-barb]{\mbox{$P\vdash x$ or $Q\vdash x$}}{\binpar{P}{Q} \vdash x}
\end{mathpar}

\subsubsection{Contexts}

One of the principle advantages of computational calculi like the
$\pi$-calculus is a well-defined notion of context,
contextual-equivalence and a correlation between
contextual-equivalence and notions of bisimulation. The notion of
context allows the decomposition of a process into (sub-)process and
its syntactic environment, its context. Thus, a context may be
thought of as a process with a ``hole'' (written $\Box$) in it. The
application of a context $M$ to a process $P$, written $M[P]$, is
tantamount to filling the hole in $M$ with $P$. In this paper we do
not need the full weight of this theory, but do make use of the notion
of context in the proof the main theorem. 

\begin{mathpar}
  \inferrule* [lab=summation] {} {{M_{M},M_{N}} \bc \Box \;|\; x.M_{A} \;|\; M_{M}+M_{N}}
  \and
  \inferrule* [lab=agent] {} {{M_{A}} \bc (\vec{x})M_{P} \;| \; \clift{P_0,\ldots,M_{P},\ldots,P_N}}
  \and \\
  \inferrule* [lab=process] {} {{M_{P}} \bc M_{N} \;| \;P|M_{P} }
\end{mathpar} 

\begin{mathpar}
  \inferrule* [lab=sychronization] {} {M_{N} \bc \Box \;|\; x?M_{F} \;|\; x!M_{C}}
  \and
  \inferrule* [lab=abstraction] {} {{M_{F}} \bc (x)M_{P} }
  \and
  \inferrule* [lab=concretion] {} {{M_{C}} \bc \langle M_{P} \rangle }
  \and \\
  \inferrule* [lab=process] {} {{M_{P}} \bc M_{N} \;| \;P|M_{P} }
\end{mathpar}

\begin{definition}[contextual application] Given a context $M$, and
  process $P$, we define the \emph{contextual application}, $M[P] :=
  M\{P/\Box\}$. That is, the contextual application of M to P is the
  substitution of $P$ for $\Box$ in $M$.
\end{definition}

$\meaningof{-} : L \to \mathcal{P}(\pi)$

\begin{mathpar}
  \inferrule* [lab=collection] {} {\meaningof{true} = \pi, \and \meaningof{~E} = \pi \setminus \meaningof{E}, \and \meaningof{E_{1} \& E_{2}} = \meaningof{E_{1}} \cap \meaningof{E_{2}}}
\end{mathpar}

\begin{mathpar}
  \inferrule* [lab=structure] {} {\meaningof{0} = \{ P \in \pi | P \equiv 0 \}, \and \\ \meaningof{E_1 | E_2} = \{ P \in \pi | P \equiv P_{1} | P_{2}, P_{1} \in \meaningof{E_{1}}, P_{2} \in \meaningof{E_2}\} }
\end{mathpar}

\begin{mathpar}
 \inferrule* [lab=behavior] {} {\meaningof{\langle a?b \rangle E} = \{ P \in \pi | P \equiv Q | u?(y)P', \\ \and \\\\ \and \\ \;\;\; u \in \meaningof{a}, \forall z.P'\{z/y\} \in \meaningof{E\{z/b\}}\}, \and \\ \meaningof{a!E} = \{ P \in \pi | P \equiv Q | x!\langle P' \rangle, x \in \meaningof{a} P' \in \meaningof{E}\} }
\end{mathpar}

\begin{mathpar}
 \inferrule* [lab=nominal] {} {\meaningof{\quotep{E}} = \{ \quotep{P} \in \quotep{\pi} | P \in \meaningof{E} \}, \and \meaningof{\quotep{P}} = \{ \quotep{Q} \in \quotep{\pi} | P \equiv Q \} \and \\ \meaningof{@\quotep{E}} = \{ P \in \pi | P \equiv @x, x \in \meaningof{E} \}}
\end{mathpar}

\begin{eqnarray*}
  \\
  \meaningof{-} : TS \to ST
\end{eqnarray*}

\begin{eqnarray*}
  \\
  L : TS \to ST
\end{eqnarray*}

\begin{eqnarray*}
  \\
  P \models E \iff P \in \meaningof{E}
\end{eqnarray*}

\begin{eqnarray*}
  P \approx_{L} Q \iff \forall E \in L. P \models E \iff Q \models E
\end{eqnarray*}

\begin{eqnarray*}
  P \approx_{K} Q
\end{eqnarray*}

\begin{eqnarray*}
  P \approx Q
\end{eqnarray*}

$\approx_{K} = \approx = \approx_{L}$

\subsubsection{Contextual duality}

Note that contexts extend the quotation operation to a family of
operations from processes to names. Given a context, $M$, we can
define a \emph{nominal context}, $\quotep{M}$ by $\quotep{M}[P] :=
\quotep{M[P]}$. To foreshadow what is to come we observe that these
operations enjoy a duality with processes very much like the duality
between vectors and maps from vectors to scalars.

Further, because the calculus is essentially higher-order, we have a
correspondence between contexts and processes. More specifically,
given a name $x$ and a context $M$ we can construct $M^{*}_{x}$ such
that 

\begin{mathpar}
  M^{*}_{x} | \lift{x}{P} \red M[P]
\end{mathpar}

namely,

\begin{mathpar}
  M^{*}_{x} := x?(u).M[\dropn{u}]
\end{mathpar}

The dependence of $M^{*}_{x}$ on a name makes it an abstraction, 

\begin{mathpar}
  M^{*} := (x)x?(u).M[\dropn{u}]
\end{mathpar}

\subsection{Additional notation}

It will sometimes be convenient to denote the process a name
quotes. We already have the notation $x = \quotep{P}$, but it will be
convenient to introduce an alternate notation, $\procn{x}$, when we
want to emphasize the connection to the use of the name. Note that, by
virtue of name equivalence, $\quotep{\procn{x}} \nameeq x$; so, the
notation is consistent with previous definitions.

Further, because names have structure it is possible to effect
substitutions on the basis of that structure. This means we need to
upgrade our notation for substitutions, which we accomplish by
adapting comprehension notation. Thus,

\begin{mathpar}
  P\{ y / x : x \in S \}
\end{mathpar}

is interpreted to mean the process derived from P by replacing (in a
capture-avoiding manner) each occurrence of $x$ in $S$ by $y$. For example,

\begin{mathpar}
  P\{ \quotep{\procn{x}|\procn{x}} / x : x \in \freenames{P} \}
\end{mathpar}

will replace each (occurrence) of a free name $x$ in $P$ by
$\quotep{\procn{x}|\procn{x}}$.

Also, we will avail ourselves of the notation $x^{L}$ and $x^{R}$ to
denote injections of a name into disjoint copies of the name
space. There are numerous ways to accomplish this. One example can be
found in \cite{MeredithR05}. This notation overloads to vectors of
names: $\vec{x}^{\pi} := (x_{i}^{\pi} \; : \; 0 \leq i < |\vec{x}| )$ where $\pi \in \{L,R\}$.

We also use $P^{\Box} := P|\Box$.

In \cite{MeredithR05} an interpretation of the new operator is
given. It turns out that there are several possible interpretations
all enjoying the requisite algebraic properties of the operator (see
\cite{milner91polyadicpi}). We will therefore make liberal use of
$(\nu\; \vec{x})P$.

% subsection the_syntax_and_semantics_of_the_notation_system (end)   

\input{qm2pi.qmops} 

\input{qm2pi.sterngerlach} 

\input{qm2pi.metric} 

% section concurrent_process_calculi (end)

%\input{qm2pi.proofsketch}

% section proof sketch (end)

%\input{qm2pi.slviaknots} 

% section spatial logic via knots (end)

\input{qm2pi.conclusion}

% section conclusion (end)

%\input{qm2pi.dtcodes} 

% section wiring algorithm (end)

\input{qm2pi.ack} 

% section acknowledgments (end)

\newpage


\bibliographystyle{plain}   
\bibliography{../../biblios/main.bib}

\input{qm2pi.rhodetails}

\end{document}

 

% section wiring algorithm (end)

\documentclass[12pt]{llncs}
%\documentclass{jktr}

\usepackage[pdftex]{hyperref}                   
\usepackage {listings}
\usepackage {mathpartir}
\usepackage{bcprules}
%\usepackage{listings}
                       
\usepackage{graphicx} 
%\usepackage[margins=2.5cm,nohead,nofoot]{geometry}
%\usepackage{geometry}
\usepackage{amsfonts}
\usepackage{amstext}
\usepackage{latexsym}
\usepackage{amssymb}
\usepackage{color}


%\include{myPreamble}
\include{qm2pi.local} 

%\ifpdf
%\usepackage[pdftex]{graphicx}
%\else
%\usepackage{graphicx}
%\fi

 % \ifpdf
%  \usepackage{pdfsync}
%  \if


%\title{Brief Article}
%\author{David F. Snyder}
%\author{L.G. Meredith}

%\address{Dept. of Math., Texas State University--San Marcos, San Marcos, TX 78666}
       
\pagestyle{empty}


\begin{document}

\lstset{language=[Objective]Caml,frame=shadowbox}

\input{qm2pi.front}

% section front matter (end)

\input{qm2pi.intro} 
 
% section introduction (end)

% \input{qm2pi.knotations} 

% section notation (end)

\input{qm2pi.process.calculi} 

% section concurrent_process_calculi_and_spatial_logics_ (end)
    
%\input{qm2pi.knots2pi} 

%\input{qm2pi.trefoil} 

%\input{qm2pi.mainthm} 

% subsection basic_interpretation (end)

%\input{qm2pi.rho.presentation} 
\subsection{The syntax and semantics of the notation system}\label{sub:the_syntax_and_semantics_of_the_notation_system} % (fold)

We now summarize a technical presentation of the calculus that
embodies our theory of dynamics. The typical presentation of such a
calculus follows the style of giving generators and relations on
them. The grammar, below, describing term constructors, freely
generates the set of processes, $\Proc$. This set is then quotiented
by a relation known as structural congruence and it is over this set
that the notion of dynamics is expressed. This presentation is
essentially that of \cite{MeredithR05} with the addition of
polyadicity and summation. For readability we have relegated some of
the technical subtleties to an appendix.

\subsubsection{Process grammar}\label{subsub:process_grammar}

\begin{mathpar}
  \inferrule* [lab=synchronization] {} {{M} \bc \pzero \;|\; x?F \;|\; x!C }
  \and
  \inferrule* [lab=abstraction] {} {{F} \bc (x)P}
  \and
  \inferrule* [lab=concretion] {} {{C} \bc \langle Q \rangle}
  \and
  \inferrule* [lab=process] {} {{P,Q} \bc M \;| \;P|Q \;|\; @{x}}
  \and
  \inferrule* [lab=name] {} {{x} \bc \quotep{P}}
\end{mathpar} 

Note that $\vec{x}$ (resp. $\vec{P}$) denotes a vector of names
(resp. processes) of length $|\vec{x}|$ (resp. $|\vec{P}|$). We adopt
the following useful abbreviations.

\begin{mathpar}
   x?(\vec{y}).P := x.(\vec{y})P \and  x\clift{\vec{P}} := x.\clift{\vec{P}}
   \and x!(y) := \lift{x}{\dropn{y}}
   \and \Pi_{i=0}^{n-1}P_i := P_0 | \ldots | P_{n-1}
\end{mathpar}

\subsubsection{Structural congruence}

\paragraph{Free and bound names and alpha-equivalence.} At the
core of structural equivalence is alpha-equivalence which identifies
process that are the same up to a change of variable. Formally, we
recognize the distinction between free and bound names. The free names
of a process, $\freenames{P}$, may be calculated recursively as
follows:

\begin{mathpar}
\freenames{\pzero} := \emptyset
  \and \\
  \freenames{x?(y).P} := \{ x \} \cup (\freenames{P} \setminus \{ y \})
  \and 
  \freenames{x!\langle P \rangle} := \{ x \} \cup \{ P \} 
  \and \\
  \freenames{P|Q} := \freenames{P} \cup \freenames{Q}
  \and \\
  \freenames{@{x}} := \{ x \}
\end{mathpar}

$\pi$
$\quotep{\pi}$

$\freenames{-} : \pi \to \mathcal{P}(\quotep{\pi})$

\begin{eqnarray*}
  \freenames{\pzero} & := & \emptyset \\
  \freenames{x?(y).P} & := & \{ x \} \cup (\freenames{P} \setminus \{ y \}) \\
  \freenames{x!\langle P \rangle} & := & \{ x \} \cup \{ P \} \\
  \freenames{P|Q} & := & \freenames{P} \cup \freenames{Q} \\
  \freenames{\dropn{x}} & := & \{ x \}
\end{eqnarray*}

The bound names of a process, $\boundnames{P}$, are those names occurring in $P$
that are not free. For example, in $x?(y).0$, the name $x$ is free, while $y$ is bound.

\begin{mathpar}
  \inferrule* [lab=monoidal-laws] {} { P|Q \equiv Q|P \and P|0 \equiv P \and P|(Q|R) \equiv (P|Q)|R }
\end{mathpar}

\begin{mathpar}
  \inferrule* [lab=alpha-equivalence] {} { (x)P \equiv (y)P\{y/x\} \and y \not\in \freenames{P} }
\end{mathpar}

\begin{definition}
Then two processes, $P,Q$, are alpha-equivalent if $P = Q\{\vec{y}/\vec{x}\}$ for
some $\vec{x} \in \boundnames{Q},\vec{y} \in \boundnames{P}$, where $Q\{\vec{y}/\vec{x}\}$
denotes the capture-avoiding substitution of $\vec{y}$ for $\vec{x}$ in $Q$.
\end{definition}

\begin{definition}
  The {\em structural congruence} \cite{SangiorgiWalker} , $\equiv$,
  between processes is the least congruence containing
  alpha-equivalence, satisfying the abelian monoid laws
  (associativity, commutativity and $\pzero$ as identity) for parallel
  composition $|$ and for summation $+$.
\end{definition}

\subsection{Name equivalence}

We take name equivalence, written $\nameeq$, to be the smallest
equivalence relation generated by the following rules.

\begin{mathpar}
\inferrule*[lab=Quote-drop]
{ }
{ \quotep{@{x}} \nameeq x }

\inferrule*[lab=Struct-equiv]
{ P \scong Q }
{ \quotep{P} \nameeq \quotep{Q} }
\end{mathpar}

The astute reader will have noticed that the mutual recursion of names
and processes imposes a mutual recursion on alpha-equivalence and
structural equivalence via name-equivalence. Fortunately, all of this
works out pleasantly and we may calculate in the natural way, free of
concern. The reader interested in the details is referred to the
appendix \ref{appendix:rho_details}.

\subsection{Substitution}

We use $\Proc$ for the set of processes, $\QProc$ for the set of
names, and $\id{\{}\vec{y} / \vec{x} \id{\}}$ to denote partial maps,
$s : \QProc \rightarrow \QProc$. A map, $s$ lifts, uniquely, to a map
on process terms, $\widehat{s} : \Proc \rightarrow \Proc$ by the
following equations.

\begin{mathpar}
  (0) \psubstp{Q}{P} := 0 \\
  (R \juxtap S) \psubstp{Q}{P}
  :=    
  (R)\psubstp{Q}{P} \juxtap (S) \psubstp{Q}{P} \\
  (x?(y).R) \psubstp{Q}{P}    
  :=    
  (x)\substp{Q}{P} (z)\concat( (R \psubstn{z}{y}) \psubstp{Q}{P} ) \\
  (\lift{x}{R}) \psubstp{Q}{P}  
  :=
  \lift{(x)\substp{Q}{P}}{ R \psubstp{Q}{P} } \\
%   (\dropn{x})  \psubstp{Q}{P}       
%   := 
%   \left\{ 
%     \begin{array}{ccc} 
%       \dropn{\quotep{Q}} & & x \nameeq \quotep{P} \\
%       \dropn{x} & & otherwise \\
%     \end{array}
%   \right. 
  (\dropn{x})  \psubstp{Q}{P}       
  := 
  \left\{ 
    \begin{array}{ccc} 
      Q & & x \nameeq \quotep{P} \\
      \dropn{x} & & otherwise \\
    \end{array}
  \right.
\end{mathpar}
 

where

\begin{eqnarray}
  (x)\id{\{} \lpquote Q \rpquote / \lpquote P \rpquote \id{\}}            = 
  \left\{ 
    \begin{array}{ccc}
      \lpquote Q \rpquote & & x \nameeq \lpquote P \rpquote \\
      x & & otherwise \\
    \end{array}
  \right. \nonumber
\end{eqnarray}

and $z$ is chosen distinct from $\quotep{P}$, $\quotep{Q}$, the free
names in $Q$, and all the names in $R$. Our $\alpha$-equivalence will
be built in the standard way from this substitution.

\begin{remark}\label{rem:no_self_referential_names}
  One consequence of these definitions is that $\forall P. \quotep{P}
  \not\in \freenames{P}$.
\end{remark}

\subsection{ Dynamic quote: an example }

Anticipating something of what's to come, consider applying the
substitution, $\widehat{\id{\{}u / z \id{\}}}$, to the following pair
of processes, $\lift{w}{y!(z)}$ and $w[ \lpquote y!(z) \rpquote ]$.

\begin{eqnarray}
	\lift{w}{y!(z)}\widehat{\id{\{}u / z \id{\}}}
		& = &
		\lift{w}{y!(u)} \nonumber\\
	w[ \lpquote y!(z) \rpquote ] \widehat{ \id{\{}u / z \id{\}} }
		& = &
		w[ \lpquote y!(z) \rpquote ] \nonumber
\end{eqnarray}

Because the body of the process between quotes is impervious to
substitution, we get radically different answers. In fact, by
examining the first process in an input context,
e.g. $x?(z).\lift{w}{y!(z)}$, we see that the process under the lift
operator may be shaped by prefixed inputs binding a name inside it. In
this sense, the lift operator will be seen as a way to dynamically
construct processes before reifying them as names.

Finally equipped with these standard features we can present the
dynamics of the calculus.

\subsubsection{Operational semantics} 

Finally, we introduce the computational dynamics. What marks these
algebras as distinct from other more traditionally studied algebraic
structures, e.g. vector spaces or polynomial rings, is the manner in
which dynamics is captured. In traditional structures, dynamics is typically
expressed through morphisms between such structures, as in linear maps
between vector spaces or morphisms between rings. In algebras
associated with the semantics of computation, the dynamics is
expressed as part of the algebraic structure itself, through a
reduction reduction relation typically denoted by $\red$. Below, we
give a recursive presentation of this relation for the calculus used
in the encoding.

$\red \subseteq \pi \times \pi$
$\red : \pi \to \mathcal{P}(\pi)$

\begin{mathpar}
  \inferrule* [lab=Comm] { \textsf{match}( x_{src}, x_{trgt} ) } { x_{trgt}?(y)P \; | \; x_{src}!\langle {Q} \rangle \red P\{\quotep{Q}/y}\} }
  \and \\
  \inferrule* [lab=Par] {{P} \red {P}'} {{{P} | {Q}} \red {{P}' | {Q}}}
  \and
  \inferrule* [lab=Equiv]{{{P} \scong {P}'} \andalso {{P}' \red {Q}'} \andalso {{Q}' \scong {Q}}}{{P} \red {Q}}
\end{mathpar}

\begin{eqnarray*}
  match_{\equiv} (\quotep{P},\quotep{Q}) & := & P \equiv Q \\
  match_{\dagger}(\quotep{P},\quotep{Q}) & := & \forall R. P|Q \red^{*} R => R \red^{*} 0 \\
  match_{K}(\quotep{P},\quotep{Q}) & := & K \mbox{ for some context } K
\end{eqnarray*}

$u?(x)P | u!\langle Q \rangle \red P\{\quotep{Q}/x\}$

%We write $\wred$ for $\red^*$, and $P\red$ if $\exists Q $ such that $ P \red Q$.
We write $P\red$ if $\exists Q $ such that $ P \red Q$ and $P\not\red$, otherwise.

\section{Replication}

As mentioned before, it is known that replication (and hence
recursion) can be implemented in a higher-order process algebra
\cite{SangiorgiWalker}. As our first example of calculation with the
machinery thus far presented we give the construction explicitly in
the {\rhoc}.

\begin{eqnarray}
	D_{x} & := & \prefix{x}{y}{(\binpar{\outputp{x}{y}}{@{y}})} \nonumber\\
	\bangp_{x}{P} & := & \binpar{{x}!\langle{\binpar{D_{x}}{P}}\rangle}{D_{x}} \nonumber
\end{eqnarray}

\begin{eqnarray}
	\bangp_{x}{P} & & \nonumber\\
	=
	& {x}!\langle{(\prefix{x}{y}{(\outputp{x}{y} | @{y})) | P}}\rangle 
	      | \prefix{x}{y}{(\outputp{x}{y} | @{y})} & \nonumber\\
	\red
	& (\outputp{x}{y} | @{y})\substn{\quotep{(\prefix{x}{y}{(@{y} | \outputp{x}{y})) | P}}}{y} & \nonumber\\
	=
	& \outputp{x}{\quotep{(\prefix{x}{y}{(\outputp{x}{y} | @{y})) | P}}}
	  | {(\prefix{x}{y}{(\outputp{x}{y} | @{y})) | P}} & \nonumber\\
	\red
	& \ldots & \nonumber\\
	\red^*
	& P | P | \ldots & \nonumber
\end{eqnarray}

Of course, this encoding, as an implementation, runs away, unfolding
$\bangp{P}$ eagerly. A lazier and more implementable replication
operator, restricted to input-guarded processes, may be obtained as follows.

\begin{eqnarray}
\bangp{\prefix{u}{v}{P}} 
	:= 
	\binpar{\lift{x}{\prefix{u}{v}{(\binpar{D(x)}{P})}}}{D(x)} \nonumber
\end{eqnarray}

\begin{remark}
  Note that the lazier definition still does not deal with summation
  or mixed summation (i.e. sums over input and output). The reader is
  invited to construct definitions of replication that deal with these
  features. 

  Further, the definitions are parameterized in a name, $x$. Can you,
  gentle reader, make a definition that eliminates this parameter and
  guarantees no accidental interaction between the replication
  machinery and the process being replicated -- i.e. no accidental
  sharing of names used by the process to get its work done and the
  name(s) used by the replication to effect copying. This latter
  revision of the definition of replication is crucial to obtaining
  the expected identity $!!P \sim !P$.
\end{remark}

\begin{remark}\label{rem:paradoxical_combinator}
  The reader familiar with the lambda calculus will have noticed the
  similarity between $D$ and the paradoxical combinator.

  [Ed. note: the existence of this seems to suggest we have to be more
  restrictive on the set of processes and names we admit if we are to
  support no-cloning.]
\end{remark}

\subsubsection{Bisimulation}

The computational dynamics gives rise to another kind of equivalence,
the equivalence of computational behavior. As previously mentioned
this is typically captured \emph{via} some form of bisimulation.

% The notion we use in this paper is weak barbed bisimulation
% \cite{milner91polyadicpi}.

The notion we use in this paper is derived from weak barbed
bisimulation \cite{milner91polyadicpi}. 

\begin{definition}
An \emph{observation relation}, $\downarrow_{\mathcal N}$, over a set
of names, $\mathcal N$, is the smallest relation satisfying the rules
below.

\infrule[Out-barb]{y \in {\mathcal N}, \; x \nameeq y}
		  {\outputp{x}{v} \downarrow_{\mathcal N} x}
\infrule[Par-barb]{\mbox{$P\downarrow_{\mathcal N} x$ or $Q\downarrow_{\mathcal N} x$}}
		  {\binpar{P}{Q} \downarrow_{\mathcal N} x}

We write $P \Downarrow_{\mathcal N} x$ if there is $Q$ such that 
$P \wred Q$ and $Q \downarrow_{\mathcal N} x$.
\end{definition}

\begin{definition}
%\label{def.bbisim}
An  ${\mathcal N}$-\emph{barbed bisimulation} over a set of names, ${\mathcal N}$, is a symmetric binary relation 
${\mathcal S}_{\mathcal N}$ between agents such that $P\rel{S}_{\mathcal N}Q$ implies:
\begin{enumerate}
\item If $P \red P'$ then $Q \wred Q'$ and $P'\rel{S}_{\mathcal N} Q'$.
\item If $P\downarrow_{\mathcal N} x$, then $Q\Downarrow_{\mathcal N} x$.
\end{enumerate}
$P$ is ${\mathcal N}$-barbed bisimilar to $Q$, written
$P \wbbisim_{\mathcal N} Q$, if $P \rel{S}_{\mathcal N} Q$ for some ${\mathcal N}$-barbed bisimulation ${\mathcal S}_{\mathcal N}$.
\end{definition}

$\mathcal{R} \subseteq \pi \times \pi$

$P \mathcal{R} Q => \forall P'. P \red P' \Rightarrow \exists Q'. Q \red Q', P' \mathcal{R} Q'$

$P \vdash x \Rightarrow Q \vdash x$

\begin{mathpar}
  \inferrule*[lab=Out-barb]{x \nameeq y}{{y}!\langle{Q}\rangle \vdash x}
  \and
  \inferrule*[lab=Par-barb]{\mbox{$P\vdash x$ or $Q\vdash x$}}{\binpar{P}{Q} \vdash x}
\end{mathpar}

\subsubsection{Contexts}

One of the principle advantages of computational calculi like the
$\pi$-calculus is a well-defined notion of context,
contextual-equivalence and a correlation between
contextual-equivalence and notions of bisimulation. The notion of
context allows the decomposition of a process into (sub-)process and
its syntactic environment, its context. Thus, a context may be
thought of as a process with a ``hole'' (written $\Box$) in it. The
application of a context $M$ to a process $P$, written $M[P]$, is
tantamount to filling the hole in $M$ with $P$. In this paper we do
not need the full weight of this theory, but do make use of the notion
of context in the proof the main theorem. 

\begin{mathpar}
  \inferrule* [lab=summation] {} {{M_{M},M_{N}} \bc \Box \;|\; x.M_{A} \;|\; M_{M}+M_{N}}
  \and
  \inferrule* [lab=agent] {} {{M_{A}} \bc (\vec{x})M_{P} \;| \; \clift{P_0,\ldots,M_{P},\ldots,P_N}}
  \and \\
  \inferrule* [lab=process] {} {{M_{P}} \bc M_{N} \;| \;P|M_{P} }
\end{mathpar} 

\begin{mathpar}
  \inferrule* [lab=sychronization] {} {M_{N} \bc \Box \;|\; x?M_{F} \;|\; x!M_{C}}
  \and
  \inferrule* [lab=abstraction] {} {{M_{F}} \bc (x)M_{P} }
  \and
  \inferrule* [lab=concretion] {} {{M_{C}} \bc \langle M_{P} \rangle }
  \and \\
  \inferrule* [lab=process] {} {{M_{P}} \bc M_{N} \;| \;P|M_{P} }
\end{mathpar}

\begin{definition}[contextual application] Given a context $M$, and
  process $P$, we define the \emph{contextual application}, $M[P] :=
  M\{P/\Box\}$. That is, the contextual application of M to P is the
  substitution of $P$ for $\Box$ in $M$.
\end{definition}

$\meaningof{-} : L \to \mathcal{P}(\pi)$

\begin{mathpar}
  \inferrule* [lab=collection] {} {\meaningof{true} = \pi, \and \meaningof{~E} = \pi \setminus \meaningof{E}, \and \meaningof{E_{1} \& E_{2}} = \meaningof{E_{1}} \cap \meaningof{E_{2}}}
\end{mathpar}

\begin{mathpar}
  \inferrule* [lab=structure] {} {\meaningof{0} = \{ P \in \pi | P \equiv 0 \}, \and \\ \meaningof{E_1 | E_2} = \{ P \in \pi | P \equiv P_{1} | P_{2}, P_{1} \in \meaningof{E_{1}}, P_{2} \in \meaningof{E_2}\} }
\end{mathpar}

\begin{mathpar}
 \inferrule* [lab=behavior] {} {\meaningof{\langle a?b \rangle E} = \{ P \in \pi | P \equiv Q | u?(y)P', \\ \and \\\\ \and \\ \;\;\; u \in \meaningof{a}, \forall z.P'\{z/y\} \in \meaningof{E\{z/b\}}\}, \and \\ \meaningof{a!E} = \{ P \in \pi | P \equiv Q | x!\langle P' \rangle, x \in \meaningof{a} P' \in \meaningof{E}\} }
\end{mathpar}

\begin{mathpar}
 \inferrule* [lab=nominal] {} {\meaningof{\quotep{E}} = \{ \quotep{P} \in \quotep{\pi} | P \in \meaningof{E} \}, \and \meaningof{\quotep{P}} = \{ \quotep{Q} \in \quotep{\pi} | P \equiv Q \} \and \\ \meaningof{@\quotep{E}} = \{ P \in \pi | P \equiv @x, x \in \meaningof{E} \}}
\end{mathpar}

\begin{eqnarray*}
  \\
  \meaningof{-} : TS \to ST
\end{eqnarray*}

\begin{eqnarray*}
  \\
  L : TS \to ST
\end{eqnarray*}

\begin{eqnarray*}
  \\
  P \models E \iff P \in \meaningof{E}
\end{eqnarray*}

\begin{eqnarray*}
  P \approx_{L} Q \iff \forall E \in L. P \models E \iff Q \models E
\end{eqnarray*}

\begin{eqnarray*}
  P \approx_{K} Q
\end{eqnarray*}

\begin{eqnarray*}
  P \approx Q
\end{eqnarray*}

$\approx_{K} = \approx = \approx_{L}$

\subsubsection{Contextual duality}

Note that contexts extend the quotation operation to a family of
operations from processes to names. Given a context, $M$, we can
define a \emph{nominal context}, $\quotep{M}$ by $\quotep{M}[P] :=
\quotep{M[P]}$. To foreshadow what is to come we observe that these
operations enjoy a duality with processes very much like the duality
between vectors and maps from vectors to scalars.

Further, because the calculus is essentially higher-order, we have a
correspondence between contexts and processes. More specifically,
given a name $x$ and a context $M$ we can construct $M^{*}_{x}$ such
that 

\begin{mathpar}
  M^{*}_{x} | \lift{x}{P} \red M[P]
\end{mathpar}

namely,

\begin{mathpar}
  M^{*}_{x} := x?(u).M[\dropn{u}]
\end{mathpar}

The dependence of $M^{*}_{x}$ on a name makes it an abstraction, 

\begin{mathpar}
  M^{*} := (x)x?(u).M[\dropn{u}]
\end{mathpar}

\subsection{Additional notation}

It will sometimes be convenient to denote the process a name
quotes. We already have the notation $x = \quotep{P}$, but it will be
convenient to introduce an alternate notation, $\procn{x}$, when we
want to emphasize the connection to the use of the name. Note that, by
virtue of name equivalence, $\quotep{\procn{x}} \nameeq x$; so, the
notation is consistent with previous definitions.

Further, because names have structure it is possible to effect
substitutions on the basis of that structure. This means we need to
upgrade our notation for substitutions, which we accomplish by
adapting comprehension notation. Thus,

\begin{mathpar}
  P\{ y / x : x \in S \}
\end{mathpar}

is interpreted to mean the process derived from P by replacing (in a
capture-avoiding manner) each occurrence of $x$ in $S$ by $y$. For example,

\begin{mathpar}
  P\{ \quotep{\procn{x}|\procn{x}} / x : x \in \freenames{P} \}
\end{mathpar}

will replace each (occurrence) of a free name $x$ in $P$ by
$\quotep{\procn{x}|\procn{x}}$.

Also, we will avail ourselves of the notation $x^{L}$ and $x^{R}$ to
denote injections of a name into disjoint copies of the name
space. There are numerous ways to accomplish this. One example can be
found in \cite{MeredithR05}. This notation overloads to vectors of
names: $\vec{x}^{\pi} := (x_{i}^{\pi} \; : \; 0 \leq i < |\vec{x}| )$ where $\pi \in \{L,R\}$.

We also use $P^{\Box} := P|\Box$.

In \cite{MeredithR05} an interpretation of the new operator is
given. It turns out that there are several possible interpretations
all enjoying the requisite algebraic properties of the operator (see
\cite{milner91polyadicpi}). We will therefore make liberal use of
$(\nu\; \vec{x})P$.

% subsection the_syntax_and_semantics_of_the_notation_system (end)   

\input{qm2pi.qmops} 

\input{qm2pi.sterngerlach} 

\input{qm2pi.metric} 

% section concurrent_process_calculi (end)

%\input{qm2pi.proofsketch}

% section proof sketch (end)

%\input{qm2pi.slviaknots} 

% section spatial logic via knots (end)

\input{qm2pi.conclusion}

% section conclusion (end)

%\input{qm2pi.dtcodes} 

% section wiring algorithm (end)

\input{qm2pi.ack} 

% section acknowledgments (end)

\newpage


\bibliographystyle{plain}   
\bibliography{../../biblios/main.bib}

\input{qm2pi.rhodetails}

\end{document}

 

% section acknowledgments (end)

\newpage


\bibliographystyle{plain}   
\bibliography{../../biblios/main.bib}

\documentclass[12pt]{llncs}
%\documentclass{jktr}

\usepackage[pdftex]{hyperref}                   
\usepackage {listings}
\usepackage {mathpartir}
\usepackage{bcprules}
%\usepackage{listings}
                       
\usepackage{graphicx} 
%\usepackage[margins=2.5cm,nohead,nofoot]{geometry}
%\usepackage{geometry}
\usepackage{amsfonts}
\usepackage{amstext}
\usepackage{latexsym}
\usepackage{amssymb}
\usepackage{color}


%\include{myPreamble}
\include{qm2pi.local} 

%\ifpdf
%\usepackage[pdftex]{graphicx}
%\else
%\usepackage{graphicx}
%\fi

 % \ifpdf
%  \usepackage{pdfsync}
%  \if


%\title{Brief Article}
%\author{David F. Snyder}
%\author{L.G. Meredith}

%\address{Dept. of Math., Texas State University--San Marcos, San Marcos, TX 78666}
       
\pagestyle{empty}


\begin{document}

\lstset{language=[Objective]Caml,frame=shadowbox}

\input{qm2pi.front}

% section front matter (end)

\input{qm2pi.intro} 
 
% section introduction (end)

% \input{qm2pi.knotations} 

% section notation (end)

\input{qm2pi.process.calculi} 

% section concurrent_process_calculi_and_spatial_logics_ (end)
    
%\input{qm2pi.knots2pi} 

%\input{qm2pi.trefoil} 

%\input{qm2pi.mainthm} 

% subsection basic_interpretation (end)

%\input{qm2pi.rho.presentation} 
\subsection{The syntax and semantics of the notation system}\label{sub:the_syntax_and_semantics_of_the_notation_system} % (fold)

We now summarize a technical presentation of the calculus that
embodies our theory of dynamics. The typical presentation of such a
calculus follows the style of giving generators and relations on
them. The grammar, below, describing term constructors, freely
generates the set of processes, $\Proc$. This set is then quotiented
by a relation known as structural congruence and it is over this set
that the notion of dynamics is expressed. This presentation is
essentially that of \cite{MeredithR05} with the addition of
polyadicity and summation. For readability we have relegated some of
the technical subtleties to an appendix.

\subsubsection{Process grammar}\label{subsub:process_grammar}

\begin{mathpar}
  \inferrule* [lab=synchronization] {} {{M} \bc \pzero \;|\; x?F \;|\; x!C }
  \and
  \inferrule* [lab=abstraction] {} {{F} \bc (x)P}
  \and
  \inferrule* [lab=concretion] {} {{C} \bc \langle Q \rangle}
  \and
  \inferrule* [lab=process] {} {{P,Q} \bc M \;| \;P|Q \;|\; @{x}}
  \and
  \inferrule* [lab=name] {} {{x} \bc \quotep{P}}
\end{mathpar} 

Note that $\vec{x}$ (resp. $\vec{P}$) denotes a vector of names
(resp. processes) of length $|\vec{x}|$ (resp. $|\vec{P}|$). We adopt
the following useful abbreviations.

\begin{mathpar}
   x?(\vec{y}).P := x.(\vec{y})P \and  x\clift{\vec{P}} := x.\clift{\vec{P}}
   \and x!(y) := \lift{x}{\dropn{y}}
   \and \Pi_{i=0}^{n-1}P_i := P_0 | \ldots | P_{n-1}
\end{mathpar}

\subsubsection{Structural congruence}

\paragraph{Free and bound names and alpha-equivalence.} At the
core of structural equivalence is alpha-equivalence which identifies
process that are the same up to a change of variable. Formally, we
recognize the distinction between free and bound names. The free names
of a process, $\freenames{P}$, may be calculated recursively as
follows:

\begin{mathpar}
\freenames{\pzero} := \emptyset
  \and \\
  \freenames{x?(y).P} := \{ x \} \cup (\freenames{P} \setminus \{ y \})
  \and 
  \freenames{x!\langle P \rangle} := \{ x \} \cup \{ P \} 
  \and \\
  \freenames{P|Q} := \freenames{P} \cup \freenames{Q}
  \and \\
  \freenames{@{x}} := \{ x \}
\end{mathpar}

$\pi$
$\quotep{\pi}$

$\freenames{-} : \pi \to \mathcal{P}(\quotep{\pi})$

\begin{eqnarray*}
  \freenames{\pzero} & := & \emptyset \\
  \freenames{x?(y).P} & := & \{ x \} \cup (\freenames{P} \setminus \{ y \}) \\
  \freenames{x!\langle P \rangle} & := & \{ x \} \cup \{ P \} \\
  \freenames{P|Q} & := & \freenames{P} \cup \freenames{Q} \\
  \freenames{\dropn{x}} & := & \{ x \}
\end{eqnarray*}

The bound names of a process, $\boundnames{P}$, are those names occurring in $P$
that are not free. For example, in $x?(y).0$, the name $x$ is free, while $y$ is bound.

\begin{mathpar}
  \inferrule* [lab=monoidal-laws] {} { P|Q \equiv Q|P \and P|0 \equiv P \and P|(Q|R) \equiv (P|Q)|R }
\end{mathpar}

\begin{mathpar}
  \inferrule* [lab=alpha-equivalence] {} { (x)P \equiv (y)P\{y/x\} \and y \not\in \freenames{P} }
\end{mathpar}

\begin{definition}
Then two processes, $P,Q$, are alpha-equivalent if $P = Q\{\vec{y}/\vec{x}\}$ for
some $\vec{x} \in \boundnames{Q},\vec{y} \in \boundnames{P}$, where $Q\{\vec{y}/\vec{x}\}$
denotes the capture-avoiding substitution of $\vec{y}$ for $\vec{x}$ in $Q$.
\end{definition}

\begin{definition}
  The {\em structural congruence} \cite{SangiorgiWalker} , $\equiv$,
  between processes is the least congruence containing
  alpha-equivalence, satisfying the abelian monoid laws
  (associativity, commutativity and $\pzero$ as identity) for parallel
  composition $|$ and for summation $+$.
\end{definition}

\subsection{Name equivalence}

We take name equivalence, written $\nameeq$, to be the smallest
equivalence relation generated by the following rules.

\begin{mathpar}
\inferrule*[lab=Quote-drop]
{ }
{ \quotep{@{x}} \nameeq x }

\inferrule*[lab=Struct-equiv]
{ P \scong Q }
{ \quotep{P} \nameeq \quotep{Q} }
\end{mathpar}

The astute reader will have noticed that the mutual recursion of names
and processes imposes a mutual recursion on alpha-equivalence and
structural equivalence via name-equivalence. Fortunately, all of this
works out pleasantly and we may calculate in the natural way, free of
concern. The reader interested in the details is referred to the
appendix \ref{appendix:rho_details}.

\subsection{Substitution}

We use $\Proc$ for the set of processes, $\QProc$ for the set of
names, and $\id{\{}\vec{y} / \vec{x} \id{\}}$ to denote partial maps,
$s : \QProc \rightarrow \QProc$. A map, $s$ lifts, uniquely, to a map
on process terms, $\widehat{s} : \Proc \rightarrow \Proc$ by the
following equations.

\begin{mathpar}
  (0) \psubstp{Q}{P} := 0 \\
  (R \juxtap S) \psubstp{Q}{P}
  :=    
  (R)\psubstp{Q}{P} \juxtap (S) \psubstp{Q}{P} \\
  (x?(y).R) \psubstp{Q}{P}    
  :=    
  (x)\substp{Q}{P} (z)\concat( (R \psubstn{z}{y}) \psubstp{Q}{P} ) \\
  (\lift{x}{R}) \psubstp{Q}{P}  
  :=
  \lift{(x)\substp{Q}{P}}{ R \psubstp{Q}{P} } \\
%   (\dropn{x})  \psubstp{Q}{P}       
%   := 
%   \left\{ 
%     \begin{array}{ccc} 
%       \dropn{\quotep{Q}} & & x \nameeq \quotep{P} \\
%       \dropn{x} & & otherwise \\
%     \end{array}
%   \right. 
  (\dropn{x})  \psubstp{Q}{P}       
  := 
  \left\{ 
    \begin{array}{ccc} 
      Q & & x \nameeq \quotep{P} \\
      \dropn{x} & & otherwise \\
    \end{array}
  \right.
\end{mathpar}
 

where

\begin{eqnarray}
  (x)\id{\{} \lpquote Q \rpquote / \lpquote P \rpquote \id{\}}            = 
  \left\{ 
    \begin{array}{ccc}
      \lpquote Q \rpquote & & x \nameeq \lpquote P \rpquote \\
      x & & otherwise \\
    \end{array}
  \right. \nonumber
\end{eqnarray}

and $z$ is chosen distinct from $\quotep{P}$, $\quotep{Q}$, the free
names in $Q$, and all the names in $R$. Our $\alpha$-equivalence will
be built in the standard way from this substitution.

\begin{remark}\label{rem:no_self_referential_names}
  One consequence of these definitions is that $\forall P. \quotep{P}
  \not\in \freenames{P}$.
\end{remark}

\subsection{ Dynamic quote: an example }

Anticipating something of what's to come, consider applying the
substitution, $\widehat{\id{\{}u / z \id{\}}}$, to the following pair
of processes, $\lift{w}{y!(z)}$ and $w[ \lpquote y!(z) \rpquote ]$.

\begin{eqnarray}
	\lift{w}{y!(z)}\widehat{\id{\{}u / z \id{\}}}
		& = &
		\lift{w}{y!(u)} \nonumber\\
	w[ \lpquote y!(z) \rpquote ] \widehat{ \id{\{}u / z \id{\}} }
		& = &
		w[ \lpquote y!(z) \rpquote ] \nonumber
\end{eqnarray}

Because the body of the process between quotes is impervious to
substitution, we get radically different answers. In fact, by
examining the first process in an input context,
e.g. $x?(z).\lift{w}{y!(z)}$, we see that the process under the lift
operator may be shaped by prefixed inputs binding a name inside it. In
this sense, the lift operator will be seen as a way to dynamically
construct processes before reifying them as names.

Finally equipped with these standard features we can present the
dynamics of the calculus.

\subsubsection{Operational semantics} 

Finally, we introduce the computational dynamics. What marks these
algebras as distinct from other more traditionally studied algebraic
structures, e.g. vector spaces or polynomial rings, is the manner in
which dynamics is captured. In traditional structures, dynamics is typically
expressed through morphisms between such structures, as in linear maps
between vector spaces or morphisms between rings. In algebras
associated with the semantics of computation, the dynamics is
expressed as part of the algebraic structure itself, through a
reduction reduction relation typically denoted by $\red$. Below, we
give a recursive presentation of this relation for the calculus used
in the encoding.

$\red \subseteq \pi \times \pi$
$\red : \pi \to \mathcal{P}(\pi)$

\begin{mathpar}
  \inferrule* [lab=Comm] { \textsf{match}( x_{src}, x_{trgt} ) } { x_{trgt}?(y)P \; | \; x_{src}!\langle {Q} \rangle \red P\{\quotep{Q}/y}\} }
  \and \\
  \inferrule* [lab=Par] {{P} \red {P}'} {{{P} | {Q}} \red {{P}' | {Q}}}
  \and
  \inferrule* [lab=Equiv]{{{P} \scong {P}'} \andalso {{P}' \red {Q}'} \andalso {{Q}' \scong {Q}}}{{P} \red {Q}}
\end{mathpar}

\begin{eqnarray*}
  match_{\equiv} (\quotep{P},\quotep{Q}) & := & P \equiv Q \\
  match_{\dagger}(\quotep{P},\quotep{Q}) & := & \forall R. P|Q \red^{*} R => R \red^{*} 0 \\
  match_{K}(\quotep{P},\quotep{Q}) & := & K \mbox{ for some context } K
\end{eqnarray*}

$u?(x)P | u!\langle Q \rangle \red P\{\quotep{Q}/x\}$

%We write $\wred$ for $\red^*$, and $P\red$ if $\exists Q $ such that $ P \red Q$.
We write $P\red$ if $\exists Q $ such that $ P \red Q$ and $P\not\red$, otherwise.

\section{Replication}

As mentioned before, it is known that replication (and hence
recursion) can be implemented in a higher-order process algebra
\cite{SangiorgiWalker}. As our first example of calculation with the
machinery thus far presented we give the construction explicitly in
the {\rhoc}.

\begin{eqnarray}
	D_{x} & := & \prefix{x}{y}{(\binpar{\outputp{x}{y}}{@{y}})} \nonumber\\
	\bangp_{x}{P} & := & \binpar{{x}!\langle{\binpar{D_{x}}{P}}\rangle}{D_{x}} \nonumber
\end{eqnarray}

\begin{eqnarray}
	\bangp_{x}{P} & & \nonumber\\
	=
	& {x}!\langle{(\prefix{x}{y}{(\outputp{x}{y} | @{y})) | P}}\rangle 
	      | \prefix{x}{y}{(\outputp{x}{y} | @{y})} & \nonumber\\
	\red
	& (\outputp{x}{y} | @{y})\substn{\quotep{(\prefix{x}{y}{(@{y} | \outputp{x}{y})) | P}}}{y} & \nonumber\\
	=
	& \outputp{x}{\quotep{(\prefix{x}{y}{(\outputp{x}{y} | @{y})) | P}}}
	  | {(\prefix{x}{y}{(\outputp{x}{y} | @{y})) | P}} & \nonumber\\
	\red
	& \ldots & \nonumber\\
	\red^*
	& P | P | \ldots & \nonumber
\end{eqnarray}

Of course, this encoding, as an implementation, runs away, unfolding
$\bangp{P}$ eagerly. A lazier and more implementable replication
operator, restricted to input-guarded processes, may be obtained as follows.

\begin{eqnarray}
\bangp{\prefix{u}{v}{P}} 
	:= 
	\binpar{\lift{x}{\prefix{u}{v}{(\binpar{D(x)}{P})}}}{D(x)} \nonumber
\end{eqnarray}

\begin{remark}
  Note that the lazier definition still does not deal with summation
  or mixed summation (i.e. sums over input and output). The reader is
  invited to construct definitions of replication that deal with these
  features. 

  Further, the definitions are parameterized in a name, $x$. Can you,
  gentle reader, make a definition that eliminates this parameter and
  guarantees no accidental interaction between the replication
  machinery and the process being replicated -- i.e. no accidental
  sharing of names used by the process to get its work done and the
  name(s) used by the replication to effect copying. This latter
  revision of the definition of replication is crucial to obtaining
  the expected identity $!!P \sim !P$.
\end{remark}

\begin{remark}\label{rem:paradoxical_combinator}
  The reader familiar with the lambda calculus will have noticed the
  similarity between $D$ and the paradoxical combinator.

  [Ed. note: the existence of this seems to suggest we have to be more
  restrictive on the set of processes and names we admit if we are to
  support no-cloning.]
\end{remark}

\subsubsection{Bisimulation}

The computational dynamics gives rise to another kind of equivalence,
the equivalence of computational behavior. As previously mentioned
this is typically captured \emph{via} some form of bisimulation.

% The notion we use in this paper is weak barbed bisimulation
% \cite{milner91polyadicpi}.

The notion we use in this paper is derived from weak barbed
bisimulation \cite{milner91polyadicpi}. 

\begin{definition}
An \emph{observation relation}, $\downarrow_{\mathcal N}$, over a set
of names, $\mathcal N$, is the smallest relation satisfying the rules
below.

\infrule[Out-barb]{y \in {\mathcal N}, \; x \nameeq y}
		  {\outputp{x}{v} \downarrow_{\mathcal N} x}
\infrule[Par-barb]{\mbox{$P\downarrow_{\mathcal N} x$ or $Q\downarrow_{\mathcal N} x$}}
		  {\binpar{P}{Q} \downarrow_{\mathcal N} x}

We write $P \Downarrow_{\mathcal N} x$ if there is $Q$ such that 
$P \wred Q$ and $Q \downarrow_{\mathcal N} x$.
\end{definition}

\begin{definition}
%\label{def.bbisim}
An  ${\mathcal N}$-\emph{barbed bisimulation} over a set of names, ${\mathcal N}$, is a symmetric binary relation 
${\mathcal S}_{\mathcal N}$ between agents such that $P\rel{S}_{\mathcal N}Q$ implies:
\begin{enumerate}
\item If $P \red P'$ then $Q \wred Q'$ and $P'\rel{S}_{\mathcal N} Q'$.
\item If $P\downarrow_{\mathcal N} x$, then $Q\Downarrow_{\mathcal N} x$.
\end{enumerate}
$P$ is ${\mathcal N}$-barbed bisimilar to $Q$, written
$P \wbbisim_{\mathcal N} Q$, if $P \rel{S}_{\mathcal N} Q$ for some ${\mathcal N}$-barbed bisimulation ${\mathcal S}_{\mathcal N}$.
\end{definition}

$\mathcal{R} \subseteq \pi \times \pi$

$P \mathcal{R} Q => \forall P'. P \red P' \Rightarrow \exists Q'. Q \red Q', P' \mathcal{R} Q'$

$P \vdash x \Rightarrow Q \vdash x$

\begin{mathpar}
  \inferrule*[lab=Out-barb]{x \nameeq y}{{y}!\langle{Q}\rangle \vdash x}
  \and
  \inferrule*[lab=Par-barb]{\mbox{$P\vdash x$ or $Q\vdash x$}}{\binpar{P}{Q} \vdash x}
\end{mathpar}

\subsubsection{Contexts}

One of the principle advantages of computational calculi like the
$\pi$-calculus is a well-defined notion of context,
contextual-equivalence and a correlation between
contextual-equivalence and notions of bisimulation. The notion of
context allows the decomposition of a process into (sub-)process and
its syntactic environment, its context. Thus, a context may be
thought of as a process with a ``hole'' (written $\Box$) in it. The
application of a context $M$ to a process $P$, written $M[P]$, is
tantamount to filling the hole in $M$ with $P$. In this paper we do
not need the full weight of this theory, but do make use of the notion
of context in the proof the main theorem. 

\begin{mathpar}
  \inferrule* [lab=summation] {} {{M_{M},M_{N}} \bc \Box \;|\; x.M_{A} \;|\; M_{M}+M_{N}}
  \and
  \inferrule* [lab=agent] {} {{M_{A}} \bc (\vec{x})M_{P} \;| \; \clift{P_0,\ldots,M_{P},\ldots,P_N}}
  \and \\
  \inferrule* [lab=process] {} {{M_{P}} \bc M_{N} \;| \;P|M_{P} }
\end{mathpar} 

\begin{mathpar}
  \inferrule* [lab=sychronization] {} {M_{N} \bc \Box \;|\; x?M_{F} \;|\; x!M_{C}}
  \and
  \inferrule* [lab=abstraction] {} {{M_{F}} \bc (x)M_{P} }
  \and
  \inferrule* [lab=concretion] {} {{M_{C}} \bc \langle M_{P} \rangle }
  \and \\
  \inferrule* [lab=process] {} {{M_{P}} \bc M_{N} \;| \;P|M_{P} }
\end{mathpar}

\begin{definition}[contextual application] Given a context $M$, and
  process $P$, we define the \emph{contextual application}, $M[P] :=
  M\{P/\Box\}$. That is, the contextual application of M to P is the
  substitution of $P$ for $\Box$ in $M$.
\end{definition}

$\meaningof{-} : L \to \mathcal{P}(\pi)$

\begin{mathpar}
  \inferrule* [lab=collection] {} {\meaningof{true} = \pi, \and \meaningof{~E} = \pi \setminus \meaningof{E}, \and \meaningof{E_{1} \& E_{2}} = \meaningof{E_{1}} \cap \meaningof{E_{2}}}
\end{mathpar}

\begin{mathpar}
  \inferrule* [lab=structure] {} {\meaningof{0} = \{ P \in \pi | P \equiv 0 \}, \and \\ \meaningof{E_1 | E_2} = \{ P \in \pi | P \equiv P_{1} | P_{2}, P_{1} \in \meaningof{E_{1}}, P_{2} \in \meaningof{E_2}\} }
\end{mathpar}

\begin{mathpar}
 \inferrule* [lab=behavior] {} {\meaningof{\langle a?b \rangle E} = \{ P \in \pi | P \equiv Q | u?(y)P', \\ \and \\\\ \and \\ \;\;\; u \in \meaningof{a}, \forall z.P'\{z/y\} \in \meaningof{E\{z/b\}}\}, \and \\ \meaningof{a!E} = \{ P \in \pi | P \equiv Q | x!\langle P' \rangle, x \in \meaningof{a} P' \in \meaningof{E}\} }
\end{mathpar}

\begin{mathpar}
 \inferrule* [lab=nominal] {} {\meaningof{\quotep{E}} = \{ \quotep{P} \in \quotep{\pi} | P \in \meaningof{E} \}, \and \meaningof{\quotep{P}} = \{ \quotep{Q} \in \quotep{\pi} | P \equiv Q \} \and \\ \meaningof{@\quotep{E}} = \{ P \in \pi | P \equiv @x, x \in \meaningof{E} \}}
\end{mathpar}

\begin{eqnarray*}
  \\
  \meaningof{-} : TS \to ST
\end{eqnarray*}

\begin{eqnarray*}
  \\
  L : TS \to ST
\end{eqnarray*}

\begin{eqnarray*}
  \\
  P \models E \iff P \in \meaningof{E}
\end{eqnarray*}

\begin{eqnarray*}
  P \approx_{L} Q \iff \forall E \in L. P \models E \iff Q \models E
\end{eqnarray*}

\begin{eqnarray*}
  P \approx_{K} Q
\end{eqnarray*}

\begin{eqnarray*}
  P \approx Q
\end{eqnarray*}

$\approx_{K} = \approx = \approx_{L}$

\subsubsection{Contextual duality}

Note that contexts extend the quotation operation to a family of
operations from processes to names. Given a context, $M$, we can
define a \emph{nominal context}, $\quotep{M}$ by $\quotep{M}[P] :=
\quotep{M[P]}$. To foreshadow what is to come we observe that these
operations enjoy a duality with processes very much like the duality
between vectors and maps from vectors to scalars.

Further, because the calculus is essentially higher-order, we have a
correspondence between contexts and processes. More specifically,
given a name $x$ and a context $M$ we can construct $M^{*}_{x}$ such
that 

\begin{mathpar}
  M^{*}_{x} | \lift{x}{P} \red M[P]
\end{mathpar}

namely,

\begin{mathpar}
  M^{*}_{x} := x?(u).M[\dropn{u}]
\end{mathpar}

The dependence of $M^{*}_{x}$ on a name makes it an abstraction, 

\begin{mathpar}
  M^{*} := (x)x?(u).M[\dropn{u}]
\end{mathpar}

\subsection{Additional notation}

It will sometimes be convenient to denote the process a name
quotes. We already have the notation $x = \quotep{P}$, but it will be
convenient to introduce an alternate notation, $\procn{x}$, when we
want to emphasize the connection to the use of the name. Note that, by
virtue of name equivalence, $\quotep{\procn{x}} \nameeq x$; so, the
notation is consistent with previous definitions.

Further, because names have structure it is possible to effect
substitutions on the basis of that structure. This means we need to
upgrade our notation for substitutions, which we accomplish by
adapting comprehension notation. Thus,

\begin{mathpar}
  P\{ y / x : x \in S \}
\end{mathpar}

is interpreted to mean the process derived from P by replacing (in a
capture-avoiding manner) each occurrence of $x$ in $S$ by $y$. For example,

\begin{mathpar}
  P\{ \quotep{\procn{x}|\procn{x}} / x : x \in \freenames{P} \}
\end{mathpar}

will replace each (occurrence) of a free name $x$ in $P$ by
$\quotep{\procn{x}|\procn{x}}$.

Also, we will avail ourselves of the notation $x^{L}$ and $x^{R}$ to
denote injections of a name into disjoint copies of the name
space. There are numerous ways to accomplish this. One example can be
found in \cite{MeredithR05}. This notation overloads to vectors of
names: $\vec{x}^{\pi} := (x_{i}^{\pi} \; : \; 0 \leq i < |\vec{x}| )$ where $\pi \in \{L,R\}$.

We also use $P^{\Box} := P|\Box$.

In \cite{MeredithR05} an interpretation of the new operator is
given. It turns out that there are several possible interpretations
all enjoying the requisite algebraic properties of the operator (see
\cite{milner91polyadicpi}). We will therefore make liberal use of
$(\nu\; \vec{x})P$.

% subsection the_syntax_and_semantics_of_the_notation_system (end)   

\input{qm2pi.qmops} 

\input{qm2pi.sterngerlach} 

\input{qm2pi.metric} 

% section concurrent_process_calculi (end)

%\input{qm2pi.proofsketch}

% section proof sketch (end)

%\input{qm2pi.slviaknots} 

% section spatial logic via knots (end)

\input{qm2pi.conclusion}

% section conclusion (end)

%\input{qm2pi.dtcodes} 

% section wiring algorithm (end)

\input{qm2pi.ack} 

% section acknowledgments (end)

\newpage


\bibliographystyle{plain}   
\bibliography{../../biblios/main.bib}

\input{qm2pi.rhodetails}

\end{document}



\end{document}

 

%\ifpdf
%\usepackage[pdftex]{graphicx}
%\else
%\usepackage{graphicx}
%\fi

 % \ifpdf
%  \usepackage{pdfsync}
%  \if


%\title{Brief Article}
%\author{David F. Snyder}
%\author{L.G. Meredith}

%\address{Dept. of Math., Texas State University--San Marcos, San Marcos, TX 78666}
       
\pagestyle{empty}


\begin{document}

\lstset{language=[Objective]Caml,frame=shadowbox}

\documentclass[12pt]{llncs}
%\documentclass{jktr}

\usepackage[pdftex]{hyperref}                   
\usepackage {listings}
\usepackage {mathpartir}
\usepackage{bcprules}
%\usepackage{listings}
                       
\usepackage{graphicx} 
%\usepackage[margins=2.5cm,nohead,nofoot]{geometry}
%\usepackage{geometry}
\usepackage{amsfonts}
\usepackage{amstext}
\usepackage{latexsym}
\usepackage{amssymb}
\usepackage{color}


%\include{myPreamble}
\documentclass[12pt]{llncs}
%\documentclass{jktr}

\usepackage[pdftex]{hyperref}                   
\usepackage {listings}
\usepackage {mathpartir}
\usepackage{bcprules}
%\usepackage{listings}
                       
\usepackage{graphicx} 
%\usepackage[margins=2.5cm,nohead,nofoot]{geometry}
%\usepackage{geometry}
\usepackage{amsfonts}
\usepackage{amstext}
\usepackage{latexsym}
\usepackage{amssymb}
\usepackage{color}


%\include{myPreamble}
\include{qm2pi.local} 

%\ifpdf
%\usepackage[pdftex]{graphicx}
%\else
%\usepackage{graphicx}
%\fi

 % \ifpdf
%  \usepackage{pdfsync}
%  \if


%\title{Brief Article}
%\author{David F. Snyder}
%\author{L.G. Meredith}

%\address{Dept. of Math., Texas State University--San Marcos, San Marcos, TX 78666}
       
\pagestyle{empty}


\begin{document}

\lstset{language=[Objective]Caml,frame=shadowbox}

\input{qm2pi.front}

% section front matter (end)

\input{qm2pi.intro} 
 
% section introduction (end)

% \input{qm2pi.knotations} 

% section notation (end)

\input{qm2pi.process.calculi} 

% section concurrent_process_calculi_and_spatial_logics_ (end)
    
%\input{qm2pi.knots2pi} 

%\input{qm2pi.trefoil} 

%\input{qm2pi.mainthm} 

% subsection basic_interpretation (end)

%\input{qm2pi.rho.presentation} 
\subsection{The syntax and semantics of the notation system}\label{sub:the_syntax_and_semantics_of_the_notation_system} % (fold)

We now summarize a technical presentation of the calculus that
embodies our theory of dynamics. The typical presentation of such a
calculus follows the style of giving generators and relations on
them. The grammar, below, describing term constructors, freely
generates the set of processes, $\Proc$. This set is then quotiented
by a relation known as structural congruence and it is over this set
that the notion of dynamics is expressed. This presentation is
essentially that of \cite{MeredithR05} with the addition of
polyadicity and summation. For readability we have relegated some of
the technical subtleties to an appendix.

\subsubsection{Process grammar}\label{subsub:process_grammar}

\begin{mathpar}
  \inferrule* [lab=synchronization] {} {{M} \bc \pzero \;|\; x?F \;|\; x!C }
  \and
  \inferrule* [lab=abstraction] {} {{F} \bc (x)P}
  \and
  \inferrule* [lab=concretion] {} {{C} \bc \langle Q \rangle}
  \and
  \inferrule* [lab=process] {} {{P,Q} \bc M \;| \;P|Q \;|\; @{x}}
  \and
  \inferrule* [lab=name] {} {{x} \bc \quotep{P}}
\end{mathpar} 

Note that $\vec{x}$ (resp. $\vec{P}$) denotes a vector of names
(resp. processes) of length $|\vec{x}|$ (resp. $|\vec{P}|$). We adopt
the following useful abbreviations.

\begin{mathpar}
   x?(\vec{y}).P := x.(\vec{y})P \and  x\clift{\vec{P}} := x.\clift{\vec{P}}
   \and x!(y) := \lift{x}{\dropn{y}}
   \and \Pi_{i=0}^{n-1}P_i := P_0 | \ldots | P_{n-1}
\end{mathpar}

\subsubsection{Structural congruence}

\paragraph{Free and bound names and alpha-equivalence.} At the
core of structural equivalence is alpha-equivalence which identifies
process that are the same up to a change of variable. Formally, we
recognize the distinction between free and bound names. The free names
of a process, $\freenames{P}$, may be calculated recursively as
follows:

\begin{mathpar}
\freenames{\pzero} := \emptyset
  \and \\
  \freenames{x?(y).P} := \{ x \} \cup (\freenames{P} \setminus \{ y \})
  \and 
  \freenames{x!\langle P \rangle} := \{ x \} \cup \{ P \} 
  \and \\
  \freenames{P|Q} := \freenames{P} \cup \freenames{Q}
  \and \\
  \freenames{@{x}} := \{ x \}
\end{mathpar}

$\pi$
$\quotep{\pi}$

$\freenames{-} : \pi \to \mathcal{P}(\quotep{\pi})$

\begin{eqnarray*}
  \freenames{\pzero} & := & \emptyset \\
  \freenames{x?(y).P} & := & \{ x \} \cup (\freenames{P} \setminus \{ y \}) \\
  \freenames{x!\langle P \rangle} & := & \{ x \} \cup \{ P \} \\
  \freenames{P|Q} & := & \freenames{P} \cup \freenames{Q} \\
  \freenames{\dropn{x}} & := & \{ x \}
\end{eqnarray*}

The bound names of a process, $\boundnames{P}$, are those names occurring in $P$
that are not free. For example, in $x?(y).0$, the name $x$ is free, while $y$ is bound.

\begin{mathpar}
  \inferrule* [lab=monoidal-laws] {} { P|Q \equiv Q|P \and P|0 \equiv P \and P|(Q|R) \equiv (P|Q)|R }
\end{mathpar}

\begin{mathpar}
  \inferrule* [lab=alpha-equivalence] {} { (x)P \equiv (y)P\{y/x\} \and y \not\in \freenames{P} }
\end{mathpar}

\begin{definition}
Then two processes, $P,Q$, are alpha-equivalent if $P = Q\{\vec{y}/\vec{x}\}$ for
some $\vec{x} \in \boundnames{Q},\vec{y} \in \boundnames{P}$, where $Q\{\vec{y}/\vec{x}\}$
denotes the capture-avoiding substitution of $\vec{y}$ for $\vec{x}$ in $Q$.
\end{definition}

\begin{definition}
  The {\em structural congruence} \cite{SangiorgiWalker} , $\equiv$,
  between processes is the least congruence containing
  alpha-equivalence, satisfying the abelian monoid laws
  (associativity, commutativity and $\pzero$ as identity) for parallel
  composition $|$ and for summation $+$.
\end{definition}

\subsection{Name equivalence}

We take name equivalence, written $\nameeq$, to be the smallest
equivalence relation generated by the following rules.

\begin{mathpar}
\inferrule*[lab=Quote-drop]
{ }
{ \quotep{@{x}} \nameeq x }

\inferrule*[lab=Struct-equiv]
{ P \scong Q }
{ \quotep{P} \nameeq \quotep{Q} }
\end{mathpar}

The astute reader will have noticed that the mutual recursion of names
and processes imposes a mutual recursion on alpha-equivalence and
structural equivalence via name-equivalence. Fortunately, all of this
works out pleasantly and we may calculate in the natural way, free of
concern. The reader interested in the details is referred to the
appendix \ref{appendix:rho_details}.

\subsection{Substitution}

We use $\Proc$ for the set of processes, $\QProc$ for the set of
names, and $\id{\{}\vec{y} / \vec{x} \id{\}}$ to denote partial maps,
$s : \QProc \rightarrow \QProc$. A map, $s$ lifts, uniquely, to a map
on process terms, $\widehat{s} : \Proc \rightarrow \Proc$ by the
following equations.

\begin{mathpar}
  (0) \psubstp{Q}{P} := 0 \\
  (R \juxtap S) \psubstp{Q}{P}
  :=    
  (R)\psubstp{Q}{P} \juxtap (S) \psubstp{Q}{P} \\
  (x?(y).R) \psubstp{Q}{P}    
  :=    
  (x)\substp{Q}{P} (z)\concat( (R \psubstn{z}{y}) \psubstp{Q}{P} ) \\
  (\lift{x}{R}) \psubstp{Q}{P}  
  :=
  \lift{(x)\substp{Q}{P}}{ R \psubstp{Q}{P} } \\
%   (\dropn{x})  \psubstp{Q}{P}       
%   := 
%   \left\{ 
%     \begin{array}{ccc} 
%       \dropn{\quotep{Q}} & & x \nameeq \quotep{P} \\
%       \dropn{x} & & otherwise \\
%     \end{array}
%   \right. 
  (\dropn{x})  \psubstp{Q}{P}       
  := 
  \left\{ 
    \begin{array}{ccc} 
      Q & & x \nameeq \quotep{P} \\
      \dropn{x} & & otherwise \\
    \end{array}
  \right.
\end{mathpar}
 

where

\begin{eqnarray}
  (x)\id{\{} \lpquote Q \rpquote / \lpquote P \rpquote \id{\}}            = 
  \left\{ 
    \begin{array}{ccc}
      \lpquote Q \rpquote & & x \nameeq \lpquote P \rpquote \\
      x & & otherwise \\
    \end{array}
  \right. \nonumber
\end{eqnarray}

and $z$ is chosen distinct from $\quotep{P}$, $\quotep{Q}$, the free
names in $Q$, and all the names in $R$. Our $\alpha$-equivalence will
be built in the standard way from this substitution.

\begin{remark}\label{rem:no_self_referential_names}
  One consequence of these definitions is that $\forall P. \quotep{P}
  \not\in \freenames{P}$.
\end{remark}

\subsection{ Dynamic quote: an example }

Anticipating something of what's to come, consider applying the
substitution, $\widehat{\id{\{}u / z \id{\}}}$, to the following pair
of processes, $\lift{w}{y!(z)}$ and $w[ \lpquote y!(z) \rpquote ]$.

\begin{eqnarray}
	\lift{w}{y!(z)}\widehat{\id{\{}u / z \id{\}}}
		& = &
		\lift{w}{y!(u)} \nonumber\\
	w[ \lpquote y!(z) \rpquote ] \widehat{ \id{\{}u / z \id{\}} }
		& = &
		w[ \lpquote y!(z) \rpquote ] \nonumber
\end{eqnarray}

Because the body of the process between quotes is impervious to
substitution, we get radically different answers. In fact, by
examining the first process in an input context,
e.g. $x?(z).\lift{w}{y!(z)}$, we see that the process under the lift
operator may be shaped by prefixed inputs binding a name inside it. In
this sense, the lift operator will be seen as a way to dynamically
construct processes before reifying them as names.

Finally equipped with these standard features we can present the
dynamics of the calculus.

\subsubsection{Operational semantics} 

Finally, we introduce the computational dynamics. What marks these
algebras as distinct from other more traditionally studied algebraic
structures, e.g. vector spaces or polynomial rings, is the manner in
which dynamics is captured. In traditional structures, dynamics is typically
expressed through morphisms between such structures, as in linear maps
between vector spaces or morphisms between rings. In algebras
associated with the semantics of computation, the dynamics is
expressed as part of the algebraic structure itself, through a
reduction reduction relation typically denoted by $\red$. Below, we
give a recursive presentation of this relation for the calculus used
in the encoding.

$\red \subseteq \pi \times \pi$
$\red : \pi \to \mathcal{P}(\pi)$

\begin{mathpar}
  \inferrule* [lab=Comm] { \textsf{match}( x_{src}, x_{trgt} ) } { x_{trgt}?(y)P \; | \; x_{src}!\langle {Q} \rangle \red P\{\quotep{Q}/y}\} }
  \and \\
  \inferrule* [lab=Par] {{P} \red {P}'} {{{P} | {Q}} \red {{P}' | {Q}}}
  \and
  \inferrule* [lab=Equiv]{{{P} \scong {P}'} \andalso {{P}' \red {Q}'} \andalso {{Q}' \scong {Q}}}{{P} \red {Q}}
\end{mathpar}

\begin{eqnarray*}
  match_{\equiv} (\quotep{P},\quotep{Q}) & := & P \equiv Q \\
  match_{\dagger}(\quotep{P},\quotep{Q}) & := & \forall R. P|Q \red^{*} R => R \red^{*} 0 \\
  match_{K}(\quotep{P},\quotep{Q}) & := & K \mbox{ for some context } K
\end{eqnarray*}

$u?(x)P | u!\langle Q \rangle \red P\{\quotep{Q}/x\}$

%We write $\wred$ for $\red^*$, and $P\red$ if $\exists Q $ such that $ P \red Q$.
We write $P\red$ if $\exists Q $ such that $ P \red Q$ and $P\not\red$, otherwise.

\section{Replication}

As mentioned before, it is known that replication (and hence
recursion) can be implemented in a higher-order process algebra
\cite{SangiorgiWalker}. As our first example of calculation with the
machinery thus far presented we give the construction explicitly in
the {\rhoc}.

\begin{eqnarray}
	D_{x} & := & \prefix{x}{y}{(\binpar{\outputp{x}{y}}{@{y}})} \nonumber\\
	\bangp_{x}{P} & := & \binpar{{x}!\langle{\binpar{D_{x}}{P}}\rangle}{D_{x}} \nonumber
\end{eqnarray}

\begin{eqnarray}
	\bangp_{x}{P} & & \nonumber\\
	=
	& {x}!\langle{(\prefix{x}{y}{(\outputp{x}{y} | @{y})) | P}}\rangle 
	      | \prefix{x}{y}{(\outputp{x}{y} | @{y})} & \nonumber\\
	\red
	& (\outputp{x}{y} | @{y})\substn{\quotep{(\prefix{x}{y}{(@{y} | \outputp{x}{y})) | P}}}{y} & \nonumber\\
	=
	& \outputp{x}{\quotep{(\prefix{x}{y}{(\outputp{x}{y} | @{y})) | P}}}
	  | {(\prefix{x}{y}{(\outputp{x}{y} | @{y})) | P}} & \nonumber\\
	\red
	& \ldots & \nonumber\\
	\red^*
	& P | P | \ldots & \nonumber
\end{eqnarray}

Of course, this encoding, as an implementation, runs away, unfolding
$\bangp{P}$ eagerly. A lazier and more implementable replication
operator, restricted to input-guarded processes, may be obtained as follows.

\begin{eqnarray}
\bangp{\prefix{u}{v}{P}} 
	:= 
	\binpar{\lift{x}{\prefix{u}{v}{(\binpar{D(x)}{P})}}}{D(x)} \nonumber
\end{eqnarray}

\begin{remark}
  Note that the lazier definition still does not deal with summation
  or mixed summation (i.e. sums over input and output). The reader is
  invited to construct definitions of replication that deal with these
  features. 

  Further, the definitions are parameterized in a name, $x$. Can you,
  gentle reader, make a definition that eliminates this parameter and
  guarantees no accidental interaction between the replication
  machinery and the process being replicated -- i.e. no accidental
  sharing of names used by the process to get its work done and the
  name(s) used by the replication to effect copying. This latter
  revision of the definition of replication is crucial to obtaining
  the expected identity $!!P \sim !P$.
\end{remark}

\begin{remark}\label{rem:paradoxical_combinator}
  The reader familiar with the lambda calculus will have noticed the
  similarity between $D$ and the paradoxical combinator.

  [Ed. note: the existence of this seems to suggest we have to be more
  restrictive on the set of processes and names we admit if we are to
  support no-cloning.]
\end{remark}

\subsubsection{Bisimulation}

The computational dynamics gives rise to another kind of equivalence,
the equivalence of computational behavior. As previously mentioned
this is typically captured \emph{via} some form of bisimulation.

% The notion we use in this paper is weak barbed bisimulation
% \cite{milner91polyadicpi}.

The notion we use in this paper is derived from weak barbed
bisimulation \cite{milner91polyadicpi}. 

\begin{definition}
An \emph{observation relation}, $\downarrow_{\mathcal N}$, over a set
of names, $\mathcal N$, is the smallest relation satisfying the rules
below.

\infrule[Out-barb]{y \in {\mathcal N}, \; x \nameeq y}
		  {\outputp{x}{v} \downarrow_{\mathcal N} x}
\infrule[Par-barb]{\mbox{$P\downarrow_{\mathcal N} x$ or $Q\downarrow_{\mathcal N} x$}}
		  {\binpar{P}{Q} \downarrow_{\mathcal N} x}

We write $P \Downarrow_{\mathcal N} x$ if there is $Q$ such that 
$P \wred Q$ and $Q \downarrow_{\mathcal N} x$.
\end{definition}

\begin{definition}
%\label{def.bbisim}
An  ${\mathcal N}$-\emph{barbed bisimulation} over a set of names, ${\mathcal N}$, is a symmetric binary relation 
${\mathcal S}_{\mathcal N}$ between agents such that $P\rel{S}_{\mathcal N}Q$ implies:
\begin{enumerate}
\item If $P \red P'$ then $Q \wred Q'$ and $P'\rel{S}_{\mathcal N} Q'$.
\item If $P\downarrow_{\mathcal N} x$, then $Q\Downarrow_{\mathcal N} x$.
\end{enumerate}
$P$ is ${\mathcal N}$-barbed bisimilar to $Q$, written
$P \wbbisim_{\mathcal N} Q$, if $P \rel{S}_{\mathcal N} Q$ for some ${\mathcal N}$-barbed bisimulation ${\mathcal S}_{\mathcal N}$.
\end{definition}

$\mathcal{R} \subseteq \pi \times \pi$

$P \mathcal{R} Q => \forall P'. P \red P' \Rightarrow \exists Q'. Q \red Q', P' \mathcal{R} Q'$

$P \vdash x \Rightarrow Q \vdash x$

\begin{mathpar}
  \inferrule*[lab=Out-barb]{x \nameeq y}{{y}!\langle{Q}\rangle \vdash x}
  \and
  \inferrule*[lab=Par-barb]{\mbox{$P\vdash x$ or $Q\vdash x$}}{\binpar{P}{Q} \vdash x}
\end{mathpar}

\subsubsection{Contexts}

One of the principle advantages of computational calculi like the
$\pi$-calculus is a well-defined notion of context,
contextual-equivalence and a correlation between
contextual-equivalence and notions of bisimulation. The notion of
context allows the decomposition of a process into (sub-)process and
its syntactic environment, its context. Thus, a context may be
thought of as a process with a ``hole'' (written $\Box$) in it. The
application of a context $M$ to a process $P$, written $M[P]$, is
tantamount to filling the hole in $M$ with $P$. In this paper we do
not need the full weight of this theory, but do make use of the notion
of context in the proof the main theorem. 

\begin{mathpar}
  \inferrule* [lab=summation] {} {{M_{M},M_{N}} \bc \Box \;|\; x.M_{A} \;|\; M_{M}+M_{N}}
  \and
  \inferrule* [lab=agent] {} {{M_{A}} \bc (\vec{x})M_{P} \;| \; \clift{P_0,\ldots,M_{P},\ldots,P_N}}
  \and \\
  \inferrule* [lab=process] {} {{M_{P}} \bc M_{N} \;| \;P|M_{P} }
\end{mathpar} 

\begin{mathpar}
  \inferrule* [lab=sychronization] {} {M_{N} \bc \Box \;|\; x?M_{F} \;|\; x!M_{C}}
  \and
  \inferrule* [lab=abstraction] {} {{M_{F}} \bc (x)M_{P} }
  \and
  \inferrule* [lab=concretion] {} {{M_{C}} \bc \langle M_{P} \rangle }
  \and \\
  \inferrule* [lab=process] {} {{M_{P}} \bc M_{N} \;| \;P|M_{P} }
\end{mathpar}

\begin{definition}[contextual application] Given a context $M$, and
  process $P$, we define the \emph{contextual application}, $M[P] :=
  M\{P/\Box\}$. That is, the contextual application of M to P is the
  substitution of $P$ for $\Box$ in $M$.
\end{definition}

$\meaningof{-} : L \to \mathcal{P}(\pi)$

\begin{mathpar}
  \inferrule* [lab=collection] {} {\meaningof{true} = \pi, \and \meaningof{~E} = \pi \setminus \meaningof{E}, \and \meaningof{E_{1} \& E_{2}} = \meaningof{E_{1}} \cap \meaningof{E_{2}}}
\end{mathpar}

\begin{mathpar}
  \inferrule* [lab=structure] {} {\meaningof{0} = \{ P \in \pi | P \equiv 0 \}, \and \\ \meaningof{E_1 | E_2} = \{ P \in \pi | P \equiv P_{1} | P_{2}, P_{1} \in \meaningof{E_{1}}, P_{2} \in \meaningof{E_2}\} }
\end{mathpar}

\begin{mathpar}
 \inferrule* [lab=behavior] {} {\meaningof{\langle a?b \rangle E} = \{ P \in \pi | P \equiv Q | u?(y)P', \\ \and \\\\ \and \\ \;\;\; u \in \meaningof{a}, \forall z.P'\{z/y\} \in \meaningof{E\{z/b\}}\}, \and \\ \meaningof{a!E} = \{ P \in \pi | P \equiv Q | x!\langle P' \rangle, x \in \meaningof{a} P' \in \meaningof{E}\} }
\end{mathpar}

\begin{mathpar}
 \inferrule* [lab=nominal] {} {\meaningof{\quotep{E}} = \{ \quotep{P} \in \quotep{\pi} | P \in \meaningof{E} \}, \and \meaningof{\quotep{P}} = \{ \quotep{Q} \in \quotep{\pi} | P \equiv Q \} \and \\ \meaningof{@\quotep{E}} = \{ P \in \pi | P \equiv @x, x \in \meaningof{E} \}}
\end{mathpar}

\begin{eqnarray*}
  \\
  \meaningof{-} : TS \to ST
\end{eqnarray*}

\begin{eqnarray*}
  \\
  L : TS \to ST
\end{eqnarray*}

\begin{eqnarray*}
  \\
  P \models E \iff P \in \meaningof{E}
\end{eqnarray*}

\begin{eqnarray*}
  P \approx_{L} Q \iff \forall E \in L. P \models E \iff Q \models E
\end{eqnarray*}

\begin{eqnarray*}
  P \approx_{K} Q
\end{eqnarray*}

\begin{eqnarray*}
  P \approx Q
\end{eqnarray*}

$\approx_{K} = \approx = \approx_{L}$

\subsubsection{Contextual duality}

Note that contexts extend the quotation operation to a family of
operations from processes to names. Given a context, $M$, we can
define a \emph{nominal context}, $\quotep{M}$ by $\quotep{M}[P] :=
\quotep{M[P]}$. To foreshadow what is to come we observe that these
operations enjoy a duality with processes very much like the duality
between vectors and maps from vectors to scalars.

Further, because the calculus is essentially higher-order, we have a
correspondence between contexts and processes. More specifically,
given a name $x$ and a context $M$ we can construct $M^{*}_{x}$ such
that 

\begin{mathpar}
  M^{*}_{x} | \lift{x}{P} \red M[P]
\end{mathpar}

namely,

\begin{mathpar}
  M^{*}_{x} := x?(u).M[\dropn{u}]
\end{mathpar}

The dependence of $M^{*}_{x}$ on a name makes it an abstraction, 

\begin{mathpar}
  M^{*} := (x)x?(u).M[\dropn{u}]
\end{mathpar}

\subsection{Additional notation}

It will sometimes be convenient to denote the process a name
quotes. We already have the notation $x = \quotep{P}$, but it will be
convenient to introduce an alternate notation, $\procn{x}$, when we
want to emphasize the connection to the use of the name. Note that, by
virtue of name equivalence, $\quotep{\procn{x}} \nameeq x$; so, the
notation is consistent with previous definitions.

Further, because names have structure it is possible to effect
substitutions on the basis of that structure. This means we need to
upgrade our notation for substitutions, which we accomplish by
adapting comprehension notation. Thus,

\begin{mathpar}
  P\{ y / x : x \in S \}
\end{mathpar}

is interpreted to mean the process derived from P by replacing (in a
capture-avoiding manner) each occurrence of $x$ in $S$ by $y$. For example,

\begin{mathpar}
  P\{ \quotep{\procn{x}|\procn{x}} / x : x \in \freenames{P} \}
\end{mathpar}

will replace each (occurrence) of a free name $x$ in $P$ by
$\quotep{\procn{x}|\procn{x}}$.

Also, we will avail ourselves of the notation $x^{L}$ and $x^{R}$ to
denote injections of a name into disjoint copies of the name
space. There are numerous ways to accomplish this. One example can be
found in \cite{MeredithR05}. This notation overloads to vectors of
names: $\vec{x}^{\pi} := (x_{i}^{\pi} \; : \; 0 \leq i < |\vec{x}| )$ where $\pi \in \{L,R\}$.

We also use $P^{\Box} := P|\Box$.

In \cite{MeredithR05} an interpretation of the new operator is
given. It turns out that there are several possible interpretations
all enjoying the requisite algebraic properties of the operator (see
\cite{milner91polyadicpi}). We will therefore make liberal use of
$(\nu\; \vec{x})P$.

% subsection the_syntax_and_semantics_of_the_notation_system (end)   

\input{qm2pi.qmops} 

\input{qm2pi.sterngerlach} 

\input{qm2pi.metric} 

% section concurrent_process_calculi (end)

%\input{qm2pi.proofsketch}

% section proof sketch (end)

%\input{qm2pi.slviaknots} 

% section spatial logic via knots (end)

\input{qm2pi.conclusion}

% section conclusion (end)

%\input{qm2pi.dtcodes} 

% section wiring algorithm (end)

\input{qm2pi.ack} 

% section acknowledgments (end)

\newpage


\bibliographystyle{plain}   
\bibliography{../../biblios/main.bib}

\input{qm2pi.rhodetails}

\end{document}

 

%\ifpdf
%\usepackage[pdftex]{graphicx}
%\else
%\usepackage{graphicx}
%\fi

 % \ifpdf
%  \usepackage{pdfsync}
%  \if


%\title{Brief Article}
%\author{David F. Snyder}
%\author{L.G. Meredith}

%\address{Dept. of Math., Texas State University--San Marcos, San Marcos, TX 78666}
       
\pagestyle{empty}


\begin{document}

\lstset{language=[Objective]Caml,frame=shadowbox}

\documentclass[12pt]{llncs}
%\documentclass{jktr}

\usepackage[pdftex]{hyperref}                   
\usepackage {listings}
\usepackage {mathpartir}
\usepackage{bcprules}
%\usepackage{listings}
                       
\usepackage{graphicx} 
%\usepackage[margins=2.5cm,nohead,nofoot]{geometry}
%\usepackage{geometry}
\usepackage{amsfonts}
\usepackage{amstext}
\usepackage{latexsym}
\usepackage{amssymb}
\usepackage{color}


%\include{myPreamble}
\include{qm2pi.local} 

%\ifpdf
%\usepackage[pdftex]{graphicx}
%\else
%\usepackage{graphicx}
%\fi

 % \ifpdf
%  \usepackage{pdfsync}
%  \if


%\title{Brief Article}
%\author{David F. Snyder}
%\author{L.G. Meredith}

%\address{Dept. of Math., Texas State University--San Marcos, San Marcos, TX 78666}
       
\pagestyle{empty}


\begin{document}

\lstset{language=[Objective]Caml,frame=shadowbox}

\input{qm2pi.front}

% section front matter (end)

\input{qm2pi.intro} 
 
% section introduction (end)

% \input{qm2pi.knotations} 

% section notation (end)

\input{qm2pi.process.calculi} 

% section concurrent_process_calculi_and_spatial_logics_ (end)
    
%\input{qm2pi.knots2pi} 

%\input{qm2pi.trefoil} 

%\input{qm2pi.mainthm} 

% subsection basic_interpretation (end)

%\input{qm2pi.rho.presentation} 
\subsection{The syntax and semantics of the notation system}\label{sub:the_syntax_and_semantics_of_the_notation_system} % (fold)

We now summarize a technical presentation of the calculus that
embodies our theory of dynamics. The typical presentation of such a
calculus follows the style of giving generators and relations on
them. The grammar, below, describing term constructors, freely
generates the set of processes, $\Proc$. This set is then quotiented
by a relation known as structural congruence and it is over this set
that the notion of dynamics is expressed. This presentation is
essentially that of \cite{MeredithR05} with the addition of
polyadicity and summation. For readability we have relegated some of
the technical subtleties to an appendix.

\subsubsection{Process grammar}\label{subsub:process_grammar}

\begin{mathpar}
  \inferrule* [lab=synchronization] {} {{M} \bc \pzero \;|\; x?F \;|\; x!C }
  \and
  \inferrule* [lab=abstraction] {} {{F} \bc (x)P}
  \and
  \inferrule* [lab=concretion] {} {{C} \bc \langle Q \rangle}
  \and
  \inferrule* [lab=process] {} {{P,Q} \bc M \;| \;P|Q \;|\; @{x}}
  \and
  \inferrule* [lab=name] {} {{x} \bc \quotep{P}}
\end{mathpar} 

Note that $\vec{x}$ (resp. $\vec{P}$) denotes a vector of names
(resp. processes) of length $|\vec{x}|$ (resp. $|\vec{P}|$). We adopt
the following useful abbreviations.

\begin{mathpar}
   x?(\vec{y}).P := x.(\vec{y})P \and  x\clift{\vec{P}} := x.\clift{\vec{P}}
   \and x!(y) := \lift{x}{\dropn{y}}
   \and \Pi_{i=0}^{n-1}P_i := P_0 | \ldots | P_{n-1}
\end{mathpar}

\subsubsection{Structural congruence}

\paragraph{Free and bound names and alpha-equivalence.} At the
core of structural equivalence is alpha-equivalence which identifies
process that are the same up to a change of variable. Formally, we
recognize the distinction between free and bound names. The free names
of a process, $\freenames{P}$, may be calculated recursively as
follows:

\begin{mathpar}
\freenames{\pzero} := \emptyset
  \and \\
  \freenames{x?(y).P} := \{ x \} \cup (\freenames{P} \setminus \{ y \})
  \and 
  \freenames{x!\langle P \rangle} := \{ x \} \cup \{ P \} 
  \and \\
  \freenames{P|Q} := \freenames{P} \cup \freenames{Q}
  \and \\
  \freenames{@{x}} := \{ x \}
\end{mathpar}

$\pi$
$\quotep{\pi}$

$\freenames{-} : \pi \to \mathcal{P}(\quotep{\pi})$

\begin{eqnarray*}
  \freenames{\pzero} & := & \emptyset \\
  \freenames{x?(y).P} & := & \{ x \} \cup (\freenames{P} \setminus \{ y \}) \\
  \freenames{x!\langle P \rangle} & := & \{ x \} \cup \{ P \} \\
  \freenames{P|Q} & := & \freenames{P} \cup \freenames{Q} \\
  \freenames{\dropn{x}} & := & \{ x \}
\end{eqnarray*}

The bound names of a process, $\boundnames{P}$, are those names occurring in $P$
that are not free. For example, in $x?(y).0$, the name $x$ is free, while $y$ is bound.

\begin{mathpar}
  \inferrule* [lab=monoidal-laws] {} { P|Q \equiv Q|P \and P|0 \equiv P \and P|(Q|R) \equiv (P|Q)|R }
\end{mathpar}

\begin{mathpar}
  \inferrule* [lab=alpha-equivalence] {} { (x)P \equiv (y)P\{y/x\} \and y \not\in \freenames{P} }
\end{mathpar}

\begin{definition}
Then two processes, $P,Q$, are alpha-equivalent if $P = Q\{\vec{y}/\vec{x}\}$ for
some $\vec{x} \in \boundnames{Q},\vec{y} \in \boundnames{P}$, where $Q\{\vec{y}/\vec{x}\}$
denotes the capture-avoiding substitution of $\vec{y}$ for $\vec{x}$ in $Q$.
\end{definition}

\begin{definition}
  The {\em structural congruence} \cite{SangiorgiWalker} , $\equiv$,
  between processes is the least congruence containing
  alpha-equivalence, satisfying the abelian monoid laws
  (associativity, commutativity and $\pzero$ as identity) for parallel
  composition $|$ and for summation $+$.
\end{definition}

\subsection{Name equivalence}

We take name equivalence, written $\nameeq$, to be the smallest
equivalence relation generated by the following rules.

\begin{mathpar}
\inferrule*[lab=Quote-drop]
{ }
{ \quotep{@{x}} \nameeq x }

\inferrule*[lab=Struct-equiv]
{ P \scong Q }
{ \quotep{P} \nameeq \quotep{Q} }
\end{mathpar}

The astute reader will have noticed that the mutual recursion of names
and processes imposes a mutual recursion on alpha-equivalence and
structural equivalence via name-equivalence. Fortunately, all of this
works out pleasantly and we may calculate in the natural way, free of
concern. The reader interested in the details is referred to the
appendix \ref{appendix:rho_details}.

\subsection{Substitution}

We use $\Proc$ for the set of processes, $\QProc$ for the set of
names, and $\id{\{}\vec{y} / \vec{x} \id{\}}$ to denote partial maps,
$s : \QProc \rightarrow \QProc$. A map, $s$ lifts, uniquely, to a map
on process terms, $\widehat{s} : \Proc \rightarrow \Proc$ by the
following equations.

\begin{mathpar}
  (0) \psubstp{Q}{P} := 0 \\
  (R \juxtap S) \psubstp{Q}{P}
  :=    
  (R)\psubstp{Q}{P} \juxtap (S) \psubstp{Q}{P} \\
  (x?(y).R) \psubstp{Q}{P}    
  :=    
  (x)\substp{Q}{P} (z)\concat( (R \psubstn{z}{y}) \psubstp{Q}{P} ) \\
  (\lift{x}{R}) \psubstp{Q}{P}  
  :=
  \lift{(x)\substp{Q}{P}}{ R \psubstp{Q}{P} } \\
%   (\dropn{x})  \psubstp{Q}{P}       
%   := 
%   \left\{ 
%     \begin{array}{ccc} 
%       \dropn{\quotep{Q}} & & x \nameeq \quotep{P} \\
%       \dropn{x} & & otherwise \\
%     \end{array}
%   \right. 
  (\dropn{x})  \psubstp{Q}{P}       
  := 
  \left\{ 
    \begin{array}{ccc} 
      Q & & x \nameeq \quotep{P} \\
      \dropn{x} & & otherwise \\
    \end{array}
  \right.
\end{mathpar}
 

where

\begin{eqnarray}
  (x)\id{\{} \lpquote Q \rpquote / \lpquote P \rpquote \id{\}}            = 
  \left\{ 
    \begin{array}{ccc}
      \lpquote Q \rpquote & & x \nameeq \lpquote P \rpquote \\
      x & & otherwise \\
    \end{array}
  \right. \nonumber
\end{eqnarray}

and $z$ is chosen distinct from $\quotep{P}$, $\quotep{Q}$, the free
names in $Q$, and all the names in $R$. Our $\alpha$-equivalence will
be built in the standard way from this substitution.

\begin{remark}\label{rem:no_self_referential_names}
  One consequence of these definitions is that $\forall P. \quotep{P}
  \not\in \freenames{P}$.
\end{remark}

\subsection{ Dynamic quote: an example }

Anticipating something of what's to come, consider applying the
substitution, $\widehat{\id{\{}u / z \id{\}}}$, to the following pair
of processes, $\lift{w}{y!(z)}$ and $w[ \lpquote y!(z) \rpquote ]$.

\begin{eqnarray}
	\lift{w}{y!(z)}\widehat{\id{\{}u / z \id{\}}}
		& = &
		\lift{w}{y!(u)} \nonumber\\
	w[ \lpquote y!(z) \rpquote ] \widehat{ \id{\{}u / z \id{\}} }
		& = &
		w[ \lpquote y!(z) \rpquote ] \nonumber
\end{eqnarray}

Because the body of the process between quotes is impervious to
substitution, we get radically different answers. In fact, by
examining the first process in an input context,
e.g. $x?(z).\lift{w}{y!(z)}$, we see that the process under the lift
operator may be shaped by prefixed inputs binding a name inside it. In
this sense, the lift operator will be seen as a way to dynamically
construct processes before reifying them as names.

Finally equipped with these standard features we can present the
dynamics of the calculus.

\subsubsection{Operational semantics} 

Finally, we introduce the computational dynamics. What marks these
algebras as distinct from other more traditionally studied algebraic
structures, e.g. vector spaces or polynomial rings, is the manner in
which dynamics is captured. In traditional structures, dynamics is typically
expressed through morphisms between such structures, as in linear maps
between vector spaces or morphisms between rings. In algebras
associated with the semantics of computation, the dynamics is
expressed as part of the algebraic structure itself, through a
reduction reduction relation typically denoted by $\red$. Below, we
give a recursive presentation of this relation for the calculus used
in the encoding.

$\red \subseteq \pi \times \pi$
$\red : \pi \to \mathcal{P}(\pi)$

\begin{mathpar}
  \inferrule* [lab=Comm] { \textsf{match}( x_{src}, x_{trgt} ) } { x_{trgt}?(y)P \; | \; x_{src}!\langle {Q} \rangle \red P\{\quotep{Q}/y}\} }
  \and \\
  \inferrule* [lab=Par] {{P} \red {P}'} {{{P} | {Q}} \red {{P}' | {Q}}}
  \and
  \inferrule* [lab=Equiv]{{{P} \scong {P}'} \andalso {{P}' \red {Q}'} \andalso {{Q}' \scong {Q}}}{{P} \red {Q}}
\end{mathpar}

\begin{eqnarray*}
  match_{\equiv} (\quotep{P},\quotep{Q}) & := & P \equiv Q \\
  match_{\dagger}(\quotep{P},\quotep{Q}) & := & \forall R. P|Q \red^{*} R => R \red^{*} 0 \\
  match_{K}(\quotep{P},\quotep{Q}) & := & K \mbox{ for some context } K
\end{eqnarray*}

$u?(x)P | u!\langle Q \rangle \red P\{\quotep{Q}/x\}$

%We write $\wred$ for $\red^*$, and $P\red$ if $\exists Q $ such that $ P \red Q$.
We write $P\red$ if $\exists Q $ such that $ P \red Q$ and $P\not\red$, otherwise.

\section{Replication}

As mentioned before, it is known that replication (and hence
recursion) can be implemented in a higher-order process algebra
\cite{SangiorgiWalker}. As our first example of calculation with the
machinery thus far presented we give the construction explicitly in
the {\rhoc}.

\begin{eqnarray}
	D_{x} & := & \prefix{x}{y}{(\binpar{\outputp{x}{y}}{@{y}})} \nonumber\\
	\bangp_{x}{P} & := & \binpar{{x}!\langle{\binpar{D_{x}}{P}}\rangle}{D_{x}} \nonumber
\end{eqnarray}

\begin{eqnarray}
	\bangp_{x}{P} & & \nonumber\\
	=
	& {x}!\langle{(\prefix{x}{y}{(\outputp{x}{y} | @{y})) | P}}\rangle 
	      | \prefix{x}{y}{(\outputp{x}{y} | @{y})} & \nonumber\\
	\red
	& (\outputp{x}{y} | @{y})\substn{\quotep{(\prefix{x}{y}{(@{y} | \outputp{x}{y})) | P}}}{y} & \nonumber\\
	=
	& \outputp{x}{\quotep{(\prefix{x}{y}{(\outputp{x}{y} | @{y})) | P}}}
	  | {(\prefix{x}{y}{(\outputp{x}{y} | @{y})) | P}} & \nonumber\\
	\red
	& \ldots & \nonumber\\
	\red^*
	& P | P | \ldots & \nonumber
\end{eqnarray}

Of course, this encoding, as an implementation, runs away, unfolding
$\bangp{P}$ eagerly. A lazier and more implementable replication
operator, restricted to input-guarded processes, may be obtained as follows.

\begin{eqnarray}
\bangp{\prefix{u}{v}{P}} 
	:= 
	\binpar{\lift{x}{\prefix{u}{v}{(\binpar{D(x)}{P})}}}{D(x)} \nonumber
\end{eqnarray}

\begin{remark}
  Note that the lazier definition still does not deal with summation
  or mixed summation (i.e. sums over input and output). The reader is
  invited to construct definitions of replication that deal with these
  features. 

  Further, the definitions are parameterized in a name, $x$. Can you,
  gentle reader, make a definition that eliminates this parameter and
  guarantees no accidental interaction between the replication
  machinery and the process being replicated -- i.e. no accidental
  sharing of names used by the process to get its work done and the
  name(s) used by the replication to effect copying. This latter
  revision of the definition of replication is crucial to obtaining
  the expected identity $!!P \sim !P$.
\end{remark}

\begin{remark}\label{rem:paradoxical_combinator}
  The reader familiar with the lambda calculus will have noticed the
  similarity between $D$ and the paradoxical combinator.

  [Ed. note: the existence of this seems to suggest we have to be more
  restrictive on the set of processes and names we admit if we are to
  support no-cloning.]
\end{remark}

\subsubsection{Bisimulation}

The computational dynamics gives rise to another kind of equivalence,
the equivalence of computational behavior. As previously mentioned
this is typically captured \emph{via} some form of bisimulation.

% The notion we use in this paper is weak barbed bisimulation
% \cite{milner91polyadicpi}.

The notion we use in this paper is derived from weak barbed
bisimulation \cite{milner91polyadicpi}. 

\begin{definition}
An \emph{observation relation}, $\downarrow_{\mathcal N}$, over a set
of names, $\mathcal N$, is the smallest relation satisfying the rules
below.

\infrule[Out-barb]{y \in {\mathcal N}, \; x \nameeq y}
		  {\outputp{x}{v} \downarrow_{\mathcal N} x}
\infrule[Par-barb]{\mbox{$P\downarrow_{\mathcal N} x$ or $Q\downarrow_{\mathcal N} x$}}
		  {\binpar{P}{Q} \downarrow_{\mathcal N} x}

We write $P \Downarrow_{\mathcal N} x$ if there is $Q$ such that 
$P \wred Q$ and $Q \downarrow_{\mathcal N} x$.
\end{definition}

\begin{definition}
%\label{def.bbisim}
An  ${\mathcal N}$-\emph{barbed bisimulation} over a set of names, ${\mathcal N}$, is a symmetric binary relation 
${\mathcal S}_{\mathcal N}$ between agents such that $P\rel{S}_{\mathcal N}Q$ implies:
\begin{enumerate}
\item If $P \red P'$ then $Q \wred Q'$ and $P'\rel{S}_{\mathcal N} Q'$.
\item If $P\downarrow_{\mathcal N} x$, then $Q\Downarrow_{\mathcal N} x$.
\end{enumerate}
$P$ is ${\mathcal N}$-barbed bisimilar to $Q$, written
$P \wbbisim_{\mathcal N} Q$, if $P \rel{S}_{\mathcal N} Q$ for some ${\mathcal N}$-barbed bisimulation ${\mathcal S}_{\mathcal N}$.
\end{definition}

$\mathcal{R} \subseteq \pi \times \pi$

$P \mathcal{R} Q => \forall P'. P \red P' \Rightarrow \exists Q'. Q \red Q', P' \mathcal{R} Q'$

$P \vdash x \Rightarrow Q \vdash x$

\begin{mathpar}
  \inferrule*[lab=Out-barb]{x \nameeq y}{{y}!\langle{Q}\rangle \vdash x}
  \and
  \inferrule*[lab=Par-barb]{\mbox{$P\vdash x$ or $Q\vdash x$}}{\binpar{P}{Q} \vdash x}
\end{mathpar}

\subsubsection{Contexts}

One of the principle advantages of computational calculi like the
$\pi$-calculus is a well-defined notion of context,
contextual-equivalence and a correlation between
contextual-equivalence and notions of bisimulation. The notion of
context allows the decomposition of a process into (sub-)process and
its syntactic environment, its context. Thus, a context may be
thought of as a process with a ``hole'' (written $\Box$) in it. The
application of a context $M$ to a process $P$, written $M[P]$, is
tantamount to filling the hole in $M$ with $P$. In this paper we do
not need the full weight of this theory, but do make use of the notion
of context in the proof the main theorem. 

\begin{mathpar}
  \inferrule* [lab=summation] {} {{M_{M},M_{N}} \bc \Box \;|\; x.M_{A} \;|\; M_{M}+M_{N}}
  \and
  \inferrule* [lab=agent] {} {{M_{A}} \bc (\vec{x})M_{P} \;| \; \clift{P_0,\ldots,M_{P},\ldots,P_N}}
  \and \\
  \inferrule* [lab=process] {} {{M_{P}} \bc M_{N} \;| \;P|M_{P} }
\end{mathpar} 

\begin{mathpar}
  \inferrule* [lab=sychronization] {} {M_{N} \bc \Box \;|\; x?M_{F} \;|\; x!M_{C}}
  \and
  \inferrule* [lab=abstraction] {} {{M_{F}} \bc (x)M_{P} }
  \and
  \inferrule* [lab=concretion] {} {{M_{C}} \bc \langle M_{P} \rangle }
  \and \\
  \inferrule* [lab=process] {} {{M_{P}} \bc M_{N} \;| \;P|M_{P} }
\end{mathpar}

\begin{definition}[contextual application] Given a context $M$, and
  process $P$, we define the \emph{contextual application}, $M[P] :=
  M\{P/\Box\}$. That is, the contextual application of M to P is the
  substitution of $P$ for $\Box$ in $M$.
\end{definition}

$\meaningof{-} : L \to \mathcal{P}(\pi)$

\begin{mathpar}
  \inferrule* [lab=collection] {} {\meaningof{true} = \pi, \and \meaningof{~E} = \pi \setminus \meaningof{E}, \and \meaningof{E_{1} \& E_{2}} = \meaningof{E_{1}} \cap \meaningof{E_{2}}}
\end{mathpar}

\begin{mathpar}
  \inferrule* [lab=structure] {} {\meaningof{0} = \{ P \in \pi | P \equiv 0 \}, \and \\ \meaningof{E_1 | E_2} = \{ P \in \pi | P \equiv P_{1} | P_{2}, P_{1} \in \meaningof{E_{1}}, P_{2} \in \meaningof{E_2}\} }
\end{mathpar}

\begin{mathpar}
 \inferrule* [lab=behavior] {} {\meaningof{\langle a?b \rangle E} = \{ P \in \pi | P \equiv Q | u?(y)P', \\ \and \\\\ \and \\ \;\;\; u \in \meaningof{a}, \forall z.P'\{z/y\} \in \meaningof{E\{z/b\}}\}, \and \\ \meaningof{a!E} = \{ P \in \pi | P \equiv Q | x!\langle P' \rangle, x \in \meaningof{a} P' \in \meaningof{E}\} }
\end{mathpar}

\begin{mathpar}
 \inferrule* [lab=nominal] {} {\meaningof{\quotep{E}} = \{ \quotep{P} \in \quotep{\pi} | P \in \meaningof{E} \}, \and \meaningof{\quotep{P}} = \{ \quotep{Q} \in \quotep{\pi} | P \equiv Q \} \and \\ \meaningof{@\quotep{E}} = \{ P \in \pi | P \equiv @x, x \in \meaningof{E} \}}
\end{mathpar}

\begin{eqnarray*}
  \\
  \meaningof{-} : TS \to ST
\end{eqnarray*}

\begin{eqnarray*}
  \\
  L : TS \to ST
\end{eqnarray*}

\begin{eqnarray*}
  \\
  P \models E \iff P \in \meaningof{E}
\end{eqnarray*}

\begin{eqnarray*}
  P \approx_{L} Q \iff \forall E \in L. P \models E \iff Q \models E
\end{eqnarray*}

\begin{eqnarray*}
  P \approx_{K} Q
\end{eqnarray*}

\begin{eqnarray*}
  P \approx Q
\end{eqnarray*}

$\approx_{K} = \approx = \approx_{L}$

\subsubsection{Contextual duality}

Note that contexts extend the quotation operation to a family of
operations from processes to names. Given a context, $M$, we can
define a \emph{nominal context}, $\quotep{M}$ by $\quotep{M}[P] :=
\quotep{M[P]}$. To foreshadow what is to come we observe that these
operations enjoy a duality with processes very much like the duality
between vectors and maps from vectors to scalars.

Further, because the calculus is essentially higher-order, we have a
correspondence between contexts and processes. More specifically,
given a name $x$ and a context $M$ we can construct $M^{*}_{x}$ such
that 

\begin{mathpar}
  M^{*}_{x} | \lift{x}{P} \red M[P]
\end{mathpar}

namely,

\begin{mathpar}
  M^{*}_{x} := x?(u).M[\dropn{u}]
\end{mathpar}

The dependence of $M^{*}_{x}$ on a name makes it an abstraction, 

\begin{mathpar}
  M^{*} := (x)x?(u).M[\dropn{u}]
\end{mathpar}

\subsection{Additional notation}

It will sometimes be convenient to denote the process a name
quotes. We already have the notation $x = \quotep{P}$, but it will be
convenient to introduce an alternate notation, $\procn{x}$, when we
want to emphasize the connection to the use of the name. Note that, by
virtue of name equivalence, $\quotep{\procn{x}} \nameeq x$; so, the
notation is consistent with previous definitions.

Further, because names have structure it is possible to effect
substitutions on the basis of that structure. This means we need to
upgrade our notation for substitutions, which we accomplish by
adapting comprehension notation. Thus,

\begin{mathpar}
  P\{ y / x : x \in S \}
\end{mathpar}

is interpreted to mean the process derived from P by replacing (in a
capture-avoiding manner) each occurrence of $x$ in $S$ by $y$. For example,

\begin{mathpar}
  P\{ \quotep{\procn{x}|\procn{x}} / x : x \in \freenames{P} \}
\end{mathpar}

will replace each (occurrence) of a free name $x$ in $P$ by
$\quotep{\procn{x}|\procn{x}}$.

Also, we will avail ourselves of the notation $x^{L}$ and $x^{R}$ to
denote injections of a name into disjoint copies of the name
space. There are numerous ways to accomplish this. One example can be
found in \cite{MeredithR05}. This notation overloads to vectors of
names: $\vec{x}^{\pi} := (x_{i}^{\pi} \; : \; 0 \leq i < |\vec{x}| )$ where $\pi \in \{L,R\}$.

We also use $P^{\Box} := P|\Box$.

In \cite{MeredithR05} an interpretation of the new operator is
given. It turns out that there are several possible interpretations
all enjoying the requisite algebraic properties of the operator (see
\cite{milner91polyadicpi}). We will therefore make liberal use of
$(\nu\; \vec{x})P$.

% subsection the_syntax_and_semantics_of_the_notation_system (end)   

\input{qm2pi.qmops} 

\input{qm2pi.sterngerlach} 

\input{qm2pi.metric} 

% section concurrent_process_calculi (end)

%\input{qm2pi.proofsketch}

% section proof sketch (end)

%\input{qm2pi.slviaknots} 

% section spatial logic via knots (end)

\input{qm2pi.conclusion}

% section conclusion (end)

%\input{qm2pi.dtcodes} 

% section wiring algorithm (end)

\input{qm2pi.ack} 

% section acknowledgments (end)

\newpage


\bibliographystyle{plain}   
\bibliography{../../biblios/main.bib}

\input{qm2pi.rhodetails}

\end{document}



% section front matter (end)

\section{Introduction}\label{sec:introduction} % (fold)
In this draft of the material i am going to have to dispense with the
usual writing conventions adopted in papers on these topics. i'm going
to have adopt whatever tone i need at the time i'm writing up the
calculations. Sometimes this may be very conversational; others it may
be the barest mathematical grunts; others still it may be that i have
lifted text from one of my other papers because the exposition of some
point was better said there. i hope that my readers are not unduly put
out by this decision. i'm not doing this to flout convention or be
rebellious. i find these calculations very technically challenging. To
keep everything going technically, something has to give; i have to
let go of some cognitive burden. So, the academic writing style --
with all of its trade-offs in terms of facilitating technical
communication -- is what i'm letting go of. Perhaps subsequent drafts
can be tightened and polished, but for now, i'm going to speak as if
we were sitting together in a coffee shop with a laptop, wifi and a
pad of paper and a pencil.

So, here's what i have to say. We -- you and i, comfortably ensconced
in our coffee shop and well-equipped with our tools -- can realize and
carry out the calculations of quantum mechanics over a very different
formal theory of dynamics, a formal theory of dynamics that
corresponds to a theory of concurrent computation with
\emph{reflection}. It has the advantage that the underlying theory is
already `quantized', but supports analogues all of the continuuous
operations. Strikingly, this underlying theory has recently been
connected with a notion of metric that we can show, by calculating
together, coincides with the metric induced by the inner product.

There are a lot of reasons why you might be interested in seeing
calculations of this form. Here's why i'm interested. For the past
several centuries there has been no competitor to the ``Newtonian''
account of dynamics. As a result the predominant share of accounts of
dynamical systems and situations have had to be formulated in terms of
the Newtonian machinery. i view this as an intellectually dangerous
position to occupy. Everything, despite it's intrinsic shape, turns
into a nail to be hit with this hammer. Recently, however, the theory
of computation has matured to the point where we have candidates for
theories of dynamics that offer very different perspective on
reasoning about dynamical systems and situations. Testing these
candidates against very successful accounts of dynamical situations,
like quantum mechanics, is going to give us some sense of how mature
they are and some measure of the quality of these accounts of
dynamics.

\subsection{Summary of contributions and outline of paper}

So, we're going to develop an interpretation of the operations of
quantum mechanics normally interpreted by Hilbert spaces and
operators. We're going to do this over a theory of computation. Note
that this is very different than the usual quantum computation program
which develops notions of computation over quantum mechanics. Rather,
we are developing a story that aligns with Wheeler's slogan: It from
Bit. To do this we will first provide an account of the theory of
computation at play here. Then we will dive into a calculation-driven
interpretation of the operations of quantum mechanics.

The reason we take this approach is that -- until very recently --
there hasn't been an axiomatic account of quantum mechanics. As a
result there has been no sharp delineation of the mathematical theory
supporting interpretation of the physical theory and the physical
theory, itself. So, ambient features of the maths are free to be
exploited (or supressed) without a real accounting of their physical
relevance. There is no sharp statement ``here's the physical theory''
qua \emph{theory} and ``here's the mathematical interpretation''
enabling a judgment of how faithful the interpretation is -- apart
from experimental observation. When there is an axiomatic account we
can judge how well a given mathematical formalism supports an
interpretation of the axioms, independent of
experimentation. Likewise, we can judge how well we have captured our
physical evidence and experience with our axiomatics, independent of
any specific mathematical implementation, with accidental detail that
may or may not have physical significance. 

In lieu of a fully fleshed out and vetted axiomatic account of quantum
mechanics, interpreting the operational notions in service of modeling
physical systems will have to suffice. In other words, we are not in
the business of providing a model of Hilbert spaces and operators. We
are in the business of providing a model of quantum mechanics because
we are motivated by testing our notions of dynamics against physical
theory; and, the predictive calculations of the physical theory must
serve as the best formulation -- shy of a fully fleshed out axiomatic
account -- of the physical theory itself (as they have for scientific
theories since time immemorial). Put another way, despite a
whole-hearted commitment to an It-from-Bit ontology, we are firmly
aligned with the shut-up-and-calculate camp as the best way to obtain
results either from the physical perspective or as a quality assurance
measure of our fledgling theory of dynamics.

In detail, we present a reflective process calculus. Then we develop
intuitive correspondences between the notions available in this
calculus and the usual physical notions supporting quantum mechanical
calculations. Thus, 

\begin{table}[htp]
  \center{
    \fbox{
      \begin{tabular}{c|c}
        quantum mechanics & process calculus \\
        \hline
        scalar & name \\
        state vector & process \\
        dual & contextual duals \\
        matrix & formal sums of process-context-dual pairs \\
        orthogonality & process annihilation \\
        inner product & execution-formula + quoting
      \end{tabular}
    }
  }
  \caption{QM - process calculi correspondences}
\end{table}

Then we tighten up these intuitions to operational definitions. We
employ the Dirac notation as the best proxy we can find for an
abstract syntax of the quantum mechanical notions. The definitions we
develop put us in contact with equational constraints coming from the
theory that we demonstrate the definitions and calculations satisfy.

This puts us in a position to shut up and calculate for the
Stern-Gerlach experimental set up, showing how these predictive
calculations become calculations on processes in our theory of a
reflective process calculus.

Penultimately, we demonstrate that the notion of metric coming from
the inner product coincides with the notion of metric available from
the theory of bisimulation. This demonstration gives us the right to
think of space as arising from behavior. Finally, we consider where we
might go from the new vantage point we have obtained.

% section introduction (end) 
 
% section introduction (end)

% \documentclass[12pt]{llncs}
%\documentclass{jktr}

\usepackage[pdftex]{hyperref}                   
\usepackage {listings}
\usepackage {mathpartir}
\usepackage{bcprules}
%\usepackage{listings}
                       
\usepackage{graphicx} 
%\usepackage[margins=2.5cm,nohead,nofoot]{geometry}
%\usepackage{geometry}
\usepackage{amsfonts}
\usepackage{amstext}
\usepackage{latexsym}
\usepackage{amssymb}
\usepackage{color}


%\include{myPreamble}
\include{qm2pi.local} 

%\ifpdf
%\usepackage[pdftex]{graphicx}
%\else
%\usepackage{graphicx}
%\fi

 % \ifpdf
%  \usepackage{pdfsync}
%  \if


%\title{Brief Article}
%\author{David F. Snyder}
%\author{L.G. Meredith}

%\address{Dept. of Math., Texas State University--San Marcos, San Marcos, TX 78666}
       
\pagestyle{empty}


\begin{document}

\lstset{language=[Objective]Caml,frame=shadowbox}

\input{qm2pi.front}

% section front matter (end)

\input{qm2pi.intro} 
 
% section introduction (end)

% \input{qm2pi.knotations} 

% section notation (end)

\input{qm2pi.process.calculi} 

% section concurrent_process_calculi_and_spatial_logics_ (end)
    
%\input{qm2pi.knots2pi} 

%\input{qm2pi.trefoil} 

%\input{qm2pi.mainthm} 

% subsection basic_interpretation (end)

%\input{qm2pi.rho.presentation} 
\subsection{The syntax and semantics of the notation system}\label{sub:the_syntax_and_semantics_of_the_notation_system} % (fold)

We now summarize a technical presentation of the calculus that
embodies our theory of dynamics. The typical presentation of such a
calculus follows the style of giving generators and relations on
them. The grammar, below, describing term constructors, freely
generates the set of processes, $\Proc$. This set is then quotiented
by a relation known as structural congruence and it is over this set
that the notion of dynamics is expressed. This presentation is
essentially that of \cite{MeredithR05} with the addition of
polyadicity and summation. For readability we have relegated some of
the technical subtleties to an appendix.

\subsubsection{Process grammar}\label{subsub:process_grammar}

\begin{mathpar}
  \inferrule* [lab=synchronization] {} {{M} \bc \pzero \;|\; x?F \;|\; x!C }
  \and
  \inferrule* [lab=abstraction] {} {{F} \bc (x)P}
  \and
  \inferrule* [lab=concretion] {} {{C} \bc \langle Q \rangle}
  \and
  \inferrule* [lab=process] {} {{P,Q} \bc M \;| \;P|Q \;|\; @{x}}
  \and
  \inferrule* [lab=name] {} {{x} \bc \quotep{P}}
\end{mathpar} 

Note that $\vec{x}$ (resp. $\vec{P}$) denotes a vector of names
(resp. processes) of length $|\vec{x}|$ (resp. $|\vec{P}|$). We adopt
the following useful abbreviations.

\begin{mathpar}
   x?(\vec{y}).P := x.(\vec{y})P \and  x\clift{\vec{P}} := x.\clift{\vec{P}}
   \and x!(y) := \lift{x}{\dropn{y}}
   \and \Pi_{i=0}^{n-1}P_i := P_0 | \ldots | P_{n-1}
\end{mathpar}

\subsubsection{Structural congruence}

\paragraph{Free and bound names and alpha-equivalence.} At the
core of structural equivalence is alpha-equivalence which identifies
process that are the same up to a change of variable. Formally, we
recognize the distinction between free and bound names. The free names
of a process, $\freenames{P}$, may be calculated recursively as
follows:

\begin{mathpar}
\freenames{\pzero} := \emptyset
  \and \\
  \freenames{x?(y).P} := \{ x \} \cup (\freenames{P} \setminus \{ y \})
  \and 
  \freenames{x!\langle P \rangle} := \{ x \} \cup \{ P \} 
  \and \\
  \freenames{P|Q} := \freenames{P} \cup \freenames{Q}
  \and \\
  \freenames{@{x}} := \{ x \}
\end{mathpar}

$\pi$
$\quotep{\pi}$

$\freenames{-} : \pi \to \mathcal{P}(\quotep{\pi})$

\begin{eqnarray*}
  \freenames{\pzero} & := & \emptyset \\
  \freenames{x?(y).P} & := & \{ x \} \cup (\freenames{P} \setminus \{ y \}) \\
  \freenames{x!\langle P \rangle} & := & \{ x \} \cup \{ P \} \\
  \freenames{P|Q} & := & \freenames{P} \cup \freenames{Q} \\
  \freenames{\dropn{x}} & := & \{ x \}
\end{eqnarray*}

The bound names of a process, $\boundnames{P}$, are those names occurring in $P$
that are not free. For example, in $x?(y).0$, the name $x$ is free, while $y$ is bound.

\begin{mathpar}
  \inferrule* [lab=monoidal-laws] {} { P|Q \equiv Q|P \and P|0 \equiv P \and P|(Q|R) \equiv (P|Q)|R }
\end{mathpar}

\begin{mathpar}
  \inferrule* [lab=alpha-equivalence] {} { (x)P \equiv (y)P\{y/x\} \and y \not\in \freenames{P} }
\end{mathpar}

\begin{definition}
Then two processes, $P,Q$, are alpha-equivalent if $P = Q\{\vec{y}/\vec{x}\}$ for
some $\vec{x} \in \boundnames{Q},\vec{y} \in \boundnames{P}$, where $Q\{\vec{y}/\vec{x}\}$
denotes the capture-avoiding substitution of $\vec{y}$ for $\vec{x}$ in $Q$.
\end{definition}

\begin{definition}
  The {\em structural congruence} \cite{SangiorgiWalker} , $\equiv$,
  between processes is the least congruence containing
  alpha-equivalence, satisfying the abelian monoid laws
  (associativity, commutativity and $\pzero$ as identity) for parallel
  composition $|$ and for summation $+$.
\end{definition}

\subsection{Name equivalence}

We take name equivalence, written $\nameeq$, to be the smallest
equivalence relation generated by the following rules.

\begin{mathpar}
\inferrule*[lab=Quote-drop]
{ }
{ \quotep{@{x}} \nameeq x }

\inferrule*[lab=Struct-equiv]
{ P \scong Q }
{ \quotep{P} \nameeq \quotep{Q} }
\end{mathpar}

The astute reader will have noticed that the mutual recursion of names
and processes imposes a mutual recursion on alpha-equivalence and
structural equivalence via name-equivalence. Fortunately, all of this
works out pleasantly and we may calculate in the natural way, free of
concern. The reader interested in the details is referred to the
appendix \ref{appendix:rho_details}.

\subsection{Substitution}

We use $\Proc$ for the set of processes, $\QProc$ for the set of
names, and $\id{\{}\vec{y} / \vec{x} \id{\}}$ to denote partial maps,
$s : \QProc \rightarrow \QProc$. A map, $s$ lifts, uniquely, to a map
on process terms, $\widehat{s} : \Proc \rightarrow \Proc$ by the
following equations.

\begin{mathpar}
  (0) \psubstp{Q}{P} := 0 \\
  (R \juxtap S) \psubstp{Q}{P}
  :=    
  (R)\psubstp{Q}{P} \juxtap (S) \psubstp{Q}{P} \\
  (x?(y).R) \psubstp{Q}{P}    
  :=    
  (x)\substp{Q}{P} (z)\concat( (R \psubstn{z}{y}) \psubstp{Q}{P} ) \\
  (\lift{x}{R}) \psubstp{Q}{P}  
  :=
  \lift{(x)\substp{Q}{P}}{ R \psubstp{Q}{P} } \\
%   (\dropn{x})  \psubstp{Q}{P}       
%   := 
%   \left\{ 
%     \begin{array}{ccc} 
%       \dropn{\quotep{Q}} & & x \nameeq \quotep{P} \\
%       \dropn{x} & & otherwise \\
%     \end{array}
%   \right. 
  (\dropn{x})  \psubstp{Q}{P}       
  := 
  \left\{ 
    \begin{array}{ccc} 
      Q & & x \nameeq \quotep{P} \\
      \dropn{x} & & otherwise \\
    \end{array}
  \right.
\end{mathpar}
 

where

\begin{eqnarray}
  (x)\id{\{} \lpquote Q \rpquote / \lpquote P \rpquote \id{\}}            = 
  \left\{ 
    \begin{array}{ccc}
      \lpquote Q \rpquote & & x \nameeq \lpquote P \rpquote \\
      x & & otherwise \\
    \end{array}
  \right. \nonumber
\end{eqnarray}

and $z$ is chosen distinct from $\quotep{P}$, $\quotep{Q}$, the free
names in $Q$, and all the names in $R$. Our $\alpha$-equivalence will
be built in the standard way from this substitution.

\begin{remark}\label{rem:no_self_referential_names}
  One consequence of these definitions is that $\forall P. \quotep{P}
  \not\in \freenames{P}$.
\end{remark}

\subsection{ Dynamic quote: an example }

Anticipating something of what's to come, consider applying the
substitution, $\widehat{\id{\{}u / z \id{\}}}$, to the following pair
of processes, $\lift{w}{y!(z)}$ and $w[ \lpquote y!(z) \rpquote ]$.

\begin{eqnarray}
	\lift{w}{y!(z)}\widehat{\id{\{}u / z \id{\}}}
		& = &
		\lift{w}{y!(u)} \nonumber\\
	w[ \lpquote y!(z) \rpquote ] \widehat{ \id{\{}u / z \id{\}} }
		& = &
		w[ \lpquote y!(z) \rpquote ] \nonumber
\end{eqnarray}

Because the body of the process between quotes is impervious to
substitution, we get radically different answers. In fact, by
examining the first process in an input context,
e.g. $x?(z).\lift{w}{y!(z)}$, we see that the process under the lift
operator may be shaped by prefixed inputs binding a name inside it. In
this sense, the lift operator will be seen as a way to dynamically
construct processes before reifying them as names.

Finally equipped with these standard features we can present the
dynamics of the calculus.

\subsubsection{Operational semantics} 

Finally, we introduce the computational dynamics. What marks these
algebras as distinct from other more traditionally studied algebraic
structures, e.g. vector spaces or polynomial rings, is the manner in
which dynamics is captured. In traditional structures, dynamics is typically
expressed through morphisms between such structures, as in linear maps
between vector spaces or morphisms between rings. In algebras
associated with the semantics of computation, the dynamics is
expressed as part of the algebraic structure itself, through a
reduction reduction relation typically denoted by $\red$. Below, we
give a recursive presentation of this relation for the calculus used
in the encoding.

$\red \subseteq \pi \times \pi$
$\red : \pi \to \mathcal{P}(\pi)$

\begin{mathpar}
  \inferrule* [lab=Comm] { \textsf{match}( x_{src}, x_{trgt} ) } { x_{trgt}?(y)P \; | \; x_{src}!\langle {Q} \rangle \red P\{\quotep{Q}/y}\} }
  \and \\
  \inferrule* [lab=Par] {{P} \red {P}'} {{{P} | {Q}} \red {{P}' | {Q}}}
  \and
  \inferrule* [lab=Equiv]{{{P} \scong {P}'} \andalso {{P}' \red {Q}'} \andalso {{Q}' \scong {Q}}}{{P} \red {Q}}
\end{mathpar}

\begin{eqnarray*}
  match_{\equiv} (\quotep{P},\quotep{Q}) & := & P \equiv Q \\
  match_{\dagger}(\quotep{P},\quotep{Q}) & := & \forall R. P|Q \red^{*} R => R \red^{*} 0 \\
  match_{K}(\quotep{P},\quotep{Q}) & := & K \mbox{ for some context } K
\end{eqnarray*}

$u?(x)P | u!\langle Q \rangle \red P\{\quotep{Q}/x\}$

%We write $\wred$ for $\red^*$, and $P\red$ if $\exists Q $ such that $ P \red Q$.
We write $P\red$ if $\exists Q $ such that $ P \red Q$ and $P\not\red$, otherwise.

\section{Replication}

As mentioned before, it is known that replication (and hence
recursion) can be implemented in a higher-order process algebra
\cite{SangiorgiWalker}. As our first example of calculation with the
machinery thus far presented we give the construction explicitly in
the {\rhoc}.

\begin{eqnarray}
	D_{x} & := & \prefix{x}{y}{(\binpar{\outputp{x}{y}}{@{y}})} \nonumber\\
	\bangp_{x}{P} & := & \binpar{{x}!\langle{\binpar{D_{x}}{P}}\rangle}{D_{x}} \nonumber
\end{eqnarray}

\begin{eqnarray}
	\bangp_{x}{P} & & \nonumber\\
	=
	& {x}!\langle{(\prefix{x}{y}{(\outputp{x}{y} | @{y})) | P}}\rangle 
	      | \prefix{x}{y}{(\outputp{x}{y} | @{y})} & \nonumber\\
	\red
	& (\outputp{x}{y} | @{y})\substn{\quotep{(\prefix{x}{y}{(@{y} | \outputp{x}{y})) | P}}}{y} & \nonumber\\
	=
	& \outputp{x}{\quotep{(\prefix{x}{y}{(\outputp{x}{y} | @{y})) | P}}}
	  | {(\prefix{x}{y}{(\outputp{x}{y} | @{y})) | P}} & \nonumber\\
	\red
	& \ldots & \nonumber\\
	\red^*
	& P | P | \ldots & \nonumber
\end{eqnarray}

Of course, this encoding, as an implementation, runs away, unfolding
$\bangp{P}$ eagerly. A lazier and more implementable replication
operator, restricted to input-guarded processes, may be obtained as follows.

\begin{eqnarray}
\bangp{\prefix{u}{v}{P}} 
	:= 
	\binpar{\lift{x}{\prefix{u}{v}{(\binpar{D(x)}{P})}}}{D(x)} \nonumber
\end{eqnarray}

\begin{remark}
  Note that the lazier definition still does not deal with summation
  or mixed summation (i.e. sums over input and output). The reader is
  invited to construct definitions of replication that deal with these
  features. 

  Further, the definitions are parameterized in a name, $x$. Can you,
  gentle reader, make a definition that eliminates this parameter and
  guarantees no accidental interaction between the replication
  machinery and the process being replicated -- i.e. no accidental
  sharing of names used by the process to get its work done and the
  name(s) used by the replication to effect copying. This latter
  revision of the definition of replication is crucial to obtaining
  the expected identity $!!P \sim !P$.
\end{remark}

\begin{remark}\label{rem:paradoxical_combinator}
  The reader familiar with the lambda calculus will have noticed the
  similarity between $D$ and the paradoxical combinator.

  [Ed. note: the existence of this seems to suggest we have to be more
  restrictive on the set of processes and names we admit if we are to
  support no-cloning.]
\end{remark}

\subsubsection{Bisimulation}

The computational dynamics gives rise to another kind of equivalence,
the equivalence of computational behavior. As previously mentioned
this is typically captured \emph{via} some form of bisimulation.

% The notion we use in this paper is weak barbed bisimulation
% \cite{milner91polyadicpi}.

The notion we use in this paper is derived from weak barbed
bisimulation \cite{milner91polyadicpi}. 

\begin{definition}
An \emph{observation relation}, $\downarrow_{\mathcal N}$, over a set
of names, $\mathcal N$, is the smallest relation satisfying the rules
below.

\infrule[Out-barb]{y \in {\mathcal N}, \; x \nameeq y}
		  {\outputp{x}{v} \downarrow_{\mathcal N} x}
\infrule[Par-barb]{\mbox{$P\downarrow_{\mathcal N} x$ or $Q\downarrow_{\mathcal N} x$}}
		  {\binpar{P}{Q} \downarrow_{\mathcal N} x}

We write $P \Downarrow_{\mathcal N} x$ if there is $Q$ such that 
$P \wred Q$ and $Q \downarrow_{\mathcal N} x$.
\end{definition}

\begin{definition}
%\label{def.bbisim}
An  ${\mathcal N}$-\emph{barbed bisimulation} over a set of names, ${\mathcal N}$, is a symmetric binary relation 
${\mathcal S}_{\mathcal N}$ between agents such that $P\rel{S}_{\mathcal N}Q$ implies:
\begin{enumerate}
\item If $P \red P'$ then $Q \wred Q'$ and $P'\rel{S}_{\mathcal N} Q'$.
\item If $P\downarrow_{\mathcal N} x$, then $Q\Downarrow_{\mathcal N} x$.
\end{enumerate}
$P$ is ${\mathcal N}$-barbed bisimilar to $Q$, written
$P \wbbisim_{\mathcal N} Q$, if $P \rel{S}_{\mathcal N} Q$ for some ${\mathcal N}$-barbed bisimulation ${\mathcal S}_{\mathcal N}$.
\end{definition}

$\mathcal{R} \subseteq \pi \times \pi$

$P \mathcal{R} Q => \forall P'. P \red P' \Rightarrow \exists Q'. Q \red Q', P' \mathcal{R} Q'$

$P \vdash x \Rightarrow Q \vdash x$

\begin{mathpar}
  \inferrule*[lab=Out-barb]{x \nameeq y}{{y}!\langle{Q}\rangle \vdash x}
  \and
  \inferrule*[lab=Par-barb]{\mbox{$P\vdash x$ or $Q\vdash x$}}{\binpar{P}{Q} \vdash x}
\end{mathpar}

\subsubsection{Contexts}

One of the principle advantages of computational calculi like the
$\pi$-calculus is a well-defined notion of context,
contextual-equivalence and a correlation between
contextual-equivalence and notions of bisimulation. The notion of
context allows the decomposition of a process into (sub-)process and
its syntactic environment, its context. Thus, a context may be
thought of as a process with a ``hole'' (written $\Box$) in it. The
application of a context $M$ to a process $P$, written $M[P]$, is
tantamount to filling the hole in $M$ with $P$. In this paper we do
not need the full weight of this theory, but do make use of the notion
of context in the proof the main theorem. 

\begin{mathpar}
  \inferrule* [lab=summation] {} {{M_{M},M_{N}} \bc \Box \;|\; x.M_{A} \;|\; M_{M}+M_{N}}
  \and
  \inferrule* [lab=agent] {} {{M_{A}} \bc (\vec{x})M_{P} \;| \; \clift{P_0,\ldots,M_{P},\ldots,P_N}}
  \and \\
  \inferrule* [lab=process] {} {{M_{P}} \bc M_{N} \;| \;P|M_{P} }
\end{mathpar} 

\begin{mathpar}
  \inferrule* [lab=sychronization] {} {M_{N} \bc \Box \;|\; x?M_{F} \;|\; x!M_{C}}
  \and
  \inferrule* [lab=abstraction] {} {{M_{F}} \bc (x)M_{P} }
  \and
  \inferrule* [lab=concretion] {} {{M_{C}} \bc \langle M_{P} \rangle }
  \and \\
  \inferrule* [lab=process] {} {{M_{P}} \bc M_{N} \;| \;P|M_{P} }
\end{mathpar}

\begin{definition}[contextual application] Given a context $M$, and
  process $P$, we define the \emph{contextual application}, $M[P] :=
  M\{P/\Box\}$. That is, the contextual application of M to P is the
  substitution of $P$ for $\Box$ in $M$.
\end{definition}

$\meaningof{-} : L \to \mathcal{P}(\pi)$

\begin{mathpar}
  \inferrule* [lab=collection] {} {\meaningof{true} = \pi, \and \meaningof{~E} = \pi \setminus \meaningof{E}, \and \meaningof{E_{1} \& E_{2}} = \meaningof{E_{1}} \cap \meaningof{E_{2}}}
\end{mathpar}

\begin{mathpar}
  \inferrule* [lab=structure] {} {\meaningof{0} = \{ P \in \pi | P \equiv 0 \}, \and \\ \meaningof{E_1 | E_2} = \{ P \in \pi | P \equiv P_{1} | P_{2}, P_{1} \in \meaningof{E_{1}}, P_{2} \in \meaningof{E_2}\} }
\end{mathpar}

\begin{mathpar}
 \inferrule* [lab=behavior] {} {\meaningof{\langle a?b \rangle E} = \{ P \in \pi | P \equiv Q | u?(y)P', \\ \and \\\\ \and \\ \;\;\; u \in \meaningof{a}, \forall z.P'\{z/y\} \in \meaningof{E\{z/b\}}\}, \and \\ \meaningof{a!E} = \{ P \in \pi | P \equiv Q | x!\langle P' \rangle, x \in \meaningof{a} P' \in \meaningof{E}\} }
\end{mathpar}

\begin{mathpar}
 \inferrule* [lab=nominal] {} {\meaningof{\quotep{E}} = \{ \quotep{P} \in \quotep{\pi} | P \in \meaningof{E} \}, \and \meaningof{\quotep{P}} = \{ \quotep{Q} \in \quotep{\pi} | P \equiv Q \} \and \\ \meaningof{@\quotep{E}} = \{ P \in \pi | P \equiv @x, x \in \meaningof{E} \}}
\end{mathpar}

\begin{eqnarray*}
  \\
  \meaningof{-} : TS \to ST
\end{eqnarray*}

\begin{eqnarray*}
  \\
  L : TS \to ST
\end{eqnarray*}

\begin{eqnarray*}
  \\
  P \models E \iff P \in \meaningof{E}
\end{eqnarray*}

\begin{eqnarray*}
  P \approx_{L} Q \iff \forall E \in L. P \models E \iff Q \models E
\end{eqnarray*}

\begin{eqnarray*}
  P \approx_{K} Q
\end{eqnarray*}

\begin{eqnarray*}
  P \approx Q
\end{eqnarray*}

$\approx_{K} = \approx = \approx_{L}$

\subsubsection{Contextual duality}

Note that contexts extend the quotation operation to a family of
operations from processes to names. Given a context, $M$, we can
define a \emph{nominal context}, $\quotep{M}$ by $\quotep{M}[P] :=
\quotep{M[P]}$. To foreshadow what is to come we observe that these
operations enjoy a duality with processes very much like the duality
between vectors and maps from vectors to scalars.

Further, because the calculus is essentially higher-order, we have a
correspondence between contexts and processes. More specifically,
given a name $x$ and a context $M$ we can construct $M^{*}_{x}$ such
that 

\begin{mathpar}
  M^{*}_{x} | \lift{x}{P} \red M[P]
\end{mathpar}

namely,

\begin{mathpar}
  M^{*}_{x} := x?(u).M[\dropn{u}]
\end{mathpar}

The dependence of $M^{*}_{x}$ on a name makes it an abstraction, 

\begin{mathpar}
  M^{*} := (x)x?(u).M[\dropn{u}]
\end{mathpar}

\subsection{Additional notation}

It will sometimes be convenient to denote the process a name
quotes. We already have the notation $x = \quotep{P}$, but it will be
convenient to introduce an alternate notation, $\procn{x}$, when we
want to emphasize the connection to the use of the name. Note that, by
virtue of name equivalence, $\quotep{\procn{x}} \nameeq x$; so, the
notation is consistent with previous definitions.

Further, because names have structure it is possible to effect
substitutions on the basis of that structure. This means we need to
upgrade our notation for substitutions, which we accomplish by
adapting comprehension notation. Thus,

\begin{mathpar}
  P\{ y / x : x \in S \}
\end{mathpar}

is interpreted to mean the process derived from P by replacing (in a
capture-avoiding manner) each occurrence of $x$ in $S$ by $y$. For example,

\begin{mathpar}
  P\{ \quotep{\procn{x}|\procn{x}} / x : x \in \freenames{P} \}
\end{mathpar}

will replace each (occurrence) of a free name $x$ in $P$ by
$\quotep{\procn{x}|\procn{x}}$.

Also, we will avail ourselves of the notation $x^{L}$ and $x^{R}$ to
denote injections of a name into disjoint copies of the name
space. There are numerous ways to accomplish this. One example can be
found in \cite{MeredithR05}. This notation overloads to vectors of
names: $\vec{x}^{\pi} := (x_{i}^{\pi} \; : \; 0 \leq i < |\vec{x}| )$ where $\pi \in \{L,R\}$.

We also use $P^{\Box} := P|\Box$.

In \cite{MeredithR05} an interpretation of the new operator is
given. It turns out that there are several possible interpretations
all enjoying the requisite algebraic properties of the operator (see
\cite{milner91polyadicpi}). We will therefore make liberal use of
$(\nu\; \vec{x})P$.

% subsection the_syntax_and_semantics_of_the_notation_system (end)   

\input{qm2pi.qmops} 

\input{qm2pi.sterngerlach} 

\input{qm2pi.metric} 

% section concurrent_process_calculi (end)

%\input{qm2pi.proofsketch}

% section proof sketch (end)

%\input{qm2pi.slviaknots} 

% section spatial logic via knots (end)

\input{qm2pi.conclusion}

% section conclusion (end)

%\input{qm2pi.dtcodes} 

% section wiring algorithm (end)

\input{qm2pi.ack} 

% section acknowledgments (end)

\newpage


\bibliographystyle{plain}   
\bibliography{../../biblios/main.bib}

\input{qm2pi.rhodetails}

\end{document}

 

% section notation (end)

\input{qm2pi.process.calculi} 

% section concurrent_process_calculi_and_spatial_logics_ (end)
    
%\documentclass[12pt]{llncs}
%\documentclass{jktr}

\usepackage[pdftex]{hyperref}                   
\usepackage {listings}
\usepackage {mathpartir}
\usepackage{bcprules}
%\usepackage{listings}
                       
\usepackage{graphicx} 
%\usepackage[margins=2.5cm,nohead,nofoot]{geometry}
%\usepackage{geometry}
\usepackage{amsfonts}
\usepackage{amstext}
\usepackage{latexsym}
\usepackage{amssymb}
\usepackage{color}


%\include{myPreamble}
\include{qm2pi.local} 

%\ifpdf
%\usepackage[pdftex]{graphicx}
%\else
%\usepackage{graphicx}
%\fi

 % \ifpdf
%  \usepackage{pdfsync}
%  \if


%\title{Brief Article}
%\author{David F. Snyder}
%\author{L.G. Meredith}

%\address{Dept. of Math., Texas State University--San Marcos, San Marcos, TX 78666}
       
\pagestyle{empty}


\begin{document}

\lstset{language=[Objective]Caml,frame=shadowbox}

\input{qm2pi.front}

% section front matter (end)

\input{qm2pi.intro} 
 
% section introduction (end)

% \input{qm2pi.knotations} 

% section notation (end)

\input{qm2pi.process.calculi} 

% section concurrent_process_calculi_and_spatial_logics_ (end)
    
%\input{qm2pi.knots2pi} 

%\input{qm2pi.trefoil} 

%\input{qm2pi.mainthm} 

% subsection basic_interpretation (end)

%\input{qm2pi.rho.presentation} 
\subsection{The syntax and semantics of the notation system}\label{sub:the_syntax_and_semantics_of_the_notation_system} % (fold)

We now summarize a technical presentation of the calculus that
embodies our theory of dynamics. The typical presentation of such a
calculus follows the style of giving generators and relations on
them. The grammar, below, describing term constructors, freely
generates the set of processes, $\Proc$. This set is then quotiented
by a relation known as structural congruence and it is over this set
that the notion of dynamics is expressed. This presentation is
essentially that of \cite{MeredithR05} with the addition of
polyadicity and summation. For readability we have relegated some of
the technical subtleties to an appendix.

\subsubsection{Process grammar}\label{subsub:process_grammar}

\begin{mathpar}
  \inferrule* [lab=synchronization] {} {{M} \bc \pzero \;|\; x?F \;|\; x!C }
  \and
  \inferrule* [lab=abstraction] {} {{F} \bc (x)P}
  \and
  \inferrule* [lab=concretion] {} {{C} \bc \langle Q \rangle}
  \and
  \inferrule* [lab=process] {} {{P,Q} \bc M \;| \;P|Q \;|\; @{x}}
  \and
  \inferrule* [lab=name] {} {{x} \bc \quotep{P}}
\end{mathpar} 

Note that $\vec{x}$ (resp. $\vec{P}$) denotes a vector of names
(resp. processes) of length $|\vec{x}|$ (resp. $|\vec{P}|$). We adopt
the following useful abbreviations.

\begin{mathpar}
   x?(\vec{y}).P := x.(\vec{y})P \and  x\clift{\vec{P}} := x.\clift{\vec{P}}
   \and x!(y) := \lift{x}{\dropn{y}}
   \and \Pi_{i=0}^{n-1}P_i := P_0 | \ldots | P_{n-1}
\end{mathpar}

\subsubsection{Structural congruence}

\paragraph{Free and bound names and alpha-equivalence.} At the
core of structural equivalence is alpha-equivalence which identifies
process that are the same up to a change of variable. Formally, we
recognize the distinction between free and bound names. The free names
of a process, $\freenames{P}$, may be calculated recursively as
follows:

\begin{mathpar}
\freenames{\pzero} := \emptyset
  \and \\
  \freenames{x?(y).P} := \{ x \} \cup (\freenames{P} \setminus \{ y \})
  \and 
  \freenames{x!\langle P \rangle} := \{ x \} \cup \{ P \} 
  \and \\
  \freenames{P|Q} := \freenames{P} \cup \freenames{Q}
  \and \\
  \freenames{@{x}} := \{ x \}
\end{mathpar}

$\pi$
$\quotep{\pi}$

$\freenames{-} : \pi \to \mathcal{P}(\quotep{\pi})$

\begin{eqnarray*}
  \freenames{\pzero} & := & \emptyset \\
  \freenames{x?(y).P} & := & \{ x \} \cup (\freenames{P} \setminus \{ y \}) \\
  \freenames{x!\langle P \rangle} & := & \{ x \} \cup \{ P \} \\
  \freenames{P|Q} & := & \freenames{P} \cup \freenames{Q} \\
  \freenames{\dropn{x}} & := & \{ x \}
\end{eqnarray*}

The bound names of a process, $\boundnames{P}$, are those names occurring in $P$
that are not free. For example, in $x?(y).0$, the name $x$ is free, while $y$ is bound.

\begin{mathpar}
  \inferrule* [lab=monoidal-laws] {} { P|Q \equiv Q|P \and P|0 \equiv P \and P|(Q|R) \equiv (P|Q)|R }
\end{mathpar}

\begin{mathpar}
  \inferrule* [lab=alpha-equivalence] {} { (x)P \equiv (y)P\{y/x\} \and y \not\in \freenames{P} }
\end{mathpar}

\begin{definition}
Then two processes, $P,Q$, are alpha-equivalent if $P = Q\{\vec{y}/\vec{x}\}$ for
some $\vec{x} \in \boundnames{Q},\vec{y} \in \boundnames{P}$, where $Q\{\vec{y}/\vec{x}\}$
denotes the capture-avoiding substitution of $\vec{y}$ for $\vec{x}$ in $Q$.
\end{definition}

\begin{definition}
  The {\em structural congruence} \cite{SangiorgiWalker} , $\equiv$,
  between processes is the least congruence containing
  alpha-equivalence, satisfying the abelian monoid laws
  (associativity, commutativity and $\pzero$ as identity) for parallel
  composition $|$ and for summation $+$.
\end{definition}

\subsection{Name equivalence}

We take name equivalence, written $\nameeq$, to be the smallest
equivalence relation generated by the following rules.

\begin{mathpar}
\inferrule*[lab=Quote-drop]
{ }
{ \quotep{@{x}} \nameeq x }

\inferrule*[lab=Struct-equiv]
{ P \scong Q }
{ \quotep{P} \nameeq \quotep{Q} }
\end{mathpar}

The astute reader will have noticed that the mutual recursion of names
and processes imposes a mutual recursion on alpha-equivalence and
structural equivalence via name-equivalence. Fortunately, all of this
works out pleasantly and we may calculate in the natural way, free of
concern. The reader interested in the details is referred to the
appendix \ref{appendix:rho_details}.

\subsection{Substitution}

We use $\Proc$ for the set of processes, $\QProc$ for the set of
names, and $\id{\{}\vec{y} / \vec{x} \id{\}}$ to denote partial maps,
$s : \QProc \rightarrow \QProc$. A map, $s$ lifts, uniquely, to a map
on process terms, $\widehat{s} : \Proc \rightarrow \Proc$ by the
following equations.

\begin{mathpar}
  (0) \psubstp{Q}{P} := 0 \\
  (R \juxtap S) \psubstp{Q}{P}
  :=    
  (R)\psubstp{Q}{P} \juxtap (S) \psubstp{Q}{P} \\
  (x?(y).R) \psubstp{Q}{P}    
  :=    
  (x)\substp{Q}{P} (z)\concat( (R \psubstn{z}{y}) \psubstp{Q}{P} ) \\
  (\lift{x}{R}) \psubstp{Q}{P}  
  :=
  \lift{(x)\substp{Q}{P}}{ R \psubstp{Q}{P} } \\
%   (\dropn{x})  \psubstp{Q}{P}       
%   := 
%   \left\{ 
%     \begin{array}{ccc} 
%       \dropn{\quotep{Q}} & & x \nameeq \quotep{P} \\
%       \dropn{x} & & otherwise \\
%     \end{array}
%   \right. 
  (\dropn{x})  \psubstp{Q}{P}       
  := 
  \left\{ 
    \begin{array}{ccc} 
      Q & & x \nameeq \quotep{P} \\
      \dropn{x} & & otherwise \\
    \end{array}
  \right.
\end{mathpar}
 

where

\begin{eqnarray}
  (x)\id{\{} \lpquote Q \rpquote / \lpquote P \rpquote \id{\}}            = 
  \left\{ 
    \begin{array}{ccc}
      \lpquote Q \rpquote & & x \nameeq \lpquote P \rpquote \\
      x & & otherwise \\
    \end{array}
  \right. \nonumber
\end{eqnarray}

and $z$ is chosen distinct from $\quotep{P}$, $\quotep{Q}$, the free
names in $Q$, and all the names in $R$. Our $\alpha$-equivalence will
be built in the standard way from this substitution.

\begin{remark}\label{rem:no_self_referential_names}
  One consequence of these definitions is that $\forall P. \quotep{P}
  \not\in \freenames{P}$.
\end{remark}

\subsection{ Dynamic quote: an example }

Anticipating something of what's to come, consider applying the
substitution, $\widehat{\id{\{}u / z \id{\}}}$, to the following pair
of processes, $\lift{w}{y!(z)}$ and $w[ \lpquote y!(z) \rpquote ]$.

\begin{eqnarray}
	\lift{w}{y!(z)}\widehat{\id{\{}u / z \id{\}}}
		& = &
		\lift{w}{y!(u)} \nonumber\\
	w[ \lpquote y!(z) \rpquote ] \widehat{ \id{\{}u / z \id{\}} }
		& = &
		w[ \lpquote y!(z) \rpquote ] \nonumber
\end{eqnarray}

Because the body of the process between quotes is impervious to
substitution, we get radically different answers. In fact, by
examining the first process in an input context,
e.g. $x?(z).\lift{w}{y!(z)}$, we see that the process under the lift
operator may be shaped by prefixed inputs binding a name inside it. In
this sense, the lift operator will be seen as a way to dynamically
construct processes before reifying them as names.

Finally equipped with these standard features we can present the
dynamics of the calculus.

\subsubsection{Operational semantics} 

Finally, we introduce the computational dynamics. What marks these
algebras as distinct from other more traditionally studied algebraic
structures, e.g. vector spaces or polynomial rings, is the manner in
which dynamics is captured. In traditional structures, dynamics is typically
expressed through morphisms between such structures, as in linear maps
between vector spaces or morphisms between rings. In algebras
associated with the semantics of computation, the dynamics is
expressed as part of the algebraic structure itself, through a
reduction reduction relation typically denoted by $\red$. Below, we
give a recursive presentation of this relation for the calculus used
in the encoding.

$\red \subseteq \pi \times \pi$
$\red : \pi \to \mathcal{P}(\pi)$

\begin{mathpar}
  \inferrule* [lab=Comm] { \textsf{match}( x_{src}, x_{trgt} ) } { x_{trgt}?(y)P \; | \; x_{src}!\langle {Q} \rangle \red P\{\quotep{Q}/y}\} }
  \and \\
  \inferrule* [lab=Par] {{P} \red {P}'} {{{P} | {Q}} \red {{P}' | {Q}}}
  \and
  \inferrule* [lab=Equiv]{{{P} \scong {P}'} \andalso {{P}' \red {Q}'} \andalso {{Q}' \scong {Q}}}{{P} \red {Q}}
\end{mathpar}

\begin{eqnarray*}
  match_{\equiv} (\quotep{P},\quotep{Q}) & := & P \equiv Q \\
  match_{\dagger}(\quotep{P},\quotep{Q}) & := & \forall R. P|Q \red^{*} R => R \red^{*} 0 \\
  match_{K}(\quotep{P},\quotep{Q}) & := & K \mbox{ for some context } K
\end{eqnarray*}

$u?(x)P | u!\langle Q \rangle \red P\{\quotep{Q}/x\}$

%We write $\wred$ for $\red^*$, and $P\red$ if $\exists Q $ such that $ P \red Q$.
We write $P\red$ if $\exists Q $ such that $ P \red Q$ and $P\not\red$, otherwise.

\section{Replication}

As mentioned before, it is known that replication (and hence
recursion) can be implemented in a higher-order process algebra
\cite{SangiorgiWalker}. As our first example of calculation with the
machinery thus far presented we give the construction explicitly in
the {\rhoc}.

\begin{eqnarray}
	D_{x} & := & \prefix{x}{y}{(\binpar{\outputp{x}{y}}{@{y}})} \nonumber\\
	\bangp_{x}{P} & := & \binpar{{x}!\langle{\binpar{D_{x}}{P}}\rangle}{D_{x}} \nonumber
\end{eqnarray}

\begin{eqnarray}
	\bangp_{x}{P} & & \nonumber\\
	=
	& {x}!\langle{(\prefix{x}{y}{(\outputp{x}{y} | @{y})) | P}}\rangle 
	      | \prefix{x}{y}{(\outputp{x}{y} | @{y})} & \nonumber\\
	\red
	& (\outputp{x}{y} | @{y})\substn{\quotep{(\prefix{x}{y}{(@{y} | \outputp{x}{y})) | P}}}{y} & \nonumber\\
	=
	& \outputp{x}{\quotep{(\prefix{x}{y}{(\outputp{x}{y} | @{y})) | P}}}
	  | {(\prefix{x}{y}{(\outputp{x}{y} | @{y})) | P}} & \nonumber\\
	\red
	& \ldots & \nonumber\\
	\red^*
	& P | P | \ldots & \nonumber
\end{eqnarray}

Of course, this encoding, as an implementation, runs away, unfolding
$\bangp{P}$ eagerly. A lazier and more implementable replication
operator, restricted to input-guarded processes, may be obtained as follows.

\begin{eqnarray}
\bangp{\prefix{u}{v}{P}} 
	:= 
	\binpar{\lift{x}{\prefix{u}{v}{(\binpar{D(x)}{P})}}}{D(x)} \nonumber
\end{eqnarray}

\begin{remark}
  Note that the lazier definition still does not deal with summation
  or mixed summation (i.e. sums over input and output). The reader is
  invited to construct definitions of replication that deal with these
  features. 

  Further, the definitions are parameterized in a name, $x$. Can you,
  gentle reader, make a definition that eliminates this parameter and
  guarantees no accidental interaction between the replication
  machinery and the process being replicated -- i.e. no accidental
  sharing of names used by the process to get its work done and the
  name(s) used by the replication to effect copying. This latter
  revision of the definition of replication is crucial to obtaining
  the expected identity $!!P \sim !P$.
\end{remark}

\begin{remark}\label{rem:paradoxical_combinator}
  The reader familiar with the lambda calculus will have noticed the
  similarity between $D$ and the paradoxical combinator.

  [Ed. note: the existence of this seems to suggest we have to be more
  restrictive on the set of processes and names we admit if we are to
  support no-cloning.]
\end{remark}

\subsubsection{Bisimulation}

The computational dynamics gives rise to another kind of equivalence,
the equivalence of computational behavior. As previously mentioned
this is typically captured \emph{via} some form of bisimulation.

% The notion we use in this paper is weak barbed bisimulation
% \cite{milner91polyadicpi}.

The notion we use in this paper is derived from weak barbed
bisimulation \cite{milner91polyadicpi}. 

\begin{definition}
An \emph{observation relation}, $\downarrow_{\mathcal N}$, over a set
of names, $\mathcal N$, is the smallest relation satisfying the rules
below.

\infrule[Out-barb]{y \in {\mathcal N}, \; x \nameeq y}
		  {\outputp{x}{v} \downarrow_{\mathcal N} x}
\infrule[Par-barb]{\mbox{$P\downarrow_{\mathcal N} x$ or $Q\downarrow_{\mathcal N} x$}}
		  {\binpar{P}{Q} \downarrow_{\mathcal N} x}

We write $P \Downarrow_{\mathcal N} x$ if there is $Q$ such that 
$P \wred Q$ and $Q \downarrow_{\mathcal N} x$.
\end{definition}

\begin{definition}
%\label{def.bbisim}
An  ${\mathcal N}$-\emph{barbed bisimulation} over a set of names, ${\mathcal N}$, is a symmetric binary relation 
${\mathcal S}_{\mathcal N}$ between agents such that $P\rel{S}_{\mathcal N}Q$ implies:
\begin{enumerate}
\item If $P \red P'$ then $Q \wred Q'$ and $P'\rel{S}_{\mathcal N} Q'$.
\item If $P\downarrow_{\mathcal N} x$, then $Q\Downarrow_{\mathcal N} x$.
\end{enumerate}
$P$ is ${\mathcal N}$-barbed bisimilar to $Q$, written
$P \wbbisim_{\mathcal N} Q$, if $P \rel{S}_{\mathcal N} Q$ for some ${\mathcal N}$-barbed bisimulation ${\mathcal S}_{\mathcal N}$.
\end{definition}

$\mathcal{R} \subseteq \pi \times \pi$

$P \mathcal{R} Q => \forall P'. P \red P' \Rightarrow \exists Q'. Q \red Q', P' \mathcal{R} Q'$

$P \vdash x \Rightarrow Q \vdash x$

\begin{mathpar}
  \inferrule*[lab=Out-barb]{x \nameeq y}{{y}!\langle{Q}\rangle \vdash x}
  \and
  \inferrule*[lab=Par-barb]{\mbox{$P\vdash x$ or $Q\vdash x$}}{\binpar{P}{Q} \vdash x}
\end{mathpar}

\subsubsection{Contexts}

One of the principle advantages of computational calculi like the
$\pi$-calculus is a well-defined notion of context,
contextual-equivalence and a correlation between
contextual-equivalence and notions of bisimulation. The notion of
context allows the decomposition of a process into (sub-)process and
its syntactic environment, its context. Thus, a context may be
thought of as a process with a ``hole'' (written $\Box$) in it. The
application of a context $M$ to a process $P$, written $M[P]$, is
tantamount to filling the hole in $M$ with $P$. In this paper we do
not need the full weight of this theory, but do make use of the notion
of context in the proof the main theorem. 

\begin{mathpar}
  \inferrule* [lab=summation] {} {{M_{M},M_{N}} \bc \Box \;|\; x.M_{A} \;|\; M_{M}+M_{N}}
  \and
  \inferrule* [lab=agent] {} {{M_{A}} \bc (\vec{x})M_{P} \;| \; \clift{P_0,\ldots,M_{P},\ldots,P_N}}
  \and \\
  \inferrule* [lab=process] {} {{M_{P}} \bc M_{N} \;| \;P|M_{P} }
\end{mathpar} 

\begin{mathpar}
  \inferrule* [lab=sychronization] {} {M_{N} \bc \Box \;|\; x?M_{F} \;|\; x!M_{C}}
  \and
  \inferrule* [lab=abstraction] {} {{M_{F}} \bc (x)M_{P} }
  \and
  \inferrule* [lab=concretion] {} {{M_{C}} \bc \langle M_{P} \rangle }
  \and \\
  \inferrule* [lab=process] {} {{M_{P}} \bc M_{N} \;| \;P|M_{P} }
\end{mathpar}

\begin{definition}[contextual application] Given a context $M$, and
  process $P$, we define the \emph{contextual application}, $M[P] :=
  M\{P/\Box\}$. That is, the contextual application of M to P is the
  substitution of $P$ for $\Box$ in $M$.
\end{definition}

$\meaningof{-} : L \to \mathcal{P}(\pi)$

\begin{mathpar}
  \inferrule* [lab=collection] {} {\meaningof{true} = \pi, \and \meaningof{~E} = \pi \setminus \meaningof{E}, \and \meaningof{E_{1} \& E_{2}} = \meaningof{E_{1}} \cap \meaningof{E_{2}}}
\end{mathpar}

\begin{mathpar}
  \inferrule* [lab=structure] {} {\meaningof{0} = \{ P \in \pi | P \equiv 0 \}, \and \\ \meaningof{E_1 | E_2} = \{ P \in \pi | P \equiv P_{1} | P_{2}, P_{1} \in \meaningof{E_{1}}, P_{2} \in \meaningof{E_2}\} }
\end{mathpar}

\begin{mathpar}
 \inferrule* [lab=behavior] {} {\meaningof{\langle a?b \rangle E} = \{ P \in \pi | P \equiv Q | u?(y)P', \\ \and \\\\ \and \\ \;\;\; u \in \meaningof{a}, \forall z.P'\{z/y\} \in \meaningof{E\{z/b\}}\}, \and \\ \meaningof{a!E} = \{ P \in \pi | P \equiv Q | x!\langle P' \rangle, x \in \meaningof{a} P' \in \meaningof{E}\} }
\end{mathpar}

\begin{mathpar}
 \inferrule* [lab=nominal] {} {\meaningof{\quotep{E}} = \{ \quotep{P} \in \quotep{\pi} | P \in \meaningof{E} \}, \and \meaningof{\quotep{P}} = \{ \quotep{Q} \in \quotep{\pi} | P \equiv Q \} \and \\ \meaningof{@\quotep{E}} = \{ P \in \pi | P \equiv @x, x \in \meaningof{E} \}}
\end{mathpar}

\begin{eqnarray*}
  \\
  \meaningof{-} : TS \to ST
\end{eqnarray*}

\begin{eqnarray*}
  \\
  L : TS \to ST
\end{eqnarray*}

\begin{eqnarray*}
  \\
  P \models E \iff P \in \meaningof{E}
\end{eqnarray*}

\begin{eqnarray*}
  P \approx_{L} Q \iff \forall E \in L. P \models E \iff Q \models E
\end{eqnarray*}

\begin{eqnarray*}
  P \approx_{K} Q
\end{eqnarray*}

\begin{eqnarray*}
  P \approx Q
\end{eqnarray*}

$\approx_{K} = \approx = \approx_{L}$

\subsubsection{Contextual duality}

Note that contexts extend the quotation operation to a family of
operations from processes to names. Given a context, $M$, we can
define a \emph{nominal context}, $\quotep{M}$ by $\quotep{M}[P] :=
\quotep{M[P]}$. To foreshadow what is to come we observe that these
operations enjoy a duality with processes very much like the duality
between vectors and maps from vectors to scalars.

Further, because the calculus is essentially higher-order, we have a
correspondence between contexts and processes. More specifically,
given a name $x$ and a context $M$ we can construct $M^{*}_{x}$ such
that 

\begin{mathpar}
  M^{*}_{x} | \lift{x}{P} \red M[P]
\end{mathpar}

namely,

\begin{mathpar}
  M^{*}_{x} := x?(u).M[\dropn{u}]
\end{mathpar}

The dependence of $M^{*}_{x}$ on a name makes it an abstraction, 

\begin{mathpar}
  M^{*} := (x)x?(u).M[\dropn{u}]
\end{mathpar}

\subsection{Additional notation}

It will sometimes be convenient to denote the process a name
quotes. We already have the notation $x = \quotep{P}$, but it will be
convenient to introduce an alternate notation, $\procn{x}$, when we
want to emphasize the connection to the use of the name. Note that, by
virtue of name equivalence, $\quotep{\procn{x}} \nameeq x$; so, the
notation is consistent with previous definitions.

Further, because names have structure it is possible to effect
substitutions on the basis of that structure. This means we need to
upgrade our notation for substitutions, which we accomplish by
adapting comprehension notation. Thus,

\begin{mathpar}
  P\{ y / x : x \in S \}
\end{mathpar}

is interpreted to mean the process derived from P by replacing (in a
capture-avoiding manner) each occurrence of $x$ in $S$ by $y$. For example,

\begin{mathpar}
  P\{ \quotep{\procn{x}|\procn{x}} / x : x \in \freenames{P} \}
\end{mathpar}

will replace each (occurrence) of a free name $x$ in $P$ by
$\quotep{\procn{x}|\procn{x}}$.

Also, we will avail ourselves of the notation $x^{L}$ and $x^{R}$ to
denote injections of a name into disjoint copies of the name
space. There are numerous ways to accomplish this. One example can be
found in \cite{MeredithR05}. This notation overloads to vectors of
names: $\vec{x}^{\pi} := (x_{i}^{\pi} \; : \; 0 \leq i < |\vec{x}| )$ where $\pi \in \{L,R\}$.

We also use $P^{\Box} := P|\Box$.

In \cite{MeredithR05} an interpretation of the new operator is
given. It turns out that there are several possible interpretations
all enjoying the requisite algebraic properties of the operator (see
\cite{milner91polyadicpi}). We will therefore make liberal use of
$(\nu\; \vec{x})P$.

% subsection the_syntax_and_semantics_of_the_notation_system (end)   

\input{qm2pi.qmops} 

\input{qm2pi.sterngerlach} 

\input{qm2pi.metric} 

% section concurrent_process_calculi (end)

%\input{qm2pi.proofsketch}

% section proof sketch (end)

%\input{qm2pi.slviaknots} 

% section spatial logic via knots (end)

\input{qm2pi.conclusion}

% section conclusion (end)

%\input{qm2pi.dtcodes} 

% section wiring algorithm (end)

\input{qm2pi.ack} 

% section acknowledgments (end)

\newpage


\bibliographystyle{plain}   
\bibliography{../../biblios/main.bib}

\input{qm2pi.rhodetails}

\end{document}

 

%\documentclass[12pt]{llncs}
%\documentclass{jktr}

\usepackage[pdftex]{hyperref}                   
\usepackage {listings}
\usepackage {mathpartir}
\usepackage{bcprules}
%\usepackage{listings}
                       
\usepackage{graphicx} 
%\usepackage[margins=2.5cm,nohead,nofoot]{geometry}
%\usepackage{geometry}
\usepackage{amsfonts}
\usepackage{amstext}
\usepackage{latexsym}
\usepackage{amssymb}
\usepackage{color}


%\include{myPreamble}
\include{qm2pi.local} 

%\ifpdf
%\usepackage[pdftex]{graphicx}
%\else
%\usepackage{graphicx}
%\fi

 % \ifpdf
%  \usepackage{pdfsync}
%  \if


%\title{Brief Article}
%\author{David F. Snyder}
%\author{L.G. Meredith}

%\address{Dept. of Math., Texas State University--San Marcos, San Marcos, TX 78666}
       
\pagestyle{empty}


\begin{document}

\lstset{language=[Objective]Caml,frame=shadowbox}

\input{qm2pi.front}

% section front matter (end)

\input{qm2pi.intro} 
 
% section introduction (end)

% \input{qm2pi.knotations} 

% section notation (end)

\input{qm2pi.process.calculi} 

% section concurrent_process_calculi_and_spatial_logics_ (end)
    
%\input{qm2pi.knots2pi} 

%\input{qm2pi.trefoil} 

%\input{qm2pi.mainthm} 

% subsection basic_interpretation (end)

%\input{qm2pi.rho.presentation} 
\subsection{The syntax and semantics of the notation system}\label{sub:the_syntax_and_semantics_of_the_notation_system} % (fold)

We now summarize a technical presentation of the calculus that
embodies our theory of dynamics. The typical presentation of such a
calculus follows the style of giving generators and relations on
them. The grammar, below, describing term constructors, freely
generates the set of processes, $\Proc$. This set is then quotiented
by a relation known as structural congruence and it is over this set
that the notion of dynamics is expressed. This presentation is
essentially that of \cite{MeredithR05} with the addition of
polyadicity and summation. For readability we have relegated some of
the technical subtleties to an appendix.

\subsubsection{Process grammar}\label{subsub:process_grammar}

\begin{mathpar}
  \inferrule* [lab=synchronization] {} {{M} \bc \pzero \;|\; x?F \;|\; x!C }
  \and
  \inferrule* [lab=abstraction] {} {{F} \bc (x)P}
  \and
  \inferrule* [lab=concretion] {} {{C} \bc \langle Q \rangle}
  \and
  \inferrule* [lab=process] {} {{P,Q} \bc M \;| \;P|Q \;|\; @{x}}
  \and
  \inferrule* [lab=name] {} {{x} \bc \quotep{P}}
\end{mathpar} 

Note that $\vec{x}$ (resp. $\vec{P}$) denotes a vector of names
(resp. processes) of length $|\vec{x}|$ (resp. $|\vec{P}|$). We adopt
the following useful abbreviations.

\begin{mathpar}
   x?(\vec{y}).P := x.(\vec{y})P \and  x\clift{\vec{P}} := x.\clift{\vec{P}}
   \and x!(y) := \lift{x}{\dropn{y}}
   \and \Pi_{i=0}^{n-1}P_i := P_0 | \ldots | P_{n-1}
\end{mathpar}

\subsubsection{Structural congruence}

\paragraph{Free and bound names and alpha-equivalence.} At the
core of structural equivalence is alpha-equivalence which identifies
process that are the same up to a change of variable. Formally, we
recognize the distinction between free and bound names. The free names
of a process, $\freenames{P}$, may be calculated recursively as
follows:

\begin{mathpar}
\freenames{\pzero} := \emptyset
  \and \\
  \freenames{x?(y).P} := \{ x \} \cup (\freenames{P} \setminus \{ y \})
  \and 
  \freenames{x!\langle P \rangle} := \{ x \} \cup \{ P \} 
  \and \\
  \freenames{P|Q} := \freenames{P} \cup \freenames{Q}
  \and \\
  \freenames{@{x}} := \{ x \}
\end{mathpar}

$\pi$
$\quotep{\pi}$

$\freenames{-} : \pi \to \mathcal{P}(\quotep{\pi})$

\begin{eqnarray*}
  \freenames{\pzero} & := & \emptyset \\
  \freenames{x?(y).P} & := & \{ x \} \cup (\freenames{P} \setminus \{ y \}) \\
  \freenames{x!\langle P \rangle} & := & \{ x \} \cup \{ P \} \\
  \freenames{P|Q} & := & \freenames{P} \cup \freenames{Q} \\
  \freenames{\dropn{x}} & := & \{ x \}
\end{eqnarray*}

The bound names of a process, $\boundnames{P}$, are those names occurring in $P$
that are not free. For example, in $x?(y).0$, the name $x$ is free, while $y$ is bound.

\begin{mathpar}
  \inferrule* [lab=monoidal-laws] {} { P|Q \equiv Q|P \and P|0 \equiv P \and P|(Q|R) \equiv (P|Q)|R }
\end{mathpar}

\begin{mathpar}
  \inferrule* [lab=alpha-equivalence] {} { (x)P \equiv (y)P\{y/x\} \and y \not\in \freenames{P} }
\end{mathpar}

\begin{definition}
Then two processes, $P,Q$, are alpha-equivalent if $P = Q\{\vec{y}/\vec{x}\}$ for
some $\vec{x} \in \boundnames{Q},\vec{y} \in \boundnames{P}$, where $Q\{\vec{y}/\vec{x}\}$
denotes the capture-avoiding substitution of $\vec{y}$ for $\vec{x}$ in $Q$.
\end{definition}

\begin{definition}
  The {\em structural congruence} \cite{SangiorgiWalker} , $\equiv$,
  between processes is the least congruence containing
  alpha-equivalence, satisfying the abelian monoid laws
  (associativity, commutativity and $\pzero$ as identity) for parallel
  composition $|$ and for summation $+$.
\end{definition}

\subsection{Name equivalence}

We take name equivalence, written $\nameeq$, to be the smallest
equivalence relation generated by the following rules.

\begin{mathpar}
\inferrule*[lab=Quote-drop]
{ }
{ \quotep{@{x}} \nameeq x }

\inferrule*[lab=Struct-equiv]
{ P \scong Q }
{ \quotep{P} \nameeq \quotep{Q} }
\end{mathpar}

The astute reader will have noticed that the mutual recursion of names
and processes imposes a mutual recursion on alpha-equivalence and
structural equivalence via name-equivalence. Fortunately, all of this
works out pleasantly and we may calculate in the natural way, free of
concern. The reader interested in the details is referred to the
appendix \ref{appendix:rho_details}.

\subsection{Substitution}

We use $\Proc$ for the set of processes, $\QProc$ for the set of
names, and $\id{\{}\vec{y} / \vec{x} \id{\}}$ to denote partial maps,
$s : \QProc \rightarrow \QProc$. A map, $s$ lifts, uniquely, to a map
on process terms, $\widehat{s} : \Proc \rightarrow \Proc$ by the
following equations.

\begin{mathpar}
  (0) \psubstp{Q}{P} := 0 \\
  (R \juxtap S) \psubstp{Q}{P}
  :=    
  (R)\psubstp{Q}{P} \juxtap (S) \psubstp{Q}{P} \\
  (x?(y).R) \psubstp{Q}{P}    
  :=    
  (x)\substp{Q}{P} (z)\concat( (R \psubstn{z}{y}) \psubstp{Q}{P} ) \\
  (\lift{x}{R}) \psubstp{Q}{P}  
  :=
  \lift{(x)\substp{Q}{P}}{ R \psubstp{Q}{P} } \\
%   (\dropn{x})  \psubstp{Q}{P}       
%   := 
%   \left\{ 
%     \begin{array}{ccc} 
%       \dropn{\quotep{Q}} & & x \nameeq \quotep{P} \\
%       \dropn{x} & & otherwise \\
%     \end{array}
%   \right. 
  (\dropn{x})  \psubstp{Q}{P}       
  := 
  \left\{ 
    \begin{array}{ccc} 
      Q & & x \nameeq \quotep{P} \\
      \dropn{x} & & otherwise \\
    \end{array}
  \right.
\end{mathpar}
 

where

\begin{eqnarray}
  (x)\id{\{} \lpquote Q \rpquote / \lpquote P \rpquote \id{\}}            = 
  \left\{ 
    \begin{array}{ccc}
      \lpquote Q \rpquote & & x \nameeq \lpquote P \rpquote \\
      x & & otherwise \\
    \end{array}
  \right. \nonumber
\end{eqnarray}

and $z$ is chosen distinct from $\quotep{P}$, $\quotep{Q}$, the free
names in $Q$, and all the names in $R$. Our $\alpha$-equivalence will
be built in the standard way from this substitution.

\begin{remark}\label{rem:no_self_referential_names}
  One consequence of these definitions is that $\forall P. \quotep{P}
  \not\in \freenames{P}$.
\end{remark}

\subsection{ Dynamic quote: an example }

Anticipating something of what's to come, consider applying the
substitution, $\widehat{\id{\{}u / z \id{\}}}$, to the following pair
of processes, $\lift{w}{y!(z)}$ and $w[ \lpquote y!(z) \rpquote ]$.

\begin{eqnarray}
	\lift{w}{y!(z)}\widehat{\id{\{}u / z \id{\}}}
		& = &
		\lift{w}{y!(u)} \nonumber\\
	w[ \lpquote y!(z) \rpquote ] \widehat{ \id{\{}u / z \id{\}} }
		& = &
		w[ \lpquote y!(z) \rpquote ] \nonumber
\end{eqnarray}

Because the body of the process between quotes is impervious to
substitution, we get radically different answers. In fact, by
examining the first process in an input context,
e.g. $x?(z).\lift{w}{y!(z)}$, we see that the process under the lift
operator may be shaped by prefixed inputs binding a name inside it. In
this sense, the lift operator will be seen as a way to dynamically
construct processes before reifying them as names.

Finally equipped with these standard features we can present the
dynamics of the calculus.

\subsubsection{Operational semantics} 

Finally, we introduce the computational dynamics. What marks these
algebras as distinct from other more traditionally studied algebraic
structures, e.g. vector spaces or polynomial rings, is the manner in
which dynamics is captured. In traditional structures, dynamics is typically
expressed through morphisms between such structures, as in linear maps
between vector spaces or morphisms between rings. In algebras
associated with the semantics of computation, the dynamics is
expressed as part of the algebraic structure itself, through a
reduction reduction relation typically denoted by $\red$. Below, we
give a recursive presentation of this relation for the calculus used
in the encoding.

$\red \subseteq \pi \times \pi$
$\red : \pi \to \mathcal{P}(\pi)$

\begin{mathpar}
  \inferrule* [lab=Comm] { \textsf{match}( x_{src}, x_{trgt} ) } { x_{trgt}?(y)P \; | \; x_{src}!\langle {Q} \rangle \red P\{\quotep{Q}/y}\} }
  \and \\
  \inferrule* [lab=Par] {{P} \red {P}'} {{{P} | {Q}} \red {{P}' | {Q}}}
  \and
  \inferrule* [lab=Equiv]{{{P} \scong {P}'} \andalso {{P}' \red {Q}'} \andalso {{Q}' \scong {Q}}}{{P} \red {Q}}
\end{mathpar}

\begin{eqnarray*}
  match_{\equiv} (\quotep{P},\quotep{Q}) & := & P \equiv Q \\
  match_{\dagger}(\quotep{P},\quotep{Q}) & := & \forall R. P|Q \red^{*} R => R \red^{*} 0 \\
  match_{K}(\quotep{P},\quotep{Q}) & := & K \mbox{ for some context } K
\end{eqnarray*}

$u?(x)P | u!\langle Q \rangle \red P\{\quotep{Q}/x\}$

%We write $\wred$ for $\red^*$, and $P\red$ if $\exists Q $ such that $ P \red Q$.
We write $P\red$ if $\exists Q $ such that $ P \red Q$ and $P\not\red$, otherwise.

\section{Replication}

As mentioned before, it is known that replication (and hence
recursion) can be implemented in a higher-order process algebra
\cite{SangiorgiWalker}. As our first example of calculation with the
machinery thus far presented we give the construction explicitly in
the {\rhoc}.

\begin{eqnarray}
	D_{x} & := & \prefix{x}{y}{(\binpar{\outputp{x}{y}}{@{y}})} \nonumber\\
	\bangp_{x}{P} & := & \binpar{{x}!\langle{\binpar{D_{x}}{P}}\rangle}{D_{x}} \nonumber
\end{eqnarray}

\begin{eqnarray}
	\bangp_{x}{P} & & \nonumber\\
	=
	& {x}!\langle{(\prefix{x}{y}{(\outputp{x}{y} | @{y})) | P}}\rangle 
	      | \prefix{x}{y}{(\outputp{x}{y} | @{y})} & \nonumber\\
	\red
	& (\outputp{x}{y} | @{y})\substn{\quotep{(\prefix{x}{y}{(@{y} | \outputp{x}{y})) | P}}}{y} & \nonumber\\
	=
	& \outputp{x}{\quotep{(\prefix{x}{y}{(\outputp{x}{y} | @{y})) | P}}}
	  | {(\prefix{x}{y}{(\outputp{x}{y} | @{y})) | P}} & \nonumber\\
	\red
	& \ldots & \nonumber\\
	\red^*
	& P | P | \ldots & \nonumber
\end{eqnarray}

Of course, this encoding, as an implementation, runs away, unfolding
$\bangp{P}$ eagerly. A lazier and more implementable replication
operator, restricted to input-guarded processes, may be obtained as follows.

\begin{eqnarray}
\bangp{\prefix{u}{v}{P}} 
	:= 
	\binpar{\lift{x}{\prefix{u}{v}{(\binpar{D(x)}{P})}}}{D(x)} \nonumber
\end{eqnarray}

\begin{remark}
  Note that the lazier definition still does not deal with summation
  or mixed summation (i.e. sums over input and output). The reader is
  invited to construct definitions of replication that deal with these
  features. 

  Further, the definitions are parameterized in a name, $x$. Can you,
  gentle reader, make a definition that eliminates this parameter and
  guarantees no accidental interaction between the replication
  machinery and the process being replicated -- i.e. no accidental
  sharing of names used by the process to get its work done and the
  name(s) used by the replication to effect copying. This latter
  revision of the definition of replication is crucial to obtaining
  the expected identity $!!P \sim !P$.
\end{remark}

\begin{remark}\label{rem:paradoxical_combinator}
  The reader familiar with the lambda calculus will have noticed the
  similarity between $D$ and the paradoxical combinator.

  [Ed. note: the existence of this seems to suggest we have to be more
  restrictive on the set of processes and names we admit if we are to
  support no-cloning.]
\end{remark}

\subsubsection{Bisimulation}

The computational dynamics gives rise to another kind of equivalence,
the equivalence of computational behavior. As previously mentioned
this is typically captured \emph{via} some form of bisimulation.

% The notion we use in this paper is weak barbed bisimulation
% \cite{milner91polyadicpi}.

The notion we use in this paper is derived from weak barbed
bisimulation \cite{milner91polyadicpi}. 

\begin{definition}
An \emph{observation relation}, $\downarrow_{\mathcal N}$, over a set
of names, $\mathcal N$, is the smallest relation satisfying the rules
below.

\infrule[Out-barb]{y \in {\mathcal N}, \; x \nameeq y}
		  {\outputp{x}{v} \downarrow_{\mathcal N} x}
\infrule[Par-barb]{\mbox{$P\downarrow_{\mathcal N} x$ or $Q\downarrow_{\mathcal N} x$}}
		  {\binpar{P}{Q} \downarrow_{\mathcal N} x}

We write $P \Downarrow_{\mathcal N} x$ if there is $Q$ such that 
$P \wred Q$ and $Q \downarrow_{\mathcal N} x$.
\end{definition}

\begin{definition}
%\label{def.bbisim}
An  ${\mathcal N}$-\emph{barbed bisimulation} over a set of names, ${\mathcal N}$, is a symmetric binary relation 
${\mathcal S}_{\mathcal N}$ between agents such that $P\rel{S}_{\mathcal N}Q$ implies:
\begin{enumerate}
\item If $P \red P'$ then $Q \wred Q'$ and $P'\rel{S}_{\mathcal N} Q'$.
\item If $P\downarrow_{\mathcal N} x$, then $Q\Downarrow_{\mathcal N} x$.
\end{enumerate}
$P$ is ${\mathcal N}$-barbed bisimilar to $Q$, written
$P \wbbisim_{\mathcal N} Q$, if $P \rel{S}_{\mathcal N} Q$ for some ${\mathcal N}$-barbed bisimulation ${\mathcal S}_{\mathcal N}$.
\end{definition}

$\mathcal{R} \subseteq \pi \times \pi$

$P \mathcal{R} Q => \forall P'. P \red P' \Rightarrow \exists Q'. Q \red Q', P' \mathcal{R} Q'$

$P \vdash x \Rightarrow Q \vdash x$

\begin{mathpar}
  \inferrule*[lab=Out-barb]{x \nameeq y}{{y}!\langle{Q}\rangle \vdash x}
  \and
  \inferrule*[lab=Par-barb]{\mbox{$P\vdash x$ or $Q\vdash x$}}{\binpar{P}{Q} \vdash x}
\end{mathpar}

\subsubsection{Contexts}

One of the principle advantages of computational calculi like the
$\pi$-calculus is a well-defined notion of context,
contextual-equivalence and a correlation between
contextual-equivalence and notions of bisimulation. The notion of
context allows the decomposition of a process into (sub-)process and
its syntactic environment, its context. Thus, a context may be
thought of as a process with a ``hole'' (written $\Box$) in it. The
application of a context $M$ to a process $P$, written $M[P]$, is
tantamount to filling the hole in $M$ with $P$. In this paper we do
not need the full weight of this theory, but do make use of the notion
of context in the proof the main theorem. 

\begin{mathpar}
  \inferrule* [lab=summation] {} {{M_{M},M_{N}} \bc \Box \;|\; x.M_{A} \;|\; M_{M}+M_{N}}
  \and
  \inferrule* [lab=agent] {} {{M_{A}} \bc (\vec{x})M_{P} \;| \; \clift{P_0,\ldots,M_{P},\ldots,P_N}}
  \and \\
  \inferrule* [lab=process] {} {{M_{P}} \bc M_{N} \;| \;P|M_{P} }
\end{mathpar} 

\begin{mathpar}
  \inferrule* [lab=sychronization] {} {M_{N} \bc \Box \;|\; x?M_{F} \;|\; x!M_{C}}
  \and
  \inferrule* [lab=abstraction] {} {{M_{F}} \bc (x)M_{P} }
  \and
  \inferrule* [lab=concretion] {} {{M_{C}} \bc \langle M_{P} \rangle }
  \and \\
  \inferrule* [lab=process] {} {{M_{P}} \bc M_{N} \;| \;P|M_{P} }
\end{mathpar}

\begin{definition}[contextual application] Given a context $M$, and
  process $P$, we define the \emph{contextual application}, $M[P] :=
  M\{P/\Box\}$. That is, the contextual application of M to P is the
  substitution of $P$ for $\Box$ in $M$.
\end{definition}

$\meaningof{-} : L \to \mathcal{P}(\pi)$

\begin{mathpar}
  \inferrule* [lab=collection] {} {\meaningof{true} = \pi, \and \meaningof{~E} = \pi \setminus \meaningof{E}, \and \meaningof{E_{1} \& E_{2}} = \meaningof{E_{1}} \cap \meaningof{E_{2}}}
\end{mathpar}

\begin{mathpar}
  \inferrule* [lab=structure] {} {\meaningof{0} = \{ P \in \pi | P \equiv 0 \}, \and \\ \meaningof{E_1 | E_2} = \{ P \in \pi | P \equiv P_{1} | P_{2}, P_{1} \in \meaningof{E_{1}}, P_{2} \in \meaningof{E_2}\} }
\end{mathpar}

\begin{mathpar}
 \inferrule* [lab=behavior] {} {\meaningof{\langle a?b \rangle E} = \{ P \in \pi | P \equiv Q | u?(y)P', \\ \and \\\\ \and \\ \;\;\; u \in \meaningof{a}, \forall z.P'\{z/y\} \in \meaningof{E\{z/b\}}\}, \and \\ \meaningof{a!E} = \{ P \in \pi | P \equiv Q | x!\langle P' \rangle, x \in \meaningof{a} P' \in \meaningof{E}\} }
\end{mathpar}

\begin{mathpar}
 \inferrule* [lab=nominal] {} {\meaningof{\quotep{E}} = \{ \quotep{P} \in \quotep{\pi} | P \in \meaningof{E} \}, \and \meaningof{\quotep{P}} = \{ \quotep{Q} \in \quotep{\pi} | P \equiv Q \} \and \\ \meaningof{@\quotep{E}} = \{ P \in \pi | P \equiv @x, x \in \meaningof{E} \}}
\end{mathpar}

\begin{eqnarray*}
  \\
  \meaningof{-} : TS \to ST
\end{eqnarray*}

\begin{eqnarray*}
  \\
  L : TS \to ST
\end{eqnarray*}

\begin{eqnarray*}
  \\
  P \models E \iff P \in \meaningof{E}
\end{eqnarray*}

\begin{eqnarray*}
  P \approx_{L} Q \iff \forall E \in L. P \models E \iff Q \models E
\end{eqnarray*}

\begin{eqnarray*}
  P \approx_{K} Q
\end{eqnarray*}

\begin{eqnarray*}
  P \approx Q
\end{eqnarray*}

$\approx_{K} = \approx = \approx_{L}$

\subsubsection{Contextual duality}

Note that contexts extend the quotation operation to a family of
operations from processes to names. Given a context, $M$, we can
define a \emph{nominal context}, $\quotep{M}$ by $\quotep{M}[P] :=
\quotep{M[P]}$. To foreshadow what is to come we observe that these
operations enjoy a duality with processes very much like the duality
between vectors and maps from vectors to scalars.

Further, because the calculus is essentially higher-order, we have a
correspondence between contexts and processes. More specifically,
given a name $x$ and a context $M$ we can construct $M^{*}_{x}$ such
that 

\begin{mathpar}
  M^{*}_{x} | \lift{x}{P} \red M[P]
\end{mathpar}

namely,

\begin{mathpar}
  M^{*}_{x} := x?(u).M[\dropn{u}]
\end{mathpar}

The dependence of $M^{*}_{x}$ on a name makes it an abstraction, 

\begin{mathpar}
  M^{*} := (x)x?(u).M[\dropn{u}]
\end{mathpar}

\subsection{Additional notation}

It will sometimes be convenient to denote the process a name
quotes. We already have the notation $x = \quotep{P}$, but it will be
convenient to introduce an alternate notation, $\procn{x}$, when we
want to emphasize the connection to the use of the name. Note that, by
virtue of name equivalence, $\quotep{\procn{x}} \nameeq x$; so, the
notation is consistent with previous definitions.

Further, because names have structure it is possible to effect
substitutions on the basis of that structure. This means we need to
upgrade our notation for substitutions, which we accomplish by
adapting comprehension notation. Thus,

\begin{mathpar}
  P\{ y / x : x \in S \}
\end{mathpar}

is interpreted to mean the process derived from P by replacing (in a
capture-avoiding manner) each occurrence of $x$ in $S$ by $y$. For example,

\begin{mathpar}
  P\{ \quotep{\procn{x}|\procn{x}} / x : x \in \freenames{P} \}
\end{mathpar}

will replace each (occurrence) of a free name $x$ in $P$ by
$\quotep{\procn{x}|\procn{x}}$.

Also, we will avail ourselves of the notation $x^{L}$ and $x^{R}$ to
denote injections of a name into disjoint copies of the name
space. There are numerous ways to accomplish this. One example can be
found in \cite{MeredithR05}. This notation overloads to vectors of
names: $\vec{x}^{\pi} := (x_{i}^{\pi} \; : \; 0 \leq i < |\vec{x}| )$ where $\pi \in \{L,R\}$.

We also use $P^{\Box} := P|\Box$.

In \cite{MeredithR05} an interpretation of the new operator is
given. It turns out that there are several possible interpretations
all enjoying the requisite algebraic properties of the operator (see
\cite{milner91polyadicpi}). We will therefore make liberal use of
$(\nu\; \vec{x})P$.

% subsection the_syntax_and_semantics_of_the_notation_system (end)   

\input{qm2pi.qmops} 

\input{qm2pi.sterngerlach} 

\input{qm2pi.metric} 

% section concurrent_process_calculi (end)

%\input{qm2pi.proofsketch}

% section proof sketch (end)

%\input{qm2pi.slviaknots} 

% section spatial logic via knots (end)

\input{qm2pi.conclusion}

% section conclusion (end)

%\input{qm2pi.dtcodes} 

% section wiring algorithm (end)

\input{qm2pi.ack} 

% section acknowledgments (end)

\newpage


\bibliographystyle{plain}   
\bibliography{../../biblios/main.bib}

\input{qm2pi.rhodetails}

\end{document}

 

%\documentclass[12pt]{llncs}
%\documentclass{jktr}

\usepackage[pdftex]{hyperref}                   
\usepackage {listings}
\usepackage {mathpartir}
\usepackage{bcprules}
%\usepackage{listings}
                       
\usepackage{graphicx} 
%\usepackage[margins=2.5cm,nohead,nofoot]{geometry}
%\usepackage{geometry}
\usepackage{amsfonts}
\usepackage{amstext}
\usepackage{latexsym}
\usepackage{amssymb}
\usepackage{color}


%\include{myPreamble}
\include{qm2pi.local} 

%\ifpdf
%\usepackage[pdftex]{graphicx}
%\else
%\usepackage{graphicx}
%\fi

 % \ifpdf
%  \usepackage{pdfsync}
%  \if


%\title{Brief Article}
%\author{David F. Snyder}
%\author{L.G. Meredith}

%\address{Dept. of Math., Texas State University--San Marcos, San Marcos, TX 78666}
       
\pagestyle{empty}


\begin{document}

\lstset{language=[Objective]Caml,frame=shadowbox}

\input{qm2pi.front}

% section front matter (end)

\input{qm2pi.intro} 
 
% section introduction (end)

% \input{qm2pi.knotations} 

% section notation (end)

\input{qm2pi.process.calculi} 

% section concurrent_process_calculi_and_spatial_logics_ (end)
    
%\input{qm2pi.knots2pi} 

%\input{qm2pi.trefoil} 

%\input{qm2pi.mainthm} 

% subsection basic_interpretation (end)

%\input{qm2pi.rho.presentation} 
\subsection{The syntax and semantics of the notation system}\label{sub:the_syntax_and_semantics_of_the_notation_system} % (fold)

We now summarize a technical presentation of the calculus that
embodies our theory of dynamics. The typical presentation of such a
calculus follows the style of giving generators and relations on
them. The grammar, below, describing term constructors, freely
generates the set of processes, $\Proc$. This set is then quotiented
by a relation known as structural congruence and it is over this set
that the notion of dynamics is expressed. This presentation is
essentially that of \cite{MeredithR05} with the addition of
polyadicity and summation. For readability we have relegated some of
the technical subtleties to an appendix.

\subsubsection{Process grammar}\label{subsub:process_grammar}

\begin{mathpar}
  \inferrule* [lab=synchronization] {} {{M} \bc \pzero \;|\; x?F \;|\; x!C }
  \and
  \inferrule* [lab=abstraction] {} {{F} \bc (x)P}
  \and
  \inferrule* [lab=concretion] {} {{C} \bc \langle Q \rangle}
  \and
  \inferrule* [lab=process] {} {{P,Q} \bc M \;| \;P|Q \;|\; @{x}}
  \and
  \inferrule* [lab=name] {} {{x} \bc \quotep{P}}
\end{mathpar} 

Note that $\vec{x}$ (resp. $\vec{P}$) denotes a vector of names
(resp. processes) of length $|\vec{x}|$ (resp. $|\vec{P}|$). We adopt
the following useful abbreviations.

\begin{mathpar}
   x?(\vec{y}).P := x.(\vec{y})P \and  x\clift{\vec{P}} := x.\clift{\vec{P}}
   \and x!(y) := \lift{x}{\dropn{y}}
   \and \Pi_{i=0}^{n-1}P_i := P_0 | \ldots | P_{n-1}
\end{mathpar}

\subsubsection{Structural congruence}

\paragraph{Free and bound names and alpha-equivalence.} At the
core of structural equivalence is alpha-equivalence which identifies
process that are the same up to a change of variable. Formally, we
recognize the distinction between free and bound names. The free names
of a process, $\freenames{P}$, may be calculated recursively as
follows:

\begin{mathpar}
\freenames{\pzero} := \emptyset
  \and \\
  \freenames{x?(y).P} := \{ x \} \cup (\freenames{P} \setminus \{ y \})
  \and 
  \freenames{x!\langle P \rangle} := \{ x \} \cup \{ P \} 
  \and \\
  \freenames{P|Q} := \freenames{P} \cup \freenames{Q}
  \and \\
  \freenames{@{x}} := \{ x \}
\end{mathpar}

$\pi$
$\quotep{\pi}$

$\freenames{-} : \pi \to \mathcal{P}(\quotep{\pi})$

\begin{eqnarray*}
  \freenames{\pzero} & := & \emptyset \\
  \freenames{x?(y).P} & := & \{ x \} \cup (\freenames{P} \setminus \{ y \}) \\
  \freenames{x!\langle P \rangle} & := & \{ x \} \cup \{ P \} \\
  \freenames{P|Q} & := & \freenames{P} \cup \freenames{Q} \\
  \freenames{\dropn{x}} & := & \{ x \}
\end{eqnarray*}

The bound names of a process, $\boundnames{P}$, are those names occurring in $P$
that are not free. For example, in $x?(y).0$, the name $x$ is free, while $y$ is bound.

\begin{mathpar}
  \inferrule* [lab=monoidal-laws] {} { P|Q \equiv Q|P \and P|0 \equiv P \and P|(Q|R) \equiv (P|Q)|R }
\end{mathpar}

\begin{mathpar}
  \inferrule* [lab=alpha-equivalence] {} { (x)P \equiv (y)P\{y/x\} \and y \not\in \freenames{P} }
\end{mathpar}

\begin{definition}
Then two processes, $P,Q$, are alpha-equivalent if $P = Q\{\vec{y}/\vec{x}\}$ for
some $\vec{x} \in \boundnames{Q},\vec{y} \in \boundnames{P}$, where $Q\{\vec{y}/\vec{x}\}$
denotes the capture-avoiding substitution of $\vec{y}$ for $\vec{x}$ in $Q$.
\end{definition}

\begin{definition}
  The {\em structural congruence} \cite{SangiorgiWalker} , $\equiv$,
  between processes is the least congruence containing
  alpha-equivalence, satisfying the abelian monoid laws
  (associativity, commutativity and $\pzero$ as identity) for parallel
  composition $|$ and for summation $+$.
\end{definition}

\subsection{Name equivalence}

We take name equivalence, written $\nameeq$, to be the smallest
equivalence relation generated by the following rules.

\begin{mathpar}
\inferrule*[lab=Quote-drop]
{ }
{ \quotep{@{x}} \nameeq x }

\inferrule*[lab=Struct-equiv]
{ P \scong Q }
{ \quotep{P} \nameeq \quotep{Q} }
\end{mathpar}

The astute reader will have noticed that the mutual recursion of names
and processes imposes a mutual recursion on alpha-equivalence and
structural equivalence via name-equivalence. Fortunately, all of this
works out pleasantly and we may calculate in the natural way, free of
concern. The reader interested in the details is referred to the
appendix \ref{appendix:rho_details}.

\subsection{Substitution}

We use $\Proc$ for the set of processes, $\QProc$ for the set of
names, and $\id{\{}\vec{y} / \vec{x} \id{\}}$ to denote partial maps,
$s : \QProc \rightarrow \QProc$. A map, $s$ lifts, uniquely, to a map
on process terms, $\widehat{s} : \Proc \rightarrow \Proc$ by the
following equations.

\begin{mathpar}
  (0) \psubstp{Q}{P} := 0 \\
  (R \juxtap S) \psubstp{Q}{P}
  :=    
  (R)\psubstp{Q}{P} \juxtap (S) \psubstp{Q}{P} \\
  (x?(y).R) \psubstp{Q}{P}    
  :=    
  (x)\substp{Q}{P} (z)\concat( (R \psubstn{z}{y}) \psubstp{Q}{P} ) \\
  (\lift{x}{R}) \psubstp{Q}{P}  
  :=
  \lift{(x)\substp{Q}{P}}{ R \psubstp{Q}{P} } \\
%   (\dropn{x})  \psubstp{Q}{P}       
%   := 
%   \left\{ 
%     \begin{array}{ccc} 
%       \dropn{\quotep{Q}} & & x \nameeq \quotep{P} \\
%       \dropn{x} & & otherwise \\
%     \end{array}
%   \right. 
  (\dropn{x})  \psubstp{Q}{P}       
  := 
  \left\{ 
    \begin{array}{ccc} 
      Q & & x \nameeq \quotep{P} \\
      \dropn{x} & & otherwise \\
    \end{array}
  \right.
\end{mathpar}
 

where

\begin{eqnarray}
  (x)\id{\{} \lpquote Q \rpquote / \lpquote P \rpquote \id{\}}            = 
  \left\{ 
    \begin{array}{ccc}
      \lpquote Q \rpquote & & x \nameeq \lpquote P \rpquote \\
      x & & otherwise \\
    \end{array}
  \right. \nonumber
\end{eqnarray}

and $z$ is chosen distinct from $\quotep{P}$, $\quotep{Q}$, the free
names in $Q$, and all the names in $R$. Our $\alpha$-equivalence will
be built in the standard way from this substitution.

\begin{remark}\label{rem:no_self_referential_names}
  One consequence of these definitions is that $\forall P. \quotep{P}
  \not\in \freenames{P}$.
\end{remark}

\subsection{ Dynamic quote: an example }

Anticipating something of what's to come, consider applying the
substitution, $\widehat{\id{\{}u / z \id{\}}}$, to the following pair
of processes, $\lift{w}{y!(z)}$ and $w[ \lpquote y!(z) \rpquote ]$.

\begin{eqnarray}
	\lift{w}{y!(z)}\widehat{\id{\{}u / z \id{\}}}
		& = &
		\lift{w}{y!(u)} \nonumber\\
	w[ \lpquote y!(z) \rpquote ] \widehat{ \id{\{}u / z \id{\}} }
		& = &
		w[ \lpquote y!(z) \rpquote ] \nonumber
\end{eqnarray}

Because the body of the process between quotes is impervious to
substitution, we get radically different answers. In fact, by
examining the first process in an input context,
e.g. $x?(z).\lift{w}{y!(z)}$, we see that the process under the lift
operator may be shaped by prefixed inputs binding a name inside it. In
this sense, the lift operator will be seen as a way to dynamically
construct processes before reifying them as names.

Finally equipped with these standard features we can present the
dynamics of the calculus.

\subsubsection{Operational semantics} 

Finally, we introduce the computational dynamics. What marks these
algebras as distinct from other more traditionally studied algebraic
structures, e.g. vector spaces or polynomial rings, is the manner in
which dynamics is captured. In traditional structures, dynamics is typically
expressed through morphisms between such structures, as in linear maps
between vector spaces or morphisms between rings. In algebras
associated with the semantics of computation, the dynamics is
expressed as part of the algebraic structure itself, through a
reduction reduction relation typically denoted by $\red$. Below, we
give a recursive presentation of this relation for the calculus used
in the encoding.

$\red \subseteq \pi \times \pi$
$\red : \pi \to \mathcal{P}(\pi)$

\begin{mathpar}
  \inferrule* [lab=Comm] { \textsf{match}( x_{src}, x_{trgt} ) } { x_{trgt}?(y)P \; | \; x_{src}!\langle {Q} \rangle \red P\{\quotep{Q}/y}\} }
  \and \\
  \inferrule* [lab=Par] {{P} \red {P}'} {{{P} | {Q}} \red {{P}' | {Q}}}
  \and
  \inferrule* [lab=Equiv]{{{P} \scong {P}'} \andalso {{P}' \red {Q}'} \andalso {{Q}' \scong {Q}}}{{P} \red {Q}}
\end{mathpar}

\begin{eqnarray*}
  match_{\equiv} (\quotep{P},\quotep{Q}) & := & P \equiv Q \\
  match_{\dagger}(\quotep{P},\quotep{Q}) & := & \forall R. P|Q \red^{*} R => R \red^{*} 0 \\
  match_{K}(\quotep{P},\quotep{Q}) & := & K \mbox{ for some context } K
\end{eqnarray*}

$u?(x)P | u!\langle Q \rangle \red P\{\quotep{Q}/x\}$

%We write $\wred$ for $\red^*$, and $P\red$ if $\exists Q $ such that $ P \red Q$.
We write $P\red$ if $\exists Q $ such that $ P \red Q$ and $P\not\red$, otherwise.

\section{Replication}

As mentioned before, it is known that replication (and hence
recursion) can be implemented in a higher-order process algebra
\cite{SangiorgiWalker}. As our first example of calculation with the
machinery thus far presented we give the construction explicitly in
the {\rhoc}.

\begin{eqnarray}
	D_{x} & := & \prefix{x}{y}{(\binpar{\outputp{x}{y}}{@{y}})} \nonumber\\
	\bangp_{x}{P} & := & \binpar{{x}!\langle{\binpar{D_{x}}{P}}\rangle}{D_{x}} \nonumber
\end{eqnarray}

\begin{eqnarray}
	\bangp_{x}{P} & & \nonumber\\
	=
	& {x}!\langle{(\prefix{x}{y}{(\outputp{x}{y} | @{y})) | P}}\rangle 
	      | \prefix{x}{y}{(\outputp{x}{y} | @{y})} & \nonumber\\
	\red
	& (\outputp{x}{y} | @{y})\substn{\quotep{(\prefix{x}{y}{(@{y} | \outputp{x}{y})) | P}}}{y} & \nonumber\\
	=
	& \outputp{x}{\quotep{(\prefix{x}{y}{(\outputp{x}{y} | @{y})) | P}}}
	  | {(\prefix{x}{y}{(\outputp{x}{y} | @{y})) | P}} & \nonumber\\
	\red
	& \ldots & \nonumber\\
	\red^*
	& P | P | \ldots & \nonumber
\end{eqnarray}

Of course, this encoding, as an implementation, runs away, unfolding
$\bangp{P}$ eagerly. A lazier and more implementable replication
operator, restricted to input-guarded processes, may be obtained as follows.

\begin{eqnarray}
\bangp{\prefix{u}{v}{P}} 
	:= 
	\binpar{\lift{x}{\prefix{u}{v}{(\binpar{D(x)}{P})}}}{D(x)} \nonumber
\end{eqnarray}

\begin{remark}
  Note that the lazier definition still does not deal with summation
  or mixed summation (i.e. sums over input and output). The reader is
  invited to construct definitions of replication that deal with these
  features. 

  Further, the definitions are parameterized in a name, $x$. Can you,
  gentle reader, make a definition that eliminates this parameter and
  guarantees no accidental interaction between the replication
  machinery and the process being replicated -- i.e. no accidental
  sharing of names used by the process to get its work done and the
  name(s) used by the replication to effect copying. This latter
  revision of the definition of replication is crucial to obtaining
  the expected identity $!!P \sim !P$.
\end{remark}

\begin{remark}\label{rem:paradoxical_combinator}
  The reader familiar with the lambda calculus will have noticed the
  similarity between $D$ and the paradoxical combinator.

  [Ed. note: the existence of this seems to suggest we have to be more
  restrictive on the set of processes and names we admit if we are to
  support no-cloning.]
\end{remark}

\subsubsection{Bisimulation}

The computational dynamics gives rise to another kind of equivalence,
the equivalence of computational behavior. As previously mentioned
this is typically captured \emph{via} some form of bisimulation.

% The notion we use in this paper is weak barbed bisimulation
% \cite{milner91polyadicpi}.

The notion we use in this paper is derived from weak barbed
bisimulation \cite{milner91polyadicpi}. 

\begin{definition}
An \emph{observation relation}, $\downarrow_{\mathcal N}$, over a set
of names, $\mathcal N$, is the smallest relation satisfying the rules
below.

\infrule[Out-barb]{y \in {\mathcal N}, \; x \nameeq y}
		  {\outputp{x}{v} \downarrow_{\mathcal N} x}
\infrule[Par-barb]{\mbox{$P\downarrow_{\mathcal N} x$ or $Q\downarrow_{\mathcal N} x$}}
		  {\binpar{P}{Q} \downarrow_{\mathcal N} x}

We write $P \Downarrow_{\mathcal N} x$ if there is $Q$ such that 
$P \wred Q$ and $Q \downarrow_{\mathcal N} x$.
\end{definition}

\begin{definition}
%\label{def.bbisim}
An  ${\mathcal N}$-\emph{barbed bisimulation} over a set of names, ${\mathcal N}$, is a symmetric binary relation 
${\mathcal S}_{\mathcal N}$ between agents such that $P\rel{S}_{\mathcal N}Q$ implies:
\begin{enumerate}
\item If $P \red P'$ then $Q \wred Q'$ and $P'\rel{S}_{\mathcal N} Q'$.
\item If $P\downarrow_{\mathcal N} x$, then $Q\Downarrow_{\mathcal N} x$.
\end{enumerate}
$P$ is ${\mathcal N}$-barbed bisimilar to $Q$, written
$P \wbbisim_{\mathcal N} Q$, if $P \rel{S}_{\mathcal N} Q$ for some ${\mathcal N}$-barbed bisimulation ${\mathcal S}_{\mathcal N}$.
\end{definition}

$\mathcal{R} \subseteq \pi \times \pi$

$P \mathcal{R} Q => \forall P'. P \red P' \Rightarrow \exists Q'. Q \red Q', P' \mathcal{R} Q'$

$P \vdash x \Rightarrow Q \vdash x$

\begin{mathpar}
  \inferrule*[lab=Out-barb]{x \nameeq y}{{y}!\langle{Q}\rangle \vdash x}
  \and
  \inferrule*[lab=Par-barb]{\mbox{$P\vdash x$ or $Q\vdash x$}}{\binpar{P}{Q} \vdash x}
\end{mathpar}

\subsubsection{Contexts}

One of the principle advantages of computational calculi like the
$\pi$-calculus is a well-defined notion of context,
contextual-equivalence and a correlation between
contextual-equivalence and notions of bisimulation. The notion of
context allows the decomposition of a process into (sub-)process and
its syntactic environment, its context. Thus, a context may be
thought of as a process with a ``hole'' (written $\Box$) in it. The
application of a context $M$ to a process $P$, written $M[P]$, is
tantamount to filling the hole in $M$ with $P$. In this paper we do
not need the full weight of this theory, but do make use of the notion
of context in the proof the main theorem. 

\begin{mathpar}
  \inferrule* [lab=summation] {} {{M_{M},M_{N}} \bc \Box \;|\; x.M_{A} \;|\; M_{M}+M_{N}}
  \and
  \inferrule* [lab=agent] {} {{M_{A}} \bc (\vec{x})M_{P} \;| \; \clift{P_0,\ldots,M_{P},\ldots,P_N}}
  \and \\
  \inferrule* [lab=process] {} {{M_{P}} \bc M_{N} \;| \;P|M_{P} }
\end{mathpar} 

\begin{mathpar}
  \inferrule* [lab=sychronization] {} {M_{N} \bc \Box \;|\; x?M_{F} \;|\; x!M_{C}}
  \and
  \inferrule* [lab=abstraction] {} {{M_{F}} \bc (x)M_{P} }
  \and
  \inferrule* [lab=concretion] {} {{M_{C}} \bc \langle M_{P} \rangle }
  \and \\
  \inferrule* [lab=process] {} {{M_{P}} \bc M_{N} \;| \;P|M_{P} }
\end{mathpar}

\begin{definition}[contextual application] Given a context $M$, and
  process $P$, we define the \emph{contextual application}, $M[P] :=
  M\{P/\Box\}$. That is, the contextual application of M to P is the
  substitution of $P$ for $\Box$ in $M$.
\end{definition}

$\meaningof{-} : L \to \mathcal{P}(\pi)$

\begin{mathpar}
  \inferrule* [lab=collection] {} {\meaningof{true} = \pi, \and \meaningof{~E} = \pi \setminus \meaningof{E}, \and \meaningof{E_{1} \& E_{2}} = \meaningof{E_{1}} \cap \meaningof{E_{2}}}
\end{mathpar}

\begin{mathpar}
  \inferrule* [lab=structure] {} {\meaningof{0} = \{ P \in \pi | P \equiv 0 \}, \and \\ \meaningof{E_1 | E_2} = \{ P \in \pi | P \equiv P_{1} | P_{2}, P_{1} \in \meaningof{E_{1}}, P_{2} \in \meaningof{E_2}\} }
\end{mathpar}

\begin{mathpar}
 \inferrule* [lab=behavior] {} {\meaningof{\langle a?b \rangle E} = \{ P \in \pi | P \equiv Q | u?(y)P', \\ \and \\\\ \and \\ \;\;\; u \in \meaningof{a}, \forall z.P'\{z/y\} \in \meaningof{E\{z/b\}}\}, \and \\ \meaningof{a!E} = \{ P \in \pi | P \equiv Q | x!\langle P' \rangle, x \in \meaningof{a} P' \in \meaningof{E}\} }
\end{mathpar}

\begin{mathpar}
 \inferrule* [lab=nominal] {} {\meaningof{\quotep{E}} = \{ \quotep{P} \in \quotep{\pi} | P \in \meaningof{E} \}, \and \meaningof{\quotep{P}} = \{ \quotep{Q} \in \quotep{\pi} | P \equiv Q \} \and \\ \meaningof{@\quotep{E}} = \{ P \in \pi | P \equiv @x, x \in \meaningof{E} \}}
\end{mathpar}

\begin{eqnarray*}
  \\
  \meaningof{-} : TS \to ST
\end{eqnarray*}

\begin{eqnarray*}
  \\
  L : TS \to ST
\end{eqnarray*}

\begin{eqnarray*}
  \\
  P \models E \iff P \in \meaningof{E}
\end{eqnarray*}

\begin{eqnarray*}
  P \approx_{L} Q \iff \forall E \in L. P \models E \iff Q \models E
\end{eqnarray*}

\begin{eqnarray*}
  P \approx_{K} Q
\end{eqnarray*}

\begin{eqnarray*}
  P \approx Q
\end{eqnarray*}

$\approx_{K} = \approx = \approx_{L}$

\subsubsection{Contextual duality}

Note that contexts extend the quotation operation to a family of
operations from processes to names. Given a context, $M$, we can
define a \emph{nominal context}, $\quotep{M}$ by $\quotep{M}[P] :=
\quotep{M[P]}$. To foreshadow what is to come we observe that these
operations enjoy a duality with processes very much like the duality
between vectors and maps from vectors to scalars.

Further, because the calculus is essentially higher-order, we have a
correspondence between contexts and processes. More specifically,
given a name $x$ and a context $M$ we can construct $M^{*}_{x}$ such
that 

\begin{mathpar}
  M^{*}_{x} | \lift{x}{P} \red M[P]
\end{mathpar}

namely,

\begin{mathpar}
  M^{*}_{x} := x?(u).M[\dropn{u}]
\end{mathpar}

The dependence of $M^{*}_{x}$ on a name makes it an abstraction, 

\begin{mathpar}
  M^{*} := (x)x?(u).M[\dropn{u}]
\end{mathpar}

\subsection{Additional notation}

It will sometimes be convenient to denote the process a name
quotes. We already have the notation $x = \quotep{P}$, but it will be
convenient to introduce an alternate notation, $\procn{x}$, when we
want to emphasize the connection to the use of the name. Note that, by
virtue of name equivalence, $\quotep{\procn{x}} \nameeq x$; so, the
notation is consistent with previous definitions.

Further, because names have structure it is possible to effect
substitutions on the basis of that structure. This means we need to
upgrade our notation for substitutions, which we accomplish by
adapting comprehension notation. Thus,

\begin{mathpar}
  P\{ y / x : x \in S \}
\end{mathpar}

is interpreted to mean the process derived from P by replacing (in a
capture-avoiding manner) each occurrence of $x$ in $S$ by $y$. For example,

\begin{mathpar}
  P\{ \quotep{\procn{x}|\procn{x}} / x : x \in \freenames{P} \}
\end{mathpar}

will replace each (occurrence) of a free name $x$ in $P$ by
$\quotep{\procn{x}|\procn{x}}$.

Also, we will avail ourselves of the notation $x^{L}$ and $x^{R}$ to
denote injections of a name into disjoint copies of the name
space. There are numerous ways to accomplish this. One example can be
found in \cite{MeredithR05}. This notation overloads to vectors of
names: $\vec{x}^{\pi} := (x_{i}^{\pi} \; : \; 0 \leq i < |\vec{x}| )$ where $\pi \in \{L,R\}$.

We also use $P^{\Box} := P|\Box$.

In \cite{MeredithR05} an interpretation of the new operator is
given. It turns out that there are several possible interpretations
all enjoying the requisite algebraic properties of the operator (see
\cite{milner91polyadicpi}). We will therefore make liberal use of
$(\nu\; \vec{x})P$.

% subsection the_syntax_and_semantics_of_the_notation_system (end)   

\input{qm2pi.qmops} 

\input{qm2pi.sterngerlach} 

\input{qm2pi.metric} 

% section concurrent_process_calculi (end)

%\input{qm2pi.proofsketch}

% section proof sketch (end)

%\input{qm2pi.slviaknots} 

% section spatial logic via knots (end)

\input{qm2pi.conclusion}

% section conclusion (end)

%\input{qm2pi.dtcodes} 

% section wiring algorithm (end)

\input{qm2pi.ack} 

% section acknowledgments (end)

\newpage


\bibliographystyle{plain}   
\bibliography{../../biblios/main.bib}

\input{qm2pi.rhodetails}

\end{document}

 

% subsection basic_interpretation (end)

%\input{qm2pi.rho.presentation} 
\subsection{The syntax and semantics of the notation system}\label{sub:the_syntax_and_semantics_of_the_notation_system} % (fold)

We now summarize a technical presentation of the calculus that
embodies our theory of dynamics. The typical presentation of such a
calculus follows the style of giving generators and relations on
them. The grammar, below, describing term constructors, freely
generates the set of processes, $\Proc$. This set is then quotiented
by a relation known as structural congruence and it is over this set
that the notion of dynamics is expressed. This presentation is
essentially that of \cite{MeredithR05} with the addition of
polyadicity and summation. For readability we have relegated some of
the technical subtleties to an appendix.

\subsubsection{Process grammar}\label{subsub:process_grammar}

\begin{mathpar}
  \inferrule* [lab=synchronization] {} {{M} \bc \pzero \;|\; x?F \;|\; x!C }
  \and
  \inferrule* [lab=abstraction] {} {{F} \bc (x)P}
  \and
  \inferrule* [lab=concretion] {} {{C} \bc \langle Q \rangle}
  \and
  \inferrule* [lab=process] {} {{P,Q} \bc M \;| \;P|Q \;|\; @{x}}
  \and
  \inferrule* [lab=name] {} {{x} \bc \quotep{P}}
\end{mathpar} 

Note that $\vec{x}$ (resp. $\vec{P}$) denotes a vector of names
(resp. processes) of length $|\vec{x}|$ (resp. $|\vec{P}|$). We adopt
the following useful abbreviations.

\begin{mathpar}
   x?(\vec{y}).P := x.(\vec{y})P \and  x\clift{\vec{P}} := x.\clift{\vec{P}}
   \and x!(y) := \lift{x}{\dropn{y}}
   \and \Pi_{i=0}^{n-1}P_i := P_0 | \ldots | P_{n-1}
\end{mathpar}

\subsubsection{Structural congruence}

\paragraph{Free and bound names and alpha-equivalence.} At the
core of structural equivalence is alpha-equivalence which identifies
process that are the same up to a change of variable. Formally, we
recognize the distinction between free and bound names. The free names
of a process, $\freenames{P}$, may be calculated recursively as
follows:

\begin{mathpar}
\freenames{\pzero} := \emptyset
  \and \\
  \freenames{x?(y).P} := \{ x \} \cup (\freenames{P} \setminus \{ y \})
  \and 
  \freenames{x!\langle P \rangle} := \{ x \} \cup \{ P \} 
  \and \\
  \freenames{P|Q} := \freenames{P} \cup \freenames{Q}
  \and \\
  \freenames{@{x}} := \{ x \}
\end{mathpar}

$\pi$
$\quotep{\pi}$

$\freenames{-} : \pi \to \mathcal{P}(\quotep{\pi})$

\begin{eqnarray*}
  \freenames{\pzero} & := & \emptyset \\
  \freenames{x?(y).P} & := & \{ x \} \cup (\freenames{P} \setminus \{ y \}) \\
  \freenames{x!\langle P \rangle} & := & \{ x \} \cup \{ P \} \\
  \freenames{P|Q} & := & \freenames{P} \cup \freenames{Q} \\
  \freenames{\dropn{x}} & := & \{ x \}
\end{eqnarray*}

The bound names of a process, $\boundnames{P}$, are those names occurring in $P$
that are not free. For example, in $x?(y).0$, the name $x$ is free, while $y$ is bound.

\begin{mathpar}
  \inferrule* [lab=monoidal-laws] {} { P|Q \equiv Q|P \and P|0 \equiv P \and P|(Q|R) \equiv (P|Q)|R }
\end{mathpar}

\begin{mathpar}
  \inferrule* [lab=alpha-equivalence] {} { (x)P \equiv (y)P\{y/x\} \and y \not\in \freenames{P} }
\end{mathpar}

\begin{definition}
Then two processes, $P,Q$, are alpha-equivalent if $P = Q\{\vec{y}/\vec{x}\}$ for
some $\vec{x} \in \boundnames{Q},\vec{y} \in \boundnames{P}$, where $Q\{\vec{y}/\vec{x}\}$
denotes the capture-avoiding substitution of $\vec{y}$ for $\vec{x}$ in $Q$.
\end{definition}

\begin{definition}
  The {\em structural congruence} \cite{SangiorgiWalker} , $\equiv$,
  between processes is the least congruence containing
  alpha-equivalence, satisfying the abelian monoid laws
  (associativity, commutativity and $\pzero$ as identity) for parallel
  composition $|$ and for summation $+$.
\end{definition}

\subsection{Name equivalence}

We take name equivalence, written $\nameeq$, to be the smallest
equivalence relation generated by the following rules.

\begin{mathpar}
\inferrule*[lab=Quote-drop]
{ }
{ \quotep{@{x}} \nameeq x }

\inferrule*[lab=Struct-equiv]
{ P \scong Q }
{ \quotep{P} \nameeq \quotep{Q} }
\end{mathpar}

The astute reader will have noticed that the mutual recursion of names
and processes imposes a mutual recursion on alpha-equivalence and
structural equivalence via name-equivalence. Fortunately, all of this
works out pleasantly and we may calculate in the natural way, free of
concern. The reader interested in the details is referred to the
appendix \ref{appendix:rho_details}.

\subsection{Substitution}

We use $\Proc$ for the set of processes, $\QProc$ for the set of
names, and $\id{\{}\vec{y} / \vec{x} \id{\}}$ to denote partial maps,
$s : \QProc \rightarrow \QProc$. A map, $s$ lifts, uniquely, to a map
on process terms, $\widehat{s} : \Proc \rightarrow \Proc$ by the
following equations.

\begin{mathpar}
  (0) \psubstp{Q}{P} := 0 \\
  (R \juxtap S) \psubstp{Q}{P}
  :=    
  (R)\psubstp{Q}{P} \juxtap (S) \psubstp{Q}{P} \\
  (x?(y).R) \psubstp{Q}{P}    
  :=    
  (x)\substp{Q}{P} (z)\concat( (R \psubstn{z}{y}) \psubstp{Q}{P} ) \\
  (\lift{x}{R}) \psubstp{Q}{P}  
  :=
  \lift{(x)\substp{Q}{P}}{ R \psubstp{Q}{P} } \\
%   (\dropn{x})  \psubstp{Q}{P}       
%   := 
%   \left\{ 
%     \begin{array}{ccc} 
%       \dropn{\quotep{Q}} & & x \nameeq \quotep{P} \\
%       \dropn{x} & & otherwise \\
%     \end{array}
%   \right. 
  (\dropn{x})  \psubstp{Q}{P}       
  := 
  \left\{ 
    \begin{array}{ccc} 
      Q & & x \nameeq \quotep{P} \\
      \dropn{x} & & otherwise \\
    \end{array}
  \right.
\end{mathpar}
 

where

\begin{eqnarray}
  (x)\id{\{} \lpquote Q \rpquote / \lpquote P \rpquote \id{\}}            = 
  \left\{ 
    \begin{array}{ccc}
      \lpquote Q \rpquote & & x \nameeq \lpquote P \rpquote \\
      x & & otherwise \\
    \end{array}
  \right. \nonumber
\end{eqnarray}

and $z$ is chosen distinct from $\quotep{P}$, $\quotep{Q}$, the free
names in $Q$, and all the names in $R$. Our $\alpha$-equivalence will
be built in the standard way from this substitution.

\begin{remark}\label{rem:no_self_referential_names}
  One consequence of these definitions is that $\forall P. \quotep{P}
  \not\in \freenames{P}$.
\end{remark}

\subsection{ Dynamic quote: an example }

Anticipating something of what's to come, consider applying the
substitution, $\widehat{\id{\{}u / z \id{\}}}$, to the following pair
of processes, $\lift{w}{y!(z)}$ and $w[ \lpquote y!(z) \rpquote ]$.

\begin{eqnarray}
	\lift{w}{y!(z)}\widehat{\id{\{}u / z \id{\}}}
		& = &
		\lift{w}{y!(u)} \nonumber\\
	w[ \lpquote y!(z) \rpquote ] \widehat{ \id{\{}u / z \id{\}} }
		& = &
		w[ \lpquote y!(z) \rpquote ] \nonumber
\end{eqnarray}

Because the body of the process between quotes is impervious to
substitution, we get radically different answers. In fact, by
examining the first process in an input context,
e.g. $x?(z).\lift{w}{y!(z)}$, we see that the process under the lift
operator may be shaped by prefixed inputs binding a name inside it. In
this sense, the lift operator will be seen as a way to dynamically
construct processes before reifying them as names.

Finally equipped with these standard features we can present the
dynamics of the calculus.

\subsubsection{Operational semantics} 

Finally, we introduce the computational dynamics. What marks these
algebras as distinct from other more traditionally studied algebraic
structures, e.g. vector spaces or polynomial rings, is the manner in
which dynamics is captured. In traditional structures, dynamics is typically
expressed through morphisms between such structures, as in linear maps
between vector spaces or morphisms between rings. In algebras
associated with the semantics of computation, the dynamics is
expressed as part of the algebraic structure itself, through a
reduction reduction relation typically denoted by $\red$. Below, we
give a recursive presentation of this relation for the calculus used
in the encoding.

$\red \subseteq \pi \times \pi$
$\red : \pi \to \mathcal{P}(\pi)$

\begin{mathpar}
  \inferrule* [lab=Comm] { \textsf{match}( x_{src}, x_{trgt} ) } { x_{trgt}?(y)P \; | \; x_{src}!\langle {Q} \rangle \red P\{\quotep{Q}/y}\} }
  \and \\
  \inferrule* [lab=Par] {{P} \red {P}'} {{{P} | {Q}} \red {{P}' | {Q}}}
  \and
  \inferrule* [lab=Equiv]{{{P} \scong {P}'} \andalso {{P}' \red {Q}'} \andalso {{Q}' \scong {Q}}}{{P} \red {Q}}
\end{mathpar}

\begin{eqnarray*}
  match_{\equiv} (\quotep{P},\quotep{Q}) & := & P \equiv Q \\
  match_{\dagger}(\quotep{P},\quotep{Q}) & := & \forall R. P|Q \red^{*} R => R \red^{*} 0 \\
  match_{K}(\quotep{P},\quotep{Q}) & := & K \mbox{ for some context } K
\end{eqnarray*}

$u?(x)P | u!\langle Q \rangle \red P\{\quotep{Q}/x\}$

%We write $\wred$ for $\red^*$, and $P\red$ if $\exists Q $ such that $ P \red Q$.
We write $P\red$ if $\exists Q $ such that $ P \red Q$ and $P\not\red$, otherwise.

\section{Replication}

As mentioned before, it is known that replication (and hence
recursion) can be implemented in a higher-order process algebra
\cite{SangiorgiWalker}. As our first example of calculation with the
machinery thus far presented we give the construction explicitly in
the {\rhoc}.

\begin{eqnarray}
	D_{x} & := & \prefix{x}{y}{(\binpar{\outputp{x}{y}}{@{y}})} \nonumber\\
	\bangp_{x}{P} & := & \binpar{{x}!\langle{\binpar{D_{x}}{P}}\rangle}{D_{x}} \nonumber
\end{eqnarray}

\begin{eqnarray}
	\bangp_{x}{P} & & \nonumber\\
	=
	& {x}!\langle{(\prefix{x}{y}{(\outputp{x}{y} | @{y})) | P}}\rangle 
	      | \prefix{x}{y}{(\outputp{x}{y} | @{y})} & \nonumber\\
	\red
	& (\outputp{x}{y} | @{y})\substn{\quotep{(\prefix{x}{y}{(@{y} | \outputp{x}{y})) | P}}}{y} & \nonumber\\
	=
	& \outputp{x}{\quotep{(\prefix{x}{y}{(\outputp{x}{y} | @{y})) | P}}}
	  | {(\prefix{x}{y}{(\outputp{x}{y} | @{y})) | P}} & \nonumber\\
	\red
	& \ldots & \nonumber\\
	\red^*
	& P | P | \ldots & \nonumber
\end{eqnarray}

Of course, this encoding, as an implementation, runs away, unfolding
$\bangp{P}$ eagerly. A lazier and more implementable replication
operator, restricted to input-guarded processes, may be obtained as follows.

\begin{eqnarray}
\bangp{\prefix{u}{v}{P}} 
	:= 
	\binpar{\lift{x}{\prefix{u}{v}{(\binpar{D(x)}{P})}}}{D(x)} \nonumber
\end{eqnarray}

\begin{remark}
  Note that the lazier definition still does not deal with summation
  or mixed summation (i.e. sums over input and output). The reader is
  invited to construct definitions of replication that deal with these
  features. 

  Further, the definitions are parameterized in a name, $x$. Can you,
  gentle reader, make a definition that eliminates this parameter and
  guarantees no accidental interaction between the replication
  machinery and the process being replicated -- i.e. no accidental
  sharing of names used by the process to get its work done and the
  name(s) used by the replication to effect copying. This latter
  revision of the definition of replication is crucial to obtaining
  the expected identity $!!P \sim !P$.
\end{remark}

\begin{remark}\label{rem:paradoxical_combinator}
  The reader familiar with the lambda calculus will have noticed the
  similarity between $D$ and the paradoxical combinator.

  [Ed. note: the existence of this seems to suggest we have to be more
  restrictive on the set of processes and names we admit if we are to
  support no-cloning.]
\end{remark}

\subsubsection{Bisimulation}

The computational dynamics gives rise to another kind of equivalence,
the equivalence of computational behavior. As previously mentioned
this is typically captured \emph{via} some form of bisimulation.

% The notion we use in this paper is weak barbed bisimulation
% \cite{milner91polyadicpi}.

The notion we use in this paper is derived from weak barbed
bisimulation \cite{milner91polyadicpi}. 

\begin{definition}
An \emph{observation relation}, $\downarrow_{\mathcal N}$, over a set
of names, $\mathcal N$, is the smallest relation satisfying the rules
below.

\infrule[Out-barb]{y \in {\mathcal N}, \; x \nameeq y}
		  {\outputp{x}{v} \downarrow_{\mathcal N} x}
\infrule[Par-barb]{\mbox{$P\downarrow_{\mathcal N} x$ or $Q\downarrow_{\mathcal N} x$}}
		  {\binpar{P}{Q} \downarrow_{\mathcal N} x}

We write $P \Downarrow_{\mathcal N} x$ if there is $Q$ such that 
$P \wred Q$ and $Q \downarrow_{\mathcal N} x$.
\end{definition}

\begin{definition}
%\label{def.bbisim}
An  ${\mathcal N}$-\emph{barbed bisimulation} over a set of names, ${\mathcal N}$, is a symmetric binary relation 
${\mathcal S}_{\mathcal N}$ between agents such that $P\rel{S}_{\mathcal N}Q$ implies:
\begin{enumerate}
\item If $P \red P'$ then $Q \wred Q'$ and $P'\rel{S}_{\mathcal N} Q'$.
\item If $P\downarrow_{\mathcal N} x$, then $Q\Downarrow_{\mathcal N} x$.
\end{enumerate}
$P$ is ${\mathcal N}$-barbed bisimilar to $Q$, written
$P \wbbisim_{\mathcal N} Q$, if $P \rel{S}_{\mathcal N} Q$ for some ${\mathcal N}$-barbed bisimulation ${\mathcal S}_{\mathcal N}$.
\end{definition}

$\mathcal{R} \subseteq \pi \times \pi$

$P \mathcal{R} Q => \forall P'. P \red P' \Rightarrow \exists Q'. Q \red Q', P' \mathcal{R} Q'$

$P \vdash x \Rightarrow Q \vdash x$

\begin{mathpar}
  \inferrule*[lab=Out-barb]{x \nameeq y}{{y}!\langle{Q}\rangle \vdash x}
  \and
  \inferrule*[lab=Par-barb]{\mbox{$P\vdash x$ or $Q\vdash x$}}{\binpar{P}{Q} \vdash x}
\end{mathpar}

\subsubsection{Contexts}

One of the principle advantages of computational calculi like the
$\pi$-calculus is a well-defined notion of context,
contextual-equivalence and a correlation between
contextual-equivalence and notions of bisimulation. The notion of
context allows the decomposition of a process into (sub-)process and
its syntactic environment, its context. Thus, a context may be
thought of as a process with a ``hole'' (written $\Box$) in it. The
application of a context $M$ to a process $P$, written $M[P]$, is
tantamount to filling the hole in $M$ with $P$. In this paper we do
not need the full weight of this theory, but do make use of the notion
of context in the proof the main theorem. 

\begin{mathpar}
  \inferrule* [lab=summation] {} {{M_{M},M_{N}} \bc \Box \;|\; x.M_{A} \;|\; M_{M}+M_{N}}
  \and
  \inferrule* [lab=agent] {} {{M_{A}} \bc (\vec{x})M_{P} \;| \; \clift{P_0,\ldots,M_{P},\ldots,P_N}}
  \and \\
  \inferrule* [lab=process] {} {{M_{P}} \bc M_{N} \;| \;P|M_{P} }
\end{mathpar} 

\begin{mathpar}
  \inferrule* [lab=sychronization] {} {M_{N} \bc \Box \;|\; x?M_{F} \;|\; x!M_{C}}
  \and
  \inferrule* [lab=abstraction] {} {{M_{F}} \bc (x)M_{P} }
  \and
  \inferrule* [lab=concretion] {} {{M_{C}} \bc \langle M_{P} \rangle }
  \and \\
  \inferrule* [lab=process] {} {{M_{P}} \bc M_{N} \;| \;P|M_{P} }
\end{mathpar}

\begin{definition}[contextual application] Given a context $M$, and
  process $P$, we define the \emph{contextual application}, $M[P] :=
  M\{P/\Box\}$. That is, the contextual application of M to P is the
  substitution of $P$ for $\Box$ in $M$.
\end{definition}

$\meaningof{-} : L \to \mathcal{P}(\pi)$

\begin{mathpar}
  \inferrule* [lab=collection] {} {\meaningof{true} = \pi, \and \meaningof{~E} = \pi \setminus \meaningof{E}, \and \meaningof{E_{1} \& E_{2}} = \meaningof{E_{1}} \cap \meaningof{E_{2}}}
\end{mathpar}

\begin{mathpar}
  \inferrule* [lab=structure] {} {\meaningof{0} = \{ P \in \pi | P \equiv 0 \}, \and \\ \meaningof{E_1 | E_2} = \{ P \in \pi | P \equiv P_{1} | P_{2}, P_{1} \in \meaningof{E_{1}}, P_{2} \in \meaningof{E_2}\} }
\end{mathpar}

\begin{mathpar}
 \inferrule* [lab=behavior] {} {\meaningof{\langle a?b \rangle E} = \{ P \in \pi | P \equiv Q | u?(y)P', \\ \and \\\\ \and \\ \;\;\; u \in \meaningof{a}, \forall z.P'\{z/y\} \in \meaningof{E\{z/b\}}\}, \and \\ \meaningof{a!E} = \{ P \in \pi | P \equiv Q | x!\langle P' \rangle, x \in \meaningof{a} P' \in \meaningof{E}\} }
\end{mathpar}

\begin{mathpar}
 \inferrule* [lab=nominal] {} {\meaningof{\quotep{E}} = \{ \quotep{P} \in \quotep{\pi} | P \in \meaningof{E} \}, \and \meaningof{\quotep{P}} = \{ \quotep{Q} \in \quotep{\pi} | P \equiv Q \} \and \\ \meaningof{@\quotep{E}} = \{ P \in \pi | P \equiv @x, x \in \meaningof{E} \}}
\end{mathpar}

\begin{eqnarray*}
  \\
  \meaningof{-} : TS \to ST
\end{eqnarray*}

\begin{eqnarray*}
  \\
  L : TS \to ST
\end{eqnarray*}

\begin{eqnarray*}
  \\
  P \models E \iff P \in \meaningof{E}
\end{eqnarray*}

\begin{eqnarray*}
  P \approx_{L} Q \iff \forall E \in L. P \models E \iff Q \models E
\end{eqnarray*}

\begin{eqnarray*}
  P \approx_{K} Q
\end{eqnarray*}

\begin{eqnarray*}
  P \approx Q
\end{eqnarray*}

$\approx_{K} = \approx = \approx_{L}$

\subsubsection{Contextual duality}

Note that contexts extend the quotation operation to a family of
operations from processes to names. Given a context, $M$, we can
define a \emph{nominal context}, $\quotep{M}$ by $\quotep{M}[P] :=
\quotep{M[P]}$. To foreshadow what is to come we observe that these
operations enjoy a duality with processes very much like the duality
between vectors and maps from vectors to scalars.

Further, because the calculus is essentially higher-order, we have a
correspondence between contexts and processes. More specifically,
given a name $x$ and a context $M$ we can construct $M^{*}_{x}$ such
that 

\begin{mathpar}
  M^{*}_{x} | \lift{x}{P} \red M[P]
\end{mathpar}

namely,

\begin{mathpar}
  M^{*}_{x} := x?(u).M[\dropn{u}]
\end{mathpar}

The dependence of $M^{*}_{x}$ on a name makes it an abstraction, 

\begin{mathpar}
  M^{*} := (x)x?(u).M[\dropn{u}]
\end{mathpar}

\subsection{Additional notation}

It will sometimes be convenient to denote the process a name
quotes. We already have the notation $x = \quotep{P}$, but it will be
convenient to introduce an alternate notation, $\procn{x}$, when we
want to emphasize the connection to the use of the name. Note that, by
virtue of name equivalence, $\quotep{\procn{x}} \nameeq x$; so, the
notation is consistent with previous definitions.

Further, because names have structure it is possible to effect
substitutions on the basis of that structure. This means we need to
upgrade our notation for substitutions, which we accomplish by
adapting comprehension notation. Thus,

\begin{mathpar}
  P\{ y / x : x \in S \}
\end{mathpar}

is interpreted to mean the process derived from P by replacing (in a
capture-avoiding manner) each occurrence of $x$ in $S$ by $y$. For example,

\begin{mathpar}
  P\{ \quotep{\procn{x}|\procn{x}} / x : x \in \freenames{P} \}
\end{mathpar}

will replace each (occurrence) of a free name $x$ in $P$ by
$\quotep{\procn{x}|\procn{x}}$.

Also, we will avail ourselves of the notation $x^{L}$ and $x^{R}$ to
denote injections of a name into disjoint copies of the name
space. There are numerous ways to accomplish this. One example can be
found in \cite{MeredithR05}. This notation overloads to vectors of
names: $\vec{x}^{\pi} := (x_{i}^{\pi} \; : \; 0 \leq i < |\vec{x}| )$ where $\pi \in \{L,R\}$.

We also use $P^{\Box} := P|\Box$.

In \cite{MeredithR05} an interpretation of the new operator is
given. It turns out that there are several possible interpretations
all enjoying the requisite algebraic properties of the operator (see
\cite{milner91polyadicpi}). We will therefore make liberal use of
$(\nu\; \vec{x})P$.

% subsection the_syntax_and_semantics_of_the_notation_system (end)   

\section{Interpretation of QM}
\subsection{Supporting definitions}
\subsubsection{Multiplication}
\begin{mathpar}
  \quotep{Q} \cdot \quotep{R} := \quotep{Q|R}
  \and \\
  \quotep{Q} \cdot P := P\{ \quotep{Q|R} / \quotep{R} : \quotep{R} \in \freenames{P} \}
\end{mathpar}

\paragraph{Discussion}
The first line needs little explanation. The second line says that
each free name of the process is replaced with the multiplication of
that name by the scalar. Multiplication of a scalar (name) by a state
(process) results in a process all the names of which have been `moved
over' by parallel composition with the process the scalar
quotes. There is a subtlety that the bound names have to be
manipulated so that multiplied names aren't accidentally
captured. There are many ways to achieve this.

\begin{remark}\label{rem:multiplication_identities}
  The reader is invited to verify that for all $x,y,z \in \QProc$ and $P \in \Proc$
  \begin{mathpar}
    x \cdot \quotep{0} \equiv x 
    \and
    x \cdot y \equiv y \cdot x
    \and
    x \cdot (y \cdot z) \equiv (x \cdot y) \cdot z
    \and \\
    \quotep{0} \cdot P \equiv P
    \and \\
    x \cdot (y \cdot P) \equiv (x \cdot y) \cdot P
    \and \\
    x \cdot (P|Q) \equiv (x \cdot P) | (x \cdot Q)
    \and \\    
  \end{mathpar}
\end{remark}

\subsubsection{Tensor product}

We define a tensor product on processes by structural induction.

\paragraph{Tensor of sums} First note that all summations, including
$\pzero$ and sequence, can be written $\Sigma_{i} x_{i}.A_{i} +
\Sigma_{j} x_{j}.C_{j}$, where we have grouped input-guarded processes
together and output-guarded processes together.

Thus, we can define the tensor product of two summations, $N_{1}\otimes N_{2}$, where

\begin{mathpar}
  N_{1} := \Sigma_{i} x_{i}.A_{i} + \Sigma_{j} x_{j}.C_{j}
  \and
  N_{2} := \Sigma_{i'} y_{i'}.B_{i'} + \Sigma_{j'} y_{j'}.D_{j'} 
\end{mathpar}

as follows.

\begin{mathpar}
  \Sigma_{i} x_{i}.A_{i} + \Sigma_{j} x_{j}.C_{j} \otimes \Sigma_{i'}
  y_{i'}.B_{i'} + \Sigma_{j'} y_{j'}.D_{j'} 
  \and \\
  := \; \Sigma_{i} \Sigma_{i'} \quotep{\stackrel{\vee}{x_{i}}| \stackrel{\vee}{y_{i'}}}.(A_{i}\otimes B_{i'}) \; | \; \Sigma_{i'} \Sigma_{i} \quotep{\stackrel{\vee}{y_{i'}}|\stackrel{\vee}{x_{i}}}.(B_{i'}\otimes A_{i})
  \and
  \;\; | \;\; \Sigma_{j} \Sigma_{j'} \quotep{\stackrel{\vee}{x_{j}}|\stackrel{\vee}{y_{j'}}}.(A_{j}\otimes B_{j'}) \; | \; \Sigma_{j'} \Sigma_{j} \quotep{\stackrel{\vee}{y_{j'}}|\stackrel{\vee}{x_{j}}}.(B_{j'}\otimes A_{j})
\end{mathpar}

\begin{remark}
  Do we need to $x^{L}$ and $y^{R}$ for this construction as well?
\end{remark}

\paragraph{Tensor of parallel compositions} Next, we distribute tensor
over par.

\begin{mathpar}
  P_{1}|P_{2} \otimes Q_{1}|Q_{2} := (P_{1} \otimes Q_{1}) | (P_{1}
  \otimes Q_{2}) | (P_{2} \otimes Q_{1}) | (P_{2} \otimes Q_{2})
\end{mathpar}

\paragraph{Tensor with dropped names} We treat tensor of a
process with a dropped name as parallel composition.

\begin{mathpar}
  P \otimes \dropn{x} := P | \dropn{x}
\end{mathpar}

\paragraph{Tensor of agents}

Finally, we need to define tensor on agents. Note that the definition
of tensor on normal products only tensors inputs with inputs and
outputs with outputs. Thus, we only have to define the operation on
``homogeneous'' pairings.

\begin{mathpar}
  (\vec{x})P \otimes (\vec{y})Q
  \and \\
  := (x_{0}^{L}|y_{0}^{R},\ldots,x_{0}^{L}|y_{n}^{R},\ldots,x_{m}^{L}|y_{0}^{R},\ldots,x_{m}^{L}|y_{n}^R)(P\{ \vec{x}^{L}/\vec{x}\} \otimes Q \{ \vec{y}^{R}/\vec{y}\})
  \and \\
  \clift{\vec{P}} \otimes \clift{\vec{Q}}
  \and \\
  := \clift{P_{0}\otimes Q_{0},\ldots,P_{0}\otimes Q_{n},\ldots,P_{m}\otimes Q_{0},\ldots,P_{m}\otimes Q_{n}}
\end{mathpar}

\begin{remark}
  Observe that arities of tensored abstractions matches arities of
  tensored concretions if the original arities matched. Note also that
  the length of the arities corresponds to the increase in dimension
  we see in ordinary vector space tensor product.
\end{remark}

\begin{remark}
  Operationally, this definition distributes the tensor down to
  components ``linked'' by summation. Tensor over summation is
  intriguing in that it mixes names. Moreover, as a consequence of the
  way it mixes names we have the identities for all $x \in \QProc$ and
  $P,Q \in \Proc$

  \begin{mathpar}
    (x \cdot P) \otimes Q \equiv x \cdot (P \otimes Q) \equiv P \otimes (x \cdot Q)
    \and
    P \otimes \pzero \equiv P
  \end{mathpar}

  that the reader is invited to verify.
\end{remark}

\subsubsection{Annihilation}
\begin{mathpar}
  P^{\perp} := \{ Q | \forall R. P|Q \red^{*} R \Rightarrow R \red^{*} \pzero \}
  \and \\
  P^{\underline{\perp}} := \Sigma_{Q \in P^{\perp}} \quotep{Q}?(y).(\dropn{y}|Q) | \Sigma_{Q \in P^{\perp}} \quotep{Q}\clift{\Box}
\end{mathpar}

\paragraph{Discussion} The reader will note that $P^{\perp}$ is a
\emph{set} of processes, while $P^{\underline{\perp}}$ is a
\emph{context}. We call the set $P^{\perp}$ the \emph{annihilators} of
$P$. The parallel composition of a process in the annihilators of $P$
with $P$ will result in a process, the state space of which has all
paths eventually leading to $\pzero$. Execution may endure loops; but
under reasonable conditions of fairness (naturally guaranteed under
most notions of bisimulation) such a composite process cannot get
stuck in such a loop and will, eventually pop out and terminate.

The context $P^{\underline{\perp}}$ is ready and willing to ``take the
$P$ out of'' the process to which it is applied. It will effectively
transmit the code of the process to which it is applied to one of the
annihilators and run the process against it.

\subsubsection{Evaluation}
We fix $M$ a domain of fully abstract interpretation with an equality
coincident with bisimulation. We take $\meaningof{\cdot} : \Proc \to
M$ to be the map interpreting processes and $\nmeaningof{\cdot} : \M
\to Proc$ to be the map running the other way. Then we define

\begin{mathpar}
  \int P := \nmeaningof{\meaningof{P}}
\end{mathpar}

\paragraph{Discussion}
There are many fully abstract interpretations of Milner's
$\pi$-calculus. Any of them can be used as a basis for interpreting
the reflective calculus here. Equipped with such a domain it is
largely a matter of grinding through to check that the Yoneda
construction for the normalization-by-evaluation program can be
extended to this setting.

\begin{remark}
  The reader is invited to verify that $\int (P^{\underline{\perp}}[P]) = 0$.
\end{remark}

\subsection{Quantum mechanics}

Table \ref{tbl:core_qm_op_defns} gives the core operational definitions

\begin{table}[htp]\label{tbl:core_qm_op_defns}
  \center{
    \fbox{
      \begin{tabular}{c|c}
        quantum mechanics & process calculus \\
        \hline
        scalar & $x := \quotep{P}$ \\
        state vector & $\state{P} := P$ \\
        dual & $\state{P}^{*} := \event{P^{\underline{\perp}}} := \quotep{P^{\underline{\perp}}}[-]$ \\
        matrix & $ \Sigma_{\alpha} \state{P_{\alpha}}x_{\alpha}\event{Q_{\alpha}}$ \\
        vector addition & $\state{P} + \state{Q} := \state{P | Q}$ \\
        tensor product & $\state{P} \otimes \state{Q} := \state{P \otimes Q}$ \\
        inner product & $\innerprod{P}{Q} := \quotep{\int P^{\underline{\perp}}[Q]}$ \\
      \end{tabular}
    }
  }
  \caption{QM - operational definitions}
\end{table}

where

\begin{mathpar}
  \prmatrix{P}{Q} := \fprmatrix{P}{\quotep{\pzero}}{Q}
  \and
  \fprmatrix{P}{x}{Q} := (\state{P},x,\event{Q})
  \and
  (\fprmatrix{P}{x}{Q})(\state{R}) := x \cdot \innerprod{Q}{R} \cdot \state{P}
  \and
  (\fprmatrix{P}{x}{Q})(\event{R}) := x \cdot \innerprod{R}{P} \cdot \event{Q}
\end{mathpar}

\paragraph{Discussion}
As promised: vectors (aka states) are represented as processes; duals
as contextual duals; inner product definition should be compared with
standard inner product definition for ....

\begin{remark}
  Assuming $\int (P^{\underline{\perp}}[P]) = 0$, the reader is
  invited to verify that $(\fprmatrix{P}{x}{P})(\state{P}) = x \cdot \state{P}$.
\end{remark}

\begin{remark}
  The reader is invited to verify that $\innerprod{P}{Q}$ could
  equally well have been written $\quotep{\int \stackrel{\vee}{x}}$
  where $x = \event{P^{\underline{\perp}}}(Q)$.

  One of the motivations for this remark is that there is another way
  to factor these operations. We could package up evaluation in the dual:

  \begin{mathpar}
    \state{P}^{*} := \event{\int P^{\underline{\perp}}} := \quotep{\int P^{\underline{\perp}}}[-]
  \end{mathpar}

  and then have inner product defined by
  
  \begin{mathpar}
    \innerprod{P}{Q} := \event{P}(Q)
  \end{mathpar}

  Hopefully, experience with the calculations will provide guidance on
  the best factoring.
\end{remark}

\begin{remark}
  Assuming $\int (P^{\underline{\perp}}[P]) = 0$, the reader is
  invited to verify that $\forall P,Q. (\prmatrix{0}{Q})(\state{0}) =
  \state{0}$ and dually $(\prmatrix{P}{0})(\event{0}) = \event{0}$.
\end{remark}

\begin{remark}
  i'm a little worried that i don't (yet) have proper support for
  complex conjugacy. But, the observation above may give us a
  clue. According to Abramsky, it must be the case that the scalars
  are iso to the homset of the identity for the tensor -- which the
  observation above characterizes. 

  For now, we will simply bookmark the notion with $\overline{x}$.
\end{remark}

\subsubsection{Adjointness}

We need to give a definition of $(\cdot)^{\dagger}$ for matrices. The
obvious candidate definition is
\begin{mathpar}
(\Sigma_{\alpha}\fprmatrix{P_{\alpha}}{x_{\alpha}}{Q_{\alpha}})^{\dagger}
= \Sigma_{\alpha}\fprmatrix{(Q_{\alpha}^{\underline{\perp}})^{*}}{\overline{x}_{\alpha}}{P_{\alpha}^{\underline{\perp}}} 
\end{mathpar}

But, $(Q_{\alpha}^{\underline{\perp}})^{*}$ requires a name along
which to communicate the process to achieve the context application.

\subsubsection{Basis for a basis}
If processes label states and ``addition'' of states (a.k.a. vector
addition) is interpreted as parallel composition, what corresponds to
notions of linear independence and basis? Here, we recall that Yoshida
has developed a set of \emph{combinators} for an asynchronous verison
of Milner's $\pi$-calculus. These are a finite set of processes such
any process can be expressed as parallel composition of these
combinators together with liberal uses of the new operator and
replication. We can simply give a translation of these into the
present calculus and have reasonable expectation that the property
carries over. That is, that the resultant set allows to express all
processes via parallel composition. Note, however, that there is no
new operator or replication in this calculus. As a result, we expect
that the corresponding set is actually infinite. That is, we expect
that the space is actually infinite dimensional.

\begin{remark}
  The attentive reader may be a bit concerned. Certainly, the
  collection $S$, $K$ and $I$ is a finite set of
  combinators. Shouldn't we expect to see a finite set of combinators
  for an effectively equivalent system? i am very sympathetic to this
  critique and feel it warrants full attention. On the other hand, i
  also have in mind the following analogy. The natural numbers, as a
  monoid under addition, has exactly $1$ generator, while the natural
  numbers, as a monoid under multiplication, has countably many
  generators (the primes). We observe that the application of the
  lambda calculus is much less resource sensitive than the parallel
  composition of the $\pi$-calculus. Could it be the case that we have
  an analogy of the form
  
  \begin{mathpar}
    m + n : MN :: m*n : M|N
  \end{mathpar}

  giving a similar blow up in the set of ``primes''?  This is such a
  wonderful thought that, even if it's not true, i think it's worth
  writing down.
\end{remark}
 

\documentclass[12pt]{llncs}
%\documentclass{jktr}

\usepackage[pdftex]{hyperref}                   
\usepackage {listings}
\usepackage {mathpartir}
\usepackage{bcprules}
%\usepackage{listings}
                       
\usepackage{graphicx} 
%\usepackage[margins=2.5cm,nohead,nofoot]{geometry}
%\usepackage{geometry}
\usepackage{amsfonts}
\usepackage{amstext}
\usepackage{latexsym}
\usepackage{amssymb}
\usepackage{color}


%\include{myPreamble}
\include{qm2pi.local} 

%\ifpdf
%\usepackage[pdftex]{graphicx}
%\else
%\usepackage{graphicx}
%\fi

 % \ifpdf
%  \usepackage{pdfsync}
%  \if


%\title{Brief Article}
%\author{David F. Snyder}
%\author{L.G. Meredith}

%\address{Dept. of Math., Texas State University--San Marcos, San Marcos, TX 78666}
       
\pagestyle{empty}


\begin{document}

\lstset{language=[Objective]Caml,frame=shadowbox}

\input{qm2pi.front}

% section front matter (end)

\input{qm2pi.intro} 
 
% section introduction (end)

% \input{qm2pi.knotations} 

% section notation (end)

\input{qm2pi.process.calculi} 

% section concurrent_process_calculi_and_spatial_logics_ (end)
    
%\input{qm2pi.knots2pi} 

%\input{qm2pi.trefoil} 

%\input{qm2pi.mainthm} 

% subsection basic_interpretation (end)

%\input{qm2pi.rho.presentation} 
\subsection{The syntax and semantics of the notation system}\label{sub:the_syntax_and_semantics_of_the_notation_system} % (fold)

We now summarize a technical presentation of the calculus that
embodies our theory of dynamics. The typical presentation of such a
calculus follows the style of giving generators and relations on
them. The grammar, below, describing term constructors, freely
generates the set of processes, $\Proc$. This set is then quotiented
by a relation known as structural congruence and it is over this set
that the notion of dynamics is expressed. This presentation is
essentially that of \cite{MeredithR05} with the addition of
polyadicity and summation. For readability we have relegated some of
the technical subtleties to an appendix.

\subsubsection{Process grammar}\label{subsub:process_grammar}

\begin{mathpar}
  \inferrule* [lab=synchronization] {} {{M} \bc \pzero \;|\; x?F \;|\; x!C }
  \and
  \inferrule* [lab=abstraction] {} {{F} \bc (x)P}
  \and
  \inferrule* [lab=concretion] {} {{C} \bc \langle Q \rangle}
  \and
  \inferrule* [lab=process] {} {{P,Q} \bc M \;| \;P|Q \;|\; @{x}}
  \and
  \inferrule* [lab=name] {} {{x} \bc \quotep{P}}
\end{mathpar} 

Note that $\vec{x}$ (resp. $\vec{P}$) denotes a vector of names
(resp. processes) of length $|\vec{x}|$ (resp. $|\vec{P}|$). We adopt
the following useful abbreviations.

\begin{mathpar}
   x?(\vec{y}).P := x.(\vec{y})P \and  x\clift{\vec{P}} := x.\clift{\vec{P}}
   \and x!(y) := \lift{x}{\dropn{y}}
   \and \Pi_{i=0}^{n-1}P_i := P_0 | \ldots | P_{n-1}
\end{mathpar}

\subsubsection{Structural congruence}

\paragraph{Free and bound names and alpha-equivalence.} At the
core of structural equivalence is alpha-equivalence which identifies
process that are the same up to a change of variable. Formally, we
recognize the distinction between free and bound names. The free names
of a process, $\freenames{P}$, may be calculated recursively as
follows:

\begin{mathpar}
\freenames{\pzero} := \emptyset
  \and \\
  \freenames{x?(y).P} := \{ x \} \cup (\freenames{P} \setminus \{ y \})
  \and 
  \freenames{x!\langle P \rangle} := \{ x \} \cup \{ P \} 
  \and \\
  \freenames{P|Q} := \freenames{P} \cup \freenames{Q}
  \and \\
  \freenames{@{x}} := \{ x \}
\end{mathpar}

$\pi$
$\quotep{\pi}$

$\freenames{-} : \pi \to \mathcal{P}(\quotep{\pi})$

\begin{eqnarray*}
  \freenames{\pzero} & := & \emptyset \\
  \freenames{x?(y).P} & := & \{ x \} \cup (\freenames{P} \setminus \{ y \}) \\
  \freenames{x!\langle P \rangle} & := & \{ x \} \cup \{ P \} \\
  \freenames{P|Q} & := & \freenames{P} \cup \freenames{Q} \\
  \freenames{\dropn{x}} & := & \{ x \}
\end{eqnarray*}

The bound names of a process, $\boundnames{P}$, are those names occurring in $P$
that are not free. For example, in $x?(y).0$, the name $x$ is free, while $y$ is bound.

\begin{mathpar}
  \inferrule* [lab=monoidal-laws] {} { P|Q \equiv Q|P \and P|0 \equiv P \and P|(Q|R) \equiv (P|Q)|R }
\end{mathpar}

\begin{mathpar}
  \inferrule* [lab=alpha-equivalence] {} { (x)P \equiv (y)P\{y/x\} \and y \not\in \freenames{P} }
\end{mathpar}

\begin{definition}
Then two processes, $P,Q$, are alpha-equivalent if $P = Q\{\vec{y}/\vec{x}\}$ for
some $\vec{x} \in \boundnames{Q},\vec{y} \in \boundnames{P}$, where $Q\{\vec{y}/\vec{x}\}$
denotes the capture-avoiding substitution of $\vec{y}$ for $\vec{x}$ in $Q$.
\end{definition}

\begin{definition}
  The {\em structural congruence} \cite{SangiorgiWalker} , $\equiv$,
  between processes is the least congruence containing
  alpha-equivalence, satisfying the abelian monoid laws
  (associativity, commutativity and $\pzero$ as identity) for parallel
  composition $|$ and for summation $+$.
\end{definition}

\subsection{Name equivalence}

We take name equivalence, written $\nameeq$, to be the smallest
equivalence relation generated by the following rules.

\begin{mathpar}
\inferrule*[lab=Quote-drop]
{ }
{ \quotep{@{x}} \nameeq x }

\inferrule*[lab=Struct-equiv]
{ P \scong Q }
{ \quotep{P} \nameeq \quotep{Q} }
\end{mathpar}

The astute reader will have noticed that the mutual recursion of names
and processes imposes a mutual recursion on alpha-equivalence and
structural equivalence via name-equivalence. Fortunately, all of this
works out pleasantly and we may calculate in the natural way, free of
concern. The reader interested in the details is referred to the
appendix \ref{appendix:rho_details}.

\subsection{Substitution}

We use $\Proc$ for the set of processes, $\QProc$ for the set of
names, and $\id{\{}\vec{y} / \vec{x} \id{\}}$ to denote partial maps,
$s : \QProc \rightarrow \QProc$. A map, $s$ lifts, uniquely, to a map
on process terms, $\widehat{s} : \Proc \rightarrow \Proc$ by the
following equations.

\begin{mathpar}
  (0) \psubstp{Q}{P} := 0 \\
  (R \juxtap S) \psubstp{Q}{P}
  :=    
  (R)\psubstp{Q}{P} \juxtap (S) \psubstp{Q}{P} \\
  (x?(y).R) \psubstp{Q}{P}    
  :=    
  (x)\substp{Q}{P} (z)\concat( (R \psubstn{z}{y}) \psubstp{Q}{P} ) \\
  (\lift{x}{R}) \psubstp{Q}{P}  
  :=
  \lift{(x)\substp{Q}{P}}{ R \psubstp{Q}{P} } \\
%   (\dropn{x})  \psubstp{Q}{P}       
%   := 
%   \left\{ 
%     \begin{array}{ccc} 
%       \dropn{\quotep{Q}} & & x \nameeq \quotep{P} \\
%       \dropn{x} & & otherwise \\
%     \end{array}
%   \right. 
  (\dropn{x})  \psubstp{Q}{P}       
  := 
  \left\{ 
    \begin{array}{ccc} 
      Q & & x \nameeq \quotep{P} \\
      \dropn{x} & & otherwise \\
    \end{array}
  \right.
\end{mathpar}
 

where

\begin{eqnarray}
  (x)\id{\{} \lpquote Q \rpquote / \lpquote P \rpquote \id{\}}            = 
  \left\{ 
    \begin{array}{ccc}
      \lpquote Q \rpquote & & x \nameeq \lpquote P \rpquote \\
      x & & otherwise \\
    \end{array}
  \right. \nonumber
\end{eqnarray}

and $z$ is chosen distinct from $\quotep{P}$, $\quotep{Q}$, the free
names in $Q$, and all the names in $R$. Our $\alpha$-equivalence will
be built in the standard way from this substitution.

\begin{remark}\label{rem:no_self_referential_names}
  One consequence of these definitions is that $\forall P. \quotep{P}
  \not\in \freenames{P}$.
\end{remark}

\subsection{ Dynamic quote: an example }

Anticipating something of what's to come, consider applying the
substitution, $\widehat{\id{\{}u / z \id{\}}}$, to the following pair
of processes, $\lift{w}{y!(z)}$ and $w[ \lpquote y!(z) \rpquote ]$.

\begin{eqnarray}
	\lift{w}{y!(z)}\widehat{\id{\{}u / z \id{\}}}
		& = &
		\lift{w}{y!(u)} \nonumber\\
	w[ \lpquote y!(z) \rpquote ] \widehat{ \id{\{}u / z \id{\}} }
		& = &
		w[ \lpquote y!(z) \rpquote ] \nonumber
\end{eqnarray}

Because the body of the process between quotes is impervious to
substitution, we get radically different answers. In fact, by
examining the first process in an input context,
e.g. $x?(z).\lift{w}{y!(z)}$, we see that the process under the lift
operator may be shaped by prefixed inputs binding a name inside it. In
this sense, the lift operator will be seen as a way to dynamically
construct processes before reifying them as names.

Finally equipped with these standard features we can present the
dynamics of the calculus.

\subsubsection{Operational semantics} 

Finally, we introduce the computational dynamics. What marks these
algebras as distinct from other more traditionally studied algebraic
structures, e.g. vector spaces or polynomial rings, is the manner in
which dynamics is captured. In traditional structures, dynamics is typically
expressed through morphisms between such structures, as in linear maps
between vector spaces or morphisms between rings. In algebras
associated with the semantics of computation, the dynamics is
expressed as part of the algebraic structure itself, through a
reduction reduction relation typically denoted by $\red$. Below, we
give a recursive presentation of this relation for the calculus used
in the encoding.

$\red \subseteq \pi \times \pi$
$\red : \pi \to \mathcal{P}(\pi)$

\begin{mathpar}
  \inferrule* [lab=Comm] { \textsf{match}( x_{src}, x_{trgt} ) } { x_{trgt}?(y)P \; | \; x_{src}!\langle {Q} \rangle \red P\{\quotep{Q}/y}\} }
  \and \\
  \inferrule* [lab=Par] {{P} \red {P}'} {{{P} | {Q}} \red {{P}' | {Q}}}
  \and
  \inferrule* [lab=Equiv]{{{P} \scong {P}'} \andalso {{P}' \red {Q}'} \andalso {{Q}' \scong {Q}}}{{P} \red {Q}}
\end{mathpar}

\begin{eqnarray*}
  match_{\equiv} (\quotep{P},\quotep{Q}) & := & P \equiv Q \\
  match_{\dagger}(\quotep{P},\quotep{Q}) & := & \forall R. P|Q \red^{*} R => R \red^{*} 0 \\
  match_{K}(\quotep{P},\quotep{Q}) & := & K \mbox{ for some context } K
\end{eqnarray*}

$u?(x)P | u!\langle Q \rangle \red P\{\quotep{Q}/x\}$

%We write $\wred$ for $\red^*$, and $P\red$ if $\exists Q $ such that $ P \red Q$.
We write $P\red$ if $\exists Q $ such that $ P \red Q$ and $P\not\red$, otherwise.

\section{Replication}

As mentioned before, it is known that replication (and hence
recursion) can be implemented in a higher-order process algebra
\cite{SangiorgiWalker}. As our first example of calculation with the
machinery thus far presented we give the construction explicitly in
the {\rhoc}.

\begin{eqnarray}
	D_{x} & := & \prefix{x}{y}{(\binpar{\outputp{x}{y}}{@{y}})} \nonumber\\
	\bangp_{x}{P} & := & \binpar{{x}!\langle{\binpar{D_{x}}{P}}\rangle}{D_{x}} \nonumber
\end{eqnarray}

\begin{eqnarray}
	\bangp_{x}{P} & & \nonumber\\
	=
	& {x}!\langle{(\prefix{x}{y}{(\outputp{x}{y} | @{y})) | P}}\rangle 
	      | \prefix{x}{y}{(\outputp{x}{y} | @{y})} & \nonumber\\
	\red
	& (\outputp{x}{y} | @{y})\substn{\quotep{(\prefix{x}{y}{(@{y} | \outputp{x}{y})) | P}}}{y} & \nonumber\\
	=
	& \outputp{x}{\quotep{(\prefix{x}{y}{(\outputp{x}{y} | @{y})) | P}}}
	  | {(\prefix{x}{y}{(\outputp{x}{y} | @{y})) | P}} & \nonumber\\
	\red
	& \ldots & \nonumber\\
	\red^*
	& P | P | \ldots & \nonumber
\end{eqnarray}

Of course, this encoding, as an implementation, runs away, unfolding
$\bangp{P}$ eagerly. A lazier and more implementable replication
operator, restricted to input-guarded processes, may be obtained as follows.

\begin{eqnarray}
\bangp{\prefix{u}{v}{P}} 
	:= 
	\binpar{\lift{x}{\prefix{u}{v}{(\binpar{D(x)}{P})}}}{D(x)} \nonumber
\end{eqnarray}

\begin{remark}
  Note that the lazier definition still does not deal with summation
  or mixed summation (i.e. sums over input and output). The reader is
  invited to construct definitions of replication that deal with these
  features. 

  Further, the definitions are parameterized in a name, $x$. Can you,
  gentle reader, make a definition that eliminates this parameter and
  guarantees no accidental interaction between the replication
  machinery and the process being replicated -- i.e. no accidental
  sharing of names used by the process to get its work done and the
  name(s) used by the replication to effect copying. This latter
  revision of the definition of replication is crucial to obtaining
  the expected identity $!!P \sim !P$.
\end{remark}

\begin{remark}\label{rem:paradoxical_combinator}
  The reader familiar with the lambda calculus will have noticed the
  similarity between $D$ and the paradoxical combinator.

  [Ed. note: the existence of this seems to suggest we have to be more
  restrictive on the set of processes and names we admit if we are to
  support no-cloning.]
\end{remark}

\subsubsection{Bisimulation}

The computational dynamics gives rise to another kind of equivalence,
the equivalence of computational behavior. As previously mentioned
this is typically captured \emph{via} some form of bisimulation.

% The notion we use in this paper is weak barbed bisimulation
% \cite{milner91polyadicpi}.

The notion we use in this paper is derived from weak barbed
bisimulation \cite{milner91polyadicpi}. 

\begin{definition}
An \emph{observation relation}, $\downarrow_{\mathcal N}$, over a set
of names, $\mathcal N$, is the smallest relation satisfying the rules
below.

\infrule[Out-barb]{y \in {\mathcal N}, \; x \nameeq y}
		  {\outputp{x}{v} \downarrow_{\mathcal N} x}
\infrule[Par-barb]{\mbox{$P\downarrow_{\mathcal N} x$ or $Q\downarrow_{\mathcal N} x$}}
		  {\binpar{P}{Q} \downarrow_{\mathcal N} x}

We write $P \Downarrow_{\mathcal N} x$ if there is $Q$ such that 
$P \wred Q$ and $Q \downarrow_{\mathcal N} x$.
\end{definition}

\begin{definition}
%\label{def.bbisim}
An  ${\mathcal N}$-\emph{barbed bisimulation} over a set of names, ${\mathcal N}$, is a symmetric binary relation 
${\mathcal S}_{\mathcal N}$ between agents such that $P\rel{S}_{\mathcal N}Q$ implies:
\begin{enumerate}
\item If $P \red P'$ then $Q \wred Q'$ and $P'\rel{S}_{\mathcal N} Q'$.
\item If $P\downarrow_{\mathcal N} x$, then $Q\Downarrow_{\mathcal N} x$.
\end{enumerate}
$P$ is ${\mathcal N}$-barbed bisimilar to $Q$, written
$P \wbbisim_{\mathcal N} Q$, if $P \rel{S}_{\mathcal N} Q$ for some ${\mathcal N}$-barbed bisimulation ${\mathcal S}_{\mathcal N}$.
\end{definition}

$\mathcal{R} \subseteq \pi \times \pi$

$P \mathcal{R} Q => \forall P'. P \red P' \Rightarrow \exists Q'. Q \red Q', P' \mathcal{R} Q'$

$P \vdash x \Rightarrow Q \vdash x$

\begin{mathpar}
  \inferrule*[lab=Out-barb]{x \nameeq y}{{y}!\langle{Q}\rangle \vdash x}
  \and
  \inferrule*[lab=Par-barb]{\mbox{$P\vdash x$ or $Q\vdash x$}}{\binpar{P}{Q} \vdash x}
\end{mathpar}

\subsubsection{Contexts}

One of the principle advantages of computational calculi like the
$\pi$-calculus is a well-defined notion of context,
contextual-equivalence and a correlation between
contextual-equivalence and notions of bisimulation. The notion of
context allows the decomposition of a process into (sub-)process and
its syntactic environment, its context. Thus, a context may be
thought of as a process with a ``hole'' (written $\Box$) in it. The
application of a context $M$ to a process $P$, written $M[P]$, is
tantamount to filling the hole in $M$ with $P$. In this paper we do
not need the full weight of this theory, but do make use of the notion
of context in the proof the main theorem. 

\begin{mathpar}
  \inferrule* [lab=summation] {} {{M_{M},M_{N}} \bc \Box \;|\; x.M_{A} \;|\; M_{M}+M_{N}}
  \and
  \inferrule* [lab=agent] {} {{M_{A}} \bc (\vec{x})M_{P} \;| \; \clift{P_0,\ldots,M_{P},\ldots,P_N}}
  \and \\
  \inferrule* [lab=process] {} {{M_{P}} \bc M_{N} \;| \;P|M_{P} }
\end{mathpar} 

\begin{mathpar}
  \inferrule* [lab=sychronization] {} {M_{N} \bc \Box \;|\; x?M_{F} \;|\; x!M_{C}}
  \and
  \inferrule* [lab=abstraction] {} {{M_{F}} \bc (x)M_{P} }
  \and
  \inferrule* [lab=concretion] {} {{M_{C}} \bc \langle M_{P} \rangle }
  \and \\
  \inferrule* [lab=process] {} {{M_{P}} \bc M_{N} \;| \;P|M_{P} }
\end{mathpar}

\begin{definition}[contextual application] Given a context $M$, and
  process $P$, we define the \emph{contextual application}, $M[P] :=
  M\{P/\Box\}$. That is, the contextual application of M to P is the
  substitution of $P$ for $\Box$ in $M$.
\end{definition}

$\meaningof{-} : L \to \mathcal{P}(\pi)$

\begin{mathpar}
  \inferrule* [lab=collection] {} {\meaningof{true} = \pi, \and \meaningof{~E} = \pi \setminus \meaningof{E}, \and \meaningof{E_{1} \& E_{2}} = \meaningof{E_{1}} \cap \meaningof{E_{2}}}
\end{mathpar}

\begin{mathpar}
  \inferrule* [lab=structure] {} {\meaningof{0} = \{ P \in \pi | P \equiv 0 \}, \and \\ \meaningof{E_1 | E_2} = \{ P \in \pi | P \equiv P_{1} | P_{2}, P_{1} \in \meaningof{E_{1}}, P_{2} \in \meaningof{E_2}\} }
\end{mathpar}

\begin{mathpar}
 \inferrule* [lab=behavior] {} {\meaningof{\langle a?b \rangle E} = \{ P \in \pi | P \equiv Q | u?(y)P', \\ \and \\\\ \and \\ \;\;\; u \in \meaningof{a}, \forall z.P'\{z/y\} \in \meaningof{E\{z/b\}}\}, \and \\ \meaningof{a!E} = \{ P \in \pi | P \equiv Q | x!\langle P' \rangle, x \in \meaningof{a} P' \in \meaningof{E}\} }
\end{mathpar}

\begin{mathpar}
 \inferrule* [lab=nominal] {} {\meaningof{\quotep{E}} = \{ \quotep{P} \in \quotep{\pi} | P \in \meaningof{E} \}, \and \meaningof{\quotep{P}} = \{ \quotep{Q} \in \quotep{\pi} | P \equiv Q \} \and \\ \meaningof{@\quotep{E}} = \{ P \in \pi | P \equiv @x, x \in \meaningof{E} \}}
\end{mathpar}

\begin{eqnarray*}
  \\
  \meaningof{-} : TS \to ST
\end{eqnarray*}

\begin{eqnarray*}
  \\
  L : TS \to ST
\end{eqnarray*}

\begin{eqnarray*}
  \\
  P \models E \iff P \in \meaningof{E}
\end{eqnarray*}

\begin{eqnarray*}
  P \approx_{L} Q \iff \forall E \in L. P \models E \iff Q \models E
\end{eqnarray*}

\begin{eqnarray*}
  P \approx_{K} Q
\end{eqnarray*}

\begin{eqnarray*}
  P \approx Q
\end{eqnarray*}

$\approx_{K} = \approx = \approx_{L}$

\subsubsection{Contextual duality}

Note that contexts extend the quotation operation to a family of
operations from processes to names. Given a context, $M$, we can
define a \emph{nominal context}, $\quotep{M}$ by $\quotep{M}[P] :=
\quotep{M[P]}$. To foreshadow what is to come we observe that these
operations enjoy a duality with processes very much like the duality
between vectors and maps from vectors to scalars.

Further, because the calculus is essentially higher-order, we have a
correspondence between contexts and processes. More specifically,
given a name $x$ and a context $M$ we can construct $M^{*}_{x}$ such
that 

\begin{mathpar}
  M^{*}_{x} | \lift{x}{P} \red M[P]
\end{mathpar}

namely,

\begin{mathpar}
  M^{*}_{x} := x?(u).M[\dropn{u}]
\end{mathpar}

The dependence of $M^{*}_{x}$ on a name makes it an abstraction, 

\begin{mathpar}
  M^{*} := (x)x?(u).M[\dropn{u}]
\end{mathpar}

\subsection{Additional notation}

It will sometimes be convenient to denote the process a name
quotes. We already have the notation $x = \quotep{P}$, but it will be
convenient to introduce an alternate notation, $\procn{x}$, when we
want to emphasize the connection to the use of the name. Note that, by
virtue of name equivalence, $\quotep{\procn{x}} \nameeq x$; so, the
notation is consistent with previous definitions.

Further, because names have structure it is possible to effect
substitutions on the basis of that structure. This means we need to
upgrade our notation for substitutions, which we accomplish by
adapting comprehension notation. Thus,

\begin{mathpar}
  P\{ y / x : x \in S \}
\end{mathpar}

is interpreted to mean the process derived from P by replacing (in a
capture-avoiding manner) each occurrence of $x$ in $S$ by $y$. For example,

\begin{mathpar}
  P\{ \quotep{\procn{x}|\procn{x}} / x : x \in \freenames{P} \}
\end{mathpar}

will replace each (occurrence) of a free name $x$ in $P$ by
$\quotep{\procn{x}|\procn{x}}$.

Also, we will avail ourselves of the notation $x^{L}$ and $x^{R}$ to
denote injections of a name into disjoint copies of the name
space. There are numerous ways to accomplish this. One example can be
found in \cite{MeredithR05}. This notation overloads to vectors of
names: $\vec{x}^{\pi} := (x_{i}^{\pi} \; : \; 0 \leq i < |\vec{x}| )$ where $\pi \in \{L,R\}$.

We also use $P^{\Box} := P|\Box$.

In \cite{MeredithR05} an interpretation of the new operator is
given. It turns out that there are several possible interpretations
all enjoying the requisite algebraic properties of the operator (see
\cite{milner91polyadicpi}). We will therefore make liberal use of
$(\nu\; \vec{x})P$.

% subsection the_syntax_and_semantics_of_the_notation_system (end)   

\input{qm2pi.qmops} 

\input{qm2pi.sterngerlach} 

\input{qm2pi.metric} 

% section concurrent_process_calculi (end)

%\input{qm2pi.proofsketch}

% section proof sketch (end)

%\input{qm2pi.slviaknots} 

% section spatial logic via knots (end)

\input{qm2pi.conclusion}

% section conclusion (end)

%\input{qm2pi.dtcodes} 

% section wiring algorithm (end)

\input{qm2pi.ack} 

% section acknowledgments (end)

\newpage


\bibliographystyle{plain}   
\bibliography{../../biblios/main.bib}

\input{qm2pi.rhodetails}

\end{document}

 

\documentclass[12pt]{llncs}
%\documentclass{jktr}

\usepackage[pdftex]{hyperref}                   
\usepackage {listings}
\usepackage {mathpartir}
\usepackage{bcprules}
%\usepackage{listings}
                       
\usepackage{graphicx} 
%\usepackage[margins=2.5cm,nohead,nofoot]{geometry}
%\usepackage{geometry}
\usepackage{amsfonts}
\usepackage{amstext}
\usepackage{latexsym}
\usepackage{amssymb}
\usepackage{color}


%\include{myPreamble}
\include{qm2pi.local} 

%\ifpdf
%\usepackage[pdftex]{graphicx}
%\else
%\usepackage{graphicx}
%\fi

 % \ifpdf
%  \usepackage{pdfsync}
%  \if


%\title{Brief Article}
%\author{David F. Snyder}
%\author{L.G. Meredith}

%\address{Dept. of Math., Texas State University--San Marcos, San Marcos, TX 78666}
       
\pagestyle{empty}


\begin{document}

\lstset{language=[Objective]Caml,frame=shadowbox}

\input{qm2pi.front}

% section front matter (end)

\input{qm2pi.intro} 
 
% section introduction (end)

% \input{qm2pi.knotations} 

% section notation (end)

\input{qm2pi.process.calculi} 

% section concurrent_process_calculi_and_spatial_logics_ (end)
    
%\input{qm2pi.knots2pi} 

%\input{qm2pi.trefoil} 

%\input{qm2pi.mainthm} 

% subsection basic_interpretation (end)

%\input{qm2pi.rho.presentation} 
\subsection{The syntax and semantics of the notation system}\label{sub:the_syntax_and_semantics_of_the_notation_system} % (fold)

We now summarize a technical presentation of the calculus that
embodies our theory of dynamics. The typical presentation of such a
calculus follows the style of giving generators and relations on
them. The grammar, below, describing term constructors, freely
generates the set of processes, $\Proc$. This set is then quotiented
by a relation known as structural congruence and it is over this set
that the notion of dynamics is expressed. This presentation is
essentially that of \cite{MeredithR05} with the addition of
polyadicity and summation. For readability we have relegated some of
the technical subtleties to an appendix.

\subsubsection{Process grammar}\label{subsub:process_grammar}

\begin{mathpar}
  \inferrule* [lab=synchronization] {} {{M} \bc \pzero \;|\; x?F \;|\; x!C }
  \and
  \inferrule* [lab=abstraction] {} {{F} \bc (x)P}
  \and
  \inferrule* [lab=concretion] {} {{C} \bc \langle Q \rangle}
  \and
  \inferrule* [lab=process] {} {{P,Q} \bc M \;| \;P|Q \;|\; @{x}}
  \and
  \inferrule* [lab=name] {} {{x} \bc \quotep{P}}
\end{mathpar} 

Note that $\vec{x}$ (resp. $\vec{P}$) denotes a vector of names
(resp. processes) of length $|\vec{x}|$ (resp. $|\vec{P}|$). We adopt
the following useful abbreviations.

\begin{mathpar}
   x?(\vec{y}).P := x.(\vec{y})P \and  x\clift{\vec{P}} := x.\clift{\vec{P}}
   \and x!(y) := \lift{x}{\dropn{y}}
   \and \Pi_{i=0}^{n-1}P_i := P_0 | \ldots | P_{n-1}
\end{mathpar}

\subsubsection{Structural congruence}

\paragraph{Free and bound names and alpha-equivalence.} At the
core of structural equivalence is alpha-equivalence which identifies
process that are the same up to a change of variable. Formally, we
recognize the distinction between free and bound names. The free names
of a process, $\freenames{P}$, may be calculated recursively as
follows:

\begin{mathpar}
\freenames{\pzero} := \emptyset
  \and \\
  \freenames{x?(y).P} := \{ x \} \cup (\freenames{P} \setminus \{ y \})
  \and 
  \freenames{x!\langle P \rangle} := \{ x \} \cup \{ P \} 
  \and \\
  \freenames{P|Q} := \freenames{P} \cup \freenames{Q}
  \and \\
  \freenames{@{x}} := \{ x \}
\end{mathpar}

$\pi$
$\quotep{\pi}$

$\freenames{-} : \pi \to \mathcal{P}(\quotep{\pi})$

\begin{eqnarray*}
  \freenames{\pzero} & := & \emptyset \\
  \freenames{x?(y).P} & := & \{ x \} \cup (\freenames{P} \setminus \{ y \}) \\
  \freenames{x!\langle P \rangle} & := & \{ x \} \cup \{ P \} \\
  \freenames{P|Q} & := & \freenames{P} \cup \freenames{Q} \\
  \freenames{\dropn{x}} & := & \{ x \}
\end{eqnarray*}

The bound names of a process, $\boundnames{P}$, are those names occurring in $P$
that are not free. For example, in $x?(y).0$, the name $x$ is free, while $y$ is bound.

\begin{mathpar}
  \inferrule* [lab=monoidal-laws] {} { P|Q \equiv Q|P \and P|0 \equiv P \and P|(Q|R) \equiv (P|Q)|R }
\end{mathpar}

\begin{mathpar}
  \inferrule* [lab=alpha-equivalence] {} { (x)P \equiv (y)P\{y/x\} \and y \not\in \freenames{P} }
\end{mathpar}

\begin{definition}
Then two processes, $P,Q$, are alpha-equivalent if $P = Q\{\vec{y}/\vec{x}\}$ for
some $\vec{x} \in \boundnames{Q},\vec{y} \in \boundnames{P}$, where $Q\{\vec{y}/\vec{x}\}$
denotes the capture-avoiding substitution of $\vec{y}$ for $\vec{x}$ in $Q$.
\end{definition}

\begin{definition}
  The {\em structural congruence} \cite{SangiorgiWalker} , $\equiv$,
  between processes is the least congruence containing
  alpha-equivalence, satisfying the abelian monoid laws
  (associativity, commutativity and $\pzero$ as identity) for parallel
  composition $|$ and for summation $+$.
\end{definition}

\subsection{Name equivalence}

We take name equivalence, written $\nameeq$, to be the smallest
equivalence relation generated by the following rules.

\begin{mathpar}
\inferrule*[lab=Quote-drop]
{ }
{ \quotep{@{x}} \nameeq x }

\inferrule*[lab=Struct-equiv]
{ P \scong Q }
{ \quotep{P} \nameeq \quotep{Q} }
\end{mathpar}

The astute reader will have noticed that the mutual recursion of names
and processes imposes a mutual recursion on alpha-equivalence and
structural equivalence via name-equivalence. Fortunately, all of this
works out pleasantly and we may calculate in the natural way, free of
concern. The reader interested in the details is referred to the
appendix \ref{appendix:rho_details}.

\subsection{Substitution}

We use $\Proc$ for the set of processes, $\QProc$ for the set of
names, and $\id{\{}\vec{y} / \vec{x} \id{\}}$ to denote partial maps,
$s : \QProc \rightarrow \QProc$. A map, $s$ lifts, uniquely, to a map
on process terms, $\widehat{s} : \Proc \rightarrow \Proc$ by the
following equations.

\begin{mathpar}
  (0) \psubstp{Q}{P} := 0 \\
  (R \juxtap S) \psubstp{Q}{P}
  :=    
  (R)\psubstp{Q}{P} \juxtap (S) \psubstp{Q}{P} \\
  (x?(y).R) \psubstp{Q}{P}    
  :=    
  (x)\substp{Q}{P} (z)\concat( (R \psubstn{z}{y}) \psubstp{Q}{P} ) \\
  (\lift{x}{R}) \psubstp{Q}{P}  
  :=
  \lift{(x)\substp{Q}{P}}{ R \psubstp{Q}{P} } \\
%   (\dropn{x})  \psubstp{Q}{P}       
%   := 
%   \left\{ 
%     \begin{array}{ccc} 
%       \dropn{\quotep{Q}} & & x \nameeq \quotep{P} \\
%       \dropn{x} & & otherwise \\
%     \end{array}
%   \right. 
  (\dropn{x})  \psubstp{Q}{P}       
  := 
  \left\{ 
    \begin{array}{ccc} 
      Q & & x \nameeq \quotep{P} \\
      \dropn{x} & & otherwise \\
    \end{array}
  \right.
\end{mathpar}
 

where

\begin{eqnarray}
  (x)\id{\{} \lpquote Q \rpquote / \lpquote P \rpquote \id{\}}            = 
  \left\{ 
    \begin{array}{ccc}
      \lpquote Q \rpquote & & x \nameeq \lpquote P \rpquote \\
      x & & otherwise \\
    \end{array}
  \right. \nonumber
\end{eqnarray}

and $z$ is chosen distinct from $\quotep{P}$, $\quotep{Q}$, the free
names in $Q$, and all the names in $R$. Our $\alpha$-equivalence will
be built in the standard way from this substitution.

\begin{remark}\label{rem:no_self_referential_names}
  One consequence of these definitions is that $\forall P. \quotep{P}
  \not\in \freenames{P}$.
\end{remark}

\subsection{ Dynamic quote: an example }

Anticipating something of what's to come, consider applying the
substitution, $\widehat{\id{\{}u / z \id{\}}}$, to the following pair
of processes, $\lift{w}{y!(z)}$ and $w[ \lpquote y!(z) \rpquote ]$.

\begin{eqnarray}
	\lift{w}{y!(z)}\widehat{\id{\{}u / z \id{\}}}
		& = &
		\lift{w}{y!(u)} \nonumber\\
	w[ \lpquote y!(z) \rpquote ] \widehat{ \id{\{}u / z \id{\}} }
		& = &
		w[ \lpquote y!(z) \rpquote ] \nonumber
\end{eqnarray}

Because the body of the process between quotes is impervious to
substitution, we get radically different answers. In fact, by
examining the first process in an input context,
e.g. $x?(z).\lift{w}{y!(z)}$, we see that the process under the lift
operator may be shaped by prefixed inputs binding a name inside it. In
this sense, the lift operator will be seen as a way to dynamically
construct processes before reifying them as names.

Finally equipped with these standard features we can present the
dynamics of the calculus.

\subsubsection{Operational semantics} 

Finally, we introduce the computational dynamics. What marks these
algebras as distinct from other more traditionally studied algebraic
structures, e.g. vector spaces or polynomial rings, is the manner in
which dynamics is captured. In traditional structures, dynamics is typically
expressed through morphisms between such structures, as in linear maps
between vector spaces or morphisms between rings. In algebras
associated with the semantics of computation, the dynamics is
expressed as part of the algebraic structure itself, through a
reduction reduction relation typically denoted by $\red$. Below, we
give a recursive presentation of this relation for the calculus used
in the encoding.

$\red \subseteq \pi \times \pi$
$\red : \pi \to \mathcal{P}(\pi)$

\begin{mathpar}
  \inferrule* [lab=Comm] { \textsf{match}( x_{src}, x_{trgt} ) } { x_{trgt}?(y)P \; | \; x_{src}!\langle {Q} \rangle \red P\{\quotep{Q}/y}\} }
  \and \\
  \inferrule* [lab=Par] {{P} \red {P}'} {{{P} | {Q}} \red {{P}' | {Q}}}
  \and
  \inferrule* [lab=Equiv]{{{P} \scong {P}'} \andalso {{P}' \red {Q}'} \andalso {{Q}' \scong {Q}}}{{P} \red {Q}}
\end{mathpar}

\begin{eqnarray*}
  match_{\equiv} (\quotep{P},\quotep{Q}) & := & P \equiv Q \\
  match_{\dagger}(\quotep{P},\quotep{Q}) & := & \forall R. P|Q \red^{*} R => R \red^{*} 0 \\
  match_{K}(\quotep{P},\quotep{Q}) & := & K \mbox{ for some context } K
\end{eqnarray*}

$u?(x)P | u!\langle Q \rangle \red P\{\quotep{Q}/x\}$

%We write $\wred$ for $\red^*$, and $P\red$ if $\exists Q $ such that $ P \red Q$.
We write $P\red$ if $\exists Q $ such that $ P \red Q$ and $P\not\red$, otherwise.

\section{Replication}

As mentioned before, it is known that replication (and hence
recursion) can be implemented in a higher-order process algebra
\cite{SangiorgiWalker}. As our first example of calculation with the
machinery thus far presented we give the construction explicitly in
the {\rhoc}.

\begin{eqnarray}
	D_{x} & := & \prefix{x}{y}{(\binpar{\outputp{x}{y}}{@{y}})} \nonumber\\
	\bangp_{x}{P} & := & \binpar{{x}!\langle{\binpar{D_{x}}{P}}\rangle}{D_{x}} \nonumber
\end{eqnarray}

\begin{eqnarray}
	\bangp_{x}{P} & & \nonumber\\
	=
	& {x}!\langle{(\prefix{x}{y}{(\outputp{x}{y} | @{y})) | P}}\rangle 
	      | \prefix{x}{y}{(\outputp{x}{y} | @{y})} & \nonumber\\
	\red
	& (\outputp{x}{y} | @{y})\substn{\quotep{(\prefix{x}{y}{(@{y} | \outputp{x}{y})) | P}}}{y} & \nonumber\\
	=
	& \outputp{x}{\quotep{(\prefix{x}{y}{(\outputp{x}{y} | @{y})) | P}}}
	  | {(\prefix{x}{y}{(\outputp{x}{y} | @{y})) | P}} & \nonumber\\
	\red
	& \ldots & \nonumber\\
	\red^*
	& P | P | \ldots & \nonumber
\end{eqnarray}

Of course, this encoding, as an implementation, runs away, unfolding
$\bangp{P}$ eagerly. A lazier and more implementable replication
operator, restricted to input-guarded processes, may be obtained as follows.

\begin{eqnarray}
\bangp{\prefix{u}{v}{P}} 
	:= 
	\binpar{\lift{x}{\prefix{u}{v}{(\binpar{D(x)}{P})}}}{D(x)} \nonumber
\end{eqnarray}

\begin{remark}
  Note that the lazier definition still does not deal with summation
  or mixed summation (i.e. sums over input and output). The reader is
  invited to construct definitions of replication that deal with these
  features. 

  Further, the definitions are parameterized in a name, $x$. Can you,
  gentle reader, make a definition that eliminates this parameter and
  guarantees no accidental interaction between the replication
  machinery and the process being replicated -- i.e. no accidental
  sharing of names used by the process to get its work done and the
  name(s) used by the replication to effect copying. This latter
  revision of the definition of replication is crucial to obtaining
  the expected identity $!!P \sim !P$.
\end{remark}

\begin{remark}\label{rem:paradoxical_combinator}
  The reader familiar with the lambda calculus will have noticed the
  similarity between $D$ and the paradoxical combinator.

  [Ed. note: the existence of this seems to suggest we have to be more
  restrictive on the set of processes and names we admit if we are to
  support no-cloning.]
\end{remark}

\subsubsection{Bisimulation}

The computational dynamics gives rise to another kind of equivalence,
the equivalence of computational behavior. As previously mentioned
this is typically captured \emph{via} some form of bisimulation.

% The notion we use in this paper is weak barbed bisimulation
% \cite{milner91polyadicpi}.

The notion we use in this paper is derived from weak barbed
bisimulation \cite{milner91polyadicpi}. 

\begin{definition}
An \emph{observation relation}, $\downarrow_{\mathcal N}$, over a set
of names, $\mathcal N$, is the smallest relation satisfying the rules
below.

\infrule[Out-barb]{y \in {\mathcal N}, \; x \nameeq y}
		  {\outputp{x}{v} \downarrow_{\mathcal N} x}
\infrule[Par-barb]{\mbox{$P\downarrow_{\mathcal N} x$ or $Q\downarrow_{\mathcal N} x$}}
		  {\binpar{P}{Q} \downarrow_{\mathcal N} x}

We write $P \Downarrow_{\mathcal N} x$ if there is $Q$ such that 
$P \wred Q$ and $Q \downarrow_{\mathcal N} x$.
\end{definition}

\begin{definition}
%\label{def.bbisim}
An  ${\mathcal N}$-\emph{barbed bisimulation} over a set of names, ${\mathcal N}$, is a symmetric binary relation 
${\mathcal S}_{\mathcal N}$ between agents such that $P\rel{S}_{\mathcal N}Q$ implies:
\begin{enumerate}
\item If $P \red P'$ then $Q \wred Q'$ and $P'\rel{S}_{\mathcal N} Q'$.
\item If $P\downarrow_{\mathcal N} x$, then $Q\Downarrow_{\mathcal N} x$.
\end{enumerate}
$P$ is ${\mathcal N}$-barbed bisimilar to $Q$, written
$P \wbbisim_{\mathcal N} Q$, if $P \rel{S}_{\mathcal N} Q$ for some ${\mathcal N}$-barbed bisimulation ${\mathcal S}_{\mathcal N}$.
\end{definition}

$\mathcal{R} \subseteq \pi \times \pi$

$P \mathcal{R} Q => \forall P'. P \red P' \Rightarrow \exists Q'. Q \red Q', P' \mathcal{R} Q'$

$P \vdash x \Rightarrow Q \vdash x$

\begin{mathpar}
  \inferrule*[lab=Out-barb]{x \nameeq y}{{y}!\langle{Q}\rangle \vdash x}
  \and
  \inferrule*[lab=Par-barb]{\mbox{$P\vdash x$ or $Q\vdash x$}}{\binpar{P}{Q} \vdash x}
\end{mathpar}

\subsubsection{Contexts}

One of the principle advantages of computational calculi like the
$\pi$-calculus is a well-defined notion of context,
contextual-equivalence and a correlation between
contextual-equivalence and notions of bisimulation. The notion of
context allows the decomposition of a process into (sub-)process and
its syntactic environment, its context. Thus, a context may be
thought of as a process with a ``hole'' (written $\Box$) in it. The
application of a context $M$ to a process $P$, written $M[P]$, is
tantamount to filling the hole in $M$ with $P$. In this paper we do
not need the full weight of this theory, but do make use of the notion
of context in the proof the main theorem. 

\begin{mathpar}
  \inferrule* [lab=summation] {} {{M_{M},M_{N}} \bc \Box \;|\; x.M_{A} \;|\; M_{M}+M_{N}}
  \and
  \inferrule* [lab=agent] {} {{M_{A}} \bc (\vec{x})M_{P} \;| \; \clift{P_0,\ldots,M_{P},\ldots,P_N}}
  \and \\
  \inferrule* [lab=process] {} {{M_{P}} \bc M_{N} \;| \;P|M_{P} }
\end{mathpar} 

\begin{mathpar}
  \inferrule* [lab=sychronization] {} {M_{N} \bc \Box \;|\; x?M_{F} \;|\; x!M_{C}}
  \and
  \inferrule* [lab=abstraction] {} {{M_{F}} \bc (x)M_{P} }
  \and
  \inferrule* [lab=concretion] {} {{M_{C}} \bc \langle M_{P} \rangle }
  \and \\
  \inferrule* [lab=process] {} {{M_{P}} \bc M_{N} \;| \;P|M_{P} }
\end{mathpar}

\begin{definition}[contextual application] Given a context $M$, and
  process $P$, we define the \emph{contextual application}, $M[P] :=
  M\{P/\Box\}$. That is, the contextual application of M to P is the
  substitution of $P$ for $\Box$ in $M$.
\end{definition}

$\meaningof{-} : L \to \mathcal{P}(\pi)$

\begin{mathpar}
  \inferrule* [lab=collection] {} {\meaningof{true} = \pi, \and \meaningof{~E} = \pi \setminus \meaningof{E}, \and \meaningof{E_{1} \& E_{2}} = \meaningof{E_{1}} \cap \meaningof{E_{2}}}
\end{mathpar}

\begin{mathpar}
  \inferrule* [lab=structure] {} {\meaningof{0} = \{ P \in \pi | P \equiv 0 \}, \and \\ \meaningof{E_1 | E_2} = \{ P \in \pi | P \equiv P_{1} | P_{2}, P_{1} \in \meaningof{E_{1}}, P_{2} \in \meaningof{E_2}\} }
\end{mathpar}

\begin{mathpar}
 \inferrule* [lab=behavior] {} {\meaningof{\langle a?b \rangle E} = \{ P \in \pi | P \equiv Q | u?(y)P', \\ \and \\\\ \and \\ \;\;\; u \in \meaningof{a}, \forall z.P'\{z/y\} \in \meaningof{E\{z/b\}}\}, \and \\ \meaningof{a!E} = \{ P \in \pi | P \equiv Q | x!\langle P' \rangle, x \in \meaningof{a} P' \in \meaningof{E}\} }
\end{mathpar}

\begin{mathpar}
 \inferrule* [lab=nominal] {} {\meaningof{\quotep{E}} = \{ \quotep{P} \in \quotep{\pi} | P \in \meaningof{E} \}, \and \meaningof{\quotep{P}} = \{ \quotep{Q} \in \quotep{\pi} | P \equiv Q \} \and \\ \meaningof{@\quotep{E}} = \{ P \in \pi | P \equiv @x, x \in \meaningof{E} \}}
\end{mathpar}

\begin{eqnarray*}
  \\
  \meaningof{-} : TS \to ST
\end{eqnarray*}

\begin{eqnarray*}
  \\
  L : TS \to ST
\end{eqnarray*}

\begin{eqnarray*}
  \\
  P \models E \iff P \in \meaningof{E}
\end{eqnarray*}

\begin{eqnarray*}
  P \approx_{L} Q \iff \forall E \in L. P \models E \iff Q \models E
\end{eqnarray*}

\begin{eqnarray*}
  P \approx_{K} Q
\end{eqnarray*}

\begin{eqnarray*}
  P \approx Q
\end{eqnarray*}

$\approx_{K} = \approx = \approx_{L}$

\subsubsection{Contextual duality}

Note that contexts extend the quotation operation to a family of
operations from processes to names. Given a context, $M$, we can
define a \emph{nominal context}, $\quotep{M}$ by $\quotep{M}[P] :=
\quotep{M[P]}$. To foreshadow what is to come we observe that these
operations enjoy a duality with processes very much like the duality
between vectors and maps from vectors to scalars.

Further, because the calculus is essentially higher-order, we have a
correspondence between contexts and processes. More specifically,
given a name $x$ and a context $M$ we can construct $M^{*}_{x}$ such
that 

\begin{mathpar}
  M^{*}_{x} | \lift{x}{P} \red M[P]
\end{mathpar}

namely,

\begin{mathpar}
  M^{*}_{x} := x?(u).M[\dropn{u}]
\end{mathpar}

The dependence of $M^{*}_{x}$ on a name makes it an abstraction, 

\begin{mathpar}
  M^{*} := (x)x?(u).M[\dropn{u}]
\end{mathpar}

\subsection{Additional notation}

It will sometimes be convenient to denote the process a name
quotes. We already have the notation $x = \quotep{P}$, but it will be
convenient to introduce an alternate notation, $\procn{x}$, when we
want to emphasize the connection to the use of the name. Note that, by
virtue of name equivalence, $\quotep{\procn{x}} \nameeq x$; so, the
notation is consistent with previous definitions.

Further, because names have structure it is possible to effect
substitutions on the basis of that structure. This means we need to
upgrade our notation for substitutions, which we accomplish by
adapting comprehension notation. Thus,

\begin{mathpar}
  P\{ y / x : x \in S \}
\end{mathpar}

is interpreted to mean the process derived from P by replacing (in a
capture-avoiding manner) each occurrence of $x$ in $S$ by $y$. For example,

\begin{mathpar}
  P\{ \quotep{\procn{x}|\procn{x}} / x : x \in \freenames{P} \}
\end{mathpar}

will replace each (occurrence) of a free name $x$ in $P$ by
$\quotep{\procn{x}|\procn{x}}$.

Also, we will avail ourselves of the notation $x^{L}$ and $x^{R}$ to
denote injections of a name into disjoint copies of the name
space. There are numerous ways to accomplish this. One example can be
found in \cite{MeredithR05}. This notation overloads to vectors of
names: $\vec{x}^{\pi} := (x_{i}^{\pi} \; : \; 0 \leq i < |\vec{x}| )$ where $\pi \in \{L,R\}$.

We also use $P^{\Box} := P|\Box$.

In \cite{MeredithR05} an interpretation of the new operator is
given. It turns out that there are several possible interpretations
all enjoying the requisite algebraic properties of the operator (see
\cite{milner91polyadicpi}). We will therefore make liberal use of
$(\nu\; \vec{x})P$.

% subsection the_syntax_and_semantics_of_the_notation_system (end)   

\input{qm2pi.qmops} 

\input{qm2pi.sterngerlach} 

\input{qm2pi.metric} 

% section concurrent_process_calculi (end)

%\input{qm2pi.proofsketch}

% section proof sketch (end)

%\input{qm2pi.slviaknots} 

% section spatial logic via knots (end)

\input{qm2pi.conclusion}

% section conclusion (end)

%\input{qm2pi.dtcodes} 

% section wiring algorithm (end)

\input{qm2pi.ack} 

% section acknowledgments (end)

\newpage


\bibliographystyle{plain}   
\bibliography{../../biblios/main.bib}

\input{qm2pi.rhodetails}

\end{document}

 

% section concurrent_process_calculi (end)

%\documentclass[12pt]{llncs}
%\documentclass{jktr}

\usepackage[pdftex]{hyperref}                   
\usepackage {listings}
\usepackage {mathpartir}
\usepackage{bcprules}
%\usepackage{listings}
                       
\usepackage{graphicx} 
%\usepackage[margins=2.5cm,nohead,nofoot]{geometry}
%\usepackage{geometry}
\usepackage{amsfonts}
\usepackage{amstext}
\usepackage{latexsym}
\usepackage{amssymb}
\usepackage{color}


%\include{myPreamble}
\include{qm2pi.local} 

%\ifpdf
%\usepackage[pdftex]{graphicx}
%\else
%\usepackage{graphicx}
%\fi

 % \ifpdf
%  \usepackage{pdfsync}
%  \if


%\title{Brief Article}
%\author{David F. Snyder}
%\author{L.G. Meredith}

%\address{Dept. of Math., Texas State University--San Marcos, San Marcos, TX 78666}
       
\pagestyle{empty}


\begin{document}

\lstset{language=[Objective]Caml,frame=shadowbox}

\input{qm2pi.front}

% section front matter (end)

\input{qm2pi.intro} 
 
% section introduction (end)

% \input{qm2pi.knotations} 

% section notation (end)

\input{qm2pi.process.calculi} 

% section concurrent_process_calculi_and_spatial_logics_ (end)
    
%\input{qm2pi.knots2pi} 

%\input{qm2pi.trefoil} 

%\input{qm2pi.mainthm} 

% subsection basic_interpretation (end)

%\input{qm2pi.rho.presentation} 
\subsection{The syntax and semantics of the notation system}\label{sub:the_syntax_and_semantics_of_the_notation_system} % (fold)

We now summarize a technical presentation of the calculus that
embodies our theory of dynamics. The typical presentation of such a
calculus follows the style of giving generators and relations on
them. The grammar, below, describing term constructors, freely
generates the set of processes, $\Proc$. This set is then quotiented
by a relation known as structural congruence and it is over this set
that the notion of dynamics is expressed. This presentation is
essentially that of \cite{MeredithR05} with the addition of
polyadicity and summation. For readability we have relegated some of
the technical subtleties to an appendix.

\subsubsection{Process grammar}\label{subsub:process_grammar}

\begin{mathpar}
  \inferrule* [lab=synchronization] {} {{M} \bc \pzero \;|\; x?F \;|\; x!C }
  \and
  \inferrule* [lab=abstraction] {} {{F} \bc (x)P}
  \and
  \inferrule* [lab=concretion] {} {{C} \bc \langle Q \rangle}
  \and
  \inferrule* [lab=process] {} {{P,Q} \bc M \;| \;P|Q \;|\; @{x}}
  \and
  \inferrule* [lab=name] {} {{x} \bc \quotep{P}}
\end{mathpar} 

Note that $\vec{x}$ (resp. $\vec{P}$) denotes a vector of names
(resp. processes) of length $|\vec{x}|$ (resp. $|\vec{P}|$). We adopt
the following useful abbreviations.

\begin{mathpar}
   x?(\vec{y}).P := x.(\vec{y})P \and  x\clift{\vec{P}} := x.\clift{\vec{P}}
   \and x!(y) := \lift{x}{\dropn{y}}
   \and \Pi_{i=0}^{n-1}P_i := P_0 | \ldots | P_{n-1}
\end{mathpar}

\subsubsection{Structural congruence}

\paragraph{Free and bound names and alpha-equivalence.} At the
core of structural equivalence is alpha-equivalence which identifies
process that are the same up to a change of variable. Formally, we
recognize the distinction between free and bound names. The free names
of a process, $\freenames{P}$, may be calculated recursively as
follows:

\begin{mathpar}
\freenames{\pzero} := \emptyset
  \and \\
  \freenames{x?(y).P} := \{ x \} \cup (\freenames{P} \setminus \{ y \})
  \and 
  \freenames{x!\langle P \rangle} := \{ x \} \cup \{ P \} 
  \and \\
  \freenames{P|Q} := \freenames{P} \cup \freenames{Q}
  \and \\
  \freenames{@{x}} := \{ x \}
\end{mathpar}

$\pi$
$\quotep{\pi}$

$\freenames{-} : \pi \to \mathcal{P}(\quotep{\pi})$

\begin{eqnarray*}
  \freenames{\pzero} & := & \emptyset \\
  \freenames{x?(y).P} & := & \{ x \} \cup (\freenames{P} \setminus \{ y \}) \\
  \freenames{x!\langle P \rangle} & := & \{ x \} \cup \{ P \} \\
  \freenames{P|Q} & := & \freenames{P} \cup \freenames{Q} \\
  \freenames{\dropn{x}} & := & \{ x \}
\end{eqnarray*}

The bound names of a process, $\boundnames{P}$, are those names occurring in $P$
that are not free. For example, in $x?(y).0$, the name $x$ is free, while $y$ is bound.

\begin{mathpar}
  \inferrule* [lab=monoidal-laws] {} { P|Q \equiv Q|P \and P|0 \equiv P \and P|(Q|R) \equiv (P|Q)|R }
\end{mathpar}

\begin{mathpar}
  \inferrule* [lab=alpha-equivalence] {} { (x)P \equiv (y)P\{y/x\} \and y \not\in \freenames{P} }
\end{mathpar}

\begin{definition}
Then two processes, $P,Q$, are alpha-equivalent if $P = Q\{\vec{y}/\vec{x}\}$ for
some $\vec{x} \in \boundnames{Q},\vec{y} \in \boundnames{P}$, where $Q\{\vec{y}/\vec{x}\}$
denotes the capture-avoiding substitution of $\vec{y}$ for $\vec{x}$ in $Q$.
\end{definition}

\begin{definition}
  The {\em structural congruence} \cite{SangiorgiWalker} , $\equiv$,
  between processes is the least congruence containing
  alpha-equivalence, satisfying the abelian monoid laws
  (associativity, commutativity and $\pzero$ as identity) for parallel
  composition $|$ and for summation $+$.
\end{definition}

\subsection{Name equivalence}

We take name equivalence, written $\nameeq$, to be the smallest
equivalence relation generated by the following rules.

\begin{mathpar}
\inferrule*[lab=Quote-drop]
{ }
{ \quotep{@{x}} \nameeq x }

\inferrule*[lab=Struct-equiv]
{ P \scong Q }
{ \quotep{P} \nameeq \quotep{Q} }
\end{mathpar}

The astute reader will have noticed that the mutual recursion of names
and processes imposes a mutual recursion on alpha-equivalence and
structural equivalence via name-equivalence. Fortunately, all of this
works out pleasantly and we may calculate in the natural way, free of
concern. The reader interested in the details is referred to the
appendix \ref{appendix:rho_details}.

\subsection{Substitution}

We use $\Proc$ for the set of processes, $\QProc$ for the set of
names, and $\id{\{}\vec{y} / \vec{x} \id{\}}$ to denote partial maps,
$s : \QProc \rightarrow \QProc$. A map, $s$ lifts, uniquely, to a map
on process terms, $\widehat{s} : \Proc \rightarrow \Proc$ by the
following equations.

\begin{mathpar}
  (0) \psubstp{Q}{P} := 0 \\
  (R \juxtap S) \psubstp{Q}{P}
  :=    
  (R)\psubstp{Q}{P} \juxtap (S) \psubstp{Q}{P} \\
  (x?(y).R) \psubstp{Q}{P}    
  :=    
  (x)\substp{Q}{P} (z)\concat( (R \psubstn{z}{y}) \psubstp{Q}{P} ) \\
  (\lift{x}{R}) \psubstp{Q}{P}  
  :=
  \lift{(x)\substp{Q}{P}}{ R \psubstp{Q}{P} } \\
%   (\dropn{x})  \psubstp{Q}{P}       
%   := 
%   \left\{ 
%     \begin{array}{ccc} 
%       \dropn{\quotep{Q}} & & x \nameeq \quotep{P} \\
%       \dropn{x} & & otherwise \\
%     \end{array}
%   \right. 
  (\dropn{x})  \psubstp{Q}{P}       
  := 
  \left\{ 
    \begin{array}{ccc} 
      Q & & x \nameeq \quotep{P} \\
      \dropn{x} & & otherwise \\
    \end{array}
  \right.
\end{mathpar}
 

where

\begin{eqnarray}
  (x)\id{\{} \lpquote Q \rpquote / \lpquote P \rpquote \id{\}}            = 
  \left\{ 
    \begin{array}{ccc}
      \lpquote Q \rpquote & & x \nameeq \lpquote P \rpquote \\
      x & & otherwise \\
    \end{array}
  \right. \nonumber
\end{eqnarray}

and $z$ is chosen distinct from $\quotep{P}$, $\quotep{Q}$, the free
names in $Q$, and all the names in $R$. Our $\alpha$-equivalence will
be built in the standard way from this substitution.

\begin{remark}\label{rem:no_self_referential_names}
  One consequence of these definitions is that $\forall P. \quotep{P}
  \not\in \freenames{P}$.
\end{remark}

\subsection{ Dynamic quote: an example }

Anticipating something of what's to come, consider applying the
substitution, $\widehat{\id{\{}u / z \id{\}}}$, to the following pair
of processes, $\lift{w}{y!(z)}$ and $w[ \lpquote y!(z) \rpquote ]$.

\begin{eqnarray}
	\lift{w}{y!(z)}\widehat{\id{\{}u / z \id{\}}}
		& = &
		\lift{w}{y!(u)} \nonumber\\
	w[ \lpquote y!(z) \rpquote ] \widehat{ \id{\{}u / z \id{\}} }
		& = &
		w[ \lpquote y!(z) \rpquote ] \nonumber
\end{eqnarray}

Because the body of the process between quotes is impervious to
substitution, we get radically different answers. In fact, by
examining the first process in an input context,
e.g. $x?(z).\lift{w}{y!(z)}$, we see that the process under the lift
operator may be shaped by prefixed inputs binding a name inside it. In
this sense, the lift operator will be seen as a way to dynamically
construct processes before reifying them as names.

Finally equipped with these standard features we can present the
dynamics of the calculus.

\subsubsection{Operational semantics} 

Finally, we introduce the computational dynamics. What marks these
algebras as distinct from other more traditionally studied algebraic
structures, e.g. vector spaces or polynomial rings, is the manner in
which dynamics is captured. In traditional structures, dynamics is typically
expressed through morphisms between such structures, as in linear maps
between vector spaces or morphisms between rings. In algebras
associated with the semantics of computation, the dynamics is
expressed as part of the algebraic structure itself, through a
reduction reduction relation typically denoted by $\red$. Below, we
give a recursive presentation of this relation for the calculus used
in the encoding.

$\red \subseteq \pi \times \pi$
$\red : \pi \to \mathcal{P}(\pi)$

\begin{mathpar}
  \inferrule* [lab=Comm] { \textsf{match}( x_{src}, x_{trgt} ) } { x_{trgt}?(y)P \; | \; x_{src}!\langle {Q} \rangle \red P\{\quotep{Q}/y}\} }
  \and \\
  \inferrule* [lab=Par] {{P} \red {P}'} {{{P} | {Q}} \red {{P}' | {Q}}}
  \and
  \inferrule* [lab=Equiv]{{{P} \scong {P}'} \andalso {{P}' \red {Q}'} \andalso {{Q}' \scong {Q}}}{{P} \red {Q}}
\end{mathpar}

\begin{eqnarray*}
  match_{\equiv} (\quotep{P},\quotep{Q}) & := & P \equiv Q \\
  match_{\dagger}(\quotep{P},\quotep{Q}) & := & \forall R. P|Q \red^{*} R => R \red^{*} 0 \\
  match_{K}(\quotep{P},\quotep{Q}) & := & K \mbox{ for some context } K
\end{eqnarray*}

$u?(x)P | u!\langle Q \rangle \red P\{\quotep{Q}/x\}$

%We write $\wred$ for $\red^*$, and $P\red$ if $\exists Q $ such that $ P \red Q$.
We write $P\red$ if $\exists Q $ such that $ P \red Q$ and $P\not\red$, otherwise.

\section{Replication}

As mentioned before, it is known that replication (and hence
recursion) can be implemented in a higher-order process algebra
\cite{SangiorgiWalker}. As our first example of calculation with the
machinery thus far presented we give the construction explicitly in
the {\rhoc}.

\begin{eqnarray}
	D_{x} & := & \prefix{x}{y}{(\binpar{\outputp{x}{y}}{@{y}})} \nonumber\\
	\bangp_{x}{P} & := & \binpar{{x}!\langle{\binpar{D_{x}}{P}}\rangle}{D_{x}} \nonumber
\end{eqnarray}

\begin{eqnarray}
	\bangp_{x}{P} & & \nonumber\\
	=
	& {x}!\langle{(\prefix{x}{y}{(\outputp{x}{y} | @{y})) | P}}\rangle 
	      | \prefix{x}{y}{(\outputp{x}{y} | @{y})} & \nonumber\\
	\red
	& (\outputp{x}{y} | @{y})\substn{\quotep{(\prefix{x}{y}{(@{y} | \outputp{x}{y})) | P}}}{y} & \nonumber\\
	=
	& \outputp{x}{\quotep{(\prefix{x}{y}{(\outputp{x}{y} | @{y})) | P}}}
	  | {(\prefix{x}{y}{(\outputp{x}{y} | @{y})) | P}} & \nonumber\\
	\red
	& \ldots & \nonumber\\
	\red^*
	& P | P | \ldots & \nonumber
\end{eqnarray}

Of course, this encoding, as an implementation, runs away, unfolding
$\bangp{P}$ eagerly. A lazier and more implementable replication
operator, restricted to input-guarded processes, may be obtained as follows.

\begin{eqnarray}
\bangp{\prefix{u}{v}{P}} 
	:= 
	\binpar{\lift{x}{\prefix{u}{v}{(\binpar{D(x)}{P})}}}{D(x)} \nonumber
\end{eqnarray}

\begin{remark}
  Note that the lazier definition still does not deal with summation
  or mixed summation (i.e. sums over input and output). The reader is
  invited to construct definitions of replication that deal with these
  features. 

  Further, the definitions are parameterized in a name, $x$. Can you,
  gentle reader, make a definition that eliminates this parameter and
  guarantees no accidental interaction between the replication
  machinery and the process being replicated -- i.e. no accidental
  sharing of names used by the process to get its work done and the
  name(s) used by the replication to effect copying. This latter
  revision of the definition of replication is crucial to obtaining
  the expected identity $!!P \sim !P$.
\end{remark}

\begin{remark}\label{rem:paradoxical_combinator}
  The reader familiar with the lambda calculus will have noticed the
  similarity between $D$ and the paradoxical combinator.

  [Ed. note: the existence of this seems to suggest we have to be more
  restrictive on the set of processes and names we admit if we are to
  support no-cloning.]
\end{remark}

\subsubsection{Bisimulation}

The computational dynamics gives rise to another kind of equivalence,
the equivalence of computational behavior. As previously mentioned
this is typically captured \emph{via} some form of bisimulation.

% The notion we use in this paper is weak barbed bisimulation
% \cite{milner91polyadicpi}.

The notion we use in this paper is derived from weak barbed
bisimulation \cite{milner91polyadicpi}. 

\begin{definition}
An \emph{observation relation}, $\downarrow_{\mathcal N}$, over a set
of names, $\mathcal N$, is the smallest relation satisfying the rules
below.

\infrule[Out-barb]{y \in {\mathcal N}, \; x \nameeq y}
		  {\outputp{x}{v} \downarrow_{\mathcal N} x}
\infrule[Par-barb]{\mbox{$P\downarrow_{\mathcal N} x$ or $Q\downarrow_{\mathcal N} x$}}
		  {\binpar{P}{Q} \downarrow_{\mathcal N} x}

We write $P \Downarrow_{\mathcal N} x$ if there is $Q$ such that 
$P \wred Q$ and $Q \downarrow_{\mathcal N} x$.
\end{definition}

\begin{definition}
%\label{def.bbisim}
An  ${\mathcal N}$-\emph{barbed bisimulation} over a set of names, ${\mathcal N}$, is a symmetric binary relation 
${\mathcal S}_{\mathcal N}$ between agents such that $P\rel{S}_{\mathcal N}Q$ implies:
\begin{enumerate}
\item If $P \red P'$ then $Q \wred Q'$ and $P'\rel{S}_{\mathcal N} Q'$.
\item If $P\downarrow_{\mathcal N} x$, then $Q\Downarrow_{\mathcal N} x$.
\end{enumerate}
$P$ is ${\mathcal N}$-barbed bisimilar to $Q$, written
$P \wbbisim_{\mathcal N} Q$, if $P \rel{S}_{\mathcal N} Q$ for some ${\mathcal N}$-barbed bisimulation ${\mathcal S}_{\mathcal N}$.
\end{definition}

$\mathcal{R} \subseteq \pi \times \pi$

$P \mathcal{R} Q => \forall P'. P \red P' \Rightarrow \exists Q'. Q \red Q', P' \mathcal{R} Q'$

$P \vdash x \Rightarrow Q \vdash x$

\begin{mathpar}
  \inferrule*[lab=Out-barb]{x \nameeq y}{{y}!\langle{Q}\rangle \vdash x}
  \and
  \inferrule*[lab=Par-barb]{\mbox{$P\vdash x$ or $Q\vdash x$}}{\binpar{P}{Q} \vdash x}
\end{mathpar}

\subsubsection{Contexts}

One of the principle advantages of computational calculi like the
$\pi$-calculus is a well-defined notion of context,
contextual-equivalence and a correlation between
contextual-equivalence and notions of bisimulation. The notion of
context allows the decomposition of a process into (sub-)process and
its syntactic environment, its context. Thus, a context may be
thought of as a process with a ``hole'' (written $\Box$) in it. The
application of a context $M$ to a process $P$, written $M[P]$, is
tantamount to filling the hole in $M$ with $P$. In this paper we do
not need the full weight of this theory, but do make use of the notion
of context in the proof the main theorem. 

\begin{mathpar}
  \inferrule* [lab=summation] {} {{M_{M},M_{N}} \bc \Box \;|\; x.M_{A} \;|\; M_{M}+M_{N}}
  \and
  \inferrule* [lab=agent] {} {{M_{A}} \bc (\vec{x})M_{P} \;| \; \clift{P_0,\ldots,M_{P},\ldots,P_N}}
  \and \\
  \inferrule* [lab=process] {} {{M_{P}} \bc M_{N} \;| \;P|M_{P} }
\end{mathpar} 

\begin{mathpar}
  \inferrule* [lab=sychronization] {} {M_{N} \bc \Box \;|\; x?M_{F} \;|\; x!M_{C}}
  \and
  \inferrule* [lab=abstraction] {} {{M_{F}} \bc (x)M_{P} }
  \and
  \inferrule* [lab=concretion] {} {{M_{C}} \bc \langle M_{P} \rangle }
  \and \\
  \inferrule* [lab=process] {} {{M_{P}} \bc M_{N} \;| \;P|M_{P} }
\end{mathpar}

\begin{definition}[contextual application] Given a context $M$, and
  process $P$, we define the \emph{contextual application}, $M[P] :=
  M\{P/\Box\}$. That is, the contextual application of M to P is the
  substitution of $P$ for $\Box$ in $M$.
\end{definition}

$\meaningof{-} : L \to \mathcal{P}(\pi)$

\begin{mathpar}
  \inferrule* [lab=collection] {} {\meaningof{true} = \pi, \and \meaningof{~E} = \pi \setminus \meaningof{E}, \and \meaningof{E_{1} \& E_{2}} = \meaningof{E_{1}} \cap \meaningof{E_{2}}}
\end{mathpar}

\begin{mathpar}
  \inferrule* [lab=structure] {} {\meaningof{0} = \{ P \in \pi | P \equiv 0 \}, \and \\ \meaningof{E_1 | E_2} = \{ P \in \pi | P \equiv P_{1} | P_{2}, P_{1} \in \meaningof{E_{1}}, P_{2} \in \meaningof{E_2}\} }
\end{mathpar}

\begin{mathpar}
 \inferrule* [lab=behavior] {} {\meaningof{\langle a?b \rangle E} = \{ P \in \pi | P \equiv Q | u?(y)P', \\ \and \\\\ \and \\ \;\;\; u \in \meaningof{a}, \forall z.P'\{z/y\} \in \meaningof{E\{z/b\}}\}, \and \\ \meaningof{a!E} = \{ P \in \pi | P \equiv Q | x!\langle P' \rangle, x \in \meaningof{a} P' \in \meaningof{E}\} }
\end{mathpar}

\begin{mathpar}
 \inferrule* [lab=nominal] {} {\meaningof{\quotep{E}} = \{ \quotep{P} \in \quotep{\pi} | P \in \meaningof{E} \}, \and \meaningof{\quotep{P}} = \{ \quotep{Q} \in \quotep{\pi} | P \equiv Q \} \and \\ \meaningof{@\quotep{E}} = \{ P \in \pi | P \equiv @x, x \in \meaningof{E} \}}
\end{mathpar}

\begin{eqnarray*}
  \\
  \meaningof{-} : TS \to ST
\end{eqnarray*}

\begin{eqnarray*}
  \\
  L : TS \to ST
\end{eqnarray*}

\begin{eqnarray*}
  \\
  P \models E \iff P \in \meaningof{E}
\end{eqnarray*}

\begin{eqnarray*}
  P \approx_{L} Q \iff \forall E \in L. P \models E \iff Q \models E
\end{eqnarray*}

\begin{eqnarray*}
  P \approx_{K} Q
\end{eqnarray*}

\begin{eqnarray*}
  P \approx Q
\end{eqnarray*}

$\approx_{K} = \approx = \approx_{L}$

\subsubsection{Contextual duality}

Note that contexts extend the quotation operation to a family of
operations from processes to names. Given a context, $M$, we can
define a \emph{nominal context}, $\quotep{M}$ by $\quotep{M}[P] :=
\quotep{M[P]}$. To foreshadow what is to come we observe that these
operations enjoy a duality with processes very much like the duality
between vectors and maps from vectors to scalars.

Further, because the calculus is essentially higher-order, we have a
correspondence between contexts and processes. More specifically,
given a name $x$ and a context $M$ we can construct $M^{*}_{x}$ such
that 

\begin{mathpar}
  M^{*}_{x} | \lift{x}{P} \red M[P]
\end{mathpar}

namely,

\begin{mathpar}
  M^{*}_{x} := x?(u).M[\dropn{u}]
\end{mathpar}

The dependence of $M^{*}_{x}$ on a name makes it an abstraction, 

\begin{mathpar}
  M^{*} := (x)x?(u).M[\dropn{u}]
\end{mathpar}

\subsection{Additional notation}

It will sometimes be convenient to denote the process a name
quotes. We already have the notation $x = \quotep{P}$, but it will be
convenient to introduce an alternate notation, $\procn{x}$, when we
want to emphasize the connection to the use of the name. Note that, by
virtue of name equivalence, $\quotep{\procn{x}} \nameeq x$; so, the
notation is consistent with previous definitions.

Further, because names have structure it is possible to effect
substitutions on the basis of that structure. This means we need to
upgrade our notation for substitutions, which we accomplish by
adapting comprehension notation. Thus,

\begin{mathpar}
  P\{ y / x : x \in S \}
\end{mathpar}

is interpreted to mean the process derived from P by replacing (in a
capture-avoiding manner) each occurrence of $x$ in $S$ by $y$. For example,

\begin{mathpar}
  P\{ \quotep{\procn{x}|\procn{x}} / x : x \in \freenames{P} \}
\end{mathpar}

will replace each (occurrence) of a free name $x$ in $P$ by
$\quotep{\procn{x}|\procn{x}}$.

Also, we will avail ourselves of the notation $x^{L}$ and $x^{R}$ to
denote injections of a name into disjoint copies of the name
space. There are numerous ways to accomplish this. One example can be
found in \cite{MeredithR05}. This notation overloads to vectors of
names: $\vec{x}^{\pi} := (x_{i}^{\pi} \; : \; 0 \leq i < |\vec{x}| )$ where $\pi \in \{L,R\}$.

We also use $P^{\Box} := P|\Box$.

In \cite{MeredithR05} an interpretation of the new operator is
given. It turns out that there are several possible interpretations
all enjoying the requisite algebraic properties of the operator (see
\cite{milner91polyadicpi}). We will therefore make liberal use of
$(\nu\; \vec{x})P$.

% subsection the_syntax_and_semantics_of_the_notation_system (end)   

\input{qm2pi.qmops} 

\input{qm2pi.sterngerlach} 

\input{qm2pi.metric} 

% section concurrent_process_calculi (end)

%\input{qm2pi.proofsketch}

% section proof sketch (end)

%\input{qm2pi.slviaknots} 

% section spatial logic via knots (end)

\input{qm2pi.conclusion}

% section conclusion (end)

%\input{qm2pi.dtcodes} 

% section wiring algorithm (end)

\input{qm2pi.ack} 

% section acknowledgments (end)

\newpage


\bibliographystyle{plain}   
\bibliography{../../biblios/main.bib}

\input{qm2pi.rhodetails}

\end{document}



% section proof sketch (end)

%\section{Unlikely characters: spatial logic for
  knots}\label{sub:characteristic_formulae} % (fold)

Associated to the mobile process calculi are a family of logics known
as the Hennessy-Milner logics. These logics typically enjoy a
semantics interpreting formulae as sets of processes that when
factored through the encoding outlined above allows an identification
of classes of knots with logical formulae. In the context of this
encoding the sub-family known as the spatial logics \cite{CairesC03}
\cite{CairesC04} \cite{Caires04} are of particular interest providing
several important features for expressing and reasoning about
properties (i.e. classes) of knots. We hint here at how this may be done.

%\begin{description}
%\item [structural connectives] 
\subsubsection{Structural connectives} The spatial logics enjoy
structural connectives corresponding, at the logical level, to the
parallel composition ($P | Q$) and new name ($(\nu \; x)P$)
connectives for processes. As illustrated in the examples below, these
connectives are extremely expressive given the shape of our encoding.
%\item [decideable satisfaction]

\subsubsection{Decideable satisfaction}
In \cite{Caires04} the satisfaction relation is shown to be decideable
for a rich class of processes. It further turns out that the image of
the our encoding is a proper subset of that class. This result
provides the basis for an algorithm by which to search for knots
enjoying a given property.
%\item [characteristic formulae]

\subsubsection{Characteristic formulae}
In the same paper \cite{Caires04} , Caires presents a means of calculating
characteristic formulae, selecting equivalence classes of processes
up to a pre--specified depth limit on the support set of names. Composed with our
encoding, this characteristic formula can be used to select
characteristic formulae for knots.
%\end{description}

\subsubsection{Spatial logic formulae}

The grammar below (segmented for comprehension) summarizes the syntax
of spatial logic formulae. We employ illustrative examples in the
sequel to provide an intuitive understanding of their meaning
referring the reader to \cite{Caires04} for a more detailed explication
of the semantics.

\begin{mathpar}
  \inferrule* [lab=boolean] {} {{A,B} \bc T \;|\; \neg A \;|\; A \wedge B \;|\; \eta = \eta'}
  \and
  \inferrule* [lab=spatial] {} {|\; \pzero \;|\; A | B \;|\; x \text{\textregistered} A \;|\; \forall x . A \;|\;  H x . A}
  \and
  \inferrule* [lab=behavioral] {} {|\; \alpha . A}
  \and 
  \inferrule* [lab=recursion] {} {|\; X(\vec{u}) \;|\; \mu X(\vec{u}) . A}
  \and
  \inferrule* [lab=action] {} {\alpha \bc \langle x?(\vec{y}) \rangle \;|\; \langle x!(\vec{y}) \rangle \;|\; \langle \tau \rangle}
  \and 
  \inferrule* [lab=name] {} {\eta \bc x \;|\; \tau}
\end{mathpar} 

% subsection characteristic_formulae (end)   	 

\subsection{Example formulae}\label{sub:example_formulae_} % (fold)

\subsubsection{Crossing as formula.}
% 
% \begin{align*}
%   \frac{d}{dx} \sin x &= \cos x 
%   & \frac{d}{dx} e^x &= e^x \\
%   \frac{d}{dx} \cos x &= - \sin x 
%   & \frac{d}{dx} \log x &= \frac{1}{x} \\
% \end{align*} 

\begin{align*}
 \mu C(x_{0},x_{1},y_{0},y_{1},u).&(\langle x_{0}?(z) \rangle(\langle u! \rangle\langle y_{1}!z \rangle C(x_{0},x_{1},y_{0},y_{1},u)) & \\
  & \wedge \langle y_{1}?(z) \rangle (\langle u! \rangle \langle x_{0}!z \rangle C(x_{0},x_{1},y_{0},y_{1},u)) & \\
  & \wedge \langle x_{1}?(z) \rangle (\langle u? \rangle \langle y_{0}!z \rangle C(x_{0},x_{1},y_{0},y_{1},u)) & \\
  & \wedge \langle y_{0}?(z) \rangle (\langle u? \rangle \langle x_{1}!z \rangle C(x_{0},x_{1},y_{0},y_{1},u))) &
\end{align*}

The lexicographical similarity between the shape of this formulae and
the shape of definition of the process representing a crossing reveals
the intuitive meaning of this formulae. It describes the capabilities
of a process that has the right to represent a crossing. For example
it picks out processes that may perform an input on the port $x_0$ in
its initial menu of capabilities. What differentiates the formula
from the process, however, is that the crossing process is the
smallest candidate to satisfy the formula. Infinitely many other
processes -- with internal behavior hidden behind this interface, so
to speak -- also satisfy this formula. Even this simple formula,
then, can be seen to open a new view onto knots, providing a
computational interpretation of \emph{virtual} knots.

Note that this formula is derived by hand. A similar formula can be
derived by employing Caires' calculation of characteristic formula
\cite{Caires04} to the process representing a crossing. In light of
this discussion, we let
$\meaningof{C}_{\phi}(x0,x1,y0,y1,u)$ denote a formula specifying the
dynamics we wish to capture of a crossing. To guarantee we preserve
the shape of the interface and minimal semantics we demand that
$\meaningof{C}_{\phi}(x0,x1,y0,y1,u) \Rightarrow
\textbf{C}(x0,x1,y0,y1,u)$ where $\textbf{C}(x0,x1,y0,y1,u)$ denotes
the formula above.
                            
\subsubsection{Crossing number constraints.}
The moral content of the context lemma (Lemma \ref{context}) is that the notion of
``locality'' in the Reidemeister moves is effectively captured by the
parallel composition operator of the process calculus. This intuition
extends through the logic. Given a formula,
$\meaningof{C}_{\phi}(x0,x1,y0,y1,u)$, we can use the structural
connectives to specify constraints on crossing numbers, such as at
least $n$ crossings, or exactly $n$ crossings.
\begin{mathpar}
  \inferrule* [lab=at-least-n] {} { K^{\geq n}_{\phi}(\vec{xs},\vec{ys}) := \Pi_{i=0}^{n-1} Hu . \meaningof{C}_{\phi}(xs_i,ys_i,u) | T }
  \and 
  \inferrule* [lab=exactly-n] {} { K^{= n}_{\phi}(\vec{xs},\vec{ys}) := \Pi_{i=0}^{n-1} Hu . \meaningof{C}_{\phi}(xs_i,ys_i,u) | \neg (\forall x_0,y_0,x_1,y_1,u . \meaningof{C}_{\phi}(x_0,y_0,x_1,y_1,u) | T) }
\end{mathpar}

To round out this section, recall that the encoding of an $n$-crossing
knot decomposes into a parallel composition of $n$ \emph{copies} of a
crossing process together with a wiring harness. To specify different
knot classes with the same crossing number amounts to specifying
logical constraints on the wiring harness. In the interest of space,
we defer examples to a forthcoming paper. Suffice it to say that both
the conditions ``alternating knot'' and ``contains the tangle
corresponding to 5/3'' are expressible. For example, it is possible to
calculate the characteristic formula of a process corresponding to the
tangle 5/3 and conjoin it into the classifying formula via the
composition connective of the logic.

Finally, we wish to observe that it is entirely within reason to
contemplate a more domain-specific version of spatial logic tailored
to the shape of processes in the image of the encoding. Such a
domain-specific logic would have a better claim to the title formal
language of knot properties.

% subsection example_formulae_ (end)

% section knots_as_processes (end) 

% section spatial logic via knots (end)

\section{Conclusions and future work}

\paragraph{Testing physical space}
You, gentle reader, may wonder why of all the theorems to be proved
given this set up we pick the one above. In some sense it's hardly
central to quantum mechanics. We see it as central in the sense that
it firmly establishes a notion of physical space arising from a notion
of the equivalence of behavior. Relating bisimulation to a metric is a
big step forward, but one is faced with interpreting the relationship
of that metric space to something more physical. Quantum mechanical
notions of ``physical'' space are still far from intuitive, but by
relating this idea of distance as testing to calculations that predict
physical circumstances we are making a not insignificant step forward
toward an understanding of the physical space we inhabit as
essentially dynamic.

\paragraph{Effectivity and simulation}
One of the observations we have yet to make is that the entire program
spelled out here is effective. We have built various interpreters for
the reflective calculus at work in this interpretation. In principle,
then, we can simulate quantum mechanics on a computer. The place where
the simulation may lose fidelity is the infinitely branching summation
for the annihilator.

In this connection i also want to point out that the evaluation style
calculation of the inner product puts the non-determinism of the
summation right at the heart of measurement. This suggests that
Milner's original reduction-based formulation of the dynamics of his
calculi in terms of sums was not just notationally suggestive of a
notion of measure-and-continue but captured some significant part of
the physics.

\paragraph{Quantum continuations}
In light of this last observation i want to point out that the
predominant account of quantum mechanics is missing a key aspect of a
truly compositional story of the physical situation. In a real lab,
when a measurement is made the observation can be made to feed into
another device that then makes another measurement conditioned on the
results of the first. This means that after the superposition was
collapsed the entire experimental set up remained in
superposition. While QM offers a means of writing this down it doesn't
quite line up well with the well-trodden formulation of computation
and continuation that we see so succinctly expressed in Milner's
calculi. This suggests that there might be advantages to this account
of dynamics waiting to be explored.

\paragraph{Quantum logic}
In this connection, we also note that by virtue of having the
Hennessy-Milner construction, we can pull the construction through the
interpretation of QM. This gives us a natural candidate for a quantum
logic that enjoys an extremely tight connection with it's domain of
interpretation, making the construction much less ad hoc (rather it is
the image of functor!).

\paragraph{Quantum probabiity}
i have questions about the basis of the interpretation of inner
product as probability amplitude. In particular, using which
axiomatization of probability theory does the notion of probability
amplitude earn the right to be so dubbed? In other words, where is the
proof that the operation for calculating a probability amplitude (and
then squaring) satisfies the axioms of what it means to calculate a
probability? Even if such a proof exists (i have yet to find it in the
literature), i wonder if it might not be possible to turn things on
their heads. Can we view the calculation of the probability amplitude
as an axiomatization of probability? If so, then the definition we
give for calculating probability amplitude may provide the basis for
an \emph{effective} theory of probability.

\paragraph{Quantum vs ``biological'' information}
Finally, i want to conclude with a more philosophical observation. At
a recent workshop in which QM was a predominant topic i noticed
something about quantum information. The speaker was giving a riveting
discussion of axiomatic QM and showing how properties of ``no
cloning'' and ``no deleting'' emerged as consequences of the
axiomatization. Theorems of this form are necessary to give us a sense
of confidence that our axioms characterize the physical theory. What
struck me, though, was that if quantum information is neither erasable
nor replicable it is markedly different from \emph{life}. Two of the
things we know about life is that

\begin{itemize}
  \item it ends;
  \item to gain some measure of persistence, to transcend it's
    finitude it is imminently copyable.
\end{itemize}

Both of these qualities are summarized succinctly in the aphorism: all
flesh is grass. For me these two kinds of ``information'' -- call them
quantum and biological -- are end points on a spectrum of strategies
for persistence. At one end, we have those curious entities that enjoy
uniqueness and permanence; at the other, we have those who in the face
of a certain end and an uncertain present make a go of passing
something on. To me one of the more remarkable aspects of the latter
strategy is that in the presence of noise (and certain features of
copying) we get a kind of dynamism, a chance for improvement against a
given persistent condition.

% subsection other_calculi_other_bisimulations_and_geometry_as_behavior (end)




% section conclusion (end)

%\documentclass[12pt]{llncs}
%\documentclass{jktr}

\usepackage[pdftex]{hyperref}                   
\usepackage {listings}
\usepackage {mathpartir}
\usepackage{bcprules}
%\usepackage{listings}
                       
\usepackage{graphicx} 
%\usepackage[margins=2.5cm,nohead,nofoot]{geometry}
%\usepackage{geometry}
\usepackage{amsfonts}
\usepackage{amstext}
\usepackage{latexsym}
\usepackage{amssymb}
\usepackage{color}


%\include{myPreamble}
\include{qm2pi.local} 

%\ifpdf
%\usepackage[pdftex]{graphicx}
%\else
%\usepackage{graphicx}
%\fi

 % \ifpdf
%  \usepackage{pdfsync}
%  \if


%\title{Brief Article}
%\author{David F. Snyder}
%\author{L.G. Meredith}

%\address{Dept. of Math., Texas State University--San Marcos, San Marcos, TX 78666}
       
\pagestyle{empty}


\begin{document}

\lstset{language=[Objective]Caml,frame=shadowbox}

\input{qm2pi.front}

% section front matter (end)

\input{qm2pi.intro} 
 
% section introduction (end)

% \input{qm2pi.knotations} 

% section notation (end)

\input{qm2pi.process.calculi} 

% section concurrent_process_calculi_and_spatial_logics_ (end)
    
%\input{qm2pi.knots2pi} 

%\input{qm2pi.trefoil} 

%\input{qm2pi.mainthm} 

% subsection basic_interpretation (end)

%\input{qm2pi.rho.presentation} 
\subsection{The syntax and semantics of the notation system}\label{sub:the_syntax_and_semantics_of_the_notation_system} % (fold)

We now summarize a technical presentation of the calculus that
embodies our theory of dynamics. The typical presentation of such a
calculus follows the style of giving generators and relations on
them. The grammar, below, describing term constructors, freely
generates the set of processes, $\Proc$. This set is then quotiented
by a relation known as structural congruence and it is over this set
that the notion of dynamics is expressed. This presentation is
essentially that of \cite{MeredithR05} with the addition of
polyadicity and summation. For readability we have relegated some of
the technical subtleties to an appendix.

\subsubsection{Process grammar}\label{subsub:process_grammar}

\begin{mathpar}
  \inferrule* [lab=synchronization] {} {{M} \bc \pzero \;|\; x?F \;|\; x!C }
  \and
  \inferrule* [lab=abstraction] {} {{F} \bc (x)P}
  \and
  \inferrule* [lab=concretion] {} {{C} \bc \langle Q \rangle}
  \and
  \inferrule* [lab=process] {} {{P,Q} \bc M \;| \;P|Q \;|\; @{x}}
  \and
  \inferrule* [lab=name] {} {{x} \bc \quotep{P}}
\end{mathpar} 

Note that $\vec{x}$ (resp. $\vec{P}$) denotes a vector of names
(resp. processes) of length $|\vec{x}|$ (resp. $|\vec{P}|$). We adopt
the following useful abbreviations.

\begin{mathpar}
   x?(\vec{y}).P := x.(\vec{y})P \and  x\clift{\vec{P}} := x.\clift{\vec{P}}
   \and x!(y) := \lift{x}{\dropn{y}}
   \and \Pi_{i=0}^{n-1}P_i := P_0 | \ldots | P_{n-1}
\end{mathpar}

\subsubsection{Structural congruence}

\paragraph{Free and bound names and alpha-equivalence.} At the
core of structural equivalence is alpha-equivalence which identifies
process that are the same up to a change of variable. Formally, we
recognize the distinction between free and bound names. The free names
of a process, $\freenames{P}$, may be calculated recursively as
follows:

\begin{mathpar}
\freenames{\pzero} := \emptyset
  \and \\
  \freenames{x?(y).P} := \{ x \} \cup (\freenames{P} \setminus \{ y \})
  \and 
  \freenames{x!\langle P \rangle} := \{ x \} \cup \{ P \} 
  \and \\
  \freenames{P|Q} := \freenames{P} \cup \freenames{Q}
  \and \\
  \freenames{@{x}} := \{ x \}
\end{mathpar}

$\pi$
$\quotep{\pi}$

$\freenames{-} : \pi \to \mathcal{P}(\quotep{\pi})$

\begin{eqnarray*}
  \freenames{\pzero} & := & \emptyset \\
  \freenames{x?(y).P} & := & \{ x \} \cup (\freenames{P} \setminus \{ y \}) \\
  \freenames{x!\langle P \rangle} & := & \{ x \} \cup \{ P \} \\
  \freenames{P|Q} & := & \freenames{P} \cup \freenames{Q} \\
  \freenames{\dropn{x}} & := & \{ x \}
\end{eqnarray*}

The bound names of a process, $\boundnames{P}$, are those names occurring in $P$
that are not free. For example, in $x?(y).0$, the name $x$ is free, while $y$ is bound.

\begin{mathpar}
  \inferrule* [lab=monoidal-laws] {} { P|Q \equiv Q|P \and P|0 \equiv P \and P|(Q|R) \equiv (P|Q)|R }
\end{mathpar}

\begin{mathpar}
  \inferrule* [lab=alpha-equivalence] {} { (x)P \equiv (y)P\{y/x\} \and y \not\in \freenames{P} }
\end{mathpar}

\begin{definition}
Then two processes, $P,Q$, are alpha-equivalent if $P = Q\{\vec{y}/\vec{x}\}$ for
some $\vec{x} \in \boundnames{Q},\vec{y} \in \boundnames{P}$, where $Q\{\vec{y}/\vec{x}\}$
denotes the capture-avoiding substitution of $\vec{y}$ for $\vec{x}$ in $Q$.
\end{definition}

\begin{definition}
  The {\em structural congruence} \cite{SangiorgiWalker} , $\equiv$,
  between processes is the least congruence containing
  alpha-equivalence, satisfying the abelian monoid laws
  (associativity, commutativity and $\pzero$ as identity) for parallel
  composition $|$ and for summation $+$.
\end{definition}

\subsection{Name equivalence}

We take name equivalence, written $\nameeq$, to be the smallest
equivalence relation generated by the following rules.

\begin{mathpar}
\inferrule*[lab=Quote-drop]
{ }
{ \quotep{@{x}} \nameeq x }

\inferrule*[lab=Struct-equiv]
{ P \scong Q }
{ \quotep{P} \nameeq \quotep{Q} }
\end{mathpar}

The astute reader will have noticed that the mutual recursion of names
and processes imposes a mutual recursion on alpha-equivalence and
structural equivalence via name-equivalence. Fortunately, all of this
works out pleasantly and we may calculate in the natural way, free of
concern. The reader interested in the details is referred to the
appendix \ref{appendix:rho_details}.

\subsection{Substitution}

We use $\Proc$ for the set of processes, $\QProc$ for the set of
names, and $\id{\{}\vec{y} / \vec{x} \id{\}}$ to denote partial maps,
$s : \QProc \rightarrow \QProc$. A map, $s$ lifts, uniquely, to a map
on process terms, $\widehat{s} : \Proc \rightarrow \Proc$ by the
following equations.

\begin{mathpar}
  (0) \psubstp{Q}{P} := 0 \\
  (R \juxtap S) \psubstp{Q}{P}
  :=    
  (R)\psubstp{Q}{P} \juxtap (S) \psubstp{Q}{P} \\
  (x?(y).R) \psubstp{Q}{P}    
  :=    
  (x)\substp{Q}{P} (z)\concat( (R \psubstn{z}{y}) \psubstp{Q}{P} ) \\
  (\lift{x}{R}) \psubstp{Q}{P}  
  :=
  \lift{(x)\substp{Q}{P}}{ R \psubstp{Q}{P} } \\
%   (\dropn{x})  \psubstp{Q}{P}       
%   := 
%   \left\{ 
%     \begin{array}{ccc} 
%       \dropn{\quotep{Q}} & & x \nameeq \quotep{P} \\
%       \dropn{x} & & otherwise \\
%     \end{array}
%   \right. 
  (\dropn{x})  \psubstp{Q}{P}       
  := 
  \left\{ 
    \begin{array}{ccc} 
      Q & & x \nameeq \quotep{P} \\
      \dropn{x} & & otherwise \\
    \end{array}
  \right.
\end{mathpar}
 

where

\begin{eqnarray}
  (x)\id{\{} \lpquote Q \rpquote / \lpquote P \rpquote \id{\}}            = 
  \left\{ 
    \begin{array}{ccc}
      \lpquote Q \rpquote & & x \nameeq \lpquote P \rpquote \\
      x & & otherwise \\
    \end{array}
  \right. \nonumber
\end{eqnarray}

and $z$ is chosen distinct from $\quotep{P}$, $\quotep{Q}$, the free
names in $Q$, and all the names in $R$. Our $\alpha$-equivalence will
be built in the standard way from this substitution.

\begin{remark}\label{rem:no_self_referential_names}
  One consequence of these definitions is that $\forall P. \quotep{P}
  \not\in \freenames{P}$.
\end{remark}

\subsection{ Dynamic quote: an example }

Anticipating something of what's to come, consider applying the
substitution, $\widehat{\id{\{}u / z \id{\}}}$, to the following pair
of processes, $\lift{w}{y!(z)}$ and $w[ \lpquote y!(z) \rpquote ]$.

\begin{eqnarray}
	\lift{w}{y!(z)}\widehat{\id{\{}u / z \id{\}}}
		& = &
		\lift{w}{y!(u)} \nonumber\\
	w[ \lpquote y!(z) \rpquote ] \widehat{ \id{\{}u / z \id{\}} }
		& = &
		w[ \lpquote y!(z) \rpquote ] \nonumber
\end{eqnarray}

Because the body of the process between quotes is impervious to
substitution, we get radically different answers. In fact, by
examining the first process in an input context,
e.g. $x?(z).\lift{w}{y!(z)}$, we see that the process under the lift
operator may be shaped by prefixed inputs binding a name inside it. In
this sense, the lift operator will be seen as a way to dynamically
construct processes before reifying them as names.

Finally equipped with these standard features we can present the
dynamics of the calculus.

\subsubsection{Operational semantics} 

Finally, we introduce the computational dynamics. What marks these
algebras as distinct from other more traditionally studied algebraic
structures, e.g. vector spaces or polynomial rings, is the manner in
which dynamics is captured. In traditional structures, dynamics is typically
expressed through morphisms between such structures, as in linear maps
between vector spaces or morphisms between rings. In algebras
associated with the semantics of computation, the dynamics is
expressed as part of the algebraic structure itself, through a
reduction reduction relation typically denoted by $\red$. Below, we
give a recursive presentation of this relation for the calculus used
in the encoding.

$\red \subseteq \pi \times \pi$
$\red : \pi \to \mathcal{P}(\pi)$

\begin{mathpar}
  \inferrule* [lab=Comm] { \textsf{match}( x_{src}, x_{trgt} ) } { x_{trgt}?(y)P \; | \; x_{src}!\langle {Q} \rangle \red P\{\quotep{Q}/y}\} }
  \and \\
  \inferrule* [lab=Par] {{P} \red {P}'} {{{P} | {Q}} \red {{P}' | {Q}}}
  \and
  \inferrule* [lab=Equiv]{{{P} \scong {P}'} \andalso {{P}' \red {Q}'} \andalso {{Q}' \scong {Q}}}{{P} \red {Q}}
\end{mathpar}

\begin{eqnarray*}
  match_{\equiv} (\quotep{P},\quotep{Q}) & := & P \equiv Q \\
  match_{\dagger}(\quotep{P},\quotep{Q}) & := & \forall R. P|Q \red^{*} R => R \red^{*} 0 \\
  match_{K}(\quotep{P},\quotep{Q}) & := & K \mbox{ for some context } K
\end{eqnarray*}

$u?(x)P | u!\langle Q \rangle \red P\{\quotep{Q}/x\}$

%We write $\wred$ for $\red^*$, and $P\red$ if $\exists Q $ such that $ P \red Q$.
We write $P\red$ if $\exists Q $ such that $ P \red Q$ and $P\not\red$, otherwise.

\section{Replication}

As mentioned before, it is known that replication (and hence
recursion) can be implemented in a higher-order process algebra
\cite{SangiorgiWalker}. As our first example of calculation with the
machinery thus far presented we give the construction explicitly in
the {\rhoc}.

\begin{eqnarray}
	D_{x} & := & \prefix{x}{y}{(\binpar{\outputp{x}{y}}{@{y}})} \nonumber\\
	\bangp_{x}{P} & := & \binpar{{x}!\langle{\binpar{D_{x}}{P}}\rangle}{D_{x}} \nonumber
\end{eqnarray}

\begin{eqnarray}
	\bangp_{x}{P} & & \nonumber\\
	=
	& {x}!\langle{(\prefix{x}{y}{(\outputp{x}{y} | @{y})) | P}}\rangle 
	      | \prefix{x}{y}{(\outputp{x}{y} | @{y})} & \nonumber\\
	\red
	& (\outputp{x}{y} | @{y})\substn{\quotep{(\prefix{x}{y}{(@{y} | \outputp{x}{y})) | P}}}{y} & \nonumber\\
	=
	& \outputp{x}{\quotep{(\prefix{x}{y}{(\outputp{x}{y} | @{y})) | P}}}
	  | {(\prefix{x}{y}{(\outputp{x}{y} | @{y})) | P}} & \nonumber\\
	\red
	& \ldots & \nonumber\\
	\red^*
	& P | P | \ldots & \nonumber
\end{eqnarray}

Of course, this encoding, as an implementation, runs away, unfolding
$\bangp{P}$ eagerly. A lazier and more implementable replication
operator, restricted to input-guarded processes, may be obtained as follows.

\begin{eqnarray}
\bangp{\prefix{u}{v}{P}} 
	:= 
	\binpar{\lift{x}{\prefix{u}{v}{(\binpar{D(x)}{P})}}}{D(x)} \nonumber
\end{eqnarray}

\begin{remark}
  Note that the lazier definition still does not deal with summation
  or mixed summation (i.e. sums over input and output). The reader is
  invited to construct definitions of replication that deal with these
  features. 

  Further, the definitions are parameterized in a name, $x$. Can you,
  gentle reader, make a definition that eliminates this parameter and
  guarantees no accidental interaction between the replication
  machinery and the process being replicated -- i.e. no accidental
  sharing of names used by the process to get its work done and the
  name(s) used by the replication to effect copying. This latter
  revision of the definition of replication is crucial to obtaining
  the expected identity $!!P \sim !P$.
\end{remark}

\begin{remark}\label{rem:paradoxical_combinator}
  The reader familiar with the lambda calculus will have noticed the
  similarity between $D$ and the paradoxical combinator.

  [Ed. note: the existence of this seems to suggest we have to be more
  restrictive on the set of processes and names we admit if we are to
  support no-cloning.]
\end{remark}

\subsubsection{Bisimulation}

The computational dynamics gives rise to another kind of equivalence,
the equivalence of computational behavior. As previously mentioned
this is typically captured \emph{via} some form of bisimulation.

% The notion we use in this paper is weak barbed bisimulation
% \cite{milner91polyadicpi}.

The notion we use in this paper is derived from weak barbed
bisimulation \cite{milner91polyadicpi}. 

\begin{definition}
An \emph{observation relation}, $\downarrow_{\mathcal N}$, over a set
of names, $\mathcal N$, is the smallest relation satisfying the rules
below.

\infrule[Out-barb]{y \in {\mathcal N}, \; x \nameeq y}
		  {\outputp{x}{v} \downarrow_{\mathcal N} x}
\infrule[Par-barb]{\mbox{$P\downarrow_{\mathcal N} x$ or $Q\downarrow_{\mathcal N} x$}}
		  {\binpar{P}{Q} \downarrow_{\mathcal N} x}

We write $P \Downarrow_{\mathcal N} x$ if there is $Q$ such that 
$P \wred Q$ and $Q \downarrow_{\mathcal N} x$.
\end{definition}

\begin{definition}
%\label{def.bbisim}
An  ${\mathcal N}$-\emph{barbed bisimulation} over a set of names, ${\mathcal N}$, is a symmetric binary relation 
${\mathcal S}_{\mathcal N}$ between agents such that $P\rel{S}_{\mathcal N}Q$ implies:
\begin{enumerate}
\item If $P \red P'$ then $Q \wred Q'$ and $P'\rel{S}_{\mathcal N} Q'$.
\item If $P\downarrow_{\mathcal N} x$, then $Q\Downarrow_{\mathcal N} x$.
\end{enumerate}
$P$ is ${\mathcal N}$-barbed bisimilar to $Q$, written
$P \wbbisim_{\mathcal N} Q$, if $P \rel{S}_{\mathcal N} Q$ for some ${\mathcal N}$-barbed bisimulation ${\mathcal S}_{\mathcal N}$.
\end{definition}

$\mathcal{R} \subseteq \pi \times \pi$

$P \mathcal{R} Q => \forall P'. P \red P' \Rightarrow \exists Q'. Q \red Q', P' \mathcal{R} Q'$

$P \vdash x \Rightarrow Q \vdash x$

\begin{mathpar}
  \inferrule*[lab=Out-barb]{x \nameeq y}{{y}!\langle{Q}\rangle \vdash x}
  \and
  \inferrule*[lab=Par-barb]{\mbox{$P\vdash x$ or $Q\vdash x$}}{\binpar{P}{Q} \vdash x}
\end{mathpar}

\subsubsection{Contexts}

One of the principle advantages of computational calculi like the
$\pi$-calculus is a well-defined notion of context,
contextual-equivalence and a correlation between
contextual-equivalence and notions of bisimulation. The notion of
context allows the decomposition of a process into (sub-)process and
its syntactic environment, its context. Thus, a context may be
thought of as a process with a ``hole'' (written $\Box$) in it. The
application of a context $M$ to a process $P$, written $M[P]$, is
tantamount to filling the hole in $M$ with $P$. In this paper we do
not need the full weight of this theory, but do make use of the notion
of context in the proof the main theorem. 

\begin{mathpar}
  \inferrule* [lab=summation] {} {{M_{M},M_{N}} \bc \Box \;|\; x.M_{A} \;|\; M_{M}+M_{N}}
  \and
  \inferrule* [lab=agent] {} {{M_{A}} \bc (\vec{x})M_{P} \;| \; \clift{P_0,\ldots,M_{P},\ldots,P_N}}
  \and \\
  \inferrule* [lab=process] {} {{M_{P}} \bc M_{N} \;| \;P|M_{P} }
\end{mathpar} 

\begin{mathpar}
  \inferrule* [lab=sychronization] {} {M_{N} \bc \Box \;|\; x?M_{F} \;|\; x!M_{C}}
  \and
  \inferrule* [lab=abstraction] {} {{M_{F}} \bc (x)M_{P} }
  \and
  \inferrule* [lab=concretion] {} {{M_{C}} \bc \langle M_{P} \rangle }
  \and \\
  \inferrule* [lab=process] {} {{M_{P}} \bc M_{N} \;| \;P|M_{P} }
\end{mathpar}

\begin{definition}[contextual application] Given a context $M$, and
  process $P$, we define the \emph{contextual application}, $M[P] :=
  M\{P/\Box\}$. That is, the contextual application of M to P is the
  substitution of $P$ for $\Box$ in $M$.
\end{definition}

$\meaningof{-} : L \to \mathcal{P}(\pi)$

\begin{mathpar}
  \inferrule* [lab=collection] {} {\meaningof{true} = \pi, \and \meaningof{~E} = \pi \setminus \meaningof{E}, \and \meaningof{E_{1} \& E_{2}} = \meaningof{E_{1}} \cap \meaningof{E_{2}}}
\end{mathpar}

\begin{mathpar}
  \inferrule* [lab=structure] {} {\meaningof{0} = \{ P \in \pi | P \equiv 0 \}, \and \\ \meaningof{E_1 | E_2} = \{ P \in \pi | P \equiv P_{1} | P_{2}, P_{1} \in \meaningof{E_{1}}, P_{2} \in \meaningof{E_2}\} }
\end{mathpar}

\begin{mathpar}
 \inferrule* [lab=behavior] {} {\meaningof{\langle a?b \rangle E} = \{ P \in \pi | P \equiv Q | u?(y)P', \\ \and \\\\ \and \\ \;\;\; u \in \meaningof{a}, \forall z.P'\{z/y\} \in \meaningof{E\{z/b\}}\}, \and \\ \meaningof{a!E} = \{ P \in \pi | P \equiv Q | x!\langle P' \rangle, x \in \meaningof{a} P' \in \meaningof{E}\} }
\end{mathpar}

\begin{mathpar}
 \inferrule* [lab=nominal] {} {\meaningof{\quotep{E}} = \{ \quotep{P} \in \quotep{\pi} | P \in \meaningof{E} \}, \and \meaningof{\quotep{P}} = \{ \quotep{Q} \in \quotep{\pi} | P \equiv Q \} \and \\ \meaningof{@\quotep{E}} = \{ P \in \pi | P \equiv @x, x \in \meaningof{E} \}}
\end{mathpar}

\begin{eqnarray*}
  \\
  \meaningof{-} : TS \to ST
\end{eqnarray*}

\begin{eqnarray*}
  \\
  L : TS \to ST
\end{eqnarray*}

\begin{eqnarray*}
  \\
  P \models E \iff P \in \meaningof{E}
\end{eqnarray*}

\begin{eqnarray*}
  P \approx_{L} Q \iff \forall E \in L. P \models E \iff Q \models E
\end{eqnarray*}

\begin{eqnarray*}
  P \approx_{K} Q
\end{eqnarray*}

\begin{eqnarray*}
  P \approx Q
\end{eqnarray*}

$\approx_{K} = \approx = \approx_{L}$

\subsubsection{Contextual duality}

Note that contexts extend the quotation operation to a family of
operations from processes to names. Given a context, $M$, we can
define a \emph{nominal context}, $\quotep{M}$ by $\quotep{M}[P] :=
\quotep{M[P]}$. To foreshadow what is to come we observe that these
operations enjoy a duality with processes very much like the duality
between vectors and maps from vectors to scalars.

Further, because the calculus is essentially higher-order, we have a
correspondence between contexts and processes. More specifically,
given a name $x$ and a context $M$ we can construct $M^{*}_{x}$ such
that 

\begin{mathpar}
  M^{*}_{x} | \lift{x}{P} \red M[P]
\end{mathpar}

namely,

\begin{mathpar}
  M^{*}_{x} := x?(u).M[\dropn{u}]
\end{mathpar}

The dependence of $M^{*}_{x}$ on a name makes it an abstraction, 

\begin{mathpar}
  M^{*} := (x)x?(u).M[\dropn{u}]
\end{mathpar}

\subsection{Additional notation}

It will sometimes be convenient to denote the process a name
quotes. We already have the notation $x = \quotep{P}$, but it will be
convenient to introduce an alternate notation, $\procn{x}$, when we
want to emphasize the connection to the use of the name. Note that, by
virtue of name equivalence, $\quotep{\procn{x}} \nameeq x$; so, the
notation is consistent with previous definitions.

Further, because names have structure it is possible to effect
substitutions on the basis of that structure. This means we need to
upgrade our notation for substitutions, which we accomplish by
adapting comprehension notation. Thus,

\begin{mathpar}
  P\{ y / x : x \in S \}
\end{mathpar}

is interpreted to mean the process derived from P by replacing (in a
capture-avoiding manner) each occurrence of $x$ in $S$ by $y$. For example,

\begin{mathpar}
  P\{ \quotep{\procn{x}|\procn{x}} / x : x \in \freenames{P} \}
\end{mathpar}

will replace each (occurrence) of a free name $x$ in $P$ by
$\quotep{\procn{x}|\procn{x}}$.

Also, we will avail ourselves of the notation $x^{L}$ and $x^{R}$ to
denote injections of a name into disjoint copies of the name
space. There are numerous ways to accomplish this. One example can be
found in \cite{MeredithR05}. This notation overloads to vectors of
names: $\vec{x}^{\pi} := (x_{i}^{\pi} \; : \; 0 \leq i < |\vec{x}| )$ where $\pi \in \{L,R\}$.

We also use $P^{\Box} := P|\Box$.

In \cite{MeredithR05} an interpretation of the new operator is
given. It turns out that there are several possible interpretations
all enjoying the requisite algebraic properties of the operator (see
\cite{milner91polyadicpi}). We will therefore make liberal use of
$(\nu\; \vec{x})P$.

% subsection the_syntax_and_semantics_of_the_notation_system (end)   

\input{qm2pi.qmops} 

\input{qm2pi.sterngerlach} 

\input{qm2pi.metric} 

% section concurrent_process_calculi (end)

%\input{qm2pi.proofsketch}

% section proof sketch (end)

%\input{qm2pi.slviaknots} 

% section spatial logic via knots (end)

\input{qm2pi.conclusion}

% section conclusion (end)

%\input{qm2pi.dtcodes} 

% section wiring algorithm (end)

\input{qm2pi.ack} 

% section acknowledgments (end)

\newpage


\bibliographystyle{plain}   
\bibliography{../../biblios/main.bib}

\input{qm2pi.rhodetails}

\end{document}

 

% section wiring algorithm (end)

\documentclass[12pt]{llncs}
%\documentclass{jktr}

\usepackage[pdftex]{hyperref}                   
\usepackage {listings}
\usepackage {mathpartir}
\usepackage{bcprules}
%\usepackage{listings}
                       
\usepackage{graphicx} 
%\usepackage[margins=2.5cm,nohead,nofoot]{geometry}
%\usepackage{geometry}
\usepackage{amsfonts}
\usepackage{amstext}
\usepackage{latexsym}
\usepackage{amssymb}
\usepackage{color}


%\include{myPreamble}
\include{qm2pi.local} 

%\ifpdf
%\usepackage[pdftex]{graphicx}
%\else
%\usepackage{graphicx}
%\fi

 % \ifpdf
%  \usepackage{pdfsync}
%  \if


%\title{Brief Article}
%\author{David F. Snyder}
%\author{L.G. Meredith}

%\address{Dept. of Math., Texas State University--San Marcos, San Marcos, TX 78666}
       
\pagestyle{empty}


\begin{document}

\lstset{language=[Objective]Caml,frame=shadowbox}

\input{qm2pi.front}

% section front matter (end)

\input{qm2pi.intro} 
 
% section introduction (end)

% \input{qm2pi.knotations} 

% section notation (end)

\input{qm2pi.process.calculi} 

% section concurrent_process_calculi_and_spatial_logics_ (end)
    
%\input{qm2pi.knots2pi} 

%\input{qm2pi.trefoil} 

%\input{qm2pi.mainthm} 

% subsection basic_interpretation (end)

%\input{qm2pi.rho.presentation} 
\subsection{The syntax and semantics of the notation system}\label{sub:the_syntax_and_semantics_of_the_notation_system} % (fold)

We now summarize a technical presentation of the calculus that
embodies our theory of dynamics. The typical presentation of such a
calculus follows the style of giving generators and relations on
them. The grammar, below, describing term constructors, freely
generates the set of processes, $\Proc$. This set is then quotiented
by a relation known as structural congruence and it is over this set
that the notion of dynamics is expressed. This presentation is
essentially that of \cite{MeredithR05} with the addition of
polyadicity and summation. For readability we have relegated some of
the technical subtleties to an appendix.

\subsubsection{Process grammar}\label{subsub:process_grammar}

\begin{mathpar}
  \inferrule* [lab=synchronization] {} {{M} \bc \pzero \;|\; x?F \;|\; x!C }
  \and
  \inferrule* [lab=abstraction] {} {{F} \bc (x)P}
  \and
  \inferrule* [lab=concretion] {} {{C} \bc \langle Q \rangle}
  \and
  \inferrule* [lab=process] {} {{P,Q} \bc M \;| \;P|Q \;|\; @{x}}
  \and
  \inferrule* [lab=name] {} {{x} \bc \quotep{P}}
\end{mathpar} 

Note that $\vec{x}$ (resp. $\vec{P}$) denotes a vector of names
(resp. processes) of length $|\vec{x}|$ (resp. $|\vec{P}|$). We adopt
the following useful abbreviations.

\begin{mathpar}
   x?(\vec{y}).P := x.(\vec{y})P \and  x\clift{\vec{P}} := x.\clift{\vec{P}}
   \and x!(y) := \lift{x}{\dropn{y}}
   \and \Pi_{i=0}^{n-1}P_i := P_0 | \ldots | P_{n-1}
\end{mathpar}

\subsubsection{Structural congruence}

\paragraph{Free and bound names and alpha-equivalence.} At the
core of structural equivalence is alpha-equivalence which identifies
process that are the same up to a change of variable. Formally, we
recognize the distinction between free and bound names. The free names
of a process, $\freenames{P}$, may be calculated recursively as
follows:

\begin{mathpar}
\freenames{\pzero} := \emptyset
  \and \\
  \freenames{x?(y).P} := \{ x \} \cup (\freenames{P} \setminus \{ y \})
  \and 
  \freenames{x!\langle P \rangle} := \{ x \} \cup \{ P \} 
  \and \\
  \freenames{P|Q} := \freenames{P} \cup \freenames{Q}
  \and \\
  \freenames{@{x}} := \{ x \}
\end{mathpar}

$\pi$
$\quotep{\pi}$

$\freenames{-} : \pi \to \mathcal{P}(\quotep{\pi})$

\begin{eqnarray*}
  \freenames{\pzero} & := & \emptyset \\
  \freenames{x?(y).P} & := & \{ x \} \cup (\freenames{P} \setminus \{ y \}) \\
  \freenames{x!\langle P \rangle} & := & \{ x \} \cup \{ P \} \\
  \freenames{P|Q} & := & \freenames{P} \cup \freenames{Q} \\
  \freenames{\dropn{x}} & := & \{ x \}
\end{eqnarray*}

The bound names of a process, $\boundnames{P}$, are those names occurring in $P$
that are not free. For example, in $x?(y).0$, the name $x$ is free, while $y$ is bound.

\begin{mathpar}
  \inferrule* [lab=monoidal-laws] {} { P|Q \equiv Q|P \and P|0 \equiv P \and P|(Q|R) \equiv (P|Q)|R }
\end{mathpar}

\begin{mathpar}
  \inferrule* [lab=alpha-equivalence] {} { (x)P \equiv (y)P\{y/x\} \and y \not\in \freenames{P} }
\end{mathpar}

\begin{definition}
Then two processes, $P,Q$, are alpha-equivalent if $P = Q\{\vec{y}/\vec{x}\}$ for
some $\vec{x} \in \boundnames{Q},\vec{y} \in \boundnames{P}$, where $Q\{\vec{y}/\vec{x}\}$
denotes the capture-avoiding substitution of $\vec{y}$ for $\vec{x}$ in $Q$.
\end{definition}

\begin{definition}
  The {\em structural congruence} \cite{SangiorgiWalker} , $\equiv$,
  between processes is the least congruence containing
  alpha-equivalence, satisfying the abelian monoid laws
  (associativity, commutativity and $\pzero$ as identity) for parallel
  composition $|$ and for summation $+$.
\end{definition}

\subsection{Name equivalence}

We take name equivalence, written $\nameeq$, to be the smallest
equivalence relation generated by the following rules.

\begin{mathpar}
\inferrule*[lab=Quote-drop]
{ }
{ \quotep{@{x}} \nameeq x }

\inferrule*[lab=Struct-equiv]
{ P \scong Q }
{ \quotep{P} \nameeq \quotep{Q} }
\end{mathpar}

The astute reader will have noticed that the mutual recursion of names
and processes imposes a mutual recursion on alpha-equivalence and
structural equivalence via name-equivalence. Fortunately, all of this
works out pleasantly and we may calculate in the natural way, free of
concern. The reader interested in the details is referred to the
appendix \ref{appendix:rho_details}.

\subsection{Substitution}

We use $\Proc$ for the set of processes, $\QProc$ for the set of
names, and $\id{\{}\vec{y} / \vec{x} \id{\}}$ to denote partial maps,
$s : \QProc \rightarrow \QProc$. A map, $s$ lifts, uniquely, to a map
on process terms, $\widehat{s} : \Proc \rightarrow \Proc$ by the
following equations.

\begin{mathpar}
  (0) \psubstp{Q}{P} := 0 \\
  (R \juxtap S) \psubstp{Q}{P}
  :=    
  (R)\psubstp{Q}{P} \juxtap (S) \psubstp{Q}{P} \\
  (x?(y).R) \psubstp{Q}{P}    
  :=    
  (x)\substp{Q}{P} (z)\concat( (R \psubstn{z}{y}) \psubstp{Q}{P} ) \\
  (\lift{x}{R}) \psubstp{Q}{P}  
  :=
  \lift{(x)\substp{Q}{P}}{ R \psubstp{Q}{P} } \\
%   (\dropn{x})  \psubstp{Q}{P}       
%   := 
%   \left\{ 
%     \begin{array}{ccc} 
%       \dropn{\quotep{Q}} & & x \nameeq \quotep{P} \\
%       \dropn{x} & & otherwise \\
%     \end{array}
%   \right. 
  (\dropn{x})  \psubstp{Q}{P}       
  := 
  \left\{ 
    \begin{array}{ccc} 
      Q & & x \nameeq \quotep{P} \\
      \dropn{x} & & otherwise \\
    \end{array}
  \right.
\end{mathpar}
 

where

\begin{eqnarray}
  (x)\id{\{} \lpquote Q \rpquote / \lpquote P \rpquote \id{\}}            = 
  \left\{ 
    \begin{array}{ccc}
      \lpquote Q \rpquote & & x \nameeq \lpquote P \rpquote \\
      x & & otherwise \\
    \end{array}
  \right. \nonumber
\end{eqnarray}

and $z$ is chosen distinct from $\quotep{P}$, $\quotep{Q}$, the free
names in $Q$, and all the names in $R$. Our $\alpha$-equivalence will
be built in the standard way from this substitution.

\begin{remark}\label{rem:no_self_referential_names}
  One consequence of these definitions is that $\forall P. \quotep{P}
  \not\in \freenames{P}$.
\end{remark}

\subsection{ Dynamic quote: an example }

Anticipating something of what's to come, consider applying the
substitution, $\widehat{\id{\{}u / z \id{\}}}$, to the following pair
of processes, $\lift{w}{y!(z)}$ and $w[ \lpquote y!(z) \rpquote ]$.

\begin{eqnarray}
	\lift{w}{y!(z)}\widehat{\id{\{}u / z \id{\}}}
		& = &
		\lift{w}{y!(u)} \nonumber\\
	w[ \lpquote y!(z) \rpquote ] \widehat{ \id{\{}u / z \id{\}} }
		& = &
		w[ \lpquote y!(z) \rpquote ] \nonumber
\end{eqnarray}

Because the body of the process between quotes is impervious to
substitution, we get radically different answers. In fact, by
examining the first process in an input context,
e.g. $x?(z).\lift{w}{y!(z)}$, we see that the process under the lift
operator may be shaped by prefixed inputs binding a name inside it. In
this sense, the lift operator will be seen as a way to dynamically
construct processes before reifying them as names.

Finally equipped with these standard features we can present the
dynamics of the calculus.

\subsubsection{Operational semantics} 

Finally, we introduce the computational dynamics. What marks these
algebras as distinct from other more traditionally studied algebraic
structures, e.g. vector spaces or polynomial rings, is the manner in
which dynamics is captured. In traditional structures, dynamics is typically
expressed through morphisms between such structures, as in linear maps
between vector spaces or morphisms between rings. In algebras
associated with the semantics of computation, the dynamics is
expressed as part of the algebraic structure itself, through a
reduction reduction relation typically denoted by $\red$. Below, we
give a recursive presentation of this relation for the calculus used
in the encoding.

$\red \subseteq \pi \times \pi$
$\red : \pi \to \mathcal{P}(\pi)$

\begin{mathpar}
  \inferrule* [lab=Comm] { \textsf{match}( x_{src}, x_{trgt} ) } { x_{trgt}?(y)P \; | \; x_{src}!\langle {Q} \rangle \red P\{\quotep{Q}/y}\} }
  \and \\
  \inferrule* [lab=Par] {{P} \red {P}'} {{{P} | {Q}} \red {{P}' | {Q}}}
  \and
  \inferrule* [lab=Equiv]{{{P} \scong {P}'} \andalso {{P}' \red {Q}'} \andalso {{Q}' \scong {Q}}}{{P} \red {Q}}
\end{mathpar}

\begin{eqnarray*}
  match_{\equiv} (\quotep{P},\quotep{Q}) & := & P \equiv Q \\
  match_{\dagger}(\quotep{P},\quotep{Q}) & := & \forall R. P|Q \red^{*} R => R \red^{*} 0 \\
  match_{K}(\quotep{P},\quotep{Q}) & := & K \mbox{ for some context } K
\end{eqnarray*}

$u?(x)P | u!\langle Q \rangle \red P\{\quotep{Q}/x\}$

%We write $\wred$ for $\red^*$, and $P\red$ if $\exists Q $ such that $ P \red Q$.
We write $P\red$ if $\exists Q $ such that $ P \red Q$ and $P\not\red$, otherwise.

\section{Replication}

As mentioned before, it is known that replication (and hence
recursion) can be implemented in a higher-order process algebra
\cite{SangiorgiWalker}. As our first example of calculation with the
machinery thus far presented we give the construction explicitly in
the {\rhoc}.

\begin{eqnarray}
	D_{x} & := & \prefix{x}{y}{(\binpar{\outputp{x}{y}}{@{y}})} \nonumber\\
	\bangp_{x}{P} & := & \binpar{{x}!\langle{\binpar{D_{x}}{P}}\rangle}{D_{x}} \nonumber
\end{eqnarray}

\begin{eqnarray}
	\bangp_{x}{P} & & \nonumber\\
	=
	& {x}!\langle{(\prefix{x}{y}{(\outputp{x}{y} | @{y})) | P}}\rangle 
	      | \prefix{x}{y}{(\outputp{x}{y} | @{y})} & \nonumber\\
	\red
	& (\outputp{x}{y} | @{y})\substn{\quotep{(\prefix{x}{y}{(@{y} | \outputp{x}{y})) | P}}}{y} & \nonumber\\
	=
	& \outputp{x}{\quotep{(\prefix{x}{y}{(\outputp{x}{y} | @{y})) | P}}}
	  | {(\prefix{x}{y}{(\outputp{x}{y} | @{y})) | P}} & \nonumber\\
	\red
	& \ldots & \nonumber\\
	\red^*
	& P | P | \ldots & \nonumber
\end{eqnarray}

Of course, this encoding, as an implementation, runs away, unfolding
$\bangp{P}$ eagerly. A lazier and more implementable replication
operator, restricted to input-guarded processes, may be obtained as follows.

\begin{eqnarray}
\bangp{\prefix{u}{v}{P}} 
	:= 
	\binpar{\lift{x}{\prefix{u}{v}{(\binpar{D(x)}{P})}}}{D(x)} \nonumber
\end{eqnarray}

\begin{remark}
  Note that the lazier definition still does not deal with summation
  or mixed summation (i.e. sums over input and output). The reader is
  invited to construct definitions of replication that deal with these
  features. 

  Further, the definitions are parameterized in a name, $x$. Can you,
  gentle reader, make a definition that eliminates this parameter and
  guarantees no accidental interaction between the replication
  machinery and the process being replicated -- i.e. no accidental
  sharing of names used by the process to get its work done and the
  name(s) used by the replication to effect copying. This latter
  revision of the definition of replication is crucial to obtaining
  the expected identity $!!P \sim !P$.
\end{remark}

\begin{remark}\label{rem:paradoxical_combinator}
  The reader familiar with the lambda calculus will have noticed the
  similarity between $D$ and the paradoxical combinator.

  [Ed. note: the existence of this seems to suggest we have to be more
  restrictive on the set of processes and names we admit if we are to
  support no-cloning.]
\end{remark}

\subsubsection{Bisimulation}

The computational dynamics gives rise to another kind of equivalence,
the equivalence of computational behavior. As previously mentioned
this is typically captured \emph{via} some form of bisimulation.

% The notion we use in this paper is weak barbed bisimulation
% \cite{milner91polyadicpi}.

The notion we use in this paper is derived from weak barbed
bisimulation \cite{milner91polyadicpi}. 

\begin{definition}
An \emph{observation relation}, $\downarrow_{\mathcal N}$, over a set
of names, $\mathcal N$, is the smallest relation satisfying the rules
below.

\infrule[Out-barb]{y \in {\mathcal N}, \; x \nameeq y}
		  {\outputp{x}{v} \downarrow_{\mathcal N} x}
\infrule[Par-barb]{\mbox{$P\downarrow_{\mathcal N} x$ or $Q\downarrow_{\mathcal N} x$}}
		  {\binpar{P}{Q} \downarrow_{\mathcal N} x}

We write $P \Downarrow_{\mathcal N} x$ if there is $Q$ such that 
$P \wred Q$ and $Q \downarrow_{\mathcal N} x$.
\end{definition}

\begin{definition}
%\label{def.bbisim}
An  ${\mathcal N}$-\emph{barbed bisimulation} over a set of names, ${\mathcal N}$, is a symmetric binary relation 
${\mathcal S}_{\mathcal N}$ between agents such that $P\rel{S}_{\mathcal N}Q$ implies:
\begin{enumerate}
\item If $P \red P'$ then $Q \wred Q'$ and $P'\rel{S}_{\mathcal N} Q'$.
\item If $P\downarrow_{\mathcal N} x$, then $Q\Downarrow_{\mathcal N} x$.
\end{enumerate}
$P$ is ${\mathcal N}$-barbed bisimilar to $Q$, written
$P \wbbisim_{\mathcal N} Q$, if $P \rel{S}_{\mathcal N} Q$ for some ${\mathcal N}$-barbed bisimulation ${\mathcal S}_{\mathcal N}$.
\end{definition}

$\mathcal{R} \subseteq \pi \times \pi$

$P \mathcal{R} Q => \forall P'. P \red P' \Rightarrow \exists Q'. Q \red Q', P' \mathcal{R} Q'$

$P \vdash x \Rightarrow Q \vdash x$

\begin{mathpar}
  \inferrule*[lab=Out-barb]{x \nameeq y}{{y}!\langle{Q}\rangle \vdash x}
  \and
  \inferrule*[lab=Par-barb]{\mbox{$P\vdash x$ or $Q\vdash x$}}{\binpar{P}{Q} \vdash x}
\end{mathpar}

\subsubsection{Contexts}

One of the principle advantages of computational calculi like the
$\pi$-calculus is a well-defined notion of context,
contextual-equivalence and a correlation between
contextual-equivalence and notions of bisimulation. The notion of
context allows the decomposition of a process into (sub-)process and
its syntactic environment, its context. Thus, a context may be
thought of as a process with a ``hole'' (written $\Box$) in it. The
application of a context $M$ to a process $P$, written $M[P]$, is
tantamount to filling the hole in $M$ with $P$. In this paper we do
not need the full weight of this theory, but do make use of the notion
of context in the proof the main theorem. 

\begin{mathpar}
  \inferrule* [lab=summation] {} {{M_{M},M_{N}} \bc \Box \;|\; x.M_{A} \;|\; M_{M}+M_{N}}
  \and
  \inferrule* [lab=agent] {} {{M_{A}} \bc (\vec{x})M_{P} \;| \; \clift{P_0,\ldots,M_{P},\ldots,P_N}}
  \and \\
  \inferrule* [lab=process] {} {{M_{P}} \bc M_{N} \;| \;P|M_{P} }
\end{mathpar} 

\begin{mathpar}
  \inferrule* [lab=sychronization] {} {M_{N} \bc \Box \;|\; x?M_{F} \;|\; x!M_{C}}
  \and
  \inferrule* [lab=abstraction] {} {{M_{F}} \bc (x)M_{P} }
  \and
  \inferrule* [lab=concretion] {} {{M_{C}} \bc \langle M_{P} \rangle }
  \and \\
  \inferrule* [lab=process] {} {{M_{P}} \bc M_{N} \;| \;P|M_{P} }
\end{mathpar}

\begin{definition}[contextual application] Given a context $M$, and
  process $P$, we define the \emph{contextual application}, $M[P] :=
  M\{P/\Box\}$. That is, the contextual application of M to P is the
  substitution of $P$ for $\Box$ in $M$.
\end{definition}

$\meaningof{-} : L \to \mathcal{P}(\pi)$

\begin{mathpar}
  \inferrule* [lab=collection] {} {\meaningof{true} = \pi, \and \meaningof{~E} = \pi \setminus \meaningof{E}, \and \meaningof{E_{1} \& E_{2}} = \meaningof{E_{1}} \cap \meaningof{E_{2}}}
\end{mathpar}

\begin{mathpar}
  \inferrule* [lab=structure] {} {\meaningof{0} = \{ P \in \pi | P \equiv 0 \}, \and \\ \meaningof{E_1 | E_2} = \{ P \in \pi | P \equiv P_{1} | P_{2}, P_{1} \in \meaningof{E_{1}}, P_{2} \in \meaningof{E_2}\} }
\end{mathpar}

\begin{mathpar}
 \inferrule* [lab=behavior] {} {\meaningof{\langle a?b \rangle E} = \{ P \in \pi | P \equiv Q | u?(y)P', \\ \and \\\\ \and \\ \;\;\; u \in \meaningof{a}, \forall z.P'\{z/y\} \in \meaningof{E\{z/b\}}\}, \and \\ \meaningof{a!E} = \{ P \in \pi | P \equiv Q | x!\langle P' \rangle, x \in \meaningof{a} P' \in \meaningof{E}\} }
\end{mathpar}

\begin{mathpar}
 \inferrule* [lab=nominal] {} {\meaningof{\quotep{E}} = \{ \quotep{P} \in \quotep{\pi} | P \in \meaningof{E} \}, \and \meaningof{\quotep{P}} = \{ \quotep{Q} \in \quotep{\pi} | P \equiv Q \} \and \\ \meaningof{@\quotep{E}} = \{ P \in \pi | P \equiv @x, x \in \meaningof{E} \}}
\end{mathpar}

\begin{eqnarray*}
  \\
  \meaningof{-} : TS \to ST
\end{eqnarray*}

\begin{eqnarray*}
  \\
  L : TS \to ST
\end{eqnarray*}

\begin{eqnarray*}
  \\
  P \models E \iff P \in \meaningof{E}
\end{eqnarray*}

\begin{eqnarray*}
  P \approx_{L} Q \iff \forall E \in L. P \models E \iff Q \models E
\end{eqnarray*}

\begin{eqnarray*}
  P \approx_{K} Q
\end{eqnarray*}

\begin{eqnarray*}
  P \approx Q
\end{eqnarray*}

$\approx_{K} = \approx = \approx_{L}$

\subsubsection{Contextual duality}

Note that contexts extend the quotation operation to a family of
operations from processes to names. Given a context, $M$, we can
define a \emph{nominal context}, $\quotep{M}$ by $\quotep{M}[P] :=
\quotep{M[P]}$. To foreshadow what is to come we observe that these
operations enjoy a duality with processes very much like the duality
between vectors and maps from vectors to scalars.

Further, because the calculus is essentially higher-order, we have a
correspondence between contexts and processes. More specifically,
given a name $x$ and a context $M$ we can construct $M^{*}_{x}$ such
that 

\begin{mathpar}
  M^{*}_{x} | \lift{x}{P} \red M[P]
\end{mathpar}

namely,

\begin{mathpar}
  M^{*}_{x} := x?(u).M[\dropn{u}]
\end{mathpar}

The dependence of $M^{*}_{x}$ on a name makes it an abstraction, 

\begin{mathpar}
  M^{*} := (x)x?(u).M[\dropn{u}]
\end{mathpar}

\subsection{Additional notation}

It will sometimes be convenient to denote the process a name
quotes. We already have the notation $x = \quotep{P}$, but it will be
convenient to introduce an alternate notation, $\procn{x}$, when we
want to emphasize the connection to the use of the name. Note that, by
virtue of name equivalence, $\quotep{\procn{x}} \nameeq x$; so, the
notation is consistent with previous definitions.

Further, because names have structure it is possible to effect
substitutions on the basis of that structure. This means we need to
upgrade our notation for substitutions, which we accomplish by
adapting comprehension notation. Thus,

\begin{mathpar}
  P\{ y / x : x \in S \}
\end{mathpar}

is interpreted to mean the process derived from P by replacing (in a
capture-avoiding manner) each occurrence of $x$ in $S$ by $y$. For example,

\begin{mathpar}
  P\{ \quotep{\procn{x}|\procn{x}} / x : x \in \freenames{P} \}
\end{mathpar}

will replace each (occurrence) of a free name $x$ in $P$ by
$\quotep{\procn{x}|\procn{x}}$.

Also, we will avail ourselves of the notation $x^{L}$ and $x^{R}$ to
denote injections of a name into disjoint copies of the name
space. There are numerous ways to accomplish this. One example can be
found in \cite{MeredithR05}. This notation overloads to vectors of
names: $\vec{x}^{\pi} := (x_{i}^{\pi} \; : \; 0 \leq i < |\vec{x}| )$ where $\pi \in \{L,R\}$.

We also use $P^{\Box} := P|\Box$.

In \cite{MeredithR05} an interpretation of the new operator is
given. It turns out that there are several possible interpretations
all enjoying the requisite algebraic properties of the operator (see
\cite{milner91polyadicpi}). We will therefore make liberal use of
$(\nu\; \vec{x})P$.

% subsection the_syntax_and_semantics_of_the_notation_system (end)   

\input{qm2pi.qmops} 

\input{qm2pi.sterngerlach} 

\input{qm2pi.metric} 

% section concurrent_process_calculi (end)

%\input{qm2pi.proofsketch}

% section proof sketch (end)

%\input{qm2pi.slviaknots} 

% section spatial logic via knots (end)

\input{qm2pi.conclusion}

% section conclusion (end)

%\input{qm2pi.dtcodes} 

% section wiring algorithm (end)

\input{qm2pi.ack} 

% section acknowledgments (end)

\newpage


\bibliographystyle{plain}   
\bibliography{../../biblios/main.bib}

\input{qm2pi.rhodetails}

\end{document}

 

% section acknowledgments (end)

\newpage


\bibliographystyle{plain}   
\bibliography{../../biblios/main.bib}

\documentclass[12pt]{llncs}
%\documentclass{jktr}

\usepackage[pdftex]{hyperref}                   
\usepackage {listings}
\usepackage {mathpartir}
\usepackage{bcprules}
%\usepackage{listings}
                       
\usepackage{graphicx} 
%\usepackage[margins=2.5cm,nohead,nofoot]{geometry}
%\usepackage{geometry}
\usepackage{amsfonts}
\usepackage{amstext}
\usepackage{latexsym}
\usepackage{amssymb}
\usepackage{color}


%\include{myPreamble}
\include{qm2pi.local} 

%\ifpdf
%\usepackage[pdftex]{graphicx}
%\else
%\usepackage{graphicx}
%\fi

 % \ifpdf
%  \usepackage{pdfsync}
%  \if


%\title{Brief Article}
%\author{David F. Snyder}
%\author{L.G. Meredith}

%\address{Dept. of Math., Texas State University--San Marcos, San Marcos, TX 78666}
       
\pagestyle{empty}


\begin{document}

\lstset{language=[Objective]Caml,frame=shadowbox}

\input{qm2pi.front}

% section front matter (end)

\input{qm2pi.intro} 
 
% section introduction (end)

% \input{qm2pi.knotations} 

% section notation (end)

\input{qm2pi.process.calculi} 

% section concurrent_process_calculi_and_spatial_logics_ (end)
    
%\input{qm2pi.knots2pi} 

%\input{qm2pi.trefoil} 

%\input{qm2pi.mainthm} 

% subsection basic_interpretation (end)

%\input{qm2pi.rho.presentation} 
\subsection{The syntax and semantics of the notation system}\label{sub:the_syntax_and_semantics_of_the_notation_system} % (fold)

We now summarize a technical presentation of the calculus that
embodies our theory of dynamics. The typical presentation of such a
calculus follows the style of giving generators and relations on
them. The grammar, below, describing term constructors, freely
generates the set of processes, $\Proc$. This set is then quotiented
by a relation known as structural congruence and it is over this set
that the notion of dynamics is expressed. This presentation is
essentially that of \cite{MeredithR05} with the addition of
polyadicity and summation. For readability we have relegated some of
the technical subtleties to an appendix.

\subsubsection{Process grammar}\label{subsub:process_grammar}

\begin{mathpar}
  \inferrule* [lab=synchronization] {} {{M} \bc \pzero \;|\; x?F \;|\; x!C }
  \and
  \inferrule* [lab=abstraction] {} {{F} \bc (x)P}
  \and
  \inferrule* [lab=concretion] {} {{C} \bc \langle Q \rangle}
  \and
  \inferrule* [lab=process] {} {{P,Q} \bc M \;| \;P|Q \;|\; @{x}}
  \and
  \inferrule* [lab=name] {} {{x} \bc \quotep{P}}
\end{mathpar} 

Note that $\vec{x}$ (resp. $\vec{P}$) denotes a vector of names
(resp. processes) of length $|\vec{x}|$ (resp. $|\vec{P}|$). We adopt
the following useful abbreviations.

\begin{mathpar}
   x?(\vec{y}).P := x.(\vec{y})P \and  x\clift{\vec{P}} := x.\clift{\vec{P}}
   \and x!(y) := \lift{x}{\dropn{y}}
   \and \Pi_{i=0}^{n-1}P_i := P_0 | \ldots | P_{n-1}
\end{mathpar}

\subsubsection{Structural congruence}

\paragraph{Free and bound names and alpha-equivalence.} At the
core of structural equivalence is alpha-equivalence which identifies
process that are the same up to a change of variable. Formally, we
recognize the distinction between free and bound names. The free names
of a process, $\freenames{P}$, may be calculated recursively as
follows:

\begin{mathpar}
\freenames{\pzero} := \emptyset
  \and \\
  \freenames{x?(y).P} := \{ x \} \cup (\freenames{P} \setminus \{ y \})
  \and 
  \freenames{x!\langle P \rangle} := \{ x \} \cup \{ P \} 
  \and \\
  \freenames{P|Q} := \freenames{P} \cup \freenames{Q}
  \and \\
  \freenames{@{x}} := \{ x \}
\end{mathpar}

$\pi$
$\quotep{\pi}$

$\freenames{-} : \pi \to \mathcal{P}(\quotep{\pi})$

\begin{eqnarray*}
  \freenames{\pzero} & := & \emptyset \\
  \freenames{x?(y).P} & := & \{ x \} \cup (\freenames{P} \setminus \{ y \}) \\
  \freenames{x!\langle P \rangle} & := & \{ x \} \cup \{ P \} \\
  \freenames{P|Q} & := & \freenames{P} \cup \freenames{Q} \\
  \freenames{\dropn{x}} & := & \{ x \}
\end{eqnarray*}

The bound names of a process, $\boundnames{P}$, are those names occurring in $P$
that are not free. For example, in $x?(y).0$, the name $x$ is free, while $y$ is bound.

\begin{mathpar}
  \inferrule* [lab=monoidal-laws] {} { P|Q \equiv Q|P \and P|0 \equiv P \and P|(Q|R) \equiv (P|Q)|R }
\end{mathpar}

\begin{mathpar}
  \inferrule* [lab=alpha-equivalence] {} { (x)P \equiv (y)P\{y/x\} \and y \not\in \freenames{P} }
\end{mathpar}

\begin{definition}
Then two processes, $P,Q$, are alpha-equivalent if $P = Q\{\vec{y}/\vec{x}\}$ for
some $\vec{x} \in \boundnames{Q},\vec{y} \in \boundnames{P}$, where $Q\{\vec{y}/\vec{x}\}$
denotes the capture-avoiding substitution of $\vec{y}$ for $\vec{x}$ in $Q$.
\end{definition}

\begin{definition}
  The {\em structural congruence} \cite{SangiorgiWalker} , $\equiv$,
  between processes is the least congruence containing
  alpha-equivalence, satisfying the abelian monoid laws
  (associativity, commutativity and $\pzero$ as identity) for parallel
  composition $|$ and for summation $+$.
\end{definition}

\subsection{Name equivalence}

We take name equivalence, written $\nameeq$, to be the smallest
equivalence relation generated by the following rules.

\begin{mathpar}
\inferrule*[lab=Quote-drop]
{ }
{ \quotep{@{x}} \nameeq x }

\inferrule*[lab=Struct-equiv]
{ P \scong Q }
{ \quotep{P} \nameeq \quotep{Q} }
\end{mathpar}

The astute reader will have noticed that the mutual recursion of names
and processes imposes a mutual recursion on alpha-equivalence and
structural equivalence via name-equivalence. Fortunately, all of this
works out pleasantly and we may calculate in the natural way, free of
concern. The reader interested in the details is referred to the
appendix \ref{appendix:rho_details}.

\subsection{Substitution}

We use $\Proc$ for the set of processes, $\QProc$ for the set of
names, and $\id{\{}\vec{y} / \vec{x} \id{\}}$ to denote partial maps,
$s : \QProc \rightarrow \QProc$. A map, $s$ lifts, uniquely, to a map
on process terms, $\widehat{s} : \Proc \rightarrow \Proc$ by the
following equations.

\begin{mathpar}
  (0) \psubstp{Q}{P} := 0 \\
  (R \juxtap S) \psubstp{Q}{P}
  :=    
  (R)\psubstp{Q}{P} \juxtap (S) \psubstp{Q}{P} \\
  (x?(y).R) \psubstp{Q}{P}    
  :=    
  (x)\substp{Q}{P} (z)\concat( (R \psubstn{z}{y}) \psubstp{Q}{P} ) \\
  (\lift{x}{R}) \psubstp{Q}{P}  
  :=
  \lift{(x)\substp{Q}{P}}{ R \psubstp{Q}{P} } \\
%   (\dropn{x})  \psubstp{Q}{P}       
%   := 
%   \left\{ 
%     \begin{array}{ccc} 
%       \dropn{\quotep{Q}} & & x \nameeq \quotep{P} \\
%       \dropn{x} & & otherwise \\
%     \end{array}
%   \right. 
  (\dropn{x})  \psubstp{Q}{P}       
  := 
  \left\{ 
    \begin{array}{ccc} 
      Q & & x \nameeq \quotep{P} \\
      \dropn{x} & & otherwise \\
    \end{array}
  \right.
\end{mathpar}
 

where

\begin{eqnarray}
  (x)\id{\{} \lpquote Q \rpquote / \lpquote P \rpquote \id{\}}            = 
  \left\{ 
    \begin{array}{ccc}
      \lpquote Q \rpquote & & x \nameeq \lpquote P \rpquote \\
      x & & otherwise \\
    \end{array}
  \right. \nonumber
\end{eqnarray}

and $z$ is chosen distinct from $\quotep{P}$, $\quotep{Q}$, the free
names in $Q$, and all the names in $R$. Our $\alpha$-equivalence will
be built in the standard way from this substitution.

\begin{remark}\label{rem:no_self_referential_names}
  One consequence of these definitions is that $\forall P. \quotep{P}
  \not\in \freenames{P}$.
\end{remark}

\subsection{ Dynamic quote: an example }

Anticipating something of what's to come, consider applying the
substitution, $\widehat{\id{\{}u / z \id{\}}}$, to the following pair
of processes, $\lift{w}{y!(z)}$ and $w[ \lpquote y!(z) \rpquote ]$.

\begin{eqnarray}
	\lift{w}{y!(z)}\widehat{\id{\{}u / z \id{\}}}
		& = &
		\lift{w}{y!(u)} \nonumber\\
	w[ \lpquote y!(z) \rpquote ] \widehat{ \id{\{}u / z \id{\}} }
		& = &
		w[ \lpquote y!(z) \rpquote ] \nonumber
\end{eqnarray}

Because the body of the process between quotes is impervious to
substitution, we get radically different answers. In fact, by
examining the first process in an input context,
e.g. $x?(z).\lift{w}{y!(z)}$, we see that the process under the lift
operator may be shaped by prefixed inputs binding a name inside it. In
this sense, the lift operator will be seen as a way to dynamically
construct processes before reifying them as names.

Finally equipped with these standard features we can present the
dynamics of the calculus.

\subsubsection{Operational semantics} 

Finally, we introduce the computational dynamics. What marks these
algebras as distinct from other more traditionally studied algebraic
structures, e.g. vector spaces or polynomial rings, is the manner in
which dynamics is captured. In traditional structures, dynamics is typically
expressed through morphisms between such structures, as in linear maps
between vector spaces or morphisms between rings. In algebras
associated with the semantics of computation, the dynamics is
expressed as part of the algebraic structure itself, through a
reduction reduction relation typically denoted by $\red$. Below, we
give a recursive presentation of this relation for the calculus used
in the encoding.

$\red \subseteq \pi \times \pi$
$\red : \pi \to \mathcal{P}(\pi)$

\begin{mathpar}
  \inferrule* [lab=Comm] { \textsf{match}( x_{src}, x_{trgt} ) } { x_{trgt}?(y)P \; | \; x_{src}!\langle {Q} \rangle \red P\{\quotep{Q}/y}\} }
  \and \\
  \inferrule* [lab=Par] {{P} \red {P}'} {{{P} | {Q}} \red {{P}' | {Q}}}
  \and
  \inferrule* [lab=Equiv]{{{P} \scong {P}'} \andalso {{P}' \red {Q}'} \andalso {{Q}' \scong {Q}}}{{P} \red {Q}}
\end{mathpar}

\begin{eqnarray*}
  match_{\equiv} (\quotep{P},\quotep{Q}) & := & P \equiv Q \\
  match_{\dagger}(\quotep{P},\quotep{Q}) & := & \forall R. P|Q \red^{*} R => R \red^{*} 0 \\
  match_{K}(\quotep{P},\quotep{Q}) & := & K \mbox{ for some context } K
\end{eqnarray*}

$u?(x)P | u!\langle Q \rangle \red P\{\quotep{Q}/x\}$

%We write $\wred$ for $\red^*$, and $P\red$ if $\exists Q $ such that $ P \red Q$.
We write $P\red$ if $\exists Q $ such that $ P \red Q$ and $P\not\red$, otherwise.

\section{Replication}

As mentioned before, it is known that replication (and hence
recursion) can be implemented in a higher-order process algebra
\cite{SangiorgiWalker}. As our first example of calculation with the
machinery thus far presented we give the construction explicitly in
the {\rhoc}.

\begin{eqnarray}
	D_{x} & := & \prefix{x}{y}{(\binpar{\outputp{x}{y}}{@{y}})} \nonumber\\
	\bangp_{x}{P} & := & \binpar{{x}!\langle{\binpar{D_{x}}{P}}\rangle}{D_{x}} \nonumber
\end{eqnarray}

\begin{eqnarray}
	\bangp_{x}{P} & & \nonumber\\
	=
	& {x}!\langle{(\prefix{x}{y}{(\outputp{x}{y} | @{y})) | P}}\rangle 
	      | \prefix{x}{y}{(\outputp{x}{y} | @{y})} & \nonumber\\
	\red
	& (\outputp{x}{y} | @{y})\substn{\quotep{(\prefix{x}{y}{(@{y} | \outputp{x}{y})) | P}}}{y} & \nonumber\\
	=
	& \outputp{x}{\quotep{(\prefix{x}{y}{(\outputp{x}{y} | @{y})) | P}}}
	  | {(\prefix{x}{y}{(\outputp{x}{y} | @{y})) | P}} & \nonumber\\
	\red
	& \ldots & \nonumber\\
	\red^*
	& P | P | \ldots & \nonumber
\end{eqnarray}

Of course, this encoding, as an implementation, runs away, unfolding
$\bangp{P}$ eagerly. A lazier and more implementable replication
operator, restricted to input-guarded processes, may be obtained as follows.

\begin{eqnarray}
\bangp{\prefix{u}{v}{P}} 
	:= 
	\binpar{\lift{x}{\prefix{u}{v}{(\binpar{D(x)}{P})}}}{D(x)} \nonumber
\end{eqnarray}

\begin{remark}
  Note that the lazier definition still does not deal with summation
  or mixed summation (i.e. sums over input and output). The reader is
  invited to construct definitions of replication that deal with these
  features. 

  Further, the definitions are parameterized in a name, $x$. Can you,
  gentle reader, make a definition that eliminates this parameter and
  guarantees no accidental interaction between the replication
  machinery and the process being replicated -- i.e. no accidental
  sharing of names used by the process to get its work done and the
  name(s) used by the replication to effect copying. This latter
  revision of the definition of replication is crucial to obtaining
  the expected identity $!!P \sim !P$.
\end{remark}

\begin{remark}\label{rem:paradoxical_combinator}
  The reader familiar with the lambda calculus will have noticed the
  similarity between $D$ and the paradoxical combinator.

  [Ed. note: the existence of this seems to suggest we have to be more
  restrictive on the set of processes and names we admit if we are to
  support no-cloning.]
\end{remark}

\subsubsection{Bisimulation}

The computational dynamics gives rise to another kind of equivalence,
the equivalence of computational behavior. As previously mentioned
this is typically captured \emph{via} some form of bisimulation.

% The notion we use in this paper is weak barbed bisimulation
% \cite{milner91polyadicpi}.

The notion we use in this paper is derived from weak barbed
bisimulation \cite{milner91polyadicpi}. 

\begin{definition}
An \emph{observation relation}, $\downarrow_{\mathcal N}$, over a set
of names, $\mathcal N$, is the smallest relation satisfying the rules
below.

\infrule[Out-barb]{y \in {\mathcal N}, \; x \nameeq y}
		  {\outputp{x}{v} \downarrow_{\mathcal N} x}
\infrule[Par-barb]{\mbox{$P\downarrow_{\mathcal N} x$ or $Q\downarrow_{\mathcal N} x$}}
		  {\binpar{P}{Q} \downarrow_{\mathcal N} x}

We write $P \Downarrow_{\mathcal N} x$ if there is $Q$ such that 
$P \wred Q$ and $Q \downarrow_{\mathcal N} x$.
\end{definition}

\begin{definition}
%\label{def.bbisim}
An  ${\mathcal N}$-\emph{barbed bisimulation} over a set of names, ${\mathcal N}$, is a symmetric binary relation 
${\mathcal S}_{\mathcal N}$ between agents such that $P\rel{S}_{\mathcal N}Q$ implies:
\begin{enumerate}
\item If $P \red P'$ then $Q \wred Q'$ and $P'\rel{S}_{\mathcal N} Q'$.
\item If $P\downarrow_{\mathcal N} x$, then $Q\Downarrow_{\mathcal N} x$.
\end{enumerate}
$P$ is ${\mathcal N}$-barbed bisimilar to $Q$, written
$P \wbbisim_{\mathcal N} Q$, if $P \rel{S}_{\mathcal N} Q$ for some ${\mathcal N}$-barbed bisimulation ${\mathcal S}_{\mathcal N}$.
\end{definition}

$\mathcal{R} \subseteq \pi \times \pi$

$P \mathcal{R} Q => \forall P'. P \red P' \Rightarrow \exists Q'. Q \red Q', P' \mathcal{R} Q'$

$P \vdash x \Rightarrow Q \vdash x$

\begin{mathpar}
  \inferrule*[lab=Out-barb]{x \nameeq y}{{y}!\langle{Q}\rangle \vdash x}
  \and
  \inferrule*[lab=Par-barb]{\mbox{$P\vdash x$ or $Q\vdash x$}}{\binpar{P}{Q} \vdash x}
\end{mathpar}

\subsubsection{Contexts}

One of the principle advantages of computational calculi like the
$\pi$-calculus is a well-defined notion of context,
contextual-equivalence and a correlation between
contextual-equivalence and notions of bisimulation. The notion of
context allows the decomposition of a process into (sub-)process and
its syntactic environment, its context. Thus, a context may be
thought of as a process with a ``hole'' (written $\Box$) in it. The
application of a context $M$ to a process $P$, written $M[P]$, is
tantamount to filling the hole in $M$ with $P$. In this paper we do
not need the full weight of this theory, but do make use of the notion
of context in the proof the main theorem. 

\begin{mathpar}
  \inferrule* [lab=summation] {} {{M_{M},M_{N}} \bc \Box \;|\; x.M_{A} \;|\; M_{M}+M_{N}}
  \and
  \inferrule* [lab=agent] {} {{M_{A}} \bc (\vec{x})M_{P} \;| \; \clift{P_0,\ldots,M_{P},\ldots,P_N}}
  \and \\
  \inferrule* [lab=process] {} {{M_{P}} \bc M_{N} \;| \;P|M_{P} }
\end{mathpar} 

\begin{mathpar}
  \inferrule* [lab=sychronization] {} {M_{N} \bc \Box \;|\; x?M_{F} \;|\; x!M_{C}}
  \and
  \inferrule* [lab=abstraction] {} {{M_{F}} \bc (x)M_{P} }
  \and
  \inferrule* [lab=concretion] {} {{M_{C}} \bc \langle M_{P} \rangle }
  \and \\
  \inferrule* [lab=process] {} {{M_{P}} \bc M_{N} \;| \;P|M_{P} }
\end{mathpar}

\begin{definition}[contextual application] Given a context $M$, and
  process $P$, we define the \emph{contextual application}, $M[P] :=
  M\{P/\Box\}$. That is, the contextual application of M to P is the
  substitution of $P$ for $\Box$ in $M$.
\end{definition}

$\meaningof{-} : L \to \mathcal{P}(\pi)$

\begin{mathpar}
  \inferrule* [lab=collection] {} {\meaningof{true} = \pi, \and \meaningof{~E} = \pi \setminus \meaningof{E}, \and \meaningof{E_{1} \& E_{2}} = \meaningof{E_{1}} \cap \meaningof{E_{2}}}
\end{mathpar}

\begin{mathpar}
  \inferrule* [lab=structure] {} {\meaningof{0} = \{ P \in \pi | P \equiv 0 \}, \and \\ \meaningof{E_1 | E_2} = \{ P \in \pi | P \equiv P_{1} | P_{2}, P_{1} \in \meaningof{E_{1}}, P_{2} \in \meaningof{E_2}\} }
\end{mathpar}

\begin{mathpar}
 \inferrule* [lab=behavior] {} {\meaningof{\langle a?b \rangle E} = \{ P \in \pi | P \equiv Q | u?(y)P', \\ \and \\\\ \and \\ \;\;\; u \in \meaningof{a}, \forall z.P'\{z/y\} \in \meaningof{E\{z/b\}}\}, \and \\ \meaningof{a!E} = \{ P \in \pi | P \equiv Q | x!\langle P' \rangle, x \in \meaningof{a} P' \in \meaningof{E}\} }
\end{mathpar}

\begin{mathpar}
 \inferrule* [lab=nominal] {} {\meaningof{\quotep{E}} = \{ \quotep{P} \in \quotep{\pi} | P \in \meaningof{E} \}, \and \meaningof{\quotep{P}} = \{ \quotep{Q} \in \quotep{\pi} | P \equiv Q \} \and \\ \meaningof{@\quotep{E}} = \{ P \in \pi | P \equiv @x, x \in \meaningof{E} \}}
\end{mathpar}

\begin{eqnarray*}
  \\
  \meaningof{-} : TS \to ST
\end{eqnarray*}

\begin{eqnarray*}
  \\
  L : TS \to ST
\end{eqnarray*}

\begin{eqnarray*}
  \\
  P \models E \iff P \in \meaningof{E}
\end{eqnarray*}

\begin{eqnarray*}
  P \approx_{L} Q \iff \forall E \in L. P \models E \iff Q \models E
\end{eqnarray*}

\begin{eqnarray*}
  P \approx_{K} Q
\end{eqnarray*}

\begin{eqnarray*}
  P \approx Q
\end{eqnarray*}

$\approx_{K} = \approx = \approx_{L}$

\subsubsection{Contextual duality}

Note that contexts extend the quotation operation to a family of
operations from processes to names. Given a context, $M$, we can
define a \emph{nominal context}, $\quotep{M}$ by $\quotep{M}[P] :=
\quotep{M[P]}$. To foreshadow what is to come we observe that these
operations enjoy a duality with processes very much like the duality
between vectors and maps from vectors to scalars.

Further, because the calculus is essentially higher-order, we have a
correspondence between contexts and processes. More specifically,
given a name $x$ and a context $M$ we can construct $M^{*}_{x}$ such
that 

\begin{mathpar}
  M^{*}_{x} | \lift{x}{P} \red M[P]
\end{mathpar}

namely,

\begin{mathpar}
  M^{*}_{x} := x?(u).M[\dropn{u}]
\end{mathpar}

The dependence of $M^{*}_{x}$ on a name makes it an abstraction, 

\begin{mathpar}
  M^{*} := (x)x?(u).M[\dropn{u}]
\end{mathpar}

\subsection{Additional notation}

It will sometimes be convenient to denote the process a name
quotes. We already have the notation $x = \quotep{P}$, but it will be
convenient to introduce an alternate notation, $\procn{x}$, when we
want to emphasize the connection to the use of the name. Note that, by
virtue of name equivalence, $\quotep{\procn{x}} \nameeq x$; so, the
notation is consistent with previous definitions.

Further, because names have structure it is possible to effect
substitutions on the basis of that structure. This means we need to
upgrade our notation for substitutions, which we accomplish by
adapting comprehension notation. Thus,

\begin{mathpar}
  P\{ y / x : x \in S \}
\end{mathpar}

is interpreted to mean the process derived from P by replacing (in a
capture-avoiding manner) each occurrence of $x$ in $S$ by $y$. For example,

\begin{mathpar}
  P\{ \quotep{\procn{x}|\procn{x}} / x : x \in \freenames{P} \}
\end{mathpar}

will replace each (occurrence) of a free name $x$ in $P$ by
$\quotep{\procn{x}|\procn{x}}$.

Also, we will avail ourselves of the notation $x^{L}$ and $x^{R}$ to
denote injections of a name into disjoint copies of the name
space. There are numerous ways to accomplish this. One example can be
found in \cite{MeredithR05}. This notation overloads to vectors of
names: $\vec{x}^{\pi} := (x_{i}^{\pi} \; : \; 0 \leq i < |\vec{x}| )$ where $\pi \in \{L,R\}$.

We also use $P^{\Box} := P|\Box$.

In \cite{MeredithR05} an interpretation of the new operator is
given. It turns out that there are several possible interpretations
all enjoying the requisite algebraic properties of the operator (see
\cite{milner91polyadicpi}). We will therefore make liberal use of
$(\nu\; \vec{x})P$.

% subsection the_syntax_and_semantics_of_the_notation_system (end)   

\input{qm2pi.qmops} 

\input{qm2pi.sterngerlach} 

\input{qm2pi.metric} 

% section concurrent_process_calculi (end)

%\input{qm2pi.proofsketch}

% section proof sketch (end)

%\input{qm2pi.slviaknots} 

% section spatial logic via knots (end)

\input{qm2pi.conclusion}

% section conclusion (end)

%\input{qm2pi.dtcodes} 

% section wiring algorithm (end)

\input{qm2pi.ack} 

% section acknowledgments (end)

\newpage


\bibliographystyle{plain}   
\bibliography{../../biblios/main.bib}

\input{qm2pi.rhodetails}

\end{document}



\end{document}



% section front matter (end)

\section{Introduction}\label{sec:introduction} % (fold)
In this draft of the material i am going to have to dispense with the
usual writing conventions adopted in papers on these topics. i'm going
to have adopt whatever tone i need at the time i'm writing up the
calculations. Sometimes this may be very conversational; others it may
be the barest mathematical grunts; others still it may be that i have
lifted text from one of my other papers because the exposition of some
point was better said there. i hope that my readers are not unduly put
out by this decision. i'm not doing this to flout convention or be
rebellious. i find these calculations very technically challenging. To
keep everything going technically, something has to give; i have to
let go of some cognitive burden. So, the academic writing style --
with all of its trade-offs in terms of facilitating technical
communication -- is what i'm letting go of. Perhaps subsequent drafts
can be tightened and polished, but for now, i'm going to speak as if
we were sitting together in a coffee shop with a laptop, wifi and a
pad of paper and a pencil.

So, here's what i have to say. We -- you and i, comfortably ensconced
in our coffee shop and well-equipped with our tools -- can realize and
carry out the calculations of quantum mechanics over a very different
formal theory of dynamics, a formal theory of dynamics that
corresponds to a theory of concurrent computation with
\emph{reflection}. It has the advantage that the underlying theory is
already `quantized', but supports analogues all of the continuuous
operations. Strikingly, this underlying theory has recently been
connected with a notion of metric that we can show, by calculating
together, coincides with the metric induced by the inner product.

There are a lot of reasons why you might be interested in seeing
calculations of this form. Here's why i'm interested. For the past
several centuries there has been no competitor to the ``Newtonian''
account of dynamics. As a result the predominant share of accounts of
dynamical systems and situations have had to be formulated in terms of
the Newtonian machinery. i view this as an intellectually dangerous
position to occupy. Everything, despite it's intrinsic shape, turns
into a nail to be hit with this hammer. Recently, however, the theory
of computation has matured to the point where we have candidates for
theories of dynamics that offer very different perspective on
reasoning about dynamical systems and situations. Testing these
candidates against very successful accounts of dynamical situations,
like quantum mechanics, is going to give us some sense of how mature
they are and some measure of the quality of these accounts of
dynamics.

\subsection{Summary of contributions and outline of paper}

So, we're going to develop an interpretation of the operations of
quantum mechanics normally interpreted by Hilbert spaces and
operators. We're going to do this over a theory of computation. Note
that this is very different than the usual quantum computation program
which develops notions of computation over quantum mechanics. Rather,
we are developing a story that aligns with Wheeler's slogan: It from
Bit. To do this we will first provide an account of the theory of
computation at play here. Then we will dive into a calculation-driven
interpretation of the operations of quantum mechanics.

The reason we take this approach is that -- until very recently --
there hasn't been an axiomatic account of quantum mechanics. As a
result there has been no sharp delineation of the mathematical theory
supporting interpretation of the physical theory and the physical
theory, itself. So, ambient features of the maths are free to be
exploited (or supressed) without a real accounting of their physical
relevance. There is no sharp statement ``here's the physical theory''
qua \emph{theory} and ``here's the mathematical interpretation''
enabling a judgment of how faithful the interpretation is -- apart
from experimental observation. When there is an axiomatic account we
can judge how well a given mathematical formalism supports an
interpretation of the axioms, independent of
experimentation. Likewise, we can judge how well we have captured our
physical evidence and experience with our axiomatics, independent of
any specific mathematical implementation, with accidental detail that
may or may not have physical significance. 

In lieu of a fully fleshed out and vetted axiomatic account of quantum
mechanics, interpreting the operational notions in service of modeling
physical systems will have to suffice. In other words, we are not in
the business of providing a model of Hilbert spaces and operators. We
are in the business of providing a model of quantum mechanics because
we are motivated by testing our notions of dynamics against physical
theory; and, the predictive calculations of the physical theory must
serve as the best formulation -- shy of a fully fleshed out axiomatic
account -- of the physical theory itself (as they have for scientific
theories since time immemorial). Put another way, despite a
whole-hearted commitment to an It-from-Bit ontology, we are firmly
aligned with the shut-up-and-calculate camp as the best way to obtain
results either from the physical perspective or as a quality assurance
measure of our fledgling theory of dynamics.

In detail, we present a reflective process calculus. Then we develop
intuitive correspondences between the notions available in this
calculus and the usual physical notions supporting quantum mechanical
calculations. Thus, 

\begin{table}[htp]
  \center{
    \fbox{
      \begin{tabular}{c|c}
        quantum mechanics & process calculus \\
        \hline
        scalar & name \\
        state vector & process \\
        dual & contextual duals \\
        matrix & formal sums of process-context-dual pairs \\
        orthogonality & process annihilation \\
        inner product & execution-formula + quoting
      \end{tabular}
    }
  }
  \caption{QM - process calculi correspondences}
\end{table}

Then we tighten up these intuitions to operational definitions. We
employ the Dirac notation as the best proxy we can find for an
abstract syntax of the quantum mechanical notions. The definitions we
develop put us in contact with equational constraints coming from the
theory that we demonstrate the definitions and calculations satisfy.

This puts us in a position to shut up and calculate for the
Stern-Gerlach experimental set up, showing how these predictive
calculations become calculations on processes in our theory of a
reflective process calculus.

Penultimately, we demonstrate that the notion of metric coming from
the inner product coincides with the notion of metric available from
the theory of bisimulation. This demonstration gives us the right to
think of space as arising from behavior. Finally, we consider where we
might go from the new vantage point we have obtained.

% section introduction (end) 
 
% section introduction (end)

% \documentclass[12pt]{llncs}
%\documentclass{jktr}

\usepackage[pdftex]{hyperref}                   
\usepackage {listings}
\usepackage {mathpartir}
\usepackage{bcprules}
%\usepackage{listings}
                       
\usepackage{graphicx} 
%\usepackage[margins=2.5cm,nohead,nofoot]{geometry}
%\usepackage{geometry}
\usepackage{amsfonts}
\usepackage{amstext}
\usepackage{latexsym}
\usepackage{amssymb}
\usepackage{color}


%\include{myPreamble}
\documentclass[12pt]{llncs}
%\documentclass{jktr}

\usepackage[pdftex]{hyperref}                   
\usepackage {listings}
\usepackage {mathpartir}
\usepackage{bcprules}
%\usepackage{listings}
                       
\usepackage{graphicx} 
%\usepackage[margins=2.5cm,nohead,nofoot]{geometry}
%\usepackage{geometry}
\usepackage{amsfonts}
\usepackage{amstext}
\usepackage{latexsym}
\usepackage{amssymb}
\usepackage{color}


%\include{myPreamble}
\include{qm2pi.local} 

%\ifpdf
%\usepackage[pdftex]{graphicx}
%\else
%\usepackage{graphicx}
%\fi

 % \ifpdf
%  \usepackage{pdfsync}
%  \if


%\title{Brief Article}
%\author{David F. Snyder}
%\author{L.G. Meredith}

%\address{Dept. of Math., Texas State University--San Marcos, San Marcos, TX 78666}
       
\pagestyle{empty}


\begin{document}

\lstset{language=[Objective]Caml,frame=shadowbox}

\input{qm2pi.front}

% section front matter (end)

\input{qm2pi.intro} 
 
% section introduction (end)

% \input{qm2pi.knotations} 

% section notation (end)

\input{qm2pi.process.calculi} 

% section concurrent_process_calculi_and_spatial_logics_ (end)
    
%\input{qm2pi.knots2pi} 

%\input{qm2pi.trefoil} 

%\input{qm2pi.mainthm} 

% subsection basic_interpretation (end)

%\input{qm2pi.rho.presentation} 
\subsection{The syntax and semantics of the notation system}\label{sub:the_syntax_and_semantics_of_the_notation_system} % (fold)

We now summarize a technical presentation of the calculus that
embodies our theory of dynamics. The typical presentation of such a
calculus follows the style of giving generators and relations on
them. The grammar, below, describing term constructors, freely
generates the set of processes, $\Proc$. This set is then quotiented
by a relation known as structural congruence and it is over this set
that the notion of dynamics is expressed. This presentation is
essentially that of \cite{MeredithR05} with the addition of
polyadicity and summation. For readability we have relegated some of
the technical subtleties to an appendix.

\subsubsection{Process grammar}\label{subsub:process_grammar}

\begin{mathpar}
  \inferrule* [lab=synchronization] {} {{M} \bc \pzero \;|\; x?F \;|\; x!C }
  \and
  \inferrule* [lab=abstraction] {} {{F} \bc (x)P}
  \and
  \inferrule* [lab=concretion] {} {{C} \bc \langle Q \rangle}
  \and
  \inferrule* [lab=process] {} {{P,Q} \bc M \;| \;P|Q \;|\; @{x}}
  \and
  \inferrule* [lab=name] {} {{x} \bc \quotep{P}}
\end{mathpar} 

Note that $\vec{x}$ (resp. $\vec{P}$) denotes a vector of names
(resp. processes) of length $|\vec{x}|$ (resp. $|\vec{P}|$). We adopt
the following useful abbreviations.

\begin{mathpar}
   x?(\vec{y}).P := x.(\vec{y})P \and  x\clift{\vec{P}} := x.\clift{\vec{P}}
   \and x!(y) := \lift{x}{\dropn{y}}
   \and \Pi_{i=0}^{n-1}P_i := P_0 | \ldots | P_{n-1}
\end{mathpar}

\subsubsection{Structural congruence}

\paragraph{Free and bound names and alpha-equivalence.} At the
core of structural equivalence is alpha-equivalence which identifies
process that are the same up to a change of variable. Formally, we
recognize the distinction between free and bound names. The free names
of a process, $\freenames{P}$, may be calculated recursively as
follows:

\begin{mathpar}
\freenames{\pzero} := \emptyset
  \and \\
  \freenames{x?(y).P} := \{ x \} \cup (\freenames{P} \setminus \{ y \})
  \and 
  \freenames{x!\langle P \rangle} := \{ x \} \cup \{ P \} 
  \and \\
  \freenames{P|Q} := \freenames{P} \cup \freenames{Q}
  \and \\
  \freenames{@{x}} := \{ x \}
\end{mathpar}

$\pi$
$\quotep{\pi}$

$\freenames{-} : \pi \to \mathcal{P}(\quotep{\pi})$

\begin{eqnarray*}
  \freenames{\pzero} & := & \emptyset \\
  \freenames{x?(y).P} & := & \{ x \} \cup (\freenames{P} \setminus \{ y \}) \\
  \freenames{x!\langle P \rangle} & := & \{ x \} \cup \{ P \} \\
  \freenames{P|Q} & := & \freenames{P} \cup \freenames{Q} \\
  \freenames{\dropn{x}} & := & \{ x \}
\end{eqnarray*}

The bound names of a process, $\boundnames{P}$, are those names occurring in $P$
that are not free. For example, in $x?(y).0$, the name $x$ is free, while $y$ is bound.

\begin{mathpar}
  \inferrule* [lab=monoidal-laws] {} { P|Q \equiv Q|P \and P|0 \equiv P \and P|(Q|R) \equiv (P|Q)|R }
\end{mathpar}

\begin{mathpar}
  \inferrule* [lab=alpha-equivalence] {} { (x)P \equiv (y)P\{y/x\} \and y \not\in \freenames{P} }
\end{mathpar}

\begin{definition}
Then two processes, $P,Q$, are alpha-equivalent if $P = Q\{\vec{y}/\vec{x}\}$ for
some $\vec{x} \in \boundnames{Q},\vec{y} \in \boundnames{P}$, where $Q\{\vec{y}/\vec{x}\}$
denotes the capture-avoiding substitution of $\vec{y}$ for $\vec{x}$ in $Q$.
\end{definition}

\begin{definition}
  The {\em structural congruence} \cite{SangiorgiWalker} , $\equiv$,
  between processes is the least congruence containing
  alpha-equivalence, satisfying the abelian monoid laws
  (associativity, commutativity and $\pzero$ as identity) for parallel
  composition $|$ and for summation $+$.
\end{definition}

\subsection{Name equivalence}

We take name equivalence, written $\nameeq$, to be the smallest
equivalence relation generated by the following rules.

\begin{mathpar}
\inferrule*[lab=Quote-drop]
{ }
{ \quotep{@{x}} \nameeq x }

\inferrule*[lab=Struct-equiv]
{ P \scong Q }
{ \quotep{P} \nameeq \quotep{Q} }
\end{mathpar}

The astute reader will have noticed that the mutual recursion of names
and processes imposes a mutual recursion on alpha-equivalence and
structural equivalence via name-equivalence. Fortunately, all of this
works out pleasantly and we may calculate in the natural way, free of
concern. The reader interested in the details is referred to the
appendix \ref{appendix:rho_details}.

\subsection{Substitution}

We use $\Proc$ for the set of processes, $\QProc$ for the set of
names, and $\id{\{}\vec{y} / \vec{x} \id{\}}$ to denote partial maps,
$s : \QProc \rightarrow \QProc$. A map, $s$ lifts, uniquely, to a map
on process terms, $\widehat{s} : \Proc \rightarrow \Proc$ by the
following equations.

\begin{mathpar}
  (0) \psubstp{Q}{P} := 0 \\
  (R \juxtap S) \psubstp{Q}{P}
  :=    
  (R)\psubstp{Q}{P} \juxtap (S) \psubstp{Q}{P} \\
  (x?(y).R) \psubstp{Q}{P}    
  :=    
  (x)\substp{Q}{P} (z)\concat( (R \psubstn{z}{y}) \psubstp{Q}{P} ) \\
  (\lift{x}{R}) \psubstp{Q}{P}  
  :=
  \lift{(x)\substp{Q}{P}}{ R \psubstp{Q}{P} } \\
%   (\dropn{x})  \psubstp{Q}{P}       
%   := 
%   \left\{ 
%     \begin{array}{ccc} 
%       \dropn{\quotep{Q}} & & x \nameeq \quotep{P} \\
%       \dropn{x} & & otherwise \\
%     \end{array}
%   \right. 
  (\dropn{x})  \psubstp{Q}{P}       
  := 
  \left\{ 
    \begin{array}{ccc} 
      Q & & x \nameeq \quotep{P} \\
      \dropn{x} & & otherwise \\
    \end{array}
  \right.
\end{mathpar}
 

where

\begin{eqnarray}
  (x)\id{\{} \lpquote Q \rpquote / \lpquote P \rpquote \id{\}}            = 
  \left\{ 
    \begin{array}{ccc}
      \lpquote Q \rpquote & & x \nameeq \lpquote P \rpquote \\
      x & & otherwise \\
    \end{array}
  \right. \nonumber
\end{eqnarray}

and $z$ is chosen distinct from $\quotep{P}$, $\quotep{Q}$, the free
names in $Q$, and all the names in $R$. Our $\alpha$-equivalence will
be built in the standard way from this substitution.

\begin{remark}\label{rem:no_self_referential_names}
  One consequence of these definitions is that $\forall P. \quotep{P}
  \not\in \freenames{P}$.
\end{remark}

\subsection{ Dynamic quote: an example }

Anticipating something of what's to come, consider applying the
substitution, $\widehat{\id{\{}u / z \id{\}}}$, to the following pair
of processes, $\lift{w}{y!(z)}$ and $w[ \lpquote y!(z) \rpquote ]$.

\begin{eqnarray}
	\lift{w}{y!(z)}\widehat{\id{\{}u / z \id{\}}}
		& = &
		\lift{w}{y!(u)} \nonumber\\
	w[ \lpquote y!(z) \rpquote ] \widehat{ \id{\{}u / z \id{\}} }
		& = &
		w[ \lpquote y!(z) \rpquote ] \nonumber
\end{eqnarray}

Because the body of the process between quotes is impervious to
substitution, we get radically different answers. In fact, by
examining the first process in an input context,
e.g. $x?(z).\lift{w}{y!(z)}$, we see that the process under the lift
operator may be shaped by prefixed inputs binding a name inside it. In
this sense, the lift operator will be seen as a way to dynamically
construct processes before reifying them as names.

Finally equipped with these standard features we can present the
dynamics of the calculus.

\subsubsection{Operational semantics} 

Finally, we introduce the computational dynamics. What marks these
algebras as distinct from other more traditionally studied algebraic
structures, e.g. vector spaces or polynomial rings, is the manner in
which dynamics is captured. In traditional structures, dynamics is typically
expressed through morphisms between such structures, as in linear maps
between vector spaces or morphisms between rings. In algebras
associated with the semantics of computation, the dynamics is
expressed as part of the algebraic structure itself, through a
reduction reduction relation typically denoted by $\red$. Below, we
give a recursive presentation of this relation for the calculus used
in the encoding.

$\red \subseteq \pi \times \pi$
$\red : \pi \to \mathcal{P}(\pi)$

\begin{mathpar}
  \inferrule* [lab=Comm] { \textsf{match}( x_{src}, x_{trgt} ) } { x_{trgt}?(y)P \; | \; x_{src}!\langle {Q} \rangle \red P\{\quotep{Q}/y}\} }
  \and \\
  \inferrule* [lab=Par] {{P} \red {P}'} {{{P} | {Q}} \red {{P}' | {Q}}}
  \and
  \inferrule* [lab=Equiv]{{{P} \scong {P}'} \andalso {{P}' \red {Q}'} \andalso {{Q}' \scong {Q}}}{{P} \red {Q}}
\end{mathpar}

\begin{eqnarray*}
  match_{\equiv} (\quotep{P},\quotep{Q}) & := & P \equiv Q \\
  match_{\dagger}(\quotep{P},\quotep{Q}) & := & \forall R. P|Q \red^{*} R => R \red^{*} 0 \\
  match_{K}(\quotep{P},\quotep{Q}) & := & K \mbox{ for some context } K
\end{eqnarray*}

$u?(x)P | u!\langle Q \rangle \red P\{\quotep{Q}/x\}$

%We write $\wred$ for $\red^*$, and $P\red$ if $\exists Q $ such that $ P \red Q$.
We write $P\red$ if $\exists Q $ such that $ P \red Q$ and $P\not\red$, otherwise.

\section{Replication}

As mentioned before, it is known that replication (and hence
recursion) can be implemented in a higher-order process algebra
\cite{SangiorgiWalker}. As our first example of calculation with the
machinery thus far presented we give the construction explicitly in
the {\rhoc}.

\begin{eqnarray}
	D_{x} & := & \prefix{x}{y}{(\binpar{\outputp{x}{y}}{@{y}})} \nonumber\\
	\bangp_{x}{P} & := & \binpar{{x}!\langle{\binpar{D_{x}}{P}}\rangle}{D_{x}} \nonumber
\end{eqnarray}

\begin{eqnarray}
	\bangp_{x}{P} & & \nonumber\\
	=
	& {x}!\langle{(\prefix{x}{y}{(\outputp{x}{y} | @{y})) | P}}\rangle 
	      | \prefix{x}{y}{(\outputp{x}{y} | @{y})} & \nonumber\\
	\red
	& (\outputp{x}{y} | @{y})\substn{\quotep{(\prefix{x}{y}{(@{y} | \outputp{x}{y})) | P}}}{y} & \nonumber\\
	=
	& \outputp{x}{\quotep{(\prefix{x}{y}{(\outputp{x}{y} | @{y})) | P}}}
	  | {(\prefix{x}{y}{(\outputp{x}{y} | @{y})) | P}} & \nonumber\\
	\red
	& \ldots & \nonumber\\
	\red^*
	& P | P | \ldots & \nonumber
\end{eqnarray}

Of course, this encoding, as an implementation, runs away, unfolding
$\bangp{P}$ eagerly. A lazier and more implementable replication
operator, restricted to input-guarded processes, may be obtained as follows.

\begin{eqnarray}
\bangp{\prefix{u}{v}{P}} 
	:= 
	\binpar{\lift{x}{\prefix{u}{v}{(\binpar{D(x)}{P})}}}{D(x)} \nonumber
\end{eqnarray}

\begin{remark}
  Note that the lazier definition still does not deal with summation
  or mixed summation (i.e. sums over input and output). The reader is
  invited to construct definitions of replication that deal with these
  features. 

  Further, the definitions are parameterized in a name, $x$. Can you,
  gentle reader, make a definition that eliminates this parameter and
  guarantees no accidental interaction between the replication
  machinery and the process being replicated -- i.e. no accidental
  sharing of names used by the process to get its work done and the
  name(s) used by the replication to effect copying. This latter
  revision of the definition of replication is crucial to obtaining
  the expected identity $!!P \sim !P$.
\end{remark}

\begin{remark}\label{rem:paradoxical_combinator}
  The reader familiar with the lambda calculus will have noticed the
  similarity between $D$ and the paradoxical combinator.

  [Ed. note: the existence of this seems to suggest we have to be more
  restrictive on the set of processes and names we admit if we are to
  support no-cloning.]
\end{remark}

\subsubsection{Bisimulation}

The computational dynamics gives rise to another kind of equivalence,
the equivalence of computational behavior. As previously mentioned
this is typically captured \emph{via} some form of bisimulation.

% The notion we use in this paper is weak barbed bisimulation
% \cite{milner91polyadicpi}.

The notion we use in this paper is derived from weak barbed
bisimulation \cite{milner91polyadicpi}. 

\begin{definition}
An \emph{observation relation}, $\downarrow_{\mathcal N}$, over a set
of names, $\mathcal N$, is the smallest relation satisfying the rules
below.

\infrule[Out-barb]{y \in {\mathcal N}, \; x \nameeq y}
		  {\outputp{x}{v} \downarrow_{\mathcal N} x}
\infrule[Par-barb]{\mbox{$P\downarrow_{\mathcal N} x$ or $Q\downarrow_{\mathcal N} x$}}
		  {\binpar{P}{Q} \downarrow_{\mathcal N} x}

We write $P \Downarrow_{\mathcal N} x$ if there is $Q$ such that 
$P \wred Q$ and $Q \downarrow_{\mathcal N} x$.
\end{definition}

\begin{definition}
%\label{def.bbisim}
An  ${\mathcal N}$-\emph{barbed bisimulation} over a set of names, ${\mathcal N}$, is a symmetric binary relation 
${\mathcal S}_{\mathcal N}$ between agents such that $P\rel{S}_{\mathcal N}Q$ implies:
\begin{enumerate}
\item If $P \red P'$ then $Q \wred Q'$ and $P'\rel{S}_{\mathcal N} Q'$.
\item If $P\downarrow_{\mathcal N} x$, then $Q\Downarrow_{\mathcal N} x$.
\end{enumerate}
$P$ is ${\mathcal N}$-barbed bisimilar to $Q$, written
$P \wbbisim_{\mathcal N} Q$, if $P \rel{S}_{\mathcal N} Q$ for some ${\mathcal N}$-barbed bisimulation ${\mathcal S}_{\mathcal N}$.
\end{definition}

$\mathcal{R} \subseteq \pi \times \pi$

$P \mathcal{R} Q => \forall P'. P \red P' \Rightarrow \exists Q'. Q \red Q', P' \mathcal{R} Q'$

$P \vdash x \Rightarrow Q \vdash x$

\begin{mathpar}
  \inferrule*[lab=Out-barb]{x \nameeq y}{{y}!\langle{Q}\rangle \vdash x}
  \and
  \inferrule*[lab=Par-barb]{\mbox{$P\vdash x$ or $Q\vdash x$}}{\binpar{P}{Q} \vdash x}
\end{mathpar}

\subsubsection{Contexts}

One of the principle advantages of computational calculi like the
$\pi$-calculus is a well-defined notion of context,
contextual-equivalence and a correlation between
contextual-equivalence and notions of bisimulation. The notion of
context allows the decomposition of a process into (sub-)process and
its syntactic environment, its context. Thus, a context may be
thought of as a process with a ``hole'' (written $\Box$) in it. The
application of a context $M$ to a process $P$, written $M[P]$, is
tantamount to filling the hole in $M$ with $P$. In this paper we do
not need the full weight of this theory, but do make use of the notion
of context in the proof the main theorem. 

\begin{mathpar}
  \inferrule* [lab=summation] {} {{M_{M},M_{N}} \bc \Box \;|\; x.M_{A} \;|\; M_{M}+M_{N}}
  \and
  \inferrule* [lab=agent] {} {{M_{A}} \bc (\vec{x})M_{P} \;| \; \clift{P_0,\ldots,M_{P},\ldots,P_N}}
  \and \\
  \inferrule* [lab=process] {} {{M_{P}} \bc M_{N} \;| \;P|M_{P} }
\end{mathpar} 

\begin{mathpar}
  \inferrule* [lab=sychronization] {} {M_{N} \bc \Box \;|\; x?M_{F} \;|\; x!M_{C}}
  \and
  \inferrule* [lab=abstraction] {} {{M_{F}} \bc (x)M_{P} }
  \and
  \inferrule* [lab=concretion] {} {{M_{C}} \bc \langle M_{P} \rangle }
  \and \\
  \inferrule* [lab=process] {} {{M_{P}} \bc M_{N} \;| \;P|M_{P} }
\end{mathpar}

\begin{definition}[contextual application] Given a context $M$, and
  process $P$, we define the \emph{contextual application}, $M[P] :=
  M\{P/\Box\}$. That is, the contextual application of M to P is the
  substitution of $P$ for $\Box$ in $M$.
\end{definition}

$\meaningof{-} : L \to \mathcal{P}(\pi)$

\begin{mathpar}
  \inferrule* [lab=collection] {} {\meaningof{true} = \pi, \and \meaningof{~E} = \pi \setminus \meaningof{E}, \and \meaningof{E_{1} \& E_{2}} = \meaningof{E_{1}} \cap \meaningof{E_{2}}}
\end{mathpar}

\begin{mathpar}
  \inferrule* [lab=structure] {} {\meaningof{0} = \{ P \in \pi | P \equiv 0 \}, \and \\ \meaningof{E_1 | E_2} = \{ P \in \pi | P \equiv P_{1} | P_{2}, P_{1} \in \meaningof{E_{1}}, P_{2} \in \meaningof{E_2}\} }
\end{mathpar}

\begin{mathpar}
 \inferrule* [lab=behavior] {} {\meaningof{\langle a?b \rangle E} = \{ P \in \pi | P \equiv Q | u?(y)P', \\ \and \\\\ \and \\ \;\;\; u \in \meaningof{a}, \forall z.P'\{z/y\} \in \meaningof{E\{z/b\}}\}, \and \\ \meaningof{a!E} = \{ P \in \pi | P \equiv Q | x!\langle P' \rangle, x \in \meaningof{a} P' \in \meaningof{E}\} }
\end{mathpar}

\begin{mathpar}
 \inferrule* [lab=nominal] {} {\meaningof{\quotep{E}} = \{ \quotep{P} \in \quotep{\pi} | P \in \meaningof{E} \}, \and \meaningof{\quotep{P}} = \{ \quotep{Q} \in \quotep{\pi} | P \equiv Q \} \and \\ \meaningof{@\quotep{E}} = \{ P \in \pi | P \equiv @x, x \in \meaningof{E} \}}
\end{mathpar}

\begin{eqnarray*}
  \\
  \meaningof{-} : TS \to ST
\end{eqnarray*}

\begin{eqnarray*}
  \\
  L : TS \to ST
\end{eqnarray*}

\begin{eqnarray*}
  \\
  P \models E \iff P \in \meaningof{E}
\end{eqnarray*}

\begin{eqnarray*}
  P \approx_{L} Q \iff \forall E \in L. P \models E \iff Q \models E
\end{eqnarray*}

\begin{eqnarray*}
  P \approx_{K} Q
\end{eqnarray*}

\begin{eqnarray*}
  P \approx Q
\end{eqnarray*}

$\approx_{K} = \approx = \approx_{L}$

\subsubsection{Contextual duality}

Note that contexts extend the quotation operation to a family of
operations from processes to names. Given a context, $M$, we can
define a \emph{nominal context}, $\quotep{M}$ by $\quotep{M}[P] :=
\quotep{M[P]}$. To foreshadow what is to come we observe that these
operations enjoy a duality with processes very much like the duality
between vectors and maps from vectors to scalars.

Further, because the calculus is essentially higher-order, we have a
correspondence between contexts and processes. More specifically,
given a name $x$ and a context $M$ we can construct $M^{*}_{x}$ such
that 

\begin{mathpar}
  M^{*}_{x} | \lift{x}{P} \red M[P]
\end{mathpar}

namely,

\begin{mathpar}
  M^{*}_{x} := x?(u).M[\dropn{u}]
\end{mathpar}

The dependence of $M^{*}_{x}$ on a name makes it an abstraction, 

\begin{mathpar}
  M^{*} := (x)x?(u).M[\dropn{u}]
\end{mathpar}

\subsection{Additional notation}

It will sometimes be convenient to denote the process a name
quotes. We already have the notation $x = \quotep{P}$, but it will be
convenient to introduce an alternate notation, $\procn{x}$, when we
want to emphasize the connection to the use of the name. Note that, by
virtue of name equivalence, $\quotep{\procn{x}} \nameeq x$; so, the
notation is consistent with previous definitions.

Further, because names have structure it is possible to effect
substitutions on the basis of that structure. This means we need to
upgrade our notation for substitutions, which we accomplish by
adapting comprehension notation. Thus,

\begin{mathpar}
  P\{ y / x : x \in S \}
\end{mathpar}

is interpreted to mean the process derived from P by replacing (in a
capture-avoiding manner) each occurrence of $x$ in $S$ by $y$. For example,

\begin{mathpar}
  P\{ \quotep{\procn{x}|\procn{x}} / x : x \in \freenames{P} \}
\end{mathpar}

will replace each (occurrence) of a free name $x$ in $P$ by
$\quotep{\procn{x}|\procn{x}}$.

Also, we will avail ourselves of the notation $x^{L}$ and $x^{R}$ to
denote injections of a name into disjoint copies of the name
space. There are numerous ways to accomplish this. One example can be
found in \cite{MeredithR05}. This notation overloads to vectors of
names: $\vec{x}^{\pi} := (x_{i}^{\pi} \; : \; 0 \leq i < |\vec{x}| )$ where $\pi \in \{L,R\}$.

We also use $P^{\Box} := P|\Box$.

In \cite{MeredithR05} an interpretation of the new operator is
given. It turns out that there are several possible interpretations
all enjoying the requisite algebraic properties of the operator (see
\cite{milner91polyadicpi}). We will therefore make liberal use of
$(\nu\; \vec{x})P$.

% subsection the_syntax_and_semantics_of_the_notation_system (end)   

\input{qm2pi.qmops} 

\input{qm2pi.sterngerlach} 

\input{qm2pi.metric} 

% section concurrent_process_calculi (end)

%\input{qm2pi.proofsketch}

% section proof sketch (end)

%\input{qm2pi.slviaknots} 

% section spatial logic via knots (end)

\input{qm2pi.conclusion}

% section conclusion (end)

%\input{qm2pi.dtcodes} 

% section wiring algorithm (end)

\input{qm2pi.ack} 

% section acknowledgments (end)

\newpage


\bibliographystyle{plain}   
\bibliography{../../biblios/main.bib}

\input{qm2pi.rhodetails}

\end{document}

 

%\ifpdf
%\usepackage[pdftex]{graphicx}
%\else
%\usepackage{graphicx}
%\fi

 % \ifpdf
%  \usepackage{pdfsync}
%  \if


%\title{Brief Article}
%\author{David F. Snyder}
%\author{L.G. Meredith}

%\address{Dept. of Math., Texas State University--San Marcos, San Marcos, TX 78666}
       
\pagestyle{empty}


\begin{document}

\lstset{language=[Objective]Caml,frame=shadowbox}

\documentclass[12pt]{llncs}
%\documentclass{jktr}

\usepackage[pdftex]{hyperref}                   
\usepackage {listings}
\usepackage {mathpartir}
\usepackage{bcprules}
%\usepackage{listings}
                       
\usepackage{graphicx} 
%\usepackage[margins=2.5cm,nohead,nofoot]{geometry}
%\usepackage{geometry}
\usepackage{amsfonts}
\usepackage{amstext}
\usepackage{latexsym}
\usepackage{amssymb}
\usepackage{color}


%\include{myPreamble}
\include{qm2pi.local} 

%\ifpdf
%\usepackage[pdftex]{graphicx}
%\else
%\usepackage{graphicx}
%\fi

 % \ifpdf
%  \usepackage{pdfsync}
%  \if


%\title{Brief Article}
%\author{David F. Snyder}
%\author{L.G. Meredith}

%\address{Dept. of Math., Texas State University--San Marcos, San Marcos, TX 78666}
       
\pagestyle{empty}


\begin{document}

\lstset{language=[Objective]Caml,frame=shadowbox}

\input{qm2pi.front}

% section front matter (end)

\input{qm2pi.intro} 
 
% section introduction (end)

% \input{qm2pi.knotations} 

% section notation (end)

\input{qm2pi.process.calculi} 

% section concurrent_process_calculi_and_spatial_logics_ (end)
    
%\input{qm2pi.knots2pi} 

%\input{qm2pi.trefoil} 

%\input{qm2pi.mainthm} 

% subsection basic_interpretation (end)

%\input{qm2pi.rho.presentation} 
\subsection{The syntax and semantics of the notation system}\label{sub:the_syntax_and_semantics_of_the_notation_system} % (fold)

We now summarize a technical presentation of the calculus that
embodies our theory of dynamics. The typical presentation of such a
calculus follows the style of giving generators and relations on
them. The grammar, below, describing term constructors, freely
generates the set of processes, $\Proc$. This set is then quotiented
by a relation known as structural congruence and it is over this set
that the notion of dynamics is expressed. This presentation is
essentially that of \cite{MeredithR05} with the addition of
polyadicity and summation. For readability we have relegated some of
the technical subtleties to an appendix.

\subsubsection{Process grammar}\label{subsub:process_grammar}

\begin{mathpar}
  \inferrule* [lab=synchronization] {} {{M} \bc \pzero \;|\; x?F \;|\; x!C }
  \and
  \inferrule* [lab=abstraction] {} {{F} \bc (x)P}
  \and
  \inferrule* [lab=concretion] {} {{C} \bc \langle Q \rangle}
  \and
  \inferrule* [lab=process] {} {{P,Q} \bc M \;| \;P|Q \;|\; @{x}}
  \and
  \inferrule* [lab=name] {} {{x} \bc \quotep{P}}
\end{mathpar} 

Note that $\vec{x}$ (resp. $\vec{P}$) denotes a vector of names
(resp. processes) of length $|\vec{x}|$ (resp. $|\vec{P}|$). We adopt
the following useful abbreviations.

\begin{mathpar}
   x?(\vec{y}).P := x.(\vec{y})P \and  x\clift{\vec{P}} := x.\clift{\vec{P}}
   \and x!(y) := \lift{x}{\dropn{y}}
   \and \Pi_{i=0}^{n-1}P_i := P_0 | \ldots | P_{n-1}
\end{mathpar}

\subsubsection{Structural congruence}

\paragraph{Free and bound names and alpha-equivalence.} At the
core of structural equivalence is alpha-equivalence which identifies
process that are the same up to a change of variable. Formally, we
recognize the distinction between free and bound names. The free names
of a process, $\freenames{P}$, may be calculated recursively as
follows:

\begin{mathpar}
\freenames{\pzero} := \emptyset
  \and \\
  \freenames{x?(y).P} := \{ x \} \cup (\freenames{P} \setminus \{ y \})
  \and 
  \freenames{x!\langle P \rangle} := \{ x \} \cup \{ P \} 
  \and \\
  \freenames{P|Q} := \freenames{P} \cup \freenames{Q}
  \and \\
  \freenames{@{x}} := \{ x \}
\end{mathpar}

$\pi$
$\quotep{\pi}$

$\freenames{-} : \pi \to \mathcal{P}(\quotep{\pi})$

\begin{eqnarray*}
  \freenames{\pzero} & := & \emptyset \\
  \freenames{x?(y).P} & := & \{ x \} \cup (\freenames{P} \setminus \{ y \}) \\
  \freenames{x!\langle P \rangle} & := & \{ x \} \cup \{ P \} \\
  \freenames{P|Q} & := & \freenames{P} \cup \freenames{Q} \\
  \freenames{\dropn{x}} & := & \{ x \}
\end{eqnarray*}

The bound names of a process, $\boundnames{P}$, are those names occurring in $P$
that are not free. For example, in $x?(y).0$, the name $x$ is free, while $y$ is bound.

\begin{mathpar}
  \inferrule* [lab=monoidal-laws] {} { P|Q \equiv Q|P \and P|0 \equiv P \and P|(Q|R) \equiv (P|Q)|R }
\end{mathpar}

\begin{mathpar}
  \inferrule* [lab=alpha-equivalence] {} { (x)P \equiv (y)P\{y/x\} \and y \not\in \freenames{P} }
\end{mathpar}

\begin{definition}
Then two processes, $P,Q$, are alpha-equivalent if $P = Q\{\vec{y}/\vec{x}\}$ for
some $\vec{x} \in \boundnames{Q},\vec{y} \in \boundnames{P}$, where $Q\{\vec{y}/\vec{x}\}$
denotes the capture-avoiding substitution of $\vec{y}$ for $\vec{x}$ in $Q$.
\end{definition}

\begin{definition}
  The {\em structural congruence} \cite{SangiorgiWalker} , $\equiv$,
  between processes is the least congruence containing
  alpha-equivalence, satisfying the abelian monoid laws
  (associativity, commutativity and $\pzero$ as identity) for parallel
  composition $|$ and for summation $+$.
\end{definition}

\subsection{Name equivalence}

We take name equivalence, written $\nameeq$, to be the smallest
equivalence relation generated by the following rules.

\begin{mathpar}
\inferrule*[lab=Quote-drop]
{ }
{ \quotep{@{x}} \nameeq x }

\inferrule*[lab=Struct-equiv]
{ P \scong Q }
{ \quotep{P} \nameeq \quotep{Q} }
\end{mathpar}

The astute reader will have noticed that the mutual recursion of names
and processes imposes a mutual recursion on alpha-equivalence and
structural equivalence via name-equivalence. Fortunately, all of this
works out pleasantly and we may calculate in the natural way, free of
concern. The reader interested in the details is referred to the
appendix \ref{appendix:rho_details}.

\subsection{Substitution}

We use $\Proc$ for the set of processes, $\QProc$ for the set of
names, and $\id{\{}\vec{y} / \vec{x} \id{\}}$ to denote partial maps,
$s : \QProc \rightarrow \QProc$. A map, $s$ lifts, uniquely, to a map
on process terms, $\widehat{s} : \Proc \rightarrow \Proc$ by the
following equations.

\begin{mathpar}
  (0) \psubstp{Q}{P} := 0 \\
  (R \juxtap S) \psubstp{Q}{P}
  :=    
  (R)\psubstp{Q}{P} \juxtap (S) \psubstp{Q}{P} \\
  (x?(y).R) \psubstp{Q}{P}    
  :=    
  (x)\substp{Q}{P} (z)\concat( (R \psubstn{z}{y}) \psubstp{Q}{P} ) \\
  (\lift{x}{R}) \psubstp{Q}{P}  
  :=
  \lift{(x)\substp{Q}{P}}{ R \psubstp{Q}{P} } \\
%   (\dropn{x})  \psubstp{Q}{P}       
%   := 
%   \left\{ 
%     \begin{array}{ccc} 
%       \dropn{\quotep{Q}} & & x \nameeq \quotep{P} \\
%       \dropn{x} & & otherwise \\
%     \end{array}
%   \right. 
  (\dropn{x})  \psubstp{Q}{P}       
  := 
  \left\{ 
    \begin{array}{ccc} 
      Q & & x \nameeq \quotep{P} \\
      \dropn{x} & & otherwise \\
    \end{array}
  \right.
\end{mathpar}
 

where

\begin{eqnarray}
  (x)\id{\{} \lpquote Q \rpquote / \lpquote P \rpquote \id{\}}            = 
  \left\{ 
    \begin{array}{ccc}
      \lpquote Q \rpquote & & x \nameeq \lpquote P \rpquote \\
      x & & otherwise \\
    \end{array}
  \right. \nonumber
\end{eqnarray}

and $z$ is chosen distinct from $\quotep{P}$, $\quotep{Q}$, the free
names in $Q$, and all the names in $R$. Our $\alpha$-equivalence will
be built in the standard way from this substitution.

\begin{remark}\label{rem:no_self_referential_names}
  One consequence of these definitions is that $\forall P. \quotep{P}
  \not\in \freenames{P}$.
\end{remark}

\subsection{ Dynamic quote: an example }

Anticipating something of what's to come, consider applying the
substitution, $\widehat{\id{\{}u / z \id{\}}}$, to the following pair
of processes, $\lift{w}{y!(z)}$ and $w[ \lpquote y!(z) \rpquote ]$.

\begin{eqnarray}
	\lift{w}{y!(z)}\widehat{\id{\{}u / z \id{\}}}
		& = &
		\lift{w}{y!(u)} \nonumber\\
	w[ \lpquote y!(z) \rpquote ] \widehat{ \id{\{}u / z \id{\}} }
		& = &
		w[ \lpquote y!(z) \rpquote ] \nonumber
\end{eqnarray}

Because the body of the process between quotes is impervious to
substitution, we get radically different answers. In fact, by
examining the first process in an input context,
e.g. $x?(z).\lift{w}{y!(z)}$, we see that the process under the lift
operator may be shaped by prefixed inputs binding a name inside it. In
this sense, the lift operator will be seen as a way to dynamically
construct processes before reifying them as names.

Finally equipped with these standard features we can present the
dynamics of the calculus.

\subsubsection{Operational semantics} 

Finally, we introduce the computational dynamics. What marks these
algebras as distinct from other more traditionally studied algebraic
structures, e.g. vector spaces or polynomial rings, is the manner in
which dynamics is captured. In traditional structures, dynamics is typically
expressed through morphisms between such structures, as in linear maps
between vector spaces or morphisms between rings. In algebras
associated with the semantics of computation, the dynamics is
expressed as part of the algebraic structure itself, through a
reduction reduction relation typically denoted by $\red$. Below, we
give a recursive presentation of this relation for the calculus used
in the encoding.

$\red \subseteq \pi \times \pi$
$\red : \pi \to \mathcal{P}(\pi)$

\begin{mathpar}
  \inferrule* [lab=Comm] { \textsf{match}( x_{src}, x_{trgt} ) } { x_{trgt}?(y)P \; | \; x_{src}!\langle {Q} \rangle \red P\{\quotep{Q}/y}\} }
  \and \\
  \inferrule* [lab=Par] {{P} \red {P}'} {{{P} | {Q}} \red {{P}' | {Q}}}
  \and
  \inferrule* [lab=Equiv]{{{P} \scong {P}'} \andalso {{P}' \red {Q}'} \andalso {{Q}' \scong {Q}}}{{P} \red {Q}}
\end{mathpar}

\begin{eqnarray*}
  match_{\equiv} (\quotep{P},\quotep{Q}) & := & P \equiv Q \\
  match_{\dagger}(\quotep{P},\quotep{Q}) & := & \forall R. P|Q \red^{*} R => R \red^{*} 0 \\
  match_{K}(\quotep{P},\quotep{Q}) & := & K \mbox{ for some context } K
\end{eqnarray*}

$u?(x)P | u!\langle Q \rangle \red P\{\quotep{Q}/x\}$

%We write $\wred$ for $\red^*$, and $P\red$ if $\exists Q $ such that $ P \red Q$.
We write $P\red$ if $\exists Q $ such that $ P \red Q$ and $P\not\red$, otherwise.

\section{Replication}

As mentioned before, it is known that replication (and hence
recursion) can be implemented in a higher-order process algebra
\cite{SangiorgiWalker}. As our first example of calculation with the
machinery thus far presented we give the construction explicitly in
the {\rhoc}.

\begin{eqnarray}
	D_{x} & := & \prefix{x}{y}{(\binpar{\outputp{x}{y}}{@{y}})} \nonumber\\
	\bangp_{x}{P} & := & \binpar{{x}!\langle{\binpar{D_{x}}{P}}\rangle}{D_{x}} \nonumber
\end{eqnarray}

\begin{eqnarray}
	\bangp_{x}{P} & & \nonumber\\
	=
	& {x}!\langle{(\prefix{x}{y}{(\outputp{x}{y} | @{y})) | P}}\rangle 
	      | \prefix{x}{y}{(\outputp{x}{y} | @{y})} & \nonumber\\
	\red
	& (\outputp{x}{y} | @{y})\substn{\quotep{(\prefix{x}{y}{(@{y} | \outputp{x}{y})) | P}}}{y} & \nonumber\\
	=
	& \outputp{x}{\quotep{(\prefix{x}{y}{(\outputp{x}{y} | @{y})) | P}}}
	  | {(\prefix{x}{y}{(\outputp{x}{y} | @{y})) | P}} & \nonumber\\
	\red
	& \ldots & \nonumber\\
	\red^*
	& P | P | \ldots & \nonumber
\end{eqnarray}

Of course, this encoding, as an implementation, runs away, unfolding
$\bangp{P}$ eagerly. A lazier and more implementable replication
operator, restricted to input-guarded processes, may be obtained as follows.

\begin{eqnarray}
\bangp{\prefix{u}{v}{P}} 
	:= 
	\binpar{\lift{x}{\prefix{u}{v}{(\binpar{D(x)}{P})}}}{D(x)} \nonumber
\end{eqnarray}

\begin{remark}
  Note that the lazier definition still does not deal with summation
  or mixed summation (i.e. sums over input and output). The reader is
  invited to construct definitions of replication that deal with these
  features. 

  Further, the definitions are parameterized in a name, $x$. Can you,
  gentle reader, make a definition that eliminates this parameter and
  guarantees no accidental interaction between the replication
  machinery and the process being replicated -- i.e. no accidental
  sharing of names used by the process to get its work done and the
  name(s) used by the replication to effect copying. This latter
  revision of the definition of replication is crucial to obtaining
  the expected identity $!!P \sim !P$.
\end{remark}

\begin{remark}\label{rem:paradoxical_combinator}
  The reader familiar with the lambda calculus will have noticed the
  similarity between $D$ and the paradoxical combinator.

  [Ed. note: the existence of this seems to suggest we have to be more
  restrictive on the set of processes and names we admit if we are to
  support no-cloning.]
\end{remark}

\subsubsection{Bisimulation}

The computational dynamics gives rise to another kind of equivalence,
the equivalence of computational behavior. As previously mentioned
this is typically captured \emph{via} some form of bisimulation.

% The notion we use in this paper is weak barbed bisimulation
% \cite{milner91polyadicpi}.

The notion we use in this paper is derived from weak barbed
bisimulation \cite{milner91polyadicpi}. 

\begin{definition}
An \emph{observation relation}, $\downarrow_{\mathcal N}$, over a set
of names, $\mathcal N$, is the smallest relation satisfying the rules
below.

\infrule[Out-barb]{y \in {\mathcal N}, \; x \nameeq y}
		  {\outputp{x}{v} \downarrow_{\mathcal N} x}
\infrule[Par-barb]{\mbox{$P\downarrow_{\mathcal N} x$ or $Q\downarrow_{\mathcal N} x$}}
		  {\binpar{P}{Q} \downarrow_{\mathcal N} x}

We write $P \Downarrow_{\mathcal N} x$ if there is $Q$ such that 
$P \wred Q$ and $Q \downarrow_{\mathcal N} x$.
\end{definition}

\begin{definition}
%\label{def.bbisim}
An  ${\mathcal N}$-\emph{barbed bisimulation} over a set of names, ${\mathcal N}$, is a symmetric binary relation 
${\mathcal S}_{\mathcal N}$ between agents such that $P\rel{S}_{\mathcal N}Q$ implies:
\begin{enumerate}
\item If $P \red P'$ then $Q \wred Q'$ and $P'\rel{S}_{\mathcal N} Q'$.
\item If $P\downarrow_{\mathcal N} x$, then $Q\Downarrow_{\mathcal N} x$.
\end{enumerate}
$P$ is ${\mathcal N}$-barbed bisimilar to $Q$, written
$P \wbbisim_{\mathcal N} Q$, if $P \rel{S}_{\mathcal N} Q$ for some ${\mathcal N}$-barbed bisimulation ${\mathcal S}_{\mathcal N}$.
\end{definition}

$\mathcal{R} \subseteq \pi \times \pi$

$P \mathcal{R} Q => \forall P'. P \red P' \Rightarrow \exists Q'. Q \red Q', P' \mathcal{R} Q'$

$P \vdash x \Rightarrow Q \vdash x$

\begin{mathpar}
  \inferrule*[lab=Out-barb]{x \nameeq y}{{y}!\langle{Q}\rangle \vdash x}
  \and
  \inferrule*[lab=Par-barb]{\mbox{$P\vdash x$ or $Q\vdash x$}}{\binpar{P}{Q} \vdash x}
\end{mathpar}

\subsubsection{Contexts}

One of the principle advantages of computational calculi like the
$\pi$-calculus is a well-defined notion of context,
contextual-equivalence and a correlation between
contextual-equivalence and notions of bisimulation. The notion of
context allows the decomposition of a process into (sub-)process and
its syntactic environment, its context. Thus, a context may be
thought of as a process with a ``hole'' (written $\Box$) in it. The
application of a context $M$ to a process $P$, written $M[P]$, is
tantamount to filling the hole in $M$ with $P$. In this paper we do
not need the full weight of this theory, but do make use of the notion
of context in the proof the main theorem. 

\begin{mathpar}
  \inferrule* [lab=summation] {} {{M_{M},M_{N}} \bc \Box \;|\; x.M_{A} \;|\; M_{M}+M_{N}}
  \and
  \inferrule* [lab=agent] {} {{M_{A}} \bc (\vec{x})M_{P} \;| \; \clift{P_0,\ldots,M_{P},\ldots,P_N}}
  \and \\
  \inferrule* [lab=process] {} {{M_{P}} \bc M_{N} \;| \;P|M_{P} }
\end{mathpar} 

\begin{mathpar}
  \inferrule* [lab=sychronization] {} {M_{N} \bc \Box \;|\; x?M_{F} \;|\; x!M_{C}}
  \and
  \inferrule* [lab=abstraction] {} {{M_{F}} \bc (x)M_{P} }
  \and
  \inferrule* [lab=concretion] {} {{M_{C}} \bc \langle M_{P} \rangle }
  \and \\
  \inferrule* [lab=process] {} {{M_{P}} \bc M_{N} \;| \;P|M_{P} }
\end{mathpar}

\begin{definition}[contextual application] Given a context $M$, and
  process $P$, we define the \emph{contextual application}, $M[P] :=
  M\{P/\Box\}$. That is, the contextual application of M to P is the
  substitution of $P$ for $\Box$ in $M$.
\end{definition}

$\meaningof{-} : L \to \mathcal{P}(\pi)$

\begin{mathpar}
  \inferrule* [lab=collection] {} {\meaningof{true} = \pi, \and \meaningof{~E} = \pi \setminus \meaningof{E}, \and \meaningof{E_{1} \& E_{2}} = \meaningof{E_{1}} \cap \meaningof{E_{2}}}
\end{mathpar}

\begin{mathpar}
  \inferrule* [lab=structure] {} {\meaningof{0} = \{ P \in \pi | P \equiv 0 \}, \and \\ \meaningof{E_1 | E_2} = \{ P \in \pi | P \equiv P_{1} | P_{2}, P_{1} \in \meaningof{E_{1}}, P_{2} \in \meaningof{E_2}\} }
\end{mathpar}

\begin{mathpar}
 \inferrule* [lab=behavior] {} {\meaningof{\langle a?b \rangle E} = \{ P \in \pi | P \equiv Q | u?(y)P', \\ \and \\\\ \and \\ \;\;\; u \in \meaningof{a}, \forall z.P'\{z/y\} \in \meaningof{E\{z/b\}}\}, \and \\ \meaningof{a!E} = \{ P \in \pi | P \equiv Q | x!\langle P' \rangle, x \in \meaningof{a} P' \in \meaningof{E}\} }
\end{mathpar}

\begin{mathpar}
 \inferrule* [lab=nominal] {} {\meaningof{\quotep{E}} = \{ \quotep{P} \in \quotep{\pi} | P \in \meaningof{E} \}, \and \meaningof{\quotep{P}} = \{ \quotep{Q} \in \quotep{\pi} | P \equiv Q \} \and \\ \meaningof{@\quotep{E}} = \{ P \in \pi | P \equiv @x, x \in \meaningof{E} \}}
\end{mathpar}

\begin{eqnarray*}
  \\
  \meaningof{-} : TS \to ST
\end{eqnarray*}

\begin{eqnarray*}
  \\
  L : TS \to ST
\end{eqnarray*}

\begin{eqnarray*}
  \\
  P \models E \iff P \in \meaningof{E}
\end{eqnarray*}

\begin{eqnarray*}
  P \approx_{L} Q \iff \forall E \in L. P \models E \iff Q \models E
\end{eqnarray*}

\begin{eqnarray*}
  P \approx_{K} Q
\end{eqnarray*}

\begin{eqnarray*}
  P \approx Q
\end{eqnarray*}

$\approx_{K} = \approx = \approx_{L}$

\subsubsection{Contextual duality}

Note that contexts extend the quotation operation to a family of
operations from processes to names. Given a context, $M$, we can
define a \emph{nominal context}, $\quotep{M}$ by $\quotep{M}[P] :=
\quotep{M[P]}$. To foreshadow what is to come we observe that these
operations enjoy a duality with processes very much like the duality
between vectors and maps from vectors to scalars.

Further, because the calculus is essentially higher-order, we have a
correspondence between contexts and processes. More specifically,
given a name $x$ and a context $M$ we can construct $M^{*}_{x}$ such
that 

\begin{mathpar}
  M^{*}_{x} | \lift{x}{P} \red M[P]
\end{mathpar}

namely,

\begin{mathpar}
  M^{*}_{x} := x?(u).M[\dropn{u}]
\end{mathpar}

The dependence of $M^{*}_{x}$ on a name makes it an abstraction, 

\begin{mathpar}
  M^{*} := (x)x?(u).M[\dropn{u}]
\end{mathpar}

\subsection{Additional notation}

It will sometimes be convenient to denote the process a name
quotes. We already have the notation $x = \quotep{P}$, but it will be
convenient to introduce an alternate notation, $\procn{x}$, when we
want to emphasize the connection to the use of the name. Note that, by
virtue of name equivalence, $\quotep{\procn{x}} \nameeq x$; so, the
notation is consistent with previous definitions.

Further, because names have structure it is possible to effect
substitutions on the basis of that structure. This means we need to
upgrade our notation for substitutions, which we accomplish by
adapting comprehension notation. Thus,

\begin{mathpar}
  P\{ y / x : x \in S \}
\end{mathpar}

is interpreted to mean the process derived from P by replacing (in a
capture-avoiding manner) each occurrence of $x$ in $S$ by $y$. For example,

\begin{mathpar}
  P\{ \quotep{\procn{x}|\procn{x}} / x : x \in \freenames{P} \}
\end{mathpar}

will replace each (occurrence) of a free name $x$ in $P$ by
$\quotep{\procn{x}|\procn{x}}$.

Also, we will avail ourselves of the notation $x^{L}$ and $x^{R}$ to
denote injections of a name into disjoint copies of the name
space. There are numerous ways to accomplish this. One example can be
found in \cite{MeredithR05}. This notation overloads to vectors of
names: $\vec{x}^{\pi} := (x_{i}^{\pi} \; : \; 0 \leq i < |\vec{x}| )$ where $\pi \in \{L,R\}$.

We also use $P^{\Box} := P|\Box$.

In \cite{MeredithR05} an interpretation of the new operator is
given. It turns out that there are several possible interpretations
all enjoying the requisite algebraic properties of the operator (see
\cite{milner91polyadicpi}). We will therefore make liberal use of
$(\nu\; \vec{x})P$.

% subsection the_syntax_and_semantics_of_the_notation_system (end)   

\input{qm2pi.qmops} 

\input{qm2pi.sterngerlach} 

\input{qm2pi.metric} 

% section concurrent_process_calculi (end)

%\input{qm2pi.proofsketch}

% section proof sketch (end)

%\input{qm2pi.slviaknots} 

% section spatial logic via knots (end)

\input{qm2pi.conclusion}

% section conclusion (end)

%\input{qm2pi.dtcodes} 

% section wiring algorithm (end)

\input{qm2pi.ack} 

% section acknowledgments (end)

\newpage


\bibliographystyle{plain}   
\bibliography{../../biblios/main.bib}

\input{qm2pi.rhodetails}

\end{document}



% section front matter (end)

\section{Introduction}\label{sec:introduction} % (fold)
In this draft of the material i am going to have to dispense with the
usual writing conventions adopted in papers on these topics. i'm going
to have adopt whatever tone i need at the time i'm writing up the
calculations. Sometimes this may be very conversational; others it may
be the barest mathematical grunts; others still it may be that i have
lifted text from one of my other papers because the exposition of some
point was better said there. i hope that my readers are not unduly put
out by this decision. i'm not doing this to flout convention or be
rebellious. i find these calculations very technically challenging. To
keep everything going technically, something has to give; i have to
let go of some cognitive burden. So, the academic writing style --
with all of its trade-offs in terms of facilitating technical
communication -- is what i'm letting go of. Perhaps subsequent drafts
can be tightened and polished, but for now, i'm going to speak as if
we were sitting together in a coffee shop with a laptop, wifi and a
pad of paper and a pencil.

So, here's what i have to say. We -- you and i, comfortably ensconced
in our coffee shop and well-equipped with our tools -- can realize and
carry out the calculations of quantum mechanics over a very different
formal theory of dynamics, a formal theory of dynamics that
corresponds to a theory of concurrent computation with
\emph{reflection}. It has the advantage that the underlying theory is
already `quantized', but supports analogues all of the continuuous
operations. Strikingly, this underlying theory has recently been
connected with a notion of metric that we can show, by calculating
together, coincides with the metric induced by the inner product.

There are a lot of reasons why you might be interested in seeing
calculations of this form. Here's why i'm interested. For the past
several centuries there has been no competitor to the ``Newtonian''
account of dynamics. As a result the predominant share of accounts of
dynamical systems and situations have had to be formulated in terms of
the Newtonian machinery. i view this as an intellectually dangerous
position to occupy. Everything, despite it's intrinsic shape, turns
into a nail to be hit with this hammer. Recently, however, the theory
of computation has matured to the point where we have candidates for
theories of dynamics that offer very different perspective on
reasoning about dynamical systems and situations. Testing these
candidates against very successful accounts of dynamical situations,
like quantum mechanics, is going to give us some sense of how mature
they are and some measure of the quality of these accounts of
dynamics.

\subsection{Summary of contributions and outline of paper}

So, we're going to develop an interpretation of the operations of
quantum mechanics normally interpreted by Hilbert spaces and
operators. We're going to do this over a theory of computation. Note
that this is very different than the usual quantum computation program
which develops notions of computation over quantum mechanics. Rather,
we are developing a story that aligns with Wheeler's slogan: It from
Bit. To do this we will first provide an account of the theory of
computation at play here. Then we will dive into a calculation-driven
interpretation of the operations of quantum mechanics.

The reason we take this approach is that -- until very recently --
there hasn't been an axiomatic account of quantum mechanics. As a
result there has been no sharp delineation of the mathematical theory
supporting interpretation of the physical theory and the physical
theory, itself. So, ambient features of the maths are free to be
exploited (or supressed) without a real accounting of their physical
relevance. There is no sharp statement ``here's the physical theory''
qua \emph{theory} and ``here's the mathematical interpretation''
enabling a judgment of how faithful the interpretation is -- apart
from experimental observation. When there is an axiomatic account we
can judge how well a given mathematical formalism supports an
interpretation of the axioms, independent of
experimentation. Likewise, we can judge how well we have captured our
physical evidence and experience with our axiomatics, independent of
any specific mathematical implementation, with accidental detail that
may or may not have physical significance. 

In lieu of a fully fleshed out and vetted axiomatic account of quantum
mechanics, interpreting the operational notions in service of modeling
physical systems will have to suffice. In other words, we are not in
the business of providing a model of Hilbert spaces and operators. We
are in the business of providing a model of quantum mechanics because
we are motivated by testing our notions of dynamics against physical
theory; and, the predictive calculations of the physical theory must
serve as the best formulation -- shy of a fully fleshed out axiomatic
account -- of the physical theory itself (as they have for scientific
theories since time immemorial). Put another way, despite a
whole-hearted commitment to an It-from-Bit ontology, we are firmly
aligned with the shut-up-and-calculate camp as the best way to obtain
results either from the physical perspective or as a quality assurance
measure of our fledgling theory of dynamics.

In detail, we present a reflective process calculus. Then we develop
intuitive correspondences between the notions available in this
calculus and the usual physical notions supporting quantum mechanical
calculations. Thus, 

\begin{table}[htp]
  \center{
    \fbox{
      \begin{tabular}{c|c}
        quantum mechanics & process calculus \\
        \hline
        scalar & name \\
        state vector & process \\
        dual & contextual duals \\
        matrix & formal sums of process-context-dual pairs \\
        orthogonality & process annihilation \\
        inner product & execution-formula + quoting
      \end{tabular}
    }
  }
  \caption{QM - process calculi correspondences}
\end{table}

Then we tighten up these intuitions to operational definitions. We
employ the Dirac notation as the best proxy we can find for an
abstract syntax of the quantum mechanical notions. The definitions we
develop put us in contact with equational constraints coming from the
theory that we demonstrate the definitions and calculations satisfy.

This puts us in a position to shut up and calculate for the
Stern-Gerlach experimental set up, showing how these predictive
calculations become calculations on processes in our theory of a
reflective process calculus.

Penultimately, we demonstrate that the notion of metric coming from
the inner product coincides with the notion of metric available from
the theory of bisimulation. This demonstration gives us the right to
think of space as arising from behavior. Finally, we consider where we
might go from the new vantage point we have obtained.

% section introduction (end) 
 
% section introduction (end)

% \documentclass[12pt]{llncs}
%\documentclass{jktr}

\usepackage[pdftex]{hyperref}                   
\usepackage {listings}
\usepackage {mathpartir}
\usepackage{bcprules}
%\usepackage{listings}
                       
\usepackage{graphicx} 
%\usepackage[margins=2.5cm,nohead,nofoot]{geometry}
%\usepackage{geometry}
\usepackage{amsfonts}
\usepackage{amstext}
\usepackage{latexsym}
\usepackage{amssymb}
\usepackage{color}


%\include{myPreamble}
\include{qm2pi.local} 

%\ifpdf
%\usepackage[pdftex]{graphicx}
%\else
%\usepackage{graphicx}
%\fi

 % \ifpdf
%  \usepackage{pdfsync}
%  \if


%\title{Brief Article}
%\author{David F. Snyder}
%\author{L.G. Meredith}

%\address{Dept. of Math., Texas State University--San Marcos, San Marcos, TX 78666}
       
\pagestyle{empty}


\begin{document}

\lstset{language=[Objective]Caml,frame=shadowbox}

\input{qm2pi.front}

% section front matter (end)

\input{qm2pi.intro} 
 
% section introduction (end)

% \input{qm2pi.knotations} 

% section notation (end)

\input{qm2pi.process.calculi} 

% section concurrent_process_calculi_and_spatial_logics_ (end)
    
%\input{qm2pi.knots2pi} 

%\input{qm2pi.trefoil} 

%\input{qm2pi.mainthm} 

% subsection basic_interpretation (end)

%\input{qm2pi.rho.presentation} 
\subsection{The syntax and semantics of the notation system}\label{sub:the_syntax_and_semantics_of_the_notation_system} % (fold)

We now summarize a technical presentation of the calculus that
embodies our theory of dynamics. The typical presentation of such a
calculus follows the style of giving generators and relations on
them. The grammar, below, describing term constructors, freely
generates the set of processes, $\Proc$. This set is then quotiented
by a relation known as structural congruence and it is over this set
that the notion of dynamics is expressed. This presentation is
essentially that of \cite{MeredithR05} with the addition of
polyadicity and summation. For readability we have relegated some of
the technical subtleties to an appendix.

\subsubsection{Process grammar}\label{subsub:process_grammar}

\begin{mathpar}
  \inferrule* [lab=synchronization] {} {{M} \bc \pzero \;|\; x?F \;|\; x!C }
  \and
  \inferrule* [lab=abstraction] {} {{F} \bc (x)P}
  \and
  \inferrule* [lab=concretion] {} {{C} \bc \langle Q \rangle}
  \and
  \inferrule* [lab=process] {} {{P,Q} \bc M \;| \;P|Q \;|\; @{x}}
  \and
  \inferrule* [lab=name] {} {{x} \bc \quotep{P}}
\end{mathpar} 

Note that $\vec{x}$ (resp. $\vec{P}$) denotes a vector of names
(resp. processes) of length $|\vec{x}|$ (resp. $|\vec{P}|$). We adopt
the following useful abbreviations.

\begin{mathpar}
   x?(\vec{y}).P := x.(\vec{y})P \and  x\clift{\vec{P}} := x.\clift{\vec{P}}
   \and x!(y) := \lift{x}{\dropn{y}}
   \and \Pi_{i=0}^{n-1}P_i := P_0 | \ldots | P_{n-1}
\end{mathpar}

\subsubsection{Structural congruence}

\paragraph{Free and bound names and alpha-equivalence.} At the
core of structural equivalence is alpha-equivalence which identifies
process that are the same up to a change of variable. Formally, we
recognize the distinction between free and bound names. The free names
of a process, $\freenames{P}$, may be calculated recursively as
follows:

\begin{mathpar}
\freenames{\pzero} := \emptyset
  \and \\
  \freenames{x?(y).P} := \{ x \} \cup (\freenames{P} \setminus \{ y \})
  \and 
  \freenames{x!\langle P \rangle} := \{ x \} \cup \{ P \} 
  \and \\
  \freenames{P|Q} := \freenames{P} \cup \freenames{Q}
  \and \\
  \freenames{@{x}} := \{ x \}
\end{mathpar}

$\pi$
$\quotep{\pi}$

$\freenames{-} : \pi \to \mathcal{P}(\quotep{\pi})$

\begin{eqnarray*}
  \freenames{\pzero} & := & \emptyset \\
  \freenames{x?(y).P} & := & \{ x \} \cup (\freenames{P} \setminus \{ y \}) \\
  \freenames{x!\langle P \rangle} & := & \{ x \} \cup \{ P \} \\
  \freenames{P|Q} & := & \freenames{P} \cup \freenames{Q} \\
  \freenames{\dropn{x}} & := & \{ x \}
\end{eqnarray*}

The bound names of a process, $\boundnames{P}$, are those names occurring in $P$
that are not free. For example, in $x?(y).0$, the name $x$ is free, while $y$ is bound.

\begin{mathpar}
  \inferrule* [lab=monoidal-laws] {} { P|Q \equiv Q|P \and P|0 \equiv P \and P|(Q|R) \equiv (P|Q)|R }
\end{mathpar}

\begin{mathpar}
  \inferrule* [lab=alpha-equivalence] {} { (x)P \equiv (y)P\{y/x\} \and y \not\in \freenames{P} }
\end{mathpar}

\begin{definition}
Then two processes, $P,Q$, are alpha-equivalent if $P = Q\{\vec{y}/\vec{x}\}$ for
some $\vec{x} \in \boundnames{Q},\vec{y} \in \boundnames{P}$, where $Q\{\vec{y}/\vec{x}\}$
denotes the capture-avoiding substitution of $\vec{y}$ for $\vec{x}$ in $Q$.
\end{definition}

\begin{definition}
  The {\em structural congruence} \cite{SangiorgiWalker} , $\equiv$,
  between processes is the least congruence containing
  alpha-equivalence, satisfying the abelian monoid laws
  (associativity, commutativity and $\pzero$ as identity) for parallel
  composition $|$ and for summation $+$.
\end{definition}

\subsection{Name equivalence}

We take name equivalence, written $\nameeq$, to be the smallest
equivalence relation generated by the following rules.

\begin{mathpar}
\inferrule*[lab=Quote-drop]
{ }
{ \quotep{@{x}} \nameeq x }

\inferrule*[lab=Struct-equiv]
{ P \scong Q }
{ \quotep{P} \nameeq \quotep{Q} }
\end{mathpar}

The astute reader will have noticed that the mutual recursion of names
and processes imposes a mutual recursion on alpha-equivalence and
structural equivalence via name-equivalence. Fortunately, all of this
works out pleasantly and we may calculate in the natural way, free of
concern. The reader interested in the details is referred to the
appendix \ref{appendix:rho_details}.

\subsection{Substitution}

We use $\Proc$ for the set of processes, $\QProc$ for the set of
names, and $\id{\{}\vec{y} / \vec{x} \id{\}}$ to denote partial maps,
$s : \QProc \rightarrow \QProc$. A map, $s$ lifts, uniquely, to a map
on process terms, $\widehat{s} : \Proc \rightarrow \Proc$ by the
following equations.

\begin{mathpar}
  (0) \psubstp{Q}{P} := 0 \\
  (R \juxtap S) \psubstp{Q}{P}
  :=    
  (R)\psubstp{Q}{P} \juxtap (S) \psubstp{Q}{P} \\
  (x?(y).R) \psubstp{Q}{P}    
  :=    
  (x)\substp{Q}{P} (z)\concat( (R \psubstn{z}{y}) \psubstp{Q}{P} ) \\
  (\lift{x}{R}) \psubstp{Q}{P}  
  :=
  \lift{(x)\substp{Q}{P}}{ R \psubstp{Q}{P} } \\
%   (\dropn{x})  \psubstp{Q}{P}       
%   := 
%   \left\{ 
%     \begin{array}{ccc} 
%       \dropn{\quotep{Q}} & & x \nameeq \quotep{P} \\
%       \dropn{x} & & otherwise \\
%     \end{array}
%   \right. 
  (\dropn{x})  \psubstp{Q}{P}       
  := 
  \left\{ 
    \begin{array}{ccc} 
      Q & & x \nameeq \quotep{P} \\
      \dropn{x} & & otherwise \\
    \end{array}
  \right.
\end{mathpar}
 

where

\begin{eqnarray}
  (x)\id{\{} \lpquote Q \rpquote / \lpquote P \rpquote \id{\}}            = 
  \left\{ 
    \begin{array}{ccc}
      \lpquote Q \rpquote & & x \nameeq \lpquote P \rpquote \\
      x & & otherwise \\
    \end{array}
  \right. \nonumber
\end{eqnarray}

and $z$ is chosen distinct from $\quotep{P}$, $\quotep{Q}$, the free
names in $Q$, and all the names in $R$. Our $\alpha$-equivalence will
be built in the standard way from this substitution.

\begin{remark}\label{rem:no_self_referential_names}
  One consequence of these definitions is that $\forall P. \quotep{P}
  \not\in \freenames{P}$.
\end{remark}

\subsection{ Dynamic quote: an example }

Anticipating something of what's to come, consider applying the
substitution, $\widehat{\id{\{}u / z \id{\}}}$, to the following pair
of processes, $\lift{w}{y!(z)}$ and $w[ \lpquote y!(z) \rpquote ]$.

\begin{eqnarray}
	\lift{w}{y!(z)}\widehat{\id{\{}u / z \id{\}}}
		& = &
		\lift{w}{y!(u)} \nonumber\\
	w[ \lpquote y!(z) \rpquote ] \widehat{ \id{\{}u / z \id{\}} }
		& = &
		w[ \lpquote y!(z) \rpquote ] \nonumber
\end{eqnarray}

Because the body of the process between quotes is impervious to
substitution, we get radically different answers. In fact, by
examining the first process in an input context,
e.g. $x?(z).\lift{w}{y!(z)}$, we see that the process under the lift
operator may be shaped by prefixed inputs binding a name inside it. In
this sense, the lift operator will be seen as a way to dynamically
construct processes before reifying them as names.

Finally equipped with these standard features we can present the
dynamics of the calculus.

\subsubsection{Operational semantics} 

Finally, we introduce the computational dynamics. What marks these
algebras as distinct from other more traditionally studied algebraic
structures, e.g. vector spaces or polynomial rings, is the manner in
which dynamics is captured. In traditional structures, dynamics is typically
expressed through morphisms between such structures, as in linear maps
between vector spaces or morphisms between rings. In algebras
associated with the semantics of computation, the dynamics is
expressed as part of the algebraic structure itself, through a
reduction reduction relation typically denoted by $\red$. Below, we
give a recursive presentation of this relation for the calculus used
in the encoding.

$\red \subseteq \pi \times \pi$
$\red : \pi \to \mathcal{P}(\pi)$

\begin{mathpar}
  \inferrule* [lab=Comm] { \textsf{match}( x_{src}, x_{trgt} ) } { x_{trgt}?(y)P \; | \; x_{src}!\langle {Q} \rangle \red P\{\quotep{Q}/y}\} }
  \and \\
  \inferrule* [lab=Par] {{P} \red {P}'} {{{P} | {Q}} \red {{P}' | {Q}}}
  \and
  \inferrule* [lab=Equiv]{{{P} \scong {P}'} \andalso {{P}' \red {Q}'} \andalso {{Q}' \scong {Q}}}{{P} \red {Q}}
\end{mathpar}

\begin{eqnarray*}
  match_{\equiv} (\quotep{P},\quotep{Q}) & := & P \equiv Q \\
  match_{\dagger}(\quotep{P},\quotep{Q}) & := & \forall R. P|Q \red^{*} R => R \red^{*} 0 \\
  match_{K}(\quotep{P},\quotep{Q}) & := & K \mbox{ for some context } K
\end{eqnarray*}

$u?(x)P | u!\langle Q \rangle \red P\{\quotep{Q}/x\}$

%We write $\wred$ for $\red^*$, and $P\red$ if $\exists Q $ such that $ P \red Q$.
We write $P\red$ if $\exists Q $ such that $ P \red Q$ and $P\not\red$, otherwise.

\section{Replication}

As mentioned before, it is known that replication (and hence
recursion) can be implemented in a higher-order process algebra
\cite{SangiorgiWalker}. As our first example of calculation with the
machinery thus far presented we give the construction explicitly in
the {\rhoc}.

\begin{eqnarray}
	D_{x} & := & \prefix{x}{y}{(\binpar{\outputp{x}{y}}{@{y}})} \nonumber\\
	\bangp_{x}{P} & := & \binpar{{x}!\langle{\binpar{D_{x}}{P}}\rangle}{D_{x}} \nonumber
\end{eqnarray}

\begin{eqnarray}
	\bangp_{x}{P} & & \nonumber\\
	=
	& {x}!\langle{(\prefix{x}{y}{(\outputp{x}{y} | @{y})) | P}}\rangle 
	      | \prefix{x}{y}{(\outputp{x}{y} | @{y})} & \nonumber\\
	\red
	& (\outputp{x}{y} | @{y})\substn{\quotep{(\prefix{x}{y}{(@{y} | \outputp{x}{y})) | P}}}{y} & \nonumber\\
	=
	& \outputp{x}{\quotep{(\prefix{x}{y}{(\outputp{x}{y} | @{y})) | P}}}
	  | {(\prefix{x}{y}{(\outputp{x}{y} | @{y})) | P}} & \nonumber\\
	\red
	& \ldots & \nonumber\\
	\red^*
	& P | P | \ldots & \nonumber
\end{eqnarray}

Of course, this encoding, as an implementation, runs away, unfolding
$\bangp{P}$ eagerly. A lazier and more implementable replication
operator, restricted to input-guarded processes, may be obtained as follows.

\begin{eqnarray}
\bangp{\prefix{u}{v}{P}} 
	:= 
	\binpar{\lift{x}{\prefix{u}{v}{(\binpar{D(x)}{P})}}}{D(x)} \nonumber
\end{eqnarray}

\begin{remark}
  Note that the lazier definition still does not deal with summation
  or mixed summation (i.e. sums over input and output). The reader is
  invited to construct definitions of replication that deal with these
  features. 

  Further, the definitions are parameterized in a name, $x$. Can you,
  gentle reader, make a definition that eliminates this parameter and
  guarantees no accidental interaction between the replication
  machinery and the process being replicated -- i.e. no accidental
  sharing of names used by the process to get its work done and the
  name(s) used by the replication to effect copying. This latter
  revision of the definition of replication is crucial to obtaining
  the expected identity $!!P \sim !P$.
\end{remark}

\begin{remark}\label{rem:paradoxical_combinator}
  The reader familiar with the lambda calculus will have noticed the
  similarity between $D$ and the paradoxical combinator.

  [Ed. note: the existence of this seems to suggest we have to be more
  restrictive on the set of processes and names we admit if we are to
  support no-cloning.]
\end{remark}

\subsubsection{Bisimulation}

The computational dynamics gives rise to another kind of equivalence,
the equivalence of computational behavior. As previously mentioned
this is typically captured \emph{via} some form of bisimulation.

% The notion we use in this paper is weak barbed bisimulation
% \cite{milner91polyadicpi}.

The notion we use in this paper is derived from weak barbed
bisimulation \cite{milner91polyadicpi}. 

\begin{definition}
An \emph{observation relation}, $\downarrow_{\mathcal N}$, over a set
of names, $\mathcal N$, is the smallest relation satisfying the rules
below.

\infrule[Out-barb]{y \in {\mathcal N}, \; x \nameeq y}
		  {\outputp{x}{v} \downarrow_{\mathcal N} x}
\infrule[Par-barb]{\mbox{$P\downarrow_{\mathcal N} x$ or $Q\downarrow_{\mathcal N} x$}}
		  {\binpar{P}{Q} \downarrow_{\mathcal N} x}

We write $P \Downarrow_{\mathcal N} x$ if there is $Q$ such that 
$P \wred Q$ and $Q \downarrow_{\mathcal N} x$.
\end{definition}

\begin{definition}
%\label{def.bbisim}
An  ${\mathcal N}$-\emph{barbed bisimulation} over a set of names, ${\mathcal N}$, is a symmetric binary relation 
${\mathcal S}_{\mathcal N}$ between agents such that $P\rel{S}_{\mathcal N}Q$ implies:
\begin{enumerate}
\item If $P \red P'$ then $Q \wred Q'$ and $P'\rel{S}_{\mathcal N} Q'$.
\item If $P\downarrow_{\mathcal N} x$, then $Q\Downarrow_{\mathcal N} x$.
\end{enumerate}
$P$ is ${\mathcal N}$-barbed bisimilar to $Q$, written
$P \wbbisim_{\mathcal N} Q$, if $P \rel{S}_{\mathcal N} Q$ for some ${\mathcal N}$-barbed bisimulation ${\mathcal S}_{\mathcal N}$.
\end{definition}

$\mathcal{R} \subseteq \pi \times \pi$

$P \mathcal{R} Q => \forall P'. P \red P' \Rightarrow \exists Q'. Q \red Q', P' \mathcal{R} Q'$

$P \vdash x \Rightarrow Q \vdash x$

\begin{mathpar}
  \inferrule*[lab=Out-barb]{x \nameeq y}{{y}!\langle{Q}\rangle \vdash x}
  \and
  \inferrule*[lab=Par-barb]{\mbox{$P\vdash x$ or $Q\vdash x$}}{\binpar{P}{Q} \vdash x}
\end{mathpar}

\subsubsection{Contexts}

One of the principle advantages of computational calculi like the
$\pi$-calculus is a well-defined notion of context,
contextual-equivalence and a correlation between
contextual-equivalence and notions of bisimulation. The notion of
context allows the decomposition of a process into (sub-)process and
its syntactic environment, its context. Thus, a context may be
thought of as a process with a ``hole'' (written $\Box$) in it. The
application of a context $M$ to a process $P$, written $M[P]$, is
tantamount to filling the hole in $M$ with $P$. In this paper we do
not need the full weight of this theory, but do make use of the notion
of context in the proof the main theorem. 

\begin{mathpar}
  \inferrule* [lab=summation] {} {{M_{M},M_{N}} \bc \Box \;|\; x.M_{A} \;|\; M_{M}+M_{N}}
  \and
  \inferrule* [lab=agent] {} {{M_{A}} \bc (\vec{x})M_{P} \;| \; \clift{P_0,\ldots,M_{P},\ldots,P_N}}
  \and \\
  \inferrule* [lab=process] {} {{M_{P}} \bc M_{N} \;| \;P|M_{P} }
\end{mathpar} 

\begin{mathpar}
  \inferrule* [lab=sychronization] {} {M_{N} \bc \Box \;|\; x?M_{F} \;|\; x!M_{C}}
  \and
  \inferrule* [lab=abstraction] {} {{M_{F}} \bc (x)M_{P} }
  \and
  \inferrule* [lab=concretion] {} {{M_{C}} \bc \langle M_{P} \rangle }
  \and \\
  \inferrule* [lab=process] {} {{M_{P}} \bc M_{N} \;| \;P|M_{P} }
\end{mathpar}

\begin{definition}[contextual application] Given a context $M$, and
  process $P$, we define the \emph{contextual application}, $M[P] :=
  M\{P/\Box\}$. That is, the contextual application of M to P is the
  substitution of $P$ for $\Box$ in $M$.
\end{definition}

$\meaningof{-} : L \to \mathcal{P}(\pi)$

\begin{mathpar}
  \inferrule* [lab=collection] {} {\meaningof{true} = \pi, \and \meaningof{~E} = \pi \setminus \meaningof{E}, \and \meaningof{E_{1} \& E_{2}} = \meaningof{E_{1}} \cap \meaningof{E_{2}}}
\end{mathpar}

\begin{mathpar}
  \inferrule* [lab=structure] {} {\meaningof{0} = \{ P \in \pi | P \equiv 0 \}, \and \\ \meaningof{E_1 | E_2} = \{ P \in \pi | P \equiv P_{1} | P_{2}, P_{1} \in \meaningof{E_{1}}, P_{2} \in \meaningof{E_2}\} }
\end{mathpar}

\begin{mathpar}
 \inferrule* [lab=behavior] {} {\meaningof{\langle a?b \rangle E} = \{ P \in \pi | P \equiv Q | u?(y)P', \\ \and \\\\ \and \\ \;\;\; u \in \meaningof{a}, \forall z.P'\{z/y\} \in \meaningof{E\{z/b\}}\}, \and \\ \meaningof{a!E} = \{ P \in \pi | P \equiv Q | x!\langle P' \rangle, x \in \meaningof{a} P' \in \meaningof{E}\} }
\end{mathpar}

\begin{mathpar}
 \inferrule* [lab=nominal] {} {\meaningof{\quotep{E}} = \{ \quotep{P} \in \quotep{\pi} | P \in \meaningof{E} \}, \and \meaningof{\quotep{P}} = \{ \quotep{Q} \in \quotep{\pi} | P \equiv Q \} \and \\ \meaningof{@\quotep{E}} = \{ P \in \pi | P \equiv @x, x \in \meaningof{E} \}}
\end{mathpar}

\begin{eqnarray*}
  \\
  \meaningof{-} : TS \to ST
\end{eqnarray*}

\begin{eqnarray*}
  \\
  L : TS \to ST
\end{eqnarray*}

\begin{eqnarray*}
  \\
  P \models E \iff P \in \meaningof{E}
\end{eqnarray*}

\begin{eqnarray*}
  P \approx_{L} Q \iff \forall E \in L. P \models E \iff Q \models E
\end{eqnarray*}

\begin{eqnarray*}
  P \approx_{K} Q
\end{eqnarray*}

\begin{eqnarray*}
  P \approx Q
\end{eqnarray*}

$\approx_{K} = \approx = \approx_{L}$

\subsubsection{Contextual duality}

Note that contexts extend the quotation operation to a family of
operations from processes to names. Given a context, $M$, we can
define a \emph{nominal context}, $\quotep{M}$ by $\quotep{M}[P] :=
\quotep{M[P]}$. To foreshadow what is to come we observe that these
operations enjoy a duality with processes very much like the duality
between vectors and maps from vectors to scalars.

Further, because the calculus is essentially higher-order, we have a
correspondence between contexts and processes. More specifically,
given a name $x$ and a context $M$ we can construct $M^{*}_{x}$ such
that 

\begin{mathpar}
  M^{*}_{x} | \lift{x}{P} \red M[P]
\end{mathpar}

namely,

\begin{mathpar}
  M^{*}_{x} := x?(u).M[\dropn{u}]
\end{mathpar}

The dependence of $M^{*}_{x}$ on a name makes it an abstraction, 

\begin{mathpar}
  M^{*} := (x)x?(u).M[\dropn{u}]
\end{mathpar}

\subsection{Additional notation}

It will sometimes be convenient to denote the process a name
quotes. We already have the notation $x = \quotep{P}$, but it will be
convenient to introduce an alternate notation, $\procn{x}$, when we
want to emphasize the connection to the use of the name. Note that, by
virtue of name equivalence, $\quotep{\procn{x}} \nameeq x$; so, the
notation is consistent with previous definitions.

Further, because names have structure it is possible to effect
substitutions on the basis of that structure. This means we need to
upgrade our notation for substitutions, which we accomplish by
adapting comprehension notation. Thus,

\begin{mathpar}
  P\{ y / x : x \in S \}
\end{mathpar}

is interpreted to mean the process derived from P by replacing (in a
capture-avoiding manner) each occurrence of $x$ in $S$ by $y$. For example,

\begin{mathpar}
  P\{ \quotep{\procn{x}|\procn{x}} / x : x \in \freenames{P} \}
\end{mathpar}

will replace each (occurrence) of a free name $x$ in $P$ by
$\quotep{\procn{x}|\procn{x}}$.

Also, we will avail ourselves of the notation $x^{L}$ and $x^{R}$ to
denote injections of a name into disjoint copies of the name
space. There are numerous ways to accomplish this. One example can be
found in \cite{MeredithR05}. This notation overloads to vectors of
names: $\vec{x}^{\pi} := (x_{i}^{\pi} \; : \; 0 \leq i < |\vec{x}| )$ where $\pi \in \{L,R\}$.

We also use $P^{\Box} := P|\Box$.

In \cite{MeredithR05} an interpretation of the new operator is
given. It turns out that there are several possible interpretations
all enjoying the requisite algebraic properties of the operator (see
\cite{milner91polyadicpi}). We will therefore make liberal use of
$(\nu\; \vec{x})P$.

% subsection the_syntax_and_semantics_of_the_notation_system (end)   

\input{qm2pi.qmops} 

\input{qm2pi.sterngerlach} 

\input{qm2pi.metric} 

% section concurrent_process_calculi (end)

%\input{qm2pi.proofsketch}

% section proof sketch (end)

%\input{qm2pi.slviaknots} 

% section spatial logic via knots (end)

\input{qm2pi.conclusion}

% section conclusion (end)

%\input{qm2pi.dtcodes} 

% section wiring algorithm (end)

\input{qm2pi.ack} 

% section acknowledgments (end)

\newpage


\bibliographystyle{plain}   
\bibliography{../../biblios/main.bib}

\input{qm2pi.rhodetails}

\end{document}

 

% section notation (end)

\input{qm2pi.process.calculi} 

% section concurrent_process_calculi_and_spatial_logics_ (end)
    
%\documentclass[12pt]{llncs}
%\documentclass{jktr}

\usepackage[pdftex]{hyperref}                   
\usepackage {listings}
\usepackage {mathpartir}
\usepackage{bcprules}
%\usepackage{listings}
                       
\usepackage{graphicx} 
%\usepackage[margins=2.5cm,nohead,nofoot]{geometry}
%\usepackage{geometry}
\usepackage{amsfonts}
\usepackage{amstext}
\usepackage{latexsym}
\usepackage{amssymb}
\usepackage{color}


%\include{myPreamble}
\include{qm2pi.local} 

%\ifpdf
%\usepackage[pdftex]{graphicx}
%\else
%\usepackage{graphicx}
%\fi

 % \ifpdf
%  \usepackage{pdfsync}
%  \if


%\title{Brief Article}
%\author{David F. Snyder}
%\author{L.G. Meredith}

%\address{Dept. of Math., Texas State University--San Marcos, San Marcos, TX 78666}
       
\pagestyle{empty}


\begin{document}

\lstset{language=[Objective]Caml,frame=shadowbox}

\input{qm2pi.front}

% section front matter (end)

\input{qm2pi.intro} 
 
% section introduction (end)

% \input{qm2pi.knotations} 

% section notation (end)

\input{qm2pi.process.calculi} 

% section concurrent_process_calculi_and_spatial_logics_ (end)
    
%\input{qm2pi.knots2pi} 

%\input{qm2pi.trefoil} 

%\input{qm2pi.mainthm} 

% subsection basic_interpretation (end)

%\input{qm2pi.rho.presentation} 
\subsection{The syntax and semantics of the notation system}\label{sub:the_syntax_and_semantics_of_the_notation_system} % (fold)

We now summarize a technical presentation of the calculus that
embodies our theory of dynamics. The typical presentation of such a
calculus follows the style of giving generators and relations on
them. The grammar, below, describing term constructors, freely
generates the set of processes, $\Proc$. This set is then quotiented
by a relation known as structural congruence and it is over this set
that the notion of dynamics is expressed. This presentation is
essentially that of \cite{MeredithR05} with the addition of
polyadicity and summation. For readability we have relegated some of
the technical subtleties to an appendix.

\subsubsection{Process grammar}\label{subsub:process_grammar}

\begin{mathpar}
  \inferrule* [lab=synchronization] {} {{M} \bc \pzero \;|\; x?F \;|\; x!C }
  \and
  \inferrule* [lab=abstraction] {} {{F} \bc (x)P}
  \and
  \inferrule* [lab=concretion] {} {{C} \bc \langle Q \rangle}
  \and
  \inferrule* [lab=process] {} {{P,Q} \bc M \;| \;P|Q \;|\; @{x}}
  \and
  \inferrule* [lab=name] {} {{x} \bc \quotep{P}}
\end{mathpar} 

Note that $\vec{x}$ (resp. $\vec{P}$) denotes a vector of names
(resp. processes) of length $|\vec{x}|$ (resp. $|\vec{P}|$). We adopt
the following useful abbreviations.

\begin{mathpar}
   x?(\vec{y}).P := x.(\vec{y})P \and  x\clift{\vec{P}} := x.\clift{\vec{P}}
   \and x!(y) := \lift{x}{\dropn{y}}
   \and \Pi_{i=0}^{n-1}P_i := P_0 | \ldots | P_{n-1}
\end{mathpar}

\subsubsection{Structural congruence}

\paragraph{Free and bound names and alpha-equivalence.} At the
core of structural equivalence is alpha-equivalence which identifies
process that are the same up to a change of variable. Formally, we
recognize the distinction between free and bound names. The free names
of a process, $\freenames{P}$, may be calculated recursively as
follows:

\begin{mathpar}
\freenames{\pzero} := \emptyset
  \and \\
  \freenames{x?(y).P} := \{ x \} \cup (\freenames{P} \setminus \{ y \})
  \and 
  \freenames{x!\langle P \rangle} := \{ x \} \cup \{ P \} 
  \and \\
  \freenames{P|Q} := \freenames{P} \cup \freenames{Q}
  \and \\
  \freenames{@{x}} := \{ x \}
\end{mathpar}

$\pi$
$\quotep{\pi}$

$\freenames{-} : \pi \to \mathcal{P}(\quotep{\pi})$

\begin{eqnarray*}
  \freenames{\pzero} & := & \emptyset \\
  \freenames{x?(y).P} & := & \{ x \} \cup (\freenames{P} \setminus \{ y \}) \\
  \freenames{x!\langle P \rangle} & := & \{ x \} \cup \{ P \} \\
  \freenames{P|Q} & := & \freenames{P} \cup \freenames{Q} \\
  \freenames{\dropn{x}} & := & \{ x \}
\end{eqnarray*}

The bound names of a process, $\boundnames{P}$, are those names occurring in $P$
that are not free. For example, in $x?(y).0$, the name $x$ is free, while $y$ is bound.

\begin{mathpar}
  \inferrule* [lab=monoidal-laws] {} { P|Q \equiv Q|P \and P|0 \equiv P \and P|(Q|R) \equiv (P|Q)|R }
\end{mathpar}

\begin{mathpar}
  \inferrule* [lab=alpha-equivalence] {} { (x)P \equiv (y)P\{y/x\} \and y \not\in \freenames{P} }
\end{mathpar}

\begin{definition}
Then two processes, $P,Q$, are alpha-equivalent if $P = Q\{\vec{y}/\vec{x}\}$ for
some $\vec{x} \in \boundnames{Q},\vec{y} \in \boundnames{P}$, where $Q\{\vec{y}/\vec{x}\}$
denotes the capture-avoiding substitution of $\vec{y}$ for $\vec{x}$ in $Q$.
\end{definition}

\begin{definition}
  The {\em structural congruence} \cite{SangiorgiWalker} , $\equiv$,
  between processes is the least congruence containing
  alpha-equivalence, satisfying the abelian monoid laws
  (associativity, commutativity and $\pzero$ as identity) for parallel
  composition $|$ and for summation $+$.
\end{definition}

\subsection{Name equivalence}

We take name equivalence, written $\nameeq$, to be the smallest
equivalence relation generated by the following rules.

\begin{mathpar}
\inferrule*[lab=Quote-drop]
{ }
{ \quotep{@{x}} \nameeq x }

\inferrule*[lab=Struct-equiv]
{ P \scong Q }
{ \quotep{P} \nameeq \quotep{Q} }
\end{mathpar}

The astute reader will have noticed that the mutual recursion of names
and processes imposes a mutual recursion on alpha-equivalence and
structural equivalence via name-equivalence. Fortunately, all of this
works out pleasantly and we may calculate in the natural way, free of
concern. The reader interested in the details is referred to the
appendix \ref{appendix:rho_details}.

\subsection{Substitution}

We use $\Proc$ for the set of processes, $\QProc$ for the set of
names, and $\id{\{}\vec{y} / \vec{x} \id{\}}$ to denote partial maps,
$s : \QProc \rightarrow \QProc$. A map, $s$ lifts, uniquely, to a map
on process terms, $\widehat{s} : \Proc \rightarrow \Proc$ by the
following equations.

\begin{mathpar}
  (0) \psubstp{Q}{P} := 0 \\
  (R \juxtap S) \psubstp{Q}{P}
  :=    
  (R)\psubstp{Q}{P} \juxtap (S) \psubstp{Q}{P} \\
  (x?(y).R) \psubstp{Q}{P}    
  :=    
  (x)\substp{Q}{P} (z)\concat( (R \psubstn{z}{y}) \psubstp{Q}{P} ) \\
  (\lift{x}{R}) \psubstp{Q}{P}  
  :=
  \lift{(x)\substp{Q}{P}}{ R \psubstp{Q}{P} } \\
%   (\dropn{x})  \psubstp{Q}{P}       
%   := 
%   \left\{ 
%     \begin{array}{ccc} 
%       \dropn{\quotep{Q}} & & x \nameeq \quotep{P} \\
%       \dropn{x} & & otherwise \\
%     \end{array}
%   \right. 
  (\dropn{x})  \psubstp{Q}{P}       
  := 
  \left\{ 
    \begin{array}{ccc} 
      Q & & x \nameeq \quotep{P} \\
      \dropn{x} & & otherwise \\
    \end{array}
  \right.
\end{mathpar}
 

where

\begin{eqnarray}
  (x)\id{\{} \lpquote Q \rpquote / \lpquote P \rpquote \id{\}}            = 
  \left\{ 
    \begin{array}{ccc}
      \lpquote Q \rpquote & & x \nameeq \lpquote P \rpquote \\
      x & & otherwise \\
    \end{array}
  \right. \nonumber
\end{eqnarray}

and $z$ is chosen distinct from $\quotep{P}$, $\quotep{Q}$, the free
names in $Q$, and all the names in $R$. Our $\alpha$-equivalence will
be built in the standard way from this substitution.

\begin{remark}\label{rem:no_self_referential_names}
  One consequence of these definitions is that $\forall P. \quotep{P}
  \not\in \freenames{P}$.
\end{remark}

\subsection{ Dynamic quote: an example }

Anticipating something of what's to come, consider applying the
substitution, $\widehat{\id{\{}u / z \id{\}}}$, to the following pair
of processes, $\lift{w}{y!(z)}$ and $w[ \lpquote y!(z) \rpquote ]$.

\begin{eqnarray}
	\lift{w}{y!(z)}\widehat{\id{\{}u / z \id{\}}}
		& = &
		\lift{w}{y!(u)} \nonumber\\
	w[ \lpquote y!(z) \rpquote ] \widehat{ \id{\{}u / z \id{\}} }
		& = &
		w[ \lpquote y!(z) \rpquote ] \nonumber
\end{eqnarray}

Because the body of the process between quotes is impervious to
substitution, we get radically different answers. In fact, by
examining the first process in an input context,
e.g. $x?(z).\lift{w}{y!(z)}$, we see that the process under the lift
operator may be shaped by prefixed inputs binding a name inside it. In
this sense, the lift operator will be seen as a way to dynamically
construct processes before reifying them as names.

Finally equipped with these standard features we can present the
dynamics of the calculus.

\subsubsection{Operational semantics} 

Finally, we introduce the computational dynamics. What marks these
algebras as distinct from other more traditionally studied algebraic
structures, e.g. vector spaces or polynomial rings, is the manner in
which dynamics is captured. In traditional structures, dynamics is typically
expressed through morphisms between such structures, as in linear maps
between vector spaces or morphisms between rings. In algebras
associated with the semantics of computation, the dynamics is
expressed as part of the algebraic structure itself, through a
reduction reduction relation typically denoted by $\red$. Below, we
give a recursive presentation of this relation for the calculus used
in the encoding.

$\red \subseteq \pi \times \pi$
$\red : \pi \to \mathcal{P}(\pi)$

\begin{mathpar}
  \inferrule* [lab=Comm] { \textsf{match}( x_{src}, x_{trgt} ) } { x_{trgt}?(y)P \; | \; x_{src}!\langle {Q} \rangle \red P\{\quotep{Q}/y}\} }
  \and \\
  \inferrule* [lab=Par] {{P} \red {P}'} {{{P} | {Q}} \red {{P}' | {Q}}}
  \and
  \inferrule* [lab=Equiv]{{{P} \scong {P}'} \andalso {{P}' \red {Q}'} \andalso {{Q}' \scong {Q}}}{{P} \red {Q}}
\end{mathpar}

\begin{eqnarray*}
  match_{\equiv} (\quotep{P},\quotep{Q}) & := & P \equiv Q \\
  match_{\dagger}(\quotep{P},\quotep{Q}) & := & \forall R. P|Q \red^{*} R => R \red^{*} 0 \\
  match_{K}(\quotep{P},\quotep{Q}) & := & K \mbox{ for some context } K
\end{eqnarray*}

$u?(x)P | u!\langle Q \rangle \red P\{\quotep{Q}/x\}$

%We write $\wred$ for $\red^*$, and $P\red$ if $\exists Q $ such that $ P \red Q$.
We write $P\red$ if $\exists Q $ such that $ P \red Q$ and $P\not\red$, otherwise.

\section{Replication}

As mentioned before, it is known that replication (and hence
recursion) can be implemented in a higher-order process algebra
\cite{SangiorgiWalker}. As our first example of calculation with the
machinery thus far presented we give the construction explicitly in
the {\rhoc}.

\begin{eqnarray}
	D_{x} & := & \prefix{x}{y}{(\binpar{\outputp{x}{y}}{@{y}})} \nonumber\\
	\bangp_{x}{P} & := & \binpar{{x}!\langle{\binpar{D_{x}}{P}}\rangle}{D_{x}} \nonumber
\end{eqnarray}

\begin{eqnarray}
	\bangp_{x}{P} & & \nonumber\\
	=
	& {x}!\langle{(\prefix{x}{y}{(\outputp{x}{y} | @{y})) | P}}\rangle 
	      | \prefix{x}{y}{(\outputp{x}{y} | @{y})} & \nonumber\\
	\red
	& (\outputp{x}{y} | @{y})\substn{\quotep{(\prefix{x}{y}{(@{y} | \outputp{x}{y})) | P}}}{y} & \nonumber\\
	=
	& \outputp{x}{\quotep{(\prefix{x}{y}{(\outputp{x}{y} | @{y})) | P}}}
	  | {(\prefix{x}{y}{(\outputp{x}{y} | @{y})) | P}} & \nonumber\\
	\red
	& \ldots & \nonumber\\
	\red^*
	& P | P | \ldots & \nonumber
\end{eqnarray}

Of course, this encoding, as an implementation, runs away, unfolding
$\bangp{P}$ eagerly. A lazier and more implementable replication
operator, restricted to input-guarded processes, may be obtained as follows.

\begin{eqnarray}
\bangp{\prefix{u}{v}{P}} 
	:= 
	\binpar{\lift{x}{\prefix{u}{v}{(\binpar{D(x)}{P})}}}{D(x)} \nonumber
\end{eqnarray}

\begin{remark}
  Note that the lazier definition still does not deal with summation
  or mixed summation (i.e. sums over input and output). The reader is
  invited to construct definitions of replication that deal with these
  features. 

  Further, the definitions are parameterized in a name, $x$. Can you,
  gentle reader, make a definition that eliminates this parameter and
  guarantees no accidental interaction between the replication
  machinery and the process being replicated -- i.e. no accidental
  sharing of names used by the process to get its work done and the
  name(s) used by the replication to effect copying. This latter
  revision of the definition of replication is crucial to obtaining
  the expected identity $!!P \sim !P$.
\end{remark}

\begin{remark}\label{rem:paradoxical_combinator}
  The reader familiar with the lambda calculus will have noticed the
  similarity between $D$ and the paradoxical combinator.

  [Ed. note: the existence of this seems to suggest we have to be more
  restrictive on the set of processes and names we admit if we are to
  support no-cloning.]
\end{remark}

\subsubsection{Bisimulation}

The computational dynamics gives rise to another kind of equivalence,
the equivalence of computational behavior. As previously mentioned
this is typically captured \emph{via} some form of bisimulation.

% The notion we use in this paper is weak barbed bisimulation
% \cite{milner91polyadicpi}.

The notion we use in this paper is derived from weak barbed
bisimulation \cite{milner91polyadicpi}. 

\begin{definition}
An \emph{observation relation}, $\downarrow_{\mathcal N}$, over a set
of names, $\mathcal N$, is the smallest relation satisfying the rules
below.

\infrule[Out-barb]{y \in {\mathcal N}, \; x \nameeq y}
		  {\outputp{x}{v} \downarrow_{\mathcal N} x}
\infrule[Par-barb]{\mbox{$P\downarrow_{\mathcal N} x$ or $Q\downarrow_{\mathcal N} x$}}
		  {\binpar{P}{Q} \downarrow_{\mathcal N} x}

We write $P \Downarrow_{\mathcal N} x$ if there is $Q$ such that 
$P \wred Q$ and $Q \downarrow_{\mathcal N} x$.
\end{definition}

\begin{definition}
%\label{def.bbisim}
An  ${\mathcal N}$-\emph{barbed bisimulation} over a set of names, ${\mathcal N}$, is a symmetric binary relation 
${\mathcal S}_{\mathcal N}$ between agents such that $P\rel{S}_{\mathcal N}Q$ implies:
\begin{enumerate}
\item If $P \red P'$ then $Q \wred Q'$ and $P'\rel{S}_{\mathcal N} Q'$.
\item If $P\downarrow_{\mathcal N} x$, then $Q\Downarrow_{\mathcal N} x$.
\end{enumerate}
$P$ is ${\mathcal N}$-barbed bisimilar to $Q$, written
$P \wbbisim_{\mathcal N} Q$, if $P \rel{S}_{\mathcal N} Q$ for some ${\mathcal N}$-barbed bisimulation ${\mathcal S}_{\mathcal N}$.
\end{definition}

$\mathcal{R} \subseteq \pi \times \pi$

$P \mathcal{R} Q => \forall P'. P \red P' \Rightarrow \exists Q'. Q \red Q', P' \mathcal{R} Q'$

$P \vdash x \Rightarrow Q \vdash x$

\begin{mathpar}
  \inferrule*[lab=Out-barb]{x \nameeq y}{{y}!\langle{Q}\rangle \vdash x}
  \and
  \inferrule*[lab=Par-barb]{\mbox{$P\vdash x$ or $Q\vdash x$}}{\binpar{P}{Q} \vdash x}
\end{mathpar}

\subsubsection{Contexts}

One of the principle advantages of computational calculi like the
$\pi$-calculus is a well-defined notion of context,
contextual-equivalence and a correlation between
contextual-equivalence and notions of bisimulation. The notion of
context allows the decomposition of a process into (sub-)process and
its syntactic environment, its context. Thus, a context may be
thought of as a process with a ``hole'' (written $\Box$) in it. The
application of a context $M$ to a process $P$, written $M[P]$, is
tantamount to filling the hole in $M$ with $P$. In this paper we do
not need the full weight of this theory, but do make use of the notion
of context in the proof the main theorem. 

\begin{mathpar}
  \inferrule* [lab=summation] {} {{M_{M},M_{N}} \bc \Box \;|\; x.M_{A} \;|\; M_{M}+M_{N}}
  \and
  \inferrule* [lab=agent] {} {{M_{A}} \bc (\vec{x})M_{P} \;| \; \clift{P_0,\ldots,M_{P},\ldots,P_N}}
  \and \\
  \inferrule* [lab=process] {} {{M_{P}} \bc M_{N} \;| \;P|M_{P} }
\end{mathpar} 

\begin{mathpar}
  \inferrule* [lab=sychronization] {} {M_{N} \bc \Box \;|\; x?M_{F} \;|\; x!M_{C}}
  \and
  \inferrule* [lab=abstraction] {} {{M_{F}} \bc (x)M_{P} }
  \and
  \inferrule* [lab=concretion] {} {{M_{C}} \bc \langle M_{P} \rangle }
  \and \\
  \inferrule* [lab=process] {} {{M_{P}} \bc M_{N} \;| \;P|M_{P} }
\end{mathpar}

\begin{definition}[contextual application] Given a context $M$, and
  process $P$, we define the \emph{contextual application}, $M[P] :=
  M\{P/\Box\}$. That is, the contextual application of M to P is the
  substitution of $P$ for $\Box$ in $M$.
\end{definition}

$\meaningof{-} : L \to \mathcal{P}(\pi)$

\begin{mathpar}
  \inferrule* [lab=collection] {} {\meaningof{true} = \pi, \and \meaningof{~E} = \pi \setminus \meaningof{E}, \and \meaningof{E_{1} \& E_{2}} = \meaningof{E_{1}} \cap \meaningof{E_{2}}}
\end{mathpar}

\begin{mathpar}
  \inferrule* [lab=structure] {} {\meaningof{0} = \{ P \in \pi | P \equiv 0 \}, \and \\ \meaningof{E_1 | E_2} = \{ P \in \pi | P \equiv P_{1} | P_{2}, P_{1} \in \meaningof{E_{1}}, P_{2} \in \meaningof{E_2}\} }
\end{mathpar}

\begin{mathpar}
 \inferrule* [lab=behavior] {} {\meaningof{\langle a?b \rangle E} = \{ P \in \pi | P \equiv Q | u?(y)P', \\ \and \\\\ \and \\ \;\;\; u \in \meaningof{a}, \forall z.P'\{z/y\} \in \meaningof{E\{z/b\}}\}, \and \\ \meaningof{a!E} = \{ P \in \pi | P \equiv Q | x!\langle P' \rangle, x \in \meaningof{a} P' \in \meaningof{E}\} }
\end{mathpar}

\begin{mathpar}
 \inferrule* [lab=nominal] {} {\meaningof{\quotep{E}} = \{ \quotep{P} \in \quotep{\pi} | P \in \meaningof{E} \}, \and \meaningof{\quotep{P}} = \{ \quotep{Q} \in \quotep{\pi} | P \equiv Q \} \and \\ \meaningof{@\quotep{E}} = \{ P \in \pi | P \equiv @x, x \in \meaningof{E} \}}
\end{mathpar}

\begin{eqnarray*}
  \\
  \meaningof{-} : TS \to ST
\end{eqnarray*}

\begin{eqnarray*}
  \\
  L : TS \to ST
\end{eqnarray*}

\begin{eqnarray*}
  \\
  P \models E \iff P \in \meaningof{E}
\end{eqnarray*}

\begin{eqnarray*}
  P \approx_{L} Q \iff \forall E \in L. P \models E \iff Q \models E
\end{eqnarray*}

\begin{eqnarray*}
  P \approx_{K} Q
\end{eqnarray*}

\begin{eqnarray*}
  P \approx Q
\end{eqnarray*}

$\approx_{K} = \approx = \approx_{L}$

\subsubsection{Contextual duality}

Note that contexts extend the quotation operation to a family of
operations from processes to names. Given a context, $M$, we can
define a \emph{nominal context}, $\quotep{M}$ by $\quotep{M}[P] :=
\quotep{M[P]}$. To foreshadow what is to come we observe that these
operations enjoy a duality with processes very much like the duality
between vectors and maps from vectors to scalars.

Further, because the calculus is essentially higher-order, we have a
correspondence between contexts and processes. More specifically,
given a name $x$ and a context $M$ we can construct $M^{*}_{x}$ such
that 

\begin{mathpar}
  M^{*}_{x} | \lift{x}{P} \red M[P]
\end{mathpar}

namely,

\begin{mathpar}
  M^{*}_{x} := x?(u).M[\dropn{u}]
\end{mathpar}

The dependence of $M^{*}_{x}$ on a name makes it an abstraction, 

\begin{mathpar}
  M^{*} := (x)x?(u).M[\dropn{u}]
\end{mathpar}

\subsection{Additional notation}

It will sometimes be convenient to denote the process a name
quotes. We already have the notation $x = \quotep{P}$, but it will be
convenient to introduce an alternate notation, $\procn{x}$, when we
want to emphasize the connection to the use of the name. Note that, by
virtue of name equivalence, $\quotep{\procn{x}} \nameeq x$; so, the
notation is consistent with previous definitions.

Further, because names have structure it is possible to effect
substitutions on the basis of that structure. This means we need to
upgrade our notation for substitutions, which we accomplish by
adapting comprehension notation. Thus,

\begin{mathpar}
  P\{ y / x : x \in S \}
\end{mathpar}

is interpreted to mean the process derived from P by replacing (in a
capture-avoiding manner) each occurrence of $x$ in $S$ by $y$. For example,

\begin{mathpar}
  P\{ \quotep{\procn{x}|\procn{x}} / x : x \in \freenames{P} \}
\end{mathpar}

will replace each (occurrence) of a free name $x$ in $P$ by
$\quotep{\procn{x}|\procn{x}}$.

Also, we will avail ourselves of the notation $x^{L}$ and $x^{R}$ to
denote injections of a name into disjoint copies of the name
space. There are numerous ways to accomplish this. One example can be
found in \cite{MeredithR05}. This notation overloads to vectors of
names: $\vec{x}^{\pi} := (x_{i}^{\pi} \; : \; 0 \leq i < |\vec{x}| )$ where $\pi \in \{L,R\}$.

We also use $P^{\Box} := P|\Box$.

In \cite{MeredithR05} an interpretation of the new operator is
given. It turns out that there are several possible interpretations
all enjoying the requisite algebraic properties of the operator (see
\cite{milner91polyadicpi}). We will therefore make liberal use of
$(\nu\; \vec{x})P$.

% subsection the_syntax_and_semantics_of_the_notation_system (end)   

\input{qm2pi.qmops} 

\input{qm2pi.sterngerlach} 

\input{qm2pi.metric} 

% section concurrent_process_calculi (end)

%\input{qm2pi.proofsketch}

% section proof sketch (end)

%\input{qm2pi.slviaknots} 

% section spatial logic via knots (end)

\input{qm2pi.conclusion}

% section conclusion (end)

%\input{qm2pi.dtcodes} 

% section wiring algorithm (end)

\input{qm2pi.ack} 

% section acknowledgments (end)

\newpage


\bibliographystyle{plain}   
\bibliography{../../biblios/main.bib}

\input{qm2pi.rhodetails}

\end{document}

 

%\documentclass[12pt]{llncs}
%\documentclass{jktr}

\usepackage[pdftex]{hyperref}                   
\usepackage {listings}
\usepackage {mathpartir}
\usepackage{bcprules}
%\usepackage{listings}
                       
\usepackage{graphicx} 
%\usepackage[margins=2.5cm,nohead,nofoot]{geometry}
%\usepackage{geometry}
\usepackage{amsfonts}
\usepackage{amstext}
\usepackage{latexsym}
\usepackage{amssymb}
\usepackage{color}


%\include{myPreamble}
\include{qm2pi.local} 

%\ifpdf
%\usepackage[pdftex]{graphicx}
%\else
%\usepackage{graphicx}
%\fi

 % \ifpdf
%  \usepackage{pdfsync}
%  \if


%\title{Brief Article}
%\author{David F. Snyder}
%\author{L.G. Meredith}

%\address{Dept. of Math., Texas State University--San Marcos, San Marcos, TX 78666}
       
\pagestyle{empty}


\begin{document}

\lstset{language=[Objective]Caml,frame=shadowbox}

\input{qm2pi.front}

% section front matter (end)

\input{qm2pi.intro} 
 
% section introduction (end)

% \input{qm2pi.knotations} 

% section notation (end)

\input{qm2pi.process.calculi} 

% section concurrent_process_calculi_and_spatial_logics_ (end)
    
%\input{qm2pi.knots2pi} 

%\input{qm2pi.trefoil} 

%\input{qm2pi.mainthm} 

% subsection basic_interpretation (end)

%\input{qm2pi.rho.presentation} 
\subsection{The syntax and semantics of the notation system}\label{sub:the_syntax_and_semantics_of_the_notation_system} % (fold)

We now summarize a technical presentation of the calculus that
embodies our theory of dynamics. The typical presentation of such a
calculus follows the style of giving generators and relations on
them. The grammar, below, describing term constructors, freely
generates the set of processes, $\Proc$. This set is then quotiented
by a relation known as structural congruence and it is over this set
that the notion of dynamics is expressed. This presentation is
essentially that of \cite{MeredithR05} with the addition of
polyadicity and summation. For readability we have relegated some of
the technical subtleties to an appendix.

\subsubsection{Process grammar}\label{subsub:process_grammar}

\begin{mathpar}
  \inferrule* [lab=synchronization] {} {{M} \bc \pzero \;|\; x?F \;|\; x!C }
  \and
  \inferrule* [lab=abstraction] {} {{F} \bc (x)P}
  \and
  \inferrule* [lab=concretion] {} {{C} \bc \langle Q \rangle}
  \and
  \inferrule* [lab=process] {} {{P,Q} \bc M \;| \;P|Q \;|\; @{x}}
  \and
  \inferrule* [lab=name] {} {{x} \bc \quotep{P}}
\end{mathpar} 

Note that $\vec{x}$ (resp. $\vec{P}$) denotes a vector of names
(resp. processes) of length $|\vec{x}|$ (resp. $|\vec{P}|$). We adopt
the following useful abbreviations.

\begin{mathpar}
   x?(\vec{y}).P := x.(\vec{y})P \and  x\clift{\vec{P}} := x.\clift{\vec{P}}
   \and x!(y) := \lift{x}{\dropn{y}}
   \and \Pi_{i=0}^{n-1}P_i := P_0 | \ldots | P_{n-1}
\end{mathpar}

\subsubsection{Structural congruence}

\paragraph{Free and bound names and alpha-equivalence.} At the
core of structural equivalence is alpha-equivalence which identifies
process that are the same up to a change of variable. Formally, we
recognize the distinction between free and bound names. The free names
of a process, $\freenames{P}$, may be calculated recursively as
follows:

\begin{mathpar}
\freenames{\pzero} := \emptyset
  \and \\
  \freenames{x?(y).P} := \{ x \} \cup (\freenames{P} \setminus \{ y \})
  \and 
  \freenames{x!\langle P \rangle} := \{ x \} \cup \{ P \} 
  \and \\
  \freenames{P|Q} := \freenames{P} \cup \freenames{Q}
  \and \\
  \freenames{@{x}} := \{ x \}
\end{mathpar}

$\pi$
$\quotep{\pi}$

$\freenames{-} : \pi \to \mathcal{P}(\quotep{\pi})$

\begin{eqnarray*}
  \freenames{\pzero} & := & \emptyset \\
  \freenames{x?(y).P} & := & \{ x \} \cup (\freenames{P} \setminus \{ y \}) \\
  \freenames{x!\langle P \rangle} & := & \{ x \} \cup \{ P \} \\
  \freenames{P|Q} & := & \freenames{P} \cup \freenames{Q} \\
  \freenames{\dropn{x}} & := & \{ x \}
\end{eqnarray*}

The bound names of a process, $\boundnames{P}$, are those names occurring in $P$
that are not free. For example, in $x?(y).0$, the name $x$ is free, while $y$ is bound.

\begin{mathpar}
  \inferrule* [lab=monoidal-laws] {} { P|Q \equiv Q|P \and P|0 \equiv P \and P|(Q|R) \equiv (P|Q)|R }
\end{mathpar}

\begin{mathpar}
  \inferrule* [lab=alpha-equivalence] {} { (x)P \equiv (y)P\{y/x\} \and y \not\in \freenames{P} }
\end{mathpar}

\begin{definition}
Then two processes, $P,Q$, are alpha-equivalent if $P = Q\{\vec{y}/\vec{x}\}$ for
some $\vec{x} \in \boundnames{Q},\vec{y} \in \boundnames{P}$, where $Q\{\vec{y}/\vec{x}\}$
denotes the capture-avoiding substitution of $\vec{y}$ for $\vec{x}$ in $Q$.
\end{definition}

\begin{definition}
  The {\em structural congruence} \cite{SangiorgiWalker} , $\equiv$,
  between processes is the least congruence containing
  alpha-equivalence, satisfying the abelian monoid laws
  (associativity, commutativity and $\pzero$ as identity) for parallel
  composition $|$ and for summation $+$.
\end{definition}

\subsection{Name equivalence}

We take name equivalence, written $\nameeq$, to be the smallest
equivalence relation generated by the following rules.

\begin{mathpar}
\inferrule*[lab=Quote-drop]
{ }
{ \quotep{@{x}} \nameeq x }

\inferrule*[lab=Struct-equiv]
{ P \scong Q }
{ \quotep{P} \nameeq \quotep{Q} }
\end{mathpar}

The astute reader will have noticed that the mutual recursion of names
and processes imposes a mutual recursion on alpha-equivalence and
structural equivalence via name-equivalence. Fortunately, all of this
works out pleasantly and we may calculate in the natural way, free of
concern. The reader interested in the details is referred to the
appendix \ref{appendix:rho_details}.

\subsection{Substitution}

We use $\Proc$ for the set of processes, $\QProc$ for the set of
names, and $\id{\{}\vec{y} / \vec{x} \id{\}}$ to denote partial maps,
$s : \QProc \rightarrow \QProc$. A map, $s$ lifts, uniquely, to a map
on process terms, $\widehat{s} : \Proc \rightarrow \Proc$ by the
following equations.

\begin{mathpar}
  (0) \psubstp{Q}{P} := 0 \\
  (R \juxtap S) \psubstp{Q}{P}
  :=    
  (R)\psubstp{Q}{P} \juxtap (S) \psubstp{Q}{P} \\
  (x?(y).R) \psubstp{Q}{P}    
  :=    
  (x)\substp{Q}{P} (z)\concat( (R \psubstn{z}{y}) \psubstp{Q}{P} ) \\
  (\lift{x}{R}) \psubstp{Q}{P}  
  :=
  \lift{(x)\substp{Q}{P}}{ R \psubstp{Q}{P} } \\
%   (\dropn{x})  \psubstp{Q}{P}       
%   := 
%   \left\{ 
%     \begin{array}{ccc} 
%       \dropn{\quotep{Q}} & & x \nameeq \quotep{P} \\
%       \dropn{x} & & otherwise \\
%     \end{array}
%   \right. 
  (\dropn{x})  \psubstp{Q}{P}       
  := 
  \left\{ 
    \begin{array}{ccc} 
      Q & & x \nameeq \quotep{P} \\
      \dropn{x} & & otherwise \\
    \end{array}
  \right.
\end{mathpar}
 

where

\begin{eqnarray}
  (x)\id{\{} \lpquote Q \rpquote / \lpquote P \rpquote \id{\}}            = 
  \left\{ 
    \begin{array}{ccc}
      \lpquote Q \rpquote & & x \nameeq \lpquote P \rpquote \\
      x & & otherwise \\
    \end{array}
  \right. \nonumber
\end{eqnarray}

and $z$ is chosen distinct from $\quotep{P}$, $\quotep{Q}$, the free
names in $Q$, and all the names in $R$. Our $\alpha$-equivalence will
be built in the standard way from this substitution.

\begin{remark}\label{rem:no_self_referential_names}
  One consequence of these definitions is that $\forall P. \quotep{P}
  \not\in \freenames{P}$.
\end{remark}

\subsection{ Dynamic quote: an example }

Anticipating something of what's to come, consider applying the
substitution, $\widehat{\id{\{}u / z \id{\}}}$, to the following pair
of processes, $\lift{w}{y!(z)}$ and $w[ \lpquote y!(z) \rpquote ]$.

\begin{eqnarray}
	\lift{w}{y!(z)}\widehat{\id{\{}u / z \id{\}}}
		& = &
		\lift{w}{y!(u)} \nonumber\\
	w[ \lpquote y!(z) \rpquote ] \widehat{ \id{\{}u / z \id{\}} }
		& = &
		w[ \lpquote y!(z) \rpquote ] \nonumber
\end{eqnarray}

Because the body of the process between quotes is impervious to
substitution, we get radically different answers. In fact, by
examining the first process in an input context,
e.g. $x?(z).\lift{w}{y!(z)}$, we see that the process under the lift
operator may be shaped by prefixed inputs binding a name inside it. In
this sense, the lift operator will be seen as a way to dynamically
construct processes before reifying them as names.

Finally equipped with these standard features we can present the
dynamics of the calculus.

\subsubsection{Operational semantics} 

Finally, we introduce the computational dynamics. What marks these
algebras as distinct from other more traditionally studied algebraic
structures, e.g. vector spaces or polynomial rings, is the manner in
which dynamics is captured. In traditional structures, dynamics is typically
expressed through morphisms between such structures, as in linear maps
between vector spaces or morphisms between rings. In algebras
associated with the semantics of computation, the dynamics is
expressed as part of the algebraic structure itself, through a
reduction reduction relation typically denoted by $\red$. Below, we
give a recursive presentation of this relation for the calculus used
in the encoding.

$\red \subseteq \pi \times \pi$
$\red : \pi \to \mathcal{P}(\pi)$

\begin{mathpar}
  \inferrule* [lab=Comm] { \textsf{match}( x_{src}, x_{trgt} ) } { x_{trgt}?(y)P \; | \; x_{src}!\langle {Q} \rangle \red P\{\quotep{Q}/y}\} }
  \and \\
  \inferrule* [lab=Par] {{P} \red {P}'} {{{P} | {Q}} \red {{P}' | {Q}}}
  \and
  \inferrule* [lab=Equiv]{{{P} \scong {P}'} \andalso {{P}' \red {Q}'} \andalso {{Q}' \scong {Q}}}{{P} \red {Q}}
\end{mathpar}

\begin{eqnarray*}
  match_{\equiv} (\quotep{P},\quotep{Q}) & := & P \equiv Q \\
  match_{\dagger}(\quotep{P},\quotep{Q}) & := & \forall R. P|Q \red^{*} R => R \red^{*} 0 \\
  match_{K}(\quotep{P},\quotep{Q}) & := & K \mbox{ for some context } K
\end{eqnarray*}

$u?(x)P | u!\langle Q \rangle \red P\{\quotep{Q}/x\}$

%We write $\wred$ for $\red^*$, and $P\red$ if $\exists Q $ such that $ P \red Q$.
We write $P\red$ if $\exists Q $ such that $ P \red Q$ and $P\not\red$, otherwise.

\section{Replication}

As mentioned before, it is known that replication (and hence
recursion) can be implemented in a higher-order process algebra
\cite{SangiorgiWalker}. As our first example of calculation with the
machinery thus far presented we give the construction explicitly in
the {\rhoc}.

\begin{eqnarray}
	D_{x} & := & \prefix{x}{y}{(\binpar{\outputp{x}{y}}{@{y}})} \nonumber\\
	\bangp_{x}{P} & := & \binpar{{x}!\langle{\binpar{D_{x}}{P}}\rangle}{D_{x}} \nonumber
\end{eqnarray}

\begin{eqnarray}
	\bangp_{x}{P} & & \nonumber\\
	=
	& {x}!\langle{(\prefix{x}{y}{(\outputp{x}{y} | @{y})) | P}}\rangle 
	      | \prefix{x}{y}{(\outputp{x}{y} | @{y})} & \nonumber\\
	\red
	& (\outputp{x}{y} | @{y})\substn{\quotep{(\prefix{x}{y}{(@{y} | \outputp{x}{y})) | P}}}{y} & \nonumber\\
	=
	& \outputp{x}{\quotep{(\prefix{x}{y}{(\outputp{x}{y} | @{y})) | P}}}
	  | {(\prefix{x}{y}{(\outputp{x}{y} | @{y})) | P}} & \nonumber\\
	\red
	& \ldots & \nonumber\\
	\red^*
	& P | P | \ldots & \nonumber
\end{eqnarray}

Of course, this encoding, as an implementation, runs away, unfolding
$\bangp{P}$ eagerly. A lazier and more implementable replication
operator, restricted to input-guarded processes, may be obtained as follows.

\begin{eqnarray}
\bangp{\prefix{u}{v}{P}} 
	:= 
	\binpar{\lift{x}{\prefix{u}{v}{(\binpar{D(x)}{P})}}}{D(x)} \nonumber
\end{eqnarray}

\begin{remark}
  Note that the lazier definition still does not deal with summation
  or mixed summation (i.e. sums over input and output). The reader is
  invited to construct definitions of replication that deal with these
  features. 

  Further, the definitions are parameterized in a name, $x$. Can you,
  gentle reader, make a definition that eliminates this parameter and
  guarantees no accidental interaction between the replication
  machinery and the process being replicated -- i.e. no accidental
  sharing of names used by the process to get its work done and the
  name(s) used by the replication to effect copying. This latter
  revision of the definition of replication is crucial to obtaining
  the expected identity $!!P \sim !P$.
\end{remark}

\begin{remark}\label{rem:paradoxical_combinator}
  The reader familiar with the lambda calculus will have noticed the
  similarity between $D$ and the paradoxical combinator.

  [Ed. note: the existence of this seems to suggest we have to be more
  restrictive on the set of processes and names we admit if we are to
  support no-cloning.]
\end{remark}

\subsubsection{Bisimulation}

The computational dynamics gives rise to another kind of equivalence,
the equivalence of computational behavior. As previously mentioned
this is typically captured \emph{via} some form of bisimulation.

% The notion we use in this paper is weak barbed bisimulation
% \cite{milner91polyadicpi}.

The notion we use in this paper is derived from weak barbed
bisimulation \cite{milner91polyadicpi}. 

\begin{definition}
An \emph{observation relation}, $\downarrow_{\mathcal N}$, over a set
of names, $\mathcal N$, is the smallest relation satisfying the rules
below.

\infrule[Out-barb]{y \in {\mathcal N}, \; x \nameeq y}
		  {\outputp{x}{v} \downarrow_{\mathcal N} x}
\infrule[Par-barb]{\mbox{$P\downarrow_{\mathcal N} x$ or $Q\downarrow_{\mathcal N} x$}}
		  {\binpar{P}{Q} \downarrow_{\mathcal N} x}

We write $P \Downarrow_{\mathcal N} x$ if there is $Q$ such that 
$P \wred Q$ and $Q \downarrow_{\mathcal N} x$.
\end{definition}

\begin{definition}
%\label{def.bbisim}
An  ${\mathcal N}$-\emph{barbed bisimulation} over a set of names, ${\mathcal N}$, is a symmetric binary relation 
${\mathcal S}_{\mathcal N}$ between agents such that $P\rel{S}_{\mathcal N}Q$ implies:
\begin{enumerate}
\item If $P \red P'$ then $Q \wred Q'$ and $P'\rel{S}_{\mathcal N} Q'$.
\item If $P\downarrow_{\mathcal N} x$, then $Q\Downarrow_{\mathcal N} x$.
\end{enumerate}
$P$ is ${\mathcal N}$-barbed bisimilar to $Q$, written
$P \wbbisim_{\mathcal N} Q$, if $P \rel{S}_{\mathcal N} Q$ for some ${\mathcal N}$-barbed bisimulation ${\mathcal S}_{\mathcal N}$.
\end{definition}

$\mathcal{R} \subseteq \pi \times \pi$

$P \mathcal{R} Q => \forall P'. P \red P' \Rightarrow \exists Q'. Q \red Q', P' \mathcal{R} Q'$

$P \vdash x \Rightarrow Q \vdash x$

\begin{mathpar}
  \inferrule*[lab=Out-barb]{x \nameeq y}{{y}!\langle{Q}\rangle \vdash x}
  \and
  \inferrule*[lab=Par-barb]{\mbox{$P\vdash x$ or $Q\vdash x$}}{\binpar{P}{Q} \vdash x}
\end{mathpar}

\subsubsection{Contexts}

One of the principle advantages of computational calculi like the
$\pi$-calculus is a well-defined notion of context,
contextual-equivalence and a correlation between
contextual-equivalence and notions of bisimulation. The notion of
context allows the decomposition of a process into (sub-)process and
its syntactic environment, its context. Thus, a context may be
thought of as a process with a ``hole'' (written $\Box$) in it. The
application of a context $M$ to a process $P$, written $M[P]$, is
tantamount to filling the hole in $M$ with $P$. In this paper we do
not need the full weight of this theory, but do make use of the notion
of context in the proof the main theorem. 

\begin{mathpar}
  \inferrule* [lab=summation] {} {{M_{M},M_{N}} \bc \Box \;|\; x.M_{A} \;|\; M_{M}+M_{N}}
  \and
  \inferrule* [lab=agent] {} {{M_{A}} \bc (\vec{x})M_{P} \;| \; \clift{P_0,\ldots,M_{P},\ldots,P_N}}
  \and \\
  \inferrule* [lab=process] {} {{M_{P}} \bc M_{N} \;| \;P|M_{P} }
\end{mathpar} 

\begin{mathpar}
  \inferrule* [lab=sychronization] {} {M_{N} \bc \Box \;|\; x?M_{F} \;|\; x!M_{C}}
  \and
  \inferrule* [lab=abstraction] {} {{M_{F}} \bc (x)M_{P} }
  \and
  \inferrule* [lab=concretion] {} {{M_{C}} \bc \langle M_{P} \rangle }
  \and \\
  \inferrule* [lab=process] {} {{M_{P}} \bc M_{N} \;| \;P|M_{P} }
\end{mathpar}

\begin{definition}[contextual application] Given a context $M$, and
  process $P$, we define the \emph{contextual application}, $M[P] :=
  M\{P/\Box\}$. That is, the contextual application of M to P is the
  substitution of $P$ for $\Box$ in $M$.
\end{definition}

$\meaningof{-} : L \to \mathcal{P}(\pi)$

\begin{mathpar}
  \inferrule* [lab=collection] {} {\meaningof{true} = \pi, \and \meaningof{~E} = \pi \setminus \meaningof{E}, \and \meaningof{E_{1} \& E_{2}} = \meaningof{E_{1}} \cap \meaningof{E_{2}}}
\end{mathpar}

\begin{mathpar}
  \inferrule* [lab=structure] {} {\meaningof{0} = \{ P \in \pi | P \equiv 0 \}, \and \\ \meaningof{E_1 | E_2} = \{ P \in \pi | P \equiv P_{1} | P_{2}, P_{1} \in \meaningof{E_{1}}, P_{2} \in \meaningof{E_2}\} }
\end{mathpar}

\begin{mathpar}
 \inferrule* [lab=behavior] {} {\meaningof{\langle a?b \rangle E} = \{ P \in \pi | P \equiv Q | u?(y)P', \\ \and \\\\ \and \\ \;\;\; u \in \meaningof{a}, \forall z.P'\{z/y\} \in \meaningof{E\{z/b\}}\}, \and \\ \meaningof{a!E} = \{ P \in \pi | P \equiv Q | x!\langle P' \rangle, x \in \meaningof{a} P' \in \meaningof{E}\} }
\end{mathpar}

\begin{mathpar}
 \inferrule* [lab=nominal] {} {\meaningof{\quotep{E}} = \{ \quotep{P} \in \quotep{\pi} | P \in \meaningof{E} \}, \and \meaningof{\quotep{P}} = \{ \quotep{Q} \in \quotep{\pi} | P \equiv Q \} \and \\ \meaningof{@\quotep{E}} = \{ P \in \pi | P \equiv @x, x \in \meaningof{E} \}}
\end{mathpar}

\begin{eqnarray*}
  \\
  \meaningof{-} : TS \to ST
\end{eqnarray*}

\begin{eqnarray*}
  \\
  L : TS \to ST
\end{eqnarray*}

\begin{eqnarray*}
  \\
  P \models E \iff P \in \meaningof{E}
\end{eqnarray*}

\begin{eqnarray*}
  P \approx_{L} Q \iff \forall E \in L. P \models E \iff Q \models E
\end{eqnarray*}

\begin{eqnarray*}
  P \approx_{K} Q
\end{eqnarray*}

\begin{eqnarray*}
  P \approx Q
\end{eqnarray*}

$\approx_{K} = \approx = \approx_{L}$

\subsubsection{Contextual duality}

Note that contexts extend the quotation operation to a family of
operations from processes to names. Given a context, $M$, we can
define a \emph{nominal context}, $\quotep{M}$ by $\quotep{M}[P] :=
\quotep{M[P]}$. To foreshadow what is to come we observe that these
operations enjoy a duality with processes very much like the duality
between vectors and maps from vectors to scalars.

Further, because the calculus is essentially higher-order, we have a
correspondence between contexts and processes. More specifically,
given a name $x$ and a context $M$ we can construct $M^{*}_{x}$ such
that 

\begin{mathpar}
  M^{*}_{x} | \lift{x}{P} \red M[P]
\end{mathpar}

namely,

\begin{mathpar}
  M^{*}_{x} := x?(u).M[\dropn{u}]
\end{mathpar}

The dependence of $M^{*}_{x}$ on a name makes it an abstraction, 

\begin{mathpar}
  M^{*} := (x)x?(u).M[\dropn{u}]
\end{mathpar}

\subsection{Additional notation}

It will sometimes be convenient to denote the process a name
quotes. We already have the notation $x = \quotep{P}$, but it will be
convenient to introduce an alternate notation, $\procn{x}$, when we
want to emphasize the connection to the use of the name. Note that, by
virtue of name equivalence, $\quotep{\procn{x}} \nameeq x$; so, the
notation is consistent with previous definitions.

Further, because names have structure it is possible to effect
substitutions on the basis of that structure. This means we need to
upgrade our notation for substitutions, which we accomplish by
adapting comprehension notation. Thus,

\begin{mathpar}
  P\{ y / x : x \in S \}
\end{mathpar}

is interpreted to mean the process derived from P by replacing (in a
capture-avoiding manner) each occurrence of $x$ in $S$ by $y$. For example,

\begin{mathpar}
  P\{ \quotep{\procn{x}|\procn{x}} / x : x \in \freenames{P} \}
\end{mathpar}

will replace each (occurrence) of a free name $x$ in $P$ by
$\quotep{\procn{x}|\procn{x}}$.

Also, we will avail ourselves of the notation $x^{L}$ and $x^{R}$ to
denote injections of a name into disjoint copies of the name
space. There are numerous ways to accomplish this. One example can be
found in \cite{MeredithR05}. This notation overloads to vectors of
names: $\vec{x}^{\pi} := (x_{i}^{\pi} \; : \; 0 \leq i < |\vec{x}| )$ where $\pi \in \{L,R\}$.

We also use $P^{\Box} := P|\Box$.

In \cite{MeredithR05} an interpretation of the new operator is
given. It turns out that there are several possible interpretations
all enjoying the requisite algebraic properties of the operator (see
\cite{milner91polyadicpi}). We will therefore make liberal use of
$(\nu\; \vec{x})P$.

% subsection the_syntax_and_semantics_of_the_notation_system (end)   

\input{qm2pi.qmops} 

\input{qm2pi.sterngerlach} 

\input{qm2pi.metric} 

% section concurrent_process_calculi (end)

%\input{qm2pi.proofsketch}

% section proof sketch (end)

%\input{qm2pi.slviaknots} 

% section spatial logic via knots (end)

\input{qm2pi.conclusion}

% section conclusion (end)

%\input{qm2pi.dtcodes} 

% section wiring algorithm (end)

\input{qm2pi.ack} 

% section acknowledgments (end)

\newpage


\bibliographystyle{plain}   
\bibliography{../../biblios/main.bib}

\input{qm2pi.rhodetails}

\end{document}

 

%\documentclass[12pt]{llncs}
%\documentclass{jktr}

\usepackage[pdftex]{hyperref}                   
\usepackage {listings}
\usepackage {mathpartir}
\usepackage{bcprules}
%\usepackage{listings}
                       
\usepackage{graphicx} 
%\usepackage[margins=2.5cm,nohead,nofoot]{geometry}
%\usepackage{geometry}
\usepackage{amsfonts}
\usepackage{amstext}
\usepackage{latexsym}
\usepackage{amssymb}
\usepackage{color}


%\include{myPreamble}
\include{qm2pi.local} 

%\ifpdf
%\usepackage[pdftex]{graphicx}
%\else
%\usepackage{graphicx}
%\fi

 % \ifpdf
%  \usepackage{pdfsync}
%  \if


%\title{Brief Article}
%\author{David F. Snyder}
%\author{L.G. Meredith}

%\address{Dept. of Math., Texas State University--San Marcos, San Marcos, TX 78666}
       
\pagestyle{empty}


\begin{document}

\lstset{language=[Objective]Caml,frame=shadowbox}

\input{qm2pi.front}

% section front matter (end)

\input{qm2pi.intro} 
 
% section introduction (end)

% \input{qm2pi.knotations} 

% section notation (end)

\input{qm2pi.process.calculi} 

% section concurrent_process_calculi_and_spatial_logics_ (end)
    
%\input{qm2pi.knots2pi} 

%\input{qm2pi.trefoil} 

%\input{qm2pi.mainthm} 

% subsection basic_interpretation (end)

%\input{qm2pi.rho.presentation} 
\subsection{The syntax and semantics of the notation system}\label{sub:the_syntax_and_semantics_of_the_notation_system} % (fold)

We now summarize a technical presentation of the calculus that
embodies our theory of dynamics. The typical presentation of such a
calculus follows the style of giving generators and relations on
them. The grammar, below, describing term constructors, freely
generates the set of processes, $\Proc$. This set is then quotiented
by a relation known as structural congruence and it is over this set
that the notion of dynamics is expressed. This presentation is
essentially that of \cite{MeredithR05} with the addition of
polyadicity and summation. For readability we have relegated some of
the technical subtleties to an appendix.

\subsubsection{Process grammar}\label{subsub:process_grammar}

\begin{mathpar}
  \inferrule* [lab=synchronization] {} {{M} \bc \pzero \;|\; x?F \;|\; x!C }
  \and
  \inferrule* [lab=abstraction] {} {{F} \bc (x)P}
  \and
  \inferrule* [lab=concretion] {} {{C} \bc \langle Q \rangle}
  \and
  \inferrule* [lab=process] {} {{P,Q} \bc M \;| \;P|Q \;|\; @{x}}
  \and
  \inferrule* [lab=name] {} {{x} \bc \quotep{P}}
\end{mathpar} 

Note that $\vec{x}$ (resp. $\vec{P}$) denotes a vector of names
(resp. processes) of length $|\vec{x}|$ (resp. $|\vec{P}|$). We adopt
the following useful abbreviations.

\begin{mathpar}
   x?(\vec{y}).P := x.(\vec{y})P \and  x\clift{\vec{P}} := x.\clift{\vec{P}}
   \and x!(y) := \lift{x}{\dropn{y}}
   \and \Pi_{i=0}^{n-1}P_i := P_0 | \ldots | P_{n-1}
\end{mathpar}

\subsubsection{Structural congruence}

\paragraph{Free and bound names and alpha-equivalence.} At the
core of structural equivalence is alpha-equivalence which identifies
process that are the same up to a change of variable. Formally, we
recognize the distinction between free and bound names. The free names
of a process, $\freenames{P}$, may be calculated recursively as
follows:

\begin{mathpar}
\freenames{\pzero} := \emptyset
  \and \\
  \freenames{x?(y).P} := \{ x \} \cup (\freenames{P} \setminus \{ y \})
  \and 
  \freenames{x!\langle P \rangle} := \{ x \} \cup \{ P \} 
  \and \\
  \freenames{P|Q} := \freenames{P} \cup \freenames{Q}
  \and \\
  \freenames{@{x}} := \{ x \}
\end{mathpar}

$\pi$
$\quotep{\pi}$

$\freenames{-} : \pi \to \mathcal{P}(\quotep{\pi})$

\begin{eqnarray*}
  \freenames{\pzero} & := & \emptyset \\
  \freenames{x?(y).P} & := & \{ x \} \cup (\freenames{P} \setminus \{ y \}) \\
  \freenames{x!\langle P \rangle} & := & \{ x \} \cup \{ P \} \\
  \freenames{P|Q} & := & \freenames{P} \cup \freenames{Q} \\
  \freenames{\dropn{x}} & := & \{ x \}
\end{eqnarray*}

The bound names of a process, $\boundnames{P}$, are those names occurring in $P$
that are not free. For example, in $x?(y).0$, the name $x$ is free, while $y$ is bound.

\begin{mathpar}
  \inferrule* [lab=monoidal-laws] {} { P|Q \equiv Q|P \and P|0 \equiv P \and P|(Q|R) \equiv (P|Q)|R }
\end{mathpar}

\begin{mathpar}
  \inferrule* [lab=alpha-equivalence] {} { (x)P \equiv (y)P\{y/x\} \and y \not\in \freenames{P} }
\end{mathpar}

\begin{definition}
Then two processes, $P,Q$, are alpha-equivalent if $P = Q\{\vec{y}/\vec{x}\}$ for
some $\vec{x} \in \boundnames{Q},\vec{y} \in \boundnames{P}$, where $Q\{\vec{y}/\vec{x}\}$
denotes the capture-avoiding substitution of $\vec{y}$ for $\vec{x}$ in $Q$.
\end{definition}

\begin{definition}
  The {\em structural congruence} \cite{SangiorgiWalker} , $\equiv$,
  between processes is the least congruence containing
  alpha-equivalence, satisfying the abelian monoid laws
  (associativity, commutativity and $\pzero$ as identity) for parallel
  composition $|$ and for summation $+$.
\end{definition}

\subsection{Name equivalence}

We take name equivalence, written $\nameeq$, to be the smallest
equivalence relation generated by the following rules.

\begin{mathpar}
\inferrule*[lab=Quote-drop]
{ }
{ \quotep{@{x}} \nameeq x }

\inferrule*[lab=Struct-equiv]
{ P \scong Q }
{ \quotep{P} \nameeq \quotep{Q} }
\end{mathpar}

The astute reader will have noticed that the mutual recursion of names
and processes imposes a mutual recursion on alpha-equivalence and
structural equivalence via name-equivalence. Fortunately, all of this
works out pleasantly and we may calculate in the natural way, free of
concern. The reader interested in the details is referred to the
appendix \ref{appendix:rho_details}.

\subsection{Substitution}

We use $\Proc$ for the set of processes, $\QProc$ for the set of
names, and $\id{\{}\vec{y} / \vec{x} \id{\}}$ to denote partial maps,
$s : \QProc \rightarrow \QProc$. A map, $s$ lifts, uniquely, to a map
on process terms, $\widehat{s} : \Proc \rightarrow \Proc$ by the
following equations.

\begin{mathpar}
  (0) \psubstp{Q}{P} := 0 \\
  (R \juxtap S) \psubstp{Q}{P}
  :=    
  (R)\psubstp{Q}{P} \juxtap (S) \psubstp{Q}{P} \\
  (x?(y).R) \psubstp{Q}{P}    
  :=    
  (x)\substp{Q}{P} (z)\concat( (R \psubstn{z}{y}) \psubstp{Q}{P} ) \\
  (\lift{x}{R}) \psubstp{Q}{P}  
  :=
  \lift{(x)\substp{Q}{P}}{ R \psubstp{Q}{P} } \\
%   (\dropn{x})  \psubstp{Q}{P}       
%   := 
%   \left\{ 
%     \begin{array}{ccc} 
%       \dropn{\quotep{Q}} & & x \nameeq \quotep{P} \\
%       \dropn{x} & & otherwise \\
%     \end{array}
%   \right. 
  (\dropn{x})  \psubstp{Q}{P}       
  := 
  \left\{ 
    \begin{array}{ccc} 
      Q & & x \nameeq \quotep{P} \\
      \dropn{x} & & otherwise \\
    \end{array}
  \right.
\end{mathpar}
 

where

\begin{eqnarray}
  (x)\id{\{} \lpquote Q \rpquote / \lpquote P \rpquote \id{\}}            = 
  \left\{ 
    \begin{array}{ccc}
      \lpquote Q \rpquote & & x \nameeq \lpquote P \rpquote \\
      x & & otherwise \\
    \end{array}
  \right. \nonumber
\end{eqnarray}

and $z$ is chosen distinct from $\quotep{P}$, $\quotep{Q}$, the free
names in $Q$, and all the names in $R$. Our $\alpha$-equivalence will
be built in the standard way from this substitution.

\begin{remark}\label{rem:no_self_referential_names}
  One consequence of these definitions is that $\forall P. \quotep{P}
  \not\in \freenames{P}$.
\end{remark}

\subsection{ Dynamic quote: an example }

Anticipating something of what's to come, consider applying the
substitution, $\widehat{\id{\{}u / z \id{\}}}$, to the following pair
of processes, $\lift{w}{y!(z)}$ and $w[ \lpquote y!(z) \rpquote ]$.

\begin{eqnarray}
	\lift{w}{y!(z)}\widehat{\id{\{}u / z \id{\}}}
		& = &
		\lift{w}{y!(u)} \nonumber\\
	w[ \lpquote y!(z) \rpquote ] \widehat{ \id{\{}u / z \id{\}} }
		& = &
		w[ \lpquote y!(z) \rpquote ] \nonumber
\end{eqnarray}

Because the body of the process between quotes is impervious to
substitution, we get radically different answers. In fact, by
examining the first process in an input context,
e.g. $x?(z).\lift{w}{y!(z)}$, we see that the process under the lift
operator may be shaped by prefixed inputs binding a name inside it. In
this sense, the lift operator will be seen as a way to dynamically
construct processes before reifying them as names.

Finally equipped with these standard features we can present the
dynamics of the calculus.

\subsubsection{Operational semantics} 

Finally, we introduce the computational dynamics. What marks these
algebras as distinct from other more traditionally studied algebraic
structures, e.g. vector spaces or polynomial rings, is the manner in
which dynamics is captured. In traditional structures, dynamics is typically
expressed through morphisms between such structures, as in linear maps
between vector spaces or morphisms between rings. In algebras
associated with the semantics of computation, the dynamics is
expressed as part of the algebraic structure itself, through a
reduction reduction relation typically denoted by $\red$. Below, we
give a recursive presentation of this relation for the calculus used
in the encoding.

$\red \subseteq \pi \times \pi$
$\red : \pi \to \mathcal{P}(\pi)$

\begin{mathpar}
  \inferrule* [lab=Comm] { \textsf{match}( x_{src}, x_{trgt} ) } { x_{trgt}?(y)P \; | \; x_{src}!\langle {Q} \rangle \red P\{\quotep{Q}/y}\} }
  \and \\
  \inferrule* [lab=Par] {{P} \red {P}'} {{{P} | {Q}} \red {{P}' | {Q}}}
  \and
  \inferrule* [lab=Equiv]{{{P} \scong {P}'} \andalso {{P}' \red {Q}'} \andalso {{Q}' \scong {Q}}}{{P} \red {Q}}
\end{mathpar}

\begin{eqnarray*}
  match_{\equiv} (\quotep{P},\quotep{Q}) & := & P \equiv Q \\
  match_{\dagger}(\quotep{P},\quotep{Q}) & := & \forall R. P|Q \red^{*} R => R \red^{*} 0 \\
  match_{K}(\quotep{P},\quotep{Q}) & := & K \mbox{ for some context } K
\end{eqnarray*}

$u?(x)P | u!\langle Q \rangle \red P\{\quotep{Q}/x\}$

%We write $\wred$ for $\red^*$, and $P\red$ if $\exists Q $ such that $ P \red Q$.
We write $P\red$ if $\exists Q $ such that $ P \red Q$ and $P\not\red$, otherwise.

\section{Replication}

As mentioned before, it is known that replication (and hence
recursion) can be implemented in a higher-order process algebra
\cite{SangiorgiWalker}. As our first example of calculation with the
machinery thus far presented we give the construction explicitly in
the {\rhoc}.

\begin{eqnarray}
	D_{x} & := & \prefix{x}{y}{(\binpar{\outputp{x}{y}}{@{y}})} \nonumber\\
	\bangp_{x}{P} & := & \binpar{{x}!\langle{\binpar{D_{x}}{P}}\rangle}{D_{x}} \nonumber
\end{eqnarray}

\begin{eqnarray}
	\bangp_{x}{P} & & \nonumber\\
	=
	& {x}!\langle{(\prefix{x}{y}{(\outputp{x}{y} | @{y})) | P}}\rangle 
	      | \prefix{x}{y}{(\outputp{x}{y} | @{y})} & \nonumber\\
	\red
	& (\outputp{x}{y} | @{y})\substn{\quotep{(\prefix{x}{y}{(@{y} | \outputp{x}{y})) | P}}}{y} & \nonumber\\
	=
	& \outputp{x}{\quotep{(\prefix{x}{y}{(\outputp{x}{y} | @{y})) | P}}}
	  | {(\prefix{x}{y}{(\outputp{x}{y} | @{y})) | P}} & \nonumber\\
	\red
	& \ldots & \nonumber\\
	\red^*
	& P | P | \ldots & \nonumber
\end{eqnarray}

Of course, this encoding, as an implementation, runs away, unfolding
$\bangp{P}$ eagerly. A lazier and more implementable replication
operator, restricted to input-guarded processes, may be obtained as follows.

\begin{eqnarray}
\bangp{\prefix{u}{v}{P}} 
	:= 
	\binpar{\lift{x}{\prefix{u}{v}{(\binpar{D(x)}{P})}}}{D(x)} \nonumber
\end{eqnarray}

\begin{remark}
  Note that the lazier definition still does not deal with summation
  or mixed summation (i.e. sums over input and output). The reader is
  invited to construct definitions of replication that deal with these
  features. 

  Further, the definitions are parameterized in a name, $x$. Can you,
  gentle reader, make a definition that eliminates this parameter and
  guarantees no accidental interaction between the replication
  machinery and the process being replicated -- i.e. no accidental
  sharing of names used by the process to get its work done and the
  name(s) used by the replication to effect copying. This latter
  revision of the definition of replication is crucial to obtaining
  the expected identity $!!P \sim !P$.
\end{remark}

\begin{remark}\label{rem:paradoxical_combinator}
  The reader familiar with the lambda calculus will have noticed the
  similarity between $D$ and the paradoxical combinator.

  [Ed. note: the existence of this seems to suggest we have to be more
  restrictive on the set of processes and names we admit if we are to
  support no-cloning.]
\end{remark}

\subsubsection{Bisimulation}

The computational dynamics gives rise to another kind of equivalence,
the equivalence of computational behavior. As previously mentioned
this is typically captured \emph{via} some form of bisimulation.

% The notion we use in this paper is weak barbed bisimulation
% \cite{milner91polyadicpi}.

The notion we use in this paper is derived from weak barbed
bisimulation \cite{milner91polyadicpi}. 

\begin{definition}
An \emph{observation relation}, $\downarrow_{\mathcal N}$, over a set
of names, $\mathcal N$, is the smallest relation satisfying the rules
below.

\infrule[Out-barb]{y \in {\mathcal N}, \; x \nameeq y}
		  {\outputp{x}{v} \downarrow_{\mathcal N} x}
\infrule[Par-barb]{\mbox{$P\downarrow_{\mathcal N} x$ or $Q\downarrow_{\mathcal N} x$}}
		  {\binpar{P}{Q} \downarrow_{\mathcal N} x}

We write $P \Downarrow_{\mathcal N} x$ if there is $Q$ such that 
$P \wred Q$ and $Q \downarrow_{\mathcal N} x$.
\end{definition}

\begin{definition}
%\label{def.bbisim}
An  ${\mathcal N}$-\emph{barbed bisimulation} over a set of names, ${\mathcal N}$, is a symmetric binary relation 
${\mathcal S}_{\mathcal N}$ between agents such that $P\rel{S}_{\mathcal N}Q$ implies:
\begin{enumerate}
\item If $P \red P'$ then $Q \wred Q'$ and $P'\rel{S}_{\mathcal N} Q'$.
\item If $P\downarrow_{\mathcal N} x$, then $Q\Downarrow_{\mathcal N} x$.
\end{enumerate}
$P$ is ${\mathcal N}$-barbed bisimilar to $Q$, written
$P \wbbisim_{\mathcal N} Q$, if $P \rel{S}_{\mathcal N} Q$ for some ${\mathcal N}$-barbed bisimulation ${\mathcal S}_{\mathcal N}$.
\end{definition}

$\mathcal{R} \subseteq \pi \times \pi$

$P \mathcal{R} Q => \forall P'. P \red P' \Rightarrow \exists Q'. Q \red Q', P' \mathcal{R} Q'$

$P \vdash x \Rightarrow Q \vdash x$

\begin{mathpar}
  \inferrule*[lab=Out-barb]{x \nameeq y}{{y}!\langle{Q}\rangle \vdash x}
  \and
  \inferrule*[lab=Par-barb]{\mbox{$P\vdash x$ or $Q\vdash x$}}{\binpar{P}{Q} \vdash x}
\end{mathpar}

\subsubsection{Contexts}

One of the principle advantages of computational calculi like the
$\pi$-calculus is a well-defined notion of context,
contextual-equivalence and a correlation between
contextual-equivalence and notions of bisimulation. The notion of
context allows the decomposition of a process into (sub-)process and
its syntactic environment, its context. Thus, a context may be
thought of as a process with a ``hole'' (written $\Box$) in it. The
application of a context $M$ to a process $P$, written $M[P]$, is
tantamount to filling the hole in $M$ with $P$. In this paper we do
not need the full weight of this theory, but do make use of the notion
of context in the proof the main theorem. 

\begin{mathpar}
  \inferrule* [lab=summation] {} {{M_{M},M_{N}} \bc \Box \;|\; x.M_{A} \;|\; M_{M}+M_{N}}
  \and
  \inferrule* [lab=agent] {} {{M_{A}} \bc (\vec{x})M_{P} \;| \; \clift{P_0,\ldots,M_{P},\ldots,P_N}}
  \and \\
  \inferrule* [lab=process] {} {{M_{P}} \bc M_{N} \;| \;P|M_{P} }
\end{mathpar} 

\begin{mathpar}
  \inferrule* [lab=sychronization] {} {M_{N} \bc \Box \;|\; x?M_{F} \;|\; x!M_{C}}
  \and
  \inferrule* [lab=abstraction] {} {{M_{F}} \bc (x)M_{P} }
  \and
  \inferrule* [lab=concretion] {} {{M_{C}} \bc \langle M_{P} \rangle }
  \and \\
  \inferrule* [lab=process] {} {{M_{P}} \bc M_{N} \;| \;P|M_{P} }
\end{mathpar}

\begin{definition}[contextual application] Given a context $M$, and
  process $P$, we define the \emph{contextual application}, $M[P] :=
  M\{P/\Box\}$. That is, the contextual application of M to P is the
  substitution of $P$ for $\Box$ in $M$.
\end{definition}

$\meaningof{-} : L \to \mathcal{P}(\pi)$

\begin{mathpar}
  \inferrule* [lab=collection] {} {\meaningof{true} = \pi, \and \meaningof{~E} = \pi \setminus \meaningof{E}, \and \meaningof{E_{1} \& E_{2}} = \meaningof{E_{1}} \cap \meaningof{E_{2}}}
\end{mathpar}

\begin{mathpar}
  \inferrule* [lab=structure] {} {\meaningof{0} = \{ P \in \pi | P \equiv 0 \}, \and \\ \meaningof{E_1 | E_2} = \{ P \in \pi | P \equiv P_{1} | P_{2}, P_{1} \in \meaningof{E_{1}}, P_{2} \in \meaningof{E_2}\} }
\end{mathpar}

\begin{mathpar}
 \inferrule* [lab=behavior] {} {\meaningof{\langle a?b \rangle E} = \{ P \in \pi | P \equiv Q | u?(y)P', \\ \and \\\\ \and \\ \;\;\; u \in \meaningof{a}, \forall z.P'\{z/y\} \in \meaningof{E\{z/b\}}\}, \and \\ \meaningof{a!E} = \{ P \in \pi | P \equiv Q | x!\langle P' \rangle, x \in \meaningof{a} P' \in \meaningof{E}\} }
\end{mathpar}

\begin{mathpar}
 \inferrule* [lab=nominal] {} {\meaningof{\quotep{E}} = \{ \quotep{P} \in \quotep{\pi} | P \in \meaningof{E} \}, \and \meaningof{\quotep{P}} = \{ \quotep{Q} \in \quotep{\pi} | P \equiv Q \} \and \\ \meaningof{@\quotep{E}} = \{ P \in \pi | P \equiv @x, x \in \meaningof{E} \}}
\end{mathpar}

\begin{eqnarray*}
  \\
  \meaningof{-} : TS \to ST
\end{eqnarray*}

\begin{eqnarray*}
  \\
  L : TS \to ST
\end{eqnarray*}

\begin{eqnarray*}
  \\
  P \models E \iff P \in \meaningof{E}
\end{eqnarray*}

\begin{eqnarray*}
  P \approx_{L} Q \iff \forall E \in L. P \models E \iff Q \models E
\end{eqnarray*}

\begin{eqnarray*}
  P \approx_{K} Q
\end{eqnarray*}

\begin{eqnarray*}
  P \approx Q
\end{eqnarray*}

$\approx_{K} = \approx = \approx_{L}$

\subsubsection{Contextual duality}

Note that contexts extend the quotation operation to a family of
operations from processes to names. Given a context, $M$, we can
define a \emph{nominal context}, $\quotep{M}$ by $\quotep{M}[P] :=
\quotep{M[P]}$. To foreshadow what is to come we observe that these
operations enjoy a duality with processes very much like the duality
between vectors and maps from vectors to scalars.

Further, because the calculus is essentially higher-order, we have a
correspondence between contexts and processes. More specifically,
given a name $x$ and a context $M$ we can construct $M^{*}_{x}$ such
that 

\begin{mathpar}
  M^{*}_{x} | \lift{x}{P} \red M[P]
\end{mathpar}

namely,

\begin{mathpar}
  M^{*}_{x} := x?(u).M[\dropn{u}]
\end{mathpar}

The dependence of $M^{*}_{x}$ on a name makes it an abstraction, 

\begin{mathpar}
  M^{*} := (x)x?(u).M[\dropn{u}]
\end{mathpar}

\subsection{Additional notation}

It will sometimes be convenient to denote the process a name
quotes. We already have the notation $x = \quotep{P}$, but it will be
convenient to introduce an alternate notation, $\procn{x}$, when we
want to emphasize the connection to the use of the name. Note that, by
virtue of name equivalence, $\quotep{\procn{x}} \nameeq x$; so, the
notation is consistent with previous definitions.

Further, because names have structure it is possible to effect
substitutions on the basis of that structure. This means we need to
upgrade our notation for substitutions, which we accomplish by
adapting comprehension notation. Thus,

\begin{mathpar}
  P\{ y / x : x \in S \}
\end{mathpar}

is interpreted to mean the process derived from P by replacing (in a
capture-avoiding manner) each occurrence of $x$ in $S$ by $y$. For example,

\begin{mathpar}
  P\{ \quotep{\procn{x}|\procn{x}} / x : x \in \freenames{P} \}
\end{mathpar}

will replace each (occurrence) of a free name $x$ in $P$ by
$\quotep{\procn{x}|\procn{x}}$.

Also, we will avail ourselves of the notation $x^{L}$ and $x^{R}$ to
denote injections of a name into disjoint copies of the name
space. There are numerous ways to accomplish this. One example can be
found in \cite{MeredithR05}. This notation overloads to vectors of
names: $\vec{x}^{\pi} := (x_{i}^{\pi} \; : \; 0 \leq i < |\vec{x}| )$ where $\pi \in \{L,R\}$.

We also use $P^{\Box} := P|\Box$.

In \cite{MeredithR05} an interpretation of the new operator is
given. It turns out that there are several possible interpretations
all enjoying the requisite algebraic properties of the operator (see
\cite{milner91polyadicpi}). We will therefore make liberal use of
$(\nu\; \vec{x})P$.

% subsection the_syntax_and_semantics_of_the_notation_system (end)   

\input{qm2pi.qmops} 

\input{qm2pi.sterngerlach} 

\input{qm2pi.metric} 

% section concurrent_process_calculi (end)

%\input{qm2pi.proofsketch}

% section proof sketch (end)

%\input{qm2pi.slviaknots} 

% section spatial logic via knots (end)

\input{qm2pi.conclusion}

% section conclusion (end)

%\input{qm2pi.dtcodes} 

% section wiring algorithm (end)

\input{qm2pi.ack} 

% section acknowledgments (end)

\newpage


\bibliographystyle{plain}   
\bibliography{../../biblios/main.bib}

\input{qm2pi.rhodetails}

\end{document}

 

% subsection basic_interpretation (end)

%\input{qm2pi.rho.presentation} 
\subsection{The syntax and semantics of the notation system}\label{sub:the_syntax_and_semantics_of_the_notation_system} % (fold)

We now summarize a technical presentation of the calculus that
embodies our theory of dynamics. The typical presentation of such a
calculus follows the style of giving generators and relations on
them. The grammar, below, describing term constructors, freely
generates the set of processes, $\Proc$. This set is then quotiented
by a relation known as structural congruence and it is over this set
that the notion of dynamics is expressed. This presentation is
essentially that of \cite{MeredithR05} with the addition of
polyadicity and summation. For readability we have relegated some of
the technical subtleties to an appendix.

\subsubsection{Process grammar}\label{subsub:process_grammar}

\begin{mathpar}
  \inferrule* [lab=synchronization] {} {{M} \bc \pzero \;|\; x?F \;|\; x!C }
  \and
  \inferrule* [lab=abstraction] {} {{F} \bc (x)P}
  \and
  \inferrule* [lab=concretion] {} {{C} \bc \langle Q \rangle}
  \and
  \inferrule* [lab=process] {} {{P,Q} \bc M \;| \;P|Q \;|\; @{x}}
  \and
  \inferrule* [lab=name] {} {{x} \bc \quotep{P}}
\end{mathpar} 

Note that $\vec{x}$ (resp. $\vec{P}$) denotes a vector of names
(resp. processes) of length $|\vec{x}|$ (resp. $|\vec{P}|$). We adopt
the following useful abbreviations.

\begin{mathpar}
   x?(\vec{y}).P := x.(\vec{y})P \and  x\clift{\vec{P}} := x.\clift{\vec{P}}
   \and x!(y) := \lift{x}{\dropn{y}}
   \and \Pi_{i=0}^{n-1}P_i := P_0 | \ldots | P_{n-1}
\end{mathpar}

\subsubsection{Structural congruence}

\paragraph{Free and bound names and alpha-equivalence.} At the
core of structural equivalence is alpha-equivalence which identifies
process that are the same up to a change of variable. Formally, we
recognize the distinction between free and bound names. The free names
of a process, $\freenames{P}$, may be calculated recursively as
follows:

\begin{mathpar}
\freenames{\pzero} := \emptyset
  \and \\
  \freenames{x?(y).P} := \{ x \} \cup (\freenames{P} \setminus \{ y \})
  \and 
  \freenames{x!\langle P \rangle} := \{ x \} \cup \{ P \} 
  \and \\
  \freenames{P|Q} := \freenames{P} \cup \freenames{Q}
  \and \\
  \freenames{@{x}} := \{ x \}
\end{mathpar}

$\pi$
$\quotep{\pi}$

$\freenames{-} : \pi \to \mathcal{P}(\quotep{\pi})$

\begin{eqnarray*}
  \freenames{\pzero} & := & \emptyset \\
  \freenames{x?(y).P} & := & \{ x \} \cup (\freenames{P} \setminus \{ y \}) \\
  \freenames{x!\langle P \rangle} & := & \{ x \} \cup \{ P \} \\
  \freenames{P|Q} & := & \freenames{P} \cup \freenames{Q} \\
  \freenames{\dropn{x}} & := & \{ x \}
\end{eqnarray*}

The bound names of a process, $\boundnames{P}$, are those names occurring in $P$
that are not free. For example, in $x?(y).0$, the name $x$ is free, while $y$ is bound.

\begin{mathpar}
  \inferrule* [lab=monoidal-laws] {} { P|Q \equiv Q|P \and P|0 \equiv P \and P|(Q|R) \equiv (P|Q)|R }
\end{mathpar}

\begin{mathpar}
  \inferrule* [lab=alpha-equivalence] {} { (x)P \equiv (y)P\{y/x\} \and y \not\in \freenames{P} }
\end{mathpar}

\begin{definition}
Then two processes, $P,Q$, are alpha-equivalent if $P = Q\{\vec{y}/\vec{x}\}$ for
some $\vec{x} \in \boundnames{Q},\vec{y} \in \boundnames{P}$, where $Q\{\vec{y}/\vec{x}\}$
denotes the capture-avoiding substitution of $\vec{y}$ for $\vec{x}$ in $Q$.
\end{definition}

\begin{definition}
  The {\em structural congruence} \cite{SangiorgiWalker} , $\equiv$,
  between processes is the least congruence containing
  alpha-equivalence, satisfying the abelian monoid laws
  (associativity, commutativity and $\pzero$ as identity) for parallel
  composition $|$ and for summation $+$.
\end{definition}

\subsection{Name equivalence}

We take name equivalence, written $\nameeq$, to be the smallest
equivalence relation generated by the following rules.

\begin{mathpar}
\inferrule*[lab=Quote-drop]
{ }
{ \quotep{@{x}} \nameeq x }

\inferrule*[lab=Struct-equiv]
{ P \scong Q }
{ \quotep{P} \nameeq \quotep{Q} }
\end{mathpar}

The astute reader will have noticed that the mutual recursion of names
and processes imposes a mutual recursion on alpha-equivalence and
structural equivalence via name-equivalence. Fortunately, all of this
works out pleasantly and we may calculate in the natural way, free of
concern. The reader interested in the details is referred to the
appendix \ref{appendix:rho_details}.

\subsection{Substitution}

We use $\Proc$ for the set of processes, $\QProc$ for the set of
names, and $\id{\{}\vec{y} / \vec{x} \id{\}}$ to denote partial maps,
$s : \QProc \rightarrow \QProc$. A map, $s$ lifts, uniquely, to a map
on process terms, $\widehat{s} : \Proc \rightarrow \Proc$ by the
following equations.

\begin{mathpar}
  (0) \psubstp{Q}{P} := 0 \\
  (R \juxtap S) \psubstp{Q}{P}
  :=    
  (R)\psubstp{Q}{P} \juxtap (S) \psubstp{Q}{P} \\
  (x?(y).R) \psubstp{Q}{P}    
  :=    
  (x)\substp{Q}{P} (z)\concat( (R \psubstn{z}{y}) \psubstp{Q}{P} ) \\
  (\lift{x}{R}) \psubstp{Q}{P}  
  :=
  \lift{(x)\substp{Q}{P}}{ R \psubstp{Q}{P} } \\
%   (\dropn{x})  \psubstp{Q}{P}       
%   := 
%   \left\{ 
%     \begin{array}{ccc} 
%       \dropn{\quotep{Q}} & & x \nameeq \quotep{P} \\
%       \dropn{x} & & otherwise \\
%     \end{array}
%   \right. 
  (\dropn{x})  \psubstp{Q}{P}       
  := 
  \left\{ 
    \begin{array}{ccc} 
      Q & & x \nameeq \quotep{P} \\
      \dropn{x} & & otherwise \\
    \end{array}
  \right.
\end{mathpar}
 

where

\begin{eqnarray}
  (x)\id{\{} \lpquote Q \rpquote / \lpquote P \rpquote \id{\}}            = 
  \left\{ 
    \begin{array}{ccc}
      \lpquote Q \rpquote & & x \nameeq \lpquote P \rpquote \\
      x & & otherwise \\
    \end{array}
  \right. \nonumber
\end{eqnarray}

and $z$ is chosen distinct from $\quotep{P}$, $\quotep{Q}$, the free
names in $Q$, and all the names in $R$. Our $\alpha$-equivalence will
be built in the standard way from this substitution.

\begin{remark}\label{rem:no_self_referential_names}
  One consequence of these definitions is that $\forall P. \quotep{P}
  \not\in \freenames{P}$.
\end{remark}

\subsection{ Dynamic quote: an example }

Anticipating something of what's to come, consider applying the
substitution, $\widehat{\id{\{}u / z \id{\}}}$, to the following pair
of processes, $\lift{w}{y!(z)}$ and $w[ \lpquote y!(z) \rpquote ]$.

\begin{eqnarray}
	\lift{w}{y!(z)}\widehat{\id{\{}u / z \id{\}}}
		& = &
		\lift{w}{y!(u)} \nonumber\\
	w[ \lpquote y!(z) \rpquote ] \widehat{ \id{\{}u / z \id{\}} }
		& = &
		w[ \lpquote y!(z) \rpquote ] \nonumber
\end{eqnarray}

Because the body of the process between quotes is impervious to
substitution, we get radically different answers. In fact, by
examining the first process in an input context,
e.g. $x?(z).\lift{w}{y!(z)}$, we see that the process under the lift
operator may be shaped by prefixed inputs binding a name inside it. In
this sense, the lift operator will be seen as a way to dynamically
construct processes before reifying them as names.

Finally equipped with these standard features we can present the
dynamics of the calculus.

\subsubsection{Operational semantics} 

Finally, we introduce the computational dynamics. What marks these
algebras as distinct from other more traditionally studied algebraic
structures, e.g. vector spaces or polynomial rings, is the manner in
which dynamics is captured. In traditional structures, dynamics is typically
expressed through morphisms between such structures, as in linear maps
between vector spaces or morphisms between rings. In algebras
associated with the semantics of computation, the dynamics is
expressed as part of the algebraic structure itself, through a
reduction reduction relation typically denoted by $\red$. Below, we
give a recursive presentation of this relation for the calculus used
in the encoding.

$\red \subseteq \pi \times \pi$
$\red : \pi \to \mathcal{P}(\pi)$

\begin{mathpar}
  \inferrule* [lab=Comm] { \textsf{match}( x_{src}, x_{trgt} ) } { x_{trgt}?(y)P \; | \; x_{src}!\langle {Q} \rangle \red P\{\quotep{Q}/y}\} }
  \and \\
  \inferrule* [lab=Par] {{P} \red {P}'} {{{P} | {Q}} \red {{P}' | {Q}}}
  \and
  \inferrule* [lab=Equiv]{{{P} \scong {P}'} \andalso {{P}' \red {Q}'} \andalso {{Q}' \scong {Q}}}{{P} \red {Q}}
\end{mathpar}

\begin{eqnarray*}
  match_{\equiv} (\quotep{P},\quotep{Q}) & := & P \equiv Q \\
  match_{\dagger}(\quotep{P},\quotep{Q}) & := & \forall R. P|Q \red^{*} R => R \red^{*} 0 \\
  match_{K}(\quotep{P},\quotep{Q}) & := & K \mbox{ for some context } K
\end{eqnarray*}

$u?(x)P | u!\langle Q \rangle \red P\{\quotep{Q}/x\}$

%We write $\wred$ for $\red^*$, and $P\red$ if $\exists Q $ such that $ P \red Q$.
We write $P\red$ if $\exists Q $ such that $ P \red Q$ and $P\not\red$, otherwise.

\section{Replication}

As mentioned before, it is known that replication (and hence
recursion) can be implemented in a higher-order process algebra
\cite{SangiorgiWalker}. As our first example of calculation with the
machinery thus far presented we give the construction explicitly in
the {\rhoc}.

\begin{eqnarray}
	D_{x} & := & \prefix{x}{y}{(\binpar{\outputp{x}{y}}{@{y}})} \nonumber\\
	\bangp_{x}{P} & := & \binpar{{x}!\langle{\binpar{D_{x}}{P}}\rangle}{D_{x}} \nonumber
\end{eqnarray}

\begin{eqnarray}
	\bangp_{x}{P} & & \nonumber\\
	=
	& {x}!\langle{(\prefix{x}{y}{(\outputp{x}{y} | @{y})) | P}}\rangle 
	      | \prefix{x}{y}{(\outputp{x}{y} | @{y})} & \nonumber\\
	\red
	& (\outputp{x}{y} | @{y})\substn{\quotep{(\prefix{x}{y}{(@{y} | \outputp{x}{y})) | P}}}{y} & \nonumber\\
	=
	& \outputp{x}{\quotep{(\prefix{x}{y}{(\outputp{x}{y} | @{y})) | P}}}
	  | {(\prefix{x}{y}{(\outputp{x}{y} | @{y})) | P}} & \nonumber\\
	\red
	& \ldots & \nonumber\\
	\red^*
	& P | P | \ldots & \nonumber
\end{eqnarray}

Of course, this encoding, as an implementation, runs away, unfolding
$\bangp{P}$ eagerly. A lazier and more implementable replication
operator, restricted to input-guarded processes, may be obtained as follows.

\begin{eqnarray}
\bangp{\prefix{u}{v}{P}} 
	:= 
	\binpar{\lift{x}{\prefix{u}{v}{(\binpar{D(x)}{P})}}}{D(x)} \nonumber
\end{eqnarray}

\begin{remark}
  Note that the lazier definition still does not deal with summation
  or mixed summation (i.e. sums over input and output). The reader is
  invited to construct definitions of replication that deal with these
  features. 

  Further, the definitions are parameterized in a name, $x$. Can you,
  gentle reader, make a definition that eliminates this parameter and
  guarantees no accidental interaction between the replication
  machinery and the process being replicated -- i.e. no accidental
  sharing of names used by the process to get its work done and the
  name(s) used by the replication to effect copying. This latter
  revision of the definition of replication is crucial to obtaining
  the expected identity $!!P \sim !P$.
\end{remark}

\begin{remark}\label{rem:paradoxical_combinator}
  The reader familiar with the lambda calculus will have noticed the
  similarity between $D$ and the paradoxical combinator.

  [Ed. note: the existence of this seems to suggest we have to be more
  restrictive on the set of processes and names we admit if we are to
  support no-cloning.]
\end{remark}

\subsubsection{Bisimulation}

The computational dynamics gives rise to another kind of equivalence,
the equivalence of computational behavior. As previously mentioned
this is typically captured \emph{via} some form of bisimulation.

% The notion we use in this paper is weak barbed bisimulation
% \cite{milner91polyadicpi}.

The notion we use in this paper is derived from weak barbed
bisimulation \cite{milner91polyadicpi}. 

\begin{definition}
An \emph{observation relation}, $\downarrow_{\mathcal N}$, over a set
of names, $\mathcal N$, is the smallest relation satisfying the rules
below.

\infrule[Out-barb]{y \in {\mathcal N}, \; x \nameeq y}
		  {\outputp{x}{v} \downarrow_{\mathcal N} x}
\infrule[Par-barb]{\mbox{$P\downarrow_{\mathcal N} x$ or $Q\downarrow_{\mathcal N} x$}}
		  {\binpar{P}{Q} \downarrow_{\mathcal N} x}

We write $P \Downarrow_{\mathcal N} x$ if there is $Q$ such that 
$P \wred Q$ and $Q \downarrow_{\mathcal N} x$.
\end{definition}

\begin{definition}
%\label{def.bbisim}
An  ${\mathcal N}$-\emph{barbed bisimulation} over a set of names, ${\mathcal N}$, is a symmetric binary relation 
${\mathcal S}_{\mathcal N}$ between agents such that $P\rel{S}_{\mathcal N}Q$ implies:
\begin{enumerate}
\item If $P \red P'$ then $Q \wred Q'$ and $P'\rel{S}_{\mathcal N} Q'$.
\item If $P\downarrow_{\mathcal N} x$, then $Q\Downarrow_{\mathcal N} x$.
\end{enumerate}
$P$ is ${\mathcal N}$-barbed bisimilar to $Q$, written
$P \wbbisim_{\mathcal N} Q$, if $P \rel{S}_{\mathcal N} Q$ for some ${\mathcal N}$-barbed bisimulation ${\mathcal S}_{\mathcal N}$.
\end{definition}

$\mathcal{R} \subseteq \pi \times \pi$

$P \mathcal{R} Q => \forall P'. P \red P' \Rightarrow \exists Q'. Q \red Q', P' \mathcal{R} Q'$

$P \vdash x \Rightarrow Q \vdash x$

\begin{mathpar}
  \inferrule*[lab=Out-barb]{x \nameeq y}{{y}!\langle{Q}\rangle \vdash x}
  \and
  \inferrule*[lab=Par-barb]{\mbox{$P\vdash x$ or $Q\vdash x$}}{\binpar{P}{Q} \vdash x}
\end{mathpar}

\subsubsection{Contexts}

One of the principle advantages of computational calculi like the
$\pi$-calculus is a well-defined notion of context,
contextual-equivalence and a correlation between
contextual-equivalence and notions of bisimulation. The notion of
context allows the decomposition of a process into (sub-)process and
its syntactic environment, its context. Thus, a context may be
thought of as a process with a ``hole'' (written $\Box$) in it. The
application of a context $M$ to a process $P$, written $M[P]$, is
tantamount to filling the hole in $M$ with $P$. In this paper we do
not need the full weight of this theory, but do make use of the notion
of context in the proof the main theorem. 

\begin{mathpar}
  \inferrule* [lab=summation] {} {{M_{M},M_{N}} \bc \Box \;|\; x.M_{A} \;|\; M_{M}+M_{N}}
  \and
  \inferrule* [lab=agent] {} {{M_{A}} \bc (\vec{x})M_{P} \;| \; \clift{P_0,\ldots,M_{P},\ldots,P_N}}
  \and \\
  \inferrule* [lab=process] {} {{M_{P}} \bc M_{N} \;| \;P|M_{P} }
\end{mathpar} 

\begin{mathpar}
  \inferrule* [lab=sychronization] {} {M_{N} \bc \Box \;|\; x?M_{F} \;|\; x!M_{C}}
  \and
  \inferrule* [lab=abstraction] {} {{M_{F}} \bc (x)M_{P} }
  \and
  \inferrule* [lab=concretion] {} {{M_{C}} \bc \langle M_{P} \rangle }
  \and \\
  \inferrule* [lab=process] {} {{M_{P}} \bc M_{N} \;| \;P|M_{P} }
\end{mathpar}

\begin{definition}[contextual application] Given a context $M$, and
  process $P$, we define the \emph{contextual application}, $M[P] :=
  M\{P/\Box\}$. That is, the contextual application of M to P is the
  substitution of $P$ for $\Box$ in $M$.
\end{definition}

$\meaningof{-} : L \to \mathcal{P}(\pi)$

\begin{mathpar}
  \inferrule* [lab=collection] {} {\meaningof{true} = \pi, \and \meaningof{~E} = \pi \setminus \meaningof{E}, \and \meaningof{E_{1} \& E_{2}} = \meaningof{E_{1}} \cap \meaningof{E_{2}}}
\end{mathpar}

\begin{mathpar}
  \inferrule* [lab=structure] {} {\meaningof{0} = \{ P \in \pi | P \equiv 0 \}, \and \\ \meaningof{E_1 | E_2} = \{ P \in \pi | P \equiv P_{1} | P_{2}, P_{1} \in \meaningof{E_{1}}, P_{2} \in \meaningof{E_2}\} }
\end{mathpar}

\begin{mathpar}
 \inferrule* [lab=behavior] {} {\meaningof{\langle a?b \rangle E} = \{ P \in \pi | P \equiv Q | u?(y)P', \\ \and \\\\ \and \\ \;\;\; u \in \meaningof{a}, \forall z.P'\{z/y\} \in \meaningof{E\{z/b\}}\}, \and \\ \meaningof{a!E} = \{ P \in \pi | P \equiv Q | x!\langle P' \rangle, x \in \meaningof{a} P' \in \meaningof{E}\} }
\end{mathpar}

\begin{mathpar}
 \inferrule* [lab=nominal] {} {\meaningof{\quotep{E}} = \{ \quotep{P} \in \quotep{\pi} | P \in \meaningof{E} \}, \and \meaningof{\quotep{P}} = \{ \quotep{Q} \in \quotep{\pi} | P \equiv Q \} \and \\ \meaningof{@\quotep{E}} = \{ P \in \pi | P \equiv @x, x \in \meaningof{E} \}}
\end{mathpar}

\begin{eqnarray*}
  \\
  \meaningof{-} : TS \to ST
\end{eqnarray*}

\begin{eqnarray*}
  \\
  L : TS \to ST
\end{eqnarray*}

\begin{eqnarray*}
  \\
  P \models E \iff P \in \meaningof{E}
\end{eqnarray*}

\begin{eqnarray*}
  P \approx_{L} Q \iff \forall E \in L. P \models E \iff Q \models E
\end{eqnarray*}

\begin{eqnarray*}
  P \approx_{K} Q
\end{eqnarray*}

\begin{eqnarray*}
  P \approx Q
\end{eqnarray*}

$\approx_{K} = \approx = \approx_{L}$

\subsubsection{Contextual duality}

Note that contexts extend the quotation operation to a family of
operations from processes to names. Given a context, $M$, we can
define a \emph{nominal context}, $\quotep{M}$ by $\quotep{M}[P] :=
\quotep{M[P]}$. To foreshadow what is to come we observe that these
operations enjoy a duality with processes very much like the duality
between vectors and maps from vectors to scalars.

Further, because the calculus is essentially higher-order, we have a
correspondence between contexts and processes. More specifically,
given a name $x$ and a context $M$ we can construct $M^{*}_{x}$ such
that 

\begin{mathpar}
  M^{*}_{x} | \lift{x}{P} \red M[P]
\end{mathpar}

namely,

\begin{mathpar}
  M^{*}_{x} := x?(u).M[\dropn{u}]
\end{mathpar}

The dependence of $M^{*}_{x}$ on a name makes it an abstraction, 

\begin{mathpar}
  M^{*} := (x)x?(u).M[\dropn{u}]
\end{mathpar}

\subsection{Additional notation}

It will sometimes be convenient to denote the process a name
quotes. We already have the notation $x = \quotep{P}$, but it will be
convenient to introduce an alternate notation, $\procn{x}$, when we
want to emphasize the connection to the use of the name. Note that, by
virtue of name equivalence, $\quotep{\procn{x}} \nameeq x$; so, the
notation is consistent with previous definitions.

Further, because names have structure it is possible to effect
substitutions on the basis of that structure. This means we need to
upgrade our notation for substitutions, which we accomplish by
adapting comprehension notation. Thus,

\begin{mathpar}
  P\{ y / x : x \in S \}
\end{mathpar}

is interpreted to mean the process derived from P by replacing (in a
capture-avoiding manner) each occurrence of $x$ in $S$ by $y$. For example,

\begin{mathpar}
  P\{ \quotep{\procn{x}|\procn{x}} / x : x \in \freenames{P} \}
\end{mathpar}

will replace each (occurrence) of a free name $x$ in $P$ by
$\quotep{\procn{x}|\procn{x}}$.

Also, we will avail ourselves of the notation $x^{L}$ and $x^{R}$ to
denote injections of a name into disjoint copies of the name
space. There are numerous ways to accomplish this. One example can be
found in \cite{MeredithR05}. This notation overloads to vectors of
names: $\vec{x}^{\pi} := (x_{i}^{\pi} \; : \; 0 \leq i < |\vec{x}| )$ where $\pi \in \{L,R\}$.

We also use $P^{\Box} := P|\Box$.

In \cite{MeredithR05} an interpretation of the new operator is
given. It turns out that there are several possible interpretations
all enjoying the requisite algebraic properties of the operator (see
\cite{milner91polyadicpi}). We will therefore make liberal use of
$(\nu\; \vec{x})P$.

% subsection the_syntax_and_semantics_of_the_notation_system (end)   

\section{Interpretation of QM}
\subsection{Supporting definitions}
\subsubsection{Multiplication}
\begin{mathpar}
  \quotep{Q} \cdot \quotep{R} := \quotep{Q|R}
  \and \\
  \quotep{Q} \cdot P := P\{ \quotep{Q|R} / \quotep{R} : \quotep{R} \in \freenames{P} \}
\end{mathpar}

\paragraph{Discussion}
The first line needs little explanation. The second line says that
each free name of the process is replaced with the multiplication of
that name by the scalar. Multiplication of a scalar (name) by a state
(process) results in a process all the names of which have been `moved
over' by parallel composition with the process the scalar
quotes. There is a subtlety that the bound names have to be
manipulated so that multiplied names aren't accidentally
captured. There are many ways to achieve this.

\begin{remark}\label{rem:multiplication_identities}
  The reader is invited to verify that for all $x,y,z \in \QProc$ and $P \in \Proc$
  \begin{mathpar}
    x \cdot \quotep{0} \equiv x 
    \and
    x \cdot y \equiv y \cdot x
    \and
    x \cdot (y \cdot z) \equiv (x \cdot y) \cdot z
    \and \\
    \quotep{0} \cdot P \equiv P
    \and \\
    x \cdot (y \cdot P) \equiv (x \cdot y) \cdot P
    \and \\
    x \cdot (P|Q) \equiv (x \cdot P) | (x \cdot Q)
    \and \\    
  \end{mathpar}
\end{remark}

\subsubsection{Tensor product}

We define a tensor product on processes by structural induction.

\paragraph{Tensor of sums} First note that all summations, including
$\pzero$ and sequence, can be written $\Sigma_{i} x_{i}.A_{i} +
\Sigma_{j} x_{j}.C_{j}$, where we have grouped input-guarded processes
together and output-guarded processes together.

Thus, we can define the tensor product of two summations, $N_{1}\otimes N_{2}$, where

\begin{mathpar}
  N_{1} := \Sigma_{i} x_{i}.A_{i} + \Sigma_{j} x_{j}.C_{j}
  \and
  N_{2} := \Sigma_{i'} y_{i'}.B_{i'} + \Sigma_{j'} y_{j'}.D_{j'} 
\end{mathpar}

as follows.

\begin{mathpar}
  \Sigma_{i} x_{i}.A_{i} + \Sigma_{j} x_{j}.C_{j} \otimes \Sigma_{i'}
  y_{i'}.B_{i'} + \Sigma_{j'} y_{j'}.D_{j'} 
  \and \\
  := \; \Sigma_{i} \Sigma_{i'} \quotep{\stackrel{\vee}{x_{i}}| \stackrel{\vee}{y_{i'}}}.(A_{i}\otimes B_{i'}) \; | \; \Sigma_{i'} \Sigma_{i} \quotep{\stackrel{\vee}{y_{i'}}|\stackrel{\vee}{x_{i}}}.(B_{i'}\otimes A_{i})
  \and
  \;\; | \;\; \Sigma_{j} \Sigma_{j'} \quotep{\stackrel{\vee}{x_{j}}|\stackrel{\vee}{y_{j'}}}.(A_{j}\otimes B_{j'}) \; | \; \Sigma_{j'} \Sigma_{j} \quotep{\stackrel{\vee}{y_{j'}}|\stackrel{\vee}{x_{j}}}.(B_{j'}\otimes A_{j})
\end{mathpar}

\begin{remark}
  Do we need to $x^{L}$ and $y^{R}$ for this construction as well?
\end{remark}

\paragraph{Tensor of parallel compositions} Next, we distribute tensor
over par.

\begin{mathpar}
  P_{1}|P_{2} \otimes Q_{1}|Q_{2} := (P_{1} \otimes Q_{1}) | (P_{1}
  \otimes Q_{2}) | (P_{2} \otimes Q_{1}) | (P_{2} \otimes Q_{2})
\end{mathpar}

\paragraph{Tensor with dropped names} We treat tensor of a
process with a dropped name as parallel composition.

\begin{mathpar}
  P \otimes \dropn{x} := P | \dropn{x}
\end{mathpar}

\paragraph{Tensor of agents}

Finally, we need to define tensor on agents. Note that the definition
of tensor on normal products only tensors inputs with inputs and
outputs with outputs. Thus, we only have to define the operation on
``homogeneous'' pairings.

\begin{mathpar}
  (\vec{x})P \otimes (\vec{y})Q
  \and \\
  := (x_{0}^{L}|y_{0}^{R},\ldots,x_{0}^{L}|y_{n}^{R},\ldots,x_{m}^{L}|y_{0}^{R},\ldots,x_{m}^{L}|y_{n}^R)(P\{ \vec{x}^{L}/\vec{x}\} \otimes Q \{ \vec{y}^{R}/\vec{y}\})
  \and \\
  \clift{\vec{P}} \otimes \clift{\vec{Q}}
  \and \\
  := \clift{P_{0}\otimes Q_{0},\ldots,P_{0}\otimes Q_{n},\ldots,P_{m}\otimes Q_{0},\ldots,P_{m}\otimes Q_{n}}
\end{mathpar}

\begin{remark}
  Observe that arities of tensored abstractions matches arities of
  tensored concretions if the original arities matched. Note also that
  the length of the arities corresponds to the increase in dimension
  we see in ordinary vector space tensor product.
\end{remark}

\begin{remark}
  Operationally, this definition distributes the tensor down to
  components ``linked'' by summation. Tensor over summation is
  intriguing in that it mixes names. Moreover, as a consequence of the
  way it mixes names we have the identities for all $x \in \QProc$ and
  $P,Q \in \Proc$

  \begin{mathpar}
    (x \cdot P) \otimes Q \equiv x \cdot (P \otimes Q) \equiv P \otimes (x \cdot Q)
    \and
    P \otimes \pzero \equiv P
  \end{mathpar}

  that the reader is invited to verify.
\end{remark}

\subsubsection{Annihilation}
\begin{mathpar}
  P^{\perp} := \{ Q | \forall R. P|Q \red^{*} R \Rightarrow R \red^{*} \pzero \}
  \and \\
  P^{\underline{\perp}} := \Sigma_{Q \in P^{\perp}} \quotep{Q}?(y).(\dropn{y}|Q) | \Sigma_{Q \in P^{\perp}} \quotep{Q}\clift{\Box}
\end{mathpar}

\paragraph{Discussion} The reader will note that $P^{\perp}$ is a
\emph{set} of processes, while $P^{\underline{\perp}}$ is a
\emph{context}. We call the set $P^{\perp}$ the \emph{annihilators} of
$P$. The parallel composition of a process in the annihilators of $P$
with $P$ will result in a process, the state space of which has all
paths eventually leading to $\pzero$. Execution may endure loops; but
under reasonable conditions of fairness (naturally guaranteed under
most notions of bisimulation) such a composite process cannot get
stuck in such a loop and will, eventually pop out and terminate.

The context $P^{\underline{\perp}}$ is ready and willing to ``take the
$P$ out of'' the process to which it is applied. It will effectively
transmit the code of the process to which it is applied to one of the
annihilators and run the process against it.

\subsubsection{Evaluation}
We fix $M$ a domain of fully abstract interpretation with an equality
coincident with bisimulation. We take $\meaningof{\cdot} : \Proc \to
M$ to be the map interpreting processes and $\nmeaningof{\cdot} : \M
\to Proc$ to be the map running the other way. Then we define

\begin{mathpar}
  \int P := \nmeaningof{\meaningof{P}}
\end{mathpar}

\paragraph{Discussion}
There are many fully abstract interpretations of Milner's
$\pi$-calculus. Any of them can be used as a basis for interpreting
the reflective calculus here. Equipped with such a domain it is
largely a matter of grinding through to check that the Yoneda
construction for the normalization-by-evaluation program can be
extended to this setting.

\begin{remark}
  The reader is invited to verify that $\int (P^{\underline{\perp}}[P]) = 0$.
\end{remark}

\subsection{Quantum mechanics}

Table \ref{tbl:core_qm_op_defns} gives the core operational definitions

\begin{table}[htp]\label{tbl:core_qm_op_defns}
  \center{
    \fbox{
      \begin{tabular}{c|c}
        quantum mechanics & process calculus \\
        \hline
        scalar & $x := \quotep{P}$ \\
        state vector & $\state{P} := P$ \\
        dual & $\state{P}^{*} := \event{P^{\underline{\perp}}} := \quotep{P^{\underline{\perp}}}[-]$ \\
        matrix & $ \Sigma_{\alpha} \state{P_{\alpha}}x_{\alpha}\event{Q_{\alpha}}$ \\
        vector addition & $\state{P} + \state{Q} := \state{P | Q}$ \\
        tensor product & $\state{P} \otimes \state{Q} := \state{P \otimes Q}$ \\
        inner product & $\innerprod{P}{Q} := \quotep{\int P^{\underline{\perp}}[Q]}$ \\
      \end{tabular}
    }
  }
  \caption{QM - operational definitions}
\end{table}

where

\begin{mathpar}
  \prmatrix{P}{Q} := \fprmatrix{P}{\quotep{\pzero}}{Q}
  \and
  \fprmatrix{P}{x}{Q} := (\state{P},x,\event{Q})
  \and
  (\fprmatrix{P}{x}{Q})(\state{R}) := x \cdot \innerprod{Q}{R} \cdot \state{P}
  \and
  (\fprmatrix{P}{x}{Q})(\event{R}) := x \cdot \innerprod{R}{P} \cdot \event{Q}
\end{mathpar}

\paragraph{Discussion}
As promised: vectors (aka states) are represented as processes; duals
as contextual duals; inner product definition should be compared with
standard inner product definition for ....

\begin{remark}
  Assuming $\int (P^{\underline{\perp}}[P]) = 0$, the reader is
  invited to verify that $(\fprmatrix{P}{x}{P})(\state{P}) = x \cdot \state{P}$.
\end{remark}

\begin{remark}
  The reader is invited to verify that $\innerprod{P}{Q}$ could
  equally well have been written $\quotep{\int \stackrel{\vee}{x}}$
  where $x = \event{P^{\underline{\perp}}}(Q)$.

  One of the motivations for this remark is that there is another way
  to factor these operations. We could package up evaluation in the dual:

  \begin{mathpar}
    \state{P}^{*} := \event{\int P^{\underline{\perp}}} := \quotep{\int P^{\underline{\perp}}}[-]
  \end{mathpar}

  and then have inner product defined by
  
  \begin{mathpar}
    \innerprod{P}{Q} := \event{P}(Q)
  \end{mathpar}

  Hopefully, experience with the calculations will provide guidance on
  the best factoring.
\end{remark}

\begin{remark}
  Assuming $\int (P^{\underline{\perp}}[P]) = 0$, the reader is
  invited to verify that $\forall P,Q. (\prmatrix{0}{Q})(\state{0}) =
  \state{0}$ and dually $(\prmatrix{P}{0})(\event{0}) = \event{0}$.
\end{remark}

\begin{remark}
  i'm a little worried that i don't (yet) have proper support for
  complex conjugacy. But, the observation above may give us a
  clue. According to Abramsky, it must be the case that the scalars
  are iso to the homset of the identity for the tensor -- which the
  observation above characterizes. 

  For now, we will simply bookmark the notion with $\overline{x}$.
\end{remark}

\subsubsection{Adjointness}

We need to give a definition of $(\cdot)^{\dagger}$ for matrices. The
obvious candidate definition is
\begin{mathpar}
(\Sigma_{\alpha}\fprmatrix{P_{\alpha}}{x_{\alpha}}{Q_{\alpha}})^{\dagger}
= \Sigma_{\alpha}\fprmatrix{(Q_{\alpha}^{\underline{\perp}})^{*}}{\overline{x}_{\alpha}}{P_{\alpha}^{\underline{\perp}}} 
\end{mathpar}

But, $(Q_{\alpha}^{\underline{\perp}})^{*}$ requires a name along
which to communicate the process to achieve the context application.

\subsubsection{Basis for a basis}
If processes label states and ``addition'' of states (a.k.a. vector
addition) is interpreted as parallel composition, what corresponds to
notions of linear independence and basis? Here, we recall that Yoshida
has developed a set of \emph{combinators} for an asynchronous verison
of Milner's $\pi$-calculus. These are a finite set of processes such
any process can be expressed as parallel composition of these
combinators together with liberal uses of the new operator and
replication. We can simply give a translation of these into the
present calculus and have reasonable expectation that the property
carries over. That is, that the resultant set allows to express all
processes via parallel composition. Note, however, that there is no
new operator or replication in this calculus. As a result, we expect
that the corresponding set is actually infinite. That is, we expect
that the space is actually infinite dimensional.

\begin{remark}
  The attentive reader may be a bit concerned. Certainly, the
  collection $S$, $K$ and $I$ is a finite set of
  combinators. Shouldn't we expect to see a finite set of combinators
  for an effectively equivalent system? i am very sympathetic to this
  critique and feel it warrants full attention. On the other hand, i
  also have in mind the following analogy. The natural numbers, as a
  monoid under addition, has exactly $1$ generator, while the natural
  numbers, as a monoid under multiplication, has countably many
  generators (the primes). We observe that the application of the
  lambda calculus is much less resource sensitive than the parallel
  composition of the $\pi$-calculus. Could it be the case that we have
  an analogy of the form
  
  \begin{mathpar}
    m + n : MN :: m*n : M|N
  \end{mathpar}

  giving a similar blow up in the set of ``primes''?  This is such a
  wonderful thought that, even if it's not true, i think it's worth
  writing down.
\end{remark}
 

\documentclass[12pt]{llncs}
%\documentclass{jktr}

\usepackage[pdftex]{hyperref}                   
\usepackage {listings}
\usepackage {mathpartir}
\usepackage{bcprules}
%\usepackage{listings}
                       
\usepackage{graphicx} 
%\usepackage[margins=2.5cm,nohead,nofoot]{geometry}
%\usepackage{geometry}
\usepackage{amsfonts}
\usepackage{amstext}
\usepackage{latexsym}
\usepackage{amssymb}
\usepackage{color}


%\include{myPreamble}
\include{qm2pi.local} 

%\ifpdf
%\usepackage[pdftex]{graphicx}
%\else
%\usepackage{graphicx}
%\fi

 % \ifpdf
%  \usepackage{pdfsync}
%  \if


%\title{Brief Article}
%\author{David F. Snyder}
%\author{L.G. Meredith}

%\address{Dept. of Math., Texas State University--San Marcos, San Marcos, TX 78666}
       
\pagestyle{empty}


\begin{document}

\lstset{language=[Objective]Caml,frame=shadowbox}

\input{qm2pi.front}

% section front matter (end)

\input{qm2pi.intro} 
 
% section introduction (end)

% \input{qm2pi.knotations} 

% section notation (end)

\input{qm2pi.process.calculi} 

% section concurrent_process_calculi_and_spatial_logics_ (end)
    
%\input{qm2pi.knots2pi} 

%\input{qm2pi.trefoil} 

%\input{qm2pi.mainthm} 

% subsection basic_interpretation (end)

%\input{qm2pi.rho.presentation} 
\subsection{The syntax and semantics of the notation system}\label{sub:the_syntax_and_semantics_of_the_notation_system} % (fold)

We now summarize a technical presentation of the calculus that
embodies our theory of dynamics. The typical presentation of such a
calculus follows the style of giving generators and relations on
them. The grammar, below, describing term constructors, freely
generates the set of processes, $\Proc$. This set is then quotiented
by a relation known as structural congruence and it is over this set
that the notion of dynamics is expressed. This presentation is
essentially that of \cite{MeredithR05} with the addition of
polyadicity and summation. For readability we have relegated some of
the technical subtleties to an appendix.

\subsubsection{Process grammar}\label{subsub:process_grammar}

\begin{mathpar}
  \inferrule* [lab=synchronization] {} {{M} \bc \pzero \;|\; x?F \;|\; x!C }
  \and
  \inferrule* [lab=abstraction] {} {{F} \bc (x)P}
  \and
  \inferrule* [lab=concretion] {} {{C} \bc \langle Q \rangle}
  \and
  \inferrule* [lab=process] {} {{P,Q} \bc M \;| \;P|Q \;|\; @{x}}
  \and
  \inferrule* [lab=name] {} {{x} \bc \quotep{P}}
\end{mathpar} 

Note that $\vec{x}$ (resp. $\vec{P}$) denotes a vector of names
(resp. processes) of length $|\vec{x}|$ (resp. $|\vec{P}|$). We adopt
the following useful abbreviations.

\begin{mathpar}
   x?(\vec{y}).P := x.(\vec{y})P \and  x\clift{\vec{P}} := x.\clift{\vec{P}}
   \and x!(y) := \lift{x}{\dropn{y}}
   \and \Pi_{i=0}^{n-1}P_i := P_0 | \ldots | P_{n-1}
\end{mathpar}

\subsubsection{Structural congruence}

\paragraph{Free and bound names and alpha-equivalence.} At the
core of structural equivalence is alpha-equivalence which identifies
process that are the same up to a change of variable. Formally, we
recognize the distinction between free and bound names. The free names
of a process, $\freenames{P}$, may be calculated recursively as
follows:

\begin{mathpar}
\freenames{\pzero} := \emptyset
  \and \\
  \freenames{x?(y).P} := \{ x \} \cup (\freenames{P} \setminus \{ y \})
  \and 
  \freenames{x!\langle P \rangle} := \{ x \} \cup \{ P \} 
  \and \\
  \freenames{P|Q} := \freenames{P} \cup \freenames{Q}
  \and \\
  \freenames{@{x}} := \{ x \}
\end{mathpar}

$\pi$
$\quotep{\pi}$

$\freenames{-} : \pi \to \mathcal{P}(\quotep{\pi})$

\begin{eqnarray*}
  \freenames{\pzero} & := & \emptyset \\
  \freenames{x?(y).P} & := & \{ x \} \cup (\freenames{P} \setminus \{ y \}) \\
  \freenames{x!\langle P \rangle} & := & \{ x \} \cup \{ P \} \\
  \freenames{P|Q} & := & \freenames{P} \cup \freenames{Q} \\
  \freenames{\dropn{x}} & := & \{ x \}
\end{eqnarray*}

The bound names of a process, $\boundnames{P}$, are those names occurring in $P$
that are not free. For example, in $x?(y).0$, the name $x$ is free, while $y$ is bound.

\begin{mathpar}
  \inferrule* [lab=monoidal-laws] {} { P|Q \equiv Q|P \and P|0 \equiv P \and P|(Q|R) \equiv (P|Q)|R }
\end{mathpar}

\begin{mathpar}
  \inferrule* [lab=alpha-equivalence] {} { (x)P \equiv (y)P\{y/x\} \and y \not\in \freenames{P} }
\end{mathpar}

\begin{definition}
Then two processes, $P,Q$, are alpha-equivalent if $P = Q\{\vec{y}/\vec{x}\}$ for
some $\vec{x} \in \boundnames{Q},\vec{y} \in \boundnames{P}$, where $Q\{\vec{y}/\vec{x}\}$
denotes the capture-avoiding substitution of $\vec{y}$ for $\vec{x}$ in $Q$.
\end{definition}

\begin{definition}
  The {\em structural congruence} \cite{SangiorgiWalker} , $\equiv$,
  between processes is the least congruence containing
  alpha-equivalence, satisfying the abelian monoid laws
  (associativity, commutativity and $\pzero$ as identity) for parallel
  composition $|$ and for summation $+$.
\end{definition}

\subsection{Name equivalence}

We take name equivalence, written $\nameeq$, to be the smallest
equivalence relation generated by the following rules.

\begin{mathpar}
\inferrule*[lab=Quote-drop]
{ }
{ \quotep{@{x}} \nameeq x }

\inferrule*[lab=Struct-equiv]
{ P \scong Q }
{ \quotep{P} \nameeq \quotep{Q} }
\end{mathpar}

The astute reader will have noticed that the mutual recursion of names
and processes imposes a mutual recursion on alpha-equivalence and
structural equivalence via name-equivalence. Fortunately, all of this
works out pleasantly and we may calculate in the natural way, free of
concern. The reader interested in the details is referred to the
appendix \ref{appendix:rho_details}.

\subsection{Substitution}

We use $\Proc$ for the set of processes, $\QProc$ for the set of
names, and $\id{\{}\vec{y} / \vec{x} \id{\}}$ to denote partial maps,
$s : \QProc \rightarrow \QProc$. A map, $s$ lifts, uniquely, to a map
on process terms, $\widehat{s} : \Proc \rightarrow \Proc$ by the
following equations.

\begin{mathpar}
  (0) \psubstp{Q}{P} := 0 \\
  (R \juxtap S) \psubstp{Q}{P}
  :=    
  (R)\psubstp{Q}{P} \juxtap (S) \psubstp{Q}{P} \\
  (x?(y).R) \psubstp{Q}{P}    
  :=    
  (x)\substp{Q}{P} (z)\concat( (R \psubstn{z}{y}) \psubstp{Q}{P} ) \\
  (\lift{x}{R}) \psubstp{Q}{P}  
  :=
  \lift{(x)\substp{Q}{P}}{ R \psubstp{Q}{P} } \\
%   (\dropn{x})  \psubstp{Q}{P}       
%   := 
%   \left\{ 
%     \begin{array}{ccc} 
%       \dropn{\quotep{Q}} & & x \nameeq \quotep{P} \\
%       \dropn{x} & & otherwise \\
%     \end{array}
%   \right. 
  (\dropn{x})  \psubstp{Q}{P}       
  := 
  \left\{ 
    \begin{array}{ccc} 
      Q & & x \nameeq \quotep{P} \\
      \dropn{x} & & otherwise \\
    \end{array}
  \right.
\end{mathpar}
 

where

\begin{eqnarray}
  (x)\id{\{} \lpquote Q \rpquote / \lpquote P \rpquote \id{\}}            = 
  \left\{ 
    \begin{array}{ccc}
      \lpquote Q \rpquote & & x \nameeq \lpquote P \rpquote \\
      x & & otherwise \\
    \end{array}
  \right. \nonumber
\end{eqnarray}

and $z$ is chosen distinct from $\quotep{P}$, $\quotep{Q}$, the free
names in $Q$, and all the names in $R$. Our $\alpha$-equivalence will
be built in the standard way from this substitution.

\begin{remark}\label{rem:no_self_referential_names}
  One consequence of these definitions is that $\forall P. \quotep{P}
  \not\in \freenames{P}$.
\end{remark}

\subsection{ Dynamic quote: an example }

Anticipating something of what's to come, consider applying the
substitution, $\widehat{\id{\{}u / z \id{\}}}$, to the following pair
of processes, $\lift{w}{y!(z)}$ and $w[ \lpquote y!(z) \rpquote ]$.

\begin{eqnarray}
	\lift{w}{y!(z)}\widehat{\id{\{}u / z \id{\}}}
		& = &
		\lift{w}{y!(u)} \nonumber\\
	w[ \lpquote y!(z) \rpquote ] \widehat{ \id{\{}u / z \id{\}} }
		& = &
		w[ \lpquote y!(z) \rpquote ] \nonumber
\end{eqnarray}

Because the body of the process between quotes is impervious to
substitution, we get radically different answers. In fact, by
examining the first process in an input context,
e.g. $x?(z).\lift{w}{y!(z)}$, we see that the process under the lift
operator may be shaped by prefixed inputs binding a name inside it. In
this sense, the lift operator will be seen as a way to dynamically
construct processes before reifying them as names.

Finally equipped with these standard features we can present the
dynamics of the calculus.

\subsubsection{Operational semantics} 

Finally, we introduce the computational dynamics. What marks these
algebras as distinct from other more traditionally studied algebraic
structures, e.g. vector spaces or polynomial rings, is the manner in
which dynamics is captured. In traditional structures, dynamics is typically
expressed through morphisms between such structures, as in linear maps
between vector spaces or morphisms between rings. In algebras
associated with the semantics of computation, the dynamics is
expressed as part of the algebraic structure itself, through a
reduction reduction relation typically denoted by $\red$. Below, we
give a recursive presentation of this relation for the calculus used
in the encoding.

$\red \subseteq \pi \times \pi$
$\red : \pi \to \mathcal{P}(\pi)$

\begin{mathpar}
  \inferrule* [lab=Comm] { \textsf{match}( x_{src}, x_{trgt} ) } { x_{trgt}?(y)P \; | \; x_{src}!\langle {Q} \rangle \red P\{\quotep{Q}/y}\} }
  \and \\
  \inferrule* [lab=Par] {{P} \red {P}'} {{{P} | {Q}} \red {{P}' | {Q}}}
  \and
  \inferrule* [lab=Equiv]{{{P} \scong {P}'} \andalso {{P}' \red {Q}'} \andalso {{Q}' \scong {Q}}}{{P} \red {Q}}
\end{mathpar}

\begin{eqnarray*}
  match_{\equiv} (\quotep{P},\quotep{Q}) & := & P \equiv Q \\
  match_{\dagger}(\quotep{P},\quotep{Q}) & := & \forall R. P|Q \red^{*} R => R \red^{*} 0 \\
  match_{K}(\quotep{P},\quotep{Q}) & := & K \mbox{ for some context } K
\end{eqnarray*}

$u?(x)P | u!\langle Q \rangle \red P\{\quotep{Q}/x\}$

%We write $\wred$ for $\red^*$, and $P\red$ if $\exists Q $ such that $ P \red Q$.
We write $P\red$ if $\exists Q $ such that $ P \red Q$ and $P\not\red$, otherwise.

\section{Replication}

As mentioned before, it is known that replication (and hence
recursion) can be implemented in a higher-order process algebra
\cite{SangiorgiWalker}. As our first example of calculation with the
machinery thus far presented we give the construction explicitly in
the {\rhoc}.

\begin{eqnarray}
	D_{x} & := & \prefix{x}{y}{(\binpar{\outputp{x}{y}}{@{y}})} \nonumber\\
	\bangp_{x}{P} & := & \binpar{{x}!\langle{\binpar{D_{x}}{P}}\rangle}{D_{x}} \nonumber
\end{eqnarray}

\begin{eqnarray}
	\bangp_{x}{P} & & \nonumber\\
	=
	& {x}!\langle{(\prefix{x}{y}{(\outputp{x}{y} | @{y})) | P}}\rangle 
	      | \prefix{x}{y}{(\outputp{x}{y} | @{y})} & \nonumber\\
	\red
	& (\outputp{x}{y} | @{y})\substn{\quotep{(\prefix{x}{y}{(@{y} | \outputp{x}{y})) | P}}}{y} & \nonumber\\
	=
	& \outputp{x}{\quotep{(\prefix{x}{y}{(\outputp{x}{y} | @{y})) | P}}}
	  | {(\prefix{x}{y}{(\outputp{x}{y} | @{y})) | P}} & \nonumber\\
	\red
	& \ldots & \nonumber\\
	\red^*
	& P | P | \ldots & \nonumber
\end{eqnarray}

Of course, this encoding, as an implementation, runs away, unfolding
$\bangp{P}$ eagerly. A lazier and more implementable replication
operator, restricted to input-guarded processes, may be obtained as follows.

\begin{eqnarray}
\bangp{\prefix{u}{v}{P}} 
	:= 
	\binpar{\lift{x}{\prefix{u}{v}{(\binpar{D(x)}{P})}}}{D(x)} \nonumber
\end{eqnarray}

\begin{remark}
  Note that the lazier definition still does not deal with summation
  or mixed summation (i.e. sums over input and output). The reader is
  invited to construct definitions of replication that deal with these
  features. 

  Further, the definitions are parameterized in a name, $x$. Can you,
  gentle reader, make a definition that eliminates this parameter and
  guarantees no accidental interaction between the replication
  machinery and the process being replicated -- i.e. no accidental
  sharing of names used by the process to get its work done and the
  name(s) used by the replication to effect copying. This latter
  revision of the definition of replication is crucial to obtaining
  the expected identity $!!P \sim !P$.
\end{remark}

\begin{remark}\label{rem:paradoxical_combinator}
  The reader familiar with the lambda calculus will have noticed the
  similarity between $D$ and the paradoxical combinator.

  [Ed. note: the existence of this seems to suggest we have to be more
  restrictive on the set of processes and names we admit if we are to
  support no-cloning.]
\end{remark}

\subsubsection{Bisimulation}

The computational dynamics gives rise to another kind of equivalence,
the equivalence of computational behavior. As previously mentioned
this is typically captured \emph{via} some form of bisimulation.

% The notion we use in this paper is weak barbed bisimulation
% \cite{milner91polyadicpi}.

The notion we use in this paper is derived from weak barbed
bisimulation \cite{milner91polyadicpi}. 

\begin{definition}
An \emph{observation relation}, $\downarrow_{\mathcal N}$, over a set
of names, $\mathcal N$, is the smallest relation satisfying the rules
below.

\infrule[Out-barb]{y \in {\mathcal N}, \; x \nameeq y}
		  {\outputp{x}{v} \downarrow_{\mathcal N} x}
\infrule[Par-barb]{\mbox{$P\downarrow_{\mathcal N} x$ or $Q\downarrow_{\mathcal N} x$}}
		  {\binpar{P}{Q} \downarrow_{\mathcal N} x}

We write $P \Downarrow_{\mathcal N} x$ if there is $Q$ such that 
$P \wred Q$ and $Q \downarrow_{\mathcal N} x$.
\end{definition}

\begin{definition}
%\label{def.bbisim}
An  ${\mathcal N}$-\emph{barbed bisimulation} over a set of names, ${\mathcal N}$, is a symmetric binary relation 
${\mathcal S}_{\mathcal N}$ between agents such that $P\rel{S}_{\mathcal N}Q$ implies:
\begin{enumerate}
\item If $P \red P'$ then $Q \wred Q'$ and $P'\rel{S}_{\mathcal N} Q'$.
\item If $P\downarrow_{\mathcal N} x$, then $Q\Downarrow_{\mathcal N} x$.
\end{enumerate}
$P$ is ${\mathcal N}$-barbed bisimilar to $Q$, written
$P \wbbisim_{\mathcal N} Q$, if $P \rel{S}_{\mathcal N} Q$ for some ${\mathcal N}$-barbed bisimulation ${\mathcal S}_{\mathcal N}$.
\end{definition}

$\mathcal{R} \subseteq \pi \times \pi$

$P \mathcal{R} Q => \forall P'. P \red P' \Rightarrow \exists Q'. Q \red Q', P' \mathcal{R} Q'$

$P \vdash x \Rightarrow Q \vdash x$

\begin{mathpar}
  \inferrule*[lab=Out-barb]{x \nameeq y}{{y}!\langle{Q}\rangle \vdash x}
  \and
  \inferrule*[lab=Par-barb]{\mbox{$P\vdash x$ or $Q\vdash x$}}{\binpar{P}{Q} \vdash x}
\end{mathpar}

\subsubsection{Contexts}

One of the principle advantages of computational calculi like the
$\pi$-calculus is a well-defined notion of context,
contextual-equivalence and a correlation between
contextual-equivalence and notions of bisimulation. The notion of
context allows the decomposition of a process into (sub-)process and
its syntactic environment, its context. Thus, a context may be
thought of as a process with a ``hole'' (written $\Box$) in it. The
application of a context $M$ to a process $P$, written $M[P]$, is
tantamount to filling the hole in $M$ with $P$. In this paper we do
not need the full weight of this theory, but do make use of the notion
of context in the proof the main theorem. 

\begin{mathpar}
  \inferrule* [lab=summation] {} {{M_{M},M_{N}} \bc \Box \;|\; x.M_{A} \;|\; M_{M}+M_{N}}
  \and
  \inferrule* [lab=agent] {} {{M_{A}} \bc (\vec{x})M_{P} \;| \; \clift{P_0,\ldots,M_{P},\ldots,P_N}}
  \and \\
  \inferrule* [lab=process] {} {{M_{P}} \bc M_{N} \;| \;P|M_{P} }
\end{mathpar} 

\begin{mathpar}
  \inferrule* [lab=sychronization] {} {M_{N} \bc \Box \;|\; x?M_{F} \;|\; x!M_{C}}
  \and
  \inferrule* [lab=abstraction] {} {{M_{F}} \bc (x)M_{P} }
  \and
  \inferrule* [lab=concretion] {} {{M_{C}} \bc \langle M_{P} \rangle }
  \and \\
  \inferrule* [lab=process] {} {{M_{P}} \bc M_{N} \;| \;P|M_{P} }
\end{mathpar}

\begin{definition}[contextual application] Given a context $M$, and
  process $P$, we define the \emph{contextual application}, $M[P] :=
  M\{P/\Box\}$. That is, the contextual application of M to P is the
  substitution of $P$ for $\Box$ in $M$.
\end{definition}

$\meaningof{-} : L \to \mathcal{P}(\pi)$

\begin{mathpar}
  \inferrule* [lab=collection] {} {\meaningof{true} = \pi, \and \meaningof{~E} = \pi \setminus \meaningof{E}, \and \meaningof{E_{1} \& E_{2}} = \meaningof{E_{1}} \cap \meaningof{E_{2}}}
\end{mathpar}

\begin{mathpar}
  \inferrule* [lab=structure] {} {\meaningof{0} = \{ P \in \pi | P \equiv 0 \}, \and \\ \meaningof{E_1 | E_2} = \{ P \in \pi | P \equiv P_{1} | P_{2}, P_{1} \in \meaningof{E_{1}}, P_{2} \in \meaningof{E_2}\} }
\end{mathpar}

\begin{mathpar}
 \inferrule* [lab=behavior] {} {\meaningof{\langle a?b \rangle E} = \{ P \in \pi | P \equiv Q | u?(y)P', \\ \and \\\\ \and \\ \;\;\; u \in \meaningof{a}, \forall z.P'\{z/y\} \in \meaningof{E\{z/b\}}\}, \and \\ \meaningof{a!E} = \{ P \in \pi | P \equiv Q | x!\langle P' \rangle, x \in \meaningof{a} P' \in \meaningof{E}\} }
\end{mathpar}

\begin{mathpar}
 \inferrule* [lab=nominal] {} {\meaningof{\quotep{E}} = \{ \quotep{P} \in \quotep{\pi} | P \in \meaningof{E} \}, \and \meaningof{\quotep{P}} = \{ \quotep{Q} \in \quotep{\pi} | P \equiv Q \} \and \\ \meaningof{@\quotep{E}} = \{ P \in \pi | P \equiv @x, x \in \meaningof{E} \}}
\end{mathpar}

\begin{eqnarray*}
  \\
  \meaningof{-} : TS \to ST
\end{eqnarray*}

\begin{eqnarray*}
  \\
  L : TS \to ST
\end{eqnarray*}

\begin{eqnarray*}
  \\
  P \models E \iff P \in \meaningof{E}
\end{eqnarray*}

\begin{eqnarray*}
  P \approx_{L} Q \iff \forall E \in L. P \models E \iff Q \models E
\end{eqnarray*}

\begin{eqnarray*}
  P \approx_{K} Q
\end{eqnarray*}

\begin{eqnarray*}
  P \approx Q
\end{eqnarray*}

$\approx_{K} = \approx = \approx_{L}$

\subsubsection{Contextual duality}

Note that contexts extend the quotation operation to a family of
operations from processes to names. Given a context, $M$, we can
define a \emph{nominal context}, $\quotep{M}$ by $\quotep{M}[P] :=
\quotep{M[P]}$. To foreshadow what is to come we observe that these
operations enjoy a duality with processes very much like the duality
between vectors and maps from vectors to scalars.

Further, because the calculus is essentially higher-order, we have a
correspondence between contexts and processes. More specifically,
given a name $x$ and a context $M$ we can construct $M^{*}_{x}$ such
that 

\begin{mathpar}
  M^{*}_{x} | \lift{x}{P} \red M[P]
\end{mathpar}

namely,

\begin{mathpar}
  M^{*}_{x} := x?(u).M[\dropn{u}]
\end{mathpar}

The dependence of $M^{*}_{x}$ on a name makes it an abstraction, 

\begin{mathpar}
  M^{*} := (x)x?(u).M[\dropn{u}]
\end{mathpar}

\subsection{Additional notation}

It will sometimes be convenient to denote the process a name
quotes. We already have the notation $x = \quotep{P}$, but it will be
convenient to introduce an alternate notation, $\procn{x}$, when we
want to emphasize the connection to the use of the name. Note that, by
virtue of name equivalence, $\quotep{\procn{x}} \nameeq x$; so, the
notation is consistent with previous definitions.

Further, because names have structure it is possible to effect
substitutions on the basis of that structure. This means we need to
upgrade our notation for substitutions, which we accomplish by
adapting comprehension notation. Thus,

\begin{mathpar}
  P\{ y / x : x \in S \}
\end{mathpar}

is interpreted to mean the process derived from P by replacing (in a
capture-avoiding manner) each occurrence of $x$ in $S$ by $y$. For example,

\begin{mathpar}
  P\{ \quotep{\procn{x}|\procn{x}} / x : x \in \freenames{P} \}
\end{mathpar}

will replace each (occurrence) of a free name $x$ in $P$ by
$\quotep{\procn{x}|\procn{x}}$.

Also, we will avail ourselves of the notation $x^{L}$ and $x^{R}$ to
denote injections of a name into disjoint copies of the name
space. There are numerous ways to accomplish this. One example can be
found in \cite{MeredithR05}. This notation overloads to vectors of
names: $\vec{x}^{\pi} := (x_{i}^{\pi} \; : \; 0 \leq i < |\vec{x}| )$ where $\pi \in \{L,R\}$.

We also use $P^{\Box} := P|\Box$.

In \cite{MeredithR05} an interpretation of the new operator is
given. It turns out that there are several possible interpretations
all enjoying the requisite algebraic properties of the operator (see
\cite{milner91polyadicpi}). We will therefore make liberal use of
$(\nu\; \vec{x})P$.

% subsection the_syntax_and_semantics_of_the_notation_system (end)   

\input{qm2pi.qmops} 

\input{qm2pi.sterngerlach} 

\input{qm2pi.metric} 

% section concurrent_process_calculi (end)

%\input{qm2pi.proofsketch}

% section proof sketch (end)

%\input{qm2pi.slviaknots} 

% section spatial logic via knots (end)

\input{qm2pi.conclusion}

% section conclusion (end)

%\input{qm2pi.dtcodes} 

% section wiring algorithm (end)

\input{qm2pi.ack} 

% section acknowledgments (end)

\newpage


\bibliographystyle{plain}   
\bibliography{../../biblios/main.bib}

\input{qm2pi.rhodetails}

\end{document}

 

\documentclass[12pt]{llncs}
%\documentclass{jktr}

\usepackage[pdftex]{hyperref}                   
\usepackage {listings}
\usepackage {mathpartir}
\usepackage{bcprules}
%\usepackage{listings}
                       
\usepackage{graphicx} 
%\usepackage[margins=2.5cm,nohead,nofoot]{geometry}
%\usepackage{geometry}
\usepackage{amsfonts}
\usepackage{amstext}
\usepackage{latexsym}
\usepackage{amssymb}
\usepackage{color}


%\include{myPreamble}
\include{qm2pi.local} 

%\ifpdf
%\usepackage[pdftex]{graphicx}
%\else
%\usepackage{graphicx}
%\fi

 % \ifpdf
%  \usepackage{pdfsync}
%  \if


%\title{Brief Article}
%\author{David F. Snyder}
%\author{L.G. Meredith}

%\address{Dept. of Math., Texas State University--San Marcos, San Marcos, TX 78666}
       
\pagestyle{empty}


\begin{document}

\lstset{language=[Objective]Caml,frame=shadowbox}

\input{qm2pi.front}

% section front matter (end)

\input{qm2pi.intro} 
 
% section introduction (end)

% \input{qm2pi.knotations} 

% section notation (end)

\input{qm2pi.process.calculi} 

% section concurrent_process_calculi_and_spatial_logics_ (end)
    
%\input{qm2pi.knots2pi} 

%\input{qm2pi.trefoil} 

%\input{qm2pi.mainthm} 

% subsection basic_interpretation (end)

%\input{qm2pi.rho.presentation} 
\subsection{The syntax and semantics of the notation system}\label{sub:the_syntax_and_semantics_of_the_notation_system} % (fold)

We now summarize a technical presentation of the calculus that
embodies our theory of dynamics. The typical presentation of such a
calculus follows the style of giving generators and relations on
them. The grammar, below, describing term constructors, freely
generates the set of processes, $\Proc$. This set is then quotiented
by a relation known as structural congruence and it is over this set
that the notion of dynamics is expressed. This presentation is
essentially that of \cite{MeredithR05} with the addition of
polyadicity and summation. For readability we have relegated some of
the technical subtleties to an appendix.

\subsubsection{Process grammar}\label{subsub:process_grammar}

\begin{mathpar}
  \inferrule* [lab=synchronization] {} {{M} \bc \pzero \;|\; x?F \;|\; x!C }
  \and
  \inferrule* [lab=abstraction] {} {{F} \bc (x)P}
  \and
  \inferrule* [lab=concretion] {} {{C} \bc \langle Q \rangle}
  \and
  \inferrule* [lab=process] {} {{P,Q} \bc M \;| \;P|Q \;|\; @{x}}
  \and
  \inferrule* [lab=name] {} {{x} \bc \quotep{P}}
\end{mathpar} 

Note that $\vec{x}$ (resp. $\vec{P}$) denotes a vector of names
(resp. processes) of length $|\vec{x}|$ (resp. $|\vec{P}|$). We adopt
the following useful abbreviations.

\begin{mathpar}
   x?(\vec{y}).P := x.(\vec{y})P \and  x\clift{\vec{P}} := x.\clift{\vec{P}}
   \and x!(y) := \lift{x}{\dropn{y}}
   \and \Pi_{i=0}^{n-1}P_i := P_0 | \ldots | P_{n-1}
\end{mathpar}

\subsubsection{Structural congruence}

\paragraph{Free and bound names and alpha-equivalence.} At the
core of structural equivalence is alpha-equivalence which identifies
process that are the same up to a change of variable. Formally, we
recognize the distinction between free and bound names. The free names
of a process, $\freenames{P}$, may be calculated recursively as
follows:

\begin{mathpar}
\freenames{\pzero} := \emptyset
  \and \\
  \freenames{x?(y).P} := \{ x \} \cup (\freenames{P} \setminus \{ y \})
  \and 
  \freenames{x!\langle P \rangle} := \{ x \} \cup \{ P \} 
  \and \\
  \freenames{P|Q} := \freenames{P} \cup \freenames{Q}
  \and \\
  \freenames{@{x}} := \{ x \}
\end{mathpar}

$\pi$
$\quotep{\pi}$

$\freenames{-} : \pi \to \mathcal{P}(\quotep{\pi})$

\begin{eqnarray*}
  \freenames{\pzero} & := & \emptyset \\
  \freenames{x?(y).P} & := & \{ x \} \cup (\freenames{P} \setminus \{ y \}) \\
  \freenames{x!\langle P \rangle} & := & \{ x \} \cup \{ P \} \\
  \freenames{P|Q} & := & \freenames{P} \cup \freenames{Q} \\
  \freenames{\dropn{x}} & := & \{ x \}
\end{eqnarray*}

The bound names of a process, $\boundnames{P}$, are those names occurring in $P$
that are not free. For example, in $x?(y).0$, the name $x$ is free, while $y$ is bound.

\begin{mathpar}
  \inferrule* [lab=monoidal-laws] {} { P|Q \equiv Q|P \and P|0 \equiv P \and P|(Q|R) \equiv (P|Q)|R }
\end{mathpar}

\begin{mathpar}
  \inferrule* [lab=alpha-equivalence] {} { (x)P \equiv (y)P\{y/x\} \and y \not\in \freenames{P} }
\end{mathpar}

\begin{definition}
Then two processes, $P,Q$, are alpha-equivalent if $P = Q\{\vec{y}/\vec{x}\}$ for
some $\vec{x} \in \boundnames{Q},\vec{y} \in \boundnames{P}$, where $Q\{\vec{y}/\vec{x}\}$
denotes the capture-avoiding substitution of $\vec{y}$ for $\vec{x}$ in $Q$.
\end{definition}

\begin{definition}
  The {\em structural congruence} \cite{SangiorgiWalker} , $\equiv$,
  between processes is the least congruence containing
  alpha-equivalence, satisfying the abelian monoid laws
  (associativity, commutativity and $\pzero$ as identity) for parallel
  composition $|$ and for summation $+$.
\end{definition}

\subsection{Name equivalence}

We take name equivalence, written $\nameeq$, to be the smallest
equivalence relation generated by the following rules.

\begin{mathpar}
\inferrule*[lab=Quote-drop]
{ }
{ \quotep{@{x}} \nameeq x }

\inferrule*[lab=Struct-equiv]
{ P \scong Q }
{ \quotep{P} \nameeq \quotep{Q} }
\end{mathpar}

The astute reader will have noticed that the mutual recursion of names
and processes imposes a mutual recursion on alpha-equivalence and
structural equivalence via name-equivalence. Fortunately, all of this
works out pleasantly and we may calculate in the natural way, free of
concern. The reader interested in the details is referred to the
appendix \ref{appendix:rho_details}.

\subsection{Substitution}

We use $\Proc$ for the set of processes, $\QProc$ for the set of
names, and $\id{\{}\vec{y} / \vec{x} \id{\}}$ to denote partial maps,
$s : \QProc \rightarrow \QProc$. A map, $s$ lifts, uniquely, to a map
on process terms, $\widehat{s} : \Proc \rightarrow \Proc$ by the
following equations.

\begin{mathpar}
  (0) \psubstp{Q}{P} := 0 \\
  (R \juxtap S) \psubstp{Q}{P}
  :=    
  (R)\psubstp{Q}{P} \juxtap (S) \psubstp{Q}{P} \\
  (x?(y).R) \psubstp{Q}{P}    
  :=    
  (x)\substp{Q}{P} (z)\concat( (R \psubstn{z}{y}) \psubstp{Q}{P} ) \\
  (\lift{x}{R}) \psubstp{Q}{P}  
  :=
  \lift{(x)\substp{Q}{P}}{ R \psubstp{Q}{P} } \\
%   (\dropn{x})  \psubstp{Q}{P}       
%   := 
%   \left\{ 
%     \begin{array}{ccc} 
%       \dropn{\quotep{Q}} & & x \nameeq \quotep{P} \\
%       \dropn{x} & & otherwise \\
%     \end{array}
%   \right. 
  (\dropn{x})  \psubstp{Q}{P}       
  := 
  \left\{ 
    \begin{array}{ccc} 
      Q & & x \nameeq \quotep{P} \\
      \dropn{x} & & otherwise \\
    \end{array}
  \right.
\end{mathpar}
 

where

\begin{eqnarray}
  (x)\id{\{} \lpquote Q \rpquote / \lpquote P \rpquote \id{\}}            = 
  \left\{ 
    \begin{array}{ccc}
      \lpquote Q \rpquote & & x \nameeq \lpquote P \rpquote \\
      x & & otherwise \\
    \end{array}
  \right. \nonumber
\end{eqnarray}

and $z$ is chosen distinct from $\quotep{P}$, $\quotep{Q}$, the free
names in $Q$, and all the names in $R$. Our $\alpha$-equivalence will
be built in the standard way from this substitution.

\begin{remark}\label{rem:no_self_referential_names}
  One consequence of these definitions is that $\forall P. \quotep{P}
  \not\in \freenames{P}$.
\end{remark}

\subsection{ Dynamic quote: an example }

Anticipating something of what's to come, consider applying the
substitution, $\widehat{\id{\{}u / z \id{\}}}$, to the following pair
of processes, $\lift{w}{y!(z)}$ and $w[ \lpquote y!(z) \rpquote ]$.

\begin{eqnarray}
	\lift{w}{y!(z)}\widehat{\id{\{}u / z \id{\}}}
		& = &
		\lift{w}{y!(u)} \nonumber\\
	w[ \lpquote y!(z) \rpquote ] \widehat{ \id{\{}u / z \id{\}} }
		& = &
		w[ \lpquote y!(z) \rpquote ] \nonumber
\end{eqnarray}

Because the body of the process between quotes is impervious to
substitution, we get radically different answers. In fact, by
examining the first process in an input context,
e.g. $x?(z).\lift{w}{y!(z)}$, we see that the process under the lift
operator may be shaped by prefixed inputs binding a name inside it. In
this sense, the lift operator will be seen as a way to dynamically
construct processes before reifying them as names.

Finally equipped with these standard features we can present the
dynamics of the calculus.

\subsubsection{Operational semantics} 

Finally, we introduce the computational dynamics. What marks these
algebras as distinct from other more traditionally studied algebraic
structures, e.g. vector spaces or polynomial rings, is the manner in
which dynamics is captured. In traditional structures, dynamics is typically
expressed through morphisms between such structures, as in linear maps
between vector spaces or morphisms between rings. In algebras
associated with the semantics of computation, the dynamics is
expressed as part of the algebraic structure itself, through a
reduction reduction relation typically denoted by $\red$. Below, we
give a recursive presentation of this relation for the calculus used
in the encoding.

$\red \subseteq \pi \times \pi$
$\red : \pi \to \mathcal{P}(\pi)$

\begin{mathpar}
  \inferrule* [lab=Comm] { \textsf{match}( x_{src}, x_{trgt} ) } { x_{trgt}?(y)P \; | \; x_{src}!\langle {Q} \rangle \red P\{\quotep{Q}/y}\} }
  \and \\
  \inferrule* [lab=Par] {{P} \red {P}'} {{{P} | {Q}} \red {{P}' | {Q}}}
  \and
  \inferrule* [lab=Equiv]{{{P} \scong {P}'} \andalso {{P}' \red {Q}'} \andalso {{Q}' \scong {Q}}}{{P} \red {Q}}
\end{mathpar}

\begin{eqnarray*}
  match_{\equiv} (\quotep{P},\quotep{Q}) & := & P \equiv Q \\
  match_{\dagger}(\quotep{P},\quotep{Q}) & := & \forall R. P|Q \red^{*} R => R \red^{*} 0 \\
  match_{K}(\quotep{P},\quotep{Q}) & := & K \mbox{ for some context } K
\end{eqnarray*}

$u?(x)P | u!\langle Q \rangle \red P\{\quotep{Q}/x\}$

%We write $\wred$ for $\red^*$, and $P\red$ if $\exists Q $ such that $ P \red Q$.
We write $P\red$ if $\exists Q $ such that $ P \red Q$ and $P\not\red$, otherwise.

\section{Replication}

As mentioned before, it is known that replication (and hence
recursion) can be implemented in a higher-order process algebra
\cite{SangiorgiWalker}. As our first example of calculation with the
machinery thus far presented we give the construction explicitly in
the {\rhoc}.

\begin{eqnarray}
	D_{x} & := & \prefix{x}{y}{(\binpar{\outputp{x}{y}}{@{y}})} \nonumber\\
	\bangp_{x}{P} & := & \binpar{{x}!\langle{\binpar{D_{x}}{P}}\rangle}{D_{x}} \nonumber
\end{eqnarray}

\begin{eqnarray}
	\bangp_{x}{P} & & \nonumber\\
	=
	& {x}!\langle{(\prefix{x}{y}{(\outputp{x}{y} | @{y})) | P}}\rangle 
	      | \prefix{x}{y}{(\outputp{x}{y} | @{y})} & \nonumber\\
	\red
	& (\outputp{x}{y} | @{y})\substn{\quotep{(\prefix{x}{y}{(@{y} | \outputp{x}{y})) | P}}}{y} & \nonumber\\
	=
	& \outputp{x}{\quotep{(\prefix{x}{y}{(\outputp{x}{y} | @{y})) | P}}}
	  | {(\prefix{x}{y}{(\outputp{x}{y} | @{y})) | P}} & \nonumber\\
	\red
	& \ldots & \nonumber\\
	\red^*
	& P | P | \ldots & \nonumber
\end{eqnarray}

Of course, this encoding, as an implementation, runs away, unfolding
$\bangp{P}$ eagerly. A lazier and more implementable replication
operator, restricted to input-guarded processes, may be obtained as follows.

\begin{eqnarray}
\bangp{\prefix{u}{v}{P}} 
	:= 
	\binpar{\lift{x}{\prefix{u}{v}{(\binpar{D(x)}{P})}}}{D(x)} \nonumber
\end{eqnarray}

\begin{remark}
  Note that the lazier definition still does not deal with summation
  or mixed summation (i.e. sums over input and output). The reader is
  invited to construct definitions of replication that deal with these
  features. 

  Further, the definitions are parameterized in a name, $x$. Can you,
  gentle reader, make a definition that eliminates this parameter and
  guarantees no accidental interaction between the replication
  machinery and the process being replicated -- i.e. no accidental
  sharing of names used by the process to get its work done and the
  name(s) used by the replication to effect copying. This latter
  revision of the definition of replication is crucial to obtaining
  the expected identity $!!P \sim !P$.
\end{remark}

\begin{remark}\label{rem:paradoxical_combinator}
  The reader familiar with the lambda calculus will have noticed the
  similarity between $D$ and the paradoxical combinator.

  [Ed. note: the existence of this seems to suggest we have to be more
  restrictive on the set of processes and names we admit if we are to
  support no-cloning.]
\end{remark}

\subsubsection{Bisimulation}

The computational dynamics gives rise to another kind of equivalence,
the equivalence of computational behavior. As previously mentioned
this is typically captured \emph{via} some form of bisimulation.

% The notion we use in this paper is weak barbed bisimulation
% \cite{milner91polyadicpi}.

The notion we use in this paper is derived from weak barbed
bisimulation \cite{milner91polyadicpi}. 

\begin{definition}
An \emph{observation relation}, $\downarrow_{\mathcal N}$, over a set
of names, $\mathcal N$, is the smallest relation satisfying the rules
below.

\infrule[Out-barb]{y \in {\mathcal N}, \; x \nameeq y}
		  {\outputp{x}{v} \downarrow_{\mathcal N} x}
\infrule[Par-barb]{\mbox{$P\downarrow_{\mathcal N} x$ or $Q\downarrow_{\mathcal N} x$}}
		  {\binpar{P}{Q} \downarrow_{\mathcal N} x}

We write $P \Downarrow_{\mathcal N} x$ if there is $Q$ such that 
$P \wred Q$ and $Q \downarrow_{\mathcal N} x$.
\end{definition}

\begin{definition}
%\label{def.bbisim}
An  ${\mathcal N}$-\emph{barbed bisimulation} over a set of names, ${\mathcal N}$, is a symmetric binary relation 
${\mathcal S}_{\mathcal N}$ between agents such that $P\rel{S}_{\mathcal N}Q$ implies:
\begin{enumerate}
\item If $P \red P'$ then $Q \wred Q'$ and $P'\rel{S}_{\mathcal N} Q'$.
\item If $P\downarrow_{\mathcal N} x$, then $Q\Downarrow_{\mathcal N} x$.
\end{enumerate}
$P$ is ${\mathcal N}$-barbed bisimilar to $Q$, written
$P \wbbisim_{\mathcal N} Q$, if $P \rel{S}_{\mathcal N} Q$ for some ${\mathcal N}$-barbed bisimulation ${\mathcal S}_{\mathcal N}$.
\end{definition}

$\mathcal{R} \subseteq \pi \times \pi$

$P \mathcal{R} Q => \forall P'. P \red P' \Rightarrow \exists Q'. Q \red Q', P' \mathcal{R} Q'$

$P \vdash x \Rightarrow Q \vdash x$

\begin{mathpar}
  \inferrule*[lab=Out-barb]{x \nameeq y}{{y}!\langle{Q}\rangle \vdash x}
  \and
  \inferrule*[lab=Par-barb]{\mbox{$P\vdash x$ or $Q\vdash x$}}{\binpar{P}{Q} \vdash x}
\end{mathpar}

\subsubsection{Contexts}

One of the principle advantages of computational calculi like the
$\pi$-calculus is a well-defined notion of context,
contextual-equivalence and a correlation between
contextual-equivalence and notions of bisimulation. The notion of
context allows the decomposition of a process into (sub-)process and
its syntactic environment, its context. Thus, a context may be
thought of as a process with a ``hole'' (written $\Box$) in it. The
application of a context $M$ to a process $P$, written $M[P]$, is
tantamount to filling the hole in $M$ with $P$. In this paper we do
not need the full weight of this theory, but do make use of the notion
of context in the proof the main theorem. 

\begin{mathpar}
  \inferrule* [lab=summation] {} {{M_{M},M_{N}} \bc \Box \;|\; x.M_{A} \;|\; M_{M}+M_{N}}
  \and
  \inferrule* [lab=agent] {} {{M_{A}} \bc (\vec{x})M_{P} \;| \; \clift{P_0,\ldots,M_{P},\ldots,P_N}}
  \and \\
  \inferrule* [lab=process] {} {{M_{P}} \bc M_{N} \;| \;P|M_{P} }
\end{mathpar} 

\begin{mathpar}
  \inferrule* [lab=sychronization] {} {M_{N} \bc \Box \;|\; x?M_{F} \;|\; x!M_{C}}
  \and
  \inferrule* [lab=abstraction] {} {{M_{F}} \bc (x)M_{P} }
  \and
  \inferrule* [lab=concretion] {} {{M_{C}} \bc \langle M_{P} \rangle }
  \and \\
  \inferrule* [lab=process] {} {{M_{P}} \bc M_{N} \;| \;P|M_{P} }
\end{mathpar}

\begin{definition}[contextual application] Given a context $M$, and
  process $P$, we define the \emph{contextual application}, $M[P] :=
  M\{P/\Box\}$. That is, the contextual application of M to P is the
  substitution of $P$ for $\Box$ in $M$.
\end{definition}

$\meaningof{-} : L \to \mathcal{P}(\pi)$

\begin{mathpar}
  \inferrule* [lab=collection] {} {\meaningof{true} = \pi, \and \meaningof{~E} = \pi \setminus \meaningof{E}, \and \meaningof{E_{1} \& E_{2}} = \meaningof{E_{1}} \cap \meaningof{E_{2}}}
\end{mathpar}

\begin{mathpar}
  \inferrule* [lab=structure] {} {\meaningof{0} = \{ P \in \pi | P \equiv 0 \}, \and \\ \meaningof{E_1 | E_2} = \{ P \in \pi | P \equiv P_{1} | P_{2}, P_{1} \in \meaningof{E_{1}}, P_{2} \in \meaningof{E_2}\} }
\end{mathpar}

\begin{mathpar}
 \inferrule* [lab=behavior] {} {\meaningof{\langle a?b \rangle E} = \{ P \in \pi | P \equiv Q | u?(y)P', \\ \and \\\\ \and \\ \;\;\; u \in \meaningof{a}, \forall z.P'\{z/y\} \in \meaningof{E\{z/b\}}\}, \and \\ \meaningof{a!E} = \{ P \in \pi | P \equiv Q | x!\langle P' \rangle, x \in \meaningof{a} P' \in \meaningof{E}\} }
\end{mathpar}

\begin{mathpar}
 \inferrule* [lab=nominal] {} {\meaningof{\quotep{E}} = \{ \quotep{P} \in \quotep{\pi} | P \in \meaningof{E} \}, \and \meaningof{\quotep{P}} = \{ \quotep{Q} \in \quotep{\pi} | P \equiv Q \} \and \\ \meaningof{@\quotep{E}} = \{ P \in \pi | P \equiv @x, x \in \meaningof{E} \}}
\end{mathpar}

\begin{eqnarray*}
  \\
  \meaningof{-} : TS \to ST
\end{eqnarray*}

\begin{eqnarray*}
  \\
  L : TS \to ST
\end{eqnarray*}

\begin{eqnarray*}
  \\
  P \models E \iff P \in \meaningof{E}
\end{eqnarray*}

\begin{eqnarray*}
  P \approx_{L} Q \iff \forall E \in L. P \models E \iff Q \models E
\end{eqnarray*}

\begin{eqnarray*}
  P \approx_{K} Q
\end{eqnarray*}

\begin{eqnarray*}
  P \approx Q
\end{eqnarray*}

$\approx_{K} = \approx = \approx_{L}$

\subsubsection{Contextual duality}

Note that contexts extend the quotation operation to a family of
operations from processes to names. Given a context, $M$, we can
define a \emph{nominal context}, $\quotep{M}$ by $\quotep{M}[P] :=
\quotep{M[P]}$. To foreshadow what is to come we observe that these
operations enjoy a duality with processes very much like the duality
between vectors and maps from vectors to scalars.

Further, because the calculus is essentially higher-order, we have a
correspondence between contexts and processes. More specifically,
given a name $x$ and a context $M$ we can construct $M^{*}_{x}$ such
that 

\begin{mathpar}
  M^{*}_{x} | \lift{x}{P} \red M[P]
\end{mathpar}

namely,

\begin{mathpar}
  M^{*}_{x} := x?(u).M[\dropn{u}]
\end{mathpar}

The dependence of $M^{*}_{x}$ on a name makes it an abstraction, 

\begin{mathpar}
  M^{*} := (x)x?(u).M[\dropn{u}]
\end{mathpar}

\subsection{Additional notation}

It will sometimes be convenient to denote the process a name
quotes. We already have the notation $x = \quotep{P}$, but it will be
convenient to introduce an alternate notation, $\procn{x}$, when we
want to emphasize the connection to the use of the name. Note that, by
virtue of name equivalence, $\quotep{\procn{x}} \nameeq x$; so, the
notation is consistent with previous definitions.

Further, because names have structure it is possible to effect
substitutions on the basis of that structure. This means we need to
upgrade our notation for substitutions, which we accomplish by
adapting comprehension notation. Thus,

\begin{mathpar}
  P\{ y / x : x \in S \}
\end{mathpar}

is interpreted to mean the process derived from P by replacing (in a
capture-avoiding manner) each occurrence of $x$ in $S$ by $y$. For example,

\begin{mathpar}
  P\{ \quotep{\procn{x}|\procn{x}} / x : x \in \freenames{P} \}
\end{mathpar}

will replace each (occurrence) of a free name $x$ in $P$ by
$\quotep{\procn{x}|\procn{x}}$.

Also, we will avail ourselves of the notation $x^{L}$ and $x^{R}$ to
denote injections of a name into disjoint copies of the name
space. There are numerous ways to accomplish this. One example can be
found in \cite{MeredithR05}. This notation overloads to vectors of
names: $\vec{x}^{\pi} := (x_{i}^{\pi} \; : \; 0 \leq i < |\vec{x}| )$ where $\pi \in \{L,R\}$.

We also use $P^{\Box} := P|\Box$.

In \cite{MeredithR05} an interpretation of the new operator is
given. It turns out that there are several possible interpretations
all enjoying the requisite algebraic properties of the operator (see
\cite{milner91polyadicpi}). We will therefore make liberal use of
$(\nu\; \vec{x})P$.

% subsection the_syntax_and_semantics_of_the_notation_system (end)   

\input{qm2pi.qmops} 

\input{qm2pi.sterngerlach} 

\input{qm2pi.metric} 

% section concurrent_process_calculi (end)

%\input{qm2pi.proofsketch}

% section proof sketch (end)

%\input{qm2pi.slviaknots} 

% section spatial logic via knots (end)

\input{qm2pi.conclusion}

% section conclusion (end)

%\input{qm2pi.dtcodes} 

% section wiring algorithm (end)

\input{qm2pi.ack} 

% section acknowledgments (end)

\newpage


\bibliographystyle{plain}   
\bibliography{../../biblios/main.bib}

\input{qm2pi.rhodetails}

\end{document}

 

% section concurrent_process_calculi (end)

%\documentclass[12pt]{llncs}
%\documentclass{jktr}

\usepackage[pdftex]{hyperref}                   
\usepackage {listings}
\usepackage {mathpartir}
\usepackage{bcprules}
%\usepackage{listings}
                       
\usepackage{graphicx} 
%\usepackage[margins=2.5cm,nohead,nofoot]{geometry}
%\usepackage{geometry}
\usepackage{amsfonts}
\usepackage{amstext}
\usepackage{latexsym}
\usepackage{amssymb}
\usepackage{color}


%\include{myPreamble}
\include{qm2pi.local} 

%\ifpdf
%\usepackage[pdftex]{graphicx}
%\else
%\usepackage{graphicx}
%\fi

 % \ifpdf
%  \usepackage{pdfsync}
%  \if


%\title{Brief Article}
%\author{David F. Snyder}
%\author{L.G. Meredith}

%\address{Dept. of Math., Texas State University--San Marcos, San Marcos, TX 78666}
       
\pagestyle{empty}


\begin{document}

\lstset{language=[Objective]Caml,frame=shadowbox}

\input{qm2pi.front}

% section front matter (end)

\input{qm2pi.intro} 
 
% section introduction (end)

% \input{qm2pi.knotations} 

% section notation (end)

\input{qm2pi.process.calculi} 

% section concurrent_process_calculi_and_spatial_logics_ (end)
    
%\input{qm2pi.knots2pi} 

%\input{qm2pi.trefoil} 

%\input{qm2pi.mainthm} 

% subsection basic_interpretation (end)

%\input{qm2pi.rho.presentation} 
\subsection{The syntax and semantics of the notation system}\label{sub:the_syntax_and_semantics_of_the_notation_system} % (fold)

We now summarize a technical presentation of the calculus that
embodies our theory of dynamics. The typical presentation of such a
calculus follows the style of giving generators and relations on
them. The grammar, below, describing term constructors, freely
generates the set of processes, $\Proc$. This set is then quotiented
by a relation known as structural congruence and it is over this set
that the notion of dynamics is expressed. This presentation is
essentially that of \cite{MeredithR05} with the addition of
polyadicity and summation. For readability we have relegated some of
the technical subtleties to an appendix.

\subsubsection{Process grammar}\label{subsub:process_grammar}

\begin{mathpar}
  \inferrule* [lab=synchronization] {} {{M} \bc \pzero \;|\; x?F \;|\; x!C }
  \and
  \inferrule* [lab=abstraction] {} {{F} \bc (x)P}
  \and
  \inferrule* [lab=concretion] {} {{C} \bc \langle Q \rangle}
  \and
  \inferrule* [lab=process] {} {{P,Q} \bc M \;| \;P|Q \;|\; @{x}}
  \and
  \inferrule* [lab=name] {} {{x} \bc \quotep{P}}
\end{mathpar} 

Note that $\vec{x}$ (resp. $\vec{P}$) denotes a vector of names
(resp. processes) of length $|\vec{x}|$ (resp. $|\vec{P}|$). We adopt
the following useful abbreviations.

\begin{mathpar}
   x?(\vec{y}).P := x.(\vec{y})P \and  x\clift{\vec{P}} := x.\clift{\vec{P}}
   \and x!(y) := \lift{x}{\dropn{y}}
   \and \Pi_{i=0}^{n-1}P_i := P_0 | \ldots | P_{n-1}
\end{mathpar}

\subsubsection{Structural congruence}

\paragraph{Free and bound names and alpha-equivalence.} At the
core of structural equivalence is alpha-equivalence which identifies
process that are the same up to a change of variable. Formally, we
recognize the distinction between free and bound names. The free names
of a process, $\freenames{P}$, may be calculated recursively as
follows:

\begin{mathpar}
\freenames{\pzero} := \emptyset
  \and \\
  \freenames{x?(y).P} := \{ x \} \cup (\freenames{P} \setminus \{ y \})
  \and 
  \freenames{x!\langle P \rangle} := \{ x \} \cup \{ P \} 
  \and \\
  \freenames{P|Q} := \freenames{P} \cup \freenames{Q}
  \and \\
  \freenames{@{x}} := \{ x \}
\end{mathpar}

$\pi$
$\quotep{\pi}$

$\freenames{-} : \pi \to \mathcal{P}(\quotep{\pi})$

\begin{eqnarray*}
  \freenames{\pzero} & := & \emptyset \\
  \freenames{x?(y).P} & := & \{ x \} \cup (\freenames{P} \setminus \{ y \}) \\
  \freenames{x!\langle P \rangle} & := & \{ x \} \cup \{ P \} \\
  \freenames{P|Q} & := & \freenames{P} \cup \freenames{Q} \\
  \freenames{\dropn{x}} & := & \{ x \}
\end{eqnarray*}

The bound names of a process, $\boundnames{P}$, are those names occurring in $P$
that are not free. For example, in $x?(y).0$, the name $x$ is free, while $y$ is bound.

\begin{mathpar}
  \inferrule* [lab=monoidal-laws] {} { P|Q \equiv Q|P \and P|0 \equiv P \and P|(Q|R) \equiv (P|Q)|R }
\end{mathpar}

\begin{mathpar}
  \inferrule* [lab=alpha-equivalence] {} { (x)P \equiv (y)P\{y/x\} \and y \not\in \freenames{P} }
\end{mathpar}

\begin{definition}
Then two processes, $P,Q$, are alpha-equivalent if $P = Q\{\vec{y}/\vec{x}\}$ for
some $\vec{x} \in \boundnames{Q},\vec{y} \in \boundnames{P}$, where $Q\{\vec{y}/\vec{x}\}$
denotes the capture-avoiding substitution of $\vec{y}$ for $\vec{x}$ in $Q$.
\end{definition}

\begin{definition}
  The {\em structural congruence} \cite{SangiorgiWalker} , $\equiv$,
  between processes is the least congruence containing
  alpha-equivalence, satisfying the abelian monoid laws
  (associativity, commutativity and $\pzero$ as identity) for parallel
  composition $|$ and for summation $+$.
\end{definition}

\subsection{Name equivalence}

We take name equivalence, written $\nameeq$, to be the smallest
equivalence relation generated by the following rules.

\begin{mathpar}
\inferrule*[lab=Quote-drop]
{ }
{ \quotep{@{x}} \nameeq x }

\inferrule*[lab=Struct-equiv]
{ P \scong Q }
{ \quotep{P} \nameeq \quotep{Q} }
\end{mathpar}

The astute reader will have noticed that the mutual recursion of names
and processes imposes a mutual recursion on alpha-equivalence and
structural equivalence via name-equivalence. Fortunately, all of this
works out pleasantly and we may calculate in the natural way, free of
concern. The reader interested in the details is referred to the
appendix \ref{appendix:rho_details}.

\subsection{Substitution}

We use $\Proc$ for the set of processes, $\QProc$ for the set of
names, and $\id{\{}\vec{y} / \vec{x} \id{\}}$ to denote partial maps,
$s : \QProc \rightarrow \QProc$. A map, $s$ lifts, uniquely, to a map
on process terms, $\widehat{s} : \Proc \rightarrow \Proc$ by the
following equations.

\begin{mathpar}
  (0) \psubstp{Q}{P} := 0 \\
  (R \juxtap S) \psubstp{Q}{P}
  :=    
  (R)\psubstp{Q}{P} \juxtap (S) \psubstp{Q}{P} \\
  (x?(y).R) \psubstp{Q}{P}    
  :=    
  (x)\substp{Q}{P} (z)\concat( (R \psubstn{z}{y}) \psubstp{Q}{P} ) \\
  (\lift{x}{R}) \psubstp{Q}{P}  
  :=
  \lift{(x)\substp{Q}{P}}{ R \psubstp{Q}{P} } \\
%   (\dropn{x})  \psubstp{Q}{P}       
%   := 
%   \left\{ 
%     \begin{array}{ccc} 
%       \dropn{\quotep{Q}} & & x \nameeq \quotep{P} \\
%       \dropn{x} & & otherwise \\
%     \end{array}
%   \right. 
  (\dropn{x})  \psubstp{Q}{P}       
  := 
  \left\{ 
    \begin{array}{ccc} 
      Q & & x \nameeq \quotep{P} \\
      \dropn{x} & & otherwise \\
    \end{array}
  \right.
\end{mathpar}
 

where

\begin{eqnarray}
  (x)\id{\{} \lpquote Q \rpquote / \lpquote P \rpquote \id{\}}            = 
  \left\{ 
    \begin{array}{ccc}
      \lpquote Q \rpquote & & x \nameeq \lpquote P \rpquote \\
      x & & otherwise \\
    \end{array}
  \right. \nonumber
\end{eqnarray}

and $z$ is chosen distinct from $\quotep{P}$, $\quotep{Q}$, the free
names in $Q$, and all the names in $R$. Our $\alpha$-equivalence will
be built in the standard way from this substitution.

\begin{remark}\label{rem:no_self_referential_names}
  One consequence of these definitions is that $\forall P. \quotep{P}
  \not\in \freenames{P}$.
\end{remark}

\subsection{ Dynamic quote: an example }

Anticipating something of what's to come, consider applying the
substitution, $\widehat{\id{\{}u / z \id{\}}}$, to the following pair
of processes, $\lift{w}{y!(z)}$ and $w[ \lpquote y!(z) \rpquote ]$.

\begin{eqnarray}
	\lift{w}{y!(z)}\widehat{\id{\{}u / z \id{\}}}
		& = &
		\lift{w}{y!(u)} \nonumber\\
	w[ \lpquote y!(z) \rpquote ] \widehat{ \id{\{}u / z \id{\}} }
		& = &
		w[ \lpquote y!(z) \rpquote ] \nonumber
\end{eqnarray}

Because the body of the process between quotes is impervious to
substitution, we get radically different answers. In fact, by
examining the first process in an input context,
e.g. $x?(z).\lift{w}{y!(z)}$, we see that the process under the lift
operator may be shaped by prefixed inputs binding a name inside it. In
this sense, the lift operator will be seen as a way to dynamically
construct processes before reifying them as names.

Finally equipped with these standard features we can present the
dynamics of the calculus.

\subsubsection{Operational semantics} 

Finally, we introduce the computational dynamics. What marks these
algebras as distinct from other more traditionally studied algebraic
structures, e.g. vector spaces or polynomial rings, is the manner in
which dynamics is captured. In traditional structures, dynamics is typically
expressed through morphisms between such structures, as in linear maps
between vector spaces or morphisms between rings. In algebras
associated with the semantics of computation, the dynamics is
expressed as part of the algebraic structure itself, through a
reduction reduction relation typically denoted by $\red$. Below, we
give a recursive presentation of this relation for the calculus used
in the encoding.

$\red \subseteq \pi \times \pi$
$\red : \pi \to \mathcal{P}(\pi)$

\begin{mathpar}
  \inferrule* [lab=Comm] { \textsf{match}( x_{src}, x_{trgt} ) } { x_{trgt}?(y)P \; | \; x_{src}!\langle {Q} \rangle \red P\{\quotep{Q}/y}\} }
  \and \\
  \inferrule* [lab=Par] {{P} \red {P}'} {{{P} | {Q}} \red {{P}' | {Q}}}
  \and
  \inferrule* [lab=Equiv]{{{P} \scong {P}'} \andalso {{P}' \red {Q}'} \andalso {{Q}' \scong {Q}}}{{P} \red {Q}}
\end{mathpar}

\begin{eqnarray*}
  match_{\equiv} (\quotep{P},\quotep{Q}) & := & P \equiv Q \\
  match_{\dagger}(\quotep{P},\quotep{Q}) & := & \forall R. P|Q \red^{*} R => R \red^{*} 0 \\
  match_{K}(\quotep{P},\quotep{Q}) & := & K \mbox{ for some context } K
\end{eqnarray*}

$u?(x)P | u!\langle Q \rangle \red P\{\quotep{Q}/x\}$

%We write $\wred$ for $\red^*$, and $P\red$ if $\exists Q $ such that $ P \red Q$.
We write $P\red$ if $\exists Q $ such that $ P \red Q$ and $P\not\red$, otherwise.

\section{Replication}

As mentioned before, it is known that replication (and hence
recursion) can be implemented in a higher-order process algebra
\cite{SangiorgiWalker}. As our first example of calculation with the
machinery thus far presented we give the construction explicitly in
the {\rhoc}.

\begin{eqnarray}
	D_{x} & := & \prefix{x}{y}{(\binpar{\outputp{x}{y}}{@{y}})} \nonumber\\
	\bangp_{x}{P} & := & \binpar{{x}!\langle{\binpar{D_{x}}{P}}\rangle}{D_{x}} \nonumber
\end{eqnarray}

\begin{eqnarray}
	\bangp_{x}{P} & & \nonumber\\
	=
	& {x}!\langle{(\prefix{x}{y}{(\outputp{x}{y} | @{y})) | P}}\rangle 
	      | \prefix{x}{y}{(\outputp{x}{y} | @{y})} & \nonumber\\
	\red
	& (\outputp{x}{y} | @{y})\substn{\quotep{(\prefix{x}{y}{(@{y} | \outputp{x}{y})) | P}}}{y} & \nonumber\\
	=
	& \outputp{x}{\quotep{(\prefix{x}{y}{(\outputp{x}{y} | @{y})) | P}}}
	  | {(\prefix{x}{y}{(\outputp{x}{y} | @{y})) | P}} & \nonumber\\
	\red
	& \ldots & \nonumber\\
	\red^*
	& P | P | \ldots & \nonumber
\end{eqnarray}

Of course, this encoding, as an implementation, runs away, unfolding
$\bangp{P}$ eagerly. A lazier and more implementable replication
operator, restricted to input-guarded processes, may be obtained as follows.

\begin{eqnarray}
\bangp{\prefix{u}{v}{P}} 
	:= 
	\binpar{\lift{x}{\prefix{u}{v}{(\binpar{D(x)}{P})}}}{D(x)} \nonumber
\end{eqnarray}

\begin{remark}
  Note that the lazier definition still does not deal with summation
  or mixed summation (i.e. sums over input and output). The reader is
  invited to construct definitions of replication that deal with these
  features. 

  Further, the definitions are parameterized in a name, $x$. Can you,
  gentle reader, make a definition that eliminates this parameter and
  guarantees no accidental interaction between the replication
  machinery and the process being replicated -- i.e. no accidental
  sharing of names used by the process to get its work done and the
  name(s) used by the replication to effect copying. This latter
  revision of the definition of replication is crucial to obtaining
  the expected identity $!!P \sim !P$.
\end{remark}

\begin{remark}\label{rem:paradoxical_combinator}
  The reader familiar with the lambda calculus will have noticed the
  similarity between $D$ and the paradoxical combinator.

  [Ed. note: the existence of this seems to suggest we have to be more
  restrictive on the set of processes and names we admit if we are to
  support no-cloning.]
\end{remark}

\subsubsection{Bisimulation}

The computational dynamics gives rise to another kind of equivalence,
the equivalence of computational behavior. As previously mentioned
this is typically captured \emph{via} some form of bisimulation.

% The notion we use in this paper is weak barbed bisimulation
% \cite{milner91polyadicpi}.

The notion we use in this paper is derived from weak barbed
bisimulation \cite{milner91polyadicpi}. 

\begin{definition}
An \emph{observation relation}, $\downarrow_{\mathcal N}$, over a set
of names, $\mathcal N$, is the smallest relation satisfying the rules
below.

\infrule[Out-barb]{y \in {\mathcal N}, \; x \nameeq y}
		  {\outputp{x}{v} \downarrow_{\mathcal N} x}
\infrule[Par-barb]{\mbox{$P\downarrow_{\mathcal N} x$ or $Q\downarrow_{\mathcal N} x$}}
		  {\binpar{P}{Q} \downarrow_{\mathcal N} x}

We write $P \Downarrow_{\mathcal N} x$ if there is $Q$ such that 
$P \wred Q$ and $Q \downarrow_{\mathcal N} x$.
\end{definition}

\begin{definition}
%\label{def.bbisim}
An  ${\mathcal N}$-\emph{barbed bisimulation} over a set of names, ${\mathcal N}$, is a symmetric binary relation 
${\mathcal S}_{\mathcal N}$ between agents such that $P\rel{S}_{\mathcal N}Q$ implies:
\begin{enumerate}
\item If $P \red P'$ then $Q \wred Q'$ and $P'\rel{S}_{\mathcal N} Q'$.
\item If $P\downarrow_{\mathcal N} x$, then $Q\Downarrow_{\mathcal N} x$.
\end{enumerate}
$P$ is ${\mathcal N}$-barbed bisimilar to $Q$, written
$P \wbbisim_{\mathcal N} Q$, if $P \rel{S}_{\mathcal N} Q$ for some ${\mathcal N}$-barbed bisimulation ${\mathcal S}_{\mathcal N}$.
\end{definition}

$\mathcal{R} \subseteq \pi \times \pi$

$P \mathcal{R} Q => \forall P'. P \red P' \Rightarrow \exists Q'. Q \red Q', P' \mathcal{R} Q'$

$P \vdash x \Rightarrow Q \vdash x$

\begin{mathpar}
  \inferrule*[lab=Out-barb]{x \nameeq y}{{y}!\langle{Q}\rangle \vdash x}
  \and
  \inferrule*[lab=Par-barb]{\mbox{$P\vdash x$ or $Q\vdash x$}}{\binpar{P}{Q} \vdash x}
\end{mathpar}

\subsubsection{Contexts}

One of the principle advantages of computational calculi like the
$\pi$-calculus is a well-defined notion of context,
contextual-equivalence and a correlation between
contextual-equivalence and notions of bisimulation. The notion of
context allows the decomposition of a process into (sub-)process and
its syntactic environment, its context. Thus, a context may be
thought of as a process with a ``hole'' (written $\Box$) in it. The
application of a context $M$ to a process $P$, written $M[P]$, is
tantamount to filling the hole in $M$ with $P$. In this paper we do
not need the full weight of this theory, but do make use of the notion
of context in the proof the main theorem. 

\begin{mathpar}
  \inferrule* [lab=summation] {} {{M_{M},M_{N}} \bc \Box \;|\; x.M_{A} \;|\; M_{M}+M_{N}}
  \and
  \inferrule* [lab=agent] {} {{M_{A}} \bc (\vec{x})M_{P} \;| \; \clift{P_0,\ldots,M_{P},\ldots,P_N}}
  \and \\
  \inferrule* [lab=process] {} {{M_{P}} \bc M_{N} \;| \;P|M_{P} }
\end{mathpar} 

\begin{mathpar}
  \inferrule* [lab=sychronization] {} {M_{N} \bc \Box \;|\; x?M_{F} \;|\; x!M_{C}}
  \and
  \inferrule* [lab=abstraction] {} {{M_{F}} \bc (x)M_{P} }
  \and
  \inferrule* [lab=concretion] {} {{M_{C}} \bc \langle M_{P} \rangle }
  \and \\
  \inferrule* [lab=process] {} {{M_{P}} \bc M_{N} \;| \;P|M_{P} }
\end{mathpar}

\begin{definition}[contextual application] Given a context $M$, and
  process $P$, we define the \emph{contextual application}, $M[P] :=
  M\{P/\Box\}$. That is, the contextual application of M to P is the
  substitution of $P$ for $\Box$ in $M$.
\end{definition}

$\meaningof{-} : L \to \mathcal{P}(\pi)$

\begin{mathpar}
  \inferrule* [lab=collection] {} {\meaningof{true} = \pi, \and \meaningof{~E} = \pi \setminus \meaningof{E}, \and \meaningof{E_{1} \& E_{2}} = \meaningof{E_{1}} \cap \meaningof{E_{2}}}
\end{mathpar}

\begin{mathpar}
  \inferrule* [lab=structure] {} {\meaningof{0} = \{ P \in \pi | P \equiv 0 \}, \and \\ \meaningof{E_1 | E_2} = \{ P \in \pi | P \equiv P_{1} | P_{2}, P_{1} \in \meaningof{E_{1}}, P_{2} \in \meaningof{E_2}\} }
\end{mathpar}

\begin{mathpar}
 \inferrule* [lab=behavior] {} {\meaningof{\langle a?b \rangle E} = \{ P \in \pi | P \equiv Q | u?(y)P', \\ \and \\\\ \and \\ \;\;\; u \in \meaningof{a}, \forall z.P'\{z/y\} \in \meaningof{E\{z/b\}}\}, \and \\ \meaningof{a!E} = \{ P \in \pi | P \equiv Q | x!\langle P' \rangle, x \in \meaningof{a} P' \in \meaningof{E}\} }
\end{mathpar}

\begin{mathpar}
 \inferrule* [lab=nominal] {} {\meaningof{\quotep{E}} = \{ \quotep{P} \in \quotep{\pi} | P \in \meaningof{E} \}, \and \meaningof{\quotep{P}} = \{ \quotep{Q} \in \quotep{\pi} | P \equiv Q \} \and \\ \meaningof{@\quotep{E}} = \{ P \in \pi | P \equiv @x, x \in \meaningof{E} \}}
\end{mathpar}

\begin{eqnarray*}
  \\
  \meaningof{-} : TS \to ST
\end{eqnarray*}

\begin{eqnarray*}
  \\
  L : TS \to ST
\end{eqnarray*}

\begin{eqnarray*}
  \\
  P \models E \iff P \in \meaningof{E}
\end{eqnarray*}

\begin{eqnarray*}
  P \approx_{L} Q \iff \forall E \in L. P \models E \iff Q \models E
\end{eqnarray*}

\begin{eqnarray*}
  P \approx_{K} Q
\end{eqnarray*}

\begin{eqnarray*}
  P \approx Q
\end{eqnarray*}

$\approx_{K} = \approx = \approx_{L}$

\subsubsection{Contextual duality}

Note that contexts extend the quotation operation to a family of
operations from processes to names. Given a context, $M$, we can
define a \emph{nominal context}, $\quotep{M}$ by $\quotep{M}[P] :=
\quotep{M[P]}$. To foreshadow what is to come we observe that these
operations enjoy a duality with processes very much like the duality
between vectors and maps from vectors to scalars.

Further, because the calculus is essentially higher-order, we have a
correspondence between contexts and processes. More specifically,
given a name $x$ and a context $M$ we can construct $M^{*}_{x}$ such
that 

\begin{mathpar}
  M^{*}_{x} | \lift{x}{P} \red M[P]
\end{mathpar}

namely,

\begin{mathpar}
  M^{*}_{x} := x?(u).M[\dropn{u}]
\end{mathpar}

The dependence of $M^{*}_{x}$ on a name makes it an abstraction, 

\begin{mathpar}
  M^{*} := (x)x?(u).M[\dropn{u}]
\end{mathpar}

\subsection{Additional notation}

It will sometimes be convenient to denote the process a name
quotes. We already have the notation $x = \quotep{P}$, but it will be
convenient to introduce an alternate notation, $\procn{x}$, when we
want to emphasize the connection to the use of the name. Note that, by
virtue of name equivalence, $\quotep{\procn{x}} \nameeq x$; so, the
notation is consistent with previous definitions.

Further, because names have structure it is possible to effect
substitutions on the basis of that structure. This means we need to
upgrade our notation for substitutions, which we accomplish by
adapting comprehension notation. Thus,

\begin{mathpar}
  P\{ y / x : x \in S \}
\end{mathpar}

is interpreted to mean the process derived from P by replacing (in a
capture-avoiding manner) each occurrence of $x$ in $S$ by $y$. For example,

\begin{mathpar}
  P\{ \quotep{\procn{x}|\procn{x}} / x : x \in \freenames{P} \}
\end{mathpar}

will replace each (occurrence) of a free name $x$ in $P$ by
$\quotep{\procn{x}|\procn{x}}$.

Also, we will avail ourselves of the notation $x^{L}$ and $x^{R}$ to
denote injections of a name into disjoint copies of the name
space. There are numerous ways to accomplish this. One example can be
found in \cite{MeredithR05}. This notation overloads to vectors of
names: $\vec{x}^{\pi} := (x_{i}^{\pi} \; : \; 0 \leq i < |\vec{x}| )$ where $\pi \in \{L,R\}$.

We also use $P^{\Box} := P|\Box$.

In \cite{MeredithR05} an interpretation of the new operator is
given. It turns out that there are several possible interpretations
all enjoying the requisite algebraic properties of the operator (see
\cite{milner91polyadicpi}). We will therefore make liberal use of
$(\nu\; \vec{x})P$.

% subsection the_syntax_and_semantics_of_the_notation_system (end)   

\input{qm2pi.qmops} 

\input{qm2pi.sterngerlach} 

\input{qm2pi.metric} 

% section concurrent_process_calculi (end)

%\input{qm2pi.proofsketch}

% section proof sketch (end)

%\input{qm2pi.slviaknots} 

% section spatial logic via knots (end)

\input{qm2pi.conclusion}

% section conclusion (end)

%\input{qm2pi.dtcodes} 

% section wiring algorithm (end)

\input{qm2pi.ack} 

% section acknowledgments (end)

\newpage


\bibliographystyle{plain}   
\bibliography{../../biblios/main.bib}

\input{qm2pi.rhodetails}

\end{document}



% section proof sketch (end)

%\section{Unlikely characters: spatial logic for
  knots}\label{sub:characteristic_formulae} % (fold)

Associated to the mobile process calculi are a family of logics known
as the Hennessy-Milner logics. These logics typically enjoy a
semantics interpreting formulae as sets of processes that when
factored through the encoding outlined above allows an identification
of classes of knots with logical formulae. In the context of this
encoding the sub-family known as the spatial logics \cite{CairesC03}
\cite{CairesC04} \cite{Caires04} are of particular interest providing
several important features for expressing and reasoning about
properties (i.e. classes) of knots. We hint here at how this may be done.

%\begin{description}
%\item [structural connectives] 
\subsubsection{Structural connectives} The spatial logics enjoy
structural connectives corresponding, at the logical level, to the
parallel composition ($P | Q$) and new name ($(\nu \; x)P$)
connectives for processes. As illustrated in the examples below, these
connectives are extremely expressive given the shape of our encoding.
%\item [decideable satisfaction]

\subsubsection{Decideable satisfaction}
In \cite{Caires04} the satisfaction relation is shown to be decideable
for a rich class of processes. It further turns out that the image of
the our encoding is a proper subset of that class. This result
provides the basis for an algorithm by which to search for knots
enjoying a given property.
%\item [characteristic formulae]

\subsubsection{Characteristic formulae}
In the same paper \cite{Caires04} , Caires presents a means of calculating
characteristic formulae, selecting equivalence classes of processes
up to a pre--specified depth limit on the support set of names. Composed with our
encoding, this characteristic formula can be used to select
characteristic formulae for knots.
%\end{description}

\subsubsection{Spatial logic formulae}

The grammar below (segmented for comprehension) summarizes the syntax
of spatial logic formulae. We employ illustrative examples in the
sequel to provide an intuitive understanding of their meaning
referring the reader to \cite{Caires04} for a more detailed explication
of the semantics.

\begin{mathpar}
  \inferrule* [lab=boolean] {} {{A,B} \bc T \;|\; \neg A \;|\; A \wedge B \;|\; \eta = \eta'}
  \and
  \inferrule* [lab=spatial] {} {|\; \pzero \;|\; A | B \;|\; x \text{\textregistered} A \;|\; \forall x . A \;|\;  H x . A}
  \and
  \inferrule* [lab=behavioral] {} {|\; \alpha . A}
  \and 
  \inferrule* [lab=recursion] {} {|\; X(\vec{u}) \;|\; \mu X(\vec{u}) . A}
  \and
  \inferrule* [lab=action] {} {\alpha \bc \langle x?(\vec{y}) \rangle \;|\; \langle x!(\vec{y}) \rangle \;|\; \langle \tau \rangle}
  \and 
  \inferrule* [lab=name] {} {\eta \bc x \;|\; \tau}
\end{mathpar} 

% subsection characteristic_formulae (end)   	 

\subsection{Example formulae}\label{sub:example_formulae_} % (fold)

\subsubsection{Crossing as formula.}
% 
% \begin{align*}
%   \frac{d}{dx} \sin x &= \cos x 
%   & \frac{d}{dx} e^x &= e^x \\
%   \frac{d}{dx} \cos x &= - \sin x 
%   & \frac{d}{dx} \log x &= \frac{1}{x} \\
% \end{align*} 

\begin{align*}
 \mu C(x_{0},x_{1},y_{0},y_{1},u).&(\langle x_{0}?(z) \rangle(\langle u! \rangle\langle y_{1}!z \rangle C(x_{0},x_{1},y_{0},y_{1},u)) & \\
  & \wedge \langle y_{1}?(z) \rangle (\langle u! \rangle \langle x_{0}!z \rangle C(x_{0},x_{1},y_{0},y_{1},u)) & \\
  & \wedge \langle x_{1}?(z) \rangle (\langle u? \rangle \langle y_{0}!z \rangle C(x_{0},x_{1},y_{0},y_{1},u)) & \\
  & \wedge \langle y_{0}?(z) \rangle (\langle u? \rangle \langle x_{1}!z \rangle C(x_{0},x_{1},y_{0},y_{1},u))) &
\end{align*}

The lexicographical similarity between the shape of this formulae and
the shape of definition of the process representing a crossing reveals
the intuitive meaning of this formulae. It describes the capabilities
of a process that has the right to represent a crossing. For example
it picks out processes that may perform an input on the port $x_0$ in
its initial menu of capabilities. What differentiates the formula
from the process, however, is that the crossing process is the
smallest candidate to satisfy the formula. Infinitely many other
processes -- with internal behavior hidden behind this interface, so
to speak -- also satisfy this formula. Even this simple formula,
then, can be seen to open a new view onto knots, providing a
computational interpretation of \emph{virtual} knots.

Note that this formula is derived by hand. A similar formula can be
derived by employing Caires' calculation of characteristic formula
\cite{Caires04} to the process representing a crossing. In light of
this discussion, we let
$\meaningof{C}_{\phi}(x0,x1,y0,y1,u)$ denote a formula specifying the
dynamics we wish to capture of a crossing. To guarantee we preserve
the shape of the interface and minimal semantics we demand that
$\meaningof{C}_{\phi}(x0,x1,y0,y1,u) \Rightarrow
\textbf{C}(x0,x1,y0,y1,u)$ where $\textbf{C}(x0,x1,y0,y1,u)$ denotes
the formula above.
                            
\subsubsection{Crossing number constraints.}
The moral content of the context lemma (Lemma \ref{context}) is that the notion of
``locality'' in the Reidemeister moves is effectively captured by the
parallel composition operator of the process calculus. This intuition
extends through the logic. Given a formula,
$\meaningof{C}_{\phi}(x0,x1,y0,y1,u)$, we can use the structural
connectives to specify constraints on crossing numbers, such as at
least $n$ crossings, or exactly $n$ crossings.
\begin{mathpar}
  \inferrule* [lab=at-least-n] {} { K^{\geq n}_{\phi}(\vec{xs},\vec{ys}) := \Pi_{i=0}^{n-1} Hu . \meaningof{C}_{\phi}(xs_i,ys_i,u) | T }
  \and 
  \inferrule* [lab=exactly-n] {} { K^{= n}_{\phi}(\vec{xs},\vec{ys}) := \Pi_{i=0}^{n-1} Hu . \meaningof{C}_{\phi}(xs_i,ys_i,u) | \neg (\forall x_0,y_0,x_1,y_1,u . \meaningof{C}_{\phi}(x_0,y_0,x_1,y_1,u) | T) }
\end{mathpar}

To round out this section, recall that the encoding of an $n$-crossing
knot decomposes into a parallel composition of $n$ \emph{copies} of a
crossing process together with a wiring harness. To specify different
knot classes with the same crossing number amounts to specifying
logical constraints on the wiring harness. In the interest of space,
we defer examples to a forthcoming paper. Suffice it to say that both
the conditions ``alternating knot'' and ``contains the tangle
corresponding to 5/3'' are expressible. For example, it is possible to
calculate the characteristic formula of a process corresponding to the
tangle 5/3 and conjoin it into the classifying formula via the
composition connective of the logic.

Finally, we wish to observe that it is entirely within reason to
contemplate a more domain-specific version of spatial logic tailored
to the shape of processes in the image of the encoding. Such a
domain-specific logic would have a better claim to the title formal
language of knot properties.

% subsection example_formulae_ (end)

% section knots_as_processes (end) 

% section spatial logic via knots (end)

\section{Conclusions and future work}

\paragraph{Testing physical space}
You, gentle reader, may wonder why of all the theorems to be proved
given this set up we pick the one above. In some sense it's hardly
central to quantum mechanics. We see it as central in the sense that
it firmly establishes a notion of physical space arising from a notion
of the equivalence of behavior. Relating bisimulation to a metric is a
big step forward, but one is faced with interpreting the relationship
of that metric space to something more physical. Quantum mechanical
notions of ``physical'' space are still far from intuitive, but by
relating this idea of distance as testing to calculations that predict
physical circumstances we are making a not insignificant step forward
toward an understanding of the physical space we inhabit as
essentially dynamic.

\paragraph{Effectivity and simulation}
One of the observations we have yet to make is that the entire program
spelled out here is effective. We have built various interpreters for
the reflective calculus at work in this interpretation. In principle,
then, we can simulate quantum mechanics on a computer. The place where
the simulation may lose fidelity is the infinitely branching summation
for the annihilator.

In this connection i also want to point out that the evaluation style
calculation of the inner product puts the non-determinism of the
summation right at the heart of measurement. This suggests that
Milner's original reduction-based formulation of the dynamics of his
calculi in terms of sums was not just notationally suggestive of a
notion of measure-and-continue but captured some significant part of
the physics.

\paragraph{Quantum continuations}
In light of this last observation i want to point out that the
predominant account of quantum mechanics is missing a key aspect of a
truly compositional story of the physical situation. In a real lab,
when a measurement is made the observation can be made to feed into
another device that then makes another measurement conditioned on the
results of the first. This means that after the superposition was
collapsed the entire experimental set up remained in
superposition. While QM offers a means of writing this down it doesn't
quite line up well with the well-trodden formulation of computation
and continuation that we see so succinctly expressed in Milner's
calculi. This suggests that there might be advantages to this account
of dynamics waiting to be explored.

\paragraph{Quantum logic}
In this connection, we also note that by virtue of having the
Hennessy-Milner construction, we can pull the construction through the
interpretation of QM. This gives us a natural candidate for a quantum
logic that enjoys an extremely tight connection with it's domain of
interpretation, making the construction much less ad hoc (rather it is
the image of functor!).

\paragraph{Quantum probabiity}
i have questions about the basis of the interpretation of inner
product as probability amplitude. In particular, using which
axiomatization of probability theory does the notion of probability
amplitude earn the right to be so dubbed? In other words, where is the
proof that the operation for calculating a probability amplitude (and
then squaring) satisfies the axioms of what it means to calculate a
probability? Even if such a proof exists (i have yet to find it in the
literature), i wonder if it might not be possible to turn things on
their heads. Can we view the calculation of the probability amplitude
as an axiomatization of probability? If so, then the definition we
give for calculating probability amplitude may provide the basis for
an \emph{effective} theory of probability.

\paragraph{Quantum vs ``biological'' information}
Finally, i want to conclude with a more philosophical observation. At
a recent workshop in which QM was a predominant topic i noticed
something about quantum information. The speaker was giving a riveting
discussion of axiomatic QM and showing how properties of ``no
cloning'' and ``no deleting'' emerged as consequences of the
axiomatization. Theorems of this form are necessary to give us a sense
of confidence that our axioms characterize the physical theory. What
struck me, though, was that if quantum information is neither erasable
nor replicable it is markedly different from \emph{life}. Two of the
things we know about life is that

\begin{itemize}
  \item it ends;
  \item to gain some measure of persistence, to transcend it's
    finitude it is imminently copyable.
\end{itemize}

Both of these qualities are summarized succinctly in the aphorism: all
flesh is grass. For me these two kinds of ``information'' -- call them
quantum and biological -- are end points on a spectrum of strategies
for persistence. At one end, we have those curious entities that enjoy
uniqueness and permanence; at the other, we have those who in the face
of a certain end and an uncertain present make a go of passing
something on. To me one of the more remarkable aspects of the latter
strategy is that in the presence of noise (and certain features of
copying) we get a kind of dynamism, a chance for improvement against a
given persistent condition.

% subsection other_calculi_other_bisimulations_and_geometry_as_behavior (end)




% section conclusion (end)

%\documentclass[12pt]{llncs}
%\documentclass{jktr}

\usepackage[pdftex]{hyperref}                   
\usepackage {listings}
\usepackage {mathpartir}
\usepackage{bcprules}
%\usepackage{listings}
                       
\usepackage{graphicx} 
%\usepackage[margins=2.5cm,nohead,nofoot]{geometry}
%\usepackage{geometry}
\usepackage{amsfonts}
\usepackage{amstext}
\usepackage{latexsym}
\usepackage{amssymb}
\usepackage{color}


%\include{myPreamble}
\include{qm2pi.local} 

%\ifpdf
%\usepackage[pdftex]{graphicx}
%\else
%\usepackage{graphicx}
%\fi

 % \ifpdf
%  \usepackage{pdfsync}
%  \if


%\title{Brief Article}
%\author{David F. Snyder}
%\author{L.G. Meredith}

%\address{Dept. of Math., Texas State University--San Marcos, San Marcos, TX 78666}
       
\pagestyle{empty}


\begin{document}

\lstset{language=[Objective]Caml,frame=shadowbox}

\input{qm2pi.front}

% section front matter (end)

\input{qm2pi.intro} 
 
% section introduction (end)

% \input{qm2pi.knotations} 

% section notation (end)

\input{qm2pi.process.calculi} 

% section concurrent_process_calculi_and_spatial_logics_ (end)
    
%\input{qm2pi.knots2pi} 

%\input{qm2pi.trefoil} 

%\input{qm2pi.mainthm} 

% subsection basic_interpretation (end)

%\input{qm2pi.rho.presentation} 
\subsection{The syntax and semantics of the notation system}\label{sub:the_syntax_and_semantics_of_the_notation_system} % (fold)

We now summarize a technical presentation of the calculus that
embodies our theory of dynamics. The typical presentation of such a
calculus follows the style of giving generators and relations on
them. The grammar, below, describing term constructors, freely
generates the set of processes, $\Proc$. This set is then quotiented
by a relation known as structural congruence and it is over this set
that the notion of dynamics is expressed. This presentation is
essentially that of \cite{MeredithR05} with the addition of
polyadicity and summation. For readability we have relegated some of
the technical subtleties to an appendix.

\subsubsection{Process grammar}\label{subsub:process_grammar}

\begin{mathpar}
  \inferrule* [lab=synchronization] {} {{M} \bc \pzero \;|\; x?F \;|\; x!C }
  \and
  \inferrule* [lab=abstraction] {} {{F} \bc (x)P}
  \and
  \inferrule* [lab=concretion] {} {{C} \bc \langle Q \rangle}
  \and
  \inferrule* [lab=process] {} {{P,Q} \bc M \;| \;P|Q \;|\; @{x}}
  \and
  \inferrule* [lab=name] {} {{x} \bc \quotep{P}}
\end{mathpar} 

Note that $\vec{x}$ (resp. $\vec{P}$) denotes a vector of names
(resp. processes) of length $|\vec{x}|$ (resp. $|\vec{P}|$). We adopt
the following useful abbreviations.

\begin{mathpar}
   x?(\vec{y}).P := x.(\vec{y})P \and  x\clift{\vec{P}} := x.\clift{\vec{P}}
   \and x!(y) := \lift{x}{\dropn{y}}
   \and \Pi_{i=0}^{n-1}P_i := P_0 | \ldots | P_{n-1}
\end{mathpar}

\subsubsection{Structural congruence}

\paragraph{Free and bound names and alpha-equivalence.} At the
core of structural equivalence is alpha-equivalence which identifies
process that are the same up to a change of variable. Formally, we
recognize the distinction between free and bound names. The free names
of a process, $\freenames{P}$, may be calculated recursively as
follows:

\begin{mathpar}
\freenames{\pzero} := \emptyset
  \and \\
  \freenames{x?(y).P} := \{ x \} \cup (\freenames{P} \setminus \{ y \})
  \and 
  \freenames{x!\langle P \rangle} := \{ x \} \cup \{ P \} 
  \and \\
  \freenames{P|Q} := \freenames{P} \cup \freenames{Q}
  \and \\
  \freenames{@{x}} := \{ x \}
\end{mathpar}

$\pi$
$\quotep{\pi}$

$\freenames{-} : \pi \to \mathcal{P}(\quotep{\pi})$

\begin{eqnarray*}
  \freenames{\pzero} & := & \emptyset \\
  \freenames{x?(y).P} & := & \{ x \} \cup (\freenames{P} \setminus \{ y \}) \\
  \freenames{x!\langle P \rangle} & := & \{ x \} \cup \{ P \} \\
  \freenames{P|Q} & := & \freenames{P} \cup \freenames{Q} \\
  \freenames{\dropn{x}} & := & \{ x \}
\end{eqnarray*}

The bound names of a process, $\boundnames{P}$, are those names occurring in $P$
that are not free. For example, in $x?(y).0$, the name $x$ is free, while $y$ is bound.

\begin{mathpar}
  \inferrule* [lab=monoidal-laws] {} { P|Q \equiv Q|P \and P|0 \equiv P \and P|(Q|R) \equiv (P|Q)|R }
\end{mathpar}

\begin{mathpar}
  \inferrule* [lab=alpha-equivalence] {} { (x)P \equiv (y)P\{y/x\} \and y \not\in \freenames{P} }
\end{mathpar}

\begin{definition}
Then two processes, $P,Q$, are alpha-equivalent if $P = Q\{\vec{y}/\vec{x}\}$ for
some $\vec{x} \in \boundnames{Q},\vec{y} \in \boundnames{P}$, where $Q\{\vec{y}/\vec{x}\}$
denotes the capture-avoiding substitution of $\vec{y}$ for $\vec{x}$ in $Q$.
\end{definition}

\begin{definition}
  The {\em structural congruence} \cite{SangiorgiWalker} , $\equiv$,
  between processes is the least congruence containing
  alpha-equivalence, satisfying the abelian monoid laws
  (associativity, commutativity and $\pzero$ as identity) for parallel
  composition $|$ and for summation $+$.
\end{definition}

\subsection{Name equivalence}

We take name equivalence, written $\nameeq$, to be the smallest
equivalence relation generated by the following rules.

\begin{mathpar}
\inferrule*[lab=Quote-drop]
{ }
{ \quotep{@{x}} \nameeq x }

\inferrule*[lab=Struct-equiv]
{ P \scong Q }
{ \quotep{P} \nameeq \quotep{Q} }
\end{mathpar}

The astute reader will have noticed that the mutual recursion of names
and processes imposes a mutual recursion on alpha-equivalence and
structural equivalence via name-equivalence. Fortunately, all of this
works out pleasantly and we may calculate in the natural way, free of
concern. The reader interested in the details is referred to the
appendix \ref{appendix:rho_details}.

\subsection{Substitution}

We use $\Proc$ for the set of processes, $\QProc$ for the set of
names, and $\id{\{}\vec{y} / \vec{x} \id{\}}$ to denote partial maps,
$s : \QProc \rightarrow \QProc$. A map, $s$ lifts, uniquely, to a map
on process terms, $\widehat{s} : \Proc \rightarrow \Proc$ by the
following equations.

\begin{mathpar}
  (0) \psubstp{Q}{P} := 0 \\
  (R \juxtap S) \psubstp{Q}{P}
  :=    
  (R)\psubstp{Q}{P} \juxtap (S) \psubstp{Q}{P} \\
  (x?(y).R) \psubstp{Q}{P}    
  :=    
  (x)\substp{Q}{P} (z)\concat( (R \psubstn{z}{y}) \psubstp{Q}{P} ) \\
  (\lift{x}{R}) \psubstp{Q}{P}  
  :=
  \lift{(x)\substp{Q}{P}}{ R \psubstp{Q}{P} } \\
%   (\dropn{x})  \psubstp{Q}{P}       
%   := 
%   \left\{ 
%     \begin{array}{ccc} 
%       \dropn{\quotep{Q}} & & x \nameeq \quotep{P} \\
%       \dropn{x} & & otherwise \\
%     \end{array}
%   \right. 
  (\dropn{x})  \psubstp{Q}{P}       
  := 
  \left\{ 
    \begin{array}{ccc} 
      Q & & x \nameeq \quotep{P} \\
      \dropn{x} & & otherwise \\
    \end{array}
  \right.
\end{mathpar}
 

where

\begin{eqnarray}
  (x)\id{\{} \lpquote Q \rpquote / \lpquote P \rpquote \id{\}}            = 
  \left\{ 
    \begin{array}{ccc}
      \lpquote Q \rpquote & & x \nameeq \lpquote P \rpquote \\
      x & & otherwise \\
    \end{array}
  \right. \nonumber
\end{eqnarray}

and $z$ is chosen distinct from $\quotep{P}$, $\quotep{Q}$, the free
names in $Q$, and all the names in $R$. Our $\alpha$-equivalence will
be built in the standard way from this substitution.

\begin{remark}\label{rem:no_self_referential_names}
  One consequence of these definitions is that $\forall P. \quotep{P}
  \not\in \freenames{P}$.
\end{remark}

\subsection{ Dynamic quote: an example }

Anticipating something of what's to come, consider applying the
substitution, $\widehat{\id{\{}u / z \id{\}}}$, to the following pair
of processes, $\lift{w}{y!(z)}$ and $w[ \lpquote y!(z) \rpquote ]$.

\begin{eqnarray}
	\lift{w}{y!(z)}\widehat{\id{\{}u / z \id{\}}}
		& = &
		\lift{w}{y!(u)} \nonumber\\
	w[ \lpquote y!(z) \rpquote ] \widehat{ \id{\{}u / z \id{\}} }
		& = &
		w[ \lpquote y!(z) \rpquote ] \nonumber
\end{eqnarray}

Because the body of the process between quotes is impervious to
substitution, we get radically different answers. In fact, by
examining the first process in an input context,
e.g. $x?(z).\lift{w}{y!(z)}$, we see that the process under the lift
operator may be shaped by prefixed inputs binding a name inside it. In
this sense, the lift operator will be seen as a way to dynamically
construct processes before reifying them as names.

Finally equipped with these standard features we can present the
dynamics of the calculus.

\subsubsection{Operational semantics} 

Finally, we introduce the computational dynamics. What marks these
algebras as distinct from other more traditionally studied algebraic
structures, e.g. vector spaces or polynomial rings, is the manner in
which dynamics is captured. In traditional structures, dynamics is typically
expressed through morphisms between such structures, as in linear maps
between vector spaces or morphisms between rings. In algebras
associated with the semantics of computation, the dynamics is
expressed as part of the algebraic structure itself, through a
reduction reduction relation typically denoted by $\red$. Below, we
give a recursive presentation of this relation for the calculus used
in the encoding.

$\red \subseteq \pi \times \pi$
$\red : \pi \to \mathcal{P}(\pi)$

\begin{mathpar}
  \inferrule* [lab=Comm] { \textsf{match}( x_{src}, x_{trgt} ) } { x_{trgt}?(y)P \; | \; x_{src}!\langle {Q} \rangle \red P\{\quotep{Q}/y}\} }
  \and \\
  \inferrule* [lab=Par] {{P} \red {P}'} {{{P} | {Q}} \red {{P}' | {Q}}}
  \and
  \inferrule* [lab=Equiv]{{{P} \scong {P}'} \andalso {{P}' \red {Q}'} \andalso {{Q}' \scong {Q}}}{{P} \red {Q}}
\end{mathpar}

\begin{eqnarray*}
  match_{\equiv} (\quotep{P},\quotep{Q}) & := & P \equiv Q \\
  match_{\dagger}(\quotep{P},\quotep{Q}) & := & \forall R. P|Q \red^{*} R => R \red^{*} 0 \\
  match_{K}(\quotep{P},\quotep{Q}) & := & K \mbox{ for some context } K
\end{eqnarray*}

$u?(x)P | u!\langle Q \rangle \red P\{\quotep{Q}/x\}$

%We write $\wred$ for $\red^*$, and $P\red$ if $\exists Q $ such that $ P \red Q$.
We write $P\red$ if $\exists Q $ such that $ P \red Q$ and $P\not\red$, otherwise.

\section{Replication}

As mentioned before, it is known that replication (and hence
recursion) can be implemented in a higher-order process algebra
\cite{SangiorgiWalker}. As our first example of calculation with the
machinery thus far presented we give the construction explicitly in
the {\rhoc}.

\begin{eqnarray}
	D_{x} & := & \prefix{x}{y}{(\binpar{\outputp{x}{y}}{@{y}})} \nonumber\\
	\bangp_{x}{P} & := & \binpar{{x}!\langle{\binpar{D_{x}}{P}}\rangle}{D_{x}} \nonumber
\end{eqnarray}

\begin{eqnarray}
	\bangp_{x}{P} & & \nonumber\\
	=
	& {x}!\langle{(\prefix{x}{y}{(\outputp{x}{y} | @{y})) | P}}\rangle 
	      | \prefix{x}{y}{(\outputp{x}{y} | @{y})} & \nonumber\\
	\red
	& (\outputp{x}{y} | @{y})\substn{\quotep{(\prefix{x}{y}{(@{y} | \outputp{x}{y})) | P}}}{y} & \nonumber\\
	=
	& \outputp{x}{\quotep{(\prefix{x}{y}{(\outputp{x}{y} | @{y})) | P}}}
	  | {(\prefix{x}{y}{(\outputp{x}{y} | @{y})) | P}} & \nonumber\\
	\red
	& \ldots & \nonumber\\
	\red^*
	& P | P | \ldots & \nonumber
\end{eqnarray}

Of course, this encoding, as an implementation, runs away, unfolding
$\bangp{P}$ eagerly. A lazier and more implementable replication
operator, restricted to input-guarded processes, may be obtained as follows.

\begin{eqnarray}
\bangp{\prefix{u}{v}{P}} 
	:= 
	\binpar{\lift{x}{\prefix{u}{v}{(\binpar{D(x)}{P})}}}{D(x)} \nonumber
\end{eqnarray}

\begin{remark}
  Note that the lazier definition still does not deal with summation
  or mixed summation (i.e. sums over input and output). The reader is
  invited to construct definitions of replication that deal with these
  features. 

  Further, the definitions are parameterized in a name, $x$. Can you,
  gentle reader, make a definition that eliminates this parameter and
  guarantees no accidental interaction between the replication
  machinery and the process being replicated -- i.e. no accidental
  sharing of names used by the process to get its work done and the
  name(s) used by the replication to effect copying. This latter
  revision of the definition of replication is crucial to obtaining
  the expected identity $!!P \sim !P$.
\end{remark}

\begin{remark}\label{rem:paradoxical_combinator}
  The reader familiar with the lambda calculus will have noticed the
  similarity between $D$ and the paradoxical combinator.

  [Ed. note: the existence of this seems to suggest we have to be more
  restrictive on the set of processes and names we admit if we are to
  support no-cloning.]
\end{remark}

\subsubsection{Bisimulation}

The computational dynamics gives rise to another kind of equivalence,
the equivalence of computational behavior. As previously mentioned
this is typically captured \emph{via} some form of bisimulation.

% The notion we use in this paper is weak barbed bisimulation
% \cite{milner91polyadicpi}.

The notion we use in this paper is derived from weak barbed
bisimulation \cite{milner91polyadicpi}. 

\begin{definition}
An \emph{observation relation}, $\downarrow_{\mathcal N}$, over a set
of names, $\mathcal N$, is the smallest relation satisfying the rules
below.

\infrule[Out-barb]{y \in {\mathcal N}, \; x \nameeq y}
		  {\outputp{x}{v} \downarrow_{\mathcal N} x}
\infrule[Par-barb]{\mbox{$P\downarrow_{\mathcal N} x$ or $Q\downarrow_{\mathcal N} x$}}
		  {\binpar{P}{Q} \downarrow_{\mathcal N} x}

We write $P \Downarrow_{\mathcal N} x$ if there is $Q$ such that 
$P \wred Q$ and $Q \downarrow_{\mathcal N} x$.
\end{definition}

\begin{definition}
%\label{def.bbisim}
An  ${\mathcal N}$-\emph{barbed bisimulation} over a set of names, ${\mathcal N}$, is a symmetric binary relation 
${\mathcal S}_{\mathcal N}$ between agents such that $P\rel{S}_{\mathcal N}Q$ implies:
\begin{enumerate}
\item If $P \red P'$ then $Q \wred Q'$ and $P'\rel{S}_{\mathcal N} Q'$.
\item If $P\downarrow_{\mathcal N} x$, then $Q\Downarrow_{\mathcal N} x$.
\end{enumerate}
$P$ is ${\mathcal N}$-barbed bisimilar to $Q$, written
$P \wbbisim_{\mathcal N} Q$, if $P \rel{S}_{\mathcal N} Q$ for some ${\mathcal N}$-barbed bisimulation ${\mathcal S}_{\mathcal N}$.
\end{definition}

$\mathcal{R} \subseteq \pi \times \pi$

$P \mathcal{R} Q => \forall P'. P \red P' \Rightarrow \exists Q'. Q \red Q', P' \mathcal{R} Q'$

$P \vdash x \Rightarrow Q \vdash x$

\begin{mathpar}
  \inferrule*[lab=Out-barb]{x \nameeq y}{{y}!\langle{Q}\rangle \vdash x}
  \and
  \inferrule*[lab=Par-barb]{\mbox{$P\vdash x$ or $Q\vdash x$}}{\binpar{P}{Q} \vdash x}
\end{mathpar}

\subsubsection{Contexts}

One of the principle advantages of computational calculi like the
$\pi$-calculus is a well-defined notion of context,
contextual-equivalence and a correlation between
contextual-equivalence and notions of bisimulation. The notion of
context allows the decomposition of a process into (sub-)process and
its syntactic environment, its context. Thus, a context may be
thought of as a process with a ``hole'' (written $\Box$) in it. The
application of a context $M$ to a process $P$, written $M[P]$, is
tantamount to filling the hole in $M$ with $P$. In this paper we do
not need the full weight of this theory, but do make use of the notion
of context in the proof the main theorem. 

\begin{mathpar}
  \inferrule* [lab=summation] {} {{M_{M},M_{N}} \bc \Box \;|\; x.M_{A} \;|\; M_{M}+M_{N}}
  \and
  \inferrule* [lab=agent] {} {{M_{A}} \bc (\vec{x})M_{P} \;| \; \clift{P_0,\ldots,M_{P},\ldots,P_N}}
  \and \\
  \inferrule* [lab=process] {} {{M_{P}} \bc M_{N} \;| \;P|M_{P} }
\end{mathpar} 

\begin{mathpar}
  \inferrule* [lab=sychronization] {} {M_{N} \bc \Box \;|\; x?M_{F} \;|\; x!M_{C}}
  \and
  \inferrule* [lab=abstraction] {} {{M_{F}} \bc (x)M_{P} }
  \and
  \inferrule* [lab=concretion] {} {{M_{C}} \bc \langle M_{P} \rangle }
  \and \\
  \inferrule* [lab=process] {} {{M_{P}} \bc M_{N} \;| \;P|M_{P} }
\end{mathpar}

\begin{definition}[contextual application] Given a context $M$, and
  process $P$, we define the \emph{contextual application}, $M[P] :=
  M\{P/\Box\}$. That is, the contextual application of M to P is the
  substitution of $P$ for $\Box$ in $M$.
\end{definition}

$\meaningof{-} : L \to \mathcal{P}(\pi)$

\begin{mathpar}
  \inferrule* [lab=collection] {} {\meaningof{true} = \pi, \and \meaningof{~E} = \pi \setminus \meaningof{E}, \and \meaningof{E_{1} \& E_{2}} = \meaningof{E_{1}} \cap \meaningof{E_{2}}}
\end{mathpar}

\begin{mathpar}
  \inferrule* [lab=structure] {} {\meaningof{0} = \{ P \in \pi | P \equiv 0 \}, \and \\ \meaningof{E_1 | E_2} = \{ P \in \pi | P \equiv P_{1} | P_{2}, P_{1} \in \meaningof{E_{1}}, P_{2} \in \meaningof{E_2}\} }
\end{mathpar}

\begin{mathpar}
 \inferrule* [lab=behavior] {} {\meaningof{\langle a?b \rangle E} = \{ P \in \pi | P \equiv Q | u?(y)P', \\ \and \\\\ \and \\ \;\;\; u \in \meaningof{a}, \forall z.P'\{z/y\} \in \meaningof{E\{z/b\}}\}, \and \\ \meaningof{a!E} = \{ P \in \pi | P \equiv Q | x!\langle P' \rangle, x \in \meaningof{a} P' \in \meaningof{E}\} }
\end{mathpar}

\begin{mathpar}
 \inferrule* [lab=nominal] {} {\meaningof{\quotep{E}} = \{ \quotep{P} \in \quotep{\pi} | P \in \meaningof{E} \}, \and \meaningof{\quotep{P}} = \{ \quotep{Q} \in \quotep{\pi} | P \equiv Q \} \and \\ \meaningof{@\quotep{E}} = \{ P \in \pi | P \equiv @x, x \in \meaningof{E} \}}
\end{mathpar}

\begin{eqnarray*}
  \\
  \meaningof{-} : TS \to ST
\end{eqnarray*}

\begin{eqnarray*}
  \\
  L : TS \to ST
\end{eqnarray*}

\begin{eqnarray*}
  \\
  P \models E \iff P \in \meaningof{E}
\end{eqnarray*}

\begin{eqnarray*}
  P \approx_{L} Q \iff \forall E \in L. P \models E \iff Q \models E
\end{eqnarray*}

\begin{eqnarray*}
  P \approx_{K} Q
\end{eqnarray*}

\begin{eqnarray*}
  P \approx Q
\end{eqnarray*}

$\approx_{K} = \approx = \approx_{L}$

\subsubsection{Contextual duality}

Note that contexts extend the quotation operation to a family of
operations from processes to names. Given a context, $M$, we can
define a \emph{nominal context}, $\quotep{M}$ by $\quotep{M}[P] :=
\quotep{M[P]}$. To foreshadow what is to come we observe that these
operations enjoy a duality with processes very much like the duality
between vectors and maps from vectors to scalars.

Further, because the calculus is essentially higher-order, we have a
correspondence between contexts and processes. More specifically,
given a name $x$ and a context $M$ we can construct $M^{*}_{x}$ such
that 

\begin{mathpar}
  M^{*}_{x} | \lift{x}{P} \red M[P]
\end{mathpar}

namely,

\begin{mathpar}
  M^{*}_{x} := x?(u).M[\dropn{u}]
\end{mathpar}

The dependence of $M^{*}_{x}$ on a name makes it an abstraction, 

\begin{mathpar}
  M^{*} := (x)x?(u).M[\dropn{u}]
\end{mathpar}

\subsection{Additional notation}

It will sometimes be convenient to denote the process a name
quotes. We already have the notation $x = \quotep{P}$, but it will be
convenient to introduce an alternate notation, $\procn{x}$, when we
want to emphasize the connection to the use of the name. Note that, by
virtue of name equivalence, $\quotep{\procn{x}} \nameeq x$; so, the
notation is consistent with previous definitions.

Further, because names have structure it is possible to effect
substitutions on the basis of that structure. This means we need to
upgrade our notation for substitutions, which we accomplish by
adapting comprehension notation. Thus,

\begin{mathpar}
  P\{ y / x : x \in S \}
\end{mathpar}

is interpreted to mean the process derived from P by replacing (in a
capture-avoiding manner) each occurrence of $x$ in $S$ by $y$. For example,

\begin{mathpar}
  P\{ \quotep{\procn{x}|\procn{x}} / x : x \in \freenames{P} \}
\end{mathpar}

will replace each (occurrence) of a free name $x$ in $P$ by
$\quotep{\procn{x}|\procn{x}}$.

Also, we will avail ourselves of the notation $x^{L}$ and $x^{R}$ to
denote injections of a name into disjoint copies of the name
space. There are numerous ways to accomplish this. One example can be
found in \cite{MeredithR05}. This notation overloads to vectors of
names: $\vec{x}^{\pi} := (x_{i}^{\pi} \; : \; 0 \leq i < |\vec{x}| )$ where $\pi \in \{L,R\}$.

We also use $P^{\Box} := P|\Box$.

In \cite{MeredithR05} an interpretation of the new operator is
given. It turns out that there are several possible interpretations
all enjoying the requisite algebraic properties of the operator (see
\cite{milner91polyadicpi}). We will therefore make liberal use of
$(\nu\; \vec{x})P$.

% subsection the_syntax_and_semantics_of_the_notation_system (end)   

\input{qm2pi.qmops} 

\input{qm2pi.sterngerlach} 

\input{qm2pi.metric} 

% section concurrent_process_calculi (end)

%\input{qm2pi.proofsketch}

% section proof sketch (end)

%\input{qm2pi.slviaknots} 

% section spatial logic via knots (end)

\input{qm2pi.conclusion}

% section conclusion (end)

%\input{qm2pi.dtcodes} 

% section wiring algorithm (end)

\input{qm2pi.ack} 

% section acknowledgments (end)

\newpage


\bibliographystyle{plain}   
\bibliography{../../biblios/main.bib}

\input{qm2pi.rhodetails}

\end{document}

 

% section wiring algorithm (end)

\documentclass[12pt]{llncs}
%\documentclass{jktr}

\usepackage[pdftex]{hyperref}                   
\usepackage {listings}
\usepackage {mathpartir}
\usepackage{bcprules}
%\usepackage{listings}
                       
\usepackage{graphicx} 
%\usepackage[margins=2.5cm,nohead,nofoot]{geometry}
%\usepackage{geometry}
\usepackage{amsfonts}
\usepackage{amstext}
\usepackage{latexsym}
\usepackage{amssymb}
\usepackage{color}


%\include{myPreamble}
\include{qm2pi.local} 

%\ifpdf
%\usepackage[pdftex]{graphicx}
%\else
%\usepackage{graphicx}
%\fi

 % \ifpdf
%  \usepackage{pdfsync}
%  \if


%\title{Brief Article}
%\author{David F. Snyder}
%\author{L.G. Meredith}

%\address{Dept. of Math., Texas State University--San Marcos, San Marcos, TX 78666}
       
\pagestyle{empty}


\begin{document}

\lstset{language=[Objective]Caml,frame=shadowbox}

\input{qm2pi.front}

% section front matter (end)

\input{qm2pi.intro} 
 
% section introduction (end)

% \input{qm2pi.knotations} 

% section notation (end)

\input{qm2pi.process.calculi} 

% section concurrent_process_calculi_and_spatial_logics_ (end)
    
%\input{qm2pi.knots2pi} 

%\input{qm2pi.trefoil} 

%\input{qm2pi.mainthm} 

% subsection basic_interpretation (end)

%\input{qm2pi.rho.presentation} 
\subsection{The syntax and semantics of the notation system}\label{sub:the_syntax_and_semantics_of_the_notation_system} % (fold)

We now summarize a technical presentation of the calculus that
embodies our theory of dynamics. The typical presentation of such a
calculus follows the style of giving generators and relations on
them. The grammar, below, describing term constructors, freely
generates the set of processes, $\Proc$. This set is then quotiented
by a relation known as structural congruence and it is over this set
that the notion of dynamics is expressed. This presentation is
essentially that of \cite{MeredithR05} with the addition of
polyadicity and summation. For readability we have relegated some of
the technical subtleties to an appendix.

\subsubsection{Process grammar}\label{subsub:process_grammar}

\begin{mathpar}
  \inferrule* [lab=synchronization] {} {{M} \bc \pzero \;|\; x?F \;|\; x!C }
  \and
  \inferrule* [lab=abstraction] {} {{F} \bc (x)P}
  \and
  \inferrule* [lab=concretion] {} {{C} \bc \langle Q \rangle}
  \and
  \inferrule* [lab=process] {} {{P,Q} \bc M \;| \;P|Q \;|\; @{x}}
  \and
  \inferrule* [lab=name] {} {{x} \bc \quotep{P}}
\end{mathpar} 

Note that $\vec{x}$ (resp. $\vec{P}$) denotes a vector of names
(resp. processes) of length $|\vec{x}|$ (resp. $|\vec{P}|$). We adopt
the following useful abbreviations.

\begin{mathpar}
   x?(\vec{y}).P := x.(\vec{y})P \and  x\clift{\vec{P}} := x.\clift{\vec{P}}
   \and x!(y) := \lift{x}{\dropn{y}}
   \and \Pi_{i=0}^{n-1}P_i := P_0 | \ldots | P_{n-1}
\end{mathpar}

\subsubsection{Structural congruence}

\paragraph{Free and bound names and alpha-equivalence.} At the
core of structural equivalence is alpha-equivalence which identifies
process that are the same up to a change of variable. Formally, we
recognize the distinction between free and bound names. The free names
of a process, $\freenames{P}$, may be calculated recursively as
follows:

\begin{mathpar}
\freenames{\pzero} := \emptyset
  \and \\
  \freenames{x?(y).P} := \{ x \} \cup (\freenames{P} \setminus \{ y \})
  \and 
  \freenames{x!\langle P \rangle} := \{ x \} \cup \{ P \} 
  \and \\
  \freenames{P|Q} := \freenames{P} \cup \freenames{Q}
  \and \\
  \freenames{@{x}} := \{ x \}
\end{mathpar}

$\pi$
$\quotep{\pi}$

$\freenames{-} : \pi \to \mathcal{P}(\quotep{\pi})$

\begin{eqnarray*}
  \freenames{\pzero} & := & \emptyset \\
  \freenames{x?(y).P} & := & \{ x \} \cup (\freenames{P} \setminus \{ y \}) \\
  \freenames{x!\langle P \rangle} & := & \{ x \} \cup \{ P \} \\
  \freenames{P|Q} & := & \freenames{P} \cup \freenames{Q} \\
  \freenames{\dropn{x}} & := & \{ x \}
\end{eqnarray*}

The bound names of a process, $\boundnames{P}$, are those names occurring in $P$
that are not free. For example, in $x?(y).0$, the name $x$ is free, while $y$ is bound.

\begin{mathpar}
  \inferrule* [lab=monoidal-laws] {} { P|Q \equiv Q|P \and P|0 \equiv P \and P|(Q|R) \equiv (P|Q)|R }
\end{mathpar}

\begin{mathpar}
  \inferrule* [lab=alpha-equivalence] {} { (x)P \equiv (y)P\{y/x\} \and y \not\in \freenames{P} }
\end{mathpar}

\begin{definition}
Then two processes, $P,Q$, are alpha-equivalent if $P = Q\{\vec{y}/\vec{x}\}$ for
some $\vec{x} \in \boundnames{Q},\vec{y} \in \boundnames{P}$, where $Q\{\vec{y}/\vec{x}\}$
denotes the capture-avoiding substitution of $\vec{y}$ for $\vec{x}$ in $Q$.
\end{definition}

\begin{definition}
  The {\em structural congruence} \cite{SangiorgiWalker} , $\equiv$,
  between processes is the least congruence containing
  alpha-equivalence, satisfying the abelian monoid laws
  (associativity, commutativity and $\pzero$ as identity) for parallel
  composition $|$ and for summation $+$.
\end{definition}

\subsection{Name equivalence}

We take name equivalence, written $\nameeq$, to be the smallest
equivalence relation generated by the following rules.

\begin{mathpar}
\inferrule*[lab=Quote-drop]
{ }
{ \quotep{@{x}} \nameeq x }

\inferrule*[lab=Struct-equiv]
{ P \scong Q }
{ \quotep{P} \nameeq \quotep{Q} }
\end{mathpar}

The astute reader will have noticed that the mutual recursion of names
and processes imposes a mutual recursion on alpha-equivalence and
structural equivalence via name-equivalence. Fortunately, all of this
works out pleasantly and we may calculate in the natural way, free of
concern. The reader interested in the details is referred to the
appendix \ref{appendix:rho_details}.

\subsection{Substitution}

We use $\Proc$ for the set of processes, $\QProc$ for the set of
names, and $\id{\{}\vec{y} / \vec{x} \id{\}}$ to denote partial maps,
$s : \QProc \rightarrow \QProc$. A map, $s$ lifts, uniquely, to a map
on process terms, $\widehat{s} : \Proc \rightarrow \Proc$ by the
following equations.

\begin{mathpar}
  (0) \psubstp{Q}{P} := 0 \\
  (R \juxtap S) \psubstp{Q}{P}
  :=    
  (R)\psubstp{Q}{P} \juxtap (S) \psubstp{Q}{P} \\
  (x?(y).R) \psubstp{Q}{P}    
  :=    
  (x)\substp{Q}{P} (z)\concat( (R \psubstn{z}{y}) \psubstp{Q}{P} ) \\
  (\lift{x}{R}) \psubstp{Q}{P}  
  :=
  \lift{(x)\substp{Q}{P}}{ R \psubstp{Q}{P} } \\
%   (\dropn{x})  \psubstp{Q}{P}       
%   := 
%   \left\{ 
%     \begin{array}{ccc} 
%       \dropn{\quotep{Q}} & & x \nameeq \quotep{P} \\
%       \dropn{x} & & otherwise \\
%     \end{array}
%   \right. 
  (\dropn{x})  \psubstp{Q}{P}       
  := 
  \left\{ 
    \begin{array}{ccc} 
      Q & & x \nameeq \quotep{P} \\
      \dropn{x} & & otherwise \\
    \end{array}
  \right.
\end{mathpar}
 

where

\begin{eqnarray}
  (x)\id{\{} \lpquote Q \rpquote / \lpquote P \rpquote \id{\}}            = 
  \left\{ 
    \begin{array}{ccc}
      \lpquote Q \rpquote & & x \nameeq \lpquote P \rpquote \\
      x & & otherwise \\
    \end{array}
  \right. \nonumber
\end{eqnarray}

and $z$ is chosen distinct from $\quotep{P}$, $\quotep{Q}$, the free
names in $Q$, and all the names in $R$. Our $\alpha$-equivalence will
be built in the standard way from this substitution.

\begin{remark}\label{rem:no_self_referential_names}
  One consequence of these definitions is that $\forall P. \quotep{P}
  \not\in \freenames{P}$.
\end{remark}

\subsection{ Dynamic quote: an example }

Anticipating something of what's to come, consider applying the
substitution, $\widehat{\id{\{}u / z \id{\}}}$, to the following pair
of processes, $\lift{w}{y!(z)}$ and $w[ \lpquote y!(z) \rpquote ]$.

\begin{eqnarray}
	\lift{w}{y!(z)}\widehat{\id{\{}u / z \id{\}}}
		& = &
		\lift{w}{y!(u)} \nonumber\\
	w[ \lpquote y!(z) \rpquote ] \widehat{ \id{\{}u / z \id{\}} }
		& = &
		w[ \lpquote y!(z) \rpquote ] \nonumber
\end{eqnarray}

Because the body of the process between quotes is impervious to
substitution, we get radically different answers. In fact, by
examining the first process in an input context,
e.g. $x?(z).\lift{w}{y!(z)}$, we see that the process under the lift
operator may be shaped by prefixed inputs binding a name inside it. In
this sense, the lift operator will be seen as a way to dynamically
construct processes before reifying them as names.

Finally equipped with these standard features we can present the
dynamics of the calculus.

\subsubsection{Operational semantics} 

Finally, we introduce the computational dynamics. What marks these
algebras as distinct from other more traditionally studied algebraic
structures, e.g. vector spaces or polynomial rings, is the manner in
which dynamics is captured. In traditional structures, dynamics is typically
expressed through morphisms between such structures, as in linear maps
between vector spaces or morphisms between rings. In algebras
associated with the semantics of computation, the dynamics is
expressed as part of the algebraic structure itself, through a
reduction reduction relation typically denoted by $\red$. Below, we
give a recursive presentation of this relation for the calculus used
in the encoding.

$\red \subseteq \pi \times \pi$
$\red : \pi \to \mathcal{P}(\pi)$

\begin{mathpar}
  \inferrule* [lab=Comm] { \textsf{match}( x_{src}, x_{trgt} ) } { x_{trgt}?(y)P \; | \; x_{src}!\langle {Q} \rangle \red P\{\quotep{Q}/y}\} }
  \and \\
  \inferrule* [lab=Par] {{P} \red {P}'} {{{P} | {Q}} \red {{P}' | {Q}}}
  \and
  \inferrule* [lab=Equiv]{{{P} \scong {P}'} \andalso {{P}' \red {Q}'} \andalso {{Q}' \scong {Q}}}{{P} \red {Q}}
\end{mathpar}

\begin{eqnarray*}
  match_{\equiv} (\quotep{P},\quotep{Q}) & := & P \equiv Q \\
  match_{\dagger}(\quotep{P},\quotep{Q}) & := & \forall R. P|Q \red^{*} R => R \red^{*} 0 \\
  match_{K}(\quotep{P},\quotep{Q}) & := & K \mbox{ for some context } K
\end{eqnarray*}

$u?(x)P | u!\langle Q \rangle \red P\{\quotep{Q}/x\}$

%We write $\wred$ for $\red^*$, and $P\red$ if $\exists Q $ such that $ P \red Q$.
We write $P\red$ if $\exists Q $ such that $ P \red Q$ and $P\not\red$, otherwise.

\section{Replication}

As mentioned before, it is known that replication (and hence
recursion) can be implemented in a higher-order process algebra
\cite{SangiorgiWalker}. As our first example of calculation with the
machinery thus far presented we give the construction explicitly in
the {\rhoc}.

\begin{eqnarray}
	D_{x} & := & \prefix{x}{y}{(\binpar{\outputp{x}{y}}{@{y}})} \nonumber\\
	\bangp_{x}{P} & := & \binpar{{x}!\langle{\binpar{D_{x}}{P}}\rangle}{D_{x}} \nonumber
\end{eqnarray}

\begin{eqnarray}
	\bangp_{x}{P} & & \nonumber\\
	=
	& {x}!\langle{(\prefix{x}{y}{(\outputp{x}{y} | @{y})) | P}}\rangle 
	      | \prefix{x}{y}{(\outputp{x}{y} | @{y})} & \nonumber\\
	\red
	& (\outputp{x}{y} | @{y})\substn{\quotep{(\prefix{x}{y}{(@{y} | \outputp{x}{y})) | P}}}{y} & \nonumber\\
	=
	& \outputp{x}{\quotep{(\prefix{x}{y}{(\outputp{x}{y} | @{y})) | P}}}
	  | {(\prefix{x}{y}{(\outputp{x}{y} | @{y})) | P}} & \nonumber\\
	\red
	& \ldots & \nonumber\\
	\red^*
	& P | P | \ldots & \nonumber
\end{eqnarray}

Of course, this encoding, as an implementation, runs away, unfolding
$\bangp{P}$ eagerly. A lazier and more implementable replication
operator, restricted to input-guarded processes, may be obtained as follows.

\begin{eqnarray}
\bangp{\prefix{u}{v}{P}} 
	:= 
	\binpar{\lift{x}{\prefix{u}{v}{(\binpar{D(x)}{P})}}}{D(x)} \nonumber
\end{eqnarray}

\begin{remark}
  Note that the lazier definition still does not deal with summation
  or mixed summation (i.e. sums over input and output). The reader is
  invited to construct definitions of replication that deal with these
  features. 

  Further, the definitions are parameterized in a name, $x$. Can you,
  gentle reader, make a definition that eliminates this parameter and
  guarantees no accidental interaction between the replication
  machinery and the process being replicated -- i.e. no accidental
  sharing of names used by the process to get its work done and the
  name(s) used by the replication to effect copying. This latter
  revision of the definition of replication is crucial to obtaining
  the expected identity $!!P \sim !P$.
\end{remark}

\begin{remark}\label{rem:paradoxical_combinator}
  The reader familiar with the lambda calculus will have noticed the
  similarity between $D$ and the paradoxical combinator.

  [Ed. note: the existence of this seems to suggest we have to be more
  restrictive on the set of processes and names we admit if we are to
  support no-cloning.]
\end{remark}

\subsubsection{Bisimulation}

The computational dynamics gives rise to another kind of equivalence,
the equivalence of computational behavior. As previously mentioned
this is typically captured \emph{via} some form of bisimulation.

% The notion we use in this paper is weak barbed bisimulation
% \cite{milner91polyadicpi}.

The notion we use in this paper is derived from weak barbed
bisimulation \cite{milner91polyadicpi}. 

\begin{definition}
An \emph{observation relation}, $\downarrow_{\mathcal N}$, over a set
of names, $\mathcal N$, is the smallest relation satisfying the rules
below.

\infrule[Out-barb]{y \in {\mathcal N}, \; x \nameeq y}
		  {\outputp{x}{v} \downarrow_{\mathcal N} x}
\infrule[Par-barb]{\mbox{$P\downarrow_{\mathcal N} x$ or $Q\downarrow_{\mathcal N} x$}}
		  {\binpar{P}{Q} \downarrow_{\mathcal N} x}

We write $P \Downarrow_{\mathcal N} x$ if there is $Q$ such that 
$P \wred Q$ and $Q \downarrow_{\mathcal N} x$.
\end{definition}

\begin{definition}
%\label{def.bbisim}
An  ${\mathcal N}$-\emph{barbed bisimulation} over a set of names, ${\mathcal N}$, is a symmetric binary relation 
${\mathcal S}_{\mathcal N}$ between agents such that $P\rel{S}_{\mathcal N}Q$ implies:
\begin{enumerate}
\item If $P \red P'$ then $Q \wred Q'$ and $P'\rel{S}_{\mathcal N} Q'$.
\item If $P\downarrow_{\mathcal N} x$, then $Q\Downarrow_{\mathcal N} x$.
\end{enumerate}
$P$ is ${\mathcal N}$-barbed bisimilar to $Q$, written
$P \wbbisim_{\mathcal N} Q$, if $P \rel{S}_{\mathcal N} Q$ for some ${\mathcal N}$-barbed bisimulation ${\mathcal S}_{\mathcal N}$.
\end{definition}

$\mathcal{R} \subseteq \pi \times \pi$

$P \mathcal{R} Q => \forall P'. P \red P' \Rightarrow \exists Q'. Q \red Q', P' \mathcal{R} Q'$

$P \vdash x \Rightarrow Q \vdash x$

\begin{mathpar}
  \inferrule*[lab=Out-barb]{x \nameeq y}{{y}!\langle{Q}\rangle \vdash x}
  \and
  \inferrule*[lab=Par-barb]{\mbox{$P\vdash x$ or $Q\vdash x$}}{\binpar{P}{Q} \vdash x}
\end{mathpar}

\subsubsection{Contexts}

One of the principle advantages of computational calculi like the
$\pi$-calculus is a well-defined notion of context,
contextual-equivalence and a correlation between
contextual-equivalence and notions of bisimulation. The notion of
context allows the decomposition of a process into (sub-)process and
its syntactic environment, its context. Thus, a context may be
thought of as a process with a ``hole'' (written $\Box$) in it. The
application of a context $M$ to a process $P$, written $M[P]$, is
tantamount to filling the hole in $M$ with $P$. In this paper we do
not need the full weight of this theory, but do make use of the notion
of context in the proof the main theorem. 

\begin{mathpar}
  \inferrule* [lab=summation] {} {{M_{M},M_{N}} \bc \Box \;|\; x.M_{A} \;|\; M_{M}+M_{N}}
  \and
  \inferrule* [lab=agent] {} {{M_{A}} \bc (\vec{x})M_{P} \;| \; \clift{P_0,\ldots,M_{P},\ldots,P_N}}
  \and \\
  \inferrule* [lab=process] {} {{M_{P}} \bc M_{N} \;| \;P|M_{P} }
\end{mathpar} 

\begin{mathpar}
  \inferrule* [lab=sychronization] {} {M_{N} \bc \Box \;|\; x?M_{F} \;|\; x!M_{C}}
  \and
  \inferrule* [lab=abstraction] {} {{M_{F}} \bc (x)M_{P} }
  \and
  \inferrule* [lab=concretion] {} {{M_{C}} \bc \langle M_{P} \rangle }
  \and \\
  \inferrule* [lab=process] {} {{M_{P}} \bc M_{N} \;| \;P|M_{P} }
\end{mathpar}

\begin{definition}[contextual application] Given a context $M$, and
  process $P$, we define the \emph{contextual application}, $M[P] :=
  M\{P/\Box\}$. That is, the contextual application of M to P is the
  substitution of $P$ for $\Box$ in $M$.
\end{definition}

$\meaningof{-} : L \to \mathcal{P}(\pi)$

\begin{mathpar}
  \inferrule* [lab=collection] {} {\meaningof{true} = \pi, \and \meaningof{~E} = \pi \setminus \meaningof{E}, \and \meaningof{E_{1} \& E_{2}} = \meaningof{E_{1}} \cap \meaningof{E_{2}}}
\end{mathpar}

\begin{mathpar}
  \inferrule* [lab=structure] {} {\meaningof{0} = \{ P \in \pi | P \equiv 0 \}, \and \\ \meaningof{E_1 | E_2} = \{ P \in \pi | P \equiv P_{1} | P_{2}, P_{1} \in \meaningof{E_{1}}, P_{2} \in \meaningof{E_2}\} }
\end{mathpar}

\begin{mathpar}
 \inferrule* [lab=behavior] {} {\meaningof{\langle a?b \rangle E} = \{ P \in \pi | P \equiv Q | u?(y)P', \\ \and \\\\ \and \\ \;\;\; u \in \meaningof{a}, \forall z.P'\{z/y\} \in \meaningof{E\{z/b\}}\}, \and \\ \meaningof{a!E} = \{ P \in \pi | P \equiv Q | x!\langle P' \rangle, x \in \meaningof{a} P' \in \meaningof{E}\} }
\end{mathpar}

\begin{mathpar}
 \inferrule* [lab=nominal] {} {\meaningof{\quotep{E}} = \{ \quotep{P} \in \quotep{\pi} | P \in \meaningof{E} \}, \and \meaningof{\quotep{P}} = \{ \quotep{Q} \in \quotep{\pi} | P \equiv Q \} \and \\ \meaningof{@\quotep{E}} = \{ P \in \pi | P \equiv @x, x \in \meaningof{E} \}}
\end{mathpar}

\begin{eqnarray*}
  \\
  \meaningof{-} : TS \to ST
\end{eqnarray*}

\begin{eqnarray*}
  \\
  L : TS \to ST
\end{eqnarray*}

\begin{eqnarray*}
  \\
  P \models E \iff P \in \meaningof{E}
\end{eqnarray*}

\begin{eqnarray*}
  P \approx_{L} Q \iff \forall E \in L. P \models E \iff Q \models E
\end{eqnarray*}

\begin{eqnarray*}
  P \approx_{K} Q
\end{eqnarray*}

\begin{eqnarray*}
  P \approx Q
\end{eqnarray*}

$\approx_{K} = \approx = \approx_{L}$

\subsubsection{Contextual duality}

Note that contexts extend the quotation operation to a family of
operations from processes to names. Given a context, $M$, we can
define a \emph{nominal context}, $\quotep{M}$ by $\quotep{M}[P] :=
\quotep{M[P]}$. To foreshadow what is to come we observe that these
operations enjoy a duality with processes very much like the duality
between vectors and maps from vectors to scalars.

Further, because the calculus is essentially higher-order, we have a
correspondence between contexts and processes. More specifically,
given a name $x$ and a context $M$ we can construct $M^{*}_{x}$ such
that 

\begin{mathpar}
  M^{*}_{x} | \lift{x}{P} \red M[P]
\end{mathpar}

namely,

\begin{mathpar}
  M^{*}_{x} := x?(u).M[\dropn{u}]
\end{mathpar}

The dependence of $M^{*}_{x}$ on a name makes it an abstraction, 

\begin{mathpar}
  M^{*} := (x)x?(u).M[\dropn{u}]
\end{mathpar}

\subsection{Additional notation}

It will sometimes be convenient to denote the process a name
quotes. We already have the notation $x = \quotep{P}$, but it will be
convenient to introduce an alternate notation, $\procn{x}$, when we
want to emphasize the connection to the use of the name. Note that, by
virtue of name equivalence, $\quotep{\procn{x}} \nameeq x$; so, the
notation is consistent with previous definitions.

Further, because names have structure it is possible to effect
substitutions on the basis of that structure. This means we need to
upgrade our notation for substitutions, which we accomplish by
adapting comprehension notation. Thus,

\begin{mathpar}
  P\{ y / x : x \in S \}
\end{mathpar}

is interpreted to mean the process derived from P by replacing (in a
capture-avoiding manner) each occurrence of $x$ in $S$ by $y$. For example,

\begin{mathpar}
  P\{ \quotep{\procn{x}|\procn{x}} / x : x \in \freenames{P} \}
\end{mathpar}

will replace each (occurrence) of a free name $x$ in $P$ by
$\quotep{\procn{x}|\procn{x}}$.

Also, we will avail ourselves of the notation $x^{L}$ and $x^{R}$ to
denote injections of a name into disjoint copies of the name
space. There are numerous ways to accomplish this. One example can be
found in \cite{MeredithR05}. This notation overloads to vectors of
names: $\vec{x}^{\pi} := (x_{i}^{\pi} \; : \; 0 \leq i < |\vec{x}| )$ where $\pi \in \{L,R\}$.

We also use $P^{\Box} := P|\Box$.

In \cite{MeredithR05} an interpretation of the new operator is
given. It turns out that there are several possible interpretations
all enjoying the requisite algebraic properties of the operator (see
\cite{milner91polyadicpi}). We will therefore make liberal use of
$(\nu\; \vec{x})P$.

% subsection the_syntax_and_semantics_of_the_notation_system (end)   

\input{qm2pi.qmops} 

\input{qm2pi.sterngerlach} 

\input{qm2pi.metric} 

% section concurrent_process_calculi (end)

%\input{qm2pi.proofsketch}

% section proof sketch (end)

%\input{qm2pi.slviaknots} 

% section spatial logic via knots (end)

\input{qm2pi.conclusion}

% section conclusion (end)

%\input{qm2pi.dtcodes} 

% section wiring algorithm (end)

\input{qm2pi.ack} 

% section acknowledgments (end)

\newpage


\bibliographystyle{plain}   
\bibliography{../../biblios/main.bib}

\input{qm2pi.rhodetails}

\end{document}

 

% section acknowledgments (end)

\newpage


\bibliographystyle{plain}   
\bibliography{../../biblios/main.bib}

\documentclass[12pt]{llncs}
%\documentclass{jktr}

\usepackage[pdftex]{hyperref}                   
\usepackage {listings}
\usepackage {mathpartir}
\usepackage{bcprules}
%\usepackage{listings}
                       
\usepackage{graphicx} 
%\usepackage[margins=2.5cm,nohead,nofoot]{geometry}
%\usepackage{geometry}
\usepackage{amsfonts}
\usepackage{amstext}
\usepackage{latexsym}
\usepackage{amssymb}
\usepackage{color}


%\include{myPreamble}
\include{qm2pi.local} 

%\ifpdf
%\usepackage[pdftex]{graphicx}
%\else
%\usepackage{graphicx}
%\fi

 % \ifpdf
%  \usepackage{pdfsync}
%  \if


%\title{Brief Article}
%\author{David F. Snyder}
%\author{L.G. Meredith}

%\address{Dept. of Math., Texas State University--San Marcos, San Marcos, TX 78666}
       
\pagestyle{empty}


\begin{document}

\lstset{language=[Objective]Caml,frame=shadowbox}

\input{qm2pi.front}

% section front matter (end)

\input{qm2pi.intro} 
 
% section introduction (end)

% \input{qm2pi.knotations} 

% section notation (end)

\input{qm2pi.process.calculi} 

% section concurrent_process_calculi_and_spatial_logics_ (end)
    
%\input{qm2pi.knots2pi} 

%\input{qm2pi.trefoil} 

%\input{qm2pi.mainthm} 

% subsection basic_interpretation (end)

%\input{qm2pi.rho.presentation} 
\subsection{The syntax and semantics of the notation system}\label{sub:the_syntax_and_semantics_of_the_notation_system} % (fold)

We now summarize a technical presentation of the calculus that
embodies our theory of dynamics. The typical presentation of such a
calculus follows the style of giving generators and relations on
them. The grammar, below, describing term constructors, freely
generates the set of processes, $\Proc$. This set is then quotiented
by a relation known as structural congruence and it is over this set
that the notion of dynamics is expressed. This presentation is
essentially that of \cite{MeredithR05} with the addition of
polyadicity and summation. For readability we have relegated some of
the technical subtleties to an appendix.

\subsubsection{Process grammar}\label{subsub:process_grammar}

\begin{mathpar}
  \inferrule* [lab=synchronization] {} {{M} \bc \pzero \;|\; x?F \;|\; x!C }
  \and
  \inferrule* [lab=abstraction] {} {{F} \bc (x)P}
  \and
  \inferrule* [lab=concretion] {} {{C} \bc \langle Q \rangle}
  \and
  \inferrule* [lab=process] {} {{P,Q} \bc M \;| \;P|Q \;|\; @{x}}
  \and
  \inferrule* [lab=name] {} {{x} \bc \quotep{P}}
\end{mathpar} 

Note that $\vec{x}$ (resp. $\vec{P}$) denotes a vector of names
(resp. processes) of length $|\vec{x}|$ (resp. $|\vec{P}|$). We adopt
the following useful abbreviations.

\begin{mathpar}
   x?(\vec{y}).P := x.(\vec{y})P \and  x\clift{\vec{P}} := x.\clift{\vec{P}}
   \and x!(y) := \lift{x}{\dropn{y}}
   \and \Pi_{i=0}^{n-1}P_i := P_0 | \ldots | P_{n-1}
\end{mathpar}

\subsubsection{Structural congruence}

\paragraph{Free and bound names and alpha-equivalence.} At the
core of structural equivalence is alpha-equivalence which identifies
process that are the same up to a change of variable. Formally, we
recognize the distinction between free and bound names. The free names
of a process, $\freenames{P}$, may be calculated recursively as
follows:

\begin{mathpar}
\freenames{\pzero} := \emptyset
  \and \\
  \freenames{x?(y).P} := \{ x \} \cup (\freenames{P} \setminus \{ y \})
  \and 
  \freenames{x!\langle P \rangle} := \{ x \} \cup \{ P \} 
  \and \\
  \freenames{P|Q} := \freenames{P} \cup \freenames{Q}
  \and \\
  \freenames{@{x}} := \{ x \}
\end{mathpar}

$\pi$
$\quotep{\pi}$

$\freenames{-} : \pi \to \mathcal{P}(\quotep{\pi})$

\begin{eqnarray*}
  \freenames{\pzero} & := & \emptyset \\
  \freenames{x?(y).P} & := & \{ x \} \cup (\freenames{P} \setminus \{ y \}) \\
  \freenames{x!\langle P \rangle} & := & \{ x \} \cup \{ P \} \\
  \freenames{P|Q} & := & \freenames{P} \cup \freenames{Q} \\
  \freenames{\dropn{x}} & := & \{ x \}
\end{eqnarray*}

The bound names of a process, $\boundnames{P}$, are those names occurring in $P$
that are not free. For example, in $x?(y).0$, the name $x$ is free, while $y$ is bound.

\begin{mathpar}
  \inferrule* [lab=monoidal-laws] {} { P|Q \equiv Q|P \and P|0 \equiv P \and P|(Q|R) \equiv (P|Q)|R }
\end{mathpar}

\begin{mathpar}
  \inferrule* [lab=alpha-equivalence] {} { (x)P \equiv (y)P\{y/x\} \and y \not\in \freenames{P} }
\end{mathpar}

\begin{definition}
Then two processes, $P,Q$, are alpha-equivalent if $P = Q\{\vec{y}/\vec{x}\}$ for
some $\vec{x} \in \boundnames{Q},\vec{y} \in \boundnames{P}$, where $Q\{\vec{y}/\vec{x}\}$
denotes the capture-avoiding substitution of $\vec{y}$ for $\vec{x}$ in $Q$.
\end{definition}

\begin{definition}
  The {\em structural congruence} \cite{SangiorgiWalker} , $\equiv$,
  between processes is the least congruence containing
  alpha-equivalence, satisfying the abelian monoid laws
  (associativity, commutativity and $\pzero$ as identity) for parallel
  composition $|$ and for summation $+$.
\end{definition}

\subsection{Name equivalence}

We take name equivalence, written $\nameeq$, to be the smallest
equivalence relation generated by the following rules.

\begin{mathpar}
\inferrule*[lab=Quote-drop]
{ }
{ \quotep{@{x}} \nameeq x }

\inferrule*[lab=Struct-equiv]
{ P \scong Q }
{ \quotep{P} \nameeq \quotep{Q} }
\end{mathpar}

The astute reader will have noticed that the mutual recursion of names
and processes imposes a mutual recursion on alpha-equivalence and
structural equivalence via name-equivalence. Fortunately, all of this
works out pleasantly and we may calculate in the natural way, free of
concern. The reader interested in the details is referred to the
appendix \ref{appendix:rho_details}.

\subsection{Substitution}

We use $\Proc$ for the set of processes, $\QProc$ for the set of
names, and $\id{\{}\vec{y} / \vec{x} \id{\}}$ to denote partial maps,
$s : \QProc \rightarrow \QProc$. A map, $s$ lifts, uniquely, to a map
on process terms, $\widehat{s} : \Proc \rightarrow \Proc$ by the
following equations.

\begin{mathpar}
  (0) \psubstp{Q}{P} := 0 \\
  (R \juxtap S) \psubstp{Q}{P}
  :=    
  (R)\psubstp{Q}{P} \juxtap (S) \psubstp{Q}{P} \\
  (x?(y).R) \psubstp{Q}{P}    
  :=    
  (x)\substp{Q}{P} (z)\concat( (R \psubstn{z}{y}) \psubstp{Q}{P} ) \\
  (\lift{x}{R}) \psubstp{Q}{P}  
  :=
  \lift{(x)\substp{Q}{P}}{ R \psubstp{Q}{P} } \\
%   (\dropn{x})  \psubstp{Q}{P}       
%   := 
%   \left\{ 
%     \begin{array}{ccc} 
%       \dropn{\quotep{Q}} & & x \nameeq \quotep{P} \\
%       \dropn{x} & & otherwise \\
%     \end{array}
%   \right. 
  (\dropn{x})  \psubstp{Q}{P}       
  := 
  \left\{ 
    \begin{array}{ccc} 
      Q & & x \nameeq \quotep{P} \\
      \dropn{x} & & otherwise \\
    \end{array}
  \right.
\end{mathpar}
 

where

\begin{eqnarray}
  (x)\id{\{} \lpquote Q \rpquote / \lpquote P \rpquote \id{\}}            = 
  \left\{ 
    \begin{array}{ccc}
      \lpquote Q \rpquote & & x \nameeq \lpquote P \rpquote \\
      x & & otherwise \\
    \end{array}
  \right. \nonumber
\end{eqnarray}

and $z$ is chosen distinct from $\quotep{P}$, $\quotep{Q}$, the free
names in $Q$, and all the names in $R$. Our $\alpha$-equivalence will
be built in the standard way from this substitution.

\begin{remark}\label{rem:no_self_referential_names}
  One consequence of these definitions is that $\forall P. \quotep{P}
  \not\in \freenames{P}$.
\end{remark}

\subsection{ Dynamic quote: an example }

Anticipating something of what's to come, consider applying the
substitution, $\widehat{\id{\{}u / z \id{\}}}$, to the following pair
of processes, $\lift{w}{y!(z)}$ and $w[ \lpquote y!(z) \rpquote ]$.

\begin{eqnarray}
	\lift{w}{y!(z)}\widehat{\id{\{}u / z \id{\}}}
		& = &
		\lift{w}{y!(u)} \nonumber\\
	w[ \lpquote y!(z) \rpquote ] \widehat{ \id{\{}u / z \id{\}} }
		& = &
		w[ \lpquote y!(z) \rpquote ] \nonumber
\end{eqnarray}

Because the body of the process between quotes is impervious to
substitution, we get radically different answers. In fact, by
examining the first process in an input context,
e.g. $x?(z).\lift{w}{y!(z)}$, we see that the process under the lift
operator may be shaped by prefixed inputs binding a name inside it. In
this sense, the lift operator will be seen as a way to dynamically
construct processes before reifying them as names.

Finally equipped with these standard features we can present the
dynamics of the calculus.

\subsubsection{Operational semantics} 

Finally, we introduce the computational dynamics. What marks these
algebras as distinct from other more traditionally studied algebraic
structures, e.g. vector spaces or polynomial rings, is the manner in
which dynamics is captured. In traditional structures, dynamics is typically
expressed through morphisms between such structures, as in linear maps
between vector spaces or morphisms between rings. In algebras
associated with the semantics of computation, the dynamics is
expressed as part of the algebraic structure itself, through a
reduction reduction relation typically denoted by $\red$. Below, we
give a recursive presentation of this relation for the calculus used
in the encoding.

$\red \subseteq \pi \times \pi$
$\red : \pi \to \mathcal{P}(\pi)$

\begin{mathpar}
  \inferrule* [lab=Comm] { \textsf{match}( x_{src}, x_{trgt} ) } { x_{trgt}?(y)P \; | \; x_{src}!\langle {Q} \rangle \red P\{\quotep{Q}/y}\} }
  \and \\
  \inferrule* [lab=Par] {{P} \red {P}'} {{{P} | {Q}} \red {{P}' | {Q}}}
  \and
  \inferrule* [lab=Equiv]{{{P} \scong {P}'} \andalso {{P}' \red {Q}'} \andalso {{Q}' \scong {Q}}}{{P} \red {Q}}
\end{mathpar}

\begin{eqnarray*}
  match_{\equiv} (\quotep{P},\quotep{Q}) & := & P \equiv Q \\
  match_{\dagger}(\quotep{P},\quotep{Q}) & := & \forall R. P|Q \red^{*} R => R \red^{*} 0 \\
  match_{K}(\quotep{P},\quotep{Q}) & := & K \mbox{ for some context } K
\end{eqnarray*}

$u?(x)P | u!\langle Q \rangle \red P\{\quotep{Q}/x\}$

%We write $\wred$ for $\red^*$, and $P\red$ if $\exists Q $ such that $ P \red Q$.
We write $P\red$ if $\exists Q $ such that $ P \red Q$ and $P\not\red$, otherwise.

\section{Replication}

As mentioned before, it is known that replication (and hence
recursion) can be implemented in a higher-order process algebra
\cite{SangiorgiWalker}. As our first example of calculation with the
machinery thus far presented we give the construction explicitly in
the {\rhoc}.

\begin{eqnarray}
	D_{x} & := & \prefix{x}{y}{(\binpar{\outputp{x}{y}}{@{y}})} \nonumber\\
	\bangp_{x}{P} & := & \binpar{{x}!\langle{\binpar{D_{x}}{P}}\rangle}{D_{x}} \nonumber
\end{eqnarray}

\begin{eqnarray}
	\bangp_{x}{P} & & \nonumber\\
	=
	& {x}!\langle{(\prefix{x}{y}{(\outputp{x}{y} | @{y})) | P}}\rangle 
	      | \prefix{x}{y}{(\outputp{x}{y} | @{y})} & \nonumber\\
	\red
	& (\outputp{x}{y} | @{y})\substn{\quotep{(\prefix{x}{y}{(@{y} | \outputp{x}{y})) | P}}}{y} & \nonumber\\
	=
	& \outputp{x}{\quotep{(\prefix{x}{y}{(\outputp{x}{y} | @{y})) | P}}}
	  | {(\prefix{x}{y}{(\outputp{x}{y} | @{y})) | P}} & \nonumber\\
	\red
	& \ldots & \nonumber\\
	\red^*
	& P | P | \ldots & \nonumber
\end{eqnarray}

Of course, this encoding, as an implementation, runs away, unfolding
$\bangp{P}$ eagerly. A lazier and more implementable replication
operator, restricted to input-guarded processes, may be obtained as follows.

\begin{eqnarray}
\bangp{\prefix{u}{v}{P}} 
	:= 
	\binpar{\lift{x}{\prefix{u}{v}{(\binpar{D(x)}{P})}}}{D(x)} \nonumber
\end{eqnarray}

\begin{remark}
  Note that the lazier definition still does not deal with summation
  or mixed summation (i.e. sums over input and output). The reader is
  invited to construct definitions of replication that deal with these
  features. 

  Further, the definitions are parameterized in a name, $x$. Can you,
  gentle reader, make a definition that eliminates this parameter and
  guarantees no accidental interaction between the replication
  machinery and the process being replicated -- i.e. no accidental
  sharing of names used by the process to get its work done and the
  name(s) used by the replication to effect copying. This latter
  revision of the definition of replication is crucial to obtaining
  the expected identity $!!P \sim !P$.
\end{remark}

\begin{remark}\label{rem:paradoxical_combinator}
  The reader familiar with the lambda calculus will have noticed the
  similarity between $D$ and the paradoxical combinator.

  [Ed. note: the existence of this seems to suggest we have to be more
  restrictive on the set of processes and names we admit if we are to
  support no-cloning.]
\end{remark}

\subsubsection{Bisimulation}

The computational dynamics gives rise to another kind of equivalence,
the equivalence of computational behavior. As previously mentioned
this is typically captured \emph{via} some form of bisimulation.

% The notion we use in this paper is weak barbed bisimulation
% \cite{milner91polyadicpi}.

The notion we use in this paper is derived from weak barbed
bisimulation \cite{milner91polyadicpi}. 

\begin{definition}
An \emph{observation relation}, $\downarrow_{\mathcal N}$, over a set
of names, $\mathcal N$, is the smallest relation satisfying the rules
below.

\infrule[Out-barb]{y \in {\mathcal N}, \; x \nameeq y}
		  {\outputp{x}{v} \downarrow_{\mathcal N} x}
\infrule[Par-barb]{\mbox{$P\downarrow_{\mathcal N} x$ or $Q\downarrow_{\mathcal N} x$}}
		  {\binpar{P}{Q} \downarrow_{\mathcal N} x}

We write $P \Downarrow_{\mathcal N} x$ if there is $Q$ such that 
$P \wred Q$ and $Q \downarrow_{\mathcal N} x$.
\end{definition}

\begin{definition}
%\label{def.bbisim}
An  ${\mathcal N}$-\emph{barbed bisimulation} over a set of names, ${\mathcal N}$, is a symmetric binary relation 
${\mathcal S}_{\mathcal N}$ between agents such that $P\rel{S}_{\mathcal N}Q$ implies:
\begin{enumerate}
\item If $P \red P'$ then $Q \wred Q'$ and $P'\rel{S}_{\mathcal N} Q'$.
\item If $P\downarrow_{\mathcal N} x$, then $Q\Downarrow_{\mathcal N} x$.
\end{enumerate}
$P$ is ${\mathcal N}$-barbed bisimilar to $Q$, written
$P \wbbisim_{\mathcal N} Q$, if $P \rel{S}_{\mathcal N} Q$ for some ${\mathcal N}$-barbed bisimulation ${\mathcal S}_{\mathcal N}$.
\end{definition}

$\mathcal{R} \subseteq \pi \times \pi$

$P \mathcal{R} Q => \forall P'. P \red P' \Rightarrow \exists Q'. Q \red Q', P' \mathcal{R} Q'$

$P \vdash x \Rightarrow Q \vdash x$

\begin{mathpar}
  \inferrule*[lab=Out-barb]{x \nameeq y}{{y}!\langle{Q}\rangle \vdash x}
  \and
  \inferrule*[lab=Par-barb]{\mbox{$P\vdash x$ or $Q\vdash x$}}{\binpar{P}{Q} \vdash x}
\end{mathpar}

\subsubsection{Contexts}

One of the principle advantages of computational calculi like the
$\pi$-calculus is a well-defined notion of context,
contextual-equivalence and a correlation between
contextual-equivalence and notions of bisimulation. The notion of
context allows the decomposition of a process into (sub-)process and
its syntactic environment, its context. Thus, a context may be
thought of as a process with a ``hole'' (written $\Box$) in it. The
application of a context $M$ to a process $P$, written $M[P]$, is
tantamount to filling the hole in $M$ with $P$. In this paper we do
not need the full weight of this theory, but do make use of the notion
of context in the proof the main theorem. 

\begin{mathpar}
  \inferrule* [lab=summation] {} {{M_{M},M_{N}} \bc \Box \;|\; x.M_{A} \;|\; M_{M}+M_{N}}
  \and
  \inferrule* [lab=agent] {} {{M_{A}} \bc (\vec{x})M_{P} \;| \; \clift{P_0,\ldots,M_{P},\ldots,P_N}}
  \and \\
  \inferrule* [lab=process] {} {{M_{P}} \bc M_{N} \;| \;P|M_{P} }
\end{mathpar} 

\begin{mathpar}
  \inferrule* [lab=sychronization] {} {M_{N} \bc \Box \;|\; x?M_{F} \;|\; x!M_{C}}
  \and
  \inferrule* [lab=abstraction] {} {{M_{F}} \bc (x)M_{P} }
  \and
  \inferrule* [lab=concretion] {} {{M_{C}} \bc \langle M_{P} \rangle }
  \and \\
  \inferrule* [lab=process] {} {{M_{P}} \bc M_{N} \;| \;P|M_{P} }
\end{mathpar}

\begin{definition}[contextual application] Given a context $M$, and
  process $P$, we define the \emph{contextual application}, $M[P] :=
  M\{P/\Box\}$. That is, the contextual application of M to P is the
  substitution of $P$ for $\Box$ in $M$.
\end{definition}

$\meaningof{-} : L \to \mathcal{P}(\pi)$

\begin{mathpar}
  \inferrule* [lab=collection] {} {\meaningof{true} = \pi, \and \meaningof{~E} = \pi \setminus \meaningof{E}, \and \meaningof{E_{1} \& E_{2}} = \meaningof{E_{1}} \cap \meaningof{E_{2}}}
\end{mathpar}

\begin{mathpar}
  \inferrule* [lab=structure] {} {\meaningof{0} = \{ P \in \pi | P \equiv 0 \}, \and \\ \meaningof{E_1 | E_2} = \{ P \in \pi | P \equiv P_{1} | P_{2}, P_{1} \in \meaningof{E_{1}}, P_{2} \in \meaningof{E_2}\} }
\end{mathpar}

\begin{mathpar}
 \inferrule* [lab=behavior] {} {\meaningof{\langle a?b \rangle E} = \{ P \in \pi | P \equiv Q | u?(y)P', \\ \and \\\\ \and \\ \;\;\; u \in \meaningof{a}, \forall z.P'\{z/y\} \in \meaningof{E\{z/b\}}\}, \and \\ \meaningof{a!E} = \{ P \in \pi | P \equiv Q | x!\langle P' \rangle, x \in \meaningof{a} P' \in \meaningof{E}\} }
\end{mathpar}

\begin{mathpar}
 \inferrule* [lab=nominal] {} {\meaningof{\quotep{E}} = \{ \quotep{P} \in \quotep{\pi} | P \in \meaningof{E} \}, \and \meaningof{\quotep{P}} = \{ \quotep{Q} \in \quotep{\pi} | P \equiv Q \} \and \\ \meaningof{@\quotep{E}} = \{ P \in \pi | P \equiv @x, x \in \meaningof{E} \}}
\end{mathpar}

\begin{eqnarray*}
  \\
  \meaningof{-} : TS \to ST
\end{eqnarray*}

\begin{eqnarray*}
  \\
  L : TS \to ST
\end{eqnarray*}

\begin{eqnarray*}
  \\
  P \models E \iff P \in \meaningof{E}
\end{eqnarray*}

\begin{eqnarray*}
  P \approx_{L} Q \iff \forall E \in L. P \models E \iff Q \models E
\end{eqnarray*}

\begin{eqnarray*}
  P \approx_{K} Q
\end{eqnarray*}

\begin{eqnarray*}
  P \approx Q
\end{eqnarray*}

$\approx_{K} = \approx = \approx_{L}$

\subsubsection{Contextual duality}

Note that contexts extend the quotation operation to a family of
operations from processes to names. Given a context, $M$, we can
define a \emph{nominal context}, $\quotep{M}$ by $\quotep{M}[P] :=
\quotep{M[P]}$. To foreshadow what is to come we observe that these
operations enjoy a duality with processes very much like the duality
between vectors and maps from vectors to scalars.

Further, because the calculus is essentially higher-order, we have a
correspondence between contexts and processes. More specifically,
given a name $x$ and a context $M$ we can construct $M^{*}_{x}$ such
that 

\begin{mathpar}
  M^{*}_{x} | \lift{x}{P} \red M[P]
\end{mathpar}

namely,

\begin{mathpar}
  M^{*}_{x} := x?(u).M[\dropn{u}]
\end{mathpar}

The dependence of $M^{*}_{x}$ on a name makes it an abstraction, 

\begin{mathpar}
  M^{*} := (x)x?(u).M[\dropn{u}]
\end{mathpar}

\subsection{Additional notation}

It will sometimes be convenient to denote the process a name
quotes. We already have the notation $x = \quotep{P}$, but it will be
convenient to introduce an alternate notation, $\procn{x}$, when we
want to emphasize the connection to the use of the name. Note that, by
virtue of name equivalence, $\quotep{\procn{x}} \nameeq x$; so, the
notation is consistent with previous definitions.

Further, because names have structure it is possible to effect
substitutions on the basis of that structure. This means we need to
upgrade our notation for substitutions, which we accomplish by
adapting comprehension notation. Thus,

\begin{mathpar}
  P\{ y / x : x \in S \}
\end{mathpar}

is interpreted to mean the process derived from P by replacing (in a
capture-avoiding manner) each occurrence of $x$ in $S$ by $y$. For example,

\begin{mathpar}
  P\{ \quotep{\procn{x}|\procn{x}} / x : x \in \freenames{P} \}
\end{mathpar}

will replace each (occurrence) of a free name $x$ in $P$ by
$\quotep{\procn{x}|\procn{x}}$.

Also, we will avail ourselves of the notation $x^{L}$ and $x^{R}$ to
denote injections of a name into disjoint copies of the name
space. There are numerous ways to accomplish this. One example can be
found in \cite{MeredithR05}. This notation overloads to vectors of
names: $\vec{x}^{\pi} := (x_{i}^{\pi} \; : \; 0 \leq i < |\vec{x}| )$ where $\pi \in \{L,R\}$.

We also use $P^{\Box} := P|\Box$.

In \cite{MeredithR05} an interpretation of the new operator is
given. It turns out that there are several possible interpretations
all enjoying the requisite algebraic properties of the operator (see
\cite{milner91polyadicpi}). We will therefore make liberal use of
$(\nu\; \vec{x})P$.

% subsection the_syntax_and_semantics_of_the_notation_system (end)   

\input{qm2pi.qmops} 

\input{qm2pi.sterngerlach} 

\input{qm2pi.metric} 

% section concurrent_process_calculi (end)

%\input{qm2pi.proofsketch}

% section proof sketch (end)

%\input{qm2pi.slviaknots} 

% section spatial logic via knots (end)

\input{qm2pi.conclusion}

% section conclusion (end)

%\input{qm2pi.dtcodes} 

% section wiring algorithm (end)

\input{qm2pi.ack} 

% section acknowledgments (end)

\newpage


\bibliographystyle{plain}   
\bibliography{../../biblios/main.bib}

\input{qm2pi.rhodetails}

\end{document}



\end{document}

 

% section notation (end)

\input{qm2pi.process.calculi} 

% section concurrent_process_calculi_and_spatial_logics_ (end)
    
%\documentclass[12pt]{llncs}
%\documentclass{jktr}

\usepackage[pdftex]{hyperref}                   
\usepackage {listings}
\usepackage {mathpartir}
\usepackage{bcprules}
%\usepackage{listings}
                       
\usepackage{graphicx} 
%\usepackage[margins=2.5cm,nohead,nofoot]{geometry}
%\usepackage{geometry}
\usepackage{amsfonts}
\usepackage{amstext}
\usepackage{latexsym}
\usepackage{amssymb}
\usepackage{color}


%\include{myPreamble}
\documentclass[12pt]{llncs}
%\documentclass{jktr}

\usepackage[pdftex]{hyperref}                   
\usepackage {listings}
\usepackage {mathpartir}
\usepackage{bcprules}
%\usepackage{listings}
                       
\usepackage{graphicx} 
%\usepackage[margins=2.5cm,nohead,nofoot]{geometry}
%\usepackage{geometry}
\usepackage{amsfonts}
\usepackage{amstext}
\usepackage{latexsym}
\usepackage{amssymb}
\usepackage{color}


%\include{myPreamble}
\include{qm2pi.local} 

%\ifpdf
%\usepackage[pdftex]{graphicx}
%\else
%\usepackage{graphicx}
%\fi

 % \ifpdf
%  \usepackage{pdfsync}
%  \if


%\title{Brief Article}
%\author{David F. Snyder}
%\author{L.G. Meredith}

%\address{Dept. of Math., Texas State University--San Marcos, San Marcos, TX 78666}
       
\pagestyle{empty}


\begin{document}

\lstset{language=[Objective]Caml,frame=shadowbox}

\input{qm2pi.front}

% section front matter (end)

\input{qm2pi.intro} 
 
% section introduction (end)

% \input{qm2pi.knotations} 

% section notation (end)

\input{qm2pi.process.calculi} 

% section concurrent_process_calculi_and_spatial_logics_ (end)
    
%\input{qm2pi.knots2pi} 

%\input{qm2pi.trefoil} 

%\input{qm2pi.mainthm} 

% subsection basic_interpretation (end)

%\input{qm2pi.rho.presentation} 
\subsection{The syntax and semantics of the notation system}\label{sub:the_syntax_and_semantics_of_the_notation_system} % (fold)

We now summarize a technical presentation of the calculus that
embodies our theory of dynamics. The typical presentation of such a
calculus follows the style of giving generators and relations on
them. The grammar, below, describing term constructors, freely
generates the set of processes, $\Proc$. This set is then quotiented
by a relation known as structural congruence and it is over this set
that the notion of dynamics is expressed. This presentation is
essentially that of \cite{MeredithR05} with the addition of
polyadicity and summation. For readability we have relegated some of
the technical subtleties to an appendix.

\subsubsection{Process grammar}\label{subsub:process_grammar}

\begin{mathpar}
  \inferrule* [lab=synchronization] {} {{M} \bc \pzero \;|\; x?F \;|\; x!C }
  \and
  \inferrule* [lab=abstraction] {} {{F} \bc (x)P}
  \and
  \inferrule* [lab=concretion] {} {{C} \bc \langle Q \rangle}
  \and
  \inferrule* [lab=process] {} {{P,Q} \bc M \;| \;P|Q \;|\; @{x}}
  \and
  \inferrule* [lab=name] {} {{x} \bc \quotep{P}}
\end{mathpar} 

Note that $\vec{x}$ (resp. $\vec{P}$) denotes a vector of names
(resp. processes) of length $|\vec{x}|$ (resp. $|\vec{P}|$). We adopt
the following useful abbreviations.

\begin{mathpar}
   x?(\vec{y}).P := x.(\vec{y})P \and  x\clift{\vec{P}} := x.\clift{\vec{P}}
   \and x!(y) := \lift{x}{\dropn{y}}
   \and \Pi_{i=0}^{n-1}P_i := P_0 | \ldots | P_{n-1}
\end{mathpar}

\subsubsection{Structural congruence}

\paragraph{Free and bound names and alpha-equivalence.} At the
core of structural equivalence is alpha-equivalence which identifies
process that are the same up to a change of variable. Formally, we
recognize the distinction between free and bound names. The free names
of a process, $\freenames{P}$, may be calculated recursively as
follows:

\begin{mathpar}
\freenames{\pzero} := \emptyset
  \and \\
  \freenames{x?(y).P} := \{ x \} \cup (\freenames{P} \setminus \{ y \})
  \and 
  \freenames{x!\langle P \rangle} := \{ x \} \cup \{ P \} 
  \and \\
  \freenames{P|Q} := \freenames{P} \cup \freenames{Q}
  \and \\
  \freenames{@{x}} := \{ x \}
\end{mathpar}

$\pi$
$\quotep{\pi}$

$\freenames{-} : \pi \to \mathcal{P}(\quotep{\pi})$

\begin{eqnarray*}
  \freenames{\pzero} & := & \emptyset \\
  \freenames{x?(y).P} & := & \{ x \} \cup (\freenames{P} \setminus \{ y \}) \\
  \freenames{x!\langle P \rangle} & := & \{ x \} \cup \{ P \} \\
  \freenames{P|Q} & := & \freenames{P} \cup \freenames{Q} \\
  \freenames{\dropn{x}} & := & \{ x \}
\end{eqnarray*}

The bound names of a process, $\boundnames{P}$, are those names occurring in $P$
that are not free. For example, in $x?(y).0$, the name $x$ is free, while $y$ is bound.

\begin{mathpar}
  \inferrule* [lab=monoidal-laws] {} { P|Q \equiv Q|P \and P|0 \equiv P \and P|(Q|R) \equiv (P|Q)|R }
\end{mathpar}

\begin{mathpar}
  \inferrule* [lab=alpha-equivalence] {} { (x)P \equiv (y)P\{y/x\} \and y \not\in \freenames{P} }
\end{mathpar}

\begin{definition}
Then two processes, $P,Q$, are alpha-equivalent if $P = Q\{\vec{y}/\vec{x}\}$ for
some $\vec{x} \in \boundnames{Q},\vec{y} \in \boundnames{P}$, where $Q\{\vec{y}/\vec{x}\}$
denotes the capture-avoiding substitution of $\vec{y}$ for $\vec{x}$ in $Q$.
\end{definition}

\begin{definition}
  The {\em structural congruence} \cite{SangiorgiWalker} , $\equiv$,
  between processes is the least congruence containing
  alpha-equivalence, satisfying the abelian monoid laws
  (associativity, commutativity and $\pzero$ as identity) for parallel
  composition $|$ and for summation $+$.
\end{definition}

\subsection{Name equivalence}

We take name equivalence, written $\nameeq$, to be the smallest
equivalence relation generated by the following rules.

\begin{mathpar}
\inferrule*[lab=Quote-drop]
{ }
{ \quotep{@{x}} \nameeq x }

\inferrule*[lab=Struct-equiv]
{ P \scong Q }
{ \quotep{P} \nameeq \quotep{Q} }
\end{mathpar}

The astute reader will have noticed that the mutual recursion of names
and processes imposes a mutual recursion on alpha-equivalence and
structural equivalence via name-equivalence. Fortunately, all of this
works out pleasantly and we may calculate in the natural way, free of
concern. The reader interested in the details is referred to the
appendix \ref{appendix:rho_details}.

\subsection{Substitution}

We use $\Proc$ for the set of processes, $\QProc$ for the set of
names, and $\id{\{}\vec{y} / \vec{x} \id{\}}$ to denote partial maps,
$s : \QProc \rightarrow \QProc$. A map, $s$ lifts, uniquely, to a map
on process terms, $\widehat{s} : \Proc \rightarrow \Proc$ by the
following equations.

\begin{mathpar}
  (0) \psubstp{Q}{P} := 0 \\
  (R \juxtap S) \psubstp{Q}{P}
  :=    
  (R)\psubstp{Q}{P} \juxtap (S) \psubstp{Q}{P} \\
  (x?(y).R) \psubstp{Q}{P}    
  :=    
  (x)\substp{Q}{P} (z)\concat( (R \psubstn{z}{y}) \psubstp{Q}{P} ) \\
  (\lift{x}{R}) \psubstp{Q}{P}  
  :=
  \lift{(x)\substp{Q}{P}}{ R \psubstp{Q}{P} } \\
%   (\dropn{x})  \psubstp{Q}{P}       
%   := 
%   \left\{ 
%     \begin{array}{ccc} 
%       \dropn{\quotep{Q}} & & x \nameeq \quotep{P} \\
%       \dropn{x} & & otherwise \\
%     \end{array}
%   \right. 
  (\dropn{x})  \psubstp{Q}{P}       
  := 
  \left\{ 
    \begin{array}{ccc} 
      Q & & x \nameeq \quotep{P} \\
      \dropn{x} & & otherwise \\
    \end{array}
  \right.
\end{mathpar}
 

where

\begin{eqnarray}
  (x)\id{\{} \lpquote Q \rpquote / \lpquote P \rpquote \id{\}}            = 
  \left\{ 
    \begin{array}{ccc}
      \lpquote Q \rpquote & & x \nameeq \lpquote P \rpquote \\
      x & & otherwise \\
    \end{array}
  \right. \nonumber
\end{eqnarray}

and $z$ is chosen distinct from $\quotep{P}$, $\quotep{Q}$, the free
names in $Q$, and all the names in $R$. Our $\alpha$-equivalence will
be built in the standard way from this substitution.

\begin{remark}\label{rem:no_self_referential_names}
  One consequence of these definitions is that $\forall P. \quotep{P}
  \not\in \freenames{P}$.
\end{remark}

\subsection{ Dynamic quote: an example }

Anticipating something of what's to come, consider applying the
substitution, $\widehat{\id{\{}u / z \id{\}}}$, to the following pair
of processes, $\lift{w}{y!(z)}$ and $w[ \lpquote y!(z) \rpquote ]$.

\begin{eqnarray}
	\lift{w}{y!(z)}\widehat{\id{\{}u / z \id{\}}}
		& = &
		\lift{w}{y!(u)} \nonumber\\
	w[ \lpquote y!(z) \rpquote ] \widehat{ \id{\{}u / z \id{\}} }
		& = &
		w[ \lpquote y!(z) \rpquote ] \nonumber
\end{eqnarray}

Because the body of the process between quotes is impervious to
substitution, we get radically different answers. In fact, by
examining the first process in an input context,
e.g. $x?(z).\lift{w}{y!(z)}$, we see that the process under the lift
operator may be shaped by prefixed inputs binding a name inside it. In
this sense, the lift operator will be seen as a way to dynamically
construct processes before reifying them as names.

Finally equipped with these standard features we can present the
dynamics of the calculus.

\subsubsection{Operational semantics} 

Finally, we introduce the computational dynamics. What marks these
algebras as distinct from other more traditionally studied algebraic
structures, e.g. vector spaces or polynomial rings, is the manner in
which dynamics is captured. In traditional structures, dynamics is typically
expressed through morphisms between such structures, as in linear maps
between vector spaces or morphisms between rings. In algebras
associated with the semantics of computation, the dynamics is
expressed as part of the algebraic structure itself, through a
reduction reduction relation typically denoted by $\red$. Below, we
give a recursive presentation of this relation for the calculus used
in the encoding.

$\red \subseteq \pi \times \pi$
$\red : \pi \to \mathcal{P}(\pi)$

\begin{mathpar}
  \inferrule* [lab=Comm] { \textsf{match}( x_{src}, x_{trgt} ) } { x_{trgt}?(y)P \; | \; x_{src}!\langle {Q} \rangle \red P\{\quotep{Q}/y}\} }
  \and \\
  \inferrule* [lab=Par] {{P} \red {P}'} {{{P} | {Q}} \red {{P}' | {Q}}}
  \and
  \inferrule* [lab=Equiv]{{{P} \scong {P}'} \andalso {{P}' \red {Q}'} \andalso {{Q}' \scong {Q}}}{{P} \red {Q}}
\end{mathpar}

\begin{eqnarray*}
  match_{\equiv} (\quotep{P},\quotep{Q}) & := & P \equiv Q \\
  match_{\dagger}(\quotep{P},\quotep{Q}) & := & \forall R. P|Q \red^{*} R => R \red^{*} 0 \\
  match_{K}(\quotep{P},\quotep{Q}) & := & K \mbox{ for some context } K
\end{eqnarray*}

$u?(x)P | u!\langle Q \rangle \red P\{\quotep{Q}/x\}$

%We write $\wred$ for $\red^*$, and $P\red$ if $\exists Q $ such that $ P \red Q$.
We write $P\red$ if $\exists Q $ such that $ P \red Q$ and $P\not\red$, otherwise.

\section{Replication}

As mentioned before, it is known that replication (and hence
recursion) can be implemented in a higher-order process algebra
\cite{SangiorgiWalker}. As our first example of calculation with the
machinery thus far presented we give the construction explicitly in
the {\rhoc}.

\begin{eqnarray}
	D_{x} & := & \prefix{x}{y}{(\binpar{\outputp{x}{y}}{@{y}})} \nonumber\\
	\bangp_{x}{P} & := & \binpar{{x}!\langle{\binpar{D_{x}}{P}}\rangle}{D_{x}} \nonumber
\end{eqnarray}

\begin{eqnarray}
	\bangp_{x}{P} & & \nonumber\\
	=
	& {x}!\langle{(\prefix{x}{y}{(\outputp{x}{y} | @{y})) | P}}\rangle 
	      | \prefix{x}{y}{(\outputp{x}{y} | @{y})} & \nonumber\\
	\red
	& (\outputp{x}{y} | @{y})\substn{\quotep{(\prefix{x}{y}{(@{y} | \outputp{x}{y})) | P}}}{y} & \nonumber\\
	=
	& \outputp{x}{\quotep{(\prefix{x}{y}{(\outputp{x}{y} | @{y})) | P}}}
	  | {(\prefix{x}{y}{(\outputp{x}{y} | @{y})) | P}} & \nonumber\\
	\red
	& \ldots & \nonumber\\
	\red^*
	& P | P | \ldots & \nonumber
\end{eqnarray}

Of course, this encoding, as an implementation, runs away, unfolding
$\bangp{P}$ eagerly. A lazier and more implementable replication
operator, restricted to input-guarded processes, may be obtained as follows.

\begin{eqnarray}
\bangp{\prefix{u}{v}{P}} 
	:= 
	\binpar{\lift{x}{\prefix{u}{v}{(\binpar{D(x)}{P})}}}{D(x)} \nonumber
\end{eqnarray}

\begin{remark}
  Note that the lazier definition still does not deal with summation
  or mixed summation (i.e. sums over input and output). The reader is
  invited to construct definitions of replication that deal with these
  features. 

  Further, the definitions are parameterized in a name, $x$. Can you,
  gentle reader, make a definition that eliminates this parameter and
  guarantees no accidental interaction between the replication
  machinery and the process being replicated -- i.e. no accidental
  sharing of names used by the process to get its work done and the
  name(s) used by the replication to effect copying. This latter
  revision of the definition of replication is crucial to obtaining
  the expected identity $!!P \sim !P$.
\end{remark}

\begin{remark}\label{rem:paradoxical_combinator}
  The reader familiar with the lambda calculus will have noticed the
  similarity between $D$ and the paradoxical combinator.

  [Ed. note: the existence of this seems to suggest we have to be more
  restrictive on the set of processes and names we admit if we are to
  support no-cloning.]
\end{remark}

\subsubsection{Bisimulation}

The computational dynamics gives rise to another kind of equivalence,
the equivalence of computational behavior. As previously mentioned
this is typically captured \emph{via} some form of bisimulation.

% The notion we use in this paper is weak barbed bisimulation
% \cite{milner91polyadicpi}.

The notion we use in this paper is derived from weak barbed
bisimulation \cite{milner91polyadicpi}. 

\begin{definition}
An \emph{observation relation}, $\downarrow_{\mathcal N}$, over a set
of names, $\mathcal N$, is the smallest relation satisfying the rules
below.

\infrule[Out-barb]{y \in {\mathcal N}, \; x \nameeq y}
		  {\outputp{x}{v} \downarrow_{\mathcal N} x}
\infrule[Par-barb]{\mbox{$P\downarrow_{\mathcal N} x$ or $Q\downarrow_{\mathcal N} x$}}
		  {\binpar{P}{Q} \downarrow_{\mathcal N} x}

We write $P \Downarrow_{\mathcal N} x$ if there is $Q$ such that 
$P \wred Q$ and $Q \downarrow_{\mathcal N} x$.
\end{definition}

\begin{definition}
%\label{def.bbisim}
An  ${\mathcal N}$-\emph{barbed bisimulation} over a set of names, ${\mathcal N}$, is a symmetric binary relation 
${\mathcal S}_{\mathcal N}$ between agents such that $P\rel{S}_{\mathcal N}Q$ implies:
\begin{enumerate}
\item If $P \red P'$ then $Q \wred Q'$ and $P'\rel{S}_{\mathcal N} Q'$.
\item If $P\downarrow_{\mathcal N} x$, then $Q\Downarrow_{\mathcal N} x$.
\end{enumerate}
$P$ is ${\mathcal N}$-barbed bisimilar to $Q$, written
$P \wbbisim_{\mathcal N} Q$, if $P \rel{S}_{\mathcal N} Q$ for some ${\mathcal N}$-barbed bisimulation ${\mathcal S}_{\mathcal N}$.
\end{definition}

$\mathcal{R} \subseteq \pi \times \pi$

$P \mathcal{R} Q => \forall P'. P \red P' \Rightarrow \exists Q'. Q \red Q', P' \mathcal{R} Q'$

$P \vdash x \Rightarrow Q \vdash x$

\begin{mathpar}
  \inferrule*[lab=Out-barb]{x \nameeq y}{{y}!\langle{Q}\rangle \vdash x}
  \and
  \inferrule*[lab=Par-barb]{\mbox{$P\vdash x$ or $Q\vdash x$}}{\binpar{P}{Q} \vdash x}
\end{mathpar}

\subsubsection{Contexts}

One of the principle advantages of computational calculi like the
$\pi$-calculus is a well-defined notion of context,
contextual-equivalence and a correlation between
contextual-equivalence and notions of bisimulation. The notion of
context allows the decomposition of a process into (sub-)process and
its syntactic environment, its context. Thus, a context may be
thought of as a process with a ``hole'' (written $\Box$) in it. The
application of a context $M$ to a process $P$, written $M[P]$, is
tantamount to filling the hole in $M$ with $P$. In this paper we do
not need the full weight of this theory, but do make use of the notion
of context in the proof the main theorem. 

\begin{mathpar}
  \inferrule* [lab=summation] {} {{M_{M},M_{N}} \bc \Box \;|\; x.M_{A} \;|\; M_{M}+M_{N}}
  \and
  \inferrule* [lab=agent] {} {{M_{A}} \bc (\vec{x})M_{P} \;| \; \clift{P_0,\ldots,M_{P},\ldots,P_N}}
  \and \\
  \inferrule* [lab=process] {} {{M_{P}} \bc M_{N} \;| \;P|M_{P} }
\end{mathpar} 

\begin{mathpar}
  \inferrule* [lab=sychronization] {} {M_{N} \bc \Box \;|\; x?M_{F} \;|\; x!M_{C}}
  \and
  \inferrule* [lab=abstraction] {} {{M_{F}} \bc (x)M_{P} }
  \and
  \inferrule* [lab=concretion] {} {{M_{C}} \bc \langle M_{P} \rangle }
  \and \\
  \inferrule* [lab=process] {} {{M_{P}} \bc M_{N} \;| \;P|M_{P} }
\end{mathpar}

\begin{definition}[contextual application] Given a context $M$, and
  process $P$, we define the \emph{contextual application}, $M[P] :=
  M\{P/\Box\}$. That is, the contextual application of M to P is the
  substitution of $P$ for $\Box$ in $M$.
\end{definition}

$\meaningof{-} : L \to \mathcal{P}(\pi)$

\begin{mathpar}
  \inferrule* [lab=collection] {} {\meaningof{true} = \pi, \and \meaningof{~E} = \pi \setminus \meaningof{E}, \and \meaningof{E_{1} \& E_{2}} = \meaningof{E_{1}} \cap \meaningof{E_{2}}}
\end{mathpar}

\begin{mathpar}
  \inferrule* [lab=structure] {} {\meaningof{0} = \{ P \in \pi | P \equiv 0 \}, \and \\ \meaningof{E_1 | E_2} = \{ P \in \pi | P \equiv P_{1} | P_{2}, P_{1} \in \meaningof{E_{1}}, P_{2} \in \meaningof{E_2}\} }
\end{mathpar}

\begin{mathpar}
 \inferrule* [lab=behavior] {} {\meaningof{\langle a?b \rangle E} = \{ P \in \pi | P \equiv Q | u?(y)P', \\ \and \\\\ \and \\ \;\;\; u \in \meaningof{a}, \forall z.P'\{z/y\} \in \meaningof{E\{z/b\}}\}, \and \\ \meaningof{a!E} = \{ P \in \pi | P \equiv Q | x!\langle P' \rangle, x \in \meaningof{a} P' \in \meaningof{E}\} }
\end{mathpar}

\begin{mathpar}
 \inferrule* [lab=nominal] {} {\meaningof{\quotep{E}} = \{ \quotep{P} \in \quotep{\pi} | P \in \meaningof{E} \}, \and \meaningof{\quotep{P}} = \{ \quotep{Q} \in \quotep{\pi} | P \equiv Q \} \and \\ \meaningof{@\quotep{E}} = \{ P \in \pi | P \equiv @x, x \in \meaningof{E} \}}
\end{mathpar}

\begin{eqnarray*}
  \\
  \meaningof{-} : TS \to ST
\end{eqnarray*}

\begin{eqnarray*}
  \\
  L : TS \to ST
\end{eqnarray*}

\begin{eqnarray*}
  \\
  P \models E \iff P \in \meaningof{E}
\end{eqnarray*}

\begin{eqnarray*}
  P \approx_{L} Q \iff \forall E \in L. P \models E \iff Q \models E
\end{eqnarray*}

\begin{eqnarray*}
  P \approx_{K} Q
\end{eqnarray*}

\begin{eqnarray*}
  P \approx Q
\end{eqnarray*}

$\approx_{K} = \approx = \approx_{L}$

\subsubsection{Contextual duality}

Note that contexts extend the quotation operation to a family of
operations from processes to names. Given a context, $M$, we can
define a \emph{nominal context}, $\quotep{M}$ by $\quotep{M}[P] :=
\quotep{M[P]}$. To foreshadow what is to come we observe that these
operations enjoy a duality with processes very much like the duality
between vectors and maps from vectors to scalars.

Further, because the calculus is essentially higher-order, we have a
correspondence between contexts and processes. More specifically,
given a name $x$ and a context $M$ we can construct $M^{*}_{x}$ such
that 

\begin{mathpar}
  M^{*}_{x} | \lift{x}{P} \red M[P]
\end{mathpar}

namely,

\begin{mathpar}
  M^{*}_{x} := x?(u).M[\dropn{u}]
\end{mathpar}

The dependence of $M^{*}_{x}$ on a name makes it an abstraction, 

\begin{mathpar}
  M^{*} := (x)x?(u).M[\dropn{u}]
\end{mathpar}

\subsection{Additional notation}

It will sometimes be convenient to denote the process a name
quotes. We already have the notation $x = \quotep{P}$, but it will be
convenient to introduce an alternate notation, $\procn{x}$, when we
want to emphasize the connection to the use of the name. Note that, by
virtue of name equivalence, $\quotep{\procn{x}} \nameeq x$; so, the
notation is consistent with previous definitions.

Further, because names have structure it is possible to effect
substitutions on the basis of that structure. This means we need to
upgrade our notation for substitutions, which we accomplish by
adapting comprehension notation. Thus,

\begin{mathpar}
  P\{ y / x : x \in S \}
\end{mathpar}

is interpreted to mean the process derived from P by replacing (in a
capture-avoiding manner) each occurrence of $x$ in $S$ by $y$. For example,

\begin{mathpar}
  P\{ \quotep{\procn{x}|\procn{x}} / x : x \in \freenames{P} \}
\end{mathpar}

will replace each (occurrence) of a free name $x$ in $P$ by
$\quotep{\procn{x}|\procn{x}}$.

Also, we will avail ourselves of the notation $x^{L}$ and $x^{R}$ to
denote injections of a name into disjoint copies of the name
space. There are numerous ways to accomplish this. One example can be
found in \cite{MeredithR05}. This notation overloads to vectors of
names: $\vec{x}^{\pi} := (x_{i}^{\pi} \; : \; 0 \leq i < |\vec{x}| )$ where $\pi \in \{L,R\}$.

We also use $P^{\Box} := P|\Box$.

In \cite{MeredithR05} an interpretation of the new operator is
given. It turns out that there are several possible interpretations
all enjoying the requisite algebraic properties of the operator (see
\cite{milner91polyadicpi}). We will therefore make liberal use of
$(\nu\; \vec{x})P$.

% subsection the_syntax_and_semantics_of_the_notation_system (end)   

\input{qm2pi.qmops} 

\input{qm2pi.sterngerlach} 

\input{qm2pi.metric} 

% section concurrent_process_calculi (end)

%\input{qm2pi.proofsketch}

% section proof sketch (end)

%\input{qm2pi.slviaknots} 

% section spatial logic via knots (end)

\input{qm2pi.conclusion}

% section conclusion (end)

%\input{qm2pi.dtcodes} 

% section wiring algorithm (end)

\input{qm2pi.ack} 

% section acknowledgments (end)

\newpage


\bibliographystyle{plain}   
\bibliography{../../biblios/main.bib}

\input{qm2pi.rhodetails}

\end{document}

 

%\ifpdf
%\usepackage[pdftex]{graphicx}
%\else
%\usepackage{graphicx}
%\fi

 % \ifpdf
%  \usepackage{pdfsync}
%  \if


%\title{Brief Article}
%\author{David F. Snyder}
%\author{L.G. Meredith}

%\address{Dept. of Math., Texas State University--San Marcos, San Marcos, TX 78666}
       
\pagestyle{empty}


\begin{document}

\lstset{language=[Objective]Caml,frame=shadowbox}

\documentclass[12pt]{llncs}
%\documentclass{jktr}

\usepackage[pdftex]{hyperref}                   
\usepackage {listings}
\usepackage {mathpartir}
\usepackage{bcprules}
%\usepackage{listings}
                       
\usepackage{graphicx} 
%\usepackage[margins=2.5cm,nohead,nofoot]{geometry}
%\usepackage{geometry}
\usepackage{amsfonts}
\usepackage{amstext}
\usepackage{latexsym}
\usepackage{amssymb}
\usepackage{color}


%\include{myPreamble}
\include{qm2pi.local} 

%\ifpdf
%\usepackage[pdftex]{graphicx}
%\else
%\usepackage{graphicx}
%\fi

 % \ifpdf
%  \usepackage{pdfsync}
%  \if


%\title{Brief Article}
%\author{David F. Snyder}
%\author{L.G. Meredith}

%\address{Dept. of Math., Texas State University--San Marcos, San Marcos, TX 78666}
       
\pagestyle{empty}


\begin{document}

\lstset{language=[Objective]Caml,frame=shadowbox}

\input{qm2pi.front}

% section front matter (end)

\input{qm2pi.intro} 
 
% section introduction (end)

% \input{qm2pi.knotations} 

% section notation (end)

\input{qm2pi.process.calculi} 

% section concurrent_process_calculi_and_spatial_logics_ (end)
    
%\input{qm2pi.knots2pi} 

%\input{qm2pi.trefoil} 

%\input{qm2pi.mainthm} 

% subsection basic_interpretation (end)

%\input{qm2pi.rho.presentation} 
\subsection{The syntax and semantics of the notation system}\label{sub:the_syntax_and_semantics_of_the_notation_system} % (fold)

We now summarize a technical presentation of the calculus that
embodies our theory of dynamics. The typical presentation of such a
calculus follows the style of giving generators and relations on
them. The grammar, below, describing term constructors, freely
generates the set of processes, $\Proc$. This set is then quotiented
by a relation known as structural congruence and it is over this set
that the notion of dynamics is expressed. This presentation is
essentially that of \cite{MeredithR05} with the addition of
polyadicity and summation. For readability we have relegated some of
the technical subtleties to an appendix.

\subsubsection{Process grammar}\label{subsub:process_grammar}

\begin{mathpar}
  \inferrule* [lab=synchronization] {} {{M} \bc \pzero \;|\; x?F \;|\; x!C }
  \and
  \inferrule* [lab=abstraction] {} {{F} \bc (x)P}
  \and
  \inferrule* [lab=concretion] {} {{C} \bc \langle Q \rangle}
  \and
  \inferrule* [lab=process] {} {{P,Q} \bc M \;| \;P|Q \;|\; @{x}}
  \and
  \inferrule* [lab=name] {} {{x} \bc \quotep{P}}
\end{mathpar} 

Note that $\vec{x}$ (resp. $\vec{P}$) denotes a vector of names
(resp. processes) of length $|\vec{x}|$ (resp. $|\vec{P}|$). We adopt
the following useful abbreviations.

\begin{mathpar}
   x?(\vec{y}).P := x.(\vec{y})P \and  x\clift{\vec{P}} := x.\clift{\vec{P}}
   \and x!(y) := \lift{x}{\dropn{y}}
   \and \Pi_{i=0}^{n-1}P_i := P_0 | \ldots | P_{n-1}
\end{mathpar}

\subsubsection{Structural congruence}

\paragraph{Free and bound names and alpha-equivalence.} At the
core of structural equivalence is alpha-equivalence which identifies
process that are the same up to a change of variable. Formally, we
recognize the distinction between free and bound names. The free names
of a process, $\freenames{P}$, may be calculated recursively as
follows:

\begin{mathpar}
\freenames{\pzero} := \emptyset
  \and \\
  \freenames{x?(y).P} := \{ x \} \cup (\freenames{P} \setminus \{ y \})
  \and 
  \freenames{x!\langle P \rangle} := \{ x \} \cup \{ P \} 
  \and \\
  \freenames{P|Q} := \freenames{P} \cup \freenames{Q}
  \and \\
  \freenames{@{x}} := \{ x \}
\end{mathpar}

$\pi$
$\quotep{\pi}$

$\freenames{-} : \pi \to \mathcal{P}(\quotep{\pi})$

\begin{eqnarray*}
  \freenames{\pzero} & := & \emptyset \\
  \freenames{x?(y).P} & := & \{ x \} \cup (\freenames{P} \setminus \{ y \}) \\
  \freenames{x!\langle P \rangle} & := & \{ x \} \cup \{ P \} \\
  \freenames{P|Q} & := & \freenames{P} \cup \freenames{Q} \\
  \freenames{\dropn{x}} & := & \{ x \}
\end{eqnarray*}

The bound names of a process, $\boundnames{P}$, are those names occurring in $P$
that are not free. For example, in $x?(y).0$, the name $x$ is free, while $y$ is bound.

\begin{mathpar}
  \inferrule* [lab=monoidal-laws] {} { P|Q \equiv Q|P \and P|0 \equiv P \and P|(Q|R) \equiv (P|Q)|R }
\end{mathpar}

\begin{mathpar}
  \inferrule* [lab=alpha-equivalence] {} { (x)P \equiv (y)P\{y/x\} \and y \not\in \freenames{P} }
\end{mathpar}

\begin{definition}
Then two processes, $P,Q$, are alpha-equivalent if $P = Q\{\vec{y}/\vec{x}\}$ for
some $\vec{x} \in \boundnames{Q},\vec{y} \in \boundnames{P}$, where $Q\{\vec{y}/\vec{x}\}$
denotes the capture-avoiding substitution of $\vec{y}$ for $\vec{x}$ in $Q$.
\end{definition}

\begin{definition}
  The {\em structural congruence} \cite{SangiorgiWalker} , $\equiv$,
  between processes is the least congruence containing
  alpha-equivalence, satisfying the abelian monoid laws
  (associativity, commutativity and $\pzero$ as identity) for parallel
  composition $|$ and for summation $+$.
\end{definition}

\subsection{Name equivalence}

We take name equivalence, written $\nameeq$, to be the smallest
equivalence relation generated by the following rules.

\begin{mathpar}
\inferrule*[lab=Quote-drop]
{ }
{ \quotep{@{x}} \nameeq x }

\inferrule*[lab=Struct-equiv]
{ P \scong Q }
{ \quotep{P} \nameeq \quotep{Q} }
\end{mathpar}

The astute reader will have noticed that the mutual recursion of names
and processes imposes a mutual recursion on alpha-equivalence and
structural equivalence via name-equivalence. Fortunately, all of this
works out pleasantly and we may calculate in the natural way, free of
concern. The reader interested in the details is referred to the
appendix \ref{appendix:rho_details}.

\subsection{Substitution}

We use $\Proc$ for the set of processes, $\QProc$ for the set of
names, and $\id{\{}\vec{y} / \vec{x} \id{\}}$ to denote partial maps,
$s : \QProc \rightarrow \QProc$. A map, $s$ lifts, uniquely, to a map
on process terms, $\widehat{s} : \Proc \rightarrow \Proc$ by the
following equations.

\begin{mathpar}
  (0) \psubstp{Q}{P} := 0 \\
  (R \juxtap S) \psubstp{Q}{P}
  :=    
  (R)\psubstp{Q}{P} \juxtap (S) \psubstp{Q}{P} \\
  (x?(y).R) \psubstp{Q}{P}    
  :=    
  (x)\substp{Q}{P} (z)\concat( (R \psubstn{z}{y}) \psubstp{Q}{P} ) \\
  (\lift{x}{R}) \psubstp{Q}{P}  
  :=
  \lift{(x)\substp{Q}{P}}{ R \psubstp{Q}{P} } \\
%   (\dropn{x})  \psubstp{Q}{P}       
%   := 
%   \left\{ 
%     \begin{array}{ccc} 
%       \dropn{\quotep{Q}} & & x \nameeq \quotep{P} \\
%       \dropn{x} & & otherwise \\
%     \end{array}
%   \right. 
  (\dropn{x})  \psubstp{Q}{P}       
  := 
  \left\{ 
    \begin{array}{ccc} 
      Q & & x \nameeq \quotep{P} \\
      \dropn{x} & & otherwise \\
    \end{array}
  \right.
\end{mathpar}
 

where

\begin{eqnarray}
  (x)\id{\{} \lpquote Q \rpquote / \lpquote P \rpquote \id{\}}            = 
  \left\{ 
    \begin{array}{ccc}
      \lpquote Q \rpquote & & x \nameeq \lpquote P \rpquote \\
      x & & otherwise \\
    \end{array}
  \right. \nonumber
\end{eqnarray}

and $z$ is chosen distinct from $\quotep{P}$, $\quotep{Q}$, the free
names in $Q$, and all the names in $R$. Our $\alpha$-equivalence will
be built in the standard way from this substitution.

\begin{remark}\label{rem:no_self_referential_names}
  One consequence of these definitions is that $\forall P. \quotep{P}
  \not\in \freenames{P}$.
\end{remark}

\subsection{ Dynamic quote: an example }

Anticipating something of what's to come, consider applying the
substitution, $\widehat{\id{\{}u / z \id{\}}}$, to the following pair
of processes, $\lift{w}{y!(z)}$ and $w[ \lpquote y!(z) \rpquote ]$.

\begin{eqnarray}
	\lift{w}{y!(z)}\widehat{\id{\{}u / z \id{\}}}
		& = &
		\lift{w}{y!(u)} \nonumber\\
	w[ \lpquote y!(z) \rpquote ] \widehat{ \id{\{}u / z \id{\}} }
		& = &
		w[ \lpquote y!(z) \rpquote ] \nonumber
\end{eqnarray}

Because the body of the process between quotes is impervious to
substitution, we get radically different answers. In fact, by
examining the first process in an input context,
e.g. $x?(z).\lift{w}{y!(z)}$, we see that the process under the lift
operator may be shaped by prefixed inputs binding a name inside it. In
this sense, the lift operator will be seen as a way to dynamically
construct processes before reifying them as names.

Finally equipped with these standard features we can present the
dynamics of the calculus.

\subsubsection{Operational semantics} 

Finally, we introduce the computational dynamics. What marks these
algebras as distinct from other more traditionally studied algebraic
structures, e.g. vector spaces or polynomial rings, is the manner in
which dynamics is captured. In traditional structures, dynamics is typically
expressed through morphisms between such structures, as in linear maps
between vector spaces or morphisms between rings. In algebras
associated with the semantics of computation, the dynamics is
expressed as part of the algebraic structure itself, through a
reduction reduction relation typically denoted by $\red$. Below, we
give a recursive presentation of this relation for the calculus used
in the encoding.

$\red \subseteq \pi \times \pi$
$\red : \pi \to \mathcal{P}(\pi)$

\begin{mathpar}
  \inferrule* [lab=Comm] { \textsf{match}( x_{src}, x_{trgt} ) } { x_{trgt}?(y)P \; | \; x_{src}!\langle {Q} \rangle \red P\{\quotep{Q}/y}\} }
  \and \\
  \inferrule* [lab=Par] {{P} \red {P}'} {{{P} | {Q}} \red {{P}' | {Q}}}
  \and
  \inferrule* [lab=Equiv]{{{P} \scong {P}'} \andalso {{P}' \red {Q}'} \andalso {{Q}' \scong {Q}}}{{P} \red {Q}}
\end{mathpar}

\begin{eqnarray*}
  match_{\equiv} (\quotep{P},\quotep{Q}) & := & P \equiv Q \\
  match_{\dagger}(\quotep{P},\quotep{Q}) & := & \forall R. P|Q \red^{*} R => R \red^{*} 0 \\
  match_{K}(\quotep{P},\quotep{Q}) & := & K \mbox{ for some context } K
\end{eqnarray*}

$u?(x)P | u!\langle Q \rangle \red P\{\quotep{Q}/x\}$

%We write $\wred$ for $\red^*$, and $P\red$ if $\exists Q $ such that $ P \red Q$.
We write $P\red$ if $\exists Q $ such that $ P \red Q$ and $P\not\red$, otherwise.

\section{Replication}

As mentioned before, it is known that replication (and hence
recursion) can be implemented in a higher-order process algebra
\cite{SangiorgiWalker}. As our first example of calculation with the
machinery thus far presented we give the construction explicitly in
the {\rhoc}.

\begin{eqnarray}
	D_{x} & := & \prefix{x}{y}{(\binpar{\outputp{x}{y}}{@{y}})} \nonumber\\
	\bangp_{x}{P} & := & \binpar{{x}!\langle{\binpar{D_{x}}{P}}\rangle}{D_{x}} \nonumber
\end{eqnarray}

\begin{eqnarray}
	\bangp_{x}{P} & & \nonumber\\
	=
	& {x}!\langle{(\prefix{x}{y}{(\outputp{x}{y} | @{y})) | P}}\rangle 
	      | \prefix{x}{y}{(\outputp{x}{y} | @{y})} & \nonumber\\
	\red
	& (\outputp{x}{y} | @{y})\substn{\quotep{(\prefix{x}{y}{(@{y} | \outputp{x}{y})) | P}}}{y} & \nonumber\\
	=
	& \outputp{x}{\quotep{(\prefix{x}{y}{(\outputp{x}{y} | @{y})) | P}}}
	  | {(\prefix{x}{y}{(\outputp{x}{y} | @{y})) | P}} & \nonumber\\
	\red
	& \ldots & \nonumber\\
	\red^*
	& P | P | \ldots & \nonumber
\end{eqnarray}

Of course, this encoding, as an implementation, runs away, unfolding
$\bangp{P}$ eagerly. A lazier and more implementable replication
operator, restricted to input-guarded processes, may be obtained as follows.

\begin{eqnarray}
\bangp{\prefix{u}{v}{P}} 
	:= 
	\binpar{\lift{x}{\prefix{u}{v}{(\binpar{D(x)}{P})}}}{D(x)} \nonumber
\end{eqnarray}

\begin{remark}
  Note that the lazier definition still does not deal with summation
  or mixed summation (i.e. sums over input and output). The reader is
  invited to construct definitions of replication that deal with these
  features. 

  Further, the definitions are parameterized in a name, $x$. Can you,
  gentle reader, make a definition that eliminates this parameter and
  guarantees no accidental interaction between the replication
  machinery and the process being replicated -- i.e. no accidental
  sharing of names used by the process to get its work done and the
  name(s) used by the replication to effect copying. This latter
  revision of the definition of replication is crucial to obtaining
  the expected identity $!!P \sim !P$.
\end{remark}

\begin{remark}\label{rem:paradoxical_combinator}
  The reader familiar with the lambda calculus will have noticed the
  similarity between $D$ and the paradoxical combinator.

  [Ed. note: the existence of this seems to suggest we have to be more
  restrictive on the set of processes and names we admit if we are to
  support no-cloning.]
\end{remark}

\subsubsection{Bisimulation}

The computational dynamics gives rise to another kind of equivalence,
the equivalence of computational behavior. As previously mentioned
this is typically captured \emph{via} some form of bisimulation.

% The notion we use in this paper is weak barbed bisimulation
% \cite{milner91polyadicpi}.

The notion we use in this paper is derived from weak barbed
bisimulation \cite{milner91polyadicpi}. 

\begin{definition}
An \emph{observation relation}, $\downarrow_{\mathcal N}$, over a set
of names, $\mathcal N$, is the smallest relation satisfying the rules
below.

\infrule[Out-barb]{y \in {\mathcal N}, \; x \nameeq y}
		  {\outputp{x}{v} \downarrow_{\mathcal N} x}
\infrule[Par-barb]{\mbox{$P\downarrow_{\mathcal N} x$ or $Q\downarrow_{\mathcal N} x$}}
		  {\binpar{P}{Q} \downarrow_{\mathcal N} x}

We write $P \Downarrow_{\mathcal N} x$ if there is $Q$ such that 
$P \wred Q$ and $Q \downarrow_{\mathcal N} x$.
\end{definition}

\begin{definition}
%\label{def.bbisim}
An  ${\mathcal N}$-\emph{barbed bisimulation} over a set of names, ${\mathcal N}$, is a symmetric binary relation 
${\mathcal S}_{\mathcal N}$ between agents such that $P\rel{S}_{\mathcal N}Q$ implies:
\begin{enumerate}
\item If $P \red P'$ then $Q \wred Q'$ and $P'\rel{S}_{\mathcal N} Q'$.
\item If $P\downarrow_{\mathcal N} x$, then $Q\Downarrow_{\mathcal N} x$.
\end{enumerate}
$P$ is ${\mathcal N}$-barbed bisimilar to $Q$, written
$P \wbbisim_{\mathcal N} Q$, if $P \rel{S}_{\mathcal N} Q$ for some ${\mathcal N}$-barbed bisimulation ${\mathcal S}_{\mathcal N}$.
\end{definition}

$\mathcal{R} \subseteq \pi \times \pi$

$P \mathcal{R} Q => \forall P'. P \red P' \Rightarrow \exists Q'. Q \red Q', P' \mathcal{R} Q'$

$P \vdash x \Rightarrow Q \vdash x$

\begin{mathpar}
  \inferrule*[lab=Out-barb]{x \nameeq y}{{y}!\langle{Q}\rangle \vdash x}
  \and
  \inferrule*[lab=Par-barb]{\mbox{$P\vdash x$ or $Q\vdash x$}}{\binpar{P}{Q} \vdash x}
\end{mathpar}

\subsubsection{Contexts}

One of the principle advantages of computational calculi like the
$\pi$-calculus is a well-defined notion of context,
contextual-equivalence and a correlation between
contextual-equivalence and notions of bisimulation. The notion of
context allows the decomposition of a process into (sub-)process and
its syntactic environment, its context. Thus, a context may be
thought of as a process with a ``hole'' (written $\Box$) in it. The
application of a context $M$ to a process $P$, written $M[P]$, is
tantamount to filling the hole in $M$ with $P$. In this paper we do
not need the full weight of this theory, but do make use of the notion
of context in the proof the main theorem. 

\begin{mathpar}
  \inferrule* [lab=summation] {} {{M_{M},M_{N}} \bc \Box \;|\; x.M_{A} \;|\; M_{M}+M_{N}}
  \and
  \inferrule* [lab=agent] {} {{M_{A}} \bc (\vec{x})M_{P} \;| \; \clift{P_0,\ldots,M_{P},\ldots,P_N}}
  \and \\
  \inferrule* [lab=process] {} {{M_{P}} \bc M_{N} \;| \;P|M_{P} }
\end{mathpar} 

\begin{mathpar}
  \inferrule* [lab=sychronization] {} {M_{N} \bc \Box \;|\; x?M_{F} \;|\; x!M_{C}}
  \and
  \inferrule* [lab=abstraction] {} {{M_{F}} \bc (x)M_{P} }
  \and
  \inferrule* [lab=concretion] {} {{M_{C}} \bc \langle M_{P} \rangle }
  \and \\
  \inferrule* [lab=process] {} {{M_{P}} \bc M_{N} \;| \;P|M_{P} }
\end{mathpar}

\begin{definition}[contextual application] Given a context $M$, and
  process $P$, we define the \emph{contextual application}, $M[P] :=
  M\{P/\Box\}$. That is, the contextual application of M to P is the
  substitution of $P$ for $\Box$ in $M$.
\end{definition}

$\meaningof{-} : L \to \mathcal{P}(\pi)$

\begin{mathpar}
  \inferrule* [lab=collection] {} {\meaningof{true} = \pi, \and \meaningof{~E} = \pi \setminus \meaningof{E}, \and \meaningof{E_{1} \& E_{2}} = \meaningof{E_{1}} \cap \meaningof{E_{2}}}
\end{mathpar}

\begin{mathpar}
  \inferrule* [lab=structure] {} {\meaningof{0} = \{ P \in \pi | P \equiv 0 \}, \and \\ \meaningof{E_1 | E_2} = \{ P \in \pi | P \equiv P_{1} | P_{2}, P_{1} \in \meaningof{E_{1}}, P_{2} \in \meaningof{E_2}\} }
\end{mathpar}

\begin{mathpar}
 \inferrule* [lab=behavior] {} {\meaningof{\langle a?b \rangle E} = \{ P \in \pi | P \equiv Q | u?(y)P', \\ \and \\\\ \and \\ \;\;\; u \in \meaningof{a}, \forall z.P'\{z/y\} \in \meaningof{E\{z/b\}}\}, \and \\ \meaningof{a!E} = \{ P \in \pi | P \equiv Q | x!\langle P' \rangle, x \in \meaningof{a} P' \in \meaningof{E}\} }
\end{mathpar}

\begin{mathpar}
 \inferrule* [lab=nominal] {} {\meaningof{\quotep{E}} = \{ \quotep{P} \in \quotep{\pi} | P \in \meaningof{E} \}, \and \meaningof{\quotep{P}} = \{ \quotep{Q} \in \quotep{\pi} | P \equiv Q \} \and \\ \meaningof{@\quotep{E}} = \{ P \in \pi | P \equiv @x, x \in \meaningof{E} \}}
\end{mathpar}

\begin{eqnarray*}
  \\
  \meaningof{-} : TS \to ST
\end{eqnarray*}

\begin{eqnarray*}
  \\
  L : TS \to ST
\end{eqnarray*}

\begin{eqnarray*}
  \\
  P \models E \iff P \in \meaningof{E}
\end{eqnarray*}

\begin{eqnarray*}
  P \approx_{L} Q \iff \forall E \in L. P \models E \iff Q \models E
\end{eqnarray*}

\begin{eqnarray*}
  P \approx_{K} Q
\end{eqnarray*}

\begin{eqnarray*}
  P \approx Q
\end{eqnarray*}

$\approx_{K} = \approx = \approx_{L}$

\subsubsection{Contextual duality}

Note that contexts extend the quotation operation to a family of
operations from processes to names. Given a context, $M$, we can
define a \emph{nominal context}, $\quotep{M}$ by $\quotep{M}[P] :=
\quotep{M[P]}$. To foreshadow what is to come we observe that these
operations enjoy a duality with processes very much like the duality
between vectors and maps from vectors to scalars.

Further, because the calculus is essentially higher-order, we have a
correspondence between contexts and processes. More specifically,
given a name $x$ and a context $M$ we can construct $M^{*}_{x}$ such
that 

\begin{mathpar}
  M^{*}_{x} | \lift{x}{P} \red M[P]
\end{mathpar}

namely,

\begin{mathpar}
  M^{*}_{x} := x?(u).M[\dropn{u}]
\end{mathpar}

The dependence of $M^{*}_{x}$ on a name makes it an abstraction, 

\begin{mathpar}
  M^{*} := (x)x?(u).M[\dropn{u}]
\end{mathpar}

\subsection{Additional notation}

It will sometimes be convenient to denote the process a name
quotes. We already have the notation $x = \quotep{P}$, but it will be
convenient to introduce an alternate notation, $\procn{x}$, when we
want to emphasize the connection to the use of the name. Note that, by
virtue of name equivalence, $\quotep{\procn{x}} \nameeq x$; so, the
notation is consistent with previous definitions.

Further, because names have structure it is possible to effect
substitutions on the basis of that structure. This means we need to
upgrade our notation for substitutions, which we accomplish by
adapting comprehension notation. Thus,

\begin{mathpar}
  P\{ y / x : x \in S \}
\end{mathpar}

is interpreted to mean the process derived from P by replacing (in a
capture-avoiding manner) each occurrence of $x$ in $S$ by $y$. For example,

\begin{mathpar}
  P\{ \quotep{\procn{x}|\procn{x}} / x : x \in \freenames{P} \}
\end{mathpar}

will replace each (occurrence) of a free name $x$ in $P$ by
$\quotep{\procn{x}|\procn{x}}$.

Also, we will avail ourselves of the notation $x^{L}$ and $x^{R}$ to
denote injections of a name into disjoint copies of the name
space. There are numerous ways to accomplish this. One example can be
found in \cite{MeredithR05}. This notation overloads to vectors of
names: $\vec{x}^{\pi} := (x_{i}^{\pi} \; : \; 0 \leq i < |\vec{x}| )$ where $\pi \in \{L,R\}$.

We also use $P^{\Box} := P|\Box$.

In \cite{MeredithR05} an interpretation of the new operator is
given. It turns out that there are several possible interpretations
all enjoying the requisite algebraic properties of the operator (see
\cite{milner91polyadicpi}). We will therefore make liberal use of
$(\nu\; \vec{x})P$.

% subsection the_syntax_and_semantics_of_the_notation_system (end)   

\input{qm2pi.qmops} 

\input{qm2pi.sterngerlach} 

\input{qm2pi.metric} 

% section concurrent_process_calculi (end)

%\input{qm2pi.proofsketch}

% section proof sketch (end)

%\input{qm2pi.slviaknots} 

% section spatial logic via knots (end)

\input{qm2pi.conclusion}

% section conclusion (end)

%\input{qm2pi.dtcodes} 

% section wiring algorithm (end)

\input{qm2pi.ack} 

% section acknowledgments (end)

\newpage


\bibliographystyle{plain}   
\bibliography{../../biblios/main.bib}

\input{qm2pi.rhodetails}

\end{document}



% section front matter (end)

\section{Introduction}\label{sec:introduction} % (fold)
In this draft of the material i am going to have to dispense with the
usual writing conventions adopted in papers on these topics. i'm going
to have adopt whatever tone i need at the time i'm writing up the
calculations. Sometimes this may be very conversational; others it may
be the barest mathematical grunts; others still it may be that i have
lifted text from one of my other papers because the exposition of some
point was better said there. i hope that my readers are not unduly put
out by this decision. i'm not doing this to flout convention or be
rebellious. i find these calculations very technically challenging. To
keep everything going technically, something has to give; i have to
let go of some cognitive burden. So, the academic writing style --
with all of its trade-offs in terms of facilitating technical
communication -- is what i'm letting go of. Perhaps subsequent drafts
can be tightened and polished, but for now, i'm going to speak as if
we were sitting together in a coffee shop with a laptop, wifi and a
pad of paper and a pencil.

So, here's what i have to say. We -- you and i, comfortably ensconced
in our coffee shop and well-equipped with our tools -- can realize and
carry out the calculations of quantum mechanics over a very different
formal theory of dynamics, a formal theory of dynamics that
corresponds to a theory of concurrent computation with
\emph{reflection}. It has the advantage that the underlying theory is
already `quantized', but supports analogues all of the continuuous
operations. Strikingly, this underlying theory has recently been
connected with a notion of metric that we can show, by calculating
together, coincides with the metric induced by the inner product.

There are a lot of reasons why you might be interested in seeing
calculations of this form. Here's why i'm interested. For the past
several centuries there has been no competitor to the ``Newtonian''
account of dynamics. As a result the predominant share of accounts of
dynamical systems and situations have had to be formulated in terms of
the Newtonian machinery. i view this as an intellectually dangerous
position to occupy. Everything, despite it's intrinsic shape, turns
into a nail to be hit with this hammer. Recently, however, the theory
of computation has matured to the point where we have candidates for
theories of dynamics that offer very different perspective on
reasoning about dynamical systems and situations. Testing these
candidates against very successful accounts of dynamical situations,
like quantum mechanics, is going to give us some sense of how mature
they are and some measure of the quality of these accounts of
dynamics.

\subsection{Summary of contributions and outline of paper}

So, we're going to develop an interpretation of the operations of
quantum mechanics normally interpreted by Hilbert spaces and
operators. We're going to do this over a theory of computation. Note
that this is very different than the usual quantum computation program
which develops notions of computation over quantum mechanics. Rather,
we are developing a story that aligns with Wheeler's slogan: It from
Bit. To do this we will first provide an account of the theory of
computation at play here. Then we will dive into a calculation-driven
interpretation of the operations of quantum mechanics.

The reason we take this approach is that -- until very recently --
there hasn't been an axiomatic account of quantum mechanics. As a
result there has been no sharp delineation of the mathematical theory
supporting interpretation of the physical theory and the physical
theory, itself. So, ambient features of the maths are free to be
exploited (or supressed) without a real accounting of their physical
relevance. There is no sharp statement ``here's the physical theory''
qua \emph{theory} and ``here's the mathematical interpretation''
enabling a judgment of how faithful the interpretation is -- apart
from experimental observation. When there is an axiomatic account we
can judge how well a given mathematical formalism supports an
interpretation of the axioms, independent of
experimentation. Likewise, we can judge how well we have captured our
physical evidence and experience with our axiomatics, independent of
any specific mathematical implementation, with accidental detail that
may or may not have physical significance. 

In lieu of a fully fleshed out and vetted axiomatic account of quantum
mechanics, interpreting the operational notions in service of modeling
physical systems will have to suffice. In other words, we are not in
the business of providing a model of Hilbert spaces and operators. We
are in the business of providing a model of quantum mechanics because
we are motivated by testing our notions of dynamics against physical
theory; and, the predictive calculations of the physical theory must
serve as the best formulation -- shy of a fully fleshed out axiomatic
account -- of the physical theory itself (as they have for scientific
theories since time immemorial). Put another way, despite a
whole-hearted commitment to an It-from-Bit ontology, we are firmly
aligned with the shut-up-and-calculate camp as the best way to obtain
results either from the physical perspective or as a quality assurance
measure of our fledgling theory of dynamics.

In detail, we present a reflective process calculus. Then we develop
intuitive correspondences between the notions available in this
calculus and the usual physical notions supporting quantum mechanical
calculations. Thus, 

\begin{table}[htp]
  \center{
    \fbox{
      \begin{tabular}{c|c}
        quantum mechanics & process calculus \\
        \hline
        scalar & name \\
        state vector & process \\
        dual & contextual duals \\
        matrix & formal sums of process-context-dual pairs \\
        orthogonality & process annihilation \\
        inner product & execution-formula + quoting
      \end{tabular}
    }
  }
  \caption{QM - process calculi correspondences}
\end{table}

Then we tighten up these intuitions to operational definitions. We
employ the Dirac notation as the best proxy we can find for an
abstract syntax of the quantum mechanical notions. The definitions we
develop put us in contact with equational constraints coming from the
theory that we demonstrate the definitions and calculations satisfy.

This puts us in a position to shut up and calculate for the
Stern-Gerlach experimental set up, showing how these predictive
calculations become calculations on processes in our theory of a
reflective process calculus.

Penultimately, we demonstrate that the notion of metric coming from
the inner product coincides with the notion of metric available from
the theory of bisimulation. This demonstration gives us the right to
think of space as arising from behavior. Finally, we consider where we
might go from the new vantage point we have obtained.

% section introduction (end) 
 
% section introduction (end)

% \documentclass[12pt]{llncs}
%\documentclass{jktr}

\usepackage[pdftex]{hyperref}                   
\usepackage {listings}
\usepackage {mathpartir}
\usepackage{bcprules}
%\usepackage{listings}
                       
\usepackage{graphicx} 
%\usepackage[margins=2.5cm,nohead,nofoot]{geometry}
%\usepackage{geometry}
\usepackage{amsfonts}
\usepackage{amstext}
\usepackage{latexsym}
\usepackage{amssymb}
\usepackage{color}


%\include{myPreamble}
\include{qm2pi.local} 

%\ifpdf
%\usepackage[pdftex]{graphicx}
%\else
%\usepackage{graphicx}
%\fi

 % \ifpdf
%  \usepackage{pdfsync}
%  \if


%\title{Brief Article}
%\author{David F. Snyder}
%\author{L.G. Meredith}

%\address{Dept. of Math., Texas State University--San Marcos, San Marcos, TX 78666}
       
\pagestyle{empty}


\begin{document}

\lstset{language=[Objective]Caml,frame=shadowbox}

\input{qm2pi.front}

% section front matter (end)

\input{qm2pi.intro} 
 
% section introduction (end)

% \input{qm2pi.knotations} 

% section notation (end)

\input{qm2pi.process.calculi} 

% section concurrent_process_calculi_and_spatial_logics_ (end)
    
%\input{qm2pi.knots2pi} 

%\input{qm2pi.trefoil} 

%\input{qm2pi.mainthm} 

% subsection basic_interpretation (end)

%\input{qm2pi.rho.presentation} 
\subsection{The syntax and semantics of the notation system}\label{sub:the_syntax_and_semantics_of_the_notation_system} % (fold)

We now summarize a technical presentation of the calculus that
embodies our theory of dynamics. The typical presentation of such a
calculus follows the style of giving generators and relations on
them. The grammar, below, describing term constructors, freely
generates the set of processes, $\Proc$. This set is then quotiented
by a relation known as structural congruence and it is over this set
that the notion of dynamics is expressed. This presentation is
essentially that of \cite{MeredithR05} with the addition of
polyadicity and summation. For readability we have relegated some of
the technical subtleties to an appendix.

\subsubsection{Process grammar}\label{subsub:process_grammar}

\begin{mathpar}
  \inferrule* [lab=synchronization] {} {{M} \bc \pzero \;|\; x?F \;|\; x!C }
  \and
  \inferrule* [lab=abstraction] {} {{F} \bc (x)P}
  \and
  \inferrule* [lab=concretion] {} {{C} \bc \langle Q \rangle}
  \and
  \inferrule* [lab=process] {} {{P,Q} \bc M \;| \;P|Q \;|\; @{x}}
  \and
  \inferrule* [lab=name] {} {{x} \bc \quotep{P}}
\end{mathpar} 

Note that $\vec{x}$ (resp. $\vec{P}$) denotes a vector of names
(resp. processes) of length $|\vec{x}|$ (resp. $|\vec{P}|$). We adopt
the following useful abbreviations.

\begin{mathpar}
   x?(\vec{y}).P := x.(\vec{y})P \and  x\clift{\vec{P}} := x.\clift{\vec{P}}
   \and x!(y) := \lift{x}{\dropn{y}}
   \and \Pi_{i=0}^{n-1}P_i := P_0 | \ldots | P_{n-1}
\end{mathpar}

\subsubsection{Structural congruence}

\paragraph{Free and bound names and alpha-equivalence.} At the
core of structural equivalence is alpha-equivalence which identifies
process that are the same up to a change of variable. Formally, we
recognize the distinction between free and bound names. The free names
of a process, $\freenames{P}$, may be calculated recursively as
follows:

\begin{mathpar}
\freenames{\pzero} := \emptyset
  \and \\
  \freenames{x?(y).P} := \{ x \} \cup (\freenames{P} \setminus \{ y \})
  \and 
  \freenames{x!\langle P \rangle} := \{ x \} \cup \{ P \} 
  \and \\
  \freenames{P|Q} := \freenames{P} \cup \freenames{Q}
  \and \\
  \freenames{@{x}} := \{ x \}
\end{mathpar}

$\pi$
$\quotep{\pi}$

$\freenames{-} : \pi \to \mathcal{P}(\quotep{\pi})$

\begin{eqnarray*}
  \freenames{\pzero} & := & \emptyset \\
  \freenames{x?(y).P} & := & \{ x \} \cup (\freenames{P} \setminus \{ y \}) \\
  \freenames{x!\langle P \rangle} & := & \{ x \} \cup \{ P \} \\
  \freenames{P|Q} & := & \freenames{P} \cup \freenames{Q} \\
  \freenames{\dropn{x}} & := & \{ x \}
\end{eqnarray*}

The bound names of a process, $\boundnames{P}$, are those names occurring in $P$
that are not free. For example, in $x?(y).0$, the name $x$ is free, while $y$ is bound.

\begin{mathpar}
  \inferrule* [lab=monoidal-laws] {} { P|Q \equiv Q|P \and P|0 \equiv P \and P|(Q|R) \equiv (P|Q)|R }
\end{mathpar}

\begin{mathpar}
  \inferrule* [lab=alpha-equivalence] {} { (x)P \equiv (y)P\{y/x\} \and y \not\in \freenames{P} }
\end{mathpar}

\begin{definition}
Then two processes, $P,Q$, are alpha-equivalent if $P = Q\{\vec{y}/\vec{x}\}$ for
some $\vec{x} \in \boundnames{Q},\vec{y} \in \boundnames{P}$, where $Q\{\vec{y}/\vec{x}\}$
denotes the capture-avoiding substitution of $\vec{y}$ for $\vec{x}$ in $Q$.
\end{definition}

\begin{definition}
  The {\em structural congruence} \cite{SangiorgiWalker} , $\equiv$,
  between processes is the least congruence containing
  alpha-equivalence, satisfying the abelian monoid laws
  (associativity, commutativity and $\pzero$ as identity) for parallel
  composition $|$ and for summation $+$.
\end{definition}

\subsection{Name equivalence}

We take name equivalence, written $\nameeq$, to be the smallest
equivalence relation generated by the following rules.

\begin{mathpar}
\inferrule*[lab=Quote-drop]
{ }
{ \quotep{@{x}} \nameeq x }

\inferrule*[lab=Struct-equiv]
{ P \scong Q }
{ \quotep{P} \nameeq \quotep{Q} }
\end{mathpar}

The astute reader will have noticed that the mutual recursion of names
and processes imposes a mutual recursion on alpha-equivalence and
structural equivalence via name-equivalence. Fortunately, all of this
works out pleasantly and we may calculate in the natural way, free of
concern. The reader interested in the details is referred to the
appendix \ref{appendix:rho_details}.

\subsection{Substitution}

We use $\Proc$ for the set of processes, $\QProc$ for the set of
names, and $\id{\{}\vec{y} / \vec{x} \id{\}}$ to denote partial maps,
$s : \QProc \rightarrow \QProc$. A map, $s$ lifts, uniquely, to a map
on process terms, $\widehat{s} : \Proc \rightarrow \Proc$ by the
following equations.

\begin{mathpar}
  (0) \psubstp{Q}{P} := 0 \\
  (R \juxtap S) \psubstp{Q}{P}
  :=    
  (R)\psubstp{Q}{P} \juxtap (S) \psubstp{Q}{P} \\
  (x?(y).R) \psubstp{Q}{P}    
  :=    
  (x)\substp{Q}{P} (z)\concat( (R \psubstn{z}{y}) \psubstp{Q}{P} ) \\
  (\lift{x}{R}) \psubstp{Q}{P}  
  :=
  \lift{(x)\substp{Q}{P}}{ R \psubstp{Q}{P} } \\
%   (\dropn{x})  \psubstp{Q}{P}       
%   := 
%   \left\{ 
%     \begin{array}{ccc} 
%       \dropn{\quotep{Q}} & & x \nameeq \quotep{P} \\
%       \dropn{x} & & otherwise \\
%     \end{array}
%   \right. 
  (\dropn{x})  \psubstp{Q}{P}       
  := 
  \left\{ 
    \begin{array}{ccc} 
      Q & & x \nameeq \quotep{P} \\
      \dropn{x} & & otherwise \\
    \end{array}
  \right.
\end{mathpar}
 

where

\begin{eqnarray}
  (x)\id{\{} \lpquote Q \rpquote / \lpquote P \rpquote \id{\}}            = 
  \left\{ 
    \begin{array}{ccc}
      \lpquote Q \rpquote & & x \nameeq \lpquote P \rpquote \\
      x & & otherwise \\
    \end{array}
  \right. \nonumber
\end{eqnarray}

and $z$ is chosen distinct from $\quotep{P}$, $\quotep{Q}$, the free
names in $Q$, and all the names in $R$. Our $\alpha$-equivalence will
be built in the standard way from this substitution.

\begin{remark}\label{rem:no_self_referential_names}
  One consequence of these definitions is that $\forall P. \quotep{P}
  \not\in \freenames{P}$.
\end{remark}

\subsection{ Dynamic quote: an example }

Anticipating something of what's to come, consider applying the
substitution, $\widehat{\id{\{}u / z \id{\}}}$, to the following pair
of processes, $\lift{w}{y!(z)}$ and $w[ \lpquote y!(z) \rpquote ]$.

\begin{eqnarray}
	\lift{w}{y!(z)}\widehat{\id{\{}u / z \id{\}}}
		& = &
		\lift{w}{y!(u)} \nonumber\\
	w[ \lpquote y!(z) \rpquote ] \widehat{ \id{\{}u / z \id{\}} }
		& = &
		w[ \lpquote y!(z) \rpquote ] \nonumber
\end{eqnarray}

Because the body of the process between quotes is impervious to
substitution, we get radically different answers. In fact, by
examining the first process in an input context,
e.g. $x?(z).\lift{w}{y!(z)}$, we see that the process under the lift
operator may be shaped by prefixed inputs binding a name inside it. In
this sense, the lift operator will be seen as a way to dynamically
construct processes before reifying them as names.

Finally equipped with these standard features we can present the
dynamics of the calculus.

\subsubsection{Operational semantics} 

Finally, we introduce the computational dynamics. What marks these
algebras as distinct from other more traditionally studied algebraic
structures, e.g. vector spaces or polynomial rings, is the manner in
which dynamics is captured. In traditional structures, dynamics is typically
expressed through morphisms between such structures, as in linear maps
between vector spaces or morphisms between rings. In algebras
associated with the semantics of computation, the dynamics is
expressed as part of the algebraic structure itself, through a
reduction reduction relation typically denoted by $\red$. Below, we
give a recursive presentation of this relation for the calculus used
in the encoding.

$\red \subseteq \pi \times \pi$
$\red : \pi \to \mathcal{P}(\pi)$

\begin{mathpar}
  \inferrule* [lab=Comm] { \textsf{match}( x_{src}, x_{trgt} ) } { x_{trgt}?(y)P \; | \; x_{src}!\langle {Q} \rangle \red P\{\quotep{Q}/y}\} }
  \and \\
  \inferrule* [lab=Par] {{P} \red {P}'} {{{P} | {Q}} \red {{P}' | {Q}}}
  \and
  \inferrule* [lab=Equiv]{{{P} \scong {P}'} \andalso {{P}' \red {Q}'} \andalso {{Q}' \scong {Q}}}{{P} \red {Q}}
\end{mathpar}

\begin{eqnarray*}
  match_{\equiv} (\quotep{P},\quotep{Q}) & := & P \equiv Q \\
  match_{\dagger}(\quotep{P},\quotep{Q}) & := & \forall R. P|Q \red^{*} R => R \red^{*} 0 \\
  match_{K}(\quotep{P},\quotep{Q}) & := & K \mbox{ for some context } K
\end{eqnarray*}

$u?(x)P | u!\langle Q \rangle \red P\{\quotep{Q}/x\}$

%We write $\wred$ for $\red^*$, and $P\red$ if $\exists Q $ such that $ P \red Q$.
We write $P\red$ if $\exists Q $ such that $ P \red Q$ and $P\not\red$, otherwise.

\section{Replication}

As mentioned before, it is known that replication (and hence
recursion) can be implemented in a higher-order process algebra
\cite{SangiorgiWalker}. As our first example of calculation with the
machinery thus far presented we give the construction explicitly in
the {\rhoc}.

\begin{eqnarray}
	D_{x} & := & \prefix{x}{y}{(\binpar{\outputp{x}{y}}{@{y}})} \nonumber\\
	\bangp_{x}{P} & := & \binpar{{x}!\langle{\binpar{D_{x}}{P}}\rangle}{D_{x}} \nonumber
\end{eqnarray}

\begin{eqnarray}
	\bangp_{x}{P} & & \nonumber\\
	=
	& {x}!\langle{(\prefix{x}{y}{(\outputp{x}{y} | @{y})) | P}}\rangle 
	      | \prefix{x}{y}{(\outputp{x}{y} | @{y})} & \nonumber\\
	\red
	& (\outputp{x}{y} | @{y})\substn{\quotep{(\prefix{x}{y}{(@{y} | \outputp{x}{y})) | P}}}{y} & \nonumber\\
	=
	& \outputp{x}{\quotep{(\prefix{x}{y}{(\outputp{x}{y} | @{y})) | P}}}
	  | {(\prefix{x}{y}{(\outputp{x}{y} | @{y})) | P}} & \nonumber\\
	\red
	& \ldots & \nonumber\\
	\red^*
	& P | P | \ldots & \nonumber
\end{eqnarray}

Of course, this encoding, as an implementation, runs away, unfolding
$\bangp{P}$ eagerly. A lazier and more implementable replication
operator, restricted to input-guarded processes, may be obtained as follows.

\begin{eqnarray}
\bangp{\prefix{u}{v}{P}} 
	:= 
	\binpar{\lift{x}{\prefix{u}{v}{(\binpar{D(x)}{P})}}}{D(x)} \nonumber
\end{eqnarray}

\begin{remark}
  Note that the lazier definition still does not deal with summation
  or mixed summation (i.e. sums over input and output). The reader is
  invited to construct definitions of replication that deal with these
  features. 

  Further, the definitions are parameterized in a name, $x$. Can you,
  gentle reader, make a definition that eliminates this parameter and
  guarantees no accidental interaction between the replication
  machinery and the process being replicated -- i.e. no accidental
  sharing of names used by the process to get its work done and the
  name(s) used by the replication to effect copying. This latter
  revision of the definition of replication is crucial to obtaining
  the expected identity $!!P \sim !P$.
\end{remark}

\begin{remark}\label{rem:paradoxical_combinator}
  The reader familiar with the lambda calculus will have noticed the
  similarity between $D$ and the paradoxical combinator.

  [Ed. note: the existence of this seems to suggest we have to be more
  restrictive on the set of processes and names we admit if we are to
  support no-cloning.]
\end{remark}

\subsubsection{Bisimulation}

The computational dynamics gives rise to another kind of equivalence,
the equivalence of computational behavior. As previously mentioned
this is typically captured \emph{via} some form of bisimulation.

% The notion we use in this paper is weak barbed bisimulation
% \cite{milner91polyadicpi}.

The notion we use in this paper is derived from weak barbed
bisimulation \cite{milner91polyadicpi}. 

\begin{definition}
An \emph{observation relation}, $\downarrow_{\mathcal N}$, over a set
of names, $\mathcal N$, is the smallest relation satisfying the rules
below.

\infrule[Out-barb]{y \in {\mathcal N}, \; x \nameeq y}
		  {\outputp{x}{v} \downarrow_{\mathcal N} x}
\infrule[Par-barb]{\mbox{$P\downarrow_{\mathcal N} x$ or $Q\downarrow_{\mathcal N} x$}}
		  {\binpar{P}{Q} \downarrow_{\mathcal N} x}

We write $P \Downarrow_{\mathcal N} x$ if there is $Q$ such that 
$P \wred Q$ and $Q \downarrow_{\mathcal N} x$.
\end{definition}

\begin{definition}
%\label{def.bbisim}
An  ${\mathcal N}$-\emph{barbed bisimulation} over a set of names, ${\mathcal N}$, is a symmetric binary relation 
${\mathcal S}_{\mathcal N}$ between agents such that $P\rel{S}_{\mathcal N}Q$ implies:
\begin{enumerate}
\item If $P \red P'$ then $Q \wred Q'$ and $P'\rel{S}_{\mathcal N} Q'$.
\item If $P\downarrow_{\mathcal N} x$, then $Q\Downarrow_{\mathcal N} x$.
\end{enumerate}
$P$ is ${\mathcal N}$-barbed bisimilar to $Q$, written
$P \wbbisim_{\mathcal N} Q$, if $P \rel{S}_{\mathcal N} Q$ for some ${\mathcal N}$-barbed bisimulation ${\mathcal S}_{\mathcal N}$.
\end{definition}

$\mathcal{R} \subseteq \pi \times \pi$

$P \mathcal{R} Q => \forall P'. P \red P' \Rightarrow \exists Q'. Q \red Q', P' \mathcal{R} Q'$

$P \vdash x \Rightarrow Q \vdash x$

\begin{mathpar}
  \inferrule*[lab=Out-barb]{x \nameeq y}{{y}!\langle{Q}\rangle \vdash x}
  \and
  \inferrule*[lab=Par-barb]{\mbox{$P\vdash x$ or $Q\vdash x$}}{\binpar{P}{Q} \vdash x}
\end{mathpar}

\subsubsection{Contexts}

One of the principle advantages of computational calculi like the
$\pi$-calculus is a well-defined notion of context,
contextual-equivalence and a correlation between
contextual-equivalence and notions of bisimulation. The notion of
context allows the decomposition of a process into (sub-)process and
its syntactic environment, its context. Thus, a context may be
thought of as a process with a ``hole'' (written $\Box$) in it. The
application of a context $M$ to a process $P$, written $M[P]$, is
tantamount to filling the hole in $M$ with $P$. In this paper we do
not need the full weight of this theory, but do make use of the notion
of context in the proof the main theorem. 

\begin{mathpar}
  \inferrule* [lab=summation] {} {{M_{M},M_{N}} \bc \Box \;|\; x.M_{A} \;|\; M_{M}+M_{N}}
  \and
  \inferrule* [lab=agent] {} {{M_{A}} \bc (\vec{x})M_{P} \;| \; \clift{P_0,\ldots,M_{P},\ldots,P_N}}
  \and \\
  \inferrule* [lab=process] {} {{M_{P}} \bc M_{N} \;| \;P|M_{P} }
\end{mathpar} 

\begin{mathpar}
  \inferrule* [lab=sychronization] {} {M_{N} \bc \Box \;|\; x?M_{F} \;|\; x!M_{C}}
  \and
  \inferrule* [lab=abstraction] {} {{M_{F}} \bc (x)M_{P} }
  \and
  \inferrule* [lab=concretion] {} {{M_{C}} \bc \langle M_{P} \rangle }
  \and \\
  \inferrule* [lab=process] {} {{M_{P}} \bc M_{N} \;| \;P|M_{P} }
\end{mathpar}

\begin{definition}[contextual application] Given a context $M$, and
  process $P$, we define the \emph{contextual application}, $M[P] :=
  M\{P/\Box\}$. That is, the contextual application of M to P is the
  substitution of $P$ for $\Box$ in $M$.
\end{definition}

$\meaningof{-} : L \to \mathcal{P}(\pi)$

\begin{mathpar}
  \inferrule* [lab=collection] {} {\meaningof{true} = \pi, \and \meaningof{~E} = \pi \setminus \meaningof{E}, \and \meaningof{E_{1} \& E_{2}} = \meaningof{E_{1}} \cap \meaningof{E_{2}}}
\end{mathpar}

\begin{mathpar}
  \inferrule* [lab=structure] {} {\meaningof{0} = \{ P \in \pi | P \equiv 0 \}, \and \\ \meaningof{E_1 | E_2} = \{ P \in \pi | P \equiv P_{1} | P_{2}, P_{1} \in \meaningof{E_{1}}, P_{2} \in \meaningof{E_2}\} }
\end{mathpar}

\begin{mathpar}
 \inferrule* [lab=behavior] {} {\meaningof{\langle a?b \rangle E} = \{ P \in \pi | P \equiv Q | u?(y)P', \\ \and \\\\ \and \\ \;\;\; u \in \meaningof{a}, \forall z.P'\{z/y\} \in \meaningof{E\{z/b\}}\}, \and \\ \meaningof{a!E} = \{ P \in \pi | P \equiv Q | x!\langle P' \rangle, x \in \meaningof{a} P' \in \meaningof{E}\} }
\end{mathpar}

\begin{mathpar}
 \inferrule* [lab=nominal] {} {\meaningof{\quotep{E}} = \{ \quotep{P} \in \quotep{\pi} | P \in \meaningof{E} \}, \and \meaningof{\quotep{P}} = \{ \quotep{Q} \in \quotep{\pi} | P \equiv Q \} \and \\ \meaningof{@\quotep{E}} = \{ P \in \pi | P \equiv @x, x \in \meaningof{E} \}}
\end{mathpar}

\begin{eqnarray*}
  \\
  \meaningof{-} : TS \to ST
\end{eqnarray*}

\begin{eqnarray*}
  \\
  L : TS \to ST
\end{eqnarray*}

\begin{eqnarray*}
  \\
  P \models E \iff P \in \meaningof{E}
\end{eqnarray*}

\begin{eqnarray*}
  P \approx_{L} Q \iff \forall E \in L. P \models E \iff Q \models E
\end{eqnarray*}

\begin{eqnarray*}
  P \approx_{K} Q
\end{eqnarray*}

\begin{eqnarray*}
  P \approx Q
\end{eqnarray*}

$\approx_{K} = \approx = \approx_{L}$

\subsubsection{Contextual duality}

Note that contexts extend the quotation operation to a family of
operations from processes to names. Given a context, $M$, we can
define a \emph{nominal context}, $\quotep{M}$ by $\quotep{M}[P] :=
\quotep{M[P]}$. To foreshadow what is to come we observe that these
operations enjoy a duality with processes very much like the duality
between vectors and maps from vectors to scalars.

Further, because the calculus is essentially higher-order, we have a
correspondence between contexts and processes. More specifically,
given a name $x$ and a context $M$ we can construct $M^{*}_{x}$ such
that 

\begin{mathpar}
  M^{*}_{x} | \lift{x}{P} \red M[P]
\end{mathpar}

namely,

\begin{mathpar}
  M^{*}_{x} := x?(u).M[\dropn{u}]
\end{mathpar}

The dependence of $M^{*}_{x}$ on a name makes it an abstraction, 

\begin{mathpar}
  M^{*} := (x)x?(u).M[\dropn{u}]
\end{mathpar}

\subsection{Additional notation}

It will sometimes be convenient to denote the process a name
quotes. We already have the notation $x = \quotep{P}$, but it will be
convenient to introduce an alternate notation, $\procn{x}$, when we
want to emphasize the connection to the use of the name. Note that, by
virtue of name equivalence, $\quotep{\procn{x}} \nameeq x$; so, the
notation is consistent with previous definitions.

Further, because names have structure it is possible to effect
substitutions on the basis of that structure. This means we need to
upgrade our notation for substitutions, which we accomplish by
adapting comprehension notation. Thus,

\begin{mathpar}
  P\{ y / x : x \in S \}
\end{mathpar}

is interpreted to mean the process derived from P by replacing (in a
capture-avoiding manner) each occurrence of $x$ in $S$ by $y$. For example,

\begin{mathpar}
  P\{ \quotep{\procn{x}|\procn{x}} / x : x \in \freenames{P} \}
\end{mathpar}

will replace each (occurrence) of a free name $x$ in $P$ by
$\quotep{\procn{x}|\procn{x}}$.

Also, we will avail ourselves of the notation $x^{L}$ and $x^{R}$ to
denote injections of a name into disjoint copies of the name
space. There are numerous ways to accomplish this. One example can be
found in \cite{MeredithR05}. This notation overloads to vectors of
names: $\vec{x}^{\pi} := (x_{i}^{\pi} \; : \; 0 \leq i < |\vec{x}| )$ where $\pi \in \{L,R\}$.

We also use $P^{\Box} := P|\Box$.

In \cite{MeredithR05} an interpretation of the new operator is
given. It turns out that there are several possible interpretations
all enjoying the requisite algebraic properties of the operator (see
\cite{milner91polyadicpi}). We will therefore make liberal use of
$(\nu\; \vec{x})P$.

% subsection the_syntax_and_semantics_of_the_notation_system (end)   

\input{qm2pi.qmops} 

\input{qm2pi.sterngerlach} 

\input{qm2pi.metric} 

% section concurrent_process_calculi (end)

%\input{qm2pi.proofsketch}

% section proof sketch (end)

%\input{qm2pi.slviaknots} 

% section spatial logic via knots (end)

\input{qm2pi.conclusion}

% section conclusion (end)

%\input{qm2pi.dtcodes} 

% section wiring algorithm (end)

\input{qm2pi.ack} 

% section acknowledgments (end)

\newpage


\bibliographystyle{plain}   
\bibliography{../../biblios/main.bib}

\input{qm2pi.rhodetails}

\end{document}

 

% section notation (end)

\input{qm2pi.process.calculi} 

% section concurrent_process_calculi_and_spatial_logics_ (end)
    
%\documentclass[12pt]{llncs}
%\documentclass{jktr}

\usepackage[pdftex]{hyperref}                   
\usepackage {listings}
\usepackage {mathpartir}
\usepackage{bcprules}
%\usepackage{listings}
                       
\usepackage{graphicx} 
%\usepackage[margins=2.5cm,nohead,nofoot]{geometry}
%\usepackage{geometry}
\usepackage{amsfonts}
\usepackage{amstext}
\usepackage{latexsym}
\usepackage{amssymb}
\usepackage{color}


%\include{myPreamble}
\include{qm2pi.local} 

%\ifpdf
%\usepackage[pdftex]{graphicx}
%\else
%\usepackage{graphicx}
%\fi

 % \ifpdf
%  \usepackage{pdfsync}
%  \if


%\title{Brief Article}
%\author{David F. Snyder}
%\author{L.G. Meredith}

%\address{Dept. of Math., Texas State University--San Marcos, San Marcos, TX 78666}
       
\pagestyle{empty}


\begin{document}

\lstset{language=[Objective]Caml,frame=shadowbox}

\input{qm2pi.front}

% section front matter (end)

\input{qm2pi.intro} 
 
% section introduction (end)

% \input{qm2pi.knotations} 

% section notation (end)

\input{qm2pi.process.calculi} 

% section concurrent_process_calculi_and_spatial_logics_ (end)
    
%\input{qm2pi.knots2pi} 

%\input{qm2pi.trefoil} 

%\input{qm2pi.mainthm} 

% subsection basic_interpretation (end)

%\input{qm2pi.rho.presentation} 
\subsection{The syntax and semantics of the notation system}\label{sub:the_syntax_and_semantics_of_the_notation_system} % (fold)

We now summarize a technical presentation of the calculus that
embodies our theory of dynamics. The typical presentation of such a
calculus follows the style of giving generators and relations on
them. The grammar, below, describing term constructors, freely
generates the set of processes, $\Proc$. This set is then quotiented
by a relation known as structural congruence and it is over this set
that the notion of dynamics is expressed. This presentation is
essentially that of \cite{MeredithR05} with the addition of
polyadicity and summation. For readability we have relegated some of
the technical subtleties to an appendix.

\subsubsection{Process grammar}\label{subsub:process_grammar}

\begin{mathpar}
  \inferrule* [lab=synchronization] {} {{M} \bc \pzero \;|\; x?F \;|\; x!C }
  \and
  \inferrule* [lab=abstraction] {} {{F} \bc (x)P}
  \and
  \inferrule* [lab=concretion] {} {{C} \bc \langle Q \rangle}
  \and
  \inferrule* [lab=process] {} {{P,Q} \bc M \;| \;P|Q \;|\; @{x}}
  \and
  \inferrule* [lab=name] {} {{x} \bc \quotep{P}}
\end{mathpar} 

Note that $\vec{x}$ (resp. $\vec{P}$) denotes a vector of names
(resp. processes) of length $|\vec{x}|$ (resp. $|\vec{P}|$). We adopt
the following useful abbreviations.

\begin{mathpar}
   x?(\vec{y}).P := x.(\vec{y})P \and  x\clift{\vec{P}} := x.\clift{\vec{P}}
   \and x!(y) := \lift{x}{\dropn{y}}
   \and \Pi_{i=0}^{n-1}P_i := P_0 | \ldots | P_{n-1}
\end{mathpar}

\subsubsection{Structural congruence}

\paragraph{Free and bound names and alpha-equivalence.} At the
core of structural equivalence is alpha-equivalence which identifies
process that are the same up to a change of variable. Formally, we
recognize the distinction between free and bound names. The free names
of a process, $\freenames{P}$, may be calculated recursively as
follows:

\begin{mathpar}
\freenames{\pzero} := \emptyset
  \and \\
  \freenames{x?(y).P} := \{ x \} \cup (\freenames{P} \setminus \{ y \})
  \and 
  \freenames{x!\langle P \rangle} := \{ x \} \cup \{ P \} 
  \and \\
  \freenames{P|Q} := \freenames{P} \cup \freenames{Q}
  \and \\
  \freenames{@{x}} := \{ x \}
\end{mathpar}

$\pi$
$\quotep{\pi}$

$\freenames{-} : \pi \to \mathcal{P}(\quotep{\pi})$

\begin{eqnarray*}
  \freenames{\pzero} & := & \emptyset \\
  \freenames{x?(y).P} & := & \{ x \} \cup (\freenames{P} \setminus \{ y \}) \\
  \freenames{x!\langle P \rangle} & := & \{ x \} \cup \{ P \} \\
  \freenames{P|Q} & := & \freenames{P} \cup \freenames{Q} \\
  \freenames{\dropn{x}} & := & \{ x \}
\end{eqnarray*}

The bound names of a process, $\boundnames{P}$, are those names occurring in $P$
that are not free. For example, in $x?(y).0$, the name $x$ is free, while $y$ is bound.

\begin{mathpar}
  \inferrule* [lab=monoidal-laws] {} { P|Q \equiv Q|P \and P|0 \equiv P \and P|(Q|R) \equiv (P|Q)|R }
\end{mathpar}

\begin{mathpar}
  \inferrule* [lab=alpha-equivalence] {} { (x)P \equiv (y)P\{y/x\} \and y \not\in \freenames{P} }
\end{mathpar}

\begin{definition}
Then two processes, $P,Q$, are alpha-equivalent if $P = Q\{\vec{y}/\vec{x}\}$ for
some $\vec{x} \in \boundnames{Q},\vec{y} \in \boundnames{P}$, where $Q\{\vec{y}/\vec{x}\}$
denotes the capture-avoiding substitution of $\vec{y}$ for $\vec{x}$ in $Q$.
\end{definition}

\begin{definition}
  The {\em structural congruence} \cite{SangiorgiWalker} , $\equiv$,
  between processes is the least congruence containing
  alpha-equivalence, satisfying the abelian monoid laws
  (associativity, commutativity and $\pzero$ as identity) for parallel
  composition $|$ and for summation $+$.
\end{definition}

\subsection{Name equivalence}

We take name equivalence, written $\nameeq$, to be the smallest
equivalence relation generated by the following rules.

\begin{mathpar}
\inferrule*[lab=Quote-drop]
{ }
{ \quotep{@{x}} \nameeq x }

\inferrule*[lab=Struct-equiv]
{ P \scong Q }
{ \quotep{P} \nameeq \quotep{Q} }
\end{mathpar}

The astute reader will have noticed that the mutual recursion of names
and processes imposes a mutual recursion on alpha-equivalence and
structural equivalence via name-equivalence. Fortunately, all of this
works out pleasantly and we may calculate in the natural way, free of
concern. The reader interested in the details is referred to the
appendix \ref{appendix:rho_details}.

\subsection{Substitution}

We use $\Proc$ for the set of processes, $\QProc$ for the set of
names, and $\id{\{}\vec{y} / \vec{x} \id{\}}$ to denote partial maps,
$s : \QProc \rightarrow \QProc$. A map, $s$ lifts, uniquely, to a map
on process terms, $\widehat{s} : \Proc \rightarrow \Proc$ by the
following equations.

\begin{mathpar}
  (0) \psubstp{Q}{P} := 0 \\
  (R \juxtap S) \psubstp{Q}{P}
  :=    
  (R)\psubstp{Q}{P} \juxtap (S) \psubstp{Q}{P} \\
  (x?(y).R) \psubstp{Q}{P}    
  :=    
  (x)\substp{Q}{P} (z)\concat( (R \psubstn{z}{y}) \psubstp{Q}{P} ) \\
  (\lift{x}{R}) \psubstp{Q}{P}  
  :=
  \lift{(x)\substp{Q}{P}}{ R \psubstp{Q}{P} } \\
%   (\dropn{x})  \psubstp{Q}{P}       
%   := 
%   \left\{ 
%     \begin{array}{ccc} 
%       \dropn{\quotep{Q}} & & x \nameeq \quotep{P} \\
%       \dropn{x} & & otherwise \\
%     \end{array}
%   \right. 
  (\dropn{x})  \psubstp{Q}{P}       
  := 
  \left\{ 
    \begin{array}{ccc} 
      Q & & x \nameeq \quotep{P} \\
      \dropn{x} & & otherwise \\
    \end{array}
  \right.
\end{mathpar}
 

where

\begin{eqnarray}
  (x)\id{\{} \lpquote Q \rpquote / \lpquote P \rpquote \id{\}}            = 
  \left\{ 
    \begin{array}{ccc}
      \lpquote Q \rpquote & & x \nameeq \lpquote P \rpquote \\
      x & & otherwise \\
    \end{array}
  \right. \nonumber
\end{eqnarray}

and $z$ is chosen distinct from $\quotep{P}$, $\quotep{Q}$, the free
names in $Q$, and all the names in $R$. Our $\alpha$-equivalence will
be built in the standard way from this substitution.

\begin{remark}\label{rem:no_self_referential_names}
  One consequence of these definitions is that $\forall P. \quotep{P}
  \not\in \freenames{P}$.
\end{remark}

\subsection{ Dynamic quote: an example }

Anticipating something of what's to come, consider applying the
substitution, $\widehat{\id{\{}u / z \id{\}}}$, to the following pair
of processes, $\lift{w}{y!(z)}$ and $w[ \lpquote y!(z) \rpquote ]$.

\begin{eqnarray}
	\lift{w}{y!(z)}\widehat{\id{\{}u / z \id{\}}}
		& = &
		\lift{w}{y!(u)} \nonumber\\
	w[ \lpquote y!(z) \rpquote ] \widehat{ \id{\{}u / z \id{\}} }
		& = &
		w[ \lpquote y!(z) \rpquote ] \nonumber
\end{eqnarray}

Because the body of the process between quotes is impervious to
substitution, we get radically different answers. In fact, by
examining the first process in an input context,
e.g. $x?(z).\lift{w}{y!(z)}$, we see that the process under the lift
operator may be shaped by prefixed inputs binding a name inside it. In
this sense, the lift operator will be seen as a way to dynamically
construct processes before reifying them as names.

Finally equipped with these standard features we can present the
dynamics of the calculus.

\subsubsection{Operational semantics} 

Finally, we introduce the computational dynamics. What marks these
algebras as distinct from other more traditionally studied algebraic
structures, e.g. vector spaces or polynomial rings, is the manner in
which dynamics is captured. In traditional structures, dynamics is typically
expressed through morphisms between such structures, as in linear maps
between vector spaces or morphisms between rings. In algebras
associated with the semantics of computation, the dynamics is
expressed as part of the algebraic structure itself, through a
reduction reduction relation typically denoted by $\red$. Below, we
give a recursive presentation of this relation for the calculus used
in the encoding.

$\red \subseteq \pi \times \pi$
$\red : \pi \to \mathcal{P}(\pi)$

\begin{mathpar}
  \inferrule* [lab=Comm] { \textsf{match}( x_{src}, x_{trgt} ) } { x_{trgt}?(y)P \; | \; x_{src}!\langle {Q} \rangle \red P\{\quotep{Q}/y}\} }
  \and \\
  \inferrule* [lab=Par] {{P} \red {P}'} {{{P} | {Q}} \red {{P}' | {Q}}}
  \and
  \inferrule* [lab=Equiv]{{{P} \scong {P}'} \andalso {{P}' \red {Q}'} \andalso {{Q}' \scong {Q}}}{{P} \red {Q}}
\end{mathpar}

\begin{eqnarray*}
  match_{\equiv} (\quotep{P},\quotep{Q}) & := & P \equiv Q \\
  match_{\dagger}(\quotep{P},\quotep{Q}) & := & \forall R. P|Q \red^{*} R => R \red^{*} 0 \\
  match_{K}(\quotep{P},\quotep{Q}) & := & K \mbox{ for some context } K
\end{eqnarray*}

$u?(x)P | u!\langle Q \rangle \red P\{\quotep{Q}/x\}$

%We write $\wred$ for $\red^*$, and $P\red$ if $\exists Q $ such that $ P \red Q$.
We write $P\red$ if $\exists Q $ such that $ P \red Q$ and $P\not\red$, otherwise.

\section{Replication}

As mentioned before, it is known that replication (and hence
recursion) can be implemented in a higher-order process algebra
\cite{SangiorgiWalker}. As our first example of calculation with the
machinery thus far presented we give the construction explicitly in
the {\rhoc}.

\begin{eqnarray}
	D_{x} & := & \prefix{x}{y}{(\binpar{\outputp{x}{y}}{@{y}})} \nonumber\\
	\bangp_{x}{P} & := & \binpar{{x}!\langle{\binpar{D_{x}}{P}}\rangle}{D_{x}} \nonumber
\end{eqnarray}

\begin{eqnarray}
	\bangp_{x}{P} & & \nonumber\\
	=
	& {x}!\langle{(\prefix{x}{y}{(\outputp{x}{y} | @{y})) | P}}\rangle 
	      | \prefix{x}{y}{(\outputp{x}{y} | @{y})} & \nonumber\\
	\red
	& (\outputp{x}{y} | @{y})\substn{\quotep{(\prefix{x}{y}{(@{y} | \outputp{x}{y})) | P}}}{y} & \nonumber\\
	=
	& \outputp{x}{\quotep{(\prefix{x}{y}{(\outputp{x}{y} | @{y})) | P}}}
	  | {(\prefix{x}{y}{(\outputp{x}{y} | @{y})) | P}} & \nonumber\\
	\red
	& \ldots & \nonumber\\
	\red^*
	& P | P | \ldots & \nonumber
\end{eqnarray}

Of course, this encoding, as an implementation, runs away, unfolding
$\bangp{P}$ eagerly. A lazier and more implementable replication
operator, restricted to input-guarded processes, may be obtained as follows.

\begin{eqnarray}
\bangp{\prefix{u}{v}{P}} 
	:= 
	\binpar{\lift{x}{\prefix{u}{v}{(\binpar{D(x)}{P})}}}{D(x)} \nonumber
\end{eqnarray}

\begin{remark}
  Note that the lazier definition still does not deal with summation
  or mixed summation (i.e. sums over input and output). The reader is
  invited to construct definitions of replication that deal with these
  features. 

  Further, the definitions are parameterized in a name, $x$. Can you,
  gentle reader, make a definition that eliminates this parameter and
  guarantees no accidental interaction between the replication
  machinery and the process being replicated -- i.e. no accidental
  sharing of names used by the process to get its work done and the
  name(s) used by the replication to effect copying. This latter
  revision of the definition of replication is crucial to obtaining
  the expected identity $!!P \sim !P$.
\end{remark}

\begin{remark}\label{rem:paradoxical_combinator}
  The reader familiar with the lambda calculus will have noticed the
  similarity between $D$ and the paradoxical combinator.

  [Ed. note: the existence of this seems to suggest we have to be more
  restrictive on the set of processes and names we admit if we are to
  support no-cloning.]
\end{remark}

\subsubsection{Bisimulation}

The computational dynamics gives rise to another kind of equivalence,
the equivalence of computational behavior. As previously mentioned
this is typically captured \emph{via} some form of bisimulation.

% The notion we use in this paper is weak barbed bisimulation
% \cite{milner91polyadicpi}.

The notion we use in this paper is derived from weak barbed
bisimulation \cite{milner91polyadicpi}. 

\begin{definition}
An \emph{observation relation}, $\downarrow_{\mathcal N}$, over a set
of names, $\mathcal N$, is the smallest relation satisfying the rules
below.

\infrule[Out-barb]{y \in {\mathcal N}, \; x \nameeq y}
		  {\outputp{x}{v} \downarrow_{\mathcal N} x}
\infrule[Par-barb]{\mbox{$P\downarrow_{\mathcal N} x$ or $Q\downarrow_{\mathcal N} x$}}
		  {\binpar{P}{Q} \downarrow_{\mathcal N} x}

We write $P \Downarrow_{\mathcal N} x$ if there is $Q$ such that 
$P \wred Q$ and $Q \downarrow_{\mathcal N} x$.
\end{definition}

\begin{definition}
%\label{def.bbisim}
An  ${\mathcal N}$-\emph{barbed bisimulation} over a set of names, ${\mathcal N}$, is a symmetric binary relation 
${\mathcal S}_{\mathcal N}$ between agents such that $P\rel{S}_{\mathcal N}Q$ implies:
\begin{enumerate}
\item If $P \red P'$ then $Q \wred Q'$ and $P'\rel{S}_{\mathcal N} Q'$.
\item If $P\downarrow_{\mathcal N} x$, then $Q\Downarrow_{\mathcal N} x$.
\end{enumerate}
$P$ is ${\mathcal N}$-barbed bisimilar to $Q$, written
$P \wbbisim_{\mathcal N} Q$, if $P \rel{S}_{\mathcal N} Q$ for some ${\mathcal N}$-barbed bisimulation ${\mathcal S}_{\mathcal N}$.
\end{definition}

$\mathcal{R} \subseteq \pi \times \pi$

$P \mathcal{R} Q => \forall P'. P \red P' \Rightarrow \exists Q'. Q \red Q', P' \mathcal{R} Q'$

$P \vdash x \Rightarrow Q \vdash x$

\begin{mathpar}
  \inferrule*[lab=Out-barb]{x \nameeq y}{{y}!\langle{Q}\rangle \vdash x}
  \and
  \inferrule*[lab=Par-barb]{\mbox{$P\vdash x$ or $Q\vdash x$}}{\binpar{P}{Q} \vdash x}
\end{mathpar}

\subsubsection{Contexts}

One of the principle advantages of computational calculi like the
$\pi$-calculus is a well-defined notion of context,
contextual-equivalence and a correlation between
contextual-equivalence and notions of bisimulation. The notion of
context allows the decomposition of a process into (sub-)process and
its syntactic environment, its context. Thus, a context may be
thought of as a process with a ``hole'' (written $\Box$) in it. The
application of a context $M$ to a process $P$, written $M[P]$, is
tantamount to filling the hole in $M$ with $P$. In this paper we do
not need the full weight of this theory, but do make use of the notion
of context in the proof the main theorem. 

\begin{mathpar}
  \inferrule* [lab=summation] {} {{M_{M},M_{N}} \bc \Box \;|\; x.M_{A} \;|\; M_{M}+M_{N}}
  \and
  \inferrule* [lab=agent] {} {{M_{A}} \bc (\vec{x})M_{P} \;| \; \clift{P_0,\ldots,M_{P},\ldots,P_N}}
  \and \\
  \inferrule* [lab=process] {} {{M_{P}} \bc M_{N} \;| \;P|M_{P} }
\end{mathpar} 

\begin{mathpar}
  \inferrule* [lab=sychronization] {} {M_{N} \bc \Box \;|\; x?M_{F} \;|\; x!M_{C}}
  \and
  \inferrule* [lab=abstraction] {} {{M_{F}} \bc (x)M_{P} }
  \and
  \inferrule* [lab=concretion] {} {{M_{C}} \bc \langle M_{P} \rangle }
  \and \\
  \inferrule* [lab=process] {} {{M_{P}} \bc M_{N} \;| \;P|M_{P} }
\end{mathpar}

\begin{definition}[contextual application] Given a context $M$, and
  process $P$, we define the \emph{contextual application}, $M[P] :=
  M\{P/\Box\}$. That is, the contextual application of M to P is the
  substitution of $P$ for $\Box$ in $M$.
\end{definition}

$\meaningof{-} : L \to \mathcal{P}(\pi)$

\begin{mathpar}
  \inferrule* [lab=collection] {} {\meaningof{true} = \pi, \and \meaningof{~E} = \pi \setminus \meaningof{E}, \and \meaningof{E_{1} \& E_{2}} = \meaningof{E_{1}} \cap \meaningof{E_{2}}}
\end{mathpar}

\begin{mathpar}
  \inferrule* [lab=structure] {} {\meaningof{0} = \{ P \in \pi | P \equiv 0 \}, \and \\ \meaningof{E_1 | E_2} = \{ P \in \pi | P \equiv P_{1} | P_{2}, P_{1} \in \meaningof{E_{1}}, P_{2} \in \meaningof{E_2}\} }
\end{mathpar}

\begin{mathpar}
 \inferrule* [lab=behavior] {} {\meaningof{\langle a?b \rangle E} = \{ P \in \pi | P \equiv Q | u?(y)P', \\ \and \\\\ \and \\ \;\;\; u \in \meaningof{a}, \forall z.P'\{z/y\} \in \meaningof{E\{z/b\}}\}, \and \\ \meaningof{a!E} = \{ P \in \pi | P \equiv Q | x!\langle P' \rangle, x \in \meaningof{a} P' \in \meaningof{E}\} }
\end{mathpar}

\begin{mathpar}
 \inferrule* [lab=nominal] {} {\meaningof{\quotep{E}} = \{ \quotep{P} \in \quotep{\pi} | P \in \meaningof{E} \}, \and \meaningof{\quotep{P}} = \{ \quotep{Q} \in \quotep{\pi} | P \equiv Q \} \and \\ \meaningof{@\quotep{E}} = \{ P \in \pi | P \equiv @x, x \in \meaningof{E} \}}
\end{mathpar}

\begin{eqnarray*}
  \\
  \meaningof{-} : TS \to ST
\end{eqnarray*}

\begin{eqnarray*}
  \\
  L : TS \to ST
\end{eqnarray*}

\begin{eqnarray*}
  \\
  P \models E \iff P \in \meaningof{E}
\end{eqnarray*}

\begin{eqnarray*}
  P \approx_{L} Q \iff \forall E \in L. P \models E \iff Q \models E
\end{eqnarray*}

\begin{eqnarray*}
  P \approx_{K} Q
\end{eqnarray*}

\begin{eqnarray*}
  P \approx Q
\end{eqnarray*}

$\approx_{K} = \approx = \approx_{L}$

\subsubsection{Contextual duality}

Note that contexts extend the quotation operation to a family of
operations from processes to names. Given a context, $M$, we can
define a \emph{nominal context}, $\quotep{M}$ by $\quotep{M}[P] :=
\quotep{M[P]}$. To foreshadow what is to come we observe that these
operations enjoy a duality with processes very much like the duality
between vectors and maps from vectors to scalars.

Further, because the calculus is essentially higher-order, we have a
correspondence between contexts and processes. More specifically,
given a name $x$ and a context $M$ we can construct $M^{*}_{x}$ such
that 

\begin{mathpar}
  M^{*}_{x} | \lift{x}{P} \red M[P]
\end{mathpar}

namely,

\begin{mathpar}
  M^{*}_{x} := x?(u).M[\dropn{u}]
\end{mathpar}

The dependence of $M^{*}_{x}$ on a name makes it an abstraction, 

\begin{mathpar}
  M^{*} := (x)x?(u).M[\dropn{u}]
\end{mathpar}

\subsection{Additional notation}

It will sometimes be convenient to denote the process a name
quotes. We already have the notation $x = \quotep{P}$, but it will be
convenient to introduce an alternate notation, $\procn{x}$, when we
want to emphasize the connection to the use of the name. Note that, by
virtue of name equivalence, $\quotep{\procn{x}} \nameeq x$; so, the
notation is consistent with previous definitions.

Further, because names have structure it is possible to effect
substitutions on the basis of that structure. This means we need to
upgrade our notation for substitutions, which we accomplish by
adapting comprehension notation. Thus,

\begin{mathpar}
  P\{ y / x : x \in S \}
\end{mathpar}

is interpreted to mean the process derived from P by replacing (in a
capture-avoiding manner) each occurrence of $x$ in $S$ by $y$. For example,

\begin{mathpar}
  P\{ \quotep{\procn{x}|\procn{x}} / x : x \in \freenames{P} \}
\end{mathpar}

will replace each (occurrence) of a free name $x$ in $P$ by
$\quotep{\procn{x}|\procn{x}}$.

Also, we will avail ourselves of the notation $x^{L}$ and $x^{R}$ to
denote injections of a name into disjoint copies of the name
space. There are numerous ways to accomplish this. One example can be
found in \cite{MeredithR05}. This notation overloads to vectors of
names: $\vec{x}^{\pi} := (x_{i}^{\pi} \; : \; 0 \leq i < |\vec{x}| )$ where $\pi \in \{L,R\}$.

We also use $P^{\Box} := P|\Box$.

In \cite{MeredithR05} an interpretation of the new operator is
given. It turns out that there are several possible interpretations
all enjoying the requisite algebraic properties of the operator (see
\cite{milner91polyadicpi}). We will therefore make liberal use of
$(\nu\; \vec{x})P$.

% subsection the_syntax_and_semantics_of_the_notation_system (end)   

\input{qm2pi.qmops} 

\input{qm2pi.sterngerlach} 

\input{qm2pi.metric} 

% section concurrent_process_calculi (end)

%\input{qm2pi.proofsketch}

% section proof sketch (end)

%\input{qm2pi.slviaknots} 

% section spatial logic via knots (end)

\input{qm2pi.conclusion}

% section conclusion (end)

%\input{qm2pi.dtcodes} 

% section wiring algorithm (end)

\input{qm2pi.ack} 

% section acknowledgments (end)

\newpage


\bibliographystyle{plain}   
\bibliography{../../biblios/main.bib}

\input{qm2pi.rhodetails}

\end{document}

 

%\documentclass[12pt]{llncs}
%\documentclass{jktr}

\usepackage[pdftex]{hyperref}                   
\usepackage {listings}
\usepackage {mathpartir}
\usepackage{bcprules}
%\usepackage{listings}
                       
\usepackage{graphicx} 
%\usepackage[margins=2.5cm,nohead,nofoot]{geometry}
%\usepackage{geometry}
\usepackage{amsfonts}
\usepackage{amstext}
\usepackage{latexsym}
\usepackage{amssymb}
\usepackage{color}


%\include{myPreamble}
\include{qm2pi.local} 

%\ifpdf
%\usepackage[pdftex]{graphicx}
%\else
%\usepackage{graphicx}
%\fi

 % \ifpdf
%  \usepackage{pdfsync}
%  \if


%\title{Brief Article}
%\author{David F. Snyder}
%\author{L.G. Meredith}

%\address{Dept. of Math., Texas State University--San Marcos, San Marcos, TX 78666}
       
\pagestyle{empty}


\begin{document}

\lstset{language=[Objective]Caml,frame=shadowbox}

\input{qm2pi.front}

% section front matter (end)

\input{qm2pi.intro} 
 
% section introduction (end)

% \input{qm2pi.knotations} 

% section notation (end)

\input{qm2pi.process.calculi} 

% section concurrent_process_calculi_and_spatial_logics_ (end)
    
%\input{qm2pi.knots2pi} 

%\input{qm2pi.trefoil} 

%\input{qm2pi.mainthm} 

% subsection basic_interpretation (end)

%\input{qm2pi.rho.presentation} 
\subsection{The syntax and semantics of the notation system}\label{sub:the_syntax_and_semantics_of_the_notation_system} % (fold)

We now summarize a technical presentation of the calculus that
embodies our theory of dynamics. The typical presentation of such a
calculus follows the style of giving generators and relations on
them. The grammar, below, describing term constructors, freely
generates the set of processes, $\Proc$. This set is then quotiented
by a relation known as structural congruence and it is over this set
that the notion of dynamics is expressed. This presentation is
essentially that of \cite{MeredithR05} with the addition of
polyadicity and summation. For readability we have relegated some of
the technical subtleties to an appendix.

\subsubsection{Process grammar}\label{subsub:process_grammar}

\begin{mathpar}
  \inferrule* [lab=synchronization] {} {{M} \bc \pzero \;|\; x?F \;|\; x!C }
  \and
  \inferrule* [lab=abstraction] {} {{F} \bc (x)P}
  \and
  \inferrule* [lab=concretion] {} {{C} \bc \langle Q \rangle}
  \and
  \inferrule* [lab=process] {} {{P,Q} \bc M \;| \;P|Q \;|\; @{x}}
  \and
  \inferrule* [lab=name] {} {{x} \bc \quotep{P}}
\end{mathpar} 

Note that $\vec{x}$ (resp. $\vec{P}$) denotes a vector of names
(resp. processes) of length $|\vec{x}|$ (resp. $|\vec{P}|$). We adopt
the following useful abbreviations.

\begin{mathpar}
   x?(\vec{y}).P := x.(\vec{y})P \and  x\clift{\vec{P}} := x.\clift{\vec{P}}
   \and x!(y) := \lift{x}{\dropn{y}}
   \and \Pi_{i=0}^{n-1}P_i := P_0 | \ldots | P_{n-1}
\end{mathpar}

\subsubsection{Structural congruence}

\paragraph{Free and bound names and alpha-equivalence.} At the
core of structural equivalence is alpha-equivalence which identifies
process that are the same up to a change of variable. Formally, we
recognize the distinction between free and bound names. The free names
of a process, $\freenames{P}$, may be calculated recursively as
follows:

\begin{mathpar}
\freenames{\pzero} := \emptyset
  \and \\
  \freenames{x?(y).P} := \{ x \} \cup (\freenames{P} \setminus \{ y \})
  \and 
  \freenames{x!\langle P \rangle} := \{ x \} \cup \{ P \} 
  \and \\
  \freenames{P|Q} := \freenames{P} \cup \freenames{Q}
  \and \\
  \freenames{@{x}} := \{ x \}
\end{mathpar}

$\pi$
$\quotep{\pi}$

$\freenames{-} : \pi \to \mathcal{P}(\quotep{\pi})$

\begin{eqnarray*}
  \freenames{\pzero} & := & \emptyset \\
  \freenames{x?(y).P} & := & \{ x \} \cup (\freenames{P} \setminus \{ y \}) \\
  \freenames{x!\langle P \rangle} & := & \{ x \} \cup \{ P \} \\
  \freenames{P|Q} & := & \freenames{P} \cup \freenames{Q} \\
  \freenames{\dropn{x}} & := & \{ x \}
\end{eqnarray*}

The bound names of a process, $\boundnames{P}$, are those names occurring in $P$
that are not free. For example, in $x?(y).0$, the name $x$ is free, while $y$ is bound.

\begin{mathpar}
  \inferrule* [lab=monoidal-laws] {} { P|Q \equiv Q|P \and P|0 \equiv P \and P|(Q|R) \equiv (P|Q)|R }
\end{mathpar}

\begin{mathpar}
  \inferrule* [lab=alpha-equivalence] {} { (x)P \equiv (y)P\{y/x\} \and y \not\in \freenames{P} }
\end{mathpar}

\begin{definition}
Then two processes, $P,Q$, are alpha-equivalent if $P = Q\{\vec{y}/\vec{x}\}$ for
some $\vec{x} \in \boundnames{Q},\vec{y} \in \boundnames{P}$, where $Q\{\vec{y}/\vec{x}\}$
denotes the capture-avoiding substitution of $\vec{y}$ for $\vec{x}$ in $Q$.
\end{definition}

\begin{definition}
  The {\em structural congruence} \cite{SangiorgiWalker} , $\equiv$,
  between processes is the least congruence containing
  alpha-equivalence, satisfying the abelian monoid laws
  (associativity, commutativity and $\pzero$ as identity) for parallel
  composition $|$ and for summation $+$.
\end{definition}

\subsection{Name equivalence}

We take name equivalence, written $\nameeq$, to be the smallest
equivalence relation generated by the following rules.

\begin{mathpar}
\inferrule*[lab=Quote-drop]
{ }
{ \quotep{@{x}} \nameeq x }

\inferrule*[lab=Struct-equiv]
{ P \scong Q }
{ \quotep{P} \nameeq \quotep{Q} }
\end{mathpar}

The astute reader will have noticed that the mutual recursion of names
and processes imposes a mutual recursion on alpha-equivalence and
structural equivalence via name-equivalence. Fortunately, all of this
works out pleasantly and we may calculate in the natural way, free of
concern. The reader interested in the details is referred to the
appendix \ref{appendix:rho_details}.

\subsection{Substitution}

We use $\Proc$ for the set of processes, $\QProc$ for the set of
names, and $\id{\{}\vec{y} / \vec{x} \id{\}}$ to denote partial maps,
$s : \QProc \rightarrow \QProc$. A map, $s$ lifts, uniquely, to a map
on process terms, $\widehat{s} : \Proc \rightarrow \Proc$ by the
following equations.

\begin{mathpar}
  (0) \psubstp{Q}{P} := 0 \\
  (R \juxtap S) \psubstp{Q}{P}
  :=    
  (R)\psubstp{Q}{P} \juxtap (S) \psubstp{Q}{P} \\
  (x?(y).R) \psubstp{Q}{P}    
  :=    
  (x)\substp{Q}{P} (z)\concat( (R \psubstn{z}{y}) \psubstp{Q}{P} ) \\
  (\lift{x}{R}) \psubstp{Q}{P}  
  :=
  \lift{(x)\substp{Q}{P}}{ R \psubstp{Q}{P} } \\
%   (\dropn{x})  \psubstp{Q}{P}       
%   := 
%   \left\{ 
%     \begin{array}{ccc} 
%       \dropn{\quotep{Q}} & & x \nameeq \quotep{P} \\
%       \dropn{x} & & otherwise \\
%     \end{array}
%   \right. 
  (\dropn{x})  \psubstp{Q}{P}       
  := 
  \left\{ 
    \begin{array}{ccc} 
      Q & & x \nameeq \quotep{P} \\
      \dropn{x} & & otherwise \\
    \end{array}
  \right.
\end{mathpar}
 

where

\begin{eqnarray}
  (x)\id{\{} \lpquote Q \rpquote / \lpquote P \rpquote \id{\}}            = 
  \left\{ 
    \begin{array}{ccc}
      \lpquote Q \rpquote & & x \nameeq \lpquote P \rpquote \\
      x & & otherwise \\
    \end{array}
  \right. \nonumber
\end{eqnarray}

and $z$ is chosen distinct from $\quotep{P}$, $\quotep{Q}$, the free
names in $Q$, and all the names in $R$. Our $\alpha$-equivalence will
be built in the standard way from this substitution.

\begin{remark}\label{rem:no_self_referential_names}
  One consequence of these definitions is that $\forall P. \quotep{P}
  \not\in \freenames{P}$.
\end{remark}

\subsection{ Dynamic quote: an example }

Anticipating something of what's to come, consider applying the
substitution, $\widehat{\id{\{}u / z \id{\}}}$, to the following pair
of processes, $\lift{w}{y!(z)}$ and $w[ \lpquote y!(z) \rpquote ]$.

\begin{eqnarray}
	\lift{w}{y!(z)}\widehat{\id{\{}u / z \id{\}}}
		& = &
		\lift{w}{y!(u)} \nonumber\\
	w[ \lpquote y!(z) \rpquote ] \widehat{ \id{\{}u / z \id{\}} }
		& = &
		w[ \lpquote y!(z) \rpquote ] \nonumber
\end{eqnarray}

Because the body of the process between quotes is impervious to
substitution, we get radically different answers. In fact, by
examining the first process in an input context,
e.g. $x?(z).\lift{w}{y!(z)}$, we see that the process under the lift
operator may be shaped by prefixed inputs binding a name inside it. In
this sense, the lift operator will be seen as a way to dynamically
construct processes before reifying them as names.

Finally equipped with these standard features we can present the
dynamics of the calculus.

\subsubsection{Operational semantics} 

Finally, we introduce the computational dynamics. What marks these
algebras as distinct from other more traditionally studied algebraic
structures, e.g. vector spaces or polynomial rings, is the manner in
which dynamics is captured. In traditional structures, dynamics is typically
expressed through morphisms between such structures, as in linear maps
between vector spaces or morphisms between rings. In algebras
associated with the semantics of computation, the dynamics is
expressed as part of the algebraic structure itself, through a
reduction reduction relation typically denoted by $\red$. Below, we
give a recursive presentation of this relation for the calculus used
in the encoding.

$\red \subseteq \pi \times \pi$
$\red : \pi \to \mathcal{P}(\pi)$

\begin{mathpar}
  \inferrule* [lab=Comm] { \textsf{match}( x_{src}, x_{trgt} ) } { x_{trgt}?(y)P \; | \; x_{src}!\langle {Q} \rangle \red P\{\quotep{Q}/y}\} }
  \and \\
  \inferrule* [lab=Par] {{P} \red {P}'} {{{P} | {Q}} \red {{P}' | {Q}}}
  \and
  \inferrule* [lab=Equiv]{{{P} \scong {P}'} \andalso {{P}' \red {Q}'} \andalso {{Q}' \scong {Q}}}{{P} \red {Q}}
\end{mathpar}

\begin{eqnarray*}
  match_{\equiv} (\quotep{P},\quotep{Q}) & := & P \equiv Q \\
  match_{\dagger}(\quotep{P},\quotep{Q}) & := & \forall R. P|Q \red^{*} R => R \red^{*} 0 \\
  match_{K}(\quotep{P},\quotep{Q}) & := & K \mbox{ for some context } K
\end{eqnarray*}

$u?(x)P | u!\langle Q \rangle \red P\{\quotep{Q}/x\}$

%We write $\wred$ for $\red^*$, and $P\red$ if $\exists Q $ such that $ P \red Q$.
We write $P\red$ if $\exists Q $ such that $ P \red Q$ and $P\not\red$, otherwise.

\section{Replication}

As mentioned before, it is known that replication (and hence
recursion) can be implemented in a higher-order process algebra
\cite{SangiorgiWalker}. As our first example of calculation with the
machinery thus far presented we give the construction explicitly in
the {\rhoc}.

\begin{eqnarray}
	D_{x} & := & \prefix{x}{y}{(\binpar{\outputp{x}{y}}{@{y}})} \nonumber\\
	\bangp_{x}{P} & := & \binpar{{x}!\langle{\binpar{D_{x}}{P}}\rangle}{D_{x}} \nonumber
\end{eqnarray}

\begin{eqnarray}
	\bangp_{x}{P} & & \nonumber\\
	=
	& {x}!\langle{(\prefix{x}{y}{(\outputp{x}{y} | @{y})) | P}}\rangle 
	      | \prefix{x}{y}{(\outputp{x}{y} | @{y})} & \nonumber\\
	\red
	& (\outputp{x}{y} | @{y})\substn{\quotep{(\prefix{x}{y}{(@{y} | \outputp{x}{y})) | P}}}{y} & \nonumber\\
	=
	& \outputp{x}{\quotep{(\prefix{x}{y}{(\outputp{x}{y} | @{y})) | P}}}
	  | {(\prefix{x}{y}{(\outputp{x}{y} | @{y})) | P}} & \nonumber\\
	\red
	& \ldots & \nonumber\\
	\red^*
	& P | P | \ldots & \nonumber
\end{eqnarray}

Of course, this encoding, as an implementation, runs away, unfolding
$\bangp{P}$ eagerly. A lazier and more implementable replication
operator, restricted to input-guarded processes, may be obtained as follows.

\begin{eqnarray}
\bangp{\prefix{u}{v}{P}} 
	:= 
	\binpar{\lift{x}{\prefix{u}{v}{(\binpar{D(x)}{P})}}}{D(x)} \nonumber
\end{eqnarray}

\begin{remark}
  Note that the lazier definition still does not deal with summation
  or mixed summation (i.e. sums over input and output). The reader is
  invited to construct definitions of replication that deal with these
  features. 

  Further, the definitions are parameterized in a name, $x$. Can you,
  gentle reader, make a definition that eliminates this parameter and
  guarantees no accidental interaction between the replication
  machinery and the process being replicated -- i.e. no accidental
  sharing of names used by the process to get its work done and the
  name(s) used by the replication to effect copying. This latter
  revision of the definition of replication is crucial to obtaining
  the expected identity $!!P \sim !P$.
\end{remark}

\begin{remark}\label{rem:paradoxical_combinator}
  The reader familiar with the lambda calculus will have noticed the
  similarity between $D$ and the paradoxical combinator.

  [Ed. note: the existence of this seems to suggest we have to be more
  restrictive on the set of processes and names we admit if we are to
  support no-cloning.]
\end{remark}

\subsubsection{Bisimulation}

The computational dynamics gives rise to another kind of equivalence,
the equivalence of computational behavior. As previously mentioned
this is typically captured \emph{via} some form of bisimulation.

% The notion we use in this paper is weak barbed bisimulation
% \cite{milner91polyadicpi}.

The notion we use in this paper is derived from weak barbed
bisimulation \cite{milner91polyadicpi}. 

\begin{definition}
An \emph{observation relation}, $\downarrow_{\mathcal N}$, over a set
of names, $\mathcal N$, is the smallest relation satisfying the rules
below.

\infrule[Out-barb]{y \in {\mathcal N}, \; x \nameeq y}
		  {\outputp{x}{v} \downarrow_{\mathcal N} x}
\infrule[Par-barb]{\mbox{$P\downarrow_{\mathcal N} x$ or $Q\downarrow_{\mathcal N} x$}}
		  {\binpar{P}{Q} \downarrow_{\mathcal N} x}

We write $P \Downarrow_{\mathcal N} x$ if there is $Q$ such that 
$P \wred Q$ and $Q \downarrow_{\mathcal N} x$.
\end{definition}

\begin{definition}
%\label{def.bbisim}
An  ${\mathcal N}$-\emph{barbed bisimulation} over a set of names, ${\mathcal N}$, is a symmetric binary relation 
${\mathcal S}_{\mathcal N}$ between agents such that $P\rel{S}_{\mathcal N}Q$ implies:
\begin{enumerate}
\item If $P \red P'$ then $Q \wred Q'$ and $P'\rel{S}_{\mathcal N} Q'$.
\item If $P\downarrow_{\mathcal N} x$, then $Q\Downarrow_{\mathcal N} x$.
\end{enumerate}
$P$ is ${\mathcal N}$-barbed bisimilar to $Q$, written
$P \wbbisim_{\mathcal N} Q$, if $P \rel{S}_{\mathcal N} Q$ for some ${\mathcal N}$-barbed bisimulation ${\mathcal S}_{\mathcal N}$.
\end{definition}

$\mathcal{R} \subseteq \pi \times \pi$

$P \mathcal{R} Q => \forall P'. P \red P' \Rightarrow \exists Q'. Q \red Q', P' \mathcal{R} Q'$

$P \vdash x \Rightarrow Q \vdash x$

\begin{mathpar}
  \inferrule*[lab=Out-barb]{x \nameeq y}{{y}!\langle{Q}\rangle \vdash x}
  \and
  \inferrule*[lab=Par-barb]{\mbox{$P\vdash x$ or $Q\vdash x$}}{\binpar{P}{Q} \vdash x}
\end{mathpar}

\subsubsection{Contexts}

One of the principle advantages of computational calculi like the
$\pi$-calculus is a well-defined notion of context,
contextual-equivalence and a correlation between
contextual-equivalence and notions of bisimulation. The notion of
context allows the decomposition of a process into (sub-)process and
its syntactic environment, its context. Thus, a context may be
thought of as a process with a ``hole'' (written $\Box$) in it. The
application of a context $M$ to a process $P$, written $M[P]$, is
tantamount to filling the hole in $M$ with $P$. In this paper we do
not need the full weight of this theory, but do make use of the notion
of context in the proof the main theorem. 

\begin{mathpar}
  \inferrule* [lab=summation] {} {{M_{M},M_{N}} \bc \Box \;|\; x.M_{A} \;|\; M_{M}+M_{N}}
  \and
  \inferrule* [lab=agent] {} {{M_{A}} \bc (\vec{x})M_{P} \;| \; \clift{P_0,\ldots,M_{P},\ldots,P_N}}
  \and \\
  \inferrule* [lab=process] {} {{M_{P}} \bc M_{N} \;| \;P|M_{P} }
\end{mathpar} 

\begin{mathpar}
  \inferrule* [lab=sychronization] {} {M_{N} \bc \Box \;|\; x?M_{F} \;|\; x!M_{C}}
  \and
  \inferrule* [lab=abstraction] {} {{M_{F}} \bc (x)M_{P} }
  \and
  \inferrule* [lab=concretion] {} {{M_{C}} \bc \langle M_{P} \rangle }
  \and \\
  \inferrule* [lab=process] {} {{M_{P}} \bc M_{N} \;| \;P|M_{P} }
\end{mathpar}

\begin{definition}[contextual application] Given a context $M$, and
  process $P$, we define the \emph{contextual application}, $M[P] :=
  M\{P/\Box\}$. That is, the contextual application of M to P is the
  substitution of $P$ for $\Box$ in $M$.
\end{definition}

$\meaningof{-} : L \to \mathcal{P}(\pi)$

\begin{mathpar}
  \inferrule* [lab=collection] {} {\meaningof{true} = \pi, \and \meaningof{~E} = \pi \setminus \meaningof{E}, \and \meaningof{E_{1} \& E_{2}} = \meaningof{E_{1}} \cap \meaningof{E_{2}}}
\end{mathpar}

\begin{mathpar}
  \inferrule* [lab=structure] {} {\meaningof{0} = \{ P \in \pi | P \equiv 0 \}, \and \\ \meaningof{E_1 | E_2} = \{ P \in \pi | P \equiv P_{1} | P_{2}, P_{1} \in \meaningof{E_{1}}, P_{2} \in \meaningof{E_2}\} }
\end{mathpar}

\begin{mathpar}
 \inferrule* [lab=behavior] {} {\meaningof{\langle a?b \rangle E} = \{ P \in \pi | P \equiv Q | u?(y)P', \\ \and \\\\ \and \\ \;\;\; u \in \meaningof{a}, \forall z.P'\{z/y\} \in \meaningof{E\{z/b\}}\}, \and \\ \meaningof{a!E} = \{ P \in \pi | P \equiv Q | x!\langle P' \rangle, x \in \meaningof{a} P' \in \meaningof{E}\} }
\end{mathpar}

\begin{mathpar}
 \inferrule* [lab=nominal] {} {\meaningof{\quotep{E}} = \{ \quotep{P} \in \quotep{\pi} | P \in \meaningof{E} \}, \and \meaningof{\quotep{P}} = \{ \quotep{Q} \in \quotep{\pi} | P \equiv Q \} \and \\ \meaningof{@\quotep{E}} = \{ P \in \pi | P \equiv @x, x \in \meaningof{E} \}}
\end{mathpar}

\begin{eqnarray*}
  \\
  \meaningof{-} : TS \to ST
\end{eqnarray*}

\begin{eqnarray*}
  \\
  L : TS \to ST
\end{eqnarray*}

\begin{eqnarray*}
  \\
  P \models E \iff P \in \meaningof{E}
\end{eqnarray*}

\begin{eqnarray*}
  P \approx_{L} Q \iff \forall E \in L. P \models E \iff Q \models E
\end{eqnarray*}

\begin{eqnarray*}
  P \approx_{K} Q
\end{eqnarray*}

\begin{eqnarray*}
  P \approx Q
\end{eqnarray*}

$\approx_{K} = \approx = \approx_{L}$

\subsubsection{Contextual duality}

Note that contexts extend the quotation operation to a family of
operations from processes to names. Given a context, $M$, we can
define a \emph{nominal context}, $\quotep{M}$ by $\quotep{M}[P] :=
\quotep{M[P]}$. To foreshadow what is to come we observe that these
operations enjoy a duality with processes very much like the duality
between vectors and maps from vectors to scalars.

Further, because the calculus is essentially higher-order, we have a
correspondence between contexts and processes. More specifically,
given a name $x$ and a context $M$ we can construct $M^{*}_{x}$ such
that 

\begin{mathpar}
  M^{*}_{x} | \lift{x}{P} \red M[P]
\end{mathpar}

namely,

\begin{mathpar}
  M^{*}_{x} := x?(u).M[\dropn{u}]
\end{mathpar}

The dependence of $M^{*}_{x}$ on a name makes it an abstraction, 

\begin{mathpar}
  M^{*} := (x)x?(u).M[\dropn{u}]
\end{mathpar}

\subsection{Additional notation}

It will sometimes be convenient to denote the process a name
quotes. We already have the notation $x = \quotep{P}$, but it will be
convenient to introduce an alternate notation, $\procn{x}$, when we
want to emphasize the connection to the use of the name. Note that, by
virtue of name equivalence, $\quotep{\procn{x}} \nameeq x$; so, the
notation is consistent with previous definitions.

Further, because names have structure it is possible to effect
substitutions on the basis of that structure. This means we need to
upgrade our notation for substitutions, which we accomplish by
adapting comprehension notation. Thus,

\begin{mathpar}
  P\{ y / x : x \in S \}
\end{mathpar}

is interpreted to mean the process derived from P by replacing (in a
capture-avoiding manner) each occurrence of $x$ in $S$ by $y$. For example,

\begin{mathpar}
  P\{ \quotep{\procn{x}|\procn{x}} / x : x \in \freenames{P} \}
\end{mathpar}

will replace each (occurrence) of a free name $x$ in $P$ by
$\quotep{\procn{x}|\procn{x}}$.

Also, we will avail ourselves of the notation $x^{L}$ and $x^{R}$ to
denote injections of a name into disjoint copies of the name
space. There are numerous ways to accomplish this. One example can be
found in \cite{MeredithR05}. This notation overloads to vectors of
names: $\vec{x}^{\pi} := (x_{i}^{\pi} \; : \; 0 \leq i < |\vec{x}| )$ where $\pi \in \{L,R\}$.

We also use $P^{\Box} := P|\Box$.

In \cite{MeredithR05} an interpretation of the new operator is
given. It turns out that there are several possible interpretations
all enjoying the requisite algebraic properties of the operator (see
\cite{milner91polyadicpi}). We will therefore make liberal use of
$(\nu\; \vec{x})P$.

% subsection the_syntax_and_semantics_of_the_notation_system (end)   

\input{qm2pi.qmops} 

\input{qm2pi.sterngerlach} 

\input{qm2pi.metric} 

% section concurrent_process_calculi (end)

%\input{qm2pi.proofsketch}

% section proof sketch (end)

%\input{qm2pi.slviaknots} 

% section spatial logic via knots (end)

\input{qm2pi.conclusion}

% section conclusion (end)

%\input{qm2pi.dtcodes} 

% section wiring algorithm (end)

\input{qm2pi.ack} 

% section acknowledgments (end)

\newpage


\bibliographystyle{plain}   
\bibliography{../../biblios/main.bib}

\input{qm2pi.rhodetails}

\end{document}

 

%\documentclass[12pt]{llncs}
%\documentclass{jktr}

\usepackage[pdftex]{hyperref}                   
\usepackage {listings}
\usepackage {mathpartir}
\usepackage{bcprules}
%\usepackage{listings}
                       
\usepackage{graphicx} 
%\usepackage[margins=2.5cm,nohead,nofoot]{geometry}
%\usepackage{geometry}
\usepackage{amsfonts}
\usepackage{amstext}
\usepackage{latexsym}
\usepackage{amssymb}
\usepackage{color}


%\include{myPreamble}
\include{qm2pi.local} 

%\ifpdf
%\usepackage[pdftex]{graphicx}
%\else
%\usepackage{graphicx}
%\fi

 % \ifpdf
%  \usepackage{pdfsync}
%  \if


%\title{Brief Article}
%\author{David F. Snyder}
%\author{L.G. Meredith}

%\address{Dept. of Math., Texas State University--San Marcos, San Marcos, TX 78666}
       
\pagestyle{empty}


\begin{document}

\lstset{language=[Objective]Caml,frame=shadowbox}

\input{qm2pi.front}

% section front matter (end)

\input{qm2pi.intro} 
 
% section introduction (end)

% \input{qm2pi.knotations} 

% section notation (end)

\input{qm2pi.process.calculi} 

% section concurrent_process_calculi_and_spatial_logics_ (end)
    
%\input{qm2pi.knots2pi} 

%\input{qm2pi.trefoil} 

%\input{qm2pi.mainthm} 

% subsection basic_interpretation (end)

%\input{qm2pi.rho.presentation} 
\subsection{The syntax and semantics of the notation system}\label{sub:the_syntax_and_semantics_of_the_notation_system} % (fold)

We now summarize a technical presentation of the calculus that
embodies our theory of dynamics. The typical presentation of such a
calculus follows the style of giving generators and relations on
them. The grammar, below, describing term constructors, freely
generates the set of processes, $\Proc$. This set is then quotiented
by a relation known as structural congruence and it is over this set
that the notion of dynamics is expressed. This presentation is
essentially that of \cite{MeredithR05} with the addition of
polyadicity and summation. For readability we have relegated some of
the technical subtleties to an appendix.

\subsubsection{Process grammar}\label{subsub:process_grammar}

\begin{mathpar}
  \inferrule* [lab=synchronization] {} {{M} \bc \pzero \;|\; x?F \;|\; x!C }
  \and
  \inferrule* [lab=abstraction] {} {{F} \bc (x)P}
  \and
  \inferrule* [lab=concretion] {} {{C} \bc \langle Q \rangle}
  \and
  \inferrule* [lab=process] {} {{P,Q} \bc M \;| \;P|Q \;|\; @{x}}
  \and
  \inferrule* [lab=name] {} {{x} \bc \quotep{P}}
\end{mathpar} 

Note that $\vec{x}$ (resp. $\vec{P}$) denotes a vector of names
(resp. processes) of length $|\vec{x}|$ (resp. $|\vec{P}|$). We adopt
the following useful abbreviations.

\begin{mathpar}
   x?(\vec{y}).P := x.(\vec{y})P \and  x\clift{\vec{P}} := x.\clift{\vec{P}}
   \and x!(y) := \lift{x}{\dropn{y}}
   \and \Pi_{i=0}^{n-1}P_i := P_0 | \ldots | P_{n-1}
\end{mathpar}

\subsubsection{Structural congruence}

\paragraph{Free and bound names and alpha-equivalence.} At the
core of structural equivalence is alpha-equivalence which identifies
process that are the same up to a change of variable. Formally, we
recognize the distinction between free and bound names. The free names
of a process, $\freenames{P}$, may be calculated recursively as
follows:

\begin{mathpar}
\freenames{\pzero} := \emptyset
  \and \\
  \freenames{x?(y).P} := \{ x \} \cup (\freenames{P} \setminus \{ y \})
  \and 
  \freenames{x!\langle P \rangle} := \{ x \} \cup \{ P \} 
  \and \\
  \freenames{P|Q} := \freenames{P} \cup \freenames{Q}
  \and \\
  \freenames{@{x}} := \{ x \}
\end{mathpar}

$\pi$
$\quotep{\pi}$

$\freenames{-} : \pi \to \mathcal{P}(\quotep{\pi})$

\begin{eqnarray*}
  \freenames{\pzero} & := & \emptyset \\
  \freenames{x?(y).P} & := & \{ x \} \cup (\freenames{P} \setminus \{ y \}) \\
  \freenames{x!\langle P \rangle} & := & \{ x \} \cup \{ P \} \\
  \freenames{P|Q} & := & \freenames{P} \cup \freenames{Q} \\
  \freenames{\dropn{x}} & := & \{ x \}
\end{eqnarray*}

The bound names of a process, $\boundnames{P}$, are those names occurring in $P$
that are not free. For example, in $x?(y).0$, the name $x$ is free, while $y$ is bound.

\begin{mathpar}
  \inferrule* [lab=monoidal-laws] {} { P|Q \equiv Q|P \and P|0 \equiv P \and P|(Q|R) \equiv (P|Q)|R }
\end{mathpar}

\begin{mathpar}
  \inferrule* [lab=alpha-equivalence] {} { (x)P \equiv (y)P\{y/x\} \and y \not\in \freenames{P} }
\end{mathpar}

\begin{definition}
Then two processes, $P,Q$, are alpha-equivalent if $P = Q\{\vec{y}/\vec{x}\}$ for
some $\vec{x} \in \boundnames{Q},\vec{y} \in \boundnames{P}$, where $Q\{\vec{y}/\vec{x}\}$
denotes the capture-avoiding substitution of $\vec{y}$ for $\vec{x}$ in $Q$.
\end{definition}

\begin{definition}
  The {\em structural congruence} \cite{SangiorgiWalker} , $\equiv$,
  between processes is the least congruence containing
  alpha-equivalence, satisfying the abelian monoid laws
  (associativity, commutativity and $\pzero$ as identity) for parallel
  composition $|$ and for summation $+$.
\end{definition}

\subsection{Name equivalence}

We take name equivalence, written $\nameeq$, to be the smallest
equivalence relation generated by the following rules.

\begin{mathpar}
\inferrule*[lab=Quote-drop]
{ }
{ \quotep{@{x}} \nameeq x }

\inferrule*[lab=Struct-equiv]
{ P \scong Q }
{ \quotep{P} \nameeq \quotep{Q} }
\end{mathpar}

The astute reader will have noticed that the mutual recursion of names
and processes imposes a mutual recursion on alpha-equivalence and
structural equivalence via name-equivalence. Fortunately, all of this
works out pleasantly and we may calculate in the natural way, free of
concern. The reader interested in the details is referred to the
appendix \ref{appendix:rho_details}.

\subsection{Substitution}

We use $\Proc$ for the set of processes, $\QProc$ for the set of
names, and $\id{\{}\vec{y} / \vec{x} \id{\}}$ to denote partial maps,
$s : \QProc \rightarrow \QProc$. A map, $s$ lifts, uniquely, to a map
on process terms, $\widehat{s} : \Proc \rightarrow \Proc$ by the
following equations.

\begin{mathpar}
  (0) \psubstp{Q}{P} := 0 \\
  (R \juxtap S) \psubstp{Q}{P}
  :=    
  (R)\psubstp{Q}{P} \juxtap (S) \psubstp{Q}{P} \\
  (x?(y).R) \psubstp{Q}{P}    
  :=    
  (x)\substp{Q}{P} (z)\concat( (R \psubstn{z}{y}) \psubstp{Q}{P} ) \\
  (\lift{x}{R}) \psubstp{Q}{P}  
  :=
  \lift{(x)\substp{Q}{P}}{ R \psubstp{Q}{P} } \\
%   (\dropn{x})  \psubstp{Q}{P}       
%   := 
%   \left\{ 
%     \begin{array}{ccc} 
%       \dropn{\quotep{Q}} & & x \nameeq \quotep{P} \\
%       \dropn{x} & & otherwise \\
%     \end{array}
%   \right. 
  (\dropn{x})  \psubstp{Q}{P}       
  := 
  \left\{ 
    \begin{array}{ccc} 
      Q & & x \nameeq \quotep{P} \\
      \dropn{x} & & otherwise \\
    \end{array}
  \right.
\end{mathpar}
 

where

\begin{eqnarray}
  (x)\id{\{} \lpquote Q \rpquote / \lpquote P \rpquote \id{\}}            = 
  \left\{ 
    \begin{array}{ccc}
      \lpquote Q \rpquote & & x \nameeq \lpquote P \rpquote \\
      x & & otherwise \\
    \end{array}
  \right. \nonumber
\end{eqnarray}

and $z$ is chosen distinct from $\quotep{P}$, $\quotep{Q}$, the free
names in $Q$, and all the names in $R$. Our $\alpha$-equivalence will
be built in the standard way from this substitution.

\begin{remark}\label{rem:no_self_referential_names}
  One consequence of these definitions is that $\forall P. \quotep{P}
  \not\in \freenames{P}$.
\end{remark}

\subsection{ Dynamic quote: an example }

Anticipating something of what's to come, consider applying the
substitution, $\widehat{\id{\{}u / z \id{\}}}$, to the following pair
of processes, $\lift{w}{y!(z)}$ and $w[ \lpquote y!(z) \rpquote ]$.

\begin{eqnarray}
	\lift{w}{y!(z)}\widehat{\id{\{}u / z \id{\}}}
		& = &
		\lift{w}{y!(u)} \nonumber\\
	w[ \lpquote y!(z) \rpquote ] \widehat{ \id{\{}u / z \id{\}} }
		& = &
		w[ \lpquote y!(z) \rpquote ] \nonumber
\end{eqnarray}

Because the body of the process between quotes is impervious to
substitution, we get radically different answers. In fact, by
examining the first process in an input context,
e.g. $x?(z).\lift{w}{y!(z)}$, we see that the process under the lift
operator may be shaped by prefixed inputs binding a name inside it. In
this sense, the lift operator will be seen as a way to dynamically
construct processes before reifying them as names.

Finally equipped with these standard features we can present the
dynamics of the calculus.

\subsubsection{Operational semantics} 

Finally, we introduce the computational dynamics. What marks these
algebras as distinct from other more traditionally studied algebraic
structures, e.g. vector spaces or polynomial rings, is the manner in
which dynamics is captured. In traditional structures, dynamics is typically
expressed through morphisms between such structures, as in linear maps
between vector spaces or morphisms between rings. In algebras
associated with the semantics of computation, the dynamics is
expressed as part of the algebraic structure itself, through a
reduction reduction relation typically denoted by $\red$. Below, we
give a recursive presentation of this relation for the calculus used
in the encoding.

$\red \subseteq \pi \times \pi$
$\red : \pi \to \mathcal{P}(\pi)$

\begin{mathpar}
  \inferrule* [lab=Comm] { \textsf{match}( x_{src}, x_{trgt} ) } { x_{trgt}?(y)P \; | \; x_{src}!\langle {Q} \rangle \red P\{\quotep{Q}/y}\} }
  \and \\
  \inferrule* [lab=Par] {{P} \red {P}'} {{{P} | {Q}} \red {{P}' | {Q}}}
  \and
  \inferrule* [lab=Equiv]{{{P} \scong {P}'} \andalso {{P}' \red {Q}'} \andalso {{Q}' \scong {Q}}}{{P} \red {Q}}
\end{mathpar}

\begin{eqnarray*}
  match_{\equiv} (\quotep{P},\quotep{Q}) & := & P \equiv Q \\
  match_{\dagger}(\quotep{P},\quotep{Q}) & := & \forall R. P|Q \red^{*} R => R \red^{*} 0 \\
  match_{K}(\quotep{P},\quotep{Q}) & := & K \mbox{ for some context } K
\end{eqnarray*}

$u?(x)P | u!\langle Q \rangle \red P\{\quotep{Q}/x\}$

%We write $\wred$ for $\red^*$, and $P\red$ if $\exists Q $ such that $ P \red Q$.
We write $P\red$ if $\exists Q $ such that $ P \red Q$ and $P\not\red$, otherwise.

\section{Replication}

As mentioned before, it is known that replication (and hence
recursion) can be implemented in a higher-order process algebra
\cite{SangiorgiWalker}. As our first example of calculation with the
machinery thus far presented we give the construction explicitly in
the {\rhoc}.

\begin{eqnarray}
	D_{x} & := & \prefix{x}{y}{(\binpar{\outputp{x}{y}}{@{y}})} \nonumber\\
	\bangp_{x}{P} & := & \binpar{{x}!\langle{\binpar{D_{x}}{P}}\rangle}{D_{x}} \nonumber
\end{eqnarray}

\begin{eqnarray}
	\bangp_{x}{P} & & \nonumber\\
	=
	& {x}!\langle{(\prefix{x}{y}{(\outputp{x}{y} | @{y})) | P}}\rangle 
	      | \prefix{x}{y}{(\outputp{x}{y} | @{y})} & \nonumber\\
	\red
	& (\outputp{x}{y} | @{y})\substn{\quotep{(\prefix{x}{y}{(@{y} | \outputp{x}{y})) | P}}}{y} & \nonumber\\
	=
	& \outputp{x}{\quotep{(\prefix{x}{y}{(\outputp{x}{y} | @{y})) | P}}}
	  | {(\prefix{x}{y}{(\outputp{x}{y} | @{y})) | P}} & \nonumber\\
	\red
	& \ldots & \nonumber\\
	\red^*
	& P | P | \ldots & \nonumber
\end{eqnarray}

Of course, this encoding, as an implementation, runs away, unfolding
$\bangp{P}$ eagerly. A lazier and more implementable replication
operator, restricted to input-guarded processes, may be obtained as follows.

\begin{eqnarray}
\bangp{\prefix{u}{v}{P}} 
	:= 
	\binpar{\lift{x}{\prefix{u}{v}{(\binpar{D(x)}{P})}}}{D(x)} \nonumber
\end{eqnarray}

\begin{remark}
  Note that the lazier definition still does not deal with summation
  or mixed summation (i.e. sums over input and output). The reader is
  invited to construct definitions of replication that deal with these
  features. 

  Further, the definitions are parameterized in a name, $x$. Can you,
  gentle reader, make a definition that eliminates this parameter and
  guarantees no accidental interaction between the replication
  machinery and the process being replicated -- i.e. no accidental
  sharing of names used by the process to get its work done and the
  name(s) used by the replication to effect copying. This latter
  revision of the definition of replication is crucial to obtaining
  the expected identity $!!P \sim !P$.
\end{remark}

\begin{remark}\label{rem:paradoxical_combinator}
  The reader familiar with the lambda calculus will have noticed the
  similarity between $D$ and the paradoxical combinator.

  [Ed. note: the existence of this seems to suggest we have to be more
  restrictive on the set of processes and names we admit if we are to
  support no-cloning.]
\end{remark}

\subsubsection{Bisimulation}

The computational dynamics gives rise to another kind of equivalence,
the equivalence of computational behavior. As previously mentioned
this is typically captured \emph{via} some form of bisimulation.

% The notion we use in this paper is weak barbed bisimulation
% \cite{milner91polyadicpi}.

The notion we use in this paper is derived from weak barbed
bisimulation \cite{milner91polyadicpi}. 

\begin{definition}
An \emph{observation relation}, $\downarrow_{\mathcal N}$, over a set
of names, $\mathcal N$, is the smallest relation satisfying the rules
below.

\infrule[Out-barb]{y \in {\mathcal N}, \; x \nameeq y}
		  {\outputp{x}{v} \downarrow_{\mathcal N} x}
\infrule[Par-barb]{\mbox{$P\downarrow_{\mathcal N} x$ or $Q\downarrow_{\mathcal N} x$}}
		  {\binpar{P}{Q} \downarrow_{\mathcal N} x}

We write $P \Downarrow_{\mathcal N} x$ if there is $Q$ such that 
$P \wred Q$ and $Q \downarrow_{\mathcal N} x$.
\end{definition}

\begin{definition}
%\label{def.bbisim}
An  ${\mathcal N}$-\emph{barbed bisimulation} over a set of names, ${\mathcal N}$, is a symmetric binary relation 
${\mathcal S}_{\mathcal N}$ between agents such that $P\rel{S}_{\mathcal N}Q$ implies:
\begin{enumerate}
\item If $P \red P'$ then $Q \wred Q'$ and $P'\rel{S}_{\mathcal N} Q'$.
\item If $P\downarrow_{\mathcal N} x$, then $Q\Downarrow_{\mathcal N} x$.
\end{enumerate}
$P$ is ${\mathcal N}$-barbed bisimilar to $Q$, written
$P \wbbisim_{\mathcal N} Q$, if $P \rel{S}_{\mathcal N} Q$ for some ${\mathcal N}$-barbed bisimulation ${\mathcal S}_{\mathcal N}$.
\end{definition}

$\mathcal{R} \subseteq \pi \times \pi$

$P \mathcal{R} Q => \forall P'. P \red P' \Rightarrow \exists Q'. Q \red Q', P' \mathcal{R} Q'$

$P \vdash x \Rightarrow Q \vdash x$

\begin{mathpar}
  \inferrule*[lab=Out-barb]{x \nameeq y}{{y}!\langle{Q}\rangle \vdash x}
  \and
  \inferrule*[lab=Par-barb]{\mbox{$P\vdash x$ or $Q\vdash x$}}{\binpar{P}{Q} \vdash x}
\end{mathpar}

\subsubsection{Contexts}

One of the principle advantages of computational calculi like the
$\pi$-calculus is a well-defined notion of context,
contextual-equivalence and a correlation between
contextual-equivalence and notions of bisimulation. The notion of
context allows the decomposition of a process into (sub-)process and
its syntactic environment, its context. Thus, a context may be
thought of as a process with a ``hole'' (written $\Box$) in it. The
application of a context $M$ to a process $P$, written $M[P]$, is
tantamount to filling the hole in $M$ with $P$. In this paper we do
not need the full weight of this theory, but do make use of the notion
of context in the proof the main theorem. 

\begin{mathpar}
  \inferrule* [lab=summation] {} {{M_{M},M_{N}} \bc \Box \;|\; x.M_{A} \;|\; M_{M}+M_{N}}
  \and
  \inferrule* [lab=agent] {} {{M_{A}} \bc (\vec{x})M_{P} \;| \; \clift{P_0,\ldots,M_{P},\ldots,P_N}}
  \and \\
  \inferrule* [lab=process] {} {{M_{P}} \bc M_{N} \;| \;P|M_{P} }
\end{mathpar} 

\begin{mathpar}
  \inferrule* [lab=sychronization] {} {M_{N} \bc \Box \;|\; x?M_{F} \;|\; x!M_{C}}
  \and
  \inferrule* [lab=abstraction] {} {{M_{F}} \bc (x)M_{P} }
  \and
  \inferrule* [lab=concretion] {} {{M_{C}} \bc \langle M_{P} \rangle }
  \and \\
  \inferrule* [lab=process] {} {{M_{P}} \bc M_{N} \;| \;P|M_{P} }
\end{mathpar}

\begin{definition}[contextual application] Given a context $M$, and
  process $P$, we define the \emph{contextual application}, $M[P] :=
  M\{P/\Box\}$. That is, the contextual application of M to P is the
  substitution of $P$ for $\Box$ in $M$.
\end{definition}

$\meaningof{-} : L \to \mathcal{P}(\pi)$

\begin{mathpar}
  \inferrule* [lab=collection] {} {\meaningof{true} = \pi, \and \meaningof{~E} = \pi \setminus \meaningof{E}, \and \meaningof{E_{1} \& E_{2}} = \meaningof{E_{1}} \cap \meaningof{E_{2}}}
\end{mathpar}

\begin{mathpar}
  \inferrule* [lab=structure] {} {\meaningof{0} = \{ P \in \pi | P \equiv 0 \}, \and \\ \meaningof{E_1 | E_2} = \{ P \in \pi | P \equiv P_{1} | P_{2}, P_{1} \in \meaningof{E_{1}}, P_{2} \in \meaningof{E_2}\} }
\end{mathpar}

\begin{mathpar}
 \inferrule* [lab=behavior] {} {\meaningof{\langle a?b \rangle E} = \{ P \in \pi | P \equiv Q | u?(y)P', \\ \and \\\\ \and \\ \;\;\; u \in \meaningof{a}, \forall z.P'\{z/y\} \in \meaningof{E\{z/b\}}\}, \and \\ \meaningof{a!E} = \{ P \in \pi | P \equiv Q | x!\langle P' \rangle, x \in \meaningof{a} P' \in \meaningof{E}\} }
\end{mathpar}

\begin{mathpar}
 \inferrule* [lab=nominal] {} {\meaningof{\quotep{E}} = \{ \quotep{P} \in \quotep{\pi} | P \in \meaningof{E} \}, \and \meaningof{\quotep{P}} = \{ \quotep{Q} \in \quotep{\pi} | P \equiv Q \} \and \\ \meaningof{@\quotep{E}} = \{ P \in \pi | P \equiv @x, x \in \meaningof{E} \}}
\end{mathpar}

\begin{eqnarray*}
  \\
  \meaningof{-} : TS \to ST
\end{eqnarray*}

\begin{eqnarray*}
  \\
  L : TS \to ST
\end{eqnarray*}

\begin{eqnarray*}
  \\
  P \models E \iff P \in \meaningof{E}
\end{eqnarray*}

\begin{eqnarray*}
  P \approx_{L} Q \iff \forall E \in L. P \models E \iff Q \models E
\end{eqnarray*}

\begin{eqnarray*}
  P \approx_{K} Q
\end{eqnarray*}

\begin{eqnarray*}
  P \approx Q
\end{eqnarray*}

$\approx_{K} = \approx = \approx_{L}$

\subsubsection{Contextual duality}

Note that contexts extend the quotation operation to a family of
operations from processes to names. Given a context, $M$, we can
define a \emph{nominal context}, $\quotep{M}$ by $\quotep{M}[P] :=
\quotep{M[P]}$. To foreshadow what is to come we observe that these
operations enjoy a duality with processes very much like the duality
between vectors and maps from vectors to scalars.

Further, because the calculus is essentially higher-order, we have a
correspondence between contexts and processes. More specifically,
given a name $x$ and a context $M$ we can construct $M^{*}_{x}$ such
that 

\begin{mathpar}
  M^{*}_{x} | \lift{x}{P} \red M[P]
\end{mathpar}

namely,

\begin{mathpar}
  M^{*}_{x} := x?(u).M[\dropn{u}]
\end{mathpar}

The dependence of $M^{*}_{x}$ on a name makes it an abstraction, 

\begin{mathpar}
  M^{*} := (x)x?(u).M[\dropn{u}]
\end{mathpar}

\subsection{Additional notation}

It will sometimes be convenient to denote the process a name
quotes. We already have the notation $x = \quotep{P}$, but it will be
convenient to introduce an alternate notation, $\procn{x}$, when we
want to emphasize the connection to the use of the name. Note that, by
virtue of name equivalence, $\quotep{\procn{x}} \nameeq x$; so, the
notation is consistent with previous definitions.

Further, because names have structure it is possible to effect
substitutions on the basis of that structure. This means we need to
upgrade our notation for substitutions, which we accomplish by
adapting comprehension notation. Thus,

\begin{mathpar}
  P\{ y / x : x \in S \}
\end{mathpar}

is interpreted to mean the process derived from P by replacing (in a
capture-avoiding manner) each occurrence of $x$ in $S$ by $y$. For example,

\begin{mathpar}
  P\{ \quotep{\procn{x}|\procn{x}} / x : x \in \freenames{P} \}
\end{mathpar}

will replace each (occurrence) of a free name $x$ in $P$ by
$\quotep{\procn{x}|\procn{x}}$.

Also, we will avail ourselves of the notation $x^{L}$ and $x^{R}$ to
denote injections of a name into disjoint copies of the name
space. There are numerous ways to accomplish this. One example can be
found in \cite{MeredithR05}. This notation overloads to vectors of
names: $\vec{x}^{\pi} := (x_{i}^{\pi} \; : \; 0 \leq i < |\vec{x}| )$ where $\pi \in \{L,R\}$.

We also use $P^{\Box} := P|\Box$.

In \cite{MeredithR05} an interpretation of the new operator is
given. It turns out that there are several possible interpretations
all enjoying the requisite algebraic properties of the operator (see
\cite{milner91polyadicpi}). We will therefore make liberal use of
$(\nu\; \vec{x})P$.

% subsection the_syntax_and_semantics_of_the_notation_system (end)   

\input{qm2pi.qmops} 

\input{qm2pi.sterngerlach} 

\input{qm2pi.metric} 

% section concurrent_process_calculi (end)

%\input{qm2pi.proofsketch}

% section proof sketch (end)

%\input{qm2pi.slviaknots} 

% section spatial logic via knots (end)

\input{qm2pi.conclusion}

% section conclusion (end)

%\input{qm2pi.dtcodes} 

% section wiring algorithm (end)

\input{qm2pi.ack} 

% section acknowledgments (end)

\newpage


\bibliographystyle{plain}   
\bibliography{../../biblios/main.bib}

\input{qm2pi.rhodetails}

\end{document}

 

% subsection basic_interpretation (end)

%\input{qm2pi.rho.presentation} 
\subsection{The syntax and semantics of the notation system}\label{sub:the_syntax_and_semantics_of_the_notation_system} % (fold)

We now summarize a technical presentation of the calculus that
embodies our theory of dynamics. The typical presentation of such a
calculus follows the style of giving generators and relations on
them. The grammar, below, describing term constructors, freely
generates the set of processes, $\Proc$. This set is then quotiented
by a relation known as structural congruence and it is over this set
that the notion of dynamics is expressed. This presentation is
essentially that of \cite{MeredithR05} with the addition of
polyadicity and summation. For readability we have relegated some of
the technical subtleties to an appendix.

\subsubsection{Process grammar}\label{subsub:process_grammar}

\begin{mathpar}
  \inferrule* [lab=synchronization] {} {{M} \bc \pzero \;|\; x?F \;|\; x!C }
  \and
  \inferrule* [lab=abstraction] {} {{F} \bc (x)P}
  \and
  \inferrule* [lab=concretion] {} {{C} \bc \langle Q \rangle}
  \and
  \inferrule* [lab=process] {} {{P,Q} \bc M \;| \;P|Q \;|\; @{x}}
  \and
  \inferrule* [lab=name] {} {{x} \bc \quotep{P}}
\end{mathpar} 

Note that $\vec{x}$ (resp. $\vec{P}$) denotes a vector of names
(resp. processes) of length $|\vec{x}|$ (resp. $|\vec{P}|$). We adopt
the following useful abbreviations.

\begin{mathpar}
   x?(\vec{y}).P := x.(\vec{y})P \and  x\clift{\vec{P}} := x.\clift{\vec{P}}
   \and x!(y) := \lift{x}{\dropn{y}}
   \and \Pi_{i=0}^{n-1}P_i := P_0 | \ldots | P_{n-1}
\end{mathpar}

\subsubsection{Structural congruence}

\paragraph{Free and bound names and alpha-equivalence.} At the
core of structural equivalence is alpha-equivalence which identifies
process that are the same up to a change of variable. Formally, we
recognize the distinction between free and bound names. The free names
of a process, $\freenames{P}$, may be calculated recursively as
follows:

\begin{mathpar}
\freenames{\pzero} := \emptyset
  \and \\
  \freenames{x?(y).P} := \{ x \} \cup (\freenames{P} \setminus \{ y \})
  \and 
  \freenames{x!\langle P \rangle} := \{ x \} \cup \{ P \} 
  \and \\
  \freenames{P|Q} := \freenames{P} \cup \freenames{Q}
  \and \\
  \freenames{@{x}} := \{ x \}
\end{mathpar}

$\pi$
$\quotep{\pi}$

$\freenames{-} : \pi \to \mathcal{P}(\quotep{\pi})$

\begin{eqnarray*}
  \freenames{\pzero} & := & \emptyset \\
  \freenames{x?(y).P} & := & \{ x \} \cup (\freenames{P} \setminus \{ y \}) \\
  \freenames{x!\langle P \rangle} & := & \{ x \} \cup \{ P \} \\
  \freenames{P|Q} & := & \freenames{P} \cup \freenames{Q} \\
  \freenames{\dropn{x}} & := & \{ x \}
\end{eqnarray*}

The bound names of a process, $\boundnames{P}$, are those names occurring in $P$
that are not free. For example, in $x?(y).0$, the name $x$ is free, while $y$ is bound.

\begin{mathpar}
  \inferrule* [lab=monoidal-laws] {} { P|Q \equiv Q|P \and P|0 \equiv P \and P|(Q|R) \equiv (P|Q)|R }
\end{mathpar}

\begin{mathpar}
  \inferrule* [lab=alpha-equivalence] {} { (x)P \equiv (y)P\{y/x\} \and y \not\in \freenames{P} }
\end{mathpar}

\begin{definition}
Then two processes, $P,Q$, are alpha-equivalent if $P = Q\{\vec{y}/\vec{x}\}$ for
some $\vec{x} \in \boundnames{Q},\vec{y} \in \boundnames{P}$, where $Q\{\vec{y}/\vec{x}\}$
denotes the capture-avoiding substitution of $\vec{y}$ for $\vec{x}$ in $Q$.
\end{definition}

\begin{definition}
  The {\em structural congruence} \cite{SangiorgiWalker} , $\equiv$,
  between processes is the least congruence containing
  alpha-equivalence, satisfying the abelian monoid laws
  (associativity, commutativity and $\pzero$ as identity) for parallel
  composition $|$ and for summation $+$.
\end{definition}

\subsection{Name equivalence}

We take name equivalence, written $\nameeq$, to be the smallest
equivalence relation generated by the following rules.

\begin{mathpar}
\inferrule*[lab=Quote-drop]
{ }
{ \quotep{@{x}} \nameeq x }

\inferrule*[lab=Struct-equiv]
{ P \scong Q }
{ \quotep{P} \nameeq \quotep{Q} }
\end{mathpar}

The astute reader will have noticed that the mutual recursion of names
and processes imposes a mutual recursion on alpha-equivalence and
structural equivalence via name-equivalence. Fortunately, all of this
works out pleasantly and we may calculate in the natural way, free of
concern. The reader interested in the details is referred to the
appendix \ref{appendix:rho_details}.

\subsection{Substitution}

We use $\Proc$ for the set of processes, $\QProc$ for the set of
names, and $\id{\{}\vec{y} / \vec{x} \id{\}}$ to denote partial maps,
$s : \QProc \rightarrow \QProc$. A map, $s$ lifts, uniquely, to a map
on process terms, $\widehat{s} : \Proc \rightarrow \Proc$ by the
following equations.

\begin{mathpar}
  (0) \psubstp{Q}{P} := 0 \\
  (R \juxtap S) \psubstp{Q}{P}
  :=    
  (R)\psubstp{Q}{P} \juxtap (S) \psubstp{Q}{P} \\
  (x?(y).R) \psubstp{Q}{P}    
  :=    
  (x)\substp{Q}{P} (z)\concat( (R \psubstn{z}{y}) \psubstp{Q}{P} ) \\
  (\lift{x}{R}) \psubstp{Q}{P}  
  :=
  \lift{(x)\substp{Q}{P}}{ R \psubstp{Q}{P} } \\
%   (\dropn{x})  \psubstp{Q}{P}       
%   := 
%   \left\{ 
%     \begin{array}{ccc} 
%       \dropn{\quotep{Q}} & & x \nameeq \quotep{P} \\
%       \dropn{x} & & otherwise \\
%     \end{array}
%   \right. 
  (\dropn{x})  \psubstp{Q}{P}       
  := 
  \left\{ 
    \begin{array}{ccc} 
      Q & & x \nameeq \quotep{P} \\
      \dropn{x} & & otherwise \\
    \end{array}
  \right.
\end{mathpar}
 

where

\begin{eqnarray}
  (x)\id{\{} \lpquote Q \rpquote / \lpquote P \rpquote \id{\}}            = 
  \left\{ 
    \begin{array}{ccc}
      \lpquote Q \rpquote & & x \nameeq \lpquote P \rpquote \\
      x & & otherwise \\
    \end{array}
  \right. \nonumber
\end{eqnarray}

and $z$ is chosen distinct from $\quotep{P}$, $\quotep{Q}$, the free
names in $Q$, and all the names in $R$. Our $\alpha$-equivalence will
be built in the standard way from this substitution.

\begin{remark}\label{rem:no_self_referential_names}
  One consequence of these definitions is that $\forall P. \quotep{P}
  \not\in \freenames{P}$.
\end{remark}

\subsection{ Dynamic quote: an example }

Anticipating something of what's to come, consider applying the
substitution, $\widehat{\id{\{}u / z \id{\}}}$, to the following pair
of processes, $\lift{w}{y!(z)}$ and $w[ \lpquote y!(z) \rpquote ]$.

\begin{eqnarray}
	\lift{w}{y!(z)}\widehat{\id{\{}u / z \id{\}}}
		& = &
		\lift{w}{y!(u)} \nonumber\\
	w[ \lpquote y!(z) \rpquote ] \widehat{ \id{\{}u / z \id{\}} }
		& = &
		w[ \lpquote y!(z) \rpquote ] \nonumber
\end{eqnarray}

Because the body of the process between quotes is impervious to
substitution, we get radically different answers. In fact, by
examining the first process in an input context,
e.g. $x?(z).\lift{w}{y!(z)}$, we see that the process under the lift
operator may be shaped by prefixed inputs binding a name inside it. In
this sense, the lift operator will be seen as a way to dynamically
construct processes before reifying them as names.

Finally equipped with these standard features we can present the
dynamics of the calculus.

\subsubsection{Operational semantics} 

Finally, we introduce the computational dynamics. What marks these
algebras as distinct from other more traditionally studied algebraic
structures, e.g. vector spaces or polynomial rings, is the manner in
which dynamics is captured. In traditional structures, dynamics is typically
expressed through morphisms between such structures, as in linear maps
between vector spaces or morphisms between rings. In algebras
associated with the semantics of computation, the dynamics is
expressed as part of the algebraic structure itself, through a
reduction reduction relation typically denoted by $\red$. Below, we
give a recursive presentation of this relation for the calculus used
in the encoding.

$\red \subseteq \pi \times \pi$
$\red : \pi \to \mathcal{P}(\pi)$

\begin{mathpar}
  \inferrule* [lab=Comm] { \textsf{match}( x_{src}, x_{trgt} ) } { x_{trgt}?(y)P \; | \; x_{src}!\langle {Q} \rangle \red P\{\quotep{Q}/y}\} }
  \and \\
  \inferrule* [lab=Par] {{P} \red {P}'} {{{P} | {Q}} \red {{P}' | {Q}}}
  \and
  \inferrule* [lab=Equiv]{{{P} \scong {P}'} \andalso {{P}' \red {Q}'} \andalso {{Q}' \scong {Q}}}{{P} \red {Q}}
\end{mathpar}

\begin{eqnarray*}
  match_{\equiv} (\quotep{P},\quotep{Q}) & := & P \equiv Q \\
  match_{\dagger}(\quotep{P},\quotep{Q}) & := & \forall R. P|Q \red^{*} R => R \red^{*} 0 \\
  match_{K}(\quotep{P},\quotep{Q}) & := & K \mbox{ for some context } K
\end{eqnarray*}

$u?(x)P | u!\langle Q \rangle \red P\{\quotep{Q}/x\}$

%We write $\wred$ for $\red^*$, and $P\red$ if $\exists Q $ such that $ P \red Q$.
We write $P\red$ if $\exists Q $ such that $ P \red Q$ and $P\not\red$, otherwise.

\section{Replication}

As mentioned before, it is known that replication (and hence
recursion) can be implemented in a higher-order process algebra
\cite{SangiorgiWalker}. As our first example of calculation with the
machinery thus far presented we give the construction explicitly in
the {\rhoc}.

\begin{eqnarray}
	D_{x} & := & \prefix{x}{y}{(\binpar{\outputp{x}{y}}{@{y}})} \nonumber\\
	\bangp_{x}{P} & := & \binpar{{x}!\langle{\binpar{D_{x}}{P}}\rangle}{D_{x}} \nonumber
\end{eqnarray}

\begin{eqnarray}
	\bangp_{x}{P} & & \nonumber\\
	=
	& {x}!\langle{(\prefix{x}{y}{(\outputp{x}{y} | @{y})) | P}}\rangle 
	      | \prefix{x}{y}{(\outputp{x}{y} | @{y})} & \nonumber\\
	\red
	& (\outputp{x}{y} | @{y})\substn{\quotep{(\prefix{x}{y}{(@{y} | \outputp{x}{y})) | P}}}{y} & \nonumber\\
	=
	& \outputp{x}{\quotep{(\prefix{x}{y}{(\outputp{x}{y} | @{y})) | P}}}
	  | {(\prefix{x}{y}{(\outputp{x}{y} | @{y})) | P}} & \nonumber\\
	\red
	& \ldots & \nonumber\\
	\red^*
	& P | P | \ldots & \nonumber
\end{eqnarray}

Of course, this encoding, as an implementation, runs away, unfolding
$\bangp{P}$ eagerly. A lazier and more implementable replication
operator, restricted to input-guarded processes, may be obtained as follows.

\begin{eqnarray}
\bangp{\prefix{u}{v}{P}} 
	:= 
	\binpar{\lift{x}{\prefix{u}{v}{(\binpar{D(x)}{P})}}}{D(x)} \nonumber
\end{eqnarray}

\begin{remark}
  Note that the lazier definition still does not deal with summation
  or mixed summation (i.e. sums over input and output). The reader is
  invited to construct definitions of replication that deal with these
  features. 

  Further, the definitions are parameterized in a name, $x$. Can you,
  gentle reader, make a definition that eliminates this parameter and
  guarantees no accidental interaction between the replication
  machinery and the process being replicated -- i.e. no accidental
  sharing of names used by the process to get its work done and the
  name(s) used by the replication to effect copying. This latter
  revision of the definition of replication is crucial to obtaining
  the expected identity $!!P \sim !P$.
\end{remark}

\begin{remark}\label{rem:paradoxical_combinator}
  The reader familiar with the lambda calculus will have noticed the
  similarity between $D$ and the paradoxical combinator.

  [Ed. note: the existence of this seems to suggest we have to be more
  restrictive on the set of processes and names we admit if we are to
  support no-cloning.]
\end{remark}

\subsubsection{Bisimulation}

The computational dynamics gives rise to another kind of equivalence,
the equivalence of computational behavior. As previously mentioned
this is typically captured \emph{via} some form of bisimulation.

% The notion we use in this paper is weak barbed bisimulation
% \cite{milner91polyadicpi}.

The notion we use in this paper is derived from weak barbed
bisimulation \cite{milner91polyadicpi}. 

\begin{definition}
An \emph{observation relation}, $\downarrow_{\mathcal N}$, over a set
of names, $\mathcal N$, is the smallest relation satisfying the rules
below.

\infrule[Out-barb]{y \in {\mathcal N}, \; x \nameeq y}
		  {\outputp{x}{v} \downarrow_{\mathcal N} x}
\infrule[Par-barb]{\mbox{$P\downarrow_{\mathcal N} x$ or $Q\downarrow_{\mathcal N} x$}}
		  {\binpar{P}{Q} \downarrow_{\mathcal N} x}

We write $P \Downarrow_{\mathcal N} x$ if there is $Q$ such that 
$P \wred Q$ and $Q \downarrow_{\mathcal N} x$.
\end{definition}

\begin{definition}
%\label{def.bbisim}
An  ${\mathcal N}$-\emph{barbed bisimulation} over a set of names, ${\mathcal N}$, is a symmetric binary relation 
${\mathcal S}_{\mathcal N}$ between agents such that $P\rel{S}_{\mathcal N}Q$ implies:
\begin{enumerate}
\item If $P \red P'$ then $Q \wred Q'$ and $P'\rel{S}_{\mathcal N} Q'$.
\item If $P\downarrow_{\mathcal N} x$, then $Q\Downarrow_{\mathcal N} x$.
\end{enumerate}
$P$ is ${\mathcal N}$-barbed bisimilar to $Q$, written
$P \wbbisim_{\mathcal N} Q$, if $P \rel{S}_{\mathcal N} Q$ for some ${\mathcal N}$-barbed bisimulation ${\mathcal S}_{\mathcal N}$.
\end{definition}

$\mathcal{R} \subseteq \pi \times \pi$

$P \mathcal{R} Q => \forall P'. P \red P' \Rightarrow \exists Q'. Q \red Q', P' \mathcal{R} Q'$

$P \vdash x \Rightarrow Q \vdash x$

\begin{mathpar}
  \inferrule*[lab=Out-barb]{x \nameeq y}{{y}!\langle{Q}\rangle \vdash x}
  \and
  \inferrule*[lab=Par-barb]{\mbox{$P\vdash x$ or $Q\vdash x$}}{\binpar{P}{Q} \vdash x}
\end{mathpar}

\subsubsection{Contexts}

One of the principle advantages of computational calculi like the
$\pi$-calculus is a well-defined notion of context,
contextual-equivalence and a correlation between
contextual-equivalence and notions of bisimulation. The notion of
context allows the decomposition of a process into (sub-)process and
its syntactic environment, its context. Thus, a context may be
thought of as a process with a ``hole'' (written $\Box$) in it. The
application of a context $M$ to a process $P$, written $M[P]$, is
tantamount to filling the hole in $M$ with $P$. In this paper we do
not need the full weight of this theory, but do make use of the notion
of context in the proof the main theorem. 

\begin{mathpar}
  \inferrule* [lab=summation] {} {{M_{M},M_{N}} \bc \Box \;|\; x.M_{A} \;|\; M_{M}+M_{N}}
  \and
  \inferrule* [lab=agent] {} {{M_{A}} \bc (\vec{x})M_{P} \;| \; \clift{P_0,\ldots,M_{P},\ldots,P_N}}
  \and \\
  \inferrule* [lab=process] {} {{M_{P}} \bc M_{N} \;| \;P|M_{P} }
\end{mathpar} 

\begin{mathpar}
  \inferrule* [lab=sychronization] {} {M_{N} \bc \Box \;|\; x?M_{F} \;|\; x!M_{C}}
  \and
  \inferrule* [lab=abstraction] {} {{M_{F}} \bc (x)M_{P} }
  \and
  \inferrule* [lab=concretion] {} {{M_{C}} \bc \langle M_{P} \rangle }
  \and \\
  \inferrule* [lab=process] {} {{M_{P}} \bc M_{N} \;| \;P|M_{P} }
\end{mathpar}

\begin{definition}[contextual application] Given a context $M$, and
  process $P$, we define the \emph{contextual application}, $M[P] :=
  M\{P/\Box\}$. That is, the contextual application of M to P is the
  substitution of $P$ for $\Box$ in $M$.
\end{definition}

$\meaningof{-} : L \to \mathcal{P}(\pi)$

\begin{mathpar}
  \inferrule* [lab=collection] {} {\meaningof{true} = \pi, \and \meaningof{~E} = \pi \setminus \meaningof{E}, \and \meaningof{E_{1} \& E_{2}} = \meaningof{E_{1}} \cap \meaningof{E_{2}}}
\end{mathpar}

\begin{mathpar}
  \inferrule* [lab=structure] {} {\meaningof{0} = \{ P \in \pi | P \equiv 0 \}, \and \\ \meaningof{E_1 | E_2} = \{ P \in \pi | P \equiv P_{1} | P_{2}, P_{1} \in \meaningof{E_{1}}, P_{2} \in \meaningof{E_2}\} }
\end{mathpar}

\begin{mathpar}
 \inferrule* [lab=behavior] {} {\meaningof{\langle a?b \rangle E} = \{ P \in \pi | P \equiv Q | u?(y)P', \\ \and \\\\ \and \\ \;\;\; u \in \meaningof{a}, \forall z.P'\{z/y\} \in \meaningof{E\{z/b\}}\}, \and \\ \meaningof{a!E} = \{ P \in \pi | P \equiv Q | x!\langle P' \rangle, x \in \meaningof{a} P' \in \meaningof{E}\} }
\end{mathpar}

\begin{mathpar}
 \inferrule* [lab=nominal] {} {\meaningof{\quotep{E}} = \{ \quotep{P} \in \quotep{\pi} | P \in \meaningof{E} \}, \and \meaningof{\quotep{P}} = \{ \quotep{Q} \in \quotep{\pi} | P \equiv Q \} \and \\ \meaningof{@\quotep{E}} = \{ P \in \pi | P \equiv @x, x \in \meaningof{E} \}}
\end{mathpar}

\begin{eqnarray*}
  \\
  \meaningof{-} : TS \to ST
\end{eqnarray*}

\begin{eqnarray*}
  \\
  L : TS \to ST
\end{eqnarray*}

\begin{eqnarray*}
  \\
  P \models E \iff P \in \meaningof{E}
\end{eqnarray*}

\begin{eqnarray*}
  P \approx_{L} Q \iff \forall E \in L. P \models E \iff Q \models E
\end{eqnarray*}

\begin{eqnarray*}
  P \approx_{K} Q
\end{eqnarray*}

\begin{eqnarray*}
  P \approx Q
\end{eqnarray*}

$\approx_{K} = \approx = \approx_{L}$

\subsubsection{Contextual duality}

Note that contexts extend the quotation operation to a family of
operations from processes to names. Given a context, $M$, we can
define a \emph{nominal context}, $\quotep{M}$ by $\quotep{M}[P] :=
\quotep{M[P]}$. To foreshadow what is to come we observe that these
operations enjoy a duality with processes very much like the duality
between vectors and maps from vectors to scalars.

Further, because the calculus is essentially higher-order, we have a
correspondence between contexts and processes. More specifically,
given a name $x$ and a context $M$ we can construct $M^{*}_{x}$ such
that 

\begin{mathpar}
  M^{*}_{x} | \lift{x}{P} \red M[P]
\end{mathpar}

namely,

\begin{mathpar}
  M^{*}_{x} := x?(u).M[\dropn{u}]
\end{mathpar}

The dependence of $M^{*}_{x}$ on a name makes it an abstraction, 

\begin{mathpar}
  M^{*} := (x)x?(u).M[\dropn{u}]
\end{mathpar}

\subsection{Additional notation}

It will sometimes be convenient to denote the process a name
quotes. We already have the notation $x = \quotep{P}$, but it will be
convenient to introduce an alternate notation, $\procn{x}$, when we
want to emphasize the connection to the use of the name. Note that, by
virtue of name equivalence, $\quotep{\procn{x}} \nameeq x$; so, the
notation is consistent with previous definitions.

Further, because names have structure it is possible to effect
substitutions on the basis of that structure. This means we need to
upgrade our notation for substitutions, which we accomplish by
adapting comprehension notation. Thus,

\begin{mathpar}
  P\{ y / x : x \in S \}
\end{mathpar}

is interpreted to mean the process derived from P by replacing (in a
capture-avoiding manner) each occurrence of $x$ in $S$ by $y$. For example,

\begin{mathpar}
  P\{ \quotep{\procn{x}|\procn{x}} / x : x \in \freenames{P} \}
\end{mathpar}

will replace each (occurrence) of a free name $x$ in $P$ by
$\quotep{\procn{x}|\procn{x}}$.

Also, we will avail ourselves of the notation $x^{L}$ and $x^{R}$ to
denote injections of a name into disjoint copies of the name
space. There are numerous ways to accomplish this. One example can be
found in \cite{MeredithR05}. This notation overloads to vectors of
names: $\vec{x}^{\pi} := (x_{i}^{\pi} \; : \; 0 \leq i < |\vec{x}| )$ where $\pi \in \{L,R\}$.

We also use $P^{\Box} := P|\Box$.

In \cite{MeredithR05} an interpretation of the new operator is
given. It turns out that there are several possible interpretations
all enjoying the requisite algebraic properties of the operator (see
\cite{milner91polyadicpi}). We will therefore make liberal use of
$(\nu\; \vec{x})P$.

% subsection the_syntax_and_semantics_of_the_notation_system (end)   

\section{Interpretation of QM}
\subsection{Supporting definitions}
\subsubsection{Multiplication}
\begin{mathpar}
  \quotep{Q} \cdot \quotep{R} := \quotep{Q|R}
  \and \\
  \quotep{Q} \cdot P := P\{ \quotep{Q|R} / \quotep{R} : \quotep{R} \in \freenames{P} \}
\end{mathpar}

\paragraph{Discussion}
The first line needs little explanation. The second line says that
each free name of the process is replaced with the multiplication of
that name by the scalar. Multiplication of a scalar (name) by a state
(process) results in a process all the names of which have been `moved
over' by parallel composition with the process the scalar
quotes. There is a subtlety that the bound names have to be
manipulated so that multiplied names aren't accidentally
captured. There are many ways to achieve this.

\begin{remark}\label{rem:multiplication_identities}
  The reader is invited to verify that for all $x,y,z \in \QProc$ and $P \in \Proc$
  \begin{mathpar}
    x \cdot \quotep{0} \equiv x 
    \and
    x \cdot y \equiv y \cdot x
    \and
    x \cdot (y \cdot z) \equiv (x \cdot y) \cdot z
    \and \\
    \quotep{0} \cdot P \equiv P
    \and \\
    x \cdot (y \cdot P) \equiv (x \cdot y) \cdot P
    \and \\
    x \cdot (P|Q) \equiv (x \cdot P) | (x \cdot Q)
    \and \\    
  \end{mathpar}
\end{remark}

\subsubsection{Tensor product}

We define a tensor product on processes by structural induction.

\paragraph{Tensor of sums} First note that all summations, including
$\pzero$ and sequence, can be written $\Sigma_{i} x_{i}.A_{i} +
\Sigma_{j} x_{j}.C_{j}$, where we have grouped input-guarded processes
together and output-guarded processes together.

Thus, we can define the tensor product of two summations, $N_{1}\otimes N_{2}$, where

\begin{mathpar}
  N_{1} := \Sigma_{i} x_{i}.A_{i} + \Sigma_{j} x_{j}.C_{j}
  \and
  N_{2} := \Sigma_{i'} y_{i'}.B_{i'} + \Sigma_{j'} y_{j'}.D_{j'} 
\end{mathpar}

as follows.

\begin{mathpar}
  \Sigma_{i} x_{i}.A_{i} + \Sigma_{j} x_{j}.C_{j} \otimes \Sigma_{i'}
  y_{i'}.B_{i'} + \Sigma_{j'} y_{j'}.D_{j'} 
  \and \\
  := \; \Sigma_{i} \Sigma_{i'} \quotep{\stackrel{\vee}{x_{i}}| \stackrel{\vee}{y_{i'}}}.(A_{i}\otimes B_{i'}) \; | \; \Sigma_{i'} \Sigma_{i} \quotep{\stackrel{\vee}{y_{i'}}|\stackrel{\vee}{x_{i}}}.(B_{i'}\otimes A_{i})
  \and
  \;\; | \;\; \Sigma_{j} \Sigma_{j'} \quotep{\stackrel{\vee}{x_{j}}|\stackrel{\vee}{y_{j'}}}.(A_{j}\otimes B_{j'}) \; | \; \Sigma_{j'} \Sigma_{j} \quotep{\stackrel{\vee}{y_{j'}}|\stackrel{\vee}{x_{j}}}.(B_{j'}\otimes A_{j})
\end{mathpar}

\begin{remark}
  Do we need to $x^{L}$ and $y^{R}$ for this construction as well?
\end{remark}

\paragraph{Tensor of parallel compositions} Next, we distribute tensor
over par.

\begin{mathpar}
  P_{1}|P_{2} \otimes Q_{1}|Q_{2} := (P_{1} \otimes Q_{1}) | (P_{1}
  \otimes Q_{2}) | (P_{2} \otimes Q_{1}) | (P_{2} \otimes Q_{2})
\end{mathpar}

\paragraph{Tensor with dropped names} We treat tensor of a
process with a dropped name as parallel composition.

\begin{mathpar}
  P \otimes \dropn{x} := P | \dropn{x}
\end{mathpar}

\paragraph{Tensor of agents}

Finally, we need to define tensor on agents. Note that the definition
of tensor on normal products only tensors inputs with inputs and
outputs with outputs. Thus, we only have to define the operation on
``homogeneous'' pairings.

\begin{mathpar}
  (\vec{x})P \otimes (\vec{y})Q
  \and \\
  := (x_{0}^{L}|y_{0}^{R},\ldots,x_{0}^{L}|y_{n}^{R},\ldots,x_{m}^{L}|y_{0}^{R},\ldots,x_{m}^{L}|y_{n}^R)(P\{ \vec{x}^{L}/\vec{x}\} \otimes Q \{ \vec{y}^{R}/\vec{y}\})
  \and \\
  \clift{\vec{P}} \otimes \clift{\vec{Q}}
  \and \\
  := \clift{P_{0}\otimes Q_{0},\ldots,P_{0}\otimes Q_{n},\ldots,P_{m}\otimes Q_{0},\ldots,P_{m}\otimes Q_{n}}
\end{mathpar}

\begin{remark}
  Observe that arities of tensored abstractions matches arities of
  tensored concretions if the original arities matched. Note also that
  the length of the arities corresponds to the increase in dimension
  we see in ordinary vector space tensor product.
\end{remark}

\begin{remark}
  Operationally, this definition distributes the tensor down to
  components ``linked'' by summation. Tensor over summation is
  intriguing in that it mixes names. Moreover, as a consequence of the
  way it mixes names we have the identities for all $x \in \QProc$ and
  $P,Q \in \Proc$

  \begin{mathpar}
    (x \cdot P) \otimes Q \equiv x \cdot (P \otimes Q) \equiv P \otimes (x \cdot Q)
    \and
    P \otimes \pzero \equiv P
  \end{mathpar}

  that the reader is invited to verify.
\end{remark}

\subsubsection{Annihilation}
\begin{mathpar}
  P^{\perp} := \{ Q | \forall R. P|Q \red^{*} R \Rightarrow R \red^{*} \pzero \}
  \and \\
  P^{\underline{\perp}} := \Sigma_{Q \in P^{\perp}} \quotep{Q}?(y).(\dropn{y}|Q) | \Sigma_{Q \in P^{\perp}} \quotep{Q}\clift{\Box}
\end{mathpar}

\paragraph{Discussion} The reader will note that $P^{\perp}$ is a
\emph{set} of processes, while $P^{\underline{\perp}}$ is a
\emph{context}. We call the set $P^{\perp}$ the \emph{annihilators} of
$P$. The parallel composition of a process in the annihilators of $P$
with $P$ will result in a process, the state space of which has all
paths eventually leading to $\pzero$. Execution may endure loops; but
under reasonable conditions of fairness (naturally guaranteed under
most notions of bisimulation) such a composite process cannot get
stuck in such a loop and will, eventually pop out and terminate.

The context $P^{\underline{\perp}}$ is ready and willing to ``take the
$P$ out of'' the process to which it is applied. It will effectively
transmit the code of the process to which it is applied to one of the
annihilators and run the process against it.

\subsubsection{Evaluation}
We fix $M$ a domain of fully abstract interpretation with an equality
coincident with bisimulation. We take $\meaningof{\cdot} : \Proc \to
M$ to be the map interpreting processes and $\nmeaningof{\cdot} : \M
\to Proc$ to be the map running the other way. Then we define

\begin{mathpar}
  \int P := \nmeaningof{\meaningof{P}}
\end{mathpar}

\paragraph{Discussion}
There are many fully abstract interpretations of Milner's
$\pi$-calculus. Any of them can be used as a basis for interpreting
the reflective calculus here. Equipped with such a domain it is
largely a matter of grinding through to check that the Yoneda
construction for the normalization-by-evaluation program can be
extended to this setting.

\begin{remark}
  The reader is invited to verify that $\int (P^{\underline{\perp}}[P]) = 0$.
\end{remark}

\subsection{Quantum mechanics}

Table \ref{tbl:core_qm_op_defns} gives the core operational definitions

\begin{table}[htp]\label{tbl:core_qm_op_defns}
  \center{
    \fbox{
      \begin{tabular}{c|c}
        quantum mechanics & process calculus \\
        \hline
        scalar & $x := \quotep{P}$ \\
        state vector & $\state{P} := P$ \\
        dual & $\state{P}^{*} := \event{P^{\underline{\perp}}} := \quotep{P^{\underline{\perp}}}[-]$ \\
        matrix & $ \Sigma_{\alpha} \state{P_{\alpha}}x_{\alpha}\event{Q_{\alpha}}$ \\
        vector addition & $\state{P} + \state{Q} := \state{P | Q}$ \\
        tensor product & $\state{P} \otimes \state{Q} := \state{P \otimes Q}$ \\
        inner product & $\innerprod{P}{Q} := \quotep{\int P^{\underline{\perp}}[Q]}$ \\
      \end{tabular}
    }
  }
  \caption{QM - operational definitions}
\end{table}

where

\begin{mathpar}
  \prmatrix{P}{Q} := \fprmatrix{P}{\quotep{\pzero}}{Q}
  \and
  \fprmatrix{P}{x}{Q} := (\state{P},x,\event{Q})
  \and
  (\fprmatrix{P}{x}{Q})(\state{R}) := x \cdot \innerprod{Q}{R} \cdot \state{P}
  \and
  (\fprmatrix{P}{x}{Q})(\event{R}) := x \cdot \innerprod{R}{P} \cdot \event{Q}
\end{mathpar}

\paragraph{Discussion}
As promised: vectors (aka states) are represented as processes; duals
as contextual duals; inner product definition should be compared with
standard inner product definition for ....

\begin{remark}
  Assuming $\int (P^{\underline{\perp}}[P]) = 0$, the reader is
  invited to verify that $(\fprmatrix{P}{x}{P})(\state{P}) = x \cdot \state{P}$.
\end{remark}

\begin{remark}
  The reader is invited to verify that $\innerprod{P}{Q}$ could
  equally well have been written $\quotep{\int \stackrel{\vee}{x}}$
  where $x = \event{P^{\underline{\perp}}}(Q)$.

  One of the motivations for this remark is that there is another way
  to factor these operations. We could package up evaluation in the dual:

  \begin{mathpar}
    \state{P}^{*} := \event{\int P^{\underline{\perp}}} := \quotep{\int P^{\underline{\perp}}}[-]
  \end{mathpar}

  and then have inner product defined by
  
  \begin{mathpar}
    \innerprod{P}{Q} := \event{P}(Q)
  \end{mathpar}

  Hopefully, experience with the calculations will provide guidance on
  the best factoring.
\end{remark}

\begin{remark}
  Assuming $\int (P^{\underline{\perp}}[P]) = 0$, the reader is
  invited to verify that $\forall P,Q. (\prmatrix{0}{Q})(\state{0}) =
  \state{0}$ and dually $(\prmatrix{P}{0})(\event{0}) = \event{0}$.
\end{remark}

\begin{remark}
  i'm a little worried that i don't (yet) have proper support for
  complex conjugacy. But, the observation above may give us a
  clue. According to Abramsky, it must be the case that the scalars
  are iso to the homset of the identity for the tensor -- which the
  observation above characterizes. 

  For now, we will simply bookmark the notion with $\overline{x}$.
\end{remark}

\subsubsection{Adjointness}

We need to give a definition of $(\cdot)^{\dagger}$ for matrices. The
obvious candidate definition is
\begin{mathpar}
(\Sigma_{\alpha}\fprmatrix{P_{\alpha}}{x_{\alpha}}{Q_{\alpha}})^{\dagger}
= \Sigma_{\alpha}\fprmatrix{(Q_{\alpha}^{\underline{\perp}})^{*}}{\overline{x}_{\alpha}}{P_{\alpha}^{\underline{\perp}}} 
\end{mathpar}

But, $(Q_{\alpha}^{\underline{\perp}})^{*}$ requires a name along
which to communicate the process to achieve the context application.

\subsubsection{Basis for a basis}
If processes label states and ``addition'' of states (a.k.a. vector
addition) is interpreted as parallel composition, what corresponds to
notions of linear independence and basis? Here, we recall that Yoshida
has developed a set of \emph{combinators} for an asynchronous verison
of Milner's $\pi$-calculus. These are a finite set of processes such
any process can be expressed as parallel composition of these
combinators together with liberal uses of the new operator and
replication. We can simply give a translation of these into the
present calculus and have reasonable expectation that the property
carries over. That is, that the resultant set allows to express all
processes via parallel composition. Note, however, that there is no
new operator or replication in this calculus. As a result, we expect
that the corresponding set is actually infinite. That is, we expect
that the space is actually infinite dimensional.

\begin{remark}
  The attentive reader may be a bit concerned. Certainly, the
  collection $S$, $K$ and $I$ is a finite set of
  combinators. Shouldn't we expect to see a finite set of combinators
  for an effectively equivalent system? i am very sympathetic to this
  critique and feel it warrants full attention. On the other hand, i
  also have in mind the following analogy. The natural numbers, as a
  monoid under addition, has exactly $1$ generator, while the natural
  numbers, as a monoid under multiplication, has countably many
  generators (the primes). We observe that the application of the
  lambda calculus is much less resource sensitive than the parallel
  composition of the $\pi$-calculus. Could it be the case that we have
  an analogy of the form
  
  \begin{mathpar}
    m + n : MN :: m*n : M|N
  \end{mathpar}

  giving a similar blow up in the set of ``primes''?  This is such a
  wonderful thought that, even if it's not true, i think it's worth
  writing down.
\end{remark}
 

\documentclass[12pt]{llncs}
%\documentclass{jktr}

\usepackage[pdftex]{hyperref}                   
\usepackage {listings}
\usepackage {mathpartir}
\usepackage{bcprules}
%\usepackage{listings}
                       
\usepackage{graphicx} 
%\usepackage[margins=2.5cm,nohead,nofoot]{geometry}
%\usepackage{geometry}
\usepackage{amsfonts}
\usepackage{amstext}
\usepackage{latexsym}
\usepackage{amssymb}
\usepackage{color}


%\include{myPreamble}
\include{qm2pi.local} 

%\ifpdf
%\usepackage[pdftex]{graphicx}
%\else
%\usepackage{graphicx}
%\fi

 % \ifpdf
%  \usepackage{pdfsync}
%  \if


%\title{Brief Article}
%\author{David F. Snyder}
%\author{L.G. Meredith}

%\address{Dept. of Math., Texas State University--San Marcos, San Marcos, TX 78666}
       
\pagestyle{empty}


\begin{document}

\lstset{language=[Objective]Caml,frame=shadowbox}

\input{qm2pi.front}

% section front matter (end)

\input{qm2pi.intro} 
 
% section introduction (end)

% \input{qm2pi.knotations} 

% section notation (end)

\input{qm2pi.process.calculi} 

% section concurrent_process_calculi_and_spatial_logics_ (end)
    
%\input{qm2pi.knots2pi} 

%\input{qm2pi.trefoil} 

%\input{qm2pi.mainthm} 

% subsection basic_interpretation (end)

%\input{qm2pi.rho.presentation} 
\subsection{The syntax and semantics of the notation system}\label{sub:the_syntax_and_semantics_of_the_notation_system} % (fold)

We now summarize a technical presentation of the calculus that
embodies our theory of dynamics. The typical presentation of such a
calculus follows the style of giving generators and relations on
them. The grammar, below, describing term constructors, freely
generates the set of processes, $\Proc$. This set is then quotiented
by a relation known as structural congruence and it is over this set
that the notion of dynamics is expressed. This presentation is
essentially that of \cite{MeredithR05} with the addition of
polyadicity and summation. For readability we have relegated some of
the technical subtleties to an appendix.

\subsubsection{Process grammar}\label{subsub:process_grammar}

\begin{mathpar}
  \inferrule* [lab=synchronization] {} {{M} \bc \pzero \;|\; x?F \;|\; x!C }
  \and
  \inferrule* [lab=abstraction] {} {{F} \bc (x)P}
  \and
  \inferrule* [lab=concretion] {} {{C} \bc \langle Q \rangle}
  \and
  \inferrule* [lab=process] {} {{P,Q} \bc M \;| \;P|Q \;|\; @{x}}
  \and
  \inferrule* [lab=name] {} {{x} \bc \quotep{P}}
\end{mathpar} 

Note that $\vec{x}$ (resp. $\vec{P}$) denotes a vector of names
(resp. processes) of length $|\vec{x}|$ (resp. $|\vec{P}|$). We adopt
the following useful abbreviations.

\begin{mathpar}
   x?(\vec{y}).P := x.(\vec{y})P \and  x\clift{\vec{P}} := x.\clift{\vec{P}}
   \and x!(y) := \lift{x}{\dropn{y}}
   \and \Pi_{i=0}^{n-1}P_i := P_0 | \ldots | P_{n-1}
\end{mathpar}

\subsubsection{Structural congruence}

\paragraph{Free and bound names and alpha-equivalence.} At the
core of structural equivalence is alpha-equivalence which identifies
process that are the same up to a change of variable. Formally, we
recognize the distinction between free and bound names. The free names
of a process, $\freenames{P}$, may be calculated recursively as
follows:

\begin{mathpar}
\freenames{\pzero} := \emptyset
  \and \\
  \freenames{x?(y).P} := \{ x \} \cup (\freenames{P} \setminus \{ y \})
  \and 
  \freenames{x!\langle P \rangle} := \{ x \} \cup \{ P \} 
  \and \\
  \freenames{P|Q} := \freenames{P} \cup \freenames{Q}
  \and \\
  \freenames{@{x}} := \{ x \}
\end{mathpar}

$\pi$
$\quotep{\pi}$

$\freenames{-} : \pi \to \mathcal{P}(\quotep{\pi})$

\begin{eqnarray*}
  \freenames{\pzero} & := & \emptyset \\
  \freenames{x?(y).P} & := & \{ x \} \cup (\freenames{P} \setminus \{ y \}) \\
  \freenames{x!\langle P \rangle} & := & \{ x \} \cup \{ P \} \\
  \freenames{P|Q} & := & \freenames{P} \cup \freenames{Q} \\
  \freenames{\dropn{x}} & := & \{ x \}
\end{eqnarray*}

The bound names of a process, $\boundnames{P}$, are those names occurring in $P$
that are not free. For example, in $x?(y).0$, the name $x$ is free, while $y$ is bound.

\begin{mathpar}
  \inferrule* [lab=monoidal-laws] {} { P|Q \equiv Q|P \and P|0 \equiv P \and P|(Q|R) \equiv (P|Q)|R }
\end{mathpar}

\begin{mathpar}
  \inferrule* [lab=alpha-equivalence] {} { (x)P \equiv (y)P\{y/x\} \and y \not\in \freenames{P} }
\end{mathpar}

\begin{definition}
Then two processes, $P,Q$, are alpha-equivalent if $P = Q\{\vec{y}/\vec{x}\}$ for
some $\vec{x} \in \boundnames{Q},\vec{y} \in \boundnames{P}$, where $Q\{\vec{y}/\vec{x}\}$
denotes the capture-avoiding substitution of $\vec{y}$ for $\vec{x}$ in $Q$.
\end{definition}

\begin{definition}
  The {\em structural congruence} \cite{SangiorgiWalker} , $\equiv$,
  between processes is the least congruence containing
  alpha-equivalence, satisfying the abelian monoid laws
  (associativity, commutativity and $\pzero$ as identity) for parallel
  composition $|$ and for summation $+$.
\end{definition}

\subsection{Name equivalence}

We take name equivalence, written $\nameeq$, to be the smallest
equivalence relation generated by the following rules.

\begin{mathpar}
\inferrule*[lab=Quote-drop]
{ }
{ \quotep{@{x}} \nameeq x }

\inferrule*[lab=Struct-equiv]
{ P \scong Q }
{ \quotep{P} \nameeq \quotep{Q} }
\end{mathpar}

The astute reader will have noticed that the mutual recursion of names
and processes imposes a mutual recursion on alpha-equivalence and
structural equivalence via name-equivalence. Fortunately, all of this
works out pleasantly and we may calculate in the natural way, free of
concern. The reader interested in the details is referred to the
appendix \ref{appendix:rho_details}.

\subsection{Substitution}

We use $\Proc$ for the set of processes, $\QProc$ for the set of
names, and $\id{\{}\vec{y} / \vec{x} \id{\}}$ to denote partial maps,
$s : \QProc \rightarrow \QProc$. A map, $s$ lifts, uniquely, to a map
on process terms, $\widehat{s} : \Proc \rightarrow \Proc$ by the
following equations.

\begin{mathpar}
  (0) \psubstp{Q}{P} := 0 \\
  (R \juxtap S) \psubstp{Q}{P}
  :=    
  (R)\psubstp{Q}{P} \juxtap (S) \psubstp{Q}{P} \\
  (x?(y).R) \psubstp{Q}{P}    
  :=    
  (x)\substp{Q}{P} (z)\concat( (R \psubstn{z}{y}) \psubstp{Q}{P} ) \\
  (\lift{x}{R}) \psubstp{Q}{P}  
  :=
  \lift{(x)\substp{Q}{P}}{ R \psubstp{Q}{P} } \\
%   (\dropn{x})  \psubstp{Q}{P}       
%   := 
%   \left\{ 
%     \begin{array}{ccc} 
%       \dropn{\quotep{Q}} & & x \nameeq \quotep{P} \\
%       \dropn{x} & & otherwise \\
%     \end{array}
%   \right. 
  (\dropn{x})  \psubstp{Q}{P}       
  := 
  \left\{ 
    \begin{array}{ccc} 
      Q & & x \nameeq \quotep{P} \\
      \dropn{x} & & otherwise \\
    \end{array}
  \right.
\end{mathpar}
 

where

\begin{eqnarray}
  (x)\id{\{} \lpquote Q \rpquote / \lpquote P \rpquote \id{\}}            = 
  \left\{ 
    \begin{array}{ccc}
      \lpquote Q \rpquote & & x \nameeq \lpquote P \rpquote \\
      x & & otherwise \\
    \end{array}
  \right. \nonumber
\end{eqnarray}

and $z$ is chosen distinct from $\quotep{P}$, $\quotep{Q}$, the free
names in $Q$, and all the names in $R$. Our $\alpha$-equivalence will
be built in the standard way from this substitution.

\begin{remark}\label{rem:no_self_referential_names}
  One consequence of these definitions is that $\forall P. \quotep{P}
  \not\in \freenames{P}$.
\end{remark}

\subsection{ Dynamic quote: an example }

Anticipating something of what's to come, consider applying the
substitution, $\widehat{\id{\{}u / z \id{\}}}$, to the following pair
of processes, $\lift{w}{y!(z)}$ and $w[ \lpquote y!(z) \rpquote ]$.

\begin{eqnarray}
	\lift{w}{y!(z)}\widehat{\id{\{}u / z \id{\}}}
		& = &
		\lift{w}{y!(u)} \nonumber\\
	w[ \lpquote y!(z) \rpquote ] \widehat{ \id{\{}u / z \id{\}} }
		& = &
		w[ \lpquote y!(z) \rpquote ] \nonumber
\end{eqnarray}

Because the body of the process between quotes is impervious to
substitution, we get radically different answers. In fact, by
examining the first process in an input context,
e.g. $x?(z).\lift{w}{y!(z)}$, we see that the process under the lift
operator may be shaped by prefixed inputs binding a name inside it. In
this sense, the lift operator will be seen as a way to dynamically
construct processes before reifying them as names.

Finally equipped with these standard features we can present the
dynamics of the calculus.

\subsubsection{Operational semantics} 

Finally, we introduce the computational dynamics. What marks these
algebras as distinct from other more traditionally studied algebraic
structures, e.g. vector spaces or polynomial rings, is the manner in
which dynamics is captured. In traditional structures, dynamics is typically
expressed through morphisms between such structures, as in linear maps
between vector spaces or morphisms between rings. In algebras
associated with the semantics of computation, the dynamics is
expressed as part of the algebraic structure itself, through a
reduction reduction relation typically denoted by $\red$. Below, we
give a recursive presentation of this relation for the calculus used
in the encoding.

$\red \subseteq \pi \times \pi$
$\red : \pi \to \mathcal{P}(\pi)$

\begin{mathpar}
  \inferrule* [lab=Comm] { \textsf{match}( x_{src}, x_{trgt} ) } { x_{trgt}?(y)P \; | \; x_{src}!\langle {Q} \rangle \red P\{\quotep{Q}/y}\} }
  \and \\
  \inferrule* [lab=Par] {{P} \red {P}'} {{{P} | {Q}} \red {{P}' | {Q}}}
  \and
  \inferrule* [lab=Equiv]{{{P} \scong {P}'} \andalso {{P}' \red {Q}'} \andalso {{Q}' \scong {Q}}}{{P} \red {Q}}
\end{mathpar}

\begin{eqnarray*}
  match_{\equiv} (\quotep{P},\quotep{Q}) & := & P \equiv Q \\
  match_{\dagger}(\quotep{P},\quotep{Q}) & := & \forall R. P|Q \red^{*} R => R \red^{*} 0 \\
  match_{K}(\quotep{P},\quotep{Q}) & := & K \mbox{ for some context } K
\end{eqnarray*}

$u?(x)P | u!\langle Q \rangle \red P\{\quotep{Q}/x\}$

%We write $\wred$ for $\red^*$, and $P\red$ if $\exists Q $ such that $ P \red Q$.
We write $P\red$ if $\exists Q $ such that $ P \red Q$ and $P\not\red$, otherwise.

\section{Replication}

As mentioned before, it is known that replication (and hence
recursion) can be implemented in a higher-order process algebra
\cite{SangiorgiWalker}. As our first example of calculation with the
machinery thus far presented we give the construction explicitly in
the {\rhoc}.

\begin{eqnarray}
	D_{x} & := & \prefix{x}{y}{(\binpar{\outputp{x}{y}}{@{y}})} \nonumber\\
	\bangp_{x}{P} & := & \binpar{{x}!\langle{\binpar{D_{x}}{P}}\rangle}{D_{x}} \nonumber
\end{eqnarray}

\begin{eqnarray}
	\bangp_{x}{P} & & \nonumber\\
	=
	& {x}!\langle{(\prefix{x}{y}{(\outputp{x}{y} | @{y})) | P}}\rangle 
	      | \prefix{x}{y}{(\outputp{x}{y} | @{y})} & \nonumber\\
	\red
	& (\outputp{x}{y} | @{y})\substn{\quotep{(\prefix{x}{y}{(@{y} | \outputp{x}{y})) | P}}}{y} & \nonumber\\
	=
	& \outputp{x}{\quotep{(\prefix{x}{y}{(\outputp{x}{y} | @{y})) | P}}}
	  | {(\prefix{x}{y}{(\outputp{x}{y} | @{y})) | P}} & \nonumber\\
	\red
	& \ldots & \nonumber\\
	\red^*
	& P | P | \ldots & \nonumber
\end{eqnarray}

Of course, this encoding, as an implementation, runs away, unfolding
$\bangp{P}$ eagerly. A lazier and more implementable replication
operator, restricted to input-guarded processes, may be obtained as follows.

\begin{eqnarray}
\bangp{\prefix{u}{v}{P}} 
	:= 
	\binpar{\lift{x}{\prefix{u}{v}{(\binpar{D(x)}{P})}}}{D(x)} \nonumber
\end{eqnarray}

\begin{remark}
  Note that the lazier definition still does not deal with summation
  or mixed summation (i.e. sums over input and output). The reader is
  invited to construct definitions of replication that deal with these
  features. 

  Further, the definitions are parameterized in a name, $x$. Can you,
  gentle reader, make a definition that eliminates this parameter and
  guarantees no accidental interaction between the replication
  machinery and the process being replicated -- i.e. no accidental
  sharing of names used by the process to get its work done and the
  name(s) used by the replication to effect copying. This latter
  revision of the definition of replication is crucial to obtaining
  the expected identity $!!P \sim !P$.
\end{remark}

\begin{remark}\label{rem:paradoxical_combinator}
  The reader familiar with the lambda calculus will have noticed the
  similarity between $D$ and the paradoxical combinator.

  [Ed. note: the existence of this seems to suggest we have to be more
  restrictive on the set of processes and names we admit if we are to
  support no-cloning.]
\end{remark}

\subsubsection{Bisimulation}

The computational dynamics gives rise to another kind of equivalence,
the equivalence of computational behavior. As previously mentioned
this is typically captured \emph{via} some form of bisimulation.

% The notion we use in this paper is weak barbed bisimulation
% \cite{milner91polyadicpi}.

The notion we use in this paper is derived from weak barbed
bisimulation \cite{milner91polyadicpi}. 

\begin{definition}
An \emph{observation relation}, $\downarrow_{\mathcal N}$, over a set
of names, $\mathcal N$, is the smallest relation satisfying the rules
below.

\infrule[Out-barb]{y \in {\mathcal N}, \; x \nameeq y}
		  {\outputp{x}{v} \downarrow_{\mathcal N} x}
\infrule[Par-barb]{\mbox{$P\downarrow_{\mathcal N} x$ or $Q\downarrow_{\mathcal N} x$}}
		  {\binpar{P}{Q} \downarrow_{\mathcal N} x}

We write $P \Downarrow_{\mathcal N} x$ if there is $Q$ such that 
$P \wred Q$ and $Q \downarrow_{\mathcal N} x$.
\end{definition}

\begin{definition}
%\label{def.bbisim}
An  ${\mathcal N}$-\emph{barbed bisimulation} over a set of names, ${\mathcal N}$, is a symmetric binary relation 
${\mathcal S}_{\mathcal N}$ between agents such that $P\rel{S}_{\mathcal N}Q$ implies:
\begin{enumerate}
\item If $P \red P'$ then $Q \wred Q'$ and $P'\rel{S}_{\mathcal N} Q'$.
\item If $P\downarrow_{\mathcal N} x$, then $Q\Downarrow_{\mathcal N} x$.
\end{enumerate}
$P$ is ${\mathcal N}$-barbed bisimilar to $Q$, written
$P \wbbisim_{\mathcal N} Q$, if $P \rel{S}_{\mathcal N} Q$ for some ${\mathcal N}$-barbed bisimulation ${\mathcal S}_{\mathcal N}$.
\end{definition}

$\mathcal{R} \subseteq \pi \times \pi$

$P \mathcal{R} Q => \forall P'. P \red P' \Rightarrow \exists Q'. Q \red Q', P' \mathcal{R} Q'$

$P \vdash x \Rightarrow Q \vdash x$

\begin{mathpar}
  \inferrule*[lab=Out-barb]{x \nameeq y}{{y}!\langle{Q}\rangle \vdash x}
  \and
  \inferrule*[lab=Par-barb]{\mbox{$P\vdash x$ or $Q\vdash x$}}{\binpar{P}{Q} \vdash x}
\end{mathpar}

\subsubsection{Contexts}

One of the principle advantages of computational calculi like the
$\pi$-calculus is a well-defined notion of context,
contextual-equivalence and a correlation between
contextual-equivalence and notions of bisimulation. The notion of
context allows the decomposition of a process into (sub-)process and
its syntactic environment, its context. Thus, a context may be
thought of as a process with a ``hole'' (written $\Box$) in it. The
application of a context $M$ to a process $P$, written $M[P]$, is
tantamount to filling the hole in $M$ with $P$. In this paper we do
not need the full weight of this theory, but do make use of the notion
of context in the proof the main theorem. 

\begin{mathpar}
  \inferrule* [lab=summation] {} {{M_{M},M_{N}} \bc \Box \;|\; x.M_{A} \;|\; M_{M}+M_{N}}
  \and
  \inferrule* [lab=agent] {} {{M_{A}} \bc (\vec{x})M_{P} \;| \; \clift{P_0,\ldots,M_{P},\ldots,P_N}}
  \and \\
  \inferrule* [lab=process] {} {{M_{P}} \bc M_{N} \;| \;P|M_{P} }
\end{mathpar} 

\begin{mathpar}
  \inferrule* [lab=sychronization] {} {M_{N} \bc \Box \;|\; x?M_{F} \;|\; x!M_{C}}
  \and
  \inferrule* [lab=abstraction] {} {{M_{F}} \bc (x)M_{P} }
  \and
  \inferrule* [lab=concretion] {} {{M_{C}} \bc \langle M_{P} \rangle }
  \and \\
  \inferrule* [lab=process] {} {{M_{P}} \bc M_{N} \;| \;P|M_{P} }
\end{mathpar}

\begin{definition}[contextual application] Given a context $M$, and
  process $P$, we define the \emph{contextual application}, $M[P] :=
  M\{P/\Box\}$. That is, the contextual application of M to P is the
  substitution of $P$ for $\Box$ in $M$.
\end{definition}

$\meaningof{-} : L \to \mathcal{P}(\pi)$

\begin{mathpar}
  \inferrule* [lab=collection] {} {\meaningof{true} = \pi, \and \meaningof{~E} = \pi \setminus \meaningof{E}, \and \meaningof{E_{1} \& E_{2}} = \meaningof{E_{1}} \cap \meaningof{E_{2}}}
\end{mathpar}

\begin{mathpar}
  \inferrule* [lab=structure] {} {\meaningof{0} = \{ P \in \pi | P \equiv 0 \}, \and \\ \meaningof{E_1 | E_2} = \{ P \in \pi | P \equiv P_{1} | P_{2}, P_{1} \in \meaningof{E_{1}}, P_{2} \in \meaningof{E_2}\} }
\end{mathpar}

\begin{mathpar}
 \inferrule* [lab=behavior] {} {\meaningof{\langle a?b \rangle E} = \{ P \in \pi | P \equiv Q | u?(y)P', \\ \and \\\\ \and \\ \;\;\; u \in \meaningof{a}, \forall z.P'\{z/y\} \in \meaningof{E\{z/b\}}\}, \and \\ \meaningof{a!E} = \{ P \in \pi | P \equiv Q | x!\langle P' \rangle, x \in \meaningof{a} P' \in \meaningof{E}\} }
\end{mathpar}

\begin{mathpar}
 \inferrule* [lab=nominal] {} {\meaningof{\quotep{E}} = \{ \quotep{P} \in \quotep{\pi} | P \in \meaningof{E} \}, \and \meaningof{\quotep{P}} = \{ \quotep{Q} \in \quotep{\pi} | P \equiv Q \} \and \\ \meaningof{@\quotep{E}} = \{ P \in \pi | P \equiv @x, x \in \meaningof{E} \}}
\end{mathpar}

\begin{eqnarray*}
  \\
  \meaningof{-} : TS \to ST
\end{eqnarray*}

\begin{eqnarray*}
  \\
  L : TS \to ST
\end{eqnarray*}

\begin{eqnarray*}
  \\
  P \models E \iff P \in \meaningof{E}
\end{eqnarray*}

\begin{eqnarray*}
  P \approx_{L} Q \iff \forall E \in L. P \models E \iff Q \models E
\end{eqnarray*}

\begin{eqnarray*}
  P \approx_{K} Q
\end{eqnarray*}

\begin{eqnarray*}
  P \approx Q
\end{eqnarray*}

$\approx_{K} = \approx = \approx_{L}$

\subsubsection{Contextual duality}

Note that contexts extend the quotation operation to a family of
operations from processes to names. Given a context, $M$, we can
define a \emph{nominal context}, $\quotep{M}$ by $\quotep{M}[P] :=
\quotep{M[P]}$. To foreshadow what is to come we observe that these
operations enjoy a duality with processes very much like the duality
between vectors and maps from vectors to scalars.

Further, because the calculus is essentially higher-order, we have a
correspondence between contexts and processes. More specifically,
given a name $x$ and a context $M$ we can construct $M^{*}_{x}$ such
that 

\begin{mathpar}
  M^{*}_{x} | \lift{x}{P} \red M[P]
\end{mathpar}

namely,

\begin{mathpar}
  M^{*}_{x} := x?(u).M[\dropn{u}]
\end{mathpar}

The dependence of $M^{*}_{x}$ on a name makes it an abstraction, 

\begin{mathpar}
  M^{*} := (x)x?(u).M[\dropn{u}]
\end{mathpar}

\subsection{Additional notation}

It will sometimes be convenient to denote the process a name
quotes. We already have the notation $x = \quotep{P}$, but it will be
convenient to introduce an alternate notation, $\procn{x}$, when we
want to emphasize the connection to the use of the name. Note that, by
virtue of name equivalence, $\quotep{\procn{x}} \nameeq x$; so, the
notation is consistent with previous definitions.

Further, because names have structure it is possible to effect
substitutions on the basis of that structure. This means we need to
upgrade our notation for substitutions, which we accomplish by
adapting comprehension notation. Thus,

\begin{mathpar}
  P\{ y / x : x \in S \}
\end{mathpar}

is interpreted to mean the process derived from P by replacing (in a
capture-avoiding manner) each occurrence of $x$ in $S$ by $y$. For example,

\begin{mathpar}
  P\{ \quotep{\procn{x}|\procn{x}} / x : x \in \freenames{P} \}
\end{mathpar}

will replace each (occurrence) of a free name $x$ in $P$ by
$\quotep{\procn{x}|\procn{x}}$.

Also, we will avail ourselves of the notation $x^{L}$ and $x^{R}$ to
denote injections of a name into disjoint copies of the name
space. There are numerous ways to accomplish this. One example can be
found in \cite{MeredithR05}. This notation overloads to vectors of
names: $\vec{x}^{\pi} := (x_{i}^{\pi} \; : \; 0 \leq i < |\vec{x}| )$ where $\pi \in \{L,R\}$.

We also use $P^{\Box} := P|\Box$.

In \cite{MeredithR05} an interpretation of the new operator is
given. It turns out that there are several possible interpretations
all enjoying the requisite algebraic properties of the operator (see
\cite{milner91polyadicpi}). We will therefore make liberal use of
$(\nu\; \vec{x})P$.

% subsection the_syntax_and_semantics_of_the_notation_system (end)   

\input{qm2pi.qmops} 

\input{qm2pi.sterngerlach} 

\input{qm2pi.metric} 

% section concurrent_process_calculi (end)

%\input{qm2pi.proofsketch}

% section proof sketch (end)

%\input{qm2pi.slviaknots} 

% section spatial logic via knots (end)

\input{qm2pi.conclusion}

% section conclusion (end)

%\input{qm2pi.dtcodes} 

% section wiring algorithm (end)

\input{qm2pi.ack} 

% section acknowledgments (end)

\newpage


\bibliographystyle{plain}   
\bibliography{../../biblios/main.bib}

\input{qm2pi.rhodetails}

\end{document}

 

\documentclass[12pt]{llncs}
%\documentclass{jktr}

\usepackage[pdftex]{hyperref}                   
\usepackage {listings}
\usepackage {mathpartir}
\usepackage{bcprules}
%\usepackage{listings}
                       
\usepackage{graphicx} 
%\usepackage[margins=2.5cm,nohead,nofoot]{geometry}
%\usepackage{geometry}
\usepackage{amsfonts}
\usepackage{amstext}
\usepackage{latexsym}
\usepackage{amssymb}
\usepackage{color}


%\include{myPreamble}
\include{qm2pi.local} 

%\ifpdf
%\usepackage[pdftex]{graphicx}
%\else
%\usepackage{graphicx}
%\fi

 % \ifpdf
%  \usepackage{pdfsync}
%  \if


%\title{Brief Article}
%\author{David F. Snyder}
%\author{L.G. Meredith}

%\address{Dept. of Math., Texas State University--San Marcos, San Marcos, TX 78666}
       
\pagestyle{empty}


\begin{document}

\lstset{language=[Objective]Caml,frame=shadowbox}

\input{qm2pi.front}

% section front matter (end)

\input{qm2pi.intro} 
 
% section introduction (end)

% \input{qm2pi.knotations} 

% section notation (end)

\input{qm2pi.process.calculi} 

% section concurrent_process_calculi_and_spatial_logics_ (end)
    
%\input{qm2pi.knots2pi} 

%\input{qm2pi.trefoil} 

%\input{qm2pi.mainthm} 

% subsection basic_interpretation (end)

%\input{qm2pi.rho.presentation} 
\subsection{The syntax and semantics of the notation system}\label{sub:the_syntax_and_semantics_of_the_notation_system} % (fold)

We now summarize a technical presentation of the calculus that
embodies our theory of dynamics. The typical presentation of such a
calculus follows the style of giving generators and relations on
them. The grammar, below, describing term constructors, freely
generates the set of processes, $\Proc$. This set is then quotiented
by a relation known as structural congruence and it is over this set
that the notion of dynamics is expressed. This presentation is
essentially that of \cite{MeredithR05} with the addition of
polyadicity and summation. For readability we have relegated some of
the technical subtleties to an appendix.

\subsubsection{Process grammar}\label{subsub:process_grammar}

\begin{mathpar}
  \inferrule* [lab=synchronization] {} {{M} \bc \pzero \;|\; x?F \;|\; x!C }
  \and
  \inferrule* [lab=abstraction] {} {{F} \bc (x)P}
  \and
  \inferrule* [lab=concretion] {} {{C} \bc \langle Q \rangle}
  \and
  \inferrule* [lab=process] {} {{P,Q} \bc M \;| \;P|Q \;|\; @{x}}
  \and
  \inferrule* [lab=name] {} {{x} \bc \quotep{P}}
\end{mathpar} 

Note that $\vec{x}$ (resp. $\vec{P}$) denotes a vector of names
(resp. processes) of length $|\vec{x}|$ (resp. $|\vec{P}|$). We adopt
the following useful abbreviations.

\begin{mathpar}
   x?(\vec{y}).P := x.(\vec{y})P \and  x\clift{\vec{P}} := x.\clift{\vec{P}}
   \and x!(y) := \lift{x}{\dropn{y}}
   \and \Pi_{i=0}^{n-1}P_i := P_0 | \ldots | P_{n-1}
\end{mathpar}

\subsubsection{Structural congruence}

\paragraph{Free and bound names and alpha-equivalence.} At the
core of structural equivalence is alpha-equivalence which identifies
process that are the same up to a change of variable. Formally, we
recognize the distinction between free and bound names. The free names
of a process, $\freenames{P}$, may be calculated recursively as
follows:

\begin{mathpar}
\freenames{\pzero} := \emptyset
  \and \\
  \freenames{x?(y).P} := \{ x \} \cup (\freenames{P} \setminus \{ y \})
  \and 
  \freenames{x!\langle P \rangle} := \{ x \} \cup \{ P \} 
  \and \\
  \freenames{P|Q} := \freenames{P} \cup \freenames{Q}
  \and \\
  \freenames{@{x}} := \{ x \}
\end{mathpar}

$\pi$
$\quotep{\pi}$

$\freenames{-} : \pi \to \mathcal{P}(\quotep{\pi})$

\begin{eqnarray*}
  \freenames{\pzero} & := & \emptyset \\
  \freenames{x?(y).P} & := & \{ x \} \cup (\freenames{P} \setminus \{ y \}) \\
  \freenames{x!\langle P \rangle} & := & \{ x \} \cup \{ P \} \\
  \freenames{P|Q} & := & \freenames{P} \cup \freenames{Q} \\
  \freenames{\dropn{x}} & := & \{ x \}
\end{eqnarray*}

The bound names of a process, $\boundnames{P}$, are those names occurring in $P$
that are not free. For example, in $x?(y).0$, the name $x$ is free, while $y$ is bound.

\begin{mathpar}
  \inferrule* [lab=monoidal-laws] {} { P|Q \equiv Q|P \and P|0 \equiv P \and P|(Q|R) \equiv (P|Q)|R }
\end{mathpar}

\begin{mathpar}
  \inferrule* [lab=alpha-equivalence] {} { (x)P \equiv (y)P\{y/x\} \and y \not\in \freenames{P} }
\end{mathpar}

\begin{definition}
Then two processes, $P,Q$, are alpha-equivalent if $P = Q\{\vec{y}/\vec{x}\}$ for
some $\vec{x} \in \boundnames{Q},\vec{y} \in \boundnames{P}$, where $Q\{\vec{y}/\vec{x}\}$
denotes the capture-avoiding substitution of $\vec{y}$ for $\vec{x}$ in $Q$.
\end{definition}

\begin{definition}
  The {\em structural congruence} \cite{SangiorgiWalker} , $\equiv$,
  between processes is the least congruence containing
  alpha-equivalence, satisfying the abelian monoid laws
  (associativity, commutativity and $\pzero$ as identity) for parallel
  composition $|$ and for summation $+$.
\end{definition}

\subsection{Name equivalence}

We take name equivalence, written $\nameeq$, to be the smallest
equivalence relation generated by the following rules.

\begin{mathpar}
\inferrule*[lab=Quote-drop]
{ }
{ \quotep{@{x}} \nameeq x }

\inferrule*[lab=Struct-equiv]
{ P \scong Q }
{ \quotep{P} \nameeq \quotep{Q} }
\end{mathpar}

The astute reader will have noticed that the mutual recursion of names
and processes imposes a mutual recursion on alpha-equivalence and
structural equivalence via name-equivalence. Fortunately, all of this
works out pleasantly and we may calculate in the natural way, free of
concern. The reader interested in the details is referred to the
appendix \ref{appendix:rho_details}.

\subsection{Substitution}

We use $\Proc$ for the set of processes, $\QProc$ for the set of
names, and $\id{\{}\vec{y} / \vec{x} \id{\}}$ to denote partial maps,
$s : \QProc \rightarrow \QProc$. A map, $s$ lifts, uniquely, to a map
on process terms, $\widehat{s} : \Proc \rightarrow \Proc$ by the
following equations.

\begin{mathpar}
  (0) \psubstp{Q}{P} := 0 \\
  (R \juxtap S) \psubstp{Q}{P}
  :=    
  (R)\psubstp{Q}{P} \juxtap (S) \psubstp{Q}{P} \\
  (x?(y).R) \psubstp{Q}{P}    
  :=    
  (x)\substp{Q}{P} (z)\concat( (R \psubstn{z}{y}) \psubstp{Q}{P} ) \\
  (\lift{x}{R}) \psubstp{Q}{P}  
  :=
  \lift{(x)\substp{Q}{P}}{ R \psubstp{Q}{P} } \\
%   (\dropn{x})  \psubstp{Q}{P}       
%   := 
%   \left\{ 
%     \begin{array}{ccc} 
%       \dropn{\quotep{Q}} & & x \nameeq \quotep{P} \\
%       \dropn{x} & & otherwise \\
%     \end{array}
%   \right. 
  (\dropn{x})  \psubstp{Q}{P}       
  := 
  \left\{ 
    \begin{array}{ccc} 
      Q & & x \nameeq \quotep{P} \\
      \dropn{x} & & otherwise \\
    \end{array}
  \right.
\end{mathpar}
 

where

\begin{eqnarray}
  (x)\id{\{} \lpquote Q \rpquote / \lpquote P \rpquote \id{\}}            = 
  \left\{ 
    \begin{array}{ccc}
      \lpquote Q \rpquote & & x \nameeq \lpquote P \rpquote \\
      x & & otherwise \\
    \end{array}
  \right. \nonumber
\end{eqnarray}

and $z$ is chosen distinct from $\quotep{P}$, $\quotep{Q}$, the free
names in $Q$, and all the names in $R$. Our $\alpha$-equivalence will
be built in the standard way from this substitution.

\begin{remark}\label{rem:no_self_referential_names}
  One consequence of these definitions is that $\forall P. \quotep{P}
  \not\in \freenames{P}$.
\end{remark}

\subsection{ Dynamic quote: an example }

Anticipating something of what's to come, consider applying the
substitution, $\widehat{\id{\{}u / z \id{\}}}$, to the following pair
of processes, $\lift{w}{y!(z)}$ and $w[ \lpquote y!(z) \rpquote ]$.

\begin{eqnarray}
	\lift{w}{y!(z)}\widehat{\id{\{}u / z \id{\}}}
		& = &
		\lift{w}{y!(u)} \nonumber\\
	w[ \lpquote y!(z) \rpquote ] \widehat{ \id{\{}u / z \id{\}} }
		& = &
		w[ \lpquote y!(z) \rpquote ] \nonumber
\end{eqnarray}

Because the body of the process between quotes is impervious to
substitution, we get radically different answers. In fact, by
examining the first process in an input context,
e.g. $x?(z).\lift{w}{y!(z)}$, we see that the process under the lift
operator may be shaped by prefixed inputs binding a name inside it. In
this sense, the lift operator will be seen as a way to dynamically
construct processes before reifying them as names.

Finally equipped with these standard features we can present the
dynamics of the calculus.

\subsubsection{Operational semantics} 

Finally, we introduce the computational dynamics. What marks these
algebras as distinct from other more traditionally studied algebraic
structures, e.g. vector spaces or polynomial rings, is the manner in
which dynamics is captured. In traditional structures, dynamics is typically
expressed through morphisms between such structures, as in linear maps
between vector spaces or morphisms between rings. In algebras
associated with the semantics of computation, the dynamics is
expressed as part of the algebraic structure itself, through a
reduction reduction relation typically denoted by $\red$. Below, we
give a recursive presentation of this relation for the calculus used
in the encoding.

$\red \subseteq \pi \times \pi$
$\red : \pi \to \mathcal{P}(\pi)$

\begin{mathpar}
  \inferrule* [lab=Comm] { \textsf{match}( x_{src}, x_{trgt} ) } { x_{trgt}?(y)P \; | \; x_{src}!\langle {Q} \rangle \red P\{\quotep{Q}/y}\} }
  \and \\
  \inferrule* [lab=Par] {{P} \red {P}'} {{{P} | {Q}} \red {{P}' | {Q}}}
  \and
  \inferrule* [lab=Equiv]{{{P} \scong {P}'} \andalso {{P}' \red {Q}'} \andalso {{Q}' \scong {Q}}}{{P} \red {Q}}
\end{mathpar}

\begin{eqnarray*}
  match_{\equiv} (\quotep{P},\quotep{Q}) & := & P \equiv Q \\
  match_{\dagger}(\quotep{P},\quotep{Q}) & := & \forall R. P|Q \red^{*} R => R \red^{*} 0 \\
  match_{K}(\quotep{P},\quotep{Q}) & := & K \mbox{ for some context } K
\end{eqnarray*}

$u?(x)P | u!\langle Q \rangle \red P\{\quotep{Q}/x\}$

%We write $\wred$ for $\red^*$, and $P\red$ if $\exists Q $ such that $ P \red Q$.
We write $P\red$ if $\exists Q $ such that $ P \red Q$ and $P\not\red$, otherwise.

\section{Replication}

As mentioned before, it is known that replication (and hence
recursion) can be implemented in a higher-order process algebra
\cite{SangiorgiWalker}. As our first example of calculation with the
machinery thus far presented we give the construction explicitly in
the {\rhoc}.

\begin{eqnarray}
	D_{x} & := & \prefix{x}{y}{(\binpar{\outputp{x}{y}}{@{y}})} \nonumber\\
	\bangp_{x}{P} & := & \binpar{{x}!\langle{\binpar{D_{x}}{P}}\rangle}{D_{x}} \nonumber
\end{eqnarray}

\begin{eqnarray}
	\bangp_{x}{P} & & \nonumber\\
	=
	& {x}!\langle{(\prefix{x}{y}{(\outputp{x}{y} | @{y})) | P}}\rangle 
	      | \prefix{x}{y}{(\outputp{x}{y} | @{y})} & \nonumber\\
	\red
	& (\outputp{x}{y} | @{y})\substn{\quotep{(\prefix{x}{y}{(@{y} | \outputp{x}{y})) | P}}}{y} & \nonumber\\
	=
	& \outputp{x}{\quotep{(\prefix{x}{y}{(\outputp{x}{y} | @{y})) | P}}}
	  | {(\prefix{x}{y}{(\outputp{x}{y} | @{y})) | P}} & \nonumber\\
	\red
	& \ldots & \nonumber\\
	\red^*
	& P | P | \ldots & \nonumber
\end{eqnarray}

Of course, this encoding, as an implementation, runs away, unfolding
$\bangp{P}$ eagerly. A lazier and more implementable replication
operator, restricted to input-guarded processes, may be obtained as follows.

\begin{eqnarray}
\bangp{\prefix{u}{v}{P}} 
	:= 
	\binpar{\lift{x}{\prefix{u}{v}{(\binpar{D(x)}{P})}}}{D(x)} \nonumber
\end{eqnarray}

\begin{remark}
  Note that the lazier definition still does not deal with summation
  or mixed summation (i.e. sums over input and output). The reader is
  invited to construct definitions of replication that deal with these
  features. 

  Further, the definitions are parameterized in a name, $x$. Can you,
  gentle reader, make a definition that eliminates this parameter and
  guarantees no accidental interaction between the replication
  machinery and the process being replicated -- i.e. no accidental
  sharing of names used by the process to get its work done and the
  name(s) used by the replication to effect copying. This latter
  revision of the definition of replication is crucial to obtaining
  the expected identity $!!P \sim !P$.
\end{remark}

\begin{remark}\label{rem:paradoxical_combinator}
  The reader familiar with the lambda calculus will have noticed the
  similarity between $D$ and the paradoxical combinator.

  [Ed. note: the existence of this seems to suggest we have to be more
  restrictive on the set of processes and names we admit if we are to
  support no-cloning.]
\end{remark}

\subsubsection{Bisimulation}

The computational dynamics gives rise to another kind of equivalence,
the equivalence of computational behavior. As previously mentioned
this is typically captured \emph{via} some form of bisimulation.

% The notion we use in this paper is weak barbed bisimulation
% \cite{milner91polyadicpi}.

The notion we use in this paper is derived from weak barbed
bisimulation \cite{milner91polyadicpi}. 

\begin{definition}
An \emph{observation relation}, $\downarrow_{\mathcal N}$, over a set
of names, $\mathcal N$, is the smallest relation satisfying the rules
below.

\infrule[Out-barb]{y \in {\mathcal N}, \; x \nameeq y}
		  {\outputp{x}{v} \downarrow_{\mathcal N} x}
\infrule[Par-barb]{\mbox{$P\downarrow_{\mathcal N} x$ or $Q\downarrow_{\mathcal N} x$}}
		  {\binpar{P}{Q} \downarrow_{\mathcal N} x}

We write $P \Downarrow_{\mathcal N} x$ if there is $Q$ such that 
$P \wred Q$ and $Q \downarrow_{\mathcal N} x$.
\end{definition}

\begin{definition}
%\label{def.bbisim}
An  ${\mathcal N}$-\emph{barbed bisimulation} over a set of names, ${\mathcal N}$, is a symmetric binary relation 
${\mathcal S}_{\mathcal N}$ between agents such that $P\rel{S}_{\mathcal N}Q$ implies:
\begin{enumerate}
\item If $P \red P'$ then $Q \wred Q'$ and $P'\rel{S}_{\mathcal N} Q'$.
\item If $P\downarrow_{\mathcal N} x$, then $Q\Downarrow_{\mathcal N} x$.
\end{enumerate}
$P$ is ${\mathcal N}$-barbed bisimilar to $Q$, written
$P \wbbisim_{\mathcal N} Q$, if $P \rel{S}_{\mathcal N} Q$ for some ${\mathcal N}$-barbed bisimulation ${\mathcal S}_{\mathcal N}$.
\end{definition}

$\mathcal{R} \subseteq \pi \times \pi$

$P \mathcal{R} Q => \forall P'. P \red P' \Rightarrow \exists Q'. Q \red Q', P' \mathcal{R} Q'$

$P \vdash x \Rightarrow Q \vdash x$

\begin{mathpar}
  \inferrule*[lab=Out-barb]{x \nameeq y}{{y}!\langle{Q}\rangle \vdash x}
  \and
  \inferrule*[lab=Par-barb]{\mbox{$P\vdash x$ or $Q\vdash x$}}{\binpar{P}{Q} \vdash x}
\end{mathpar}

\subsubsection{Contexts}

One of the principle advantages of computational calculi like the
$\pi$-calculus is a well-defined notion of context,
contextual-equivalence and a correlation between
contextual-equivalence and notions of bisimulation. The notion of
context allows the decomposition of a process into (sub-)process and
its syntactic environment, its context. Thus, a context may be
thought of as a process with a ``hole'' (written $\Box$) in it. The
application of a context $M$ to a process $P$, written $M[P]$, is
tantamount to filling the hole in $M$ with $P$. In this paper we do
not need the full weight of this theory, but do make use of the notion
of context in the proof the main theorem. 

\begin{mathpar}
  \inferrule* [lab=summation] {} {{M_{M},M_{N}} \bc \Box \;|\; x.M_{A} \;|\; M_{M}+M_{N}}
  \and
  \inferrule* [lab=agent] {} {{M_{A}} \bc (\vec{x})M_{P} \;| \; \clift{P_0,\ldots,M_{P},\ldots,P_N}}
  \and \\
  \inferrule* [lab=process] {} {{M_{P}} \bc M_{N} \;| \;P|M_{P} }
\end{mathpar} 

\begin{mathpar}
  \inferrule* [lab=sychronization] {} {M_{N} \bc \Box \;|\; x?M_{F} \;|\; x!M_{C}}
  \and
  \inferrule* [lab=abstraction] {} {{M_{F}} \bc (x)M_{P} }
  \and
  \inferrule* [lab=concretion] {} {{M_{C}} \bc \langle M_{P} \rangle }
  \and \\
  \inferrule* [lab=process] {} {{M_{P}} \bc M_{N} \;| \;P|M_{P} }
\end{mathpar}

\begin{definition}[contextual application] Given a context $M$, and
  process $P$, we define the \emph{contextual application}, $M[P] :=
  M\{P/\Box\}$. That is, the contextual application of M to P is the
  substitution of $P$ for $\Box$ in $M$.
\end{definition}

$\meaningof{-} : L \to \mathcal{P}(\pi)$

\begin{mathpar}
  \inferrule* [lab=collection] {} {\meaningof{true} = \pi, \and \meaningof{~E} = \pi \setminus \meaningof{E}, \and \meaningof{E_{1} \& E_{2}} = \meaningof{E_{1}} \cap \meaningof{E_{2}}}
\end{mathpar}

\begin{mathpar}
  \inferrule* [lab=structure] {} {\meaningof{0} = \{ P \in \pi | P \equiv 0 \}, \and \\ \meaningof{E_1 | E_2} = \{ P \in \pi | P \equiv P_{1} | P_{2}, P_{1} \in \meaningof{E_{1}}, P_{2} \in \meaningof{E_2}\} }
\end{mathpar}

\begin{mathpar}
 \inferrule* [lab=behavior] {} {\meaningof{\langle a?b \rangle E} = \{ P \in \pi | P \equiv Q | u?(y)P', \\ \and \\\\ \and \\ \;\;\; u \in \meaningof{a}, \forall z.P'\{z/y\} \in \meaningof{E\{z/b\}}\}, \and \\ \meaningof{a!E} = \{ P \in \pi | P \equiv Q | x!\langle P' \rangle, x \in \meaningof{a} P' \in \meaningof{E}\} }
\end{mathpar}

\begin{mathpar}
 \inferrule* [lab=nominal] {} {\meaningof{\quotep{E}} = \{ \quotep{P} \in \quotep{\pi} | P \in \meaningof{E} \}, \and \meaningof{\quotep{P}} = \{ \quotep{Q} \in \quotep{\pi} | P \equiv Q \} \and \\ \meaningof{@\quotep{E}} = \{ P \in \pi | P \equiv @x, x \in \meaningof{E} \}}
\end{mathpar}

\begin{eqnarray*}
  \\
  \meaningof{-} : TS \to ST
\end{eqnarray*}

\begin{eqnarray*}
  \\
  L : TS \to ST
\end{eqnarray*}

\begin{eqnarray*}
  \\
  P \models E \iff P \in \meaningof{E}
\end{eqnarray*}

\begin{eqnarray*}
  P \approx_{L} Q \iff \forall E \in L. P \models E \iff Q \models E
\end{eqnarray*}

\begin{eqnarray*}
  P \approx_{K} Q
\end{eqnarray*}

\begin{eqnarray*}
  P \approx Q
\end{eqnarray*}

$\approx_{K} = \approx = \approx_{L}$

\subsubsection{Contextual duality}

Note that contexts extend the quotation operation to a family of
operations from processes to names. Given a context, $M$, we can
define a \emph{nominal context}, $\quotep{M}$ by $\quotep{M}[P] :=
\quotep{M[P]}$. To foreshadow what is to come we observe that these
operations enjoy a duality with processes very much like the duality
between vectors and maps from vectors to scalars.

Further, because the calculus is essentially higher-order, we have a
correspondence between contexts and processes. More specifically,
given a name $x$ and a context $M$ we can construct $M^{*}_{x}$ such
that 

\begin{mathpar}
  M^{*}_{x} | \lift{x}{P} \red M[P]
\end{mathpar}

namely,

\begin{mathpar}
  M^{*}_{x} := x?(u).M[\dropn{u}]
\end{mathpar}

The dependence of $M^{*}_{x}$ on a name makes it an abstraction, 

\begin{mathpar}
  M^{*} := (x)x?(u).M[\dropn{u}]
\end{mathpar}

\subsection{Additional notation}

It will sometimes be convenient to denote the process a name
quotes. We already have the notation $x = \quotep{P}$, but it will be
convenient to introduce an alternate notation, $\procn{x}$, when we
want to emphasize the connection to the use of the name. Note that, by
virtue of name equivalence, $\quotep{\procn{x}} \nameeq x$; so, the
notation is consistent with previous definitions.

Further, because names have structure it is possible to effect
substitutions on the basis of that structure. This means we need to
upgrade our notation for substitutions, which we accomplish by
adapting comprehension notation. Thus,

\begin{mathpar}
  P\{ y / x : x \in S \}
\end{mathpar}

is interpreted to mean the process derived from P by replacing (in a
capture-avoiding manner) each occurrence of $x$ in $S$ by $y$. For example,

\begin{mathpar}
  P\{ \quotep{\procn{x}|\procn{x}} / x : x \in \freenames{P} \}
\end{mathpar}

will replace each (occurrence) of a free name $x$ in $P$ by
$\quotep{\procn{x}|\procn{x}}$.

Also, we will avail ourselves of the notation $x^{L}$ and $x^{R}$ to
denote injections of a name into disjoint copies of the name
space. There are numerous ways to accomplish this. One example can be
found in \cite{MeredithR05}. This notation overloads to vectors of
names: $\vec{x}^{\pi} := (x_{i}^{\pi} \; : \; 0 \leq i < |\vec{x}| )$ where $\pi \in \{L,R\}$.

We also use $P^{\Box} := P|\Box$.

In \cite{MeredithR05} an interpretation of the new operator is
given. It turns out that there are several possible interpretations
all enjoying the requisite algebraic properties of the operator (see
\cite{milner91polyadicpi}). We will therefore make liberal use of
$(\nu\; \vec{x})P$.

% subsection the_syntax_and_semantics_of_the_notation_system (end)   

\input{qm2pi.qmops} 

\input{qm2pi.sterngerlach} 

\input{qm2pi.metric} 

% section concurrent_process_calculi (end)

%\input{qm2pi.proofsketch}

% section proof sketch (end)

%\input{qm2pi.slviaknots} 

% section spatial logic via knots (end)

\input{qm2pi.conclusion}

% section conclusion (end)

%\input{qm2pi.dtcodes} 

% section wiring algorithm (end)

\input{qm2pi.ack} 

% section acknowledgments (end)

\newpage


\bibliographystyle{plain}   
\bibliography{../../biblios/main.bib}

\input{qm2pi.rhodetails}

\end{document}

 

% section concurrent_process_calculi (end)

%\documentclass[12pt]{llncs}
%\documentclass{jktr}

\usepackage[pdftex]{hyperref}                   
\usepackage {listings}
\usepackage {mathpartir}
\usepackage{bcprules}
%\usepackage{listings}
                       
\usepackage{graphicx} 
%\usepackage[margins=2.5cm,nohead,nofoot]{geometry}
%\usepackage{geometry}
\usepackage{amsfonts}
\usepackage{amstext}
\usepackage{latexsym}
\usepackage{amssymb}
\usepackage{color}


%\include{myPreamble}
\include{qm2pi.local} 

%\ifpdf
%\usepackage[pdftex]{graphicx}
%\else
%\usepackage{graphicx}
%\fi

 % \ifpdf
%  \usepackage{pdfsync}
%  \if


%\title{Brief Article}
%\author{David F. Snyder}
%\author{L.G. Meredith}

%\address{Dept. of Math., Texas State University--San Marcos, San Marcos, TX 78666}
       
\pagestyle{empty}


\begin{document}

\lstset{language=[Objective]Caml,frame=shadowbox}

\input{qm2pi.front}

% section front matter (end)

\input{qm2pi.intro} 
 
% section introduction (end)

% \input{qm2pi.knotations} 

% section notation (end)

\input{qm2pi.process.calculi} 

% section concurrent_process_calculi_and_spatial_logics_ (end)
    
%\input{qm2pi.knots2pi} 

%\input{qm2pi.trefoil} 

%\input{qm2pi.mainthm} 

% subsection basic_interpretation (end)

%\input{qm2pi.rho.presentation} 
\subsection{The syntax and semantics of the notation system}\label{sub:the_syntax_and_semantics_of_the_notation_system} % (fold)

We now summarize a technical presentation of the calculus that
embodies our theory of dynamics. The typical presentation of such a
calculus follows the style of giving generators and relations on
them. The grammar, below, describing term constructors, freely
generates the set of processes, $\Proc$. This set is then quotiented
by a relation known as structural congruence and it is over this set
that the notion of dynamics is expressed. This presentation is
essentially that of \cite{MeredithR05} with the addition of
polyadicity and summation. For readability we have relegated some of
the technical subtleties to an appendix.

\subsubsection{Process grammar}\label{subsub:process_grammar}

\begin{mathpar}
  \inferrule* [lab=synchronization] {} {{M} \bc \pzero \;|\; x?F \;|\; x!C }
  \and
  \inferrule* [lab=abstraction] {} {{F} \bc (x)P}
  \and
  \inferrule* [lab=concretion] {} {{C} \bc \langle Q \rangle}
  \and
  \inferrule* [lab=process] {} {{P,Q} \bc M \;| \;P|Q \;|\; @{x}}
  \and
  \inferrule* [lab=name] {} {{x} \bc \quotep{P}}
\end{mathpar} 

Note that $\vec{x}$ (resp. $\vec{P}$) denotes a vector of names
(resp. processes) of length $|\vec{x}|$ (resp. $|\vec{P}|$). We adopt
the following useful abbreviations.

\begin{mathpar}
   x?(\vec{y}).P := x.(\vec{y})P \and  x\clift{\vec{P}} := x.\clift{\vec{P}}
   \and x!(y) := \lift{x}{\dropn{y}}
   \and \Pi_{i=0}^{n-1}P_i := P_0 | \ldots | P_{n-1}
\end{mathpar}

\subsubsection{Structural congruence}

\paragraph{Free and bound names and alpha-equivalence.} At the
core of structural equivalence is alpha-equivalence which identifies
process that are the same up to a change of variable. Formally, we
recognize the distinction between free and bound names. The free names
of a process, $\freenames{P}$, may be calculated recursively as
follows:

\begin{mathpar}
\freenames{\pzero} := \emptyset
  \and \\
  \freenames{x?(y).P} := \{ x \} \cup (\freenames{P} \setminus \{ y \})
  \and 
  \freenames{x!\langle P \rangle} := \{ x \} \cup \{ P \} 
  \and \\
  \freenames{P|Q} := \freenames{P} \cup \freenames{Q}
  \and \\
  \freenames{@{x}} := \{ x \}
\end{mathpar}

$\pi$
$\quotep{\pi}$

$\freenames{-} : \pi \to \mathcal{P}(\quotep{\pi})$

\begin{eqnarray*}
  \freenames{\pzero} & := & \emptyset \\
  \freenames{x?(y).P} & := & \{ x \} \cup (\freenames{P} \setminus \{ y \}) \\
  \freenames{x!\langle P \rangle} & := & \{ x \} \cup \{ P \} \\
  \freenames{P|Q} & := & \freenames{P} \cup \freenames{Q} \\
  \freenames{\dropn{x}} & := & \{ x \}
\end{eqnarray*}

The bound names of a process, $\boundnames{P}$, are those names occurring in $P$
that are not free. For example, in $x?(y).0$, the name $x$ is free, while $y$ is bound.

\begin{mathpar}
  \inferrule* [lab=monoidal-laws] {} { P|Q \equiv Q|P \and P|0 \equiv P \and P|(Q|R) \equiv (P|Q)|R }
\end{mathpar}

\begin{mathpar}
  \inferrule* [lab=alpha-equivalence] {} { (x)P \equiv (y)P\{y/x\} \and y \not\in \freenames{P} }
\end{mathpar}

\begin{definition}
Then two processes, $P,Q$, are alpha-equivalent if $P = Q\{\vec{y}/\vec{x}\}$ for
some $\vec{x} \in \boundnames{Q},\vec{y} \in \boundnames{P}$, where $Q\{\vec{y}/\vec{x}\}$
denotes the capture-avoiding substitution of $\vec{y}$ for $\vec{x}$ in $Q$.
\end{definition}

\begin{definition}
  The {\em structural congruence} \cite{SangiorgiWalker} , $\equiv$,
  between processes is the least congruence containing
  alpha-equivalence, satisfying the abelian monoid laws
  (associativity, commutativity and $\pzero$ as identity) for parallel
  composition $|$ and for summation $+$.
\end{definition}

\subsection{Name equivalence}

We take name equivalence, written $\nameeq$, to be the smallest
equivalence relation generated by the following rules.

\begin{mathpar}
\inferrule*[lab=Quote-drop]
{ }
{ \quotep{@{x}} \nameeq x }

\inferrule*[lab=Struct-equiv]
{ P \scong Q }
{ \quotep{P} \nameeq \quotep{Q} }
\end{mathpar}

The astute reader will have noticed that the mutual recursion of names
and processes imposes a mutual recursion on alpha-equivalence and
structural equivalence via name-equivalence. Fortunately, all of this
works out pleasantly and we may calculate in the natural way, free of
concern. The reader interested in the details is referred to the
appendix \ref{appendix:rho_details}.

\subsection{Substitution}

We use $\Proc$ for the set of processes, $\QProc$ for the set of
names, and $\id{\{}\vec{y} / \vec{x} \id{\}}$ to denote partial maps,
$s : \QProc \rightarrow \QProc$. A map, $s$ lifts, uniquely, to a map
on process terms, $\widehat{s} : \Proc \rightarrow \Proc$ by the
following equations.

\begin{mathpar}
  (0) \psubstp{Q}{P} := 0 \\
  (R \juxtap S) \psubstp{Q}{P}
  :=    
  (R)\psubstp{Q}{P} \juxtap (S) \psubstp{Q}{P} \\
  (x?(y).R) \psubstp{Q}{P}    
  :=    
  (x)\substp{Q}{P} (z)\concat( (R \psubstn{z}{y}) \psubstp{Q}{P} ) \\
  (\lift{x}{R}) \psubstp{Q}{P}  
  :=
  \lift{(x)\substp{Q}{P}}{ R \psubstp{Q}{P} } \\
%   (\dropn{x})  \psubstp{Q}{P}       
%   := 
%   \left\{ 
%     \begin{array}{ccc} 
%       \dropn{\quotep{Q}} & & x \nameeq \quotep{P} \\
%       \dropn{x} & & otherwise \\
%     \end{array}
%   \right. 
  (\dropn{x})  \psubstp{Q}{P}       
  := 
  \left\{ 
    \begin{array}{ccc} 
      Q & & x \nameeq \quotep{P} \\
      \dropn{x} & & otherwise \\
    \end{array}
  \right.
\end{mathpar}
 

where

\begin{eqnarray}
  (x)\id{\{} \lpquote Q \rpquote / \lpquote P \rpquote \id{\}}            = 
  \left\{ 
    \begin{array}{ccc}
      \lpquote Q \rpquote & & x \nameeq \lpquote P \rpquote \\
      x & & otherwise \\
    \end{array}
  \right. \nonumber
\end{eqnarray}

and $z$ is chosen distinct from $\quotep{P}$, $\quotep{Q}$, the free
names in $Q$, and all the names in $R$. Our $\alpha$-equivalence will
be built in the standard way from this substitution.

\begin{remark}\label{rem:no_self_referential_names}
  One consequence of these definitions is that $\forall P. \quotep{P}
  \not\in \freenames{P}$.
\end{remark}

\subsection{ Dynamic quote: an example }

Anticipating something of what's to come, consider applying the
substitution, $\widehat{\id{\{}u / z \id{\}}}$, to the following pair
of processes, $\lift{w}{y!(z)}$ and $w[ \lpquote y!(z) \rpquote ]$.

\begin{eqnarray}
	\lift{w}{y!(z)}\widehat{\id{\{}u / z \id{\}}}
		& = &
		\lift{w}{y!(u)} \nonumber\\
	w[ \lpquote y!(z) \rpquote ] \widehat{ \id{\{}u / z \id{\}} }
		& = &
		w[ \lpquote y!(z) \rpquote ] \nonumber
\end{eqnarray}

Because the body of the process between quotes is impervious to
substitution, we get radically different answers. In fact, by
examining the first process in an input context,
e.g. $x?(z).\lift{w}{y!(z)}$, we see that the process under the lift
operator may be shaped by prefixed inputs binding a name inside it. In
this sense, the lift operator will be seen as a way to dynamically
construct processes before reifying them as names.

Finally equipped with these standard features we can present the
dynamics of the calculus.

\subsubsection{Operational semantics} 

Finally, we introduce the computational dynamics. What marks these
algebras as distinct from other more traditionally studied algebraic
structures, e.g. vector spaces or polynomial rings, is the manner in
which dynamics is captured. In traditional structures, dynamics is typically
expressed through morphisms between such structures, as in linear maps
between vector spaces or morphisms between rings. In algebras
associated with the semantics of computation, the dynamics is
expressed as part of the algebraic structure itself, through a
reduction reduction relation typically denoted by $\red$. Below, we
give a recursive presentation of this relation for the calculus used
in the encoding.

$\red \subseteq \pi \times \pi$
$\red : \pi \to \mathcal{P}(\pi)$

\begin{mathpar}
  \inferrule* [lab=Comm] { \textsf{match}( x_{src}, x_{trgt} ) } { x_{trgt}?(y)P \; | \; x_{src}!\langle {Q} \rangle \red P\{\quotep{Q}/y}\} }
  \and \\
  \inferrule* [lab=Par] {{P} \red {P}'} {{{P} | {Q}} \red {{P}' | {Q}}}
  \and
  \inferrule* [lab=Equiv]{{{P} \scong {P}'} \andalso {{P}' \red {Q}'} \andalso {{Q}' \scong {Q}}}{{P} \red {Q}}
\end{mathpar}

\begin{eqnarray*}
  match_{\equiv} (\quotep{P},\quotep{Q}) & := & P \equiv Q \\
  match_{\dagger}(\quotep{P},\quotep{Q}) & := & \forall R. P|Q \red^{*} R => R \red^{*} 0 \\
  match_{K}(\quotep{P},\quotep{Q}) & := & K \mbox{ for some context } K
\end{eqnarray*}

$u?(x)P | u!\langle Q \rangle \red P\{\quotep{Q}/x\}$

%We write $\wred$ for $\red^*$, and $P\red$ if $\exists Q $ such that $ P \red Q$.
We write $P\red$ if $\exists Q $ such that $ P \red Q$ and $P\not\red$, otherwise.

\section{Replication}

As mentioned before, it is known that replication (and hence
recursion) can be implemented in a higher-order process algebra
\cite{SangiorgiWalker}. As our first example of calculation with the
machinery thus far presented we give the construction explicitly in
the {\rhoc}.

\begin{eqnarray}
	D_{x} & := & \prefix{x}{y}{(\binpar{\outputp{x}{y}}{@{y}})} \nonumber\\
	\bangp_{x}{P} & := & \binpar{{x}!\langle{\binpar{D_{x}}{P}}\rangle}{D_{x}} \nonumber
\end{eqnarray}

\begin{eqnarray}
	\bangp_{x}{P} & & \nonumber\\
	=
	& {x}!\langle{(\prefix{x}{y}{(\outputp{x}{y} | @{y})) | P}}\rangle 
	      | \prefix{x}{y}{(\outputp{x}{y} | @{y})} & \nonumber\\
	\red
	& (\outputp{x}{y} | @{y})\substn{\quotep{(\prefix{x}{y}{(@{y} | \outputp{x}{y})) | P}}}{y} & \nonumber\\
	=
	& \outputp{x}{\quotep{(\prefix{x}{y}{(\outputp{x}{y} | @{y})) | P}}}
	  | {(\prefix{x}{y}{(\outputp{x}{y} | @{y})) | P}} & \nonumber\\
	\red
	& \ldots & \nonumber\\
	\red^*
	& P | P | \ldots & \nonumber
\end{eqnarray}

Of course, this encoding, as an implementation, runs away, unfolding
$\bangp{P}$ eagerly. A lazier and more implementable replication
operator, restricted to input-guarded processes, may be obtained as follows.

\begin{eqnarray}
\bangp{\prefix{u}{v}{P}} 
	:= 
	\binpar{\lift{x}{\prefix{u}{v}{(\binpar{D(x)}{P})}}}{D(x)} \nonumber
\end{eqnarray}

\begin{remark}
  Note that the lazier definition still does not deal with summation
  or mixed summation (i.e. sums over input and output). The reader is
  invited to construct definitions of replication that deal with these
  features. 

  Further, the definitions are parameterized in a name, $x$. Can you,
  gentle reader, make a definition that eliminates this parameter and
  guarantees no accidental interaction between the replication
  machinery and the process being replicated -- i.e. no accidental
  sharing of names used by the process to get its work done and the
  name(s) used by the replication to effect copying. This latter
  revision of the definition of replication is crucial to obtaining
  the expected identity $!!P \sim !P$.
\end{remark}

\begin{remark}\label{rem:paradoxical_combinator}
  The reader familiar with the lambda calculus will have noticed the
  similarity between $D$ and the paradoxical combinator.

  [Ed. note: the existence of this seems to suggest we have to be more
  restrictive on the set of processes and names we admit if we are to
  support no-cloning.]
\end{remark}

\subsubsection{Bisimulation}

The computational dynamics gives rise to another kind of equivalence,
the equivalence of computational behavior. As previously mentioned
this is typically captured \emph{via} some form of bisimulation.

% The notion we use in this paper is weak barbed bisimulation
% \cite{milner91polyadicpi}.

The notion we use in this paper is derived from weak barbed
bisimulation \cite{milner91polyadicpi}. 

\begin{definition}
An \emph{observation relation}, $\downarrow_{\mathcal N}$, over a set
of names, $\mathcal N$, is the smallest relation satisfying the rules
below.

\infrule[Out-barb]{y \in {\mathcal N}, \; x \nameeq y}
		  {\outputp{x}{v} \downarrow_{\mathcal N} x}
\infrule[Par-barb]{\mbox{$P\downarrow_{\mathcal N} x$ or $Q\downarrow_{\mathcal N} x$}}
		  {\binpar{P}{Q} \downarrow_{\mathcal N} x}

We write $P \Downarrow_{\mathcal N} x$ if there is $Q$ such that 
$P \wred Q$ and $Q \downarrow_{\mathcal N} x$.
\end{definition}

\begin{definition}
%\label{def.bbisim}
An  ${\mathcal N}$-\emph{barbed bisimulation} over a set of names, ${\mathcal N}$, is a symmetric binary relation 
${\mathcal S}_{\mathcal N}$ between agents such that $P\rel{S}_{\mathcal N}Q$ implies:
\begin{enumerate}
\item If $P \red P'$ then $Q \wred Q'$ and $P'\rel{S}_{\mathcal N} Q'$.
\item If $P\downarrow_{\mathcal N} x$, then $Q\Downarrow_{\mathcal N} x$.
\end{enumerate}
$P$ is ${\mathcal N}$-barbed bisimilar to $Q$, written
$P \wbbisim_{\mathcal N} Q$, if $P \rel{S}_{\mathcal N} Q$ for some ${\mathcal N}$-barbed bisimulation ${\mathcal S}_{\mathcal N}$.
\end{definition}

$\mathcal{R} \subseteq \pi \times \pi$

$P \mathcal{R} Q => \forall P'. P \red P' \Rightarrow \exists Q'. Q \red Q', P' \mathcal{R} Q'$

$P \vdash x \Rightarrow Q \vdash x$

\begin{mathpar}
  \inferrule*[lab=Out-barb]{x \nameeq y}{{y}!\langle{Q}\rangle \vdash x}
  \and
  \inferrule*[lab=Par-barb]{\mbox{$P\vdash x$ or $Q\vdash x$}}{\binpar{P}{Q} \vdash x}
\end{mathpar}

\subsubsection{Contexts}

One of the principle advantages of computational calculi like the
$\pi$-calculus is a well-defined notion of context,
contextual-equivalence and a correlation between
contextual-equivalence and notions of bisimulation. The notion of
context allows the decomposition of a process into (sub-)process and
its syntactic environment, its context. Thus, a context may be
thought of as a process with a ``hole'' (written $\Box$) in it. The
application of a context $M$ to a process $P$, written $M[P]$, is
tantamount to filling the hole in $M$ with $P$. In this paper we do
not need the full weight of this theory, but do make use of the notion
of context in the proof the main theorem. 

\begin{mathpar}
  \inferrule* [lab=summation] {} {{M_{M},M_{N}} \bc \Box \;|\; x.M_{A} \;|\; M_{M}+M_{N}}
  \and
  \inferrule* [lab=agent] {} {{M_{A}} \bc (\vec{x})M_{P} \;| \; \clift{P_0,\ldots,M_{P},\ldots,P_N}}
  \and \\
  \inferrule* [lab=process] {} {{M_{P}} \bc M_{N} \;| \;P|M_{P} }
\end{mathpar} 

\begin{mathpar}
  \inferrule* [lab=sychronization] {} {M_{N} \bc \Box \;|\; x?M_{F} \;|\; x!M_{C}}
  \and
  \inferrule* [lab=abstraction] {} {{M_{F}} \bc (x)M_{P} }
  \and
  \inferrule* [lab=concretion] {} {{M_{C}} \bc \langle M_{P} \rangle }
  \and \\
  \inferrule* [lab=process] {} {{M_{P}} \bc M_{N} \;| \;P|M_{P} }
\end{mathpar}

\begin{definition}[contextual application] Given a context $M$, and
  process $P$, we define the \emph{contextual application}, $M[P] :=
  M\{P/\Box\}$. That is, the contextual application of M to P is the
  substitution of $P$ for $\Box$ in $M$.
\end{definition}

$\meaningof{-} : L \to \mathcal{P}(\pi)$

\begin{mathpar}
  \inferrule* [lab=collection] {} {\meaningof{true} = \pi, \and \meaningof{~E} = \pi \setminus \meaningof{E}, \and \meaningof{E_{1} \& E_{2}} = \meaningof{E_{1}} \cap \meaningof{E_{2}}}
\end{mathpar}

\begin{mathpar}
  \inferrule* [lab=structure] {} {\meaningof{0} = \{ P \in \pi | P \equiv 0 \}, \and \\ \meaningof{E_1 | E_2} = \{ P \in \pi | P \equiv P_{1} | P_{2}, P_{1} \in \meaningof{E_{1}}, P_{2} \in \meaningof{E_2}\} }
\end{mathpar}

\begin{mathpar}
 \inferrule* [lab=behavior] {} {\meaningof{\langle a?b \rangle E} = \{ P \in \pi | P \equiv Q | u?(y)P', \\ \and \\\\ \and \\ \;\;\; u \in \meaningof{a}, \forall z.P'\{z/y\} \in \meaningof{E\{z/b\}}\}, \and \\ \meaningof{a!E} = \{ P \in \pi | P \equiv Q | x!\langle P' \rangle, x \in \meaningof{a} P' \in \meaningof{E}\} }
\end{mathpar}

\begin{mathpar}
 \inferrule* [lab=nominal] {} {\meaningof{\quotep{E}} = \{ \quotep{P} \in \quotep{\pi} | P \in \meaningof{E} \}, \and \meaningof{\quotep{P}} = \{ \quotep{Q} \in \quotep{\pi} | P \equiv Q \} \and \\ \meaningof{@\quotep{E}} = \{ P \in \pi | P \equiv @x, x \in \meaningof{E} \}}
\end{mathpar}

\begin{eqnarray*}
  \\
  \meaningof{-} : TS \to ST
\end{eqnarray*}

\begin{eqnarray*}
  \\
  L : TS \to ST
\end{eqnarray*}

\begin{eqnarray*}
  \\
  P \models E \iff P \in \meaningof{E}
\end{eqnarray*}

\begin{eqnarray*}
  P \approx_{L} Q \iff \forall E \in L. P \models E \iff Q \models E
\end{eqnarray*}

\begin{eqnarray*}
  P \approx_{K} Q
\end{eqnarray*}

\begin{eqnarray*}
  P \approx Q
\end{eqnarray*}

$\approx_{K} = \approx = \approx_{L}$

\subsubsection{Contextual duality}

Note that contexts extend the quotation operation to a family of
operations from processes to names. Given a context, $M$, we can
define a \emph{nominal context}, $\quotep{M}$ by $\quotep{M}[P] :=
\quotep{M[P]}$. To foreshadow what is to come we observe that these
operations enjoy a duality with processes very much like the duality
between vectors and maps from vectors to scalars.

Further, because the calculus is essentially higher-order, we have a
correspondence between contexts and processes. More specifically,
given a name $x$ and a context $M$ we can construct $M^{*}_{x}$ such
that 

\begin{mathpar}
  M^{*}_{x} | \lift{x}{P} \red M[P]
\end{mathpar}

namely,

\begin{mathpar}
  M^{*}_{x} := x?(u).M[\dropn{u}]
\end{mathpar}

The dependence of $M^{*}_{x}$ on a name makes it an abstraction, 

\begin{mathpar}
  M^{*} := (x)x?(u).M[\dropn{u}]
\end{mathpar}

\subsection{Additional notation}

It will sometimes be convenient to denote the process a name
quotes. We already have the notation $x = \quotep{P}$, but it will be
convenient to introduce an alternate notation, $\procn{x}$, when we
want to emphasize the connection to the use of the name. Note that, by
virtue of name equivalence, $\quotep{\procn{x}} \nameeq x$; so, the
notation is consistent with previous definitions.

Further, because names have structure it is possible to effect
substitutions on the basis of that structure. This means we need to
upgrade our notation for substitutions, which we accomplish by
adapting comprehension notation. Thus,

\begin{mathpar}
  P\{ y / x : x \in S \}
\end{mathpar}

is interpreted to mean the process derived from P by replacing (in a
capture-avoiding manner) each occurrence of $x$ in $S$ by $y$. For example,

\begin{mathpar}
  P\{ \quotep{\procn{x}|\procn{x}} / x : x \in \freenames{P} \}
\end{mathpar}

will replace each (occurrence) of a free name $x$ in $P$ by
$\quotep{\procn{x}|\procn{x}}$.

Also, we will avail ourselves of the notation $x^{L}$ and $x^{R}$ to
denote injections of a name into disjoint copies of the name
space. There are numerous ways to accomplish this. One example can be
found in \cite{MeredithR05}. This notation overloads to vectors of
names: $\vec{x}^{\pi} := (x_{i}^{\pi} \; : \; 0 \leq i < |\vec{x}| )$ where $\pi \in \{L,R\}$.

We also use $P^{\Box} := P|\Box$.

In \cite{MeredithR05} an interpretation of the new operator is
given. It turns out that there are several possible interpretations
all enjoying the requisite algebraic properties of the operator (see
\cite{milner91polyadicpi}). We will therefore make liberal use of
$(\nu\; \vec{x})P$.

% subsection the_syntax_and_semantics_of_the_notation_system (end)   

\input{qm2pi.qmops} 

\input{qm2pi.sterngerlach} 

\input{qm2pi.metric} 

% section concurrent_process_calculi (end)

%\input{qm2pi.proofsketch}

% section proof sketch (end)

%\input{qm2pi.slviaknots} 

% section spatial logic via knots (end)

\input{qm2pi.conclusion}

% section conclusion (end)

%\input{qm2pi.dtcodes} 

% section wiring algorithm (end)

\input{qm2pi.ack} 

% section acknowledgments (end)

\newpage


\bibliographystyle{plain}   
\bibliography{../../biblios/main.bib}

\input{qm2pi.rhodetails}

\end{document}



% section proof sketch (end)

%\section{Unlikely characters: spatial logic for
  knots}\label{sub:characteristic_formulae} % (fold)

Associated to the mobile process calculi are a family of logics known
as the Hennessy-Milner logics. These logics typically enjoy a
semantics interpreting formulae as sets of processes that when
factored through the encoding outlined above allows an identification
of classes of knots with logical formulae. In the context of this
encoding the sub-family known as the spatial logics \cite{CairesC03}
\cite{CairesC04} \cite{Caires04} are of particular interest providing
several important features for expressing and reasoning about
properties (i.e. classes) of knots. We hint here at how this may be done.

%\begin{description}
%\item [structural connectives] 
\subsubsection{Structural connectives} The spatial logics enjoy
structural connectives corresponding, at the logical level, to the
parallel composition ($P | Q$) and new name ($(\nu \; x)P$)
connectives for processes. As illustrated in the examples below, these
connectives are extremely expressive given the shape of our encoding.
%\item [decideable satisfaction]

\subsubsection{Decideable satisfaction}
In \cite{Caires04} the satisfaction relation is shown to be decideable
for a rich class of processes. It further turns out that the image of
the our encoding is a proper subset of that class. This result
provides the basis for an algorithm by which to search for knots
enjoying a given property.
%\item [characteristic formulae]

\subsubsection{Characteristic formulae}
In the same paper \cite{Caires04} , Caires presents a means of calculating
characteristic formulae, selecting equivalence classes of processes
up to a pre--specified depth limit on the support set of names. Composed with our
encoding, this characteristic formula can be used to select
characteristic formulae for knots.
%\end{description}

\subsubsection{Spatial logic formulae}

The grammar below (segmented for comprehension) summarizes the syntax
of spatial logic formulae. We employ illustrative examples in the
sequel to provide an intuitive understanding of their meaning
referring the reader to \cite{Caires04} for a more detailed explication
of the semantics.

\begin{mathpar}
  \inferrule* [lab=boolean] {} {{A,B} \bc T \;|\; \neg A \;|\; A \wedge B \;|\; \eta = \eta'}
  \and
  \inferrule* [lab=spatial] {} {|\; \pzero \;|\; A | B \;|\; x \text{\textregistered} A \;|\; \forall x . A \;|\;  H x . A}
  \and
  \inferrule* [lab=behavioral] {} {|\; \alpha . A}
  \and 
  \inferrule* [lab=recursion] {} {|\; X(\vec{u}) \;|\; \mu X(\vec{u}) . A}
  \and
  \inferrule* [lab=action] {} {\alpha \bc \langle x?(\vec{y}) \rangle \;|\; \langle x!(\vec{y}) \rangle \;|\; \langle \tau \rangle}
  \and 
  \inferrule* [lab=name] {} {\eta \bc x \;|\; \tau}
\end{mathpar} 

% subsection characteristic_formulae (end)   	 

\subsection{Example formulae}\label{sub:example_formulae_} % (fold)

\subsubsection{Crossing as formula.}
% 
% \begin{align*}
%   \frac{d}{dx} \sin x &= \cos x 
%   & \frac{d}{dx} e^x &= e^x \\
%   \frac{d}{dx} \cos x &= - \sin x 
%   & \frac{d}{dx} \log x &= \frac{1}{x} \\
% \end{align*} 

\begin{align*}
 \mu C(x_{0},x_{1},y_{0},y_{1},u).&(\langle x_{0}?(z) \rangle(\langle u! \rangle\langle y_{1}!z \rangle C(x_{0},x_{1},y_{0},y_{1},u)) & \\
  & \wedge \langle y_{1}?(z) \rangle (\langle u! \rangle \langle x_{0}!z \rangle C(x_{0},x_{1},y_{0},y_{1},u)) & \\
  & \wedge \langle x_{1}?(z) \rangle (\langle u? \rangle \langle y_{0}!z \rangle C(x_{0},x_{1},y_{0},y_{1},u)) & \\
  & \wedge \langle y_{0}?(z) \rangle (\langle u? \rangle \langle x_{1}!z \rangle C(x_{0},x_{1},y_{0},y_{1},u))) &
\end{align*}

The lexicographical similarity between the shape of this formulae and
the shape of definition of the process representing a crossing reveals
the intuitive meaning of this formulae. It describes the capabilities
of a process that has the right to represent a crossing. For example
it picks out processes that may perform an input on the port $x_0$ in
its initial menu of capabilities. What differentiates the formula
from the process, however, is that the crossing process is the
smallest candidate to satisfy the formula. Infinitely many other
processes -- with internal behavior hidden behind this interface, so
to speak -- also satisfy this formula. Even this simple formula,
then, can be seen to open a new view onto knots, providing a
computational interpretation of \emph{virtual} knots.

Note that this formula is derived by hand. A similar formula can be
derived by employing Caires' calculation of characteristic formula
\cite{Caires04} to the process representing a crossing. In light of
this discussion, we let
$\meaningof{C}_{\phi}(x0,x1,y0,y1,u)$ denote a formula specifying the
dynamics we wish to capture of a crossing. To guarantee we preserve
the shape of the interface and minimal semantics we demand that
$\meaningof{C}_{\phi}(x0,x1,y0,y1,u) \Rightarrow
\textbf{C}(x0,x1,y0,y1,u)$ where $\textbf{C}(x0,x1,y0,y1,u)$ denotes
the formula above.
                            
\subsubsection{Crossing number constraints.}
The moral content of the context lemma (Lemma \ref{context}) is that the notion of
``locality'' in the Reidemeister moves is effectively captured by the
parallel composition operator of the process calculus. This intuition
extends through the logic. Given a formula,
$\meaningof{C}_{\phi}(x0,x1,y0,y1,u)$, we can use the structural
connectives to specify constraints on crossing numbers, such as at
least $n$ crossings, or exactly $n$ crossings.
\begin{mathpar}
  \inferrule* [lab=at-least-n] {} { K^{\geq n}_{\phi}(\vec{xs},\vec{ys}) := \Pi_{i=0}^{n-1} Hu . \meaningof{C}_{\phi}(xs_i,ys_i,u) | T }
  \and 
  \inferrule* [lab=exactly-n] {} { K^{= n}_{\phi}(\vec{xs},\vec{ys}) := \Pi_{i=0}^{n-1} Hu . \meaningof{C}_{\phi}(xs_i,ys_i,u) | \neg (\forall x_0,y_0,x_1,y_1,u . \meaningof{C}_{\phi}(x_0,y_0,x_1,y_1,u) | T) }
\end{mathpar}

To round out this section, recall that the encoding of an $n$-crossing
knot decomposes into a parallel composition of $n$ \emph{copies} of a
crossing process together with a wiring harness. To specify different
knot classes with the same crossing number amounts to specifying
logical constraints on the wiring harness. In the interest of space,
we defer examples to a forthcoming paper. Suffice it to say that both
the conditions ``alternating knot'' and ``contains the tangle
corresponding to 5/3'' are expressible. For example, it is possible to
calculate the characteristic formula of a process corresponding to the
tangle 5/3 and conjoin it into the classifying formula via the
composition connective of the logic.

Finally, we wish to observe that it is entirely within reason to
contemplate a more domain-specific version of spatial logic tailored
to the shape of processes in the image of the encoding. Such a
domain-specific logic would have a better claim to the title formal
language of knot properties.

% subsection example_formulae_ (end)

% section knots_as_processes (end) 

% section spatial logic via knots (end)

\section{Conclusions and future work}

\paragraph{Testing physical space}
You, gentle reader, may wonder why of all the theorems to be proved
given this set up we pick the one above. In some sense it's hardly
central to quantum mechanics. We see it as central in the sense that
it firmly establishes a notion of physical space arising from a notion
of the equivalence of behavior. Relating bisimulation to a metric is a
big step forward, but one is faced with interpreting the relationship
of that metric space to something more physical. Quantum mechanical
notions of ``physical'' space are still far from intuitive, but by
relating this idea of distance as testing to calculations that predict
physical circumstances we are making a not insignificant step forward
toward an understanding of the physical space we inhabit as
essentially dynamic.

\paragraph{Effectivity and simulation}
One of the observations we have yet to make is that the entire program
spelled out here is effective. We have built various interpreters for
the reflective calculus at work in this interpretation. In principle,
then, we can simulate quantum mechanics on a computer. The place where
the simulation may lose fidelity is the infinitely branching summation
for the annihilator.

In this connection i also want to point out that the evaluation style
calculation of the inner product puts the non-determinism of the
summation right at the heart of measurement. This suggests that
Milner's original reduction-based formulation of the dynamics of his
calculi in terms of sums was not just notationally suggestive of a
notion of measure-and-continue but captured some significant part of
the physics.

\paragraph{Quantum continuations}
In light of this last observation i want to point out that the
predominant account of quantum mechanics is missing a key aspect of a
truly compositional story of the physical situation. In a real lab,
when a measurement is made the observation can be made to feed into
another device that then makes another measurement conditioned on the
results of the first. This means that after the superposition was
collapsed the entire experimental set up remained in
superposition. While QM offers a means of writing this down it doesn't
quite line up well with the well-trodden formulation of computation
and continuation that we see so succinctly expressed in Milner's
calculi. This suggests that there might be advantages to this account
of dynamics waiting to be explored.

\paragraph{Quantum logic}
In this connection, we also note that by virtue of having the
Hennessy-Milner construction, we can pull the construction through the
interpretation of QM. This gives us a natural candidate for a quantum
logic that enjoys an extremely tight connection with it's domain of
interpretation, making the construction much less ad hoc (rather it is
the image of functor!).

\paragraph{Quantum probabiity}
i have questions about the basis of the interpretation of inner
product as probability amplitude. In particular, using which
axiomatization of probability theory does the notion of probability
amplitude earn the right to be so dubbed? In other words, where is the
proof that the operation for calculating a probability amplitude (and
then squaring) satisfies the axioms of what it means to calculate a
probability? Even if such a proof exists (i have yet to find it in the
literature), i wonder if it might not be possible to turn things on
their heads. Can we view the calculation of the probability amplitude
as an axiomatization of probability? If so, then the definition we
give for calculating probability amplitude may provide the basis for
an \emph{effective} theory of probability.

\paragraph{Quantum vs ``biological'' information}
Finally, i want to conclude with a more philosophical observation. At
a recent workshop in which QM was a predominant topic i noticed
something about quantum information. The speaker was giving a riveting
discussion of axiomatic QM and showing how properties of ``no
cloning'' and ``no deleting'' emerged as consequences of the
axiomatization. Theorems of this form are necessary to give us a sense
of confidence that our axioms characterize the physical theory. What
struck me, though, was that if quantum information is neither erasable
nor replicable it is markedly different from \emph{life}. Two of the
things we know about life is that

\begin{itemize}
  \item it ends;
  \item to gain some measure of persistence, to transcend it's
    finitude it is imminently copyable.
\end{itemize}

Both of these qualities are summarized succinctly in the aphorism: all
flesh is grass. For me these two kinds of ``information'' -- call them
quantum and biological -- are end points on a spectrum of strategies
for persistence. At one end, we have those curious entities that enjoy
uniqueness and permanence; at the other, we have those who in the face
of a certain end and an uncertain present make a go of passing
something on. To me one of the more remarkable aspects of the latter
strategy is that in the presence of noise (and certain features of
copying) we get a kind of dynamism, a chance for improvement against a
given persistent condition.

% subsection other_calculi_other_bisimulations_and_geometry_as_behavior (end)




% section conclusion (end)

%\documentclass[12pt]{llncs}
%\documentclass{jktr}

\usepackage[pdftex]{hyperref}                   
\usepackage {listings}
\usepackage {mathpartir}
\usepackage{bcprules}
%\usepackage{listings}
                       
\usepackage{graphicx} 
%\usepackage[margins=2.5cm,nohead,nofoot]{geometry}
%\usepackage{geometry}
\usepackage{amsfonts}
\usepackage{amstext}
\usepackage{latexsym}
\usepackage{amssymb}
\usepackage{color}


%\include{myPreamble}
\include{qm2pi.local} 

%\ifpdf
%\usepackage[pdftex]{graphicx}
%\else
%\usepackage{graphicx}
%\fi

 % \ifpdf
%  \usepackage{pdfsync}
%  \if


%\title{Brief Article}
%\author{David F. Snyder}
%\author{L.G. Meredith}

%\address{Dept. of Math., Texas State University--San Marcos, San Marcos, TX 78666}
       
\pagestyle{empty}


\begin{document}

\lstset{language=[Objective]Caml,frame=shadowbox}

\input{qm2pi.front}

% section front matter (end)

\input{qm2pi.intro} 
 
% section introduction (end)

% \input{qm2pi.knotations} 

% section notation (end)

\input{qm2pi.process.calculi} 

% section concurrent_process_calculi_and_spatial_logics_ (end)
    
%\input{qm2pi.knots2pi} 

%\input{qm2pi.trefoil} 

%\input{qm2pi.mainthm} 

% subsection basic_interpretation (end)

%\input{qm2pi.rho.presentation} 
\subsection{The syntax and semantics of the notation system}\label{sub:the_syntax_and_semantics_of_the_notation_system} % (fold)

We now summarize a technical presentation of the calculus that
embodies our theory of dynamics. The typical presentation of such a
calculus follows the style of giving generators and relations on
them. The grammar, below, describing term constructors, freely
generates the set of processes, $\Proc$. This set is then quotiented
by a relation known as structural congruence and it is over this set
that the notion of dynamics is expressed. This presentation is
essentially that of \cite{MeredithR05} with the addition of
polyadicity and summation. For readability we have relegated some of
the technical subtleties to an appendix.

\subsubsection{Process grammar}\label{subsub:process_grammar}

\begin{mathpar}
  \inferrule* [lab=synchronization] {} {{M} \bc \pzero \;|\; x?F \;|\; x!C }
  \and
  \inferrule* [lab=abstraction] {} {{F} \bc (x)P}
  \and
  \inferrule* [lab=concretion] {} {{C} \bc \langle Q \rangle}
  \and
  \inferrule* [lab=process] {} {{P,Q} \bc M \;| \;P|Q \;|\; @{x}}
  \and
  \inferrule* [lab=name] {} {{x} \bc \quotep{P}}
\end{mathpar} 

Note that $\vec{x}$ (resp. $\vec{P}$) denotes a vector of names
(resp. processes) of length $|\vec{x}|$ (resp. $|\vec{P}|$). We adopt
the following useful abbreviations.

\begin{mathpar}
   x?(\vec{y}).P := x.(\vec{y})P \and  x\clift{\vec{P}} := x.\clift{\vec{P}}
   \and x!(y) := \lift{x}{\dropn{y}}
   \and \Pi_{i=0}^{n-1}P_i := P_0 | \ldots | P_{n-1}
\end{mathpar}

\subsubsection{Structural congruence}

\paragraph{Free and bound names and alpha-equivalence.} At the
core of structural equivalence is alpha-equivalence which identifies
process that are the same up to a change of variable. Formally, we
recognize the distinction between free and bound names. The free names
of a process, $\freenames{P}$, may be calculated recursively as
follows:

\begin{mathpar}
\freenames{\pzero} := \emptyset
  \and \\
  \freenames{x?(y).P} := \{ x \} \cup (\freenames{P} \setminus \{ y \})
  \and 
  \freenames{x!\langle P \rangle} := \{ x \} \cup \{ P \} 
  \and \\
  \freenames{P|Q} := \freenames{P} \cup \freenames{Q}
  \and \\
  \freenames{@{x}} := \{ x \}
\end{mathpar}

$\pi$
$\quotep{\pi}$

$\freenames{-} : \pi \to \mathcal{P}(\quotep{\pi})$

\begin{eqnarray*}
  \freenames{\pzero} & := & \emptyset \\
  \freenames{x?(y).P} & := & \{ x \} \cup (\freenames{P} \setminus \{ y \}) \\
  \freenames{x!\langle P \rangle} & := & \{ x \} \cup \{ P \} \\
  \freenames{P|Q} & := & \freenames{P} \cup \freenames{Q} \\
  \freenames{\dropn{x}} & := & \{ x \}
\end{eqnarray*}

The bound names of a process, $\boundnames{P}$, are those names occurring in $P$
that are not free. For example, in $x?(y).0$, the name $x$ is free, while $y$ is bound.

\begin{mathpar}
  \inferrule* [lab=monoidal-laws] {} { P|Q \equiv Q|P \and P|0 \equiv P \and P|(Q|R) \equiv (P|Q)|R }
\end{mathpar}

\begin{mathpar}
  \inferrule* [lab=alpha-equivalence] {} { (x)P \equiv (y)P\{y/x\} \and y \not\in \freenames{P} }
\end{mathpar}

\begin{definition}
Then two processes, $P,Q$, are alpha-equivalent if $P = Q\{\vec{y}/\vec{x}\}$ for
some $\vec{x} \in \boundnames{Q},\vec{y} \in \boundnames{P}$, where $Q\{\vec{y}/\vec{x}\}$
denotes the capture-avoiding substitution of $\vec{y}$ for $\vec{x}$ in $Q$.
\end{definition}

\begin{definition}
  The {\em structural congruence} \cite{SangiorgiWalker} , $\equiv$,
  between processes is the least congruence containing
  alpha-equivalence, satisfying the abelian monoid laws
  (associativity, commutativity and $\pzero$ as identity) for parallel
  composition $|$ and for summation $+$.
\end{definition}

\subsection{Name equivalence}

We take name equivalence, written $\nameeq$, to be the smallest
equivalence relation generated by the following rules.

\begin{mathpar}
\inferrule*[lab=Quote-drop]
{ }
{ \quotep{@{x}} \nameeq x }

\inferrule*[lab=Struct-equiv]
{ P \scong Q }
{ \quotep{P} \nameeq \quotep{Q} }
\end{mathpar}

The astute reader will have noticed that the mutual recursion of names
and processes imposes a mutual recursion on alpha-equivalence and
structural equivalence via name-equivalence. Fortunately, all of this
works out pleasantly and we may calculate in the natural way, free of
concern. The reader interested in the details is referred to the
appendix \ref{appendix:rho_details}.

\subsection{Substitution}

We use $\Proc$ for the set of processes, $\QProc$ for the set of
names, and $\id{\{}\vec{y} / \vec{x} \id{\}}$ to denote partial maps,
$s : \QProc \rightarrow \QProc$. A map, $s$ lifts, uniquely, to a map
on process terms, $\widehat{s} : \Proc \rightarrow \Proc$ by the
following equations.

\begin{mathpar}
  (0) \psubstp{Q}{P} := 0 \\
  (R \juxtap S) \psubstp{Q}{P}
  :=    
  (R)\psubstp{Q}{P} \juxtap (S) \psubstp{Q}{P} \\
  (x?(y).R) \psubstp{Q}{P}    
  :=    
  (x)\substp{Q}{P} (z)\concat( (R \psubstn{z}{y}) \psubstp{Q}{P} ) \\
  (\lift{x}{R}) \psubstp{Q}{P}  
  :=
  \lift{(x)\substp{Q}{P}}{ R \psubstp{Q}{P} } \\
%   (\dropn{x})  \psubstp{Q}{P}       
%   := 
%   \left\{ 
%     \begin{array}{ccc} 
%       \dropn{\quotep{Q}} & & x \nameeq \quotep{P} \\
%       \dropn{x} & & otherwise \\
%     \end{array}
%   \right. 
  (\dropn{x})  \psubstp{Q}{P}       
  := 
  \left\{ 
    \begin{array}{ccc} 
      Q & & x \nameeq \quotep{P} \\
      \dropn{x} & & otherwise \\
    \end{array}
  \right.
\end{mathpar}
 

where

\begin{eqnarray}
  (x)\id{\{} \lpquote Q \rpquote / \lpquote P \rpquote \id{\}}            = 
  \left\{ 
    \begin{array}{ccc}
      \lpquote Q \rpquote & & x \nameeq \lpquote P \rpquote \\
      x & & otherwise \\
    \end{array}
  \right. \nonumber
\end{eqnarray}

and $z$ is chosen distinct from $\quotep{P}$, $\quotep{Q}$, the free
names in $Q$, and all the names in $R$. Our $\alpha$-equivalence will
be built in the standard way from this substitution.

\begin{remark}\label{rem:no_self_referential_names}
  One consequence of these definitions is that $\forall P. \quotep{P}
  \not\in \freenames{P}$.
\end{remark}

\subsection{ Dynamic quote: an example }

Anticipating something of what's to come, consider applying the
substitution, $\widehat{\id{\{}u / z \id{\}}}$, to the following pair
of processes, $\lift{w}{y!(z)}$ and $w[ \lpquote y!(z) \rpquote ]$.

\begin{eqnarray}
	\lift{w}{y!(z)}\widehat{\id{\{}u / z \id{\}}}
		& = &
		\lift{w}{y!(u)} \nonumber\\
	w[ \lpquote y!(z) \rpquote ] \widehat{ \id{\{}u / z \id{\}} }
		& = &
		w[ \lpquote y!(z) \rpquote ] \nonumber
\end{eqnarray}

Because the body of the process between quotes is impervious to
substitution, we get radically different answers. In fact, by
examining the first process in an input context,
e.g. $x?(z).\lift{w}{y!(z)}$, we see that the process under the lift
operator may be shaped by prefixed inputs binding a name inside it. In
this sense, the lift operator will be seen as a way to dynamically
construct processes before reifying them as names.

Finally equipped with these standard features we can present the
dynamics of the calculus.

\subsubsection{Operational semantics} 

Finally, we introduce the computational dynamics. What marks these
algebras as distinct from other more traditionally studied algebraic
structures, e.g. vector spaces or polynomial rings, is the manner in
which dynamics is captured. In traditional structures, dynamics is typically
expressed through morphisms between such structures, as in linear maps
between vector spaces or morphisms between rings. In algebras
associated with the semantics of computation, the dynamics is
expressed as part of the algebraic structure itself, through a
reduction reduction relation typically denoted by $\red$. Below, we
give a recursive presentation of this relation for the calculus used
in the encoding.

$\red \subseteq \pi \times \pi$
$\red : \pi \to \mathcal{P}(\pi)$

\begin{mathpar}
  \inferrule* [lab=Comm] { \textsf{match}( x_{src}, x_{trgt} ) } { x_{trgt}?(y)P \; | \; x_{src}!\langle {Q} \rangle \red P\{\quotep{Q}/y}\} }
  \and \\
  \inferrule* [lab=Par] {{P} \red {P}'} {{{P} | {Q}} \red {{P}' | {Q}}}
  \and
  \inferrule* [lab=Equiv]{{{P} \scong {P}'} \andalso {{P}' \red {Q}'} \andalso {{Q}' \scong {Q}}}{{P} \red {Q}}
\end{mathpar}

\begin{eqnarray*}
  match_{\equiv} (\quotep{P},\quotep{Q}) & := & P \equiv Q \\
  match_{\dagger}(\quotep{P},\quotep{Q}) & := & \forall R. P|Q \red^{*} R => R \red^{*} 0 \\
  match_{K}(\quotep{P},\quotep{Q}) & := & K \mbox{ for some context } K
\end{eqnarray*}

$u?(x)P | u!\langle Q \rangle \red P\{\quotep{Q}/x\}$

%We write $\wred$ for $\red^*$, and $P\red$ if $\exists Q $ such that $ P \red Q$.
We write $P\red$ if $\exists Q $ such that $ P \red Q$ and $P\not\red$, otherwise.

\section{Replication}

As mentioned before, it is known that replication (and hence
recursion) can be implemented in a higher-order process algebra
\cite{SangiorgiWalker}. As our first example of calculation with the
machinery thus far presented we give the construction explicitly in
the {\rhoc}.

\begin{eqnarray}
	D_{x} & := & \prefix{x}{y}{(\binpar{\outputp{x}{y}}{@{y}})} \nonumber\\
	\bangp_{x}{P} & := & \binpar{{x}!\langle{\binpar{D_{x}}{P}}\rangle}{D_{x}} \nonumber
\end{eqnarray}

\begin{eqnarray}
	\bangp_{x}{P} & & \nonumber\\
	=
	& {x}!\langle{(\prefix{x}{y}{(\outputp{x}{y} | @{y})) | P}}\rangle 
	      | \prefix{x}{y}{(\outputp{x}{y} | @{y})} & \nonumber\\
	\red
	& (\outputp{x}{y} | @{y})\substn{\quotep{(\prefix{x}{y}{(@{y} | \outputp{x}{y})) | P}}}{y} & \nonumber\\
	=
	& \outputp{x}{\quotep{(\prefix{x}{y}{(\outputp{x}{y} | @{y})) | P}}}
	  | {(\prefix{x}{y}{(\outputp{x}{y} | @{y})) | P}} & \nonumber\\
	\red
	& \ldots & \nonumber\\
	\red^*
	& P | P | \ldots & \nonumber
\end{eqnarray}

Of course, this encoding, as an implementation, runs away, unfolding
$\bangp{P}$ eagerly. A lazier and more implementable replication
operator, restricted to input-guarded processes, may be obtained as follows.

\begin{eqnarray}
\bangp{\prefix{u}{v}{P}} 
	:= 
	\binpar{\lift{x}{\prefix{u}{v}{(\binpar{D(x)}{P})}}}{D(x)} \nonumber
\end{eqnarray}

\begin{remark}
  Note that the lazier definition still does not deal with summation
  or mixed summation (i.e. sums over input and output). The reader is
  invited to construct definitions of replication that deal with these
  features. 

  Further, the definitions are parameterized in a name, $x$. Can you,
  gentle reader, make a definition that eliminates this parameter and
  guarantees no accidental interaction between the replication
  machinery and the process being replicated -- i.e. no accidental
  sharing of names used by the process to get its work done and the
  name(s) used by the replication to effect copying. This latter
  revision of the definition of replication is crucial to obtaining
  the expected identity $!!P \sim !P$.
\end{remark}

\begin{remark}\label{rem:paradoxical_combinator}
  The reader familiar with the lambda calculus will have noticed the
  similarity between $D$ and the paradoxical combinator.

  [Ed. note: the existence of this seems to suggest we have to be more
  restrictive on the set of processes and names we admit if we are to
  support no-cloning.]
\end{remark}

\subsubsection{Bisimulation}

The computational dynamics gives rise to another kind of equivalence,
the equivalence of computational behavior. As previously mentioned
this is typically captured \emph{via} some form of bisimulation.

% The notion we use in this paper is weak barbed bisimulation
% \cite{milner91polyadicpi}.

The notion we use in this paper is derived from weak barbed
bisimulation \cite{milner91polyadicpi}. 

\begin{definition}
An \emph{observation relation}, $\downarrow_{\mathcal N}$, over a set
of names, $\mathcal N$, is the smallest relation satisfying the rules
below.

\infrule[Out-barb]{y \in {\mathcal N}, \; x \nameeq y}
		  {\outputp{x}{v} \downarrow_{\mathcal N} x}
\infrule[Par-barb]{\mbox{$P\downarrow_{\mathcal N} x$ or $Q\downarrow_{\mathcal N} x$}}
		  {\binpar{P}{Q} \downarrow_{\mathcal N} x}

We write $P \Downarrow_{\mathcal N} x$ if there is $Q$ such that 
$P \wred Q$ and $Q \downarrow_{\mathcal N} x$.
\end{definition}

\begin{definition}
%\label{def.bbisim}
An  ${\mathcal N}$-\emph{barbed bisimulation} over a set of names, ${\mathcal N}$, is a symmetric binary relation 
${\mathcal S}_{\mathcal N}$ between agents such that $P\rel{S}_{\mathcal N}Q$ implies:
\begin{enumerate}
\item If $P \red P'$ then $Q \wred Q'$ and $P'\rel{S}_{\mathcal N} Q'$.
\item If $P\downarrow_{\mathcal N} x$, then $Q\Downarrow_{\mathcal N} x$.
\end{enumerate}
$P$ is ${\mathcal N}$-barbed bisimilar to $Q$, written
$P \wbbisim_{\mathcal N} Q$, if $P \rel{S}_{\mathcal N} Q$ for some ${\mathcal N}$-barbed bisimulation ${\mathcal S}_{\mathcal N}$.
\end{definition}

$\mathcal{R} \subseteq \pi \times \pi$

$P \mathcal{R} Q => \forall P'. P \red P' \Rightarrow \exists Q'. Q \red Q', P' \mathcal{R} Q'$

$P \vdash x \Rightarrow Q \vdash x$

\begin{mathpar}
  \inferrule*[lab=Out-barb]{x \nameeq y}{{y}!\langle{Q}\rangle \vdash x}
  \and
  \inferrule*[lab=Par-barb]{\mbox{$P\vdash x$ or $Q\vdash x$}}{\binpar{P}{Q} \vdash x}
\end{mathpar}

\subsubsection{Contexts}

One of the principle advantages of computational calculi like the
$\pi$-calculus is a well-defined notion of context,
contextual-equivalence and a correlation between
contextual-equivalence and notions of bisimulation. The notion of
context allows the decomposition of a process into (sub-)process and
its syntactic environment, its context. Thus, a context may be
thought of as a process with a ``hole'' (written $\Box$) in it. The
application of a context $M$ to a process $P$, written $M[P]$, is
tantamount to filling the hole in $M$ with $P$. In this paper we do
not need the full weight of this theory, but do make use of the notion
of context in the proof the main theorem. 

\begin{mathpar}
  \inferrule* [lab=summation] {} {{M_{M},M_{N}} \bc \Box \;|\; x.M_{A} \;|\; M_{M}+M_{N}}
  \and
  \inferrule* [lab=agent] {} {{M_{A}} \bc (\vec{x})M_{P} \;| \; \clift{P_0,\ldots,M_{P},\ldots,P_N}}
  \and \\
  \inferrule* [lab=process] {} {{M_{P}} \bc M_{N} \;| \;P|M_{P} }
\end{mathpar} 

\begin{mathpar}
  \inferrule* [lab=sychronization] {} {M_{N} \bc \Box \;|\; x?M_{F} \;|\; x!M_{C}}
  \and
  \inferrule* [lab=abstraction] {} {{M_{F}} \bc (x)M_{P} }
  \and
  \inferrule* [lab=concretion] {} {{M_{C}} \bc \langle M_{P} \rangle }
  \and \\
  \inferrule* [lab=process] {} {{M_{P}} \bc M_{N} \;| \;P|M_{P} }
\end{mathpar}

\begin{definition}[contextual application] Given a context $M$, and
  process $P$, we define the \emph{contextual application}, $M[P] :=
  M\{P/\Box\}$. That is, the contextual application of M to P is the
  substitution of $P$ for $\Box$ in $M$.
\end{definition}

$\meaningof{-} : L \to \mathcal{P}(\pi)$

\begin{mathpar}
  \inferrule* [lab=collection] {} {\meaningof{true} = \pi, \and \meaningof{~E} = \pi \setminus \meaningof{E}, \and \meaningof{E_{1} \& E_{2}} = \meaningof{E_{1}} \cap \meaningof{E_{2}}}
\end{mathpar}

\begin{mathpar}
  \inferrule* [lab=structure] {} {\meaningof{0} = \{ P \in \pi | P \equiv 0 \}, \and \\ \meaningof{E_1 | E_2} = \{ P \in \pi | P \equiv P_{1} | P_{2}, P_{1} \in \meaningof{E_{1}}, P_{2} \in \meaningof{E_2}\} }
\end{mathpar}

\begin{mathpar}
 \inferrule* [lab=behavior] {} {\meaningof{\langle a?b \rangle E} = \{ P \in \pi | P \equiv Q | u?(y)P', \\ \and \\\\ \and \\ \;\;\; u \in \meaningof{a}, \forall z.P'\{z/y\} \in \meaningof{E\{z/b\}}\}, \and \\ \meaningof{a!E} = \{ P \in \pi | P \equiv Q | x!\langle P' \rangle, x \in \meaningof{a} P' \in \meaningof{E}\} }
\end{mathpar}

\begin{mathpar}
 \inferrule* [lab=nominal] {} {\meaningof{\quotep{E}} = \{ \quotep{P} \in \quotep{\pi} | P \in \meaningof{E} \}, \and \meaningof{\quotep{P}} = \{ \quotep{Q} \in \quotep{\pi} | P \equiv Q \} \and \\ \meaningof{@\quotep{E}} = \{ P \in \pi | P \equiv @x, x \in \meaningof{E} \}}
\end{mathpar}

\begin{eqnarray*}
  \\
  \meaningof{-} : TS \to ST
\end{eqnarray*}

\begin{eqnarray*}
  \\
  L : TS \to ST
\end{eqnarray*}

\begin{eqnarray*}
  \\
  P \models E \iff P \in \meaningof{E}
\end{eqnarray*}

\begin{eqnarray*}
  P \approx_{L} Q \iff \forall E \in L. P \models E \iff Q \models E
\end{eqnarray*}

\begin{eqnarray*}
  P \approx_{K} Q
\end{eqnarray*}

\begin{eqnarray*}
  P \approx Q
\end{eqnarray*}

$\approx_{K} = \approx = \approx_{L}$

\subsubsection{Contextual duality}

Note that contexts extend the quotation operation to a family of
operations from processes to names. Given a context, $M$, we can
define a \emph{nominal context}, $\quotep{M}$ by $\quotep{M}[P] :=
\quotep{M[P]}$. To foreshadow what is to come we observe that these
operations enjoy a duality with processes very much like the duality
between vectors and maps from vectors to scalars.

Further, because the calculus is essentially higher-order, we have a
correspondence between contexts and processes. More specifically,
given a name $x$ and a context $M$ we can construct $M^{*}_{x}$ such
that 

\begin{mathpar}
  M^{*}_{x} | \lift{x}{P} \red M[P]
\end{mathpar}

namely,

\begin{mathpar}
  M^{*}_{x} := x?(u).M[\dropn{u}]
\end{mathpar}

The dependence of $M^{*}_{x}$ on a name makes it an abstraction, 

\begin{mathpar}
  M^{*} := (x)x?(u).M[\dropn{u}]
\end{mathpar}

\subsection{Additional notation}

It will sometimes be convenient to denote the process a name
quotes. We already have the notation $x = \quotep{P}$, but it will be
convenient to introduce an alternate notation, $\procn{x}$, when we
want to emphasize the connection to the use of the name. Note that, by
virtue of name equivalence, $\quotep{\procn{x}} \nameeq x$; so, the
notation is consistent with previous definitions.

Further, because names have structure it is possible to effect
substitutions on the basis of that structure. This means we need to
upgrade our notation for substitutions, which we accomplish by
adapting comprehension notation. Thus,

\begin{mathpar}
  P\{ y / x : x \in S \}
\end{mathpar}

is interpreted to mean the process derived from P by replacing (in a
capture-avoiding manner) each occurrence of $x$ in $S$ by $y$. For example,

\begin{mathpar}
  P\{ \quotep{\procn{x}|\procn{x}} / x : x \in \freenames{P} \}
\end{mathpar}

will replace each (occurrence) of a free name $x$ in $P$ by
$\quotep{\procn{x}|\procn{x}}$.

Also, we will avail ourselves of the notation $x^{L}$ and $x^{R}$ to
denote injections of a name into disjoint copies of the name
space. There are numerous ways to accomplish this. One example can be
found in \cite{MeredithR05}. This notation overloads to vectors of
names: $\vec{x}^{\pi} := (x_{i}^{\pi} \; : \; 0 \leq i < |\vec{x}| )$ where $\pi \in \{L,R\}$.

We also use $P^{\Box} := P|\Box$.

In \cite{MeredithR05} an interpretation of the new operator is
given. It turns out that there are several possible interpretations
all enjoying the requisite algebraic properties of the operator (see
\cite{milner91polyadicpi}). We will therefore make liberal use of
$(\nu\; \vec{x})P$.

% subsection the_syntax_and_semantics_of_the_notation_system (end)   

\input{qm2pi.qmops} 

\input{qm2pi.sterngerlach} 

\input{qm2pi.metric} 

% section concurrent_process_calculi (end)

%\input{qm2pi.proofsketch}

% section proof sketch (end)

%\input{qm2pi.slviaknots} 

% section spatial logic via knots (end)

\input{qm2pi.conclusion}

% section conclusion (end)

%\input{qm2pi.dtcodes} 

% section wiring algorithm (end)

\input{qm2pi.ack} 

% section acknowledgments (end)

\newpage


\bibliographystyle{plain}   
\bibliography{../../biblios/main.bib}

\input{qm2pi.rhodetails}

\end{document}

 

% section wiring algorithm (end)

\documentclass[12pt]{llncs}
%\documentclass{jktr}

\usepackage[pdftex]{hyperref}                   
\usepackage {listings}
\usepackage {mathpartir}
\usepackage{bcprules}
%\usepackage{listings}
                       
\usepackage{graphicx} 
%\usepackage[margins=2.5cm,nohead,nofoot]{geometry}
%\usepackage{geometry}
\usepackage{amsfonts}
\usepackage{amstext}
\usepackage{latexsym}
\usepackage{amssymb}
\usepackage{color}


%\include{myPreamble}
\include{qm2pi.local} 

%\ifpdf
%\usepackage[pdftex]{graphicx}
%\else
%\usepackage{graphicx}
%\fi

 % \ifpdf
%  \usepackage{pdfsync}
%  \if


%\title{Brief Article}
%\author{David F. Snyder}
%\author{L.G. Meredith}

%\address{Dept. of Math., Texas State University--San Marcos, San Marcos, TX 78666}
       
\pagestyle{empty}


\begin{document}

\lstset{language=[Objective]Caml,frame=shadowbox}

\input{qm2pi.front}

% section front matter (end)

\input{qm2pi.intro} 
 
% section introduction (end)

% \input{qm2pi.knotations} 

% section notation (end)

\input{qm2pi.process.calculi} 

% section concurrent_process_calculi_and_spatial_logics_ (end)
    
%\input{qm2pi.knots2pi} 

%\input{qm2pi.trefoil} 

%\input{qm2pi.mainthm} 

% subsection basic_interpretation (end)

%\input{qm2pi.rho.presentation} 
\subsection{The syntax and semantics of the notation system}\label{sub:the_syntax_and_semantics_of_the_notation_system} % (fold)

We now summarize a technical presentation of the calculus that
embodies our theory of dynamics. The typical presentation of such a
calculus follows the style of giving generators and relations on
them. The grammar, below, describing term constructors, freely
generates the set of processes, $\Proc$. This set is then quotiented
by a relation known as structural congruence and it is over this set
that the notion of dynamics is expressed. This presentation is
essentially that of \cite{MeredithR05} with the addition of
polyadicity and summation. For readability we have relegated some of
the technical subtleties to an appendix.

\subsubsection{Process grammar}\label{subsub:process_grammar}

\begin{mathpar}
  \inferrule* [lab=synchronization] {} {{M} \bc \pzero \;|\; x?F \;|\; x!C }
  \and
  \inferrule* [lab=abstraction] {} {{F} \bc (x)P}
  \and
  \inferrule* [lab=concretion] {} {{C} \bc \langle Q \rangle}
  \and
  \inferrule* [lab=process] {} {{P,Q} \bc M \;| \;P|Q \;|\; @{x}}
  \and
  \inferrule* [lab=name] {} {{x} \bc \quotep{P}}
\end{mathpar} 

Note that $\vec{x}$ (resp. $\vec{P}$) denotes a vector of names
(resp. processes) of length $|\vec{x}|$ (resp. $|\vec{P}|$). We adopt
the following useful abbreviations.

\begin{mathpar}
   x?(\vec{y}).P := x.(\vec{y})P \and  x\clift{\vec{P}} := x.\clift{\vec{P}}
   \and x!(y) := \lift{x}{\dropn{y}}
   \and \Pi_{i=0}^{n-1}P_i := P_0 | \ldots | P_{n-1}
\end{mathpar}

\subsubsection{Structural congruence}

\paragraph{Free and bound names and alpha-equivalence.} At the
core of structural equivalence is alpha-equivalence which identifies
process that are the same up to a change of variable. Formally, we
recognize the distinction between free and bound names. The free names
of a process, $\freenames{P}$, may be calculated recursively as
follows:

\begin{mathpar}
\freenames{\pzero} := \emptyset
  \and \\
  \freenames{x?(y).P} := \{ x \} \cup (\freenames{P} \setminus \{ y \})
  \and 
  \freenames{x!\langle P \rangle} := \{ x \} \cup \{ P \} 
  \and \\
  \freenames{P|Q} := \freenames{P} \cup \freenames{Q}
  \and \\
  \freenames{@{x}} := \{ x \}
\end{mathpar}

$\pi$
$\quotep{\pi}$

$\freenames{-} : \pi \to \mathcal{P}(\quotep{\pi})$

\begin{eqnarray*}
  \freenames{\pzero} & := & \emptyset \\
  \freenames{x?(y).P} & := & \{ x \} \cup (\freenames{P} \setminus \{ y \}) \\
  \freenames{x!\langle P \rangle} & := & \{ x \} \cup \{ P \} \\
  \freenames{P|Q} & := & \freenames{P} \cup \freenames{Q} \\
  \freenames{\dropn{x}} & := & \{ x \}
\end{eqnarray*}

The bound names of a process, $\boundnames{P}$, are those names occurring in $P$
that are not free. For example, in $x?(y).0$, the name $x$ is free, while $y$ is bound.

\begin{mathpar}
  \inferrule* [lab=monoidal-laws] {} { P|Q \equiv Q|P \and P|0 \equiv P \and P|(Q|R) \equiv (P|Q)|R }
\end{mathpar}

\begin{mathpar}
  \inferrule* [lab=alpha-equivalence] {} { (x)P \equiv (y)P\{y/x\} \and y \not\in \freenames{P} }
\end{mathpar}

\begin{definition}
Then two processes, $P,Q$, are alpha-equivalent if $P = Q\{\vec{y}/\vec{x}\}$ for
some $\vec{x} \in \boundnames{Q},\vec{y} \in \boundnames{P}$, where $Q\{\vec{y}/\vec{x}\}$
denotes the capture-avoiding substitution of $\vec{y}$ for $\vec{x}$ in $Q$.
\end{definition}

\begin{definition}
  The {\em structural congruence} \cite{SangiorgiWalker} , $\equiv$,
  between processes is the least congruence containing
  alpha-equivalence, satisfying the abelian monoid laws
  (associativity, commutativity and $\pzero$ as identity) for parallel
  composition $|$ and for summation $+$.
\end{definition}

\subsection{Name equivalence}

We take name equivalence, written $\nameeq$, to be the smallest
equivalence relation generated by the following rules.

\begin{mathpar}
\inferrule*[lab=Quote-drop]
{ }
{ \quotep{@{x}} \nameeq x }

\inferrule*[lab=Struct-equiv]
{ P \scong Q }
{ \quotep{P} \nameeq \quotep{Q} }
\end{mathpar}

The astute reader will have noticed that the mutual recursion of names
and processes imposes a mutual recursion on alpha-equivalence and
structural equivalence via name-equivalence. Fortunately, all of this
works out pleasantly and we may calculate in the natural way, free of
concern. The reader interested in the details is referred to the
appendix \ref{appendix:rho_details}.

\subsection{Substitution}

We use $\Proc$ for the set of processes, $\QProc$ for the set of
names, and $\id{\{}\vec{y} / \vec{x} \id{\}}$ to denote partial maps,
$s : \QProc \rightarrow \QProc$. A map, $s$ lifts, uniquely, to a map
on process terms, $\widehat{s} : \Proc \rightarrow \Proc$ by the
following equations.

\begin{mathpar}
  (0) \psubstp{Q}{P} := 0 \\
  (R \juxtap S) \psubstp{Q}{P}
  :=    
  (R)\psubstp{Q}{P} \juxtap (S) \psubstp{Q}{P} \\
  (x?(y).R) \psubstp{Q}{P}    
  :=    
  (x)\substp{Q}{P} (z)\concat( (R \psubstn{z}{y}) \psubstp{Q}{P} ) \\
  (\lift{x}{R}) \psubstp{Q}{P}  
  :=
  \lift{(x)\substp{Q}{P}}{ R \psubstp{Q}{P} } \\
%   (\dropn{x})  \psubstp{Q}{P}       
%   := 
%   \left\{ 
%     \begin{array}{ccc} 
%       \dropn{\quotep{Q}} & & x \nameeq \quotep{P} \\
%       \dropn{x} & & otherwise \\
%     \end{array}
%   \right. 
  (\dropn{x})  \psubstp{Q}{P}       
  := 
  \left\{ 
    \begin{array}{ccc} 
      Q & & x \nameeq \quotep{P} \\
      \dropn{x} & & otherwise \\
    \end{array}
  \right.
\end{mathpar}
 

where

\begin{eqnarray}
  (x)\id{\{} \lpquote Q \rpquote / \lpquote P \rpquote \id{\}}            = 
  \left\{ 
    \begin{array}{ccc}
      \lpquote Q \rpquote & & x \nameeq \lpquote P \rpquote \\
      x & & otherwise \\
    \end{array}
  \right. \nonumber
\end{eqnarray}

and $z$ is chosen distinct from $\quotep{P}$, $\quotep{Q}$, the free
names in $Q$, and all the names in $R$. Our $\alpha$-equivalence will
be built in the standard way from this substitution.

\begin{remark}\label{rem:no_self_referential_names}
  One consequence of these definitions is that $\forall P. \quotep{P}
  \not\in \freenames{P}$.
\end{remark}

\subsection{ Dynamic quote: an example }

Anticipating something of what's to come, consider applying the
substitution, $\widehat{\id{\{}u / z \id{\}}}$, to the following pair
of processes, $\lift{w}{y!(z)}$ and $w[ \lpquote y!(z) \rpquote ]$.

\begin{eqnarray}
	\lift{w}{y!(z)}\widehat{\id{\{}u / z \id{\}}}
		& = &
		\lift{w}{y!(u)} \nonumber\\
	w[ \lpquote y!(z) \rpquote ] \widehat{ \id{\{}u / z \id{\}} }
		& = &
		w[ \lpquote y!(z) \rpquote ] \nonumber
\end{eqnarray}

Because the body of the process between quotes is impervious to
substitution, we get radically different answers. In fact, by
examining the first process in an input context,
e.g. $x?(z).\lift{w}{y!(z)}$, we see that the process under the lift
operator may be shaped by prefixed inputs binding a name inside it. In
this sense, the lift operator will be seen as a way to dynamically
construct processes before reifying them as names.

Finally equipped with these standard features we can present the
dynamics of the calculus.

\subsubsection{Operational semantics} 

Finally, we introduce the computational dynamics. What marks these
algebras as distinct from other more traditionally studied algebraic
structures, e.g. vector spaces or polynomial rings, is the manner in
which dynamics is captured. In traditional structures, dynamics is typically
expressed through morphisms between such structures, as in linear maps
between vector spaces or morphisms between rings. In algebras
associated with the semantics of computation, the dynamics is
expressed as part of the algebraic structure itself, through a
reduction reduction relation typically denoted by $\red$. Below, we
give a recursive presentation of this relation for the calculus used
in the encoding.

$\red \subseteq \pi \times \pi$
$\red : \pi \to \mathcal{P}(\pi)$

\begin{mathpar}
  \inferrule* [lab=Comm] { \textsf{match}( x_{src}, x_{trgt} ) } { x_{trgt}?(y)P \; | \; x_{src}!\langle {Q} \rangle \red P\{\quotep{Q}/y}\} }
  \and \\
  \inferrule* [lab=Par] {{P} \red {P}'} {{{P} | {Q}} \red {{P}' | {Q}}}
  \and
  \inferrule* [lab=Equiv]{{{P} \scong {P}'} \andalso {{P}' \red {Q}'} \andalso {{Q}' \scong {Q}}}{{P} \red {Q}}
\end{mathpar}

\begin{eqnarray*}
  match_{\equiv} (\quotep{P},\quotep{Q}) & := & P \equiv Q \\
  match_{\dagger}(\quotep{P},\quotep{Q}) & := & \forall R. P|Q \red^{*} R => R \red^{*} 0 \\
  match_{K}(\quotep{P},\quotep{Q}) & := & K \mbox{ for some context } K
\end{eqnarray*}

$u?(x)P | u!\langle Q \rangle \red P\{\quotep{Q}/x\}$

%We write $\wred$ for $\red^*$, and $P\red$ if $\exists Q $ such that $ P \red Q$.
We write $P\red$ if $\exists Q $ such that $ P \red Q$ and $P\not\red$, otherwise.

\section{Replication}

As mentioned before, it is known that replication (and hence
recursion) can be implemented in a higher-order process algebra
\cite{SangiorgiWalker}. As our first example of calculation with the
machinery thus far presented we give the construction explicitly in
the {\rhoc}.

\begin{eqnarray}
	D_{x} & := & \prefix{x}{y}{(\binpar{\outputp{x}{y}}{@{y}})} \nonumber\\
	\bangp_{x}{P} & := & \binpar{{x}!\langle{\binpar{D_{x}}{P}}\rangle}{D_{x}} \nonumber
\end{eqnarray}

\begin{eqnarray}
	\bangp_{x}{P} & & \nonumber\\
	=
	& {x}!\langle{(\prefix{x}{y}{(\outputp{x}{y} | @{y})) | P}}\rangle 
	      | \prefix{x}{y}{(\outputp{x}{y} | @{y})} & \nonumber\\
	\red
	& (\outputp{x}{y} | @{y})\substn{\quotep{(\prefix{x}{y}{(@{y} | \outputp{x}{y})) | P}}}{y} & \nonumber\\
	=
	& \outputp{x}{\quotep{(\prefix{x}{y}{(\outputp{x}{y} | @{y})) | P}}}
	  | {(\prefix{x}{y}{(\outputp{x}{y} | @{y})) | P}} & \nonumber\\
	\red
	& \ldots & \nonumber\\
	\red^*
	& P | P | \ldots & \nonumber
\end{eqnarray}

Of course, this encoding, as an implementation, runs away, unfolding
$\bangp{P}$ eagerly. A lazier and more implementable replication
operator, restricted to input-guarded processes, may be obtained as follows.

\begin{eqnarray}
\bangp{\prefix{u}{v}{P}} 
	:= 
	\binpar{\lift{x}{\prefix{u}{v}{(\binpar{D(x)}{P})}}}{D(x)} \nonumber
\end{eqnarray}

\begin{remark}
  Note that the lazier definition still does not deal with summation
  or mixed summation (i.e. sums over input and output). The reader is
  invited to construct definitions of replication that deal with these
  features. 

  Further, the definitions are parameterized in a name, $x$. Can you,
  gentle reader, make a definition that eliminates this parameter and
  guarantees no accidental interaction between the replication
  machinery and the process being replicated -- i.e. no accidental
  sharing of names used by the process to get its work done and the
  name(s) used by the replication to effect copying. This latter
  revision of the definition of replication is crucial to obtaining
  the expected identity $!!P \sim !P$.
\end{remark}

\begin{remark}\label{rem:paradoxical_combinator}
  The reader familiar with the lambda calculus will have noticed the
  similarity between $D$ and the paradoxical combinator.

  [Ed. note: the existence of this seems to suggest we have to be more
  restrictive on the set of processes and names we admit if we are to
  support no-cloning.]
\end{remark}

\subsubsection{Bisimulation}

The computational dynamics gives rise to another kind of equivalence,
the equivalence of computational behavior. As previously mentioned
this is typically captured \emph{via} some form of bisimulation.

% The notion we use in this paper is weak barbed bisimulation
% \cite{milner91polyadicpi}.

The notion we use in this paper is derived from weak barbed
bisimulation \cite{milner91polyadicpi}. 

\begin{definition}
An \emph{observation relation}, $\downarrow_{\mathcal N}$, over a set
of names, $\mathcal N$, is the smallest relation satisfying the rules
below.

\infrule[Out-barb]{y \in {\mathcal N}, \; x \nameeq y}
		  {\outputp{x}{v} \downarrow_{\mathcal N} x}
\infrule[Par-barb]{\mbox{$P\downarrow_{\mathcal N} x$ or $Q\downarrow_{\mathcal N} x$}}
		  {\binpar{P}{Q} \downarrow_{\mathcal N} x}

We write $P \Downarrow_{\mathcal N} x$ if there is $Q$ such that 
$P \wred Q$ and $Q \downarrow_{\mathcal N} x$.
\end{definition}

\begin{definition}
%\label{def.bbisim}
An  ${\mathcal N}$-\emph{barbed bisimulation} over a set of names, ${\mathcal N}$, is a symmetric binary relation 
${\mathcal S}_{\mathcal N}$ between agents such that $P\rel{S}_{\mathcal N}Q$ implies:
\begin{enumerate}
\item If $P \red P'$ then $Q \wred Q'$ and $P'\rel{S}_{\mathcal N} Q'$.
\item If $P\downarrow_{\mathcal N} x$, then $Q\Downarrow_{\mathcal N} x$.
\end{enumerate}
$P$ is ${\mathcal N}$-barbed bisimilar to $Q$, written
$P \wbbisim_{\mathcal N} Q$, if $P \rel{S}_{\mathcal N} Q$ for some ${\mathcal N}$-barbed bisimulation ${\mathcal S}_{\mathcal N}$.
\end{definition}

$\mathcal{R} \subseteq \pi \times \pi$

$P \mathcal{R} Q => \forall P'. P \red P' \Rightarrow \exists Q'. Q \red Q', P' \mathcal{R} Q'$

$P \vdash x \Rightarrow Q \vdash x$

\begin{mathpar}
  \inferrule*[lab=Out-barb]{x \nameeq y}{{y}!\langle{Q}\rangle \vdash x}
  \and
  \inferrule*[lab=Par-barb]{\mbox{$P\vdash x$ or $Q\vdash x$}}{\binpar{P}{Q} \vdash x}
\end{mathpar}

\subsubsection{Contexts}

One of the principle advantages of computational calculi like the
$\pi$-calculus is a well-defined notion of context,
contextual-equivalence and a correlation between
contextual-equivalence and notions of bisimulation. The notion of
context allows the decomposition of a process into (sub-)process and
its syntactic environment, its context. Thus, a context may be
thought of as a process with a ``hole'' (written $\Box$) in it. The
application of a context $M$ to a process $P$, written $M[P]$, is
tantamount to filling the hole in $M$ with $P$. In this paper we do
not need the full weight of this theory, but do make use of the notion
of context in the proof the main theorem. 

\begin{mathpar}
  \inferrule* [lab=summation] {} {{M_{M},M_{N}} \bc \Box \;|\; x.M_{A} \;|\; M_{M}+M_{N}}
  \and
  \inferrule* [lab=agent] {} {{M_{A}} \bc (\vec{x})M_{P} \;| \; \clift{P_0,\ldots,M_{P},\ldots,P_N}}
  \and \\
  \inferrule* [lab=process] {} {{M_{P}} \bc M_{N} \;| \;P|M_{P} }
\end{mathpar} 

\begin{mathpar}
  \inferrule* [lab=sychronization] {} {M_{N} \bc \Box \;|\; x?M_{F} \;|\; x!M_{C}}
  \and
  \inferrule* [lab=abstraction] {} {{M_{F}} \bc (x)M_{P} }
  \and
  \inferrule* [lab=concretion] {} {{M_{C}} \bc \langle M_{P} \rangle }
  \and \\
  \inferrule* [lab=process] {} {{M_{P}} \bc M_{N} \;| \;P|M_{P} }
\end{mathpar}

\begin{definition}[contextual application] Given a context $M$, and
  process $P$, we define the \emph{contextual application}, $M[P] :=
  M\{P/\Box\}$. That is, the contextual application of M to P is the
  substitution of $P$ for $\Box$ in $M$.
\end{definition}

$\meaningof{-} : L \to \mathcal{P}(\pi)$

\begin{mathpar}
  \inferrule* [lab=collection] {} {\meaningof{true} = \pi, \and \meaningof{~E} = \pi \setminus \meaningof{E}, \and \meaningof{E_{1} \& E_{2}} = \meaningof{E_{1}} \cap \meaningof{E_{2}}}
\end{mathpar}

\begin{mathpar}
  \inferrule* [lab=structure] {} {\meaningof{0} = \{ P \in \pi | P \equiv 0 \}, \and \\ \meaningof{E_1 | E_2} = \{ P \in \pi | P \equiv P_{1} | P_{2}, P_{1} \in \meaningof{E_{1}}, P_{2} \in \meaningof{E_2}\} }
\end{mathpar}

\begin{mathpar}
 \inferrule* [lab=behavior] {} {\meaningof{\langle a?b \rangle E} = \{ P \in \pi | P \equiv Q | u?(y)P', \\ \and \\\\ \and \\ \;\;\; u \in \meaningof{a}, \forall z.P'\{z/y\} \in \meaningof{E\{z/b\}}\}, \and \\ \meaningof{a!E} = \{ P \in \pi | P \equiv Q | x!\langle P' \rangle, x \in \meaningof{a} P' \in \meaningof{E}\} }
\end{mathpar}

\begin{mathpar}
 \inferrule* [lab=nominal] {} {\meaningof{\quotep{E}} = \{ \quotep{P} \in \quotep{\pi} | P \in \meaningof{E} \}, \and \meaningof{\quotep{P}} = \{ \quotep{Q} \in \quotep{\pi} | P \equiv Q \} \and \\ \meaningof{@\quotep{E}} = \{ P \in \pi | P \equiv @x, x \in \meaningof{E} \}}
\end{mathpar}

\begin{eqnarray*}
  \\
  \meaningof{-} : TS \to ST
\end{eqnarray*}

\begin{eqnarray*}
  \\
  L : TS \to ST
\end{eqnarray*}

\begin{eqnarray*}
  \\
  P \models E \iff P \in \meaningof{E}
\end{eqnarray*}

\begin{eqnarray*}
  P \approx_{L} Q \iff \forall E \in L. P \models E \iff Q \models E
\end{eqnarray*}

\begin{eqnarray*}
  P \approx_{K} Q
\end{eqnarray*}

\begin{eqnarray*}
  P \approx Q
\end{eqnarray*}

$\approx_{K} = \approx = \approx_{L}$

\subsubsection{Contextual duality}

Note that contexts extend the quotation operation to a family of
operations from processes to names. Given a context, $M$, we can
define a \emph{nominal context}, $\quotep{M}$ by $\quotep{M}[P] :=
\quotep{M[P]}$. To foreshadow what is to come we observe that these
operations enjoy a duality with processes very much like the duality
between vectors and maps from vectors to scalars.

Further, because the calculus is essentially higher-order, we have a
correspondence between contexts and processes. More specifically,
given a name $x$ and a context $M$ we can construct $M^{*}_{x}$ such
that 

\begin{mathpar}
  M^{*}_{x} | \lift{x}{P} \red M[P]
\end{mathpar}

namely,

\begin{mathpar}
  M^{*}_{x} := x?(u).M[\dropn{u}]
\end{mathpar}

The dependence of $M^{*}_{x}$ on a name makes it an abstraction, 

\begin{mathpar}
  M^{*} := (x)x?(u).M[\dropn{u}]
\end{mathpar}

\subsection{Additional notation}

It will sometimes be convenient to denote the process a name
quotes. We already have the notation $x = \quotep{P}$, but it will be
convenient to introduce an alternate notation, $\procn{x}$, when we
want to emphasize the connection to the use of the name. Note that, by
virtue of name equivalence, $\quotep{\procn{x}} \nameeq x$; so, the
notation is consistent with previous definitions.

Further, because names have structure it is possible to effect
substitutions on the basis of that structure. This means we need to
upgrade our notation for substitutions, which we accomplish by
adapting comprehension notation. Thus,

\begin{mathpar}
  P\{ y / x : x \in S \}
\end{mathpar}

is interpreted to mean the process derived from P by replacing (in a
capture-avoiding manner) each occurrence of $x$ in $S$ by $y$. For example,

\begin{mathpar}
  P\{ \quotep{\procn{x}|\procn{x}} / x : x \in \freenames{P} \}
\end{mathpar}

will replace each (occurrence) of a free name $x$ in $P$ by
$\quotep{\procn{x}|\procn{x}}$.

Also, we will avail ourselves of the notation $x^{L}$ and $x^{R}$ to
denote injections of a name into disjoint copies of the name
space. There are numerous ways to accomplish this. One example can be
found in \cite{MeredithR05}. This notation overloads to vectors of
names: $\vec{x}^{\pi} := (x_{i}^{\pi} \; : \; 0 \leq i < |\vec{x}| )$ where $\pi \in \{L,R\}$.

We also use $P^{\Box} := P|\Box$.

In \cite{MeredithR05} an interpretation of the new operator is
given. It turns out that there are several possible interpretations
all enjoying the requisite algebraic properties of the operator (see
\cite{milner91polyadicpi}). We will therefore make liberal use of
$(\nu\; \vec{x})P$.

% subsection the_syntax_and_semantics_of_the_notation_system (end)   

\input{qm2pi.qmops} 

\input{qm2pi.sterngerlach} 

\input{qm2pi.metric} 

% section concurrent_process_calculi (end)

%\input{qm2pi.proofsketch}

% section proof sketch (end)

%\input{qm2pi.slviaknots} 

% section spatial logic via knots (end)

\input{qm2pi.conclusion}

% section conclusion (end)

%\input{qm2pi.dtcodes} 

% section wiring algorithm (end)

\input{qm2pi.ack} 

% section acknowledgments (end)

\newpage


\bibliographystyle{plain}   
\bibliography{../../biblios/main.bib}

\input{qm2pi.rhodetails}

\end{document}

 

% section acknowledgments (end)

\newpage


\bibliographystyle{plain}   
\bibliography{../../biblios/main.bib}

\documentclass[12pt]{llncs}
%\documentclass{jktr}

\usepackage[pdftex]{hyperref}                   
\usepackage {listings}
\usepackage {mathpartir}
\usepackage{bcprules}
%\usepackage{listings}
                       
\usepackage{graphicx} 
%\usepackage[margins=2.5cm,nohead,nofoot]{geometry}
%\usepackage{geometry}
\usepackage{amsfonts}
\usepackage{amstext}
\usepackage{latexsym}
\usepackage{amssymb}
\usepackage{color}


%\include{myPreamble}
\include{qm2pi.local} 

%\ifpdf
%\usepackage[pdftex]{graphicx}
%\else
%\usepackage{graphicx}
%\fi

 % \ifpdf
%  \usepackage{pdfsync}
%  \if


%\title{Brief Article}
%\author{David F. Snyder}
%\author{L.G. Meredith}

%\address{Dept. of Math., Texas State University--San Marcos, San Marcos, TX 78666}
       
\pagestyle{empty}


\begin{document}

\lstset{language=[Objective]Caml,frame=shadowbox}

\input{qm2pi.front}

% section front matter (end)

\input{qm2pi.intro} 
 
% section introduction (end)

% \input{qm2pi.knotations} 

% section notation (end)

\input{qm2pi.process.calculi} 

% section concurrent_process_calculi_and_spatial_logics_ (end)
    
%\input{qm2pi.knots2pi} 

%\input{qm2pi.trefoil} 

%\input{qm2pi.mainthm} 

% subsection basic_interpretation (end)

%\input{qm2pi.rho.presentation} 
\subsection{The syntax and semantics of the notation system}\label{sub:the_syntax_and_semantics_of_the_notation_system} % (fold)

We now summarize a technical presentation of the calculus that
embodies our theory of dynamics. The typical presentation of such a
calculus follows the style of giving generators and relations on
them. The grammar, below, describing term constructors, freely
generates the set of processes, $\Proc$. This set is then quotiented
by a relation known as structural congruence and it is over this set
that the notion of dynamics is expressed. This presentation is
essentially that of \cite{MeredithR05} with the addition of
polyadicity and summation. For readability we have relegated some of
the technical subtleties to an appendix.

\subsubsection{Process grammar}\label{subsub:process_grammar}

\begin{mathpar}
  \inferrule* [lab=synchronization] {} {{M} \bc \pzero \;|\; x?F \;|\; x!C }
  \and
  \inferrule* [lab=abstraction] {} {{F} \bc (x)P}
  \and
  \inferrule* [lab=concretion] {} {{C} \bc \langle Q \rangle}
  \and
  \inferrule* [lab=process] {} {{P,Q} \bc M \;| \;P|Q \;|\; @{x}}
  \and
  \inferrule* [lab=name] {} {{x} \bc \quotep{P}}
\end{mathpar} 

Note that $\vec{x}$ (resp. $\vec{P}$) denotes a vector of names
(resp. processes) of length $|\vec{x}|$ (resp. $|\vec{P}|$). We adopt
the following useful abbreviations.

\begin{mathpar}
   x?(\vec{y}).P := x.(\vec{y})P \and  x\clift{\vec{P}} := x.\clift{\vec{P}}
   \and x!(y) := \lift{x}{\dropn{y}}
   \and \Pi_{i=0}^{n-1}P_i := P_0 | \ldots | P_{n-1}
\end{mathpar}

\subsubsection{Structural congruence}

\paragraph{Free and bound names and alpha-equivalence.} At the
core of structural equivalence is alpha-equivalence which identifies
process that are the same up to a change of variable. Formally, we
recognize the distinction between free and bound names. The free names
of a process, $\freenames{P}$, may be calculated recursively as
follows:

\begin{mathpar}
\freenames{\pzero} := \emptyset
  \and \\
  \freenames{x?(y).P} := \{ x \} \cup (\freenames{P} \setminus \{ y \})
  \and 
  \freenames{x!\langle P \rangle} := \{ x \} \cup \{ P \} 
  \and \\
  \freenames{P|Q} := \freenames{P} \cup \freenames{Q}
  \and \\
  \freenames{@{x}} := \{ x \}
\end{mathpar}

$\pi$
$\quotep{\pi}$

$\freenames{-} : \pi \to \mathcal{P}(\quotep{\pi})$

\begin{eqnarray*}
  \freenames{\pzero} & := & \emptyset \\
  \freenames{x?(y).P} & := & \{ x \} \cup (\freenames{P} \setminus \{ y \}) \\
  \freenames{x!\langle P \rangle} & := & \{ x \} \cup \{ P \} \\
  \freenames{P|Q} & := & \freenames{P} \cup \freenames{Q} \\
  \freenames{\dropn{x}} & := & \{ x \}
\end{eqnarray*}

The bound names of a process, $\boundnames{P}$, are those names occurring in $P$
that are not free. For example, in $x?(y).0$, the name $x$ is free, while $y$ is bound.

\begin{mathpar}
  \inferrule* [lab=monoidal-laws] {} { P|Q \equiv Q|P \and P|0 \equiv P \and P|(Q|R) \equiv (P|Q)|R }
\end{mathpar}

\begin{mathpar}
  \inferrule* [lab=alpha-equivalence] {} { (x)P \equiv (y)P\{y/x\} \and y \not\in \freenames{P} }
\end{mathpar}

\begin{definition}
Then two processes, $P,Q$, are alpha-equivalent if $P = Q\{\vec{y}/\vec{x}\}$ for
some $\vec{x} \in \boundnames{Q},\vec{y} \in \boundnames{P}$, where $Q\{\vec{y}/\vec{x}\}$
denotes the capture-avoiding substitution of $\vec{y}$ for $\vec{x}$ in $Q$.
\end{definition}

\begin{definition}
  The {\em structural congruence} \cite{SangiorgiWalker} , $\equiv$,
  between processes is the least congruence containing
  alpha-equivalence, satisfying the abelian monoid laws
  (associativity, commutativity and $\pzero$ as identity) for parallel
  composition $|$ and for summation $+$.
\end{definition}

\subsection{Name equivalence}

We take name equivalence, written $\nameeq$, to be the smallest
equivalence relation generated by the following rules.

\begin{mathpar}
\inferrule*[lab=Quote-drop]
{ }
{ \quotep{@{x}} \nameeq x }

\inferrule*[lab=Struct-equiv]
{ P \scong Q }
{ \quotep{P} \nameeq \quotep{Q} }
\end{mathpar}

The astute reader will have noticed that the mutual recursion of names
and processes imposes a mutual recursion on alpha-equivalence and
structural equivalence via name-equivalence. Fortunately, all of this
works out pleasantly and we may calculate in the natural way, free of
concern. The reader interested in the details is referred to the
appendix \ref{appendix:rho_details}.

\subsection{Substitution}

We use $\Proc$ for the set of processes, $\QProc$ for the set of
names, and $\id{\{}\vec{y} / \vec{x} \id{\}}$ to denote partial maps,
$s : \QProc \rightarrow \QProc$. A map, $s$ lifts, uniquely, to a map
on process terms, $\widehat{s} : \Proc \rightarrow \Proc$ by the
following equations.

\begin{mathpar}
  (0) \psubstp{Q}{P} := 0 \\
  (R \juxtap S) \psubstp{Q}{P}
  :=    
  (R)\psubstp{Q}{P} \juxtap (S) \psubstp{Q}{P} \\
  (x?(y).R) \psubstp{Q}{P}    
  :=    
  (x)\substp{Q}{P} (z)\concat( (R \psubstn{z}{y}) \psubstp{Q}{P} ) \\
  (\lift{x}{R}) \psubstp{Q}{P}  
  :=
  \lift{(x)\substp{Q}{P}}{ R \psubstp{Q}{P} } \\
%   (\dropn{x})  \psubstp{Q}{P}       
%   := 
%   \left\{ 
%     \begin{array}{ccc} 
%       \dropn{\quotep{Q}} & & x \nameeq \quotep{P} \\
%       \dropn{x} & & otherwise \\
%     \end{array}
%   \right. 
  (\dropn{x})  \psubstp{Q}{P}       
  := 
  \left\{ 
    \begin{array}{ccc} 
      Q & & x \nameeq \quotep{P} \\
      \dropn{x} & & otherwise \\
    \end{array}
  \right.
\end{mathpar}
 

where

\begin{eqnarray}
  (x)\id{\{} \lpquote Q \rpquote / \lpquote P \rpquote \id{\}}            = 
  \left\{ 
    \begin{array}{ccc}
      \lpquote Q \rpquote & & x \nameeq \lpquote P \rpquote \\
      x & & otherwise \\
    \end{array}
  \right. \nonumber
\end{eqnarray}

and $z$ is chosen distinct from $\quotep{P}$, $\quotep{Q}$, the free
names in $Q$, and all the names in $R$. Our $\alpha$-equivalence will
be built in the standard way from this substitution.

\begin{remark}\label{rem:no_self_referential_names}
  One consequence of these definitions is that $\forall P. \quotep{P}
  \not\in \freenames{P}$.
\end{remark}

\subsection{ Dynamic quote: an example }

Anticipating something of what's to come, consider applying the
substitution, $\widehat{\id{\{}u / z \id{\}}}$, to the following pair
of processes, $\lift{w}{y!(z)}$ and $w[ \lpquote y!(z) \rpquote ]$.

\begin{eqnarray}
	\lift{w}{y!(z)}\widehat{\id{\{}u / z \id{\}}}
		& = &
		\lift{w}{y!(u)} \nonumber\\
	w[ \lpquote y!(z) \rpquote ] \widehat{ \id{\{}u / z \id{\}} }
		& = &
		w[ \lpquote y!(z) \rpquote ] \nonumber
\end{eqnarray}

Because the body of the process between quotes is impervious to
substitution, we get radically different answers. In fact, by
examining the first process in an input context,
e.g. $x?(z).\lift{w}{y!(z)}$, we see that the process under the lift
operator may be shaped by prefixed inputs binding a name inside it. In
this sense, the lift operator will be seen as a way to dynamically
construct processes before reifying them as names.

Finally equipped with these standard features we can present the
dynamics of the calculus.

\subsubsection{Operational semantics} 

Finally, we introduce the computational dynamics. What marks these
algebras as distinct from other more traditionally studied algebraic
structures, e.g. vector spaces or polynomial rings, is the manner in
which dynamics is captured. In traditional structures, dynamics is typically
expressed through morphisms between such structures, as in linear maps
between vector spaces or morphisms between rings. In algebras
associated with the semantics of computation, the dynamics is
expressed as part of the algebraic structure itself, through a
reduction reduction relation typically denoted by $\red$. Below, we
give a recursive presentation of this relation for the calculus used
in the encoding.

$\red \subseteq \pi \times \pi$
$\red : \pi \to \mathcal{P}(\pi)$

\begin{mathpar}
  \inferrule* [lab=Comm] { \textsf{match}( x_{src}, x_{trgt} ) } { x_{trgt}?(y)P \; | \; x_{src}!\langle {Q} \rangle \red P\{\quotep{Q}/y}\} }
  \and \\
  \inferrule* [lab=Par] {{P} \red {P}'} {{{P} | {Q}} \red {{P}' | {Q}}}
  \and
  \inferrule* [lab=Equiv]{{{P} \scong {P}'} \andalso {{P}' \red {Q}'} \andalso {{Q}' \scong {Q}}}{{P} \red {Q}}
\end{mathpar}

\begin{eqnarray*}
  match_{\equiv} (\quotep{P},\quotep{Q}) & := & P \equiv Q \\
  match_{\dagger}(\quotep{P},\quotep{Q}) & := & \forall R. P|Q \red^{*} R => R \red^{*} 0 \\
  match_{K}(\quotep{P},\quotep{Q}) & := & K \mbox{ for some context } K
\end{eqnarray*}

$u?(x)P | u!\langle Q \rangle \red P\{\quotep{Q}/x\}$

%We write $\wred$ for $\red^*$, and $P\red$ if $\exists Q $ such that $ P \red Q$.
We write $P\red$ if $\exists Q $ such that $ P \red Q$ and $P\not\red$, otherwise.

\section{Replication}

As mentioned before, it is known that replication (and hence
recursion) can be implemented in a higher-order process algebra
\cite{SangiorgiWalker}. As our first example of calculation with the
machinery thus far presented we give the construction explicitly in
the {\rhoc}.

\begin{eqnarray}
	D_{x} & := & \prefix{x}{y}{(\binpar{\outputp{x}{y}}{@{y}})} \nonumber\\
	\bangp_{x}{P} & := & \binpar{{x}!\langle{\binpar{D_{x}}{P}}\rangle}{D_{x}} \nonumber
\end{eqnarray}

\begin{eqnarray}
	\bangp_{x}{P} & & \nonumber\\
	=
	& {x}!\langle{(\prefix{x}{y}{(\outputp{x}{y} | @{y})) | P}}\rangle 
	      | \prefix{x}{y}{(\outputp{x}{y} | @{y})} & \nonumber\\
	\red
	& (\outputp{x}{y} | @{y})\substn{\quotep{(\prefix{x}{y}{(@{y} | \outputp{x}{y})) | P}}}{y} & \nonumber\\
	=
	& \outputp{x}{\quotep{(\prefix{x}{y}{(\outputp{x}{y} | @{y})) | P}}}
	  | {(\prefix{x}{y}{(\outputp{x}{y} | @{y})) | P}} & \nonumber\\
	\red
	& \ldots & \nonumber\\
	\red^*
	& P | P | \ldots & \nonumber
\end{eqnarray}

Of course, this encoding, as an implementation, runs away, unfolding
$\bangp{P}$ eagerly. A lazier and more implementable replication
operator, restricted to input-guarded processes, may be obtained as follows.

\begin{eqnarray}
\bangp{\prefix{u}{v}{P}} 
	:= 
	\binpar{\lift{x}{\prefix{u}{v}{(\binpar{D(x)}{P})}}}{D(x)} \nonumber
\end{eqnarray}

\begin{remark}
  Note that the lazier definition still does not deal with summation
  or mixed summation (i.e. sums over input and output). The reader is
  invited to construct definitions of replication that deal with these
  features. 

  Further, the definitions are parameterized in a name, $x$. Can you,
  gentle reader, make a definition that eliminates this parameter and
  guarantees no accidental interaction between the replication
  machinery and the process being replicated -- i.e. no accidental
  sharing of names used by the process to get its work done and the
  name(s) used by the replication to effect copying. This latter
  revision of the definition of replication is crucial to obtaining
  the expected identity $!!P \sim !P$.
\end{remark}

\begin{remark}\label{rem:paradoxical_combinator}
  The reader familiar with the lambda calculus will have noticed the
  similarity between $D$ and the paradoxical combinator.

  [Ed. note: the existence of this seems to suggest we have to be more
  restrictive on the set of processes and names we admit if we are to
  support no-cloning.]
\end{remark}

\subsubsection{Bisimulation}

The computational dynamics gives rise to another kind of equivalence,
the equivalence of computational behavior. As previously mentioned
this is typically captured \emph{via} some form of bisimulation.

% The notion we use in this paper is weak barbed bisimulation
% \cite{milner91polyadicpi}.

The notion we use in this paper is derived from weak barbed
bisimulation \cite{milner91polyadicpi}. 

\begin{definition}
An \emph{observation relation}, $\downarrow_{\mathcal N}$, over a set
of names, $\mathcal N$, is the smallest relation satisfying the rules
below.

\infrule[Out-barb]{y \in {\mathcal N}, \; x \nameeq y}
		  {\outputp{x}{v} \downarrow_{\mathcal N} x}
\infrule[Par-barb]{\mbox{$P\downarrow_{\mathcal N} x$ or $Q\downarrow_{\mathcal N} x$}}
		  {\binpar{P}{Q} \downarrow_{\mathcal N} x}

We write $P \Downarrow_{\mathcal N} x$ if there is $Q$ such that 
$P \wred Q$ and $Q \downarrow_{\mathcal N} x$.
\end{definition}

\begin{definition}
%\label{def.bbisim}
An  ${\mathcal N}$-\emph{barbed bisimulation} over a set of names, ${\mathcal N}$, is a symmetric binary relation 
${\mathcal S}_{\mathcal N}$ between agents such that $P\rel{S}_{\mathcal N}Q$ implies:
\begin{enumerate}
\item If $P \red P'$ then $Q \wred Q'$ and $P'\rel{S}_{\mathcal N} Q'$.
\item If $P\downarrow_{\mathcal N} x$, then $Q\Downarrow_{\mathcal N} x$.
\end{enumerate}
$P$ is ${\mathcal N}$-barbed bisimilar to $Q$, written
$P \wbbisim_{\mathcal N} Q$, if $P \rel{S}_{\mathcal N} Q$ for some ${\mathcal N}$-barbed bisimulation ${\mathcal S}_{\mathcal N}$.
\end{definition}

$\mathcal{R} \subseteq \pi \times \pi$

$P \mathcal{R} Q => \forall P'. P \red P' \Rightarrow \exists Q'. Q \red Q', P' \mathcal{R} Q'$

$P \vdash x \Rightarrow Q \vdash x$

\begin{mathpar}
  \inferrule*[lab=Out-barb]{x \nameeq y}{{y}!\langle{Q}\rangle \vdash x}
  \and
  \inferrule*[lab=Par-barb]{\mbox{$P\vdash x$ or $Q\vdash x$}}{\binpar{P}{Q} \vdash x}
\end{mathpar}

\subsubsection{Contexts}

One of the principle advantages of computational calculi like the
$\pi$-calculus is a well-defined notion of context,
contextual-equivalence and a correlation between
contextual-equivalence and notions of bisimulation. The notion of
context allows the decomposition of a process into (sub-)process and
its syntactic environment, its context. Thus, a context may be
thought of as a process with a ``hole'' (written $\Box$) in it. The
application of a context $M$ to a process $P$, written $M[P]$, is
tantamount to filling the hole in $M$ with $P$. In this paper we do
not need the full weight of this theory, but do make use of the notion
of context in the proof the main theorem. 

\begin{mathpar}
  \inferrule* [lab=summation] {} {{M_{M},M_{N}} \bc \Box \;|\; x.M_{A} \;|\; M_{M}+M_{N}}
  \and
  \inferrule* [lab=agent] {} {{M_{A}} \bc (\vec{x})M_{P} \;| \; \clift{P_0,\ldots,M_{P},\ldots,P_N}}
  \and \\
  \inferrule* [lab=process] {} {{M_{P}} \bc M_{N} \;| \;P|M_{P} }
\end{mathpar} 

\begin{mathpar}
  \inferrule* [lab=sychronization] {} {M_{N} \bc \Box \;|\; x?M_{F} \;|\; x!M_{C}}
  \and
  \inferrule* [lab=abstraction] {} {{M_{F}} \bc (x)M_{P} }
  \and
  \inferrule* [lab=concretion] {} {{M_{C}} \bc \langle M_{P} \rangle }
  \and \\
  \inferrule* [lab=process] {} {{M_{P}} \bc M_{N} \;| \;P|M_{P} }
\end{mathpar}

\begin{definition}[contextual application] Given a context $M$, and
  process $P$, we define the \emph{contextual application}, $M[P] :=
  M\{P/\Box\}$. That is, the contextual application of M to P is the
  substitution of $P$ for $\Box$ in $M$.
\end{definition}

$\meaningof{-} : L \to \mathcal{P}(\pi)$

\begin{mathpar}
  \inferrule* [lab=collection] {} {\meaningof{true} = \pi, \and \meaningof{~E} = \pi \setminus \meaningof{E}, \and \meaningof{E_{1} \& E_{2}} = \meaningof{E_{1}} \cap \meaningof{E_{2}}}
\end{mathpar}

\begin{mathpar}
  \inferrule* [lab=structure] {} {\meaningof{0} = \{ P \in \pi | P \equiv 0 \}, \and \\ \meaningof{E_1 | E_2} = \{ P \in \pi | P \equiv P_{1} | P_{2}, P_{1} \in \meaningof{E_{1}}, P_{2} \in \meaningof{E_2}\} }
\end{mathpar}

\begin{mathpar}
 \inferrule* [lab=behavior] {} {\meaningof{\langle a?b \rangle E} = \{ P \in \pi | P \equiv Q | u?(y)P', \\ \and \\\\ \and \\ \;\;\; u \in \meaningof{a}, \forall z.P'\{z/y\} \in \meaningof{E\{z/b\}}\}, \and \\ \meaningof{a!E} = \{ P \in \pi | P \equiv Q | x!\langle P' \rangle, x \in \meaningof{a} P' \in \meaningof{E}\} }
\end{mathpar}

\begin{mathpar}
 \inferrule* [lab=nominal] {} {\meaningof{\quotep{E}} = \{ \quotep{P} \in \quotep{\pi} | P \in \meaningof{E} \}, \and \meaningof{\quotep{P}} = \{ \quotep{Q} \in \quotep{\pi} | P \equiv Q \} \and \\ \meaningof{@\quotep{E}} = \{ P \in \pi | P \equiv @x, x \in \meaningof{E} \}}
\end{mathpar}

\begin{eqnarray*}
  \\
  \meaningof{-} : TS \to ST
\end{eqnarray*}

\begin{eqnarray*}
  \\
  L : TS \to ST
\end{eqnarray*}

\begin{eqnarray*}
  \\
  P \models E \iff P \in \meaningof{E}
\end{eqnarray*}

\begin{eqnarray*}
  P \approx_{L} Q \iff \forall E \in L. P \models E \iff Q \models E
\end{eqnarray*}

\begin{eqnarray*}
  P \approx_{K} Q
\end{eqnarray*}

\begin{eqnarray*}
  P \approx Q
\end{eqnarray*}

$\approx_{K} = \approx = \approx_{L}$

\subsubsection{Contextual duality}

Note that contexts extend the quotation operation to a family of
operations from processes to names. Given a context, $M$, we can
define a \emph{nominal context}, $\quotep{M}$ by $\quotep{M}[P] :=
\quotep{M[P]}$. To foreshadow what is to come we observe that these
operations enjoy a duality with processes very much like the duality
between vectors and maps from vectors to scalars.

Further, because the calculus is essentially higher-order, we have a
correspondence between contexts and processes. More specifically,
given a name $x$ and a context $M$ we can construct $M^{*}_{x}$ such
that 

\begin{mathpar}
  M^{*}_{x} | \lift{x}{P} \red M[P]
\end{mathpar}

namely,

\begin{mathpar}
  M^{*}_{x} := x?(u).M[\dropn{u}]
\end{mathpar}

The dependence of $M^{*}_{x}$ on a name makes it an abstraction, 

\begin{mathpar}
  M^{*} := (x)x?(u).M[\dropn{u}]
\end{mathpar}

\subsection{Additional notation}

It will sometimes be convenient to denote the process a name
quotes. We already have the notation $x = \quotep{P}$, but it will be
convenient to introduce an alternate notation, $\procn{x}$, when we
want to emphasize the connection to the use of the name. Note that, by
virtue of name equivalence, $\quotep{\procn{x}} \nameeq x$; so, the
notation is consistent with previous definitions.

Further, because names have structure it is possible to effect
substitutions on the basis of that structure. This means we need to
upgrade our notation for substitutions, which we accomplish by
adapting comprehension notation. Thus,

\begin{mathpar}
  P\{ y / x : x \in S \}
\end{mathpar}

is interpreted to mean the process derived from P by replacing (in a
capture-avoiding manner) each occurrence of $x$ in $S$ by $y$. For example,

\begin{mathpar}
  P\{ \quotep{\procn{x}|\procn{x}} / x : x \in \freenames{P} \}
\end{mathpar}

will replace each (occurrence) of a free name $x$ in $P$ by
$\quotep{\procn{x}|\procn{x}}$.

Also, we will avail ourselves of the notation $x^{L}$ and $x^{R}$ to
denote injections of a name into disjoint copies of the name
space. There are numerous ways to accomplish this. One example can be
found in \cite{MeredithR05}. This notation overloads to vectors of
names: $\vec{x}^{\pi} := (x_{i}^{\pi} \; : \; 0 \leq i < |\vec{x}| )$ where $\pi \in \{L,R\}$.

We also use $P^{\Box} := P|\Box$.

In \cite{MeredithR05} an interpretation of the new operator is
given. It turns out that there are several possible interpretations
all enjoying the requisite algebraic properties of the operator (see
\cite{milner91polyadicpi}). We will therefore make liberal use of
$(\nu\; \vec{x})P$.

% subsection the_syntax_and_semantics_of_the_notation_system (end)   

\input{qm2pi.qmops} 

\input{qm2pi.sterngerlach} 

\input{qm2pi.metric} 

% section concurrent_process_calculi (end)

%\input{qm2pi.proofsketch}

% section proof sketch (end)

%\input{qm2pi.slviaknots} 

% section spatial logic via knots (end)

\input{qm2pi.conclusion}

% section conclusion (end)

%\input{qm2pi.dtcodes} 

% section wiring algorithm (end)

\input{qm2pi.ack} 

% section acknowledgments (end)

\newpage


\bibliographystyle{plain}   
\bibliography{../../biblios/main.bib}

\input{qm2pi.rhodetails}

\end{document}



\end{document}

 

%\documentclass[12pt]{llncs}
%\documentclass{jktr}

\usepackage[pdftex]{hyperref}                   
\usepackage {listings}
\usepackage {mathpartir}
\usepackage{bcprules}
%\usepackage{listings}
                       
\usepackage{graphicx} 
%\usepackage[margins=2.5cm,nohead,nofoot]{geometry}
%\usepackage{geometry}
\usepackage{amsfonts}
\usepackage{amstext}
\usepackage{latexsym}
\usepackage{amssymb}
\usepackage{color}


%\include{myPreamble}
\documentclass[12pt]{llncs}
%\documentclass{jktr}

\usepackage[pdftex]{hyperref}                   
\usepackage {listings}
\usepackage {mathpartir}
\usepackage{bcprules}
%\usepackage{listings}
                       
\usepackage{graphicx} 
%\usepackage[margins=2.5cm,nohead,nofoot]{geometry}
%\usepackage{geometry}
\usepackage{amsfonts}
\usepackage{amstext}
\usepackage{latexsym}
\usepackage{amssymb}
\usepackage{color}


%\include{myPreamble}
\include{qm2pi.local} 

%\ifpdf
%\usepackage[pdftex]{graphicx}
%\else
%\usepackage{graphicx}
%\fi

 % \ifpdf
%  \usepackage{pdfsync}
%  \if


%\title{Brief Article}
%\author{David F. Snyder}
%\author{L.G. Meredith}

%\address{Dept. of Math., Texas State University--San Marcos, San Marcos, TX 78666}
       
\pagestyle{empty}


\begin{document}

\lstset{language=[Objective]Caml,frame=shadowbox}

\input{qm2pi.front}

% section front matter (end)

\input{qm2pi.intro} 
 
% section introduction (end)

% \input{qm2pi.knotations} 

% section notation (end)

\input{qm2pi.process.calculi} 

% section concurrent_process_calculi_and_spatial_logics_ (end)
    
%\input{qm2pi.knots2pi} 

%\input{qm2pi.trefoil} 

%\input{qm2pi.mainthm} 

% subsection basic_interpretation (end)

%\input{qm2pi.rho.presentation} 
\subsection{The syntax and semantics of the notation system}\label{sub:the_syntax_and_semantics_of_the_notation_system} % (fold)

We now summarize a technical presentation of the calculus that
embodies our theory of dynamics. The typical presentation of such a
calculus follows the style of giving generators and relations on
them. The grammar, below, describing term constructors, freely
generates the set of processes, $\Proc$. This set is then quotiented
by a relation known as structural congruence and it is over this set
that the notion of dynamics is expressed. This presentation is
essentially that of \cite{MeredithR05} with the addition of
polyadicity and summation. For readability we have relegated some of
the technical subtleties to an appendix.

\subsubsection{Process grammar}\label{subsub:process_grammar}

\begin{mathpar}
  \inferrule* [lab=synchronization] {} {{M} \bc \pzero \;|\; x?F \;|\; x!C }
  \and
  \inferrule* [lab=abstraction] {} {{F} \bc (x)P}
  \and
  \inferrule* [lab=concretion] {} {{C} \bc \langle Q \rangle}
  \and
  \inferrule* [lab=process] {} {{P,Q} \bc M \;| \;P|Q \;|\; @{x}}
  \and
  \inferrule* [lab=name] {} {{x} \bc \quotep{P}}
\end{mathpar} 

Note that $\vec{x}$ (resp. $\vec{P}$) denotes a vector of names
(resp. processes) of length $|\vec{x}|$ (resp. $|\vec{P}|$). We adopt
the following useful abbreviations.

\begin{mathpar}
   x?(\vec{y}).P := x.(\vec{y})P \and  x\clift{\vec{P}} := x.\clift{\vec{P}}
   \and x!(y) := \lift{x}{\dropn{y}}
   \and \Pi_{i=0}^{n-1}P_i := P_0 | \ldots | P_{n-1}
\end{mathpar}

\subsubsection{Structural congruence}

\paragraph{Free and bound names and alpha-equivalence.} At the
core of structural equivalence is alpha-equivalence which identifies
process that are the same up to a change of variable. Formally, we
recognize the distinction between free and bound names. The free names
of a process, $\freenames{P}$, may be calculated recursively as
follows:

\begin{mathpar}
\freenames{\pzero} := \emptyset
  \and \\
  \freenames{x?(y).P} := \{ x \} \cup (\freenames{P} \setminus \{ y \})
  \and 
  \freenames{x!\langle P \rangle} := \{ x \} \cup \{ P \} 
  \and \\
  \freenames{P|Q} := \freenames{P} \cup \freenames{Q}
  \and \\
  \freenames{@{x}} := \{ x \}
\end{mathpar}

$\pi$
$\quotep{\pi}$

$\freenames{-} : \pi \to \mathcal{P}(\quotep{\pi})$

\begin{eqnarray*}
  \freenames{\pzero} & := & \emptyset \\
  \freenames{x?(y).P} & := & \{ x \} \cup (\freenames{P} \setminus \{ y \}) \\
  \freenames{x!\langle P \rangle} & := & \{ x \} \cup \{ P \} \\
  \freenames{P|Q} & := & \freenames{P} \cup \freenames{Q} \\
  \freenames{\dropn{x}} & := & \{ x \}
\end{eqnarray*}

The bound names of a process, $\boundnames{P}$, are those names occurring in $P$
that are not free. For example, in $x?(y).0$, the name $x$ is free, while $y$ is bound.

\begin{mathpar}
  \inferrule* [lab=monoidal-laws] {} { P|Q \equiv Q|P \and P|0 \equiv P \and P|(Q|R) \equiv (P|Q)|R }
\end{mathpar}

\begin{mathpar}
  \inferrule* [lab=alpha-equivalence] {} { (x)P \equiv (y)P\{y/x\} \and y \not\in \freenames{P} }
\end{mathpar}

\begin{definition}
Then two processes, $P,Q$, are alpha-equivalent if $P = Q\{\vec{y}/\vec{x}\}$ for
some $\vec{x} \in \boundnames{Q},\vec{y} \in \boundnames{P}$, where $Q\{\vec{y}/\vec{x}\}$
denotes the capture-avoiding substitution of $\vec{y}$ for $\vec{x}$ in $Q$.
\end{definition}

\begin{definition}
  The {\em structural congruence} \cite{SangiorgiWalker} , $\equiv$,
  between processes is the least congruence containing
  alpha-equivalence, satisfying the abelian monoid laws
  (associativity, commutativity and $\pzero$ as identity) for parallel
  composition $|$ and for summation $+$.
\end{definition}

\subsection{Name equivalence}

We take name equivalence, written $\nameeq$, to be the smallest
equivalence relation generated by the following rules.

\begin{mathpar}
\inferrule*[lab=Quote-drop]
{ }
{ \quotep{@{x}} \nameeq x }

\inferrule*[lab=Struct-equiv]
{ P \scong Q }
{ \quotep{P} \nameeq \quotep{Q} }
\end{mathpar}

The astute reader will have noticed that the mutual recursion of names
and processes imposes a mutual recursion on alpha-equivalence and
structural equivalence via name-equivalence. Fortunately, all of this
works out pleasantly and we may calculate in the natural way, free of
concern. The reader interested in the details is referred to the
appendix \ref{appendix:rho_details}.

\subsection{Substitution}

We use $\Proc$ for the set of processes, $\QProc$ for the set of
names, and $\id{\{}\vec{y} / \vec{x} \id{\}}$ to denote partial maps,
$s : \QProc \rightarrow \QProc$. A map, $s$ lifts, uniquely, to a map
on process terms, $\widehat{s} : \Proc \rightarrow \Proc$ by the
following equations.

\begin{mathpar}
  (0) \psubstp{Q}{P} := 0 \\
  (R \juxtap S) \psubstp{Q}{P}
  :=    
  (R)\psubstp{Q}{P} \juxtap (S) \psubstp{Q}{P} \\
  (x?(y).R) \psubstp{Q}{P}    
  :=    
  (x)\substp{Q}{P} (z)\concat( (R \psubstn{z}{y}) \psubstp{Q}{P} ) \\
  (\lift{x}{R}) \psubstp{Q}{P}  
  :=
  \lift{(x)\substp{Q}{P}}{ R \psubstp{Q}{P} } \\
%   (\dropn{x})  \psubstp{Q}{P}       
%   := 
%   \left\{ 
%     \begin{array}{ccc} 
%       \dropn{\quotep{Q}} & & x \nameeq \quotep{P} \\
%       \dropn{x} & & otherwise \\
%     \end{array}
%   \right. 
  (\dropn{x})  \psubstp{Q}{P}       
  := 
  \left\{ 
    \begin{array}{ccc} 
      Q & & x \nameeq \quotep{P} \\
      \dropn{x} & & otherwise \\
    \end{array}
  \right.
\end{mathpar}
 

where

\begin{eqnarray}
  (x)\id{\{} \lpquote Q \rpquote / \lpquote P \rpquote \id{\}}            = 
  \left\{ 
    \begin{array}{ccc}
      \lpquote Q \rpquote & & x \nameeq \lpquote P \rpquote \\
      x & & otherwise \\
    \end{array}
  \right. \nonumber
\end{eqnarray}

and $z$ is chosen distinct from $\quotep{P}$, $\quotep{Q}$, the free
names in $Q$, and all the names in $R$. Our $\alpha$-equivalence will
be built in the standard way from this substitution.

\begin{remark}\label{rem:no_self_referential_names}
  One consequence of these definitions is that $\forall P. \quotep{P}
  \not\in \freenames{P}$.
\end{remark}

\subsection{ Dynamic quote: an example }

Anticipating something of what's to come, consider applying the
substitution, $\widehat{\id{\{}u / z \id{\}}}$, to the following pair
of processes, $\lift{w}{y!(z)}$ and $w[ \lpquote y!(z) \rpquote ]$.

\begin{eqnarray}
	\lift{w}{y!(z)}\widehat{\id{\{}u / z \id{\}}}
		& = &
		\lift{w}{y!(u)} \nonumber\\
	w[ \lpquote y!(z) \rpquote ] \widehat{ \id{\{}u / z \id{\}} }
		& = &
		w[ \lpquote y!(z) \rpquote ] \nonumber
\end{eqnarray}

Because the body of the process between quotes is impervious to
substitution, we get radically different answers. In fact, by
examining the first process in an input context,
e.g. $x?(z).\lift{w}{y!(z)}$, we see that the process under the lift
operator may be shaped by prefixed inputs binding a name inside it. In
this sense, the lift operator will be seen as a way to dynamically
construct processes before reifying them as names.

Finally equipped with these standard features we can present the
dynamics of the calculus.

\subsubsection{Operational semantics} 

Finally, we introduce the computational dynamics. What marks these
algebras as distinct from other more traditionally studied algebraic
structures, e.g. vector spaces or polynomial rings, is the manner in
which dynamics is captured. In traditional structures, dynamics is typically
expressed through morphisms between such structures, as in linear maps
between vector spaces or morphisms between rings. In algebras
associated with the semantics of computation, the dynamics is
expressed as part of the algebraic structure itself, through a
reduction reduction relation typically denoted by $\red$. Below, we
give a recursive presentation of this relation for the calculus used
in the encoding.

$\red \subseteq \pi \times \pi$
$\red : \pi \to \mathcal{P}(\pi)$

\begin{mathpar}
  \inferrule* [lab=Comm] { \textsf{match}( x_{src}, x_{trgt} ) } { x_{trgt}?(y)P \; | \; x_{src}!\langle {Q} \rangle \red P\{\quotep{Q}/y}\} }
  \and \\
  \inferrule* [lab=Par] {{P} \red {P}'} {{{P} | {Q}} \red {{P}' | {Q}}}
  \and
  \inferrule* [lab=Equiv]{{{P} \scong {P}'} \andalso {{P}' \red {Q}'} \andalso {{Q}' \scong {Q}}}{{P} \red {Q}}
\end{mathpar}

\begin{eqnarray*}
  match_{\equiv} (\quotep{P},\quotep{Q}) & := & P \equiv Q \\
  match_{\dagger}(\quotep{P},\quotep{Q}) & := & \forall R. P|Q \red^{*} R => R \red^{*} 0 \\
  match_{K}(\quotep{P},\quotep{Q}) & := & K \mbox{ for some context } K
\end{eqnarray*}

$u?(x)P | u!\langle Q \rangle \red P\{\quotep{Q}/x\}$

%We write $\wred$ for $\red^*$, and $P\red$ if $\exists Q $ such that $ P \red Q$.
We write $P\red$ if $\exists Q $ such that $ P \red Q$ and $P\not\red$, otherwise.

\section{Replication}

As mentioned before, it is known that replication (and hence
recursion) can be implemented in a higher-order process algebra
\cite{SangiorgiWalker}. As our first example of calculation with the
machinery thus far presented we give the construction explicitly in
the {\rhoc}.

\begin{eqnarray}
	D_{x} & := & \prefix{x}{y}{(\binpar{\outputp{x}{y}}{@{y}})} \nonumber\\
	\bangp_{x}{P} & := & \binpar{{x}!\langle{\binpar{D_{x}}{P}}\rangle}{D_{x}} \nonumber
\end{eqnarray}

\begin{eqnarray}
	\bangp_{x}{P} & & \nonumber\\
	=
	& {x}!\langle{(\prefix{x}{y}{(\outputp{x}{y} | @{y})) | P}}\rangle 
	      | \prefix{x}{y}{(\outputp{x}{y} | @{y})} & \nonumber\\
	\red
	& (\outputp{x}{y} | @{y})\substn{\quotep{(\prefix{x}{y}{(@{y} | \outputp{x}{y})) | P}}}{y} & \nonumber\\
	=
	& \outputp{x}{\quotep{(\prefix{x}{y}{(\outputp{x}{y} | @{y})) | P}}}
	  | {(\prefix{x}{y}{(\outputp{x}{y} | @{y})) | P}} & \nonumber\\
	\red
	& \ldots & \nonumber\\
	\red^*
	& P | P | \ldots & \nonumber
\end{eqnarray}

Of course, this encoding, as an implementation, runs away, unfolding
$\bangp{P}$ eagerly. A lazier and more implementable replication
operator, restricted to input-guarded processes, may be obtained as follows.

\begin{eqnarray}
\bangp{\prefix{u}{v}{P}} 
	:= 
	\binpar{\lift{x}{\prefix{u}{v}{(\binpar{D(x)}{P})}}}{D(x)} \nonumber
\end{eqnarray}

\begin{remark}
  Note that the lazier definition still does not deal with summation
  or mixed summation (i.e. sums over input and output). The reader is
  invited to construct definitions of replication that deal with these
  features. 

  Further, the definitions are parameterized in a name, $x$. Can you,
  gentle reader, make a definition that eliminates this parameter and
  guarantees no accidental interaction between the replication
  machinery and the process being replicated -- i.e. no accidental
  sharing of names used by the process to get its work done and the
  name(s) used by the replication to effect copying. This latter
  revision of the definition of replication is crucial to obtaining
  the expected identity $!!P \sim !P$.
\end{remark}

\begin{remark}\label{rem:paradoxical_combinator}
  The reader familiar with the lambda calculus will have noticed the
  similarity between $D$ and the paradoxical combinator.

  [Ed. note: the existence of this seems to suggest we have to be more
  restrictive on the set of processes and names we admit if we are to
  support no-cloning.]
\end{remark}

\subsubsection{Bisimulation}

The computational dynamics gives rise to another kind of equivalence,
the equivalence of computational behavior. As previously mentioned
this is typically captured \emph{via} some form of bisimulation.

% The notion we use in this paper is weak barbed bisimulation
% \cite{milner91polyadicpi}.

The notion we use in this paper is derived from weak barbed
bisimulation \cite{milner91polyadicpi}. 

\begin{definition}
An \emph{observation relation}, $\downarrow_{\mathcal N}$, over a set
of names, $\mathcal N$, is the smallest relation satisfying the rules
below.

\infrule[Out-barb]{y \in {\mathcal N}, \; x \nameeq y}
		  {\outputp{x}{v} \downarrow_{\mathcal N} x}
\infrule[Par-barb]{\mbox{$P\downarrow_{\mathcal N} x$ or $Q\downarrow_{\mathcal N} x$}}
		  {\binpar{P}{Q} \downarrow_{\mathcal N} x}

We write $P \Downarrow_{\mathcal N} x$ if there is $Q$ such that 
$P \wred Q$ and $Q \downarrow_{\mathcal N} x$.
\end{definition}

\begin{definition}
%\label{def.bbisim}
An  ${\mathcal N}$-\emph{barbed bisimulation} over a set of names, ${\mathcal N}$, is a symmetric binary relation 
${\mathcal S}_{\mathcal N}$ between agents such that $P\rel{S}_{\mathcal N}Q$ implies:
\begin{enumerate}
\item If $P \red P'$ then $Q \wred Q'$ and $P'\rel{S}_{\mathcal N} Q'$.
\item If $P\downarrow_{\mathcal N} x$, then $Q\Downarrow_{\mathcal N} x$.
\end{enumerate}
$P$ is ${\mathcal N}$-barbed bisimilar to $Q$, written
$P \wbbisim_{\mathcal N} Q$, if $P \rel{S}_{\mathcal N} Q$ for some ${\mathcal N}$-barbed bisimulation ${\mathcal S}_{\mathcal N}$.
\end{definition}

$\mathcal{R} \subseteq \pi \times \pi$

$P \mathcal{R} Q => \forall P'. P \red P' \Rightarrow \exists Q'. Q \red Q', P' \mathcal{R} Q'$

$P \vdash x \Rightarrow Q \vdash x$

\begin{mathpar}
  \inferrule*[lab=Out-barb]{x \nameeq y}{{y}!\langle{Q}\rangle \vdash x}
  \and
  \inferrule*[lab=Par-barb]{\mbox{$P\vdash x$ or $Q\vdash x$}}{\binpar{P}{Q} \vdash x}
\end{mathpar}

\subsubsection{Contexts}

One of the principle advantages of computational calculi like the
$\pi$-calculus is a well-defined notion of context,
contextual-equivalence and a correlation between
contextual-equivalence and notions of bisimulation. The notion of
context allows the decomposition of a process into (sub-)process and
its syntactic environment, its context. Thus, a context may be
thought of as a process with a ``hole'' (written $\Box$) in it. The
application of a context $M$ to a process $P$, written $M[P]$, is
tantamount to filling the hole in $M$ with $P$. In this paper we do
not need the full weight of this theory, but do make use of the notion
of context in the proof the main theorem. 

\begin{mathpar}
  \inferrule* [lab=summation] {} {{M_{M},M_{N}} \bc \Box \;|\; x.M_{A} \;|\; M_{M}+M_{N}}
  \and
  \inferrule* [lab=agent] {} {{M_{A}} \bc (\vec{x})M_{P} \;| \; \clift{P_0,\ldots,M_{P},\ldots,P_N}}
  \and \\
  \inferrule* [lab=process] {} {{M_{P}} \bc M_{N} \;| \;P|M_{P} }
\end{mathpar} 

\begin{mathpar}
  \inferrule* [lab=sychronization] {} {M_{N} \bc \Box \;|\; x?M_{F} \;|\; x!M_{C}}
  \and
  \inferrule* [lab=abstraction] {} {{M_{F}} \bc (x)M_{P} }
  \and
  \inferrule* [lab=concretion] {} {{M_{C}} \bc \langle M_{P} \rangle }
  \and \\
  \inferrule* [lab=process] {} {{M_{P}} \bc M_{N} \;| \;P|M_{P} }
\end{mathpar}

\begin{definition}[contextual application] Given a context $M$, and
  process $P$, we define the \emph{contextual application}, $M[P] :=
  M\{P/\Box\}$. That is, the contextual application of M to P is the
  substitution of $P$ for $\Box$ in $M$.
\end{definition}

$\meaningof{-} : L \to \mathcal{P}(\pi)$

\begin{mathpar}
  \inferrule* [lab=collection] {} {\meaningof{true} = \pi, \and \meaningof{~E} = \pi \setminus \meaningof{E}, \and \meaningof{E_{1} \& E_{2}} = \meaningof{E_{1}} \cap \meaningof{E_{2}}}
\end{mathpar}

\begin{mathpar}
  \inferrule* [lab=structure] {} {\meaningof{0} = \{ P \in \pi | P \equiv 0 \}, \and \\ \meaningof{E_1 | E_2} = \{ P \in \pi | P \equiv P_{1} | P_{2}, P_{1} \in \meaningof{E_{1}}, P_{2} \in \meaningof{E_2}\} }
\end{mathpar}

\begin{mathpar}
 \inferrule* [lab=behavior] {} {\meaningof{\langle a?b \rangle E} = \{ P \in \pi | P \equiv Q | u?(y)P', \\ \and \\\\ \and \\ \;\;\; u \in \meaningof{a}, \forall z.P'\{z/y\} \in \meaningof{E\{z/b\}}\}, \and \\ \meaningof{a!E} = \{ P \in \pi | P \equiv Q | x!\langle P' \rangle, x \in \meaningof{a} P' \in \meaningof{E}\} }
\end{mathpar}

\begin{mathpar}
 \inferrule* [lab=nominal] {} {\meaningof{\quotep{E}} = \{ \quotep{P} \in \quotep{\pi} | P \in \meaningof{E} \}, \and \meaningof{\quotep{P}} = \{ \quotep{Q} \in \quotep{\pi} | P \equiv Q \} \and \\ \meaningof{@\quotep{E}} = \{ P \in \pi | P \equiv @x, x \in \meaningof{E} \}}
\end{mathpar}

\begin{eqnarray*}
  \\
  \meaningof{-} : TS \to ST
\end{eqnarray*}

\begin{eqnarray*}
  \\
  L : TS \to ST
\end{eqnarray*}

\begin{eqnarray*}
  \\
  P \models E \iff P \in \meaningof{E}
\end{eqnarray*}

\begin{eqnarray*}
  P \approx_{L} Q \iff \forall E \in L. P \models E \iff Q \models E
\end{eqnarray*}

\begin{eqnarray*}
  P \approx_{K} Q
\end{eqnarray*}

\begin{eqnarray*}
  P \approx Q
\end{eqnarray*}

$\approx_{K} = \approx = \approx_{L}$

\subsubsection{Contextual duality}

Note that contexts extend the quotation operation to a family of
operations from processes to names. Given a context, $M$, we can
define a \emph{nominal context}, $\quotep{M}$ by $\quotep{M}[P] :=
\quotep{M[P]}$. To foreshadow what is to come we observe that these
operations enjoy a duality with processes very much like the duality
between vectors and maps from vectors to scalars.

Further, because the calculus is essentially higher-order, we have a
correspondence between contexts and processes. More specifically,
given a name $x$ and a context $M$ we can construct $M^{*}_{x}$ such
that 

\begin{mathpar}
  M^{*}_{x} | \lift{x}{P} \red M[P]
\end{mathpar}

namely,

\begin{mathpar}
  M^{*}_{x} := x?(u).M[\dropn{u}]
\end{mathpar}

The dependence of $M^{*}_{x}$ on a name makes it an abstraction, 

\begin{mathpar}
  M^{*} := (x)x?(u).M[\dropn{u}]
\end{mathpar}

\subsection{Additional notation}

It will sometimes be convenient to denote the process a name
quotes. We already have the notation $x = \quotep{P}$, but it will be
convenient to introduce an alternate notation, $\procn{x}$, when we
want to emphasize the connection to the use of the name. Note that, by
virtue of name equivalence, $\quotep{\procn{x}} \nameeq x$; so, the
notation is consistent with previous definitions.

Further, because names have structure it is possible to effect
substitutions on the basis of that structure. This means we need to
upgrade our notation for substitutions, which we accomplish by
adapting comprehension notation. Thus,

\begin{mathpar}
  P\{ y / x : x \in S \}
\end{mathpar}

is interpreted to mean the process derived from P by replacing (in a
capture-avoiding manner) each occurrence of $x$ in $S$ by $y$. For example,

\begin{mathpar}
  P\{ \quotep{\procn{x}|\procn{x}} / x : x \in \freenames{P} \}
\end{mathpar}

will replace each (occurrence) of a free name $x$ in $P$ by
$\quotep{\procn{x}|\procn{x}}$.

Also, we will avail ourselves of the notation $x^{L}$ and $x^{R}$ to
denote injections of a name into disjoint copies of the name
space. There are numerous ways to accomplish this. One example can be
found in \cite{MeredithR05}. This notation overloads to vectors of
names: $\vec{x}^{\pi} := (x_{i}^{\pi} \; : \; 0 \leq i < |\vec{x}| )$ where $\pi \in \{L,R\}$.

We also use $P^{\Box} := P|\Box$.

In \cite{MeredithR05} an interpretation of the new operator is
given. It turns out that there are several possible interpretations
all enjoying the requisite algebraic properties of the operator (see
\cite{milner91polyadicpi}). We will therefore make liberal use of
$(\nu\; \vec{x})P$.

% subsection the_syntax_and_semantics_of_the_notation_system (end)   

\input{qm2pi.qmops} 

\input{qm2pi.sterngerlach} 

\input{qm2pi.metric} 

% section concurrent_process_calculi (end)

%\input{qm2pi.proofsketch}

% section proof sketch (end)

%\input{qm2pi.slviaknots} 

% section spatial logic via knots (end)

\input{qm2pi.conclusion}

% section conclusion (end)

%\input{qm2pi.dtcodes} 

% section wiring algorithm (end)

\input{qm2pi.ack} 

% section acknowledgments (end)

\newpage


\bibliographystyle{plain}   
\bibliography{../../biblios/main.bib}

\input{qm2pi.rhodetails}

\end{document}

 

%\ifpdf
%\usepackage[pdftex]{graphicx}
%\else
%\usepackage{graphicx}
%\fi

 % \ifpdf
%  \usepackage{pdfsync}
%  \if


%\title{Brief Article}
%\author{David F. Snyder}
%\author{L.G. Meredith}

%\address{Dept. of Math., Texas State University--San Marcos, San Marcos, TX 78666}
       
\pagestyle{empty}


\begin{document}

\lstset{language=[Objective]Caml,frame=shadowbox}

\documentclass[12pt]{llncs}
%\documentclass{jktr}

\usepackage[pdftex]{hyperref}                   
\usepackage {listings}
\usepackage {mathpartir}
\usepackage{bcprules}
%\usepackage{listings}
                       
\usepackage{graphicx} 
%\usepackage[margins=2.5cm,nohead,nofoot]{geometry}
%\usepackage{geometry}
\usepackage{amsfonts}
\usepackage{amstext}
\usepackage{latexsym}
\usepackage{amssymb}
\usepackage{color}


%\include{myPreamble}
\include{qm2pi.local} 

%\ifpdf
%\usepackage[pdftex]{graphicx}
%\else
%\usepackage{graphicx}
%\fi

 % \ifpdf
%  \usepackage{pdfsync}
%  \if


%\title{Brief Article}
%\author{David F. Snyder}
%\author{L.G. Meredith}

%\address{Dept. of Math., Texas State University--San Marcos, San Marcos, TX 78666}
       
\pagestyle{empty}


\begin{document}

\lstset{language=[Objective]Caml,frame=shadowbox}

\input{qm2pi.front}

% section front matter (end)

\input{qm2pi.intro} 
 
% section introduction (end)

% \input{qm2pi.knotations} 

% section notation (end)

\input{qm2pi.process.calculi} 

% section concurrent_process_calculi_and_spatial_logics_ (end)
    
%\input{qm2pi.knots2pi} 

%\input{qm2pi.trefoil} 

%\input{qm2pi.mainthm} 

% subsection basic_interpretation (end)

%\input{qm2pi.rho.presentation} 
\subsection{The syntax and semantics of the notation system}\label{sub:the_syntax_and_semantics_of_the_notation_system} % (fold)

We now summarize a technical presentation of the calculus that
embodies our theory of dynamics. The typical presentation of such a
calculus follows the style of giving generators and relations on
them. The grammar, below, describing term constructors, freely
generates the set of processes, $\Proc$. This set is then quotiented
by a relation known as structural congruence and it is over this set
that the notion of dynamics is expressed. This presentation is
essentially that of \cite{MeredithR05} with the addition of
polyadicity and summation. For readability we have relegated some of
the technical subtleties to an appendix.

\subsubsection{Process grammar}\label{subsub:process_grammar}

\begin{mathpar}
  \inferrule* [lab=synchronization] {} {{M} \bc \pzero \;|\; x?F \;|\; x!C }
  \and
  \inferrule* [lab=abstraction] {} {{F} \bc (x)P}
  \and
  \inferrule* [lab=concretion] {} {{C} \bc \langle Q \rangle}
  \and
  \inferrule* [lab=process] {} {{P,Q} \bc M \;| \;P|Q \;|\; @{x}}
  \and
  \inferrule* [lab=name] {} {{x} \bc \quotep{P}}
\end{mathpar} 

Note that $\vec{x}$ (resp. $\vec{P}$) denotes a vector of names
(resp. processes) of length $|\vec{x}|$ (resp. $|\vec{P}|$). We adopt
the following useful abbreviations.

\begin{mathpar}
   x?(\vec{y}).P := x.(\vec{y})P \and  x\clift{\vec{P}} := x.\clift{\vec{P}}
   \and x!(y) := \lift{x}{\dropn{y}}
   \and \Pi_{i=0}^{n-1}P_i := P_0 | \ldots | P_{n-1}
\end{mathpar}

\subsubsection{Structural congruence}

\paragraph{Free and bound names and alpha-equivalence.} At the
core of structural equivalence is alpha-equivalence which identifies
process that are the same up to a change of variable. Formally, we
recognize the distinction between free and bound names. The free names
of a process, $\freenames{P}$, may be calculated recursively as
follows:

\begin{mathpar}
\freenames{\pzero} := \emptyset
  \and \\
  \freenames{x?(y).P} := \{ x \} \cup (\freenames{P} \setminus \{ y \})
  \and 
  \freenames{x!\langle P \rangle} := \{ x \} \cup \{ P \} 
  \and \\
  \freenames{P|Q} := \freenames{P} \cup \freenames{Q}
  \and \\
  \freenames{@{x}} := \{ x \}
\end{mathpar}

$\pi$
$\quotep{\pi}$

$\freenames{-} : \pi \to \mathcal{P}(\quotep{\pi})$

\begin{eqnarray*}
  \freenames{\pzero} & := & \emptyset \\
  \freenames{x?(y).P} & := & \{ x \} \cup (\freenames{P} \setminus \{ y \}) \\
  \freenames{x!\langle P \rangle} & := & \{ x \} \cup \{ P \} \\
  \freenames{P|Q} & := & \freenames{P} \cup \freenames{Q} \\
  \freenames{\dropn{x}} & := & \{ x \}
\end{eqnarray*}

The bound names of a process, $\boundnames{P}$, are those names occurring in $P$
that are not free. For example, in $x?(y).0$, the name $x$ is free, while $y$ is bound.

\begin{mathpar}
  \inferrule* [lab=monoidal-laws] {} { P|Q \equiv Q|P \and P|0 \equiv P \and P|(Q|R) \equiv (P|Q)|R }
\end{mathpar}

\begin{mathpar}
  \inferrule* [lab=alpha-equivalence] {} { (x)P \equiv (y)P\{y/x\} \and y \not\in \freenames{P} }
\end{mathpar}

\begin{definition}
Then two processes, $P,Q$, are alpha-equivalent if $P = Q\{\vec{y}/\vec{x}\}$ for
some $\vec{x} \in \boundnames{Q},\vec{y} \in \boundnames{P}$, where $Q\{\vec{y}/\vec{x}\}$
denotes the capture-avoiding substitution of $\vec{y}$ for $\vec{x}$ in $Q$.
\end{definition}

\begin{definition}
  The {\em structural congruence} \cite{SangiorgiWalker} , $\equiv$,
  between processes is the least congruence containing
  alpha-equivalence, satisfying the abelian monoid laws
  (associativity, commutativity and $\pzero$ as identity) for parallel
  composition $|$ and for summation $+$.
\end{definition}

\subsection{Name equivalence}

We take name equivalence, written $\nameeq$, to be the smallest
equivalence relation generated by the following rules.

\begin{mathpar}
\inferrule*[lab=Quote-drop]
{ }
{ \quotep{@{x}} \nameeq x }

\inferrule*[lab=Struct-equiv]
{ P \scong Q }
{ \quotep{P} \nameeq \quotep{Q} }
\end{mathpar}

The astute reader will have noticed that the mutual recursion of names
and processes imposes a mutual recursion on alpha-equivalence and
structural equivalence via name-equivalence. Fortunately, all of this
works out pleasantly and we may calculate in the natural way, free of
concern. The reader interested in the details is referred to the
appendix \ref{appendix:rho_details}.

\subsection{Substitution}

We use $\Proc$ for the set of processes, $\QProc$ for the set of
names, and $\id{\{}\vec{y} / \vec{x} \id{\}}$ to denote partial maps,
$s : \QProc \rightarrow \QProc$. A map, $s$ lifts, uniquely, to a map
on process terms, $\widehat{s} : \Proc \rightarrow \Proc$ by the
following equations.

\begin{mathpar}
  (0) \psubstp{Q}{P} := 0 \\
  (R \juxtap S) \psubstp{Q}{P}
  :=    
  (R)\psubstp{Q}{P} \juxtap (S) \psubstp{Q}{P} \\
  (x?(y).R) \psubstp{Q}{P}    
  :=    
  (x)\substp{Q}{P} (z)\concat( (R \psubstn{z}{y}) \psubstp{Q}{P} ) \\
  (\lift{x}{R}) \psubstp{Q}{P}  
  :=
  \lift{(x)\substp{Q}{P}}{ R \psubstp{Q}{P} } \\
%   (\dropn{x})  \psubstp{Q}{P}       
%   := 
%   \left\{ 
%     \begin{array}{ccc} 
%       \dropn{\quotep{Q}} & & x \nameeq \quotep{P} \\
%       \dropn{x} & & otherwise \\
%     \end{array}
%   \right. 
  (\dropn{x})  \psubstp{Q}{P}       
  := 
  \left\{ 
    \begin{array}{ccc} 
      Q & & x \nameeq \quotep{P} \\
      \dropn{x} & & otherwise \\
    \end{array}
  \right.
\end{mathpar}
 

where

\begin{eqnarray}
  (x)\id{\{} \lpquote Q \rpquote / \lpquote P \rpquote \id{\}}            = 
  \left\{ 
    \begin{array}{ccc}
      \lpquote Q \rpquote & & x \nameeq \lpquote P \rpquote \\
      x & & otherwise \\
    \end{array}
  \right. \nonumber
\end{eqnarray}

and $z$ is chosen distinct from $\quotep{P}$, $\quotep{Q}$, the free
names in $Q$, and all the names in $R$. Our $\alpha$-equivalence will
be built in the standard way from this substitution.

\begin{remark}\label{rem:no_self_referential_names}
  One consequence of these definitions is that $\forall P. \quotep{P}
  \not\in \freenames{P}$.
\end{remark}

\subsection{ Dynamic quote: an example }

Anticipating something of what's to come, consider applying the
substitution, $\widehat{\id{\{}u / z \id{\}}}$, to the following pair
of processes, $\lift{w}{y!(z)}$ and $w[ \lpquote y!(z) \rpquote ]$.

\begin{eqnarray}
	\lift{w}{y!(z)}\widehat{\id{\{}u / z \id{\}}}
		& = &
		\lift{w}{y!(u)} \nonumber\\
	w[ \lpquote y!(z) \rpquote ] \widehat{ \id{\{}u / z \id{\}} }
		& = &
		w[ \lpquote y!(z) \rpquote ] \nonumber
\end{eqnarray}

Because the body of the process between quotes is impervious to
substitution, we get radically different answers. In fact, by
examining the first process in an input context,
e.g. $x?(z).\lift{w}{y!(z)}$, we see that the process under the lift
operator may be shaped by prefixed inputs binding a name inside it. In
this sense, the lift operator will be seen as a way to dynamically
construct processes before reifying them as names.

Finally equipped with these standard features we can present the
dynamics of the calculus.

\subsubsection{Operational semantics} 

Finally, we introduce the computational dynamics. What marks these
algebras as distinct from other more traditionally studied algebraic
structures, e.g. vector spaces or polynomial rings, is the manner in
which dynamics is captured. In traditional structures, dynamics is typically
expressed through morphisms between such structures, as in linear maps
between vector spaces or morphisms between rings. In algebras
associated with the semantics of computation, the dynamics is
expressed as part of the algebraic structure itself, through a
reduction reduction relation typically denoted by $\red$. Below, we
give a recursive presentation of this relation for the calculus used
in the encoding.

$\red \subseteq \pi \times \pi$
$\red : \pi \to \mathcal{P}(\pi)$

\begin{mathpar}
  \inferrule* [lab=Comm] { \textsf{match}( x_{src}, x_{trgt} ) } { x_{trgt}?(y)P \; | \; x_{src}!\langle {Q} \rangle \red P\{\quotep{Q}/y}\} }
  \and \\
  \inferrule* [lab=Par] {{P} \red {P}'} {{{P} | {Q}} \red {{P}' | {Q}}}
  \and
  \inferrule* [lab=Equiv]{{{P} \scong {P}'} \andalso {{P}' \red {Q}'} \andalso {{Q}' \scong {Q}}}{{P} \red {Q}}
\end{mathpar}

\begin{eqnarray*}
  match_{\equiv} (\quotep{P},\quotep{Q}) & := & P \equiv Q \\
  match_{\dagger}(\quotep{P},\quotep{Q}) & := & \forall R. P|Q \red^{*} R => R \red^{*} 0 \\
  match_{K}(\quotep{P},\quotep{Q}) & := & K \mbox{ for some context } K
\end{eqnarray*}

$u?(x)P | u!\langle Q \rangle \red P\{\quotep{Q}/x\}$

%We write $\wred$ for $\red^*$, and $P\red$ if $\exists Q $ such that $ P \red Q$.
We write $P\red$ if $\exists Q $ such that $ P \red Q$ and $P\not\red$, otherwise.

\section{Replication}

As mentioned before, it is known that replication (and hence
recursion) can be implemented in a higher-order process algebra
\cite{SangiorgiWalker}. As our first example of calculation with the
machinery thus far presented we give the construction explicitly in
the {\rhoc}.

\begin{eqnarray}
	D_{x} & := & \prefix{x}{y}{(\binpar{\outputp{x}{y}}{@{y}})} \nonumber\\
	\bangp_{x}{P} & := & \binpar{{x}!\langle{\binpar{D_{x}}{P}}\rangle}{D_{x}} \nonumber
\end{eqnarray}

\begin{eqnarray}
	\bangp_{x}{P} & & \nonumber\\
	=
	& {x}!\langle{(\prefix{x}{y}{(\outputp{x}{y} | @{y})) | P}}\rangle 
	      | \prefix{x}{y}{(\outputp{x}{y} | @{y})} & \nonumber\\
	\red
	& (\outputp{x}{y} | @{y})\substn{\quotep{(\prefix{x}{y}{(@{y} | \outputp{x}{y})) | P}}}{y} & \nonumber\\
	=
	& \outputp{x}{\quotep{(\prefix{x}{y}{(\outputp{x}{y} | @{y})) | P}}}
	  | {(\prefix{x}{y}{(\outputp{x}{y} | @{y})) | P}} & \nonumber\\
	\red
	& \ldots & \nonumber\\
	\red^*
	& P | P | \ldots & \nonumber
\end{eqnarray}

Of course, this encoding, as an implementation, runs away, unfolding
$\bangp{P}$ eagerly. A lazier and more implementable replication
operator, restricted to input-guarded processes, may be obtained as follows.

\begin{eqnarray}
\bangp{\prefix{u}{v}{P}} 
	:= 
	\binpar{\lift{x}{\prefix{u}{v}{(\binpar{D(x)}{P})}}}{D(x)} \nonumber
\end{eqnarray}

\begin{remark}
  Note that the lazier definition still does not deal with summation
  or mixed summation (i.e. sums over input and output). The reader is
  invited to construct definitions of replication that deal with these
  features. 

  Further, the definitions are parameterized in a name, $x$. Can you,
  gentle reader, make a definition that eliminates this parameter and
  guarantees no accidental interaction between the replication
  machinery and the process being replicated -- i.e. no accidental
  sharing of names used by the process to get its work done and the
  name(s) used by the replication to effect copying. This latter
  revision of the definition of replication is crucial to obtaining
  the expected identity $!!P \sim !P$.
\end{remark}

\begin{remark}\label{rem:paradoxical_combinator}
  The reader familiar with the lambda calculus will have noticed the
  similarity between $D$ and the paradoxical combinator.

  [Ed. note: the existence of this seems to suggest we have to be more
  restrictive on the set of processes and names we admit if we are to
  support no-cloning.]
\end{remark}

\subsubsection{Bisimulation}

The computational dynamics gives rise to another kind of equivalence,
the equivalence of computational behavior. As previously mentioned
this is typically captured \emph{via} some form of bisimulation.

% The notion we use in this paper is weak barbed bisimulation
% \cite{milner91polyadicpi}.

The notion we use in this paper is derived from weak barbed
bisimulation \cite{milner91polyadicpi}. 

\begin{definition}
An \emph{observation relation}, $\downarrow_{\mathcal N}$, over a set
of names, $\mathcal N$, is the smallest relation satisfying the rules
below.

\infrule[Out-barb]{y \in {\mathcal N}, \; x \nameeq y}
		  {\outputp{x}{v} \downarrow_{\mathcal N} x}
\infrule[Par-barb]{\mbox{$P\downarrow_{\mathcal N} x$ or $Q\downarrow_{\mathcal N} x$}}
		  {\binpar{P}{Q} \downarrow_{\mathcal N} x}

We write $P \Downarrow_{\mathcal N} x$ if there is $Q$ such that 
$P \wred Q$ and $Q \downarrow_{\mathcal N} x$.
\end{definition}

\begin{definition}
%\label{def.bbisim}
An  ${\mathcal N}$-\emph{barbed bisimulation} over a set of names, ${\mathcal N}$, is a symmetric binary relation 
${\mathcal S}_{\mathcal N}$ between agents such that $P\rel{S}_{\mathcal N}Q$ implies:
\begin{enumerate}
\item If $P \red P'$ then $Q \wred Q'$ and $P'\rel{S}_{\mathcal N} Q'$.
\item If $P\downarrow_{\mathcal N} x$, then $Q\Downarrow_{\mathcal N} x$.
\end{enumerate}
$P$ is ${\mathcal N}$-barbed bisimilar to $Q$, written
$P \wbbisim_{\mathcal N} Q$, if $P \rel{S}_{\mathcal N} Q$ for some ${\mathcal N}$-barbed bisimulation ${\mathcal S}_{\mathcal N}$.
\end{definition}

$\mathcal{R} \subseteq \pi \times \pi$

$P \mathcal{R} Q => \forall P'. P \red P' \Rightarrow \exists Q'. Q \red Q', P' \mathcal{R} Q'$

$P \vdash x \Rightarrow Q \vdash x$

\begin{mathpar}
  \inferrule*[lab=Out-barb]{x \nameeq y}{{y}!\langle{Q}\rangle \vdash x}
  \and
  \inferrule*[lab=Par-barb]{\mbox{$P\vdash x$ or $Q\vdash x$}}{\binpar{P}{Q} \vdash x}
\end{mathpar}

\subsubsection{Contexts}

One of the principle advantages of computational calculi like the
$\pi$-calculus is a well-defined notion of context,
contextual-equivalence and a correlation between
contextual-equivalence and notions of bisimulation. The notion of
context allows the decomposition of a process into (sub-)process and
its syntactic environment, its context. Thus, a context may be
thought of as a process with a ``hole'' (written $\Box$) in it. The
application of a context $M$ to a process $P$, written $M[P]$, is
tantamount to filling the hole in $M$ with $P$. In this paper we do
not need the full weight of this theory, but do make use of the notion
of context in the proof the main theorem. 

\begin{mathpar}
  \inferrule* [lab=summation] {} {{M_{M},M_{N}} \bc \Box \;|\; x.M_{A} \;|\; M_{M}+M_{N}}
  \and
  \inferrule* [lab=agent] {} {{M_{A}} \bc (\vec{x})M_{P} \;| \; \clift{P_0,\ldots,M_{P},\ldots,P_N}}
  \and \\
  \inferrule* [lab=process] {} {{M_{P}} \bc M_{N} \;| \;P|M_{P} }
\end{mathpar} 

\begin{mathpar}
  \inferrule* [lab=sychronization] {} {M_{N} \bc \Box \;|\; x?M_{F} \;|\; x!M_{C}}
  \and
  \inferrule* [lab=abstraction] {} {{M_{F}} \bc (x)M_{P} }
  \and
  \inferrule* [lab=concretion] {} {{M_{C}} \bc \langle M_{P} \rangle }
  \and \\
  \inferrule* [lab=process] {} {{M_{P}} \bc M_{N} \;| \;P|M_{P} }
\end{mathpar}

\begin{definition}[contextual application] Given a context $M$, and
  process $P$, we define the \emph{contextual application}, $M[P] :=
  M\{P/\Box\}$. That is, the contextual application of M to P is the
  substitution of $P$ for $\Box$ in $M$.
\end{definition}

$\meaningof{-} : L \to \mathcal{P}(\pi)$

\begin{mathpar}
  \inferrule* [lab=collection] {} {\meaningof{true} = \pi, \and \meaningof{~E} = \pi \setminus \meaningof{E}, \and \meaningof{E_{1} \& E_{2}} = \meaningof{E_{1}} \cap \meaningof{E_{2}}}
\end{mathpar}

\begin{mathpar}
  \inferrule* [lab=structure] {} {\meaningof{0} = \{ P \in \pi | P \equiv 0 \}, \and \\ \meaningof{E_1 | E_2} = \{ P \in \pi | P \equiv P_{1} | P_{2}, P_{1} \in \meaningof{E_{1}}, P_{2} \in \meaningof{E_2}\} }
\end{mathpar}

\begin{mathpar}
 \inferrule* [lab=behavior] {} {\meaningof{\langle a?b \rangle E} = \{ P \in \pi | P \equiv Q | u?(y)P', \\ \and \\\\ \and \\ \;\;\; u \in \meaningof{a}, \forall z.P'\{z/y\} \in \meaningof{E\{z/b\}}\}, \and \\ \meaningof{a!E} = \{ P \in \pi | P \equiv Q | x!\langle P' \rangle, x \in \meaningof{a} P' \in \meaningof{E}\} }
\end{mathpar}

\begin{mathpar}
 \inferrule* [lab=nominal] {} {\meaningof{\quotep{E}} = \{ \quotep{P} \in \quotep{\pi} | P \in \meaningof{E} \}, \and \meaningof{\quotep{P}} = \{ \quotep{Q} \in \quotep{\pi} | P \equiv Q \} \and \\ \meaningof{@\quotep{E}} = \{ P \in \pi | P \equiv @x, x \in \meaningof{E} \}}
\end{mathpar}

\begin{eqnarray*}
  \\
  \meaningof{-} : TS \to ST
\end{eqnarray*}

\begin{eqnarray*}
  \\
  L : TS \to ST
\end{eqnarray*}

\begin{eqnarray*}
  \\
  P \models E \iff P \in \meaningof{E}
\end{eqnarray*}

\begin{eqnarray*}
  P \approx_{L} Q \iff \forall E \in L. P \models E \iff Q \models E
\end{eqnarray*}

\begin{eqnarray*}
  P \approx_{K} Q
\end{eqnarray*}

\begin{eqnarray*}
  P \approx Q
\end{eqnarray*}

$\approx_{K} = \approx = \approx_{L}$

\subsubsection{Contextual duality}

Note that contexts extend the quotation operation to a family of
operations from processes to names. Given a context, $M$, we can
define a \emph{nominal context}, $\quotep{M}$ by $\quotep{M}[P] :=
\quotep{M[P]}$. To foreshadow what is to come we observe that these
operations enjoy a duality with processes very much like the duality
between vectors and maps from vectors to scalars.

Further, because the calculus is essentially higher-order, we have a
correspondence between contexts and processes. More specifically,
given a name $x$ and a context $M$ we can construct $M^{*}_{x}$ such
that 

\begin{mathpar}
  M^{*}_{x} | \lift{x}{P} \red M[P]
\end{mathpar}

namely,

\begin{mathpar}
  M^{*}_{x} := x?(u).M[\dropn{u}]
\end{mathpar}

The dependence of $M^{*}_{x}$ on a name makes it an abstraction, 

\begin{mathpar}
  M^{*} := (x)x?(u).M[\dropn{u}]
\end{mathpar}

\subsection{Additional notation}

It will sometimes be convenient to denote the process a name
quotes. We already have the notation $x = \quotep{P}$, but it will be
convenient to introduce an alternate notation, $\procn{x}$, when we
want to emphasize the connection to the use of the name. Note that, by
virtue of name equivalence, $\quotep{\procn{x}} \nameeq x$; so, the
notation is consistent with previous definitions.

Further, because names have structure it is possible to effect
substitutions on the basis of that structure. This means we need to
upgrade our notation for substitutions, which we accomplish by
adapting comprehension notation. Thus,

\begin{mathpar}
  P\{ y / x : x \in S \}
\end{mathpar}

is interpreted to mean the process derived from P by replacing (in a
capture-avoiding manner) each occurrence of $x$ in $S$ by $y$. For example,

\begin{mathpar}
  P\{ \quotep{\procn{x}|\procn{x}} / x : x \in \freenames{P} \}
\end{mathpar}

will replace each (occurrence) of a free name $x$ in $P$ by
$\quotep{\procn{x}|\procn{x}}$.

Also, we will avail ourselves of the notation $x^{L}$ and $x^{R}$ to
denote injections of a name into disjoint copies of the name
space. There are numerous ways to accomplish this. One example can be
found in \cite{MeredithR05}. This notation overloads to vectors of
names: $\vec{x}^{\pi} := (x_{i}^{\pi} \; : \; 0 \leq i < |\vec{x}| )$ where $\pi \in \{L,R\}$.

We also use $P^{\Box} := P|\Box$.

In \cite{MeredithR05} an interpretation of the new operator is
given. It turns out that there are several possible interpretations
all enjoying the requisite algebraic properties of the operator (see
\cite{milner91polyadicpi}). We will therefore make liberal use of
$(\nu\; \vec{x})P$.

% subsection the_syntax_and_semantics_of_the_notation_system (end)   

\input{qm2pi.qmops} 

\input{qm2pi.sterngerlach} 

\input{qm2pi.metric} 

% section concurrent_process_calculi (end)

%\input{qm2pi.proofsketch}

% section proof sketch (end)

%\input{qm2pi.slviaknots} 

% section spatial logic via knots (end)

\input{qm2pi.conclusion}

% section conclusion (end)

%\input{qm2pi.dtcodes} 

% section wiring algorithm (end)

\input{qm2pi.ack} 

% section acknowledgments (end)

\newpage


\bibliographystyle{plain}   
\bibliography{../../biblios/main.bib}

\input{qm2pi.rhodetails}

\end{document}



% section front matter (end)

\section{Introduction}\label{sec:introduction} % (fold)
In this draft of the material i am going to have to dispense with the
usual writing conventions adopted in papers on these topics. i'm going
to have adopt whatever tone i need at the time i'm writing up the
calculations. Sometimes this may be very conversational; others it may
be the barest mathematical grunts; others still it may be that i have
lifted text from one of my other papers because the exposition of some
point was better said there. i hope that my readers are not unduly put
out by this decision. i'm not doing this to flout convention or be
rebellious. i find these calculations very technically challenging. To
keep everything going technically, something has to give; i have to
let go of some cognitive burden. So, the academic writing style --
with all of its trade-offs in terms of facilitating technical
communication -- is what i'm letting go of. Perhaps subsequent drafts
can be tightened and polished, but for now, i'm going to speak as if
we were sitting together in a coffee shop with a laptop, wifi and a
pad of paper and a pencil.

So, here's what i have to say. We -- you and i, comfortably ensconced
in our coffee shop and well-equipped with our tools -- can realize and
carry out the calculations of quantum mechanics over a very different
formal theory of dynamics, a formal theory of dynamics that
corresponds to a theory of concurrent computation with
\emph{reflection}. It has the advantage that the underlying theory is
already `quantized', but supports analogues all of the continuuous
operations. Strikingly, this underlying theory has recently been
connected with a notion of metric that we can show, by calculating
together, coincides with the metric induced by the inner product.

There are a lot of reasons why you might be interested in seeing
calculations of this form. Here's why i'm interested. For the past
several centuries there has been no competitor to the ``Newtonian''
account of dynamics. As a result the predominant share of accounts of
dynamical systems and situations have had to be formulated in terms of
the Newtonian machinery. i view this as an intellectually dangerous
position to occupy. Everything, despite it's intrinsic shape, turns
into a nail to be hit with this hammer. Recently, however, the theory
of computation has matured to the point where we have candidates for
theories of dynamics that offer very different perspective on
reasoning about dynamical systems and situations. Testing these
candidates against very successful accounts of dynamical situations,
like quantum mechanics, is going to give us some sense of how mature
they are and some measure of the quality of these accounts of
dynamics.

\subsection{Summary of contributions and outline of paper}

So, we're going to develop an interpretation of the operations of
quantum mechanics normally interpreted by Hilbert spaces and
operators. We're going to do this over a theory of computation. Note
that this is very different than the usual quantum computation program
which develops notions of computation over quantum mechanics. Rather,
we are developing a story that aligns with Wheeler's slogan: It from
Bit. To do this we will first provide an account of the theory of
computation at play here. Then we will dive into a calculation-driven
interpretation of the operations of quantum mechanics.

The reason we take this approach is that -- until very recently --
there hasn't been an axiomatic account of quantum mechanics. As a
result there has been no sharp delineation of the mathematical theory
supporting interpretation of the physical theory and the physical
theory, itself. So, ambient features of the maths are free to be
exploited (or supressed) without a real accounting of their physical
relevance. There is no sharp statement ``here's the physical theory''
qua \emph{theory} and ``here's the mathematical interpretation''
enabling a judgment of how faithful the interpretation is -- apart
from experimental observation. When there is an axiomatic account we
can judge how well a given mathematical formalism supports an
interpretation of the axioms, independent of
experimentation. Likewise, we can judge how well we have captured our
physical evidence and experience with our axiomatics, independent of
any specific mathematical implementation, with accidental detail that
may or may not have physical significance. 

In lieu of a fully fleshed out and vetted axiomatic account of quantum
mechanics, interpreting the operational notions in service of modeling
physical systems will have to suffice. In other words, we are not in
the business of providing a model of Hilbert spaces and operators. We
are in the business of providing a model of quantum mechanics because
we are motivated by testing our notions of dynamics against physical
theory; and, the predictive calculations of the physical theory must
serve as the best formulation -- shy of a fully fleshed out axiomatic
account -- of the physical theory itself (as they have for scientific
theories since time immemorial). Put another way, despite a
whole-hearted commitment to an It-from-Bit ontology, we are firmly
aligned with the shut-up-and-calculate camp as the best way to obtain
results either from the physical perspective or as a quality assurance
measure of our fledgling theory of dynamics.

In detail, we present a reflective process calculus. Then we develop
intuitive correspondences between the notions available in this
calculus and the usual physical notions supporting quantum mechanical
calculations. Thus, 

\begin{table}[htp]
  \center{
    \fbox{
      \begin{tabular}{c|c}
        quantum mechanics & process calculus \\
        \hline
        scalar & name \\
        state vector & process \\
        dual & contextual duals \\
        matrix & formal sums of process-context-dual pairs \\
        orthogonality & process annihilation \\
        inner product & execution-formula + quoting
      \end{tabular}
    }
  }
  \caption{QM - process calculi correspondences}
\end{table}

Then we tighten up these intuitions to operational definitions. We
employ the Dirac notation as the best proxy we can find for an
abstract syntax of the quantum mechanical notions. The definitions we
develop put us in contact with equational constraints coming from the
theory that we demonstrate the definitions and calculations satisfy.

This puts us in a position to shut up and calculate for the
Stern-Gerlach experimental set up, showing how these predictive
calculations become calculations on processes in our theory of a
reflective process calculus.

Penultimately, we demonstrate that the notion of metric coming from
the inner product coincides with the notion of metric available from
the theory of bisimulation. This demonstration gives us the right to
think of space as arising from behavior. Finally, we consider where we
might go from the new vantage point we have obtained.

% section introduction (end) 
 
% section introduction (end)

% \documentclass[12pt]{llncs}
%\documentclass{jktr}

\usepackage[pdftex]{hyperref}                   
\usepackage {listings}
\usepackage {mathpartir}
\usepackage{bcprules}
%\usepackage{listings}
                       
\usepackage{graphicx} 
%\usepackage[margins=2.5cm,nohead,nofoot]{geometry}
%\usepackage{geometry}
\usepackage{amsfonts}
\usepackage{amstext}
\usepackage{latexsym}
\usepackage{amssymb}
\usepackage{color}


%\include{myPreamble}
\include{qm2pi.local} 

%\ifpdf
%\usepackage[pdftex]{graphicx}
%\else
%\usepackage{graphicx}
%\fi

 % \ifpdf
%  \usepackage{pdfsync}
%  \if


%\title{Brief Article}
%\author{David F. Snyder}
%\author{L.G. Meredith}

%\address{Dept. of Math., Texas State University--San Marcos, San Marcos, TX 78666}
       
\pagestyle{empty}


\begin{document}

\lstset{language=[Objective]Caml,frame=shadowbox}

\input{qm2pi.front}

% section front matter (end)

\input{qm2pi.intro} 
 
% section introduction (end)

% \input{qm2pi.knotations} 

% section notation (end)

\input{qm2pi.process.calculi} 

% section concurrent_process_calculi_and_spatial_logics_ (end)
    
%\input{qm2pi.knots2pi} 

%\input{qm2pi.trefoil} 

%\input{qm2pi.mainthm} 

% subsection basic_interpretation (end)

%\input{qm2pi.rho.presentation} 
\subsection{The syntax and semantics of the notation system}\label{sub:the_syntax_and_semantics_of_the_notation_system} % (fold)

We now summarize a technical presentation of the calculus that
embodies our theory of dynamics. The typical presentation of such a
calculus follows the style of giving generators and relations on
them. The grammar, below, describing term constructors, freely
generates the set of processes, $\Proc$. This set is then quotiented
by a relation known as structural congruence and it is over this set
that the notion of dynamics is expressed. This presentation is
essentially that of \cite{MeredithR05} with the addition of
polyadicity and summation. For readability we have relegated some of
the technical subtleties to an appendix.

\subsubsection{Process grammar}\label{subsub:process_grammar}

\begin{mathpar}
  \inferrule* [lab=synchronization] {} {{M} \bc \pzero \;|\; x?F \;|\; x!C }
  \and
  \inferrule* [lab=abstraction] {} {{F} \bc (x)P}
  \and
  \inferrule* [lab=concretion] {} {{C} \bc \langle Q \rangle}
  \and
  \inferrule* [lab=process] {} {{P,Q} \bc M \;| \;P|Q \;|\; @{x}}
  \and
  \inferrule* [lab=name] {} {{x} \bc \quotep{P}}
\end{mathpar} 

Note that $\vec{x}$ (resp. $\vec{P}$) denotes a vector of names
(resp. processes) of length $|\vec{x}|$ (resp. $|\vec{P}|$). We adopt
the following useful abbreviations.

\begin{mathpar}
   x?(\vec{y}).P := x.(\vec{y})P \and  x\clift{\vec{P}} := x.\clift{\vec{P}}
   \and x!(y) := \lift{x}{\dropn{y}}
   \and \Pi_{i=0}^{n-1}P_i := P_0 | \ldots | P_{n-1}
\end{mathpar}

\subsubsection{Structural congruence}

\paragraph{Free and bound names and alpha-equivalence.} At the
core of structural equivalence is alpha-equivalence which identifies
process that are the same up to a change of variable. Formally, we
recognize the distinction between free and bound names. The free names
of a process, $\freenames{P}$, may be calculated recursively as
follows:

\begin{mathpar}
\freenames{\pzero} := \emptyset
  \and \\
  \freenames{x?(y).P} := \{ x \} \cup (\freenames{P} \setminus \{ y \})
  \and 
  \freenames{x!\langle P \rangle} := \{ x \} \cup \{ P \} 
  \and \\
  \freenames{P|Q} := \freenames{P} \cup \freenames{Q}
  \and \\
  \freenames{@{x}} := \{ x \}
\end{mathpar}

$\pi$
$\quotep{\pi}$

$\freenames{-} : \pi \to \mathcal{P}(\quotep{\pi})$

\begin{eqnarray*}
  \freenames{\pzero} & := & \emptyset \\
  \freenames{x?(y).P} & := & \{ x \} \cup (\freenames{P} \setminus \{ y \}) \\
  \freenames{x!\langle P \rangle} & := & \{ x \} \cup \{ P \} \\
  \freenames{P|Q} & := & \freenames{P} \cup \freenames{Q} \\
  \freenames{\dropn{x}} & := & \{ x \}
\end{eqnarray*}

The bound names of a process, $\boundnames{P}$, are those names occurring in $P$
that are not free. For example, in $x?(y).0$, the name $x$ is free, while $y$ is bound.

\begin{mathpar}
  \inferrule* [lab=monoidal-laws] {} { P|Q \equiv Q|P \and P|0 \equiv P \and P|(Q|R) \equiv (P|Q)|R }
\end{mathpar}

\begin{mathpar}
  \inferrule* [lab=alpha-equivalence] {} { (x)P \equiv (y)P\{y/x\} \and y \not\in \freenames{P} }
\end{mathpar}

\begin{definition}
Then two processes, $P,Q$, are alpha-equivalent if $P = Q\{\vec{y}/\vec{x}\}$ for
some $\vec{x} \in \boundnames{Q},\vec{y} \in \boundnames{P}$, where $Q\{\vec{y}/\vec{x}\}$
denotes the capture-avoiding substitution of $\vec{y}$ for $\vec{x}$ in $Q$.
\end{definition}

\begin{definition}
  The {\em structural congruence} \cite{SangiorgiWalker} , $\equiv$,
  between processes is the least congruence containing
  alpha-equivalence, satisfying the abelian monoid laws
  (associativity, commutativity and $\pzero$ as identity) for parallel
  composition $|$ and for summation $+$.
\end{definition}

\subsection{Name equivalence}

We take name equivalence, written $\nameeq$, to be the smallest
equivalence relation generated by the following rules.

\begin{mathpar}
\inferrule*[lab=Quote-drop]
{ }
{ \quotep{@{x}} \nameeq x }

\inferrule*[lab=Struct-equiv]
{ P \scong Q }
{ \quotep{P} \nameeq \quotep{Q} }
\end{mathpar}

The astute reader will have noticed that the mutual recursion of names
and processes imposes a mutual recursion on alpha-equivalence and
structural equivalence via name-equivalence. Fortunately, all of this
works out pleasantly and we may calculate in the natural way, free of
concern. The reader interested in the details is referred to the
appendix \ref{appendix:rho_details}.

\subsection{Substitution}

We use $\Proc$ for the set of processes, $\QProc$ for the set of
names, and $\id{\{}\vec{y} / \vec{x} \id{\}}$ to denote partial maps,
$s : \QProc \rightarrow \QProc$. A map, $s$ lifts, uniquely, to a map
on process terms, $\widehat{s} : \Proc \rightarrow \Proc$ by the
following equations.

\begin{mathpar}
  (0) \psubstp{Q}{P} := 0 \\
  (R \juxtap S) \psubstp{Q}{P}
  :=    
  (R)\psubstp{Q}{P} \juxtap (S) \psubstp{Q}{P} \\
  (x?(y).R) \psubstp{Q}{P}    
  :=    
  (x)\substp{Q}{P} (z)\concat( (R \psubstn{z}{y}) \psubstp{Q}{P} ) \\
  (\lift{x}{R}) \psubstp{Q}{P}  
  :=
  \lift{(x)\substp{Q}{P}}{ R \psubstp{Q}{P} } \\
%   (\dropn{x})  \psubstp{Q}{P}       
%   := 
%   \left\{ 
%     \begin{array}{ccc} 
%       \dropn{\quotep{Q}} & & x \nameeq \quotep{P} \\
%       \dropn{x} & & otherwise \\
%     \end{array}
%   \right. 
  (\dropn{x})  \psubstp{Q}{P}       
  := 
  \left\{ 
    \begin{array}{ccc} 
      Q & & x \nameeq \quotep{P} \\
      \dropn{x} & & otherwise \\
    \end{array}
  \right.
\end{mathpar}
 

where

\begin{eqnarray}
  (x)\id{\{} \lpquote Q \rpquote / \lpquote P \rpquote \id{\}}            = 
  \left\{ 
    \begin{array}{ccc}
      \lpquote Q \rpquote & & x \nameeq \lpquote P \rpquote \\
      x & & otherwise \\
    \end{array}
  \right. \nonumber
\end{eqnarray}

and $z$ is chosen distinct from $\quotep{P}$, $\quotep{Q}$, the free
names in $Q$, and all the names in $R$. Our $\alpha$-equivalence will
be built in the standard way from this substitution.

\begin{remark}\label{rem:no_self_referential_names}
  One consequence of these definitions is that $\forall P. \quotep{P}
  \not\in \freenames{P}$.
\end{remark}

\subsection{ Dynamic quote: an example }

Anticipating something of what's to come, consider applying the
substitution, $\widehat{\id{\{}u / z \id{\}}}$, to the following pair
of processes, $\lift{w}{y!(z)}$ and $w[ \lpquote y!(z) \rpquote ]$.

\begin{eqnarray}
	\lift{w}{y!(z)}\widehat{\id{\{}u / z \id{\}}}
		& = &
		\lift{w}{y!(u)} \nonumber\\
	w[ \lpquote y!(z) \rpquote ] \widehat{ \id{\{}u / z \id{\}} }
		& = &
		w[ \lpquote y!(z) \rpquote ] \nonumber
\end{eqnarray}

Because the body of the process between quotes is impervious to
substitution, we get radically different answers. In fact, by
examining the first process in an input context,
e.g. $x?(z).\lift{w}{y!(z)}$, we see that the process under the lift
operator may be shaped by prefixed inputs binding a name inside it. In
this sense, the lift operator will be seen as a way to dynamically
construct processes before reifying them as names.

Finally equipped with these standard features we can present the
dynamics of the calculus.

\subsubsection{Operational semantics} 

Finally, we introduce the computational dynamics. What marks these
algebras as distinct from other more traditionally studied algebraic
structures, e.g. vector spaces or polynomial rings, is the manner in
which dynamics is captured. In traditional structures, dynamics is typically
expressed through morphisms between such structures, as in linear maps
between vector spaces or morphisms between rings. In algebras
associated with the semantics of computation, the dynamics is
expressed as part of the algebraic structure itself, through a
reduction reduction relation typically denoted by $\red$. Below, we
give a recursive presentation of this relation for the calculus used
in the encoding.

$\red \subseteq \pi \times \pi$
$\red : \pi \to \mathcal{P}(\pi)$

\begin{mathpar}
  \inferrule* [lab=Comm] { \textsf{match}( x_{src}, x_{trgt} ) } { x_{trgt}?(y)P \; | \; x_{src}!\langle {Q} \rangle \red P\{\quotep{Q}/y}\} }
  \and \\
  \inferrule* [lab=Par] {{P} \red {P}'} {{{P} | {Q}} \red {{P}' | {Q}}}
  \and
  \inferrule* [lab=Equiv]{{{P} \scong {P}'} \andalso {{P}' \red {Q}'} \andalso {{Q}' \scong {Q}}}{{P} \red {Q}}
\end{mathpar}

\begin{eqnarray*}
  match_{\equiv} (\quotep{P},\quotep{Q}) & := & P \equiv Q \\
  match_{\dagger}(\quotep{P},\quotep{Q}) & := & \forall R. P|Q \red^{*} R => R \red^{*} 0 \\
  match_{K}(\quotep{P},\quotep{Q}) & := & K \mbox{ for some context } K
\end{eqnarray*}

$u?(x)P | u!\langle Q \rangle \red P\{\quotep{Q}/x\}$

%We write $\wred$ for $\red^*$, and $P\red$ if $\exists Q $ such that $ P \red Q$.
We write $P\red$ if $\exists Q $ such that $ P \red Q$ and $P\not\red$, otherwise.

\section{Replication}

As mentioned before, it is known that replication (and hence
recursion) can be implemented in a higher-order process algebra
\cite{SangiorgiWalker}. As our first example of calculation with the
machinery thus far presented we give the construction explicitly in
the {\rhoc}.

\begin{eqnarray}
	D_{x} & := & \prefix{x}{y}{(\binpar{\outputp{x}{y}}{@{y}})} \nonumber\\
	\bangp_{x}{P} & := & \binpar{{x}!\langle{\binpar{D_{x}}{P}}\rangle}{D_{x}} \nonumber
\end{eqnarray}

\begin{eqnarray}
	\bangp_{x}{P} & & \nonumber\\
	=
	& {x}!\langle{(\prefix{x}{y}{(\outputp{x}{y} | @{y})) | P}}\rangle 
	      | \prefix{x}{y}{(\outputp{x}{y} | @{y})} & \nonumber\\
	\red
	& (\outputp{x}{y} | @{y})\substn{\quotep{(\prefix{x}{y}{(@{y} | \outputp{x}{y})) | P}}}{y} & \nonumber\\
	=
	& \outputp{x}{\quotep{(\prefix{x}{y}{(\outputp{x}{y} | @{y})) | P}}}
	  | {(\prefix{x}{y}{(\outputp{x}{y} | @{y})) | P}} & \nonumber\\
	\red
	& \ldots & \nonumber\\
	\red^*
	& P | P | \ldots & \nonumber
\end{eqnarray}

Of course, this encoding, as an implementation, runs away, unfolding
$\bangp{P}$ eagerly. A lazier and more implementable replication
operator, restricted to input-guarded processes, may be obtained as follows.

\begin{eqnarray}
\bangp{\prefix{u}{v}{P}} 
	:= 
	\binpar{\lift{x}{\prefix{u}{v}{(\binpar{D(x)}{P})}}}{D(x)} \nonumber
\end{eqnarray}

\begin{remark}
  Note that the lazier definition still does not deal with summation
  or mixed summation (i.e. sums over input and output). The reader is
  invited to construct definitions of replication that deal with these
  features. 

  Further, the definitions are parameterized in a name, $x$. Can you,
  gentle reader, make a definition that eliminates this parameter and
  guarantees no accidental interaction between the replication
  machinery and the process being replicated -- i.e. no accidental
  sharing of names used by the process to get its work done and the
  name(s) used by the replication to effect copying. This latter
  revision of the definition of replication is crucial to obtaining
  the expected identity $!!P \sim !P$.
\end{remark}

\begin{remark}\label{rem:paradoxical_combinator}
  The reader familiar with the lambda calculus will have noticed the
  similarity between $D$ and the paradoxical combinator.

  [Ed. note: the existence of this seems to suggest we have to be more
  restrictive on the set of processes and names we admit if we are to
  support no-cloning.]
\end{remark}

\subsubsection{Bisimulation}

The computational dynamics gives rise to another kind of equivalence,
the equivalence of computational behavior. As previously mentioned
this is typically captured \emph{via} some form of bisimulation.

% The notion we use in this paper is weak barbed bisimulation
% \cite{milner91polyadicpi}.

The notion we use in this paper is derived from weak barbed
bisimulation \cite{milner91polyadicpi}. 

\begin{definition}
An \emph{observation relation}, $\downarrow_{\mathcal N}$, over a set
of names, $\mathcal N$, is the smallest relation satisfying the rules
below.

\infrule[Out-barb]{y \in {\mathcal N}, \; x \nameeq y}
		  {\outputp{x}{v} \downarrow_{\mathcal N} x}
\infrule[Par-barb]{\mbox{$P\downarrow_{\mathcal N} x$ or $Q\downarrow_{\mathcal N} x$}}
		  {\binpar{P}{Q} \downarrow_{\mathcal N} x}

We write $P \Downarrow_{\mathcal N} x$ if there is $Q$ such that 
$P \wred Q$ and $Q \downarrow_{\mathcal N} x$.
\end{definition}

\begin{definition}
%\label{def.bbisim}
An  ${\mathcal N}$-\emph{barbed bisimulation} over a set of names, ${\mathcal N}$, is a symmetric binary relation 
${\mathcal S}_{\mathcal N}$ between agents such that $P\rel{S}_{\mathcal N}Q$ implies:
\begin{enumerate}
\item If $P \red P'$ then $Q \wred Q'$ and $P'\rel{S}_{\mathcal N} Q'$.
\item If $P\downarrow_{\mathcal N} x$, then $Q\Downarrow_{\mathcal N} x$.
\end{enumerate}
$P$ is ${\mathcal N}$-barbed bisimilar to $Q$, written
$P \wbbisim_{\mathcal N} Q$, if $P \rel{S}_{\mathcal N} Q$ for some ${\mathcal N}$-barbed bisimulation ${\mathcal S}_{\mathcal N}$.
\end{definition}

$\mathcal{R} \subseteq \pi \times \pi$

$P \mathcal{R} Q => \forall P'. P \red P' \Rightarrow \exists Q'. Q \red Q', P' \mathcal{R} Q'$

$P \vdash x \Rightarrow Q \vdash x$

\begin{mathpar}
  \inferrule*[lab=Out-barb]{x \nameeq y}{{y}!\langle{Q}\rangle \vdash x}
  \and
  \inferrule*[lab=Par-barb]{\mbox{$P\vdash x$ or $Q\vdash x$}}{\binpar{P}{Q} \vdash x}
\end{mathpar}

\subsubsection{Contexts}

One of the principle advantages of computational calculi like the
$\pi$-calculus is a well-defined notion of context,
contextual-equivalence and a correlation between
contextual-equivalence and notions of bisimulation. The notion of
context allows the decomposition of a process into (sub-)process and
its syntactic environment, its context. Thus, a context may be
thought of as a process with a ``hole'' (written $\Box$) in it. The
application of a context $M$ to a process $P$, written $M[P]$, is
tantamount to filling the hole in $M$ with $P$. In this paper we do
not need the full weight of this theory, but do make use of the notion
of context in the proof the main theorem. 

\begin{mathpar}
  \inferrule* [lab=summation] {} {{M_{M},M_{N}} \bc \Box \;|\; x.M_{A} \;|\; M_{M}+M_{N}}
  \and
  \inferrule* [lab=agent] {} {{M_{A}} \bc (\vec{x})M_{P} \;| \; \clift{P_0,\ldots,M_{P},\ldots,P_N}}
  \and \\
  \inferrule* [lab=process] {} {{M_{P}} \bc M_{N} \;| \;P|M_{P} }
\end{mathpar} 

\begin{mathpar}
  \inferrule* [lab=sychronization] {} {M_{N} \bc \Box \;|\; x?M_{F} \;|\; x!M_{C}}
  \and
  \inferrule* [lab=abstraction] {} {{M_{F}} \bc (x)M_{P} }
  \and
  \inferrule* [lab=concretion] {} {{M_{C}} \bc \langle M_{P} \rangle }
  \and \\
  \inferrule* [lab=process] {} {{M_{P}} \bc M_{N} \;| \;P|M_{P} }
\end{mathpar}

\begin{definition}[contextual application] Given a context $M$, and
  process $P$, we define the \emph{contextual application}, $M[P] :=
  M\{P/\Box\}$. That is, the contextual application of M to P is the
  substitution of $P$ for $\Box$ in $M$.
\end{definition}

$\meaningof{-} : L \to \mathcal{P}(\pi)$

\begin{mathpar}
  \inferrule* [lab=collection] {} {\meaningof{true} = \pi, \and \meaningof{~E} = \pi \setminus \meaningof{E}, \and \meaningof{E_{1} \& E_{2}} = \meaningof{E_{1}} \cap \meaningof{E_{2}}}
\end{mathpar}

\begin{mathpar}
  \inferrule* [lab=structure] {} {\meaningof{0} = \{ P \in \pi | P \equiv 0 \}, \and \\ \meaningof{E_1 | E_2} = \{ P \in \pi | P \equiv P_{1} | P_{2}, P_{1} \in \meaningof{E_{1}}, P_{2} \in \meaningof{E_2}\} }
\end{mathpar}

\begin{mathpar}
 \inferrule* [lab=behavior] {} {\meaningof{\langle a?b \rangle E} = \{ P \in \pi | P \equiv Q | u?(y)P', \\ \and \\\\ \and \\ \;\;\; u \in \meaningof{a}, \forall z.P'\{z/y\} \in \meaningof{E\{z/b\}}\}, \and \\ \meaningof{a!E} = \{ P \in \pi | P \equiv Q | x!\langle P' \rangle, x \in \meaningof{a} P' \in \meaningof{E}\} }
\end{mathpar}

\begin{mathpar}
 \inferrule* [lab=nominal] {} {\meaningof{\quotep{E}} = \{ \quotep{P} \in \quotep{\pi} | P \in \meaningof{E} \}, \and \meaningof{\quotep{P}} = \{ \quotep{Q} \in \quotep{\pi} | P \equiv Q \} \and \\ \meaningof{@\quotep{E}} = \{ P \in \pi | P \equiv @x, x \in \meaningof{E} \}}
\end{mathpar}

\begin{eqnarray*}
  \\
  \meaningof{-} : TS \to ST
\end{eqnarray*}

\begin{eqnarray*}
  \\
  L : TS \to ST
\end{eqnarray*}

\begin{eqnarray*}
  \\
  P \models E \iff P \in \meaningof{E}
\end{eqnarray*}

\begin{eqnarray*}
  P \approx_{L} Q \iff \forall E \in L. P \models E \iff Q \models E
\end{eqnarray*}

\begin{eqnarray*}
  P \approx_{K} Q
\end{eqnarray*}

\begin{eqnarray*}
  P \approx Q
\end{eqnarray*}

$\approx_{K} = \approx = \approx_{L}$

\subsubsection{Contextual duality}

Note that contexts extend the quotation operation to a family of
operations from processes to names. Given a context, $M$, we can
define a \emph{nominal context}, $\quotep{M}$ by $\quotep{M}[P] :=
\quotep{M[P]}$. To foreshadow what is to come we observe that these
operations enjoy a duality with processes very much like the duality
between vectors and maps from vectors to scalars.

Further, because the calculus is essentially higher-order, we have a
correspondence between contexts and processes. More specifically,
given a name $x$ and a context $M$ we can construct $M^{*}_{x}$ such
that 

\begin{mathpar}
  M^{*}_{x} | \lift{x}{P} \red M[P]
\end{mathpar}

namely,

\begin{mathpar}
  M^{*}_{x} := x?(u).M[\dropn{u}]
\end{mathpar}

The dependence of $M^{*}_{x}$ on a name makes it an abstraction, 

\begin{mathpar}
  M^{*} := (x)x?(u).M[\dropn{u}]
\end{mathpar}

\subsection{Additional notation}

It will sometimes be convenient to denote the process a name
quotes. We already have the notation $x = \quotep{P}$, but it will be
convenient to introduce an alternate notation, $\procn{x}$, when we
want to emphasize the connection to the use of the name. Note that, by
virtue of name equivalence, $\quotep{\procn{x}} \nameeq x$; so, the
notation is consistent with previous definitions.

Further, because names have structure it is possible to effect
substitutions on the basis of that structure. This means we need to
upgrade our notation for substitutions, which we accomplish by
adapting comprehension notation. Thus,

\begin{mathpar}
  P\{ y / x : x \in S \}
\end{mathpar}

is interpreted to mean the process derived from P by replacing (in a
capture-avoiding manner) each occurrence of $x$ in $S$ by $y$. For example,

\begin{mathpar}
  P\{ \quotep{\procn{x}|\procn{x}} / x : x \in \freenames{P} \}
\end{mathpar}

will replace each (occurrence) of a free name $x$ in $P$ by
$\quotep{\procn{x}|\procn{x}}$.

Also, we will avail ourselves of the notation $x^{L}$ and $x^{R}$ to
denote injections of a name into disjoint copies of the name
space. There are numerous ways to accomplish this. One example can be
found in \cite{MeredithR05}. This notation overloads to vectors of
names: $\vec{x}^{\pi} := (x_{i}^{\pi} \; : \; 0 \leq i < |\vec{x}| )$ where $\pi \in \{L,R\}$.

We also use $P^{\Box} := P|\Box$.

In \cite{MeredithR05} an interpretation of the new operator is
given. It turns out that there are several possible interpretations
all enjoying the requisite algebraic properties of the operator (see
\cite{milner91polyadicpi}). We will therefore make liberal use of
$(\nu\; \vec{x})P$.

% subsection the_syntax_and_semantics_of_the_notation_system (end)   

\input{qm2pi.qmops} 

\input{qm2pi.sterngerlach} 

\input{qm2pi.metric} 

% section concurrent_process_calculi (end)

%\input{qm2pi.proofsketch}

% section proof sketch (end)

%\input{qm2pi.slviaknots} 

% section spatial logic via knots (end)

\input{qm2pi.conclusion}

% section conclusion (end)

%\input{qm2pi.dtcodes} 

% section wiring algorithm (end)

\input{qm2pi.ack} 

% section acknowledgments (end)

\newpage


\bibliographystyle{plain}   
\bibliography{../../biblios/main.bib}

\input{qm2pi.rhodetails}

\end{document}

 

% section notation (end)

\input{qm2pi.process.calculi} 

% section concurrent_process_calculi_and_spatial_logics_ (end)
    
%\documentclass[12pt]{llncs}
%\documentclass{jktr}

\usepackage[pdftex]{hyperref}                   
\usepackage {listings}
\usepackage {mathpartir}
\usepackage{bcprules}
%\usepackage{listings}
                       
\usepackage{graphicx} 
%\usepackage[margins=2.5cm,nohead,nofoot]{geometry}
%\usepackage{geometry}
\usepackage{amsfonts}
\usepackage{amstext}
\usepackage{latexsym}
\usepackage{amssymb}
\usepackage{color}


%\include{myPreamble}
\include{qm2pi.local} 

%\ifpdf
%\usepackage[pdftex]{graphicx}
%\else
%\usepackage{graphicx}
%\fi

 % \ifpdf
%  \usepackage{pdfsync}
%  \if


%\title{Brief Article}
%\author{David F. Snyder}
%\author{L.G. Meredith}

%\address{Dept. of Math., Texas State University--San Marcos, San Marcos, TX 78666}
       
\pagestyle{empty}


\begin{document}

\lstset{language=[Objective]Caml,frame=shadowbox}

\input{qm2pi.front}

% section front matter (end)

\input{qm2pi.intro} 
 
% section introduction (end)

% \input{qm2pi.knotations} 

% section notation (end)

\input{qm2pi.process.calculi} 

% section concurrent_process_calculi_and_spatial_logics_ (end)
    
%\input{qm2pi.knots2pi} 

%\input{qm2pi.trefoil} 

%\input{qm2pi.mainthm} 

% subsection basic_interpretation (end)

%\input{qm2pi.rho.presentation} 
\subsection{The syntax and semantics of the notation system}\label{sub:the_syntax_and_semantics_of_the_notation_system} % (fold)

We now summarize a technical presentation of the calculus that
embodies our theory of dynamics. The typical presentation of such a
calculus follows the style of giving generators and relations on
them. The grammar, below, describing term constructors, freely
generates the set of processes, $\Proc$. This set is then quotiented
by a relation known as structural congruence and it is over this set
that the notion of dynamics is expressed. This presentation is
essentially that of \cite{MeredithR05} with the addition of
polyadicity and summation. For readability we have relegated some of
the technical subtleties to an appendix.

\subsubsection{Process grammar}\label{subsub:process_grammar}

\begin{mathpar}
  \inferrule* [lab=synchronization] {} {{M} \bc \pzero \;|\; x?F \;|\; x!C }
  \and
  \inferrule* [lab=abstraction] {} {{F} \bc (x)P}
  \and
  \inferrule* [lab=concretion] {} {{C} \bc \langle Q \rangle}
  \and
  \inferrule* [lab=process] {} {{P,Q} \bc M \;| \;P|Q \;|\; @{x}}
  \and
  \inferrule* [lab=name] {} {{x} \bc \quotep{P}}
\end{mathpar} 

Note that $\vec{x}$ (resp. $\vec{P}$) denotes a vector of names
(resp. processes) of length $|\vec{x}|$ (resp. $|\vec{P}|$). We adopt
the following useful abbreviations.

\begin{mathpar}
   x?(\vec{y}).P := x.(\vec{y})P \and  x\clift{\vec{P}} := x.\clift{\vec{P}}
   \and x!(y) := \lift{x}{\dropn{y}}
   \and \Pi_{i=0}^{n-1}P_i := P_0 | \ldots | P_{n-1}
\end{mathpar}

\subsubsection{Structural congruence}

\paragraph{Free and bound names and alpha-equivalence.} At the
core of structural equivalence is alpha-equivalence which identifies
process that are the same up to a change of variable. Formally, we
recognize the distinction between free and bound names. The free names
of a process, $\freenames{P}$, may be calculated recursively as
follows:

\begin{mathpar}
\freenames{\pzero} := \emptyset
  \and \\
  \freenames{x?(y).P} := \{ x \} \cup (\freenames{P} \setminus \{ y \})
  \and 
  \freenames{x!\langle P \rangle} := \{ x \} \cup \{ P \} 
  \and \\
  \freenames{P|Q} := \freenames{P} \cup \freenames{Q}
  \and \\
  \freenames{@{x}} := \{ x \}
\end{mathpar}

$\pi$
$\quotep{\pi}$

$\freenames{-} : \pi \to \mathcal{P}(\quotep{\pi})$

\begin{eqnarray*}
  \freenames{\pzero} & := & \emptyset \\
  \freenames{x?(y).P} & := & \{ x \} \cup (\freenames{P} \setminus \{ y \}) \\
  \freenames{x!\langle P \rangle} & := & \{ x \} \cup \{ P \} \\
  \freenames{P|Q} & := & \freenames{P} \cup \freenames{Q} \\
  \freenames{\dropn{x}} & := & \{ x \}
\end{eqnarray*}

The bound names of a process, $\boundnames{P}$, are those names occurring in $P$
that are not free. For example, in $x?(y).0$, the name $x$ is free, while $y$ is bound.

\begin{mathpar}
  \inferrule* [lab=monoidal-laws] {} { P|Q \equiv Q|P \and P|0 \equiv P \and P|(Q|R) \equiv (P|Q)|R }
\end{mathpar}

\begin{mathpar}
  \inferrule* [lab=alpha-equivalence] {} { (x)P \equiv (y)P\{y/x\} \and y \not\in \freenames{P} }
\end{mathpar}

\begin{definition}
Then two processes, $P,Q$, are alpha-equivalent if $P = Q\{\vec{y}/\vec{x}\}$ for
some $\vec{x} \in \boundnames{Q},\vec{y} \in \boundnames{P}$, where $Q\{\vec{y}/\vec{x}\}$
denotes the capture-avoiding substitution of $\vec{y}$ for $\vec{x}$ in $Q$.
\end{definition}

\begin{definition}
  The {\em structural congruence} \cite{SangiorgiWalker} , $\equiv$,
  between processes is the least congruence containing
  alpha-equivalence, satisfying the abelian monoid laws
  (associativity, commutativity and $\pzero$ as identity) for parallel
  composition $|$ and for summation $+$.
\end{definition}

\subsection{Name equivalence}

We take name equivalence, written $\nameeq$, to be the smallest
equivalence relation generated by the following rules.

\begin{mathpar}
\inferrule*[lab=Quote-drop]
{ }
{ \quotep{@{x}} \nameeq x }

\inferrule*[lab=Struct-equiv]
{ P \scong Q }
{ \quotep{P} \nameeq \quotep{Q} }
\end{mathpar}

The astute reader will have noticed that the mutual recursion of names
and processes imposes a mutual recursion on alpha-equivalence and
structural equivalence via name-equivalence. Fortunately, all of this
works out pleasantly and we may calculate in the natural way, free of
concern. The reader interested in the details is referred to the
appendix \ref{appendix:rho_details}.

\subsection{Substitution}

We use $\Proc$ for the set of processes, $\QProc$ for the set of
names, and $\id{\{}\vec{y} / \vec{x} \id{\}}$ to denote partial maps,
$s : \QProc \rightarrow \QProc$. A map, $s$ lifts, uniquely, to a map
on process terms, $\widehat{s} : \Proc \rightarrow \Proc$ by the
following equations.

\begin{mathpar}
  (0) \psubstp{Q}{P} := 0 \\
  (R \juxtap S) \psubstp{Q}{P}
  :=    
  (R)\psubstp{Q}{P} \juxtap (S) \psubstp{Q}{P} \\
  (x?(y).R) \psubstp{Q}{P}    
  :=    
  (x)\substp{Q}{P} (z)\concat( (R \psubstn{z}{y}) \psubstp{Q}{P} ) \\
  (\lift{x}{R}) \psubstp{Q}{P}  
  :=
  \lift{(x)\substp{Q}{P}}{ R \psubstp{Q}{P} } \\
%   (\dropn{x})  \psubstp{Q}{P}       
%   := 
%   \left\{ 
%     \begin{array}{ccc} 
%       \dropn{\quotep{Q}} & & x \nameeq \quotep{P} \\
%       \dropn{x} & & otherwise \\
%     \end{array}
%   \right. 
  (\dropn{x})  \psubstp{Q}{P}       
  := 
  \left\{ 
    \begin{array}{ccc} 
      Q & & x \nameeq \quotep{P} \\
      \dropn{x} & & otherwise \\
    \end{array}
  \right.
\end{mathpar}
 

where

\begin{eqnarray}
  (x)\id{\{} \lpquote Q \rpquote / \lpquote P \rpquote \id{\}}            = 
  \left\{ 
    \begin{array}{ccc}
      \lpquote Q \rpquote & & x \nameeq \lpquote P \rpquote \\
      x & & otherwise \\
    \end{array}
  \right. \nonumber
\end{eqnarray}

and $z$ is chosen distinct from $\quotep{P}$, $\quotep{Q}$, the free
names in $Q$, and all the names in $R$. Our $\alpha$-equivalence will
be built in the standard way from this substitution.

\begin{remark}\label{rem:no_self_referential_names}
  One consequence of these definitions is that $\forall P. \quotep{P}
  \not\in \freenames{P}$.
\end{remark}

\subsection{ Dynamic quote: an example }

Anticipating something of what's to come, consider applying the
substitution, $\widehat{\id{\{}u / z \id{\}}}$, to the following pair
of processes, $\lift{w}{y!(z)}$ and $w[ \lpquote y!(z) \rpquote ]$.

\begin{eqnarray}
	\lift{w}{y!(z)}\widehat{\id{\{}u / z \id{\}}}
		& = &
		\lift{w}{y!(u)} \nonumber\\
	w[ \lpquote y!(z) \rpquote ] \widehat{ \id{\{}u / z \id{\}} }
		& = &
		w[ \lpquote y!(z) \rpquote ] \nonumber
\end{eqnarray}

Because the body of the process between quotes is impervious to
substitution, we get radically different answers. In fact, by
examining the first process in an input context,
e.g. $x?(z).\lift{w}{y!(z)}$, we see that the process under the lift
operator may be shaped by prefixed inputs binding a name inside it. In
this sense, the lift operator will be seen as a way to dynamically
construct processes before reifying them as names.

Finally equipped with these standard features we can present the
dynamics of the calculus.

\subsubsection{Operational semantics} 

Finally, we introduce the computational dynamics. What marks these
algebras as distinct from other more traditionally studied algebraic
structures, e.g. vector spaces or polynomial rings, is the manner in
which dynamics is captured. In traditional structures, dynamics is typically
expressed through morphisms between such structures, as in linear maps
between vector spaces or morphisms between rings. In algebras
associated with the semantics of computation, the dynamics is
expressed as part of the algebraic structure itself, through a
reduction reduction relation typically denoted by $\red$. Below, we
give a recursive presentation of this relation for the calculus used
in the encoding.

$\red \subseteq \pi \times \pi$
$\red : \pi \to \mathcal{P}(\pi)$

\begin{mathpar}
  \inferrule* [lab=Comm] { \textsf{match}( x_{src}, x_{trgt} ) } { x_{trgt}?(y)P \; | \; x_{src}!\langle {Q} \rangle \red P\{\quotep{Q}/y}\} }
  \and \\
  \inferrule* [lab=Par] {{P} \red {P}'} {{{P} | {Q}} \red {{P}' | {Q}}}
  \and
  \inferrule* [lab=Equiv]{{{P} \scong {P}'} \andalso {{P}' \red {Q}'} \andalso {{Q}' \scong {Q}}}{{P} \red {Q}}
\end{mathpar}

\begin{eqnarray*}
  match_{\equiv} (\quotep{P},\quotep{Q}) & := & P \equiv Q \\
  match_{\dagger}(\quotep{P},\quotep{Q}) & := & \forall R. P|Q \red^{*} R => R \red^{*} 0 \\
  match_{K}(\quotep{P},\quotep{Q}) & := & K \mbox{ for some context } K
\end{eqnarray*}

$u?(x)P | u!\langle Q \rangle \red P\{\quotep{Q}/x\}$

%We write $\wred$ for $\red^*$, and $P\red$ if $\exists Q $ such that $ P \red Q$.
We write $P\red$ if $\exists Q $ such that $ P \red Q$ and $P\not\red$, otherwise.

\section{Replication}

As mentioned before, it is known that replication (and hence
recursion) can be implemented in a higher-order process algebra
\cite{SangiorgiWalker}. As our first example of calculation with the
machinery thus far presented we give the construction explicitly in
the {\rhoc}.

\begin{eqnarray}
	D_{x} & := & \prefix{x}{y}{(\binpar{\outputp{x}{y}}{@{y}})} \nonumber\\
	\bangp_{x}{P} & := & \binpar{{x}!\langle{\binpar{D_{x}}{P}}\rangle}{D_{x}} \nonumber
\end{eqnarray}

\begin{eqnarray}
	\bangp_{x}{P} & & \nonumber\\
	=
	& {x}!\langle{(\prefix{x}{y}{(\outputp{x}{y} | @{y})) | P}}\rangle 
	      | \prefix{x}{y}{(\outputp{x}{y} | @{y})} & \nonumber\\
	\red
	& (\outputp{x}{y} | @{y})\substn{\quotep{(\prefix{x}{y}{(@{y} | \outputp{x}{y})) | P}}}{y} & \nonumber\\
	=
	& \outputp{x}{\quotep{(\prefix{x}{y}{(\outputp{x}{y} | @{y})) | P}}}
	  | {(\prefix{x}{y}{(\outputp{x}{y} | @{y})) | P}} & \nonumber\\
	\red
	& \ldots & \nonumber\\
	\red^*
	& P | P | \ldots & \nonumber
\end{eqnarray}

Of course, this encoding, as an implementation, runs away, unfolding
$\bangp{P}$ eagerly. A lazier and more implementable replication
operator, restricted to input-guarded processes, may be obtained as follows.

\begin{eqnarray}
\bangp{\prefix{u}{v}{P}} 
	:= 
	\binpar{\lift{x}{\prefix{u}{v}{(\binpar{D(x)}{P})}}}{D(x)} \nonumber
\end{eqnarray}

\begin{remark}
  Note that the lazier definition still does not deal with summation
  or mixed summation (i.e. sums over input and output). The reader is
  invited to construct definitions of replication that deal with these
  features. 

  Further, the definitions are parameterized in a name, $x$. Can you,
  gentle reader, make a definition that eliminates this parameter and
  guarantees no accidental interaction between the replication
  machinery and the process being replicated -- i.e. no accidental
  sharing of names used by the process to get its work done and the
  name(s) used by the replication to effect copying. This latter
  revision of the definition of replication is crucial to obtaining
  the expected identity $!!P \sim !P$.
\end{remark}

\begin{remark}\label{rem:paradoxical_combinator}
  The reader familiar with the lambda calculus will have noticed the
  similarity between $D$ and the paradoxical combinator.

  [Ed. note: the existence of this seems to suggest we have to be more
  restrictive on the set of processes and names we admit if we are to
  support no-cloning.]
\end{remark}

\subsubsection{Bisimulation}

The computational dynamics gives rise to another kind of equivalence,
the equivalence of computational behavior. As previously mentioned
this is typically captured \emph{via} some form of bisimulation.

% The notion we use in this paper is weak barbed bisimulation
% \cite{milner91polyadicpi}.

The notion we use in this paper is derived from weak barbed
bisimulation \cite{milner91polyadicpi}. 

\begin{definition}
An \emph{observation relation}, $\downarrow_{\mathcal N}$, over a set
of names, $\mathcal N$, is the smallest relation satisfying the rules
below.

\infrule[Out-barb]{y \in {\mathcal N}, \; x \nameeq y}
		  {\outputp{x}{v} \downarrow_{\mathcal N} x}
\infrule[Par-barb]{\mbox{$P\downarrow_{\mathcal N} x$ or $Q\downarrow_{\mathcal N} x$}}
		  {\binpar{P}{Q} \downarrow_{\mathcal N} x}

We write $P \Downarrow_{\mathcal N} x$ if there is $Q$ such that 
$P \wred Q$ and $Q \downarrow_{\mathcal N} x$.
\end{definition}

\begin{definition}
%\label{def.bbisim}
An  ${\mathcal N}$-\emph{barbed bisimulation} over a set of names, ${\mathcal N}$, is a symmetric binary relation 
${\mathcal S}_{\mathcal N}$ between agents such that $P\rel{S}_{\mathcal N}Q$ implies:
\begin{enumerate}
\item If $P \red P'$ then $Q \wred Q'$ and $P'\rel{S}_{\mathcal N} Q'$.
\item If $P\downarrow_{\mathcal N} x$, then $Q\Downarrow_{\mathcal N} x$.
\end{enumerate}
$P$ is ${\mathcal N}$-barbed bisimilar to $Q$, written
$P \wbbisim_{\mathcal N} Q$, if $P \rel{S}_{\mathcal N} Q$ for some ${\mathcal N}$-barbed bisimulation ${\mathcal S}_{\mathcal N}$.
\end{definition}

$\mathcal{R} \subseteq \pi \times \pi$

$P \mathcal{R} Q => \forall P'. P \red P' \Rightarrow \exists Q'. Q \red Q', P' \mathcal{R} Q'$

$P \vdash x \Rightarrow Q \vdash x$

\begin{mathpar}
  \inferrule*[lab=Out-barb]{x \nameeq y}{{y}!\langle{Q}\rangle \vdash x}
  \and
  \inferrule*[lab=Par-barb]{\mbox{$P\vdash x$ or $Q\vdash x$}}{\binpar{P}{Q} \vdash x}
\end{mathpar}

\subsubsection{Contexts}

One of the principle advantages of computational calculi like the
$\pi$-calculus is a well-defined notion of context,
contextual-equivalence and a correlation between
contextual-equivalence and notions of bisimulation. The notion of
context allows the decomposition of a process into (sub-)process and
its syntactic environment, its context. Thus, a context may be
thought of as a process with a ``hole'' (written $\Box$) in it. The
application of a context $M$ to a process $P$, written $M[P]$, is
tantamount to filling the hole in $M$ with $P$. In this paper we do
not need the full weight of this theory, but do make use of the notion
of context in the proof the main theorem. 

\begin{mathpar}
  \inferrule* [lab=summation] {} {{M_{M},M_{N}} \bc \Box \;|\; x.M_{A} \;|\; M_{M}+M_{N}}
  \and
  \inferrule* [lab=agent] {} {{M_{A}} \bc (\vec{x})M_{P} \;| \; \clift{P_0,\ldots,M_{P},\ldots,P_N}}
  \and \\
  \inferrule* [lab=process] {} {{M_{P}} \bc M_{N} \;| \;P|M_{P} }
\end{mathpar} 

\begin{mathpar}
  \inferrule* [lab=sychronization] {} {M_{N} \bc \Box \;|\; x?M_{F} \;|\; x!M_{C}}
  \and
  \inferrule* [lab=abstraction] {} {{M_{F}} \bc (x)M_{P} }
  \and
  \inferrule* [lab=concretion] {} {{M_{C}} \bc \langle M_{P} \rangle }
  \and \\
  \inferrule* [lab=process] {} {{M_{P}} \bc M_{N} \;| \;P|M_{P} }
\end{mathpar}

\begin{definition}[contextual application] Given a context $M$, and
  process $P$, we define the \emph{contextual application}, $M[P] :=
  M\{P/\Box\}$. That is, the contextual application of M to P is the
  substitution of $P$ for $\Box$ in $M$.
\end{definition}

$\meaningof{-} : L \to \mathcal{P}(\pi)$

\begin{mathpar}
  \inferrule* [lab=collection] {} {\meaningof{true} = \pi, \and \meaningof{~E} = \pi \setminus \meaningof{E}, \and \meaningof{E_{1} \& E_{2}} = \meaningof{E_{1}} \cap \meaningof{E_{2}}}
\end{mathpar}

\begin{mathpar}
  \inferrule* [lab=structure] {} {\meaningof{0} = \{ P \in \pi | P \equiv 0 \}, \and \\ \meaningof{E_1 | E_2} = \{ P \in \pi | P \equiv P_{1} | P_{2}, P_{1} \in \meaningof{E_{1}}, P_{2} \in \meaningof{E_2}\} }
\end{mathpar}

\begin{mathpar}
 \inferrule* [lab=behavior] {} {\meaningof{\langle a?b \rangle E} = \{ P \in \pi | P \equiv Q | u?(y)P', \\ \and \\\\ \and \\ \;\;\; u \in \meaningof{a}, \forall z.P'\{z/y\} \in \meaningof{E\{z/b\}}\}, \and \\ \meaningof{a!E} = \{ P \in \pi | P \equiv Q | x!\langle P' \rangle, x \in \meaningof{a} P' \in \meaningof{E}\} }
\end{mathpar}

\begin{mathpar}
 \inferrule* [lab=nominal] {} {\meaningof{\quotep{E}} = \{ \quotep{P} \in \quotep{\pi} | P \in \meaningof{E} \}, \and \meaningof{\quotep{P}} = \{ \quotep{Q} \in \quotep{\pi} | P \equiv Q \} \and \\ \meaningof{@\quotep{E}} = \{ P \in \pi | P \equiv @x, x \in \meaningof{E} \}}
\end{mathpar}

\begin{eqnarray*}
  \\
  \meaningof{-} : TS \to ST
\end{eqnarray*}

\begin{eqnarray*}
  \\
  L : TS \to ST
\end{eqnarray*}

\begin{eqnarray*}
  \\
  P \models E \iff P \in \meaningof{E}
\end{eqnarray*}

\begin{eqnarray*}
  P \approx_{L} Q \iff \forall E \in L. P \models E \iff Q \models E
\end{eqnarray*}

\begin{eqnarray*}
  P \approx_{K} Q
\end{eqnarray*}

\begin{eqnarray*}
  P \approx Q
\end{eqnarray*}

$\approx_{K} = \approx = \approx_{L}$

\subsubsection{Contextual duality}

Note that contexts extend the quotation operation to a family of
operations from processes to names. Given a context, $M$, we can
define a \emph{nominal context}, $\quotep{M}$ by $\quotep{M}[P] :=
\quotep{M[P]}$. To foreshadow what is to come we observe that these
operations enjoy a duality with processes very much like the duality
between vectors and maps from vectors to scalars.

Further, because the calculus is essentially higher-order, we have a
correspondence between contexts and processes. More specifically,
given a name $x$ and a context $M$ we can construct $M^{*}_{x}$ such
that 

\begin{mathpar}
  M^{*}_{x} | \lift{x}{P} \red M[P]
\end{mathpar}

namely,

\begin{mathpar}
  M^{*}_{x} := x?(u).M[\dropn{u}]
\end{mathpar}

The dependence of $M^{*}_{x}$ on a name makes it an abstraction, 

\begin{mathpar}
  M^{*} := (x)x?(u).M[\dropn{u}]
\end{mathpar}

\subsection{Additional notation}

It will sometimes be convenient to denote the process a name
quotes. We already have the notation $x = \quotep{P}$, but it will be
convenient to introduce an alternate notation, $\procn{x}$, when we
want to emphasize the connection to the use of the name. Note that, by
virtue of name equivalence, $\quotep{\procn{x}} \nameeq x$; so, the
notation is consistent with previous definitions.

Further, because names have structure it is possible to effect
substitutions on the basis of that structure. This means we need to
upgrade our notation for substitutions, which we accomplish by
adapting comprehension notation. Thus,

\begin{mathpar}
  P\{ y / x : x \in S \}
\end{mathpar}

is interpreted to mean the process derived from P by replacing (in a
capture-avoiding manner) each occurrence of $x$ in $S$ by $y$. For example,

\begin{mathpar}
  P\{ \quotep{\procn{x}|\procn{x}} / x : x \in \freenames{P} \}
\end{mathpar}

will replace each (occurrence) of a free name $x$ in $P$ by
$\quotep{\procn{x}|\procn{x}}$.

Also, we will avail ourselves of the notation $x^{L}$ and $x^{R}$ to
denote injections of a name into disjoint copies of the name
space. There are numerous ways to accomplish this. One example can be
found in \cite{MeredithR05}. This notation overloads to vectors of
names: $\vec{x}^{\pi} := (x_{i}^{\pi} \; : \; 0 \leq i < |\vec{x}| )$ where $\pi \in \{L,R\}$.

We also use $P^{\Box} := P|\Box$.

In \cite{MeredithR05} an interpretation of the new operator is
given. It turns out that there are several possible interpretations
all enjoying the requisite algebraic properties of the operator (see
\cite{milner91polyadicpi}). We will therefore make liberal use of
$(\nu\; \vec{x})P$.

% subsection the_syntax_and_semantics_of_the_notation_system (end)   

\input{qm2pi.qmops} 

\input{qm2pi.sterngerlach} 

\input{qm2pi.metric} 

% section concurrent_process_calculi (end)

%\input{qm2pi.proofsketch}

% section proof sketch (end)

%\input{qm2pi.slviaknots} 

% section spatial logic via knots (end)

\input{qm2pi.conclusion}

% section conclusion (end)

%\input{qm2pi.dtcodes} 

% section wiring algorithm (end)

\input{qm2pi.ack} 

% section acknowledgments (end)

\newpage


\bibliographystyle{plain}   
\bibliography{../../biblios/main.bib}

\input{qm2pi.rhodetails}

\end{document}

 

%\documentclass[12pt]{llncs}
%\documentclass{jktr}

\usepackage[pdftex]{hyperref}                   
\usepackage {listings}
\usepackage {mathpartir}
\usepackage{bcprules}
%\usepackage{listings}
                       
\usepackage{graphicx} 
%\usepackage[margins=2.5cm,nohead,nofoot]{geometry}
%\usepackage{geometry}
\usepackage{amsfonts}
\usepackage{amstext}
\usepackage{latexsym}
\usepackage{amssymb}
\usepackage{color}


%\include{myPreamble}
\include{qm2pi.local} 

%\ifpdf
%\usepackage[pdftex]{graphicx}
%\else
%\usepackage{graphicx}
%\fi

 % \ifpdf
%  \usepackage{pdfsync}
%  \if


%\title{Brief Article}
%\author{David F. Snyder}
%\author{L.G. Meredith}

%\address{Dept. of Math., Texas State University--San Marcos, San Marcos, TX 78666}
       
\pagestyle{empty}


\begin{document}

\lstset{language=[Objective]Caml,frame=shadowbox}

\input{qm2pi.front}

% section front matter (end)

\input{qm2pi.intro} 
 
% section introduction (end)

% \input{qm2pi.knotations} 

% section notation (end)

\input{qm2pi.process.calculi} 

% section concurrent_process_calculi_and_spatial_logics_ (end)
    
%\input{qm2pi.knots2pi} 

%\input{qm2pi.trefoil} 

%\input{qm2pi.mainthm} 

% subsection basic_interpretation (end)

%\input{qm2pi.rho.presentation} 
\subsection{The syntax and semantics of the notation system}\label{sub:the_syntax_and_semantics_of_the_notation_system} % (fold)

We now summarize a technical presentation of the calculus that
embodies our theory of dynamics. The typical presentation of such a
calculus follows the style of giving generators and relations on
them. The grammar, below, describing term constructors, freely
generates the set of processes, $\Proc$. This set is then quotiented
by a relation known as structural congruence and it is over this set
that the notion of dynamics is expressed. This presentation is
essentially that of \cite{MeredithR05} with the addition of
polyadicity and summation. For readability we have relegated some of
the technical subtleties to an appendix.

\subsubsection{Process grammar}\label{subsub:process_grammar}

\begin{mathpar}
  \inferrule* [lab=synchronization] {} {{M} \bc \pzero \;|\; x?F \;|\; x!C }
  \and
  \inferrule* [lab=abstraction] {} {{F} \bc (x)P}
  \and
  \inferrule* [lab=concretion] {} {{C} \bc \langle Q \rangle}
  \and
  \inferrule* [lab=process] {} {{P,Q} \bc M \;| \;P|Q \;|\; @{x}}
  \and
  \inferrule* [lab=name] {} {{x} \bc \quotep{P}}
\end{mathpar} 

Note that $\vec{x}$ (resp. $\vec{P}$) denotes a vector of names
(resp. processes) of length $|\vec{x}|$ (resp. $|\vec{P}|$). We adopt
the following useful abbreviations.

\begin{mathpar}
   x?(\vec{y}).P := x.(\vec{y})P \and  x\clift{\vec{P}} := x.\clift{\vec{P}}
   \and x!(y) := \lift{x}{\dropn{y}}
   \and \Pi_{i=0}^{n-1}P_i := P_0 | \ldots | P_{n-1}
\end{mathpar}

\subsubsection{Structural congruence}

\paragraph{Free and bound names and alpha-equivalence.} At the
core of structural equivalence is alpha-equivalence which identifies
process that are the same up to a change of variable. Formally, we
recognize the distinction between free and bound names. The free names
of a process, $\freenames{P}$, may be calculated recursively as
follows:

\begin{mathpar}
\freenames{\pzero} := \emptyset
  \and \\
  \freenames{x?(y).P} := \{ x \} \cup (\freenames{P} \setminus \{ y \})
  \and 
  \freenames{x!\langle P \rangle} := \{ x \} \cup \{ P \} 
  \and \\
  \freenames{P|Q} := \freenames{P} \cup \freenames{Q}
  \and \\
  \freenames{@{x}} := \{ x \}
\end{mathpar}

$\pi$
$\quotep{\pi}$

$\freenames{-} : \pi \to \mathcal{P}(\quotep{\pi})$

\begin{eqnarray*}
  \freenames{\pzero} & := & \emptyset \\
  \freenames{x?(y).P} & := & \{ x \} \cup (\freenames{P} \setminus \{ y \}) \\
  \freenames{x!\langle P \rangle} & := & \{ x \} \cup \{ P \} \\
  \freenames{P|Q} & := & \freenames{P} \cup \freenames{Q} \\
  \freenames{\dropn{x}} & := & \{ x \}
\end{eqnarray*}

The bound names of a process, $\boundnames{P}$, are those names occurring in $P$
that are not free. For example, in $x?(y).0$, the name $x$ is free, while $y$ is bound.

\begin{mathpar}
  \inferrule* [lab=monoidal-laws] {} { P|Q \equiv Q|P \and P|0 \equiv P \and P|(Q|R) \equiv (P|Q)|R }
\end{mathpar}

\begin{mathpar}
  \inferrule* [lab=alpha-equivalence] {} { (x)P \equiv (y)P\{y/x\} \and y \not\in \freenames{P} }
\end{mathpar}

\begin{definition}
Then two processes, $P,Q$, are alpha-equivalent if $P = Q\{\vec{y}/\vec{x}\}$ for
some $\vec{x} \in \boundnames{Q},\vec{y} \in \boundnames{P}$, where $Q\{\vec{y}/\vec{x}\}$
denotes the capture-avoiding substitution of $\vec{y}$ for $\vec{x}$ in $Q$.
\end{definition}

\begin{definition}
  The {\em structural congruence} \cite{SangiorgiWalker} , $\equiv$,
  between processes is the least congruence containing
  alpha-equivalence, satisfying the abelian monoid laws
  (associativity, commutativity and $\pzero$ as identity) for parallel
  composition $|$ and for summation $+$.
\end{definition}

\subsection{Name equivalence}

We take name equivalence, written $\nameeq$, to be the smallest
equivalence relation generated by the following rules.

\begin{mathpar}
\inferrule*[lab=Quote-drop]
{ }
{ \quotep{@{x}} \nameeq x }

\inferrule*[lab=Struct-equiv]
{ P \scong Q }
{ \quotep{P} \nameeq \quotep{Q} }
\end{mathpar}

The astute reader will have noticed that the mutual recursion of names
and processes imposes a mutual recursion on alpha-equivalence and
structural equivalence via name-equivalence. Fortunately, all of this
works out pleasantly and we may calculate in the natural way, free of
concern. The reader interested in the details is referred to the
appendix \ref{appendix:rho_details}.

\subsection{Substitution}

We use $\Proc$ for the set of processes, $\QProc$ for the set of
names, and $\id{\{}\vec{y} / \vec{x} \id{\}}$ to denote partial maps,
$s : \QProc \rightarrow \QProc$. A map, $s$ lifts, uniquely, to a map
on process terms, $\widehat{s} : \Proc \rightarrow \Proc$ by the
following equations.

\begin{mathpar}
  (0) \psubstp{Q}{P} := 0 \\
  (R \juxtap S) \psubstp{Q}{P}
  :=    
  (R)\psubstp{Q}{P} \juxtap (S) \psubstp{Q}{P} \\
  (x?(y).R) \psubstp{Q}{P}    
  :=    
  (x)\substp{Q}{P} (z)\concat( (R \psubstn{z}{y}) \psubstp{Q}{P} ) \\
  (\lift{x}{R}) \psubstp{Q}{P}  
  :=
  \lift{(x)\substp{Q}{P}}{ R \psubstp{Q}{P} } \\
%   (\dropn{x})  \psubstp{Q}{P}       
%   := 
%   \left\{ 
%     \begin{array}{ccc} 
%       \dropn{\quotep{Q}} & & x \nameeq \quotep{P} \\
%       \dropn{x} & & otherwise \\
%     \end{array}
%   \right. 
  (\dropn{x})  \psubstp{Q}{P}       
  := 
  \left\{ 
    \begin{array}{ccc} 
      Q & & x \nameeq \quotep{P} \\
      \dropn{x} & & otherwise \\
    \end{array}
  \right.
\end{mathpar}
 

where

\begin{eqnarray}
  (x)\id{\{} \lpquote Q \rpquote / \lpquote P \rpquote \id{\}}            = 
  \left\{ 
    \begin{array}{ccc}
      \lpquote Q \rpquote & & x \nameeq \lpquote P \rpquote \\
      x & & otherwise \\
    \end{array}
  \right. \nonumber
\end{eqnarray}

and $z$ is chosen distinct from $\quotep{P}$, $\quotep{Q}$, the free
names in $Q$, and all the names in $R$. Our $\alpha$-equivalence will
be built in the standard way from this substitution.

\begin{remark}\label{rem:no_self_referential_names}
  One consequence of these definitions is that $\forall P. \quotep{P}
  \not\in \freenames{P}$.
\end{remark}

\subsection{ Dynamic quote: an example }

Anticipating something of what's to come, consider applying the
substitution, $\widehat{\id{\{}u / z \id{\}}}$, to the following pair
of processes, $\lift{w}{y!(z)}$ and $w[ \lpquote y!(z) \rpquote ]$.

\begin{eqnarray}
	\lift{w}{y!(z)}\widehat{\id{\{}u / z \id{\}}}
		& = &
		\lift{w}{y!(u)} \nonumber\\
	w[ \lpquote y!(z) \rpquote ] \widehat{ \id{\{}u / z \id{\}} }
		& = &
		w[ \lpquote y!(z) \rpquote ] \nonumber
\end{eqnarray}

Because the body of the process between quotes is impervious to
substitution, we get radically different answers. In fact, by
examining the first process in an input context,
e.g. $x?(z).\lift{w}{y!(z)}$, we see that the process under the lift
operator may be shaped by prefixed inputs binding a name inside it. In
this sense, the lift operator will be seen as a way to dynamically
construct processes before reifying them as names.

Finally equipped with these standard features we can present the
dynamics of the calculus.

\subsubsection{Operational semantics} 

Finally, we introduce the computational dynamics. What marks these
algebras as distinct from other more traditionally studied algebraic
structures, e.g. vector spaces or polynomial rings, is the manner in
which dynamics is captured. In traditional structures, dynamics is typically
expressed through morphisms between such structures, as in linear maps
between vector spaces or morphisms between rings. In algebras
associated with the semantics of computation, the dynamics is
expressed as part of the algebraic structure itself, through a
reduction reduction relation typically denoted by $\red$. Below, we
give a recursive presentation of this relation for the calculus used
in the encoding.

$\red \subseteq \pi \times \pi$
$\red : \pi \to \mathcal{P}(\pi)$

\begin{mathpar}
  \inferrule* [lab=Comm] { \textsf{match}( x_{src}, x_{trgt} ) } { x_{trgt}?(y)P \; | \; x_{src}!\langle {Q} \rangle \red P\{\quotep{Q}/y}\} }
  \and \\
  \inferrule* [lab=Par] {{P} \red {P}'} {{{P} | {Q}} \red {{P}' | {Q}}}
  \and
  \inferrule* [lab=Equiv]{{{P} \scong {P}'} \andalso {{P}' \red {Q}'} \andalso {{Q}' \scong {Q}}}{{P} \red {Q}}
\end{mathpar}

\begin{eqnarray*}
  match_{\equiv} (\quotep{P},\quotep{Q}) & := & P \equiv Q \\
  match_{\dagger}(\quotep{P},\quotep{Q}) & := & \forall R. P|Q \red^{*} R => R \red^{*} 0 \\
  match_{K}(\quotep{P},\quotep{Q}) & := & K \mbox{ for some context } K
\end{eqnarray*}

$u?(x)P | u!\langle Q \rangle \red P\{\quotep{Q}/x\}$

%We write $\wred$ for $\red^*$, and $P\red$ if $\exists Q $ such that $ P \red Q$.
We write $P\red$ if $\exists Q $ such that $ P \red Q$ and $P\not\red$, otherwise.

\section{Replication}

As mentioned before, it is known that replication (and hence
recursion) can be implemented in a higher-order process algebra
\cite{SangiorgiWalker}. As our first example of calculation with the
machinery thus far presented we give the construction explicitly in
the {\rhoc}.

\begin{eqnarray}
	D_{x} & := & \prefix{x}{y}{(\binpar{\outputp{x}{y}}{@{y}})} \nonumber\\
	\bangp_{x}{P} & := & \binpar{{x}!\langle{\binpar{D_{x}}{P}}\rangle}{D_{x}} \nonumber
\end{eqnarray}

\begin{eqnarray}
	\bangp_{x}{P} & & \nonumber\\
	=
	& {x}!\langle{(\prefix{x}{y}{(\outputp{x}{y} | @{y})) | P}}\rangle 
	      | \prefix{x}{y}{(\outputp{x}{y} | @{y})} & \nonumber\\
	\red
	& (\outputp{x}{y} | @{y})\substn{\quotep{(\prefix{x}{y}{(@{y} | \outputp{x}{y})) | P}}}{y} & \nonumber\\
	=
	& \outputp{x}{\quotep{(\prefix{x}{y}{(\outputp{x}{y} | @{y})) | P}}}
	  | {(\prefix{x}{y}{(\outputp{x}{y} | @{y})) | P}} & \nonumber\\
	\red
	& \ldots & \nonumber\\
	\red^*
	& P | P | \ldots & \nonumber
\end{eqnarray}

Of course, this encoding, as an implementation, runs away, unfolding
$\bangp{P}$ eagerly. A lazier and more implementable replication
operator, restricted to input-guarded processes, may be obtained as follows.

\begin{eqnarray}
\bangp{\prefix{u}{v}{P}} 
	:= 
	\binpar{\lift{x}{\prefix{u}{v}{(\binpar{D(x)}{P})}}}{D(x)} \nonumber
\end{eqnarray}

\begin{remark}
  Note that the lazier definition still does not deal with summation
  or mixed summation (i.e. sums over input and output). The reader is
  invited to construct definitions of replication that deal with these
  features. 

  Further, the definitions are parameterized in a name, $x$. Can you,
  gentle reader, make a definition that eliminates this parameter and
  guarantees no accidental interaction between the replication
  machinery and the process being replicated -- i.e. no accidental
  sharing of names used by the process to get its work done and the
  name(s) used by the replication to effect copying. This latter
  revision of the definition of replication is crucial to obtaining
  the expected identity $!!P \sim !P$.
\end{remark}

\begin{remark}\label{rem:paradoxical_combinator}
  The reader familiar with the lambda calculus will have noticed the
  similarity between $D$ and the paradoxical combinator.

  [Ed. note: the existence of this seems to suggest we have to be more
  restrictive on the set of processes and names we admit if we are to
  support no-cloning.]
\end{remark}

\subsubsection{Bisimulation}

The computational dynamics gives rise to another kind of equivalence,
the equivalence of computational behavior. As previously mentioned
this is typically captured \emph{via} some form of bisimulation.

% The notion we use in this paper is weak barbed bisimulation
% \cite{milner91polyadicpi}.

The notion we use in this paper is derived from weak barbed
bisimulation \cite{milner91polyadicpi}. 

\begin{definition}
An \emph{observation relation}, $\downarrow_{\mathcal N}$, over a set
of names, $\mathcal N$, is the smallest relation satisfying the rules
below.

\infrule[Out-barb]{y \in {\mathcal N}, \; x \nameeq y}
		  {\outputp{x}{v} \downarrow_{\mathcal N} x}
\infrule[Par-barb]{\mbox{$P\downarrow_{\mathcal N} x$ or $Q\downarrow_{\mathcal N} x$}}
		  {\binpar{P}{Q} \downarrow_{\mathcal N} x}

We write $P \Downarrow_{\mathcal N} x$ if there is $Q$ such that 
$P \wred Q$ and $Q \downarrow_{\mathcal N} x$.
\end{definition}

\begin{definition}
%\label{def.bbisim}
An  ${\mathcal N}$-\emph{barbed bisimulation} over a set of names, ${\mathcal N}$, is a symmetric binary relation 
${\mathcal S}_{\mathcal N}$ between agents such that $P\rel{S}_{\mathcal N}Q$ implies:
\begin{enumerate}
\item If $P \red P'$ then $Q \wred Q'$ and $P'\rel{S}_{\mathcal N} Q'$.
\item If $P\downarrow_{\mathcal N} x$, then $Q\Downarrow_{\mathcal N} x$.
\end{enumerate}
$P$ is ${\mathcal N}$-barbed bisimilar to $Q$, written
$P \wbbisim_{\mathcal N} Q$, if $P \rel{S}_{\mathcal N} Q$ for some ${\mathcal N}$-barbed bisimulation ${\mathcal S}_{\mathcal N}$.
\end{definition}

$\mathcal{R} \subseteq \pi \times \pi$

$P \mathcal{R} Q => \forall P'. P \red P' \Rightarrow \exists Q'. Q \red Q', P' \mathcal{R} Q'$

$P \vdash x \Rightarrow Q \vdash x$

\begin{mathpar}
  \inferrule*[lab=Out-barb]{x \nameeq y}{{y}!\langle{Q}\rangle \vdash x}
  \and
  \inferrule*[lab=Par-barb]{\mbox{$P\vdash x$ or $Q\vdash x$}}{\binpar{P}{Q} \vdash x}
\end{mathpar}

\subsubsection{Contexts}

One of the principle advantages of computational calculi like the
$\pi$-calculus is a well-defined notion of context,
contextual-equivalence and a correlation between
contextual-equivalence and notions of bisimulation. The notion of
context allows the decomposition of a process into (sub-)process and
its syntactic environment, its context. Thus, a context may be
thought of as a process with a ``hole'' (written $\Box$) in it. The
application of a context $M$ to a process $P$, written $M[P]$, is
tantamount to filling the hole in $M$ with $P$. In this paper we do
not need the full weight of this theory, but do make use of the notion
of context in the proof the main theorem. 

\begin{mathpar}
  \inferrule* [lab=summation] {} {{M_{M},M_{N}} \bc \Box \;|\; x.M_{A} \;|\; M_{M}+M_{N}}
  \and
  \inferrule* [lab=agent] {} {{M_{A}} \bc (\vec{x})M_{P} \;| \; \clift{P_0,\ldots,M_{P},\ldots,P_N}}
  \and \\
  \inferrule* [lab=process] {} {{M_{P}} \bc M_{N} \;| \;P|M_{P} }
\end{mathpar} 

\begin{mathpar}
  \inferrule* [lab=sychronization] {} {M_{N} \bc \Box \;|\; x?M_{F} \;|\; x!M_{C}}
  \and
  \inferrule* [lab=abstraction] {} {{M_{F}} \bc (x)M_{P} }
  \and
  \inferrule* [lab=concretion] {} {{M_{C}} \bc \langle M_{P} \rangle }
  \and \\
  \inferrule* [lab=process] {} {{M_{P}} \bc M_{N} \;| \;P|M_{P} }
\end{mathpar}

\begin{definition}[contextual application] Given a context $M$, and
  process $P$, we define the \emph{contextual application}, $M[P] :=
  M\{P/\Box\}$. That is, the contextual application of M to P is the
  substitution of $P$ for $\Box$ in $M$.
\end{definition}

$\meaningof{-} : L \to \mathcal{P}(\pi)$

\begin{mathpar}
  \inferrule* [lab=collection] {} {\meaningof{true} = \pi, \and \meaningof{~E} = \pi \setminus \meaningof{E}, \and \meaningof{E_{1} \& E_{2}} = \meaningof{E_{1}} \cap \meaningof{E_{2}}}
\end{mathpar}

\begin{mathpar}
  \inferrule* [lab=structure] {} {\meaningof{0} = \{ P \in \pi | P \equiv 0 \}, \and \\ \meaningof{E_1 | E_2} = \{ P \in \pi | P \equiv P_{1} | P_{2}, P_{1} \in \meaningof{E_{1}}, P_{2} \in \meaningof{E_2}\} }
\end{mathpar}

\begin{mathpar}
 \inferrule* [lab=behavior] {} {\meaningof{\langle a?b \rangle E} = \{ P \in \pi | P \equiv Q | u?(y)P', \\ \and \\\\ \and \\ \;\;\; u \in \meaningof{a}, \forall z.P'\{z/y\} \in \meaningof{E\{z/b\}}\}, \and \\ \meaningof{a!E} = \{ P \in \pi | P \equiv Q | x!\langle P' \rangle, x \in \meaningof{a} P' \in \meaningof{E}\} }
\end{mathpar}

\begin{mathpar}
 \inferrule* [lab=nominal] {} {\meaningof{\quotep{E}} = \{ \quotep{P} \in \quotep{\pi} | P \in \meaningof{E} \}, \and \meaningof{\quotep{P}} = \{ \quotep{Q} \in \quotep{\pi} | P \equiv Q \} \and \\ \meaningof{@\quotep{E}} = \{ P \in \pi | P \equiv @x, x \in \meaningof{E} \}}
\end{mathpar}

\begin{eqnarray*}
  \\
  \meaningof{-} : TS \to ST
\end{eqnarray*}

\begin{eqnarray*}
  \\
  L : TS \to ST
\end{eqnarray*}

\begin{eqnarray*}
  \\
  P \models E \iff P \in \meaningof{E}
\end{eqnarray*}

\begin{eqnarray*}
  P \approx_{L} Q \iff \forall E \in L. P \models E \iff Q \models E
\end{eqnarray*}

\begin{eqnarray*}
  P \approx_{K} Q
\end{eqnarray*}

\begin{eqnarray*}
  P \approx Q
\end{eqnarray*}

$\approx_{K} = \approx = \approx_{L}$

\subsubsection{Contextual duality}

Note that contexts extend the quotation operation to a family of
operations from processes to names. Given a context, $M$, we can
define a \emph{nominal context}, $\quotep{M}$ by $\quotep{M}[P] :=
\quotep{M[P]}$. To foreshadow what is to come we observe that these
operations enjoy a duality with processes very much like the duality
between vectors and maps from vectors to scalars.

Further, because the calculus is essentially higher-order, we have a
correspondence between contexts and processes. More specifically,
given a name $x$ and a context $M$ we can construct $M^{*}_{x}$ such
that 

\begin{mathpar}
  M^{*}_{x} | \lift{x}{P} \red M[P]
\end{mathpar}

namely,

\begin{mathpar}
  M^{*}_{x} := x?(u).M[\dropn{u}]
\end{mathpar}

The dependence of $M^{*}_{x}$ on a name makes it an abstraction, 

\begin{mathpar}
  M^{*} := (x)x?(u).M[\dropn{u}]
\end{mathpar}

\subsection{Additional notation}

It will sometimes be convenient to denote the process a name
quotes. We already have the notation $x = \quotep{P}$, but it will be
convenient to introduce an alternate notation, $\procn{x}$, when we
want to emphasize the connection to the use of the name. Note that, by
virtue of name equivalence, $\quotep{\procn{x}} \nameeq x$; so, the
notation is consistent with previous definitions.

Further, because names have structure it is possible to effect
substitutions on the basis of that structure. This means we need to
upgrade our notation for substitutions, which we accomplish by
adapting comprehension notation. Thus,

\begin{mathpar}
  P\{ y / x : x \in S \}
\end{mathpar}

is interpreted to mean the process derived from P by replacing (in a
capture-avoiding manner) each occurrence of $x$ in $S$ by $y$. For example,

\begin{mathpar}
  P\{ \quotep{\procn{x}|\procn{x}} / x : x \in \freenames{P} \}
\end{mathpar}

will replace each (occurrence) of a free name $x$ in $P$ by
$\quotep{\procn{x}|\procn{x}}$.

Also, we will avail ourselves of the notation $x^{L}$ and $x^{R}$ to
denote injections of a name into disjoint copies of the name
space. There are numerous ways to accomplish this. One example can be
found in \cite{MeredithR05}. This notation overloads to vectors of
names: $\vec{x}^{\pi} := (x_{i}^{\pi} \; : \; 0 \leq i < |\vec{x}| )$ where $\pi \in \{L,R\}$.

We also use $P^{\Box} := P|\Box$.

In \cite{MeredithR05} an interpretation of the new operator is
given. It turns out that there are several possible interpretations
all enjoying the requisite algebraic properties of the operator (see
\cite{milner91polyadicpi}). We will therefore make liberal use of
$(\nu\; \vec{x})P$.

% subsection the_syntax_and_semantics_of_the_notation_system (end)   

\input{qm2pi.qmops} 

\input{qm2pi.sterngerlach} 

\input{qm2pi.metric} 

% section concurrent_process_calculi (end)

%\input{qm2pi.proofsketch}

% section proof sketch (end)

%\input{qm2pi.slviaknots} 

% section spatial logic via knots (end)

\input{qm2pi.conclusion}

% section conclusion (end)

%\input{qm2pi.dtcodes} 

% section wiring algorithm (end)

\input{qm2pi.ack} 

% section acknowledgments (end)

\newpage


\bibliographystyle{plain}   
\bibliography{../../biblios/main.bib}

\input{qm2pi.rhodetails}

\end{document}

 

%\documentclass[12pt]{llncs}
%\documentclass{jktr}

\usepackage[pdftex]{hyperref}                   
\usepackage {listings}
\usepackage {mathpartir}
\usepackage{bcprules}
%\usepackage{listings}
                       
\usepackage{graphicx} 
%\usepackage[margins=2.5cm,nohead,nofoot]{geometry}
%\usepackage{geometry}
\usepackage{amsfonts}
\usepackage{amstext}
\usepackage{latexsym}
\usepackage{amssymb}
\usepackage{color}


%\include{myPreamble}
\include{qm2pi.local} 

%\ifpdf
%\usepackage[pdftex]{graphicx}
%\else
%\usepackage{graphicx}
%\fi

 % \ifpdf
%  \usepackage{pdfsync}
%  \if


%\title{Brief Article}
%\author{David F. Snyder}
%\author{L.G. Meredith}

%\address{Dept. of Math., Texas State University--San Marcos, San Marcos, TX 78666}
       
\pagestyle{empty}


\begin{document}

\lstset{language=[Objective]Caml,frame=shadowbox}

\input{qm2pi.front}

% section front matter (end)

\input{qm2pi.intro} 
 
% section introduction (end)

% \input{qm2pi.knotations} 

% section notation (end)

\input{qm2pi.process.calculi} 

% section concurrent_process_calculi_and_spatial_logics_ (end)
    
%\input{qm2pi.knots2pi} 

%\input{qm2pi.trefoil} 

%\input{qm2pi.mainthm} 

% subsection basic_interpretation (end)

%\input{qm2pi.rho.presentation} 
\subsection{The syntax and semantics of the notation system}\label{sub:the_syntax_and_semantics_of_the_notation_system} % (fold)

We now summarize a technical presentation of the calculus that
embodies our theory of dynamics. The typical presentation of such a
calculus follows the style of giving generators and relations on
them. The grammar, below, describing term constructors, freely
generates the set of processes, $\Proc$. This set is then quotiented
by a relation known as structural congruence and it is over this set
that the notion of dynamics is expressed. This presentation is
essentially that of \cite{MeredithR05} with the addition of
polyadicity and summation. For readability we have relegated some of
the technical subtleties to an appendix.

\subsubsection{Process grammar}\label{subsub:process_grammar}

\begin{mathpar}
  \inferrule* [lab=synchronization] {} {{M} \bc \pzero \;|\; x?F \;|\; x!C }
  \and
  \inferrule* [lab=abstraction] {} {{F} \bc (x)P}
  \and
  \inferrule* [lab=concretion] {} {{C} \bc \langle Q \rangle}
  \and
  \inferrule* [lab=process] {} {{P,Q} \bc M \;| \;P|Q \;|\; @{x}}
  \and
  \inferrule* [lab=name] {} {{x} \bc \quotep{P}}
\end{mathpar} 

Note that $\vec{x}$ (resp. $\vec{P}$) denotes a vector of names
(resp. processes) of length $|\vec{x}|$ (resp. $|\vec{P}|$). We adopt
the following useful abbreviations.

\begin{mathpar}
   x?(\vec{y}).P := x.(\vec{y})P \and  x\clift{\vec{P}} := x.\clift{\vec{P}}
   \and x!(y) := \lift{x}{\dropn{y}}
   \and \Pi_{i=0}^{n-1}P_i := P_0 | \ldots | P_{n-1}
\end{mathpar}

\subsubsection{Structural congruence}

\paragraph{Free and bound names and alpha-equivalence.} At the
core of structural equivalence is alpha-equivalence which identifies
process that are the same up to a change of variable. Formally, we
recognize the distinction between free and bound names. The free names
of a process, $\freenames{P}$, may be calculated recursively as
follows:

\begin{mathpar}
\freenames{\pzero} := \emptyset
  \and \\
  \freenames{x?(y).P} := \{ x \} \cup (\freenames{P} \setminus \{ y \})
  \and 
  \freenames{x!\langle P \rangle} := \{ x \} \cup \{ P \} 
  \and \\
  \freenames{P|Q} := \freenames{P} \cup \freenames{Q}
  \and \\
  \freenames{@{x}} := \{ x \}
\end{mathpar}

$\pi$
$\quotep{\pi}$

$\freenames{-} : \pi \to \mathcal{P}(\quotep{\pi})$

\begin{eqnarray*}
  \freenames{\pzero} & := & \emptyset \\
  \freenames{x?(y).P} & := & \{ x \} \cup (\freenames{P} \setminus \{ y \}) \\
  \freenames{x!\langle P \rangle} & := & \{ x \} \cup \{ P \} \\
  \freenames{P|Q} & := & \freenames{P} \cup \freenames{Q} \\
  \freenames{\dropn{x}} & := & \{ x \}
\end{eqnarray*}

The bound names of a process, $\boundnames{P}$, are those names occurring in $P$
that are not free. For example, in $x?(y).0$, the name $x$ is free, while $y$ is bound.

\begin{mathpar}
  \inferrule* [lab=monoidal-laws] {} { P|Q \equiv Q|P \and P|0 \equiv P \and P|(Q|R) \equiv (P|Q)|R }
\end{mathpar}

\begin{mathpar}
  \inferrule* [lab=alpha-equivalence] {} { (x)P \equiv (y)P\{y/x\} \and y \not\in \freenames{P} }
\end{mathpar}

\begin{definition}
Then two processes, $P,Q$, are alpha-equivalent if $P = Q\{\vec{y}/\vec{x}\}$ for
some $\vec{x} \in \boundnames{Q},\vec{y} \in \boundnames{P}$, where $Q\{\vec{y}/\vec{x}\}$
denotes the capture-avoiding substitution of $\vec{y}$ for $\vec{x}$ in $Q$.
\end{definition}

\begin{definition}
  The {\em structural congruence} \cite{SangiorgiWalker} , $\equiv$,
  between processes is the least congruence containing
  alpha-equivalence, satisfying the abelian monoid laws
  (associativity, commutativity and $\pzero$ as identity) for parallel
  composition $|$ and for summation $+$.
\end{definition}

\subsection{Name equivalence}

We take name equivalence, written $\nameeq$, to be the smallest
equivalence relation generated by the following rules.

\begin{mathpar}
\inferrule*[lab=Quote-drop]
{ }
{ \quotep{@{x}} \nameeq x }

\inferrule*[lab=Struct-equiv]
{ P \scong Q }
{ \quotep{P} \nameeq \quotep{Q} }
\end{mathpar}

The astute reader will have noticed that the mutual recursion of names
and processes imposes a mutual recursion on alpha-equivalence and
structural equivalence via name-equivalence. Fortunately, all of this
works out pleasantly and we may calculate in the natural way, free of
concern. The reader interested in the details is referred to the
appendix \ref{appendix:rho_details}.

\subsection{Substitution}

We use $\Proc$ for the set of processes, $\QProc$ for the set of
names, and $\id{\{}\vec{y} / \vec{x} \id{\}}$ to denote partial maps,
$s : \QProc \rightarrow \QProc$. A map, $s$ lifts, uniquely, to a map
on process terms, $\widehat{s} : \Proc \rightarrow \Proc$ by the
following equations.

\begin{mathpar}
  (0) \psubstp{Q}{P} := 0 \\
  (R \juxtap S) \psubstp{Q}{P}
  :=    
  (R)\psubstp{Q}{P} \juxtap (S) \psubstp{Q}{P} \\
  (x?(y).R) \psubstp{Q}{P}    
  :=    
  (x)\substp{Q}{P} (z)\concat( (R \psubstn{z}{y}) \psubstp{Q}{P} ) \\
  (\lift{x}{R}) \psubstp{Q}{P}  
  :=
  \lift{(x)\substp{Q}{P}}{ R \psubstp{Q}{P} } \\
%   (\dropn{x})  \psubstp{Q}{P}       
%   := 
%   \left\{ 
%     \begin{array}{ccc} 
%       \dropn{\quotep{Q}} & & x \nameeq \quotep{P} \\
%       \dropn{x} & & otherwise \\
%     \end{array}
%   \right. 
  (\dropn{x})  \psubstp{Q}{P}       
  := 
  \left\{ 
    \begin{array}{ccc} 
      Q & & x \nameeq \quotep{P} \\
      \dropn{x} & & otherwise \\
    \end{array}
  \right.
\end{mathpar}
 

where

\begin{eqnarray}
  (x)\id{\{} \lpquote Q \rpquote / \lpquote P \rpquote \id{\}}            = 
  \left\{ 
    \begin{array}{ccc}
      \lpquote Q \rpquote & & x \nameeq \lpquote P \rpquote \\
      x & & otherwise \\
    \end{array}
  \right. \nonumber
\end{eqnarray}

and $z$ is chosen distinct from $\quotep{P}$, $\quotep{Q}$, the free
names in $Q$, and all the names in $R$. Our $\alpha$-equivalence will
be built in the standard way from this substitution.

\begin{remark}\label{rem:no_self_referential_names}
  One consequence of these definitions is that $\forall P. \quotep{P}
  \not\in \freenames{P}$.
\end{remark}

\subsection{ Dynamic quote: an example }

Anticipating something of what's to come, consider applying the
substitution, $\widehat{\id{\{}u / z \id{\}}}$, to the following pair
of processes, $\lift{w}{y!(z)}$ and $w[ \lpquote y!(z) \rpquote ]$.

\begin{eqnarray}
	\lift{w}{y!(z)}\widehat{\id{\{}u / z \id{\}}}
		& = &
		\lift{w}{y!(u)} \nonumber\\
	w[ \lpquote y!(z) \rpquote ] \widehat{ \id{\{}u / z \id{\}} }
		& = &
		w[ \lpquote y!(z) \rpquote ] \nonumber
\end{eqnarray}

Because the body of the process between quotes is impervious to
substitution, we get radically different answers. In fact, by
examining the first process in an input context,
e.g. $x?(z).\lift{w}{y!(z)}$, we see that the process under the lift
operator may be shaped by prefixed inputs binding a name inside it. In
this sense, the lift operator will be seen as a way to dynamically
construct processes before reifying them as names.

Finally equipped with these standard features we can present the
dynamics of the calculus.

\subsubsection{Operational semantics} 

Finally, we introduce the computational dynamics. What marks these
algebras as distinct from other more traditionally studied algebraic
structures, e.g. vector spaces or polynomial rings, is the manner in
which dynamics is captured. In traditional structures, dynamics is typically
expressed through morphisms between such structures, as in linear maps
between vector spaces or morphisms between rings. In algebras
associated with the semantics of computation, the dynamics is
expressed as part of the algebraic structure itself, through a
reduction reduction relation typically denoted by $\red$. Below, we
give a recursive presentation of this relation for the calculus used
in the encoding.

$\red \subseteq \pi \times \pi$
$\red : \pi \to \mathcal{P}(\pi)$

\begin{mathpar}
  \inferrule* [lab=Comm] { \textsf{match}( x_{src}, x_{trgt} ) } { x_{trgt}?(y)P \; | \; x_{src}!\langle {Q} \rangle \red P\{\quotep{Q}/y}\} }
  \and \\
  \inferrule* [lab=Par] {{P} \red {P}'} {{{P} | {Q}} \red {{P}' | {Q}}}
  \and
  \inferrule* [lab=Equiv]{{{P} \scong {P}'} \andalso {{P}' \red {Q}'} \andalso {{Q}' \scong {Q}}}{{P} \red {Q}}
\end{mathpar}

\begin{eqnarray*}
  match_{\equiv} (\quotep{P},\quotep{Q}) & := & P \equiv Q \\
  match_{\dagger}(\quotep{P},\quotep{Q}) & := & \forall R. P|Q \red^{*} R => R \red^{*} 0 \\
  match_{K}(\quotep{P},\quotep{Q}) & := & K \mbox{ for some context } K
\end{eqnarray*}

$u?(x)P | u!\langle Q \rangle \red P\{\quotep{Q}/x\}$

%We write $\wred$ for $\red^*$, and $P\red$ if $\exists Q $ such that $ P \red Q$.
We write $P\red$ if $\exists Q $ such that $ P \red Q$ and $P\not\red$, otherwise.

\section{Replication}

As mentioned before, it is known that replication (and hence
recursion) can be implemented in a higher-order process algebra
\cite{SangiorgiWalker}. As our first example of calculation with the
machinery thus far presented we give the construction explicitly in
the {\rhoc}.

\begin{eqnarray}
	D_{x} & := & \prefix{x}{y}{(\binpar{\outputp{x}{y}}{@{y}})} \nonumber\\
	\bangp_{x}{P} & := & \binpar{{x}!\langle{\binpar{D_{x}}{P}}\rangle}{D_{x}} \nonumber
\end{eqnarray}

\begin{eqnarray}
	\bangp_{x}{P} & & \nonumber\\
	=
	& {x}!\langle{(\prefix{x}{y}{(\outputp{x}{y} | @{y})) | P}}\rangle 
	      | \prefix{x}{y}{(\outputp{x}{y} | @{y})} & \nonumber\\
	\red
	& (\outputp{x}{y} | @{y})\substn{\quotep{(\prefix{x}{y}{(@{y} | \outputp{x}{y})) | P}}}{y} & \nonumber\\
	=
	& \outputp{x}{\quotep{(\prefix{x}{y}{(\outputp{x}{y} | @{y})) | P}}}
	  | {(\prefix{x}{y}{(\outputp{x}{y} | @{y})) | P}} & \nonumber\\
	\red
	& \ldots & \nonumber\\
	\red^*
	& P | P | \ldots & \nonumber
\end{eqnarray}

Of course, this encoding, as an implementation, runs away, unfolding
$\bangp{P}$ eagerly. A lazier and more implementable replication
operator, restricted to input-guarded processes, may be obtained as follows.

\begin{eqnarray}
\bangp{\prefix{u}{v}{P}} 
	:= 
	\binpar{\lift{x}{\prefix{u}{v}{(\binpar{D(x)}{P})}}}{D(x)} \nonumber
\end{eqnarray}

\begin{remark}
  Note that the lazier definition still does not deal with summation
  or mixed summation (i.e. sums over input and output). The reader is
  invited to construct definitions of replication that deal with these
  features. 

  Further, the definitions are parameterized in a name, $x$. Can you,
  gentle reader, make a definition that eliminates this parameter and
  guarantees no accidental interaction between the replication
  machinery and the process being replicated -- i.e. no accidental
  sharing of names used by the process to get its work done and the
  name(s) used by the replication to effect copying. This latter
  revision of the definition of replication is crucial to obtaining
  the expected identity $!!P \sim !P$.
\end{remark}

\begin{remark}\label{rem:paradoxical_combinator}
  The reader familiar with the lambda calculus will have noticed the
  similarity between $D$ and the paradoxical combinator.

  [Ed. note: the existence of this seems to suggest we have to be more
  restrictive on the set of processes and names we admit if we are to
  support no-cloning.]
\end{remark}

\subsubsection{Bisimulation}

The computational dynamics gives rise to another kind of equivalence,
the equivalence of computational behavior. As previously mentioned
this is typically captured \emph{via} some form of bisimulation.

% The notion we use in this paper is weak barbed bisimulation
% \cite{milner91polyadicpi}.

The notion we use in this paper is derived from weak barbed
bisimulation \cite{milner91polyadicpi}. 

\begin{definition}
An \emph{observation relation}, $\downarrow_{\mathcal N}$, over a set
of names, $\mathcal N$, is the smallest relation satisfying the rules
below.

\infrule[Out-barb]{y \in {\mathcal N}, \; x \nameeq y}
		  {\outputp{x}{v} \downarrow_{\mathcal N} x}
\infrule[Par-barb]{\mbox{$P\downarrow_{\mathcal N} x$ or $Q\downarrow_{\mathcal N} x$}}
		  {\binpar{P}{Q} \downarrow_{\mathcal N} x}

We write $P \Downarrow_{\mathcal N} x$ if there is $Q$ such that 
$P \wred Q$ and $Q \downarrow_{\mathcal N} x$.
\end{definition}

\begin{definition}
%\label{def.bbisim}
An  ${\mathcal N}$-\emph{barbed bisimulation} over a set of names, ${\mathcal N}$, is a symmetric binary relation 
${\mathcal S}_{\mathcal N}$ between agents such that $P\rel{S}_{\mathcal N}Q$ implies:
\begin{enumerate}
\item If $P \red P'$ then $Q \wred Q'$ and $P'\rel{S}_{\mathcal N} Q'$.
\item If $P\downarrow_{\mathcal N} x$, then $Q\Downarrow_{\mathcal N} x$.
\end{enumerate}
$P$ is ${\mathcal N}$-barbed bisimilar to $Q$, written
$P \wbbisim_{\mathcal N} Q$, if $P \rel{S}_{\mathcal N} Q$ for some ${\mathcal N}$-barbed bisimulation ${\mathcal S}_{\mathcal N}$.
\end{definition}

$\mathcal{R} \subseteq \pi \times \pi$

$P \mathcal{R} Q => \forall P'. P \red P' \Rightarrow \exists Q'. Q \red Q', P' \mathcal{R} Q'$

$P \vdash x \Rightarrow Q \vdash x$

\begin{mathpar}
  \inferrule*[lab=Out-barb]{x \nameeq y}{{y}!\langle{Q}\rangle \vdash x}
  \and
  \inferrule*[lab=Par-barb]{\mbox{$P\vdash x$ or $Q\vdash x$}}{\binpar{P}{Q} \vdash x}
\end{mathpar}

\subsubsection{Contexts}

One of the principle advantages of computational calculi like the
$\pi$-calculus is a well-defined notion of context,
contextual-equivalence and a correlation between
contextual-equivalence and notions of bisimulation. The notion of
context allows the decomposition of a process into (sub-)process and
its syntactic environment, its context. Thus, a context may be
thought of as a process with a ``hole'' (written $\Box$) in it. The
application of a context $M$ to a process $P$, written $M[P]$, is
tantamount to filling the hole in $M$ with $P$. In this paper we do
not need the full weight of this theory, but do make use of the notion
of context in the proof the main theorem. 

\begin{mathpar}
  \inferrule* [lab=summation] {} {{M_{M},M_{N}} \bc \Box \;|\; x.M_{A} \;|\; M_{M}+M_{N}}
  \and
  \inferrule* [lab=agent] {} {{M_{A}} \bc (\vec{x})M_{P} \;| \; \clift{P_0,\ldots,M_{P},\ldots,P_N}}
  \and \\
  \inferrule* [lab=process] {} {{M_{P}} \bc M_{N} \;| \;P|M_{P} }
\end{mathpar} 

\begin{mathpar}
  \inferrule* [lab=sychronization] {} {M_{N} \bc \Box \;|\; x?M_{F} \;|\; x!M_{C}}
  \and
  \inferrule* [lab=abstraction] {} {{M_{F}} \bc (x)M_{P} }
  \and
  \inferrule* [lab=concretion] {} {{M_{C}} \bc \langle M_{P} \rangle }
  \and \\
  \inferrule* [lab=process] {} {{M_{P}} \bc M_{N} \;| \;P|M_{P} }
\end{mathpar}

\begin{definition}[contextual application] Given a context $M$, and
  process $P$, we define the \emph{contextual application}, $M[P] :=
  M\{P/\Box\}$. That is, the contextual application of M to P is the
  substitution of $P$ for $\Box$ in $M$.
\end{definition}

$\meaningof{-} : L \to \mathcal{P}(\pi)$

\begin{mathpar}
  \inferrule* [lab=collection] {} {\meaningof{true} = \pi, \and \meaningof{~E} = \pi \setminus \meaningof{E}, \and \meaningof{E_{1} \& E_{2}} = \meaningof{E_{1}} \cap \meaningof{E_{2}}}
\end{mathpar}

\begin{mathpar}
  \inferrule* [lab=structure] {} {\meaningof{0} = \{ P \in \pi | P \equiv 0 \}, \and \\ \meaningof{E_1 | E_2} = \{ P \in \pi | P \equiv P_{1} | P_{2}, P_{1} \in \meaningof{E_{1}}, P_{2} \in \meaningof{E_2}\} }
\end{mathpar}

\begin{mathpar}
 \inferrule* [lab=behavior] {} {\meaningof{\langle a?b \rangle E} = \{ P \in \pi | P \equiv Q | u?(y)P', \\ \and \\\\ \and \\ \;\;\; u \in \meaningof{a}, \forall z.P'\{z/y\} \in \meaningof{E\{z/b\}}\}, \and \\ \meaningof{a!E} = \{ P \in \pi | P \equiv Q | x!\langle P' \rangle, x \in \meaningof{a} P' \in \meaningof{E}\} }
\end{mathpar}

\begin{mathpar}
 \inferrule* [lab=nominal] {} {\meaningof{\quotep{E}} = \{ \quotep{P} \in \quotep{\pi} | P \in \meaningof{E} \}, \and \meaningof{\quotep{P}} = \{ \quotep{Q} \in \quotep{\pi} | P \equiv Q \} \and \\ \meaningof{@\quotep{E}} = \{ P \in \pi | P \equiv @x, x \in \meaningof{E} \}}
\end{mathpar}

\begin{eqnarray*}
  \\
  \meaningof{-} : TS \to ST
\end{eqnarray*}

\begin{eqnarray*}
  \\
  L : TS \to ST
\end{eqnarray*}

\begin{eqnarray*}
  \\
  P \models E \iff P \in \meaningof{E}
\end{eqnarray*}

\begin{eqnarray*}
  P \approx_{L} Q \iff \forall E \in L. P \models E \iff Q \models E
\end{eqnarray*}

\begin{eqnarray*}
  P \approx_{K} Q
\end{eqnarray*}

\begin{eqnarray*}
  P \approx Q
\end{eqnarray*}

$\approx_{K} = \approx = \approx_{L}$

\subsubsection{Contextual duality}

Note that contexts extend the quotation operation to a family of
operations from processes to names. Given a context, $M$, we can
define a \emph{nominal context}, $\quotep{M}$ by $\quotep{M}[P] :=
\quotep{M[P]}$. To foreshadow what is to come we observe that these
operations enjoy a duality with processes very much like the duality
between vectors and maps from vectors to scalars.

Further, because the calculus is essentially higher-order, we have a
correspondence between contexts and processes. More specifically,
given a name $x$ and a context $M$ we can construct $M^{*}_{x}$ such
that 

\begin{mathpar}
  M^{*}_{x} | \lift{x}{P} \red M[P]
\end{mathpar}

namely,

\begin{mathpar}
  M^{*}_{x} := x?(u).M[\dropn{u}]
\end{mathpar}

The dependence of $M^{*}_{x}$ on a name makes it an abstraction, 

\begin{mathpar}
  M^{*} := (x)x?(u).M[\dropn{u}]
\end{mathpar}

\subsection{Additional notation}

It will sometimes be convenient to denote the process a name
quotes. We already have the notation $x = \quotep{P}$, but it will be
convenient to introduce an alternate notation, $\procn{x}$, when we
want to emphasize the connection to the use of the name. Note that, by
virtue of name equivalence, $\quotep{\procn{x}} \nameeq x$; so, the
notation is consistent with previous definitions.

Further, because names have structure it is possible to effect
substitutions on the basis of that structure. This means we need to
upgrade our notation for substitutions, which we accomplish by
adapting comprehension notation. Thus,

\begin{mathpar}
  P\{ y / x : x \in S \}
\end{mathpar}

is interpreted to mean the process derived from P by replacing (in a
capture-avoiding manner) each occurrence of $x$ in $S$ by $y$. For example,

\begin{mathpar}
  P\{ \quotep{\procn{x}|\procn{x}} / x : x \in \freenames{P} \}
\end{mathpar}

will replace each (occurrence) of a free name $x$ in $P$ by
$\quotep{\procn{x}|\procn{x}}$.

Also, we will avail ourselves of the notation $x^{L}$ and $x^{R}$ to
denote injections of a name into disjoint copies of the name
space. There are numerous ways to accomplish this. One example can be
found in \cite{MeredithR05}. This notation overloads to vectors of
names: $\vec{x}^{\pi} := (x_{i}^{\pi} \; : \; 0 \leq i < |\vec{x}| )$ where $\pi \in \{L,R\}$.

We also use $P^{\Box} := P|\Box$.

In \cite{MeredithR05} an interpretation of the new operator is
given. It turns out that there are several possible interpretations
all enjoying the requisite algebraic properties of the operator (see
\cite{milner91polyadicpi}). We will therefore make liberal use of
$(\nu\; \vec{x})P$.

% subsection the_syntax_and_semantics_of_the_notation_system (end)   

\input{qm2pi.qmops} 

\input{qm2pi.sterngerlach} 

\input{qm2pi.metric} 

% section concurrent_process_calculi (end)

%\input{qm2pi.proofsketch}

% section proof sketch (end)

%\input{qm2pi.slviaknots} 

% section spatial logic via knots (end)

\input{qm2pi.conclusion}

% section conclusion (end)

%\input{qm2pi.dtcodes} 

% section wiring algorithm (end)

\input{qm2pi.ack} 

% section acknowledgments (end)

\newpage


\bibliographystyle{plain}   
\bibliography{../../biblios/main.bib}

\input{qm2pi.rhodetails}

\end{document}

 

% subsection basic_interpretation (end)

%\input{qm2pi.rho.presentation} 
\subsection{The syntax and semantics of the notation system}\label{sub:the_syntax_and_semantics_of_the_notation_system} % (fold)

We now summarize a technical presentation of the calculus that
embodies our theory of dynamics. The typical presentation of such a
calculus follows the style of giving generators and relations on
them. The grammar, below, describing term constructors, freely
generates the set of processes, $\Proc$. This set is then quotiented
by a relation known as structural congruence and it is over this set
that the notion of dynamics is expressed. This presentation is
essentially that of \cite{MeredithR05} with the addition of
polyadicity and summation. For readability we have relegated some of
the technical subtleties to an appendix.

\subsubsection{Process grammar}\label{subsub:process_grammar}

\begin{mathpar}
  \inferrule* [lab=synchronization] {} {{M} \bc \pzero \;|\; x?F \;|\; x!C }
  \and
  \inferrule* [lab=abstraction] {} {{F} \bc (x)P}
  \and
  \inferrule* [lab=concretion] {} {{C} \bc \langle Q \rangle}
  \and
  \inferrule* [lab=process] {} {{P,Q} \bc M \;| \;P|Q \;|\; @{x}}
  \and
  \inferrule* [lab=name] {} {{x} \bc \quotep{P}}
\end{mathpar} 

Note that $\vec{x}$ (resp. $\vec{P}$) denotes a vector of names
(resp. processes) of length $|\vec{x}|$ (resp. $|\vec{P}|$). We adopt
the following useful abbreviations.

\begin{mathpar}
   x?(\vec{y}).P := x.(\vec{y})P \and  x\clift{\vec{P}} := x.\clift{\vec{P}}
   \and x!(y) := \lift{x}{\dropn{y}}
   \and \Pi_{i=0}^{n-1}P_i := P_0 | \ldots | P_{n-1}
\end{mathpar}

\subsubsection{Structural congruence}

\paragraph{Free and bound names and alpha-equivalence.} At the
core of structural equivalence is alpha-equivalence which identifies
process that are the same up to a change of variable. Formally, we
recognize the distinction between free and bound names. The free names
of a process, $\freenames{P}$, may be calculated recursively as
follows:

\begin{mathpar}
\freenames{\pzero} := \emptyset
  \and \\
  \freenames{x?(y).P} := \{ x \} \cup (\freenames{P} \setminus \{ y \})
  \and 
  \freenames{x!\langle P \rangle} := \{ x \} \cup \{ P \} 
  \and \\
  \freenames{P|Q} := \freenames{P} \cup \freenames{Q}
  \and \\
  \freenames{@{x}} := \{ x \}
\end{mathpar}

$\pi$
$\quotep{\pi}$

$\freenames{-} : \pi \to \mathcal{P}(\quotep{\pi})$

\begin{eqnarray*}
  \freenames{\pzero} & := & \emptyset \\
  \freenames{x?(y).P} & := & \{ x \} \cup (\freenames{P} \setminus \{ y \}) \\
  \freenames{x!\langle P \rangle} & := & \{ x \} \cup \{ P \} \\
  \freenames{P|Q} & := & \freenames{P} \cup \freenames{Q} \\
  \freenames{\dropn{x}} & := & \{ x \}
\end{eqnarray*}

The bound names of a process, $\boundnames{P}$, are those names occurring in $P$
that are not free. For example, in $x?(y).0$, the name $x$ is free, while $y$ is bound.

\begin{mathpar}
  \inferrule* [lab=monoidal-laws] {} { P|Q \equiv Q|P \and P|0 \equiv P \and P|(Q|R) \equiv (P|Q)|R }
\end{mathpar}

\begin{mathpar}
  \inferrule* [lab=alpha-equivalence] {} { (x)P \equiv (y)P\{y/x\} \and y \not\in \freenames{P} }
\end{mathpar}

\begin{definition}
Then two processes, $P,Q$, are alpha-equivalent if $P = Q\{\vec{y}/\vec{x}\}$ for
some $\vec{x} \in \boundnames{Q},\vec{y} \in \boundnames{P}$, where $Q\{\vec{y}/\vec{x}\}$
denotes the capture-avoiding substitution of $\vec{y}$ for $\vec{x}$ in $Q$.
\end{definition}

\begin{definition}
  The {\em structural congruence} \cite{SangiorgiWalker} , $\equiv$,
  between processes is the least congruence containing
  alpha-equivalence, satisfying the abelian monoid laws
  (associativity, commutativity and $\pzero$ as identity) for parallel
  composition $|$ and for summation $+$.
\end{definition}

\subsection{Name equivalence}

We take name equivalence, written $\nameeq$, to be the smallest
equivalence relation generated by the following rules.

\begin{mathpar}
\inferrule*[lab=Quote-drop]
{ }
{ \quotep{@{x}} \nameeq x }

\inferrule*[lab=Struct-equiv]
{ P \scong Q }
{ \quotep{P} \nameeq \quotep{Q} }
\end{mathpar}

The astute reader will have noticed that the mutual recursion of names
and processes imposes a mutual recursion on alpha-equivalence and
structural equivalence via name-equivalence. Fortunately, all of this
works out pleasantly and we may calculate in the natural way, free of
concern. The reader interested in the details is referred to the
appendix \ref{appendix:rho_details}.

\subsection{Substitution}

We use $\Proc$ for the set of processes, $\QProc$ for the set of
names, and $\id{\{}\vec{y} / \vec{x} \id{\}}$ to denote partial maps,
$s : \QProc \rightarrow \QProc$. A map, $s$ lifts, uniquely, to a map
on process terms, $\widehat{s} : \Proc \rightarrow \Proc$ by the
following equations.

\begin{mathpar}
  (0) \psubstp{Q}{P} := 0 \\
  (R \juxtap S) \psubstp{Q}{P}
  :=    
  (R)\psubstp{Q}{P} \juxtap (S) \psubstp{Q}{P} \\
  (x?(y).R) \psubstp{Q}{P}    
  :=    
  (x)\substp{Q}{P} (z)\concat( (R \psubstn{z}{y}) \psubstp{Q}{P} ) \\
  (\lift{x}{R}) \psubstp{Q}{P}  
  :=
  \lift{(x)\substp{Q}{P}}{ R \psubstp{Q}{P} } \\
%   (\dropn{x})  \psubstp{Q}{P}       
%   := 
%   \left\{ 
%     \begin{array}{ccc} 
%       \dropn{\quotep{Q}} & & x \nameeq \quotep{P} \\
%       \dropn{x} & & otherwise \\
%     \end{array}
%   \right. 
  (\dropn{x})  \psubstp{Q}{P}       
  := 
  \left\{ 
    \begin{array}{ccc} 
      Q & & x \nameeq \quotep{P} \\
      \dropn{x} & & otherwise \\
    \end{array}
  \right.
\end{mathpar}
 

where

\begin{eqnarray}
  (x)\id{\{} \lpquote Q \rpquote / \lpquote P \rpquote \id{\}}            = 
  \left\{ 
    \begin{array}{ccc}
      \lpquote Q \rpquote & & x \nameeq \lpquote P \rpquote \\
      x & & otherwise \\
    \end{array}
  \right. \nonumber
\end{eqnarray}

and $z$ is chosen distinct from $\quotep{P}$, $\quotep{Q}$, the free
names in $Q$, and all the names in $R$. Our $\alpha$-equivalence will
be built in the standard way from this substitution.

\begin{remark}\label{rem:no_self_referential_names}
  One consequence of these definitions is that $\forall P. \quotep{P}
  \not\in \freenames{P}$.
\end{remark}

\subsection{ Dynamic quote: an example }

Anticipating something of what's to come, consider applying the
substitution, $\widehat{\id{\{}u / z \id{\}}}$, to the following pair
of processes, $\lift{w}{y!(z)}$ and $w[ \lpquote y!(z) \rpquote ]$.

\begin{eqnarray}
	\lift{w}{y!(z)}\widehat{\id{\{}u / z \id{\}}}
		& = &
		\lift{w}{y!(u)} \nonumber\\
	w[ \lpquote y!(z) \rpquote ] \widehat{ \id{\{}u / z \id{\}} }
		& = &
		w[ \lpquote y!(z) \rpquote ] \nonumber
\end{eqnarray}

Because the body of the process between quotes is impervious to
substitution, we get radically different answers. In fact, by
examining the first process in an input context,
e.g. $x?(z).\lift{w}{y!(z)}$, we see that the process under the lift
operator may be shaped by prefixed inputs binding a name inside it. In
this sense, the lift operator will be seen as a way to dynamically
construct processes before reifying them as names.

Finally equipped with these standard features we can present the
dynamics of the calculus.

\subsubsection{Operational semantics} 

Finally, we introduce the computational dynamics. What marks these
algebras as distinct from other more traditionally studied algebraic
structures, e.g. vector spaces or polynomial rings, is the manner in
which dynamics is captured. In traditional structures, dynamics is typically
expressed through morphisms between such structures, as in linear maps
between vector spaces or morphisms between rings. In algebras
associated with the semantics of computation, the dynamics is
expressed as part of the algebraic structure itself, through a
reduction reduction relation typically denoted by $\red$. Below, we
give a recursive presentation of this relation for the calculus used
in the encoding.

$\red \subseteq \pi \times \pi$
$\red : \pi \to \mathcal{P}(\pi)$

\begin{mathpar}
  \inferrule* [lab=Comm] { \textsf{match}( x_{src}, x_{trgt} ) } { x_{trgt}?(y)P \; | \; x_{src}!\langle {Q} \rangle \red P\{\quotep{Q}/y}\} }
  \and \\
  \inferrule* [lab=Par] {{P} \red {P}'} {{{P} | {Q}} \red {{P}' | {Q}}}
  \and
  \inferrule* [lab=Equiv]{{{P} \scong {P}'} \andalso {{P}' \red {Q}'} \andalso {{Q}' \scong {Q}}}{{P} \red {Q}}
\end{mathpar}

\begin{eqnarray*}
  match_{\equiv} (\quotep{P},\quotep{Q}) & := & P \equiv Q \\
  match_{\dagger}(\quotep{P},\quotep{Q}) & := & \forall R. P|Q \red^{*} R => R \red^{*} 0 \\
  match_{K}(\quotep{P},\quotep{Q}) & := & K \mbox{ for some context } K
\end{eqnarray*}

$u?(x)P | u!\langle Q \rangle \red P\{\quotep{Q}/x\}$

%We write $\wred$ for $\red^*$, and $P\red$ if $\exists Q $ such that $ P \red Q$.
We write $P\red$ if $\exists Q $ such that $ P \red Q$ and $P\not\red$, otherwise.

\section{Replication}

As mentioned before, it is known that replication (and hence
recursion) can be implemented in a higher-order process algebra
\cite{SangiorgiWalker}. As our first example of calculation with the
machinery thus far presented we give the construction explicitly in
the {\rhoc}.

\begin{eqnarray}
	D_{x} & := & \prefix{x}{y}{(\binpar{\outputp{x}{y}}{@{y}})} \nonumber\\
	\bangp_{x}{P} & := & \binpar{{x}!\langle{\binpar{D_{x}}{P}}\rangle}{D_{x}} \nonumber
\end{eqnarray}

\begin{eqnarray}
	\bangp_{x}{P} & & \nonumber\\
	=
	& {x}!\langle{(\prefix{x}{y}{(\outputp{x}{y} | @{y})) | P}}\rangle 
	      | \prefix{x}{y}{(\outputp{x}{y} | @{y})} & \nonumber\\
	\red
	& (\outputp{x}{y} | @{y})\substn{\quotep{(\prefix{x}{y}{(@{y} | \outputp{x}{y})) | P}}}{y} & \nonumber\\
	=
	& \outputp{x}{\quotep{(\prefix{x}{y}{(\outputp{x}{y} | @{y})) | P}}}
	  | {(\prefix{x}{y}{(\outputp{x}{y} | @{y})) | P}} & \nonumber\\
	\red
	& \ldots & \nonumber\\
	\red^*
	& P | P | \ldots & \nonumber
\end{eqnarray}

Of course, this encoding, as an implementation, runs away, unfolding
$\bangp{P}$ eagerly. A lazier and more implementable replication
operator, restricted to input-guarded processes, may be obtained as follows.

\begin{eqnarray}
\bangp{\prefix{u}{v}{P}} 
	:= 
	\binpar{\lift{x}{\prefix{u}{v}{(\binpar{D(x)}{P})}}}{D(x)} \nonumber
\end{eqnarray}

\begin{remark}
  Note that the lazier definition still does not deal with summation
  or mixed summation (i.e. sums over input and output). The reader is
  invited to construct definitions of replication that deal with these
  features. 

  Further, the definitions are parameterized in a name, $x$. Can you,
  gentle reader, make a definition that eliminates this parameter and
  guarantees no accidental interaction between the replication
  machinery and the process being replicated -- i.e. no accidental
  sharing of names used by the process to get its work done and the
  name(s) used by the replication to effect copying. This latter
  revision of the definition of replication is crucial to obtaining
  the expected identity $!!P \sim !P$.
\end{remark}

\begin{remark}\label{rem:paradoxical_combinator}
  The reader familiar with the lambda calculus will have noticed the
  similarity between $D$ and the paradoxical combinator.

  [Ed. note: the existence of this seems to suggest we have to be more
  restrictive on the set of processes and names we admit if we are to
  support no-cloning.]
\end{remark}

\subsubsection{Bisimulation}

The computational dynamics gives rise to another kind of equivalence,
the equivalence of computational behavior. As previously mentioned
this is typically captured \emph{via} some form of bisimulation.

% The notion we use in this paper is weak barbed bisimulation
% \cite{milner91polyadicpi}.

The notion we use in this paper is derived from weak barbed
bisimulation \cite{milner91polyadicpi}. 

\begin{definition}
An \emph{observation relation}, $\downarrow_{\mathcal N}$, over a set
of names, $\mathcal N$, is the smallest relation satisfying the rules
below.

\infrule[Out-barb]{y \in {\mathcal N}, \; x \nameeq y}
		  {\outputp{x}{v} \downarrow_{\mathcal N} x}
\infrule[Par-barb]{\mbox{$P\downarrow_{\mathcal N} x$ or $Q\downarrow_{\mathcal N} x$}}
		  {\binpar{P}{Q} \downarrow_{\mathcal N} x}

We write $P \Downarrow_{\mathcal N} x$ if there is $Q$ such that 
$P \wred Q$ and $Q \downarrow_{\mathcal N} x$.
\end{definition}

\begin{definition}
%\label{def.bbisim}
An  ${\mathcal N}$-\emph{barbed bisimulation} over a set of names, ${\mathcal N}$, is a symmetric binary relation 
${\mathcal S}_{\mathcal N}$ between agents such that $P\rel{S}_{\mathcal N}Q$ implies:
\begin{enumerate}
\item If $P \red P'$ then $Q \wred Q'$ and $P'\rel{S}_{\mathcal N} Q'$.
\item If $P\downarrow_{\mathcal N} x$, then $Q\Downarrow_{\mathcal N} x$.
\end{enumerate}
$P$ is ${\mathcal N}$-barbed bisimilar to $Q$, written
$P \wbbisim_{\mathcal N} Q$, if $P \rel{S}_{\mathcal N} Q$ for some ${\mathcal N}$-barbed bisimulation ${\mathcal S}_{\mathcal N}$.
\end{definition}

$\mathcal{R} \subseteq \pi \times \pi$

$P \mathcal{R} Q => \forall P'. P \red P' \Rightarrow \exists Q'. Q \red Q', P' \mathcal{R} Q'$

$P \vdash x \Rightarrow Q \vdash x$

\begin{mathpar}
  \inferrule*[lab=Out-barb]{x \nameeq y}{{y}!\langle{Q}\rangle \vdash x}
  \and
  \inferrule*[lab=Par-barb]{\mbox{$P\vdash x$ or $Q\vdash x$}}{\binpar{P}{Q} \vdash x}
\end{mathpar}

\subsubsection{Contexts}

One of the principle advantages of computational calculi like the
$\pi$-calculus is a well-defined notion of context,
contextual-equivalence and a correlation between
contextual-equivalence and notions of bisimulation. The notion of
context allows the decomposition of a process into (sub-)process and
its syntactic environment, its context. Thus, a context may be
thought of as a process with a ``hole'' (written $\Box$) in it. The
application of a context $M$ to a process $P$, written $M[P]$, is
tantamount to filling the hole in $M$ with $P$. In this paper we do
not need the full weight of this theory, but do make use of the notion
of context in the proof the main theorem. 

\begin{mathpar}
  \inferrule* [lab=summation] {} {{M_{M},M_{N}} \bc \Box \;|\; x.M_{A} \;|\; M_{M}+M_{N}}
  \and
  \inferrule* [lab=agent] {} {{M_{A}} \bc (\vec{x})M_{P} \;| \; \clift{P_0,\ldots,M_{P},\ldots,P_N}}
  \and \\
  \inferrule* [lab=process] {} {{M_{P}} \bc M_{N} \;| \;P|M_{P} }
\end{mathpar} 

\begin{mathpar}
  \inferrule* [lab=sychronization] {} {M_{N} \bc \Box \;|\; x?M_{F} \;|\; x!M_{C}}
  \and
  \inferrule* [lab=abstraction] {} {{M_{F}} \bc (x)M_{P} }
  \and
  \inferrule* [lab=concretion] {} {{M_{C}} \bc \langle M_{P} \rangle }
  \and \\
  \inferrule* [lab=process] {} {{M_{P}} \bc M_{N} \;| \;P|M_{P} }
\end{mathpar}

\begin{definition}[contextual application] Given a context $M$, and
  process $P$, we define the \emph{contextual application}, $M[P] :=
  M\{P/\Box\}$. That is, the contextual application of M to P is the
  substitution of $P$ for $\Box$ in $M$.
\end{definition}

$\meaningof{-} : L \to \mathcal{P}(\pi)$

\begin{mathpar}
  \inferrule* [lab=collection] {} {\meaningof{true} = \pi, \and \meaningof{~E} = \pi \setminus \meaningof{E}, \and \meaningof{E_{1} \& E_{2}} = \meaningof{E_{1}} \cap \meaningof{E_{2}}}
\end{mathpar}

\begin{mathpar}
  \inferrule* [lab=structure] {} {\meaningof{0} = \{ P \in \pi | P \equiv 0 \}, \and \\ \meaningof{E_1 | E_2} = \{ P \in \pi | P \equiv P_{1} | P_{2}, P_{1} \in \meaningof{E_{1}}, P_{2} \in \meaningof{E_2}\} }
\end{mathpar}

\begin{mathpar}
 \inferrule* [lab=behavior] {} {\meaningof{\langle a?b \rangle E} = \{ P \in \pi | P \equiv Q | u?(y)P', \\ \and \\\\ \and \\ \;\;\; u \in \meaningof{a}, \forall z.P'\{z/y\} \in \meaningof{E\{z/b\}}\}, \and \\ \meaningof{a!E} = \{ P \in \pi | P \equiv Q | x!\langle P' \rangle, x \in \meaningof{a} P' \in \meaningof{E}\} }
\end{mathpar}

\begin{mathpar}
 \inferrule* [lab=nominal] {} {\meaningof{\quotep{E}} = \{ \quotep{P} \in \quotep{\pi} | P \in \meaningof{E} \}, \and \meaningof{\quotep{P}} = \{ \quotep{Q} \in \quotep{\pi} | P \equiv Q \} \and \\ \meaningof{@\quotep{E}} = \{ P \in \pi | P \equiv @x, x \in \meaningof{E} \}}
\end{mathpar}

\begin{eqnarray*}
  \\
  \meaningof{-} : TS \to ST
\end{eqnarray*}

\begin{eqnarray*}
  \\
  L : TS \to ST
\end{eqnarray*}

\begin{eqnarray*}
  \\
  P \models E \iff P \in \meaningof{E}
\end{eqnarray*}

\begin{eqnarray*}
  P \approx_{L} Q \iff \forall E \in L. P \models E \iff Q \models E
\end{eqnarray*}

\begin{eqnarray*}
  P \approx_{K} Q
\end{eqnarray*}

\begin{eqnarray*}
  P \approx Q
\end{eqnarray*}

$\approx_{K} = \approx = \approx_{L}$

\subsubsection{Contextual duality}

Note that contexts extend the quotation operation to a family of
operations from processes to names. Given a context, $M$, we can
define a \emph{nominal context}, $\quotep{M}$ by $\quotep{M}[P] :=
\quotep{M[P]}$. To foreshadow what is to come we observe that these
operations enjoy a duality with processes very much like the duality
between vectors and maps from vectors to scalars.

Further, because the calculus is essentially higher-order, we have a
correspondence between contexts and processes. More specifically,
given a name $x$ and a context $M$ we can construct $M^{*}_{x}$ such
that 

\begin{mathpar}
  M^{*}_{x} | \lift{x}{P} \red M[P]
\end{mathpar}

namely,

\begin{mathpar}
  M^{*}_{x} := x?(u).M[\dropn{u}]
\end{mathpar}

The dependence of $M^{*}_{x}$ on a name makes it an abstraction, 

\begin{mathpar}
  M^{*} := (x)x?(u).M[\dropn{u}]
\end{mathpar}

\subsection{Additional notation}

It will sometimes be convenient to denote the process a name
quotes. We already have the notation $x = \quotep{P}$, but it will be
convenient to introduce an alternate notation, $\procn{x}$, when we
want to emphasize the connection to the use of the name. Note that, by
virtue of name equivalence, $\quotep{\procn{x}} \nameeq x$; so, the
notation is consistent with previous definitions.

Further, because names have structure it is possible to effect
substitutions on the basis of that structure. This means we need to
upgrade our notation for substitutions, which we accomplish by
adapting comprehension notation. Thus,

\begin{mathpar}
  P\{ y / x : x \in S \}
\end{mathpar}

is interpreted to mean the process derived from P by replacing (in a
capture-avoiding manner) each occurrence of $x$ in $S$ by $y$. For example,

\begin{mathpar}
  P\{ \quotep{\procn{x}|\procn{x}} / x : x \in \freenames{P} \}
\end{mathpar}

will replace each (occurrence) of a free name $x$ in $P$ by
$\quotep{\procn{x}|\procn{x}}$.

Also, we will avail ourselves of the notation $x^{L}$ and $x^{R}$ to
denote injections of a name into disjoint copies of the name
space. There are numerous ways to accomplish this. One example can be
found in \cite{MeredithR05}. This notation overloads to vectors of
names: $\vec{x}^{\pi} := (x_{i}^{\pi} \; : \; 0 \leq i < |\vec{x}| )$ where $\pi \in \{L,R\}$.

We also use $P^{\Box} := P|\Box$.

In \cite{MeredithR05} an interpretation of the new operator is
given. It turns out that there are several possible interpretations
all enjoying the requisite algebraic properties of the operator (see
\cite{milner91polyadicpi}). We will therefore make liberal use of
$(\nu\; \vec{x})P$.

% subsection the_syntax_and_semantics_of_the_notation_system (end)   

\section{Interpretation of QM}
\subsection{Supporting definitions}
\subsubsection{Multiplication}
\begin{mathpar}
  \quotep{Q} \cdot \quotep{R} := \quotep{Q|R}
  \and \\
  \quotep{Q} \cdot P := P\{ \quotep{Q|R} / \quotep{R} : \quotep{R} \in \freenames{P} \}
\end{mathpar}

\paragraph{Discussion}
The first line needs little explanation. The second line says that
each free name of the process is replaced with the multiplication of
that name by the scalar. Multiplication of a scalar (name) by a state
(process) results in a process all the names of which have been `moved
over' by parallel composition with the process the scalar
quotes. There is a subtlety that the bound names have to be
manipulated so that multiplied names aren't accidentally
captured. There are many ways to achieve this.

\begin{remark}\label{rem:multiplication_identities}
  The reader is invited to verify that for all $x,y,z \in \QProc$ and $P \in \Proc$
  \begin{mathpar}
    x \cdot \quotep{0} \equiv x 
    \and
    x \cdot y \equiv y \cdot x
    \and
    x \cdot (y \cdot z) \equiv (x \cdot y) \cdot z
    \and \\
    \quotep{0} \cdot P \equiv P
    \and \\
    x \cdot (y \cdot P) \equiv (x \cdot y) \cdot P
    \and \\
    x \cdot (P|Q) \equiv (x \cdot P) | (x \cdot Q)
    \and \\    
  \end{mathpar}
\end{remark}

\subsubsection{Tensor product}

We define a tensor product on processes by structural induction.

\paragraph{Tensor of sums} First note that all summations, including
$\pzero$ and sequence, can be written $\Sigma_{i} x_{i}.A_{i} +
\Sigma_{j} x_{j}.C_{j}$, where we have grouped input-guarded processes
together and output-guarded processes together.

Thus, we can define the tensor product of two summations, $N_{1}\otimes N_{2}$, where

\begin{mathpar}
  N_{1} := \Sigma_{i} x_{i}.A_{i} + \Sigma_{j} x_{j}.C_{j}
  \and
  N_{2} := \Sigma_{i'} y_{i'}.B_{i'} + \Sigma_{j'} y_{j'}.D_{j'} 
\end{mathpar}

as follows.

\begin{mathpar}
  \Sigma_{i} x_{i}.A_{i} + \Sigma_{j} x_{j}.C_{j} \otimes \Sigma_{i'}
  y_{i'}.B_{i'} + \Sigma_{j'} y_{j'}.D_{j'} 
  \and \\
  := \; \Sigma_{i} \Sigma_{i'} \quotep{\stackrel{\vee}{x_{i}}| \stackrel{\vee}{y_{i'}}}.(A_{i}\otimes B_{i'}) \; | \; \Sigma_{i'} \Sigma_{i} \quotep{\stackrel{\vee}{y_{i'}}|\stackrel{\vee}{x_{i}}}.(B_{i'}\otimes A_{i})
  \and
  \;\; | \;\; \Sigma_{j} \Sigma_{j'} \quotep{\stackrel{\vee}{x_{j}}|\stackrel{\vee}{y_{j'}}}.(A_{j}\otimes B_{j'}) \; | \; \Sigma_{j'} \Sigma_{j} \quotep{\stackrel{\vee}{y_{j'}}|\stackrel{\vee}{x_{j}}}.(B_{j'}\otimes A_{j})
\end{mathpar}

\begin{remark}
  Do we need to $x^{L}$ and $y^{R}$ for this construction as well?
\end{remark}

\paragraph{Tensor of parallel compositions} Next, we distribute tensor
over par.

\begin{mathpar}
  P_{1}|P_{2} \otimes Q_{1}|Q_{2} := (P_{1} \otimes Q_{1}) | (P_{1}
  \otimes Q_{2}) | (P_{2} \otimes Q_{1}) | (P_{2} \otimes Q_{2})
\end{mathpar}

\paragraph{Tensor with dropped names} We treat tensor of a
process with a dropped name as parallel composition.

\begin{mathpar}
  P \otimes \dropn{x} := P | \dropn{x}
\end{mathpar}

\paragraph{Tensor of agents}

Finally, we need to define tensor on agents. Note that the definition
of tensor on normal products only tensors inputs with inputs and
outputs with outputs. Thus, we only have to define the operation on
``homogeneous'' pairings.

\begin{mathpar}
  (\vec{x})P \otimes (\vec{y})Q
  \and \\
  := (x_{0}^{L}|y_{0}^{R},\ldots,x_{0}^{L}|y_{n}^{R},\ldots,x_{m}^{L}|y_{0}^{R},\ldots,x_{m}^{L}|y_{n}^R)(P\{ \vec{x}^{L}/\vec{x}\} \otimes Q \{ \vec{y}^{R}/\vec{y}\})
  \and \\
  \clift{\vec{P}} \otimes \clift{\vec{Q}}
  \and \\
  := \clift{P_{0}\otimes Q_{0},\ldots,P_{0}\otimes Q_{n},\ldots,P_{m}\otimes Q_{0},\ldots,P_{m}\otimes Q_{n}}
\end{mathpar}

\begin{remark}
  Observe that arities of tensored abstractions matches arities of
  tensored concretions if the original arities matched. Note also that
  the length of the arities corresponds to the increase in dimension
  we see in ordinary vector space tensor product.
\end{remark}

\begin{remark}
  Operationally, this definition distributes the tensor down to
  components ``linked'' by summation. Tensor over summation is
  intriguing in that it mixes names. Moreover, as a consequence of the
  way it mixes names we have the identities for all $x \in \QProc$ and
  $P,Q \in \Proc$

  \begin{mathpar}
    (x \cdot P) \otimes Q \equiv x \cdot (P \otimes Q) \equiv P \otimes (x \cdot Q)
    \and
    P \otimes \pzero \equiv P
  \end{mathpar}

  that the reader is invited to verify.
\end{remark}

\subsubsection{Annihilation}
\begin{mathpar}
  P^{\perp} := \{ Q | \forall R. P|Q \red^{*} R \Rightarrow R \red^{*} \pzero \}
  \and \\
  P^{\underline{\perp}} := \Sigma_{Q \in P^{\perp}} \quotep{Q}?(y).(\dropn{y}|Q) | \Sigma_{Q \in P^{\perp}} \quotep{Q}\clift{\Box}
\end{mathpar}

\paragraph{Discussion} The reader will note that $P^{\perp}$ is a
\emph{set} of processes, while $P^{\underline{\perp}}$ is a
\emph{context}. We call the set $P^{\perp}$ the \emph{annihilators} of
$P$. The parallel composition of a process in the annihilators of $P$
with $P$ will result in a process, the state space of which has all
paths eventually leading to $\pzero$. Execution may endure loops; but
under reasonable conditions of fairness (naturally guaranteed under
most notions of bisimulation) such a composite process cannot get
stuck in such a loop and will, eventually pop out and terminate.

The context $P^{\underline{\perp}}$ is ready and willing to ``take the
$P$ out of'' the process to which it is applied. It will effectively
transmit the code of the process to which it is applied to one of the
annihilators and run the process against it.

\subsubsection{Evaluation}
We fix $M$ a domain of fully abstract interpretation with an equality
coincident with bisimulation. We take $\meaningof{\cdot} : \Proc \to
M$ to be the map interpreting processes and $\nmeaningof{\cdot} : \M
\to Proc$ to be the map running the other way. Then we define

\begin{mathpar}
  \int P := \nmeaningof{\meaningof{P}}
\end{mathpar}

\paragraph{Discussion}
There are many fully abstract interpretations of Milner's
$\pi$-calculus. Any of them can be used as a basis for interpreting
the reflective calculus here. Equipped with such a domain it is
largely a matter of grinding through to check that the Yoneda
construction for the normalization-by-evaluation program can be
extended to this setting.

\begin{remark}
  The reader is invited to verify that $\int (P^{\underline{\perp}}[P]) = 0$.
\end{remark}

\subsection{Quantum mechanics}

Table \ref{tbl:core_qm_op_defns} gives the core operational definitions

\begin{table}[htp]\label{tbl:core_qm_op_defns}
  \center{
    \fbox{
      \begin{tabular}{c|c}
        quantum mechanics & process calculus \\
        \hline
        scalar & $x := \quotep{P}$ \\
        state vector & $\state{P} := P$ \\
        dual & $\state{P}^{*} := \event{P^{\underline{\perp}}} := \quotep{P^{\underline{\perp}}}[-]$ \\
        matrix & $ \Sigma_{\alpha} \state{P_{\alpha}}x_{\alpha}\event{Q_{\alpha}}$ \\
        vector addition & $\state{P} + \state{Q} := \state{P | Q}$ \\
        tensor product & $\state{P} \otimes \state{Q} := \state{P \otimes Q}$ \\
        inner product & $\innerprod{P}{Q} := \quotep{\int P^{\underline{\perp}}[Q]}$ \\
      \end{tabular}
    }
  }
  \caption{QM - operational definitions}
\end{table}

where

\begin{mathpar}
  \prmatrix{P}{Q} := \fprmatrix{P}{\quotep{\pzero}}{Q}
  \and
  \fprmatrix{P}{x}{Q} := (\state{P},x,\event{Q})
  \and
  (\fprmatrix{P}{x}{Q})(\state{R}) := x \cdot \innerprod{Q}{R} \cdot \state{P}
  \and
  (\fprmatrix{P}{x}{Q})(\event{R}) := x \cdot \innerprod{R}{P} \cdot \event{Q}
\end{mathpar}

\paragraph{Discussion}
As promised: vectors (aka states) are represented as processes; duals
as contextual duals; inner product definition should be compared with
standard inner product definition for ....

\begin{remark}
  Assuming $\int (P^{\underline{\perp}}[P]) = 0$, the reader is
  invited to verify that $(\fprmatrix{P}{x}{P})(\state{P}) = x \cdot \state{P}$.
\end{remark}

\begin{remark}
  The reader is invited to verify that $\innerprod{P}{Q}$ could
  equally well have been written $\quotep{\int \stackrel{\vee}{x}}$
  where $x = \event{P^{\underline{\perp}}}(Q)$.

  One of the motivations for this remark is that there is another way
  to factor these operations. We could package up evaluation in the dual:

  \begin{mathpar}
    \state{P}^{*} := \event{\int P^{\underline{\perp}}} := \quotep{\int P^{\underline{\perp}}}[-]
  \end{mathpar}

  and then have inner product defined by
  
  \begin{mathpar}
    \innerprod{P}{Q} := \event{P}(Q)
  \end{mathpar}

  Hopefully, experience with the calculations will provide guidance on
  the best factoring.
\end{remark}

\begin{remark}
  Assuming $\int (P^{\underline{\perp}}[P]) = 0$, the reader is
  invited to verify that $\forall P,Q. (\prmatrix{0}{Q})(\state{0}) =
  \state{0}$ and dually $(\prmatrix{P}{0})(\event{0}) = \event{0}$.
\end{remark}

\begin{remark}
  i'm a little worried that i don't (yet) have proper support for
  complex conjugacy. But, the observation above may give us a
  clue. According to Abramsky, it must be the case that the scalars
  are iso to the homset of the identity for the tensor -- which the
  observation above characterizes. 

  For now, we will simply bookmark the notion with $\overline{x}$.
\end{remark}

\subsubsection{Adjointness}

We need to give a definition of $(\cdot)^{\dagger}$ for matrices. The
obvious candidate definition is
\begin{mathpar}
(\Sigma_{\alpha}\fprmatrix{P_{\alpha}}{x_{\alpha}}{Q_{\alpha}})^{\dagger}
= \Sigma_{\alpha}\fprmatrix{(Q_{\alpha}^{\underline{\perp}})^{*}}{\overline{x}_{\alpha}}{P_{\alpha}^{\underline{\perp}}} 
\end{mathpar}

But, $(Q_{\alpha}^{\underline{\perp}})^{*}$ requires a name along
which to communicate the process to achieve the context application.

\subsubsection{Basis for a basis}
If processes label states and ``addition'' of states (a.k.a. vector
addition) is interpreted as parallel composition, what corresponds to
notions of linear independence and basis? Here, we recall that Yoshida
has developed a set of \emph{combinators} for an asynchronous verison
of Milner's $\pi$-calculus. These are a finite set of processes such
any process can be expressed as parallel composition of these
combinators together with liberal uses of the new operator and
replication. We can simply give a translation of these into the
present calculus and have reasonable expectation that the property
carries over. That is, that the resultant set allows to express all
processes via parallel composition. Note, however, that there is no
new operator or replication in this calculus. As a result, we expect
that the corresponding set is actually infinite. That is, we expect
that the space is actually infinite dimensional.

\begin{remark}
  The attentive reader may be a bit concerned. Certainly, the
  collection $S$, $K$ and $I$ is a finite set of
  combinators. Shouldn't we expect to see a finite set of combinators
  for an effectively equivalent system? i am very sympathetic to this
  critique and feel it warrants full attention. On the other hand, i
  also have in mind the following analogy. The natural numbers, as a
  monoid under addition, has exactly $1$ generator, while the natural
  numbers, as a monoid under multiplication, has countably many
  generators (the primes). We observe that the application of the
  lambda calculus is much less resource sensitive than the parallel
  composition of the $\pi$-calculus. Could it be the case that we have
  an analogy of the form
  
  \begin{mathpar}
    m + n : MN :: m*n : M|N
  \end{mathpar}

  giving a similar blow up in the set of ``primes''?  This is such a
  wonderful thought that, even if it's not true, i think it's worth
  writing down.
\end{remark}
 

\documentclass[12pt]{llncs}
%\documentclass{jktr}

\usepackage[pdftex]{hyperref}                   
\usepackage {listings}
\usepackage {mathpartir}
\usepackage{bcprules}
%\usepackage{listings}
                       
\usepackage{graphicx} 
%\usepackage[margins=2.5cm,nohead,nofoot]{geometry}
%\usepackage{geometry}
\usepackage{amsfonts}
\usepackage{amstext}
\usepackage{latexsym}
\usepackage{amssymb}
\usepackage{color}


%\include{myPreamble}
\include{qm2pi.local} 

%\ifpdf
%\usepackage[pdftex]{graphicx}
%\else
%\usepackage{graphicx}
%\fi

 % \ifpdf
%  \usepackage{pdfsync}
%  \if


%\title{Brief Article}
%\author{David F. Snyder}
%\author{L.G. Meredith}

%\address{Dept. of Math., Texas State University--San Marcos, San Marcos, TX 78666}
       
\pagestyle{empty}


\begin{document}

\lstset{language=[Objective]Caml,frame=shadowbox}

\input{qm2pi.front}

% section front matter (end)

\input{qm2pi.intro} 
 
% section introduction (end)

% \input{qm2pi.knotations} 

% section notation (end)

\input{qm2pi.process.calculi} 

% section concurrent_process_calculi_and_spatial_logics_ (end)
    
%\input{qm2pi.knots2pi} 

%\input{qm2pi.trefoil} 

%\input{qm2pi.mainthm} 

% subsection basic_interpretation (end)

%\input{qm2pi.rho.presentation} 
\subsection{The syntax and semantics of the notation system}\label{sub:the_syntax_and_semantics_of_the_notation_system} % (fold)

We now summarize a technical presentation of the calculus that
embodies our theory of dynamics. The typical presentation of such a
calculus follows the style of giving generators and relations on
them. The grammar, below, describing term constructors, freely
generates the set of processes, $\Proc$. This set is then quotiented
by a relation known as structural congruence and it is over this set
that the notion of dynamics is expressed. This presentation is
essentially that of \cite{MeredithR05} with the addition of
polyadicity and summation. For readability we have relegated some of
the technical subtleties to an appendix.

\subsubsection{Process grammar}\label{subsub:process_grammar}

\begin{mathpar}
  \inferrule* [lab=synchronization] {} {{M} \bc \pzero \;|\; x?F \;|\; x!C }
  \and
  \inferrule* [lab=abstraction] {} {{F} \bc (x)P}
  \and
  \inferrule* [lab=concretion] {} {{C} \bc \langle Q \rangle}
  \and
  \inferrule* [lab=process] {} {{P,Q} \bc M \;| \;P|Q \;|\; @{x}}
  \and
  \inferrule* [lab=name] {} {{x} \bc \quotep{P}}
\end{mathpar} 

Note that $\vec{x}$ (resp. $\vec{P}$) denotes a vector of names
(resp. processes) of length $|\vec{x}|$ (resp. $|\vec{P}|$). We adopt
the following useful abbreviations.

\begin{mathpar}
   x?(\vec{y}).P := x.(\vec{y})P \and  x\clift{\vec{P}} := x.\clift{\vec{P}}
   \and x!(y) := \lift{x}{\dropn{y}}
   \and \Pi_{i=0}^{n-1}P_i := P_0 | \ldots | P_{n-1}
\end{mathpar}

\subsubsection{Structural congruence}

\paragraph{Free and bound names and alpha-equivalence.} At the
core of structural equivalence is alpha-equivalence which identifies
process that are the same up to a change of variable. Formally, we
recognize the distinction between free and bound names. The free names
of a process, $\freenames{P}$, may be calculated recursively as
follows:

\begin{mathpar}
\freenames{\pzero} := \emptyset
  \and \\
  \freenames{x?(y).P} := \{ x \} \cup (\freenames{P} \setminus \{ y \})
  \and 
  \freenames{x!\langle P \rangle} := \{ x \} \cup \{ P \} 
  \and \\
  \freenames{P|Q} := \freenames{P} \cup \freenames{Q}
  \and \\
  \freenames{@{x}} := \{ x \}
\end{mathpar}

$\pi$
$\quotep{\pi}$

$\freenames{-} : \pi \to \mathcal{P}(\quotep{\pi})$

\begin{eqnarray*}
  \freenames{\pzero} & := & \emptyset \\
  \freenames{x?(y).P} & := & \{ x \} \cup (\freenames{P} \setminus \{ y \}) \\
  \freenames{x!\langle P \rangle} & := & \{ x \} \cup \{ P \} \\
  \freenames{P|Q} & := & \freenames{P} \cup \freenames{Q} \\
  \freenames{\dropn{x}} & := & \{ x \}
\end{eqnarray*}

The bound names of a process, $\boundnames{P}$, are those names occurring in $P$
that are not free. For example, in $x?(y).0$, the name $x$ is free, while $y$ is bound.

\begin{mathpar}
  \inferrule* [lab=monoidal-laws] {} { P|Q \equiv Q|P \and P|0 \equiv P \and P|(Q|R) \equiv (P|Q)|R }
\end{mathpar}

\begin{mathpar}
  \inferrule* [lab=alpha-equivalence] {} { (x)P \equiv (y)P\{y/x\} \and y \not\in \freenames{P} }
\end{mathpar}

\begin{definition}
Then two processes, $P,Q$, are alpha-equivalent if $P = Q\{\vec{y}/\vec{x}\}$ for
some $\vec{x} \in \boundnames{Q},\vec{y} \in \boundnames{P}$, where $Q\{\vec{y}/\vec{x}\}$
denotes the capture-avoiding substitution of $\vec{y}$ for $\vec{x}$ in $Q$.
\end{definition}

\begin{definition}
  The {\em structural congruence} \cite{SangiorgiWalker} , $\equiv$,
  between processes is the least congruence containing
  alpha-equivalence, satisfying the abelian monoid laws
  (associativity, commutativity and $\pzero$ as identity) for parallel
  composition $|$ and for summation $+$.
\end{definition}

\subsection{Name equivalence}

We take name equivalence, written $\nameeq$, to be the smallest
equivalence relation generated by the following rules.

\begin{mathpar}
\inferrule*[lab=Quote-drop]
{ }
{ \quotep{@{x}} \nameeq x }

\inferrule*[lab=Struct-equiv]
{ P \scong Q }
{ \quotep{P} \nameeq \quotep{Q} }
\end{mathpar}

The astute reader will have noticed that the mutual recursion of names
and processes imposes a mutual recursion on alpha-equivalence and
structural equivalence via name-equivalence. Fortunately, all of this
works out pleasantly and we may calculate in the natural way, free of
concern. The reader interested in the details is referred to the
appendix \ref{appendix:rho_details}.

\subsection{Substitution}

We use $\Proc$ for the set of processes, $\QProc$ for the set of
names, and $\id{\{}\vec{y} / \vec{x} \id{\}}$ to denote partial maps,
$s : \QProc \rightarrow \QProc$. A map, $s$ lifts, uniquely, to a map
on process terms, $\widehat{s} : \Proc \rightarrow \Proc$ by the
following equations.

\begin{mathpar}
  (0) \psubstp{Q}{P} := 0 \\
  (R \juxtap S) \psubstp{Q}{P}
  :=    
  (R)\psubstp{Q}{P} \juxtap (S) \psubstp{Q}{P} \\
  (x?(y).R) \psubstp{Q}{P}    
  :=    
  (x)\substp{Q}{P} (z)\concat( (R \psubstn{z}{y}) \psubstp{Q}{P} ) \\
  (\lift{x}{R}) \psubstp{Q}{P}  
  :=
  \lift{(x)\substp{Q}{P}}{ R \psubstp{Q}{P} } \\
%   (\dropn{x})  \psubstp{Q}{P}       
%   := 
%   \left\{ 
%     \begin{array}{ccc} 
%       \dropn{\quotep{Q}} & & x \nameeq \quotep{P} \\
%       \dropn{x} & & otherwise \\
%     \end{array}
%   \right. 
  (\dropn{x})  \psubstp{Q}{P}       
  := 
  \left\{ 
    \begin{array}{ccc} 
      Q & & x \nameeq \quotep{P} \\
      \dropn{x} & & otherwise \\
    \end{array}
  \right.
\end{mathpar}
 

where

\begin{eqnarray}
  (x)\id{\{} \lpquote Q \rpquote / \lpquote P \rpquote \id{\}}            = 
  \left\{ 
    \begin{array}{ccc}
      \lpquote Q \rpquote & & x \nameeq \lpquote P \rpquote \\
      x & & otherwise \\
    \end{array}
  \right. \nonumber
\end{eqnarray}

and $z$ is chosen distinct from $\quotep{P}$, $\quotep{Q}$, the free
names in $Q$, and all the names in $R$. Our $\alpha$-equivalence will
be built in the standard way from this substitution.

\begin{remark}\label{rem:no_self_referential_names}
  One consequence of these definitions is that $\forall P. \quotep{P}
  \not\in \freenames{P}$.
\end{remark}

\subsection{ Dynamic quote: an example }

Anticipating something of what's to come, consider applying the
substitution, $\widehat{\id{\{}u / z \id{\}}}$, to the following pair
of processes, $\lift{w}{y!(z)}$ and $w[ \lpquote y!(z) \rpquote ]$.

\begin{eqnarray}
	\lift{w}{y!(z)}\widehat{\id{\{}u / z \id{\}}}
		& = &
		\lift{w}{y!(u)} \nonumber\\
	w[ \lpquote y!(z) \rpquote ] \widehat{ \id{\{}u / z \id{\}} }
		& = &
		w[ \lpquote y!(z) \rpquote ] \nonumber
\end{eqnarray}

Because the body of the process between quotes is impervious to
substitution, we get radically different answers. In fact, by
examining the first process in an input context,
e.g. $x?(z).\lift{w}{y!(z)}$, we see that the process under the lift
operator may be shaped by prefixed inputs binding a name inside it. In
this sense, the lift operator will be seen as a way to dynamically
construct processes before reifying them as names.

Finally equipped with these standard features we can present the
dynamics of the calculus.

\subsubsection{Operational semantics} 

Finally, we introduce the computational dynamics. What marks these
algebras as distinct from other more traditionally studied algebraic
structures, e.g. vector spaces or polynomial rings, is the manner in
which dynamics is captured. In traditional structures, dynamics is typically
expressed through morphisms between such structures, as in linear maps
between vector spaces or morphisms between rings. In algebras
associated with the semantics of computation, the dynamics is
expressed as part of the algebraic structure itself, through a
reduction reduction relation typically denoted by $\red$. Below, we
give a recursive presentation of this relation for the calculus used
in the encoding.

$\red \subseteq \pi \times \pi$
$\red : \pi \to \mathcal{P}(\pi)$

\begin{mathpar}
  \inferrule* [lab=Comm] { \textsf{match}( x_{src}, x_{trgt} ) } { x_{trgt}?(y)P \; | \; x_{src}!\langle {Q} \rangle \red P\{\quotep{Q}/y}\} }
  \and \\
  \inferrule* [lab=Par] {{P} \red {P}'} {{{P} | {Q}} \red {{P}' | {Q}}}
  \and
  \inferrule* [lab=Equiv]{{{P} \scong {P}'} \andalso {{P}' \red {Q}'} \andalso {{Q}' \scong {Q}}}{{P} \red {Q}}
\end{mathpar}

\begin{eqnarray*}
  match_{\equiv} (\quotep{P},\quotep{Q}) & := & P \equiv Q \\
  match_{\dagger}(\quotep{P},\quotep{Q}) & := & \forall R. P|Q \red^{*} R => R \red^{*} 0 \\
  match_{K}(\quotep{P},\quotep{Q}) & := & K \mbox{ for some context } K
\end{eqnarray*}

$u?(x)P | u!\langle Q \rangle \red P\{\quotep{Q}/x\}$

%We write $\wred$ for $\red^*$, and $P\red$ if $\exists Q $ such that $ P \red Q$.
We write $P\red$ if $\exists Q $ such that $ P \red Q$ and $P\not\red$, otherwise.

\section{Replication}

As mentioned before, it is known that replication (and hence
recursion) can be implemented in a higher-order process algebra
\cite{SangiorgiWalker}. As our first example of calculation with the
machinery thus far presented we give the construction explicitly in
the {\rhoc}.

\begin{eqnarray}
	D_{x} & := & \prefix{x}{y}{(\binpar{\outputp{x}{y}}{@{y}})} \nonumber\\
	\bangp_{x}{P} & := & \binpar{{x}!\langle{\binpar{D_{x}}{P}}\rangle}{D_{x}} \nonumber
\end{eqnarray}

\begin{eqnarray}
	\bangp_{x}{P} & & \nonumber\\
	=
	& {x}!\langle{(\prefix{x}{y}{(\outputp{x}{y} | @{y})) | P}}\rangle 
	      | \prefix{x}{y}{(\outputp{x}{y} | @{y})} & \nonumber\\
	\red
	& (\outputp{x}{y} | @{y})\substn{\quotep{(\prefix{x}{y}{(@{y} | \outputp{x}{y})) | P}}}{y} & \nonumber\\
	=
	& \outputp{x}{\quotep{(\prefix{x}{y}{(\outputp{x}{y} | @{y})) | P}}}
	  | {(\prefix{x}{y}{(\outputp{x}{y} | @{y})) | P}} & \nonumber\\
	\red
	& \ldots & \nonumber\\
	\red^*
	& P | P | \ldots & \nonumber
\end{eqnarray}

Of course, this encoding, as an implementation, runs away, unfolding
$\bangp{P}$ eagerly. A lazier and more implementable replication
operator, restricted to input-guarded processes, may be obtained as follows.

\begin{eqnarray}
\bangp{\prefix{u}{v}{P}} 
	:= 
	\binpar{\lift{x}{\prefix{u}{v}{(\binpar{D(x)}{P})}}}{D(x)} \nonumber
\end{eqnarray}

\begin{remark}
  Note that the lazier definition still does not deal with summation
  or mixed summation (i.e. sums over input and output). The reader is
  invited to construct definitions of replication that deal with these
  features. 

  Further, the definitions are parameterized in a name, $x$. Can you,
  gentle reader, make a definition that eliminates this parameter and
  guarantees no accidental interaction between the replication
  machinery and the process being replicated -- i.e. no accidental
  sharing of names used by the process to get its work done and the
  name(s) used by the replication to effect copying. This latter
  revision of the definition of replication is crucial to obtaining
  the expected identity $!!P \sim !P$.
\end{remark}

\begin{remark}\label{rem:paradoxical_combinator}
  The reader familiar with the lambda calculus will have noticed the
  similarity between $D$ and the paradoxical combinator.

  [Ed. note: the existence of this seems to suggest we have to be more
  restrictive on the set of processes and names we admit if we are to
  support no-cloning.]
\end{remark}

\subsubsection{Bisimulation}

The computational dynamics gives rise to another kind of equivalence,
the equivalence of computational behavior. As previously mentioned
this is typically captured \emph{via} some form of bisimulation.

% The notion we use in this paper is weak barbed bisimulation
% \cite{milner91polyadicpi}.

The notion we use in this paper is derived from weak barbed
bisimulation \cite{milner91polyadicpi}. 

\begin{definition}
An \emph{observation relation}, $\downarrow_{\mathcal N}$, over a set
of names, $\mathcal N$, is the smallest relation satisfying the rules
below.

\infrule[Out-barb]{y \in {\mathcal N}, \; x \nameeq y}
		  {\outputp{x}{v} \downarrow_{\mathcal N} x}
\infrule[Par-barb]{\mbox{$P\downarrow_{\mathcal N} x$ or $Q\downarrow_{\mathcal N} x$}}
		  {\binpar{P}{Q} \downarrow_{\mathcal N} x}

We write $P \Downarrow_{\mathcal N} x$ if there is $Q$ such that 
$P \wred Q$ and $Q \downarrow_{\mathcal N} x$.
\end{definition}

\begin{definition}
%\label{def.bbisim}
An  ${\mathcal N}$-\emph{barbed bisimulation} over a set of names, ${\mathcal N}$, is a symmetric binary relation 
${\mathcal S}_{\mathcal N}$ between agents such that $P\rel{S}_{\mathcal N}Q$ implies:
\begin{enumerate}
\item If $P \red P'$ then $Q \wred Q'$ and $P'\rel{S}_{\mathcal N} Q'$.
\item If $P\downarrow_{\mathcal N} x$, then $Q\Downarrow_{\mathcal N} x$.
\end{enumerate}
$P$ is ${\mathcal N}$-barbed bisimilar to $Q$, written
$P \wbbisim_{\mathcal N} Q$, if $P \rel{S}_{\mathcal N} Q$ for some ${\mathcal N}$-barbed bisimulation ${\mathcal S}_{\mathcal N}$.
\end{definition}

$\mathcal{R} \subseteq \pi \times \pi$

$P \mathcal{R} Q => \forall P'. P \red P' \Rightarrow \exists Q'. Q \red Q', P' \mathcal{R} Q'$

$P \vdash x \Rightarrow Q \vdash x$

\begin{mathpar}
  \inferrule*[lab=Out-barb]{x \nameeq y}{{y}!\langle{Q}\rangle \vdash x}
  \and
  \inferrule*[lab=Par-barb]{\mbox{$P\vdash x$ or $Q\vdash x$}}{\binpar{P}{Q} \vdash x}
\end{mathpar}

\subsubsection{Contexts}

One of the principle advantages of computational calculi like the
$\pi$-calculus is a well-defined notion of context,
contextual-equivalence and a correlation between
contextual-equivalence and notions of bisimulation. The notion of
context allows the decomposition of a process into (sub-)process and
its syntactic environment, its context. Thus, a context may be
thought of as a process with a ``hole'' (written $\Box$) in it. The
application of a context $M$ to a process $P$, written $M[P]$, is
tantamount to filling the hole in $M$ with $P$. In this paper we do
not need the full weight of this theory, but do make use of the notion
of context in the proof the main theorem. 

\begin{mathpar}
  \inferrule* [lab=summation] {} {{M_{M},M_{N}} \bc \Box \;|\; x.M_{A} \;|\; M_{M}+M_{N}}
  \and
  \inferrule* [lab=agent] {} {{M_{A}} \bc (\vec{x})M_{P} \;| \; \clift{P_0,\ldots,M_{P},\ldots,P_N}}
  \and \\
  \inferrule* [lab=process] {} {{M_{P}} \bc M_{N} \;| \;P|M_{P} }
\end{mathpar} 

\begin{mathpar}
  \inferrule* [lab=sychronization] {} {M_{N} \bc \Box \;|\; x?M_{F} \;|\; x!M_{C}}
  \and
  \inferrule* [lab=abstraction] {} {{M_{F}} \bc (x)M_{P} }
  \and
  \inferrule* [lab=concretion] {} {{M_{C}} \bc \langle M_{P} \rangle }
  \and \\
  \inferrule* [lab=process] {} {{M_{P}} \bc M_{N} \;| \;P|M_{P} }
\end{mathpar}

\begin{definition}[contextual application] Given a context $M$, and
  process $P$, we define the \emph{contextual application}, $M[P] :=
  M\{P/\Box\}$. That is, the contextual application of M to P is the
  substitution of $P$ for $\Box$ in $M$.
\end{definition}

$\meaningof{-} : L \to \mathcal{P}(\pi)$

\begin{mathpar}
  \inferrule* [lab=collection] {} {\meaningof{true} = \pi, \and \meaningof{~E} = \pi \setminus \meaningof{E}, \and \meaningof{E_{1} \& E_{2}} = \meaningof{E_{1}} \cap \meaningof{E_{2}}}
\end{mathpar}

\begin{mathpar}
  \inferrule* [lab=structure] {} {\meaningof{0} = \{ P \in \pi | P \equiv 0 \}, \and \\ \meaningof{E_1 | E_2} = \{ P \in \pi | P \equiv P_{1} | P_{2}, P_{1} \in \meaningof{E_{1}}, P_{2} \in \meaningof{E_2}\} }
\end{mathpar}

\begin{mathpar}
 \inferrule* [lab=behavior] {} {\meaningof{\langle a?b \rangle E} = \{ P \in \pi | P \equiv Q | u?(y)P', \\ \and \\\\ \and \\ \;\;\; u \in \meaningof{a}, \forall z.P'\{z/y\} \in \meaningof{E\{z/b\}}\}, \and \\ \meaningof{a!E} = \{ P \in \pi | P \equiv Q | x!\langle P' \rangle, x \in \meaningof{a} P' \in \meaningof{E}\} }
\end{mathpar}

\begin{mathpar}
 \inferrule* [lab=nominal] {} {\meaningof{\quotep{E}} = \{ \quotep{P} \in \quotep{\pi} | P \in \meaningof{E} \}, \and \meaningof{\quotep{P}} = \{ \quotep{Q} \in \quotep{\pi} | P \equiv Q \} \and \\ \meaningof{@\quotep{E}} = \{ P \in \pi | P \equiv @x, x \in \meaningof{E} \}}
\end{mathpar}

\begin{eqnarray*}
  \\
  \meaningof{-} : TS \to ST
\end{eqnarray*}

\begin{eqnarray*}
  \\
  L : TS \to ST
\end{eqnarray*}

\begin{eqnarray*}
  \\
  P \models E \iff P \in \meaningof{E}
\end{eqnarray*}

\begin{eqnarray*}
  P \approx_{L} Q \iff \forall E \in L. P \models E \iff Q \models E
\end{eqnarray*}

\begin{eqnarray*}
  P \approx_{K} Q
\end{eqnarray*}

\begin{eqnarray*}
  P \approx Q
\end{eqnarray*}

$\approx_{K} = \approx = \approx_{L}$

\subsubsection{Contextual duality}

Note that contexts extend the quotation operation to a family of
operations from processes to names. Given a context, $M$, we can
define a \emph{nominal context}, $\quotep{M}$ by $\quotep{M}[P] :=
\quotep{M[P]}$. To foreshadow what is to come we observe that these
operations enjoy a duality with processes very much like the duality
between vectors and maps from vectors to scalars.

Further, because the calculus is essentially higher-order, we have a
correspondence between contexts and processes. More specifically,
given a name $x$ and a context $M$ we can construct $M^{*}_{x}$ such
that 

\begin{mathpar}
  M^{*}_{x} | \lift{x}{P} \red M[P]
\end{mathpar}

namely,

\begin{mathpar}
  M^{*}_{x} := x?(u).M[\dropn{u}]
\end{mathpar}

The dependence of $M^{*}_{x}$ on a name makes it an abstraction, 

\begin{mathpar}
  M^{*} := (x)x?(u).M[\dropn{u}]
\end{mathpar}

\subsection{Additional notation}

It will sometimes be convenient to denote the process a name
quotes. We already have the notation $x = \quotep{P}$, but it will be
convenient to introduce an alternate notation, $\procn{x}$, when we
want to emphasize the connection to the use of the name. Note that, by
virtue of name equivalence, $\quotep{\procn{x}} \nameeq x$; so, the
notation is consistent with previous definitions.

Further, because names have structure it is possible to effect
substitutions on the basis of that structure. This means we need to
upgrade our notation for substitutions, which we accomplish by
adapting comprehension notation. Thus,

\begin{mathpar}
  P\{ y / x : x \in S \}
\end{mathpar}

is interpreted to mean the process derived from P by replacing (in a
capture-avoiding manner) each occurrence of $x$ in $S$ by $y$. For example,

\begin{mathpar}
  P\{ \quotep{\procn{x}|\procn{x}} / x : x \in \freenames{P} \}
\end{mathpar}

will replace each (occurrence) of a free name $x$ in $P$ by
$\quotep{\procn{x}|\procn{x}}$.

Also, we will avail ourselves of the notation $x^{L}$ and $x^{R}$ to
denote injections of a name into disjoint copies of the name
space. There are numerous ways to accomplish this. One example can be
found in \cite{MeredithR05}. This notation overloads to vectors of
names: $\vec{x}^{\pi} := (x_{i}^{\pi} \; : \; 0 \leq i < |\vec{x}| )$ where $\pi \in \{L,R\}$.

We also use $P^{\Box} := P|\Box$.

In \cite{MeredithR05} an interpretation of the new operator is
given. It turns out that there are several possible interpretations
all enjoying the requisite algebraic properties of the operator (see
\cite{milner91polyadicpi}). We will therefore make liberal use of
$(\nu\; \vec{x})P$.

% subsection the_syntax_and_semantics_of_the_notation_system (end)   

\input{qm2pi.qmops} 

\input{qm2pi.sterngerlach} 

\input{qm2pi.metric} 

% section concurrent_process_calculi (end)

%\input{qm2pi.proofsketch}

% section proof sketch (end)

%\input{qm2pi.slviaknots} 

% section spatial logic via knots (end)

\input{qm2pi.conclusion}

% section conclusion (end)

%\input{qm2pi.dtcodes} 

% section wiring algorithm (end)

\input{qm2pi.ack} 

% section acknowledgments (end)

\newpage


\bibliographystyle{plain}   
\bibliography{../../biblios/main.bib}

\input{qm2pi.rhodetails}

\end{document}

 

\documentclass[12pt]{llncs}
%\documentclass{jktr}

\usepackage[pdftex]{hyperref}                   
\usepackage {listings}
\usepackage {mathpartir}
\usepackage{bcprules}
%\usepackage{listings}
                       
\usepackage{graphicx} 
%\usepackage[margins=2.5cm,nohead,nofoot]{geometry}
%\usepackage{geometry}
\usepackage{amsfonts}
\usepackage{amstext}
\usepackage{latexsym}
\usepackage{amssymb}
\usepackage{color}


%\include{myPreamble}
\include{qm2pi.local} 

%\ifpdf
%\usepackage[pdftex]{graphicx}
%\else
%\usepackage{graphicx}
%\fi

 % \ifpdf
%  \usepackage{pdfsync}
%  \if


%\title{Brief Article}
%\author{David F. Snyder}
%\author{L.G. Meredith}

%\address{Dept. of Math., Texas State University--San Marcos, San Marcos, TX 78666}
       
\pagestyle{empty}


\begin{document}

\lstset{language=[Objective]Caml,frame=shadowbox}

\input{qm2pi.front}

% section front matter (end)

\input{qm2pi.intro} 
 
% section introduction (end)

% \input{qm2pi.knotations} 

% section notation (end)

\input{qm2pi.process.calculi} 

% section concurrent_process_calculi_and_spatial_logics_ (end)
    
%\input{qm2pi.knots2pi} 

%\input{qm2pi.trefoil} 

%\input{qm2pi.mainthm} 

% subsection basic_interpretation (end)

%\input{qm2pi.rho.presentation} 
\subsection{The syntax and semantics of the notation system}\label{sub:the_syntax_and_semantics_of_the_notation_system} % (fold)

We now summarize a technical presentation of the calculus that
embodies our theory of dynamics. The typical presentation of such a
calculus follows the style of giving generators and relations on
them. The grammar, below, describing term constructors, freely
generates the set of processes, $\Proc$. This set is then quotiented
by a relation known as structural congruence and it is over this set
that the notion of dynamics is expressed. This presentation is
essentially that of \cite{MeredithR05} with the addition of
polyadicity and summation. For readability we have relegated some of
the technical subtleties to an appendix.

\subsubsection{Process grammar}\label{subsub:process_grammar}

\begin{mathpar}
  \inferrule* [lab=synchronization] {} {{M} \bc \pzero \;|\; x?F \;|\; x!C }
  \and
  \inferrule* [lab=abstraction] {} {{F} \bc (x)P}
  \and
  \inferrule* [lab=concretion] {} {{C} \bc \langle Q \rangle}
  \and
  \inferrule* [lab=process] {} {{P,Q} \bc M \;| \;P|Q \;|\; @{x}}
  \and
  \inferrule* [lab=name] {} {{x} \bc \quotep{P}}
\end{mathpar} 

Note that $\vec{x}$ (resp. $\vec{P}$) denotes a vector of names
(resp. processes) of length $|\vec{x}|$ (resp. $|\vec{P}|$). We adopt
the following useful abbreviations.

\begin{mathpar}
   x?(\vec{y}).P := x.(\vec{y})P \and  x\clift{\vec{P}} := x.\clift{\vec{P}}
   \and x!(y) := \lift{x}{\dropn{y}}
   \and \Pi_{i=0}^{n-1}P_i := P_0 | \ldots | P_{n-1}
\end{mathpar}

\subsubsection{Structural congruence}

\paragraph{Free and bound names and alpha-equivalence.} At the
core of structural equivalence is alpha-equivalence which identifies
process that are the same up to a change of variable. Formally, we
recognize the distinction between free and bound names. The free names
of a process, $\freenames{P}$, may be calculated recursively as
follows:

\begin{mathpar}
\freenames{\pzero} := \emptyset
  \and \\
  \freenames{x?(y).P} := \{ x \} \cup (\freenames{P} \setminus \{ y \})
  \and 
  \freenames{x!\langle P \rangle} := \{ x \} \cup \{ P \} 
  \and \\
  \freenames{P|Q} := \freenames{P} \cup \freenames{Q}
  \and \\
  \freenames{@{x}} := \{ x \}
\end{mathpar}

$\pi$
$\quotep{\pi}$

$\freenames{-} : \pi \to \mathcal{P}(\quotep{\pi})$

\begin{eqnarray*}
  \freenames{\pzero} & := & \emptyset \\
  \freenames{x?(y).P} & := & \{ x \} \cup (\freenames{P} \setminus \{ y \}) \\
  \freenames{x!\langle P \rangle} & := & \{ x \} \cup \{ P \} \\
  \freenames{P|Q} & := & \freenames{P} \cup \freenames{Q} \\
  \freenames{\dropn{x}} & := & \{ x \}
\end{eqnarray*}

The bound names of a process, $\boundnames{P}$, are those names occurring in $P$
that are not free. For example, in $x?(y).0$, the name $x$ is free, while $y$ is bound.

\begin{mathpar}
  \inferrule* [lab=monoidal-laws] {} { P|Q \equiv Q|P \and P|0 \equiv P \and P|(Q|R) \equiv (P|Q)|R }
\end{mathpar}

\begin{mathpar}
  \inferrule* [lab=alpha-equivalence] {} { (x)P \equiv (y)P\{y/x\} \and y \not\in \freenames{P} }
\end{mathpar}

\begin{definition}
Then two processes, $P,Q$, are alpha-equivalent if $P = Q\{\vec{y}/\vec{x}\}$ for
some $\vec{x} \in \boundnames{Q},\vec{y} \in \boundnames{P}$, where $Q\{\vec{y}/\vec{x}\}$
denotes the capture-avoiding substitution of $\vec{y}$ for $\vec{x}$ in $Q$.
\end{definition}

\begin{definition}
  The {\em structural congruence} \cite{SangiorgiWalker} , $\equiv$,
  between processes is the least congruence containing
  alpha-equivalence, satisfying the abelian monoid laws
  (associativity, commutativity and $\pzero$ as identity) for parallel
  composition $|$ and for summation $+$.
\end{definition}

\subsection{Name equivalence}

We take name equivalence, written $\nameeq$, to be the smallest
equivalence relation generated by the following rules.

\begin{mathpar}
\inferrule*[lab=Quote-drop]
{ }
{ \quotep{@{x}} \nameeq x }

\inferrule*[lab=Struct-equiv]
{ P \scong Q }
{ \quotep{P} \nameeq \quotep{Q} }
\end{mathpar}

The astute reader will have noticed that the mutual recursion of names
and processes imposes a mutual recursion on alpha-equivalence and
structural equivalence via name-equivalence. Fortunately, all of this
works out pleasantly and we may calculate in the natural way, free of
concern. The reader interested in the details is referred to the
appendix \ref{appendix:rho_details}.

\subsection{Substitution}

We use $\Proc$ for the set of processes, $\QProc$ for the set of
names, and $\id{\{}\vec{y} / \vec{x} \id{\}}$ to denote partial maps,
$s : \QProc \rightarrow \QProc$. A map, $s$ lifts, uniquely, to a map
on process terms, $\widehat{s} : \Proc \rightarrow \Proc$ by the
following equations.

\begin{mathpar}
  (0) \psubstp{Q}{P} := 0 \\
  (R \juxtap S) \psubstp{Q}{P}
  :=    
  (R)\psubstp{Q}{P} \juxtap (S) \psubstp{Q}{P} \\
  (x?(y).R) \psubstp{Q}{P}    
  :=    
  (x)\substp{Q}{P} (z)\concat( (R \psubstn{z}{y}) \psubstp{Q}{P} ) \\
  (\lift{x}{R}) \psubstp{Q}{P}  
  :=
  \lift{(x)\substp{Q}{P}}{ R \psubstp{Q}{P} } \\
%   (\dropn{x})  \psubstp{Q}{P}       
%   := 
%   \left\{ 
%     \begin{array}{ccc} 
%       \dropn{\quotep{Q}} & & x \nameeq \quotep{P} \\
%       \dropn{x} & & otherwise \\
%     \end{array}
%   \right. 
  (\dropn{x})  \psubstp{Q}{P}       
  := 
  \left\{ 
    \begin{array}{ccc} 
      Q & & x \nameeq \quotep{P} \\
      \dropn{x} & & otherwise \\
    \end{array}
  \right.
\end{mathpar}
 

where

\begin{eqnarray}
  (x)\id{\{} \lpquote Q \rpquote / \lpquote P \rpquote \id{\}}            = 
  \left\{ 
    \begin{array}{ccc}
      \lpquote Q \rpquote & & x \nameeq \lpquote P \rpquote \\
      x & & otherwise \\
    \end{array}
  \right. \nonumber
\end{eqnarray}

and $z$ is chosen distinct from $\quotep{P}$, $\quotep{Q}$, the free
names in $Q$, and all the names in $R$. Our $\alpha$-equivalence will
be built in the standard way from this substitution.

\begin{remark}\label{rem:no_self_referential_names}
  One consequence of these definitions is that $\forall P. \quotep{P}
  \not\in \freenames{P}$.
\end{remark}

\subsection{ Dynamic quote: an example }

Anticipating something of what's to come, consider applying the
substitution, $\widehat{\id{\{}u / z \id{\}}}$, to the following pair
of processes, $\lift{w}{y!(z)}$ and $w[ \lpquote y!(z) \rpquote ]$.

\begin{eqnarray}
	\lift{w}{y!(z)}\widehat{\id{\{}u / z \id{\}}}
		& = &
		\lift{w}{y!(u)} \nonumber\\
	w[ \lpquote y!(z) \rpquote ] \widehat{ \id{\{}u / z \id{\}} }
		& = &
		w[ \lpquote y!(z) \rpquote ] \nonumber
\end{eqnarray}

Because the body of the process between quotes is impervious to
substitution, we get radically different answers. In fact, by
examining the first process in an input context,
e.g. $x?(z).\lift{w}{y!(z)}$, we see that the process under the lift
operator may be shaped by prefixed inputs binding a name inside it. In
this sense, the lift operator will be seen as a way to dynamically
construct processes before reifying them as names.

Finally equipped with these standard features we can present the
dynamics of the calculus.

\subsubsection{Operational semantics} 

Finally, we introduce the computational dynamics. What marks these
algebras as distinct from other more traditionally studied algebraic
structures, e.g. vector spaces or polynomial rings, is the manner in
which dynamics is captured. In traditional structures, dynamics is typically
expressed through morphisms between such structures, as in linear maps
between vector spaces or morphisms between rings. In algebras
associated with the semantics of computation, the dynamics is
expressed as part of the algebraic structure itself, through a
reduction reduction relation typically denoted by $\red$. Below, we
give a recursive presentation of this relation for the calculus used
in the encoding.

$\red \subseteq \pi \times \pi$
$\red : \pi \to \mathcal{P}(\pi)$

\begin{mathpar}
  \inferrule* [lab=Comm] { \textsf{match}( x_{src}, x_{trgt} ) } { x_{trgt}?(y)P \; | \; x_{src}!\langle {Q} \rangle \red P\{\quotep{Q}/y}\} }
  \and \\
  \inferrule* [lab=Par] {{P} \red {P}'} {{{P} | {Q}} \red {{P}' | {Q}}}
  \and
  \inferrule* [lab=Equiv]{{{P} \scong {P}'} \andalso {{P}' \red {Q}'} \andalso {{Q}' \scong {Q}}}{{P} \red {Q}}
\end{mathpar}

\begin{eqnarray*}
  match_{\equiv} (\quotep{P},\quotep{Q}) & := & P \equiv Q \\
  match_{\dagger}(\quotep{P},\quotep{Q}) & := & \forall R. P|Q \red^{*} R => R \red^{*} 0 \\
  match_{K}(\quotep{P},\quotep{Q}) & := & K \mbox{ for some context } K
\end{eqnarray*}

$u?(x)P | u!\langle Q \rangle \red P\{\quotep{Q}/x\}$

%We write $\wred$ for $\red^*$, and $P\red$ if $\exists Q $ such that $ P \red Q$.
We write $P\red$ if $\exists Q $ such that $ P \red Q$ and $P\not\red$, otherwise.

\section{Replication}

As mentioned before, it is known that replication (and hence
recursion) can be implemented in a higher-order process algebra
\cite{SangiorgiWalker}. As our first example of calculation with the
machinery thus far presented we give the construction explicitly in
the {\rhoc}.

\begin{eqnarray}
	D_{x} & := & \prefix{x}{y}{(\binpar{\outputp{x}{y}}{@{y}})} \nonumber\\
	\bangp_{x}{P} & := & \binpar{{x}!\langle{\binpar{D_{x}}{P}}\rangle}{D_{x}} \nonumber
\end{eqnarray}

\begin{eqnarray}
	\bangp_{x}{P} & & \nonumber\\
	=
	& {x}!\langle{(\prefix{x}{y}{(\outputp{x}{y} | @{y})) | P}}\rangle 
	      | \prefix{x}{y}{(\outputp{x}{y} | @{y})} & \nonumber\\
	\red
	& (\outputp{x}{y} | @{y})\substn{\quotep{(\prefix{x}{y}{(@{y} | \outputp{x}{y})) | P}}}{y} & \nonumber\\
	=
	& \outputp{x}{\quotep{(\prefix{x}{y}{(\outputp{x}{y} | @{y})) | P}}}
	  | {(\prefix{x}{y}{(\outputp{x}{y} | @{y})) | P}} & \nonumber\\
	\red
	& \ldots & \nonumber\\
	\red^*
	& P | P | \ldots & \nonumber
\end{eqnarray}

Of course, this encoding, as an implementation, runs away, unfolding
$\bangp{P}$ eagerly. A lazier and more implementable replication
operator, restricted to input-guarded processes, may be obtained as follows.

\begin{eqnarray}
\bangp{\prefix{u}{v}{P}} 
	:= 
	\binpar{\lift{x}{\prefix{u}{v}{(\binpar{D(x)}{P})}}}{D(x)} \nonumber
\end{eqnarray}

\begin{remark}
  Note that the lazier definition still does not deal with summation
  or mixed summation (i.e. sums over input and output). The reader is
  invited to construct definitions of replication that deal with these
  features. 

  Further, the definitions are parameterized in a name, $x$. Can you,
  gentle reader, make a definition that eliminates this parameter and
  guarantees no accidental interaction between the replication
  machinery and the process being replicated -- i.e. no accidental
  sharing of names used by the process to get its work done and the
  name(s) used by the replication to effect copying. This latter
  revision of the definition of replication is crucial to obtaining
  the expected identity $!!P \sim !P$.
\end{remark}

\begin{remark}\label{rem:paradoxical_combinator}
  The reader familiar with the lambda calculus will have noticed the
  similarity between $D$ and the paradoxical combinator.

  [Ed. note: the existence of this seems to suggest we have to be more
  restrictive on the set of processes and names we admit if we are to
  support no-cloning.]
\end{remark}

\subsubsection{Bisimulation}

The computational dynamics gives rise to another kind of equivalence,
the equivalence of computational behavior. As previously mentioned
this is typically captured \emph{via} some form of bisimulation.

% The notion we use in this paper is weak barbed bisimulation
% \cite{milner91polyadicpi}.

The notion we use in this paper is derived from weak barbed
bisimulation \cite{milner91polyadicpi}. 

\begin{definition}
An \emph{observation relation}, $\downarrow_{\mathcal N}$, over a set
of names, $\mathcal N$, is the smallest relation satisfying the rules
below.

\infrule[Out-barb]{y \in {\mathcal N}, \; x \nameeq y}
		  {\outputp{x}{v} \downarrow_{\mathcal N} x}
\infrule[Par-barb]{\mbox{$P\downarrow_{\mathcal N} x$ or $Q\downarrow_{\mathcal N} x$}}
		  {\binpar{P}{Q} \downarrow_{\mathcal N} x}

We write $P \Downarrow_{\mathcal N} x$ if there is $Q$ such that 
$P \wred Q$ and $Q \downarrow_{\mathcal N} x$.
\end{definition}

\begin{definition}
%\label{def.bbisim}
An  ${\mathcal N}$-\emph{barbed bisimulation} over a set of names, ${\mathcal N}$, is a symmetric binary relation 
${\mathcal S}_{\mathcal N}$ between agents such that $P\rel{S}_{\mathcal N}Q$ implies:
\begin{enumerate}
\item If $P \red P'$ then $Q \wred Q'$ and $P'\rel{S}_{\mathcal N} Q'$.
\item If $P\downarrow_{\mathcal N} x$, then $Q\Downarrow_{\mathcal N} x$.
\end{enumerate}
$P$ is ${\mathcal N}$-barbed bisimilar to $Q$, written
$P \wbbisim_{\mathcal N} Q$, if $P \rel{S}_{\mathcal N} Q$ for some ${\mathcal N}$-barbed bisimulation ${\mathcal S}_{\mathcal N}$.
\end{definition}

$\mathcal{R} \subseteq \pi \times \pi$

$P \mathcal{R} Q => \forall P'. P \red P' \Rightarrow \exists Q'. Q \red Q', P' \mathcal{R} Q'$

$P \vdash x \Rightarrow Q \vdash x$

\begin{mathpar}
  \inferrule*[lab=Out-barb]{x \nameeq y}{{y}!\langle{Q}\rangle \vdash x}
  \and
  \inferrule*[lab=Par-barb]{\mbox{$P\vdash x$ or $Q\vdash x$}}{\binpar{P}{Q} \vdash x}
\end{mathpar}

\subsubsection{Contexts}

One of the principle advantages of computational calculi like the
$\pi$-calculus is a well-defined notion of context,
contextual-equivalence and a correlation between
contextual-equivalence and notions of bisimulation. The notion of
context allows the decomposition of a process into (sub-)process and
its syntactic environment, its context. Thus, a context may be
thought of as a process with a ``hole'' (written $\Box$) in it. The
application of a context $M$ to a process $P$, written $M[P]$, is
tantamount to filling the hole in $M$ with $P$. In this paper we do
not need the full weight of this theory, but do make use of the notion
of context in the proof the main theorem. 

\begin{mathpar}
  \inferrule* [lab=summation] {} {{M_{M},M_{N}} \bc \Box \;|\; x.M_{A} \;|\; M_{M}+M_{N}}
  \and
  \inferrule* [lab=agent] {} {{M_{A}} \bc (\vec{x})M_{P} \;| \; \clift{P_0,\ldots,M_{P},\ldots,P_N}}
  \and \\
  \inferrule* [lab=process] {} {{M_{P}} \bc M_{N} \;| \;P|M_{P} }
\end{mathpar} 

\begin{mathpar}
  \inferrule* [lab=sychronization] {} {M_{N} \bc \Box \;|\; x?M_{F} \;|\; x!M_{C}}
  \and
  \inferrule* [lab=abstraction] {} {{M_{F}} \bc (x)M_{P} }
  \and
  \inferrule* [lab=concretion] {} {{M_{C}} \bc \langle M_{P} \rangle }
  \and \\
  \inferrule* [lab=process] {} {{M_{P}} \bc M_{N} \;| \;P|M_{P} }
\end{mathpar}

\begin{definition}[contextual application] Given a context $M$, and
  process $P$, we define the \emph{contextual application}, $M[P] :=
  M\{P/\Box\}$. That is, the contextual application of M to P is the
  substitution of $P$ for $\Box$ in $M$.
\end{definition}

$\meaningof{-} : L \to \mathcal{P}(\pi)$

\begin{mathpar}
  \inferrule* [lab=collection] {} {\meaningof{true} = \pi, \and \meaningof{~E} = \pi \setminus \meaningof{E}, \and \meaningof{E_{1} \& E_{2}} = \meaningof{E_{1}} \cap \meaningof{E_{2}}}
\end{mathpar}

\begin{mathpar}
  \inferrule* [lab=structure] {} {\meaningof{0} = \{ P \in \pi | P \equiv 0 \}, \and \\ \meaningof{E_1 | E_2} = \{ P \in \pi | P \equiv P_{1} | P_{2}, P_{1} \in \meaningof{E_{1}}, P_{2} \in \meaningof{E_2}\} }
\end{mathpar}

\begin{mathpar}
 \inferrule* [lab=behavior] {} {\meaningof{\langle a?b \rangle E} = \{ P \in \pi | P \equiv Q | u?(y)P', \\ \and \\\\ \and \\ \;\;\; u \in \meaningof{a}, \forall z.P'\{z/y\} \in \meaningof{E\{z/b\}}\}, \and \\ \meaningof{a!E} = \{ P \in \pi | P \equiv Q | x!\langle P' \rangle, x \in \meaningof{a} P' \in \meaningof{E}\} }
\end{mathpar}

\begin{mathpar}
 \inferrule* [lab=nominal] {} {\meaningof{\quotep{E}} = \{ \quotep{P} \in \quotep{\pi} | P \in \meaningof{E} \}, \and \meaningof{\quotep{P}} = \{ \quotep{Q} \in \quotep{\pi} | P \equiv Q \} \and \\ \meaningof{@\quotep{E}} = \{ P \in \pi | P \equiv @x, x \in \meaningof{E} \}}
\end{mathpar}

\begin{eqnarray*}
  \\
  \meaningof{-} : TS \to ST
\end{eqnarray*}

\begin{eqnarray*}
  \\
  L : TS \to ST
\end{eqnarray*}

\begin{eqnarray*}
  \\
  P \models E \iff P \in \meaningof{E}
\end{eqnarray*}

\begin{eqnarray*}
  P \approx_{L} Q \iff \forall E \in L. P \models E \iff Q \models E
\end{eqnarray*}

\begin{eqnarray*}
  P \approx_{K} Q
\end{eqnarray*}

\begin{eqnarray*}
  P \approx Q
\end{eqnarray*}

$\approx_{K} = \approx = \approx_{L}$

\subsubsection{Contextual duality}

Note that contexts extend the quotation operation to a family of
operations from processes to names. Given a context, $M$, we can
define a \emph{nominal context}, $\quotep{M}$ by $\quotep{M}[P] :=
\quotep{M[P]}$. To foreshadow what is to come we observe that these
operations enjoy a duality with processes very much like the duality
between vectors and maps from vectors to scalars.

Further, because the calculus is essentially higher-order, we have a
correspondence between contexts and processes. More specifically,
given a name $x$ and a context $M$ we can construct $M^{*}_{x}$ such
that 

\begin{mathpar}
  M^{*}_{x} | \lift{x}{P} \red M[P]
\end{mathpar}

namely,

\begin{mathpar}
  M^{*}_{x} := x?(u).M[\dropn{u}]
\end{mathpar}

The dependence of $M^{*}_{x}$ on a name makes it an abstraction, 

\begin{mathpar}
  M^{*} := (x)x?(u).M[\dropn{u}]
\end{mathpar}

\subsection{Additional notation}

It will sometimes be convenient to denote the process a name
quotes. We already have the notation $x = \quotep{P}$, but it will be
convenient to introduce an alternate notation, $\procn{x}$, when we
want to emphasize the connection to the use of the name. Note that, by
virtue of name equivalence, $\quotep{\procn{x}} \nameeq x$; so, the
notation is consistent with previous definitions.

Further, because names have structure it is possible to effect
substitutions on the basis of that structure. This means we need to
upgrade our notation for substitutions, which we accomplish by
adapting comprehension notation. Thus,

\begin{mathpar}
  P\{ y / x : x \in S \}
\end{mathpar}

is interpreted to mean the process derived from P by replacing (in a
capture-avoiding manner) each occurrence of $x$ in $S$ by $y$. For example,

\begin{mathpar}
  P\{ \quotep{\procn{x}|\procn{x}} / x : x \in \freenames{P} \}
\end{mathpar}

will replace each (occurrence) of a free name $x$ in $P$ by
$\quotep{\procn{x}|\procn{x}}$.

Also, we will avail ourselves of the notation $x^{L}$ and $x^{R}$ to
denote injections of a name into disjoint copies of the name
space. There are numerous ways to accomplish this. One example can be
found in \cite{MeredithR05}. This notation overloads to vectors of
names: $\vec{x}^{\pi} := (x_{i}^{\pi} \; : \; 0 \leq i < |\vec{x}| )$ where $\pi \in \{L,R\}$.

We also use $P^{\Box} := P|\Box$.

In \cite{MeredithR05} an interpretation of the new operator is
given. It turns out that there are several possible interpretations
all enjoying the requisite algebraic properties of the operator (see
\cite{milner91polyadicpi}). We will therefore make liberal use of
$(\nu\; \vec{x})P$.

% subsection the_syntax_and_semantics_of_the_notation_system (end)   

\input{qm2pi.qmops} 

\input{qm2pi.sterngerlach} 

\input{qm2pi.metric} 

% section concurrent_process_calculi (end)

%\input{qm2pi.proofsketch}

% section proof sketch (end)

%\input{qm2pi.slviaknots} 

% section spatial logic via knots (end)

\input{qm2pi.conclusion}

% section conclusion (end)

%\input{qm2pi.dtcodes} 

% section wiring algorithm (end)

\input{qm2pi.ack} 

% section acknowledgments (end)

\newpage


\bibliographystyle{plain}   
\bibliography{../../biblios/main.bib}

\input{qm2pi.rhodetails}

\end{document}

 

% section concurrent_process_calculi (end)

%\documentclass[12pt]{llncs}
%\documentclass{jktr}

\usepackage[pdftex]{hyperref}                   
\usepackage {listings}
\usepackage {mathpartir}
\usepackage{bcprules}
%\usepackage{listings}
                       
\usepackage{graphicx} 
%\usepackage[margins=2.5cm,nohead,nofoot]{geometry}
%\usepackage{geometry}
\usepackage{amsfonts}
\usepackage{amstext}
\usepackage{latexsym}
\usepackage{amssymb}
\usepackage{color}


%\include{myPreamble}
\include{qm2pi.local} 

%\ifpdf
%\usepackage[pdftex]{graphicx}
%\else
%\usepackage{graphicx}
%\fi

 % \ifpdf
%  \usepackage{pdfsync}
%  \if


%\title{Brief Article}
%\author{David F. Snyder}
%\author{L.G. Meredith}

%\address{Dept. of Math., Texas State University--San Marcos, San Marcos, TX 78666}
       
\pagestyle{empty}


\begin{document}

\lstset{language=[Objective]Caml,frame=shadowbox}

\input{qm2pi.front}

% section front matter (end)

\input{qm2pi.intro} 
 
% section introduction (end)

% \input{qm2pi.knotations} 

% section notation (end)

\input{qm2pi.process.calculi} 

% section concurrent_process_calculi_and_spatial_logics_ (end)
    
%\input{qm2pi.knots2pi} 

%\input{qm2pi.trefoil} 

%\input{qm2pi.mainthm} 

% subsection basic_interpretation (end)

%\input{qm2pi.rho.presentation} 
\subsection{The syntax and semantics of the notation system}\label{sub:the_syntax_and_semantics_of_the_notation_system} % (fold)

We now summarize a technical presentation of the calculus that
embodies our theory of dynamics. The typical presentation of such a
calculus follows the style of giving generators and relations on
them. The grammar, below, describing term constructors, freely
generates the set of processes, $\Proc$. This set is then quotiented
by a relation known as structural congruence and it is over this set
that the notion of dynamics is expressed. This presentation is
essentially that of \cite{MeredithR05} with the addition of
polyadicity and summation. For readability we have relegated some of
the technical subtleties to an appendix.

\subsubsection{Process grammar}\label{subsub:process_grammar}

\begin{mathpar}
  \inferrule* [lab=synchronization] {} {{M} \bc \pzero \;|\; x?F \;|\; x!C }
  \and
  \inferrule* [lab=abstraction] {} {{F} \bc (x)P}
  \and
  \inferrule* [lab=concretion] {} {{C} \bc \langle Q \rangle}
  \and
  \inferrule* [lab=process] {} {{P,Q} \bc M \;| \;P|Q \;|\; @{x}}
  \and
  \inferrule* [lab=name] {} {{x} \bc \quotep{P}}
\end{mathpar} 

Note that $\vec{x}$ (resp. $\vec{P}$) denotes a vector of names
(resp. processes) of length $|\vec{x}|$ (resp. $|\vec{P}|$). We adopt
the following useful abbreviations.

\begin{mathpar}
   x?(\vec{y}).P := x.(\vec{y})P \and  x\clift{\vec{P}} := x.\clift{\vec{P}}
   \and x!(y) := \lift{x}{\dropn{y}}
   \and \Pi_{i=0}^{n-1}P_i := P_0 | \ldots | P_{n-1}
\end{mathpar}

\subsubsection{Structural congruence}

\paragraph{Free and bound names and alpha-equivalence.} At the
core of structural equivalence is alpha-equivalence which identifies
process that are the same up to a change of variable. Formally, we
recognize the distinction between free and bound names. The free names
of a process, $\freenames{P}$, may be calculated recursively as
follows:

\begin{mathpar}
\freenames{\pzero} := \emptyset
  \and \\
  \freenames{x?(y).P} := \{ x \} \cup (\freenames{P} \setminus \{ y \})
  \and 
  \freenames{x!\langle P \rangle} := \{ x \} \cup \{ P \} 
  \and \\
  \freenames{P|Q} := \freenames{P} \cup \freenames{Q}
  \and \\
  \freenames{@{x}} := \{ x \}
\end{mathpar}

$\pi$
$\quotep{\pi}$

$\freenames{-} : \pi \to \mathcal{P}(\quotep{\pi})$

\begin{eqnarray*}
  \freenames{\pzero} & := & \emptyset \\
  \freenames{x?(y).P} & := & \{ x \} \cup (\freenames{P} \setminus \{ y \}) \\
  \freenames{x!\langle P \rangle} & := & \{ x \} \cup \{ P \} \\
  \freenames{P|Q} & := & \freenames{P} \cup \freenames{Q} \\
  \freenames{\dropn{x}} & := & \{ x \}
\end{eqnarray*}

The bound names of a process, $\boundnames{P}$, are those names occurring in $P$
that are not free. For example, in $x?(y).0$, the name $x$ is free, while $y$ is bound.

\begin{mathpar}
  \inferrule* [lab=monoidal-laws] {} { P|Q \equiv Q|P \and P|0 \equiv P \and P|(Q|R) \equiv (P|Q)|R }
\end{mathpar}

\begin{mathpar}
  \inferrule* [lab=alpha-equivalence] {} { (x)P \equiv (y)P\{y/x\} \and y \not\in \freenames{P} }
\end{mathpar}

\begin{definition}
Then two processes, $P,Q$, are alpha-equivalent if $P = Q\{\vec{y}/\vec{x}\}$ for
some $\vec{x} \in \boundnames{Q},\vec{y} \in \boundnames{P}$, where $Q\{\vec{y}/\vec{x}\}$
denotes the capture-avoiding substitution of $\vec{y}$ for $\vec{x}$ in $Q$.
\end{definition}

\begin{definition}
  The {\em structural congruence} \cite{SangiorgiWalker} , $\equiv$,
  between processes is the least congruence containing
  alpha-equivalence, satisfying the abelian monoid laws
  (associativity, commutativity and $\pzero$ as identity) for parallel
  composition $|$ and for summation $+$.
\end{definition}

\subsection{Name equivalence}

We take name equivalence, written $\nameeq$, to be the smallest
equivalence relation generated by the following rules.

\begin{mathpar}
\inferrule*[lab=Quote-drop]
{ }
{ \quotep{@{x}} \nameeq x }

\inferrule*[lab=Struct-equiv]
{ P \scong Q }
{ \quotep{P} \nameeq \quotep{Q} }
\end{mathpar}

The astute reader will have noticed that the mutual recursion of names
and processes imposes a mutual recursion on alpha-equivalence and
structural equivalence via name-equivalence. Fortunately, all of this
works out pleasantly and we may calculate in the natural way, free of
concern. The reader interested in the details is referred to the
appendix \ref{appendix:rho_details}.

\subsection{Substitution}

We use $\Proc$ for the set of processes, $\QProc$ for the set of
names, and $\id{\{}\vec{y} / \vec{x} \id{\}}$ to denote partial maps,
$s : \QProc \rightarrow \QProc$. A map, $s$ lifts, uniquely, to a map
on process terms, $\widehat{s} : \Proc \rightarrow \Proc$ by the
following equations.

\begin{mathpar}
  (0) \psubstp{Q}{P} := 0 \\
  (R \juxtap S) \psubstp{Q}{P}
  :=    
  (R)\psubstp{Q}{P} \juxtap (S) \psubstp{Q}{P} \\
  (x?(y).R) \psubstp{Q}{P}    
  :=    
  (x)\substp{Q}{P} (z)\concat( (R \psubstn{z}{y}) \psubstp{Q}{P} ) \\
  (\lift{x}{R}) \psubstp{Q}{P}  
  :=
  \lift{(x)\substp{Q}{P}}{ R \psubstp{Q}{P} } \\
%   (\dropn{x})  \psubstp{Q}{P}       
%   := 
%   \left\{ 
%     \begin{array}{ccc} 
%       \dropn{\quotep{Q}} & & x \nameeq \quotep{P} \\
%       \dropn{x} & & otherwise \\
%     \end{array}
%   \right. 
  (\dropn{x})  \psubstp{Q}{P}       
  := 
  \left\{ 
    \begin{array}{ccc} 
      Q & & x \nameeq \quotep{P} \\
      \dropn{x} & & otherwise \\
    \end{array}
  \right.
\end{mathpar}
 

where

\begin{eqnarray}
  (x)\id{\{} \lpquote Q \rpquote / \lpquote P \rpquote \id{\}}            = 
  \left\{ 
    \begin{array}{ccc}
      \lpquote Q \rpquote & & x \nameeq \lpquote P \rpquote \\
      x & & otherwise \\
    \end{array}
  \right. \nonumber
\end{eqnarray}

and $z$ is chosen distinct from $\quotep{P}$, $\quotep{Q}$, the free
names in $Q$, and all the names in $R$. Our $\alpha$-equivalence will
be built in the standard way from this substitution.

\begin{remark}\label{rem:no_self_referential_names}
  One consequence of these definitions is that $\forall P. \quotep{P}
  \not\in \freenames{P}$.
\end{remark}

\subsection{ Dynamic quote: an example }

Anticipating something of what's to come, consider applying the
substitution, $\widehat{\id{\{}u / z \id{\}}}$, to the following pair
of processes, $\lift{w}{y!(z)}$ and $w[ \lpquote y!(z) \rpquote ]$.

\begin{eqnarray}
	\lift{w}{y!(z)}\widehat{\id{\{}u / z \id{\}}}
		& = &
		\lift{w}{y!(u)} \nonumber\\
	w[ \lpquote y!(z) \rpquote ] \widehat{ \id{\{}u / z \id{\}} }
		& = &
		w[ \lpquote y!(z) \rpquote ] \nonumber
\end{eqnarray}

Because the body of the process between quotes is impervious to
substitution, we get radically different answers. In fact, by
examining the first process in an input context,
e.g. $x?(z).\lift{w}{y!(z)}$, we see that the process under the lift
operator may be shaped by prefixed inputs binding a name inside it. In
this sense, the lift operator will be seen as a way to dynamically
construct processes before reifying them as names.

Finally equipped with these standard features we can present the
dynamics of the calculus.

\subsubsection{Operational semantics} 

Finally, we introduce the computational dynamics. What marks these
algebras as distinct from other more traditionally studied algebraic
structures, e.g. vector spaces or polynomial rings, is the manner in
which dynamics is captured. In traditional structures, dynamics is typically
expressed through morphisms between such structures, as in linear maps
between vector spaces or morphisms between rings. In algebras
associated with the semantics of computation, the dynamics is
expressed as part of the algebraic structure itself, through a
reduction reduction relation typically denoted by $\red$. Below, we
give a recursive presentation of this relation for the calculus used
in the encoding.

$\red \subseteq \pi \times \pi$
$\red : \pi \to \mathcal{P}(\pi)$

\begin{mathpar}
  \inferrule* [lab=Comm] { \textsf{match}( x_{src}, x_{trgt} ) } { x_{trgt}?(y)P \; | \; x_{src}!\langle {Q} \rangle \red P\{\quotep{Q}/y}\} }
  \and \\
  \inferrule* [lab=Par] {{P} \red {P}'} {{{P} | {Q}} \red {{P}' | {Q}}}
  \and
  \inferrule* [lab=Equiv]{{{P} \scong {P}'} \andalso {{P}' \red {Q}'} \andalso {{Q}' \scong {Q}}}{{P} \red {Q}}
\end{mathpar}

\begin{eqnarray*}
  match_{\equiv} (\quotep{P},\quotep{Q}) & := & P \equiv Q \\
  match_{\dagger}(\quotep{P},\quotep{Q}) & := & \forall R. P|Q \red^{*} R => R \red^{*} 0 \\
  match_{K}(\quotep{P},\quotep{Q}) & := & K \mbox{ for some context } K
\end{eqnarray*}

$u?(x)P | u!\langle Q \rangle \red P\{\quotep{Q}/x\}$

%We write $\wred$ for $\red^*$, and $P\red$ if $\exists Q $ such that $ P \red Q$.
We write $P\red$ if $\exists Q $ such that $ P \red Q$ and $P\not\red$, otherwise.

\section{Replication}

As mentioned before, it is known that replication (and hence
recursion) can be implemented in a higher-order process algebra
\cite{SangiorgiWalker}. As our first example of calculation with the
machinery thus far presented we give the construction explicitly in
the {\rhoc}.

\begin{eqnarray}
	D_{x} & := & \prefix{x}{y}{(\binpar{\outputp{x}{y}}{@{y}})} \nonumber\\
	\bangp_{x}{P} & := & \binpar{{x}!\langle{\binpar{D_{x}}{P}}\rangle}{D_{x}} \nonumber
\end{eqnarray}

\begin{eqnarray}
	\bangp_{x}{P} & & \nonumber\\
	=
	& {x}!\langle{(\prefix{x}{y}{(\outputp{x}{y} | @{y})) | P}}\rangle 
	      | \prefix{x}{y}{(\outputp{x}{y} | @{y})} & \nonumber\\
	\red
	& (\outputp{x}{y} | @{y})\substn{\quotep{(\prefix{x}{y}{(@{y} | \outputp{x}{y})) | P}}}{y} & \nonumber\\
	=
	& \outputp{x}{\quotep{(\prefix{x}{y}{(\outputp{x}{y} | @{y})) | P}}}
	  | {(\prefix{x}{y}{(\outputp{x}{y} | @{y})) | P}} & \nonumber\\
	\red
	& \ldots & \nonumber\\
	\red^*
	& P | P | \ldots & \nonumber
\end{eqnarray}

Of course, this encoding, as an implementation, runs away, unfolding
$\bangp{P}$ eagerly. A lazier and more implementable replication
operator, restricted to input-guarded processes, may be obtained as follows.

\begin{eqnarray}
\bangp{\prefix{u}{v}{P}} 
	:= 
	\binpar{\lift{x}{\prefix{u}{v}{(\binpar{D(x)}{P})}}}{D(x)} \nonumber
\end{eqnarray}

\begin{remark}
  Note that the lazier definition still does not deal with summation
  or mixed summation (i.e. sums over input and output). The reader is
  invited to construct definitions of replication that deal with these
  features. 

  Further, the definitions are parameterized in a name, $x$. Can you,
  gentle reader, make a definition that eliminates this parameter and
  guarantees no accidental interaction between the replication
  machinery and the process being replicated -- i.e. no accidental
  sharing of names used by the process to get its work done and the
  name(s) used by the replication to effect copying. This latter
  revision of the definition of replication is crucial to obtaining
  the expected identity $!!P \sim !P$.
\end{remark}

\begin{remark}\label{rem:paradoxical_combinator}
  The reader familiar with the lambda calculus will have noticed the
  similarity between $D$ and the paradoxical combinator.

  [Ed. note: the existence of this seems to suggest we have to be more
  restrictive on the set of processes and names we admit if we are to
  support no-cloning.]
\end{remark}

\subsubsection{Bisimulation}

The computational dynamics gives rise to another kind of equivalence,
the equivalence of computational behavior. As previously mentioned
this is typically captured \emph{via} some form of bisimulation.

% The notion we use in this paper is weak barbed bisimulation
% \cite{milner91polyadicpi}.

The notion we use in this paper is derived from weak barbed
bisimulation \cite{milner91polyadicpi}. 

\begin{definition}
An \emph{observation relation}, $\downarrow_{\mathcal N}$, over a set
of names, $\mathcal N$, is the smallest relation satisfying the rules
below.

\infrule[Out-barb]{y \in {\mathcal N}, \; x \nameeq y}
		  {\outputp{x}{v} \downarrow_{\mathcal N} x}
\infrule[Par-barb]{\mbox{$P\downarrow_{\mathcal N} x$ or $Q\downarrow_{\mathcal N} x$}}
		  {\binpar{P}{Q} \downarrow_{\mathcal N} x}

We write $P \Downarrow_{\mathcal N} x$ if there is $Q$ such that 
$P \wred Q$ and $Q \downarrow_{\mathcal N} x$.
\end{definition}

\begin{definition}
%\label{def.bbisim}
An  ${\mathcal N}$-\emph{barbed bisimulation} over a set of names, ${\mathcal N}$, is a symmetric binary relation 
${\mathcal S}_{\mathcal N}$ between agents such that $P\rel{S}_{\mathcal N}Q$ implies:
\begin{enumerate}
\item If $P \red P'$ then $Q \wred Q'$ and $P'\rel{S}_{\mathcal N} Q'$.
\item If $P\downarrow_{\mathcal N} x$, then $Q\Downarrow_{\mathcal N} x$.
\end{enumerate}
$P$ is ${\mathcal N}$-barbed bisimilar to $Q$, written
$P \wbbisim_{\mathcal N} Q$, if $P \rel{S}_{\mathcal N} Q$ for some ${\mathcal N}$-barbed bisimulation ${\mathcal S}_{\mathcal N}$.
\end{definition}

$\mathcal{R} \subseteq \pi \times \pi$

$P \mathcal{R} Q => \forall P'. P \red P' \Rightarrow \exists Q'. Q \red Q', P' \mathcal{R} Q'$

$P \vdash x \Rightarrow Q \vdash x$

\begin{mathpar}
  \inferrule*[lab=Out-barb]{x \nameeq y}{{y}!\langle{Q}\rangle \vdash x}
  \and
  \inferrule*[lab=Par-barb]{\mbox{$P\vdash x$ or $Q\vdash x$}}{\binpar{P}{Q} \vdash x}
\end{mathpar}

\subsubsection{Contexts}

One of the principle advantages of computational calculi like the
$\pi$-calculus is a well-defined notion of context,
contextual-equivalence and a correlation between
contextual-equivalence and notions of bisimulation. The notion of
context allows the decomposition of a process into (sub-)process and
its syntactic environment, its context. Thus, a context may be
thought of as a process with a ``hole'' (written $\Box$) in it. The
application of a context $M$ to a process $P$, written $M[P]$, is
tantamount to filling the hole in $M$ with $P$. In this paper we do
not need the full weight of this theory, but do make use of the notion
of context in the proof the main theorem. 

\begin{mathpar}
  \inferrule* [lab=summation] {} {{M_{M},M_{N}} \bc \Box \;|\; x.M_{A} \;|\; M_{M}+M_{N}}
  \and
  \inferrule* [lab=agent] {} {{M_{A}} \bc (\vec{x})M_{P} \;| \; \clift{P_0,\ldots,M_{P},\ldots,P_N}}
  \and \\
  \inferrule* [lab=process] {} {{M_{P}} \bc M_{N} \;| \;P|M_{P} }
\end{mathpar} 

\begin{mathpar}
  \inferrule* [lab=sychronization] {} {M_{N} \bc \Box \;|\; x?M_{F} \;|\; x!M_{C}}
  \and
  \inferrule* [lab=abstraction] {} {{M_{F}} \bc (x)M_{P} }
  \and
  \inferrule* [lab=concretion] {} {{M_{C}} \bc \langle M_{P} \rangle }
  \and \\
  \inferrule* [lab=process] {} {{M_{P}} \bc M_{N} \;| \;P|M_{P} }
\end{mathpar}

\begin{definition}[contextual application] Given a context $M$, and
  process $P$, we define the \emph{contextual application}, $M[P] :=
  M\{P/\Box\}$. That is, the contextual application of M to P is the
  substitution of $P$ for $\Box$ in $M$.
\end{definition}

$\meaningof{-} : L \to \mathcal{P}(\pi)$

\begin{mathpar}
  \inferrule* [lab=collection] {} {\meaningof{true} = \pi, \and \meaningof{~E} = \pi \setminus \meaningof{E}, \and \meaningof{E_{1} \& E_{2}} = \meaningof{E_{1}} \cap \meaningof{E_{2}}}
\end{mathpar}

\begin{mathpar}
  \inferrule* [lab=structure] {} {\meaningof{0} = \{ P \in \pi | P \equiv 0 \}, \and \\ \meaningof{E_1 | E_2} = \{ P \in \pi | P \equiv P_{1} | P_{2}, P_{1} \in \meaningof{E_{1}}, P_{2} \in \meaningof{E_2}\} }
\end{mathpar}

\begin{mathpar}
 \inferrule* [lab=behavior] {} {\meaningof{\langle a?b \rangle E} = \{ P \in \pi | P \equiv Q | u?(y)P', \\ \and \\\\ \and \\ \;\;\; u \in \meaningof{a}, \forall z.P'\{z/y\} \in \meaningof{E\{z/b\}}\}, \and \\ \meaningof{a!E} = \{ P \in \pi | P \equiv Q | x!\langle P' \rangle, x \in \meaningof{a} P' \in \meaningof{E}\} }
\end{mathpar}

\begin{mathpar}
 \inferrule* [lab=nominal] {} {\meaningof{\quotep{E}} = \{ \quotep{P} \in \quotep{\pi} | P \in \meaningof{E} \}, \and \meaningof{\quotep{P}} = \{ \quotep{Q} \in \quotep{\pi} | P \equiv Q \} \and \\ \meaningof{@\quotep{E}} = \{ P \in \pi | P \equiv @x, x \in \meaningof{E} \}}
\end{mathpar}

\begin{eqnarray*}
  \\
  \meaningof{-} : TS \to ST
\end{eqnarray*}

\begin{eqnarray*}
  \\
  L : TS \to ST
\end{eqnarray*}

\begin{eqnarray*}
  \\
  P \models E \iff P \in \meaningof{E}
\end{eqnarray*}

\begin{eqnarray*}
  P \approx_{L} Q \iff \forall E \in L. P \models E \iff Q \models E
\end{eqnarray*}

\begin{eqnarray*}
  P \approx_{K} Q
\end{eqnarray*}

\begin{eqnarray*}
  P \approx Q
\end{eqnarray*}

$\approx_{K} = \approx = \approx_{L}$

\subsubsection{Contextual duality}

Note that contexts extend the quotation operation to a family of
operations from processes to names. Given a context, $M$, we can
define a \emph{nominal context}, $\quotep{M}$ by $\quotep{M}[P] :=
\quotep{M[P]}$. To foreshadow what is to come we observe that these
operations enjoy a duality with processes very much like the duality
between vectors and maps from vectors to scalars.

Further, because the calculus is essentially higher-order, we have a
correspondence between contexts and processes. More specifically,
given a name $x$ and a context $M$ we can construct $M^{*}_{x}$ such
that 

\begin{mathpar}
  M^{*}_{x} | \lift{x}{P} \red M[P]
\end{mathpar}

namely,

\begin{mathpar}
  M^{*}_{x} := x?(u).M[\dropn{u}]
\end{mathpar}

The dependence of $M^{*}_{x}$ on a name makes it an abstraction, 

\begin{mathpar}
  M^{*} := (x)x?(u).M[\dropn{u}]
\end{mathpar}

\subsection{Additional notation}

It will sometimes be convenient to denote the process a name
quotes. We already have the notation $x = \quotep{P}$, but it will be
convenient to introduce an alternate notation, $\procn{x}$, when we
want to emphasize the connection to the use of the name. Note that, by
virtue of name equivalence, $\quotep{\procn{x}} \nameeq x$; so, the
notation is consistent with previous definitions.

Further, because names have structure it is possible to effect
substitutions on the basis of that structure. This means we need to
upgrade our notation for substitutions, which we accomplish by
adapting comprehension notation. Thus,

\begin{mathpar}
  P\{ y / x : x \in S \}
\end{mathpar}

is interpreted to mean the process derived from P by replacing (in a
capture-avoiding manner) each occurrence of $x$ in $S$ by $y$. For example,

\begin{mathpar}
  P\{ \quotep{\procn{x}|\procn{x}} / x : x \in \freenames{P} \}
\end{mathpar}

will replace each (occurrence) of a free name $x$ in $P$ by
$\quotep{\procn{x}|\procn{x}}$.

Also, we will avail ourselves of the notation $x^{L}$ and $x^{R}$ to
denote injections of a name into disjoint copies of the name
space. There are numerous ways to accomplish this. One example can be
found in \cite{MeredithR05}. This notation overloads to vectors of
names: $\vec{x}^{\pi} := (x_{i}^{\pi} \; : \; 0 \leq i < |\vec{x}| )$ where $\pi \in \{L,R\}$.

We also use $P^{\Box} := P|\Box$.

In \cite{MeredithR05} an interpretation of the new operator is
given. It turns out that there are several possible interpretations
all enjoying the requisite algebraic properties of the operator (see
\cite{milner91polyadicpi}). We will therefore make liberal use of
$(\nu\; \vec{x})P$.

% subsection the_syntax_and_semantics_of_the_notation_system (end)   

\input{qm2pi.qmops} 

\input{qm2pi.sterngerlach} 

\input{qm2pi.metric} 

% section concurrent_process_calculi (end)

%\input{qm2pi.proofsketch}

% section proof sketch (end)

%\input{qm2pi.slviaknots} 

% section spatial logic via knots (end)

\input{qm2pi.conclusion}

% section conclusion (end)

%\input{qm2pi.dtcodes} 

% section wiring algorithm (end)

\input{qm2pi.ack} 

% section acknowledgments (end)

\newpage


\bibliographystyle{plain}   
\bibliography{../../biblios/main.bib}

\input{qm2pi.rhodetails}

\end{document}



% section proof sketch (end)

%\section{Unlikely characters: spatial logic for
  knots}\label{sub:characteristic_formulae} % (fold)

Associated to the mobile process calculi are a family of logics known
as the Hennessy-Milner logics. These logics typically enjoy a
semantics interpreting formulae as sets of processes that when
factored through the encoding outlined above allows an identification
of classes of knots with logical formulae. In the context of this
encoding the sub-family known as the spatial logics \cite{CairesC03}
\cite{CairesC04} \cite{Caires04} are of particular interest providing
several important features for expressing and reasoning about
properties (i.e. classes) of knots. We hint here at how this may be done.

%\begin{description}
%\item [structural connectives] 
\subsubsection{Structural connectives} The spatial logics enjoy
structural connectives corresponding, at the logical level, to the
parallel composition ($P | Q$) and new name ($(\nu \; x)P$)
connectives for processes. As illustrated in the examples below, these
connectives are extremely expressive given the shape of our encoding.
%\item [decideable satisfaction]

\subsubsection{Decideable satisfaction}
In \cite{Caires04} the satisfaction relation is shown to be decideable
for a rich class of processes. It further turns out that the image of
the our encoding is a proper subset of that class. This result
provides the basis for an algorithm by which to search for knots
enjoying a given property.
%\item [characteristic formulae]

\subsubsection{Characteristic formulae}
In the same paper \cite{Caires04} , Caires presents a means of calculating
characteristic formulae, selecting equivalence classes of processes
up to a pre--specified depth limit on the support set of names. Composed with our
encoding, this characteristic formula can be used to select
characteristic formulae for knots.
%\end{description}

\subsubsection{Spatial logic formulae}

The grammar below (segmented for comprehension) summarizes the syntax
of spatial logic formulae. We employ illustrative examples in the
sequel to provide an intuitive understanding of their meaning
referring the reader to \cite{Caires04} for a more detailed explication
of the semantics.

\begin{mathpar}
  \inferrule* [lab=boolean] {} {{A,B} \bc T \;|\; \neg A \;|\; A \wedge B \;|\; \eta = \eta'}
  \and
  \inferrule* [lab=spatial] {} {|\; \pzero \;|\; A | B \;|\; x \text{\textregistered} A \;|\; \forall x . A \;|\;  H x . A}
  \and
  \inferrule* [lab=behavioral] {} {|\; \alpha . A}
  \and 
  \inferrule* [lab=recursion] {} {|\; X(\vec{u}) \;|\; \mu X(\vec{u}) . A}
  \and
  \inferrule* [lab=action] {} {\alpha \bc \langle x?(\vec{y}) \rangle \;|\; \langle x!(\vec{y}) \rangle \;|\; \langle \tau \rangle}
  \and 
  \inferrule* [lab=name] {} {\eta \bc x \;|\; \tau}
\end{mathpar} 

% subsection characteristic_formulae (end)   	 

\subsection{Example formulae}\label{sub:example_formulae_} % (fold)

\subsubsection{Crossing as formula.}
% 
% \begin{align*}
%   \frac{d}{dx} \sin x &= \cos x 
%   & \frac{d}{dx} e^x &= e^x \\
%   \frac{d}{dx} \cos x &= - \sin x 
%   & \frac{d}{dx} \log x &= \frac{1}{x} \\
% \end{align*} 

\begin{align*}
 \mu C(x_{0},x_{1},y_{0},y_{1},u).&(\langle x_{0}?(z) \rangle(\langle u! \rangle\langle y_{1}!z \rangle C(x_{0},x_{1},y_{0},y_{1},u)) & \\
  & \wedge \langle y_{1}?(z) \rangle (\langle u! \rangle \langle x_{0}!z \rangle C(x_{0},x_{1},y_{0},y_{1},u)) & \\
  & \wedge \langle x_{1}?(z) \rangle (\langle u? \rangle \langle y_{0}!z \rangle C(x_{0},x_{1},y_{0},y_{1},u)) & \\
  & \wedge \langle y_{0}?(z) \rangle (\langle u? \rangle \langle x_{1}!z \rangle C(x_{0},x_{1},y_{0},y_{1},u))) &
\end{align*}

The lexicographical similarity between the shape of this formulae and
the shape of definition of the process representing a crossing reveals
the intuitive meaning of this formulae. It describes the capabilities
of a process that has the right to represent a crossing. For example
it picks out processes that may perform an input on the port $x_0$ in
its initial menu of capabilities. What differentiates the formula
from the process, however, is that the crossing process is the
smallest candidate to satisfy the formula. Infinitely many other
processes -- with internal behavior hidden behind this interface, so
to speak -- also satisfy this formula. Even this simple formula,
then, can be seen to open a new view onto knots, providing a
computational interpretation of \emph{virtual} knots.

Note that this formula is derived by hand. A similar formula can be
derived by employing Caires' calculation of characteristic formula
\cite{Caires04} to the process representing a crossing. In light of
this discussion, we let
$\meaningof{C}_{\phi}(x0,x1,y0,y1,u)$ denote a formula specifying the
dynamics we wish to capture of a crossing. To guarantee we preserve
the shape of the interface and minimal semantics we demand that
$\meaningof{C}_{\phi}(x0,x1,y0,y1,u) \Rightarrow
\textbf{C}(x0,x1,y0,y1,u)$ where $\textbf{C}(x0,x1,y0,y1,u)$ denotes
the formula above.
                            
\subsubsection{Crossing number constraints.}
The moral content of the context lemma (Lemma \ref{context}) is that the notion of
``locality'' in the Reidemeister moves is effectively captured by the
parallel composition operator of the process calculus. This intuition
extends through the logic. Given a formula,
$\meaningof{C}_{\phi}(x0,x1,y0,y1,u)$, we can use the structural
connectives to specify constraints on crossing numbers, such as at
least $n$ crossings, or exactly $n$ crossings.
\begin{mathpar}
  \inferrule* [lab=at-least-n] {} { K^{\geq n}_{\phi}(\vec{xs},\vec{ys}) := \Pi_{i=0}^{n-1} Hu . \meaningof{C}_{\phi}(xs_i,ys_i,u) | T }
  \and 
  \inferrule* [lab=exactly-n] {} { K^{= n}_{\phi}(\vec{xs},\vec{ys}) := \Pi_{i=0}^{n-1} Hu . \meaningof{C}_{\phi}(xs_i,ys_i,u) | \neg (\forall x_0,y_0,x_1,y_1,u . \meaningof{C}_{\phi}(x_0,y_0,x_1,y_1,u) | T) }
\end{mathpar}

To round out this section, recall that the encoding of an $n$-crossing
knot decomposes into a parallel composition of $n$ \emph{copies} of a
crossing process together with a wiring harness. To specify different
knot classes with the same crossing number amounts to specifying
logical constraints on the wiring harness. In the interest of space,
we defer examples to a forthcoming paper. Suffice it to say that both
the conditions ``alternating knot'' and ``contains the tangle
corresponding to 5/3'' are expressible. For example, it is possible to
calculate the characteristic formula of a process corresponding to the
tangle 5/3 and conjoin it into the classifying formula via the
composition connective of the logic.

Finally, we wish to observe that it is entirely within reason to
contemplate a more domain-specific version of spatial logic tailored
to the shape of processes in the image of the encoding. Such a
domain-specific logic would have a better claim to the title formal
language of knot properties.

% subsection example_formulae_ (end)

% section knots_as_processes (end) 

% section spatial logic via knots (end)

\section{Conclusions and future work}

\paragraph{Testing physical space}
You, gentle reader, may wonder why of all the theorems to be proved
given this set up we pick the one above. In some sense it's hardly
central to quantum mechanics. We see it as central in the sense that
it firmly establishes a notion of physical space arising from a notion
of the equivalence of behavior. Relating bisimulation to a metric is a
big step forward, but one is faced with interpreting the relationship
of that metric space to something more physical. Quantum mechanical
notions of ``physical'' space are still far from intuitive, but by
relating this idea of distance as testing to calculations that predict
physical circumstances we are making a not insignificant step forward
toward an understanding of the physical space we inhabit as
essentially dynamic.

\paragraph{Effectivity and simulation}
One of the observations we have yet to make is that the entire program
spelled out here is effective. We have built various interpreters for
the reflective calculus at work in this interpretation. In principle,
then, we can simulate quantum mechanics on a computer. The place where
the simulation may lose fidelity is the infinitely branching summation
for the annihilator.

In this connection i also want to point out that the evaluation style
calculation of the inner product puts the non-determinism of the
summation right at the heart of measurement. This suggests that
Milner's original reduction-based formulation of the dynamics of his
calculi in terms of sums was not just notationally suggestive of a
notion of measure-and-continue but captured some significant part of
the physics.

\paragraph{Quantum continuations}
In light of this last observation i want to point out that the
predominant account of quantum mechanics is missing a key aspect of a
truly compositional story of the physical situation. In a real lab,
when a measurement is made the observation can be made to feed into
another device that then makes another measurement conditioned on the
results of the first. This means that after the superposition was
collapsed the entire experimental set up remained in
superposition. While QM offers a means of writing this down it doesn't
quite line up well with the well-trodden formulation of computation
and continuation that we see so succinctly expressed in Milner's
calculi. This suggests that there might be advantages to this account
of dynamics waiting to be explored.

\paragraph{Quantum logic}
In this connection, we also note that by virtue of having the
Hennessy-Milner construction, we can pull the construction through the
interpretation of QM. This gives us a natural candidate for a quantum
logic that enjoys an extremely tight connection with it's domain of
interpretation, making the construction much less ad hoc (rather it is
the image of functor!).

\paragraph{Quantum probabiity}
i have questions about the basis of the interpretation of inner
product as probability amplitude. In particular, using which
axiomatization of probability theory does the notion of probability
amplitude earn the right to be so dubbed? In other words, where is the
proof that the operation for calculating a probability amplitude (and
then squaring) satisfies the axioms of what it means to calculate a
probability? Even if such a proof exists (i have yet to find it in the
literature), i wonder if it might not be possible to turn things on
their heads. Can we view the calculation of the probability amplitude
as an axiomatization of probability? If so, then the definition we
give for calculating probability amplitude may provide the basis for
an \emph{effective} theory of probability.

\paragraph{Quantum vs ``biological'' information}
Finally, i want to conclude with a more philosophical observation. At
a recent workshop in which QM was a predominant topic i noticed
something about quantum information. The speaker was giving a riveting
discussion of axiomatic QM and showing how properties of ``no
cloning'' and ``no deleting'' emerged as consequences of the
axiomatization. Theorems of this form are necessary to give us a sense
of confidence that our axioms characterize the physical theory. What
struck me, though, was that if quantum information is neither erasable
nor replicable it is markedly different from \emph{life}. Two of the
things we know about life is that

\begin{itemize}
  \item it ends;
  \item to gain some measure of persistence, to transcend it's
    finitude it is imminently copyable.
\end{itemize}

Both of these qualities are summarized succinctly in the aphorism: all
flesh is grass. For me these two kinds of ``information'' -- call them
quantum and biological -- are end points on a spectrum of strategies
for persistence. At one end, we have those curious entities that enjoy
uniqueness and permanence; at the other, we have those who in the face
of a certain end and an uncertain present make a go of passing
something on. To me one of the more remarkable aspects of the latter
strategy is that in the presence of noise (and certain features of
copying) we get a kind of dynamism, a chance for improvement against a
given persistent condition.

% subsection other_calculi_other_bisimulations_and_geometry_as_behavior (end)




% section conclusion (end)

%\documentclass[12pt]{llncs}
%\documentclass{jktr}

\usepackage[pdftex]{hyperref}                   
\usepackage {listings}
\usepackage {mathpartir}
\usepackage{bcprules}
%\usepackage{listings}
                       
\usepackage{graphicx} 
%\usepackage[margins=2.5cm,nohead,nofoot]{geometry}
%\usepackage{geometry}
\usepackage{amsfonts}
\usepackage{amstext}
\usepackage{latexsym}
\usepackage{amssymb}
\usepackage{color}


%\include{myPreamble}
\include{qm2pi.local} 

%\ifpdf
%\usepackage[pdftex]{graphicx}
%\else
%\usepackage{graphicx}
%\fi

 % \ifpdf
%  \usepackage{pdfsync}
%  \if


%\title{Brief Article}
%\author{David F. Snyder}
%\author{L.G. Meredith}

%\address{Dept. of Math., Texas State University--San Marcos, San Marcos, TX 78666}
       
\pagestyle{empty}


\begin{document}

\lstset{language=[Objective]Caml,frame=shadowbox}

\input{qm2pi.front}

% section front matter (end)

\input{qm2pi.intro} 
 
% section introduction (end)

% \input{qm2pi.knotations} 

% section notation (end)

\input{qm2pi.process.calculi} 

% section concurrent_process_calculi_and_spatial_logics_ (end)
    
%\input{qm2pi.knots2pi} 

%\input{qm2pi.trefoil} 

%\input{qm2pi.mainthm} 

% subsection basic_interpretation (end)

%\input{qm2pi.rho.presentation} 
\subsection{The syntax and semantics of the notation system}\label{sub:the_syntax_and_semantics_of_the_notation_system} % (fold)

We now summarize a technical presentation of the calculus that
embodies our theory of dynamics. The typical presentation of such a
calculus follows the style of giving generators and relations on
them. The grammar, below, describing term constructors, freely
generates the set of processes, $\Proc$. This set is then quotiented
by a relation known as structural congruence and it is over this set
that the notion of dynamics is expressed. This presentation is
essentially that of \cite{MeredithR05} with the addition of
polyadicity and summation. For readability we have relegated some of
the technical subtleties to an appendix.

\subsubsection{Process grammar}\label{subsub:process_grammar}

\begin{mathpar}
  \inferrule* [lab=synchronization] {} {{M} \bc \pzero \;|\; x?F \;|\; x!C }
  \and
  \inferrule* [lab=abstraction] {} {{F} \bc (x)P}
  \and
  \inferrule* [lab=concretion] {} {{C} \bc \langle Q \rangle}
  \and
  \inferrule* [lab=process] {} {{P,Q} \bc M \;| \;P|Q \;|\; @{x}}
  \and
  \inferrule* [lab=name] {} {{x} \bc \quotep{P}}
\end{mathpar} 

Note that $\vec{x}$ (resp. $\vec{P}$) denotes a vector of names
(resp. processes) of length $|\vec{x}|$ (resp. $|\vec{P}|$). We adopt
the following useful abbreviations.

\begin{mathpar}
   x?(\vec{y}).P := x.(\vec{y})P \and  x\clift{\vec{P}} := x.\clift{\vec{P}}
   \and x!(y) := \lift{x}{\dropn{y}}
   \and \Pi_{i=0}^{n-1}P_i := P_0 | \ldots | P_{n-1}
\end{mathpar}

\subsubsection{Structural congruence}

\paragraph{Free and bound names and alpha-equivalence.} At the
core of structural equivalence is alpha-equivalence which identifies
process that are the same up to a change of variable. Formally, we
recognize the distinction between free and bound names. The free names
of a process, $\freenames{P}$, may be calculated recursively as
follows:

\begin{mathpar}
\freenames{\pzero} := \emptyset
  \and \\
  \freenames{x?(y).P} := \{ x \} \cup (\freenames{P} \setminus \{ y \})
  \and 
  \freenames{x!\langle P \rangle} := \{ x \} \cup \{ P \} 
  \and \\
  \freenames{P|Q} := \freenames{P} \cup \freenames{Q}
  \and \\
  \freenames{@{x}} := \{ x \}
\end{mathpar}

$\pi$
$\quotep{\pi}$

$\freenames{-} : \pi \to \mathcal{P}(\quotep{\pi})$

\begin{eqnarray*}
  \freenames{\pzero} & := & \emptyset \\
  \freenames{x?(y).P} & := & \{ x \} \cup (\freenames{P} \setminus \{ y \}) \\
  \freenames{x!\langle P \rangle} & := & \{ x \} \cup \{ P \} \\
  \freenames{P|Q} & := & \freenames{P} \cup \freenames{Q} \\
  \freenames{\dropn{x}} & := & \{ x \}
\end{eqnarray*}

The bound names of a process, $\boundnames{P}$, are those names occurring in $P$
that are not free. For example, in $x?(y).0$, the name $x$ is free, while $y$ is bound.

\begin{mathpar}
  \inferrule* [lab=monoidal-laws] {} { P|Q \equiv Q|P \and P|0 \equiv P \and P|(Q|R) \equiv (P|Q)|R }
\end{mathpar}

\begin{mathpar}
  \inferrule* [lab=alpha-equivalence] {} { (x)P \equiv (y)P\{y/x\} \and y \not\in \freenames{P} }
\end{mathpar}

\begin{definition}
Then two processes, $P,Q$, are alpha-equivalent if $P = Q\{\vec{y}/\vec{x}\}$ for
some $\vec{x} \in \boundnames{Q},\vec{y} \in \boundnames{P}$, where $Q\{\vec{y}/\vec{x}\}$
denotes the capture-avoiding substitution of $\vec{y}$ for $\vec{x}$ in $Q$.
\end{definition}

\begin{definition}
  The {\em structural congruence} \cite{SangiorgiWalker} , $\equiv$,
  between processes is the least congruence containing
  alpha-equivalence, satisfying the abelian monoid laws
  (associativity, commutativity and $\pzero$ as identity) for parallel
  composition $|$ and for summation $+$.
\end{definition}

\subsection{Name equivalence}

We take name equivalence, written $\nameeq$, to be the smallest
equivalence relation generated by the following rules.

\begin{mathpar}
\inferrule*[lab=Quote-drop]
{ }
{ \quotep{@{x}} \nameeq x }

\inferrule*[lab=Struct-equiv]
{ P \scong Q }
{ \quotep{P} \nameeq \quotep{Q} }
\end{mathpar}

The astute reader will have noticed that the mutual recursion of names
and processes imposes a mutual recursion on alpha-equivalence and
structural equivalence via name-equivalence. Fortunately, all of this
works out pleasantly and we may calculate in the natural way, free of
concern. The reader interested in the details is referred to the
appendix \ref{appendix:rho_details}.

\subsection{Substitution}

We use $\Proc$ for the set of processes, $\QProc$ for the set of
names, and $\id{\{}\vec{y} / \vec{x} \id{\}}$ to denote partial maps,
$s : \QProc \rightarrow \QProc$. A map, $s$ lifts, uniquely, to a map
on process terms, $\widehat{s} : \Proc \rightarrow \Proc$ by the
following equations.

\begin{mathpar}
  (0) \psubstp{Q}{P} := 0 \\
  (R \juxtap S) \psubstp{Q}{P}
  :=    
  (R)\psubstp{Q}{P} \juxtap (S) \psubstp{Q}{P} \\
  (x?(y).R) \psubstp{Q}{P}    
  :=    
  (x)\substp{Q}{P} (z)\concat( (R \psubstn{z}{y}) \psubstp{Q}{P} ) \\
  (\lift{x}{R}) \psubstp{Q}{P}  
  :=
  \lift{(x)\substp{Q}{P}}{ R \psubstp{Q}{P} } \\
%   (\dropn{x})  \psubstp{Q}{P}       
%   := 
%   \left\{ 
%     \begin{array}{ccc} 
%       \dropn{\quotep{Q}} & & x \nameeq \quotep{P} \\
%       \dropn{x} & & otherwise \\
%     \end{array}
%   \right. 
  (\dropn{x})  \psubstp{Q}{P}       
  := 
  \left\{ 
    \begin{array}{ccc} 
      Q & & x \nameeq \quotep{P} \\
      \dropn{x} & & otherwise \\
    \end{array}
  \right.
\end{mathpar}
 

where

\begin{eqnarray}
  (x)\id{\{} \lpquote Q \rpquote / \lpquote P \rpquote \id{\}}            = 
  \left\{ 
    \begin{array}{ccc}
      \lpquote Q \rpquote & & x \nameeq \lpquote P \rpquote \\
      x & & otherwise \\
    \end{array}
  \right. \nonumber
\end{eqnarray}

and $z$ is chosen distinct from $\quotep{P}$, $\quotep{Q}$, the free
names in $Q$, and all the names in $R$. Our $\alpha$-equivalence will
be built in the standard way from this substitution.

\begin{remark}\label{rem:no_self_referential_names}
  One consequence of these definitions is that $\forall P. \quotep{P}
  \not\in \freenames{P}$.
\end{remark}

\subsection{ Dynamic quote: an example }

Anticipating something of what's to come, consider applying the
substitution, $\widehat{\id{\{}u / z \id{\}}}$, to the following pair
of processes, $\lift{w}{y!(z)}$ and $w[ \lpquote y!(z) \rpquote ]$.

\begin{eqnarray}
	\lift{w}{y!(z)}\widehat{\id{\{}u / z \id{\}}}
		& = &
		\lift{w}{y!(u)} \nonumber\\
	w[ \lpquote y!(z) \rpquote ] \widehat{ \id{\{}u / z \id{\}} }
		& = &
		w[ \lpquote y!(z) \rpquote ] \nonumber
\end{eqnarray}

Because the body of the process between quotes is impervious to
substitution, we get radically different answers. In fact, by
examining the first process in an input context,
e.g. $x?(z).\lift{w}{y!(z)}$, we see that the process under the lift
operator may be shaped by prefixed inputs binding a name inside it. In
this sense, the lift operator will be seen as a way to dynamically
construct processes before reifying them as names.

Finally equipped with these standard features we can present the
dynamics of the calculus.

\subsubsection{Operational semantics} 

Finally, we introduce the computational dynamics. What marks these
algebras as distinct from other more traditionally studied algebraic
structures, e.g. vector spaces or polynomial rings, is the manner in
which dynamics is captured. In traditional structures, dynamics is typically
expressed through morphisms between such structures, as in linear maps
between vector spaces or morphisms between rings. In algebras
associated with the semantics of computation, the dynamics is
expressed as part of the algebraic structure itself, through a
reduction reduction relation typically denoted by $\red$. Below, we
give a recursive presentation of this relation for the calculus used
in the encoding.

$\red \subseteq \pi \times \pi$
$\red : \pi \to \mathcal{P}(\pi)$

\begin{mathpar}
  \inferrule* [lab=Comm] { \textsf{match}( x_{src}, x_{trgt} ) } { x_{trgt}?(y)P \; | \; x_{src}!\langle {Q} \rangle \red P\{\quotep{Q}/y}\} }
  \and \\
  \inferrule* [lab=Par] {{P} \red {P}'} {{{P} | {Q}} \red {{P}' | {Q}}}
  \and
  \inferrule* [lab=Equiv]{{{P} \scong {P}'} \andalso {{P}' \red {Q}'} \andalso {{Q}' \scong {Q}}}{{P} \red {Q}}
\end{mathpar}

\begin{eqnarray*}
  match_{\equiv} (\quotep{P},\quotep{Q}) & := & P \equiv Q \\
  match_{\dagger}(\quotep{P},\quotep{Q}) & := & \forall R. P|Q \red^{*} R => R \red^{*} 0 \\
  match_{K}(\quotep{P},\quotep{Q}) & := & K \mbox{ for some context } K
\end{eqnarray*}

$u?(x)P | u!\langle Q \rangle \red P\{\quotep{Q}/x\}$

%We write $\wred$ for $\red^*$, and $P\red$ if $\exists Q $ such that $ P \red Q$.
We write $P\red$ if $\exists Q $ such that $ P \red Q$ and $P\not\red$, otherwise.

\section{Replication}

As mentioned before, it is known that replication (and hence
recursion) can be implemented in a higher-order process algebra
\cite{SangiorgiWalker}. As our first example of calculation with the
machinery thus far presented we give the construction explicitly in
the {\rhoc}.

\begin{eqnarray}
	D_{x} & := & \prefix{x}{y}{(\binpar{\outputp{x}{y}}{@{y}})} \nonumber\\
	\bangp_{x}{P} & := & \binpar{{x}!\langle{\binpar{D_{x}}{P}}\rangle}{D_{x}} \nonumber
\end{eqnarray}

\begin{eqnarray}
	\bangp_{x}{P} & & \nonumber\\
	=
	& {x}!\langle{(\prefix{x}{y}{(\outputp{x}{y} | @{y})) | P}}\rangle 
	      | \prefix{x}{y}{(\outputp{x}{y} | @{y})} & \nonumber\\
	\red
	& (\outputp{x}{y} | @{y})\substn{\quotep{(\prefix{x}{y}{(@{y} | \outputp{x}{y})) | P}}}{y} & \nonumber\\
	=
	& \outputp{x}{\quotep{(\prefix{x}{y}{(\outputp{x}{y} | @{y})) | P}}}
	  | {(\prefix{x}{y}{(\outputp{x}{y} | @{y})) | P}} & \nonumber\\
	\red
	& \ldots & \nonumber\\
	\red^*
	& P | P | \ldots & \nonumber
\end{eqnarray}

Of course, this encoding, as an implementation, runs away, unfolding
$\bangp{P}$ eagerly. A lazier and more implementable replication
operator, restricted to input-guarded processes, may be obtained as follows.

\begin{eqnarray}
\bangp{\prefix{u}{v}{P}} 
	:= 
	\binpar{\lift{x}{\prefix{u}{v}{(\binpar{D(x)}{P})}}}{D(x)} \nonumber
\end{eqnarray}

\begin{remark}
  Note that the lazier definition still does not deal with summation
  or mixed summation (i.e. sums over input and output). The reader is
  invited to construct definitions of replication that deal with these
  features. 

  Further, the definitions are parameterized in a name, $x$. Can you,
  gentle reader, make a definition that eliminates this parameter and
  guarantees no accidental interaction between the replication
  machinery and the process being replicated -- i.e. no accidental
  sharing of names used by the process to get its work done and the
  name(s) used by the replication to effect copying. This latter
  revision of the definition of replication is crucial to obtaining
  the expected identity $!!P \sim !P$.
\end{remark}

\begin{remark}\label{rem:paradoxical_combinator}
  The reader familiar with the lambda calculus will have noticed the
  similarity between $D$ and the paradoxical combinator.

  [Ed. note: the existence of this seems to suggest we have to be more
  restrictive on the set of processes and names we admit if we are to
  support no-cloning.]
\end{remark}

\subsubsection{Bisimulation}

The computational dynamics gives rise to another kind of equivalence,
the equivalence of computational behavior. As previously mentioned
this is typically captured \emph{via} some form of bisimulation.

% The notion we use in this paper is weak barbed bisimulation
% \cite{milner91polyadicpi}.

The notion we use in this paper is derived from weak barbed
bisimulation \cite{milner91polyadicpi}. 

\begin{definition}
An \emph{observation relation}, $\downarrow_{\mathcal N}$, over a set
of names, $\mathcal N$, is the smallest relation satisfying the rules
below.

\infrule[Out-barb]{y \in {\mathcal N}, \; x \nameeq y}
		  {\outputp{x}{v} \downarrow_{\mathcal N} x}
\infrule[Par-barb]{\mbox{$P\downarrow_{\mathcal N} x$ or $Q\downarrow_{\mathcal N} x$}}
		  {\binpar{P}{Q} \downarrow_{\mathcal N} x}

We write $P \Downarrow_{\mathcal N} x$ if there is $Q$ such that 
$P \wred Q$ and $Q \downarrow_{\mathcal N} x$.
\end{definition}

\begin{definition}
%\label{def.bbisim}
An  ${\mathcal N}$-\emph{barbed bisimulation} over a set of names, ${\mathcal N}$, is a symmetric binary relation 
${\mathcal S}_{\mathcal N}$ between agents such that $P\rel{S}_{\mathcal N}Q$ implies:
\begin{enumerate}
\item If $P \red P'$ then $Q \wred Q'$ and $P'\rel{S}_{\mathcal N} Q'$.
\item If $P\downarrow_{\mathcal N} x$, then $Q\Downarrow_{\mathcal N} x$.
\end{enumerate}
$P$ is ${\mathcal N}$-barbed bisimilar to $Q$, written
$P \wbbisim_{\mathcal N} Q$, if $P \rel{S}_{\mathcal N} Q$ for some ${\mathcal N}$-barbed bisimulation ${\mathcal S}_{\mathcal N}$.
\end{definition}

$\mathcal{R} \subseteq \pi \times \pi$

$P \mathcal{R} Q => \forall P'. P \red P' \Rightarrow \exists Q'. Q \red Q', P' \mathcal{R} Q'$

$P \vdash x \Rightarrow Q \vdash x$

\begin{mathpar}
  \inferrule*[lab=Out-barb]{x \nameeq y}{{y}!\langle{Q}\rangle \vdash x}
  \and
  \inferrule*[lab=Par-barb]{\mbox{$P\vdash x$ or $Q\vdash x$}}{\binpar{P}{Q} \vdash x}
\end{mathpar}

\subsubsection{Contexts}

One of the principle advantages of computational calculi like the
$\pi$-calculus is a well-defined notion of context,
contextual-equivalence and a correlation between
contextual-equivalence and notions of bisimulation. The notion of
context allows the decomposition of a process into (sub-)process and
its syntactic environment, its context. Thus, a context may be
thought of as a process with a ``hole'' (written $\Box$) in it. The
application of a context $M$ to a process $P$, written $M[P]$, is
tantamount to filling the hole in $M$ with $P$. In this paper we do
not need the full weight of this theory, but do make use of the notion
of context in the proof the main theorem. 

\begin{mathpar}
  \inferrule* [lab=summation] {} {{M_{M},M_{N}} \bc \Box \;|\; x.M_{A} \;|\; M_{M}+M_{N}}
  \and
  \inferrule* [lab=agent] {} {{M_{A}} \bc (\vec{x})M_{P} \;| \; \clift{P_0,\ldots,M_{P},\ldots,P_N}}
  \and \\
  \inferrule* [lab=process] {} {{M_{P}} \bc M_{N} \;| \;P|M_{P} }
\end{mathpar} 

\begin{mathpar}
  \inferrule* [lab=sychronization] {} {M_{N} \bc \Box \;|\; x?M_{F} \;|\; x!M_{C}}
  \and
  \inferrule* [lab=abstraction] {} {{M_{F}} \bc (x)M_{P} }
  \and
  \inferrule* [lab=concretion] {} {{M_{C}} \bc \langle M_{P} \rangle }
  \and \\
  \inferrule* [lab=process] {} {{M_{P}} \bc M_{N} \;| \;P|M_{P} }
\end{mathpar}

\begin{definition}[contextual application] Given a context $M$, and
  process $P$, we define the \emph{contextual application}, $M[P] :=
  M\{P/\Box\}$. That is, the contextual application of M to P is the
  substitution of $P$ for $\Box$ in $M$.
\end{definition}

$\meaningof{-} : L \to \mathcal{P}(\pi)$

\begin{mathpar}
  \inferrule* [lab=collection] {} {\meaningof{true} = \pi, \and \meaningof{~E} = \pi \setminus \meaningof{E}, \and \meaningof{E_{1} \& E_{2}} = \meaningof{E_{1}} \cap \meaningof{E_{2}}}
\end{mathpar}

\begin{mathpar}
  \inferrule* [lab=structure] {} {\meaningof{0} = \{ P \in \pi | P \equiv 0 \}, \and \\ \meaningof{E_1 | E_2} = \{ P \in \pi | P \equiv P_{1} | P_{2}, P_{1} \in \meaningof{E_{1}}, P_{2} \in \meaningof{E_2}\} }
\end{mathpar}

\begin{mathpar}
 \inferrule* [lab=behavior] {} {\meaningof{\langle a?b \rangle E} = \{ P \in \pi | P \equiv Q | u?(y)P', \\ \and \\\\ \and \\ \;\;\; u \in \meaningof{a}, \forall z.P'\{z/y\} \in \meaningof{E\{z/b\}}\}, \and \\ \meaningof{a!E} = \{ P \in \pi | P \equiv Q | x!\langle P' \rangle, x \in \meaningof{a} P' \in \meaningof{E}\} }
\end{mathpar}

\begin{mathpar}
 \inferrule* [lab=nominal] {} {\meaningof{\quotep{E}} = \{ \quotep{P} \in \quotep{\pi} | P \in \meaningof{E} \}, \and \meaningof{\quotep{P}} = \{ \quotep{Q} \in \quotep{\pi} | P \equiv Q \} \and \\ \meaningof{@\quotep{E}} = \{ P \in \pi | P \equiv @x, x \in \meaningof{E} \}}
\end{mathpar}

\begin{eqnarray*}
  \\
  \meaningof{-} : TS \to ST
\end{eqnarray*}

\begin{eqnarray*}
  \\
  L : TS \to ST
\end{eqnarray*}

\begin{eqnarray*}
  \\
  P \models E \iff P \in \meaningof{E}
\end{eqnarray*}

\begin{eqnarray*}
  P \approx_{L} Q \iff \forall E \in L. P \models E \iff Q \models E
\end{eqnarray*}

\begin{eqnarray*}
  P \approx_{K} Q
\end{eqnarray*}

\begin{eqnarray*}
  P \approx Q
\end{eqnarray*}

$\approx_{K} = \approx = \approx_{L}$

\subsubsection{Contextual duality}

Note that contexts extend the quotation operation to a family of
operations from processes to names. Given a context, $M$, we can
define a \emph{nominal context}, $\quotep{M}$ by $\quotep{M}[P] :=
\quotep{M[P]}$. To foreshadow what is to come we observe that these
operations enjoy a duality with processes very much like the duality
between vectors and maps from vectors to scalars.

Further, because the calculus is essentially higher-order, we have a
correspondence between contexts and processes. More specifically,
given a name $x$ and a context $M$ we can construct $M^{*}_{x}$ such
that 

\begin{mathpar}
  M^{*}_{x} | \lift{x}{P} \red M[P]
\end{mathpar}

namely,

\begin{mathpar}
  M^{*}_{x} := x?(u).M[\dropn{u}]
\end{mathpar}

The dependence of $M^{*}_{x}$ on a name makes it an abstraction, 

\begin{mathpar}
  M^{*} := (x)x?(u).M[\dropn{u}]
\end{mathpar}

\subsection{Additional notation}

It will sometimes be convenient to denote the process a name
quotes. We already have the notation $x = \quotep{P}$, but it will be
convenient to introduce an alternate notation, $\procn{x}$, when we
want to emphasize the connection to the use of the name. Note that, by
virtue of name equivalence, $\quotep{\procn{x}} \nameeq x$; so, the
notation is consistent with previous definitions.

Further, because names have structure it is possible to effect
substitutions on the basis of that structure. This means we need to
upgrade our notation for substitutions, which we accomplish by
adapting comprehension notation. Thus,

\begin{mathpar}
  P\{ y / x : x \in S \}
\end{mathpar}

is interpreted to mean the process derived from P by replacing (in a
capture-avoiding manner) each occurrence of $x$ in $S$ by $y$. For example,

\begin{mathpar}
  P\{ \quotep{\procn{x}|\procn{x}} / x : x \in \freenames{P} \}
\end{mathpar}

will replace each (occurrence) of a free name $x$ in $P$ by
$\quotep{\procn{x}|\procn{x}}$.

Also, we will avail ourselves of the notation $x^{L}$ and $x^{R}$ to
denote injections of a name into disjoint copies of the name
space. There are numerous ways to accomplish this. One example can be
found in \cite{MeredithR05}. This notation overloads to vectors of
names: $\vec{x}^{\pi} := (x_{i}^{\pi} \; : \; 0 \leq i < |\vec{x}| )$ where $\pi \in \{L,R\}$.

We also use $P^{\Box} := P|\Box$.

In \cite{MeredithR05} an interpretation of the new operator is
given. It turns out that there are several possible interpretations
all enjoying the requisite algebraic properties of the operator (see
\cite{milner91polyadicpi}). We will therefore make liberal use of
$(\nu\; \vec{x})P$.

% subsection the_syntax_and_semantics_of_the_notation_system (end)   

\input{qm2pi.qmops} 

\input{qm2pi.sterngerlach} 

\input{qm2pi.metric} 

% section concurrent_process_calculi (end)

%\input{qm2pi.proofsketch}

% section proof sketch (end)

%\input{qm2pi.slviaknots} 

% section spatial logic via knots (end)

\input{qm2pi.conclusion}

% section conclusion (end)

%\input{qm2pi.dtcodes} 

% section wiring algorithm (end)

\input{qm2pi.ack} 

% section acknowledgments (end)

\newpage


\bibliographystyle{plain}   
\bibliography{../../biblios/main.bib}

\input{qm2pi.rhodetails}

\end{document}

 

% section wiring algorithm (end)

\documentclass[12pt]{llncs}
%\documentclass{jktr}

\usepackage[pdftex]{hyperref}                   
\usepackage {listings}
\usepackage {mathpartir}
\usepackage{bcprules}
%\usepackage{listings}
                       
\usepackage{graphicx} 
%\usepackage[margins=2.5cm,nohead,nofoot]{geometry}
%\usepackage{geometry}
\usepackage{amsfonts}
\usepackage{amstext}
\usepackage{latexsym}
\usepackage{amssymb}
\usepackage{color}


%\include{myPreamble}
\include{qm2pi.local} 

%\ifpdf
%\usepackage[pdftex]{graphicx}
%\else
%\usepackage{graphicx}
%\fi

 % \ifpdf
%  \usepackage{pdfsync}
%  \if


%\title{Brief Article}
%\author{David F. Snyder}
%\author{L.G. Meredith}

%\address{Dept. of Math., Texas State University--San Marcos, San Marcos, TX 78666}
       
\pagestyle{empty}


\begin{document}

\lstset{language=[Objective]Caml,frame=shadowbox}

\input{qm2pi.front}

% section front matter (end)

\input{qm2pi.intro} 
 
% section introduction (end)

% \input{qm2pi.knotations} 

% section notation (end)

\input{qm2pi.process.calculi} 

% section concurrent_process_calculi_and_spatial_logics_ (end)
    
%\input{qm2pi.knots2pi} 

%\input{qm2pi.trefoil} 

%\input{qm2pi.mainthm} 

% subsection basic_interpretation (end)

%\input{qm2pi.rho.presentation} 
\subsection{The syntax and semantics of the notation system}\label{sub:the_syntax_and_semantics_of_the_notation_system} % (fold)

We now summarize a technical presentation of the calculus that
embodies our theory of dynamics. The typical presentation of such a
calculus follows the style of giving generators and relations on
them. The grammar, below, describing term constructors, freely
generates the set of processes, $\Proc$. This set is then quotiented
by a relation known as structural congruence and it is over this set
that the notion of dynamics is expressed. This presentation is
essentially that of \cite{MeredithR05} with the addition of
polyadicity and summation. For readability we have relegated some of
the technical subtleties to an appendix.

\subsubsection{Process grammar}\label{subsub:process_grammar}

\begin{mathpar}
  \inferrule* [lab=synchronization] {} {{M} \bc \pzero \;|\; x?F \;|\; x!C }
  \and
  \inferrule* [lab=abstraction] {} {{F} \bc (x)P}
  \and
  \inferrule* [lab=concretion] {} {{C} \bc \langle Q \rangle}
  \and
  \inferrule* [lab=process] {} {{P,Q} \bc M \;| \;P|Q \;|\; @{x}}
  \and
  \inferrule* [lab=name] {} {{x} \bc \quotep{P}}
\end{mathpar} 

Note that $\vec{x}$ (resp. $\vec{P}$) denotes a vector of names
(resp. processes) of length $|\vec{x}|$ (resp. $|\vec{P}|$). We adopt
the following useful abbreviations.

\begin{mathpar}
   x?(\vec{y}).P := x.(\vec{y})P \and  x\clift{\vec{P}} := x.\clift{\vec{P}}
   \and x!(y) := \lift{x}{\dropn{y}}
   \and \Pi_{i=0}^{n-1}P_i := P_0 | \ldots | P_{n-1}
\end{mathpar}

\subsubsection{Structural congruence}

\paragraph{Free and bound names and alpha-equivalence.} At the
core of structural equivalence is alpha-equivalence which identifies
process that are the same up to a change of variable. Formally, we
recognize the distinction between free and bound names. The free names
of a process, $\freenames{P}$, may be calculated recursively as
follows:

\begin{mathpar}
\freenames{\pzero} := \emptyset
  \and \\
  \freenames{x?(y).P} := \{ x \} \cup (\freenames{P} \setminus \{ y \})
  \and 
  \freenames{x!\langle P \rangle} := \{ x \} \cup \{ P \} 
  \and \\
  \freenames{P|Q} := \freenames{P} \cup \freenames{Q}
  \and \\
  \freenames{@{x}} := \{ x \}
\end{mathpar}

$\pi$
$\quotep{\pi}$

$\freenames{-} : \pi \to \mathcal{P}(\quotep{\pi})$

\begin{eqnarray*}
  \freenames{\pzero} & := & \emptyset \\
  \freenames{x?(y).P} & := & \{ x \} \cup (\freenames{P} \setminus \{ y \}) \\
  \freenames{x!\langle P \rangle} & := & \{ x \} \cup \{ P \} \\
  \freenames{P|Q} & := & \freenames{P} \cup \freenames{Q} \\
  \freenames{\dropn{x}} & := & \{ x \}
\end{eqnarray*}

The bound names of a process, $\boundnames{P}$, are those names occurring in $P$
that are not free. For example, in $x?(y).0$, the name $x$ is free, while $y$ is bound.

\begin{mathpar}
  \inferrule* [lab=monoidal-laws] {} { P|Q \equiv Q|P \and P|0 \equiv P \and P|(Q|R) \equiv (P|Q)|R }
\end{mathpar}

\begin{mathpar}
  \inferrule* [lab=alpha-equivalence] {} { (x)P \equiv (y)P\{y/x\} \and y \not\in \freenames{P} }
\end{mathpar}

\begin{definition}
Then two processes, $P,Q$, are alpha-equivalent if $P = Q\{\vec{y}/\vec{x}\}$ for
some $\vec{x} \in \boundnames{Q},\vec{y} \in \boundnames{P}$, where $Q\{\vec{y}/\vec{x}\}$
denotes the capture-avoiding substitution of $\vec{y}$ for $\vec{x}$ in $Q$.
\end{definition}

\begin{definition}
  The {\em structural congruence} \cite{SangiorgiWalker} , $\equiv$,
  between processes is the least congruence containing
  alpha-equivalence, satisfying the abelian monoid laws
  (associativity, commutativity and $\pzero$ as identity) for parallel
  composition $|$ and for summation $+$.
\end{definition}

\subsection{Name equivalence}

We take name equivalence, written $\nameeq$, to be the smallest
equivalence relation generated by the following rules.

\begin{mathpar}
\inferrule*[lab=Quote-drop]
{ }
{ \quotep{@{x}} \nameeq x }

\inferrule*[lab=Struct-equiv]
{ P \scong Q }
{ \quotep{P} \nameeq \quotep{Q} }
\end{mathpar}

The astute reader will have noticed that the mutual recursion of names
and processes imposes a mutual recursion on alpha-equivalence and
structural equivalence via name-equivalence. Fortunately, all of this
works out pleasantly and we may calculate in the natural way, free of
concern. The reader interested in the details is referred to the
appendix \ref{appendix:rho_details}.

\subsection{Substitution}

We use $\Proc$ for the set of processes, $\QProc$ for the set of
names, and $\id{\{}\vec{y} / \vec{x} \id{\}}$ to denote partial maps,
$s : \QProc \rightarrow \QProc$. A map, $s$ lifts, uniquely, to a map
on process terms, $\widehat{s} : \Proc \rightarrow \Proc$ by the
following equations.

\begin{mathpar}
  (0) \psubstp{Q}{P} := 0 \\
  (R \juxtap S) \psubstp{Q}{P}
  :=    
  (R)\psubstp{Q}{P} \juxtap (S) \psubstp{Q}{P} \\
  (x?(y).R) \psubstp{Q}{P}    
  :=    
  (x)\substp{Q}{P} (z)\concat( (R \psubstn{z}{y}) \psubstp{Q}{P} ) \\
  (\lift{x}{R}) \psubstp{Q}{P}  
  :=
  \lift{(x)\substp{Q}{P}}{ R \psubstp{Q}{P} } \\
%   (\dropn{x})  \psubstp{Q}{P}       
%   := 
%   \left\{ 
%     \begin{array}{ccc} 
%       \dropn{\quotep{Q}} & & x \nameeq \quotep{P} \\
%       \dropn{x} & & otherwise \\
%     \end{array}
%   \right. 
  (\dropn{x})  \psubstp{Q}{P}       
  := 
  \left\{ 
    \begin{array}{ccc} 
      Q & & x \nameeq \quotep{P} \\
      \dropn{x} & & otherwise \\
    \end{array}
  \right.
\end{mathpar}
 

where

\begin{eqnarray}
  (x)\id{\{} \lpquote Q \rpquote / \lpquote P \rpquote \id{\}}            = 
  \left\{ 
    \begin{array}{ccc}
      \lpquote Q \rpquote & & x \nameeq \lpquote P \rpquote \\
      x & & otherwise \\
    \end{array}
  \right. \nonumber
\end{eqnarray}

and $z$ is chosen distinct from $\quotep{P}$, $\quotep{Q}$, the free
names in $Q$, and all the names in $R$. Our $\alpha$-equivalence will
be built in the standard way from this substitution.

\begin{remark}\label{rem:no_self_referential_names}
  One consequence of these definitions is that $\forall P. \quotep{P}
  \not\in \freenames{P}$.
\end{remark}

\subsection{ Dynamic quote: an example }

Anticipating something of what's to come, consider applying the
substitution, $\widehat{\id{\{}u / z \id{\}}}$, to the following pair
of processes, $\lift{w}{y!(z)}$ and $w[ \lpquote y!(z) \rpquote ]$.

\begin{eqnarray}
	\lift{w}{y!(z)}\widehat{\id{\{}u / z \id{\}}}
		& = &
		\lift{w}{y!(u)} \nonumber\\
	w[ \lpquote y!(z) \rpquote ] \widehat{ \id{\{}u / z \id{\}} }
		& = &
		w[ \lpquote y!(z) \rpquote ] \nonumber
\end{eqnarray}

Because the body of the process between quotes is impervious to
substitution, we get radically different answers. In fact, by
examining the first process in an input context,
e.g. $x?(z).\lift{w}{y!(z)}$, we see that the process under the lift
operator may be shaped by prefixed inputs binding a name inside it. In
this sense, the lift operator will be seen as a way to dynamically
construct processes before reifying them as names.

Finally equipped with these standard features we can present the
dynamics of the calculus.

\subsubsection{Operational semantics} 

Finally, we introduce the computational dynamics. What marks these
algebras as distinct from other more traditionally studied algebraic
structures, e.g. vector spaces or polynomial rings, is the manner in
which dynamics is captured. In traditional structures, dynamics is typically
expressed through morphisms between such structures, as in linear maps
between vector spaces or morphisms between rings. In algebras
associated with the semantics of computation, the dynamics is
expressed as part of the algebraic structure itself, through a
reduction reduction relation typically denoted by $\red$. Below, we
give a recursive presentation of this relation for the calculus used
in the encoding.

$\red \subseteq \pi \times \pi$
$\red : \pi \to \mathcal{P}(\pi)$

\begin{mathpar}
  \inferrule* [lab=Comm] { \textsf{match}( x_{src}, x_{trgt} ) } { x_{trgt}?(y)P \; | \; x_{src}!\langle {Q} \rangle \red P\{\quotep{Q}/y}\} }
  \and \\
  \inferrule* [lab=Par] {{P} \red {P}'} {{{P} | {Q}} \red {{P}' | {Q}}}
  \and
  \inferrule* [lab=Equiv]{{{P} \scong {P}'} \andalso {{P}' \red {Q}'} \andalso {{Q}' \scong {Q}}}{{P} \red {Q}}
\end{mathpar}

\begin{eqnarray*}
  match_{\equiv} (\quotep{P},\quotep{Q}) & := & P \equiv Q \\
  match_{\dagger}(\quotep{P},\quotep{Q}) & := & \forall R. P|Q \red^{*} R => R \red^{*} 0 \\
  match_{K}(\quotep{P},\quotep{Q}) & := & K \mbox{ for some context } K
\end{eqnarray*}

$u?(x)P | u!\langle Q \rangle \red P\{\quotep{Q}/x\}$

%We write $\wred$ for $\red^*$, and $P\red$ if $\exists Q $ such that $ P \red Q$.
We write $P\red$ if $\exists Q $ such that $ P \red Q$ and $P\not\red$, otherwise.

\section{Replication}

As mentioned before, it is known that replication (and hence
recursion) can be implemented in a higher-order process algebra
\cite{SangiorgiWalker}. As our first example of calculation with the
machinery thus far presented we give the construction explicitly in
the {\rhoc}.

\begin{eqnarray}
	D_{x} & := & \prefix{x}{y}{(\binpar{\outputp{x}{y}}{@{y}})} \nonumber\\
	\bangp_{x}{P} & := & \binpar{{x}!\langle{\binpar{D_{x}}{P}}\rangle}{D_{x}} \nonumber
\end{eqnarray}

\begin{eqnarray}
	\bangp_{x}{P} & & \nonumber\\
	=
	& {x}!\langle{(\prefix{x}{y}{(\outputp{x}{y} | @{y})) | P}}\rangle 
	      | \prefix{x}{y}{(\outputp{x}{y} | @{y})} & \nonumber\\
	\red
	& (\outputp{x}{y} | @{y})\substn{\quotep{(\prefix{x}{y}{(@{y} | \outputp{x}{y})) | P}}}{y} & \nonumber\\
	=
	& \outputp{x}{\quotep{(\prefix{x}{y}{(\outputp{x}{y} | @{y})) | P}}}
	  | {(\prefix{x}{y}{(\outputp{x}{y} | @{y})) | P}} & \nonumber\\
	\red
	& \ldots & \nonumber\\
	\red^*
	& P | P | \ldots & \nonumber
\end{eqnarray}

Of course, this encoding, as an implementation, runs away, unfolding
$\bangp{P}$ eagerly. A lazier and more implementable replication
operator, restricted to input-guarded processes, may be obtained as follows.

\begin{eqnarray}
\bangp{\prefix{u}{v}{P}} 
	:= 
	\binpar{\lift{x}{\prefix{u}{v}{(\binpar{D(x)}{P})}}}{D(x)} \nonumber
\end{eqnarray}

\begin{remark}
  Note that the lazier definition still does not deal with summation
  or mixed summation (i.e. sums over input and output). The reader is
  invited to construct definitions of replication that deal with these
  features. 

  Further, the definitions are parameterized in a name, $x$. Can you,
  gentle reader, make a definition that eliminates this parameter and
  guarantees no accidental interaction between the replication
  machinery and the process being replicated -- i.e. no accidental
  sharing of names used by the process to get its work done and the
  name(s) used by the replication to effect copying. This latter
  revision of the definition of replication is crucial to obtaining
  the expected identity $!!P \sim !P$.
\end{remark}

\begin{remark}\label{rem:paradoxical_combinator}
  The reader familiar with the lambda calculus will have noticed the
  similarity between $D$ and the paradoxical combinator.

  [Ed. note: the existence of this seems to suggest we have to be more
  restrictive on the set of processes and names we admit if we are to
  support no-cloning.]
\end{remark}

\subsubsection{Bisimulation}

The computational dynamics gives rise to another kind of equivalence,
the equivalence of computational behavior. As previously mentioned
this is typically captured \emph{via} some form of bisimulation.

% The notion we use in this paper is weak barbed bisimulation
% \cite{milner91polyadicpi}.

The notion we use in this paper is derived from weak barbed
bisimulation \cite{milner91polyadicpi}. 

\begin{definition}
An \emph{observation relation}, $\downarrow_{\mathcal N}$, over a set
of names, $\mathcal N$, is the smallest relation satisfying the rules
below.

\infrule[Out-barb]{y \in {\mathcal N}, \; x \nameeq y}
		  {\outputp{x}{v} \downarrow_{\mathcal N} x}
\infrule[Par-barb]{\mbox{$P\downarrow_{\mathcal N} x$ or $Q\downarrow_{\mathcal N} x$}}
		  {\binpar{P}{Q} \downarrow_{\mathcal N} x}

We write $P \Downarrow_{\mathcal N} x$ if there is $Q$ such that 
$P \wred Q$ and $Q \downarrow_{\mathcal N} x$.
\end{definition}

\begin{definition}
%\label{def.bbisim}
An  ${\mathcal N}$-\emph{barbed bisimulation} over a set of names, ${\mathcal N}$, is a symmetric binary relation 
${\mathcal S}_{\mathcal N}$ between agents such that $P\rel{S}_{\mathcal N}Q$ implies:
\begin{enumerate}
\item If $P \red P'$ then $Q \wred Q'$ and $P'\rel{S}_{\mathcal N} Q'$.
\item If $P\downarrow_{\mathcal N} x$, then $Q\Downarrow_{\mathcal N} x$.
\end{enumerate}
$P$ is ${\mathcal N}$-barbed bisimilar to $Q$, written
$P \wbbisim_{\mathcal N} Q$, if $P \rel{S}_{\mathcal N} Q$ for some ${\mathcal N}$-barbed bisimulation ${\mathcal S}_{\mathcal N}$.
\end{definition}

$\mathcal{R} \subseteq \pi \times \pi$

$P \mathcal{R} Q => \forall P'. P \red P' \Rightarrow \exists Q'. Q \red Q', P' \mathcal{R} Q'$

$P \vdash x \Rightarrow Q \vdash x$

\begin{mathpar}
  \inferrule*[lab=Out-barb]{x \nameeq y}{{y}!\langle{Q}\rangle \vdash x}
  \and
  \inferrule*[lab=Par-barb]{\mbox{$P\vdash x$ or $Q\vdash x$}}{\binpar{P}{Q} \vdash x}
\end{mathpar}

\subsubsection{Contexts}

One of the principle advantages of computational calculi like the
$\pi$-calculus is a well-defined notion of context,
contextual-equivalence and a correlation between
contextual-equivalence and notions of bisimulation. The notion of
context allows the decomposition of a process into (sub-)process and
its syntactic environment, its context. Thus, a context may be
thought of as a process with a ``hole'' (written $\Box$) in it. The
application of a context $M$ to a process $P$, written $M[P]$, is
tantamount to filling the hole in $M$ with $P$. In this paper we do
not need the full weight of this theory, but do make use of the notion
of context in the proof the main theorem. 

\begin{mathpar}
  \inferrule* [lab=summation] {} {{M_{M},M_{N}} \bc \Box \;|\; x.M_{A} \;|\; M_{M}+M_{N}}
  \and
  \inferrule* [lab=agent] {} {{M_{A}} \bc (\vec{x})M_{P} \;| \; \clift{P_0,\ldots,M_{P},\ldots,P_N}}
  \and \\
  \inferrule* [lab=process] {} {{M_{P}} \bc M_{N} \;| \;P|M_{P} }
\end{mathpar} 

\begin{mathpar}
  \inferrule* [lab=sychronization] {} {M_{N} \bc \Box \;|\; x?M_{F} \;|\; x!M_{C}}
  \and
  \inferrule* [lab=abstraction] {} {{M_{F}} \bc (x)M_{P} }
  \and
  \inferrule* [lab=concretion] {} {{M_{C}} \bc \langle M_{P} \rangle }
  \and \\
  \inferrule* [lab=process] {} {{M_{P}} \bc M_{N} \;| \;P|M_{P} }
\end{mathpar}

\begin{definition}[contextual application] Given a context $M$, and
  process $P$, we define the \emph{contextual application}, $M[P] :=
  M\{P/\Box\}$. That is, the contextual application of M to P is the
  substitution of $P$ for $\Box$ in $M$.
\end{definition}

$\meaningof{-} : L \to \mathcal{P}(\pi)$

\begin{mathpar}
  \inferrule* [lab=collection] {} {\meaningof{true} = \pi, \and \meaningof{~E} = \pi \setminus \meaningof{E}, \and \meaningof{E_{1} \& E_{2}} = \meaningof{E_{1}} \cap \meaningof{E_{2}}}
\end{mathpar}

\begin{mathpar}
  \inferrule* [lab=structure] {} {\meaningof{0} = \{ P \in \pi | P \equiv 0 \}, \and \\ \meaningof{E_1 | E_2} = \{ P \in \pi | P \equiv P_{1} | P_{2}, P_{1} \in \meaningof{E_{1}}, P_{2} \in \meaningof{E_2}\} }
\end{mathpar}

\begin{mathpar}
 \inferrule* [lab=behavior] {} {\meaningof{\langle a?b \rangle E} = \{ P \in \pi | P \equiv Q | u?(y)P', \\ \and \\\\ \and \\ \;\;\; u \in \meaningof{a}, \forall z.P'\{z/y\} \in \meaningof{E\{z/b\}}\}, \and \\ \meaningof{a!E} = \{ P \in \pi | P \equiv Q | x!\langle P' \rangle, x \in \meaningof{a} P' \in \meaningof{E}\} }
\end{mathpar}

\begin{mathpar}
 \inferrule* [lab=nominal] {} {\meaningof{\quotep{E}} = \{ \quotep{P} \in \quotep{\pi} | P \in \meaningof{E} \}, \and \meaningof{\quotep{P}} = \{ \quotep{Q} \in \quotep{\pi} | P \equiv Q \} \and \\ \meaningof{@\quotep{E}} = \{ P \in \pi | P \equiv @x, x \in \meaningof{E} \}}
\end{mathpar}

\begin{eqnarray*}
  \\
  \meaningof{-} : TS \to ST
\end{eqnarray*}

\begin{eqnarray*}
  \\
  L : TS \to ST
\end{eqnarray*}

\begin{eqnarray*}
  \\
  P \models E \iff P \in \meaningof{E}
\end{eqnarray*}

\begin{eqnarray*}
  P \approx_{L} Q \iff \forall E \in L. P \models E \iff Q \models E
\end{eqnarray*}

\begin{eqnarray*}
  P \approx_{K} Q
\end{eqnarray*}

\begin{eqnarray*}
  P \approx Q
\end{eqnarray*}

$\approx_{K} = \approx = \approx_{L}$

\subsubsection{Contextual duality}

Note that contexts extend the quotation operation to a family of
operations from processes to names. Given a context, $M$, we can
define a \emph{nominal context}, $\quotep{M}$ by $\quotep{M}[P] :=
\quotep{M[P]}$. To foreshadow what is to come we observe that these
operations enjoy a duality with processes very much like the duality
between vectors and maps from vectors to scalars.

Further, because the calculus is essentially higher-order, we have a
correspondence between contexts and processes. More specifically,
given a name $x$ and a context $M$ we can construct $M^{*}_{x}$ such
that 

\begin{mathpar}
  M^{*}_{x} | \lift{x}{P} \red M[P]
\end{mathpar}

namely,

\begin{mathpar}
  M^{*}_{x} := x?(u).M[\dropn{u}]
\end{mathpar}

The dependence of $M^{*}_{x}$ on a name makes it an abstraction, 

\begin{mathpar}
  M^{*} := (x)x?(u).M[\dropn{u}]
\end{mathpar}

\subsection{Additional notation}

It will sometimes be convenient to denote the process a name
quotes. We already have the notation $x = \quotep{P}$, but it will be
convenient to introduce an alternate notation, $\procn{x}$, when we
want to emphasize the connection to the use of the name. Note that, by
virtue of name equivalence, $\quotep{\procn{x}} \nameeq x$; so, the
notation is consistent with previous definitions.

Further, because names have structure it is possible to effect
substitutions on the basis of that structure. This means we need to
upgrade our notation for substitutions, which we accomplish by
adapting comprehension notation. Thus,

\begin{mathpar}
  P\{ y / x : x \in S \}
\end{mathpar}

is interpreted to mean the process derived from P by replacing (in a
capture-avoiding manner) each occurrence of $x$ in $S$ by $y$. For example,

\begin{mathpar}
  P\{ \quotep{\procn{x}|\procn{x}} / x : x \in \freenames{P} \}
\end{mathpar}

will replace each (occurrence) of a free name $x$ in $P$ by
$\quotep{\procn{x}|\procn{x}}$.

Also, we will avail ourselves of the notation $x^{L}$ and $x^{R}$ to
denote injections of a name into disjoint copies of the name
space. There are numerous ways to accomplish this. One example can be
found in \cite{MeredithR05}. This notation overloads to vectors of
names: $\vec{x}^{\pi} := (x_{i}^{\pi} \; : \; 0 \leq i < |\vec{x}| )$ where $\pi \in \{L,R\}$.

We also use $P^{\Box} := P|\Box$.

In \cite{MeredithR05} an interpretation of the new operator is
given. It turns out that there are several possible interpretations
all enjoying the requisite algebraic properties of the operator (see
\cite{milner91polyadicpi}). We will therefore make liberal use of
$(\nu\; \vec{x})P$.

% subsection the_syntax_and_semantics_of_the_notation_system (end)   

\input{qm2pi.qmops} 

\input{qm2pi.sterngerlach} 

\input{qm2pi.metric} 

% section concurrent_process_calculi (end)

%\input{qm2pi.proofsketch}

% section proof sketch (end)

%\input{qm2pi.slviaknots} 

% section spatial logic via knots (end)

\input{qm2pi.conclusion}

% section conclusion (end)

%\input{qm2pi.dtcodes} 

% section wiring algorithm (end)

\input{qm2pi.ack} 

% section acknowledgments (end)

\newpage


\bibliographystyle{plain}   
\bibliography{../../biblios/main.bib}

\input{qm2pi.rhodetails}

\end{document}

 

% section acknowledgments (end)

\newpage


\bibliographystyle{plain}   
\bibliography{../../biblios/main.bib}

\documentclass[12pt]{llncs}
%\documentclass{jktr}

\usepackage[pdftex]{hyperref}                   
\usepackage {listings}
\usepackage {mathpartir}
\usepackage{bcprules}
%\usepackage{listings}
                       
\usepackage{graphicx} 
%\usepackage[margins=2.5cm,nohead,nofoot]{geometry}
%\usepackage{geometry}
\usepackage{amsfonts}
\usepackage{amstext}
\usepackage{latexsym}
\usepackage{amssymb}
\usepackage{color}


%\include{myPreamble}
\include{qm2pi.local} 

%\ifpdf
%\usepackage[pdftex]{graphicx}
%\else
%\usepackage{graphicx}
%\fi

 % \ifpdf
%  \usepackage{pdfsync}
%  \if


%\title{Brief Article}
%\author{David F. Snyder}
%\author{L.G. Meredith}

%\address{Dept. of Math., Texas State University--San Marcos, San Marcos, TX 78666}
       
\pagestyle{empty}


\begin{document}

\lstset{language=[Objective]Caml,frame=shadowbox}

\input{qm2pi.front}

% section front matter (end)

\input{qm2pi.intro} 
 
% section introduction (end)

% \input{qm2pi.knotations} 

% section notation (end)

\input{qm2pi.process.calculi} 

% section concurrent_process_calculi_and_spatial_logics_ (end)
    
%\input{qm2pi.knots2pi} 

%\input{qm2pi.trefoil} 

%\input{qm2pi.mainthm} 

% subsection basic_interpretation (end)

%\input{qm2pi.rho.presentation} 
\subsection{The syntax and semantics of the notation system}\label{sub:the_syntax_and_semantics_of_the_notation_system} % (fold)

We now summarize a technical presentation of the calculus that
embodies our theory of dynamics. The typical presentation of such a
calculus follows the style of giving generators and relations on
them. The grammar, below, describing term constructors, freely
generates the set of processes, $\Proc$. This set is then quotiented
by a relation known as structural congruence and it is over this set
that the notion of dynamics is expressed. This presentation is
essentially that of \cite{MeredithR05} with the addition of
polyadicity and summation. For readability we have relegated some of
the technical subtleties to an appendix.

\subsubsection{Process grammar}\label{subsub:process_grammar}

\begin{mathpar}
  \inferrule* [lab=synchronization] {} {{M} \bc \pzero \;|\; x?F \;|\; x!C }
  \and
  \inferrule* [lab=abstraction] {} {{F} \bc (x)P}
  \and
  \inferrule* [lab=concretion] {} {{C} \bc \langle Q \rangle}
  \and
  \inferrule* [lab=process] {} {{P,Q} \bc M \;| \;P|Q \;|\; @{x}}
  \and
  \inferrule* [lab=name] {} {{x} \bc \quotep{P}}
\end{mathpar} 

Note that $\vec{x}$ (resp. $\vec{P}$) denotes a vector of names
(resp. processes) of length $|\vec{x}|$ (resp. $|\vec{P}|$). We adopt
the following useful abbreviations.

\begin{mathpar}
   x?(\vec{y}).P := x.(\vec{y})P \and  x\clift{\vec{P}} := x.\clift{\vec{P}}
   \and x!(y) := \lift{x}{\dropn{y}}
   \and \Pi_{i=0}^{n-1}P_i := P_0 | \ldots | P_{n-1}
\end{mathpar}

\subsubsection{Structural congruence}

\paragraph{Free and bound names and alpha-equivalence.} At the
core of structural equivalence is alpha-equivalence which identifies
process that are the same up to a change of variable. Formally, we
recognize the distinction between free and bound names. The free names
of a process, $\freenames{P}$, may be calculated recursively as
follows:

\begin{mathpar}
\freenames{\pzero} := \emptyset
  \and \\
  \freenames{x?(y).P} := \{ x \} \cup (\freenames{P} \setminus \{ y \})
  \and 
  \freenames{x!\langle P \rangle} := \{ x \} \cup \{ P \} 
  \and \\
  \freenames{P|Q} := \freenames{P} \cup \freenames{Q}
  \and \\
  \freenames{@{x}} := \{ x \}
\end{mathpar}

$\pi$
$\quotep{\pi}$

$\freenames{-} : \pi \to \mathcal{P}(\quotep{\pi})$

\begin{eqnarray*}
  \freenames{\pzero} & := & \emptyset \\
  \freenames{x?(y).P} & := & \{ x \} \cup (\freenames{P} \setminus \{ y \}) \\
  \freenames{x!\langle P \rangle} & := & \{ x \} \cup \{ P \} \\
  \freenames{P|Q} & := & \freenames{P} \cup \freenames{Q} \\
  \freenames{\dropn{x}} & := & \{ x \}
\end{eqnarray*}

The bound names of a process, $\boundnames{P}$, are those names occurring in $P$
that are not free. For example, in $x?(y).0$, the name $x$ is free, while $y$ is bound.

\begin{mathpar}
  \inferrule* [lab=monoidal-laws] {} { P|Q \equiv Q|P \and P|0 \equiv P \and P|(Q|R) \equiv (P|Q)|R }
\end{mathpar}

\begin{mathpar}
  \inferrule* [lab=alpha-equivalence] {} { (x)P \equiv (y)P\{y/x\} \and y \not\in \freenames{P} }
\end{mathpar}

\begin{definition}
Then two processes, $P,Q$, are alpha-equivalent if $P = Q\{\vec{y}/\vec{x}\}$ for
some $\vec{x} \in \boundnames{Q},\vec{y} \in \boundnames{P}$, where $Q\{\vec{y}/\vec{x}\}$
denotes the capture-avoiding substitution of $\vec{y}$ for $\vec{x}$ in $Q$.
\end{definition}

\begin{definition}
  The {\em structural congruence} \cite{SangiorgiWalker} , $\equiv$,
  between processes is the least congruence containing
  alpha-equivalence, satisfying the abelian monoid laws
  (associativity, commutativity and $\pzero$ as identity) for parallel
  composition $|$ and for summation $+$.
\end{definition}

\subsection{Name equivalence}

We take name equivalence, written $\nameeq$, to be the smallest
equivalence relation generated by the following rules.

\begin{mathpar}
\inferrule*[lab=Quote-drop]
{ }
{ \quotep{@{x}} \nameeq x }

\inferrule*[lab=Struct-equiv]
{ P \scong Q }
{ \quotep{P} \nameeq \quotep{Q} }
\end{mathpar}

The astute reader will have noticed that the mutual recursion of names
and processes imposes a mutual recursion on alpha-equivalence and
structural equivalence via name-equivalence. Fortunately, all of this
works out pleasantly and we may calculate in the natural way, free of
concern. The reader interested in the details is referred to the
appendix \ref{appendix:rho_details}.

\subsection{Substitution}

We use $\Proc$ for the set of processes, $\QProc$ for the set of
names, and $\id{\{}\vec{y} / \vec{x} \id{\}}$ to denote partial maps,
$s : \QProc \rightarrow \QProc$. A map, $s$ lifts, uniquely, to a map
on process terms, $\widehat{s} : \Proc \rightarrow \Proc$ by the
following equations.

\begin{mathpar}
  (0) \psubstp{Q}{P} := 0 \\
  (R \juxtap S) \psubstp{Q}{P}
  :=    
  (R)\psubstp{Q}{P} \juxtap (S) \psubstp{Q}{P} \\
  (x?(y).R) \psubstp{Q}{P}    
  :=    
  (x)\substp{Q}{P} (z)\concat( (R \psubstn{z}{y}) \psubstp{Q}{P} ) \\
  (\lift{x}{R}) \psubstp{Q}{P}  
  :=
  \lift{(x)\substp{Q}{P}}{ R \psubstp{Q}{P} } \\
%   (\dropn{x})  \psubstp{Q}{P}       
%   := 
%   \left\{ 
%     \begin{array}{ccc} 
%       \dropn{\quotep{Q}} & & x \nameeq \quotep{P} \\
%       \dropn{x} & & otherwise \\
%     \end{array}
%   \right. 
  (\dropn{x})  \psubstp{Q}{P}       
  := 
  \left\{ 
    \begin{array}{ccc} 
      Q & & x \nameeq \quotep{P} \\
      \dropn{x} & & otherwise \\
    \end{array}
  \right.
\end{mathpar}
 

where

\begin{eqnarray}
  (x)\id{\{} \lpquote Q \rpquote / \lpquote P \rpquote \id{\}}            = 
  \left\{ 
    \begin{array}{ccc}
      \lpquote Q \rpquote & & x \nameeq \lpquote P \rpquote \\
      x & & otherwise \\
    \end{array}
  \right. \nonumber
\end{eqnarray}

and $z$ is chosen distinct from $\quotep{P}$, $\quotep{Q}$, the free
names in $Q$, and all the names in $R$. Our $\alpha$-equivalence will
be built in the standard way from this substitution.

\begin{remark}\label{rem:no_self_referential_names}
  One consequence of these definitions is that $\forall P. \quotep{P}
  \not\in \freenames{P}$.
\end{remark}

\subsection{ Dynamic quote: an example }

Anticipating something of what's to come, consider applying the
substitution, $\widehat{\id{\{}u / z \id{\}}}$, to the following pair
of processes, $\lift{w}{y!(z)}$ and $w[ \lpquote y!(z) \rpquote ]$.

\begin{eqnarray}
	\lift{w}{y!(z)}\widehat{\id{\{}u / z \id{\}}}
		& = &
		\lift{w}{y!(u)} \nonumber\\
	w[ \lpquote y!(z) \rpquote ] \widehat{ \id{\{}u / z \id{\}} }
		& = &
		w[ \lpquote y!(z) \rpquote ] \nonumber
\end{eqnarray}

Because the body of the process between quotes is impervious to
substitution, we get radically different answers. In fact, by
examining the first process in an input context,
e.g. $x?(z).\lift{w}{y!(z)}$, we see that the process under the lift
operator may be shaped by prefixed inputs binding a name inside it. In
this sense, the lift operator will be seen as a way to dynamically
construct processes before reifying them as names.

Finally equipped with these standard features we can present the
dynamics of the calculus.

\subsubsection{Operational semantics} 

Finally, we introduce the computational dynamics. What marks these
algebras as distinct from other more traditionally studied algebraic
structures, e.g. vector spaces or polynomial rings, is the manner in
which dynamics is captured. In traditional structures, dynamics is typically
expressed through morphisms between such structures, as in linear maps
between vector spaces or morphisms between rings. In algebras
associated with the semantics of computation, the dynamics is
expressed as part of the algebraic structure itself, through a
reduction reduction relation typically denoted by $\red$. Below, we
give a recursive presentation of this relation for the calculus used
in the encoding.

$\red \subseteq \pi \times \pi$
$\red : \pi \to \mathcal{P}(\pi)$

\begin{mathpar}
  \inferrule* [lab=Comm] { \textsf{match}( x_{src}, x_{trgt} ) } { x_{trgt}?(y)P \; | \; x_{src}!\langle {Q} \rangle \red P\{\quotep{Q}/y}\} }
  \and \\
  \inferrule* [lab=Par] {{P} \red {P}'} {{{P} | {Q}} \red {{P}' | {Q}}}
  \and
  \inferrule* [lab=Equiv]{{{P} \scong {P}'} \andalso {{P}' \red {Q}'} \andalso {{Q}' \scong {Q}}}{{P} \red {Q}}
\end{mathpar}

\begin{eqnarray*}
  match_{\equiv} (\quotep{P},\quotep{Q}) & := & P \equiv Q \\
  match_{\dagger}(\quotep{P},\quotep{Q}) & := & \forall R. P|Q \red^{*} R => R \red^{*} 0 \\
  match_{K}(\quotep{P},\quotep{Q}) & := & K \mbox{ for some context } K
\end{eqnarray*}

$u?(x)P | u!\langle Q \rangle \red P\{\quotep{Q}/x\}$

%We write $\wred$ for $\red^*$, and $P\red$ if $\exists Q $ such that $ P \red Q$.
We write $P\red$ if $\exists Q $ such that $ P \red Q$ and $P\not\red$, otherwise.

\section{Replication}

As mentioned before, it is known that replication (and hence
recursion) can be implemented in a higher-order process algebra
\cite{SangiorgiWalker}. As our first example of calculation with the
machinery thus far presented we give the construction explicitly in
the {\rhoc}.

\begin{eqnarray}
	D_{x} & := & \prefix{x}{y}{(\binpar{\outputp{x}{y}}{@{y}})} \nonumber\\
	\bangp_{x}{P} & := & \binpar{{x}!\langle{\binpar{D_{x}}{P}}\rangle}{D_{x}} \nonumber
\end{eqnarray}

\begin{eqnarray}
	\bangp_{x}{P} & & \nonumber\\
	=
	& {x}!\langle{(\prefix{x}{y}{(\outputp{x}{y} | @{y})) | P}}\rangle 
	      | \prefix{x}{y}{(\outputp{x}{y} | @{y})} & \nonumber\\
	\red
	& (\outputp{x}{y} | @{y})\substn{\quotep{(\prefix{x}{y}{(@{y} | \outputp{x}{y})) | P}}}{y} & \nonumber\\
	=
	& \outputp{x}{\quotep{(\prefix{x}{y}{(\outputp{x}{y} | @{y})) | P}}}
	  | {(\prefix{x}{y}{(\outputp{x}{y} | @{y})) | P}} & \nonumber\\
	\red
	& \ldots & \nonumber\\
	\red^*
	& P | P | \ldots & \nonumber
\end{eqnarray}

Of course, this encoding, as an implementation, runs away, unfolding
$\bangp{P}$ eagerly. A lazier and more implementable replication
operator, restricted to input-guarded processes, may be obtained as follows.

\begin{eqnarray}
\bangp{\prefix{u}{v}{P}} 
	:= 
	\binpar{\lift{x}{\prefix{u}{v}{(\binpar{D(x)}{P})}}}{D(x)} \nonumber
\end{eqnarray}

\begin{remark}
  Note that the lazier definition still does not deal with summation
  or mixed summation (i.e. sums over input and output). The reader is
  invited to construct definitions of replication that deal with these
  features. 

  Further, the definitions are parameterized in a name, $x$. Can you,
  gentle reader, make a definition that eliminates this parameter and
  guarantees no accidental interaction between the replication
  machinery and the process being replicated -- i.e. no accidental
  sharing of names used by the process to get its work done and the
  name(s) used by the replication to effect copying. This latter
  revision of the definition of replication is crucial to obtaining
  the expected identity $!!P \sim !P$.
\end{remark}

\begin{remark}\label{rem:paradoxical_combinator}
  The reader familiar with the lambda calculus will have noticed the
  similarity between $D$ and the paradoxical combinator.

  [Ed. note: the existence of this seems to suggest we have to be more
  restrictive on the set of processes and names we admit if we are to
  support no-cloning.]
\end{remark}

\subsubsection{Bisimulation}

The computational dynamics gives rise to another kind of equivalence,
the equivalence of computational behavior. As previously mentioned
this is typically captured \emph{via} some form of bisimulation.

% The notion we use in this paper is weak barbed bisimulation
% \cite{milner91polyadicpi}.

The notion we use in this paper is derived from weak barbed
bisimulation \cite{milner91polyadicpi}. 

\begin{definition}
An \emph{observation relation}, $\downarrow_{\mathcal N}$, over a set
of names, $\mathcal N$, is the smallest relation satisfying the rules
below.

\infrule[Out-barb]{y \in {\mathcal N}, \; x \nameeq y}
		  {\outputp{x}{v} \downarrow_{\mathcal N} x}
\infrule[Par-barb]{\mbox{$P\downarrow_{\mathcal N} x$ or $Q\downarrow_{\mathcal N} x$}}
		  {\binpar{P}{Q} \downarrow_{\mathcal N} x}

We write $P \Downarrow_{\mathcal N} x$ if there is $Q$ such that 
$P \wred Q$ and $Q \downarrow_{\mathcal N} x$.
\end{definition}

\begin{definition}
%\label{def.bbisim}
An  ${\mathcal N}$-\emph{barbed bisimulation} over a set of names, ${\mathcal N}$, is a symmetric binary relation 
${\mathcal S}_{\mathcal N}$ between agents such that $P\rel{S}_{\mathcal N}Q$ implies:
\begin{enumerate}
\item If $P \red P'$ then $Q \wred Q'$ and $P'\rel{S}_{\mathcal N} Q'$.
\item If $P\downarrow_{\mathcal N} x$, then $Q\Downarrow_{\mathcal N} x$.
\end{enumerate}
$P$ is ${\mathcal N}$-barbed bisimilar to $Q$, written
$P \wbbisim_{\mathcal N} Q$, if $P \rel{S}_{\mathcal N} Q$ for some ${\mathcal N}$-barbed bisimulation ${\mathcal S}_{\mathcal N}$.
\end{definition}

$\mathcal{R} \subseteq \pi \times \pi$

$P \mathcal{R} Q => \forall P'. P \red P' \Rightarrow \exists Q'. Q \red Q', P' \mathcal{R} Q'$

$P \vdash x \Rightarrow Q \vdash x$

\begin{mathpar}
  \inferrule*[lab=Out-barb]{x \nameeq y}{{y}!\langle{Q}\rangle \vdash x}
  \and
  \inferrule*[lab=Par-barb]{\mbox{$P\vdash x$ or $Q\vdash x$}}{\binpar{P}{Q} \vdash x}
\end{mathpar}

\subsubsection{Contexts}

One of the principle advantages of computational calculi like the
$\pi$-calculus is a well-defined notion of context,
contextual-equivalence and a correlation between
contextual-equivalence and notions of bisimulation. The notion of
context allows the decomposition of a process into (sub-)process and
its syntactic environment, its context. Thus, a context may be
thought of as a process with a ``hole'' (written $\Box$) in it. The
application of a context $M$ to a process $P$, written $M[P]$, is
tantamount to filling the hole in $M$ with $P$. In this paper we do
not need the full weight of this theory, but do make use of the notion
of context in the proof the main theorem. 

\begin{mathpar}
  \inferrule* [lab=summation] {} {{M_{M},M_{N}} \bc \Box \;|\; x.M_{A} \;|\; M_{M}+M_{N}}
  \and
  \inferrule* [lab=agent] {} {{M_{A}} \bc (\vec{x})M_{P} \;| \; \clift{P_0,\ldots,M_{P},\ldots,P_N}}
  \and \\
  \inferrule* [lab=process] {} {{M_{P}} \bc M_{N} \;| \;P|M_{P} }
\end{mathpar} 

\begin{mathpar}
  \inferrule* [lab=sychronization] {} {M_{N} \bc \Box \;|\; x?M_{F} \;|\; x!M_{C}}
  \and
  \inferrule* [lab=abstraction] {} {{M_{F}} \bc (x)M_{P} }
  \and
  \inferrule* [lab=concretion] {} {{M_{C}} \bc \langle M_{P} \rangle }
  \and \\
  \inferrule* [lab=process] {} {{M_{P}} \bc M_{N} \;| \;P|M_{P} }
\end{mathpar}

\begin{definition}[contextual application] Given a context $M$, and
  process $P$, we define the \emph{contextual application}, $M[P] :=
  M\{P/\Box\}$. That is, the contextual application of M to P is the
  substitution of $P$ for $\Box$ in $M$.
\end{definition}

$\meaningof{-} : L \to \mathcal{P}(\pi)$

\begin{mathpar}
  \inferrule* [lab=collection] {} {\meaningof{true} = \pi, \and \meaningof{~E} = \pi \setminus \meaningof{E}, \and \meaningof{E_{1} \& E_{2}} = \meaningof{E_{1}} \cap \meaningof{E_{2}}}
\end{mathpar}

\begin{mathpar}
  \inferrule* [lab=structure] {} {\meaningof{0} = \{ P \in \pi | P \equiv 0 \}, \and \\ \meaningof{E_1 | E_2} = \{ P \in \pi | P \equiv P_{1} | P_{2}, P_{1} \in \meaningof{E_{1}}, P_{2} \in \meaningof{E_2}\} }
\end{mathpar}

\begin{mathpar}
 \inferrule* [lab=behavior] {} {\meaningof{\langle a?b \rangle E} = \{ P \in \pi | P \equiv Q | u?(y)P', \\ \and \\\\ \and \\ \;\;\; u \in \meaningof{a}, \forall z.P'\{z/y\} \in \meaningof{E\{z/b\}}\}, \and \\ \meaningof{a!E} = \{ P \in \pi | P \equiv Q | x!\langle P' \rangle, x \in \meaningof{a} P' \in \meaningof{E}\} }
\end{mathpar}

\begin{mathpar}
 \inferrule* [lab=nominal] {} {\meaningof{\quotep{E}} = \{ \quotep{P} \in \quotep{\pi} | P \in \meaningof{E} \}, \and \meaningof{\quotep{P}} = \{ \quotep{Q} \in \quotep{\pi} | P \equiv Q \} \and \\ \meaningof{@\quotep{E}} = \{ P \in \pi | P \equiv @x, x \in \meaningof{E} \}}
\end{mathpar}

\begin{eqnarray*}
  \\
  \meaningof{-} : TS \to ST
\end{eqnarray*}

\begin{eqnarray*}
  \\
  L : TS \to ST
\end{eqnarray*}

\begin{eqnarray*}
  \\
  P \models E \iff P \in \meaningof{E}
\end{eqnarray*}

\begin{eqnarray*}
  P \approx_{L} Q \iff \forall E \in L. P \models E \iff Q \models E
\end{eqnarray*}

\begin{eqnarray*}
  P \approx_{K} Q
\end{eqnarray*}

\begin{eqnarray*}
  P \approx Q
\end{eqnarray*}

$\approx_{K} = \approx = \approx_{L}$

\subsubsection{Contextual duality}

Note that contexts extend the quotation operation to a family of
operations from processes to names. Given a context, $M$, we can
define a \emph{nominal context}, $\quotep{M}$ by $\quotep{M}[P] :=
\quotep{M[P]}$. To foreshadow what is to come we observe that these
operations enjoy a duality with processes very much like the duality
between vectors and maps from vectors to scalars.

Further, because the calculus is essentially higher-order, we have a
correspondence between contexts and processes. More specifically,
given a name $x$ and a context $M$ we can construct $M^{*}_{x}$ such
that 

\begin{mathpar}
  M^{*}_{x} | \lift{x}{P} \red M[P]
\end{mathpar}

namely,

\begin{mathpar}
  M^{*}_{x} := x?(u).M[\dropn{u}]
\end{mathpar}

The dependence of $M^{*}_{x}$ on a name makes it an abstraction, 

\begin{mathpar}
  M^{*} := (x)x?(u).M[\dropn{u}]
\end{mathpar}

\subsection{Additional notation}

It will sometimes be convenient to denote the process a name
quotes. We already have the notation $x = \quotep{P}$, but it will be
convenient to introduce an alternate notation, $\procn{x}$, when we
want to emphasize the connection to the use of the name. Note that, by
virtue of name equivalence, $\quotep{\procn{x}} \nameeq x$; so, the
notation is consistent with previous definitions.

Further, because names have structure it is possible to effect
substitutions on the basis of that structure. This means we need to
upgrade our notation for substitutions, which we accomplish by
adapting comprehension notation. Thus,

\begin{mathpar}
  P\{ y / x : x \in S \}
\end{mathpar}

is interpreted to mean the process derived from P by replacing (in a
capture-avoiding manner) each occurrence of $x$ in $S$ by $y$. For example,

\begin{mathpar}
  P\{ \quotep{\procn{x}|\procn{x}} / x : x \in \freenames{P} \}
\end{mathpar}

will replace each (occurrence) of a free name $x$ in $P$ by
$\quotep{\procn{x}|\procn{x}}$.

Also, we will avail ourselves of the notation $x^{L}$ and $x^{R}$ to
denote injections of a name into disjoint copies of the name
space. There are numerous ways to accomplish this. One example can be
found in \cite{MeredithR05}. This notation overloads to vectors of
names: $\vec{x}^{\pi} := (x_{i}^{\pi} \; : \; 0 \leq i < |\vec{x}| )$ where $\pi \in \{L,R\}$.

We also use $P^{\Box} := P|\Box$.

In \cite{MeredithR05} an interpretation of the new operator is
given. It turns out that there are several possible interpretations
all enjoying the requisite algebraic properties of the operator (see
\cite{milner91polyadicpi}). We will therefore make liberal use of
$(\nu\; \vec{x})P$.

% subsection the_syntax_and_semantics_of_the_notation_system (end)   

\input{qm2pi.qmops} 

\input{qm2pi.sterngerlach} 

\input{qm2pi.metric} 

% section concurrent_process_calculi (end)

%\input{qm2pi.proofsketch}

% section proof sketch (end)

%\input{qm2pi.slviaknots} 

% section spatial logic via knots (end)

\input{qm2pi.conclusion}

% section conclusion (end)

%\input{qm2pi.dtcodes} 

% section wiring algorithm (end)

\input{qm2pi.ack} 

% section acknowledgments (end)

\newpage


\bibliographystyle{plain}   
\bibliography{../../biblios/main.bib}

\input{qm2pi.rhodetails}

\end{document}



\end{document}

 

%\documentclass[12pt]{llncs}
%\documentclass{jktr}

\usepackage[pdftex]{hyperref}                   
\usepackage {listings}
\usepackage {mathpartir}
\usepackage{bcprules}
%\usepackage{listings}
                       
\usepackage{graphicx} 
%\usepackage[margins=2.5cm,nohead,nofoot]{geometry}
%\usepackage{geometry}
\usepackage{amsfonts}
\usepackage{amstext}
\usepackage{latexsym}
\usepackage{amssymb}
\usepackage{color}


%\include{myPreamble}
\documentclass[12pt]{llncs}
%\documentclass{jktr}

\usepackage[pdftex]{hyperref}                   
\usepackage {listings}
\usepackage {mathpartir}
\usepackage{bcprules}
%\usepackage{listings}
                       
\usepackage{graphicx} 
%\usepackage[margins=2.5cm,nohead,nofoot]{geometry}
%\usepackage{geometry}
\usepackage{amsfonts}
\usepackage{amstext}
\usepackage{latexsym}
\usepackage{amssymb}
\usepackage{color}


%\include{myPreamble}
\include{qm2pi.local} 

%\ifpdf
%\usepackage[pdftex]{graphicx}
%\else
%\usepackage{graphicx}
%\fi

 % \ifpdf
%  \usepackage{pdfsync}
%  \if


%\title{Brief Article}
%\author{David F. Snyder}
%\author{L.G. Meredith}

%\address{Dept. of Math., Texas State University--San Marcos, San Marcos, TX 78666}
       
\pagestyle{empty}


\begin{document}

\lstset{language=[Objective]Caml,frame=shadowbox}

\input{qm2pi.front}

% section front matter (end)

\input{qm2pi.intro} 
 
% section introduction (end)

% \input{qm2pi.knotations} 

% section notation (end)

\input{qm2pi.process.calculi} 

% section concurrent_process_calculi_and_spatial_logics_ (end)
    
%\input{qm2pi.knots2pi} 

%\input{qm2pi.trefoil} 

%\input{qm2pi.mainthm} 

% subsection basic_interpretation (end)

%\input{qm2pi.rho.presentation} 
\subsection{The syntax and semantics of the notation system}\label{sub:the_syntax_and_semantics_of_the_notation_system} % (fold)

We now summarize a technical presentation of the calculus that
embodies our theory of dynamics. The typical presentation of such a
calculus follows the style of giving generators and relations on
them. The grammar, below, describing term constructors, freely
generates the set of processes, $\Proc$. This set is then quotiented
by a relation known as structural congruence and it is over this set
that the notion of dynamics is expressed. This presentation is
essentially that of \cite{MeredithR05} with the addition of
polyadicity and summation. For readability we have relegated some of
the technical subtleties to an appendix.

\subsubsection{Process grammar}\label{subsub:process_grammar}

\begin{mathpar}
  \inferrule* [lab=synchronization] {} {{M} \bc \pzero \;|\; x?F \;|\; x!C }
  \and
  \inferrule* [lab=abstraction] {} {{F} \bc (x)P}
  \and
  \inferrule* [lab=concretion] {} {{C} \bc \langle Q \rangle}
  \and
  \inferrule* [lab=process] {} {{P,Q} \bc M \;| \;P|Q \;|\; @{x}}
  \and
  \inferrule* [lab=name] {} {{x} \bc \quotep{P}}
\end{mathpar} 

Note that $\vec{x}$ (resp. $\vec{P}$) denotes a vector of names
(resp. processes) of length $|\vec{x}|$ (resp. $|\vec{P}|$). We adopt
the following useful abbreviations.

\begin{mathpar}
   x?(\vec{y}).P := x.(\vec{y})P \and  x\clift{\vec{P}} := x.\clift{\vec{P}}
   \and x!(y) := \lift{x}{\dropn{y}}
   \and \Pi_{i=0}^{n-1}P_i := P_0 | \ldots | P_{n-1}
\end{mathpar}

\subsubsection{Structural congruence}

\paragraph{Free and bound names and alpha-equivalence.} At the
core of structural equivalence is alpha-equivalence which identifies
process that are the same up to a change of variable. Formally, we
recognize the distinction between free and bound names. The free names
of a process, $\freenames{P}$, may be calculated recursively as
follows:

\begin{mathpar}
\freenames{\pzero} := \emptyset
  \and \\
  \freenames{x?(y).P} := \{ x \} \cup (\freenames{P} \setminus \{ y \})
  \and 
  \freenames{x!\langle P \rangle} := \{ x \} \cup \{ P \} 
  \and \\
  \freenames{P|Q} := \freenames{P} \cup \freenames{Q}
  \and \\
  \freenames{@{x}} := \{ x \}
\end{mathpar}

$\pi$
$\quotep{\pi}$

$\freenames{-} : \pi \to \mathcal{P}(\quotep{\pi})$

\begin{eqnarray*}
  \freenames{\pzero} & := & \emptyset \\
  \freenames{x?(y).P} & := & \{ x \} \cup (\freenames{P} \setminus \{ y \}) \\
  \freenames{x!\langle P \rangle} & := & \{ x \} \cup \{ P \} \\
  \freenames{P|Q} & := & \freenames{P} \cup \freenames{Q} \\
  \freenames{\dropn{x}} & := & \{ x \}
\end{eqnarray*}

The bound names of a process, $\boundnames{P}$, are those names occurring in $P$
that are not free. For example, in $x?(y).0$, the name $x$ is free, while $y$ is bound.

\begin{mathpar}
  \inferrule* [lab=monoidal-laws] {} { P|Q \equiv Q|P \and P|0 \equiv P \and P|(Q|R) \equiv (P|Q)|R }
\end{mathpar}

\begin{mathpar}
  \inferrule* [lab=alpha-equivalence] {} { (x)P \equiv (y)P\{y/x\} \and y \not\in \freenames{P} }
\end{mathpar}

\begin{definition}
Then two processes, $P,Q$, are alpha-equivalent if $P = Q\{\vec{y}/\vec{x}\}$ for
some $\vec{x} \in \boundnames{Q},\vec{y} \in \boundnames{P}$, where $Q\{\vec{y}/\vec{x}\}$
denotes the capture-avoiding substitution of $\vec{y}$ for $\vec{x}$ in $Q$.
\end{definition}

\begin{definition}
  The {\em structural congruence} \cite{SangiorgiWalker} , $\equiv$,
  between processes is the least congruence containing
  alpha-equivalence, satisfying the abelian monoid laws
  (associativity, commutativity and $\pzero$ as identity) for parallel
  composition $|$ and for summation $+$.
\end{definition}

\subsection{Name equivalence}

We take name equivalence, written $\nameeq$, to be the smallest
equivalence relation generated by the following rules.

\begin{mathpar}
\inferrule*[lab=Quote-drop]
{ }
{ \quotep{@{x}} \nameeq x }

\inferrule*[lab=Struct-equiv]
{ P \scong Q }
{ \quotep{P} \nameeq \quotep{Q} }
\end{mathpar}

The astute reader will have noticed that the mutual recursion of names
and processes imposes a mutual recursion on alpha-equivalence and
structural equivalence via name-equivalence. Fortunately, all of this
works out pleasantly and we may calculate in the natural way, free of
concern. The reader interested in the details is referred to the
appendix \ref{appendix:rho_details}.

\subsection{Substitution}

We use $\Proc$ for the set of processes, $\QProc$ for the set of
names, and $\id{\{}\vec{y} / \vec{x} \id{\}}$ to denote partial maps,
$s : \QProc \rightarrow \QProc$. A map, $s$ lifts, uniquely, to a map
on process terms, $\widehat{s} : \Proc \rightarrow \Proc$ by the
following equations.

\begin{mathpar}
  (0) \psubstp{Q}{P} := 0 \\
  (R \juxtap S) \psubstp{Q}{P}
  :=    
  (R)\psubstp{Q}{P} \juxtap (S) \psubstp{Q}{P} \\
  (x?(y).R) \psubstp{Q}{P}    
  :=    
  (x)\substp{Q}{P} (z)\concat( (R \psubstn{z}{y}) \psubstp{Q}{P} ) \\
  (\lift{x}{R}) \psubstp{Q}{P}  
  :=
  \lift{(x)\substp{Q}{P}}{ R \psubstp{Q}{P} } \\
%   (\dropn{x})  \psubstp{Q}{P}       
%   := 
%   \left\{ 
%     \begin{array}{ccc} 
%       \dropn{\quotep{Q}} & & x \nameeq \quotep{P} \\
%       \dropn{x} & & otherwise \\
%     \end{array}
%   \right. 
  (\dropn{x})  \psubstp{Q}{P}       
  := 
  \left\{ 
    \begin{array}{ccc} 
      Q & & x \nameeq \quotep{P} \\
      \dropn{x} & & otherwise \\
    \end{array}
  \right.
\end{mathpar}
 

where

\begin{eqnarray}
  (x)\id{\{} \lpquote Q \rpquote / \lpquote P \rpquote \id{\}}            = 
  \left\{ 
    \begin{array}{ccc}
      \lpquote Q \rpquote & & x \nameeq \lpquote P \rpquote \\
      x & & otherwise \\
    \end{array}
  \right. \nonumber
\end{eqnarray}

and $z$ is chosen distinct from $\quotep{P}$, $\quotep{Q}$, the free
names in $Q$, and all the names in $R$. Our $\alpha$-equivalence will
be built in the standard way from this substitution.

\begin{remark}\label{rem:no_self_referential_names}
  One consequence of these definitions is that $\forall P. \quotep{P}
  \not\in \freenames{P}$.
\end{remark}

\subsection{ Dynamic quote: an example }

Anticipating something of what's to come, consider applying the
substitution, $\widehat{\id{\{}u / z \id{\}}}$, to the following pair
of processes, $\lift{w}{y!(z)}$ and $w[ \lpquote y!(z) \rpquote ]$.

\begin{eqnarray}
	\lift{w}{y!(z)}\widehat{\id{\{}u / z \id{\}}}
		& = &
		\lift{w}{y!(u)} \nonumber\\
	w[ \lpquote y!(z) \rpquote ] \widehat{ \id{\{}u / z \id{\}} }
		& = &
		w[ \lpquote y!(z) \rpquote ] \nonumber
\end{eqnarray}

Because the body of the process between quotes is impervious to
substitution, we get radically different answers. In fact, by
examining the first process in an input context,
e.g. $x?(z).\lift{w}{y!(z)}$, we see that the process under the lift
operator may be shaped by prefixed inputs binding a name inside it. In
this sense, the lift operator will be seen as a way to dynamically
construct processes before reifying them as names.

Finally equipped with these standard features we can present the
dynamics of the calculus.

\subsubsection{Operational semantics} 

Finally, we introduce the computational dynamics. What marks these
algebras as distinct from other more traditionally studied algebraic
structures, e.g. vector spaces or polynomial rings, is the manner in
which dynamics is captured. In traditional structures, dynamics is typically
expressed through morphisms between such structures, as in linear maps
between vector spaces or morphisms between rings. In algebras
associated with the semantics of computation, the dynamics is
expressed as part of the algebraic structure itself, through a
reduction reduction relation typically denoted by $\red$. Below, we
give a recursive presentation of this relation for the calculus used
in the encoding.

$\red \subseteq \pi \times \pi$
$\red : \pi \to \mathcal{P}(\pi)$

\begin{mathpar}
  \inferrule* [lab=Comm] { \textsf{match}( x_{src}, x_{trgt} ) } { x_{trgt}?(y)P \; | \; x_{src}!\langle {Q} \rangle \red P\{\quotep{Q}/y}\} }
  \and \\
  \inferrule* [lab=Par] {{P} \red {P}'} {{{P} | {Q}} \red {{P}' | {Q}}}
  \and
  \inferrule* [lab=Equiv]{{{P} \scong {P}'} \andalso {{P}' \red {Q}'} \andalso {{Q}' \scong {Q}}}{{P} \red {Q}}
\end{mathpar}

\begin{eqnarray*}
  match_{\equiv} (\quotep{P},\quotep{Q}) & := & P \equiv Q \\
  match_{\dagger}(\quotep{P},\quotep{Q}) & := & \forall R. P|Q \red^{*} R => R \red^{*} 0 \\
  match_{K}(\quotep{P},\quotep{Q}) & := & K \mbox{ for some context } K
\end{eqnarray*}

$u?(x)P | u!\langle Q \rangle \red P\{\quotep{Q}/x\}$

%We write $\wred$ for $\red^*$, and $P\red$ if $\exists Q $ such that $ P \red Q$.
We write $P\red$ if $\exists Q $ such that $ P \red Q$ and $P\not\red$, otherwise.

\section{Replication}

As mentioned before, it is known that replication (and hence
recursion) can be implemented in a higher-order process algebra
\cite{SangiorgiWalker}. As our first example of calculation with the
machinery thus far presented we give the construction explicitly in
the {\rhoc}.

\begin{eqnarray}
	D_{x} & := & \prefix{x}{y}{(\binpar{\outputp{x}{y}}{@{y}})} \nonumber\\
	\bangp_{x}{P} & := & \binpar{{x}!\langle{\binpar{D_{x}}{P}}\rangle}{D_{x}} \nonumber
\end{eqnarray}

\begin{eqnarray}
	\bangp_{x}{P} & & \nonumber\\
	=
	& {x}!\langle{(\prefix{x}{y}{(\outputp{x}{y} | @{y})) | P}}\rangle 
	      | \prefix{x}{y}{(\outputp{x}{y} | @{y})} & \nonumber\\
	\red
	& (\outputp{x}{y} | @{y})\substn{\quotep{(\prefix{x}{y}{(@{y} | \outputp{x}{y})) | P}}}{y} & \nonumber\\
	=
	& \outputp{x}{\quotep{(\prefix{x}{y}{(\outputp{x}{y} | @{y})) | P}}}
	  | {(\prefix{x}{y}{(\outputp{x}{y} | @{y})) | P}} & \nonumber\\
	\red
	& \ldots & \nonumber\\
	\red^*
	& P | P | \ldots & \nonumber
\end{eqnarray}

Of course, this encoding, as an implementation, runs away, unfolding
$\bangp{P}$ eagerly. A lazier and more implementable replication
operator, restricted to input-guarded processes, may be obtained as follows.

\begin{eqnarray}
\bangp{\prefix{u}{v}{P}} 
	:= 
	\binpar{\lift{x}{\prefix{u}{v}{(\binpar{D(x)}{P})}}}{D(x)} \nonumber
\end{eqnarray}

\begin{remark}
  Note that the lazier definition still does not deal with summation
  or mixed summation (i.e. sums over input and output). The reader is
  invited to construct definitions of replication that deal with these
  features. 

  Further, the definitions are parameterized in a name, $x$. Can you,
  gentle reader, make a definition that eliminates this parameter and
  guarantees no accidental interaction between the replication
  machinery and the process being replicated -- i.e. no accidental
  sharing of names used by the process to get its work done and the
  name(s) used by the replication to effect copying. This latter
  revision of the definition of replication is crucial to obtaining
  the expected identity $!!P \sim !P$.
\end{remark}

\begin{remark}\label{rem:paradoxical_combinator}
  The reader familiar with the lambda calculus will have noticed the
  similarity between $D$ and the paradoxical combinator.

  [Ed. note: the existence of this seems to suggest we have to be more
  restrictive on the set of processes and names we admit if we are to
  support no-cloning.]
\end{remark}

\subsubsection{Bisimulation}

The computational dynamics gives rise to another kind of equivalence,
the equivalence of computational behavior. As previously mentioned
this is typically captured \emph{via} some form of bisimulation.

% The notion we use in this paper is weak barbed bisimulation
% \cite{milner91polyadicpi}.

The notion we use in this paper is derived from weak barbed
bisimulation \cite{milner91polyadicpi}. 

\begin{definition}
An \emph{observation relation}, $\downarrow_{\mathcal N}$, over a set
of names, $\mathcal N$, is the smallest relation satisfying the rules
below.

\infrule[Out-barb]{y \in {\mathcal N}, \; x \nameeq y}
		  {\outputp{x}{v} \downarrow_{\mathcal N} x}
\infrule[Par-barb]{\mbox{$P\downarrow_{\mathcal N} x$ or $Q\downarrow_{\mathcal N} x$}}
		  {\binpar{P}{Q} \downarrow_{\mathcal N} x}

We write $P \Downarrow_{\mathcal N} x$ if there is $Q$ such that 
$P \wred Q$ and $Q \downarrow_{\mathcal N} x$.
\end{definition}

\begin{definition}
%\label{def.bbisim}
An  ${\mathcal N}$-\emph{barbed bisimulation} over a set of names, ${\mathcal N}$, is a symmetric binary relation 
${\mathcal S}_{\mathcal N}$ between agents such that $P\rel{S}_{\mathcal N}Q$ implies:
\begin{enumerate}
\item If $P \red P'$ then $Q \wred Q'$ and $P'\rel{S}_{\mathcal N} Q'$.
\item If $P\downarrow_{\mathcal N} x$, then $Q\Downarrow_{\mathcal N} x$.
\end{enumerate}
$P$ is ${\mathcal N}$-barbed bisimilar to $Q$, written
$P \wbbisim_{\mathcal N} Q$, if $P \rel{S}_{\mathcal N} Q$ for some ${\mathcal N}$-barbed bisimulation ${\mathcal S}_{\mathcal N}$.
\end{definition}

$\mathcal{R} \subseteq \pi \times \pi$

$P \mathcal{R} Q => \forall P'. P \red P' \Rightarrow \exists Q'. Q \red Q', P' \mathcal{R} Q'$

$P \vdash x \Rightarrow Q \vdash x$

\begin{mathpar}
  \inferrule*[lab=Out-barb]{x \nameeq y}{{y}!\langle{Q}\rangle \vdash x}
  \and
  \inferrule*[lab=Par-barb]{\mbox{$P\vdash x$ or $Q\vdash x$}}{\binpar{P}{Q} \vdash x}
\end{mathpar}

\subsubsection{Contexts}

One of the principle advantages of computational calculi like the
$\pi$-calculus is a well-defined notion of context,
contextual-equivalence and a correlation between
contextual-equivalence and notions of bisimulation. The notion of
context allows the decomposition of a process into (sub-)process and
its syntactic environment, its context. Thus, a context may be
thought of as a process with a ``hole'' (written $\Box$) in it. The
application of a context $M$ to a process $P$, written $M[P]$, is
tantamount to filling the hole in $M$ with $P$. In this paper we do
not need the full weight of this theory, but do make use of the notion
of context in the proof the main theorem. 

\begin{mathpar}
  \inferrule* [lab=summation] {} {{M_{M},M_{N}} \bc \Box \;|\; x.M_{A} \;|\; M_{M}+M_{N}}
  \and
  \inferrule* [lab=agent] {} {{M_{A}} \bc (\vec{x})M_{P} \;| \; \clift{P_0,\ldots,M_{P},\ldots,P_N}}
  \and \\
  \inferrule* [lab=process] {} {{M_{P}} \bc M_{N} \;| \;P|M_{P} }
\end{mathpar} 

\begin{mathpar}
  \inferrule* [lab=sychronization] {} {M_{N} \bc \Box \;|\; x?M_{F} \;|\; x!M_{C}}
  \and
  \inferrule* [lab=abstraction] {} {{M_{F}} \bc (x)M_{P} }
  \and
  \inferrule* [lab=concretion] {} {{M_{C}} \bc \langle M_{P} \rangle }
  \and \\
  \inferrule* [lab=process] {} {{M_{P}} \bc M_{N} \;| \;P|M_{P} }
\end{mathpar}

\begin{definition}[contextual application] Given a context $M$, and
  process $P$, we define the \emph{contextual application}, $M[P] :=
  M\{P/\Box\}$. That is, the contextual application of M to P is the
  substitution of $P$ for $\Box$ in $M$.
\end{definition}

$\meaningof{-} : L \to \mathcal{P}(\pi)$

\begin{mathpar}
  \inferrule* [lab=collection] {} {\meaningof{true} = \pi, \and \meaningof{~E} = \pi \setminus \meaningof{E}, \and \meaningof{E_{1} \& E_{2}} = \meaningof{E_{1}} \cap \meaningof{E_{2}}}
\end{mathpar}

\begin{mathpar}
  \inferrule* [lab=structure] {} {\meaningof{0} = \{ P \in \pi | P \equiv 0 \}, \and \\ \meaningof{E_1 | E_2} = \{ P \in \pi | P \equiv P_{1} | P_{2}, P_{1} \in \meaningof{E_{1}}, P_{2} \in \meaningof{E_2}\} }
\end{mathpar}

\begin{mathpar}
 \inferrule* [lab=behavior] {} {\meaningof{\langle a?b \rangle E} = \{ P \in \pi | P \equiv Q | u?(y)P', \\ \and \\\\ \and \\ \;\;\; u \in \meaningof{a}, \forall z.P'\{z/y\} \in \meaningof{E\{z/b\}}\}, \and \\ \meaningof{a!E} = \{ P \in \pi | P \equiv Q | x!\langle P' \rangle, x \in \meaningof{a} P' \in \meaningof{E}\} }
\end{mathpar}

\begin{mathpar}
 \inferrule* [lab=nominal] {} {\meaningof{\quotep{E}} = \{ \quotep{P} \in \quotep{\pi} | P \in \meaningof{E} \}, \and \meaningof{\quotep{P}} = \{ \quotep{Q} \in \quotep{\pi} | P \equiv Q \} \and \\ \meaningof{@\quotep{E}} = \{ P \in \pi | P \equiv @x, x \in \meaningof{E} \}}
\end{mathpar}

\begin{eqnarray*}
  \\
  \meaningof{-} : TS \to ST
\end{eqnarray*}

\begin{eqnarray*}
  \\
  L : TS \to ST
\end{eqnarray*}

\begin{eqnarray*}
  \\
  P \models E \iff P \in \meaningof{E}
\end{eqnarray*}

\begin{eqnarray*}
  P \approx_{L} Q \iff \forall E \in L. P \models E \iff Q \models E
\end{eqnarray*}

\begin{eqnarray*}
  P \approx_{K} Q
\end{eqnarray*}

\begin{eqnarray*}
  P \approx Q
\end{eqnarray*}

$\approx_{K} = \approx = \approx_{L}$

\subsubsection{Contextual duality}

Note that contexts extend the quotation operation to a family of
operations from processes to names. Given a context, $M$, we can
define a \emph{nominal context}, $\quotep{M}$ by $\quotep{M}[P] :=
\quotep{M[P]}$. To foreshadow what is to come we observe that these
operations enjoy a duality with processes very much like the duality
between vectors and maps from vectors to scalars.

Further, because the calculus is essentially higher-order, we have a
correspondence between contexts and processes. More specifically,
given a name $x$ and a context $M$ we can construct $M^{*}_{x}$ such
that 

\begin{mathpar}
  M^{*}_{x} | \lift{x}{P} \red M[P]
\end{mathpar}

namely,

\begin{mathpar}
  M^{*}_{x} := x?(u).M[\dropn{u}]
\end{mathpar}

The dependence of $M^{*}_{x}$ on a name makes it an abstraction, 

\begin{mathpar}
  M^{*} := (x)x?(u).M[\dropn{u}]
\end{mathpar}

\subsection{Additional notation}

It will sometimes be convenient to denote the process a name
quotes. We already have the notation $x = \quotep{P}$, but it will be
convenient to introduce an alternate notation, $\procn{x}$, when we
want to emphasize the connection to the use of the name. Note that, by
virtue of name equivalence, $\quotep{\procn{x}} \nameeq x$; so, the
notation is consistent with previous definitions.

Further, because names have structure it is possible to effect
substitutions on the basis of that structure. This means we need to
upgrade our notation for substitutions, which we accomplish by
adapting comprehension notation. Thus,

\begin{mathpar}
  P\{ y / x : x \in S \}
\end{mathpar}

is interpreted to mean the process derived from P by replacing (in a
capture-avoiding manner) each occurrence of $x$ in $S$ by $y$. For example,

\begin{mathpar}
  P\{ \quotep{\procn{x}|\procn{x}} / x : x \in \freenames{P} \}
\end{mathpar}

will replace each (occurrence) of a free name $x$ in $P$ by
$\quotep{\procn{x}|\procn{x}}$.

Also, we will avail ourselves of the notation $x^{L}$ and $x^{R}$ to
denote injections of a name into disjoint copies of the name
space. There are numerous ways to accomplish this. One example can be
found in \cite{MeredithR05}. This notation overloads to vectors of
names: $\vec{x}^{\pi} := (x_{i}^{\pi} \; : \; 0 \leq i < |\vec{x}| )$ where $\pi \in \{L,R\}$.

We also use $P^{\Box} := P|\Box$.

In \cite{MeredithR05} an interpretation of the new operator is
given. It turns out that there are several possible interpretations
all enjoying the requisite algebraic properties of the operator (see
\cite{milner91polyadicpi}). We will therefore make liberal use of
$(\nu\; \vec{x})P$.

% subsection the_syntax_and_semantics_of_the_notation_system (end)   

\input{qm2pi.qmops} 

\input{qm2pi.sterngerlach} 

\input{qm2pi.metric} 

% section concurrent_process_calculi (end)

%\input{qm2pi.proofsketch}

% section proof sketch (end)

%\input{qm2pi.slviaknots} 

% section spatial logic via knots (end)

\input{qm2pi.conclusion}

% section conclusion (end)

%\input{qm2pi.dtcodes} 

% section wiring algorithm (end)

\input{qm2pi.ack} 

% section acknowledgments (end)

\newpage


\bibliographystyle{plain}   
\bibliography{../../biblios/main.bib}

\input{qm2pi.rhodetails}

\end{document}

 

%\ifpdf
%\usepackage[pdftex]{graphicx}
%\else
%\usepackage{graphicx}
%\fi

 % \ifpdf
%  \usepackage{pdfsync}
%  \if


%\title{Brief Article}
%\author{David F. Snyder}
%\author{L.G. Meredith}

%\address{Dept. of Math., Texas State University--San Marcos, San Marcos, TX 78666}
       
\pagestyle{empty}


\begin{document}

\lstset{language=[Objective]Caml,frame=shadowbox}

\documentclass[12pt]{llncs}
%\documentclass{jktr}

\usepackage[pdftex]{hyperref}                   
\usepackage {listings}
\usepackage {mathpartir}
\usepackage{bcprules}
%\usepackage{listings}
                       
\usepackage{graphicx} 
%\usepackage[margins=2.5cm,nohead,nofoot]{geometry}
%\usepackage{geometry}
\usepackage{amsfonts}
\usepackage{amstext}
\usepackage{latexsym}
\usepackage{amssymb}
\usepackage{color}


%\include{myPreamble}
\include{qm2pi.local} 

%\ifpdf
%\usepackage[pdftex]{graphicx}
%\else
%\usepackage{graphicx}
%\fi

 % \ifpdf
%  \usepackage{pdfsync}
%  \if


%\title{Brief Article}
%\author{David F. Snyder}
%\author{L.G. Meredith}

%\address{Dept. of Math., Texas State University--San Marcos, San Marcos, TX 78666}
       
\pagestyle{empty}


\begin{document}

\lstset{language=[Objective]Caml,frame=shadowbox}

\input{qm2pi.front}

% section front matter (end)

\input{qm2pi.intro} 
 
% section introduction (end)

% \input{qm2pi.knotations} 

% section notation (end)

\input{qm2pi.process.calculi} 

% section concurrent_process_calculi_and_spatial_logics_ (end)
    
%\input{qm2pi.knots2pi} 

%\input{qm2pi.trefoil} 

%\input{qm2pi.mainthm} 

% subsection basic_interpretation (end)

%\input{qm2pi.rho.presentation} 
\subsection{The syntax and semantics of the notation system}\label{sub:the_syntax_and_semantics_of_the_notation_system} % (fold)

We now summarize a technical presentation of the calculus that
embodies our theory of dynamics. The typical presentation of such a
calculus follows the style of giving generators and relations on
them. The grammar, below, describing term constructors, freely
generates the set of processes, $\Proc$. This set is then quotiented
by a relation known as structural congruence and it is over this set
that the notion of dynamics is expressed. This presentation is
essentially that of \cite{MeredithR05} with the addition of
polyadicity and summation. For readability we have relegated some of
the technical subtleties to an appendix.

\subsubsection{Process grammar}\label{subsub:process_grammar}

\begin{mathpar}
  \inferrule* [lab=synchronization] {} {{M} \bc \pzero \;|\; x?F \;|\; x!C }
  \and
  \inferrule* [lab=abstraction] {} {{F} \bc (x)P}
  \and
  \inferrule* [lab=concretion] {} {{C} \bc \langle Q \rangle}
  \and
  \inferrule* [lab=process] {} {{P,Q} \bc M \;| \;P|Q \;|\; @{x}}
  \and
  \inferrule* [lab=name] {} {{x} \bc \quotep{P}}
\end{mathpar} 

Note that $\vec{x}$ (resp. $\vec{P}$) denotes a vector of names
(resp. processes) of length $|\vec{x}|$ (resp. $|\vec{P}|$). We adopt
the following useful abbreviations.

\begin{mathpar}
   x?(\vec{y}).P := x.(\vec{y})P \and  x\clift{\vec{P}} := x.\clift{\vec{P}}
   \and x!(y) := \lift{x}{\dropn{y}}
   \and \Pi_{i=0}^{n-1}P_i := P_0 | \ldots | P_{n-1}
\end{mathpar}

\subsubsection{Structural congruence}

\paragraph{Free and bound names and alpha-equivalence.} At the
core of structural equivalence is alpha-equivalence which identifies
process that are the same up to a change of variable. Formally, we
recognize the distinction between free and bound names. The free names
of a process, $\freenames{P}$, may be calculated recursively as
follows:

\begin{mathpar}
\freenames{\pzero} := \emptyset
  \and \\
  \freenames{x?(y).P} := \{ x \} \cup (\freenames{P} \setminus \{ y \})
  \and 
  \freenames{x!\langle P \rangle} := \{ x \} \cup \{ P \} 
  \and \\
  \freenames{P|Q} := \freenames{P} \cup \freenames{Q}
  \and \\
  \freenames{@{x}} := \{ x \}
\end{mathpar}

$\pi$
$\quotep{\pi}$

$\freenames{-} : \pi \to \mathcal{P}(\quotep{\pi})$

\begin{eqnarray*}
  \freenames{\pzero} & := & \emptyset \\
  \freenames{x?(y).P} & := & \{ x \} \cup (\freenames{P} \setminus \{ y \}) \\
  \freenames{x!\langle P \rangle} & := & \{ x \} \cup \{ P \} \\
  \freenames{P|Q} & := & \freenames{P} \cup \freenames{Q} \\
  \freenames{\dropn{x}} & := & \{ x \}
\end{eqnarray*}

The bound names of a process, $\boundnames{P}$, are those names occurring in $P$
that are not free. For example, in $x?(y).0$, the name $x$ is free, while $y$ is bound.

\begin{mathpar}
  \inferrule* [lab=monoidal-laws] {} { P|Q \equiv Q|P \and P|0 \equiv P \and P|(Q|R) \equiv (P|Q)|R }
\end{mathpar}

\begin{mathpar}
  \inferrule* [lab=alpha-equivalence] {} { (x)P \equiv (y)P\{y/x\} \and y \not\in \freenames{P} }
\end{mathpar}

\begin{definition}
Then two processes, $P,Q$, are alpha-equivalent if $P = Q\{\vec{y}/\vec{x}\}$ for
some $\vec{x} \in \boundnames{Q},\vec{y} \in \boundnames{P}$, where $Q\{\vec{y}/\vec{x}\}$
denotes the capture-avoiding substitution of $\vec{y}$ for $\vec{x}$ in $Q$.
\end{definition}

\begin{definition}
  The {\em structural congruence} \cite{SangiorgiWalker} , $\equiv$,
  between processes is the least congruence containing
  alpha-equivalence, satisfying the abelian monoid laws
  (associativity, commutativity and $\pzero$ as identity) for parallel
  composition $|$ and for summation $+$.
\end{definition}

\subsection{Name equivalence}

We take name equivalence, written $\nameeq$, to be the smallest
equivalence relation generated by the following rules.

\begin{mathpar}
\inferrule*[lab=Quote-drop]
{ }
{ \quotep{@{x}} \nameeq x }

\inferrule*[lab=Struct-equiv]
{ P \scong Q }
{ \quotep{P} \nameeq \quotep{Q} }
\end{mathpar}

The astute reader will have noticed that the mutual recursion of names
and processes imposes a mutual recursion on alpha-equivalence and
structural equivalence via name-equivalence. Fortunately, all of this
works out pleasantly and we may calculate in the natural way, free of
concern. The reader interested in the details is referred to the
appendix \ref{appendix:rho_details}.

\subsection{Substitution}

We use $\Proc$ for the set of processes, $\QProc$ for the set of
names, and $\id{\{}\vec{y} / \vec{x} \id{\}}$ to denote partial maps,
$s : \QProc \rightarrow \QProc$. A map, $s$ lifts, uniquely, to a map
on process terms, $\widehat{s} : \Proc \rightarrow \Proc$ by the
following equations.

\begin{mathpar}
  (0) \psubstp{Q}{P} := 0 \\
  (R \juxtap S) \psubstp{Q}{P}
  :=    
  (R)\psubstp{Q}{P} \juxtap (S) \psubstp{Q}{P} \\
  (x?(y).R) \psubstp{Q}{P}    
  :=    
  (x)\substp{Q}{P} (z)\concat( (R \psubstn{z}{y}) \psubstp{Q}{P} ) \\
  (\lift{x}{R}) \psubstp{Q}{P}  
  :=
  \lift{(x)\substp{Q}{P}}{ R \psubstp{Q}{P} } \\
%   (\dropn{x})  \psubstp{Q}{P}       
%   := 
%   \left\{ 
%     \begin{array}{ccc} 
%       \dropn{\quotep{Q}} & & x \nameeq \quotep{P} \\
%       \dropn{x} & & otherwise \\
%     \end{array}
%   \right. 
  (\dropn{x})  \psubstp{Q}{P}       
  := 
  \left\{ 
    \begin{array}{ccc} 
      Q & & x \nameeq \quotep{P} \\
      \dropn{x} & & otherwise \\
    \end{array}
  \right.
\end{mathpar}
 

where

\begin{eqnarray}
  (x)\id{\{} \lpquote Q \rpquote / \lpquote P \rpquote \id{\}}            = 
  \left\{ 
    \begin{array}{ccc}
      \lpquote Q \rpquote & & x \nameeq \lpquote P \rpquote \\
      x & & otherwise \\
    \end{array}
  \right. \nonumber
\end{eqnarray}

and $z$ is chosen distinct from $\quotep{P}$, $\quotep{Q}$, the free
names in $Q$, and all the names in $R$. Our $\alpha$-equivalence will
be built in the standard way from this substitution.

\begin{remark}\label{rem:no_self_referential_names}
  One consequence of these definitions is that $\forall P. \quotep{P}
  \not\in \freenames{P}$.
\end{remark}

\subsection{ Dynamic quote: an example }

Anticipating something of what's to come, consider applying the
substitution, $\widehat{\id{\{}u / z \id{\}}}$, to the following pair
of processes, $\lift{w}{y!(z)}$ and $w[ \lpquote y!(z) \rpquote ]$.

\begin{eqnarray}
	\lift{w}{y!(z)}\widehat{\id{\{}u / z \id{\}}}
		& = &
		\lift{w}{y!(u)} \nonumber\\
	w[ \lpquote y!(z) \rpquote ] \widehat{ \id{\{}u / z \id{\}} }
		& = &
		w[ \lpquote y!(z) \rpquote ] \nonumber
\end{eqnarray}

Because the body of the process between quotes is impervious to
substitution, we get radically different answers. In fact, by
examining the first process in an input context,
e.g. $x?(z).\lift{w}{y!(z)}$, we see that the process under the lift
operator may be shaped by prefixed inputs binding a name inside it. In
this sense, the lift operator will be seen as a way to dynamically
construct processes before reifying them as names.

Finally equipped with these standard features we can present the
dynamics of the calculus.

\subsubsection{Operational semantics} 

Finally, we introduce the computational dynamics. What marks these
algebras as distinct from other more traditionally studied algebraic
structures, e.g. vector spaces or polynomial rings, is the manner in
which dynamics is captured. In traditional structures, dynamics is typically
expressed through morphisms between such structures, as in linear maps
between vector spaces or morphisms between rings. In algebras
associated with the semantics of computation, the dynamics is
expressed as part of the algebraic structure itself, through a
reduction reduction relation typically denoted by $\red$. Below, we
give a recursive presentation of this relation for the calculus used
in the encoding.

$\red \subseteq \pi \times \pi$
$\red : \pi \to \mathcal{P}(\pi)$

\begin{mathpar}
  \inferrule* [lab=Comm] { \textsf{match}( x_{src}, x_{trgt} ) } { x_{trgt}?(y)P \; | \; x_{src}!\langle {Q} \rangle \red P\{\quotep{Q}/y}\} }
  \and \\
  \inferrule* [lab=Par] {{P} \red {P}'} {{{P} | {Q}} \red {{P}' | {Q}}}
  \and
  \inferrule* [lab=Equiv]{{{P} \scong {P}'} \andalso {{P}' \red {Q}'} \andalso {{Q}' \scong {Q}}}{{P} \red {Q}}
\end{mathpar}

\begin{eqnarray*}
  match_{\equiv} (\quotep{P},\quotep{Q}) & := & P \equiv Q \\
  match_{\dagger}(\quotep{P},\quotep{Q}) & := & \forall R. P|Q \red^{*} R => R \red^{*} 0 \\
  match_{K}(\quotep{P},\quotep{Q}) & := & K \mbox{ for some context } K
\end{eqnarray*}

$u?(x)P | u!\langle Q \rangle \red P\{\quotep{Q}/x\}$

%We write $\wred$ for $\red^*$, and $P\red$ if $\exists Q $ such that $ P \red Q$.
We write $P\red$ if $\exists Q $ such that $ P \red Q$ and $P\not\red$, otherwise.

\section{Replication}

As mentioned before, it is known that replication (and hence
recursion) can be implemented in a higher-order process algebra
\cite{SangiorgiWalker}. As our first example of calculation with the
machinery thus far presented we give the construction explicitly in
the {\rhoc}.

\begin{eqnarray}
	D_{x} & := & \prefix{x}{y}{(\binpar{\outputp{x}{y}}{@{y}})} \nonumber\\
	\bangp_{x}{P} & := & \binpar{{x}!\langle{\binpar{D_{x}}{P}}\rangle}{D_{x}} \nonumber
\end{eqnarray}

\begin{eqnarray}
	\bangp_{x}{P} & & \nonumber\\
	=
	& {x}!\langle{(\prefix{x}{y}{(\outputp{x}{y} | @{y})) | P}}\rangle 
	      | \prefix{x}{y}{(\outputp{x}{y} | @{y})} & \nonumber\\
	\red
	& (\outputp{x}{y} | @{y})\substn{\quotep{(\prefix{x}{y}{(@{y} | \outputp{x}{y})) | P}}}{y} & \nonumber\\
	=
	& \outputp{x}{\quotep{(\prefix{x}{y}{(\outputp{x}{y} | @{y})) | P}}}
	  | {(\prefix{x}{y}{(\outputp{x}{y} | @{y})) | P}} & \nonumber\\
	\red
	& \ldots & \nonumber\\
	\red^*
	& P | P | \ldots & \nonumber
\end{eqnarray}

Of course, this encoding, as an implementation, runs away, unfolding
$\bangp{P}$ eagerly. A lazier and more implementable replication
operator, restricted to input-guarded processes, may be obtained as follows.

\begin{eqnarray}
\bangp{\prefix{u}{v}{P}} 
	:= 
	\binpar{\lift{x}{\prefix{u}{v}{(\binpar{D(x)}{P})}}}{D(x)} \nonumber
\end{eqnarray}

\begin{remark}
  Note that the lazier definition still does not deal with summation
  or mixed summation (i.e. sums over input and output). The reader is
  invited to construct definitions of replication that deal with these
  features. 

  Further, the definitions are parameterized in a name, $x$. Can you,
  gentle reader, make a definition that eliminates this parameter and
  guarantees no accidental interaction between the replication
  machinery and the process being replicated -- i.e. no accidental
  sharing of names used by the process to get its work done and the
  name(s) used by the replication to effect copying. This latter
  revision of the definition of replication is crucial to obtaining
  the expected identity $!!P \sim !P$.
\end{remark}

\begin{remark}\label{rem:paradoxical_combinator}
  The reader familiar with the lambda calculus will have noticed the
  similarity between $D$ and the paradoxical combinator.

  [Ed. note: the existence of this seems to suggest we have to be more
  restrictive on the set of processes and names we admit if we are to
  support no-cloning.]
\end{remark}

\subsubsection{Bisimulation}

The computational dynamics gives rise to another kind of equivalence,
the equivalence of computational behavior. As previously mentioned
this is typically captured \emph{via} some form of bisimulation.

% The notion we use in this paper is weak barbed bisimulation
% \cite{milner91polyadicpi}.

The notion we use in this paper is derived from weak barbed
bisimulation \cite{milner91polyadicpi}. 

\begin{definition}
An \emph{observation relation}, $\downarrow_{\mathcal N}$, over a set
of names, $\mathcal N$, is the smallest relation satisfying the rules
below.

\infrule[Out-barb]{y \in {\mathcal N}, \; x \nameeq y}
		  {\outputp{x}{v} \downarrow_{\mathcal N} x}
\infrule[Par-barb]{\mbox{$P\downarrow_{\mathcal N} x$ or $Q\downarrow_{\mathcal N} x$}}
		  {\binpar{P}{Q} \downarrow_{\mathcal N} x}

We write $P \Downarrow_{\mathcal N} x$ if there is $Q$ such that 
$P \wred Q$ and $Q \downarrow_{\mathcal N} x$.
\end{definition}

\begin{definition}
%\label{def.bbisim}
An  ${\mathcal N}$-\emph{barbed bisimulation} over a set of names, ${\mathcal N}$, is a symmetric binary relation 
${\mathcal S}_{\mathcal N}$ between agents such that $P\rel{S}_{\mathcal N}Q$ implies:
\begin{enumerate}
\item If $P \red P'$ then $Q \wred Q'$ and $P'\rel{S}_{\mathcal N} Q'$.
\item If $P\downarrow_{\mathcal N} x$, then $Q\Downarrow_{\mathcal N} x$.
\end{enumerate}
$P$ is ${\mathcal N}$-barbed bisimilar to $Q$, written
$P \wbbisim_{\mathcal N} Q$, if $P \rel{S}_{\mathcal N} Q$ for some ${\mathcal N}$-barbed bisimulation ${\mathcal S}_{\mathcal N}$.
\end{definition}

$\mathcal{R} \subseteq \pi \times \pi$

$P \mathcal{R} Q => \forall P'. P \red P' \Rightarrow \exists Q'. Q \red Q', P' \mathcal{R} Q'$

$P \vdash x \Rightarrow Q \vdash x$

\begin{mathpar}
  \inferrule*[lab=Out-barb]{x \nameeq y}{{y}!\langle{Q}\rangle \vdash x}
  \and
  \inferrule*[lab=Par-barb]{\mbox{$P\vdash x$ or $Q\vdash x$}}{\binpar{P}{Q} \vdash x}
\end{mathpar}

\subsubsection{Contexts}

One of the principle advantages of computational calculi like the
$\pi$-calculus is a well-defined notion of context,
contextual-equivalence and a correlation between
contextual-equivalence and notions of bisimulation. The notion of
context allows the decomposition of a process into (sub-)process and
its syntactic environment, its context. Thus, a context may be
thought of as a process with a ``hole'' (written $\Box$) in it. The
application of a context $M$ to a process $P$, written $M[P]$, is
tantamount to filling the hole in $M$ with $P$. In this paper we do
not need the full weight of this theory, but do make use of the notion
of context in the proof the main theorem. 

\begin{mathpar}
  \inferrule* [lab=summation] {} {{M_{M},M_{N}} \bc \Box \;|\; x.M_{A} \;|\; M_{M}+M_{N}}
  \and
  \inferrule* [lab=agent] {} {{M_{A}} \bc (\vec{x})M_{P} \;| \; \clift{P_0,\ldots,M_{P},\ldots,P_N}}
  \and \\
  \inferrule* [lab=process] {} {{M_{P}} \bc M_{N} \;| \;P|M_{P} }
\end{mathpar} 

\begin{mathpar}
  \inferrule* [lab=sychronization] {} {M_{N} \bc \Box \;|\; x?M_{F} \;|\; x!M_{C}}
  \and
  \inferrule* [lab=abstraction] {} {{M_{F}} \bc (x)M_{P} }
  \and
  \inferrule* [lab=concretion] {} {{M_{C}} \bc \langle M_{P} \rangle }
  \and \\
  \inferrule* [lab=process] {} {{M_{P}} \bc M_{N} \;| \;P|M_{P} }
\end{mathpar}

\begin{definition}[contextual application] Given a context $M$, and
  process $P$, we define the \emph{contextual application}, $M[P] :=
  M\{P/\Box\}$. That is, the contextual application of M to P is the
  substitution of $P$ for $\Box$ in $M$.
\end{definition}

$\meaningof{-} : L \to \mathcal{P}(\pi)$

\begin{mathpar}
  \inferrule* [lab=collection] {} {\meaningof{true} = \pi, \and \meaningof{~E} = \pi \setminus \meaningof{E}, \and \meaningof{E_{1} \& E_{2}} = \meaningof{E_{1}} \cap \meaningof{E_{2}}}
\end{mathpar}

\begin{mathpar}
  \inferrule* [lab=structure] {} {\meaningof{0} = \{ P \in \pi | P \equiv 0 \}, \and \\ \meaningof{E_1 | E_2} = \{ P \in \pi | P \equiv P_{1} | P_{2}, P_{1} \in \meaningof{E_{1}}, P_{2} \in \meaningof{E_2}\} }
\end{mathpar}

\begin{mathpar}
 \inferrule* [lab=behavior] {} {\meaningof{\langle a?b \rangle E} = \{ P \in \pi | P \equiv Q | u?(y)P', \\ \and \\\\ \and \\ \;\;\; u \in \meaningof{a}, \forall z.P'\{z/y\} \in \meaningof{E\{z/b\}}\}, \and \\ \meaningof{a!E} = \{ P \in \pi | P \equiv Q | x!\langle P' \rangle, x \in \meaningof{a} P' \in \meaningof{E}\} }
\end{mathpar}

\begin{mathpar}
 \inferrule* [lab=nominal] {} {\meaningof{\quotep{E}} = \{ \quotep{P} \in \quotep{\pi} | P \in \meaningof{E} \}, \and \meaningof{\quotep{P}} = \{ \quotep{Q} \in \quotep{\pi} | P \equiv Q \} \and \\ \meaningof{@\quotep{E}} = \{ P \in \pi | P \equiv @x, x \in \meaningof{E} \}}
\end{mathpar}

\begin{eqnarray*}
  \\
  \meaningof{-} : TS \to ST
\end{eqnarray*}

\begin{eqnarray*}
  \\
  L : TS \to ST
\end{eqnarray*}

\begin{eqnarray*}
  \\
  P \models E \iff P \in \meaningof{E}
\end{eqnarray*}

\begin{eqnarray*}
  P \approx_{L} Q \iff \forall E \in L. P \models E \iff Q \models E
\end{eqnarray*}

\begin{eqnarray*}
  P \approx_{K} Q
\end{eqnarray*}

\begin{eqnarray*}
  P \approx Q
\end{eqnarray*}

$\approx_{K} = \approx = \approx_{L}$

\subsubsection{Contextual duality}

Note that contexts extend the quotation operation to a family of
operations from processes to names. Given a context, $M$, we can
define a \emph{nominal context}, $\quotep{M}$ by $\quotep{M}[P] :=
\quotep{M[P]}$. To foreshadow what is to come we observe that these
operations enjoy a duality with processes very much like the duality
between vectors and maps from vectors to scalars.

Further, because the calculus is essentially higher-order, we have a
correspondence between contexts and processes. More specifically,
given a name $x$ and a context $M$ we can construct $M^{*}_{x}$ such
that 

\begin{mathpar}
  M^{*}_{x} | \lift{x}{P} \red M[P]
\end{mathpar}

namely,

\begin{mathpar}
  M^{*}_{x} := x?(u).M[\dropn{u}]
\end{mathpar}

The dependence of $M^{*}_{x}$ on a name makes it an abstraction, 

\begin{mathpar}
  M^{*} := (x)x?(u).M[\dropn{u}]
\end{mathpar}

\subsection{Additional notation}

It will sometimes be convenient to denote the process a name
quotes. We already have the notation $x = \quotep{P}$, but it will be
convenient to introduce an alternate notation, $\procn{x}$, when we
want to emphasize the connection to the use of the name. Note that, by
virtue of name equivalence, $\quotep{\procn{x}} \nameeq x$; so, the
notation is consistent with previous definitions.

Further, because names have structure it is possible to effect
substitutions on the basis of that structure. This means we need to
upgrade our notation for substitutions, which we accomplish by
adapting comprehension notation. Thus,

\begin{mathpar}
  P\{ y / x : x \in S \}
\end{mathpar}

is interpreted to mean the process derived from P by replacing (in a
capture-avoiding manner) each occurrence of $x$ in $S$ by $y$. For example,

\begin{mathpar}
  P\{ \quotep{\procn{x}|\procn{x}} / x : x \in \freenames{P} \}
\end{mathpar}

will replace each (occurrence) of a free name $x$ in $P$ by
$\quotep{\procn{x}|\procn{x}}$.

Also, we will avail ourselves of the notation $x^{L}$ and $x^{R}$ to
denote injections of a name into disjoint copies of the name
space. There are numerous ways to accomplish this. One example can be
found in \cite{MeredithR05}. This notation overloads to vectors of
names: $\vec{x}^{\pi} := (x_{i}^{\pi} \; : \; 0 \leq i < |\vec{x}| )$ where $\pi \in \{L,R\}$.

We also use $P^{\Box} := P|\Box$.

In \cite{MeredithR05} an interpretation of the new operator is
given. It turns out that there are several possible interpretations
all enjoying the requisite algebraic properties of the operator (see
\cite{milner91polyadicpi}). We will therefore make liberal use of
$(\nu\; \vec{x})P$.

% subsection the_syntax_and_semantics_of_the_notation_system (end)   

\input{qm2pi.qmops} 

\input{qm2pi.sterngerlach} 

\input{qm2pi.metric} 

% section concurrent_process_calculi (end)

%\input{qm2pi.proofsketch}

% section proof sketch (end)

%\input{qm2pi.slviaknots} 

% section spatial logic via knots (end)

\input{qm2pi.conclusion}

% section conclusion (end)

%\input{qm2pi.dtcodes} 

% section wiring algorithm (end)

\input{qm2pi.ack} 

% section acknowledgments (end)

\newpage


\bibliographystyle{plain}   
\bibliography{../../biblios/main.bib}

\input{qm2pi.rhodetails}

\end{document}



% section front matter (end)

\section{Introduction}\label{sec:introduction} % (fold)
In this draft of the material i am going to have to dispense with the
usual writing conventions adopted in papers on these topics. i'm going
to have adopt whatever tone i need at the time i'm writing up the
calculations. Sometimes this may be very conversational; others it may
be the barest mathematical grunts; others still it may be that i have
lifted text from one of my other papers because the exposition of some
point was better said there. i hope that my readers are not unduly put
out by this decision. i'm not doing this to flout convention or be
rebellious. i find these calculations very technically challenging. To
keep everything going technically, something has to give; i have to
let go of some cognitive burden. So, the academic writing style --
with all of its trade-offs in terms of facilitating technical
communication -- is what i'm letting go of. Perhaps subsequent drafts
can be tightened and polished, but for now, i'm going to speak as if
we were sitting together in a coffee shop with a laptop, wifi and a
pad of paper and a pencil.

So, here's what i have to say. We -- you and i, comfortably ensconced
in our coffee shop and well-equipped with our tools -- can realize and
carry out the calculations of quantum mechanics over a very different
formal theory of dynamics, a formal theory of dynamics that
corresponds to a theory of concurrent computation with
\emph{reflection}. It has the advantage that the underlying theory is
already `quantized', but supports analogues all of the continuuous
operations. Strikingly, this underlying theory has recently been
connected with a notion of metric that we can show, by calculating
together, coincides with the metric induced by the inner product.

There are a lot of reasons why you might be interested in seeing
calculations of this form. Here's why i'm interested. For the past
several centuries there has been no competitor to the ``Newtonian''
account of dynamics. As a result the predominant share of accounts of
dynamical systems and situations have had to be formulated in terms of
the Newtonian machinery. i view this as an intellectually dangerous
position to occupy. Everything, despite it's intrinsic shape, turns
into a nail to be hit with this hammer. Recently, however, the theory
of computation has matured to the point where we have candidates for
theories of dynamics that offer very different perspective on
reasoning about dynamical systems and situations. Testing these
candidates against very successful accounts of dynamical situations,
like quantum mechanics, is going to give us some sense of how mature
they are and some measure of the quality of these accounts of
dynamics.

\subsection{Summary of contributions and outline of paper}

So, we're going to develop an interpretation of the operations of
quantum mechanics normally interpreted by Hilbert spaces and
operators. We're going to do this over a theory of computation. Note
that this is very different than the usual quantum computation program
which develops notions of computation over quantum mechanics. Rather,
we are developing a story that aligns with Wheeler's slogan: It from
Bit. To do this we will first provide an account of the theory of
computation at play here. Then we will dive into a calculation-driven
interpretation of the operations of quantum mechanics.

The reason we take this approach is that -- until very recently --
there hasn't been an axiomatic account of quantum mechanics. As a
result there has been no sharp delineation of the mathematical theory
supporting interpretation of the physical theory and the physical
theory, itself. So, ambient features of the maths are free to be
exploited (or supressed) without a real accounting of their physical
relevance. There is no sharp statement ``here's the physical theory''
qua \emph{theory} and ``here's the mathematical interpretation''
enabling a judgment of how faithful the interpretation is -- apart
from experimental observation. When there is an axiomatic account we
can judge how well a given mathematical formalism supports an
interpretation of the axioms, independent of
experimentation. Likewise, we can judge how well we have captured our
physical evidence and experience with our axiomatics, independent of
any specific mathematical implementation, with accidental detail that
may or may not have physical significance. 

In lieu of a fully fleshed out and vetted axiomatic account of quantum
mechanics, interpreting the operational notions in service of modeling
physical systems will have to suffice. In other words, we are not in
the business of providing a model of Hilbert spaces and operators. We
are in the business of providing a model of quantum mechanics because
we are motivated by testing our notions of dynamics against physical
theory; and, the predictive calculations of the physical theory must
serve as the best formulation -- shy of a fully fleshed out axiomatic
account -- of the physical theory itself (as they have for scientific
theories since time immemorial). Put another way, despite a
whole-hearted commitment to an It-from-Bit ontology, we are firmly
aligned with the shut-up-and-calculate camp as the best way to obtain
results either from the physical perspective or as a quality assurance
measure of our fledgling theory of dynamics.

In detail, we present a reflective process calculus. Then we develop
intuitive correspondences between the notions available in this
calculus and the usual physical notions supporting quantum mechanical
calculations. Thus, 

\begin{table}[htp]
  \center{
    \fbox{
      \begin{tabular}{c|c}
        quantum mechanics & process calculus \\
        \hline
        scalar & name \\
        state vector & process \\
        dual & contextual duals \\
        matrix & formal sums of process-context-dual pairs \\
        orthogonality & process annihilation \\
        inner product & execution-formula + quoting
      \end{tabular}
    }
  }
  \caption{QM - process calculi correspondences}
\end{table}

Then we tighten up these intuitions to operational definitions. We
employ the Dirac notation as the best proxy we can find for an
abstract syntax of the quantum mechanical notions. The definitions we
develop put us in contact with equational constraints coming from the
theory that we demonstrate the definitions and calculations satisfy.

This puts us in a position to shut up and calculate for the
Stern-Gerlach experimental set up, showing how these predictive
calculations become calculations on processes in our theory of a
reflective process calculus.

Penultimately, we demonstrate that the notion of metric coming from
the inner product coincides with the notion of metric available from
the theory of bisimulation. This demonstration gives us the right to
think of space as arising from behavior. Finally, we consider where we
might go from the new vantage point we have obtained.

% section introduction (end) 
 
% section introduction (end)

% \documentclass[12pt]{llncs}
%\documentclass{jktr}

\usepackage[pdftex]{hyperref}                   
\usepackage {listings}
\usepackage {mathpartir}
\usepackage{bcprules}
%\usepackage{listings}
                       
\usepackage{graphicx} 
%\usepackage[margins=2.5cm,nohead,nofoot]{geometry}
%\usepackage{geometry}
\usepackage{amsfonts}
\usepackage{amstext}
\usepackage{latexsym}
\usepackage{amssymb}
\usepackage{color}


%\include{myPreamble}
\include{qm2pi.local} 

%\ifpdf
%\usepackage[pdftex]{graphicx}
%\else
%\usepackage{graphicx}
%\fi

 % \ifpdf
%  \usepackage{pdfsync}
%  \if


%\title{Brief Article}
%\author{David F. Snyder}
%\author{L.G. Meredith}

%\address{Dept. of Math., Texas State University--San Marcos, San Marcos, TX 78666}
       
\pagestyle{empty}


\begin{document}

\lstset{language=[Objective]Caml,frame=shadowbox}

\input{qm2pi.front}

% section front matter (end)

\input{qm2pi.intro} 
 
% section introduction (end)

% \input{qm2pi.knotations} 

% section notation (end)

\input{qm2pi.process.calculi} 

% section concurrent_process_calculi_and_spatial_logics_ (end)
    
%\input{qm2pi.knots2pi} 

%\input{qm2pi.trefoil} 

%\input{qm2pi.mainthm} 

% subsection basic_interpretation (end)

%\input{qm2pi.rho.presentation} 
\subsection{The syntax and semantics of the notation system}\label{sub:the_syntax_and_semantics_of_the_notation_system} % (fold)

We now summarize a technical presentation of the calculus that
embodies our theory of dynamics. The typical presentation of such a
calculus follows the style of giving generators and relations on
them. The grammar, below, describing term constructors, freely
generates the set of processes, $\Proc$. This set is then quotiented
by a relation known as structural congruence and it is over this set
that the notion of dynamics is expressed. This presentation is
essentially that of \cite{MeredithR05} with the addition of
polyadicity and summation. For readability we have relegated some of
the technical subtleties to an appendix.

\subsubsection{Process grammar}\label{subsub:process_grammar}

\begin{mathpar}
  \inferrule* [lab=synchronization] {} {{M} \bc \pzero \;|\; x?F \;|\; x!C }
  \and
  \inferrule* [lab=abstraction] {} {{F} \bc (x)P}
  \and
  \inferrule* [lab=concretion] {} {{C} \bc \langle Q \rangle}
  \and
  \inferrule* [lab=process] {} {{P,Q} \bc M \;| \;P|Q \;|\; @{x}}
  \and
  \inferrule* [lab=name] {} {{x} \bc \quotep{P}}
\end{mathpar} 

Note that $\vec{x}$ (resp. $\vec{P}$) denotes a vector of names
(resp. processes) of length $|\vec{x}|$ (resp. $|\vec{P}|$). We adopt
the following useful abbreviations.

\begin{mathpar}
   x?(\vec{y}).P := x.(\vec{y})P \and  x\clift{\vec{P}} := x.\clift{\vec{P}}
   \and x!(y) := \lift{x}{\dropn{y}}
   \and \Pi_{i=0}^{n-1}P_i := P_0 | \ldots | P_{n-1}
\end{mathpar}

\subsubsection{Structural congruence}

\paragraph{Free and bound names and alpha-equivalence.} At the
core of structural equivalence is alpha-equivalence which identifies
process that are the same up to a change of variable. Formally, we
recognize the distinction between free and bound names. The free names
of a process, $\freenames{P}$, may be calculated recursively as
follows:

\begin{mathpar}
\freenames{\pzero} := \emptyset
  \and \\
  \freenames{x?(y).P} := \{ x \} \cup (\freenames{P} \setminus \{ y \})
  \and 
  \freenames{x!\langle P \rangle} := \{ x \} \cup \{ P \} 
  \and \\
  \freenames{P|Q} := \freenames{P} \cup \freenames{Q}
  \and \\
  \freenames{@{x}} := \{ x \}
\end{mathpar}

$\pi$
$\quotep{\pi}$

$\freenames{-} : \pi \to \mathcal{P}(\quotep{\pi})$

\begin{eqnarray*}
  \freenames{\pzero} & := & \emptyset \\
  \freenames{x?(y).P} & := & \{ x \} \cup (\freenames{P} \setminus \{ y \}) \\
  \freenames{x!\langle P \rangle} & := & \{ x \} \cup \{ P \} \\
  \freenames{P|Q} & := & \freenames{P} \cup \freenames{Q} \\
  \freenames{\dropn{x}} & := & \{ x \}
\end{eqnarray*}

The bound names of a process, $\boundnames{P}$, are those names occurring in $P$
that are not free. For example, in $x?(y).0$, the name $x$ is free, while $y$ is bound.

\begin{mathpar}
  \inferrule* [lab=monoidal-laws] {} { P|Q \equiv Q|P \and P|0 \equiv P \and P|(Q|R) \equiv (P|Q)|R }
\end{mathpar}

\begin{mathpar}
  \inferrule* [lab=alpha-equivalence] {} { (x)P \equiv (y)P\{y/x\} \and y \not\in \freenames{P} }
\end{mathpar}

\begin{definition}
Then two processes, $P,Q$, are alpha-equivalent if $P = Q\{\vec{y}/\vec{x}\}$ for
some $\vec{x} \in \boundnames{Q},\vec{y} \in \boundnames{P}$, where $Q\{\vec{y}/\vec{x}\}$
denotes the capture-avoiding substitution of $\vec{y}$ for $\vec{x}$ in $Q$.
\end{definition}

\begin{definition}
  The {\em structural congruence} \cite{SangiorgiWalker} , $\equiv$,
  between processes is the least congruence containing
  alpha-equivalence, satisfying the abelian monoid laws
  (associativity, commutativity and $\pzero$ as identity) for parallel
  composition $|$ and for summation $+$.
\end{definition}

\subsection{Name equivalence}

We take name equivalence, written $\nameeq$, to be the smallest
equivalence relation generated by the following rules.

\begin{mathpar}
\inferrule*[lab=Quote-drop]
{ }
{ \quotep{@{x}} \nameeq x }

\inferrule*[lab=Struct-equiv]
{ P \scong Q }
{ \quotep{P} \nameeq \quotep{Q} }
\end{mathpar}

The astute reader will have noticed that the mutual recursion of names
and processes imposes a mutual recursion on alpha-equivalence and
structural equivalence via name-equivalence. Fortunately, all of this
works out pleasantly and we may calculate in the natural way, free of
concern. The reader interested in the details is referred to the
appendix \ref{appendix:rho_details}.

\subsection{Substitution}

We use $\Proc$ for the set of processes, $\QProc$ for the set of
names, and $\id{\{}\vec{y} / \vec{x} \id{\}}$ to denote partial maps,
$s : \QProc \rightarrow \QProc$. A map, $s$ lifts, uniquely, to a map
on process terms, $\widehat{s} : \Proc \rightarrow \Proc$ by the
following equations.

\begin{mathpar}
  (0) \psubstp{Q}{P} := 0 \\
  (R \juxtap S) \psubstp{Q}{P}
  :=    
  (R)\psubstp{Q}{P} \juxtap (S) \psubstp{Q}{P} \\
  (x?(y).R) \psubstp{Q}{P}    
  :=    
  (x)\substp{Q}{P} (z)\concat( (R \psubstn{z}{y}) \psubstp{Q}{P} ) \\
  (\lift{x}{R}) \psubstp{Q}{P}  
  :=
  \lift{(x)\substp{Q}{P}}{ R \psubstp{Q}{P} } \\
%   (\dropn{x})  \psubstp{Q}{P}       
%   := 
%   \left\{ 
%     \begin{array}{ccc} 
%       \dropn{\quotep{Q}} & & x \nameeq \quotep{P} \\
%       \dropn{x} & & otherwise \\
%     \end{array}
%   \right. 
  (\dropn{x})  \psubstp{Q}{P}       
  := 
  \left\{ 
    \begin{array}{ccc} 
      Q & & x \nameeq \quotep{P} \\
      \dropn{x} & & otherwise \\
    \end{array}
  \right.
\end{mathpar}
 

where

\begin{eqnarray}
  (x)\id{\{} \lpquote Q \rpquote / \lpquote P \rpquote \id{\}}            = 
  \left\{ 
    \begin{array}{ccc}
      \lpquote Q \rpquote & & x \nameeq \lpquote P \rpquote \\
      x & & otherwise \\
    \end{array}
  \right. \nonumber
\end{eqnarray}

and $z$ is chosen distinct from $\quotep{P}$, $\quotep{Q}$, the free
names in $Q$, and all the names in $R$. Our $\alpha$-equivalence will
be built in the standard way from this substitution.

\begin{remark}\label{rem:no_self_referential_names}
  One consequence of these definitions is that $\forall P. \quotep{P}
  \not\in \freenames{P}$.
\end{remark}

\subsection{ Dynamic quote: an example }

Anticipating something of what's to come, consider applying the
substitution, $\widehat{\id{\{}u / z \id{\}}}$, to the following pair
of processes, $\lift{w}{y!(z)}$ and $w[ \lpquote y!(z) \rpquote ]$.

\begin{eqnarray}
	\lift{w}{y!(z)}\widehat{\id{\{}u / z \id{\}}}
		& = &
		\lift{w}{y!(u)} \nonumber\\
	w[ \lpquote y!(z) \rpquote ] \widehat{ \id{\{}u / z \id{\}} }
		& = &
		w[ \lpquote y!(z) \rpquote ] \nonumber
\end{eqnarray}

Because the body of the process between quotes is impervious to
substitution, we get radically different answers. In fact, by
examining the first process in an input context,
e.g. $x?(z).\lift{w}{y!(z)}$, we see that the process under the lift
operator may be shaped by prefixed inputs binding a name inside it. In
this sense, the lift operator will be seen as a way to dynamically
construct processes before reifying them as names.

Finally equipped with these standard features we can present the
dynamics of the calculus.

\subsubsection{Operational semantics} 

Finally, we introduce the computational dynamics. What marks these
algebras as distinct from other more traditionally studied algebraic
structures, e.g. vector spaces or polynomial rings, is the manner in
which dynamics is captured. In traditional structures, dynamics is typically
expressed through morphisms between such structures, as in linear maps
between vector spaces or morphisms between rings. In algebras
associated with the semantics of computation, the dynamics is
expressed as part of the algebraic structure itself, through a
reduction reduction relation typically denoted by $\red$. Below, we
give a recursive presentation of this relation for the calculus used
in the encoding.

$\red \subseteq \pi \times \pi$
$\red : \pi \to \mathcal{P}(\pi)$

\begin{mathpar}
  \inferrule* [lab=Comm] { \textsf{match}( x_{src}, x_{trgt} ) } { x_{trgt}?(y)P \; | \; x_{src}!\langle {Q} \rangle \red P\{\quotep{Q}/y}\} }
  \and \\
  \inferrule* [lab=Par] {{P} \red {P}'} {{{P} | {Q}} \red {{P}' | {Q}}}
  \and
  \inferrule* [lab=Equiv]{{{P} \scong {P}'} \andalso {{P}' \red {Q}'} \andalso {{Q}' \scong {Q}}}{{P} \red {Q}}
\end{mathpar}

\begin{eqnarray*}
  match_{\equiv} (\quotep{P},\quotep{Q}) & := & P \equiv Q \\
  match_{\dagger}(\quotep{P},\quotep{Q}) & := & \forall R. P|Q \red^{*} R => R \red^{*} 0 \\
  match_{K}(\quotep{P},\quotep{Q}) & := & K \mbox{ for some context } K
\end{eqnarray*}

$u?(x)P | u!\langle Q \rangle \red P\{\quotep{Q}/x\}$

%We write $\wred$ for $\red^*$, and $P\red$ if $\exists Q $ such that $ P \red Q$.
We write $P\red$ if $\exists Q $ such that $ P \red Q$ and $P\not\red$, otherwise.

\section{Replication}

As mentioned before, it is known that replication (and hence
recursion) can be implemented in a higher-order process algebra
\cite{SangiorgiWalker}. As our first example of calculation with the
machinery thus far presented we give the construction explicitly in
the {\rhoc}.

\begin{eqnarray}
	D_{x} & := & \prefix{x}{y}{(\binpar{\outputp{x}{y}}{@{y}})} \nonumber\\
	\bangp_{x}{P} & := & \binpar{{x}!\langle{\binpar{D_{x}}{P}}\rangle}{D_{x}} \nonumber
\end{eqnarray}

\begin{eqnarray}
	\bangp_{x}{P} & & \nonumber\\
	=
	& {x}!\langle{(\prefix{x}{y}{(\outputp{x}{y} | @{y})) | P}}\rangle 
	      | \prefix{x}{y}{(\outputp{x}{y} | @{y})} & \nonumber\\
	\red
	& (\outputp{x}{y} | @{y})\substn{\quotep{(\prefix{x}{y}{(@{y} | \outputp{x}{y})) | P}}}{y} & \nonumber\\
	=
	& \outputp{x}{\quotep{(\prefix{x}{y}{(\outputp{x}{y} | @{y})) | P}}}
	  | {(\prefix{x}{y}{(\outputp{x}{y} | @{y})) | P}} & \nonumber\\
	\red
	& \ldots & \nonumber\\
	\red^*
	& P | P | \ldots & \nonumber
\end{eqnarray}

Of course, this encoding, as an implementation, runs away, unfolding
$\bangp{P}$ eagerly. A lazier and more implementable replication
operator, restricted to input-guarded processes, may be obtained as follows.

\begin{eqnarray}
\bangp{\prefix{u}{v}{P}} 
	:= 
	\binpar{\lift{x}{\prefix{u}{v}{(\binpar{D(x)}{P})}}}{D(x)} \nonumber
\end{eqnarray}

\begin{remark}
  Note that the lazier definition still does not deal with summation
  or mixed summation (i.e. sums over input and output). The reader is
  invited to construct definitions of replication that deal with these
  features. 

  Further, the definitions are parameterized in a name, $x$. Can you,
  gentle reader, make a definition that eliminates this parameter and
  guarantees no accidental interaction between the replication
  machinery and the process being replicated -- i.e. no accidental
  sharing of names used by the process to get its work done and the
  name(s) used by the replication to effect copying. This latter
  revision of the definition of replication is crucial to obtaining
  the expected identity $!!P \sim !P$.
\end{remark}

\begin{remark}\label{rem:paradoxical_combinator}
  The reader familiar with the lambda calculus will have noticed the
  similarity between $D$ and the paradoxical combinator.

  [Ed. note: the existence of this seems to suggest we have to be more
  restrictive on the set of processes and names we admit if we are to
  support no-cloning.]
\end{remark}

\subsubsection{Bisimulation}

The computational dynamics gives rise to another kind of equivalence,
the equivalence of computational behavior. As previously mentioned
this is typically captured \emph{via} some form of bisimulation.

% The notion we use in this paper is weak barbed bisimulation
% \cite{milner91polyadicpi}.

The notion we use in this paper is derived from weak barbed
bisimulation \cite{milner91polyadicpi}. 

\begin{definition}
An \emph{observation relation}, $\downarrow_{\mathcal N}$, over a set
of names, $\mathcal N$, is the smallest relation satisfying the rules
below.

\infrule[Out-barb]{y \in {\mathcal N}, \; x \nameeq y}
		  {\outputp{x}{v} \downarrow_{\mathcal N} x}
\infrule[Par-barb]{\mbox{$P\downarrow_{\mathcal N} x$ or $Q\downarrow_{\mathcal N} x$}}
		  {\binpar{P}{Q} \downarrow_{\mathcal N} x}

We write $P \Downarrow_{\mathcal N} x$ if there is $Q$ such that 
$P \wred Q$ and $Q \downarrow_{\mathcal N} x$.
\end{definition}

\begin{definition}
%\label{def.bbisim}
An  ${\mathcal N}$-\emph{barbed bisimulation} over a set of names, ${\mathcal N}$, is a symmetric binary relation 
${\mathcal S}_{\mathcal N}$ between agents such that $P\rel{S}_{\mathcal N}Q$ implies:
\begin{enumerate}
\item If $P \red P'$ then $Q \wred Q'$ and $P'\rel{S}_{\mathcal N} Q'$.
\item If $P\downarrow_{\mathcal N} x$, then $Q\Downarrow_{\mathcal N} x$.
\end{enumerate}
$P$ is ${\mathcal N}$-barbed bisimilar to $Q$, written
$P \wbbisim_{\mathcal N} Q$, if $P \rel{S}_{\mathcal N} Q$ for some ${\mathcal N}$-barbed bisimulation ${\mathcal S}_{\mathcal N}$.
\end{definition}

$\mathcal{R} \subseteq \pi \times \pi$

$P \mathcal{R} Q => \forall P'. P \red P' \Rightarrow \exists Q'. Q \red Q', P' \mathcal{R} Q'$

$P \vdash x \Rightarrow Q \vdash x$

\begin{mathpar}
  \inferrule*[lab=Out-barb]{x \nameeq y}{{y}!\langle{Q}\rangle \vdash x}
  \and
  \inferrule*[lab=Par-barb]{\mbox{$P\vdash x$ or $Q\vdash x$}}{\binpar{P}{Q} \vdash x}
\end{mathpar}

\subsubsection{Contexts}

One of the principle advantages of computational calculi like the
$\pi$-calculus is a well-defined notion of context,
contextual-equivalence and a correlation between
contextual-equivalence and notions of bisimulation. The notion of
context allows the decomposition of a process into (sub-)process and
its syntactic environment, its context. Thus, a context may be
thought of as a process with a ``hole'' (written $\Box$) in it. The
application of a context $M$ to a process $P$, written $M[P]$, is
tantamount to filling the hole in $M$ with $P$. In this paper we do
not need the full weight of this theory, but do make use of the notion
of context in the proof the main theorem. 

\begin{mathpar}
  \inferrule* [lab=summation] {} {{M_{M},M_{N}} \bc \Box \;|\; x.M_{A} \;|\; M_{M}+M_{N}}
  \and
  \inferrule* [lab=agent] {} {{M_{A}} \bc (\vec{x})M_{P} \;| \; \clift{P_0,\ldots,M_{P},\ldots,P_N}}
  \and \\
  \inferrule* [lab=process] {} {{M_{P}} \bc M_{N} \;| \;P|M_{P} }
\end{mathpar} 

\begin{mathpar}
  \inferrule* [lab=sychronization] {} {M_{N} \bc \Box \;|\; x?M_{F} \;|\; x!M_{C}}
  \and
  \inferrule* [lab=abstraction] {} {{M_{F}} \bc (x)M_{P} }
  \and
  \inferrule* [lab=concretion] {} {{M_{C}} \bc \langle M_{P} \rangle }
  \and \\
  \inferrule* [lab=process] {} {{M_{P}} \bc M_{N} \;| \;P|M_{P} }
\end{mathpar}

\begin{definition}[contextual application] Given a context $M$, and
  process $P$, we define the \emph{contextual application}, $M[P] :=
  M\{P/\Box\}$. That is, the contextual application of M to P is the
  substitution of $P$ for $\Box$ in $M$.
\end{definition}

$\meaningof{-} : L \to \mathcal{P}(\pi)$

\begin{mathpar}
  \inferrule* [lab=collection] {} {\meaningof{true} = \pi, \and \meaningof{~E} = \pi \setminus \meaningof{E}, \and \meaningof{E_{1} \& E_{2}} = \meaningof{E_{1}} \cap \meaningof{E_{2}}}
\end{mathpar}

\begin{mathpar}
  \inferrule* [lab=structure] {} {\meaningof{0} = \{ P \in \pi | P \equiv 0 \}, \and \\ \meaningof{E_1 | E_2} = \{ P \in \pi | P \equiv P_{1} | P_{2}, P_{1} \in \meaningof{E_{1}}, P_{2} \in \meaningof{E_2}\} }
\end{mathpar}

\begin{mathpar}
 \inferrule* [lab=behavior] {} {\meaningof{\langle a?b \rangle E} = \{ P \in \pi | P \equiv Q | u?(y)P', \\ \and \\\\ \and \\ \;\;\; u \in \meaningof{a}, \forall z.P'\{z/y\} \in \meaningof{E\{z/b\}}\}, \and \\ \meaningof{a!E} = \{ P \in \pi | P \equiv Q | x!\langle P' \rangle, x \in \meaningof{a} P' \in \meaningof{E}\} }
\end{mathpar}

\begin{mathpar}
 \inferrule* [lab=nominal] {} {\meaningof{\quotep{E}} = \{ \quotep{P} \in \quotep{\pi} | P \in \meaningof{E} \}, \and \meaningof{\quotep{P}} = \{ \quotep{Q} \in \quotep{\pi} | P \equiv Q \} \and \\ \meaningof{@\quotep{E}} = \{ P \in \pi | P \equiv @x, x \in \meaningof{E} \}}
\end{mathpar}

\begin{eqnarray*}
  \\
  \meaningof{-} : TS \to ST
\end{eqnarray*}

\begin{eqnarray*}
  \\
  L : TS \to ST
\end{eqnarray*}

\begin{eqnarray*}
  \\
  P \models E \iff P \in \meaningof{E}
\end{eqnarray*}

\begin{eqnarray*}
  P \approx_{L} Q \iff \forall E \in L. P \models E \iff Q \models E
\end{eqnarray*}

\begin{eqnarray*}
  P \approx_{K} Q
\end{eqnarray*}

\begin{eqnarray*}
  P \approx Q
\end{eqnarray*}

$\approx_{K} = \approx = \approx_{L}$

\subsubsection{Contextual duality}

Note that contexts extend the quotation operation to a family of
operations from processes to names. Given a context, $M$, we can
define a \emph{nominal context}, $\quotep{M}$ by $\quotep{M}[P] :=
\quotep{M[P]}$. To foreshadow what is to come we observe that these
operations enjoy a duality with processes very much like the duality
between vectors and maps from vectors to scalars.

Further, because the calculus is essentially higher-order, we have a
correspondence between contexts and processes. More specifically,
given a name $x$ and a context $M$ we can construct $M^{*}_{x}$ such
that 

\begin{mathpar}
  M^{*}_{x} | \lift{x}{P} \red M[P]
\end{mathpar}

namely,

\begin{mathpar}
  M^{*}_{x} := x?(u).M[\dropn{u}]
\end{mathpar}

The dependence of $M^{*}_{x}$ on a name makes it an abstraction, 

\begin{mathpar}
  M^{*} := (x)x?(u).M[\dropn{u}]
\end{mathpar}

\subsection{Additional notation}

It will sometimes be convenient to denote the process a name
quotes. We already have the notation $x = \quotep{P}$, but it will be
convenient to introduce an alternate notation, $\procn{x}$, when we
want to emphasize the connection to the use of the name. Note that, by
virtue of name equivalence, $\quotep{\procn{x}} \nameeq x$; so, the
notation is consistent with previous definitions.

Further, because names have structure it is possible to effect
substitutions on the basis of that structure. This means we need to
upgrade our notation for substitutions, which we accomplish by
adapting comprehension notation. Thus,

\begin{mathpar}
  P\{ y / x : x \in S \}
\end{mathpar}

is interpreted to mean the process derived from P by replacing (in a
capture-avoiding manner) each occurrence of $x$ in $S$ by $y$. For example,

\begin{mathpar}
  P\{ \quotep{\procn{x}|\procn{x}} / x : x \in \freenames{P} \}
\end{mathpar}

will replace each (occurrence) of a free name $x$ in $P$ by
$\quotep{\procn{x}|\procn{x}}$.

Also, we will avail ourselves of the notation $x^{L}$ and $x^{R}$ to
denote injections of a name into disjoint copies of the name
space. There are numerous ways to accomplish this. One example can be
found in \cite{MeredithR05}. This notation overloads to vectors of
names: $\vec{x}^{\pi} := (x_{i}^{\pi} \; : \; 0 \leq i < |\vec{x}| )$ where $\pi \in \{L,R\}$.

We also use $P^{\Box} := P|\Box$.

In \cite{MeredithR05} an interpretation of the new operator is
given. It turns out that there are several possible interpretations
all enjoying the requisite algebraic properties of the operator (see
\cite{milner91polyadicpi}). We will therefore make liberal use of
$(\nu\; \vec{x})P$.

% subsection the_syntax_and_semantics_of_the_notation_system (end)   

\input{qm2pi.qmops} 

\input{qm2pi.sterngerlach} 

\input{qm2pi.metric} 

% section concurrent_process_calculi (end)

%\input{qm2pi.proofsketch}

% section proof sketch (end)

%\input{qm2pi.slviaknots} 

% section spatial logic via knots (end)

\input{qm2pi.conclusion}

% section conclusion (end)

%\input{qm2pi.dtcodes} 

% section wiring algorithm (end)

\input{qm2pi.ack} 

% section acknowledgments (end)

\newpage


\bibliographystyle{plain}   
\bibliography{../../biblios/main.bib}

\input{qm2pi.rhodetails}

\end{document}

 

% section notation (end)

\input{qm2pi.process.calculi} 

% section concurrent_process_calculi_and_spatial_logics_ (end)
    
%\documentclass[12pt]{llncs}
%\documentclass{jktr}

\usepackage[pdftex]{hyperref}                   
\usepackage {listings}
\usepackage {mathpartir}
\usepackage{bcprules}
%\usepackage{listings}
                       
\usepackage{graphicx} 
%\usepackage[margins=2.5cm,nohead,nofoot]{geometry}
%\usepackage{geometry}
\usepackage{amsfonts}
\usepackage{amstext}
\usepackage{latexsym}
\usepackage{amssymb}
\usepackage{color}


%\include{myPreamble}
\include{qm2pi.local} 

%\ifpdf
%\usepackage[pdftex]{graphicx}
%\else
%\usepackage{graphicx}
%\fi

 % \ifpdf
%  \usepackage{pdfsync}
%  \if


%\title{Brief Article}
%\author{David F. Snyder}
%\author{L.G. Meredith}

%\address{Dept. of Math., Texas State University--San Marcos, San Marcos, TX 78666}
       
\pagestyle{empty}


\begin{document}

\lstset{language=[Objective]Caml,frame=shadowbox}

\input{qm2pi.front}

% section front matter (end)

\input{qm2pi.intro} 
 
% section introduction (end)

% \input{qm2pi.knotations} 

% section notation (end)

\input{qm2pi.process.calculi} 

% section concurrent_process_calculi_and_spatial_logics_ (end)
    
%\input{qm2pi.knots2pi} 

%\input{qm2pi.trefoil} 

%\input{qm2pi.mainthm} 

% subsection basic_interpretation (end)

%\input{qm2pi.rho.presentation} 
\subsection{The syntax and semantics of the notation system}\label{sub:the_syntax_and_semantics_of_the_notation_system} % (fold)

We now summarize a technical presentation of the calculus that
embodies our theory of dynamics. The typical presentation of such a
calculus follows the style of giving generators and relations on
them. The grammar, below, describing term constructors, freely
generates the set of processes, $\Proc$. This set is then quotiented
by a relation known as structural congruence and it is over this set
that the notion of dynamics is expressed. This presentation is
essentially that of \cite{MeredithR05} with the addition of
polyadicity and summation. For readability we have relegated some of
the technical subtleties to an appendix.

\subsubsection{Process grammar}\label{subsub:process_grammar}

\begin{mathpar}
  \inferrule* [lab=synchronization] {} {{M} \bc \pzero \;|\; x?F \;|\; x!C }
  \and
  \inferrule* [lab=abstraction] {} {{F} \bc (x)P}
  \and
  \inferrule* [lab=concretion] {} {{C} \bc \langle Q \rangle}
  \and
  \inferrule* [lab=process] {} {{P,Q} \bc M \;| \;P|Q \;|\; @{x}}
  \and
  \inferrule* [lab=name] {} {{x} \bc \quotep{P}}
\end{mathpar} 

Note that $\vec{x}$ (resp. $\vec{P}$) denotes a vector of names
(resp. processes) of length $|\vec{x}|$ (resp. $|\vec{P}|$). We adopt
the following useful abbreviations.

\begin{mathpar}
   x?(\vec{y}).P := x.(\vec{y})P \and  x\clift{\vec{P}} := x.\clift{\vec{P}}
   \and x!(y) := \lift{x}{\dropn{y}}
   \and \Pi_{i=0}^{n-1}P_i := P_0 | \ldots | P_{n-1}
\end{mathpar}

\subsubsection{Structural congruence}

\paragraph{Free and bound names and alpha-equivalence.} At the
core of structural equivalence is alpha-equivalence which identifies
process that are the same up to a change of variable. Formally, we
recognize the distinction between free and bound names. The free names
of a process, $\freenames{P}$, may be calculated recursively as
follows:

\begin{mathpar}
\freenames{\pzero} := \emptyset
  \and \\
  \freenames{x?(y).P} := \{ x \} \cup (\freenames{P} \setminus \{ y \})
  \and 
  \freenames{x!\langle P \rangle} := \{ x \} \cup \{ P \} 
  \and \\
  \freenames{P|Q} := \freenames{P} \cup \freenames{Q}
  \and \\
  \freenames{@{x}} := \{ x \}
\end{mathpar}

$\pi$
$\quotep{\pi}$

$\freenames{-} : \pi \to \mathcal{P}(\quotep{\pi})$

\begin{eqnarray*}
  \freenames{\pzero} & := & \emptyset \\
  \freenames{x?(y).P} & := & \{ x \} \cup (\freenames{P} \setminus \{ y \}) \\
  \freenames{x!\langle P \rangle} & := & \{ x \} \cup \{ P \} \\
  \freenames{P|Q} & := & \freenames{P} \cup \freenames{Q} \\
  \freenames{\dropn{x}} & := & \{ x \}
\end{eqnarray*}

The bound names of a process, $\boundnames{P}$, are those names occurring in $P$
that are not free. For example, in $x?(y).0$, the name $x$ is free, while $y$ is bound.

\begin{mathpar}
  \inferrule* [lab=monoidal-laws] {} { P|Q \equiv Q|P \and P|0 \equiv P \and P|(Q|R) \equiv (P|Q)|R }
\end{mathpar}

\begin{mathpar}
  \inferrule* [lab=alpha-equivalence] {} { (x)P \equiv (y)P\{y/x\} \and y \not\in \freenames{P} }
\end{mathpar}

\begin{definition}
Then two processes, $P,Q$, are alpha-equivalent if $P = Q\{\vec{y}/\vec{x}\}$ for
some $\vec{x} \in \boundnames{Q},\vec{y} \in \boundnames{P}$, where $Q\{\vec{y}/\vec{x}\}$
denotes the capture-avoiding substitution of $\vec{y}$ for $\vec{x}$ in $Q$.
\end{definition}

\begin{definition}
  The {\em structural congruence} \cite{SangiorgiWalker} , $\equiv$,
  between processes is the least congruence containing
  alpha-equivalence, satisfying the abelian monoid laws
  (associativity, commutativity and $\pzero$ as identity) for parallel
  composition $|$ and for summation $+$.
\end{definition}

\subsection{Name equivalence}

We take name equivalence, written $\nameeq$, to be the smallest
equivalence relation generated by the following rules.

\begin{mathpar}
\inferrule*[lab=Quote-drop]
{ }
{ \quotep{@{x}} \nameeq x }

\inferrule*[lab=Struct-equiv]
{ P \scong Q }
{ \quotep{P} \nameeq \quotep{Q} }
\end{mathpar}

The astute reader will have noticed that the mutual recursion of names
and processes imposes a mutual recursion on alpha-equivalence and
structural equivalence via name-equivalence. Fortunately, all of this
works out pleasantly and we may calculate in the natural way, free of
concern. The reader interested in the details is referred to the
appendix \ref{appendix:rho_details}.

\subsection{Substitution}

We use $\Proc$ for the set of processes, $\QProc$ for the set of
names, and $\id{\{}\vec{y} / \vec{x} \id{\}}$ to denote partial maps,
$s : \QProc \rightarrow \QProc$. A map, $s$ lifts, uniquely, to a map
on process terms, $\widehat{s} : \Proc \rightarrow \Proc$ by the
following equations.

\begin{mathpar}
  (0) \psubstp{Q}{P} := 0 \\
  (R \juxtap S) \psubstp{Q}{P}
  :=    
  (R)\psubstp{Q}{P} \juxtap (S) \psubstp{Q}{P} \\
  (x?(y).R) \psubstp{Q}{P}    
  :=    
  (x)\substp{Q}{P} (z)\concat( (R \psubstn{z}{y}) \psubstp{Q}{P} ) \\
  (\lift{x}{R}) \psubstp{Q}{P}  
  :=
  \lift{(x)\substp{Q}{P}}{ R \psubstp{Q}{P} } \\
%   (\dropn{x})  \psubstp{Q}{P}       
%   := 
%   \left\{ 
%     \begin{array}{ccc} 
%       \dropn{\quotep{Q}} & & x \nameeq \quotep{P} \\
%       \dropn{x} & & otherwise \\
%     \end{array}
%   \right. 
  (\dropn{x})  \psubstp{Q}{P}       
  := 
  \left\{ 
    \begin{array}{ccc} 
      Q & & x \nameeq \quotep{P} \\
      \dropn{x} & & otherwise \\
    \end{array}
  \right.
\end{mathpar}
 

where

\begin{eqnarray}
  (x)\id{\{} \lpquote Q \rpquote / \lpquote P \rpquote \id{\}}            = 
  \left\{ 
    \begin{array}{ccc}
      \lpquote Q \rpquote & & x \nameeq \lpquote P \rpquote \\
      x & & otherwise \\
    \end{array}
  \right. \nonumber
\end{eqnarray}

and $z$ is chosen distinct from $\quotep{P}$, $\quotep{Q}$, the free
names in $Q$, and all the names in $R$. Our $\alpha$-equivalence will
be built in the standard way from this substitution.

\begin{remark}\label{rem:no_self_referential_names}
  One consequence of these definitions is that $\forall P. \quotep{P}
  \not\in \freenames{P}$.
\end{remark}

\subsection{ Dynamic quote: an example }

Anticipating something of what's to come, consider applying the
substitution, $\widehat{\id{\{}u / z \id{\}}}$, to the following pair
of processes, $\lift{w}{y!(z)}$ and $w[ \lpquote y!(z) \rpquote ]$.

\begin{eqnarray}
	\lift{w}{y!(z)}\widehat{\id{\{}u / z \id{\}}}
		& = &
		\lift{w}{y!(u)} \nonumber\\
	w[ \lpquote y!(z) \rpquote ] \widehat{ \id{\{}u / z \id{\}} }
		& = &
		w[ \lpquote y!(z) \rpquote ] \nonumber
\end{eqnarray}

Because the body of the process between quotes is impervious to
substitution, we get radically different answers. In fact, by
examining the first process in an input context,
e.g. $x?(z).\lift{w}{y!(z)}$, we see that the process under the lift
operator may be shaped by prefixed inputs binding a name inside it. In
this sense, the lift operator will be seen as a way to dynamically
construct processes before reifying them as names.

Finally equipped with these standard features we can present the
dynamics of the calculus.

\subsubsection{Operational semantics} 

Finally, we introduce the computational dynamics. What marks these
algebras as distinct from other more traditionally studied algebraic
structures, e.g. vector spaces or polynomial rings, is the manner in
which dynamics is captured. In traditional structures, dynamics is typically
expressed through morphisms between such structures, as in linear maps
between vector spaces or morphisms between rings. In algebras
associated with the semantics of computation, the dynamics is
expressed as part of the algebraic structure itself, through a
reduction reduction relation typically denoted by $\red$. Below, we
give a recursive presentation of this relation for the calculus used
in the encoding.

$\red \subseteq \pi \times \pi$
$\red : \pi \to \mathcal{P}(\pi)$

\begin{mathpar}
  \inferrule* [lab=Comm] { \textsf{match}( x_{src}, x_{trgt} ) } { x_{trgt}?(y)P \; | \; x_{src}!\langle {Q} \rangle \red P\{\quotep{Q}/y}\} }
  \and \\
  \inferrule* [lab=Par] {{P} \red {P}'} {{{P} | {Q}} \red {{P}' | {Q}}}
  \and
  \inferrule* [lab=Equiv]{{{P} \scong {P}'} \andalso {{P}' \red {Q}'} \andalso {{Q}' \scong {Q}}}{{P} \red {Q}}
\end{mathpar}

\begin{eqnarray*}
  match_{\equiv} (\quotep{P},\quotep{Q}) & := & P \equiv Q \\
  match_{\dagger}(\quotep{P},\quotep{Q}) & := & \forall R. P|Q \red^{*} R => R \red^{*} 0 \\
  match_{K}(\quotep{P},\quotep{Q}) & := & K \mbox{ for some context } K
\end{eqnarray*}

$u?(x)P | u!\langle Q \rangle \red P\{\quotep{Q}/x\}$

%We write $\wred$ for $\red^*$, and $P\red$ if $\exists Q $ such that $ P \red Q$.
We write $P\red$ if $\exists Q $ such that $ P \red Q$ and $P\not\red$, otherwise.

\section{Replication}

As mentioned before, it is known that replication (and hence
recursion) can be implemented in a higher-order process algebra
\cite{SangiorgiWalker}. As our first example of calculation with the
machinery thus far presented we give the construction explicitly in
the {\rhoc}.

\begin{eqnarray}
	D_{x} & := & \prefix{x}{y}{(\binpar{\outputp{x}{y}}{@{y}})} \nonumber\\
	\bangp_{x}{P} & := & \binpar{{x}!\langle{\binpar{D_{x}}{P}}\rangle}{D_{x}} \nonumber
\end{eqnarray}

\begin{eqnarray}
	\bangp_{x}{P} & & \nonumber\\
	=
	& {x}!\langle{(\prefix{x}{y}{(\outputp{x}{y} | @{y})) | P}}\rangle 
	      | \prefix{x}{y}{(\outputp{x}{y} | @{y})} & \nonumber\\
	\red
	& (\outputp{x}{y} | @{y})\substn{\quotep{(\prefix{x}{y}{(@{y} | \outputp{x}{y})) | P}}}{y} & \nonumber\\
	=
	& \outputp{x}{\quotep{(\prefix{x}{y}{(\outputp{x}{y} | @{y})) | P}}}
	  | {(\prefix{x}{y}{(\outputp{x}{y} | @{y})) | P}} & \nonumber\\
	\red
	& \ldots & \nonumber\\
	\red^*
	& P | P | \ldots & \nonumber
\end{eqnarray}

Of course, this encoding, as an implementation, runs away, unfolding
$\bangp{P}$ eagerly. A lazier and more implementable replication
operator, restricted to input-guarded processes, may be obtained as follows.

\begin{eqnarray}
\bangp{\prefix{u}{v}{P}} 
	:= 
	\binpar{\lift{x}{\prefix{u}{v}{(\binpar{D(x)}{P})}}}{D(x)} \nonumber
\end{eqnarray}

\begin{remark}
  Note that the lazier definition still does not deal with summation
  or mixed summation (i.e. sums over input and output). The reader is
  invited to construct definitions of replication that deal with these
  features. 

  Further, the definitions are parameterized in a name, $x$. Can you,
  gentle reader, make a definition that eliminates this parameter and
  guarantees no accidental interaction between the replication
  machinery and the process being replicated -- i.e. no accidental
  sharing of names used by the process to get its work done and the
  name(s) used by the replication to effect copying. This latter
  revision of the definition of replication is crucial to obtaining
  the expected identity $!!P \sim !P$.
\end{remark}

\begin{remark}\label{rem:paradoxical_combinator}
  The reader familiar with the lambda calculus will have noticed the
  similarity between $D$ and the paradoxical combinator.

  [Ed. note: the existence of this seems to suggest we have to be more
  restrictive on the set of processes and names we admit if we are to
  support no-cloning.]
\end{remark}

\subsubsection{Bisimulation}

The computational dynamics gives rise to another kind of equivalence,
the equivalence of computational behavior. As previously mentioned
this is typically captured \emph{via} some form of bisimulation.

% The notion we use in this paper is weak barbed bisimulation
% \cite{milner91polyadicpi}.

The notion we use in this paper is derived from weak barbed
bisimulation \cite{milner91polyadicpi}. 

\begin{definition}
An \emph{observation relation}, $\downarrow_{\mathcal N}$, over a set
of names, $\mathcal N$, is the smallest relation satisfying the rules
below.

\infrule[Out-barb]{y \in {\mathcal N}, \; x \nameeq y}
		  {\outputp{x}{v} \downarrow_{\mathcal N} x}
\infrule[Par-barb]{\mbox{$P\downarrow_{\mathcal N} x$ or $Q\downarrow_{\mathcal N} x$}}
		  {\binpar{P}{Q} \downarrow_{\mathcal N} x}

We write $P \Downarrow_{\mathcal N} x$ if there is $Q$ such that 
$P \wred Q$ and $Q \downarrow_{\mathcal N} x$.
\end{definition}

\begin{definition}
%\label{def.bbisim}
An  ${\mathcal N}$-\emph{barbed bisimulation} over a set of names, ${\mathcal N}$, is a symmetric binary relation 
${\mathcal S}_{\mathcal N}$ between agents such that $P\rel{S}_{\mathcal N}Q$ implies:
\begin{enumerate}
\item If $P \red P'$ then $Q \wred Q'$ and $P'\rel{S}_{\mathcal N} Q'$.
\item If $P\downarrow_{\mathcal N} x$, then $Q\Downarrow_{\mathcal N} x$.
\end{enumerate}
$P$ is ${\mathcal N}$-barbed bisimilar to $Q$, written
$P \wbbisim_{\mathcal N} Q$, if $P \rel{S}_{\mathcal N} Q$ for some ${\mathcal N}$-barbed bisimulation ${\mathcal S}_{\mathcal N}$.
\end{definition}

$\mathcal{R} \subseteq \pi \times \pi$

$P \mathcal{R} Q => \forall P'. P \red P' \Rightarrow \exists Q'. Q \red Q', P' \mathcal{R} Q'$

$P \vdash x \Rightarrow Q \vdash x$

\begin{mathpar}
  \inferrule*[lab=Out-barb]{x \nameeq y}{{y}!\langle{Q}\rangle \vdash x}
  \and
  \inferrule*[lab=Par-barb]{\mbox{$P\vdash x$ or $Q\vdash x$}}{\binpar{P}{Q} \vdash x}
\end{mathpar}

\subsubsection{Contexts}

One of the principle advantages of computational calculi like the
$\pi$-calculus is a well-defined notion of context,
contextual-equivalence and a correlation between
contextual-equivalence and notions of bisimulation. The notion of
context allows the decomposition of a process into (sub-)process and
its syntactic environment, its context. Thus, a context may be
thought of as a process with a ``hole'' (written $\Box$) in it. The
application of a context $M$ to a process $P$, written $M[P]$, is
tantamount to filling the hole in $M$ with $P$. In this paper we do
not need the full weight of this theory, but do make use of the notion
of context in the proof the main theorem. 

\begin{mathpar}
  \inferrule* [lab=summation] {} {{M_{M},M_{N}} \bc \Box \;|\; x.M_{A} \;|\; M_{M}+M_{N}}
  \and
  \inferrule* [lab=agent] {} {{M_{A}} \bc (\vec{x})M_{P} \;| \; \clift{P_0,\ldots,M_{P},\ldots,P_N}}
  \and \\
  \inferrule* [lab=process] {} {{M_{P}} \bc M_{N} \;| \;P|M_{P} }
\end{mathpar} 

\begin{mathpar}
  \inferrule* [lab=sychronization] {} {M_{N} \bc \Box \;|\; x?M_{F} \;|\; x!M_{C}}
  \and
  \inferrule* [lab=abstraction] {} {{M_{F}} \bc (x)M_{P} }
  \and
  \inferrule* [lab=concretion] {} {{M_{C}} \bc \langle M_{P} \rangle }
  \and \\
  \inferrule* [lab=process] {} {{M_{P}} \bc M_{N} \;| \;P|M_{P} }
\end{mathpar}

\begin{definition}[contextual application] Given a context $M$, and
  process $P$, we define the \emph{contextual application}, $M[P] :=
  M\{P/\Box\}$. That is, the contextual application of M to P is the
  substitution of $P$ for $\Box$ in $M$.
\end{definition}

$\meaningof{-} : L \to \mathcal{P}(\pi)$

\begin{mathpar}
  \inferrule* [lab=collection] {} {\meaningof{true} = \pi, \and \meaningof{~E} = \pi \setminus \meaningof{E}, \and \meaningof{E_{1} \& E_{2}} = \meaningof{E_{1}} \cap \meaningof{E_{2}}}
\end{mathpar}

\begin{mathpar}
  \inferrule* [lab=structure] {} {\meaningof{0} = \{ P \in \pi | P \equiv 0 \}, \and \\ \meaningof{E_1 | E_2} = \{ P \in \pi | P \equiv P_{1} | P_{2}, P_{1} \in \meaningof{E_{1}}, P_{2} \in \meaningof{E_2}\} }
\end{mathpar}

\begin{mathpar}
 \inferrule* [lab=behavior] {} {\meaningof{\langle a?b \rangle E} = \{ P \in \pi | P \equiv Q | u?(y)P', \\ \and \\\\ \and \\ \;\;\; u \in \meaningof{a}, \forall z.P'\{z/y\} \in \meaningof{E\{z/b\}}\}, \and \\ \meaningof{a!E} = \{ P \in \pi | P \equiv Q | x!\langle P' \rangle, x \in \meaningof{a} P' \in \meaningof{E}\} }
\end{mathpar}

\begin{mathpar}
 \inferrule* [lab=nominal] {} {\meaningof{\quotep{E}} = \{ \quotep{P} \in \quotep{\pi} | P \in \meaningof{E} \}, \and \meaningof{\quotep{P}} = \{ \quotep{Q} \in \quotep{\pi} | P \equiv Q \} \and \\ \meaningof{@\quotep{E}} = \{ P \in \pi | P \equiv @x, x \in \meaningof{E} \}}
\end{mathpar}

\begin{eqnarray*}
  \\
  \meaningof{-} : TS \to ST
\end{eqnarray*}

\begin{eqnarray*}
  \\
  L : TS \to ST
\end{eqnarray*}

\begin{eqnarray*}
  \\
  P \models E \iff P \in \meaningof{E}
\end{eqnarray*}

\begin{eqnarray*}
  P \approx_{L} Q \iff \forall E \in L. P \models E \iff Q \models E
\end{eqnarray*}

\begin{eqnarray*}
  P \approx_{K} Q
\end{eqnarray*}

\begin{eqnarray*}
  P \approx Q
\end{eqnarray*}

$\approx_{K} = \approx = \approx_{L}$

\subsubsection{Contextual duality}

Note that contexts extend the quotation operation to a family of
operations from processes to names. Given a context, $M$, we can
define a \emph{nominal context}, $\quotep{M}$ by $\quotep{M}[P] :=
\quotep{M[P]}$. To foreshadow what is to come we observe that these
operations enjoy a duality with processes very much like the duality
between vectors and maps from vectors to scalars.

Further, because the calculus is essentially higher-order, we have a
correspondence between contexts and processes. More specifically,
given a name $x$ and a context $M$ we can construct $M^{*}_{x}$ such
that 

\begin{mathpar}
  M^{*}_{x} | \lift{x}{P} \red M[P]
\end{mathpar}

namely,

\begin{mathpar}
  M^{*}_{x} := x?(u).M[\dropn{u}]
\end{mathpar}

The dependence of $M^{*}_{x}$ on a name makes it an abstraction, 

\begin{mathpar}
  M^{*} := (x)x?(u).M[\dropn{u}]
\end{mathpar}

\subsection{Additional notation}

It will sometimes be convenient to denote the process a name
quotes. We already have the notation $x = \quotep{P}$, but it will be
convenient to introduce an alternate notation, $\procn{x}$, when we
want to emphasize the connection to the use of the name. Note that, by
virtue of name equivalence, $\quotep{\procn{x}} \nameeq x$; so, the
notation is consistent with previous definitions.

Further, because names have structure it is possible to effect
substitutions on the basis of that structure. This means we need to
upgrade our notation for substitutions, which we accomplish by
adapting comprehension notation. Thus,

\begin{mathpar}
  P\{ y / x : x \in S \}
\end{mathpar}

is interpreted to mean the process derived from P by replacing (in a
capture-avoiding manner) each occurrence of $x$ in $S$ by $y$. For example,

\begin{mathpar}
  P\{ \quotep{\procn{x}|\procn{x}} / x : x \in \freenames{P} \}
\end{mathpar}

will replace each (occurrence) of a free name $x$ in $P$ by
$\quotep{\procn{x}|\procn{x}}$.

Also, we will avail ourselves of the notation $x^{L}$ and $x^{R}$ to
denote injections of a name into disjoint copies of the name
space. There are numerous ways to accomplish this. One example can be
found in \cite{MeredithR05}. This notation overloads to vectors of
names: $\vec{x}^{\pi} := (x_{i}^{\pi} \; : \; 0 \leq i < |\vec{x}| )$ where $\pi \in \{L,R\}$.

We also use $P^{\Box} := P|\Box$.

In \cite{MeredithR05} an interpretation of the new operator is
given. It turns out that there are several possible interpretations
all enjoying the requisite algebraic properties of the operator (see
\cite{milner91polyadicpi}). We will therefore make liberal use of
$(\nu\; \vec{x})P$.

% subsection the_syntax_and_semantics_of_the_notation_system (end)   

\input{qm2pi.qmops} 

\input{qm2pi.sterngerlach} 

\input{qm2pi.metric} 

% section concurrent_process_calculi (end)

%\input{qm2pi.proofsketch}

% section proof sketch (end)

%\input{qm2pi.slviaknots} 

% section spatial logic via knots (end)

\input{qm2pi.conclusion}

% section conclusion (end)

%\input{qm2pi.dtcodes} 

% section wiring algorithm (end)

\input{qm2pi.ack} 

% section acknowledgments (end)

\newpage


\bibliographystyle{plain}   
\bibliography{../../biblios/main.bib}

\input{qm2pi.rhodetails}

\end{document}

 

%\documentclass[12pt]{llncs}
%\documentclass{jktr}

\usepackage[pdftex]{hyperref}                   
\usepackage {listings}
\usepackage {mathpartir}
\usepackage{bcprules}
%\usepackage{listings}
                       
\usepackage{graphicx} 
%\usepackage[margins=2.5cm,nohead,nofoot]{geometry}
%\usepackage{geometry}
\usepackage{amsfonts}
\usepackage{amstext}
\usepackage{latexsym}
\usepackage{amssymb}
\usepackage{color}


%\include{myPreamble}
\include{qm2pi.local} 

%\ifpdf
%\usepackage[pdftex]{graphicx}
%\else
%\usepackage{graphicx}
%\fi

 % \ifpdf
%  \usepackage{pdfsync}
%  \if


%\title{Brief Article}
%\author{David F. Snyder}
%\author{L.G. Meredith}

%\address{Dept. of Math., Texas State University--San Marcos, San Marcos, TX 78666}
       
\pagestyle{empty}


\begin{document}

\lstset{language=[Objective]Caml,frame=shadowbox}

\input{qm2pi.front}

% section front matter (end)

\input{qm2pi.intro} 
 
% section introduction (end)

% \input{qm2pi.knotations} 

% section notation (end)

\input{qm2pi.process.calculi} 

% section concurrent_process_calculi_and_spatial_logics_ (end)
    
%\input{qm2pi.knots2pi} 

%\input{qm2pi.trefoil} 

%\input{qm2pi.mainthm} 

% subsection basic_interpretation (end)

%\input{qm2pi.rho.presentation} 
\subsection{The syntax and semantics of the notation system}\label{sub:the_syntax_and_semantics_of_the_notation_system} % (fold)

We now summarize a technical presentation of the calculus that
embodies our theory of dynamics. The typical presentation of such a
calculus follows the style of giving generators and relations on
them. The grammar, below, describing term constructors, freely
generates the set of processes, $\Proc$. This set is then quotiented
by a relation known as structural congruence and it is over this set
that the notion of dynamics is expressed. This presentation is
essentially that of \cite{MeredithR05} with the addition of
polyadicity and summation. For readability we have relegated some of
the technical subtleties to an appendix.

\subsubsection{Process grammar}\label{subsub:process_grammar}

\begin{mathpar}
  \inferrule* [lab=synchronization] {} {{M} \bc \pzero \;|\; x?F \;|\; x!C }
  \and
  \inferrule* [lab=abstraction] {} {{F} \bc (x)P}
  \and
  \inferrule* [lab=concretion] {} {{C} \bc \langle Q \rangle}
  \and
  \inferrule* [lab=process] {} {{P,Q} \bc M \;| \;P|Q \;|\; @{x}}
  \and
  \inferrule* [lab=name] {} {{x} \bc \quotep{P}}
\end{mathpar} 

Note that $\vec{x}$ (resp. $\vec{P}$) denotes a vector of names
(resp. processes) of length $|\vec{x}|$ (resp. $|\vec{P}|$). We adopt
the following useful abbreviations.

\begin{mathpar}
   x?(\vec{y}).P := x.(\vec{y})P \and  x\clift{\vec{P}} := x.\clift{\vec{P}}
   \and x!(y) := \lift{x}{\dropn{y}}
   \and \Pi_{i=0}^{n-1}P_i := P_0 | \ldots | P_{n-1}
\end{mathpar}

\subsubsection{Structural congruence}

\paragraph{Free and bound names and alpha-equivalence.} At the
core of structural equivalence is alpha-equivalence which identifies
process that are the same up to a change of variable. Formally, we
recognize the distinction between free and bound names. The free names
of a process, $\freenames{P}$, may be calculated recursively as
follows:

\begin{mathpar}
\freenames{\pzero} := \emptyset
  \and \\
  \freenames{x?(y).P} := \{ x \} \cup (\freenames{P} \setminus \{ y \})
  \and 
  \freenames{x!\langle P \rangle} := \{ x \} \cup \{ P \} 
  \and \\
  \freenames{P|Q} := \freenames{P} \cup \freenames{Q}
  \and \\
  \freenames{@{x}} := \{ x \}
\end{mathpar}

$\pi$
$\quotep{\pi}$

$\freenames{-} : \pi \to \mathcal{P}(\quotep{\pi})$

\begin{eqnarray*}
  \freenames{\pzero} & := & \emptyset \\
  \freenames{x?(y).P} & := & \{ x \} \cup (\freenames{P} \setminus \{ y \}) \\
  \freenames{x!\langle P \rangle} & := & \{ x \} \cup \{ P \} \\
  \freenames{P|Q} & := & \freenames{P} \cup \freenames{Q} \\
  \freenames{\dropn{x}} & := & \{ x \}
\end{eqnarray*}

The bound names of a process, $\boundnames{P}$, are those names occurring in $P$
that are not free. For example, in $x?(y).0$, the name $x$ is free, while $y$ is bound.

\begin{mathpar}
  \inferrule* [lab=monoidal-laws] {} { P|Q \equiv Q|P \and P|0 \equiv P \and P|(Q|R) \equiv (P|Q)|R }
\end{mathpar}

\begin{mathpar}
  \inferrule* [lab=alpha-equivalence] {} { (x)P \equiv (y)P\{y/x\} \and y \not\in \freenames{P} }
\end{mathpar}

\begin{definition}
Then two processes, $P,Q$, are alpha-equivalent if $P = Q\{\vec{y}/\vec{x}\}$ for
some $\vec{x} \in \boundnames{Q},\vec{y} \in \boundnames{P}$, where $Q\{\vec{y}/\vec{x}\}$
denotes the capture-avoiding substitution of $\vec{y}$ for $\vec{x}$ in $Q$.
\end{definition}

\begin{definition}
  The {\em structural congruence} \cite{SangiorgiWalker} , $\equiv$,
  between processes is the least congruence containing
  alpha-equivalence, satisfying the abelian monoid laws
  (associativity, commutativity and $\pzero$ as identity) for parallel
  composition $|$ and for summation $+$.
\end{definition}

\subsection{Name equivalence}

We take name equivalence, written $\nameeq$, to be the smallest
equivalence relation generated by the following rules.

\begin{mathpar}
\inferrule*[lab=Quote-drop]
{ }
{ \quotep{@{x}} \nameeq x }

\inferrule*[lab=Struct-equiv]
{ P \scong Q }
{ \quotep{P} \nameeq \quotep{Q} }
\end{mathpar}

The astute reader will have noticed that the mutual recursion of names
and processes imposes a mutual recursion on alpha-equivalence and
structural equivalence via name-equivalence. Fortunately, all of this
works out pleasantly and we may calculate in the natural way, free of
concern. The reader interested in the details is referred to the
appendix \ref{appendix:rho_details}.

\subsection{Substitution}

We use $\Proc$ for the set of processes, $\QProc$ for the set of
names, and $\id{\{}\vec{y} / \vec{x} \id{\}}$ to denote partial maps,
$s : \QProc \rightarrow \QProc$. A map, $s$ lifts, uniquely, to a map
on process terms, $\widehat{s} : \Proc \rightarrow \Proc$ by the
following equations.

\begin{mathpar}
  (0) \psubstp{Q}{P} := 0 \\
  (R \juxtap S) \psubstp{Q}{P}
  :=    
  (R)\psubstp{Q}{P} \juxtap (S) \psubstp{Q}{P} \\
  (x?(y).R) \psubstp{Q}{P}    
  :=    
  (x)\substp{Q}{P} (z)\concat( (R \psubstn{z}{y}) \psubstp{Q}{P} ) \\
  (\lift{x}{R}) \psubstp{Q}{P}  
  :=
  \lift{(x)\substp{Q}{P}}{ R \psubstp{Q}{P} } \\
%   (\dropn{x})  \psubstp{Q}{P}       
%   := 
%   \left\{ 
%     \begin{array}{ccc} 
%       \dropn{\quotep{Q}} & & x \nameeq \quotep{P} \\
%       \dropn{x} & & otherwise \\
%     \end{array}
%   \right. 
  (\dropn{x})  \psubstp{Q}{P}       
  := 
  \left\{ 
    \begin{array}{ccc} 
      Q & & x \nameeq \quotep{P} \\
      \dropn{x} & & otherwise \\
    \end{array}
  \right.
\end{mathpar}
 

where

\begin{eqnarray}
  (x)\id{\{} \lpquote Q \rpquote / \lpquote P \rpquote \id{\}}            = 
  \left\{ 
    \begin{array}{ccc}
      \lpquote Q \rpquote & & x \nameeq \lpquote P \rpquote \\
      x & & otherwise \\
    \end{array}
  \right. \nonumber
\end{eqnarray}

and $z$ is chosen distinct from $\quotep{P}$, $\quotep{Q}$, the free
names in $Q$, and all the names in $R$. Our $\alpha$-equivalence will
be built in the standard way from this substitution.

\begin{remark}\label{rem:no_self_referential_names}
  One consequence of these definitions is that $\forall P. \quotep{P}
  \not\in \freenames{P}$.
\end{remark}

\subsection{ Dynamic quote: an example }

Anticipating something of what's to come, consider applying the
substitution, $\widehat{\id{\{}u / z \id{\}}}$, to the following pair
of processes, $\lift{w}{y!(z)}$ and $w[ \lpquote y!(z) \rpquote ]$.

\begin{eqnarray}
	\lift{w}{y!(z)}\widehat{\id{\{}u / z \id{\}}}
		& = &
		\lift{w}{y!(u)} \nonumber\\
	w[ \lpquote y!(z) \rpquote ] \widehat{ \id{\{}u / z \id{\}} }
		& = &
		w[ \lpquote y!(z) \rpquote ] \nonumber
\end{eqnarray}

Because the body of the process between quotes is impervious to
substitution, we get radically different answers. In fact, by
examining the first process in an input context,
e.g. $x?(z).\lift{w}{y!(z)}$, we see that the process under the lift
operator may be shaped by prefixed inputs binding a name inside it. In
this sense, the lift operator will be seen as a way to dynamically
construct processes before reifying them as names.

Finally equipped with these standard features we can present the
dynamics of the calculus.

\subsubsection{Operational semantics} 

Finally, we introduce the computational dynamics. What marks these
algebras as distinct from other more traditionally studied algebraic
structures, e.g. vector spaces or polynomial rings, is the manner in
which dynamics is captured. In traditional structures, dynamics is typically
expressed through morphisms between such structures, as in linear maps
between vector spaces or morphisms between rings. In algebras
associated with the semantics of computation, the dynamics is
expressed as part of the algebraic structure itself, through a
reduction reduction relation typically denoted by $\red$. Below, we
give a recursive presentation of this relation for the calculus used
in the encoding.

$\red \subseteq \pi \times \pi$
$\red : \pi \to \mathcal{P}(\pi)$

\begin{mathpar}
  \inferrule* [lab=Comm] { \textsf{match}( x_{src}, x_{trgt} ) } { x_{trgt}?(y)P \; | \; x_{src}!\langle {Q} \rangle \red P\{\quotep{Q}/y}\} }
  \and \\
  \inferrule* [lab=Par] {{P} \red {P}'} {{{P} | {Q}} \red {{P}' | {Q}}}
  \and
  \inferrule* [lab=Equiv]{{{P} \scong {P}'} \andalso {{P}' \red {Q}'} \andalso {{Q}' \scong {Q}}}{{P} \red {Q}}
\end{mathpar}

\begin{eqnarray*}
  match_{\equiv} (\quotep{P},\quotep{Q}) & := & P \equiv Q \\
  match_{\dagger}(\quotep{P},\quotep{Q}) & := & \forall R. P|Q \red^{*} R => R \red^{*} 0 \\
  match_{K}(\quotep{P},\quotep{Q}) & := & K \mbox{ for some context } K
\end{eqnarray*}

$u?(x)P | u!\langle Q \rangle \red P\{\quotep{Q}/x\}$

%We write $\wred$ for $\red^*$, and $P\red$ if $\exists Q $ such that $ P \red Q$.
We write $P\red$ if $\exists Q $ such that $ P \red Q$ and $P\not\red$, otherwise.

\section{Replication}

As mentioned before, it is known that replication (and hence
recursion) can be implemented in a higher-order process algebra
\cite{SangiorgiWalker}. As our first example of calculation with the
machinery thus far presented we give the construction explicitly in
the {\rhoc}.

\begin{eqnarray}
	D_{x} & := & \prefix{x}{y}{(\binpar{\outputp{x}{y}}{@{y}})} \nonumber\\
	\bangp_{x}{P} & := & \binpar{{x}!\langle{\binpar{D_{x}}{P}}\rangle}{D_{x}} \nonumber
\end{eqnarray}

\begin{eqnarray}
	\bangp_{x}{P} & & \nonumber\\
	=
	& {x}!\langle{(\prefix{x}{y}{(\outputp{x}{y} | @{y})) | P}}\rangle 
	      | \prefix{x}{y}{(\outputp{x}{y} | @{y})} & \nonumber\\
	\red
	& (\outputp{x}{y} | @{y})\substn{\quotep{(\prefix{x}{y}{(@{y} | \outputp{x}{y})) | P}}}{y} & \nonumber\\
	=
	& \outputp{x}{\quotep{(\prefix{x}{y}{(\outputp{x}{y} | @{y})) | P}}}
	  | {(\prefix{x}{y}{(\outputp{x}{y} | @{y})) | P}} & \nonumber\\
	\red
	& \ldots & \nonumber\\
	\red^*
	& P | P | \ldots & \nonumber
\end{eqnarray}

Of course, this encoding, as an implementation, runs away, unfolding
$\bangp{P}$ eagerly. A lazier and more implementable replication
operator, restricted to input-guarded processes, may be obtained as follows.

\begin{eqnarray}
\bangp{\prefix{u}{v}{P}} 
	:= 
	\binpar{\lift{x}{\prefix{u}{v}{(\binpar{D(x)}{P})}}}{D(x)} \nonumber
\end{eqnarray}

\begin{remark}
  Note that the lazier definition still does not deal with summation
  or mixed summation (i.e. sums over input and output). The reader is
  invited to construct definitions of replication that deal with these
  features. 

  Further, the definitions are parameterized in a name, $x$. Can you,
  gentle reader, make a definition that eliminates this parameter and
  guarantees no accidental interaction between the replication
  machinery and the process being replicated -- i.e. no accidental
  sharing of names used by the process to get its work done and the
  name(s) used by the replication to effect copying. This latter
  revision of the definition of replication is crucial to obtaining
  the expected identity $!!P \sim !P$.
\end{remark}

\begin{remark}\label{rem:paradoxical_combinator}
  The reader familiar with the lambda calculus will have noticed the
  similarity between $D$ and the paradoxical combinator.

  [Ed. note: the existence of this seems to suggest we have to be more
  restrictive on the set of processes and names we admit if we are to
  support no-cloning.]
\end{remark}

\subsubsection{Bisimulation}

The computational dynamics gives rise to another kind of equivalence,
the equivalence of computational behavior. As previously mentioned
this is typically captured \emph{via} some form of bisimulation.

% The notion we use in this paper is weak barbed bisimulation
% \cite{milner91polyadicpi}.

The notion we use in this paper is derived from weak barbed
bisimulation \cite{milner91polyadicpi}. 

\begin{definition}
An \emph{observation relation}, $\downarrow_{\mathcal N}$, over a set
of names, $\mathcal N$, is the smallest relation satisfying the rules
below.

\infrule[Out-barb]{y \in {\mathcal N}, \; x \nameeq y}
		  {\outputp{x}{v} \downarrow_{\mathcal N} x}
\infrule[Par-barb]{\mbox{$P\downarrow_{\mathcal N} x$ or $Q\downarrow_{\mathcal N} x$}}
		  {\binpar{P}{Q} \downarrow_{\mathcal N} x}

We write $P \Downarrow_{\mathcal N} x$ if there is $Q$ such that 
$P \wred Q$ and $Q \downarrow_{\mathcal N} x$.
\end{definition}

\begin{definition}
%\label{def.bbisim}
An  ${\mathcal N}$-\emph{barbed bisimulation} over a set of names, ${\mathcal N}$, is a symmetric binary relation 
${\mathcal S}_{\mathcal N}$ between agents such that $P\rel{S}_{\mathcal N}Q$ implies:
\begin{enumerate}
\item If $P \red P'$ then $Q \wred Q'$ and $P'\rel{S}_{\mathcal N} Q'$.
\item If $P\downarrow_{\mathcal N} x$, then $Q\Downarrow_{\mathcal N} x$.
\end{enumerate}
$P$ is ${\mathcal N}$-barbed bisimilar to $Q$, written
$P \wbbisim_{\mathcal N} Q$, if $P \rel{S}_{\mathcal N} Q$ for some ${\mathcal N}$-barbed bisimulation ${\mathcal S}_{\mathcal N}$.
\end{definition}

$\mathcal{R} \subseteq \pi \times \pi$

$P \mathcal{R} Q => \forall P'. P \red P' \Rightarrow \exists Q'. Q \red Q', P' \mathcal{R} Q'$

$P \vdash x \Rightarrow Q \vdash x$

\begin{mathpar}
  \inferrule*[lab=Out-barb]{x \nameeq y}{{y}!\langle{Q}\rangle \vdash x}
  \and
  \inferrule*[lab=Par-barb]{\mbox{$P\vdash x$ or $Q\vdash x$}}{\binpar{P}{Q} \vdash x}
\end{mathpar}

\subsubsection{Contexts}

One of the principle advantages of computational calculi like the
$\pi$-calculus is a well-defined notion of context,
contextual-equivalence and a correlation between
contextual-equivalence and notions of bisimulation. The notion of
context allows the decomposition of a process into (sub-)process and
its syntactic environment, its context. Thus, a context may be
thought of as a process with a ``hole'' (written $\Box$) in it. The
application of a context $M$ to a process $P$, written $M[P]$, is
tantamount to filling the hole in $M$ with $P$. In this paper we do
not need the full weight of this theory, but do make use of the notion
of context in the proof the main theorem. 

\begin{mathpar}
  \inferrule* [lab=summation] {} {{M_{M},M_{N}} \bc \Box \;|\; x.M_{A} \;|\; M_{M}+M_{N}}
  \and
  \inferrule* [lab=agent] {} {{M_{A}} \bc (\vec{x})M_{P} \;| \; \clift{P_0,\ldots,M_{P},\ldots,P_N}}
  \and \\
  \inferrule* [lab=process] {} {{M_{P}} \bc M_{N} \;| \;P|M_{P} }
\end{mathpar} 

\begin{mathpar}
  \inferrule* [lab=sychronization] {} {M_{N} \bc \Box \;|\; x?M_{F} \;|\; x!M_{C}}
  \and
  \inferrule* [lab=abstraction] {} {{M_{F}} \bc (x)M_{P} }
  \and
  \inferrule* [lab=concretion] {} {{M_{C}} \bc \langle M_{P} \rangle }
  \and \\
  \inferrule* [lab=process] {} {{M_{P}} \bc M_{N} \;| \;P|M_{P} }
\end{mathpar}

\begin{definition}[contextual application] Given a context $M$, and
  process $P$, we define the \emph{contextual application}, $M[P] :=
  M\{P/\Box\}$. That is, the contextual application of M to P is the
  substitution of $P$ for $\Box$ in $M$.
\end{definition}

$\meaningof{-} : L \to \mathcal{P}(\pi)$

\begin{mathpar}
  \inferrule* [lab=collection] {} {\meaningof{true} = \pi, \and \meaningof{~E} = \pi \setminus \meaningof{E}, \and \meaningof{E_{1} \& E_{2}} = \meaningof{E_{1}} \cap \meaningof{E_{2}}}
\end{mathpar}

\begin{mathpar}
  \inferrule* [lab=structure] {} {\meaningof{0} = \{ P \in \pi | P \equiv 0 \}, \and \\ \meaningof{E_1 | E_2} = \{ P \in \pi | P \equiv P_{1} | P_{2}, P_{1} \in \meaningof{E_{1}}, P_{2} \in \meaningof{E_2}\} }
\end{mathpar}

\begin{mathpar}
 \inferrule* [lab=behavior] {} {\meaningof{\langle a?b \rangle E} = \{ P \in \pi | P \equiv Q | u?(y)P', \\ \and \\\\ \and \\ \;\;\; u \in \meaningof{a}, \forall z.P'\{z/y\} \in \meaningof{E\{z/b\}}\}, \and \\ \meaningof{a!E} = \{ P \in \pi | P \equiv Q | x!\langle P' \rangle, x \in \meaningof{a} P' \in \meaningof{E}\} }
\end{mathpar}

\begin{mathpar}
 \inferrule* [lab=nominal] {} {\meaningof{\quotep{E}} = \{ \quotep{P} \in \quotep{\pi} | P \in \meaningof{E} \}, \and \meaningof{\quotep{P}} = \{ \quotep{Q} \in \quotep{\pi} | P \equiv Q \} \and \\ \meaningof{@\quotep{E}} = \{ P \in \pi | P \equiv @x, x \in \meaningof{E} \}}
\end{mathpar}

\begin{eqnarray*}
  \\
  \meaningof{-} : TS \to ST
\end{eqnarray*}

\begin{eqnarray*}
  \\
  L : TS \to ST
\end{eqnarray*}

\begin{eqnarray*}
  \\
  P \models E \iff P \in \meaningof{E}
\end{eqnarray*}

\begin{eqnarray*}
  P \approx_{L} Q \iff \forall E \in L. P \models E \iff Q \models E
\end{eqnarray*}

\begin{eqnarray*}
  P \approx_{K} Q
\end{eqnarray*}

\begin{eqnarray*}
  P \approx Q
\end{eqnarray*}

$\approx_{K} = \approx = \approx_{L}$

\subsubsection{Contextual duality}

Note that contexts extend the quotation operation to a family of
operations from processes to names. Given a context, $M$, we can
define a \emph{nominal context}, $\quotep{M}$ by $\quotep{M}[P] :=
\quotep{M[P]}$. To foreshadow what is to come we observe that these
operations enjoy a duality with processes very much like the duality
between vectors and maps from vectors to scalars.

Further, because the calculus is essentially higher-order, we have a
correspondence between contexts and processes. More specifically,
given a name $x$ and a context $M$ we can construct $M^{*}_{x}$ such
that 

\begin{mathpar}
  M^{*}_{x} | \lift{x}{P} \red M[P]
\end{mathpar}

namely,

\begin{mathpar}
  M^{*}_{x} := x?(u).M[\dropn{u}]
\end{mathpar}

The dependence of $M^{*}_{x}$ on a name makes it an abstraction, 

\begin{mathpar}
  M^{*} := (x)x?(u).M[\dropn{u}]
\end{mathpar}

\subsection{Additional notation}

It will sometimes be convenient to denote the process a name
quotes. We already have the notation $x = \quotep{P}$, but it will be
convenient to introduce an alternate notation, $\procn{x}$, when we
want to emphasize the connection to the use of the name. Note that, by
virtue of name equivalence, $\quotep{\procn{x}} \nameeq x$; so, the
notation is consistent with previous definitions.

Further, because names have structure it is possible to effect
substitutions on the basis of that structure. This means we need to
upgrade our notation for substitutions, which we accomplish by
adapting comprehension notation. Thus,

\begin{mathpar}
  P\{ y / x : x \in S \}
\end{mathpar}

is interpreted to mean the process derived from P by replacing (in a
capture-avoiding manner) each occurrence of $x$ in $S$ by $y$. For example,

\begin{mathpar}
  P\{ \quotep{\procn{x}|\procn{x}} / x : x \in \freenames{P} \}
\end{mathpar}

will replace each (occurrence) of a free name $x$ in $P$ by
$\quotep{\procn{x}|\procn{x}}$.

Also, we will avail ourselves of the notation $x^{L}$ and $x^{R}$ to
denote injections of a name into disjoint copies of the name
space. There are numerous ways to accomplish this. One example can be
found in \cite{MeredithR05}. This notation overloads to vectors of
names: $\vec{x}^{\pi} := (x_{i}^{\pi} \; : \; 0 \leq i < |\vec{x}| )$ where $\pi \in \{L,R\}$.

We also use $P^{\Box} := P|\Box$.

In \cite{MeredithR05} an interpretation of the new operator is
given. It turns out that there are several possible interpretations
all enjoying the requisite algebraic properties of the operator (see
\cite{milner91polyadicpi}). We will therefore make liberal use of
$(\nu\; \vec{x})P$.

% subsection the_syntax_and_semantics_of_the_notation_system (end)   

\input{qm2pi.qmops} 

\input{qm2pi.sterngerlach} 

\input{qm2pi.metric} 

% section concurrent_process_calculi (end)

%\input{qm2pi.proofsketch}

% section proof sketch (end)

%\input{qm2pi.slviaknots} 

% section spatial logic via knots (end)

\input{qm2pi.conclusion}

% section conclusion (end)

%\input{qm2pi.dtcodes} 

% section wiring algorithm (end)

\input{qm2pi.ack} 

% section acknowledgments (end)

\newpage


\bibliographystyle{plain}   
\bibliography{../../biblios/main.bib}

\input{qm2pi.rhodetails}

\end{document}

 

%\documentclass[12pt]{llncs}
%\documentclass{jktr}

\usepackage[pdftex]{hyperref}                   
\usepackage {listings}
\usepackage {mathpartir}
\usepackage{bcprules}
%\usepackage{listings}
                       
\usepackage{graphicx} 
%\usepackage[margins=2.5cm,nohead,nofoot]{geometry}
%\usepackage{geometry}
\usepackage{amsfonts}
\usepackage{amstext}
\usepackage{latexsym}
\usepackage{amssymb}
\usepackage{color}


%\include{myPreamble}
\include{qm2pi.local} 

%\ifpdf
%\usepackage[pdftex]{graphicx}
%\else
%\usepackage{graphicx}
%\fi

 % \ifpdf
%  \usepackage{pdfsync}
%  \if


%\title{Brief Article}
%\author{David F. Snyder}
%\author{L.G. Meredith}

%\address{Dept. of Math., Texas State University--San Marcos, San Marcos, TX 78666}
       
\pagestyle{empty}


\begin{document}

\lstset{language=[Objective]Caml,frame=shadowbox}

\input{qm2pi.front}

% section front matter (end)

\input{qm2pi.intro} 
 
% section introduction (end)

% \input{qm2pi.knotations} 

% section notation (end)

\input{qm2pi.process.calculi} 

% section concurrent_process_calculi_and_spatial_logics_ (end)
    
%\input{qm2pi.knots2pi} 

%\input{qm2pi.trefoil} 

%\input{qm2pi.mainthm} 

% subsection basic_interpretation (end)

%\input{qm2pi.rho.presentation} 
\subsection{The syntax and semantics of the notation system}\label{sub:the_syntax_and_semantics_of_the_notation_system} % (fold)

We now summarize a technical presentation of the calculus that
embodies our theory of dynamics. The typical presentation of such a
calculus follows the style of giving generators and relations on
them. The grammar, below, describing term constructors, freely
generates the set of processes, $\Proc$. This set is then quotiented
by a relation known as structural congruence and it is over this set
that the notion of dynamics is expressed. This presentation is
essentially that of \cite{MeredithR05} with the addition of
polyadicity and summation. For readability we have relegated some of
the technical subtleties to an appendix.

\subsubsection{Process grammar}\label{subsub:process_grammar}

\begin{mathpar}
  \inferrule* [lab=synchronization] {} {{M} \bc \pzero \;|\; x?F \;|\; x!C }
  \and
  \inferrule* [lab=abstraction] {} {{F} \bc (x)P}
  \and
  \inferrule* [lab=concretion] {} {{C} \bc \langle Q \rangle}
  \and
  \inferrule* [lab=process] {} {{P,Q} \bc M \;| \;P|Q \;|\; @{x}}
  \and
  \inferrule* [lab=name] {} {{x} \bc \quotep{P}}
\end{mathpar} 

Note that $\vec{x}$ (resp. $\vec{P}$) denotes a vector of names
(resp. processes) of length $|\vec{x}|$ (resp. $|\vec{P}|$). We adopt
the following useful abbreviations.

\begin{mathpar}
   x?(\vec{y}).P := x.(\vec{y})P \and  x\clift{\vec{P}} := x.\clift{\vec{P}}
   \and x!(y) := \lift{x}{\dropn{y}}
   \and \Pi_{i=0}^{n-1}P_i := P_0 | \ldots | P_{n-1}
\end{mathpar}

\subsubsection{Structural congruence}

\paragraph{Free and bound names and alpha-equivalence.} At the
core of structural equivalence is alpha-equivalence which identifies
process that are the same up to a change of variable. Formally, we
recognize the distinction between free and bound names. The free names
of a process, $\freenames{P}$, may be calculated recursively as
follows:

\begin{mathpar}
\freenames{\pzero} := \emptyset
  \and \\
  \freenames{x?(y).P} := \{ x \} \cup (\freenames{P} \setminus \{ y \})
  \and 
  \freenames{x!\langle P \rangle} := \{ x \} \cup \{ P \} 
  \and \\
  \freenames{P|Q} := \freenames{P} \cup \freenames{Q}
  \and \\
  \freenames{@{x}} := \{ x \}
\end{mathpar}

$\pi$
$\quotep{\pi}$

$\freenames{-} : \pi \to \mathcal{P}(\quotep{\pi})$

\begin{eqnarray*}
  \freenames{\pzero} & := & \emptyset \\
  \freenames{x?(y).P} & := & \{ x \} \cup (\freenames{P} \setminus \{ y \}) \\
  \freenames{x!\langle P \rangle} & := & \{ x \} \cup \{ P \} \\
  \freenames{P|Q} & := & \freenames{P} \cup \freenames{Q} \\
  \freenames{\dropn{x}} & := & \{ x \}
\end{eqnarray*}

The bound names of a process, $\boundnames{P}$, are those names occurring in $P$
that are not free. For example, in $x?(y).0$, the name $x$ is free, while $y$ is bound.

\begin{mathpar}
  \inferrule* [lab=monoidal-laws] {} { P|Q \equiv Q|P \and P|0 \equiv P \and P|(Q|R) \equiv (P|Q)|R }
\end{mathpar}

\begin{mathpar}
  \inferrule* [lab=alpha-equivalence] {} { (x)P \equiv (y)P\{y/x\} \and y \not\in \freenames{P} }
\end{mathpar}

\begin{definition}
Then two processes, $P,Q$, are alpha-equivalent if $P = Q\{\vec{y}/\vec{x}\}$ for
some $\vec{x} \in \boundnames{Q},\vec{y} \in \boundnames{P}$, where $Q\{\vec{y}/\vec{x}\}$
denotes the capture-avoiding substitution of $\vec{y}$ for $\vec{x}$ in $Q$.
\end{definition}

\begin{definition}
  The {\em structural congruence} \cite{SangiorgiWalker} , $\equiv$,
  between processes is the least congruence containing
  alpha-equivalence, satisfying the abelian monoid laws
  (associativity, commutativity and $\pzero$ as identity) for parallel
  composition $|$ and for summation $+$.
\end{definition}

\subsection{Name equivalence}

We take name equivalence, written $\nameeq$, to be the smallest
equivalence relation generated by the following rules.

\begin{mathpar}
\inferrule*[lab=Quote-drop]
{ }
{ \quotep{@{x}} \nameeq x }

\inferrule*[lab=Struct-equiv]
{ P \scong Q }
{ \quotep{P} \nameeq \quotep{Q} }
\end{mathpar}

The astute reader will have noticed that the mutual recursion of names
and processes imposes a mutual recursion on alpha-equivalence and
structural equivalence via name-equivalence. Fortunately, all of this
works out pleasantly and we may calculate in the natural way, free of
concern. The reader interested in the details is referred to the
appendix \ref{appendix:rho_details}.

\subsection{Substitution}

We use $\Proc$ for the set of processes, $\QProc$ for the set of
names, and $\id{\{}\vec{y} / \vec{x} \id{\}}$ to denote partial maps,
$s : \QProc \rightarrow \QProc$. A map, $s$ lifts, uniquely, to a map
on process terms, $\widehat{s} : \Proc \rightarrow \Proc$ by the
following equations.

\begin{mathpar}
  (0) \psubstp{Q}{P} := 0 \\
  (R \juxtap S) \psubstp{Q}{P}
  :=    
  (R)\psubstp{Q}{P} \juxtap (S) \psubstp{Q}{P} \\
  (x?(y).R) \psubstp{Q}{P}    
  :=    
  (x)\substp{Q}{P} (z)\concat( (R \psubstn{z}{y}) \psubstp{Q}{P} ) \\
  (\lift{x}{R}) \psubstp{Q}{P}  
  :=
  \lift{(x)\substp{Q}{P}}{ R \psubstp{Q}{P} } \\
%   (\dropn{x})  \psubstp{Q}{P}       
%   := 
%   \left\{ 
%     \begin{array}{ccc} 
%       \dropn{\quotep{Q}} & & x \nameeq \quotep{P} \\
%       \dropn{x} & & otherwise \\
%     \end{array}
%   \right. 
  (\dropn{x})  \psubstp{Q}{P}       
  := 
  \left\{ 
    \begin{array}{ccc} 
      Q & & x \nameeq \quotep{P} \\
      \dropn{x} & & otherwise \\
    \end{array}
  \right.
\end{mathpar}
 

where

\begin{eqnarray}
  (x)\id{\{} \lpquote Q \rpquote / \lpquote P \rpquote \id{\}}            = 
  \left\{ 
    \begin{array}{ccc}
      \lpquote Q \rpquote & & x \nameeq \lpquote P \rpquote \\
      x & & otherwise \\
    \end{array}
  \right. \nonumber
\end{eqnarray}

and $z$ is chosen distinct from $\quotep{P}$, $\quotep{Q}$, the free
names in $Q$, and all the names in $R$. Our $\alpha$-equivalence will
be built in the standard way from this substitution.

\begin{remark}\label{rem:no_self_referential_names}
  One consequence of these definitions is that $\forall P. \quotep{P}
  \not\in \freenames{P}$.
\end{remark}

\subsection{ Dynamic quote: an example }

Anticipating something of what's to come, consider applying the
substitution, $\widehat{\id{\{}u / z \id{\}}}$, to the following pair
of processes, $\lift{w}{y!(z)}$ and $w[ \lpquote y!(z) \rpquote ]$.

\begin{eqnarray}
	\lift{w}{y!(z)}\widehat{\id{\{}u / z \id{\}}}
		& = &
		\lift{w}{y!(u)} \nonumber\\
	w[ \lpquote y!(z) \rpquote ] \widehat{ \id{\{}u / z \id{\}} }
		& = &
		w[ \lpquote y!(z) \rpquote ] \nonumber
\end{eqnarray}

Because the body of the process between quotes is impervious to
substitution, we get radically different answers. In fact, by
examining the first process in an input context,
e.g. $x?(z).\lift{w}{y!(z)}$, we see that the process under the lift
operator may be shaped by prefixed inputs binding a name inside it. In
this sense, the lift operator will be seen as a way to dynamically
construct processes before reifying them as names.

Finally equipped with these standard features we can present the
dynamics of the calculus.

\subsubsection{Operational semantics} 

Finally, we introduce the computational dynamics. What marks these
algebras as distinct from other more traditionally studied algebraic
structures, e.g. vector spaces or polynomial rings, is the manner in
which dynamics is captured. In traditional structures, dynamics is typically
expressed through morphisms between such structures, as in linear maps
between vector spaces or morphisms between rings. In algebras
associated with the semantics of computation, the dynamics is
expressed as part of the algebraic structure itself, through a
reduction reduction relation typically denoted by $\red$. Below, we
give a recursive presentation of this relation for the calculus used
in the encoding.

$\red \subseteq \pi \times \pi$
$\red : \pi \to \mathcal{P}(\pi)$

\begin{mathpar}
  \inferrule* [lab=Comm] { \textsf{match}( x_{src}, x_{trgt} ) } { x_{trgt}?(y)P \; | \; x_{src}!\langle {Q} \rangle \red P\{\quotep{Q}/y}\} }
  \and \\
  \inferrule* [lab=Par] {{P} \red {P}'} {{{P} | {Q}} \red {{P}' | {Q}}}
  \and
  \inferrule* [lab=Equiv]{{{P} \scong {P}'} \andalso {{P}' \red {Q}'} \andalso {{Q}' \scong {Q}}}{{P} \red {Q}}
\end{mathpar}

\begin{eqnarray*}
  match_{\equiv} (\quotep{P},\quotep{Q}) & := & P \equiv Q \\
  match_{\dagger}(\quotep{P},\quotep{Q}) & := & \forall R. P|Q \red^{*} R => R \red^{*} 0 \\
  match_{K}(\quotep{P},\quotep{Q}) & := & K \mbox{ for some context } K
\end{eqnarray*}

$u?(x)P | u!\langle Q \rangle \red P\{\quotep{Q}/x\}$

%We write $\wred$ for $\red^*$, and $P\red$ if $\exists Q $ such that $ P \red Q$.
We write $P\red$ if $\exists Q $ such that $ P \red Q$ and $P\not\red$, otherwise.

\section{Replication}

As mentioned before, it is known that replication (and hence
recursion) can be implemented in a higher-order process algebra
\cite{SangiorgiWalker}. As our first example of calculation with the
machinery thus far presented we give the construction explicitly in
the {\rhoc}.

\begin{eqnarray}
	D_{x} & := & \prefix{x}{y}{(\binpar{\outputp{x}{y}}{@{y}})} \nonumber\\
	\bangp_{x}{P} & := & \binpar{{x}!\langle{\binpar{D_{x}}{P}}\rangle}{D_{x}} \nonumber
\end{eqnarray}

\begin{eqnarray}
	\bangp_{x}{P} & & \nonumber\\
	=
	& {x}!\langle{(\prefix{x}{y}{(\outputp{x}{y} | @{y})) | P}}\rangle 
	      | \prefix{x}{y}{(\outputp{x}{y} | @{y})} & \nonumber\\
	\red
	& (\outputp{x}{y} | @{y})\substn{\quotep{(\prefix{x}{y}{(@{y} | \outputp{x}{y})) | P}}}{y} & \nonumber\\
	=
	& \outputp{x}{\quotep{(\prefix{x}{y}{(\outputp{x}{y} | @{y})) | P}}}
	  | {(\prefix{x}{y}{(\outputp{x}{y} | @{y})) | P}} & \nonumber\\
	\red
	& \ldots & \nonumber\\
	\red^*
	& P | P | \ldots & \nonumber
\end{eqnarray}

Of course, this encoding, as an implementation, runs away, unfolding
$\bangp{P}$ eagerly. A lazier and more implementable replication
operator, restricted to input-guarded processes, may be obtained as follows.

\begin{eqnarray}
\bangp{\prefix{u}{v}{P}} 
	:= 
	\binpar{\lift{x}{\prefix{u}{v}{(\binpar{D(x)}{P})}}}{D(x)} \nonumber
\end{eqnarray}

\begin{remark}
  Note that the lazier definition still does not deal with summation
  or mixed summation (i.e. sums over input and output). The reader is
  invited to construct definitions of replication that deal with these
  features. 

  Further, the definitions are parameterized in a name, $x$. Can you,
  gentle reader, make a definition that eliminates this parameter and
  guarantees no accidental interaction between the replication
  machinery and the process being replicated -- i.e. no accidental
  sharing of names used by the process to get its work done and the
  name(s) used by the replication to effect copying. This latter
  revision of the definition of replication is crucial to obtaining
  the expected identity $!!P \sim !P$.
\end{remark}

\begin{remark}\label{rem:paradoxical_combinator}
  The reader familiar with the lambda calculus will have noticed the
  similarity between $D$ and the paradoxical combinator.

  [Ed. note: the existence of this seems to suggest we have to be more
  restrictive on the set of processes and names we admit if we are to
  support no-cloning.]
\end{remark}

\subsubsection{Bisimulation}

The computational dynamics gives rise to another kind of equivalence,
the equivalence of computational behavior. As previously mentioned
this is typically captured \emph{via} some form of bisimulation.

% The notion we use in this paper is weak barbed bisimulation
% \cite{milner91polyadicpi}.

The notion we use in this paper is derived from weak barbed
bisimulation \cite{milner91polyadicpi}. 

\begin{definition}
An \emph{observation relation}, $\downarrow_{\mathcal N}$, over a set
of names, $\mathcal N$, is the smallest relation satisfying the rules
below.

\infrule[Out-barb]{y \in {\mathcal N}, \; x \nameeq y}
		  {\outputp{x}{v} \downarrow_{\mathcal N} x}
\infrule[Par-barb]{\mbox{$P\downarrow_{\mathcal N} x$ or $Q\downarrow_{\mathcal N} x$}}
		  {\binpar{P}{Q} \downarrow_{\mathcal N} x}

We write $P \Downarrow_{\mathcal N} x$ if there is $Q$ such that 
$P \wred Q$ and $Q \downarrow_{\mathcal N} x$.
\end{definition}

\begin{definition}
%\label{def.bbisim}
An  ${\mathcal N}$-\emph{barbed bisimulation} over a set of names, ${\mathcal N}$, is a symmetric binary relation 
${\mathcal S}_{\mathcal N}$ between agents such that $P\rel{S}_{\mathcal N}Q$ implies:
\begin{enumerate}
\item If $P \red P'$ then $Q \wred Q'$ and $P'\rel{S}_{\mathcal N} Q'$.
\item If $P\downarrow_{\mathcal N} x$, then $Q\Downarrow_{\mathcal N} x$.
\end{enumerate}
$P$ is ${\mathcal N}$-barbed bisimilar to $Q$, written
$P \wbbisim_{\mathcal N} Q$, if $P \rel{S}_{\mathcal N} Q$ for some ${\mathcal N}$-barbed bisimulation ${\mathcal S}_{\mathcal N}$.
\end{definition}

$\mathcal{R} \subseteq \pi \times \pi$

$P \mathcal{R} Q => \forall P'. P \red P' \Rightarrow \exists Q'. Q \red Q', P' \mathcal{R} Q'$

$P \vdash x \Rightarrow Q \vdash x$

\begin{mathpar}
  \inferrule*[lab=Out-barb]{x \nameeq y}{{y}!\langle{Q}\rangle \vdash x}
  \and
  \inferrule*[lab=Par-barb]{\mbox{$P\vdash x$ or $Q\vdash x$}}{\binpar{P}{Q} \vdash x}
\end{mathpar}

\subsubsection{Contexts}

One of the principle advantages of computational calculi like the
$\pi$-calculus is a well-defined notion of context,
contextual-equivalence and a correlation between
contextual-equivalence and notions of bisimulation. The notion of
context allows the decomposition of a process into (sub-)process and
its syntactic environment, its context. Thus, a context may be
thought of as a process with a ``hole'' (written $\Box$) in it. The
application of a context $M$ to a process $P$, written $M[P]$, is
tantamount to filling the hole in $M$ with $P$. In this paper we do
not need the full weight of this theory, but do make use of the notion
of context in the proof the main theorem. 

\begin{mathpar}
  \inferrule* [lab=summation] {} {{M_{M},M_{N}} \bc \Box \;|\; x.M_{A} \;|\; M_{M}+M_{N}}
  \and
  \inferrule* [lab=agent] {} {{M_{A}} \bc (\vec{x})M_{P} \;| \; \clift{P_0,\ldots,M_{P},\ldots,P_N}}
  \and \\
  \inferrule* [lab=process] {} {{M_{P}} \bc M_{N} \;| \;P|M_{P} }
\end{mathpar} 

\begin{mathpar}
  \inferrule* [lab=sychronization] {} {M_{N} \bc \Box \;|\; x?M_{F} \;|\; x!M_{C}}
  \and
  \inferrule* [lab=abstraction] {} {{M_{F}} \bc (x)M_{P} }
  \and
  \inferrule* [lab=concretion] {} {{M_{C}} \bc \langle M_{P} \rangle }
  \and \\
  \inferrule* [lab=process] {} {{M_{P}} \bc M_{N} \;| \;P|M_{P} }
\end{mathpar}

\begin{definition}[contextual application] Given a context $M$, and
  process $P$, we define the \emph{contextual application}, $M[P] :=
  M\{P/\Box\}$. That is, the contextual application of M to P is the
  substitution of $P$ for $\Box$ in $M$.
\end{definition}

$\meaningof{-} : L \to \mathcal{P}(\pi)$

\begin{mathpar}
  \inferrule* [lab=collection] {} {\meaningof{true} = \pi, \and \meaningof{~E} = \pi \setminus \meaningof{E}, \and \meaningof{E_{1} \& E_{2}} = \meaningof{E_{1}} \cap \meaningof{E_{2}}}
\end{mathpar}

\begin{mathpar}
  \inferrule* [lab=structure] {} {\meaningof{0} = \{ P \in \pi | P \equiv 0 \}, \and \\ \meaningof{E_1 | E_2} = \{ P \in \pi | P \equiv P_{1} | P_{2}, P_{1} \in \meaningof{E_{1}}, P_{2} \in \meaningof{E_2}\} }
\end{mathpar}

\begin{mathpar}
 \inferrule* [lab=behavior] {} {\meaningof{\langle a?b \rangle E} = \{ P \in \pi | P \equiv Q | u?(y)P', \\ \and \\\\ \and \\ \;\;\; u \in \meaningof{a}, \forall z.P'\{z/y\} \in \meaningof{E\{z/b\}}\}, \and \\ \meaningof{a!E} = \{ P \in \pi | P \equiv Q | x!\langle P' \rangle, x \in \meaningof{a} P' \in \meaningof{E}\} }
\end{mathpar}

\begin{mathpar}
 \inferrule* [lab=nominal] {} {\meaningof{\quotep{E}} = \{ \quotep{P} \in \quotep{\pi} | P \in \meaningof{E} \}, \and \meaningof{\quotep{P}} = \{ \quotep{Q} \in \quotep{\pi} | P \equiv Q \} \and \\ \meaningof{@\quotep{E}} = \{ P \in \pi | P \equiv @x, x \in \meaningof{E} \}}
\end{mathpar}

\begin{eqnarray*}
  \\
  \meaningof{-} : TS \to ST
\end{eqnarray*}

\begin{eqnarray*}
  \\
  L : TS \to ST
\end{eqnarray*}

\begin{eqnarray*}
  \\
  P \models E \iff P \in \meaningof{E}
\end{eqnarray*}

\begin{eqnarray*}
  P \approx_{L} Q \iff \forall E \in L. P \models E \iff Q \models E
\end{eqnarray*}

\begin{eqnarray*}
  P \approx_{K} Q
\end{eqnarray*}

\begin{eqnarray*}
  P \approx Q
\end{eqnarray*}

$\approx_{K} = \approx = \approx_{L}$

\subsubsection{Contextual duality}

Note that contexts extend the quotation operation to a family of
operations from processes to names. Given a context, $M$, we can
define a \emph{nominal context}, $\quotep{M}$ by $\quotep{M}[P] :=
\quotep{M[P]}$. To foreshadow what is to come we observe that these
operations enjoy a duality with processes very much like the duality
between vectors and maps from vectors to scalars.

Further, because the calculus is essentially higher-order, we have a
correspondence between contexts and processes. More specifically,
given a name $x$ and a context $M$ we can construct $M^{*}_{x}$ such
that 

\begin{mathpar}
  M^{*}_{x} | \lift{x}{P} \red M[P]
\end{mathpar}

namely,

\begin{mathpar}
  M^{*}_{x} := x?(u).M[\dropn{u}]
\end{mathpar}

The dependence of $M^{*}_{x}$ on a name makes it an abstraction, 

\begin{mathpar}
  M^{*} := (x)x?(u).M[\dropn{u}]
\end{mathpar}

\subsection{Additional notation}

It will sometimes be convenient to denote the process a name
quotes. We already have the notation $x = \quotep{P}$, but it will be
convenient to introduce an alternate notation, $\procn{x}$, when we
want to emphasize the connection to the use of the name. Note that, by
virtue of name equivalence, $\quotep{\procn{x}} \nameeq x$; so, the
notation is consistent with previous definitions.

Further, because names have structure it is possible to effect
substitutions on the basis of that structure. This means we need to
upgrade our notation for substitutions, which we accomplish by
adapting comprehension notation. Thus,

\begin{mathpar}
  P\{ y / x : x \in S \}
\end{mathpar}

is interpreted to mean the process derived from P by replacing (in a
capture-avoiding manner) each occurrence of $x$ in $S$ by $y$. For example,

\begin{mathpar}
  P\{ \quotep{\procn{x}|\procn{x}} / x : x \in \freenames{P} \}
\end{mathpar}

will replace each (occurrence) of a free name $x$ in $P$ by
$\quotep{\procn{x}|\procn{x}}$.

Also, we will avail ourselves of the notation $x^{L}$ and $x^{R}$ to
denote injections of a name into disjoint copies of the name
space. There are numerous ways to accomplish this. One example can be
found in \cite{MeredithR05}. This notation overloads to vectors of
names: $\vec{x}^{\pi} := (x_{i}^{\pi} \; : \; 0 \leq i < |\vec{x}| )$ where $\pi \in \{L,R\}$.

We also use $P^{\Box} := P|\Box$.

In \cite{MeredithR05} an interpretation of the new operator is
given. It turns out that there are several possible interpretations
all enjoying the requisite algebraic properties of the operator (see
\cite{milner91polyadicpi}). We will therefore make liberal use of
$(\nu\; \vec{x})P$.

% subsection the_syntax_and_semantics_of_the_notation_system (end)   

\input{qm2pi.qmops} 

\input{qm2pi.sterngerlach} 

\input{qm2pi.metric} 

% section concurrent_process_calculi (end)

%\input{qm2pi.proofsketch}

% section proof sketch (end)

%\input{qm2pi.slviaknots} 

% section spatial logic via knots (end)

\input{qm2pi.conclusion}

% section conclusion (end)

%\input{qm2pi.dtcodes} 

% section wiring algorithm (end)

\input{qm2pi.ack} 

% section acknowledgments (end)

\newpage


\bibliographystyle{plain}   
\bibliography{../../biblios/main.bib}

\input{qm2pi.rhodetails}

\end{document}

 

% subsection basic_interpretation (end)

%\input{qm2pi.rho.presentation} 
\subsection{The syntax and semantics of the notation system}\label{sub:the_syntax_and_semantics_of_the_notation_system} % (fold)

We now summarize a technical presentation of the calculus that
embodies our theory of dynamics. The typical presentation of such a
calculus follows the style of giving generators and relations on
them. The grammar, below, describing term constructors, freely
generates the set of processes, $\Proc$. This set is then quotiented
by a relation known as structural congruence and it is over this set
that the notion of dynamics is expressed. This presentation is
essentially that of \cite{MeredithR05} with the addition of
polyadicity and summation. For readability we have relegated some of
the technical subtleties to an appendix.

\subsubsection{Process grammar}\label{subsub:process_grammar}

\begin{mathpar}
  \inferrule* [lab=synchronization] {} {{M} \bc \pzero \;|\; x?F \;|\; x!C }
  \and
  \inferrule* [lab=abstraction] {} {{F} \bc (x)P}
  \and
  \inferrule* [lab=concretion] {} {{C} \bc \langle Q \rangle}
  \and
  \inferrule* [lab=process] {} {{P,Q} \bc M \;| \;P|Q \;|\; @{x}}
  \and
  \inferrule* [lab=name] {} {{x} \bc \quotep{P}}
\end{mathpar} 

Note that $\vec{x}$ (resp. $\vec{P}$) denotes a vector of names
(resp. processes) of length $|\vec{x}|$ (resp. $|\vec{P}|$). We adopt
the following useful abbreviations.

\begin{mathpar}
   x?(\vec{y}).P := x.(\vec{y})P \and  x\clift{\vec{P}} := x.\clift{\vec{P}}
   \and x!(y) := \lift{x}{\dropn{y}}
   \and \Pi_{i=0}^{n-1}P_i := P_0 | \ldots | P_{n-1}
\end{mathpar}

\subsubsection{Structural congruence}

\paragraph{Free and bound names and alpha-equivalence.} At the
core of structural equivalence is alpha-equivalence which identifies
process that are the same up to a change of variable. Formally, we
recognize the distinction between free and bound names. The free names
of a process, $\freenames{P}$, may be calculated recursively as
follows:

\begin{mathpar}
\freenames{\pzero} := \emptyset
  \and \\
  \freenames{x?(y).P} := \{ x \} \cup (\freenames{P} \setminus \{ y \})
  \and 
  \freenames{x!\langle P \rangle} := \{ x \} \cup \{ P \} 
  \and \\
  \freenames{P|Q} := \freenames{P} \cup \freenames{Q}
  \and \\
  \freenames{@{x}} := \{ x \}
\end{mathpar}

$\pi$
$\quotep{\pi}$

$\freenames{-} : \pi \to \mathcal{P}(\quotep{\pi})$

\begin{eqnarray*}
  \freenames{\pzero} & := & \emptyset \\
  \freenames{x?(y).P} & := & \{ x \} \cup (\freenames{P} \setminus \{ y \}) \\
  \freenames{x!\langle P \rangle} & := & \{ x \} \cup \{ P \} \\
  \freenames{P|Q} & := & \freenames{P} \cup \freenames{Q} \\
  \freenames{\dropn{x}} & := & \{ x \}
\end{eqnarray*}

The bound names of a process, $\boundnames{P}$, are those names occurring in $P$
that are not free. For example, in $x?(y).0$, the name $x$ is free, while $y$ is bound.

\begin{mathpar}
  \inferrule* [lab=monoidal-laws] {} { P|Q \equiv Q|P \and P|0 \equiv P \and P|(Q|R) \equiv (P|Q)|R }
\end{mathpar}

\begin{mathpar}
  \inferrule* [lab=alpha-equivalence] {} { (x)P \equiv (y)P\{y/x\} \and y \not\in \freenames{P} }
\end{mathpar}

\begin{definition}
Then two processes, $P,Q$, are alpha-equivalent if $P = Q\{\vec{y}/\vec{x}\}$ for
some $\vec{x} \in \boundnames{Q},\vec{y} \in \boundnames{P}$, where $Q\{\vec{y}/\vec{x}\}$
denotes the capture-avoiding substitution of $\vec{y}$ for $\vec{x}$ in $Q$.
\end{definition}

\begin{definition}
  The {\em structural congruence} \cite{SangiorgiWalker} , $\equiv$,
  between processes is the least congruence containing
  alpha-equivalence, satisfying the abelian monoid laws
  (associativity, commutativity and $\pzero$ as identity) for parallel
  composition $|$ and for summation $+$.
\end{definition}

\subsection{Name equivalence}

We take name equivalence, written $\nameeq$, to be the smallest
equivalence relation generated by the following rules.

\begin{mathpar}
\inferrule*[lab=Quote-drop]
{ }
{ \quotep{@{x}} \nameeq x }

\inferrule*[lab=Struct-equiv]
{ P \scong Q }
{ \quotep{P} \nameeq \quotep{Q} }
\end{mathpar}

The astute reader will have noticed that the mutual recursion of names
and processes imposes a mutual recursion on alpha-equivalence and
structural equivalence via name-equivalence. Fortunately, all of this
works out pleasantly and we may calculate in the natural way, free of
concern. The reader interested in the details is referred to the
appendix \ref{appendix:rho_details}.

\subsection{Substitution}

We use $\Proc$ for the set of processes, $\QProc$ for the set of
names, and $\id{\{}\vec{y} / \vec{x} \id{\}}$ to denote partial maps,
$s : \QProc \rightarrow \QProc$. A map, $s$ lifts, uniquely, to a map
on process terms, $\widehat{s} : \Proc \rightarrow \Proc$ by the
following equations.

\begin{mathpar}
  (0) \psubstp{Q}{P} := 0 \\
  (R \juxtap S) \psubstp{Q}{P}
  :=    
  (R)\psubstp{Q}{P} \juxtap (S) \psubstp{Q}{P} \\
  (x?(y).R) \psubstp{Q}{P}    
  :=    
  (x)\substp{Q}{P} (z)\concat( (R \psubstn{z}{y}) \psubstp{Q}{P} ) \\
  (\lift{x}{R}) \psubstp{Q}{P}  
  :=
  \lift{(x)\substp{Q}{P}}{ R \psubstp{Q}{P} } \\
%   (\dropn{x})  \psubstp{Q}{P}       
%   := 
%   \left\{ 
%     \begin{array}{ccc} 
%       \dropn{\quotep{Q}} & & x \nameeq \quotep{P} \\
%       \dropn{x} & & otherwise \\
%     \end{array}
%   \right. 
  (\dropn{x})  \psubstp{Q}{P}       
  := 
  \left\{ 
    \begin{array}{ccc} 
      Q & & x \nameeq \quotep{P} \\
      \dropn{x} & & otherwise \\
    \end{array}
  \right.
\end{mathpar}
 

where

\begin{eqnarray}
  (x)\id{\{} \lpquote Q \rpquote / \lpquote P \rpquote \id{\}}            = 
  \left\{ 
    \begin{array}{ccc}
      \lpquote Q \rpquote & & x \nameeq \lpquote P \rpquote \\
      x & & otherwise \\
    \end{array}
  \right. \nonumber
\end{eqnarray}

and $z$ is chosen distinct from $\quotep{P}$, $\quotep{Q}$, the free
names in $Q$, and all the names in $R$. Our $\alpha$-equivalence will
be built in the standard way from this substitution.

\begin{remark}\label{rem:no_self_referential_names}
  One consequence of these definitions is that $\forall P. \quotep{P}
  \not\in \freenames{P}$.
\end{remark}

\subsection{ Dynamic quote: an example }

Anticipating something of what's to come, consider applying the
substitution, $\widehat{\id{\{}u / z \id{\}}}$, to the following pair
of processes, $\lift{w}{y!(z)}$ and $w[ \lpquote y!(z) \rpquote ]$.

\begin{eqnarray}
	\lift{w}{y!(z)}\widehat{\id{\{}u / z \id{\}}}
		& = &
		\lift{w}{y!(u)} \nonumber\\
	w[ \lpquote y!(z) \rpquote ] \widehat{ \id{\{}u / z \id{\}} }
		& = &
		w[ \lpquote y!(z) \rpquote ] \nonumber
\end{eqnarray}

Because the body of the process between quotes is impervious to
substitution, we get radically different answers. In fact, by
examining the first process in an input context,
e.g. $x?(z).\lift{w}{y!(z)}$, we see that the process under the lift
operator may be shaped by prefixed inputs binding a name inside it. In
this sense, the lift operator will be seen as a way to dynamically
construct processes before reifying them as names.

Finally equipped with these standard features we can present the
dynamics of the calculus.

\subsubsection{Operational semantics} 

Finally, we introduce the computational dynamics. What marks these
algebras as distinct from other more traditionally studied algebraic
structures, e.g. vector spaces or polynomial rings, is the manner in
which dynamics is captured. In traditional structures, dynamics is typically
expressed through morphisms between such structures, as in linear maps
between vector spaces or morphisms between rings. In algebras
associated with the semantics of computation, the dynamics is
expressed as part of the algebraic structure itself, through a
reduction reduction relation typically denoted by $\red$. Below, we
give a recursive presentation of this relation for the calculus used
in the encoding.

$\red \subseteq \pi \times \pi$
$\red : \pi \to \mathcal{P}(\pi)$

\begin{mathpar}
  \inferrule* [lab=Comm] { \textsf{match}( x_{src}, x_{trgt} ) } { x_{trgt}?(y)P \; | \; x_{src}!\langle {Q} \rangle \red P\{\quotep{Q}/y}\} }
  \and \\
  \inferrule* [lab=Par] {{P} \red {P}'} {{{P} | {Q}} \red {{P}' | {Q}}}
  \and
  \inferrule* [lab=Equiv]{{{P} \scong {P}'} \andalso {{P}' \red {Q}'} \andalso {{Q}' \scong {Q}}}{{P} \red {Q}}
\end{mathpar}

\begin{eqnarray*}
  match_{\equiv} (\quotep{P},\quotep{Q}) & := & P \equiv Q \\
  match_{\dagger}(\quotep{P},\quotep{Q}) & := & \forall R. P|Q \red^{*} R => R \red^{*} 0 \\
  match_{K}(\quotep{P},\quotep{Q}) & := & K \mbox{ for some context } K
\end{eqnarray*}

$u?(x)P | u!\langle Q \rangle \red P\{\quotep{Q}/x\}$

%We write $\wred$ for $\red^*$, and $P\red$ if $\exists Q $ such that $ P \red Q$.
We write $P\red$ if $\exists Q $ such that $ P \red Q$ and $P\not\red$, otherwise.

\section{Replication}

As mentioned before, it is known that replication (and hence
recursion) can be implemented in a higher-order process algebra
\cite{SangiorgiWalker}. As our first example of calculation with the
machinery thus far presented we give the construction explicitly in
the {\rhoc}.

\begin{eqnarray}
	D_{x} & := & \prefix{x}{y}{(\binpar{\outputp{x}{y}}{@{y}})} \nonumber\\
	\bangp_{x}{P} & := & \binpar{{x}!\langle{\binpar{D_{x}}{P}}\rangle}{D_{x}} \nonumber
\end{eqnarray}

\begin{eqnarray}
	\bangp_{x}{P} & & \nonumber\\
	=
	& {x}!\langle{(\prefix{x}{y}{(\outputp{x}{y} | @{y})) | P}}\rangle 
	      | \prefix{x}{y}{(\outputp{x}{y} | @{y})} & \nonumber\\
	\red
	& (\outputp{x}{y} | @{y})\substn{\quotep{(\prefix{x}{y}{(@{y} | \outputp{x}{y})) | P}}}{y} & \nonumber\\
	=
	& \outputp{x}{\quotep{(\prefix{x}{y}{(\outputp{x}{y} | @{y})) | P}}}
	  | {(\prefix{x}{y}{(\outputp{x}{y} | @{y})) | P}} & \nonumber\\
	\red
	& \ldots & \nonumber\\
	\red^*
	& P | P | \ldots & \nonumber
\end{eqnarray}

Of course, this encoding, as an implementation, runs away, unfolding
$\bangp{P}$ eagerly. A lazier and more implementable replication
operator, restricted to input-guarded processes, may be obtained as follows.

\begin{eqnarray}
\bangp{\prefix{u}{v}{P}} 
	:= 
	\binpar{\lift{x}{\prefix{u}{v}{(\binpar{D(x)}{P})}}}{D(x)} \nonumber
\end{eqnarray}

\begin{remark}
  Note that the lazier definition still does not deal with summation
  or mixed summation (i.e. sums over input and output). The reader is
  invited to construct definitions of replication that deal with these
  features. 

  Further, the definitions are parameterized in a name, $x$. Can you,
  gentle reader, make a definition that eliminates this parameter and
  guarantees no accidental interaction between the replication
  machinery and the process being replicated -- i.e. no accidental
  sharing of names used by the process to get its work done and the
  name(s) used by the replication to effect copying. This latter
  revision of the definition of replication is crucial to obtaining
  the expected identity $!!P \sim !P$.
\end{remark}

\begin{remark}\label{rem:paradoxical_combinator}
  The reader familiar with the lambda calculus will have noticed the
  similarity between $D$ and the paradoxical combinator.

  [Ed. note: the existence of this seems to suggest we have to be more
  restrictive on the set of processes and names we admit if we are to
  support no-cloning.]
\end{remark}

\subsubsection{Bisimulation}

The computational dynamics gives rise to another kind of equivalence,
the equivalence of computational behavior. As previously mentioned
this is typically captured \emph{via} some form of bisimulation.

% The notion we use in this paper is weak barbed bisimulation
% \cite{milner91polyadicpi}.

The notion we use in this paper is derived from weak barbed
bisimulation \cite{milner91polyadicpi}. 

\begin{definition}
An \emph{observation relation}, $\downarrow_{\mathcal N}$, over a set
of names, $\mathcal N$, is the smallest relation satisfying the rules
below.

\infrule[Out-barb]{y \in {\mathcal N}, \; x \nameeq y}
		  {\outputp{x}{v} \downarrow_{\mathcal N} x}
\infrule[Par-barb]{\mbox{$P\downarrow_{\mathcal N} x$ or $Q\downarrow_{\mathcal N} x$}}
		  {\binpar{P}{Q} \downarrow_{\mathcal N} x}

We write $P \Downarrow_{\mathcal N} x$ if there is $Q$ such that 
$P \wred Q$ and $Q \downarrow_{\mathcal N} x$.
\end{definition}

\begin{definition}
%\label{def.bbisim}
An  ${\mathcal N}$-\emph{barbed bisimulation} over a set of names, ${\mathcal N}$, is a symmetric binary relation 
${\mathcal S}_{\mathcal N}$ between agents such that $P\rel{S}_{\mathcal N}Q$ implies:
\begin{enumerate}
\item If $P \red P'$ then $Q \wred Q'$ and $P'\rel{S}_{\mathcal N} Q'$.
\item If $P\downarrow_{\mathcal N} x$, then $Q\Downarrow_{\mathcal N} x$.
\end{enumerate}
$P$ is ${\mathcal N}$-barbed bisimilar to $Q$, written
$P \wbbisim_{\mathcal N} Q$, if $P \rel{S}_{\mathcal N} Q$ for some ${\mathcal N}$-barbed bisimulation ${\mathcal S}_{\mathcal N}$.
\end{definition}

$\mathcal{R} \subseteq \pi \times \pi$

$P \mathcal{R} Q => \forall P'. P \red P' \Rightarrow \exists Q'. Q \red Q', P' \mathcal{R} Q'$

$P \vdash x \Rightarrow Q \vdash x$

\begin{mathpar}
  \inferrule*[lab=Out-barb]{x \nameeq y}{{y}!\langle{Q}\rangle \vdash x}
  \and
  \inferrule*[lab=Par-barb]{\mbox{$P\vdash x$ or $Q\vdash x$}}{\binpar{P}{Q} \vdash x}
\end{mathpar}

\subsubsection{Contexts}

One of the principle advantages of computational calculi like the
$\pi$-calculus is a well-defined notion of context,
contextual-equivalence and a correlation between
contextual-equivalence and notions of bisimulation. The notion of
context allows the decomposition of a process into (sub-)process and
its syntactic environment, its context. Thus, a context may be
thought of as a process with a ``hole'' (written $\Box$) in it. The
application of a context $M$ to a process $P$, written $M[P]$, is
tantamount to filling the hole in $M$ with $P$. In this paper we do
not need the full weight of this theory, but do make use of the notion
of context in the proof the main theorem. 

\begin{mathpar}
  \inferrule* [lab=summation] {} {{M_{M},M_{N}} \bc \Box \;|\; x.M_{A} \;|\; M_{M}+M_{N}}
  \and
  \inferrule* [lab=agent] {} {{M_{A}} \bc (\vec{x})M_{P} \;| \; \clift{P_0,\ldots,M_{P},\ldots,P_N}}
  \and \\
  \inferrule* [lab=process] {} {{M_{P}} \bc M_{N} \;| \;P|M_{P} }
\end{mathpar} 

\begin{mathpar}
  \inferrule* [lab=sychronization] {} {M_{N} \bc \Box \;|\; x?M_{F} \;|\; x!M_{C}}
  \and
  \inferrule* [lab=abstraction] {} {{M_{F}} \bc (x)M_{P} }
  \and
  \inferrule* [lab=concretion] {} {{M_{C}} \bc \langle M_{P} \rangle }
  \and \\
  \inferrule* [lab=process] {} {{M_{P}} \bc M_{N} \;| \;P|M_{P} }
\end{mathpar}

\begin{definition}[contextual application] Given a context $M$, and
  process $P$, we define the \emph{contextual application}, $M[P] :=
  M\{P/\Box\}$. That is, the contextual application of M to P is the
  substitution of $P$ for $\Box$ in $M$.
\end{definition}

$\meaningof{-} : L \to \mathcal{P}(\pi)$

\begin{mathpar}
  \inferrule* [lab=collection] {} {\meaningof{true} = \pi, \and \meaningof{~E} = \pi \setminus \meaningof{E}, \and \meaningof{E_{1} \& E_{2}} = \meaningof{E_{1}} \cap \meaningof{E_{2}}}
\end{mathpar}

\begin{mathpar}
  \inferrule* [lab=structure] {} {\meaningof{0} = \{ P \in \pi | P \equiv 0 \}, \and \\ \meaningof{E_1 | E_2} = \{ P \in \pi | P \equiv P_{1} | P_{2}, P_{1} \in \meaningof{E_{1}}, P_{2} \in \meaningof{E_2}\} }
\end{mathpar}

\begin{mathpar}
 \inferrule* [lab=behavior] {} {\meaningof{\langle a?b \rangle E} = \{ P \in \pi | P \equiv Q | u?(y)P', \\ \and \\\\ \and \\ \;\;\; u \in \meaningof{a}, \forall z.P'\{z/y\} \in \meaningof{E\{z/b\}}\}, \and \\ \meaningof{a!E} = \{ P \in \pi | P \equiv Q | x!\langle P' \rangle, x \in \meaningof{a} P' \in \meaningof{E}\} }
\end{mathpar}

\begin{mathpar}
 \inferrule* [lab=nominal] {} {\meaningof{\quotep{E}} = \{ \quotep{P} \in \quotep{\pi} | P \in \meaningof{E} \}, \and \meaningof{\quotep{P}} = \{ \quotep{Q} \in \quotep{\pi} | P \equiv Q \} \and \\ \meaningof{@\quotep{E}} = \{ P \in \pi | P \equiv @x, x \in \meaningof{E} \}}
\end{mathpar}

\begin{eqnarray*}
  \\
  \meaningof{-} : TS \to ST
\end{eqnarray*}

\begin{eqnarray*}
  \\
  L : TS \to ST
\end{eqnarray*}

\begin{eqnarray*}
  \\
  P \models E \iff P \in \meaningof{E}
\end{eqnarray*}

\begin{eqnarray*}
  P \approx_{L} Q \iff \forall E \in L. P \models E \iff Q \models E
\end{eqnarray*}

\begin{eqnarray*}
  P \approx_{K} Q
\end{eqnarray*}

\begin{eqnarray*}
  P \approx Q
\end{eqnarray*}

$\approx_{K} = \approx = \approx_{L}$

\subsubsection{Contextual duality}

Note that contexts extend the quotation operation to a family of
operations from processes to names. Given a context, $M$, we can
define a \emph{nominal context}, $\quotep{M}$ by $\quotep{M}[P] :=
\quotep{M[P]}$. To foreshadow what is to come we observe that these
operations enjoy a duality with processes very much like the duality
between vectors and maps from vectors to scalars.

Further, because the calculus is essentially higher-order, we have a
correspondence between contexts and processes. More specifically,
given a name $x$ and a context $M$ we can construct $M^{*}_{x}$ such
that 

\begin{mathpar}
  M^{*}_{x} | \lift{x}{P} \red M[P]
\end{mathpar}

namely,

\begin{mathpar}
  M^{*}_{x} := x?(u).M[\dropn{u}]
\end{mathpar}

The dependence of $M^{*}_{x}$ on a name makes it an abstraction, 

\begin{mathpar}
  M^{*} := (x)x?(u).M[\dropn{u}]
\end{mathpar}

\subsection{Additional notation}

It will sometimes be convenient to denote the process a name
quotes. We already have the notation $x = \quotep{P}$, but it will be
convenient to introduce an alternate notation, $\procn{x}$, when we
want to emphasize the connection to the use of the name. Note that, by
virtue of name equivalence, $\quotep{\procn{x}} \nameeq x$; so, the
notation is consistent with previous definitions.

Further, because names have structure it is possible to effect
substitutions on the basis of that structure. This means we need to
upgrade our notation for substitutions, which we accomplish by
adapting comprehension notation. Thus,

\begin{mathpar}
  P\{ y / x : x \in S \}
\end{mathpar}

is interpreted to mean the process derived from P by replacing (in a
capture-avoiding manner) each occurrence of $x$ in $S$ by $y$. For example,

\begin{mathpar}
  P\{ \quotep{\procn{x}|\procn{x}} / x : x \in \freenames{P} \}
\end{mathpar}

will replace each (occurrence) of a free name $x$ in $P$ by
$\quotep{\procn{x}|\procn{x}}$.

Also, we will avail ourselves of the notation $x^{L}$ and $x^{R}$ to
denote injections of a name into disjoint copies of the name
space. There are numerous ways to accomplish this. One example can be
found in \cite{MeredithR05}. This notation overloads to vectors of
names: $\vec{x}^{\pi} := (x_{i}^{\pi} \; : \; 0 \leq i < |\vec{x}| )$ where $\pi \in \{L,R\}$.

We also use $P^{\Box} := P|\Box$.

In \cite{MeredithR05} an interpretation of the new operator is
given. It turns out that there are several possible interpretations
all enjoying the requisite algebraic properties of the operator (see
\cite{milner91polyadicpi}). We will therefore make liberal use of
$(\nu\; \vec{x})P$.

% subsection the_syntax_and_semantics_of_the_notation_system (end)   

\section{Interpretation of QM}
\subsection{Supporting definitions}
\subsubsection{Multiplication}
\begin{mathpar}
  \quotep{Q} \cdot \quotep{R} := \quotep{Q|R}
  \and \\
  \quotep{Q} \cdot P := P\{ \quotep{Q|R} / \quotep{R} : \quotep{R} \in \freenames{P} \}
\end{mathpar}

\paragraph{Discussion}
The first line needs little explanation. The second line says that
each free name of the process is replaced with the multiplication of
that name by the scalar. Multiplication of a scalar (name) by a state
(process) results in a process all the names of which have been `moved
over' by parallel composition with the process the scalar
quotes. There is a subtlety that the bound names have to be
manipulated so that multiplied names aren't accidentally
captured. There are many ways to achieve this.

\begin{remark}\label{rem:multiplication_identities}
  The reader is invited to verify that for all $x,y,z \in \QProc$ and $P \in \Proc$
  \begin{mathpar}
    x \cdot \quotep{0} \equiv x 
    \and
    x \cdot y \equiv y \cdot x
    \and
    x \cdot (y \cdot z) \equiv (x \cdot y) \cdot z
    \and \\
    \quotep{0} \cdot P \equiv P
    \and \\
    x \cdot (y \cdot P) \equiv (x \cdot y) \cdot P
    \and \\
    x \cdot (P|Q) \equiv (x \cdot P) | (x \cdot Q)
    \and \\    
  \end{mathpar}
\end{remark}

\subsubsection{Tensor product}

We define a tensor product on processes by structural induction.

\paragraph{Tensor of sums} First note that all summations, including
$\pzero$ and sequence, can be written $\Sigma_{i} x_{i}.A_{i} +
\Sigma_{j} x_{j}.C_{j}$, where we have grouped input-guarded processes
together and output-guarded processes together.

Thus, we can define the tensor product of two summations, $N_{1}\otimes N_{2}$, where

\begin{mathpar}
  N_{1} := \Sigma_{i} x_{i}.A_{i} + \Sigma_{j} x_{j}.C_{j}
  \and
  N_{2} := \Sigma_{i'} y_{i'}.B_{i'} + \Sigma_{j'} y_{j'}.D_{j'} 
\end{mathpar}

as follows.

\begin{mathpar}
  \Sigma_{i} x_{i}.A_{i} + \Sigma_{j} x_{j}.C_{j} \otimes \Sigma_{i'}
  y_{i'}.B_{i'} + \Sigma_{j'} y_{j'}.D_{j'} 
  \and \\
  := \; \Sigma_{i} \Sigma_{i'} \quotep{\stackrel{\vee}{x_{i}}| \stackrel{\vee}{y_{i'}}}.(A_{i}\otimes B_{i'}) \; | \; \Sigma_{i'} \Sigma_{i} \quotep{\stackrel{\vee}{y_{i'}}|\stackrel{\vee}{x_{i}}}.(B_{i'}\otimes A_{i})
  \and
  \;\; | \;\; \Sigma_{j} \Sigma_{j'} \quotep{\stackrel{\vee}{x_{j}}|\stackrel{\vee}{y_{j'}}}.(A_{j}\otimes B_{j'}) \; | \; \Sigma_{j'} \Sigma_{j} \quotep{\stackrel{\vee}{y_{j'}}|\stackrel{\vee}{x_{j}}}.(B_{j'}\otimes A_{j})
\end{mathpar}

\begin{remark}
  Do we need to $x^{L}$ and $y^{R}$ for this construction as well?
\end{remark}

\paragraph{Tensor of parallel compositions} Next, we distribute tensor
over par.

\begin{mathpar}
  P_{1}|P_{2} \otimes Q_{1}|Q_{2} := (P_{1} \otimes Q_{1}) | (P_{1}
  \otimes Q_{2}) | (P_{2} \otimes Q_{1}) | (P_{2} \otimes Q_{2})
\end{mathpar}

\paragraph{Tensor with dropped names} We treat tensor of a
process with a dropped name as parallel composition.

\begin{mathpar}
  P \otimes \dropn{x} := P | \dropn{x}
\end{mathpar}

\paragraph{Tensor of agents}

Finally, we need to define tensor on agents. Note that the definition
of tensor on normal products only tensors inputs with inputs and
outputs with outputs. Thus, we only have to define the operation on
``homogeneous'' pairings.

\begin{mathpar}
  (\vec{x})P \otimes (\vec{y})Q
  \and \\
  := (x_{0}^{L}|y_{0}^{R},\ldots,x_{0}^{L}|y_{n}^{R},\ldots,x_{m}^{L}|y_{0}^{R},\ldots,x_{m}^{L}|y_{n}^R)(P\{ \vec{x}^{L}/\vec{x}\} \otimes Q \{ \vec{y}^{R}/\vec{y}\})
  \and \\
  \clift{\vec{P}} \otimes \clift{\vec{Q}}
  \and \\
  := \clift{P_{0}\otimes Q_{0},\ldots,P_{0}\otimes Q_{n},\ldots,P_{m}\otimes Q_{0},\ldots,P_{m}\otimes Q_{n}}
\end{mathpar}

\begin{remark}
  Observe that arities of tensored abstractions matches arities of
  tensored concretions if the original arities matched. Note also that
  the length of the arities corresponds to the increase in dimension
  we see in ordinary vector space tensor product.
\end{remark}

\begin{remark}
  Operationally, this definition distributes the tensor down to
  components ``linked'' by summation. Tensor over summation is
  intriguing in that it mixes names. Moreover, as a consequence of the
  way it mixes names we have the identities for all $x \in \QProc$ and
  $P,Q \in \Proc$

  \begin{mathpar}
    (x \cdot P) \otimes Q \equiv x \cdot (P \otimes Q) \equiv P \otimes (x \cdot Q)
    \and
    P \otimes \pzero \equiv P
  \end{mathpar}

  that the reader is invited to verify.
\end{remark}

\subsubsection{Annihilation}
\begin{mathpar}
  P^{\perp} := \{ Q | \forall R. P|Q \red^{*} R \Rightarrow R \red^{*} \pzero \}
  \and \\
  P^{\underline{\perp}} := \Sigma_{Q \in P^{\perp}} \quotep{Q}?(y).(\dropn{y}|Q) | \Sigma_{Q \in P^{\perp}} \quotep{Q}\clift{\Box}
\end{mathpar}

\paragraph{Discussion} The reader will note that $P^{\perp}$ is a
\emph{set} of processes, while $P^{\underline{\perp}}$ is a
\emph{context}. We call the set $P^{\perp}$ the \emph{annihilators} of
$P$. The parallel composition of a process in the annihilators of $P$
with $P$ will result in a process, the state space of which has all
paths eventually leading to $\pzero$. Execution may endure loops; but
under reasonable conditions of fairness (naturally guaranteed under
most notions of bisimulation) such a composite process cannot get
stuck in such a loop and will, eventually pop out and terminate.

The context $P^{\underline{\perp}}$ is ready and willing to ``take the
$P$ out of'' the process to which it is applied. It will effectively
transmit the code of the process to which it is applied to one of the
annihilators and run the process against it.

\subsubsection{Evaluation}
We fix $M$ a domain of fully abstract interpretation with an equality
coincident with bisimulation. We take $\meaningof{\cdot} : \Proc \to
M$ to be the map interpreting processes and $\nmeaningof{\cdot} : \M
\to Proc$ to be the map running the other way. Then we define

\begin{mathpar}
  \int P := \nmeaningof{\meaningof{P}}
\end{mathpar}

\paragraph{Discussion}
There are many fully abstract interpretations of Milner's
$\pi$-calculus. Any of them can be used as a basis for interpreting
the reflective calculus here. Equipped with such a domain it is
largely a matter of grinding through to check that the Yoneda
construction for the normalization-by-evaluation program can be
extended to this setting.

\begin{remark}
  The reader is invited to verify that $\int (P^{\underline{\perp}}[P]) = 0$.
\end{remark}

\subsection{Quantum mechanics}

Table \ref{tbl:core_qm_op_defns} gives the core operational definitions

\begin{table}[htp]\label{tbl:core_qm_op_defns}
  \center{
    \fbox{
      \begin{tabular}{c|c}
        quantum mechanics & process calculus \\
        \hline
        scalar & $x := \quotep{P}$ \\
        state vector & $\state{P} := P$ \\
        dual & $\state{P}^{*} := \event{P^{\underline{\perp}}} := \quotep{P^{\underline{\perp}}}[-]$ \\
        matrix & $ \Sigma_{\alpha} \state{P_{\alpha}}x_{\alpha}\event{Q_{\alpha}}$ \\
        vector addition & $\state{P} + \state{Q} := \state{P | Q}$ \\
        tensor product & $\state{P} \otimes \state{Q} := \state{P \otimes Q}$ \\
        inner product & $\innerprod{P}{Q} := \quotep{\int P^{\underline{\perp}}[Q]}$ \\
      \end{tabular}
    }
  }
  \caption{QM - operational definitions}
\end{table}

where

\begin{mathpar}
  \prmatrix{P}{Q} := \fprmatrix{P}{\quotep{\pzero}}{Q}
  \and
  \fprmatrix{P}{x}{Q} := (\state{P},x,\event{Q})
  \and
  (\fprmatrix{P}{x}{Q})(\state{R}) := x \cdot \innerprod{Q}{R} \cdot \state{P}
  \and
  (\fprmatrix{P}{x}{Q})(\event{R}) := x \cdot \innerprod{R}{P} \cdot \event{Q}
\end{mathpar}

\paragraph{Discussion}
As promised: vectors (aka states) are represented as processes; duals
as contextual duals; inner product definition should be compared with
standard inner product definition for ....

\begin{remark}
  Assuming $\int (P^{\underline{\perp}}[P]) = 0$, the reader is
  invited to verify that $(\fprmatrix{P}{x}{P})(\state{P}) = x \cdot \state{P}$.
\end{remark}

\begin{remark}
  The reader is invited to verify that $\innerprod{P}{Q}$ could
  equally well have been written $\quotep{\int \stackrel{\vee}{x}}$
  where $x = \event{P^{\underline{\perp}}}(Q)$.

  One of the motivations for this remark is that there is another way
  to factor these operations. We could package up evaluation in the dual:

  \begin{mathpar}
    \state{P}^{*} := \event{\int P^{\underline{\perp}}} := \quotep{\int P^{\underline{\perp}}}[-]
  \end{mathpar}

  and then have inner product defined by
  
  \begin{mathpar}
    \innerprod{P}{Q} := \event{P}(Q)
  \end{mathpar}

  Hopefully, experience with the calculations will provide guidance on
  the best factoring.
\end{remark}

\begin{remark}
  Assuming $\int (P^{\underline{\perp}}[P]) = 0$, the reader is
  invited to verify that $\forall P,Q. (\prmatrix{0}{Q})(\state{0}) =
  \state{0}$ and dually $(\prmatrix{P}{0})(\event{0}) = \event{0}$.
\end{remark}

\begin{remark}
  i'm a little worried that i don't (yet) have proper support for
  complex conjugacy. But, the observation above may give us a
  clue. According to Abramsky, it must be the case that the scalars
  are iso to the homset of the identity for the tensor -- which the
  observation above characterizes. 

  For now, we will simply bookmark the notion with $\overline{x}$.
\end{remark}

\subsubsection{Adjointness}

We need to give a definition of $(\cdot)^{\dagger}$ for matrices. The
obvious candidate definition is
\begin{mathpar}
(\Sigma_{\alpha}\fprmatrix{P_{\alpha}}{x_{\alpha}}{Q_{\alpha}})^{\dagger}
= \Sigma_{\alpha}\fprmatrix{(Q_{\alpha}^{\underline{\perp}})^{*}}{\overline{x}_{\alpha}}{P_{\alpha}^{\underline{\perp}}} 
\end{mathpar}

But, $(Q_{\alpha}^{\underline{\perp}})^{*}$ requires a name along
which to communicate the process to achieve the context application.

\subsubsection{Basis for a basis}
If processes label states and ``addition'' of states (a.k.a. vector
addition) is interpreted as parallel composition, what corresponds to
notions of linear independence and basis? Here, we recall that Yoshida
has developed a set of \emph{combinators} for an asynchronous verison
of Milner's $\pi$-calculus. These are a finite set of processes such
any process can be expressed as parallel composition of these
combinators together with liberal uses of the new operator and
replication. We can simply give a translation of these into the
present calculus and have reasonable expectation that the property
carries over. That is, that the resultant set allows to express all
processes via parallel composition. Note, however, that there is no
new operator or replication in this calculus. As a result, we expect
that the corresponding set is actually infinite. That is, we expect
that the space is actually infinite dimensional.

\begin{remark}
  The attentive reader may be a bit concerned. Certainly, the
  collection $S$, $K$ and $I$ is a finite set of
  combinators. Shouldn't we expect to see a finite set of combinators
  for an effectively equivalent system? i am very sympathetic to this
  critique and feel it warrants full attention. On the other hand, i
  also have in mind the following analogy. The natural numbers, as a
  monoid under addition, has exactly $1$ generator, while the natural
  numbers, as a monoid under multiplication, has countably many
  generators (the primes). We observe that the application of the
  lambda calculus is much less resource sensitive than the parallel
  composition of the $\pi$-calculus. Could it be the case that we have
  an analogy of the form
  
  \begin{mathpar}
    m + n : MN :: m*n : M|N
  \end{mathpar}

  giving a similar blow up in the set of ``primes''?  This is such a
  wonderful thought that, even if it's not true, i think it's worth
  writing down.
\end{remark}
 

\documentclass[12pt]{llncs}
%\documentclass{jktr}

\usepackage[pdftex]{hyperref}                   
\usepackage {listings}
\usepackage {mathpartir}
\usepackage{bcprules}
%\usepackage{listings}
                       
\usepackage{graphicx} 
%\usepackage[margins=2.5cm,nohead,nofoot]{geometry}
%\usepackage{geometry}
\usepackage{amsfonts}
\usepackage{amstext}
\usepackage{latexsym}
\usepackage{amssymb}
\usepackage{color}


%\include{myPreamble}
\include{qm2pi.local} 

%\ifpdf
%\usepackage[pdftex]{graphicx}
%\else
%\usepackage{graphicx}
%\fi

 % \ifpdf
%  \usepackage{pdfsync}
%  \if


%\title{Brief Article}
%\author{David F. Snyder}
%\author{L.G. Meredith}

%\address{Dept. of Math., Texas State University--San Marcos, San Marcos, TX 78666}
       
\pagestyle{empty}


\begin{document}

\lstset{language=[Objective]Caml,frame=shadowbox}

\input{qm2pi.front}

% section front matter (end)

\input{qm2pi.intro} 
 
% section introduction (end)

% \input{qm2pi.knotations} 

% section notation (end)

\input{qm2pi.process.calculi} 

% section concurrent_process_calculi_and_spatial_logics_ (end)
    
%\input{qm2pi.knots2pi} 

%\input{qm2pi.trefoil} 

%\input{qm2pi.mainthm} 

% subsection basic_interpretation (end)

%\input{qm2pi.rho.presentation} 
\subsection{The syntax and semantics of the notation system}\label{sub:the_syntax_and_semantics_of_the_notation_system} % (fold)

We now summarize a technical presentation of the calculus that
embodies our theory of dynamics. The typical presentation of such a
calculus follows the style of giving generators and relations on
them. The grammar, below, describing term constructors, freely
generates the set of processes, $\Proc$. This set is then quotiented
by a relation known as structural congruence and it is over this set
that the notion of dynamics is expressed. This presentation is
essentially that of \cite{MeredithR05} with the addition of
polyadicity and summation. For readability we have relegated some of
the technical subtleties to an appendix.

\subsubsection{Process grammar}\label{subsub:process_grammar}

\begin{mathpar}
  \inferrule* [lab=synchronization] {} {{M} \bc \pzero \;|\; x?F \;|\; x!C }
  \and
  \inferrule* [lab=abstraction] {} {{F} \bc (x)P}
  \and
  \inferrule* [lab=concretion] {} {{C} \bc \langle Q \rangle}
  \and
  \inferrule* [lab=process] {} {{P,Q} \bc M \;| \;P|Q \;|\; @{x}}
  \and
  \inferrule* [lab=name] {} {{x} \bc \quotep{P}}
\end{mathpar} 

Note that $\vec{x}$ (resp. $\vec{P}$) denotes a vector of names
(resp. processes) of length $|\vec{x}|$ (resp. $|\vec{P}|$). We adopt
the following useful abbreviations.

\begin{mathpar}
   x?(\vec{y}).P := x.(\vec{y})P \and  x\clift{\vec{P}} := x.\clift{\vec{P}}
   \and x!(y) := \lift{x}{\dropn{y}}
   \and \Pi_{i=0}^{n-1}P_i := P_0 | \ldots | P_{n-1}
\end{mathpar}

\subsubsection{Structural congruence}

\paragraph{Free and bound names and alpha-equivalence.} At the
core of structural equivalence is alpha-equivalence which identifies
process that are the same up to a change of variable. Formally, we
recognize the distinction between free and bound names. The free names
of a process, $\freenames{P}$, may be calculated recursively as
follows:

\begin{mathpar}
\freenames{\pzero} := \emptyset
  \and \\
  \freenames{x?(y).P} := \{ x \} \cup (\freenames{P} \setminus \{ y \})
  \and 
  \freenames{x!\langle P \rangle} := \{ x \} \cup \{ P \} 
  \and \\
  \freenames{P|Q} := \freenames{P} \cup \freenames{Q}
  \and \\
  \freenames{@{x}} := \{ x \}
\end{mathpar}

$\pi$
$\quotep{\pi}$

$\freenames{-} : \pi \to \mathcal{P}(\quotep{\pi})$

\begin{eqnarray*}
  \freenames{\pzero} & := & \emptyset \\
  \freenames{x?(y).P} & := & \{ x \} \cup (\freenames{P} \setminus \{ y \}) \\
  \freenames{x!\langle P \rangle} & := & \{ x \} \cup \{ P \} \\
  \freenames{P|Q} & := & \freenames{P} \cup \freenames{Q} \\
  \freenames{\dropn{x}} & := & \{ x \}
\end{eqnarray*}

The bound names of a process, $\boundnames{P}$, are those names occurring in $P$
that are not free. For example, in $x?(y).0$, the name $x$ is free, while $y$ is bound.

\begin{mathpar}
  \inferrule* [lab=monoidal-laws] {} { P|Q \equiv Q|P \and P|0 \equiv P \and P|(Q|R) \equiv (P|Q)|R }
\end{mathpar}

\begin{mathpar}
  \inferrule* [lab=alpha-equivalence] {} { (x)P \equiv (y)P\{y/x\} \and y \not\in \freenames{P} }
\end{mathpar}

\begin{definition}
Then two processes, $P,Q$, are alpha-equivalent if $P = Q\{\vec{y}/\vec{x}\}$ for
some $\vec{x} \in \boundnames{Q},\vec{y} \in \boundnames{P}$, where $Q\{\vec{y}/\vec{x}\}$
denotes the capture-avoiding substitution of $\vec{y}$ for $\vec{x}$ in $Q$.
\end{definition}

\begin{definition}
  The {\em structural congruence} \cite{SangiorgiWalker} , $\equiv$,
  between processes is the least congruence containing
  alpha-equivalence, satisfying the abelian monoid laws
  (associativity, commutativity and $\pzero$ as identity) for parallel
  composition $|$ and for summation $+$.
\end{definition}

\subsection{Name equivalence}

We take name equivalence, written $\nameeq$, to be the smallest
equivalence relation generated by the following rules.

\begin{mathpar}
\inferrule*[lab=Quote-drop]
{ }
{ \quotep{@{x}} \nameeq x }

\inferrule*[lab=Struct-equiv]
{ P \scong Q }
{ \quotep{P} \nameeq \quotep{Q} }
\end{mathpar}

The astute reader will have noticed that the mutual recursion of names
and processes imposes a mutual recursion on alpha-equivalence and
structural equivalence via name-equivalence. Fortunately, all of this
works out pleasantly and we may calculate in the natural way, free of
concern. The reader interested in the details is referred to the
appendix \ref{appendix:rho_details}.

\subsection{Substitution}

We use $\Proc$ for the set of processes, $\QProc$ for the set of
names, and $\id{\{}\vec{y} / \vec{x} \id{\}}$ to denote partial maps,
$s : \QProc \rightarrow \QProc$. A map, $s$ lifts, uniquely, to a map
on process terms, $\widehat{s} : \Proc \rightarrow \Proc$ by the
following equations.

\begin{mathpar}
  (0) \psubstp{Q}{P} := 0 \\
  (R \juxtap S) \psubstp{Q}{P}
  :=    
  (R)\psubstp{Q}{P} \juxtap (S) \psubstp{Q}{P} \\
  (x?(y).R) \psubstp{Q}{P}    
  :=    
  (x)\substp{Q}{P} (z)\concat( (R \psubstn{z}{y}) \psubstp{Q}{P} ) \\
  (\lift{x}{R}) \psubstp{Q}{P}  
  :=
  \lift{(x)\substp{Q}{P}}{ R \psubstp{Q}{P} } \\
%   (\dropn{x})  \psubstp{Q}{P}       
%   := 
%   \left\{ 
%     \begin{array}{ccc} 
%       \dropn{\quotep{Q}} & & x \nameeq \quotep{P} \\
%       \dropn{x} & & otherwise \\
%     \end{array}
%   \right. 
  (\dropn{x})  \psubstp{Q}{P}       
  := 
  \left\{ 
    \begin{array}{ccc} 
      Q & & x \nameeq \quotep{P} \\
      \dropn{x} & & otherwise \\
    \end{array}
  \right.
\end{mathpar}
 

where

\begin{eqnarray}
  (x)\id{\{} \lpquote Q \rpquote / \lpquote P \rpquote \id{\}}            = 
  \left\{ 
    \begin{array}{ccc}
      \lpquote Q \rpquote & & x \nameeq \lpquote P \rpquote \\
      x & & otherwise \\
    \end{array}
  \right. \nonumber
\end{eqnarray}

and $z$ is chosen distinct from $\quotep{P}$, $\quotep{Q}$, the free
names in $Q$, and all the names in $R$. Our $\alpha$-equivalence will
be built in the standard way from this substitution.

\begin{remark}\label{rem:no_self_referential_names}
  One consequence of these definitions is that $\forall P. \quotep{P}
  \not\in \freenames{P}$.
\end{remark}

\subsection{ Dynamic quote: an example }

Anticipating something of what's to come, consider applying the
substitution, $\widehat{\id{\{}u / z \id{\}}}$, to the following pair
of processes, $\lift{w}{y!(z)}$ and $w[ \lpquote y!(z) \rpquote ]$.

\begin{eqnarray}
	\lift{w}{y!(z)}\widehat{\id{\{}u / z \id{\}}}
		& = &
		\lift{w}{y!(u)} \nonumber\\
	w[ \lpquote y!(z) \rpquote ] \widehat{ \id{\{}u / z \id{\}} }
		& = &
		w[ \lpquote y!(z) \rpquote ] \nonumber
\end{eqnarray}

Because the body of the process between quotes is impervious to
substitution, we get radically different answers. In fact, by
examining the first process in an input context,
e.g. $x?(z).\lift{w}{y!(z)}$, we see that the process under the lift
operator may be shaped by prefixed inputs binding a name inside it. In
this sense, the lift operator will be seen as a way to dynamically
construct processes before reifying them as names.

Finally equipped with these standard features we can present the
dynamics of the calculus.

\subsubsection{Operational semantics} 

Finally, we introduce the computational dynamics. What marks these
algebras as distinct from other more traditionally studied algebraic
structures, e.g. vector spaces or polynomial rings, is the manner in
which dynamics is captured. In traditional structures, dynamics is typically
expressed through morphisms between such structures, as in linear maps
between vector spaces or morphisms between rings. In algebras
associated with the semantics of computation, the dynamics is
expressed as part of the algebraic structure itself, through a
reduction reduction relation typically denoted by $\red$. Below, we
give a recursive presentation of this relation for the calculus used
in the encoding.

$\red \subseteq \pi \times \pi$
$\red : \pi \to \mathcal{P}(\pi)$

\begin{mathpar}
  \inferrule* [lab=Comm] { \textsf{match}( x_{src}, x_{trgt} ) } { x_{trgt}?(y)P \; | \; x_{src}!\langle {Q} \rangle \red P\{\quotep{Q}/y}\} }
  \and \\
  \inferrule* [lab=Par] {{P} \red {P}'} {{{P} | {Q}} \red {{P}' | {Q}}}
  \and
  \inferrule* [lab=Equiv]{{{P} \scong {P}'} \andalso {{P}' \red {Q}'} \andalso {{Q}' \scong {Q}}}{{P} \red {Q}}
\end{mathpar}

\begin{eqnarray*}
  match_{\equiv} (\quotep{P},\quotep{Q}) & := & P \equiv Q \\
  match_{\dagger}(\quotep{P},\quotep{Q}) & := & \forall R. P|Q \red^{*} R => R \red^{*} 0 \\
  match_{K}(\quotep{P},\quotep{Q}) & := & K \mbox{ for some context } K
\end{eqnarray*}

$u?(x)P | u!\langle Q \rangle \red P\{\quotep{Q}/x\}$

%We write $\wred$ for $\red^*$, and $P\red$ if $\exists Q $ such that $ P \red Q$.
We write $P\red$ if $\exists Q $ such that $ P \red Q$ and $P\not\red$, otherwise.

\section{Replication}

As mentioned before, it is known that replication (and hence
recursion) can be implemented in a higher-order process algebra
\cite{SangiorgiWalker}. As our first example of calculation with the
machinery thus far presented we give the construction explicitly in
the {\rhoc}.

\begin{eqnarray}
	D_{x} & := & \prefix{x}{y}{(\binpar{\outputp{x}{y}}{@{y}})} \nonumber\\
	\bangp_{x}{P} & := & \binpar{{x}!\langle{\binpar{D_{x}}{P}}\rangle}{D_{x}} \nonumber
\end{eqnarray}

\begin{eqnarray}
	\bangp_{x}{P} & & \nonumber\\
	=
	& {x}!\langle{(\prefix{x}{y}{(\outputp{x}{y} | @{y})) | P}}\rangle 
	      | \prefix{x}{y}{(\outputp{x}{y} | @{y})} & \nonumber\\
	\red
	& (\outputp{x}{y} | @{y})\substn{\quotep{(\prefix{x}{y}{(@{y} | \outputp{x}{y})) | P}}}{y} & \nonumber\\
	=
	& \outputp{x}{\quotep{(\prefix{x}{y}{(\outputp{x}{y} | @{y})) | P}}}
	  | {(\prefix{x}{y}{(\outputp{x}{y} | @{y})) | P}} & \nonumber\\
	\red
	& \ldots & \nonumber\\
	\red^*
	& P | P | \ldots & \nonumber
\end{eqnarray}

Of course, this encoding, as an implementation, runs away, unfolding
$\bangp{P}$ eagerly. A lazier and more implementable replication
operator, restricted to input-guarded processes, may be obtained as follows.

\begin{eqnarray}
\bangp{\prefix{u}{v}{P}} 
	:= 
	\binpar{\lift{x}{\prefix{u}{v}{(\binpar{D(x)}{P})}}}{D(x)} \nonumber
\end{eqnarray}

\begin{remark}
  Note that the lazier definition still does not deal with summation
  or mixed summation (i.e. sums over input and output). The reader is
  invited to construct definitions of replication that deal with these
  features. 

  Further, the definitions are parameterized in a name, $x$. Can you,
  gentle reader, make a definition that eliminates this parameter and
  guarantees no accidental interaction between the replication
  machinery and the process being replicated -- i.e. no accidental
  sharing of names used by the process to get its work done and the
  name(s) used by the replication to effect copying. This latter
  revision of the definition of replication is crucial to obtaining
  the expected identity $!!P \sim !P$.
\end{remark}

\begin{remark}\label{rem:paradoxical_combinator}
  The reader familiar with the lambda calculus will have noticed the
  similarity between $D$ and the paradoxical combinator.

  [Ed. note: the existence of this seems to suggest we have to be more
  restrictive on the set of processes and names we admit if we are to
  support no-cloning.]
\end{remark}

\subsubsection{Bisimulation}

The computational dynamics gives rise to another kind of equivalence,
the equivalence of computational behavior. As previously mentioned
this is typically captured \emph{via} some form of bisimulation.

% The notion we use in this paper is weak barbed bisimulation
% \cite{milner91polyadicpi}.

The notion we use in this paper is derived from weak barbed
bisimulation \cite{milner91polyadicpi}. 

\begin{definition}
An \emph{observation relation}, $\downarrow_{\mathcal N}$, over a set
of names, $\mathcal N$, is the smallest relation satisfying the rules
below.

\infrule[Out-barb]{y \in {\mathcal N}, \; x \nameeq y}
		  {\outputp{x}{v} \downarrow_{\mathcal N} x}
\infrule[Par-barb]{\mbox{$P\downarrow_{\mathcal N} x$ or $Q\downarrow_{\mathcal N} x$}}
		  {\binpar{P}{Q} \downarrow_{\mathcal N} x}

We write $P \Downarrow_{\mathcal N} x$ if there is $Q$ such that 
$P \wred Q$ and $Q \downarrow_{\mathcal N} x$.
\end{definition}

\begin{definition}
%\label{def.bbisim}
An  ${\mathcal N}$-\emph{barbed bisimulation} over a set of names, ${\mathcal N}$, is a symmetric binary relation 
${\mathcal S}_{\mathcal N}$ between agents such that $P\rel{S}_{\mathcal N}Q$ implies:
\begin{enumerate}
\item If $P \red P'$ then $Q \wred Q'$ and $P'\rel{S}_{\mathcal N} Q'$.
\item If $P\downarrow_{\mathcal N} x$, then $Q\Downarrow_{\mathcal N} x$.
\end{enumerate}
$P$ is ${\mathcal N}$-barbed bisimilar to $Q$, written
$P \wbbisim_{\mathcal N} Q$, if $P \rel{S}_{\mathcal N} Q$ for some ${\mathcal N}$-barbed bisimulation ${\mathcal S}_{\mathcal N}$.
\end{definition}

$\mathcal{R} \subseteq \pi \times \pi$

$P \mathcal{R} Q => \forall P'. P \red P' \Rightarrow \exists Q'. Q \red Q', P' \mathcal{R} Q'$

$P \vdash x \Rightarrow Q \vdash x$

\begin{mathpar}
  \inferrule*[lab=Out-barb]{x \nameeq y}{{y}!\langle{Q}\rangle \vdash x}
  \and
  \inferrule*[lab=Par-barb]{\mbox{$P\vdash x$ or $Q\vdash x$}}{\binpar{P}{Q} \vdash x}
\end{mathpar}

\subsubsection{Contexts}

One of the principle advantages of computational calculi like the
$\pi$-calculus is a well-defined notion of context,
contextual-equivalence and a correlation between
contextual-equivalence and notions of bisimulation. The notion of
context allows the decomposition of a process into (sub-)process and
its syntactic environment, its context. Thus, a context may be
thought of as a process with a ``hole'' (written $\Box$) in it. The
application of a context $M$ to a process $P$, written $M[P]$, is
tantamount to filling the hole in $M$ with $P$. In this paper we do
not need the full weight of this theory, but do make use of the notion
of context in the proof the main theorem. 

\begin{mathpar}
  \inferrule* [lab=summation] {} {{M_{M},M_{N}} \bc \Box \;|\; x.M_{A} \;|\; M_{M}+M_{N}}
  \and
  \inferrule* [lab=agent] {} {{M_{A}} \bc (\vec{x})M_{P} \;| \; \clift{P_0,\ldots,M_{P},\ldots,P_N}}
  \and \\
  \inferrule* [lab=process] {} {{M_{P}} \bc M_{N} \;| \;P|M_{P} }
\end{mathpar} 

\begin{mathpar}
  \inferrule* [lab=sychronization] {} {M_{N} \bc \Box \;|\; x?M_{F} \;|\; x!M_{C}}
  \and
  \inferrule* [lab=abstraction] {} {{M_{F}} \bc (x)M_{P} }
  \and
  \inferrule* [lab=concretion] {} {{M_{C}} \bc \langle M_{P} \rangle }
  \and \\
  \inferrule* [lab=process] {} {{M_{P}} \bc M_{N} \;| \;P|M_{P} }
\end{mathpar}

\begin{definition}[contextual application] Given a context $M$, and
  process $P$, we define the \emph{contextual application}, $M[P] :=
  M\{P/\Box\}$. That is, the contextual application of M to P is the
  substitution of $P$ for $\Box$ in $M$.
\end{definition}

$\meaningof{-} : L \to \mathcal{P}(\pi)$

\begin{mathpar}
  \inferrule* [lab=collection] {} {\meaningof{true} = \pi, \and \meaningof{~E} = \pi \setminus \meaningof{E}, \and \meaningof{E_{1} \& E_{2}} = \meaningof{E_{1}} \cap \meaningof{E_{2}}}
\end{mathpar}

\begin{mathpar}
  \inferrule* [lab=structure] {} {\meaningof{0} = \{ P \in \pi | P \equiv 0 \}, \and \\ \meaningof{E_1 | E_2} = \{ P \in \pi | P \equiv P_{1} | P_{2}, P_{1} \in \meaningof{E_{1}}, P_{2} \in \meaningof{E_2}\} }
\end{mathpar}

\begin{mathpar}
 \inferrule* [lab=behavior] {} {\meaningof{\langle a?b \rangle E} = \{ P \in \pi | P \equiv Q | u?(y)P', \\ \and \\\\ \and \\ \;\;\; u \in \meaningof{a}, \forall z.P'\{z/y\} \in \meaningof{E\{z/b\}}\}, \and \\ \meaningof{a!E} = \{ P \in \pi | P \equiv Q | x!\langle P' \rangle, x \in \meaningof{a} P' \in \meaningof{E}\} }
\end{mathpar}

\begin{mathpar}
 \inferrule* [lab=nominal] {} {\meaningof{\quotep{E}} = \{ \quotep{P} \in \quotep{\pi} | P \in \meaningof{E} \}, \and \meaningof{\quotep{P}} = \{ \quotep{Q} \in \quotep{\pi} | P \equiv Q \} \and \\ \meaningof{@\quotep{E}} = \{ P \in \pi | P \equiv @x, x \in \meaningof{E} \}}
\end{mathpar}

\begin{eqnarray*}
  \\
  \meaningof{-} : TS \to ST
\end{eqnarray*}

\begin{eqnarray*}
  \\
  L : TS \to ST
\end{eqnarray*}

\begin{eqnarray*}
  \\
  P \models E \iff P \in \meaningof{E}
\end{eqnarray*}

\begin{eqnarray*}
  P \approx_{L} Q \iff \forall E \in L. P \models E \iff Q \models E
\end{eqnarray*}

\begin{eqnarray*}
  P \approx_{K} Q
\end{eqnarray*}

\begin{eqnarray*}
  P \approx Q
\end{eqnarray*}

$\approx_{K} = \approx = \approx_{L}$

\subsubsection{Contextual duality}

Note that contexts extend the quotation operation to a family of
operations from processes to names. Given a context, $M$, we can
define a \emph{nominal context}, $\quotep{M}$ by $\quotep{M}[P] :=
\quotep{M[P]}$. To foreshadow what is to come we observe that these
operations enjoy a duality with processes very much like the duality
between vectors and maps from vectors to scalars.

Further, because the calculus is essentially higher-order, we have a
correspondence between contexts and processes. More specifically,
given a name $x$ and a context $M$ we can construct $M^{*}_{x}$ such
that 

\begin{mathpar}
  M^{*}_{x} | \lift{x}{P} \red M[P]
\end{mathpar}

namely,

\begin{mathpar}
  M^{*}_{x} := x?(u).M[\dropn{u}]
\end{mathpar}

The dependence of $M^{*}_{x}$ on a name makes it an abstraction, 

\begin{mathpar}
  M^{*} := (x)x?(u).M[\dropn{u}]
\end{mathpar}

\subsection{Additional notation}

It will sometimes be convenient to denote the process a name
quotes. We already have the notation $x = \quotep{P}$, but it will be
convenient to introduce an alternate notation, $\procn{x}$, when we
want to emphasize the connection to the use of the name. Note that, by
virtue of name equivalence, $\quotep{\procn{x}} \nameeq x$; so, the
notation is consistent with previous definitions.

Further, because names have structure it is possible to effect
substitutions on the basis of that structure. This means we need to
upgrade our notation for substitutions, which we accomplish by
adapting comprehension notation. Thus,

\begin{mathpar}
  P\{ y / x : x \in S \}
\end{mathpar}

is interpreted to mean the process derived from P by replacing (in a
capture-avoiding manner) each occurrence of $x$ in $S$ by $y$. For example,

\begin{mathpar}
  P\{ \quotep{\procn{x}|\procn{x}} / x : x \in \freenames{P} \}
\end{mathpar}

will replace each (occurrence) of a free name $x$ in $P$ by
$\quotep{\procn{x}|\procn{x}}$.

Also, we will avail ourselves of the notation $x^{L}$ and $x^{R}$ to
denote injections of a name into disjoint copies of the name
space. There are numerous ways to accomplish this. One example can be
found in \cite{MeredithR05}. This notation overloads to vectors of
names: $\vec{x}^{\pi} := (x_{i}^{\pi} \; : \; 0 \leq i < |\vec{x}| )$ where $\pi \in \{L,R\}$.

We also use $P^{\Box} := P|\Box$.

In \cite{MeredithR05} an interpretation of the new operator is
given. It turns out that there are several possible interpretations
all enjoying the requisite algebraic properties of the operator (see
\cite{milner91polyadicpi}). We will therefore make liberal use of
$(\nu\; \vec{x})P$.

% subsection the_syntax_and_semantics_of_the_notation_system (end)   

\input{qm2pi.qmops} 

\input{qm2pi.sterngerlach} 

\input{qm2pi.metric} 

% section concurrent_process_calculi (end)

%\input{qm2pi.proofsketch}

% section proof sketch (end)

%\input{qm2pi.slviaknots} 

% section spatial logic via knots (end)

\input{qm2pi.conclusion}

% section conclusion (end)

%\input{qm2pi.dtcodes} 

% section wiring algorithm (end)

\input{qm2pi.ack} 

% section acknowledgments (end)

\newpage


\bibliographystyle{plain}   
\bibliography{../../biblios/main.bib}

\input{qm2pi.rhodetails}

\end{document}

 

\documentclass[12pt]{llncs}
%\documentclass{jktr}

\usepackage[pdftex]{hyperref}                   
\usepackage {listings}
\usepackage {mathpartir}
\usepackage{bcprules}
%\usepackage{listings}
                       
\usepackage{graphicx} 
%\usepackage[margins=2.5cm,nohead,nofoot]{geometry}
%\usepackage{geometry}
\usepackage{amsfonts}
\usepackage{amstext}
\usepackage{latexsym}
\usepackage{amssymb}
\usepackage{color}


%\include{myPreamble}
\include{qm2pi.local} 

%\ifpdf
%\usepackage[pdftex]{graphicx}
%\else
%\usepackage{graphicx}
%\fi

 % \ifpdf
%  \usepackage{pdfsync}
%  \if


%\title{Brief Article}
%\author{David F. Snyder}
%\author{L.G. Meredith}

%\address{Dept. of Math., Texas State University--San Marcos, San Marcos, TX 78666}
       
\pagestyle{empty}


\begin{document}

\lstset{language=[Objective]Caml,frame=shadowbox}

\input{qm2pi.front}

% section front matter (end)

\input{qm2pi.intro} 
 
% section introduction (end)

% \input{qm2pi.knotations} 

% section notation (end)

\input{qm2pi.process.calculi} 

% section concurrent_process_calculi_and_spatial_logics_ (end)
    
%\input{qm2pi.knots2pi} 

%\input{qm2pi.trefoil} 

%\input{qm2pi.mainthm} 

% subsection basic_interpretation (end)

%\input{qm2pi.rho.presentation} 
\subsection{The syntax and semantics of the notation system}\label{sub:the_syntax_and_semantics_of_the_notation_system} % (fold)

We now summarize a technical presentation of the calculus that
embodies our theory of dynamics. The typical presentation of such a
calculus follows the style of giving generators and relations on
them. The grammar, below, describing term constructors, freely
generates the set of processes, $\Proc$. This set is then quotiented
by a relation known as structural congruence and it is over this set
that the notion of dynamics is expressed. This presentation is
essentially that of \cite{MeredithR05} with the addition of
polyadicity and summation. For readability we have relegated some of
the technical subtleties to an appendix.

\subsubsection{Process grammar}\label{subsub:process_grammar}

\begin{mathpar}
  \inferrule* [lab=synchronization] {} {{M} \bc \pzero \;|\; x?F \;|\; x!C }
  \and
  \inferrule* [lab=abstraction] {} {{F} \bc (x)P}
  \and
  \inferrule* [lab=concretion] {} {{C} \bc \langle Q \rangle}
  \and
  \inferrule* [lab=process] {} {{P,Q} \bc M \;| \;P|Q \;|\; @{x}}
  \and
  \inferrule* [lab=name] {} {{x} \bc \quotep{P}}
\end{mathpar} 

Note that $\vec{x}$ (resp. $\vec{P}$) denotes a vector of names
(resp. processes) of length $|\vec{x}|$ (resp. $|\vec{P}|$). We adopt
the following useful abbreviations.

\begin{mathpar}
   x?(\vec{y}).P := x.(\vec{y})P \and  x\clift{\vec{P}} := x.\clift{\vec{P}}
   \and x!(y) := \lift{x}{\dropn{y}}
   \and \Pi_{i=0}^{n-1}P_i := P_0 | \ldots | P_{n-1}
\end{mathpar}

\subsubsection{Structural congruence}

\paragraph{Free and bound names and alpha-equivalence.} At the
core of structural equivalence is alpha-equivalence which identifies
process that are the same up to a change of variable. Formally, we
recognize the distinction between free and bound names. The free names
of a process, $\freenames{P}$, may be calculated recursively as
follows:

\begin{mathpar}
\freenames{\pzero} := \emptyset
  \and \\
  \freenames{x?(y).P} := \{ x \} \cup (\freenames{P} \setminus \{ y \})
  \and 
  \freenames{x!\langle P \rangle} := \{ x \} \cup \{ P \} 
  \and \\
  \freenames{P|Q} := \freenames{P} \cup \freenames{Q}
  \and \\
  \freenames{@{x}} := \{ x \}
\end{mathpar}

$\pi$
$\quotep{\pi}$

$\freenames{-} : \pi \to \mathcal{P}(\quotep{\pi})$

\begin{eqnarray*}
  \freenames{\pzero} & := & \emptyset \\
  \freenames{x?(y).P} & := & \{ x \} \cup (\freenames{P} \setminus \{ y \}) \\
  \freenames{x!\langle P \rangle} & := & \{ x \} \cup \{ P \} \\
  \freenames{P|Q} & := & \freenames{P} \cup \freenames{Q} \\
  \freenames{\dropn{x}} & := & \{ x \}
\end{eqnarray*}

The bound names of a process, $\boundnames{P}$, are those names occurring in $P$
that are not free. For example, in $x?(y).0$, the name $x$ is free, while $y$ is bound.

\begin{mathpar}
  \inferrule* [lab=monoidal-laws] {} { P|Q \equiv Q|P \and P|0 \equiv P \and P|(Q|R) \equiv (P|Q)|R }
\end{mathpar}

\begin{mathpar}
  \inferrule* [lab=alpha-equivalence] {} { (x)P \equiv (y)P\{y/x\} \and y \not\in \freenames{P} }
\end{mathpar}

\begin{definition}
Then two processes, $P,Q$, are alpha-equivalent if $P = Q\{\vec{y}/\vec{x}\}$ for
some $\vec{x} \in \boundnames{Q},\vec{y} \in \boundnames{P}$, where $Q\{\vec{y}/\vec{x}\}$
denotes the capture-avoiding substitution of $\vec{y}$ for $\vec{x}$ in $Q$.
\end{definition}

\begin{definition}
  The {\em structural congruence} \cite{SangiorgiWalker} , $\equiv$,
  between processes is the least congruence containing
  alpha-equivalence, satisfying the abelian monoid laws
  (associativity, commutativity and $\pzero$ as identity) for parallel
  composition $|$ and for summation $+$.
\end{definition}

\subsection{Name equivalence}

We take name equivalence, written $\nameeq$, to be the smallest
equivalence relation generated by the following rules.

\begin{mathpar}
\inferrule*[lab=Quote-drop]
{ }
{ \quotep{@{x}} \nameeq x }

\inferrule*[lab=Struct-equiv]
{ P \scong Q }
{ \quotep{P} \nameeq \quotep{Q} }
\end{mathpar}

The astute reader will have noticed that the mutual recursion of names
and processes imposes a mutual recursion on alpha-equivalence and
structural equivalence via name-equivalence. Fortunately, all of this
works out pleasantly and we may calculate in the natural way, free of
concern. The reader interested in the details is referred to the
appendix \ref{appendix:rho_details}.

\subsection{Substitution}

We use $\Proc$ for the set of processes, $\QProc$ for the set of
names, and $\id{\{}\vec{y} / \vec{x} \id{\}}$ to denote partial maps,
$s : \QProc \rightarrow \QProc$. A map, $s$ lifts, uniquely, to a map
on process terms, $\widehat{s} : \Proc \rightarrow \Proc$ by the
following equations.

\begin{mathpar}
  (0) \psubstp{Q}{P} := 0 \\
  (R \juxtap S) \psubstp{Q}{P}
  :=    
  (R)\psubstp{Q}{P} \juxtap (S) \psubstp{Q}{P} \\
  (x?(y).R) \psubstp{Q}{P}    
  :=    
  (x)\substp{Q}{P} (z)\concat( (R \psubstn{z}{y}) \psubstp{Q}{P} ) \\
  (\lift{x}{R}) \psubstp{Q}{P}  
  :=
  \lift{(x)\substp{Q}{P}}{ R \psubstp{Q}{P} } \\
%   (\dropn{x})  \psubstp{Q}{P}       
%   := 
%   \left\{ 
%     \begin{array}{ccc} 
%       \dropn{\quotep{Q}} & & x \nameeq \quotep{P} \\
%       \dropn{x} & & otherwise \\
%     \end{array}
%   \right. 
  (\dropn{x})  \psubstp{Q}{P}       
  := 
  \left\{ 
    \begin{array}{ccc} 
      Q & & x \nameeq \quotep{P} \\
      \dropn{x} & & otherwise \\
    \end{array}
  \right.
\end{mathpar}
 

where

\begin{eqnarray}
  (x)\id{\{} \lpquote Q \rpquote / \lpquote P \rpquote \id{\}}            = 
  \left\{ 
    \begin{array}{ccc}
      \lpquote Q \rpquote & & x \nameeq \lpquote P \rpquote \\
      x & & otherwise \\
    \end{array}
  \right. \nonumber
\end{eqnarray}

and $z$ is chosen distinct from $\quotep{P}$, $\quotep{Q}$, the free
names in $Q$, and all the names in $R$. Our $\alpha$-equivalence will
be built in the standard way from this substitution.

\begin{remark}\label{rem:no_self_referential_names}
  One consequence of these definitions is that $\forall P. \quotep{P}
  \not\in \freenames{P}$.
\end{remark}

\subsection{ Dynamic quote: an example }

Anticipating something of what's to come, consider applying the
substitution, $\widehat{\id{\{}u / z \id{\}}}$, to the following pair
of processes, $\lift{w}{y!(z)}$ and $w[ \lpquote y!(z) \rpquote ]$.

\begin{eqnarray}
	\lift{w}{y!(z)}\widehat{\id{\{}u / z \id{\}}}
		& = &
		\lift{w}{y!(u)} \nonumber\\
	w[ \lpquote y!(z) \rpquote ] \widehat{ \id{\{}u / z \id{\}} }
		& = &
		w[ \lpquote y!(z) \rpquote ] \nonumber
\end{eqnarray}

Because the body of the process between quotes is impervious to
substitution, we get radically different answers. In fact, by
examining the first process in an input context,
e.g. $x?(z).\lift{w}{y!(z)}$, we see that the process under the lift
operator may be shaped by prefixed inputs binding a name inside it. In
this sense, the lift operator will be seen as a way to dynamically
construct processes before reifying them as names.

Finally equipped with these standard features we can present the
dynamics of the calculus.

\subsubsection{Operational semantics} 

Finally, we introduce the computational dynamics. What marks these
algebras as distinct from other more traditionally studied algebraic
structures, e.g. vector spaces or polynomial rings, is the manner in
which dynamics is captured. In traditional structures, dynamics is typically
expressed through morphisms between such structures, as in linear maps
between vector spaces or morphisms between rings. In algebras
associated with the semantics of computation, the dynamics is
expressed as part of the algebraic structure itself, through a
reduction reduction relation typically denoted by $\red$. Below, we
give a recursive presentation of this relation for the calculus used
in the encoding.

$\red \subseteq \pi \times \pi$
$\red : \pi \to \mathcal{P}(\pi)$

\begin{mathpar}
  \inferrule* [lab=Comm] { \textsf{match}( x_{src}, x_{trgt} ) } { x_{trgt}?(y)P \; | \; x_{src}!\langle {Q} \rangle \red P\{\quotep{Q}/y}\} }
  \and \\
  \inferrule* [lab=Par] {{P} \red {P}'} {{{P} | {Q}} \red {{P}' | {Q}}}
  \and
  \inferrule* [lab=Equiv]{{{P} \scong {P}'} \andalso {{P}' \red {Q}'} \andalso {{Q}' \scong {Q}}}{{P} \red {Q}}
\end{mathpar}

\begin{eqnarray*}
  match_{\equiv} (\quotep{P},\quotep{Q}) & := & P \equiv Q \\
  match_{\dagger}(\quotep{P},\quotep{Q}) & := & \forall R. P|Q \red^{*} R => R \red^{*} 0 \\
  match_{K}(\quotep{P},\quotep{Q}) & := & K \mbox{ for some context } K
\end{eqnarray*}

$u?(x)P | u!\langle Q \rangle \red P\{\quotep{Q}/x\}$

%We write $\wred$ for $\red^*$, and $P\red$ if $\exists Q $ such that $ P \red Q$.
We write $P\red$ if $\exists Q $ such that $ P \red Q$ and $P\not\red$, otherwise.

\section{Replication}

As mentioned before, it is known that replication (and hence
recursion) can be implemented in a higher-order process algebra
\cite{SangiorgiWalker}. As our first example of calculation with the
machinery thus far presented we give the construction explicitly in
the {\rhoc}.

\begin{eqnarray}
	D_{x} & := & \prefix{x}{y}{(\binpar{\outputp{x}{y}}{@{y}})} \nonumber\\
	\bangp_{x}{P} & := & \binpar{{x}!\langle{\binpar{D_{x}}{P}}\rangle}{D_{x}} \nonumber
\end{eqnarray}

\begin{eqnarray}
	\bangp_{x}{P} & & \nonumber\\
	=
	& {x}!\langle{(\prefix{x}{y}{(\outputp{x}{y} | @{y})) | P}}\rangle 
	      | \prefix{x}{y}{(\outputp{x}{y} | @{y})} & \nonumber\\
	\red
	& (\outputp{x}{y} | @{y})\substn{\quotep{(\prefix{x}{y}{(@{y} | \outputp{x}{y})) | P}}}{y} & \nonumber\\
	=
	& \outputp{x}{\quotep{(\prefix{x}{y}{(\outputp{x}{y} | @{y})) | P}}}
	  | {(\prefix{x}{y}{(\outputp{x}{y} | @{y})) | P}} & \nonumber\\
	\red
	& \ldots & \nonumber\\
	\red^*
	& P | P | \ldots & \nonumber
\end{eqnarray}

Of course, this encoding, as an implementation, runs away, unfolding
$\bangp{P}$ eagerly. A lazier and more implementable replication
operator, restricted to input-guarded processes, may be obtained as follows.

\begin{eqnarray}
\bangp{\prefix{u}{v}{P}} 
	:= 
	\binpar{\lift{x}{\prefix{u}{v}{(\binpar{D(x)}{P})}}}{D(x)} \nonumber
\end{eqnarray}

\begin{remark}
  Note that the lazier definition still does not deal with summation
  or mixed summation (i.e. sums over input and output). The reader is
  invited to construct definitions of replication that deal with these
  features. 

  Further, the definitions are parameterized in a name, $x$. Can you,
  gentle reader, make a definition that eliminates this parameter and
  guarantees no accidental interaction between the replication
  machinery and the process being replicated -- i.e. no accidental
  sharing of names used by the process to get its work done and the
  name(s) used by the replication to effect copying. This latter
  revision of the definition of replication is crucial to obtaining
  the expected identity $!!P \sim !P$.
\end{remark}

\begin{remark}\label{rem:paradoxical_combinator}
  The reader familiar with the lambda calculus will have noticed the
  similarity between $D$ and the paradoxical combinator.

  [Ed. note: the existence of this seems to suggest we have to be more
  restrictive on the set of processes and names we admit if we are to
  support no-cloning.]
\end{remark}

\subsubsection{Bisimulation}

The computational dynamics gives rise to another kind of equivalence,
the equivalence of computational behavior. As previously mentioned
this is typically captured \emph{via} some form of bisimulation.

% The notion we use in this paper is weak barbed bisimulation
% \cite{milner91polyadicpi}.

The notion we use in this paper is derived from weak barbed
bisimulation \cite{milner91polyadicpi}. 

\begin{definition}
An \emph{observation relation}, $\downarrow_{\mathcal N}$, over a set
of names, $\mathcal N$, is the smallest relation satisfying the rules
below.

\infrule[Out-barb]{y \in {\mathcal N}, \; x \nameeq y}
		  {\outputp{x}{v} \downarrow_{\mathcal N} x}
\infrule[Par-barb]{\mbox{$P\downarrow_{\mathcal N} x$ or $Q\downarrow_{\mathcal N} x$}}
		  {\binpar{P}{Q} \downarrow_{\mathcal N} x}

We write $P \Downarrow_{\mathcal N} x$ if there is $Q$ such that 
$P \wred Q$ and $Q \downarrow_{\mathcal N} x$.
\end{definition}

\begin{definition}
%\label{def.bbisim}
An  ${\mathcal N}$-\emph{barbed bisimulation} over a set of names, ${\mathcal N}$, is a symmetric binary relation 
${\mathcal S}_{\mathcal N}$ between agents such that $P\rel{S}_{\mathcal N}Q$ implies:
\begin{enumerate}
\item If $P \red P'$ then $Q \wred Q'$ and $P'\rel{S}_{\mathcal N} Q'$.
\item If $P\downarrow_{\mathcal N} x$, then $Q\Downarrow_{\mathcal N} x$.
\end{enumerate}
$P$ is ${\mathcal N}$-barbed bisimilar to $Q$, written
$P \wbbisim_{\mathcal N} Q$, if $P \rel{S}_{\mathcal N} Q$ for some ${\mathcal N}$-barbed bisimulation ${\mathcal S}_{\mathcal N}$.
\end{definition}

$\mathcal{R} \subseteq \pi \times \pi$

$P \mathcal{R} Q => \forall P'. P \red P' \Rightarrow \exists Q'. Q \red Q', P' \mathcal{R} Q'$

$P \vdash x \Rightarrow Q \vdash x$

\begin{mathpar}
  \inferrule*[lab=Out-barb]{x \nameeq y}{{y}!\langle{Q}\rangle \vdash x}
  \and
  \inferrule*[lab=Par-barb]{\mbox{$P\vdash x$ or $Q\vdash x$}}{\binpar{P}{Q} \vdash x}
\end{mathpar}

\subsubsection{Contexts}

One of the principle advantages of computational calculi like the
$\pi$-calculus is a well-defined notion of context,
contextual-equivalence and a correlation between
contextual-equivalence and notions of bisimulation. The notion of
context allows the decomposition of a process into (sub-)process and
its syntactic environment, its context. Thus, a context may be
thought of as a process with a ``hole'' (written $\Box$) in it. The
application of a context $M$ to a process $P$, written $M[P]$, is
tantamount to filling the hole in $M$ with $P$. In this paper we do
not need the full weight of this theory, but do make use of the notion
of context in the proof the main theorem. 

\begin{mathpar}
  \inferrule* [lab=summation] {} {{M_{M},M_{N}} \bc \Box \;|\; x.M_{A} \;|\; M_{M}+M_{N}}
  \and
  \inferrule* [lab=agent] {} {{M_{A}} \bc (\vec{x})M_{P} \;| \; \clift{P_0,\ldots,M_{P},\ldots,P_N}}
  \and \\
  \inferrule* [lab=process] {} {{M_{P}} \bc M_{N} \;| \;P|M_{P} }
\end{mathpar} 

\begin{mathpar}
  \inferrule* [lab=sychronization] {} {M_{N} \bc \Box \;|\; x?M_{F} \;|\; x!M_{C}}
  \and
  \inferrule* [lab=abstraction] {} {{M_{F}} \bc (x)M_{P} }
  \and
  \inferrule* [lab=concretion] {} {{M_{C}} \bc \langle M_{P} \rangle }
  \and \\
  \inferrule* [lab=process] {} {{M_{P}} \bc M_{N} \;| \;P|M_{P} }
\end{mathpar}

\begin{definition}[contextual application] Given a context $M$, and
  process $P$, we define the \emph{contextual application}, $M[P] :=
  M\{P/\Box\}$. That is, the contextual application of M to P is the
  substitution of $P$ for $\Box$ in $M$.
\end{definition}

$\meaningof{-} : L \to \mathcal{P}(\pi)$

\begin{mathpar}
  \inferrule* [lab=collection] {} {\meaningof{true} = \pi, \and \meaningof{~E} = \pi \setminus \meaningof{E}, \and \meaningof{E_{1} \& E_{2}} = \meaningof{E_{1}} \cap \meaningof{E_{2}}}
\end{mathpar}

\begin{mathpar}
  \inferrule* [lab=structure] {} {\meaningof{0} = \{ P \in \pi | P \equiv 0 \}, \and \\ \meaningof{E_1 | E_2} = \{ P \in \pi | P \equiv P_{1} | P_{2}, P_{1} \in \meaningof{E_{1}}, P_{2} \in \meaningof{E_2}\} }
\end{mathpar}

\begin{mathpar}
 \inferrule* [lab=behavior] {} {\meaningof{\langle a?b \rangle E} = \{ P \in \pi | P \equiv Q | u?(y)P', \\ \and \\\\ \and \\ \;\;\; u \in \meaningof{a}, \forall z.P'\{z/y\} \in \meaningof{E\{z/b\}}\}, \and \\ \meaningof{a!E} = \{ P \in \pi | P \equiv Q | x!\langle P' \rangle, x \in \meaningof{a} P' \in \meaningof{E}\} }
\end{mathpar}

\begin{mathpar}
 \inferrule* [lab=nominal] {} {\meaningof{\quotep{E}} = \{ \quotep{P} \in \quotep{\pi} | P \in \meaningof{E} \}, \and \meaningof{\quotep{P}} = \{ \quotep{Q} \in \quotep{\pi} | P \equiv Q \} \and \\ \meaningof{@\quotep{E}} = \{ P \in \pi | P \equiv @x, x \in \meaningof{E} \}}
\end{mathpar}

\begin{eqnarray*}
  \\
  \meaningof{-} : TS \to ST
\end{eqnarray*}

\begin{eqnarray*}
  \\
  L : TS \to ST
\end{eqnarray*}

\begin{eqnarray*}
  \\
  P \models E \iff P \in \meaningof{E}
\end{eqnarray*}

\begin{eqnarray*}
  P \approx_{L} Q \iff \forall E \in L. P \models E \iff Q \models E
\end{eqnarray*}

\begin{eqnarray*}
  P \approx_{K} Q
\end{eqnarray*}

\begin{eqnarray*}
  P \approx Q
\end{eqnarray*}

$\approx_{K} = \approx = \approx_{L}$

\subsubsection{Contextual duality}

Note that contexts extend the quotation operation to a family of
operations from processes to names. Given a context, $M$, we can
define a \emph{nominal context}, $\quotep{M}$ by $\quotep{M}[P] :=
\quotep{M[P]}$. To foreshadow what is to come we observe that these
operations enjoy a duality with processes very much like the duality
between vectors and maps from vectors to scalars.

Further, because the calculus is essentially higher-order, we have a
correspondence between contexts and processes. More specifically,
given a name $x$ and a context $M$ we can construct $M^{*}_{x}$ such
that 

\begin{mathpar}
  M^{*}_{x} | \lift{x}{P} \red M[P]
\end{mathpar}

namely,

\begin{mathpar}
  M^{*}_{x} := x?(u).M[\dropn{u}]
\end{mathpar}

The dependence of $M^{*}_{x}$ on a name makes it an abstraction, 

\begin{mathpar}
  M^{*} := (x)x?(u).M[\dropn{u}]
\end{mathpar}

\subsection{Additional notation}

It will sometimes be convenient to denote the process a name
quotes. We already have the notation $x = \quotep{P}$, but it will be
convenient to introduce an alternate notation, $\procn{x}$, when we
want to emphasize the connection to the use of the name. Note that, by
virtue of name equivalence, $\quotep{\procn{x}} \nameeq x$; so, the
notation is consistent with previous definitions.

Further, because names have structure it is possible to effect
substitutions on the basis of that structure. This means we need to
upgrade our notation for substitutions, which we accomplish by
adapting comprehension notation. Thus,

\begin{mathpar}
  P\{ y / x : x \in S \}
\end{mathpar}

is interpreted to mean the process derived from P by replacing (in a
capture-avoiding manner) each occurrence of $x$ in $S$ by $y$. For example,

\begin{mathpar}
  P\{ \quotep{\procn{x}|\procn{x}} / x : x \in \freenames{P} \}
\end{mathpar}

will replace each (occurrence) of a free name $x$ in $P$ by
$\quotep{\procn{x}|\procn{x}}$.

Also, we will avail ourselves of the notation $x^{L}$ and $x^{R}$ to
denote injections of a name into disjoint copies of the name
space. There are numerous ways to accomplish this. One example can be
found in \cite{MeredithR05}. This notation overloads to vectors of
names: $\vec{x}^{\pi} := (x_{i}^{\pi} \; : \; 0 \leq i < |\vec{x}| )$ where $\pi \in \{L,R\}$.

We also use $P^{\Box} := P|\Box$.

In \cite{MeredithR05} an interpretation of the new operator is
given. It turns out that there are several possible interpretations
all enjoying the requisite algebraic properties of the operator (see
\cite{milner91polyadicpi}). We will therefore make liberal use of
$(\nu\; \vec{x})P$.

% subsection the_syntax_and_semantics_of_the_notation_system (end)   

\input{qm2pi.qmops} 

\input{qm2pi.sterngerlach} 

\input{qm2pi.metric} 

% section concurrent_process_calculi (end)

%\input{qm2pi.proofsketch}

% section proof sketch (end)

%\input{qm2pi.slviaknots} 

% section spatial logic via knots (end)

\input{qm2pi.conclusion}

% section conclusion (end)

%\input{qm2pi.dtcodes} 

% section wiring algorithm (end)

\input{qm2pi.ack} 

% section acknowledgments (end)

\newpage


\bibliographystyle{plain}   
\bibliography{../../biblios/main.bib}

\input{qm2pi.rhodetails}

\end{document}

 

% section concurrent_process_calculi (end)

%\documentclass[12pt]{llncs}
%\documentclass{jktr}

\usepackage[pdftex]{hyperref}                   
\usepackage {listings}
\usepackage {mathpartir}
\usepackage{bcprules}
%\usepackage{listings}
                       
\usepackage{graphicx} 
%\usepackage[margins=2.5cm,nohead,nofoot]{geometry}
%\usepackage{geometry}
\usepackage{amsfonts}
\usepackage{amstext}
\usepackage{latexsym}
\usepackage{amssymb}
\usepackage{color}


%\include{myPreamble}
\include{qm2pi.local} 

%\ifpdf
%\usepackage[pdftex]{graphicx}
%\else
%\usepackage{graphicx}
%\fi

 % \ifpdf
%  \usepackage{pdfsync}
%  \if


%\title{Brief Article}
%\author{David F. Snyder}
%\author{L.G. Meredith}

%\address{Dept. of Math., Texas State University--San Marcos, San Marcos, TX 78666}
       
\pagestyle{empty}


\begin{document}

\lstset{language=[Objective]Caml,frame=shadowbox}

\input{qm2pi.front}

% section front matter (end)

\input{qm2pi.intro} 
 
% section introduction (end)

% \input{qm2pi.knotations} 

% section notation (end)

\input{qm2pi.process.calculi} 

% section concurrent_process_calculi_and_spatial_logics_ (end)
    
%\input{qm2pi.knots2pi} 

%\input{qm2pi.trefoil} 

%\input{qm2pi.mainthm} 

% subsection basic_interpretation (end)

%\input{qm2pi.rho.presentation} 
\subsection{The syntax and semantics of the notation system}\label{sub:the_syntax_and_semantics_of_the_notation_system} % (fold)

We now summarize a technical presentation of the calculus that
embodies our theory of dynamics. The typical presentation of such a
calculus follows the style of giving generators and relations on
them. The grammar, below, describing term constructors, freely
generates the set of processes, $\Proc$. This set is then quotiented
by a relation known as structural congruence and it is over this set
that the notion of dynamics is expressed. This presentation is
essentially that of \cite{MeredithR05} with the addition of
polyadicity and summation. For readability we have relegated some of
the technical subtleties to an appendix.

\subsubsection{Process grammar}\label{subsub:process_grammar}

\begin{mathpar}
  \inferrule* [lab=synchronization] {} {{M} \bc \pzero \;|\; x?F \;|\; x!C }
  \and
  \inferrule* [lab=abstraction] {} {{F} \bc (x)P}
  \and
  \inferrule* [lab=concretion] {} {{C} \bc \langle Q \rangle}
  \and
  \inferrule* [lab=process] {} {{P,Q} \bc M \;| \;P|Q \;|\; @{x}}
  \and
  \inferrule* [lab=name] {} {{x} \bc \quotep{P}}
\end{mathpar} 

Note that $\vec{x}$ (resp. $\vec{P}$) denotes a vector of names
(resp. processes) of length $|\vec{x}|$ (resp. $|\vec{P}|$). We adopt
the following useful abbreviations.

\begin{mathpar}
   x?(\vec{y}).P := x.(\vec{y})P \and  x\clift{\vec{P}} := x.\clift{\vec{P}}
   \and x!(y) := \lift{x}{\dropn{y}}
   \and \Pi_{i=0}^{n-1}P_i := P_0 | \ldots | P_{n-1}
\end{mathpar}

\subsubsection{Structural congruence}

\paragraph{Free and bound names and alpha-equivalence.} At the
core of structural equivalence is alpha-equivalence which identifies
process that are the same up to a change of variable. Formally, we
recognize the distinction between free and bound names. The free names
of a process, $\freenames{P}$, may be calculated recursively as
follows:

\begin{mathpar}
\freenames{\pzero} := \emptyset
  \and \\
  \freenames{x?(y).P} := \{ x \} \cup (\freenames{P} \setminus \{ y \})
  \and 
  \freenames{x!\langle P \rangle} := \{ x \} \cup \{ P \} 
  \and \\
  \freenames{P|Q} := \freenames{P} \cup \freenames{Q}
  \and \\
  \freenames{@{x}} := \{ x \}
\end{mathpar}

$\pi$
$\quotep{\pi}$

$\freenames{-} : \pi \to \mathcal{P}(\quotep{\pi})$

\begin{eqnarray*}
  \freenames{\pzero} & := & \emptyset \\
  \freenames{x?(y).P} & := & \{ x \} \cup (\freenames{P} \setminus \{ y \}) \\
  \freenames{x!\langle P \rangle} & := & \{ x \} \cup \{ P \} \\
  \freenames{P|Q} & := & \freenames{P} \cup \freenames{Q} \\
  \freenames{\dropn{x}} & := & \{ x \}
\end{eqnarray*}

The bound names of a process, $\boundnames{P}$, are those names occurring in $P$
that are not free. For example, in $x?(y).0$, the name $x$ is free, while $y$ is bound.

\begin{mathpar}
  \inferrule* [lab=monoidal-laws] {} { P|Q \equiv Q|P \and P|0 \equiv P \and P|(Q|R) \equiv (P|Q)|R }
\end{mathpar}

\begin{mathpar}
  \inferrule* [lab=alpha-equivalence] {} { (x)P \equiv (y)P\{y/x\} \and y \not\in \freenames{P} }
\end{mathpar}

\begin{definition}
Then two processes, $P,Q$, are alpha-equivalent if $P = Q\{\vec{y}/\vec{x}\}$ for
some $\vec{x} \in \boundnames{Q},\vec{y} \in \boundnames{P}$, where $Q\{\vec{y}/\vec{x}\}$
denotes the capture-avoiding substitution of $\vec{y}$ for $\vec{x}$ in $Q$.
\end{definition}

\begin{definition}
  The {\em structural congruence} \cite{SangiorgiWalker} , $\equiv$,
  between processes is the least congruence containing
  alpha-equivalence, satisfying the abelian monoid laws
  (associativity, commutativity and $\pzero$ as identity) for parallel
  composition $|$ and for summation $+$.
\end{definition}

\subsection{Name equivalence}

We take name equivalence, written $\nameeq$, to be the smallest
equivalence relation generated by the following rules.

\begin{mathpar}
\inferrule*[lab=Quote-drop]
{ }
{ \quotep{@{x}} \nameeq x }

\inferrule*[lab=Struct-equiv]
{ P \scong Q }
{ \quotep{P} \nameeq \quotep{Q} }
\end{mathpar}

The astute reader will have noticed that the mutual recursion of names
and processes imposes a mutual recursion on alpha-equivalence and
structural equivalence via name-equivalence. Fortunately, all of this
works out pleasantly and we may calculate in the natural way, free of
concern. The reader interested in the details is referred to the
appendix \ref{appendix:rho_details}.

\subsection{Substitution}

We use $\Proc$ for the set of processes, $\QProc$ for the set of
names, and $\id{\{}\vec{y} / \vec{x} \id{\}}$ to denote partial maps,
$s : \QProc \rightarrow \QProc$. A map, $s$ lifts, uniquely, to a map
on process terms, $\widehat{s} : \Proc \rightarrow \Proc$ by the
following equations.

\begin{mathpar}
  (0) \psubstp{Q}{P} := 0 \\
  (R \juxtap S) \psubstp{Q}{P}
  :=    
  (R)\psubstp{Q}{P} \juxtap (S) \psubstp{Q}{P} \\
  (x?(y).R) \psubstp{Q}{P}    
  :=    
  (x)\substp{Q}{P} (z)\concat( (R \psubstn{z}{y}) \psubstp{Q}{P} ) \\
  (\lift{x}{R}) \psubstp{Q}{P}  
  :=
  \lift{(x)\substp{Q}{P}}{ R \psubstp{Q}{P} } \\
%   (\dropn{x})  \psubstp{Q}{P}       
%   := 
%   \left\{ 
%     \begin{array}{ccc} 
%       \dropn{\quotep{Q}} & & x \nameeq \quotep{P} \\
%       \dropn{x} & & otherwise \\
%     \end{array}
%   \right. 
  (\dropn{x})  \psubstp{Q}{P}       
  := 
  \left\{ 
    \begin{array}{ccc} 
      Q & & x \nameeq \quotep{P} \\
      \dropn{x} & & otherwise \\
    \end{array}
  \right.
\end{mathpar}
 

where

\begin{eqnarray}
  (x)\id{\{} \lpquote Q \rpquote / \lpquote P \rpquote \id{\}}            = 
  \left\{ 
    \begin{array}{ccc}
      \lpquote Q \rpquote & & x \nameeq \lpquote P \rpquote \\
      x & & otherwise \\
    \end{array}
  \right. \nonumber
\end{eqnarray}

and $z$ is chosen distinct from $\quotep{P}$, $\quotep{Q}$, the free
names in $Q$, and all the names in $R$. Our $\alpha$-equivalence will
be built in the standard way from this substitution.

\begin{remark}\label{rem:no_self_referential_names}
  One consequence of these definitions is that $\forall P. \quotep{P}
  \not\in \freenames{P}$.
\end{remark}

\subsection{ Dynamic quote: an example }

Anticipating something of what's to come, consider applying the
substitution, $\widehat{\id{\{}u / z \id{\}}}$, to the following pair
of processes, $\lift{w}{y!(z)}$ and $w[ \lpquote y!(z) \rpquote ]$.

\begin{eqnarray}
	\lift{w}{y!(z)}\widehat{\id{\{}u / z \id{\}}}
		& = &
		\lift{w}{y!(u)} \nonumber\\
	w[ \lpquote y!(z) \rpquote ] \widehat{ \id{\{}u / z \id{\}} }
		& = &
		w[ \lpquote y!(z) \rpquote ] \nonumber
\end{eqnarray}

Because the body of the process between quotes is impervious to
substitution, we get radically different answers. In fact, by
examining the first process in an input context,
e.g. $x?(z).\lift{w}{y!(z)}$, we see that the process under the lift
operator may be shaped by prefixed inputs binding a name inside it. In
this sense, the lift operator will be seen as a way to dynamically
construct processes before reifying them as names.

Finally equipped with these standard features we can present the
dynamics of the calculus.

\subsubsection{Operational semantics} 

Finally, we introduce the computational dynamics. What marks these
algebras as distinct from other more traditionally studied algebraic
structures, e.g. vector spaces or polynomial rings, is the manner in
which dynamics is captured. In traditional structures, dynamics is typically
expressed through morphisms between such structures, as in linear maps
between vector spaces or morphisms between rings. In algebras
associated with the semantics of computation, the dynamics is
expressed as part of the algebraic structure itself, through a
reduction reduction relation typically denoted by $\red$. Below, we
give a recursive presentation of this relation for the calculus used
in the encoding.

$\red \subseteq \pi \times \pi$
$\red : \pi \to \mathcal{P}(\pi)$

\begin{mathpar}
  \inferrule* [lab=Comm] { \textsf{match}( x_{src}, x_{trgt} ) } { x_{trgt}?(y)P \; | \; x_{src}!\langle {Q} \rangle \red P\{\quotep{Q}/y}\} }
  \and \\
  \inferrule* [lab=Par] {{P} \red {P}'} {{{P} | {Q}} \red {{P}' | {Q}}}
  \and
  \inferrule* [lab=Equiv]{{{P} \scong {P}'} \andalso {{P}' \red {Q}'} \andalso {{Q}' \scong {Q}}}{{P} \red {Q}}
\end{mathpar}

\begin{eqnarray*}
  match_{\equiv} (\quotep{P},\quotep{Q}) & := & P \equiv Q \\
  match_{\dagger}(\quotep{P},\quotep{Q}) & := & \forall R. P|Q \red^{*} R => R \red^{*} 0 \\
  match_{K}(\quotep{P},\quotep{Q}) & := & K \mbox{ for some context } K
\end{eqnarray*}

$u?(x)P | u!\langle Q \rangle \red P\{\quotep{Q}/x\}$

%We write $\wred$ for $\red^*$, and $P\red$ if $\exists Q $ such that $ P \red Q$.
We write $P\red$ if $\exists Q $ such that $ P \red Q$ and $P\not\red$, otherwise.

\section{Replication}

As mentioned before, it is known that replication (and hence
recursion) can be implemented in a higher-order process algebra
\cite{SangiorgiWalker}. As our first example of calculation with the
machinery thus far presented we give the construction explicitly in
the {\rhoc}.

\begin{eqnarray}
	D_{x} & := & \prefix{x}{y}{(\binpar{\outputp{x}{y}}{@{y}})} \nonumber\\
	\bangp_{x}{P} & := & \binpar{{x}!\langle{\binpar{D_{x}}{P}}\rangle}{D_{x}} \nonumber
\end{eqnarray}

\begin{eqnarray}
	\bangp_{x}{P} & & \nonumber\\
	=
	& {x}!\langle{(\prefix{x}{y}{(\outputp{x}{y} | @{y})) | P}}\rangle 
	      | \prefix{x}{y}{(\outputp{x}{y} | @{y})} & \nonumber\\
	\red
	& (\outputp{x}{y} | @{y})\substn{\quotep{(\prefix{x}{y}{(@{y} | \outputp{x}{y})) | P}}}{y} & \nonumber\\
	=
	& \outputp{x}{\quotep{(\prefix{x}{y}{(\outputp{x}{y} | @{y})) | P}}}
	  | {(\prefix{x}{y}{(\outputp{x}{y} | @{y})) | P}} & \nonumber\\
	\red
	& \ldots & \nonumber\\
	\red^*
	& P | P | \ldots & \nonumber
\end{eqnarray}

Of course, this encoding, as an implementation, runs away, unfolding
$\bangp{P}$ eagerly. A lazier and more implementable replication
operator, restricted to input-guarded processes, may be obtained as follows.

\begin{eqnarray}
\bangp{\prefix{u}{v}{P}} 
	:= 
	\binpar{\lift{x}{\prefix{u}{v}{(\binpar{D(x)}{P})}}}{D(x)} \nonumber
\end{eqnarray}

\begin{remark}
  Note that the lazier definition still does not deal with summation
  or mixed summation (i.e. sums over input and output). The reader is
  invited to construct definitions of replication that deal with these
  features. 

  Further, the definitions are parameterized in a name, $x$. Can you,
  gentle reader, make a definition that eliminates this parameter and
  guarantees no accidental interaction between the replication
  machinery and the process being replicated -- i.e. no accidental
  sharing of names used by the process to get its work done and the
  name(s) used by the replication to effect copying. This latter
  revision of the definition of replication is crucial to obtaining
  the expected identity $!!P \sim !P$.
\end{remark}

\begin{remark}\label{rem:paradoxical_combinator}
  The reader familiar with the lambda calculus will have noticed the
  similarity between $D$ and the paradoxical combinator.

  [Ed. note: the existence of this seems to suggest we have to be more
  restrictive on the set of processes and names we admit if we are to
  support no-cloning.]
\end{remark}

\subsubsection{Bisimulation}

The computational dynamics gives rise to another kind of equivalence,
the equivalence of computational behavior. As previously mentioned
this is typically captured \emph{via} some form of bisimulation.

% The notion we use in this paper is weak barbed bisimulation
% \cite{milner91polyadicpi}.

The notion we use in this paper is derived from weak barbed
bisimulation \cite{milner91polyadicpi}. 

\begin{definition}
An \emph{observation relation}, $\downarrow_{\mathcal N}$, over a set
of names, $\mathcal N$, is the smallest relation satisfying the rules
below.

\infrule[Out-barb]{y \in {\mathcal N}, \; x \nameeq y}
		  {\outputp{x}{v} \downarrow_{\mathcal N} x}
\infrule[Par-barb]{\mbox{$P\downarrow_{\mathcal N} x$ or $Q\downarrow_{\mathcal N} x$}}
		  {\binpar{P}{Q} \downarrow_{\mathcal N} x}

We write $P \Downarrow_{\mathcal N} x$ if there is $Q$ such that 
$P \wred Q$ and $Q \downarrow_{\mathcal N} x$.
\end{definition}

\begin{definition}
%\label{def.bbisim}
An  ${\mathcal N}$-\emph{barbed bisimulation} over a set of names, ${\mathcal N}$, is a symmetric binary relation 
${\mathcal S}_{\mathcal N}$ between agents such that $P\rel{S}_{\mathcal N}Q$ implies:
\begin{enumerate}
\item If $P \red P'$ then $Q \wred Q'$ and $P'\rel{S}_{\mathcal N} Q'$.
\item If $P\downarrow_{\mathcal N} x$, then $Q\Downarrow_{\mathcal N} x$.
\end{enumerate}
$P$ is ${\mathcal N}$-barbed bisimilar to $Q$, written
$P \wbbisim_{\mathcal N} Q$, if $P \rel{S}_{\mathcal N} Q$ for some ${\mathcal N}$-barbed bisimulation ${\mathcal S}_{\mathcal N}$.
\end{definition}

$\mathcal{R} \subseteq \pi \times \pi$

$P \mathcal{R} Q => \forall P'. P \red P' \Rightarrow \exists Q'. Q \red Q', P' \mathcal{R} Q'$

$P \vdash x \Rightarrow Q \vdash x$

\begin{mathpar}
  \inferrule*[lab=Out-barb]{x \nameeq y}{{y}!\langle{Q}\rangle \vdash x}
  \and
  \inferrule*[lab=Par-barb]{\mbox{$P\vdash x$ or $Q\vdash x$}}{\binpar{P}{Q} \vdash x}
\end{mathpar}

\subsubsection{Contexts}

One of the principle advantages of computational calculi like the
$\pi$-calculus is a well-defined notion of context,
contextual-equivalence and a correlation between
contextual-equivalence and notions of bisimulation. The notion of
context allows the decomposition of a process into (sub-)process and
its syntactic environment, its context. Thus, a context may be
thought of as a process with a ``hole'' (written $\Box$) in it. The
application of a context $M$ to a process $P$, written $M[P]$, is
tantamount to filling the hole in $M$ with $P$. In this paper we do
not need the full weight of this theory, but do make use of the notion
of context in the proof the main theorem. 

\begin{mathpar}
  \inferrule* [lab=summation] {} {{M_{M},M_{N}} \bc \Box \;|\; x.M_{A} \;|\; M_{M}+M_{N}}
  \and
  \inferrule* [lab=agent] {} {{M_{A}} \bc (\vec{x})M_{P} \;| \; \clift{P_0,\ldots,M_{P},\ldots,P_N}}
  \and \\
  \inferrule* [lab=process] {} {{M_{P}} \bc M_{N} \;| \;P|M_{P} }
\end{mathpar} 

\begin{mathpar}
  \inferrule* [lab=sychronization] {} {M_{N} \bc \Box \;|\; x?M_{F} \;|\; x!M_{C}}
  \and
  \inferrule* [lab=abstraction] {} {{M_{F}} \bc (x)M_{P} }
  \and
  \inferrule* [lab=concretion] {} {{M_{C}} \bc \langle M_{P} \rangle }
  \and \\
  \inferrule* [lab=process] {} {{M_{P}} \bc M_{N} \;| \;P|M_{P} }
\end{mathpar}

\begin{definition}[contextual application] Given a context $M$, and
  process $P$, we define the \emph{contextual application}, $M[P] :=
  M\{P/\Box\}$. That is, the contextual application of M to P is the
  substitution of $P$ for $\Box$ in $M$.
\end{definition}

$\meaningof{-} : L \to \mathcal{P}(\pi)$

\begin{mathpar}
  \inferrule* [lab=collection] {} {\meaningof{true} = \pi, \and \meaningof{~E} = \pi \setminus \meaningof{E}, \and \meaningof{E_{1} \& E_{2}} = \meaningof{E_{1}} \cap \meaningof{E_{2}}}
\end{mathpar}

\begin{mathpar}
  \inferrule* [lab=structure] {} {\meaningof{0} = \{ P \in \pi | P \equiv 0 \}, \and \\ \meaningof{E_1 | E_2} = \{ P \in \pi | P \equiv P_{1} | P_{2}, P_{1} \in \meaningof{E_{1}}, P_{2} \in \meaningof{E_2}\} }
\end{mathpar}

\begin{mathpar}
 \inferrule* [lab=behavior] {} {\meaningof{\langle a?b \rangle E} = \{ P \in \pi | P \equiv Q | u?(y)P', \\ \and \\\\ \and \\ \;\;\; u \in \meaningof{a}, \forall z.P'\{z/y\} \in \meaningof{E\{z/b\}}\}, \and \\ \meaningof{a!E} = \{ P \in \pi | P \equiv Q | x!\langle P' \rangle, x \in \meaningof{a} P' \in \meaningof{E}\} }
\end{mathpar}

\begin{mathpar}
 \inferrule* [lab=nominal] {} {\meaningof{\quotep{E}} = \{ \quotep{P} \in \quotep{\pi} | P \in \meaningof{E} \}, \and \meaningof{\quotep{P}} = \{ \quotep{Q} \in \quotep{\pi} | P \equiv Q \} \and \\ \meaningof{@\quotep{E}} = \{ P \in \pi | P \equiv @x, x \in \meaningof{E} \}}
\end{mathpar}

\begin{eqnarray*}
  \\
  \meaningof{-} : TS \to ST
\end{eqnarray*}

\begin{eqnarray*}
  \\
  L : TS \to ST
\end{eqnarray*}

\begin{eqnarray*}
  \\
  P \models E \iff P \in \meaningof{E}
\end{eqnarray*}

\begin{eqnarray*}
  P \approx_{L} Q \iff \forall E \in L. P \models E \iff Q \models E
\end{eqnarray*}

\begin{eqnarray*}
  P \approx_{K} Q
\end{eqnarray*}

\begin{eqnarray*}
  P \approx Q
\end{eqnarray*}

$\approx_{K} = \approx = \approx_{L}$

\subsubsection{Contextual duality}

Note that contexts extend the quotation operation to a family of
operations from processes to names. Given a context, $M$, we can
define a \emph{nominal context}, $\quotep{M}$ by $\quotep{M}[P] :=
\quotep{M[P]}$. To foreshadow what is to come we observe that these
operations enjoy a duality with processes very much like the duality
between vectors and maps from vectors to scalars.

Further, because the calculus is essentially higher-order, we have a
correspondence between contexts and processes. More specifically,
given a name $x$ and a context $M$ we can construct $M^{*}_{x}$ such
that 

\begin{mathpar}
  M^{*}_{x} | \lift{x}{P} \red M[P]
\end{mathpar}

namely,

\begin{mathpar}
  M^{*}_{x} := x?(u).M[\dropn{u}]
\end{mathpar}

The dependence of $M^{*}_{x}$ on a name makes it an abstraction, 

\begin{mathpar}
  M^{*} := (x)x?(u).M[\dropn{u}]
\end{mathpar}

\subsection{Additional notation}

It will sometimes be convenient to denote the process a name
quotes. We already have the notation $x = \quotep{P}$, but it will be
convenient to introduce an alternate notation, $\procn{x}$, when we
want to emphasize the connection to the use of the name. Note that, by
virtue of name equivalence, $\quotep{\procn{x}} \nameeq x$; so, the
notation is consistent with previous definitions.

Further, because names have structure it is possible to effect
substitutions on the basis of that structure. This means we need to
upgrade our notation for substitutions, which we accomplish by
adapting comprehension notation. Thus,

\begin{mathpar}
  P\{ y / x : x \in S \}
\end{mathpar}

is interpreted to mean the process derived from P by replacing (in a
capture-avoiding manner) each occurrence of $x$ in $S$ by $y$. For example,

\begin{mathpar}
  P\{ \quotep{\procn{x}|\procn{x}} / x : x \in \freenames{P} \}
\end{mathpar}

will replace each (occurrence) of a free name $x$ in $P$ by
$\quotep{\procn{x}|\procn{x}}$.

Also, we will avail ourselves of the notation $x^{L}$ and $x^{R}$ to
denote injections of a name into disjoint copies of the name
space. There are numerous ways to accomplish this. One example can be
found in \cite{MeredithR05}. This notation overloads to vectors of
names: $\vec{x}^{\pi} := (x_{i}^{\pi} \; : \; 0 \leq i < |\vec{x}| )$ where $\pi \in \{L,R\}$.

We also use $P^{\Box} := P|\Box$.

In \cite{MeredithR05} an interpretation of the new operator is
given. It turns out that there are several possible interpretations
all enjoying the requisite algebraic properties of the operator (see
\cite{milner91polyadicpi}). We will therefore make liberal use of
$(\nu\; \vec{x})P$.

% subsection the_syntax_and_semantics_of_the_notation_system (end)   

\input{qm2pi.qmops} 

\input{qm2pi.sterngerlach} 

\input{qm2pi.metric} 

% section concurrent_process_calculi (end)

%\input{qm2pi.proofsketch}

% section proof sketch (end)

%\input{qm2pi.slviaknots} 

% section spatial logic via knots (end)

\input{qm2pi.conclusion}

% section conclusion (end)

%\input{qm2pi.dtcodes} 

% section wiring algorithm (end)

\input{qm2pi.ack} 

% section acknowledgments (end)

\newpage


\bibliographystyle{plain}   
\bibliography{../../biblios/main.bib}

\input{qm2pi.rhodetails}

\end{document}



% section proof sketch (end)

%\section{Unlikely characters: spatial logic for
  knots}\label{sub:characteristic_formulae} % (fold)

Associated to the mobile process calculi are a family of logics known
as the Hennessy-Milner logics. These logics typically enjoy a
semantics interpreting formulae as sets of processes that when
factored through the encoding outlined above allows an identification
of classes of knots with logical formulae. In the context of this
encoding the sub-family known as the spatial logics \cite{CairesC03}
\cite{CairesC04} \cite{Caires04} are of particular interest providing
several important features for expressing and reasoning about
properties (i.e. classes) of knots. We hint here at how this may be done.

%\begin{description}
%\item [structural connectives] 
\subsubsection{Structural connectives} The spatial logics enjoy
structural connectives corresponding, at the logical level, to the
parallel composition ($P | Q$) and new name ($(\nu \; x)P$)
connectives for processes. As illustrated in the examples below, these
connectives are extremely expressive given the shape of our encoding.
%\item [decideable satisfaction]

\subsubsection{Decideable satisfaction}
In \cite{Caires04} the satisfaction relation is shown to be decideable
for a rich class of processes. It further turns out that the image of
the our encoding is a proper subset of that class. This result
provides the basis for an algorithm by which to search for knots
enjoying a given property.
%\item [characteristic formulae]

\subsubsection{Characteristic formulae}
In the same paper \cite{Caires04} , Caires presents a means of calculating
characteristic formulae, selecting equivalence classes of processes
up to a pre--specified depth limit on the support set of names. Composed with our
encoding, this characteristic formula can be used to select
characteristic formulae for knots.
%\end{description}

\subsubsection{Spatial logic formulae}

The grammar below (segmented for comprehension) summarizes the syntax
of spatial logic formulae. We employ illustrative examples in the
sequel to provide an intuitive understanding of their meaning
referring the reader to \cite{Caires04} for a more detailed explication
of the semantics.

\begin{mathpar}
  \inferrule* [lab=boolean] {} {{A,B} \bc T \;|\; \neg A \;|\; A \wedge B \;|\; \eta = \eta'}
  \and
  \inferrule* [lab=spatial] {} {|\; \pzero \;|\; A | B \;|\; x \text{\textregistered} A \;|\; \forall x . A \;|\;  H x . A}
  \and
  \inferrule* [lab=behavioral] {} {|\; \alpha . A}
  \and 
  \inferrule* [lab=recursion] {} {|\; X(\vec{u}) \;|\; \mu X(\vec{u}) . A}
  \and
  \inferrule* [lab=action] {} {\alpha \bc \langle x?(\vec{y}) \rangle \;|\; \langle x!(\vec{y}) \rangle \;|\; \langle \tau \rangle}
  \and 
  \inferrule* [lab=name] {} {\eta \bc x \;|\; \tau}
\end{mathpar} 

% subsection characteristic_formulae (end)   	 

\subsection{Example formulae}\label{sub:example_formulae_} % (fold)

\subsubsection{Crossing as formula.}
% 
% \begin{align*}
%   \frac{d}{dx} \sin x &= \cos x 
%   & \frac{d}{dx} e^x &= e^x \\
%   \frac{d}{dx} \cos x &= - \sin x 
%   & \frac{d}{dx} \log x &= \frac{1}{x} \\
% \end{align*} 

\begin{align*}
 \mu C(x_{0},x_{1},y_{0},y_{1},u).&(\langle x_{0}?(z) \rangle(\langle u! \rangle\langle y_{1}!z \rangle C(x_{0},x_{1},y_{0},y_{1},u)) & \\
  & \wedge \langle y_{1}?(z) \rangle (\langle u! \rangle \langle x_{0}!z \rangle C(x_{0},x_{1},y_{0},y_{1},u)) & \\
  & \wedge \langle x_{1}?(z) \rangle (\langle u? \rangle \langle y_{0}!z \rangle C(x_{0},x_{1},y_{0},y_{1},u)) & \\
  & \wedge \langle y_{0}?(z) \rangle (\langle u? \rangle \langle x_{1}!z \rangle C(x_{0},x_{1},y_{0},y_{1},u))) &
\end{align*}

The lexicographical similarity between the shape of this formulae and
the shape of definition of the process representing a crossing reveals
the intuitive meaning of this formulae. It describes the capabilities
of a process that has the right to represent a crossing. For example
it picks out processes that may perform an input on the port $x_0$ in
its initial menu of capabilities. What differentiates the formula
from the process, however, is that the crossing process is the
smallest candidate to satisfy the formula. Infinitely many other
processes -- with internal behavior hidden behind this interface, so
to speak -- also satisfy this formula. Even this simple formula,
then, can be seen to open a new view onto knots, providing a
computational interpretation of \emph{virtual} knots.

Note that this formula is derived by hand. A similar formula can be
derived by employing Caires' calculation of characteristic formula
\cite{Caires04} to the process representing a crossing. In light of
this discussion, we let
$\meaningof{C}_{\phi}(x0,x1,y0,y1,u)$ denote a formula specifying the
dynamics we wish to capture of a crossing. To guarantee we preserve
the shape of the interface and minimal semantics we demand that
$\meaningof{C}_{\phi}(x0,x1,y0,y1,u) \Rightarrow
\textbf{C}(x0,x1,y0,y1,u)$ where $\textbf{C}(x0,x1,y0,y1,u)$ denotes
the formula above.
                            
\subsubsection{Crossing number constraints.}
The moral content of the context lemma (Lemma \ref{context}) is that the notion of
``locality'' in the Reidemeister moves is effectively captured by the
parallel composition operator of the process calculus. This intuition
extends through the logic. Given a formula,
$\meaningof{C}_{\phi}(x0,x1,y0,y1,u)$, we can use the structural
connectives to specify constraints on crossing numbers, such as at
least $n$ crossings, or exactly $n$ crossings.
\begin{mathpar}
  \inferrule* [lab=at-least-n] {} { K^{\geq n}_{\phi}(\vec{xs},\vec{ys}) := \Pi_{i=0}^{n-1} Hu . \meaningof{C}_{\phi}(xs_i,ys_i,u) | T }
  \and 
  \inferrule* [lab=exactly-n] {} { K^{= n}_{\phi}(\vec{xs},\vec{ys}) := \Pi_{i=0}^{n-1} Hu . \meaningof{C}_{\phi}(xs_i,ys_i,u) | \neg (\forall x_0,y_0,x_1,y_1,u . \meaningof{C}_{\phi}(x_0,y_0,x_1,y_1,u) | T) }
\end{mathpar}

To round out this section, recall that the encoding of an $n$-crossing
knot decomposes into a parallel composition of $n$ \emph{copies} of a
crossing process together with a wiring harness. To specify different
knot classes with the same crossing number amounts to specifying
logical constraints on the wiring harness. In the interest of space,
we defer examples to a forthcoming paper. Suffice it to say that both
the conditions ``alternating knot'' and ``contains the tangle
corresponding to 5/3'' are expressible. For example, it is possible to
calculate the characteristic formula of a process corresponding to the
tangle 5/3 and conjoin it into the classifying formula via the
composition connective of the logic.

Finally, we wish to observe that it is entirely within reason to
contemplate a more domain-specific version of spatial logic tailored
to the shape of processes in the image of the encoding. Such a
domain-specific logic would have a better claim to the title formal
language of knot properties.

% subsection example_formulae_ (end)

% section knots_as_processes (end) 

% section spatial logic via knots (end)

\section{Conclusions and future work}

\paragraph{Testing physical space}
You, gentle reader, may wonder why of all the theorems to be proved
given this set up we pick the one above. In some sense it's hardly
central to quantum mechanics. We see it as central in the sense that
it firmly establishes a notion of physical space arising from a notion
of the equivalence of behavior. Relating bisimulation to a metric is a
big step forward, but one is faced with interpreting the relationship
of that metric space to something more physical. Quantum mechanical
notions of ``physical'' space are still far from intuitive, but by
relating this idea of distance as testing to calculations that predict
physical circumstances we are making a not insignificant step forward
toward an understanding of the physical space we inhabit as
essentially dynamic.

\paragraph{Effectivity and simulation}
One of the observations we have yet to make is that the entire program
spelled out here is effective. We have built various interpreters for
the reflective calculus at work in this interpretation. In principle,
then, we can simulate quantum mechanics on a computer. The place where
the simulation may lose fidelity is the infinitely branching summation
for the annihilator.

In this connection i also want to point out that the evaluation style
calculation of the inner product puts the non-determinism of the
summation right at the heart of measurement. This suggests that
Milner's original reduction-based formulation of the dynamics of his
calculi in terms of sums was not just notationally suggestive of a
notion of measure-and-continue but captured some significant part of
the physics.

\paragraph{Quantum continuations}
In light of this last observation i want to point out that the
predominant account of quantum mechanics is missing a key aspect of a
truly compositional story of the physical situation. In a real lab,
when a measurement is made the observation can be made to feed into
another device that then makes another measurement conditioned on the
results of the first. This means that after the superposition was
collapsed the entire experimental set up remained in
superposition. While QM offers a means of writing this down it doesn't
quite line up well with the well-trodden formulation of computation
and continuation that we see so succinctly expressed in Milner's
calculi. This suggests that there might be advantages to this account
of dynamics waiting to be explored.

\paragraph{Quantum logic}
In this connection, we also note that by virtue of having the
Hennessy-Milner construction, we can pull the construction through the
interpretation of QM. This gives us a natural candidate for a quantum
logic that enjoys an extremely tight connection with it's domain of
interpretation, making the construction much less ad hoc (rather it is
the image of functor!).

\paragraph{Quantum probabiity}
i have questions about the basis of the interpretation of inner
product as probability amplitude. In particular, using which
axiomatization of probability theory does the notion of probability
amplitude earn the right to be so dubbed? In other words, where is the
proof that the operation for calculating a probability amplitude (and
then squaring) satisfies the axioms of what it means to calculate a
probability? Even if such a proof exists (i have yet to find it in the
literature), i wonder if it might not be possible to turn things on
their heads. Can we view the calculation of the probability amplitude
as an axiomatization of probability? If so, then the definition we
give for calculating probability amplitude may provide the basis for
an \emph{effective} theory of probability.

\paragraph{Quantum vs ``biological'' information}
Finally, i want to conclude with a more philosophical observation. At
a recent workshop in which QM was a predominant topic i noticed
something about quantum information. The speaker was giving a riveting
discussion of axiomatic QM and showing how properties of ``no
cloning'' and ``no deleting'' emerged as consequences of the
axiomatization. Theorems of this form are necessary to give us a sense
of confidence that our axioms characterize the physical theory. What
struck me, though, was that if quantum information is neither erasable
nor replicable it is markedly different from \emph{life}. Two of the
things we know about life is that

\begin{itemize}
  \item it ends;
  \item to gain some measure of persistence, to transcend it's
    finitude it is imminently copyable.
\end{itemize}

Both of these qualities are summarized succinctly in the aphorism: all
flesh is grass. For me these two kinds of ``information'' -- call them
quantum and biological -- are end points on a spectrum of strategies
for persistence. At one end, we have those curious entities that enjoy
uniqueness and permanence; at the other, we have those who in the face
of a certain end and an uncertain present make a go of passing
something on. To me one of the more remarkable aspects of the latter
strategy is that in the presence of noise (and certain features of
copying) we get a kind of dynamism, a chance for improvement against a
given persistent condition.

% subsection other_calculi_other_bisimulations_and_geometry_as_behavior (end)




% section conclusion (end)

%\documentclass[12pt]{llncs}
%\documentclass{jktr}

\usepackage[pdftex]{hyperref}                   
\usepackage {listings}
\usepackage {mathpartir}
\usepackage{bcprules}
%\usepackage{listings}
                       
\usepackage{graphicx} 
%\usepackage[margins=2.5cm,nohead,nofoot]{geometry}
%\usepackage{geometry}
\usepackage{amsfonts}
\usepackage{amstext}
\usepackage{latexsym}
\usepackage{amssymb}
\usepackage{color}


%\include{myPreamble}
\include{qm2pi.local} 

%\ifpdf
%\usepackage[pdftex]{graphicx}
%\else
%\usepackage{graphicx}
%\fi

 % \ifpdf
%  \usepackage{pdfsync}
%  \if


%\title{Brief Article}
%\author{David F. Snyder}
%\author{L.G. Meredith}

%\address{Dept. of Math., Texas State University--San Marcos, San Marcos, TX 78666}
       
\pagestyle{empty}


\begin{document}

\lstset{language=[Objective]Caml,frame=shadowbox}

\input{qm2pi.front}

% section front matter (end)

\input{qm2pi.intro} 
 
% section introduction (end)

% \input{qm2pi.knotations} 

% section notation (end)

\input{qm2pi.process.calculi} 

% section concurrent_process_calculi_and_spatial_logics_ (end)
    
%\input{qm2pi.knots2pi} 

%\input{qm2pi.trefoil} 

%\input{qm2pi.mainthm} 

% subsection basic_interpretation (end)

%\input{qm2pi.rho.presentation} 
\subsection{The syntax and semantics of the notation system}\label{sub:the_syntax_and_semantics_of_the_notation_system} % (fold)

We now summarize a technical presentation of the calculus that
embodies our theory of dynamics. The typical presentation of such a
calculus follows the style of giving generators and relations on
them. The grammar, below, describing term constructors, freely
generates the set of processes, $\Proc$. This set is then quotiented
by a relation known as structural congruence and it is over this set
that the notion of dynamics is expressed. This presentation is
essentially that of \cite{MeredithR05} with the addition of
polyadicity and summation. For readability we have relegated some of
the technical subtleties to an appendix.

\subsubsection{Process grammar}\label{subsub:process_grammar}

\begin{mathpar}
  \inferrule* [lab=synchronization] {} {{M} \bc \pzero \;|\; x?F \;|\; x!C }
  \and
  \inferrule* [lab=abstraction] {} {{F} \bc (x)P}
  \and
  \inferrule* [lab=concretion] {} {{C} \bc \langle Q \rangle}
  \and
  \inferrule* [lab=process] {} {{P,Q} \bc M \;| \;P|Q \;|\; @{x}}
  \and
  \inferrule* [lab=name] {} {{x} \bc \quotep{P}}
\end{mathpar} 

Note that $\vec{x}$ (resp. $\vec{P}$) denotes a vector of names
(resp. processes) of length $|\vec{x}|$ (resp. $|\vec{P}|$). We adopt
the following useful abbreviations.

\begin{mathpar}
   x?(\vec{y}).P := x.(\vec{y})P \and  x\clift{\vec{P}} := x.\clift{\vec{P}}
   \and x!(y) := \lift{x}{\dropn{y}}
   \and \Pi_{i=0}^{n-1}P_i := P_0 | \ldots | P_{n-1}
\end{mathpar}

\subsubsection{Structural congruence}

\paragraph{Free and bound names and alpha-equivalence.} At the
core of structural equivalence is alpha-equivalence which identifies
process that are the same up to a change of variable. Formally, we
recognize the distinction between free and bound names. The free names
of a process, $\freenames{P}$, may be calculated recursively as
follows:

\begin{mathpar}
\freenames{\pzero} := \emptyset
  \and \\
  \freenames{x?(y).P} := \{ x \} \cup (\freenames{P} \setminus \{ y \})
  \and 
  \freenames{x!\langle P \rangle} := \{ x \} \cup \{ P \} 
  \and \\
  \freenames{P|Q} := \freenames{P} \cup \freenames{Q}
  \and \\
  \freenames{@{x}} := \{ x \}
\end{mathpar}

$\pi$
$\quotep{\pi}$

$\freenames{-} : \pi \to \mathcal{P}(\quotep{\pi})$

\begin{eqnarray*}
  \freenames{\pzero} & := & \emptyset \\
  \freenames{x?(y).P} & := & \{ x \} \cup (\freenames{P} \setminus \{ y \}) \\
  \freenames{x!\langle P \rangle} & := & \{ x \} \cup \{ P \} \\
  \freenames{P|Q} & := & \freenames{P} \cup \freenames{Q} \\
  \freenames{\dropn{x}} & := & \{ x \}
\end{eqnarray*}

The bound names of a process, $\boundnames{P}$, are those names occurring in $P$
that are not free. For example, in $x?(y).0$, the name $x$ is free, while $y$ is bound.

\begin{mathpar}
  \inferrule* [lab=monoidal-laws] {} { P|Q \equiv Q|P \and P|0 \equiv P \and P|(Q|R) \equiv (P|Q)|R }
\end{mathpar}

\begin{mathpar}
  \inferrule* [lab=alpha-equivalence] {} { (x)P \equiv (y)P\{y/x\} \and y \not\in \freenames{P} }
\end{mathpar}

\begin{definition}
Then two processes, $P,Q$, are alpha-equivalent if $P = Q\{\vec{y}/\vec{x}\}$ for
some $\vec{x} \in \boundnames{Q},\vec{y} \in \boundnames{P}$, where $Q\{\vec{y}/\vec{x}\}$
denotes the capture-avoiding substitution of $\vec{y}$ for $\vec{x}$ in $Q$.
\end{definition}

\begin{definition}
  The {\em structural congruence} \cite{SangiorgiWalker} , $\equiv$,
  between processes is the least congruence containing
  alpha-equivalence, satisfying the abelian monoid laws
  (associativity, commutativity and $\pzero$ as identity) for parallel
  composition $|$ and for summation $+$.
\end{definition}

\subsection{Name equivalence}

We take name equivalence, written $\nameeq$, to be the smallest
equivalence relation generated by the following rules.

\begin{mathpar}
\inferrule*[lab=Quote-drop]
{ }
{ \quotep{@{x}} \nameeq x }

\inferrule*[lab=Struct-equiv]
{ P \scong Q }
{ \quotep{P} \nameeq \quotep{Q} }
\end{mathpar}

The astute reader will have noticed that the mutual recursion of names
and processes imposes a mutual recursion on alpha-equivalence and
structural equivalence via name-equivalence. Fortunately, all of this
works out pleasantly and we may calculate in the natural way, free of
concern. The reader interested in the details is referred to the
appendix \ref{appendix:rho_details}.

\subsection{Substitution}

We use $\Proc$ for the set of processes, $\QProc$ for the set of
names, and $\id{\{}\vec{y} / \vec{x} \id{\}}$ to denote partial maps,
$s : \QProc \rightarrow \QProc$. A map, $s$ lifts, uniquely, to a map
on process terms, $\widehat{s} : \Proc \rightarrow \Proc$ by the
following equations.

\begin{mathpar}
  (0) \psubstp{Q}{P} := 0 \\
  (R \juxtap S) \psubstp{Q}{P}
  :=    
  (R)\psubstp{Q}{P} \juxtap (S) \psubstp{Q}{P} \\
  (x?(y).R) \psubstp{Q}{P}    
  :=    
  (x)\substp{Q}{P} (z)\concat( (R \psubstn{z}{y}) \psubstp{Q}{P} ) \\
  (\lift{x}{R}) \psubstp{Q}{P}  
  :=
  \lift{(x)\substp{Q}{P}}{ R \psubstp{Q}{P} } \\
%   (\dropn{x})  \psubstp{Q}{P}       
%   := 
%   \left\{ 
%     \begin{array}{ccc} 
%       \dropn{\quotep{Q}} & & x \nameeq \quotep{P} \\
%       \dropn{x} & & otherwise \\
%     \end{array}
%   \right. 
  (\dropn{x})  \psubstp{Q}{P}       
  := 
  \left\{ 
    \begin{array}{ccc} 
      Q & & x \nameeq \quotep{P} \\
      \dropn{x} & & otherwise \\
    \end{array}
  \right.
\end{mathpar}
 

where

\begin{eqnarray}
  (x)\id{\{} \lpquote Q \rpquote / \lpquote P \rpquote \id{\}}            = 
  \left\{ 
    \begin{array}{ccc}
      \lpquote Q \rpquote & & x \nameeq \lpquote P \rpquote \\
      x & & otherwise \\
    \end{array}
  \right. \nonumber
\end{eqnarray}

and $z$ is chosen distinct from $\quotep{P}$, $\quotep{Q}$, the free
names in $Q$, and all the names in $R$. Our $\alpha$-equivalence will
be built in the standard way from this substitution.

\begin{remark}\label{rem:no_self_referential_names}
  One consequence of these definitions is that $\forall P. \quotep{P}
  \not\in \freenames{P}$.
\end{remark}

\subsection{ Dynamic quote: an example }

Anticipating something of what's to come, consider applying the
substitution, $\widehat{\id{\{}u / z \id{\}}}$, to the following pair
of processes, $\lift{w}{y!(z)}$ and $w[ \lpquote y!(z) \rpquote ]$.

\begin{eqnarray}
	\lift{w}{y!(z)}\widehat{\id{\{}u / z \id{\}}}
		& = &
		\lift{w}{y!(u)} \nonumber\\
	w[ \lpquote y!(z) \rpquote ] \widehat{ \id{\{}u / z \id{\}} }
		& = &
		w[ \lpquote y!(z) \rpquote ] \nonumber
\end{eqnarray}

Because the body of the process between quotes is impervious to
substitution, we get radically different answers. In fact, by
examining the first process in an input context,
e.g. $x?(z).\lift{w}{y!(z)}$, we see that the process under the lift
operator may be shaped by prefixed inputs binding a name inside it. In
this sense, the lift operator will be seen as a way to dynamically
construct processes before reifying them as names.

Finally equipped with these standard features we can present the
dynamics of the calculus.

\subsubsection{Operational semantics} 

Finally, we introduce the computational dynamics. What marks these
algebras as distinct from other more traditionally studied algebraic
structures, e.g. vector spaces or polynomial rings, is the manner in
which dynamics is captured. In traditional structures, dynamics is typically
expressed through morphisms between such structures, as in linear maps
between vector spaces or morphisms between rings. In algebras
associated with the semantics of computation, the dynamics is
expressed as part of the algebraic structure itself, through a
reduction reduction relation typically denoted by $\red$. Below, we
give a recursive presentation of this relation for the calculus used
in the encoding.

$\red \subseteq \pi \times \pi$
$\red : \pi \to \mathcal{P}(\pi)$

\begin{mathpar}
  \inferrule* [lab=Comm] { \textsf{match}( x_{src}, x_{trgt} ) } { x_{trgt}?(y)P \; | \; x_{src}!\langle {Q} \rangle \red P\{\quotep{Q}/y}\} }
  \and \\
  \inferrule* [lab=Par] {{P} \red {P}'} {{{P} | {Q}} \red {{P}' | {Q}}}
  \and
  \inferrule* [lab=Equiv]{{{P} \scong {P}'} \andalso {{P}' \red {Q}'} \andalso {{Q}' \scong {Q}}}{{P} \red {Q}}
\end{mathpar}

\begin{eqnarray*}
  match_{\equiv} (\quotep{P},\quotep{Q}) & := & P \equiv Q \\
  match_{\dagger}(\quotep{P},\quotep{Q}) & := & \forall R. P|Q \red^{*} R => R \red^{*} 0 \\
  match_{K}(\quotep{P},\quotep{Q}) & := & K \mbox{ for some context } K
\end{eqnarray*}

$u?(x)P | u!\langle Q \rangle \red P\{\quotep{Q}/x\}$

%We write $\wred$ for $\red^*$, and $P\red$ if $\exists Q $ such that $ P \red Q$.
We write $P\red$ if $\exists Q $ such that $ P \red Q$ and $P\not\red$, otherwise.

\section{Replication}

As mentioned before, it is known that replication (and hence
recursion) can be implemented in a higher-order process algebra
\cite{SangiorgiWalker}. As our first example of calculation with the
machinery thus far presented we give the construction explicitly in
the {\rhoc}.

\begin{eqnarray}
	D_{x} & := & \prefix{x}{y}{(\binpar{\outputp{x}{y}}{@{y}})} \nonumber\\
	\bangp_{x}{P} & := & \binpar{{x}!\langle{\binpar{D_{x}}{P}}\rangle}{D_{x}} \nonumber
\end{eqnarray}

\begin{eqnarray}
	\bangp_{x}{P} & & \nonumber\\
	=
	& {x}!\langle{(\prefix{x}{y}{(\outputp{x}{y} | @{y})) | P}}\rangle 
	      | \prefix{x}{y}{(\outputp{x}{y} | @{y})} & \nonumber\\
	\red
	& (\outputp{x}{y} | @{y})\substn{\quotep{(\prefix{x}{y}{(@{y} | \outputp{x}{y})) | P}}}{y} & \nonumber\\
	=
	& \outputp{x}{\quotep{(\prefix{x}{y}{(\outputp{x}{y} | @{y})) | P}}}
	  | {(\prefix{x}{y}{(\outputp{x}{y} | @{y})) | P}} & \nonumber\\
	\red
	& \ldots & \nonumber\\
	\red^*
	& P | P | \ldots & \nonumber
\end{eqnarray}

Of course, this encoding, as an implementation, runs away, unfolding
$\bangp{P}$ eagerly. A lazier and more implementable replication
operator, restricted to input-guarded processes, may be obtained as follows.

\begin{eqnarray}
\bangp{\prefix{u}{v}{P}} 
	:= 
	\binpar{\lift{x}{\prefix{u}{v}{(\binpar{D(x)}{P})}}}{D(x)} \nonumber
\end{eqnarray}

\begin{remark}
  Note that the lazier definition still does not deal with summation
  or mixed summation (i.e. sums over input and output). The reader is
  invited to construct definitions of replication that deal with these
  features. 

  Further, the definitions are parameterized in a name, $x$. Can you,
  gentle reader, make a definition that eliminates this parameter and
  guarantees no accidental interaction between the replication
  machinery and the process being replicated -- i.e. no accidental
  sharing of names used by the process to get its work done and the
  name(s) used by the replication to effect copying. This latter
  revision of the definition of replication is crucial to obtaining
  the expected identity $!!P \sim !P$.
\end{remark}

\begin{remark}\label{rem:paradoxical_combinator}
  The reader familiar with the lambda calculus will have noticed the
  similarity between $D$ and the paradoxical combinator.

  [Ed. note: the existence of this seems to suggest we have to be more
  restrictive on the set of processes and names we admit if we are to
  support no-cloning.]
\end{remark}

\subsubsection{Bisimulation}

The computational dynamics gives rise to another kind of equivalence,
the equivalence of computational behavior. As previously mentioned
this is typically captured \emph{via} some form of bisimulation.

% The notion we use in this paper is weak barbed bisimulation
% \cite{milner91polyadicpi}.

The notion we use in this paper is derived from weak barbed
bisimulation \cite{milner91polyadicpi}. 

\begin{definition}
An \emph{observation relation}, $\downarrow_{\mathcal N}$, over a set
of names, $\mathcal N$, is the smallest relation satisfying the rules
below.

\infrule[Out-barb]{y \in {\mathcal N}, \; x \nameeq y}
		  {\outputp{x}{v} \downarrow_{\mathcal N} x}
\infrule[Par-barb]{\mbox{$P\downarrow_{\mathcal N} x$ or $Q\downarrow_{\mathcal N} x$}}
		  {\binpar{P}{Q} \downarrow_{\mathcal N} x}

We write $P \Downarrow_{\mathcal N} x$ if there is $Q$ such that 
$P \wred Q$ and $Q \downarrow_{\mathcal N} x$.
\end{definition}

\begin{definition}
%\label{def.bbisim}
An  ${\mathcal N}$-\emph{barbed bisimulation} over a set of names, ${\mathcal N}$, is a symmetric binary relation 
${\mathcal S}_{\mathcal N}$ between agents such that $P\rel{S}_{\mathcal N}Q$ implies:
\begin{enumerate}
\item If $P \red P'$ then $Q \wred Q'$ and $P'\rel{S}_{\mathcal N} Q'$.
\item If $P\downarrow_{\mathcal N} x$, then $Q\Downarrow_{\mathcal N} x$.
\end{enumerate}
$P$ is ${\mathcal N}$-barbed bisimilar to $Q$, written
$P \wbbisim_{\mathcal N} Q$, if $P \rel{S}_{\mathcal N} Q$ for some ${\mathcal N}$-barbed bisimulation ${\mathcal S}_{\mathcal N}$.
\end{definition}

$\mathcal{R} \subseteq \pi \times \pi$

$P \mathcal{R} Q => \forall P'. P \red P' \Rightarrow \exists Q'. Q \red Q', P' \mathcal{R} Q'$

$P \vdash x \Rightarrow Q \vdash x$

\begin{mathpar}
  \inferrule*[lab=Out-barb]{x \nameeq y}{{y}!\langle{Q}\rangle \vdash x}
  \and
  \inferrule*[lab=Par-barb]{\mbox{$P\vdash x$ or $Q\vdash x$}}{\binpar{P}{Q} \vdash x}
\end{mathpar}

\subsubsection{Contexts}

One of the principle advantages of computational calculi like the
$\pi$-calculus is a well-defined notion of context,
contextual-equivalence and a correlation between
contextual-equivalence and notions of bisimulation. The notion of
context allows the decomposition of a process into (sub-)process and
its syntactic environment, its context. Thus, a context may be
thought of as a process with a ``hole'' (written $\Box$) in it. The
application of a context $M$ to a process $P$, written $M[P]$, is
tantamount to filling the hole in $M$ with $P$. In this paper we do
not need the full weight of this theory, but do make use of the notion
of context in the proof the main theorem. 

\begin{mathpar}
  \inferrule* [lab=summation] {} {{M_{M},M_{N}} \bc \Box \;|\; x.M_{A} \;|\; M_{M}+M_{N}}
  \and
  \inferrule* [lab=agent] {} {{M_{A}} \bc (\vec{x})M_{P} \;| \; \clift{P_0,\ldots,M_{P},\ldots,P_N}}
  \and \\
  \inferrule* [lab=process] {} {{M_{P}} \bc M_{N} \;| \;P|M_{P} }
\end{mathpar} 

\begin{mathpar}
  \inferrule* [lab=sychronization] {} {M_{N} \bc \Box \;|\; x?M_{F} \;|\; x!M_{C}}
  \and
  \inferrule* [lab=abstraction] {} {{M_{F}} \bc (x)M_{P} }
  \and
  \inferrule* [lab=concretion] {} {{M_{C}} \bc \langle M_{P} \rangle }
  \and \\
  \inferrule* [lab=process] {} {{M_{P}} \bc M_{N} \;| \;P|M_{P} }
\end{mathpar}

\begin{definition}[contextual application] Given a context $M$, and
  process $P$, we define the \emph{contextual application}, $M[P] :=
  M\{P/\Box\}$. That is, the contextual application of M to P is the
  substitution of $P$ for $\Box$ in $M$.
\end{definition}

$\meaningof{-} : L \to \mathcal{P}(\pi)$

\begin{mathpar}
  \inferrule* [lab=collection] {} {\meaningof{true} = \pi, \and \meaningof{~E} = \pi \setminus \meaningof{E}, \and \meaningof{E_{1} \& E_{2}} = \meaningof{E_{1}} \cap \meaningof{E_{2}}}
\end{mathpar}

\begin{mathpar}
  \inferrule* [lab=structure] {} {\meaningof{0} = \{ P \in \pi | P \equiv 0 \}, \and \\ \meaningof{E_1 | E_2} = \{ P \in \pi | P \equiv P_{1} | P_{2}, P_{1} \in \meaningof{E_{1}}, P_{2} \in \meaningof{E_2}\} }
\end{mathpar}

\begin{mathpar}
 \inferrule* [lab=behavior] {} {\meaningof{\langle a?b \rangle E} = \{ P \in \pi | P \equiv Q | u?(y)P', \\ \and \\\\ \and \\ \;\;\; u \in \meaningof{a}, \forall z.P'\{z/y\} \in \meaningof{E\{z/b\}}\}, \and \\ \meaningof{a!E} = \{ P \in \pi | P \equiv Q | x!\langle P' \rangle, x \in \meaningof{a} P' \in \meaningof{E}\} }
\end{mathpar}

\begin{mathpar}
 \inferrule* [lab=nominal] {} {\meaningof{\quotep{E}} = \{ \quotep{P} \in \quotep{\pi} | P \in \meaningof{E} \}, \and \meaningof{\quotep{P}} = \{ \quotep{Q} \in \quotep{\pi} | P \equiv Q \} \and \\ \meaningof{@\quotep{E}} = \{ P \in \pi | P \equiv @x, x \in \meaningof{E} \}}
\end{mathpar}

\begin{eqnarray*}
  \\
  \meaningof{-} : TS \to ST
\end{eqnarray*}

\begin{eqnarray*}
  \\
  L : TS \to ST
\end{eqnarray*}

\begin{eqnarray*}
  \\
  P \models E \iff P \in \meaningof{E}
\end{eqnarray*}

\begin{eqnarray*}
  P \approx_{L} Q \iff \forall E \in L. P \models E \iff Q \models E
\end{eqnarray*}

\begin{eqnarray*}
  P \approx_{K} Q
\end{eqnarray*}

\begin{eqnarray*}
  P \approx Q
\end{eqnarray*}

$\approx_{K} = \approx = \approx_{L}$

\subsubsection{Contextual duality}

Note that contexts extend the quotation operation to a family of
operations from processes to names. Given a context, $M$, we can
define a \emph{nominal context}, $\quotep{M}$ by $\quotep{M}[P] :=
\quotep{M[P]}$. To foreshadow what is to come we observe that these
operations enjoy a duality with processes very much like the duality
between vectors and maps from vectors to scalars.

Further, because the calculus is essentially higher-order, we have a
correspondence between contexts and processes. More specifically,
given a name $x$ and a context $M$ we can construct $M^{*}_{x}$ such
that 

\begin{mathpar}
  M^{*}_{x} | \lift{x}{P} \red M[P]
\end{mathpar}

namely,

\begin{mathpar}
  M^{*}_{x} := x?(u).M[\dropn{u}]
\end{mathpar}

The dependence of $M^{*}_{x}$ on a name makes it an abstraction, 

\begin{mathpar}
  M^{*} := (x)x?(u).M[\dropn{u}]
\end{mathpar}

\subsection{Additional notation}

It will sometimes be convenient to denote the process a name
quotes. We already have the notation $x = \quotep{P}$, but it will be
convenient to introduce an alternate notation, $\procn{x}$, when we
want to emphasize the connection to the use of the name. Note that, by
virtue of name equivalence, $\quotep{\procn{x}} \nameeq x$; so, the
notation is consistent with previous definitions.

Further, because names have structure it is possible to effect
substitutions on the basis of that structure. This means we need to
upgrade our notation for substitutions, which we accomplish by
adapting comprehension notation. Thus,

\begin{mathpar}
  P\{ y / x : x \in S \}
\end{mathpar}

is interpreted to mean the process derived from P by replacing (in a
capture-avoiding manner) each occurrence of $x$ in $S$ by $y$. For example,

\begin{mathpar}
  P\{ \quotep{\procn{x}|\procn{x}} / x : x \in \freenames{P} \}
\end{mathpar}

will replace each (occurrence) of a free name $x$ in $P$ by
$\quotep{\procn{x}|\procn{x}}$.

Also, we will avail ourselves of the notation $x^{L}$ and $x^{R}$ to
denote injections of a name into disjoint copies of the name
space. There are numerous ways to accomplish this. One example can be
found in \cite{MeredithR05}. This notation overloads to vectors of
names: $\vec{x}^{\pi} := (x_{i}^{\pi} \; : \; 0 \leq i < |\vec{x}| )$ where $\pi \in \{L,R\}$.

We also use $P^{\Box} := P|\Box$.

In \cite{MeredithR05} an interpretation of the new operator is
given. It turns out that there are several possible interpretations
all enjoying the requisite algebraic properties of the operator (see
\cite{milner91polyadicpi}). We will therefore make liberal use of
$(\nu\; \vec{x})P$.

% subsection the_syntax_and_semantics_of_the_notation_system (end)   

\input{qm2pi.qmops} 

\input{qm2pi.sterngerlach} 

\input{qm2pi.metric} 

% section concurrent_process_calculi (end)

%\input{qm2pi.proofsketch}

% section proof sketch (end)

%\input{qm2pi.slviaknots} 

% section spatial logic via knots (end)

\input{qm2pi.conclusion}

% section conclusion (end)

%\input{qm2pi.dtcodes} 

% section wiring algorithm (end)

\input{qm2pi.ack} 

% section acknowledgments (end)

\newpage


\bibliographystyle{plain}   
\bibliography{../../biblios/main.bib}

\input{qm2pi.rhodetails}

\end{document}

 

% section wiring algorithm (end)

\documentclass[12pt]{llncs}
%\documentclass{jktr}

\usepackage[pdftex]{hyperref}                   
\usepackage {listings}
\usepackage {mathpartir}
\usepackage{bcprules}
%\usepackage{listings}
                       
\usepackage{graphicx} 
%\usepackage[margins=2.5cm,nohead,nofoot]{geometry}
%\usepackage{geometry}
\usepackage{amsfonts}
\usepackage{amstext}
\usepackage{latexsym}
\usepackage{amssymb}
\usepackage{color}


%\include{myPreamble}
\include{qm2pi.local} 

%\ifpdf
%\usepackage[pdftex]{graphicx}
%\else
%\usepackage{graphicx}
%\fi

 % \ifpdf
%  \usepackage{pdfsync}
%  \if


%\title{Brief Article}
%\author{David F. Snyder}
%\author{L.G. Meredith}

%\address{Dept. of Math., Texas State University--San Marcos, San Marcos, TX 78666}
       
\pagestyle{empty}


\begin{document}

\lstset{language=[Objective]Caml,frame=shadowbox}

\input{qm2pi.front}

% section front matter (end)

\input{qm2pi.intro} 
 
% section introduction (end)

% \input{qm2pi.knotations} 

% section notation (end)

\input{qm2pi.process.calculi} 

% section concurrent_process_calculi_and_spatial_logics_ (end)
    
%\input{qm2pi.knots2pi} 

%\input{qm2pi.trefoil} 

%\input{qm2pi.mainthm} 

% subsection basic_interpretation (end)

%\input{qm2pi.rho.presentation} 
\subsection{The syntax and semantics of the notation system}\label{sub:the_syntax_and_semantics_of_the_notation_system} % (fold)

We now summarize a technical presentation of the calculus that
embodies our theory of dynamics. The typical presentation of such a
calculus follows the style of giving generators and relations on
them. The grammar, below, describing term constructors, freely
generates the set of processes, $\Proc$. This set is then quotiented
by a relation known as structural congruence and it is over this set
that the notion of dynamics is expressed. This presentation is
essentially that of \cite{MeredithR05} with the addition of
polyadicity and summation. For readability we have relegated some of
the technical subtleties to an appendix.

\subsubsection{Process grammar}\label{subsub:process_grammar}

\begin{mathpar}
  \inferrule* [lab=synchronization] {} {{M} \bc \pzero \;|\; x?F \;|\; x!C }
  \and
  \inferrule* [lab=abstraction] {} {{F} \bc (x)P}
  \and
  \inferrule* [lab=concretion] {} {{C} \bc \langle Q \rangle}
  \and
  \inferrule* [lab=process] {} {{P,Q} \bc M \;| \;P|Q \;|\; @{x}}
  \and
  \inferrule* [lab=name] {} {{x} \bc \quotep{P}}
\end{mathpar} 

Note that $\vec{x}$ (resp. $\vec{P}$) denotes a vector of names
(resp. processes) of length $|\vec{x}|$ (resp. $|\vec{P}|$). We adopt
the following useful abbreviations.

\begin{mathpar}
   x?(\vec{y}).P := x.(\vec{y})P \and  x\clift{\vec{P}} := x.\clift{\vec{P}}
   \and x!(y) := \lift{x}{\dropn{y}}
   \and \Pi_{i=0}^{n-1}P_i := P_0 | \ldots | P_{n-1}
\end{mathpar}

\subsubsection{Structural congruence}

\paragraph{Free and bound names and alpha-equivalence.} At the
core of structural equivalence is alpha-equivalence which identifies
process that are the same up to a change of variable. Formally, we
recognize the distinction between free and bound names. The free names
of a process, $\freenames{P}$, may be calculated recursively as
follows:

\begin{mathpar}
\freenames{\pzero} := \emptyset
  \and \\
  \freenames{x?(y).P} := \{ x \} \cup (\freenames{P} \setminus \{ y \})
  \and 
  \freenames{x!\langle P \rangle} := \{ x \} \cup \{ P \} 
  \and \\
  \freenames{P|Q} := \freenames{P} \cup \freenames{Q}
  \and \\
  \freenames{@{x}} := \{ x \}
\end{mathpar}

$\pi$
$\quotep{\pi}$

$\freenames{-} : \pi \to \mathcal{P}(\quotep{\pi})$

\begin{eqnarray*}
  \freenames{\pzero} & := & \emptyset \\
  \freenames{x?(y).P} & := & \{ x \} \cup (\freenames{P} \setminus \{ y \}) \\
  \freenames{x!\langle P \rangle} & := & \{ x \} \cup \{ P \} \\
  \freenames{P|Q} & := & \freenames{P} \cup \freenames{Q} \\
  \freenames{\dropn{x}} & := & \{ x \}
\end{eqnarray*}

The bound names of a process, $\boundnames{P}$, are those names occurring in $P$
that are not free. For example, in $x?(y).0$, the name $x$ is free, while $y$ is bound.

\begin{mathpar}
  \inferrule* [lab=monoidal-laws] {} { P|Q \equiv Q|P \and P|0 \equiv P \and P|(Q|R) \equiv (P|Q)|R }
\end{mathpar}

\begin{mathpar}
  \inferrule* [lab=alpha-equivalence] {} { (x)P \equiv (y)P\{y/x\} \and y \not\in \freenames{P} }
\end{mathpar}

\begin{definition}
Then two processes, $P,Q$, are alpha-equivalent if $P = Q\{\vec{y}/\vec{x}\}$ for
some $\vec{x} \in \boundnames{Q},\vec{y} \in \boundnames{P}$, where $Q\{\vec{y}/\vec{x}\}$
denotes the capture-avoiding substitution of $\vec{y}$ for $\vec{x}$ in $Q$.
\end{definition}

\begin{definition}
  The {\em structural congruence} \cite{SangiorgiWalker} , $\equiv$,
  between processes is the least congruence containing
  alpha-equivalence, satisfying the abelian monoid laws
  (associativity, commutativity and $\pzero$ as identity) for parallel
  composition $|$ and for summation $+$.
\end{definition}

\subsection{Name equivalence}

We take name equivalence, written $\nameeq$, to be the smallest
equivalence relation generated by the following rules.

\begin{mathpar}
\inferrule*[lab=Quote-drop]
{ }
{ \quotep{@{x}} \nameeq x }

\inferrule*[lab=Struct-equiv]
{ P \scong Q }
{ \quotep{P} \nameeq \quotep{Q} }
\end{mathpar}

The astute reader will have noticed that the mutual recursion of names
and processes imposes a mutual recursion on alpha-equivalence and
structural equivalence via name-equivalence. Fortunately, all of this
works out pleasantly and we may calculate in the natural way, free of
concern. The reader interested in the details is referred to the
appendix \ref{appendix:rho_details}.

\subsection{Substitution}

We use $\Proc$ for the set of processes, $\QProc$ for the set of
names, and $\id{\{}\vec{y} / \vec{x} \id{\}}$ to denote partial maps,
$s : \QProc \rightarrow \QProc$. A map, $s$ lifts, uniquely, to a map
on process terms, $\widehat{s} : \Proc \rightarrow \Proc$ by the
following equations.

\begin{mathpar}
  (0) \psubstp{Q}{P} := 0 \\
  (R \juxtap S) \psubstp{Q}{P}
  :=    
  (R)\psubstp{Q}{P} \juxtap (S) \psubstp{Q}{P} \\
  (x?(y).R) \psubstp{Q}{P}    
  :=    
  (x)\substp{Q}{P} (z)\concat( (R \psubstn{z}{y}) \psubstp{Q}{P} ) \\
  (\lift{x}{R}) \psubstp{Q}{P}  
  :=
  \lift{(x)\substp{Q}{P}}{ R \psubstp{Q}{P} } \\
%   (\dropn{x})  \psubstp{Q}{P}       
%   := 
%   \left\{ 
%     \begin{array}{ccc} 
%       \dropn{\quotep{Q}} & & x \nameeq \quotep{P} \\
%       \dropn{x} & & otherwise \\
%     \end{array}
%   \right. 
  (\dropn{x})  \psubstp{Q}{P}       
  := 
  \left\{ 
    \begin{array}{ccc} 
      Q & & x \nameeq \quotep{P} \\
      \dropn{x} & & otherwise \\
    \end{array}
  \right.
\end{mathpar}
 

where

\begin{eqnarray}
  (x)\id{\{} \lpquote Q \rpquote / \lpquote P \rpquote \id{\}}            = 
  \left\{ 
    \begin{array}{ccc}
      \lpquote Q \rpquote & & x \nameeq \lpquote P \rpquote \\
      x & & otherwise \\
    \end{array}
  \right. \nonumber
\end{eqnarray}

and $z$ is chosen distinct from $\quotep{P}$, $\quotep{Q}$, the free
names in $Q$, and all the names in $R$. Our $\alpha$-equivalence will
be built in the standard way from this substitution.

\begin{remark}\label{rem:no_self_referential_names}
  One consequence of these definitions is that $\forall P. \quotep{P}
  \not\in \freenames{P}$.
\end{remark}

\subsection{ Dynamic quote: an example }

Anticipating something of what's to come, consider applying the
substitution, $\widehat{\id{\{}u / z \id{\}}}$, to the following pair
of processes, $\lift{w}{y!(z)}$ and $w[ \lpquote y!(z) \rpquote ]$.

\begin{eqnarray}
	\lift{w}{y!(z)}\widehat{\id{\{}u / z \id{\}}}
		& = &
		\lift{w}{y!(u)} \nonumber\\
	w[ \lpquote y!(z) \rpquote ] \widehat{ \id{\{}u / z \id{\}} }
		& = &
		w[ \lpquote y!(z) \rpquote ] \nonumber
\end{eqnarray}

Because the body of the process between quotes is impervious to
substitution, we get radically different answers. In fact, by
examining the first process in an input context,
e.g. $x?(z).\lift{w}{y!(z)}$, we see that the process under the lift
operator may be shaped by prefixed inputs binding a name inside it. In
this sense, the lift operator will be seen as a way to dynamically
construct processes before reifying them as names.

Finally equipped with these standard features we can present the
dynamics of the calculus.

\subsubsection{Operational semantics} 

Finally, we introduce the computational dynamics. What marks these
algebras as distinct from other more traditionally studied algebraic
structures, e.g. vector spaces or polynomial rings, is the manner in
which dynamics is captured. In traditional structures, dynamics is typically
expressed through morphisms between such structures, as in linear maps
between vector spaces or morphisms between rings. In algebras
associated with the semantics of computation, the dynamics is
expressed as part of the algebraic structure itself, through a
reduction reduction relation typically denoted by $\red$. Below, we
give a recursive presentation of this relation for the calculus used
in the encoding.

$\red \subseteq \pi \times \pi$
$\red : \pi \to \mathcal{P}(\pi)$

\begin{mathpar}
  \inferrule* [lab=Comm] { \textsf{match}( x_{src}, x_{trgt} ) } { x_{trgt}?(y)P \; | \; x_{src}!\langle {Q} \rangle \red P\{\quotep{Q}/y}\} }
  \and \\
  \inferrule* [lab=Par] {{P} \red {P}'} {{{P} | {Q}} \red {{P}' | {Q}}}
  \and
  \inferrule* [lab=Equiv]{{{P} \scong {P}'} \andalso {{P}' \red {Q}'} \andalso {{Q}' \scong {Q}}}{{P} \red {Q}}
\end{mathpar}

\begin{eqnarray*}
  match_{\equiv} (\quotep{P},\quotep{Q}) & := & P \equiv Q \\
  match_{\dagger}(\quotep{P},\quotep{Q}) & := & \forall R. P|Q \red^{*} R => R \red^{*} 0 \\
  match_{K}(\quotep{P},\quotep{Q}) & := & K \mbox{ for some context } K
\end{eqnarray*}

$u?(x)P | u!\langle Q \rangle \red P\{\quotep{Q}/x\}$

%We write $\wred$ for $\red^*$, and $P\red$ if $\exists Q $ such that $ P \red Q$.
We write $P\red$ if $\exists Q $ such that $ P \red Q$ and $P\not\red$, otherwise.

\section{Replication}

As mentioned before, it is known that replication (and hence
recursion) can be implemented in a higher-order process algebra
\cite{SangiorgiWalker}. As our first example of calculation with the
machinery thus far presented we give the construction explicitly in
the {\rhoc}.

\begin{eqnarray}
	D_{x} & := & \prefix{x}{y}{(\binpar{\outputp{x}{y}}{@{y}})} \nonumber\\
	\bangp_{x}{P} & := & \binpar{{x}!\langle{\binpar{D_{x}}{P}}\rangle}{D_{x}} \nonumber
\end{eqnarray}

\begin{eqnarray}
	\bangp_{x}{P} & & \nonumber\\
	=
	& {x}!\langle{(\prefix{x}{y}{(\outputp{x}{y} | @{y})) | P}}\rangle 
	      | \prefix{x}{y}{(\outputp{x}{y} | @{y})} & \nonumber\\
	\red
	& (\outputp{x}{y} | @{y})\substn{\quotep{(\prefix{x}{y}{(@{y} | \outputp{x}{y})) | P}}}{y} & \nonumber\\
	=
	& \outputp{x}{\quotep{(\prefix{x}{y}{(\outputp{x}{y} | @{y})) | P}}}
	  | {(\prefix{x}{y}{(\outputp{x}{y} | @{y})) | P}} & \nonumber\\
	\red
	& \ldots & \nonumber\\
	\red^*
	& P | P | \ldots & \nonumber
\end{eqnarray}

Of course, this encoding, as an implementation, runs away, unfolding
$\bangp{P}$ eagerly. A lazier and more implementable replication
operator, restricted to input-guarded processes, may be obtained as follows.

\begin{eqnarray}
\bangp{\prefix{u}{v}{P}} 
	:= 
	\binpar{\lift{x}{\prefix{u}{v}{(\binpar{D(x)}{P})}}}{D(x)} \nonumber
\end{eqnarray}

\begin{remark}
  Note that the lazier definition still does not deal with summation
  or mixed summation (i.e. sums over input and output). The reader is
  invited to construct definitions of replication that deal with these
  features. 

  Further, the definitions are parameterized in a name, $x$. Can you,
  gentle reader, make a definition that eliminates this parameter and
  guarantees no accidental interaction between the replication
  machinery and the process being replicated -- i.e. no accidental
  sharing of names used by the process to get its work done and the
  name(s) used by the replication to effect copying. This latter
  revision of the definition of replication is crucial to obtaining
  the expected identity $!!P \sim !P$.
\end{remark}

\begin{remark}\label{rem:paradoxical_combinator}
  The reader familiar with the lambda calculus will have noticed the
  similarity between $D$ and the paradoxical combinator.

  [Ed. note: the existence of this seems to suggest we have to be more
  restrictive on the set of processes and names we admit if we are to
  support no-cloning.]
\end{remark}

\subsubsection{Bisimulation}

The computational dynamics gives rise to another kind of equivalence,
the equivalence of computational behavior. As previously mentioned
this is typically captured \emph{via} some form of bisimulation.

% The notion we use in this paper is weak barbed bisimulation
% \cite{milner91polyadicpi}.

The notion we use in this paper is derived from weak barbed
bisimulation \cite{milner91polyadicpi}. 

\begin{definition}
An \emph{observation relation}, $\downarrow_{\mathcal N}$, over a set
of names, $\mathcal N$, is the smallest relation satisfying the rules
below.

\infrule[Out-barb]{y \in {\mathcal N}, \; x \nameeq y}
		  {\outputp{x}{v} \downarrow_{\mathcal N} x}
\infrule[Par-barb]{\mbox{$P\downarrow_{\mathcal N} x$ or $Q\downarrow_{\mathcal N} x$}}
		  {\binpar{P}{Q} \downarrow_{\mathcal N} x}

We write $P \Downarrow_{\mathcal N} x$ if there is $Q$ such that 
$P \wred Q$ and $Q \downarrow_{\mathcal N} x$.
\end{definition}

\begin{definition}
%\label{def.bbisim}
An  ${\mathcal N}$-\emph{barbed bisimulation} over a set of names, ${\mathcal N}$, is a symmetric binary relation 
${\mathcal S}_{\mathcal N}$ between agents such that $P\rel{S}_{\mathcal N}Q$ implies:
\begin{enumerate}
\item If $P \red P'$ then $Q \wred Q'$ and $P'\rel{S}_{\mathcal N} Q'$.
\item If $P\downarrow_{\mathcal N} x$, then $Q\Downarrow_{\mathcal N} x$.
\end{enumerate}
$P$ is ${\mathcal N}$-barbed bisimilar to $Q$, written
$P \wbbisim_{\mathcal N} Q$, if $P \rel{S}_{\mathcal N} Q$ for some ${\mathcal N}$-barbed bisimulation ${\mathcal S}_{\mathcal N}$.
\end{definition}

$\mathcal{R} \subseteq \pi \times \pi$

$P \mathcal{R} Q => \forall P'. P \red P' \Rightarrow \exists Q'. Q \red Q', P' \mathcal{R} Q'$

$P \vdash x \Rightarrow Q \vdash x$

\begin{mathpar}
  \inferrule*[lab=Out-barb]{x \nameeq y}{{y}!\langle{Q}\rangle \vdash x}
  \and
  \inferrule*[lab=Par-barb]{\mbox{$P\vdash x$ or $Q\vdash x$}}{\binpar{P}{Q} \vdash x}
\end{mathpar}

\subsubsection{Contexts}

One of the principle advantages of computational calculi like the
$\pi$-calculus is a well-defined notion of context,
contextual-equivalence and a correlation between
contextual-equivalence and notions of bisimulation. The notion of
context allows the decomposition of a process into (sub-)process and
its syntactic environment, its context. Thus, a context may be
thought of as a process with a ``hole'' (written $\Box$) in it. The
application of a context $M$ to a process $P$, written $M[P]$, is
tantamount to filling the hole in $M$ with $P$. In this paper we do
not need the full weight of this theory, but do make use of the notion
of context in the proof the main theorem. 

\begin{mathpar}
  \inferrule* [lab=summation] {} {{M_{M},M_{N}} \bc \Box \;|\; x.M_{A} \;|\; M_{M}+M_{N}}
  \and
  \inferrule* [lab=agent] {} {{M_{A}} \bc (\vec{x})M_{P} \;| \; \clift{P_0,\ldots,M_{P},\ldots,P_N}}
  \and \\
  \inferrule* [lab=process] {} {{M_{P}} \bc M_{N} \;| \;P|M_{P} }
\end{mathpar} 

\begin{mathpar}
  \inferrule* [lab=sychronization] {} {M_{N} \bc \Box \;|\; x?M_{F} \;|\; x!M_{C}}
  \and
  \inferrule* [lab=abstraction] {} {{M_{F}} \bc (x)M_{P} }
  \and
  \inferrule* [lab=concretion] {} {{M_{C}} \bc \langle M_{P} \rangle }
  \and \\
  \inferrule* [lab=process] {} {{M_{P}} \bc M_{N} \;| \;P|M_{P} }
\end{mathpar}

\begin{definition}[contextual application] Given a context $M$, and
  process $P$, we define the \emph{contextual application}, $M[P] :=
  M\{P/\Box\}$. That is, the contextual application of M to P is the
  substitution of $P$ for $\Box$ in $M$.
\end{definition}

$\meaningof{-} : L \to \mathcal{P}(\pi)$

\begin{mathpar}
  \inferrule* [lab=collection] {} {\meaningof{true} = \pi, \and \meaningof{~E} = \pi \setminus \meaningof{E}, \and \meaningof{E_{1} \& E_{2}} = \meaningof{E_{1}} \cap \meaningof{E_{2}}}
\end{mathpar}

\begin{mathpar}
  \inferrule* [lab=structure] {} {\meaningof{0} = \{ P \in \pi | P \equiv 0 \}, \and \\ \meaningof{E_1 | E_2} = \{ P \in \pi | P \equiv P_{1} | P_{2}, P_{1} \in \meaningof{E_{1}}, P_{2} \in \meaningof{E_2}\} }
\end{mathpar}

\begin{mathpar}
 \inferrule* [lab=behavior] {} {\meaningof{\langle a?b \rangle E} = \{ P \in \pi | P \equiv Q | u?(y)P', \\ \and \\\\ \and \\ \;\;\; u \in \meaningof{a}, \forall z.P'\{z/y\} \in \meaningof{E\{z/b\}}\}, \and \\ \meaningof{a!E} = \{ P \in \pi | P \equiv Q | x!\langle P' \rangle, x \in \meaningof{a} P' \in \meaningof{E}\} }
\end{mathpar}

\begin{mathpar}
 \inferrule* [lab=nominal] {} {\meaningof{\quotep{E}} = \{ \quotep{P} \in \quotep{\pi} | P \in \meaningof{E} \}, \and \meaningof{\quotep{P}} = \{ \quotep{Q} \in \quotep{\pi} | P \equiv Q \} \and \\ \meaningof{@\quotep{E}} = \{ P \in \pi | P \equiv @x, x \in \meaningof{E} \}}
\end{mathpar}

\begin{eqnarray*}
  \\
  \meaningof{-} : TS \to ST
\end{eqnarray*}

\begin{eqnarray*}
  \\
  L : TS \to ST
\end{eqnarray*}

\begin{eqnarray*}
  \\
  P \models E \iff P \in \meaningof{E}
\end{eqnarray*}

\begin{eqnarray*}
  P \approx_{L} Q \iff \forall E \in L. P \models E \iff Q \models E
\end{eqnarray*}

\begin{eqnarray*}
  P \approx_{K} Q
\end{eqnarray*}

\begin{eqnarray*}
  P \approx Q
\end{eqnarray*}

$\approx_{K} = \approx = \approx_{L}$

\subsubsection{Contextual duality}

Note that contexts extend the quotation operation to a family of
operations from processes to names. Given a context, $M$, we can
define a \emph{nominal context}, $\quotep{M}$ by $\quotep{M}[P] :=
\quotep{M[P]}$. To foreshadow what is to come we observe that these
operations enjoy a duality with processes very much like the duality
between vectors and maps from vectors to scalars.

Further, because the calculus is essentially higher-order, we have a
correspondence between contexts and processes. More specifically,
given a name $x$ and a context $M$ we can construct $M^{*}_{x}$ such
that 

\begin{mathpar}
  M^{*}_{x} | \lift{x}{P} \red M[P]
\end{mathpar}

namely,

\begin{mathpar}
  M^{*}_{x} := x?(u).M[\dropn{u}]
\end{mathpar}

The dependence of $M^{*}_{x}$ on a name makes it an abstraction, 

\begin{mathpar}
  M^{*} := (x)x?(u).M[\dropn{u}]
\end{mathpar}

\subsection{Additional notation}

It will sometimes be convenient to denote the process a name
quotes. We already have the notation $x = \quotep{P}$, but it will be
convenient to introduce an alternate notation, $\procn{x}$, when we
want to emphasize the connection to the use of the name. Note that, by
virtue of name equivalence, $\quotep{\procn{x}} \nameeq x$; so, the
notation is consistent with previous definitions.

Further, because names have structure it is possible to effect
substitutions on the basis of that structure. This means we need to
upgrade our notation for substitutions, which we accomplish by
adapting comprehension notation. Thus,

\begin{mathpar}
  P\{ y / x : x \in S \}
\end{mathpar}

is interpreted to mean the process derived from P by replacing (in a
capture-avoiding manner) each occurrence of $x$ in $S$ by $y$. For example,

\begin{mathpar}
  P\{ \quotep{\procn{x}|\procn{x}} / x : x \in \freenames{P} \}
\end{mathpar}

will replace each (occurrence) of a free name $x$ in $P$ by
$\quotep{\procn{x}|\procn{x}}$.

Also, we will avail ourselves of the notation $x^{L}$ and $x^{R}$ to
denote injections of a name into disjoint copies of the name
space. There are numerous ways to accomplish this. One example can be
found in \cite{MeredithR05}. This notation overloads to vectors of
names: $\vec{x}^{\pi} := (x_{i}^{\pi} \; : \; 0 \leq i < |\vec{x}| )$ where $\pi \in \{L,R\}$.

We also use $P^{\Box} := P|\Box$.

In \cite{MeredithR05} an interpretation of the new operator is
given. It turns out that there are several possible interpretations
all enjoying the requisite algebraic properties of the operator (see
\cite{milner91polyadicpi}). We will therefore make liberal use of
$(\nu\; \vec{x})P$.

% subsection the_syntax_and_semantics_of_the_notation_system (end)   

\input{qm2pi.qmops} 

\input{qm2pi.sterngerlach} 

\input{qm2pi.metric} 

% section concurrent_process_calculi (end)

%\input{qm2pi.proofsketch}

% section proof sketch (end)

%\input{qm2pi.slviaknots} 

% section spatial logic via knots (end)

\input{qm2pi.conclusion}

% section conclusion (end)

%\input{qm2pi.dtcodes} 

% section wiring algorithm (end)

\input{qm2pi.ack} 

% section acknowledgments (end)

\newpage


\bibliographystyle{plain}   
\bibliography{../../biblios/main.bib}

\input{qm2pi.rhodetails}

\end{document}

 

% section acknowledgments (end)

\newpage


\bibliographystyle{plain}   
\bibliography{../../biblios/main.bib}

\documentclass[12pt]{llncs}
%\documentclass{jktr}

\usepackage[pdftex]{hyperref}                   
\usepackage {listings}
\usepackage {mathpartir}
\usepackage{bcprules}
%\usepackage{listings}
                       
\usepackage{graphicx} 
%\usepackage[margins=2.5cm,nohead,nofoot]{geometry}
%\usepackage{geometry}
\usepackage{amsfonts}
\usepackage{amstext}
\usepackage{latexsym}
\usepackage{amssymb}
\usepackage{color}


%\include{myPreamble}
\include{qm2pi.local} 

%\ifpdf
%\usepackage[pdftex]{graphicx}
%\else
%\usepackage{graphicx}
%\fi

 % \ifpdf
%  \usepackage{pdfsync}
%  \if


%\title{Brief Article}
%\author{David F. Snyder}
%\author{L.G. Meredith}

%\address{Dept. of Math., Texas State University--San Marcos, San Marcos, TX 78666}
       
\pagestyle{empty}


\begin{document}

\lstset{language=[Objective]Caml,frame=shadowbox}

\input{qm2pi.front}

% section front matter (end)

\input{qm2pi.intro} 
 
% section introduction (end)

% \input{qm2pi.knotations} 

% section notation (end)

\input{qm2pi.process.calculi} 

% section concurrent_process_calculi_and_spatial_logics_ (end)
    
%\input{qm2pi.knots2pi} 

%\input{qm2pi.trefoil} 

%\input{qm2pi.mainthm} 

% subsection basic_interpretation (end)

%\input{qm2pi.rho.presentation} 
\subsection{The syntax and semantics of the notation system}\label{sub:the_syntax_and_semantics_of_the_notation_system} % (fold)

We now summarize a technical presentation of the calculus that
embodies our theory of dynamics. The typical presentation of such a
calculus follows the style of giving generators and relations on
them. The grammar, below, describing term constructors, freely
generates the set of processes, $\Proc$. This set is then quotiented
by a relation known as structural congruence and it is over this set
that the notion of dynamics is expressed. This presentation is
essentially that of \cite{MeredithR05} with the addition of
polyadicity and summation. For readability we have relegated some of
the technical subtleties to an appendix.

\subsubsection{Process grammar}\label{subsub:process_grammar}

\begin{mathpar}
  \inferrule* [lab=synchronization] {} {{M} \bc \pzero \;|\; x?F \;|\; x!C }
  \and
  \inferrule* [lab=abstraction] {} {{F} \bc (x)P}
  \and
  \inferrule* [lab=concretion] {} {{C} \bc \langle Q \rangle}
  \and
  \inferrule* [lab=process] {} {{P,Q} \bc M \;| \;P|Q \;|\; @{x}}
  \and
  \inferrule* [lab=name] {} {{x} \bc \quotep{P}}
\end{mathpar} 

Note that $\vec{x}$ (resp. $\vec{P}$) denotes a vector of names
(resp. processes) of length $|\vec{x}|$ (resp. $|\vec{P}|$). We adopt
the following useful abbreviations.

\begin{mathpar}
   x?(\vec{y}).P := x.(\vec{y})P \and  x\clift{\vec{P}} := x.\clift{\vec{P}}
   \and x!(y) := \lift{x}{\dropn{y}}
   \and \Pi_{i=0}^{n-1}P_i := P_0 | \ldots | P_{n-1}
\end{mathpar}

\subsubsection{Structural congruence}

\paragraph{Free and bound names and alpha-equivalence.} At the
core of structural equivalence is alpha-equivalence which identifies
process that are the same up to a change of variable. Formally, we
recognize the distinction between free and bound names. The free names
of a process, $\freenames{P}$, may be calculated recursively as
follows:

\begin{mathpar}
\freenames{\pzero} := \emptyset
  \and \\
  \freenames{x?(y).P} := \{ x \} \cup (\freenames{P} \setminus \{ y \})
  \and 
  \freenames{x!\langle P \rangle} := \{ x \} \cup \{ P \} 
  \and \\
  \freenames{P|Q} := \freenames{P} \cup \freenames{Q}
  \and \\
  \freenames{@{x}} := \{ x \}
\end{mathpar}

$\pi$
$\quotep{\pi}$

$\freenames{-} : \pi \to \mathcal{P}(\quotep{\pi})$

\begin{eqnarray*}
  \freenames{\pzero} & := & \emptyset \\
  \freenames{x?(y).P} & := & \{ x \} \cup (\freenames{P} \setminus \{ y \}) \\
  \freenames{x!\langle P \rangle} & := & \{ x \} \cup \{ P \} \\
  \freenames{P|Q} & := & \freenames{P} \cup \freenames{Q} \\
  \freenames{\dropn{x}} & := & \{ x \}
\end{eqnarray*}

The bound names of a process, $\boundnames{P}$, are those names occurring in $P$
that are not free. For example, in $x?(y).0$, the name $x$ is free, while $y$ is bound.

\begin{mathpar}
  \inferrule* [lab=monoidal-laws] {} { P|Q \equiv Q|P \and P|0 \equiv P \and P|(Q|R) \equiv (P|Q)|R }
\end{mathpar}

\begin{mathpar}
  \inferrule* [lab=alpha-equivalence] {} { (x)P \equiv (y)P\{y/x\} \and y \not\in \freenames{P} }
\end{mathpar}

\begin{definition}
Then two processes, $P,Q$, are alpha-equivalent if $P = Q\{\vec{y}/\vec{x}\}$ for
some $\vec{x} \in \boundnames{Q},\vec{y} \in \boundnames{P}$, where $Q\{\vec{y}/\vec{x}\}$
denotes the capture-avoiding substitution of $\vec{y}$ for $\vec{x}$ in $Q$.
\end{definition}

\begin{definition}
  The {\em structural congruence} \cite{SangiorgiWalker} , $\equiv$,
  between processes is the least congruence containing
  alpha-equivalence, satisfying the abelian monoid laws
  (associativity, commutativity and $\pzero$ as identity) for parallel
  composition $|$ and for summation $+$.
\end{definition}

\subsection{Name equivalence}

We take name equivalence, written $\nameeq$, to be the smallest
equivalence relation generated by the following rules.

\begin{mathpar}
\inferrule*[lab=Quote-drop]
{ }
{ \quotep{@{x}} \nameeq x }

\inferrule*[lab=Struct-equiv]
{ P \scong Q }
{ \quotep{P} \nameeq \quotep{Q} }
\end{mathpar}

The astute reader will have noticed that the mutual recursion of names
and processes imposes a mutual recursion on alpha-equivalence and
structural equivalence via name-equivalence. Fortunately, all of this
works out pleasantly and we may calculate in the natural way, free of
concern. The reader interested in the details is referred to the
appendix \ref{appendix:rho_details}.

\subsection{Substitution}

We use $\Proc$ for the set of processes, $\QProc$ for the set of
names, and $\id{\{}\vec{y} / \vec{x} \id{\}}$ to denote partial maps,
$s : \QProc \rightarrow \QProc$. A map, $s$ lifts, uniquely, to a map
on process terms, $\widehat{s} : \Proc \rightarrow \Proc$ by the
following equations.

\begin{mathpar}
  (0) \psubstp{Q}{P} := 0 \\
  (R \juxtap S) \psubstp{Q}{P}
  :=    
  (R)\psubstp{Q}{P} \juxtap (S) \psubstp{Q}{P} \\
  (x?(y).R) \psubstp{Q}{P}    
  :=    
  (x)\substp{Q}{P} (z)\concat( (R \psubstn{z}{y}) \psubstp{Q}{P} ) \\
  (\lift{x}{R}) \psubstp{Q}{P}  
  :=
  \lift{(x)\substp{Q}{P}}{ R \psubstp{Q}{P} } \\
%   (\dropn{x})  \psubstp{Q}{P}       
%   := 
%   \left\{ 
%     \begin{array}{ccc} 
%       \dropn{\quotep{Q}} & & x \nameeq \quotep{P} \\
%       \dropn{x} & & otherwise \\
%     \end{array}
%   \right. 
  (\dropn{x})  \psubstp{Q}{P}       
  := 
  \left\{ 
    \begin{array}{ccc} 
      Q & & x \nameeq \quotep{P} \\
      \dropn{x} & & otherwise \\
    \end{array}
  \right.
\end{mathpar}
 

where

\begin{eqnarray}
  (x)\id{\{} \lpquote Q \rpquote / \lpquote P \rpquote \id{\}}            = 
  \left\{ 
    \begin{array}{ccc}
      \lpquote Q \rpquote & & x \nameeq \lpquote P \rpquote \\
      x & & otherwise \\
    \end{array}
  \right. \nonumber
\end{eqnarray}

and $z$ is chosen distinct from $\quotep{P}$, $\quotep{Q}$, the free
names in $Q$, and all the names in $R$. Our $\alpha$-equivalence will
be built in the standard way from this substitution.

\begin{remark}\label{rem:no_self_referential_names}
  One consequence of these definitions is that $\forall P. \quotep{P}
  \not\in \freenames{P}$.
\end{remark}

\subsection{ Dynamic quote: an example }

Anticipating something of what's to come, consider applying the
substitution, $\widehat{\id{\{}u / z \id{\}}}$, to the following pair
of processes, $\lift{w}{y!(z)}$ and $w[ \lpquote y!(z) \rpquote ]$.

\begin{eqnarray}
	\lift{w}{y!(z)}\widehat{\id{\{}u / z \id{\}}}
		& = &
		\lift{w}{y!(u)} \nonumber\\
	w[ \lpquote y!(z) \rpquote ] \widehat{ \id{\{}u / z \id{\}} }
		& = &
		w[ \lpquote y!(z) \rpquote ] \nonumber
\end{eqnarray}

Because the body of the process between quotes is impervious to
substitution, we get radically different answers. In fact, by
examining the first process in an input context,
e.g. $x?(z).\lift{w}{y!(z)}$, we see that the process under the lift
operator may be shaped by prefixed inputs binding a name inside it. In
this sense, the lift operator will be seen as a way to dynamically
construct processes before reifying them as names.

Finally equipped with these standard features we can present the
dynamics of the calculus.

\subsubsection{Operational semantics} 

Finally, we introduce the computational dynamics. What marks these
algebras as distinct from other more traditionally studied algebraic
structures, e.g. vector spaces or polynomial rings, is the manner in
which dynamics is captured. In traditional structures, dynamics is typically
expressed through morphisms between such structures, as in linear maps
between vector spaces or morphisms between rings. In algebras
associated with the semantics of computation, the dynamics is
expressed as part of the algebraic structure itself, through a
reduction reduction relation typically denoted by $\red$. Below, we
give a recursive presentation of this relation for the calculus used
in the encoding.

$\red \subseteq \pi \times \pi$
$\red : \pi \to \mathcal{P}(\pi)$

\begin{mathpar}
  \inferrule* [lab=Comm] { \textsf{match}( x_{src}, x_{trgt} ) } { x_{trgt}?(y)P \; | \; x_{src}!\langle {Q} \rangle \red P\{\quotep{Q}/y}\} }
  \and \\
  \inferrule* [lab=Par] {{P} \red {P}'} {{{P} | {Q}} \red {{P}' | {Q}}}
  \and
  \inferrule* [lab=Equiv]{{{P} \scong {P}'} \andalso {{P}' \red {Q}'} \andalso {{Q}' \scong {Q}}}{{P} \red {Q}}
\end{mathpar}

\begin{eqnarray*}
  match_{\equiv} (\quotep{P},\quotep{Q}) & := & P \equiv Q \\
  match_{\dagger}(\quotep{P},\quotep{Q}) & := & \forall R. P|Q \red^{*} R => R \red^{*} 0 \\
  match_{K}(\quotep{P},\quotep{Q}) & := & K \mbox{ for some context } K
\end{eqnarray*}

$u?(x)P | u!\langle Q \rangle \red P\{\quotep{Q}/x\}$

%We write $\wred$ for $\red^*$, and $P\red$ if $\exists Q $ such that $ P \red Q$.
We write $P\red$ if $\exists Q $ such that $ P \red Q$ and $P\not\red$, otherwise.

\section{Replication}

As mentioned before, it is known that replication (and hence
recursion) can be implemented in a higher-order process algebra
\cite{SangiorgiWalker}. As our first example of calculation with the
machinery thus far presented we give the construction explicitly in
the {\rhoc}.

\begin{eqnarray}
	D_{x} & := & \prefix{x}{y}{(\binpar{\outputp{x}{y}}{@{y}})} \nonumber\\
	\bangp_{x}{P} & := & \binpar{{x}!\langle{\binpar{D_{x}}{P}}\rangle}{D_{x}} \nonumber
\end{eqnarray}

\begin{eqnarray}
	\bangp_{x}{P} & & \nonumber\\
	=
	& {x}!\langle{(\prefix{x}{y}{(\outputp{x}{y} | @{y})) | P}}\rangle 
	      | \prefix{x}{y}{(\outputp{x}{y} | @{y})} & \nonumber\\
	\red
	& (\outputp{x}{y} | @{y})\substn{\quotep{(\prefix{x}{y}{(@{y} | \outputp{x}{y})) | P}}}{y} & \nonumber\\
	=
	& \outputp{x}{\quotep{(\prefix{x}{y}{(\outputp{x}{y} | @{y})) | P}}}
	  | {(\prefix{x}{y}{(\outputp{x}{y} | @{y})) | P}} & \nonumber\\
	\red
	& \ldots & \nonumber\\
	\red^*
	& P | P | \ldots & \nonumber
\end{eqnarray}

Of course, this encoding, as an implementation, runs away, unfolding
$\bangp{P}$ eagerly. A lazier and more implementable replication
operator, restricted to input-guarded processes, may be obtained as follows.

\begin{eqnarray}
\bangp{\prefix{u}{v}{P}} 
	:= 
	\binpar{\lift{x}{\prefix{u}{v}{(\binpar{D(x)}{P})}}}{D(x)} \nonumber
\end{eqnarray}

\begin{remark}
  Note that the lazier definition still does not deal with summation
  or mixed summation (i.e. sums over input and output). The reader is
  invited to construct definitions of replication that deal with these
  features. 

  Further, the definitions are parameterized in a name, $x$. Can you,
  gentle reader, make a definition that eliminates this parameter and
  guarantees no accidental interaction between the replication
  machinery and the process being replicated -- i.e. no accidental
  sharing of names used by the process to get its work done and the
  name(s) used by the replication to effect copying. This latter
  revision of the definition of replication is crucial to obtaining
  the expected identity $!!P \sim !P$.
\end{remark}

\begin{remark}\label{rem:paradoxical_combinator}
  The reader familiar with the lambda calculus will have noticed the
  similarity between $D$ and the paradoxical combinator.

  [Ed. note: the existence of this seems to suggest we have to be more
  restrictive on the set of processes and names we admit if we are to
  support no-cloning.]
\end{remark}

\subsubsection{Bisimulation}

The computational dynamics gives rise to another kind of equivalence,
the equivalence of computational behavior. As previously mentioned
this is typically captured \emph{via} some form of bisimulation.

% The notion we use in this paper is weak barbed bisimulation
% \cite{milner91polyadicpi}.

The notion we use in this paper is derived from weak barbed
bisimulation \cite{milner91polyadicpi}. 

\begin{definition}
An \emph{observation relation}, $\downarrow_{\mathcal N}$, over a set
of names, $\mathcal N$, is the smallest relation satisfying the rules
below.

\infrule[Out-barb]{y \in {\mathcal N}, \; x \nameeq y}
		  {\outputp{x}{v} \downarrow_{\mathcal N} x}
\infrule[Par-barb]{\mbox{$P\downarrow_{\mathcal N} x$ or $Q\downarrow_{\mathcal N} x$}}
		  {\binpar{P}{Q} \downarrow_{\mathcal N} x}

We write $P \Downarrow_{\mathcal N} x$ if there is $Q$ such that 
$P \wred Q$ and $Q \downarrow_{\mathcal N} x$.
\end{definition}

\begin{definition}
%\label{def.bbisim}
An  ${\mathcal N}$-\emph{barbed bisimulation} over a set of names, ${\mathcal N}$, is a symmetric binary relation 
${\mathcal S}_{\mathcal N}$ between agents such that $P\rel{S}_{\mathcal N}Q$ implies:
\begin{enumerate}
\item If $P \red P'$ then $Q \wred Q'$ and $P'\rel{S}_{\mathcal N} Q'$.
\item If $P\downarrow_{\mathcal N} x$, then $Q\Downarrow_{\mathcal N} x$.
\end{enumerate}
$P$ is ${\mathcal N}$-barbed bisimilar to $Q$, written
$P \wbbisim_{\mathcal N} Q$, if $P \rel{S}_{\mathcal N} Q$ for some ${\mathcal N}$-barbed bisimulation ${\mathcal S}_{\mathcal N}$.
\end{definition}

$\mathcal{R} \subseteq \pi \times \pi$

$P \mathcal{R} Q => \forall P'. P \red P' \Rightarrow \exists Q'. Q \red Q', P' \mathcal{R} Q'$

$P \vdash x \Rightarrow Q \vdash x$

\begin{mathpar}
  \inferrule*[lab=Out-barb]{x \nameeq y}{{y}!\langle{Q}\rangle \vdash x}
  \and
  \inferrule*[lab=Par-barb]{\mbox{$P\vdash x$ or $Q\vdash x$}}{\binpar{P}{Q} \vdash x}
\end{mathpar}

\subsubsection{Contexts}

One of the principle advantages of computational calculi like the
$\pi$-calculus is a well-defined notion of context,
contextual-equivalence and a correlation between
contextual-equivalence and notions of bisimulation. The notion of
context allows the decomposition of a process into (sub-)process and
its syntactic environment, its context. Thus, a context may be
thought of as a process with a ``hole'' (written $\Box$) in it. The
application of a context $M$ to a process $P$, written $M[P]$, is
tantamount to filling the hole in $M$ with $P$. In this paper we do
not need the full weight of this theory, but do make use of the notion
of context in the proof the main theorem. 

\begin{mathpar}
  \inferrule* [lab=summation] {} {{M_{M},M_{N}} \bc \Box \;|\; x.M_{A} \;|\; M_{M}+M_{N}}
  \and
  \inferrule* [lab=agent] {} {{M_{A}} \bc (\vec{x})M_{P} \;| \; \clift{P_0,\ldots,M_{P},\ldots,P_N}}
  \and \\
  \inferrule* [lab=process] {} {{M_{P}} \bc M_{N} \;| \;P|M_{P} }
\end{mathpar} 

\begin{mathpar}
  \inferrule* [lab=sychronization] {} {M_{N} \bc \Box \;|\; x?M_{F} \;|\; x!M_{C}}
  \and
  \inferrule* [lab=abstraction] {} {{M_{F}} \bc (x)M_{P} }
  \and
  \inferrule* [lab=concretion] {} {{M_{C}} \bc \langle M_{P} \rangle }
  \and \\
  \inferrule* [lab=process] {} {{M_{P}} \bc M_{N} \;| \;P|M_{P} }
\end{mathpar}

\begin{definition}[contextual application] Given a context $M$, and
  process $P$, we define the \emph{contextual application}, $M[P] :=
  M\{P/\Box\}$. That is, the contextual application of M to P is the
  substitution of $P$ for $\Box$ in $M$.
\end{definition}

$\meaningof{-} : L \to \mathcal{P}(\pi)$

\begin{mathpar}
  \inferrule* [lab=collection] {} {\meaningof{true} = \pi, \and \meaningof{~E} = \pi \setminus \meaningof{E}, \and \meaningof{E_{1} \& E_{2}} = \meaningof{E_{1}} \cap \meaningof{E_{2}}}
\end{mathpar}

\begin{mathpar}
  \inferrule* [lab=structure] {} {\meaningof{0} = \{ P \in \pi | P \equiv 0 \}, \and \\ \meaningof{E_1 | E_2} = \{ P \in \pi | P \equiv P_{1} | P_{2}, P_{1} \in \meaningof{E_{1}}, P_{2} \in \meaningof{E_2}\} }
\end{mathpar}

\begin{mathpar}
 \inferrule* [lab=behavior] {} {\meaningof{\langle a?b \rangle E} = \{ P \in \pi | P \equiv Q | u?(y)P', \\ \and \\\\ \and \\ \;\;\; u \in \meaningof{a}, \forall z.P'\{z/y\} \in \meaningof{E\{z/b\}}\}, \and \\ \meaningof{a!E} = \{ P \in \pi | P \equiv Q | x!\langle P' \rangle, x \in \meaningof{a} P' \in \meaningof{E}\} }
\end{mathpar}

\begin{mathpar}
 \inferrule* [lab=nominal] {} {\meaningof{\quotep{E}} = \{ \quotep{P} \in \quotep{\pi} | P \in \meaningof{E} \}, \and \meaningof{\quotep{P}} = \{ \quotep{Q} \in \quotep{\pi} | P \equiv Q \} \and \\ \meaningof{@\quotep{E}} = \{ P \in \pi | P \equiv @x, x \in \meaningof{E} \}}
\end{mathpar}

\begin{eqnarray*}
  \\
  \meaningof{-} : TS \to ST
\end{eqnarray*}

\begin{eqnarray*}
  \\
  L : TS \to ST
\end{eqnarray*}

\begin{eqnarray*}
  \\
  P \models E \iff P \in \meaningof{E}
\end{eqnarray*}

\begin{eqnarray*}
  P \approx_{L} Q \iff \forall E \in L. P \models E \iff Q \models E
\end{eqnarray*}

\begin{eqnarray*}
  P \approx_{K} Q
\end{eqnarray*}

\begin{eqnarray*}
  P \approx Q
\end{eqnarray*}

$\approx_{K} = \approx = \approx_{L}$

\subsubsection{Contextual duality}

Note that contexts extend the quotation operation to a family of
operations from processes to names. Given a context, $M$, we can
define a \emph{nominal context}, $\quotep{M}$ by $\quotep{M}[P] :=
\quotep{M[P]}$. To foreshadow what is to come we observe that these
operations enjoy a duality with processes very much like the duality
between vectors and maps from vectors to scalars.

Further, because the calculus is essentially higher-order, we have a
correspondence between contexts and processes. More specifically,
given a name $x$ and a context $M$ we can construct $M^{*}_{x}$ such
that 

\begin{mathpar}
  M^{*}_{x} | \lift{x}{P} \red M[P]
\end{mathpar}

namely,

\begin{mathpar}
  M^{*}_{x} := x?(u).M[\dropn{u}]
\end{mathpar}

The dependence of $M^{*}_{x}$ on a name makes it an abstraction, 

\begin{mathpar}
  M^{*} := (x)x?(u).M[\dropn{u}]
\end{mathpar}

\subsection{Additional notation}

It will sometimes be convenient to denote the process a name
quotes. We already have the notation $x = \quotep{P}$, but it will be
convenient to introduce an alternate notation, $\procn{x}$, when we
want to emphasize the connection to the use of the name. Note that, by
virtue of name equivalence, $\quotep{\procn{x}} \nameeq x$; so, the
notation is consistent with previous definitions.

Further, because names have structure it is possible to effect
substitutions on the basis of that structure. This means we need to
upgrade our notation for substitutions, which we accomplish by
adapting comprehension notation. Thus,

\begin{mathpar}
  P\{ y / x : x \in S \}
\end{mathpar}

is interpreted to mean the process derived from P by replacing (in a
capture-avoiding manner) each occurrence of $x$ in $S$ by $y$. For example,

\begin{mathpar}
  P\{ \quotep{\procn{x}|\procn{x}} / x : x \in \freenames{P} \}
\end{mathpar}

will replace each (occurrence) of a free name $x$ in $P$ by
$\quotep{\procn{x}|\procn{x}}$.

Also, we will avail ourselves of the notation $x^{L}$ and $x^{R}$ to
denote injections of a name into disjoint copies of the name
space. There are numerous ways to accomplish this. One example can be
found in \cite{MeredithR05}. This notation overloads to vectors of
names: $\vec{x}^{\pi} := (x_{i}^{\pi} \; : \; 0 \leq i < |\vec{x}| )$ where $\pi \in \{L,R\}$.

We also use $P^{\Box} := P|\Box$.

In \cite{MeredithR05} an interpretation of the new operator is
given. It turns out that there are several possible interpretations
all enjoying the requisite algebraic properties of the operator (see
\cite{milner91polyadicpi}). We will therefore make liberal use of
$(\nu\; \vec{x})P$.

% subsection the_syntax_and_semantics_of_the_notation_system (end)   

\input{qm2pi.qmops} 

\input{qm2pi.sterngerlach} 

\input{qm2pi.metric} 

% section concurrent_process_calculi (end)

%\input{qm2pi.proofsketch}

% section proof sketch (end)

%\input{qm2pi.slviaknots} 

% section spatial logic via knots (end)

\input{qm2pi.conclusion}

% section conclusion (end)

%\input{qm2pi.dtcodes} 

% section wiring algorithm (end)

\input{qm2pi.ack} 

% section acknowledgments (end)

\newpage


\bibliographystyle{plain}   
\bibliography{../../biblios/main.bib}

\input{qm2pi.rhodetails}

\end{document}



\end{document}

 

% subsection basic_interpretation (end)

%\input{qm2pi.rho.presentation} 
\subsection{The syntax and semantics of the notation system}\label{sub:the_syntax_and_semantics_of_the_notation_system} % (fold)

We now summarize a technical presentation of the calculus that
embodies our theory of dynamics. The typical presentation of such a
calculus follows the style of giving generators and relations on
them. The grammar, below, describing term constructors, freely
generates the set of processes, $\Proc$. This set is then quotiented
by a relation known as structural congruence and it is over this set
that the notion of dynamics is expressed. This presentation is
essentially that of \cite{MeredithR05} with the addition of
polyadicity and summation. For readability we have relegated some of
the technical subtleties to an appendix.

\subsubsection{Process grammar}\label{subsub:process_grammar}

\begin{mathpar}
  \inferrule* [lab=synchronization] {} {{M} \bc \pzero \;|\; x?F \;|\; x!C }
  \and
  \inferrule* [lab=abstraction] {} {{F} \bc (x)P}
  \and
  \inferrule* [lab=concretion] {} {{C} \bc \langle Q \rangle}
  \and
  \inferrule* [lab=process] {} {{P,Q} \bc M \;| \;P|Q \;|\; @{x}}
  \and
  \inferrule* [lab=name] {} {{x} \bc \quotep{P}}
\end{mathpar} 

Note that $\vec{x}$ (resp. $\vec{P}$) denotes a vector of names
(resp. processes) of length $|\vec{x}|$ (resp. $|\vec{P}|$). We adopt
the following useful abbreviations.

\begin{mathpar}
   x?(\vec{y}).P := x.(\vec{y})P \and  x\clift{\vec{P}} := x.\clift{\vec{P}}
   \and x!(y) := \lift{x}{\dropn{y}}
   \and \Pi_{i=0}^{n-1}P_i := P_0 | \ldots | P_{n-1}
\end{mathpar}

\subsubsection{Structural congruence}

\paragraph{Free and bound names and alpha-equivalence.} At the
core of structural equivalence is alpha-equivalence which identifies
process that are the same up to a change of variable. Formally, we
recognize the distinction between free and bound names. The free names
of a process, $\freenames{P}$, may be calculated recursively as
follows:

\begin{mathpar}
\freenames{\pzero} := \emptyset
  \and \\
  \freenames{x?(y).P} := \{ x \} \cup (\freenames{P} \setminus \{ y \})
  \and 
  \freenames{x!\langle P \rangle} := \{ x \} \cup \{ P \} 
  \and \\
  \freenames{P|Q} := \freenames{P} \cup \freenames{Q}
  \and \\
  \freenames{@{x}} := \{ x \}
\end{mathpar}

$\pi$
$\quotep{\pi}$

$\freenames{-} : \pi \to \mathcal{P}(\quotep{\pi})$

\begin{eqnarray*}
  \freenames{\pzero} & := & \emptyset \\
  \freenames{x?(y).P} & := & \{ x \} \cup (\freenames{P} \setminus \{ y \}) \\
  \freenames{x!\langle P \rangle} & := & \{ x \} \cup \{ P \} \\
  \freenames{P|Q} & := & \freenames{P} \cup \freenames{Q} \\
  \freenames{\dropn{x}} & := & \{ x \}
\end{eqnarray*}

The bound names of a process, $\boundnames{P}$, are those names occurring in $P$
that are not free. For example, in $x?(y).0$, the name $x$ is free, while $y$ is bound.

\begin{mathpar}
  \inferrule* [lab=monoidal-laws] {} { P|Q \equiv Q|P \and P|0 \equiv P \and P|(Q|R) \equiv (P|Q)|R }
\end{mathpar}

\begin{mathpar}
  \inferrule* [lab=alpha-equivalence] {} { (x)P \equiv (y)P\{y/x\} \and y \not\in \freenames{P} }
\end{mathpar}

\begin{definition}
Then two processes, $P,Q$, are alpha-equivalent if $P = Q\{\vec{y}/\vec{x}\}$ for
some $\vec{x} \in \boundnames{Q},\vec{y} \in \boundnames{P}$, where $Q\{\vec{y}/\vec{x}\}$
denotes the capture-avoiding substitution of $\vec{y}$ for $\vec{x}$ in $Q$.
\end{definition}

\begin{definition}
  The {\em structural congruence} \cite{SangiorgiWalker} , $\equiv$,
  between processes is the least congruence containing
  alpha-equivalence, satisfying the abelian monoid laws
  (associativity, commutativity and $\pzero$ as identity) for parallel
  composition $|$ and for summation $+$.
\end{definition}

\subsection{Name equivalence}

We take name equivalence, written $\nameeq$, to be the smallest
equivalence relation generated by the following rules.

\begin{mathpar}
\inferrule*[lab=Quote-drop]
{ }
{ \quotep{@{x}} \nameeq x }

\inferrule*[lab=Struct-equiv]
{ P \scong Q }
{ \quotep{P} \nameeq \quotep{Q} }
\end{mathpar}

The astute reader will have noticed that the mutual recursion of names
and processes imposes a mutual recursion on alpha-equivalence and
structural equivalence via name-equivalence. Fortunately, all of this
works out pleasantly and we may calculate in the natural way, free of
concern. The reader interested in the details is referred to the
appendix \ref{appendix:rho_details}.

\subsection{Substitution}

We use $\Proc$ for the set of processes, $\QProc$ for the set of
names, and $\id{\{}\vec{y} / \vec{x} \id{\}}$ to denote partial maps,
$s : \QProc \rightarrow \QProc$. A map, $s$ lifts, uniquely, to a map
on process terms, $\widehat{s} : \Proc \rightarrow \Proc$ by the
following equations.

\begin{mathpar}
  (0) \psubstp{Q}{P} := 0 \\
  (R \juxtap S) \psubstp{Q}{P}
  :=    
  (R)\psubstp{Q}{P} \juxtap (S) \psubstp{Q}{P} \\
  (x?(y).R) \psubstp{Q}{P}    
  :=    
  (x)\substp{Q}{P} (z)\concat( (R \psubstn{z}{y}) \psubstp{Q}{P} ) \\
  (\lift{x}{R}) \psubstp{Q}{P}  
  :=
  \lift{(x)\substp{Q}{P}}{ R \psubstp{Q}{P} } \\
%   (\dropn{x})  \psubstp{Q}{P}       
%   := 
%   \left\{ 
%     \begin{array}{ccc} 
%       \dropn{\quotep{Q}} & & x \nameeq \quotep{P} \\
%       \dropn{x} & & otherwise \\
%     \end{array}
%   \right. 
  (\dropn{x})  \psubstp{Q}{P}       
  := 
  \left\{ 
    \begin{array}{ccc} 
      Q & & x \nameeq \quotep{P} \\
      \dropn{x} & & otherwise \\
    \end{array}
  \right.
\end{mathpar}
 

where

\begin{eqnarray}
  (x)\id{\{} \lpquote Q \rpquote / \lpquote P \rpquote \id{\}}            = 
  \left\{ 
    \begin{array}{ccc}
      \lpquote Q \rpquote & & x \nameeq \lpquote P \rpquote \\
      x & & otherwise \\
    \end{array}
  \right. \nonumber
\end{eqnarray}

and $z$ is chosen distinct from $\quotep{P}$, $\quotep{Q}$, the free
names in $Q$, and all the names in $R$. Our $\alpha$-equivalence will
be built in the standard way from this substitution.

\begin{remark}\label{rem:no_self_referential_names}
  One consequence of these definitions is that $\forall P. \quotep{P}
  \not\in \freenames{P}$.
\end{remark}

\subsection{ Dynamic quote: an example }

Anticipating something of what's to come, consider applying the
substitution, $\widehat{\id{\{}u / z \id{\}}}$, to the following pair
of processes, $\lift{w}{y!(z)}$ and $w[ \lpquote y!(z) \rpquote ]$.

\begin{eqnarray}
	\lift{w}{y!(z)}\widehat{\id{\{}u / z \id{\}}}
		& = &
		\lift{w}{y!(u)} \nonumber\\
	w[ \lpquote y!(z) \rpquote ] \widehat{ \id{\{}u / z \id{\}} }
		& = &
		w[ \lpquote y!(z) \rpquote ] \nonumber
\end{eqnarray}

Because the body of the process between quotes is impervious to
substitution, we get radically different answers. In fact, by
examining the first process in an input context,
e.g. $x?(z).\lift{w}{y!(z)}$, we see that the process under the lift
operator may be shaped by prefixed inputs binding a name inside it. In
this sense, the lift operator will be seen as a way to dynamically
construct processes before reifying them as names.

Finally equipped with these standard features we can present the
dynamics of the calculus.

\subsubsection{Operational semantics} 

Finally, we introduce the computational dynamics. What marks these
algebras as distinct from other more traditionally studied algebraic
structures, e.g. vector spaces or polynomial rings, is the manner in
which dynamics is captured. In traditional structures, dynamics is typically
expressed through morphisms between such structures, as in linear maps
between vector spaces or morphisms between rings. In algebras
associated with the semantics of computation, the dynamics is
expressed as part of the algebraic structure itself, through a
reduction reduction relation typically denoted by $\red$. Below, we
give a recursive presentation of this relation for the calculus used
in the encoding.

$\red \subseteq \pi \times \pi$
$\red : \pi \to \mathcal{P}(\pi)$

\begin{mathpar}
  \inferrule* [lab=Comm] { \textsf{match}( x_{src}, x_{trgt} ) } { x_{trgt}?(y)P \; | \; x_{src}!\langle {Q} \rangle \red P\{\quotep{Q}/y}\} }
  \and \\
  \inferrule* [lab=Par] {{P} \red {P}'} {{{P} | {Q}} \red {{P}' | {Q}}}
  \and
  \inferrule* [lab=Equiv]{{{P} \scong {P}'} \andalso {{P}' \red {Q}'} \andalso {{Q}' \scong {Q}}}{{P} \red {Q}}
\end{mathpar}

\begin{eqnarray*}
  match_{\equiv} (\quotep{P},\quotep{Q}) & := & P \equiv Q \\
  match_{\dagger}(\quotep{P},\quotep{Q}) & := & \forall R. P|Q \red^{*} R => R \red^{*} 0 \\
  match_{K}(\quotep{P},\quotep{Q}) & := & K \mbox{ for some context } K
\end{eqnarray*}

$u?(x)P | u!\langle Q \rangle \red P\{\quotep{Q}/x\}$

%We write $\wred$ for $\red^*$, and $P\red$ if $\exists Q $ such that $ P \red Q$.
We write $P\red$ if $\exists Q $ such that $ P \red Q$ and $P\not\red$, otherwise.

\section{Replication}

As mentioned before, it is known that replication (and hence
recursion) can be implemented in a higher-order process algebra
\cite{SangiorgiWalker}. As our first example of calculation with the
machinery thus far presented we give the construction explicitly in
the {\rhoc}.

\begin{eqnarray}
	D_{x} & := & \prefix{x}{y}{(\binpar{\outputp{x}{y}}{@{y}})} \nonumber\\
	\bangp_{x}{P} & := & \binpar{{x}!\langle{\binpar{D_{x}}{P}}\rangle}{D_{x}} \nonumber
\end{eqnarray}

\begin{eqnarray}
	\bangp_{x}{P} & & \nonumber\\
	=
	& {x}!\langle{(\prefix{x}{y}{(\outputp{x}{y} | @{y})) | P}}\rangle 
	      | \prefix{x}{y}{(\outputp{x}{y} | @{y})} & \nonumber\\
	\red
	& (\outputp{x}{y} | @{y})\substn{\quotep{(\prefix{x}{y}{(@{y} | \outputp{x}{y})) | P}}}{y} & \nonumber\\
	=
	& \outputp{x}{\quotep{(\prefix{x}{y}{(\outputp{x}{y} | @{y})) | P}}}
	  | {(\prefix{x}{y}{(\outputp{x}{y} | @{y})) | P}} & \nonumber\\
	\red
	& \ldots & \nonumber\\
	\red^*
	& P | P | \ldots & \nonumber
\end{eqnarray}

Of course, this encoding, as an implementation, runs away, unfolding
$\bangp{P}$ eagerly. A lazier and more implementable replication
operator, restricted to input-guarded processes, may be obtained as follows.

\begin{eqnarray}
\bangp{\prefix{u}{v}{P}} 
	:= 
	\binpar{\lift{x}{\prefix{u}{v}{(\binpar{D(x)}{P})}}}{D(x)} \nonumber
\end{eqnarray}

\begin{remark}
  Note that the lazier definition still does not deal with summation
  or mixed summation (i.e. sums over input and output). The reader is
  invited to construct definitions of replication that deal with these
  features. 

  Further, the definitions are parameterized in a name, $x$. Can you,
  gentle reader, make a definition that eliminates this parameter and
  guarantees no accidental interaction between the replication
  machinery and the process being replicated -- i.e. no accidental
  sharing of names used by the process to get its work done and the
  name(s) used by the replication to effect copying. This latter
  revision of the definition of replication is crucial to obtaining
  the expected identity $!!P \sim !P$.
\end{remark}

\begin{remark}\label{rem:paradoxical_combinator}
  The reader familiar with the lambda calculus will have noticed the
  similarity between $D$ and the paradoxical combinator.

  [Ed. note: the existence of this seems to suggest we have to be more
  restrictive on the set of processes and names we admit if we are to
  support no-cloning.]
\end{remark}

\subsubsection{Bisimulation}

The computational dynamics gives rise to another kind of equivalence,
the equivalence of computational behavior. As previously mentioned
this is typically captured \emph{via} some form of bisimulation.

% The notion we use in this paper is weak barbed bisimulation
% \cite{milner91polyadicpi}.

The notion we use in this paper is derived from weak barbed
bisimulation \cite{milner91polyadicpi}. 

\begin{definition}
An \emph{observation relation}, $\downarrow_{\mathcal N}$, over a set
of names, $\mathcal N$, is the smallest relation satisfying the rules
below.

\infrule[Out-barb]{y \in {\mathcal N}, \; x \nameeq y}
		  {\outputp{x}{v} \downarrow_{\mathcal N} x}
\infrule[Par-barb]{\mbox{$P\downarrow_{\mathcal N} x$ or $Q\downarrow_{\mathcal N} x$}}
		  {\binpar{P}{Q} \downarrow_{\mathcal N} x}

We write $P \Downarrow_{\mathcal N} x$ if there is $Q$ such that 
$P \wred Q$ and $Q \downarrow_{\mathcal N} x$.
\end{definition}

\begin{definition}
%\label{def.bbisim}
An  ${\mathcal N}$-\emph{barbed bisimulation} over a set of names, ${\mathcal N}$, is a symmetric binary relation 
${\mathcal S}_{\mathcal N}$ between agents such that $P\rel{S}_{\mathcal N}Q$ implies:
\begin{enumerate}
\item If $P \red P'$ then $Q \wred Q'$ and $P'\rel{S}_{\mathcal N} Q'$.
\item If $P\downarrow_{\mathcal N} x$, then $Q\Downarrow_{\mathcal N} x$.
\end{enumerate}
$P$ is ${\mathcal N}$-barbed bisimilar to $Q$, written
$P \wbbisim_{\mathcal N} Q$, if $P \rel{S}_{\mathcal N} Q$ for some ${\mathcal N}$-barbed bisimulation ${\mathcal S}_{\mathcal N}$.
\end{definition}

$\mathcal{R} \subseteq \pi \times \pi$

$P \mathcal{R} Q => \forall P'. P \red P' \Rightarrow \exists Q'. Q \red Q', P' \mathcal{R} Q'$

$P \vdash x \Rightarrow Q \vdash x$

\begin{mathpar}
  \inferrule*[lab=Out-barb]{x \nameeq y}{{y}!\langle{Q}\rangle \vdash x}
  \and
  \inferrule*[lab=Par-barb]{\mbox{$P\vdash x$ or $Q\vdash x$}}{\binpar{P}{Q} \vdash x}
\end{mathpar}

\subsubsection{Contexts}

One of the principle advantages of computational calculi like the
$\pi$-calculus is a well-defined notion of context,
contextual-equivalence and a correlation between
contextual-equivalence and notions of bisimulation. The notion of
context allows the decomposition of a process into (sub-)process and
its syntactic environment, its context. Thus, a context may be
thought of as a process with a ``hole'' (written $\Box$) in it. The
application of a context $M$ to a process $P$, written $M[P]$, is
tantamount to filling the hole in $M$ with $P$. In this paper we do
not need the full weight of this theory, but do make use of the notion
of context in the proof the main theorem. 

\begin{mathpar}
  \inferrule* [lab=summation] {} {{M_{M},M_{N}} \bc \Box \;|\; x.M_{A} \;|\; M_{M}+M_{N}}
  \and
  \inferrule* [lab=agent] {} {{M_{A}} \bc (\vec{x})M_{P} \;| \; \clift{P_0,\ldots,M_{P},\ldots,P_N}}
  \and \\
  \inferrule* [lab=process] {} {{M_{P}} \bc M_{N} \;| \;P|M_{P} }
\end{mathpar} 

\begin{mathpar}
  \inferrule* [lab=sychronization] {} {M_{N} \bc \Box \;|\; x?M_{F} \;|\; x!M_{C}}
  \and
  \inferrule* [lab=abstraction] {} {{M_{F}} \bc (x)M_{P} }
  \and
  \inferrule* [lab=concretion] {} {{M_{C}} \bc \langle M_{P} \rangle }
  \and \\
  \inferrule* [lab=process] {} {{M_{P}} \bc M_{N} \;| \;P|M_{P} }
\end{mathpar}

\begin{definition}[contextual application] Given a context $M$, and
  process $P$, we define the \emph{contextual application}, $M[P] :=
  M\{P/\Box\}$. That is, the contextual application of M to P is the
  substitution of $P$ for $\Box$ in $M$.
\end{definition}

$\meaningof{-} : L \to \mathcal{P}(\pi)$

\begin{mathpar}
  \inferrule* [lab=collection] {} {\meaningof{true} = \pi, \and \meaningof{~E} = \pi \setminus \meaningof{E}, \and \meaningof{E_{1} \& E_{2}} = \meaningof{E_{1}} \cap \meaningof{E_{2}}}
\end{mathpar}

\begin{mathpar}
  \inferrule* [lab=structure] {} {\meaningof{0} = \{ P \in \pi | P \equiv 0 \}, \and \\ \meaningof{E_1 | E_2} = \{ P \in \pi | P \equiv P_{1} | P_{2}, P_{1} \in \meaningof{E_{1}}, P_{2} \in \meaningof{E_2}\} }
\end{mathpar}

\begin{mathpar}
 \inferrule* [lab=behavior] {} {\meaningof{\langle a?b \rangle E} = \{ P \in \pi | P \equiv Q | u?(y)P', \\ \and \\\\ \and \\ \;\;\; u \in \meaningof{a}, \forall z.P'\{z/y\} \in \meaningof{E\{z/b\}}\}, \and \\ \meaningof{a!E} = \{ P \in \pi | P \equiv Q | x!\langle P' \rangle, x \in \meaningof{a} P' \in \meaningof{E}\} }
\end{mathpar}

\begin{mathpar}
 \inferrule* [lab=nominal] {} {\meaningof{\quotep{E}} = \{ \quotep{P} \in \quotep{\pi} | P \in \meaningof{E} \}, \and \meaningof{\quotep{P}} = \{ \quotep{Q} \in \quotep{\pi} | P \equiv Q \} \and \\ \meaningof{@\quotep{E}} = \{ P \in \pi | P \equiv @x, x \in \meaningof{E} \}}
\end{mathpar}

\begin{eqnarray*}
  \\
  \meaningof{-} : TS \to ST
\end{eqnarray*}

\begin{eqnarray*}
  \\
  L : TS \to ST
\end{eqnarray*}

\begin{eqnarray*}
  \\
  P \models E \iff P \in \meaningof{E}
\end{eqnarray*}

\begin{eqnarray*}
  P \approx_{L} Q \iff \forall E \in L. P \models E \iff Q \models E
\end{eqnarray*}

\begin{eqnarray*}
  P \approx_{K} Q
\end{eqnarray*}

\begin{eqnarray*}
  P \approx Q
\end{eqnarray*}

$\approx_{K} = \approx = \approx_{L}$

\subsubsection{Contextual duality}

Note that contexts extend the quotation operation to a family of
operations from processes to names. Given a context, $M$, we can
define a \emph{nominal context}, $\quotep{M}$ by $\quotep{M}[P] :=
\quotep{M[P]}$. To foreshadow what is to come we observe that these
operations enjoy a duality with processes very much like the duality
between vectors and maps from vectors to scalars.

Further, because the calculus is essentially higher-order, we have a
correspondence between contexts and processes. More specifically,
given a name $x$ and a context $M$ we can construct $M^{*}_{x}$ such
that 

\begin{mathpar}
  M^{*}_{x} | \lift{x}{P} \red M[P]
\end{mathpar}

namely,

\begin{mathpar}
  M^{*}_{x} := x?(u).M[\dropn{u}]
\end{mathpar}

The dependence of $M^{*}_{x}$ on a name makes it an abstraction, 

\begin{mathpar}
  M^{*} := (x)x?(u).M[\dropn{u}]
\end{mathpar}

\subsection{Additional notation}

It will sometimes be convenient to denote the process a name
quotes. We already have the notation $x = \quotep{P}$, but it will be
convenient to introduce an alternate notation, $\procn{x}$, when we
want to emphasize the connection to the use of the name. Note that, by
virtue of name equivalence, $\quotep{\procn{x}} \nameeq x$; so, the
notation is consistent with previous definitions.

Further, because names have structure it is possible to effect
substitutions on the basis of that structure. This means we need to
upgrade our notation for substitutions, which we accomplish by
adapting comprehension notation. Thus,

\begin{mathpar}
  P\{ y / x : x \in S \}
\end{mathpar}

is interpreted to mean the process derived from P by replacing (in a
capture-avoiding manner) each occurrence of $x$ in $S$ by $y$. For example,

\begin{mathpar}
  P\{ \quotep{\procn{x}|\procn{x}} / x : x \in \freenames{P} \}
\end{mathpar}

will replace each (occurrence) of a free name $x$ in $P$ by
$\quotep{\procn{x}|\procn{x}}$.

Also, we will avail ourselves of the notation $x^{L}$ and $x^{R}$ to
denote injections of a name into disjoint copies of the name
space. There are numerous ways to accomplish this. One example can be
found in \cite{MeredithR05}. This notation overloads to vectors of
names: $\vec{x}^{\pi} := (x_{i}^{\pi} \; : \; 0 \leq i < |\vec{x}| )$ where $\pi \in \{L,R\}$.

We also use $P^{\Box} := P|\Box$.

In \cite{MeredithR05} an interpretation of the new operator is
given. It turns out that there are several possible interpretations
all enjoying the requisite algebraic properties of the operator (see
\cite{milner91polyadicpi}). We will therefore make liberal use of
$(\nu\; \vec{x})P$.

% subsection the_syntax_and_semantics_of_the_notation_system (end)   

\section{Interpretation of QM}
\subsection{Supporting definitions}
\subsubsection{Multiplication}
\begin{mathpar}
  \quotep{Q} \cdot \quotep{R} := \quotep{Q|R}
  \and \\
  \quotep{Q} \cdot P := P\{ \quotep{Q|R} / \quotep{R} : \quotep{R} \in \freenames{P} \}
\end{mathpar}

\paragraph{Discussion}
The first line needs little explanation. The second line says that
each free name of the process is replaced with the multiplication of
that name by the scalar. Multiplication of a scalar (name) by a state
(process) results in a process all the names of which have been `moved
over' by parallel composition with the process the scalar
quotes. There is a subtlety that the bound names have to be
manipulated so that multiplied names aren't accidentally
captured. There are many ways to achieve this.

\begin{remark}\label{rem:multiplication_identities}
  The reader is invited to verify that for all $x,y,z \in \QProc$ and $P \in \Proc$
  \begin{mathpar}
    x \cdot \quotep{0} \equiv x 
    \and
    x \cdot y \equiv y \cdot x
    \and
    x \cdot (y \cdot z) \equiv (x \cdot y) \cdot z
    \and \\
    \quotep{0} \cdot P \equiv P
    \and \\
    x \cdot (y \cdot P) \equiv (x \cdot y) \cdot P
    \and \\
    x \cdot (P|Q) \equiv (x \cdot P) | (x \cdot Q)
    \and \\    
  \end{mathpar}
\end{remark}

\subsubsection{Tensor product}

We define a tensor product on processes by structural induction.

\paragraph{Tensor of sums} First note that all summations, including
$\pzero$ and sequence, can be written $\Sigma_{i} x_{i}.A_{i} +
\Sigma_{j} x_{j}.C_{j}$, where we have grouped input-guarded processes
together and output-guarded processes together.

Thus, we can define the tensor product of two summations, $N_{1}\otimes N_{2}$, where

\begin{mathpar}
  N_{1} := \Sigma_{i} x_{i}.A_{i} + \Sigma_{j} x_{j}.C_{j}
  \and
  N_{2} := \Sigma_{i'} y_{i'}.B_{i'} + \Sigma_{j'} y_{j'}.D_{j'} 
\end{mathpar}

as follows.

\begin{mathpar}
  \Sigma_{i} x_{i}.A_{i} + \Sigma_{j} x_{j}.C_{j} \otimes \Sigma_{i'}
  y_{i'}.B_{i'} + \Sigma_{j'} y_{j'}.D_{j'} 
  \and \\
  := \; \Sigma_{i} \Sigma_{i'} \quotep{\stackrel{\vee}{x_{i}}| \stackrel{\vee}{y_{i'}}}.(A_{i}\otimes B_{i'}) \; | \; \Sigma_{i'} \Sigma_{i} \quotep{\stackrel{\vee}{y_{i'}}|\stackrel{\vee}{x_{i}}}.(B_{i'}\otimes A_{i})
  \and
  \;\; | \;\; \Sigma_{j} \Sigma_{j'} \quotep{\stackrel{\vee}{x_{j}}|\stackrel{\vee}{y_{j'}}}.(A_{j}\otimes B_{j'}) \; | \; \Sigma_{j'} \Sigma_{j} \quotep{\stackrel{\vee}{y_{j'}}|\stackrel{\vee}{x_{j}}}.(B_{j'}\otimes A_{j})
\end{mathpar}

\begin{remark}
  Do we need to $x^{L}$ and $y^{R}$ for this construction as well?
\end{remark}

\paragraph{Tensor of parallel compositions} Next, we distribute tensor
over par.

\begin{mathpar}
  P_{1}|P_{2} \otimes Q_{1}|Q_{2} := (P_{1} \otimes Q_{1}) | (P_{1}
  \otimes Q_{2}) | (P_{2} \otimes Q_{1}) | (P_{2} \otimes Q_{2})
\end{mathpar}

\paragraph{Tensor with dropped names} We treat tensor of a
process with a dropped name as parallel composition.

\begin{mathpar}
  P \otimes \dropn{x} := P | \dropn{x}
\end{mathpar}

\paragraph{Tensor of agents}

Finally, we need to define tensor on agents. Note that the definition
of tensor on normal products only tensors inputs with inputs and
outputs with outputs. Thus, we only have to define the operation on
``homogeneous'' pairings.

\begin{mathpar}
  (\vec{x})P \otimes (\vec{y})Q
  \and \\
  := (x_{0}^{L}|y_{0}^{R},\ldots,x_{0}^{L}|y_{n}^{R},\ldots,x_{m}^{L}|y_{0}^{R},\ldots,x_{m}^{L}|y_{n}^R)(P\{ \vec{x}^{L}/\vec{x}\} \otimes Q \{ \vec{y}^{R}/\vec{y}\})
  \and \\
  \clift{\vec{P}} \otimes \clift{\vec{Q}}
  \and \\
  := \clift{P_{0}\otimes Q_{0},\ldots,P_{0}\otimes Q_{n},\ldots,P_{m}\otimes Q_{0},\ldots,P_{m}\otimes Q_{n}}
\end{mathpar}

\begin{remark}
  Observe that arities of tensored abstractions matches arities of
  tensored concretions if the original arities matched. Note also that
  the length of the arities corresponds to the increase in dimension
  we see in ordinary vector space tensor product.
\end{remark}

\begin{remark}
  Operationally, this definition distributes the tensor down to
  components ``linked'' by summation. Tensor over summation is
  intriguing in that it mixes names. Moreover, as a consequence of the
  way it mixes names we have the identities for all $x \in \QProc$ and
  $P,Q \in \Proc$

  \begin{mathpar}
    (x \cdot P) \otimes Q \equiv x \cdot (P \otimes Q) \equiv P \otimes (x \cdot Q)
    \and
    P \otimes \pzero \equiv P
  \end{mathpar}

  that the reader is invited to verify.
\end{remark}

\subsubsection{Annihilation}
\begin{mathpar}
  P^{\perp} := \{ Q | \forall R. P|Q \red^{*} R \Rightarrow R \red^{*} \pzero \}
  \and \\
  P^{\underline{\perp}} := \Sigma_{Q \in P^{\perp}} \quotep{Q}?(y).(\dropn{y}|Q) | \Sigma_{Q \in P^{\perp}} \quotep{Q}\clift{\Box}
\end{mathpar}

\paragraph{Discussion} The reader will note that $P^{\perp}$ is a
\emph{set} of processes, while $P^{\underline{\perp}}$ is a
\emph{context}. We call the set $P^{\perp}$ the \emph{annihilators} of
$P$. The parallel composition of a process in the annihilators of $P$
with $P$ will result in a process, the state space of which has all
paths eventually leading to $\pzero$. Execution may endure loops; but
under reasonable conditions of fairness (naturally guaranteed under
most notions of bisimulation) such a composite process cannot get
stuck in such a loop and will, eventually pop out and terminate.

The context $P^{\underline{\perp}}$ is ready and willing to ``take the
$P$ out of'' the process to which it is applied. It will effectively
transmit the code of the process to which it is applied to one of the
annihilators and run the process against it.

\subsubsection{Evaluation}
We fix $M$ a domain of fully abstract interpretation with an equality
coincident with bisimulation. We take $\meaningof{\cdot} : \Proc \to
M$ to be the map interpreting processes and $\nmeaningof{\cdot} : \M
\to Proc$ to be the map running the other way. Then we define

\begin{mathpar}
  \int P := \nmeaningof{\meaningof{P}}
\end{mathpar}

\paragraph{Discussion}
There are many fully abstract interpretations of Milner's
$\pi$-calculus. Any of them can be used as a basis for interpreting
the reflective calculus here. Equipped with such a domain it is
largely a matter of grinding through to check that the Yoneda
construction for the normalization-by-evaluation program can be
extended to this setting.

\begin{remark}
  The reader is invited to verify that $\int (P^{\underline{\perp}}[P]) = 0$.
\end{remark}

\subsection{Quantum mechanics}

Table \ref{tbl:core_qm_op_defns} gives the core operational definitions

\begin{table}[htp]\label{tbl:core_qm_op_defns}
  \center{
    \fbox{
      \begin{tabular}{c|c}
        quantum mechanics & process calculus \\
        \hline
        scalar & $x := \quotep{P}$ \\
        state vector & $\state{P} := P$ \\
        dual & $\state{P}^{*} := \event{P^{\underline{\perp}}} := \quotep{P^{\underline{\perp}}}[-]$ \\
        matrix & $ \Sigma_{\alpha} \state{P_{\alpha}}x_{\alpha}\event{Q_{\alpha}}$ \\
        vector addition & $\state{P} + \state{Q} := \state{P | Q}$ \\
        tensor product & $\state{P} \otimes \state{Q} := \state{P \otimes Q}$ \\
        inner product & $\innerprod{P}{Q} := \quotep{\int P^{\underline{\perp}}[Q]}$ \\
      \end{tabular}
    }
  }
  \caption{QM - operational definitions}
\end{table}

where

\begin{mathpar}
  \prmatrix{P}{Q} := \fprmatrix{P}{\quotep{\pzero}}{Q}
  \and
  \fprmatrix{P}{x}{Q} := (\state{P},x,\event{Q})
  \and
  (\fprmatrix{P}{x}{Q})(\state{R}) := x \cdot \innerprod{Q}{R} \cdot \state{P}
  \and
  (\fprmatrix{P}{x}{Q})(\event{R}) := x \cdot \innerprod{R}{P} \cdot \event{Q}
\end{mathpar}

\paragraph{Discussion}
As promised: vectors (aka states) are represented as processes; duals
as contextual duals; inner product definition should be compared with
standard inner product definition for ....

\begin{remark}
  Assuming $\int (P^{\underline{\perp}}[P]) = 0$, the reader is
  invited to verify that $(\fprmatrix{P}{x}{P})(\state{P}) = x \cdot \state{P}$.
\end{remark}

\begin{remark}
  The reader is invited to verify that $\innerprod{P}{Q}$ could
  equally well have been written $\quotep{\int \stackrel{\vee}{x}}$
  where $x = \event{P^{\underline{\perp}}}(Q)$.

  One of the motivations for this remark is that there is another way
  to factor these operations. We could package up evaluation in the dual:

  \begin{mathpar}
    \state{P}^{*} := \event{\int P^{\underline{\perp}}} := \quotep{\int P^{\underline{\perp}}}[-]
  \end{mathpar}

  and then have inner product defined by
  
  \begin{mathpar}
    \innerprod{P}{Q} := \event{P}(Q)
  \end{mathpar}

  Hopefully, experience with the calculations will provide guidance on
  the best factoring.
\end{remark}

\begin{remark}
  Assuming $\int (P^{\underline{\perp}}[P]) = 0$, the reader is
  invited to verify that $\forall P,Q. (\prmatrix{0}{Q})(\state{0}) =
  \state{0}$ and dually $(\prmatrix{P}{0})(\event{0}) = \event{0}$.
\end{remark}

\begin{remark}
  i'm a little worried that i don't (yet) have proper support for
  complex conjugacy. But, the observation above may give us a
  clue. According to Abramsky, it must be the case that the scalars
  are iso to the homset of the identity for the tensor -- which the
  observation above characterizes. 

  For now, we will simply bookmark the notion with $\overline{x}$.
\end{remark}

\subsubsection{Adjointness}

We need to give a definition of $(\cdot)^{\dagger}$ for matrices. The
obvious candidate definition is
\begin{mathpar}
(\Sigma_{\alpha}\fprmatrix{P_{\alpha}}{x_{\alpha}}{Q_{\alpha}})^{\dagger}
= \Sigma_{\alpha}\fprmatrix{(Q_{\alpha}^{\underline{\perp}})^{*}}{\overline{x}_{\alpha}}{P_{\alpha}^{\underline{\perp}}} 
\end{mathpar}

But, $(Q_{\alpha}^{\underline{\perp}})^{*}$ requires a name along
which to communicate the process to achieve the context application.

\subsubsection{Basis for a basis}
If processes label states and ``addition'' of states (a.k.a. vector
addition) is interpreted as parallel composition, what corresponds to
notions of linear independence and basis? Here, we recall that Yoshida
has developed a set of \emph{combinators} for an asynchronous verison
of Milner's $\pi$-calculus. These are a finite set of processes such
any process can be expressed as parallel composition of these
combinators together with liberal uses of the new operator and
replication. We can simply give a translation of these into the
present calculus and have reasonable expectation that the property
carries over. That is, that the resultant set allows to express all
processes via parallel composition. Note, however, that there is no
new operator or replication in this calculus. As a result, we expect
that the corresponding set is actually infinite. That is, we expect
that the space is actually infinite dimensional.

\begin{remark}
  The attentive reader may be a bit concerned. Certainly, the
  collection $S$, $K$ and $I$ is a finite set of
  combinators. Shouldn't we expect to see a finite set of combinators
  for an effectively equivalent system? i am very sympathetic to this
  critique and feel it warrants full attention. On the other hand, i
  also have in mind the following analogy. The natural numbers, as a
  monoid under addition, has exactly $1$ generator, while the natural
  numbers, as a monoid under multiplication, has countably many
  generators (the primes). We observe that the application of the
  lambda calculus is much less resource sensitive than the parallel
  composition of the $\pi$-calculus. Could it be the case that we have
  an analogy of the form
  
  \begin{mathpar}
    m + n : MN :: m*n : M|N
  \end{mathpar}

  giving a similar blow up in the set of ``primes''?  This is such a
  wonderful thought that, even if it's not true, i think it's worth
  writing down.
\end{remark}
 

\documentclass[12pt]{llncs}
%\documentclass{jktr}

\usepackage[pdftex]{hyperref}                   
\usepackage {listings}
\usepackage {mathpartir}
\usepackage{bcprules}
%\usepackage{listings}
                       
\usepackage{graphicx} 
%\usepackage[margins=2.5cm,nohead,nofoot]{geometry}
%\usepackage{geometry}
\usepackage{amsfonts}
\usepackage{amstext}
\usepackage{latexsym}
\usepackage{amssymb}
\usepackage{color}


%\include{myPreamble}
\documentclass[12pt]{llncs}
%\documentclass{jktr}

\usepackage[pdftex]{hyperref}                   
\usepackage {listings}
\usepackage {mathpartir}
\usepackage{bcprules}
%\usepackage{listings}
                       
\usepackage{graphicx} 
%\usepackage[margins=2.5cm,nohead,nofoot]{geometry}
%\usepackage{geometry}
\usepackage{amsfonts}
\usepackage{amstext}
\usepackage{latexsym}
\usepackage{amssymb}
\usepackage{color}


%\include{myPreamble}
\include{qm2pi.local} 

%\ifpdf
%\usepackage[pdftex]{graphicx}
%\else
%\usepackage{graphicx}
%\fi

 % \ifpdf
%  \usepackage{pdfsync}
%  \if


%\title{Brief Article}
%\author{David F. Snyder}
%\author{L.G. Meredith}

%\address{Dept. of Math., Texas State University--San Marcos, San Marcos, TX 78666}
       
\pagestyle{empty}


\begin{document}

\lstset{language=[Objective]Caml,frame=shadowbox}

\input{qm2pi.front}

% section front matter (end)

\input{qm2pi.intro} 
 
% section introduction (end)

% \input{qm2pi.knotations} 

% section notation (end)

\input{qm2pi.process.calculi} 

% section concurrent_process_calculi_and_spatial_logics_ (end)
    
%\input{qm2pi.knots2pi} 

%\input{qm2pi.trefoil} 

%\input{qm2pi.mainthm} 

% subsection basic_interpretation (end)

%\input{qm2pi.rho.presentation} 
\subsection{The syntax and semantics of the notation system}\label{sub:the_syntax_and_semantics_of_the_notation_system} % (fold)

We now summarize a technical presentation of the calculus that
embodies our theory of dynamics. The typical presentation of such a
calculus follows the style of giving generators and relations on
them. The grammar, below, describing term constructors, freely
generates the set of processes, $\Proc$. This set is then quotiented
by a relation known as structural congruence and it is over this set
that the notion of dynamics is expressed. This presentation is
essentially that of \cite{MeredithR05} with the addition of
polyadicity and summation. For readability we have relegated some of
the technical subtleties to an appendix.

\subsubsection{Process grammar}\label{subsub:process_grammar}

\begin{mathpar}
  \inferrule* [lab=synchronization] {} {{M} \bc \pzero \;|\; x?F \;|\; x!C }
  \and
  \inferrule* [lab=abstraction] {} {{F} \bc (x)P}
  \and
  \inferrule* [lab=concretion] {} {{C} \bc \langle Q \rangle}
  \and
  \inferrule* [lab=process] {} {{P,Q} \bc M \;| \;P|Q \;|\; @{x}}
  \and
  \inferrule* [lab=name] {} {{x} \bc \quotep{P}}
\end{mathpar} 

Note that $\vec{x}$ (resp. $\vec{P}$) denotes a vector of names
(resp. processes) of length $|\vec{x}|$ (resp. $|\vec{P}|$). We adopt
the following useful abbreviations.

\begin{mathpar}
   x?(\vec{y}).P := x.(\vec{y})P \and  x\clift{\vec{P}} := x.\clift{\vec{P}}
   \and x!(y) := \lift{x}{\dropn{y}}
   \and \Pi_{i=0}^{n-1}P_i := P_0 | \ldots | P_{n-1}
\end{mathpar}

\subsubsection{Structural congruence}

\paragraph{Free and bound names and alpha-equivalence.} At the
core of structural equivalence is alpha-equivalence which identifies
process that are the same up to a change of variable. Formally, we
recognize the distinction between free and bound names. The free names
of a process, $\freenames{P}$, may be calculated recursively as
follows:

\begin{mathpar}
\freenames{\pzero} := \emptyset
  \and \\
  \freenames{x?(y).P} := \{ x \} \cup (\freenames{P} \setminus \{ y \})
  \and 
  \freenames{x!\langle P \rangle} := \{ x \} \cup \{ P \} 
  \and \\
  \freenames{P|Q} := \freenames{P} \cup \freenames{Q}
  \and \\
  \freenames{@{x}} := \{ x \}
\end{mathpar}

$\pi$
$\quotep{\pi}$

$\freenames{-} : \pi \to \mathcal{P}(\quotep{\pi})$

\begin{eqnarray*}
  \freenames{\pzero} & := & \emptyset \\
  \freenames{x?(y).P} & := & \{ x \} \cup (\freenames{P} \setminus \{ y \}) \\
  \freenames{x!\langle P \rangle} & := & \{ x \} \cup \{ P \} \\
  \freenames{P|Q} & := & \freenames{P} \cup \freenames{Q} \\
  \freenames{\dropn{x}} & := & \{ x \}
\end{eqnarray*}

The bound names of a process, $\boundnames{P}$, are those names occurring in $P$
that are not free. For example, in $x?(y).0$, the name $x$ is free, while $y$ is bound.

\begin{mathpar}
  \inferrule* [lab=monoidal-laws] {} { P|Q \equiv Q|P \and P|0 \equiv P \and P|(Q|R) \equiv (P|Q)|R }
\end{mathpar}

\begin{mathpar}
  \inferrule* [lab=alpha-equivalence] {} { (x)P \equiv (y)P\{y/x\} \and y \not\in \freenames{P} }
\end{mathpar}

\begin{definition}
Then two processes, $P,Q$, are alpha-equivalent if $P = Q\{\vec{y}/\vec{x}\}$ for
some $\vec{x} \in \boundnames{Q},\vec{y} \in \boundnames{P}$, where $Q\{\vec{y}/\vec{x}\}$
denotes the capture-avoiding substitution of $\vec{y}$ for $\vec{x}$ in $Q$.
\end{definition}

\begin{definition}
  The {\em structural congruence} \cite{SangiorgiWalker} , $\equiv$,
  between processes is the least congruence containing
  alpha-equivalence, satisfying the abelian monoid laws
  (associativity, commutativity and $\pzero$ as identity) for parallel
  composition $|$ and for summation $+$.
\end{definition}

\subsection{Name equivalence}

We take name equivalence, written $\nameeq$, to be the smallest
equivalence relation generated by the following rules.

\begin{mathpar}
\inferrule*[lab=Quote-drop]
{ }
{ \quotep{@{x}} \nameeq x }

\inferrule*[lab=Struct-equiv]
{ P \scong Q }
{ \quotep{P} \nameeq \quotep{Q} }
\end{mathpar}

The astute reader will have noticed that the mutual recursion of names
and processes imposes a mutual recursion on alpha-equivalence and
structural equivalence via name-equivalence. Fortunately, all of this
works out pleasantly and we may calculate in the natural way, free of
concern. The reader interested in the details is referred to the
appendix \ref{appendix:rho_details}.

\subsection{Substitution}

We use $\Proc$ for the set of processes, $\QProc$ for the set of
names, and $\id{\{}\vec{y} / \vec{x} \id{\}}$ to denote partial maps,
$s : \QProc \rightarrow \QProc$. A map, $s$ lifts, uniquely, to a map
on process terms, $\widehat{s} : \Proc \rightarrow \Proc$ by the
following equations.

\begin{mathpar}
  (0) \psubstp{Q}{P} := 0 \\
  (R \juxtap S) \psubstp{Q}{P}
  :=    
  (R)\psubstp{Q}{P} \juxtap (S) \psubstp{Q}{P} \\
  (x?(y).R) \psubstp{Q}{P}    
  :=    
  (x)\substp{Q}{P} (z)\concat( (R \psubstn{z}{y}) \psubstp{Q}{P} ) \\
  (\lift{x}{R}) \psubstp{Q}{P}  
  :=
  \lift{(x)\substp{Q}{P}}{ R \psubstp{Q}{P} } \\
%   (\dropn{x})  \psubstp{Q}{P}       
%   := 
%   \left\{ 
%     \begin{array}{ccc} 
%       \dropn{\quotep{Q}} & & x \nameeq \quotep{P} \\
%       \dropn{x} & & otherwise \\
%     \end{array}
%   \right. 
  (\dropn{x})  \psubstp{Q}{P}       
  := 
  \left\{ 
    \begin{array}{ccc} 
      Q & & x \nameeq \quotep{P} \\
      \dropn{x} & & otherwise \\
    \end{array}
  \right.
\end{mathpar}
 

where

\begin{eqnarray}
  (x)\id{\{} \lpquote Q \rpquote / \lpquote P \rpquote \id{\}}            = 
  \left\{ 
    \begin{array}{ccc}
      \lpquote Q \rpquote & & x \nameeq \lpquote P \rpquote \\
      x & & otherwise \\
    \end{array}
  \right. \nonumber
\end{eqnarray}

and $z$ is chosen distinct from $\quotep{P}$, $\quotep{Q}$, the free
names in $Q$, and all the names in $R$. Our $\alpha$-equivalence will
be built in the standard way from this substitution.

\begin{remark}\label{rem:no_self_referential_names}
  One consequence of these definitions is that $\forall P. \quotep{P}
  \not\in \freenames{P}$.
\end{remark}

\subsection{ Dynamic quote: an example }

Anticipating something of what's to come, consider applying the
substitution, $\widehat{\id{\{}u / z \id{\}}}$, to the following pair
of processes, $\lift{w}{y!(z)}$ and $w[ \lpquote y!(z) \rpquote ]$.

\begin{eqnarray}
	\lift{w}{y!(z)}\widehat{\id{\{}u / z \id{\}}}
		& = &
		\lift{w}{y!(u)} \nonumber\\
	w[ \lpquote y!(z) \rpquote ] \widehat{ \id{\{}u / z \id{\}} }
		& = &
		w[ \lpquote y!(z) \rpquote ] \nonumber
\end{eqnarray}

Because the body of the process between quotes is impervious to
substitution, we get radically different answers. In fact, by
examining the first process in an input context,
e.g. $x?(z).\lift{w}{y!(z)}$, we see that the process under the lift
operator may be shaped by prefixed inputs binding a name inside it. In
this sense, the lift operator will be seen as a way to dynamically
construct processes before reifying them as names.

Finally equipped with these standard features we can present the
dynamics of the calculus.

\subsubsection{Operational semantics} 

Finally, we introduce the computational dynamics. What marks these
algebras as distinct from other more traditionally studied algebraic
structures, e.g. vector spaces or polynomial rings, is the manner in
which dynamics is captured. In traditional structures, dynamics is typically
expressed through morphisms between such structures, as in linear maps
between vector spaces or morphisms between rings. In algebras
associated with the semantics of computation, the dynamics is
expressed as part of the algebraic structure itself, through a
reduction reduction relation typically denoted by $\red$. Below, we
give a recursive presentation of this relation for the calculus used
in the encoding.

$\red \subseteq \pi \times \pi$
$\red : \pi \to \mathcal{P}(\pi)$

\begin{mathpar}
  \inferrule* [lab=Comm] { \textsf{match}( x_{src}, x_{trgt} ) } { x_{trgt}?(y)P \; | \; x_{src}!\langle {Q} \rangle \red P\{\quotep{Q}/y}\} }
  \and \\
  \inferrule* [lab=Par] {{P} \red {P}'} {{{P} | {Q}} \red {{P}' | {Q}}}
  \and
  \inferrule* [lab=Equiv]{{{P} \scong {P}'} \andalso {{P}' \red {Q}'} \andalso {{Q}' \scong {Q}}}{{P} \red {Q}}
\end{mathpar}

\begin{eqnarray*}
  match_{\equiv} (\quotep{P},\quotep{Q}) & := & P \equiv Q \\
  match_{\dagger}(\quotep{P},\quotep{Q}) & := & \forall R. P|Q \red^{*} R => R \red^{*} 0 \\
  match_{K}(\quotep{P},\quotep{Q}) & := & K \mbox{ for some context } K
\end{eqnarray*}

$u?(x)P | u!\langle Q \rangle \red P\{\quotep{Q}/x\}$

%We write $\wred$ for $\red^*$, and $P\red$ if $\exists Q $ such that $ P \red Q$.
We write $P\red$ if $\exists Q $ such that $ P \red Q$ and $P\not\red$, otherwise.

\section{Replication}

As mentioned before, it is known that replication (and hence
recursion) can be implemented in a higher-order process algebra
\cite{SangiorgiWalker}. As our first example of calculation with the
machinery thus far presented we give the construction explicitly in
the {\rhoc}.

\begin{eqnarray}
	D_{x} & := & \prefix{x}{y}{(\binpar{\outputp{x}{y}}{@{y}})} \nonumber\\
	\bangp_{x}{P} & := & \binpar{{x}!\langle{\binpar{D_{x}}{P}}\rangle}{D_{x}} \nonumber
\end{eqnarray}

\begin{eqnarray}
	\bangp_{x}{P} & & \nonumber\\
	=
	& {x}!\langle{(\prefix{x}{y}{(\outputp{x}{y} | @{y})) | P}}\rangle 
	      | \prefix{x}{y}{(\outputp{x}{y} | @{y})} & \nonumber\\
	\red
	& (\outputp{x}{y} | @{y})\substn{\quotep{(\prefix{x}{y}{(@{y} | \outputp{x}{y})) | P}}}{y} & \nonumber\\
	=
	& \outputp{x}{\quotep{(\prefix{x}{y}{(\outputp{x}{y} | @{y})) | P}}}
	  | {(\prefix{x}{y}{(\outputp{x}{y} | @{y})) | P}} & \nonumber\\
	\red
	& \ldots & \nonumber\\
	\red^*
	& P | P | \ldots & \nonumber
\end{eqnarray}

Of course, this encoding, as an implementation, runs away, unfolding
$\bangp{P}$ eagerly. A lazier and more implementable replication
operator, restricted to input-guarded processes, may be obtained as follows.

\begin{eqnarray}
\bangp{\prefix{u}{v}{P}} 
	:= 
	\binpar{\lift{x}{\prefix{u}{v}{(\binpar{D(x)}{P})}}}{D(x)} \nonumber
\end{eqnarray}

\begin{remark}
  Note that the lazier definition still does not deal with summation
  or mixed summation (i.e. sums over input and output). The reader is
  invited to construct definitions of replication that deal with these
  features. 

  Further, the definitions are parameterized in a name, $x$. Can you,
  gentle reader, make a definition that eliminates this parameter and
  guarantees no accidental interaction between the replication
  machinery and the process being replicated -- i.e. no accidental
  sharing of names used by the process to get its work done and the
  name(s) used by the replication to effect copying. This latter
  revision of the definition of replication is crucial to obtaining
  the expected identity $!!P \sim !P$.
\end{remark}

\begin{remark}\label{rem:paradoxical_combinator}
  The reader familiar with the lambda calculus will have noticed the
  similarity between $D$ and the paradoxical combinator.

  [Ed. note: the existence of this seems to suggest we have to be more
  restrictive on the set of processes and names we admit if we are to
  support no-cloning.]
\end{remark}

\subsubsection{Bisimulation}

The computational dynamics gives rise to another kind of equivalence,
the equivalence of computational behavior. As previously mentioned
this is typically captured \emph{via} some form of bisimulation.

% The notion we use in this paper is weak barbed bisimulation
% \cite{milner91polyadicpi}.

The notion we use in this paper is derived from weak barbed
bisimulation \cite{milner91polyadicpi}. 

\begin{definition}
An \emph{observation relation}, $\downarrow_{\mathcal N}$, over a set
of names, $\mathcal N$, is the smallest relation satisfying the rules
below.

\infrule[Out-barb]{y \in {\mathcal N}, \; x \nameeq y}
		  {\outputp{x}{v} \downarrow_{\mathcal N} x}
\infrule[Par-barb]{\mbox{$P\downarrow_{\mathcal N} x$ or $Q\downarrow_{\mathcal N} x$}}
		  {\binpar{P}{Q} \downarrow_{\mathcal N} x}

We write $P \Downarrow_{\mathcal N} x$ if there is $Q$ such that 
$P \wred Q$ and $Q \downarrow_{\mathcal N} x$.
\end{definition}

\begin{definition}
%\label{def.bbisim}
An  ${\mathcal N}$-\emph{barbed bisimulation} over a set of names, ${\mathcal N}$, is a symmetric binary relation 
${\mathcal S}_{\mathcal N}$ between agents such that $P\rel{S}_{\mathcal N}Q$ implies:
\begin{enumerate}
\item If $P \red P'$ then $Q \wred Q'$ and $P'\rel{S}_{\mathcal N} Q'$.
\item If $P\downarrow_{\mathcal N} x$, then $Q\Downarrow_{\mathcal N} x$.
\end{enumerate}
$P$ is ${\mathcal N}$-barbed bisimilar to $Q$, written
$P \wbbisim_{\mathcal N} Q$, if $P \rel{S}_{\mathcal N} Q$ for some ${\mathcal N}$-barbed bisimulation ${\mathcal S}_{\mathcal N}$.
\end{definition}

$\mathcal{R} \subseteq \pi \times \pi$

$P \mathcal{R} Q => \forall P'. P \red P' \Rightarrow \exists Q'. Q \red Q', P' \mathcal{R} Q'$

$P \vdash x \Rightarrow Q \vdash x$

\begin{mathpar}
  \inferrule*[lab=Out-barb]{x \nameeq y}{{y}!\langle{Q}\rangle \vdash x}
  \and
  \inferrule*[lab=Par-barb]{\mbox{$P\vdash x$ or $Q\vdash x$}}{\binpar{P}{Q} \vdash x}
\end{mathpar}

\subsubsection{Contexts}

One of the principle advantages of computational calculi like the
$\pi$-calculus is a well-defined notion of context,
contextual-equivalence and a correlation between
contextual-equivalence and notions of bisimulation. The notion of
context allows the decomposition of a process into (sub-)process and
its syntactic environment, its context. Thus, a context may be
thought of as a process with a ``hole'' (written $\Box$) in it. The
application of a context $M$ to a process $P$, written $M[P]$, is
tantamount to filling the hole in $M$ with $P$. In this paper we do
not need the full weight of this theory, but do make use of the notion
of context in the proof the main theorem. 

\begin{mathpar}
  \inferrule* [lab=summation] {} {{M_{M},M_{N}} \bc \Box \;|\; x.M_{A} \;|\; M_{M}+M_{N}}
  \and
  \inferrule* [lab=agent] {} {{M_{A}} \bc (\vec{x})M_{P} \;| \; \clift{P_0,\ldots,M_{P},\ldots,P_N}}
  \and \\
  \inferrule* [lab=process] {} {{M_{P}} \bc M_{N} \;| \;P|M_{P} }
\end{mathpar} 

\begin{mathpar}
  \inferrule* [lab=sychronization] {} {M_{N} \bc \Box \;|\; x?M_{F} \;|\; x!M_{C}}
  \and
  \inferrule* [lab=abstraction] {} {{M_{F}} \bc (x)M_{P} }
  \and
  \inferrule* [lab=concretion] {} {{M_{C}} \bc \langle M_{P} \rangle }
  \and \\
  \inferrule* [lab=process] {} {{M_{P}} \bc M_{N} \;| \;P|M_{P} }
\end{mathpar}

\begin{definition}[contextual application] Given a context $M$, and
  process $P$, we define the \emph{contextual application}, $M[P] :=
  M\{P/\Box\}$. That is, the contextual application of M to P is the
  substitution of $P$ for $\Box$ in $M$.
\end{definition}

$\meaningof{-} : L \to \mathcal{P}(\pi)$

\begin{mathpar}
  \inferrule* [lab=collection] {} {\meaningof{true} = \pi, \and \meaningof{~E} = \pi \setminus \meaningof{E}, \and \meaningof{E_{1} \& E_{2}} = \meaningof{E_{1}} \cap \meaningof{E_{2}}}
\end{mathpar}

\begin{mathpar}
  \inferrule* [lab=structure] {} {\meaningof{0} = \{ P \in \pi | P \equiv 0 \}, \and \\ \meaningof{E_1 | E_2} = \{ P \in \pi | P \equiv P_{1} | P_{2}, P_{1} \in \meaningof{E_{1}}, P_{2} \in \meaningof{E_2}\} }
\end{mathpar}

\begin{mathpar}
 \inferrule* [lab=behavior] {} {\meaningof{\langle a?b \rangle E} = \{ P \in \pi | P \equiv Q | u?(y)P', \\ \and \\\\ \and \\ \;\;\; u \in \meaningof{a}, \forall z.P'\{z/y\} \in \meaningof{E\{z/b\}}\}, \and \\ \meaningof{a!E} = \{ P \in \pi | P \equiv Q | x!\langle P' \rangle, x \in \meaningof{a} P' \in \meaningof{E}\} }
\end{mathpar}

\begin{mathpar}
 \inferrule* [lab=nominal] {} {\meaningof{\quotep{E}} = \{ \quotep{P} \in \quotep{\pi} | P \in \meaningof{E} \}, \and \meaningof{\quotep{P}} = \{ \quotep{Q} \in \quotep{\pi} | P \equiv Q \} \and \\ \meaningof{@\quotep{E}} = \{ P \in \pi | P \equiv @x, x \in \meaningof{E} \}}
\end{mathpar}

\begin{eqnarray*}
  \\
  \meaningof{-} : TS \to ST
\end{eqnarray*}

\begin{eqnarray*}
  \\
  L : TS \to ST
\end{eqnarray*}

\begin{eqnarray*}
  \\
  P \models E \iff P \in \meaningof{E}
\end{eqnarray*}

\begin{eqnarray*}
  P \approx_{L} Q \iff \forall E \in L. P \models E \iff Q \models E
\end{eqnarray*}

\begin{eqnarray*}
  P \approx_{K} Q
\end{eqnarray*}

\begin{eqnarray*}
  P \approx Q
\end{eqnarray*}

$\approx_{K} = \approx = \approx_{L}$

\subsubsection{Contextual duality}

Note that contexts extend the quotation operation to a family of
operations from processes to names. Given a context, $M$, we can
define a \emph{nominal context}, $\quotep{M}$ by $\quotep{M}[P] :=
\quotep{M[P]}$. To foreshadow what is to come we observe that these
operations enjoy a duality with processes very much like the duality
between vectors and maps from vectors to scalars.

Further, because the calculus is essentially higher-order, we have a
correspondence between contexts and processes. More specifically,
given a name $x$ and a context $M$ we can construct $M^{*}_{x}$ such
that 

\begin{mathpar}
  M^{*}_{x} | \lift{x}{P} \red M[P]
\end{mathpar}

namely,

\begin{mathpar}
  M^{*}_{x} := x?(u).M[\dropn{u}]
\end{mathpar}

The dependence of $M^{*}_{x}$ on a name makes it an abstraction, 

\begin{mathpar}
  M^{*} := (x)x?(u).M[\dropn{u}]
\end{mathpar}

\subsection{Additional notation}

It will sometimes be convenient to denote the process a name
quotes. We already have the notation $x = \quotep{P}$, but it will be
convenient to introduce an alternate notation, $\procn{x}$, when we
want to emphasize the connection to the use of the name. Note that, by
virtue of name equivalence, $\quotep{\procn{x}} \nameeq x$; so, the
notation is consistent with previous definitions.

Further, because names have structure it is possible to effect
substitutions on the basis of that structure. This means we need to
upgrade our notation for substitutions, which we accomplish by
adapting comprehension notation. Thus,

\begin{mathpar}
  P\{ y / x : x \in S \}
\end{mathpar}

is interpreted to mean the process derived from P by replacing (in a
capture-avoiding manner) each occurrence of $x$ in $S$ by $y$. For example,

\begin{mathpar}
  P\{ \quotep{\procn{x}|\procn{x}} / x : x \in \freenames{P} \}
\end{mathpar}

will replace each (occurrence) of a free name $x$ in $P$ by
$\quotep{\procn{x}|\procn{x}}$.

Also, we will avail ourselves of the notation $x^{L}$ and $x^{R}$ to
denote injections of a name into disjoint copies of the name
space. There are numerous ways to accomplish this. One example can be
found in \cite{MeredithR05}. This notation overloads to vectors of
names: $\vec{x}^{\pi} := (x_{i}^{\pi} \; : \; 0 \leq i < |\vec{x}| )$ where $\pi \in \{L,R\}$.

We also use $P^{\Box} := P|\Box$.

In \cite{MeredithR05} an interpretation of the new operator is
given. It turns out that there are several possible interpretations
all enjoying the requisite algebraic properties of the operator (see
\cite{milner91polyadicpi}). We will therefore make liberal use of
$(\nu\; \vec{x})P$.

% subsection the_syntax_and_semantics_of_the_notation_system (end)   

\input{qm2pi.qmops} 

\input{qm2pi.sterngerlach} 

\input{qm2pi.metric} 

% section concurrent_process_calculi (end)

%\input{qm2pi.proofsketch}

% section proof sketch (end)

%\input{qm2pi.slviaknots} 

% section spatial logic via knots (end)

\input{qm2pi.conclusion}

% section conclusion (end)

%\input{qm2pi.dtcodes} 

% section wiring algorithm (end)

\input{qm2pi.ack} 

% section acknowledgments (end)

\newpage


\bibliographystyle{plain}   
\bibliography{../../biblios/main.bib}

\input{qm2pi.rhodetails}

\end{document}

 

%\ifpdf
%\usepackage[pdftex]{graphicx}
%\else
%\usepackage{graphicx}
%\fi

 % \ifpdf
%  \usepackage{pdfsync}
%  \if


%\title{Brief Article}
%\author{David F. Snyder}
%\author{L.G. Meredith}

%\address{Dept. of Math., Texas State University--San Marcos, San Marcos, TX 78666}
       
\pagestyle{empty}


\begin{document}

\lstset{language=[Objective]Caml,frame=shadowbox}

\documentclass[12pt]{llncs}
%\documentclass{jktr}

\usepackage[pdftex]{hyperref}                   
\usepackage {listings}
\usepackage {mathpartir}
\usepackage{bcprules}
%\usepackage{listings}
                       
\usepackage{graphicx} 
%\usepackage[margins=2.5cm,nohead,nofoot]{geometry}
%\usepackage{geometry}
\usepackage{amsfonts}
\usepackage{amstext}
\usepackage{latexsym}
\usepackage{amssymb}
\usepackage{color}


%\include{myPreamble}
\include{qm2pi.local} 

%\ifpdf
%\usepackage[pdftex]{graphicx}
%\else
%\usepackage{graphicx}
%\fi

 % \ifpdf
%  \usepackage{pdfsync}
%  \if


%\title{Brief Article}
%\author{David F. Snyder}
%\author{L.G. Meredith}

%\address{Dept. of Math., Texas State University--San Marcos, San Marcos, TX 78666}
       
\pagestyle{empty}


\begin{document}

\lstset{language=[Objective]Caml,frame=shadowbox}

\input{qm2pi.front}

% section front matter (end)

\input{qm2pi.intro} 
 
% section introduction (end)

% \input{qm2pi.knotations} 

% section notation (end)

\input{qm2pi.process.calculi} 

% section concurrent_process_calculi_and_spatial_logics_ (end)
    
%\input{qm2pi.knots2pi} 

%\input{qm2pi.trefoil} 

%\input{qm2pi.mainthm} 

% subsection basic_interpretation (end)

%\input{qm2pi.rho.presentation} 
\subsection{The syntax and semantics of the notation system}\label{sub:the_syntax_and_semantics_of_the_notation_system} % (fold)

We now summarize a technical presentation of the calculus that
embodies our theory of dynamics. The typical presentation of such a
calculus follows the style of giving generators and relations on
them. The grammar, below, describing term constructors, freely
generates the set of processes, $\Proc$. This set is then quotiented
by a relation known as structural congruence and it is over this set
that the notion of dynamics is expressed. This presentation is
essentially that of \cite{MeredithR05} with the addition of
polyadicity and summation. For readability we have relegated some of
the technical subtleties to an appendix.

\subsubsection{Process grammar}\label{subsub:process_grammar}

\begin{mathpar}
  \inferrule* [lab=synchronization] {} {{M} \bc \pzero \;|\; x?F \;|\; x!C }
  \and
  \inferrule* [lab=abstraction] {} {{F} \bc (x)P}
  \and
  \inferrule* [lab=concretion] {} {{C} \bc \langle Q \rangle}
  \and
  \inferrule* [lab=process] {} {{P,Q} \bc M \;| \;P|Q \;|\; @{x}}
  \and
  \inferrule* [lab=name] {} {{x} \bc \quotep{P}}
\end{mathpar} 

Note that $\vec{x}$ (resp. $\vec{P}$) denotes a vector of names
(resp. processes) of length $|\vec{x}|$ (resp. $|\vec{P}|$). We adopt
the following useful abbreviations.

\begin{mathpar}
   x?(\vec{y}).P := x.(\vec{y})P \and  x\clift{\vec{P}} := x.\clift{\vec{P}}
   \and x!(y) := \lift{x}{\dropn{y}}
   \and \Pi_{i=0}^{n-1}P_i := P_0 | \ldots | P_{n-1}
\end{mathpar}

\subsubsection{Structural congruence}

\paragraph{Free and bound names and alpha-equivalence.} At the
core of structural equivalence is alpha-equivalence which identifies
process that are the same up to a change of variable. Formally, we
recognize the distinction between free and bound names. The free names
of a process, $\freenames{P}$, may be calculated recursively as
follows:

\begin{mathpar}
\freenames{\pzero} := \emptyset
  \and \\
  \freenames{x?(y).P} := \{ x \} \cup (\freenames{P} \setminus \{ y \})
  \and 
  \freenames{x!\langle P \rangle} := \{ x \} \cup \{ P \} 
  \and \\
  \freenames{P|Q} := \freenames{P} \cup \freenames{Q}
  \and \\
  \freenames{@{x}} := \{ x \}
\end{mathpar}

$\pi$
$\quotep{\pi}$

$\freenames{-} : \pi \to \mathcal{P}(\quotep{\pi})$

\begin{eqnarray*}
  \freenames{\pzero} & := & \emptyset \\
  \freenames{x?(y).P} & := & \{ x \} \cup (\freenames{P} \setminus \{ y \}) \\
  \freenames{x!\langle P \rangle} & := & \{ x \} \cup \{ P \} \\
  \freenames{P|Q} & := & \freenames{P} \cup \freenames{Q} \\
  \freenames{\dropn{x}} & := & \{ x \}
\end{eqnarray*}

The bound names of a process, $\boundnames{P}$, are those names occurring in $P$
that are not free. For example, in $x?(y).0$, the name $x$ is free, while $y$ is bound.

\begin{mathpar}
  \inferrule* [lab=monoidal-laws] {} { P|Q \equiv Q|P \and P|0 \equiv P \and P|(Q|R) \equiv (P|Q)|R }
\end{mathpar}

\begin{mathpar}
  \inferrule* [lab=alpha-equivalence] {} { (x)P \equiv (y)P\{y/x\} \and y \not\in \freenames{P} }
\end{mathpar}

\begin{definition}
Then two processes, $P,Q$, are alpha-equivalent if $P = Q\{\vec{y}/\vec{x}\}$ for
some $\vec{x} \in \boundnames{Q},\vec{y} \in \boundnames{P}$, where $Q\{\vec{y}/\vec{x}\}$
denotes the capture-avoiding substitution of $\vec{y}$ for $\vec{x}$ in $Q$.
\end{definition}

\begin{definition}
  The {\em structural congruence} \cite{SangiorgiWalker} , $\equiv$,
  between processes is the least congruence containing
  alpha-equivalence, satisfying the abelian monoid laws
  (associativity, commutativity and $\pzero$ as identity) for parallel
  composition $|$ and for summation $+$.
\end{definition}

\subsection{Name equivalence}

We take name equivalence, written $\nameeq$, to be the smallest
equivalence relation generated by the following rules.

\begin{mathpar}
\inferrule*[lab=Quote-drop]
{ }
{ \quotep{@{x}} \nameeq x }

\inferrule*[lab=Struct-equiv]
{ P \scong Q }
{ \quotep{P} \nameeq \quotep{Q} }
\end{mathpar}

The astute reader will have noticed that the mutual recursion of names
and processes imposes a mutual recursion on alpha-equivalence and
structural equivalence via name-equivalence. Fortunately, all of this
works out pleasantly and we may calculate in the natural way, free of
concern. The reader interested in the details is referred to the
appendix \ref{appendix:rho_details}.

\subsection{Substitution}

We use $\Proc$ for the set of processes, $\QProc$ for the set of
names, and $\id{\{}\vec{y} / \vec{x} \id{\}}$ to denote partial maps,
$s : \QProc \rightarrow \QProc$. A map, $s$ lifts, uniquely, to a map
on process terms, $\widehat{s} : \Proc \rightarrow \Proc$ by the
following equations.

\begin{mathpar}
  (0) \psubstp{Q}{P} := 0 \\
  (R \juxtap S) \psubstp{Q}{P}
  :=    
  (R)\psubstp{Q}{P} \juxtap (S) \psubstp{Q}{P} \\
  (x?(y).R) \psubstp{Q}{P}    
  :=    
  (x)\substp{Q}{P} (z)\concat( (R \psubstn{z}{y}) \psubstp{Q}{P} ) \\
  (\lift{x}{R}) \psubstp{Q}{P}  
  :=
  \lift{(x)\substp{Q}{P}}{ R \psubstp{Q}{P} } \\
%   (\dropn{x})  \psubstp{Q}{P}       
%   := 
%   \left\{ 
%     \begin{array}{ccc} 
%       \dropn{\quotep{Q}} & & x \nameeq \quotep{P} \\
%       \dropn{x} & & otherwise \\
%     \end{array}
%   \right. 
  (\dropn{x})  \psubstp{Q}{P}       
  := 
  \left\{ 
    \begin{array}{ccc} 
      Q & & x \nameeq \quotep{P} \\
      \dropn{x} & & otherwise \\
    \end{array}
  \right.
\end{mathpar}
 

where

\begin{eqnarray}
  (x)\id{\{} \lpquote Q \rpquote / \lpquote P \rpquote \id{\}}            = 
  \left\{ 
    \begin{array}{ccc}
      \lpquote Q \rpquote & & x \nameeq \lpquote P \rpquote \\
      x & & otherwise \\
    \end{array}
  \right. \nonumber
\end{eqnarray}

and $z$ is chosen distinct from $\quotep{P}$, $\quotep{Q}$, the free
names in $Q$, and all the names in $R$. Our $\alpha$-equivalence will
be built in the standard way from this substitution.

\begin{remark}\label{rem:no_self_referential_names}
  One consequence of these definitions is that $\forall P. \quotep{P}
  \not\in \freenames{P}$.
\end{remark}

\subsection{ Dynamic quote: an example }

Anticipating something of what's to come, consider applying the
substitution, $\widehat{\id{\{}u / z \id{\}}}$, to the following pair
of processes, $\lift{w}{y!(z)}$ and $w[ \lpquote y!(z) \rpquote ]$.

\begin{eqnarray}
	\lift{w}{y!(z)}\widehat{\id{\{}u / z \id{\}}}
		& = &
		\lift{w}{y!(u)} \nonumber\\
	w[ \lpquote y!(z) \rpquote ] \widehat{ \id{\{}u / z \id{\}} }
		& = &
		w[ \lpquote y!(z) \rpquote ] \nonumber
\end{eqnarray}

Because the body of the process between quotes is impervious to
substitution, we get radically different answers. In fact, by
examining the first process in an input context,
e.g. $x?(z).\lift{w}{y!(z)}$, we see that the process under the lift
operator may be shaped by prefixed inputs binding a name inside it. In
this sense, the lift operator will be seen as a way to dynamically
construct processes before reifying them as names.

Finally equipped with these standard features we can present the
dynamics of the calculus.

\subsubsection{Operational semantics} 

Finally, we introduce the computational dynamics. What marks these
algebras as distinct from other more traditionally studied algebraic
structures, e.g. vector spaces or polynomial rings, is the manner in
which dynamics is captured. In traditional structures, dynamics is typically
expressed through morphisms between such structures, as in linear maps
between vector spaces or morphisms between rings. In algebras
associated with the semantics of computation, the dynamics is
expressed as part of the algebraic structure itself, through a
reduction reduction relation typically denoted by $\red$. Below, we
give a recursive presentation of this relation for the calculus used
in the encoding.

$\red \subseteq \pi \times \pi$
$\red : \pi \to \mathcal{P}(\pi)$

\begin{mathpar}
  \inferrule* [lab=Comm] { \textsf{match}( x_{src}, x_{trgt} ) } { x_{trgt}?(y)P \; | \; x_{src}!\langle {Q} \rangle \red P\{\quotep{Q}/y}\} }
  \and \\
  \inferrule* [lab=Par] {{P} \red {P}'} {{{P} | {Q}} \red {{P}' | {Q}}}
  \and
  \inferrule* [lab=Equiv]{{{P} \scong {P}'} \andalso {{P}' \red {Q}'} \andalso {{Q}' \scong {Q}}}{{P} \red {Q}}
\end{mathpar}

\begin{eqnarray*}
  match_{\equiv} (\quotep{P},\quotep{Q}) & := & P \equiv Q \\
  match_{\dagger}(\quotep{P},\quotep{Q}) & := & \forall R. P|Q \red^{*} R => R \red^{*} 0 \\
  match_{K}(\quotep{P},\quotep{Q}) & := & K \mbox{ for some context } K
\end{eqnarray*}

$u?(x)P | u!\langle Q \rangle \red P\{\quotep{Q}/x\}$

%We write $\wred$ for $\red^*$, and $P\red$ if $\exists Q $ such that $ P \red Q$.
We write $P\red$ if $\exists Q $ such that $ P \red Q$ and $P\not\red$, otherwise.

\section{Replication}

As mentioned before, it is known that replication (and hence
recursion) can be implemented in a higher-order process algebra
\cite{SangiorgiWalker}. As our first example of calculation with the
machinery thus far presented we give the construction explicitly in
the {\rhoc}.

\begin{eqnarray}
	D_{x} & := & \prefix{x}{y}{(\binpar{\outputp{x}{y}}{@{y}})} \nonumber\\
	\bangp_{x}{P} & := & \binpar{{x}!\langle{\binpar{D_{x}}{P}}\rangle}{D_{x}} \nonumber
\end{eqnarray}

\begin{eqnarray}
	\bangp_{x}{P} & & \nonumber\\
	=
	& {x}!\langle{(\prefix{x}{y}{(\outputp{x}{y} | @{y})) | P}}\rangle 
	      | \prefix{x}{y}{(\outputp{x}{y} | @{y})} & \nonumber\\
	\red
	& (\outputp{x}{y} | @{y})\substn{\quotep{(\prefix{x}{y}{(@{y} | \outputp{x}{y})) | P}}}{y} & \nonumber\\
	=
	& \outputp{x}{\quotep{(\prefix{x}{y}{(\outputp{x}{y} | @{y})) | P}}}
	  | {(\prefix{x}{y}{(\outputp{x}{y} | @{y})) | P}} & \nonumber\\
	\red
	& \ldots & \nonumber\\
	\red^*
	& P | P | \ldots & \nonumber
\end{eqnarray}

Of course, this encoding, as an implementation, runs away, unfolding
$\bangp{P}$ eagerly. A lazier and more implementable replication
operator, restricted to input-guarded processes, may be obtained as follows.

\begin{eqnarray}
\bangp{\prefix{u}{v}{P}} 
	:= 
	\binpar{\lift{x}{\prefix{u}{v}{(\binpar{D(x)}{P})}}}{D(x)} \nonumber
\end{eqnarray}

\begin{remark}
  Note that the lazier definition still does not deal with summation
  or mixed summation (i.e. sums over input and output). The reader is
  invited to construct definitions of replication that deal with these
  features. 

  Further, the definitions are parameterized in a name, $x$. Can you,
  gentle reader, make a definition that eliminates this parameter and
  guarantees no accidental interaction between the replication
  machinery and the process being replicated -- i.e. no accidental
  sharing of names used by the process to get its work done and the
  name(s) used by the replication to effect copying. This latter
  revision of the definition of replication is crucial to obtaining
  the expected identity $!!P \sim !P$.
\end{remark}

\begin{remark}\label{rem:paradoxical_combinator}
  The reader familiar with the lambda calculus will have noticed the
  similarity between $D$ and the paradoxical combinator.

  [Ed. note: the existence of this seems to suggest we have to be more
  restrictive on the set of processes and names we admit if we are to
  support no-cloning.]
\end{remark}

\subsubsection{Bisimulation}

The computational dynamics gives rise to another kind of equivalence,
the equivalence of computational behavior. As previously mentioned
this is typically captured \emph{via} some form of bisimulation.

% The notion we use in this paper is weak barbed bisimulation
% \cite{milner91polyadicpi}.

The notion we use in this paper is derived from weak barbed
bisimulation \cite{milner91polyadicpi}. 

\begin{definition}
An \emph{observation relation}, $\downarrow_{\mathcal N}$, over a set
of names, $\mathcal N$, is the smallest relation satisfying the rules
below.

\infrule[Out-barb]{y \in {\mathcal N}, \; x \nameeq y}
		  {\outputp{x}{v} \downarrow_{\mathcal N} x}
\infrule[Par-barb]{\mbox{$P\downarrow_{\mathcal N} x$ or $Q\downarrow_{\mathcal N} x$}}
		  {\binpar{P}{Q} \downarrow_{\mathcal N} x}

We write $P \Downarrow_{\mathcal N} x$ if there is $Q$ such that 
$P \wred Q$ and $Q \downarrow_{\mathcal N} x$.
\end{definition}

\begin{definition}
%\label{def.bbisim}
An  ${\mathcal N}$-\emph{barbed bisimulation} over a set of names, ${\mathcal N}$, is a symmetric binary relation 
${\mathcal S}_{\mathcal N}$ between agents such that $P\rel{S}_{\mathcal N}Q$ implies:
\begin{enumerate}
\item If $P \red P'$ then $Q \wred Q'$ and $P'\rel{S}_{\mathcal N} Q'$.
\item If $P\downarrow_{\mathcal N} x$, then $Q\Downarrow_{\mathcal N} x$.
\end{enumerate}
$P$ is ${\mathcal N}$-barbed bisimilar to $Q$, written
$P \wbbisim_{\mathcal N} Q$, if $P \rel{S}_{\mathcal N} Q$ for some ${\mathcal N}$-barbed bisimulation ${\mathcal S}_{\mathcal N}$.
\end{definition}

$\mathcal{R} \subseteq \pi \times \pi$

$P \mathcal{R} Q => \forall P'. P \red P' \Rightarrow \exists Q'. Q \red Q', P' \mathcal{R} Q'$

$P \vdash x \Rightarrow Q \vdash x$

\begin{mathpar}
  \inferrule*[lab=Out-barb]{x \nameeq y}{{y}!\langle{Q}\rangle \vdash x}
  \and
  \inferrule*[lab=Par-barb]{\mbox{$P\vdash x$ or $Q\vdash x$}}{\binpar{P}{Q} \vdash x}
\end{mathpar}

\subsubsection{Contexts}

One of the principle advantages of computational calculi like the
$\pi$-calculus is a well-defined notion of context,
contextual-equivalence and a correlation between
contextual-equivalence and notions of bisimulation. The notion of
context allows the decomposition of a process into (sub-)process and
its syntactic environment, its context. Thus, a context may be
thought of as a process with a ``hole'' (written $\Box$) in it. The
application of a context $M$ to a process $P$, written $M[P]$, is
tantamount to filling the hole in $M$ with $P$. In this paper we do
not need the full weight of this theory, but do make use of the notion
of context in the proof the main theorem. 

\begin{mathpar}
  \inferrule* [lab=summation] {} {{M_{M},M_{N}} \bc \Box \;|\; x.M_{A} \;|\; M_{M}+M_{N}}
  \and
  \inferrule* [lab=agent] {} {{M_{A}} \bc (\vec{x})M_{P} \;| \; \clift{P_0,\ldots,M_{P},\ldots,P_N}}
  \and \\
  \inferrule* [lab=process] {} {{M_{P}} \bc M_{N} \;| \;P|M_{P} }
\end{mathpar} 

\begin{mathpar}
  \inferrule* [lab=sychronization] {} {M_{N} \bc \Box \;|\; x?M_{F} \;|\; x!M_{C}}
  \and
  \inferrule* [lab=abstraction] {} {{M_{F}} \bc (x)M_{P} }
  \and
  \inferrule* [lab=concretion] {} {{M_{C}} \bc \langle M_{P} \rangle }
  \and \\
  \inferrule* [lab=process] {} {{M_{P}} \bc M_{N} \;| \;P|M_{P} }
\end{mathpar}

\begin{definition}[contextual application] Given a context $M$, and
  process $P$, we define the \emph{contextual application}, $M[P] :=
  M\{P/\Box\}$. That is, the contextual application of M to P is the
  substitution of $P$ for $\Box$ in $M$.
\end{definition}

$\meaningof{-} : L \to \mathcal{P}(\pi)$

\begin{mathpar}
  \inferrule* [lab=collection] {} {\meaningof{true} = \pi, \and \meaningof{~E} = \pi \setminus \meaningof{E}, \and \meaningof{E_{1} \& E_{2}} = \meaningof{E_{1}} \cap \meaningof{E_{2}}}
\end{mathpar}

\begin{mathpar}
  \inferrule* [lab=structure] {} {\meaningof{0} = \{ P \in \pi | P \equiv 0 \}, \and \\ \meaningof{E_1 | E_2} = \{ P \in \pi | P \equiv P_{1} | P_{2}, P_{1} \in \meaningof{E_{1}}, P_{2} \in \meaningof{E_2}\} }
\end{mathpar}

\begin{mathpar}
 \inferrule* [lab=behavior] {} {\meaningof{\langle a?b \rangle E} = \{ P \in \pi | P \equiv Q | u?(y)P', \\ \and \\\\ \and \\ \;\;\; u \in \meaningof{a}, \forall z.P'\{z/y\} \in \meaningof{E\{z/b\}}\}, \and \\ \meaningof{a!E} = \{ P \in \pi | P \equiv Q | x!\langle P' \rangle, x \in \meaningof{a} P' \in \meaningof{E}\} }
\end{mathpar}

\begin{mathpar}
 \inferrule* [lab=nominal] {} {\meaningof{\quotep{E}} = \{ \quotep{P} \in \quotep{\pi} | P \in \meaningof{E} \}, \and \meaningof{\quotep{P}} = \{ \quotep{Q} \in \quotep{\pi} | P \equiv Q \} \and \\ \meaningof{@\quotep{E}} = \{ P \in \pi | P \equiv @x, x \in \meaningof{E} \}}
\end{mathpar}

\begin{eqnarray*}
  \\
  \meaningof{-} : TS \to ST
\end{eqnarray*}

\begin{eqnarray*}
  \\
  L : TS \to ST
\end{eqnarray*}

\begin{eqnarray*}
  \\
  P \models E \iff P \in \meaningof{E}
\end{eqnarray*}

\begin{eqnarray*}
  P \approx_{L} Q \iff \forall E \in L. P \models E \iff Q \models E
\end{eqnarray*}

\begin{eqnarray*}
  P \approx_{K} Q
\end{eqnarray*}

\begin{eqnarray*}
  P \approx Q
\end{eqnarray*}

$\approx_{K} = \approx = \approx_{L}$

\subsubsection{Contextual duality}

Note that contexts extend the quotation operation to a family of
operations from processes to names. Given a context, $M$, we can
define a \emph{nominal context}, $\quotep{M}$ by $\quotep{M}[P] :=
\quotep{M[P]}$. To foreshadow what is to come we observe that these
operations enjoy a duality with processes very much like the duality
between vectors and maps from vectors to scalars.

Further, because the calculus is essentially higher-order, we have a
correspondence between contexts and processes. More specifically,
given a name $x$ and a context $M$ we can construct $M^{*}_{x}$ such
that 

\begin{mathpar}
  M^{*}_{x} | \lift{x}{P} \red M[P]
\end{mathpar}

namely,

\begin{mathpar}
  M^{*}_{x} := x?(u).M[\dropn{u}]
\end{mathpar}

The dependence of $M^{*}_{x}$ on a name makes it an abstraction, 

\begin{mathpar}
  M^{*} := (x)x?(u).M[\dropn{u}]
\end{mathpar}

\subsection{Additional notation}

It will sometimes be convenient to denote the process a name
quotes. We already have the notation $x = \quotep{P}$, but it will be
convenient to introduce an alternate notation, $\procn{x}$, when we
want to emphasize the connection to the use of the name. Note that, by
virtue of name equivalence, $\quotep{\procn{x}} \nameeq x$; so, the
notation is consistent with previous definitions.

Further, because names have structure it is possible to effect
substitutions on the basis of that structure. This means we need to
upgrade our notation for substitutions, which we accomplish by
adapting comprehension notation. Thus,

\begin{mathpar}
  P\{ y / x : x \in S \}
\end{mathpar}

is interpreted to mean the process derived from P by replacing (in a
capture-avoiding manner) each occurrence of $x$ in $S$ by $y$. For example,

\begin{mathpar}
  P\{ \quotep{\procn{x}|\procn{x}} / x : x \in \freenames{P} \}
\end{mathpar}

will replace each (occurrence) of a free name $x$ in $P$ by
$\quotep{\procn{x}|\procn{x}}$.

Also, we will avail ourselves of the notation $x^{L}$ and $x^{R}$ to
denote injections of a name into disjoint copies of the name
space. There are numerous ways to accomplish this. One example can be
found in \cite{MeredithR05}. This notation overloads to vectors of
names: $\vec{x}^{\pi} := (x_{i}^{\pi} \; : \; 0 \leq i < |\vec{x}| )$ where $\pi \in \{L,R\}$.

We also use $P^{\Box} := P|\Box$.

In \cite{MeredithR05} an interpretation of the new operator is
given. It turns out that there are several possible interpretations
all enjoying the requisite algebraic properties of the operator (see
\cite{milner91polyadicpi}). We will therefore make liberal use of
$(\nu\; \vec{x})P$.

% subsection the_syntax_and_semantics_of_the_notation_system (end)   

\input{qm2pi.qmops} 

\input{qm2pi.sterngerlach} 

\input{qm2pi.metric} 

% section concurrent_process_calculi (end)

%\input{qm2pi.proofsketch}

% section proof sketch (end)

%\input{qm2pi.slviaknots} 

% section spatial logic via knots (end)

\input{qm2pi.conclusion}

% section conclusion (end)

%\input{qm2pi.dtcodes} 

% section wiring algorithm (end)

\input{qm2pi.ack} 

% section acknowledgments (end)

\newpage


\bibliographystyle{plain}   
\bibliography{../../biblios/main.bib}

\input{qm2pi.rhodetails}

\end{document}



% section front matter (end)

\section{Introduction}\label{sec:introduction} % (fold)
In this draft of the material i am going to have to dispense with the
usual writing conventions adopted in papers on these topics. i'm going
to have adopt whatever tone i need at the time i'm writing up the
calculations. Sometimes this may be very conversational; others it may
be the barest mathematical grunts; others still it may be that i have
lifted text from one of my other papers because the exposition of some
point was better said there. i hope that my readers are not unduly put
out by this decision. i'm not doing this to flout convention or be
rebellious. i find these calculations very technically challenging. To
keep everything going technically, something has to give; i have to
let go of some cognitive burden. So, the academic writing style --
with all of its trade-offs in terms of facilitating technical
communication -- is what i'm letting go of. Perhaps subsequent drafts
can be tightened and polished, but for now, i'm going to speak as if
we were sitting together in a coffee shop with a laptop, wifi and a
pad of paper and a pencil.

So, here's what i have to say. We -- you and i, comfortably ensconced
in our coffee shop and well-equipped with our tools -- can realize and
carry out the calculations of quantum mechanics over a very different
formal theory of dynamics, a formal theory of dynamics that
corresponds to a theory of concurrent computation with
\emph{reflection}. It has the advantage that the underlying theory is
already `quantized', but supports analogues all of the continuuous
operations. Strikingly, this underlying theory has recently been
connected with a notion of metric that we can show, by calculating
together, coincides with the metric induced by the inner product.

There are a lot of reasons why you might be interested in seeing
calculations of this form. Here's why i'm interested. For the past
several centuries there has been no competitor to the ``Newtonian''
account of dynamics. As a result the predominant share of accounts of
dynamical systems and situations have had to be formulated in terms of
the Newtonian machinery. i view this as an intellectually dangerous
position to occupy. Everything, despite it's intrinsic shape, turns
into a nail to be hit with this hammer. Recently, however, the theory
of computation has matured to the point where we have candidates for
theories of dynamics that offer very different perspective on
reasoning about dynamical systems and situations. Testing these
candidates against very successful accounts of dynamical situations,
like quantum mechanics, is going to give us some sense of how mature
they are and some measure of the quality of these accounts of
dynamics.

\subsection{Summary of contributions and outline of paper}

So, we're going to develop an interpretation of the operations of
quantum mechanics normally interpreted by Hilbert spaces and
operators. We're going to do this over a theory of computation. Note
that this is very different than the usual quantum computation program
which develops notions of computation over quantum mechanics. Rather,
we are developing a story that aligns with Wheeler's slogan: It from
Bit. To do this we will first provide an account of the theory of
computation at play here. Then we will dive into a calculation-driven
interpretation of the operations of quantum mechanics.

The reason we take this approach is that -- until very recently --
there hasn't been an axiomatic account of quantum mechanics. As a
result there has been no sharp delineation of the mathematical theory
supporting interpretation of the physical theory and the physical
theory, itself. So, ambient features of the maths are free to be
exploited (or supressed) without a real accounting of their physical
relevance. There is no sharp statement ``here's the physical theory''
qua \emph{theory} and ``here's the mathematical interpretation''
enabling a judgment of how faithful the interpretation is -- apart
from experimental observation. When there is an axiomatic account we
can judge how well a given mathematical formalism supports an
interpretation of the axioms, independent of
experimentation. Likewise, we can judge how well we have captured our
physical evidence and experience with our axiomatics, independent of
any specific mathematical implementation, with accidental detail that
may or may not have physical significance. 

In lieu of a fully fleshed out and vetted axiomatic account of quantum
mechanics, interpreting the operational notions in service of modeling
physical systems will have to suffice. In other words, we are not in
the business of providing a model of Hilbert spaces and operators. We
are in the business of providing a model of quantum mechanics because
we are motivated by testing our notions of dynamics against physical
theory; and, the predictive calculations of the physical theory must
serve as the best formulation -- shy of a fully fleshed out axiomatic
account -- of the physical theory itself (as they have for scientific
theories since time immemorial). Put another way, despite a
whole-hearted commitment to an It-from-Bit ontology, we are firmly
aligned with the shut-up-and-calculate camp as the best way to obtain
results either from the physical perspective or as a quality assurance
measure of our fledgling theory of dynamics.

In detail, we present a reflective process calculus. Then we develop
intuitive correspondences between the notions available in this
calculus and the usual physical notions supporting quantum mechanical
calculations. Thus, 

\begin{table}[htp]
  \center{
    \fbox{
      \begin{tabular}{c|c}
        quantum mechanics & process calculus \\
        \hline
        scalar & name \\
        state vector & process \\
        dual & contextual duals \\
        matrix & formal sums of process-context-dual pairs \\
        orthogonality & process annihilation \\
        inner product & execution-formula + quoting
      \end{tabular}
    }
  }
  \caption{QM - process calculi correspondences}
\end{table}

Then we tighten up these intuitions to operational definitions. We
employ the Dirac notation as the best proxy we can find for an
abstract syntax of the quantum mechanical notions. The definitions we
develop put us in contact with equational constraints coming from the
theory that we demonstrate the definitions and calculations satisfy.

This puts us in a position to shut up and calculate for the
Stern-Gerlach experimental set up, showing how these predictive
calculations become calculations on processes in our theory of a
reflective process calculus.

Penultimately, we demonstrate that the notion of metric coming from
the inner product coincides with the notion of metric available from
the theory of bisimulation. This demonstration gives us the right to
think of space as arising from behavior. Finally, we consider where we
might go from the new vantage point we have obtained.

% section introduction (end) 
 
% section introduction (end)

% \documentclass[12pt]{llncs}
%\documentclass{jktr}

\usepackage[pdftex]{hyperref}                   
\usepackage {listings}
\usepackage {mathpartir}
\usepackage{bcprules}
%\usepackage{listings}
                       
\usepackage{graphicx} 
%\usepackage[margins=2.5cm,nohead,nofoot]{geometry}
%\usepackage{geometry}
\usepackage{amsfonts}
\usepackage{amstext}
\usepackage{latexsym}
\usepackage{amssymb}
\usepackage{color}


%\include{myPreamble}
\include{qm2pi.local} 

%\ifpdf
%\usepackage[pdftex]{graphicx}
%\else
%\usepackage{graphicx}
%\fi

 % \ifpdf
%  \usepackage{pdfsync}
%  \if


%\title{Brief Article}
%\author{David F. Snyder}
%\author{L.G. Meredith}

%\address{Dept. of Math., Texas State University--San Marcos, San Marcos, TX 78666}
       
\pagestyle{empty}


\begin{document}

\lstset{language=[Objective]Caml,frame=shadowbox}

\input{qm2pi.front}

% section front matter (end)

\input{qm2pi.intro} 
 
% section introduction (end)

% \input{qm2pi.knotations} 

% section notation (end)

\input{qm2pi.process.calculi} 

% section concurrent_process_calculi_and_spatial_logics_ (end)
    
%\input{qm2pi.knots2pi} 

%\input{qm2pi.trefoil} 

%\input{qm2pi.mainthm} 

% subsection basic_interpretation (end)

%\input{qm2pi.rho.presentation} 
\subsection{The syntax and semantics of the notation system}\label{sub:the_syntax_and_semantics_of_the_notation_system} % (fold)

We now summarize a technical presentation of the calculus that
embodies our theory of dynamics. The typical presentation of such a
calculus follows the style of giving generators and relations on
them. The grammar, below, describing term constructors, freely
generates the set of processes, $\Proc$. This set is then quotiented
by a relation known as structural congruence and it is over this set
that the notion of dynamics is expressed. This presentation is
essentially that of \cite{MeredithR05} with the addition of
polyadicity and summation. For readability we have relegated some of
the technical subtleties to an appendix.

\subsubsection{Process grammar}\label{subsub:process_grammar}

\begin{mathpar}
  \inferrule* [lab=synchronization] {} {{M} \bc \pzero \;|\; x?F \;|\; x!C }
  \and
  \inferrule* [lab=abstraction] {} {{F} \bc (x)P}
  \and
  \inferrule* [lab=concretion] {} {{C} \bc \langle Q \rangle}
  \and
  \inferrule* [lab=process] {} {{P,Q} \bc M \;| \;P|Q \;|\; @{x}}
  \and
  \inferrule* [lab=name] {} {{x} \bc \quotep{P}}
\end{mathpar} 

Note that $\vec{x}$ (resp. $\vec{P}$) denotes a vector of names
(resp. processes) of length $|\vec{x}|$ (resp. $|\vec{P}|$). We adopt
the following useful abbreviations.

\begin{mathpar}
   x?(\vec{y}).P := x.(\vec{y})P \and  x\clift{\vec{P}} := x.\clift{\vec{P}}
   \and x!(y) := \lift{x}{\dropn{y}}
   \and \Pi_{i=0}^{n-1}P_i := P_0 | \ldots | P_{n-1}
\end{mathpar}

\subsubsection{Structural congruence}

\paragraph{Free and bound names and alpha-equivalence.} At the
core of structural equivalence is alpha-equivalence which identifies
process that are the same up to a change of variable. Formally, we
recognize the distinction between free and bound names. The free names
of a process, $\freenames{P}$, may be calculated recursively as
follows:

\begin{mathpar}
\freenames{\pzero} := \emptyset
  \and \\
  \freenames{x?(y).P} := \{ x \} \cup (\freenames{P} \setminus \{ y \})
  \and 
  \freenames{x!\langle P \rangle} := \{ x \} \cup \{ P \} 
  \and \\
  \freenames{P|Q} := \freenames{P} \cup \freenames{Q}
  \and \\
  \freenames{@{x}} := \{ x \}
\end{mathpar}

$\pi$
$\quotep{\pi}$

$\freenames{-} : \pi \to \mathcal{P}(\quotep{\pi})$

\begin{eqnarray*}
  \freenames{\pzero} & := & \emptyset \\
  \freenames{x?(y).P} & := & \{ x \} \cup (\freenames{P} \setminus \{ y \}) \\
  \freenames{x!\langle P \rangle} & := & \{ x \} \cup \{ P \} \\
  \freenames{P|Q} & := & \freenames{P} \cup \freenames{Q} \\
  \freenames{\dropn{x}} & := & \{ x \}
\end{eqnarray*}

The bound names of a process, $\boundnames{P}$, are those names occurring in $P$
that are not free. For example, in $x?(y).0$, the name $x$ is free, while $y$ is bound.

\begin{mathpar}
  \inferrule* [lab=monoidal-laws] {} { P|Q \equiv Q|P \and P|0 \equiv P \and P|(Q|R) \equiv (P|Q)|R }
\end{mathpar}

\begin{mathpar}
  \inferrule* [lab=alpha-equivalence] {} { (x)P \equiv (y)P\{y/x\} \and y \not\in \freenames{P} }
\end{mathpar}

\begin{definition}
Then two processes, $P,Q$, are alpha-equivalent if $P = Q\{\vec{y}/\vec{x}\}$ for
some $\vec{x} \in \boundnames{Q},\vec{y} \in \boundnames{P}$, where $Q\{\vec{y}/\vec{x}\}$
denotes the capture-avoiding substitution of $\vec{y}$ for $\vec{x}$ in $Q$.
\end{definition}

\begin{definition}
  The {\em structural congruence} \cite{SangiorgiWalker} , $\equiv$,
  between processes is the least congruence containing
  alpha-equivalence, satisfying the abelian monoid laws
  (associativity, commutativity and $\pzero$ as identity) for parallel
  composition $|$ and for summation $+$.
\end{definition}

\subsection{Name equivalence}

We take name equivalence, written $\nameeq$, to be the smallest
equivalence relation generated by the following rules.

\begin{mathpar}
\inferrule*[lab=Quote-drop]
{ }
{ \quotep{@{x}} \nameeq x }

\inferrule*[lab=Struct-equiv]
{ P \scong Q }
{ \quotep{P} \nameeq \quotep{Q} }
\end{mathpar}

The astute reader will have noticed that the mutual recursion of names
and processes imposes a mutual recursion on alpha-equivalence and
structural equivalence via name-equivalence. Fortunately, all of this
works out pleasantly and we may calculate in the natural way, free of
concern. The reader interested in the details is referred to the
appendix \ref{appendix:rho_details}.

\subsection{Substitution}

We use $\Proc$ for the set of processes, $\QProc$ for the set of
names, and $\id{\{}\vec{y} / \vec{x} \id{\}}$ to denote partial maps,
$s : \QProc \rightarrow \QProc$. A map, $s$ lifts, uniquely, to a map
on process terms, $\widehat{s} : \Proc \rightarrow \Proc$ by the
following equations.

\begin{mathpar}
  (0) \psubstp{Q}{P} := 0 \\
  (R \juxtap S) \psubstp{Q}{P}
  :=    
  (R)\psubstp{Q}{P} \juxtap (S) \psubstp{Q}{P} \\
  (x?(y).R) \psubstp{Q}{P}    
  :=    
  (x)\substp{Q}{P} (z)\concat( (R \psubstn{z}{y}) \psubstp{Q}{P} ) \\
  (\lift{x}{R}) \psubstp{Q}{P}  
  :=
  \lift{(x)\substp{Q}{P}}{ R \psubstp{Q}{P} } \\
%   (\dropn{x})  \psubstp{Q}{P}       
%   := 
%   \left\{ 
%     \begin{array}{ccc} 
%       \dropn{\quotep{Q}} & & x \nameeq \quotep{P} \\
%       \dropn{x} & & otherwise \\
%     \end{array}
%   \right. 
  (\dropn{x})  \psubstp{Q}{P}       
  := 
  \left\{ 
    \begin{array}{ccc} 
      Q & & x \nameeq \quotep{P} \\
      \dropn{x} & & otherwise \\
    \end{array}
  \right.
\end{mathpar}
 

where

\begin{eqnarray}
  (x)\id{\{} \lpquote Q \rpquote / \lpquote P \rpquote \id{\}}            = 
  \left\{ 
    \begin{array}{ccc}
      \lpquote Q \rpquote & & x \nameeq \lpquote P \rpquote \\
      x & & otherwise \\
    \end{array}
  \right. \nonumber
\end{eqnarray}

and $z$ is chosen distinct from $\quotep{P}$, $\quotep{Q}$, the free
names in $Q$, and all the names in $R$. Our $\alpha$-equivalence will
be built in the standard way from this substitution.

\begin{remark}\label{rem:no_self_referential_names}
  One consequence of these definitions is that $\forall P. \quotep{P}
  \not\in \freenames{P}$.
\end{remark}

\subsection{ Dynamic quote: an example }

Anticipating something of what's to come, consider applying the
substitution, $\widehat{\id{\{}u / z \id{\}}}$, to the following pair
of processes, $\lift{w}{y!(z)}$ and $w[ \lpquote y!(z) \rpquote ]$.

\begin{eqnarray}
	\lift{w}{y!(z)}\widehat{\id{\{}u / z \id{\}}}
		& = &
		\lift{w}{y!(u)} \nonumber\\
	w[ \lpquote y!(z) \rpquote ] \widehat{ \id{\{}u / z \id{\}} }
		& = &
		w[ \lpquote y!(z) \rpquote ] \nonumber
\end{eqnarray}

Because the body of the process between quotes is impervious to
substitution, we get radically different answers. In fact, by
examining the first process in an input context,
e.g. $x?(z).\lift{w}{y!(z)}$, we see that the process under the lift
operator may be shaped by prefixed inputs binding a name inside it. In
this sense, the lift operator will be seen as a way to dynamically
construct processes before reifying them as names.

Finally equipped with these standard features we can present the
dynamics of the calculus.

\subsubsection{Operational semantics} 

Finally, we introduce the computational dynamics. What marks these
algebras as distinct from other more traditionally studied algebraic
structures, e.g. vector spaces or polynomial rings, is the manner in
which dynamics is captured. In traditional structures, dynamics is typically
expressed through morphisms between such structures, as in linear maps
between vector spaces or morphisms between rings. In algebras
associated with the semantics of computation, the dynamics is
expressed as part of the algebraic structure itself, through a
reduction reduction relation typically denoted by $\red$. Below, we
give a recursive presentation of this relation for the calculus used
in the encoding.

$\red \subseteq \pi \times \pi$
$\red : \pi \to \mathcal{P}(\pi)$

\begin{mathpar}
  \inferrule* [lab=Comm] { \textsf{match}( x_{src}, x_{trgt} ) } { x_{trgt}?(y)P \; | \; x_{src}!\langle {Q} \rangle \red P\{\quotep{Q}/y}\} }
  \and \\
  \inferrule* [lab=Par] {{P} \red {P}'} {{{P} | {Q}} \red {{P}' | {Q}}}
  \and
  \inferrule* [lab=Equiv]{{{P} \scong {P}'} \andalso {{P}' \red {Q}'} \andalso {{Q}' \scong {Q}}}{{P} \red {Q}}
\end{mathpar}

\begin{eqnarray*}
  match_{\equiv} (\quotep{P},\quotep{Q}) & := & P \equiv Q \\
  match_{\dagger}(\quotep{P},\quotep{Q}) & := & \forall R. P|Q \red^{*} R => R \red^{*} 0 \\
  match_{K}(\quotep{P},\quotep{Q}) & := & K \mbox{ for some context } K
\end{eqnarray*}

$u?(x)P | u!\langle Q \rangle \red P\{\quotep{Q}/x\}$

%We write $\wred$ for $\red^*$, and $P\red$ if $\exists Q $ such that $ P \red Q$.
We write $P\red$ if $\exists Q $ such that $ P \red Q$ and $P\not\red$, otherwise.

\section{Replication}

As mentioned before, it is known that replication (and hence
recursion) can be implemented in a higher-order process algebra
\cite{SangiorgiWalker}. As our first example of calculation with the
machinery thus far presented we give the construction explicitly in
the {\rhoc}.

\begin{eqnarray}
	D_{x} & := & \prefix{x}{y}{(\binpar{\outputp{x}{y}}{@{y}})} \nonumber\\
	\bangp_{x}{P} & := & \binpar{{x}!\langle{\binpar{D_{x}}{P}}\rangle}{D_{x}} \nonumber
\end{eqnarray}

\begin{eqnarray}
	\bangp_{x}{P} & & \nonumber\\
	=
	& {x}!\langle{(\prefix{x}{y}{(\outputp{x}{y} | @{y})) | P}}\rangle 
	      | \prefix{x}{y}{(\outputp{x}{y} | @{y})} & \nonumber\\
	\red
	& (\outputp{x}{y} | @{y})\substn{\quotep{(\prefix{x}{y}{(@{y} | \outputp{x}{y})) | P}}}{y} & \nonumber\\
	=
	& \outputp{x}{\quotep{(\prefix{x}{y}{(\outputp{x}{y} | @{y})) | P}}}
	  | {(\prefix{x}{y}{(\outputp{x}{y} | @{y})) | P}} & \nonumber\\
	\red
	& \ldots & \nonumber\\
	\red^*
	& P | P | \ldots & \nonumber
\end{eqnarray}

Of course, this encoding, as an implementation, runs away, unfolding
$\bangp{P}$ eagerly. A lazier and more implementable replication
operator, restricted to input-guarded processes, may be obtained as follows.

\begin{eqnarray}
\bangp{\prefix{u}{v}{P}} 
	:= 
	\binpar{\lift{x}{\prefix{u}{v}{(\binpar{D(x)}{P})}}}{D(x)} \nonumber
\end{eqnarray}

\begin{remark}
  Note that the lazier definition still does not deal with summation
  or mixed summation (i.e. sums over input and output). The reader is
  invited to construct definitions of replication that deal with these
  features. 

  Further, the definitions are parameterized in a name, $x$. Can you,
  gentle reader, make a definition that eliminates this parameter and
  guarantees no accidental interaction between the replication
  machinery and the process being replicated -- i.e. no accidental
  sharing of names used by the process to get its work done and the
  name(s) used by the replication to effect copying. This latter
  revision of the definition of replication is crucial to obtaining
  the expected identity $!!P \sim !P$.
\end{remark}

\begin{remark}\label{rem:paradoxical_combinator}
  The reader familiar with the lambda calculus will have noticed the
  similarity between $D$ and the paradoxical combinator.

  [Ed. note: the existence of this seems to suggest we have to be more
  restrictive on the set of processes and names we admit if we are to
  support no-cloning.]
\end{remark}

\subsubsection{Bisimulation}

The computational dynamics gives rise to another kind of equivalence,
the equivalence of computational behavior. As previously mentioned
this is typically captured \emph{via} some form of bisimulation.

% The notion we use in this paper is weak barbed bisimulation
% \cite{milner91polyadicpi}.

The notion we use in this paper is derived from weak barbed
bisimulation \cite{milner91polyadicpi}. 

\begin{definition}
An \emph{observation relation}, $\downarrow_{\mathcal N}$, over a set
of names, $\mathcal N$, is the smallest relation satisfying the rules
below.

\infrule[Out-barb]{y \in {\mathcal N}, \; x \nameeq y}
		  {\outputp{x}{v} \downarrow_{\mathcal N} x}
\infrule[Par-barb]{\mbox{$P\downarrow_{\mathcal N} x$ or $Q\downarrow_{\mathcal N} x$}}
		  {\binpar{P}{Q} \downarrow_{\mathcal N} x}

We write $P \Downarrow_{\mathcal N} x$ if there is $Q$ such that 
$P \wred Q$ and $Q \downarrow_{\mathcal N} x$.
\end{definition}

\begin{definition}
%\label{def.bbisim}
An  ${\mathcal N}$-\emph{barbed bisimulation} over a set of names, ${\mathcal N}$, is a symmetric binary relation 
${\mathcal S}_{\mathcal N}$ between agents such that $P\rel{S}_{\mathcal N}Q$ implies:
\begin{enumerate}
\item If $P \red P'$ then $Q \wred Q'$ and $P'\rel{S}_{\mathcal N} Q'$.
\item If $P\downarrow_{\mathcal N} x$, then $Q\Downarrow_{\mathcal N} x$.
\end{enumerate}
$P$ is ${\mathcal N}$-barbed bisimilar to $Q$, written
$P \wbbisim_{\mathcal N} Q$, if $P \rel{S}_{\mathcal N} Q$ for some ${\mathcal N}$-barbed bisimulation ${\mathcal S}_{\mathcal N}$.
\end{definition}

$\mathcal{R} \subseteq \pi \times \pi$

$P \mathcal{R} Q => \forall P'. P \red P' \Rightarrow \exists Q'. Q \red Q', P' \mathcal{R} Q'$

$P \vdash x \Rightarrow Q \vdash x$

\begin{mathpar}
  \inferrule*[lab=Out-barb]{x \nameeq y}{{y}!\langle{Q}\rangle \vdash x}
  \and
  \inferrule*[lab=Par-barb]{\mbox{$P\vdash x$ or $Q\vdash x$}}{\binpar{P}{Q} \vdash x}
\end{mathpar}

\subsubsection{Contexts}

One of the principle advantages of computational calculi like the
$\pi$-calculus is a well-defined notion of context,
contextual-equivalence and a correlation between
contextual-equivalence and notions of bisimulation. The notion of
context allows the decomposition of a process into (sub-)process and
its syntactic environment, its context. Thus, a context may be
thought of as a process with a ``hole'' (written $\Box$) in it. The
application of a context $M$ to a process $P$, written $M[P]$, is
tantamount to filling the hole in $M$ with $P$. In this paper we do
not need the full weight of this theory, but do make use of the notion
of context in the proof the main theorem. 

\begin{mathpar}
  \inferrule* [lab=summation] {} {{M_{M},M_{N}} \bc \Box \;|\; x.M_{A} \;|\; M_{M}+M_{N}}
  \and
  \inferrule* [lab=agent] {} {{M_{A}} \bc (\vec{x})M_{P} \;| \; \clift{P_0,\ldots,M_{P},\ldots,P_N}}
  \and \\
  \inferrule* [lab=process] {} {{M_{P}} \bc M_{N} \;| \;P|M_{P} }
\end{mathpar} 

\begin{mathpar}
  \inferrule* [lab=sychronization] {} {M_{N} \bc \Box \;|\; x?M_{F} \;|\; x!M_{C}}
  \and
  \inferrule* [lab=abstraction] {} {{M_{F}} \bc (x)M_{P} }
  \and
  \inferrule* [lab=concretion] {} {{M_{C}} \bc \langle M_{P} \rangle }
  \and \\
  \inferrule* [lab=process] {} {{M_{P}} \bc M_{N} \;| \;P|M_{P} }
\end{mathpar}

\begin{definition}[contextual application] Given a context $M$, and
  process $P$, we define the \emph{contextual application}, $M[P] :=
  M\{P/\Box\}$. That is, the contextual application of M to P is the
  substitution of $P$ for $\Box$ in $M$.
\end{definition}

$\meaningof{-} : L \to \mathcal{P}(\pi)$

\begin{mathpar}
  \inferrule* [lab=collection] {} {\meaningof{true} = \pi, \and \meaningof{~E} = \pi \setminus \meaningof{E}, \and \meaningof{E_{1} \& E_{2}} = \meaningof{E_{1}} \cap \meaningof{E_{2}}}
\end{mathpar}

\begin{mathpar}
  \inferrule* [lab=structure] {} {\meaningof{0} = \{ P \in \pi | P \equiv 0 \}, \and \\ \meaningof{E_1 | E_2} = \{ P \in \pi | P \equiv P_{1} | P_{2}, P_{1} \in \meaningof{E_{1}}, P_{2} \in \meaningof{E_2}\} }
\end{mathpar}

\begin{mathpar}
 \inferrule* [lab=behavior] {} {\meaningof{\langle a?b \rangle E} = \{ P \in \pi | P \equiv Q | u?(y)P', \\ \and \\\\ \and \\ \;\;\; u \in \meaningof{a}, \forall z.P'\{z/y\} \in \meaningof{E\{z/b\}}\}, \and \\ \meaningof{a!E} = \{ P \in \pi | P \equiv Q | x!\langle P' \rangle, x \in \meaningof{a} P' \in \meaningof{E}\} }
\end{mathpar}

\begin{mathpar}
 \inferrule* [lab=nominal] {} {\meaningof{\quotep{E}} = \{ \quotep{P} \in \quotep{\pi} | P \in \meaningof{E} \}, \and \meaningof{\quotep{P}} = \{ \quotep{Q} \in \quotep{\pi} | P \equiv Q \} \and \\ \meaningof{@\quotep{E}} = \{ P \in \pi | P \equiv @x, x \in \meaningof{E} \}}
\end{mathpar}

\begin{eqnarray*}
  \\
  \meaningof{-} : TS \to ST
\end{eqnarray*}

\begin{eqnarray*}
  \\
  L : TS \to ST
\end{eqnarray*}

\begin{eqnarray*}
  \\
  P \models E \iff P \in \meaningof{E}
\end{eqnarray*}

\begin{eqnarray*}
  P \approx_{L} Q \iff \forall E \in L. P \models E \iff Q \models E
\end{eqnarray*}

\begin{eqnarray*}
  P \approx_{K} Q
\end{eqnarray*}

\begin{eqnarray*}
  P \approx Q
\end{eqnarray*}

$\approx_{K} = \approx = \approx_{L}$

\subsubsection{Contextual duality}

Note that contexts extend the quotation operation to a family of
operations from processes to names. Given a context, $M$, we can
define a \emph{nominal context}, $\quotep{M}$ by $\quotep{M}[P] :=
\quotep{M[P]}$. To foreshadow what is to come we observe that these
operations enjoy a duality with processes very much like the duality
between vectors and maps from vectors to scalars.

Further, because the calculus is essentially higher-order, we have a
correspondence between contexts and processes. More specifically,
given a name $x$ and a context $M$ we can construct $M^{*}_{x}$ such
that 

\begin{mathpar}
  M^{*}_{x} | \lift{x}{P} \red M[P]
\end{mathpar}

namely,

\begin{mathpar}
  M^{*}_{x} := x?(u).M[\dropn{u}]
\end{mathpar}

The dependence of $M^{*}_{x}$ on a name makes it an abstraction, 

\begin{mathpar}
  M^{*} := (x)x?(u).M[\dropn{u}]
\end{mathpar}

\subsection{Additional notation}

It will sometimes be convenient to denote the process a name
quotes. We already have the notation $x = \quotep{P}$, but it will be
convenient to introduce an alternate notation, $\procn{x}$, when we
want to emphasize the connection to the use of the name. Note that, by
virtue of name equivalence, $\quotep{\procn{x}} \nameeq x$; so, the
notation is consistent with previous definitions.

Further, because names have structure it is possible to effect
substitutions on the basis of that structure. This means we need to
upgrade our notation for substitutions, which we accomplish by
adapting comprehension notation. Thus,

\begin{mathpar}
  P\{ y / x : x \in S \}
\end{mathpar}

is interpreted to mean the process derived from P by replacing (in a
capture-avoiding manner) each occurrence of $x$ in $S$ by $y$. For example,

\begin{mathpar}
  P\{ \quotep{\procn{x}|\procn{x}} / x : x \in \freenames{P} \}
\end{mathpar}

will replace each (occurrence) of a free name $x$ in $P$ by
$\quotep{\procn{x}|\procn{x}}$.

Also, we will avail ourselves of the notation $x^{L}$ and $x^{R}$ to
denote injections of a name into disjoint copies of the name
space. There are numerous ways to accomplish this. One example can be
found in \cite{MeredithR05}. This notation overloads to vectors of
names: $\vec{x}^{\pi} := (x_{i}^{\pi} \; : \; 0 \leq i < |\vec{x}| )$ where $\pi \in \{L,R\}$.

We also use $P^{\Box} := P|\Box$.

In \cite{MeredithR05} an interpretation of the new operator is
given. It turns out that there are several possible interpretations
all enjoying the requisite algebraic properties of the operator (see
\cite{milner91polyadicpi}). We will therefore make liberal use of
$(\nu\; \vec{x})P$.

% subsection the_syntax_and_semantics_of_the_notation_system (end)   

\input{qm2pi.qmops} 

\input{qm2pi.sterngerlach} 

\input{qm2pi.metric} 

% section concurrent_process_calculi (end)

%\input{qm2pi.proofsketch}

% section proof sketch (end)

%\input{qm2pi.slviaknots} 

% section spatial logic via knots (end)

\input{qm2pi.conclusion}

% section conclusion (end)

%\input{qm2pi.dtcodes} 

% section wiring algorithm (end)

\input{qm2pi.ack} 

% section acknowledgments (end)

\newpage


\bibliographystyle{plain}   
\bibliography{../../biblios/main.bib}

\input{qm2pi.rhodetails}

\end{document}

 

% section notation (end)

\input{qm2pi.process.calculi} 

% section concurrent_process_calculi_and_spatial_logics_ (end)
    
%\documentclass[12pt]{llncs}
%\documentclass{jktr}

\usepackage[pdftex]{hyperref}                   
\usepackage {listings}
\usepackage {mathpartir}
\usepackage{bcprules}
%\usepackage{listings}
                       
\usepackage{graphicx} 
%\usepackage[margins=2.5cm,nohead,nofoot]{geometry}
%\usepackage{geometry}
\usepackage{amsfonts}
\usepackage{amstext}
\usepackage{latexsym}
\usepackage{amssymb}
\usepackage{color}


%\include{myPreamble}
\include{qm2pi.local} 

%\ifpdf
%\usepackage[pdftex]{graphicx}
%\else
%\usepackage{graphicx}
%\fi

 % \ifpdf
%  \usepackage{pdfsync}
%  \if


%\title{Brief Article}
%\author{David F. Snyder}
%\author{L.G. Meredith}

%\address{Dept. of Math., Texas State University--San Marcos, San Marcos, TX 78666}
       
\pagestyle{empty}


\begin{document}

\lstset{language=[Objective]Caml,frame=shadowbox}

\input{qm2pi.front}

% section front matter (end)

\input{qm2pi.intro} 
 
% section introduction (end)

% \input{qm2pi.knotations} 

% section notation (end)

\input{qm2pi.process.calculi} 

% section concurrent_process_calculi_and_spatial_logics_ (end)
    
%\input{qm2pi.knots2pi} 

%\input{qm2pi.trefoil} 

%\input{qm2pi.mainthm} 

% subsection basic_interpretation (end)

%\input{qm2pi.rho.presentation} 
\subsection{The syntax and semantics of the notation system}\label{sub:the_syntax_and_semantics_of_the_notation_system} % (fold)

We now summarize a technical presentation of the calculus that
embodies our theory of dynamics. The typical presentation of such a
calculus follows the style of giving generators and relations on
them. The grammar, below, describing term constructors, freely
generates the set of processes, $\Proc$. This set is then quotiented
by a relation known as structural congruence and it is over this set
that the notion of dynamics is expressed. This presentation is
essentially that of \cite{MeredithR05} with the addition of
polyadicity and summation. For readability we have relegated some of
the technical subtleties to an appendix.

\subsubsection{Process grammar}\label{subsub:process_grammar}

\begin{mathpar}
  \inferrule* [lab=synchronization] {} {{M} \bc \pzero \;|\; x?F \;|\; x!C }
  \and
  \inferrule* [lab=abstraction] {} {{F} \bc (x)P}
  \and
  \inferrule* [lab=concretion] {} {{C} \bc \langle Q \rangle}
  \and
  \inferrule* [lab=process] {} {{P,Q} \bc M \;| \;P|Q \;|\; @{x}}
  \and
  \inferrule* [lab=name] {} {{x} \bc \quotep{P}}
\end{mathpar} 

Note that $\vec{x}$ (resp. $\vec{P}$) denotes a vector of names
(resp. processes) of length $|\vec{x}|$ (resp. $|\vec{P}|$). We adopt
the following useful abbreviations.

\begin{mathpar}
   x?(\vec{y}).P := x.(\vec{y})P \and  x\clift{\vec{P}} := x.\clift{\vec{P}}
   \and x!(y) := \lift{x}{\dropn{y}}
   \and \Pi_{i=0}^{n-1}P_i := P_0 | \ldots | P_{n-1}
\end{mathpar}

\subsubsection{Structural congruence}

\paragraph{Free and bound names and alpha-equivalence.} At the
core of structural equivalence is alpha-equivalence which identifies
process that are the same up to a change of variable. Formally, we
recognize the distinction between free and bound names. The free names
of a process, $\freenames{P}$, may be calculated recursively as
follows:

\begin{mathpar}
\freenames{\pzero} := \emptyset
  \and \\
  \freenames{x?(y).P} := \{ x \} \cup (\freenames{P} \setminus \{ y \})
  \and 
  \freenames{x!\langle P \rangle} := \{ x \} \cup \{ P \} 
  \and \\
  \freenames{P|Q} := \freenames{P} \cup \freenames{Q}
  \and \\
  \freenames{@{x}} := \{ x \}
\end{mathpar}

$\pi$
$\quotep{\pi}$

$\freenames{-} : \pi \to \mathcal{P}(\quotep{\pi})$

\begin{eqnarray*}
  \freenames{\pzero} & := & \emptyset \\
  \freenames{x?(y).P} & := & \{ x \} \cup (\freenames{P} \setminus \{ y \}) \\
  \freenames{x!\langle P \rangle} & := & \{ x \} \cup \{ P \} \\
  \freenames{P|Q} & := & \freenames{P} \cup \freenames{Q} \\
  \freenames{\dropn{x}} & := & \{ x \}
\end{eqnarray*}

The bound names of a process, $\boundnames{P}$, are those names occurring in $P$
that are not free. For example, in $x?(y).0$, the name $x$ is free, while $y$ is bound.

\begin{mathpar}
  \inferrule* [lab=monoidal-laws] {} { P|Q \equiv Q|P \and P|0 \equiv P \and P|(Q|R) \equiv (P|Q)|R }
\end{mathpar}

\begin{mathpar}
  \inferrule* [lab=alpha-equivalence] {} { (x)P \equiv (y)P\{y/x\} \and y \not\in \freenames{P} }
\end{mathpar}

\begin{definition}
Then two processes, $P,Q$, are alpha-equivalent if $P = Q\{\vec{y}/\vec{x}\}$ for
some $\vec{x} \in \boundnames{Q},\vec{y} \in \boundnames{P}$, where $Q\{\vec{y}/\vec{x}\}$
denotes the capture-avoiding substitution of $\vec{y}$ for $\vec{x}$ in $Q$.
\end{definition}

\begin{definition}
  The {\em structural congruence} \cite{SangiorgiWalker} , $\equiv$,
  between processes is the least congruence containing
  alpha-equivalence, satisfying the abelian monoid laws
  (associativity, commutativity and $\pzero$ as identity) for parallel
  composition $|$ and for summation $+$.
\end{definition}

\subsection{Name equivalence}

We take name equivalence, written $\nameeq$, to be the smallest
equivalence relation generated by the following rules.

\begin{mathpar}
\inferrule*[lab=Quote-drop]
{ }
{ \quotep{@{x}} \nameeq x }

\inferrule*[lab=Struct-equiv]
{ P \scong Q }
{ \quotep{P} \nameeq \quotep{Q} }
\end{mathpar}

The astute reader will have noticed that the mutual recursion of names
and processes imposes a mutual recursion on alpha-equivalence and
structural equivalence via name-equivalence. Fortunately, all of this
works out pleasantly and we may calculate in the natural way, free of
concern. The reader interested in the details is referred to the
appendix \ref{appendix:rho_details}.

\subsection{Substitution}

We use $\Proc$ for the set of processes, $\QProc$ for the set of
names, and $\id{\{}\vec{y} / \vec{x} \id{\}}$ to denote partial maps,
$s : \QProc \rightarrow \QProc$. A map, $s$ lifts, uniquely, to a map
on process terms, $\widehat{s} : \Proc \rightarrow \Proc$ by the
following equations.

\begin{mathpar}
  (0) \psubstp{Q}{P} := 0 \\
  (R \juxtap S) \psubstp{Q}{P}
  :=    
  (R)\psubstp{Q}{P} \juxtap (S) \psubstp{Q}{P} \\
  (x?(y).R) \psubstp{Q}{P}    
  :=    
  (x)\substp{Q}{P} (z)\concat( (R \psubstn{z}{y}) \psubstp{Q}{P} ) \\
  (\lift{x}{R}) \psubstp{Q}{P}  
  :=
  \lift{(x)\substp{Q}{P}}{ R \psubstp{Q}{P} } \\
%   (\dropn{x})  \psubstp{Q}{P}       
%   := 
%   \left\{ 
%     \begin{array}{ccc} 
%       \dropn{\quotep{Q}} & & x \nameeq \quotep{P} \\
%       \dropn{x} & & otherwise \\
%     \end{array}
%   \right. 
  (\dropn{x})  \psubstp{Q}{P}       
  := 
  \left\{ 
    \begin{array}{ccc} 
      Q & & x \nameeq \quotep{P} \\
      \dropn{x} & & otherwise \\
    \end{array}
  \right.
\end{mathpar}
 

where

\begin{eqnarray}
  (x)\id{\{} \lpquote Q \rpquote / \lpquote P \rpquote \id{\}}            = 
  \left\{ 
    \begin{array}{ccc}
      \lpquote Q \rpquote & & x \nameeq \lpquote P \rpquote \\
      x & & otherwise \\
    \end{array}
  \right. \nonumber
\end{eqnarray}

and $z$ is chosen distinct from $\quotep{P}$, $\quotep{Q}$, the free
names in $Q$, and all the names in $R$. Our $\alpha$-equivalence will
be built in the standard way from this substitution.

\begin{remark}\label{rem:no_self_referential_names}
  One consequence of these definitions is that $\forall P. \quotep{P}
  \not\in \freenames{P}$.
\end{remark}

\subsection{ Dynamic quote: an example }

Anticipating something of what's to come, consider applying the
substitution, $\widehat{\id{\{}u / z \id{\}}}$, to the following pair
of processes, $\lift{w}{y!(z)}$ and $w[ \lpquote y!(z) \rpquote ]$.

\begin{eqnarray}
	\lift{w}{y!(z)}\widehat{\id{\{}u / z \id{\}}}
		& = &
		\lift{w}{y!(u)} \nonumber\\
	w[ \lpquote y!(z) \rpquote ] \widehat{ \id{\{}u / z \id{\}} }
		& = &
		w[ \lpquote y!(z) \rpquote ] \nonumber
\end{eqnarray}

Because the body of the process between quotes is impervious to
substitution, we get radically different answers. In fact, by
examining the first process in an input context,
e.g. $x?(z).\lift{w}{y!(z)}$, we see that the process under the lift
operator may be shaped by prefixed inputs binding a name inside it. In
this sense, the lift operator will be seen as a way to dynamically
construct processes before reifying them as names.

Finally equipped with these standard features we can present the
dynamics of the calculus.

\subsubsection{Operational semantics} 

Finally, we introduce the computational dynamics. What marks these
algebras as distinct from other more traditionally studied algebraic
structures, e.g. vector spaces or polynomial rings, is the manner in
which dynamics is captured. In traditional structures, dynamics is typically
expressed through morphisms between such structures, as in linear maps
between vector spaces or morphisms between rings. In algebras
associated with the semantics of computation, the dynamics is
expressed as part of the algebraic structure itself, through a
reduction reduction relation typically denoted by $\red$. Below, we
give a recursive presentation of this relation for the calculus used
in the encoding.

$\red \subseteq \pi \times \pi$
$\red : \pi \to \mathcal{P}(\pi)$

\begin{mathpar}
  \inferrule* [lab=Comm] { \textsf{match}( x_{src}, x_{trgt} ) } { x_{trgt}?(y)P \; | \; x_{src}!\langle {Q} \rangle \red P\{\quotep{Q}/y}\} }
  \and \\
  \inferrule* [lab=Par] {{P} \red {P}'} {{{P} | {Q}} \red {{P}' | {Q}}}
  \and
  \inferrule* [lab=Equiv]{{{P} \scong {P}'} \andalso {{P}' \red {Q}'} \andalso {{Q}' \scong {Q}}}{{P} \red {Q}}
\end{mathpar}

\begin{eqnarray*}
  match_{\equiv} (\quotep{P},\quotep{Q}) & := & P \equiv Q \\
  match_{\dagger}(\quotep{P},\quotep{Q}) & := & \forall R. P|Q \red^{*} R => R \red^{*} 0 \\
  match_{K}(\quotep{P},\quotep{Q}) & := & K \mbox{ for some context } K
\end{eqnarray*}

$u?(x)P | u!\langle Q \rangle \red P\{\quotep{Q}/x\}$

%We write $\wred$ for $\red^*$, and $P\red$ if $\exists Q $ such that $ P \red Q$.
We write $P\red$ if $\exists Q $ such that $ P \red Q$ and $P\not\red$, otherwise.

\section{Replication}

As mentioned before, it is known that replication (and hence
recursion) can be implemented in a higher-order process algebra
\cite{SangiorgiWalker}. As our first example of calculation with the
machinery thus far presented we give the construction explicitly in
the {\rhoc}.

\begin{eqnarray}
	D_{x} & := & \prefix{x}{y}{(\binpar{\outputp{x}{y}}{@{y}})} \nonumber\\
	\bangp_{x}{P} & := & \binpar{{x}!\langle{\binpar{D_{x}}{P}}\rangle}{D_{x}} \nonumber
\end{eqnarray}

\begin{eqnarray}
	\bangp_{x}{P} & & \nonumber\\
	=
	& {x}!\langle{(\prefix{x}{y}{(\outputp{x}{y} | @{y})) | P}}\rangle 
	      | \prefix{x}{y}{(\outputp{x}{y} | @{y})} & \nonumber\\
	\red
	& (\outputp{x}{y} | @{y})\substn{\quotep{(\prefix{x}{y}{(@{y} | \outputp{x}{y})) | P}}}{y} & \nonumber\\
	=
	& \outputp{x}{\quotep{(\prefix{x}{y}{(\outputp{x}{y} | @{y})) | P}}}
	  | {(\prefix{x}{y}{(\outputp{x}{y} | @{y})) | P}} & \nonumber\\
	\red
	& \ldots & \nonumber\\
	\red^*
	& P | P | \ldots & \nonumber
\end{eqnarray}

Of course, this encoding, as an implementation, runs away, unfolding
$\bangp{P}$ eagerly. A lazier and more implementable replication
operator, restricted to input-guarded processes, may be obtained as follows.

\begin{eqnarray}
\bangp{\prefix{u}{v}{P}} 
	:= 
	\binpar{\lift{x}{\prefix{u}{v}{(\binpar{D(x)}{P})}}}{D(x)} \nonumber
\end{eqnarray}

\begin{remark}
  Note that the lazier definition still does not deal with summation
  or mixed summation (i.e. sums over input and output). The reader is
  invited to construct definitions of replication that deal with these
  features. 

  Further, the definitions are parameterized in a name, $x$. Can you,
  gentle reader, make a definition that eliminates this parameter and
  guarantees no accidental interaction between the replication
  machinery and the process being replicated -- i.e. no accidental
  sharing of names used by the process to get its work done and the
  name(s) used by the replication to effect copying. This latter
  revision of the definition of replication is crucial to obtaining
  the expected identity $!!P \sim !P$.
\end{remark}

\begin{remark}\label{rem:paradoxical_combinator}
  The reader familiar with the lambda calculus will have noticed the
  similarity between $D$ and the paradoxical combinator.

  [Ed. note: the existence of this seems to suggest we have to be more
  restrictive on the set of processes and names we admit if we are to
  support no-cloning.]
\end{remark}

\subsubsection{Bisimulation}

The computational dynamics gives rise to another kind of equivalence,
the equivalence of computational behavior. As previously mentioned
this is typically captured \emph{via} some form of bisimulation.

% The notion we use in this paper is weak barbed bisimulation
% \cite{milner91polyadicpi}.

The notion we use in this paper is derived from weak barbed
bisimulation \cite{milner91polyadicpi}. 

\begin{definition}
An \emph{observation relation}, $\downarrow_{\mathcal N}$, over a set
of names, $\mathcal N$, is the smallest relation satisfying the rules
below.

\infrule[Out-barb]{y \in {\mathcal N}, \; x \nameeq y}
		  {\outputp{x}{v} \downarrow_{\mathcal N} x}
\infrule[Par-barb]{\mbox{$P\downarrow_{\mathcal N} x$ or $Q\downarrow_{\mathcal N} x$}}
		  {\binpar{P}{Q} \downarrow_{\mathcal N} x}

We write $P \Downarrow_{\mathcal N} x$ if there is $Q$ such that 
$P \wred Q$ and $Q \downarrow_{\mathcal N} x$.
\end{definition}

\begin{definition}
%\label{def.bbisim}
An  ${\mathcal N}$-\emph{barbed bisimulation} over a set of names, ${\mathcal N}$, is a symmetric binary relation 
${\mathcal S}_{\mathcal N}$ between agents such that $P\rel{S}_{\mathcal N}Q$ implies:
\begin{enumerate}
\item If $P \red P'$ then $Q \wred Q'$ and $P'\rel{S}_{\mathcal N} Q'$.
\item If $P\downarrow_{\mathcal N} x$, then $Q\Downarrow_{\mathcal N} x$.
\end{enumerate}
$P$ is ${\mathcal N}$-barbed bisimilar to $Q$, written
$P \wbbisim_{\mathcal N} Q$, if $P \rel{S}_{\mathcal N} Q$ for some ${\mathcal N}$-barbed bisimulation ${\mathcal S}_{\mathcal N}$.
\end{definition}

$\mathcal{R} \subseteq \pi \times \pi$

$P \mathcal{R} Q => \forall P'. P \red P' \Rightarrow \exists Q'. Q \red Q', P' \mathcal{R} Q'$

$P \vdash x \Rightarrow Q \vdash x$

\begin{mathpar}
  \inferrule*[lab=Out-barb]{x \nameeq y}{{y}!\langle{Q}\rangle \vdash x}
  \and
  \inferrule*[lab=Par-barb]{\mbox{$P\vdash x$ or $Q\vdash x$}}{\binpar{P}{Q} \vdash x}
\end{mathpar}

\subsubsection{Contexts}

One of the principle advantages of computational calculi like the
$\pi$-calculus is a well-defined notion of context,
contextual-equivalence and a correlation between
contextual-equivalence and notions of bisimulation. The notion of
context allows the decomposition of a process into (sub-)process and
its syntactic environment, its context. Thus, a context may be
thought of as a process with a ``hole'' (written $\Box$) in it. The
application of a context $M$ to a process $P$, written $M[P]$, is
tantamount to filling the hole in $M$ with $P$. In this paper we do
not need the full weight of this theory, but do make use of the notion
of context in the proof the main theorem. 

\begin{mathpar}
  \inferrule* [lab=summation] {} {{M_{M},M_{N}} \bc \Box \;|\; x.M_{A} \;|\; M_{M}+M_{N}}
  \and
  \inferrule* [lab=agent] {} {{M_{A}} \bc (\vec{x})M_{P} \;| \; \clift{P_0,\ldots,M_{P},\ldots,P_N}}
  \and \\
  \inferrule* [lab=process] {} {{M_{P}} \bc M_{N} \;| \;P|M_{P} }
\end{mathpar} 

\begin{mathpar}
  \inferrule* [lab=sychronization] {} {M_{N} \bc \Box \;|\; x?M_{F} \;|\; x!M_{C}}
  \and
  \inferrule* [lab=abstraction] {} {{M_{F}} \bc (x)M_{P} }
  \and
  \inferrule* [lab=concretion] {} {{M_{C}} \bc \langle M_{P} \rangle }
  \and \\
  \inferrule* [lab=process] {} {{M_{P}} \bc M_{N} \;| \;P|M_{P} }
\end{mathpar}

\begin{definition}[contextual application] Given a context $M$, and
  process $P$, we define the \emph{contextual application}, $M[P] :=
  M\{P/\Box\}$. That is, the contextual application of M to P is the
  substitution of $P$ for $\Box$ in $M$.
\end{definition}

$\meaningof{-} : L \to \mathcal{P}(\pi)$

\begin{mathpar}
  \inferrule* [lab=collection] {} {\meaningof{true} = \pi, \and \meaningof{~E} = \pi \setminus \meaningof{E}, \and \meaningof{E_{1} \& E_{2}} = \meaningof{E_{1}} \cap \meaningof{E_{2}}}
\end{mathpar}

\begin{mathpar}
  \inferrule* [lab=structure] {} {\meaningof{0} = \{ P \in \pi | P \equiv 0 \}, \and \\ \meaningof{E_1 | E_2} = \{ P \in \pi | P \equiv P_{1} | P_{2}, P_{1} \in \meaningof{E_{1}}, P_{2} \in \meaningof{E_2}\} }
\end{mathpar}

\begin{mathpar}
 \inferrule* [lab=behavior] {} {\meaningof{\langle a?b \rangle E} = \{ P \in \pi | P \equiv Q | u?(y)P', \\ \and \\\\ \and \\ \;\;\; u \in \meaningof{a}, \forall z.P'\{z/y\} \in \meaningof{E\{z/b\}}\}, \and \\ \meaningof{a!E} = \{ P \in \pi | P \equiv Q | x!\langle P' \rangle, x \in \meaningof{a} P' \in \meaningof{E}\} }
\end{mathpar}

\begin{mathpar}
 \inferrule* [lab=nominal] {} {\meaningof{\quotep{E}} = \{ \quotep{P} \in \quotep{\pi} | P \in \meaningof{E} \}, \and \meaningof{\quotep{P}} = \{ \quotep{Q} \in \quotep{\pi} | P \equiv Q \} \and \\ \meaningof{@\quotep{E}} = \{ P \in \pi | P \equiv @x, x \in \meaningof{E} \}}
\end{mathpar}

\begin{eqnarray*}
  \\
  \meaningof{-} : TS \to ST
\end{eqnarray*}

\begin{eqnarray*}
  \\
  L : TS \to ST
\end{eqnarray*}

\begin{eqnarray*}
  \\
  P \models E \iff P \in \meaningof{E}
\end{eqnarray*}

\begin{eqnarray*}
  P \approx_{L} Q \iff \forall E \in L. P \models E \iff Q \models E
\end{eqnarray*}

\begin{eqnarray*}
  P \approx_{K} Q
\end{eqnarray*}

\begin{eqnarray*}
  P \approx Q
\end{eqnarray*}

$\approx_{K} = \approx = \approx_{L}$

\subsubsection{Contextual duality}

Note that contexts extend the quotation operation to a family of
operations from processes to names. Given a context, $M$, we can
define a \emph{nominal context}, $\quotep{M}$ by $\quotep{M}[P] :=
\quotep{M[P]}$. To foreshadow what is to come we observe that these
operations enjoy a duality with processes very much like the duality
between vectors and maps from vectors to scalars.

Further, because the calculus is essentially higher-order, we have a
correspondence between contexts and processes. More specifically,
given a name $x$ and a context $M$ we can construct $M^{*}_{x}$ such
that 

\begin{mathpar}
  M^{*}_{x} | \lift{x}{P} \red M[P]
\end{mathpar}

namely,

\begin{mathpar}
  M^{*}_{x} := x?(u).M[\dropn{u}]
\end{mathpar}

The dependence of $M^{*}_{x}$ on a name makes it an abstraction, 

\begin{mathpar}
  M^{*} := (x)x?(u).M[\dropn{u}]
\end{mathpar}

\subsection{Additional notation}

It will sometimes be convenient to denote the process a name
quotes. We already have the notation $x = \quotep{P}$, but it will be
convenient to introduce an alternate notation, $\procn{x}$, when we
want to emphasize the connection to the use of the name. Note that, by
virtue of name equivalence, $\quotep{\procn{x}} \nameeq x$; so, the
notation is consistent with previous definitions.

Further, because names have structure it is possible to effect
substitutions on the basis of that structure. This means we need to
upgrade our notation for substitutions, which we accomplish by
adapting comprehension notation. Thus,

\begin{mathpar}
  P\{ y / x : x \in S \}
\end{mathpar}

is interpreted to mean the process derived from P by replacing (in a
capture-avoiding manner) each occurrence of $x$ in $S$ by $y$. For example,

\begin{mathpar}
  P\{ \quotep{\procn{x}|\procn{x}} / x : x \in \freenames{P} \}
\end{mathpar}

will replace each (occurrence) of a free name $x$ in $P$ by
$\quotep{\procn{x}|\procn{x}}$.

Also, we will avail ourselves of the notation $x^{L}$ and $x^{R}$ to
denote injections of a name into disjoint copies of the name
space. There are numerous ways to accomplish this. One example can be
found in \cite{MeredithR05}. This notation overloads to vectors of
names: $\vec{x}^{\pi} := (x_{i}^{\pi} \; : \; 0 \leq i < |\vec{x}| )$ where $\pi \in \{L,R\}$.

We also use $P^{\Box} := P|\Box$.

In \cite{MeredithR05} an interpretation of the new operator is
given. It turns out that there are several possible interpretations
all enjoying the requisite algebraic properties of the operator (see
\cite{milner91polyadicpi}). We will therefore make liberal use of
$(\nu\; \vec{x})P$.

% subsection the_syntax_and_semantics_of_the_notation_system (end)   

\input{qm2pi.qmops} 

\input{qm2pi.sterngerlach} 

\input{qm2pi.metric} 

% section concurrent_process_calculi (end)

%\input{qm2pi.proofsketch}

% section proof sketch (end)

%\input{qm2pi.slviaknots} 

% section spatial logic via knots (end)

\input{qm2pi.conclusion}

% section conclusion (end)

%\input{qm2pi.dtcodes} 

% section wiring algorithm (end)

\input{qm2pi.ack} 

% section acknowledgments (end)

\newpage


\bibliographystyle{plain}   
\bibliography{../../biblios/main.bib}

\input{qm2pi.rhodetails}

\end{document}

 

%\documentclass[12pt]{llncs}
%\documentclass{jktr}

\usepackage[pdftex]{hyperref}                   
\usepackage {listings}
\usepackage {mathpartir}
\usepackage{bcprules}
%\usepackage{listings}
                       
\usepackage{graphicx} 
%\usepackage[margins=2.5cm,nohead,nofoot]{geometry}
%\usepackage{geometry}
\usepackage{amsfonts}
\usepackage{amstext}
\usepackage{latexsym}
\usepackage{amssymb}
\usepackage{color}


%\include{myPreamble}
\include{qm2pi.local} 

%\ifpdf
%\usepackage[pdftex]{graphicx}
%\else
%\usepackage{graphicx}
%\fi

 % \ifpdf
%  \usepackage{pdfsync}
%  \if


%\title{Brief Article}
%\author{David F. Snyder}
%\author{L.G. Meredith}

%\address{Dept. of Math., Texas State University--San Marcos, San Marcos, TX 78666}
       
\pagestyle{empty}


\begin{document}

\lstset{language=[Objective]Caml,frame=shadowbox}

\input{qm2pi.front}

% section front matter (end)

\input{qm2pi.intro} 
 
% section introduction (end)

% \input{qm2pi.knotations} 

% section notation (end)

\input{qm2pi.process.calculi} 

% section concurrent_process_calculi_and_spatial_logics_ (end)
    
%\input{qm2pi.knots2pi} 

%\input{qm2pi.trefoil} 

%\input{qm2pi.mainthm} 

% subsection basic_interpretation (end)

%\input{qm2pi.rho.presentation} 
\subsection{The syntax and semantics of the notation system}\label{sub:the_syntax_and_semantics_of_the_notation_system} % (fold)

We now summarize a technical presentation of the calculus that
embodies our theory of dynamics. The typical presentation of such a
calculus follows the style of giving generators and relations on
them. The grammar, below, describing term constructors, freely
generates the set of processes, $\Proc$. This set is then quotiented
by a relation known as structural congruence and it is over this set
that the notion of dynamics is expressed. This presentation is
essentially that of \cite{MeredithR05} with the addition of
polyadicity and summation. For readability we have relegated some of
the technical subtleties to an appendix.

\subsubsection{Process grammar}\label{subsub:process_grammar}

\begin{mathpar}
  \inferrule* [lab=synchronization] {} {{M} \bc \pzero \;|\; x?F \;|\; x!C }
  \and
  \inferrule* [lab=abstraction] {} {{F} \bc (x)P}
  \and
  \inferrule* [lab=concretion] {} {{C} \bc \langle Q \rangle}
  \and
  \inferrule* [lab=process] {} {{P,Q} \bc M \;| \;P|Q \;|\; @{x}}
  \and
  \inferrule* [lab=name] {} {{x} \bc \quotep{P}}
\end{mathpar} 

Note that $\vec{x}$ (resp. $\vec{P}$) denotes a vector of names
(resp. processes) of length $|\vec{x}|$ (resp. $|\vec{P}|$). We adopt
the following useful abbreviations.

\begin{mathpar}
   x?(\vec{y}).P := x.(\vec{y})P \and  x\clift{\vec{P}} := x.\clift{\vec{P}}
   \and x!(y) := \lift{x}{\dropn{y}}
   \and \Pi_{i=0}^{n-1}P_i := P_0 | \ldots | P_{n-1}
\end{mathpar}

\subsubsection{Structural congruence}

\paragraph{Free and bound names and alpha-equivalence.} At the
core of structural equivalence is alpha-equivalence which identifies
process that are the same up to a change of variable. Formally, we
recognize the distinction between free and bound names. The free names
of a process, $\freenames{P}$, may be calculated recursively as
follows:

\begin{mathpar}
\freenames{\pzero} := \emptyset
  \and \\
  \freenames{x?(y).P} := \{ x \} \cup (\freenames{P} \setminus \{ y \})
  \and 
  \freenames{x!\langle P \rangle} := \{ x \} \cup \{ P \} 
  \and \\
  \freenames{P|Q} := \freenames{P} \cup \freenames{Q}
  \and \\
  \freenames{@{x}} := \{ x \}
\end{mathpar}

$\pi$
$\quotep{\pi}$

$\freenames{-} : \pi \to \mathcal{P}(\quotep{\pi})$

\begin{eqnarray*}
  \freenames{\pzero} & := & \emptyset \\
  \freenames{x?(y).P} & := & \{ x \} \cup (\freenames{P} \setminus \{ y \}) \\
  \freenames{x!\langle P \rangle} & := & \{ x \} \cup \{ P \} \\
  \freenames{P|Q} & := & \freenames{P} \cup \freenames{Q} \\
  \freenames{\dropn{x}} & := & \{ x \}
\end{eqnarray*}

The bound names of a process, $\boundnames{P}$, are those names occurring in $P$
that are not free. For example, in $x?(y).0$, the name $x$ is free, while $y$ is bound.

\begin{mathpar}
  \inferrule* [lab=monoidal-laws] {} { P|Q \equiv Q|P \and P|0 \equiv P \and P|(Q|R) \equiv (P|Q)|R }
\end{mathpar}

\begin{mathpar}
  \inferrule* [lab=alpha-equivalence] {} { (x)P \equiv (y)P\{y/x\} \and y \not\in \freenames{P} }
\end{mathpar}

\begin{definition}
Then two processes, $P,Q$, are alpha-equivalent if $P = Q\{\vec{y}/\vec{x}\}$ for
some $\vec{x} \in \boundnames{Q},\vec{y} \in \boundnames{P}$, where $Q\{\vec{y}/\vec{x}\}$
denotes the capture-avoiding substitution of $\vec{y}$ for $\vec{x}$ in $Q$.
\end{definition}

\begin{definition}
  The {\em structural congruence} \cite{SangiorgiWalker} , $\equiv$,
  between processes is the least congruence containing
  alpha-equivalence, satisfying the abelian monoid laws
  (associativity, commutativity and $\pzero$ as identity) for parallel
  composition $|$ and for summation $+$.
\end{definition}

\subsection{Name equivalence}

We take name equivalence, written $\nameeq$, to be the smallest
equivalence relation generated by the following rules.

\begin{mathpar}
\inferrule*[lab=Quote-drop]
{ }
{ \quotep{@{x}} \nameeq x }

\inferrule*[lab=Struct-equiv]
{ P \scong Q }
{ \quotep{P} \nameeq \quotep{Q} }
\end{mathpar}

The astute reader will have noticed that the mutual recursion of names
and processes imposes a mutual recursion on alpha-equivalence and
structural equivalence via name-equivalence. Fortunately, all of this
works out pleasantly and we may calculate in the natural way, free of
concern. The reader interested in the details is referred to the
appendix \ref{appendix:rho_details}.

\subsection{Substitution}

We use $\Proc$ for the set of processes, $\QProc$ for the set of
names, and $\id{\{}\vec{y} / \vec{x} \id{\}}$ to denote partial maps,
$s : \QProc \rightarrow \QProc$. A map, $s$ lifts, uniquely, to a map
on process terms, $\widehat{s} : \Proc \rightarrow \Proc$ by the
following equations.

\begin{mathpar}
  (0) \psubstp{Q}{P} := 0 \\
  (R \juxtap S) \psubstp{Q}{P}
  :=    
  (R)\psubstp{Q}{P} \juxtap (S) \psubstp{Q}{P} \\
  (x?(y).R) \psubstp{Q}{P}    
  :=    
  (x)\substp{Q}{P} (z)\concat( (R \psubstn{z}{y}) \psubstp{Q}{P} ) \\
  (\lift{x}{R}) \psubstp{Q}{P}  
  :=
  \lift{(x)\substp{Q}{P}}{ R \psubstp{Q}{P} } \\
%   (\dropn{x})  \psubstp{Q}{P}       
%   := 
%   \left\{ 
%     \begin{array}{ccc} 
%       \dropn{\quotep{Q}} & & x \nameeq \quotep{P} \\
%       \dropn{x} & & otherwise \\
%     \end{array}
%   \right. 
  (\dropn{x})  \psubstp{Q}{P}       
  := 
  \left\{ 
    \begin{array}{ccc} 
      Q & & x \nameeq \quotep{P} \\
      \dropn{x} & & otherwise \\
    \end{array}
  \right.
\end{mathpar}
 

where

\begin{eqnarray}
  (x)\id{\{} \lpquote Q \rpquote / \lpquote P \rpquote \id{\}}            = 
  \left\{ 
    \begin{array}{ccc}
      \lpquote Q \rpquote & & x \nameeq \lpquote P \rpquote \\
      x & & otherwise \\
    \end{array}
  \right. \nonumber
\end{eqnarray}

and $z$ is chosen distinct from $\quotep{P}$, $\quotep{Q}$, the free
names in $Q$, and all the names in $R$. Our $\alpha$-equivalence will
be built in the standard way from this substitution.

\begin{remark}\label{rem:no_self_referential_names}
  One consequence of these definitions is that $\forall P. \quotep{P}
  \not\in \freenames{P}$.
\end{remark}

\subsection{ Dynamic quote: an example }

Anticipating something of what's to come, consider applying the
substitution, $\widehat{\id{\{}u / z \id{\}}}$, to the following pair
of processes, $\lift{w}{y!(z)}$ and $w[ \lpquote y!(z) \rpquote ]$.

\begin{eqnarray}
	\lift{w}{y!(z)}\widehat{\id{\{}u / z \id{\}}}
		& = &
		\lift{w}{y!(u)} \nonumber\\
	w[ \lpquote y!(z) \rpquote ] \widehat{ \id{\{}u / z \id{\}} }
		& = &
		w[ \lpquote y!(z) \rpquote ] \nonumber
\end{eqnarray}

Because the body of the process between quotes is impervious to
substitution, we get radically different answers. In fact, by
examining the first process in an input context,
e.g. $x?(z).\lift{w}{y!(z)}$, we see that the process under the lift
operator may be shaped by prefixed inputs binding a name inside it. In
this sense, the lift operator will be seen as a way to dynamically
construct processes before reifying them as names.

Finally equipped with these standard features we can present the
dynamics of the calculus.

\subsubsection{Operational semantics} 

Finally, we introduce the computational dynamics. What marks these
algebras as distinct from other more traditionally studied algebraic
structures, e.g. vector spaces or polynomial rings, is the manner in
which dynamics is captured. In traditional structures, dynamics is typically
expressed through morphisms between such structures, as in linear maps
between vector spaces or morphisms between rings. In algebras
associated with the semantics of computation, the dynamics is
expressed as part of the algebraic structure itself, through a
reduction reduction relation typically denoted by $\red$. Below, we
give a recursive presentation of this relation for the calculus used
in the encoding.

$\red \subseteq \pi \times \pi$
$\red : \pi \to \mathcal{P}(\pi)$

\begin{mathpar}
  \inferrule* [lab=Comm] { \textsf{match}( x_{src}, x_{trgt} ) } { x_{trgt}?(y)P \; | \; x_{src}!\langle {Q} \rangle \red P\{\quotep{Q}/y}\} }
  \and \\
  \inferrule* [lab=Par] {{P} \red {P}'} {{{P} | {Q}} \red {{P}' | {Q}}}
  \and
  \inferrule* [lab=Equiv]{{{P} \scong {P}'} \andalso {{P}' \red {Q}'} \andalso {{Q}' \scong {Q}}}{{P} \red {Q}}
\end{mathpar}

\begin{eqnarray*}
  match_{\equiv} (\quotep{P},\quotep{Q}) & := & P \equiv Q \\
  match_{\dagger}(\quotep{P},\quotep{Q}) & := & \forall R. P|Q \red^{*} R => R \red^{*} 0 \\
  match_{K}(\quotep{P},\quotep{Q}) & := & K \mbox{ for some context } K
\end{eqnarray*}

$u?(x)P | u!\langle Q \rangle \red P\{\quotep{Q}/x\}$

%We write $\wred$ for $\red^*$, and $P\red$ if $\exists Q $ such that $ P \red Q$.
We write $P\red$ if $\exists Q $ such that $ P \red Q$ and $P\not\red$, otherwise.

\section{Replication}

As mentioned before, it is known that replication (and hence
recursion) can be implemented in a higher-order process algebra
\cite{SangiorgiWalker}. As our first example of calculation with the
machinery thus far presented we give the construction explicitly in
the {\rhoc}.

\begin{eqnarray}
	D_{x} & := & \prefix{x}{y}{(\binpar{\outputp{x}{y}}{@{y}})} \nonumber\\
	\bangp_{x}{P} & := & \binpar{{x}!\langle{\binpar{D_{x}}{P}}\rangle}{D_{x}} \nonumber
\end{eqnarray}

\begin{eqnarray}
	\bangp_{x}{P} & & \nonumber\\
	=
	& {x}!\langle{(\prefix{x}{y}{(\outputp{x}{y} | @{y})) | P}}\rangle 
	      | \prefix{x}{y}{(\outputp{x}{y} | @{y})} & \nonumber\\
	\red
	& (\outputp{x}{y} | @{y})\substn{\quotep{(\prefix{x}{y}{(@{y} | \outputp{x}{y})) | P}}}{y} & \nonumber\\
	=
	& \outputp{x}{\quotep{(\prefix{x}{y}{(\outputp{x}{y} | @{y})) | P}}}
	  | {(\prefix{x}{y}{(\outputp{x}{y} | @{y})) | P}} & \nonumber\\
	\red
	& \ldots & \nonumber\\
	\red^*
	& P | P | \ldots & \nonumber
\end{eqnarray}

Of course, this encoding, as an implementation, runs away, unfolding
$\bangp{P}$ eagerly. A lazier and more implementable replication
operator, restricted to input-guarded processes, may be obtained as follows.

\begin{eqnarray}
\bangp{\prefix{u}{v}{P}} 
	:= 
	\binpar{\lift{x}{\prefix{u}{v}{(\binpar{D(x)}{P})}}}{D(x)} \nonumber
\end{eqnarray}

\begin{remark}
  Note that the lazier definition still does not deal with summation
  or mixed summation (i.e. sums over input and output). The reader is
  invited to construct definitions of replication that deal with these
  features. 

  Further, the definitions are parameterized in a name, $x$. Can you,
  gentle reader, make a definition that eliminates this parameter and
  guarantees no accidental interaction between the replication
  machinery and the process being replicated -- i.e. no accidental
  sharing of names used by the process to get its work done and the
  name(s) used by the replication to effect copying. This latter
  revision of the definition of replication is crucial to obtaining
  the expected identity $!!P \sim !P$.
\end{remark}

\begin{remark}\label{rem:paradoxical_combinator}
  The reader familiar with the lambda calculus will have noticed the
  similarity between $D$ and the paradoxical combinator.

  [Ed. note: the existence of this seems to suggest we have to be more
  restrictive on the set of processes and names we admit if we are to
  support no-cloning.]
\end{remark}

\subsubsection{Bisimulation}

The computational dynamics gives rise to another kind of equivalence,
the equivalence of computational behavior. As previously mentioned
this is typically captured \emph{via} some form of bisimulation.

% The notion we use in this paper is weak barbed bisimulation
% \cite{milner91polyadicpi}.

The notion we use in this paper is derived from weak barbed
bisimulation \cite{milner91polyadicpi}. 

\begin{definition}
An \emph{observation relation}, $\downarrow_{\mathcal N}$, over a set
of names, $\mathcal N$, is the smallest relation satisfying the rules
below.

\infrule[Out-barb]{y \in {\mathcal N}, \; x \nameeq y}
		  {\outputp{x}{v} \downarrow_{\mathcal N} x}
\infrule[Par-barb]{\mbox{$P\downarrow_{\mathcal N} x$ or $Q\downarrow_{\mathcal N} x$}}
		  {\binpar{P}{Q} \downarrow_{\mathcal N} x}

We write $P \Downarrow_{\mathcal N} x$ if there is $Q$ such that 
$P \wred Q$ and $Q \downarrow_{\mathcal N} x$.
\end{definition}

\begin{definition}
%\label{def.bbisim}
An  ${\mathcal N}$-\emph{barbed bisimulation} over a set of names, ${\mathcal N}$, is a symmetric binary relation 
${\mathcal S}_{\mathcal N}$ between agents such that $P\rel{S}_{\mathcal N}Q$ implies:
\begin{enumerate}
\item If $P \red P'$ then $Q \wred Q'$ and $P'\rel{S}_{\mathcal N} Q'$.
\item If $P\downarrow_{\mathcal N} x$, then $Q\Downarrow_{\mathcal N} x$.
\end{enumerate}
$P$ is ${\mathcal N}$-barbed bisimilar to $Q$, written
$P \wbbisim_{\mathcal N} Q$, if $P \rel{S}_{\mathcal N} Q$ for some ${\mathcal N}$-barbed bisimulation ${\mathcal S}_{\mathcal N}$.
\end{definition}

$\mathcal{R} \subseteq \pi \times \pi$

$P \mathcal{R} Q => \forall P'. P \red P' \Rightarrow \exists Q'. Q \red Q', P' \mathcal{R} Q'$

$P \vdash x \Rightarrow Q \vdash x$

\begin{mathpar}
  \inferrule*[lab=Out-barb]{x \nameeq y}{{y}!\langle{Q}\rangle \vdash x}
  \and
  \inferrule*[lab=Par-barb]{\mbox{$P\vdash x$ or $Q\vdash x$}}{\binpar{P}{Q} \vdash x}
\end{mathpar}

\subsubsection{Contexts}

One of the principle advantages of computational calculi like the
$\pi$-calculus is a well-defined notion of context,
contextual-equivalence and a correlation between
contextual-equivalence and notions of bisimulation. The notion of
context allows the decomposition of a process into (sub-)process and
its syntactic environment, its context. Thus, a context may be
thought of as a process with a ``hole'' (written $\Box$) in it. The
application of a context $M$ to a process $P$, written $M[P]$, is
tantamount to filling the hole in $M$ with $P$. In this paper we do
not need the full weight of this theory, but do make use of the notion
of context in the proof the main theorem. 

\begin{mathpar}
  \inferrule* [lab=summation] {} {{M_{M},M_{N}} \bc \Box \;|\; x.M_{A} \;|\; M_{M}+M_{N}}
  \and
  \inferrule* [lab=agent] {} {{M_{A}} \bc (\vec{x})M_{P} \;| \; \clift{P_0,\ldots,M_{P},\ldots,P_N}}
  \and \\
  \inferrule* [lab=process] {} {{M_{P}} \bc M_{N} \;| \;P|M_{P} }
\end{mathpar} 

\begin{mathpar}
  \inferrule* [lab=sychronization] {} {M_{N} \bc \Box \;|\; x?M_{F} \;|\; x!M_{C}}
  \and
  \inferrule* [lab=abstraction] {} {{M_{F}} \bc (x)M_{P} }
  \and
  \inferrule* [lab=concretion] {} {{M_{C}} \bc \langle M_{P} \rangle }
  \and \\
  \inferrule* [lab=process] {} {{M_{P}} \bc M_{N} \;| \;P|M_{P} }
\end{mathpar}

\begin{definition}[contextual application] Given a context $M$, and
  process $P$, we define the \emph{contextual application}, $M[P] :=
  M\{P/\Box\}$. That is, the contextual application of M to P is the
  substitution of $P$ for $\Box$ in $M$.
\end{definition}

$\meaningof{-} : L \to \mathcal{P}(\pi)$

\begin{mathpar}
  \inferrule* [lab=collection] {} {\meaningof{true} = \pi, \and \meaningof{~E} = \pi \setminus \meaningof{E}, \and \meaningof{E_{1} \& E_{2}} = \meaningof{E_{1}} \cap \meaningof{E_{2}}}
\end{mathpar}

\begin{mathpar}
  \inferrule* [lab=structure] {} {\meaningof{0} = \{ P \in \pi | P \equiv 0 \}, \and \\ \meaningof{E_1 | E_2} = \{ P \in \pi | P \equiv P_{1} | P_{2}, P_{1} \in \meaningof{E_{1}}, P_{2} \in \meaningof{E_2}\} }
\end{mathpar}

\begin{mathpar}
 \inferrule* [lab=behavior] {} {\meaningof{\langle a?b \rangle E} = \{ P \in \pi | P \equiv Q | u?(y)P', \\ \and \\\\ \and \\ \;\;\; u \in \meaningof{a}, \forall z.P'\{z/y\} \in \meaningof{E\{z/b\}}\}, \and \\ \meaningof{a!E} = \{ P \in \pi | P \equiv Q | x!\langle P' \rangle, x \in \meaningof{a} P' \in \meaningof{E}\} }
\end{mathpar}

\begin{mathpar}
 \inferrule* [lab=nominal] {} {\meaningof{\quotep{E}} = \{ \quotep{P} \in \quotep{\pi} | P \in \meaningof{E} \}, \and \meaningof{\quotep{P}} = \{ \quotep{Q} \in \quotep{\pi} | P \equiv Q \} \and \\ \meaningof{@\quotep{E}} = \{ P \in \pi | P \equiv @x, x \in \meaningof{E} \}}
\end{mathpar}

\begin{eqnarray*}
  \\
  \meaningof{-} : TS \to ST
\end{eqnarray*}

\begin{eqnarray*}
  \\
  L : TS \to ST
\end{eqnarray*}

\begin{eqnarray*}
  \\
  P \models E \iff P \in \meaningof{E}
\end{eqnarray*}

\begin{eqnarray*}
  P \approx_{L} Q \iff \forall E \in L. P \models E \iff Q \models E
\end{eqnarray*}

\begin{eqnarray*}
  P \approx_{K} Q
\end{eqnarray*}

\begin{eqnarray*}
  P \approx Q
\end{eqnarray*}

$\approx_{K} = \approx = \approx_{L}$

\subsubsection{Contextual duality}

Note that contexts extend the quotation operation to a family of
operations from processes to names. Given a context, $M$, we can
define a \emph{nominal context}, $\quotep{M}$ by $\quotep{M}[P] :=
\quotep{M[P]}$. To foreshadow what is to come we observe that these
operations enjoy a duality with processes very much like the duality
between vectors and maps from vectors to scalars.

Further, because the calculus is essentially higher-order, we have a
correspondence between contexts and processes. More specifically,
given a name $x$ and a context $M$ we can construct $M^{*}_{x}$ such
that 

\begin{mathpar}
  M^{*}_{x} | \lift{x}{P} \red M[P]
\end{mathpar}

namely,

\begin{mathpar}
  M^{*}_{x} := x?(u).M[\dropn{u}]
\end{mathpar}

The dependence of $M^{*}_{x}$ on a name makes it an abstraction, 

\begin{mathpar}
  M^{*} := (x)x?(u).M[\dropn{u}]
\end{mathpar}

\subsection{Additional notation}

It will sometimes be convenient to denote the process a name
quotes. We already have the notation $x = \quotep{P}$, but it will be
convenient to introduce an alternate notation, $\procn{x}$, when we
want to emphasize the connection to the use of the name. Note that, by
virtue of name equivalence, $\quotep{\procn{x}} \nameeq x$; so, the
notation is consistent with previous definitions.

Further, because names have structure it is possible to effect
substitutions on the basis of that structure. This means we need to
upgrade our notation for substitutions, which we accomplish by
adapting comprehension notation. Thus,

\begin{mathpar}
  P\{ y / x : x \in S \}
\end{mathpar}

is interpreted to mean the process derived from P by replacing (in a
capture-avoiding manner) each occurrence of $x$ in $S$ by $y$. For example,

\begin{mathpar}
  P\{ \quotep{\procn{x}|\procn{x}} / x : x \in \freenames{P} \}
\end{mathpar}

will replace each (occurrence) of a free name $x$ in $P$ by
$\quotep{\procn{x}|\procn{x}}$.

Also, we will avail ourselves of the notation $x^{L}$ and $x^{R}$ to
denote injections of a name into disjoint copies of the name
space. There are numerous ways to accomplish this. One example can be
found in \cite{MeredithR05}. This notation overloads to vectors of
names: $\vec{x}^{\pi} := (x_{i}^{\pi} \; : \; 0 \leq i < |\vec{x}| )$ where $\pi \in \{L,R\}$.

We also use $P^{\Box} := P|\Box$.

In \cite{MeredithR05} an interpretation of the new operator is
given. It turns out that there are several possible interpretations
all enjoying the requisite algebraic properties of the operator (see
\cite{milner91polyadicpi}). We will therefore make liberal use of
$(\nu\; \vec{x})P$.

% subsection the_syntax_and_semantics_of_the_notation_system (end)   

\input{qm2pi.qmops} 

\input{qm2pi.sterngerlach} 

\input{qm2pi.metric} 

% section concurrent_process_calculi (end)

%\input{qm2pi.proofsketch}

% section proof sketch (end)

%\input{qm2pi.slviaknots} 

% section spatial logic via knots (end)

\input{qm2pi.conclusion}

% section conclusion (end)

%\input{qm2pi.dtcodes} 

% section wiring algorithm (end)

\input{qm2pi.ack} 

% section acknowledgments (end)

\newpage


\bibliographystyle{plain}   
\bibliography{../../biblios/main.bib}

\input{qm2pi.rhodetails}

\end{document}

 

%\documentclass[12pt]{llncs}
%\documentclass{jktr}

\usepackage[pdftex]{hyperref}                   
\usepackage {listings}
\usepackage {mathpartir}
\usepackage{bcprules}
%\usepackage{listings}
                       
\usepackage{graphicx} 
%\usepackage[margins=2.5cm,nohead,nofoot]{geometry}
%\usepackage{geometry}
\usepackage{amsfonts}
\usepackage{amstext}
\usepackage{latexsym}
\usepackage{amssymb}
\usepackage{color}


%\include{myPreamble}
\include{qm2pi.local} 

%\ifpdf
%\usepackage[pdftex]{graphicx}
%\else
%\usepackage{graphicx}
%\fi

 % \ifpdf
%  \usepackage{pdfsync}
%  \if


%\title{Brief Article}
%\author{David F. Snyder}
%\author{L.G. Meredith}

%\address{Dept. of Math., Texas State University--San Marcos, San Marcos, TX 78666}
       
\pagestyle{empty}


\begin{document}

\lstset{language=[Objective]Caml,frame=shadowbox}

\input{qm2pi.front}

% section front matter (end)

\input{qm2pi.intro} 
 
% section introduction (end)

% \input{qm2pi.knotations} 

% section notation (end)

\input{qm2pi.process.calculi} 

% section concurrent_process_calculi_and_spatial_logics_ (end)
    
%\input{qm2pi.knots2pi} 

%\input{qm2pi.trefoil} 

%\input{qm2pi.mainthm} 

% subsection basic_interpretation (end)

%\input{qm2pi.rho.presentation} 
\subsection{The syntax and semantics of the notation system}\label{sub:the_syntax_and_semantics_of_the_notation_system} % (fold)

We now summarize a technical presentation of the calculus that
embodies our theory of dynamics. The typical presentation of such a
calculus follows the style of giving generators and relations on
them. The grammar, below, describing term constructors, freely
generates the set of processes, $\Proc$. This set is then quotiented
by a relation known as structural congruence and it is over this set
that the notion of dynamics is expressed. This presentation is
essentially that of \cite{MeredithR05} with the addition of
polyadicity and summation. For readability we have relegated some of
the technical subtleties to an appendix.

\subsubsection{Process grammar}\label{subsub:process_grammar}

\begin{mathpar}
  \inferrule* [lab=synchronization] {} {{M} \bc \pzero \;|\; x?F \;|\; x!C }
  \and
  \inferrule* [lab=abstraction] {} {{F} \bc (x)P}
  \and
  \inferrule* [lab=concretion] {} {{C} \bc \langle Q \rangle}
  \and
  \inferrule* [lab=process] {} {{P,Q} \bc M \;| \;P|Q \;|\; @{x}}
  \and
  \inferrule* [lab=name] {} {{x} \bc \quotep{P}}
\end{mathpar} 

Note that $\vec{x}$ (resp. $\vec{P}$) denotes a vector of names
(resp. processes) of length $|\vec{x}|$ (resp. $|\vec{P}|$). We adopt
the following useful abbreviations.

\begin{mathpar}
   x?(\vec{y}).P := x.(\vec{y})P \and  x\clift{\vec{P}} := x.\clift{\vec{P}}
   \and x!(y) := \lift{x}{\dropn{y}}
   \and \Pi_{i=0}^{n-1}P_i := P_0 | \ldots | P_{n-1}
\end{mathpar}

\subsubsection{Structural congruence}

\paragraph{Free and bound names and alpha-equivalence.} At the
core of structural equivalence is alpha-equivalence which identifies
process that are the same up to a change of variable. Formally, we
recognize the distinction between free and bound names. The free names
of a process, $\freenames{P}$, may be calculated recursively as
follows:

\begin{mathpar}
\freenames{\pzero} := \emptyset
  \and \\
  \freenames{x?(y).P} := \{ x \} \cup (\freenames{P} \setminus \{ y \})
  \and 
  \freenames{x!\langle P \rangle} := \{ x \} \cup \{ P \} 
  \and \\
  \freenames{P|Q} := \freenames{P} \cup \freenames{Q}
  \and \\
  \freenames{@{x}} := \{ x \}
\end{mathpar}

$\pi$
$\quotep{\pi}$

$\freenames{-} : \pi \to \mathcal{P}(\quotep{\pi})$

\begin{eqnarray*}
  \freenames{\pzero} & := & \emptyset \\
  \freenames{x?(y).P} & := & \{ x \} \cup (\freenames{P} \setminus \{ y \}) \\
  \freenames{x!\langle P \rangle} & := & \{ x \} \cup \{ P \} \\
  \freenames{P|Q} & := & \freenames{P} \cup \freenames{Q} \\
  \freenames{\dropn{x}} & := & \{ x \}
\end{eqnarray*}

The bound names of a process, $\boundnames{P}$, are those names occurring in $P$
that are not free. For example, in $x?(y).0$, the name $x$ is free, while $y$ is bound.

\begin{mathpar}
  \inferrule* [lab=monoidal-laws] {} { P|Q \equiv Q|P \and P|0 \equiv P \and P|(Q|R) \equiv (P|Q)|R }
\end{mathpar}

\begin{mathpar}
  \inferrule* [lab=alpha-equivalence] {} { (x)P \equiv (y)P\{y/x\} \and y \not\in \freenames{P} }
\end{mathpar}

\begin{definition}
Then two processes, $P,Q$, are alpha-equivalent if $P = Q\{\vec{y}/\vec{x}\}$ for
some $\vec{x} \in \boundnames{Q},\vec{y} \in \boundnames{P}$, where $Q\{\vec{y}/\vec{x}\}$
denotes the capture-avoiding substitution of $\vec{y}$ for $\vec{x}$ in $Q$.
\end{definition}

\begin{definition}
  The {\em structural congruence} \cite{SangiorgiWalker} , $\equiv$,
  between processes is the least congruence containing
  alpha-equivalence, satisfying the abelian monoid laws
  (associativity, commutativity and $\pzero$ as identity) for parallel
  composition $|$ and for summation $+$.
\end{definition}

\subsection{Name equivalence}

We take name equivalence, written $\nameeq$, to be the smallest
equivalence relation generated by the following rules.

\begin{mathpar}
\inferrule*[lab=Quote-drop]
{ }
{ \quotep{@{x}} \nameeq x }

\inferrule*[lab=Struct-equiv]
{ P \scong Q }
{ \quotep{P} \nameeq \quotep{Q} }
\end{mathpar}

The astute reader will have noticed that the mutual recursion of names
and processes imposes a mutual recursion on alpha-equivalence and
structural equivalence via name-equivalence. Fortunately, all of this
works out pleasantly and we may calculate in the natural way, free of
concern. The reader interested in the details is referred to the
appendix \ref{appendix:rho_details}.

\subsection{Substitution}

We use $\Proc$ for the set of processes, $\QProc$ for the set of
names, and $\id{\{}\vec{y} / \vec{x} \id{\}}$ to denote partial maps,
$s : \QProc \rightarrow \QProc$. A map, $s$ lifts, uniquely, to a map
on process terms, $\widehat{s} : \Proc \rightarrow \Proc$ by the
following equations.

\begin{mathpar}
  (0) \psubstp{Q}{P} := 0 \\
  (R \juxtap S) \psubstp{Q}{P}
  :=    
  (R)\psubstp{Q}{P} \juxtap (S) \psubstp{Q}{P} \\
  (x?(y).R) \psubstp{Q}{P}    
  :=    
  (x)\substp{Q}{P} (z)\concat( (R \psubstn{z}{y}) \psubstp{Q}{P} ) \\
  (\lift{x}{R}) \psubstp{Q}{P}  
  :=
  \lift{(x)\substp{Q}{P}}{ R \psubstp{Q}{P} } \\
%   (\dropn{x})  \psubstp{Q}{P}       
%   := 
%   \left\{ 
%     \begin{array}{ccc} 
%       \dropn{\quotep{Q}} & & x \nameeq \quotep{P} \\
%       \dropn{x} & & otherwise \\
%     \end{array}
%   \right. 
  (\dropn{x})  \psubstp{Q}{P}       
  := 
  \left\{ 
    \begin{array}{ccc} 
      Q & & x \nameeq \quotep{P} \\
      \dropn{x} & & otherwise \\
    \end{array}
  \right.
\end{mathpar}
 

where

\begin{eqnarray}
  (x)\id{\{} \lpquote Q \rpquote / \lpquote P \rpquote \id{\}}            = 
  \left\{ 
    \begin{array}{ccc}
      \lpquote Q \rpquote & & x \nameeq \lpquote P \rpquote \\
      x & & otherwise \\
    \end{array}
  \right. \nonumber
\end{eqnarray}

and $z$ is chosen distinct from $\quotep{P}$, $\quotep{Q}$, the free
names in $Q$, and all the names in $R$. Our $\alpha$-equivalence will
be built in the standard way from this substitution.

\begin{remark}\label{rem:no_self_referential_names}
  One consequence of these definitions is that $\forall P. \quotep{P}
  \not\in \freenames{P}$.
\end{remark}

\subsection{ Dynamic quote: an example }

Anticipating something of what's to come, consider applying the
substitution, $\widehat{\id{\{}u / z \id{\}}}$, to the following pair
of processes, $\lift{w}{y!(z)}$ and $w[ \lpquote y!(z) \rpquote ]$.

\begin{eqnarray}
	\lift{w}{y!(z)}\widehat{\id{\{}u / z \id{\}}}
		& = &
		\lift{w}{y!(u)} \nonumber\\
	w[ \lpquote y!(z) \rpquote ] \widehat{ \id{\{}u / z \id{\}} }
		& = &
		w[ \lpquote y!(z) \rpquote ] \nonumber
\end{eqnarray}

Because the body of the process between quotes is impervious to
substitution, we get radically different answers. In fact, by
examining the first process in an input context,
e.g. $x?(z).\lift{w}{y!(z)}$, we see that the process under the lift
operator may be shaped by prefixed inputs binding a name inside it. In
this sense, the lift operator will be seen as a way to dynamically
construct processes before reifying them as names.

Finally equipped with these standard features we can present the
dynamics of the calculus.

\subsubsection{Operational semantics} 

Finally, we introduce the computational dynamics. What marks these
algebras as distinct from other more traditionally studied algebraic
structures, e.g. vector spaces or polynomial rings, is the manner in
which dynamics is captured. In traditional structures, dynamics is typically
expressed through morphisms between such structures, as in linear maps
between vector spaces or morphisms between rings. In algebras
associated with the semantics of computation, the dynamics is
expressed as part of the algebraic structure itself, through a
reduction reduction relation typically denoted by $\red$. Below, we
give a recursive presentation of this relation for the calculus used
in the encoding.

$\red \subseteq \pi \times \pi$
$\red : \pi \to \mathcal{P}(\pi)$

\begin{mathpar}
  \inferrule* [lab=Comm] { \textsf{match}( x_{src}, x_{trgt} ) } { x_{trgt}?(y)P \; | \; x_{src}!\langle {Q} \rangle \red P\{\quotep{Q}/y}\} }
  \and \\
  \inferrule* [lab=Par] {{P} \red {P}'} {{{P} | {Q}} \red {{P}' | {Q}}}
  \and
  \inferrule* [lab=Equiv]{{{P} \scong {P}'} \andalso {{P}' \red {Q}'} \andalso {{Q}' \scong {Q}}}{{P} \red {Q}}
\end{mathpar}

\begin{eqnarray*}
  match_{\equiv} (\quotep{P},\quotep{Q}) & := & P \equiv Q \\
  match_{\dagger}(\quotep{P},\quotep{Q}) & := & \forall R. P|Q \red^{*} R => R \red^{*} 0 \\
  match_{K}(\quotep{P},\quotep{Q}) & := & K \mbox{ for some context } K
\end{eqnarray*}

$u?(x)P | u!\langle Q \rangle \red P\{\quotep{Q}/x\}$

%We write $\wred$ for $\red^*$, and $P\red$ if $\exists Q $ such that $ P \red Q$.
We write $P\red$ if $\exists Q $ such that $ P \red Q$ and $P\not\red$, otherwise.

\section{Replication}

As mentioned before, it is known that replication (and hence
recursion) can be implemented in a higher-order process algebra
\cite{SangiorgiWalker}. As our first example of calculation with the
machinery thus far presented we give the construction explicitly in
the {\rhoc}.

\begin{eqnarray}
	D_{x} & := & \prefix{x}{y}{(\binpar{\outputp{x}{y}}{@{y}})} \nonumber\\
	\bangp_{x}{P} & := & \binpar{{x}!\langle{\binpar{D_{x}}{P}}\rangle}{D_{x}} \nonumber
\end{eqnarray}

\begin{eqnarray}
	\bangp_{x}{P} & & \nonumber\\
	=
	& {x}!\langle{(\prefix{x}{y}{(\outputp{x}{y} | @{y})) | P}}\rangle 
	      | \prefix{x}{y}{(\outputp{x}{y} | @{y})} & \nonumber\\
	\red
	& (\outputp{x}{y} | @{y})\substn{\quotep{(\prefix{x}{y}{(@{y} | \outputp{x}{y})) | P}}}{y} & \nonumber\\
	=
	& \outputp{x}{\quotep{(\prefix{x}{y}{(\outputp{x}{y} | @{y})) | P}}}
	  | {(\prefix{x}{y}{(\outputp{x}{y} | @{y})) | P}} & \nonumber\\
	\red
	& \ldots & \nonumber\\
	\red^*
	& P | P | \ldots & \nonumber
\end{eqnarray}

Of course, this encoding, as an implementation, runs away, unfolding
$\bangp{P}$ eagerly. A lazier and more implementable replication
operator, restricted to input-guarded processes, may be obtained as follows.

\begin{eqnarray}
\bangp{\prefix{u}{v}{P}} 
	:= 
	\binpar{\lift{x}{\prefix{u}{v}{(\binpar{D(x)}{P})}}}{D(x)} \nonumber
\end{eqnarray}

\begin{remark}
  Note that the lazier definition still does not deal with summation
  or mixed summation (i.e. sums over input and output). The reader is
  invited to construct definitions of replication that deal with these
  features. 

  Further, the definitions are parameterized in a name, $x$. Can you,
  gentle reader, make a definition that eliminates this parameter and
  guarantees no accidental interaction between the replication
  machinery and the process being replicated -- i.e. no accidental
  sharing of names used by the process to get its work done and the
  name(s) used by the replication to effect copying. This latter
  revision of the definition of replication is crucial to obtaining
  the expected identity $!!P \sim !P$.
\end{remark}

\begin{remark}\label{rem:paradoxical_combinator}
  The reader familiar with the lambda calculus will have noticed the
  similarity between $D$ and the paradoxical combinator.

  [Ed. note: the existence of this seems to suggest we have to be more
  restrictive on the set of processes and names we admit if we are to
  support no-cloning.]
\end{remark}

\subsubsection{Bisimulation}

The computational dynamics gives rise to another kind of equivalence,
the equivalence of computational behavior. As previously mentioned
this is typically captured \emph{via} some form of bisimulation.

% The notion we use in this paper is weak barbed bisimulation
% \cite{milner91polyadicpi}.

The notion we use in this paper is derived from weak barbed
bisimulation \cite{milner91polyadicpi}. 

\begin{definition}
An \emph{observation relation}, $\downarrow_{\mathcal N}$, over a set
of names, $\mathcal N$, is the smallest relation satisfying the rules
below.

\infrule[Out-barb]{y \in {\mathcal N}, \; x \nameeq y}
		  {\outputp{x}{v} \downarrow_{\mathcal N} x}
\infrule[Par-barb]{\mbox{$P\downarrow_{\mathcal N} x$ or $Q\downarrow_{\mathcal N} x$}}
		  {\binpar{P}{Q} \downarrow_{\mathcal N} x}

We write $P \Downarrow_{\mathcal N} x$ if there is $Q$ such that 
$P \wred Q$ and $Q \downarrow_{\mathcal N} x$.
\end{definition}

\begin{definition}
%\label{def.bbisim}
An  ${\mathcal N}$-\emph{barbed bisimulation} over a set of names, ${\mathcal N}$, is a symmetric binary relation 
${\mathcal S}_{\mathcal N}$ between agents such that $P\rel{S}_{\mathcal N}Q$ implies:
\begin{enumerate}
\item If $P \red P'$ then $Q \wred Q'$ and $P'\rel{S}_{\mathcal N} Q'$.
\item If $P\downarrow_{\mathcal N} x$, then $Q\Downarrow_{\mathcal N} x$.
\end{enumerate}
$P$ is ${\mathcal N}$-barbed bisimilar to $Q$, written
$P \wbbisim_{\mathcal N} Q$, if $P \rel{S}_{\mathcal N} Q$ for some ${\mathcal N}$-barbed bisimulation ${\mathcal S}_{\mathcal N}$.
\end{definition}

$\mathcal{R} \subseteq \pi \times \pi$

$P \mathcal{R} Q => \forall P'. P \red P' \Rightarrow \exists Q'. Q \red Q', P' \mathcal{R} Q'$

$P \vdash x \Rightarrow Q \vdash x$

\begin{mathpar}
  \inferrule*[lab=Out-barb]{x \nameeq y}{{y}!\langle{Q}\rangle \vdash x}
  \and
  \inferrule*[lab=Par-barb]{\mbox{$P\vdash x$ or $Q\vdash x$}}{\binpar{P}{Q} \vdash x}
\end{mathpar}

\subsubsection{Contexts}

One of the principle advantages of computational calculi like the
$\pi$-calculus is a well-defined notion of context,
contextual-equivalence and a correlation between
contextual-equivalence and notions of bisimulation. The notion of
context allows the decomposition of a process into (sub-)process and
its syntactic environment, its context. Thus, a context may be
thought of as a process with a ``hole'' (written $\Box$) in it. The
application of a context $M$ to a process $P$, written $M[P]$, is
tantamount to filling the hole in $M$ with $P$. In this paper we do
not need the full weight of this theory, but do make use of the notion
of context in the proof the main theorem. 

\begin{mathpar}
  \inferrule* [lab=summation] {} {{M_{M},M_{N}} \bc \Box \;|\; x.M_{A} \;|\; M_{M}+M_{N}}
  \and
  \inferrule* [lab=agent] {} {{M_{A}} \bc (\vec{x})M_{P} \;| \; \clift{P_0,\ldots,M_{P},\ldots,P_N}}
  \and \\
  \inferrule* [lab=process] {} {{M_{P}} \bc M_{N} \;| \;P|M_{P} }
\end{mathpar} 

\begin{mathpar}
  \inferrule* [lab=sychronization] {} {M_{N} \bc \Box \;|\; x?M_{F} \;|\; x!M_{C}}
  \and
  \inferrule* [lab=abstraction] {} {{M_{F}} \bc (x)M_{P} }
  \and
  \inferrule* [lab=concretion] {} {{M_{C}} \bc \langle M_{P} \rangle }
  \and \\
  \inferrule* [lab=process] {} {{M_{P}} \bc M_{N} \;| \;P|M_{P} }
\end{mathpar}

\begin{definition}[contextual application] Given a context $M$, and
  process $P$, we define the \emph{contextual application}, $M[P] :=
  M\{P/\Box\}$. That is, the contextual application of M to P is the
  substitution of $P$ for $\Box$ in $M$.
\end{definition}

$\meaningof{-} : L \to \mathcal{P}(\pi)$

\begin{mathpar}
  \inferrule* [lab=collection] {} {\meaningof{true} = \pi, \and \meaningof{~E} = \pi \setminus \meaningof{E}, \and \meaningof{E_{1} \& E_{2}} = \meaningof{E_{1}} \cap \meaningof{E_{2}}}
\end{mathpar}

\begin{mathpar}
  \inferrule* [lab=structure] {} {\meaningof{0} = \{ P \in \pi | P \equiv 0 \}, \and \\ \meaningof{E_1 | E_2} = \{ P \in \pi | P \equiv P_{1} | P_{2}, P_{1} \in \meaningof{E_{1}}, P_{2} \in \meaningof{E_2}\} }
\end{mathpar}

\begin{mathpar}
 \inferrule* [lab=behavior] {} {\meaningof{\langle a?b \rangle E} = \{ P \in \pi | P \equiv Q | u?(y)P', \\ \and \\\\ \and \\ \;\;\; u \in \meaningof{a}, \forall z.P'\{z/y\} \in \meaningof{E\{z/b\}}\}, \and \\ \meaningof{a!E} = \{ P \in \pi | P \equiv Q | x!\langle P' \rangle, x \in \meaningof{a} P' \in \meaningof{E}\} }
\end{mathpar}

\begin{mathpar}
 \inferrule* [lab=nominal] {} {\meaningof{\quotep{E}} = \{ \quotep{P} \in \quotep{\pi} | P \in \meaningof{E} \}, \and \meaningof{\quotep{P}} = \{ \quotep{Q} \in \quotep{\pi} | P \equiv Q \} \and \\ \meaningof{@\quotep{E}} = \{ P \in \pi | P \equiv @x, x \in \meaningof{E} \}}
\end{mathpar}

\begin{eqnarray*}
  \\
  \meaningof{-} : TS \to ST
\end{eqnarray*}

\begin{eqnarray*}
  \\
  L : TS \to ST
\end{eqnarray*}

\begin{eqnarray*}
  \\
  P \models E \iff P \in \meaningof{E}
\end{eqnarray*}

\begin{eqnarray*}
  P \approx_{L} Q \iff \forall E \in L. P \models E \iff Q \models E
\end{eqnarray*}

\begin{eqnarray*}
  P \approx_{K} Q
\end{eqnarray*}

\begin{eqnarray*}
  P \approx Q
\end{eqnarray*}

$\approx_{K} = \approx = \approx_{L}$

\subsubsection{Contextual duality}

Note that contexts extend the quotation operation to a family of
operations from processes to names. Given a context, $M$, we can
define a \emph{nominal context}, $\quotep{M}$ by $\quotep{M}[P] :=
\quotep{M[P]}$. To foreshadow what is to come we observe that these
operations enjoy a duality with processes very much like the duality
between vectors and maps from vectors to scalars.

Further, because the calculus is essentially higher-order, we have a
correspondence between contexts and processes. More specifically,
given a name $x$ and a context $M$ we can construct $M^{*}_{x}$ such
that 

\begin{mathpar}
  M^{*}_{x} | \lift{x}{P} \red M[P]
\end{mathpar}

namely,

\begin{mathpar}
  M^{*}_{x} := x?(u).M[\dropn{u}]
\end{mathpar}

The dependence of $M^{*}_{x}$ on a name makes it an abstraction, 

\begin{mathpar}
  M^{*} := (x)x?(u).M[\dropn{u}]
\end{mathpar}

\subsection{Additional notation}

It will sometimes be convenient to denote the process a name
quotes. We already have the notation $x = \quotep{P}$, but it will be
convenient to introduce an alternate notation, $\procn{x}$, when we
want to emphasize the connection to the use of the name. Note that, by
virtue of name equivalence, $\quotep{\procn{x}} \nameeq x$; so, the
notation is consistent with previous definitions.

Further, because names have structure it is possible to effect
substitutions on the basis of that structure. This means we need to
upgrade our notation for substitutions, which we accomplish by
adapting comprehension notation. Thus,

\begin{mathpar}
  P\{ y / x : x \in S \}
\end{mathpar}

is interpreted to mean the process derived from P by replacing (in a
capture-avoiding manner) each occurrence of $x$ in $S$ by $y$. For example,

\begin{mathpar}
  P\{ \quotep{\procn{x}|\procn{x}} / x : x \in \freenames{P} \}
\end{mathpar}

will replace each (occurrence) of a free name $x$ in $P$ by
$\quotep{\procn{x}|\procn{x}}$.

Also, we will avail ourselves of the notation $x^{L}$ and $x^{R}$ to
denote injections of a name into disjoint copies of the name
space. There are numerous ways to accomplish this. One example can be
found in \cite{MeredithR05}. This notation overloads to vectors of
names: $\vec{x}^{\pi} := (x_{i}^{\pi} \; : \; 0 \leq i < |\vec{x}| )$ where $\pi \in \{L,R\}$.

We also use $P^{\Box} := P|\Box$.

In \cite{MeredithR05} an interpretation of the new operator is
given. It turns out that there are several possible interpretations
all enjoying the requisite algebraic properties of the operator (see
\cite{milner91polyadicpi}). We will therefore make liberal use of
$(\nu\; \vec{x})P$.

% subsection the_syntax_and_semantics_of_the_notation_system (end)   

\input{qm2pi.qmops} 

\input{qm2pi.sterngerlach} 

\input{qm2pi.metric} 

% section concurrent_process_calculi (end)

%\input{qm2pi.proofsketch}

% section proof sketch (end)

%\input{qm2pi.slviaknots} 

% section spatial logic via knots (end)

\input{qm2pi.conclusion}

% section conclusion (end)

%\input{qm2pi.dtcodes} 

% section wiring algorithm (end)

\input{qm2pi.ack} 

% section acknowledgments (end)

\newpage


\bibliographystyle{plain}   
\bibliography{../../biblios/main.bib}

\input{qm2pi.rhodetails}

\end{document}

 

% subsection basic_interpretation (end)

%\input{qm2pi.rho.presentation} 
\subsection{The syntax and semantics of the notation system}\label{sub:the_syntax_and_semantics_of_the_notation_system} % (fold)

We now summarize a technical presentation of the calculus that
embodies our theory of dynamics. The typical presentation of such a
calculus follows the style of giving generators and relations on
them. The grammar, below, describing term constructors, freely
generates the set of processes, $\Proc$. This set is then quotiented
by a relation known as structural congruence and it is over this set
that the notion of dynamics is expressed. This presentation is
essentially that of \cite{MeredithR05} with the addition of
polyadicity and summation. For readability we have relegated some of
the technical subtleties to an appendix.

\subsubsection{Process grammar}\label{subsub:process_grammar}

\begin{mathpar}
  \inferrule* [lab=synchronization] {} {{M} \bc \pzero \;|\; x?F \;|\; x!C }
  \and
  \inferrule* [lab=abstraction] {} {{F} \bc (x)P}
  \and
  \inferrule* [lab=concretion] {} {{C} \bc \langle Q \rangle}
  \and
  \inferrule* [lab=process] {} {{P,Q} \bc M \;| \;P|Q \;|\; @{x}}
  \and
  \inferrule* [lab=name] {} {{x} \bc \quotep{P}}
\end{mathpar} 

Note that $\vec{x}$ (resp. $\vec{P}$) denotes a vector of names
(resp. processes) of length $|\vec{x}|$ (resp. $|\vec{P}|$). We adopt
the following useful abbreviations.

\begin{mathpar}
   x?(\vec{y}).P := x.(\vec{y})P \and  x\clift{\vec{P}} := x.\clift{\vec{P}}
   \and x!(y) := \lift{x}{\dropn{y}}
   \and \Pi_{i=0}^{n-1}P_i := P_0 | \ldots | P_{n-1}
\end{mathpar}

\subsubsection{Structural congruence}

\paragraph{Free and bound names and alpha-equivalence.} At the
core of structural equivalence is alpha-equivalence which identifies
process that are the same up to a change of variable. Formally, we
recognize the distinction between free and bound names. The free names
of a process, $\freenames{P}$, may be calculated recursively as
follows:

\begin{mathpar}
\freenames{\pzero} := \emptyset
  \and \\
  \freenames{x?(y).P} := \{ x \} \cup (\freenames{P} \setminus \{ y \})
  \and 
  \freenames{x!\langle P \rangle} := \{ x \} \cup \{ P \} 
  \and \\
  \freenames{P|Q} := \freenames{P} \cup \freenames{Q}
  \and \\
  \freenames{@{x}} := \{ x \}
\end{mathpar}

$\pi$
$\quotep{\pi}$

$\freenames{-} : \pi \to \mathcal{P}(\quotep{\pi})$

\begin{eqnarray*}
  \freenames{\pzero} & := & \emptyset \\
  \freenames{x?(y).P} & := & \{ x \} \cup (\freenames{P} \setminus \{ y \}) \\
  \freenames{x!\langle P \rangle} & := & \{ x \} \cup \{ P \} \\
  \freenames{P|Q} & := & \freenames{P} \cup \freenames{Q} \\
  \freenames{\dropn{x}} & := & \{ x \}
\end{eqnarray*}

The bound names of a process, $\boundnames{P}$, are those names occurring in $P$
that are not free. For example, in $x?(y).0$, the name $x$ is free, while $y$ is bound.

\begin{mathpar}
  \inferrule* [lab=monoidal-laws] {} { P|Q \equiv Q|P \and P|0 \equiv P \and P|(Q|R) \equiv (P|Q)|R }
\end{mathpar}

\begin{mathpar}
  \inferrule* [lab=alpha-equivalence] {} { (x)P \equiv (y)P\{y/x\} \and y \not\in \freenames{P} }
\end{mathpar}

\begin{definition}
Then two processes, $P,Q$, are alpha-equivalent if $P = Q\{\vec{y}/\vec{x}\}$ for
some $\vec{x} \in \boundnames{Q},\vec{y} \in \boundnames{P}$, where $Q\{\vec{y}/\vec{x}\}$
denotes the capture-avoiding substitution of $\vec{y}$ for $\vec{x}$ in $Q$.
\end{definition}

\begin{definition}
  The {\em structural congruence} \cite{SangiorgiWalker} , $\equiv$,
  between processes is the least congruence containing
  alpha-equivalence, satisfying the abelian monoid laws
  (associativity, commutativity and $\pzero$ as identity) for parallel
  composition $|$ and for summation $+$.
\end{definition}

\subsection{Name equivalence}

We take name equivalence, written $\nameeq$, to be the smallest
equivalence relation generated by the following rules.

\begin{mathpar}
\inferrule*[lab=Quote-drop]
{ }
{ \quotep{@{x}} \nameeq x }

\inferrule*[lab=Struct-equiv]
{ P \scong Q }
{ \quotep{P} \nameeq \quotep{Q} }
\end{mathpar}

The astute reader will have noticed that the mutual recursion of names
and processes imposes a mutual recursion on alpha-equivalence and
structural equivalence via name-equivalence. Fortunately, all of this
works out pleasantly and we may calculate in the natural way, free of
concern. The reader interested in the details is referred to the
appendix \ref{appendix:rho_details}.

\subsection{Substitution}

We use $\Proc$ for the set of processes, $\QProc$ for the set of
names, and $\id{\{}\vec{y} / \vec{x} \id{\}}$ to denote partial maps,
$s : \QProc \rightarrow \QProc$. A map, $s$ lifts, uniquely, to a map
on process terms, $\widehat{s} : \Proc \rightarrow \Proc$ by the
following equations.

\begin{mathpar}
  (0) \psubstp{Q}{P} := 0 \\
  (R \juxtap S) \psubstp{Q}{P}
  :=    
  (R)\psubstp{Q}{P} \juxtap (S) \psubstp{Q}{P} \\
  (x?(y).R) \psubstp{Q}{P}    
  :=    
  (x)\substp{Q}{P} (z)\concat( (R \psubstn{z}{y}) \psubstp{Q}{P} ) \\
  (\lift{x}{R}) \psubstp{Q}{P}  
  :=
  \lift{(x)\substp{Q}{P}}{ R \psubstp{Q}{P} } \\
%   (\dropn{x})  \psubstp{Q}{P}       
%   := 
%   \left\{ 
%     \begin{array}{ccc} 
%       \dropn{\quotep{Q}} & & x \nameeq \quotep{P} \\
%       \dropn{x} & & otherwise \\
%     \end{array}
%   \right. 
  (\dropn{x})  \psubstp{Q}{P}       
  := 
  \left\{ 
    \begin{array}{ccc} 
      Q & & x \nameeq \quotep{P} \\
      \dropn{x} & & otherwise \\
    \end{array}
  \right.
\end{mathpar}
 

where

\begin{eqnarray}
  (x)\id{\{} \lpquote Q \rpquote / \lpquote P \rpquote \id{\}}            = 
  \left\{ 
    \begin{array}{ccc}
      \lpquote Q \rpquote & & x \nameeq \lpquote P \rpquote \\
      x & & otherwise \\
    \end{array}
  \right. \nonumber
\end{eqnarray}

and $z$ is chosen distinct from $\quotep{P}$, $\quotep{Q}$, the free
names in $Q$, and all the names in $R$. Our $\alpha$-equivalence will
be built in the standard way from this substitution.

\begin{remark}\label{rem:no_self_referential_names}
  One consequence of these definitions is that $\forall P. \quotep{P}
  \not\in \freenames{P}$.
\end{remark}

\subsection{ Dynamic quote: an example }

Anticipating something of what's to come, consider applying the
substitution, $\widehat{\id{\{}u / z \id{\}}}$, to the following pair
of processes, $\lift{w}{y!(z)}$ and $w[ \lpquote y!(z) \rpquote ]$.

\begin{eqnarray}
	\lift{w}{y!(z)}\widehat{\id{\{}u / z \id{\}}}
		& = &
		\lift{w}{y!(u)} \nonumber\\
	w[ \lpquote y!(z) \rpquote ] \widehat{ \id{\{}u / z \id{\}} }
		& = &
		w[ \lpquote y!(z) \rpquote ] \nonumber
\end{eqnarray}

Because the body of the process between quotes is impervious to
substitution, we get radically different answers. In fact, by
examining the first process in an input context,
e.g. $x?(z).\lift{w}{y!(z)}$, we see that the process under the lift
operator may be shaped by prefixed inputs binding a name inside it. In
this sense, the lift operator will be seen as a way to dynamically
construct processes before reifying them as names.

Finally equipped with these standard features we can present the
dynamics of the calculus.

\subsubsection{Operational semantics} 

Finally, we introduce the computational dynamics. What marks these
algebras as distinct from other more traditionally studied algebraic
structures, e.g. vector spaces or polynomial rings, is the manner in
which dynamics is captured. In traditional structures, dynamics is typically
expressed through morphisms between such structures, as in linear maps
between vector spaces or morphisms between rings. In algebras
associated with the semantics of computation, the dynamics is
expressed as part of the algebraic structure itself, through a
reduction reduction relation typically denoted by $\red$. Below, we
give a recursive presentation of this relation for the calculus used
in the encoding.

$\red \subseteq \pi \times \pi$
$\red : \pi \to \mathcal{P}(\pi)$

\begin{mathpar}
  \inferrule* [lab=Comm] { \textsf{match}( x_{src}, x_{trgt} ) } { x_{trgt}?(y)P \; | \; x_{src}!\langle {Q} \rangle \red P\{\quotep{Q}/y}\} }
  \and \\
  \inferrule* [lab=Par] {{P} \red {P}'} {{{P} | {Q}} \red {{P}' | {Q}}}
  \and
  \inferrule* [lab=Equiv]{{{P} \scong {P}'} \andalso {{P}' \red {Q}'} \andalso {{Q}' \scong {Q}}}{{P} \red {Q}}
\end{mathpar}

\begin{eqnarray*}
  match_{\equiv} (\quotep{P},\quotep{Q}) & := & P \equiv Q \\
  match_{\dagger}(\quotep{P},\quotep{Q}) & := & \forall R. P|Q \red^{*} R => R \red^{*} 0 \\
  match_{K}(\quotep{P},\quotep{Q}) & := & K \mbox{ for some context } K
\end{eqnarray*}

$u?(x)P | u!\langle Q \rangle \red P\{\quotep{Q}/x\}$

%We write $\wred$ for $\red^*$, and $P\red$ if $\exists Q $ such that $ P \red Q$.
We write $P\red$ if $\exists Q $ such that $ P \red Q$ and $P\not\red$, otherwise.

\section{Replication}

As mentioned before, it is known that replication (and hence
recursion) can be implemented in a higher-order process algebra
\cite{SangiorgiWalker}. As our first example of calculation with the
machinery thus far presented we give the construction explicitly in
the {\rhoc}.

\begin{eqnarray}
	D_{x} & := & \prefix{x}{y}{(\binpar{\outputp{x}{y}}{@{y}})} \nonumber\\
	\bangp_{x}{P} & := & \binpar{{x}!\langle{\binpar{D_{x}}{P}}\rangle}{D_{x}} \nonumber
\end{eqnarray}

\begin{eqnarray}
	\bangp_{x}{P} & & \nonumber\\
	=
	& {x}!\langle{(\prefix{x}{y}{(\outputp{x}{y} | @{y})) | P}}\rangle 
	      | \prefix{x}{y}{(\outputp{x}{y} | @{y})} & \nonumber\\
	\red
	& (\outputp{x}{y} | @{y})\substn{\quotep{(\prefix{x}{y}{(@{y} | \outputp{x}{y})) | P}}}{y} & \nonumber\\
	=
	& \outputp{x}{\quotep{(\prefix{x}{y}{(\outputp{x}{y} | @{y})) | P}}}
	  | {(\prefix{x}{y}{(\outputp{x}{y} | @{y})) | P}} & \nonumber\\
	\red
	& \ldots & \nonumber\\
	\red^*
	& P | P | \ldots & \nonumber
\end{eqnarray}

Of course, this encoding, as an implementation, runs away, unfolding
$\bangp{P}$ eagerly. A lazier and more implementable replication
operator, restricted to input-guarded processes, may be obtained as follows.

\begin{eqnarray}
\bangp{\prefix{u}{v}{P}} 
	:= 
	\binpar{\lift{x}{\prefix{u}{v}{(\binpar{D(x)}{P})}}}{D(x)} \nonumber
\end{eqnarray}

\begin{remark}
  Note that the lazier definition still does not deal with summation
  or mixed summation (i.e. sums over input and output). The reader is
  invited to construct definitions of replication that deal with these
  features. 

  Further, the definitions are parameterized in a name, $x$. Can you,
  gentle reader, make a definition that eliminates this parameter and
  guarantees no accidental interaction between the replication
  machinery and the process being replicated -- i.e. no accidental
  sharing of names used by the process to get its work done and the
  name(s) used by the replication to effect copying. This latter
  revision of the definition of replication is crucial to obtaining
  the expected identity $!!P \sim !P$.
\end{remark}

\begin{remark}\label{rem:paradoxical_combinator}
  The reader familiar with the lambda calculus will have noticed the
  similarity between $D$ and the paradoxical combinator.

  [Ed. note: the existence of this seems to suggest we have to be more
  restrictive on the set of processes and names we admit if we are to
  support no-cloning.]
\end{remark}

\subsubsection{Bisimulation}

The computational dynamics gives rise to another kind of equivalence,
the equivalence of computational behavior. As previously mentioned
this is typically captured \emph{via} some form of bisimulation.

% The notion we use in this paper is weak barbed bisimulation
% \cite{milner91polyadicpi}.

The notion we use in this paper is derived from weak barbed
bisimulation \cite{milner91polyadicpi}. 

\begin{definition}
An \emph{observation relation}, $\downarrow_{\mathcal N}$, over a set
of names, $\mathcal N$, is the smallest relation satisfying the rules
below.

\infrule[Out-barb]{y \in {\mathcal N}, \; x \nameeq y}
		  {\outputp{x}{v} \downarrow_{\mathcal N} x}
\infrule[Par-barb]{\mbox{$P\downarrow_{\mathcal N} x$ or $Q\downarrow_{\mathcal N} x$}}
		  {\binpar{P}{Q} \downarrow_{\mathcal N} x}

We write $P \Downarrow_{\mathcal N} x$ if there is $Q$ such that 
$P \wred Q$ and $Q \downarrow_{\mathcal N} x$.
\end{definition}

\begin{definition}
%\label{def.bbisim}
An  ${\mathcal N}$-\emph{barbed bisimulation} over a set of names, ${\mathcal N}$, is a symmetric binary relation 
${\mathcal S}_{\mathcal N}$ between agents such that $P\rel{S}_{\mathcal N}Q$ implies:
\begin{enumerate}
\item If $P \red P'$ then $Q \wred Q'$ and $P'\rel{S}_{\mathcal N} Q'$.
\item If $P\downarrow_{\mathcal N} x$, then $Q\Downarrow_{\mathcal N} x$.
\end{enumerate}
$P$ is ${\mathcal N}$-barbed bisimilar to $Q$, written
$P \wbbisim_{\mathcal N} Q$, if $P \rel{S}_{\mathcal N} Q$ for some ${\mathcal N}$-barbed bisimulation ${\mathcal S}_{\mathcal N}$.
\end{definition}

$\mathcal{R} \subseteq \pi \times \pi$

$P \mathcal{R} Q => \forall P'. P \red P' \Rightarrow \exists Q'. Q \red Q', P' \mathcal{R} Q'$

$P \vdash x \Rightarrow Q \vdash x$

\begin{mathpar}
  \inferrule*[lab=Out-barb]{x \nameeq y}{{y}!\langle{Q}\rangle \vdash x}
  \and
  \inferrule*[lab=Par-barb]{\mbox{$P\vdash x$ or $Q\vdash x$}}{\binpar{P}{Q} \vdash x}
\end{mathpar}

\subsubsection{Contexts}

One of the principle advantages of computational calculi like the
$\pi$-calculus is a well-defined notion of context,
contextual-equivalence and a correlation between
contextual-equivalence and notions of bisimulation. The notion of
context allows the decomposition of a process into (sub-)process and
its syntactic environment, its context. Thus, a context may be
thought of as a process with a ``hole'' (written $\Box$) in it. The
application of a context $M$ to a process $P$, written $M[P]$, is
tantamount to filling the hole in $M$ with $P$. In this paper we do
not need the full weight of this theory, but do make use of the notion
of context in the proof the main theorem. 

\begin{mathpar}
  \inferrule* [lab=summation] {} {{M_{M},M_{N}} \bc \Box \;|\; x.M_{A} \;|\; M_{M}+M_{N}}
  \and
  \inferrule* [lab=agent] {} {{M_{A}} \bc (\vec{x})M_{P} \;| \; \clift{P_0,\ldots,M_{P},\ldots,P_N}}
  \and \\
  \inferrule* [lab=process] {} {{M_{P}} \bc M_{N} \;| \;P|M_{P} }
\end{mathpar} 

\begin{mathpar}
  \inferrule* [lab=sychronization] {} {M_{N} \bc \Box \;|\; x?M_{F} \;|\; x!M_{C}}
  \and
  \inferrule* [lab=abstraction] {} {{M_{F}} \bc (x)M_{P} }
  \and
  \inferrule* [lab=concretion] {} {{M_{C}} \bc \langle M_{P} \rangle }
  \and \\
  \inferrule* [lab=process] {} {{M_{P}} \bc M_{N} \;| \;P|M_{P} }
\end{mathpar}

\begin{definition}[contextual application] Given a context $M$, and
  process $P$, we define the \emph{contextual application}, $M[P] :=
  M\{P/\Box\}$. That is, the contextual application of M to P is the
  substitution of $P$ for $\Box$ in $M$.
\end{definition}

$\meaningof{-} : L \to \mathcal{P}(\pi)$

\begin{mathpar}
  \inferrule* [lab=collection] {} {\meaningof{true} = \pi, \and \meaningof{~E} = \pi \setminus \meaningof{E}, \and \meaningof{E_{1} \& E_{2}} = \meaningof{E_{1}} \cap \meaningof{E_{2}}}
\end{mathpar}

\begin{mathpar}
  \inferrule* [lab=structure] {} {\meaningof{0} = \{ P \in \pi | P \equiv 0 \}, \and \\ \meaningof{E_1 | E_2} = \{ P \in \pi | P \equiv P_{1} | P_{2}, P_{1} \in \meaningof{E_{1}}, P_{2} \in \meaningof{E_2}\} }
\end{mathpar}

\begin{mathpar}
 \inferrule* [lab=behavior] {} {\meaningof{\langle a?b \rangle E} = \{ P \in \pi | P \equiv Q | u?(y)P', \\ \and \\\\ \and \\ \;\;\; u \in \meaningof{a}, \forall z.P'\{z/y\} \in \meaningof{E\{z/b\}}\}, \and \\ \meaningof{a!E} = \{ P \in \pi | P \equiv Q | x!\langle P' \rangle, x \in \meaningof{a} P' \in \meaningof{E}\} }
\end{mathpar}

\begin{mathpar}
 \inferrule* [lab=nominal] {} {\meaningof{\quotep{E}} = \{ \quotep{P} \in \quotep{\pi} | P \in \meaningof{E} \}, \and \meaningof{\quotep{P}} = \{ \quotep{Q} \in \quotep{\pi} | P \equiv Q \} \and \\ \meaningof{@\quotep{E}} = \{ P \in \pi | P \equiv @x, x \in \meaningof{E} \}}
\end{mathpar}

\begin{eqnarray*}
  \\
  \meaningof{-} : TS \to ST
\end{eqnarray*}

\begin{eqnarray*}
  \\
  L : TS \to ST
\end{eqnarray*}

\begin{eqnarray*}
  \\
  P \models E \iff P \in \meaningof{E}
\end{eqnarray*}

\begin{eqnarray*}
  P \approx_{L} Q \iff \forall E \in L. P \models E \iff Q \models E
\end{eqnarray*}

\begin{eqnarray*}
  P \approx_{K} Q
\end{eqnarray*}

\begin{eqnarray*}
  P \approx Q
\end{eqnarray*}

$\approx_{K} = \approx = \approx_{L}$

\subsubsection{Contextual duality}

Note that contexts extend the quotation operation to a family of
operations from processes to names. Given a context, $M$, we can
define a \emph{nominal context}, $\quotep{M}$ by $\quotep{M}[P] :=
\quotep{M[P]}$. To foreshadow what is to come we observe that these
operations enjoy a duality with processes very much like the duality
between vectors and maps from vectors to scalars.

Further, because the calculus is essentially higher-order, we have a
correspondence between contexts and processes. More specifically,
given a name $x$ and a context $M$ we can construct $M^{*}_{x}$ such
that 

\begin{mathpar}
  M^{*}_{x} | \lift{x}{P} \red M[P]
\end{mathpar}

namely,

\begin{mathpar}
  M^{*}_{x} := x?(u).M[\dropn{u}]
\end{mathpar}

The dependence of $M^{*}_{x}$ on a name makes it an abstraction, 

\begin{mathpar}
  M^{*} := (x)x?(u).M[\dropn{u}]
\end{mathpar}

\subsection{Additional notation}

It will sometimes be convenient to denote the process a name
quotes. We already have the notation $x = \quotep{P}$, but it will be
convenient to introduce an alternate notation, $\procn{x}$, when we
want to emphasize the connection to the use of the name. Note that, by
virtue of name equivalence, $\quotep{\procn{x}} \nameeq x$; so, the
notation is consistent with previous definitions.

Further, because names have structure it is possible to effect
substitutions on the basis of that structure. This means we need to
upgrade our notation for substitutions, which we accomplish by
adapting comprehension notation. Thus,

\begin{mathpar}
  P\{ y / x : x \in S \}
\end{mathpar}

is interpreted to mean the process derived from P by replacing (in a
capture-avoiding manner) each occurrence of $x$ in $S$ by $y$. For example,

\begin{mathpar}
  P\{ \quotep{\procn{x}|\procn{x}} / x : x \in \freenames{P} \}
\end{mathpar}

will replace each (occurrence) of a free name $x$ in $P$ by
$\quotep{\procn{x}|\procn{x}}$.

Also, we will avail ourselves of the notation $x^{L}$ and $x^{R}$ to
denote injections of a name into disjoint copies of the name
space. There are numerous ways to accomplish this. One example can be
found in \cite{MeredithR05}. This notation overloads to vectors of
names: $\vec{x}^{\pi} := (x_{i}^{\pi} \; : \; 0 \leq i < |\vec{x}| )$ where $\pi \in \{L,R\}$.

We also use $P^{\Box} := P|\Box$.

In \cite{MeredithR05} an interpretation of the new operator is
given. It turns out that there are several possible interpretations
all enjoying the requisite algebraic properties of the operator (see
\cite{milner91polyadicpi}). We will therefore make liberal use of
$(\nu\; \vec{x})P$.

% subsection the_syntax_and_semantics_of_the_notation_system (end)   

\section{Interpretation of QM}
\subsection{Supporting definitions}
\subsubsection{Multiplication}
\begin{mathpar}
  \quotep{Q} \cdot \quotep{R} := \quotep{Q|R}
  \and \\
  \quotep{Q} \cdot P := P\{ \quotep{Q|R} / \quotep{R} : \quotep{R} \in \freenames{P} \}
\end{mathpar}

\paragraph{Discussion}
The first line needs little explanation. The second line says that
each free name of the process is replaced with the multiplication of
that name by the scalar. Multiplication of a scalar (name) by a state
(process) results in a process all the names of which have been `moved
over' by parallel composition with the process the scalar
quotes. There is a subtlety that the bound names have to be
manipulated so that multiplied names aren't accidentally
captured. There are many ways to achieve this.

\begin{remark}\label{rem:multiplication_identities}
  The reader is invited to verify that for all $x,y,z \in \QProc$ and $P \in \Proc$
  \begin{mathpar}
    x \cdot \quotep{0} \equiv x 
    \and
    x \cdot y \equiv y \cdot x
    \and
    x \cdot (y \cdot z) \equiv (x \cdot y) \cdot z
    \and \\
    \quotep{0} \cdot P \equiv P
    \and \\
    x \cdot (y \cdot P) \equiv (x \cdot y) \cdot P
    \and \\
    x \cdot (P|Q) \equiv (x \cdot P) | (x \cdot Q)
    \and \\    
  \end{mathpar}
\end{remark}

\subsubsection{Tensor product}

We define a tensor product on processes by structural induction.

\paragraph{Tensor of sums} First note that all summations, including
$\pzero$ and sequence, can be written $\Sigma_{i} x_{i}.A_{i} +
\Sigma_{j} x_{j}.C_{j}$, where we have grouped input-guarded processes
together and output-guarded processes together.

Thus, we can define the tensor product of two summations, $N_{1}\otimes N_{2}$, where

\begin{mathpar}
  N_{1} := \Sigma_{i} x_{i}.A_{i} + \Sigma_{j} x_{j}.C_{j}
  \and
  N_{2} := \Sigma_{i'} y_{i'}.B_{i'} + \Sigma_{j'} y_{j'}.D_{j'} 
\end{mathpar}

as follows.

\begin{mathpar}
  \Sigma_{i} x_{i}.A_{i} + \Sigma_{j} x_{j}.C_{j} \otimes \Sigma_{i'}
  y_{i'}.B_{i'} + \Sigma_{j'} y_{j'}.D_{j'} 
  \and \\
  := \; \Sigma_{i} \Sigma_{i'} \quotep{\stackrel{\vee}{x_{i}}| \stackrel{\vee}{y_{i'}}}.(A_{i}\otimes B_{i'}) \; | \; \Sigma_{i'} \Sigma_{i} \quotep{\stackrel{\vee}{y_{i'}}|\stackrel{\vee}{x_{i}}}.(B_{i'}\otimes A_{i})
  \and
  \;\; | \;\; \Sigma_{j} \Sigma_{j'} \quotep{\stackrel{\vee}{x_{j}}|\stackrel{\vee}{y_{j'}}}.(A_{j}\otimes B_{j'}) \; | \; \Sigma_{j'} \Sigma_{j} \quotep{\stackrel{\vee}{y_{j'}}|\stackrel{\vee}{x_{j}}}.(B_{j'}\otimes A_{j})
\end{mathpar}

\begin{remark}
  Do we need to $x^{L}$ and $y^{R}$ for this construction as well?
\end{remark}

\paragraph{Tensor of parallel compositions} Next, we distribute tensor
over par.

\begin{mathpar}
  P_{1}|P_{2} \otimes Q_{1}|Q_{2} := (P_{1} \otimes Q_{1}) | (P_{1}
  \otimes Q_{2}) | (P_{2} \otimes Q_{1}) | (P_{2} \otimes Q_{2})
\end{mathpar}

\paragraph{Tensor with dropped names} We treat tensor of a
process with a dropped name as parallel composition.

\begin{mathpar}
  P \otimes \dropn{x} := P | \dropn{x}
\end{mathpar}

\paragraph{Tensor of agents}

Finally, we need to define tensor on agents. Note that the definition
of tensor on normal products only tensors inputs with inputs and
outputs with outputs. Thus, we only have to define the operation on
``homogeneous'' pairings.

\begin{mathpar}
  (\vec{x})P \otimes (\vec{y})Q
  \and \\
  := (x_{0}^{L}|y_{0}^{R},\ldots,x_{0}^{L}|y_{n}^{R},\ldots,x_{m}^{L}|y_{0}^{R},\ldots,x_{m}^{L}|y_{n}^R)(P\{ \vec{x}^{L}/\vec{x}\} \otimes Q \{ \vec{y}^{R}/\vec{y}\})
  \and \\
  \clift{\vec{P}} \otimes \clift{\vec{Q}}
  \and \\
  := \clift{P_{0}\otimes Q_{0},\ldots,P_{0}\otimes Q_{n},\ldots,P_{m}\otimes Q_{0},\ldots,P_{m}\otimes Q_{n}}
\end{mathpar}

\begin{remark}
  Observe that arities of tensored abstractions matches arities of
  tensored concretions if the original arities matched. Note also that
  the length of the arities corresponds to the increase in dimension
  we see in ordinary vector space tensor product.
\end{remark}

\begin{remark}
  Operationally, this definition distributes the tensor down to
  components ``linked'' by summation. Tensor over summation is
  intriguing in that it mixes names. Moreover, as a consequence of the
  way it mixes names we have the identities for all $x \in \QProc$ and
  $P,Q \in \Proc$

  \begin{mathpar}
    (x \cdot P) \otimes Q \equiv x \cdot (P \otimes Q) \equiv P \otimes (x \cdot Q)
    \and
    P \otimes \pzero \equiv P
  \end{mathpar}

  that the reader is invited to verify.
\end{remark}

\subsubsection{Annihilation}
\begin{mathpar}
  P^{\perp} := \{ Q | \forall R. P|Q \red^{*} R \Rightarrow R \red^{*} \pzero \}
  \and \\
  P^{\underline{\perp}} := \Sigma_{Q \in P^{\perp}} \quotep{Q}?(y).(\dropn{y}|Q) | \Sigma_{Q \in P^{\perp}} \quotep{Q}\clift{\Box}
\end{mathpar}

\paragraph{Discussion} The reader will note that $P^{\perp}$ is a
\emph{set} of processes, while $P^{\underline{\perp}}$ is a
\emph{context}. We call the set $P^{\perp}$ the \emph{annihilators} of
$P$. The parallel composition of a process in the annihilators of $P$
with $P$ will result in a process, the state space of which has all
paths eventually leading to $\pzero$. Execution may endure loops; but
under reasonable conditions of fairness (naturally guaranteed under
most notions of bisimulation) such a composite process cannot get
stuck in such a loop and will, eventually pop out and terminate.

The context $P^{\underline{\perp}}$ is ready and willing to ``take the
$P$ out of'' the process to which it is applied. It will effectively
transmit the code of the process to which it is applied to one of the
annihilators and run the process against it.

\subsubsection{Evaluation}
We fix $M$ a domain of fully abstract interpretation with an equality
coincident with bisimulation. We take $\meaningof{\cdot} : \Proc \to
M$ to be the map interpreting processes and $\nmeaningof{\cdot} : \M
\to Proc$ to be the map running the other way. Then we define

\begin{mathpar}
  \int P := \nmeaningof{\meaningof{P}}
\end{mathpar}

\paragraph{Discussion}
There are many fully abstract interpretations of Milner's
$\pi$-calculus. Any of them can be used as a basis for interpreting
the reflective calculus here. Equipped with such a domain it is
largely a matter of grinding through to check that the Yoneda
construction for the normalization-by-evaluation program can be
extended to this setting.

\begin{remark}
  The reader is invited to verify that $\int (P^{\underline{\perp}}[P]) = 0$.
\end{remark}

\subsection{Quantum mechanics}

Table \ref{tbl:core_qm_op_defns} gives the core operational definitions

\begin{table}[htp]\label{tbl:core_qm_op_defns}
  \center{
    \fbox{
      \begin{tabular}{c|c}
        quantum mechanics & process calculus \\
        \hline
        scalar & $x := \quotep{P}$ \\
        state vector & $\state{P} := P$ \\
        dual & $\state{P}^{*} := \event{P^{\underline{\perp}}} := \quotep{P^{\underline{\perp}}}[-]$ \\
        matrix & $ \Sigma_{\alpha} \state{P_{\alpha}}x_{\alpha}\event{Q_{\alpha}}$ \\
        vector addition & $\state{P} + \state{Q} := \state{P | Q}$ \\
        tensor product & $\state{P} \otimes \state{Q} := \state{P \otimes Q}$ \\
        inner product & $\innerprod{P}{Q} := \quotep{\int P^{\underline{\perp}}[Q]}$ \\
      \end{tabular}
    }
  }
  \caption{QM - operational definitions}
\end{table}

where

\begin{mathpar}
  \prmatrix{P}{Q} := \fprmatrix{P}{\quotep{\pzero}}{Q}
  \and
  \fprmatrix{P}{x}{Q} := (\state{P},x,\event{Q})
  \and
  (\fprmatrix{P}{x}{Q})(\state{R}) := x \cdot \innerprod{Q}{R} \cdot \state{P}
  \and
  (\fprmatrix{P}{x}{Q})(\event{R}) := x \cdot \innerprod{R}{P} \cdot \event{Q}
\end{mathpar}

\paragraph{Discussion}
As promised: vectors (aka states) are represented as processes; duals
as contextual duals; inner product definition should be compared with
standard inner product definition for ....

\begin{remark}
  Assuming $\int (P^{\underline{\perp}}[P]) = 0$, the reader is
  invited to verify that $(\fprmatrix{P}{x}{P})(\state{P}) = x \cdot \state{P}$.
\end{remark}

\begin{remark}
  The reader is invited to verify that $\innerprod{P}{Q}$ could
  equally well have been written $\quotep{\int \stackrel{\vee}{x}}$
  where $x = \event{P^{\underline{\perp}}}(Q)$.

  One of the motivations for this remark is that there is another way
  to factor these operations. We could package up evaluation in the dual:

  \begin{mathpar}
    \state{P}^{*} := \event{\int P^{\underline{\perp}}} := \quotep{\int P^{\underline{\perp}}}[-]
  \end{mathpar}

  and then have inner product defined by
  
  \begin{mathpar}
    \innerprod{P}{Q} := \event{P}(Q)
  \end{mathpar}

  Hopefully, experience with the calculations will provide guidance on
  the best factoring.
\end{remark}

\begin{remark}
  Assuming $\int (P^{\underline{\perp}}[P]) = 0$, the reader is
  invited to verify that $\forall P,Q. (\prmatrix{0}{Q})(\state{0}) =
  \state{0}$ and dually $(\prmatrix{P}{0})(\event{0}) = \event{0}$.
\end{remark}

\begin{remark}
  i'm a little worried that i don't (yet) have proper support for
  complex conjugacy. But, the observation above may give us a
  clue. According to Abramsky, it must be the case that the scalars
  are iso to the homset of the identity for the tensor -- which the
  observation above characterizes. 

  For now, we will simply bookmark the notion with $\overline{x}$.
\end{remark}

\subsubsection{Adjointness}

We need to give a definition of $(\cdot)^{\dagger}$ for matrices. The
obvious candidate definition is
\begin{mathpar}
(\Sigma_{\alpha}\fprmatrix{P_{\alpha}}{x_{\alpha}}{Q_{\alpha}})^{\dagger}
= \Sigma_{\alpha}\fprmatrix{(Q_{\alpha}^{\underline{\perp}})^{*}}{\overline{x}_{\alpha}}{P_{\alpha}^{\underline{\perp}}} 
\end{mathpar}

But, $(Q_{\alpha}^{\underline{\perp}})^{*}$ requires a name along
which to communicate the process to achieve the context application.

\subsubsection{Basis for a basis}
If processes label states and ``addition'' of states (a.k.a. vector
addition) is interpreted as parallel composition, what corresponds to
notions of linear independence and basis? Here, we recall that Yoshida
has developed a set of \emph{combinators} for an asynchronous verison
of Milner's $\pi$-calculus. These are a finite set of processes such
any process can be expressed as parallel composition of these
combinators together with liberal uses of the new operator and
replication. We can simply give a translation of these into the
present calculus and have reasonable expectation that the property
carries over. That is, that the resultant set allows to express all
processes via parallel composition. Note, however, that there is no
new operator or replication in this calculus. As a result, we expect
that the corresponding set is actually infinite. That is, we expect
that the space is actually infinite dimensional.

\begin{remark}
  The attentive reader may be a bit concerned. Certainly, the
  collection $S$, $K$ and $I$ is a finite set of
  combinators. Shouldn't we expect to see a finite set of combinators
  for an effectively equivalent system? i am very sympathetic to this
  critique and feel it warrants full attention. On the other hand, i
  also have in mind the following analogy. The natural numbers, as a
  monoid under addition, has exactly $1$ generator, while the natural
  numbers, as a monoid under multiplication, has countably many
  generators (the primes). We observe that the application of the
  lambda calculus is much less resource sensitive than the parallel
  composition of the $\pi$-calculus. Could it be the case that we have
  an analogy of the form
  
  \begin{mathpar}
    m + n : MN :: m*n : M|N
  \end{mathpar}

  giving a similar blow up in the set of ``primes''?  This is such a
  wonderful thought that, even if it's not true, i think it's worth
  writing down.
\end{remark}
 

\documentclass[12pt]{llncs}
%\documentclass{jktr}

\usepackage[pdftex]{hyperref}                   
\usepackage {listings}
\usepackage {mathpartir}
\usepackage{bcprules}
%\usepackage{listings}
                       
\usepackage{graphicx} 
%\usepackage[margins=2.5cm,nohead,nofoot]{geometry}
%\usepackage{geometry}
\usepackage{amsfonts}
\usepackage{amstext}
\usepackage{latexsym}
\usepackage{amssymb}
\usepackage{color}


%\include{myPreamble}
\include{qm2pi.local} 

%\ifpdf
%\usepackage[pdftex]{graphicx}
%\else
%\usepackage{graphicx}
%\fi

 % \ifpdf
%  \usepackage{pdfsync}
%  \if


%\title{Brief Article}
%\author{David F. Snyder}
%\author{L.G. Meredith}

%\address{Dept. of Math., Texas State University--San Marcos, San Marcos, TX 78666}
       
\pagestyle{empty}


\begin{document}

\lstset{language=[Objective]Caml,frame=shadowbox}

\input{qm2pi.front}

% section front matter (end)

\input{qm2pi.intro} 
 
% section introduction (end)

% \input{qm2pi.knotations} 

% section notation (end)

\input{qm2pi.process.calculi} 

% section concurrent_process_calculi_and_spatial_logics_ (end)
    
%\input{qm2pi.knots2pi} 

%\input{qm2pi.trefoil} 

%\input{qm2pi.mainthm} 

% subsection basic_interpretation (end)

%\input{qm2pi.rho.presentation} 
\subsection{The syntax and semantics of the notation system}\label{sub:the_syntax_and_semantics_of_the_notation_system} % (fold)

We now summarize a technical presentation of the calculus that
embodies our theory of dynamics. The typical presentation of such a
calculus follows the style of giving generators and relations on
them. The grammar, below, describing term constructors, freely
generates the set of processes, $\Proc$. This set is then quotiented
by a relation known as structural congruence and it is over this set
that the notion of dynamics is expressed. This presentation is
essentially that of \cite{MeredithR05} with the addition of
polyadicity and summation. For readability we have relegated some of
the technical subtleties to an appendix.

\subsubsection{Process grammar}\label{subsub:process_grammar}

\begin{mathpar}
  \inferrule* [lab=synchronization] {} {{M} \bc \pzero \;|\; x?F \;|\; x!C }
  \and
  \inferrule* [lab=abstraction] {} {{F} \bc (x)P}
  \and
  \inferrule* [lab=concretion] {} {{C} \bc \langle Q \rangle}
  \and
  \inferrule* [lab=process] {} {{P,Q} \bc M \;| \;P|Q \;|\; @{x}}
  \and
  \inferrule* [lab=name] {} {{x} \bc \quotep{P}}
\end{mathpar} 

Note that $\vec{x}$ (resp. $\vec{P}$) denotes a vector of names
(resp. processes) of length $|\vec{x}|$ (resp. $|\vec{P}|$). We adopt
the following useful abbreviations.

\begin{mathpar}
   x?(\vec{y}).P := x.(\vec{y})P \and  x\clift{\vec{P}} := x.\clift{\vec{P}}
   \and x!(y) := \lift{x}{\dropn{y}}
   \and \Pi_{i=0}^{n-1}P_i := P_0 | \ldots | P_{n-1}
\end{mathpar}

\subsubsection{Structural congruence}

\paragraph{Free and bound names and alpha-equivalence.} At the
core of structural equivalence is alpha-equivalence which identifies
process that are the same up to a change of variable. Formally, we
recognize the distinction between free and bound names. The free names
of a process, $\freenames{P}$, may be calculated recursively as
follows:

\begin{mathpar}
\freenames{\pzero} := \emptyset
  \and \\
  \freenames{x?(y).P} := \{ x \} \cup (\freenames{P} \setminus \{ y \})
  \and 
  \freenames{x!\langle P \rangle} := \{ x \} \cup \{ P \} 
  \and \\
  \freenames{P|Q} := \freenames{P} \cup \freenames{Q}
  \and \\
  \freenames{@{x}} := \{ x \}
\end{mathpar}

$\pi$
$\quotep{\pi}$

$\freenames{-} : \pi \to \mathcal{P}(\quotep{\pi})$

\begin{eqnarray*}
  \freenames{\pzero} & := & \emptyset \\
  \freenames{x?(y).P} & := & \{ x \} \cup (\freenames{P} \setminus \{ y \}) \\
  \freenames{x!\langle P \rangle} & := & \{ x \} \cup \{ P \} \\
  \freenames{P|Q} & := & \freenames{P} \cup \freenames{Q} \\
  \freenames{\dropn{x}} & := & \{ x \}
\end{eqnarray*}

The bound names of a process, $\boundnames{P}$, are those names occurring in $P$
that are not free. For example, in $x?(y).0$, the name $x$ is free, while $y$ is bound.

\begin{mathpar}
  \inferrule* [lab=monoidal-laws] {} { P|Q \equiv Q|P \and P|0 \equiv P \and P|(Q|R) \equiv (P|Q)|R }
\end{mathpar}

\begin{mathpar}
  \inferrule* [lab=alpha-equivalence] {} { (x)P \equiv (y)P\{y/x\} \and y \not\in \freenames{P} }
\end{mathpar}

\begin{definition}
Then two processes, $P,Q$, are alpha-equivalent if $P = Q\{\vec{y}/\vec{x}\}$ for
some $\vec{x} \in \boundnames{Q},\vec{y} \in \boundnames{P}$, where $Q\{\vec{y}/\vec{x}\}$
denotes the capture-avoiding substitution of $\vec{y}$ for $\vec{x}$ in $Q$.
\end{definition}

\begin{definition}
  The {\em structural congruence} \cite{SangiorgiWalker} , $\equiv$,
  between processes is the least congruence containing
  alpha-equivalence, satisfying the abelian monoid laws
  (associativity, commutativity and $\pzero$ as identity) for parallel
  composition $|$ and for summation $+$.
\end{definition}

\subsection{Name equivalence}

We take name equivalence, written $\nameeq$, to be the smallest
equivalence relation generated by the following rules.

\begin{mathpar}
\inferrule*[lab=Quote-drop]
{ }
{ \quotep{@{x}} \nameeq x }

\inferrule*[lab=Struct-equiv]
{ P \scong Q }
{ \quotep{P} \nameeq \quotep{Q} }
\end{mathpar}

The astute reader will have noticed that the mutual recursion of names
and processes imposes a mutual recursion on alpha-equivalence and
structural equivalence via name-equivalence. Fortunately, all of this
works out pleasantly and we may calculate in the natural way, free of
concern. The reader interested in the details is referred to the
appendix \ref{appendix:rho_details}.

\subsection{Substitution}

We use $\Proc$ for the set of processes, $\QProc$ for the set of
names, and $\id{\{}\vec{y} / \vec{x} \id{\}}$ to denote partial maps,
$s : \QProc \rightarrow \QProc$. A map, $s$ lifts, uniquely, to a map
on process terms, $\widehat{s} : \Proc \rightarrow \Proc$ by the
following equations.

\begin{mathpar}
  (0) \psubstp{Q}{P} := 0 \\
  (R \juxtap S) \psubstp{Q}{P}
  :=    
  (R)\psubstp{Q}{P} \juxtap (S) \psubstp{Q}{P} \\
  (x?(y).R) \psubstp{Q}{P}    
  :=    
  (x)\substp{Q}{P} (z)\concat( (R \psubstn{z}{y}) \psubstp{Q}{P} ) \\
  (\lift{x}{R}) \psubstp{Q}{P}  
  :=
  \lift{(x)\substp{Q}{P}}{ R \psubstp{Q}{P} } \\
%   (\dropn{x})  \psubstp{Q}{P}       
%   := 
%   \left\{ 
%     \begin{array}{ccc} 
%       \dropn{\quotep{Q}} & & x \nameeq \quotep{P} \\
%       \dropn{x} & & otherwise \\
%     \end{array}
%   \right. 
  (\dropn{x})  \psubstp{Q}{P}       
  := 
  \left\{ 
    \begin{array}{ccc} 
      Q & & x \nameeq \quotep{P} \\
      \dropn{x} & & otherwise \\
    \end{array}
  \right.
\end{mathpar}
 

where

\begin{eqnarray}
  (x)\id{\{} \lpquote Q \rpquote / \lpquote P \rpquote \id{\}}            = 
  \left\{ 
    \begin{array}{ccc}
      \lpquote Q \rpquote & & x \nameeq \lpquote P \rpquote \\
      x & & otherwise \\
    \end{array}
  \right. \nonumber
\end{eqnarray}

and $z$ is chosen distinct from $\quotep{P}$, $\quotep{Q}$, the free
names in $Q$, and all the names in $R$. Our $\alpha$-equivalence will
be built in the standard way from this substitution.

\begin{remark}\label{rem:no_self_referential_names}
  One consequence of these definitions is that $\forall P. \quotep{P}
  \not\in \freenames{P}$.
\end{remark}

\subsection{ Dynamic quote: an example }

Anticipating something of what's to come, consider applying the
substitution, $\widehat{\id{\{}u / z \id{\}}}$, to the following pair
of processes, $\lift{w}{y!(z)}$ and $w[ \lpquote y!(z) \rpquote ]$.

\begin{eqnarray}
	\lift{w}{y!(z)}\widehat{\id{\{}u / z \id{\}}}
		& = &
		\lift{w}{y!(u)} \nonumber\\
	w[ \lpquote y!(z) \rpquote ] \widehat{ \id{\{}u / z \id{\}} }
		& = &
		w[ \lpquote y!(z) \rpquote ] \nonumber
\end{eqnarray}

Because the body of the process between quotes is impervious to
substitution, we get radically different answers. In fact, by
examining the first process in an input context,
e.g. $x?(z).\lift{w}{y!(z)}$, we see that the process under the lift
operator may be shaped by prefixed inputs binding a name inside it. In
this sense, the lift operator will be seen as a way to dynamically
construct processes before reifying them as names.

Finally equipped with these standard features we can present the
dynamics of the calculus.

\subsubsection{Operational semantics} 

Finally, we introduce the computational dynamics. What marks these
algebras as distinct from other more traditionally studied algebraic
structures, e.g. vector spaces or polynomial rings, is the manner in
which dynamics is captured. In traditional structures, dynamics is typically
expressed through morphisms between such structures, as in linear maps
between vector spaces or morphisms between rings. In algebras
associated with the semantics of computation, the dynamics is
expressed as part of the algebraic structure itself, through a
reduction reduction relation typically denoted by $\red$. Below, we
give a recursive presentation of this relation for the calculus used
in the encoding.

$\red \subseteq \pi \times \pi$
$\red : \pi \to \mathcal{P}(\pi)$

\begin{mathpar}
  \inferrule* [lab=Comm] { \textsf{match}( x_{src}, x_{trgt} ) } { x_{trgt}?(y)P \; | \; x_{src}!\langle {Q} \rangle \red P\{\quotep{Q}/y}\} }
  \and \\
  \inferrule* [lab=Par] {{P} \red {P}'} {{{P} | {Q}} \red {{P}' | {Q}}}
  \and
  \inferrule* [lab=Equiv]{{{P} \scong {P}'} \andalso {{P}' \red {Q}'} \andalso {{Q}' \scong {Q}}}{{P} \red {Q}}
\end{mathpar}

\begin{eqnarray*}
  match_{\equiv} (\quotep{P},\quotep{Q}) & := & P \equiv Q \\
  match_{\dagger}(\quotep{P},\quotep{Q}) & := & \forall R. P|Q \red^{*} R => R \red^{*} 0 \\
  match_{K}(\quotep{P},\quotep{Q}) & := & K \mbox{ for some context } K
\end{eqnarray*}

$u?(x)P | u!\langle Q \rangle \red P\{\quotep{Q}/x\}$

%We write $\wred$ for $\red^*$, and $P\red$ if $\exists Q $ such that $ P \red Q$.
We write $P\red$ if $\exists Q $ such that $ P \red Q$ and $P\not\red$, otherwise.

\section{Replication}

As mentioned before, it is known that replication (and hence
recursion) can be implemented in a higher-order process algebra
\cite{SangiorgiWalker}. As our first example of calculation with the
machinery thus far presented we give the construction explicitly in
the {\rhoc}.

\begin{eqnarray}
	D_{x} & := & \prefix{x}{y}{(\binpar{\outputp{x}{y}}{@{y}})} \nonumber\\
	\bangp_{x}{P} & := & \binpar{{x}!\langle{\binpar{D_{x}}{P}}\rangle}{D_{x}} \nonumber
\end{eqnarray}

\begin{eqnarray}
	\bangp_{x}{P} & & \nonumber\\
	=
	& {x}!\langle{(\prefix{x}{y}{(\outputp{x}{y} | @{y})) | P}}\rangle 
	      | \prefix{x}{y}{(\outputp{x}{y} | @{y})} & \nonumber\\
	\red
	& (\outputp{x}{y} | @{y})\substn{\quotep{(\prefix{x}{y}{(@{y} | \outputp{x}{y})) | P}}}{y} & \nonumber\\
	=
	& \outputp{x}{\quotep{(\prefix{x}{y}{(\outputp{x}{y} | @{y})) | P}}}
	  | {(\prefix{x}{y}{(\outputp{x}{y} | @{y})) | P}} & \nonumber\\
	\red
	& \ldots & \nonumber\\
	\red^*
	& P | P | \ldots & \nonumber
\end{eqnarray}

Of course, this encoding, as an implementation, runs away, unfolding
$\bangp{P}$ eagerly. A lazier and more implementable replication
operator, restricted to input-guarded processes, may be obtained as follows.

\begin{eqnarray}
\bangp{\prefix{u}{v}{P}} 
	:= 
	\binpar{\lift{x}{\prefix{u}{v}{(\binpar{D(x)}{P})}}}{D(x)} \nonumber
\end{eqnarray}

\begin{remark}
  Note that the lazier definition still does not deal with summation
  or mixed summation (i.e. sums over input and output). The reader is
  invited to construct definitions of replication that deal with these
  features. 

  Further, the definitions are parameterized in a name, $x$. Can you,
  gentle reader, make a definition that eliminates this parameter and
  guarantees no accidental interaction between the replication
  machinery and the process being replicated -- i.e. no accidental
  sharing of names used by the process to get its work done and the
  name(s) used by the replication to effect copying. This latter
  revision of the definition of replication is crucial to obtaining
  the expected identity $!!P \sim !P$.
\end{remark}

\begin{remark}\label{rem:paradoxical_combinator}
  The reader familiar with the lambda calculus will have noticed the
  similarity between $D$ and the paradoxical combinator.

  [Ed. note: the existence of this seems to suggest we have to be more
  restrictive on the set of processes and names we admit if we are to
  support no-cloning.]
\end{remark}

\subsubsection{Bisimulation}

The computational dynamics gives rise to another kind of equivalence,
the equivalence of computational behavior. As previously mentioned
this is typically captured \emph{via} some form of bisimulation.

% The notion we use in this paper is weak barbed bisimulation
% \cite{milner91polyadicpi}.

The notion we use in this paper is derived from weak barbed
bisimulation \cite{milner91polyadicpi}. 

\begin{definition}
An \emph{observation relation}, $\downarrow_{\mathcal N}$, over a set
of names, $\mathcal N$, is the smallest relation satisfying the rules
below.

\infrule[Out-barb]{y \in {\mathcal N}, \; x \nameeq y}
		  {\outputp{x}{v} \downarrow_{\mathcal N} x}
\infrule[Par-barb]{\mbox{$P\downarrow_{\mathcal N} x$ or $Q\downarrow_{\mathcal N} x$}}
		  {\binpar{P}{Q} \downarrow_{\mathcal N} x}

We write $P \Downarrow_{\mathcal N} x$ if there is $Q$ such that 
$P \wred Q$ and $Q \downarrow_{\mathcal N} x$.
\end{definition}

\begin{definition}
%\label{def.bbisim}
An  ${\mathcal N}$-\emph{barbed bisimulation} over a set of names, ${\mathcal N}$, is a symmetric binary relation 
${\mathcal S}_{\mathcal N}$ between agents such that $P\rel{S}_{\mathcal N}Q$ implies:
\begin{enumerate}
\item If $P \red P'$ then $Q \wred Q'$ and $P'\rel{S}_{\mathcal N} Q'$.
\item If $P\downarrow_{\mathcal N} x$, then $Q\Downarrow_{\mathcal N} x$.
\end{enumerate}
$P$ is ${\mathcal N}$-barbed bisimilar to $Q$, written
$P \wbbisim_{\mathcal N} Q$, if $P \rel{S}_{\mathcal N} Q$ for some ${\mathcal N}$-barbed bisimulation ${\mathcal S}_{\mathcal N}$.
\end{definition}

$\mathcal{R} \subseteq \pi \times \pi$

$P \mathcal{R} Q => \forall P'. P \red P' \Rightarrow \exists Q'. Q \red Q', P' \mathcal{R} Q'$

$P \vdash x \Rightarrow Q \vdash x$

\begin{mathpar}
  \inferrule*[lab=Out-barb]{x \nameeq y}{{y}!\langle{Q}\rangle \vdash x}
  \and
  \inferrule*[lab=Par-barb]{\mbox{$P\vdash x$ or $Q\vdash x$}}{\binpar{P}{Q} \vdash x}
\end{mathpar}

\subsubsection{Contexts}

One of the principle advantages of computational calculi like the
$\pi$-calculus is a well-defined notion of context,
contextual-equivalence and a correlation between
contextual-equivalence and notions of bisimulation. The notion of
context allows the decomposition of a process into (sub-)process and
its syntactic environment, its context. Thus, a context may be
thought of as a process with a ``hole'' (written $\Box$) in it. The
application of a context $M$ to a process $P$, written $M[P]$, is
tantamount to filling the hole in $M$ with $P$. In this paper we do
not need the full weight of this theory, but do make use of the notion
of context in the proof the main theorem. 

\begin{mathpar}
  \inferrule* [lab=summation] {} {{M_{M},M_{N}} \bc \Box \;|\; x.M_{A} \;|\; M_{M}+M_{N}}
  \and
  \inferrule* [lab=agent] {} {{M_{A}} \bc (\vec{x})M_{P} \;| \; \clift{P_0,\ldots,M_{P},\ldots,P_N}}
  \and \\
  \inferrule* [lab=process] {} {{M_{P}} \bc M_{N} \;| \;P|M_{P} }
\end{mathpar} 

\begin{mathpar}
  \inferrule* [lab=sychronization] {} {M_{N} \bc \Box \;|\; x?M_{F} \;|\; x!M_{C}}
  \and
  \inferrule* [lab=abstraction] {} {{M_{F}} \bc (x)M_{P} }
  \and
  \inferrule* [lab=concretion] {} {{M_{C}} \bc \langle M_{P} \rangle }
  \and \\
  \inferrule* [lab=process] {} {{M_{P}} \bc M_{N} \;| \;P|M_{P} }
\end{mathpar}

\begin{definition}[contextual application] Given a context $M$, and
  process $P$, we define the \emph{contextual application}, $M[P] :=
  M\{P/\Box\}$. That is, the contextual application of M to P is the
  substitution of $P$ for $\Box$ in $M$.
\end{definition}

$\meaningof{-} : L \to \mathcal{P}(\pi)$

\begin{mathpar}
  \inferrule* [lab=collection] {} {\meaningof{true} = \pi, \and \meaningof{~E} = \pi \setminus \meaningof{E}, \and \meaningof{E_{1} \& E_{2}} = \meaningof{E_{1}} \cap \meaningof{E_{2}}}
\end{mathpar}

\begin{mathpar}
  \inferrule* [lab=structure] {} {\meaningof{0} = \{ P \in \pi | P \equiv 0 \}, \and \\ \meaningof{E_1 | E_2} = \{ P \in \pi | P \equiv P_{1} | P_{2}, P_{1} \in \meaningof{E_{1}}, P_{2} \in \meaningof{E_2}\} }
\end{mathpar}

\begin{mathpar}
 \inferrule* [lab=behavior] {} {\meaningof{\langle a?b \rangle E} = \{ P \in \pi | P \equiv Q | u?(y)P', \\ \and \\\\ \and \\ \;\;\; u \in \meaningof{a}, \forall z.P'\{z/y\} \in \meaningof{E\{z/b\}}\}, \and \\ \meaningof{a!E} = \{ P \in \pi | P \equiv Q | x!\langle P' \rangle, x \in \meaningof{a} P' \in \meaningof{E}\} }
\end{mathpar}

\begin{mathpar}
 \inferrule* [lab=nominal] {} {\meaningof{\quotep{E}} = \{ \quotep{P} \in \quotep{\pi} | P \in \meaningof{E} \}, \and \meaningof{\quotep{P}} = \{ \quotep{Q} \in \quotep{\pi} | P \equiv Q \} \and \\ \meaningof{@\quotep{E}} = \{ P \in \pi | P \equiv @x, x \in \meaningof{E} \}}
\end{mathpar}

\begin{eqnarray*}
  \\
  \meaningof{-} : TS \to ST
\end{eqnarray*}

\begin{eqnarray*}
  \\
  L : TS \to ST
\end{eqnarray*}

\begin{eqnarray*}
  \\
  P \models E \iff P \in \meaningof{E}
\end{eqnarray*}

\begin{eqnarray*}
  P \approx_{L} Q \iff \forall E \in L. P \models E \iff Q \models E
\end{eqnarray*}

\begin{eqnarray*}
  P \approx_{K} Q
\end{eqnarray*}

\begin{eqnarray*}
  P \approx Q
\end{eqnarray*}

$\approx_{K} = \approx = \approx_{L}$

\subsubsection{Contextual duality}

Note that contexts extend the quotation operation to a family of
operations from processes to names. Given a context, $M$, we can
define a \emph{nominal context}, $\quotep{M}$ by $\quotep{M}[P] :=
\quotep{M[P]}$. To foreshadow what is to come we observe that these
operations enjoy a duality with processes very much like the duality
between vectors and maps from vectors to scalars.

Further, because the calculus is essentially higher-order, we have a
correspondence between contexts and processes. More specifically,
given a name $x$ and a context $M$ we can construct $M^{*}_{x}$ such
that 

\begin{mathpar}
  M^{*}_{x} | \lift{x}{P} \red M[P]
\end{mathpar}

namely,

\begin{mathpar}
  M^{*}_{x} := x?(u).M[\dropn{u}]
\end{mathpar}

The dependence of $M^{*}_{x}$ on a name makes it an abstraction, 

\begin{mathpar}
  M^{*} := (x)x?(u).M[\dropn{u}]
\end{mathpar}

\subsection{Additional notation}

It will sometimes be convenient to denote the process a name
quotes. We already have the notation $x = \quotep{P}$, but it will be
convenient to introduce an alternate notation, $\procn{x}$, when we
want to emphasize the connection to the use of the name. Note that, by
virtue of name equivalence, $\quotep{\procn{x}} \nameeq x$; so, the
notation is consistent with previous definitions.

Further, because names have structure it is possible to effect
substitutions on the basis of that structure. This means we need to
upgrade our notation for substitutions, which we accomplish by
adapting comprehension notation. Thus,

\begin{mathpar}
  P\{ y / x : x \in S \}
\end{mathpar}

is interpreted to mean the process derived from P by replacing (in a
capture-avoiding manner) each occurrence of $x$ in $S$ by $y$. For example,

\begin{mathpar}
  P\{ \quotep{\procn{x}|\procn{x}} / x : x \in \freenames{P} \}
\end{mathpar}

will replace each (occurrence) of a free name $x$ in $P$ by
$\quotep{\procn{x}|\procn{x}}$.

Also, we will avail ourselves of the notation $x^{L}$ and $x^{R}$ to
denote injections of a name into disjoint copies of the name
space. There are numerous ways to accomplish this. One example can be
found in \cite{MeredithR05}. This notation overloads to vectors of
names: $\vec{x}^{\pi} := (x_{i}^{\pi} \; : \; 0 \leq i < |\vec{x}| )$ where $\pi \in \{L,R\}$.

We also use $P^{\Box} := P|\Box$.

In \cite{MeredithR05} an interpretation of the new operator is
given. It turns out that there are several possible interpretations
all enjoying the requisite algebraic properties of the operator (see
\cite{milner91polyadicpi}). We will therefore make liberal use of
$(\nu\; \vec{x})P$.

% subsection the_syntax_and_semantics_of_the_notation_system (end)   

\input{qm2pi.qmops} 

\input{qm2pi.sterngerlach} 

\input{qm2pi.metric} 

% section concurrent_process_calculi (end)

%\input{qm2pi.proofsketch}

% section proof sketch (end)

%\input{qm2pi.slviaknots} 

% section spatial logic via knots (end)

\input{qm2pi.conclusion}

% section conclusion (end)

%\input{qm2pi.dtcodes} 

% section wiring algorithm (end)

\input{qm2pi.ack} 

% section acknowledgments (end)

\newpage


\bibliographystyle{plain}   
\bibliography{../../biblios/main.bib}

\input{qm2pi.rhodetails}

\end{document}

 

\documentclass[12pt]{llncs}
%\documentclass{jktr}

\usepackage[pdftex]{hyperref}                   
\usepackage {listings}
\usepackage {mathpartir}
\usepackage{bcprules}
%\usepackage{listings}
                       
\usepackage{graphicx} 
%\usepackage[margins=2.5cm,nohead,nofoot]{geometry}
%\usepackage{geometry}
\usepackage{amsfonts}
\usepackage{amstext}
\usepackage{latexsym}
\usepackage{amssymb}
\usepackage{color}


%\include{myPreamble}
\include{qm2pi.local} 

%\ifpdf
%\usepackage[pdftex]{graphicx}
%\else
%\usepackage{graphicx}
%\fi

 % \ifpdf
%  \usepackage{pdfsync}
%  \if


%\title{Brief Article}
%\author{David F. Snyder}
%\author{L.G. Meredith}

%\address{Dept. of Math., Texas State University--San Marcos, San Marcos, TX 78666}
       
\pagestyle{empty}


\begin{document}

\lstset{language=[Objective]Caml,frame=shadowbox}

\input{qm2pi.front}

% section front matter (end)

\input{qm2pi.intro} 
 
% section introduction (end)

% \input{qm2pi.knotations} 

% section notation (end)

\input{qm2pi.process.calculi} 

% section concurrent_process_calculi_and_spatial_logics_ (end)
    
%\input{qm2pi.knots2pi} 

%\input{qm2pi.trefoil} 

%\input{qm2pi.mainthm} 

% subsection basic_interpretation (end)

%\input{qm2pi.rho.presentation} 
\subsection{The syntax and semantics of the notation system}\label{sub:the_syntax_and_semantics_of_the_notation_system} % (fold)

We now summarize a technical presentation of the calculus that
embodies our theory of dynamics. The typical presentation of such a
calculus follows the style of giving generators and relations on
them. The grammar, below, describing term constructors, freely
generates the set of processes, $\Proc$. This set is then quotiented
by a relation known as structural congruence and it is over this set
that the notion of dynamics is expressed. This presentation is
essentially that of \cite{MeredithR05} with the addition of
polyadicity and summation. For readability we have relegated some of
the technical subtleties to an appendix.

\subsubsection{Process grammar}\label{subsub:process_grammar}

\begin{mathpar}
  \inferrule* [lab=synchronization] {} {{M} \bc \pzero \;|\; x?F \;|\; x!C }
  \and
  \inferrule* [lab=abstraction] {} {{F} \bc (x)P}
  \and
  \inferrule* [lab=concretion] {} {{C} \bc \langle Q \rangle}
  \and
  \inferrule* [lab=process] {} {{P,Q} \bc M \;| \;P|Q \;|\; @{x}}
  \and
  \inferrule* [lab=name] {} {{x} \bc \quotep{P}}
\end{mathpar} 

Note that $\vec{x}$ (resp. $\vec{P}$) denotes a vector of names
(resp. processes) of length $|\vec{x}|$ (resp. $|\vec{P}|$). We adopt
the following useful abbreviations.

\begin{mathpar}
   x?(\vec{y}).P := x.(\vec{y})P \and  x\clift{\vec{P}} := x.\clift{\vec{P}}
   \and x!(y) := \lift{x}{\dropn{y}}
   \and \Pi_{i=0}^{n-1}P_i := P_0 | \ldots | P_{n-1}
\end{mathpar}

\subsubsection{Structural congruence}

\paragraph{Free and bound names and alpha-equivalence.} At the
core of structural equivalence is alpha-equivalence which identifies
process that are the same up to a change of variable. Formally, we
recognize the distinction between free and bound names. The free names
of a process, $\freenames{P}$, may be calculated recursively as
follows:

\begin{mathpar}
\freenames{\pzero} := \emptyset
  \and \\
  \freenames{x?(y).P} := \{ x \} \cup (\freenames{P} \setminus \{ y \})
  \and 
  \freenames{x!\langle P \rangle} := \{ x \} \cup \{ P \} 
  \and \\
  \freenames{P|Q} := \freenames{P} \cup \freenames{Q}
  \and \\
  \freenames{@{x}} := \{ x \}
\end{mathpar}

$\pi$
$\quotep{\pi}$

$\freenames{-} : \pi \to \mathcal{P}(\quotep{\pi})$

\begin{eqnarray*}
  \freenames{\pzero} & := & \emptyset \\
  \freenames{x?(y).P} & := & \{ x \} \cup (\freenames{P} \setminus \{ y \}) \\
  \freenames{x!\langle P \rangle} & := & \{ x \} \cup \{ P \} \\
  \freenames{P|Q} & := & \freenames{P} \cup \freenames{Q} \\
  \freenames{\dropn{x}} & := & \{ x \}
\end{eqnarray*}

The bound names of a process, $\boundnames{P}$, are those names occurring in $P$
that are not free. For example, in $x?(y).0$, the name $x$ is free, while $y$ is bound.

\begin{mathpar}
  \inferrule* [lab=monoidal-laws] {} { P|Q \equiv Q|P \and P|0 \equiv P \and P|(Q|R) \equiv (P|Q)|R }
\end{mathpar}

\begin{mathpar}
  \inferrule* [lab=alpha-equivalence] {} { (x)P \equiv (y)P\{y/x\} \and y \not\in \freenames{P} }
\end{mathpar}

\begin{definition}
Then two processes, $P,Q$, are alpha-equivalent if $P = Q\{\vec{y}/\vec{x}\}$ for
some $\vec{x} \in \boundnames{Q},\vec{y} \in \boundnames{P}$, where $Q\{\vec{y}/\vec{x}\}$
denotes the capture-avoiding substitution of $\vec{y}$ for $\vec{x}$ in $Q$.
\end{definition}

\begin{definition}
  The {\em structural congruence} \cite{SangiorgiWalker} , $\equiv$,
  between processes is the least congruence containing
  alpha-equivalence, satisfying the abelian monoid laws
  (associativity, commutativity and $\pzero$ as identity) for parallel
  composition $|$ and for summation $+$.
\end{definition}

\subsection{Name equivalence}

We take name equivalence, written $\nameeq$, to be the smallest
equivalence relation generated by the following rules.

\begin{mathpar}
\inferrule*[lab=Quote-drop]
{ }
{ \quotep{@{x}} \nameeq x }

\inferrule*[lab=Struct-equiv]
{ P \scong Q }
{ \quotep{P} \nameeq \quotep{Q} }
\end{mathpar}

The astute reader will have noticed that the mutual recursion of names
and processes imposes a mutual recursion on alpha-equivalence and
structural equivalence via name-equivalence. Fortunately, all of this
works out pleasantly and we may calculate in the natural way, free of
concern. The reader interested in the details is referred to the
appendix \ref{appendix:rho_details}.

\subsection{Substitution}

We use $\Proc$ for the set of processes, $\QProc$ for the set of
names, and $\id{\{}\vec{y} / \vec{x} \id{\}}$ to denote partial maps,
$s : \QProc \rightarrow \QProc$. A map, $s$ lifts, uniquely, to a map
on process terms, $\widehat{s} : \Proc \rightarrow \Proc$ by the
following equations.

\begin{mathpar}
  (0) \psubstp{Q}{P} := 0 \\
  (R \juxtap S) \psubstp{Q}{P}
  :=    
  (R)\psubstp{Q}{P} \juxtap (S) \psubstp{Q}{P} \\
  (x?(y).R) \psubstp{Q}{P}    
  :=    
  (x)\substp{Q}{P} (z)\concat( (R \psubstn{z}{y}) \psubstp{Q}{P} ) \\
  (\lift{x}{R}) \psubstp{Q}{P}  
  :=
  \lift{(x)\substp{Q}{P}}{ R \psubstp{Q}{P} } \\
%   (\dropn{x})  \psubstp{Q}{P}       
%   := 
%   \left\{ 
%     \begin{array}{ccc} 
%       \dropn{\quotep{Q}} & & x \nameeq \quotep{P} \\
%       \dropn{x} & & otherwise \\
%     \end{array}
%   \right. 
  (\dropn{x})  \psubstp{Q}{P}       
  := 
  \left\{ 
    \begin{array}{ccc} 
      Q & & x \nameeq \quotep{P} \\
      \dropn{x} & & otherwise \\
    \end{array}
  \right.
\end{mathpar}
 

where

\begin{eqnarray}
  (x)\id{\{} \lpquote Q \rpquote / \lpquote P \rpquote \id{\}}            = 
  \left\{ 
    \begin{array}{ccc}
      \lpquote Q \rpquote & & x \nameeq \lpquote P \rpquote \\
      x & & otherwise \\
    \end{array}
  \right. \nonumber
\end{eqnarray}

and $z$ is chosen distinct from $\quotep{P}$, $\quotep{Q}$, the free
names in $Q$, and all the names in $R$. Our $\alpha$-equivalence will
be built in the standard way from this substitution.

\begin{remark}\label{rem:no_self_referential_names}
  One consequence of these definitions is that $\forall P. \quotep{P}
  \not\in \freenames{P}$.
\end{remark}

\subsection{ Dynamic quote: an example }

Anticipating something of what's to come, consider applying the
substitution, $\widehat{\id{\{}u / z \id{\}}}$, to the following pair
of processes, $\lift{w}{y!(z)}$ and $w[ \lpquote y!(z) \rpquote ]$.

\begin{eqnarray}
	\lift{w}{y!(z)}\widehat{\id{\{}u / z \id{\}}}
		& = &
		\lift{w}{y!(u)} \nonumber\\
	w[ \lpquote y!(z) \rpquote ] \widehat{ \id{\{}u / z \id{\}} }
		& = &
		w[ \lpquote y!(z) \rpquote ] \nonumber
\end{eqnarray}

Because the body of the process between quotes is impervious to
substitution, we get radically different answers. In fact, by
examining the first process in an input context,
e.g. $x?(z).\lift{w}{y!(z)}$, we see that the process under the lift
operator may be shaped by prefixed inputs binding a name inside it. In
this sense, the lift operator will be seen as a way to dynamically
construct processes before reifying them as names.

Finally equipped with these standard features we can present the
dynamics of the calculus.

\subsubsection{Operational semantics} 

Finally, we introduce the computational dynamics. What marks these
algebras as distinct from other more traditionally studied algebraic
structures, e.g. vector spaces or polynomial rings, is the manner in
which dynamics is captured. In traditional structures, dynamics is typically
expressed through morphisms between such structures, as in linear maps
between vector spaces or morphisms between rings. In algebras
associated with the semantics of computation, the dynamics is
expressed as part of the algebraic structure itself, through a
reduction reduction relation typically denoted by $\red$. Below, we
give a recursive presentation of this relation for the calculus used
in the encoding.

$\red \subseteq \pi \times \pi$
$\red : \pi \to \mathcal{P}(\pi)$

\begin{mathpar}
  \inferrule* [lab=Comm] { \textsf{match}( x_{src}, x_{trgt} ) } { x_{trgt}?(y)P \; | \; x_{src}!\langle {Q} \rangle \red P\{\quotep{Q}/y}\} }
  \and \\
  \inferrule* [lab=Par] {{P} \red {P}'} {{{P} | {Q}} \red {{P}' | {Q}}}
  \and
  \inferrule* [lab=Equiv]{{{P} \scong {P}'} \andalso {{P}' \red {Q}'} \andalso {{Q}' \scong {Q}}}{{P} \red {Q}}
\end{mathpar}

\begin{eqnarray*}
  match_{\equiv} (\quotep{P},\quotep{Q}) & := & P \equiv Q \\
  match_{\dagger}(\quotep{P},\quotep{Q}) & := & \forall R. P|Q \red^{*} R => R \red^{*} 0 \\
  match_{K}(\quotep{P},\quotep{Q}) & := & K \mbox{ for some context } K
\end{eqnarray*}

$u?(x)P | u!\langle Q \rangle \red P\{\quotep{Q}/x\}$

%We write $\wred$ for $\red^*$, and $P\red$ if $\exists Q $ such that $ P \red Q$.
We write $P\red$ if $\exists Q $ such that $ P \red Q$ and $P\not\red$, otherwise.

\section{Replication}

As mentioned before, it is known that replication (and hence
recursion) can be implemented in a higher-order process algebra
\cite{SangiorgiWalker}. As our first example of calculation with the
machinery thus far presented we give the construction explicitly in
the {\rhoc}.

\begin{eqnarray}
	D_{x} & := & \prefix{x}{y}{(\binpar{\outputp{x}{y}}{@{y}})} \nonumber\\
	\bangp_{x}{P} & := & \binpar{{x}!\langle{\binpar{D_{x}}{P}}\rangle}{D_{x}} \nonumber
\end{eqnarray}

\begin{eqnarray}
	\bangp_{x}{P} & & \nonumber\\
	=
	& {x}!\langle{(\prefix{x}{y}{(\outputp{x}{y} | @{y})) | P}}\rangle 
	      | \prefix{x}{y}{(\outputp{x}{y} | @{y})} & \nonumber\\
	\red
	& (\outputp{x}{y} | @{y})\substn{\quotep{(\prefix{x}{y}{(@{y} | \outputp{x}{y})) | P}}}{y} & \nonumber\\
	=
	& \outputp{x}{\quotep{(\prefix{x}{y}{(\outputp{x}{y} | @{y})) | P}}}
	  | {(\prefix{x}{y}{(\outputp{x}{y} | @{y})) | P}} & \nonumber\\
	\red
	& \ldots & \nonumber\\
	\red^*
	& P | P | \ldots & \nonumber
\end{eqnarray}

Of course, this encoding, as an implementation, runs away, unfolding
$\bangp{P}$ eagerly. A lazier and more implementable replication
operator, restricted to input-guarded processes, may be obtained as follows.

\begin{eqnarray}
\bangp{\prefix{u}{v}{P}} 
	:= 
	\binpar{\lift{x}{\prefix{u}{v}{(\binpar{D(x)}{P})}}}{D(x)} \nonumber
\end{eqnarray}

\begin{remark}
  Note that the lazier definition still does not deal with summation
  or mixed summation (i.e. sums over input and output). The reader is
  invited to construct definitions of replication that deal with these
  features. 

  Further, the definitions are parameterized in a name, $x$. Can you,
  gentle reader, make a definition that eliminates this parameter and
  guarantees no accidental interaction between the replication
  machinery and the process being replicated -- i.e. no accidental
  sharing of names used by the process to get its work done and the
  name(s) used by the replication to effect copying. This latter
  revision of the definition of replication is crucial to obtaining
  the expected identity $!!P \sim !P$.
\end{remark}

\begin{remark}\label{rem:paradoxical_combinator}
  The reader familiar with the lambda calculus will have noticed the
  similarity between $D$ and the paradoxical combinator.

  [Ed. note: the existence of this seems to suggest we have to be more
  restrictive on the set of processes and names we admit if we are to
  support no-cloning.]
\end{remark}

\subsubsection{Bisimulation}

The computational dynamics gives rise to another kind of equivalence,
the equivalence of computational behavior. As previously mentioned
this is typically captured \emph{via} some form of bisimulation.

% The notion we use in this paper is weak barbed bisimulation
% \cite{milner91polyadicpi}.

The notion we use in this paper is derived from weak barbed
bisimulation \cite{milner91polyadicpi}. 

\begin{definition}
An \emph{observation relation}, $\downarrow_{\mathcal N}$, over a set
of names, $\mathcal N$, is the smallest relation satisfying the rules
below.

\infrule[Out-barb]{y \in {\mathcal N}, \; x \nameeq y}
		  {\outputp{x}{v} \downarrow_{\mathcal N} x}
\infrule[Par-barb]{\mbox{$P\downarrow_{\mathcal N} x$ or $Q\downarrow_{\mathcal N} x$}}
		  {\binpar{P}{Q} \downarrow_{\mathcal N} x}

We write $P \Downarrow_{\mathcal N} x$ if there is $Q$ such that 
$P \wred Q$ and $Q \downarrow_{\mathcal N} x$.
\end{definition}

\begin{definition}
%\label{def.bbisim}
An  ${\mathcal N}$-\emph{barbed bisimulation} over a set of names, ${\mathcal N}$, is a symmetric binary relation 
${\mathcal S}_{\mathcal N}$ between agents such that $P\rel{S}_{\mathcal N}Q$ implies:
\begin{enumerate}
\item If $P \red P'$ then $Q \wred Q'$ and $P'\rel{S}_{\mathcal N} Q'$.
\item If $P\downarrow_{\mathcal N} x$, then $Q\Downarrow_{\mathcal N} x$.
\end{enumerate}
$P$ is ${\mathcal N}$-barbed bisimilar to $Q$, written
$P \wbbisim_{\mathcal N} Q$, if $P \rel{S}_{\mathcal N} Q$ for some ${\mathcal N}$-barbed bisimulation ${\mathcal S}_{\mathcal N}$.
\end{definition}

$\mathcal{R} \subseteq \pi \times \pi$

$P \mathcal{R} Q => \forall P'. P \red P' \Rightarrow \exists Q'. Q \red Q', P' \mathcal{R} Q'$

$P \vdash x \Rightarrow Q \vdash x$

\begin{mathpar}
  \inferrule*[lab=Out-barb]{x \nameeq y}{{y}!\langle{Q}\rangle \vdash x}
  \and
  \inferrule*[lab=Par-barb]{\mbox{$P\vdash x$ or $Q\vdash x$}}{\binpar{P}{Q} \vdash x}
\end{mathpar}

\subsubsection{Contexts}

One of the principle advantages of computational calculi like the
$\pi$-calculus is a well-defined notion of context,
contextual-equivalence and a correlation between
contextual-equivalence and notions of bisimulation. The notion of
context allows the decomposition of a process into (sub-)process and
its syntactic environment, its context. Thus, a context may be
thought of as a process with a ``hole'' (written $\Box$) in it. The
application of a context $M$ to a process $P$, written $M[P]$, is
tantamount to filling the hole in $M$ with $P$. In this paper we do
not need the full weight of this theory, but do make use of the notion
of context in the proof the main theorem. 

\begin{mathpar}
  \inferrule* [lab=summation] {} {{M_{M},M_{N}} \bc \Box \;|\; x.M_{A} \;|\; M_{M}+M_{N}}
  \and
  \inferrule* [lab=agent] {} {{M_{A}} \bc (\vec{x})M_{P} \;| \; \clift{P_0,\ldots,M_{P},\ldots,P_N}}
  \and \\
  \inferrule* [lab=process] {} {{M_{P}} \bc M_{N} \;| \;P|M_{P} }
\end{mathpar} 

\begin{mathpar}
  \inferrule* [lab=sychronization] {} {M_{N} \bc \Box \;|\; x?M_{F} \;|\; x!M_{C}}
  \and
  \inferrule* [lab=abstraction] {} {{M_{F}} \bc (x)M_{P} }
  \and
  \inferrule* [lab=concretion] {} {{M_{C}} \bc \langle M_{P} \rangle }
  \and \\
  \inferrule* [lab=process] {} {{M_{P}} \bc M_{N} \;| \;P|M_{P} }
\end{mathpar}

\begin{definition}[contextual application] Given a context $M$, and
  process $P$, we define the \emph{contextual application}, $M[P] :=
  M\{P/\Box\}$. That is, the contextual application of M to P is the
  substitution of $P$ for $\Box$ in $M$.
\end{definition}

$\meaningof{-} : L \to \mathcal{P}(\pi)$

\begin{mathpar}
  \inferrule* [lab=collection] {} {\meaningof{true} = \pi, \and \meaningof{~E} = \pi \setminus \meaningof{E}, \and \meaningof{E_{1} \& E_{2}} = \meaningof{E_{1}} \cap \meaningof{E_{2}}}
\end{mathpar}

\begin{mathpar}
  \inferrule* [lab=structure] {} {\meaningof{0} = \{ P \in \pi | P \equiv 0 \}, \and \\ \meaningof{E_1 | E_2} = \{ P \in \pi | P \equiv P_{1} | P_{2}, P_{1} \in \meaningof{E_{1}}, P_{2} \in \meaningof{E_2}\} }
\end{mathpar}

\begin{mathpar}
 \inferrule* [lab=behavior] {} {\meaningof{\langle a?b \rangle E} = \{ P \in \pi | P \equiv Q | u?(y)P', \\ \and \\\\ \and \\ \;\;\; u \in \meaningof{a}, \forall z.P'\{z/y\} \in \meaningof{E\{z/b\}}\}, \and \\ \meaningof{a!E} = \{ P \in \pi | P \equiv Q | x!\langle P' \rangle, x \in \meaningof{a} P' \in \meaningof{E}\} }
\end{mathpar}

\begin{mathpar}
 \inferrule* [lab=nominal] {} {\meaningof{\quotep{E}} = \{ \quotep{P} \in \quotep{\pi} | P \in \meaningof{E} \}, \and \meaningof{\quotep{P}} = \{ \quotep{Q} \in \quotep{\pi} | P \equiv Q \} \and \\ \meaningof{@\quotep{E}} = \{ P \in \pi | P \equiv @x, x \in \meaningof{E} \}}
\end{mathpar}

\begin{eqnarray*}
  \\
  \meaningof{-} : TS \to ST
\end{eqnarray*}

\begin{eqnarray*}
  \\
  L : TS \to ST
\end{eqnarray*}

\begin{eqnarray*}
  \\
  P \models E \iff P \in \meaningof{E}
\end{eqnarray*}

\begin{eqnarray*}
  P \approx_{L} Q \iff \forall E \in L. P \models E \iff Q \models E
\end{eqnarray*}

\begin{eqnarray*}
  P \approx_{K} Q
\end{eqnarray*}

\begin{eqnarray*}
  P \approx Q
\end{eqnarray*}

$\approx_{K} = \approx = \approx_{L}$

\subsubsection{Contextual duality}

Note that contexts extend the quotation operation to a family of
operations from processes to names. Given a context, $M$, we can
define a \emph{nominal context}, $\quotep{M}$ by $\quotep{M}[P] :=
\quotep{M[P]}$. To foreshadow what is to come we observe that these
operations enjoy a duality with processes very much like the duality
between vectors and maps from vectors to scalars.

Further, because the calculus is essentially higher-order, we have a
correspondence between contexts and processes. More specifically,
given a name $x$ and a context $M$ we can construct $M^{*}_{x}$ such
that 

\begin{mathpar}
  M^{*}_{x} | \lift{x}{P} \red M[P]
\end{mathpar}

namely,

\begin{mathpar}
  M^{*}_{x} := x?(u).M[\dropn{u}]
\end{mathpar}

The dependence of $M^{*}_{x}$ on a name makes it an abstraction, 

\begin{mathpar}
  M^{*} := (x)x?(u).M[\dropn{u}]
\end{mathpar}

\subsection{Additional notation}

It will sometimes be convenient to denote the process a name
quotes. We already have the notation $x = \quotep{P}$, but it will be
convenient to introduce an alternate notation, $\procn{x}$, when we
want to emphasize the connection to the use of the name. Note that, by
virtue of name equivalence, $\quotep{\procn{x}} \nameeq x$; so, the
notation is consistent with previous definitions.

Further, because names have structure it is possible to effect
substitutions on the basis of that structure. This means we need to
upgrade our notation for substitutions, which we accomplish by
adapting comprehension notation. Thus,

\begin{mathpar}
  P\{ y / x : x \in S \}
\end{mathpar}

is interpreted to mean the process derived from P by replacing (in a
capture-avoiding manner) each occurrence of $x$ in $S$ by $y$. For example,

\begin{mathpar}
  P\{ \quotep{\procn{x}|\procn{x}} / x : x \in \freenames{P} \}
\end{mathpar}

will replace each (occurrence) of a free name $x$ in $P$ by
$\quotep{\procn{x}|\procn{x}}$.

Also, we will avail ourselves of the notation $x^{L}$ and $x^{R}$ to
denote injections of a name into disjoint copies of the name
space. There are numerous ways to accomplish this. One example can be
found in \cite{MeredithR05}. This notation overloads to vectors of
names: $\vec{x}^{\pi} := (x_{i}^{\pi} \; : \; 0 \leq i < |\vec{x}| )$ where $\pi \in \{L,R\}$.

We also use $P^{\Box} := P|\Box$.

In \cite{MeredithR05} an interpretation of the new operator is
given. It turns out that there are several possible interpretations
all enjoying the requisite algebraic properties of the operator (see
\cite{milner91polyadicpi}). We will therefore make liberal use of
$(\nu\; \vec{x})P$.

% subsection the_syntax_and_semantics_of_the_notation_system (end)   

\input{qm2pi.qmops} 

\input{qm2pi.sterngerlach} 

\input{qm2pi.metric} 

% section concurrent_process_calculi (end)

%\input{qm2pi.proofsketch}

% section proof sketch (end)

%\input{qm2pi.slviaknots} 

% section spatial logic via knots (end)

\input{qm2pi.conclusion}

% section conclusion (end)

%\input{qm2pi.dtcodes} 

% section wiring algorithm (end)

\input{qm2pi.ack} 

% section acknowledgments (end)

\newpage


\bibliographystyle{plain}   
\bibliography{../../biblios/main.bib}

\input{qm2pi.rhodetails}

\end{document}

 

% section concurrent_process_calculi (end)

%\documentclass[12pt]{llncs}
%\documentclass{jktr}

\usepackage[pdftex]{hyperref}                   
\usepackage {listings}
\usepackage {mathpartir}
\usepackage{bcprules}
%\usepackage{listings}
                       
\usepackage{graphicx} 
%\usepackage[margins=2.5cm,nohead,nofoot]{geometry}
%\usepackage{geometry}
\usepackage{amsfonts}
\usepackage{amstext}
\usepackage{latexsym}
\usepackage{amssymb}
\usepackage{color}


%\include{myPreamble}
\include{qm2pi.local} 

%\ifpdf
%\usepackage[pdftex]{graphicx}
%\else
%\usepackage{graphicx}
%\fi

 % \ifpdf
%  \usepackage{pdfsync}
%  \if


%\title{Brief Article}
%\author{David F. Snyder}
%\author{L.G. Meredith}

%\address{Dept. of Math., Texas State University--San Marcos, San Marcos, TX 78666}
       
\pagestyle{empty}


\begin{document}

\lstset{language=[Objective]Caml,frame=shadowbox}

\input{qm2pi.front}

% section front matter (end)

\input{qm2pi.intro} 
 
% section introduction (end)

% \input{qm2pi.knotations} 

% section notation (end)

\input{qm2pi.process.calculi} 

% section concurrent_process_calculi_and_spatial_logics_ (end)
    
%\input{qm2pi.knots2pi} 

%\input{qm2pi.trefoil} 

%\input{qm2pi.mainthm} 

% subsection basic_interpretation (end)

%\input{qm2pi.rho.presentation} 
\subsection{The syntax and semantics of the notation system}\label{sub:the_syntax_and_semantics_of_the_notation_system} % (fold)

We now summarize a technical presentation of the calculus that
embodies our theory of dynamics. The typical presentation of such a
calculus follows the style of giving generators and relations on
them. The grammar, below, describing term constructors, freely
generates the set of processes, $\Proc$. This set is then quotiented
by a relation known as structural congruence and it is over this set
that the notion of dynamics is expressed. This presentation is
essentially that of \cite{MeredithR05} with the addition of
polyadicity and summation. For readability we have relegated some of
the technical subtleties to an appendix.

\subsubsection{Process grammar}\label{subsub:process_grammar}

\begin{mathpar}
  \inferrule* [lab=synchronization] {} {{M} \bc \pzero \;|\; x?F \;|\; x!C }
  \and
  \inferrule* [lab=abstraction] {} {{F} \bc (x)P}
  \and
  \inferrule* [lab=concretion] {} {{C} \bc \langle Q \rangle}
  \and
  \inferrule* [lab=process] {} {{P,Q} \bc M \;| \;P|Q \;|\; @{x}}
  \and
  \inferrule* [lab=name] {} {{x} \bc \quotep{P}}
\end{mathpar} 

Note that $\vec{x}$ (resp. $\vec{P}$) denotes a vector of names
(resp. processes) of length $|\vec{x}|$ (resp. $|\vec{P}|$). We adopt
the following useful abbreviations.

\begin{mathpar}
   x?(\vec{y}).P := x.(\vec{y})P \and  x\clift{\vec{P}} := x.\clift{\vec{P}}
   \and x!(y) := \lift{x}{\dropn{y}}
   \and \Pi_{i=0}^{n-1}P_i := P_0 | \ldots | P_{n-1}
\end{mathpar}

\subsubsection{Structural congruence}

\paragraph{Free and bound names and alpha-equivalence.} At the
core of structural equivalence is alpha-equivalence which identifies
process that are the same up to a change of variable. Formally, we
recognize the distinction between free and bound names. The free names
of a process, $\freenames{P}$, may be calculated recursively as
follows:

\begin{mathpar}
\freenames{\pzero} := \emptyset
  \and \\
  \freenames{x?(y).P} := \{ x \} \cup (\freenames{P} \setminus \{ y \})
  \and 
  \freenames{x!\langle P \rangle} := \{ x \} \cup \{ P \} 
  \and \\
  \freenames{P|Q} := \freenames{P} \cup \freenames{Q}
  \and \\
  \freenames{@{x}} := \{ x \}
\end{mathpar}

$\pi$
$\quotep{\pi}$

$\freenames{-} : \pi \to \mathcal{P}(\quotep{\pi})$

\begin{eqnarray*}
  \freenames{\pzero} & := & \emptyset \\
  \freenames{x?(y).P} & := & \{ x \} \cup (\freenames{P} \setminus \{ y \}) \\
  \freenames{x!\langle P \rangle} & := & \{ x \} \cup \{ P \} \\
  \freenames{P|Q} & := & \freenames{P} \cup \freenames{Q} \\
  \freenames{\dropn{x}} & := & \{ x \}
\end{eqnarray*}

The bound names of a process, $\boundnames{P}$, are those names occurring in $P$
that are not free. For example, in $x?(y).0$, the name $x$ is free, while $y$ is bound.

\begin{mathpar}
  \inferrule* [lab=monoidal-laws] {} { P|Q \equiv Q|P \and P|0 \equiv P \and P|(Q|R) \equiv (P|Q)|R }
\end{mathpar}

\begin{mathpar}
  \inferrule* [lab=alpha-equivalence] {} { (x)P \equiv (y)P\{y/x\} \and y \not\in \freenames{P} }
\end{mathpar}

\begin{definition}
Then two processes, $P,Q$, are alpha-equivalent if $P = Q\{\vec{y}/\vec{x}\}$ for
some $\vec{x} \in \boundnames{Q},\vec{y} \in \boundnames{P}$, where $Q\{\vec{y}/\vec{x}\}$
denotes the capture-avoiding substitution of $\vec{y}$ for $\vec{x}$ in $Q$.
\end{definition}

\begin{definition}
  The {\em structural congruence} \cite{SangiorgiWalker} , $\equiv$,
  between processes is the least congruence containing
  alpha-equivalence, satisfying the abelian monoid laws
  (associativity, commutativity and $\pzero$ as identity) for parallel
  composition $|$ and for summation $+$.
\end{definition}

\subsection{Name equivalence}

We take name equivalence, written $\nameeq$, to be the smallest
equivalence relation generated by the following rules.

\begin{mathpar}
\inferrule*[lab=Quote-drop]
{ }
{ \quotep{@{x}} \nameeq x }

\inferrule*[lab=Struct-equiv]
{ P \scong Q }
{ \quotep{P} \nameeq \quotep{Q} }
\end{mathpar}

The astute reader will have noticed that the mutual recursion of names
and processes imposes a mutual recursion on alpha-equivalence and
structural equivalence via name-equivalence. Fortunately, all of this
works out pleasantly and we may calculate in the natural way, free of
concern. The reader interested in the details is referred to the
appendix \ref{appendix:rho_details}.

\subsection{Substitution}

We use $\Proc$ for the set of processes, $\QProc$ for the set of
names, and $\id{\{}\vec{y} / \vec{x} \id{\}}$ to denote partial maps,
$s : \QProc \rightarrow \QProc$. A map, $s$ lifts, uniquely, to a map
on process terms, $\widehat{s} : \Proc \rightarrow \Proc$ by the
following equations.

\begin{mathpar}
  (0) \psubstp{Q}{P} := 0 \\
  (R \juxtap S) \psubstp{Q}{P}
  :=    
  (R)\psubstp{Q}{P} \juxtap (S) \psubstp{Q}{P} \\
  (x?(y).R) \psubstp{Q}{P}    
  :=    
  (x)\substp{Q}{P} (z)\concat( (R \psubstn{z}{y}) \psubstp{Q}{P} ) \\
  (\lift{x}{R}) \psubstp{Q}{P}  
  :=
  \lift{(x)\substp{Q}{P}}{ R \psubstp{Q}{P} } \\
%   (\dropn{x})  \psubstp{Q}{P}       
%   := 
%   \left\{ 
%     \begin{array}{ccc} 
%       \dropn{\quotep{Q}} & & x \nameeq \quotep{P} \\
%       \dropn{x} & & otherwise \\
%     \end{array}
%   \right. 
  (\dropn{x})  \psubstp{Q}{P}       
  := 
  \left\{ 
    \begin{array}{ccc} 
      Q & & x \nameeq \quotep{P} \\
      \dropn{x} & & otherwise \\
    \end{array}
  \right.
\end{mathpar}
 

where

\begin{eqnarray}
  (x)\id{\{} \lpquote Q \rpquote / \lpquote P \rpquote \id{\}}            = 
  \left\{ 
    \begin{array}{ccc}
      \lpquote Q \rpquote & & x \nameeq \lpquote P \rpquote \\
      x & & otherwise \\
    \end{array}
  \right. \nonumber
\end{eqnarray}

and $z$ is chosen distinct from $\quotep{P}$, $\quotep{Q}$, the free
names in $Q$, and all the names in $R$. Our $\alpha$-equivalence will
be built in the standard way from this substitution.

\begin{remark}\label{rem:no_self_referential_names}
  One consequence of these definitions is that $\forall P. \quotep{P}
  \not\in \freenames{P}$.
\end{remark}

\subsection{ Dynamic quote: an example }

Anticipating something of what's to come, consider applying the
substitution, $\widehat{\id{\{}u / z \id{\}}}$, to the following pair
of processes, $\lift{w}{y!(z)}$ and $w[ \lpquote y!(z) \rpquote ]$.

\begin{eqnarray}
	\lift{w}{y!(z)}\widehat{\id{\{}u / z \id{\}}}
		& = &
		\lift{w}{y!(u)} \nonumber\\
	w[ \lpquote y!(z) \rpquote ] \widehat{ \id{\{}u / z \id{\}} }
		& = &
		w[ \lpquote y!(z) \rpquote ] \nonumber
\end{eqnarray}

Because the body of the process between quotes is impervious to
substitution, we get radically different answers. In fact, by
examining the first process in an input context,
e.g. $x?(z).\lift{w}{y!(z)}$, we see that the process under the lift
operator may be shaped by prefixed inputs binding a name inside it. In
this sense, the lift operator will be seen as a way to dynamically
construct processes before reifying them as names.

Finally equipped with these standard features we can present the
dynamics of the calculus.

\subsubsection{Operational semantics} 

Finally, we introduce the computational dynamics. What marks these
algebras as distinct from other more traditionally studied algebraic
structures, e.g. vector spaces or polynomial rings, is the manner in
which dynamics is captured. In traditional structures, dynamics is typically
expressed through morphisms between such structures, as in linear maps
between vector spaces or morphisms between rings. In algebras
associated with the semantics of computation, the dynamics is
expressed as part of the algebraic structure itself, through a
reduction reduction relation typically denoted by $\red$. Below, we
give a recursive presentation of this relation for the calculus used
in the encoding.

$\red \subseteq \pi \times \pi$
$\red : \pi \to \mathcal{P}(\pi)$

\begin{mathpar}
  \inferrule* [lab=Comm] { \textsf{match}( x_{src}, x_{trgt} ) } { x_{trgt}?(y)P \; | \; x_{src}!\langle {Q} \rangle \red P\{\quotep{Q}/y}\} }
  \and \\
  \inferrule* [lab=Par] {{P} \red {P}'} {{{P} | {Q}} \red {{P}' | {Q}}}
  \and
  \inferrule* [lab=Equiv]{{{P} \scong {P}'} \andalso {{P}' \red {Q}'} \andalso {{Q}' \scong {Q}}}{{P} \red {Q}}
\end{mathpar}

\begin{eqnarray*}
  match_{\equiv} (\quotep{P},\quotep{Q}) & := & P \equiv Q \\
  match_{\dagger}(\quotep{P},\quotep{Q}) & := & \forall R. P|Q \red^{*} R => R \red^{*} 0 \\
  match_{K}(\quotep{P},\quotep{Q}) & := & K \mbox{ for some context } K
\end{eqnarray*}

$u?(x)P | u!\langle Q \rangle \red P\{\quotep{Q}/x\}$

%We write $\wred$ for $\red^*$, and $P\red$ if $\exists Q $ such that $ P \red Q$.
We write $P\red$ if $\exists Q $ such that $ P \red Q$ and $P\not\red$, otherwise.

\section{Replication}

As mentioned before, it is known that replication (and hence
recursion) can be implemented in a higher-order process algebra
\cite{SangiorgiWalker}. As our first example of calculation with the
machinery thus far presented we give the construction explicitly in
the {\rhoc}.

\begin{eqnarray}
	D_{x} & := & \prefix{x}{y}{(\binpar{\outputp{x}{y}}{@{y}})} \nonumber\\
	\bangp_{x}{P} & := & \binpar{{x}!\langle{\binpar{D_{x}}{P}}\rangle}{D_{x}} \nonumber
\end{eqnarray}

\begin{eqnarray}
	\bangp_{x}{P} & & \nonumber\\
	=
	& {x}!\langle{(\prefix{x}{y}{(\outputp{x}{y} | @{y})) | P}}\rangle 
	      | \prefix{x}{y}{(\outputp{x}{y} | @{y})} & \nonumber\\
	\red
	& (\outputp{x}{y} | @{y})\substn{\quotep{(\prefix{x}{y}{(@{y} | \outputp{x}{y})) | P}}}{y} & \nonumber\\
	=
	& \outputp{x}{\quotep{(\prefix{x}{y}{(\outputp{x}{y} | @{y})) | P}}}
	  | {(\prefix{x}{y}{(\outputp{x}{y} | @{y})) | P}} & \nonumber\\
	\red
	& \ldots & \nonumber\\
	\red^*
	& P | P | \ldots & \nonumber
\end{eqnarray}

Of course, this encoding, as an implementation, runs away, unfolding
$\bangp{P}$ eagerly. A lazier and more implementable replication
operator, restricted to input-guarded processes, may be obtained as follows.

\begin{eqnarray}
\bangp{\prefix{u}{v}{P}} 
	:= 
	\binpar{\lift{x}{\prefix{u}{v}{(\binpar{D(x)}{P})}}}{D(x)} \nonumber
\end{eqnarray}

\begin{remark}
  Note that the lazier definition still does not deal with summation
  or mixed summation (i.e. sums over input and output). The reader is
  invited to construct definitions of replication that deal with these
  features. 

  Further, the definitions are parameterized in a name, $x$. Can you,
  gentle reader, make a definition that eliminates this parameter and
  guarantees no accidental interaction between the replication
  machinery and the process being replicated -- i.e. no accidental
  sharing of names used by the process to get its work done and the
  name(s) used by the replication to effect copying. This latter
  revision of the definition of replication is crucial to obtaining
  the expected identity $!!P \sim !P$.
\end{remark}

\begin{remark}\label{rem:paradoxical_combinator}
  The reader familiar with the lambda calculus will have noticed the
  similarity between $D$ and the paradoxical combinator.

  [Ed. note: the existence of this seems to suggest we have to be more
  restrictive on the set of processes and names we admit if we are to
  support no-cloning.]
\end{remark}

\subsubsection{Bisimulation}

The computational dynamics gives rise to another kind of equivalence,
the equivalence of computational behavior. As previously mentioned
this is typically captured \emph{via} some form of bisimulation.

% The notion we use in this paper is weak barbed bisimulation
% \cite{milner91polyadicpi}.

The notion we use in this paper is derived from weak barbed
bisimulation \cite{milner91polyadicpi}. 

\begin{definition}
An \emph{observation relation}, $\downarrow_{\mathcal N}$, over a set
of names, $\mathcal N$, is the smallest relation satisfying the rules
below.

\infrule[Out-barb]{y \in {\mathcal N}, \; x \nameeq y}
		  {\outputp{x}{v} \downarrow_{\mathcal N} x}
\infrule[Par-barb]{\mbox{$P\downarrow_{\mathcal N} x$ or $Q\downarrow_{\mathcal N} x$}}
		  {\binpar{P}{Q} \downarrow_{\mathcal N} x}

We write $P \Downarrow_{\mathcal N} x$ if there is $Q$ such that 
$P \wred Q$ and $Q \downarrow_{\mathcal N} x$.
\end{definition}

\begin{definition}
%\label{def.bbisim}
An  ${\mathcal N}$-\emph{barbed bisimulation} over a set of names, ${\mathcal N}$, is a symmetric binary relation 
${\mathcal S}_{\mathcal N}$ between agents such that $P\rel{S}_{\mathcal N}Q$ implies:
\begin{enumerate}
\item If $P \red P'$ then $Q \wred Q'$ and $P'\rel{S}_{\mathcal N} Q'$.
\item If $P\downarrow_{\mathcal N} x$, then $Q\Downarrow_{\mathcal N} x$.
\end{enumerate}
$P$ is ${\mathcal N}$-barbed bisimilar to $Q$, written
$P \wbbisim_{\mathcal N} Q$, if $P \rel{S}_{\mathcal N} Q$ for some ${\mathcal N}$-barbed bisimulation ${\mathcal S}_{\mathcal N}$.
\end{definition}

$\mathcal{R} \subseteq \pi \times \pi$

$P \mathcal{R} Q => \forall P'. P \red P' \Rightarrow \exists Q'. Q \red Q', P' \mathcal{R} Q'$

$P \vdash x \Rightarrow Q \vdash x$

\begin{mathpar}
  \inferrule*[lab=Out-barb]{x \nameeq y}{{y}!\langle{Q}\rangle \vdash x}
  \and
  \inferrule*[lab=Par-barb]{\mbox{$P\vdash x$ or $Q\vdash x$}}{\binpar{P}{Q} \vdash x}
\end{mathpar}

\subsubsection{Contexts}

One of the principle advantages of computational calculi like the
$\pi$-calculus is a well-defined notion of context,
contextual-equivalence and a correlation between
contextual-equivalence and notions of bisimulation. The notion of
context allows the decomposition of a process into (sub-)process and
its syntactic environment, its context. Thus, a context may be
thought of as a process with a ``hole'' (written $\Box$) in it. The
application of a context $M$ to a process $P$, written $M[P]$, is
tantamount to filling the hole in $M$ with $P$. In this paper we do
not need the full weight of this theory, but do make use of the notion
of context in the proof the main theorem. 

\begin{mathpar}
  \inferrule* [lab=summation] {} {{M_{M},M_{N}} \bc \Box \;|\; x.M_{A} \;|\; M_{M}+M_{N}}
  \and
  \inferrule* [lab=agent] {} {{M_{A}} \bc (\vec{x})M_{P} \;| \; \clift{P_0,\ldots,M_{P},\ldots,P_N}}
  \and \\
  \inferrule* [lab=process] {} {{M_{P}} \bc M_{N} \;| \;P|M_{P} }
\end{mathpar} 

\begin{mathpar}
  \inferrule* [lab=sychronization] {} {M_{N} \bc \Box \;|\; x?M_{F} \;|\; x!M_{C}}
  \and
  \inferrule* [lab=abstraction] {} {{M_{F}} \bc (x)M_{P} }
  \and
  \inferrule* [lab=concretion] {} {{M_{C}} \bc \langle M_{P} \rangle }
  \and \\
  \inferrule* [lab=process] {} {{M_{P}} \bc M_{N} \;| \;P|M_{P} }
\end{mathpar}

\begin{definition}[contextual application] Given a context $M$, and
  process $P$, we define the \emph{contextual application}, $M[P] :=
  M\{P/\Box\}$. That is, the contextual application of M to P is the
  substitution of $P$ for $\Box$ in $M$.
\end{definition}

$\meaningof{-} : L \to \mathcal{P}(\pi)$

\begin{mathpar}
  \inferrule* [lab=collection] {} {\meaningof{true} = \pi, \and \meaningof{~E} = \pi \setminus \meaningof{E}, \and \meaningof{E_{1} \& E_{2}} = \meaningof{E_{1}} \cap \meaningof{E_{2}}}
\end{mathpar}

\begin{mathpar}
  \inferrule* [lab=structure] {} {\meaningof{0} = \{ P \in \pi | P \equiv 0 \}, \and \\ \meaningof{E_1 | E_2} = \{ P \in \pi | P \equiv P_{1} | P_{2}, P_{1} \in \meaningof{E_{1}}, P_{2} \in \meaningof{E_2}\} }
\end{mathpar}

\begin{mathpar}
 \inferrule* [lab=behavior] {} {\meaningof{\langle a?b \rangle E} = \{ P \in \pi | P \equiv Q | u?(y)P', \\ \and \\\\ \and \\ \;\;\; u \in \meaningof{a}, \forall z.P'\{z/y\} \in \meaningof{E\{z/b\}}\}, \and \\ \meaningof{a!E} = \{ P \in \pi | P \equiv Q | x!\langle P' \rangle, x \in \meaningof{a} P' \in \meaningof{E}\} }
\end{mathpar}

\begin{mathpar}
 \inferrule* [lab=nominal] {} {\meaningof{\quotep{E}} = \{ \quotep{P} \in \quotep{\pi} | P \in \meaningof{E} \}, \and \meaningof{\quotep{P}} = \{ \quotep{Q} \in \quotep{\pi} | P \equiv Q \} \and \\ \meaningof{@\quotep{E}} = \{ P \in \pi | P \equiv @x, x \in \meaningof{E} \}}
\end{mathpar}

\begin{eqnarray*}
  \\
  \meaningof{-} : TS \to ST
\end{eqnarray*}

\begin{eqnarray*}
  \\
  L : TS \to ST
\end{eqnarray*}

\begin{eqnarray*}
  \\
  P \models E \iff P \in \meaningof{E}
\end{eqnarray*}

\begin{eqnarray*}
  P \approx_{L} Q \iff \forall E \in L. P \models E \iff Q \models E
\end{eqnarray*}

\begin{eqnarray*}
  P \approx_{K} Q
\end{eqnarray*}

\begin{eqnarray*}
  P \approx Q
\end{eqnarray*}

$\approx_{K} = \approx = \approx_{L}$

\subsubsection{Contextual duality}

Note that contexts extend the quotation operation to a family of
operations from processes to names. Given a context, $M$, we can
define a \emph{nominal context}, $\quotep{M}$ by $\quotep{M}[P] :=
\quotep{M[P]}$. To foreshadow what is to come we observe that these
operations enjoy a duality with processes very much like the duality
between vectors and maps from vectors to scalars.

Further, because the calculus is essentially higher-order, we have a
correspondence between contexts and processes. More specifically,
given a name $x$ and a context $M$ we can construct $M^{*}_{x}$ such
that 

\begin{mathpar}
  M^{*}_{x} | \lift{x}{P} \red M[P]
\end{mathpar}

namely,

\begin{mathpar}
  M^{*}_{x} := x?(u).M[\dropn{u}]
\end{mathpar}

The dependence of $M^{*}_{x}$ on a name makes it an abstraction, 

\begin{mathpar}
  M^{*} := (x)x?(u).M[\dropn{u}]
\end{mathpar}

\subsection{Additional notation}

It will sometimes be convenient to denote the process a name
quotes. We already have the notation $x = \quotep{P}$, but it will be
convenient to introduce an alternate notation, $\procn{x}$, when we
want to emphasize the connection to the use of the name. Note that, by
virtue of name equivalence, $\quotep{\procn{x}} \nameeq x$; so, the
notation is consistent with previous definitions.

Further, because names have structure it is possible to effect
substitutions on the basis of that structure. This means we need to
upgrade our notation for substitutions, which we accomplish by
adapting comprehension notation. Thus,

\begin{mathpar}
  P\{ y / x : x \in S \}
\end{mathpar}

is interpreted to mean the process derived from P by replacing (in a
capture-avoiding manner) each occurrence of $x$ in $S$ by $y$. For example,

\begin{mathpar}
  P\{ \quotep{\procn{x}|\procn{x}} / x : x \in \freenames{P} \}
\end{mathpar}

will replace each (occurrence) of a free name $x$ in $P$ by
$\quotep{\procn{x}|\procn{x}}$.

Also, we will avail ourselves of the notation $x^{L}$ and $x^{R}$ to
denote injections of a name into disjoint copies of the name
space. There are numerous ways to accomplish this. One example can be
found in \cite{MeredithR05}. This notation overloads to vectors of
names: $\vec{x}^{\pi} := (x_{i}^{\pi} \; : \; 0 \leq i < |\vec{x}| )$ where $\pi \in \{L,R\}$.

We also use $P^{\Box} := P|\Box$.

In \cite{MeredithR05} an interpretation of the new operator is
given. It turns out that there are several possible interpretations
all enjoying the requisite algebraic properties of the operator (see
\cite{milner91polyadicpi}). We will therefore make liberal use of
$(\nu\; \vec{x})P$.

% subsection the_syntax_and_semantics_of_the_notation_system (end)   

\input{qm2pi.qmops} 

\input{qm2pi.sterngerlach} 

\input{qm2pi.metric} 

% section concurrent_process_calculi (end)

%\input{qm2pi.proofsketch}

% section proof sketch (end)

%\input{qm2pi.slviaknots} 

% section spatial logic via knots (end)

\input{qm2pi.conclusion}

% section conclusion (end)

%\input{qm2pi.dtcodes} 

% section wiring algorithm (end)

\input{qm2pi.ack} 

% section acknowledgments (end)

\newpage


\bibliographystyle{plain}   
\bibliography{../../biblios/main.bib}

\input{qm2pi.rhodetails}

\end{document}



% section proof sketch (end)

%\section{Unlikely characters: spatial logic for
  knots}\label{sub:characteristic_formulae} % (fold)

Associated to the mobile process calculi are a family of logics known
as the Hennessy-Milner logics. These logics typically enjoy a
semantics interpreting formulae as sets of processes that when
factored through the encoding outlined above allows an identification
of classes of knots with logical formulae. In the context of this
encoding the sub-family known as the spatial logics \cite{CairesC03}
\cite{CairesC04} \cite{Caires04} are of particular interest providing
several important features for expressing and reasoning about
properties (i.e. classes) of knots. We hint here at how this may be done.

%\begin{description}
%\item [structural connectives] 
\subsubsection{Structural connectives} The spatial logics enjoy
structural connectives corresponding, at the logical level, to the
parallel composition ($P | Q$) and new name ($(\nu \; x)P$)
connectives for processes. As illustrated in the examples below, these
connectives are extremely expressive given the shape of our encoding.
%\item [decideable satisfaction]

\subsubsection{Decideable satisfaction}
In \cite{Caires04} the satisfaction relation is shown to be decideable
for a rich class of processes. It further turns out that the image of
the our encoding is a proper subset of that class. This result
provides the basis for an algorithm by which to search for knots
enjoying a given property.
%\item [characteristic formulae]

\subsubsection{Characteristic formulae}
In the same paper \cite{Caires04} , Caires presents a means of calculating
characteristic formulae, selecting equivalence classes of processes
up to a pre--specified depth limit on the support set of names. Composed with our
encoding, this characteristic formula can be used to select
characteristic formulae for knots.
%\end{description}

\subsubsection{Spatial logic formulae}

The grammar below (segmented for comprehension) summarizes the syntax
of spatial logic formulae. We employ illustrative examples in the
sequel to provide an intuitive understanding of their meaning
referring the reader to \cite{Caires04} for a more detailed explication
of the semantics.

\begin{mathpar}
  \inferrule* [lab=boolean] {} {{A,B} \bc T \;|\; \neg A \;|\; A \wedge B \;|\; \eta = \eta'}
  \and
  \inferrule* [lab=spatial] {} {|\; \pzero \;|\; A | B \;|\; x \text{\textregistered} A \;|\; \forall x . A \;|\;  H x . A}
  \and
  \inferrule* [lab=behavioral] {} {|\; \alpha . A}
  \and 
  \inferrule* [lab=recursion] {} {|\; X(\vec{u}) \;|\; \mu X(\vec{u}) . A}
  \and
  \inferrule* [lab=action] {} {\alpha \bc \langle x?(\vec{y}) \rangle \;|\; \langle x!(\vec{y}) \rangle \;|\; \langle \tau \rangle}
  \and 
  \inferrule* [lab=name] {} {\eta \bc x \;|\; \tau}
\end{mathpar} 

% subsection characteristic_formulae (end)   	 

\subsection{Example formulae}\label{sub:example_formulae_} % (fold)

\subsubsection{Crossing as formula.}
% 
% \begin{align*}
%   \frac{d}{dx} \sin x &= \cos x 
%   & \frac{d}{dx} e^x &= e^x \\
%   \frac{d}{dx} \cos x &= - \sin x 
%   & \frac{d}{dx} \log x &= \frac{1}{x} \\
% \end{align*} 

\begin{align*}
 \mu C(x_{0},x_{1},y_{0},y_{1},u).&(\langle x_{0}?(z) \rangle(\langle u! \rangle\langle y_{1}!z \rangle C(x_{0},x_{1},y_{0},y_{1},u)) & \\
  & \wedge \langle y_{1}?(z) \rangle (\langle u! \rangle \langle x_{0}!z \rangle C(x_{0},x_{1},y_{0},y_{1},u)) & \\
  & \wedge \langle x_{1}?(z) \rangle (\langle u? \rangle \langle y_{0}!z \rangle C(x_{0},x_{1},y_{0},y_{1},u)) & \\
  & \wedge \langle y_{0}?(z) \rangle (\langle u? \rangle \langle x_{1}!z \rangle C(x_{0},x_{1},y_{0},y_{1},u))) &
\end{align*}

The lexicographical similarity between the shape of this formulae and
the shape of definition of the process representing a crossing reveals
the intuitive meaning of this formulae. It describes the capabilities
of a process that has the right to represent a crossing. For example
it picks out processes that may perform an input on the port $x_0$ in
its initial menu of capabilities. What differentiates the formula
from the process, however, is that the crossing process is the
smallest candidate to satisfy the formula. Infinitely many other
processes -- with internal behavior hidden behind this interface, so
to speak -- also satisfy this formula. Even this simple formula,
then, can be seen to open a new view onto knots, providing a
computational interpretation of \emph{virtual} knots.

Note that this formula is derived by hand. A similar formula can be
derived by employing Caires' calculation of characteristic formula
\cite{Caires04} to the process representing a crossing. In light of
this discussion, we let
$\meaningof{C}_{\phi}(x0,x1,y0,y1,u)$ denote a formula specifying the
dynamics we wish to capture of a crossing. To guarantee we preserve
the shape of the interface and minimal semantics we demand that
$\meaningof{C}_{\phi}(x0,x1,y0,y1,u) \Rightarrow
\textbf{C}(x0,x1,y0,y1,u)$ where $\textbf{C}(x0,x1,y0,y1,u)$ denotes
the formula above.
                            
\subsubsection{Crossing number constraints.}
The moral content of the context lemma (Lemma \ref{context}) is that the notion of
``locality'' in the Reidemeister moves is effectively captured by the
parallel composition operator of the process calculus. This intuition
extends through the logic. Given a formula,
$\meaningof{C}_{\phi}(x0,x1,y0,y1,u)$, we can use the structural
connectives to specify constraints on crossing numbers, such as at
least $n$ crossings, or exactly $n$ crossings.
\begin{mathpar}
  \inferrule* [lab=at-least-n] {} { K^{\geq n}_{\phi}(\vec{xs},\vec{ys}) := \Pi_{i=0}^{n-1} Hu . \meaningof{C}_{\phi}(xs_i,ys_i,u) | T }
  \and 
  \inferrule* [lab=exactly-n] {} { K^{= n}_{\phi}(\vec{xs},\vec{ys}) := \Pi_{i=0}^{n-1} Hu . \meaningof{C}_{\phi}(xs_i,ys_i,u) | \neg (\forall x_0,y_0,x_1,y_1,u . \meaningof{C}_{\phi}(x_0,y_0,x_1,y_1,u) | T) }
\end{mathpar}

To round out this section, recall that the encoding of an $n$-crossing
knot decomposes into a parallel composition of $n$ \emph{copies} of a
crossing process together with a wiring harness. To specify different
knot classes with the same crossing number amounts to specifying
logical constraints on the wiring harness. In the interest of space,
we defer examples to a forthcoming paper. Suffice it to say that both
the conditions ``alternating knot'' and ``contains the tangle
corresponding to 5/3'' are expressible. For example, it is possible to
calculate the characteristic formula of a process corresponding to the
tangle 5/3 and conjoin it into the classifying formula via the
composition connective of the logic.

Finally, we wish to observe that it is entirely within reason to
contemplate a more domain-specific version of spatial logic tailored
to the shape of processes in the image of the encoding. Such a
domain-specific logic would have a better claim to the title formal
language of knot properties.

% subsection example_formulae_ (end)

% section knots_as_processes (end) 

% section spatial logic via knots (end)

\section{Conclusions and future work}

\paragraph{Testing physical space}
You, gentle reader, may wonder why of all the theorems to be proved
given this set up we pick the one above. In some sense it's hardly
central to quantum mechanics. We see it as central in the sense that
it firmly establishes a notion of physical space arising from a notion
of the equivalence of behavior. Relating bisimulation to a metric is a
big step forward, but one is faced with interpreting the relationship
of that metric space to something more physical. Quantum mechanical
notions of ``physical'' space are still far from intuitive, but by
relating this idea of distance as testing to calculations that predict
physical circumstances we are making a not insignificant step forward
toward an understanding of the physical space we inhabit as
essentially dynamic.

\paragraph{Effectivity and simulation}
One of the observations we have yet to make is that the entire program
spelled out here is effective. We have built various interpreters for
the reflective calculus at work in this interpretation. In principle,
then, we can simulate quantum mechanics on a computer. The place where
the simulation may lose fidelity is the infinitely branching summation
for the annihilator.

In this connection i also want to point out that the evaluation style
calculation of the inner product puts the non-determinism of the
summation right at the heart of measurement. This suggests that
Milner's original reduction-based formulation of the dynamics of his
calculi in terms of sums was not just notationally suggestive of a
notion of measure-and-continue but captured some significant part of
the physics.

\paragraph{Quantum continuations}
In light of this last observation i want to point out that the
predominant account of quantum mechanics is missing a key aspect of a
truly compositional story of the physical situation. In a real lab,
when a measurement is made the observation can be made to feed into
another device that then makes another measurement conditioned on the
results of the first. This means that after the superposition was
collapsed the entire experimental set up remained in
superposition. While QM offers a means of writing this down it doesn't
quite line up well with the well-trodden formulation of computation
and continuation that we see so succinctly expressed in Milner's
calculi. This suggests that there might be advantages to this account
of dynamics waiting to be explored.

\paragraph{Quantum logic}
In this connection, we also note that by virtue of having the
Hennessy-Milner construction, we can pull the construction through the
interpretation of QM. This gives us a natural candidate for a quantum
logic that enjoys an extremely tight connection with it's domain of
interpretation, making the construction much less ad hoc (rather it is
the image of functor!).

\paragraph{Quantum probabiity}
i have questions about the basis of the interpretation of inner
product as probability amplitude. In particular, using which
axiomatization of probability theory does the notion of probability
amplitude earn the right to be so dubbed? In other words, where is the
proof that the operation for calculating a probability amplitude (and
then squaring) satisfies the axioms of what it means to calculate a
probability? Even if such a proof exists (i have yet to find it in the
literature), i wonder if it might not be possible to turn things on
their heads. Can we view the calculation of the probability amplitude
as an axiomatization of probability? If so, then the definition we
give for calculating probability amplitude may provide the basis for
an \emph{effective} theory of probability.

\paragraph{Quantum vs ``biological'' information}
Finally, i want to conclude with a more philosophical observation. At
a recent workshop in which QM was a predominant topic i noticed
something about quantum information. The speaker was giving a riveting
discussion of axiomatic QM and showing how properties of ``no
cloning'' and ``no deleting'' emerged as consequences of the
axiomatization. Theorems of this form are necessary to give us a sense
of confidence that our axioms characterize the physical theory. What
struck me, though, was that if quantum information is neither erasable
nor replicable it is markedly different from \emph{life}. Two of the
things we know about life is that

\begin{itemize}
  \item it ends;
  \item to gain some measure of persistence, to transcend it's
    finitude it is imminently copyable.
\end{itemize}

Both of these qualities are summarized succinctly in the aphorism: all
flesh is grass. For me these two kinds of ``information'' -- call them
quantum and biological -- are end points on a spectrum of strategies
for persistence. At one end, we have those curious entities that enjoy
uniqueness and permanence; at the other, we have those who in the face
of a certain end and an uncertain present make a go of passing
something on. To me one of the more remarkable aspects of the latter
strategy is that in the presence of noise (and certain features of
copying) we get a kind of dynamism, a chance for improvement against a
given persistent condition.

% subsection other_calculi_other_bisimulations_and_geometry_as_behavior (end)




% section conclusion (end)

%\documentclass[12pt]{llncs}
%\documentclass{jktr}

\usepackage[pdftex]{hyperref}                   
\usepackage {listings}
\usepackage {mathpartir}
\usepackage{bcprules}
%\usepackage{listings}
                       
\usepackage{graphicx} 
%\usepackage[margins=2.5cm,nohead,nofoot]{geometry}
%\usepackage{geometry}
\usepackage{amsfonts}
\usepackage{amstext}
\usepackage{latexsym}
\usepackage{amssymb}
\usepackage{color}


%\include{myPreamble}
\include{qm2pi.local} 

%\ifpdf
%\usepackage[pdftex]{graphicx}
%\else
%\usepackage{graphicx}
%\fi

 % \ifpdf
%  \usepackage{pdfsync}
%  \if


%\title{Brief Article}
%\author{David F. Snyder}
%\author{L.G. Meredith}

%\address{Dept. of Math., Texas State University--San Marcos, San Marcos, TX 78666}
       
\pagestyle{empty}


\begin{document}

\lstset{language=[Objective]Caml,frame=shadowbox}

\input{qm2pi.front}

% section front matter (end)

\input{qm2pi.intro} 
 
% section introduction (end)

% \input{qm2pi.knotations} 

% section notation (end)

\input{qm2pi.process.calculi} 

% section concurrent_process_calculi_and_spatial_logics_ (end)
    
%\input{qm2pi.knots2pi} 

%\input{qm2pi.trefoil} 

%\input{qm2pi.mainthm} 

% subsection basic_interpretation (end)

%\input{qm2pi.rho.presentation} 
\subsection{The syntax and semantics of the notation system}\label{sub:the_syntax_and_semantics_of_the_notation_system} % (fold)

We now summarize a technical presentation of the calculus that
embodies our theory of dynamics. The typical presentation of such a
calculus follows the style of giving generators and relations on
them. The grammar, below, describing term constructors, freely
generates the set of processes, $\Proc$. This set is then quotiented
by a relation known as structural congruence and it is over this set
that the notion of dynamics is expressed. This presentation is
essentially that of \cite{MeredithR05} with the addition of
polyadicity and summation. For readability we have relegated some of
the technical subtleties to an appendix.

\subsubsection{Process grammar}\label{subsub:process_grammar}

\begin{mathpar}
  \inferrule* [lab=synchronization] {} {{M} \bc \pzero \;|\; x?F \;|\; x!C }
  \and
  \inferrule* [lab=abstraction] {} {{F} \bc (x)P}
  \and
  \inferrule* [lab=concretion] {} {{C} \bc \langle Q \rangle}
  \and
  \inferrule* [lab=process] {} {{P,Q} \bc M \;| \;P|Q \;|\; @{x}}
  \and
  \inferrule* [lab=name] {} {{x} \bc \quotep{P}}
\end{mathpar} 

Note that $\vec{x}$ (resp. $\vec{P}$) denotes a vector of names
(resp. processes) of length $|\vec{x}|$ (resp. $|\vec{P}|$). We adopt
the following useful abbreviations.

\begin{mathpar}
   x?(\vec{y}).P := x.(\vec{y})P \and  x\clift{\vec{P}} := x.\clift{\vec{P}}
   \and x!(y) := \lift{x}{\dropn{y}}
   \and \Pi_{i=0}^{n-1}P_i := P_0 | \ldots | P_{n-1}
\end{mathpar}

\subsubsection{Structural congruence}

\paragraph{Free and bound names and alpha-equivalence.} At the
core of structural equivalence is alpha-equivalence which identifies
process that are the same up to a change of variable. Formally, we
recognize the distinction between free and bound names. The free names
of a process, $\freenames{P}$, may be calculated recursively as
follows:

\begin{mathpar}
\freenames{\pzero} := \emptyset
  \and \\
  \freenames{x?(y).P} := \{ x \} \cup (\freenames{P} \setminus \{ y \})
  \and 
  \freenames{x!\langle P \rangle} := \{ x \} \cup \{ P \} 
  \and \\
  \freenames{P|Q} := \freenames{P} \cup \freenames{Q}
  \and \\
  \freenames{@{x}} := \{ x \}
\end{mathpar}

$\pi$
$\quotep{\pi}$

$\freenames{-} : \pi \to \mathcal{P}(\quotep{\pi})$

\begin{eqnarray*}
  \freenames{\pzero} & := & \emptyset \\
  \freenames{x?(y).P} & := & \{ x \} \cup (\freenames{P} \setminus \{ y \}) \\
  \freenames{x!\langle P \rangle} & := & \{ x \} \cup \{ P \} \\
  \freenames{P|Q} & := & \freenames{P} \cup \freenames{Q} \\
  \freenames{\dropn{x}} & := & \{ x \}
\end{eqnarray*}

The bound names of a process, $\boundnames{P}$, are those names occurring in $P$
that are not free. For example, in $x?(y).0$, the name $x$ is free, while $y$ is bound.

\begin{mathpar}
  \inferrule* [lab=monoidal-laws] {} { P|Q \equiv Q|P \and P|0 \equiv P \and P|(Q|R) \equiv (P|Q)|R }
\end{mathpar}

\begin{mathpar}
  \inferrule* [lab=alpha-equivalence] {} { (x)P \equiv (y)P\{y/x\} \and y \not\in \freenames{P} }
\end{mathpar}

\begin{definition}
Then two processes, $P,Q$, are alpha-equivalent if $P = Q\{\vec{y}/\vec{x}\}$ for
some $\vec{x} \in \boundnames{Q},\vec{y} \in \boundnames{P}$, where $Q\{\vec{y}/\vec{x}\}$
denotes the capture-avoiding substitution of $\vec{y}$ for $\vec{x}$ in $Q$.
\end{definition}

\begin{definition}
  The {\em structural congruence} \cite{SangiorgiWalker} , $\equiv$,
  between processes is the least congruence containing
  alpha-equivalence, satisfying the abelian monoid laws
  (associativity, commutativity and $\pzero$ as identity) for parallel
  composition $|$ and for summation $+$.
\end{definition}

\subsection{Name equivalence}

We take name equivalence, written $\nameeq$, to be the smallest
equivalence relation generated by the following rules.

\begin{mathpar}
\inferrule*[lab=Quote-drop]
{ }
{ \quotep{@{x}} \nameeq x }

\inferrule*[lab=Struct-equiv]
{ P \scong Q }
{ \quotep{P} \nameeq \quotep{Q} }
\end{mathpar}

The astute reader will have noticed that the mutual recursion of names
and processes imposes a mutual recursion on alpha-equivalence and
structural equivalence via name-equivalence. Fortunately, all of this
works out pleasantly and we may calculate in the natural way, free of
concern. The reader interested in the details is referred to the
appendix \ref{appendix:rho_details}.

\subsection{Substitution}

We use $\Proc$ for the set of processes, $\QProc$ for the set of
names, and $\id{\{}\vec{y} / \vec{x} \id{\}}$ to denote partial maps,
$s : \QProc \rightarrow \QProc$. A map, $s$ lifts, uniquely, to a map
on process terms, $\widehat{s} : \Proc \rightarrow \Proc$ by the
following equations.

\begin{mathpar}
  (0) \psubstp{Q}{P} := 0 \\
  (R \juxtap S) \psubstp{Q}{P}
  :=    
  (R)\psubstp{Q}{P} \juxtap (S) \psubstp{Q}{P} \\
  (x?(y).R) \psubstp{Q}{P}    
  :=    
  (x)\substp{Q}{P} (z)\concat( (R \psubstn{z}{y}) \psubstp{Q}{P} ) \\
  (\lift{x}{R}) \psubstp{Q}{P}  
  :=
  \lift{(x)\substp{Q}{P}}{ R \psubstp{Q}{P} } \\
%   (\dropn{x})  \psubstp{Q}{P}       
%   := 
%   \left\{ 
%     \begin{array}{ccc} 
%       \dropn{\quotep{Q}} & & x \nameeq \quotep{P} \\
%       \dropn{x} & & otherwise \\
%     \end{array}
%   \right. 
  (\dropn{x})  \psubstp{Q}{P}       
  := 
  \left\{ 
    \begin{array}{ccc} 
      Q & & x \nameeq \quotep{P} \\
      \dropn{x} & & otherwise \\
    \end{array}
  \right.
\end{mathpar}
 

where

\begin{eqnarray}
  (x)\id{\{} \lpquote Q \rpquote / \lpquote P \rpquote \id{\}}            = 
  \left\{ 
    \begin{array}{ccc}
      \lpquote Q \rpquote & & x \nameeq \lpquote P \rpquote \\
      x & & otherwise \\
    \end{array}
  \right. \nonumber
\end{eqnarray}

and $z$ is chosen distinct from $\quotep{P}$, $\quotep{Q}$, the free
names in $Q$, and all the names in $R$. Our $\alpha$-equivalence will
be built in the standard way from this substitution.

\begin{remark}\label{rem:no_self_referential_names}
  One consequence of these definitions is that $\forall P. \quotep{P}
  \not\in \freenames{P}$.
\end{remark}

\subsection{ Dynamic quote: an example }

Anticipating something of what's to come, consider applying the
substitution, $\widehat{\id{\{}u / z \id{\}}}$, to the following pair
of processes, $\lift{w}{y!(z)}$ and $w[ \lpquote y!(z) \rpquote ]$.

\begin{eqnarray}
	\lift{w}{y!(z)}\widehat{\id{\{}u / z \id{\}}}
		& = &
		\lift{w}{y!(u)} \nonumber\\
	w[ \lpquote y!(z) \rpquote ] \widehat{ \id{\{}u / z \id{\}} }
		& = &
		w[ \lpquote y!(z) \rpquote ] \nonumber
\end{eqnarray}

Because the body of the process between quotes is impervious to
substitution, we get radically different answers. In fact, by
examining the first process in an input context,
e.g. $x?(z).\lift{w}{y!(z)}$, we see that the process under the lift
operator may be shaped by prefixed inputs binding a name inside it. In
this sense, the lift operator will be seen as a way to dynamically
construct processes before reifying them as names.

Finally equipped with these standard features we can present the
dynamics of the calculus.

\subsubsection{Operational semantics} 

Finally, we introduce the computational dynamics. What marks these
algebras as distinct from other more traditionally studied algebraic
structures, e.g. vector spaces or polynomial rings, is the manner in
which dynamics is captured. In traditional structures, dynamics is typically
expressed through morphisms between such structures, as in linear maps
between vector spaces or morphisms between rings. In algebras
associated with the semantics of computation, the dynamics is
expressed as part of the algebraic structure itself, through a
reduction reduction relation typically denoted by $\red$. Below, we
give a recursive presentation of this relation for the calculus used
in the encoding.

$\red \subseteq \pi \times \pi$
$\red : \pi \to \mathcal{P}(\pi)$

\begin{mathpar}
  \inferrule* [lab=Comm] { \textsf{match}( x_{src}, x_{trgt} ) } { x_{trgt}?(y)P \; | \; x_{src}!\langle {Q} \rangle \red P\{\quotep{Q}/y}\} }
  \and \\
  \inferrule* [lab=Par] {{P} \red {P}'} {{{P} | {Q}} \red {{P}' | {Q}}}
  \and
  \inferrule* [lab=Equiv]{{{P} \scong {P}'} \andalso {{P}' \red {Q}'} \andalso {{Q}' \scong {Q}}}{{P} \red {Q}}
\end{mathpar}

\begin{eqnarray*}
  match_{\equiv} (\quotep{P},\quotep{Q}) & := & P \equiv Q \\
  match_{\dagger}(\quotep{P},\quotep{Q}) & := & \forall R. P|Q \red^{*} R => R \red^{*} 0 \\
  match_{K}(\quotep{P},\quotep{Q}) & := & K \mbox{ for some context } K
\end{eqnarray*}

$u?(x)P | u!\langle Q \rangle \red P\{\quotep{Q}/x\}$

%We write $\wred$ for $\red^*$, and $P\red$ if $\exists Q $ such that $ P \red Q$.
We write $P\red$ if $\exists Q $ such that $ P \red Q$ and $P\not\red$, otherwise.

\section{Replication}

As mentioned before, it is known that replication (and hence
recursion) can be implemented in a higher-order process algebra
\cite{SangiorgiWalker}. As our first example of calculation with the
machinery thus far presented we give the construction explicitly in
the {\rhoc}.

\begin{eqnarray}
	D_{x} & := & \prefix{x}{y}{(\binpar{\outputp{x}{y}}{@{y}})} \nonumber\\
	\bangp_{x}{P} & := & \binpar{{x}!\langle{\binpar{D_{x}}{P}}\rangle}{D_{x}} \nonumber
\end{eqnarray}

\begin{eqnarray}
	\bangp_{x}{P} & & \nonumber\\
	=
	& {x}!\langle{(\prefix{x}{y}{(\outputp{x}{y} | @{y})) | P}}\rangle 
	      | \prefix{x}{y}{(\outputp{x}{y} | @{y})} & \nonumber\\
	\red
	& (\outputp{x}{y} | @{y})\substn{\quotep{(\prefix{x}{y}{(@{y} | \outputp{x}{y})) | P}}}{y} & \nonumber\\
	=
	& \outputp{x}{\quotep{(\prefix{x}{y}{(\outputp{x}{y} | @{y})) | P}}}
	  | {(\prefix{x}{y}{(\outputp{x}{y} | @{y})) | P}} & \nonumber\\
	\red
	& \ldots & \nonumber\\
	\red^*
	& P | P | \ldots & \nonumber
\end{eqnarray}

Of course, this encoding, as an implementation, runs away, unfolding
$\bangp{P}$ eagerly. A lazier and more implementable replication
operator, restricted to input-guarded processes, may be obtained as follows.

\begin{eqnarray}
\bangp{\prefix{u}{v}{P}} 
	:= 
	\binpar{\lift{x}{\prefix{u}{v}{(\binpar{D(x)}{P})}}}{D(x)} \nonumber
\end{eqnarray}

\begin{remark}
  Note that the lazier definition still does not deal with summation
  or mixed summation (i.e. sums over input and output). The reader is
  invited to construct definitions of replication that deal with these
  features. 

  Further, the definitions are parameterized in a name, $x$. Can you,
  gentle reader, make a definition that eliminates this parameter and
  guarantees no accidental interaction between the replication
  machinery and the process being replicated -- i.e. no accidental
  sharing of names used by the process to get its work done and the
  name(s) used by the replication to effect copying. This latter
  revision of the definition of replication is crucial to obtaining
  the expected identity $!!P \sim !P$.
\end{remark}

\begin{remark}\label{rem:paradoxical_combinator}
  The reader familiar with the lambda calculus will have noticed the
  similarity between $D$ and the paradoxical combinator.

  [Ed. note: the existence of this seems to suggest we have to be more
  restrictive on the set of processes and names we admit if we are to
  support no-cloning.]
\end{remark}

\subsubsection{Bisimulation}

The computational dynamics gives rise to another kind of equivalence,
the equivalence of computational behavior. As previously mentioned
this is typically captured \emph{via} some form of bisimulation.

% The notion we use in this paper is weak barbed bisimulation
% \cite{milner91polyadicpi}.

The notion we use in this paper is derived from weak barbed
bisimulation \cite{milner91polyadicpi}. 

\begin{definition}
An \emph{observation relation}, $\downarrow_{\mathcal N}$, over a set
of names, $\mathcal N$, is the smallest relation satisfying the rules
below.

\infrule[Out-barb]{y \in {\mathcal N}, \; x \nameeq y}
		  {\outputp{x}{v} \downarrow_{\mathcal N} x}
\infrule[Par-barb]{\mbox{$P\downarrow_{\mathcal N} x$ or $Q\downarrow_{\mathcal N} x$}}
		  {\binpar{P}{Q} \downarrow_{\mathcal N} x}

We write $P \Downarrow_{\mathcal N} x$ if there is $Q$ such that 
$P \wred Q$ and $Q \downarrow_{\mathcal N} x$.
\end{definition}

\begin{definition}
%\label{def.bbisim}
An  ${\mathcal N}$-\emph{barbed bisimulation} over a set of names, ${\mathcal N}$, is a symmetric binary relation 
${\mathcal S}_{\mathcal N}$ between agents such that $P\rel{S}_{\mathcal N}Q$ implies:
\begin{enumerate}
\item If $P \red P'$ then $Q \wred Q'$ and $P'\rel{S}_{\mathcal N} Q'$.
\item If $P\downarrow_{\mathcal N} x$, then $Q\Downarrow_{\mathcal N} x$.
\end{enumerate}
$P$ is ${\mathcal N}$-barbed bisimilar to $Q$, written
$P \wbbisim_{\mathcal N} Q$, if $P \rel{S}_{\mathcal N} Q$ for some ${\mathcal N}$-barbed bisimulation ${\mathcal S}_{\mathcal N}$.
\end{definition}

$\mathcal{R} \subseteq \pi \times \pi$

$P \mathcal{R} Q => \forall P'. P \red P' \Rightarrow \exists Q'. Q \red Q', P' \mathcal{R} Q'$

$P \vdash x \Rightarrow Q \vdash x$

\begin{mathpar}
  \inferrule*[lab=Out-barb]{x \nameeq y}{{y}!\langle{Q}\rangle \vdash x}
  \and
  \inferrule*[lab=Par-barb]{\mbox{$P\vdash x$ or $Q\vdash x$}}{\binpar{P}{Q} \vdash x}
\end{mathpar}

\subsubsection{Contexts}

One of the principle advantages of computational calculi like the
$\pi$-calculus is a well-defined notion of context,
contextual-equivalence and a correlation between
contextual-equivalence and notions of bisimulation. The notion of
context allows the decomposition of a process into (sub-)process and
its syntactic environment, its context. Thus, a context may be
thought of as a process with a ``hole'' (written $\Box$) in it. The
application of a context $M$ to a process $P$, written $M[P]$, is
tantamount to filling the hole in $M$ with $P$. In this paper we do
not need the full weight of this theory, but do make use of the notion
of context in the proof the main theorem. 

\begin{mathpar}
  \inferrule* [lab=summation] {} {{M_{M},M_{N}} \bc \Box \;|\; x.M_{A} \;|\; M_{M}+M_{N}}
  \and
  \inferrule* [lab=agent] {} {{M_{A}} \bc (\vec{x})M_{P} \;| \; \clift{P_0,\ldots,M_{P},\ldots,P_N}}
  \and \\
  \inferrule* [lab=process] {} {{M_{P}} \bc M_{N} \;| \;P|M_{P} }
\end{mathpar} 

\begin{mathpar}
  \inferrule* [lab=sychronization] {} {M_{N} \bc \Box \;|\; x?M_{F} \;|\; x!M_{C}}
  \and
  \inferrule* [lab=abstraction] {} {{M_{F}} \bc (x)M_{P} }
  \and
  \inferrule* [lab=concretion] {} {{M_{C}} \bc \langle M_{P} \rangle }
  \and \\
  \inferrule* [lab=process] {} {{M_{P}} \bc M_{N} \;| \;P|M_{P} }
\end{mathpar}

\begin{definition}[contextual application] Given a context $M$, and
  process $P$, we define the \emph{contextual application}, $M[P] :=
  M\{P/\Box\}$. That is, the contextual application of M to P is the
  substitution of $P$ for $\Box$ in $M$.
\end{definition}

$\meaningof{-} : L \to \mathcal{P}(\pi)$

\begin{mathpar}
  \inferrule* [lab=collection] {} {\meaningof{true} = \pi, \and \meaningof{~E} = \pi \setminus \meaningof{E}, \and \meaningof{E_{1} \& E_{2}} = \meaningof{E_{1}} \cap \meaningof{E_{2}}}
\end{mathpar}

\begin{mathpar}
  \inferrule* [lab=structure] {} {\meaningof{0} = \{ P \in \pi | P \equiv 0 \}, \and \\ \meaningof{E_1 | E_2} = \{ P \in \pi | P \equiv P_{1} | P_{2}, P_{1} \in \meaningof{E_{1}}, P_{2} \in \meaningof{E_2}\} }
\end{mathpar}

\begin{mathpar}
 \inferrule* [lab=behavior] {} {\meaningof{\langle a?b \rangle E} = \{ P \in \pi | P \equiv Q | u?(y)P', \\ \and \\\\ \and \\ \;\;\; u \in \meaningof{a}, \forall z.P'\{z/y\} \in \meaningof{E\{z/b\}}\}, \and \\ \meaningof{a!E} = \{ P \in \pi | P \equiv Q | x!\langle P' \rangle, x \in \meaningof{a} P' \in \meaningof{E}\} }
\end{mathpar}

\begin{mathpar}
 \inferrule* [lab=nominal] {} {\meaningof{\quotep{E}} = \{ \quotep{P} \in \quotep{\pi} | P \in \meaningof{E} \}, \and \meaningof{\quotep{P}} = \{ \quotep{Q} \in \quotep{\pi} | P \equiv Q \} \and \\ \meaningof{@\quotep{E}} = \{ P \in \pi | P \equiv @x, x \in \meaningof{E} \}}
\end{mathpar}

\begin{eqnarray*}
  \\
  \meaningof{-} : TS \to ST
\end{eqnarray*}

\begin{eqnarray*}
  \\
  L : TS \to ST
\end{eqnarray*}

\begin{eqnarray*}
  \\
  P \models E \iff P \in \meaningof{E}
\end{eqnarray*}

\begin{eqnarray*}
  P \approx_{L} Q \iff \forall E \in L. P \models E \iff Q \models E
\end{eqnarray*}

\begin{eqnarray*}
  P \approx_{K} Q
\end{eqnarray*}

\begin{eqnarray*}
  P \approx Q
\end{eqnarray*}

$\approx_{K} = \approx = \approx_{L}$

\subsubsection{Contextual duality}

Note that contexts extend the quotation operation to a family of
operations from processes to names. Given a context, $M$, we can
define a \emph{nominal context}, $\quotep{M}$ by $\quotep{M}[P] :=
\quotep{M[P]}$. To foreshadow what is to come we observe that these
operations enjoy a duality with processes very much like the duality
between vectors and maps from vectors to scalars.

Further, because the calculus is essentially higher-order, we have a
correspondence between contexts and processes. More specifically,
given a name $x$ and a context $M$ we can construct $M^{*}_{x}$ such
that 

\begin{mathpar}
  M^{*}_{x} | \lift{x}{P} \red M[P]
\end{mathpar}

namely,

\begin{mathpar}
  M^{*}_{x} := x?(u).M[\dropn{u}]
\end{mathpar}

The dependence of $M^{*}_{x}$ on a name makes it an abstraction, 

\begin{mathpar}
  M^{*} := (x)x?(u).M[\dropn{u}]
\end{mathpar}

\subsection{Additional notation}

It will sometimes be convenient to denote the process a name
quotes. We already have the notation $x = \quotep{P}$, but it will be
convenient to introduce an alternate notation, $\procn{x}$, when we
want to emphasize the connection to the use of the name. Note that, by
virtue of name equivalence, $\quotep{\procn{x}} \nameeq x$; so, the
notation is consistent with previous definitions.

Further, because names have structure it is possible to effect
substitutions on the basis of that structure. This means we need to
upgrade our notation for substitutions, which we accomplish by
adapting comprehension notation. Thus,

\begin{mathpar}
  P\{ y / x : x \in S \}
\end{mathpar}

is interpreted to mean the process derived from P by replacing (in a
capture-avoiding manner) each occurrence of $x$ in $S$ by $y$. For example,

\begin{mathpar}
  P\{ \quotep{\procn{x}|\procn{x}} / x : x \in \freenames{P} \}
\end{mathpar}

will replace each (occurrence) of a free name $x$ in $P$ by
$\quotep{\procn{x}|\procn{x}}$.

Also, we will avail ourselves of the notation $x^{L}$ and $x^{R}$ to
denote injections of a name into disjoint copies of the name
space. There are numerous ways to accomplish this. One example can be
found in \cite{MeredithR05}. This notation overloads to vectors of
names: $\vec{x}^{\pi} := (x_{i}^{\pi} \; : \; 0 \leq i < |\vec{x}| )$ where $\pi \in \{L,R\}$.

We also use $P^{\Box} := P|\Box$.

In \cite{MeredithR05} an interpretation of the new operator is
given. It turns out that there are several possible interpretations
all enjoying the requisite algebraic properties of the operator (see
\cite{milner91polyadicpi}). We will therefore make liberal use of
$(\nu\; \vec{x})P$.

% subsection the_syntax_and_semantics_of_the_notation_system (end)   

\input{qm2pi.qmops} 

\input{qm2pi.sterngerlach} 

\input{qm2pi.metric} 

% section concurrent_process_calculi (end)

%\input{qm2pi.proofsketch}

% section proof sketch (end)

%\input{qm2pi.slviaknots} 

% section spatial logic via knots (end)

\input{qm2pi.conclusion}

% section conclusion (end)

%\input{qm2pi.dtcodes} 

% section wiring algorithm (end)

\input{qm2pi.ack} 

% section acknowledgments (end)

\newpage


\bibliographystyle{plain}   
\bibliography{../../biblios/main.bib}

\input{qm2pi.rhodetails}

\end{document}

 

% section wiring algorithm (end)

\documentclass[12pt]{llncs}
%\documentclass{jktr}

\usepackage[pdftex]{hyperref}                   
\usepackage {listings}
\usepackage {mathpartir}
\usepackage{bcprules}
%\usepackage{listings}
                       
\usepackage{graphicx} 
%\usepackage[margins=2.5cm,nohead,nofoot]{geometry}
%\usepackage{geometry}
\usepackage{amsfonts}
\usepackage{amstext}
\usepackage{latexsym}
\usepackage{amssymb}
\usepackage{color}


%\include{myPreamble}
\include{qm2pi.local} 

%\ifpdf
%\usepackage[pdftex]{graphicx}
%\else
%\usepackage{graphicx}
%\fi

 % \ifpdf
%  \usepackage{pdfsync}
%  \if


%\title{Brief Article}
%\author{David F. Snyder}
%\author{L.G. Meredith}

%\address{Dept. of Math., Texas State University--San Marcos, San Marcos, TX 78666}
       
\pagestyle{empty}


\begin{document}

\lstset{language=[Objective]Caml,frame=shadowbox}

\input{qm2pi.front}

% section front matter (end)

\input{qm2pi.intro} 
 
% section introduction (end)

% \input{qm2pi.knotations} 

% section notation (end)

\input{qm2pi.process.calculi} 

% section concurrent_process_calculi_and_spatial_logics_ (end)
    
%\input{qm2pi.knots2pi} 

%\input{qm2pi.trefoil} 

%\input{qm2pi.mainthm} 

% subsection basic_interpretation (end)

%\input{qm2pi.rho.presentation} 
\subsection{The syntax and semantics of the notation system}\label{sub:the_syntax_and_semantics_of_the_notation_system} % (fold)

We now summarize a technical presentation of the calculus that
embodies our theory of dynamics. The typical presentation of such a
calculus follows the style of giving generators and relations on
them. The grammar, below, describing term constructors, freely
generates the set of processes, $\Proc$. This set is then quotiented
by a relation known as structural congruence and it is over this set
that the notion of dynamics is expressed. This presentation is
essentially that of \cite{MeredithR05} with the addition of
polyadicity and summation. For readability we have relegated some of
the technical subtleties to an appendix.

\subsubsection{Process grammar}\label{subsub:process_grammar}

\begin{mathpar}
  \inferrule* [lab=synchronization] {} {{M} \bc \pzero \;|\; x?F \;|\; x!C }
  \and
  \inferrule* [lab=abstraction] {} {{F} \bc (x)P}
  \and
  \inferrule* [lab=concretion] {} {{C} \bc \langle Q \rangle}
  \and
  \inferrule* [lab=process] {} {{P,Q} \bc M \;| \;P|Q \;|\; @{x}}
  \and
  \inferrule* [lab=name] {} {{x} \bc \quotep{P}}
\end{mathpar} 

Note that $\vec{x}$ (resp. $\vec{P}$) denotes a vector of names
(resp. processes) of length $|\vec{x}|$ (resp. $|\vec{P}|$). We adopt
the following useful abbreviations.

\begin{mathpar}
   x?(\vec{y}).P := x.(\vec{y})P \and  x\clift{\vec{P}} := x.\clift{\vec{P}}
   \and x!(y) := \lift{x}{\dropn{y}}
   \and \Pi_{i=0}^{n-1}P_i := P_0 | \ldots | P_{n-1}
\end{mathpar}

\subsubsection{Structural congruence}

\paragraph{Free and bound names and alpha-equivalence.} At the
core of structural equivalence is alpha-equivalence which identifies
process that are the same up to a change of variable. Formally, we
recognize the distinction between free and bound names. The free names
of a process, $\freenames{P}$, may be calculated recursively as
follows:

\begin{mathpar}
\freenames{\pzero} := \emptyset
  \and \\
  \freenames{x?(y).P} := \{ x \} \cup (\freenames{P} \setminus \{ y \})
  \and 
  \freenames{x!\langle P \rangle} := \{ x \} \cup \{ P \} 
  \and \\
  \freenames{P|Q} := \freenames{P} \cup \freenames{Q}
  \and \\
  \freenames{@{x}} := \{ x \}
\end{mathpar}

$\pi$
$\quotep{\pi}$

$\freenames{-} : \pi \to \mathcal{P}(\quotep{\pi})$

\begin{eqnarray*}
  \freenames{\pzero} & := & \emptyset \\
  \freenames{x?(y).P} & := & \{ x \} \cup (\freenames{P} \setminus \{ y \}) \\
  \freenames{x!\langle P \rangle} & := & \{ x \} \cup \{ P \} \\
  \freenames{P|Q} & := & \freenames{P} \cup \freenames{Q} \\
  \freenames{\dropn{x}} & := & \{ x \}
\end{eqnarray*}

The bound names of a process, $\boundnames{P}$, are those names occurring in $P$
that are not free. For example, in $x?(y).0$, the name $x$ is free, while $y$ is bound.

\begin{mathpar}
  \inferrule* [lab=monoidal-laws] {} { P|Q \equiv Q|P \and P|0 \equiv P \and P|(Q|R) \equiv (P|Q)|R }
\end{mathpar}

\begin{mathpar}
  \inferrule* [lab=alpha-equivalence] {} { (x)P \equiv (y)P\{y/x\} \and y \not\in \freenames{P} }
\end{mathpar}

\begin{definition}
Then two processes, $P,Q$, are alpha-equivalent if $P = Q\{\vec{y}/\vec{x}\}$ for
some $\vec{x} \in \boundnames{Q},\vec{y} \in \boundnames{P}$, where $Q\{\vec{y}/\vec{x}\}$
denotes the capture-avoiding substitution of $\vec{y}$ for $\vec{x}$ in $Q$.
\end{definition}

\begin{definition}
  The {\em structural congruence} \cite{SangiorgiWalker} , $\equiv$,
  between processes is the least congruence containing
  alpha-equivalence, satisfying the abelian monoid laws
  (associativity, commutativity and $\pzero$ as identity) for parallel
  composition $|$ and for summation $+$.
\end{definition}

\subsection{Name equivalence}

We take name equivalence, written $\nameeq$, to be the smallest
equivalence relation generated by the following rules.

\begin{mathpar}
\inferrule*[lab=Quote-drop]
{ }
{ \quotep{@{x}} \nameeq x }

\inferrule*[lab=Struct-equiv]
{ P \scong Q }
{ \quotep{P} \nameeq \quotep{Q} }
\end{mathpar}

The astute reader will have noticed that the mutual recursion of names
and processes imposes a mutual recursion on alpha-equivalence and
structural equivalence via name-equivalence. Fortunately, all of this
works out pleasantly and we may calculate in the natural way, free of
concern. The reader interested in the details is referred to the
appendix \ref{appendix:rho_details}.

\subsection{Substitution}

We use $\Proc$ for the set of processes, $\QProc$ for the set of
names, and $\id{\{}\vec{y} / \vec{x} \id{\}}$ to denote partial maps,
$s : \QProc \rightarrow \QProc$. A map, $s$ lifts, uniquely, to a map
on process terms, $\widehat{s} : \Proc \rightarrow \Proc$ by the
following equations.

\begin{mathpar}
  (0) \psubstp{Q}{P} := 0 \\
  (R \juxtap S) \psubstp{Q}{P}
  :=    
  (R)\psubstp{Q}{P} \juxtap (S) \psubstp{Q}{P} \\
  (x?(y).R) \psubstp{Q}{P}    
  :=    
  (x)\substp{Q}{P} (z)\concat( (R \psubstn{z}{y}) \psubstp{Q}{P} ) \\
  (\lift{x}{R}) \psubstp{Q}{P}  
  :=
  \lift{(x)\substp{Q}{P}}{ R \psubstp{Q}{P} } \\
%   (\dropn{x})  \psubstp{Q}{P}       
%   := 
%   \left\{ 
%     \begin{array}{ccc} 
%       \dropn{\quotep{Q}} & & x \nameeq \quotep{P} \\
%       \dropn{x} & & otherwise \\
%     \end{array}
%   \right. 
  (\dropn{x})  \psubstp{Q}{P}       
  := 
  \left\{ 
    \begin{array}{ccc} 
      Q & & x \nameeq \quotep{P} \\
      \dropn{x} & & otherwise \\
    \end{array}
  \right.
\end{mathpar}
 

where

\begin{eqnarray}
  (x)\id{\{} \lpquote Q \rpquote / \lpquote P \rpquote \id{\}}            = 
  \left\{ 
    \begin{array}{ccc}
      \lpquote Q \rpquote & & x \nameeq \lpquote P \rpquote \\
      x & & otherwise \\
    \end{array}
  \right. \nonumber
\end{eqnarray}

and $z$ is chosen distinct from $\quotep{P}$, $\quotep{Q}$, the free
names in $Q$, and all the names in $R$. Our $\alpha$-equivalence will
be built in the standard way from this substitution.

\begin{remark}\label{rem:no_self_referential_names}
  One consequence of these definitions is that $\forall P. \quotep{P}
  \not\in \freenames{P}$.
\end{remark}

\subsection{ Dynamic quote: an example }

Anticipating something of what's to come, consider applying the
substitution, $\widehat{\id{\{}u / z \id{\}}}$, to the following pair
of processes, $\lift{w}{y!(z)}$ and $w[ \lpquote y!(z) \rpquote ]$.

\begin{eqnarray}
	\lift{w}{y!(z)}\widehat{\id{\{}u / z \id{\}}}
		& = &
		\lift{w}{y!(u)} \nonumber\\
	w[ \lpquote y!(z) \rpquote ] \widehat{ \id{\{}u / z \id{\}} }
		& = &
		w[ \lpquote y!(z) \rpquote ] \nonumber
\end{eqnarray}

Because the body of the process between quotes is impervious to
substitution, we get radically different answers. In fact, by
examining the first process in an input context,
e.g. $x?(z).\lift{w}{y!(z)}$, we see that the process under the lift
operator may be shaped by prefixed inputs binding a name inside it. In
this sense, the lift operator will be seen as a way to dynamically
construct processes before reifying them as names.

Finally equipped with these standard features we can present the
dynamics of the calculus.

\subsubsection{Operational semantics} 

Finally, we introduce the computational dynamics. What marks these
algebras as distinct from other more traditionally studied algebraic
structures, e.g. vector spaces or polynomial rings, is the manner in
which dynamics is captured. In traditional structures, dynamics is typically
expressed through morphisms between such structures, as in linear maps
between vector spaces or morphisms between rings. In algebras
associated with the semantics of computation, the dynamics is
expressed as part of the algebraic structure itself, through a
reduction reduction relation typically denoted by $\red$. Below, we
give a recursive presentation of this relation for the calculus used
in the encoding.

$\red \subseteq \pi \times \pi$
$\red : \pi \to \mathcal{P}(\pi)$

\begin{mathpar}
  \inferrule* [lab=Comm] { \textsf{match}( x_{src}, x_{trgt} ) } { x_{trgt}?(y)P \; | \; x_{src}!\langle {Q} \rangle \red P\{\quotep{Q}/y}\} }
  \and \\
  \inferrule* [lab=Par] {{P} \red {P}'} {{{P} | {Q}} \red {{P}' | {Q}}}
  \and
  \inferrule* [lab=Equiv]{{{P} \scong {P}'} \andalso {{P}' \red {Q}'} \andalso {{Q}' \scong {Q}}}{{P} \red {Q}}
\end{mathpar}

\begin{eqnarray*}
  match_{\equiv} (\quotep{P},\quotep{Q}) & := & P \equiv Q \\
  match_{\dagger}(\quotep{P},\quotep{Q}) & := & \forall R. P|Q \red^{*} R => R \red^{*} 0 \\
  match_{K}(\quotep{P},\quotep{Q}) & := & K \mbox{ for some context } K
\end{eqnarray*}

$u?(x)P | u!\langle Q \rangle \red P\{\quotep{Q}/x\}$

%We write $\wred$ for $\red^*$, and $P\red$ if $\exists Q $ such that $ P \red Q$.
We write $P\red$ if $\exists Q $ such that $ P \red Q$ and $P\not\red$, otherwise.

\section{Replication}

As mentioned before, it is known that replication (and hence
recursion) can be implemented in a higher-order process algebra
\cite{SangiorgiWalker}. As our first example of calculation with the
machinery thus far presented we give the construction explicitly in
the {\rhoc}.

\begin{eqnarray}
	D_{x} & := & \prefix{x}{y}{(\binpar{\outputp{x}{y}}{@{y}})} \nonumber\\
	\bangp_{x}{P} & := & \binpar{{x}!\langle{\binpar{D_{x}}{P}}\rangle}{D_{x}} \nonumber
\end{eqnarray}

\begin{eqnarray}
	\bangp_{x}{P} & & \nonumber\\
	=
	& {x}!\langle{(\prefix{x}{y}{(\outputp{x}{y} | @{y})) | P}}\rangle 
	      | \prefix{x}{y}{(\outputp{x}{y} | @{y})} & \nonumber\\
	\red
	& (\outputp{x}{y} | @{y})\substn{\quotep{(\prefix{x}{y}{(@{y} | \outputp{x}{y})) | P}}}{y} & \nonumber\\
	=
	& \outputp{x}{\quotep{(\prefix{x}{y}{(\outputp{x}{y} | @{y})) | P}}}
	  | {(\prefix{x}{y}{(\outputp{x}{y} | @{y})) | P}} & \nonumber\\
	\red
	& \ldots & \nonumber\\
	\red^*
	& P | P | \ldots & \nonumber
\end{eqnarray}

Of course, this encoding, as an implementation, runs away, unfolding
$\bangp{P}$ eagerly. A lazier and more implementable replication
operator, restricted to input-guarded processes, may be obtained as follows.

\begin{eqnarray}
\bangp{\prefix{u}{v}{P}} 
	:= 
	\binpar{\lift{x}{\prefix{u}{v}{(\binpar{D(x)}{P})}}}{D(x)} \nonumber
\end{eqnarray}

\begin{remark}
  Note that the lazier definition still does not deal with summation
  or mixed summation (i.e. sums over input and output). The reader is
  invited to construct definitions of replication that deal with these
  features. 

  Further, the definitions are parameterized in a name, $x$. Can you,
  gentle reader, make a definition that eliminates this parameter and
  guarantees no accidental interaction between the replication
  machinery and the process being replicated -- i.e. no accidental
  sharing of names used by the process to get its work done and the
  name(s) used by the replication to effect copying. This latter
  revision of the definition of replication is crucial to obtaining
  the expected identity $!!P \sim !P$.
\end{remark}

\begin{remark}\label{rem:paradoxical_combinator}
  The reader familiar with the lambda calculus will have noticed the
  similarity between $D$ and the paradoxical combinator.

  [Ed. note: the existence of this seems to suggest we have to be more
  restrictive on the set of processes and names we admit if we are to
  support no-cloning.]
\end{remark}

\subsubsection{Bisimulation}

The computational dynamics gives rise to another kind of equivalence,
the equivalence of computational behavior. As previously mentioned
this is typically captured \emph{via} some form of bisimulation.

% The notion we use in this paper is weak barbed bisimulation
% \cite{milner91polyadicpi}.

The notion we use in this paper is derived from weak barbed
bisimulation \cite{milner91polyadicpi}. 

\begin{definition}
An \emph{observation relation}, $\downarrow_{\mathcal N}$, over a set
of names, $\mathcal N$, is the smallest relation satisfying the rules
below.

\infrule[Out-barb]{y \in {\mathcal N}, \; x \nameeq y}
		  {\outputp{x}{v} \downarrow_{\mathcal N} x}
\infrule[Par-barb]{\mbox{$P\downarrow_{\mathcal N} x$ or $Q\downarrow_{\mathcal N} x$}}
		  {\binpar{P}{Q} \downarrow_{\mathcal N} x}

We write $P \Downarrow_{\mathcal N} x$ if there is $Q$ such that 
$P \wred Q$ and $Q \downarrow_{\mathcal N} x$.
\end{definition}

\begin{definition}
%\label{def.bbisim}
An  ${\mathcal N}$-\emph{barbed bisimulation} over a set of names, ${\mathcal N}$, is a symmetric binary relation 
${\mathcal S}_{\mathcal N}$ between agents such that $P\rel{S}_{\mathcal N}Q$ implies:
\begin{enumerate}
\item If $P \red P'$ then $Q \wred Q'$ and $P'\rel{S}_{\mathcal N} Q'$.
\item If $P\downarrow_{\mathcal N} x$, then $Q\Downarrow_{\mathcal N} x$.
\end{enumerate}
$P$ is ${\mathcal N}$-barbed bisimilar to $Q$, written
$P \wbbisim_{\mathcal N} Q$, if $P \rel{S}_{\mathcal N} Q$ for some ${\mathcal N}$-barbed bisimulation ${\mathcal S}_{\mathcal N}$.
\end{definition}

$\mathcal{R} \subseteq \pi \times \pi$

$P \mathcal{R} Q => \forall P'. P \red P' \Rightarrow \exists Q'. Q \red Q', P' \mathcal{R} Q'$

$P \vdash x \Rightarrow Q \vdash x$

\begin{mathpar}
  \inferrule*[lab=Out-barb]{x \nameeq y}{{y}!\langle{Q}\rangle \vdash x}
  \and
  \inferrule*[lab=Par-barb]{\mbox{$P\vdash x$ or $Q\vdash x$}}{\binpar{P}{Q} \vdash x}
\end{mathpar}

\subsubsection{Contexts}

One of the principle advantages of computational calculi like the
$\pi$-calculus is a well-defined notion of context,
contextual-equivalence and a correlation between
contextual-equivalence and notions of bisimulation. The notion of
context allows the decomposition of a process into (sub-)process and
its syntactic environment, its context. Thus, a context may be
thought of as a process with a ``hole'' (written $\Box$) in it. The
application of a context $M$ to a process $P$, written $M[P]$, is
tantamount to filling the hole in $M$ with $P$. In this paper we do
not need the full weight of this theory, but do make use of the notion
of context in the proof the main theorem. 

\begin{mathpar}
  \inferrule* [lab=summation] {} {{M_{M},M_{N}} \bc \Box \;|\; x.M_{A} \;|\; M_{M}+M_{N}}
  \and
  \inferrule* [lab=agent] {} {{M_{A}} \bc (\vec{x})M_{P} \;| \; \clift{P_0,\ldots,M_{P},\ldots,P_N}}
  \and \\
  \inferrule* [lab=process] {} {{M_{P}} \bc M_{N} \;| \;P|M_{P} }
\end{mathpar} 

\begin{mathpar}
  \inferrule* [lab=sychronization] {} {M_{N} \bc \Box \;|\; x?M_{F} \;|\; x!M_{C}}
  \and
  \inferrule* [lab=abstraction] {} {{M_{F}} \bc (x)M_{P} }
  \and
  \inferrule* [lab=concretion] {} {{M_{C}} \bc \langle M_{P} \rangle }
  \and \\
  \inferrule* [lab=process] {} {{M_{P}} \bc M_{N} \;| \;P|M_{P} }
\end{mathpar}

\begin{definition}[contextual application] Given a context $M$, and
  process $P$, we define the \emph{contextual application}, $M[P] :=
  M\{P/\Box\}$. That is, the contextual application of M to P is the
  substitution of $P$ for $\Box$ in $M$.
\end{definition}

$\meaningof{-} : L \to \mathcal{P}(\pi)$

\begin{mathpar}
  \inferrule* [lab=collection] {} {\meaningof{true} = \pi, \and \meaningof{~E} = \pi \setminus \meaningof{E}, \and \meaningof{E_{1} \& E_{2}} = \meaningof{E_{1}} \cap \meaningof{E_{2}}}
\end{mathpar}

\begin{mathpar}
  \inferrule* [lab=structure] {} {\meaningof{0} = \{ P \in \pi | P \equiv 0 \}, \and \\ \meaningof{E_1 | E_2} = \{ P \in \pi | P \equiv P_{1} | P_{2}, P_{1} \in \meaningof{E_{1}}, P_{2} \in \meaningof{E_2}\} }
\end{mathpar}

\begin{mathpar}
 \inferrule* [lab=behavior] {} {\meaningof{\langle a?b \rangle E} = \{ P \in \pi | P \equiv Q | u?(y)P', \\ \and \\\\ \and \\ \;\;\; u \in \meaningof{a}, \forall z.P'\{z/y\} \in \meaningof{E\{z/b\}}\}, \and \\ \meaningof{a!E} = \{ P \in \pi | P \equiv Q | x!\langle P' \rangle, x \in \meaningof{a} P' \in \meaningof{E}\} }
\end{mathpar}

\begin{mathpar}
 \inferrule* [lab=nominal] {} {\meaningof{\quotep{E}} = \{ \quotep{P} \in \quotep{\pi} | P \in \meaningof{E} \}, \and \meaningof{\quotep{P}} = \{ \quotep{Q} \in \quotep{\pi} | P \equiv Q \} \and \\ \meaningof{@\quotep{E}} = \{ P \in \pi | P \equiv @x, x \in \meaningof{E} \}}
\end{mathpar}

\begin{eqnarray*}
  \\
  \meaningof{-} : TS \to ST
\end{eqnarray*}

\begin{eqnarray*}
  \\
  L : TS \to ST
\end{eqnarray*}

\begin{eqnarray*}
  \\
  P \models E \iff P \in \meaningof{E}
\end{eqnarray*}

\begin{eqnarray*}
  P \approx_{L} Q \iff \forall E \in L. P \models E \iff Q \models E
\end{eqnarray*}

\begin{eqnarray*}
  P \approx_{K} Q
\end{eqnarray*}

\begin{eqnarray*}
  P \approx Q
\end{eqnarray*}

$\approx_{K} = \approx = \approx_{L}$

\subsubsection{Contextual duality}

Note that contexts extend the quotation operation to a family of
operations from processes to names. Given a context, $M$, we can
define a \emph{nominal context}, $\quotep{M}$ by $\quotep{M}[P] :=
\quotep{M[P]}$. To foreshadow what is to come we observe that these
operations enjoy a duality with processes very much like the duality
between vectors and maps from vectors to scalars.

Further, because the calculus is essentially higher-order, we have a
correspondence between contexts and processes. More specifically,
given a name $x$ and a context $M$ we can construct $M^{*}_{x}$ such
that 

\begin{mathpar}
  M^{*}_{x} | \lift{x}{P} \red M[P]
\end{mathpar}

namely,

\begin{mathpar}
  M^{*}_{x} := x?(u).M[\dropn{u}]
\end{mathpar}

The dependence of $M^{*}_{x}$ on a name makes it an abstraction, 

\begin{mathpar}
  M^{*} := (x)x?(u).M[\dropn{u}]
\end{mathpar}

\subsection{Additional notation}

It will sometimes be convenient to denote the process a name
quotes. We already have the notation $x = \quotep{P}$, but it will be
convenient to introduce an alternate notation, $\procn{x}$, when we
want to emphasize the connection to the use of the name. Note that, by
virtue of name equivalence, $\quotep{\procn{x}} \nameeq x$; so, the
notation is consistent with previous definitions.

Further, because names have structure it is possible to effect
substitutions on the basis of that structure. This means we need to
upgrade our notation for substitutions, which we accomplish by
adapting comprehension notation. Thus,

\begin{mathpar}
  P\{ y / x : x \in S \}
\end{mathpar}

is interpreted to mean the process derived from P by replacing (in a
capture-avoiding manner) each occurrence of $x$ in $S$ by $y$. For example,

\begin{mathpar}
  P\{ \quotep{\procn{x}|\procn{x}} / x : x \in \freenames{P} \}
\end{mathpar}

will replace each (occurrence) of a free name $x$ in $P$ by
$\quotep{\procn{x}|\procn{x}}$.

Also, we will avail ourselves of the notation $x^{L}$ and $x^{R}$ to
denote injections of a name into disjoint copies of the name
space. There are numerous ways to accomplish this. One example can be
found in \cite{MeredithR05}. This notation overloads to vectors of
names: $\vec{x}^{\pi} := (x_{i}^{\pi} \; : \; 0 \leq i < |\vec{x}| )$ where $\pi \in \{L,R\}$.

We also use $P^{\Box} := P|\Box$.

In \cite{MeredithR05} an interpretation of the new operator is
given. It turns out that there are several possible interpretations
all enjoying the requisite algebraic properties of the operator (see
\cite{milner91polyadicpi}). We will therefore make liberal use of
$(\nu\; \vec{x})P$.

% subsection the_syntax_and_semantics_of_the_notation_system (end)   

\input{qm2pi.qmops} 

\input{qm2pi.sterngerlach} 

\input{qm2pi.metric} 

% section concurrent_process_calculi (end)

%\input{qm2pi.proofsketch}

% section proof sketch (end)

%\input{qm2pi.slviaknots} 

% section spatial logic via knots (end)

\input{qm2pi.conclusion}

% section conclusion (end)

%\input{qm2pi.dtcodes} 

% section wiring algorithm (end)

\input{qm2pi.ack} 

% section acknowledgments (end)

\newpage


\bibliographystyle{plain}   
\bibliography{../../biblios/main.bib}

\input{qm2pi.rhodetails}

\end{document}

 

% section acknowledgments (end)

\newpage


\bibliographystyle{plain}   
\bibliography{../../biblios/main.bib}

\documentclass[12pt]{llncs}
%\documentclass{jktr}

\usepackage[pdftex]{hyperref}                   
\usepackage {listings}
\usepackage {mathpartir}
\usepackage{bcprules}
%\usepackage{listings}
                       
\usepackage{graphicx} 
%\usepackage[margins=2.5cm,nohead,nofoot]{geometry}
%\usepackage{geometry}
\usepackage{amsfonts}
\usepackage{amstext}
\usepackage{latexsym}
\usepackage{amssymb}
\usepackage{color}


%\include{myPreamble}
\include{qm2pi.local} 

%\ifpdf
%\usepackage[pdftex]{graphicx}
%\else
%\usepackage{graphicx}
%\fi

 % \ifpdf
%  \usepackage{pdfsync}
%  \if


%\title{Brief Article}
%\author{David F. Snyder}
%\author{L.G. Meredith}

%\address{Dept. of Math., Texas State University--San Marcos, San Marcos, TX 78666}
       
\pagestyle{empty}


\begin{document}

\lstset{language=[Objective]Caml,frame=shadowbox}

\input{qm2pi.front}

% section front matter (end)

\input{qm2pi.intro} 
 
% section introduction (end)

% \input{qm2pi.knotations} 

% section notation (end)

\input{qm2pi.process.calculi} 

% section concurrent_process_calculi_and_spatial_logics_ (end)
    
%\input{qm2pi.knots2pi} 

%\input{qm2pi.trefoil} 

%\input{qm2pi.mainthm} 

% subsection basic_interpretation (end)

%\input{qm2pi.rho.presentation} 
\subsection{The syntax and semantics of the notation system}\label{sub:the_syntax_and_semantics_of_the_notation_system} % (fold)

We now summarize a technical presentation of the calculus that
embodies our theory of dynamics. The typical presentation of such a
calculus follows the style of giving generators and relations on
them. The grammar, below, describing term constructors, freely
generates the set of processes, $\Proc$. This set is then quotiented
by a relation known as structural congruence and it is over this set
that the notion of dynamics is expressed. This presentation is
essentially that of \cite{MeredithR05} with the addition of
polyadicity and summation. For readability we have relegated some of
the technical subtleties to an appendix.

\subsubsection{Process grammar}\label{subsub:process_grammar}

\begin{mathpar}
  \inferrule* [lab=synchronization] {} {{M} \bc \pzero \;|\; x?F \;|\; x!C }
  \and
  \inferrule* [lab=abstraction] {} {{F} \bc (x)P}
  \and
  \inferrule* [lab=concretion] {} {{C} \bc \langle Q \rangle}
  \and
  \inferrule* [lab=process] {} {{P,Q} \bc M \;| \;P|Q \;|\; @{x}}
  \and
  \inferrule* [lab=name] {} {{x} \bc \quotep{P}}
\end{mathpar} 

Note that $\vec{x}$ (resp. $\vec{P}$) denotes a vector of names
(resp. processes) of length $|\vec{x}|$ (resp. $|\vec{P}|$). We adopt
the following useful abbreviations.

\begin{mathpar}
   x?(\vec{y}).P := x.(\vec{y})P \and  x\clift{\vec{P}} := x.\clift{\vec{P}}
   \and x!(y) := \lift{x}{\dropn{y}}
   \and \Pi_{i=0}^{n-1}P_i := P_0 | \ldots | P_{n-1}
\end{mathpar}

\subsubsection{Structural congruence}

\paragraph{Free and bound names and alpha-equivalence.} At the
core of structural equivalence is alpha-equivalence which identifies
process that are the same up to a change of variable. Formally, we
recognize the distinction between free and bound names. The free names
of a process, $\freenames{P}$, may be calculated recursively as
follows:

\begin{mathpar}
\freenames{\pzero} := \emptyset
  \and \\
  \freenames{x?(y).P} := \{ x \} \cup (\freenames{P} \setminus \{ y \})
  \and 
  \freenames{x!\langle P \rangle} := \{ x \} \cup \{ P \} 
  \and \\
  \freenames{P|Q} := \freenames{P} \cup \freenames{Q}
  \and \\
  \freenames{@{x}} := \{ x \}
\end{mathpar}

$\pi$
$\quotep{\pi}$

$\freenames{-} : \pi \to \mathcal{P}(\quotep{\pi})$

\begin{eqnarray*}
  \freenames{\pzero} & := & \emptyset \\
  \freenames{x?(y).P} & := & \{ x \} \cup (\freenames{P} \setminus \{ y \}) \\
  \freenames{x!\langle P \rangle} & := & \{ x \} \cup \{ P \} \\
  \freenames{P|Q} & := & \freenames{P} \cup \freenames{Q} \\
  \freenames{\dropn{x}} & := & \{ x \}
\end{eqnarray*}

The bound names of a process, $\boundnames{P}$, are those names occurring in $P$
that are not free. For example, in $x?(y).0$, the name $x$ is free, while $y$ is bound.

\begin{mathpar}
  \inferrule* [lab=monoidal-laws] {} { P|Q \equiv Q|P \and P|0 \equiv P \and P|(Q|R) \equiv (P|Q)|R }
\end{mathpar}

\begin{mathpar}
  \inferrule* [lab=alpha-equivalence] {} { (x)P \equiv (y)P\{y/x\} \and y \not\in \freenames{P} }
\end{mathpar}

\begin{definition}
Then two processes, $P,Q$, are alpha-equivalent if $P = Q\{\vec{y}/\vec{x}\}$ for
some $\vec{x} \in \boundnames{Q},\vec{y} \in \boundnames{P}$, where $Q\{\vec{y}/\vec{x}\}$
denotes the capture-avoiding substitution of $\vec{y}$ for $\vec{x}$ in $Q$.
\end{definition}

\begin{definition}
  The {\em structural congruence} \cite{SangiorgiWalker} , $\equiv$,
  between processes is the least congruence containing
  alpha-equivalence, satisfying the abelian monoid laws
  (associativity, commutativity and $\pzero$ as identity) for parallel
  composition $|$ and for summation $+$.
\end{definition}

\subsection{Name equivalence}

We take name equivalence, written $\nameeq$, to be the smallest
equivalence relation generated by the following rules.

\begin{mathpar}
\inferrule*[lab=Quote-drop]
{ }
{ \quotep{@{x}} \nameeq x }

\inferrule*[lab=Struct-equiv]
{ P \scong Q }
{ \quotep{P} \nameeq \quotep{Q} }
\end{mathpar}

The astute reader will have noticed that the mutual recursion of names
and processes imposes a mutual recursion on alpha-equivalence and
structural equivalence via name-equivalence. Fortunately, all of this
works out pleasantly and we may calculate in the natural way, free of
concern. The reader interested in the details is referred to the
appendix \ref{appendix:rho_details}.

\subsection{Substitution}

We use $\Proc$ for the set of processes, $\QProc$ for the set of
names, and $\id{\{}\vec{y} / \vec{x} \id{\}}$ to denote partial maps,
$s : \QProc \rightarrow \QProc$. A map, $s$ lifts, uniquely, to a map
on process terms, $\widehat{s} : \Proc \rightarrow \Proc$ by the
following equations.

\begin{mathpar}
  (0) \psubstp{Q}{P} := 0 \\
  (R \juxtap S) \psubstp{Q}{P}
  :=    
  (R)\psubstp{Q}{P} \juxtap (S) \psubstp{Q}{P} \\
  (x?(y).R) \psubstp{Q}{P}    
  :=    
  (x)\substp{Q}{P} (z)\concat( (R \psubstn{z}{y}) \psubstp{Q}{P} ) \\
  (\lift{x}{R}) \psubstp{Q}{P}  
  :=
  \lift{(x)\substp{Q}{P}}{ R \psubstp{Q}{P} } \\
%   (\dropn{x})  \psubstp{Q}{P}       
%   := 
%   \left\{ 
%     \begin{array}{ccc} 
%       \dropn{\quotep{Q}} & & x \nameeq \quotep{P} \\
%       \dropn{x} & & otherwise \\
%     \end{array}
%   \right. 
  (\dropn{x})  \psubstp{Q}{P}       
  := 
  \left\{ 
    \begin{array}{ccc} 
      Q & & x \nameeq \quotep{P} \\
      \dropn{x} & & otherwise \\
    \end{array}
  \right.
\end{mathpar}
 

where

\begin{eqnarray}
  (x)\id{\{} \lpquote Q \rpquote / \lpquote P \rpquote \id{\}}            = 
  \left\{ 
    \begin{array}{ccc}
      \lpquote Q \rpquote & & x \nameeq \lpquote P \rpquote \\
      x & & otherwise \\
    \end{array}
  \right. \nonumber
\end{eqnarray}

and $z$ is chosen distinct from $\quotep{P}$, $\quotep{Q}$, the free
names in $Q$, and all the names in $R$. Our $\alpha$-equivalence will
be built in the standard way from this substitution.

\begin{remark}\label{rem:no_self_referential_names}
  One consequence of these definitions is that $\forall P. \quotep{P}
  \not\in \freenames{P}$.
\end{remark}

\subsection{ Dynamic quote: an example }

Anticipating something of what's to come, consider applying the
substitution, $\widehat{\id{\{}u / z \id{\}}}$, to the following pair
of processes, $\lift{w}{y!(z)}$ and $w[ \lpquote y!(z) \rpquote ]$.

\begin{eqnarray}
	\lift{w}{y!(z)}\widehat{\id{\{}u / z \id{\}}}
		& = &
		\lift{w}{y!(u)} \nonumber\\
	w[ \lpquote y!(z) \rpquote ] \widehat{ \id{\{}u / z \id{\}} }
		& = &
		w[ \lpquote y!(z) \rpquote ] \nonumber
\end{eqnarray}

Because the body of the process between quotes is impervious to
substitution, we get radically different answers. In fact, by
examining the first process in an input context,
e.g. $x?(z).\lift{w}{y!(z)}$, we see that the process under the lift
operator may be shaped by prefixed inputs binding a name inside it. In
this sense, the lift operator will be seen as a way to dynamically
construct processes before reifying them as names.

Finally equipped with these standard features we can present the
dynamics of the calculus.

\subsubsection{Operational semantics} 

Finally, we introduce the computational dynamics. What marks these
algebras as distinct from other more traditionally studied algebraic
structures, e.g. vector spaces or polynomial rings, is the manner in
which dynamics is captured. In traditional structures, dynamics is typically
expressed through morphisms between such structures, as in linear maps
between vector spaces or morphisms between rings. In algebras
associated with the semantics of computation, the dynamics is
expressed as part of the algebraic structure itself, through a
reduction reduction relation typically denoted by $\red$. Below, we
give a recursive presentation of this relation for the calculus used
in the encoding.

$\red \subseteq \pi \times \pi$
$\red : \pi \to \mathcal{P}(\pi)$

\begin{mathpar}
  \inferrule* [lab=Comm] { \textsf{match}( x_{src}, x_{trgt} ) } { x_{trgt}?(y)P \; | \; x_{src}!\langle {Q} \rangle \red P\{\quotep{Q}/y}\} }
  \and \\
  \inferrule* [lab=Par] {{P} \red {P}'} {{{P} | {Q}} \red {{P}' | {Q}}}
  \and
  \inferrule* [lab=Equiv]{{{P} \scong {P}'} \andalso {{P}' \red {Q}'} \andalso {{Q}' \scong {Q}}}{{P} \red {Q}}
\end{mathpar}

\begin{eqnarray*}
  match_{\equiv} (\quotep{P},\quotep{Q}) & := & P \equiv Q \\
  match_{\dagger}(\quotep{P},\quotep{Q}) & := & \forall R. P|Q \red^{*} R => R \red^{*} 0 \\
  match_{K}(\quotep{P},\quotep{Q}) & := & K \mbox{ for some context } K
\end{eqnarray*}

$u?(x)P | u!\langle Q \rangle \red P\{\quotep{Q}/x\}$

%We write $\wred$ for $\red^*$, and $P\red$ if $\exists Q $ such that $ P \red Q$.
We write $P\red$ if $\exists Q $ such that $ P \red Q$ and $P\not\red$, otherwise.

\section{Replication}

As mentioned before, it is known that replication (and hence
recursion) can be implemented in a higher-order process algebra
\cite{SangiorgiWalker}. As our first example of calculation with the
machinery thus far presented we give the construction explicitly in
the {\rhoc}.

\begin{eqnarray}
	D_{x} & := & \prefix{x}{y}{(\binpar{\outputp{x}{y}}{@{y}})} \nonumber\\
	\bangp_{x}{P} & := & \binpar{{x}!\langle{\binpar{D_{x}}{P}}\rangle}{D_{x}} \nonumber
\end{eqnarray}

\begin{eqnarray}
	\bangp_{x}{P} & & \nonumber\\
	=
	& {x}!\langle{(\prefix{x}{y}{(\outputp{x}{y} | @{y})) | P}}\rangle 
	      | \prefix{x}{y}{(\outputp{x}{y} | @{y})} & \nonumber\\
	\red
	& (\outputp{x}{y} | @{y})\substn{\quotep{(\prefix{x}{y}{(@{y} | \outputp{x}{y})) | P}}}{y} & \nonumber\\
	=
	& \outputp{x}{\quotep{(\prefix{x}{y}{(\outputp{x}{y} | @{y})) | P}}}
	  | {(\prefix{x}{y}{(\outputp{x}{y} | @{y})) | P}} & \nonumber\\
	\red
	& \ldots & \nonumber\\
	\red^*
	& P | P | \ldots & \nonumber
\end{eqnarray}

Of course, this encoding, as an implementation, runs away, unfolding
$\bangp{P}$ eagerly. A lazier and more implementable replication
operator, restricted to input-guarded processes, may be obtained as follows.

\begin{eqnarray}
\bangp{\prefix{u}{v}{P}} 
	:= 
	\binpar{\lift{x}{\prefix{u}{v}{(\binpar{D(x)}{P})}}}{D(x)} \nonumber
\end{eqnarray}

\begin{remark}
  Note that the lazier definition still does not deal with summation
  or mixed summation (i.e. sums over input and output). The reader is
  invited to construct definitions of replication that deal with these
  features. 

  Further, the definitions are parameterized in a name, $x$. Can you,
  gentle reader, make a definition that eliminates this parameter and
  guarantees no accidental interaction between the replication
  machinery and the process being replicated -- i.e. no accidental
  sharing of names used by the process to get its work done and the
  name(s) used by the replication to effect copying. This latter
  revision of the definition of replication is crucial to obtaining
  the expected identity $!!P \sim !P$.
\end{remark}

\begin{remark}\label{rem:paradoxical_combinator}
  The reader familiar with the lambda calculus will have noticed the
  similarity between $D$ and the paradoxical combinator.

  [Ed. note: the existence of this seems to suggest we have to be more
  restrictive on the set of processes and names we admit if we are to
  support no-cloning.]
\end{remark}

\subsubsection{Bisimulation}

The computational dynamics gives rise to another kind of equivalence,
the equivalence of computational behavior. As previously mentioned
this is typically captured \emph{via} some form of bisimulation.

% The notion we use in this paper is weak barbed bisimulation
% \cite{milner91polyadicpi}.

The notion we use in this paper is derived from weak barbed
bisimulation \cite{milner91polyadicpi}. 

\begin{definition}
An \emph{observation relation}, $\downarrow_{\mathcal N}$, over a set
of names, $\mathcal N$, is the smallest relation satisfying the rules
below.

\infrule[Out-barb]{y \in {\mathcal N}, \; x \nameeq y}
		  {\outputp{x}{v} \downarrow_{\mathcal N} x}
\infrule[Par-barb]{\mbox{$P\downarrow_{\mathcal N} x$ or $Q\downarrow_{\mathcal N} x$}}
		  {\binpar{P}{Q} \downarrow_{\mathcal N} x}

We write $P \Downarrow_{\mathcal N} x$ if there is $Q$ such that 
$P \wred Q$ and $Q \downarrow_{\mathcal N} x$.
\end{definition}

\begin{definition}
%\label{def.bbisim}
An  ${\mathcal N}$-\emph{barbed bisimulation} over a set of names, ${\mathcal N}$, is a symmetric binary relation 
${\mathcal S}_{\mathcal N}$ between agents such that $P\rel{S}_{\mathcal N}Q$ implies:
\begin{enumerate}
\item If $P \red P'$ then $Q \wred Q'$ and $P'\rel{S}_{\mathcal N} Q'$.
\item If $P\downarrow_{\mathcal N} x$, then $Q\Downarrow_{\mathcal N} x$.
\end{enumerate}
$P$ is ${\mathcal N}$-barbed bisimilar to $Q$, written
$P \wbbisim_{\mathcal N} Q$, if $P \rel{S}_{\mathcal N} Q$ for some ${\mathcal N}$-barbed bisimulation ${\mathcal S}_{\mathcal N}$.
\end{definition}

$\mathcal{R} \subseteq \pi \times \pi$

$P \mathcal{R} Q => \forall P'. P \red P' \Rightarrow \exists Q'. Q \red Q', P' \mathcal{R} Q'$

$P \vdash x \Rightarrow Q \vdash x$

\begin{mathpar}
  \inferrule*[lab=Out-barb]{x \nameeq y}{{y}!\langle{Q}\rangle \vdash x}
  \and
  \inferrule*[lab=Par-barb]{\mbox{$P\vdash x$ or $Q\vdash x$}}{\binpar{P}{Q} \vdash x}
\end{mathpar}

\subsubsection{Contexts}

One of the principle advantages of computational calculi like the
$\pi$-calculus is a well-defined notion of context,
contextual-equivalence and a correlation between
contextual-equivalence and notions of bisimulation. The notion of
context allows the decomposition of a process into (sub-)process and
its syntactic environment, its context. Thus, a context may be
thought of as a process with a ``hole'' (written $\Box$) in it. The
application of a context $M$ to a process $P$, written $M[P]$, is
tantamount to filling the hole in $M$ with $P$. In this paper we do
not need the full weight of this theory, but do make use of the notion
of context in the proof the main theorem. 

\begin{mathpar}
  \inferrule* [lab=summation] {} {{M_{M},M_{N}} \bc \Box \;|\; x.M_{A} \;|\; M_{M}+M_{N}}
  \and
  \inferrule* [lab=agent] {} {{M_{A}} \bc (\vec{x})M_{P} \;| \; \clift{P_0,\ldots,M_{P},\ldots,P_N}}
  \and \\
  \inferrule* [lab=process] {} {{M_{P}} \bc M_{N} \;| \;P|M_{P} }
\end{mathpar} 

\begin{mathpar}
  \inferrule* [lab=sychronization] {} {M_{N} \bc \Box \;|\; x?M_{F} \;|\; x!M_{C}}
  \and
  \inferrule* [lab=abstraction] {} {{M_{F}} \bc (x)M_{P} }
  \and
  \inferrule* [lab=concretion] {} {{M_{C}} \bc \langle M_{P} \rangle }
  \and \\
  \inferrule* [lab=process] {} {{M_{P}} \bc M_{N} \;| \;P|M_{P} }
\end{mathpar}

\begin{definition}[contextual application] Given a context $M$, and
  process $P$, we define the \emph{contextual application}, $M[P] :=
  M\{P/\Box\}$. That is, the contextual application of M to P is the
  substitution of $P$ for $\Box$ in $M$.
\end{definition}

$\meaningof{-} : L \to \mathcal{P}(\pi)$

\begin{mathpar}
  \inferrule* [lab=collection] {} {\meaningof{true} = \pi, \and \meaningof{~E} = \pi \setminus \meaningof{E}, \and \meaningof{E_{1} \& E_{2}} = \meaningof{E_{1}} \cap \meaningof{E_{2}}}
\end{mathpar}

\begin{mathpar}
  \inferrule* [lab=structure] {} {\meaningof{0} = \{ P \in \pi | P \equiv 0 \}, \and \\ \meaningof{E_1 | E_2} = \{ P \in \pi | P \equiv P_{1} | P_{2}, P_{1} \in \meaningof{E_{1}}, P_{2} \in \meaningof{E_2}\} }
\end{mathpar}

\begin{mathpar}
 \inferrule* [lab=behavior] {} {\meaningof{\langle a?b \rangle E} = \{ P \in \pi | P \equiv Q | u?(y)P', \\ \and \\\\ \and \\ \;\;\; u \in \meaningof{a}, \forall z.P'\{z/y\} \in \meaningof{E\{z/b\}}\}, \and \\ \meaningof{a!E} = \{ P \in \pi | P \equiv Q | x!\langle P' \rangle, x \in \meaningof{a} P' \in \meaningof{E}\} }
\end{mathpar}

\begin{mathpar}
 \inferrule* [lab=nominal] {} {\meaningof{\quotep{E}} = \{ \quotep{P} \in \quotep{\pi} | P \in \meaningof{E} \}, \and \meaningof{\quotep{P}} = \{ \quotep{Q} \in \quotep{\pi} | P \equiv Q \} \and \\ \meaningof{@\quotep{E}} = \{ P \in \pi | P \equiv @x, x \in \meaningof{E} \}}
\end{mathpar}

\begin{eqnarray*}
  \\
  \meaningof{-} : TS \to ST
\end{eqnarray*}

\begin{eqnarray*}
  \\
  L : TS \to ST
\end{eqnarray*}

\begin{eqnarray*}
  \\
  P \models E \iff P \in \meaningof{E}
\end{eqnarray*}

\begin{eqnarray*}
  P \approx_{L} Q \iff \forall E \in L. P \models E \iff Q \models E
\end{eqnarray*}

\begin{eqnarray*}
  P \approx_{K} Q
\end{eqnarray*}

\begin{eqnarray*}
  P \approx Q
\end{eqnarray*}

$\approx_{K} = \approx = \approx_{L}$

\subsubsection{Contextual duality}

Note that contexts extend the quotation operation to a family of
operations from processes to names. Given a context, $M$, we can
define a \emph{nominal context}, $\quotep{M}$ by $\quotep{M}[P] :=
\quotep{M[P]}$. To foreshadow what is to come we observe that these
operations enjoy a duality with processes very much like the duality
between vectors and maps from vectors to scalars.

Further, because the calculus is essentially higher-order, we have a
correspondence between contexts and processes. More specifically,
given a name $x$ and a context $M$ we can construct $M^{*}_{x}$ such
that 

\begin{mathpar}
  M^{*}_{x} | \lift{x}{P} \red M[P]
\end{mathpar}

namely,

\begin{mathpar}
  M^{*}_{x} := x?(u).M[\dropn{u}]
\end{mathpar}

The dependence of $M^{*}_{x}$ on a name makes it an abstraction, 

\begin{mathpar}
  M^{*} := (x)x?(u).M[\dropn{u}]
\end{mathpar}

\subsection{Additional notation}

It will sometimes be convenient to denote the process a name
quotes. We already have the notation $x = \quotep{P}$, but it will be
convenient to introduce an alternate notation, $\procn{x}$, when we
want to emphasize the connection to the use of the name. Note that, by
virtue of name equivalence, $\quotep{\procn{x}} \nameeq x$; so, the
notation is consistent with previous definitions.

Further, because names have structure it is possible to effect
substitutions on the basis of that structure. This means we need to
upgrade our notation for substitutions, which we accomplish by
adapting comprehension notation. Thus,

\begin{mathpar}
  P\{ y / x : x \in S \}
\end{mathpar}

is interpreted to mean the process derived from P by replacing (in a
capture-avoiding manner) each occurrence of $x$ in $S$ by $y$. For example,

\begin{mathpar}
  P\{ \quotep{\procn{x}|\procn{x}} / x : x \in \freenames{P} \}
\end{mathpar}

will replace each (occurrence) of a free name $x$ in $P$ by
$\quotep{\procn{x}|\procn{x}}$.

Also, we will avail ourselves of the notation $x^{L}$ and $x^{R}$ to
denote injections of a name into disjoint copies of the name
space. There are numerous ways to accomplish this. One example can be
found in \cite{MeredithR05}. This notation overloads to vectors of
names: $\vec{x}^{\pi} := (x_{i}^{\pi} \; : \; 0 \leq i < |\vec{x}| )$ where $\pi \in \{L,R\}$.

We also use $P^{\Box} := P|\Box$.

In \cite{MeredithR05} an interpretation of the new operator is
given. It turns out that there are several possible interpretations
all enjoying the requisite algebraic properties of the operator (see
\cite{milner91polyadicpi}). We will therefore make liberal use of
$(\nu\; \vec{x})P$.

% subsection the_syntax_and_semantics_of_the_notation_system (end)   

\input{qm2pi.qmops} 

\input{qm2pi.sterngerlach} 

\input{qm2pi.metric} 

% section concurrent_process_calculi (end)

%\input{qm2pi.proofsketch}

% section proof sketch (end)

%\input{qm2pi.slviaknots} 

% section spatial logic via knots (end)

\input{qm2pi.conclusion}

% section conclusion (end)

%\input{qm2pi.dtcodes} 

% section wiring algorithm (end)

\input{qm2pi.ack} 

% section acknowledgments (end)

\newpage


\bibliographystyle{plain}   
\bibliography{../../biblios/main.bib}

\input{qm2pi.rhodetails}

\end{document}



\end{document}

 

\documentclass[12pt]{llncs}
%\documentclass{jktr}

\usepackage[pdftex]{hyperref}                   
\usepackage {listings}
\usepackage {mathpartir}
\usepackage{bcprules}
%\usepackage{listings}
                       
\usepackage{graphicx} 
%\usepackage[margins=2.5cm,nohead,nofoot]{geometry}
%\usepackage{geometry}
\usepackage{amsfonts}
\usepackage{amstext}
\usepackage{latexsym}
\usepackage{amssymb}
\usepackage{color}


%\include{myPreamble}
\documentclass[12pt]{llncs}
%\documentclass{jktr}

\usepackage[pdftex]{hyperref}                   
\usepackage {listings}
\usepackage {mathpartir}
\usepackage{bcprules}
%\usepackage{listings}
                       
\usepackage{graphicx} 
%\usepackage[margins=2.5cm,nohead,nofoot]{geometry}
%\usepackage{geometry}
\usepackage{amsfonts}
\usepackage{amstext}
\usepackage{latexsym}
\usepackage{amssymb}
\usepackage{color}


%\include{myPreamble}
\include{qm2pi.local} 

%\ifpdf
%\usepackage[pdftex]{graphicx}
%\else
%\usepackage{graphicx}
%\fi

 % \ifpdf
%  \usepackage{pdfsync}
%  \if


%\title{Brief Article}
%\author{David F. Snyder}
%\author{L.G. Meredith}

%\address{Dept. of Math., Texas State University--San Marcos, San Marcos, TX 78666}
       
\pagestyle{empty}


\begin{document}

\lstset{language=[Objective]Caml,frame=shadowbox}

\input{qm2pi.front}

% section front matter (end)

\input{qm2pi.intro} 
 
% section introduction (end)

% \input{qm2pi.knotations} 

% section notation (end)

\input{qm2pi.process.calculi} 

% section concurrent_process_calculi_and_spatial_logics_ (end)
    
%\input{qm2pi.knots2pi} 

%\input{qm2pi.trefoil} 

%\input{qm2pi.mainthm} 

% subsection basic_interpretation (end)

%\input{qm2pi.rho.presentation} 
\subsection{The syntax and semantics of the notation system}\label{sub:the_syntax_and_semantics_of_the_notation_system} % (fold)

We now summarize a technical presentation of the calculus that
embodies our theory of dynamics. The typical presentation of such a
calculus follows the style of giving generators and relations on
them. The grammar, below, describing term constructors, freely
generates the set of processes, $\Proc$. This set is then quotiented
by a relation known as structural congruence and it is over this set
that the notion of dynamics is expressed. This presentation is
essentially that of \cite{MeredithR05} with the addition of
polyadicity and summation. For readability we have relegated some of
the technical subtleties to an appendix.

\subsubsection{Process grammar}\label{subsub:process_grammar}

\begin{mathpar}
  \inferrule* [lab=synchronization] {} {{M} \bc \pzero \;|\; x?F \;|\; x!C }
  \and
  \inferrule* [lab=abstraction] {} {{F} \bc (x)P}
  \and
  \inferrule* [lab=concretion] {} {{C} \bc \langle Q \rangle}
  \and
  \inferrule* [lab=process] {} {{P,Q} \bc M \;| \;P|Q \;|\; @{x}}
  \and
  \inferrule* [lab=name] {} {{x} \bc \quotep{P}}
\end{mathpar} 

Note that $\vec{x}$ (resp. $\vec{P}$) denotes a vector of names
(resp. processes) of length $|\vec{x}|$ (resp. $|\vec{P}|$). We adopt
the following useful abbreviations.

\begin{mathpar}
   x?(\vec{y}).P := x.(\vec{y})P \and  x\clift{\vec{P}} := x.\clift{\vec{P}}
   \and x!(y) := \lift{x}{\dropn{y}}
   \and \Pi_{i=0}^{n-1}P_i := P_0 | \ldots | P_{n-1}
\end{mathpar}

\subsubsection{Structural congruence}

\paragraph{Free and bound names and alpha-equivalence.} At the
core of structural equivalence is alpha-equivalence which identifies
process that are the same up to a change of variable. Formally, we
recognize the distinction between free and bound names. The free names
of a process, $\freenames{P}$, may be calculated recursively as
follows:

\begin{mathpar}
\freenames{\pzero} := \emptyset
  \and \\
  \freenames{x?(y).P} := \{ x \} \cup (\freenames{P} \setminus \{ y \})
  \and 
  \freenames{x!\langle P \rangle} := \{ x \} \cup \{ P \} 
  \and \\
  \freenames{P|Q} := \freenames{P} \cup \freenames{Q}
  \and \\
  \freenames{@{x}} := \{ x \}
\end{mathpar}

$\pi$
$\quotep{\pi}$

$\freenames{-} : \pi \to \mathcal{P}(\quotep{\pi})$

\begin{eqnarray*}
  \freenames{\pzero} & := & \emptyset \\
  \freenames{x?(y).P} & := & \{ x \} \cup (\freenames{P} \setminus \{ y \}) \\
  \freenames{x!\langle P \rangle} & := & \{ x \} \cup \{ P \} \\
  \freenames{P|Q} & := & \freenames{P} \cup \freenames{Q} \\
  \freenames{\dropn{x}} & := & \{ x \}
\end{eqnarray*}

The bound names of a process, $\boundnames{P}$, are those names occurring in $P$
that are not free. For example, in $x?(y).0$, the name $x$ is free, while $y$ is bound.

\begin{mathpar}
  \inferrule* [lab=monoidal-laws] {} { P|Q \equiv Q|P \and P|0 \equiv P \and P|(Q|R) \equiv (P|Q)|R }
\end{mathpar}

\begin{mathpar}
  \inferrule* [lab=alpha-equivalence] {} { (x)P \equiv (y)P\{y/x\} \and y \not\in \freenames{P} }
\end{mathpar}

\begin{definition}
Then two processes, $P,Q$, are alpha-equivalent if $P = Q\{\vec{y}/\vec{x}\}$ for
some $\vec{x} \in \boundnames{Q},\vec{y} \in \boundnames{P}$, where $Q\{\vec{y}/\vec{x}\}$
denotes the capture-avoiding substitution of $\vec{y}$ for $\vec{x}$ in $Q$.
\end{definition}

\begin{definition}
  The {\em structural congruence} \cite{SangiorgiWalker} , $\equiv$,
  between processes is the least congruence containing
  alpha-equivalence, satisfying the abelian monoid laws
  (associativity, commutativity and $\pzero$ as identity) for parallel
  composition $|$ and for summation $+$.
\end{definition}

\subsection{Name equivalence}

We take name equivalence, written $\nameeq$, to be the smallest
equivalence relation generated by the following rules.

\begin{mathpar}
\inferrule*[lab=Quote-drop]
{ }
{ \quotep{@{x}} \nameeq x }

\inferrule*[lab=Struct-equiv]
{ P \scong Q }
{ \quotep{P} \nameeq \quotep{Q} }
\end{mathpar}

The astute reader will have noticed that the mutual recursion of names
and processes imposes a mutual recursion on alpha-equivalence and
structural equivalence via name-equivalence. Fortunately, all of this
works out pleasantly and we may calculate in the natural way, free of
concern. The reader interested in the details is referred to the
appendix \ref{appendix:rho_details}.

\subsection{Substitution}

We use $\Proc$ for the set of processes, $\QProc$ for the set of
names, and $\id{\{}\vec{y} / \vec{x} \id{\}}$ to denote partial maps,
$s : \QProc \rightarrow \QProc$. A map, $s$ lifts, uniquely, to a map
on process terms, $\widehat{s} : \Proc \rightarrow \Proc$ by the
following equations.

\begin{mathpar}
  (0) \psubstp{Q}{P} := 0 \\
  (R \juxtap S) \psubstp{Q}{P}
  :=    
  (R)\psubstp{Q}{P} \juxtap (S) \psubstp{Q}{P} \\
  (x?(y).R) \psubstp{Q}{P}    
  :=    
  (x)\substp{Q}{P} (z)\concat( (R \psubstn{z}{y}) \psubstp{Q}{P} ) \\
  (\lift{x}{R}) \psubstp{Q}{P}  
  :=
  \lift{(x)\substp{Q}{P}}{ R \psubstp{Q}{P} } \\
%   (\dropn{x})  \psubstp{Q}{P}       
%   := 
%   \left\{ 
%     \begin{array}{ccc} 
%       \dropn{\quotep{Q}} & & x \nameeq \quotep{P} \\
%       \dropn{x} & & otherwise \\
%     \end{array}
%   \right. 
  (\dropn{x})  \psubstp{Q}{P}       
  := 
  \left\{ 
    \begin{array}{ccc} 
      Q & & x \nameeq \quotep{P} \\
      \dropn{x} & & otherwise \\
    \end{array}
  \right.
\end{mathpar}
 

where

\begin{eqnarray}
  (x)\id{\{} \lpquote Q \rpquote / \lpquote P \rpquote \id{\}}            = 
  \left\{ 
    \begin{array}{ccc}
      \lpquote Q \rpquote & & x \nameeq \lpquote P \rpquote \\
      x & & otherwise \\
    \end{array}
  \right. \nonumber
\end{eqnarray}

and $z$ is chosen distinct from $\quotep{P}$, $\quotep{Q}$, the free
names in $Q$, and all the names in $R$. Our $\alpha$-equivalence will
be built in the standard way from this substitution.

\begin{remark}\label{rem:no_self_referential_names}
  One consequence of these definitions is that $\forall P. \quotep{P}
  \not\in \freenames{P}$.
\end{remark}

\subsection{ Dynamic quote: an example }

Anticipating something of what's to come, consider applying the
substitution, $\widehat{\id{\{}u / z \id{\}}}$, to the following pair
of processes, $\lift{w}{y!(z)}$ and $w[ \lpquote y!(z) \rpquote ]$.

\begin{eqnarray}
	\lift{w}{y!(z)}\widehat{\id{\{}u / z \id{\}}}
		& = &
		\lift{w}{y!(u)} \nonumber\\
	w[ \lpquote y!(z) \rpquote ] \widehat{ \id{\{}u / z \id{\}} }
		& = &
		w[ \lpquote y!(z) \rpquote ] \nonumber
\end{eqnarray}

Because the body of the process between quotes is impervious to
substitution, we get radically different answers. In fact, by
examining the first process in an input context,
e.g. $x?(z).\lift{w}{y!(z)}$, we see that the process under the lift
operator may be shaped by prefixed inputs binding a name inside it. In
this sense, the lift operator will be seen as a way to dynamically
construct processes before reifying them as names.

Finally equipped with these standard features we can present the
dynamics of the calculus.

\subsubsection{Operational semantics} 

Finally, we introduce the computational dynamics. What marks these
algebras as distinct from other more traditionally studied algebraic
structures, e.g. vector spaces or polynomial rings, is the manner in
which dynamics is captured. In traditional structures, dynamics is typically
expressed through morphisms between such structures, as in linear maps
between vector spaces or morphisms between rings. In algebras
associated with the semantics of computation, the dynamics is
expressed as part of the algebraic structure itself, through a
reduction reduction relation typically denoted by $\red$. Below, we
give a recursive presentation of this relation for the calculus used
in the encoding.

$\red \subseteq \pi \times \pi$
$\red : \pi \to \mathcal{P}(\pi)$

\begin{mathpar}
  \inferrule* [lab=Comm] { \textsf{match}( x_{src}, x_{trgt} ) } { x_{trgt}?(y)P \; | \; x_{src}!\langle {Q} \rangle \red P\{\quotep{Q}/y}\} }
  \and \\
  \inferrule* [lab=Par] {{P} \red {P}'} {{{P} | {Q}} \red {{P}' | {Q}}}
  \and
  \inferrule* [lab=Equiv]{{{P} \scong {P}'} \andalso {{P}' \red {Q}'} \andalso {{Q}' \scong {Q}}}{{P} \red {Q}}
\end{mathpar}

\begin{eqnarray*}
  match_{\equiv} (\quotep{P},\quotep{Q}) & := & P \equiv Q \\
  match_{\dagger}(\quotep{P},\quotep{Q}) & := & \forall R. P|Q \red^{*} R => R \red^{*} 0 \\
  match_{K}(\quotep{P},\quotep{Q}) & := & K \mbox{ for some context } K
\end{eqnarray*}

$u?(x)P | u!\langle Q \rangle \red P\{\quotep{Q}/x\}$

%We write $\wred$ for $\red^*$, and $P\red$ if $\exists Q $ such that $ P \red Q$.
We write $P\red$ if $\exists Q $ such that $ P \red Q$ and $P\not\red$, otherwise.

\section{Replication}

As mentioned before, it is known that replication (and hence
recursion) can be implemented in a higher-order process algebra
\cite{SangiorgiWalker}. As our first example of calculation with the
machinery thus far presented we give the construction explicitly in
the {\rhoc}.

\begin{eqnarray}
	D_{x} & := & \prefix{x}{y}{(\binpar{\outputp{x}{y}}{@{y}})} \nonumber\\
	\bangp_{x}{P} & := & \binpar{{x}!\langle{\binpar{D_{x}}{P}}\rangle}{D_{x}} \nonumber
\end{eqnarray}

\begin{eqnarray}
	\bangp_{x}{P} & & \nonumber\\
	=
	& {x}!\langle{(\prefix{x}{y}{(\outputp{x}{y} | @{y})) | P}}\rangle 
	      | \prefix{x}{y}{(\outputp{x}{y} | @{y})} & \nonumber\\
	\red
	& (\outputp{x}{y} | @{y})\substn{\quotep{(\prefix{x}{y}{(@{y} | \outputp{x}{y})) | P}}}{y} & \nonumber\\
	=
	& \outputp{x}{\quotep{(\prefix{x}{y}{(\outputp{x}{y} | @{y})) | P}}}
	  | {(\prefix{x}{y}{(\outputp{x}{y} | @{y})) | P}} & \nonumber\\
	\red
	& \ldots & \nonumber\\
	\red^*
	& P | P | \ldots & \nonumber
\end{eqnarray}

Of course, this encoding, as an implementation, runs away, unfolding
$\bangp{P}$ eagerly. A lazier and more implementable replication
operator, restricted to input-guarded processes, may be obtained as follows.

\begin{eqnarray}
\bangp{\prefix{u}{v}{P}} 
	:= 
	\binpar{\lift{x}{\prefix{u}{v}{(\binpar{D(x)}{P})}}}{D(x)} \nonumber
\end{eqnarray}

\begin{remark}
  Note that the lazier definition still does not deal with summation
  or mixed summation (i.e. sums over input and output). The reader is
  invited to construct definitions of replication that deal with these
  features. 

  Further, the definitions are parameterized in a name, $x$. Can you,
  gentle reader, make a definition that eliminates this parameter and
  guarantees no accidental interaction between the replication
  machinery and the process being replicated -- i.e. no accidental
  sharing of names used by the process to get its work done and the
  name(s) used by the replication to effect copying. This latter
  revision of the definition of replication is crucial to obtaining
  the expected identity $!!P \sim !P$.
\end{remark}

\begin{remark}\label{rem:paradoxical_combinator}
  The reader familiar with the lambda calculus will have noticed the
  similarity between $D$ and the paradoxical combinator.

  [Ed. note: the existence of this seems to suggest we have to be more
  restrictive on the set of processes and names we admit if we are to
  support no-cloning.]
\end{remark}

\subsubsection{Bisimulation}

The computational dynamics gives rise to another kind of equivalence,
the equivalence of computational behavior. As previously mentioned
this is typically captured \emph{via} some form of bisimulation.

% The notion we use in this paper is weak barbed bisimulation
% \cite{milner91polyadicpi}.

The notion we use in this paper is derived from weak barbed
bisimulation \cite{milner91polyadicpi}. 

\begin{definition}
An \emph{observation relation}, $\downarrow_{\mathcal N}$, over a set
of names, $\mathcal N$, is the smallest relation satisfying the rules
below.

\infrule[Out-barb]{y \in {\mathcal N}, \; x \nameeq y}
		  {\outputp{x}{v} \downarrow_{\mathcal N} x}
\infrule[Par-barb]{\mbox{$P\downarrow_{\mathcal N} x$ or $Q\downarrow_{\mathcal N} x$}}
		  {\binpar{P}{Q} \downarrow_{\mathcal N} x}

We write $P \Downarrow_{\mathcal N} x$ if there is $Q$ such that 
$P \wred Q$ and $Q \downarrow_{\mathcal N} x$.
\end{definition}

\begin{definition}
%\label{def.bbisim}
An  ${\mathcal N}$-\emph{barbed bisimulation} over a set of names, ${\mathcal N}$, is a symmetric binary relation 
${\mathcal S}_{\mathcal N}$ between agents such that $P\rel{S}_{\mathcal N}Q$ implies:
\begin{enumerate}
\item If $P \red P'$ then $Q \wred Q'$ and $P'\rel{S}_{\mathcal N} Q'$.
\item If $P\downarrow_{\mathcal N} x$, then $Q\Downarrow_{\mathcal N} x$.
\end{enumerate}
$P$ is ${\mathcal N}$-barbed bisimilar to $Q$, written
$P \wbbisim_{\mathcal N} Q$, if $P \rel{S}_{\mathcal N} Q$ for some ${\mathcal N}$-barbed bisimulation ${\mathcal S}_{\mathcal N}$.
\end{definition}

$\mathcal{R} \subseteq \pi \times \pi$

$P \mathcal{R} Q => \forall P'. P \red P' \Rightarrow \exists Q'. Q \red Q', P' \mathcal{R} Q'$

$P \vdash x \Rightarrow Q \vdash x$

\begin{mathpar}
  \inferrule*[lab=Out-barb]{x \nameeq y}{{y}!\langle{Q}\rangle \vdash x}
  \and
  \inferrule*[lab=Par-barb]{\mbox{$P\vdash x$ or $Q\vdash x$}}{\binpar{P}{Q} \vdash x}
\end{mathpar}

\subsubsection{Contexts}

One of the principle advantages of computational calculi like the
$\pi$-calculus is a well-defined notion of context,
contextual-equivalence and a correlation between
contextual-equivalence and notions of bisimulation. The notion of
context allows the decomposition of a process into (sub-)process and
its syntactic environment, its context. Thus, a context may be
thought of as a process with a ``hole'' (written $\Box$) in it. The
application of a context $M$ to a process $P$, written $M[P]$, is
tantamount to filling the hole in $M$ with $P$. In this paper we do
not need the full weight of this theory, but do make use of the notion
of context in the proof the main theorem. 

\begin{mathpar}
  \inferrule* [lab=summation] {} {{M_{M},M_{N}} \bc \Box \;|\; x.M_{A} \;|\; M_{M}+M_{N}}
  \and
  \inferrule* [lab=agent] {} {{M_{A}} \bc (\vec{x})M_{P} \;| \; \clift{P_0,\ldots,M_{P},\ldots,P_N}}
  \and \\
  \inferrule* [lab=process] {} {{M_{P}} \bc M_{N} \;| \;P|M_{P} }
\end{mathpar} 

\begin{mathpar}
  \inferrule* [lab=sychronization] {} {M_{N} \bc \Box \;|\; x?M_{F} \;|\; x!M_{C}}
  \and
  \inferrule* [lab=abstraction] {} {{M_{F}} \bc (x)M_{P} }
  \and
  \inferrule* [lab=concretion] {} {{M_{C}} \bc \langle M_{P} \rangle }
  \and \\
  \inferrule* [lab=process] {} {{M_{P}} \bc M_{N} \;| \;P|M_{P} }
\end{mathpar}

\begin{definition}[contextual application] Given a context $M$, and
  process $P$, we define the \emph{contextual application}, $M[P] :=
  M\{P/\Box\}$. That is, the contextual application of M to P is the
  substitution of $P$ for $\Box$ in $M$.
\end{definition}

$\meaningof{-} : L \to \mathcal{P}(\pi)$

\begin{mathpar}
  \inferrule* [lab=collection] {} {\meaningof{true} = \pi, \and \meaningof{~E} = \pi \setminus \meaningof{E}, \and \meaningof{E_{1} \& E_{2}} = \meaningof{E_{1}} \cap \meaningof{E_{2}}}
\end{mathpar}

\begin{mathpar}
  \inferrule* [lab=structure] {} {\meaningof{0} = \{ P \in \pi | P \equiv 0 \}, \and \\ \meaningof{E_1 | E_2} = \{ P \in \pi | P \equiv P_{1} | P_{2}, P_{1} \in \meaningof{E_{1}}, P_{2} \in \meaningof{E_2}\} }
\end{mathpar}

\begin{mathpar}
 \inferrule* [lab=behavior] {} {\meaningof{\langle a?b \rangle E} = \{ P \in \pi | P \equiv Q | u?(y)P', \\ \and \\\\ \and \\ \;\;\; u \in \meaningof{a}, \forall z.P'\{z/y\} \in \meaningof{E\{z/b\}}\}, \and \\ \meaningof{a!E} = \{ P \in \pi | P \equiv Q | x!\langle P' \rangle, x \in \meaningof{a} P' \in \meaningof{E}\} }
\end{mathpar}

\begin{mathpar}
 \inferrule* [lab=nominal] {} {\meaningof{\quotep{E}} = \{ \quotep{P} \in \quotep{\pi} | P \in \meaningof{E} \}, \and \meaningof{\quotep{P}} = \{ \quotep{Q} \in \quotep{\pi} | P \equiv Q \} \and \\ \meaningof{@\quotep{E}} = \{ P \in \pi | P \equiv @x, x \in \meaningof{E} \}}
\end{mathpar}

\begin{eqnarray*}
  \\
  \meaningof{-} : TS \to ST
\end{eqnarray*}

\begin{eqnarray*}
  \\
  L : TS \to ST
\end{eqnarray*}

\begin{eqnarray*}
  \\
  P \models E \iff P \in \meaningof{E}
\end{eqnarray*}

\begin{eqnarray*}
  P \approx_{L} Q \iff \forall E \in L. P \models E \iff Q \models E
\end{eqnarray*}

\begin{eqnarray*}
  P \approx_{K} Q
\end{eqnarray*}

\begin{eqnarray*}
  P \approx Q
\end{eqnarray*}

$\approx_{K} = \approx = \approx_{L}$

\subsubsection{Contextual duality}

Note that contexts extend the quotation operation to a family of
operations from processes to names. Given a context, $M$, we can
define a \emph{nominal context}, $\quotep{M}$ by $\quotep{M}[P] :=
\quotep{M[P]}$. To foreshadow what is to come we observe that these
operations enjoy a duality with processes very much like the duality
between vectors and maps from vectors to scalars.

Further, because the calculus is essentially higher-order, we have a
correspondence between contexts and processes. More specifically,
given a name $x$ and a context $M$ we can construct $M^{*}_{x}$ such
that 

\begin{mathpar}
  M^{*}_{x} | \lift{x}{P} \red M[P]
\end{mathpar}

namely,

\begin{mathpar}
  M^{*}_{x} := x?(u).M[\dropn{u}]
\end{mathpar}

The dependence of $M^{*}_{x}$ on a name makes it an abstraction, 

\begin{mathpar}
  M^{*} := (x)x?(u).M[\dropn{u}]
\end{mathpar}

\subsection{Additional notation}

It will sometimes be convenient to denote the process a name
quotes. We already have the notation $x = \quotep{P}$, but it will be
convenient to introduce an alternate notation, $\procn{x}$, when we
want to emphasize the connection to the use of the name. Note that, by
virtue of name equivalence, $\quotep{\procn{x}} \nameeq x$; so, the
notation is consistent with previous definitions.

Further, because names have structure it is possible to effect
substitutions on the basis of that structure. This means we need to
upgrade our notation for substitutions, which we accomplish by
adapting comprehension notation. Thus,

\begin{mathpar}
  P\{ y / x : x \in S \}
\end{mathpar}

is interpreted to mean the process derived from P by replacing (in a
capture-avoiding manner) each occurrence of $x$ in $S$ by $y$. For example,

\begin{mathpar}
  P\{ \quotep{\procn{x}|\procn{x}} / x : x \in \freenames{P} \}
\end{mathpar}

will replace each (occurrence) of a free name $x$ in $P$ by
$\quotep{\procn{x}|\procn{x}}$.

Also, we will avail ourselves of the notation $x^{L}$ and $x^{R}$ to
denote injections of a name into disjoint copies of the name
space. There are numerous ways to accomplish this. One example can be
found in \cite{MeredithR05}. This notation overloads to vectors of
names: $\vec{x}^{\pi} := (x_{i}^{\pi} \; : \; 0 \leq i < |\vec{x}| )$ where $\pi \in \{L,R\}$.

We also use $P^{\Box} := P|\Box$.

In \cite{MeredithR05} an interpretation of the new operator is
given. It turns out that there are several possible interpretations
all enjoying the requisite algebraic properties of the operator (see
\cite{milner91polyadicpi}). We will therefore make liberal use of
$(\nu\; \vec{x})P$.

% subsection the_syntax_and_semantics_of_the_notation_system (end)   

\input{qm2pi.qmops} 

\input{qm2pi.sterngerlach} 

\input{qm2pi.metric} 

% section concurrent_process_calculi (end)

%\input{qm2pi.proofsketch}

% section proof sketch (end)

%\input{qm2pi.slviaknots} 

% section spatial logic via knots (end)

\input{qm2pi.conclusion}

% section conclusion (end)

%\input{qm2pi.dtcodes} 

% section wiring algorithm (end)

\input{qm2pi.ack} 

% section acknowledgments (end)

\newpage


\bibliographystyle{plain}   
\bibliography{../../biblios/main.bib}

\input{qm2pi.rhodetails}

\end{document}

 

%\ifpdf
%\usepackage[pdftex]{graphicx}
%\else
%\usepackage{graphicx}
%\fi

 % \ifpdf
%  \usepackage{pdfsync}
%  \if


%\title{Brief Article}
%\author{David F. Snyder}
%\author{L.G. Meredith}

%\address{Dept. of Math., Texas State University--San Marcos, San Marcos, TX 78666}
       
\pagestyle{empty}


\begin{document}

\lstset{language=[Objective]Caml,frame=shadowbox}

\documentclass[12pt]{llncs}
%\documentclass{jktr}

\usepackage[pdftex]{hyperref}                   
\usepackage {listings}
\usepackage {mathpartir}
\usepackage{bcprules}
%\usepackage{listings}
                       
\usepackage{graphicx} 
%\usepackage[margins=2.5cm,nohead,nofoot]{geometry}
%\usepackage{geometry}
\usepackage{amsfonts}
\usepackage{amstext}
\usepackage{latexsym}
\usepackage{amssymb}
\usepackage{color}


%\include{myPreamble}
\include{qm2pi.local} 

%\ifpdf
%\usepackage[pdftex]{graphicx}
%\else
%\usepackage{graphicx}
%\fi

 % \ifpdf
%  \usepackage{pdfsync}
%  \if


%\title{Brief Article}
%\author{David F. Snyder}
%\author{L.G. Meredith}

%\address{Dept. of Math., Texas State University--San Marcos, San Marcos, TX 78666}
       
\pagestyle{empty}


\begin{document}

\lstset{language=[Objective]Caml,frame=shadowbox}

\input{qm2pi.front}

% section front matter (end)

\input{qm2pi.intro} 
 
% section introduction (end)

% \input{qm2pi.knotations} 

% section notation (end)

\input{qm2pi.process.calculi} 

% section concurrent_process_calculi_and_spatial_logics_ (end)
    
%\input{qm2pi.knots2pi} 

%\input{qm2pi.trefoil} 

%\input{qm2pi.mainthm} 

% subsection basic_interpretation (end)

%\input{qm2pi.rho.presentation} 
\subsection{The syntax and semantics of the notation system}\label{sub:the_syntax_and_semantics_of_the_notation_system} % (fold)

We now summarize a technical presentation of the calculus that
embodies our theory of dynamics. The typical presentation of such a
calculus follows the style of giving generators and relations on
them. The grammar, below, describing term constructors, freely
generates the set of processes, $\Proc$. This set is then quotiented
by a relation known as structural congruence and it is over this set
that the notion of dynamics is expressed. This presentation is
essentially that of \cite{MeredithR05} with the addition of
polyadicity and summation. For readability we have relegated some of
the technical subtleties to an appendix.

\subsubsection{Process grammar}\label{subsub:process_grammar}

\begin{mathpar}
  \inferrule* [lab=synchronization] {} {{M} \bc \pzero \;|\; x?F \;|\; x!C }
  \and
  \inferrule* [lab=abstraction] {} {{F} \bc (x)P}
  \and
  \inferrule* [lab=concretion] {} {{C} \bc \langle Q \rangle}
  \and
  \inferrule* [lab=process] {} {{P,Q} \bc M \;| \;P|Q \;|\; @{x}}
  \and
  \inferrule* [lab=name] {} {{x} \bc \quotep{P}}
\end{mathpar} 

Note that $\vec{x}$ (resp. $\vec{P}$) denotes a vector of names
(resp. processes) of length $|\vec{x}|$ (resp. $|\vec{P}|$). We adopt
the following useful abbreviations.

\begin{mathpar}
   x?(\vec{y}).P := x.(\vec{y})P \and  x\clift{\vec{P}} := x.\clift{\vec{P}}
   \and x!(y) := \lift{x}{\dropn{y}}
   \and \Pi_{i=0}^{n-1}P_i := P_0 | \ldots | P_{n-1}
\end{mathpar}

\subsubsection{Structural congruence}

\paragraph{Free and bound names and alpha-equivalence.} At the
core of structural equivalence is alpha-equivalence which identifies
process that are the same up to a change of variable. Formally, we
recognize the distinction between free and bound names. The free names
of a process, $\freenames{P}$, may be calculated recursively as
follows:

\begin{mathpar}
\freenames{\pzero} := \emptyset
  \and \\
  \freenames{x?(y).P} := \{ x \} \cup (\freenames{P} \setminus \{ y \})
  \and 
  \freenames{x!\langle P \rangle} := \{ x \} \cup \{ P \} 
  \and \\
  \freenames{P|Q} := \freenames{P} \cup \freenames{Q}
  \and \\
  \freenames{@{x}} := \{ x \}
\end{mathpar}

$\pi$
$\quotep{\pi}$

$\freenames{-} : \pi \to \mathcal{P}(\quotep{\pi})$

\begin{eqnarray*}
  \freenames{\pzero} & := & \emptyset \\
  \freenames{x?(y).P} & := & \{ x \} \cup (\freenames{P} \setminus \{ y \}) \\
  \freenames{x!\langle P \rangle} & := & \{ x \} \cup \{ P \} \\
  \freenames{P|Q} & := & \freenames{P} \cup \freenames{Q} \\
  \freenames{\dropn{x}} & := & \{ x \}
\end{eqnarray*}

The bound names of a process, $\boundnames{P}$, are those names occurring in $P$
that are not free. For example, in $x?(y).0$, the name $x$ is free, while $y$ is bound.

\begin{mathpar}
  \inferrule* [lab=monoidal-laws] {} { P|Q \equiv Q|P \and P|0 \equiv P \and P|(Q|R) \equiv (P|Q)|R }
\end{mathpar}

\begin{mathpar}
  \inferrule* [lab=alpha-equivalence] {} { (x)P \equiv (y)P\{y/x\} \and y \not\in \freenames{P} }
\end{mathpar}

\begin{definition}
Then two processes, $P,Q$, are alpha-equivalent if $P = Q\{\vec{y}/\vec{x}\}$ for
some $\vec{x} \in \boundnames{Q},\vec{y} \in \boundnames{P}$, where $Q\{\vec{y}/\vec{x}\}$
denotes the capture-avoiding substitution of $\vec{y}$ for $\vec{x}$ in $Q$.
\end{definition}

\begin{definition}
  The {\em structural congruence} \cite{SangiorgiWalker} , $\equiv$,
  between processes is the least congruence containing
  alpha-equivalence, satisfying the abelian monoid laws
  (associativity, commutativity and $\pzero$ as identity) for parallel
  composition $|$ and for summation $+$.
\end{definition}

\subsection{Name equivalence}

We take name equivalence, written $\nameeq$, to be the smallest
equivalence relation generated by the following rules.

\begin{mathpar}
\inferrule*[lab=Quote-drop]
{ }
{ \quotep{@{x}} \nameeq x }

\inferrule*[lab=Struct-equiv]
{ P \scong Q }
{ \quotep{P} \nameeq \quotep{Q} }
\end{mathpar}

The astute reader will have noticed that the mutual recursion of names
and processes imposes a mutual recursion on alpha-equivalence and
structural equivalence via name-equivalence. Fortunately, all of this
works out pleasantly and we may calculate in the natural way, free of
concern. The reader interested in the details is referred to the
appendix \ref{appendix:rho_details}.

\subsection{Substitution}

We use $\Proc$ for the set of processes, $\QProc$ for the set of
names, and $\id{\{}\vec{y} / \vec{x} \id{\}}$ to denote partial maps,
$s : \QProc \rightarrow \QProc$. A map, $s$ lifts, uniquely, to a map
on process terms, $\widehat{s} : \Proc \rightarrow \Proc$ by the
following equations.

\begin{mathpar}
  (0) \psubstp{Q}{P} := 0 \\
  (R \juxtap S) \psubstp{Q}{P}
  :=    
  (R)\psubstp{Q}{P} \juxtap (S) \psubstp{Q}{P} \\
  (x?(y).R) \psubstp{Q}{P}    
  :=    
  (x)\substp{Q}{P} (z)\concat( (R \psubstn{z}{y}) \psubstp{Q}{P} ) \\
  (\lift{x}{R}) \psubstp{Q}{P}  
  :=
  \lift{(x)\substp{Q}{P}}{ R \psubstp{Q}{P} } \\
%   (\dropn{x})  \psubstp{Q}{P}       
%   := 
%   \left\{ 
%     \begin{array}{ccc} 
%       \dropn{\quotep{Q}} & & x \nameeq \quotep{P} \\
%       \dropn{x} & & otherwise \\
%     \end{array}
%   \right. 
  (\dropn{x})  \psubstp{Q}{P}       
  := 
  \left\{ 
    \begin{array}{ccc} 
      Q & & x \nameeq \quotep{P} \\
      \dropn{x} & & otherwise \\
    \end{array}
  \right.
\end{mathpar}
 

where

\begin{eqnarray}
  (x)\id{\{} \lpquote Q \rpquote / \lpquote P \rpquote \id{\}}            = 
  \left\{ 
    \begin{array}{ccc}
      \lpquote Q \rpquote & & x \nameeq \lpquote P \rpquote \\
      x & & otherwise \\
    \end{array}
  \right. \nonumber
\end{eqnarray}

and $z$ is chosen distinct from $\quotep{P}$, $\quotep{Q}$, the free
names in $Q$, and all the names in $R$. Our $\alpha$-equivalence will
be built in the standard way from this substitution.

\begin{remark}\label{rem:no_self_referential_names}
  One consequence of these definitions is that $\forall P. \quotep{P}
  \not\in \freenames{P}$.
\end{remark}

\subsection{ Dynamic quote: an example }

Anticipating something of what's to come, consider applying the
substitution, $\widehat{\id{\{}u / z \id{\}}}$, to the following pair
of processes, $\lift{w}{y!(z)}$ and $w[ \lpquote y!(z) \rpquote ]$.

\begin{eqnarray}
	\lift{w}{y!(z)}\widehat{\id{\{}u / z \id{\}}}
		& = &
		\lift{w}{y!(u)} \nonumber\\
	w[ \lpquote y!(z) \rpquote ] \widehat{ \id{\{}u / z \id{\}} }
		& = &
		w[ \lpquote y!(z) \rpquote ] \nonumber
\end{eqnarray}

Because the body of the process between quotes is impervious to
substitution, we get radically different answers. In fact, by
examining the first process in an input context,
e.g. $x?(z).\lift{w}{y!(z)}$, we see that the process under the lift
operator may be shaped by prefixed inputs binding a name inside it. In
this sense, the lift operator will be seen as a way to dynamically
construct processes before reifying them as names.

Finally equipped with these standard features we can present the
dynamics of the calculus.

\subsubsection{Operational semantics} 

Finally, we introduce the computational dynamics. What marks these
algebras as distinct from other more traditionally studied algebraic
structures, e.g. vector spaces or polynomial rings, is the manner in
which dynamics is captured. In traditional structures, dynamics is typically
expressed through morphisms between such structures, as in linear maps
between vector spaces or morphisms between rings. In algebras
associated with the semantics of computation, the dynamics is
expressed as part of the algebraic structure itself, through a
reduction reduction relation typically denoted by $\red$. Below, we
give a recursive presentation of this relation for the calculus used
in the encoding.

$\red \subseteq \pi \times \pi$
$\red : \pi \to \mathcal{P}(\pi)$

\begin{mathpar}
  \inferrule* [lab=Comm] { \textsf{match}( x_{src}, x_{trgt} ) } { x_{trgt}?(y)P \; | \; x_{src}!\langle {Q} \rangle \red P\{\quotep{Q}/y}\} }
  \and \\
  \inferrule* [lab=Par] {{P} \red {P}'} {{{P} | {Q}} \red {{P}' | {Q}}}
  \and
  \inferrule* [lab=Equiv]{{{P} \scong {P}'} \andalso {{P}' \red {Q}'} \andalso {{Q}' \scong {Q}}}{{P} \red {Q}}
\end{mathpar}

\begin{eqnarray*}
  match_{\equiv} (\quotep{P},\quotep{Q}) & := & P \equiv Q \\
  match_{\dagger}(\quotep{P},\quotep{Q}) & := & \forall R. P|Q \red^{*} R => R \red^{*} 0 \\
  match_{K}(\quotep{P},\quotep{Q}) & := & K \mbox{ for some context } K
\end{eqnarray*}

$u?(x)P | u!\langle Q \rangle \red P\{\quotep{Q}/x\}$

%We write $\wred$ for $\red^*$, and $P\red$ if $\exists Q $ such that $ P \red Q$.
We write $P\red$ if $\exists Q $ such that $ P \red Q$ and $P\not\red$, otherwise.

\section{Replication}

As mentioned before, it is known that replication (and hence
recursion) can be implemented in a higher-order process algebra
\cite{SangiorgiWalker}. As our first example of calculation with the
machinery thus far presented we give the construction explicitly in
the {\rhoc}.

\begin{eqnarray}
	D_{x} & := & \prefix{x}{y}{(\binpar{\outputp{x}{y}}{@{y}})} \nonumber\\
	\bangp_{x}{P} & := & \binpar{{x}!\langle{\binpar{D_{x}}{P}}\rangle}{D_{x}} \nonumber
\end{eqnarray}

\begin{eqnarray}
	\bangp_{x}{P} & & \nonumber\\
	=
	& {x}!\langle{(\prefix{x}{y}{(\outputp{x}{y} | @{y})) | P}}\rangle 
	      | \prefix{x}{y}{(\outputp{x}{y} | @{y})} & \nonumber\\
	\red
	& (\outputp{x}{y} | @{y})\substn{\quotep{(\prefix{x}{y}{(@{y} | \outputp{x}{y})) | P}}}{y} & \nonumber\\
	=
	& \outputp{x}{\quotep{(\prefix{x}{y}{(\outputp{x}{y} | @{y})) | P}}}
	  | {(\prefix{x}{y}{(\outputp{x}{y} | @{y})) | P}} & \nonumber\\
	\red
	& \ldots & \nonumber\\
	\red^*
	& P | P | \ldots & \nonumber
\end{eqnarray}

Of course, this encoding, as an implementation, runs away, unfolding
$\bangp{P}$ eagerly. A lazier and more implementable replication
operator, restricted to input-guarded processes, may be obtained as follows.

\begin{eqnarray}
\bangp{\prefix{u}{v}{P}} 
	:= 
	\binpar{\lift{x}{\prefix{u}{v}{(\binpar{D(x)}{P})}}}{D(x)} \nonumber
\end{eqnarray}

\begin{remark}
  Note that the lazier definition still does not deal with summation
  or mixed summation (i.e. sums over input and output). The reader is
  invited to construct definitions of replication that deal with these
  features. 

  Further, the definitions are parameterized in a name, $x$. Can you,
  gentle reader, make a definition that eliminates this parameter and
  guarantees no accidental interaction between the replication
  machinery and the process being replicated -- i.e. no accidental
  sharing of names used by the process to get its work done and the
  name(s) used by the replication to effect copying. This latter
  revision of the definition of replication is crucial to obtaining
  the expected identity $!!P \sim !P$.
\end{remark}

\begin{remark}\label{rem:paradoxical_combinator}
  The reader familiar with the lambda calculus will have noticed the
  similarity between $D$ and the paradoxical combinator.

  [Ed. note: the existence of this seems to suggest we have to be more
  restrictive on the set of processes and names we admit if we are to
  support no-cloning.]
\end{remark}

\subsubsection{Bisimulation}

The computational dynamics gives rise to another kind of equivalence,
the equivalence of computational behavior. As previously mentioned
this is typically captured \emph{via} some form of bisimulation.

% The notion we use in this paper is weak barbed bisimulation
% \cite{milner91polyadicpi}.

The notion we use in this paper is derived from weak barbed
bisimulation \cite{milner91polyadicpi}. 

\begin{definition}
An \emph{observation relation}, $\downarrow_{\mathcal N}$, over a set
of names, $\mathcal N$, is the smallest relation satisfying the rules
below.

\infrule[Out-barb]{y \in {\mathcal N}, \; x \nameeq y}
		  {\outputp{x}{v} \downarrow_{\mathcal N} x}
\infrule[Par-barb]{\mbox{$P\downarrow_{\mathcal N} x$ or $Q\downarrow_{\mathcal N} x$}}
		  {\binpar{P}{Q} \downarrow_{\mathcal N} x}

We write $P \Downarrow_{\mathcal N} x$ if there is $Q$ such that 
$P \wred Q$ and $Q \downarrow_{\mathcal N} x$.
\end{definition}

\begin{definition}
%\label{def.bbisim}
An  ${\mathcal N}$-\emph{barbed bisimulation} over a set of names, ${\mathcal N}$, is a symmetric binary relation 
${\mathcal S}_{\mathcal N}$ between agents such that $P\rel{S}_{\mathcal N}Q$ implies:
\begin{enumerate}
\item If $P \red P'$ then $Q \wred Q'$ and $P'\rel{S}_{\mathcal N} Q'$.
\item If $P\downarrow_{\mathcal N} x$, then $Q\Downarrow_{\mathcal N} x$.
\end{enumerate}
$P$ is ${\mathcal N}$-barbed bisimilar to $Q$, written
$P \wbbisim_{\mathcal N} Q$, if $P \rel{S}_{\mathcal N} Q$ for some ${\mathcal N}$-barbed bisimulation ${\mathcal S}_{\mathcal N}$.
\end{definition}

$\mathcal{R} \subseteq \pi \times \pi$

$P \mathcal{R} Q => \forall P'. P \red P' \Rightarrow \exists Q'. Q \red Q', P' \mathcal{R} Q'$

$P \vdash x \Rightarrow Q \vdash x$

\begin{mathpar}
  \inferrule*[lab=Out-barb]{x \nameeq y}{{y}!\langle{Q}\rangle \vdash x}
  \and
  \inferrule*[lab=Par-barb]{\mbox{$P\vdash x$ or $Q\vdash x$}}{\binpar{P}{Q} \vdash x}
\end{mathpar}

\subsubsection{Contexts}

One of the principle advantages of computational calculi like the
$\pi$-calculus is a well-defined notion of context,
contextual-equivalence and a correlation between
contextual-equivalence and notions of bisimulation. The notion of
context allows the decomposition of a process into (sub-)process and
its syntactic environment, its context. Thus, a context may be
thought of as a process with a ``hole'' (written $\Box$) in it. The
application of a context $M$ to a process $P$, written $M[P]$, is
tantamount to filling the hole in $M$ with $P$. In this paper we do
not need the full weight of this theory, but do make use of the notion
of context in the proof the main theorem. 

\begin{mathpar}
  \inferrule* [lab=summation] {} {{M_{M},M_{N}} \bc \Box \;|\; x.M_{A} \;|\; M_{M}+M_{N}}
  \and
  \inferrule* [lab=agent] {} {{M_{A}} \bc (\vec{x})M_{P} \;| \; \clift{P_0,\ldots,M_{P},\ldots,P_N}}
  \and \\
  \inferrule* [lab=process] {} {{M_{P}} \bc M_{N} \;| \;P|M_{P} }
\end{mathpar} 

\begin{mathpar}
  \inferrule* [lab=sychronization] {} {M_{N} \bc \Box \;|\; x?M_{F} \;|\; x!M_{C}}
  \and
  \inferrule* [lab=abstraction] {} {{M_{F}} \bc (x)M_{P} }
  \and
  \inferrule* [lab=concretion] {} {{M_{C}} \bc \langle M_{P} \rangle }
  \and \\
  \inferrule* [lab=process] {} {{M_{P}} \bc M_{N} \;| \;P|M_{P} }
\end{mathpar}

\begin{definition}[contextual application] Given a context $M$, and
  process $P$, we define the \emph{contextual application}, $M[P] :=
  M\{P/\Box\}$. That is, the contextual application of M to P is the
  substitution of $P$ for $\Box$ in $M$.
\end{definition}

$\meaningof{-} : L \to \mathcal{P}(\pi)$

\begin{mathpar}
  \inferrule* [lab=collection] {} {\meaningof{true} = \pi, \and \meaningof{~E} = \pi \setminus \meaningof{E}, \and \meaningof{E_{1} \& E_{2}} = \meaningof{E_{1}} \cap \meaningof{E_{2}}}
\end{mathpar}

\begin{mathpar}
  \inferrule* [lab=structure] {} {\meaningof{0} = \{ P \in \pi | P \equiv 0 \}, \and \\ \meaningof{E_1 | E_2} = \{ P \in \pi | P \equiv P_{1} | P_{2}, P_{1} \in \meaningof{E_{1}}, P_{2} \in \meaningof{E_2}\} }
\end{mathpar}

\begin{mathpar}
 \inferrule* [lab=behavior] {} {\meaningof{\langle a?b \rangle E} = \{ P \in \pi | P \equiv Q | u?(y)P', \\ \and \\\\ \and \\ \;\;\; u \in \meaningof{a}, \forall z.P'\{z/y\} \in \meaningof{E\{z/b\}}\}, \and \\ \meaningof{a!E} = \{ P \in \pi | P \equiv Q | x!\langle P' \rangle, x \in \meaningof{a} P' \in \meaningof{E}\} }
\end{mathpar}

\begin{mathpar}
 \inferrule* [lab=nominal] {} {\meaningof{\quotep{E}} = \{ \quotep{P} \in \quotep{\pi} | P \in \meaningof{E} \}, \and \meaningof{\quotep{P}} = \{ \quotep{Q} \in \quotep{\pi} | P \equiv Q \} \and \\ \meaningof{@\quotep{E}} = \{ P \in \pi | P \equiv @x, x \in \meaningof{E} \}}
\end{mathpar}

\begin{eqnarray*}
  \\
  \meaningof{-} : TS \to ST
\end{eqnarray*}

\begin{eqnarray*}
  \\
  L : TS \to ST
\end{eqnarray*}

\begin{eqnarray*}
  \\
  P \models E \iff P \in \meaningof{E}
\end{eqnarray*}

\begin{eqnarray*}
  P \approx_{L} Q \iff \forall E \in L. P \models E \iff Q \models E
\end{eqnarray*}

\begin{eqnarray*}
  P \approx_{K} Q
\end{eqnarray*}

\begin{eqnarray*}
  P \approx Q
\end{eqnarray*}

$\approx_{K} = \approx = \approx_{L}$

\subsubsection{Contextual duality}

Note that contexts extend the quotation operation to a family of
operations from processes to names. Given a context, $M$, we can
define a \emph{nominal context}, $\quotep{M}$ by $\quotep{M}[P] :=
\quotep{M[P]}$. To foreshadow what is to come we observe that these
operations enjoy a duality with processes very much like the duality
between vectors and maps from vectors to scalars.

Further, because the calculus is essentially higher-order, we have a
correspondence between contexts and processes. More specifically,
given a name $x$ and a context $M$ we can construct $M^{*}_{x}$ such
that 

\begin{mathpar}
  M^{*}_{x} | \lift{x}{P} \red M[P]
\end{mathpar}

namely,

\begin{mathpar}
  M^{*}_{x} := x?(u).M[\dropn{u}]
\end{mathpar}

The dependence of $M^{*}_{x}$ on a name makes it an abstraction, 

\begin{mathpar}
  M^{*} := (x)x?(u).M[\dropn{u}]
\end{mathpar}

\subsection{Additional notation}

It will sometimes be convenient to denote the process a name
quotes. We already have the notation $x = \quotep{P}$, but it will be
convenient to introduce an alternate notation, $\procn{x}$, when we
want to emphasize the connection to the use of the name. Note that, by
virtue of name equivalence, $\quotep{\procn{x}} \nameeq x$; so, the
notation is consistent with previous definitions.

Further, because names have structure it is possible to effect
substitutions on the basis of that structure. This means we need to
upgrade our notation for substitutions, which we accomplish by
adapting comprehension notation. Thus,

\begin{mathpar}
  P\{ y / x : x \in S \}
\end{mathpar}

is interpreted to mean the process derived from P by replacing (in a
capture-avoiding manner) each occurrence of $x$ in $S$ by $y$. For example,

\begin{mathpar}
  P\{ \quotep{\procn{x}|\procn{x}} / x : x \in \freenames{P} \}
\end{mathpar}

will replace each (occurrence) of a free name $x$ in $P$ by
$\quotep{\procn{x}|\procn{x}}$.

Also, we will avail ourselves of the notation $x^{L}$ and $x^{R}$ to
denote injections of a name into disjoint copies of the name
space. There are numerous ways to accomplish this. One example can be
found in \cite{MeredithR05}. This notation overloads to vectors of
names: $\vec{x}^{\pi} := (x_{i}^{\pi} \; : \; 0 \leq i < |\vec{x}| )$ where $\pi \in \{L,R\}$.

We also use $P^{\Box} := P|\Box$.

In \cite{MeredithR05} an interpretation of the new operator is
given. It turns out that there are several possible interpretations
all enjoying the requisite algebraic properties of the operator (see
\cite{milner91polyadicpi}). We will therefore make liberal use of
$(\nu\; \vec{x})P$.

% subsection the_syntax_and_semantics_of_the_notation_system (end)   

\input{qm2pi.qmops} 

\input{qm2pi.sterngerlach} 

\input{qm2pi.metric} 

% section concurrent_process_calculi (end)

%\input{qm2pi.proofsketch}

% section proof sketch (end)

%\input{qm2pi.slviaknots} 

% section spatial logic via knots (end)

\input{qm2pi.conclusion}

% section conclusion (end)

%\input{qm2pi.dtcodes} 

% section wiring algorithm (end)

\input{qm2pi.ack} 

% section acknowledgments (end)

\newpage


\bibliographystyle{plain}   
\bibliography{../../biblios/main.bib}

\input{qm2pi.rhodetails}

\end{document}



% section front matter (end)

\section{Introduction}\label{sec:introduction} % (fold)
In this draft of the material i am going to have to dispense with the
usual writing conventions adopted in papers on these topics. i'm going
to have adopt whatever tone i need at the time i'm writing up the
calculations. Sometimes this may be very conversational; others it may
be the barest mathematical grunts; others still it may be that i have
lifted text from one of my other papers because the exposition of some
point was better said there. i hope that my readers are not unduly put
out by this decision. i'm not doing this to flout convention or be
rebellious. i find these calculations very technically challenging. To
keep everything going technically, something has to give; i have to
let go of some cognitive burden. So, the academic writing style --
with all of its trade-offs in terms of facilitating technical
communication -- is what i'm letting go of. Perhaps subsequent drafts
can be tightened and polished, but for now, i'm going to speak as if
we were sitting together in a coffee shop with a laptop, wifi and a
pad of paper and a pencil.

So, here's what i have to say. We -- you and i, comfortably ensconced
in our coffee shop and well-equipped with our tools -- can realize and
carry out the calculations of quantum mechanics over a very different
formal theory of dynamics, a formal theory of dynamics that
corresponds to a theory of concurrent computation with
\emph{reflection}. It has the advantage that the underlying theory is
already `quantized', but supports analogues all of the continuuous
operations. Strikingly, this underlying theory has recently been
connected with a notion of metric that we can show, by calculating
together, coincides with the metric induced by the inner product.

There are a lot of reasons why you might be interested in seeing
calculations of this form. Here's why i'm interested. For the past
several centuries there has been no competitor to the ``Newtonian''
account of dynamics. As a result the predominant share of accounts of
dynamical systems and situations have had to be formulated in terms of
the Newtonian machinery. i view this as an intellectually dangerous
position to occupy. Everything, despite it's intrinsic shape, turns
into a nail to be hit with this hammer. Recently, however, the theory
of computation has matured to the point where we have candidates for
theories of dynamics that offer very different perspective on
reasoning about dynamical systems and situations. Testing these
candidates against very successful accounts of dynamical situations,
like quantum mechanics, is going to give us some sense of how mature
they are and some measure of the quality of these accounts of
dynamics.

\subsection{Summary of contributions and outline of paper}

So, we're going to develop an interpretation of the operations of
quantum mechanics normally interpreted by Hilbert spaces and
operators. We're going to do this over a theory of computation. Note
that this is very different than the usual quantum computation program
which develops notions of computation over quantum mechanics. Rather,
we are developing a story that aligns with Wheeler's slogan: It from
Bit. To do this we will first provide an account of the theory of
computation at play here. Then we will dive into a calculation-driven
interpretation of the operations of quantum mechanics.

The reason we take this approach is that -- until very recently --
there hasn't been an axiomatic account of quantum mechanics. As a
result there has been no sharp delineation of the mathematical theory
supporting interpretation of the physical theory and the physical
theory, itself. So, ambient features of the maths are free to be
exploited (or supressed) without a real accounting of their physical
relevance. There is no sharp statement ``here's the physical theory''
qua \emph{theory} and ``here's the mathematical interpretation''
enabling a judgment of how faithful the interpretation is -- apart
from experimental observation. When there is an axiomatic account we
can judge how well a given mathematical formalism supports an
interpretation of the axioms, independent of
experimentation. Likewise, we can judge how well we have captured our
physical evidence and experience with our axiomatics, independent of
any specific mathematical implementation, with accidental detail that
may or may not have physical significance. 

In lieu of a fully fleshed out and vetted axiomatic account of quantum
mechanics, interpreting the operational notions in service of modeling
physical systems will have to suffice. In other words, we are not in
the business of providing a model of Hilbert spaces and operators. We
are in the business of providing a model of quantum mechanics because
we are motivated by testing our notions of dynamics against physical
theory; and, the predictive calculations of the physical theory must
serve as the best formulation -- shy of a fully fleshed out axiomatic
account -- of the physical theory itself (as they have for scientific
theories since time immemorial). Put another way, despite a
whole-hearted commitment to an It-from-Bit ontology, we are firmly
aligned with the shut-up-and-calculate camp as the best way to obtain
results either from the physical perspective or as a quality assurance
measure of our fledgling theory of dynamics.

In detail, we present a reflective process calculus. Then we develop
intuitive correspondences between the notions available in this
calculus and the usual physical notions supporting quantum mechanical
calculations. Thus, 

\begin{table}[htp]
  \center{
    \fbox{
      \begin{tabular}{c|c}
        quantum mechanics & process calculus \\
        \hline
        scalar & name \\
        state vector & process \\
        dual & contextual duals \\
        matrix & formal sums of process-context-dual pairs \\
        orthogonality & process annihilation \\
        inner product & execution-formula + quoting
      \end{tabular}
    }
  }
  \caption{QM - process calculi correspondences}
\end{table}

Then we tighten up these intuitions to operational definitions. We
employ the Dirac notation as the best proxy we can find for an
abstract syntax of the quantum mechanical notions. The definitions we
develop put us in contact with equational constraints coming from the
theory that we demonstrate the definitions and calculations satisfy.

This puts us in a position to shut up and calculate for the
Stern-Gerlach experimental set up, showing how these predictive
calculations become calculations on processes in our theory of a
reflective process calculus.

Penultimately, we demonstrate that the notion of metric coming from
the inner product coincides with the notion of metric available from
the theory of bisimulation. This demonstration gives us the right to
think of space as arising from behavior. Finally, we consider where we
might go from the new vantage point we have obtained.

% section introduction (end) 
 
% section introduction (end)

% \documentclass[12pt]{llncs}
%\documentclass{jktr}

\usepackage[pdftex]{hyperref}                   
\usepackage {listings}
\usepackage {mathpartir}
\usepackage{bcprules}
%\usepackage{listings}
                       
\usepackage{graphicx} 
%\usepackage[margins=2.5cm,nohead,nofoot]{geometry}
%\usepackage{geometry}
\usepackage{amsfonts}
\usepackage{amstext}
\usepackage{latexsym}
\usepackage{amssymb}
\usepackage{color}


%\include{myPreamble}
\include{qm2pi.local} 

%\ifpdf
%\usepackage[pdftex]{graphicx}
%\else
%\usepackage{graphicx}
%\fi

 % \ifpdf
%  \usepackage{pdfsync}
%  \if


%\title{Brief Article}
%\author{David F. Snyder}
%\author{L.G. Meredith}

%\address{Dept. of Math., Texas State University--San Marcos, San Marcos, TX 78666}
       
\pagestyle{empty}


\begin{document}

\lstset{language=[Objective]Caml,frame=shadowbox}

\input{qm2pi.front}

% section front matter (end)

\input{qm2pi.intro} 
 
% section introduction (end)

% \input{qm2pi.knotations} 

% section notation (end)

\input{qm2pi.process.calculi} 

% section concurrent_process_calculi_and_spatial_logics_ (end)
    
%\input{qm2pi.knots2pi} 

%\input{qm2pi.trefoil} 

%\input{qm2pi.mainthm} 

% subsection basic_interpretation (end)

%\input{qm2pi.rho.presentation} 
\subsection{The syntax and semantics of the notation system}\label{sub:the_syntax_and_semantics_of_the_notation_system} % (fold)

We now summarize a technical presentation of the calculus that
embodies our theory of dynamics. The typical presentation of such a
calculus follows the style of giving generators and relations on
them. The grammar, below, describing term constructors, freely
generates the set of processes, $\Proc$. This set is then quotiented
by a relation known as structural congruence and it is over this set
that the notion of dynamics is expressed. This presentation is
essentially that of \cite{MeredithR05} with the addition of
polyadicity and summation. For readability we have relegated some of
the technical subtleties to an appendix.

\subsubsection{Process grammar}\label{subsub:process_grammar}

\begin{mathpar}
  \inferrule* [lab=synchronization] {} {{M} \bc \pzero \;|\; x?F \;|\; x!C }
  \and
  \inferrule* [lab=abstraction] {} {{F} \bc (x)P}
  \and
  \inferrule* [lab=concretion] {} {{C} \bc \langle Q \rangle}
  \and
  \inferrule* [lab=process] {} {{P,Q} \bc M \;| \;P|Q \;|\; @{x}}
  \and
  \inferrule* [lab=name] {} {{x} \bc \quotep{P}}
\end{mathpar} 

Note that $\vec{x}$ (resp. $\vec{P}$) denotes a vector of names
(resp. processes) of length $|\vec{x}|$ (resp. $|\vec{P}|$). We adopt
the following useful abbreviations.

\begin{mathpar}
   x?(\vec{y}).P := x.(\vec{y})P \and  x\clift{\vec{P}} := x.\clift{\vec{P}}
   \and x!(y) := \lift{x}{\dropn{y}}
   \and \Pi_{i=0}^{n-1}P_i := P_0 | \ldots | P_{n-1}
\end{mathpar}

\subsubsection{Structural congruence}

\paragraph{Free and bound names and alpha-equivalence.} At the
core of structural equivalence is alpha-equivalence which identifies
process that are the same up to a change of variable. Formally, we
recognize the distinction between free and bound names. The free names
of a process, $\freenames{P}$, may be calculated recursively as
follows:

\begin{mathpar}
\freenames{\pzero} := \emptyset
  \and \\
  \freenames{x?(y).P} := \{ x \} \cup (\freenames{P} \setminus \{ y \})
  \and 
  \freenames{x!\langle P \rangle} := \{ x \} \cup \{ P \} 
  \and \\
  \freenames{P|Q} := \freenames{P} \cup \freenames{Q}
  \and \\
  \freenames{@{x}} := \{ x \}
\end{mathpar}

$\pi$
$\quotep{\pi}$

$\freenames{-} : \pi \to \mathcal{P}(\quotep{\pi})$

\begin{eqnarray*}
  \freenames{\pzero} & := & \emptyset \\
  \freenames{x?(y).P} & := & \{ x \} \cup (\freenames{P} \setminus \{ y \}) \\
  \freenames{x!\langle P \rangle} & := & \{ x \} \cup \{ P \} \\
  \freenames{P|Q} & := & \freenames{P} \cup \freenames{Q} \\
  \freenames{\dropn{x}} & := & \{ x \}
\end{eqnarray*}

The bound names of a process, $\boundnames{P}$, are those names occurring in $P$
that are not free. For example, in $x?(y).0$, the name $x$ is free, while $y$ is bound.

\begin{mathpar}
  \inferrule* [lab=monoidal-laws] {} { P|Q \equiv Q|P \and P|0 \equiv P \and P|(Q|R) \equiv (P|Q)|R }
\end{mathpar}

\begin{mathpar}
  \inferrule* [lab=alpha-equivalence] {} { (x)P \equiv (y)P\{y/x\} \and y \not\in \freenames{P} }
\end{mathpar}

\begin{definition}
Then two processes, $P,Q$, are alpha-equivalent if $P = Q\{\vec{y}/\vec{x}\}$ for
some $\vec{x} \in \boundnames{Q},\vec{y} \in \boundnames{P}$, where $Q\{\vec{y}/\vec{x}\}$
denotes the capture-avoiding substitution of $\vec{y}$ for $\vec{x}$ in $Q$.
\end{definition}

\begin{definition}
  The {\em structural congruence} \cite{SangiorgiWalker} , $\equiv$,
  between processes is the least congruence containing
  alpha-equivalence, satisfying the abelian monoid laws
  (associativity, commutativity and $\pzero$ as identity) for parallel
  composition $|$ and for summation $+$.
\end{definition}

\subsection{Name equivalence}

We take name equivalence, written $\nameeq$, to be the smallest
equivalence relation generated by the following rules.

\begin{mathpar}
\inferrule*[lab=Quote-drop]
{ }
{ \quotep{@{x}} \nameeq x }

\inferrule*[lab=Struct-equiv]
{ P \scong Q }
{ \quotep{P} \nameeq \quotep{Q} }
\end{mathpar}

The astute reader will have noticed that the mutual recursion of names
and processes imposes a mutual recursion on alpha-equivalence and
structural equivalence via name-equivalence. Fortunately, all of this
works out pleasantly and we may calculate in the natural way, free of
concern. The reader interested in the details is referred to the
appendix \ref{appendix:rho_details}.

\subsection{Substitution}

We use $\Proc$ for the set of processes, $\QProc$ for the set of
names, and $\id{\{}\vec{y} / \vec{x} \id{\}}$ to denote partial maps,
$s : \QProc \rightarrow \QProc$. A map, $s$ lifts, uniquely, to a map
on process terms, $\widehat{s} : \Proc \rightarrow \Proc$ by the
following equations.

\begin{mathpar}
  (0) \psubstp{Q}{P} := 0 \\
  (R \juxtap S) \psubstp{Q}{P}
  :=    
  (R)\psubstp{Q}{P} \juxtap (S) \psubstp{Q}{P} \\
  (x?(y).R) \psubstp{Q}{P}    
  :=    
  (x)\substp{Q}{P} (z)\concat( (R \psubstn{z}{y}) \psubstp{Q}{P} ) \\
  (\lift{x}{R}) \psubstp{Q}{P}  
  :=
  \lift{(x)\substp{Q}{P}}{ R \psubstp{Q}{P} } \\
%   (\dropn{x})  \psubstp{Q}{P}       
%   := 
%   \left\{ 
%     \begin{array}{ccc} 
%       \dropn{\quotep{Q}} & & x \nameeq \quotep{P} \\
%       \dropn{x} & & otherwise \\
%     \end{array}
%   \right. 
  (\dropn{x})  \psubstp{Q}{P}       
  := 
  \left\{ 
    \begin{array}{ccc} 
      Q & & x \nameeq \quotep{P} \\
      \dropn{x} & & otherwise \\
    \end{array}
  \right.
\end{mathpar}
 

where

\begin{eqnarray}
  (x)\id{\{} \lpquote Q \rpquote / \lpquote P \rpquote \id{\}}            = 
  \left\{ 
    \begin{array}{ccc}
      \lpquote Q \rpquote & & x \nameeq \lpquote P \rpquote \\
      x & & otherwise \\
    \end{array}
  \right. \nonumber
\end{eqnarray}

and $z$ is chosen distinct from $\quotep{P}$, $\quotep{Q}$, the free
names in $Q$, and all the names in $R$. Our $\alpha$-equivalence will
be built in the standard way from this substitution.

\begin{remark}\label{rem:no_self_referential_names}
  One consequence of these definitions is that $\forall P. \quotep{P}
  \not\in \freenames{P}$.
\end{remark}

\subsection{ Dynamic quote: an example }

Anticipating something of what's to come, consider applying the
substitution, $\widehat{\id{\{}u / z \id{\}}}$, to the following pair
of processes, $\lift{w}{y!(z)}$ and $w[ \lpquote y!(z) \rpquote ]$.

\begin{eqnarray}
	\lift{w}{y!(z)}\widehat{\id{\{}u / z \id{\}}}
		& = &
		\lift{w}{y!(u)} \nonumber\\
	w[ \lpquote y!(z) \rpquote ] \widehat{ \id{\{}u / z \id{\}} }
		& = &
		w[ \lpquote y!(z) \rpquote ] \nonumber
\end{eqnarray}

Because the body of the process between quotes is impervious to
substitution, we get radically different answers. In fact, by
examining the first process in an input context,
e.g. $x?(z).\lift{w}{y!(z)}$, we see that the process under the lift
operator may be shaped by prefixed inputs binding a name inside it. In
this sense, the lift operator will be seen as a way to dynamically
construct processes before reifying them as names.

Finally equipped with these standard features we can present the
dynamics of the calculus.

\subsubsection{Operational semantics} 

Finally, we introduce the computational dynamics. What marks these
algebras as distinct from other more traditionally studied algebraic
structures, e.g. vector spaces or polynomial rings, is the manner in
which dynamics is captured. In traditional structures, dynamics is typically
expressed through morphisms between such structures, as in linear maps
between vector spaces or morphisms between rings. In algebras
associated with the semantics of computation, the dynamics is
expressed as part of the algebraic structure itself, through a
reduction reduction relation typically denoted by $\red$. Below, we
give a recursive presentation of this relation for the calculus used
in the encoding.

$\red \subseteq \pi \times \pi$
$\red : \pi \to \mathcal{P}(\pi)$

\begin{mathpar}
  \inferrule* [lab=Comm] { \textsf{match}( x_{src}, x_{trgt} ) } { x_{trgt}?(y)P \; | \; x_{src}!\langle {Q} \rangle \red P\{\quotep{Q}/y}\} }
  \and \\
  \inferrule* [lab=Par] {{P} \red {P}'} {{{P} | {Q}} \red {{P}' | {Q}}}
  \and
  \inferrule* [lab=Equiv]{{{P} \scong {P}'} \andalso {{P}' \red {Q}'} \andalso {{Q}' \scong {Q}}}{{P} \red {Q}}
\end{mathpar}

\begin{eqnarray*}
  match_{\equiv} (\quotep{P},\quotep{Q}) & := & P \equiv Q \\
  match_{\dagger}(\quotep{P},\quotep{Q}) & := & \forall R. P|Q \red^{*} R => R \red^{*} 0 \\
  match_{K}(\quotep{P},\quotep{Q}) & := & K \mbox{ for some context } K
\end{eqnarray*}

$u?(x)P | u!\langle Q \rangle \red P\{\quotep{Q}/x\}$

%We write $\wred$ for $\red^*$, and $P\red$ if $\exists Q $ such that $ P \red Q$.
We write $P\red$ if $\exists Q $ such that $ P \red Q$ and $P\not\red$, otherwise.

\section{Replication}

As mentioned before, it is known that replication (and hence
recursion) can be implemented in a higher-order process algebra
\cite{SangiorgiWalker}. As our first example of calculation with the
machinery thus far presented we give the construction explicitly in
the {\rhoc}.

\begin{eqnarray}
	D_{x} & := & \prefix{x}{y}{(\binpar{\outputp{x}{y}}{@{y}})} \nonumber\\
	\bangp_{x}{P} & := & \binpar{{x}!\langle{\binpar{D_{x}}{P}}\rangle}{D_{x}} \nonumber
\end{eqnarray}

\begin{eqnarray}
	\bangp_{x}{P} & & \nonumber\\
	=
	& {x}!\langle{(\prefix{x}{y}{(\outputp{x}{y} | @{y})) | P}}\rangle 
	      | \prefix{x}{y}{(\outputp{x}{y} | @{y})} & \nonumber\\
	\red
	& (\outputp{x}{y} | @{y})\substn{\quotep{(\prefix{x}{y}{(@{y} | \outputp{x}{y})) | P}}}{y} & \nonumber\\
	=
	& \outputp{x}{\quotep{(\prefix{x}{y}{(\outputp{x}{y} | @{y})) | P}}}
	  | {(\prefix{x}{y}{(\outputp{x}{y} | @{y})) | P}} & \nonumber\\
	\red
	& \ldots & \nonumber\\
	\red^*
	& P | P | \ldots & \nonumber
\end{eqnarray}

Of course, this encoding, as an implementation, runs away, unfolding
$\bangp{P}$ eagerly. A lazier and more implementable replication
operator, restricted to input-guarded processes, may be obtained as follows.

\begin{eqnarray}
\bangp{\prefix{u}{v}{P}} 
	:= 
	\binpar{\lift{x}{\prefix{u}{v}{(\binpar{D(x)}{P})}}}{D(x)} \nonumber
\end{eqnarray}

\begin{remark}
  Note that the lazier definition still does not deal with summation
  or mixed summation (i.e. sums over input and output). The reader is
  invited to construct definitions of replication that deal with these
  features. 

  Further, the definitions are parameterized in a name, $x$. Can you,
  gentle reader, make a definition that eliminates this parameter and
  guarantees no accidental interaction between the replication
  machinery and the process being replicated -- i.e. no accidental
  sharing of names used by the process to get its work done and the
  name(s) used by the replication to effect copying. This latter
  revision of the definition of replication is crucial to obtaining
  the expected identity $!!P \sim !P$.
\end{remark}

\begin{remark}\label{rem:paradoxical_combinator}
  The reader familiar with the lambda calculus will have noticed the
  similarity between $D$ and the paradoxical combinator.

  [Ed. note: the existence of this seems to suggest we have to be more
  restrictive on the set of processes and names we admit if we are to
  support no-cloning.]
\end{remark}

\subsubsection{Bisimulation}

The computational dynamics gives rise to another kind of equivalence,
the equivalence of computational behavior. As previously mentioned
this is typically captured \emph{via} some form of bisimulation.

% The notion we use in this paper is weak barbed bisimulation
% \cite{milner91polyadicpi}.

The notion we use in this paper is derived from weak barbed
bisimulation \cite{milner91polyadicpi}. 

\begin{definition}
An \emph{observation relation}, $\downarrow_{\mathcal N}$, over a set
of names, $\mathcal N$, is the smallest relation satisfying the rules
below.

\infrule[Out-barb]{y \in {\mathcal N}, \; x \nameeq y}
		  {\outputp{x}{v} \downarrow_{\mathcal N} x}
\infrule[Par-barb]{\mbox{$P\downarrow_{\mathcal N} x$ or $Q\downarrow_{\mathcal N} x$}}
		  {\binpar{P}{Q} \downarrow_{\mathcal N} x}

We write $P \Downarrow_{\mathcal N} x$ if there is $Q$ such that 
$P \wred Q$ and $Q \downarrow_{\mathcal N} x$.
\end{definition}

\begin{definition}
%\label{def.bbisim}
An  ${\mathcal N}$-\emph{barbed bisimulation} over a set of names, ${\mathcal N}$, is a symmetric binary relation 
${\mathcal S}_{\mathcal N}$ between agents such that $P\rel{S}_{\mathcal N}Q$ implies:
\begin{enumerate}
\item If $P \red P'$ then $Q \wred Q'$ and $P'\rel{S}_{\mathcal N} Q'$.
\item If $P\downarrow_{\mathcal N} x$, then $Q\Downarrow_{\mathcal N} x$.
\end{enumerate}
$P$ is ${\mathcal N}$-barbed bisimilar to $Q$, written
$P \wbbisim_{\mathcal N} Q$, if $P \rel{S}_{\mathcal N} Q$ for some ${\mathcal N}$-barbed bisimulation ${\mathcal S}_{\mathcal N}$.
\end{definition}

$\mathcal{R} \subseteq \pi \times \pi$

$P \mathcal{R} Q => \forall P'. P \red P' \Rightarrow \exists Q'. Q \red Q', P' \mathcal{R} Q'$

$P \vdash x \Rightarrow Q \vdash x$

\begin{mathpar}
  \inferrule*[lab=Out-barb]{x \nameeq y}{{y}!\langle{Q}\rangle \vdash x}
  \and
  \inferrule*[lab=Par-barb]{\mbox{$P\vdash x$ or $Q\vdash x$}}{\binpar{P}{Q} \vdash x}
\end{mathpar}

\subsubsection{Contexts}

One of the principle advantages of computational calculi like the
$\pi$-calculus is a well-defined notion of context,
contextual-equivalence and a correlation between
contextual-equivalence and notions of bisimulation. The notion of
context allows the decomposition of a process into (sub-)process and
its syntactic environment, its context. Thus, a context may be
thought of as a process with a ``hole'' (written $\Box$) in it. The
application of a context $M$ to a process $P$, written $M[P]$, is
tantamount to filling the hole in $M$ with $P$. In this paper we do
not need the full weight of this theory, but do make use of the notion
of context in the proof the main theorem. 

\begin{mathpar}
  \inferrule* [lab=summation] {} {{M_{M},M_{N}} \bc \Box \;|\; x.M_{A} \;|\; M_{M}+M_{N}}
  \and
  \inferrule* [lab=agent] {} {{M_{A}} \bc (\vec{x})M_{P} \;| \; \clift{P_0,\ldots,M_{P},\ldots,P_N}}
  \and \\
  \inferrule* [lab=process] {} {{M_{P}} \bc M_{N} \;| \;P|M_{P} }
\end{mathpar} 

\begin{mathpar}
  \inferrule* [lab=sychronization] {} {M_{N} \bc \Box \;|\; x?M_{F} \;|\; x!M_{C}}
  \and
  \inferrule* [lab=abstraction] {} {{M_{F}} \bc (x)M_{P} }
  \and
  \inferrule* [lab=concretion] {} {{M_{C}} \bc \langle M_{P} \rangle }
  \and \\
  \inferrule* [lab=process] {} {{M_{P}} \bc M_{N} \;| \;P|M_{P} }
\end{mathpar}

\begin{definition}[contextual application] Given a context $M$, and
  process $P$, we define the \emph{contextual application}, $M[P] :=
  M\{P/\Box\}$. That is, the contextual application of M to P is the
  substitution of $P$ for $\Box$ in $M$.
\end{definition}

$\meaningof{-} : L \to \mathcal{P}(\pi)$

\begin{mathpar}
  \inferrule* [lab=collection] {} {\meaningof{true} = \pi, \and \meaningof{~E} = \pi \setminus \meaningof{E}, \and \meaningof{E_{1} \& E_{2}} = \meaningof{E_{1}} \cap \meaningof{E_{2}}}
\end{mathpar}

\begin{mathpar}
  \inferrule* [lab=structure] {} {\meaningof{0} = \{ P \in \pi | P \equiv 0 \}, \and \\ \meaningof{E_1 | E_2} = \{ P \in \pi | P \equiv P_{1} | P_{2}, P_{1} \in \meaningof{E_{1}}, P_{2} \in \meaningof{E_2}\} }
\end{mathpar}

\begin{mathpar}
 \inferrule* [lab=behavior] {} {\meaningof{\langle a?b \rangle E} = \{ P \in \pi | P \equiv Q | u?(y)P', \\ \and \\\\ \and \\ \;\;\; u \in \meaningof{a}, \forall z.P'\{z/y\} \in \meaningof{E\{z/b\}}\}, \and \\ \meaningof{a!E} = \{ P \in \pi | P \equiv Q | x!\langle P' \rangle, x \in \meaningof{a} P' \in \meaningof{E}\} }
\end{mathpar}

\begin{mathpar}
 \inferrule* [lab=nominal] {} {\meaningof{\quotep{E}} = \{ \quotep{P} \in \quotep{\pi} | P \in \meaningof{E} \}, \and \meaningof{\quotep{P}} = \{ \quotep{Q} \in \quotep{\pi} | P \equiv Q \} \and \\ \meaningof{@\quotep{E}} = \{ P \in \pi | P \equiv @x, x \in \meaningof{E} \}}
\end{mathpar}

\begin{eqnarray*}
  \\
  \meaningof{-} : TS \to ST
\end{eqnarray*}

\begin{eqnarray*}
  \\
  L : TS \to ST
\end{eqnarray*}

\begin{eqnarray*}
  \\
  P \models E \iff P \in \meaningof{E}
\end{eqnarray*}

\begin{eqnarray*}
  P \approx_{L} Q \iff \forall E \in L. P \models E \iff Q \models E
\end{eqnarray*}

\begin{eqnarray*}
  P \approx_{K} Q
\end{eqnarray*}

\begin{eqnarray*}
  P \approx Q
\end{eqnarray*}

$\approx_{K} = \approx = \approx_{L}$

\subsubsection{Contextual duality}

Note that contexts extend the quotation operation to a family of
operations from processes to names. Given a context, $M$, we can
define a \emph{nominal context}, $\quotep{M}$ by $\quotep{M}[P] :=
\quotep{M[P]}$. To foreshadow what is to come we observe that these
operations enjoy a duality with processes very much like the duality
between vectors and maps from vectors to scalars.

Further, because the calculus is essentially higher-order, we have a
correspondence between contexts and processes. More specifically,
given a name $x$ and a context $M$ we can construct $M^{*}_{x}$ such
that 

\begin{mathpar}
  M^{*}_{x} | \lift{x}{P} \red M[P]
\end{mathpar}

namely,

\begin{mathpar}
  M^{*}_{x} := x?(u).M[\dropn{u}]
\end{mathpar}

The dependence of $M^{*}_{x}$ on a name makes it an abstraction, 

\begin{mathpar}
  M^{*} := (x)x?(u).M[\dropn{u}]
\end{mathpar}

\subsection{Additional notation}

It will sometimes be convenient to denote the process a name
quotes. We already have the notation $x = \quotep{P}$, but it will be
convenient to introduce an alternate notation, $\procn{x}$, when we
want to emphasize the connection to the use of the name. Note that, by
virtue of name equivalence, $\quotep{\procn{x}} \nameeq x$; so, the
notation is consistent with previous definitions.

Further, because names have structure it is possible to effect
substitutions on the basis of that structure. This means we need to
upgrade our notation for substitutions, which we accomplish by
adapting comprehension notation. Thus,

\begin{mathpar}
  P\{ y / x : x \in S \}
\end{mathpar}

is interpreted to mean the process derived from P by replacing (in a
capture-avoiding manner) each occurrence of $x$ in $S$ by $y$. For example,

\begin{mathpar}
  P\{ \quotep{\procn{x}|\procn{x}} / x : x \in \freenames{P} \}
\end{mathpar}

will replace each (occurrence) of a free name $x$ in $P$ by
$\quotep{\procn{x}|\procn{x}}$.

Also, we will avail ourselves of the notation $x^{L}$ and $x^{R}$ to
denote injections of a name into disjoint copies of the name
space. There are numerous ways to accomplish this. One example can be
found in \cite{MeredithR05}. This notation overloads to vectors of
names: $\vec{x}^{\pi} := (x_{i}^{\pi} \; : \; 0 \leq i < |\vec{x}| )$ where $\pi \in \{L,R\}$.

We also use $P^{\Box} := P|\Box$.

In \cite{MeredithR05} an interpretation of the new operator is
given. It turns out that there are several possible interpretations
all enjoying the requisite algebraic properties of the operator (see
\cite{milner91polyadicpi}). We will therefore make liberal use of
$(\nu\; \vec{x})P$.

% subsection the_syntax_and_semantics_of_the_notation_system (end)   

\input{qm2pi.qmops} 

\input{qm2pi.sterngerlach} 

\input{qm2pi.metric} 

% section concurrent_process_calculi (end)

%\input{qm2pi.proofsketch}

% section proof sketch (end)

%\input{qm2pi.slviaknots} 

% section spatial logic via knots (end)

\input{qm2pi.conclusion}

% section conclusion (end)

%\input{qm2pi.dtcodes} 

% section wiring algorithm (end)

\input{qm2pi.ack} 

% section acknowledgments (end)

\newpage


\bibliographystyle{plain}   
\bibliography{../../biblios/main.bib}

\input{qm2pi.rhodetails}

\end{document}

 

% section notation (end)

\input{qm2pi.process.calculi} 

% section concurrent_process_calculi_and_spatial_logics_ (end)
    
%\documentclass[12pt]{llncs}
%\documentclass{jktr}

\usepackage[pdftex]{hyperref}                   
\usepackage {listings}
\usepackage {mathpartir}
\usepackage{bcprules}
%\usepackage{listings}
                       
\usepackage{graphicx} 
%\usepackage[margins=2.5cm,nohead,nofoot]{geometry}
%\usepackage{geometry}
\usepackage{amsfonts}
\usepackage{amstext}
\usepackage{latexsym}
\usepackage{amssymb}
\usepackage{color}


%\include{myPreamble}
\include{qm2pi.local} 

%\ifpdf
%\usepackage[pdftex]{graphicx}
%\else
%\usepackage{graphicx}
%\fi

 % \ifpdf
%  \usepackage{pdfsync}
%  \if


%\title{Brief Article}
%\author{David F. Snyder}
%\author{L.G. Meredith}

%\address{Dept. of Math., Texas State University--San Marcos, San Marcos, TX 78666}
       
\pagestyle{empty}


\begin{document}

\lstset{language=[Objective]Caml,frame=shadowbox}

\input{qm2pi.front}

% section front matter (end)

\input{qm2pi.intro} 
 
% section introduction (end)

% \input{qm2pi.knotations} 

% section notation (end)

\input{qm2pi.process.calculi} 

% section concurrent_process_calculi_and_spatial_logics_ (end)
    
%\input{qm2pi.knots2pi} 

%\input{qm2pi.trefoil} 

%\input{qm2pi.mainthm} 

% subsection basic_interpretation (end)

%\input{qm2pi.rho.presentation} 
\subsection{The syntax and semantics of the notation system}\label{sub:the_syntax_and_semantics_of_the_notation_system} % (fold)

We now summarize a technical presentation of the calculus that
embodies our theory of dynamics. The typical presentation of such a
calculus follows the style of giving generators and relations on
them. The grammar, below, describing term constructors, freely
generates the set of processes, $\Proc$. This set is then quotiented
by a relation known as structural congruence and it is over this set
that the notion of dynamics is expressed. This presentation is
essentially that of \cite{MeredithR05} with the addition of
polyadicity and summation. For readability we have relegated some of
the technical subtleties to an appendix.

\subsubsection{Process grammar}\label{subsub:process_grammar}

\begin{mathpar}
  \inferrule* [lab=synchronization] {} {{M} \bc \pzero \;|\; x?F \;|\; x!C }
  \and
  \inferrule* [lab=abstraction] {} {{F} \bc (x)P}
  \and
  \inferrule* [lab=concretion] {} {{C} \bc \langle Q \rangle}
  \and
  \inferrule* [lab=process] {} {{P,Q} \bc M \;| \;P|Q \;|\; @{x}}
  \and
  \inferrule* [lab=name] {} {{x} \bc \quotep{P}}
\end{mathpar} 

Note that $\vec{x}$ (resp. $\vec{P}$) denotes a vector of names
(resp. processes) of length $|\vec{x}|$ (resp. $|\vec{P}|$). We adopt
the following useful abbreviations.

\begin{mathpar}
   x?(\vec{y}).P := x.(\vec{y})P \and  x\clift{\vec{P}} := x.\clift{\vec{P}}
   \and x!(y) := \lift{x}{\dropn{y}}
   \and \Pi_{i=0}^{n-1}P_i := P_0 | \ldots | P_{n-1}
\end{mathpar}

\subsubsection{Structural congruence}

\paragraph{Free and bound names and alpha-equivalence.} At the
core of structural equivalence is alpha-equivalence which identifies
process that are the same up to a change of variable. Formally, we
recognize the distinction between free and bound names. The free names
of a process, $\freenames{P}$, may be calculated recursively as
follows:

\begin{mathpar}
\freenames{\pzero} := \emptyset
  \and \\
  \freenames{x?(y).P} := \{ x \} \cup (\freenames{P} \setminus \{ y \})
  \and 
  \freenames{x!\langle P \rangle} := \{ x \} \cup \{ P \} 
  \and \\
  \freenames{P|Q} := \freenames{P} \cup \freenames{Q}
  \and \\
  \freenames{@{x}} := \{ x \}
\end{mathpar}

$\pi$
$\quotep{\pi}$

$\freenames{-} : \pi \to \mathcal{P}(\quotep{\pi})$

\begin{eqnarray*}
  \freenames{\pzero} & := & \emptyset \\
  \freenames{x?(y).P} & := & \{ x \} \cup (\freenames{P} \setminus \{ y \}) \\
  \freenames{x!\langle P \rangle} & := & \{ x \} \cup \{ P \} \\
  \freenames{P|Q} & := & \freenames{P} \cup \freenames{Q} \\
  \freenames{\dropn{x}} & := & \{ x \}
\end{eqnarray*}

The bound names of a process, $\boundnames{P}$, are those names occurring in $P$
that are not free. For example, in $x?(y).0$, the name $x$ is free, while $y$ is bound.

\begin{mathpar}
  \inferrule* [lab=monoidal-laws] {} { P|Q \equiv Q|P \and P|0 \equiv P \and P|(Q|R) \equiv (P|Q)|R }
\end{mathpar}

\begin{mathpar}
  \inferrule* [lab=alpha-equivalence] {} { (x)P \equiv (y)P\{y/x\} \and y \not\in \freenames{P} }
\end{mathpar}

\begin{definition}
Then two processes, $P,Q$, are alpha-equivalent if $P = Q\{\vec{y}/\vec{x}\}$ for
some $\vec{x} \in \boundnames{Q},\vec{y} \in \boundnames{P}$, where $Q\{\vec{y}/\vec{x}\}$
denotes the capture-avoiding substitution of $\vec{y}$ for $\vec{x}$ in $Q$.
\end{definition}

\begin{definition}
  The {\em structural congruence} \cite{SangiorgiWalker} , $\equiv$,
  between processes is the least congruence containing
  alpha-equivalence, satisfying the abelian monoid laws
  (associativity, commutativity and $\pzero$ as identity) for parallel
  composition $|$ and for summation $+$.
\end{definition}

\subsection{Name equivalence}

We take name equivalence, written $\nameeq$, to be the smallest
equivalence relation generated by the following rules.

\begin{mathpar}
\inferrule*[lab=Quote-drop]
{ }
{ \quotep{@{x}} \nameeq x }

\inferrule*[lab=Struct-equiv]
{ P \scong Q }
{ \quotep{P} \nameeq \quotep{Q} }
\end{mathpar}

The astute reader will have noticed that the mutual recursion of names
and processes imposes a mutual recursion on alpha-equivalence and
structural equivalence via name-equivalence. Fortunately, all of this
works out pleasantly and we may calculate in the natural way, free of
concern. The reader interested in the details is referred to the
appendix \ref{appendix:rho_details}.

\subsection{Substitution}

We use $\Proc$ for the set of processes, $\QProc$ for the set of
names, and $\id{\{}\vec{y} / \vec{x} \id{\}}$ to denote partial maps,
$s : \QProc \rightarrow \QProc$. A map, $s$ lifts, uniquely, to a map
on process terms, $\widehat{s} : \Proc \rightarrow \Proc$ by the
following equations.

\begin{mathpar}
  (0) \psubstp{Q}{P} := 0 \\
  (R \juxtap S) \psubstp{Q}{P}
  :=    
  (R)\psubstp{Q}{P} \juxtap (S) \psubstp{Q}{P} \\
  (x?(y).R) \psubstp{Q}{P}    
  :=    
  (x)\substp{Q}{P} (z)\concat( (R \psubstn{z}{y}) \psubstp{Q}{P} ) \\
  (\lift{x}{R}) \psubstp{Q}{P}  
  :=
  \lift{(x)\substp{Q}{P}}{ R \psubstp{Q}{P} } \\
%   (\dropn{x})  \psubstp{Q}{P}       
%   := 
%   \left\{ 
%     \begin{array}{ccc} 
%       \dropn{\quotep{Q}} & & x \nameeq \quotep{P} \\
%       \dropn{x} & & otherwise \\
%     \end{array}
%   \right. 
  (\dropn{x})  \psubstp{Q}{P}       
  := 
  \left\{ 
    \begin{array}{ccc} 
      Q & & x \nameeq \quotep{P} \\
      \dropn{x} & & otherwise \\
    \end{array}
  \right.
\end{mathpar}
 

where

\begin{eqnarray}
  (x)\id{\{} \lpquote Q \rpquote / \lpquote P \rpquote \id{\}}            = 
  \left\{ 
    \begin{array}{ccc}
      \lpquote Q \rpquote & & x \nameeq \lpquote P \rpquote \\
      x & & otherwise \\
    \end{array}
  \right. \nonumber
\end{eqnarray}

and $z$ is chosen distinct from $\quotep{P}$, $\quotep{Q}$, the free
names in $Q$, and all the names in $R$. Our $\alpha$-equivalence will
be built in the standard way from this substitution.

\begin{remark}\label{rem:no_self_referential_names}
  One consequence of these definitions is that $\forall P. \quotep{P}
  \not\in \freenames{P}$.
\end{remark}

\subsection{ Dynamic quote: an example }

Anticipating something of what's to come, consider applying the
substitution, $\widehat{\id{\{}u / z \id{\}}}$, to the following pair
of processes, $\lift{w}{y!(z)}$ and $w[ \lpquote y!(z) \rpquote ]$.

\begin{eqnarray}
	\lift{w}{y!(z)}\widehat{\id{\{}u / z \id{\}}}
		& = &
		\lift{w}{y!(u)} \nonumber\\
	w[ \lpquote y!(z) \rpquote ] \widehat{ \id{\{}u / z \id{\}} }
		& = &
		w[ \lpquote y!(z) \rpquote ] \nonumber
\end{eqnarray}

Because the body of the process between quotes is impervious to
substitution, we get radically different answers. In fact, by
examining the first process in an input context,
e.g. $x?(z).\lift{w}{y!(z)}$, we see that the process under the lift
operator may be shaped by prefixed inputs binding a name inside it. In
this sense, the lift operator will be seen as a way to dynamically
construct processes before reifying them as names.

Finally equipped with these standard features we can present the
dynamics of the calculus.

\subsubsection{Operational semantics} 

Finally, we introduce the computational dynamics. What marks these
algebras as distinct from other more traditionally studied algebraic
structures, e.g. vector spaces or polynomial rings, is the manner in
which dynamics is captured. In traditional structures, dynamics is typically
expressed through morphisms between such structures, as in linear maps
between vector spaces or morphisms between rings. In algebras
associated with the semantics of computation, the dynamics is
expressed as part of the algebraic structure itself, through a
reduction reduction relation typically denoted by $\red$. Below, we
give a recursive presentation of this relation for the calculus used
in the encoding.

$\red \subseteq \pi \times \pi$
$\red : \pi \to \mathcal{P}(\pi)$

\begin{mathpar}
  \inferrule* [lab=Comm] { \textsf{match}( x_{src}, x_{trgt} ) } { x_{trgt}?(y)P \; | \; x_{src}!\langle {Q} \rangle \red P\{\quotep{Q}/y}\} }
  \and \\
  \inferrule* [lab=Par] {{P} \red {P}'} {{{P} | {Q}} \red {{P}' | {Q}}}
  \and
  \inferrule* [lab=Equiv]{{{P} \scong {P}'} \andalso {{P}' \red {Q}'} \andalso {{Q}' \scong {Q}}}{{P} \red {Q}}
\end{mathpar}

\begin{eqnarray*}
  match_{\equiv} (\quotep{P},\quotep{Q}) & := & P \equiv Q \\
  match_{\dagger}(\quotep{P},\quotep{Q}) & := & \forall R. P|Q \red^{*} R => R \red^{*} 0 \\
  match_{K}(\quotep{P},\quotep{Q}) & := & K \mbox{ for some context } K
\end{eqnarray*}

$u?(x)P | u!\langle Q \rangle \red P\{\quotep{Q}/x\}$

%We write $\wred$ for $\red^*$, and $P\red$ if $\exists Q $ such that $ P \red Q$.
We write $P\red$ if $\exists Q $ such that $ P \red Q$ and $P\not\red$, otherwise.

\section{Replication}

As mentioned before, it is known that replication (and hence
recursion) can be implemented in a higher-order process algebra
\cite{SangiorgiWalker}. As our first example of calculation with the
machinery thus far presented we give the construction explicitly in
the {\rhoc}.

\begin{eqnarray}
	D_{x} & := & \prefix{x}{y}{(\binpar{\outputp{x}{y}}{@{y}})} \nonumber\\
	\bangp_{x}{P} & := & \binpar{{x}!\langle{\binpar{D_{x}}{P}}\rangle}{D_{x}} \nonumber
\end{eqnarray}

\begin{eqnarray}
	\bangp_{x}{P} & & \nonumber\\
	=
	& {x}!\langle{(\prefix{x}{y}{(\outputp{x}{y} | @{y})) | P}}\rangle 
	      | \prefix{x}{y}{(\outputp{x}{y} | @{y})} & \nonumber\\
	\red
	& (\outputp{x}{y} | @{y})\substn{\quotep{(\prefix{x}{y}{(@{y} | \outputp{x}{y})) | P}}}{y} & \nonumber\\
	=
	& \outputp{x}{\quotep{(\prefix{x}{y}{(\outputp{x}{y} | @{y})) | P}}}
	  | {(\prefix{x}{y}{(\outputp{x}{y} | @{y})) | P}} & \nonumber\\
	\red
	& \ldots & \nonumber\\
	\red^*
	& P | P | \ldots & \nonumber
\end{eqnarray}

Of course, this encoding, as an implementation, runs away, unfolding
$\bangp{P}$ eagerly. A lazier and more implementable replication
operator, restricted to input-guarded processes, may be obtained as follows.

\begin{eqnarray}
\bangp{\prefix{u}{v}{P}} 
	:= 
	\binpar{\lift{x}{\prefix{u}{v}{(\binpar{D(x)}{P})}}}{D(x)} \nonumber
\end{eqnarray}

\begin{remark}
  Note that the lazier definition still does not deal with summation
  or mixed summation (i.e. sums over input and output). The reader is
  invited to construct definitions of replication that deal with these
  features. 

  Further, the definitions are parameterized in a name, $x$. Can you,
  gentle reader, make a definition that eliminates this parameter and
  guarantees no accidental interaction between the replication
  machinery and the process being replicated -- i.e. no accidental
  sharing of names used by the process to get its work done and the
  name(s) used by the replication to effect copying. This latter
  revision of the definition of replication is crucial to obtaining
  the expected identity $!!P \sim !P$.
\end{remark}

\begin{remark}\label{rem:paradoxical_combinator}
  The reader familiar with the lambda calculus will have noticed the
  similarity between $D$ and the paradoxical combinator.

  [Ed. note: the existence of this seems to suggest we have to be more
  restrictive on the set of processes and names we admit if we are to
  support no-cloning.]
\end{remark}

\subsubsection{Bisimulation}

The computational dynamics gives rise to another kind of equivalence,
the equivalence of computational behavior. As previously mentioned
this is typically captured \emph{via} some form of bisimulation.

% The notion we use in this paper is weak barbed bisimulation
% \cite{milner91polyadicpi}.

The notion we use in this paper is derived from weak barbed
bisimulation \cite{milner91polyadicpi}. 

\begin{definition}
An \emph{observation relation}, $\downarrow_{\mathcal N}$, over a set
of names, $\mathcal N$, is the smallest relation satisfying the rules
below.

\infrule[Out-barb]{y \in {\mathcal N}, \; x \nameeq y}
		  {\outputp{x}{v} \downarrow_{\mathcal N} x}
\infrule[Par-barb]{\mbox{$P\downarrow_{\mathcal N} x$ or $Q\downarrow_{\mathcal N} x$}}
		  {\binpar{P}{Q} \downarrow_{\mathcal N} x}

We write $P \Downarrow_{\mathcal N} x$ if there is $Q$ such that 
$P \wred Q$ and $Q \downarrow_{\mathcal N} x$.
\end{definition}

\begin{definition}
%\label{def.bbisim}
An  ${\mathcal N}$-\emph{barbed bisimulation} over a set of names, ${\mathcal N}$, is a symmetric binary relation 
${\mathcal S}_{\mathcal N}$ between agents such that $P\rel{S}_{\mathcal N}Q$ implies:
\begin{enumerate}
\item If $P \red P'$ then $Q \wred Q'$ and $P'\rel{S}_{\mathcal N} Q'$.
\item If $P\downarrow_{\mathcal N} x$, then $Q\Downarrow_{\mathcal N} x$.
\end{enumerate}
$P$ is ${\mathcal N}$-barbed bisimilar to $Q$, written
$P \wbbisim_{\mathcal N} Q$, if $P \rel{S}_{\mathcal N} Q$ for some ${\mathcal N}$-barbed bisimulation ${\mathcal S}_{\mathcal N}$.
\end{definition}

$\mathcal{R} \subseteq \pi \times \pi$

$P \mathcal{R} Q => \forall P'. P \red P' \Rightarrow \exists Q'. Q \red Q', P' \mathcal{R} Q'$

$P \vdash x \Rightarrow Q \vdash x$

\begin{mathpar}
  \inferrule*[lab=Out-barb]{x \nameeq y}{{y}!\langle{Q}\rangle \vdash x}
  \and
  \inferrule*[lab=Par-barb]{\mbox{$P\vdash x$ or $Q\vdash x$}}{\binpar{P}{Q} \vdash x}
\end{mathpar}

\subsubsection{Contexts}

One of the principle advantages of computational calculi like the
$\pi$-calculus is a well-defined notion of context,
contextual-equivalence and a correlation between
contextual-equivalence and notions of bisimulation. The notion of
context allows the decomposition of a process into (sub-)process and
its syntactic environment, its context. Thus, a context may be
thought of as a process with a ``hole'' (written $\Box$) in it. The
application of a context $M$ to a process $P$, written $M[P]$, is
tantamount to filling the hole in $M$ with $P$. In this paper we do
not need the full weight of this theory, but do make use of the notion
of context in the proof the main theorem. 

\begin{mathpar}
  \inferrule* [lab=summation] {} {{M_{M},M_{N}} \bc \Box \;|\; x.M_{A} \;|\; M_{M}+M_{N}}
  \and
  \inferrule* [lab=agent] {} {{M_{A}} \bc (\vec{x})M_{P} \;| \; \clift{P_0,\ldots,M_{P},\ldots,P_N}}
  \and \\
  \inferrule* [lab=process] {} {{M_{P}} \bc M_{N} \;| \;P|M_{P} }
\end{mathpar} 

\begin{mathpar}
  \inferrule* [lab=sychronization] {} {M_{N} \bc \Box \;|\; x?M_{F} \;|\; x!M_{C}}
  \and
  \inferrule* [lab=abstraction] {} {{M_{F}} \bc (x)M_{P} }
  \and
  \inferrule* [lab=concretion] {} {{M_{C}} \bc \langle M_{P} \rangle }
  \and \\
  \inferrule* [lab=process] {} {{M_{P}} \bc M_{N} \;| \;P|M_{P} }
\end{mathpar}

\begin{definition}[contextual application] Given a context $M$, and
  process $P$, we define the \emph{contextual application}, $M[P] :=
  M\{P/\Box\}$. That is, the contextual application of M to P is the
  substitution of $P$ for $\Box$ in $M$.
\end{definition}

$\meaningof{-} : L \to \mathcal{P}(\pi)$

\begin{mathpar}
  \inferrule* [lab=collection] {} {\meaningof{true} = \pi, \and \meaningof{~E} = \pi \setminus \meaningof{E}, \and \meaningof{E_{1} \& E_{2}} = \meaningof{E_{1}} \cap \meaningof{E_{2}}}
\end{mathpar}

\begin{mathpar}
  \inferrule* [lab=structure] {} {\meaningof{0} = \{ P \in \pi | P \equiv 0 \}, \and \\ \meaningof{E_1 | E_2} = \{ P \in \pi | P \equiv P_{1} | P_{2}, P_{1} \in \meaningof{E_{1}}, P_{2} \in \meaningof{E_2}\} }
\end{mathpar}

\begin{mathpar}
 \inferrule* [lab=behavior] {} {\meaningof{\langle a?b \rangle E} = \{ P \in \pi | P \equiv Q | u?(y)P', \\ \and \\\\ \and \\ \;\;\; u \in \meaningof{a}, \forall z.P'\{z/y\} \in \meaningof{E\{z/b\}}\}, \and \\ \meaningof{a!E} = \{ P \in \pi | P \equiv Q | x!\langle P' \rangle, x \in \meaningof{a} P' \in \meaningof{E}\} }
\end{mathpar}

\begin{mathpar}
 \inferrule* [lab=nominal] {} {\meaningof{\quotep{E}} = \{ \quotep{P} \in \quotep{\pi} | P \in \meaningof{E} \}, \and \meaningof{\quotep{P}} = \{ \quotep{Q} \in \quotep{\pi} | P \equiv Q \} \and \\ \meaningof{@\quotep{E}} = \{ P \in \pi | P \equiv @x, x \in \meaningof{E} \}}
\end{mathpar}

\begin{eqnarray*}
  \\
  \meaningof{-} : TS \to ST
\end{eqnarray*}

\begin{eqnarray*}
  \\
  L : TS \to ST
\end{eqnarray*}

\begin{eqnarray*}
  \\
  P \models E \iff P \in \meaningof{E}
\end{eqnarray*}

\begin{eqnarray*}
  P \approx_{L} Q \iff \forall E \in L. P \models E \iff Q \models E
\end{eqnarray*}

\begin{eqnarray*}
  P \approx_{K} Q
\end{eqnarray*}

\begin{eqnarray*}
  P \approx Q
\end{eqnarray*}

$\approx_{K} = \approx = \approx_{L}$

\subsubsection{Contextual duality}

Note that contexts extend the quotation operation to a family of
operations from processes to names. Given a context, $M$, we can
define a \emph{nominal context}, $\quotep{M}$ by $\quotep{M}[P] :=
\quotep{M[P]}$. To foreshadow what is to come we observe that these
operations enjoy a duality with processes very much like the duality
between vectors and maps from vectors to scalars.

Further, because the calculus is essentially higher-order, we have a
correspondence between contexts and processes. More specifically,
given a name $x$ and a context $M$ we can construct $M^{*}_{x}$ such
that 

\begin{mathpar}
  M^{*}_{x} | \lift{x}{P} \red M[P]
\end{mathpar}

namely,

\begin{mathpar}
  M^{*}_{x} := x?(u).M[\dropn{u}]
\end{mathpar}

The dependence of $M^{*}_{x}$ on a name makes it an abstraction, 

\begin{mathpar}
  M^{*} := (x)x?(u).M[\dropn{u}]
\end{mathpar}

\subsection{Additional notation}

It will sometimes be convenient to denote the process a name
quotes. We already have the notation $x = \quotep{P}$, but it will be
convenient to introduce an alternate notation, $\procn{x}$, when we
want to emphasize the connection to the use of the name. Note that, by
virtue of name equivalence, $\quotep{\procn{x}} \nameeq x$; so, the
notation is consistent with previous definitions.

Further, because names have structure it is possible to effect
substitutions on the basis of that structure. This means we need to
upgrade our notation for substitutions, which we accomplish by
adapting comprehension notation. Thus,

\begin{mathpar}
  P\{ y / x : x \in S \}
\end{mathpar}

is interpreted to mean the process derived from P by replacing (in a
capture-avoiding manner) each occurrence of $x$ in $S$ by $y$. For example,

\begin{mathpar}
  P\{ \quotep{\procn{x}|\procn{x}} / x : x \in \freenames{P} \}
\end{mathpar}

will replace each (occurrence) of a free name $x$ in $P$ by
$\quotep{\procn{x}|\procn{x}}$.

Also, we will avail ourselves of the notation $x^{L}$ and $x^{R}$ to
denote injections of a name into disjoint copies of the name
space. There are numerous ways to accomplish this. One example can be
found in \cite{MeredithR05}. This notation overloads to vectors of
names: $\vec{x}^{\pi} := (x_{i}^{\pi} \; : \; 0 \leq i < |\vec{x}| )$ where $\pi \in \{L,R\}$.

We also use $P^{\Box} := P|\Box$.

In \cite{MeredithR05} an interpretation of the new operator is
given. It turns out that there are several possible interpretations
all enjoying the requisite algebraic properties of the operator (see
\cite{milner91polyadicpi}). We will therefore make liberal use of
$(\nu\; \vec{x})P$.

% subsection the_syntax_and_semantics_of_the_notation_system (end)   

\input{qm2pi.qmops} 

\input{qm2pi.sterngerlach} 

\input{qm2pi.metric} 

% section concurrent_process_calculi (end)

%\input{qm2pi.proofsketch}

% section proof sketch (end)

%\input{qm2pi.slviaknots} 

% section spatial logic via knots (end)

\input{qm2pi.conclusion}

% section conclusion (end)

%\input{qm2pi.dtcodes} 

% section wiring algorithm (end)

\input{qm2pi.ack} 

% section acknowledgments (end)

\newpage


\bibliographystyle{plain}   
\bibliography{../../biblios/main.bib}

\input{qm2pi.rhodetails}

\end{document}

 

%\documentclass[12pt]{llncs}
%\documentclass{jktr}

\usepackage[pdftex]{hyperref}                   
\usepackage {listings}
\usepackage {mathpartir}
\usepackage{bcprules}
%\usepackage{listings}
                       
\usepackage{graphicx} 
%\usepackage[margins=2.5cm,nohead,nofoot]{geometry}
%\usepackage{geometry}
\usepackage{amsfonts}
\usepackage{amstext}
\usepackage{latexsym}
\usepackage{amssymb}
\usepackage{color}


%\include{myPreamble}
\include{qm2pi.local} 

%\ifpdf
%\usepackage[pdftex]{graphicx}
%\else
%\usepackage{graphicx}
%\fi

 % \ifpdf
%  \usepackage{pdfsync}
%  \if


%\title{Brief Article}
%\author{David F. Snyder}
%\author{L.G. Meredith}

%\address{Dept. of Math., Texas State University--San Marcos, San Marcos, TX 78666}
       
\pagestyle{empty}


\begin{document}

\lstset{language=[Objective]Caml,frame=shadowbox}

\input{qm2pi.front}

% section front matter (end)

\input{qm2pi.intro} 
 
% section introduction (end)

% \input{qm2pi.knotations} 

% section notation (end)

\input{qm2pi.process.calculi} 

% section concurrent_process_calculi_and_spatial_logics_ (end)
    
%\input{qm2pi.knots2pi} 

%\input{qm2pi.trefoil} 

%\input{qm2pi.mainthm} 

% subsection basic_interpretation (end)

%\input{qm2pi.rho.presentation} 
\subsection{The syntax and semantics of the notation system}\label{sub:the_syntax_and_semantics_of_the_notation_system} % (fold)

We now summarize a technical presentation of the calculus that
embodies our theory of dynamics. The typical presentation of such a
calculus follows the style of giving generators and relations on
them. The grammar, below, describing term constructors, freely
generates the set of processes, $\Proc$. This set is then quotiented
by a relation known as structural congruence and it is over this set
that the notion of dynamics is expressed. This presentation is
essentially that of \cite{MeredithR05} with the addition of
polyadicity and summation. For readability we have relegated some of
the technical subtleties to an appendix.

\subsubsection{Process grammar}\label{subsub:process_grammar}

\begin{mathpar}
  \inferrule* [lab=synchronization] {} {{M} \bc \pzero \;|\; x?F \;|\; x!C }
  \and
  \inferrule* [lab=abstraction] {} {{F} \bc (x)P}
  \and
  \inferrule* [lab=concretion] {} {{C} \bc \langle Q \rangle}
  \and
  \inferrule* [lab=process] {} {{P,Q} \bc M \;| \;P|Q \;|\; @{x}}
  \and
  \inferrule* [lab=name] {} {{x} \bc \quotep{P}}
\end{mathpar} 

Note that $\vec{x}$ (resp. $\vec{P}$) denotes a vector of names
(resp. processes) of length $|\vec{x}|$ (resp. $|\vec{P}|$). We adopt
the following useful abbreviations.

\begin{mathpar}
   x?(\vec{y}).P := x.(\vec{y})P \and  x\clift{\vec{P}} := x.\clift{\vec{P}}
   \and x!(y) := \lift{x}{\dropn{y}}
   \and \Pi_{i=0}^{n-1}P_i := P_0 | \ldots | P_{n-1}
\end{mathpar}

\subsubsection{Structural congruence}

\paragraph{Free and bound names and alpha-equivalence.} At the
core of structural equivalence is alpha-equivalence which identifies
process that are the same up to a change of variable. Formally, we
recognize the distinction between free and bound names. The free names
of a process, $\freenames{P}$, may be calculated recursively as
follows:

\begin{mathpar}
\freenames{\pzero} := \emptyset
  \and \\
  \freenames{x?(y).P} := \{ x \} \cup (\freenames{P} \setminus \{ y \})
  \and 
  \freenames{x!\langle P \rangle} := \{ x \} \cup \{ P \} 
  \and \\
  \freenames{P|Q} := \freenames{P} \cup \freenames{Q}
  \and \\
  \freenames{@{x}} := \{ x \}
\end{mathpar}

$\pi$
$\quotep{\pi}$

$\freenames{-} : \pi \to \mathcal{P}(\quotep{\pi})$

\begin{eqnarray*}
  \freenames{\pzero} & := & \emptyset \\
  \freenames{x?(y).P} & := & \{ x \} \cup (\freenames{P} \setminus \{ y \}) \\
  \freenames{x!\langle P \rangle} & := & \{ x \} \cup \{ P \} \\
  \freenames{P|Q} & := & \freenames{P} \cup \freenames{Q} \\
  \freenames{\dropn{x}} & := & \{ x \}
\end{eqnarray*}

The bound names of a process, $\boundnames{P}$, are those names occurring in $P$
that are not free. For example, in $x?(y).0$, the name $x$ is free, while $y$ is bound.

\begin{mathpar}
  \inferrule* [lab=monoidal-laws] {} { P|Q \equiv Q|P \and P|0 \equiv P \and P|(Q|R) \equiv (P|Q)|R }
\end{mathpar}

\begin{mathpar}
  \inferrule* [lab=alpha-equivalence] {} { (x)P \equiv (y)P\{y/x\} \and y \not\in \freenames{P} }
\end{mathpar}

\begin{definition}
Then two processes, $P,Q$, are alpha-equivalent if $P = Q\{\vec{y}/\vec{x}\}$ for
some $\vec{x} \in \boundnames{Q},\vec{y} \in \boundnames{P}$, where $Q\{\vec{y}/\vec{x}\}$
denotes the capture-avoiding substitution of $\vec{y}$ for $\vec{x}$ in $Q$.
\end{definition}

\begin{definition}
  The {\em structural congruence} \cite{SangiorgiWalker} , $\equiv$,
  between processes is the least congruence containing
  alpha-equivalence, satisfying the abelian monoid laws
  (associativity, commutativity and $\pzero$ as identity) for parallel
  composition $|$ and for summation $+$.
\end{definition}

\subsection{Name equivalence}

We take name equivalence, written $\nameeq$, to be the smallest
equivalence relation generated by the following rules.

\begin{mathpar}
\inferrule*[lab=Quote-drop]
{ }
{ \quotep{@{x}} \nameeq x }

\inferrule*[lab=Struct-equiv]
{ P \scong Q }
{ \quotep{P} \nameeq \quotep{Q} }
\end{mathpar}

The astute reader will have noticed that the mutual recursion of names
and processes imposes a mutual recursion on alpha-equivalence and
structural equivalence via name-equivalence. Fortunately, all of this
works out pleasantly and we may calculate in the natural way, free of
concern. The reader interested in the details is referred to the
appendix \ref{appendix:rho_details}.

\subsection{Substitution}

We use $\Proc$ for the set of processes, $\QProc$ for the set of
names, and $\id{\{}\vec{y} / \vec{x} \id{\}}$ to denote partial maps,
$s : \QProc \rightarrow \QProc$. A map, $s$ lifts, uniquely, to a map
on process terms, $\widehat{s} : \Proc \rightarrow \Proc$ by the
following equations.

\begin{mathpar}
  (0) \psubstp{Q}{P} := 0 \\
  (R \juxtap S) \psubstp{Q}{P}
  :=    
  (R)\psubstp{Q}{P} \juxtap (S) \psubstp{Q}{P} \\
  (x?(y).R) \psubstp{Q}{P}    
  :=    
  (x)\substp{Q}{P} (z)\concat( (R \psubstn{z}{y}) \psubstp{Q}{P} ) \\
  (\lift{x}{R}) \psubstp{Q}{P}  
  :=
  \lift{(x)\substp{Q}{P}}{ R \psubstp{Q}{P} } \\
%   (\dropn{x})  \psubstp{Q}{P}       
%   := 
%   \left\{ 
%     \begin{array}{ccc} 
%       \dropn{\quotep{Q}} & & x \nameeq \quotep{P} \\
%       \dropn{x} & & otherwise \\
%     \end{array}
%   \right. 
  (\dropn{x})  \psubstp{Q}{P}       
  := 
  \left\{ 
    \begin{array}{ccc} 
      Q & & x \nameeq \quotep{P} \\
      \dropn{x} & & otherwise \\
    \end{array}
  \right.
\end{mathpar}
 

where

\begin{eqnarray}
  (x)\id{\{} \lpquote Q \rpquote / \lpquote P \rpquote \id{\}}            = 
  \left\{ 
    \begin{array}{ccc}
      \lpquote Q \rpquote & & x \nameeq \lpquote P \rpquote \\
      x & & otherwise \\
    \end{array}
  \right. \nonumber
\end{eqnarray}

and $z$ is chosen distinct from $\quotep{P}$, $\quotep{Q}$, the free
names in $Q$, and all the names in $R$. Our $\alpha$-equivalence will
be built in the standard way from this substitution.

\begin{remark}\label{rem:no_self_referential_names}
  One consequence of these definitions is that $\forall P. \quotep{P}
  \not\in \freenames{P}$.
\end{remark}

\subsection{ Dynamic quote: an example }

Anticipating something of what's to come, consider applying the
substitution, $\widehat{\id{\{}u / z \id{\}}}$, to the following pair
of processes, $\lift{w}{y!(z)}$ and $w[ \lpquote y!(z) \rpquote ]$.

\begin{eqnarray}
	\lift{w}{y!(z)}\widehat{\id{\{}u / z \id{\}}}
		& = &
		\lift{w}{y!(u)} \nonumber\\
	w[ \lpquote y!(z) \rpquote ] \widehat{ \id{\{}u / z \id{\}} }
		& = &
		w[ \lpquote y!(z) \rpquote ] \nonumber
\end{eqnarray}

Because the body of the process between quotes is impervious to
substitution, we get radically different answers. In fact, by
examining the first process in an input context,
e.g. $x?(z).\lift{w}{y!(z)}$, we see that the process under the lift
operator may be shaped by prefixed inputs binding a name inside it. In
this sense, the lift operator will be seen as a way to dynamically
construct processes before reifying them as names.

Finally equipped with these standard features we can present the
dynamics of the calculus.

\subsubsection{Operational semantics} 

Finally, we introduce the computational dynamics. What marks these
algebras as distinct from other more traditionally studied algebraic
structures, e.g. vector spaces or polynomial rings, is the manner in
which dynamics is captured. In traditional structures, dynamics is typically
expressed through morphisms between such structures, as in linear maps
between vector spaces or morphisms between rings. In algebras
associated with the semantics of computation, the dynamics is
expressed as part of the algebraic structure itself, through a
reduction reduction relation typically denoted by $\red$. Below, we
give a recursive presentation of this relation for the calculus used
in the encoding.

$\red \subseteq \pi \times \pi$
$\red : \pi \to \mathcal{P}(\pi)$

\begin{mathpar}
  \inferrule* [lab=Comm] { \textsf{match}( x_{src}, x_{trgt} ) } { x_{trgt}?(y)P \; | \; x_{src}!\langle {Q} \rangle \red P\{\quotep{Q}/y}\} }
  \and \\
  \inferrule* [lab=Par] {{P} \red {P}'} {{{P} | {Q}} \red {{P}' | {Q}}}
  \and
  \inferrule* [lab=Equiv]{{{P} \scong {P}'} \andalso {{P}' \red {Q}'} \andalso {{Q}' \scong {Q}}}{{P} \red {Q}}
\end{mathpar}

\begin{eqnarray*}
  match_{\equiv} (\quotep{P},\quotep{Q}) & := & P \equiv Q \\
  match_{\dagger}(\quotep{P},\quotep{Q}) & := & \forall R. P|Q \red^{*} R => R \red^{*} 0 \\
  match_{K}(\quotep{P},\quotep{Q}) & := & K \mbox{ for some context } K
\end{eqnarray*}

$u?(x)P | u!\langle Q \rangle \red P\{\quotep{Q}/x\}$

%We write $\wred$ for $\red^*$, and $P\red$ if $\exists Q $ such that $ P \red Q$.
We write $P\red$ if $\exists Q $ such that $ P \red Q$ and $P\not\red$, otherwise.

\section{Replication}

As mentioned before, it is known that replication (and hence
recursion) can be implemented in a higher-order process algebra
\cite{SangiorgiWalker}. As our first example of calculation with the
machinery thus far presented we give the construction explicitly in
the {\rhoc}.

\begin{eqnarray}
	D_{x} & := & \prefix{x}{y}{(\binpar{\outputp{x}{y}}{@{y}})} \nonumber\\
	\bangp_{x}{P} & := & \binpar{{x}!\langle{\binpar{D_{x}}{P}}\rangle}{D_{x}} \nonumber
\end{eqnarray}

\begin{eqnarray}
	\bangp_{x}{P} & & \nonumber\\
	=
	& {x}!\langle{(\prefix{x}{y}{(\outputp{x}{y} | @{y})) | P}}\rangle 
	      | \prefix{x}{y}{(\outputp{x}{y} | @{y})} & \nonumber\\
	\red
	& (\outputp{x}{y} | @{y})\substn{\quotep{(\prefix{x}{y}{(@{y} | \outputp{x}{y})) | P}}}{y} & \nonumber\\
	=
	& \outputp{x}{\quotep{(\prefix{x}{y}{(\outputp{x}{y} | @{y})) | P}}}
	  | {(\prefix{x}{y}{(\outputp{x}{y} | @{y})) | P}} & \nonumber\\
	\red
	& \ldots & \nonumber\\
	\red^*
	& P | P | \ldots & \nonumber
\end{eqnarray}

Of course, this encoding, as an implementation, runs away, unfolding
$\bangp{P}$ eagerly. A lazier and more implementable replication
operator, restricted to input-guarded processes, may be obtained as follows.

\begin{eqnarray}
\bangp{\prefix{u}{v}{P}} 
	:= 
	\binpar{\lift{x}{\prefix{u}{v}{(\binpar{D(x)}{P})}}}{D(x)} \nonumber
\end{eqnarray}

\begin{remark}
  Note that the lazier definition still does not deal with summation
  or mixed summation (i.e. sums over input and output). The reader is
  invited to construct definitions of replication that deal with these
  features. 

  Further, the definitions are parameterized in a name, $x$. Can you,
  gentle reader, make a definition that eliminates this parameter and
  guarantees no accidental interaction between the replication
  machinery and the process being replicated -- i.e. no accidental
  sharing of names used by the process to get its work done and the
  name(s) used by the replication to effect copying. This latter
  revision of the definition of replication is crucial to obtaining
  the expected identity $!!P \sim !P$.
\end{remark}

\begin{remark}\label{rem:paradoxical_combinator}
  The reader familiar with the lambda calculus will have noticed the
  similarity between $D$ and the paradoxical combinator.

  [Ed. note: the existence of this seems to suggest we have to be more
  restrictive on the set of processes and names we admit if we are to
  support no-cloning.]
\end{remark}

\subsubsection{Bisimulation}

The computational dynamics gives rise to another kind of equivalence,
the equivalence of computational behavior. As previously mentioned
this is typically captured \emph{via} some form of bisimulation.

% The notion we use in this paper is weak barbed bisimulation
% \cite{milner91polyadicpi}.

The notion we use in this paper is derived from weak barbed
bisimulation \cite{milner91polyadicpi}. 

\begin{definition}
An \emph{observation relation}, $\downarrow_{\mathcal N}$, over a set
of names, $\mathcal N$, is the smallest relation satisfying the rules
below.

\infrule[Out-barb]{y \in {\mathcal N}, \; x \nameeq y}
		  {\outputp{x}{v} \downarrow_{\mathcal N} x}
\infrule[Par-barb]{\mbox{$P\downarrow_{\mathcal N} x$ or $Q\downarrow_{\mathcal N} x$}}
		  {\binpar{P}{Q} \downarrow_{\mathcal N} x}

We write $P \Downarrow_{\mathcal N} x$ if there is $Q$ such that 
$P \wred Q$ and $Q \downarrow_{\mathcal N} x$.
\end{definition}

\begin{definition}
%\label{def.bbisim}
An  ${\mathcal N}$-\emph{barbed bisimulation} over a set of names, ${\mathcal N}$, is a symmetric binary relation 
${\mathcal S}_{\mathcal N}$ between agents such that $P\rel{S}_{\mathcal N}Q$ implies:
\begin{enumerate}
\item If $P \red P'$ then $Q \wred Q'$ and $P'\rel{S}_{\mathcal N} Q'$.
\item If $P\downarrow_{\mathcal N} x$, then $Q\Downarrow_{\mathcal N} x$.
\end{enumerate}
$P$ is ${\mathcal N}$-barbed bisimilar to $Q$, written
$P \wbbisim_{\mathcal N} Q$, if $P \rel{S}_{\mathcal N} Q$ for some ${\mathcal N}$-barbed bisimulation ${\mathcal S}_{\mathcal N}$.
\end{definition}

$\mathcal{R} \subseteq \pi \times \pi$

$P \mathcal{R} Q => \forall P'. P \red P' \Rightarrow \exists Q'. Q \red Q', P' \mathcal{R} Q'$

$P \vdash x \Rightarrow Q \vdash x$

\begin{mathpar}
  \inferrule*[lab=Out-barb]{x \nameeq y}{{y}!\langle{Q}\rangle \vdash x}
  \and
  \inferrule*[lab=Par-barb]{\mbox{$P\vdash x$ or $Q\vdash x$}}{\binpar{P}{Q} \vdash x}
\end{mathpar}

\subsubsection{Contexts}

One of the principle advantages of computational calculi like the
$\pi$-calculus is a well-defined notion of context,
contextual-equivalence and a correlation between
contextual-equivalence and notions of bisimulation. The notion of
context allows the decomposition of a process into (sub-)process and
its syntactic environment, its context. Thus, a context may be
thought of as a process with a ``hole'' (written $\Box$) in it. The
application of a context $M$ to a process $P$, written $M[P]$, is
tantamount to filling the hole in $M$ with $P$. In this paper we do
not need the full weight of this theory, but do make use of the notion
of context in the proof the main theorem. 

\begin{mathpar}
  \inferrule* [lab=summation] {} {{M_{M},M_{N}} \bc \Box \;|\; x.M_{A} \;|\; M_{M}+M_{N}}
  \and
  \inferrule* [lab=agent] {} {{M_{A}} \bc (\vec{x})M_{P} \;| \; \clift{P_0,\ldots,M_{P},\ldots,P_N}}
  \and \\
  \inferrule* [lab=process] {} {{M_{P}} \bc M_{N} \;| \;P|M_{P} }
\end{mathpar} 

\begin{mathpar}
  \inferrule* [lab=sychronization] {} {M_{N} \bc \Box \;|\; x?M_{F} \;|\; x!M_{C}}
  \and
  \inferrule* [lab=abstraction] {} {{M_{F}} \bc (x)M_{P} }
  \and
  \inferrule* [lab=concretion] {} {{M_{C}} \bc \langle M_{P} \rangle }
  \and \\
  \inferrule* [lab=process] {} {{M_{P}} \bc M_{N} \;| \;P|M_{P} }
\end{mathpar}

\begin{definition}[contextual application] Given a context $M$, and
  process $P$, we define the \emph{contextual application}, $M[P] :=
  M\{P/\Box\}$. That is, the contextual application of M to P is the
  substitution of $P$ for $\Box$ in $M$.
\end{definition}

$\meaningof{-} : L \to \mathcal{P}(\pi)$

\begin{mathpar}
  \inferrule* [lab=collection] {} {\meaningof{true} = \pi, \and \meaningof{~E} = \pi \setminus \meaningof{E}, \and \meaningof{E_{1} \& E_{2}} = \meaningof{E_{1}} \cap \meaningof{E_{2}}}
\end{mathpar}

\begin{mathpar}
  \inferrule* [lab=structure] {} {\meaningof{0} = \{ P \in \pi | P \equiv 0 \}, \and \\ \meaningof{E_1 | E_2} = \{ P \in \pi | P \equiv P_{1} | P_{2}, P_{1} \in \meaningof{E_{1}}, P_{2} \in \meaningof{E_2}\} }
\end{mathpar}

\begin{mathpar}
 \inferrule* [lab=behavior] {} {\meaningof{\langle a?b \rangle E} = \{ P \in \pi | P \equiv Q | u?(y)P', \\ \and \\\\ \and \\ \;\;\; u \in \meaningof{a}, \forall z.P'\{z/y\} \in \meaningof{E\{z/b\}}\}, \and \\ \meaningof{a!E} = \{ P \in \pi | P \equiv Q | x!\langle P' \rangle, x \in \meaningof{a} P' \in \meaningof{E}\} }
\end{mathpar}

\begin{mathpar}
 \inferrule* [lab=nominal] {} {\meaningof{\quotep{E}} = \{ \quotep{P} \in \quotep{\pi} | P \in \meaningof{E} \}, \and \meaningof{\quotep{P}} = \{ \quotep{Q} \in \quotep{\pi} | P \equiv Q \} \and \\ \meaningof{@\quotep{E}} = \{ P \in \pi | P \equiv @x, x \in \meaningof{E} \}}
\end{mathpar}

\begin{eqnarray*}
  \\
  \meaningof{-} : TS \to ST
\end{eqnarray*}

\begin{eqnarray*}
  \\
  L : TS \to ST
\end{eqnarray*}

\begin{eqnarray*}
  \\
  P \models E \iff P \in \meaningof{E}
\end{eqnarray*}

\begin{eqnarray*}
  P \approx_{L} Q \iff \forall E \in L. P \models E \iff Q \models E
\end{eqnarray*}

\begin{eqnarray*}
  P \approx_{K} Q
\end{eqnarray*}

\begin{eqnarray*}
  P \approx Q
\end{eqnarray*}

$\approx_{K} = \approx = \approx_{L}$

\subsubsection{Contextual duality}

Note that contexts extend the quotation operation to a family of
operations from processes to names. Given a context, $M$, we can
define a \emph{nominal context}, $\quotep{M}$ by $\quotep{M}[P] :=
\quotep{M[P]}$. To foreshadow what is to come we observe that these
operations enjoy a duality with processes very much like the duality
between vectors and maps from vectors to scalars.

Further, because the calculus is essentially higher-order, we have a
correspondence between contexts and processes. More specifically,
given a name $x$ and a context $M$ we can construct $M^{*}_{x}$ such
that 

\begin{mathpar}
  M^{*}_{x} | \lift{x}{P} \red M[P]
\end{mathpar}

namely,

\begin{mathpar}
  M^{*}_{x} := x?(u).M[\dropn{u}]
\end{mathpar}

The dependence of $M^{*}_{x}$ on a name makes it an abstraction, 

\begin{mathpar}
  M^{*} := (x)x?(u).M[\dropn{u}]
\end{mathpar}

\subsection{Additional notation}

It will sometimes be convenient to denote the process a name
quotes. We already have the notation $x = \quotep{P}$, but it will be
convenient to introduce an alternate notation, $\procn{x}$, when we
want to emphasize the connection to the use of the name. Note that, by
virtue of name equivalence, $\quotep{\procn{x}} \nameeq x$; so, the
notation is consistent with previous definitions.

Further, because names have structure it is possible to effect
substitutions on the basis of that structure. This means we need to
upgrade our notation for substitutions, which we accomplish by
adapting comprehension notation. Thus,

\begin{mathpar}
  P\{ y / x : x \in S \}
\end{mathpar}

is interpreted to mean the process derived from P by replacing (in a
capture-avoiding manner) each occurrence of $x$ in $S$ by $y$. For example,

\begin{mathpar}
  P\{ \quotep{\procn{x}|\procn{x}} / x : x \in \freenames{P} \}
\end{mathpar}

will replace each (occurrence) of a free name $x$ in $P$ by
$\quotep{\procn{x}|\procn{x}}$.

Also, we will avail ourselves of the notation $x^{L}$ and $x^{R}$ to
denote injections of a name into disjoint copies of the name
space. There are numerous ways to accomplish this. One example can be
found in \cite{MeredithR05}. This notation overloads to vectors of
names: $\vec{x}^{\pi} := (x_{i}^{\pi} \; : \; 0 \leq i < |\vec{x}| )$ where $\pi \in \{L,R\}$.

We also use $P^{\Box} := P|\Box$.

In \cite{MeredithR05} an interpretation of the new operator is
given. It turns out that there are several possible interpretations
all enjoying the requisite algebraic properties of the operator (see
\cite{milner91polyadicpi}). We will therefore make liberal use of
$(\nu\; \vec{x})P$.

% subsection the_syntax_and_semantics_of_the_notation_system (end)   

\input{qm2pi.qmops} 

\input{qm2pi.sterngerlach} 

\input{qm2pi.metric} 

% section concurrent_process_calculi (end)

%\input{qm2pi.proofsketch}

% section proof sketch (end)

%\input{qm2pi.slviaknots} 

% section spatial logic via knots (end)

\input{qm2pi.conclusion}

% section conclusion (end)

%\input{qm2pi.dtcodes} 

% section wiring algorithm (end)

\input{qm2pi.ack} 

% section acknowledgments (end)

\newpage


\bibliographystyle{plain}   
\bibliography{../../biblios/main.bib}

\input{qm2pi.rhodetails}

\end{document}

 

%\documentclass[12pt]{llncs}
%\documentclass{jktr}

\usepackage[pdftex]{hyperref}                   
\usepackage {listings}
\usepackage {mathpartir}
\usepackage{bcprules}
%\usepackage{listings}
                       
\usepackage{graphicx} 
%\usepackage[margins=2.5cm,nohead,nofoot]{geometry}
%\usepackage{geometry}
\usepackage{amsfonts}
\usepackage{amstext}
\usepackage{latexsym}
\usepackage{amssymb}
\usepackage{color}


%\include{myPreamble}
\include{qm2pi.local} 

%\ifpdf
%\usepackage[pdftex]{graphicx}
%\else
%\usepackage{graphicx}
%\fi

 % \ifpdf
%  \usepackage{pdfsync}
%  \if


%\title{Brief Article}
%\author{David F. Snyder}
%\author{L.G. Meredith}

%\address{Dept. of Math., Texas State University--San Marcos, San Marcos, TX 78666}
       
\pagestyle{empty}


\begin{document}

\lstset{language=[Objective]Caml,frame=shadowbox}

\input{qm2pi.front}

% section front matter (end)

\input{qm2pi.intro} 
 
% section introduction (end)

% \input{qm2pi.knotations} 

% section notation (end)

\input{qm2pi.process.calculi} 

% section concurrent_process_calculi_and_spatial_logics_ (end)
    
%\input{qm2pi.knots2pi} 

%\input{qm2pi.trefoil} 

%\input{qm2pi.mainthm} 

% subsection basic_interpretation (end)

%\input{qm2pi.rho.presentation} 
\subsection{The syntax and semantics of the notation system}\label{sub:the_syntax_and_semantics_of_the_notation_system} % (fold)

We now summarize a technical presentation of the calculus that
embodies our theory of dynamics. The typical presentation of such a
calculus follows the style of giving generators and relations on
them. The grammar, below, describing term constructors, freely
generates the set of processes, $\Proc$. This set is then quotiented
by a relation known as structural congruence and it is over this set
that the notion of dynamics is expressed. This presentation is
essentially that of \cite{MeredithR05} with the addition of
polyadicity and summation. For readability we have relegated some of
the technical subtleties to an appendix.

\subsubsection{Process grammar}\label{subsub:process_grammar}

\begin{mathpar}
  \inferrule* [lab=synchronization] {} {{M} \bc \pzero \;|\; x?F \;|\; x!C }
  \and
  \inferrule* [lab=abstraction] {} {{F} \bc (x)P}
  \and
  \inferrule* [lab=concretion] {} {{C} \bc \langle Q \rangle}
  \and
  \inferrule* [lab=process] {} {{P,Q} \bc M \;| \;P|Q \;|\; @{x}}
  \and
  \inferrule* [lab=name] {} {{x} \bc \quotep{P}}
\end{mathpar} 

Note that $\vec{x}$ (resp. $\vec{P}$) denotes a vector of names
(resp. processes) of length $|\vec{x}|$ (resp. $|\vec{P}|$). We adopt
the following useful abbreviations.

\begin{mathpar}
   x?(\vec{y}).P := x.(\vec{y})P \and  x\clift{\vec{P}} := x.\clift{\vec{P}}
   \and x!(y) := \lift{x}{\dropn{y}}
   \and \Pi_{i=0}^{n-1}P_i := P_0 | \ldots | P_{n-1}
\end{mathpar}

\subsubsection{Structural congruence}

\paragraph{Free and bound names and alpha-equivalence.} At the
core of structural equivalence is alpha-equivalence which identifies
process that are the same up to a change of variable. Formally, we
recognize the distinction between free and bound names. The free names
of a process, $\freenames{P}$, may be calculated recursively as
follows:

\begin{mathpar}
\freenames{\pzero} := \emptyset
  \and \\
  \freenames{x?(y).P} := \{ x \} \cup (\freenames{P} \setminus \{ y \})
  \and 
  \freenames{x!\langle P \rangle} := \{ x \} \cup \{ P \} 
  \and \\
  \freenames{P|Q} := \freenames{P} \cup \freenames{Q}
  \and \\
  \freenames{@{x}} := \{ x \}
\end{mathpar}

$\pi$
$\quotep{\pi}$

$\freenames{-} : \pi \to \mathcal{P}(\quotep{\pi})$

\begin{eqnarray*}
  \freenames{\pzero} & := & \emptyset \\
  \freenames{x?(y).P} & := & \{ x \} \cup (\freenames{P} \setminus \{ y \}) \\
  \freenames{x!\langle P \rangle} & := & \{ x \} \cup \{ P \} \\
  \freenames{P|Q} & := & \freenames{P} \cup \freenames{Q} \\
  \freenames{\dropn{x}} & := & \{ x \}
\end{eqnarray*}

The bound names of a process, $\boundnames{P}$, are those names occurring in $P$
that are not free. For example, in $x?(y).0$, the name $x$ is free, while $y$ is bound.

\begin{mathpar}
  \inferrule* [lab=monoidal-laws] {} { P|Q \equiv Q|P \and P|0 \equiv P \and P|(Q|R) \equiv (P|Q)|R }
\end{mathpar}

\begin{mathpar}
  \inferrule* [lab=alpha-equivalence] {} { (x)P \equiv (y)P\{y/x\} \and y \not\in \freenames{P} }
\end{mathpar}

\begin{definition}
Then two processes, $P,Q$, are alpha-equivalent if $P = Q\{\vec{y}/\vec{x}\}$ for
some $\vec{x} \in \boundnames{Q},\vec{y} \in \boundnames{P}$, where $Q\{\vec{y}/\vec{x}\}$
denotes the capture-avoiding substitution of $\vec{y}$ for $\vec{x}$ in $Q$.
\end{definition}

\begin{definition}
  The {\em structural congruence} \cite{SangiorgiWalker} , $\equiv$,
  between processes is the least congruence containing
  alpha-equivalence, satisfying the abelian monoid laws
  (associativity, commutativity and $\pzero$ as identity) for parallel
  composition $|$ and for summation $+$.
\end{definition}

\subsection{Name equivalence}

We take name equivalence, written $\nameeq$, to be the smallest
equivalence relation generated by the following rules.

\begin{mathpar}
\inferrule*[lab=Quote-drop]
{ }
{ \quotep{@{x}} \nameeq x }

\inferrule*[lab=Struct-equiv]
{ P \scong Q }
{ \quotep{P} \nameeq \quotep{Q} }
\end{mathpar}

The astute reader will have noticed that the mutual recursion of names
and processes imposes a mutual recursion on alpha-equivalence and
structural equivalence via name-equivalence. Fortunately, all of this
works out pleasantly and we may calculate in the natural way, free of
concern. The reader interested in the details is referred to the
appendix \ref{appendix:rho_details}.

\subsection{Substitution}

We use $\Proc$ for the set of processes, $\QProc$ for the set of
names, and $\id{\{}\vec{y} / \vec{x} \id{\}}$ to denote partial maps,
$s : \QProc \rightarrow \QProc$. A map, $s$ lifts, uniquely, to a map
on process terms, $\widehat{s} : \Proc \rightarrow \Proc$ by the
following equations.

\begin{mathpar}
  (0) \psubstp{Q}{P} := 0 \\
  (R \juxtap S) \psubstp{Q}{P}
  :=    
  (R)\psubstp{Q}{P} \juxtap (S) \psubstp{Q}{P} \\
  (x?(y).R) \psubstp{Q}{P}    
  :=    
  (x)\substp{Q}{P} (z)\concat( (R \psubstn{z}{y}) \psubstp{Q}{P} ) \\
  (\lift{x}{R}) \psubstp{Q}{P}  
  :=
  \lift{(x)\substp{Q}{P}}{ R \psubstp{Q}{P} } \\
%   (\dropn{x})  \psubstp{Q}{P}       
%   := 
%   \left\{ 
%     \begin{array}{ccc} 
%       \dropn{\quotep{Q}} & & x \nameeq \quotep{P} \\
%       \dropn{x} & & otherwise \\
%     \end{array}
%   \right. 
  (\dropn{x})  \psubstp{Q}{P}       
  := 
  \left\{ 
    \begin{array}{ccc} 
      Q & & x \nameeq \quotep{P} \\
      \dropn{x} & & otherwise \\
    \end{array}
  \right.
\end{mathpar}
 

where

\begin{eqnarray}
  (x)\id{\{} \lpquote Q \rpquote / \lpquote P \rpquote \id{\}}            = 
  \left\{ 
    \begin{array}{ccc}
      \lpquote Q \rpquote & & x \nameeq \lpquote P \rpquote \\
      x & & otherwise \\
    \end{array}
  \right. \nonumber
\end{eqnarray}

and $z$ is chosen distinct from $\quotep{P}$, $\quotep{Q}$, the free
names in $Q$, and all the names in $R$. Our $\alpha$-equivalence will
be built in the standard way from this substitution.

\begin{remark}\label{rem:no_self_referential_names}
  One consequence of these definitions is that $\forall P. \quotep{P}
  \not\in \freenames{P}$.
\end{remark}

\subsection{ Dynamic quote: an example }

Anticipating something of what's to come, consider applying the
substitution, $\widehat{\id{\{}u / z \id{\}}}$, to the following pair
of processes, $\lift{w}{y!(z)}$ and $w[ \lpquote y!(z) \rpquote ]$.

\begin{eqnarray}
	\lift{w}{y!(z)}\widehat{\id{\{}u / z \id{\}}}
		& = &
		\lift{w}{y!(u)} \nonumber\\
	w[ \lpquote y!(z) \rpquote ] \widehat{ \id{\{}u / z \id{\}} }
		& = &
		w[ \lpquote y!(z) \rpquote ] \nonumber
\end{eqnarray}

Because the body of the process between quotes is impervious to
substitution, we get radically different answers. In fact, by
examining the first process in an input context,
e.g. $x?(z).\lift{w}{y!(z)}$, we see that the process under the lift
operator may be shaped by prefixed inputs binding a name inside it. In
this sense, the lift operator will be seen as a way to dynamically
construct processes before reifying them as names.

Finally equipped with these standard features we can present the
dynamics of the calculus.

\subsubsection{Operational semantics} 

Finally, we introduce the computational dynamics. What marks these
algebras as distinct from other more traditionally studied algebraic
structures, e.g. vector spaces or polynomial rings, is the manner in
which dynamics is captured. In traditional structures, dynamics is typically
expressed through morphisms between such structures, as in linear maps
between vector spaces or morphisms between rings. In algebras
associated with the semantics of computation, the dynamics is
expressed as part of the algebraic structure itself, through a
reduction reduction relation typically denoted by $\red$. Below, we
give a recursive presentation of this relation for the calculus used
in the encoding.

$\red \subseteq \pi \times \pi$
$\red : \pi \to \mathcal{P}(\pi)$

\begin{mathpar}
  \inferrule* [lab=Comm] { \textsf{match}( x_{src}, x_{trgt} ) } { x_{trgt}?(y)P \; | \; x_{src}!\langle {Q} \rangle \red P\{\quotep{Q}/y}\} }
  \and \\
  \inferrule* [lab=Par] {{P} \red {P}'} {{{P} | {Q}} \red {{P}' | {Q}}}
  \and
  \inferrule* [lab=Equiv]{{{P} \scong {P}'} \andalso {{P}' \red {Q}'} \andalso {{Q}' \scong {Q}}}{{P} \red {Q}}
\end{mathpar}

\begin{eqnarray*}
  match_{\equiv} (\quotep{P},\quotep{Q}) & := & P \equiv Q \\
  match_{\dagger}(\quotep{P},\quotep{Q}) & := & \forall R. P|Q \red^{*} R => R \red^{*} 0 \\
  match_{K}(\quotep{P},\quotep{Q}) & := & K \mbox{ for some context } K
\end{eqnarray*}

$u?(x)P | u!\langle Q \rangle \red P\{\quotep{Q}/x\}$

%We write $\wred$ for $\red^*$, and $P\red$ if $\exists Q $ such that $ P \red Q$.
We write $P\red$ if $\exists Q $ such that $ P \red Q$ and $P\not\red$, otherwise.

\section{Replication}

As mentioned before, it is known that replication (and hence
recursion) can be implemented in a higher-order process algebra
\cite{SangiorgiWalker}. As our first example of calculation with the
machinery thus far presented we give the construction explicitly in
the {\rhoc}.

\begin{eqnarray}
	D_{x} & := & \prefix{x}{y}{(\binpar{\outputp{x}{y}}{@{y}})} \nonumber\\
	\bangp_{x}{P} & := & \binpar{{x}!\langle{\binpar{D_{x}}{P}}\rangle}{D_{x}} \nonumber
\end{eqnarray}

\begin{eqnarray}
	\bangp_{x}{P} & & \nonumber\\
	=
	& {x}!\langle{(\prefix{x}{y}{(\outputp{x}{y} | @{y})) | P}}\rangle 
	      | \prefix{x}{y}{(\outputp{x}{y} | @{y})} & \nonumber\\
	\red
	& (\outputp{x}{y} | @{y})\substn{\quotep{(\prefix{x}{y}{(@{y} | \outputp{x}{y})) | P}}}{y} & \nonumber\\
	=
	& \outputp{x}{\quotep{(\prefix{x}{y}{(\outputp{x}{y} | @{y})) | P}}}
	  | {(\prefix{x}{y}{(\outputp{x}{y} | @{y})) | P}} & \nonumber\\
	\red
	& \ldots & \nonumber\\
	\red^*
	& P | P | \ldots & \nonumber
\end{eqnarray}

Of course, this encoding, as an implementation, runs away, unfolding
$\bangp{P}$ eagerly. A lazier and more implementable replication
operator, restricted to input-guarded processes, may be obtained as follows.

\begin{eqnarray}
\bangp{\prefix{u}{v}{P}} 
	:= 
	\binpar{\lift{x}{\prefix{u}{v}{(\binpar{D(x)}{P})}}}{D(x)} \nonumber
\end{eqnarray}

\begin{remark}
  Note that the lazier definition still does not deal with summation
  or mixed summation (i.e. sums over input and output). The reader is
  invited to construct definitions of replication that deal with these
  features. 

  Further, the definitions are parameterized in a name, $x$. Can you,
  gentle reader, make a definition that eliminates this parameter and
  guarantees no accidental interaction between the replication
  machinery and the process being replicated -- i.e. no accidental
  sharing of names used by the process to get its work done and the
  name(s) used by the replication to effect copying. This latter
  revision of the definition of replication is crucial to obtaining
  the expected identity $!!P \sim !P$.
\end{remark}

\begin{remark}\label{rem:paradoxical_combinator}
  The reader familiar with the lambda calculus will have noticed the
  similarity between $D$ and the paradoxical combinator.

  [Ed. note: the existence of this seems to suggest we have to be more
  restrictive on the set of processes and names we admit if we are to
  support no-cloning.]
\end{remark}

\subsubsection{Bisimulation}

The computational dynamics gives rise to another kind of equivalence,
the equivalence of computational behavior. As previously mentioned
this is typically captured \emph{via} some form of bisimulation.

% The notion we use in this paper is weak barbed bisimulation
% \cite{milner91polyadicpi}.

The notion we use in this paper is derived from weak barbed
bisimulation \cite{milner91polyadicpi}. 

\begin{definition}
An \emph{observation relation}, $\downarrow_{\mathcal N}$, over a set
of names, $\mathcal N$, is the smallest relation satisfying the rules
below.

\infrule[Out-barb]{y \in {\mathcal N}, \; x \nameeq y}
		  {\outputp{x}{v} \downarrow_{\mathcal N} x}
\infrule[Par-barb]{\mbox{$P\downarrow_{\mathcal N} x$ or $Q\downarrow_{\mathcal N} x$}}
		  {\binpar{P}{Q} \downarrow_{\mathcal N} x}

We write $P \Downarrow_{\mathcal N} x$ if there is $Q$ such that 
$P \wred Q$ and $Q \downarrow_{\mathcal N} x$.
\end{definition}

\begin{definition}
%\label{def.bbisim}
An  ${\mathcal N}$-\emph{barbed bisimulation} over a set of names, ${\mathcal N}$, is a symmetric binary relation 
${\mathcal S}_{\mathcal N}$ between agents such that $P\rel{S}_{\mathcal N}Q$ implies:
\begin{enumerate}
\item If $P \red P'$ then $Q \wred Q'$ and $P'\rel{S}_{\mathcal N} Q'$.
\item If $P\downarrow_{\mathcal N} x$, then $Q\Downarrow_{\mathcal N} x$.
\end{enumerate}
$P$ is ${\mathcal N}$-barbed bisimilar to $Q$, written
$P \wbbisim_{\mathcal N} Q$, if $P \rel{S}_{\mathcal N} Q$ for some ${\mathcal N}$-barbed bisimulation ${\mathcal S}_{\mathcal N}$.
\end{definition}

$\mathcal{R} \subseteq \pi \times \pi$

$P \mathcal{R} Q => \forall P'. P \red P' \Rightarrow \exists Q'. Q \red Q', P' \mathcal{R} Q'$

$P \vdash x \Rightarrow Q \vdash x$

\begin{mathpar}
  \inferrule*[lab=Out-barb]{x \nameeq y}{{y}!\langle{Q}\rangle \vdash x}
  \and
  \inferrule*[lab=Par-barb]{\mbox{$P\vdash x$ or $Q\vdash x$}}{\binpar{P}{Q} \vdash x}
\end{mathpar}

\subsubsection{Contexts}

One of the principle advantages of computational calculi like the
$\pi$-calculus is a well-defined notion of context,
contextual-equivalence and a correlation between
contextual-equivalence and notions of bisimulation. The notion of
context allows the decomposition of a process into (sub-)process and
its syntactic environment, its context. Thus, a context may be
thought of as a process with a ``hole'' (written $\Box$) in it. The
application of a context $M$ to a process $P$, written $M[P]$, is
tantamount to filling the hole in $M$ with $P$. In this paper we do
not need the full weight of this theory, but do make use of the notion
of context in the proof the main theorem. 

\begin{mathpar}
  \inferrule* [lab=summation] {} {{M_{M},M_{N}} \bc \Box \;|\; x.M_{A} \;|\; M_{M}+M_{N}}
  \and
  \inferrule* [lab=agent] {} {{M_{A}} \bc (\vec{x})M_{P} \;| \; \clift{P_0,\ldots,M_{P},\ldots,P_N}}
  \and \\
  \inferrule* [lab=process] {} {{M_{P}} \bc M_{N} \;| \;P|M_{P} }
\end{mathpar} 

\begin{mathpar}
  \inferrule* [lab=sychronization] {} {M_{N} \bc \Box \;|\; x?M_{F} \;|\; x!M_{C}}
  \and
  \inferrule* [lab=abstraction] {} {{M_{F}} \bc (x)M_{P} }
  \and
  \inferrule* [lab=concretion] {} {{M_{C}} \bc \langle M_{P} \rangle }
  \and \\
  \inferrule* [lab=process] {} {{M_{P}} \bc M_{N} \;| \;P|M_{P} }
\end{mathpar}

\begin{definition}[contextual application] Given a context $M$, and
  process $P$, we define the \emph{contextual application}, $M[P] :=
  M\{P/\Box\}$. That is, the contextual application of M to P is the
  substitution of $P$ for $\Box$ in $M$.
\end{definition}

$\meaningof{-} : L \to \mathcal{P}(\pi)$

\begin{mathpar}
  \inferrule* [lab=collection] {} {\meaningof{true} = \pi, \and \meaningof{~E} = \pi \setminus \meaningof{E}, \and \meaningof{E_{1} \& E_{2}} = \meaningof{E_{1}} \cap \meaningof{E_{2}}}
\end{mathpar}

\begin{mathpar}
  \inferrule* [lab=structure] {} {\meaningof{0} = \{ P \in \pi | P \equiv 0 \}, \and \\ \meaningof{E_1 | E_2} = \{ P \in \pi | P \equiv P_{1} | P_{2}, P_{1} \in \meaningof{E_{1}}, P_{2} \in \meaningof{E_2}\} }
\end{mathpar}

\begin{mathpar}
 \inferrule* [lab=behavior] {} {\meaningof{\langle a?b \rangle E} = \{ P \in \pi | P \equiv Q | u?(y)P', \\ \and \\\\ \and \\ \;\;\; u \in \meaningof{a}, \forall z.P'\{z/y\} \in \meaningof{E\{z/b\}}\}, \and \\ \meaningof{a!E} = \{ P \in \pi | P \equiv Q | x!\langle P' \rangle, x \in \meaningof{a} P' \in \meaningof{E}\} }
\end{mathpar}

\begin{mathpar}
 \inferrule* [lab=nominal] {} {\meaningof{\quotep{E}} = \{ \quotep{P} \in \quotep{\pi} | P \in \meaningof{E} \}, \and \meaningof{\quotep{P}} = \{ \quotep{Q} \in \quotep{\pi} | P \equiv Q \} \and \\ \meaningof{@\quotep{E}} = \{ P \in \pi | P \equiv @x, x \in \meaningof{E} \}}
\end{mathpar}

\begin{eqnarray*}
  \\
  \meaningof{-} : TS \to ST
\end{eqnarray*}

\begin{eqnarray*}
  \\
  L : TS \to ST
\end{eqnarray*}

\begin{eqnarray*}
  \\
  P \models E \iff P \in \meaningof{E}
\end{eqnarray*}

\begin{eqnarray*}
  P \approx_{L} Q \iff \forall E \in L. P \models E \iff Q \models E
\end{eqnarray*}

\begin{eqnarray*}
  P \approx_{K} Q
\end{eqnarray*}

\begin{eqnarray*}
  P \approx Q
\end{eqnarray*}

$\approx_{K} = \approx = \approx_{L}$

\subsubsection{Contextual duality}

Note that contexts extend the quotation operation to a family of
operations from processes to names. Given a context, $M$, we can
define a \emph{nominal context}, $\quotep{M}$ by $\quotep{M}[P] :=
\quotep{M[P]}$. To foreshadow what is to come we observe that these
operations enjoy a duality with processes very much like the duality
between vectors and maps from vectors to scalars.

Further, because the calculus is essentially higher-order, we have a
correspondence between contexts and processes. More specifically,
given a name $x$ and a context $M$ we can construct $M^{*}_{x}$ such
that 

\begin{mathpar}
  M^{*}_{x} | \lift{x}{P} \red M[P]
\end{mathpar}

namely,

\begin{mathpar}
  M^{*}_{x} := x?(u).M[\dropn{u}]
\end{mathpar}

The dependence of $M^{*}_{x}$ on a name makes it an abstraction, 

\begin{mathpar}
  M^{*} := (x)x?(u).M[\dropn{u}]
\end{mathpar}

\subsection{Additional notation}

It will sometimes be convenient to denote the process a name
quotes. We already have the notation $x = \quotep{P}$, but it will be
convenient to introduce an alternate notation, $\procn{x}$, when we
want to emphasize the connection to the use of the name. Note that, by
virtue of name equivalence, $\quotep{\procn{x}} \nameeq x$; so, the
notation is consistent with previous definitions.

Further, because names have structure it is possible to effect
substitutions on the basis of that structure. This means we need to
upgrade our notation for substitutions, which we accomplish by
adapting comprehension notation. Thus,

\begin{mathpar}
  P\{ y / x : x \in S \}
\end{mathpar}

is interpreted to mean the process derived from P by replacing (in a
capture-avoiding manner) each occurrence of $x$ in $S$ by $y$. For example,

\begin{mathpar}
  P\{ \quotep{\procn{x}|\procn{x}} / x : x \in \freenames{P} \}
\end{mathpar}

will replace each (occurrence) of a free name $x$ in $P$ by
$\quotep{\procn{x}|\procn{x}}$.

Also, we will avail ourselves of the notation $x^{L}$ and $x^{R}$ to
denote injections of a name into disjoint copies of the name
space. There are numerous ways to accomplish this. One example can be
found in \cite{MeredithR05}. This notation overloads to vectors of
names: $\vec{x}^{\pi} := (x_{i}^{\pi} \; : \; 0 \leq i < |\vec{x}| )$ where $\pi \in \{L,R\}$.

We also use $P^{\Box} := P|\Box$.

In \cite{MeredithR05} an interpretation of the new operator is
given. It turns out that there are several possible interpretations
all enjoying the requisite algebraic properties of the operator (see
\cite{milner91polyadicpi}). We will therefore make liberal use of
$(\nu\; \vec{x})P$.

% subsection the_syntax_and_semantics_of_the_notation_system (end)   

\input{qm2pi.qmops} 

\input{qm2pi.sterngerlach} 

\input{qm2pi.metric} 

% section concurrent_process_calculi (end)

%\input{qm2pi.proofsketch}

% section proof sketch (end)

%\input{qm2pi.slviaknots} 

% section spatial logic via knots (end)

\input{qm2pi.conclusion}

% section conclusion (end)

%\input{qm2pi.dtcodes} 

% section wiring algorithm (end)

\input{qm2pi.ack} 

% section acknowledgments (end)

\newpage


\bibliographystyle{plain}   
\bibliography{../../biblios/main.bib}

\input{qm2pi.rhodetails}

\end{document}

 

% subsection basic_interpretation (end)

%\input{qm2pi.rho.presentation} 
\subsection{The syntax and semantics of the notation system}\label{sub:the_syntax_and_semantics_of_the_notation_system} % (fold)

We now summarize a technical presentation of the calculus that
embodies our theory of dynamics. The typical presentation of such a
calculus follows the style of giving generators and relations on
them. The grammar, below, describing term constructors, freely
generates the set of processes, $\Proc$. This set is then quotiented
by a relation known as structural congruence and it is over this set
that the notion of dynamics is expressed. This presentation is
essentially that of \cite{MeredithR05} with the addition of
polyadicity and summation. For readability we have relegated some of
the technical subtleties to an appendix.

\subsubsection{Process grammar}\label{subsub:process_grammar}

\begin{mathpar}
  \inferrule* [lab=synchronization] {} {{M} \bc \pzero \;|\; x?F \;|\; x!C }
  \and
  \inferrule* [lab=abstraction] {} {{F} \bc (x)P}
  \and
  \inferrule* [lab=concretion] {} {{C} \bc \langle Q \rangle}
  \and
  \inferrule* [lab=process] {} {{P,Q} \bc M \;| \;P|Q \;|\; @{x}}
  \and
  \inferrule* [lab=name] {} {{x} \bc \quotep{P}}
\end{mathpar} 

Note that $\vec{x}$ (resp. $\vec{P}$) denotes a vector of names
(resp. processes) of length $|\vec{x}|$ (resp. $|\vec{P}|$). We adopt
the following useful abbreviations.

\begin{mathpar}
   x?(\vec{y}).P := x.(\vec{y})P \and  x\clift{\vec{P}} := x.\clift{\vec{P}}
   \and x!(y) := \lift{x}{\dropn{y}}
   \and \Pi_{i=0}^{n-1}P_i := P_0 | \ldots | P_{n-1}
\end{mathpar}

\subsubsection{Structural congruence}

\paragraph{Free and bound names and alpha-equivalence.} At the
core of structural equivalence is alpha-equivalence which identifies
process that are the same up to a change of variable. Formally, we
recognize the distinction between free and bound names. The free names
of a process, $\freenames{P}$, may be calculated recursively as
follows:

\begin{mathpar}
\freenames{\pzero} := \emptyset
  \and \\
  \freenames{x?(y).P} := \{ x \} \cup (\freenames{P} \setminus \{ y \})
  \and 
  \freenames{x!\langle P \rangle} := \{ x \} \cup \{ P \} 
  \and \\
  \freenames{P|Q} := \freenames{P} \cup \freenames{Q}
  \and \\
  \freenames{@{x}} := \{ x \}
\end{mathpar}

$\pi$
$\quotep{\pi}$

$\freenames{-} : \pi \to \mathcal{P}(\quotep{\pi})$

\begin{eqnarray*}
  \freenames{\pzero} & := & \emptyset \\
  \freenames{x?(y).P} & := & \{ x \} \cup (\freenames{P} \setminus \{ y \}) \\
  \freenames{x!\langle P \rangle} & := & \{ x \} \cup \{ P \} \\
  \freenames{P|Q} & := & \freenames{P} \cup \freenames{Q} \\
  \freenames{\dropn{x}} & := & \{ x \}
\end{eqnarray*}

The bound names of a process, $\boundnames{P}$, are those names occurring in $P$
that are not free. For example, in $x?(y).0$, the name $x$ is free, while $y$ is bound.

\begin{mathpar}
  \inferrule* [lab=monoidal-laws] {} { P|Q \equiv Q|P \and P|0 \equiv P \and P|(Q|R) \equiv (P|Q)|R }
\end{mathpar}

\begin{mathpar}
  \inferrule* [lab=alpha-equivalence] {} { (x)P \equiv (y)P\{y/x\} \and y \not\in \freenames{P} }
\end{mathpar}

\begin{definition}
Then two processes, $P,Q$, are alpha-equivalent if $P = Q\{\vec{y}/\vec{x}\}$ for
some $\vec{x} \in \boundnames{Q},\vec{y} \in \boundnames{P}$, where $Q\{\vec{y}/\vec{x}\}$
denotes the capture-avoiding substitution of $\vec{y}$ for $\vec{x}$ in $Q$.
\end{definition}

\begin{definition}
  The {\em structural congruence} \cite{SangiorgiWalker} , $\equiv$,
  between processes is the least congruence containing
  alpha-equivalence, satisfying the abelian monoid laws
  (associativity, commutativity and $\pzero$ as identity) for parallel
  composition $|$ and for summation $+$.
\end{definition}

\subsection{Name equivalence}

We take name equivalence, written $\nameeq$, to be the smallest
equivalence relation generated by the following rules.

\begin{mathpar}
\inferrule*[lab=Quote-drop]
{ }
{ \quotep{@{x}} \nameeq x }

\inferrule*[lab=Struct-equiv]
{ P \scong Q }
{ \quotep{P} \nameeq \quotep{Q} }
\end{mathpar}

The astute reader will have noticed that the mutual recursion of names
and processes imposes a mutual recursion on alpha-equivalence and
structural equivalence via name-equivalence. Fortunately, all of this
works out pleasantly and we may calculate in the natural way, free of
concern. The reader interested in the details is referred to the
appendix \ref{appendix:rho_details}.

\subsection{Substitution}

We use $\Proc$ for the set of processes, $\QProc$ for the set of
names, and $\id{\{}\vec{y} / \vec{x} \id{\}}$ to denote partial maps,
$s : \QProc \rightarrow \QProc$. A map, $s$ lifts, uniquely, to a map
on process terms, $\widehat{s} : \Proc \rightarrow \Proc$ by the
following equations.

\begin{mathpar}
  (0) \psubstp{Q}{P} := 0 \\
  (R \juxtap S) \psubstp{Q}{P}
  :=    
  (R)\psubstp{Q}{P} \juxtap (S) \psubstp{Q}{P} \\
  (x?(y).R) \psubstp{Q}{P}    
  :=    
  (x)\substp{Q}{P} (z)\concat( (R \psubstn{z}{y}) \psubstp{Q}{P} ) \\
  (\lift{x}{R}) \psubstp{Q}{P}  
  :=
  \lift{(x)\substp{Q}{P}}{ R \psubstp{Q}{P} } \\
%   (\dropn{x})  \psubstp{Q}{P}       
%   := 
%   \left\{ 
%     \begin{array}{ccc} 
%       \dropn{\quotep{Q}} & & x \nameeq \quotep{P} \\
%       \dropn{x} & & otherwise \\
%     \end{array}
%   \right. 
  (\dropn{x})  \psubstp{Q}{P}       
  := 
  \left\{ 
    \begin{array}{ccc} 
      Q & & x \nameeq \quotep{P} \\
      \dropn{x} & & otherwise \\
    \end{array}
  \right.
\end{mathpar}
 

where

\begin{eqnarray}
  (x)\id{\{} \lpquote Q \rpquote / \lpquote P \rpquote \id{\}}            = 
  \left\{ 
    \begin{array}{ccc}
      \lpquote Q \rpquote & & x \nameeq \lpquote P \rpquote \\
      x & & otherwise \\
    \end{array}
  \right. \nonumber
\end{eqnarray}

and $z$ is chosen distinct from $\quotep{P}$, $\quotep{Q}$, the free
names in $Q$, and all the names in $R$. Our $\alpha$-equivalence will
be built in the standard way from this substitution.

\begin{remark}\label{rem:no_self_referential_names}
  One consequence of these definitions is that $\forall P. \quotep{P}
  \not\in \freenames{P}$.
\end{remark}

\subsection{ Dynamic quote: an example }

Anticipating something of what's to come, consider applying the
substitution, $\widehat{\id{\{}u / z \id{\}}}$, to the following pair
of processes, $\lift{w}{y!(z)}$ and $w[ \lpquote y!(z) \rpquote ]$.

\begin{eqnarray}
	\lift{w}{y!(z)}\widehat{\id{\{}u / z \id{\}}}
		& = &
		\lift{w}{y!(u)} \nonumber\\
	w[ \lpquote y!(z) \rpquote ] \widehat{ \id{\{}u / z \id{\}} }
		& = &
		w[ \lpquote y!(z) \rpquote ] \nonumber
\end{eqnarray}

Because the body of the process between quotes is impervious to
substitution, we get radically different answers. In fact, by
examining the first process in an input context,
e.g. $x?(z).\lift{w}{y!(z)}$, we see that the process under the lift
operator may be shaped by prefixed inputs binding a name inside it. In
this sense, the lift operator will be seen as a way to dynamically
construct processes before reifying them as names.

Finally equipped with these standard features we can present the
dynamics of the calculus.

\subsubsection{Operational semantics} 

Finally, we introduce the computational dynamics. What marks these
algebras as distinct from other more traditionally studied algebraic
structures, e.g. vector spaces or polynomial rings, is the manner in
which dynamics is captured. In traditional structures, dynamics is typically
expressed through morphisms between such structures, as in linear maps
between vector spaces or morphisms between rings. In algebras
associated with the semantics of computation, the dynamics is
expressed as part of the algebraic structure itself, through a
reduction reduction relation typically denoted by $\red$. Below, we
give a recursive presentation of this relation for the calculus used
in the encoding.

$\red \subseteq \pi \times \pi$
$\red : \pi \to \mathcal{P}(\pi)$

\begin{mathpar}
  \inferrule* [lab=Comm] { \textsf{match}( x_{src}, x_{trgt} ) } { x_{trgt}?(y)P \; | \; x_{src}!\langle {Q} \rangle \red P\{\quotep{Q}/y}\} }
  \and \\
  \inferrule* [lab=Par] {{P} \red {P}'} {{{P} | {Q}} \red {{P}' | {Q}}}
  \and
  \inferrule* [lab=Equiv]{{{P} \scong {P}'} \andalso {{P}' \red {Q}'} \andalso {{Q}' \scong {Q}}}{{P} \red {Q}}
\end{mathpar}

\begin{eqnarray*}
  match_{\equiv} (\quotep{P},\quotep{Q}) & := & P \equiv Q \\
  match_{\dagger}(\quotep{P},\quotep{Q}) & := & \forall R. P|Q \red^{*} R => R \red^{*} 0 \\
  match_{K}(\quotep{P},\quotep{Q}) & := & K \mbox{ for some context } K
\end{eqnarray*}

$u?(x)P | u!\langle Q \rangle \red P\{\quotep{Q}/x\}$

%We write $\wred$ for $\red^*$, and $P\red$ if $\exists Q $ such that $ P \red Q$.
We write $P\red$ if $\exists Q $ such that $ P \red Q$ and $P\not\red$, otherwise.

\section{Replication}

As mentioned before, it is known that replication (and hence
recursion) can be implemented in a higher-order process algebra
\cite{SangiorgiWalker}. As our first example of calculation with the
machinery thus far presented we give the construction explicitly in
the {\rhoc}.

\begin{eqnarray}
	D_{x} & := & \prefix{x}{y}{(\binpar{\outputp{x}{y}}{@{y}})} \nonumber\\
	\bangp_{x}{P} & := & \binpar{{x}!\langle{\binpar{D_{x}}{P}}\rangle}{D_{x}} \nonumber
\end{eqnarray}

\begin{eqnarray}
	\bangp_{x}{P} & & \nonumber\\
	=
	& {x}!\langle{(\prefix{x}{y}{(\outputp{x}{y} | @{y})) | P}}\rangle 
	      | \prefix{x}{y}{(\outputp{x}{y} | @{y})} & \nonumber\\
	\red
	& (\outputp{x}{y} | @{y})\substn{\quotep{(\prefix{x}{y}{(@{y} | \outputp{x}{y})) | P}}}{y} & \nonumber\\
	=
	& \outputp{x}{\quotep{(\prefix{x}{y}{(\outputp{x}{y} | @{y})) | P}}}
	  | {(\prefix{x}{y}{(\outputp{x}{y} | @{y})) | P}} & \nonumber\\
	\red
	& \ldots & \nonumber\\
	\red^*
	& P | P | \ldots & \nonumber
\end{eqnarray}

Of course, this encoding, as an implementation, runs away, unfolding
$\bangp{P}$ eagerly. A lazier and more implementable replication
operator, restricted to input-guarded processes, may be obtained as follows.

\begin{eqnarray}
\bangp{\prefix{u}{v}{P}} 
	:= 
	\binpar{\lift{x}{\prefix{u}{v}{(\binpar{D(x)}{P})}}}{D(x)} \nonumber
\end{eqnarray}

\begin{remark}
  Note that the lazier definition still does not deal with summation
  or mixed summation (i.e. sums over input and output). The reader is
  invited to construct definitions of replication that deal with these
  features. 

  Further, the definitions are parameterized in a name, $x$. Can you,
  gentle reader, make a definition that eliminates this parameter and
  guarantees no accidental interaction between the replication
  machinery and the process being replicated -- i.e. no accidental
  sharing of names used by the process to get its work done and the
  name(s) used by the replication to effect copying. This latter
  revision of the definition of replication is crucial to obtaining
  the expected identity $!!P \sim !P$.
\end{remark}

\begin{remark}\label{rem:paradoxical_combinator}
  The reader familiar with the lambda calculus will have noticed the
  similarity between $D$ and the paradoxical combinator.

  [Ed. note: the existence of this seems to suggest we have to be more
  restrictive on the set of processes and names we admit if we are to
  support no-cloning.]
\end{remark}

\subsubsection{Bisimulation}

The computational dynamics gives rise to another kind of equivalence,
the equivalence of computational behavior. As previously mentioned
this is typically captured \emph{via} some form of bisimulation.

% The notion we use in this paper is weak barbed bisimulation
% \cite{milner91polyadicpi}.

The notion we use in this paper is derived from weak barbed
bisimulation \cite{milner91polyadicpi}. 

\begin{definition}
An \emph{observation relation}, $\downarrow_{\mathcal N}$, over a set
of names, $\mathcal N$, is the smallest relation satisfying the rules
below.

\infrule[Out-barb]{y \in {\mathcal N}, \; x \nameeq y}
		  {\outputp{x}{v} \downarrow_{\mathcal N} x}
\infrule[Par-barb]{\mbox{$P\downarrow_{\mathcal N} x$ or $Q\downarrow_{\mathcal N} x$}}
		  {\binpar{P}{Q} \downarrow_{\mathcal N} x}

We write $P \Downarrow_{\mathcal N} x$ if there is $Q$ such that 
$P \wred Q$ and $Q \downarrow_{\mathcal N} x$.
\end{definition}

\begin{definition}
%\label{def.bbisim}
An  ${\mathcal N}$-\emph{barbed bisimulation} over a set of names, ${\mathcal N}$, is a symmetric binary relation 
${\mathcal S}_{\mathcal N}$ between agents such that $P\rel{S}_{\mathcal N}Q$ implies:
\begin{enumerate}
\item If $P \red P'$ then $Q \wred Q'$ and $P'\rel{S}_{\mathcal N} Q'$.
\item If $P\downarrow_{\mathcal N} x$, then $Q\Downarrow_{\mathcal N} x$.
\end{enumerate}
$P$ is ${\mathcal N}$-barbed bisimilar to $Q$, written
$P \wbbisim_{\mathcal N} Q$, if $P \rel{S}_{\mathcal N} Q$ for some ${\mathcal N}$-barbed bisimulation ${\mathcal S}_{\mathcal N}$.
\end{definition}

$\mathcal{R} \subseteq \pi \times \pi$

$P \mathcal{R} Q => \forall P'. P \red P' \Rightarrow \exists Q'. Q \red Q', P' \mathcal{R} Q'$

$P \vdash x \Rightarrow Q \vdash x$

\begin{mathpar}
  \inferrule*[lab=Out-barb]{x \nameeq y}{{y}!\langle{Q}\rangle \vdash x}
  \and
  \inferrule*[lab=Par-barb]{\mbox{$P\vdash x$ or $Q\vdash x$}}{\binpar{P}{Q} \vdash x}
\end{mathpar}

\subsubsection{Contexts}

One of the principle advantages of computational calculi like the
$\pi$-calculus is a well-defined notion of context,
contextual-equivalence and a correlation between
contextual-equivalence and notions of bisimulation. The notion of
context allows the decomposition of a process into (sub-)process and
its syntactic environment, its context. Thus, a context may be
thought of as a process with a ``hole'' (written $\Box$) in it. The
application of a context $M$ to a process $P$, written $M[P]$, is
tantamount to filling the hole in $M$ with $P$. In this paper we do
not need the full weight of this theory, but do make use of the notion
of context in the proof the main theorem. 

\begin{mathpar}
  \inferrule* [lab=summation] {} {{M_{M},M_{N}} \bc \Box \;|\; x.M_{A} \;|\; M_{M}+M_{N}}
  \and
  \inferrule* [lab=agent] {} {{M_{A}} \bc (\vec{x})M_{P} \;| \; \clift{P_0,\ldots,M_{P},\ldots,P_N}}
  \and \\
  \inferrule* [lab=process] {} {{M_{P}} \bc M_{N} \;| \;P|M_{P} }
\end{mathpar} 

\begin{mathpar}
  \inferrule* [lab=sychronization] {} {M_{N} \bc \Box \;|\; x?M_{F} \;|\; x!M_{C}}
  \and
  \inferrule* [lab=abstraction] {} {{M_{F}} \bc (x)M_{P} }
  \and
  \inferrule* [lab=concretion] {} {{M_{C}} \bc \langle M_{P} \rangle }
  \and \\
  \inferrule* [lab=process] {} {{M_{P}} \bc M_{N} \;| \;P|M_{P} }
\end{mathpar}

\begin{definition}[contextual application] Given a context $M$, and
  process $P$, we define the \emph{contextual application}, $M[P] :=
  M\{P/\Box\}$. That is, the contextual application of M to P is the
  substitution of $P$ for $\Box$ in $M$.
\end{definition}

$\meaningof{-} : L \to \mathcal{P}(\pi)$

\begin{mathpar}
  \inferrule* [lab=collection] {} {\meaningof{true} = \pi, \and \meaningof{~E} = \pi \setminus \meaningof{E}, \and \meaningof{E_{1} \& E_{2}} = \meaningof{E_{1}} \cap \meaningof{E_{2}}}
\end{mathpar}

\begin{mathpar}
  \inferrule* [lab=structure] {} {\meaningof{0} = \{ P \in \pi | P \equiv 0 \}, \and \\ \meaningof{E_1 | E_2} = \{ P \in \pi | P \equiv P_{1} | P_{2}, P_{1} \in \meaningof{E_{1}}, P_{2} \in \meaningof{E_2}\} }
\end{mathpar}

\begin{mathpar}
 \inferrule* [lab=behavior] {} {\meaningof{\langle a?b \rangle E} = \{ P \in \pi | P \equiv Q | u?(y)P', \\ \and \\\\ \and \\ \;\;\; u \in \meaningof{a}, \forall z.P'\{z/y\} \in \meaningof{E\{z/b\}}\}, \and \\ \meaningof{a!E} = \{ P \in \pi | P \equiv Q | x!\langle P' \rangle, x \in \meaningof{a} P' \in \meaningof{E}\} }
\end{mathpar}

\begin{mathpar}
 \inferrule* [lab=nominal] {} {\meaningof{\quotep{E}} = \{ \quotep{P} \in \quotep{\pi} | P \in \meaningof{E} \}, \and \meaningof{\quotep{P}} = \{ \quotep{Q} \in \quotep{\pi} | P \equiv Q \} \and \\ \meaningof{@\quotep{E}} = \{ P \in \pi | P \equiv @x, x \in \meaningof{E} \}}
\end{mathpar}

\begin{eqnarray*}
  \\
  \meaningof{-} : TS \to ST
\end{eqnarray*}

\begin{eqnarray*}
  \\
  L : TS \to ST
\end{eqnarray*}

\begin{eqnarray*}
  \\
  P \models E \iff P \in \meaningof{E}
\end{eqnarray*}

\begin{eqnarray*}
  P \approx_{L} Q \iff \forall E \in L. P \models E \iff Q \models E
\end{eqnarray*}

\begin{eqnarray*}
  P \approx_{K} Q
\end{eqnarray*}

\begin{eqnarray*}
  P \approx Q
\end{eqnarray*}

$\approx_{K} = \approx = \approx_{L}$

\subsubsection{Contextual duality}

Note that contexts extend the quotation operation to a family of
operations from processes to names. Given a context, $M$, we can
define a \emph{nominal context}, $\quotep{M}$ by $\quotep{M}[P] :=
\quotep{M[P]}$. To foreshadow what is to come we observe that these
operations enjoy a duality with processes very much like the duality
between vectors and maps from vectors to scalars.

Further, because the calculus is essentially higher-order, we have a
correspondence between contexts and processes. More specifically,
given a name $x$ and a context $M$ we can construct $M^{*}_{x}$ such
that 

\begin{mathpar}
  M^{*}_{x} | \lift{x}{P} \red M[P]
\end{mathpar}

namely,

\begin{mathpar}
  M^{*}_{x} := x?(u).M[\dropn{u}]
\end{mathpar}

The dependence of $M^{*}_{x}$ on a name makes it an abstraction, 

\begin{mathpar}
  M^{*} := (x)x?(u).M[\dropn{u}]
\end{mathpar}

\subsection{Additional notation}

It will sometimes be convenient to denote the process a name
quotes. We already have the notation $x = \quotep{P}$, but it will be
convenient to introduce an alternate notation, $\procn{x}$, when we
want to emphasize the connection to the use of the name. Note that, by
virtue of name equivalence, $\quotep{\procn{x}} \nameeq x$; so, the
notation is consistent with previous definitions.

Further, because names have structure it is possible to effect
substitutions on the basis of that structure. This means we need to
upgrade our notation for substitutions, which we accomplish by
adapting comprehension notation. Thus,

\begin{mathpar}
  P\{ y / x : x \in S \}
\end{mathpar}

is interpreted to mean the process derived from P by replacing (in a
capture-avoiding manner) each occurrence of $x$ in $S$ by $y$. For example,

\begin{mathpar}
  P\{ \quotep{\procn{x}|\procn{x}} / x : x \in \freenames{P} \}
\end{mathpar}

will replace each (occurrence) of a free name $x$ in $P$ by
$\quotep{\procn{x}|\procn{x}}$.

Also, we will avail ourselves of the notation $x^{L}$ and $x^{R}$ to
denote injections of a name into disjoint copies of the name
space. There are numerous ways to accomplish this. One example can be
found in \cite{MeredithR05}. This notation overloads to vectors of
names: $\vec{x}^{\pi} := (x_{i}^{\pi} \; : \; 0 \leq i < |\vec{x}| )$ where $\pi \in \{L,R\}$.

We also use $P^{\Box} := P|\Box$.

In \cite{MeredithR05} an interpretation of the new operator is
given. It turns out that there are several possible interpretations
all enjoying the requisite algebraic properties of the operator (see
\cite{milner91polyadicpi}). We will therefore make liberal use of
$(\nu\; \vec{x})P$.

% subsection the_syntax_and_semantics_of_the_notation_system (end)   

\section{Interpretation of QM}
\subsection{Supporting definitions}
\subsubsection{Multiplication}
\begin{mathpar}
  \quotep{Q} \cdot \quotep{R} := \quotep{Q|R}
  \and \\
  \quotep{Q} \cdot P := P\{ \quotep{Q|R} / \quotep{R} : \quotep{R} \in \freenames{P} \}
\end{mathpar}

\paragraph{Discussion}
The first line needs little explanation. The second line says that
each free name of the process is replaced with the multiplication of
that name by the scalar. Multiplication of a scalar (name) by a state
(process) results in a process all the names of which have been `moved
over' by parallel composition with the process the scalar
quotes. There is a subtlety that the bound names have to be
manipulated so that multiplied names aren't accidentally
captured. There are many ways to achieve this.

\begin{remark}\label{rem:multiplication_identities}
  The reader is invited to verify that for all $x,y,z \in \QProc$ and $P \in \Proc$
  \begin{mathpar}
    x \cdot \quotep{0} \equiv x 
    \and
    x \cdot y \equiv y \cdot x
    \and
    x \cdot (y \cdot z) \equiv (x \cdot y) \cdot z
    \and \\
    \quotep{0} \cdot P \equiv P
    \and \\
    x \cdot (y \cdot P) \equiv (x \cdot y) \cdot P
    \and \\
    x \cdot (P|Q) \equiv (x \cdot P) | (x \cdot Q)
    \and \\    
  \end{mathpar}
\end{remark}

\subsubsection{Tensor product}

We define a tensor product on processes by structural induction.

\paragraph{Tensor of sums} First note that all summations, including
$\pzero$ and sequence, can be written $\Sigma_{i} x_{i}.A_{i} +
\Sigma_{j} x_{j}.C_{j}$, where we have grouped input-guarded processes
together and output-guarded processes together.

Thus, we can define the tensor product of two summations, $N_{1}\otimes N_{2}$, where

\begin{mathpar}
  N_{1} := \Sigma_{i} x_{i}.A_{i} + \Sigma_{j} x_{j}.C_{j}
  \and
  N_{2} := \Sigma_{i'} y_{i'}.B_{i'} + \Sigma_{j'} y_{j'}.D_{j'} 
\end{mathpar}

as follows.

\begin{mathpar}
  \Sigma_{i} x_{i}.A_{i} + \Sigma_{j} x_{j}.C_{j} \otimes \Sigma_{i'}
  y_{i'}.B_{i'} + \Sigma_{j'} y_{j'}.D_{j'} 
  \and \\
  := \; \Sigma_{i} \Sigma_{i'} \quotep{\stackrel{\vee}{x_{i}}| \stackrel{\vee}{y_{i'}}}.(A_{i}\otimes B_{i'}) \; | \; \Sigma_{i'} \Sigma_{i} \quotep{\stackrel{\vee}{y_{i'}}|\stackrel{\vee}{x_{i}}}.(B_{i'}\otimes A_{i})
  \and
  \;\; | \;\; \Sigma_{j} \Sigma_{j'} \quotep{\stackrel{\vee}{x_{j}}|\stackrel{\vee}{y_{j'}}}.(A_{j}\otimes B_{j'}) \; | \; \Sigma_{j'} \Sigma_{j} \quotep{\stackrel{\vee}{y_{j'}}|\stackrel{\vee}{x_{j}}}.(B_{j'}\otimes A_{j})
\end{mathpar}

\begin{remark}
  Do we need to $x^{L}$ and $y^{R}$ for this construction as well?
\end{remark}

\paragraph{Tensor of parallel compositions} Next, we distribute tensor
over par.

\begin{mathpar}
  P_{1}|P_{2} \otimes Q_{1}|Q_{2} := (P_{1} \otimes Q_{1}) | (P_{1}
  \otimes Q_{2}) | (P_{2} \otimes Q_{1}) | (P_{2} \otimes Q_{2})
\end{mathpar}

\paragraph{Tensor with dropped names} We treat tensor of a
process with a dropped name as parallel composition.

\begin{mathpar}
  P \otimes \dropn{x} := P | \dropn{x}
\end{mathpar}

\paragraph{Tensor of agents}

Finally, we need to define tensor on agents. Note that the definition
of tensor on normal products only tensors inputs with inputs and
outputs with outputs. Thus, we only have to define the operation on
``homogeneous'' pairings.

\begin{mathpar}
  (\vec{x})P \otimes (\vec{y})Q
  \and \\
  := (x_{0}^{L}|y_{0}^{R},\ldots,x_{0}^{L}|y_{n}^{R},\ldots,x_{m}^{L}|y_{0}^{R},\ldots,x_{m}^{L}|y_{n}^R)(P\{ \vec{x}^{L}/\vec{x}\} \otimes Q \{ \vec{y}^{R}/\vec{y}\})
  \and \\
  \clift{\vec{P}} \otimes \clift{\vec{Q}}
  \and \\
  := \clift{P_{0}\otimes Q_{0},\ldots,P_{0}\otimes Q_{n},\ldots,P_{m}\otimes Q_{0},\ldots,P_{m}\otimes Q_{n}}
\end{mathpar}

\begin{remark}
  Observe that arities of tensored abstractions matches arities of
  tensored concretions if the original arities matched. Note also that
  the length of the arities corresponds to the increase in dimension
  we see in ordinary vector space tensor product.
\end{remark}

\begin{remark}
  Operationally, this definition distributes the tensor down to
  components ``linked'' by summation. Tensor over summation is
  intriguing in that it mixes names. Moreover, as a consequence of the
  way it mixes names we have the identities for all $x \in \QProc$ and
  $P,Q \in \Proc$

  \begin{mathpar}
    (x \cdot P) \otimes Q \equiv x \cdot (P \otimes Q) \equiv P \otimes (x \cdot Q)
    \and
    P \otimes \pzero \equiv P
  \end{mathpar}

  that the reader is invited to verify.
\end{remark}

\subsubsection{Annihilation}
\begin{mathpar}
  P^{\perp} := \{ Q | \forall R. P|Q \red^{*} R \Rightarrow R \red^{*} \pzero \}
  \and \\
  P^{\underline{\perp}} := \Sigma_{Q \in P^{\perp}} \quotep{Q}?(y).(\dropn{y}|Q) | \Sigma_{Q \in P^{\perp}} \quotep{Q}\clift{\Box}
\end{mathpar}

\paragraph{Discussion} The reader will note that $P^{\perp}$ is a
\emph{set} of processes, while $P^{\underline{\perp}}$ is a
\emph{context}. We call the set $P^{\perp}$ the \emph{annihilators} of
$P$. The parallel composition of a process in the annihilators of $P$
with $P$ will result in a process, the state space of which has all
paths eventually leading to $\pzero$. Execution may endure loops; but
under reasonable conditions of fairness (naturally guaranteed under
most notions of bisimulation) such a composite process cannot get
stuck in such a loop and will, eventually pop out and terminate.

The context $P^{\underline{\perp}}$ is ready and willing to ``take the
$P$ out of'' the process to which it is applied. It will effectively
transmit the code of the process to which it is applied to one of the
annihilators and run the process against it.

\subsubsection{Evaluation}
We fix $M$ a domain of fully abstract interpretation with an equality
coincident with bisimulation. We take $\meaningof{\cdot} : \Proc \to
M$ to be the map interpreting processes and $\nmeaningof{\cdot} : \M
\to Proc$ to be the map running the other way. Then we define

\begin{mathpar}
  \int P := \nmeaningof{\meaningof{P}}
\end{mathpar}

\paragraph{Discussion}
There are many fully abstract interpretations of Milner's
$\pi$-calculus. Any of them can be used as a basis for interpreting
the reflective calculus here. Equipped with such a domain it is
largely a matter of grinding through to check that the Yoneda
construction for the normalization-by-evaluation program can be
extended to this setting.

\begin{remark}
  The reader is invited to verify that $\int (P^{\underline{\perp}}[P]) = 0$.
\end{remark}

\subsection{Quantum mechanics}

Table \ref{tbl:core_qm_op_defns} gives the core operational definitions

\begin{table}[htp]\label{tbl:core_qm_op_defns}
  \center{
    \fbox{
      \begin{tabular}{c|c}
        quantum mechanics & process calculus \\
        \hline
        scalar & $x := \quotep{P}$ \\
        state vector & $\state{P} := P$ \\
        dual & $\state{P}^{*} := \event{P^{\underline{\perp}}} := \quotep{P^{\underline{\perp}}}[-]$ \\
        matrix & $ \Sigma_{\alpha} \state{P_{\alpha}}x_{\alpha}\event{Q_{\alpha}}$ \\
        vector addition & $\state{P} + \state{Q} := \state{P | Q}$ \\
        tensor product & $\state{P} \otimes \state{Q} := \state{P \otimes Q}$ \\
        inner product & $\innerprod{P}{Q} := \quotep{\int P^{\underline{\perp}}[Q]}$ \\
      \end{tabular}
    }
  }
  \caption{QM - operational definitions}
\end{table}

where

\begin{mathpar}
  \prmatrix{P}{Q} := \fprmatrix{P}{\quotep{\pzero}}{Q}
  \and
  \fprmatrix{P}{x}{Q} := (\state{P},x,\event{Q})
  \and
  (\fprmatrix{P}{x}{Q})(\state{R}) := x \cdot \innerprod{Q}{R} \cdot \state{P}
  \and
  (\fprmatrix{P}{x}{Q})(\event{R}) := x \cdot \innerprod{R}{P} \cdot \event{Q}
\end{mathpar}

\paragraph{Discussion}
As promised: vectors (aka states) are represented as processes; duals
as contextual duals; inner product definition should be compared with
standard inner product definition for ....

\begin{remark}
  Assuming $\int (P^{\underline{\perp}}[P]) = 0$, the reader is
  invited to verify that $(\fprmatrix{P}{x}{P})(\state{P}) = x \cdot \state{P}$.
\end{remark}

\begin{remark}
  The reader is invited to verify that $\innerprod{P}{Q}$ could
  equally well have been written $\quotep{\int \stackrel{\vee}{x}}$
  where $x = \event{P^{\underline{\perp}}}(Q)$.

  One of the motivations for this remark is that there is another way
  to factor these operations. We could package up evaluation in the dual:

  \begin{mathpar}
    \state{P}^{*} := \event{\int P^{\underline{\perp}}} := \quotep{\int P^{\underline{\perp}}}[-]
  \end{mathpar}

  and then have inner product defined by
  
  \begin{mathpar}
    \innerprod{P}{Q} := \event{P}(Q)
  \end{mathpar}

  Hopefully, experience with the calculations will provide guidance on
  the best factoring.
\end{remark}

\begin{remark}
  Assuming $\int (P^{\underline{\perp}}[P]) = 0$, the reader is
  invited to verify that $\forall P,Q. (\prmatrix{0}{Q})(\state{0}) =
  \state{0}$ and dually $(\prmatrix{P}{0})(\event{0}) = \event{0}$.
\end{remark}

\begin{remark}
  i'm a little worried that i don't (yet) have proper support for
  complex conjugacy. But, the observation above may give us a
  clue. According to Abramsky, it must be the case that the scalars
  are iso to the homset of the identity for the tensor -- which the
  observation above characterizes. 

  For now, we will simply bookmark the notion with $\overline{x}$.
\end{remark}

\subsubsection{Adjointness}

We need to give a definition of $(\cdot)^{\dagger}$ for matrices. The
obvious candidate definition is
\begin{mathpar}
(\Sigma_{\alpha}\fprmatrix{P_{\alpha}}{x_{\alpha}}{Q_{\alpha}})^{\dagger}
= \Sigma_{\alpha}\fprmatrix{(Q_{\alpha}^{\underline{\perp}})^{*}}{\overline{x}_{\alpha}}{P_{\alpha}^{\underline{\perp}}} 
\end{mathpar}

But, $(Q_{\alpha}^{\underline{\perp}})^{*}$ requires a name along
which to communicate the process to achieve the context application.

\subsubsection{Basis for a basis}
If processes label states and ``addition'' of states (a.k.a. vector
addition) is interpreted as parallel composition, what corresponds to
notions of linear independence and basis? Here, we recall that Yoshida
has developed a set of \emph{combinators} for an asynchronous verison
of Milner's $\pi$-calculus. These are a finite set of processes such
any process can be expressed as parallel composition of these
combinators together with liberal uses of the new operator and
replication. We can simply give a translation of these into the
present calculus and have reasonable expectation that the property
carries over. That is, that the resultant set allows to express all
processes via parallel composition. Note, however, that there is no
new operator or replication in this calculus. As a result, we expect
that the corresponding set is actually infinite. That is, we expect
that the space is actually infinite dimensional.

\begin{remark}
  The attentive reader may be a bit concerned. Certainly, the
  collection $S$, $K$ and $I$ is a finite set of
  combinators. Shouldn't we expect to see a finite set of combinators
  for an effectively equivalent system? i am very sympathetic to this
  critique and feel it warrants full attention. On the other hand, i
  also have in mind the following analogy. The natural numbers, as a
  monoid under addition, has exactly $1$ generator, while the natural
  numbers, as a monoid under multiplication, has countably many
  generators (the primes). We observe that the application of the
  lambda calculus is much less resource sensitive than the parallel
  composition of the $\pi$-calculus. Could it be the case that we have
  an analogy of the form
  
  \begin{mathpar}
    m + n : MN :: m*n : M|N
  \end{mathpar}

  giving a similar blow up in the set of ``primes''?  This is such a
  wonderful thought that, even if it's not true, i think it's worth
  writing down.
\end{remark}
 

\documentclass[12pt]{llncs}
%\documentclass{jktr}

\usepackage[pdftex]{hyperref}                   
\usepackage {listings}
\usepackage {mathpartir}
\usepackage{bcprules}
%\usepackage{listings}
                       
\usepackage{graphicx} 
%\usepackage[margins=2.5cm,nohead,nofoot]{geometry}
%\usepackage{geometry}
\usepackage{amsfonts}
\usepackage{amstext}
\usepackage{latexsym}
\usepackage{amssymb}
\usepackage{color}


%\include{myPreamble}
\include{qm2pi.local} 

%\ifpdf
%\usepackage[pdftex]{graphicx}
%\else
%\usepackage{graphicx}
%\fi

 % \ifpdf
%  \usepackage{pdfsync}
%  \if


%\title{Brief Article}
%\author{David F. Snyder}
%\author{L.G. Meredith}

%\address{Dept. of Math., Texas State University--San Marcos, San Marcos, TX 78666}
       
\pagestyle{empty}


\begin{document}

\lstset{language=[Objective]Caml,frame=shadowbox}

\input{qm2pi.front}

% section front matter (end)

\input{qm2pi.intro} 
 
% section introduction (end)

% \input{qm2pi.knotations} 

% section notation (end)

\input{qm2pi.process.calculi} 

% section concurrent_process_calculi_and_spatial_logics_ (end)
    
%\input{qm2pi.knots2pi} 

%\input{qm2pi.trefoil} 

%\input{qm2pi.mainthm} 

% subsection basic_interpretation (end)

%\input{qm2pi.rho.presentation} 
\subsection{The syntax and semantics of the notation system}\label{sub:the_syntax_and_semantics_of_the_notation_system} % (fold)

We now summarize a technical presentation of the calculus that
embodies our theory of dynamics. The typical presentation of such a
calculus follows the style of giving generators and relations on
them. The grammar, below, describing term constructors, freely
generates the set of processes, $\Proc$. This set is then quotiented
by a relation known as structural congruence and it is over this set
that the notion of dynamics is expressed. This presentation is
essentially that of \cite{MeredithR05} with the addition of
polyadicity and summation. For readability we have relegated some of
the technical subtleties to an appendix.

\subsubsection{Process grammar}\label{subsub:process_grammar}

\begin{mathpar}
  \inferrule* [lab=synchronization] {} {{M} \bc \pzero \;|\; x?F \;|\; x!C }
  \and
  \inferrule* [lab=abstraction] {} {{F} \bc (x)P}
  \and
  \inferrule* [lab=concretion] {} {{C} \bc \langle Q \rangle}
  \and
  \inferrule* [lab=process] {} {{P,Q} \bc M \;| \;P|Q \;|\; @{x}}
  \and
  \inferrule* [lab=name] {} {{x} \bc \quotep{P}}
\end{mathpar} 

Note that $\vec{x}$ (resp. $\vec{P}$) denotes a vector of names
(resp. processes) of length $|\vec{x}|$ (resp. $|\vec{P}|$). We adopt
the following useful abbreviations.

\begin{mathpar}
   x?(\vec{y}).P := x.(\vec{y})P \and  x\clift{\vec{P}} := x.\clift{\vec{P}}
   \and x!(y) := \lift{x}{\dropn{y}}
   \and \Pi_{i=0}^{n-1}P_i := P_0 | \ldots | P_{n-1}
\end{mathpar}

\subsubsection{Structural congruence}

\paragraph{Free and bound names and alpha-equivalence.} At the
core of structural equivalence is alpha-equivalence which identifies
process that are the same up to a change of variable. Formally, we
recognize the distinction between free and bound names. The free names
of a process, $\freenames{P}$, may be calculated recursively as
follows:

\begin{mathpar}
\freenames{\pzero} := \emptyset
  \and \\
  \freenames{x?(y).P} := \{ x \} \cup (\freenames{P} \setminus \{ y \})
  \and 
  \freenames{x!\langle P \rangle} := \{ x \} \cup \{ P \} 
  \and \\
  \freenames{P|Q} := \freenames{P} \cup \freenames{Q}
  \and \\
  \freenames{@{x}} := \{ x \}
\end{mathpar}

$\pi$
$\quotep{\pi}$

$\freenames{-} : \pi \to \mathcal{P}(\quotep{\pi})$

\begin{eqnarray*}
  \freenames{\pzero} & := & \emptyset \\
  \freenames{x?(y).P} & := & \{ x \} \cup (\freenames{P} \setminus \{ y \}) \\
  \freenames{x!\langle P \rangle} & := & \{ x \} \cup \{ P \} \\
  \freenames{P|Q} & := & \freenames{P} \cup \freenames{Q} \\
  \freenames{\dropn{x}} & := & \{ x \}
\end{eqnarray*}

The bound names of a process, $\boundnames{P}$, are those names occurring in $P$
that are not free. For example, in $x?(y).0$, the name $x$ is free, while $y$ is bound.

\begin{mathpar}
  \inferrule* [lab=monoidal-laws] {} { P|Q \equiv Q|P \and P|0 \equiv P \and P|(Q|R) \equiv (P|Q)|R }
\end{mathpar}

\begin{mathpar}
  \inferrule* [lab=alpha-equivalence] {} { (x)P \equiv (y)P\{y/x\} \and y \not\in \freenames{P} }
\end{mathpar}

\begin{definition}
Then two processes, $P,Q$, are alpha-equivalent if $P = Q\{\vec{y}/\vec{x}\}$ for
some $\vec{x} \in \boundnames{Q},\vec{y} \in \boundnames{P}$, where $Q\{\vec{y}/\vec{x}\}$
denotes the capture-avoiding substitution of $\vec{y}$ for $\vec{x}$ in $Q$.
\end{definition}

\begin{definition}
  The {\em structural congruence} \cite{SangiorgiWalker} , $\equiv$,
  between processes is the least congruence containing
  alpha-equivalence, satisfying the abelian monoid laws
  (associativity, commutativity and $\pzero$ as identity) for parallel
  composition $|$ and for summation $+$.
\end{definition}

\subsection{Name equivalence}

We take name equivalence, written $\nameeq$, to be the smallest
equivalence relation generated by the following rules.

\begin{mathpar}
\inferrule*[lab=Quote-drop]
{ }
{ \quotep{@{x}} \nameeq x }

\inferrule*[lab=Struct-equiv]
{ P \scong Q }
{ \quotep{P} \nameeq \quotep{Q} }
\end{mathpar}

The astute reader will have noticed that the mutual recursion of names
and processes imposes a mutual recursion on alpha-equivalence and
structural equivalence via name-equivalence. Fortunately, all of this
works out pleasantly and we may calculate in the natural way, free of
concern. The reader interested in the details is referred to the
appendix \ref{appendix:rho_details}.

\subsection{Substitution}

We use $\Proc$ for the set of processes, $\QProc$ for the set of
names, and $\id{\{}\vec{y} / \vec{x} \id{\}}$ to denote partial maps,
$s : \QProc \rightarrow \QProc$. A map, $s$ lifts, uniquely, to a map
on process terms, $\widehat{s} : \Proc \rightarrow \Proc$ by the
following equations.

\begin{mathpar}
  (0) \psubstp{Q}{P} := 0 \\
  (R \juxtap S) \psubstp{Q}{P}
  :=    
  (R)\psubstp{Q}{P} \juxtap (S) \psubstp{Q}{P} \\
  (x?(y).R) \psubstp{Q}{P}    
  :=    
  (x)\substp{Q}{P} (z)\concat( (R \psubstn{z}{y}) \psubstp{Q}{P} ) \\
  (\lift{x}{R}) \psubstp{Q}{P}  
  :=
  \lift{(x)\substp{Q}{P}}{ R \psubstp{Q}{P} } \\
%   (\dropn{x})  \psubstp{Q}{P}       
%   := 
%   \left\{ 
%     \begin{array}{ccc} 
%       \dropn{\quotep{Q}} & & x \nameeq \quotep{P} \\
%       \dropn{x} & & otherwise \\
%     \end{array}
%   \right. 
  (\dropn{x})  \psubstp{Q}{P}       
  := 
  \left\{ 
    \begin{array}{ccc} 
      Q & & x \nameeq \quotep{P} \\
      \dropn{x} & & otherwise \\
    \end{array}
  \right.
\end{mathpar}
 

where

\begin{eqnarray}
  (x)\id{\{} \lpquote Q \rpquote / \lpquote P \rpquote \id{\}}            = 
  \left\{ 
    \begin{array}{ccc}
      \lpquote Q \rpquote & & x \nameeq \lpquote P \rpquote \\
      x & & otherwise \\
    \end{array}
  \right. \nonumber
\end{eqnarray}

and $z$ is chosen distinct from $\quotep{P}$, $\quotep{Q}$, the free
names in $Q$, and all the names in $R$. Our $\alpha$-equivalence will
be built in the standard way from this substitution.

\begin{remark}\label{rem:no_self_referential_names}
  One consequence of these definitions is that $\forall P. \quotep{P}
  \not\in \freenames{P}$.
\end{remark}

\subsection{ Dynamic quote: an example }

Anticipating something of what's to come, consider applying the
substitution, $\widehat{\id{\{}u / z \id{\}}}$, to the following pair
of processes, $\lift{w}{y!(z)}$ and $w[ \lpquote y!(z) \rpquote ]$.

\begin{eqnarray}
	\lift{w}{y!(z)}\widehat{\id{\{}u / z \id{\}}}
		& = &
		\lift{w}{y!(u)} \nonumber\\
	w[ \lpquote y!(z) \rpquote ] \widehat{ \id{\{}u / z \id{\}} }
		& = &
		w[ \lpquote y!(z) \rpquote ] \nonumber
\end{eqnarray}

Because the body of the process between quotes is impervious to
substitution, we get radically different answers. In fact, by
examining the first process in an input context,
e.g. $x?(z).\lift{w}{y!(z)}$, we see that the process under the lift
operator may be shaped by prefixed inputs binding a name inside it. In
this sense, the lift operator will be seen as a way to dynamically
construct processes before reifying them as names.

Finally equipped with these standard features we can present the
dynamics of the calculus.

\subsubsection{Operational semantics} 

Finally, we introduce the computational dynamics. What marks these
algebras as distinct from other more traditionally studied algebraic
structures, e.g. vector spaces or polynomial rings, is the manner in
which dynamics is captured. In traditional structures, dynamics is typically
expressed through morphisms between such structures, as in linear maps
between vector spaces or morphisms between rings. In algebras
associated with the semantics of computation, the dynamics is
expressed as part of the algebraic structure itself, through a
reduction reduction relation typically denoted by $\red$. Below, we
give a recursive presentation of this relation for the calculus used
in the encoding.

$\red \subseteq \pi \times \pi$
$\red : \pi \to \mathcal{P}(\pi)$

\begin{mathpar}
  \inferrule* [lab=Comm] { \textsf{match}( x_{src}, x_{trgt} ) } { x_{trgt}?(y)P \; | \; x_{src}!\langle {Q} \rangle \red P\{\quotep{Q}/y}\} }
  \and \\
  \inferrule* [lab=Par] {{P} \red {P}'} {{{P} | {Q}} \red {{P}' | {Q}}}
  \and
  \inferrule* [lab=Equiv]{{{P} \scong {P}'} \andalso {{P}' \red {Q}'} \andalso {{Q}' \scong {Q}}}{{P} \red {Q}}
\end{mathpar}

\begin{eqnarray*}
  match_{\equiv} (\quotep{P},\quotep{Q}) & := & P \equiv Q \\
  match_{\dagger}(\quotep{P},\quotep{Q}) & := & \forall R. P|Q \red^{*} R => R \red^{*} 0 \\
  match_{K}(\quotep{P},\quotep{Q}) & := & K \mbox{ for some context } K
\end{eqnarray*}

$u?(x)P | u!\langle Q \rangle \red P\{\quotep{Q}/x\}$

%We write $\wred$ for $\red^*$, and $P\red$ if $\exists Q $ such that $ P \red Q$.
We write $P\red$ if $\exists Q $ such that $ P \red Q$ and $P\not\red$, otherwise.

\section{Replication}

As mentioned before, it is known that replication (and hence
recursion) can be implemented in a higher-order process algebra
\cite{SangiorgiWalker}. As our first example of calculation with the
machinery thus far presented we give the construction explicitly in
the {\rhoc}.

\begin{eqnarray}
	D_{x} & := & \prefix{x}{y}{(\binpar{\outputp{x}{y}}{@{y}})} \nonumber\\
	\bangp_{x}{P} & := & \binpar{{x}!\langle{\binpar{D_{x}}{P}}\rangle}{D_{x}} \nonumber
\end{eqnarray}

\begin{eqnarray}
	\bangp_{x}{P} & & \nonumber\\
	=
	& {x}!\langle{(\prefix{x}{y}{(\outputp{x}{y} | @{y})) | P}}\rangle 
	      | \prefix{x}{y}{(\outputp{x}{y} | @{y})} & \nonumber\\
	\red
	& (\outputp{x}{y} | @{y})\substn{\quotep{(\prefix{x}{y}{(@{y} | \outputp{x}{y})) | P}}}{y} & \nonumber\\
	=
	& \outputp{x}{\quotep{(\prefix{x}{y}{(\outputp{x}{y} | @{y})) | P}}}
	  | {(\prefix{x}{y}{(\outputp{x}{y} | @{y})) | P}} & \nonumber\\
	\red
	& \ldots & \nonumber\\
	\red^*
	& P | P | \ldots & \nonumber
\end{eqnarray}

Of course, this encoding, as an implementation, runs away, unfolding
$\bangp{P}$ eagerly. A lazier and more implementable replication
operator, restricted to input-guarded processes, may be obtained as follows.

\begin{eqnarray}
\bangp{\prefix{u}{v}{P}} 
	:= 
	\binpar{\lift{x}{\prefix{u}{v}{(\binpar{D(x)}{P})}}}{D(x)} \nonumber
\end{eqnarray}

\begin{remark}
  Note that the lazier definition still does not deal with summation
  or mixed summation (i.e. sums over input and output). The reader is
  invited to construct definitions of replication that deal with these
  features. 

  Further, the definitions are parameterized in a name, $x$. Can you,
  gentle reader, make a definition that eliminates this parameter and
  guarantees no accidental interaction between the replication
  machinery and the process being replicated -- i.e. no accidental
  sharing of names used by the process to get its work done and the
  name(s) used by the replication to effect copying. This latter
  revision of the definition of replication is crucial to obtaining
  the expected identity $!!P \sim !P$.
\end{remark}

\begin{remark}\label{rem:paradoxical_combinator}
  The reader familiar with the lambda calculus will have noticed the
  similarity between $D$ and the paradoxical combinator.

  [Ed. note: the existence of this seems to suggest we have to be more
  restrictive on the set of processes and names we admit if we are to
  support no-cloning.]
\end{remark}

\subsubsection{Bisimulation}

The computational dynamics gives rise to another kind of equivalence,
the equivalence of computational behavior. As previously mentioned
this is typically captured \emph{via} some form of bisimulation.

% The notion we use in this paper is weak barbed bisimulation
% \cite{milner91polyadicpi}.

The notion we use in this paper is derived from weak barbed
bisimulation \cite{milner91polyadicpi}. 

\begin{definition}
An \emph{observation relation}, $\downarrow_{\mathcal N}$, over a set
of names, $\mathcal N$, is the smallest relation satisfying the rules
below.

\infrule[Out-barb]{y \in {\mathcal N}, \; x \nameeq y}
		  {\outputp{x}{v} \downarrow_{\mathcal N} x}
\infrule[Par-barb]{\mbox{$P\downarrow_{\mathcal N} x$ or $Q\downarrow_{\mathcal N} x$}}
		  {\binpar{P}{Q} \downarrow_{\mathcal N} x}

We write $P \Downarrow_{\mathcal N} x$ if there is $Q$ such that 
$P \wred Q$ and $Q \downarrow_{\mathcal N} x$.
\end{definition}

\begin{definition}
%\label{def.bbisim}
An  ${\mathcal N}$-\emph{barbed bisimulation} over a set of names, ${\mathcal N}$, is a symmetric binary relation 
${\mathcal S}_{\mathcal N}$ between agents such that $P\rel{S}_{\mathcal N}Q$ implies:
\begin{enumerate}
\item If $P \red P'$ then $Q \wred Q'$ and $P'\rel{S}_{\mathcal N} Q'$.
\item If $P\downarrow_{\mathcal N} x$, then $Q\Downarrow_{\mathcal N} x$.
\end{enumerate}
$P$ is ${\mathcal N}$-barbed bisimilar to $Q$, written
$P \wbbisim_{\mathcal N} Q$, if $P \rel{S}_{\mathcal N} Q$ for some ${\mathcal N}$-barbed bisimulation ${\mathcal S}_{\mathcal N}$.
\end{definition}

$\mathcal{R} \subseteq \pi \times \pi$

$P \mathcal{R} Q => \forall P'. P \red P' \Rightarrow \exists Q'. Q \red Q', P' \mathcal{R} Q'$

$P \vdash x \Rightarrow Q \vdash x$

\begin{mathpar}
  \inferrule*[lab=Out-barb]{x \nameeq y}{{y}!\langle{Q}\rangle \vdash x}
  \and
  \inferrule*[lab=Par-barb]{\mbox{$P\vdash x$ or $Q\vdash x$}}{\binpar{P}{Q} \vdash x}
\end{mathpar}

\subsubsection{Contexts}

One of the principle advantages of computational calculi like the
$\pi$-calculus is a well-defined notion of context,
contextual-equivalence and a correlation between
contextual-equivalence and notions of bisimulation. The notion of
context allows the decomposition of a process into (sub-)process and
its syntactic environment, its context. Thus, a context may be
thought of as a process with a ``hole'' (written $\Box$) in it. The
application of a context $M$ to a process $P$, written $M[P]$, is
tantamount to filling the hole in $M$ with $P$. In this paper we do
not need the full weight of this theory, but do make use of the notion
of context in the proof the main theorem. 

\begin{mathpar}
  \inferrule* [lab=summation] {} {{M_{M},M_{N}} \bc \Box \;|\; x.M_{A} \;|\; M_{M}+M_{N}}
  \and
  \inferrule* [lab=agent] {} {{M_{A}} \bc (\vec{x})M_{P} \;| \; \clift{P_0,\ldots,M_{P},\ldots,P_N}}
  \and \\
  \inferrule* [lab=process] {} {{M_{P}} \bc M_{N} \;| \;P|M_{P} }
\end{mathpar} 

\begin{mathpar}
  \inferrule* [lab=sychronization] {} {M_{N} \bc \Box \;|\; x?M_{F} \;|\; x!M_{C}}
  \and
  \inferrule* [lab=abstraction] {} {{M_{F}} \bc (x)M_{P} }
  \and
  \inferrule* [lab=concretion] {} {{M_{C}} \bc \langle M_{P} \rangle }
  \and \\
  \inferrule* [lab=process] {} {{M_{P}} \bc M_{N} \;| \;P|M_{P} }
\end{mathpar}

\begin{definition}[contextual application] Given a context $M$, and
  process $P$, we define the \emph{contextual application}, $M[P] :=
  M\{P/\Box\}$. That is, the contextual application of M to P is the
  substitution of $P$ for $\Box$ in $M$.
\end{definition}

$\meaningof{-} : L \to \mathcal{P}(\pi)$

\begin{mathpar}
  \inferrule* [lab=collection] {} {\meaningof{true} = \pi, \and \meaningof{~E} = \pi \setminus \meaningof{E}, \and \meaningof{E_{1} \& E_{2}} = \meaningof{E_{1}} \cap \meaningof{E_{2}}}
\end{mathpar}

\begin{mathpar}
  \inferrule* [lab=structure] {} {\meaningof{0} = \{ P \in \pi | P \equiv 0 \}, \and \\ \meaningof{E_1 | E_2} = \{ P \in \pi | P \equiv P_{1} | P_{2}, P_{1} \in \meaningof{E_{1}}, P_{2} \in \meaningof{E_2}\} }
\end{mathpar}

\begin{mathpar}
 \inferrule* [lab=behavior] {} {\meaningof{\langle a?b \rangle E} = \{ P \in \pi | P \equiv Q | u?(y)P', \\ \and \\\\ \and \\ \;\;\; u \in \meaningof{a}, \forall z.P'\{z/y\} \in \meaningof{E\{z/b\}}\}, \and \\ \meaningof{a!E} = \{ P \in \pi | P \equiv Q | x!\langle P' \rangle, x \in \meaningof{a} P' \in \meaningof{E}\} }
\end{mathpar}

\begin{mathpar}
 \inferrule* [lab=nominal] {} {\meaningof{\quotep{E}} = \{ \quotep{P} \in \quotep{\pi} | P \in \meaningof{E} \}, \and \meaningof{\quotep{P}} = \{ \quotep{Q} \in \quotep{\pi} | P \equiv Q \} \and \\ \meaningof{@\quotep{E}} = \{ P \in \pi | P \equiv @x, x \in \meaningof{E} \}}
\end{mathpar}

\begin{eqnarray*}
  \\
  \meaningof{-} : TS \to ST
\end{eqnarray*}

\begin{eqnarray*}
  \\
  L : TS \to ST
\end{eqnarray*}

\begin{eqnarray*}
  \\
  P \models E \iff P \in \meaningof{E}
\end{eqnarray*}

\begin{eqnarray*}
  P \approx_{L} Q \iff \forall E \in L. P \models E \iff Q \models E
\end{eqnarray*}

\begin{eqnarray*}
  P \approx_{K} Q
\end{eqnarray*}

\begin{eqnarray*}
  P \approx Q
\end{eqnarray*}

$\approx_{K} = \approx = \approx_{L}$

\subsubsection{Contextual duality}

Note that contexts extend the quotation operation to a family of
operations from processes to names. Given a context, $M$, we can
define a \emph{nominal context}, $\quotep{M}$ by $\quotep{M}[P] :=
\quotep{M[P]}$. To foreshadow what is to come we observe that these
operations enjoy a duality with processes very much like the duality
between vectors and maps from vectors to scalars.

Further, because the calculus is essentially higher-order, we have a
correspondence between contexts and processes. More specifically,
given a name $x$ and a context $M$ we can construct $M^{*}_{x}$ such
that 

\begin{mathpar}
  M^{*}_{x} | \lift{x}{P} \red M[P]
\end{mathpar}

namely,

\begin{mathpar}
  M^{*}_{x} := x?(u).M[\dropn{u}]
\end{mathpar}

The dependence of $M^{*}_{x}$ on a name makes it an abstraction, 

\begin{mathpar}
  M^{*} := (x)x?(u).M[\dropn{u}]
\end{mathpar}

\subsection{Additional notation}

It will sometimes be convenient to denote the process a name
quotes. We already have the notation $x = \quotep{P}$, but it will be
convenient to introduce an alternate notation, $\procn{x}$, when we
want to emphasize the connection to the use of the name. Note that, by
virtue of name equivalence, $\quotep{\procn{x}} \nameeq x$; so, the
notation is consistent with previous definitions.

Further, because names have structure it is possible to effect
substitutions on the basis of that structure. This means we need to
upgrade our notation for substitutions, which we accomplish by
adapting comprehension notation. Thus,

\begin{mathpar}
  P\{ y / x : x \in S \}
\end{mathpar}

is interpreted to mean the process derived from P by replacing (in a
capture-avoiding manner) each occurrence of $x$ in $S$ by $y$. For example,

\begin{mathpar}
  P\{ \quotep{\procn{x}|\procn{x}} / x : x \in \freenames{P} \}
\end{mathpar}

will replace each (occurrence) of a free name $x$ in $P$ by
$\quotep{\procn{x}|\procn{x}}$.

Also, we will avail ourselves of the notation $x^{L}$ and $x^{R}$ to
denote injections of a name into disjoint copies of the name
space. There are numerous ways to accomplish this. One example can be
found in \cite{MeredithR05}. This notation overloads to vectors of
names: $\vec{x}^{\pi} := (x_{i}^{\pi} \; : \; 0 \leq i < |\vec{x}| )$ where $\pi \in \{L,R\}$.

We also use $P^{\Box} := P|\Box$.

In \cite{MeredithR05} an interpretation of the new operator is
given. It turns out that there are several possible interpretations
all enjoying the requisite algebraic properties of the operator (see
\cite{milner91polyadicpi}). We will therefore make liberal use of
$(\nu\; \vec{x})P$.

% subsection the_syntax_and_semantics_of_the_notation_system (end)   

\input{qm2pi.qmops} 

\input{qm2pi.sterngerlach} 

\input{qm2pi.metric} 

% section concurrent_process_calculi (end)

%\input{qm2pi.proofsketch}

% section proof sketch (end)

%\input{qm2pi.slviaknots} 

% section spatial logic via knots (end)

\input{qm2pi.conclusion}

% section conclusion (end)

%\input{qm2pi.dtcodes} 

% section wiring algorithm (end)

\input{qm2pi.ack} 

% section acknowledgments (end)

\newpage


\bibliographystyle{plain}   
\bibliography{../../biblios/main.bib}

\input{qm2pi.rhodetails}

\end{document}

 

\documentclass[12pt]{llncs}
%\documentclass{jktr}

\usepackage[pdftex]{hyperref}                   
\usepackage {listings}
\usepackage {mathpartir}
\usepackage{bcprules}
%\usepackage{listings}
                       
\usepackage{graphicx} 
%\usepackage[margins=2.5cm,nohead,nofoot]{geometry}
%\usepackage{geometry}
\usepackage{amsfonts}
\usepackage{amstext}
\usepackage{latexsym}
\usepackage{amssymb}
\usepackage{color}


%\include{myPreamble}
\include{qm2pi.local} 

%\ifpdf
%\usepackage[pdftex]{graphicx}
%\else
%\usepackage{graphicx}
%\fi

 % \ifpdf
%  \usepackage{pdfsync}
%  \if


%\title{Brief Article}
%\author{David F. Snyder}
%\author{L.G. Meredith}

%\address{Dept. of Math., Texas State University--San Marcos, San Marcos, TX 78666}
       
\pagestyle{empty}


\begin{document}

\lstset{language=[Objective]Caml,frame=shadowbox}

\input{qm2pi.front}

% section front matter (end)

\input{qm2pi.intro} 
 
% section introduction (end)

% \input{qm2pi.knotations} 

% section notation (end)

\input{qm2pi.process.calculi} 

% section concurrent_process_calculi_and_spatial_logics_ (end)
    
%\input{qm2pi.knots2pi} 

%\input{qm2pi.trefoil} 

%\input{qm2pi.mainthm} 

% subsection basic_interpretation (end)

%\input{qm2pi.rho.presentation} 
\subsection{The syntax and semantics of the notation system}\label{sub:the_syntax_and_semantics_of_the_notation_system} % (fold)

We now summarize a technical presentation of the calculus that
embodies our theory of dynamics. The typical presentation of such a
calculus follows the style of giving generators and relations on
them. The grammar, below, describing term constructors, freely
generates the set of processes, $\Proc$. This set is then quotiented
by a relation known as structural congruence and it is over this set
that the notion of dynamics is expressed. This presentation is
essentially that of \cite{MeredithR05} with the addition of
polyadicity and summation. For readability we have relegated some of
the technical subtleties to an appendix.

\subsubsection{Process grammar}\label{subsub:process_grammar}

\begin{mathpar}
  \inferrule* [lab=synchronization] {} {{M} \bc \pzero \;|\; x?F \;|\; x!C }
  \and
  \inferrule* [lab=abstraction] {} {{F} \bc (x)P}
  \and
  \inferrule* [lab=concretion] {} {{C} \bc \langle Q \rangle}
  \and
  \inferrule* [lab=process] {} {{P,Q} \bc M \;| \;P|Q \;|\; @{x}}
  \and
  \inferrule* [lab=name] {} {{x} \bc \quotep{P}}
\end{mathpar} 

Note that $\vec{x}$ (resp. $\vec{P}$) denotes a vector of names
(resp. processes) of length $|\vec{x}|$ (resp. $|\vec{P}|$). We adopt
the following useful abbreviations.

\begin{mathpar}
   x?(\vec{y}).P := x.(\vec{y})P \and  x\clift{\vec{P}} := x.\clift{\vec{P}}
   \and x!(y) := \lift{x}{\dropn{y}}
   \and \Pi_{i=0}^{n-1}P_i := P_0 | \ldots | P_{n-1}
\end{mathpar}

\subsubsection{Structural congruence}

\paragraph{Free and bound names and alpha-equivalence.} At the
core of structural equivalence is alpha-equivalence which identifies
process that are the same up to a change of variable. Formally, we
recognize the distinction between free and bound names. The free names
of a process, $\freenames{P}$, may be calculated recursively as
follows:

\begin{mathpar}
\freenames{\pzero} := \emptyset
  \and \\
  \freenames{x?(y).P} := \{ x \} \cup (\freenames{P} \setminus \{ y \})
  \and 
  \freenames{x!\langle P \rangle} := \{ x \} \cup \{ P \} 
  \and \\
  \freenames{P|Q} := \freenames{P} \cup \freenames{Q}
  \and \\
  \freenames{@{x}} := \{ x \}
\end{mathpar}

$\pi$
$\quotep{\pi}$

$\freenames{-} : \pi \to \mathcal{P}(\quotep{\pi})$

\begin{eqnarray*}
  \freenames{\pzero} & := & \emptyset \\
  \freenames{x?(y).P} & := & \{ x \} \cup (\freenames{P} \setminus \{ y \}) \\
  \freenames{x!\langle P \rangle} & := & \{ x \} \cup \{ P \} \\
  \freenames{P|Q} & := & \freenames{P} \cup \freenames{Q} \\
  \freenames{\dropn{x}} & := & \{ x \}
\end{eqnarray*}

The bound names of a process, $\boundnames{P}$, are those names occurring in $P$
that are not free. For example, in $x?(y).0$, the name $x$ is free, while $y$ is bound.

\begin{mathpar}
  \inferrule* [lab=monoidal-laws] {} { P|Q \equiv Q|P \and P|0 \equiv P \and P|(Q|R) \equiv (P|Q)|R }
\end{mathpar}

\begin{mathpar}
  \inferrule* [lab=alpha-equivalence] {} { (x)P \equiv (y)P\{y/x\} \and y \not\in \freenames{P} }
\end{mathpar}

\begin{definition}
Then two processes, $P,Q$, are alpha-equivalent if $P = Q\{\vec{y}/\vec{x}\}$ for
some $\vec{x} \in \boundnames{Q},\vec{y} \in \boundnames{P}$, where $Q\{\vec{y}/\vec{x}\}$
denotes the capture-avoiding substitution of $\vec{y}$ for $\vec{x}$ in $Q$.
\end{definition}

\begin{definition}
  The {\em structural congruence} \cite{SangiorgiWalker} , $\equiv$,
  between processes is the least congruence containing
  alpha-equivalence, satisfying the abelian monoid laws
  (associativity, commutativity and $\pzero$ as identity) for parallel
  composition $|$ and for summation $+$.
\end{definition}

\subsection{Name equivalence}

We take name equivalence, written $\nameeq$, to be the smallest
equivalence relation generated by the following rules.

\begin{mathpar}
\inferrule*[lab=Quote-drop]
{ }
{ \quotep{@{x}} \nameeq x }

\inferrule*[lab=Struct-equiv]
{ P \scong Q }
{ \quotep{P} \nameeq \quotep{Q} }
\end{mathpar}

The astute reader will have noticed that the mutual recursion of names
and processes imposes a mutual recursion on alpha-equivalence and
structural equivalence via name-equivalence. Fortunately, all of this
works out pleasantly and we may calculate in the natural way, free of
concern. The reader interested in the details is referred to the
appendix \ref{appendix:rho_details}.

\subsection{Substitution}

We use $\Proc$ for the set of processes, $\QProc$ for the set of
names, and $\id{\{}\vec{y} / \vec{x} \id{\}}$ to denote partial maps,
$s : \QProc \rightarrow \QProc$. A map, $s$ lifts, uniquely, to a map
on process terms, $\widehat{s} : \Proc \rightarrow \Proc$ by the
following equations.

\begin{mathpar}
  (0) \psubstp{Q}{P} := 0 \\
  (R \juxtap S) \psubstp{Q}{P}
  :=    
  (R)\psubstp{Q}{P} \juxtap (S) \psubstp{Q}{P} \\
  (x?(y).R) \psubstp{Q}{P}    
  :=    
  (x)\substp{Q}{P} (z)\concat( (R \psubstn{z}{y}) \psubstp{Q}{P} ) \\
  (\lift{x}{R}) \psubstp{Q}{P}  
  :=
  \lift{(x)\substp{Q}{P}}{ R \psubstp{Q}{P} } \\
%   (\dropn{x})  \psubstp{Q}{P}       
%   := 
%   \left\{ 
%     \begin{array}{ccc} 
%       \dropn{\quotep{Q}} & & x \nameeq \quotep{P} \\
%       \dropn{x} & & otherwise \\
%     \end{array}
%   \right. 
  (\dropn{x})  \psubstp{Q}{P}       
  := 
  \left\{ 
    \begin{array}{ccc} 
      Q & & x \nameeq \quotep{P} \\
      \dropn{x} & & otherwise \\
    \end{array}
  \right.
\end{mathpar}
 

where

\begin{eqnarray}
  (x)\id{\{} \lpquote Q \rpquote / \lpquote P \rpquote \id{\}}            = 
  \left\{ 
    \begin{array}{ccc}
      \lpquote Q \rpquote & & x \nameeq \lpquote P \rpquote \\
      x & & otherwise \\
    \end{array}
  \right. \nonumber
\end{eqnarray}

and $z$ is chosen distinct from $\quotep{P}$, $\quotep{Q}$, the free
names in $Q$, and all the names in $R$. Our $\alpha$-equivalence will
be built in the standard way from this substitution.

\begin{remark}\label{rem:no_self_referential_names}
  One consequence of these definitions is that $\forall P. \quotep{P}
  \not\in \freenames{P}$.
\end{remark}

\subsection{ Dynamic quote: an example }

Anticipating something of what's to come, consider applying the
substitution, $\widehat{\id{\{}u / z \id{\}}}$, to the following pair
of processes, $\lift{w}{y!(z)}$ and $w[ \lpquote y!(z) \rpquote ]$.

\begin{eqnarray}
	\lift{w}{y!(z)}\widehat{\id{\{}u / z \id{\}}}
		& = &
		\lift{w}{y!(u)} \nonumber\\
	w[ \lpquote y!(z) \rpquote ] \widehat{ \id{\{}u / z \id{\}} }
		& = &
		w[ \lpquote y!(z) \rpquote ] \nonumber
\end{eqnarray}

Because the body of the process between quotes is impervious to
substitution, we get radically different answers. In fact, by
examining the first process in an input context,
e.g. $x?(z).\lift{w}{y!(z)}$, we see that the process under the lift
operator may be shaped by prefixed inputs binding a name inside it. In
this sense, the lift operator will be seen as a way to dynamically
construct processes before reifying them as names.

Finally equipped with these standard features we can present the
dynamics of the calculus.

\subsubsection{Operational semantics} 

Finally, we introduce the computational dynamics. What marks these
algebras as distinct from other more traditionally studied algebraic
structures, e.g. vector spaces or polynomial rings, is the manner in
which dynamics is captured. In traditional structures, dynamics is typically
expressed through morphisms between such structures, as in linear maps
between vector spaces or morphisms between rings. In algebras
associated with the semantics of computation, the dynamics is
expressed as part of the algebraic structure itself, through a
reduction reduction relation typically denoted by $\red$. Below, we
give a recursive presentation of this relation for the calculus used
in the encoding.

$\red \subseteq \pi \times \pi$
$\red : \pi \to \mathcal{P}(\pi)$

\begin{mathpar}
  \inferrule* [lab=Comm] { \textsf{match}( x_{src}, x_{trgt} ) } { x_{trgt}?(y)P \; | \; x_{src}!\langle {Q} \rangle \red P\{\quotep{Q}/y}\} }
  \and \\
  \inferrule* [lab=Par] {{P} \red {P}'} {{{P} | {Q}} \red {{P}' | {Q}}}
  \and
  \inferrule* [lab=Equiv]{{{P} \scong {P}'} \andalso {{P}' \red {Q}'} \andalso {{Q}' \scong {Q}}}{{P} \red {Q}}
\end{mathpar}

\begin{eqnarray*}
  match_{\equiv} (\quotep{P},\quotep{Q}) & := & P \equiv Q \\
  match_{\dagger}(\quotep{P},\quotep{Q}) & := & \forall R. P|Q \red^{*} R => R \red^{*} 0 \\
  match_{K}(\quotep{P},\quotep{Q}) & := & K \mbox{ for some context } K
\end{eqnarray*}

$u?(x)P | u!\langle Q \rangle \red P\{\quotep{Q}/x\}$

%We write $\wred$ for $\red^*$, and $P\red$ if $\exists Q $ such that $ P \red Q$.
We write $P\red$ if $\exists Q $ such that $ P \red Q$ and $P\not\red$, otherwise.

\section{Replication}

As mentioned before, it is known that replication (and hence
recursion) can be implemented in a higher-order process algebra
\cite{SangiorgiWalker}. As our first example of calculation with the
machinery thus far presented we give the construction explicitly in
the {\rhoc}.

\begin{eqnarray}
	D_{x} & := & \prefix{x}{y}{(\binpar{\outputp{x}{y}}{@{y}})} \nonumber\\
	\bangp_{x}{P} & := & \binpar{{x}!\langle{\binpar{D_{x}}{P}}\rangle}{D_{x}} \nonumber
\end{eqnarray}

\begin{eqnarray}
	\bangp_{x}{P} & & \nonumber\\
	=
	& {x}!\langle{(\prefix{x}{y}{(\outputp{x}{y} | @{y})) | P}}\rangle 
	      | \prefix{x}{y}{(\outputp{x}{y} | @{y})} & \nonumber\\
	\red
	& (\outputp{x}{y} | @{y})\substn{\quotep{(\prefix{x}{y}{(@{y} | \outputp{x}{y})) | P}}}{y} & \nonumber\\
	=
	& \outputp{x}{\quotep{(\prefix{x}{y}{(\outputp{x}{y} | @{y})) | P}}}
	  | {(\prefix{x}{y}{(\outputp{x}{y} | @{y})) | P}} & \nonumber\\
	\red
	& \ldots & \nonumber\\
	\red^*
	& P | P | \ldots & \nonumber
\end{eqnarray}

Of course, this encoding, as an implementation, runs away, unfolding
$\bangp{P}$ eagerly. A lazier and more implementable replication
operator, restricted to input-guarded processes, may be obtained as follows.

\begin{eqnarray}
\bangp{\prefix{u}{v}{P}} 
	:= 
	\binpar{\lift{x}{\prefix{u}{v}{(\binpar{D(x)}{P})}}}{D(x)} \nonumber
\end{eqnarray}

\begin{remark}
  Note that the lazier definition still does not deal with summation
  or mixed summation (i.e. sums over input and output). The reader is
  invited to construct definitions of replication that deal with these
  features. 

  Further, the definitions are parameterized in a name, $x$. Can you,
  gentle reader, make a definition that eliminates this parameter and
  guarantees no accidental interaction between the replication
  machinery and the process being replicated -- i.e. no accidental
  sharing of names used by the process to get its work done and the
  name(s) used by the replication to effect copying. This latter
  revision of the definition of replication is crucial to obtaining
  the expected identity $!!P \sim !P$.
\end{remark}

\begin{remark}\label{rem:paradoxical_combinator}
  The reader familiar with the lambda calculus will have noticed the
  similarity between $D$ and the paradoxical combinator.

  [Ed. note: the existence of this seems to suggest we have to be more
  restrictive on the set of processes and names we admit if we are to
  support no-cloning.]
\end{remark}

\subsubsection{Bisimulation}

The computational dynamics gives rise to another kind of equivalence,
the equivalence of computational behavior. As previously mentioned
this is typically captured \emph{via} some form of bisimulation.

% The notion we use in this paper is weak barbed bisimulation
% \cite{milner91polyadicpi}.

The notion we use in this paper is derived from weak barbed
bisimulation \cite{milner91polyadicpi}. 

\begin{definition}
An \emph{observation relation}, $\downarrow_{\mathcal N}$, over a set
of names, $\mathcal N$, is the smallest relation satisfying the rules
below.

\infrule[Out-barb]{y \in {\mathcal N}, \; x \nameeq y}
		  {\outputp{x}{v} \downarrow_{\mathcal N} x}
\infrule[Par-barb]{\mbox{$P\downarrow_{\mathcal N} x$ or $Q\downarrow_{\mathcal N} x$}}
		  {\binpar{P}{Q} \downarrow_{\mathcal N} x}

We write $P \Downarrow_{\mathcal N} x$ if there is $Q$ such that 
$P \wred Q$ and $Q \downarrow_{\mathcal N} x$.
\end{definition}

\begin{definition}
%\label{def.bbisim}
An  ${\mathcal N}$-\emph{barbed bisimulation} over a set of names, ${\mathcal N}$, is a symmetric binary relation 
${\mathcal S}_{\mathcal N}$ between agents such that $P\rel{S}_{\mathcal N}Q$ implies:
\begin{enumerate}
\item If $P \red P'$ then $Q \wred Q'$ and $P'\rel{S}_{\mathcal N} Q'$.
\item If $P\downarrow_{\mathcal N} x$, then $Q\Downarrow_{\mathcal N} x$.
\end{enumerate}
$P$ is ${\mathcal N}$-barbed bisimilar to $Q$, written
$P \wbbisim_{\mathcal N} Q$, if $P \rel{S}_{\mathcal N} Q$ for some ${\mathcal N}$-barbed bisimulation ${\mathcal S}_{\mathcal N}$.
\end{definition}

$\mathcal{R} \subseteq \pi \times \pi$

$P \mathcal{R} Q => \forall P'. P \red P' \Rightarrow \exists Q'. Q \red Q', P' \mathcal{R} Q'$

$P \vdash x \Rightarrow Q \vdash x$

\begin{mathpar}
  \inferrule*[lab=Out-barb]{x \nameeq y}{{y}!\langle{Q}\rangle \vdash x}
  \and
  \inferrule*[lab=Par-barb]{\mbox{$P\vdash x$ or $Q\vdash x$}}{\binpar{P}{Q} \vdash x}
\end{mathpar}

\subsubsection{Contexts}

One of the principle advantages of computational calculi like the
$\pi$-calculus is a well-defined notion of context,
contextual-equivalence and a correlation between
contextual-equivalence and notions of bisimulation. The notion of
context allows the decomposition of a process into (sub-)process and
its syntactic environment, its context. Thus, a context may be
thought of as a process with a ``hole'' (written $\Box$) in it. The
application of a context $M$ to a process $P$, written $M[P]$, is
tantamount to filling the hole in $M$ with $P$. In this paper we do
not need the full weight of this theory, but do make use of the notion
of context in the proof the main theorem. 

\begin{mathpar}
  \inferrule* [lab=summation] {} {{M_{M},M_{N}} \bc \Box \;|\; x.M_{A} \;|\; M_{M}+M_{N}}
  \and
  \inferrule* [lab=agent] {} {{M_{A}} \bc (\vec{x})M_{P} \;| \; \clift{P_0,\ldots,M_{P},\ldots,P_N}}
  \and \\
  \inferrule* [lab=process] {} {{M_{P}} \bc M_{N} \;| \;P|M_{P} }
\end{mathpar} 

\begin{mathpar}
  \inferrule* [lab=sychronization] {} {M_{N} \bc \Box \;|\; x?M_{F} \;|\; x!M_{C}}
  \and
  \inferrule* [lab=abstraction] {} {{M_{F}} \bc (x)M_{P} }
  \and
  \inferrule* [lab=concretion] {} {{M_{C}} \bc \langle M_{P} \rangle }
  \and \\
  \inferrule* [lab=process] {} {{M_{P}} \bc M_{N} \;| \;P|M_{P} }
\end{mathpar}

\begin{definition}[contextual application] Given a context $M$, and
  process $P$, we define the \emph{contextual application}, $M[P] :=
  M\{P/\Box\}$. That is, the contextual application of M to P is the
  substitution of $P$ for $\Box$ in $M$.
\end{definition}

$\meaningof{-} : L \to \mathcal{P}(\pi)$

\begin{mathpar}
  \inferrule* [lab=collection] {} {\meaningof{true} = \pi, \and \meaningof{~E} = \pi \setminus \meaningof{E}, \and \meaningof{E_{1} \& E_{2}} = \meaningof{E_{1}} \cap \meaningof{E_{2}}}
\end{mathpar}

\begin{mathpar}
  \inferrule* [lab=structure] {} {\meaningof{0} = \{ P \in \pi | P \equiv 0 \}, \and \\ \meaningof{E_1 | E_2} = \{ P \in \pi | P \equiv P_{1} | P_{2}, P_{1} \in \meaningof{E_{1}}, P_{2} \in \meaningof{E_2}\} }
\end{mathpar}

\begin{mathpar}
 \inferrule* [lab=behavior] {} {\meaningof{\langle a?b \rangle E} = \{ P \in \pi | P \equiv Q | u?(y)P', \\ \and \\\\ \and \\ \;\;\; u \in \meaningof{a}, \forall z.P'\{z/y\} \in \meaningof{E\{z/b\}}\}, \and \\ \meaningof{a!E} = \{ P \in \pi | P \equiv Q | x!\langle P' \rangle, x \in \meaningof{a} P' \in \meaningof{E}\} }
\end{mathpar}

\begin{mathpar}
 \inferrule* [lab=nominal] {} {\meaningof{\quotep{E}} = \{ \quotep{P} \in \quotep{\pi} | P \in \meaningof{E} \}, \and \meaningof{\quotep{P}} = \{ \quotep{Q} \in \quotep{\pi} | P \equiv Q \} \and \\ \meaningof{@\quotep{E}} = \{ P \in \pi | P \equiv @x, x \in \meaningof{E} \}}
\end{mathpar}

\begin{eqnarray*}
  \\
  \meaningof{-} : TS \to ST
\end{eqnarray*}

\begin{eqnarray*}
  \\
  L : TS \to ST
\end{eqnarray*}

\begin{eqnarray*}
  \\
  P \models E \iff P \in \meaningof{E}
\end{eqnarray*}

\begin{eqnarray*}
  P \approx_{L} Q \iff \forall E \in L. P \models E \iff Q \models E
\end{eqnarray*}

\begin{eqnarray*}
  P \approx_{K} Q
\end{eqnarray*}

\begin{eqnarray*}
  P \approx Q
\end{eqnarray*}

$\approx_{K} = \approx = \approx_{L}$

\subsubsection{Contextual duality}

Note that contexts extend the quotation operation to a family of
operations from processes to names. Given a context, $M$, we can
define a \emph{nominal context}, $\quotep{M}$ by $\quotep{M}[P] :=
\quotep{M[P]}$. To foreshadow what is to come we observe that these
operations enjoy a duality with processes very much like the duality
between vectors and maps from vectors to scalars.

Further, because the calculus is essentially higher-order, we have a
correspondence between contexts and processes. More specifically,
given a name $x$ and a context $M$ we can construct $M^{*}_{x}$ such
that 

\begin{mathpar}
  M^{*}_{x} | \lift{x}{P} \red M[P]
\end{mathpar}

namely,

\begin{mathpar}
  M^{*}_{x} := x?(u).M[\dropn{u}]
\end{mathpar}

The dependence of $M^{*}_{x}$ on a name makes it an abstraction, 

\begin{mathpar}
  M^{*} := (x)x?(u).M[\dropn{u}]
\end{mathpar}

\subsection{Additional notation}

It will sometimes be convenient to denote the process a name
quotes. We already have the notation $x = \quotep{P}$, but it will be
convenient to introduce an alternate notation, $\procn{x}$, when we
want to emphasize the connection to the use of the name. Note that, by
virtue of name equivalence, $\quotep{\procn{x}} \nameeq x$; so, the
notation is consistent with previous definitions.

Further, because names have structure it is possible to effect
substitutions on the basis of that structure. This means we need to
upgrade our notation for substitutions, which we accomplish by
adapting comprehension notation. Thus,

\begin{mathpar}
  P\{ y / x : x \in S \}
\end{mathpar}

is interpreted to mean the process derived from P by replacing (in a
capture-avoiding manner) each occurrence of $x$ in $S$ by $y$. For example,

\begin{mathpar}
  P\{ \quotep{\procn{x}|\procn{x}} / x : x \in \freenames{P} \}
\end{mathpar}

will replace each (occurrence) of a free name $x$ in $P$ by
$\quotep{\procn{x}|\procn{x}}$.

Also, we will avail ourselves of the notation $x^{L}$ and $x^{R}$ to
denote injections of a name into disjoint copies of the name
space. There are numerous ways to accomplish this. One example can be
found in \cite{MeredithR05}. This notation overloads to vectors of
names: $\vec{x}^{\pi} := (x_{i}^{\pi} \; : \; 0 \leq i < |\vec{x}| )$ where $\pi \in \{L,R\}$.

We also use $P^{\Box} := P|\Box$.

In \cite{MeredithR05} an interpretation of the new operator is
given. It turns out that there are several possible interpretations
all enjoying the requisite algebraic properties of the operator (see
\cite{milner91polyadicpi}). We will therefore make liberal use of
$(\nu\; \vec{x})P$.

% subsection the_syntax_and_semantics_of_the_notation_system (end)   

\input{qm2pi.qmops} 

\input{qm2pi.sterngerlach} 

\input{qm2pi.metric} 

% section concurrent_process_calculi (end)

%\input{qm2pi.proofsketch}

% section proof sketch (end)

%\input{qm2pi.slviaknots} 

% section spatial logic via knots (end)

\input{qm2pi.conclusion}

% section conclusion (end)

%\input{qm2pi.dtcodes} 

% section wiring algorithm (end)

\input{qm2pi.ack} 

% section acknowledgments (end)

\newpage


\bibliographystyle{plain}   
\bibliography{../../biblios/main.bib}

\input{qm2pi.rhodetails}

\end{document}

 

% section concurrent_process_calculi (end)

%\documentclass[12pt]{llncs}
%\documentclass{jktr}

\usepackage[pdftex]{hyperref}                   
\usepackage {listings}
\usepackage {mathpartir}
\usepackage{bcprules}
%\usepackage{listings}
                       
\usepackage{graphicx} 
%\usepackage[margins=2.5cm,nohead,nofoot]{geometry}
%\usepackage{geometry}
\usepackage{amsfonts}
\usepackage{amstext}
\usepackage{latexsym}
\usepackage{amssymb}
\usepackage{color}


%\include{myPreamble}
\include{qm2pi.local} 

%\ifpdf
%\usepackage[pdftex]{graphicx}
%\else
%\usepackage{graphicx}
%\fi

 % \ifpdf
%  \usepackage{pdfsync}
%  \if


%\title{Brief Article}
%\author{David F. Snyder}
%\author{L.G. Meredith}

%\address{Dept. of Math., Texas State University--San Marcos, San Marcos, TX 78666}
       
\pagestyle{empty}


\begin{document}

\lstset{language=[Objective]Caml,frame=shadowbox}

\input{qm2pi.front}

% section front matter (end)

\input{qm2pi.intro} 
 
% section introduction (end)

% \input{qm2pi.knotations} 

% section notation (end)

\input{qm2pi.process.calculi} 

% section concurrent_process_calculi_and_spatial_logics_ (end)
    
%\input{qm2pi.knots2pi} 

%\input{qm2pi.trefoil} 

%\input{qm2pi.mainthm} 

% subsection basic_interpretation (end)

%\input{qm2pi.rho.presentation} 
\subsection{The syntax and semantics of the notation system}\label{sub:the_syntax_and_semantics_of_the_notation_system} % (fold)

We now summarize a technical presentation of the calculus that
embodies our theory of dynamics. The typical presentation of such a
calculus follows the style of giving generators and relations on
them. The grammar, below, describing term constructors, freely
generates the set of processes, $\Proc$. This set is then quotiented
by a relation known as structural congruence and it is over this set
that the notion of dynamics is expressed. This presentation is
essentially that of \cite{MeredithR05} with the addition of
polyadicity and summation. For readability we have relegated some of
the technical subtleties to an appendix.

\subsubsection{Process grammar}\label{subsub:process_grammar}

\begin{mathpar}
  \inferrule* [lab=synchronization] {} {{M} \bc \pzero \;|\; x?F \;|\; x!C }
  \and
  \inferrule* [lab=abstraction] {} {{F} \bc (x)P}
  \and
  \inferrule* [lab=concretion] {} {{C} \bc \langle Q \rangle}
  \and
  \inferrule* [lab=process] {} {{P,Q} \bc M \;| \;P|Q \;|\; @{x}}
  \and
  \inferrule* [lab=name] {} {{x} \bc \quotep{P}}
\end{mathpar} 

Note that $\vec{x}$ (resp. $\vec{P}$) denotes a vector of names
(resp. processes) of length $|\vec{x}|$ (resp. $|\vec{P}|$). We adopt
the following useful abbreviations.

\begin{mathpar}
   x?(\vec{y}).P := x.(\vec{y})P \and  x\clift{\vec{P}} := x.\clift{\vec{P}}
   \and x!(y) := \lift{x}{\dropn{y}}
   \and \Pi_{i=0}^{n-1}P_i := P_0 | \ldots | P_{n-1}
\end{mathpar}

\subsubsection{Structural congruence}

\paragraph{Free and bound names and alpha-equivalence.} At the
core of structural equivalence is alpha-equivalence which identifies
process that are the same up to a change of variable. Formally, we
recognize the distinction between free and bound names. The free names
of a process, $\freenames{P}$, may be calculated recursively as
follows:

\begin{mathpar}
\freenames{\pzero} := \emptyset
  \and \\
  \freenames{x?(y).P} := \{ x \} \cup (\freenames{P} \setminus \{ y \})
  \and 
  \freenames{x!\langle P \rangle} := \{ x \} \cup \{ P \} 
  \and \\
  \freenames{P|Q} := \freenames{P} \cup \freenames{Q}
  \and \\
  \freenames{@{x}} := \{ x \}
\end{mathpar}

$\pi$
$\quotep{\pi}$

$\freenames{-} : \pi \to \mathcal{P}(\quotep{\pi})$

\begin{eqnarray*}
  \freenames{\pzero} & := & \emptyset \\
  \freenames{x?(y).P} & := & \{ x \} \cup (\freenames{P} \setminus \{ y \}) \\
  \freenames{x!\langle P \rangle} & := & \{ x \} \cup \{ P \} \\
  \freenames{P|Q} & := & \freenames{P} \cup \freenames{Q} \\
  \freenames{\dropn{x}} & := & \{ x \}
\end{eqnarray*}

The bound names of a process, $\boundnames{P}$, are those names occurring in $P$
that are not free. For example, in $x?(y).0$, the name $x$ is free, while $y$ is bound.

\begin{mathpar}
  \inferrule* [lab=monoidal-laws] {} { P|Q \equiv Q|P \and P|0 \equiv P \and P|(Q|R) \equiv (P|Q)|R }
\end{mathpar}

\begin{mathpar}
  \inferrule* [lab=alpha-equivalence] {} { (x)P \equiv (y)P\{y/x\} \and y \not\in \freenames{P} }
\end{mathpar}

\begin{definition}
Then two processes, $P,Q$, are alpha-equivalent if $P = Q\{\vec{y}/\vec{x}\}$ for
some $\vec{x} \in \boundnames{Q},\vec{y} \in \boundnames{P}$, where $Q\{\vec{y}/\vec{x}\}$
denotes the capture-avoiding substitution of $\vec{y}$ for $\vec{x}$ in $Q$.
\end{definition}

\begin{definition}
  The {\em structural congruence} \cite{SangiorgiWalker} , $\equiv$,
  between processes is the least congruence containing
  alpha-equivalence, satisfying the abelian monoid laws
  (associativity, commutativity and $\pzero$ as identity) for parallel
  composition $|$ and for summation $+$.
\end{definition}

\subsection{Name equivalence}

We take name equivalence, written $\nameeq$, to be the smallest
equivalence relation generated by the following rules.

\begin{mathpar}
\inferrule*[lab=Quote-drop]
{ }
{ \quotep{@{x}} \nameeq x }

\inferrule*[lab=Struct-equiv]
{ P \scong Q }
{ \quotep{P} \nameeq \quotep{Q} }
\end{mathpar}

The astute reader will have noticed that the mutual recursion of names
and processes imposes a mutual recursion on alpha-equivalence and
structural equivalence via name-equivalence. Fortunately, all of this
works out pleasantly and we may calculate in the natural way, free of
concern. The reader interested in the details is referred to the
appendix \ref{appendix:rho_details}.

\subsection{Substitution}

We use $\Proc$ for the set of processes, $\QProc$ for the set of
names, and $\id{\{}\vec{y} / \vec{x} \id{\}}$ to denote partial maps,
$s : \QProc \rightarrow \QProc$. A map, $s$ lifts, uniquely, to a map
on process terms, $\widehat{s} : \Proc \rightarrow \Proc$ by the
following equations.

\begin{mathpar}
  (0) \psubstp{Q}{P} := 0 \\
  (R \juxtap S) \psubstp{Q}{P}
  :=    
  (R)\psubstp{Q}{P} \juxtap (S) \psubstp{Q}{P} \\
  (x?(y).R) \psubstp{Q}{P}    
  :=    
  (x)\substp{Q}{P} (z)\concat( (R \psubstn{z}{y}) \psubstp{Q}{P} ) \\
  (\lift{x}{R}) \psubstp{Q}{P}  
  :=
  \lift{(x)\substp{Q}{P}}{ R \psubstp{Q}{P} } \\
%   (\dropn{x})  \psubstp{Q}{P}       
%   := 
%   \left\{ 
%     \begin{array}{ccc} 
%       \dropn{\quotep{Q}} & & x \nameeq \quotep{P} \\
%       \dropn{x} & & otherwise \\
%     \end{array}
%   \right. 
  (\dropn{x})  \psubstp{Q}{P}       
  := 
  \left\{ 
    \begin{array}{ccc} 
      Q & & x \nameeq \quotep{P} \\
      \dropn{x} & & otherwise \\
    \end{array}
  \right.
\end{mathpar}
 

where

\begin{eqnarray}
  (x)\id{\{} \lpquote Q \rpquote / \lpquote P \rpquote \id{\}}            = 
  \left\{ 
    \begin{array}{ccc}
      \lpquote Q \rpquote & & x \nameeq \lpquote P \rpquote \\
      x & & otherwise \\
    \end{array}
  \right. \nonumber
\end{eqnarray}

and $z$ is chosen distinct from $\quotep{P}$, $\quotep{Q}$, the free
names in $Q$, and all the names in $R$. Our $\alpha$-equivalence will
be built in the standard way from this substitution.

\begin{remark}\label{rem:no_self_referential_names}
  One consequence of these definitions is that $\forall P. \quotep{P}
  \not\in \freenames{P}$.
\end{remark}

\subsection{ Dynamic quote: an example }

Anticipating something of what's to come, consider applying the
substitution, $\widehat{\id{\{}u / z \id{\}}}$, to the following pair
of processes, $\lift{w}{y!(z)}$ and $w[ \lpquote y!(z) \rpquote ]$.

\begin{eqnarray}
	\lift{w}{y!(z)}\widehat{\id{\{}u / z \id{\}}}
		& = &
		\lift{w}{y!(u)} \nonumber\\
	w[ \lpquote y!(z) \rpquote ] \widehat{ \id{\{}u / z \id{\}} }
		& = &
		w[ \lpquote y!(z) \rpquote ] \nonumber
\end{eqnarray}

Because the body of the process between quotes is impervious to
substitution, we get radically different answers. In fact, by
examining the first process in an input context,
e.g. $x?(z).\lift{w}{y!(z)}$, we see that the process under the lift
operator may be shaped by prefixed inputs binding a name inside it. In
this sense, the lift operator will be seen as a way to dynamically
construct processes before reifying them as names.

Finally equipped with these standard features we can present the
dynamics of the calculus.

\subsubsection{Operational semantics} 

Finally, we introduce the computational dynamics. What marks these
algebras as distinct from other more traditionally studied algebraic
structures, e.g. vector spaces or polynomial rings, is the manner in
which dynamics is captured. In traditional structures, dynamics is typically
expressed through morphisms between such structures, as in linear maps
between vector spaces or morphisms between rings. In algebras
associated with the semantics of computation, the dynamics is
expressed as part of the algebraic structure itself, through a
reduction reduction relation typically denoted by $\red$. Below, we
give a recursive presentation of this relation for the calculus used
in the encoding.

$\red \subseteq \pi \times \pi$
$\red : \pi \to \mathcal{P}(\pi)$

\begin{mathpar}
  \inferrule* [lab=Comm] { \textsf{match}( x_{src}, x_{trgt} ) } { x_{trgt}?(y)P \; | \; x_{src}!\langle {Q} \rangle \red P\{\quotep{Q}/y}\} }
  \and \\
  \inferrule* [lab=Par] {{P} \red {P}'} {{{P} | {Q}} \red {{P}' | {Q}}}
  \and
  \inferrule* [lab=Equiv]{{{P} \scong {P}'} \andalso {{P}' \red {Q}'} \andalso {{Q}' \scong {Q}}}{{P} \red {Q}}
\end{mathpar}

\begin{eqnarray*}
  match_{\equiv} (\quotep{P},\quotep{Q}) & := & P \equiv Q \\
  match_{\dagger}(\quotep{P},\quotep{Q}) & := & \forall R. P|Q \red^{*} R => R \red^{*} 0 \\
  match_{K}(\quotep{P},\quotep{Q}) & := & K \mbox{ for some context } K
\end{eqnarray*}

$u?(x)P | u!\langle Q \rangle \red P\{\quotep{Q}/x\}$

%We write $\wred$ for $\red^*$, and $P\red$ if $\exists Q $ such that $ P \red Q$.
We write $P\red$ if $\exists Q $ such that $ P \red Q$ and $P\not\red$, otherwise.

\section{Replication}

As mentioned before, it is known that replication (and hence
recursion) can be implemented in a higher-order process algebra
\cite{SangiorgiWalker}. As our first example of calculation with the
machinery thus far presented we give the construction explicitly in
the {\rhoc}.

\begin{eqnarray}
	D_{x} & := & \prefix{x}{y}{(\binpar{\outputp{x}{y}}{@{y}})} \nonumber\\
	\bangp_{x}{P} & := & \binpar{{x}!\langle{\binpar{D_{x}}{P}}\rangle}{D_{x}} \nonumber
\end{eqnarray}

\begin{eqnarray}
	\bangp_{x}{P} & & \nonumber\\
	=
	& {x}!\langle{(\prefix{x}{y}{(\outputp{x}{y} | @{y})) | P}}\rangle 
	      | \prefix{x}{y}{(\outputp{x}{y} | @{y})} & \nonumber\\
	\red
	& (\outputp{x}{y} | @{y})\substn{\quotep{(\prefix{x}{y}{(@{y} | \outputp{x}{y})) | P}}}{y} & \nonumber\\
	=
	& \outputp{x}{\quotep{(\prefix{x}{y}{(\outputp{x}{y} | @{y})) | P}}}
	  | {(\prefix{x}{y}{(\outputp{x}{y} | @{y})) | P}} & \nonumber\\
	\red
	& \ldots & \nonumber\\
	\red^*
	& P | P | \ldots & \nonumber
\end{eqnarray}

Of course, this encoding, as an implementation, runs away, unfolding
$\bangp{P}$ eagerly. A lazier and more implementable replication
operator, restricted to input-guarded processes, may be obtained as follows.

\begin{eqnarray}
\bangp{\prefix{u}{v}{P}} 
	:= 
	\binpar{\lift{x}{\prefix{u}{v}{(\binpar{D(x)}{P})}}}{D(x)} \nonumber
\end{eqnarray}

\begin{remark}
  Note that the lazier definition still does not deal with summation
  or mixed summation (i.e. sums over input and output). The reader is
  invited to construct definitions of replication that deal with these
  features. 

  Further, the definitions are parameterized in a name, $x$. Can you,
  gentle reader, make a definition that eliminates this parameter and
  guarantees no accidental interaction between the replication
  machinery and the process being replicated -- i.e. no accidental
  sharing of names used by the process to get its work done and the
  name(s) used by the replication to effect copying. This latter
  revision of the definition of replication is crucial to obtaining
  the expected identity $!!P \sim !P$.
\end{remark}

\begin{remark}\label{rem:paradoxical_combinator}
  The reader familiar with the lambda calculus will have noticed the
  similarity between $D$ and the paradoxical combinator.

  [Ed. note: the existence of this seems to suggest we have to be more
  restrictive on the set of processes and names we admit if we are to
  support no-cloning.]
\end{remark}

\subsubsection{Bisimulation}

The computational dynamics gives rise to another kind of equivalence,
the equivalence of computational behavior. As previously mentioned
this is typically captured \emph{via} some form of bisimulation.

% The notion we use in this paper is weak barbed bisimulation
% \cite{milner91polyadicpi}.

The notion we use in this paper is derived from weak barbed
bisimulation \cite{milner91polyadicpi}. 

\begin{definition}
An \emph{observation relation}, $\downarrow_{\mathcal N}$, over a set
of names, $\mathcal N$, is the smallest relation satisfying the rules
below.

\infrule[Out-barb]{y \in {\mathcal N}, \; x \nameeq y}
		  {\outputp{x}{v} \downarrow_{\mathcal N} x}
\infrule[Par-barb]{\mbox{$P\downarrow_{\mathcal N} x$ or $Q\downarrow_{\mathcal N} x$}}
		  {\binpar{P}{Q} \downarrow_{\mathcal N} x}

We write $P \Downarrow_{\mathcal N} x$ if there is $Q$ such that 
$P \wred Q$ and $Q \downarrow_{\mathcal N} x$.
\end{definition}

\begin{definition}
%\label{def.bbisim}
An  ${\mathcal N}$-\emph{barbed bisimulation} over a set of names, ${\mathcal N}$, is a symmetric binary relation 
${\mathcal S}_{\mathcal N}$ between agents such that $P\rel{S}_{\mathcal N}Q$ implies:
\begin{enumerate}
\item If $P \red P'$ then $Q \wred Q'$ and $P'\rel{S}_{\mathcal N} Q'$.
\item If $P\downarrow_{\mathcal N} x$, then $Q\Downarrow_{\mathcal N} x$.
\end{enumerate}
$P$ is ${\mathcal N}$-barbed bisimilar to $Q$, written
$P \wbbisim_{\mathcal N} Q$, if $P \rel{S}_{\mathcal N} Q$ for some ${\mathcal N}$-barbed bisimulation ${\mathcal S}_{\mathcal N}$.
\end{definition}

$\mathcal{R} \subseteq \pi \times \pi$

$P \mathcal{R} Q => \forall P'. P \red P' \Rightarrow \exists Q'. Q \red Q', P' \mathcal{R} Q'$

$P \vdash x \Rightarrow Q \vdash x$

\begin{mathpar}
  \inferrule*[lab=Out-barb]{x \nameeq y}{{y}!\langle{Q}\rangle \vdash x}
  \and
  \inferrule*[lab=Par-barb]{\mbox{$P\vdash x$ or $Q\vdash x$}}{\binpar{P}{Q} \vdash x}
\end{mathpar}

\subsubsection{Contexts}

One of the principle advantages of computational calculi like the
$\pi$-calculus is a well-defined notion of context,
contextual-equivalence and a correlation between
contextual-equivalence and notions of bisimulation. The notion of
context allows the decomposition of a process into (sub-)process and
its syntactic environment, its context. Thus, a context may be
thought of as a process with a ``hole'' (written $\Box$) in it. The
application of a context $M$ to a process $P$, written $M[P]$, is
tantamount to filling the hole in $M$ with $P$. In this paper we do
not need the full weight of this theory, but do make use of the notion
of context in the proof the main theorem. 

\begin{mathpar}
  \inferrule* [lab=summation] {} {{M_{M},M_{N}} \bc \Box \;|\; x.M_{A} \;|\; M_{M}+M_{N}}
  \and
  \inferrule* [lab=agent] {} {{M_{A}} \bc (\vec{x})M_{P} \;| \; \clift{P_0,\ldots,M_{P},\ldots,P_N}}
  \and \\
  \inferrule* [lab=process] {} {{M_{P}} \bc M_{N} \;| \;P|M_{P} }
\end{mathpar} 

\begin{mathpar}
  \inferrule* [lab=sychronization] {} {M_{N} \bc \Box \;|\; x?M_{F} \;|\; x!M_{C}}
  \and
  \inferrule* [lab=abstraction] {} {{M_{F}} \bc (x)M_{P} }
  \and
  \inferrule* [lab=concretion] {} {{M_{C}} \bc \langle M_{P} \rangle }
  \and \\
  \inferrule* [lab=process] {} {{M_{P}} \bc M_{N} \;| \;P|M_{P} }
\end{mathpar}

\begin{definition}[contextual application] Given a context $M$, and
  process $P$, we define the \emph{contextual application}, $M[P] :=
  M\{P/\Box\}$. That is, the contextual application of M to P is the
  substitution of $P$ for $\Box$ in $M$.
\end{definition}

$\meaningof{-} : L \to \mathcal{P}(\pi)$

\begin{mathpar}
  \inferrule* [lab=collection] {} {\meaningof{true} = \pi, \and \meaningof{~E} = \pi \setminus \meaningof{E}, \and \meaningof{E_{1} \& E_{2}} = \meaningof{E_{1}} \cap \meaningof{E_{2}}}
\end{mathpar}

\begin{mathpar}
  \inferrule* [lab=structure] {} {\meaningof{0} = \{ P \in \pi | P \equiv 0 \}, \and \\ \meaningof{E_1 | E_2} = \{ P \in \pi | P \equiv P_{1} | P_{2}, P_{1} \in \meaningof{E_{1}}, P_{2} \in \meaningof{E_2}\} }
\end{mathpar}

\begin{mathpar}
 \inferrule* [lab=behavior] {} {\meaningof{\langle a?b \rangle E} = \{ P \in \pi | P \equiv Q | u?(y)P', \\ \and \\\\ \and \\ \;\;\; u \in \meaningof{a}, \forall z.P'\{z/y\} \in \meaningof{E\{z/b\}}\}, \and \\ \meaningof{a!E} = \{ P \in \pi | P \equiv Q | x!\langle P' \rangle, x \in \meaningof{a} P' \in \meaningof{E}\} }
\end{mathpar}

\begin{mathpar}
 \inferrule* [lab=nominal] {} {\meaningof{\quotep{E}} = \{ \quotep{P} \in \quotep{\pi} | P \in \meaningof{E} \}, \and \meaningof{\quotep{P}} = \{ \quotep{Q} \in \quotep{\pi} | P \equiv Q \} \and \\ \meaningof{@\quotep{E}} = \{ P \in \pi | P \equiv @x, x \in \meaningof{E} \}}
\end{mathpar}

\begin{eqnarray*}
  \\
  \meaningof{-} : TS \to ST
\end{eqnarray*}

\begin{eqnarray*}
  \\
  L : TS \to ST
\end{eqnarray*}

\begin{eqnarray*}
  \\
  P \models E \iff P \in \meaningof{E}
\end{eqnarray*}

\begin{eqnarray*}
  P \approx_{L} Q \iff \forall E \in L. P \models E \iff Q \models E
\end{eqnarray*}

\begin{eqnarray*}
  P \approx_{K} Q
\end{eqnarray*}

\begin{eqnarray*}
  P \approx Q
\end{eqnarray*}

$\approx_{K} = \approx = \approx_{L}$

\subsubsection{Contextual duality}

Note that contexts extend the quotation operation to a family of
operations from processes to names. Given a context, $M$, we can
define a \emph{nominal context}, $\quotep{M}$ by $\quotep{M}[P] :=
\quotep{M[P]}$. To foreshadow what is to come we observe that these
operations enjoy a duality with processes very much like the duality
between vectors and maps from vectors to scalars.

Further, because the calculus is essentially higher-order, we have a
correspondence between contexts and processes. More specifically,
given a name $x$ and a context $M$ we can construct $M^{*}_{x}$ such
that 

\begin{mathpar}
  M^{*}_{x} | \lift{x}{P} \red M[P]
\end{mathpar}

namely,

\begin{mathpar}
  M^{*}_{x} := x?(u).M[\dropn{u}]
\end{mathpar}

The dependence of $M^{*}_{x}$ on a name makes it an abstraction, 

\begin{mathpar}
  M^{*} := (x)x?(u).M[\dropn{u}]
\end{mathpar}

\subsection{Additional notation}

It will sometimes be convenient to denote the process a name
quotes. We already have the notation $x = \quotep{P}$, but it will be
convenient to introduce an alternate notation, $\procn{x}$, when we
want to emphasize the connection to the use of the name. Note that, by
virtue of name equivalence, $\quotep{\procn{x}} \nameeq x$; so, the
notation is consistent with previous definitions.

Further, because names have structure it is possible to effect
substitutions on the basis of that structure. This means we need to
upgrade our notation for substitutions, which we accomplish by
adapting comprehension notation. Thus,

\begin{mathpar}
  P\{ y / x : x \in S \}
\end{mathpar}

is interpreted to mean the process derived from P by replacing (in a
capture-avoiding manner) each occurrence of $x$ in $S$ by $y$. For example,

\begin{mathpar}
  P\{ \quotep{\procn{x}|\procn{x}} / x : x \in \freenames{P} \}
\end{mathpar}

will replace each (occurrence) of a free name $x$ in $P$ by
$\quotep{\procn{x}|\procn{x}}$.

Also, we will avail ourselves of the notation $x^{L}$ and $x^{R}$ to
denote injections of a name into disjoint copies of the name
space. There are numerous ways to accomplish this. One example can be
found in \cite{MeredithR05}. This notation overloads to vectors of
names: $\vec{x}^{\pi} := (x_{i}^{\pi} \; : \; 0 \leq i < |\vec{x}| )$ where $\pi \in \{L,R\}$.

We also use $P^{\Box} := P|\Box$.

In \cite{MeredithR05} an interpretation of the new operator is
given. It turns out that there are several possible interpretations
all enjoying the requisite algebraic properties of the operator (see
\cite{milner91polyadicpi}). We will therefore make liberal use of
$(\nu\; \vec{x})P$.

% subsection the_syntax_and_semantics_of_the_notation_system (end)   

\input{qm2pi.qmops} 

\input{qm2pi.sterngerlach} 

\input{qm2pi.metric} 

% section concurrent_process_calculi (end)

%\input{qm2pi.proofsketch}

% section proof sketch (end)

%\input{qm2pi.slviaknots} 

% section spatial logic via knots (end)

\input{qm2pi.conclusion}

% section conclusion (end)

%\input{qm2pi.dtcodes} 

% section wiring algorithm (end)

\input{qm2pi.ack} 

% section acknowledgments (end)

\newpage


\bibliographystyle{plain}   
\bibliography{../../biblios/main.bib}

\input{qm2pi.rhodetails}

\end{document}



% section proof sketch (end)

%\section{Unlikely characters: spatial logic for
  knots}\label{sub:characteristic_formulae} % (fold)

Associated to the mobile process calculi are a family of logics known
as the Hennessy-Milner logics. These logics typically enjoy a
semantics interpreting formulae as sets of processes that when
factored through the encoding outlined above allows an identification
of classes of knots with logical formulae. In the context of this
encoding the sub-family known as the spatial logics \cite{CairesC03}
\cite{CairesC04} \cite{Caires04} are of particular interest providing
several important features for expressing and reasoning about
properties (i.e. classes) of knots. We hint here at how this may be done.

%\begin{description}
%\item [structural connectives] 
\subsubsection{Structural connectives} The spatial logics enjoy
structural connectives corresponding, at the logical level, to the
parallel composition ($P | Q$) and new name ($(\nu \; x)P$)
connectives for processes. As illustrated in the examples below, these
connectives are extremely expressive given the shape of our encoding.
%\item [decideable satisfaction]

\subsubsection{Decideable satisfaction}
In \cite{Caires04} the satisfaction relation is shown to be decideable
for a rich class of processes. It further turns out that the image of
the our encoding is a proper subset of that class. This result
provides the basis for an algorithm by which to search for knots
enjoying a given property.
%\item [characteristic formulae]

\subsubsection{Characteristic formulae}
In the same paper \cite{Caires04} , Caires presents a means of calculating
characteristic formulae, selecting equivalence classes of processes
up to a pre--specified depth limit on the support set of names. Composed with our
encoding, this characteristic formula can be used to select
characteristic formulae for knots.
%\end{description}

\subsubsection{Spatial logic formulae}

The grammar below (segmented for comprehension) summarizes the syntax
of spatial logic formulae. We employ illustrative examples in the
sequel to provide an intuitive understanding of their meaning
referring the reader to \cite{Caires04} for a more detailed explication
of the semantics.

\begin{mathpar}
  \inferrule* [lab=boolean] {} {{A,B} \bc T \;|\; \neg A \;|\; A \wedge B \;|\; \eta = \eta'}
  \and
  \inferrule* [lab=spatial] {} {|\; \pzero \;|\; A | B \;|\; x \text{\textregistered} A \;|\; \forall x . A \;|\;  H x . A}
  \and
  \inferrule* [lab=behavioral] {} {|\; \alpha . A}
  \and 
  \inferrule* [lab=recursion] {} {|\; X(\vec{u}) \;|\; \mu X(\vec{u}) . A}
  \and
  \inferrule* [lab=action] {} {\alpha \bc \langle x?(\vec{y}) \rangle \;|\; \langle x!(\vec{y}) \rangle \;|\; \langle \tau \rangle}
  \and 
  \inferrule* [lab=name] {} {\eta \bc x \;|\; \tau}
\end{mathpar} 

% subsection characteristic_formulae (end)   	 

\subsection{Example formulae}\label{sub:example_formulae_} % (fold)

\subsubsection{Crossing as formula.}
% 
% \begin{align*}
%   \frac{d}{dx} \sin x &= \cos x 
%   & \frac{d}{dx} e^x &= e^x \\
%   \frac{d}{dx} \cos x &= - \sin x 
%   & \frac{d}{dx} \log x &= \frac{1}{x} \\
% \end{align*} 

\begin{align*}
 \mu C(x_{0},x_{1},y_{0},y_{1},u).&(\langle x_{0}?(z) \rangle(\langle u! \rangle\langle y_{1}!z \rangle C(x_{0},x_{1},y_{0},y_{1},u)) & \\
  & \wedge \langle y_{1}?(z) \rangle (\langle u! \rangle \langle x_{0}!z \rangle C(x_{0},x_{1},y_{0},y_{1},u)) & \\
  & \wedge \langle x_{1}?(z) \rangle (\langle u? \rangle \langle y_{0}!z \rangle C(x_{0},x_{1},y_{0},y_{1},u)) & \\
  & \wedge \langle y_{0}?(z) \rangle (\langle u? \rangle \langle x_{1}!z \rangle C(x_{0},x_{1},y_{0},y_{1},u))) &
\end{align*}

The lexicographical similarity between the shape of this formulae and
the shape of definition of the process representing a crossing reveals
the intuitive meaning of this formulae. It describes the capabilities
of a process that has the right to represent a crossing. For example
it picks out processes that may perform an input on the port $x_0$ in
its initial menu of capabilities. What differentiates the formula
from the process, however, is that the crossing process is the
smallest candidate to satisfy the formula. Infinitely many other
processes -- with internal behavior hidden behind this interface, so
to speak -- also satisfy this formula. Even this simple formula,
then, can be seen to open a new view onto knots, providing a
computational interpretation of \emph{virtual} knots.

Note that this formula is derived by hand. A similar formula can be
derived by employing Caires' calculation of characteristic formula
\cite{Caires04} to the process representing a crossing. In light of
this discussion, we let
$\meaningof{C}_{\phi}(x0,x1,y0,y1,u)$ denote a formula specifying the
dynamics we wish to capture of a crossing. To guarantee we preserve
the shape of the interface and minimal semantics we demand that
$\meaningof{C}_{\phi}(x0,x1,y0,y1,u) \Rightarrow
\textbf{C}(x0,x1,y0,y1,u)$ where $\textbf{C}(x0,x1,y0,y1,u)$ denotes
the formula above.
                            
\subsubsection{Crossing number constraints.}
The moral content of the context lemma (Lemma \ref{context}) is that the notion of
``locality'' in the Reidemeister moves is effectively captured by the
parallel composition operator of the process calculus. This intuition
extends through the logic. Given a formula,
$\meaningof{C}_{\phi}(x0,x1,y0,y1,u)$, we can use the structural
connectives to specify constraints on crossing numbers, such as at
least $n$ crossings, or exactly $n$ crossings.
\begin{mathpar}
  \inferrule* [lab=at-least-n] {} { K^{\geq n}_{\phi}(\vec{xs},\vec{ys}) := \Pi_{i=0}^{n-1} Hu . \meaningof{C}_{\phi}(xs_i,ys_i,u) | T }
  \and 
  \inferrule* [lab=exactly-n] {} { K^{= n}_{\phi}(\vec{xs},\vec{ys}) := \Pi_{i=0}^{n-1} Hu . \meaningof{C}_{\phi}(xs_i,ys_i,u) | \neg (\forall x_0,y_0,x_1,y_1,u . \meaningof{C}_{\phi}(x_0,y_0,x_1,y_1,u) | T) }
\end{mathpar}

To round out this section, recall that the encoding of an $n$-crossing
knot decomposes into a parallel composition of $n$ \emph{copies} of a
crossing process together with a wiring harness. To specify different
knot classes with the same crossing number amounts to specifying
logical constraints on the wiring harness. In the interest of space,
we defer examples to a forthcoming paper. Suffice it to say that both
the conditions ``alternating knot'' and ``contains the tangle
corresponding to 5/3'' are expressible. For example, it is possible to
calculate the characteristic formula of a process corresponding to the
tangle 5/3 and conjoin it into the classifying formula via the
composition connective of the logic.

Finally, we wish to observe that it is entirely within reason to
contemplate a more domain-specific version of spatial logic tailored
to the shape of processes in the image of the encoding. Such a
domain-specific logic would have a better claim to the title formal
language of knot properties.

% subsection example_formulae_ (end)

% section knots_as_processes (end) 

% section spatial logic via knots (end)

\section{Conclusions and future work}

\paragraph{Testing physical space}
You, gentle reader, may wonder why of all the theorems to be proved
given this set up we pick the one above. In some sense it's hardly
central to quantum mechanics. We see it as central in the sense that
it firmly establishes a notion of physical space arising from a notion
of the equivalence of behavior. Relating bisimulation to a metric is a
big step forward, but one is faced with interpreting the relationship
of that metric space to something more physical. Quantum mechanical
notions of ``physical'' space are still far from intuitive, but by
relating this idea of distance as testing to calculations that predict
physical circumstances we are making a not insignificant step forward
toward an understanding of the physical space we inhabit as
essentially dynamic.

\paragraph{Effectivity and simulation}
One of the observations we have yet to make is that the entire program
spelled out here is effective. We have built various interpreters for
the reflective calculus at work in this interpretation. In principle,
then, we can simulate quantum mechanics on a computer. The place where
the simulation may lose fidelity is the infinitely branching summation
for the annihilator.

In this connection i also want to point out that the evaluation style
calculation of the inner product puts the non-determinism of the
summation right at the heart of measurement. This suggests that
Milner's original reduction-based formulation of the dynamics of his
calculi in terms of sums was not just notationally suggestive of a
notion of measure-and-continue but captured some significant part of
the physics.

\paragraph{Quantum continuations}
In light of this last observation i want to point out that the
predominant account of quantum mechanics is missing a key aspect of a
truly compositional story of the physical situation. In a real lab,
when a measurement is made the observation can be made to feed into
another device that then makes another measurement conditioned on the
results of the first. This means that after the superposition was
collapsed the entire experimental set up remained in
superposition. While QM offers a means of writing this down it doesn't
quite line up well with the well-trodden formulation of computation
and continuation that we see so succinctly expressed in Milner's
calculi. This suggests that there might be advantages to this account
of dynamics waiting to be explored.

\paragraph{Quantum logic}
In this connection, we also note that by virtue of having the
Hennessy-Milner construction, we can pull the construction through the
interpretation of QM. This gives us a natural candidate for a quantum
logic that enjoys an extremely tight connection with it's domain of
interpretation, making the construction much less ad hoc (rather it is
the image of functor!).

\paragraph{Quantum probabiity}
i have questions about the basis of the interpretation of inner
product as probability amplitude. In particular, using which
axiomatization of probability theory does the notion of probability
amplitude earn the right to be so dubbed? In other words, where is the
proof that the operation for calculating a probability amplitude (and
then squaring) satisfies the axioms of what it means to calculate a
probability? Even if such a proof exists (i have yet to find it in the
literature), i wonder if it might not be possible to turn things on
their heads. Can we view the calculation of the probability amplitude
as an axiomatization of probability? If so, then the definition we
give for calculating probability amplitude may provide the basis for
an \emph{effective} theory of probability.

\paragraph{Quantum vs ``biological'' information}
Finally, i want to conclude with a more philosophical observation. At
a recent workshop in which QM was a predominant topic i noticed
something about quantum information. The speaker was giving a riveting
discussion of axiomatic QM and showing how properties of ``no
cloning'' and ``no deleting'' emerged as consequences of the
axiomatization. Theorems of this form are necessary to give us a sense
of confidence that our axioms characterize the physical theory. What
struck me, though, was that if quantum information is neither erasable
nor replicable it is markedly different from \emph{life}. Two of the
things we know about life is that

\begin{itemize}
  \item it ends;
  \item to gain some measure of persistence, to transcend it's
    finitude it is imminently copyable.
\end{itemize}

Both of these qualities are summarized succinctly in the aphorism: all
flesh is grass. For me these two kinds of ``information'' -- call them
quantum and biological -- are end points on a spectrum of strategies
for persistence. At one end, we have those curious entities that enjoy
uniqueness and permanence; at the other, we have those who in the face
of a certain end and an uncertain present make a go of passing
something on. To me one of the more remarkable aspects of the latter
strategy is that in the presence of noise (and certain features of
copying) we get a kind of dynamism, a chance for improvement against a
given persistent condition.

% subsection other_calculi_other_bisimulations_and_geometry_as_behavior (end)




% section conclusion (end)

%\documentclass[12pt]{llncs}
%\documentclass{jktr}

\usepackage[pdftex]{hyperref}                   
\usepackage {listings}
\usepackage {mathpartir}
\usepackage{bcprules}
%\usepackage{listings}
                       
\usepackage{graphicx} 
%\usepackage[margins=2.5cm,nohead,nofoot]{geometry}
%\usepackage{geometry}
\usepackage{amsfonts}
\usepackage{amstext}
\usepackage{latexsym}
\usepackage{amssymb}
\usepackage{color}


%\include{myPreamble}
\include{qm2pi.local} 

%\ifpdf
%\usepackage[pdftex]{graphicx}
%\else
%\usepackage{graphicx}
%\fi

 % \ifpdf
%  \usepackage{pdfsync}
%  \if


%\title{Brief Article}
%\author{David F. Snyder}
%\author{L.G. Meredith}

%\address{Dept. of Math., Texas State University--San Marcos, San Marcos, TX 78666}
       
\pagestyle{empty}


\begin{document}

\lstset{language=[Objective]Caml,frame=shadowbox}

\input{qm2pi.front}

% section front matter (end)

\input{qm2pi.intro} 
 
% section introduction (end)

% \input{qm2pi.knotations} 

% section notation (end)

\input{qm2pi.process.calculi} 

% section concurrent_process_calculi_and_spatial_logics_ (end)
    
%\input{qm2pi.knots2pi} 

%\input{qm2pi.trefoil} 

%\input{qm2pi.mainthm} 

% subsection basic_interpretation (end)

%\input{qm2pi.rho.presentation} 
\subsection{The syntax and semantics of the notation system}\label{sub:the_syntax_and_semantics_of_the_notation_system} % (fold)

We now summarize a technical presentation of the calculus that
embodies our theory of dynamics. The typical presentation of such a
calculus follows the style of giving generators and relations on
them. The grammar, below, describing term constructors, freely
generates the set of processes, $\Proc$. This set is then quotiented
by a relation known as structural congruence and it is over this set
that the notion of dynamics is expressed. This presentation is
essentially that of \cite{MeredithR05} with the addition of
polyadicity and summation. For readability we have relegated some of
the technical subtleties to an appendix.

\subsubsection{Process grammar}\label{subsub:process_grammar}

\begin{mathpar}
  \inferrule* [lab=synchronization] {} {{M} \bc \pzero \;|\; x?F \;|\; x!C }
  \and
  \inferrule* [lab=abstraction] {} {{F} \bc (x)P}
  \and
  \inferrule* [lab=concretion] {} {{C} \bc \langle Q \rangle}
  \and
  \inferrule* [lab=process] {} {{P,Q} \bc M \;| \;P|Q \;|\; @{x}}
  \and
  \inferrule* [lab=name] {} {{x} \bc \quotep{P}}
\end{mathpar} 

Note that $\vec{x}$ (resp. $\vec{P}$) denotes a vector of names
(resp. processes) of length $|\vec{x}|$ (resp. $|\vec{P}|$). We adopt
the following useful abbreviations.

\begin{mathpar}
   x?(\vec{y}).P := x.(\vec{y})P \and  x\clift{\vec{P}} := x.\clift{\vec{P}}
   \and x!(y) := \lift{x}{\dropn{y}}
   \and \Pi_{i=0}^{n-1}P_i := P_0 | \ldots | P_{n-1}
\end{mathpar}

\subsubsection{Structural congruence}

\paragraph{Free and bound names and alpha-equivalence.} At the
core of structural equivalence is alpha-equivalence which identifies
process that are the same up to a change of variable. Formally, we
recognize the distinction between free and bound names. The free names
of a process, $\freenames{P}$, may be calculated recursively as
follows:

\begin{mathpar}
\freenames{\pzero} := \emptyset
  \and \\
  \freenames{x?(y).P} := \{ x \} \cup (\freenames{P} \setminus \{ y \})
  \and 
  \freenames{x!\langle P \rangle} := \{ x \} \cup \{ P \} 
  \and \\
  \freenames{P|Q} := \freenames{P} \cup \freenames{Q}
  \and \\
  \freenames{@{x}} := \{ x \}
\end{mathpar}

$\pi$
$\quotep{\pi}$

$\freenames{-} : \pi \to \mathcal{P}(\quotep{\pi})$

\begin{eqnarray*}
  \freenames{\pzero} & := & \emptyset \\
  \freenames{x?(y).P} & := & \{ x \} \cup (\freenames{P} \setminus \{ y \}) \\
  \freenames{x!\langle P \rangle} & := & \{ x \} \cup \{ P \} \\
  \freenames{P|Q} & := & \freenames{P} \cup \freenames{Q} \\
  \freenames{\dropn{x}} & := & \{ x \}
\end{eqnarray*}

The bound names of a process, $\boundnames{P}$, are those names occurring in $P$
that are not free. For example, in $x?(y).0$, the name $x$ is free, while $y$ is bound.

\begin{mathpar}
  \inferrule* [lab=monoidal-laws] {} { P|Q \equiv Q|P \and P|0 \equiv P \and P|(Q|R) \equiv (P|Q)|R }
\end{mathpar}

\begin{mathpar}
  \inferrule* [lab=alpha-equivalence] {} { (x)P \equiv (y)P\{y/x\} \and y \not\in \freenames{P} }
\end{mathpar}

\begin{definition}
Then two processes, $P,Q$, are alpha-equivalent if $P = Q\{\vec{y}/\vec{x}\}$ for
some $\vec{x} \in \boundnames{Q},\vec{y} \in \boundnames{P}$, where $Q\{\vec{y}/\vec{x}\}$
denotes the capture-avoiding substitution of $\vec{y}$ for $\vec{x}$ in $Q$.
\end{definition}

\begin{definition}
  The {\em structural congruence} \cite{SangiorgiWalker} , $\equiv$,
  between processes is the least congruence containing
  alpha-equivalence, satisfying the abelian monoid laws
  (associativity, commutativity and $\pzero$ as identity) for parallel
  composition $|$ and for summation $+$.
\end{definition}

\subsection{Name equivalence}

We take name equivalence, written $\nameeq$, to be the smallest
equivalence relation generated by the following rules.

\begin{mathpar}
\inferrule*[lab=Quote-drop]
{ }
{ \quotep{@{x}} \nameeq x }

\inferrule*[lab=Struct-equiv]
{ P \scong Q }
{ \quotep{P} \nameeq \quotep{Q} }
\end{mathpar}

The astute reader will have noticed that the mutual recursion of names
and processes imposes a mutual recursion on alpha-equivalence and
structural equivalence via name-equivalence. Fortunately, all of this
works out pleasantly and we may calculate in the natural way, free of
concern. The reader interested in the details is referred to the
appendix \ref{appendix:rho_details}.

\subsection{Substitution}

We use $\Proc$ for the set of processes, $\QProc$ for the set of
names, and $\id{\{}\vec{y} / \vec{x} \id{\}}$ to denote partial maps,
$s : \QProc \rightarrow \QProc$. A map, $s$ lifts, uniquely, to a map
on process terms, $\widehat{s} : \Proc \rightarrow \Proc$ by the
following equations.

\begin{mathpar}
  (0) \psubstp{Q}{P} := 0 \\
  (R \juxtap S) \psubstp{Q}{P}
  :=    
  (R)\psubstp{Q}{P} \juxtap (S) \psubstp{Q}{P} \\
  (x?(y).R) \psubstp{Q}{P}    
  :=    
  (x)\substp{Q}{P} (z)\concat( (R \psubstn{z}{y}) \psubstp{Q}{P} ) \\
  (\lift{x}{R}) \psubstp{Q}{P}  
  :=
  \lift{(x)\substp{Q}{P}}{ R \psubstp{Q}{P} } \\
%   (\dropn{x})  \psubstp{Q}{P}       
%   := 
%   \left\{ 
%     \begin{array}{ccc} 
%       \dropn{\quotep{Q}} & & x \nameeq \quotep{P} \\
%       \dropn{x} & & otherwise \\
%     \end{array}
%   \right. 
  (\dropn{x})  \psubstp{Q}{P}       
  := 
  \left\{ 
    \begin{array}{ccc} 
      Q & & x \nameeq \quotep{P} \\
      \dropn{x} & & otherwise \\
    \end{array}
  \right.
\end{mathpar}
 

where

\begin{eqnarray}
  (x)\id{\{} \lpquote Q \rpquote / \lpquote P \rpquote \id{\}}            = 
  \left\{ 
    \begin{array}{ccc}
      \lpquote Q \rpquote & & x \nameeq \lpquote P \rpquote \\
      x & & otherwise \\
    \end{array}
  \right. \nonumber
\end{eqnarray}

and $z$ is chosen distinct from $\quotep{P}$, $\quotep{Q}$, the free
names in $Q$, and all the names in $R$. Our $\alpha$-equivalence will
be built in the standard way from this substitution.

\begin{remark}\label{rem:no_self_referential_names}
  One consequence of these definitions is that $\forall P. \quotep{P}
  \not\in \freenames{P}$.
\end{remark}

\subsection{ Dynamic quote: an example }

Anticipating something of what's to come, consider applying the
substitution, $\widehat{\id{\{}u / z \id{\}}}$, to the following pair
of processes, $\lift{w}{y!(z)}$ and $w[ \lpquote y!(z) \rpquote ]$.

\begin{eqnarray}
	\lift{w}{y!(z)}\widehat{\id{\{}u / z \id{\}}}
		& = &
		\lift{w}{y!(u)} \nonumber\\
	w[ \lpquote y!(z) \rpquote ] \widehat{ \id{\{}u / z \id{\}} }
		& = &
		w[ \lpquote y!(z) \rpquote ] \nonumber
\end{eqnarray}

Because the body of the process between quotes is impervious to
substitution, we get radically different answers. In fact, by
examining the first process in an input context,
e.g. $x?(z).\lift{w}{y!(z)}$, we see that the process under the lift
operator may be shaped by prefixed inputs binding a name inside it. In
this sense, the lift operator will be seen as a way to dynamically
construct processes before reifying them as names.

Finally equipped with these standard features we can present the
dynamics of the calculus.

\subsubsection{Operational semantics} 

Finally, we introduce the computational dynamics. What marks these
algebras as distinct from other more traditionally studied algebraic
structures, e.g. vector spaces or polynomial rings, is the manner in
which dynamics is captured. In traditional structures, dynamics is typically
expressed through morphisms between such structures, as in linear maps
between vector spaces or morphisms between rings. In algebras
associated with the semantics of computation, the dynamics is
expressed as part of the algebraic structure itself, through a
reduction reduction relation typically denoted by $\red$. Below, we
give a recursive presentation of this relation for the calculus used
in the encoding.

$\red \subseteq \pi \times \pi$
$\red : \pi \to \mathcal{P}(\pi)$

\begin{mathpar}
  \inferrule* [lab=Comm] { \textsf{match}( x_{src}, x_{trgt} ) } { x_{trgt}?(y)P \; | \; x_{src}!\langle {Q} \rangle \red P\{\quotep{Q}/y}\} }
  \and \\
  \inferrule* [lab=Par] {{P} \red {P}'} {{{P} | {Q}} \red {{P}' | {Q}}}
  \and
  \inferrule* [lab=Equiv]{{{P} \scong {P}'} \andalso {{P}' \red {Q}'} \andalso {{Q}' \scong {Q}}}{{P} \red {Q}}
\end{mathpar}

\begin{eqnarray*}
  match_{\equiv} (\quotep{P},\quotep{Q}) & := & P \equiv Q \\
  match_{\dagger}(\quotep{P},\quotep{Q}) & := & \forall R. P|Q \red^{*} R => R \red^{*} 0 \\
  match_{K}(\quotep{P},\quotep{Q}) & := & K \mbox{ for some context } K
\end{eqnarray*}

$u?(x)P | u!\langle Q \rangle \red P\{\quotep{Q}/x\}$

%We write $\wred$ for $\red^*$, and $P\red$ if $\exists Q $ such that $ P \red Q$.
We write $P\red$ if $\exists Q $ such that $ P \red Q$ and $P\not\red$, otherwise.

\section{Replication}

As mentioned before, it is known that replication (and hence
recursion) can be implemented in a higher-order process algebra
\cite{SangiorgiWalker}. As our first example of calculation with the
machinery thus far presented we give the construction explicitly in
the {\rhoc}.

\begin{eqnarray}
	D_{x} & := & \prefix{x}{y}{(\binpar{\outputp{x}{y}}{@{y}})} \nonumber\\
	\bangp_{x}{P} & := & \binpar{{x}!\langle{\binpar{D_{x}}{P}}\rangle}{D_{x}} \nonumber
\end{eqnarray}

\begin{eqnarray}
	\bangp_{x}{P} & & \nonumber\\
	=
	& {x}!\langle{(\prefix{x}{y}{(\outputp{x}{y} | @{y})) | P}}\rangle 
	      | \prefix{x}{y}{(\outputp{x}{y} | @{y})} & \nonumber\\
	\red
	& (\outputp{x}{y} | @{y})\substn{\quotep{(\prefix{x}{y}{(@{y} | \outputp{x}{y})) | P}}}{y} & \nonumber\\
	=
	& \outputp{x}{\quotep{(\prefix{x}{y}{(\outputp{x}{y} | @{y})) | P}}}
	  | {(\prefix{x}{y}{(\outputp{x}{y} | @{y})) | P}} & \nonumber\\
	\red
	& \ldots & \nonumber\\
	\red^*
	& P | P | \ldots & \nonumber
\end{eqnarray}

Of course, this encoding, as an implementation, runs away, unfolding
$\bangp{P}$ eagerly. A lazier and more implementable replication
operator, restricted to input-guarded processes, may be obtained as follows.

\begin{eqnarray}
\bangp{\prefix{u}{v}{P}} 
	:= 
	\binpar{\lift{x}{\prefix{u}{v}{(\binpar{D(x)}{P})}}}{D(x)} \nonumber
\end{eqnarray}

\begin{remark}
  Note that the lazier definition still does not deal with summation
  or mixed summation (i.e. sums over input and output). The reader is
  invited to construct definitions of replication that deal with these
  features. 

  Further, the definitions are parameterized in a name, $x$. Can you,
  gentle reader, make a definition that eliminates this parameter and
  guarantees no accidental interaction between the replication
  machinery and the process being replicated -- i.e. no accidental
  sharing of names used by the process to get its work done and the
  name(s) used by the replication to effect copying. This latter
  revision of the definition of replication is crucial to obtaining
  the expected identity $!!P \sim !P$.
\end{remark}

\begin{remark}\label{rem:paradoxical_combinator}
  The reader familiar with the lambda calculus will have noticed the
  similarity between $D$ and the paradoxical combinator.

  [Ed. note: the existence of this seems to suggest we have to be more
  restrictive on the set of processes and names we admit if we are to
  support no-cloning.]
\end{remark}

\subsubsection{Bisimulation}

The computational dynamics gives rise to another kind of equivalence,
the equivalence of computational behavior. As previously mentioned
this is typically captured \emph{via} some form of bisimulation.

% The notion we use in this paper is weak barbed bisimulation
% \cite{milner91polyadicpi}.

The notion we use in this paper is derived from weak barbed
bisimulation \cite{milner91polyadicpi}. 

\begin{definition}
An \emph{observation relation}, $\downarrow_{\mathcal N}$, over a set
of names, $\mathcal N$, is the smallest relation satisfying the rules
below.

\infrule[Out-barb]{y \in {\mathcal N}, \; x \nameeq y}
		  {\outputp{x}{v} \downarrow_{\mathcal N} x}
\infrule[Par-barb]{\mbox{$P\downarrow_{\mathcal N} x$ or $Q\downarrow_{\mathcal N} x$}}
		  {\binpar{P}{Q} \downarrow_{\mathcal N} x}

We write $P \Downarrow_{\mathcal N} x$ if there is $Q$ such that 
$P \wred Q$ and $Q \downarrow_{\mathcal N} x$.
\end{definition}

\begin{definition}
%\label{def.bbisim}
An  ${\mathcal N}$-\emph{barbed bisimulation} over a set of names, ${\mathcal N}$, is a symmetric binary relation 
${\mathcal S}_{\mathcal N}$ between agents such that $P\rel{S}_{\mathcal N}Q$ implies:
\begin{enumerate}
\item If $P \red P'$ then $Q \wred Q'$ and $P'\rel{S}_{\mathcal N} Q'$.
\item If $P\downarrow_{\mathcal N} x$, then $Q\Downarrow_{\mathcal N} x$.
\end{enumerate}
$P$ is ${\mathcal N}$-barbed bisimilar to $Q$, written
$P \wbbisim_{\mathcal N} Q$, if $P \rel{S}_{\mathcal N} Q$ for some ${\mathcal N}$-barbed bisimulation ${\mathcal S}_{\mathcal N}$.
\end{definition}

$\mathcal{R} \subseteq \pi \times \pi$

$P \mathcal{R} Q => \forall P'. P \red P' \Rightarrow \exists Q'. Q \red Q', P' \mathcal{R} Q'$

$P \vdash x \Rightarrow Q \vdash x$

\begin{mathpar}
  \inferrule*[lab=Out-barb]{x \nameeq y}{{y}!\langle{Q}\rangle \vdash x}
  \and
  \inferrule*[lab=Par-barb]{\mbox{$P\vdash x$ or $Q\vdash x$}}{\binpar{P}{Q} \vdash x}
\end{mathpar}

\subsubsection{Contexts}

One of the principle advantages of computational calculi like the
$\pi$-calculus is a well-defined notion of context,
contextual-equivalence and a correlation between
contextual-equivalence and notions of bisimulation. The notion of
context allows the decomposition of a process into (sub-)process and
its syntactic environment, its context. Thus, a context may be
thought of as a process with a ``hole'' (written $\Box$) in it. The
application of a context $M$ to a process $P$, written $M[P]$, is
tantamount to filling the hole in $M$ with $P$. In this paper we do
not need the full weight of this theory, but do make use of the notion
of context in the proof the main theorem. 

\begin{mathpar}
  \inferrule* [lab=summation] {} {{M_{M},M_{N}} \bc \Box \;|\; x.M_{A} \;|\; M_{M}+M_{N}}
  \and
  \inferrule* [lab=agent] {} {{M_{A}} \bc (\vec{x})M_{P} \;| \; \clift{P_0,\ldots,M_{P},\ldots,P_N}}
  \and \\
  \inferrule* [lab=process] {} {{M_{P}} \bc M_{N} \;| \;P|M_{P} }
\end{mathpar} 

\begin{mathpar}
  \inferrule* [lab=sychronization] {} {M_{N} \bc \Box \;|\; x?M_{F} \;|\; x!M_{C}}
  \and
  \inferrule* [lab=abstraction] {} {{M_{F}} \bc (x)M_{P} }
  \and
  \inferrule* [lab=concretion] {} {{M_{C}} \bc \langle M_{P} \rangle }
  \and \\
  \inferrule* [lab=process] {} {{M_{P}} \bc M_{N} \;| \;P|M_{P} }
\end{mathpar}

\begin{definition}[contextual application] Given a context $M$, and
  process $P$, we define the \emph{contextual application}, $M[P] :=
  M\{P/\Box\}$. That is, the contextual application of M to P is the
  substitution of $P$ for $\Box$ in $M$.
\end{definition}

$\meaningof{-} : L \to \mathcal{P}(\pi)$

\begin{mathpar}
  \inferrule* [lab=collection] {} {\meaningof{true} = \pi, \and \meaningof{~E} = \pi \setminus \meaningof{E}, \and \meaningof{E_{1} \& E_{2}} = \meaningof{E_{1}} \cap \meaningof{E_{2}}}
\end{mathpar}

\begin{mathpar}
  \inferrule* [lab=structure] {} {\meaningof{0} = \{ P \in \pi | P \equiv 0 \}, \and \\ \meaningof{E_1 | E_2} = \{ P \in \pi | P \equiv P_{1} | P_{2}, P_{1} \in \meaningof{E_{1}}, P_{2} \in \meaningof{E_2}\} }
\end{mathpar}

\begin{mathpar}
 \inferrule* [lab=behavior] {} {\meaningof{\langle a?b \rangle E} = \{ P \in \pi | P \equiv Q | u?(y)P', \\ \and \\\\ \and \\ \;\;\; u \in \meaningof{a}, \forall z.P'\{z/y\} \in \meaningof{E\{z/b\}}\}, \and \\ \meaningof{a!E} = \{ P \in \pi | P \equiv Q | x!\langle P' \rangle, x \in \meaningof{a} P' \in \meaningof{E}\} }
\end{mathpar}

\begin{mathpar}
 \inferrule* [lab=nominal] {} {\meaningof{\quotep{E}} = \{ \quotep{P} \in \quotep{\pi} | P \in \meaningof{E} \}, \and \meaningof{\quotep{P}} = \{ \quotep{Q} \in \quotep{\pi} | P \equiv Q \} \and \\ \meaningof{@\quotep{E}} = \{ P \in \pi | P \equiv @x, x \in \meaningof{E} \}}
\end{mathpar}

\begin{eqnarray*}
  \\
  \meaningof{-} : TS \to ST
\end{eqnarray*}

\begin{eqnarray*}
  \\
  L : TS \to ST
\end{eqnarray*}

\begin{eqnarray*}
  \\
  P \models E \iff P \in \meaningof{E}
\end{eqnarray*}

\begin{eqnarray*}
  P \approx_{L} Q \iff \forall E \in L. P \models E \iff Q \models E
\end{eqnarray*}

\begin{eqnarray*}
  P \approx_{K} Q
\end{eqnarray*}

\begin{eqnarray*}
  P \approx Q
\end{eqnarray*}

$\approx_{K} = \approx = \approx_{L}$

\subsubsection{Contextual duality}

Note that contexts extend the quotation operation to a family of
operations from processes to names. Given a context, $M$, we can
define a \emph{nominal context}, $\quotep{M}$ by $\quotep{M}[P] :=
\quotep{M[P]}$. To foreshadow what is to come we observe that these
operations enjoy a duality with processes very much like the duality
between vectors and maps from vectors to scalars.

Further, because the calculus is essentially higher-order, we have a
correspondence between contexts and processes. More specifically,
given a name $x$ and a context $M$ we can construct $M^{*}_{x}$ such
that 

\begin{mathpar}
  M^{*}_{x} | \lift{x}{P} \red M[P]
\end{mathpar}

namely,

\begin{mathpar}
  M^{*}_{x} := x?(u).M[\dropn{u}]
\end{mathpar}

The dependence of $M^{*}_{x}$ on a name makes it an abstraction, 

\begin{mathpar}
  M^{*} := (x)x?(u).M[\dropn{u}]
\end{mathpar}

\subsection{Additional notation}

It will sometimes be convenient to denote the process a name
quotes. We already have the notation $x = \quotep{P}$, but it will be
convenient to introduce an alternate notation, $\procn{x}$, when we
want to emphasize the connection to the use of the name. Note that, by
virtue of name equivalence, $\quotep{\procn{x}} \nameeq x$; so, the
notation is consistent with previous definitions.

Further, because names have structure it is possible to effect
substitutions on the basis of that structure. This means we need to
upgrade our notation for substitutions, which we accomplish by
adapting comprehension notation. Thus,

\begin{mathpar}
  P\{ y / x : x \in S \}
\end{mathpar}

is interpreted to mean the process derived from P by replacing (in a
capture-avoiding manner) each occurrence of $x$ in $S$ by $y$. For example,

\begin{mathpar}
  P\{ \quotep{\procn{x}|\procn{x}} / x : x \in \freenames{P} \}
\end{mathpar}

will replace each (occurrence) of a free name $x$ in $P$ by
$\quotep{\procn{x}|\procn{x}}$.

Also, we will avail ourselves of the notation $x^{L}$ and $x^{R}$ to
denote injections of a name into disjoint copies of the name
space. There are numerous ways to accomplish this. One example can be
found in \cite{MeredithR05}. This notation overloads to vectors of
names: $\vec{x}^{\pi} := (x_{i}^{\pi} \; : \; 0 \leq i < |\vec{x}| )$ where $\pi \in \{L,R\}$.

We also use $P^{\Box} := P|\Box$.

In \cite{MeredithR05} an interpretation of the new operator is
given. It turns out that there are several possible interpretations
all enjoying the requisite algebraic properties of the operator (see
\cite{milner91polyadicpi}). We will therefore make liberal use of
$(\nu\; \vec{x})P$.

% subsection the_syntax_and_semantics_of_the_notation_system (end)   

\input{qm2pi.qmops} 

\input{qm2pi.sterngerlach} 

\input{qm2pi.metric} 

% section concurrent_process_calculi (end)

%\input{qm2pi.proofsketch}

% section proof sketch (end)

%\input{qm2pi.slviaknots} 

% section spatial logic via knots (end)

\input{qm2pi.conclusion}

% section conclusion (end)

%\input{qm2pi.dtcodes} 

% section wiring algorithm (end)

\input{qm2pi.ack} 

% section acknowledgments (end)

\newpage


\bibliographystyle{plain}   
\bibliography{../../biblios/main.bib}

\input{qm2pi.rhodetails}

\end{document}

 

% section wiring algorithm (end)

\documentclass[12pt]{llncs}
%\documentclass{jktr}

\usepackage[pdftex]{hyperref}                   
\usepackage {listings}
\usepackage {mathpartir}
\usepackage{bcprules}
%\usepackage{listings}
                       
\usepackage{graphicx} 
%\usepackage[margins=2.5cm,nohead,nofoot]{geometry}
%\usepackage{geometry}
\usepackage{amsfonts}
\usepackage{amstext}
\usepackage{latexsym}
\usepackage{amssymb}
\usepackage{color}


%\include{myPreamble}
\include{qm2pi.local} 

%\ifpdf
%\usepackage[pdftex]{graphicx}
%\else
%\usepackage{graphicx}
%\fi

 % \ifpdf
%  \usepackage{pdfsync}
%  \if


%\title{Brief Article}
%\author{David F. Snyder}
%\author{L.G. Meredith}

%\address{Dept. of Math., Texas State University--San Marcos, San Marcos, TX 78666}
       
\pagestyle{empty}


\begin{document}

\lstset{language=[Objective]Caml,frame=shadowbox}

\input{qm2pi.front}

% section front matter (end)

\input{qm2pi.intro} 
 
% section introduction (end)

% \input{qm2pi.knotations} 

% section notation (end)

\input{qm2pi.process.calculi} 

% section concurrent_process_calculi_and_spatial_logics_ (end)
    
%\input{qm2pi.knots2pi} 

%\input{qm2pi.trefoil} 

%\input{qm2pi.mainthm} 

% subsection basic_interpretation (end)

%\input{qm2pi.rho.presentation} 
\subsection{The syntax and semantics of the notation system}\label{sub:the_syntax_and_semantics_of_the_notation_system} % (fold)

We now summarize a technical presentation of the calculus that
embodies our theory of dynamics. The typical presentation of such a
calculus follows the style of giving generators and relations on
them. The grammar, below, describing term constructors, freely
generates the set of processes, $\Proc$. This set is then quotiented
by a relation known as structural congruence and it is over this set
that the notion of dynamics is expressed. This presentation is
essentially that of \cite{MeredithR05} with the addition of
polyadicity and summation. For readability we have relegated some of
the technical subtleties to an appendix.

\subsubsection{Process grammar}\label{subsub:process_grammar}

\begin{mathpar}
  \inferrule* [lab=synchronization] {} {{M} \bc \pzero \;|\; x?F \;|\; x!C }
  \and
  \inferrule* [lab=abstraction] {} {{F} \bc (x)P}
  \and
  \inferrule* [lab=concretion] {} {{C} \bc \langle Q \rangle}
  \and
  \inferrule* [lab=process] {} {{P,Q} \bc M \;| \;P|Q \;|\; @{x}}
  \and
  \inferrule* [lab=name] {} {{x} \bc \quotep{P}}
\end{mathpar} 

Note that $\vec{x}$ (resp. $\vec{P}$) denotes a vector of names
(resp. processes) of length $|\vec{x}|$ (resp. $|\vec{P}|$). We adopt
the following useful abbreviations.

\begin{mathpar}
   x?(\vec{y}).P := x.(\vec{y})P \and  x\clift{\vec{P}} := x.\clift{\vec{P}}
   \and x!(y) := \lift{x}{\dropn{y}}
   \and \Pi_{i=0}^{n-1}P_i := P_0 | \ldots | P_{n-1}
\end{mathpar}

\subsubsection{Structural congruence}

\paragraph{Free and bound names and alpha-equivalence.} At the
core of structural equivalence is alpha-equivalence which identifies
process that are the same up to a change of variable. Formally, we
recognize the distinction between free and bound names. The free names
of a process, $\freenames{P}$, may be calculated recursively as
follows:

\begin{mathpar}
\freenames{\pzero} := \emptyset
  \and \\
  \freenames{x?(y).P} := \{ x \} \cup (\freenames{P} \setminus \{ y \})
  \and 
  \freenames{x!\langle P \rangle} := \{ x \} \cup \{ P \} 
  \and \\
  \freenames{P|Q} := \freenames{P} \cup \freenames{Q}
  \and \\
  \freenames{@{x}} := \{ x \}
\end{mathpar}

$\pi$
$\quotep{\pi}$

$\freenames{-} : \pi \to \mathcal{P}(\quotep{\pi})$

\begin{eqnarray*}
  \freenames{\pzero} & := & \emptyset \\
  \freenames{x?(y).P} & := & \{ x \} \cup (\freenames{P} \setminus \{ y \}) \\
  \freenames{x!\langle P \rangle} & := & \{ x \} \cup \{ P \} \\
  \freenames{P|Q} & := & \freenames{P} \cup \freenames{Q} \\
  \freenames{\dropn{x}} & := & \{ x \}
\end{eqnarray*}

The bound names of a process, $\boundnames{P}$, are those names occurring in $P$
that are not free. For example, in $x?(y).0$, the name $x$ is free, while $y$ is bound.

\begin{mathpar}
  \inferrule* [lab=monoidal-laws] {} { P|Q \equiv Q|P \and P|0 \equiv P \and P|(Q|R) \equiv (P|Q)|R }
\end{mathpar}

\begin{mathpar}
  \inferrule* [lab=alpha-equivalence] {} { (x)P \equiv (y)P\{y/x\} \and y \not\in \freenames{P} }
\end{mathpar}

\begin{definition}
Then two processes, $P,Q$, are alpha-equivalent if $P = Q\{\vec{y}/\vec{x}\}$ for
some $\vec{x} \in \boundnames{Q},\vec{y} \in \boundnames{P}$, where $Q\{\vec{y}/\vec{x}\}$
denotes the capture-avoiding substitution of $\vec{y}$ for $\vec{x}$ in $Q$.
\end{definition}

\begin{definition}
  The {\em structural congruence} \cite{SangiorgiWalker} , $\equiv$,
  between processes is the least congruence containing
  alpha-equivalence, satisfying the abelian monoid laws
  (associativity, commutativity and $\pzero$ as identity) for parallel
  composition $|$ and for summation $+$.
\end{definition}

\subsection{Name equivalence}

We take name equivalence, written $\nameeq$, to be the smallest
equivalence relation generated by the following rules.

\begin{mathpar}
\inferrule*[lab=Quote-drop]
{ }
{ \quotep{@{x}} \nameeq x }

\inferrule*[lab=Struct-equiv]
{ P \scong Q }
{ \quotep{P} \nameeq \quotep{Q} }
\end{mathpar}

The astute reader will have noticed that the mutual recursion of names
and processes imposes a mutual recursion on alpha-equivalence and
structural equivalence via name-equivalence. Fortunately, all of this
works out pleasantly and we may calculate in the natural way, free of
concern. The reader interested in the details is referred to the
appendix \ref{appendix:rho_details}.

\subsection{Substitution}

We use $\Proc$ for the set of processes, $\QProc$ for the set of
names, and $\id{\{}\vec{y} / \vec{x} \id{\}}$ to denote partial maps,
$s : \QProc \rightarrow \QProc$. A map, $s$ lifts, uniquely, to a map
on process terms, $\widehat{s} : \Proc \rightarrow \Proc$ by the
following equations.

\begin{mathpar}
  (0) \psubstp{Q}{P} := 0 \\
  (R \juxtap S) \psubstp{Q}{P}
  :=    
  (R)\psubstp{Q}{P} \juxtap (S) \psubstp{Q}{P} \\
  (x?(y).R) \psubstp{Q}{P}    
  :=    
  (x)\substp{Q}{P} (z)\concat( (R \psubstn{z}{y}) \psubstp{Q}{P} ) \\
  (\lift{x}{R}) \psubstp{Q}{P}  
  :=
  \lift{(x)\substp{Q}{P}}{ R \psubstp{Q}{P} } \\
%   (\dropn{x})  \psubstp{Q}{P}       
%   := 
%   \left\{ 
%     \begin{array}{ccc} 
%       \dropn{\quotep{Q}} & & x \nameeq \quotep{P} \\
%       \dropn{x} & & otherwise \\
%     \end{array}
%   \right. 
  (\dropn{x})  \psubstp{Q}{P}       
  := 
  \left\{ 
    \begin{array}{ccc} 
      Q & & x \nameeq \quotep{P} \\
      \dropn{x} & & otherwise \\
    \end{array}
  \right.
\end{mathpar}
 

where

\begin{eqnarray}
  (x)\id{\{} \lpquote Q \rpquote / \lpquote P \rpquote \id{\}}            = 
  \left\{ 
    \begin{array}{ccc}
      \lpquote Q \rpquote & & x \nameeq \lpquote P \rpquote \\
      x & & otherwise \\
    \end{array}
  \right. \nonumber
\end{eqnarray}

and $z$ is chosen distinct from $\quotep{P}$, $\quotep{Q}$, the free
names in $Q$, and all the names in $R$. Our $\alpha$-equivalence will
be built in the standard way from this substitution.

\begin{remark}\label{rem:no_self_referential_names}
  One consequence of these definitions is that $\forall P. \quotep{P}
  \not\in \freenames{P}$.
\end{remark}

\subsection{ Dynamic quote: an example }

Anticipating something of what's to come, consider applying the
substitution, $\widehat{\id{\{}u / z \id{\}}}$, to the following pair
of processes, $\lift{w}{y!(z)}$ and $w[ \lpquote y!(z) \rpquote ]$.

\begin{eqnarray}
	\lift{w}{y!(z)}\widehat{\id{\{}u / z \id{\}}}
		& = &
		\lift{w}{y!(u)} \nonumber\\
	w[ \lpquote y!(z) \rpquote ] \widehat{ \id{\{}u / z \id{\}} }
		& = &
		w[ \lpquote y!(z) \rpquote ] \nonumber
\end{eqnarray}

Because the body of the process between quotes is impervious to
substitution, we get radically different answers. In fact, by
examining the first process in an input context,
e.g. $x?(z).\lift{w}{y!(z)}$, we see that the process under the lift
operator may be shaped by prefixed inputs binding a name inside it. In
this sense, the lift operator will be seen as a way to dynamically
construct processes before reifying them as names.

Finally equipped with these standard features we can present the
dynamics of the calculus.

\subsubsection{Operational semantics} 

Finally, we introduce the computational dynamics. What marks these
algebras as distinct from other more traditionally studied algebraic
structures, e.g. vector spaces or polynomial rings, is the manner in
which dynamics is captured. In traditional structures, dynamics is typically
expressed through morphisms between such structures, as in linear maps
between vector spaces or morphisms between rings. In algebras
associated with the semantics of computation, the dynamics is
expressed as part of the algebraic structure itself, through a
reduction reduction relation typically denoted by $\red$. Below, we
give a recursive presentation of this relation for the calculus used
in the encoding.

$\red \subseteq \pi \times \pi$
$\red : \pi \to \mathcal{P}(\pi)$

\begin{mathpar}
  \inferrule* [lab=Comm] { \textsf{match}( x_{src}, x_{trgt} ) } { x_{trgt}?(y)P \; | \; x_{src}!\langle {Q} \rangle \red P\{\quotep{Q}/y}\} }
  \and \\
  \inferrule* [lab=Par] {{P} \red {P}'} {{{P} | {Q}} \red {{P}' | {Q}}}
  \and
  \inferrule* [lab=Equiv]{{{P} \scong {P}'} \andalso {{P}' \red {Q}'} \andalso {{Q}' \scong {Q}}}{{P} \red {Q}}
\end{mathpar}

\begin{eqnarray*}
  match_{\equiv} (\quotep{P},\quotep{Q}) & := & P \equiv Q \\
  match_{\dagger}(\quotep{P},\quotep{Q}) & := & \forall R. P|Q \red^{*} R => R \red^{*} 0 \\
  match_{K}(\quotep{P},\quotep{Q}) & := & K \mbox{ for some context } K
\end{eqnarray*}

$u?(x)P | u!\langle Q \rangle \red P\{\quotep{Q}/x\}$

%We write $\wred$ for $\red^*$, and $P\red$ if $\exists Q $ such that $ P \red Q$.
We write $P\red$ if $\exists Q $ such that $ P \red Q$ and $P\not\red$, otherwise.

\section{Replication}

As mentioned before, it is known that replication (and hence
recursion) can be implemented in a higher-order process algebra
\cite{SangiorgiWalker}. As our first example of calculation with the
machinery thus far presented we give the construction explicitly in
the {\rhoc}.

\begin{eqnarray}
	D_{x} & := & \prefix{x}{y}{(\binpar{\outputp{x}{y}}{@{y}})} \nonumber\\
	\bangp_{x}{P} & := & \binpar{{x}!\langle{\binpar{D_{x}}{P}}\rangle}{D_{x}} \nonumber
\end{eqnarray}

\begin{eqnarray}
	\bangp_{x}{P} & & \nonumber\\
	=
	& {x}!\langle{(\prefix{x}{y}{(\outputp{x}{y} | @{y})) | P}}\rangle 
	      | \prefix{x}{y}{(\outputp{x}{y} | @{y})} & \nonumber\\
	\red
	& (\outputp{x}{y} | @{y})\substn{\quotep{(\prefix{x}{y}{(@{y} | \outputp{x}{y})) | P}}}{y} & \nonumber\\
	=
	& \outputp{x}{\quotep{(\prefix{x}{y}{(\outputp{x}{y} | @{y})) | P}}}
	  | {(\prefix{x}{y}{(\outputp{x}{y} | @{y})) | P}} & \nonumber\\
	\red
	& \ldots & \nonumber\\
	\red^*
	& P | P | \ldots & \nonumber
\end{eqnarray}

Of course, this encoding, as an implementation, runs away, unfolding
$\bangp{P}$ eagerly. A lazier and more implementable replication
operator, restricted to input-guarded processes, may be obtained as follows.

\begin{eqnarray}
\bangp{\prefix{u}{v}{P}} 
	:= 
	\binpar{\lift{x}{\prefix{u}{v}{(\binpar{D(x)}{P})}}}{D(x)} \nonumber
\end{eqnarray}

\begin{remark}
  Note that the lazier definition still does not deal with summation
  or mixed summation (i.e. sums over input and output). The reader is
  invited to construct definitions of replication that deal with these
  features. 

  Further, the definitions are parameterized in a name, $x$. Can you,
  gentle reader, make a definition that eliminates this parameter and
  guarantees no accidental interaction between the replication
  machinery and the process being replicated -- i.e. no accidental
  sharing of names used by the process to get its work done and the
  name(s) used by the replication to effect copying. This latter
  revision of the definition of replication is crucial to obtaining
  the expected identity $!!P \sim !P$.
\end{remark}

\begin{remark}\label{rem:paradoxical_combinator}
  The reader familiar with the lambda calculus will have noticed the
  similarity between $D$ and the paradoxical combinator.

  [Ed. note: the existence of this seems to suggest we have to be more
  restrictive on the set of processes and names we admit if we are to
  support no-cloning.]
\end{remark}

\subsubsection{Bisimulation}

The computational dynamics gives rise to another kind of equivalence,
the equivalence of computational behavior. As previously mentioned
this is typically captured \emph{via} some form of bisimulation.

% The notion we use in this paper is weak barbed bisimulation
% \cite{milner91polyadicpi}.

The notion we use in this paper is derived from weak barbed
bisimulation \cite{milner91polyadicpi}. 

\begin{definition}
An \emph{observation relation}, $\downarrow_{\mathcal N}$, over a set
of names, $\mathcal N$, is the smallest relation satisfying the rules
below.

\infrule[Out-barb]{y \in {\mathcal N}, \; x \nameeq y}
		  {\outputp{x}{v} \downarrow_{\mathcal N} x}
\infrule[Par-barb]{\mbox{$P\downarrow_{\mathcal N} x$ or $Q\downarrow_{\mathcal N} x$}}
		  {\binpar{P}{Q} \downarrow_{\mathcal N} x}

We write $P \Downarrow_{\mathcal N} x$ if there is $Q$ such that 
$P \wred Q$ and $Q \downarrow_{\mathcal N} x$.
\end{definition}

\begin{definition}
%\label{def.bbisim}
An  ${\mathcal N}$-\emph{barbed bisimulation} over a set of names, ${\mathcal N}$, is a symmetric binary relation 
${\mathcal S}_{\mathcal N}$ between agents such that $P\rel{S}_{\mathcal N}Q$ implies:
\begin{enumerate}
\item If $P \red P'$ then $Q \wred Q'$ and $P'\rel{S}_{\mathcal N} Q'$.
\item If $P\downarrow_{\mathcal N} x$, then $Q\Downarrow_{\mathcal N} x$.
\end{enumerate}
$P$ is ${\mathcal N}$-barbed bisimilar to $Q$, written
$P \wbbisim_{\mathcal N} Q$, if $P \rel{S}_{\mathcal N} Q$ for some ${\mathcal N}$-barbed bisimulation ${\mathcal S}_{\mathcal N}$.
\end{definition}

$\mathcal{R} \subseteq \pi \times \pi$

$P \mathcal{R} Q => \forall P'. P \red P' \Rightarrow \exists Q'. Q \red Q', P' \mathcal{R} Q'$

$P \vdash x \Rightarrow Q \vdash x$

\begin{mathpar}
  \inferrule*[lab=Out-barb]{x \nameeq y}{{y}!\langle{Q}\rangle \vdash x}
  \and
  \inferrule*[lab=Par-barb]{\mbox{$P\vdash x$ or $Q\vdash x$}}{\binpar{P}{Q} \vdash x}
\end{mathpar}

\subsubsection{Contexts}

One of the principle advantages of computational calculi like the
$\pi$-calculus is a well-defined notion of context,
contextual-equivalence and a correlation between
contextual-equivalence and notions of bisimulation. The notion of
context allows the decomposition of a process into (sub-)process and
its syntactic environment, its context. Thus, a context may be
thought of as a process with a ``hole'' (written $\Box$) in it. The
application of a context $M$ to a process $P$, written $M[P]$, is
tantamount to filling the hole in $M$ with $P$. In this paper we do
not need the full weight of this theory, but do make use of the notion
of context in the proof the main theorem. 

\begin{mathpar}
  \inferrule* [lab=summation] {} {{M_{M},M_{N}} \bc \Box \;|\; x.M_{A} \;|\; M_{M}+M_{N}}
  \and
  \inferrule* [lab=agent] {} {{M_{A}} \bc (\vec{x})M_{P} \;| \; \clift{P_0,\ldots,M_{P},\ldots,P_N}}
  \and \\
  \inferrule* [lab=process] {} {{M_{P}} \bc M_{N} \;| \;P|M_{P} }
\end{mathpar} 

\begin{mathpar}
  \inferrule* [lab=sychronization] {} {M_{N} \bc \Box \;|\; x?M_{F} \;|\; x!M_{C}}
  \and
  \inferrule* [lab=abstraction] {} {{M_{F}} \bc (x)M_{P} }
  \and
  \inferrule* [lab=concretion] {} {{M_{C}} \bc \langle M_{P} \rangle }
  \and \\
  \inferrule* [lab=process] {} {{M_{P}} \bc M_{N} \;| \;P|M_{P} }
\end{mathpar}

\begin{definition}[contextual application] Given a context $M$, and
  process $P$, we define the \emph{contextual application}, $M[P] :=
  M\{P/\Box\}$. That is, the contextual application of M to P is the
  substitution of $P$ for $\Box$ in $M$.
\end{definition}

$\meaningof{-} : L \to \mathcal{P}(\pi)$

\begin{mathpar}
  \inferrule* [lab=collection] {} {\meaningof{true} = \pi, \and \meaningof{~E} = \pi \setminus \meaningof{E}, \and \meaningof{E_{1} \& E_{2}} = \meaningof{E_{1}} \cap \meaningof{E_{2}}}
\end{mathpar}

\begin{mathpar}
  \inferrule* [lab=structure] {} {\meaningof{0} = \{ P \in \pi | P \equiv 0 \}, \and \\ \meaningof{E_1 | E_2} = \{ P \in \pi | P \equiv P_{1} | P_{2}, P_{1} \in \meaningof{E_{1}}, P_{2} \in \meaningof{E_2}\} }
\end{mathpar}

\begin{mathpar}
 \inferrule* [lab=behavior] {} {\meaningof{\langle a?b \rangle E} = \{ P \in \pi | P \equiv Q | u?(y)P', \\ \and \\\\ \and \\ \;\;\; u \in \meaningof{a}, \forall z.P'\{z/y\} \in \meaningof{E\{z/b\}}\}, \and \\ \meaningof{a!E} = \{ P \in \pi | P \equiv Q | x!\langle P' \rangle, x \in \meaningof{a} P' \in \meaningof{E}\} }
\end{mathpar}

\begin{mathpar}
 \inferrule* [lab=nominal] {} {\meaningof{\quotep{E}} = \{ \quotep{P} \in \quotep{\pi} | P \in \meaningof{E} \}, \and \meaningof{\quotep{P}} = \{ \quotep{Q} \in \quotep{\pi} | P \equiv Q \} \and \\ \meaningof{@\quotep{E}} = \{ P \in \pi | P \equiv @x, x \in \meaningof{E} \}}
\end{mathpar}

\begin{eqnarray*}
  \\
  \meaningof{-} : TS \to ST
\end{eqnarray*}

\begin{eqnarray*}
  \\
  L : TS \to ST
\end{eqnarray*}

\begin{eqnarray*}
  \\
  P \models E \iff P \in \meaningof{E}
\end{eqnarray*}

\begin{eqnarray*}
  P \approx_{L} Q \iff \forall E \in L. P \models E \iff Q \models E
\end{eqnarray*}

\begin{eqnarray*}
  P \approx_{K} Q
\end{eqnarray*}

\begin{eqnarray*}
  P \approx Q
\end{eqnarray*}

$\approx_{K} = \approx = \approx_{L}$

\subsubsection{Contextual duality}

Note that contexts extend the quotation operation to a family of
operations from processes to names. Given a context, $M$, we can
define a \emph{nominal context}, $\quotep{M}$ by $\quotep{M}[P] :=
\quotep{M[P]}$. To foreshadow what is to come we observe that these
operations enjoy a duality with processes very much like the duality
between vectors and maps from vectors to scalars.

Further, because the calculus is essentially higher-order, we have a
correspondence between contexts and processes. More specifically,
given a name $x$ and a context $M$ we can construct $M^{*}_{x}$ such
that 

\begin{mathpar}
  M^{*}_{x} | \lift{x}{P} \red M[P]
\end{mathpar}

namely,

\begin{mathpar}
  M^{*}_{x} := x?(u).M[\dropn{u}]
\end{mathpar}

The dependence of $M^{*}_{x}$ on a name makes it an abstraction, 

\begin{mathpar}
  M^{*} := (x)x?(u).M[\dropn{u}]
\end{mathpar}

\subsection{Additional notation}

It will sometimes be convenient to denote the process a name
quotes. We already have the notation $x = \quotep{P}$, but it will be
convenient to introduce an alternate notation, $\procn{x}$, when we
want to emphasize the connection to the use of the name. Note that, by
virtue of name equivalence, $\quotep{\procn{x}} \nameeq x$; so, the
notation is consistent with previous definitions.

Further, because names have structure it is possible to effect
substitutions on the basis of that structure. This means we need to
upgrade our notation for substitutions, which we accomplish by
adapting comprehension notation. Thus,

\begin{mathpar}
  P\{ y / x : x \in S \}
\end{mathpar}

is interpreted to mean the process derived from P by replacing (in a
capture-avoiding manner) each occurrence of $x$ in $S$ by $y$. For example,

\begin{mathpar}
  P\{ \quotep{\procn{x}|\procn{x}} / x : x \in \freenames{P} \}
\end{mathpar}

will replace each (occurrence) of a free name $x$ in $P$ by
$\quotep{\procn{x}|\procn{x}}$.

Also, we will avail ourselves of the notation $x^{L}$ and $x^{R}$ to
denote injections of a name into disjoint copies of the name
space. There are numerous ways to accomplish this. One example can be
found in \cite{MeredithR05}. This notation overloads to vectors of
names: $\vec{x}^{\pi} := (x_{i}^{\pi} \; : \; 0 \leq i < |\vec{x}| )$ where $\pi \in \{L,R\}$.

We also use $P^{\Box} := P|\Box$.

In \cite{MeredithR05} an interpretation of the new operator is
given. It turns out that there are several possible interpretations
all enjoying the requisite algebraic properties of the operator (see
\cite{milner91polyadicpi}). We will therefore make liberal use of
$(\nu\; \vec{x})P$.

% subsection the_syntax_and_semantics_of_the_notation_system (end)   

\input{qm2pi.qmops} 

\input{qm2pi.sterngerlach} 

\input{qm2pi.metric} 

% section concurrent_process_calculi (end)

%\input{qm2pi.proofsketch}

% section proof sketch (end)

%\input{qm2pi.slviaknots} 

% section spatial logic via knots (end)

\input{qm2pi.conclusion}

% section conclusion (end)

%\input{qm2pi.dtcodes} 

% section wiring algorithm (end)

\input{qm2pi.ack} 

% section acknowledgments (end)

\newpage


\bibliographystyle{plain}   
\bibliography{../../biblios/main.bib}

\input{qm2pi.rhodetails}

\end{document}

 

% section acknowledgments (end)

\newpage


\bibliographystyle{plain}   
\bibliography{../../biblios/main.bib}

\documentclass[12pt]{llncs}
%\documentclass{jktr}

\usepackage[pdftex]{hyperref}                   
\usepackage {listings}
\usepackage {mathpartir}
\usepackage{bcprules}
%\usepackage{listings}
                       
\usepackage{graphicx} 
%\usepackage[margins=2.5cm,nohead,nofoot]{geometry}
%\usepackage{geometry}
\usepackage{amsfonts}
\usepackage{amstext}
\usepackage{latexsym}
\usepackage{amssymb}
\usepackage{color}


%\include{myPreamble}
\include{qm2pi.local} 

%\ifpdf
%\usepackage[pdftex]{graphicx}
%\else
%\usepackage{graphicx}
%\fi

 % \ifpdf
%  \usepackage{pdfsync}
%  \if


%\title{Brief Article}
%\author{David F. Snyder}
%\author{L.G. Meredith}

%\address{Dept. of Math., Texas State University--San Marcos, San Marcos, TX 78666}
       
\pagestyle{empty}


\begin{document}

\lstset{language=[Objective]Caml,frame=shadowbox}

\input{qm2pi.front}

% section front matter (end)

\input{qm2pi.intro} 
 
% section introduction (end)

% \input{qm2pi.knotations} 

% section notation (end)

\input{qm2pi.process.calculi} 

% section concurrent_process_calculi_and_spatial_logics_ (end)
    
%\input{qm2pi.knots2pi} 

%\input{qm2pi.trefoil} 

%\input{qm2pi.mainthm} 

% subsection basic_interpretation (end)

%\input{qm2pi.rho.presentation} 
\subsection{The syntax and semantics of the notation system}\label{sub:the_syntax_and_semantics_of_the_notation_system} % (fold)

We now summarize a technical presentation of the calculus that
embodies our theory of dynamics. The typical presentation of such a
calculus follows the style of giving generators and relations on
them. The grammar, below, describing term constructors, freely
generates the set of processes, $\Proc$. This set is then quotiented
by a relation known as structural congruence and it is over this set
that the notion of dynamics is expressed. This presentation is
essentially that of \cite{MeredithR05} with the addition of
polyadicity and summation. For readability we have relegated some of
the technical subtleties to an appendix.

\subsubsection{Process grammar}\label{subsub:process_grammar}

\begin{mathpar}
  \inferrule* [lab=synchronization] {} {{M} \bc \pzero \;|\; x?F \;|\; x!C }
  \and
  \inferrule* [lab=abstraction] {} {{F} \bc (x)P}
  \and
  \inferrule* [lab=concretion] {} {{C} \bc \langle Q \rangle}
  \and
  \inferrule* [lab=process] {} {{P,Q} \bc M \;| \;P|Q \;|\; @{x}}
  \and
  \inferrule* [lab=name] {} {{x} \bc \quotep{P}}
\end{mathpar} 

Note that $\vec{x}$ (resp. $\vec{P}$) denotes a vector of names
(resp. processes) of length $|\vec{x}|$ (resp. $|\vec{P}|$). We adopt
the following useful abbreviations.

\begin{mathpar}
   x?(\vec{y}).P := x.(\vec{y})P \and  x\clift{\vec{P}} := x.\clift{\vec{P}}
   \and x!(y) := \lift{x}{\dropn{y}}
   \and \Pi_{i=0}^{n-1}P_i := P_0 | \ldots | P_{n-1}
\end{mathpar}

\subsubsection{Structural congruence}

\paragraph{Free and bound names and alpha-equivalence.} At the
core of structural equivalence is alpha-equivalence which identifies
process that are the same up to a change of variable. Formally, we
recognize the distinction between free and bound names. The free names
of a process, $\freenames{P}$, may be calculated recursively as
follows:

\begin{mathpar}
\freenames{\pzero} := \emptyset
  \and \\
  \freenames{x?(y).P} := \{ x \} \cup (\freenames{P} \setminus \{ y \})
  \and 
  \freenames{x!\langle P \rangle} := \{ x \} \cup \{ P \} 
  \and \\
  \freenames{P|Q} := \freenames{P} \cup \freenames{Q}
  \and \\
  \freenames{@{x}} := \{ x \}
\end{mathpar}

$\pi$
$\quotep{\pi}$

$\freenames{-} : \pi \to \mathcal{P}(\quotep{\pi})$

\begin{eqnarray*}
  \freenames{\pzero} & := & \emptyset \\
  \freenames{x?(y).P} & := & \{ x \} \cup (\freenames{P} \setminus \{ y \}) \\
  \freenames{x!\langle P \rangle} & := & \{ x \} \cup \{ P \} \\
  \freenames{P|Q} & := & \freenames{P} \cup \freenames{Q} \\
  \freenames{\dropn{x}} & := & \{ x \}
\end{eqnarray*}

The bound names of a process, $\boundnames{P}$, are those names occurring in $P$
that are not free. For example, in $x?(y).0$, the name $x$ is free, while $y$ is bound.

\begin{mathpar}
  \inferrule* [lab=monoidal-laws] {} { P|Q \equiv Q|P \and P|0 \equiv P \and P|(Q|R) \equiv (P|Q)|R }
\end{mathpar}

\begin{mathpar}
  \inferrule* [lab=alpha-equivalence] {} { (x)P \equiv (y)P\{y/x\} \and y \not\in \freenames{P} }
\end{mathpar}

\begin{definition}
Then two processes, $P,Q$, are alpha-equivalent if $P = Q\{\vec{y}/\vec{x}\}$ for
some $\vec{x} \in \boundnames{Q},\vec{y} \in \boundnames{P}$, where $Q\{\vec{y}/\vec{x}\}$
denotes the capture-avoiding substitution of $\vec{y}$ for $\vec{x}$ in $Q$.
\end{definition}

\begin{definition}
  The {\em structural congruence} \cite{SangiorgiWalker} , $\equiv$,
  between processes is the least congruence containing
  alpha-equivalence, satisfying the abelian monoid laws
  (associativity, commutativity and $\pzero$ as identity) for parallel
  composition $|$ and for summation $+$.
\end{definition}

\subsection{Name equivalence}

We take name equivalence, written $\nameeq$, to be the smallest
equivalence relation generated by the following rules.

\begin{mathpar}
\inferrule*[lab=Quote-drop]
{ }
{ \quotep{@{x}} \nameeq x }

\inferrule*[lab=Struct-equiv]
{ P \scong Q }
{ \quotep{P} \nameeq \quotep{Q} }
\end{mathpar}

The astute reader will have noticed that the mutual recursion of names
and processes imposes a mutual recursion on alpha-equivalence and
structural equivalence via name-equivalence. Fortunately, all of this
works out pleasantly and we may calculate in the natural way, free of
concern. The reader interested in the details is referred to the
appendix \ref{appendix:rho_details}.

\subsection{Substitution}

We use $\Proc$ for the set of processes, $\QProc$ for the set of
names, and $\id{\{}\vec{y} / \vec{x} \id{\}}$ to denote partial maps,
$s : \QProc \rightarrow \QProc$. A map, $s$ lifts, uniquely, to a map
on process terms, $\widehat{s} : \Proc \rightarrow \Proc$ by the
following equations.

\begin{mathpar}
  (0) \psubstp{Q}{P} := 0 \\
  (R \juxtap S) \psubstp{Q}{P}
  :=    
  (R)\psubstp{Q}{P} \juxtap (S) \psubstp{Q}{P} \\
  (x?(y).R) \psubstp{Q}{P}    
  :=    
  (x)\substp{Q}{P} (z)\concat( (R \psubstn{z}{y}) \psubstp{Q}{P} ) \\
  (\lift{x}{R}) \psubstp{Q}{P}  
  :=
  \lift{(x)\substp{Q}{P}}{ R \psubstp{Q}{P} } \\
%   (\dropn{x})  \psubstp{Q}{P}       
%   := 
%   \left\{ 
%     \begin{array}{ccc} 
%       \dropn{\quotep{Q}} & & x \nameeq \quotep{P} \\
%       \dropn{x} & & otherwise \\
%     \end{array}
%   \right. 
  (\dropn{x})  \psubstp{Q}{P}       
  := 
  \left\{ 
    \begin{array}{ccc} 
      Q & & x \nameeq \quotep{P} \\
      \dropn{x} & & otherwise \\
    \end{array}
  \right.
\end{mathpar}
 

where

\begin{eqnarray}
  (x)\id{\{} \lpquote Q \rpquote / \lpquote P \rpquote \id{\}}            = 
  \left\{ 
    \begin{array}{ccc}
      \lpquote Q \rpquote & & x \nameeq \lpquote P \rpquote \\
      x & & otherwise \\
    \end{array}
  \right. \nonumber
\end{eqnarray}

and $z$ is chosen distinct from $\quotep{P}$, $\quotep{Q}$, the free
names in $Q$, and all the names in $R$. Our $\alpha$-equivalence will
be built in the standard way from this substitution.

\begin{remark}\label{rem:no_self_referential_names}
  One consequence of these definitions is that $\forall P. \quotep{P}
  \not\in \freenames{P}$.
\end{remark}

\subsection{ Dynamic quote: an example }

Anticipating something of what's to come, consider applying the
substitution, $\widehat{\id{\{}u / z \id{\}}}$, to the following pair
of processes, $\lift{w}{y!(z)}$ and $w[ \lpquote y!(z) \rpquote ]$.

\begin{eqnarray}
	\lift{w}{y!(z)}\widehat{\id{\{}u / z \id{\}}}
		& = &
		\lift{w}{y!(u)} \nonumber\\
	w[ \lpquote y!(z) \rpquote ] \widehat{ \id{\{}u / z \id{\}} }
		& = &
		w[ \lpquote y!(z) \rpquote ] \nonumber
\end{eqnarray}

Because the body of the process between quotes is impervious to
substitution, we get radically different answers. In fact, by
examining the first process in an input context,
e.g. $x?(z).\lift{w}{y!(z)}$, we see that the process under the lift
operator may be shaped by prefixed inputs binding a name inside it. In
this sense, the lift operator will be seen as a way to dynamically
construct processes before reifying them as names.

Finally equipped with these standard features we can present the
dynamics of the calculus.

\subsubsection{Operational semantics} 

Finally, we introduce the computational dynamics. What marks these
algebras as distinct from other more traditionally studied algebraic
structures, e.g. vector spaces or polynomial rings, is the manner in
which dynamics is captured. In traditional structures, dynamics is typically
expressed through morphisms between such structures, as in linear maps
between vector spaces or morphisms between rings. In algebras
associated with the semantics of computation, the dynamics is
expressed as part of the algebraic structure itself, through a
reduction reduction relation typically denoted by $\red$. Below, we
give a recursive presentation of this relation for the calculus used
in the encoding.

$\red \subseteq \pi \times \pi$
$\red : \pi \to \mathcal{P}(\pi)$

\begin{mathpar}
  \inferrule* [lab=Comm] { \textsf{match}( x_{src}, x_{trgt} ) } { x_{trgt}?(y)P \; | \; x_{src}!\langle {Q} \rangle \red P\{\quotep{Q}/y}\} }
  \and \\
  \inferrule* [lab=Par] {{P} \red {P}'} {{{P} | {Q}} \red {{P}' | {Q}}}
  \and
  \inferrule* [lab=Equiv]{{{P} \scong {P}'} \andalso {{P}' \red {Q}'} \andalso {{Q}' \scong {Q}}}{{P} \red {Q}}
\end{mathpar}

\begin{eqnarray*}
  match_{\equiv} (\quotep{P},\quotep{Q}) & := & P \equiv Q \\
  match_{\dagger}(\quotep{P},\quotep{Q}) & := & \forall R. P|Q \red^{*} R => R \red^{*} 0 \\
  match_{K}(\quotep{P},\quotep{Q}) & := & K \mbox{ for some context } K
\end{eqnarray*}

$u?(x)P | u!\langle Q \rangle \red P\{\quotep{Q}/x\}$

%We write $\wred$ for $\red^*$, and $P\red$ if $\exists Q $ such that $ P \red Q$.
We write $P\red$ if $\exists Q $ such that $ P \red Q$ and $P\not\red$, otherwise.

\section{Replication}

As mentioned before, it is known that replication (and hence
recursion) can be implemented in a higher-order process algebra
\cite{SangiorgiWalker}. As our first example of calculation with the
machinery thus far presented we give the construction explicitly in
the {\rhoc}.

\begin{eqnarray}
	D_{x} & := & \prefix{x}{y}{(\binpar{\outputp{x}{y}}{@{y}})} \nonumber\\
	\bangp_{x}{P} & := & \binpar{{x}!\langle{\binpar{D_{x}}{P}}\rangle}{D_{x}} \nonumber
\end{eqnarray}

\begin{eqnarray}
	\bangp_{x}{P} & & \nonumber\\
	=
	& {x}!\langle{(\prefix{x}{y}{(\outputp{x}{y} | @{y})) | P}}\rangle 
	      | \prefix{x}{y}{(\outputp{x}{y} | @{y})} & \nonumber\\
	\red
	& (\outputp{x}{y} | @{y})\substn{\quotep{(\prefix{x}{y}{(@{y} | \outputp{x}{y})) | P}}}{y} & \nonumber\\
	=
	& \outputp{x}{\quotep{(\prefix{x}{y}{(\outputp{x}{y} | @{y})) | P}}}
	  | {(\prefix{x}{y}{(\outputp{x}{y} | @{y})) | P}} & \nonumber\\
	\red
	& \ldots & \nonumber\\
	\red^*
	& P | P | \ldots & \nonumber
\end{eqnarray}

Of course, this encoding, as an implementation, runs away, unfolding
$\bangp{P}$ eagerly. A lazier and more implementable replication
operator, restricted to input-guarded processes, may be obtained as follows.

\begin{eqnarray}
\bangp{\prefix{u}{v}{P}} 
	:= 
	\binpar{\lift{x}{\prefix{u}{v}{(\binpar{D(x)}{P})}}}{D(x)} \nonumber
\end{eqnarray}

\begin{remark}
  Note that the lazier definition still does not deal with summation
  or mixed summation (i.e. sums over input and output). The reader is
  invited to construct definitions of replication that deal with these
  features. 

  Further, the definitions are parameterized in a name, $x$. Can you,
  gentle reader, make a definition that eliminates this parameter and
  guarantees no accidental interaction between the replication
  machinery and the process being replicated -- i.e. no accidental
  sharing of names used by the process to get its work done and the
  name(s) used by the replication to effect copying. This latter
  revision of the definition of replication is crucial to obtaining
  the expected identity $!!P \sim !P$.
\end{remark}

\begin{remark}\label{rem:paradoxical_combinator}
  The reader familiar with the lambda calculus will have noticed the
  similarity between $D$ and the paradoxical combinator.

  [Ed. note: the existence of this seems to suggest we have to be more
  restrictive on the set of processes and names we admit if we are to
  support no-cloning.]
\end{remark}

\subsubsection{Bisimulation}

The computational dynamics gives rise to another kind of equivalence,
the equivalence of computational behavior. As previously mentioned
this is typically captured \emph{via} some form of bisimulation.

% The notion we use in this paper is weak barbed bisimulation
% \cite{milner91polyadicpi}.

The notion we use in this paper is derived from weak barbed
bisimulation \cite{milner91polyadicpi}. 

\begin{definition}
An \emph{observation relation}, $\downarrow_{\mathcal N}$, over a set
of names, $\mathcal N$, is the smallest relation satisfying the rules
below.

\infrule[Out-barb]{y \in {\mathcal N}, \; x \nameeq y}
		  {\outputp{x}{v} \downarrow_{\mathcal N} x}
\infrule[Par-barb]{\mbox{$P\downarrow_{\mathcal N} x$ or $Q\downarrow_{\mathcal N} x$}}
		  {\binpar{P}{Q} \downarrow_{\mathcal N} x}

We write $P \Downarrow_{\mathcal N} x$ if there is $Q$ such that 
$P \wred Q$ and $Q \downarrow_{\mathcal N} x$.
\end{definition}

\begin{definition}
%\label{def.bbisim}
An  ${\mathcal N}$-\emph{barbed bisimulation} over a set of names, ${\mathcal N}$, is a symmetric binary relation 
${\mathcal S}_{\mathcal N}$ between agents such that $P\rel{S}_{\mathcal N}Q$ implies:
\begin{enumerate}
\item If $P \red P'$ then $Q \wred Q'$ and $P'\rel{S}_{\mathcal N} Q'$.
\item If $P\downarrow_{\mathcal N} x$, then $Q\Downarrow_{\mathcal N} x$.
\end{enumerate}
$P$ is ${\mathcal N}$-barbed bisimilar to $Q$, written
$P \wbbisim_{\mathcal N} Q$, if $P \rel{S}_{\mathcal N} Q$ for some ${\mathcal N}$-barbed bisimulation ${\mathcal S}_{\mathcal N}$.
\end{definition}

$\mathcal{R} \subseteq \pi \times \pi$

$P \mathcal{R} Q => \forall P'. P \red P' \Rightarrow \exists Q'. Q \red Q', P' \mathcal{R} Q'$

$P \vdash x \Rightarrow Q \vdash x$

\begin{mathpar}
  \inferrule*[lab=Out-barb]{x \nameeq y}{{y}!\langle{Q}\rangle \vdash x}
  \and
  \inferrule*[lab=Par-barb]{\mbox{$P\vdash x$ or $Q\vdash x$}}{\binpar{P}{Q} \vdash x}
\end{mathpar}

\subsubsection{Contexts}

One of the principle advantages of computational calculi like the
$\pi$-calculus is a well-defined notion of context,
contextual-equivalence and a correlation between
contextual-equivalence and notions of bisimulation. The notion of
context allows the decomposition of a process into (sub-)process and
its syntactic environment, its context. Thus, a context may be
thought of as a process with a ``hole'' (written $\Box$) in it. The
application of a context $M$ to a process $P$, written $M[P]$, is
tantamount to filling the hole in $M$ with $P$. In this paper we do
not need the full weight of this theory, but do make use of the notion
of context in the proof the main theorem. 

\begin{mathpar}
  \inferrule* [lab=summation] {} {{M_{M},M_{N}} \bc \Box \;|\; x.M_{A} \;|\; M_{M}+M_{N}}
  \and
  \inferrule* [lab=agent] {} {{M_{A}} \bc (\vec{x})M_{P} \;| \; \clift{P_0,\ldots,M_{P},\ldots,P_N}}
  \and \\
  \inferrule* [lab=process] {} {{M_{P}} \bc M_{N} \;| \;P|M_{P} }
\end{mathpar} 

\begin{mathpar}
  \inferrule* [lab=sychronization] {} {M_{N} \bc \Box \;|\; x?M_{F} \;|\; x!M_{C}}
  \and
  \inferrule* [lab=abstraction] {} {{M_{F}} \bc (x)M_{P} }
  \and
  \inferrule* [lab=concretion] {} {{M_{C}} \bc \langle M_{P} \rangle }
  \and \\
  \inferrule* [lab=process] {} {{M_{P}} \bc M_{N} \;| \;P|M_{P} }
\end{mathpar}

\begin{definition}[contextual application] Given a context $M$, and
  process $P$, we define the \emph{contextual application}, $M[P] :=
  M\{P/\Box\}$. That is, the contextual application of M to P is the
  substitution of $P$ for $\Box$ in $M$.
\end{definition}

$\meaningof{-} : L \to \mathcal{P}(\pi)$

\begin{mathpar}
  \inferrule* [lab=collection] {} {\meaningof{true} = \pi, \and \meaningof{~E} = \pi \setminus \meaningof{E}, \and \meaningof{E_{1} \& E_{2}} = \meaningof{E_{1}} \cap \meaningof{E_{2}}}
\end{mathpar}

\begin{mathpar}
  \inferrule* [lab=structure] {} {\meaningof{0} = \{ P \in \pi | P \equiv 0 \}, \and \\ \meaningof{E_1 | E_2} = \{ P \in \pi | P \equiv P_{1} | P_{2}, P_{1} \in \meaningof{E_{1}}, P_{2} \in \meaningof{E_2}\} }
\end{mathpar}

\begin{mathpar}
 \inferrule* [lab=behavior] {} {\meaningof{\langle a?b \rangle E} = \{ P \in \pi | P \equiv Q | u?(y)P', \\ \and \\\\ \and \\ \;\;\; u \in \meaningof{a}, \forall z.P'\{z/y\} \in \meaningof{E\{z/b\}}\}, \and \\ \meaningof{a!E} = \{ P \in \pi | P \equiv Q | x!\langle P' \rangle, x \in \meaningof{a} P' \in \meaningof{E}\} }
\end{mathpar}

\begin{mathpar}
 \inferrule* [lab=nominal] {} {\meaningof{\quotep{E}} = \{ \quotep{P} \in \quotep{\pi} | P \in \meaningof{E} \}, \and \meaningof{\quotep{P}} = \{ \quotep{Q} \in \quotep{\pi} | P \equiv Q \} \and \\ \meaningof{@\quotep{E}} = \{ P \in \pi | P \equiv @x, x \in \meaningof{E} \}}
\end{mathpar}

\begin{eqnarray*}
  \\
  \meaningof{-} : TS \to ST
\end{eqnarray*}

\begin{eqnarray*}
  \\
  L : TS \to ST
\end{eqnarray*}

\begin{eqnarray*}
  \\
  P \models E \iff P \in \meaningof{E}
\end{eqnarray*}

\begin{eqnarray*}
  P \approx_{L} Q \iff \forall E \in L. P \models E \iff Q \models E
\end{eqnarray*}

\begin{eqnarray*}
  P \approx_{K} Q
\end{eqnarray*}

\begin{eqnarray*}
  P \approx Q
\end{eqnarray*}

$\approx_{K} = \approx = \approx_{L}$

\subsubsection{Contextual duality}

Note that contexts extend the quotation operation to a family of
operations from processes to names. Given a context, $M$, we can
define a \emph{nominal context}, $\quotep{M}$ by $\quotep{M}[P] :=
\quotep{M[P]}$. To foreshadow what is to come we observe that these
operations enjoy a duality with processes very much like the duality
between vectors and maps from vectors to scalars.

Further, because the calculus is essentially higher-order, we have a
correspondence between contexts and processes. More specifically,
given a name $x$ and a context $M$ we can construct $M^{*}_{x}$ such
that 

\begin{mathpar}
  M^{*}_{x} | \lift{x}{P} \red M[P]
\end{mathpar}

namely,

\begin{mathpar}
  M^{*}_{x} := x?(u).M[\dropn{u}]
\end{mathpar}

The dependence of $M^{*}_{x}$ on a name makes it an abstraction, 

\begin{mathpar}
  M^{*} := (x)x?(u).M[\dropn{u}]
\end{mathpar}

\subsection{Additional notation}

It will sometimes be convenient to denote the process a name
quotes. We already have the notation $x = \quotep{P}$, but it will be
convenient to introduce an alternate notation, $\procn{x}$, when we
want to emphasize the connection to the use of the name. Note that, by
virtue of name equivalence, $\quotep{\procn{x}} \nameeq x$; so, the
notation is consistent with previous definitions.

Further, because names have structure it is possible to effect
substitutions on the basis of that structure. This means we need to
upgrade our notation for substitutions, which we accomplish by
adapting comprehension notation. Thus,

\begin{mathpar}
  P\{ y / x : x \in S \}
\end{mathpar}

is interpreted to mean the process derived from P by replacing (in a
capture-avoiding manner) each occurrence of $x$ in $S$ by $y$. For example,

\begin{mathpar}
  P\{ \quotep{\procn{x}|\procn{x}} / x : x \in \freenames{P} \}
\end{mathpar}

will replace each (occurrence) of a free name $x$ in $P$ by
$\quotep{\procn{x}|\procn{x}}$.

Also, we will avail ourselves of the notation $x^{L}$ and $x^{R}$ to
denote injections of a name into disjoint copies of the name
space. There are numerous ways to accomplish this. One example can be
found in \cite{MeredithR05}. This notation overloads to vectors of
names: $\vec{x}^{\pi} := (x_{i}^{\pi} \; : \; 0 \leq i < |\vec{x}| )$ where $\pi \in \{L,R\}$.

We also use $P^{\Box} := P|\Box$.

In \cite{MeredithR05} an interpretation of the new operator is
given. It turns out that there are several possible interpretations
all enjoying the requisite algebraic properties of the operator (see
\cite{milner91polyadicpi}). We will therefore make liberal use of
$(\nu\; \vec{x})P$.

% subsection the_syntax_and_semantics_of_the_notation_system (end)   

\input{qm2pi.qmops} 

\input{qm2pi.sterngerlach} 

\input{qm2pi.metric} 

% section concurrent_process_calculi (end)

%\input{qm2pi.proofsketch}

% section proof sketch (end)

%\input{qm2pi.slviaknots} 

% section spatial logic via knots (end)

\input{qm2pi.conclusion}

% section conclusion (end)

%\input{qm2pi.dtcodes} 

% section wiring algorithm (end)

\input{qm2pi.ack} 

% section acknowledgments (end)

\newpage


\bibliographystyle{plain}   
\bibliography{../../biblios/main.bib}

\input{qm2pi.rhodetails}

\end{document}



\end{document}

 

% section concurrent_process_calculi (end)

%\documentclass[12pt]{llncs}
%\documentclass{jktr}

\usepackage[pdftex]{hyperref}                   
\usepackage {listings}
\usepackage {mathpartir}
\usepackage{bcprules}
%\usepackage{listings}
                       
\usepackage{graphicx} 
%\usepackage[margins=2.5cm,nohead,nofoot]{geometry}
%\usepackage{geometry}
\usepackage{amsfonts}
\usepackage{amstext}
\usepackage{latexsym}
\usepackage{amssymb}
\usepackage{color}


%\include{myPreamble}
\documentclass[12pt]{llncs}
%\documentclass{jktr}

\usepackage[pdftex]{hyperref}                   
\usepackage {listings}
\usepackage {mathpartir}
\usepackage{bcprules}
%\usepackage{listings}
                       
\usepackage{graphicx} 
%\usepackage[margins=2.5cm,nohead,nofoot]{geometry}
%\usepackage{geometry}
\usepackage{amsfonts}
\usepackage{amstext}
\usepackage{latexsym}
\usepackage{amssymb}
\usepackage{color}


%\include{myPreamble}
\include{qm2pi.local} 

%\ifpdf
%\usepackage[pdftex]{graphicx}
%\else
%\usepackage{graphicx}
%\fi

 % \ifpdf
%  \usepackage{pdfsync}
%  \if


%\title{Brief Article}
%\author{David F. Snyder}
%\author{L.G. Meredith}

%\address{Dept. of Math., Texas State University--San Marcos, San Marcos, TX 78666}
       
\pagestyle{empty}


\begin{document}

\lstset{language=[Objective]Caml,frame=shadowbox}

\input{qm2pi.front}

% section front matter (end)

\input{qm2pi.intro} 
 
% section introduction (end)

% \input{qm2pi.knotations} 

% section notation (end)

\input{qm2pi.process.calculi} 

% section concurrent_process_calculi_and_spatial_logics_ (end)
    
%\input{qm2pi.knots2pi} 

%\input{qm2pi.trefoil} 

%\input{qm2pi.mainthm} 

% subsection basic_interpretation (end)

%\input{qm2pi.rho.presentation} 
\subsection{The syntax and semantics of the notation system}\label{sub:the_syntax_and_semantics_of_the_notation_system} % (fold)

We now summarize a technical presentation of the calculus that
embodies our theory of dynamics. The typical presentation of such a
calculus follows the style of giving generators and relations on
them. The grammar, below, describing term constructors, freely
generates the set of processes, $\Proc$. This set is then quotiented
by a relation known as structural congruence and it is over this set
that the notion of dynamics is expressed. This presentation is
essentially that of \cite{MeredithR05} with the addition of
polyadicity and summation. For readability we have relegated some of
the technical subtleties to an appendix.

\subsubsection{Process grammar}\label{subsub:process_grammar}

\begin{mathpar}
  \inferrule* [lab=synchronization] {} {{M} \bc \pzero \;|\; x?F \;|\; x!C }
  \and
  \inferrule* [lab=abstraction] {} {{F} \bc (x)P}
  \and
  \inferrule* [lab=concretion] {} {{C} \bc \langle Q \rangle}
  \and
  \inferrule* [lab=process] {} {{P,Q} \bc M \;| \;P|Q \;|\; @{x}}
  \and
  \inferrule* [lab=name] {} {{x} \bc \quotep{P}}
\end{mathpar} 

Note that $\vec{x}$ (resp. $\vec{P}$) denotes a vector of names
(resp. processes) of length $|\vec{x}|$ (resp. $|\vec{P}|$). We adopt
the following useful abbreviations.

\begin{mathpar}
   x?(\vec{y}).P := x.(\vec{y})P \and  x\clift{\vec{P}} := x.\clift{\vec{P}}
   \and x!(y) := \lift{x}{\dropn{y}}
   \and \Pi_{i=0}^{n-1}P_i := P_0 | \ldots | P_{n-1}
\end{mathpar}

\subsubsection{Structural congruence}

\paragraph{Free and bound names and alpha-equivalence.} At the
core of structural equivalence is alpha-equivalence which identifies
process that are the same up to a change of variable. Formally, we
recognize the distinction between free and bound names. The free names
of a process, $\freenames{P}$, may be calculated recursively as
follows:

\begin{mathpar}
\freenames{\pzero} := \emptyset
  \and \\
  \freenames{x?(y).P} := \{ x \} \cup (\freenames{P} \setminus \{ y \})
  \and 
  \freenames{x!\langle P \rangle} := \{ x \} \cup \{ P \} 
  \and \\
  \freenames{P|Q} := \freenames{P} \cup \freenames{Q}
  \and \\
  \freenames{@{x}} := \{ x \}
\end{mathpar}

$\pi$
$\quotep{\pi}$

$\freenames{-} : \pi \to \mathcal{P}(\quotep{\pi})$

\begin{eqnarray*}
  \freenames{\pzero} & := & \emptyset \\
  \freenames{x?(y).P} & := & \{ x \} \cup (\freenames{P} \setminus \{ y \}) \\
  \freenames{x!\langle P \rangle} & := & \{ x \} \cup \{ P \} \\
  \freenames{P|Q} & := & \freenames{P} \cup \freenames{Q} \\
  \freenames{\dropn{x}} & := & \{ x \}
\end{eqnarray*}

The bound names of a process, $\boundnames{P}$, are those names occurring in $P$
that are not free. For example, in $x?(y).0$, the name $x$ is free, while $y$ is bound.

\begin{mathpar}
  \inferrule* [lab=monoidal-laws] {} { P|Q \equiv Q|P \and P|0 \equiv P \and P|(Q|R) \equiv (P|Q)|R }
\end{mathpar}

\begin{mathpar}
  \inferrule* [lab=alpha-equivalence] {} { (x)P \equiv (y)P\{y/x\} \and y \not\in \freenames{P} }
\end{mathpar}

\begin{definition}
Then two processes, $P,Q$, are alpha-equivalent if $P = Q\{\vec{y}/\vec{x}\}$ for
some $\vec{x} \in \boundnames{Q},\vec{y} \in \boundnames{P}$, where $Q\{\vec{y}/\vec{x}\}$
denotes the capture-avoiding substitution of $\vec{y}$ for $\vec{x}$ in $Q$.
\end{definition}

\begin{definition}
  The {\em structural congruence} \cite{SangiorgiWalker} , $\equiv$,
  between processes is the least congruence containing
  alpha-equivalence, satisfying the abelian monoid laws
  (associativity, commutativity and $\pzero$ as identity) for parallel
  composition $|$ and for summation $+$.
\end{definition}

\subsection{Name equivalence}

We take name equivalence, written $\nameeq$, to be the smallest
equivalence relation generated by the following rules.

\begin{mathpar}
\inferrule*[lab=Quote-drop]
{ }
{ \quotep{@{x}} \nameeq x }

\inferrule*[lab=Struct-equiv]
{ P \scong Q }
{ \quotep{P} \nameeq \quotep{Q} }
\end{mathpar}

The astute reader will have noticed that the mutual recursion of names
and processes imposes a mutual recursion on alpha-equivalence and
structural equivalence via name-equivalence. Fortunately, all of this
works out pleasantly and we may calculate in the natural way, free of
concern. The reader interested in the details is referred to the
appendix \ref{appendix:rho_details}.

\subsection{Substitution}

We use $\Proc$ for the set of processes, $\QProc$ for the set of
names, and $\id{\{}\vec{y} / \vec{x} \id{\}}$ to denote partial maps,
$s : \QProc \rightarrow \QProc$. A map, $s$ lifts, uniquely, to a map
on process terms, $\widehat{s} : \Proc \rightarrow \Proc$ by the
following equations.

\begin{mathpar}
  (0) \psubstp{Q}{P} := 0 \\
  (R \juxtap S) \psubstp{Q}{P}
  :=    
  (R)\psubstp{Q}{P} \juxtap (S) \psubstp{Q}{P} \\
  (x?(y).R) \psubstp{Q}{P}    
  :=    
  (x)\substp{Q}{P} (z)\concat( (R \psubstn{z}{y}) \psubstp{Q}{P} ) \\
  (\lift{x}{R}) \psubstp{Q}{P}  
  :=
  \lift{(x)\substp{Q}{P}}{ R \psubstp{Q}{P} } \\
%   (\dropn{x})  \psubstp{Q}{P}       
%   := 
%   \left\{ 
%     \begin{array}{ccc} 
%       \dropn{\quotep{Q}} & & x \nameeq \quotep{P} \\
%       \dropn{x} & & otherwise \\
%     \end{array}
%   \right. 
  (\dropn{x})  \psubstp{Q}{P}       
  := 
  \left\{ 
    \begin{array}{ccc} 
      Q & & x \nameeq \quotep{P} \\
      \dropn{x} & & otherwise \\
    \end{array}
  \right.
\end{mathpar}
 

where

\begin{eqnarray}
  (x)\id{\{} \lpquote Q \rpquote / \lpquote P \rpquote \id{\}}            = 
  \left\{ 
    \begin{array}{ccc}
      \lpquote Q \rpquote & & x \nameeq \lpquote P \rpquote \\
      x & & otherwise \\
    \end{array}
  \right. \nonumber
\end{eqnarray}

and $z$ is chosen distinct from $\quotep{P}$, $\quotep{Q}$, the free
names in $Q$, and all the names in $R$. Our $\alpha$-equivalence will
be built in the standard way from this substitution.

\begin{remark}\label{rem:no_self_referential_names}
  One consequence of these definitions is that $\forall P. \quotep{P}
  \not\in \freenames{P}$.
\end{remark}

\subsection{ Dynamic quote: an example }

Anticipating something of what's to come, consider applying the
substitution, $\widehat{\id{\{}u / z \id{\}}}$, to the following pair
of processes, $\lift{w}{y!(z)}$ and $w[ \lpquote y!(z) \rpquote ]$.

\begin{eqnarray}
	\lift{w}{y!(z)}\widehat{\id{\{}u / z \id{\}}}
		& = &
		\lift{w}{y!(u)} \nonumber\\
	w[ \lpquote y!(z) \rpquote ] \widehat{ \id{\{}u / z \id{\}} }
		& = &
		w[ \lpquote y!(z) \rpquote ] \nonumber
\end{eqnarray}

Because the body of the process between quotes is impervious to
substitution, we get radically different answers. In fact, by
examining the first process in an input context,
e.g. $x?(z).\lift{w}{y!(z)}$, we see that the process under the lift
operator may be shaped by prefixed inputs binding a name inside it. In
this sense, the lift operator will be seen as a way to dynamically
construct processes before reifying them as names.

Finally equipped with these standard features we can present the
dynamics of the calculus.

\subsubsection{Operational semantics} 

Finally, we introduce the computational dynamics. What marks these
algebras as distinct from other more traditionally studied algebraic
structures, e.g. vector spaces or polynomial rings, is the manner in
which dynamics is captured. In traditional structures, dynamics is typically
expressed through morphisms between such structures, as in linear maps
between vector spaces or morphisms between rings. In algebras
associated with the semantics of computation, the dynamics is
expressed as part of the algebraic structure itself, through a
reduction reduction relation typically denoted by $\red$. Below, we
give a recursive presentation of this relation for the calculus used
in the encoding.

$\red \subseteq \pi \times \pi$
$\red : \pi \to \mathcal{P}(\pi)$

\begin{mathpar}
  \inferrule* [lab=Comm] { \textsf{match}( x_{src}, x_{trgt} ) } { x_{trgt}?(y)P \; | \; x_{src}!\langle {Q} \rangle \red P\{\quotep{Q}/y}\} }
  \and \\
  \inferrule* [lab=Par] {{P} \red {P}'} {{{P} | {Q}} \red {{P}' | {Q}}}
  \and
  \inferrule* [lab=Equiv]{{{P} \scong {P}'} \andalso {{P}' \red {Q}'} \andalso {{Q}' \scong {Q}}}{{P} \red {Q}}
\end{mathpar}

\begin{eqnarray*}
  match_{\equiv} (\quotep{P},\quotep{Q}) & := & P \equiv Q \\
  match_{\dagger}(\quotep{P},\quotep{Q}) & := & \forall R. P|Q \red^{*} R => R \red^{*} 0 \\
  match_{K}(\quotep{P},\quotep{Q}) & := & K \mbox{ for some context } K
\end{eqnarray*}

$u?(x)P | u!\langle Q \rangle \red P\{\quotep{Q}/x\}$

%We write $\wred$ for $\red^*$, and $P\red$ if $\exists Q $ such that $ P \red Q$.
We write $P\red$ if $\exists Q $ such that $ P \red Q$ and $P\not\red$, otherwise.

\section{Replication}

As mentioned before, it is known that replication (and hence
recursion) can be implemented in a higher-order process algebra
\cite{SangiorgiWalker}. As our first example of calculation with the
machinery thus far presented we give the construction explicitly in
the {\rhoc}.

\begin{eqnarray}
	D_{x} & := & \prefix{x}{y}{(\binpar{\outputp{x}{y}}{@{y}})} \nonumber\\
	\bangp_{x}{P} & := & \binpar{{x}!\langle{\binpar{D_{x}}{P}}\rangle}{D_{x}} \nonumber
\end{eqnarray}

\begin{eqnarray}
	\bangp_{x}{P} & & \nonumber\\
	=
	& {x}!\langle{(\prefix{x}{y}{(\outputp{x}{y} | @{y})) | P}}\rangle 
	      | \prefix{x}{y}{(\outputp{x}{y} | @{y})} & \nonumber\\
	\red
	& (\outputp{x}{y} | @{y})\substn{\quotep{(\prefix{x}{y}{(@{y} | \outputp{x}{y})) | P}}}{y} & \nonumber\\
	=
	& \outputp{x}{\quotep{(\prefix{x}{y}{(\outputp{x}{y} | @{y})) | P}}}
	  | {(\prefix{x}{y}{(\outputp{x}{y} | @{y})) | P}} & \nonumber\\
	\red
	& \ldots & \nonumber\\
	\red^*
	& P | P | \ldots & \nonumber
\end{eqnarray}

Of course, this encoding, as an implementation, runs away, unfolding
$\bangp{P}$ eagerly. A lazier and more implementable replication
operator, restricted to input-guarded processes, may be obtained as follows.

\begin{eqnarray}
\bangp{\prefix{u}{v}{P}} 
	:= 
	\binpar{\lift{x}{\prefix{u}{v}{(\binpar{D(x)}{P})}}}{D(x)} \nonumber
\end{eqnarray}

\begin{remark}
  Note that the lazier definition still does not deal with summation
  or mixed summation (i.e. sums over input and output). The reader is
  invited to construct definitions of replication that deal with these
  features. 

  Further, the definitions are parameterized in a name, $x$. Can you,
  gentle reader, make a definition that eliminates this parameter and
  guarantees no accidental interaction between the replication
  machinery and the process being replicated -- i.e. no accidental
  sharing of names used by the process to get its work done and the
  name(s) used by the replication to effect copying. This latter
  revision of the definition of replication is crucial to obtaining
  the expected identity $!!P \sim !P$.
\end{remark}

\begin{remark}\label{rem:paradoxical_combinator}
  The reader familiar with the lambda calculus will have noticed the
  similarity between $D$ and the paradoxical combinator.

  [Ed. note: the existence of this seems to suggest we have to be more
  restrictive on the set of processes and names we admit if we are to
  support no-cloning.]
\end{remark}

\subsubsection{Bisimulation}

The computational dynamics gives rise to another kind of equivalence,
the equivalence of computational behavior. As previously mentioned
this is typically captured \emph{via} some form of bisimulation.

% The notion we use in this paper is weak barbed bisimulation
% \cite{milner91polyadicpi}.

The notion we use in this paper is derived from weak barbed
bisimulation \cite{milner91polyadicpi}. 

\begin{definition}
An \emph{observation relation}, $\downarrow_{\mathcal N}$, over a set
of names, $\mathcal N$, is the smallest relation satisfying the rules
below.

\infrule[Out-barb]{y \in {\mathcal N}, \; x \nameeq y}
		  {\outputp{x}{v} \downarrow_{\mathcal N} x}
\infrule[Par-barb]{\mbox{$P\downarrow_{\mathcal N} x$ or $Q\downarrow_{\mathcal N} x$}}
		  {\binpar{P}{Q} \downarrow_{\mathcal N} x}

We write $P \Downarrow_{\mathcal N} x$ if there is $Q$ such that 
$P \wred Q$ and $Q \downarrow_{\mathcal N} x$.
\end{definition}

\begin{definition}
%\label{def.bbisim}
An  ${\mathcal N}$-\emph{barbed bisimulation} over a set of names, ${\mathcal N}$, is a symmetric binary relation 
${\mathcal S}_{\mathcal N}$ between agents such that $P\rel{S}_{\mathcal N}Q$ implies:
\begin{enumerate}
\item If $P \red P'$ then $Q \wred Q'$ and $P'\rel{S}_{\mathcal N} Q'$.
\item If $P\downarrow_{\mathcal N} x$, then $Q\Downarrow_{\mathcal N} x$.
\end{enumerate}
$P$ is ${\mathcal N}$-barbed bisimilar to $Q$, written
$P \wbbisim_{\mathcal N} Q$, if $P \rel{S}_{\mathcal N} Q$ for some ${\mathcal N}$-barbed bisimulation ${\mathcal S}_{\mathcal N}$.
\end{definition}

$\mathcal{R} \subseteq \pi \times \pi$

$P \mathcal{R} Q => \forall P'. P \red P' \Rightarrow \exists Q'. Q \red Q', P' \mathcal{R} Q'$

$P \vdash x \Rightarrow Q \vdash x$

\begin{mathpar}
  \inferrule*[lab=Out-barb]{x \nameeq y}{{y}!\langle{Q}\rangle \vdash x}
  \and
  \inferrule*[lab=Par-barb]{\mbox{$P\vdash x$ or $Q\vdash x$}}{\binpar{P}{Q} \vdash x}
\end{mathpar}

\subsubsection{Contexts}

One of the principle advantages of computational calculi like the
$\pi$-calculus is a well-defined notion of context,
contextual-equivalence and a correlation between
contextual-equivalence and notions of bisimulation. The notion of
context allows the decomposition of a process into (sub-)process and
its syntactic environment, its context. Thus, a context may be
thought of as a process with a ``hole'' (written $\Box$) in it. The
application of a context $M$ to a process $P$, written $M[P]$, is
tantamount to filling the hole in $M$ with $P$. In this paper we do
not need the full weight of this theory, but do make use of the notion
of context in the proof the main theorem. 

\begin{mathpar}
  \inferrule* [lab=summation] {} {{M_{M},M_{N}} \bc \Box \;|\; x.M_{A} \;|\; M_{M}+M_{N}}
  \and
  \inferrule* [lab=agent] {} {{M_{A}} \bc (\vec{x})M_{P} \;| \; \clift{P_0,\ldots,M_{P},\ldots,P_N}}
  \and \\
  \inferrule* [lab=process] {} {{M_{P}} \bc M_{N} \;| \;P|M_{P} }
\end{mathpar} 

\begin{mathpar}
  \inferrule* [lab=sychronization] {} {M_{N} \bc \Box \;|\; x?M_{F} \;|\; x!M_{C}}
  \and
  \inferrule* [lab=abstraction] {} {{M_{F}} \bc (x)M_{P} }
  \and
  \inferrule* [lab=concretion] {} {{M_{C}} \bc \langle M_{P} \rangle }
  \and \\
  \inferrule* [lab=process] {} {{M_{P}} \bc M_{N} \;| \;P|M_{P} }
\end{mathpar}

\begin{definition}[contextual application] Given a context $M$, and
  process $P$, we define the \emph{contextual application}, $M[P] :=
  M\{P/\Box\}$. That is, the contextual application of M to P is the
  substitution of $P$ for $\Box$ in $M$.
\end{definition}

$\meaningof{-} : L \to \mathcal{P}(\pi)$

\begin{mathpar}
  \inferrule* [lab=collection] {} {\meaningof{true} = \pi, \and \meaningof{~E} = \pi \setminus \meaningof{E}, \and \meaningof{E_{1} \& E_{2}} = \meaningof{E_{1}} \cap \meaningof{E_{2}}}
\end{mathpar}

\begin{mathpar}
  \inferrule* [lab=structure] {} {\meaningof{0} = \{ P \in \pi | P \equiv 0 \}, \and \\ \meaningof{E_1 | E_2} = \{ P \in \pi | P \equiv P_{1} | P_{2}, P_{1} \in \meaningof{E_{1}}, P_{2} \in \meaningof{E_2}\} }
\end{mathpar}

\begin{mathpar}
 \inferrule* [lab=behavior] {} {\meaningof{\langle a?b \rangle E} = \{ P \in \pi | P \equiv Q | u?(y)P', \\ \and \\\\ \and \\ \;\;\; u \in \meaningof{a}, \forall z.P'\{z/y\} \in \meaningof{E\{z/b\}}\}, \and \\ \meaningof{a!E} = \{ P \in \pi | P \equiv Q | x!\langle P' \rangle, x \in \meaningof{a} P' \in \meaningof{E}\} }
\end{mathpar}

\begin{mathpar}
 \inferrule* [lab=nominal] {} {\meaningof{\quotep{E}} = \{ \quotep{P} \in \quotep{\pi} | P \in \meaningof{E} \}, \and \meaningof{\quotep{P}} = \{ \quotep{Q} \in \quotep{\pi} | P \equiv Q \} \and \\ \meaningof{@\quotep{E}} = \{ P \in \pi | P \equiv @x, x \in \meaningof{E} \}}
\end{mathpar}

\begin{eqnarray*}
  \\
  \meaningof{-} : TS \to ST
\end{eqnarray*}

\begin{eqnarray*}
  \\
  L : TS \to ST
\end{eqnarray*}

\begin{eqnarray*}
  \\
  P \models E \iff P \in \meaningof{E}
\end{eqnarray*}

\begin{eqnarray*}
  P \approx_{L} Q \iff \forall E \in L. P \models E \iff Q \models E
\end{eqnarray*}

\begin{eqnarray*}
  P \approx_{K} Q
\end{eqnarray*}

\begin{eqnarray*}
  P \approx Q
\end{eqnarray*}

$\approx_{K} = \approx = \approx_{L}$

\subsubsection{Contextual duality}

Note that contexts extend the quotation operation to a family of
operations from processes to names. Given a context, $M$, we can
define a \emph{nominal context}, $\quotep{M}$ by $\quotep{M}[P] :=
\quotep{M[P]}$. To foreshadow what is to come we observe that these
operations enjoy a duality with processes very much like the duality
between vectors and maps from vectors to scalars.

Further, because the calculus is essentially higher-order, we have a
correspondence between contexts and processes. More specifically,
given a name $x$ and a context $M$ we can construct $M^{*}_{x}$ such
that 

\begin{mathpar}
  M^{*}_{x} | \lift{x}{P} \red M[P]
\end{mathpar}

namely,

\begin{mathpar}
  M^{*}_{x} := x?(u).M[\dropn{u}]
\end{mathpar}

The dependence of $M^{*}_{x}$ on a name makes it an abstraction, 

\begin{mathpar}
  M^{*} := (x)x?(u).M[\dropn{u}]
\end{mathpar}

\subsection{Additional notation}

It will sometimes be convenient to denote the process a name
quotes. We already have the notation $x = \quotep{P}$, but it will be
convenient to introduce an alternate notation, $\procn{x}$, when we
want to emphasize the connection to the use of the name. Note that, by
virtue of name equivalence, $\quotep{\procn{x}} \nameeq x$; so, the
notation is consistent with previous definitions.

Further, because names have structure it is possible to effect
substitutions on the basis of that structure. This means we need to
upgrade our notation for substitutions, which we accomplish by
adapting comprehension notation. Thus,

\begin{mathpar}
  P\{ y / x : x \in S \}
\end{mathpar}

is interpreted to mean the process derived from P by replacing (in a
capture-avoiding manner) each occurrence of $x$ in $S$ by $y$. For example,

\begin{mathpar}
  P\{ \quotep{\procn{x}|\procn{x}} / x : x \in \freenames{P} \}
\end{mathpar}

will replace each (occurrence) of a free name $x$ in $P$ by
$\quotep{\procn{x}|\procn{x}}$.

Also, we will avail ourselves of the notation $x^{L}$ and $x^{R}$ to
denote injections of a name into disjoint copies of the name
space. There are numerous ways to accomplish this. One example can be
found in \cite{MeredithR05}. This notation overloads to vectors of
names: $\vec{x}^{\pi} := (x_{i}^{\pi} \; : \; 0 \leq i < |\vec{x}| )$ where $\pi \in \{L,R\}$.

We also use $P^{\Box} := P|\Box$.

In \cite{MeredithR05} an interpretation of the new operator is
given. It turns out that there are several possible interpretations
all enjoying the requisite algebraic properties of the operator (see
\cite{milner91polyadicpi}). We will therefore make liberal use of
$(\nu\; \vec{x})P$.

% subsection the_syntax_and_semantics_of_the_notation_system (end)   

\input{qm2pi.qmops} 

\input{qm2pi.sterngerlach} 

\input{qm2pi.metric} 

% section concurrent_process_calculi (end)

%\input{qm2pi.proofsketch}

% section proof sketch (end)

%\input{qm2pi.slviaknots} 

% section spatial logic via knots (end)

\input{qm2pi.conclusion}

% section conclusion (end)

%\input{qm2pi.dtcodes} 

% section wiring algorithm (end)

\input{qm2pi.ack} 

% section acknowledgments (end)

\newpage


\bibliographystyle{plain}   
\bibliography{../../biblios/main.bib}

\input{qm2pi.rhodetails}

\end{document}

 

%\ifpdf
%\usepackage[pdftex]{graphicx}
%\else
%\usepackage{graphicx}
%\fi

 % \ifpdf
%  \usepackage{pdfsync}
%  \if


%\title{Brief Article}
%\author{David F. Snyder}
%\author{L.G. Meredith}

%\address{Dept. of Math., Texas State University--San Marcos, San Marcos, TX 78666}
       
\pagestyle{empty}


\begin{document}

\lstset{language=[Objective]Caml,frame=shadowbox}

\documentclass[12pt]{llncs}
%\documentclass{jktr}

\usepackage[pdftex]{hyperref}                   
\usepackage {listings}
\usepackage {mathpartir}
\usepackage{bcprules}
%\usepackage{listings}
                       
\usepackage{graphicx} 
%\usepackage[margins=2.5cm,nohead,nofoot]{geometry}
%\usepackage{geometry}
\usepackage{amsfonts}
\usepackage{amstext}
\usepackage{latexsym}
\usepackage{amssymb}
\usepackage{color}


%\include{myPreamble}
\include{qm2pi.local} 

%\ifpdf
%\usepackage[pdftex]{graphicx}
%\else
%\usepackage{graphicx}
%\fi

 % \ifpdf
%  \usepackage{pdfsync}
%  \if


%\title{Brief Article}
%\author{David F. Snyder}
%\author{L.G. Meredith}

%\address{Dept. of Math., Texas State University--San Marcos, San Marcos, TX 78666}
       
\pagestyle{empty}


\begin{document}

\lstset{language=[Objective]Caml,frame=shadowbox}

\input{qm2pi.front}

% section front matter (end)

\input{qm2pi.intro} 
 
% section introduction (end)

% \input{qm2pi.knotations} 

% section notation (end)

\input{qm2pi.process.calculi} 

% section concurrent_process_calculi_and_spatial_logics_ (end)
    
%\input{qm2pi.knots2pi} 

%\input{qm2pi.trefoil} 

%\input{qm2pi.mainthm} 

% subsection basic_interpretation (end)

%\input{qm2pi.rho.presentation} 
\subsection{The syntax and semantics of the notation system}\label{sub:the_syntax_and_semantics_of_the_notation_system} % (fold)

We now summarize a technical presentation of the calculus that
embodies our theory of dynamics. The typical presentation of such a
calculus follows the style of giving generators and relations on
them. The grammar, below, describing term constructors, freely
generates the set of processes, $\Proc$. This set is then quotiented
by a relation known as structural congruence and it is over this set
that the notion of dynamics is expressed. This presentation is
essentially that of \cite{MeredithR05} with the addition of
polyadicity and summation. For readability we have relegated some of
the technical subtleties to an appendix.

\subsubsection{Process grammar}\label{subsub:process_grammar}

\begin{mathpar}
  \inferrule* [lab=synchronization] {} {{M} \bc \pzero \;|\; x?F \;|\; x!C }
  \and
  \inferrule* [lab=abstraction] {} {{F} \bc (x)P}
  \and
  \inferrule* [lab=concretion] {} {{C} \bc \langle Q \rangle}
  \and
  \inferrule* [lab=process] {} {{P,Q} \bc M \;| \;P|Q \;|\; @{x}}
  \and
  \inferrule* [lab=name] {} {{x} \bc \quotep{P}}
\end{mathpar} 

Note that $\vec{x}$ (resp. $\vec{P}$) denotes a vector of names
(resp. processes) of length $|\vec{x}|$ (resp. $|\vec{P}|$). We adopt
the following useful abbreviations.

\begin{mathpar}
   x?(\vec{y}).P := x.(\vec{y})P \and  x\clift{\vec{P}} := x.\clift{\vec{P}}
   \and x!(y) := \lift{x}{\dropn{y}}
   \and \Pi_{i=0}^{n-1}P_i := P_0 | \ldots | P_{n-1}
\end{mathpar}

\subsubsection{Structural congruence}

\paragraph{Free and bound names and alpha-equivalence.} At the
core of structural equivalence is alpha-equivalence which identifies
process that are the same up to a change of variable. Formally, we
recognize the distinction between free and bound names. The free names
of a process, $\freenames{P}$, may be calculated recursively as
follows:

\begin{mathpar}
\freenames{\pzero} := \emptyset
  \and \\
  \freenames{x?(y).P} := \{ x \} \cup (\freenames{P} \setminus \{ y \})
  \and 
  \freenames{x!\langle P \rangle} := \{ x \} \cup \{ P \} 
  \and \\
  \freenames{P|Q} := \freenames{P} \cup \freenames{Q}
  \and \\
  \freenames{@{x}} := \{ x \}
\end{mathpar}

$\pi$
$\quotep{\pi}$

$\freenames{-} : \pi \to \mathcal{P}(\quotep{\pi})$

\begin{eqnarray*}
  \freenames{\pzero} & := & \emptyset \\
  \freenames{x?(y).P} & := & \{ x \} \cup (\freenames{P} \setminus \{ y \}) \\
  \freenames{x!\langle P \rangle} & := & \{ x \} \cup \{ P \} \\
  \freenames{P|Q} & := & \freenames{P} \cup \freenames{Q} \\
  \freenames{\dropn{x}} & := & \{ x \}
\end{eqnarray*}

The bound names of a process, $\boundnames{P}$, are those names occurring in $P$
that are not free. For example, in $x?(y).0$, the name $x$ is free, while $y$ is bound.

\begin{mathpar}
  \inferrule* [lab=monoidal-laws] {} { P|Q \equiv Q|P \and P|0 \equiv P \and P|(Q|R) \equiv (P|Q)|R }
\end{mathpar}

\begin{mathpar}
  \inferrule* [lab=alpha-equivalence] {} { (x)P \equiv (y)P\{y/x\} \and y \not\in \freenames{P} }
\end{mathpar}

\begin{definition}
Then two processes, $P,Q$, are alpha-equivalent if $P = Q\{\vec{y}/\vec{x}\}$ for
some $\vec{x} \in \boundnames{Q},\vec{y} \in \boundnames{P}$, where $Q\{\vec{y}/\vec{x}\}$
denotes the capture-avoiding substitution of $\vec{y}$ for $\vec{x}$ in $Q$.
\end{definition}

\begin{definition}
  The {\em structural congruence} \cite{SangiorgiWalker} , $\equiv$,
  between processes is the least congruence containing
  alpha-equivalence, satisfying the abelian monoid laws
  (associativity, commutativity and $\pzero$ as identity) for parallel
  composition $|$ and for summation $+$.
\end{definition}

\subsection{Name equivalence}

We take name equivalence, written $\nameeq$, to be the smallest
equivalence relation generated by the following rules.

\begin{mathpar}
\inferrule*[lab=Quote-drop]
{ }
{ \quotep{@{x}} \nameeq x }

\inferrule*[lab=Struct-equiv]
{ P \scong Q }
{ \quotep{P} \nameeq \quotep{Q} }
\end{mathpar}

The astute reader will have noticed that the mutual recursion of names
and processes imposes a mutual recursion on alpha-equivalence and
structural equivalence via name-equivalence. Fortunately, all of this
works out pleasantly and we may calculate in the natural way, free of
concern. The reader interested in the details is referred to the
appendix \ref{appendix:rho_details}.

\subsection{Substitution}

We use $\Proc$ for the set of processes, $\QProc$ for the set of
names, and $\id{\{}\vec{y} / \vec{x} \id{\}}$ to denote partial maps,
$s : \QProc \rightarrow \QProc$. A map, $s$ lifts, uniquely, to a map
on process terms, $\widehat{s} : \Proc \rightarrow \Proc$ by the
following equations.

\begin{mathpar}
  (0) \psubstp{Q}{P} := 0 \\
  (R \juxtap S) \psubstp{Q}{P}
  :=    
  (R)\psubstp{Q}{P} \juxtap (S) \psubstp{Q}{P} \\
  (x?(y).R) \psubstp{Q}{P}    
  :=    
  (x)\substp{Q}{P} (z)\concat( (R \psubstn{z}{y}) \psubstp{Q}{P} ) \\
  (\lift{x}{R}) \psubstp{Q}{P}  
  :=
  \lift{(x)\substp{Q}{P}}{ R \psubstp{Q}{P} } \\
%   (\dropn{x})  \psubstp{Q}{P}       
%   := 
%   \left\{ 
%     \begin{array}{ccc} 
%       \dropn{\quotep{Q}} & & x \nameeq \quotep{P} \\
%       \dropn{x} & & otherwise \\
%     \end{array}
%   \right. 
  (\dropn{x})  \psubstp{Q}{P}       
  := 
  \left\{ 
    \begin{array}{ccc} 
      Q & & x \nameeq \quotep{P} \\
      \dropn{x} & & otherwise \\
    \end{array}
  \right.
\end{mathpar}
 

where

\begin{eqnarray}
  (x)\id{\{} \lpquote Q \rpquote / \lpquote P \rpquote \id{\}}            = 
  \left\{ 
    \begin{array}{ccc}
      \lpquote Q \rpquote & & x \nameeq \lpquote P \rpquote \\
      x & & otherwise \\
    \end{array}
  \right. \nonumber
\end{eqnarray}

and $z$ is chosen distinct from $\quotep{P}$, $\quotep{Q}$, the free
names in $Q$, and all the names in $R$. Our $\alpha$-equivalence will
be built in the standard way from this substitution.

\begin{remark}\label{rem:no_self_referential_names}
  One consequence of these definitions is that $\forall P. \quotep{P}
  \not\in \freenames{P}$.
\end{remark}

\subsection{ Dynamic quote: an example }

Anticipating something of what's to come, consider applying the
substitution, $\widehat{\id{\{}u / z \id{\}}}$, to the following pair
of processes, $\lift{w}{y!(z)}$ and $w[ \lpquote y!(z) \rpquote ]$.

\begin{eqnarray}
	\lift{w}{y!(z)}\widehat{\id{\{}u / z \id{\}}}
		& = &
		\lift{w}{y!(u)} \nonumber\\
	w[ \lpquote y!(z) \rpquote ] \widehat{ \id{\{}u / z \id{\}} }
		& = &
		w[ \lpquote y!(z) \rpquote ] \nonumber
\end{eqnarray}

Because the body of the process between quotes is impervious to
substitution, we get radically different answers. In fact, by
examining the first process in an input context,
e.g. $x?(z).\lift{w}{y!(z)}$, we see that the process under the lift
operator may be shaped by prefixed inputs binding a name inside it. In
this sense, the lift operator will be seen as a way to dynamically
construct processes before reifying them as names.

Finally equipped with these standard features we can present the
dynamics of the calculus.

\subsubsection{Operational semantics} 

Finally, we introduce the computational dynamics. What marks these
algebras as distinct from other more traditionally studied algebraic
structures, e.g. vector spaces or polynomial rings, is the manner in
which dynamics is captured. In traditional structures, dynamics is typically
expressed through morphisms between such structures, as in linear maps
between vector spaces or morphisms between rings. In algebras
associated with the semantics of computation, the dynamics is
expressed as part of the algebraic structure itself, through a
reduction reduction relation typically denoted by $\red$. Below, we
give a recursive presentation of this relation for the calculus used
in the encoding.

$\red \subseteq \pi \times \pi$
$\red : \pi \to \mathcal{P}(\pi)$

\begin{mathpar}
  \inferrule* [lab=Comm] { \textsf{match}( x_{src}, x_{trgt} ) } { x_{trgt}?(y)P \; | \; x_{src}!\langle {Q} \rangle \red P\{\quotep{Q}/y}\} }
  \and \\
  \inferrule* [lab=Par] {{P} \red {P}'} {{{P} | {Q}} \red {{P}' | {Q}}}
  \and
  \inferrule* [lab=Equiv]{{{P} \scong {P}'} \andalso {{P}' \red {Q}'} \andalso {{Q}' \scong {Q}}}{{P} \red {Q}}
\end{mathpar}

\begin{eqnarray*}
  match_{\equiv} (\quotep{P},\quotep{Q}) & := & P \equiv Q \\
  match_{\dagger}(\quotep{P},\quotep{Q}) & := & \forall R. P|Q \red^{*} R => R \red^{*} 0 \\
  match_{K}(\quotep{P},\quotep{Q}) & := & K \mbox{ for some context } K
\end{eqnarray*}

$u?(x)P | u!\langle Q \rangle \red P\{\quotep{Q}/x\}$

%We write $\wred$ for $\red^*$, and $P\red$ if $\exists Q $ such that $ P \red Q$.
We write $P\red$ if $\exists Q $ such that $ P \red Q$ and $P\not\red$, otherwise.

\section{Replication}

As mentioned before, it is known that replication (and hence
recursion) can be implemented in a higher-order process algebra
\cite{SangiorgiWalker}. As our first example of calculation with the
machinery thus far presented we give the construction explicitly in
the {\rhoc}.

\begin{eqnarray}
	D_{x} & := & \prefix{x}{y}{(\binpar{\outputp{x}{y}}{@{y}})} \nonumber\\
	\bangp_{x}{P} & := & \binpar{{x}!\langle{\binpar{D_{x}}{P}}\rangle}{D_{x}} \nonumber
\end{eqnarray}

\begin{eqnarray}
	\bangp_{x}{P} & & \nonumber\\
	=
	& {x}!\langle{(\prefix{x}{y}{(\outputp{x}{y} | @{y})) | P}}\rangle 
	      | \prefix{x}{y}{(\outputp{x}{y} | @{y})} & \nonumber\\
	\red
	& (\outputp{x}{y} | @{y})\substn{\quotep{(\prefix{x}{y}{(@{y} | \outputp{x}{y})) | P}}}{y} & \nonumber\\
	=
	& \outputp{x}{\quotep{(\prefix{x}{y}{(\outputp{x}{y} | @{y})) | P}}}
	  | {(\prefix{x}{y}{(\outputp{x}{y} | @{y})) | P}} & \nonumber\\
	\red
	& \ldots & \nonumber\\
	\red^*
	& P | P | \ldots & \nonumber
\end{eqnarray}

Of course, this encoding, as an implementation, runs away, unfolding
$\bangp{P}$ eagerly. A lazier and more implementable replication
operator, restricted to input-guarded processes, may be obtained as follows.

\begin{eqnarray}
\bangp{\prefix{u}{v}{P}} 
	:= 
	\binpar{\lift{x}{\prefix{u}{v}{(\binpar{D(x)}{P})}}}{D(x)} \nonumber
\end{eqnarray}

\begin{remark}
  Note that the lazier definition still does not deal with summation
  or mixed summation (i.e. sums over input and output). The reader is
  invited to construct definitions of replication that deal with these
  features. 

  Further, the definitions are parameterized in a name, $x$. Can you,
  gentle reader, make a definition that eliminates this parameter and
  guarantees no accidental interaction between the replication
  machinery and the process being replicated -- i.e. no accidental
  sharing of names used by the process to get its work done and the
  name(s) used by the replication to effect copying. This latter
  revision of the definition of replication is crucial to obtaining
  the expected identity $!!P \sim !P$.
\end{remark}

\begin{remark}\label{rem:paradoxical_combinator}
  The reader familiar with the lambda calculus will have noticed the
  similarity between $D$ and the paradoxical combinator.

  [Ed. note: the existence of this seems to suggest we have to be more
  restrictive on the set of processes and names we admit if we are to
  support no-cloning.]
\end{remark}

\subsubsection{Bisimulation}

The computational dynamics gives rise to another kind of equivalence,
the equivalence of computational behavior. As previously mentioned
this is typically captured \emph{via} some form of bisimulation.

% The notion we use in this paper is weak barbed bisimulation
% \cite{milner91polyadicpi}.

The notion we use in this paper is derived from weak barbed
bisimulation \cite{milner91polyadicpi}. 

\begin{definition}
An \emph{observation relation}, $\downarrow_{\mathcal N}$, over a set
of names, $\mathcal N$, is the smallest relation satisfying the rules
below.

\infrule[Out-barb]{y \in {\mathcal N}, \; x \nameeq y}
		  {\outputp{x}{v} \downarrow_{\mathcal N} x}
\infrule[Par-barb]{\mbox{$P\downarrow_{\mathcal N} x$ or $Q\downarrow_{\mathcal N} x$}}
		  {\binpar{P}{Q} \downarrow_{\mathcal N} x}

We write $P \Downarrow_{\mathcal N} x$ if there is $Q$ such that 
$P \wred Q$ and $Q \downarrow_{\mathcal N} x$.
\end{definition}

\begin{definition}
%\label{def.bbisim}
An  ${\mathcal N}$-\emph{barbed bisimulation} over a set of names, ${\mathcal N}$, is a symmetric binary relation 
${\mathcal S}_{\mathcal N}$ between agents such that $P\rel{S}_{\mathcal N}Q$ implies:
\begin{enumerate}
\item If $P \red P'$ then $Q \wred Q'$ and $P'\rel{S}_{\mathcal N} Q'$.
\item If $P\downarrow_{\mathcal N} x$, then $Q\Downarrow_{\mathcal N} x$.
\end{enumerate}
$P$ is ${\mathcal N}$-barbed bisimilar to $Q$, written
$P \wbbisim_{\mathcal N} Q$, if $P \rel{S}_{\mathcal N} Q$ for some ${\mathcal N}$-barbed bisimulation ${\mathcal S}_{\mathcal N}$.
\end{definition}

$\mathcal{R} \subseteq \pi \times \pi$

$P \mathcal{R} Q => \forall P'. P \red P' \Rightarrow \exists Q'. Q \red Q', P' \mathcal{R} Q'$

$P \vdash x \Rightarrow Q \vdash x$

\begin{mathpar}
  \inferrule*[lab=Out-barb]{x \nameeq y}{{y}!\langle{Q}\rangle \vdash x}
  \and
  \inferrule*[lab=Par-barb]{\mbox{$P\vdash x$ or $Q\vdash x$}}{\binpar{P}{Q} \vdash x}
\end{mathpar}

\subsubsection{Contexts}

One of the principle advantages of computational calculi like the
$\pi$-calculus is a well-defined notion of context,
contextual-equivalence and a correlation between
contextual-equivalence and notions of bisimulation. The notion of
context allows the decomposition of a process into (sub-)process and
its syntactic environment, its context. Thus, a context may be
thought of as a process with a ``hole'' (written $\Box$) in it. The
application of a context $M$ to a process $P$, written $M[P]$, is
tantamount to filling the hole in $M$ with $P$. In this paper we do
not need the full weight of this theory, but do make use of the notion
of context in the proof the main theorem. 

\begin{mathpar}
  \inferrule* [lab=summation] {} {{M_{M},M_{N}} \bc \Box \;|\; x.M_{A} \;|\; M_{M}+M_{N}}
  \and
  \inferrule* [lab=agent] {} {{M_{A}} \bc (\vec{x})M_{P} \;| \; \clift{P_0,\ldots,M_{P},\ldots,P_N}}
  \and \\
  \inferrule* [lab=process] {} {{M_{P}} \bc M_{N} \;| \;P|M_{P} }
\end{mathpar} 

\begin{mathpar}
  \inferrule* [lab=sychronization] {} {M_{N} \bc \Box \;|\; x?M_{F} \;|\; x!M_{C}}
  \and
  \inferrule* [lab=abstraction] {} {{M_{F}} \bc (x)M_{P} }
  \and
  \inferrule* [lab=concretion] {} {{M_{C}} \bc \langle M_{P} \rangle }
  \and \\
  \inferrule* [lab=process] {} {{M_{P}} \bc M_{N} \;| \;P|M_{P} }
\end{mathpar}

\begin{definition}[contextual application] Given a context $M$, and
  process $P$, we define the \emph{contextual application}, $M[P] :=
  M\{P/\Box\}$. That is, the contextual application of M to P is the
  substitution of $P$ for $\Box$ in $M$.
\end{definition}

$\meaningof{-} : L \to \mathcal{P}(\pi)$

\begin{mathpar}
  \inferrule* [lab=collection] {} {\meaningof{true} = \pi, \and \meaningof{~E} = \pi \setminus \meaningof{E}, \and \meaningof{E_{1} \& E_{2}} = \meaningof{E_{1}} \cap \meaningof{E_{2}}}
\end{mathpar}

\begin{mathpar}
  \inferrule* [lab=structure] {} {\meaningof{0} = \{ P \in \pi | P \equiv 0 \}, \and \\ \meaningof{E_1 | E_2} = \{ P \in \pi | P \equiv P_{1} | P_{2}, P_{1} \in \meaningof{E_{1}}, P_{2} \in \meaningof{E_2}\} }
\end{mathpar}

\begin{mathpar}
 \inferrule* [lab=behavior] {} {\meaningof{\langle a?b \rangle E} = \{ P \in \pi | P \equiv Q | u?(y)P', \\ \and \\\\ \and \\ \;\;\; u \in \meaningof{a}, \forall z.P'\{z/y\} \in \meaningof{E\{z/b\}}\}, \and \\ \meaningof{a!E} = \{ P \in \pi | P \equiv Q | x!\langle P' \rangle, x \in \meaningof{a} P' \in \meaningof{E}\} }
\end{mathpar}

\begin{mathpar}
 \inferrule* [lab=nominal] {} {\meaningof{\quotep{E}} = \{ \quotep{P} \in \quotep{\pi} | P \in \meaningof{E} \}, \and \meaningof{\quotep{P}} = \{ \quotep{Q} \in \quotep{\pi} | P \equiv Q \} \and \\ \meaningof{@\quotep{E}} = \{ P \in \pi | P \equiv @x, x \in \meaningof{E} \}}
\end{mathpar}

\begin{eqnarray*}
  \\
  \meaningof{-} : TS \to ST
\end{eqnarray*}

\begin{eqnarray*}
  \\
  L : TS \to ST
\end{eqnarray*}

\begin{eqnarray*}
  \\
  P \models E \iff P \in \meaningof{E}
\end{eqnarray*}

\begin{eqnarray*}
  P \approx_{L} Q \iff \forall E \in L. P \models E \iff Q \models E
\end{eqnarray*}

\begin{eqnarray*}
  P \approx_{K} Q
\end{eqnarray*}

\begin{eqnarray*}
  P \approx Q
\end{eqnarray*}

$\approx_{K} = \approx = \approx_{L}$

\subsubsection{Contextual duality}

Note that contexts extend the quotation operation to a family of
operations from processes to names. Given a context, $M$, we can
define a \emph{nominal context}, $\quotep{M}$ by $\quotep{M}[P] :=
\quotep{M[P]}$. To foreshadow what is to come we observe that these
operations enjoy a duality with processes very much like the duality
between vectors and maps from vectors to scalars.

Further, because the calculus is essentially higher-order, we have a
correspondence between contexts and processes. More specifically,
given a name $x$ and a context $M$ we can construct $M^{*}_{x}$ such
that 

\begin{mathpar}
  M^{*}_{x} | \lift{x}{P} \red M[P]
\end{mathpar}

namely,

\begin{mathpar}
  M^{*}_{x} := x?(u).M[\dropn{u}]
\end{mathpar}

The dependence of $M^{*}_{x}$ on a name makes it an abstraction, 

\begin{mathpar}
  M^{*} := (x)x?(u).M[\dropn{u}]
\end{mathpar}

\subsection{Additional notation}

It will sometimes be convenient to denote the process a name
quotes. We already have the notation $x = \quotep{P}$, but it will be
convenient to introduce an alternate notation, $\procn{x}$, when we
want to emphasize the connection to the use of the name. Note that, by
virtue of name equivalence, $\quotep{\procn{x}} \nameeq x$; so, the
notation is consistent with previous definitions.

Further, because names have structure it is possible to effect
substitutions on the basis of that structure. This means we need to
upgrade our notation for substitutions, which we accomplish by
adapting comprehension notation. Thus,

\begin{mathpar}
  P\{ y / x : x \in S \}
\end{mathpar}

is interpreted to mean the process derived from P by replacing (in a
capture-avoiding manner) each occurrence of $x$ in $S$ by $y$. For example,

\begin{mathpar}
  P\{ \quotep{\procn{x}|\procn{x}} / x : x \in \freenames{P} \}
\end{mathpar}

will replace each (occurrence) of a free name $x$ in $P$ by
$\quotep{\procn{x}|\procn{x}}$.

Also, we will avail ourselves of the notation $x^{L}$ and $x^{R}$ to
denote injections of a name into disjoint copies of the name
space. There are numerous ways to accomplish this. One example can be
found in \cite{MeredithR05}. This notation overloads to vectors of
names: $\vec{x}^{\pi} := (x_{i}^{\pi} \; : \; 0 \leq i < |\vec{x}| )$ where $\pi \in \{L,R\}$.

We also use $P^{\Box} := P|\Box$.

In \cite{MeredithR05} an interpretation of the new operator is
given. It turns out that there are several possible interpretations
all enjoying the requisite algebraic properties of the operator (see
\cite{milner91polyadicpi}). We will therefore make liberal use of
$(\nu\; \vec{x})P$.

% subsection the_syntax_and_semantics_of_the_notation_system (end)   

\input{qm2pi.qmops} 

\input{qm2pi.sterngerlach} 

\input{qm2pi.metric} 

% section concurrent_process_calculi (end)

%\input{qm2pi.proofsketch}

% section proof sketch (end)

%\input{qm2pi.slviaknots} 

% section spatial logic via knots (end)

\input{qm2pi.conclusion}

% section conclusion (end)

%\input{qm2pi.dtcodes} 

% section wiring algorithm (end)

\input{qm2pi.ack} 

% section acknowledgments (end)

\newpage


\bibliographystyle{plain}   
\bibliography{../../biblios/main.bib}

\input{qm2pi.rhodetails}

\end{document}



% section front matter (end)

\section{Introduction}\label{sec:introduction} % (fold)
In this draft of the material i am going to have to dispense with the
usual writing conventions adopted in papers on these topics. i'm going
to have adopt whatever tone i need at the time i'm writing up the
calculations. Sometimes this may be very conversational; others it may
be the barest mathematical grunts; others still it may be that i have
lifted text from one of my other papers because the exposition of some
point was better said there. i hope that my readers are not unduly put
out by this decision. i'm not doing this to flout convention or be
rebellious. i find these calculations very technically challenging. To
keep everything going technically, something has to give; i have to
let go of some cognitive burden. So, the academic writing style --
with all of its trade-offs in terms of facilitating technical
communication -- is what i'm letting go of. Perhaps subsequent drafts
can be tightened and polished, but for now, i'm going to speak as if
we were sitting together in a coffee shop with a laptop, wifi and a
pad of paper and a pencil.

So, here's what i have to say. We -- you and i, comfortably ensconced
in our coffee shop and well-equipped with our tools -- can realize and
carry out the calculations of quantum mechanics over a very different
formal theory of dynamics, a formal theory of dynamics that
corresponds to a theory of concurrent computation with
\emph{reflection}. It has the advantage that the underlying theory is
already `quantized', but supports analogues all of the continuuous
operations. Strikingly, this underlying theory has recently been
connected with a notion of metric that we can show, by calculating
together, coincides with the metric induced by the inner product.

There are a lot of reasons why you might be interested in seeing
calculations of this form. Here's why i'm interested. For the past
several centuries there has been no competitor to the ``Newtonian''
account of dynamics. As a result the predominant share of accounts of
dynamical systems and situations have had to be formulated in terms of
the Newtonian machinery. i view this as an intellectually dangerous
position to occupy. Everything, despite it's intrinsic shape, turns
into a nail to be hit with this hammer. Recently, however, the theory
of computation has matured to the point where we have candidates for
theories of dynamics that offer very different perspective on
reasoning about dynamical systems and situations. Testing these
candidates against very successful accounts of dynamical situations,
like quantum mechanics, is going to give us some sense of how mature
they are and some measure of the quality of these accounts of
dynamics.

\subsection{Summary of contributions and outline of paper}

So, we're going to develop an interpretation of the operations of
quantum mechanics normally interpreted by Hilbert spaces and
operators. We're going to do this over a theory of computation. Note
that this is very different than the usual quantum computation program
which develops notions of computation over quantum mechanics. Rather,
we are developing a story that aligns with Wheeler's slogan: It from
Bit. To do this we will first provide an account of the theory of
computation at play here. Then we will dive into a calculation-driven
interpretation of the operations of quantum mechanics.

The reason we take this approach is that -- until very recently --
there hasn't been an axiomatic account of quantum mechanics. As a
result there has been no sharp delineation of the mathematical theory
supporting interpretation of the physical theory and the physical
theory, itself. So, ambient features of the maths are free to be
exploited (or supressed) without a real accounting of their physical
relevance. There is no sharp statement ``here's the physical theory''
qua \emph{theory} and ``here's the mathematical interpretation''
enabling a judgment of how faithful the interpretation is -- apart
from experimental observation. When there is an axiomatic account we
can judge how well a given mathematical formalism supports an
interpretation of the axioms, independent of
experimentation. Likewise, we can judge how well we have captured our
physical evidence and experience with our axiomatics, independent of
any specific mathematical implementation, with accidental detail that
may or may not have physical significance. 

In lieu of a fully fleshed out and vetted axiomatic account of quantum
mechanics, interpreting the operational notions in service of modeling
physical systems will have to suffice. In other words, we are not in
the business of providing a model of Hilbert spaces and operators. We
are in the business of providing a model of quantum mechanics because
we are motivated by testing our notions of dynamics against physical
theory; and, the predictive calculations of the physical theory must
serve as the best formulation -- shy of a fully fleshed out axiomatic
account -- of the physical theory itself (as they have for scientific
theories since time immemorial). Put another way, despite a
whole-hearted commitment to an It-from-Bit ontology, we are firmly
aligned with the shut-up-and-calculate camp as the best way to obtain
results either from the physical perspective or as a quality assurance
measure of our fledgling theory of dynamics.

In detail, we present a reflective process calculus. Then we develop
intuitive correspondences between the notions available in this
calculus and the usual physical notions supporting quantum mechanical
calculations. Thus, 

\begin{table}[htp]
  \center{
    \fbox{
      \begin{tabular}{c|c}
        quantum mechanics & process calculus \\
        \hline
        scalar & name \\
        state vector & process \\
        dual & contextual duals \\
        matrix & formal sums of process-context-dual pairs \\
        orthogonality & process annihilation \\
        inner product & execution-formula + quoting
      \end{tabular}
    }
  }
  \caption{QM - process calculi correspondences}
\end{table}

Then we tighten up these intuitions to operational definitions. We
employ the Dirac notation as the best proxy we can find for an
abstract syntax of the quantum mechanical notions. The definitions we
develop put us in contact with equational constraints coming from the
theory that we demonstrate the definitions and calculations satisfy.

This puts us in a position to shut up and calculate for the
Stern-Gerlach experimental set up, showing how these predictive
calculations become calculations on processes in our theory of a
reflective process calculus.

Penultimately, we demonstrate that the notion of metric coming from
the inner product coincides with the notion of metric available from
the theory of bisimulation. This demonstration gives us the right to
think of space as arising from behavior. Finally, we consider where we
might go from the new vantage point we have obtained.

% section introduction (end) 
 
% section introduction (end)

% \documentclass[12pt]{llncs}
%\documentclass{jktr}

\usepackage[pdftex]{hyperref}                   
\usepackage {listings}
\usepackage {mathpartir}
\usepackage{bcprules}
%\usepackage{listings}
                       
\usepackage{graphicx} 
%\usepackage[margins=2.5cm,nohead,nofoot]{geometry}
%\usepackage{geometry}
\usepackage{amsfonts}
\usepackage{amstext}
\usepackage{latexsym}
\usepackage{amssymb}
\usepackage{color}


%\include{myPreamble}
\include{qm2pi.local} 

%\ifpdf
%\usepackage[pdftex]{graphicx}
%\else
%\usepackage{graphicx}
%\fi

 % \ifpdf
%  \usepackage{pdfsync}
%  \if


%\title{Brief Article}
%\author{David F. Snyder}
%\author{L.G. Meredith}

%\address{Dept. of Math., Texas State University--San Marcos, San Marcos, TX 78666}
       
\pagestyle{empty}


\begin{document}

\lstset{language=[Objective]Caml,frame=shadowbox}

\input{qm2pi.front}

% section front matter (end)

\input{qm2pi.intro} 
 
% section introduction (end)

% \input{qm2pi.knotations} 

% section notation (end)

\input{qm2pi.process.calculi} 

% section concurrent_process_calculi_and_spatial_logics_ (end)
    
%\input{qm2pi.knots2pi} 

%\input{qm2pi.trefoil} 

%\input{qm2pi.mainthm} 

% subsection basic_interpretation (end)

%\input{qm2pi.rho.presentation} 
\subsection{The syntax and semantics of the notation system}\label{sub:the_syntax_and_semantics_of_the_notation_system} % (fold)

We now summarize a technical presentation of the calculus that
embodies our theory of dynamics. The typical presentation of such a
calculus follows the style of giving generators and relations on
them. The grammar, below, describing term constructors, freely
generates the set of processes, $\Proc$. This set is then quotiented
by a relation known as structural congruence and it is over this set
that the notion of dynamics is expressed. This presentation is
essentially that of \cite{MeredithR05} with the addition of
polyadicity and summation. For readability we have relegated some of
the technical subtleties to an appendix.

\subsubsection{Process grammar}\label{subsub:process_grammar}

\begin{mathpar}
  \inferrule* [lab=synchronization] {} {{M} \bc \pzero \;|\; x?F \;|\; x!C }
  \and
  \inferrule* [lab=abstraction] {} {{F} \bc (x)P}
  \and
  \inferrule* [lab=concretion] {} {{C} \bc \langle Q \rangle}
  \and
  \inferrule* [lab=process] {} {{P,Q} \bc M \;| \;P|Q \;|\; @{x}}
  \and
  \inferrule* [lab=name] {} {{x} \bc \quotep{P}}
\end{mathpar} 

Note that $\vec{x}$ (resp. $\vec{P}$) denotes a vector of names
(resp. processes) of length $|\vec{x}|$ (resp. $|\vec{P}|$). We adopt
the following useful abbreviations.

\begin{mathpar}
   x?(\vec{y}).P := x.(\vec{y})P \and  x\clift{\vec{P}} := x.\clift{\vec{P}}
   \and x!(y) := \lift{x}{\dropn{y}}
   \and \Pi_{i=0}^{n-1}P_i := P_0 | \ldots | P_{n-1}
\end{mathpar}

\subsubsection{Structural congruence}

\paragraph{Free and bound names and alpha-equivalence.} At the
core of structural equivalence is alpha-equivalence which identifies
process that are the same up to a change of variable. Formally, we
recognize the distinction between free and bound names. The free names
of a process, $\freenames{P}$, may be calculated recursively as
follows:

\begin{mathpar}
\freenames{\pzero} := \emptyset
  \and \\
  \freenames{x?(y).P} := \{ x \} \cup (\freenames{P} \setminus \{ y \})
  \and 
  \freenames{x!\langle P \rangle} := \{ x \} \cup \{ P \} 
  \and \\
  \freenames{P|Q} := \freenames{P} \cup \freenames{Q}
  \and \\
  \freenames{@{x}} := \{ x \}
\end{mathpar}

$\pi$
$\quotep{\pi}$

$\freenames{-} : \pi \to \mathcal{P}(\quotep{\pi})$

\begin{eqnarray*}
  \freenames{\pzero} & := & \emptyset \\
  \freenames{x?(y).P} & := & \{ x \} \cup (\freenames{P} \setminus \{ y \}) \\
  \freenames{x!\langle P \rangle} & := & \{ x \} \cup \{ P \} \\
  \freenames{P|Q} & := & \freenames{P} \cup \freenames{Q} \\
  \freenames{\dropn{x}} & := & \{ x \}
\end{eqnarray*}

The bound names of a process, $\boundnames{P}$, are those names occurring in $P$
that are not free. For example, in $x?(y).0$, the name $x$ is free, while $y$ is bound.

\begin{mathpar}
  \inferrule* [lab=monoidal-laws] {} { P|Q \equiv Q|P \and P|0 \equiv P \and P|(Q|R) \equiv (P|Q)|R }
\end{mathpar}

\begin{mathpar}
  \inferrule* [lab=alpha-equivalence] {} { (x)P \equiv (y)P\{y/x\} \and y \not\in \freenames{P} }
\end{mathpar}

\begin{definition}
Then two processes, $P,Q$, are alpha-equivalent if $P = Q\{\vec{y}/\vec{x}\}$ for
some $\vec{x} \in \boundnames{Q},\vec{y} \in \boundnames{P}$, where $Q\{\vec{y}/\vec{x}\}$
denotes the capture-avoiding substitution of $\vec{y}$ for $\vec{x}$ in $Q$.
\end{definition}

\begin{definition}
  The {\em structural congruence} \cite{SangiorgiWalker} , $\equiv$,
  between processes is the least congruence containing
  alpha-equivalence, satisfying the abelian monoid laws
  (associativity, commutativity and $\pzero$ as identity) for parallel
  composition $|$ and for summation $+$.
\end{definition}

\subsection{Name equivalence}

We take name equivalence, written $\nameeq$, to be the smallest
equivalence relation generated by the following rules.

\begin{mathpar}
\inferrule*[lab=Quote-drop]
{ }
{ \quotep{@{x}} \nameeq x }

\inferrule*[lab=Struct-equiv]
{ P \scong Q }
{ \quotep{P} \nameeq \quotep{Q} }
\end{mathpar}

The astute reader will have noticed that the mutual recursion of names
and processes imposes a mutual recursion on alpha-equivalence and
structural equivalence via name-equivalence. Fortunately, all of this
works out pleasantly and we may calculate in the natural way, free of
concern. The reader interested in the details is referred to the
appendix \ref{appendix:rho_details}.

\subsection{Substitution}

We use $\Proc$ for the set of processes, $\QProc$ for the set of
names, and $\id{\{}\vec{y} / \vec{x} \id{\}}$ to denote partial maps,
$s : \QProc \rightarrow \QProc$. A map, $s$ lifts, uniquely, to a map
on process terms, $\widehat{s} : \Proc \rightarrow \Proc$ by the
following equations.

\begin{mathpar}
  (0) \psubstp{Q}{P} := 0 \\
  (R \juxtap S) \psubstp{Q}{P}
  :=    
  (R)\psubstp{Q}{P} \juxtap (S) \psubstp{Q}{P} \\
  (x?(y).R) \psubstp{Q}{P}    
  :=    
  (x)\substp{Q}{P} (z)\concat( (R \psubstn{z}{y}) \psubstp{Q}{P} ) \\
  (\lift{x}{R}) \psubstp{Q}{P}  
  :=
  \lift{(x)\substp{Q}{P}}{ R \psubstp{Q}{P} } \\
%   (\dropn{x})  \psubstp{Q}{P}       
%   := 
%   \left\{ 
%     \begin{array}{ccc} 
%       \dropn{\quotep{Q}} & & x \nameeq \quotep{P} \\
%       \dropn{x} & & otherwise \\
%     \end{array}
%   \right. 
  (\dropn{x})  \psubstp{Q}{P}       
  := 
  \left\{ 
    \begin{array}{ccc} 
      Q & & x \nameeq \quotep{P} \\
      \dropn{x} & & otherwise \\
    \end{array}
  \right.
\end{mathpar}
 

where

\begin{eqnarray}
  (x)\id{\{} \lpquote Q \rpquote / \lpquote P \rpquote \id{\}}            = 
  \left\{ 
    \begin{array}{ccc}
      \lpquote Q \rpquote & & x \nameeq \lpquote P \rpquote \\
      x & & otherwise \\
    \end{array}
  \right. \nonumber
\end{eqnarray}

and $z$ is chosen distinct from $\quotep{P}$, $\quotep{Q}$, the free
names in $Q$, and all the names in $R$. Our $\alpha$-equivalence will
be built in the standard way from this substitution.

\begin{remark}\label{rem:no_self_referential_names}
  One consequence of these definitions is that $\forall P. \quotep{P}
  \not\in \freenames{P}$.
\end{remark}

\subsection{ Dynamic quote: an example }

Anticipating something of what's to come, consider applying the
substitution, $\widehat{\id{\{}u / z \id{\}}}$, to the following pair
of processes, $\lift{w}{y!(z)}$ and $w[ \lpquote y!(z) \rpquote ]$.

\begin{eqnarray}
	\lift{w}{y!(z)}\widehat{\id{\{}u / z \id{\}}}
		& = &
		\lift{w}{y!(u)} \nonumber\\
	w[ \lpquote y!(z) \rpquote ] \widehat{ \id{\{}u / z \id{\}} }
		& = &
		w[ \lpquote y!(z) \rpquote ] \nonumber
\end{eqnarray}

Because the body of the process between quotes is impervious to
substitution, we get radically different answers. In fact, by
examining the first process in an input context,
e.g. $x?(z).\lift{w}{y!(z)}$, we see that the process under the lift
operator may be shaped by prefixed inputs binding a name inside it. In
this sense, the lift operator will be seen as a way to dynamically
construct processes before reifying them as names.

Finally equipped with these standard features we can present the
dynamics of the calculus.

\subsubsection{Operational semantics} 

Finally, we introduce the computational dynamics. What marks these
algebras as distinct from other more traditionally studied algebraic
structures, e.g. vector spaces or polynomial rings, is the manner in
which dynamics is captured. In traditional structures, dynamics is typically
expressed through morphisms between such structures, as in linear maps
between vector spaces or morphisms between rings. In algebras
associated with the semantics of computation, the dynamics is
expressed as part of the algebraic structure itself, through a
reduction reduction relation typically denoted by $\red$. Below, we
give a recursive presentation of this relation for the calculus used
in the encoding.

$\red \subseteq \pi \times \pi$
$\red : \pi \to \mathcal{P}(\pi)$

\begin{mathpar}
  \inferrule* [lab=Comm] { \textsf{match}( x_{src}, x_{trgt} ) } { x_{trgt}?(y)P \; | \; x_{src}!\langle {Q} \rangle \red P\{\quotep{Q}/y}\} }
  \and \\
  \inferrule* [lab=Par] {{P} \red {P}'} {{{P} | {Q}} \red {{P}' | {Q}}}
  \and
  \inferrule* [lab=Equiv]{{{P} \scong {P}'} \andalso {{P}' \red {Q}'} \andalso {{Q}' \scong {Q}}}{{P} \red {Q}}
\end{mathpar}

\begin{eqnarray*}
  match_{\equiv} (\quotep{P},\quotep{Q}) & := & P \equiv Q \\
  match_{\dagger}(\quotep{P},\quotep{Q}) & := & \forall R. P|Q \red^{*} R => R \red^{*} 0 \\
  match_{K}(\quotep{P},\quotep{Q}) & := & K \mbox{ for some context } K
\end{eqnarray*}

$u?(x)P | u!\langle Q \rangle \red P\{\quotep{Q}/x\}$

%We write $\wred$ for $\red^*$, and $P\red$ if $\exists Q $ such that $ P \red Q$.
We write $P\red$ if $\exists Q $ such that $ P \red Q$ and $P\not\red$, otherwise.

\section{Replication}

As mentioned before, it is known that replication (and hence
recursion) can be implemented in a higher-order process algebra
\cite{SangiorgiWalker}. As our first example of calculation with the
machinery thus far presented we give the construction explicitly in
the {\rhoc}.

\begin{eqnarray}
	D_{x} & := & \prefix{x}{y}{(\binpar{\outputp{x}{y}}{@{y}})} \nonumber\\
	\bangp_{x}{P} & := & \binpar{{x}!\langle{\binpar{D_{x}}{P}}\rangle}{D_{x}} \nonumber
\end{eqnarray}

\begin{eqnarray}
	\bangp_{x}{P} & & \nonumber\\
	=
	& {x}!\langle{(\prefix{x}{y}{(\outputp{x}{y} | @{y})) | P}}\rangle 
	      | \prefix{x}{y}{(\outputp{x}{y} | @{y})} & \nonumber\\
	\red
	& (\outputp{x}{y} | @{y})\substn{\quotep{(\prefix{x}{y}{(@{y} | \outputp{x}{y})) | P}}}{y} & \nonumber\\
	=
	& \outputp{x}{\quotep{(\prefix{x}{y}{(\outputp{x}{y} | @{y})) | P}}}
	  | {(\prefix{x}{y}{(\outputp{x}{y} | @{y})) | P}} & \nonumber\\
	\red
	& \ldots & \nonumber\\
	\red^*
	& P | P | \ldots & \nonumber
\end{eqnarray}

Of course, this encoding, as an implementation, runs away, unfolding
$\bangp{P}$ eagerly. A lazier and more implementable replication
operator, restricted to input-guarded processes, may be obtained as follows.

\begin{eqnarray}
\bangp{\prefix{u}{v}{P}} 
	:= 
	\binpar{\lift{x}{\prefix{u}{v}{(\binpar{D(x)}{P})}}}{D(x)} \nonumber
\end{eqnarray}

\begin{remark}
  Note that the lazier definition still does not deal with summation
  or mixed summation (i.e. sums over input and output). The reader is
  invited to construct definitions of replication that deal with these
  features. 

  Further, the definitions are parameterized in a name, $x$. Can you,
  gentle reader, make a definition that eliminates this parameter and
  guarantees no accidental interaction between the replication
  machinery and the process being replicated -- i.e. no accidental
  sharing of names used by the process to get its work done and the
  name(s) used by the replication to effect copying. This latter
  revision of the definition of replication is crucial to obtaining
  the expected identity $!!P \sim !P$.
\end{remark}

\begin{remark}\label{rem:paradoxical_combinator}
  The reader familiar with the lambda calculus will have noticed the
  similarity between $D$ and the paradoxical combinator.

  [Ed. note: the existence of this seems to suggest we have to be more
  restrictive on the set of processes and names we admit if we are to
  support no-cloning.]
\end{remark}

\subsubsection{Bisimulation}

The computational dynamics gives rise to another kind of equivalence,
the equivalence of computational behavior. As previously mentioned
this is typically captured \emph{via} some form of bisimulation.

% The notion we use in this paper is weak barbed bisimulation
% \cite{milner91polyadicpi}.

The notion we use in this paper is derived from weak barbed
bisimulation \cite{milner91polyadicpi}. 

\begin{definition}
An \emph{observation relation}, $\downarrow_{\mathcal N}$, over a set
of names, $\mathcal N$, is the smallest relation satisfying the rules
below.

\infrule[Out-barb]{y \in {\mathcal N}, \; x \nameeq y}
		  {\outputp{x}{v} \downarrow_{\mathcal N} x}
\infrule[Par-barb]{\mbox{$P\downarrow_{\mathcal N} x$ or $Q\downarrow_{\mathcal N} x$}}
		  {\binpar{P}{Q} \downarrow_{\mathcal N} x}

We write $P \Downarrow_{\mathcal N} x$ if there is $Q$ such that 
$P \wred Q$ and $Q \downarrow_{\mathcal N} x$.
\end{definition}

\begin{definition}
%\label{def.bbisim}
An  ${\mathcal N}$-\emph{barbed bisimulation} over a set of names, ${\mathcal N}$, is a symmetric binary relation 
${\mathcal S}_{\mathcal N}$ between agents such that $P\rel{S}_{\mathcal N}Q$ implies:
\begin{enumerate}
\item If $P \red P'$ then $Q \wred Q'$ and $P'\rel{S}_{\mathcal N} Q'$.
\item If $P\downarrow_{\mathcal N} x$, then $Q\Downarrow_{\mathcal N} x$.
\end{enumerate}
$P$ is ${\mathcal N}$-barbed bisimilar to $Q$, written
$P \wbbisim_{\mathcal N} Q$, if $P \rel{S}_{\mathcal N} Q$ for some ${\mathcal N}$-barbed bisimulation ${\mathcal S}_{\mathcal N}$.
\end{definition}

$\mathcal{R} \subseteq \pi \times \pi$

$P \mathcal{R} Q => \forall P'. P \red P' \Rightarrow \exists Q'. Q \red Q', P' \mathcal{R} Q'$

$P \vdash x \Rightarrow Q \vdash x$

\begin{mathpar}
  \inferrule*[lab=Out-barb]{x \nameeq y}{{y}!\langle{Q}\rangle \vdash x}
  \and
  \inferrule*[lab=Par-barb]{\mbox{$P\vdash x$ or $Q\vdash x$}}{\binpar{P}{Q} \vdash x}
\end{mathpar}

\subsubsection{Contexts}

One of the principle advantages of computational calculi like the
$\pi$-calculus is a well-defined notion of context,
contextual-equivalence and a correlation between
contextual-equivalence and notions of bisimulation. The notion of
context allows the decomposition of a process into (sub-)process and
its syntactic environment, its context. Thus, a context may be
thought of as a process with a ``hole'' (written $\Box$) in it. The
application of a context $M$ to a process $P$, written $M[P]$, is
tantamount to filling the hole in $M$ with $P$. In this paper we do
not need the full weight of this theory, but do make use of the notion
of context in the proof the main theorem. 

\begin{mathpar}
  \inferrule* [lab=summation] {} {{M_{M},M_{N}} \bc \Box \;|\; x.M_{A} \;|\; M_{M}+M_{N}}
  \and
  \inferrule* [lab=agent] {} {{M_{A}} \bc (\vec{x})M_{P} \;| \; \clift{P_0,\ldots,M_{P},\ldots,P_N}}
  \and \\
  \inferrule* [lab=process] {} {{M_{P}} \bc M_{N} \;| \;P|M_{P} }
\end{mathpar} 

\begin{mathpar}
  \inferrule* [lab=sychronization] {} {M_{N} \bc \Box \;|\; x?M_{F} \;|\; x!M_{C}}
  \and
  \inferrule* [lab=abstraction] {} {{M_{F}} \bc (x)M_{P} }
  \and
  \inferrule* [lab=concretion] {} {{M_{C}} \bc \langle M_{P} \rangle }
  \and \\
  \inferrule* [lab=process] {} {{M_{P}} \bc M_{N} \;| \;P|M_{P} }
\end{mathpar}

\begin{definition}[contextual application] Given a context $M$, and
  process $P$, we define the \emph{contextual application}, $M[P] :=
  M\{P/\Box\}$. That is, the contextual application of M to P is the
  substitution of $P$ for $\Box$ in $M$.
\end{definition}

$\meaningof{-} : L \to \mathcal{P}(\pi)$

\begin{mathpar}
  \inferrule* [lab=collection] {} {\meaningof{true} = \pi, \and \meaningof{~E} = \pi \setminus \meaningof{E}, \and \meaningof{E_{1} \& E_{2}} = \meaningof{E_{1}} \cap \meaningof{E_{2}}}
\end{mathpar}

\begin{mathpar}
  \inferrule* [lab=structure] {} {\meaningof{0} = \{ P \in \pi | P \equiv 0 \}, \and \\ \meaningof{E_1 | E_2} = \{ P \in \pi | P \equiv P_{1} | P_{2}, P_{1} \in \meaningof{E_{1}}, P_{2} \in \meaningof{E_2}\} }
\end{mathpar}

\begin{mathpar}
 \inferrule* [lab=behavior] {} {\meaningof{\langle a?b \rangle E} = \{ P \in \pi | P \equiv Q | u?(y)P', \\ \and \\\\ \and \\ \;\;\; u \in \meaningof{a}, \forall z.P'\{z/y\} \in \meaningof{E\{z/b\}}\}, \and \\ \meaningof{a!E} = \{ P \in \pi | P \equiv Q | x!\langle P' \rangle, x \in \meaningof{a} P' \in \meaningof{E}\} }
\end{mathpar}

\begin{mathpar}
 \inferrule* [lab=nominal] {} {\meaningof{\quotep{E}} = \{ \quotep{P} \in \quotep{\pi} | P \in \meaningof{E} \}, \and \meaningof{\quotep{P}} = \{ \quotep{Q} \in \quotep{\pi} | P \equiv Q \} \and \\ \meaningof{@\quotep{E}} = \{ P \in \pi | P \equiv @x, x \in \meaningof{E} \}}
\end{mathpar}

\begin{eqnarray*}
  \\
  \meaningof{-} : TS \to ST
\end{eqnarray*}

\begin{eqnarray*}
  \\
  L : TS \to ST
\end{eqnarray*}

\begin{eqnarray*}
  \\
  P \models E \iff P \in \meaningof{E}
\end{eqnarray*}

\begin{eqnarray*}
  P \approx_{L} Q \iff \forall E \in L. P \models E \iff Q \models E
\end{eqnarray*}

\begin{eqnarray*}
  P \approx_{K} Q
\end{eqnarray*}

\begin{eqnarray*}
  P \approx Q
\end{eqnarray*}

$\approx_{K} = \approx = \approx_{L}$

\subsubsection{Contextual duality}

Note that contexts extend the quotation operation to a family of
operations from processes to names. Given a context, $M$, we can
define a \emph{nominal context}, $\quotep{M}$ by $\quotep{M}[P] :=
\quotep{M[P]}$. To foreshadow what is to come we observe that these
operations enjoy a duality with processes very much like the duality
between vectors and maps from vectors to scalars.

Further, because the calculus is essentially higher-order, we have a
correspondence between contexts and processes. More specifically,
given a name $x$ and a context $M$ we can construct $M^{*}_{x}$ such
that 

\begin{mathpar}
  M^{*}_{x} | \lift{x}{P} \red M[P]
\end{mathpar}

namely,

\begin{mathpar}
  M^{*}_{x} := x?(u).M[\dropn{u}]
\end{mathpar}

The dependence of $M^{*}_{x}$ on a name makes it an abstraction, 

\begin{mathpar}
  M^{*} := (x)x?(u).M[\dropn{u}]
\end{mathpar}

\subsection{Additional notation}

It will sometimes be convenient to denote the process a name
quotes. We already have the notation $x = \quotep{P}$, but it will be
convenient to introduce an alternate notation, $\procn{x}$, when we
want to emphasize the connection to the use of the name. Note that, by
virtue of name equivalence, $\quotep{\procn{x}} \nameeq x$; so, the
notation is consistent with previous definitions.

Further, because names have structure it is possible to effect
substitutions on the basis of that structure. This means we need to
upgrade our notation for substitutions, which we accomplish by
adapting comprehension notation. Thus,

\begin{mathpar}
  P\{ y / x : x \in S \}
\end{mathpar}

is interpreted to mean the process derived from P by replacing (in a
capture-avoiding manner) each occurrence of $x$ in $S$ by $y$. For example,

\begin{mathpar}
  P\{ \quotep{\procn{x}|\procn{x}} / x : x \in \freenames{P} \}
\end{mathpar}

will replace each (occurrence) of a free name $x$ in $P$ by
$\quotep{\procn{x}|\procn{x}}$.

Also, we will avail ourselves of the notation $x^{L}$ and $x^{R}$ to
denote injections of a name into disjoint copies of the name
space. There are numerous ways to accomplish this. One example can be
found in \cite{MeredithR05}. This notation overloads to vectors of
names: $\vec{x}^{\pi} := (x_{i}^{\pi} \; : \; 0 \leq i < |\vec{x}| )$ where $\pi \in \{L,R\}$.

We also use $P^{\Box} := P|\Box$.

In \cite{MeredithR05} an interpretation of the new operator is
given. It turns out that there are several possible interpretations
all enjoying the requisite algebraic properties of the operator (see
\cite{milner91polyadicpi}). We will therefore make liberal use of
$(\nu\; \vec{x})P$.

% subsection the_syntax_and_semantics_of_the_notation_system (end)   

\input{qm2pi.qmops} 

\input{qm2pi.sterngerlach} 

\input{qm2pi.metric} 

% section concurrent_process_calculi (end)

%\input{qm2pi.proofsketch}

% section proof sketch (end)

%\input{qm2pi.slviaknots} 

% section spatial logic via knots (end)

\input{qm2pi.conclusion}

% section conclusion (end)

%\input{qm2pi.dtcodes} 

% section wiring algorithm (end)

\input{qm2pi.ack} 

% section acknowledgments (end)

\newpage


\bibliographystyle{plain}   
\bibliography{../../biblios/main.bib}

\input{qm2pi.rhodetails}

\end{document}

 

% section notation (end)

\input{qm2pi.process.calculi} 

% section concurrent_process_calculi_and_spatial_logics_ (end)
    
%\documentclass[12pt]{llncs}
%\documentclass{jktr}

\usepackage[pdftex]{hyperref}                   
\usepackage {listings}
\usepackage {mathpartir}
\usepackage{bcprules}
%\usepackage{listings}
                       
\usepackage{graphicx} 
%\usepackage[margins=2.5cm,nohead,nofoot]{geometry}
%\usepackage{geometry}
\usepackage{amsfonts}
\usepackage{amstext}
\usepackage{latexsym}
\usepackage{amssymb}
\usepackage{color}


%\include{myPreamble}
\include{qm2pi.local} 

%\ifpdf
%\usepackage[pdftex]{graphicx}
%\else
%\usepackage{graphicx}
%\fi

 % \ifpdf
%  \usepackage{pdfsync}
%  \if


%\title{Brief Article}
%\author{David F. Snyder}
%\author{L.G. Meredith}

%\address{Dept. of Math., Texas State University--San Marcos, San Marcos, TX 78666}
       
\pagestyle{empty}


\begin{document}

\lstset{language=[Objective]Caml,frame=shadowbox}

\input{qm2pi.front}

% section front matter (end)

\input{qm2pi.intro} 
 
% section introduction (end)

% \input{qm2pi.knotations} 

% section notation (end)

\input{qm2pi.process.calculi} 

% section concurrent_process_calculi_and_spatial_logics_ (end)
    
%\input{qm2pi.knots2pi} 

%\input{qm2pi.trefoil} 

%\input{qm2pi.mainthm} 

% subsection basic_interpretation (end)

%\input{qm2pi.rho.presentation} 
\subsection{The syntax and semantics of the notation system}\label{sub:the_syntax_and_semantics_of_the_notation_system} % (fold)

We now summarize a technical presentation of the calculus that
embodies our theory of dynamics. The typical presentation of such a
calculus follows the style of giving generators and relations on
them. The grammar, below, describing term constructors, freely
generates the set of processes, $\Proc$. This set is then quotiented
by a relation known as structural congruence and it is over this set
that the notion of dynamics is expressed. This presentation is
essentially that of \cite{MeredithR05} with the addition of
polyadicity and summation. For readability we have relegated some of
the technical subtleties to an appendix.

\subsubsection{Process grammar}\label{subsub:process_grammar}

\begin{mathpar}
  \inferrule* [lab=synchronization] {} {{M} \bc \pzero \;|\; x?F \;|\; x!C }
  \and
  \inferrule* [lab=abstraction] {} {{F} \bc (x)P}
  \and
  \inferrule* [lab=concretion] {} {{C} \bc \langle Q \rangle}
  \and
  \inferrule* [lab=process] {} {{P,Q} \bc M \;| \;P|Q \;|\; @{x}}
  \and
  \inferrule* [lab=name] {} {{x} \bc \quotep{P}}
\end{mathpar} 

Note that $\vec{x}$ (resp. $\vec{P}$) denotes a vector of names
(resp. processes) of length $|\vec{x}|$ (resp. $|\vec{P}|$). We adopt
the following useful abbreviations.

\begin{mathpar}
   x?(\vec{y}).P := x.(\vec{y})P \and  x\clift{\vec{P}} := x.\clift{\vec{P}}
   \and x!(y) := \lift{x}{\dropn{y}}
   \and \Pi_{i=0}^{n-1}P_i := P_0 | \ldots | P_{n-1}
\end{mathpar}

\subsubsection{Structural congruence}

\paragraph{Free and bound names and alpha-equivalence.} At the
core of structural equivalence is alpha-equivalence which identifies
process that are the same up to a change of variable. Formally, we
recognize the distinction between free and bound names. The free names
of a process, $\freenames{P}$, may be calculated recursively as
follows:

\begin{mathpar}
\freenames{\pzero} := \emptyset
  \and \\
  \freenames{x?(y).P} := \{ x \} \cup (\freenames{P} \setminus \{ y \})
  \and 
  \freenames{x!\langle P \rangle} := \{ x \} \cup \{ P \} 
  \and \\
  \freenames{P|Q} := \freenames{P} \cup \freenames{Q}
  \and \\
  \freenames{@{x}} := \{ x \}
\end{mathpar}

$\pi$
$\quotep{\pi}$

$\freenames{-} : \pi \to \mathcal{P}(\quotep{\pi})$

\begin{eqnarray*}
  \freenames{\pzero} & := & \emptyset \\
  \freenames{x?(y).P} & := & \{ x \} \cup (\freenames{P} \setminus \{ y \}) \\
  \freenames{x!\langle P \rangle} & := & \{ x \} \cup \{ P \} \\
  \freenames{P|Q} & := & \freenames{P} \cup \freenames{Q} \\
  \freenames{\dropn{x}} & := & \{ x \}
\end{eqnarray*}

The bound names of a process, $\boundnames{P}$, are those names occurring in $P$
that are not free. For example, in $x?(y).0$, the name $x$ is free, while $y$ is bound.

\begin{mathpar}
  \inferrule* [lab=monoidal-laws] {} { P|Q \equiv Q|P \and P|0 \equiv P \and P|(Q|R) \equiv (P|Q)|R }
\end{mathpar}

\begin{mathpar}
  \inferrule* [lab=alpha-equivalence] {} { (x)P \equiv (y)P\{y/x\} \and y \not\in \freenames{P} }
\end{mathpar}

\begin{definition}
Then two processes, $P,Q$, are alpha-equivalent if $P = Q\{\vec{y}/\vec{x}\}$ for
some $\vec{x} \in \boundnames{Q},\vec{y} \in \boundnames{P}$, where $Q\{\vec{y}/\vec{x}\}$
denotes the capture-avoiding substitution of $\vec{y}$ for $\vec{x}$ in $Q$.
\end{definition}

\begin{definition}
  The {\em structural congruence} \cite{SangiorgiWalker} , $\equiv$,
  between processes is the least congruence containing
  alpha-equivalence, satisfying the abelian monoid laws
  (associativity, commutativity and $\pzero$ as identity) for parallel
  composition $|$ and for summation $+$.
\end{definition}

\subsection{Name equivalence}

We take name equivalence, written $\nameeq$, to be the smallest
equivalence relation generated by the following rules.

\begin{mathpar}
\inferrule*[lab=Quote-drop]
{ }
{ \quotep{@{x}} \nameeq x }

\inferrule*[lab=Struct-equiv]
{ P \scong Q }
{ \quotep{P} \nameeq \quotep{Q} }
\end{mathpar}

The astute reader will have noticed that the mutual recursion of names
and processes imposes a mutual recursion on alpha-equivalence and
structural equivalence via name-equivalence. Fortunately, all of this
works out pleasantly and we may calculate in the natural way, free of
concern. The reader interested in the details is referred to the
appendix \ref{appendix:rho_details}.

\subsection{Substitution}

We use $\Proc$ for the set of processes, $\QProc$ for the set of
names, and $\id{\{}\vec{y} / \vec{x} \id{\}}$ to denote partial maps,
$s : \QProc \rightarrow \QProc$. A map, $s$ lifts, uniquely, to a map
on process terms, $\widehat{s} : \Proc \rightarrow \Proc$ by the
following equations.

\begin{mathpar}
  (0) \psubstp{Q}{P} := 0 \\
  (R \juxtap S) \psubstp{Q}{P}
  :=    
  (R)\psubstp{Q}{P} \juxtap (S) \psubstp{Q}{P} \\
  (x?(y).R) \psubstp{Q}{P}    
  :=    
  (x)\substp{Q}{P} (z)\concat( (R \psubstn{z}{y}) \psubstp{Q}{P} ) \\
  (\lift{x}{R}) \psubstp{Q}{P}  
  :=
  \lift{(x)\substp{Q}{P}}{ R \psubstp{Q}{P} } \\
%   (\dropn{x})  \psubstp{Q}{P}       
%   := 
%   \left\{ 
%     \begin{array}{ccc} 
%       \dropn{\quotep{Q}} & & x \nameeq \quotep{P} \\
%       \dropn{x} & & otherwise \\
%     \end{array}
%   \right. 
  (\dropn{x})  \psubstp{Q}{P}       
  := 
  \left\{ 
    \begin{array}{ccc} 
      Q & & x \nameeq \quotep{P} \\
      \dropn{x} & & otherwise \\
    \end{array}
  \right.
\end{mathpar}
 

where

\begin{eqnarray}
  (x)\id{\{} \lpquote Q \rpquote / \lpquote P \rpquote \id{\}}            = 
  \left\{ 
    \begin{array}{ccc}
      \lpquote Q \rpquote & & x \nameeq \lpquote P \rpquote \\
      x & & otherwise \\
    \end{array}
  \right. \nonumber
\end{eqnarray}

and $z$ is chosen distinct from $\quotep{P}$, $\quotep{Q}$, the free
names in $Q$, and all the names in $R$. Our $\alpha$-equivalence will
be built in the standard way from this substitution.

\begin{remark}\label{rem:no_self_referential_names}
  One consequence of these definitions is that $\forall P. \quotep{P}
  \not\in \freenames{P}$.
\end{remark}

\subsection{ Dynamic quote: an example }

Anticipating something of what's to come, consider applying the
substitution, $\widehat{\id{\{}u / z \id{\}}}$, to the following pair
of processes, $\lift{w}{y!(z)}$ and $w[ \lpquote y!(z) \rpquote ]$.

\begin{eqnarray}
	\lift{w}{y!(z)}\widehat{\id{\{}u / z \id{\}}}
		& = &
		\lift{w}{y!(u)} \nonumber\\
	w[ \lpquote y!(z) \rpquote ] \widehat{ \id{\{}u / z \id{\}} }
		& = &
		w[ \lpquote y!(z) \rpquote ] \nonumber
\end{eqnarray}

Because the body of the process between quotes is impervious to
substitution, we get radically different answers. In fact, by
examining the first process in an input context,
e.g. $x?(z).\lift{w}{y!(z)}$, we see that the process under the lift
operator may be shaped by prefixed inputs binding a name inside it. In
this sense, the lift operator will be seen as a way to dynamically
construct processes before reifying them as names.

Finally equipped with these standard features we can present the
dynamics of the calculus.

\subsubsection{Operational semantics} 

Finally, we introduce the computational dynamics. What marks these
algebras as distinct from other more traditionally studied algebraic
structures, e.g. vector spaces or polynomial rings, is the manner in
which dynamics is captured. In traditional structures, dynamics is typically
expressed through morphisms between such structures, as in linear maps
between vector spaces or morphisms between rings. In algebras
associated with the semantics of computation, the dynamics is
expressed as part of the algebraic structure itself, through a
reduction reduction relation typically denoted by $\red$. Below, we
give a recursive presentation of this relation for the calculus used
in the encoding.

$\red \subseteq \pi \times \pi$
$\red : \pi \to \mathcal{P}(\pi)$

\begin{mathpar}
  \inferrule* [lab=Comm] { \textsf{match}( x_{src}, x_{trgt} ) } { x_{trgt}?(y)P \; | \; x_{src}!\langle {Q} \rangle \red P\{\quotep{Q}/y}\} }
  \and \\
  \inferrule* [lab=Par] {{P} \red {P}'} {{{P} | {Q}} \red {{P}' | {Q}}}
  \and
  \inferrule* [lab=Equiv]{{{P} \scong {P}'} \andalso {{P}' \red {Q}'} \andalso {{Q}' \scong {Q}}}{{P} \red {Q}}
\end{mathpar}

\begin{eqnarray*}
  match_{\equiv} (\quotep{P},\quotep{Q}) & := & P \equiv Q \\
  match_{\dagger}(\quotep{P},\quotep{Q}) & := & \forall R. P|Q \red^{*} R => R \red^{*} 0 \\
  match_{K}(\quotep{P},\quotep{Q}) & := & K \mbox{ for some context } K
\end{eqnarray*}

$u?(x)P | u!\langle Q \rangle \red P\{\quotep{Q}/x\}$

%We write $\wred$ for $\red^*$, and $P\red$ if $\exists Q $ such that $ P \red Q$.
We write $P\red$ if $\exists Q $ such that $ P \red Q$ and $P\not\red$, otherwise.

\section{Replication}

As mentioned before, it is known that replication (and hence
recursion) can be implemented in a higher-order process algebra
\cite{SangiorgiWalker}. As our first example of calculation with the
machinery thus far presented we give the construction explicitly in
the {\rhoc}.

\begin{eqnarray}
	D_{x} & := & \prefix{x}{y}{(\binpar{\outputp{x}{y}}{@{y}})} \nonumber\\
	\bangp_{x}{P} & := & \binpar{{x}!\langle{\binpar{D_{x}}{P}}\rangle}{D_{x}} \nonumber
\end{eqnarray}

\begin{eqnarray}
	\bangp_{x}{P} & & \nonumber\\
	=
	& {x}!\langle{(\prefix{x}{y}{(\outputp{x}{y} | @{y})) | P}}\rangle 
	      | \prefix{x}{y}{(\outputp{x}{y} | @{y})} & \nonumber\\
	\red
	& (\outputp{x}{y} | @{y})\substn{\quotep{(\prefix{x}{y}{(@{y} | \outputp{x}{y})) | P}}}{y} & \nonumber\\
	=
	& \outputp{x}{\quotep{(\prefix{x}{y}{(\outputp{x}{y} | @{y})) | P}}}
	  | {(\prefix{x}{y}{(\outputp{x}{y} | @{y})) | P}} & \nonumber\\
	\red
	& \ldots & \nonumber\\
	\red^*
	& P | P | \ldots & \nonumber
\end{eqnarray}

Of course, this encoding, as an implementation, runs away, unfolding
$\bangp{P}$ eagerly. A lazier and more implementable replication
operator, restricted to input-guarded processes, may be obtained as follows.

\begin{eqnarray}
\bangp{\prefix{u}{v}{P}} 
	:= 
	\binpar{\lift{x}{\prefix{u}{v}{(\binpar{D(x)}{P})}}}{D(x)} \nonumber
\end{eqnarray}

\begin{remark}
  Note that the lazier definition still does not deal with summation
  or mixed summation (i.e. sums over input and output). The reader is
  invited to construct definitions of replication that deal with these
  features. 

  Further, the definitions are parameterized in a name, $x$. Can you,
  gentle reader, make a definition that eliminates this parameter and
  guarantees no accidental interaction between the replication
  machinery and the process being replicated -- i.e. no accidental
  sharing of names used by the process to get its work done and the
  name(s) used by the replication to effect copying. This latter
  revision of the definition of replication is crucial to obtaining
  the expected identity $!!P \sim !P$.
\end{remark}

\begin{remark}\label{rem:paradoxical_combinator}
  The reader familiar with the lambda calculus will have noticed the
  similarity between $D$ and the paradoxical combinator.

  [Ed. note: the existence of this seems to suggest we have to be more
  restrictive on the set of processes and names we admit if we are to
  support no-cloning.]
\end{remark}

\subsubsection{Bisimulation}

The computational dynamics gives rise to another kind of equivalence,
the equivalence of computational behavior. As previously mentioned
this is typically captured \emph{via} some form of bisimulation.

% The notion we use in this paper is weak barbed bisimulation
% \cite{milner91polyadicpi}.

The notion we use in this paper is derived from weak barbed
bisimulation \cite{milner91polyadicpi}. 

\begin{definition}
An \emph{observation relation}, $\downarrow_{\mathcal N}$, over a set
of names, $\mathcal N$, is the smallest relation satisfying the rules
below.

\infrule[Out-barb]{y \in {\mathcal N}, \; x \nameeq y}
		  {\outputp{x}{v} \downarrow_{\mathcal N} x}
\infrule[Par-barb]{\mbox{$P\downarrow_{\mathcal N} x$ or $Q\downarrow_{\mathcal N} x$}}
		  {\binpar{P}{Q} \downarrow_{\mathcal N} x}

We write $P \Downarrow_{\mathcal N} x$ if there is $Q$ such that 
$P \wred Q$ and $Q \downarrow_{\mathcal N} x$.
\end{definition}

\begin{definition}
%\label{def.bbisim}
An  ${\mathcal N}$-\emph{barbed bisimulation} over a set of names, ${\mathcal N}$, is a symmetric binary relation 
${\mathcal S}_{\mathcal N}$ between agents such that $P\rel{S}_{\mathcal N}Q$ implies:
\begin{enumerate}
\item If $P \red P'$ then $Q \wred Q'$ and $P'\rel{S}_{\mathcal N} Q'$.
\item If $P\downarrow_{\mathcal N} x$, then $Q\Downarrow_{\mathcal N} x$.
\end{enumerate}
$P$ is ${\mathcal N}$-barbed bisimilar to $Q$, written
$P \wbbisim_{\mathcal N} Q$, if $P \rel{S}_{\mathcal N} Q$ for some ${\mathcal N}$-barbed bisimulation ${\mathcal S}_{\mathcal N}$.
\end{definition}

$\mathcal{R} \subseteq \pi \times \pi$

$P \mathcal{R} Q => \forall P'. P \red P' \Rightarrow \exists Q'. Q \red Q', P' \mathcal{R} Q'$

$P \vdash x \Rightarrow Q \vdash x$

\begin{mathpar}
  \inferrule*[lab=Out-barb]{x \nameeq y}{{y}!\langle{Q}\rangle \vdash x}
  \and
  \inferrule*[lab=Par-barb]{\mbox{$P\vdash x$ or $Q\vdash x$}}{\binpar{P}{Q} \vdash x}
\end{mathpar}

\subsubsection{Contexts}

One of the principle advantages of computational calculi like the
$\pi$-calculus is a well-defined notion of context,
contextual-equivalence and a correlation between
contextual-equivalence and notions of bisimulation. The notion of
context allows the decomposition of a process into (sub-)process and
its syntactic environment, its context. Thus, a context may be
thought of as a process with a ``hole'' (written $\Box$) in it. The
application of a context $M$ to a process $P$, written $M[P]$, is
tantamount to filling the hole in $M$ with $P$. In this paper we do
not need the full weight of this theory, but do make use of the notion
of context in the proof the main theorem. 

\begin{mathpar}
  \inferrule* [lab=summation] {} {{M_{M},M_{N}} \bc \Box \;|\; x.M_{A} \;|\; M_{M}+M_{N}}
  \and
  \inferrule* [lab=agent] {} {{M_{A}} \bc (\vec{x})M_{P} \;| \; \clift{P_0,\ldots,M_{P},\ldots,P_N}}
  \and \\
  \inferrule* [lab=process] {} {{M_{P}} \bc M_{N} \;| \;P|M_{P} }
\end{mathpar} 

\begin{mathpar}
  \inferrule* [lab=sychronization] {} {M_{N} \bc \Box \;|\; x?M_{F} \;|\; x!M_{C}}
  \and
  \inferrule* [lab=abstraction] {} {{M_{F}} \bc (x)M_{P} }
  \and
  \inferrule* [lab=concretion] {} {{M_{C}} \bc \langle M_{P} \rangle }
  \and \\
  \inferrule* [lab=process] {} {{M_{P}} \bc M_{N} \;| \;P|M_{P} }
\end{mathpar}

\begin{definition}[contextual application] Given a context $M$, and
  process $P$, we define the \emph{contextual application}, $M[P] :=
  M\{P/\Box\}$. That is, the contextual application of M to P is the
  substitution of $P$ for $\Box$ in $M$.
\end{definition}

$\meaningof{-} : L \to \mathcal{P}(\pi)$

\begin{mathpar}
  \inferrule* [lab=collection] {} {\meaningof{true} = \pi, \and \meaningof{~E} = \pi \setminus \meaningof{E}, \and \meaningof{E_{1} \& E_{2}} = \meaningof{E_{1}} \cap \meaningof{E_{2}}}
\end{mathpar}

\begin{mathpar}
  \inferrule* [lab=structure] {} {\meaningof{0} = \{ P \in \pi | P \equiv 0 \}, \and \\ \meaningof{E_1 | E_2} = \{ P \in \pi | P \equiv P_{1} | P_{2}, P_{1} \in \meaningof{E_{1}}, P_{2} \in \meaningof{E_2}\} }
\end{mathpar}

\begin{mathpar}
 \inferrule* [lab=behavior] {} {\meaningof{\langle a?b \rangle E} = \{ P \in \pi | P \equiv Q | u?(y)P', \\ \and \\\\ \and \\ \;\;\; u \in \meaningof{a}, \forall z.P'\{z/y\} \in \meaningof{E\{z/b\}}\}, \and \\ \meaningof{a!E} = \{ P \in \pi | P \equiv Q | x!\langle P' \rangle, x \in \meaningof{a} P' \in \meaningof{E}\} }
\end{mathpar}

\begin{mathpar}
 \inferrule* [lab=nominal] {} {\meaningof{\quotep{E}} = \{ \quotep{P} \in \quotep{\pi} | P \in \meaningof{E} \}, \and \meaningof{\quotep{P}} = \{ \quotep{Q} \in \quotep{\pi} | P \equiv Q \} \and \\ \meaningof{@\quotep{E}} = \{ P \in \pi | P \equiv @x, x \in \meaningof{E} \}}
\end{mathpar}

\begin{eqnarray*}
  \\
  \meaningof{-} : TS \to ST
\end{eqnarray*}

\begin{eqnarray*}
  \\
  L : TS \to ST
\end{eqnarray*}

\begin{eqnarray*}
  \\
  P \models E \iff P \in \meaningof{E}
\end{eqnarray*}

\begin{eqnarray*}
  P \approx_{L} Q \iff \forall E \in L. P \models E \iff Q \models E
\end{eqnarray*}

\begin{eqnarray*}
  P \approx_{K} Q
\end{eqnarray*}

\begin{eqnarray*}
  P \approx Q
\end{eqnarray*}

$\approx_{K} = \approx = \approx_{L}$

\subsubsection{Contextual duality}

Note that contexts extend the quotation operation to a family of
operations from processes to names. Given a context, $M$, we can
define a \emph{nominal context}, $\quotep{M}$ by $\quotep{M}[P] :=
\quotep{M[P]}$. To foreshadow what is to come we observe that these
operations enjoy a duality with processes very much like the duality
between vectors and maps from vectors to scalars.

Further, because the calculus is essentially higher-order, we have a
correspondence between contexts and processes. More specifically,
given a name $x$ and a context $M$ we can construct $M^{*}_{x}$ such
that 

\begin{mathpar}
  M^{*}_{x} | \lift{x}{P} \red M[P]
\end{mathpar}

namely,

\begin{mathpar}
  M^{*}_{x} := x?(u).M[\dropn{u}]
\end{mathpar}

The dependence of $M^{*}_{x}$ on a name makes it an abstraction, 

\begin{mathpar}
  M^{*} := (x)x?(u).M[\dropn{u}]
\end{mathpar}

\subsection{Additional notation}

It will sometimes be convenient to denote the process a name
quotes. We already have the notation $x = \quotep{P}$, but it will be
convenient to introduce an alternate notation, $\procn{x}$, when we
want to emphasize the connection to the use of the name. Note that, by
virtue of name equivalence, $\quotep{\procn{x}} \nameeq x$; so, the
notation is consistent with previous definitions.

Further, because names have structure it is possible to effect
substitutions on the basis of that structure. This means we need to
upgrade our notation for substitutions, which we accomplish by
adapting comprehension notation. Thus,

\begin{mathpar}
  P\{ y / x : x \in S \}
\end{mathpar}

is interpreted to mean the process derived from P by replacing (in a
capture-avoiding manner) each occurrence of $x$ in $S$ by $y$. For example,

\begin{mathpar}
  P\{ \quotep{\procn{x}|\procn{x}} / x : x \in \freenames{P} \}
\end{mathpar}

will replace each (occurrence) of a free name $x$ in $P$ by
$\quotep{\procn{x}|\procn{x}}$.

Also, we will avail ourselves of the notation $x^{L}$ and $x^{R}$ to
denote injections of a name into disjoint copies of the name
space. There are numerous ways to accomplish this. One example can be
found in \cite{MeredithR05}. This notation overloads to vectors of
names: $\vec{x}^{\pi} := (x_{i}^{\pi} \; : \; 0 \leq i < |\vec{x}| )$ where $\pi \in \{L,R\}$.

We also use $P^{\Box} := P|\Box$.

In \cite{MeredithR05} an interpretation of the new operator is
given. It turns out that there are several possible interpretations
all enjoying the requisite algebraic properties of the operator (see
\cite{milner91polyadicpi}). We will therefore make liberal use of
$(\nu\; \vec{x})P$.

% subsection the_syntax_and_semantics_of_the_notation_system (end)   

\input{qm2pi.qmops} 

\input{qm2pi.sterngerlach} 

\input{qm2pi.metric} 

% section concurrent_process_calculi (end)

%\input{qm2pi.proofsketch}

% section proof sketch (end)

%\input{qm2pi.slviaknots} 

% section spatial logic via knots (end)

\input{qm2pi.conclusion}

% section conclusion (end)

%\input{qm2pi.dtcodes} 

% section wiring algorithm (end)

\input{qm2pi.ack} 

% section acknowledgments (end)

\newpage


\bibliographystyle{plain}   
\bibliography{../../biblios/main.bib}

\input{qm2pi.rhodetails}

\end{document}

 

%\documentclass[12pt]{llncs}
%\documentclass{jktr}

\usepackage[pdftex]{hyperref}                   
\usepackage {listings}
\usepackage {mathpartir}
\usepackage{bcprules}
%\usepackage{listings}
                       
\usepackage{graphicx} 
%\usepackage[margins=2.5cm,nohead,nofoot]{geometry}
%\usepackage{geometry}
\usepackage{amsfonts}
\usepackage{amstext}
\usepackage{latexsym}
\usepackage{amssymb}
\usepackage{color}


%\include{myPreamble}
\include{qm2pi.local} 

%\ifpdf
%\usepackage[pdftex]{graphicx}
%\else
%\usepackage{graphicx}
%\fi

 % \ifpdf
%  \usepackage{pdfsync}
%  \if


%\title{Brief Article}
%\author{David F. Snyder}
%\author{L.G. Meredith}

%\address{Dept. of Math., Texas State University--San Marcos, San Marcos, TX 78666}
       
\pagestyle{empty}


\begin{document}

\lstset{language=[Objective]Caml,frame=shadowbox}

\input{qm2pi.front}

% section front matter (end)

\input{qm2pi.intro} 
 
% section introduction (end)

% \input{qm2pi.knotations} 

% section notation (end)

\input{qm2pi.process.calculi} 

% section concurrent_process_calculi_and_spatial_logics_ (end)
    
%\input{qm2pi.knots2pi} 

%\input{qm2pi.trefoil} 

%\input{qm2pi.mainthm} 

% subsection basic_interpretation (end)

%\input{qm2pi.rho.presentation} 
\subsection{The syntax and semantics of the notation system}\label{sub:the_syntax_and_semantics_of_the_notation_system} % (fold)

We now summarize a technical presentation of the calculus that
embodies our theory of dynamics. The typical presentation of such a
calculus follows the style of giving generators and relations on
them. The grammar, below, describing term constructors, freely
generates the set of processes, $\Proc$. This set is then quotiented
by a relation known as structural congruence and it is over this set
that the notion of dynamics is expressed. This presentation is
essentially that of \cite{MeredithR05} with the addition of
polyadicity and summation. For readability we have relegated some of
the technical subtleties to an appendix.

\subsubsection{Process grammar}\label{subsub:process_grammar}

\begin{mathpar}
  \inferrule* [lab=synchronization] {} {{M} \bc \pzero \;|\; x?F \;|\; x!C }
  \and
  \inferrule* [lab=abstraction] {} {{F} \bc (x)P}
  \and
  \inferrule* [lab=concretion] {} {{C} \bc \langle Q \rangle}
  \and
  \inferrule* [lab=process] {} {{P,Q} \bc M \;| \;P|Q \;|\; @{x}}
  \and
  \inferrule* [lab=name] {} {{x} \bc \quotep{P}}
\end{mathpar} 

Note that $\vec{x}$ (resp. $\vec{P}$) denotes a vector of names
(resp. processes) of length $|\vec{x}|$ (resp. $|\vec{P}|$). We adopt
the following useful abbreviations.

\begin{mathpar}
   x?(\vec{y}).P := x.(\vec{y})P \and  x\clift{\vec{P}} := x.\clift{\vec{P}}
   \and x!(y) := \lift{x}{\dropn{y}}
   \and \Pi_{i=0}^{n-1}P_i := P_0 | \ldots | P_{n-1}
\end{mathpar}

\subsubsection{Structural congruence}

\paragraph{Free and bound names and alpha-equivalence.} At the
core of structural equivalence is alpha-equivalence which identifies
process that are the same up to a change of variable. Formally, we
recognize the distinction between free and bound names. The free names
of a process, $\freenames{P}$, may be calculated recursively as
follows:

\begin{mathpar}
\freenames{\pzero} := \emptyset
  \and \\
  \freenames{x?(y).P} := \{ x \} \cup (\freenames{P} \setminus \{ y \})
  \and 
  \freenames{x!\langle P \rangle} := \{ x \} \cup \{ P \} 
  \and \\
  \freenames{P|Q} := \freenames{P} \cup \freenames{Q}
  \and \\
  \freenames{@{x}} := \{ x \}
\end{mathpar}

$\pi$
$\quotep{\pi}$

$\freenames{-} : \pi \to \mathcal{P}(\quotep{\pi})$

\begin{eqnarray*}
  \freenames{\pzero} & := & \emptyset \\
  \freenames{x?(y).P} & := & \{ x \} \cup (\freenames{P} \setminus \{ y \}) \\
  \freenames{x!\langle P \rangle} & := & \{ x \} \cup \{ P \} \\
  \freenames{P|Q} & := & \freenames{P} \cup \freenames{Q} \\
  \freenames{\dropn{x}} & := & \{ x \}
\end{eqnarray*}

The bound names of a process, $\boundnames{P}$, are those names occurring in $P$
that are not free. For example, in $x?(y).0$, the name $x$ is free, while $y$ is bound.

\begin{mathpar}
  \inferrule* [lab=monoidal-laws] {} { P|Q \equiv Q|P \and P|0 \equiv P \and P|(Q|R) \equiv (P|Q)|R }
\end{mathpar}

\begin{mathpar}
  \inferrule* [lab=alpha-equivalence] {} { (x)P \equiv (y)P\{y/x\} \and y \not\in \freenames{P} }
\end{mathpar}

\begin{definition}
Then two processes, $P,Q$, are alpha-equivalent if $P = Q\{\vec{y}/\vec{x}\}$ for
some $\vec{x} \in \boundnames{Q},\vec{y} \in \boundnames{P}$, where $Q\{\vec{y}/\vec{x}\}$
denotes the capture-avoiding substitution of $\vec{y}$ for $\vec{x}$ in $Q$.
\end{definition}

\begin{definition}
  The {\em structural congruence} \cite{SangiorgiWalker} , $\equiv$,
  between processes is the least congruence containing
  alpha-equivalence, satisfying the abelian monoid laws
  (associativity, commutativity and $\pzero$ as identity) for parallel
  composition $|$ and for summation $+$.
\end{definition}

\subsection{Name equivalence}

We take name equivalence, written $\nameeq$, to be the smallest
equivalence relation generated by the following rules.

\begin{mathpar}
\inferrule*[lab=Quote-drop]
{ }
{ \quotep{@{x}} \nameeq x }

\inferrule*[lab=Struct-equiv]
{ P \scong Q }
{ \quotep{P} \nameeq \quotep{Q} }
\end{mathpar}

The astute reader will have noticed that the mutual recursion of names
and processes imposes a mutual recursion on alpha-equivalence and
structural equivalence via name-equivalence. Fortunately, all of this
works out pleasantly and we may calculate in the natural way, free of
concern. The reader interested in the details is referred to the
appendix \ref{appendix:rho_details}.

\subsection{Substitution}

We use $\Proc$ for the set of processes, $\QProc$ for the set of
names, and $\id{\{}\vec{y} / \vec{x} \id{\}}$ to denote partial maps,
$s : \QProc \rightarrow \QProc$. A map, $s$ lifts, uniquely, to a map
on process terms, $\widehat{s} : \Proc \rightarrow \Proc$ by the
following equations.

\begin{mathpar}
  (0) \psubstp{Q}{P} := 0 \\
  (R \juxtap S) \psubstp{Q}{P}
  :=    
  (R)\psubstp{Q}{P} \juxtap (S) \psubstp{Q}{P} \\
  (x?(y).R) \psubstp{Q}{P}    
  :=    
  (x)\substp{Q}{P} (z)\concat( (R \psubstn{z}{y}) \psubstp{Q}{P} ) \\
  (\lift{x}{R}) \psubstp{Q}{P}  
  :=
  \lift{(x)\substp{Q}{P}}{ R \psubstp{Q}{P} } \\
%   (\dropn{x})  \psubstp{Q}{P}       
%   := 
%   \left\{ 
%     \begin{array}{ccc} 
%       \dropn{\quotep{Q}} & & x \nameeq \quotep{P} \\
%       \dropn{x} & & otherwise \\
%     \end{array}
%   \right. 
  (\dropn{x})  \psubstp{Q}{P}       
  := 
  \left\{ 
    \begin{array}{ccc} 
      Q & & x \nameeq \quotep{P} \\
      \dropn{x} & & otherwise \\
    \end{array}
  \right.
\end{mathpar}
 

where

\begin{eqnarray}
  (x)\id{\{} \lpquote Q \rpquote / \lpquote P \rpquote \id{\}}            = 
  \left\{ 
    \begin{array}{ccc}
      \lpquote Q \rpquote & & x \nameeq \lpquote P \rpquote \\
      x & & otherwise \\
    \end{array}
  \right. \nonumber
\end{eqnarray}

and $z$ is chosen distinct from $\quotep{P}$, $\quotep{Q}$, the free
names in $Q$, and all the names in $R$. Our $\alpha$-equivalence will
be built in the standard way from this substitution.

\begin{remark}\label{rem:no_self_referential_names}
  One consequence of these definitions is that $\forall P. \quotep{P}
  \not\in \freenames{P}$.
\end{remark}

\subsection{ Dynamic quote: an example }

Anticipating something of what's to come, consider applying the
substitution, $\widehat{\id{\{}u / z \id{\}}}$, to the following pair
of processes, $\lift{w}{y!(z)}$ and $w[ \lpquote y!(z) \rpquote ]$.

\begin{eqnarray}
	\lift{w}{y!(z)}\widehat{\id{\{}u / z \id{\}}}
		& = &
		\lift{w}{y!(u)} \nonumber\\
	w[ \lpquote y!(z) \rpquote ] \widehat{ \id{\{}u / z \id{\}} }
		& = &
		w[ \lpquote y!(z) \rpquote ] \nonumber
\end{eqnarray}

Because the body of the process between quotes is impervious to
substitution, we get radically different answers. In fact, by
examining the first process in an input context,
e.g. $x?(z).\lift{w}{y!(z)}$, we see that the process under the lift
operator may be shaped by prefixed inputs binding a name inside it. In
this sense, the lift operator will be seen as a way to dynamically
construct processes before reifying them as names.

Finally equipped with these standard features we can present the
dynamics of the calculus.

\subsubsection{Operational semantics} 

Finally, we introduce the computational dynamics. What marks these
algebras as distinct from other more traditionally studied algebraic
structures, e.g. vector spaces or polynomial rings, is the manner in
which dynamics is captured. In traditional structures, dynamics is typically
expressed through morphisms between such structures, as in linear maps
between vector spaces or morphisms between rings. In algebras
associated with the semantics of computation, the dynamics is
expressed as part of the algebraic structure itself, through a
reduction reduction relation typically denoted by $\red$. Below, we
give a recursive presentation of this relation for the calculus used
in the encoding.

$\red \subseteq \pi \times \pi$
$\red : \pi \to \mathcal{P}(\pi)$

\begin{mathpar}
  \inferrule* [lab=Comm] { \textsf{match}( x_{src}, x_{trgt} ) } { x_{trgt}?(y)P \; | \; x_{src}!\langle {Q} \rangle \red P\{\quotep{Q}/y}\} }
  \and \\
  \inferrule* [lab=Par] {{P} \red {P}'} {{{P} | {Q}} \red {{P}' | {Q}}}
  \and
  \inferrule* [lab=Equiv]{{{P} \scong {P}'} \andalso {{P}' \red {Q}'} \andalso {{Q}' \scong {Q}}}{{P} \red {Q}}
\end{mathpar}

\begin{eqnarray*}
  match_{\equiv} (\quotep{P},\quotep{Q}) & := & P \equiv Q \\
  match_{\dagger}(\quotep{P},\quotep{Q}) & := & \forall R. P|Q \red^{*} R => R \red^{*} 0 \\
  match_{K}(\quotep{P},\quotep{Q}) & := & K \mbox{ for some context } K
\end{eqnarray*}

$u?(x)P | u!\langle Q \rangle \red P\{\quotep{Q}/x\}$

%We write $\wred$ for $\red^*$, and $P\red$ if $\exists Q $ such that $ P \red Q$.
We write $P\red$ if $\exists Q $ such that $ P \red Q$ and $P\not\red$, otherwise.

\section{Replication}

As mentioned before, it is known that replication (and hence
recursion) can be implemented in a higher-order process algebra
\cite{SangiorgiWalker}. As our first example of calculation with the
machinery thus far presented we give the construction explicitly in
the {\rhoc}.

\begin{eqnarray}
	D_{x} & := & \prefix{x}{y}{(\binpar{\outputp{x}{y}}{@{y}})} \nonumber\\
	\bangp_{x}{P} & := & \binpar{{x}!\langle{\binpar{D_{x}}{P}}\rangle}{D_{x}} \nonumber
\end{eqnarray}

\begin{eqnarray}
	\bangp_{x}{P} & & \nonumber\\
	=
	& {x}!\langle{(\prefix{x}{y}{(\outputp{x}{y} | @{y})) | P}}\rangle 
	      | \prefix{x}{y}{(\outputp{x}{y} | @{y})} & \nonumber\\
	\red
	& (\outputp{x}{y} | @{y})\substn{\quotep{(\prefix{x}{y}{(@{y} | \outputp{x}{y})) | P}}}{y} & \nonumber\\
	=
	& \outputp{x}{\quotep{(\prefix{x}{y}{(\outputp{x}{y} | @{y})) | P}}}
	  | {(\prefix{x}{y}{(\outputp{x}{y} | @{y})) | P}} & \nonumber\\
	\red
	& \ldots & \nonumber\\
	\red^*
	& P | P | \ldots & \nonumber
\end{eqnarray}

Of course, this encoding, as an implementation, runs away, unfolding
$\bangp{P}$ eagerly. A lazier and more implementable replication
operator, restricted to input-guarded processes, may be obtained as follows.

\begin{eqnarray}
\bangp{\prefix{u}{v}{P}} 
	:= 
	\binpar{\lift{x}{\prefix{u}{v}{(\binpar{D(x)}{P})}}}{D(x)} \nonumber
\end{eqnarray}

\begin{remark}
  Note that the lazier definition still does not deal with summation
  or mixed summation (i.e. sums over input and output). The reader is
  invited to construct definitions of replication that deal with these
  features. 

  Further, the definitions are parameterized in a name, $x$. Can you,
  gentle reader, make a definition that eliminates this parameter and
  guarantees no accidental interaction between the replication
  machinery and the process being replicated -- i.e. no accidental
  sharing of names used by the process to get its work done and the
  name(s) used by the replication to effect copying. This latter
  revision of the definition of replication is crucial to obtaining
  the expected identity $!!P \sim !P$.
\end{remark}

\begin{remark}\label{rem:paradoxical_combinator}
  The reader familiar with the lambda calculus will have noticed the
  similarity between $D$ and the paradoxical combinator.

  [Ed. note: the existence of this seems to suggest we have to be more
  restrictive on the set of processes and names we admit if we are to
  support no-cloning.]
\end{remark}

\subsubsection{Bisimulation}

The computational dynamics gives rise to another kind of equivalence,
the equivalence of computational behavior. As previously mentioned
this is typically captured \emph{via} some form of bisimulation.

% The notion we use in this paper is weak barbed bisimulation
% \cite{milner91polyadicpi}.

The notion we use in this paper is derived from weak barbed
bisimulation \cite{milner91polyadicpi}. 

\begin{definition}
An \emph{observation relation}, $\downarrow_{\mathcal N}$, over a set
of names, $\mathcal N$, is the smallest relation satisfying the rules
below.

\infrule[Out-barb]{y \in {\mathcal N}, \; x \nameeq y}
		  {\outputp{x}{v} \downarrow_{\mathcal N} x}
\infrule[Par-barb]{\mbox{$P\downarrow_{\mathcal N} x$ or $Q\downarrow_{\mathcal N} x$}}
		  {\binpar{P}{Q} \downarrow_{\mathcal N} x}

We write $P \Downarrow_{\mathcal N} x$ if there is $Q$ such that 
$P \wred Q$ and $Q \downarrow_{\mathcal N} x$.
\end{definition}

\begin{definition}
%\label{def.bbisim}
An  ${\mathcal N}$-\emph{barbed bisimulation} over a set of names, ${\mathcal N}$, is a symmetric binary relation 
${\mathcal S}_{\mathcal N}$ between agents such that $P\rel{S}_{\mathcal N}Q$ implies:
\begin{enumerate}
\item If $P \red P'$ then $Q \wred Q'$ and $P'\rel{S}_{\mathcal N} Q'$.
\item If $P\downarrow_{\mathcal N} x$, then $Q\Downarrow_{\mathcal N} x$.
\end{enumerate}
$P$ is ${\mathcal N}$-barbed bisimilar to $Q$, written
$P \wbbisim_{\mathcal N} Q$, if $P \rel{S}_{\mathcal N} Q$ for some ${\mathcal N}$-barbed bisimulation ${\mathcal S}_{\mathcal N}$.
\end{definition}

$\mathcal{R} \subseteq \pi \times \pi$

$P \mathcal{R} Q => \forall P'. P \red P' \Rightarrow \exists Q'. Q \red Q', P' \mathcal{R} Q'$

$P \vdash x \Rightarrow Q \vdash x$

\begin{mathpar}
  \inferrule*[lab=Out-barb]{x \nameeq y}{{y}!\langle{Q}\rangle \vdash x}
  \and
  \inferrule*[lab=Par-barb]{\mbox{$P\vdash x$ or $Q\vdash x$}}{\binpar{P}{Q} \vdash x}
\end{mathpar}

\subsubsection{Contexts}

One of the principle advantages of computational calculi like the
$\pi$-calculus is a well-defined notion of context,
contextual-equivalence and a correlation between
contextual-equivalence and notions of bisimulation. The notion of
context allows the decomposition of a process into (sub-)process and
its syntactic environment, its context. Thus, a context may be
thought of as a process with a ``hole'' (written $\Box$) in it. The
application of a context $M$ to a process $P$, written $M[P]$, is
tantamount to filling the hole in $M$ with $P$. In this paper we do
not need the full weight of this theory, but do make use of the notion
of context in the proof the main theorem. 

\begin{mathpar}
  \inferrule* [lab=summation] {} {{M_{M},M_{N}} \bc \Box \;|\; x.M_{A} \;|\; M_{M}+M_{N}}
  \and
  \inferrule* [lab=agent] {} {{M_{A}} \bc (\vec{x})M_{P} \;| \; \clift{P_0,\ldots,M_{P},\ldots,P_N}}
  \and \\
  \inferrule* [lab=process] {} {{M_{P}} \bc M_{N} \;| \;P|M_{P} }
\end{mathpar} 

\begin{mathpar}
  \inferrule* [lab=sychronization] {} {M_{N} \bc \Box \;|\; x?M_{F} \;|\; x!M_{C}}
  \and
  \inferrule* [lab=abstraction] {} {{M_{F}} \bc (x)M_{P} }
  \and
  \inferrule* [lab=concretion] {} {{M_{C}} \bc \langle M_{P} \rangle }
  \and \\
  \inferrule* [lab=process] {} {{M_{P}} \bc M_{N} \;| \;P|M_{P} }
\end{mathpar}

\begin{definition}[contextual application] Given a context $M$, and
  process $P$, we define the \emph{contextual application}, $M[P] :=
  M\{P/\Box\}$. That is, the contextual application of M to P is the
  substitution of $P$ for $\Box$ in $M$.
\end{definition}

$\meaningof{-} : L \to \mathcal{P}(\pi)$

\begin{mathpar}
  \inferrule* [lab=collection] {} {\meaningof{true} = \pi, \and \meaningof{~E} = \pi \setminus \meaningof{E}, \and \meaningof{E_{1} \& E_{2}} = \meaningof{E_{1}} \cap \meaningof{E_{2}}}
\end{mathpar}

\begin{mathpar}
  \inferrule* [lab=structure] {} {\meaningof{0} = \{ P \in \pi | P \equiv 0 \}, \and \\ \meaningof{E_1 | E_2} = \{ P \in \pi | P \equiv P_{1} | P_{2}, P_{1} \in \meaningof{E_{1}}, P_{2} \in \meaningof{E_2}\} }
\end{mathpar}

\begin{mathpar}
 \inferrule* [lab=behavior] {} {\meaningof{\langle a?b \rangle E} = \{ P \in \pi | P \equiv Q | u?(y)P', \\ \and \\\\ \and \\ \;\;\; u \in \meaningof{a}, \forall z.P'\{z/y\} \in \meaningof{E\{z/b\}}\}, \and \\ \meaningof{a!E} = \{ P \in \pi | P \equiv Q | x!\langle P' \rangle, x \in \meaningof{a} P' \in \meaningof{E}\} }
\end{mathpar}

\begin{mathpar}
 \inferrule* [lab=nominal] {} {\meaningof{\quotep{E}} = \{ \quotep{P} \in \quotep{\pi} | P \in \meaningof{E} \}, \and \meaningof{\quotep{P}} = \{ \quotep{Q} \in \quotep{\pi} | P \equiv Q \} \and \\ \meaningof{@\quotep{E}} = \{ P \in \pi | P \equiv @x, x \in \meaningof{E} \}}
\end{mathpar}

\begin{eqnarray*}
  \\
  \meaningof{-} : TS \to ST
\end{eqnarray*}

\begin{eqnarray*}
  \\
  L : TS \to ST
\end{eqnarray*}

\begin{eqnarray*}
  \\
  P \models E \iff P \in \meaningof{E}
\end{eqnarray*}

\begin{eqnarray*}
  P \approx_{L} Q \iff \forall E \in L. P \models E \iff Q \models E
\end{eqnarray*}

\begin{eqnarray*}
  P \approx_{K} Q
\end{eqnarray*}

\begin{eqnarray*}
  P \approx Q
\end{eqnarray*}

$\approx_{K} = \approx = \approx_{L}$

\subsubsection{Contextual duality}

Note that contexts extend the quotation operation to a family of
operations from processes to names. Given a context, $M$, we can
define a \emph{nominal context}, $\quotep{M}$ by $\quotep{M}[P] :=
\quotep{M[P]}$. To foreshadow what is to come we observe that these
operations enjoy a duality with processes very much like the duality
between vectors and maps from vectors to scalars.

Further, because the calculus is essentially higher-order, we have a
correspondence between contexts and processes. More specifically,
given a name $x$ and a context $M$ we can construct $M^{*}_{x}$ such
that 

\begin{mathpar}
  M^{*}_{x} | \lift{x}{P} \red M[P]
\end{mathpar}

namely,

\begin{mathpar}
  M^{*}_{x} := x?(u).M[\dropn{u}]
\end{mathpar}

The dependence of $M^{*}_{x}$ on a name makes it an abstraction, 

\begin{mathpar}
  M^{*} := (x)x?(u).M[\dropn{u}]
\end{mathpar}

\subsection{Additional notation}

It will sometimes be convenient to denote the process a name
quotes. We already have the notation $x = \quotep{P}$, but it will be
convenient to introduce an alternate notation, $\procn{x}$, when we
want to emphasize the connection to the use of the name. Note that, by
virtue of name equivalence, $\quotep{\procn{x}} \nameeq x$; so, the
notation is consistent with previous definitions.

Further, because names have structure it is possible to effect
substitutions on the basis of that structure. This means we need to
upgrade our notation for substitutions, which we accomplish by
adapting comprehension notation. Thus,

\begin{mathpar}
  P\{ y / x : x \in S \}
\end{mathpar}

is interpreted to mean the process derived from P by replacing (in a
capture-avoiding manner) each occurrence of $x$ in $S$ by $y$. For example,

\begin{mathpar}
  P\{ \quotep{\procn{x}|\procn{x}} / x : x \in \freenames{P} \}
\end{mathpar}

will replace each (occurrence) of a free name $x$ in $P$ by
$\quotep{\procn{x}|\procn{x}}$.

Also, we will avail ourselves of the notation $x^{L}$ and $x^{R}$ to
denote injections of a name into disjoint copies of the name
space. There are numerous ways to accomplish this. One example can be
found in \cite{MeredithR05}. This notation overloads to vectors of
names: $\vec{x}^{\pi} := (x_{i}^{\pi} \; : \; 0 \leq i < |\vec{x}| )$ where $\pi \in \{L,R\}$.

We also use $P^{\Box} := P|\Box$.

In \cite{MeredithR05} an interpretation of the new operator is
given. It turns out that there are several possible interpretations
all enjoying the requisite algebraic properties of the operator (see
\cite{milner91polyadicpi}). We will therefore make liberal use of
$(\nu\; \vec{x})P$.

% subsection the_syntax_and_semantics_of_the_notation_system (end)   

\input{qm2pi.qmops} 

\input{qm2pi.sterngerlach} 

\input{qm2pi.metric} 

% section concurrent_process_calculi (end)

%\input{qm2pi.proofsketch}

% section proof sketch (end)

%\input{qm2pi.slviaknots} 

% section spatial logic via knots (end)

\input{qm2pi.conclusion}

% section conclusion (end)

%\input{qm2pi.dtcodes} 

% section wiring algorithm (end)

\input{qm2pi.ack} 

% section acknowledgments (end)

\newpage


\bibliographystyle{plain}   
\bibliography{../../biblios/main.bib}

\input{qm2pi.rhodetails}

\end{document}

 

%\documentclass[12pt]{llncs}
%\documentclass{jktr}

\usepackage[pdftex]{hyperref}                   
\usepackage {listings}
\usepackage {mathpartir}
\usepackage{bcprules}
%\usepackage{listings}
                       
\usepackage{graphicx} 
%\usepackage[margins=2.5cm,nohead,nofoot]{geometry}
%\usepackage{geometry}
\usepackage{amsfonts}
\usepackage{amstext}
\usepackage{latexsym}
\usepackage{amssymb}
\usepackage{color}


%\include{myPreamble}
\include{qm2pi.local} 

%\ifpdf
%\usepackage[pdftex]{graphicx}
%\else
%\usepackage{graphicx}
%\fi

 % \ifpdf
%  \usepackage{pdfsync}
%  \if


%\title{Brief Article}
%\author{David F. Snyder}
%\author{L.G. Meredith}

%\address{Dept. of Math., Texas State University--San Marcos, San Marcos, TX 78666}
       
\pagestyle{empty}


\begin{document}

\lstset{language=[Objective]Caml,frame=shadowbox}

\input{qm2pi.front}

% section front matter (end)

\input{qm2pi.intro} 
 
% section introduction (end)

% \input{qm2pi.knotations} 

% section notation (end)

\input{qm2pi.process.calculi} 

% section concurrent_process_calculi_and_spatial_logics_ (end)
    
%\input{qm2pi.knots2pi} 

%\input{qm2pi.trefoil} 

%\input{qm2pi.mainthm} 

% subsection basic_interpretation (end)

%\input{qm2pi.rho.presentation} 
\subsection{The syntax and semantics of the notation system}\label{sub:the_syntax_and_semantics_of_the_notation_system} % (fold)

We now summarize a technical presentation of the calculus that
embodies our theory of dynamics. The typical presentation of such a
calculus follows the style of giving generators and relations on
them. The grammar, below, describing term constructors, freely
generates the set of processes, $\Proc$. This set is then quotiented
by a relation known as structural congruence and it is over this set
that the notion of dynamics is expressed. This presentation is
essentially that of \cite{MeredithR05} with the addition of
polyadicity and summation. For readability we have relegated some of
the technical subtleties to an appendix.

\subsubsection{Process grammar}\label{subsub:process_grammar}

\begin{mathpar}
  \inferrule* [lab=synchronization] {} {{M} \bc \pzero \;|\; x?F \;|\; x!C }
  \and
  \inferrule* [lab=abstraction] {} {{F} \bc (x)P}
  \and
  \inferrule* [lab=concretion] {} {{C} \bc \langle Q \rangle}
  \and
  \inferrule* [lab=process] {} {{P,Q} \bc M \;| \;P|Q \;|\; @{x}}
  \and
  \inferrule* [lab=name] {} {{x} \bc \quotep{P}}
\end{mathpar} 

Note that $\vec{x}$ (resp. $\vec{P}$) denotes a vector of names
(resp. processes) of length $|\vec{x}|$ (resp. $|\vec{P}|$). We adopt
the following useful abbreviations.

\begin{mathpar}
   x?(\vec{y}).P := x.(\vec{y})P \and  x\clift{\vec{P}} := x.\clift{\vec{P}}
   \and x!(y) := \lift{x}{\dropn{y}}
   \and \Pi_{i=0}^{n-1}P_i := P_0 | \ldots | P_{n-1}
\end{mathpar}

\subsubsection{Structural congruence}

\paragraph{Free and bound names and alpha-equivalence.} At the
core of structural equivalence is alpha-equivalence which identifies
process that are the same up to a change of variable. Formally, we
recognize the distinction between free and bound names. The free names
of a process, $\freenames{P}$, may be calculated recursively as
follows:

\begin{mathpar}
\freenames{\pzero} := \emptyset
  \and \\
  \freenames{x?(y).P} := \{ x \} \cup (\freenames{P} \setminus \{ y \})
  \and 
  \freenames{x!\langle P \rangle} := \{ x \} \cup \{ P \} 
  \and \\
  \freenames{P|Q} := \freenames{P} \cup \freenames{Q}
  \and \\
  \freenames{@{x}} := \{ x \}
\end{mathpar}

$\pi$
$\quotep{\pi}$

$\freenames{-} : \pi \to \mathcal{P}(\quotep{\pi})$

\begin{eqnarray*}
  \freenames{\pzero} & := & \emptyset \\
  \freenames{x?(y).P} & := & \{ x \} \cup (\freenames{P} \setminus \{ y \}) \\
  \freenames{x!\langle P \rangle} & := & \{ x \} \cup \{ P \} \\
  \freenames{P|Q} & := & \freenames{P} \cup \freenames{Q} \\
  \freenames{\dropn{x}} & := & \{ x \}
\end{eqnarray*}

The bound names of a process, $\boundnames{P}$, are those names occurring in $P$
that are not free. For example, in $x?(y).0$, the name $x$ is free, while $y$ is bound.

\begin{mathpar}
  \inferrule* [lab=monoidal-laws] {} { P|Q \equiv Q|P \and P|0 \equiv P \and P|(Q|R) \equiv (P|Q)|R }
\end{mathpar}

\begin{mathpar}
  \inferrule* [lab=alpha-equivalence] {} { (x)P \equiv (y)P\{y/x\} \and y \not\in \freenames{P} }
\end{mathpar}

\begin{definition}
Then two processes, $P,Q$, are alpha-equivalent if $P = Q\{\vec{y}/\vec{x}\}$ for
some $\vec{x} \in \boundnames{Q},\vec{y} \in \boundnames{P}$, where $Q\{\vec{y}/\vec{x}\}$
denotes the capture-avoiding substitution of $\vec{y}$ for $\vec{x}$ in $Q$.
\end{definition}

\begin{definition}
  The {\em structural congruence} \cite{SangiorgiWalker} , $\equiv$,
  between processes is the least congruence containing
  alpha-equivalence, satisfying the abelian monoid laws
  (associativity, commutativity and $\pzero$ as identity) for parallel
  composition $|$ and for summation $+$.
\end{definition}

\subsection{Name equivalence}

We take name equivalence, written $\nameeq$, to be the smallest
equivalence relation generated by the following rules.

\begin{mathpar}
\inferrule*[lab=Quote-drop]
{ }
{ \quotep{@{x}} \nameeq x }

\inferrule*[lab=Struct-equiv]
{ P \scong Q }
{ \quotep{P} \nameeq \quotep{Q} }
\end{mathpar}

The astute reader will have noticed that the mutual recursion of names
and processes imposes a mutual recursion on alpha-equivalence and
structural equivalence via name-equivalence. Fortunately, all of this
works out pleasantly and we may calculate in the natural way, free of
concern. The reader interested in the details is referred to the
appendix \ref{appendix:rho_details}.

\subsection{Substitution}

We use $\Proc$ for the set of processes, $\QProc$ for the set of
names, and $\id{\{}\vec{y} / \vec{x} \id{\}}$ to denote partial maps,
$s : \QProc \rightarrow \QProc$. A map, $s$ lifts, uniquely, to a map
on process terms, $\widehat{s} : \Proc \rightarrow \Proc$ by the
following equations.

\begin{mathpar}
  (0) \psubstp{Q}{P} := 0 \\
  (R \juxtap S) \psubstp{Q}{P}
  :=    
  (R)\psubstp{Q}{P} \juxtap (S) \psubstp{Q}{P} \\
  (x?(y).R) \psubstp{Q}{P}    
  :=    
  (x)\substp{Q}{P} (z)\concat( (R \psubstn{z}{y}) \psubstp{Q}{P} ) \\
  (\lift{x}{R}) \psubstp{Q}{P}  
  :=
  \lift{(x)\substp{Q}{P}}{ R \psubstp{Q}{P} } \\
%   (\dropn{x})  \psubstp{Q}{P}       
%   := 
%   \left\{ 
%     \begin{array}{ccc} 
%       \dropn{\quotep{Q}} & & x \nameeq \quotep{P} \\
%       \dropn{x} & & otherwise \\
%     \end{array}
%   \right. 
  (\dropn{x})  \psubstp{Q}{P}       
  := 
  \left\{ 
    \begin{array}{ccc} 
      Q & & x \nameeq \quotep{P} \\
      \dropn{x} & & otherwise \\
    \end{array}
  \right.
\end{mathpar}
 

where

\begin{eqnarray}
  (x)\id{\{} \lpquote Q \rpquote / \lpquote P \rpquote \id{\}}            = 
  \left\{ 
    \begin{array}{ccc}
      \lpquote Q \rpquote & & x \nameeq \lpquote P \rpquote \\
      x & & otherwise \\
    \end{array}
  \right. \nonumber
\end{eqnarray}

and $z$ is chosen distinct from $\quotep{P}$, $\quotep{Q}$, the free
names in $Q$, and all the names in $R$. Our $\alpha$-equivalence will
be built in the standard way from this substitution.

\begin{remark}\label{rem:no_self_referential_names}
  One consequence of these definitions is that $\forall P. \quotep{P}
  \not\in \freenames{P}$.
\end{remark}

\subsection{ Dynamic quote: an example }

Anticipating something of what's to come, consider applying the
substitution, $\widehat{\id{\{}u / z \id{\}}}$, to the following pair
of processes, $\lift{w}{y!(z)}$ and $w[ \lpquote y!(z) \rpquote ]$.

\begin{eqnarray}
	\lift{w}{y!(z)}\widehat{\id{\{}u / z \id{\}}}
		& = &
		\lift{w}{y!(u)} \nonumber\\
	w[ \lpquote y!(z) \rpquote ] \widehat{ \id{\{}u / z \id{\}} }
		& = &
		w[ \lpquote y!(z) \rpquote ] \nonumber
\end{eqnarray}

Because the body of the process between quotes is impervious to
substitution, we get radically different answers. In fact, by
examining the first process in an input context,
e.g. $x?(z).\lift{w}{y!(z)}$, we see that the process under the lift
operator may be shaped by prefixed inputs binding a name inside it. In
this sense, the lift operator will be seen as a way to dynamically
construct processes before reifying them as names.

Finally equipped with these standard features we can present the
dynamics of the calculus.

\subsubsection{Operational semantics} 

Finally, we introduce the computational dynamics. What marks these
algebras as distinct from other more traditionally studied algebraic
structures, e.g. vector spaces or polynomial rings, is the manner in
which dynamics is captured. In traditional structures, dynamics is typically
expressed through morphisms between such structures, as in linear maps
between vector spaces or morphisms between rings. In algebras
associated with the semantics of computation, the dynamics is
expressed as part of the algebraic structure itself, through a
reduction reduction relation typically denoted by $\red$. Below, we
give a recursive presentation of this relation for the calculus used
in the encoding.

$\red \subseteq \pi \times \pi$
$\red : \pi \to \mathcal{P}(\pi)$

\begin{mathpar}
  \inferrule* [lab=Comm] { \textsf{match}( x_{src}, x_{trgt} ) } { x_{trgt}?(y)P \; | \; x_{src}!\langle {Q} \rangle \red P\{\quotep{Q}/y}\} }
  \and \\
  \inferrule* [lab=Par] {{P} \red {P}'} {{{P} | {Q}} \red {{P}' | {Q}}}
  \and
  \inferrule* [lab=Equiv]{{{P} \scong {P}'} \andalso {{P}' \red {Q}'} \andalso {{Q}' \scong {Q}}}{{P} \red {Q}}
\end{mathpar}

\begin{eqnarray*}
  match_{\equiv} (\quotep{P},\quotep{Q}) & := & P \equiv Q \\
  match_{\dagger}(\quotep{P},\quotep{Q}) & := & \forall R. P|Q \red^{*} R => R \red^{*} 0 \\
  match_{K}(\quotep{P},\quotep{Q}) & := & K \mbox{ for some context } K
\end{eqnarray*}

$u?(x)P | u!\langle Q \rangle \red P\{\quotep{Q}/x\}$

%We write $\wred$ for $\red^*$, and $P\red$ if $\exists Q $ such that $ P \red Q$.
We write $P\red$ if $\exists Q $ such that $ P \red Q$ and $P\not\red$, otherwise.

\section{Replication}

As mentioned before, it is known that replication (and hence
recursion) can be implemented in a higher-order process algebra
\cite{SangiorgiWalker}. As our first example of calculation with the
machinery thus far presented we give the construction explicitly in
the {\rhoc}.

\begin{eqnarray}
	D_{x} & := & \prefix{x}{y}{(\binpar{\outputp{x}{y}}{@{y}})} \nonumber\\
	\bangp_{x}{P} & := & \binpar{{x}!\langle{\binpar{D_{x}}{P}}\rangle}{D_{x}} \nonumber
\end{eqnarray}

\begin{eqnarray}
	\bangp_{x}{P} & & \nonumber\\
	=
	& {x}!\langle{(\prefix{x}{y}{(\outputp{x}{y} | @{y})) | P}}\rangle 
	      | \prefix{x}{y}{(\outputp{x}{y} | @{y})} & \nonumber\\
	\red
	& (\outputp{x}{y} | @{y})\substn{\quotep{(\prefix{x}{y}{(@{y} | \outputp{x}{y})) | P}}}{y} & \nonumber\\
	=
	& \outputp{x}{\quotep{(\prefix{x}{y}{(\outputp{x}{y} | @{y})) | P}}}
	  | {(\prefix{x}{y}{(\outputp{x}{y} | @{y})) | P}} & \nonumber\\
	\red
	& \ldots & \nonumber\\
	\red^*
	& P | P | \ldots & \nonumber
\end{eqnarray}

Of course, this encoding, as an implementation, runs away, unfolding
$\bangp{P}$ eagerly. A lazier and more implementable replication
operator, restricted to input-guarded processes, may be obtained as follows.

\begin{eqnarray}
\bangp{\prefix{u}{v}{P}} 
	:= 
	\binpar{\lift{x}{\prefix{u}{v}{(\binpar{D(x)}{P})}}}{D(x)} \nonumber
\end{eqnarray}

\begin{remark}
  Note that the lazier definition still does not deal with summation
  or mixed summation (i.e. sums over input and output). The reader is
  invited to construct definitions of replication that deal with these
  features. 

  Further, the definitions are parameterized in a name, $x$. Can you,
  gentle reader, make a definition that eliminates this parameter and
  guarantees no accidental interaction between the replication
  machinery and the process being replicated -- i.e. no accidental
  sharing of names used by the process to get its work done and the
  name(s) used by the replication to effect copying. This latter
  revision of the definition of replication is crucial to obtaining
  the expected identity $!!P \sim !P$.
\end{remark}

\begin{remark}\label{rem:paradoxical_combinator}
  The reader familiar with the lambda calculus will have noticed the
  similarity between $D$ and the paradoxical combinator.

  [Ed. note: the existence of this seems to suggest we have to be more
  restrictive on the set of processes and names we admit if we are to
  support no-cloning.]
\end{remark}

\subsubsection{Bisimulation}

The computational dynamics gives rise to another kind of equivalence,
the equivalence of computational behavior. As previously mentioned
this is typically captured \emph{via} some form of bisimulation.

% The notion we use in this paper is weak barbed bisimulation
% \cite{milner91polyadicpi}.

The notion we use in this paper is derived from weak barbed
bisimulation \cite{milner91polyadicpi}. 

\begin{definition}
An \emph{observation relation}, $\downarrow_{\mathcal N}$, over a set
of names, $\mathcal N$, is the smallest relation satisfying the rules
below.

\infrule[Out-barb]{y \in {\mathcal N}, \; x \nameeq y}
		  {\outputp{x}{v} \downarrow_{\mathcal N} x}
\infrule[Par-barb]{\mbox{$P\downarrow_{\mathcal N} x$ or $Q\downarrow_{\mathcal N} x$}}
		  {\binpar{P}{Q} \downarrow_{\mathcal N} x}

We write $P \Downarrow_{\mathcal N} x$ if there is $Q$ such that 
$P \wred Q$ and $Q \downarrow_{\mathcal N} x$.
\end{definition}

\begin{definition}
%\label{def.bbisim}
An  ${\mathcal N}$-\emph{barbed bisimulation} over a set of names, ${\mathcal N}$, is a symmetric binary relation 
${\mathcal S}_{\mathcal N}$ between agents such that $P\rel{S}_{\mathcal N}Q$ implies:
\begin{enumerate}
\item If $P \red P'$ then $Q \wred Q'$ and $P'\rel{S}_{\mathcal N} Q'$.
\item If $P\downarrow_{\mathcal N} x$, then $Q\Downarrow_{\mathcal N} x$.
\end{enumerate}
$P$ is ${\mathcal N}$-barbed bisimilar to $Q$, written
$P \wbbisim_{\mathcal N} Q$, if $P \rel{S}_{\mathcal N} Q$ for some ${\mathcal N}$-barbed bisimulation ${\mathcal S}_{\mathcal N}$.
\end{definition}

$\mathcal{R} \subseteq \pi \times \pi$

$P \mathcal{R} Q => \forall P'. P \red P' \Rightarrow \exists Q'. Q \red Q', P' \mathcal{R} Q'$

$P \vdash x \Rightarrow Q \vdash x$

\begin{mathpar}
  \inferrule*[lab=Out-barb]{x \nameeq y}{{y}!\langle{Q}\rangle \vdash x}
  \and
  \inferrule*[lab=Par-barb]{\mbox{$P\vdash x$ or $Q\vdash x$}}{\binpar{P}{Q} \vdash x}
\end{mathpar}

\subsubsection{Contexts}

One of the principle advantages of computational calculi like the
$\pi$-calculus is a well-defined notion of context,
contextual-equivalence and a correlation between
contextual-equivalence and notions of bisimulation. The notion of
context allows the decomposition of a process into (sub-)process and
its syntactic environment, its context. Thus, a context may be
thought of as a process with a ``hole'' (written $\Box$) in it. The
application of a context $M$ to a process $P$, written $M[P]$, is
tantamount to filling the hole in $M$ with $P$. In this paper we do
not need the full weight of this theory, but do make use of the notion
of context in the proof the main theorem. 

\begin{mathpar}
  \inferrule* [lab=summation] {} {{M_{M},M_{N}} \bc \Box \;|\; x.M_{A} \;|\; M_{M}+M_{N}}
  \and
  \inferrule* [lab=agent] {} {{M_{A}} \bc (\vec{x})M_{P} \;| \; \clift{P_0,\ldots,M_{P},\ldots,P_N}}
  \and \\
  \inferrule* [lab=process] {} {{M_{P}} \bc M_{N} \;| \;P|M_{P} }
\end{mathpar} 

\begin{mathpar}
  \inferrule* [lab=sychronization] {} {M_{N} \bc \Box \;|\; x?M_{F} \;|\; x!M_{C}}
  \and
  \inferrule* [lab=abstraction] {} {{M_{F}} \bc (x)M_{P} }
  \and
  \inferrule* [lab=concretion] {} {{M_{C}} \bc \langle M_{P} \rangle }
  \and \\
  \inferrule* [lab=process] {} {{M_{P}} \bc M_{N} \;| \;P|M_{P} }
\end{mathpar}

\begin{definition}[contextual application] Given a context $M$, and
  process $P$, we define the \emph{contextual application}, $M[P] :=
  M\{P/\Box\}$. That is, the contextual application of M to P is the
  substitution of $P$ for $\Box$ in $M$.
\end{definition}

$\meaningof{-} : L \to \mathcal{P}(\pi)$

\begin{mathpar}
  \inferrule* [lab=collection] {} {\meaningof{true} = \pi, \and \meaningof{~E} = \pi \setminus \meaningof{E}, \and \meaningof{E_{1} \& E_{2}} = \meaningof{E_{1}} \cap \meaningof{E_{2}}}
\end{mathpar}

\begin{mathpar}
  \inferrule* [lab=structure] {} {\meaningof{0} = \{ P \in \pi | P \equiv 0 \}, \and \\ \meaningof{E_1 | E_2} = \{ P \in \pi | P \equiv P_{1} | P_{2}, P_{1} \in \meaningof{E_{1}}, P_{2} \in \meaningof{E_2}\} }
\end{mathpar}

\begin{mathpar}
 \inferrule* [lab=behavior] {} {\meaningof{\langle a?b \rangle E} = \{ P \in \pi | P \equiv Q | u?(y)P', \\ \and \\\\ \and \\ \;\;\; u \in \meaningof{a}, \forall z.P'\{z/y\} \in \meaningof{E\{z/b\}}\}, \and \\ \meaningof{a!E} = \{ P \in \pi | P \equiv Q | x!\langle P' \rangle, x \in \meaningof{a} P' \in \meaningof{E}\} }
\end{mathpar}

\begin{mathpar}
 \inferrule* [lab=nominal] {} {\meaningof{\quotep{E}} = \{ \quotep{P} \in \quotep{\pi} | P \in \meaningof{E} \}, \and \meaningof{\quotep{P}} = \{ \quotep{Q} \in \quotep{\pi} | P \equiv Q \} \and \\ \meaningof{@\quotep{E}} = \{ P \in \pi | P \equiv @x, x \in \meaningof{E} \}}
\end{mathpar}

\begin{eqnarray*}
  \\
  \meaningof{-} : TS \to ST
\end{eqnarray*}

\begin{eqnarray*}
  \\
  L : TS \to ST
\end{eqnarray*}

\begin{eqnarray*}
  \\
  P \models E \iff P \in \meaningof{E}
\end{eqnarray*}

\begin{eqnarray*}
  P \approx_{L} Q \iff \forall E \in L. P \models E \iff Q \models E
\end{eqnarray*}

\begin{eqnarray*}
  P \approx_{K} Q
\end{eqnarray*}

\begin{eqnarray*}
  P \approx Q
\end{eqnarray*}

$\approx_{K} = \approx = \approx_{L}$

\subsubsection{Contextual duality}

Note that contexts extend the quotation operation to a family of
operations from processes to names. Given a context, $M$, we can
define a \emph{nominal context}, $\quotep{M}$ by $\quotep{M}[P] :=
\quotep{M[P]}$. To foreshadow what is to come we observe that these
operations enjoy a duality with processes very much like the duality
between vectors and maps from vectors to scalars.

Further, because the calculus is essentially higher-order, we have a
correspondence between contexts and processes. More specifically,
given a name $x$ and a context $M$ we can construct $M^{*}_{x}$ such
that 

\begin{mathpar}
  M^{*}_{x} | \lift{x}{P} \red M[P]
\end{mathpar}

namely,

\begin{mathpar}
  M^{*}_{x} := x?(u).M[\dropn{u}]
\end{mathpar}

The dependence of $M^{*}_{x}$ on a name makes it an abstraction, 

\begin{mathpar}
  M^{*} := (x)x?(u).M[\dropn{u}]
\end{mathpar}

\subsection{Additional notation}

It will sometimes be convenient to denote the process a name
quotes. We already have the notation $x = \quotep{P}$, but it will be
convenient to introduce an alternate notation, $\procn{x}$, when we
want to emphasize the connection to the use of the name. Note that, by
virtue of name equivalence, $\quotep{\procn{x}} \nameeq x$; so, the
notation is consistent with previous definitions.

Further, because names have structure it is possible to effect
substitutions on the basis of that structure. This means we need to
upgrade our notation for substitutions, which we accomplish by
adapting comprehension notation. Thus,

\begin{mathpar}
  P\{ y / x : x \in S \}
\end{mathpar}

is interpreted to mean the process derived from P by replacing (in a
capture-avoiding manner) each occurrence of $x$ in $S$ by $y$. For example,

\begin{mathpar}
  P\{ \quotep{\procn{x}|\procn{x}} / x : x \in \freenames{P} \}
\end{mathpar}

will replace each (occurrence) of a free name $x$ in $P$ by
$\quotep{\procn{x}|\procn{x}}$.

Also, we will avail ourselves of the notation $x^{L}$ and $x^{R}$ to
denote injections of a name into disjoint copies of the name
space. There are numerous ways to accomplish this. One example can be
found in \cite{MeredithR05}. This notation overloads to vectors of
names: $\vec{x}^{\pi} := (x_{i}^{\pi} \; : \; 0 \leq i < |\vec{x}| )$ where $\pi \in \{L,R\}$.

We also use $P^{\Box} := P|\Box$.

In \cite{MeredithR05} an interpretation of the new operator is
given. It turns out that there are several possible interpretations
all enjoying the requisite algebraic properties of the operator (see
\cite{milner91polyadicpi}). We will therefore make liberal use of
$(\nu\; \vec{x})P$.

% subsection the_syntax_and_semantics_of_the_notation_system (end)   

\input{qm2pi.qmops} 

\input{qm2pi.sterngerlach} 

\input{qm2pi.metric} 

% section concurrent_process_calculi (end)

%\input{qm2pi.proofsketch}

% section proof sketch (end)

%\input{qm2pi.slviaknots} 

% section spatial logic via knots (end)

\input{qm2pi.conclusion}

% section conclusion (end)

%\input{qm2pi.dtcodes} 

% section wiring algorithm (end)

\input{qm2pi.ack} 

% section acknowledgments (end)

\newpage


\bibliographystyle{plain}   
\bibliography{../../biblios/main.bib}

\input{qm2pi.rhodetails}

\end{document}

 

% subsection basic_interpretation (end)

%\input{qm2pi.rho.presentation} 
\subsection{The syntax and semantics of the notation system}\label{sub:the_syntax_and_semantics_of_the_notation_system} % (fold)

We now summarize a technical presentation of the calculus that
embodies our theory of dynamics. The typical presentation of such a
calculus follows the style of giving generators and relations on
them. The grammar, below, describing term constructors, freely
generates the set of processes, $\Proc$. This set is then quotiented
by a relation known as structural congruence and it is over this set
that the notion of dynamics is expressed. This presentation is
essentially that of \cite{MeredithR05} with the addition of
polyadicity and summation. For readability we have relegated some of
the technical subtleties to an appendix.

\subsubsection{Process grammar}\label{subsub:process_grammar}

\begin{mathpar}
  \inferrule* [lab=synchronization] {} {{M} \bc \pzero \;|\; x?F \;|\; x!C }
  \and
  \inferrule* [lab=abstraction] {} {{F} \bc (x)P}
  \and
  \inferrule* [lab=concretion] {} {{C} \bc \langle Q \rangle}
  \and
  \inferrule* [lab=process] {} {{P,Q} \bc M \;| \;P|Q \;|\; @{x}}
  \and
  \inferrule* [lab=name] {} {{x} \bc \quotep{P}}
\end{mathpar} 

Note that $\vec{x}$ (resp. $\vec{P}$) denotes a vector of names
(resp. processes) of length $|\vec{x}|$ (resp. $|\vec{P}|$). We adopt
the following useful abbreviations.

\begin{mathpar}
   x?(\vec{y}).P := x.(\vec{y})P \and  x\clift{\vec{P}} := x.\clift{\vec{P}}
   \and x!(y) := \lift{x}{\dropn{y}}
   \and \Pi_{i=0}^{n-1}P_i := P_0 | \ldots | P_{n-1}
\end{mathpar}

\subsubsection{Structural congruence}

\paragraph{Free and bound names and alpha-equivalence.} At the
core of structural equivalence is alpha-equivalence which identifies
process that are the same up to a change of variable. Formally, we
recognize the distinction between free and bound names. The free names
of a process, $\freenames{P}$, may be calculated recursively as
follows:

\begin{mathpar}
\freenames{\pzero} := \emptyset
  \and \\
  \freenames{x?(y).P} := \{ x \} \cup (\freenames{P} \setminus \{ y \})
  \and 
  \freenames{x!\langle P \rangle} := \{ x \} \cup \{ P \} 
  \and \\
  \freenames{P|Q} := \freenames{P} \cup \freenames{Q}
  \and \\
  \freenames{@{x}} := \{ x \}
\end{mathpar}

$\pi$
$\quotep{\pi}$

$\freenames{-} : \pi \to \mathcal{P}(\quotep{\pi})$

\begin{eqnarray*}
  \freenames{\pzero} & := & \emptyset \\
  \freenames{x?(y).P} & := & \{ x \} \cup (\freenames{P} \setminus \{ y \}) \\
  \freenames{x!\langle P \rangle} & := & \{ x \} \cup \{ P \} \\
  \freenames{P|Q} & := & \freenames{P} \cup \freenames{Q} \\
  \freenames{\dropn{x}} & := & \{ x \}
\end{eqnarray*}

The bound names of a process, $\boundnames{P}$, are those names occurring in $P$
that are not free. For example, in $x?(y).0$, the name $x$ is free, while $y$ is bound.

\begin{mathpar}
  \inferrule* [lab=monoidal-laws] {} { P|Q \equiv Q|P \and P|0 \equiv P \and P|(Q|R) \equiv (P|Q)|R }
\end{mathpar}

\begin{mathpar}
  \inferrule* [lab=alpha-equivalence] {} { (x)P \equiv (y)P\{y/x\} \and y \not\in \freenames{P} }
\end{mathpar}

\begin{definition}
Then two processes, $P,Q$, are alpha-equivalent if $P = Q\{\vec{y}/\vec{x}\}$ for
some $\vec{x} \in \boundnames{Q},\vec{y} \in \boundnames{P}$, where $Q\{\vec{y}/\vec{x}\}$
denotes the capture-avoiding substitution of $\vec{y}$ for $\vec{x}$ in $Q$.
\end{definition}

\begin{definition}
  The {\em structural congruence} \cite{SangiorgiWalker} , $\equiv$,
  between processes is the least congruence containing
  alpha-equivalence, satisfying the abelian monoid laws
  (associativity, commutativity and $\pzero$ as identity) for parallel
  composition $|$ and for summation $+$.
\end{definition}

\subsection{Name equivalence}

We take name equivalence, written $\nameeq$, to be the smallest
equivalence relation generated by the following rules.

\begin{mathpar}
\inferrule*[lab=Quote-drop]
{ }
{ \quotep{@{x}} \nameeq x }

\inferrule*[lab=Struct-equiv]
{ P \scong Q }
{ \quotep{P} \nameeq \quotep{Q} }
\end{mathpar}

The astute reader will have noticed that the mutual recursion of names
and processes imposes a mutual recursion on alpha-equivalence and
structural equivalence via name-equivalence. Fortunately, all of this
works out pleasantly and we may calculate in the natural way, free of
concern. The reader interested in the details is referred to the
appendix \ref{appendix:rho_details}.

\subsection{Substitution}

We use $\Proc$ for the set of processes, $\QProc$ for the set of
names, and $\id{\{}\vec{y} / \vec{x} \id{\}}$ to denote partial maps,
$s : \QProc \rightarrow \QProc$. A map, $s$ lifts, uniquely, to a map
on process terms, $\widehat{s} : \Proc \rightarrow \Proc$ by the
following equations.

\begin{mathpar}
  (0) \psubstp{Q}{P} := 0 \\
  (R \juxtap S) \psubstp{Q}{P}
  :=    
  (R)\psubstp{Q}{P} \juxtap (S) \psubstp{Q}{P} \\
  (x?(y).R) \psubstp{Q}{P}    
  :=    
  (x)\substp{Q}{P} (z)\concat( (R \psubstn{z}{y}) \psubstp{Q}{P} ) \\
  (\lift{x}{R}) \psubstp{Q}{P}  
  :=
  \lift{(x)\substp{Q}{P}}{ R \psubstp{Q}{P} } \\
%   (\dropn{x})  \psubstp{Q}{P}       
%   := 
%   \left\{ 
%     \begin{array}{ccc} 
%       \dropn{\quotep{Q}} & & x \nameeq \quotep{P} \\
%       \dropn{x} & & otherwise \\
%     \end{array}
%   \right. 
  (\dropn{x})  \psubstp{Q}{P}       
  := 
  \left\{ 
    \begin{array}{ccc} 
      Q & & x \nameeq \quotep{P} \\
      \dropn{x} & & otherwise \\
    \end{array}
  \right.
\end{mathpar}
 

where

\begin{eqnarray}
  (x)\id{\{} \lpquote Q \rpquote / \lpquote P \rpquote \id{\}}            = 
  \left\{ 
    \begin{array}{ccc}
      \lpquote Q \rpquote & & x \nameeq \lpquote P \rpquote \\
      x & & otherwise \\
    \end{array}
  \right. \nonumber
\end{eqnarray}

and $z$ is chosen distinct from $\quotep{P}$, $\quotep{Q}$, the free
names in $Q$, and all the names in $R$. Our $\alpha$-equivalence will
be built in the standard way from this substitution.

\begin{remark}\label{rem:no_self_referential_names}
  One consequence of these definitions is that $\forall P. \quotep{P}
  \not\in \freenames{P}$.
\end{remark}

\subsection{ Dynamic quote: an example }

Anticipating something of what's to come, consider applying the
substitution, $\widehat{\id{\{}u / z \id{\}}}$, to the following pair
of processes, $\lift{w}{y!(z)}$ and $w[ \lpquote y!(z) \rpquote ]$.

\begin{eqnarray}
	\lift{w}{y!(z)}\widehat{\id{\{}u / z \id{\}}}
		& = &
		\lift{w}{y!(u)} \nonumber\\
	w[ \lpquote y!(z) \rpquote ] \widehat{ \id{\{}u / z \id{\}} }
		& = &
		w[ \lpquote y!(z) \rpquote ] \nonumber
\end{eqnarray}

Because the body of the process between quotes is impervious to
substitution, we get radically different answers. In fact, by
examining the first process in an input context,
e.g. $x?(z).\lift{w}{y!(z)}$, we see that the process under the lift
operator may be shaped by prefixed inputs binding a name inside it. In
this sense, the lift operator will be seen as a way to dynamically
construct processes before reifying them as names.

Finally equipped with these standard features we can present the
dynamics of the calculus.

\subsubsection{Operational semantics} 

Finally, we introduce the computational dynamics. What marks these
algebras as distinct from other more traditionally studied algebraic
structures, e.g. vector spaces or polynomial rings, is the manner in
which dynamics is captured. In traditional structures, dynamics is typically
expressed through morphisms between such structures, as in linear maps
between vector spaces or morphisms between rings. In algebras
associated with the semantics of computation, the dynamics is
expressed as part of the algebraic structure itself, through a
reduction reduction relation typically denoted by $\red$. Below, we
give a recursive presentation of this relation for the calculus used
in the encoding.

$\red \subseteq \pi \times \pi$
$\red : \pi \to \mathcal{P}(\pi)$

\begin{mathpar}
  \inferrule* [lab=Comm] { \textsf{match}( x_{src}, x_{trgt} ) } { x_{trgt}?(y)P \; | \; x_{src}!\langle {Q} \rangle \red P\{\quotep{Q}/y}\} }
  \and \\
  \inferrule* [lab=Par] {{P} \red {P}'} {{{P} | {Q}} \red {{P}' | {Q}}}
  \and
  \inferrule* [lab=Equiv]{{{P} \scong {P}'} \andalso {{P}' \red {Q}'} \andalso {{Q}' \scong {Q}}}{{P} \red {Q}}
\end{mathpar}

\begin{eqnarray*}
  match_{\equiv} (\quotep{P},\quotep{Q}) & := & P \equiv Q \\
  match_{\dagger}(\quotep{P},\quotep{Q}) & := & \forall R. P|Q \red^{*} R => R \red^{*} 0 \\
  match_{K}(\quotep{P},\quotep{Q}) & := & K \mbox{ for some context } K
\end{eqnarray*}

$u?(x)P | u!\langle Q \rangle \red P\{\quotep{Q}/x\}$

%We write $\wred$ for $\red^*$, and $P\red$ if $\exists Q $ such that $ P \red Q$.
We write $P\red$ if $\exists Q $ such that $ P \red Q$ and $P\not\red$, otherwise.

\section{Replication}

As mentioned before, it is known that replication (and hence
recursion) can be implemented in a higher-order process algebra
\cite{SangiorgiWalker}. As our first example of calculation with the
machinery thus far presented we give the construction explicitly in
the {\rhoc}.

\begin{eqnarray}
	D_{x} & := & \prefix{x}{y}{(\binpar{\outputp{x}{y}}{@{y}})} \nonumber\\
	\bangp_{x}{P} & := & \binpar{{x}!\langle{\binpar{D_{x}}{P}}\rangle}{D_{x}} \nonumber
\end{eqnarray}

\begin{eqnarray}
	\bangp_{x}{P} & & \nonumber\\
	=
	& {x}!\langle{(\prefix{x}{y}{(\outputp{x}{y} | @{y})) | P}}\rangle 
	      | \prefix{x}{y}{(\outputp{x}{y} | @{y})} & \nonumber\\
	\red
	& (\outputp{x}{y} | @{y})\substn{\quotep{(\prefix{x}{y}{(@{y} | \outputp{x}{y})) | P}}}{y} & \nonumber\\
	=
	& \outputp{x}{\quotep{(\prefix{x}{y}{(\outputp{x}{y} | @{y})) | P}}}
	  | {(\prefix{x}{y}{(\outputp{x}{y} | @{y})) | P}} & \nonumber\\
	\red
	& \ldots & \nonumber\\
	\red^*
	& P | P | \ldots & \nonumber
\end{eqnarray}

Of course, this encoding, as an implementation, runs away, unfolding
$\bangp{P}$ eagerly. A lazier and more implementable replication
operator, restricted to input-guarded processes, may be obtained as follows.

\begin{eqnarray}
\bangp{\prefix{u}{v}{P}} 
	:= 
	\binpar{\lift{x}{\prefix{u}{v}{(\binpar{D(x)}{P})}}}{D(x)} \nonumber
\end{eqnarray}

\begin{remark}
  Note that the lazier definition still does not deal with summation
  or mixed summation (i.e. sums over input and output). The reader is
  invited to construct definitions of replication that deal with these
  features. 

  Further, the definitions are parameterized in a name, $x$. Can you,
  gentle reader, make a definition that eliminates this parameter and
  guarantees no accidental interaction between the replication
  machinery and the process being replicated -- i.e. no accidental
  sharing of names used by the process to get its work done and the
  name(s) used by the replication to effect copying. This latter
  revision of the definition of replication is crucial to obtaining
  the expected identity $!!P \sim !P$.
\end{remark}

\begin{remark}\label{rem:paradoxical_combinator}
  The reader familiar with the lambda calculus will have noticed the
  similarity between $D$ and the paradoxical combinator.

  [Ed. note: the existence of this seems to suggest we have to be more
  restrictive on the set of processes and names we admit if we are to
  support no-cloning.]
\end{remark}

\subsubsection{Bisimulation}

The computational dynamics gives rise to another kind of equivalence,
the equivalence of computational behavior. As previously mentioned
this is typically captured \emph{via} some form of bisimulation.

% The notion we use in this paper is weak barbed bisimulation
% \cite{milner91polyadicpi}.

The notion we use in this paper is derived from weak barbed
bisimulation \cite{milner91polyadicpi}. 

\begin{definition}
An \emph{observation relation}, $\downarrow_{\mathcal N}$, over a set
of names, $\mathcal N$, is the smallest relation satisfying the rules
below.

\infrule[Out-barb]{y \in {\mathcal N}, \; x \nameeq y}
		  {\outputp{x}{v} \downarrow_{\mathcal N} x}
\infrule[Par-barb]{\mbox{$P\downarrow_{\mathcal N} x$ or $Q\downarrow_{\mathcal N} x$}}
		  {\binpar{P}{Q} \downarrow_{\mathcal N} x}

We write $P \Downarrow_{\mathcal N} x$ if there is $Q$ such that 
$P \wred Q$ and $Q \downarrow_{\mathcal N} x$.
\end{definition}

\begin{definition}
%\label{def.bbisim}
An  ${\mathcal N}$-\emph{barbed bisimulation} over a set of names, ${\mathcal N}$, is a symmetric binary relation 
${\mathcal S}_{\mathcal N}$ between agents such that $P\rel{S}_{\mathcal N}Q$ implies:
\begin{enumerate}
\item If $P \red P'$ then $Q \wred Q'$ and $P'\rel{S}_{\mathcal N} Q'$.
\item If $P\downarrow_{\mathcal N} x$, then $Q\Downarrow_{\mathcal N} x$.
\end{enumerate}
$P$ is ${\mathcal N}$-barbed bisimilar to $Q$, written
$P \wbbisim_{\mathcal N} Q$, if $P \rel{S}_{\mathcal N} Q$ for some ${\mathcal N}$-barbed bisimulation ${\mathcal S}_{\mathcal N}$.
\end{definition}

$\mathcal{R} \subseteq \pi \times \pi$

$P \mathcal{R} Q => \forall P'. P \red P' \Rightarrow \exists Q'. Q \red Q', P' \mathcal{R} Q'$

$P \vdash x \Rightarrow Q \vdash x$

\begin{mathpar}
  \inferrule*[lab=Out-barb]{x \nameeq y}{{y}!\langle{Q}\rangle \vdash x}
  \and
  \inferrule*[lab=Par-barb]{\mbox{$P\vdash x$ or $Q\vdash x$}}{\binpar{P}{Q} \vdash x}
\end{mathpar}

\subsubsection{Contexts}

One of the principle advantages of computational calculi like the
$\pi$-calculus is a well-defined notion of context,
contextual-equivalence and a correlation between
contextual-equivalence and notions of bisimulation. The notion of
context allows the decomposition of a process into (sub-)process and
its syntactic environment, its context. Thus, a context may be
thought of as a process with a ``hole'' (written $\Box$) in it. The
application of a context $M$ to a process $P$, written $M[P]$, is
tantamount to filling the hole in $M$ with $P$. In this paper we do
not need the full weight of this theory, but do make use of the notion
of context in the proof the main theorem. 

\begin{mathpar}
  \inferrule* [lab=summation] {} {{M_{M},M_{N}} \bc \Box \;|\; x.M_{A} \;|\; M_{M}+M_{N}}
  \and
  \inferrule* [lab=agent] {} {{M_{A}} \bc (\vec{x})M_{P} \;| \; \clift{P_0,\ldots,M_{P},\ldots,P_N}}
  \and \\
  \inferrule* [lab=process] {} {{M_{P}} \bc M_{N} \;| \;P|M_{P} }
\end{mathpar} 

\begin{mathpar}
  \inferrule* [lab=sychronization] {} {M_{N} \bc \Box \;|\; x?M_{F} \;|\; x!M_{C}}
  \and
  \inferrule* [lab=abstraction] {} {{M_{F}} \bc (x)M_{P} }
  \and
  \inferrule* [lab=concretion] {} {{M_{C}} \bc \langle M_{P} \rangle }
  \and \\
  \inferrule* [lab=process] {} {{M_{P}} \bc M_{N} \;| \;P|M_{P} }
\end{mathpar}

\begin{definition}[contextual application] Given a context $M$, and
  process $P$, we define the \emph{contextual application}, $M[P] :=
  M\{P/\Box\}$. That is, the contextual application of M to P is the
  substitution of $P$ for $\Box$ in $M$.
\end{definition}

$\meaningof{-} : L \to \mathcal{P}(\pi)$

\begin{mathpar}
  \inferrule* [lab=collection] {} {\meaningof{true} = \pi, \and \meaningof{~E} = \pi \setminus \meaningof{E}, \and \meaningof{E_{1} \& E_{2}} = \meaningof{E_{1}} \cap \meaningof{E_{2}}}
\end{mathpar}

\begin{mathpar}
  \inferrule* [lab=structure] {} {\meaningof{0} = \{ P \in \pi | P \equiv 0 \}, \and \\ \meaningof{E_1 | E_2} = \{ P \in \pi | P \equiv P_{1} | P_{2}, P_{1} \in \meaningof{E_{1}}, P_{2} \in \meaningof{E_2}\} }
\end{mathpar}

\begin{mathpar}
 \inferrule* [lab=behavior] {} {\meaningof{\langle a?b \rangle E} = \{ P \in \pi | P \equiv Q | u?(y)P', \\ \and \\\\ \and \\ \;\;\; u \in \meaningof{a}, \forall z.P'\{z/y\} \in \meaningof{E\{z/b\}}\}, \and \\ \meaningof{a!E} = \{ P \in \pi | P \equiv Q | x!\langle P' \rangle, x \in \meaningof{a} P' \in \meaningof{E}\} }
\end{mathpar}

\begin{mathpar}
 \inferrule* [lab=nominal] {} {\meaningof{\quotep{E}} = \{ \quotep{P} \in \quotep{\pi} | P \in \meaningof{E} \}, \and \meaningof{\quotep{P}} = \{ \quotep{Q} \in \quotep{\pi} | P \equiv Q \} \and \\ \meaningof{@\quotep{E}} = \{ P \in \pi | P \equiv @x, x \in \meaningof{E} \}}
\end{mathpar}

\begin{eqnarray*}
  \\
  \meaningof{-} : TS \to ST
\end{eqnarray*}

\begin{eqnarray*}
  \\
  L : TS \to ST
\end{eqnarray*}

\begin{eqnarray*}
  \\
  P \models E \iff P \in \meaningof{E}
\end{eqnarray*}

\begin{eqnarray*}
  P \approx_{L} Q \iff \forall E \in L. P \models E \iff Q \models E
\end{eqnarray*}

\begin{eqnarray*}
  P \approx_{K} Q
\end{eqnarray*}

\begin{eqnarray*}
  P \approx Q
\end{eqnarray*}

$\approx_{K} = \approx = \approx_{L}$

\subsubsection{Contextual duality}

Note that contexts extend the quotation operation to a family of
operations from processes to names. Given a context, $M$, we can
define a \emph{nominal context}, $\quotep{M}$ by $\quotep{M}[P] :=
\quotep{M[P]}$. To foreshadow what is to come we observe that these
operations enjoy a duality with processes very much like the duality
between vectors and maps from vectors to scalars.

Further, because the calculus is essentially higher-order, we have a
correspondence between contexts and processes. More specifically,
given a name $x$ and a context $M$ we can construct $M^{*}_{x}$ such
that 

\begin{mathpar}
  M^{*}_{x} | \lift{x}{P} \red M[P]
\end{mathpar}

namely,

\begin{mathpar}
  M^{*}_{x} := x?(u).M[\dropn{u}]
\end{mathpar}

The dependence of $M^{*}_{x}$ on a name makes it an abstraction, 

\begin{mathpar}
  M^{*} := (x)x?(u).M[\dropn{u}]
\end{mathpar}

\subsection{Additional notation}

It will sometimes be convenient to denote the process a name
quotes. We already have the notation $x = \quotep{P}$, but it will be
convenient to introduce an alternate notation, $\procn{x}$, when we
want to emphasize the connection to the use of the name. Note that, by
virtue of name equivalence, $\quotep{\procn{x}} \nameeq x$; so, the
notation is consistent with previous definitions.

Further, because names have structure it is possible to effect
substitutions on the basis of that structure. This means we need to
upgrade our notation for substitutions, which we accomplish by
adapting comprehension notation. Thus,

\begin{mathpar}
  P\{ y / x : x \in S \}
\end{mathpar}

is interpreted to mean the process derived from P by replacing (in a
capture-avoiding manner) each occurrence of $x$ in $S$ by $y$. For example,

\begin{mathpar}
  P\{ \quotep{\procn{x}|\procn{x}} / x : x \in \freenames{P} \}
\end{mathpar}

will replace each (occurrence) of a free name $x$ in $P$ by
$\quotep{\procn{x}|\procn{x}}$.

Also, we will avail ourselves of the notation $x^{L}$ and $x^{R}$ to
denote injections of a name into disjoint copies of the name
space. There are numerous ways to accomplish this. One example can be
found in \cite{MeredithR05}. This notation overloads to vectors of
names: $\vec{x}^{\pi} := (x_{i}^{\pi} \; : \; 0 \leq i < |\vec{x}| )$ where $\pi \in \{L,R\}$.

We also use $P^{\Box} := P|\Box$.

In \cite{MeredithR05} an interpretation of the new operator is
given. It turns out that there are several possible interpretations
all enjoying the requisite algebraic properties of the operator (see
\cite{milner91polyadicpi}). We will therefore make liberal use of
$(\nu\; \vec{x})P$.

% subsection the_syntax_and_semantics_of_the_notation_system (end)   

\section{Interpretation of QM}
\subsection{Supporting definitions}
\subsubsection{Multiplication}
\begin{mathpar}
  \quotep{Q} \cdot \quotep{R} := \quotep{Q|R}
  \and \\
  \quotep{Q} \cdot P := P\{ \quotep{Q|R} / \quotep{R} : \quotep{R} \in \freenames{P} \}
\end{mathpar}

\paragraph{Discussion}
The first line needs little explanation. The second line says that
each free name of the process is replaced with the multiplication of
that name by the scalar. Multiplication of a scalar (name) by a state
(process) results in a process all the names of which have been `moved
over' by parallel composition with the process the scalar
quotes. There is a subtlety that the bound names have to be
manipulated so that multiplied names aren't accidentally
captured. There are many ways to achieve this.

\begin{remark}\label{rem:multiplication_identities}
  The reader is invited to verify that for all $x,y,z \in \QProc$ and $P \in \Proc$
  \begin{mathpar}
    x \cdot \quotep{0} \equiv x 
    \and
    x \cdot y \equiv y \cdot x
    \and
    x \cdot (y \cdot z) \equiv (x \cdot y) \cdot z
    \and \\
    \quotep{0} \cdot P \equiv P
    \and \\
    x \cdot (y \cdot P) \equiv (x \cdot y) \cdot P
    \and \\
    x \cdot (P|Q) \equiv (x \cdot P) | (x \cdot Q)
    \and \\    
  \end{mathpar}
\end{remark}

\subsubsection{Tensor product}

We define a tensor product on processes by structural induction.

\paragraph{Tensor of sums} First note that all summations, including
$\pzero$ and sequence, can be written $\Sigma_{i} x_{i}.A_{i} +
\Sigma_{j} x_{j}.C_{j}$, where we have grouped input-guarded processes
together and output-guarded processes together.

Thus, we can define the tensor product of two summations, $N_{1}\otimes N_{2}$, where

\begin{mathpar}
  N_{1} := \Sigma_{i} x_{i}.A_{i} + \Sigma_{j} x_{j}.C_{j}
  \and
  N_{2} := \Sigma_{i'} y_{i'}.B_{i'} + \Sigma_{j'} y_{j'}.D_{j'} 
\end{mathpar}

as follows.

\begin{mathpar}
  \Sigma_{i} x_{i}.A_{i} + \Sigma_{j} x_{j}.C_{j} \otimes \Sigma_{i'}
  y_{i'}.B_{i'} + \Sigma_{j'} y_{j'}.D_{j'} 
  \and \\
  := \; \Sigma_{i} \Sigma_{i'} \quotep{\stackrel{\vee}{x_{i}}| \stackrel{\vee}{y_{i'}}}.(A_{i}\otimes B_{i'}) \; | \; \Sigma_{i'} \Sigma_{i} \quotep{\stackrel{\vee}{y_{i'}}|\stackrel{\vee}{x_{i}}}.(B_{i'}\otimes A_{i})
  \and
  \;\; | \;\; \Sigma_{j} \Sigma_{j'} \quotep{\stackrel{\vee}{x_{j}}|\stackrel{\vee}{y_{j'}}}.(A_{j}\otimes B_{j'}) \; | \; \Sigma_{j'} \Sigma_{j} \quotep{\stackrel{\vee}{y_{j'}}|\stackrel{\vee}{x_{j}}}.(B_{j'}\otimes A_{j})
\end{mathpar}

\begin{remark}
  Do we need to $x^{L}$ and $y^{R}$ for this construction as well?
\end{remark}

\paragraph{Tensor of parallel compositions} Next, we distribute tensor
over par.

\begin{mathpar}
  P_{1}|P_{2} \otimes Q_{1}|Q_{2} := (P_{1} \otimes Q_{1}) | (P_{1}
  \otimes Q_{2}) | (P_{2} \otimes Q_{1}) | (P_{2} \otimes Q_{2})
\end{mathpar}

\paragraph{Tensor with dropped names} We treat tensor of a
process with a dropped name as parallel composition.

\begin{mathpar}
  P \otimes \dropn{x} := P | \dropn{x}
\end{mathpar}

\paragraph{Tensor of agents}

Finally, we need to define tensor on agents. Note that the definition
of tensor on normal products only tensors inputs with inputs and
outputs with outputs. Thus, we only have to define the operation on
``homogeneous'' pairings.

\begin{mathpar}
  (\vec{x})P \otimes (\vec{y})Q
  \and \\
  := (x_{0}^{L}|y_{0}^{R},\ldots,x_{0}^{L}|y_{n}^{R},\ldots,x_{m}^{L}|y_{0}^{R},\ldots,x_{m}^{L}|y_{n}^R)(P\{ \vec{x}^{L}/\vec{x}\} \otimes Q \{ \vec{y}^{R}/\vec{y}\})
  \and \\
  \clift{\vec{P}} \otimes \clift{\vec{Q}}
  \and \\
  := \clift{P_{0}\otimes Q_{0},\ldots,P_{0}\otimes Q_{n},\ldots,P_{m}\otimes Q_{0},\ldots,P_{m}\otimes Q_{n}}
\end{mathpar}

\begin{remark}
  Observe that arities of tensored abstractions matches arities of
  tensored concretions if the original arities matched. Note also that
  the length of the arities corresponds to the increase in dimension
  we see in ordinary vector space tensor product.
\end{remark}

\begin{remark}
  Operationally, this definition distributes the tensor down to
  components ``linked'' by summation. Tensor over summation is
  intriguing in that it mixes names. Moreover, as a consequence of the
  way it mixes names we have the identities for all $x \in \QProc$ and
  $P,Q \in \Proc$

  \begin{mathpar}
    (x \cdot P) \otimes Q \equiv x \cdot (P \otimes Q) \equiv P \otimes (x \cdot Q)
    \and
    P \otimes \pzero \equiv P
  \end{mathpar}

  that the reader is invited to verify.
\end{remark}

\subsubsection{Annihilation}
\begin{mathpar}
  P^{\perp} := \{ Q | \forall R. P|Q \red^{*} R \Rightarrow R \red^{*} \pzero \}
  \and \\
  P^{\underline{\perp}} := \Sigma_{Q \in P^{\perp}} \quotep{Q}?(y).(\dropn{y}|Q) | \Sigma_{Q \in P^{\perp}} \quotep{Q}\clift{\Box}
\end{mathpar}

\paragraph{Discussion} The reader will note that $P^{\perp}$ is a
\emph{set} of processes, while $P^{\underline{\perp}}$ is a
\emph{context}. We call the set $P^{\perp}$ the \emph{annihilators} of
$P$. The parallel composition of a process in the annihilators of $P$
with $P$ will result in a process, the state space of which has all
paths eventually leading to $\pzero$. Execution may endure loops; but
under reasonable conditions of fairness (naturally guaranteed under
most notions of bisimulation) such a composite process cannot get
stuck in such a loop and will, eventually pop out and terminate.

The context $P^{\underline{\perp}}$ is ready and willing to ``take the
$P$ out of'' the process to which it is applied. It will effectively
transmit the code of the process to which it is applied to one of the
annihilators and run the process against it.

\subsubsection{Evaluation}
We fix $M$ a domain of fully abstract interpretation with an equality
coincident with bisimulation. We take $\meaningof{\cdot} : \Proc \to
M$ to be the map interpreting processes and $\nmeaningof{\cdot} : \M
\to Proc$ to be the map running the other way. Then we define

\begin{mathpar}
  \int P := \nmeaningof{\meaningof{P}}
\end{mathpar}

\paragraph{Discussion}
There are many fully abstract interpretations of Milner's
$\pi$-calculus. Any of them can be used as a basis for interpreting
the reflective calculus here. Equipped with such a domain it is
largely a matter of grinding through to check that the Yoneda
construction for the normalization-by-evaluation program can be
extended to this setting.

\begin{remark}
  The reader is invited to verify that $\int (P^{\underline{\perp}}[P]) = 0$.
\end{remark}

\subsection{Quantum mechanics}

Table \ref{tbl:core_qm_op_defns} gives the core operational definitions

\begin{table}[htp]\label{tbl:core_qm_op_defns}
  \center{
    \fbox{
      \begin{tabular}{c|c}
        quantum mechanics & process calculus \\
        \hline
        scalar & $x := \quotep{P}$ \\
        state vector & $\state{P} := P$ \\
        dual & $\state{P}^{*} := \event{P^{\underline{\perp}}} := \quotep{P^{\underline{\perp}}}[-]$ \\
        matrix & $ \Sigma_{\alpha} \state{P_{\alpha}}x_{\alpha}\event{Q_{\alpha}}$ \\
        vector addition & $\state{P} + \state{Q} := \state{P | Q}$ \\
        tensor product & $\state{P} \otimes \state{Q} := \state{P \otimes Q}$ \\
        inner product & $\innerprod{P}{Q} := \quotep{\int P^{\underline{\perp}}[Q]}$ \\
      \end{tabular}
    }
  }
  \caption{QM - operational definitions}
\end{table}

where

\begin{mathpar}
  \prmatrix{P}{Q} := \fprmatrix{P}{\quotep{\pzero}}{Q}
  \and
  \fprmatrix{P}{x}{Q} := (\state{P},x,\event{Q})
  \and
  (\fprmatrix{P}{x}{Q})(\state{R}) := x \cdot \innerprod{Q}{R} \cdot \state{P}
  \and
  (\fprmatrix{P}{x}{Q})(\event{R}) := x \cdot \innerprod{R}{P} \cdot \event{Q}
\end{mathpar}

\paragraph{Discussion}
As promised: vectors (aka states) are represented as processes; duals
as contextual duals; inner product definition should be compared with
standard inner product definition for ....

\begin{remark}
  Assuming $\int (P^{\underline{\perp}}[P]) = 0$, the reader is
  invited to verify that $(\fprmatrix{P}{x}{P})(\state{P}) = x \cdot \state{P}$.
\end{remark}

\begin{remark}
  The reader is invited to verify that $\innerprod{P}{Q}$ could
  equally well have been written $\quotep{\int \stackrel{\vee}{x}}$
  where $x = \event{P^{\underline{\perp}}}(Q)$.

  One of the motivations for this remark is that there is another way
  to factor these operations. We could package up evaluation in the dual:

  \begin{mathpar}
    \state{P}^{*} := \event{\int P^{\underline{\perp}}} := \quotep{\int P^{\underline{\perp}}}[-]
  \end{mathpar}

  and then have inner product defined by
  
  \begin{mathpar}
    \innerprod{P}{Q} := \event{P}(Q)
  \end{mathpar}

  Hopefully, experience with the calculations will provide guidance on
  the best factoring.
\end{remark}

\begin{remark}
  Assuming $\int (P^{\underline{\perp}}[P]) = 0$, the reader is
  invited to verify that $\forall P,Q. (\prmatrix{0}{Q})(\state{0}) =
  \state{0}$ and dually $(\prmatrix{P}{0})(\event{0}) = \event{0}$.
\end{remark}

\begin{remark}
  i'm a little worried that i don't (yet) have proper support for
  complex conjugacy. But, the observation above may give us a
  clue. According to Abramsky, it must be the case that the scalars
  are iso to the homset of the identity for the tensor -- which the
  observation above characterizes. 

  For now, we will simply bookmark the notion with $\overline{x}$.
\end{remark}

\subsubsection{Adjointness}

We need to give a definition of $(\cdot)^{\dagger}$ for matrices. The
obvious candidate definition is
\begin{mathpar}
(\Sigma_{\alpha}\fprmatrix{P_{\alpha}}{x_{\alpha}}{Q_{\alpha}})^{\dagger}
= \Sigma_{\alpha}\fprmatrix{(Q_{\alpha}^{\underline{\perp}})^{*}}{\overline{x}_{\alpha}}{P_{\alpha}^{\underline{\perp}}} 
\end{mathpar}

But, $(Q_{\alpha}^{\underline{\perp}})^{*}$ requires a name along
which to communicate the process to achieve the context application.

\subsubsection{Basis for a basis}
If processes label states and ``addition'' of states (a.k.a. vector
addition) is interpreted as parallel composition, what corresponds to
notions of linear independence and basis? Here, we recall that Yoshida
has developed a set of \emph{combinators} for an asynchronous verison
of Milner's $\pi$-calculus. These are a finite set of processes such
any process can be expressed as parallel composition of these
combinators together with liberal uses of the new operator and
replication. We can simply give a translation of these into the
present calculus and have reasonable expectation that the property
carries over. That is, that the resultant set allows to express all
processes via parallel composition. Note, however, that there is no
new operator or replication in this calculus. As a result, we expect
that the corresponding set is actually infinite. That is, we expect
that the space is actually infinite dimensional.

\begin{remark}
  The attentive reader may be a bit concerned. Certainly, the
  collection $S$, $K$ and $I$ is a finite set of
  combinators. Shouldn't we expect to see a finite set of combinators
  for an effectively equivalent system? i am very sympathetic to this
  critique and feel it warrants full attention. On the other hand, i
  also have in mind the following analogy. The natural numbers, as a
  monoid under addition, has exactly $1$ generator, while the natural
  numbers, as a monoid under multiplication, has countably many
  generators (the primes). We observe that the application of the
  lambda calculus is much less resource sensitive than the parallel
  composition of the $\pi$-calculus. Could it be the case that we have
  an analogy of the form
  
  \begin{mathpar}
    m + n : MN :: m*n : M|N
  \end{mathpar}

  giving a similar blow up in the set of ``primes''?  This is such a
  wonderful thought that, even if it's not true, i think it's worth
  writing down.
\end{remark}
 

\documentclass[12pt]{llncs}
%\documentclass{jktr}

\usepackage[pdftex]{hyperref}                   
\usepackage {listings}
\usepackage {mathpartir}
\usepackage{bcprules}
%\usepackage{listings}
                       
\usepackage{graphicx} 
%\usepackage[margins=2.5cm,nohead,nofoot]{geometry}
%\usepackage{geometry}
\usepackage{amsfonts}
\usepackage{amstext}
\usepackage{latexsym}
\usepackage{amssymb}
\usepackage{color}


%\include{myPreamble}
\include{qm2pi.local} 

%\ifpdf
%\usepackage[pdftex]{graphicx}
%\else
%\usepackage{graphicx}
%\fi

 % \ifpdf
%  \usepackage{pdfsync}
%  \if


%\title{Brief Article}
%\author{David F. Snyder}
%\author{L.G. Meredith}

%\address{Dept. of Math., Texas State University--San Marcos, San Marcos, TX 78666}
       
\pagestyle{empty}


\begin{document}

\lstset{language=[Objective]Caml,frame=shadowbox}

\input{qm2pi.front}

% section front matter (end)

\input{qm2pi.intro} 
 
% section introduction (end)

% \input{qm2pi.knotations} 

% section notation (end)

\input{qm2pi.process.calculi} 

% section concurrent_process_calculi_and_spatial_logics_ (end)
    
%\input{qm2pi.knots2pi} 

%\input{qm2pi.trefoil} 

%\input{qm2pi.mainthm} 

% subsection basic_interpretation (end)

%\input{qm2pi.rho.presentation} 
\subsection{The syntax and semantics of the notation system}\label{sub:the_syntax_and_semantics_of_the_notation_system} % (fold)

We now summarize a technical presentation of the calculus that
embodies our theory of dynamics. The typical presentation of such a
calculus follows the style of giving generators and relations on
them. The grammar, below, describing term constructors, freely
generates the set of processes, $\Proc$. This set is then quotiented
by a relation known as structural congruence and it is over this set
that the notion of dynamics is expressed. This presentation is
essentially that of \cite{MeredithR05} with the addition of
polyadicity and summation. For readability we have relegated some of
the technical subtleties to an appendix.

\subsubsection{Process grammar}\label{subsub:process_grammar}

\begin{mathpar}
  \inferrule* [lab=synchronization] {} {{M} \bc \pzero \;|\; x?F \;|\; x!C }
  \and
  \inferrule* [lab=abstraction] {} {{F} \bc (x)P}
  \and
  \inferrule* [lab=concretion] {} {{C} \bc \langle Q \rangle}
  \and
  \inferrule* [lab=process] {} {{P,Q} \bc M \;| \;P|Q \;|\; @{x}}
  \and
  \inferrule* [lab=name] {} {{x} \bc \quotep{P}}
\end{mathpar} 

Note that $\vec{x}$ (resp. $\vec{P}$) denotes a vector of names
(resp. processes) of length $|\vec{x}|$ (resp. $|\vec{P}|$). We adopt
the following useful abbreviations.

\begin{mathpar}
   x?(\vec{y}).P := x.(\vec{y})P \and  x\clift{\vec{P}} := x.\clift{\vec{P}}
   \and x!(y) := \lift{x}{\dropn{y}}
   \and \Pi_{i=0}^{n-1}P_i := P_0 | \ldots | P_{n-1}
\end{mathpar}

\subsubsection{Structural congruence}

\paragraph{Free and bound names and alpha-equivalence.} At the
core of structural equivalence is alpha-equivalence which identifies
process that are the same up to a change of variable. Formally, we
recognize the distinction between free and bound names. The free names
of a process, $\freenames{P}$, may be calculated recursively as
follows:

\begin{mathpar}
\freenames{\pzero} := \emptyset
  \and \\
  \freenames{x?(y).P} := \{ x \} \cup (\freenames{P} \setminus \{ y \})
  \and 
  \freenames{x!\langle P \rangle} := \{ x \} \cup \{ P \} 
  \and \\
  \freenames{P|Q} := \freenames{P} \cup \freenames{Q}
  \and \\
  \freenames{@{x}} := \{ x \}
\end{mathpar}

$\pi$
$\quotep{\pi}$

$\freenames{-} : \pi \to \mathcal{P}(\quotep{\pi})$

\begin{eqnarray*}
  \freenames{\pzero} & := & \emptyset \\
  \freenames{x?(y).P} & := & \{ x \} \cup (\freenames{P} \setminus \{ y \}) \\
  \freenames{x!\langle P \rangle} & := & \{ x \} \cup \{ P \} \\
  \freenames{P|Q} & := & \freenames{P} \cup \freenames{Q} \\
  \freenames{\dropn{x}} & := & \{ x \}
\end{eqnarray*}

The bound names of a process, $\boundnames{P}$, are those names occurring in $P$
that are not free. For example, in $x?(y).0$, the name $x$ is free, while $y$ is bound.

\begin{mathpar}
  \inferrule* [lab=monoidal-laws] {} { P|Q \equiv Q|P \and P|0 \equiv P \and P|(Q|R) \equiv (P|Q)|R }
\end{mathpar}

\begin{mathpar}
  \inferrule* [lab=alpha-equivalence] {} { (x)P \equiv (y)P\{y/x\} \and y \not\in \freenames{P} }
\end{mathpar}

\begin{definition}
Then two processes, $P,Q$, are alpha-equivalent if $P = Q\{\vec{y}/\vec{x}\}$ for
some $\vec{x} \in \boundnames{Q},\vec{y} \in \boundnames{P}$, where $Q\{\vec{y}/\vec{x}\}$
denotes the capture-avoiding substitution of $\vec{y}$ for $\vec{x}$ in $Q$.
\end{definition}

\begin{definition}
  The {\em structural congruence} \cite{SangiorgiWalker} , $\equiv$,
  between processes is the least congruence containing
  alpha-equivalence, satisfying the abelian monoid laws
  (associativity, commutativity and $\pzero$ as identity) for parallel
  composition $|$ and for summation $+$.
\end{definition}

\subsection{Name equivalence}

We take name equivalence, written $\nameeq$, to be the smallest
equivalence relation generated by the following rules.

\begin{mathpar}
\inferrule*[lab=Quote-drop]
{ }
{ \quotep{@{x}} \nameeq x }

\inferrule*[lab=Struct-equiv]
{ P \scong Q }
{ \quotep{P} \nameeq \quotep{Q} }
\end{mathpar}

The astute reader will have noticed that the mutual recursion of names
and processes imposes a mutual recursion on alpha-equivalence and
structural equivalence via name-equivalence. Fortunately, all of this
works out pleasantly and we may calculate in the natural way, free of
concern. The reader interested in the details is referred to the
appendix \ref{appendix:rho_details}.

\subsection{Substitution}

We use $\Proc$ for the set of processes, $\QProc$ for the set of
names, and $\id{\{}\vec{y} / \vec{x} \id{\}}$ to denote partial maps,
$s : \QProc \rightarrow \QProc$. A map, $s$ lifts, uniquely, to a map
on process terms, $\widehat{s} : \Proc \rightarrow \Proc$ by the
following equations.

\begin{mathpar}
  (0) \psubstp{Q}{P} := 0 \\
  (R \juxtap S) \psubstp{Q}{P}
  :=    
  (R)\psubstp{Q}{P} \juxtap (S) \psubstp{Q}{P} \\
  (x?(y).R) \psubstp{Q}{P}    
  :=    
  (x)\substp{Q}{P} (z)\concat( (R \psubstn{z}{y}) \psubstp{Q}{P} ) \\
  (\lift{x}{R}) \psubstp{Q}{P}  
  :=
  \lift{(x)\substp{Q}{P}}{ R \psubstp{Q}{P} } \\
%   (\dropn{x})  \psubstp{Q}{P}       
%   := 
%   \left\{ 
%     \begin{array}{ccc} 
%       \dropn{\quotep{Q}} & & x \nameeq \quotep{P} \\
%       \dropn{x} & & otherwise \\
%     \end{array}
%   \right. 
  (\dropn{x})  \psubstp{Q}{P}       
  := 
  \left\{ 
    \begin{array}{ccc} 
      Q & & x \nameeq \quotep{P} \\
      \dropn{x} & & otherwise \\
    \end{array}
  \right.
\end{mathpar}
 

where

\begin{eqnarray}
  (x)\id{\{} \lpquote Q \rpquote / \lpquote P \rpquote \id{\}}            = 
  \left\{ 
    \begin{array}{ccc}
      \lpquote Q \rpquote & & x \nameeq \lpquote P \rpquote \\
      x & & otherwise \\
    \end{array}
  \right. \nonumber
\end{eqnarray}

and $z$ is chosen distinct from $\quotep{P}$, $\quotep{Q}$, the free
names in $Q$, and all the names in $R$. Our $\alpha$-equivalence will
be built in the standard way from this substitution.

\begin{remark}\label{rem:no_self_referential_names}
  One consequence of these definitions is that $\forall P. \quotep{P}
  \not\in \freenames{P}$.
\end{remark}

\subsection{ Dynamic quote: an example }

Anticipating something of what's to come, consider applying the
substitution, $\widehat{\id{\{}u / z \id{\}}}$, to the following pair
of processes, $\lift{w}{y!(z)}$ and $w[ \lpquote y!(z) \rpquote ]$.

\begin{eqnarray}
	\lift{w}{y!(z)}\widehat{\id{\{}u / z \id{\}}}
		& = &
		\lift{w}{y!(u)} \nonumber\\
	w[ \lpquote y!(z) \rpquote ] \widehat{ \id{\{}u / z \id{\}} }
		& = &
		w[ \lpquote y!(z) \rpquote ] \nonumber
\end{eqnarray}

Because the body of the process between quotes is impervious to
substitution, we get radically different answers. In fact, by
examining the first process in an input context,
e.g. $x?(z).\lift{w}{y!(z)}$, we see that the process under the lift
operator may be shaped by prefixed inputs binding a name inside it. In
this sense, the lift operator will be seen as a way to dynamically
construct processes before reifying them as names.

Finally equipped with these standard features we can present the
dynamics of the calculus.

\subsubsection{Operational semantics} 

Finally, we introduce the computational dynamics. What marks these
algebras as distinct from other more traditionally studied algebraic
structures, e.g. vector spaces or polynomial rings, is the manner in
which dynamics is captured. In traditional structures, dynamics is typically
expressed through morphisms between such structures, as in linear maps
between vector spaces or morphisms between rings. In algebras
associated with the semantics of computation, the dynamics is
expressed as part of the algebraic structure itself, through a
reduction reduction relation typically denoted by $\red$. Below, we
give a recursive presentation of this relation for the calculus used
in the encoding.

$\red \subseteq \pi \times \pi$
$\red : \pi \to \mathcal{P}(\pi)$

\begin{mathpar}
  \inferrule* [lab=Comm] { \textsf{match}( x_{src}, x_{trgt} ) } { x_{trgt}?(y)P \; | \; x_{src}!\langle {Q} \rangle \red P\{\quotep{Q}/y}\} }
  \and \\
  \inferrule* [lab=Par] {{P} \red {P}'} {{{P} | {Q}} \red {{P}' | {Q}}}
  \and
  \inferrule* [lab=Equiv]{{{P} \scong {P}'} \andalso {{P}' \red {Q}'} \andalso {{Q}' \scong {Q}}}{{P} \red {Q}}
\end{mathpar}

\begin{eqnarray*}
  match_{\equiv} (\quotep{P},\quotep{Q}) & := & P \equiv Q \\
  match_{\dagger}(\quotep{P},\quotep{Q}) & := & \forall R. P|Q \red^{*} R => R \red^{*} 0 \\
  match_{K}(\quotep{P},\quotep{Q}) & := & K \mbox{ for some context } K
\end{eqnarray*}

$u?(x)P | u!\langle Q \rangle \red P\{\quotep{Q}/x\}$

%We write $\wred$ for $\red^*$, and $P\red$ if $\exists Q $ such that $ P \red Q$.
We write $P\red$ if $\exists Q $ such that $ P \red Q$ and $P\not\red$, otherwise.

\section{Replication}

As mentioned before, it is known that replication (and hence
recursion) can be implemented in a higher-order process algebra
\cite{SangiorgiWalker}. As our first example of calculation with the
machinery thus far presented we give the construction explicitly in
the {\rhoc}.

\begin{eqnarray}
	D_{x} & := & \prefix{x}{y}{(\binpar{\outputp{x}{y}}{@{y}})} \nonumber\\
	\bangp_{x}{P} & := & \binpar{{x}!\langle{\binpar{D_{x}}{P}}\rangle}{D_{x}} \nonumber
\end{eqnarray}

\begin{eqnarray}
	\bangp_{x}{P} & & \nonumber\\
	=
	& {x}!\langle{(\prefix{x}{y}{(\outputp{x}{y} | @{y})) | P}}\rangle 
	      | \prefix{x}{y}{(\outputp{x}{y} | @{y})} & \nonumber\\
	\red
	& (\outputp{x}{y} | @{y})\substn{\quotep{(\prefix{x}{y}{(@{y} | \outputp{x}{y})) | P}}}{y} & \nonumber\\
	=
	& \outputp{x}{\quotep{(\prefix{x}{y}{(\outputp{x}{y} | @{y})) | P}}}
	  | {(\prefix{x}{y}{(\outputp{x}{y} | @{y})) | P}} & \nonumber\\
	\red
	& \ldots & \nonumber\\
	\red^*
	& P | P | \ldots & \nonumber
\end{eqnarray}

Of course, this encoding, as an implementation, runs away, unfolding
$\bangp{P}$ eagerly. A lazier and more implementable replication
operator, restricted to input-guarded processes, may be obtained as follows.

\begin{eqnarray}
\bangp{\prefix{u}{v}{P}} 
	:= 
	\binpar{\lift{x}{\prefix{u}{v}{(\binpar{D(x)}{P})}}}{D(x)} \nonumber
\end{eqnarray}

\begin{remark}
  Note that the lazier definition still does not deal with summation
  or mixed summation (i.e. sums over input and output). The reader is
  invited to construct definitions of replication that deal with these
  features. 

  Further, the definitions are parameterized in a name, $x$. Can you,
  gentle reader, make a definition that eliminates this parameter and
  guarantees no accidental interaction between the replication
  machinery and the process being replicated -- i.e. no accidental
  sharing of names used by the process to get its work done and the
  name(s) used by the replication to effect copying. This latter
  revision of the definition of replication is crucial to obtaining
  the expected identity $!!P \sim !P$.
\end{remark}

\begin{remark}\label{rem:paradoxical_combinator}
  The reader familiar with the lambda calculus will have noticed the
  similarity between $D$ and the paradoxical combinator.

  [Ed. note: the existence of this seems to suggest we have to be more
  restrictive on the set of processes and names we admit if we are to
  support no-cloning.]
\end{remark}

\subsubsection{Bisimulation}

The computational dynamics gives rise to another kind of equivalence,
the equivalence of computational behavior. As previously mentioned
this is typically captured \emph{via} some form of bisimulation.

% The notion we use in this paper is weak barbed bisimulation
% \cite{milner91polyadicpi}.

The notion we use in this paper is derived from weak barbed
bisimulation \cite{milner91polyadicpi}. 

\begin{definition}
An \emph{observation relation}, $\downarrow_{\mathcal N}$, over a set
of names, $\mathcal N$, is the smallest relation satisfying the rules
below.

\infrule[Out-barb]{y \in {\mathcal N}, \; x \nameeq y}
		  {\outputp{x}{v} \downarrow_{\mathcal N} x}
\infrule[Par-barb]{\mbox{$P\downarrow_{\mathcal N} x$ or $Q\downarrow_{\mathcal N} x$}}
		  {\binpar{P}{Q} \downarrow_{\mathcal N} x}

We write $P \Downarrow_{\mathcal N} x$ if there is $Q$ such that 
$P \wred Q$ and $Q \downarrow_{\mathcal N} x$.
\end{definition}

\begin{definition}
%\label{def.bbisim}
An  ${\mathcal N}$-\emph{barbed bisimulation} over a set of names, ${\mathcal N}$, is a symmetric binary relation 
${\mathcal S}_{\mathcal N}$ between agents such that $P\rel{S}_{\mathcal N}Q$ implies:
\begin{enumerate}
\item If $P \red P'$ then $Q \wred Q'$ and $P'\rel{S}_{\mathcal N} Q'$.
\item If $P\downarrow_{\mathcal N} x$, then $Q\Downarrow_{\mathcal N} x$.
\end{enumerate}
$P$ is ${\mathcal N}$-barbed bisimilar to $Q$, written
$P \wbbisim_{\mathcal N} Q$, if $P \rel{S}_{\mathcal N} Q$ for some ${\mathcal N}$-barbed bisimulation ${\mathcal S}_{\mathcal N}$.
\end{definition}

$\mathcal{R} \subseteq \pi \times \pi$

$P \mathcal{R} Q => \forall P'. P \red P' \Rightarrow \exists Q'. Q \red Q', P' \mathcal{R} Q'$

$P \vdash x \Rightarrow Q \vdash x$

\begin{mathpar}
  \inferrule*[lab=Out-barb]{x \nameeq y}{{y}!\langle{Q}\rangle \vdash x}
  \and
  \inferrule*[lab=Par-barb]{\mbox{$P\vdash x$ or $Q\vdash x$}}{\binpar{P}{Q} \vdash x}
\end{mathpar}

\subsubsection{Contexts}

One of the principle advantages of computational calculi like the
$\pi$-calculus is a well-defined notion of context,
contextual-equivalence and a correlation between
contextual-equivalence and notions of bisimulation. The notion of
context allows the decomposition of a process into (sub-)process and
its syntactic environment, its context. Thus, a context may be
thought of as a process with a ``hole'' (written $\Box$) in it. The
application of a context $M$ to a process $P$, written $M[P]$, is
tantamount to filling the hole in $M$ with $P$. In this paper we do
not need the full weight of this theory, but do make use of the notion
of context in the proof the main theorem. 

\begin{mathpar}
  \inferrule* [lab=summation] {} {{M_{M},M_{N}} \bc \Box \;|\; x.M_{A} \;|\; M_{M}+M_{N}}
  \and
  \inferrule* [lab=agent] {} {{M_{A}} \bc (\vec{x})M_{P} \;| \; \clift{P_0,\ldots,M_{P},\ldots,P_N}}
  \and \\
  \inferrule* [lab=process] {} {{M_{P}} \bc M_{N} \;| \;P|M_{P} }
\end{mathpar} 

\begin{mathpar}
  \inferrule* [lab=sychronization] {} {M_{N} \bc \Box \;|\; x?M_{F} \;|\; x!M_{C}}
  \and
  \inferrule* [lab=abstraction] {} {{M_{F}} \bc (x)M_{P} }
  \and
  \inferrule* [lab=concretion] {} {{M_{C}} \bc \langle M_{P} \rangle }
  \and \\
  \inferrule* [lab=process] {} {{M_{P}} \bc M_{N} \;| \;P|M_{P} }
\end{mathpar}

\begin{definition}[contextual application] Given a context $M$, and
  process $P$, we define the \emph{contextual application}, $M[P] :=
  M\{P/\Box\}$. That is, the contextual application of M to P is the
  substitution of $P$ for $\Box$ in $M$.
\end{definition}

$\meaningof{-} : L \to \mathcal{P}(\pi)$

\begin{mathpar}
  \inferrule* [lab=collection] {} {\meaningof{true} = \pi, \and \meaningof{~E} = \pi \setminus \meaningof{E}, \and \meaningof{E_{1} \& E_{2}} = \meaningof{E_{1}} \cap \meaningof{E_{2}}}
\end{mathpar}

\begin{mathpar}
  \inferrule* [lab=structure] {} {\meaningof{0} = \{ P \in \pi | P \equiv 0 \}, \and \\ \meaningof{E_1 | E_2} = \{ P \in \pi | P \equiv P_{1} | P_{2}, P_{1} \in \meaningof{E_{1}}, P_{2} \in \meaningof{E_2}\} }
\end{mathpar}

\begin{mathpar}
 \inferrule* [lab=behavior] {} {\meaningof{\langle a?b \rangle E} = \{ P \in \pi | P \equiv Q | u?(y)P', \\ \and \\\\ \and \\ \;\;\; u \in \meaningof{a}, \forall z.P'\{z/y\} \in \meaningof{E\{z/b\}}\}, \and \\ \meaningof{a!E} = \{ P \in \pi | P \equiv Q | x!\langle P' \rangle, x \in \meaningof{a} P' \in \meaningof{E}\} }
\end{mathpar}

\begin{mathpar}
 \inferrule* [lab=nominal] {} {\meaningof{\quotep{E}} = \{ \quotep{P} \in \quotep{\pi} | P \in \meaningof{E} \}, \and \meaningof{\quotep{P}} = \{ \quotep{Q} \in \quotep{\pi} | P \equiv Q \} \and \\ \meaningof{@\quotep{E}} = \{ P \in \pi | P \equiv @x, x \in \meaningof{E} \}}
\end{mathpar}

\begin{eqnarray*}
  \\
  \meaningof{-} : TS \to ST
\end{eqnarray*}

\begin{eqnarray*}
  \\
  L : TS \to ST
\end{eqnarray*}

\begin{eqnarray*}
  \\
  P \models E \iff P \in \meaningof{E}
\end{eqnarray*}

\begin{eqnarray*}
  P \approx_{L} Q \iff \forall E \in L. P \models E \iff Q \models E
\end{eqnarray*}

\begin{eqnarray*}
  P \approx_{K} Q
\end{eqnarray*}

\begin{eqnarray*}
  P \approx Q
\end{eqnarray*}

$\approx_{K} = \approx = \approx_{L}$

\subsubsection{Contextual duality}

Note that contexts extend the quotation operation to a family of
operations from processes to names. Given a context, $M$, we can
define a \emph{nominal context}, $\quotep{M}$ by $\quotep{M}[P] :=
\quotep{M[P]}$. To foreshadow what is to come we observe that these
operations enjoy a duality with processes very much like the duality
between vectors and maps from vectors to scalars.

Further, because the calculus is essentially higher-order, we have a
correspondence between contexts and processes. More specifically,
given a name $x$ and a context $M$ we can construct $M^{*}_{x}$ such
that 

\begin{mathpar}
  M^{*}_{x} | \lift{x}{P} \red M[P]
\end{mathpar}

namely,

\begin{mathpar}
  M^{*}_{x} := x?(u).M[\dropn{u}]
\end{mathpar}

The dependence of $M^{*}_{x}$ on a name makes it an abstraction, 

\begin{mathpar}
  M^{*} := (x)x?(u).M[\dropn{u}]
\end{mathpar}

\subsection{Additional notation}

It will sometimes be convenient to denote the process a name
quotes. We already have the notation $x = \quotep{P}$, but it will be
convenient to introduce an alternate notation, $\procn{x}$, when we
want to emphasize the connection to the use of the name. Note that, by
virtue of name equivalence, $\quotep{\procn{x}} \nameeq x$; so, the
notation is consistent with previous definitions.

Further, because names have structure it is possible to effect
substitutions on the basis of that structure. This means we need to
upgrade our notation for substitutions, which we accomplish by
adapting comprehension notation. Thus,

\begin{mathpar}
  P\{ y / x : x \in S \}
\end{mathpar}

is interpreted to mean the process derived from P by replacing (in a
capture-avoiding manner) each occurrence of $x$ in $S$ by $y$. For example,

\begin{mathpar}
  P\{ \quotep{\procn{x}|\procn{x}} / x : x \in \freenames{P} \}
\end{mathpar}

will replace each (occurrence) of a free name $x$ in $P$ by
$\quotep{\procn{x}|\procn{x}}$.

Also, we will avail ourselves of the notation $x^{L}$ and $x^{R}$ to
denote injections of a name into disjoint copies of the name
space. There are numerous ways to accomplish this. One example can be
found in \cite{MeredithR05}. This notation overloads to vectors of
names: $\vec{x}^{\pi} := (x_{i}^{\pi} \; : \; 0 \leq i < |\vec{x}| )$ where $\pi \in \{L,R\}$.

We also use $P^{\Box} := P|\Box$.

In \cite{MeredithR05} an interpretation of the new operator is
given. It turns out that there are several possible interpretations
all enjoying the requisite algebraic properties of the operator (see
\cite{milner91polyadicpi}). We will therefore make liberal use of
$(\nu\; \vec{x})P$.

% subsection the_syntax_and_semantics_of_the_notation_system (end)   

\input{qm2pi.qmops} 

\input{qm2pi.sterngerlach} 

\input{qm2pi.metric} 

% section concurrent_process_calculi (end)

%\input{qm2pi.proofsketch}

% section proof sketch (end)

%\input{qm2pi.slviaknots} 

% section spatial logic via knots (end)

\input{qm2pi.conclusion}

% section conclusion (end)

%\input{qm2pi.dtcodes} 

% section wiring algorithm (end)

\input{qm2pi.ack} 

% section acknowledgments (end)

\newpage


\bibliographystyle{plain}   
\bibliography{../../biblios/main.bib}

\input{qm2pi.rhodetails}

\end{document}

 

\documentclass[12pt]{llncs}
%\documentclass{jktr}

\usepackage[pdftex]{hyperref}                   
\usepackage {listings}
\usepackage {mathpartir}
\usepackage{bcprules}
%\usepackage{listings}
                       
\usepackage{graphicx} 
%\usepackage[margins=2.5cm,nohead,nofoot]{geometry}
%\usepackage{geometry}
\usepackage{amsfonts}
\usepackage{amstext}
\usepackage{latexsym}
\usepackage{amssymb}
\usepackage{color}


%\include{myPreamble}
\include{qm2pi.local} 

%\ifpdf
%\usepackage[pdftex]{graphicx}
%\else
%\usepackage{graphicx}
%\fi

 % \ifpdf
%  \usepackage{pdfsync}
%  \if


%\title{Brief Article}
%\author{David F. Snyder}
%\author{L.G. Meredith}

%\address{Dept. of Math., Texas State University--San Marcos, San Marcos, TX 78666}
       
\pagestyle{empty}


\begin{document}

\lstset{language=[Objective]Caml,frame=shadowbox}

\input{qm2pi.front}

% section front matter (end)

\input{qm2pi.intro} 
 
% section introduction (end)

% \input{qm2pi.knotations} 

% section notation (end)

\input{qm2pi.process.calculi} 

% section concurrent_process_calculi_and_spatial_logics_ (end)
    
%\input{qm2pi.knots2pi} 

%\input{qm2pi.trefoil} 

%\input{qm2pi.mainthm} 

% subsection basic_interpretation (end)

%\input{qm2pi.rho.presentation} 
\subsection{The syntax and semantics of the notation system}\label{sub:the_syntax_and_semantics_of_the_notation_system} % (fold)

We now summarize a technical presentation of the calculus that
embodies our theory of dynamics. The typical presentation of such a
calculus follows the style of giving generators and relations on
them. The grammar, below, describing term constructors, freely
generates the set of processes, $\Proc$. This set is then quotiented
by a relation known as structural congruence and it is over this set
that the notion of dynamics is expressed. This presentation is
essentially that of \cite{MeredithR05} with the addition of
polyadicity and summation. For readability we have relegated some of
the technical subtleties to an appendix.

\subsubsection{Process grammar}\label{subsub:process_grammar}

\begin{mathpar}
  \inferrule* [lab=synchronization] {} {{M} \bc \pzero \;|\; x?F \;|\; x!C }
  \and
  \inferrule* [lab=abstraction] {} {{F} \bc (x)P}
  \and
  \inferrule* [lab=concretion] {} {{C} \bc \langle Q \rangle}
  \and
  \inferrule* [lab=process] {} {{P,Q} \bc M \;| \;P|Q \;|\; @{x}}
  \and
  \inferrule* [lab=name] {} {{x} \bc \quotep{P}}
\end{mathpar} 

Note that $\vec{x}$ (resp. $\vec{P}$) denotes a vector of names
(resp. processes) of length $|\vec{x}|$ (resp. $|\vec{P}|$). We adopt
the following useful abbreviations.

\begin{mathpar}
   x?(\vec{y}).P := x.(\vec{y})P \and  x\clift{\vec{P}} := x.\clift{\vec{P}}
   \and x!(y) := \lift{x}{\dropn{y}}
   \and \Pi_{i=0}^{n-1}P_i := P_0 | \ldots | P_{n-1}
\end{mathpar}

\subsubsection{Structural congruence}

\paragraph{Free and bound names and alpha-equivalence.} At the
core of structural equivalence is alpha-equivalence which identifies
process that are the same up to a change of variable. Formally, we
recognize the distinction between free and bound names. The free names
of a process, $\freenames{P}$, may be calculated recursively as
follows:

\begin{mathpar}
\freenames{\pzero} := \emptyset
  \and \\
  \freenames{x?(y).P} := \{ x \} \cup (\freenames{P} \setminus \{ y \})
  \and 
  \freenames{x!\langle P \rangle} := \{ x \} \cup \{ P \} 
  \and \\
  \freenames{P|Q} := \freenames{P} \cup \freenames{Q}
  \and \\
  \freenames{@{x}} := \{ x \}
\end{mathpar}

$\pi$
$\quotep{\pi}$

$\freenames{-} : \pi \to \mathcal{P}(\quotep{\pi})$

\begin{eqnarray*}
  \freenames{\pzero} & := & \emptyset \\
  \freenames{x?(y).P} & := & \{ x \} \cup (\freenames{P} \setminus \{ y \}) \\
  \freenames{x!\langle P \rangle} & := & \{ x \} \cup \{ P \} \\
  \freenames{P|Q} & := & \freenames{P} \cup \freenames{Q} \\
  \freenames{\dropn{x}} & := & \{ x \}
\end{eqnarray*}

The bound names of a process, $\boundnames{P}$, are those names occurring in $P$
that are not free. For example, in $x?(y).0$, the name $x$ is free, while $y$ is bound.

\begin{mathpar}
  \inferrule* [lab=monoidal-laws] {} { P|Q \equiv Q|P \and P|0 \equiv P \and P|(Q|R) \equiv (P|Q)|R }
\end{mathpar}

\begin{mathpar}
  \inferrule* [lab=alpha-equivalence] {} { (x)P \equiv (y)P\{y/x\} \and y \not\in \freenames{P} }
\end{mathpar}

\begin{definition}
Then two processes, $P,Q$, are alpha-equivalent if $P = Q\{\vec{y}/\vec{x}\}$ for
some $\vec{x} \in \boundnames{Q},\vec{y} \in \boundnames{P}$, where $Q\{\vec{y}/\vec{x}\}$
denotes the capture-avoiding substitution of $\vec{y}$ for $\vec{x}$ in $Q$.
\end{definition}

\begin{definition}
  The {\em structural congruence} \cite{SangiorgiWalker} , $\equiv$,
  between processes is the least congruence containing
  alpha-equivalence, satisfying the abelian monoid laws
  (associativity, commutativity and $\pzero$ as identity) for parallel
  composition $|$ and for summation $+$.
\end{definition}

\subsection{Name equivalence}

We take name equivalence, written $\nameeq$, to be the smallest
equivalence relation generated by the following rules.

\begin{mathpar}
\inferrule*[lab=Quote-drop]
{ }
{ \quotep{@{x}} \nameeq x }

\inferrule*[lab=Struct-equiv]
{ P \scong Q }
{ \quotep{P} \nameeq \quotep{Q} }
\end{mathpar}

The astute reader will have noticed that the mutual recursion of names
and processes imposes a mutual recursion on alpha-equivalence and
structural equivalence via name-equivalence. Fortunately, all of this
works out pleasantly and we may calculate in the natural way, free of
concern. The reader interested in the details is referred to the
appendix \ref{appendix:rho_details}.

\subsection{Substitution}

We use $\Proc$ for the set of processes, $\QProc$ for the set of
names, and $\id{\{}\vec{y} / \vec{x} \id{\}}$ to denote partial maps,
$s : \QProc \rightarrow \QProc$. A map, $s$ lifts, uniquely, to a map
on process terms, $\widehat{s} : \Proc \rightarrow \Proc$ by the
following equations.

\begin{mathpar}
  (0) \psubstp{Q}{P} := 0 \\
  (R \juxtap S) \psubstp{Q}{P}
  :=    
  (R)\psubstp{Q}{P} \juxtap (S) \psubstp{Q}{P} \\
  (x?(y).R) \psubstp{Q}{P}    
  :=    
  (x)\substp{Q}{P} (z)\concat( (R \psubstn{z}{y}) \psubstp{Q}{P} ) \\
  (\lift{x}{R}) \psubstp{Q}{P}  
  :=
  \lift{(x)\substp{Q}{P}}{ R \psubstp{Q}{P} } \\
%   (\dropn{x})  \psubstp{Q}{P}       
%   := 
%   \left\{ 
%     \begin{array}{ccc} 
%       \dropn{\quotep{Q}} & & x \nameeq \quotep{P} \\
%       \dropn{x} & & otherwise \\
%     \end{array}
%   \right. 
  (\dropn{x})  \psubstp{Q}{P}       
  := 
  \left\{ 
    \begin{array}{ccc} 
      Q & & x \nameeq \quotep{P} \\
      \dropn{x} & & otherwise \\
    \end{array}
  \right.
\end{mathpar}
 

where

\begin{eqnarray}
  (x)\id{\{} \lpquote Q \rpquote / \lpquote P \rpquote \id{\}}            = 
  \left\{ 
    \begin{array}{ccc}
      \lpquote Q \rpquote & & x \nameeq \lpquote P \rpquote \\
      x & & otherwise \\
    \end{array}
  \right. \nonumber
\end{eqnarray}

and $z$ is chosen distinct from $\quotep{P}$, $\quotep{Q}$, the free
names in $Q$, and all the names in $R$. Our $\alpha$-equivalence will
be built in the standard way from this substitution.

\begin{remark}\label{rem:no_self_referential_names}
  One consequence of these definitions is that $\forall P. \quotep{P}
  \not\in \freenames{P}$.
\end{remark}

\subsection{ Dynamic quote: an example }

Anticipating something of what's to come, consider applying the
substitution, $\widehat{\id{\{}u / z \id{\}}}$, to the following pair
of processes, $\lift{w}{y!(z)}$ and $w[ \lpquote y!(z) \rpquote ]$.

\begin{eqnarray}
	\lift{w}{y!(z)}\widehat{\id{\{}u / z \id{\}}}
		& = &
		\lift{w}{y!(u)} \nonumber\\
	w[ \lpquote y!(z) \rpquote ] \widehat{ \id{\{}u / z \id{\}} }
		& = &
		w[ \lpquote y!(z) \rpquote ] \nonumber
\end{eqnarray}

Because the body of the process between quotes is impervious to
substitution, we get radically different answers. In fact, by
examining the first process in an input context,
e.g. $x?(z).\lift{w}{y!(z)}$, we see that the process under the lift
operator may be shaped by prefixed inputs binding a name inside it. In
this sense, the lift operator will be seen as a way to dynamically
construct processes before reifying them as names.

Finally equipped with these standard features we can present the
dynamics of the calculus.

\subsubsection{Operational semantics} 

Finally, we introduce the computational dynamics. What marks these
algebras as distinct from other more traditionally studied algebraic
structures, e.g. vector spaces or polynomial rings, is the manner in
which dynamics is captured. In traditional structures, dynamics is typically
expressed through morphisms between such structures, as in linear maps
between vector spaces or morphisms between rings. In algebras
associated with the semantics of computation, the dynamics is
expressed as part of the algebraic structure itself, through a
reduction reduction relation typically denoted by $\red$. Below, we
give a recursive presentation of this relation for the calculus used
in the encoding.

$\red \subseteq \pi \times \pi$
$\red : \pi \to \mathcal{P}(\pi)$

\begin{mathpar}
  \inferrule* [lab=Comm] { \textsf{match}( x_{src}, x_{trgt} ) } { x_{trgt}?(y)P \; | \; x_{src}!\langle {Q} \rangle \red P\{\quotep{Q}/y}\} }
  \and \\
  \inferrule* [lab=Par] {{P} \red {P}'} {{{P} | {Q}} \red {{P}' | {Q}}}
  \and
  \inferrule* [lab=Equiv]{{{P} \scong {P}'} \andalso {{P}' \red {Q}'} \andalso {{Q}' \scong {Q}}}{{P} \red {Q}}
\end{mathpar}

\begin{eqnarray*}
  match_{\equiv} (\quotep{P},\quotep{Q}) & := & P \equiv Q \\
  match_{\dagger}(\quotep{P},\quotep{Q}) & := & \forall R. P|Q \red^{*} R => R \red^{*} 0 \\
  match_{K}(\quotep{P},\quotep{Q}) & := & K \mbox{ for some context } K
\end{eqnarray*}

$u?(x)P | u!\langle Q \rangle \red P\{\quotep{Q}/x\}$

%We write $\wred$ for $\red^*$, and $P\red$ if $\exists Q $ such that $ P \red Q$.
We write $P\red$ if $\exists Q $ such that $ P \red Q$ and $P\not\red$, otherwise.

\section{Replication}

As mentioned before, it is known that replication (and hence
recursion) can be implemented in a higher-order process algebra
\cite{SangiorgiWalker}. As our first example of calculation with the
machinery thus far presented we give the construction explicitly in
the {\rhoc}.

\begin{eqnarray}
	D_{x} & := & \prefix{x}{y}{(\binpar{\outputp{x}{y}}{@{y}})} \nonumber\\
	\bangp_{x}{P} & := & \binpar{{x}!\langle{\binpar{D_{x}}{P}}\rangle}{D_{x}} \nonumber
\end{eqnarray}

\begin{eqnarray}
	\bangp_{x}{P} & & \nonumber\\
	=
	& {x}!\langle{(\prefix{x}{y}{(\outputp{x}{y} | @{y})) | P}}\rangle 
	      | \prefix{x}{y}{(\outputp{x}{y} | @{y})} & \nonumber\\
	\red
	& (\outputp{x}{y} | @{y})\substn{\quotep{(\prefix{x}{y}{(@{y} | \outputp{x}{y})) | P}}}{y} & \nonumber\\
	=
	& \outputp{x}{\quotep{(\prefix{x}{y}{(\outputp{x}{y} | @{y})) | P}}}
	  | {(\prefix{x}{y}{(\outputp{x}{y} | @{y})) | P}} & \nonumber\\
	\red
	& \ldots & \nonumber\\
	\red^*
	& P | P | \ldots & \nonumber
\end{eqnarray}

Of course, this encoding, as an implementation, runs away, unfolding
$\bangp{P}$ eagerly. A lazier and more implementable replication
operator, restricted to input-guarded processes, may be obtained as follows.

\begin{eqnarray}
\bangp{\prefix{u}{v}{P}} 
	:= 
	\binpar{\lift{x}{\prefix{u}{v}{(\binpar{D(x)}{P})}}}{D(x)} \nonumber
\end{eqnarray}

\begin{remark}
  Note that the lazier definition still does not deal with summation
  or mixed summation (i.e. sums over input and output). The reader is
  invited to construct definitions of replication that deal with these
  features. 

  Further, the definitions are parameterized in a name, $x$. Can you,
  gentle reader, make a definition that eliminates this parameter and
  guarantees no accidental interaction between the replication
  machinery and the process being replicated -- i.e. no accidental
  sharing of names used by the process to get its work done and the
  name(s) used by the replication to effect copying. This latter
  revision of the definition of replication is crucial to obtaining
  the expected identity $!!P \sim !P$.
\end{remark}

\begin{remark}\label{rem:paradoxical_combinator}
  The reader familiar with the lambda calculus will have noticed the
  similarity between $D$ and the paradoxical combinator.

  [Ed. note: the existence of this seems to suggest we have to be more
  restrictive on the set of processes and names we admit if we are to
  support no-cloning.]
\end{remark}

\subsubsection{Bisimulation}

The computational dynamics gives rise to another kind of equivalence,
the equivalence of computational behavior. As previously mentioned
this is typically captured \emph{via} some form of bisimulation.

% The notion we use in this paper is weak barbed bisimulation
% \cite{milner91polyadicpi}.

The notion we use in this paper is derived from weak barbed
bisimulation \cite{milner91polyadicpi}. 

\begin{definition}
An \emph{observation relation}, $\downarrow_{\mathcal N}$, over a set
of names, $\mathcal N$, is the smallest relation satisfying the rules
below.

\infrule[Out-barb]{y \in {\mathcal N}, \; x \nameeq y}
		  {\outputp{x}{v} \downarrow_{\mathcal N} x}
\infrule[Par-barb]{\mbox{$P\downarrow_{\mathcal N} x$ or $Q\downarrow_{\mathcal N} x$}}
		  {\binpar{P}{Q} \downarrow_{\mathcal N} x}

We write $P \Downarrow_{\mathcal N} x$ if there is $Q$ such that 
$P \wred Q$ and $Q \downarrow_{\mathcal N} x$.
\end{definition}

\begin{definition}
%\label{def.bbisim}
An  ${\mathcal N}$-\emph{barbed bisimulation} over a set of names, ${\mathcal N}$, is a symmetric binary relation 
${\mathcal S}_{\mathcal N}$ between agents such that $P\rel{S}_{\mathcal N}Q$ implies:
\begin{enumerate}
\item If $P \red P'$ then $Q \wred Q'$ and $P'\rel{S}_{\mathcal N} Q'$.
\item If $P\downarrow_{\mathcal N} x$, then $Q\Downarrow_{\mathcal N} x$.
\end{enumerate}
$P$ is ${\mathcal N}$-barbed bisimilar to $Q$, written
$P \wbbisim_{\mathcal N} Q$, if $P \rel{S}_{\mathcal N} Q$ for some ${\mathcal N}$-barbed bisimulation ${\mathcal S}_{\mathcal N}$.
\end{definition}

$\mathcal{R} \subseteq \pi \times \pi$

$P \mathcal{R} Q => \forall P'. P \red P' \Rightarrow \exists Q'. Q \red Q', P' \mathcal{R} Q'$

$P \vdash x \Rightarrow Q \vdash x$

\begin{mathpar}
  \inferrule*[lab=Out-barb]{x \nameeq y}{{y}!\langle{Q}\rangle \vdash x}
  \and
  \inferrule*[lab=Par-barb]{\mbox{$P\vdash x$ or $Q\vdash x$}}{\binpar{P}{Q} \vdash x}
\end{mathpar}

\subsubsection{Contexts}

One of the principle advantages of computational calculi like the
$\pi$-calculus is a well-defined notion of context,
contextual-equivalence and a correlation between
contextual-equivalence and notions of bisimulation. The notion of
context allows the decomposition of a process into (sub-)process and
its syntactic environment, its context. Thus, a context may be
thought of as a process with a ``hole'' (written $\Box$) in it. The
application of a context $M$ to a process $P$, written $M[P]$, is
tantamount to filling the hole in $M$ with $P$. In this paper we do
not need the full weight of this theory, but do make use of the notion
of context in the proof the main theorem. 

\begin{mathpar}
  \inferrule* [lab=summation] {} {{M_{M},M_{N}} \bc \Box \;|\; x.M_{A} \;|\; M_{M}+M_{N}}
  \and
  \inferrule* [lab=agent] {} {{M_{A}} \bc (\vec{x})M_{P} \;| \; \clift{P_0,\ldots,M_{P},\ldots,P_N}}
  \and \\
  \inferrule* [lab=process] {} {{M_{P}} \bc M_{N} \;| \;P|M_{P} }
\end{mathpar} 

\begin{mathpar}
  \inferrule* [lab=sychronization] {} {M_{N} \bc \Box \;|\; x?M_{F} \;|\; x!M_{C}}
  \and
  \inferrule* [lab=abstraction] {} {{M_{F}} \bc (x)M_{P} }
  \and
  \inferrule* [lab=concretion] {} {{M_{C}} \bc \langle M_{P} \rangle }
  \and \\
  \inferrule* [lab=process] {} {{M_{P}} \bc M_{N} \;| \;P|M_{P} }
\end{mathpar}

\begin{definition}[contextual application] Given a context $M$, and
  process $P$, we define the \emph{contextual application}, $M[P] :=
  M\{P/\Box\}$. That is, the contextual application of M to P is the
  substitution of $P$ for $\Box$ in $M$.
\end{definition}

$\meaningof{-} : L \to \mathcal{P}(\pi)$

\begin{mathpar}
  \inferrule* [lab=collection] {} {\meaningof{true} = \pi, \and \meaningof{~E} = \pi \setminus \meaningof{E}, \and \meaningof{E_{1} \& E_{2}} = \meaningof{E_{1}} \cap \meaningof{E_{2}}}
\end{mathpar}

\begin{mathpar}
  \inferrule* [lab=structure] {} {\meaningof{0} = \{ P \in \pi | P \equiv 0 \}, \and \\ \meaningof{E_1 | E_2} = \{ P \in \pi | P \equiv P_{1} | P_{2}, P_{1} \in \meaningof{E_{1}}, P_{2} \in \meaningof{E_2}\} }
\end{mathpar}

\begin{mathpar}
 \inferrule* [lab=behavior] {} {\meaningof{\langle a?b \rangle E} = \{ P \in \pi | P \equiv Q | u?(y)P', \\ \and \\\\ \and \\ \;\;\; u \in \meaningof{a}, \forall z.P'\{z/y\} \in \meaningof{E\{z/b\}}\}, \and \\ \meaningof{a!E} = \{ P \in \pi | P \equiv Q | x!\langle P' \rangle, x \in \meaningof{a} P' \in \meaningof{E}\} }
\end{mathpar}

\begin{mathpar}
 \inferrule* [lab=nominal] {} {\meaningof{\quotep{E}} = \{ \quotep{P} \in \quotep{\pi} | P \in \meaningof{E} \}, \and \meaningof{\quotep{P}} = \{ \quotep{Q} \in \quotep{\pi} | P \equiv Q \} \and \\ \meaningof{@\quotep{E}} = \{ P \in \pi | P \equiv @x, x \in \meaningof{E} \}}
\end{mathpar}

\begin{eqnarray*}
  \\
  \meaningof{-} : TS \to ST
\end{eqnarray*}

\begin{eqnarray*}
  \\
  L : TS \to ST
\end{eqnarray*}

\begin{eqnarray*}
  \\
  P \models E \iff P \in \meaningof{E}
\end{eqnarray*}

\begin{eqnarray*}
  P \approx_{L} Q \iff \forall E \in L. P \models E \iff Q \models E
\end{eqnarray*}

\begin{eqnarray*}
  P \approx_{K} Q
\end{eqnarray*}

\begin{eqnarray*}
  P \approx Q
\end{eqnarray*}

$\approx_{K} = \approx = \approx_{L}$

\subsubsection{Contextual duality}

Note that contexts extend the quotation operation to a family of
operations from processes to names. Given a context, $M$, we can
define a \emph{nominal context}, $\quotep{M}$ by $\quotep{M}[P] :=
\quotep{M[P]}$. To foreshadow what is to come we observe that these
operations enjoy a duality with processes very much like the duality
between vectors and maps from vectors to scalars.

Further, because the calculus is essentially higher-order, we have a
correspondence between contexts and processes. More specifically,
given a name $x$ and a context $M$ we can construct $M^{*}_{x}$ such
that 

\begin{mathpar}
  M^{*}_{x} | \lift{x}{P} \red M[P]
\end{mathpar}

namely,

\begin{mathpar}
  M^{*}_{x} := x?(u).M[\dropn{u}]
\end{mathpar}

The dependence of $M^{*}_{x}$ on a name makes it an abstraction, 

\begin{mathpar}
  M^{*} := (x)x?(u).M[\dropn{u}]
\end{mathpar}

\subsection{Additional notation}

It will sometimes be convenient to denote the process a name
quotes. We already have the notation $x = \quotep{P}$, but it will be
convenient to introduce an alternate notation, $\procn{x}$, when we
want to emphasize the connection to the use of the name. Note that, by
virtue of name equivalence, $\quotep{\procn{x}} \nameeq x$; so, the
notation is consistent with previous definitions.

Further, because names have structure it is possible to effect
substitutions on the basis of that structure. This means we need to
upgrade our notation for substitutions, which we accomplish by
adapting comprehension notation. Thus,

\begin{mathpar}
  P\{ y / x : x \in S \}
\end{mathpar}

is interpreted to mean the process derived from P by replacing (in a
capture-avoiding manner) each occurrence of $x$ in $S$ by $y$. For example,

\begin{mathpar}
  P\{ \quotep{\procn{x}|\procn{x}} / x : x \in \freenames{P} \}
\end{mathpar}

will replace each (occurrence) of a free name $x$ in $P$ by
$\quotep{\procn{x}|\procn{x}}$.

Also, we will avail ourselves of the notation $x^{L}$ and $x^{R}$ to
denote injections of a name into disjoint copies of the name
space. There are numerous ways to accomplish this. One example can be
found in \cite{MeredithR05}. This notation overloads to vectors of
names: $\vec{x}^{\pi} := (x_{i}^{\pi} \; : \; 0 \leq i < |\vec{x}| )$ where $\pi \in \{L,R\}$.

We also use $P^{\Box} := P|\Box$.

In \cite{MeredithR05} an interpretation of the new operator is
given. It turns out that there are several possible interpretations
all enjoying the requisite algebraic properties of the operator (see
\cite{milner91polyadicpi}). We will therefore make liberal use of
$(\nu\; \vec{x})P$.

% subsection the_syntax_and_semantics_of_the_notation_system (end)   

\input{qm2pi.qmops} 

\input{qm2pi.sterngerlach} 

\input{qm2pi.metric} 

% section concurrent_process_calculi (end)

%\input{qm2pi.proofsketch}

% section proof sketch (end)

%\input{qm2pi.slviaknots} 

% section spatial logic via knots (end)

\input{qm2pi.conclusion}

% section conclusion (end)

%\input{qm2pi.dtcodes} 

% section wiring algorithm (end)

\input{qm2pi.ack} 

% section acknowledgments (end)

\newpage


\bibliographystyle{plain}   
\bibliography{../../biblios/main.bib}

\input{qm2pi.rhodetails}

\end{document}

 

% section concurrent_process_calculi (end)

%\documentclass[12pt]{llncs}
%\documentclass{jktr}

\usepackage[pdftex]{hyperref}                   
\usepackage {listings}
\usepackage {mathpartir}
\usepackage{bcprules}
%\usepackage{listings}
                       
\usepackage{graphicx} 
%\usepackage[margins=2.5cm,nohead,nofoot]{geometry}
%\usepackage{geometry}
\usepackage{amsfonts}
\usepackage{amstext}
\usepackage{latexsym}
\usepackage{amssymb}
\usepackage{color}


%\include{myPreamble}
\include{qm2pi.local} 

%\ifpdf
%\usepackage[pdftex]{graphicx}
%\else
%\usepackage{graphicx}
%\fi

 % \ifpdf
%  \usepackage{pdfsync}
%  \if


%\title{Brief Article}
%\author{David F. Snyder}
%\author{L.G. Meredith}

%\address{Dept. of Math., Texas State University--San Marcos, San Marcos, TX 78666}
       
\pagestyle{empty}


\begin{document}

\lstset{language=[Objective]Caml,frame=shadowbox}

\input{qm2pi.front}

% section front matter (end)

\input{qm2pi.intro} 
 
% section introduction (end)

% \input{qm2pi.knotations} 

% section notation (end)

\input{qm2pi.process.calculi} 

% section concurrent_process_calculi_and_spatial_logics_ (end)
    
%\input{qm2pi.knots2pi} 

%\input{qm2pi.trefoil} 

%\input{qm2pi.mainthm} 

% subsection basic_interpretation (end)

%\input{qm2pi.rho.presentation} 
\subsection{The syntax and semantics of the notation system}\label{sub:the_syntax_and_semantics_of_the_notation_system} % (fold)

We now summarize a technical presentation of the calculus that
embodies our theory of dynamics. The typical presentation of such a
calculus follows the style of giving generators and relations on
them. The grammar, below, describing term constructors, freely
generates the set of processes, $\Proc$. This set is then quotiented
by a relation known as structural congruence and it is over this set
that the notion of dynamics is expressed. This presentation is
essentially that of \cite{MeredithR05} with the addition of
polyadicity and summation. For readability we have relegated some of
the technical subtleties to an appendix.

\subsubsection{Process grammar}\label{subsub:process_grammar}

\begin{mathpar}
  \inferrule* [lab=synchronization] {} {{M} \bc \pzero \;|\; x?F \;|\; x!C }
  \and
  \inferrule* [lab=abstraction] {} {{F} \bc (x)P}
  \and
  \inferrule* [lab=concretion] {} {{C} \bc \langle Q \rangle}
  \and
  \inferrule* [lab=process] {} {{P,Q} \bc M \;| \;P|Q \;|\; @{x}}
  \and
  \inferrule* [lab=name] {} {{x} \bc \quotep{P}}
\end{mathpar} 

Note that $\vec{x}$ (resp. $\vec{P}$) denotes a vector of names
(resp. processes) of length $|\vec{x}|$ (resp. $|\vec{P}|$). We adopt
the following useful abbreviations.

\begin{mathpar}
   x?(\vec{y}).P := x.(\vec{y})P \and  x\clift{\vec{P}} := x.\clift{\vec{P}}
   \and x!(y) := \lift{x}{\dropn{y}}
   \and \Pi_{i=0}^{n-1}P_i := P_0 | \ldots | P_{n-1}
\end{mathpar}

\subsubsection{Structural congruence}

\paragraph{Free and bound names and alpha-equivalence.} At the
core of structural equivalence is alpha-equivalence which identifies
process that are the same up to a change of variable. Formally, we
recognize the distinction between free and bound names. The free names
of a process, $\freenames{P}$, may be calculated recursively as
follows:

\begin{mathpar}
\freenames{\pzero} := \emptyset
  \and \\
  \freenames{x?(y).P} := \{ x \} \cup (\freenames{P} \setminus \{ y \})
  \and 
  \freenames{x!\langle P \rangle} := \{ x \} \cup \{ P \} 
  \and \\
  \freenames{P|Q} := \freenames{P} \cup \freenames{Q}
  \and \\
  \freenames{@{x}} := \{ x \}
\end{mathpar}

$\pi$
$\quotep{\pi}$

$\freenames{-} : \pi \to \mathcal{P}(\quotep{\pi})$

\begin{eqnarray*}
  \freenames{\pzero} & := & \emptyset \\
  \freenames{x?(y).P} & := & \{ x \} \cup (\freenames{P} \setminus \{ y \}) \\
  \freenames{x!\langle P \rangle} & := & \{ x \} \cup \{ P \} \\
  \freenames{P|Q} & := & \freenames{P} \cup \freenames{Q} \\
  \freenames{\dropn{x}} & := & \{ x \}
\end{eqnarray*}

The bound names of a process, $\boundnames{P}$, are those names occurring in $P$
that are not free. For example, in $x?(y).0$, the name $x$ is free, while $y$ is bound.

\begin{mathpar}
  \inferrule* [lab=monoidal-laws] {} { P|Q \equiv Q|P \and P|0 \equiv P \and P|(Q|R) \equiv (P|Q)|R }
\end{mathpar}

\begin{mathpar}
  \inferrule* [lab=alpha-equivalence] {} { (x)P \equiv (y)P\{y/x\} \and y \not\in \freenames{P} }
\end{mathpar}

\begin{definition}
Then two processes, $P,Q$, are alpha-equivalent if $P = Q\{\vec{y}/\vec{x}\}$ for
some $\vec{x} \in \boundnames{Q},\vec{y} \in \boundnames{P}$, where $Q\{\vec{y}/\vec{x}\}$
denotes the capture-avoiding substitution of $\vec{y}$ for $\vec{x}$ in $Q$.
\end{definition}

\begin{definition}
  The {\em structural congruence} \cite{SangiorgiWalker} , $\equiv$,
  between processes is the least congruence containing
  alpha-equivalence, satisfying the abelian monoid laws
  (associativity, commutativity and $\pzero$ as identity) for parallel
  composition $|$ and for summation $+$.
\end{definition}

\subsection{Name equivalence}

We take name equivalence, written $\nameeq$, to be the smallest
equivalence relation generated by the following rules.

\begin{mathpar}
\inferrule*[lab=Quote-drop]
{ }
{ \quotep{@{x}} \nameeq x }

\inferrule*[lab=Struct-equiv]
{ P \scong Q }
{ \quotep{P} \nameeq \quotep{Q} }
\end{mathpar}

The astute reader will have noticed that the mutual recursion of names
and processes imposes a mutual recursion on alpha-equivalence and
structural equivalence via name-equivalence. Fortunately, all of this
works out pleasantly and we may calculate in the natural way, free of
concern. The reader interested in the details is referred to the
appendix \ref{appendix:rho_details}.

\subsection{Substitution}

We use $\Proc$ for the set of processes, $\QProc$ for the set of
names, and $\id{\{}\vec{y} / \vec{x} \id{\}}$ to denote partial maps,
$s : \QProc \rightarrow \QProc$. A map, $s$ lifts, uniquely, to a map
on process terms, $\widehat{s} : \Proc \rightarrow \Proc$ by the
following equations.

\begin{mathpar}
  (0) \psubstp{Q}{P} := 0 \\
  (R \juxtap S) \psubstp{Q}{P}
  :=    
  (R)\psubstp{Q}{P} \juxtap (S) \psubstp{Q}{P} \\
  (x?(y).R) \psubstp{Q}{P}    
  :=    
  (x)\substp{Q}{P} (z)\concat( (R \psubstn{z}{y}) \psubstp{Q}{P} ) \\
  (\lift{x}{R}) \psubstp{Q}{P}  
  :=
  \lift{(x)\substp{Q}{P}}{ R \psubstp{Q}{P} } \\
%   (\dropn{x})  \psubstp{Q}{P}       
%   := 
%   \left\{ 
%     \begin{array}{ccc} 
%       \dropn{\quotep{Q}} & & x \nameeq \quotep{P} \\
%       \dropn{x} & & otherwise \\
%     \end{array}
%   \right. 
  (\dropn{x})  \psubstp{Q}{P}       
  := 
  \left\{ 
    \begin{array}{ccc} 
      Q & & x \nameeq \quotep{P} \\
      \dropn{x} & & otherwise \\
    \end{array}
  \right.
\end{mathpar}
 

where

\begin{eqnarray}
  (x)\id{\{} \lpquote Q \rpquote / \lpquote P \rpquote \id{\}}            = 
  \left\{ 
    \begin{array}{ccc}
      \lpquote Q \rpquote & & x \nameeq \lpquote P \rpquote \\
      x & & otherwise \\
    \end{array}
  \right. \nonumber
\end{eqnarray}

and $z$ is chosen distinct from $\quotep{P}$, $\quotep{Q}$, the free
names in $Q$, and all the names in $R$. Our $\alpha$-equivalence will
be built in the standard way from this substitution.

\begin{remark}\label{rem:no_self_referential_names}
  One consequence of these definitions is that $\forall P. \quotep{P}
  \not\in \freenames{P}$.
\end{remark}

\subsection{ Dynamic quote: an example }

Anticipating something of what's to come, consider applying the
substitution, $\widehat{\id{\{}u / z \id{\}}}$, to the following pair
of processes, $\lift{w}{y!(z)}$ and $w[ \lpquote y!(z) \rpquote ]$.

\begin{eqnarray}
	\lift{w}{y!(z)}\widehat{\id{\{}u / z \id{\}}}
		& = &
		\lift{w}{y!(u)} \nonumber\\
	w[ \lpquote y!(z) \rpquote ] \widehat{ \id{\{}u / z \id{\}} }
		& = &
		w[ \lpquote y!(z) \rpquote ] \nonumber
\end{eqnarray}

Because the body of the process between quotes is impervious to
substitution, we get radically different answers. In fact, by
examining the first process in an input context,
e.g. $x?(z).\lift{w}{y!(z)}$, we see that the process under the lift
operator may be shaped by prefixed inputs binding a name inside it. In
this sense, the lift operator will be seen as a way to dynamically
construct processes before reifying them as names.

Finally equipped with these standard features we can present the
dynamics of the calculus.

\subsubsection{Operational semantics} 

Finally, we introduce the computational dynamics. What marks these
algebras as distinct from other more traditionally studied algebraic
structures, e.g. vector spaces or polynomial rings, is the manner in
which dynamics is captured. In traditional structures, dynamics is typically
expressed through morphisms between such structures, as in linear maps
between vector spaces or morphisms between rings. In algebras
associated with the semantics of computation, the dynamics is
expressed as part of the algebraic structure itself, through a
reduction reduction relation typically denoted by $\red$. Below, we
give a recursive presentation of this relation for the calculus used
in the encoding.

$\red \subseteq \pi \times \pi$
$\red : \pi \to \mathcal{P}(\pi)$

\begin{mathpar}
  \inferrule* [lab=Comm] { \textsf{match}( x_{src}, x_{trgt} ) } { x_{trgt}?(y)P \; | \; x_{src}!\langle {Q} \rangle \red P\{\quotep{Q}/y}\} }
  \and \\
  \inferrule* [lab=Par] {{P} \red {P}'} {{{P} | {Q}} \red {{P}' | {Q}}}
  \and
  \inferrule* [lab=Equiv]{{{P} \scong {P}'} \andalso {{P}' \red {Q}'} \andalso {{Q}' \scong {Q}}}{{P} \red {Q}}
\end{mathpar}

\begin{eqnarray*}
  match_{\equiv} (\quotep{P},\quotep{Q}) & := & P \equiv Q \\
  match_{\dagger}(\quotep{P},\quotep{Q}) & := & \forall R. P|Q \red^{*} R => R \red^{*} 0 \\
  match_{K}(\quotep{P},\quotep{Q}) & := & K \mbox{ for some context } K
\end{eqnarray*}

$u?(x)P | u!\langle Q \rangle \red P\{\quotep{Q}/x\}$

%We write $\wred$ for $\red^*$, and $P\red$ if $\exists Q $ such that $ P \red Q$.
We write $P\red$ if $\exists Q $ such that $ P \red Q$ and $P\not\red$, otherwise.

\section{Replication}

As mentioned before, it is known that replication (and hence
recursion) can be implemented in a higher-order process algebra
\cite{SangiorgiWalker}. As our first example of calculation with the
machinery thus far presented we give the construction explicitly in
the {\rhoc}.

\begin{eqnarray}
	D_{x} & := & \prefix{x}{y}{(\binpar{\outputp{x}{y}}{@{y}})} \nonumber\\
	\bangp_{x}{P} & := & \binpar{{x}!\langle{\binpar{D_{x}}{P}}\rangle}{D_{x}} \nonumber
\end{eqnarray}

\begin{eqnarray}
	\bangp_{x}{P} & & \nonumber\\
	=
	& {x}!\langle{(\prefix{x}{y}{(\outputp{x}{y} | @{y})) | P}}\rangle 
	      | \prefix{x}{y}{(\outputp{x}{y} | @{y})} & \nonumber\\
	\red
	& (\outputp{x}{y} | @{y})\substn{\quotep{(\prefix{x}{y}{(@{y} | \outputp{x}{y})) | P}}}{y} & \nonumber\\
	=
	& \outputp{x}{\quotep{(\prefix{x}{y}{(\outputp{x}{y} | @{y})) | P}}}
	  | {(\prefix{x}{y}{(\outputp{x}{y} | @{y})) | P}} & \nonumber\\
	\red
	& \ldots & \nonumber\\
	\red^*
	& P | P | \ldots & \nonumber
\end{eqnarray}

Of course, this encoding, as an implementation, runs away, unfolding
$\bangp{P}$ eagerly. A lazier and more implementable replication
operator, restricted to input-guarded processes, may be obtained as follows.

\begin{eqnarray}
\bangp{\prefix{u}{v}{P}} 
	:= 
	\binpar{\lift{x}{\prefix{u}{v}{(\binpar{D(x)}{P})}}}{D(x)} \nonumber
\end{eqnarray}

\begin{remark}
  Note that the lazier definition still does not deal with summation
  or mixed summation (i.e. sums over input and output). The reader is
  invited to construct definitions of replication that deal with these
  features. 

  Further, the definitions are parameterized in a name, $x$. Can you,
  gentle reader, make a definition that eliminates this parameter and
  guarantees no accidental interaction between the replication
  machinery and the process being replicated -- i.e. no accidental
  sharing of names used by the process to get its work done and the
  name(s) used by the replication to effect copying. This latter
  revision of the definition of replication is crucial to obtaining
  the expected identity $!!P \sim !P$.
\end{remark}

\begin{remark}\label{rem:paradoxical_combinator}
  The reader familiar with the lambda calculus will have noticed the
  similarity between $D$ and the paradoxical combinator.

  [Ed. note: the existence of this seems to suggest we have to be more
  restrictive on the set of processes and names we admit if we are to
  support no-cloning.]
\end{remark}

\subsubsection{Bisimulation}

The computational dynamics gives rise to another kind of equivalence,
the equivalence of computational behavior. As previously mentioned
this is typically captured \emph{via} some form of bisimulation.

% The notion we use in this paper is weak barbed bisimulation
% \cite{milner91polyadicpi}.

The notion we use in this paper is derived from weak barbed
bisimulation \cite{milner91polyadicpi}. 

\begin{definition}
An \emph{observation relation}, $\downarrow_{\mathcal N}$, over a set
of names, $\mathcal N$, is the smallest relation satisfying the rules
below.

\infrule[Out-barb]{y \in {\mathcal N}, \; x \nameeq y}
		  {\outputp{x}{v} \downarrow_{\mathcal N} x}
\infrule[Par-barb]{\mbox{$P\downarrow_{\mathcal N} x$ or $Q\downarrow_{\mathcal N} x$}}
		  {\binpar{P}{Q} \downarrow_{\mathcal N} x}

We write $P \Downarrow_{\mathcal N} x$ if there is $Q$ such that 
$P \wred Q$ and $Q \downarrow_{\mathcal N} x$.
\end{definition}

\begin{definition}
%\label{def.bbisim}
An  ${\mathcal N}$-\emph{barbed bisimulation} over a set of names, ${\mathcal N}$, is a symmetric binary relation 
${\mathcal S}_{\mathcal N}$ between agents such that $P\rel{S}_{\mathcal N}Q$ implies:
\begin{enumerate}
\item If $P \red P'$ then $Q \wred Q'$ and $P'\rel{S}_{\mathcal N} Q'$.
\item If $P\downarrow_{\mathcal N} x$, then $Q\Downarrow_{\mathcal N} x$.
\end{enumerate}
$P$ is ${\mathcal N}$-barbed bisimilar to $Q$, written
$P \wbbisim_{\mathcal N} Q$, if $P \rel{S}_{\mathcal N} Q$ for some ${\mathcal N}$-barbed bisimulation ${\mathcal S}_{\mathcal N}$.
\end{definition}

$\mathcal{R} \subseteq \pi \times \pi$

$P \mathcal{R} Q => \forall P'. P \red P' \Rightarrow \exists Q'. Q \red Q', P' \mathcal{R} Q'$

$P \vdash x \Rightarrow Q \vdash x$

\begin{mathpar}
  \inferrule*[lab=Out-barb]{x \nameeq y}{{y}!\langle{Q}\rangle \vdash x}
  \and
  \inferrule*[lab=Par-barb]{\mbox{$P\vdash x$ or $Q\vdash x$}}{\binpar{P}{Q} \vdash x}
\end{mathpar}

\subsubsection{Contexts}

One of the principle advantages of computational calculi like the
$\pi$-calculus is a well-defined notion of context,
contextual-equivalence and a correlation between
contextual-equivalence and notions of bisimulation. The notion of
context allows the decomposition of a process into (sub-)process and
its syntactic environment, its context. Thus, a context may be
thought of as a process with a ``hole'' (written $\Box$) in it. The
application of a context $M$ to a process $P$, written $M[P]$, is
tantamount to filling the hole in $M$ with $P$. In this paper we do
not need the full weight of this theory, but do make use of the notion
of context in the proof the main theorem. 

\begin{mathpar}
  \inferrule* [lab=summation] {} {{M_{M},M_{N}} \bc \Box \;|\; x.M_{A} \;|\; M_{M}+M_{N}}
  \and
  \inferrule* [lab=agent] {} {{M_{A}} \bc (\vec{x})M_{P} \;| \; \clift{P_0,\ldots,M_{P},\ldots,P_N}}
  \and \\
  \inferrule* [lab=process] {} {{M_{P}} \bc M_{N} \;| \;P|M_{P} }
\end{mathpar} 

\begin{mathpar}
  \inferrule* [lab=sychronization] {} {M_{N} \bc \Box \;|\; x?M_{F} \;|\; x!M_{C}}
  \and
  \inferrule* [lab=abstraction] {} {{M_{F}} \bc (x)M_{P} }
  \and
  \inferrule* [lab=concretion] {} {{M_{C}} \bc \langle M_{P} \rangle }
  \and \\
  \inferrule* [lab=process] {} {{M_{P}} \bc M_{N} \;| \;P|M_{P} }
\end{mathpar}

\begin{definition}[contextual application] Given a context $M$, and
  process $P$, we define the \emph{contextual application}, $M[P] :=
  M\{P/\Box\}$. That is, the contextual application of M to P is the
  substitution of $P$ for $\Box$ in $M$.
\end{definition}

$\meaningof{-} : L \to \mathcal{P}(\pi)$

\begin{mathpar}
  \inferrule* [lab=collection] {} {\meaningof{true} = \pi, \and \meaningof{~E} = \pi \setminus \meaningof{E}, \and \meaningof{E_{1} \& E_{2}} = \meaningof{E_{1}} \cap \meaningof{E_{2}}}
\end{mathpar}

\begin{mathpar}
  \inferrule* [lab=structure] {} {\meaningof{0} = \{ P \in \pi | P \equiv 0 \}, \and \\ \meaningof{E_1 | E_2} = \{ P \in \pi | P \equiv P_{1} | P_{2}, P_{1} \in \meaningof{E_{1}}, P_{2} \in \meaningof{E_2}\} }
\end{mathpar}

\begin{mathpar}
 \inferrule* [lab=behavior] {} {\meaningof{\langle a?b \rangle E} = \{ P \in \pi | P \equiv Q | u?(y)P', \\ \and \\\\ \and \\ \;\;\; u \in \meaningof{a}, \forall z.P'\{z/y\} \in \meaningof{E\{z/b\}}\}, \and \\ \meaningof{a!E} = \{ P \in \pi | P \equiv Q | x!\langle P' \rangle, x \in \meaningof{a} P' \in \meaningof{E}\} }
\end{mathpar}

\begin{mathpar}
 \inferrule* [lab=nominal] {} {\meaningof{\quotep{E}} = \{ \quotep{P} \in \quotep{\pi} | P \in \meaningof{E} \}, \and \meaningof{\quotep{P}} = \{ \quotep{Q} \in \quotep{\pi} | P \equiv Q \} \and \\ \meaningof{@\quotep{E}} = \{ P \in \pi | P \equiv @x, x \in \meaningof{E} \}}
\end{mathpar}

\begin{eqnarray*}
  \\
  \meaningof{-} : TS \to ST
\end{eqnarray*}

\begin{eqnarray*}
  \\
  L : TS \to ST
\end{eqnarray*}

\begin{eqnarray*}
  \\
  P \models E \iff P \in \meaningof{E}
\end{eqnarray*}

\begin{eqnarray*}
  P \approx_{L} Q \iff \forall E \in L. P \models E \iff Q \models E
\end{eqnarray*}

\begin{eqnarray*}
  P \approx_{K} Q
\end{eqnarray*}

\begin{eqnarray*}
  P \approx Q
\end{eqnarray*}

$\approx_{K} = \approx = \approx_{L}$

\subsubsection{Contextual duality}

Note that contexts extend the quotation operation to a family of
operations from processes to names. Given a context, $M$, we can
define a \emph{nominal context}, $\quotep{M}$ by $\quotep{M}[P] :=
\quotep{M[P]}$. To foreshadow what is to come we observe that these
operations enjoy a duality with processes very much like the duality
between vectors and maps from vectors to scalars.

Further, because the calculus is essentially higher-order, we have a
correspondence between contexts and processes. More specifically,
given a name $x$ and a context $M$ we can construct $M^{*}_{x}$ such
that 

\begin{mathpar}
  M^{*}_{x} | \lift{x}{P} \red M[P]
\end{mathpar}

namely,

\begin{mathpar}
  M^{*}_{x} := x?(u).M[\dropn{u}]
\end{mathpar}

The dependence of $M^{*}_{x}$ on a name makes it an abstraction, 

\begin{mathpar}
  M^{*} := (x)x?(u).M[\dropn{u}]
\end{mathpar}

\subsection{Additional notation}

It will sometimes be convenient to denote the process a name
quotes. We already have the notation $x = \quotep{P}$, but it will be
convenient to introduce an alternate notation, $\procn{x}$, when we
want to emphasize the connection to the use of the name. Note that, by
virtue of name equivalence, $\quotep{\procn{x}} \nameeq x$; so, the
notation is consistent with previous definitions.

Further, because names have structure it is possible to effect
substitutions on the basis of that structure. This means we need to
upgrade our notation for substitutions, which we accomplish by
adapting comprehension notation. Thus,

\begin{mathpar}
  P\{ y / x : x \in S \}
\end{mathpar}

is interpreted to mean the process derived from P by replacing (in a
capture-avoiding manner) each occurrence of $x$ in $S$ by $y$. For example,

\begin{mathpar}
  P\{ \quotep{\procn{x}|\procn{x}} / x : x \in \freenames{P} \}
\end{mathpar}

will replace each (occurrence) of a free name $x$ in $P$ by
$\quotep{\procn{x}|\procn{x}}$.

Also, we will avail ourselves of the notation $x^{L}$ and $x^{R}$ to
denote injections of a name into disjoint copies of the name
space. There are numerous ways to accomplish this. One example can be
found in \cite{MeredithR05}. This notation overloads to vectors of
names: $\vec{x}^{\pi} := (x_{i}^{\pi} \; : \; 0 \leq i < |\vec{x}| )$ where $\pi \in \{L,R\}$.

We also use $P^{\Box} := P|\Box$.

In \cite{MeredithR05} an interpretation of the new operator is
given. It turns out that there are several possible interpretations
all enjoying the requisite algebraic properties of the operator (see
\cite{milner91polyadicpi}). We will therefore make liberal use of
$(\nu\; \vec{x})P$.

% subsection the_syntax_and_semantics_of_the_notation_system (end)   

\input{qm2pi.qmops} 

\input{qm2pi.sterngerlach} 

\input{qm2pi.metric} 

% section concurrent_process_calculi (end)

%\input{qm2pi.proofsketch}

% section proof sketch (end)

%\input{qm2pi.slviaknots} 

% section spatial logic via knots (end)

\input{qm2pi.conclusion}

% section conclusion (end)

%\input{qm2pi.dtcodes} 

% section wiring algorithm (end)

\input{qm2pi.ack} 

% section acknowledgments (end)

\newpage


\bibliographystyle{plain}   
\bibliography{../../biblios/main.bib}

\input{qm2pi.rhodetails}

\end{document}



% section proof sketch (end)

%\section{Unlikely characters: spatial logic for
  knots}\label{sub:characteristic_formulae} % (fold)

Associated to the mobile process calculi are a family of logics known
as the Hennessy-Milner logics. These logics typically enjoy a
semantics interpreting formulae as sets of processes that when
factored through the encoding outlined above allows an identification
of classes of knots with logical formulae. In the context of this
encoding the sub-family known as the spatial logics \cite{CairesC03}
\cite{CairesC04} \cite{Caires04} are of particular interest providing
several important features for expressing and reasoning about
properties (i.e. classes) of knots. We hint here at how this may be done.

%\begin{description}
%\item [structural connectives] 
\subsubsection{Structural connectives} The spatial logics enjoy
structural connectives corresponding, at the logical level, to the
parallel composition ($P | Q$) and new name ($(\nu \; x)P$)
connectives for processes. As illustrated in the examples below, these
connectives are extremely expressive given the shape of our encoding.
%\item [decideable satisfaction]

\subsubsection{Decideable satisfaction}
In \cite{Caires04} the satisfaction relation is shown to be decideable
for a rich class of processes. It further turns out that the image of
the our encoding is a proper subset of that class. This result
provides the basis for an algorithm by which to search for knots
enjoying a given property.
%\item [characteristic formulae]

\subsubsection{Characteristic formulae}
In the same paper \cite{Caires04} , Caires presents a means of calculating
characteristic formulae, selecting equivalence classes of processes
up to a pre--specified depth limit on the support set of names. Composed with our
encoding, this characteristic formula can be used to select
characteristic formulae for knots.
%\end{description}

\subsubsection{Spatial logic formulae}

The grammar below (segmented for comprehension) summarizes the syntax
of spatial logic formulae. We employ illustrative examples in the
sequel to provide an intuitive understanding of their meaning
referring the reader to \cite{Caires04} for a more detailed explication
of the semantics.

\begin{mathpar}
  \inferrule* [lab=boolean] {} {{A,B} \bc T \;|\; \neg A \;|\; A \wedge B \;|\; \eta = \eta'}
  \and
  \inferrule* [lab=spatial] {} {|\; \pzero \;|\; A | B \;|\; x \text{\textregistered} A \;|\; \forall x . A \;|\;  H x . A}
  \and
  \inferrule* [lab=behavioral] {} {|\; \alpha . A}
  \and 
  \inferrule* [lab=recursion] {} {|\; X(\vec{u}) \;|\; \mu X(\vec{u}) . A}
  \and
  \inferrule* [lab=action] {} {\alpha \bc \langle x?(\vec{y}) \rangle \;|\; \langle x!(\vec{y}) \rangle \;|\; \langle \tau \rangle}
  \and 
  \inferrule* [lab=name] {} {\eta \bc x \;|\; \tau}
\end{mathpar} 

% subsection characteristic_formulae (end)   	 

\subsection{Example formulae}\label{sub:example_formulae_} % (fold)

\subsubsection{Crossing as formula.}
% 
% \begin{align*}
%   \frac{d}{dx} \sin x &= \cos x 
%   & \frac{d}{dx} e^x &= e^x \\
%   \frac{d}{dx} \cos x &= - \sin x 
%   & \frac{d}{dx} \log x &= \frac{1}{x} \\
% \end{align*} 

\begin{align*}
 \mu C(x_{0},x_{1},y_{0},y_{1},u).&(\langle x_{0}?(z) \rangle(\langle u! \rangle\langle y_{1}!z \rangle C(x_{0},x_{1},y_{0},y_{1},u)) & \\
  & \wedge \langle y_{1}?(z) \rangle (\langle u! \rangle \langle x_{0}!z \rangle C(x_{0},x_{1},y_{0},y_{1},u)) & \\
  & \wedge \langle x_{1}?(z) \rangle (\langle u? \rangle \langle y_{0}!z \rangle C(x_{0},x_{1},y_{0},y_{1},u)) & \\
  & \wedge \langle y_{0}?(z) \rangle (\langle u? \rangle \langle x_{1}!z \rangle C(x_{0},x_{1},y_{0},y_{1},u))) &
\end{align*}

The lexicographical similarity between the shape of this formulae and
the shape of definition of the process representing a crossing reveals
the intuitive meaning of this formulae. It describes the capabilities
of a process that has the right to represent a crossing. For example
it picks out processes that may perform an input on the port $x_0$ in
its initial menu of capabilities. What differentiates the formula
from the process, however, is that the crossing process is the
smallest candidate to satisfy the formula. Infinitely many other
processes -- with internal behavior hidden behind this interface, so
to speak -- also satisfy this formula. Even this simple formula,
then, can be seen to open a new view onto knots, providing a
computational interpretation of \emph{virtual} knots.

Note that this formula is derived by hand. A similar formula can be
derived by employing Caires' calculation of characteristic formula
\cite{Caires04} to the process representing a crossing. In light of
this discussion, we let
$\meaningof{C}_{\phi}(x0,x1,y0,y1,u)$ denote a formula specifying the
dynamics we wish to capture of a crossing. To guarantee we preserve
the shape of the interface and minimal semantics we demand that
$\meaningof{C}_{\phi}(x0,x1,y0,y1,u) \Rightarrow
\textbf{C}(x0,x1,y0,y1,u)$ where $\textbf{C}(x0,x1,y0,y1,u)$ denotes
the formula above.
                            
\subsubsection{Crossing number constraints.}
The moral content of the context lemma (Lemma \ref{context}) is that the notion of
``locality'' in the Reidemeister moves is effectively captured by the
parallel composition operator of the process calculus. This intuition
extends through the logic. Given a formula,
$\meaningof{C}_{\phi}(x0,x1,y0,y1,u)$, we can use the structural
connectives to specify constraints on crossing numbers, such as at
least $n$ crossings, or exactly $n$ crossings.
\begin{mathpar}
  \inferrule* [lab=at-least-n] {} { K^{\geq n}_{\phi}(\vec{xs},\vec{ys}) := \Pi_{i=0}^{n-1} Hu . \meaningof{C}_{\phi}(xs_i,ys_i,u) | T }
  \and 
  \inferrule* [lab=exactly-n] {} { K^{= n}_{\phi}(\vec{xs},\vec{ys}) := \Pi_{i=0}^{n-1} Hu . \meaningof{C}_{\phi}(xs_i,ys_i,u) | \neg (\forall x_0,y_0,x_1,y_1,u . \meaningof{C}_{\phi}(x_0,y_0,x_1,y_1,u) | T) }
\end{mathpar}

To round out this section, recall that the encoding of an $n$-crossing
knot decomposes into a parallel composition of $n$ \emph{copies} of a
crossing process together with a wiring harness. To specify different
knot classes with the same crossing number amounts to specifying
logical constraints on the wiring harness. In the interest of space,
we defer examples to a forthcoming paper. Suffice it to say that both
the conditions ``alternating knot'' and ``contains the tangle
corresponding to 5/3'' are expressible. For example, it is possible to
calculate the characteristic formula of a process corresponding to the
tangle 5/3 and conjoin it into the classifying formula via the
composition connective of the logic.

Finally, we wish to observe that it is entirely within reason to
contemplate a more domain-specific version of spatial logic tailored
to the shape of processes in the image of the encoding. Such a
domain-specific logic would have a better claim to the title formal
language of knot properties.

% subsection example_formulae_ (end)

% section knots_as_processes (end) 

% section spatial logic via knots (end)

\section{Conclusions and future work}

\paragraph{Testing physical space}
You, gentle reader, may wonder why of all the theorems to be proved
given this set up we pick the one above. In some sense it's hardly
central to quantum mechanics. We see it as central in the sense that
it firmly establishes a notion of physical space arising from a notion
of the equivalence of behavior. Relating bisimulation to a metric is a
big step forward, but one is faced with interpreting the relationship
of that metric space to something more physical. Quantum mechanical
notions of ``physical'' space are still far from intuitive, but by
relating this idea of distance as testing to calculations that predict
physical circumstances we are making a not insignificant step forward
toward an understanding of the physical space we inhabit as
essentially dynamic.

\paragraph{Effectivity and simulation}
One of the observations we have yet to make is that the entire program
spelled out here is effective. We have built various interpreters for
the reflective calculus at work in this interpretation. In principle,
then, we can simulate quantum mechanics on a computer. The place where
the simulation may lose fidelity is the infinitely branching summation
for the annihilator.

In this connection i also want to point out that the evaluation style
calculation of the inner product puts the non-determinism of the
summation right at the heart of measurement. This suggests that
Milner's original reduction-based formulation of the dynamics of his
calculi in terms of sums was not just notationally suggestive of a
notion of measure-and-continue but captured some significant part of
the physics.

\paragraph{Quantum continuations}
In light of this last observation i want to point out that the
predominant account of quantum mechanics is missing a key aspect of a
truly compositional story of the physical situation. In a real lab,
when a measurement is made the observation can be made to feed into
another device that then makes another measurement conditioned on the
results of the first. This means that after the superposition was
collapsed the entire experimental set up remained in
superposition. While QM offers a means of writing this down it doesn't
quite line up well with the well-trodden formulation of computation
and continuation that we see so succinctly expressed in Milner's
calculi. This suggests that there might be advantages to this account
of dynamics waiting to be explored.

\paragraph{Quantum logic}
In this connection, we also note that by virtue of having the
Hennessy-Milner construction, we can pull the construction through the
interpretation of QM. This gives us a natural candidate for a quantum
logic that enjoys an extremely tight connection with it's domain of
interpretation, making the construction much less ad hoc (rather it is
the image of functor!).

\paragraph{Quantum probabiity}
i have questions about the basis of the interpretation of inner
product as probability amplitude. In particular, using which
axiomatization of probability theory does the notion of probability
amplitude earn the right to be so dubbed? In other words, where is the
proof that the operation for calculating a probability amplitude (and
then squaring) satisfies the axioms of what it means to calculate a
probability? Even if such a proof exists (i have yet to find it in the
literature), i wonder if it might not be possible to turn things on
their heads. Can we view the calculation of the probability amplitude
as an axiomatization of probability? If so, then the definition we
give for calculating probability amplitude may provide the basis for
an \emph{effective} theory of probability.

\paragraph{Quantum vs ``biological'' information}
Finally, i want to conclude with a more philosophical observation. At
a recent workshop in which QM was a predominant topic i noticed
something about quantum information. The speaker was giving a riveting
discussion of axiomatic QM and showing how properties of ``no
cloning'' and ``no deleting'' emerged as consequences of the
axiomatization. Theorems of this form are necessary to give us a sense
of confidence that our axioms characterize the physical theory. What
struck me, though, was that if quantum information is neither erasable
nor replicable it is markedly different from \emph{life}. Two of the
things we know about life is that

\begin{itemize}
  \item it ends;
  \item to gain some measure of persistence, to transcend it's
    finitude it is imminently copyable.
\end{itemize}

Both of these qualities are summarized succinctly in the aphorism: all
flesh is grass. For me these two kinds of ``information'' -- call them
quantum and biological -- are end points on a spectrum of strategies
for persistence. At one end, we have those curious entities that enjoy
uniqueness and permanence; at the other, we have those who in the face
of a certain end and an uncertain present make a go of passing
something on. To me one of the more remarkable aspects of the latter
strategy is that in the presence of noise (and certain features of
copying) we get a kind of dynamism, a chance for improvement against a
given persistent condition.

% subsection other_calculi_other_bisimulations_and_geometry_as_behavior (end)




% section conclusion (end)

%\documentclass[12pt]{llncs}
%\documentclass{jktr}

\usepackage[pdftex]{hyperref}                   
\usepackage {listings}
\usepackage {mathpartir}
\usepackage{bcprules}
%\usepackage{listings}
                       
\usepackage{graphicx} 
%\usepackage[margins=2.5cm,nohead,nofoot]{geometry}
%\usepackage{geometry}
\usepackage{amsfonts}
\usepackage{amstext}
\usepackage{latexsym}
\usepackage{amssymb}
\usepackage{color}


%\include{myPreamble}
\include{qm2pi.local} 

%\ifpdf
%\usepackage[pdftex]{graphicx}
%\else
%\usepackage{graphicx}
%\fi

 % \ifpdf
%  \usepackage{pdfsync}
%  \if


%\title{Brief Article}
%\author{David F. Snyder}
%\author{L.G. Meredith}

%\address{Dept. of Math., Texas State University--San Marcos, San Marcos, TX 78666}
       
\pagestyle{empty}


\begin{document}

\lstset{language=[Objective]Caml,frame=shadowbox}

\input{qm2pi.front}

% section front matter (end)

\input{qm2pi.intro} 
 
% section introduction (end)

% \input{qm2pi.knotations} 

% section notation (end)

\input{qm2pi.process.calculi} 

% section concurrent_process_calculi_and_spatial_logics_ (end)
    
%\input{qm2pi.knots2pi} 

%\input{qm2pi.trefoil} 

%\input{qm2pi.mainthm} 

% subsection basic_interpretation (end)

%\input{qm2pi.rho.presentation} 
\subsection{The syntax and semantics of the notation system}\label{sub:the_syntax_and_semantics_of_the_notation_system} % (fold)

We now summarize a technical presentation of the calculus that
embodies our theory of dynamics. The typical presentation of such a
calculus follows the style of giving generators and relations on
them. The grammar, below, describing term constructors, freely
generates the set of processes, $\Proc$. This set is then quotiented
by a relation known as structural congruence and it is over this set
that the notion of dynamics is expressed. This presentation is
essentially that of \cite{MeredithR05} with the addition of
polyadicity and summation. For readability we have relegated some of
the technical subtleties to an appendix.

\subsubsection{Process grammar}\label{subsub:process_grammar}

\begin{mathpar}
  \inferrule* [lab=synchronization] {} {{M} \bc \pzero \;|\; x?F \;|\; x!C }
  \and
  \inferrule* [lab=abstraction] {} {{F} \bc (x)P}
  \and
  \inferrule* [lab=concretion] {} {{C} \bc \langle Q \rangle}
  \and
  \inferrule* [lab=process] {} {{P,Q} \bc M \;| \;P|Q \;|\; @{x}}
  \and
  \inferrule* [lab=name] {} {{x} \bc \quotep{P}}
\end{mathpar} 

Note that $\vec{x}$ (resp. $\vec{P}$) denotes a vector of names
(resp. processes) of length $|\vec{x}|$ (resp. $|\vec{P}|$). We adopt
the following useful abbreviations.

\begin{mathpar}
   x?(\vec{y}).P := x.(\vec{y})P \and  x\clift{\vec{P}} := x.\clift{\vec{P}}
   \and x!(y) := \lift{x}{\dropn{y}}
   \and \Pi_{i=0}^{n-1}P_i := P_0 | \ldots | P_{n-1}
\end{mathpar}

\subsubsection{Structural congruence}

\paragraph{Free and bound names and alpha-equivalence.} At the
core of structural equivalence is alpha-equivalence which identifies
process that are the same up to a change of variable. Formally, we
recognize the distinction between free and bound names. The free names
of a process, $\freenames{P}$, may be calculated recursively as
follows:

\begin{mathpar}
\freenames{\pzero} := \emptyset
  \and \\
  \freenames{x?(y).P} := \{ x \} \cup (\freenames{P} \setminus \{ y \})
  \and 
  \freenames{x!\langle P \rangle} := \{ x \} \cup \{ P \} 
  \and \\
  \freenames{P|Q} := \freenames{P} \cup \freenames{Q}
  \and \\
  \freenames{@{x}} := \{ x \}
\end{mathpar}

$\pi$
$\quotep{\pi}$

$\freenames{-} : \pi \to \mathcal{P}(\quotep{\pi})$

\begin{eqnarray*}
  \freenames{\pzero} & := & \emptyset \\
  \freenames{x?(y).P} & := & \{ x \} \cup (\freenames{P} \setminus \{ y \}) \\
  \freenames{x!\langle P \rangle} & := & \{ x \} \cup \{ P \} \\
  \freenames{P|Q} & := & \freenames{P} \cup \freenames{Q} \\
  \freenames{\dropn{x}} & := & \{ x \}
\end{eqnarray*}

The bound names of a process, $\boundnames{P}$, are those names occurring in $P$
that are not free. For example, in $x?(y).0$, the name $x$ is free, while $y$ is bound.

\begin{mathpar}
  \inferrule* [lab=monoidal-laws] {} { P|Q \equiv Q|P \and P|0 \equiv P \and P|(Q|R) \equiv (P|Q)|R }
\end{mathpar}

\begin{mathpar}
  \inferrule* [lab=alpha-equivalence] {} { (x)P \equiv (y)P\{y/x\} \and y \not\in \freenames{P} }
\end{mathpar}

\begin{definition}
Then two processes, $P,Q$, are alpha-equivalent if $P = Q\{\vec{y}/\vec{x}\}$ for
some $\vec{x} \in \boundnames{Q},\vec{y} \in \boundnames{P}$, where $Q\{\vec{y}/\vec{x}\}$
denotes the capture-avoiding substitution of $\vec{y}$ for $\vec{x}$ in $Q$.
\end{definition}

\begin{definition}
  The {\em structural congruence} \cite{SangiorgiWalker} , $\equiv$,
  between processes is the least congruence containing
  alpha-equivalence, satisfying the abelian monoid laws
  (associativity, commutativity and $\pzero$ as identity) for parallel
  composition $|$ and for summation $+$.
\end{definition}

\subsection{Name equivalence}

We take name equivalence, written $\nameeq$, to be the smallest
equivalence relation generated by the following rules.

\begin{mathpar}
\inferrule*[lab=Quote-drop]
{ }
{ \quotep{@{x}} \nameeq x }

\inferrule*[lab=Struct-equiv]
{ P \scong Q }
{ \quotep{P} \nameeq \quotep{Q} }
\end{mathpar}

The astute reader will have noticed that the mutual recursion of names
and processes imposes a mutual recursion on alpha-equivalence and
structural equivalence via name-equivalence. Fortunately, all of this
works out pleasantly and we may calculate in the natural way, free of
concern. The reader interested in the details is referred to the
appendix \ref{appendix:rho_details}.

\subsection{Substitution}

We use $\Proc$ for the set of processes, $\QProc$ for the set of
names, and $\id{\{}\vec{y} / \vec{x} \id{\}}$ to denote partial maps,
$s : \QProc \rightarrow \QProc$. A map, $s$ lifts, uniquely, to a map
on process terms, $\widehat{s} : \Proc \rightarrow \Proc$ by the
following equations.

\begin{mathpar}
  (0) \psubstp{Q}{P} := 0 \\
  (R \juxtap S) \psubstp{Q}{P}
  :=    
  (R)\psubstp{Q}{P} \juxtap (S) \psubstp{Q}{P} \\
  (x?(y).R) \psubstp{Q}{P}    
  :=    
  (x)\substp{Q}{P} (z)\concat( (R \psubstn{z}{y}) \psubstp{Q}{P} ) \\
  (\lift{x}{R}) \psubstp{Q}{P}  
  :=
  \lift{(x)\substp{Q}{P}}{ R \psubstp{Q}{P} } \\
%   (\dropn{x})  \psubstp{Q}{P}       
%   := 
%   \left\{ 
%     \begin{array}{ccc} 
%       \dropn{\quotep{Q}} & & x \nameeq \quotep{P} \\
%       \dropn{x} & & otherwise \\
%     \end{array}
%   \right. 
  (\dropn{x})  \psubstp{Q}{P}       
  := 
  \left\{ 
    \begin{array}{ccc} 
      Q & & x \nameeq \quotep{P} \\
      \dropn{x} & & otherwise \\
    \end{array}
  \right.
\end{mathpar}
 

where

\begin{eqnarray}
  (x)\id{\{} \lpquote Q \rpquote / \lpquote P \rpquote \id{\}}            = 
  \left\{ 
    \begin{array}{ccc}
      \lpquote Q \rpquote & & x \nameeq \lpquote P \rpquote \\
      x & & otherwise \\
    \end{array}
  \right. \nonumber
\end{eqnarray}

and $z$ is chosen distinct from $\quotep{P}$, $\quotep{Q}$, the free
names in $Q$, and all the names in $R$. Our $\alpha$-equivalence will
be built in the standard way from this substitution.

\begin{remark}\label{rem:no_self_referential_names}
  One consequence of these definitions is that $\forall P. \quotep{P}
  \not\in \freenames{P}$.
\end{remark}

\subsection{ Dynamic quote: an example }

Anticipating something of what's to come, consider applying the
substitution, $\widehat{\id{\{}u / z \id{\}}}$, to the following pair
of processes, $\lift{w}{y!(z)}$ and $w[ \lpquote y!(z) \rpquote ]$.

\begin{eqnarray}
	\lift{w}{y!(z)}\widehat{\id{\{}u / z \id{\}}}
		& = &
		\lift{w}{y!(u)} \nonumber\\
	w[ \lpquote y!(z) \rpquote ] \widehat{ \id{\{}u / z \id{\}} }
		& = &
		w[ \lpquote y!(z) \rpquote ] \nonumber
\end{eqnarray}

Because the body of the process between quotes is impervious to
substitution, we get radically different answers. In fact, by
examining the first process in an input context,
e.g. $x?(z).\lift{w}{y!(z)}$, we see that the process under the lift
operator may be shaped by prefixed inputs binding a name inside it. In
this sense, the lift operator will be seen as a way to dynamically
construct processes before reifying them as names.

Finally equipped with these standard features we can present the
dynamics of the calculus.

\subsubsection{Operational semantics} 

Finally, we introduce the computational dynamics. What marks these
algebras as distinct from other more traditionally studied algebraic
structures, e.g. vector spaces or polynomial rings, is the manner in
which dynamics is captured. In traditional structures, dynamics is typically
expressed through morphisms between such structures, as in linear maps
between vector spaces or morphisms between rings. In algebras
associated with the semantics of computation, the dynamics is
expressed as part of the algebraic structure itself, through a
reduction reduction relation typically denoted by $\red$. Below, we
give a recursive presentation of this relation for the calculus used
in the encoding.

$\red \subseteq \pi \times \pi$
$\red : \pi \to \mathcal{P}(\pi)$

\begin{mathpar}
  \inferrule* [lab=Comm] { \textsf{match}( x_{src}, x_{trgt} ) } { x_{trgt}?(y)P \; | \; x_{src}!\langle {Q} \rangle \red P\{\quotep{Q}/y}\} }
  \and \\
  \inferrule* [lab=Par] {{P} \red {P}'} {{{P} | {Q}} \red {{P}' | {Q}}}
  \and
  \inferrule* [lab=Equiv]{{{P} \scong {P}'} \andalso {{P}' \red {Q}'} \andalso {{Q}' \scong {Q}}}{{P} \red {Q}}
\end{mathpar}

\begin{eqnarray*}
  match_{\equiv} (\quotep{P},\quotep{Q}) & := & P \equiv Q \\
  match_{\dagger}(\quotep{P},\quotep{Q}) & := & \forall R. P|Q \red^{*} R => R \red^{*} 0 \\
  match_{K}(\quotep{P},\quotep{Q}) & := & K \mbox{ for some context } K
\end{eqnarray*}

$u?(x)P | u!\langle Q \rangle \red P\{\quotep{Q}/x\}$

%We write $\wred$ for $\red^*$, and $P\red$ if $\exists Q $ such that $ P \red Q$.
We write $P\red$ if $\exists Q $ such that $ P \red Q$ and $P\not\red$, otherwise.

\section{Replication}

As mentioned before, it is known that replication (and hence
recursion) can be implemented in a higher-order process algebra
\cite{SangiorgiWalker}. As our first example of calculation with the
machinery thus far presented we give the construction explicitly in
the {\rhoc}.

\begin{eqnarray}
	D_{x} & := & \prefix{x}{y}{(\binpar{\outputp{x}{y}}{@{y}})} \nonumber\\
	\bangp_{x}{P} & := & \binpar{{x}!\langle{\binpar{D_{x}}{P}}\rangle}{D_{x}} \nonumber
\end{eqnarray}

\begin{eqnarray}
	\bangp_{x}{P} & & \nonumber\\
	=
	& {x}!\langle{(\prefix{x}{y}{(\outputp{x}{y} | @{y})) | P}}\rangle 
	      | \prefix{x}{y}{(\outputp{x}{y} | @{y})} & \nonumber\\
	\red
	& (\outputp{x}{y} | @{y})\substn{\quotep{(\prefix{x}{y}{(@{y} | \outputp{x}{y})) | P}}}{y} & \nonumber\\
	=
	& \outputp{x}{\quotep{(\prefix{x}{y}{(\outputp{x}{y} | @{y})) | P}}}
	  | {(\prefix{x}{y}{(\outputp{x}{y} | @{y})) | P}} & \nonumber\\
	\red
	& \ldots & \nonumber\\
	\red^*
	& P | P | \ldots & \nonumber
\end{eqnarray}

Of course, this encoding, as an implementation, runs away, unfolding
$\bangp{P}$ eagerly. A lazier and more implementable replication
operator, restricted to input-guarded processes, may be obtained as follows.

\begin{eqnarray}
\bangp{\prefix{u}{v}{P}} 
	:= 
	\binpar{\lift{x}{\prefix{u}{v}{(\binpar{D(x)}{P})}}}{D(x)} \nonumber
\end{eqnarray}

\begin{remark}
  Note that the lazier definition still does not deal with summation
  or mixed summation (i.e. sums over input and output). The reader is
  invited to construct definitions of replication that deal with these
  features. 

  Further, the definitions are parameterized in a name, $x$. Can you,
  gentle reader, make a definition that eliminates this parameter and
  guarantees no accidental interaction between the replication
  machinery and the process being replicated -- i.e. no accidental
  sharing of names used by the process to get its work done and the
  name(s) used by the replication to effect copying. This latter
  revision of the definition of replication is crucial to obtaining
  the expected identity $!!P \sim !P$.
\end{remark}

\begin{remark}\label{rem:paradoxical_combinator}
  The reader familiar with the lambda calculus will have noticed the
  similarity between $D$ and the paradoxical combinator.

  [Ed. note: the existence of this seems to suggest we have to be more
  restrictive on the set of processes and names we admit if we are to
  support no-cloning.]
\end{remark}

\subsubsection{Bisimulation}

The computational dynamics gives rise to another kind of equivalence,
the equivalence of computational behavior. As previously mentioned
this is typically captured \emph{via} some form of bisimulation.

% The notion we use in this paper is weak barbed bisimulation
% \cite{milner91polyadicpi}.

The notion we use in this paper is derived from weak barbed
bisimulation \cite{milner91polyadicpi}. 

\begin{definition}
An \emph{observation relation}, $\downarrow_{\mathcal N}$, over a set
of names, $\mathcal N$, is the smallest relation satisfying the rules
below.

\infrule[Out-barb]{y \in {\mathcal N}, \; x \nameeq y}
		  {\outputp{x}{v} \downarrow_{\mathcal N} x}
\infrule[Par-barb]{\mbox{$P\downarrow_{\mathcal N} x$ or $Q\downarrow_{\mathcal N} x$}}
		  {\binpar{P}{Q} \downarrow_{\mathcal N} x}

We write $P \Downarrow_{\mathcal N} x$ if there is $Q$ such that 
$P \wred Q$ and $Q \downarrow_{\mathcal N} x$.
\end{definition}

\begin{definition}
%\label{def.bbisim}
An  ${\mathcal N}$-\emph{barbed bisimulation} over a set of names, ${\mathcal N}$, is a symmetric binary relation 
${\mathcal S}_{\mathcal N}$ between agents such that $P\rel{S}_{\mathcal N}Q$ implies:
\begin{enumerate}
\item If $P \red P'$ then $Q \wred Q'$ and $P'\rel{S}_{\mathcal N} Q'$.
\item If $P\downarrow_{\mathcal N} x$, then $Q\Downarrow_{\mathcal N} x$.
\end{enumerate}
$P$ is ${\mathcal N}$-barbed bisimilar to $Q$, written
$P \wbbisim_{\mathcal N} Q$, if $P \rel{S}_{\mathcal N} Q$ for some ${\mathcal N}$-barbed bisimulation ${\mathcal S}_{\mathcal N}$.
\end{definition}

$\mathcal{R} \subseteq \pi \times \pi$

$P \mathcal{R} Q => \forall P'. P \red P' \Rightarrow \exists Q'. Q \red Q', P' \mathcal{R} Q'$

$P \vdash x \Rightarrow Q \vdash x$

\begin{mathpar}
  \inferrule*[lab=Out-barb]{x \nameeq y}{{y}!\langle{Q}\rangle \vdash x}
  \and
  \inferrule*[lab=Par-barb]{\mbox{$P\vdash x$ or $Q\vdash x$}}{\binpar{P}{Q} \vdash x}
\end{mathpar}

\subsubsection{Contexts}

One of the principle advantages of computational calculi like the
$\pi$-calculus is a well-defined notion of context,
contextual-equivalence and a correlation between
contextual-equivalence and notions of bisimulation. The notion of
context allows the decomposition of a process into (sub-)process and
its syntactic environment, its context. Thus, a context may be
thought of as a process with a ``hole'' (written $\Box$) in it. The
application of a context $M$ to a process $P$, written $M[P]$, is
tantamount to filling the hole in $M$ with $P$. In this paper we do
not need the full weight of this theory, but do make use of the notion
of context in the proof the main theorem. 

\begin{mathpar}
  \inferrule* [lab=summation] {} {{M_{M},M_{N}} \bc \Box \;|\; x.M_{A} \;|\; M_{M}+M_{N}}
  \and
  \inferrule* [lab=agent] {} {{M_{A}} \bc (\vec{x})M_{P} \;| \; \clift{P_0,\ldots,M_{P},\ldots,P_N}}
  \and \\
  \inferrule* [lab=process] {} {{M_{P}} \bc M_{N} \;| \;P|M_{P} }
\end{mathpar} 

\begin{mathpar}
  \inferrule* [lab=sychronization] {} {M_{N} \bc \Box \;|\; x?M_{F} \;|\; x!M_{C}}
  \and
  \inferrule* [lab=abstraction] {} {{M_{F}} \bc (x)M_{P} }
  \and
  \inferrule* [lab=concretion] {} {{M_{C}} \bc \langle M_{P} \rangle }
  \and \\
  \inferrule* [lab=process] {} {{M_{P}} \bc M_{N} \;| \;P|M_{P} }
\end{mathpar}

\begin{definition}[contextual application] Given a context $M$, and
  process $P$, we define the \emph{contextual application}, $M[P] :=
  M\{P/\Box\}$. That is, the contextual application of M to P is the
  substitution of $P$ for $\Box$ in $M$.
\end{definition}

$\meaningof{-} : L \to \mathcal{P}(\pi)$

\begin{mathpar}
  \inferrule* [lab=collection] {} {\meaningof{true} = \pi, \and \meaningof{~E} = \pi \setminus \meaningof{E}, \and \meaningof{E_{1} \& E_{2}} = \meaningof{E_{1}} \cap \meaningof{E_{2}}}
\end{mathpar}

\begin{mathpar}
  \inferrule* [lab=structure] {} {\meaningof{0} = \{ P \in \pi | P \equiv 0 \}, \and \\ \meaningof{E_1 | E_2} = \{ P \in \pi | P \equiv P_{1} | P_{2}, P_{1} \in \meaningof{E_{1}}, P_{2} \in \meaningof{E_2}\} }
\end{mathpar}

\begin{mathpar}
 \inferrule* [lab=behavior] {} {\meaningof{\langle a?b \rangle E} = \{ P \in \pi | P \equiv Q | u?(y)P', \\ \and \\\\ \and \\ \;\;\; u \in \meaningof{a}, \forall z.P'\{z/y\} \in \meaningof{E\{z/b\}}\}, \and \\ \meaningof{a!E} = \{ P \in \pi | P \equiv Q | x!\langle P' \rangle, x \in \meaningof{a} P' \in \meaningof{E}\} }
\end{mathpar}

\begin{mathpar}
 \inferrule* [lab=nominal] {} {\meaningof{\quotep{E}} = \{ \quotep{P} \in \quotep{\pi} | P \in \meaningof{E} \}, \and \meaningof{\quotep{P}} = \{ \quotep{Q} \in \quotep{\pi} | P \equiv Q \} \and \\ \meaningof{@\quotep{E}} = \{ P \in \pi | P \equiv @x, x \in \meaningof{E} \}}
\end{mathpar}

\begin{eqnarray*}
  \\
  \meaningof{-} : TS \to ST
\end{eqnarray*}

\begin{eqnarray*}
  \\
  L : TS \to ST
\end{eqnarray*}

\begin{eqnarray*}
  \\
  P \models E \iff P \in \meaningof{E}
\end{eqnarray*}

\begin{eqnarray*}
  P \approx_{L} Q \iff \forall E \in L. P \models E \iff Q \models E
\end{eqnarray*}

\begin{eqnarray*}
  P \approx_{K} Q
\end{eqnarray*}

\begin{eqnarray*}
  P \approx Q
\end{eqnarray*}

$\approx_{K} = \approx = \approx_{L}$

\subsubsection{Contextual duality}

Note that contexts extend the quotation operation to a family of
operations from processes to names. Given a context, $M$, we can
define a \emph{nominal context}, $\quotep{M}$ by $\quotep{M}[P] :=
\quotep{M[P]}$. To foreshadow what is to come we observe that these
operations enjoy a duality with processes very much like the duality
between vectors and maps from vectors to scalars.

Further, because the calculus is essentially higher-order, we have a
correspondence between contexts and processes. More specifically,
given a name $x$ and a context $M$ we can construct $M^{*}_{x}$ such
that 

\begin{mathpar}
  M^{*}_{x} | \lift{x}{P} \red M[P]
\end{mathpar}

namely,

\begin{mathpar}
  M^{*}_{x} := x?(u).M[\dropn{u}]
\end{mathpar}

The dependence of $M^{*}_{x}$ on a name makes it an abstraction, 

\begin{mathpar}
  M^{*} := (x)x?(u).M[\dropn{u}]
\end{mathpar}

\subsection{Additional notation}

It will sometimes be convenient to denote the process a name
quotes. We already have the notation $x = \quotep{P}$, but it will be
convenient to introduce an alternate notation, $\procn{x}$, when we
want to emphasize the connection to the use of the name. Note that, by
virtue of name equivalence, $\quotep{\procn{x}} \nameeq x$; so, the
notation is consistent with previous definitions.

Further, because names have structure it is possible to effect
substitutions on the basis of that structure. This means we need to
upgrade our notation for substitutions, which we accomplish by
adapting comprehension notation. Thus,

\begin{mathpar}
  P\{ y / x : x \in S \}
\end{mathpar}

is interpreted to mean the process derived from P by replacing (in a
capture-avoiding manner) each occurrence of $x$ in $S$ by $y$. For example,

\begin{mathpar}
  P\{ \quotep{\procn{x}|\procn{x}} / x : x \in \freenames{P} \}
\end{mathpar}

will replace each (occurrence) of a free name $x$ in $P$ by
$\quotep{\procn{x}|\procn{x}}$.

Also, we will avail ourselves of the notation $x^{L}$ and $x^{R}$ to
denote injections of a name into disjoint copies of the name
space. There are numerous ways to accomplish this. One example can be
found in \cite{MeredithR05}. This notation overloads to vectors of
names: $\vec{x}^{\pi} := (x_{i}^{\pi} \; : \; 0 \leq i < |\vec{x}| )$ where $\pi \in \{L,R\}$.

We also use $P^{\Box} := P|\Box$.

In \cite{MeredithR05} an interpretation of the new operator is
given. It turns out that there are several possible interpretations
all enjoying the requisite algebraic properties of the operator (see
\cite{milner91polyadicpi}). We will therefore make liberal use of
$(\nu\; \vec{x})P$.

% subsection the_syntax_and_semantics_of_the_notation_system (end)   

\input{qm2pi.qmops} 

\input{qm2pi.sterngerlach} 

\input{qm2pi.metric} 

% section concurrent_process_calculi (end)

%\input{qm2pi.proofsketch}

% section proof sketch (end)

%\input{qm2pi.slviaknots} 

% section spatial logic via knots (end)

\input{qm2pi.conclusion}

% section conclusion (end)

%\input{qm2pi.dtcodes} 

% section wiring algorithm (end)

\input{qm2pi.ack} 

% section acknowledgments (end)

\newpage


\bibliographystyle{plain}   
\bibliography{../../biblios/main.bib}

\input{qm2pi.rhodetails}

\end{document}

 

% section wiring algorithm (end)

\documentclass[12pt]{llncs}
%\documentclass{jktr}

\usepackage[pdftex]{hyperref}                   
\usepackage {listings}
\usepackage {mathpartir}
\usepackage{bcprules}
%\usepackage{listings}
                       
\usepackage{graphicx} 
%\usepackage[margins=2.5cm,nohead,nofoot]{geometry}
%\usepackage{geometry}
\usepackage{amsfonts}
\usepackage{amstext}
\usepackage{latexsym}
\usepackage{amssymb}
\usepackage{color}


%\include{myPreamble}
\include{qm2pi.local} 

%\ifpdf
%\usepackage[pdftex]{graphicx}
%\else
%\usepackage{graphicx}
%\fi

 % \ifpdf
%  \usepackage{pdfsync}
%  \if


%\title{Brief Article}
%\author{David F. Snyder}
%\author{L.G. Meredith}

%\address{Dept. of Math., Texas State University--San Marcos, San Marcos, TX 78666}
       
\pagestyle{empty}


\begin{document}

\lstset{language=[Objective]Caml,frame=shadowbox}

\input{qm2pi.front}

% section front matter (end)

\input{qm2pi.intro} 
 
% section introduction (end)

% \input{qm2pi.knotations} 

% section notation (end)

\input{qm2pi.process.calculi} 

% section concurrent_process_calculi_and_spatial_logics_ (end)
    
%\input{qm2pi.knots2pi} 

%\input{qm2pi.trefoil} 

%\input{qm2pi.mainthm} 

% subsection basic_interpretation (end)

%\input{qm2pi.rho.presentation} 
\subsection{The syntax and semantics of the notation system}\label{sub:the_syntax_and_semantics_of_the_notation_system} % (fold)

We now summarize a technical presentation of the calculus that
embodies our theory of dynamics. The typical presentation of such a
calculus follows the style of giving generators and relations on
them. The grammar, below, describing term constructors, freely
generates the set of processes, $\Proc$. This set is then quotiented
by a relation known as structural congruence and it is over this set
that the notion of dynamics is expressed. This presentation is
essentially that of \cite{MeredithR05} with the addition of
polyadicity and summation. For readability we have relegated some of
the technical subtleties to an appendix.

\subsubsection{Process grammar}\label{subsub:process_grammar}

\begin{mathpar}
  \inferrule* [lab=synchronization] {} {{M} \bc \pzero \;|\; x?F \;|\; x!C }
  \and
  \inferrule* [lab=abstraction] {} {{F} \bc (x)P}
  \and
  \inferrule* [lab=concretion] {} {{C} \bc \langle Q \rangle}
  \and
  \inferrule* [lab=process] {} {{P,Q} \bc M \;| \;P|Q \;|\; @{x}}
  \and
  \inferrule* [lab=name] {} {{x} \bc \quotep{P}}
\end{mathpar} 

Note that $\vec{x}$ (resp. $\vec{P}$) denotes a vector of names
(resp. processes) of length $|\vec{x}|$ (resp. $|\vec{P}|$). We adopt
the following useful abbreviations.

\begin{mathpar}
   x?(\vec{y}).P := x.(\vec{y})P \and  x\clift{\vec{P}} := x.\clift{\vec{P}}
   \and x!(y) := \lift{x}{\dropn{y}}
   \and \Pi_{i=0}^{n-1}P_i := P_0 | \ldots | P_{n-1}
\end{mathpar}

\subsubsection{Structural congruence}

\paragraph{Free and bound names and alpha-equivalence.} At the
core of structural equivalence is alpha-equivalence which identifies
process that are the same up to a change of variable. Formally, we
recognize the distinction between free and bound names. The free names
of a process, $\freenames{P}$, may be calculated recursively as
follows:

\begin{mathpar}
\freenames{\pzero} := \emptyset
  \and \\
  \freenames{x?(y).P} := \{ x \} \cup (\freenames{P} \setminus \{ y \})
  \and 
  \freenames{x!\langle P \rangle} := \{ x \} \cup \{ P \} 
  \and \\
  \freenames{P|Q} := \freenames{P} \cup \freenames{Q}
  \and \\
  \freenames{@{x}} := \{ x \}
\end{mathpar}

$\pi$
$\quotep{\pi}$

$\freenames{-} : \pi \to \mathcal{P}(\quotep{\pi})$

\begin{eqnarray*}
  \freenames{\pzero} & := & \emptyset \\
  \freenames{x?(y).P} & := & \{ x \} \cup (\freenames{P} \setminus \{ y \}) \\
  \freenames{x!\langle P \rangle} & := & \{ x \} \cup \{ P \} \\
  \freenames{P|Q} & := & \freenames{P} \cup \freenames{Q} \\
  \freenames{\dropn{x}} & := & \{ x \}
\end{eqnarray*}

The bound names of a process, $\boundnames{P}$, are those names occurring in $P$
that are not free. For example, in $x?(y).0$, the name $x$ is free, while $y$ is bound.

\begin{mathpar}
  \inferrule* [lab=monoidal-laws] {} { P|Q \equiv Q|P \and P|0 \equiv P \and P|(Q|R) \equiv (P|Q)|R }
\end{mathpar}

\begin{mathpar}
  \inferrule* [lab=alpha-equivalence] {} { (x)P \equiv (y)P\{y/x\} \and y \not\in \freenames{P} }
\end{mathpar}

\begin{definition}
Then two processes, $P,Q$, are alpha-equivalent if $P = Q\{\vec{y}/\vec{x}\}$ for
some $\vec{x} \in \boundnames{Q},\vec{y} \in \boundnames{P}$, where $Q\{\vec{y}/\vec{x}\}$
denotes the capture-avoiding substitution of $\vec{y}$ for $\vec{x}$ in $Q$.
\end{definition}

\begin{definition}
  The {\em structural congruence} \cite{SangiorgiWalker} , $\equiv$,
  between processes is the least congruence containing
  alpha-equivalence, satisfying the abelian monoid laws
  (associativity, commutativity and $\pzero$ as identity) for parallel
  composition $|$ and for summation $+$.
\end{definition}

\subsection{Name equivalence}

We take name equivalence, written $\nameeq$, to be the smallest
equivalence relation generated by the following rules.

\begin{mathpar}
\inferrule*[lab=Quote-drop]
{ }
{ \quotep{@{x}} \nameeq x }

\inferrule*[lab=Struct-equiv]
{ P \scong Q }
{ \quotep{P} \nameeq \quotep{Q} }
\end{mathpar}

The astute reader will have noticed that the mutual recursion of names
and processes imposes a mutual recursion on alpha-equivalence and
structural equivalence via name-equivalence. Fortunately, all of this
works out pleasantly and we may calculate in the natural way, free of
concern. The reader interested in the details is referred to the
appendix \ref{appendix:rho_details}.

\subsection{Substitution}

We use $\Proc$ for the set of processes, $\QProc$ for the set of
names, and $\id{\{}\vec{y} / \vec{x} \id{\}}$ to denote partial maps,
$s : \QProc \rightarrow \QProc$. A map, $s$ lifts, uniquely, to a map
on process terms, $\widehat{s} : \Proc \rightarrow \Proc$ by the
following equations.

\begin{mathpar}
  (0) \psubstp{Q}{P} := 0 \\
  (R \juxtap S) \psubstp{Q}{P}
  :=    
  (R)\psubstp{Q}{P} \juxtap (S) \psubstp{Q}{P} \\
  (x?(y).R) \psubstp{Q}{P}    
  :=    
  (x)\substp{Q}{P} (z)\concat( (R \psubstn{z}{y}) \psubstp{Q}{P} ) \\
  (\lift{x}{R}) \psubstp{Q}{P}  
  :=
  \lift{(x)\substp{Q}{P}}{ R \psubstp{Q}{P} } \\
%   (\dropn{x})  \psubstp{Q}{P}       
%   := 
%   \left\{ 
%     \begin{array}{ccc} 
%       \dropn{\quotep{Q}} & & x \nameeq \quotep{P} \\
%       \dropn{x} & & otherwise \\
%     \end{array}
%   \right. 
  (\dropn{x})  \psubstp{Q}{P}       
  := 
  \left\{ 
    \begin{array}{ccc} 
      Q & & x \nameeq \quotep{P} \\
      \dropn{x} & & otherwise \\
    \end{array}
  \right.
\end{mathpar}
 

where

\begin{eqnarray}
  (x)\id{\{} \lpquote Q \rpquote / \lpquote P \rpquote \id{\}}            = 
  \left\{ 
    \begin{array}{ccc}
      \lpquote Q \rpquote & & x \nameeq \lpquote P \rpquote \\
      x & & otherwise \\
    \end{array}
  \right. \nonumber
\end{eqnarray}

and $z$ is chosen distinct from $\quotep{P}$, $\quotep{Q}$, the free
names in $Q$, and all the names in $R$. Our $\alpha$-equivalence will
be built in the standard way from this substitution.

\begin{remark}\label{rem:no_self_referential_names}
  One consequence of these definitions is that $\forall P. \quotep{P}
  \not\in \freenames{P}$.
\end{remark}

\subsection{ Dynamic quote: an example }

Anticipating something of what's to come, consider applying the
substitution, $\widehat{\id{\{}u / z \id{\}}}$, to the following pair
of processes, $\lift{w}{y!(z)}$ and $w[ \lpquote y!(z) \rpquote ]$.

\begin{eqnarray}
	\lift{w}{y!(z)}\widehat{\id{\{}u / z \id{\}}}
		& = &
		\lift{w}{y!(u)} \nonumber\\
	w[ \lpquote y!(z) \rpquote ] \widehat{ \id{\{}u / z \id{\}} }
		& = &
		w[ \lpquote y!(z) \rpquote ] \nonumber
\end{eqnarray}

Because the body of the process between quotes is impervious to
substitution, we get radically different answers. In fact, by
examining the first process in an input context,
e.g. $x?(z).\lift{w}{y!(z)}$, we see that the process under the lift
operator may be shaped by prefixed inputs binding a name inside it. In
this sense, the lift operator will be seen as a way to dynamically
construct processes before reifying them as names.

Finally equipped with these standard features we can present the
dynamics of the calculus.

\subsubsection{Operational semantics} 

Finally, we introduce the computational dynamics. What marks these
algebras as distinct from other more traditionally studied algebraic
structures, e.g. vector spaces or polynomial rings, is the manner in
which dynamics is captured. In traditional structures, dynamics is typically
expressed through morphisms between such structures, as in linear maps
between vector spaces or morphisms between rings. In algebras
associated with the semantics of computation, the dynamics is
expressed as part of the algebraic structure itself, through a
reduction reduction relation typically denoted by $\red$. Below, we
give a recursive presentation of this relation for the calculus used
in the encoding.

$\red \subseteq \pi \times \pi$
$\red : \pi \to \mathcal{P}(\pi)$

\begin{mathpar}
  \inferrule* [lab=Comm] { \textsf{match}( x_{src}, x_{trgt} ) } { x_{trgt}?(y)P \; | \; x_{src}!\langle {Q} \rangle \red P\{\quotep{Q}/y}\} }
  \and \\
  \inferrule* [lab=Par] {{P} \red {P}'} {{{P} | {Q}} \red {{P}' | {Q}}}
  \and
  \inferrule* [lab=Equiv]{{{P} \scong {P}'} \andalso {{P}' \red {Q}'} \andalso {{Q}' \scong {Q}}}{{P} \red {Q}}
\end{mathpar}

\begin{eqnarray*}
  match_{\equiv} (\quotep{P},\quotep{Q}) & := & P \equiv Q \\
  match_{\dagger}(\quotep{P},\quotep{Q}) & := & \forall R. P|Q \red^{*} R => R \red^{*} 0 \\
  match_{K}(\quotep{P},\quotep{Q}) & := & K \mbox{ for some context } K
\end{eqnarray*}

$u?(x)P | u!\langle Q \rangle \red P\{\quotep{Q}/x\}$

%We write $\wred$ for $\red^*$, and $P\red$ if $\exists Q $ such that $ P \red Q$.
We write $P\red$ if $\exists Q $ such that $ P \red Q$ and $P\not\red$, otherwise.

\section{Replication}

As mentioned before, it is known that replication (and hence
recursion) can be implemented in a higher-order process algebra
\cite{SangiorgiWalker}. As our first example of calculation with the
machinery thus far presented we give the construction explicitly in
the {\rhoc}.

\begin{eqnarray}
	D_{x} & := & \prefix{x}{y}{(\binpar{\outputp{x}{y}}{@{y}})} \nonumber\\
	\bangp_{x}{P} & := & \binpar{{x}!\langle{\binpar{D_{x}}{P}}\rangle}{D_{x}} \nonumber
\end{eqnarray}

\begin{eqnarray}
	\bangp_{x}{P} & & \nonumber\\
	=
	& {x}!\langle{(\prefix{x}{y}{(\outputp{x}{y} | @{y})) | P}}\rangle 
	      | \prefix{x}{y}{(\outputp{x}{y} | @{y})} & \nonumber\\
	\red
	& (\outputp{x}{y} | @{y})\substn{\quotep{(\prefix{x}{y}{(@{y} | \outputp{x}{y})) | P}}}{y} & \nonumber\\
	=
	& \outputp{x}{\quotep{(\prefix{x}{y}{(\outputp{x}{y} | @{y})) | P}}}
	  | {(\prefix{x}{y}{(\outputp{x}{y} | @{y})) | P}} & \nonumber\\
	\red
	& \ldots & \nonumber\\
	\red^*
	& P | P | \ldots & \nonumber
\end{eqnarray}

Of course, this encoding, as an implementation, runs away, unfolding
$\bangp{P}$ eagerly. A lazier and more implementable replication
operator, restricted to input-guarded processes, may be obtained as follows.

\begin{eqnarray}
\bangp{\prefix{u}{v}{P}} 
	:= 
	\binpar{\lift{x}{\prefix{u}{v}{(\binpar{D(x)}{P})}}}{D(x)} \nonumber
\end{eqnarray}

\begin{remark}
  Note that the lazier definition still does not deal with summation
  or mixed summation (i.e. sums over input and output). The reader is
  invited to construct definitions of replication that deal with these
  features. 

  Further, the definitions are parameterized in a name, $x$. Can you,
  gentle reader, make a definition that eliminates this parameter and
  guarantees no accidental interaction between the replication
  machinery and the process being replicated -- i.e. no accidental
  sharing of names used by the process to get its work done and the
  name(s) used by the replication to effect copying. This latter
  revision of the definition of replication is crucial to obtaining
  the expected identity $!!P \sim !P$.
\end{remark}

\begin{remark}\label{rem:paradoxical_combinator}
  The reader familiar with the lambda calculus will have noticed the
  similarity between $D$ and the paradoxical combinator.

  [Ed. note: the existence of this seems to suggest we have to be more
  restrictive on the set of processes and names we admit if we are to
  support no-cloning.]
\end{remark}

\subsubsection{Bisimulation}

The computational dynamics gives rise to another kind of equivalence,
the equivalence of computational behavior. As previously mentioned
this is typically captured \emph{via} some form of bisimulation.

% The notion we use in this paper is weak barbed bisimulation
% \cite{milner91polyadicpi}.

The notion we use in this paper is derived from weak barbed
bisimulation \cite{milner91polyadicpi}. 

\begin{definition}
An \emph{observation relation}, $\downarrow_{\mathcal N}$, over a set
of names, $\mathcal N$, is the smallest relation satisfying the rules
below.

\infrule[Out-barb]{y \in {\mathcal N}, \; x \nameeq y}
		  {\outputp{x}{v} \downarrow_{\mathcal N} x}
\infrule[Par-barb]{\mbox{$P\downarrow_{\mathcal N} x$ or $Q\downarrow_{\mathcal N} x$}}
		  {\binpar{P}{Q} \downarrow_{\mathcal N} x}

We write $P \Downarrow_{\mathcal N} x$ if there is $Q$ such that 
$P \wred Q$ and $Q \downarrow_{\mathcal N} x$.
\end{definition}

\begin{definition}
%\label{def.bbisim}
An  ${\mathcal N}$-\emph{barbed bisimulation} over a set of names, ${\mathcal N}$, is a symmetric binary relation 
${\mathcal S}_{\mathcal N}$ between agents such that $P\rel{S}_{\mathcal N}Q$ implies:
\begin{enumerate}
\item If $P \red P'$ then $Q \wred Q'$ and $P'\rel{S}_{\mathcal N} Q'$.
\item If $P\downarrow_{\mathcal N} x$, then $Q\Downarrow_{\mathcal N} x$.
\end{enumerate}
$P$ is ${\mathcal N}$-barbed bisimilar to $Q$, written
$P \wbbisim_{\mathcal N} Q$, if $P \rel{S}_{\mathcal N} Q$ for some ${\mathcal N}$-barbed bisimulation ${\mathcal S}_{\mathcal N}$.
\end{definition}

$\mathcal{R} \subseteq \pi \times \pi$

$P \mathcal{R} Q => \forall P'. P \red P' \Rightarrow \exists Q'. Q \red Q', P' \mathcal{R} Q'$

$P \vdash x \Rightarrow Q \vdash x$

\begin{mathpar}
  \inferrule*[lab=Out-barb]{x \nameeq y}{{y}!\langle{Q}\rangle \vdash x}
  \and
  \inferrule*[lab=Par-barb]{\mbox{$P\vdash x$ or $Q\vdash x$}}{\binpar{P}{Q} \vdash x}
\end{mathpar}

\subsubsection{Contexts}

One of the principle advantages of computational calculi like the
$\pi$-calculus is a well-defined notion of context,
contextual-equivalence and a correlation between
contextual-equivalence and notions of bisimulation. The notion of
context allows the decomposition of a process into (sub-)process and
its syntactic environment, its context. Thus, a context may be
thought of as a process with a ``hole'' (written $\Box$) in it. The
application of a context $M$ to a process $P$, written $M[P]$, is
tantamount to filling the hole in $M$ with $P$. In this paper we do
not need the full weight of this theory, but do make use of the notion
of context in the proof the main theorem. 

\begin{mathpar}
  \inferrule* [lab=summation] {} {{M_{M},M_{N}} \bc \Box \;|\; x.M_{A} \;|\; M_{M}+M_{N}}
  \and
  \inferrule* [lab=agent] {} {{M_{A}} \bc (\vec{x})M_{P} \;| \; \clift{P_0,\ldots,M_{P},\ldots,P_N}}
  \and \\
  \inferrule* [lab=process] {} {{M_{P}} \bc M_{N} \;| \;P|M_{P} }
\end{mathpar} 

\begin{mathpar}
  \inferrule* [lab=sychronization] {} {M_{N} \bc \Box \;|\; x?M_{F} \;|\; x!M_{C}}
  \and
  \inferrule* [lab=abstraction] {} {{M_{F}} \bc (x)M_{P} }
  \and
  \inferrule* [lab=concretion] {} {{M_{C}} \bc \langle M_{P} \rangle }
  \and \\
  \inferrule* [lab=process] {} {{M_{P}} \bc M_{N} \;| \;P|M_{P} }
\end{mathpar}

\begin{definition}[contextual application] Given a context $M$, and
  process $P$, we define the \emph{contextual application}, $M[P] :=
  M\{P/\Box\}$. That is, the contextual application of M to P is the
  substitution of $P$ for $\Box$ in $M$.
\end{definition}

$\meaningof{-} : L \to \mathcal{P}(\pi)$

\begin{mathpar}
  \inferrule* [lab=collection] {} {\meaningof{true} = \pi, \and \meaningof{~E} = \pi \setminus \meaningof{E}, \and \meaningof{E_{1} \& E_{2}} = \meaningof{E_{1}} \cap \meaningof{E_{2}}}
\end{mathpar}

\begin{mathpar}
  \inferrule* [lab=structure] {} {\meaningof{0} = \{ P \in \pi | P \equiv 0 \}, \and \\ \meaningof{E_1 | E_2} = \{ P \in \pi | P \equiv P_{1} | P_{2}, P_{1} \in \meaningof{E_{1}}, P_{2} \in \meaningof{E_2}\} }
\end{mathpar}

\begin{mathpar}
 \inferrule* [lab=behavior] {} {\meaningof{\langle a?b \rangle E} = \{ P \in \pi | P \equiv Q | u?(y)P', \\ \and \\\\ \and \\ \;\;\; u \in \meaningof{a}, \forall z.P'\{z/y\} \in \meaningof{E\{z/b\}}\}, \and \\ \meaningof{a!E} = \{ P \in \pi | P \equiv Q | x!\langle P' \rangle, x \in \meaningof{a} P' \in \meaningof{E}\} }
\end{mathpar}

\begin{mathpar}
 \inferrule* [lab=nominal] {} {\meaningof{\quotep{E}} = \{ \quotep{P} \in \quotep{\pi} | P \in \meaningof{E} \}, \and \meaningof{\quotep{P}} = \{ \quotep{Q} \in \quotep{\pi} | P \equiv Q \} \and \\ \meaningof{@\quotep{E}} = \{ P \in \pi | P \equiv @x, x \in \meaningof{E} \}}
\end{mathpar}

\begin{eqnarray*}
  \\
  \meaningof{-} : TS \to ST
\end{eqnarray*}

\begin{eqnarray*}
  \\
  L : TS \to ST
\end{eqnarray*}

\begin{eqnarray*}
  \\
  P \models E \iff P \in \meaningof{E}
\end{eqnarray*}

\begin{eqnarray*}
  P \approx_{L} Q \iff \forall E \in L. P \models E \iff Q \models E
\end{eqnarray*}

\begin{eqnarray*}
  P \approx_{K} Q
\end{eqnarray*}

\begin{eqnarray*}
  P \approx Q
\end{eqnarray*}

$\approx_{K} = \approx = \approx_{L}$

\subsubsection{Contextual duality}

Note that contexts extend the quotation operation to a family of
operations from processes to names. Given a context, $M$, we can
define a \emph{nominal context}, $\quotep{M}$ by $\quotep{M}[P] :=
\quotep{M[P]}$. To foreshadow what is to come we observe that these
operations enjoy a duality with processes very much like the duality
between vectors and maps from vectors to scalars.

Further, because the calculus is essentially higher-order, we have a
correspondence between contexts and processes. More specifically,
given a name $x$ and a context $M$ we can construct $M^{*}_{x}$ such
that 

\begin{mathpar}
  M^{*}_{x} | \lift{x}{P} \red M[P]
\end{mathpar}

namely,

\begin{mathpar}
  M^{*}_{x} := x?(u).M[\dropn{u}]
\end{mathpar}

The dependence of $M^{*}_{x}$ on a name makes it an abstraction, 

\begin{mathpar}
  M^{*} := (x)x?(u).M[\dropn{u}]
\end{mathpar}

\subsection{Additional notation}

It will sometimes be convenient to denote the process a name
quotes. We already have the notation $x = \quotep{P}$, but it will be
convenient to introduce an alternate notation, $\procn{x}$, when we
want to emphasize the connection to the use of the name. Note that, by
virtue of name equivalence, $\quotep{\procn{x}} \nameeq x$; so, the
notation is consistent with previous definitions.

Further, because names have structure it is possible to effect
substitutions on the basis of that structure. This means we need to
upgrade our notation for substitutions, which we accomplish by
adapting comprehension notation. Thus,

\begin{mathpar}
  P\{ y / x : x \in S \}
\end{mathpar}

is interpreted to mean the process derived from P by replacing (in a
capture-avoiding manner) each occurrence of $x$ in $S$ by $y$. For example,

\begin{mathpar}
  P\{ \quotep{\procn{x}|\procn{x}} / x : x \in \freenames{P} \}
\end{mathpar}

will replace each (occurrence) of a free name $x$ in $P$ by
$\quotep{\procn{x}|\procn{x}}$.

Also, we will avail ourselves of the notation $x^{L}$ and $x^{R}$ to
denote injections of a name into disjoint copies of the name
space. There are numerous ways to accomplish this. One example can be
found in \cite{MeredithR05}. This notation overloads to vectors of
names: $\vec{x}^{\pi} := (x_{i}^{\pi} \; : \; 0 \leq i < |\vec{x}| )$ where $\pi \in \{L,R\}$.

We also use $P^{\Box} := P|\Box$.

In \cite{MeredithR05} an interpretation of the new operator is
given. It turns out that there are several possible interpretations
all enjoying the requisite algebraic properties of the operator (see
\cite{milner91polyadicpi}). We will therefore make liberal use of
$(\nu\; \vec{x})P$.

% subsection the_syntax_and_semantics_of_the_notation_system (end)   

\input{qm2pi.qmops} 

\input{qm2pi.sterngerlach} 

\input{qm2pi.metric} 

% section concurrent_process_calculi (end)

%\input{qm2pi.proofsketch}

% section proof sketch (end)

%\input{qm2pi.slviaknots} 

% section spatial logic via knots (end)

\input{qm2pi.conclusion}

% section conclusion (end)

%\input{qm2pi.dtcodes} 

% section wiring algorithm (end)

\input{qm2pi.ack} 

% section acknowledgments (end)

\newpage


\bibliographystyle{plain}   
\bibliography{../../biblios/main.bib}

\input{qm2pi.rhodetails}

\end{document}

 

% section acknowledgments (end)

\newpage


\bibliographystyle{plain}   
\bibliography{../../biblios/main.bib}

\documentclass[12pt]{llncs}
%\documentclass{jktr}

\usepackage[pdftex]{hyperref}                   
\usepackage {listings}
\usepackage {mathpartir}
\usepackage{bcprules}
%\usepackage{listings}
                       
\usepackage{graphicx} 
%\usepackage[margins=2.5cm,nohead,nofoot]{geometry}
%\usepackage{geometry}
\usepackage{amsfonts}
\usepackage{amstext}
\usepackage{latexsym}
\usepackage{amssymb}
\usepackage{color}


%\include{myPreamble}
\include{qm2pi.local} 

%\ifpdf
%\usepackage[pdftex]{graphicx}
%\else
%\usepackage{graphicx}
%\fi

 % \ifpdf
%  \usepackage{pdfsync}
%  \if


%\title{Brief Article}
%\author{David F. Snyder}
%\author{L.G. Meredith}

%\address{Dept. of Math., Texas State University--San Marcos, San Marcos, TX 78666}
       
\pagestyle{empty}


\begin{document}

\lstset{language=[Objective]Caml,frame=shadowbox}

\input{qm2pi.front}

% section front matter (end)

\input{qm2pi.intro} 
 
% section introduction (end)

% \input{qm2pi.knotations} 

% section notation (end)

\input{qm2pi.process.calculi} 

% section concurrent_process_calculi_and_spatial_logics_ (end)
    
%\input{qm2pi.knots2pi} 

%\input{qm2pi.trefoil} 

%\input{qm2pi.mainthm} 

% subsection basic_interpretation (end)

%\input{qm2pi.rho.presentation} 
\subsection{The syntax and semantics of the notation system}\label{sub:the_syntax_and_semantics_of_the_notation_system} % (fold)

We now summarize a technical presentation of the calculus that
embodies our theory of dynamics. The typical presentation of such a
calculus follows the style of giving generators and relations on
them. The grammar, below, describing term constructors, freely
generates the set of processes, $\Proc$. This set is then quotiented
by a relation known as structural congruence and it is over this set
that the notion of dynamics is expressed. This presentation is
essentially that of \cite{MeredithR05} with the addition of
polyadicity and summation. For readability we have relegated some of
the technical subtleties to an appendix.

\subsubsection{Process grammar}\label{subsub:process_grammar}

\begin{mathpar}
  \inferrule* [lab=synchronization] {} {{M} \bc \pzero \;|\; x?F \;|\; x!C }
  \and
  \inferrule* [lab=abstraction] {} {{F} \bc (x)P}
  \and
  \inferrule* [lab=concretion] {} {{C} \bc \langle Q \rangle}
  \and
  \inferrule* [lab=process] {} {{P,Q} \bc M \;| \;P|Q \;|\; @{x}}
  \and
  \inferrule* [lab=name] {} {{x} \bc \quotep{P}}
\end{mathpar} 

Note that $\vec{x}$ (resp. $\vec{P}$) denotes a vector of names
(resp. processes) of length $|\vec{x}|$ (resp. $|\vec{P}|$). We adopt
the following useful abbreviations.

\begin{mathpar}
   x?(\vec{y}).P := x.(\vec{y})P \and  x\clift{\vec{P}} := x.\clift{\vec{P}}
   \and x!(y) := \lift{x}{\dropn{y}}
   \and \Pi_{i=0}^{n-1}P_i := P_0 | \ldots | P_{n-1}
\end{mathpar}

\subsubsection{Structural congruence}

\paragraph{Free and bound names and alpha-equivalence.} At the
core of structural equivalence is alpha-equivalence which identifies
process that are the same up to a change of variable. Formally, we
recognize the distinction between free and bound names. The free names
of a process, $\freenames{P}$, may be calculated recursively as
follows:

\begin{mathpar}
\freenames{\pzero} := \emptyset
  \and \\
  \freenames{x?(y).P} := \{ x \} \cup (\freenames{P} \setminus \{ y \})
  \and 
  \freenames{x!\langle P \rangle} := \{ x \} \cup \{ P \} 
  \and \\
  \freenames{P|Q} := \freenames{P} \cup \freenames{Q}
  \and \\
  \freenames{@{x}} := \{ x \}
\end{mathpar}

$\pi$
$\quotep{\pi}$

$\freenames{-} : \pi \to \mathcal{P}(\quotep{\pi})$

\begin{eqnarray*}
  \freenames{\pzero} & := & \emptyset \\
  \freenames{x?(y).P} & := & \{ x \} \cup (\freenames{P} \setminus \{ y \}) \\
  \freenames{x!\langle P \rangle} & := & \{ x \} \cup \{ P \} \\
  \freenames{P|Q} & := & \freenames{P} \cup \freenames{Q} \\
  \freenames{\dropn{x}} & := & \{ x \}
\end{eqnarray*}

The bound names of a process, $\boundnames{P}$, are those names occurring in $P$
that are not free. For example, in $x?(y).0$, the name $x$ is free, while $y$ is bound.

\begin{mathpar}
  \inferrule* [lab=monoidal-laws] {} { P|Q \equiv Q|P \and P|0 \equiv P \and P|(Q|R) \equiv (P|Q)|R }
\end{mathpar}

\begin{mathpar}
  \inferrule* [lab=alpha-equivalence] {} { (x)P \equiv (y)P\{y/x\} \and y \not\in \freenames{P} }
\end{mathpar}

\begin{definition}
Then two processes, $P,Q$, are alpha-equivalent if $P = Q\{\vec{y}/\vec{x}\}$ for
some $\vec{x} \in \boundnames{Q},\vec{y} \in \boundnames{P}$, where $Q\{\vec{y}/\vec{x}\}$
denotes the capture-avoiding substitution of $\vec{y}$ for $\vec{x}$ in $Q$.
\end{definition}

\begin{definition}
  The {\em structural congruence} \cite{SangiorgiWalker} , $\equiv$,
  between processes is the least congruence containing
  alpha-equivalence, satisfying the abelian monoid laws
  (associativity, commutativity and $\pzero$ as identity) for parallel
  composition $|$ and for summation $+$.
\end{definition}

\subsection{Name equivalence}

We take name equivalence, written $\nameeq$, to be the smallest
equivalence relation generated by the following rules.

\begin{mathpar}
\inferrule*[lab=Quote-drop]
{ }
{ \quotep{@{x}} \nameeq x }

\inferrule*[lab=Struct-equiv]
{ P \scong Q }
{ \quotep{P} \nameeq \quotep{Q} }
\end{mathpar}

The astute reader will have noticed that the mutual recursion of names
and processes imposes a mutual recursion on alpha-equivalence and
structural equivalence via name-equivalence. Fortunately, all of this
works out pleasantly and we may calculate in the natural way, free of
concern. The reader interested in the details is referred to the
appendix \ref{appendix:rho_details}.

\subsection{Substitution}

We use $\Proc$ for the set of processes, $\QProc$ for the set of
names, and $\id{\{}\vec{y} / \vec{x} \id{\}}$ to denote partial maps,
$s : \QProc \rightarrow \QProc$. A map, $s$ lifts, uniquely, to a map
on process terms, $\widehat{s} : \Proc \rightarrow \Proc$ by the
following equations.

\begin{mathpar}
  (0) \psubstp{Q}{P} := 0 \\
  (R \juxtap S) \psubstp{Q}{P}
  :=    
  (R)\psubstp{Q}{P} \juxtap (S) \psubstp{Q}{P} \\
  (x?(y).R) \psubstp{Q}{P}    
  :=    
  (x)\substp{Q}{P} (z)\concat( (R \psubstn{z}{y}) \psubstp{Q}{P} ) \\
  (\lift{x}{R}) \psubstp{Q}{P}  
  :=
  \lift{(x)\substp{Q}{P}}{ R \psubstp{Q}{P} } \\
%   (\dropn{x})  \psubstp{Q}{P}       
%   := 
%   \left\{ 
%     \begin{array}{ccc} 
%       \dropn{\quotep{Q}} & & x \nameeq \quotep{P} \\
%       \dropn{x} & & otherwise \\
%     \end{array}
%   \right. 
  (\dropn{x})  \psubstp{Q}{P}       
  := 
  \left\{ 
    \begin{array}{ccc} 
      Q & & x \nameeq \quotep{P} \\
      \dropn{x} & & otherwise \\
    \end{array}
  \right.
\end{mathpar}
 

where

\begin{eqnarray}
  (x)\id{\{} \lpquote Q \rpquote / \lpquote P \rpquote \id{\}}            = 
  \left\{ 
    \begin{array}{ccc}
      \lpquote Q \rpquote & & x \nameeq \lpquote P \rpquote \\
      x & & otherwise \\
    \end{array}
  \right. \nonumber
\end{eqnarray}

and $z$ is chosen distinct from $\quotep{P}$, $\quotep{Q}$, the free
names in $Q$, and all the names in $R$. Our $\alpha$-equivalence will
be built in the standard way from this substitution.

\begin{remark}\label{rem:no_self_referential_names}
  One consequence of these definitions is that $\forall P. \quotep{P}
  \not\in \freenames{P}$.
\end{remark}

\subsection{ Dynamic quote: an example }

Anticipating something of what's to come, consider applying the
substitution, $\widehat{\id{\{}u / z \id{\}}}$, to the following pair
of processes, $\lift{w}{y!(z)}$ and $w[ \lpquote y!(z) \rpquote ]$.

\begin{eqnarray}
	\lift{w}{y!(z)}\widehat{\id{\{}u / z \id{\}}}
		& = &
		\lift{w}{y!(u)} \nonumber\\
	w[ \lpquote y!(z) \rpquote ] \widehat{ \id{\{}u / z \id{\}} }
		& = &
		w[ \lpquote y!(z) \rpquote ] \nonumber
\end{eqnarray}

Because the body of the process between quotes is impervious to
substitution, we get radically different answers. In fact, by
examining the first process in an input context,
e.g. $x?(z).\lift{w}{y!(z)}$, we see that the process under the lift
operator may be shaped by prefixed inputs binding a name inside it. In
this sense, the lift operator will be seen as a way to dynamically
construct processes before reifying them as names.

Finally equipped with these standard features we can present the
dynamics of the calculus.

\subsubsection{Operational semantics} 

Finally, we introduce the computational dynamics. What marks these
algebras as distinct from other more traditionally studied algebraic
structures, e.g. vector spaces or polynomial rings, is the manner in
which dynamics is captured. In traditional structures, dynamics is typically
expressed through morphisms between such structures, as in linear maps
between vector spaces or morphisms between rings. In algebras
associated with the semantics of computation, the dynamics is
expressed as part of the algebraic structure itself, through a
reduction reduction relation typically denoted by $\red$. Below, we
give a recursive presentation of this relation for the calculus used
in the encoding.

$\red \subseteq \pi \times \pi$
$\red : \pi \to \mathcal{P}(\pi)$

\begin{mathpar}
  \inferrule* [lab=Comm] { \textsf{match}( x_{src}, x_{trgt} ) } { x_{trgt}?(y)P \; | \; x_{src}!\langle {Q} \rangle \red P\{\quotep{Q}/y}\} }
  \and \\
  \inferrule* [lab=Par] {{P} \red {P}'} {{{P} | {Q}} \red {{P}' | {Q}}}
  \and
  \inferrule* [lab=Equiv]{{{P} \scong {P}'} \andalso {{P}' \red {Q}'} \andalso {{Q}' \scong {Q}}}{{P} \red {Q}}
\end{mathpar}

\begin{eqnarray*}
  match_{\equiv} (\quotep{P},\quotep{Q}) & := & P \equiv Q \\
  match_{\dagger}(\quotep{P},\quotep{Q}) & := & \forall R. P|Q \red^{*} R => R \red^{*} 0 \\
  match_{K}(\quotep{P},\quotep{Q}) & := & K \mbox{ for some context } K
\end{eqnarray*}

$u?(x)P | u!\langle Q \rangle \red P\{\quotep{Q}/x\}$

%We write $\wred$ for $\red^*$, and $P\red$ if $\exists Q $ such that $ P \red Q$.
We write $P\red$ if $\exists Q $ such that $ P \red Q$ and $P\not\red$, otherwise.

\section{Replication}

As mentioned before, it is known that replication (and hence
recursion) can be implemented in a higher-order process algebra
\cite{SangiorgiWalker}. As our first example of calculation with the
machinery thus far presented we give the construction explicitly in
the {\rhoc}.

\begin{eqnarray}
	D_{x} & := & \prefix{x}{y}{(\binpar{\outputp{x}{y}}{@{y}})} \nonumber\\
	\bangp_{x}{P} & := & \binpar{{x}!\langle{\binpar{D_{x}}{P}}\rangle}{D_{x}} \nonumber
\end{eqnarray}

\begin{eqnarray}
	\bangp_{x}{P} & & \nonumber\\
	=
	& {x}!\langle{(\prefix{x}{y}{(\outputp{x}{y} | @{y})) | P}}\rangle 
	      | \prefix{x}{y}{(\outputp{x}{y} | @{y})} & \nonumber\\
	\red
	& (\outputp{x}{y} | @{y})\substn{\quotep{(\prefix{x}{y}{(@{y} | \outputp{x}{y})) | P}}}{y} & \nonumber\\
	=
	& \outputp{x}{\quotep{(\prefix{x}{y}{(\outputp{x}{y} | @{y})) | P}}}
	  | {(\prefix{x}{y}{(\outputp{x}{y} | @{y})) | P}} & \nonumber\\
	\red
	& \ldots & \nonumber\\
	\red^*
	& P | P | \ldots & \nonumber
\end{eqnarray}

Of course, this encoding, as an implementation, runs away, unfolding
$\bangp{P}$ eagerly. A lazier and more implementable replication
operator, restricted to input-guarded processes, may be obtained as follows.

\begin{eqnarray}
\bangp{\prefix{u}{v}{P}} 
	:= 
	\binpar{\lift{x}{\prefix{u}{v}{(\binpar{D(x)}{P})}}}{D(x)} \nonumber
\end{eqnarray}

\begin{remark}
  Note that the lazier definition still does not deal with summation
  or mixed summation (i.e. sums over input and output). The reader is
  invited to construct definitions of replication that deal with these
  features. 

  Further, the definitions are parameterized in a name, $x$. Can you,
  gentle reader, make a definition that eliminates this parameter and
  guarantees no accidental interaction between the replication
  machinery and the process being replicated -- i.e. no accidental
  sharing of names used by the process to get its work done and the
  name(s) used by the replication to effect copying. This latter
  revision of the definition of replication is crucial to obtaining
  the expected identity $!!P \sim !P$.
\end{remark}

\begin{remark}\label{rem:paradoxical_combinator}
  The reader familiar with the lambda calculus will have noticed the
  similarity between $D$ and the paradoxical combinator.

  [Ed. note: the existence of this seems to suggest we have to be more
  restrictive on the set of processes and names we admit if we are to
  support no-cloning.]
\end{remark}

\subsubsection{Bisimulation}

The computational dynamics gives rise to another kind of equivalence,
the equivalence of computational behavior. As previously mentioned
this is typically captured \emph{via} some form of bisimulation.

% The notion we use in this paper is weak barbed bisimulation
% \cite{milner91polyadicpi}.

The notion we use in this paper is derived from weak barbed
bisimulation \cite{milner91polyadicpi}. 

\begin{definition}
An \emph{observation relation}, $\downarrow_{\mathcal N}$, over a set
of names, $\mathcal N$, is the smallest relation satisfying the rules
below.

\infrule[Out-barb]{y \in {\mathcal N}, \; x \nameeq y}
		  {\outputp{x}{v} \downarrow_{\mathcal N} x}
\infrule[Par-barb]{\mbox{$P\downarrow_{\mathcal N} x$ or $Q\downarrow_{\mathcal N} x$}}
		  {\binpar{P}{Q} \downarrow_{\mathcal N} x}

We write $P \Downarrow_{\mathcal N} x$ if there is $Q$ such that 
$P \wred Q$ and $Q \downarrow_{\mathcal N} x$.
\end{definition}

\begin{definition}
%\label{def.bbisim}
An  ${\mathcal N}$-\emph{barbed bisimulation} over a set of names, ${\mathcal N}$, is a symmetric binary relation 
${\mathcal S}_{\mathcal N}$ between agents such that $P\rel{S}_{\mathcal N}Q$ implies:
\begin{enumerate}
\item If $P \red P'$ then $Q \wred Q'$ and $P'\rel{S}_{\mathcal N} Q'$.
\item If $P\downarrow_{\mathcal N} x$, then $Q\Downarrow_{\mathcal N} x$.
\end{enumerate}
$P$ is ${\mathcal N}$-barbed bisimilar to $Q$, written
$P \wbbisim_{\mathcal N} Q$, if $P \rel{S}_{\mathcal N} Q$ for some ${\mathcal N}$-barbed bisimulation ${\mathcal S}_{\mathcal N}$.
\end{definition}

$\mathcal{R} \subseteq \pi \times \pi$

$P \mathcal{R} Q => \forall P'. P \red P' \Rightarrow \exists Q'. Q \red Q', P' \mathcal{R} Q'$

$P \vdash x \Rightarrow Q \vdash x$

\begin{mathpar}
  \inferrule*[lab=Out-barb]{x \nameeq y}{{y}!\langle{Q}\rangle \vdash x}
  \and
  \inferrule*[lab=Par-barb]{\mbox{$P\vdash x$ or $Q\vdash x$}}{\binpar{P}{Q} \vdash x}
\end{mathpar}

\subsubsection{Contexts}

One of the principle advantages of computational calculi like the
$\pi$-calculus is a well-defined notion of context,
contextual-equivalence and a correlation between
contextual-equivalence and notions of bisimulation. The notion of
context allows the decomposition of a process into (sub-)process and
its syntactic environment, its context. Thus, a context may be
thought of as a process with a ``hole'' (written $\Box$) in it. The
application of a context $M$ to a process $P$, written $M[P]$, is
tantamount to filling the hole in $M$ with $P$. In this paper we do
not need the full weight of this theory, but do make use of the notion
of context in the proof the main theorem. 

\begin{mathpar}
  \inferrule* [lab=summation] {} {{M_{M},M_{N}} \bc \Box \;|\; x.M_{A} \;|\; M_{M}+M_{N}}
  \and
  \inferrule* [lab=agent] {} {{M_{A}} \bc (\vec{x})M_{P} \;| \; \clift{P_0,\ldots,M_{P},\ldots,P_N}}
  \and \\
  \inferrule* [lab=process] {} {{M_{P}} \bc M_{N} \;| \;P|M_{P} }
\end{mathpar} 

\begin{mathpar}
  \inferrule* [lab=sychronization] {} {M_{N} \bc \Box \;|\; x?M_{F} \;|\; x!M_{C}}
  \and
  \inferrule* [lab=abstraction] {} {{M_{F}} \bc (x)M_{P} }
  \and
  \inferrule* [lab=concretion] {} {{M_{C}} \bc \langle M_{P} \rangle }
  \and \\
  \inferrule* [lab=process] {} {{M_{P}} \bc M_{N} \;| \;P|M_{P} }
\end{mathpar}

\begin{definition}[contextual application] Given a context $M$, and
  process $P$, we define the \emph{contextual application}, $M[P] :=
  M\{P/\Box\}$. That is, the contextual application of M to P is the
  substitution of $P$ for $\Box$ in $M$.
\end{definition}

$\meaningof{-} : L \to \mathcal{P}(\pi)$

\begin{mathpar}
  \inferrule* [lab=collection] {} {\meaningof{true} = \pi, \and \meaningof{~E} = \pi \setminus \meaningof{E}, \and \meaningof{E_{1} \& E_{2}} = \meaningof{E_{1}} \cap \meaningof{E_{2}}}
\end{mathpar}

\begin{mathpar}
  \inferrule* [lab=structure] {} {\meaningof{0} = \{ P \in \pi | P \equiv 0 \}, \and \\ \meaningof{E_1 | E_2} = \{ P \in \pi | P \equiv P_{1} | P_{2}, P_{1} \in \meaningof{E_{1}}, P_{2} \in \meaningof{E_2}\} }
\end{mathpar}

\begin{mathpar}
 \inferrule* [lab=behavior] {} {\meaningof{\langle a?b \rangle E} = \{ P \in \pi | P \equiv Q | u?(y)P', \\ \and \\\\ \and \\ \;\;\; u \in \meaningof{a}, \forall z.P'\{z/y\} \in \meaningof{E\{z/b\}}\}, \and \\ \meaningof{a!E} = \{ P \in \pi | P \equiv Q | x!\langle P' \rangle, x \in \meaningof{a} P' \in \meaningof{E}\} }
\end{mathpar}

\begin{mathpar}
 \inferrule* [lab=nominal] {} {\meaningof{\quotep{E}} = \{ \quotep{P} \in \quotep{\pi} | P \in \meaningof{E} \}, \and \meaningof{\quotep{P}} = \{ \quotep{Q} \in \quotep{\pi} | P \equiv Q \} \and \\ \meaningof{@\quotep{E}} = \{ P \in \pi | P \equiv @x, x \in \meaningof{E} \}}
\end{mathpar}

\begin{eqnarray*}
  \\
  \meaningof{-} : TS \to ST
\end{eqnarray*}

\begin{eqnarray*}
  \\
  L : TS \to ST
\end{eqnarray*}

\begin{eqnarray*}
  \\
  P \models E \iff P \in \meaningof{E}
\end{eqnarray*}

\begin{eqnarray*}
  P \approx_{L} Q \iff \forall E \in L. P \models E \iff Q \models E
\end{eqnarray*}

\begin{eqnarray*}
  P \approx_{K} Q
\end{eqnarray*}

\begin{eqnarray*}
  P \approx Q
\end{eqnarray*}

$\approx_{K} = \approx = \approx_{L}$

\subsubsection{Contextual duality}

Note that contexts extend the quotation operation to a family of
operations from processes to names. Given a context, $M$, we can
define a \emph{nominal context}, $\quotep{M}$ by $\quotep{M}[P] :=
\quotep{M[P]}$. To foreshadow what is to come we observe that these
operations enjoy a duality with processes very much like the duality
between vectors and maps from vectors to scalars.

Further, because the calculus is essentially higher-order, we have a
correspondence between contexts and processes. More specifically,
given a name $x$ and a context $M$ we can construct $M^{*}_{x}$ such
that 

\begin{mathpar}
  M^{*}_{x} | \lift{x}{P} \red M[P]
\end{mathpar}

namely,

\begin{mathpar}
  M^{*}_{x} := x?(u).M[\dropn{u}]
\end{mathpar}

The dependence of $M^{*}_{x}$ on a name makes it an abstraction, 

\begin{mathpar}
  M^{*} := (x)x?(u).M[\dropn{u}]
\end{mathpar}

\subsection{Additional notation}

It will sometimes be convenient to denote the process a name
quotes. We already have the notation $x = \quotep{P}$, but it will be
convenient to introduce an alternate notation, $\procn{x}$, when we
want to emphasize the connection to the use of the name. Note that, by
virtue of name equivalence, $\quotep{\procn{x}} \nameeq x$; so, the
notation is consistent with previous definitions.

Further, because names have structure it is possible to effect
substitutions on the basis of that structure. This means we need to
upgrade our notation for substitutions, which we accomplish by
adapting comprehension notation. Thus,

\begin{mathpar}
  P\{ y / x : x \in S \}
\end{mathpar}

is interpreted to mean the process derived from P by replacing (in a
capture-avoiding manner) each occurrence of $x$ in $S$ by $y$. For example,

\begin{mathpar}
  P\{ \quotep{\procn{x}|\procn{x}} / x : x \in \freenames{P} \}
\end{mathpar}

will replace each (occurrence) of a free name $x$ in $P$ by
$\quotep{\procn{x}|\procn{x}}$.

Also, we will avail ourselves of the notation $x^{L}$ and $x^{R}$ to
denote injections of a name into disjoint copies of the name
space. There are numerous ways to accomplish this. One example can be
found in \cite{MeredithR05}. This notation overloads to vectors of
names: $\vec{x}^{\pi} := (x_{i}^{\pi} \; : \; 0 \leq i < |\vec{x}| )$ where $\pi \in \{L,R\}$.

We also use $P^{\Box} := P|\Box$.

In \cite{MeredithR05} an interpretation of the new operator is
given. It turns out that there are several possible interpretations
all enjoying the requisite algebraic properties of the operator (see
\cite{milner91polyadicpi}). We will therefore make liberal use of
$(\nu\; \vec{x})P$.

% subsection the_syntax_and_semantics_of_the_notation_system (end)   

\input{qm2pi.qmops} 

\input{qm2pi.sterngerlach} 

\input{qm2pi.metric} 

% section concurrent_process_calculi (end)

%\input{qm2pi.proofsketch}

% section proof sketch (end)

%\input{qm2pi.slviaknots} 

% section spatial logic via knots (end)

\input{qm2pi.conclusion}

% section conclusion (end)

%\input{qm2pi.dtcodes} 

% section wiring algorithm (end)

\input{qm2pi.ack} 

% section acknowledgments (end)

\newpage


\bibliographystyle{plain}   
\bibliography{../../biblios/main.bib}

\input{qm2pi.rhodetails}

\end{document}



\end{document}



% section proof sketch (end)

%\section{Unlikely characters: spatial logic for
  knots}\label{sub:characteristic_formulae} % (fold)

Associated to the mobile process calculi are a family of logics known
as the Hennessy-Milner logics. These logics typically enjoy a
semantics interpreting formulae as sets of processes that when
factored through the encoding outlined above allows an identification
of classes of knots with logical formulae. In the context of this
encoding the sub-family known as the spatial logics \cite{CairesC03}
\cite{CairesC04} \cite{Caires04} are of particular interest providing
several important features for expressing and reasoning about
properties (i.e. classes) of knots. We hint here at how this may be done.

%\begin{description}
%\item [structural connectives] 
\subsubsection{Structural connectives} The spatial logics enjoy
structural connectives corresponding, at the logical level, to the
parallel composition ($P | Q$) and new name ($(\nu \; x)P$)
connectives for processes. As illustrated in the examples below, these
connectives are extremely expressive given the shape of our encoding.
%\item [decideable satisfaction]

\subsubsection{Decideable satisfaction}
In \cite{Caires04} the satisfaction relation is shown to be decideable
for a rich class of processes. It further turns out that the image of
the our encoding is a proper subset of that class. This result
provides the basis for an algorithm by which to search for knots
enjoying a given property.
%\item [characteristic formulae]

\subsubsection{Characteristic formulae}
In the same paper \cite{Caires04} , Caires presents a means of calculating
characteristic formulae, selecting equivalence classes of processes
up to a pre--specified depth limit on the support set of names. Composed with our
encoding, this characteristic formula can be used to select
characteristic formulae for knots.
%\end{description}

\subsubsection{Spatial logic formulae}

The grammar below (segmented for comprehension) summarizes the syntax
of spatial logic formulae. We employ illustrative examples in the
sequel to provide an intuitive understanding of their meaning
referring the reader to \cite{Caires04} for a more detailed explication
of the semantics.

\begin{mathpar}
  \inferrule* [lab=boolean] {} {{A,B} \bc T \;|\; \neg A \;|\; A \wedge B \;|\; \eta = \eta'}
  \and
  \inferrule* [lab=spatial] {} {|\; \pzero \;|\; A | B \;|\; x \text{\textregistered} A \;|\; \forall x . A \;|\;  H x . A}
  \and
  \inferrule* [lab=behavioral] {} {|\; \alpha . A}
  \and 
  \inferrule* [lab=recursion] {} {|\; X(\vec{u}) \;|\; \mu X(\vec{u}) . A}
  \and
  \inferrule* [lab=action] {} {\alpha \bc \langle x?(\vec{y}) \rangle \;|\; \langle x!(\vec{y}) \rangle \;|\; \langle \tau \rangle}
  \and 
  \inferrule* [lab=name] {} {\eta \bc x \;|\; \tau}
\end{mathpar} 

% subsection characteristic_formulae (end)   	 

\subsection{Example formulae}\label{sub:example_formulae_} % (fold)

\subsubsection{Crossing as formula.}
% 
% \begin{align*}
%   \frac{d}{dx} \sin x &= \cos x 
%   & \frac{d}{dx} e^x &= e^x \\
%   \frac{d}{dx} \cos x &= - \sin x 
%   & \frac{d}{dx} \log x &= \frac{1}{x} \\
% \end{align*} 

\begin{align*}
 \mu C(x_{0},x_{1},y_{0},y_{1},u).&(\langle x_{0}?(z) \rangle(\langle u! \rangle\langle y_{1}!z \rangle C(x_{0},x_{1},y_{0},y_{1},u)) & \\
  & \wedge \langle y_{1}?(z) \rangle (\langle u! \rangle \langle x_{0}!z \rangle C(x_{0},x_{1},y_{0},y_{1},u)) & \\
  & \wedge \langle x_{1}?(z) \rangle (\langle u? \rangle \langle y_{0}!z \rangle C(x_{0},x_{1},y_{0},y_{1},u)) & \\
  & \wedge \langle y_{0}?(z) \rangle (\langle u? \rangle \langle x_{1}!z \rangle C(x_{0},x_{1},y_{0},y_{1},u))) &
\end{align*}

The lexicographical similarity between the shape of this formulae and
the shape of definition of the process representing a crossing reveals
the intuitive meaning of this formulae. It describes the capabilities
of a process that has the right to represent a crossing. For example
it picks out processes that may perform an input on the port $x_0$ in
its initial menu of capabilities. What differentiates the formula
from the process, however, is that the crossing process is the
smallest candidate to satisfy the formula. Infinitely many other
processes -- with internal behavior hidden behind this interface, so
to speak -- also satisfy this formula. Even this simple formula,
then, can be seen to open a new view onto knots, providing a
computational interpretation of \emph{virtual} knots.

Note that this formula is derived by hand. A similar formula can be
derived by employing Caires' calculation of characteristic formula
\cite{Caires04} to the process representing a crossing. In light of
this discussion, we let
$\meaningof{C}_{\phi}(x0,x1,y0,y1,u)$ denote a formula specifying the
dynamics we wish to capture of a crossing. To guarantee we preserve
the shape of the interface and minimal semantics we demand that
$\meaningof{C}_{\phi}(x0,x1,y0,y1,u) \Rightarrow
\textbf{C}(x0,x1,y0,y1,u)$ where $\textbf{C}(x0,x1,y0,y1,u)$ denotes
the formula above.
                            
\subsubsection{Crossing number constraints.}
The moral content of the context lemma (Lemma \ref{context}) is that the notion of
``locality'' in the Reidemeister moves is effectively captured by the
parallel composition operator of the process calculus. This intuition
extends through the logic. Given a formula,
$\meaningof{C}_{\phi}(x0,x1,y0,y1,u)$, we can use the structural
connectives to specify constraints on crossing numbers, such as at
least $n$ crossings, or exactly $n$ crossings.
\begin{mathpar}
  \inferrule* [lab=at-least-n] {} { K^{\geq n}_{\phi}(\vec{xs},\vec{ys}) := \Pi_{i=0}^{n-1} Hu . \meaningof{C}_{\phi}(xs_i,ys_i,u) | T }
  \and 
  \inferrule* [lab=exactly-n] {} { K^{= n}_{\phi}(\vec{xs},\vec{ys}) := \Pi_{i=0}^{n-1} Hu . \meaningof{C}_{\phi}(xs_i,ys_i,u) | \neg (\forall x_0,y_0,x_1,y_1,u . \meaningof{C}_{\phi}(x_0,y_0,x_1,y_1,u) | T) }
\end{mathpar}

To round out this section, recall that the encoding of an $n$-crossing
knot decomposes into a parallel composition of $n$ \emph{copies} of a
crossing process together with a wiring harness. To specify different
knot classes with the same crossing number amounts to specifying
logical constraints on the wiring harness. In the interest of space,
we defer examples to a forthcoming paper. Suffice it to say that both
the conditions ``alternating knot'' and ``contains the tangle
corresponding to 5/3'' are expressible. For example, it is possible to
calculate the characteristic formula of a process corresponding to the
tangle 5/3 and conjoin it into the classifying formula via the
composition connective of the logic.

Finally, we wish to observe that it is entirely within reason to
contemplate a more domain-specific version of spatial logic tailored
to the shape of processes in the image of the encoding. Such a
domain-specific logic would have a better claim to the title formal
language of knot properties.

% subsection example_formulae_ (end)

% section knots_as_processes (end) 

% section spatial logic via knots (end)

\section{Conclusions and future work}

\paragraph{Testing physical space}
You, gentle reader, may wonder why of all the theorems to be proved
given this set up we pick the one above. In some sense it's hardly
central to quantum mechanics. We see it as central in the sense that
it firmly establishes a notion of physical space arising from a notion
of the equivalence of behavior. Relating bisimulation to a metric is a
big step forward, but one is faced with interpreting the relationship
of that metric space to something more physical. Quantum mechanical
notions of ``physical'' space are still far from intuitive, but by
relating this idea of distance as testing to calculations that predict
physical circumstances we are making a not insignificant step forward
toward an understanding of the physical space we inhabit as
essentially dynamic.

\paragraph{Effectivity and simulation}
One of the observations we have yet to make is that the entire program
spelled out here is effective. We have built various interpreters for
the reflective calculus at work in this interpretation. In principle,
then, we can simulate quantum mechanics on a computer. The place where
the simulation may lose fidelity is the infinitely branching summation
for the annihilator.

In this connection i also want to point out that the evaluation style
calculation of the inner product puts the non-determinism of the
summation right at the heart of measurement. This suggests that
Milner's original reduction-based formulation of the dynamics of his
calculi in terms of sums was not just notationally suggestive of a
notion of measure-and-continue but captured some significant part of
the physics.

\paragraph{Quantum continuations}
In light of this last observation i want to point out that the
predominant account of quantum mechanics is missing a key aspect of a
truly compositional story of the physical situation. In a real lab,
when a measurement is made the observation can be made to feed into
another device that then makes another measurement conditioned on the
results of the first. This means that after the superposition was
collapsed the entire experimental set up remained in
superposition. While QM offers a means of writing this down it doesn't
quite line up well with the well-trodden formulation of computation
and continuation that we see so succinctly expressed in Milner's
calculi. This suggests that there might be advantages to this account
of dynamics waiting to be explored.

\paragraph{Quantum logic}
In this connection, we also note that by virtue of having the
Hennessy-Milner construction, we can pull the construction through the
interpretation of QM. This gives us a natural candidate for a quantum
logic that enjoys an extremely tight connection with it's domain of
interpretation, making the construction much less ad hoc (rather it is
the image of functor!).

\paragraph{Quantum probabiity}
i have questions about the basis of the interpretation of inner
product as probability amplitude. In particular, using which
axiomatization of probability theory does the notion of probability
amplitude earn the right to be so dubbed? In other words, where is the
proof that the operation for calculating a probability amplitude (and
then squaring) satisfies the axioms of what it means to calculate a
probability? Even if such a proof exists (i have yet to find it in the
literature), i wonder if it might not be possible to turn things on
their heads. Can we view the calculation of the probability amplitude
as an axiomatization of probability? If so, then the definition we
give for calculating probability amplitude may provide the basis for
an \emph{effective} theory of probability.

\paragraph{Quantum vs ``biological'' information}
Finally, i want to conclude with a more philosophical observation. At
a recent workshop in which QM was a predominant topic i noticed
something about quantum information. The speaker was giving a riveting
discussion of axiomatic QM and showing how properties of ``no
cloning'' and ``no deleting'' emerged as consequences of the
axiomatization. Theorems of this form are necessary to give us a sense
of confidence that our axioms characterize the physical theory. What
struck me, though, was that if quantum information is neither erasable
nor replicable it is markedly different from \emph{life}. Two of the
things we know about life is that

\begin{itemize}
  \item it ends;
  \item to gain some measure of persistence, to transcend it's
    finitude it is imminently copyable.
\end{itemize}

Both of these qualities are summarized succinctly in the aphorism: all
flesh is grass. For me these two kinds of ``information'' -- call them
quantum and biological -- are end points on a spectrum of strategies
for persistence. At one end, we have those curious entities that enjoy
uniqueness and permanence; at the other, we have those who in the face
of a certain end and an uncertain present make a go of passing
something on. To me one of the more remarkable aspects of the latter
strategy is that in the presence of noise (and certain features of
copying) we get a kind of dynamism, a chance for improvement against a
given persistent condition.

% subsection other_calculi_other_bisimulations_and_geometry_as_behavior (end)




% section conclusion (end)

%\documentclass[12pt]{llncs}
%\documentclass{jktr}

\usepackage[pdftex]{hyperref}                   
\usepackage {listings}
\usepackage {mathpartir}
\usepackage{bcprules}
%\usepackage{listings}
                       
\usepackage{graphicx} 
%\usepackage[margins=2.5cm,nohead,nofoot]{geometry}
%\usepackage{geometry}
\usepackage{amsfonts}
\usepackage{amstext}
\usepackage{latexsym}
\usepackage{amssymb}
\usepackage{color}


%\include{myPreamble}
\documentclass[12pt]{llncs}
%\documentclass{jktr}

\usepackage[pdftex]{hyperref}                   
\usepackage {listings}
\usepackage {mathpartir}
\usepackage{bcprules}
%\usepackage{listings}
                       
\usepackage{graphicx} 
%\usepackage[margins=2.5cm,nohead,nofoot]{geometry}
%\usepackage{geometry}
\usepackage{amsfonts}
\usepackage{amstext}
\usepackage{latexsym}
\usepackage{amssymb}
\usepackage{color}


%\include{myPreamble}
\include{qm2pi.local} 

%\ifpdf
%\usepackage[pdftex]{graphicx}
%\else
%\usepackage{graphicx}
%\fi

 % \ifpdf
%  \usepackage{pdfsync}
%  \if


%\title{Brief Article}
%\author{David F. Snyder}
%\author{L.G. Meredith}

%\address{Dept. of Math., Texas State University--San Marcos, San Marcos, TX 78666}
       
\pagestyle{empty}


\begin{document}

\lstset{language=[Objective]Caml,frame=shadowbox}

\input{qm2pi.front}

% section front matter (end)

\input{qm2pi.intro} 
 
% section introduction (end)

% \input{qm2pi.knotations} 

% section notation (end)

\input{qm2pi.process.calculi} 

% section concurrent_process_calculi_and_spatial_logics_ (end)
    
%\input{qm2pi.knots2pi} 

%\input{qm2pi.trefoil} 

%\input{qm2pi.mainthm} 

% subsection basic_interpretation (end)

%\input{qm2pi.rho.presentation} 
\subsection{The syntax and semantics of the notation system}\label{sub:the_syntax_and_semantics_of_the_notation_system} % (fold)

We now summarize a technical presentation of the calculus that
embodies our theory of dynamics. The typical presentation of such a
calculus follows the style of giving generators and relations on
them. The grammar, below, describing term constructors, freely
generates the set of processes, $\Proc$. This set is then quotiented
by a relation known as structural congruence and it is over this set
that the notion of dynamics is expressed. This presentation is
essentially that of \cite{MeredithR05} with the addition of
polyadicity and summation. For readability we have relegated some of
the technical subtleties to an appendix.

\subsubsection{Process grammar}\label{subsub:process_grammar}

\begin{mathpar}
  \inferrule* [lab=synchronization] {} {{M} \bc \pzero \;|\; x?F \;|\; x!C }
  \and
  \inferrule* [lab=abstraction] {} {{F} \bc (x)P}
  \and
  \inferrule* [lab=concretion] {} {{C} \bc \langle Q \rangle}
  \and
  \inferrule* [lab=process] {} {{P,Q} \bc M \;| \;P|Q \;|\; @{x}}
  \and
  \inferrule* [lab=name] {} {{x} \bc \quotep{P}}
\end{mathpar} 

Note that $\vec{x}$ (resp. $\vec{P}$) denotes a vector of names
(resp. processes) of length $|\vec{x}|$ (resp. $|\vec{P}|$). We adopt
the following useful abbreviations.

\begin{mathpar}
   x?(\vec{y}).P := x.(\vec{y})P \and  x\clift{\vec{P}} := x.\clift{\vec{P}}
   \and x!(y) := \lift{x}{\dropn{y}}
   \and \Pi_{i=0}^{n-1}P_i := P_0 | \ldots | P_{n-1}
\end{mathpar}

\subsubsection{Structural congruence}

\paragraph{Free and bound names and alpha-equivalence.} At the
core of structural equivalence is alpha-equivalence which identifies
process that are the same up to a change of variable. Formally, we
recognize the distinction between free and bound names. The free names
of a process, $\freenames{P}$, may be calculated recursively as
follows:

\begin{mathpar}
\freenames{\pzero} := \emptyset
  \and \\
  \freenames{x?(y).P} := \{ x \} \cup (\freenames{P} \setminus \{ y \})
  \and 
  \freenames{x!\langle P \rangle} := \{ x \} \cup \{ P \} 
  \and \\
  \freenames{P|Q} := \freenames{P} \cup \freenames{Q}
  \and \\
  \freenames{@{x}} := \{ x \}
\end{mathpar}

$\pi$
$\quotep{\pi}$

$\freenames{-} : \pi \to \mathcal{P}(\quotep{\pi})$

\begin{eqnarray*}
  \freenames{\pzero} & := & \emptyset \\
  \freenames{x?(y).P} & := & \{ x \} \cup (\freenames{P} \setminus \{ y \}) \\
  \freenames{x!\langle P \rangle} & := & \{ x \} \cup \{ P \} \\
  \freenames{P|Q} & := & \freenames{P} \cup \freenames{Q} \\
  \freenames{\dropn{x}} & := & \{ x \}
\end{eqnarray*}

The bound names of a process, $\boundnames{P}$, are those names occurring in $P$
that are not free. For example, in $x?(y).0$, the name $x$ is free, while $y$ is bound.

\begin{mathpar}
  \inferrule* [lab=monoidal-laws] {} { P|Q \equiv Q|P \and P|0 \equiv P \and P|(Q|R) \equiv (P|Q)|R }
\end{mathpar}

\begin{mathpar}
  \inferrule* [lab=alpha-equivalence] {} { (x)P \equiv (y)P\{y/x\} \and y \not\in \freenames{P} }
\end{mathpar}

\begin{definition}
Then two processes, $P,Q$, are alpha-equivalent if $P = Q\{\vec{y}/\vec{x}\}$ for
some $\vec{x} \in \boundnames{Q},\vec{y} \in \boundnames{P}$, where $Q\{\vec{y}/\vec{x}\}$
denotes the capture-avoiding substitution of $\vec{y}$ for $\vec{x}$ in $Q$.
\end{definition}

\begin{definition}
  The {\em structural congruence} \cite{SangiorgiWalker} , $\equiv$,
  between processes is the least congruence containing
  alpha-equivalence, satisfying the abelian monoid laws
  (associativity, commutativity and $\pzero$ as identity) for parallel
  composition $|$ and for summation $+$.
\end{definition}

\subsection{Name equivalence}

We take name equivalence, written $\nameeq$, to be the smallest
equivalence relation generated by the following rules.

\begin{mathpar}
\inferrule*[lab=Quote-drop]
{ }
{ \quotep{@{x}} \nameeq x }

\inferrule*[lab=Struct-equiv]
{ P \scong Q }
{ \quotep{P} \nameeq \quotep{Q} }
\end{mathpar}

The astute reader will have noticed that the mutual recursion of names
and processes imposes a mutual recursion on alpha-equivalence and
structural equivalence via name-equivalence. Fortunately, all of this
works out pleasantly and we may calculate in the natural way, free of
concern. The reader interested in the details is referred to the
appendix \ref{appendix:rho_details}.

\subsection{Substitution}

We use $\Proc$ for the set of processes, $\QProc$ for the set of
names, and $\id{\{}\vec{y} / \vec{x} \id{\}}$ to denote partial maps,
$s : \QProc \rightarrow \QProc$. A map, $s$ lifts, uniquely, to a map
on process terms, $\widehat{s} : \Proc \rightarrow \Proc$ by the
following equations.

\begin{mathpar}
  (0) \psubstp{Q}{P} := 0 \\
  (R \juxtap S) \psubstp{Q}{P}
  :=    
  (R)\psubstp{Q}{P} \juxtap (S) \psubstp{Q}{P} \\
  (x?(y).R) \psubstp{Q}{P}    
  :=    
  (x)\substp{Q}{P} (z)\concat( (R \psubstn{z}{y}) \psubstp{Q}{P} ) \\
  (\lift{x}{R}) \psubstp{Q}{P}  
  :=
  \lift{(x)\substp{Q}{P}}{ R \psubstp{Q}{P} } \\
%   (\dropn{x})  \psubstp{Q}{P}       
%   := 
%   \left\{ 
%     \begin{array}{ccc} 
%       \dropn{\quotep{Q}} & & x \nameeq \quotep{P} \\
%       \dropn{x} & & otherwise \\
%     \end{array}
%   \right. 
  (\dropn{x})  \psubstp{Q}{P}       
  := 
  \left\{ 
    \begin{array}{ccc} 
      Q & & x \nameeq \quotep{P} \\
      \dropn{x} & & otherwise \\
    \end{array}
  \right.
\end{mathpar}
 

where

\begin{eqnarray}
  (x)\id{\{} \lpquote Q \rpquote / \lpquote P \rpquote \id{\}}            = 
  \left\{ 
    \begin{array}{ccc}
      \lpquote Q \rpquote & & x \nameeq \lpquote P \rpquote \\
      x & & otherwise \\
    \end{array}
  \right. \nonumber
\end{eqnarray}

and $z$ is chosen distinct from $\quotep{P}$, $\quotep{Q}$, the free
names in $Q$, and all the names in $R$. Our $\alpha$-equivalence will
be built in the standard way from this substitution.

\begin{remark}\label{rem:no_self_referential_names}
  One consequence of these definitions is that $\forall P. \quotep{P}
  \not\in \freenames{P}$.
\end{remark}

\subsection{ Dynamic quote: an example }

Anticipating something of what's to come, consider applying the
substitution, $\widehat{\id{\{}u / z \id{\}}}$, to the following pair
of processes, $\lift{w}{y!(z)}$ and $w[ \lpquote y!(z) \rpquote ]$.

\begin{eqnarray}
	\lift{w}{y!(z)}\widehat{\id{\{}u / z \id{\}}}
		& = &
		\lift{w}{y!(u)} \nonumber\\
	w[ \lpquote y!(z) \rpquote ] \widehat{ \id{\{}u / z \id{\}} }
		& = &
		w[ \lpquote y!(z) \rpquote ] \nonumber
\end{eqnarray}

Because the body of the process between quotes is impervious to
substitution, we get radically different answers. In fact, by
examining the first process in an input context,
e.g. $x?(z).\lift{w}{y!(z)}$, we see that the process under the lift
operator may be shaped by prefixed inputs binding a name inside it. In
this sense, the lift operator will be seen as a way to dynamically
construct processes before reifying them as names.

Finally equipped with these standard features we can present the
dynamics of the calculus.

\subsubsection{Operational semantics} 

Finally, we introduce the computational dynamics. What marks these
algebras as distinct from other more traditionally studied algebraic
structures, e.g. vector spaces or polynomial rings, is the manner in
which dynamics is captured. In traditional structures, dynamics is typically
expressed through morphisms between such structures, as in linear maps
between vector spaces or morphisms between rings. In algebras
associated with the semantics of computation, the dynamics is
expressed as part of the algebraic structure itself, through a
reduction reduction relation typically denoted by $\red$. Below, we
give a recursive presentation of this relation for the calculus used
in the encoding.

$\red \subseteq \pi \times \pi$
$\red : \pi \to \mathcal{P}(\pi)$

\begin{mathpar}
  \inferrule* [lab=Comm] { \textsf{match}( x_{src}, x_{trgt} ) } { x_{trgt}?(y)P \; | \; x_{src}!\langle {Q} \rangle \red P\{\quotep{Q}/y}\} }
  \and \\
  \inferrule* [lab=Par] {{P} \red {P}'} {{{P} | {Q}} \red {{P}' | {Q}}}
  \and
  \inferrule* [lab=Equiv]{{{P} \scong {P}'} \andalso {{P}' \red {Q}'} \andalso {{Q}' \scong {Q}}}{{P} \red {Q}}
\end{mathpar}

\begin{eqnarray*}
  match_{\equiv} (\quotep{P},\quotep{Q}) & := & P \equiv Q \\
  match_{\dagger}(\quotep{P},\quotep{Q}) & := & \forall R. P|Q \red^{*} R => R \red^{*} 0 \\
  match_{K}(\quotep{P},\quotep{Q}) & := & K \mbox{ for some context } K
\end{eqnarray*}

$u?(x)P | u!\langle Q \rangle \red P\{\quotep{Q}/x\}$

%We write $\wred$ for $\red^*$, and $P\red$ if $\exists Q $ such that $ P \red Q$.
We write $P\red$ if $\exists Q $ such that $ P \red Q$ and $P\not\red$, otherwise.

\section{Replication}

As mentioned before, it is known that replication (and hence
recursion) can be implemented in a higher-order process algebra
\cite{SangiorgiWalker}. As our first example of calculation with the
machinery thus far presented we give the construction explicitly in
the {\rhoc}.

\begin{eqnarray}
	D_{x} & := & \prefix{x}{y}{(\binpar{\outputp{x}{y}}{@{y}})} \nonumber\\
	\bangp_{x}{P} & := & \binpar{{x}!\langle{\binpar{D_{x}}{P}}\rangle}{D_{x}} \nonumber
\end{eqnarray}

\begin{eqnarray}
	\bangp_{x}{P} & & \nonumber\\
	=
	& {x}!\langle{(\prefix{x}{y}{(\outputp{x}{y} | @{y})) | P}}\rangle 
	      | \prefix{x}{y}{(\outputp{x}{y} | @{y})} & \nonumber\\
	\red
	& (\outputp{x}{y} | @{y})\substn{\quotep{(\prefix{x}{y}{(@{y} | \outputp{x}{y})) | P}}}{y} & \nonumber\\
	=
	& \outputp{x}{\quotep{(\prefix{x}{y}{(\outputp{x}{y} | @{y})) | P}}}
	  | {(\prefix{x}{y}{(\outputp{x}{y} | @{y})) | P}} & \nonumber\\
	\red
	& \ldots & \nonumber\\
	\red^*
	& P | P | \ldots & \nonumber
\end{eqnarray}

Of course, this encoding, as an implementation, runs away, unfolding
$\bangp{P}$ eagerly. A lazier and more implementable replication
operator, restricted to input-guarded processes, may be obtained as follows.

\begin{eqnarray}
\bangp{\prefix{u}{v}{P}} 
	:= 
	\binpar{\lift{x}{\prefix{u}{v}{(\binpar{D(x)}{P})}}}{D(x)} \nonumber
\end{eqnarray}

\begin{remark}
  Note that the lazier definition still does not deal with summation
  or mixed summation (i.e. sums over input and output). The reader is
  invited to construct definitions of replication that deal with these
  features. 

  Further, the definitions are parameterized in a name, $x$. Can you,
  gentle reader, make a definition that eliminates this parameter and
  guarantees no accidental interaction between the replication
  machinery and the process being replicated -- i.e. no accidental
  sharing of names used by the process to get its work done and the
  name(s) used by the replication to effect copying. This latter
  revision of the definition of replication is crucial to obtaining
  the expected identity $!!P \sim !P$.
\end{remark}

\begin{remark}\label{rem:paradoxical_combinator}
  The reader familiar with the lambda calculus will have noticed the
  similarity between $D$ and the paradoxical combinator.

  [Ed. note: the existence of this seems to suggest we have to be more
  restrictive on the set of processes and names we admit if we are to
  support no-cloning.]
\end{remark}

\subsubsection{Bisimulation}

The computational dynamics gives rise to another kind of equivalence,
the equivalence of computational behavior. As previously mentioned
this is typically captured \emph{via} some form of bisimulation.

% The notion we use in this paper is weak barbed bisimulation
% \cite{milner91polyadicpi}.

The notion we use in this paper is derived from weak barbed
bisimulation \cite{milner91polyadicpi}. 

\begin{definition}
An \emph{observation relation}, $\downarrow_{\mathcal N}$, over a set
of names, $\mathcal N$, is the smallest relation satisfying the rules
below.

\infrule[Out-barb]{y \in {\mathcal N}, \; x \nameeq y}
		  {\outputp{x}{v} \downarrow_{\mathcal N} x}
\infrule[Par-barb]{\mbox{$P\downarrow_{\mathcal N} x$ or $Q\downarrow_{\mathcal N} x$}}
		  {\binpar{P}{Q} \downarrow_{\mathcal N} x}

We write $P \Downarrow_{\mathcal N} x$ if there is $Q$ such that 
$P \wred Q$ and $Q \downarrow_{\mathcal N} x$.
\end{definition}

\begin{definition}
%\label{def.bbisim}
An  ${\mathcal N}$-\emph{barbed bisimulation} over a set of names, ${\mathcal N}$, is a symmetric binary relation 
${\mathcal S}_{\mathcal N}$ between agents such that $P\rel{S}_{\mathcal N}Q$ implies:
\begin{enumerate}
\item If $P \red P'$ then $Q \wred Q'$ and $P'\rel{S}_{\mathcal N} Q'$.
\item If $P\downarrow_{\mathcal N} x$, then $Q\Downarrow_{\mathcal N} x$.
\end{enumerate}
$P$ is ${\mathcal N}$-barbed bisimilar to $Q$, written
$P \wbbisim_{\mathcal N} Q$, if $P \rel{S}_{\mathcal N} Q$ for some ${\mathcal N}$-barbed bisimulation ${\mathcal S}_{\mathcal N}$.
\end{definition}

$\mathcal{R} \subseteq \pi \times \pi$

$P \mathcal{R} Q => \forall P'. P \red P' \Rightarrow \exists Q'. Q \red Q', P' \mathcal{R} Q'$

$P \vdash x \Rightarrow Q \vdash x$

\begin{mathpar}
  \inferrule*[lab=Out-barb]{x \nameeq y}{{y}!\langle{Q}\rangle \vdash x}
  \and
  \inferrule*[lab=Par-barb]{\mbox{$P\vdash x$ or $Q\vdash x$}}{\binpar{P}{Q} \vdash x}
\end{mathpar}

\subsubsection{Contexts}

One of the principle advantages of computational calculi like the
$\pi$-calculus is a well-defined notion of context,
contextual-equivalence and a correlation between
contextual-equivalence and notions of bisimulation. The notion of
context allows the decomposition of a process into (sub-)process and
its syntactic environment, its context. Thus, a context may be
thought of as a process with a ``hole'' (written $\Box$) in it. The
application of a context $M$ to a process $P$, written $M[P]$, is
tantamount to filling the hole in $M$ with $P$. In this paper we do
not need the full weight of this theory, but do make use of the notion
of context in the proof the main theorem. 

\begin{mathpar}
  \inferrule* [lab=summation] {} {{M_{M},M_{N}} \bc \Box \;|\; x.M_{A} \;|\; M_{M}+M_{N}}
  \and
  \inferrule* [lab=agent] {} {{M_{A}} \bc (\vec{x})M_{P} \;| \; \clift{P_0,\ldots,M_{P},\ldots,P_N}}
  \and \\
  \inferrule* [lab=process] {} {{M_{P}} \bc M_{N} \;| \;P|M_{P} }
\end{mathpar} 

\begin{mathpar}
  \inferrule* [lab=sychronization] {} {M_{N} \bc \Box \;|\; x?M_{F} \;|\; x!M_{C}}
  \and
  \inferrule* [lab=abstraction] {} {{M_{F}} \bc (x)M_{P} }
  \and
  \inferrule* [lab=concretion] {} {{M_{C}} \bc \langle M_{P} \rangle }
  \and \\
  \inferrule* [lab=process] {} {{M_{P}} \bc M_{N} \;| \;P|M_{P} }
\end{mathpar}

\begin{definition}[contextual application] Given a context $M$, and
  process $P$, we define the \emph{contextual application}, $M[P] :=
  M\{P/\Box\}$. That is, the contextual application of M to P is the
  substitution of $P$ for $\Box$ in $M$.
\end{definition}

$\meaningof{-} : L \to \mathcal{P}(\pi)$

\begin{mathpar}
  \inferrule* [lab=collection] {} {\meaningof{true} = \pi, \and \meaningof{~E} = \pi \setminus \meaningof{E}, \and \meaningof{E_{1} \& E_{2}} = \meaningof{E_{1}} \cap \meaningof{E_{2}}}
\end{mathpar}

\begin{mathpar}
  \inferrule* [lab=structure] {} {\meaningof{0} = \{ P \in \pi | P \equiv 0 \}, \and \\ \meaningof{E_1 | E_2} = \{ P \in \pi | P \equiv P_{1} | P_{2}, P_{1} \in \meaningof{E_{1}}, P_{2} \in \meaningof{E_2}\} }
\end{mathpar}

\begin{mathpar}
 \inferrule* [lab=behavior] {} {\meaningof{\langle a?b \rangle E} = \{ P \in \pi | P \equiv Q | u?(y)P', \\ \and \\\\ \and \\ \;\;\; u \in \meaningof{a}, \forall z.P'\{z/y\} \in \meaningof{E\{z/b\}}\}, \and \\ \meaningof{a!E} = \{ P \in \pi | P \equiv Q | x!\langle P' \rangle, x \in \meaningof{a} P' \in \meaningof{E}\} }
\end{mathpar}

\begin{mathpar}
 \inferrule* [lab=nominal] {} {\meaningof{\quotep{E}} = \{ \quotep{P} \in \quotep{\pi} | P \in \meaningof{E} \}, \and \meaningof{\quotep{P}} = \{ \quotep{Q} \in \quotep{\pi} | P \equiv Q \} \and \\ \meaningof{@\quotep{E}} = \{ P \in \pi | P \equiv @x, x \in \meaningof{E} \}}
\end{mathpar}

\begin{eqnarray*}
  \\
  \meaningof{-} : TS \to ST
\end{eqnarray*}

\begin{eqnarray*}
  \\
  L : TS \to ST
\end{eqnarray*}

\begin{eqnarray*}
  \\
  P \models E \iff P \in \meaningof{E}
\end{eqnarray*}

\begin{eqnarray*}
  P \approx_{L} Q \iff \forall E \in L. P \models E \iff Q \models E
\end{eqnarray*}

\begin{eqnarray*}
  P \approx_{K} Q
\end{eqnarray*}

\begin{eqnarray*}
  P \approx Q
\end{eqnarray*}

$\approx_{K} = \approx = \approx_{L}$

\subsubsection{Contextual duality}

Note that contexts extend the quotation operation to a family of
operations from processes to names. Given a context, $M$, we can
define a \emph{nominal context}, $\quotep{M}$ by $\quotep{M}[P] :=
\quotep{M[P]}$. To foreshadow what is to come we observe that these
operations enjoy a duality with processes very much like the duality
between vectors and maps from vectors to scalars.

Further, because the calculus is essentially higher-order, we have a
correspondence between contexts and processes. More specifically,
given a name $x$ and a context $M$ we can construct $M^{*}_{x}$ such
that 

\begin{mathpar}
  M^{*}_{x} | \lift{x}{P} \red M[P]
\end{mathpar}

namely,

\begin{mathpar}
  M^{*}_{x} := x?(u).M[\dropn{u}]
\end{mathpar}

The dependence of $M^{*}_{x}$ on a name makes it an abstraction, 

\begin{mathpar}
  M^{*} := (x)x?(u).M[\dropn{u}]
\end{mathpar}

\subsection{Additional notation}

It will sometimes be convenient to denote the process a name
quotes. We already have the notation $x = \quotep{P}$, but it will be
convenient to introduce an alternate notation, $\procn{x}$, when we
want to emphasize the connection to the use of the name. Note that, by
virtue of name equivalence, $\quotep{\procn{x}} \nameeq x$; so, the
notation is consistent with previous definitions.

Further, because names have structure it is possible to effect
substitutions on the basis of that structure. This means we need to
upgrade our notation for substitutions, which we accomplish by
adapting comprehension notation. Thus,

\begin{mathpar}
  P\{ y / x : x \in S \}
\end{mathpar}

is interpreted to mean the process derived from P by replacing (in a
capture-avoiding manner) each occurrence of $x$ in $S$ by $y$. For example,

\begin{mathpar}
  P\{ \quotep{\procn{x}|\procn{x}} / x : x \in \freenames{P} \}
\end{mathpar}

will replace each (occurrence) of a free name $x$ in $P$ by
$\quotep{\procn{x}|\procn{x}}$.

Also, we will avail ourselves of the notation $x^{L}$ and $x^{R}$ to
denote injections of a name into disjoint copies of the name
space. There are numerous ways to accomplish this. One example can be
found in \cite{MeredithR05}. This notation overloads to vectors of
names: $\vec{x}^{\pi} := (x_{i}^{\pi} \; : \; 0 \leq i < |\vec{x}| )$ where $\pi \in \{L,R\}$.

We also use $P^{\Box} := P|\Box$.

In \cite{MeredithR05} an interpretation of the new operator is
given. It turns out that there are several possible interpretations
all enjoying the requisite algebraic properties of the operator (see
\cite{milner91polyadicpi}). We will therefore make liberal use of
$(\nu\; \vec{x})P$.

% subsection the_syntax_and_semantics_of_the_notation_system (end)   

\input{qm2pi.qmops} 

\input{qm2pi.sterngerlach} 

\input{qm2pi.metric} 

% section concurrent_process_calculi (end)

%\input{qm2pi.proofsketch}

% section proof sketch (end)

%\input{qm2pi.slviaknots} 

% section spatial logic via knots (end)

\input{qm2pi.conclusion}

% section conclusion (end)

%\input{qm2pi.dtcodes} 

% section wiring algorithm (end)

\input{qm2pi.ack} 

% section acknowledgments (end)

\newpage


\bibliographystyle{plain}   
\bibliography{../../biblios/main.bib}

\input{qm2pi.rhodetails}

\end{document}

 

%\ifpdf
%\usepackage[pdftex]{graphicx}
%\else
%\usepackage{graphicx}
%\fi

 % \ifpdf
%  \usepackage{pdfsync}
%  \if


%\title{Brief Article}
%\author{David F. Snyder}
%\author{L.G. Meredith}

%\address{Dept. of Math., Texas State University--San Marcos, San Marcos, TX 78666}
       
\pagestyle{empty}


\begin{document}

\lstset{language=[Objective]Caml,frame=shadowbox}

\documentclass[12pt]{llncs}
%\documentclass{jktr}

\usepackage[pdftex]{hyperref}                   
\usepackage {listings}
\usepackage {mathpartir}
\usepackage{bcprules}
%\usepackage{listings}
                       
\usepackage{graphicx} 
%\usepackage[margins=2.5cm,nohead,nofoot]{geometry}
%\usepackage{geometry}
\usepackage{amsfonts}
\usepackage{amstext}
\usepackage{latexsym}
\usepackage{amssymb}
\usepackage{color}


%\include{myPreamble}
\include{qm2pi.local} 

%\ifpdf
%\usepackage[pdftex]{graphicx}
%\else
%\usepackage{graphicx}
%\fi

 % \ifpdf
%  \usepackage{pdfsync}
%  \if


%\title{Brief Article}
%\author{David F. Snyder}
%\author{L.G. Meredith}

%\address{Dept. of Math., Texas State University--San Marcos, San Marcos, TX 78666}
       
\pagestyle{empty}


\begin{document}

\lstset{language=[Objective]Caml,frame=shadowbox}

\input{qm2pi.front}

% section front matter (end)

\input{qm2pi.intro} 
 
% section introduction (end)

% \input{qm2pi.knotations} 

% section notation (end)

\input{qm2pi.process.calculi} 

% section concurrent_process_calculi_and_spatial_logics_ (end)
    
%\input{qm2pi.knots2pi} 

%\input{qm2pi.trefoil} 

%\input{qm2pi.mainthm} 

% subsection basic_interpretation (end)

%\input{qm2pi.rho.presentation} 
\subsection{The syntax and semantics of the notation system}\label{sub:the_syntax_and_semantics_of_the_notation_system} % (fold)

We now summarize a technical presentation of the calculus that
embodies our theory of dynamics. The typical presentation of such a
calculus follows the style of giving generators and relations on
them. The grammar, below, describing term constructors, freely
generates the set of processes, $\Proc$. This set is then quotiented
by a relation known as structural congruence and it is over this set
that the notion of dynamics is expressed. This presentation is
essentially that of \cite{MeredithR05} with the addition of
polyadicity and summation. For readability we have relegated some of
the technical subtleties to an appendix.

\subsubsection{Process grammar}\label{subsub:process_grammar}

\begin{mathpar}
  \inferrule* [lab=synchronization] {} {{M} \bc \pzero \;|\; x?F \;|\; x!C }
  \and
  \inferrule* [lab=abstraction] {} {{F} \bc (x)P}
  \and
  \inferrule* [lab=concretion] {} {{C} \bc \langle Q \rangle}
  \and
  \inferrule* [lab=process] {} {{P,Q} \bc M \;| \;P|Q \;|\; @{x}}
  \and
  \inferrule* [lab=name] {} {{x} \bc \quotep{P}}
\end{mathpar} 

Note that $\vec{x}$ (resp. $\vec{P}$) denotes a vector of names
(resp. processes) of length $|\vec{x}|$ (resp. $|\vec{P}|$). We adopt
the following useful abbreviations.

\begin{mathpar}
   x?(\vec{y}).P := x.(\vec{y})P \and  x\clift{\vec{P}} := x.\clift{\vec{P}}
   \and x!(y) := \lift{x}{\dropn{y}}
   \and \Pi_{i=0}^{n-1}P_i := P_0 | \ldots | P_{n-1}
\end{mathpar}

\subsubsection{Structural congruence}

\paragraph{Free and bound names and alpha-equivalence.} At the
core of structural equivalence is alpha-equivalence which identifies
process that are the same up to a change of variable. Formally, we
recognize the distinction between free and bound names. The free names
of a process, $\freenames{P}$, may be calculated recursively as
follows:

\begin{mathpar}
\freenames{\pzero} := \emptyset
  \and \\
  \freenames{x?(y).P} := \{ x \} \cup (\freenames{P} \setminus \{ y \})
  \and 
  \freenames{x!\langle P \rangle} := \{ x \} \cup \{ P \} 
  \and \\
  \freenames{P|Q} := \freenames{P} \cup \freenames{Q}
  \and \\
  \freenames{@{x}} := \{ x \}
\end{mathpar}

$\pi$
$\quotep{\pi}$

$\freenames{-} : \pi \to \mathcal{P}(\quotep{\pi})$

\begin{eqnarray*}
  \freenames{\pzero} & := & \emptyset \\
  \freenames{x?(y).P} & := & \{ x \} \cup (\freenames{P} \setminus \{ y \}) \\
  \freenames{x!\langle P \rangle} & := & \{ x \} \cup \{ P \} \\
  \freenames{P|Q} & := & \freenames{P} \cup \freenames{Q} \\
  \freenames{\dropn{x}} & := & \{ x \}
\end{eqnarray*}

The bound names of a process, $\boundnames{P}$, are those names occurring in $P$
that are not free. For example, in $x?(y).0$, the name $x$ is free, while $y$ is bound.

\begin{mathpar}
  \inferrule* [lab=monoidal-laws] {} { P|Q \equiv Q|P \and P|0 \equiv P \and P|(Q|R) \equiv (P|Q)|R }
\end{mathpar}

\begin{mathpar}
  \inferrule* [lab=alpha-equivalence] {} { (x)P \equiv (y)P\{y/x\} \and y \not\in \freenames{P} }
\end{mathpar}

\begin{definition}
Then two processes, $P,Q$, are alpha-equivalent if $P = Q\{\vec{y}/\vec{x}\}$ for
some $\vec{x} \in \boundnames{Q},\vec{y} \in \boundnames{P}$, where $Q\{\vec{y}/\vec{x}\}$
denotes the capture-avoiding substitution of $\vec{y}$ for $\vec{x}$ in $Q$.
\end{definition}

\begin{definition}
  The {\em structural congruence} \cite{SangiorgiWalker} , $\equiv$,
  between processes is the least congruence containing
  alpha-equivalence, satisfying the abelian monoid laws
  (associativity, commutativity and $\pzero$ as identity) for parallel
  composition $|$ and for summation $+$.
\end{definition}

\subsection{Name equivalence}

We take name equivalence, written $\nameeq$, to be the smallest
equivalence relation generated by the following rules.

\begin{mathpar}
\inferrule*[lab=Quote-drop]
{ }
{ \quotep{@{x}} \nameeq x }

\inferrule*[lab=Struct-equiv]
{ P \scong Q }
{ \quotep{P} \nameeq \quotep{Q} }
\end{mathpar}

The astute reader will have noticed that the mutual recursion of names
and processes imposes a mutual recursion on alpha-equivalence and
structural equivalence via name-equivalence. Fortunately, all of this
works out pleasantly and we may calculate in the natural way, free of
concern. The reader interested in the details is referred to the
appendix \ref{appendix:rho_details}.

\subsection{Substitution}

We use $\Proc$ for the set of processes, $\QProc$ for the set of
names, and $\id{\{}\vec{y} / \vec{x} \id{\}}$ to denote partial maps,
$s : \QProc \rightarrow \QProc$. A map, $s$ lifts, uniquely, to a map
on process terms, $\widehat{s} : \Proc \rightarrow \Proc$ by the
following equations.

\begin{mathpar}
  (0) \psubstp{Q}{P} := 0 \\
  (R \juxtap S) \psubstp{Q}{P}
  :=    
  (R)\psubstp{Q}{P} \juxtap (S) \psubstp{Q}{P} \\
  (x?(y).R) \psubstp{Q}{P}    
  :=    
  (x)\substp{Q}{P} (z)\concat( (R \psubstn{z}{y}) \psubstp{Q}{P} ) \\
  (\lift{x}{R}) \psubstp{Q}{P}  
  :=
  \lift{(x)\substp{Q}{P}}{ R \psubstp{Q}{P} } \\
%   (\dropn{x})  \psubstp{Q}{P}       
%   := 
%   \left\{ 
%     \begin{array}{ccc} 
%       \dropn{\quotep{Q}} & & x \nameeq \quotep{P} \\
%       \dropn{x} & & otherwise \\
%     \end{array}
%   \right. 
  (\dropn{x})  \psubstp{Q}{P}       
  := 
  \left\{ 
    \begin{array}{ccc} 
      Q & & x \nameeq \quotep{P} \\
      \dropn{x} & & otherwise \\
    \end{array}
  \right.
\end{mathpar}
 

where

\begin{eqnarray}
  (x)\id{\{} \lpquote Q \rpquote / \lpquote P \rpquote \id{\}}            = 
  \left\{ 
    \begin{array}{ccc}
      \lpquote Q \rpquote & & x \nameeq \lpquote P \rpquote \\
      x & & otherwise \\
    \end{array}
  \right. \nonumber
\end{eqnarray}

and $z$ is chosen distinct from $\quotep{P}$, $\quotep{Q}$, the free
names in $Q$, and all the names in $R$. Our $\alpha$-equivalence will
be built in the standard way from this substitution.

\begin{remark}\label{rem:no_self_referential_names}
  One consequence of these definitions is that $\forall P. \quotep{P}
  \not\in \freenames{P}$.
\end{remark}

\subsection{ Dynamic quote: an example }

Anticipating something of what's to come, consider applying the
substitution, $\widehat{\id{\{}u / z \id{\}}}$, to the following pair
of processes, $\lift{w}{y!(z)}$ and $w[ \lpquote y!(z) \rpquote ]$.

\begin{eqnarray}
	\lift{w}{y!(z)}\widehat{\id{\{}u / z \id{\}}}
		& = &
		\lift{w}{y!(u)} \nonumber\\
	w[ \lpquote y!(z) \rpquote ] \widehat{ \id{\{}u / z \id{\}} }
		& = &
		w[ \lpquote y!(z) \rpquote ] \nonumber
\end{eqnarray}

Because the body of the process between quotes is impervious to
substitution, we get radically different answers. In fact, by
examining the first process in an input context,
e.g. $x?(z).\lift{w}{y!(z)}$, we see that the process under the lift
operator may be shaped by prefixed inputs binding a name inside it. In
this sense, the lift operator will be seen as a way to dynamically
construct processes before reifying them as names.

Finally equipped with these standard features we can present the
dynamics of the calculus.

\subsubsection{Operational semantics} 

Finally, we introduce the computational dynamics. What marks these
algebras as distinct from other more traditionally studied algebraic
structures, e.g. vector spaces or polynomial rings, is the manner in
which dynamics is captured. In traditional structures, dynamics is typically
expressed through morphisms between such structures, as in linear maps
between vector spaces or morphisms between rings. In algebras
associated with the semantics of computation, the dynamics is
expressed as part of the algebraic structure itself, through a
reduction reduction relation typically denoted by $\red$. Below, we
give a recursive presentation of this relation for the calculus used
in the encoding.

$\red \subseteq \pi \times \pi$
$\red : \pi \to \mathcal{P}(\pi)$

\begin{mathpar}
  \inferrule* [lab=Comm] { \textsf{match}( x_{src}, x_{trgt} ) } { x_{trgt}?(y)P \; | \; x_{src}!\langle {Q} \rangle \red P\{\quotep{Q}/y}\} }
  \and \\
  \inferrule* [lab=Par] {{P} \red {P}'} {{{P} | {Q}} \red {{P}' | {Q}}}
  \and
  \inferrule* [lab=Equiv]{{{P} \scong {P}'} \andalso {{P}' \red {Q}'} \andalso {{Q}' \scong {Q}}}{{P} \red {Q}}
\end{mathpar}

\begin{eqnarray*}
  match_{\equiv} (\quotep{P},\quotep{Q}) & := & P \equiv Q \\
  match_{\dagger}(\quotep{P},\quotep{Q}) & := & \forall R. P|Q \red^{*} R => R \red^{*} 0 \\
  match_{K}(\quotep{P},\quotep{Q}) & := & K \mbox{ for some context } K
\end{eqnarray*}

$u?(x)P | u!\langle Q \rangle \red P\{\quotep{Q}/x\}$

%We write $\wred$ for $\red^*$, and $P\red$ if $\exists Q $ such that $ P \red Q$.
We write $P\red$ if $\exists Q $ such that $ P \red Q$ and $P\not\red$, otherwise.

\section{Replication}

As mentioned before, it is known that replication (and hence
recursion) can be implemented in a higher-order process algebra
\cite{SangiorgiWalker}. As our first example of calculation with the
machinery thus far presented we give the construction explicitly in
the {\rhoc}.

\begin{eqnarray}
	D_{x} & := & \prefix{x}{y}{(\binpar{\outputp{x}{y}}{@{y}})} \nonumber\\
	\bangp_{x}{P} & := & \binpar{{x}!\langle{\binpar{D_{x}}{P}}\rangle}{D_{x}} \nonumber
\end{eqnarray}

\begin{eqnarray}
	\bangp_{x}{P} & & \nonumber\\
	=
	& {x}!\langle{(\prefix{x}{y}{(\outputp{x}{y} | @{y})) | P}}\rangle 
	      | \prefix{x}{y}{(\outputp{x}{y} | @{y})} & \nonumber\\
	\red
	& (\outputp{x}{y} | @{y})\substn{\quotep{(\prefix{x}{y}{(@{y} | \outputp{x}{y})) | P}}}{y} & \nonumber\\
	=
	& \outputp{x}{\quotep{(\prefix{x}{y}{(\outputp{x}{y} | @{y})) | P}}}
	  | {(\prefix{x}{y}{(\outputp{x}{y} | @{y})) | P}} & \nonumber\\
	\red
	& \ldots & \nonumber\\
	\red^*
	& P | P | \ldots & \nonumber
\end{eqnarray}

Of course, this encoding, as an implementation, runs away, unfolding
$\bangp{P}$ eagerly. A lazier and more implementable replication
operator, restricted to input-guarded processes, may be obtained as follows.

\begin{eqnarray}
\bangp{\prefix{u}{v}{P}} 
	:= 
	\binpar{\lift{x}{\prefix{u}{v}{(\binpar{D(x)}{P})}}}{D(x)} \nonumber
\end{eqnarray}

\begin{remark}
  Note that the lazier definition still does not deal with summation
  or mixed summation (i.e. sums over input and output). The reader is
  invited to construct definitions of replication that deal with these
  features. 

  Further, the definitions are parameterized in a name, $x$. Can you,
  gentle reader, make a definition that eliminates this parameter and
  guarantees no accidental interaction between the replication
  machinery and the process being replicated -- i.e. no accidental
  sharing of names used by the process to get its work done and the
  name(s) used by the replication to effect copying. This latter
  revision of the definition of replication is crucial to obtaining
  the expected identity $!!P \sim !P$.
\end{remark}

\begin{remark}\label{rem:paradoxical_combinator}
  The reader familiar with the lambda calculus will have noticed the
  similarity between $D$ and the paradoxical combinator.

  [Ed. note: the existence of this seems to suggest we have to be more
  restrictive on the set of processes and names we admit if we are to
  support no-cloning.]
\end{remark}

\subsubsection{Bisimulation}

The computational dynamics gives rise to another kind of equivalence,
the equivalence of computational behavior. As previously mentioned
this is typically captured \emph{via} some form of bisimulation.

% The notion we use in this paper is weak barbed bisimulation
% \cite{milner91polyadicpi}.

The notion we use in this paper is derived from weak barbed
bisimulation \cite{milner91polyadicpi}. 

\begin{definition}
An \emph{observation relation}, $\downarrow_{\mathcal N}$, over a set
of names, $\mathcal N$, is the smallest relation satisfying the rules
below.

\infrule[Out-barb]{y \in {\mathcal N}, \; x \nameeq y}
		  {\outputp{x}{v} \downarrow_{\mathcal N} x}
\infrule[Par-barb]{\mbox{$P\downarrow_{\mathcal N} x$ or $Q\downarrow_{\mathcal N} x$}}
		  {\binpar{P}{Q} \downarrow_{\mathcal N} x}

We write $P \Downarrow_{\mathcal N} x$ if there is $Q$ such that 
$P \wred Q$ and $Q \downarrow_{\mathcal N} x$.
\end{definition}

\begin{definition}
%\label{def.bbisim}
An  ${\mathcal N}$-\emph{barbed bisimulation} over a set of names, ${\mathcal N}$, is a symmetric binary relation 
${\mathcal S}_{\mathcal N}$ between agents such that $P\rel{S}_{\mathcal N}Q$ implies:
\begin{enumerate}
\item If $P \red P'$ then $Q \wred Q'$ and $P'\rel{S}_{\mathcal N} Q'$.
\item If $P\downarrow_{\mathcal N} x$, then $Q\Downarrow_{\mathcal N} x$.
\end{enumerate}
$P$ is ${\mathcal N}$-barbed bisimilar to $Q$, written
$P \wbbisim_{\mathcal N} Q$, if $P \rel{S}_{\mathcal N} Q$ for some ${\mathcal N}$-barbed bisimulation ${\mathcal S}_{\mathcal N}$.
\end{definition}

$\mathcal{R} \subseteq \pi \times \pi$

$P \mathcal{R} Q => \forall P'. P \red P' \Rightarrow \exists Q'. Q \red Q', P' \mathcal{R} Q'$

$P \vdash x \Rightarrow Q \vdash x$

\begin{mathpar}
  \inferrule*[lab=Out-barb]{x \nameeq y}{{y}!\langle{Q}\rangle \vdash x}
  \and
  \inferrule*[lab=Par-barb]{\mbox{$P\vdash x$ or $Q\vdash x$}}{\binpar{P}{Q} \vdash x}
\end{mathpar}

\subsubsection{Contexts}

One of the principle advantages of computational calculi like the
$\pi$-calculus is a well-defined notion of context,
contextual-equivalence and a correlation between
contextual-equivalence and notions of bisimulation. The notion of
context allows the decomposition of a process into (sub-)process and
its syntactic environment, its context. Thus, a context may be
thought of as a process with a ``hole'' (written $\Box$) in it. The
application of a context $M$ to a process $P$, written $M[P]$, is
tantamount to filling the hole in $M$ with $P$. In this paper we do
not need the full weight of this theory, but do make use of the notion
of context in the proof the main theorem. 

\begin{mathpar}
  \inferrule* [lab=summation] {} {{M_{M},M_{N}} \bc \Box \;|\; x.M_{A} \;|\; M_{M}+M_{N}}
  \and
  \inferrule* [lab=agent] {} {{M_{A}} \bc (\vec{x})M_{P} \;| \; \clift{P_0,\ldots,M_{P},\ldots,P_N}}
  \and \\
  \inferrule* [lab=process] {} {{M_{P}} \bc M_{N} \;| \;P|M_{P} }
\end{mathpar} 

\begin{mathpar}
  \inferrule* [lab=sychronization] {} {M_{N} \bc \Box \;|\; x?M_{F} \;|\; x!M_{C}}
  \and
  \inferrule* [lab=abstraction] {} {{M_{F}} \bc (x)M_{P} }
  \and
  \inferrule* [lab=concretion] {} {{M_{C}} \bc \langle M_{P} \rangle }
  \and \\
  \inferrule* [lab=process] {} {{M_{P}} \bc M_{N} \;| \;P|M_{P} }
\end{mathpar}

\begin{definition}[contextual application] Given a context $M$, and
  process $P$, we define the \emph{contextual application}, $M[P] :=
  M\{P/\Box\}$. That is, the contextual application of M to P is the
  substitution of $P$ for $\Box$ in $M$.
\end{definition}

$\meaningof{-} : L \to \mathcal{P}(\pi)$

\begin{mathpar}
  \inferrule* [lab=collection] {} {\meaningof{true} = \pi, \and \meaningof{~E} = \pi \setminus \meaningof{E}, \and \meaningof{E_{1} \& E_{2}} = \meaningof{E_{1}} \cap \meaningof{E_{2}}}
\end{mathpar}

\begin{mathpar}
  \inferrule* [lab=structure] {} {\meaningof{0} = \{ P \in \pi | P \equiv 0 \}, \and \\ \meaningof{E_1 | E_2} = \{ P \in \pi | P \equiv P_{1} | P_{2}, P_{1} \in \meaningof{E_{1}}, P_{2} \in \meaningof{E_2}\} }
\end{mathpar}

\begin{mathpar}
 \inferrule* [lab=behavior] {} {\meaningof{\langle a?b \rangle E} = \{ P \in \pi | P \equiv Q | u?(y)P', \\ \and \\\\ \and \\ \;\;\; u \in \meaningof{a}, \forall z.P'\{z/y\} \in \meaningof{E\{z/b\}}\}, \and \\ \meaningof{a!E} = \{ P \in \pi | P \equiv Q | x!\langle P' \rangle, x \in \meaningof{a} P' \in \meaningof{E}\} }
\end{mathpar}

\begin{mathpar}
 \inferrule* [lab=nominal] {} {\meaningof{\quotep{E}} = \{ \quotep{P} \in \quotep{\pi} | P \in \meaningof{E} \}, \and \meaningof{\quotep{P}} = \{ \quotep{Q} \in \quotep{\pi} | P \equiv Q \} \and \\ \meaningof{@\quotep{E}} = \{ P \in \pi | P \equiv @x, x \in \meaningof{E} \}}
\end{mathpar}

\begin{eqnarray*}
  \\
  \meaningof{-} : TS \to ST
\end{eqnarray*}

\begin{eqnarray*}
  \\
  L : TS \to ST
\end{eqnarray*}

\begin{eqnarray*}
  \\
  P \models E \iff P \in \meaningof{E}
\end{eqnarray*}

\begin{eqnarray*}
  P \approx_{L} Q \iff \forall E \in L. P \models E \iff Q \models E
\end{eqnarray*}

\begin{eqnarray*}
  P \approx_{K} Q
\end{eqnarray*}

\begin{eqnarray*}
  P \approx Q
\end{eqnarray*}

$\approx_{K} = \approx = \approx_{L}$

\subsubsection{Contextual duality}

Note that contexts extend the quotation operation to a family of
operations from processes to names. Given a context, $M$, we can
define a \emph{nominal context}, $\quotep{M}$ by $\quotep{M}[P] :=
\quotep{M[P]}$. To foreshadow what is to come we observe that these
operations enjoy a duality with processes very much like the duality
between vectors and maps from vectors to scalars.

Further, because the calculus is essentially higher-order, we have a
correspondence between contexts and processes. More specifically,
given a name $x$ and a context $M$ we can construct $M^{*}_{x}$ such
that 

\begin{mathpar}
  M^{*}_{x} | \lift{x}{P} \red M[P]
\end{mathpar}

namely,

\begin{mathpar}
  M^{*}_{x} := x?(u).M[\dropn{u}]
\end{mathpar}

The dependence of $M^{*}_{x}$ on a name makes it an abstraction, 

\begin{mathpar}
  M^{*} := (x)x?(u).M[\dropn{u}]
\end{mathpar}

\subsection{Additional notation}

It will sometimes be convenient to denote the process a name
quotes. We already have the notation $x = \quotep{P}$, but it will be
convenient to introduce an alternate notation, $\procn{x}$, when we
want to emphasize the connection to the use of the name. Note that, by
virtue of name equivalence, $\quotep{\procn{x}} \nameeq x$; so, the
notation is consistent with previous definitions.

Further, because names have structure it is possible to effect
substitutions on the basis of that structure. This means we need to
upgrade our notation for substitutions, which we accomplish by
adapting comprehension notation. Thus,

\begin{mathpar}
  P\{ y / x : x \in S \}
\end{mathpar}

is interpreted to mean the process derived from P by replacing (in a
capture-avoiding manner) each occurrence of $x$ in $S$ by $y$. For example,

\begin{mathpar}
  P\{ \quotep{\procn{x}|\procn{x}} / x : x \in \freenames{P} \}
\end{mathpar}

will replace each (occurrence) of a free name $x$ in $P$ by
$\quotep{\procn{x}|\procn{x}}$.

Also, we will avail ourselves of the notation $x^{L}$ and $x^{R}$ to
denote injections of a name into disjoint copies of the name
space. There are numerous ways to accomplish this. One example can be
found in \cite{MeredithR05}. This notation overloads to vectors of
names: $\vec{x}^{\pi} := (x_{i}^{\pi} \; : \; 0 \leq i < |\vec{x}| )$ where $\pi \in \{L,R\}$.

We also use $P^{\Box} := P|\Box$.

In \cite{MeredithR05} an interpretation of the new operator is
given. It turns out that there are several possible interpretations
all enjoying the requisite algebraic properties of the operator (see
\cite{milner91polyadicpi}). We will therefore make liberal use of
$(\nu\; \vec{x})P$.

% subsection the_syntax_and_semantics_of_the_notation_system (end)   

\input{qm2pi.qmops} 

\input{qm2pi.sterngerlach} 

\input{qm2pi.metric} 

% section concurrent_process_calculi (end)

%\input{qm2pi.proofsketch}

% section proof sketch (end)

%\input{qm2pi.slviaknots} 

% section spatial logic via knots (end)

\input{qm2pi.conclusion}

% section conclusion (end)

%\input{qm2pi.dtcodes} 

% section wiring algorithm (end)

\input{qm2pi.ack} 

% section acknowledgments (end)

\newpage


\bibliographystyle{plain}   
\bibliography{../../biblios/main.bib}

\input{qm2pi.rhodetails}

\end{document}



% section front matter (end)

\section{Introduction}\label{sec:introduction} % (fold)
In this draft of the material i am going to have to dispense with the
usual writing conventions adopted in papers on these topics. i'm going
to have adopt whatever tone i need at the time i'm writing up the
calculations. Sometimes this may be very conversational; others it may
be the barest mathematical grunts; others still it may be that i have
lifted text from one of my other papers because the exposition of some
point was better said there. i hope that my readers are not unduly put
out by this decision. i'm not doing this to flout convention or be
rebellious. i find these calculations very technically challenging. To
keep everything going technically, something has to give; i have to
let go of some cognitive burden. So, the academic writing style --
with all of its trade-offs in terms of facilitating technical
communication -- is what i'm letting go of. Perhaps subsequent drafts
can be tightened and polished, but for now, i'm going to speak as if
we were sitting together in a coffee shop with a laptop, wifi and a
pad of paper and a pencil.

So, here's what i have to say. We -- you and i, comfortably ensconced
in our coffee shop and well-equipped with our tools -- can realize and
carry out the calculations of quantum mechanics over a very different
formal theory of dynamics, a formal theory of dynamics that
corresponds to a theory of concurrent computation with
\emph{reflection}. It has the advantage that the underlying theory is
already `quantized', but supports analogues all of the continuuous
operations. Strikingly, this underlying theory has recently been
connected with a notion of metric that we can show, by calculating
together, coincides with the metric induced by the inner product.

There are a lot of reasons why you might be interested in seeing
calculations of this form. Here's why i'm interested. For the past
several centuries there has been no competitor to the ``Newtonian''
account of dynamics. As a result the predominant share of accounts of
dynamical systems and situations have had to be formulated in terms of
the Newtonian machinery. i view this as an intellectually dangerous
position to occupy. Everything, despite it's intrinsic shape, turns
into a nail to be hit with this hammer. Recently, however, the theory
of computation has matured to the point where we have candidates for
theories of dynamics that offer very different perspective on
reasoning about dynamical systems and situations. Testing these
candidates against very successful accounts of dynamical situations,
like quantum mechanics, is going to give us some sense of how mature
they are and some measure of the quality of these accounts of
dynamics.

\subsection{Summary of contributions and outline of paper}

So, we're going to develop an interpretation of the operations of
quantum mechanics normally interpreted by Hilbert spaces and
operators. We're going to do this over a theory of computation. Note
that this is very different than the usual quantum computation program
which develops notions of computation over quantum mechanics. Rather,
we are developing a story that aligns with Wheeler's slogan: It from
Bit. To do this we will first provide an account of the theory of
computation at play here. Then we will dive into a calculation-driven
interpretation of the operations of quantum mechanics.

The reason we take this approach is that -- until very recently --
there hasn't been an axiomatic account of quantum mechanics. As a
result there has been no sharp delineation of the mathematical theory
supporting interpretation of the physical theory and the physical
theory, itself. So, ambient features of the maths are free to be
exploited (or supressed) without a real accounting of their physical
relevance. There is no sharp statement ``here's the physical theory''
qua \emph{theory} and ``here's the mathematical interpretation''
enabling a judgment of how faithful the interpretation is -- apart
from experimental observation. When there is an axiomatic account we
can judge how well a given mathematical formalism supports an
interpretation of the axioms, independent of
experimentation. Likewise, we can judge how well we have captured our
physical evidence and experience with our axiomatics, independent of
any specific mathematical implementation, with accidental detail that
may or may not have physical significance. 

In lieu of a fully fleshed out and vetted axiomatic account of quantum
mechanics, interpreting the operational notions in service of modeling
physical systems will have to suffice. In other words, we are not in
the business of providing a model of Hilbert spaces and operators. We
are in the business of providing a model of quantum mechanics because
we are motivated by testing our notions of dynamics against physical
theory; and, the predictive calculations of the physical theory must
serve as the best formulation -- shy of a fully fleshed out axiomatic
account -- of the physical theory itself (as they have for scientific
theories since time immemorial). Put another way, despite a
whole-hearted commitment to an It-from-Bit ontology, we are firmly
aligned with the shut-up-and-calculate camp as the best way to obtain
results either from the physical perspective or as a quality assurance
measure of our fledgling theory of dynamics.

In detail, we present a reflective process calculus. Then we develop
intuitive correspondences between the notions available in this
calculus and the usual physical notions supporting quantum mechanical
calculations. Thus, 

\begin{table}[htp]
  \center{
    \fbox{
      \begin{tabular}{c|c}
        quantum mechanics & process calculus \\
        \hline
        scalar & name \\
        state vector & process \\
        dual & contextual duals \\
        matrix & formal sums of process-context-dual pairs \\
        orthogonality & process annihilation \\
        inner product & execution-formula + quoting
      \end{tabular}
    }
  }
  \caption{QM - process calculi correspondences}
\end{table}

Then we tighten up these intuitions to operational definitions. We
employ the Dirac notation as the best proxy we can find for an
abstract syntax of the quantum mechanical notions. The definitions we
develop put us in contact with equational constraints coming from the
theory that we demonstrate the definitions and calculations satisfy.

This puts us in a position to shut up and calculate for the
Stern-Gerlach experimental set up, showing how these predictive
calculations become calculations on processes in our theory of a
reflective process calculus.

Penultimately, we demonstrate that the notion of metric coming from
the inner product coincides with the notion of metric available from
the theory of bisimulation. This demonstration gives us the right to
think of space as arising from behavior. Finally, we consider where we
might go from the new vantage point we have obtained.

% section introduction (end) 
 
% section introduction (end)

% \documentclass[12pt]{llncs}
%\documentclass{jktr}

\usepackage[pdftex]{hyperref}                   
\usepackage {listings}
\usepackage {mathpartir}
\usepackage{bcprules}
%\usepackage{listings}
                       
\usepackage{graphicx} 
%\usepackage[margins=2.5cm,nohead,nofoot]{geometry}
%\usepackage{geometry}
\usepackage{amsfonts}
\usepackage{amstext}
\usepackage{latexsym}
\usepackage{amssymb}
\usepackage{color}


%\include{myPreamble}
\include{qm2pi.local} 

%\ifpdf
%\usepackage[pdftex]{graphicx}
%\else
%\usepackage{graphicx}
%\fi

 % \ifpdf
%  \usepackage{pdfsync}
%  \if


%\title{Brief Article}
%\author{David F. Snyder}
%\author{L.G. Meredith}

%\address{Dept. of Math., Texas State University--San Marcos, San Marcos, TX 78666}
       
\pagestyle{empty}


\begin{document}

\lstset{language=[Objective]Caml,frame=shadowbox}

\input{qm2pi.front}

% section front matter (end)

\input{qm2pi.intro} 
 
% section introduction (end)

% \input{qm2pi.knotations} 

% section notation (end)

\input{qm2pi.process.calculi} 

% section concurrent_process_calculi_and_spatial_logics_ (end)
    
%\input{qm2pi.knots2pi} 

%\input{qm2pi.trefoil} 

%\input{qm2pi.mainthm} 

% subsection basic_interpretation (end)

%\input{qm2pi.rho.presentation} 
\subsection{The syntax and semantics of the notation system}\label{sub:the_syntax_and_semantics_of_the_notation_system} % (fold)

We now summarize a technical presentation of the calculus that
embodies our theory of dynamics. The typical presentation of such a
calculus follows the style of giving generators and relations on
them. The grammar, below, describing term constructors, freely
generates the set of processes, $\Proc$. This set is then quotiented
by a relation known as structural congruence and it is over this set
that the notion of dynamics is expressed. This presentation is
essentially that of \cite{MeredithR05} with the addition of
polyadicity and summation. For readability we have relegated some of
the technical subtleties to an appendix.

\subsubsection{Process grammar}\label{subsub:process_grammar}

\begin{mathpar}
  \inferrule* [lab=synchronization] {} {{M} \bc \pzero \;|\; x?F \;|\; x!C }
  \and
  \inferrule* [lab=abstraction] {} {{F} \bc (x)P}
  \and
  \inferrule* [lab=concretion] {} {{C} \bc \langle Q \rangle}
  \and
  \inferrule* [lab=process] {} {{P,Q} \bc M \;| \;P|Q \;|\; @{x}}
  \and
  \inferrule* [lab=name] {} {{x} \bc \quotep{P}}
\end{mathpar} 

Note that $\vec{x}$ (resp. $\vec{P}$) denotes a vector of names
(resp. processes) of length $|\vec{x}|$ (resp. $|\vec{P}|$). We adopt
the following useful abbreviations.

\begin{mathpar}
   x?(\vec{y}).P := x.(\vec{y})P \and  x\clift{\vec{P}} := x.\clift{\vec{P}}
   \and x!(y) := \lift{x}{\dropn{y}}
   \and \Pi_{i=0}^{n-1}P_i := P_0 | \ldots | P_{n-1}
\end{mathpar}

\subsubsection{Structural congruence}

\paragraph{Free and bound names and alpha-equivalence.} At the
core of structural equivalence is alpha-equivalence which identifies
process that are the same up to a change of variable. Formally, we
recognize the distinction between free and bound names. The free names
of a process, $\freenames{P}$, may be calculated recursively as
follows:

\begin{mathpar}
\freenames{\pzero} := \emptyset
  \and \\
  \freenames{x?(y).P} := \{ x \} \cup (\freenames{P} \setminus \{ y \})
  \and 
  \freenames{x!\langle P \rangle} := \{ x \} \cup \{ P \} 
  \and \\
  \freenames{P|Q} := \freenames{P} \cup \freenames{Q}
  \and \\
  \freenames{@{x}} := \{ x \}
\end{mathpar}

$\pi$
$\quotep{\pi}$

$\freenames{-} : \pi \to \mathcal{P}(\quotep{\pi})$

\begin{eqnarray*}
  \freenames{\pzero} & := & \emptyset \\
  \freenames{x?(y).P} & := & \{ x \} \cup (\freenames{P} \setminus \{ y \}) \\
  \freenames{x!\langle P \rangle} & := & \{ x \} \cup \{ P \} \\
  \freenames{P|Q} & := & \freenames{P} \cup \freenames{Q} \\
  \freenames{\dropn{x}} & := & \{ x \}
\end{eqnarray*}

The bound names of a process, $\boundnames{P}$, are those names occurring in $P$
that are not free. For example, in $x?(y).0$, the name $x$ is free, while $y$ is bound.

\begin{mathpar}
  \inferrule* [lab=monoidal-laws] {} { P|Q \equiv Q|P \and P|0 \equiv P \and P|(Q|R) \equiv (P|Q)|R }
\end{mathpar}

\begin{mathpar}
  \inferrule* [lab=alpha-equivalence] {} { (x)P \equiv (y)P\{y/x\} \and y \not\in \freenames{P} }
\end{mathpar}

\begin{definition}
Then two processes, $P,Q$, are alpha-equivalent if $P = Q\{\vec{y}/\vec{x}\}$ for
some $\vec{x} \in \boundnames{Q},\vec{y} \in \boundnames{P}$, where $Q\{\vec{y}/\vec{x}\}$
denotes the capture-avoiding substitution of $\vec{y}$ for $\vec{x}$ in $Q$.
\end{definition}

\begin{definition}
  The {\em structural congruence} \cite{SangiorgiWalker} , $\equiv$,
  between processes is the least congruence containing
  alpha-equivalence, satisfying the abelian monoid laws
  (associativity, commutativity and $\pzero$ as identity) for parallel
  composition $|$ and for summation $+$.
\end{definition}

\subsection{Name equivalence}

We take name equivalence, written $\nameeq$, to be the smallest
equivalence relation generated by the following rules.

\begin{mathpar}
\inferrule*[lab=Quote-drop]
{ }
{ \quotep{@{x}} \nameeq x }

\inferrule*[lab=Struct-equiv]
{ P \scong Q }
{ \quotep{P} \nameeq \quotep{Q} }
\end{mathpar}

The astute reader will have noticed that the mutual recursion of names
and processes imposes a mutual recursion on alpha-equivalence and
structural equivalence via name-equivalence. Fortunately, all of this
works out pleasantly and we may calculate in the natural way, free of
concern. The reader interested in the details is referred to the
appendix \ref{appendix:rho_details}.

\subsection{Substitution}

We use $\Proc$ for the set of processes, $\QProc$ for the set of
names, and $\id{\{}\vec{y} / \vec{x} \id{\}}$ to denote partial maps,
$s : \QProc \rightarrow \QProc$. A map, $s$ lifts, uniquely, to a map
on process terms, $\widehat{s} : \Proc \rightarrow \Proc$ by the
following equations.

\begin{mathpar}
  (0) \psubstp{Q}{P} := 0 \\
  (R \juxtap S) \psubstp{Q}{P}
  :=    
  (R)\psubstp{Q}{P} \juxtap (S) \psubstp{Q}{P} \\
  (x?(y).R) \psubstp{Q}{P}    
  :=    
  (x)\substp{Q}{P} (z)\concat( (R \psubstn{z}{y}) \psubstp{Q}{P} ) \\
  (\lift{x}{R}) \psubstp{Q}{P}  
  :=
  \lift{(x)\substp{Q}{P}}{ R \psubstp{Q}{P} } \\
%   (\dropn{x})  \psubstp{Q}{P}       
%   := 
%   \left\{ 
%     \begin{array}{ccc} 
%       \dropn{\quotep{Q}} & & x \nameeq \quotep{P} \\
%       \dropn{x} & & otherwise \\
%     \end{array}
%   \right. 
  (\dropn{x})  \psubstp{Q}{P}       
  := 
  \left\{ 
    \begin{array}{ccc} 
      Q & & x \nameeq \quotep{P} \\
      \dropn{x} & & otherwise \\
    \end{array}
  \right.
\end{mathpar}
 

where

\begin{eqnarray}
  (x)\id{\{} \lpquote Q \rpquote / \lpquote P \rpquote \id{\}}            = 
  \left\{ 
    \begin{array}{ccc}
      \lpquote Q \rpquote & & x \nameeq \lpquote P \rpquote \\
      x & & otherwise \\
    \end{array}
  \right. \nonumber
\end{eqnarray}

and $z$ is chosen distinct from $\quotep{P}$, $\quotep{Q}$, the free
names in $Q$, and all the names in $R$. Our $\alpha$-equivalence will
be built in the standard way from this substitution.

\begin{remark}\label{rem:no_self_referential_names}
  One consequence of these definitions is that $\forall P. \quotep{P}
  \not\in \freenames{P}$.
\end{remark}

\subsection{ Dynamic quote: an example }

Anticipating something of what's to come, consider applying the
substitution, $\widehat{\id{\{}u / z \id{\}}}$, to the following pair
of processes, $\lift{w}{y!(z)}$ and $w[ \lpquote y!(z) \rpquote ]$.

\begin{eqnarray}
	\lift{w}{y!(z)}\widehat{\id{\{}u / z \id{\}}}
		& = &
		\lift{w}{y!(u)} \nonumber\\
	w[ \lpquote y!(z) \rpquote ] \widehat{ \id{\{}u / z \id{\}} }
		& = &
		w[ \lpquote y!(z) \rpquote ] \nonumber
\end{eqnarray}

Because the body of the process between quotes is impervious to
substitution, we get radically different answers. In fact, by
examining the first process in an input context,
e.g. $x?(z).\lift{w}{y!(z)}$, we see that the process under the lift
operator may be shaped by prefixed inputs binding a name inside it. In
this sense, the lift operator will be seen as a way to dynamically
construct processes before reifying them as names.

Finally equipped with these standard features we can present the
dynamics of the calculus.

\subsubsection{Operational semantics} 

Finally, we introduce the computational dynamics. What marks these
algebras as distinct from other more traditionally studied algebraic
structures, e.g. vector spaces or polynomial rings, is the manner in
which dynamics is captured. In traditional structures, dynamics is typically
expressed through morphisms between such structures, as in linear maps
between vector spaces or morphisms between rings. In algebras
associated with the semantics of computation, the dynamics is
expressed as part of the algebraic structure itself, through a
reduction reduction relation typically denoted by $\red$. Below, we
give a recursive presentation of this relation for the calculus used
in the encoding.

$\red \subseteq \pi \times \pi$
$\red : \pi \to \mathcal{P}(\pi)$

\begin{mathpar}
  \inferrule* [lab=Comm] { \textsf{match}( x_{src}, x_{trgt} ) } { x_{trgt}?(y)P \; | \; x_{src}!\langle {Q} \rangle \red P\{\quotep{Q}/y}\} }
  \and \\
  \inferrule* [lab=Par] {{P} \red {P}'} {{{P} | {Q}} \red {{P}' | {Q}}}
  \and
  \inferrule* [lab=Equiv]{{{P} \scong {P}'} \andalso {{P}' \red {Q}'} \andalso {{Q}' \scong {Q}}}{{P} \red {Q}}
\end{mathpar}

\begin{eqnarray*}
  match_{\equiv} (\quotep{P},\quotep{Q}) & := & P \equiv Q \\
  match_{\dagger}(\quotep{P},\quotep{Q}) & := & \forall R. P|Q \red^{*} R => R \red^{*} 0 \\
  match_{K}(\quotep{P},\quotep{Q}) & := & K \mbox{ for some context } K
\end{eqnarray*}

$u?(x)P | u!\langle Q \rangle \red P\{\quotep{Q}/x\}$

%We write $\wred$ for $\red^*$, and $P\red$ if $\exists Q $ such that $ P \red Q$.
We write $P\red$ if $\exists Q $ such that $ P \red Q$ and $P\not\red$, otherwise.

\section{Replication}

As mentioned before, it is known that replication (and hence
recursion) can be implemented in a higher-order process algebra
\cite{SangiorgiWalker}. As our first example of calculation with the
machinery thus far presented we give the construction explicitly in
the {\rhoc}.

\begin{eqnarray}
	D_{x} & := & \prefix{x}{y}{(\binpar{\outputp{x}{y}}{@{y}})} \nonumber\\
	\bangp_{x}{P} & := & \binpar{{x}!\langle{\binpar{D_{x}}{P}}\rangle}{D_{x}} \nonumber
\end{eqnarray}

\begin{eqnarray}
	\bangp_{x}{P} & & \nonumber\\
	=
	& {x}!\langle{(\prefix{x}{y}{(\outputp{x}{y} | @{y})) | P}}\rangle 
	      | \prefix{x}{y}{(\outputp{x}{y} | @{y})} & \nonumber\\
	\red
	& (\outputp{x}{y} | @{y})\substn{\quotep{(\prefix{x}{y}{(@{y} | \outputp{x}{y})) | P}}}{y} & \nonumber\\
	=
	& \outputp{x}{\quotep{(\prefix{x}{y}{(\outputp{x}{y} | @{y})) | P}}}
	  | {(\prefix{x}{y}{(\outputp{x}{y} | @{y})) | P}} & \nonumber\\
	\red
	& \ldots & \nonumber\\
	\red^*
	& P | P | \ldots & \nonumber
\end{eqnarray}

Of course, this encoding, as an implementation, runs away, unfolding
$\bangp{P}$ eagerly. A lazier and more implementable replication
operator, restricted to input-guarded processes, may be obtained as follows.

\begin{eqnarray}
\bangp{\prefix{u}{v}{P}} 
	:= 
	\binpar{\lift{x}{\prefix{u}{v}{(\binpar{D(x)}{P})}}}{D(x)} \nonumber
\end{eqnarray}

\begin{remark}
  Note that the lazier definition still does not deal with summation
  or mixed summation (i.e. sums over input and output). The reader is
  invited to construct definitions of replication that deal with these
  features. 

  Further, the definitions are parameterized in a name, $x$. Can you,
  gentle reader, make a definition that eliminates this parameter and
  guarantees no accidental interaction between the replication
  machinery and the process being replicated -- i.e. no accidental
  sharing of names used by the process to get its work done and the
  name(s) used by the replication to effect copying. This latter
  revision of the definition of replication is crucial to obtaining
  the expected identity $!!P \sim !P$.
\end{remark}

\begin{remark}\label{rem:paradoxical_combinator}
  The reader familiar with the lambda calculus will have noticed the
  similarity between $D$ and the paradoxical combinator.

  [Ed. note: the existence of this seems to suggest we have to be more
  restrictive on the set of processes and names we admit if we are to
  support no-cloning.]
\end{remark}

\subsubsection{Bisimulation}

The computational dynamics gives rise to another kind of equivalence,
the equivalence of computational behavior. As previously mentioned
this is typically captured \emph{via} some form of bisimulation.

% The notion we use in this paper is weak barbed bisimulation
% \cite{milner91polyadicpi}.

The notion we use in this paper is derived from weak barbed
bisimulation \cite{milner91polyadicpi}. 

\begin{definition}
An \emph{observation relation}, $\downarrow_{\mathcal N}$, over a set
of names, $\mathcal N$, is the smallest relation satisfying the rules
below.

\infrule[Out-barb]{y \in {\mathcal N}, \; x \nameeq y}
		  {\outputp{x}{v} \downarrow_{\mathcal N} x}
\infrule[Par-barb]{\mbox{$P\downarrow_{\mathcal N} x$ or $Q\downarrow_{\mathcal N} x$}}
		  {\binpar{P}{Q} \downarrow_{\mathcal N} x}

We write $P \Downarrow_{\mathcal N} x$ if there is $Q$ such that 
$P \wred Q$ and $Q \downarrow_{\mathcal N} x$.
\end{definition}

\begin{definition}
%\label{def.bbisim}
An  ${\mathcal N}$-\emph{barbed bisimulation} over a set of names, ${\mathcal N}$, is a symmetric binary relation 
${\mathcal S}_{\mathcal N}$ between agents such that $P\rel{S}_{\mathcal N}Q$ implies:
\begin{enumerate}
\item If $P \red P'$ then $Q \wred Q'$ and $P'\rel{S}_{\mathcal N} Q'$.
\item If $P\downarrow_{\mathcal N} x$, then $Q\Downarrow_{\mathcal N} x$.
\end{enumerate}
$P$ is ${\mathcal N}$-barbed bisimilar to $Q$, written
$P \wbbisim_{\mathcal N} Q$, if $P \rel{S}_{\mathcal N} Q$ for some ${\mathcal N}$-barbed bisimulation ${\mathcal S}_{\mathcal N}$.
\end{definition}

$\mathcal{R} \subseteq \pi \times \pi$

$P \mathcal{R} Q => \forall P'. P \red P' \Rightarrow \exists Q'. Q \red Q', P' \mathcal{R} Q'$

$P \vdash x \Rightarrow Q \vdash x$

\begin{mathpar}
  \inferrule*[lab=Out-barb]{x \nameeq y}{{y}!\langle{Q}\rangle \vdash x}
  \and
  \inferrule*[lab=Par-barb]{\mbox{$P\vdash x$ or $Q\vdash x$}}{\binpar{P}{Q} \vdash x}
\end{mathpar}

\subsubsection{Contexts}

One of the principle advantages of computational calculi like the
$\pi$-calculus is a well-defined notion of context,
contextual-equivalence and a correlation between
contextual-equivalence and notions of bisimulation. The notion of
context allows the decomposition of a process into (sub-)process and
its syntactic environment, its context. Thus, a context may be
thought of as a process with a ``hole'' (written $\Box$) in it. The
application of a context $M$ to a process $P$, written $M[P]$, is
tantamount to filling the hole in $M$ with $P$. In this paper we do
not need the full weight of this theory, but do make use of the notion
of context in the proof the main theorem. 

\begin{mathpar}
  \inferrule* [lab=summation] {} {{M_{M},M_{N}} \bc \Box \;|\; x.M_{A} \;|\; M_{M}+M_{N}}
  \and
  \inferrule* [lab=agent] {} {{M_{A}} \bc (\vec{x})M_{P} \;| \; \clift{P_0,\ldots,M_{P},\ldots,P_N}}
  \and \\
  \inferrule* [lab=process] {} {{M_{P}} \bc M_{N} \;| \;P|M_{P} }
\end{mathpar} 

\begin{mathpar}
  \inferrule* [lab=sychronization] {} {M_{N} \bc \Box \;|\; x?M_{F} \;|\; x!M_{C}}
  \and
  \inferrule* [lab=abstraction] {} {{M_{F}} \bc (x)M_{P} }
  \and
  \inferrule* [lab=concretion] {} {{M_{C}} \bc \langle M_{P} \rangle }
  \and \\
  \inferrule* [lab=process] {} {{M_{P}} \bc M_{N} \;| \;P|M_{P} }
\end{mathpar}

\begin{definition}[contextual application] Given a context $M$, and
  process $P$, we define the \emph{contextual application}, $M[P] :=
  M\{P/\Box\}$. That is, the contextual application of M to P is the
  substitution of $P$ for $\Box$ in $M$.
\end{definition}

$\meaningof{-} : L \to \mathcal{P}(\pi)$

\begin{mathpar}
  \inferrule* [lab=collection] {} {\meaningof{true} = \pi, \and \meaningof{~E} = \pi \setminus \meaningof{E}, \and \meaningof{E_{1} \& E_{2}} = \meaningof{E_{1}} \cap \meaningof{E_{2}}}
\end{mathpar}

\begin{mathpar}
  \inferrule* [lab=structure] {} {\meaningof{0} = \{ P \in \pi | P \equiv 0 \}, \and \\ \meaningof{E_1 | E_2} = \{ P \in \pi | P \equiv P_{1} | P_{2}, P_{1} \in \meaningof{E_{1}}, P_{2} \in \meaningof{E_2}\} }
\end{mathpar}

\begin{mathpar}
 \inferrule* [lab=behavior] {} {\meaningof{\langle a?b \rangle E} = \{ P \in \pi | P \equiv Q | u?(y)P', \\ \and \\\\ \and \\ \;\;\; u \in \meaningof{a}, \forall z.P'\{z/y\} \in \meaningof{E\{z/b\}}\}, \and \\ \meaningof{a!E} = \{ P \in \pi | P \equiv Q | x!\langle P' \rangle, x \in \meaningof{a} P' \in \meaningof{E}\} }
\end{mathpar}

\begin{mathpar}
 \inferrule* [lab=nominal] {} {\meaningof{\quotep{E}} = \{ \quotep{P} \in \quotep{\pi} | P \in \meaningof{E} \}, \and \meaningof{\quotep{P}} = \{ \quotep{Q} \in \quotep{\pi} | P \equiv Q \} \and \\ \meaningof{@\quotep{E}} = \{ P \in \pi | P \equiv @x, x \in \meaningof{E} \}}
\end{mathpar}

\begin{eqnarray*}
  \\
  \meaningof{-} : TS \to ST
\end{eqnarray*}

\begin{eqnarray*}
  \\
  L : TS \to ST
\end{eqnarray*}

\begin{eqnarray*}
  \\
  P \models E \iff P \in \meaningof{E}
\end{eqnarray*}

\begin{eqnarray*}
  P \approx_{L} Q \iff \forall E \in L. P \models E \iff Q \models E
\end{eqnarray*}

\begin{eqnarray*}
  P \approx_{K} Q
\end{eqnarray*}

\begin{eqnarray*}
  P \approx Q
\end{eqnarray*}

$\approx_{K} = \approx = \approx_{L}$

\subsubsection{Contextual duality}

Note that contexts extend the quotation operation to a family of
operations from processes to names. Given a context, $M$, we can
define a \emph{nominal context}, $\quotep{M}$ by $\quotep{M}[P] :=
\quotep{M[P]}$. To foreshadow what is to come we observe that these
operations enjoy a duality with processes very much like the duality
between vectors and maps from vectors to scalars.

Further, because the calculus is essentially higher-order, we have a
correspondence between contexts and processes. More specifically,
given a name $x$ and a context $M$ we can construct $M^{*}_{x}$ such
that 

\begin{mathpar}
  M^{*}_{x} | \lift{x}{P} \red M[P]
\end{mathpar}

namely,

\begin{mathpar}
  M^{*}_{x} := x?(u).M[\dropn{u}]
\end{mathpar}

The dependence of $M^{*}_{x}$ on a name makes it an abstraction, 

\begin{mathpar}
  M^{*} := (x)x?(u).M[\dropn{u}]
\end{mathpar}

\subsection{Additional notation}

It will sometimes be convenient to denote the process a name
quotes. We already have the notation $x = \quotep{P}$, but it will be
convenient to introduce an alternate notation, $\procn{x}$, when we
want to emphasize the connection to the use of the name. Note that, by
virtue of name equivalence, $\quotep{\procn{x}} \nameeq x$; so, the
notation is consistent with previous definitions.

Further, because names have structure it is possible to effect
substitutions on the basis of that structure. This means we need to
upgrade our notation for substitutions, which we accomplish by
adapting comprehension notation. Thus,

\begin{mathpar}
  P\{ y / x : x \in S \}
\end{mathpar}

is interpreted to mean the process derived from P by replacing (in a
capture-avoiding manner) each occurrence of $x$ in $S$ by $y$. For example,

\begin{mathpar}
  P\{ \quotep{\procn{x}|\procn{x}} / x : x \in \freenames{P} \}
\end{mathpar}

will replace each (occurrence) of a free name $x$ in $P$ by
$\quotep{\procn{x}|\procn{x}}$.

Also, we will avail ourselves of the notation $x^{L}$ and $x^{R}$ to
denote injections of a name into disjoint copies of the name
space. There are numerous ways to accomplish this. One example can be
found in \cite{MeredithR05}. This notation overloads to vectors of
names: $\vec{x}^{\pi} := (x_{i}^{\pi} \; : \; 0 \leq i < |\vec{x}| )$ where $\pi \in \{L,R\}$.

We also use $P^{\Box} := P|\Box$.

In \cite{MeredithR05} an interpretation of the new operator is
given. It turns out that there are several possible interpretations
all enjoying the requisite algebraic properties of the operator (see
\cite{milner91polyadicpi}). We will therefore make liberal use of
$(\nu\; \vec{x})P$.

% subsection the_syntax_and_semantics_of_the_notation_system (end)   

\input{qm2pi.qmops} 

\input{qm2pi.sterngerlach} 

\input{qm2pi.metric} 

% section concurrent_process_calculi (end)

%\input{qm2pi.proofsketch}

% section proof sketch (end)

%\input{qm2pi.slviaknots} 

% section spatial logic via knots (end)

\input{qm2pi.conclusion}

% section conclusion (end)

%\input{qm2pi.dtcodes} 

% section wiring algorithm (end)

\input{qm2pi.ack} 

% section acknowledgments (end)

\newpage


\bibliographystyle{plain}   
\bibliography{../../biblios/main.bib}

\input{qm2pi.rhodetails}

\end{document}

 

% section notation (end)

\input{qm2pi.process.calculi} 

% section concurrent_process_calculi_and_spatial_logics_ (end)
    
%\documentclass[12pt]{llncs}
%\documentclass{jktr}

\usepackage[pdftex]{hyperref}                   
\usepackage {listings}
\usepackage {mathpartir}
\usepackage{bcprules}
%\usepackage{listings}
                       
\usepackage{graphicx} 
%\usepackage[margins=2.5cm,nohead,nofoot]{geometry}
%\usepackage{geometry}
\usepackage{amsfonts}
\usepackage{amstext}
\usepackage{latexsym}
\usepackage{amssymb}
\usepackage{color}


%\include{myPreamble}
\include{qm2pi.local} 

%\ifpdf
%\usepackage[pdftex]{graphicx}
%\else
%\usepackage{graphicx}
%\fi

 % \ifpdf
%  \usepackage{pdfsync}
%  \if


%\title{Brief Article}
%\author{David F. Snyder}
%\author{L.G. Meredith}

%\address{Dept. of Math., Texas State University--San Marcos, San Marcos, TX 78666}
       
\pagestyle{empty}


\begin{document}

\lstset{language=[Objective]Caml,frame=shadowbox}

\input{qm2pi.front}

% section front matter (end)

\input{qm2pi.intro} 
 
% section introduction (end)

% \input{qm2pi.knotations} 

% section notation (end)

\input{qm2pi.process.calculi} 

% section concurrent_process_calculi_and_spatial_logics_ (end)
    
%\input{qm2pi.knots2pi} 

%\input{qm2pi.trefoil} 

%\input{qm2pi.mainthm} 

% subsection basic_interpretation (end)

%\input{qm2pi.rho.presentation} 
\subsection{The syntax and semantics of the notation system}\label{sub:the_syntax_and_semantics_of_the_notation_system} % (fold)

We now summarize a technical presentation of the calculus that
embodies our theory of dynamics. The typical presentation of such a
calculus follows the style of giving generators and relations on
them. The grammar, below, describing term constructors, freely
generates the set of processes, $\Proc$. This set is then quotiented
by a relation known as structural congruence and it is over this set
that the notion of dynamics is expressed. This presentation is
essentially that of \cite{MeredithR05} with the addition of
polyadicity and summation. For readability we have relegated some of
the technical subtleties to an appendix.

\subsubsection{Process grammar}\label{subsub:process_grammar}

\begin{mathpar}
  \inferrule* [lab=synchronization] {} {{M} \bc \pzero \;|\; x?F \;|\; x!C }
  \and
  \inferrule* [lab=abstraction] {} {{F} \bc (x)P}
  \and
  \inferrule* [lab=concretion] {} {{C} \bc \langle Q \rangle}
  \and
  \inferrule* [lab=process] {} {{P,Q} \bc M \;| \;P|Q \;|\; @{x}}
  \and
  \inferrule* [lab=name] {} {{x} \bc \quotep{P}}
\end{mathpar} 

Note that $\vec{x}$ (resp. $\vec{P}$) denotes a vector of names
(resp. processes) of length $|\vec{x}|$ (resp. $|\vec{P}|$). We adopt
the following useful abbreviations.

\begin{mathpar}
   x?(\vec{y}).P := x.(\vec{y})P \and  x\clift{\vec{P}} := x.\clift{\vec{P}}
   \and x!(y) := \lift{x}{\dropn{y}}
   \and \Pi_{i=0}^{n-1}P_i := P_0 | \ldots | P_{n-1}
\end{mathpar}

\subsubsection{Structural congruence}

\paragraph{Free and bound names and alpha-equivalence.} At the
core of structural equivalence is alpha-equivalence which identifies
process that are the same up to a change of variable. Formally, we
recognize the distinction between free and bound names. The free names
of a process, $\freenames{P}$, may be calculated recursively as
follows:

\begin{mathpar}
\freenames{\pzero} := \emptyset
  \and \\
  \freenames{x?(y).P} := \{ x \} \cup (\freenames{P} \setminus \{ y \})
  \and 
  \freenames{x!\langle P \rangle} := \{ x \} \cup \{ P \} 
  \and \\
  \freenames{P|Q} := \freenames{P} \cup \freenames{Q}
  \and \\
  \freenames{@{x}} := \{ x \}
\end{mathpar}

$\pi$
$\quotep{\pi}$

$\freenames{-} : \pi \to \mathcal{P}(\quotep{\pi})$

\begin{eqnarray*}
  \freenames{\pzero} & := & \emptyset \\
  \freenames{x?(y).P} & := & \{ x \} \cup (\freenames{P} \setminus \{ y \}) \\
  \freenames{x!\langle P \rangle} & := & \{ x \} \cup \{ P \} \\
  \freenames{P|Q} & := & \freenames{P} \cup \freenames{Q} \\
  \freenames{\dropn{x}} & := & \{ x \}
\end{eqnarray*}

The bound names of a process, $\boundnames{P}$, are those names occurring in $P$
that are not free. For example, in $x?(y).0$, the name $x$ is free, while $y$ is bound.

\begin{mathpar}
  \inferrule* [lab=monoidal-laws] {} { P|Q \equiv Q|P \and P|0 \equiv P \and P|(Q|R) \equiv (P|Q)|R }
\end{mathpar}

\begin{mathpar}
  \inferrule* [lab=alpha-equivalence] {} { (x)P \equiv (y)P\{y/x\} \and y \not\in \freenames{P} }
\end{mathpar}

\begin{definition}
Then two processes, $P,Q$, are alpha-equivalent if $P = Q\{\vec{y}/\vec{x}\}$ for
some $\vec{x} \in \boundnames{Q},\vec{y} \in \boundnames{P}$, where $Q\{\vec{y}/\vec{x}\}$
denotes the capture-avoiding substitution of $\vec{y}$ for $\vec{x}$ in $Q$.
\end{definition}

\begin{definition}
  The {\em structural congruence} \cite{SangiorgiWalker} , $\equiv$,
  between processes is the least congruence containing
  alpha-equivalence, satisfying the abelian monoid laws
  (associativity, commutativity and $\pzero$ as identity) for parallel
  composition $|$ and for summation $+$.
\end{definition}

\subsection{Name equivalence}

We take name equivalence, written $\nameeq$, to be the smallest
equivalence relation generated by the following rules.

\begin{mathpar}
\inferrule*[lab=Quote-drop]
{ }
{ \quotep{@{x}} \nameeq x }

\inferrule*[lab=Struct-equiv]
{ P \scong Q }
{ \quotep{P} \nameeq \quotep{Q} }
\end{mathpar}

The astute reader will have noticed that the mutual recursion of names
and processes imposes a mutual recursion on alpha-equivalence and
structural equivalence via name-equivalence. Fortunately, all of this
works out pleasantly and we may calculate in the natural way, free of
concern. The reader interested in the details is referred to the
appendix \ref{appendix:rho_details}.

\subsection{Substitution}

We use $\Proc$ for the set of processes, $\QProc$ for the set of
names, and $\id{\{}\vec{y} / \vec{x} \id{\}}$ to denote partial maps,
$s : \QProc \rightarrow \QProc$. A map, $s$ lifts, uniquely, to a map
on process terms, $\widehat{s} : \Proc \rightarrow \Proc$ by the
following equations.

\begin{mathpar}
  (0) \psubstp{Q}{P} := 0 \\
  (R \juxtap S) \psubstp{Q}{P}
  :=    
  (R)\psubstp{Q}{P} \juxtap (S) \psubstp{Q}{P} \\
  (x?(y).R) \psubstp{Q}{P}    
  :=    
  (x)\substp{Q}{P} (z)\concat( (R \psubstn{z}{y}) \psubstp{Q}{P} ) \\
  (\lift{x}{R}) \psubstp{Q}{P}  
  :=
  \lift{(x)\substp{Q}{P}}{ R \psubstp{Q}{P} } \\
%   (\dropn{x})  \psubstp{Q}{P}       
%   := 
%   \left\{ 
%     \begin{array}{ccc} 
%       \dropn{\quotep{Q}} & & x \nameeq \quotep{P} \\
%       \dropn{x} & & otherwise \\
%     \end{array}
%   \right. 
  (\dropn{x})  \psubstp{Q}{P}       
  := 
  \left\{ 
    \begin{array}{ccc} 
      Q & & x \nameeq \quotep{P} \\
      \dropn{x} & & otherwise \\
    \end{array}
  \right.
\end{mathpar}
 

where

\begin{eqnarray}
  (x)\id{\{} \lpquote Q \rpquote / \lpquote P \rpquote \id{\}}            = 
  \left\{ 
    \begin{array}{ccc}
      \lpquote Q \rpquote & & x \nameeq \lpquote P \rpquote \\
      x & & otherwise \\
    \end{array}
  \right. \nonumber
\end{eqnarray}

and $z$ is chosen distinct from $\quotep{P}$, $\quotep{Q}$, the free
names in $Q$, and all the names in $R$. Our $\alpha$-equivalence will
be built in the standard way from this substitution.

\begin{remark}\label{rem:no_self_referential_names}
  One consequence of these definitions is that $\forall P. \quotep{P}
  \not\in \freenames{P}$.
\end{remark}

\subsection{ Dynamic quote: an example }

Anticipating something of what's to come, consider applying the
substitution, $\widehat{\id{\{}u / z \id{\}}}$, to the following pair
of processes, $\lift{w}{y!(z)}$ and $w[ \lpquote y!(z) \rpquote ]$.

\begin{eqnarray}
	\lift{w}{y!(z)}\widehat{\id{\{}u / z \id{\}}}
		& = &
		\lift{w}{y!(u)} \nonumber\\
	w[ \lpquote y!(z) \rpquote ] \widehat{ \id{\{}u / z \id{\}} }
		& = &
		w[ \lpquote y!(z) \rpquote ] \nonumber
\end{eqnarray}

Because the body of the process between quotes is impervious to
substitution, we get radically different answers. In fact, by
examining the first process in an input context,
e.g. $x?(z).\lift{w}{y!(z)}$, we see that the process under the lift
operator may be shaped by prefixed inputs binding a name inside it. In
this sense, the lift operator will be seen as a way to dynamically
construct processes before reifying them as names.

Finally equipped with these standard features we can present the
dynamics of the calculus.

\subsubsection{Operational semantics} 

Finally, we introduce the computational dynamics. What marks these
algebras as distinct from other more traditionally studied algebraic
structures, e.g. vector spaces or polynomial rings, is the manner in
which dynamics is captured. In traditional structures, dynamics is typically
expressed through morphisms between such structures, as in linear maps
between vector spaces or morphisms between rings. In algebras
associated with the semantics of computation, the dynamics is
expressed as part of the algebraic structure itself, through a
reduction reduction relation typically denoted by $\red$. Below, we
give a recursive presentation of this relation for the calculus used
in the encoding.

$\red \subseteq \pi \times \pi$
$\red : \pi \to \mathcal{P}(\pi)$

\begin{mathpar}
  \inferrule* [lab=Comm] { \textsf{match}( x_{src}, x_{trgt} ) } { x_{trgt}?(y)P \; | \; x_{src}!\langle {Q} \rangle \red P\{\quotep{Q}/y}\} }
  \and \\
  \inferrule* [lab=Par] {{P} \red {P}'} {{{P} | {Q}} \red {{P}' | {Q}}}
  \and
  \inferrule* [lab=Equiv]{{{P} \scong {P}'} \andalso {{P}' \red {Q}'} \andalso {{Q}' \scong {Q}}}{{P} \red {Q}}
\end{mathpar}

\begin{eqnarray*}
  match_{\equiv} (\quotep{P},\quotep{Q}) & := & P \equiv Q \\
  match_{\dagger}(\quotep{P},\quotep{Q}) & := & \forall R. P|Q \red^{*} R => R \red^{*} 0 \\
  match_{K}(\quotep{P},\quotep{Q}) & := & K \mbox{ for some context } K
\end{eqnarray*}

$u?(x)P | u!\langle Q \rangle \red P\{\quotep{Q}/x\}$

%We write $\wred$ for $\red^*$, and $P\red$ if $\exists Q $ such that $ P \red Q$.
We write $P\red$ if $\exists Q $ such that $ P \red Q$ and $P\not\red$, otherwise.

\section{Replication}

As mentioned before, it is known that replication (and hence
recursion) can be implemented in a higher-order process algebra
\cite{SangiorgiWalker}. As our first example of calculation with the
machinery thus far presented we give the construction explicitly in
the {\rhoc}.

\begin{eqnarray}
	D_{x} & := & \prefix{x}{y}{(\binpar{\outputp{x}{y}}{@{y}})} \nonumber\\
	\bangp_{x}{P} & := & \binpar{{x}!\langle{\binpar{D_{x}}{P}}\rangle}{D_{x}} \nonumber
\end{eqnarray}

\begin{eqnarray}
	\bangp_{x}{P} & & \nonumber\\
	=
	& {x}!\langle{(\prefix{x}{y}{(\outputp{x}{y} | @{y})) | P}}\rangle 
	      | \prefix{x}{y}{(\outputp{x}{y} | @{y})} & \nonumber\\
	\red
	& (\outputp{x}{y} | @{y})\substn{\quotep{(\prefix{x}{y}{(@{y} | \outputp{x}{y})) | P}}}{y} & \nonumber\\
	=
	& \outputp{x}{\quotep{(\prefix{x}{y}{(\outputp{x}{y} | @{y})) | P}}}
	  | {(\prefix{x}{y}{(\outputp{x}{y} | @{y})) | P}} & \nonumber\\
	\red
	& \ldots & \nonumber\\
	\red^*
	& P | P | \ldots & \nonumber
\end{eqnarray}

Of course, this encoding, as an implementation, runs away, unfolding
$\bangp{P}$ eagerly. A lazier and more implementable replication
operator, restricted to input-guarded processes, may be obtained as follows.

\begin{eqnarray}
\bangp{\prefix{u}{v}{P}} 
	:= 
	\binpar{\lift{x}{\prefix{u}{v}{(\binpar{D(x)}{P})}}}{D(x)} \nonumber
\end{eqnarray}

\begin{remark}
  Note that the lazier definition still does not deal with summation
  or mixed summation (i.e. sums over input and output). The reader is
  invited to construct definitions of replication that deal with these
  features. 

  Further, the definitions are parameterized in a name, $x$. Can you,
  gentle reader, make a definition that eliminates this parameter and
  guarantees no accidental interaction between the replication
  machinery and the process being replicated -- i.e. no accidental
  sharing of names used by the process to get its work done and the
  name(s) used by the replication to effect copying. This latter
  revision of the definition of replication is crucial to obtaining
  the expected identity $!!P \sim !P$.
\end{remark}

\begin{remark}\label{rem:paradoxical_combinator}
  The reader familiar with the lambda calculus will have noticed the
  similarity between $D$ and the paradoxical combinator.

  [Ed. note: the existence of this seems to suggest we have to be more
  restrictive on the set of processes and names we admit if we are to
  support no-cloning.]
\end{remark}

\subsubsection{Bisimulation}

The computational dynamics gives rise to another kind of equivalence,
the equivalence of computational behavior. As previously mentioned
this is typically captured \emph{via} some form of bisimulation.

% The notion we use in this paper is weak barbed bisimulation
% \cite{milner91polyadicpi}.

The notion we use in this paper is derived from weak barbed
bisimulation \cite{milner91polyadicpi}. 

\begin{definition}
An \emph{observation relation}, $\downarrow_{\mathcal N}$, over a set
of names, $\mathcal N$, is the smallest relation satisfying the rules
below.

\infrule[Out-barb]{y \in {\mathcal N}, \; x \nameeq y}
		  {\outputp{x}{v} \downarrow_{\mathcal N} x}
\infrule[Par-barb]{\mbox{$P\downarrow_{\mathcal N} x$ or $Q\downarrow_{\mathcal N} x$}}
		  {\binpar{P}{Q} \downarrow_{\mathcal N} x}

We write $P \Downarrow_{\mathcal N} x$ if there is $Q$ such that 
$P \wred Q$ and $Q \downarrow_{\mathcal N} x$.
\end{definition}

\begin{definition}
%\label{def.bbisim}
An  ${\mathcal N}$-\emph{barbed bisimulation} over a set of names, ${\mathcal N}$, is a symmetric binary relation 
${\mathcal S}_{\mathcal N}$ between agents such that $P\rel{S}_{\mathcal N}Q$ implies:
\begin{enumerate}
\item If $P \red P'$ then $Q \wred Q'$ and $P'\rel{S}_{\mathcal N} Q'$.
\item If $P\downarrow_{\mathcal N} x$, then $Q\Downarrow_{\mathcal N} x$.
\end{enumerate}
$P$ is ${\mathcal N}$-barbed bisimilar to $Q$, written
$P \wbbisim_{\mathcal N} Q$, if $P \rel{S}_{\mathcal N} Q$ for some ${\mathcal N}$-barbed bisimulation ${\mathcal S}_{\mathcal N}$.
\end{definition}

$\mathcal{R} \subseteq \pi \times \pi$

$P \mathcal{R} Q => \forall P'. P \red P' \Rightarrow \exists Q'. Q \red Q', P' \mathcal{R} Q'$

$P \vdash x \Rightarrow Q \vdash x$

\begin{mathpar}
  \inferrule*[lab=Out-barb]{x \nameeq y}{{y}!\langle{Q}\rangle \vdash x}
  \and
  \inferrule*[lab=Par-barb]{\mbox{$P\vdash x$ or $Q\vdash x$}}{\binpar{P}{Q} \vdash x}
\end{mathpar}

\subsubsection{Contexts}

One of the principle advantages of computational calculi like the
$\pi$-calculus is a well-defined notion of context,
contextual-equivalence and a correlation between
contextual-equivalence and notions of bisimulation. The notion of
context allows the decomposition of a process into (sub-)process and
its syntactic environment, its context. Thus, a context may be
thought of as a process with a ``hole'' (written $\Box$) in it. The
application of a context $M$ to a process $P$, written $M[P]$, is
tantamount to filling the hole in $M$ with $P$. In this paper we do
not need the full weight of this theory, but do make use of the notion
of context in the proof the main theorem. 

\begin{mathpar}
  \inferrule* [lab=summation] {} {{M_{M},M_{N}} \bc \Box \;|\; x.M_{A} \;|\; M_{M}+M_{N}}
  \and
  \inferrule* [lab=agent] {} {{M_{A}} \bc (\vec{x})M_{P} \;| \; \clift{P_0,\ldots,M_{P},\ldots,P_N}}
  \and \\
  \inferrule* [lab=process] {} {{M_{P}} \bc M_{N} \;| \;P|M_{P} }
\end{mathpar} 

\begin{mathpar}
  \inferrule* [lab=sychronization] {} {M_{N} \bc \Box \;|\; x?M_{F} \;|\; x!M_{C}}
  \and
  \inferrule* [lab=abstraction] {} {{M_{F}} \bc (x)M_{P} }
  \and
  \inferrule* [lab=concretion] {} {{M_{C}} \bc \langle M_{P} \rangle }
  \and \\
  \inferrule* [lab=process] {} {{M_{P}} \bc M_{N} \;| \;P|M_{P} }
\end{mathpar}

\begin{definition}[contextual application] Given a context $M$, and
  process $P$, we define the \emph{contextual application}, $M[P] :=
  M\{P/\Box\}$. That is, the contextual application of M to P is the
  substitution of $P$ for $\Box$ in $M$.
\end{definition}

$\meaningof{-} : L \to \mathcal{P}(\pi)$

\begin{mathpar}
  \inferrule* [lab=collection] {} {\meaningof{true} = \pi, \and \meaningof{~E} = \pi \setminus \meaningof{E}, \and \meaningof{E_{1} \& E_{2}} = \meaningof{E_{1}} \cap \meaningof{E_{2}}}
\end{mathpar}

\begin{mathpar}
  \inferrule* [lab=structure] {} {\meaningof{0} = \{ P \in \pi | P \equiv 0 \}, \and \\ \meaningof{E_1 | E_2} = \{ P \in \pi | P \equiv P_{1} | P_{2}, P_{1} \in \meaningof{E_{1}}, P_{2} \in \meaningof{E_2}\} }
\end{mathpar}

\begin{mathpar}
 \inferrule* [lab=behavior] {} {\meaningof{\langle a?b \rangle E} = \{ P \in \pi | P \equiv Q | u?(y)P', \\ \and \\\\ \and \\ \;\;\; u \in \meaningof{a}, \forall z.P'\{z/y\} \in \meaningof{E\{z/b\}}\}, \and \\ \meaningof{a!E} = \{ P \in \pi | P \equiv Q | x!\langle P' \rangle, x \in \meaningof{a} P' \in \meaningof{E}\} }
\end{mathpar}

\begin{mathpar}
 \inferrule* [lab=nominal] {} {\meaningof{\quotep{E}} = \{ \quotep{P} \in \quotep{\pi} | P \in \meaningof{E} \}, \and \meaningof{\quotep{P}} = \{ \quotep{Q} \in \quotep{\pi} | P \equiv Q \} \and \\ \meaningof{@\quotep{E}} = \{ P \in \pi | P \equiv @x, x \in \meaningof{E} \}}
\end{mathpar}

\begin{eqnarray*}
  \\
  \meaningof{-} : TS \to ST
\end{eqnarray*}

\begin{eqnarray*}
  \\
  L : TS \to ST
\end{eqnarray*}

\begin{eqnarray*}
  \\
  P \models E \iff P \in \meaningof{E}
\end{eqnarray*}

\begin{eqnarray*}
  P \approx_{L} Q \iff \forall E \in L. P \models E \iff Q \models E
\end{eqnarray*}

\begin{eqnarray*}
  P \approx_{K} Q
\end{eqnarray*}

\begin{eqnarray*}
  P \approx Q
\end{eqnarray*}

$\approx_{K} = \approx = \approx_{L}$

\subsubsection{Contextual duality}

Note that contexts extend the quotation operation to a family of
operations from processes to names. Given a context, $M$, we can
define a \emph{nominal context}, $\quotep{M}$ by $\quotep{M}[P] :=
\quotep{M[P]}$. To foreshadow what is to come we observe that these
operations enjoy a duality with processes very much like the duality
between vectors and maps from vectors to scalars.

Further, because the calculus is essentially higher-order, we have a
correspondence between contexts and processes. More specifically,
given a name $x$ and a context $M$ we can construct $M^{*}_{x}$ such
that 

\begin{mathpar}
  M^{*}_{x} | \lift{x}{P} \red M[P]
\end{mathpar}

namely,

\begin{mathpar}
  M^{*}_{x} := x?(u).M[\dropn{u}]
\end{mathpar}

The dependence of $M^{*}_{x}$ on a name makes it an abstraction, 

\begin{mathpar}
  M^{*} := (x)x?(u).M[\dropn{u}]
\end{mathpar}

\subsection{Additional notation}

It will sometimes be convenient to denote the process a name
quotes. We already have the notation $x = \quotep{P}$, but it will be
convenient to introduce an alternate notation, $\procn{x}$, when we
want to emphasize the connection to the use of the name. Note that, by
virtue of name equivalence, $\quotep{\procn{x}} \nameeq x$; so, the
notation is consistent with previous definitions.

Further, because names have structure it is possible to effect
substitutions on the basis of that structure. This means we need to
upgrade our notation for substitutions, which we accomplish by
adapting comprehension notation. Thus,

\begin{mathpar}
  P\{ y / x : x \in S \}
\end{mathpar}

is interpreted to mean the process derived from P by replacing (in a
capture-avoiding manner) each occurrence of $x$ in $S$ by $y$. For example,

\begin{mathpar}
  P\{ \quotep{\procn{x}|\procn{x}} / x : x \in \freenames{P} \}
\end{mathpar}

will replace each (occurrence) of a free name $x$ in $P$ by
$\quotep{\procn{x}|\procn{x}}$.

Also, we will avail ourselves of the notation $x^{L}$ and $x^{R}$ to
denote injections of a name into disjoint copies of the name
space. There are numerous ways to accomplish this. One example can be
found in \cite{MeredithR05}. This notation overloads to vectors of
names: $\vec{x}^{\pi} := (x_{i}^{\pi} \; : \; 0 \leq i < |\vec{x}| )$ where $\pi \in \{L,R\}$.

We also use $P^{\Box} := P|\Box$.

In \cite{MeredithR05} an interpretation of the new operator is
given. It turns out that there are several possible interpretations
all enjoying the requisite algebraic properties of the operator (see
\cite{milner91polyadicpi}). We will therefore make liberal use of
$(\nu\; \vec{x})P$.

% subsection the_syntax_and_semantics_of_the_notation_system (end)   

\input{qm2pi.qmops} 

\input{qm2pi.sterngerlach} 

\input{qm2pi.metric} 

% section concurrent_process_calculi (end)

%\input{qm2pi.proofsketch}

% section proof sketch (end)

%\input{qm2pi.slviaknots} 

% section spatial logic via knots (end)

\input{qm2pi.conclusion}

% section conclusion (end)

%\input{qm2pi.dtcodes} 

% section wiring algorithm (end)

\input{qm2pi.ack} 

% section acknowledgments (end)

\newpage


\bibliographystyle{plain}   
\bibliography{../../biblios/main.bib}

\input{qm2pi.rhodetails}

\end{document}

 

%\documentclass[12pt]{llncs}
%\documentclass{jktr}

\usepackage[pdftex]{hyperref}                   
\usepackage {listings}
\usepackage {mathpartir}
\usepackage{bcprules}
%\usepackage{listings}
                       
\usepackage{graphicx} 
%\usepackage[margins=2.5cm,nohead,nofoot]{geometry}
%\usepackage{geometry}
\usepackage{amsfonts}
\usepackage{amstext}
\usepackage{latexsym}
\usepackage{amssymb}
\usepackage{color}


%\include{myPreamble}
\include{qm2pi.local} 

%\ifpdf
%\usepackage[pdftex]{graphicx}
%\else
%\usepackage{graphicx}
%\fi

 % \ifpdf
%  \usepackage{pdfsync}
%  \if


%\title{Brief Article}
%\author{David F. Snyder}
%\author{L.G. Meredith}

%\address{Dept. of Math., Texas State University--San Marcos, San Marcos, TX 78666}
       
\pagestyle{empty}


\begin{document}

\lstset{language=[Objective]Caml,frame=shadowbox}

\input{qm2pi.front}

% section front matter (end)

\input{qm2pi.intro} 
 
% section introduction (end)

% \input{qm2pi.knotations} 

% section notation (end)

\input{qm2pi.process.calculi} 

% section concurrent_process_calculi_and_spatial_logics_ (end)
    
%\input{qm2pi.knots2pi} 

%\input{qm2pi.trefoil} 

%\input{qm2pi.mainthm} 

% subsection basic_interpretation (end)

%\input{qm2pi.rho.presentation} 
\subsection{The syntax and semantics of the notation system}\label{sub:the_syntax_and_semantics_of_the_notation_system} % (fold)

We now summarize a technical presentation of the calculus that
embodies our theory of dynamics. The typical presentation of such a
calculus follows the style of giving generators and relations on
them. The grammar, below, describing term constructors, freely
generates the set of processes, $\Proc$. This set is then quotiented
by a relation known as structural congruence and it is over this set
that the notion of dynamics is expressed. This presentation is
essentially that of \cite{MeredithR05} with the addition of
polyadicity and summation. For readability we have relegated some of
the technical subtleties to an appendix.

\subsubsection{Process grammar}\label{subsub:process_grammar}

\begin{mathpar}
  \inferrule* [lab=synchronization] {} {{M} \bc \pzero \;|\; x?F \;|\; x!C }
  \and
  \inferrule* [lab=abstraction] {} {{F} \bc (x)P}
  \and
  \inferrule* [lab=concretion] {} {{C} \bc \langle Q \rangle}
  \and
  \inferrule* [lab=process] {} {{P,Q} \bc M \;| \;P|Q \;|\; @{x}}
  \and
  \inferrule* [lab=name] {} {{x} \bc \quotep{P}}
\end{mathpar} 

Note that $\vec{x}$ (resp. $\vec{P}$) denotes a vector of names
(resp. processes) of length $|\vec{x}|$ (resp. $|\vec{P}|$). We adopt
the following useful abbreviations.

\begin{mathpar}
   x?(\vec{y}).P := x.(\vec{y})P \and  x\clift{\vec{P}} := x.\clift{\vec{P}}
   \and x!(y) := \lift{x}{\dropn{y}}
   \and \Pi_{i=0}^{n-1}P_i := P_0 | \ldots | P_{n-1}
\end{mathpar}

\subsubsection{Structural congruence}

\paragraph{Free and bound names and alpha-equivalence.} At the
core of structural equivalence is alpha-equivalence which identifies
process that are the same up to a change of variable. Formally, we
recognize the distinction between free and bound names. The free names
of a process, $\freenames{P}$, may be calculated recursively as
follows:

\begin{mathpar}
\freenames{\pzero} := \emptyset
  \and \\
  \freenames{x?(y).P} := \{ x \} \cup (\freenames{P} \setminus \{ y \})
  \and 
  \freenames{x!\langle P \rangle} := \{ x \} \cup \{ P \} 
  \and \\
  \freenames{P|Q} := \freenames{P} \cup \freenames{Q}
  \and \\
  \freenames{@{x}} := \{ x \}
\end{mathpar}

$\pi$
$\quotep{\pi}$

$\freenames{-} : \pi \to \mathcal{P}(\quotep{\pi})$

\begin{eqnarray*}
  \freenames{\pzero} & := & \emptyset \\
  \freenames{x?(y).P} & := & \{ x \} \cup (\freenames{P} \setminus \{ y \}) \\
  \freenames{x!\langle P \rangle} & := & \{ x \} \cup \{ P \} \\
  \freenames{P|Q} & := & \freenames{P} \cup \freenames{Q} \\
  \freenames{\dropn{x}} & := & \{ x \}
\end{eqnarray*}

The bound names of a process, $\boundnames{P}$, are those names occurring in $P$
that are not free. For example, in $x?(y).0$, the name $x$ is free, while $y$ is bound.

\begin{mathpar}
  \inferrule* [lab=monoidal-laws] {} { P|Q \equiv Q|P \and P|0 \equiv P \and P|(Q|R) \equiv (P|Q)|R }
\end{mathpar}

\begin{mathpar}
  \inferrule* [lab=alpha-equivalence] {} { (x)P \equiv (y)P\{y/x\} \and y \not\in \freenames{P} }
\end{mathpar}

\begin{definition}
Then two processes, $P,Q$, are alpha-equivalent if $P = Q\{\vec{y}/\vec{x}\}$ for
some $\vec{x} \in \boundnames{Q},\vec{y} \in \boundnames{P}$, where $Q\{\vec{y}/\vec{x}\}$
denotes the capture-avoiding substitution of $\vec{y}$ for $\vec{x}$ in $Q$.
\end{definition}

\begin{definition}
  The {\em structural congruence} \cite{SangiorgiWalker} , $\equiv$,
  between processes is the least congruence containing
  alpha-equivalence, satisfying the abelian monoid laws
  (associativity, commutativity and $\pzero$ as identity) for parallel
  composition $|$ and for summation $+$.
\end{definition}

\subsection{Name equivalence}

We take name equivalence, written $\nameeq$, to be the smallest
equivalence relation generated by the following rules.

\begin{mathpar}
\inferrule*[lab=Quote-drop]
{ }
{ \quotep{@{x}} \nameeq x }

\inferrule*[lab=Struct-equiv]
{ P \scong Q }
{ \quotep{P} \nameeq \quotep{Q} }
\end{mathpar}

The astute reader will have noticed that the mutual recursion of names
and processes imposes a mutual recursion on alpha-equivalence and
structural equivalence via name-equivalence. Fortunately, all of this
works out pleasantly and we may calculate in the natural way, free of
concern. The reader interested in the details is referred to the
appendix \ref{appendix:rho_details}.

\subsection{Substitution}

We use $\Proc$ for the set of processes, $\QProc$ for the set of
names, and $\id{\{}\vec{y} / \vec{x} \id{\}}$ to denote partial maps,
$s : \QProc \rightarrow \QProc$. A map, $s$ lifts, uniquely, to a map
on process terms, $\widehat{s} : \Proc \rightarrow \Proc$ by the
following equations.

\begin{mathpar}
  (0) \psubstp{Q}{P} := 0 \\
  (R \juxtap S) \psubstp{Q}{P}
  :=    
  (R)\psubstp{Q}{P} \juxtap (S) \psubstp{Q}{P} \\
  (x?(y).R) \psubstp{Q}{P}    
  :=    
  (x)\substp{Q}{P} (z)\concat( (R \psubstn{z}{y}) \psubstp{Q}{P} ) \\
  (\lift{x}{R}) \psubstp{Q}{P}  
  :=
  \lift{(x)\substp{Q}{P}}{ R \psubstp{Q}{P} } \\
%   (\dropn{x})  \psubstp{Q}{P}       
%   := 
%   \left\{ 
%     \begin{array}{ccc} 
%       \dropn{\quotep{Q}} & & x \nameeq \quotep{P} \\
%       \dropn{x} & & otherwise \\
%     \end{array}
%   \right. 
  (\dropn{x})  \psubstp{Q}{P}       
  := 
  \left\{ 
    \begin{array}{ccc} 
      Q & & x \nameeq \quotep{P} \\
      \dropn{x} & & otherwise \\
    \end{array}
  \right.
\end{mathpar}
 

where

\begin{eqnarray}
  (x)\id{\{} \lpquote Q \rpquote / \lpquote P \rpquote \id{\}}            = 
  \left\{ 
    \begin{array}{ccc}
      \lpquote Q \rpquote & & x \nameeq \lpquote P \rpquote \\
      x & & otherwise \\
    \end{array}
  \right. \nonumber
\end{eqnarray}

and $z$ is chosen distinct from $\quotep{P}$, $\quotep{Q}$, the free
names in $Q$, and all the names in $R$. Our $\alpha$-equivalence will
be built in the standard way from this substitution.

\begin{remark}\label{rem:no_self_referential_names}
  One consequence of these definitions is that $\forall P. \quotep{P}
  \not\in \freenames{P}$.
\end{remark}

\subsection{ Dynamic quote: an example }

Anticipating something of what's to come, consider applying the
substitution, $\widehat{\id{\{}u / z \id{\}}}$, to the following pair
of processes, $\lift{w}{y!(z)}$ and $w[ \lpquote y!(z) \rpquote ]$.

\begin{eqnarray}
	\lift{w}{y!(z)}\widehat{\id{\{}u / z \id{\}}}
		& = &
		\lift{w}{y!(u)} \nonumber\\
	w[ \lpquote y!(z) \rpquote ] \widehat{ \id{\{}u / z \id{\}} }
		& = &
		w[ \lpquote y!(z) \rpquote ] \nonumber
\end{eqnarray}

Because the body of the process between quotes is impervious to
substitution, we get radically different answers. In fact, by
examining the first process in an input context,
e.g. $x?(z).\lift{w}{y!(z)}$, we see that the process under the lift
operator may be shaped by prefixed inputs binding a name inside it. In
this sense, the lift operator will be seen as a way to dynamically
construct processes before reifying them as names.

Finally equipped with these standard features we can present the
dynamics of the calculus.

\subsubsection{Operational semantics} 

Finally, we introduce the computational dynamics. What marks these
algebras as distinct from other more traditionally studied algebraic
structures, e.g. vector spaces or polynomial rings, is the manner in
which dynamics is captured. In traditional structures, dynamics is typically
expressed through morphisms between such structures, as in linear maps
between vector spaces or morphisms between rings. In algebras
associated with the semantics of computation, the dynamics is
expressed as part of the algebraic structure itself, through a
reduction reduction relation typically denoted by $\red$. Below, we
give a recursive presentation of this relation for the calculus used
in the encoding.

$\red \subseteq \pi \times \pi$
$\red : \pi \to \mathcal{P}(\pi)$

\begin{mathpar}
  \inferrule* [lab=Comm] { \textsf{match}( x_{src}, x_{trgt} ) } { x_{trgt}?(y)P \; | \; x_{src}!\langle {Q} \rangle \red P\{\quotep{Q}/y}\} }
  \and \\
  \inferrule* [lab=Par] {{P} \red {P}'} {{{P} | {Q}} \red {{P}' | {Q}}}
  \and
  \inferrule* [lab=Equiv]{{{P} \scong {P}'} \andalso {{P}' \red {Q}'} \andalso {{Q}' \scong {Q}}}{{P} \red {Q}}
\end{mathpar}

\begin{eqnarray*}
  match_{\equiv} (\quotep{P},\quotep{Q}) & := & P \equiv Q \\
  match_{\dagger}(\quotep{P},\quotep{Q}) & := & \forall R. P|Q \red^{*} R => R \red^{*} 0 \\
  match_{K}(\quotep{P},\quotep{Q}) & := & K \mbox{ for some context } K
\end{eqnarray*}

$u?(x)P | u!\langle Q \rangle \red P\{\quotep{Q}/x\}$

%We write $\wred$ for $\red^*$, and $P\red$ if $\exists Q $ such that $ P \red Q$.
We write $P\red$ if $\exists Q $ such that $ P \red Q$ and $P\not\red$, otherwise.

\section{Replication}

As mentioned before, it is known that replication (and hence
recursion) can be implemented in a higher-order process algebra
\cite{SangiorgiWalker}. As our first example of calculation with the
machinery thus far presented we give the construction explicitly in
the {\rhoc}.

\begin{eqnarray}
	D_{x} & := & \prefix{x}{y}{(\binpar{\outputp{x}{y}}{@{y}})} \nonumber\\
	\bangp_{x}{P} & := & \binpar{{x}!\langle{\binpar{D_{x}}{P}}\rangle}{D_{x}} \nonumber
\end{eqnarray}

\begin{eqnarray}
	\bangp_{x}{P} & & \nonumber\\
	=
	& {x}!\langle{(\prefix{x}{y}{(\outputp{x}{y} | @{y})) | P}}\rangle 
	      | \prefix{x}{y}{(\outputp{x}{y} | @{y})} & \nonumber\\
	\red
	& (\outputp{x}{y} | @{y})\substn{\quotep{(\prefix{x}{y}{(@{y} | \outputp{x}{y})) | P}}}{y} & \nonumber\\
	=
	& \outputp{x}{\quotep{(\prefix{x}{y}{(\outputp{x}{y} | @{y})) | P}}}
	  | {(\prefix{x}{y}{(\outputp{x}{y} | @{y})) | P}} & \nonumber\\
	\red
	& \ldots & \nonumber\\
	\red^*
	& P | P | \ldots & \nonumber
\end{eqnarray}

Of course, this encoding, as an implementation, runs away, unfolding
$\bangp{P}$ eagerly. A lazier and more implementable replication
operator, restricted to input-guarded processes, may be obtained as follows.

\begin{eqnarray}
\bangp{\prefix{u}{v}{P}} 
	:= 
	\binpar{\lift{x}{\prefix{u}{v}{(\binpar{D(x)}{P})}}}{D(x)} \nonumber
\end{eqnarray}

\begin{remark}
  Note that the lazier definition still does not deal with summation
  or mixed summation (i.e. sums over input and output). The reader is
  invited to construct definitions of replication that deal with these
  features. 

  Further, the definitions are parameterized in a name, $x$. Can you,
  gentle reader, make a definition that eliminates this parameter and
  guarantees no accidental interaction between the replication
  machinery and the process being replicated -- i.e. no accidental
  sharing of names used by the process to get its work done and the
  name(s) used by the replication to effect copying. This latter
  revision of the definition of replication is crucial to obtaining
  the expected identity $!!P \sim !P$.
\end{remark}

\begin{remark}\label{rem:paradoxical_combinator}
  The reader familiar with the lambda calculus will have noticed the
  similarity between $D$ and the paradoxical combinator.

  [Ed. note: the existence of this seems to suggest we have to be more
  restrictive on the set of processes and names we admit if we are to
  support no-cloning.]
\end{remark}

\subsubsection{Bisimulation}

The computational dynamics gives rise to another kind of equivalence,
the equivalence of computational behavior. As previously mentioned
this is typically captured \emph{via} some form of bisimulation.

% The notion we use in this paper is weak barbed bisimulation
% \cite{milner91polyadicpi}.

The notion we use in this paper is derived from weak barbed
bisimulation \cite{milner91polyadicpi}. 

\begin{definition}
An \emph{observation relation}, $\downarrow_{\mathcal N}$, over a set
of names, $\mathcal N$, is the smallest relation satisfying the rules
below.

\infrule[Out-barb]{y \in {\mathcal N}, \; x \nameeq y}
		  {\outputp{x}{v} \downarrow_{\mathcal N} x}
\infrule[Par-barb]{\mbox{$P\downarrow_{\mathcal N} x$ or $Q\downarrow_{\mathcal N} x$}}
		  {\binpar{P}{Q} \downarrow_{\mathcal N} x}

We write $P \Downarrow_{\mathcal N} x$ if there is $Q$ such that 
$P \wred Q$ and $Q \downarrow_{\mathcal N} x$.
\end{definition}

\begin{definition}
%\label{def.bbisim}
An  ${\mathcal N}$-\emph{barbed bisimulation} over a set of names, ${\mathcal N}$, is a symmetric binary relation 
${\mathcal S}_{\mathcal N}$ between agents such that $P\rel{S}_{\mathcal N}Q$ implies:
\begin{enumerate}
\item If $P \red P'$ then $Q \wred Q'$ and $P'\rel{S}_{\mathcal N} Q'$.
\item If $P\downarrow_{\mathcal N} x$, then $Q\Downarrow_{\mathcal N} x$.
\end{enumerate}
$P$ is ${\mathcal N}$-barbed bisimilar to $Q$, written
$P \wbbisim_{\mathcal N} Q$, if $P \rel{S}_{\mathcal N} Q$ for some ${\mathcal N}$-barbed bisimulation ${\mathcal S}_{\mathcal N}$.
\end{definition}

$\mathcal{R} \subseteq \pi \times \pi$

$P \mathcal{R} Q => \forall P'. P \red P' \Rightarrow \exists Q'. Q \red Q', P' \mathcal{R} Q'$

$P \vdash x \Rightarrow Q \vdash x$

\begin{mathpar}
  \inferrule*[lab=Out-barb]{x \nameeq y}{{y}!\langle{Q}\rangle \vdash x}
  \and
  \inferrule*[lab=Par-barb]{\mbox{$P\vdash x$ or $Q\vdash x$}}{\binpar{P}{Q} \vdash x}
\end{mathpar}

\subsubsection{Contexts}

One of the principle advantages of computational calculi like the
$\pi$-calculus is a well-defined notion of context,
contextual-equivalence and a correlation between
contextual-equivalence and notions of bisimulation. The notion of
context allows the decomposition of a process into (sub-)process and
its syntactic environment, its context. Thus, a context may be
thought of as a process with a ``hole'' (written $\Box$) in it. The
application of a context $M$ to a process $P$, written $M[P]$, is
tantamount to filling the hole in $M$ with $P$. In this paper we do
not need the full weight of this theory, but do make use of the notion
of context in the proof the main theorem. 

\begin{mathpar}
  \inferrule* [lab=summation] {} {{M_{M},M_{N}} \bc \Box \;|\; x.M_{A} \;|\; M_{M}+M_{N}}
  \and
  \inferrule* [lab=agent] {} {{M_{A}} \bc (\vec{x})M_{P} \;| \; \clift{P_0,\ldots,M_{P},\ldots,P_N}}
  \and \\
  \inferrule* [lab=process] {} {{M_{P}} \bc M_{N} \;| \;P|M_{P} }
\end{mathpar} 

\begin{mathpar}
  \inferrule* [lab=sychronization] {} {M_{N} \bc \Box \;|\; x?M_{F} \;|\; x!M_{C}}
  \and
  \inferrule* [lab=abstraction] {} {{M_{F}} \bc (x)M_{P} }
  \and
  \inferrule* [lab=concretion] {} {{M_{C}} \bc \langle M_{P} \rangle }
  \and \\
  \inferrule* [lab=process] {} {{M_{P}} \bc M_{N} \;| \;P|M_{P} }
\end{mathpar}

\begin{definition}[contextual application] Given a context $M$, and
  process $P$, we define the \emph{contextual application}, $M[P] :=
  M\{P/\Box\}$. That is, the contextual application of M to P is the
  substitution of $P$ for $\Box$ in $M$.
\end{definition}

$\meaningof{-} : L \to \mathcal{P}(\pi)$

\begin{mathpar}
  \inferrule* [lab=collection] {} {\meaningof{true} = \pi, \and \meaningof{~E} = \pi \setminus \meaningof{E}, \and \meaningof{E_{1} \& E_{2}} = \meaningof{E_{1}} \cap \meaningof{E_{2}}}
\end{mathpar}

\begin{mathpar}
  \inferrule* [lab=structure] {} {\meaningof{0} = \{ P \in \pi | P \equiv 0 \}, \and \\ \meaningof{E_1 | E_2} = \{ P \in \pi | P \equiv P_{1} | P_{2}, P_{1} \in \meaningof{E_{1}}, P_{2} \in \meaningof{E_2}\} }
\end{mathpar}

\begin{mathpar}
 \inferrule* [lab=behavior] {} {\meaningof{\langle a?b \rangle E} = \{ P \in \pi | P \equiv Q | u?(y)P', \\ \and \\\\ \and \\ \;\;\; u \in \meaningof{a}, \forall z.P'\{z/y\} \in \meaningof{E\{z/b\}}\}, \and \\ \meaningof{a!E} = \{ P \in \pi | P \equiv Q | x!\langle P' \rangle, x \in \meaningof{a} P' \in \meaningof{E}\} }
\end{mathpar}

\begin{mathpar}
 \inferrule* [lab=nominal] {} {\meaningof{\quotep{E}} = \{ \quotep{P} \in \quotep{\pi} | P \in \meaningof{E} \}, \and \meaningof{\quotep{P}} = \{ \quotep{Q} \in \quotep{\pi} | P \equiv Q \} \and \\ \meaningof{@\quotep{E}} = \{ P \in \pi | P \equiv @x, x \in \meaningof{E} \}}
\end{mathpar}

\begin{eqnarray*}
  \\
  \meaningof{-} : TS \to ST
\end{eqnarray*}

\begin{eqnarray*}
  \\
  L : TS \to ST
\end{eqnarray*}

\begin{eqnarray*}
  \\
  P \models E \iff P \in \meaningof{E}
\end{eqnarray*}

\begin{eqnarray*}
  P \approx_{L} Q \iff \forall E \in L. P \models E \iff Q \models E
\end{eqnarray*}

\begin{eqnarray*}
  P \approx_{K} Q
\end{eqnarray*}

\begin{eqnarray*}
  P \approx Q
\end{eqnarray*}

$\approx_{K} = \approx = \approx_{L}$

\subsubsection{Contextual duality}

Note that contexts extend the quotation operation to a family of
operations from processes to names. Given a context, $M$, we can
define a \emph{nominal context}, $\quotep{M}$ by $\quotep{M}[P] :=
\quotep{M[P]}$. To foreshadow what is to come we observe that these
operations enjoy a duality with processes very much like the duality
between vectors and maps from vectors to scalars.

Further, because the calculus is essentially higher-order, we have a
correspondence between contexts and processes. More specifically,
given a name $x$ and a context $M$ we can construct $M^{*}_{x}$ such
that 

\begin{mathpar}
  M^{*}_{x} | \lift{x}{P} \red M[P]
\end{mathpar}

namely,

\begin{mathpar}
  M^{*}_{x} := x?(u).M[\dropn{u}]
\end{mathpar}

The dependence of $M^{*}_{x}$ on a name makes it an abstraction, 

\begin{mathpar}
  M^{*} := (x)x?(u).M[\dropn{u}]
\end{mathpar}

\subsection{Additional notation}

It will sometimes be convenient to denote the process a name
quotes. We already have the notation $x = \quotep{P}$, but it will be
convenient to introduce an alternate notation, $\procn{x}$, when we
want to emphasize the connection to the use of the name. Note that, by
virtue of name equivalence, $\quotep{\procn{x}} \nameeq x$; so, the
notation is consistent with previous definitions.

Further, because names have structure it is possible to effect
substitutions on the basis of that structure. This means we need to
upgrade our notation for substitutions, which we accomplish by
adapting comprehension notation. Thus,

\begin{mathpar}
  P\{ y / x : x \in S \}
\end{mathpar}

is interpreted to mean the process derived from P by replacing (in a
capture-avoiding manner) each occurrence of $x$ in $S$ by $y$. For example,

\begin{mathpar}
  P\{ \quotep{\procn{x}|\procn{x}} / x : x \in \freenames{P} \}
\end{mathpar}

will replace each (occurrence) of a free name $x$ in $P$ by
$\quotep{\procn{x}|\procn{x}}$.

Also, we will avail ourselves of the notation $x^{L}$ and $x^{R}$ to
denote injections of a name into disjoint copies of the name
space. There are numerous ways to accomplish this. One example can be
found in \cite{MeredithR05}. This notation overloads to vectors of
names: $\vec{x}^{\pi} := (x_{i}^{\pi} \; : \; 0 \leq i < |\vec{x}| )$ where $\pi \in \{L,R\}$.

We also use $P^{\Box} := P|\Box$.

In \cite{MeredithR05} an interpretation of the new operator is
given. It turns out that there are several possible interpretations
all enjoying the requisite algebraic properties of the operator (see
\cite{milner91polyadicpi}). We will therefore make liberal use of
$(\nu\; \vec{x})P$.

% subsection the_syntax_and_semantics_of_the_notation_system (end)   

\input{qm2pi.qmops} 

\input{qm2pi.sterngerlach} 

\input{qm2pi.metric} 

% section concurrent_process_calculi (end)

%\input{qm2pi.proofsketch}

% section proof sketch (end)

%\input{qm2pi.slviaknots} 

% section spatial logic via knots (end)

\input{qm2pi.conclusion}

% section conclusion (end)

%\input{qm2pi.dtcodes} 

% section wiring algorithm (end)

\input{qm2pi.ack} 

% section acknowledgments (end)

\newpage


\bibliographystyle{plain}   
\bibliography{../../biblios/main.bib}

\input{qm2pi.rhodetails}

\end{document}

 

%\documentclass[12pt]{llncs}
%\documentclass{jktr}

\usepackage[pdftex]{hyperref}                   
\usepackage {listings}
\usepackage {mathpartir}
\usepackage{bcprules}
%\usepackage{listings}
                       
\usepackage{graphicx} 
%\usepackage[margins=2.5cm,nohead,nofoot]{geometry}
%\usepackage{geometry}
\usepackage{amsfonts}
\usepackage{amstext}
\usepackage{latexsym}
\usepackage{amssymb}
\usepackage{color}


%\include{myPreamble}
\include{qm2pi.local} 

%\ifpdf
%\usepackage[pdftex]{graphicx}
%\else
%\usepackage{graphicx}
%\fi

 % \ifpdf
%  \usepackage{pdfsync}
%  \if


%\title{Brief Article}
%\author{David F. Snyder}
%\author{L.G. Meredith}

%\address{Dept. of Math., Texas State University--San Marcos, San Marcos, TX 78666}
       
\pagestyle{empty}


\begin{document}

\lstset{language=[Objective]Caml,frame=shadowbox}

\input{qm2pi.front}

% section front matter (end)

\input{qm2pi.intro} 
 
% section introduction (end)

% \input{qm2pi.knotations} 

% section notation (end)

\input{qm2pi.process.calculi} 

% section concurrent_process_calculi_and_spatial_logics_ (end)
    
%\input{qm2pi.knots2pi} 

%\input{qm2pi.trefoil} 

%\input{qm2pi.mainthm} 

% subsection basic_interpretation (end)

%\input{qm2pi.rho.presentation} 
\subsection{The syntax and semantics of the notation system}\label{sub:the_syntax_and_semantics_of_the_notation_system} % (fold)

We now summarize a technical presentation of the calculus that
embodies our theory of dynamics. The typical presentation of such a
calculus follows the style of giving generators and relations on
them. The grammar, below, describing term constructors, freely
generates the set of processes, $\Proc$. This set is then quotiented
by a relation known as structural congruence and it is over this set
that the notion of dynamics is expressed. This presentation is
essentially that of \cite{MeredithR05} with the addition of
polyadicity and summation. For readability we have relegated some of
the technical subtleties to an appendix.

\subsubsection{Process grammar}\label{subsub:process_grammar}

\begin{mathpar}
  \inferrule* [lab=synchronization] {} {{M} \bc \pzero \;|\; x?F \;|\; x!C }
  \and
  \inferrule* [lab=abstraction] {} {{F} \bc (x)P}
  \and
  \inferrule* [lab=concretion] {} {{C} \bc \langle Q \rangle}
  \and
  \inferrule* [lab=process] {} {{P,Q} \bc M \;| \;P|Q \;|\; @{x}}
  \and
  \inferrule* [lab=name] {} {{x} \bc \quotep{P}}
\end{mathpar} 

Note that $\vec{x}$ (resp. $\vec{P}$) denotes a vector of names
(resp. processes) of length $|\vec{x}|$ (resp. $|\vec{P}|$). We adopt
the following useful abbreviations.

\begin{mathpar}
   x?(\vec{y}).P := x.(\vec{y})P \and  x\clift{\vec{P}} := x.\clift{\vec{P}}
   \and x!(y) := \lift{x}{\dropn{y}}
   \and \Pi_{i=0}^{n-1}P_i := P_0 | \ldots | P_{n-1}
\end{mathpar}

\subsubsection{Structural congruence}

\paragraph{Free and bound names and alpha-equivalence.} At the
core of structural equivalence is alpha-equivalence which identifies
process that are the same up to a change of variable. Formally, we
recognize the distinction between free and bound names. The free names
of a process, $\freenames{P}$, may be calculated recursively as
follows:

\begin{mathpar}
\freenames{\pzero} := \emptyset
  \and \\
  \freenames{x?(y).P} := \{ x \} \cup (\freenames{P} \setminus \{ y \})
  \and 
  \freenames{x!\langle P \rangle} := \{ x \} \cup \{ P \} 
  \and \\
  \freenames{P|Q} := \freenames{P} \cup \freenames{Q}
  \and \\
  \freenames{@{x}} := \{ x \}
\end{mathpar}

$\pi$
$\quotep{\pi}$

$\freenames{-} : \pi \to \mathcal{P}(\quotep{\pi})$

\begin{eqnarray*}
  \freenames{\pzero} & := & \emptyset \\
  \freenames{x?(y).P} & := & \{ x \} \cup (\freenames{P} \setminus \{ y \}) \\
  \freenames{x!\langle P \rangle} & := & \{ x \} \cup \{ P \} \\
  \freenames{P|Q} & := & \freenames{P} \cup \freenames{Q} \\
  \freenames{\dropn{x}} & := & \{ x \}
\end{eqnarray*}

The bound names of a process, $\boundnames{P}$, are those names occurring in $P$
that are not free. For example, in $x?(y).0$, the name $x$ is free, while $y$ is bound.

\begin{mathpar}
  \inferrule* [lab=monoidal-laws] {} { P|Q \equiv Q|P \and P|0 \equiv P \and P|(Q|R) \equiv (P|Q)|R }
\end{mathpar}

\begin{mathpar}
  \inferrule* [lab=alpha-equivalence] {} { (x)P \equiv (y)P\{y/x\} \and y \not\in \freenames{P} }
\end{mathpar}

\begin{definition}
Then two processes, $P,Q$, are alpha-equivalent if $P = Q\{\vec{y}/\vec{x}\}$ for
some $\vec{x} \in \boundnames{Q},\vec{y} \in \boundnames{P}$, where $Q\{\vec{y}/\vec{x}\}$
denotes the capture-avoiding substitution of $\vec{y}$ for $\vec{x}$ in $Q$.
\end{definition}

\begin{definition}
  The {\em structural congruence} \cite{SangiorgiWalker} , $\equiv$,
  between processes is the least congruence containing
  alpha-equivalence, satisfying the abelian monoid laws
  (associativity, commutativity and $\pzero$ as identity) for parallel
  composition $|$ and for summation $+$.
\end{definition}

\subsection{Name equivalence}

We take name equivalence, written $\nameeq$, to be the smallest
equivalence relation generated by the following rules.

\begin{mathpar}
\inferrule*[lab=Quote-drop]
{ }
{ \quotep{@{x}} \nameeq x }

\inferrule*[lab=Struct-equiv]
{ P \scong Q }
{ \quotep{P} \nameeq \quotep{Q} }
\end{mathpar}

The astute reader will have noticed that the mutual recursion of names
and processes imposes a mutual recursion on alpha-equivalence and
structural equivalence via name-equivalence. Fortunately, all of this
works out pleasantly and we may calculate in the natural way, free of
concern. The reader interested in the details is referred to the
appendix \ref{appendix:rho_details}.

\subsection{Substitution}

We use $\Proc$ for the set of processes, $\QProc$ for the set of
names, and $\id{\{}\vec{y} / \vec{x} \id{\}}$ to denote partial maps,
$s : \QProc \rightarrow \QProc$. A map, $s$ lifts, uniquely, to a map
on process terms, $\widehat{s} : \Proc \rightarrow \Proc$ by the
following equations.

\begin{mathpar}
  (0) \psubstp{Q}{P} := 0 \\
  (R \juxtap S) \psubstp{Q}{P}
  :=    
  (R)\psubstp{Q}{P} \juxtap (S) \psubstp{Q}{P} \\
  (x?(y).R) \psubstp{Q}{P}    
  :=    
  (x)\substp{Q}{P} (z)\concat( (R \psubstn{z}{y}) \psubstp{Q}{P} ) \\
  (\lift{x}{R}) \psubstp{Q}{P}  
  :=
  \lift{(x)\substp{Q}{P}}{ R \psubstp{Q}{P} } \\
%   (\dropn{x})  \psubstp{Q}{P}       
%   := 
%   \left\{ 
%     \begin{array}{ccc} 
%       \dropn{\quotep{Q}} & & x \nameeq \quotep{P} \\
%       \dropn{x} & & otherwise \\
%     \end{array}
%   \right. 
  (\dropn{x})  \psubstp{Q}{P}       
  := 
  \left\{ 
    \begin{array}{ccc} 
      Q & & x \nameeq \quotep{P} \\
      \dropn{x} & & otherwise \\
    \end{array}
  \right.
\end{mathpar}
 

where

\begin{eqnarray}
  (x)\id{\{} \lpquote Q \rpquote / \lpquote P \rpquote \id{\}}            = 
  \left\{ 
    \begin{array}{ccc}
      \lpquote Q \rpquote & & x \nameeq \lpquote P \rpquote \\
      x & & otherwise \\
    \end{array}
  \right. \nonumber
\end{eqnarray}

and $z$ is chosen distinct from $\quotep{P}$, $\quotep{Q}$, the free
names in $Q$, and all the names in $R$. Our $\alpha$-equivalence will
be built in the standard way from this substitution.

\begin{remark}\label{rem:no_self_referential_names}
  One consequence of these definitions is that $\forall P. \quotep{P}
  \not\in \freenames{P}$.
\end{remark}

\subsection{ Dynamic quote: an example }

Anticipating something of what's to come, consider applying the
substitution, $\widehat{\id{\{}u / z \id{\}}}$, to the following pair
of processes, $\lift{w}{y!(z)}$ and $w[ \lpquote y!(z) \rpquote ]$.

\begin{eqnarray}
	\lift{w}{y!(z)}\widehat{\id{\{}u / z \id{\}}}
		& = &
		\lift{w}{y!(u)} \nonumber\\
	w[ \lpquote y!(z) \rpquote ] \widehat{ \id{\{}u / z \id{\}} }
		& = &
		w[ \lpquote y!(z) \rpquote ] \nonumber
\end{eqnarray}

Because the body of the process between quotes is impervious to
substitution, we get radically different answers. In fact, by
examining the first process in an input context,
e.g. $x?(z).\lift{w}{y!(z)}$, we see that the process under the lift
operator may be shaped by prefixed inputs binding a name inside it. In
this sense, the lift operator will be seen as a way to dynamically
construct processes before reifying them as names.

Finally equipped with these standard features we can present the
dynamics of the calculus.

\subsubsection{Operational semantics} 

Finally, we introduce the computational dynamics. What marks these
algebras as distinct from other more traditionally studied algebraic
structures, e.g. vector spaces or polynomial rings, is the manner in
which dynamics is captured. In traditional structures, dynamics is typically
expressed through morphisms between such structures, as in linear maps
between vector spaces or morphisms between rings. In algebras
associated with the semantics of computation, the dynamics is
expressed as part of the algebraic structure itself, through a
reduction reduction relation typically denoted by $\red$. Below, we
give a recursive presentation of this relation for the calculus used
in the encoding.

$\red \subseteq \pi \times \pi$
$\red : \pi \to \mathcal{P}(\pi)$

\begin{mathpar}
  \inferrule* [lab=Comm] { \textsf{match}( x_{src}, x_{trgt} ) } { x_{trgt}?(y)P \; | \; x_{src}!\langle {Q} \rangle \red P\{\quotep{Q}/y}\} }
  \and \\
  \inferrule* [lab=Par] {{P} \red {P}'} {{{P} | {Q}} \red {{P}' | {Q}}}
  \and
  \inferrule* [lab=Equiv]{{{P} \scong {P}'} \andalso {{P}' \red {Q}'} \andalso {{Q}' \scong {Q}}}{{P} \red {Q}}
\end{mathpar}

\begin{eqnarray*}
  match_{\equiv} (\quotep{P},\quotep{Q}) & := & P \equiv Q \\
  match_{\dagger}(\quotep{P},\quotep{Q}) & := & \forall R. P|Q \red^{*} R => R \red^{*} 0 \\
  match_{K}(\quotep{P},\quotep{Q}) & := & K \mbox{ for some context } K
\end{eqnarray*}

$u?(x)P | u!\langle Q \rangle \red P\{\quotep{Q}/x\}$

%We write $\wred$ for $\red^*$, and $P\red$ if $\exists Q $ such that $ P \red Q$.
We write $P\red$ if $\exists Q $ such that $ P \red Q$ and $P\not\red$, otherwise.

\section{Replication}

As mentioned before, it is known that replication (and hence
recursion) can be implemented in a higher-order process algebra
\cite{SangiorgiWalker}. As our first example of calculation with the
machinery thus far presented we give the construction explicitly in
the {\rhoc}.

\begin{eqnarray}
	D_{x} & := & \prefix{x}{y}{(\binpar{\outputp{x}{y}}{@{y}})} \nonumber\\
	\bangp_{x}{P} & := & \binpar{{x}!\langle{\binpar{D_{x}}{P}}\rangle}{D_{x}} \nonumber
\end{eqnarray}

\begin{eqnarray}
	\bangp_{x}{P} & & \nonumber\\
	=
	& {x}!\langle{(\prefix{x}{y}{(\outputp{x}{y} | @{y})) | P}}\rangle 
	      | \prefix{x}{y}{(\outputp{x}{y} | @{y})} & \nonumber\\
	\red
	& (\outputp{x}{y} | @{y})\substn{\quotep{(\prefix{x}{y}{(@{y} | \outputp{x}{y})) | P}}}{y} & \nonumber\\
	=
	& \outputp{x}{\quotep{(\prefix{x}{y}{(\outputp{x}{y} | @{y})) | P}}}
	  | {(\prefix{x}{y}{(\outputp{x}{y} | @{y})) | P}} & \nonumber\\
	\red
	& \ldots & \nonumber\\
	\red^*
	& P | P | \ldots & \nonumber
\end{eqnarray}

Of course, this encoding, as an implementation, runs away, unfolding
$\bangp{P}$ eagerly. A lazier and more implementable replication
operator, restricted to input-guarded processes, may be obtained as follows.

\begin{eqnarray}
\bangp{\prefix{u}{v}{P}} 
	:= 
	\binpar{\lift{x}{\prefix{u}{v}{(\binpar{D(x)}{P})}}}{D(x)} \nonumber
\end{eqnarray}

\begin{remark}
  Note that the lazier definition still does not deal with summation
  or mixed summation (i.e. sums over input and output). The reader is
  invited to construct definitions of replication that deal with these
  features. 

  Further, the definitions are parameterized in a name, $x$. Can you,
  gentle reader, make a definition that eliminates this parameter and
  guarantees no accidental interaction between the replication
  machinery and the process being replicated -- i.e. no accidental
  sharing of names used by the process to get its work done and the
  name(s) used by the replication to effect copying. This latter
  revision of the definition of replication is crucial to obtaining
  the expected identity $!!P \sim !P$.
\end{remark}

\begin{remark}\label{rem:paradoxical_combinator}
  The reader familiar with the lambda calculus will have noticed the
  similarity between $D$ and the paradoxical combinator.

  [Ed. note: the existence of this seems to suggest we have to be more
  restrictive on the set of processes and names we admit if we are to
  support no-cloning.]
\end{remark}

\subsubsection{Bisimulation}

The computational dynamics gives rise to another kind of equivalence,
the equivalence of computational behavior. As previously mentioned
this is typically captured \emph{via} some form of bisimulation.

% The notion we use in this paper is weak barbed bisimulation
% \cite{milner91polyadicpi}.

The notion we use in this paper is derived from weak barbed
bisimulation \cite{milner91polyadicpi}. 

\begin{definition}
An \emph{observation relation}, $\downarrow_{\mathcal N}$, over a set
of names, $\mathcal N$, is the smallest relation satisfying the rules
below.

\infrule[Out-barb]{y \in {\mathcal N}, \; x \nameeq y}
		  {\outputp{x}{v} \downarrow_{\mathcal N} x}
\infrule[Par-barb]{\mbox{$P\downarrow_{\mathcal N} x$ or $Q\downarrow_{\mathcal N} x$}}
		  {\binpar{P}{Q} \downarrow_{\mathcal N} x}

We write $P \Downarrow_{\mathcal N} x$ if there is $Q$ such that 
$P \wred Q$ and $Q \downarrow_{\mathcal N} x$.
\end{definition}

\begin{definition}
%\label{def.bbisim}
An  ${\mathcal N}$-\emph{barbed bisimulation} over a set of names, ${\mathcal N}$, is a symmetric binary relation 
${\mathcal S}_{\mathcal N}$ between agents such that $P\rel{S}_{\mathcal N}Q$ implies:
\begin{enumerate}
\item If $P \red P'$ then $Q \wred Q'$ and $P'\rel{S}_{\mathcal N} Q'$.
\item If $P\downarrow_{\mathcal N} x$, then $Q\Downarrow_{\mathcal N} x$.
\end{enumerate}
$P$ is ${\mathcal N}$-barbed bisimilar to $Q$, written
$P \wbbisim_{\mathcal N} Q$, if $P \rel{S}_{\mathcal N} Q$ for some ${\mathcal N}$-barbed bisimulation ${\mathcal S}_{\mathcal N}$.
\end{definition}

$\mathcal{R} \subseteq \pi \times \pi$

$P \mathcal{R} Q => \forall P'. P \red P' \Rightarrow \exists Q'. Q \red Q', P' \mathcal{R} Q'$

$P \vdash x \Rightarrow Q \vdash x$

\begin{mathpar}
  \inferrule*[lab=Out-barb]{x \nameeq y}{{y}!\langle{Q}\rangle \vdash x}
  \and
  \inferrule*[lab=Par-barb]{\mbox{$P\vdash x$ or $Q\vdash x$}}{\binpar{P}{Q} \vdash x}
\end{mathpar}

\subsubsection{Contexts}

One of the principle advantages of computational calculi like the
$\pi$-calculus is a well-defined notion of context,
contextual-equivalence and a correlation between
contextual-equivalence and notions of bisimulation. The notion of
context allows the decomposition of a process into (sub-)process and
its syntactic environment, its context. Thus, a context may be
thought of as a process with a ``hole'' (written $\Box$) in it. The
application of a context $M$ to a process $P$, written $M[P]$, is
tantamount to filling the hole in $M$ with $P$. In this paper we do
not need the full weight of this theory, but do make use of the notion
of context in the proof the main theorem. 

\begin{mathpar}
  \inferrule* [lab=summation] {} {{M_{M},M_{N}} \bc \Box \;|\; x.M_{A} \;|\; M_{M}+M_{N}}
  \and
  \inferrule* [lab=agent] {} {{M_{A}} \bc (\vec{x})M_{P} \;| \; \clift{P_0,\ldots,M_{P},\ldots,P_N}}
  \and \\
  \inferrule* [lab=process] {} {{M_{P}} \bc M_{N} \;| \;P|M_{P} }
\end{mathpar} 

\begin{mathpar}
  \inferrule* [lab=sychronization] {} {M_{N} \bc \Box \;|\; x?M_{F} \;|\; x!M_{C}}
  \and
  \inferrule* [lab=abstraction] {} {{M_{F}} \bc (x)M_{P} }
  \and
  \inferrule* [lab=concretion] {} {{M_{C}} \bc \langle M_{P} \rangle }
  \and \\
  \inferrule* [lab=process] {} {{M_{P}} \bc M_{N} \;| \;P|M_{P} }
\end{mathpar}

\begin{definition}[contextual application] Given a context $M$, and
  process $P$, we define the \emph{contextual application}, $M[P] :=
  M\{P/\Box\}$. That is, the contextual application of M to P is the
  substitution of $P$ for $\Box$ in $M$.
\end{definition}

$\meaningof{-} : L \to \mathcal{P}(\pi)$

\begin{mathpar}
  \inferrule* [lab=collection] {} {\meaningof{true} = \pi, \and \meaningof{~E} = \pi \setminus \meaningof{E}, \and \meaningof{E_{1} \& E_{2}} = \meaningof{E_{1}} \cap \meaningof{E_{2}}}
\end{mathpar}

\begin{mathpar}
  \inferrule* [lab=structure] {} {\meaningof{0} = \{ P \in \pi | P \equiv 0 \}, \and \\ \meaningof{E_1 | E_2} = \{ P \in \pi | P \equiv P_{1} | P_{2}, P_{1} \in \meaningof{E_{1}}, P_{2} \in \meaningof{E_2}\} }
\end{mathpar}

\begin{mathpar}
 \inferrule* [lab=behavior] {} {\meaningof{\langle a?b \rangle E} = \{ P \in \pi | P \equiv Q | u?(y)P', \\ \and \\\\ \and \\ \;\;\; u \in \meaningof{a}, \forall z.P'\{z/y\} \in \meaningof{E\{z/b\}}\}, \and \\ \meaningof{a!E} = \{ P \in \pi | P \equiv Q | x!\langle P' \rangle, x \in \meaningof{a} P' \in \meaningof{E}\} }
\end{mathpar}

\begin{mathpar}
 \inferrule* [lab=nominal] {} {\meaningof{\quotep{E}} = \{ \quotep{P} \in \quotep{\pi} | P \in \meaningof{E} \}, \and \meaningof{\quotep{P}} = \{ \quotep{Q} \in \quotep{\pi} | P \equiv Q \} \and \\ \meaningof{@\quotep{E}} = \{ P \in \pi | P \equiv @x, x \in \meaningof{E} \}}
\end{mathpar}

\begin{eqnarray*}
  \\
  \meaningof{-} : TS \to ST
\end{eqnarray*}

\begin{eqnarray*}
  \\
  L : TS \to ST
\end{eqnarray*}

\begin{eqnarray*}
  \\
  P \models E \iff P \in \meaningof{E}
\end{eqnarray*}

\begin{eqnarray*}
  P \approx_{L} Q \iff \forall E \in L. P \models E \iff Q \models E
\end{eqnarray*}

\begin{eqnarray*}
  P \approx_{K} Q
\end{eqnarray*}

\begin{eqnarray*}
  P \approx Q
\end{eqnarray*}

$\approx_{K} = \approx = \approx_{L}$

\subsubsection{Contextual duality}

Note that contexts extend the quotation operation to a family of
operations from processes to names. Given a context, $M$, we can
define a \emph{nominal context}, $\quotep{M}$ by $\quotep{M}[P] :=
\quotep{M[P]}$. To foreshadow what is to come we observe that these
operations enjoy a duality with processes very much like the duality
between vectors and maps from vectors to scalars.

Further, because the calculus is essentially higher-order, we have a
correspondence between contexts and processes. More specifically,
given a name $x$ and a context $M$ we can construct $M^{*}_{x}$ such
that 

\begin{mathpar}
  M^{*}_{x} | \lift{x}{P} \red M[P]
\end{mathpar}

namely,

\begin{mathpar}
  M^{*}_{x} := x?(u).M[\dropn{u}]
\end{mathpar}

The dependence of $M^{*}_{x}$ on a name makes it an abstraction, 

\begin{mathpar}
  M^{*} := (x)x?(u).M[\dropn{u}]
\end{mathpar}

\subsection{Additional notation}

It will sometimes be convenient to denote the process a name
quotes. We already have the notation $x = \quotep{P}$, but it will be
convenient to introduce an alternate notation, $\procn{x}$, when we
want to emphasize the connection to the use of the name. Note that, by
virtue of name equivalence, $\quotep{\procn{x}} \nameeq x$; so, the
notation is consistent with previous definitions.

Further, because names have structure it is possible to effect
substitutions on the basis of that structure. This means we need to
upgrade our notation for substitutions, which we accomplish by
adapting comprehension notation. Thus,

\begin{mathpar}
  P\{ y / x : x \in S \}
\end{mathpar}

is interpreted to mean the process derived from P by replacing (in a
capture-avoiding manner) each occurrence of $x$ in $S$ by $y$. For example,

\begin{mathpar}
  P\{ \quotep{\procn{x}|\procn{x}} / x : x \in \freenames{P} \}
\end{mathpar}

will replace each (occurrence) of a free name $x$ in $P$ by
$\quotep{\procn{x}|\procn{x}}$.

Also, we will avail ourselves of the notation $x^{L}$ and $x^{R}$ to
denote injections of a name into disjoint copies of the name
space. There are numerous ways to accomplish this. One example can be
found in \cite{MeredithR05}. This notation overloads to vectors of
names: $\vec{x}^{\pi} := (x_{i}^{\pi} \; : \; 0 \leq i < |\vec{x}| )$ where $\pi \in \{L,R\}$.

We also use $P^{\Box} := P|\Box$.

In \cite{MeredithR05} an interpretation of the new operator is
given. It turns out that there are several possible interpretations
all enjoying the requisite algebraic properties of the operator (see
\cite{milner91polyadicpi}). We will therefore make liberal use of
$(\nu\; \vec{x})P$.

% subsection the_syntax_and_semantics_of_the_notation_system (end)   

\input{qm2pi.qmops} 

\input{qm2pi.sterngerlach} 

\input{qm2pi.metric} 

% section concurrent_process_calculi (end)

%\input{qm2pi.proofsketch}

% section proof sketch (end)

%\input{qm2pi.slviaknots} 

% section spatial logic via knots (end)

\input{qm2pi.conclusion}

% section conclusion (end)

%\input{qm2pi.dtcodes} 

% section wiring algorithm (end)

\input{qm2pi.ack} 

% section acknowledgments (end)

\newpage


\bibliographystyle{plain}   
\bibliography{../../biblios/main.bib}

\input{qm2pi.rhodetails}

\end{document}

 

% subsection basic_interpretation (end)

%\input{qm2pi.rho.presentation} 
\subsection{The syntax and semantics of the notation system}\label{sub:the_syntax_and_semantics_of_the_notation_system} % (fold)

We now summarize a technical presentation of the calculus that
embodies our theory of dynamics. The typical presentation of such a
calculus follows the style of giving generators and relations on
them. The grammar, below, describing term constructors, freely
generates the set of processes, $\Proc$. This set is then quotiented
by a relation known as structural congruence and it is over this set
that the notion of dynamics is expressed. This presentation is
essentially that of \cite{MeredithR05} with the addition of
polyadicity and summation. For readability we have relegated some of
the technical subtleties to an appendix.

\subsubsection{Process grammar}\label{subsub:process_grammar}

\begin{mathpar}
  \inferrule* [lab=synchronization] {} {{M} \bc \pzero \;|\; x?F \;|\; x!C }
  \and
  \inferrule* [lab=abstraction] {} {{F} \bc (x)P}
  \and
  \inferrule* [lab=concretion] {} {{C} \bc \langle Q \rangle}
  \and
  \inferrule* [lab=process] {} {{P,Q} \bc M \;| \;P|Q \;|\; @{x}}
  \and
  \inferrule* [lab=name] {} {{x} \bc \quotep{P}}
\end{mathpar} 

Note that $\vec{x}$ (resp. $\vec{P}$) denotes a vector of names
(resp. processes) of length $|\vec{x}|$ (resp. $|\vec{P}|$). We adopt
the following useful abbreviations.

\begin{mathpar}
   x?(\vec{y}).P := x.(\vec{y})P \and  x\clift{\vec{P}} := x.\clift{\vec{P}}
   \and x!(y) := \lift{x}{\dropn{y}}
   \and \Pi_{i=0}^{n-1}P_i := P_0 | \ldots | P_{n-1}
\end{mathpar}

\subsubsection{Structural congruence}

\paragraph{Free and bound names and alpha-equivalence.} At the
core of structural equivalence is alpha-equivalence which identifies
process that are the same up to a change of variable. Formally, we
recognize the distinction between free and bound names. The free names
of a process, $\freenames{P}$, may be calculated recursively as
follows:

\begin{mathpar}
\freenames{\pzero} := \emptyset
  \and \\
  \freenames{x?(y).P} := \{ x \} \cup (\freenames{P} \setminus \{ y \})
  \and 
  \freenames{x!\langle P \rangle} := \{ x \} \cup \{ P \} 
  \and \\
  \freenames{P|Q} := \freenames{P} \cup \freenames{Q}
  \and \\
  \freenames{@{x}} := \{ x \}
\end{mathpar}

$\pi$
$\quotep{\pi}$

$\freenames{-} : \pi \to \mathcal{P}(\quotep{\pi})$

\begin{eqnarray*}
  \freenames{\pzero} & := & \emptyset \\
  \freenames{x?(y).P} & := & \{ x \} \cup (\freenames{P} \setminus \{ y \}) \\
  \freenames{x!\langle P \rangle} & := & \{ x \} \cup \{ P \} \\
  \freenames{P|Q} & := & \freenames{P} \cup \freenames{Q} \\
  \freenames{\dropn{x}} & := & \{ x \}
\end{eqnarray*}

The bound names of a process, $\boundnames{P}$, are those names occurring in $P$
that are not free. For example, in $x?(y).0$, the name $x$ is free, while $y$ is bound.

\begin{mathpar}
  \inferrule* [lab=monoidal-laws] {} { P|Q \equiv Q|P \and P|0 \equiv P \and P|(Q|R) \equiv (P|Q)|R }
\end{mathpar}

\begin{mathpar}
  \inferrule* [lab=alpha-equivalence] {} { (x)P \equiv (y)P\{y/x\} \and y \not\in \freenames{P} }
\end{mathpar}

\begin{definition}
Then two processes, $P,Q$, are alpha-equivalent if $P = Q\{\vec{y}/\vec{x}\}$ for
some $\vec{x} \in \boundnames{Q},\vec{y} \in \boundnames{P}$, where $Q\{\vec{y}/\vec{x}\}$
denotes the capture-avoiding substitution of $\vec{y}$ for $\vec{x}$ in $Q$.
\end{definition}

\begin{definition}
  The {\em structural congruence} \cite{SangiorgiWalker} , $\equiv$,
  between processes is the least congruence containing
  alpha-equivalence, satisfying the abelian monoid laws
  (associativity, commutativity and $\pzero$ as identity) for parallel
  composition $|$ and for summation $+$.
\end{definition}

\subsection{Name equivalence}

We take name equivalence, written $\nameeq$, to be the smallest
equivalence relation generated by the following rules.

\begin{mathpar}
\inferrule*[lab=Quote-drop]
{ }
{ \quotep{@{x}} \nameeq x }

\inferrule*[lab=Struct-equiv]
{ P \scong Q }
{ \quotep{P} \nameeq \quotep{Q} }
\end{mathpar}

The astute reader will have noticed that the mutual recursion of names
and processes imposes a mutual recursion on alpha-equivalence and
structural equivalence via name-equivalence. Fortunately, all of this
works out pleasantly and we may calculate in the natural way, free of
concern. The reader interested in the details is referred to the
appendix \ref{appendix:rho_details}.

\subsection{Substitution}

We use $\Proc$ for the set of processes, $\QProc$ for the set of
names, and $\id{\{}\vec{y} / \vec{x} \id{\}}$ to denote partial maps,
$s : \QProc \rightarrow \QProc$. A map, $s$ lifts, uniquely, to a map
on process terms, $\widehat{s} : \Proc \rightarrow \Proc$ by the
following equations.

\begin{mathpar}
  (0) \psubstp{Q}{P} := 0 \\
  (R \juxtap S) \psubstp{Q}{P}
  :=    
  (R)\psubstp{Q}{P} \juxtap (S) \psubstp{Q}{P} \\
  (x?(y).R) \psubstp{Q}{P}    
  :=    
  (x)\substp{Q}{P} (z)\concat( (R \psubstn{z}{y}) \psubstp{Q}{P} ) \\
  (\lift{x}{R}) \psubstp{Q}{P}  
  :=
  \lift{(x)\substp{Q}{P}}{ R \psubstp{Q}{P} } \\
%   (\dropn{x})  \psubstp{Q}{P}       
%   := 
%   \left\{ 
%     \begin{array}{ccc} 
%       \dropn{\quotep{Q}} & & x \nameeq \quotep{P} \\
%       \dropn{x} & & otherwise \\
%     \end{array}
%   \right. 
  (\dropn{x})  \psubstp{Q}{P}       
  := 
  \left\{ 
    \begin{array}{ccc} 
      Q & & x \nameeq \quotep{P} \\
      \dropn{x} & & otherwise \\
    \end{array}
  \right.
\end{mathpar}
 

where

\begin{eqnarray}
  (x)\id{\{} \lpquote Q \rpquote / \lpquote P \rpquote \id{\}}            = 
  \left\{ 
    \begin{array}{ccc}
      \lpquote Q \rpquote & & x \nameeq \lpquote P \rpquote \\
      x & & otherwise \\
    \end{array}
  \right. \nonumber
\end{eqnarray}

and $z$ is chosen distinct from $\quotep{P}$, $\quotep{Q}$, the free
names in $Q$, and all the names in $R$. Our $\alpha$-equivalence will
be built in the standard way from this substitution.

\begin{remark}\label{rem:no_self_referential_names}
  One consequence of these definitions is that $\forall P. \quotep{P}
  \not\in \freenames{P}$.
\end{remark}

\subsection{ Dynamic quote: an example }

Anticipating something of what's to come, consider applying the
substitution, $\widehat{\id{\{}u / z \id{\}}}$, to the following pair
of processes, $\lift{w}{y!(z)}$ and $w[ \lpquote y!(z) \rpquote ]$.

\begin{eqnarray}
	\lift{w}{y!(z)}\widehat{\id{\{}u / z \id{\}}}
		& = &
		\lift{w}{y!(u)} \nonumber\\
	w[ \lpquote y!(z) \rpquote ] \widehat{ \id{\{}u / z \id{\}} }
		& = &
		w[ \lpquote y!(z) \rpquote ] \nonumber
\end{eqnarray}

Because the body of the process between quotes is impervious to
substitution, we get radically different answers. In fact, by
examining the first process in an input context,
e.g. $x?(z).\lift{w}{y!(z)}$, we see that the process under the lift
operator may be shaped by prefixed inputs binding a name inside it. In
this sense, the lift operator will be seen as a way to dynamically
construct processes before reifying them as names.

Finally equipped with these standard features we can present the
dynamics of the calculus.

\subsubsection{Operational semantics} 

Finally, we introduce the computational dynamics. What marks these
algebras as distinct from other more traditionally studied algebraic
structures, e.g. vector spaces or polynomial rings, is the manner in
which dynamics is captured. In traditional structures, dynamics is typically
expressed through morphisms between such structures, as in linear maps
between vector spaces or morphisms between rings. In algebras
associated with the semantics of computation, the dynamics is
expressed as part of the algebraic structure itself, through a
reduction reduction relation typically denoted by $\red$. Below, we
give a recursive presentation of this relation for the calculus used
in the encoding.

$\red \subseteq \pi \times \pi$
$\red : \pi \to \mathcal{P}(\pi)$

\begin{mathpar}
  \inferrule* [lab=Comm] { \textsf{match}( x_{src}, x_{trgt} ) } { x_{trgt}?(y)P \; | \; x_{src}!\langle {Q} \rangle \red P\{\quotep{Q}/y}\} }
  \and \\
  \inferrule* [lab=Par] {{P} \red {P}'} {{{P} | {Q}} \red {{P}' | {Q}}}
  \and
  \inferrule* [lab=Equiv]{{{P} \scong {P}'} \andalso {{P}' \red {Q}'} \andalso {{Q}' \scong {Q}}}{{P} \red {Q}}
\end{mathpar}

\begin{eqnarray*}
  match_{\equiv} (\quotep{P},\quotep{Q}) & := & P \equiv Q \\
  match_{\dagger}(\quotep{P},\quotep{Q}) & := & \forall R. P|Q \red^{*} R => R \red^{*} 0 \\
  match_{K}(\quotep{P},\quotep{Q}) & := & K \mbox{ for some context } K
\end{eqnarray*}

$u?(x)P | u!\langle Q \rangle \red P\{\quotep{Q}/x\}$

%We write $\wred$ for $\red^*$, and $P\red$ if $\exists Q $ such that $ P \red Q$.
We write $P\red$ if $\exists Q $ such that $ P \red Q$ and $P\not\red$, otherwise.

\section{Replication}

As mentioned before, it is known that replication (and hence
recursion) can be implemented in a higher-order process algebra
\cite{SangiorgiWalker}. As our first example of calculation with the
machinery thus far presented we give the construction explicitly in
the {\rhoc}.

\begin{eqnarray}
	D_{x} & := & \prefix{x}{y}{(\binpar{\outputp{x}{y}}{@{y}})} \nonumber\\
	\bangp_{x}{P} & := & \binpar{{x}!\langle{\binpar{D_{x}}{P}}\rangle}{D_{x}} \nonumber
\end{eqnarray}

\begin{eqnarray}
	\bangp_{x}{P} & & \nonumber\\
	=
	& {x}!\langle{(\prefix{x}{y}{(\outputp{x}{y} | @{y})) | P}}\rangle 
	      | \prefix{x}{y}{(\outputp{x}{y} | @{y})} & \nonumber\\
	\red
	& (\outputp{x}{y} | @{y})\substn{\quotep{(\prefix{x}{y}{(@{y} | \outputp{x}{y})) | P}}}{y} & \nonumber\\
	=
	& \outputp{x}{\quotep{(\prefix{x}{y}{(\outputp{x}{y} | @{y})) | P}}}
	  | {(\prefix{x}{y}{(\outputp{x}{y} | @{y})) | P}} & \nonumber\\
	\red
	& \ldots & \nonumber\\
	\red^*
	& P | P | \ldots & \nonumber
\end{eqnarray}

Of course, this encoding, as an implementation, runs away, unfolding
$\bangp{P}$ eagerly. A lazier and more implementable replication
operator, restricted to input-guarded processes, may be obtained as follows.

\begin{eqnarray}
\bangp{\prefix{u}{v}{P}} 
	:= 
	\binpar{\lift{x}{\prefix{u}{v}{(\binpar{D(x)}{P})}}}{D(x)} \nonumber
\end{eqnarray}

\begin{remark}
  Note that the lazier definition still does not deal with summation
  or mixed summation (i.e. sums over input and output). The reader is
  invited to construct definitions of replication that deal with these
  features. 

  Further, the definitions are parameterized in a name, $x$. Can you,
  gentle reader, make a definition that eliminates this parameter and
  guarantees no accidental interaction between the replication
  machinery and the process being replicated -- i.e. no accidental
  sharing of names used by the process to get its work done and the
  name(s) used by the replication to effect copying. This latter
  revision of the definition of replication is crucial to obtaining
  the expected identity $!!P \sim !P$.
\end{remark}

\begin{remark}\label{rem:paradoxical_combinator}
  The reader familiar with the lambda calculus will have noticed the
  similarity between $D$ and the paradoxical combinator.

  [Ed. note: the existence of this seems to suggest we have to be more
  restrictive on the set of processes and names we admit if we are to
  support no-cloning.]
\end{remark}

\subsubsection{Bisimulation}

The computational dynamics gives rise to another kind of equivalence,
the equivalence of computational behavior. As previously mentioned
this is typically captured \emph{via} some form of bisimulation.

% The notion we use in this paper is weak barbed bisimulation
% \cite{milner91polyadicpi}.

The notion we use in this paper is derived from weak barbed
bisimulation \cite{milner91polyadicpi}. 

\begin{definition}
An \emph{observation relation}, $\downarrow_{\mathcal N}$, over a set
of names, $\mathcal N$, is the smallest relation satisfying the rules
below.

\infrule[Out-barb]{y \in {\mathcal N}, \; x \nameeq y}
		  {\outputp{x}{v} \downarrow_{\mathcal N} x}
\infrule[Par-barb]{\mbox{$P\downarrow_{\mathcal N} x$ or $Q\downarrow_{\mathcal N} x$}}
		  {\binpar{P}{Q} \downarrow_{\mathcal N} x}

We write $P \Downarrow_{\mathcal N} x$ if there is $Q$ such that 
$P \wred Q$ and $Q \downarrow_{\mathcal N} x$.
\end{definition}

\begin{definition}
%\label{def.bbisim}
An  ${\mathcal N}$-\emph{barbed bisimulation} over a set of names, ${\mathcal N}$, is a symmetric binary relation 
${\mathcal S}_{\mathcal N}$ between agents such that $P\rel{S}_{\mathcal N}Q$ implies:
\begin{enumerate}
\item If $P \red P'$ then $Q \wred Q'$ and $P'\rel{S}_{\mathcal N} Q'$.
\item If $P\downarrow_{\mathcal N} x$, then $Q\Downarrow_{\mathcal N} x$.
\end{enumerate}
$P$ is ${\mathcal N}$-barbed bisimilar to $Q$, written
$P \wbbisim_{\mathcal N} Q$, if $P \rel{S}_{\mathcal N} Q$ for some ${\mathcal N}$-barbed bisimulation ${\mathcal S}_{\mathcal N}$.
\end{definition}

$\mathcal{R} \subseteq \pi \times \pi$

$P \mathcal{R} Q => \forall P'. P \red P' \Rightarrow \exists Q'. Q \red Q', P' \mathcal{R} Q'$

$P \vdash x \Rightarrow Q \vdash x$

\begin{mathpar}
  \inferrule*[lab=Out-barb]{x \nameeq y}{{y}!\langle{Q}\rangle \vdash x}
  \and
  \inferrule*[lab=Par-barb]{\mbox{$P\vdash x$ or $Q\vdash x$}}{\binpar{P}{Q} \vdash x}
\end{mathpar}

\subsubsection{Contexts}

One of the principle advantages of computational calculi like the
$\pi$-calculus is a well-defined notion of context,
contextual-equivalence and a correlation between
contextual-equivalence and notions of bisimulation. The notion of
context allows the decomposition of a process into (sub-)process and
its syntactic environment, its context. Thus, a context may be
thought of as a process with a ``hole'' (written $\Box$) in it. The
application of a context $M$ to a process $P$, written $M[P]$, is
tantamount to filling the hole in $M$ with $P$. In this paper we do
not need the full weight of this theory, but do make use of the notion
of context in the proof the main theorem. 

\begin{mathpar}
  \inferrule* [lab=summation] {} {{M_{M},M_{N}} \bc \Box \;|\; x.M_{A} \;|\; M_{M}+M_{N}}
  \and
  \inferrule* [lab=agent] {} {{M_{A}} \bc (\vec{x})M_{P} \;| \; \clift{P_0,\ldots,M_{P},\ldots,P_N}}
  \and \\
  \inferrule* [lab=process] {} {{M_{P}} \bc M_{N} \;| \;P|M_{P} }
\end{mathpar} 

\begin{mathpar}
  \inferrule* [lab=sychronization] {} {M_{N} \bc \Box \;|\; x?M_{F} \;|\; x!M_{C}}
  \and
  \inferrule* [lab=abstraction] {} {{M_{F}} \bc (x)M_{P} }
  \and
  \inferrule* [lab=concretion] {} {{M_{C}} \bc \langle M_{P} \rangle }
  \and \\
  \inferrule* [lab=process] {} {{M_{P}} \bc M_{N} \;| \;P|M_{P} }
\end{mathpar}

\begin{definition}[contextual application] Given a context $M$, and
  process $P$, we define the \emph{contextual application}, $M[P] :=
  M\{P/\Box\}$. That is, the contextual application of M to P is the
  substitution of $P$ for $\Box$ in $M$.
\end{definition}

$\meaningof{-} : L \to \mathcal{P}(\pi)$

\begin{mathpar}
  \inferrule* [lab=collection] {} {\meaningof{true} = \pi, \and \meaningof{~E} = \pi \setminus \meaningof{E}, \and \meaningof{E_{1} \& E_{2}} = \meaningof{E_{1}} \cap \meaningof{E_{2}}}
\end{mathpar}

\begin{mathpar}
  \inferrule* [lab=structure] {} {\meaningof{0} = \{ P \in \pi | P \equiv 0 \}, \and \\ \meaningof{E_1 | E_2} = \{ P \in \pi | P \equiv P_{1} | P_{2}, P_{1} \in \meaningof{E_{1}}, P_{2} \in \meaningof{E_2}\} }
\end{mathpar}

\begin{mathpar}
 \inferrule* [lab=behavior] {} {\meaningof{\langle a?b \rangle E} = \{ P \in \pi | P \equiv Q | u?(y)P', \\ \and \\\\ \and \\ \;\;\; u \in \meaningof{a}, \forall z.P'\{z/y\} \in \meaningof{E\{z/b\}}\}, \and \\ \meaningof{a!E} = \{ P \in \pi | P \equiv Q | x!\langle P' \rangle, x \in \meaningof{a} P' \in \meaningof{E}\} }
\end{mathpar}

\begin{mathpar}
 \inferrule* [lab=nominal] {} {\meaningof{\quotep{E}} = \{ \quotep{P} \in \quotep{\pi} | P \in \meaningof{E} \}, \and \meaningof{\quotep{P}} = \{ \quotep{Q} \in \quotep{\pi} | P \equiv Q \} \and \\ \meaningof{@\quotep{E}} = \{ P \in \pi | P \equiv @x, x \in \meaningof{E} \}}
\end{mathpar}

\begin{eqnarray*}
  \\
  \meaningof{-} : TS \to ST
\end{eqnarray*}

\begin{eqnarray*}
  \\
  L : TS \to ST
\end{eqnarray*}

\begin{eqnarray*}
  \\
  P \models E \iff P \in \meaningof{E}
\end{eqnarray*}

\begin{eqnarray*}
  P \approx_{L} Q \iff \forall E \in L. P \models E \iff Q \models E
\end{eqnarray*}

\begin{eqnarray*}
  P \approx_{K} Q
\end{eqnarray*}

\begin{eqnarray*}
  P \approx Q
\end{eqnarray*}

$\approx_{K} = \approx = \approx_{L}$

\subsubsection{Contextual duality}

Note that contexts extend the quotation operation to a family of
operations from processes to names. Given a context, $M$, we can
define a \emph{nominal context}, $\quotep{M}$ by $\quotep{M}[P] :=
\quotep{M[P]}$. To foreshadow what is to come we observe that these
operations enjoy a duality with processes very much like the duality
between vectors and maps from vectors to scalars.

Further, because the calculus is essentially higher-order, we have a
correspondence between contexts and processes. More specifically,
given a name $x$ and a context $M$ we can construct $M^{*}_{x}$ such
that 

\begin{mathpar}
  M^{*}_{x} | \lift{x}{P} \red M[P]
\end{mathpar}

namely,

\begin{mathpar}
  M^{*}_{x} := x?(u).M[\dropn{u}]
\end{mathpar}

The dependence of $M^{*}_{x}$ on a name makes it an abstraction, 

\begin{mathpar}
  M^{*} := (x)x?(u).M[\dropn{u}]
\end{mathpar}

\subsection{Additional notation}

It will sometimes be convenient to denote the process a name
quotes. We already have the notation $x = \quotep{P}$, but it will be
convenient to introduce an alternate notation, $\procn{x}$, when we
want to emphasize the connection to the use of the name. Note that, by
virtue of name equivalence, $\quotep{\procn{x}} \nameeq x$; so, the
notation is consistent with previous definitions.

Further, because names have structure it is possible to effect
substitutions on the basis of that structure. This means we need to
upgrade our notation for substitutions, which we accomplish by
adapting comprehension notation. Thus,

\begin{mathpar}
  P\{ y / x : x \in S \}
\end{mathpar}

is interpreted to mean the process derived from P by replacing (in a
capture-avoiding manner) each occurrence of $x$ in $S$ by $y$. For example,

\begin{mathpar}
  P\{ \quotep{\procn{x}|\procn{x}} / x : x \in \freenames{P} \}
\end{mathpar}

will replace each (occurrence) of a free name $x$ in $P$ by
$\quotep{\procn{x}|\procn{x}}$.

Also, we will avail ourselves of the notation $x^{L}$ and $x^{R}$ to
denote injections of a name into disjoint copies of the name
space. There are numerous ways to accomplish this. One example can be
found in \cite{MeredithR05}. This notation overloads to vectors of
names: $\vec{x}^{\pi} := (x_{i}^{\pi} \; : \; 0 \leq i < |\vec{x}| )$ where $\pi \in \{L,R\}$.

We also use $P^{\Box} := P|\Box$.

In \cite{MeredithR05} an interpretation of the new operator is
given. It turns out that there are several possible interpretations
all enjoying the requisite algebraic properties of the operator (see
\cite{milner91polyadicpi}). We will therefore make liberal use of
$(\nu\; \vec{x})P$.

% subsection the_syntax_and_semantics_of_the_notation_system (end)   

\section{Interpretation of QM}
\subsection{Supporting definitions}
\subsubsection{Multiplication}
\begin{mathpar}
  \quotep{Q} \cdot \quotep{R} := \quotep{Q|R}
  \and \\
  \quotep{Q} \cdot P := P\{ \quotep{Q|R} / \quotep{R} : \quotep{R} \in \freenames{P} \}
\end{mathpar}

\paragraph{Discussion}
The first line needs little explanation. The second line says that
each free name of the process is replaced with the multiplication of
that name by the scalar. Multiplication of a scalar (name) by a state
(process) results in a process all the names of which have been `moved
over' by parallel composition with the process the scalar
quotes. There is a subtlety that the bound names have to be
manipulated so that multiplied names aren't accidentally
captured. There are many ways to achieve this.

\begin{remark}\label{rem:multiplication_identities}
  The reader is invited to verify that for all $x,y,z \in \QProc$ and $P \in \Proc$
  \begin{mathpar}
    x \cdot \quotep{0} \equiv x 
    \and
    x \cdot y \equiv y \cdot x
    \and
    x \cdot (y \cdot z) \equiv (x \cdot y) \cdot z
    \and \\
    \quotep{0} \cdot P \equiv P
    \and \\
    x \cdot (y \cdot P) \equiv (x \cdot y) \cdot P
    \and \\
    x \cdot (P|Q) \equiv (x \cdot P) | (x \cdot Q)
    \and \\    
  \end{mathpar}
\end{remark}

\subsubsection{Tensor product}

We define a tensor product on processes by structural induction.

\paragraph{Tensor of sums} First note that all summations, including
$\pzero$ and sequence, can be written $\Sigma_{i} x_{i}.A_{i} +
\Sigma_{j} x_{j}.C_{j}$, where we have grouped input-guarded processes
together and output-guarded processes together.

Thus, we can define the tensor product of two summations, $N_{1}\otimes N_{2}$, where

\begin{mathpar}
  N_{1} := \Sigma_{i} x_{i}.A_{i} + \Sigma_{j} x_{j}.C_{j}
  \and
  N_{2} := \Sigma_{i'} y_{i'}.B_{i'} + \Sigma_{j'} y_{j'}.D_{j'} 
\end{mathpar}

as follows.

\begin{mathpar}
  \Sigma_{i} x_{i}.A_{i} + \Sigma_{j} x_{j}.C_{j} \otimes \Sigma_{i'}
  y_{i'}.B_{i'} + \Sigma_{j'} y_{j'}.D_{j'} 
  \and \\
  := \; \Sigma_{i} \Sigma_{i'} \quotep{\stackrel{\vee}{x_{i}}| \stackrel{\vee}{y_{i'}}}.(A_{i}\otimes B_{i'}) \; | \; \Sigma_{i'} \Sigma_{i} \quotep{\stackrel{\vee}{y_{i'}}|\stackrel{\vee}{x_{i}}}.(B_{i'}\otimes A_{i})
  \and
  \;\; | \;\; \Sigma_{j} \Sigma_{j'} \quotep{\stackrel{\vee}{x_{j}}|\stackrel{\vee}{y_{j'}}}.(A_{j}\otimes B_{j'}) \; | \; \Sigma_{j'} \Sigma_{j} \quotep{\stackrel{\vee}{y_{j'}}|\stackrel{\vee}{x_{j}}}.(B_{j'}\otimes A_{j})
\end{mathpar}

\begin{remark}
  Do we need to $x^{L}$ and $y^{R}$ for this construction as well?
\end{remark}

\paragraph{Tensor of parallel compositions} Next, we distribute tensor
over par.

\begin{mathpar}
  P_{1}|P_{2} \otimes Q_{1}|Q_{2} := (P_{1} \otimes Q_{1}) | (P_{1}
  \otimes Q_{2}) | (P_{2} \otimes Q_{1}) | (P_{2} \otimes Q_{2})
\end{mathpar}

\paragraph{Tensor with dropped names} We treat tensor of a
process with a dropped name as parallel composition.

\begin{mathpar}
  P \otimes \dropn{x} := P | \dropn{x}
\end{mathpar}

\paragraph{Tensor of agents}

Finally, we need to define tensor on agents. Note that the definition
of tensor on normal products only tensors inputs with inputs and
outputs with outputs. Thus, we only have to define the operation on
``homogeneous'' pairings.

\begin{mathpar}
  (\vec{x})P \otimes (\vec{y})Q
  \and \\
  := (x_{0}^{L}|y_{0}^{R},\ldots,x_{0}^{L}|y_{n}^{R},\ldots,x_{m}^{L}|y_{0}^{R},\ldots,x_{m}^{L}|y_{n}^R)(P\{ \vec{x}^{L}/\vec{x}\} \otimes Q \{ \vec{y}^{R}/\vec{y}\})
  \and \\
  \clift{\vec{P}} \otimes \clift{\vec{Q}}
  \and \\
  := \clift{P_{0}\otimes Q_{0},\ldots,P_{0}\otimes Q_{n},\ldots,P_{m}\otimes Q_{0},\ldots,P_{m}\otimes Q_{n}}
\end{mathpar}

\begin{remark}
  Observe that arities of tensored abstractions matches arities of
  tensored concretions if the original arities matched. Note also that
  the length of the arities corresponds to the increase in dimension
  we see in ordinary vector space tensor product.
\end{remark}

\begin{remark}
  Operationally, this definition distributes the tensor down to
  components ``linked'' by summation. Tensor over summation is
  intriguing in that it mixes names. Moreover, as a consequence of the
  way it mixes names we have the identities for all $x \in \QProc$ and
  $P,Q \in \Proc$

  \begin{mathpar}
    (x \cdot P) \otimes Q \equiv x \cdot (P \otimes Q) \equiv P \otimes (x \cdot Q)
    \and
    P \otimes \pzero \equiv P
  \end{mathpar}

  that the reader is invited to verify.
\end{remark}

\subsubsection{Annihilation}
\begin{mathpar}
  P^{\perp} := \{ Q | \forall R. P|Q \red^{*} R \Rightarrow R \red^{*} \pzero \}
  \and \\
  P^{\underline{\perp}} := \Sigma_{Q \in P^{\perp}} \quotep{Q}?(y).(\dropn{y}|Q) | \Sigma_{Q \in P^{\perp}} \quotep{Q}\clift{\Box}
\end{mathpar}

\paragraph{Discussion} The reader will note that $P^{\perp}$ is a
\emph{set} of processes, while $P^{\underline{\perp}}$ is a
\emph{context}. We call the set $P^{\perp}$ the \emph{annihilators} of
$P$. The parallel composition of a process in the annihilators of $P$
with $P$ will result in a process, the state space of which has all
paths eventually leading to $\pzero$. Execution may endure loops; but
under reasonable conditions of fairness (naturally guaranteed under
most notions of bisimulation) such a composite process cannot get
stuck in such a loop and will, eventually pop out and terminate.

The context $P^{\underline{\perp}}$ is ready and willing to ``take the
$P$ out of'' the process to which it is applied. It will effectively
transmit the code of the process to which it is applied to one of the
annihilators and run the process against it.

\subsubsection{Evaluation}
We fix $M$ a domain of fully abstract interpretation with an equality
coincident with bisimulation. We take $\meaningof{\cdot} : \Proc \to
M$ to be the map interpreting processes and $\nmeaningof{\cdot} : \M
\to Proc$ to be the map running the other way. Then we define

\begin{mathpar}
  \int P := \nmeaningof{\meaningof{P}}
\end{mathpar}

\paragraph{Discussion}
There are many fully abstract interpretations of Milner's
$\pi$-calculus. Any of them can be used as a basis for interpreting
the reflective calculus here. Equipped with such a domain it is
largely a matter of grinding through to check that the Yoneda
construction for the normalization-by-evaluation program can be
extended to this setting.

\begin{remark}
  The reader is invited to verify that $\int (P^{\underline{\perp}}[P]) = 0$.
\end{remark}

\subsection{Quantum mechanics}

Table \ref{tbl:core_qm_op_defns} gives the core operational definitions

\begin{table}[htp]\label{tbl:core_qm_op_defns}
  \center{
    \fbox{
      \begin{tabular}{c|c}
        quantum mechanics & process calculus \\
        \hline
        scalar & $x := \quotep{P}$ \\
        state vector & $\state{P} := P$ \\
        dual & $\state{P}^{*} := \event{P^{\underline{\perp}}} := \quotep{P^{\underline{\perp}}}[-]$ \\
        matrix & $ \Sigma_{\alpha} \state{P_{\alpha}}x_{\alpha}\event{Q_{\alpha}}$ \\
        vector addition & $\state{P} + \state{Q} := \state{P | Q}$ \\
        tensor product & $\state{P} \otimes \state{Q} := \state{P \otimes Q}$ \\
        inner product & $\innerprod{P}{Q} := \quotep{\int P^{\underline{\perp}}[Q]}$ \\
      \end{tabular}
    }
  }
  \caption{QM - operational definitions}
\end{table}

where

\begin{mathpar}
  \prmatrix{P}{Q} := \fprmatrix{P}{\quotep{\pzero}}{Q}
  \and
  \fprmatrix{P}{x}{Q} := (\state{P},x,\event{Q})
  \and
  (\fprmatrix{P}{x}{Q})(\state{R}) := x \cdot \innerprod{Q}{R} \cdot \state{P}
  \and
  (\fprmatrix{P}{x}{Q})(\event{R}) := x \cdot \innerprod{R}{P} \cdot \event{Q}
\end{mathpar}

\paragraph{Discussion}
As promised: vectors (aka states) are represented as processes; duals
as contextual duals; inner product definition should be compared with
standard inner product definition for ....

\begin{remark}
  Assuming $\int (P^{\underline{\perp}}[P]) = 0$, the reader is
  invited to verify that $(\fprmatrix{P}{x}{P})(\state{P}) = x \cdot \state{P}$.
\end{remark}

\begin{remark}
  The reader is invited to verify that $\innerprod{P}{Q}$ could
  equally well have been written $\quotep{\int \stackrel{\vee}{x}}$
  where $x = \event{P^{\underline{\perp}}}(Q)$.

  One of the motivations for this remark is that there is another way
  to factor these operations. We could package up evaluation in the dual:

  \begin{mathpar}
    \state{P}^{*} := \event{\int P^{\underline{\perp}}} := \quotep{\int P^{\underline{\perp}}}[-]
  \end{mathpar}

  and then have inner product defined by
  
  \begin{mathpar}
    \innerprod{P}{Q} := \event{P}(Q)
  \end{mathpar}

  Hopefully, experience with the calculations will provide guidance on
  the best factoring.
\end{remark}

\begin{remark}
  Assuming $\int (P^{\underline{\perp}}[P]) = 0$, the reader is
  invited to verify that $\forall P,Q. (\prmatrix{0}{Q})(\state{0}) =
  \state{0}$ and dually $(\prmatrix{P}{0})(\event{0}) = \event{0}$.
\end{remark}

\begin{remark}
  i'm a little worried that i don't (yet) have proper support for
  complex conjugacy. But, the observation above may give us a
  clue. According to Abramsky, it must be the case that the scalars
  are iso to the homset of the identity for the tensor -- which the
  observation above characterizes. 

  For now, we will simply bookmark the notion with $\overline{x}$.
\end{remark}

\subsubsection{Adjointness}

We need to give a definition of $(\cdot)^{\dagger}$ for matrices. The
obvious candidate definition is
\begin{mathpar}
(\Sigma_{\alpha}\fprmatrix{P_{\alpha}}{x_{\alpha}}{Q_{\alpha}})^{\dagger}
= \Sigma_{\alpha}\fprmatrix{(Q_{\alpha}^{\underline{\perp}})^{*}}{\overline{x}_{\alpha}}{P_{\alpha}^{\underline{\perp}}} 
\end{mathpar}

But, $(Q_{\alpha}^{\underline{\perp}})^{*}$ requires a name along
which to communicate the process to achieve the context application.

\subsubsection{Basis for a basis}
If processes label states and ``addition'' of states (a.k.a. vector
addition) is interpreted as parallel composition, what corresponds to
notions of linear independence and basis? Here, we recall that Yoshida
has developed a set of \emph{combinators} for an asynchronous verison
of Milner's $\pi$-calculus. These are a finite set of processes such
any process can be expressed as parallel composition of these
combinators together with liberal uses of the new operator and
replication. We can simply give a translation of these into the
present calculus and have reasonable expectation that the property
carries over. That is, that the resultant set allows to express all
processes via parallel composition. Note, however, that there is no
new operator or replication in this calculus. As a result, we expect
that the corresponding set is actually infinite. That is, we expect
that the space is actually infinite dimensional.

\begin{remark}
  The attentive reader may be a bit concerned. Certainly, the
  collection $S$, $K$ and $I$ is a finite set of
  combinators. Shouldn't we expect to see a finite set of combinators
  for an effectively equivalent system? i am very sympathetic to this
  critique and feel it warrants full attention. On the other hand, i
  also have in mind the following analogy. The natural numbers, as a
  monoid under addition, has exactly $1$ generator, while the natural
  numbers, as a monoid under multiplication, has countably many
  generators (the primes). We observe that the application of the
  lambda calculus is much less resource sensitive than the parallel
  composition of the $\pi$-calculus. Could it be the case that we have
  an analogy of the form
  
  \begin{mathpar}
    m + n : MN :: m*n : M|N
  \end{mathpar}

  giving a similar blow up in the set of ``primes''?  This is such a
  wonderful thought that, even if it's not true, i think it's worth
  writing down.
\end{remark}
 

\documentclass[12pt]{llncs}
%\documentclass{jktr}

\usepackage[pdftex]{hyperref}                   
\usepackage {listings}
\usepackage {mathpartir}
\usepackage{bcprules}
%\usepackage{listings}
                       
\usepackage{graphicx} 
%\usepackage[margins=2.5cm,nohead,nofoot]{geometry}
%\usepackage{geometry}
\usepackage{amsfonts}
\usepackage{amstext}
\usepackage{latexsym}
\usepackage{amssymb}
\usepackage{color}


%\include{myPreamble}
\include{qm2pi.local} 

%\ifpdf
%\usepackage[pdftex]{graphicx}
%\else
%\usepackage{graphicx}
%\fi

 % \ifpdf
%  \usepackage{pdfsync}
%  \if


%\title{Brief Article}
%\author{David F. Snyder}
%\author{L.G. Meredith}

%\address{Dept. of Math., Texas State University--San Marcos, San Marcos, TX 78666}
       
\pagestyle{empty}


\begin{document}

\lstset{language=[Objective]Caml,frame=shadowbox}

\input{qm2pi.front}

% section front matter (end)

\input{qm2pi.intro} 
 
% section introduction (end)

% \input{qm2pi.knotations} 

% section notation (end)

\input{qm2pi.process.calculi} 

% section concurrent_process_calculi_and_spatial_logics_ (end)
    
%\input{qm2pi.knots2pi} 

%\input{qm2pi.trefoil} 

%\input{qm2pi.mainthm} 

% subsection basic_interpretation (end)

%\input{qm2pi.rho.presentation} 
\subsection{The syntax and semantics of the notation system}\label{sub:the_syntax_and_semantics_of_the_notation_system} % (fold)

We now summarize a technical presentation of the calculus that
embodies our theory of dynamics. The typical presentation of such a
calculus follows the style of giving generators and relations on
them. The grammar, below, describing term constructors, freely
generates the set of processes, $\Proc$. This set is then quotiented
by a relation known as structural congruence and it is over this set
that the notion of dynamics is expressed. This presentation is
essentially that of \cite{MeredithR05} with the addition of
polyadicity and summation. For readability we have relegated some of
the technical subtleties to an appendix.

\subsubsection{Process grammar}\label{subsub:process_grammar}

\begin{mathpar}
  \inferrule* [lab=synchronization] {} {{M} \bc \pzero \;|\; x?F \;|\; x!C }
  \and
  \inferrule* [lab=abstraction] {} {{F} \bc (x)P}
  \and
  \inferrule* [lab=concretion] {} {{C} \bc \langle Q \rangle}
  \and
  \inferrule* [lab=process] {} {{P,Q} \bc M \;| \;P|Q \;|\; @{x}}
  \and
  \inferrule* [lab=name] {} {{x} \bc \quotep{P}}
\end{mathpar} 

Note that $\vec{x}$ (resp. $\vec{P}$) denotes a vector of names
(resp. processes) of length $|\vec{x}|$ (resp. $|\vec{P}|$). We adopt
the following useful abbreviations.

\begin{mathpar}
   x?(\vec{y}).P := x.(\vec{y})P \and  x\clift{\vec{P}} := x.\clift{\vec{P}}
   \and x!(y) := \lift{x}{\dropn{y}}
   \and \Pi_{i=0}^{n-1}P_i := P_0 | \ldots | P_{n-1}
\end{mathpar}

\subsubsection{Structural congruence}

\paragraph{Free and bound names and alpha-equivalence.} At the
core of structural equivalence is alpha-equivalence which identifies
process that are the same up to a change of variable. Formally, we
recognize the distinction between free and bound names. The free names
of a process, $\freenames{P}$, may be calculated recursively as
follows:

\begin{mathpar}
\freenames{\pzero} := \emptyset
  \and \\
  \freenames{x?(y).P} := \{ x \} \cup (\freenames{P} \setminus \{ y \})
  \and 
  \freenames{x!\langle P \rangle} := \{ x \} \cup \{ P \} 
  \and \\
  \freenames{P|Q} := \freenames{P} \cup \freenames{Q}
  \and \\
  \freenames{@{x}} := \{ x \}
\end{mathpar}

$\pi$
$\quotep{\pi}$

$\freenames{-} : \pi \to \mathcal{P}(\quotep{\pi})$

\begin{eqnarray*}
  \freenames{\pzero} & := & \emptyset \\
  \freenames{x?(y).P} & := & \{ x \} \cup (\freenames{P} \setminus \{ y \}) \\
  \freenames{x!\langle P \rangle} & := & \{ x \} \cup \{ P \} \\
  \freenames{P|Q} & := & \freenames{P} \cup \freenames{Q} \\
  \freenames{\dropn{x}} & := & \{ x \}
\end{eqnarray*}

The bound names of a process, $\boundnames{P}$, are those names occurring in $P$
that are not free. For example, in $x?(y).0$, the name $x$ is free, while $y$ is bound.

\begin{mathpar}
  \inferrule* [lab=monoidal-laws] {} { P|Q \equiv Q|P \and P|0 \equiv P \and P|(Q|R) \equiv (P|Q)|R }
\end{mathpar}

\begin{mathpar}
  \inferrule* [lab=alpha-equivalence] {} { (x)P \equiv (y)P\{y/x\} \and y \not\in \freenames{P} }
\end{mathpar}

\begin{definition}
Then two processes, $P,Q$, are alpha-equivalent if $P = Q\{\vec{y}/\vec{x}\}$ for
some $\vec{x} \in \boundnames{Q},\vec{y} \in \boundnames{P}$, where $Q\{\vec{y}/\vec{x}\}$
denotes the capture-avoiding substitution of $\vec{y}$ for $\vec{x}$ in $Q$.
\end{definition}

\begin{definition}
  The {\em structural congruence} \cite{SangiorgiWalker} , $\equiv$,
  between processes is the least congruence containing
  alpha-equivalence, satisfying the abelian monoid laws
  (associativity, commutativity and $\pzero$ as identity) for parallel
  composition $|$ and for summation $+$.
\end{definition}

\subsection{Name equivalence}

We take name equivalence, written $\nameeq$, to be the smallest
equivalence relation generated by the following rules.

\begin{mathpar}
\inferrule*[lab=Quote-drop]
{ }
{ \quotep{@{x}} \nameeq x }

\inferrule*[lab=Struct-equiv]
{ P \scong Q }
{ \quotep{P} \nameeq \quotep{Q} }
\end{mathpar}

The astute reader will have noticed that the mutual recursion of names
and processes imposes a mutual recursion on alpha-equivalence and
structural equivalence via name-equivalence. Fortunately, all of this
works out pleasantly and we may calculate in the natural way, free of
concern. The reader interested in the details is referred to the
appendix \ref{appendix:rho_details}.

\subsection{Substitution}

We use $\Proc$ for the set of processes, $\QProc$ for the set of
names, and $\id{\{}\vec{y} / \vec{x} \id{\}}$ to denote partial maps,
$s : \QProc \rightarrow \QProc$. A map, $s$ lifts, uniquely, to a map
on process terms, $\widehat{s} : \Proc \rightarrow \Proc$ by the
following equations.

\begin{mathpar}
  (0) \psubstp{Q}{P} := 0 \\
  (R \juxtap S) \psubstp{Q}{P}
  :=    
  (R)\psubstp{Q}{P} \juxtap (S) \psubstp{Q}{P} \\
  (x?(y).R) \psubstp{Q}{P}    
  :=    
  (x)\substp{Q}{P} (z)\concat( (R \psubstn{z}{y}) \psubstp{Q}{P} ) \\
  (\lift{x}{R}) \psubstp{Q}{P}  
  :=
  \lift{(x)\substp{Q}{P}}{ R \psubstp{Q}{P} } \\
%   (\dropn{x})  \psubstp{Q}{P}       
%   := 
%   \left\{ 
%     \begin{array}{ccc} 
%       \dropn{\quotep{Q}} & & x \nameeq \quotep{P} \\
%       \dropn{x} & & otherwise \\
%     \end{array}
%   \right. 
  (\dropn{x})  \psubstp{Q}{P}       
  := 
  \left\{ 
    \begin{array}{ccc} 
      Q & & x \nameeq \quotep{P} \\
      \dropn{x} & & otherwise \\
    \end{array}
  \right.
\end{mathpar}
 

where

\begin{eqnarray}
  (x)\id{\{} \lpquote Q \rpquote / \lpquote P \rpquote \id{\}}            = 
  \left\{ 
    \begin{array}{ccc}
      \lpquote Q \rpquote & & x \nameeq \lpquote P \rpquote \\
      x & & otherwise \\
    \end{array}
  \right. \nonumber
\end{eqnarray}

and $z$ is chosen distinct from $\quotep{P}$, $\quotep{Q}$, the free
names in $Q$, and all the names in $R$. Our $\alpha$-equivalence will
be built in the standard way from this substitution.

\begin{remark}\label{rem:no_self_referential_names}
  One consequence of these definitions is that $\forall P. \quotep{P}
  \not\in \freenames{P}$.
\end{remark}

\subsection{ Dynamic quote: an example }

Anticipating something of what's to come, consider applying the
substitution, $\widehat{\id{\{}u / z \id{\}}}$, to the following pair
of processes, $\lift{w}{y!(z)}$ and $w[ \lpquote y!(z) \rpquote ]$.

\begin{eqnarray}
	\lift{w}{y!(z)}\widehat{\id{\{}u / z \id{\}}}
		& = &
		\lift{w}{y!(u)} \nonumber\\
	w[ \lpquote y!(z) \rpquote ] \widehat{ \id{\{}u / z \id{\}} }
		& = &
		w[ \lpquote y!(z) \rpquote ] \nonumber
\end{eqnarray}

Because the body of the process between quotes is impervious to
substitution, we get radically different answers. In fact, by
examining the first process in an input context,
e.g. $x?(z).\lift{w}{y!(z)}$, we see that the process under the lift
operator may be shaped by prefixed inputs binding a name inside it. In
this sense, the lift operator will be seen as a way to dynamically
construct processes before reifying them as names.

Finally equipped with these standard features we can present the
dynamics of the calculus.

\subsubsection{Operational semantics} 

Finally, we introduce the computational dynamics. What marks these
algebras as distinct from other more traditionally studied algebraic
structures, e.g. vector spaces or polynomial rings, is the manner in
which dynamics is captured. In traditional structures, dynamics is typically
expressed through morphisms between such structures, as in linear maps
between vector spaces or morphisms between rings. In algebras
associated with the semantics of computation, the dynamics is
expressed as part of the algebraic structure itself, through a
reduction reduction relation typically denoted by $\red$. Below, we
give a recursive presentation of this relation for the calculus used
in the encoding.

$\red \subseteq \pi \times \pi$
$\red : \pi \to \mathcal{P}(\pi)$

\begin{mathpar}
  \inferrule* [lab=Comm] { \textsf{match}( x_{src}, x_{trgt} ) } { x_{trgt}?(y)P \; | \; x_{src}!\langle {Q} \rangle \red P\{\quotep{Q}/y}\} }
  \and \\
  \inferrule* [lab=Par] {{P} \red {P}'} {{{P} | {Q}} \red {{P}' | {Q}}}
  \and
  \inferrule* [lab=Equiv]{{{P} \scong {P}'} \andalso {{P}' \red {Q}'} \andalso {{Q}' \scong {Q}}}{{P} \red {Q}}
\end{mathpar}

\begin{eqnarray*}
  match_{\equiv} (\quotep{P},\quotep{Q}) & := & P \equiv Q \\
  match_{\dagger}(\quotep{P},\quotep{Q}) & := & \forall R. P|Q \red^{*} R => R \red^{*} 0 \\
  match_{K}(\quotep{P},\quotep{Q}) & := & K \mbox{ for some context } K
\end{eqnarray*}

$u?(x)P | u!\langle Q \rangle \red P\{\quotep{Q}/x\}$

%We write $\wred$ for $\red^*$, and $P\red$ if $\exists Q $ such that $ P \red Q$.
We write $P\red$ if $\exists Q $ such that $ P \red Q$ and $P\not\red$, otherwise.

\section{Replication}

As mentioned before, it is known that replication (and hence
recursion) can be implemented in a higher-order process algebra
\cite{SangiorgiWalker}. As our first example of calculation with the
machinery thus far presented we give the construction explicitly in
the {\rhoc}.

\begin{eqnarray}
	D_{x} & := & \prefix{x}{y}{(\binpar{\outputp{x}{y}}{@{y}})} \nonumber\\
	\bangp_{x}{P} & := & \binpar{{x}!\langle{\binpar{D_{x}}{P}}\rangle}{D_{x}} \nonumber
\end{eqnarray}

\begin{eqnarray}
	\bangp_{x}{P} & & \nonumber\\
	=
	& {x}!\langle{(\prefix{x}{y}{(\outputp{x}{y} | @{y})) | P}}\rangle 
	      | \prefix{x}{y}{(\outputp{x}{y} | @{y})} & \nonumber\\
	\red
	& (\outputp{x}{y} | @{y})\substn{\quotep{(\prefix{x}{y}{(@{y} | \outputp{x}{y})) | P}}}{y} & \nonumber\\
	=
	& \outputp{x}{\quotep{(\prefix{x}{y}{(\outputp{x}{y} | @{y})) | P}}}
	  | {(\prefix{x}{y}{(\outputp{x}{y} | @{y})) | P}} & \nonumber\\
	\red
	& \ldots & \nonumber\\
	\red^*
	& P | P | \ldots & \nonumber
\end{eqnarray}

Of course, this encoding, as an implementation, runs away, unfolding
$\bangp{P}$ eagerly. A lazier and more implementable replication
operator, restricted to input-guarded processes, may be obtained as follows.

\begin{eqnarray}
\bangp{\prefix{u}{v}{P}} 
	:= 
	\binpar{\lift{x}{\prefix{u}{v}{(\binpar{D(x)}{P})}}}{D(x)} \nonumber
\end{eqnarray}

\begin{remark}
  Note that the lazier definition still does not deal with summation
  or mixed summation (i.e. sums over input and output). The reader is
  invited to construct definitions of replication that deal with these
  features. 

  Further, the definitions are parameterized in a name, $x$. Can you,
  gentle reader, make a definition that eliminates this parameter and
  guarantees no accidental interaction between the replication
  machinery and the process being replicated -- i.e. no accidental
  sharing of names used by the process to get its work done and the
  name(s) used by the replication to effect copying. This latter
  revision of the definition of replication is crucial to obtaining
  the expected identity $!!P \sim !P$.
\end{remark}

\begin{remark}\label{rem:paradoxical_combinator}
  The reader familiar with the lambda calculus will have noticed the
  similarity between $D$ and the paradoxical combinator.

  [Ed. note: the existence of this seems to suggest we have to be more
  restrictive on the set of processes and names we admit if we are to
  support no-cloning.]
\end{remark}

\subsubsection{Bisimulation}

The computational dynamics gives rise to another kind of equivalence,
the equivalence of computational behavior. As previously mentioned
this is typically captured \emph{via} some form of bisimulation.

% The notion we use in this paper is weak barbed bisimulation
% \cite{milner91polyadicpi}.

The notion we use in this paper is derived from weak barbed
bisimulation \cite{milner91polyadicpi}. 

\begin{definition}
An \emph{observation relation}, $\downarrow_{\mathcal N}$, over a set
of names, $\mathcal N$, is the smallest relation satisfying the rules
below.

\infrule[Out-barb]{y \in {\mathcal N}, \; x \nameeq y}
		  {\outputp{x}{v} \downarrow_{\mathcal N} x}
\infrule[Par-barb]{\mbox{$P\downarrow_{\mathcal N} x$ or $Q\downarrow_{\mathcal N} x$}}
		  {\binpar{P}{Q} \downarrow_{\mathcal N} x}

We write $P \Downarrow_{\mathcal N} x$ if there is $Q$ such that 
$P \wred Q$ and $Q \downarrow_{\mathcal N} x$.
\end{definition}

\begin{definition}
%\label{def.bbisim}
An  ${\mathcal N}$-\emph{barbed bisimulation} over a set of names, ${\mathcal N}$, is a symmetric binary relation 
${\mathcal S}_{\mathcal N}$ between agents such that $P\rel{S}_{\mathcal N}Q$ implies:
\begin{enumerate}
\item If $P \red P'$ then $Q \wred Q'$ and $P'\rel{S}_{\mathcal N} Q'$.
\item If $P\downarrow_{\mathcal N} x$, then $Q\Downarrow_{\mathcal N} x$.
\end{enumerate}
$P$ is ${\mathcal N}$-barbed bisimilar to $Q$, written
$P \wbbisim_{\mathcal N} Q$, if $P \rel{S}_{\mathcal N} Q$ for some ${\mathcal N}$-barbed bisimulation ${\mathcal S}_{\mathcal N}$.
\end{definition}

$\mathcal{R} \subseteq \pi \times \pi$

$P \mathcal{R} Q => \forall P'. P \red P' \Rightarrow \exists Q'. Q \red Q', P' \mathcal{R} Q'$

$P \vdash x \Rightarrow Q \vdash x$

\begin{mathpar}
  \inferrule*[lab=Out-barb]{x \nameeq y}{{y}!\langle{Q}\rangle \vdash x}
  \and
  \inferrule*[lab=Par-barb]{\mbox{$P\vdash x$ or $Q\vdash x$}}{\binpar{P}{Q} \vdash x}
\end{mathpar}

\subsubsection{Contexts}

One of the principle advantages of computational calculi like the
$\pi$-calculus is a well-defined notion of context,
contextual-equivalence and a correlation between
contextual-equivalence and notions of bisimulation. The notion of
context allows the decomposition of a process into (sub-)process and
its syntactic environment, its context. Thus, a context may be
thought of as a process with a ``hole'' (written $\Box$) in it. The
application of a context $M$ to a process $P$, written $M[P]$, is
tantamount to filling the hole in $M$ with $P$. In this paper we do
not need the full weight of this theory, but do make use of the notion
of context in the proof the main theorem. 

\begin{mathpar}
  \inferrule* [lab=summation] {} {{M_{M},M_{N}} \bc \Box \;|\; x.M_{A} \;|\; M_{M}+M_{N}}
  \and
  \inferrule* [lab=agent] {} {{M_{A}} \bc (\vec{x})M_{P} \;| \; \clift{P_0,\ldots,M_{P},\ldots,P_N}}
  \and \\
  \inferrule* [lab=process] {} {{M_{P}} \bc M_{N} \;| \;P|M_{P} }
\end{mathpar} 

\begin{mathpar}
  \inferrule* [lab=sychronization] {} {M_{N} \bc \Box \;|\; x?M_{F} \;|\; x!M_{C}}
  \and
  \inferrule* [lab=abstraction] {} {{M_{F}} \bc (x)M_{P} }
  \and
  \inferrule* [lab=concretion] {} {{M_{C}} \bc \langle M_{P} \rangle }
  \and \\
  \inferrule* [lab=process] {} {{M_{P}} \bc M_{N} \;| \;P|M_{P} }
\end{mathpar}

\begin{definition}[contextual application] Given a context $M$, and
  process $P$, we define the \emph{contextual application}, $M[P] :=
  M\{P/\Box\}$. That is, the contextual application of M to P is the
  substitution of $P$ for $\Box$ in $M$.
\end{definition}

$\meaningof{-} : L \to \mathcal{P}(\pi)$

\begin{mathpar}
  \inferrule* [lab=collection] {} {\meaningof{true} = \pi, \and \meaningof{~E} = \pi \setminus \meaningof{E}, \and \meaningof{E_{1} \& E_{2}} = \meaningof{E_{1}} \cap \meaningof{E_{2}}}
\end{mathpar}

\begin{mathpar}
  \inferrule* [lab=structure] {} {\meaningof{0} = \{ P \in \pi | P \equiv 0 \}, \and \\ \meaningof{E_1 | E_2} = \{ P \in \pi | P \equiv P_{1} | P_{2}, P_{1} \in \meaningof{E_{1}}, P_{2} \in \meaningof{E_2}\} }
\end{mathpar}

\begin{mathpar}
 \inferrule* [lab=behavior] {} {\meaningof{\langle a?b \rangle E} = \{ P \in \pi | P \equiv Q | u?(y)P', \\ \and \\\\ \and \\ \;\;\; u \in \meaningof{a}, \forall z.P'\{z/y\} \in \meaningof{E\{z/b\}}\}, \and \\ \meaningof{a!E} = \{ P \in \pi | P \equiv Q | x!\langle P' \rangle, x \in \meaningof{a} P' \in \meaningof{E}\} }
\end{mathpar}

\begin{mathpar}
 \inferrule* [lab=nominal] {} {\meaningof{\quotep{E}} = \{ \quotep{P} \in \quotep{\pi} | P \in \meaningof{E} \}, \and \meaningof{\quotep{P}} = \{ \quotep{Q} \in \quotep{\pi} | P \equiv Q \} \and \\ \meaningof{@\quotep{E}} = \{ P \in \pi | P \equiv @x, x \in \meaningof{E} \}}
\end{mathpar}

\begin{eqnarray*}
  \\
  \meaningof{-} : TS \to ST
\end{eqnarray*}

\begin{eqnarray*}
  \\
  L : TS \to ST
\end{eqnarray*}

\begin{eqnarray*}
  \\
  P \models E \iff P \in \meaningof{E}
\end{eqnarray*}

\begin{eqnarray*}
  P \approx_{L} Q \iff \forall E \in L. P \models E \iff Q \models E
\end{eqnarray*}

\begin{eqnarray*}
  P \approx_{K} Q
\end{eqnarray*}

\begin{eqnarray*}
  P \approx Q
\end{eqnarray*}

$\approx_{K} = \approx = \approx_{L}$

\subsubsection{Contextual duality}

Note that contexts extend the quotation operation to a family of
operations from processes to names. Given a context, $M$, we can
define a \emph{nominal context}, $\quotep{M}$ by $\quotep{M}[P] :=
\quotep{M[P]}$. To foreshadow what is to come we observe that these
operations enjoy a duality with processes very much like the duality
between vectors and maps from vectors to scalars.

Further, because the calculus is essentially higher-order, we have a
correspondence between contexts and processes. More specifically,
given a name $x$ and a context $M$ we can construct $M^{*}_{x}$ such
that 

\begin{mathpar}
  M^{*}_{x} | \lift{x}{P} \red M[P]
\end{mathpar}

namely,

\begin{mathpar}
  M^{*}_{x} := x?(u).M[\dropn{u}]
\end{mathpar}

The dependence of $M^{*}_{x}$ on a name makes it an abstraction, 

\begin{mathpar}
  M^{*} := (x)x?(u).M[\dropn{u}]
\end{mathpar}

\subsection{Additional notation}

It will sometimes be convenient to denote the process a name
quotes. We already have the notation $x = \quotep{P}$, but it will be
convenient to introduce an alternate notation, $\procn{x}$, when we
want to emphasize the connection to the use of the name. Note that, by
virtue of name equivalence, $\quotep{\procn{x}} \nameeq x$; so, the
notation is consistent with previous definitions.

Further, because names have structure it is possible to effect
substitutions on the basis of that structure. This means we need to
upgrade our notation for substitutions, which we accomplish by
adapting comprehension notation. Thus,

\begin{mathpar}
  P\{ y / x : x \in S \}
\end{mathpar}

is interpreted to mean the process derived from P by replacing (in a
capture-avoiding manner) each occurrence of $x$ in $S$ by $y$. For example,

\begin{mathpar}
  P\{ \quotep{\procn{x}|\procn{x}} / x : x \in \freenames{P} \}
\end{mathpar}

will replace each (occurrence) of a free name $x$ in $P$ by
$\quotep{\procn{x}|\procn{x}}$.

Also, we will avail ourselves of the notation $x^{L}$ and $x^{R}$ to
denote injections of a name into disjoint copies of the name
space. There are numerous ways to accomplish this. One example can be
found in \cite{MeredithR05}. This notation overloads to vectors of
names: $\vec{x}^{\pi} := (x_{i}^{\pi} \; : \; 0 \leq i < |\vec{x}| )$ where $\pi \in \{L,R\}$.

We also use $P^{\Box} := P|\Box$.

In \cite{MeredithR05} an interpretation of the new operator is
given. It turns out that there are several possible interpretations
all enjoying the requisite algebraic properties of the operator (see
\cite{milner91polyadicpi}). We will therefore make liberal use of
$(\nu\; \vec{x})P$.

% subsection the_syntax_and_semantics_of_the_notation_system (end)   

\input{qm2pi.qmops} 

\input{qm2pi.sterngerlach} 

\input{qm2pi.metric} 

% section concurrent_process_calculi (end)

%\input{qm2pi.proofsketch}

% section proof sketch (end)

%\input{qm2pi.slviaknots} 

% section spatial logic via knots (end)

\input{qm2pi.conclusion}

% section conclusion (end)

%\input{qm2pi.dtcodes} 

% section wiring algorithm (end)

\input{qm2pi.ack} 

% section acknowledgments (end)

\newpage


\bibliographystyle{plain}   
\bibliography{../../biblios/main.bib}

\input{qm2pi.rhodetails}

\end{document}

 

\documentclass[12pt]{llncs}
%\documentclass{jktr}

\usepackage[pdftex]{hyperref}                   
\usepackage {listings}
\usepackage {mathpartir}
\usepackage{bcprules}
%\usepackage{listings}
                       
\usepackage{graphicx} 
%\usepackage[margins=2.5cm,nohead,nofoot]{geometry}
%\usepackage{geometry}
\usepackage{amsfonts}
\usepackage{amstext}
\usepackage{latexsym}
\usepackage{amssymb}
\usepackage{color}


%\include{myPreamble}
\include{qm2pi.local} 

%\ifpdf
%\usepackage[pdftex]{graphicx}
%\else
%\usepackage{graphicx}
%\fi

 % \ifpdf
%  \usepackage{pdfsync}
%  \if


%\title{Brief Article}
%\author{David F. Snyder}
%\author{L.G. Meredith}

%\address{Dept. of Math., Texas State University--San Marcos, San Marcos, TX 78666}
       
\pagestyle{empty}


\begin{document}

\lstset{language=[Objective]Caml,frame=shadowbox}

\input{qm2pi.front}

% section front matter (end)

\input{qm2pi.intro} 
 
% section introduction (end)

% \input{qm2pi.knotations} 

% section notation (end)

\input{qm2pi.process.calculi} 

% section concurrent_process_calculi_and_spatial_logics_ (end)
    
%\input{qm2pi.knots2pi} 

%\input{qm2pi.trefoil} 

%\input{qm2pi.mainthm} 

% subsection basic_interpretation (end)

%\input{qm2pi.rho.presentation} 
\subsection{The syntax and semantics of the notation system}\label{sub:the_syntax_and_semantics_of_the_notation_system} % (fold)

We now summarize a technical presentation of the calculus that
embodies our theory of dynamics. The typical presentation of such a
calculus follows the style of giving generators and relations on
them. The grammar, below, describing term constructors, freely
generates the set of processes, $\Proc$. This set is then quotiented
by a relation known as structural congruence and it is over this set
that the notion of dynamics is expressed. This presentation is
essentially that of \cite{MeredithR05} with the addition of
polyadicity and summation. For readability we have relegated some of
the technical subtleties to an appendix.

\subsubsection{Process grammar}\label{subsub:process_grammar}

\begin{mathpar}
  \inferrule* [lab=synchronization] {} {{M} \bc \pzero \;|\; x?F \;|\; x!C }
  \and
  \inferrule* [lab=abstraction] {} {{F} \bc (x)P}
  \and
  \inferrule* [lab=concretion] {} {{C} \bc \langle Q \rangle}
  \and
  \inferrule* [lab=process] {} {{P,Q} \bc M \;| \;P|Q \;|\; @{x}}
  \and
  \inferrule* [lab=name] {} {{x} \bc \quotep{P}}
\end{mathpar} 

Note that $\vec{x}$ (resp. $\vec{P}$) denotes a vector of names
(resp. processes) of length $|\vec{x}|$ (resp. $|\vec{P}|$). We adopt
the following useful abbreviations.

\begin{mathpar}
   x?(\vec{y}).P := x.(\vec{y})P \and  x\clift{\vec{P}} := x.\clift{\vec{P}}
   \and x!(y) := \lift{x}{\dropn{y}}
   \and \Pi_{i=0}^{n-1}P_i := P_0 | \ldots | P_{n-1}
\end{mathpar}

\subsubsection{Structural congruence}

\paragraph{Free and bound names and alpha-equivalence.} At the
core of structural equivalence is alpha-equivalence which identifies
process that are the same up to a change of variable. Formally, we
recognize the distinction between free and bound names. The free names
of a process, $\freenames{P}$, may be calculated recursively as
follows:

\begin{mathpar}
\freenames{\pzero} := \emptyset
  \and \\
  \freenames{x?(y).P} := \{ x \} \cup (\freenames{P} \setminus \{ y \})
  \and 
  \freenames{x!\langle P \rangle} := \{ x \} \cup \{ P \} 
  \and \\
  \freenames{P|Q} := \freenames{P} \cup \freenames{Q}
  \and \\
  \freenames{@{x}} := \{ x \}
\end{mathpar}

$\pi$
$\quotep{\pi}$

$\freenames{-} : \pi \to \mathcal{P}(\quotep{\pi})$

\begin{eqnarray*}
  \freenames{\pzero} & := & \emptyset \\
  \freenames{x?(y).P} & := & \{ x \} \cup (\freenames{P} \setminus \{ y \}) \\
  \freenames{x!\langle P \rangle} & := & \{ x \} \cup \{ P \} \\
  \freenames{P|Q} & := & \freenames{P} \cup \freenames{Q} \\
  \freenames{\dropn{x}} & := & \{ x \}
\end{eqnarray*}

The bound names of a process, $\boundnames{P}$, are those names occurring in $P$
that are not free. For example, in $x?(y).0$, the name $x$ is free, while $y$ is bound.

\begin{mathpar}
  \inferrule* [lab=monoidal-laws] {} { P|Q \equiv Q|P \and P|0 \equiv P \and P|(Q|R) \equiv (P|Q)|R }
\end{mathpar}

\begin{mathpar}
  \inferrule* [lab=alpha-equivalence] {} { (x)P \equiv (y)P\{y/x\} \and y \not\in \freenames{P} }
\end{mathpar}

\begin{definition}
Then two processes, $P,Q$, are alpha-equivalent if $P = Q\{\vec{y}/\vec{x}\}$ for
some $\vec{x} \in \boundnames{Q},\vec{y} \in \boundnames{P}$, where $Q\{\vec{y}/\vec{x}\}$
denotes the capture-avoiding substitution of $\vec{y}$ for $\vec{x}$ in $Q$.
\end{definition}

\begin{definition}
  The {\em structural congruence} \cite{SangiorgiWalker} , $\equiv$,
  between processes is the least congruence containing
  alpha-equivalence, satisfying the abelian monoid laws
  (associativity, commutativity and $\pzero$ as identity) for parallel
  composition $|$ and for summation $+$.
\end{definition}

\subsection{Name equivalence}

We take name equivalence, written $\nameeq$, to be the smallest
equivalence relation generated by the following rules.

\begin{mathpar}
\inferrule*[lab=Quote-drop]
{ }
{ \quotep{@{x}} \nameeq x }

\inferrule*[lab=Struct-equiv]
{ P \scong Q }
{ \quotep{P} \nameeq \quotep{Q} }
\end{mathpar}

The astute reader will have noticed that the mutual recursion of names
and processes imposes a mutual recursion on alpha-equivalence and
structural equivalence via name-equivalence. Fortunately, all of this
works out pleasantly and we may calculate in the natural way, free of
concern. The reader interested in the details is referred to the
appendix \ref{appendix:rho_details}.

\subsection{Substitution}

We use $\Proc$ for the set of processes, $\QProc$ for the set of
names, and $\id{\{}\vec{y} / \vec{x} \id{\}}$ to denote partial maps,
$s : \QProc \rightarrow \QProc$. A map, $s$ lifts, uniquely, to a map
on process terms, $\widehat{s} : \Proc \rightarrow \Proc$ by the
following equations.

\begin{mathpar}
  (0) \psubstp{Q}{P} := 0 \\
  (R \juxtap S) \psubstp{Q}{P}
  :=    
  (R)\psubstp{Q}{P} \juxtap (S) \psubstp{Q}{P} \\
  (x?(y).R) \psubstp{Q}{P}    
  :=    
  (x)\substp{Q}{P} (z)\concat( (R \psubstn{z}{y}) \psubstp{Q}{P} ) \\
  (\lift{x}{R}) \psubstp{Q}{P}  
  :=
  \lift{(x)\substp{Q}{P}}{ R \psubstp{Q}{P} } \\
%   (\dropn{x})  \psubstp{Q}{P}       
%   := 
%   \left\{ 
%     \begin{array}{ccc} 
%       \dropn{\quotep{Q}} & & x \nameeq \quotep{P} \\
%       \dropn{x} & & otherwise \\
%     \end{array}
%   \right. 
  (\dropn{x})  \psubstp{Q}{P}       
  := 
  \left\{ 
    \begin{array}{ccc} 
      Q & & x \nameeq \quotep{P} \\
      \dropn{x} & & otherwise \\
    \end{array}
  \right.
\end{mathpar}
 

where

\begin{eqnarray}
  (x)\id{\{} \lpquote Q \rpquote / \lpquote P \rpquote \id{\}}            = 
  \left\{ 
    \begin{array}{ccc}
      \lpquote Q \rpquote & & x \nameeq \lpquote P \rpquote \\
      x & & otherwise \\
    \end{array}
  \right. \nonumber
\end{eqnarray}

and $z$ is chosen distinct from $\quotep{P}$, $\quotep{Q}$, the free
names in $Q$, and all the names in $R$. Our $\alpha$-equivalence will
be built in the standard way from this substitution.

\begin{remark}\label{rem:no_self_referential_names}
  One consequence of these definitions is that $\forall P. \quotep{P}
  \not\in \freenames{P}$.
\end{remark}

\subsection{ Dynamic quote: an example }

Anticipating something of what's to come, consider applying the
substitution, $\widehat{\id{\{}u / z \id{\}}}$, to the following pair
of processes, $\lift{w}{y!(z)}$ and $w[ \lpquote y!(z) \rpquote ]$.

\begin{eqnarray}
	\lift{w}{y!(z)}\widehat{\id{\{}u / z \id{\}}}
		& = &
		\lift{w}{y!(u)} \nonumber\\
	w[ \lpquote y!(z) \rpquote ] \widehat{ \id{\{}u / z \id{\}} }
		& = &
		w[ \lpquote y!(z) \rpquote ] \nonumber
\end{eqnarray}

Because the body of the process between quotes is impervious to
substitution, we get radically different answers. In fact, by
examining the first process in an input context,
e.g. $x?(z).\lift{w}{y!(z)}$, we see that the process under the lift
operator may be shaped by prefixed inputs binding a name inside it. In
this sense, the lift operator will be seen as a way to dynamically
construct processes before reifying them as names.

Finally equipped with these standard features we can present the
dynamics of the calculus.

\subsubsection{Operational semantics} 

Finally, we introduce the computational dynamics. What marks these
algebras as distinct from other more traditionally studied algebraic
structures, e.g. vector spaces or polynomial rings, is the manner in
which dynamics is captured. In traditional structures, dynamics is typically
expressed through morphisms between such structures, as in linear maps
between vector spaces or morphisms between rings. In algebras
associated with the semantics of computation, the dynamics is
expressed as part of the algebraic structure itself, through a
reduction reduction relation typically denoted by $\red$. Below, we
give a recursive presentation of this relation for the calculus used
in the encoding.

$\red \subseteq \pi \times \pi$
$\red : \pi \to \mathcal{P}(\pi)$

\begin{mathpar}
  \inferrule* [lab=Comm] { \textsf{match}( x_{src}, x_{trgt} ) } { x_{trgt}?(y)P \; | \; x_{src}!\langle {Q} \rangle \red P\{\quotep{Q}/y}\} }
  \and \\
  \inferrule* [lab=Par] {{P} \red {P}'} {{{P} | {Q}} \red {{P}' | {Q}}}
  \and
  \inferrule* [lab=Equiv]{{{P} \scong {P}'} \andalso {{P}' \red {Q}'} \andalso {{Q}' \scong {Q}}}{{P} \red {Q}}
\end{mathpar}

\begin{eqnarray*}
  match_{\equiv} (\quotep{P},\quotep{Q}) & := & P \equiv Q \\
  match_{\dagger}(\quotep{P},\quotep{Q}) & := & \forall R. P|Q \red^{*} R => R \red^{*} 0 \\
  match_{K}(\quotep{P},\quotep{Q}) & := & K \mbox{ for some context } K
\end{eqnarray*}

$u?(x)P | u!\langle Q \rangle \red P\{\quotep{Q}/x\}$

%We write $\wred$ for $\red^*$, and $P\red$ if $\exists Q $ such that $ P \red Q$.
We write $P\red$ if $\exists Q $ such that $ P \red Q$ and $P\not\red$, otherwise.

\section{Replication}

As mentioned before, it is known that replication (and hence
recursion) can be implemented in a higher-order process algebra
\cite{SangiorgiWalker}. As our first example of calculation with the
machinery thus far presented we give the construction explicitly in
the {\rhoc}.

\begin{eqnarray}
	D_{x} & := & \prefix{x}{y}{(\binpar{\outputp{x}{y}}{@{y}})} \nonumber\\
	\bangp_{x}{P} & := & \binpar{{x}!\langle{\binpar{D_{x}}{P}}\rangle}{D_{x}} \nonumber
\end{eqnarray}

\begin{eqnarray}
	\bangp_{x}{P} & & \nonumber\\
	=
	& {x}!\langle{(\prefix{x}{y}{(\outputp{x}{y} | @{y})) | P}}\rangle 
	      | \prefix{x}{y}{(\outputp{x}{y} | @{y})} & \nonumber\\
	\red
	& (\outputp{x}{y} | @{y})\substn{\quotep{(\prefix{x}{y}{(@{y} | \outputp{x}{y})) | P}}}{y} & \nonumber\\
	=
	& \outputp{x}{\quotep{(\prefix{x}{y}{(\outputp{x}{y} | @{y})) | P}}}
	  | {(\prefix{x}{y}{(\outputp{x}{y} | @{y})) | P}} & \nonumber\\
	\red
	& \ldots & \nonumber\\
	\red^*
	& P | P | \ldots & \nonumber
\end{eqnarray}

Of course, this encoding, as an implementation, runs away, unfolding
$\bangp{P}$ eagerly. A lazier and more implementable replication
operator, restricted to input-guarded processes, may be obtained as follows.

\begin{eqnarray}
\bangp{\prefix{u}{v}{P}} 
	:= 
	\binpar{\lift{x}{\prefix{u}{v}{(\binpar{D(x)}{P})}}}{D(x)} \nonumber
\end{eqnarray}

\begin{remark}
  Note that the lazier definition still does not deal with summation
  or mixed summation (i.e. sums over input and output). The reader is
  invited to construct definitions of replication that deal with these
  features. 

  Further, the definitions are parameterized in a name, $x$. Can you,
  gentle reader, make a definition that eliminates this parameter and
  guarantees no accidental interaction between the replication
  machinery and the process being replicated -- i.e. no accidental
  sharing of names used by the process to get its work done and the
  name(s) used by the replication to effect copying. This latter
  revision of the definition of replication is crucial to obtaining
  the expected identity $!!P \sim !P$.
\end{remark}

\begin{remark}\label{rem:paradoxical_combinator}
  The reader familiar with the lambda calculus will have noticed the
  similarity between $D$ and the paradoxical combinator.

  [Ed. note: the existence of this seems to suggest we have to be more
  restrictive on the set of processes and names we admit if we are to
  support no-cloning.]
\end{remark}

\subsubsection{Bisimulation}

The computational dynamics gives rise to another kind of equivalence,
the equivalence of computational behavior. As previously mentioned
this is typically captured \emph{via} some form of bisimulation.

% The notion we use in this paper is weak barbed bisimulation
% \cite{milner91polyadicpi}.

The notion we use in this paper is derived from weak barbed
bisimulation \cite{milner91polyadicpi}. 

\begin{definition}
An \emph{observation relation}, $\downarrow_{\mathcal N}$, over a set
of names, $\mathcal N$, is the smallest relation satisfying the rules
below.

\infrule[Out-barb]{y \in {\mathcal N}, \; x \nameeq y}
		  {\outputp{x}{v} \downarrow_{\mathcal N} x}
\infrule[Par-barb]{\mbox{$P\downarrow_{\mathcal N} x$ or $Q\downarrow_{\mathcal N} x$}}
		  {\binpar{P}{Q} \downarrow_{\mathcal N} x}

We write $P \Downarrow_{\mathcal N} x$ if there is $Q$ such that 
$P \wred Q$ and $Q \downarrow_{\mathcal N} x$.
\end{definition}

\begin{definition}
%\label{def.bbisim}
An  ${\mathcal N}$-\emph{barbed bisimulation} over a set of names, ${\mathcal N}$, is a symmetric binary relation 
${\mathcal S}_{\mathcal N}$ between agents such that $P\rel{S}_{\mathcal N}Q$ implies:
\begin{enumerate}
\item If $P \red P'$ then $Q \wred Q'$ and $P'\rel{S}_{\mathcal N} Q'$.
\item If $P\downarrow_{\mathcal N} x$, then $Q\Downarrow_{\mathcal N} x$.
\end{enumerate}
$P$ is ${\mathcal N}$-barbed bisimilar to $Q$, written
$P \wbbisim_{\mathcal N} Q$, if $P \rel{S}_{\mathcal N} Q$ for some ${\mathcal N}$-barbed bisimulation ${\mathcal S}_{\mathcal N}$.
\end{definition}

$\mathcal{R} \subseteq \pi \times \pi$

$P \mathcal{R} Q => \forall P'. P \red P' \Rightarrow \exists Q'. Q \red Q', P' \mathcal{R} Q'$

$P \vdash x \Rightarrow Q \vdash x$

\begin{mathpar}
  \inferrule*[lab=Out-barb]{x \nameeq y}{{y}!\langle{Q}\rangle \vdash x}
  \and
  \inferrule*[lab=Par-barb]{\mbox{$P\vdash x$ or $Q\vdash x$}}{\binpar{P}{Q} \vdash x}
\end{mathpar}

\subsubsection{Contexts}

One of the principle advantages of computational calculi like the
$\pi$-calculus is a well-defined notion of context,
contextual-equivalence and a correlation between
contextual-equivalence and notions of bisimulation. The notion of
context allows the decomposition of a process into (sub-)process and
its syntactic environment, its context. Thus, a context may be
thought of as a process with a ``hole'' (written $\Box$) in it. The
application of a context $M$ to a process $P$, written $M[P]$, is
tantamount to filling the hole in $M$ with $P$. In this paper we do
not need the full weight of this theory, but do make use of the notion
of context in the proof the main theorem. 

\begin{mathpar}
  \inferrule* [lab=summation] {} {{M_{M},M_{N}} \bc \Box \;|\; x.M_{A} \;|\; M_{M}+M_{N}}
  \and
  \inferrule* [lab=agent] {} {{M_{A}} \bc (\vec{x})M_{P} \;| \; \clift{P_0,\ldots,M_{P},\ldots,P_N}}
  \and \\
  \inferrule* [lab=process] {} {{M_{P}} \bc M_{N} \;| \;P|M_{P} }
\end{mathpar} 

\begin{mathpar}
  \inferrule* [lab=sychronization] {} {M_{N} \bc \Box \;|\; x?M_{F} \;|\; x!M_{C}}
  \and
  \inferrule* [lab=abstraction] {} {{M_{F}} \bc (x)M_{P} }
  \and
  \inferrule* [lab=concretion] {} {{M_{C}} \bc \langle M_{P} \rangle }
  \and \\
  \inferrule* [lab=process] {} {{M_{P}} \bc M_{N} \;| \;P|M_{P} }
\end{mathpar}

\begin{definition}[contextual application] Given a context $M$, and
  process $P$, we define the \emph{contextual application}, $M[P] :=
  M\{P/\Box\}$. That is, the contextual application of M to P is the
  substitution of $P$ for $\Box$ in $M$.
\end{definition}

$\meaningof{-} : L \to \mathcal{P}(\pi)$

\begin{mathpar}
  \inferrule* [lab=collection] {} {\meaningof{true} = \pi, \and \meaningof{~E} = \pi \setminus \meaningof{E}, \and \meaningof{E_{1} \& E_{2}} = \meaningof{E_{1}} \cap \meaningof{E_{2}}}
\end{mathpar}

\begin{mathpar}
  \inferrule* [lab=structure] {} {\meaningof{0} = \{ P \in \pi | P \equiv 0 \}, \and \\ \meaningof{E_1 | E_2} = \{ P \in \pi | P \equiv P_{1} | P_{2}, P_{1} \in \meaningof{E_{1}}, P_{2} \in \meaningof{E_2}\} }
\end{mathpar}

\begin{mathpar}
 \inferrule* [lab=behavior] {} {\meaningof{\langle a?b \rangle E} = \{ P \in \pi | P \equiv Q | u?(y)P', \\ \and \\\\ \and \\ \;\;\; u \in \meaningof{a}, \forall z.P'\{z/y\} \in \meaningof{E\{z/b\}}\}, \and \\ \meaningof{a!E} = \{ P \in \pi | P \equiv Q | x!\langle P' \rangle, x \in \meaningof{a} P' \in \meaningof{E}\} }
\end{mathpar}

\begin{mathpar}
 \inferrule* [lab=nominal] {} {\meaningof{\quotep{E}} = \{ \quotep{P} \in \quotep{\pi} | P \in \meaningof{E} \}, \and \meaningof{\quotep{P}} = \{ \quotep{Q} \in \quotep{\pi} | P \equiv Q \} \and \\ \meaningof{@\quotep{E}} = \{ P \in \pi | P \equiv @x, x \in \meaningof{E} \}}
\end{mathpar}

\begin{eqnarray*}
  \\
  \meaningof{-} : TS \to ST
\end{eqnarray*}

\begin{eqnarray*}
  \\
  L : TS \to ST
\end{eqnarray*}

\begin{eqnarray*}
  \\
  P \models E \iff P \in \meaningof{E}
\end{eqnarray*}

\begin{eqnarray*}
  P \approx_{L} Q \iff \forall E \in L. P \models E \iff Q \models E
\end{eqnarray*}

\begin{eqnarray*}
  P \approx_{K} Q
\end{eqnarray*}

\begin{eqnarray*}
  P \approx Q
\end{eqnarray*}

$\approx_{K} = \approx = \approx_{L}$

\subsubsection{Contextual duality}

Note that contexts extend the quotation operation to a family of
operations from processes to names. Given a context, $M$, we can
define a \emph{nominal context}, $\quotep{M}$ by $\quotep{M}[P] :=
\quotep{M[P]}$. To foreshadow what is to come we observe that these
operations enjoy a duality with processes very much like the duality
between vectors and maps from vectors to scalars.

Further, because the calculus is essentially higher-order, we have a
correspondence between contexts and processes. More specifically,
given a name $x$ and a context $M$ we can construct $M^{*}_{x}$ such
that 

\begin{mathpar}
  M^{*}_{x} | \lift{x}{P} \red M[P]
\end{mathpar}

namely,

\begin{mathpar}
  M^{*}_{x} := x?(u).M[\dropn{u}]
\end{mathpar}

The dependence of $M^{*}_{x}$ on a name makes it an abstraction, 

\begin{mathpar}
  M^{*} := (x)x?(u).M[\dropn{u}]
\end{mathpar}

\subsection{Additional notation}

It will sometimes be convenient to denote the process a name
quotes. We already have the notation $x = \quotep{P}$, but it will be
convenient to introduce an alternate notation, $\procn{x}$, when we
want to emphasize the connection to the use of the name. Note that, by
virtue of name equivalence, $\quotep{\procn{x}} \nameeq x$; so, the
notation is consistent with previous definitions.

Further, because names have structure it is possible to effect
substitutions on the basis of that structure. This means we need to
upgrade our notation for substitutions, which we accomplish by
adapting comprehension notation. Thus,

\begin{mathpar}
  P\{ y / x : x \in S \}
\end{mathpar}

is interpreted to mean the process derived from P by replacing (in a
capture-avoiding manner) each occurrence of $x$ in $S$ by $y$. For example,

\begin{mathpar}
  P\{ \quotep{\procn{x}|\procn{x}} / x : x \in \freenames{P} \}
\end{mathpar}

will replace each (occurrence) of a free name $x$ in $P$ by
$\quotep{\procn{x}|\procn{x}}$.

Also, we will avail ourselves of the notation $x^{L}$ and $x^{R}$ to
denote injections of a name into disjoint copies of the name
space. There are numerous ways to accomplish this. One example can be
found in \cite{MeredithR05}. This notation overloads to vectors of
names: $\vec{x}^{\pi} := (x_{i}^{\pi} \; : \; 0 \leq i < |\vec{x}| )$ where $\pi \in \{L,R\}$.

We also use $P^{\Box} := P|\Box$.

In \cite{MeredithR05} an interpretation of the new operator is
given. It turns out that there are several possible interpretations
all enjoying the requisite algebraic properties of the operator (see
\cite{milner91polyadicpi}). We will therefore make liberal use of
$(\nu\; \vec{x})P$.

% subsection the_syntax_and_semantics_of_the_notation_system (end)   

\input{qm2pi.qmops} 

\input{qm2pi.sterngerlach} 

\input{qm2pi.metric} 

% section concurrent_process_calculi (end)

%\input{qm2pi.proofsketch}

% section proof sketch (end)

%\input{qm2pi.slviaknots} 

% section spatial logic via knots (end)

\input{qm2pi.conclusion}

% section conclusion (end)

%\input{qm2pi.dtcodes} 

% section wiring algorithm (end)

\input{qm2pi.ack} 

% section acknowledgments (end)

\newpage


\bibliographystyle{plain}   
\bibliography{../../biblios/main.bib}

\input{qm2pi.rhodetails}

\end{document}

 

% section concurrent_process_calculi (end)

%\documentclass[12pt]{llncs}
%\documentclass{jktr}

\usepackage[pdftex]{hyperref}                   
\usepackage {listings}
\usepackage {mathpartir}
\usepackage{bcprules}
%\usepackage{listings}
                       
\usepackage{graphicx} 
%\usepackage[margins=2.5cm,nohead,nofoot]{geometry}
%\usepackage{geometry}
\usepackage{amsfonts}
\usepackage{amstext}
\usepackage{latexsym}
\usepackage{amssymb}
\usepackage{color}


%\include{myPreamble}
\include{qm2pi.local} 

%\ifpdf
%\usepackage[pdftex]{graphicx}
%\else
%\usepackage{graphicx}
%\fi

 % \ifpdf
%  \usepackage{pdfsync}
%  \if


%\title{Brief Article}
%\author{David F. Snyder}
%\author{L.G. Meredith}

%\address{Dept. of Math., Texas State University--San Marcos, San Marcos, TX 78666}
       
\pagestyle{empty}


\begin{document}

\lstset{language=[Objective]Caml,frame=shadowbox}

\input{qm2pi.front}

% section front matter (end)

\input{qm2pi.intro} 
 
% section introduction (end)

% \input{qm2pi.knotations} 

% section notation (end)

\input{qm2pi.process.calculi} 

% section concurrent_process_calculi_and_spatial_logics_ (end)
    
%\input{qm2pi.knots2pi} 

%\input{qm2pi.trefoil} 

%\input{qm2pi.mainthm} 

% subsection basic_interpretation (end)

%\input{qm2pi.rho.presentation} 
\subsection{The syntax and semantics of the notation system}\label{sub:the_syntax_and_semantics_of_the_notation_system} % (fold)

We now summarize a technical presentation of the calculus that
embodies our theory of dynamics. The typical presentation of such a
calculus follows the style of giving generators and relations on
them. The grammar, below, describing term constructors, freely
generates the set of processes, $\Proc$. This set is then quotiented
by a relation known as structural congruence and it is over this set
that the notion of dynamics is expressed. This presentation is
essentially that of \cite{MeredithR05} with the addition of
polyadicity and summation. For readability we have relegated some of
the technical subtleties to an appendix.

\subsubsection{Process grammar}\label{subsub:process_grammar}

\begin{mathpar}
  \inferrule* [lab=synchronization] {} {{M} \bc \pzero \;|\; x?F \;|\; x!C }
  \and
  \inferrule* [lab=abstraction] {} {{F} \bc (x)P}
  \and
  \inferrule* [lab=concretion] {} {{C} \bc \langle Q \rangle}
  \and
  \inferrule* [lab=process] {} {{P,Q} \bc M \;| \;P|Q \;|\; @{x}}
  \and
  \inferrule* [lab=name] {} {{x} \bc \quotep{P}}
\end{mathpar} 

Note that $\vec{x}$ (resp. $\vec{P}$) denotes a vector of names
(resp. processes) of length $|\vec{x}|$ (resp. $|\vec{P}|$). We adopt
the following useful abbreviations.

\begin{mathpar}
   x?(\vec{y}).P := x.(\vec{y})P \and  x\clift{\vec{P}} := x.\clift{\vec{P}}
   \and x!(y) := \lift{x}{\dropn{y}}
   \and \Pi_{i=0}^{n-1}P_i := P_0 | \ldots | P_{n-1}
\end{mathpar}

\subsubsection{Structural congruence}

\paragraph{Free and bound names and alpha-equivalence.} At the
core of structural equivalence is alpha-equivalence which identifies
process that are the same up to a change of variable. Formally, we
recognize the distinction between free and bound names. The free names
of a process, $\freenames{P}$, may be calculated recursively as
follows:

\begin{mathpar}
\freenames{\pzero} := \emptyset
  \and \\
  \freenames{x?(y).P} := \{ x \} \cup (\freenames{P} \setminus \{ y \})
  \and 
  \freenames{x!\langle P \rangle} := \{ x \} \cup \{ P \} 
  \and \\
  \freenames{P|Q} := \freenames{P} \cup \freenames{Q}
  \and \\
  \freenames{@{x}} := \{ x \}
\end{mathpar}

$\pi$
$\quotep{\pi}$

$\freenames{-} : \pi \to \mathcal{P}(\quotep{\pi})$

\begin{eqnarray*}
  \freenames{\pzero} & := & \emptyset \\
  \freenames{x?(y).P} & := & \{ x \} \cup (\freenames{P} \setminus \{ y \}) \\
  \freenames{x!\langle P \rangle} & := & \{ x \} \cup \{ P \} \\
  \freenames{P|Q} & := & \freenames{P} \cup \freenames{Q} \\
  \freenames{\dropn{x}} & := & \{ x \}
\end{eqnarray*}

The bound names of a process, $\boundnames{P}$, are those names occurring in $P$
that are not free. For example, in $x?(y).0$, the name $x$ is free, while $y$ is bound.

\begin{mathpar}
  \inferrule* [lab=monoidal-laws] {} { P|Q \equiv Q|P \and P|0 \equiv P \and P|(Q|R) \equiv (P|Q)|R }
\end{mathpar}

\begin{mathpar}
  \inferrule* [lab=alpha-equivalence] {} { (x)P \equiv (y)P\{y/x\} \and y \not\in \freenames{P} }
\end{mathpar}

\begin{definition}
Then two processes, $P,Q$, are alpha-equivalent if $P = Q\{\vec{y}/\vec{x}\}$ for
some $\vec{x} \in \boundnames{Q},\vec{y} \in \boundnames{P}$, where $Q\{\vec{y}/\vec{x}\}$
denotes the capture-avoiding substitution of $\vec{y}$ for $\vec{x}$ in $Q$.
\end{definition}

\begin{definition}
  The {\em structural congruence} \cite{SangiorgiWalker} , $\equiv$,
  between processes is the least congruence containing
  alpha-equivalence, satisfying the abelian monoid laws
  (associativity, commutativity and $\pzero$ as identity) for parallel
  composition $|$ and for summation $+$.
\end{definition}

\subsection{Name equivalence}

We take name equivalence, written $\nameeq$, to be the smallest
equivalence relation generated by the following rules.

\begin{mathpar}
\inferrule*[lab=Quote-drop]
{ }
{ \quotep{@{x}} \nameeq x }

\inferrule*[lab=Struct-equiv]
{ P \scong Q }
{ \quotep{P} \nameeq \quotep{Q} }
\end{mathpar}

The astute reader will have noticed that the mutual recursion of names
and processes imposes a mutual recursion on alpha-equivalence and
structural equivalence via name-equivalence. Fortunately, all of this
works out pleasantly and we may calculate in the natural way, free of
concern. The reader interested in the details is referred to the
appendix \ref{appendix:rho_details}.

\subsection{Substitution}

We use $\Proc$ for the set of processes, $\QProc$ for the set of
names, and $\id{\{}\vec{y} / \vec{x} \id{\}}$ to denote partial maps,
$s : \QProc \rightarrow \QProc$. A map, $s$ lifts, uniquely, to a map
on process terms, $\widehat{s} : \Proc \rightarrow \Proc$ by the
following equations.

\begin{mathpar}
  (0) \psubstp{Q}{P} := 0 \\
  (R \juxtap S) \psubstp{Q}{P}
  :=    
  (R)\psubstp{Q}{P} \juxtap (S) \psubstp{Q}{P} \\
  (x?(y).R) \psubstp{Q}{P}    
  :=    
  (x)\substp{Q}{P} (z)\concat( (R \psubstn{z}{y}) \psubstp{Q}{P} ) \\
  (\lift{x}{R}) \psubstp{Q}{P}  
  :=
  \lift{(x)\substp{Q}{P}}{ R \psubstp{Q}{P} } \\
%   (\dropn{x})  \psubstp{Q}{P}       
%   := 
%   \left\{ 
%     \begin{array}{ccc} 
%       \dropn{\quotep{Q}} & & x \nameeq \quotep{P} \\
%       \dropn{x} & & otherwise \\
%     \end{array}
%   \right. 
  (\dropn{x})  \psubstp{Q}{P}       
  := 
  \left\{ 
    \begin{array}{ccc} 
      Q & & x \nameeq \quotep{P} \\
      \dropn{x} & & otherwise \\
    \end{array}
  \right.
\end{mathpar}
 

where

\begin{eqnarray}
  (x)\id{\{} \lpquote Q \rpquote / \lpquote P \rpquote \id{\}}            = 
  \left\{ 
    \begin{array}{ccc}
      \lpquote Q \rpquote & & x \nameeq \lpquote P \rpquote \\
      x & & otherwise \\
    \end{array}
  \right. \nonumber
\end{eqnarray}

and $z$ is chosen distinct from $\quotep{P}$, $\quotep{Q}$, the free
names in $Q$, and all the names in $R$. Our $\alpha$-equivalence will
be built in the standard way from this substitution.

\begin{remark}\label{rem:no_self_referential_names}
  One consequence of these definitions is that $\forall P. \quotep{P}
  \not\in \freenames{P}$.
\end{remark}

\subsection{ Dynamic quote: an example }

Anticipating something of what's to come, consider applying the
substitution, $\widehat{\id{\{}u / z \id{\}}}$, to the following pair
of processes, $\lift{w}{y!(z)}$ and $w[ \lpquote y!(z) \rpquote ]$.

\begin{eqnarray}
	\lift{w}{y!(z)}\widehat{\id{\{}u / z \id{\}}}
		& = &
		\lift{w}{y!(u)} \nonumber\\
	w[ \lpquote y!(z) \rpquote ] \widehat{ \id{\{}u / z \id{\}} }
		& = &
		w[ \lpquote y!(z) \rpquote ] \nonumber
\end{eqnarray}

Because the body of the process between quotes is impervious to
substitution, we get radically different answers. In fact, by
examining the first process in an input context,
e.g. $x?(z).\lift{w}{y!(z)}$, we see that the process under the lift
operator may be shaped by prefixed inputs binding a name inside it. In
this sense, the lift operator will be seen as a way to dynamically
construct processes before reifying them as names.

Finally equipped with these standard features we can present the
dynamics of the calculus.

\subsubsection{Operational semantics} 

Finally, we introduce the computational dynamics. What marks these
algebras as distinct from other more traditionally studied algebraic
structures, e.g. vector spaces or polynomial rings, is the manner in
which dynamics is captured. In traditional structures, dynamics is typically
expressed through morphisms between such structures, as in linear maps
between vector spaces or morphisms between rings. In algebras
associated with the semantics of computation, the dynamics is
expressed as part of the algebraic structure itself, through a
reduction reduction relation typically denoted by $\red$. Below, we
give a recursive presentation of this relation for the calculus used
in the encoding.

$\red \subseteq \pi \times \pi$
$\red : \pi \to \mathcal{P}(\pi)$

\begin{mathpar}
  \inferrule* [lab=Comm] { \textsf{match}( x_{src}, x_{trgt} ) } { x_{trgt}?(y)P \; | \; x_{src}!\langle {Q} \rangle \red P\{\quotep{Q}/y}\} }
  \and \\
  \inferrule* [lab=Par] {{P} \red {P}'} {{{P} | {Q}} \red {{P}' | {Q}}}
  \and
  \inferrule* [lab=Equiv]{{{P} \scong {P}'} \andalso {{P}' \red {Q}'} \andalso {{Q}' \scong {Q}}}{{P} \red {Q}}
\end{mathpar}

\begin{eqnarray*}
  match_{\equiv} (\quotep{P},\quotep{Q}) & := & P \equiv Q \\
  match_{\dagger}(\quotep{P},\quotep{Q}) & := & \forall R. P|Q \red^{*} R => R \red^{*} 0 \\
  match_{K}(\quotep{P},\quotep{Q}) & := & K \mbox{ for some context } K
\end{eqnarray*}

$u?(x)P | u!\langle Q \rangle \red P\{\quotep{Q}/x\}$

%We write $\wred$ for $\red^*$, and $P\red$ if $\exists Q $ such that $ P \red Q$.
We write $P\red$ if $\exists Q $ such that $ P \red Q$ and $P\not\red$, otherwise.

\section{Replication}

As mentioned before, it is known that replication (and hence
recursion) can be implemented in a higher-order process algebra
\cite{SangiorgiWalker}. As our first example of calculation with the
machinery thus far presented we give the construction explicitly in
the {\rhoc}.

\begin{eqnarray}
	D_{x} & := & \prefix{x}{y}{(\binpar{\outputp{x}{y}}{@{y}})} \nonumber\\
	\bangp_{x}{P} & := & \binpar{{x}!\langle{\binpar{D_{x}}{P}}\rangle}{D_{x}} \nonumber
\end{eqnarray}

\begin{eqnarray}
	\bangp_{x}{P} & & \nonumber\\
	=
	& {x}!\langle{(\prefix{x}{y}{(\outputp{x}{y} | @{y})) | P}}\rangle 
	      | \prefix{x}{y}{(\outputp{x}{y} | @{y})} & \nonumber\\
	\red
	& (\outputp{x}{y} | @{y})\substn{\quotep{(\prefix{x}{y}{(@{y} | \outputp{x}{y})) | P}}}{y} & \nonumber\\
	=
	& \outputp{x}{\quotep{(\prefix{x}{y}{(\outputp{x}{y} | @{y})) | P}}}
	  | {(\prefix{x}{y}{(\outputp{x}{y} | @{y})) | P}} & \nonumber\\
	\red
	& \ldots & \nonumber\\
	\red^*
	& P | P | \ldots & \nonumber
\end{eqnarray}

Of course, this encoding, as an implementation, runs away, unfolding
$\bangp{P}$ eagerly. A lazier and more implementable replication
operator, restricted to input-guarded processes, may be obtained as follows.

\begin{eqnarray}
\bangp{\prefix{u}{v}{P}} 
	:= 
	\binpar{\lift{x}{\prefix{u}{v}{(\binpar{D(x)}{P})}}}{D(x)} \nonumber
\end{eqnarray}

\begin{remark}
  Note that the lazier definition still does not deal with summation
  or mixed summation (i.e. sums over input and output). The reader is
  invited to construct definitions of replication that deal with these
  features. 

  Further, the definitions are parameterized in a name, $x$. Can you,
  gentle reader, make a definition that eliminates this parameter and
  guarantees no accidental interaction between the replication
  machinery and the process being replicated -- i.e. no accidental
  sharing of names used by the process to get its work done and the
  name(s) used by the replication to effect copying. This latter
  revision of the definition of replication is crucial to obtaining
  the expected identity $!!P \sim !P$.
\end{remark}

\begin{remark}\label{rem:paradoxical_combinator}
  The reader familiar with the lambda calculus will have noticed the
  similarity between $D$ and the paradoxical combinator.

  [Ed. note: the existence of this seems to suggest we have to be more
  restrictive on the set of processes and names we admit if we are to
  support no-cloning.]
\end{remark}

\subsubsection{Bisimulation}

The computational dynamics gives rise to another kind of equivalence,
the equivalence of computational behavior. As previously mentioned
this is typically captured \emph{via} some form of bisimulation.

% The notion we use in this paper is weak barbed bisimulation
% \cite{milner91polyadicpi}.

The notion we use in this paper is derived from weak barbed
bisimulation \cite{milner91polyadicpi}. 

\begin{definition}
An \emph{observation relation}, $\downarrow_{\mathcal N}$, over a set
of names, $\mathcal N$, is the smallest relation satisfying the rules
below.

\infrule[Out-barb]{y \in {\mathcal N}, \; x \nameeq y}
		  {\outputp{x}{v} \downarrow_{\mathcal N} x}
\infrule[Par-barb]{\mbox{$P\downarrow_{\mathcal N} x$ or $Q\downarrow_{\mathcal N} x$}}
		  {\binpar{P}{Q} \downarrow_{\mathcal N} x}

We write $P \Downarrow_{\mathcal N} x$ if there is $Q$ such that 
$P \wred Q$ and $Q \downarrow_{\mathcal N} x$.
\end{definition}

\begin{definition}
%\label{def.bbisim}
An  ${\mathcal N}$-\emph{barbed bisimulation} over a set of names, ${\mathcal N}$, is a symmetric binary relation 
${\mathcal S}_{\mathcal N}$ between agents such that $P\rel{S}_{\mathcal N}Q$ implies:
\begin{enumerate}
\item If $P \red P'$ then $Q \wred Q'$ and $P'\rel{S}_{\mathcal N} Q'$.
\item If $P\downarrow_{\mathcal N} x$, then $Q\Downarrow_{\mathcal N} x$.
\end{enumerate}
$P$ is ${\mathcal N}$-barbed bisimilar to $Q$, written
$P \wbbisim_{\mathcal N} Q$, if $P \rel{S}_{\mathcal N} Q$ for some ${\mathcal N}$-barbed bisimulation ${\mathcal S}_{\mathcal N}$.
\end{definition}

$\mathcal{R} \subseteq \pi \times \pi$

$P \mathcal{R} Q => \forall P'. P \red P' \Rightarrow \exists Q'. Q \red Q', P' \mathcal{R} Q'$

$P \vdash x \Rightarrow Q \vdash x$

\begin{mathpar}
  \inferrule*[lab=Out-barb]{x \nameeq y}{{y}!\langle{Q}\rangle \vdash x}
  \and
  \inferrule*[lab=Par-barb]{\mbox{$P\vdash x$ or $Q\vdash x$}}{\binpar{P}{Q} \vdash x}
\end{mathpar}

\subsubsection{Contexts}

One of the principle advantages of computational calculi like the
$\pi$-calculus is a well-defined notion of context,
contextual-equivalence and a correlation between
contextual-equivalence and notions of bisimulation. The notion of
context allows the decomposition of a process into (sub-)process and
its syntactic environment, its context. Thus, a context may be
thought of as a process with a ``hole'' (written $\Box$) in it. The
application of a context $M$ to a process $P$, written $M[P]$, is
tantamount to filling the hole in $M$ with $P$. In this paper we do
not need the full weight of this theory, but do make use of the notion
of context in the proof the main theorem. 

\begin{mathpar}
  \inferrule* [lab=summation] {} {{M_{M},M_{N}} \bc \Box \;|\; x.M_{A} \;|\; M_{M}+M_{N}}
  \and
  \inferrule* [lab=agent] {} {{M_{A}} \bc (\vec{x})M_{P} \;| \; \clift{P_0,\ldots,M_{P},\ldots,P_N}}
  \and \\
  \inferrule* [lab=process] {} {{M_{P}} \bc M_{N} \;| \;P|M_{P} }
\end{mathpar} 

\begin{mathpar}
  \inferrule* [lab=sychronization] {} {M_{N} \bc \Box \;|\; x?M_{F} \;|\; x!M_{C}}
  \and
  \inferrule* [lab=abstraction] {} {{M_{F}} \bc (x)M_{P} }
  \and
  \inferrule* [lab=concretion] {} {{M_{C}} \bc \langle M_{P} \rangle }
  \and \\
  \inferrule* [lab=process] {} {{M_{P}} \bc M_{N} \;| \;P|M_{P} }
\end{mathpar}

\begin{definition}[contextual application] Given a context $M$, and
  process $P$, we define the \emph{contextual application}, $M[P] :=
  M\{P/\Box\}$. That is, the contextual application of M to P is the
  substitution of $P$ for $\Box$ in $M$.
\end{definition}

$\meaningof{-} : L \to \mathcal{P}(\pi)$

\begin{mathpar}
  \inferrule* [lab=collection] {} {\meaningof{true} = \pi, \and \meaningof{~E} = \pi \setminus \meaningof{E}, \and \meaningof{E_{1} \& E_{2}} = \meaningof{E_{1}} \cap \meaningof{E_{2}}}
\end{mathpar}

\begin{mathpar}
  \inferrule* [lab=structure] {} {\meaningof{0} = \{ P \in \pi | P \equiv 0 \}, \and \\ \meaningof{E_1 | E_2} = \{ P \in \pi | P \equiv P_{1} | P_{2}, P_{1} \in \meaningof{E_{1}}, P_{2} \in \meaningof{E_2}\} }
\end{mathpar}

\begin{mathpar}
 \inferrule* [lab=behavior] {} {\meaningof{\langle a?b \rangle E} = \{ P \in \pi | P \equiv Q | u?(y)P', \\ \and \\\\ \and \\ \;\;\; u \in \meaningof{a}, \forall z.P'\{z/y\} \in \meaningof{E\{z/b\}}\}, \and \\ \meaningof{a!E} = \{ P \in \pi | P \equiv Q | x!\langle P' \rangle, x \in \meaningof{a} P' \in \meaningof{E}\} }
\end{mathpar}

\begin{mathpar}
 \inferrule* [lab=nominal] {} {\meaningof{\quotep{E}} = \{ \quotep{P} \in \quotep{\pi} | P \in \meaningof{E} \}, \and \meaningof{\quotep{P}} = \{ \quotep{Q} \in \quotep{\pi} | P \equiv Q \} \and \\ \meaningof{@\quotep{E}} = \{ P \in \pi | P \equiv @x, x \in \meaningof{E} \}}
\end{mathpar}

\begin{eqnarray*}
  \\
  \meaningof{-} : TS \to ST
\end{eqnarray*}

\begin{eqnarray*}
  \\
  L : TS \to ST
\end{eqnarray*}

\begin{eqnarray*}
  \\
  P \models E \iff P \in \meaningof{E}
\end{eqnarray*}

\begin{eqnarray*}
  P \approx_{L} Q \iff \forall E \in L. P \models E \iff Q \models E
\end{eqnarray*}

\begin{eqnarray*}
  P \approx_{K} Q
\end{eqnarray*}

\begin{eqnarray*}
  P \approx Q
\end{eqnarray*}

$\approx_{K} = \approx = \approx_{L}$

\subsubsection{Contextual duality}

Note that contexts extend the quotation operation to a family of
operations from processes to names. Given a context, $M$, we can
define a \emph{nominal context}, $\quotep{M}$ by $\quotep{M}[P] :=
\quotep{M[P]}$. To foreshadow what is to come we observe that these
operations enjoy a duality with processes very much like the duality
between vectors and maps from vectors to scalars.

Further, because the calculus is essentially higher-order, we have a
correspondence between contexts and processes. More specifically,
given a name $x$ and a context $M$ we can construct $M^{*}_{x}$ such
that 

\begin{mathpar}
  M^{*}_{x} | \lift{x}{P} \red M[P]
\end{mathpar}

namely,

\begin{mathpar}
  M^{*}_{x} := x?(u).M[\dropn{u}]
\end{mathpar}

The dependence of $M^{*}_{x}$ on a name makes it an abstraction, 

\begin{mathpar}
  M^{*} := (x)x?(u).M[\dropn{u}]
\end{mathpar}

\subsection{Additional notation}

It will sometimes be convenient to denote the process a name
quotes. We already have the notation $x = \quotep{P}$, but it will be
convenient to introduce an alternate notation, $\procn{x}$, when we
want to emphasize the connection to the use of the name. Note that, by
virtue of name equivalence, $\quotep{\procn{x}} \nameeq x$; so, the
notation is consistent with previous definitions.

Further, because names have structure it is possible to effect
substitutions on the basis of that structure. This means we need to
upgrade our notation for substitutions, which we accomplish by
adapting comprehension notation. Thus,

\begin{mathpar}
  P\{ y / x : x \in S \}
\end{mathpar}

is interpreted to mean the process derived from P by replacing (in a
capture-avoiding manner) each occurrence of $x$ in $S$ by $y$. For example,

\begin{mathpar}
  P\{ \quotep{\procn{x}|\procn{x}} / x : x \in \freenames{P} \}
\end{mathpar}

will replace each (occurrence) of a free name $x$ in $P$ by
$\quotep{\procn{x}|\procn{x}}$.

Also, we will avail ourselves of the notation $x^{L}$ and $x^{R}$ to
denote injections of a name into disjoint copies of the name
space. There are numerous ways to accomplish this. One example can be
found in \cite{MeredithR05}. This notation overloads to vectors of
names: $\vec{x}^{\pi} := (x_{i}^{\pi} \; : \; 0 \leq i < |\vec{x}| )$ where $\pi \in \{L,R\}$.

We also use $P^{\Box} := P|\Box$.

In \cite{MeredithR05} an interpretation of the new operator is
given. It turns out that there are several possible interpretations
all enjoying the requisite algebraic properties of the operator (see
\cite{milner91polyadicpi}). We will therefore make liberal use of
$(\nu\; \vec{x})P$.

% subsection the_syntax_and_semantics_of_the_notation_system (end)   

\input{qm2pi.qmops} 

\input{qm2pi.sterngerlach} 

\input{qm2pi.metric} 

% section concurrent_process_calculi (end)

%\input{qm2pi.proofsketch}

% section proof sketch (end)

%\input{qm2pi.slviaknots} 

% section spatial logic via knots (end)

\input{qm2pi.conclusion}

% section conclusion (end)

%\input{qm2pi.dtcodes} 

% section wiring algorithm (end)

\input{qm2pi.ack} 

% section acknowledgments (end)

\newpage


\bibliographystyle{plain}   
\bibliography{../../biblios/main.bib}

\input{qm2pi.rhodetails}

\end{document}



% section proof sketch (end)

%\section{Unlikely characters: spatial logic for
  knots}\label{sub:characteristic_formulae} % (fold)

Associated to the mobile process calculi are a family of logics known
as the Hennessy-Milner logics. These logics typically enjoy a
semantics interpreting formulae as sets of processes that when
factored through the encoding outlined above allows an identification
of classes of knots with logical formulae. In the context of this
encoding the sub-family known as the spatial logics \cite{CairesC03}
\cite{CairesC04} \cite{Caires04} are of particular interest providing
several important features for expressing and reasoning about
properties (i.e. classes) of knots. We hint here at how this may be done.

%\begin{description}
%\item [structural connectives] 
\subsubsection{Structural connectives} The spatial logics enjoy
structural connectives corresponding, at the logical level, to the
parallel composition ($P | Q$) and new name ($(\nu \; x)P$)
connectives for processes. As illustrated in the examples below, these
connectives are extremely expressive given the shape of our encoding.
%\item [decideable satisfaction]

\subsubsection{Decideable satisfaction}
In \cite{Caires04} the satisfaction relation is shown to be decideable
for a rich class of processes. It further turns out that the image of
the our encoding is a proper subset of that class. This result
provides the basis for an algorithm by which to search for knots
enjoying a given property.
%\item [characteristic formulae]

\subsubsection{Characteristic formulae}
In the same paper \cite{Caires04} , Caires presents a means of calculating
characteristic formulae, selecting equivalence classes of processes
up to a pre--specified depth limit on the support set of names. Composed with our
encoding, this characteristic formula can be used to select
characteristic formulae for knots.
%\end{description}

\subsubsection{Spatial logic formulae}

The grammar below (segmented for comprehension) summarizes the syntax
of spatial logic formulae. We employ illustrative examples in the
sequel to provide an intuitive understanding of their meaning
referring the reader to \cite{Caires04} for a more detailed explication
of the semantics.

\begin{mathpar}
  \inferrule* [lab=boolean] {} {{A,B} \bc T \;|\; \neg A \;|\; A \wedge B \;|\; \eta = \eta'}
  \and
  \inferrule* [lab=spatial] {} {|\; \pzero \;|\; A | B \;|\; x \text{\textregistered} A \;|\; \forall x . A \;|\;  H x . A}
  \and
  \inferrule* [lab=behavioral] {} {|\; \alpha . A}
  \and 
  \inferrule* [lab=recursion] {} {|\; X(\vec{u}) \;|\; \mu X(\vec{u}) . A}
  \and
  \inferrule* [lab=action] {} {\alpha \bc \langle x?(\vec{y}) \rangle \;|\; \langle x!(\vec{y}) \rangle \;|\; \langle \tau \rangle}
  \and 
  \inferrule* [lab=name] {} {\eta \bc x \;|\; \tau}
\end{mathpar} 

% subsection characteristic_formulae (end)   	 

\subsection{Example formulae}\label{sub:example_formulae_} % (fold)

\subsubsection{Crossing as formula.}
% 
% \begin{align*}
%   \frac{d}{dx} \sin x &= \cos x 
%   & \frac{d}{dx} e^x &= e^x \\
%   \frac{d}{dx} \cos x &= - \sin x 
%   & \frac{d}{dx} \log x &= \frac{1}{x} \\
% \end{align*} 

\begin{align*}
 \mu C(x_{0},x_{1},y_{0},y_{1},u).&(\langle x_{0}?(z) \rangle(\langle u! \rangle\langle y_{1}!z \rangle C(x_{0},x_{1},y_{0},y_{1},u)) & \\
  & \wedge \langle y_{1}?(z) \rangle (\langle u! \rangle \langle x_{0}!z \rangle C(x_{0},x_{1},y_{0},y_{1},u)) & \\
  & \wedge \langle x_{1}?(z) \rangle (\langle u? \rangle \langle y_{0}!z \rangle C(x_{0},x_{1},y_{0},y_{1},u)) & \\
  & \wedge \langle y_{0}?(z) \rangle (\langle u? \rangle \langle x_{1}!z \rangle C(x_{0},x_{1},y_{0},y_{1},u))) &
\end{align*}

The lexicographical similarity between the shape of this formulae and
the shape of definition of the process representing a crossing reveals
the intuitive meaning of this formulae. It describes the capabilities
of a process that has the right to represent a crossing. For example
it picks out processes that may perform an input on the port $x_0$ in
its initial menu of capabilities. What differentiates the formula
from the process, however, is that the crossing process is the
smallest candidate to satisfy the formula. Infinitely many other
processes -- with internal behavior hidden behind this interface, so
to speak -- also satisfy this formula. Even this simple formula,
then, can be seen to open a new view onto knots, providing a
computational interpretation of \emph{virtual} knots.

Note that this formula is derived by hand. A similar formula can be
derived by employing Caires' calculation of characteristic formula
\cite{Caires04} to the process representing a crossing. In light of
this discussion, we let
$\meaningof{C}_{\phi}(x0,x1,y0,y1,u)$ denote a formula specifying the
dynamics we wish to capture of a crossing. To guarantee we preserve
the shape of the interface and minimal semantics we demand that
$\meaningof{C}_{\phi}(x0,x1,y0,y1,u) \Rightarrow
\textbf{C}(x0,x1,y0,y1,u)$ where $\textbf{C}(x0,x1,y0,y1,u)$ denotes
the formula above.
                            
\subsubsection{Crossing number constraints.}
The moral content of the context lemma (Lemma \ref{context}) is that the notion of
``locality'' in the Reidemeister moves is effectively captured by the
parallel composition operator of the process calculus. This intuition
extends through the logic. Given a formula,
$\meaningof{C}_{\phi}(x0,x1,y0,y1,u)$, we can use the structural
connectives to specify constraints on crossing numbers, such as at
least $n$ crossings, or exactly $n$ crossings.
\begin{mathpar}
  \inferrule* [lab=at-least-n] {} { K^{\geq n}_{\phi}(\vec{xs},\vec{ys}) := \Pi_{i=0}^{n-1} Hu . \meaningof{C}_{\phi}(xs_i,ys_i,u) | T }
  \and 
  \inferrule* [lab=exactly-n] {} { K^{= n}_{\phi}(\vec{xs},\vec{ys}) := \Pi_{i=0}^{n-1} Hu . \meaningof{C}_{\phi}(xs_i,ys_i,u) | \neg (\forall x_0,y_0,x_1,y_1,u . \meaningof{C}_{\phi}(x_0,y_0,x_1,y_1,u) | T) }
\end{mathpar}

To round out this section, recall that the encoding of an $n$-crossing
knot decomposes into a parallel composition of $n$ \emph{copies} of a
crossing process together with a wiring harness. To specify different
knot classes with the same crossing number amounts to specifying
logical constraints on the wiring harness. In the interest of space,
we defer examples to a forthcoming paper. Suffice it to say that both
the conditions ``alternating knot'' and ``contains the tangle
corresponding to 5/3'' are expressible. For example, it is possible to
calculate the characteristic formula of a process corresponding to the
tangle 5/3 and conjoin it into the classifying formula via the
composition connective of the logic.

Finally, we wish to observe that it is entirely within reason to
contemplate a more domain-specific version of spatial logic tailored
to the shape of processes in the image of the encoding. Such a
domain-specific logic would have a better claim to the title formal
language of knot properties.

% subsection example_formulae_ (end)

% section knots_as_processes (end) 

% section spatial logic via knots (end)

\section{Conclusions and future work}

\paragraph{Testing physical space}
You, gentle reader, may wonder why of all the theorems to be proved
given this set up we pick the one above. In some sense it's hardly
central to quantum mechanics. We see it as central in the sense that
it firmly establishes a notion of physical space arising from a notion
of the equivalence of behavior. Relating bisimulation to a metric is a
big step forward, but one is faced with interpreting the relationship
of that metric space to something more physical. Quantum mechanical
notions of ``physical'' space are still far from intuitive, but by
relating this idea of distance as testing to calculations that predict
physical circumstances we are making a not insignificant step forward
toward an understanding of the physical space we inhabit as
essentially dynamic.

\paragraph{Effectivity and simulation}
One of the observations we have yet to make is that the entire program
spelled out here is effective. We have built various interpreters for
the reflective calculus at work in this interpretation. In principle,
then, we can simulate quantum mechanics on a computer. The place where
the simulation may lose fidelity is the infinitely branching summation
for the annihilator.

In this connection i also want to point out that the evaluation style
calculation of the inner product puts the non-determinism of the
summation right at the heart of measurement. This suggests that
Milner's original reduction-based formulation of the dynamics of his
calculi in terms of sums was not just notationally suggestive of a
notion of measure-and-continue but captured some significant part of
the physics.

\paragraph{Quantum continuations}
In light of this last observation i want to point out that the
predominant account of quantum mechanics is missing a key aspect of a
truly compositional story of the physical situation. In a real lab,
when a measurement is made the observation can be made to feed into
another device that then makes another measurement conditioned on the
results of the first. This means that after the superposition was
collapsed the entire experimental set up remained in
superposition. While QM offers a means of writing this down it doesn't
quite line up well with the well-trodden formulation of computation
and continuation that we see so succinctly expressed in Milner's
calculi. This suggests that there might be advantages to this account
of dynamics waiting to be explored.

\paragraph{Quantum logic}
In this connection, we also note that by virtue of having the
Hennessy-Milner construction, we can pull the construction through the
interpretation of QM. This gives us a natural candidate for a quantum
logic that enjoys an extremely tight connection with it's domain of
interpretation, making the construction much less ad hoc (rather it is
the image of functor!).

\paragraph{Quantum probabiity}
i have questions about the basis of the interpretation of inner
product as probability amplitude. In particular, using which
axiomatization of probability theory does the notion of probability
amplitude earn the right to be so dubbed? In other words, where is the
proof that the operation for calculating a probability amplitude (and
then squaring) satisfies the axioms of what it means to calculate a
probability? Even if such a proof exists (i have yet to find it in the
literature), i wonder if it might not be possible to turn things on
their heads. Can we view the calculation of the probability amplitude
as an axiomatization of probability? If so, then the definition we
give for calculating probability amplitude may provide the basis for
an \emph{effective} theory of probability.

\paragraph{Quantum vs ``biological'' information}
Finally, i want to conclude with a more philosophical observation. At
a recent workshop in which QM was a predominant topic i noticed
something about quantum information. The speaker was giving a riveting
discussion of axiomatic QM and showing how properties of ``no
cloning'' and ``no deleting'' emerged as consequences of the
axiomatization. Theorems of this form are necessary to give us a sense
of confidence that our axioms characterize the physical theory. What
struck me, though, was that if quantum information is neither erasable
nor replicable it is markedly different from \emph{life}. Two of the
things we know about life is that

\begin{itemize}
  \item it ends;
  \item to gain some measure of persistence, to transcend it's
    finitude it is imminently copyable.
\end{itemize}

Both of these qualities are summarized succinctly in the aphorism: all
flesh is grass. For me these two kinds of ``information'' -- call them
quantum and biological -- are end points on a spectrum of strategies
for persistence. At one end, we have those curious entities that enjoy
uniqueness and permanence; at the other, we have those who in the face
of a certain end and an uncertain present make a go of passing
something on. To me one of the more remarkable aspects of the latter
strategy is that in the presence of noise (and certain features of
copying) we get a kind of dynamism, a chance for improvement against a
given persistent condition.

% subsection other_calculi_other_bisimulations_and_geometry_as_behavior (end)




% section conclusion (end)

%\documentclass[12pt]{llncs}
%\documentclass{jktr}

\usepackage[pdftex]{hyperref}                   
\usepackage {listings}
\usepackage {mathpartir}
\usepackage{bcprules}
%\usepackage{listings}
                       
\usepackage{graphicx} 
%\usepackage[margins=2.5cm,nohead,nofoot]{geometry}
%\usepackage{geometry}
\usepackage{amsfonts}
\usepackage{amstext}
\usepackage{latexsym}
\usepackage{amssymb}
\usepackage{color}


%\include{myPreamble}
\include{qm2pi.local} 

%\ifpdf
%\usepackage[pdftex]{graphicx}
%\else
%\usepackage{graphicx}
%\fi

 % \ifpdf
%  \usepackage{pdfsync}
%  \if


%\title{Brief Article}
%\author{David F. Snyder}
%\author{L.G. Meredith}

%\address{Dept. of Math., Texas State University--San Marcos, San Marcos, TX 78666}
       
\pagestyle{empty}


\begin{document}

\lstset{language=[Objective]Caml,frame=shadowbox}

\input{qm2pi.front}

% section front matter (end)

\input{qm2pi.intro} 
 
% section introduction (end)

% \input{qm2pi.knotations} 

% section notation (end)

\input{qm2pi.process.calculi} 

% section concurrent_process_calculi_and_spatial_logics_ (end)
    
%\input{qm2pi.knots2pi} 

%\input{qm2pi.trefoil} 

%\input{qm2pi.mainthm} 

% subsection basic_interpretation (end)

%\input{qm2pi.rho.presentation} 
\subsection{The syntax and semantics of the notation system}\label{sub:the_syntax_and_semantics_of_the_notation_system} % (fold)

We now summarize a technical presentation of the calculus that
embodies our theory of dynamics. The typical presentation of such a
calculus follows the style of giving generators and relations on
them. The grammar, below, describing term constructors, freely
generates the set of processes, $\Proc$. This set is then quotiented
by a relation known as structural congruence and it is over this set
that the notion of dynamics is expressed. This presentation is
essentially that of \cite{MeredithR05} with the addition of
polyadicity and summation. For readability we have relegated some of
the technical subtleties to an appendix.

\subsubsection{Process grammar}\label{subsub:process_grammar}

\begin{mathpar}
  \inferrule* [lab=synchronization] {} {{M} \bc \pzero \;|\; x?F \;|\; x!C }
  \and
  \inferrule* [lab=abstraction] {} {{F} \bc (x)P}
  \and
  \inferrule* [lab=concretion] {} {{C} \bc \langle Q \rangle}
  \and
  \inferrule* [lab=process] {} {{P,Q} \bc M \;| \;P|Q \;|\; @{x}}
  \and
  \inferrule* [lab=name] {} {{x} \bc \quotep{P}}
\end{mathpar} 

Note that $\vec{x}$ (resp. $\vec{P}$) denotes a vector of names
(resp. processes) of length $|\vec{x}|$ (resp. $|\vec{P}|$). We adopt
the following useful abbreviations.

\begin{mathpar}
   x?(\vec{y}).P := x.(\vec{y})P \and  x\clift{\vec{P}} := x.\clift{\vec{P}}
   \and x!(y) := \lift{x}{\dropn{y}}
   \and \Pi_{i=0}^{n-1}P_i := P_0 | \ldots | P_{n-1}
\end{mathpar}

\subsubsection{Structural congruence}

\paragraph{Free and bound names and alpha-equivalence.} At the
core of structural equivalence is alpha-equivalence which identifies
process that are the same up to a change of variable. Formally, we
recognize the distinction between free and bound names. The free names
of a process, $\freenames{P}$, may be calculated recursively as
follows:

\begin{mathpar}
\freenames{\pzero} := \emptyset
  \and \\
  \freenames{x?(y).P} := \{ x \} \cup (\freenames{P} \setminus \{ y \})
  \and 
  \freenames{x!\langle P \rangle} := \{ x \} \cup \{ P \} 
  \and \\
  \freenames{P|Q} := \freenames{P} \cup \freenames{Q}
  \and \\
  \freenames{@{x}} := \{ x \}
\end{mathpar}

$\pi$
$\quotep{\pi}$

$\freenames{-} : \pi \to \mathcal{P}(\quotep{\pi})$

\begin{eqnarray*}
  \freenames{\pzero} & := & \emptyset \\
  \freenames{x?(y).P} & := & \{ x \} \cup (\freenames{P} \setminus \{ y \}) \\
  \freenames{x!\langle P \rangle} & := & \{ x \} \cup \{ P \} \\
  \freenames{P|Q} & := & \freenames{P} \cup \freenames{Q} \\
  \freenames{\dropn{x}} & := & \{ x \}
\end{eqnarray*}

The bound names of a process, $\boundnames{P}$, are those names occurring in $P$
that are not free. For example, in $x?(y).0$, the name $x$ is free, while $y$ is bound.

\begin{mathpar}
  \inferrule* [lab=monoidal-laws] {} { P|Q \equiv Q|P \and P|0 \equiv P \and P|(Q|R) \equiv (P|Q)|R }
\end{mathpar}

\begin{mathpar}
  \inferrule* [lab=alpha-equivalence] {} { (x)P \equiv (y)P\{y/x\} \and y \not\in \freenames{P} }
\end{mathpar}

\begin{definition}
Then two processes, $P,Q$, are alpha-equivalent if $P = Q\{\vec{y}/\vec{x}\}$ for
some $\vec{x} \in \boundnames{Q},\vec{y} \in \boundnames{P}$, where $Q\{\vec{y}/\vec{x}\}$
denotes the capture-avoiding substitution of $\vec{y}$ for $\vec{x}$ in $Q$.
\end{definition}

\begin{definition}
  The {\em structural congruence} \cite{SangiorgiWalker} , $\equiv$,
  between processes is the least congruence containing
  alpha-equivalence, satisfying the abelian monoid laws
  (associativity, commutativity and $\pzero$ as identity) for parallel
  composition $|$ and for summation $+$.
\end{definition}

\subsection{Name equivalence}

We take name equivalence, written $\nameeq$, to be the smallest
equivalence relation generated by the following rules.

\begin{mathpar}
\inferrule*[lab=Quote-drop]
{ }
{ \quotep{@{x}} \nameeq x }

\inferrule*[lab=Struct-equiv]
{ P \scong Q }
{ \quotep{P} \nameeq \quotep{Q} }
\end{mathpar}

The astute reader will have noticed that the mutual recursion of names
and processes imposes a mutual recursion on alpha-equivalence and
structural equivalence via name-equivalence. Fortunately, all of this
works out pleasantly and we may calculate in the natural way, free of
concern. The reader interested in the details is referred to the
appendix \ref{appendix:rho_details}.

\subsection{Substitution}

We use $\Proc$ for the set of processes, $\QProc$ for the set of
names, and $\id{\{}\vec{y} / \vec{x} \id{\}}$ to denote partial maps,
$s : \QProc \rightarrow \QProc$. A map, $s$ lifts, uniquely, to a map
on process terms, $\widehat{s} : \Proc \rightarrow \Proc$ by the
following equations.

\begin{mathpar}
  (0) \psubstp{Q}{P} := 0 \\
  (R \juxtap S) \psubstp{Q}{P}
  :=    
  (R)\psubstp{Q}{P} \juxtap (S) \psubstp{Q}{P} \\
  (x?(y).R) \psubstp{Q}{P}    
  :=    
  (x)\substp{Q}{P} (z)\concat( (R \psubstn{z}{y}) \psubstp{Q}{P} ) \\
  (\lift{x}{R}) \psubstp{Q}{P}  
  :=
  \lift{(x)\substp{Q}{P}}{ R \psubstp{Q}{P} } \\
%   (\dropn{x})  \psubstp{Q}{P}       
%   := 
%   \left\{ 
%     \begin{array}{ccc} 
%       \dropn{\quotep{Q}} & & x \nameeq \quotep{P} \\
%       \dropn{x} & & otherwise \\
%     \end{array}
%   \right. 
  (\dropn{x})  \psubstp{Q}{P}       
  := 
  \left\{ 
    \begin{array}{ccc} 
      Q & & x \nameeq \quotep{P} \\
      \dropn{x} & & otherwise \\
    \end{array}
  \right.
\end{mathpar}
 

where

\begin{eqnarray}
  (x)\id{\{} \lpquote Q \rpquote / \lpquote P \rpquote \id{\}}            = 
  \left\{ 
    \begin{array}{ccc}
      \lpquote Q \rpquote & & x \nameeq \lpquote P \rpquote \\
      x & & otherwise \\
    \end{array}
  \right. \nonumber
\end{eqnarray}

and $z$ is chosen distinct from $\quotep{P}$, $\quotep{Q}$, the free
names in $Q$, and all the names in $R$. Our $\alpha$-equivalence will
be built in the standard way from this substitution.

\begin{remark}\label{rem:no_self_referential_names}
  One consequence of these definitions is that $\forall P. \quotep{P}
  \not\in \freenames{P}$.
\end{remark}

\subsection{ Dynamic quote: an example }

Anticipating something of what's to come, consider applying the
substitution, $\widehat{\id{\{}u / z \id{\}}}$, to the following pair
of processes, $\lift{w}{y!(z)}$ and $w[ \lpquote y!(z) \rpquote ]$.

\begin{eqnarray}
	\lift{w}{y!(z)}\widehat{\id{\{}u / z \id{\}}}
		& = &
		\lift{w}{y!(u)} \nonumber\\
	w[ \lpquote y!(z) \rpquote ] \widehat{ \id{\{}u / z \id{\}} }
		& = &
		w[ \lpquote y!(z) \rpquote ] \nonumber
\end{eqnarray}

Because the body of the process between quotes is impervious to
substitution, we get radically different answers. In fact, by
examining the first process in an input context,
e.g. $x?(z).\lift{w}{y!(z)}$, we see that the process under the lift
operator may be shaped by prefixed inputs binding a name inside it. In
this sense, the lift operator will be seen as a way to dynamically
construct processes before reifying them as names.

Finally equipped with these standard features we can present the
dynamics of the calculus.

\subsubsection{Operational semantics} 

Finally, we introduce the computational dynamics. What marks these
algebras as distinct from other more traditionally studied algebraic
structures, e.g. vector spaces or polynomial rings, is the manner in
which dynamics is captured. In traditional structures, dynamics is typically
expressed through morphisms between such structures, as in linear maps
between vector spaces or morphisms between rings. In algebras
associated with the semantics of computation, the dynamics is
expressed as part of the algebraic structure itself, through a
reduction reduction relation typically denoted by $\red$. Below, we
give a recursive presentation of this relation for the calculus used
in the encoding.

$\red \subseteq \pi \times \pi$
$\red : \pi \to \mathcal{P}(\pi)$

\begin{mathpar}
  \inferrule* [lab=Comm] { \textsf{match}( x_{src}, x_{trgt} ) } { x_{trgt}?(y)P \; | \; x_{src}!\langle {Q} \rangle \red P\{\quotep{Q}/y}\} }
  \and \\
  \inferrule* [lab=Par] {{P} \red {P}'} {{{P} | {Q}} \red {{P}' | {Q}}}
  \and
  \inferrule* [lab=Equiv]{{{P} \scong {P}'} \andalso {{P}' \red {Q}'} \andalso {{Q}' \scong {Q}}}{{P} \red {Q}}
\end{mathpar}

\begin{eqnarray*}
  match_{\equiv} (\quotep{P},\quotep{Q}) & := & P \equiv Q \\
  match_{\dagger}(\quotep{P},\quotep{Q}) & := & \forall R. P|Q \red^{*} R => R \red^{*} 0 \\
  match_{K}(\quotep{P},\quotep{Q}) & := & K \mbox{ for some context } K
\end{eqnarray*}

$u?(x)P | u!\langle Q \rangle \red P\{\quotep{Q}/x\}$

%We write $\wred$ for $\red^*$, and $P\red$ if $\exists Q $ such that $ P \red Q$.
We write $P\red$ if $\exists Q $ such that $ P \red Q$ and $P\not\red$, otherwise.

\section{Replication}

As mentioned before, it is known that replication (and hence
recursion) can be implemented in a higher-order process algebra
\cite{SangiorgiWalker}. As our first example of calculation with the
machinery thus far presented we give the construction explicitly in
the {\rhoc}.

\begin{eqnarray}
	D_{x} & := & \prefix{x}{y}{(\binpar{\outputp{x}{y}}{@{y}})} \nonumber\\
	\bangp_{x}{P} & := & \binpar{{x}!\langle{\binpar{D_{x}}{P}}\rangle}{D_{x}} \nonumber
\end{eqnarray}

\begin{eqnarray}
	\bangp_{x}{P} & & \nonumber\\
	=
	& {x}!\langle{(\prefix{x}{y}{(\outputp{x}{y} | @{y})) | P}}\rangle 
	      | \prefix{x}{y}{(\outputp{x}{y} | @{y})} & \nonumber\\
	\red
	& (\outputp{x}{y} | @{y})\substn{\quotep{(\prefix{x}{y}{(@{y} | \outputp{x}{y})) | P}}}{y} & \nonumber\\
	=
	& \outputp{x}{\quotep{(\prefix{x}{y}{(\outputp{x}{y} | @{y})) | P}}}
	  | {(\prefix{x}{y}{(\outputp{x}{y} | @{y})) | P}} & \nonumber\\
	\red
	& \ldots & \nonumber\\
	\red^*
	& P | P | \ldots & \nonumber
\end{eqnarray}

Of course, this encoding, as an implementation, runs away, unfolding
$\bangp{P}$ eagerly. A lazier and more implementable replication
operator, restricted to input-guarded processes, may be obtained as follows.

\begin{eqnarray}
\bangp{\prefix{u}{v}{P}} 
	:= 
	\binpar{\lift{x}{\prefix{u}{v}{(\binpar{D(x)}{P})}}}{D(x)} \nonumber
\end{eqnarray}

\begin{remark}
  Note that the lazier definition still does not deal with summation
  or mixed summation (i.e. sums over input and output). The reader is
  invited to construct definitions of replication that deal with these
  features. 

  Further, the definitions are parameterized in a name, $x$. Can you,
  gentle reader, make a definition that eliminates this parameter and
  guarantees no accidental interaction between the replication
  machinery and the process being replicated -- i.e. no accidental
  sharing of names used by the process to get its work done and the
  name(s) used by the replication to effect copying. This latter
  revision of the definition of replication is crucial to obtaining
  the expected identity $!!P \sim !P$.
\end{remark}

\begin{remark}\label{rem:paradoxical_combinator}
  The reader familiar with the lambda calculus will have noticed the
  similarity between $D$ and the paradoxical combinator.

  [Ed. note: the existence of this seems to suggest we have to be more
  restrictive on the set of processes and names we admit if we are to
  support no-cloning.]
\end{remark}

\subsubsection{Bisimulation}

The computational dynamics gives rise to another kind of equivalence,
the equivalence of computational behavior. As previously mentioned
this is typically captured \emph{via} some form of bisimulation.

% The notion we use in this paper is weak barbed bisimulation
% \cite{milner91polyadicpi}.

The notion we use in this paper is derived from weak barbed
bisimulation \cite{milner91polyadicpi}. 

\begin{definition}
An \emph{observation relation}, $\downarrow_{\mathcal N}$, over a set
of names, $\mathcal N$, is the smallest relation satisfying the rules
below.

\infrule[Out-barb]{y \in {\mathcal N}, \; x \nameeq y}
		  {\outputp{x}{v} \downarrow_{\mathcal N} x}
\infrule[Par-barb]{\mbox{$P\downarrow_{\mathcal N} x$ or $Q\downarrow_{\mathcal N} x$}}
		  {\binpar{P}{Q} \downarrow_{\mathcal N} x}

We write $P \Downarrow_{\mathcal N} x$ if there is $Q$ such that 
$P \wred Q$ and $Q \downarrow_{\mathcal N} x$.
\end{definition}

\begin{definition}
%\label{def.bbisim}
An  ${\mathcal N}$-\emph{barbed bisimulation} over a set of names, ${\mathcal N}$, is a symmetric binary relation 
${\mathcal S}_{\mathcal N}$ between agents such that $P\rel{S}_{\mathcal N}Q$ implies:
\begin{enumerate}
\item If $P \red P'$ then $Q \wred Q'$ and $P'\rel{S}_{\mathcal N} Q'$.
\item If $P\downarrow_{\mathcal N} x$, then $Q\Downarrow_{\mathcal N} x$.
\end{enumerate}
$P$ is ${\mathcal N}$-barbed bisimilar to $Q$, written
$P \wbbisim_{\mathcal N} Q$, if $P \rel{S}_{\mathcal N} Q$ for some ${\mathcal N}$-barbed bisimulation ${\mathcal S}_{\mathcal N}$.
\end{definition}

$\mathcal{R} \subseteq \pi \times \pi$

$P \mathcal{R} Q => \forall P'. P \red P' \Rightarrow \exists Q'. Q \red Q', P' \mathcal{R} Q'$

$P \vdash x \Rightarrow Q \vdash x$

\begin{mathpar}
  \inferrule*[lab=Out-barb]{x \nameeq y}{{y}!\langle{Q}\rangle \vdash x}
  \and
  \inferrule*[lab=Par-barb]{\mbox{$P\vdash x$ or $Q\vdash x$}}{\binpar{P}{Q} \vdash x}
\end{mathpar}

\subsubsection{Contexts}

One of the principle advantages of computational calculi like the
$\pi$-calculus is a well-defined notion of context,
contextual-equivalence and a correlation between
contextual-equivalence and notions of bisimulation. The notion of
context allows the decomposition of a process into (sub-)process and
its syntactic environment, its context. Thus, a context may be
thought of as a process with a ``hole'' (written $\Box$) in it. The
application of a context $M$ to a process $P$, written $M[P]$, is
tantamount to filling the hole in $M$ with $P$. In this paper we do
not need the full weight of this theory, but do make use of the notion
of context in the proof the main theorem. 

\begin{mathpar}
  \inferrule* [lab=summation] {} {{M_{M},M_{N}} \bc \Box \;|\; x.M_{A} \;|\; M_{M}+M_{N}}
  \and
  \inferrule* [lab=agent] {} {{M_{A}} \bc (\vec{x})M_{P} \;| \; \clift{P_0,\ldots,M_{P},\ldots,P_N}}
  \and \\
  \inferrule* [lab=process] {} {{M_{P}} \bc M_{N} \;| \;P|M_{P} }
\end{mathpar} 

\begin{mathpar}
  \inferrule* [lab=sychronization] {} {M_{N} \bc \Box \;|\; x?M_{F} \;|\; x!M_{C}}
  \and
  \inferrule* [lab=abstraction] {} {{M_{F}} \bc (x)M_{P} }
  \and
  \inferrule* [lab=concretion] {} {{M_{C}} \bc \langle M_{P} \rangle }
  \and \\
  \inferrule* [lab=process] {} {{M_{P}} \bc M_{N} \;| \;P|M_{P} }
\end{mathpar}

\begin{definition}[contextual application] Given a context $M$, and
  process $P$, we define the \emph{contextual application}, $M[P] :=
  M\{P/\Box\}$. That is, the contextual application of M to P is the
  substitution of $P$ for $\Box$ in $M$.
\end{definition}

$\meaningof{-} : L \to \mathcal{P}(\pi)$

\begin{mathpar}
  \inferrule* [lab=collection] {} {\meaningof{true} = \pi, \and \meaningof{~E} = \pi \setminus \meaningof{E}, \and \meaningof{E_{1} \& E_{2}} = \meaningof{E_{1}} \cap \meaningof{E_{2}}}
\end{mathpar}

\begin{mathpar}
  \inferrule* [lab=structure] {} {\meaningof{0} = \{ P \in \pi | P \equiv 0 \}, \and \\ \meaningof{E_1 | E_2} = \{ P \in \pi | P \equiv P_{1} | P_{2}, P_{1} \in \meaningof{E_{1}}, P_{2} \in \meaningof{E_2}\} }
\end{mathpar}

\begin{mathpar}
 \inferrule* [lab=behavior] {} {\meaningof{\langle a?b \rangle E} = \{ P \in \pi | P \equiv Q | u?(y)P', \\ \and \\\\ \and \\ \;\;\; u \in \meaningof{a}, \forall z.P'\{z/y\} \in \meaningof{E\{z/b\}}\}, \and \\ \meaningof{a!E} = \{ P \in \pi | P \equiv Q | x!\langle P' \rangle, x \in \meaningof{a} P' \in \meaningof{E}\} }
\end{mathpar}

\begin{mathpar}
 \inferrule* [lab=nominal] {} {\meaningof{\quotep{E}} = \{ \quotep{P} \in \quotep{\pi} | P \in \meaningof{E} \}, \and \meaningof{\quotep{P}} = \{ \quotep{Q} \in \quotep{\pi} | P \equiv Q \} \and \\ \meaningof{@\quotep{E}} = \{ P \in \pi | P \equiv @x, x \in \meaningof{E} \}}
\end{mathpar}

\begin{eqnarray*}
  \\
  \meaningof{-} : TS \to ST
\end{eqnarray*}

\begin{eqnarray*}
  \\
  L : TS \to ST
\end{eqnarray*}

\begin{eqnarray*}
  \\
  P \models E \iff P \in \meaningof{E}
\end{eqnarray*}

\begin{eqnarray*}
  P \approx_{L} Q \iff \forall E \in L. P \models E \iff Q \models E
\end{eqnarray*}

\begin{eqnarray*}
  P \approx_{K} Q
\end{eqnarray*}

\begin{eqnarray*}
  P \approx Q
\end{eqnarray*}

$\approx_{K} = \approx = \approx_{L}$

\subsubsection{Contextual duality}

Note that contexts extend the quotation operation to a family of
operations from processes to names. Given a context, $M$, we can
define a \emph{nominal context}, $\quotep{M}$ by $\quotep{M}[P] :=
\quotep{M[P]}$. To foreshadow what is to come we observe that these
operations enjoy a duality with processes very much like the duality
between vectors and maps from vectors to scalars.

Further, because the calculus is essentially higher-order, we have a
correspondence between contexts and processes. More specifically,
given a name $x$ and a context $M$ we can construct $M^{*}_{x}$ such
that 

\begin{mathpar}
  M^{*}_{x} | \lift{x}{P} \red M[P]
\end{mathpar}

namely,

\begin{mathpar}
  M^{*}_{x} := x?(u).M[\dropn{u}]
\end{mathpar}

The dependence of $M^{*}_{x}$ on a name makes it an abstraction, 

\begin{mathpar}
  M^{*} := (x)x?(u).M[\dropn{u}]
\end{mathpar}

\subsection{Additional notation}

It will sometimes be convenient to denote the process a name
quotes. We already have the notation $x = \quotep{P}$, but it will be
convenient to introduce an alternate notation, $\procn{x}$, when we
want to emphasize the connection to the use of the name. Note that, by
virtue of name equivalence, $\quotep{\procn{x}} \nameeq x$; so, the
notation is consistent with previous definitions.

Further, because names have structure it is possible to effect
substitutions on the basis of that structure. This means we need to
upgrade our notation for substitutions, which we accomplish by
adapting comprehension notation. Thus,

\begin{mathpar}
  P\{ y / x : x \in S \}
\end{mathpar}

is interpreted to mean the process derived from P by replacing (in a
capture-avoiding manner) each occurrence of $x$ in $S$ by $y$. For example,

\begin{mathpar}
  P\{ \quotep{\procn{x}|\procn{x}} / x : x \in \freenames{P} \}
\end{mathpar}

will replace each (occurrence) of a free name $x$ in $P$ by
$\quotep{\procn{x}|\procn{x}}$.

Also, we will avail ourselves of the notation $x^{L}$ and $x^{R}$ to
denote injections of a name into disjoint copies of the name
space. There are numerous ways to accomplish this. One example can be
found in \cite{MeredithR05}. This notation overloads to vectors of
names: $\vec{x}^{\pi} := (x_{i}^{\pi} \; : \; 0 \leq i < |\vec{x}| )$ where $\pi \in \{L,R\}$.

We also use $P^{\Box} := P|\Box$.

In \cite{MeredithR05} an interpretation of the new operator is
given. It turns out that there are several possible interpretations
all enjoying the requisite algebraic properties of the operator (see
\cite{milner91polyadicpi}). We will therefore make liberal use of
$(\nu\; \vec{x})P$.

% subsection the_syntax_and_semantics_of_the_notation_system (end)   

\input{qm2pi.qmops} 

\input{qm2pi.sterngerlach} 

\input{qm2pi.metric} 

% section concurrent_process_calculi (end)

%\input{qm2pi.proofsketch}

% section proof sketch (end)

%\input{qm2pi.slviaknots} 

% section spatial logic via knots (end)

\input{qm2pi.conclusion}

% section conclusion (end)

%\input{qm2pi.dtcodes} 

% section wiring algorithm (end)

\input{qm2pi.ack} 

% section acknowledgments (end)

\newpage


\bibliographystyle{plain}   
\bibliography{../../biblios/main.bib}

\input{qm2pi.rhodetails}

\end{document}

 

% section wiring algorithm (end)

\documentclass[12pt]{llncs}
%\documentclass{jktr}

\usepackage[pdftex]{hyperref}                   
\usepackage {listings}
\usepackage {mathpartir}
\usepackage{bcprules}
%\usepackage{listings}
                       
\usepackage{graphicx} 
%\usepackage[margins=2.5cm,nohead,nofoot]{geometry}
%\usepackage{geometry}
\usepackage{amsfonts}
\usepackage{amstext}
\usepackage{latexsym}
\usepackage{amssymb}
\usepackage{color}


%\include{myPreamble}
\include{qm2pi.local} 

%\ifpdf
%\usepackage[pdftex]{graphicx}
%\else
%\usepackage{graphicx}
%\fi

 % \ifpdf
%  \usepackage{pdfsync}
%  \if


%\title{Brief Article}
%\author{David F. Snyder}
%\author{L.G. Meredith}

%\address{Dept. of Math., Texas State University--San Marcos, San Marcos, TX 78666}
       
\pagestyle{empty}


\begin{document}

\lstset{language=[Objective]Caml,frame=shadowbox}

\input{qm2pi.front}

% section front matter (end)

\input{qm2pi.intro} 
 
% section introduction (end)

% \input{qm2pi.knotations} 

% section notation (end)

\input{qm2pi.process.calculi} 

% section concurrent_process_calculi_and_spatial_logics_ (end)
    
%\input{qm2pi.knots2pi} 

%\input{qm2pi.trefoil} 

%\input{qm2pi.mainthm} 

% subsection basic_interpretation (end)

%\input{qm2pi.rho.presentation} 
\subsection{The syntax and semantics of the notation system}\label{sub:the_syntax_and_semantics_of_the_notation_system} % (fold)

We now summarize a technical presentation of the calculus that
embodies our theory of dynamics. The typical presentation of such a
calculus follows the style of giving generators and relations on
them. The grammar, below, describing term constructors, freely
generates the set of processes, $\Proc$. This set is then quotiented
by a relation known as structural congruence and it is over this set
that the notion of dynamics is expressed. This presentation is
essentially that of \cite{MeredithR05} with the addition of
polyadicity and summation. For readability we have relegated some of
the technical subtleties to an appendix.

\subsubsection{Process grammar}\label{subsub:process_grammar}

\begin{mathpar}
  \inferrule* [lab=synchronization] {} {{M} \bc \pzero \;|\; x?F \;|\; x!C }
  \and
  \inferrule* [lab=abstraction] {} {{F} \bc (x)P}
  \and
  \inferrule* [lab=concretion] {} {{C} \bc \langle Q \rangle}
  \and
  \inferrule* [lab=process] {} {{P,Q} \bc M \;| \;P|Q \;|\; @{x}}
  \and
  \inferrule* [lab=name] {} {{x} \bc \quotep{P}}
\end{mathpar} 

Note that $\vec{x}$ (resp. $\vec{P}$) denotes a vector of names
(resp. processes) of length $|\vec{x}|$ (resp. $|\vec{P}|$). We adopt
the following useful abbreviations.

\begin{mathpar}
   x?(\vec{y}).P := x.(\vec{y})P \and  x\clift{\vec{P}} := x.\clift{\vec{P}}
   \and x!(y) := \lift{x}{\dropn{y}}
   \and \Pi_{i=0}^{n-1}P_i := P_0 | \ldots | P_{n-1}
\end{mathpar}

\subsubsection{Structural congruence}

\paragraph{Free and bound names and alpha-equivalence.} At the
core of structural equivalence is alpha-equivalence which identifies
process that are the same up to a change of variable. Formally, we
recognize the distinction between free and bound names. The free names
of a process, $\freenames{P}$, may be calculated recursively as
follows:

\begin{mathpar}
\freenames{\pzero} := \emptyset
  \and \\
  \freenames{x?(y).P} := \{ x \} \cup (\freenames{P} \setminus \{ y \})
  \and 
  \freenames{x!\langle P \rangle} := \{ x \} \cup \{ P \} 
  \and \\
  \freenames{P|Q} := \freenames{P} \cup \freenames{Q}
  \and \\
  \freenames{@{x}} := \{ x \}
\end{mathpar}

$\pi$
$\quotep{\pi}$

$\freenames{-} : \pi \to \mathcal{P}(\quotep{\pi})$

\begin{eqnarray*}
  \freenames{\pzero} & := & \emptyset \\
  \freenames{x?(y).P} & := & \{ x \} \cup (\freenames{P} \setminus \{ y \}) \\
  \freenames{x!\langle P \rangle} & := & \{ x \} \cup \{ P \} \\
  \freenames{P|Q} & := & \freenames{P} \cup \freenames{Q} \\
  \freenames{\dropn{x}} & := & \{ x \}
\end{eqnarray*}

The bound names of a process, $\boundnames{P}$, are those names occurring in $P$
that are not free. For example, in $x?(y).0$, the name $x$ is free, while $y$ is bound.

\begin{mathpar}
  \inferrule* [lab=monoidal-laws] {} { P|Q \equiv Q|P \and P|0 \equiv P \and P|(Q|R) \equiv (P|Q)|R }
\end{mathpar}

\begin{mathpar}
  \inferrule* [lab=alpha-equivalence] {} { (x)P \equiv (y)P\{y/x\} \and y \not\in \freenames{P} }
\end{mathpar}

\begin{definition}
Then two processes, $P,Q$, are alpha-equivalent if $P = Q\{\vec{y}/\vec{x}\}$ for
some $\vec{x} \in \boundnames{Q},\vec{y} \in \boundnames{P}$, where $Q\{\vec{y}/\vec{x}\}$
denotes the capture-avoiding substitution of $\vec{y}$ for $\vec{x}$ in $Q$.
\end{definition}

\begin{definition}
  The {\em structural congruence} \cite{SangiorgiWalker} , $\equiv$,
  between processes is the least congruence containing
  alpha-equivalence, satisfying the abelian monoid laws
  (associativity, commutativity and $\pzero$ as identity) for parallel
  composition $|$ and for summation $+$.
\end{definition}

\subsection{Name equivalence}

We take name equivalence, written $\nameeq$, to be the smallest
equivalence relation generated by the following rules.

\begin{mathpar}
\inferrule*[lab=Quote-drop]
{ }
{ \quotep{@{x}} \nameeq x }

\inferrule*[lab=Struct-equiv]
{ P \scong Q }
{ \quotep{P} \nameeq \quotep{Q} }
\end{mathpar}

The astute reader will have noticed that the mutual recursion of names
and processes imposes a mutual recursion on alpha-equivalence and
structural equivalence via name-equivalence. Fortunately, all of this
works out pleasantly and we may calculate in the natural way, free of
concern. The reader interested in the details is referred to the
appendix \ref{appendix:rho_details}.

\subsection{Substitution}

We use $\Proc$ for the set of processes, $\QProc$ for the set of
names, and $\id{\{}\vec{y} / \vec{x} \id{\}}$ to denote partial maps,
$s : \QProc \rightarrow \QProc$. A map, $s$ lifts, uniquely, to a map
on process terms, $\widehat{s} : \Proc \rightarrow \Proc$ by the
following equations.

\begin{mathpar}
  (0) \psubstp{Q}{P} := 0 \\
  (R \juxtap S) \psubstp{Q}{P}
  :=    
  (R)\psubstp{Q}{P} \juxtap (S) \psubstp{Q}{P} \\
  (x?(y).R) \psubstp{Q}{P}    
  :=    
  (x)\substp{Q}{P} (z)\concat( (R \psubstn{z}{y}) \psubstp{Q}{P} ) \\
  (\lift{x}{R}) \psubstp{Q}{P}  
  :=
  \lift{(x)\substp{Q}{P}}{ R \psubstp{Q}{P} } \\
%   (\dropn{x})  \psubstp{Q}{P}       
%   := 
%   \left\{ 
%     \begin{array}{ccc} 
%       \dropn{\quotep{Q}} & & x \nameeq \quotep{P} \\
%       \dropn{x} & & otherwise \\
%     \end{array}
%   \right. 
  (\dropn{x})  \psubstp{Q}{P}       
  := 
  \left\{ 
    \begin{array}{ccc} 
      Q & & x \nameeq \quotep{P} \\
      \dropn{x} & & otherwise \\
    \end{array}
  \right.
\end{mathpar}
 

where

\begin{eqnarray}
  (x)\id{\{} \lpquote Q \rpquote / \lpquote P \rpquote \id{\}}            = 
  \left\{ 
    \begin{array}{ccc}
      \lpquote Q \rpquote & & x \nameeq \lpquote P \rpquote \\
      x & & otherwise \\
    \end{array}
  \right. \nonumber
\end{eqnarray}

and $z$ is chosen distinct from $\quotep{P}$, $\quotep{Q}$, the free
names in $Q$, and all the names in $R$. Our $\alpha$-equivalence will
be built in the standard way from this substitution.

\begin{remark}\label{rem:no_self_referential_names}
  One consequence of these definitions is that $\forall P. \quotep{P}
  \not\in \freenames{P}$.
\end{remark}

\subsection{ Dynamic quote: an example }

Anticipating something of what's to come, consider applying the
substitution, $\widehat{\id{\{}u / z \id{\}}}$, to the following pair
of processes, $\lift{w}{y!(z)}$ and $w[ \lpquote y!(z) \rpquote ]$.

\begin{eqnarray}
	\lift{w}{y!(z)}\widehat{\id{\{}u / z \id{\}}}
		& = &
		\lift{w}{y!(u)} \nonumber\\
	w[ \lpquote y!(z) \rpquote ] \widehat{ \id{\{}u / z \id{\}} }
		& = &
		w[ \lpquote y!(z) \rpquote ] \nonumber
\end{eqnarray}

Because the body of the process between quotes is impervious to
substitution, we get radically different answers. In fact, by
examining the first process in an input context,
e.g. $x?(z).\lift{w}{y!(z)}$, we see that the process under the lift
operator may be shaped by prefixed inputs binding a name inside it. In
this sense, the lift operator will be seen as a way to dynamically
construct processes before reifying them as names.

Finally equipped with these standard features we can present the
dynamics of the calculus.

\subsubsection{Operational semantics} 

Finally, we introduce the computational dynamics. What marks these
algebras as distinct from other more traditionally studied algebraic
structures, e.g. vector spaces or polynomial rings, is the manner in
which dynamics is captured. In traditional structures, dynamics is typically
expressed through morphisms between such structures, as in linear maps
between vector spaces or morphisms between rings. In algebras
associated with the semantics of computation, the dynamics is
expressed as part of the algebraic structure itself, through a
reduction reduction relation typically denoted by $\red$. Below, we
give a recursive presentation of this relation for the calculus used
in the encoding.

$\red \subseteq \pi \times \pi$
$\red : \pi \to \mathcal{P}(\pi)$

\begin{mathpar}
  \inferrule* [lab=Comm] { \textsf{match}( x_{src}, x_{trgt} ) } { x_{trgt}?(y)P \; | \; x_{src}!\langle {Q} \rangle \red P\{\quotep{Q}/y}\} }
  \and \\
  \inferrule* [lab=Par] {{P} \red {P}'} {{{P} | {Q}} \red {{P}' | {Q}}}
  \and
  \inferrule* [lab=Equiv]{{{P} \scong {P}'} \andalso {{P}' \red {Q}'} \andalso {{Q}' \scong {Q}}}{{P} \red {Q}}
\end{mathpar}

\begin{eqnarray*}
  match_{\equiv} (\quotep{P},\quotep{Q}) & := & P \equiv Q \\
  match_{\dagger}(\quotep{P},\quotep{Q}) & := & \forall R. P|Q \red^{*} R => R \red^{*} 0 \\
  match_{K}(\quotep{P},\quotep{Q}) & := & K \mbox{ for some context } K
\end{eqnarray*}

$u?(x)P | u!\langle Q \rangle \red P\{\quotep{Q}/x\}$

%We write $\wred$ for $\red^*$, and $P\red$ if $\exists Q $ such that $ P \red Q$.
We write $P\red$ if $\exists Q $ such that $ P \red Q$ and $P\not\red$, otherwise.

\section{Replication}

As mentioned before, it is known that replication (and hence
recursion) can be implemented in a higher-order process algebra
\cite{SangiorgiWalker}. As our first example of calculation with the
machinery thus far presented we give the construction explicitly in
the {\rhoc}.

\begin{eqnarray}
	D_{x} & := & \prefix{x}{y}{(\binpar{\outputp{x}{y}}{@{y}})} \nonumber\\
	\bangp_{x}{P} & := & \binpar{{x}!\langle{\binpar{D_{x}}{P}}\rangle}{D_{x}} \nonumber
\end{eqnarray}

\begin{eqnarray}
	\bangp_{x}{P} & & \nonumber\\
	=
	& {x}!\langle{(\prefix{x}{y}{(\outputp{x}{y} | @{y})) | P}}\rangle 
	      | \prefix{x}{y}{(\outputp{x}{y} | @{y})} & \nonumber\\
	\red
	& (\outputp{x}{y} | @{y})\substn{\quotep{(\prefix{x}{y}{(@{y} | \outputp{x}{y})) | P}}}{y} & \nonumber\\
	=
	& \outputp{x}{\quotep{(\prefix{x}{y}{(\outputp{x}{y} | @{y})) | P}}}
	  | {(\prefix{x}{y}{(\outputp{x}{y} | @{y})) | P}} & \nonumber\\
	\red
	& \ldots & \nonumber\\
	\red^*
	& P | P | \ldots & \nonumber
\end{eqnarray}

Of course, this encoding, as an implementation, runs away, unfolding
$\bangp{P}$ eagerly. A lazier and more implementable replication
operator, restricted to input-guarded processes, may be obtained as follows.

\begin{eqnarray}
\bangp{\prefix{u}{v}{P}} 
	:= 
	\binpar{\lift{x}{\prefix{u}{v}{(\binpar{D(x)}{P})}}}{D(x)} \nonumber
\end{eqnarray}

\begin{remark}
  Note that the lazier definition still does not deal with summation
  or mixed summation (i.e. sums over input and output). The reader is
  invited to construct definitions of replication that deal with these
  features. 

  Further, the definitions are parameterized in a name, $x$. Can you,
  gentle reader, make a definition that eliminates this parameter and
  guarantees no accidental interaction between the replication
  machinery and the process being replicated -- i.e. no accidental
  sharing of names used by the process to get its work done and the
  name(s) used by the replication to effect copying. This latter
  revision of the definition of replication is crucial to obtaining
  the expected identity $!!P \sim !P$.
\end{remark}

\begin{remark}\label{rem:paradoxical_combinator}
  The reader familiar with the lambda calculus will have noticed the
  similarity between $D$ and the paradoxical combinator.

  [Ed. note: the existence of this seems to suggest we have to be more
  restrictive on the set of processes and names we admit if we are to
  support no-cloning.]
\end{remark}

\subsubsection{Bisimulation}

The computational dynamics gives rise to another kind of equivalence,
the equivalence of computational behavior. As previously mentioned
this is typically captured \emph{via} some form of bisimulation.

% The notion we use in this paper is weak barbed bisimulation
% \cite{milner91polyadicpi}.

The notion we use in this paper is derived from weak barbed
bisimulation \cite{milner91polyadicpi}. 

\begin{definition}
An \emph{observation relation}, $\downarrow_{\mathcal N}$, over a set
of names, $\mathcal N$, is the smallest relation satisfying the rules
below.

\infrule[Out-barb]{y \in {\mathcal N}, \; x \nameeq y}
		  {\outputp{x}{v} \downarrow_{\mathcal N} x}
\infrule[Par-barb]{\mbox{$P\downarrow_{\mathcal N} x$ or $Q\downarrow_{\mathcal N} x$}}
		  {\binpar{P}{Q} \downarrow_{\mathcal N} x}

We write $P \Downarrow_{\mathcal N} x$ if there is $Q$ such that 
$P \wred Q$ and $Q \downarrow_{\mathcal N} x$.
\end{definition}

\begin{definition}
%\label{def.bbisim}
An  ${\mathcal N}$-\emph{barbed bisimulation} over a set of names, ${\mathcal N}$, is a symmetric binary relation 
${\mathcal S}_{\mathcal N}$ between agents such that $P\rel{S}_{\mathcal N}Q$ implies:
\begin{enumerate}
\item If $P \red P'$ then $Q \wred Q'$ and $P'\rel{S}_{\mathcal N} Q'$.
\item If $P\downarrow_{\mathcal N} x$, then $Q\Downarrow_{\mathcal N} x$.
\end{enumerate}
$P$ is ${\mathcal N}$-barbed bisimilar to $Q$, written
$P \wbbisim_{\mathcal N} Q$, if $P \rel{S}_{\mathcal N} Q$ for some ${\mathcal N}$-barbed bisimulation ${\mathcal S}_{\mathcal N}$.
\end{definition}

$\mathcal{R} \subseteq \pi \times \pi$

$P \mathcal{R} Q => \forall P'. P \red P' \Rightarrow \exists Q'. Q \red Q', P' \mathcal{R} Q'$

$P \vdash x \Rightarrow Q \vdash x$

\begin{mathpar}
  \inferrule*[lab=Out-barb]{x \nameeq y}{{y}!\langle{Q}\rangle \vdash x}
  \and
  \inferrule*[lab=Par-barb]{\mbox{$P\vdash x$ or $Q\vdash x$}}{\binpar{P}{Q} \vdash x}
\end{mathpar}

\subsubsection{Contexts}

One of the principle advantages of computational calculi like the
$\pi$-calculus is a well-defined notion of context,
contextual-equivalence and a correlation between
contextual-equivalence and notions of bisimulation. The notion of
context allows the decomposition of a process into (sub-)process and
its syntactic environment, its context. Thus, a context may be
thought of as a process with a ``hole'' (written $\Box$) in it. The
application of a context $M$ to a process $P$, written $M[P]$, is
tantamount to filling the hole in $M$ with $P$. In this paper we do
not need the full weight of this theory, but do make use of the notion
of context in the proof the main theorem. 

\begin{mathpar}
  \inferrule* [lab=summation] {} {{M_{M},M_{N}} \bc \Box \;|\; x.M_{A} \;|\; M_{M}+M_{N}}
  \and
  \inferrule* [lab=agent] {} {{M_{A}} \bc (\vec{x})M_{P} \;| \; \clift{P_0,\ldots,M_{P},\ldots,P_N}}
  \and \\
  \inferrule* [lab=process] {} {{M_{P}} \bc M_{N} \;| \;P|M_{P} }
\end{mathpar} 

\begin{mathpar}
  \inferrule* [lab=sychronization] {} {M_{N} \bc \Box \;|\; x?M_{F} \;|\; x!M_{C}}
  \and
  \inferrule* [lab=abstraction] {} {{M_{F}} \bc (x)M_{P} }
  \and
  \inferrule* [lab=concretion] {} {{M_{C}} \bc \langle M_{P} \rangle }
  \and \\
  \inferrule* [lab=process] {} {{M_{P}} \bc M_{N} \;| \;P|M_{P} }
\end{mathpar}

\begin{definition}[contextual application] Given a context $M$, and
  process $P$, we define the \emph{contextual application}, $M[P] :=
  M\{P/\Box\}$. That is, the contextual application of M to P is the
  substitution of $P$ for $\Box$ in $M$.
\end{definition}

$\meaningof{-} : L \to \mathcal{P}(\pi)$

\begin{mathpar}
  \inferrule* [lab=collection] {} {\meaningof{true} = \pi, \and \meaningof{~E} = \pi \setminus \meaningof{E}, \and \meaningof{E_{1} \& E_{2}} = \meaningof{E_{1}} \cap \meaningof{E_{2}}}
\end{mathpar}

\begin{mathpar}
  \inferrule* [lab=structure] {} {\meaningof{0} = \{ P \in \pi | P \equiv 0 \}, \and \\ \meaningof{E_1 | E_2} = \{ P \in \pi | P \equiv P_{1} | P_{2}, P_{1} \in \meaningof{E_{1}}, P_{2} \in \meaningof{E_2}\} }
\end{mathpar}

\begin{mathpar}
 \inferrule* [lab=behavior] {} {\meaningof{\langle a?b \rangle E} = \{ P \in \pi | P \equiv Q | u?(y)P', \\ \and \\\\ \and \\ \;\;\; u \in \meaningof{a}, \forall z.P'\{z/y\} \in \meaningof{E\{z/b\}}\}, \and \\ \meaningof{a!E} = \{ P \in \pi | P \equiv Q | x!\langle P' \rangle, x \in \meaningof{a} P' \in \meaningof{E}\} }
\end{mathpar}

\begin{mathpar}
 \inferrule* [lab=nominal] {} {\meaningof{\quotep{E}} = \{ \quotep{P} \in \quotep{\pi} | P \in \meaningof{E} \}, \and \meaningof{\quotep{P}} = \{ \quotep{Q} \in \quotep{\pi} | P \equiv Q \} \and \\ \meaningof{@\quotep{E}} = \{ P \in \pi | P \equiv @x, x \in \meaningof{E} \}}
\end{mathpar}

\begin{eqnarray*}
  \\
  \meaningof{-} : TS \to ST
\end{eqnarray*}

\begin{eqnarray*}
  \\
  L : TS \to ST
\end{eqnarray*}

\begin{eqnarray*}
  \\
  P \models E \iff P \in \meaningof{E}
\end{eqnarray*}

\begin{eqnarray*}
  P \approx_{L} Q \iff \forall E \in L. P \models E \iff Q \models E
\end{eqnarray*}

\begin{eqnarray*}
  P \approx_{K} Q
\end{eqnarray*}

\begin{eqnarray*}
  P \approx Q
\end{eqnarray*}

$\approx_{K} = \approx = \approx_{L}$

\subsubsection{Contextual duality}

Note that contexts extend the quotation operation to a family of
operations from processes to names. Given a context, $M$, we can
define a \emph{nominal context}, $\quotep{M}$ by $\quotep{M}[P] :=
\quotep{M[P]}$. To foreshadow what is to come we observe that these
operations enjoy a duality with processes very much like the duality
between vectors and maps from vectors to scalars.

Further, because the calculus is essentially higher-order, we have a
correspondence between contexts and processes. More specifically,
given a name $x$ and a context $M$ we can construct $M^{*}_{x}$ such
that 

\begin{mathpar}
  M^{*}_{x} | \lift{x}{P} \red M[P]
\end{mathpar}

namely,

\begin{mathpar}
  M^{*}_{x} := x?(u).M[\dropn{u}]
\end{mathpar}

The dependence of $M^{*}_{x}$ on a name makes it an abstraction, 

\begin{mathpar}
  M^{*} := (x)x?(u).M[\dropn{u}]
\end{mathpar}

\subsection{Additional notation}

It will sometimes be convenient to denote the process a name
quotes. We already have the notation $x = \quotep{P}$, but it will be
convenient to introduce an alternate notation, $\procn{x}$, when we
want to emphasize the connection to the use of the name. Note that, by
virtue of name equivalence, $\quotep{\procn{x}} \nameeq x$; so, the
notation is consistent with previous definitions.

Further, because names have structure it is possible to effect
substitutions on the basis of that structure. This means we need to
upgrade our notation for substitutions, which we accomplish by
adapting comprehension notation. Thus,

\begin{mathpar}
  P\{ y / x : x \in S \}
\end{mathpar}

is interpreted to mean the process derived from P by replacing (in a
capture-avoiding manner) each occurrence of $x$ in $S$ by $y$. For example,

\begin{mathpar}
  P\{ \quotep{\procn{x}|\procn{x}} / x : x \in \freenames{P} \}
\end{mathpar}

will replace each (occurrence) of a free name $x$ in $P$ by
$\quotep{\procn{x}|\procn{x}}$.

Also, we will avail ourselves of the notation $x^{L}$ and $x^{R}$ to
denote injections of a name into disjoint copies of the name
space. There are numerous ways to accomplish this. One example can be
found in \cite{MeredithR05}. This notation overloads to vectors of
names: $\vec{x}^{\pi} := (x_{i}^{\pi} \; : \; 0 \leq i < |\vec{x}| )$ where $\pi \in \{L,R\}$.

We also use $P^{\Box} := P|\Box$.

In \cite{MeredithR05} an interpretation of the new operator is
given. It turns out that there are several possible interpretations
all enjoying the requisite algebraic properties of the operator (see
\cite{milner91polyadicpi}). We will therefore make liberal use of
$(\nu\; \vec{x})P$.

% subsection the_syntax_and_semantics_of_the_notation_system (end)   

\input{qm2pi.qmops} 

\input{qm2pi.sterngerlach} 

\input{qm2pi.metric} 

% section concurrent_process_calculi (end)

%\input{qm2pi.proofsketch}

% section proof sketch (end)

%\input{qm2pi.slviaknots} 

% section spatial logic via knots (end)

\input{qm2pi.conclusion}

% section conclusion (end)

%\input{qm2pi.dtcodes} 

% section wiring algorithm (end)

\input{qm2pi.ack} 

% section acknowledgments (end)

\newpage


\bibliographystyle{plain}   
\bibliography{../../biblios/main.bib}

\input{qm2pi.rhodetails}

\end{document}

 

% section acknowledgments (end)

\newpage


\bibliographystyle{plain}   
\bibliography{../../biblios/main.bib}

\documentclass[12pt]{llncs}
%\documentclass{jktr}

\usepackage[pdftex]{hyperref}                   
\usepackage {listings}
\usepackage {mathpartir}
\usepackage{bcprules}
%\usepackage{listings}
                       
\usepackage{graphicx} 
%\usepackage[margins=2.5cm,nohead,nofoot]{geometry}
%\usepackage{geometry}
\usepackage{amsfonts}
\usepackage{amstext}
\usepackage{latexsym}
\usepackage{amssymb}
\usepackage{color}


%\include{myPreamble}
\include{qm2pi.local} 

%\ifpdf
%\usepackage[pdftex]{graphicx}
%\else
%\usepackage{graphicx}
%\fi

 % \ifpdf
%  \usepackage{pdfsync}
%  \if


%\title{Brief Article}
%\author{David F. Snyder}
%\author{L.G. Meredith}

%\address{Dept. of Math., Texas State University--San Marcos, San Marcos, TX 78666}
       
\pagestyle{empty}


\begin{document}

\lstset{language=[Objective]Caml,frame=shadowbox}

\input{qm2pi.front}

% section front matter (end)

\input{qm2pi.intro} 
 
% section introduction (end)

% \input{qm2pi.knotations} 

% section notation (end)

\input{qm2pi.process.calculi} 

% section concurrent_process_calculi_and_spatial_logics_ (end)
    
%\input{qm2pi.knots2pi} 

%\input{qm2pi.trefoil} 

%\input{qm2pi.mainthm} 

% subsection basic_interpretation (end)

%\input{qm2pi.rho.presentation} 
\subsection{The syntax and semantics of the notation system}\label{sub:the_syntax_and_semantics_of_the_notation_system} % (fold)

We now summarize a technical presentation of the calculus that
embodies our theory of dynamics. The typical presentation of such a
calculus follows the style of giving generators and relations on
them. The grammar, below, describing term constructors, freely
generates the set of processes, $\Proc$. This set is then quotiented
by a relation known as structural congruence and it is over this set
that the notion of dynamics is expressed. This presentation is
essentially that of \cite{MeredithR05} with the addition of
polyadicity and summation. For readability we have relegated some of
the technical subtleties to an appendix.

\subsubsection{Process grammar}\label{subsub:process_grammar}

\begin{mathpar}
  \inferrule* [lab=synchronization] {} {{M} \bc \pzero \;|\; x?F \;|\; x!C }
  \and
  \inferrule* [lab=abstraction] {} {{F} \bc (x)P}
  \and
  \inferrule* [lab=concretion] {} {{C} \bc \langle Q \rangle}
  \and
  \inferrule* [lab=process] {} {{P,Q} \bc M \;| \;P|Q \;|\; @{x}}
  \and
  \inferrule* [lab=name] {} {{x} \bc \quotep{P}}
\end{mathpar} 

Note that $\vec{x}$ (resp. $\vec{P}$) denotes a vector of names
(resp. processes) of length $|\vec{x}|$ (resp. $|\vec{P}|$). We adopt
the following useful abbreviations.

\begin{mathpar}
   x?(\vec{y}).P := x.(\vec{y})P \and  x\clift{\vec{P}} := x.\clift{\vec{P}}
   \and x!(y) := \lift{x}{\dropn{y}}
   \and \Pi_{i=0}^{n-1}P_i := P_0 | \ldots | P_{n-1}
\end{mathpar}

\subsubsection{Structural congruence}

\paragraph{Free and bound names and alpha-equivalence.} At the
core of structural equivalence is alpha-equivalence which identifies
process that are the same up to a change of variable. Formally, we
recognize the distinction between free and bound names. The free names
of a process, $\freenames{P}$, may be calculated recursively as
follows:

\begin{mathpar}
\freenames{\pzero} := \emptyset
  \and \\
  \freenames{x?(y).P} := \{ x \} \cup (\freenames{P} \setminus \{ y \})
  \and 
  \freenames{x!\langle P \rangle} := \{ x \} \cup \{ P \} 
  \and \\
  \freenames{P|Q} := \freenames{P} \cup \freenames{Q}
  \and \\
  \freenames{@{x}} := \{ x \}
\end{mathpar}

$\pi$
$\quotep{\pi}$

$\freenames{-} : \pi \to \mathcal{P}(\quotep{\pi})$

\begin{eqnarray*}
  \freenames{\pzero} & := & \emptyset \\
  \freenames{x?(y).P} & := & \{ x \} \cup (\freenames{P} \setminus \{ y \}) \\
  \freenames{x!\langle P \rangle} & := & \{ x \} \cup \{ P \} \\
  \freenames{P|Q} & := & \freenames{P} \cup \freenames{Q} \\
  \freenames{\dropn{x}} & := & \{ x \}
\end{eqnarray*}

The bound names of a process, $\boundnames{P}$, are those names occurring in $P$
that are not free. For example, in $x?(y).0$, the name $x$ is free, while $y$ is bound.

\begin{mathpar}
  \inferrule* [lab=monoidal-laws] {} { P|Q \equiv Q|P \and P|0 \equiv P \and P|(Q|R) \equiv (P|Q)|R }
\end{mathpar}

\begin{mathpar}
  \inferrule* [lab=alpha-equivalence] {} { (x)P \equiv (y)P\{y/x\} \and y \not\in \freenames{P} }
\end{mathpar}

\begin{definition}
Then two processes, $P,Q$, are alpha-equivalent if $P = Q\{\vec{y}/\vec{x}\}$ for
some $\vec{x} \in \boundnames{Q},\vec{y} \in \boundnames{P}$, where $Q\{\vec{y}/\vec{x}\}$
denotes the capture-avoiding substitution of $\vec{y}$ for $\vec{x}$ in $Q$.
\end{definition}

\begin{definition}
  The {\em structural congruence} \cite{SangiorgiWalker} , $\equiv$,
  between processes is the least congruence containing
  alpha-equivalence, satisfying the abelian monoid laws
  (associativity, commutativity and $\pzero$ as identity) for parallel
  composition $|$ and for summation $+$.
\end{definition}

\subsection{Name equivalence}

We take name equivalence, written $\nameeq$, to be the smallest
equivalence relation generated by the following rules.

\begin{mathpar}
\inferrule*[lab=Quote-drop]
{ }
{ \quotep{@{x}} \nameeq x }

\inferrule*[lab=Struct-equiv]
{ P \scong Q }
{ \quotep{P} \nameeq \quotep{Q} }
\end{mathpar}

The astute reader will have noticed that the mutual recursion of names
and processes imposes a mutual recursion on alpha-equivalence and
structural equivalence via name-equivalence. Fortunately, all of this
works out pleasantly and we may calculate in the natural way, free of
concern. The reader interested in the details is referred to the
appendix \ref{appendix:rho_details}.

\subsection{Substitution}

We use $\Proc$ for the set of processes, $\QProc$ for the set of
names, and $\id{\{}\vec{y} / \vec{x} \id{\}}$ to denote partial maps,
$s : \QProc \rightarrow \QProc$. A map, $s$ lifts, uniquely, to a map
on process terms, $\widehat{s} : \Proc \rightarrow \Proc$ by the
following equations.

\begin{mathpar}
  (0) \psubstp{Q}{P} := 0 \\
  (R \juxtap S) \psubstp{Q}{P}
  :=    
  (R)\psubstp{Q}{P} \juxtap (S) \psubstp{Q}{P} \\
  (x?(y).R) \psubstp{Q}{P}    
  :=    
  (x)\substp{Q}{P} (z)\concat( (R \psubstn{z}{y}) \psubstp{Q}{P} ) \\
  (\lift{x}{R}) \psubstp{Q}{P}  
  :=
  \lift{(x)\substp{Q}{P}}{ R \psubstp{Q}{P} } \\
%   (\dropn{x})  \psubstp{Q}{P}       
%   := 
%   \left\{ 
%     \begin{array}{ccc} 
%       \dropn{\quotep{Q}} & & x \nameeq \quotep{P} \\
%       \dropn{x} & & otherwise \\
%     \end{array}
%   \right. 
  (\dropn{x})  \psubstp{Q}{P}       
  := 
  \left\{ 
    \begin{array}{ccc} 
      Q & & x \nameeq \quotep{P} \\
      \dropn{x} & & otherwise \\
    \end{array}
  \right.
\end{mathpar}
 

where

\begin{eqnarray}
  (x)\id{\{} \lpquote Q \rpquote / \lpquote P \rpquote \id{\}}            = 
  \left\{ 
    \begin{array}{ccc}
      \lpquote Q \rpquote & & x \nameeq \lpquote P \rpquote \\
      x & & otherwise \\
    \end{array}
  \right. \nonumber
\end{eqnarray}

and $z$ is chosen distinct from $\quotep{P}$, $\quotep{Q}$, the free
names in $Q$, and all the names in $R$. Our $\alpha$-equivalence will
be built in the standard way from this substitution.

\begin{remark}\label{rem:no_self_referential_names}
  One consequence of these definitions is that $\forall P. \quotep{P}
  \not\in \freenames{P}$.
\end{remark}

\subsection{ Dynamic quote: an example }

Anticipating something of what's to come, consider applying the
substitution, $\widehat{\id{\{}u / z \id{\}}}$, to the following pair
of processes, $\lift{w}{y!(z)}$ and $w[ \lpquote y!(z) \rpquote ]$.

\begin{eqnarray}
	\lift{w}{y!(z)}\widehat{\id{\{}u / z \id{\}}}
		& = &
		\lift{w}{y!(u)} \nonumber\\
	w[ \lpquote y!(z) \rpquote ] \widehat{ \id{\{}u / z \id{\}} }
		& = &
		w[ \lpquote y!(z) \rpquote ] \nonumber
\end{eqnarray}

Because the body of the process between quotes is impervious to
substitution, we get radically different answers. In fact, by
examining the first process in an input context,
e.g. $x?(z).\lift{w}{y!(z)}$, we see that the process under the lift
operator may be shaped by prefixed inputs binding a name inside it. In
this sense, the lift operator will be seen as a way to dynamically
construct processes before reifying them as names.

Finally equipped with these standard features we can present the
dynamics of the calculus.

\subsubsection{Operational semantics} 

Finally, we introduce the computational dynamics. What marks these
algebras as distinct from other more traditionally studied algebraic
structures, e.g. vector spaces or polynomial rings, is the manner in
which dynamics is captured. In traditional structures, dynamics is typically
expressed through morphisms between such structures, as in linear maps
between vector spaces or morphisms between rings. In algebras
associated with the semantics of computation, the dynamics is
expressed as part of the algebraic structure itself, through a
reduction reduction relation typically denoted by $\red$. Below, we
give a recursive presentation of this relation for the calculus used
in the encoding.

$\red \subseteq \pi \times \pi$
$\red : \pi \to \mathcal{P}(\pi)$

\begin{mathpar}
  \inferrule* [lab=Comm] { \textsf{match}( x_{src}, x_{trgt} ) } { x_{trgt}?(y)P \; | \; x_{src}!\langle {Q} \rangle \red P\{\quotep{Q}/y}\} }
  \and \\
  \inferrule* [lab=Par] {{P} \red {P}'} {{{P} | {Q}} \red {{P}' | {Q}}}
  \and
  \inferrule* [lab=Equiv]{{{P} \scong {P}'} \andalso {{P}' \red {Q}'} \andalso {{Q}' \scong {Q}}}{{P} \red {Q}}
\end{mathpar}

\begin{eqnarray*}
  match_{\equiv} (\quotep{P},\quotep{Q}) & := & P \equiv Q \\
  match_{\dagger}(\quotep{P},\quotep{Q}) & := & \forall R. P|Q \red^{*} R => R \red^{*} 0 \\
  match_{K}(\quotep{P},\quotep{Q}) & := & K \mbox{ for some context } K
\end{eqnarray*}

$u?(x)P | u!\langle Q \rangle \red P\{\quotep{Q}/x\}$

%We write $\wred$ for $\red^*$, and $P\red$ if $\exists Q $ such that $ P \red Q$.
We write $P\red$ if $\exists Q $ such that $ P \red Q$ and $P\not\red$, otherwise.

\section{Replication}

As mentioned before, it is known that replication (and hence
recursion) can be implemented in a higher-order process algebra
\cite{SangiorgiWalker}. As our first example of calculation with the
machinery thus far presented we give the construction explicitly in
the {\rhoc}.

\begin{eqnarray}
	D_{x} & := & \prefix{x}{y}{(\binpar{\outputp{x}{y}}{@{y}})} \nonumber\\
	\bangp_{x}{P} & := & \binpar{{x}!\langle{\binpar{D_{x}}{P}}\rangle}{D_{x}} \nonumber
\end{eqnarray}

\begin{eqnarray}
	\bangp_{x}{P} & & \nonumber\\
	=
	& {x}!\langle{(\prefix{x}{y}{(\outputp{x}{y} | @{y})) | P}}\rangle 
	      | \prefix{x}{y}{(\outputp{x}{y} | @{y})} & \nonumber\\
	\red
	& (\outputp{x}{y} | @{y})\substn{\quotep{(\prefix{x}{y}{(@{y} | \outputp{x}{y})) | P}}}{y} & \nonumber\\
	=
	& \outputp{x}{\quotep{(\prefix{x}{y}{(\outputp{x}{y} | @{y})) | P}}}
	  | {(\prefix{x}{y}{(\outputp{x}{y} | @{y})) | P}} & \nonumber\\
	\red
	& \ldots & \nonumber\\
	\red^*
	& P | P | \ldots & \nonumber
\end{eqnarray}

Of course, this encoding, as an implementation, runs away, unfolding
$\bangp{P}$ eagerly. A lazier and more implementable replication
operator, restricted to input-guarded processes, may be obtained as follows.

\begin{eqnarray}
\bangp{\prefix{u}{v}{P}} 
	:= 
	\binpar{\lift{x}{\prefix{u}{v}{(\binpar{D(x)}{P})}}}{D(x)} \nonumber
\end{eqnarray}

\begin{remark}
  Note that the lazier definition still does not deal with summation
  or mixed summation (i.e. sums over input and output). The reader is
  invited to construct definitions of replication that deal with these
  features. 

  Further, the definitions are parameterized in a name, $x$. Can you,
  gentle reader, make a definition that eliminates this parameter and
  guarantees no accidental interaction between the replication
  machinery and the process being replicated -- i.e. no accidental
  sharing of names used by the process to get its work done and the
  name(s) used by the replication to effect copying. This latter
  revision of the definition of replication is crucial to obtaining
  the expected identity $!!P \sim !P$.
\end{remark}

\begin{remark}\label{rem:paradoxical_combinator}
  The reader familiar with the lambda calculus will have noticed the
  similarity between $D$ and the paradoxical combinator.

  [Ed. note: the existence of this seems to suggest we have to be more
  restrictive on the set of processes and names we admit if we are to
  support no-cloning.]
\end{remark}

\subsubsection{Bisimulation}

The computational dynamics gives rise to another kind of equivalence,
the equivalence of computational behavior. As previously mentioned
this is typically captured \emph{via} some form of bisimulation.

% The notion we use in this paper is weak barbed bisimulation
% \cite{milner91polyadicpi}.

The notion we use in this paper is derived from weak barbed
bisimulation \cite{milner91polyadicpi}. 

\begin{definition}
An \emph{observation relation}, $\downarrow_{\mathcal N}$, over a set
of names, $\mathcal N$, is the smallest relation satisfying the rules
below.

\infrule[Out-barb]{y \in {\mathcal N}, \; x \nameeq y}
		  {\outputp{x}{v} \downarrow_{\mathcal N} x}
\infrule[Par-barb]{\mbox{$P\downarrow_{\mathcal N} x$ or $Q\downarrow_{\mathcal N} x$}}
		  {\binpar{P}{Q} \downarrow_{\mathcal N} x}

We write $P \Downarrow_{\mathcal N} x$ if there is $Q$ such that 
$P \wred Q$ and $Q \downarrow_{\mathcal N} x$.
\end{definition}

\begin{definition}
%\label{def.bbisim}
An  ${\mathcal N}$-\emph{barbed bisimulation} over a set of names, ${\mathcal N}$, is a symmetric binary relation 
${\mathcal S}_{\mathcal N}$ between agents such that $P\rel{S}_{\mathcal N}Q$ implies:
\begin{enumerate}
\item If $P \red P'$ then $Q \wred Q'$ and $P'\rel{S}_{\mathcal N} Q'$.
\item If $P\downarrow_{\mathcal N} x$, then $Q\Downarrow_{\mathcal N} x$.
\end{enumerate}
$P$ is ${\mathcal N}$-barbed bisimilar to $Q$, written
$P \wbbisim_{\mathcal N} Q$, if $P \rel{S}_{\mathcal N} Q$ for some ${\mathcal N}$-barbed bisimulation ${\mathcal S}_{\mathcal N}$.
\end{definition}

$\mathcal{R} \subseteq \pi \times \pi$

$P \mathcal{R} Q => \forall P'. P \red P' \Rightarrow \exists Q'. Q \red Q', P' \mathcal{R} Q'$

$P \vdash x \Rightarrow Q \vdash x$

\begin{mathpar}
  \inferrule*[lab=Out-barb]{x \nameeq y}{{y}!\langle{Q}\rangle \vdash x}
  \and
  \inferrule*[lab=Par-barb]{\mbox{$P\vdash x$ or $Q\vdash x$}}{\binpar{P}{Q} \vdash x}
\end{mathpar}

\subsubsection{Contexts}

One of the principle advantages of computational calculi like the
$\pi$-calculus is a well-defined notion of context,
contextual-equivalence and a correlation between
contextual-equivalence and notions of bisimulation. The notion of
context allows the decomposition of a process into (sub-)process and
its syntactic environment, its context. Thus, a context may be
thought of as a process with a ``hole'' (written $\Box$) in it. The
application of a context $M$ to a process $P$, written $M[P]$, is
tantamount to filling the hole in $M$ with $P$. In this paper we do
not need the full weight of this theory, but do make use of the notion
of context in the proof the main theorem. 

\begin{mathpar}
  \inferrule* [lab=summation] {} {{M_{M},M_{N}} \bc \Box \;|\; x.M_{A} \;|\; M_{M}+M_{N}}
  \and
  \inferrule* [lab=agent] {} {{M_{A}} \bc (\vec{x})M_{P} \;| \; \clift{P_0,\ldots,M_{P},\ldots,P_N}}
  \and \\
  \inferrule* [lab=process] {} {{M_{P}} \bc M_{N} \;| \;P|M_{P} }
\end{mathpar} 

\begin{mathpar}
  \inferrule* [lab=sychronization] {} {M_{N} \bc \Box \;|\; x?M_{F} \;|\; x!M_{C}}
  \and
  \inferrule* [lab=abstraction] {} {{M_{F}} \bc (x)M_{P} }
  \and
  \inferrule* [lab=concretion] {} {{M_{C}} \bc \langle M_{P} \rangle }
  \and \\
  \inferrule* [lab=process] {} {{M_{P}} \bc M_{N} \;| \;P|M_{P} }
\end{mathpar}

\begin{definition}[contextual application] Given a context $M$, and
  process $P$, we define the \emph{contextual application}, $M[P] :=
  M\{P/\Box\}$. That is, the contextual application of M to P is the
  substitution of $P$ for $\Box$ in $M$.
\end{definition}

$\meaningof{-} : L \to \mathcal{P}(\pi)$

\begin{mathpar}
  \inferrule* [lab=collection] {} {\meaningof{true} = \pi, \and \meaningof{~E} = \pi \setminus \meaningof{E}, \and \meaningof{E_{1} \& E_{2}} = \meaningof{E_{1}} \cap \meaningof{E_{2}}}
\end{mathpar}

\begin{mathpar}
  \inferrule* [lab=structure] {} {\meaningof{0} = \{ P \in \pi | P \equiv 0 \}, \and \\ \meaningof{E_1 | E_2} = \{ P \in \pi | P \equiv P_{1} | P_{2}, P_{1} \in \meaningof{E_{1}}, P_{2} \in \meaningof{E_2}\} }
\end{mathpar}

\begin{mathpar}
 \inferrule* [lab=behavior] {} {\meaningof{\langle a?b \rangle E} = \{ P \in \pi | P \equiv Q | u?(y)P', \\ \and \\\\ \and \\ \;\;\; u \in \meaningof{a}, \forall z.P'\{z/y\} \in \meaningof{E\{z/b\}}\}, \and \\ \meaningof{a!E} = \{ P \in \pi | P \equiv Q | x!\langle P' \rangle, x \in \meaningof{a} P' \in \meaningof{E}\} }
\end{mathpar}

\begin{mathpar}
 \inferrule* [lab=nominal] {} {\meaningof{\quotep{E}} = \{ \quotep{P} \in \quotep{\pi} | P \in \meaningof{E} \}, \and \meaningof{\quotep{P}} = \{ \quotep{Q} \in \quotep{\pi} | P \equiv Q \} \and \\ \meaningof{@\quotep{E}} = \{ P \in \pi | P \equiv @x, x \in \meaningof{E} \}}
\end{mathpar}

\begin{eqnarray*}
  \\
  \meaningof{-} : TS \to ST
\end{eqnarray*}

\begin{eqnarray*}
  \\
  L : TS \to ST
\end{eqnarray*}

\begin{eqnarray*}
  \\
  P \models E \iff P \in \meaningof{E}
\end{eqnarray*}

\begin{eqnarray*}
  P \approx_{L} Q \iff \forall E \in L. P \models E \iff Q \models E
\end{eqnarray*}

\begin{eqnarray*}
  P \approx_{K} Q
\end{eqnarray*}

\begin{eqnarray*}
  P \approx Q
\end{eqnarray*}

$\approx_{K} = \approx = \approx_{L}$

\subsubsection{Contextual duality}

Note that contexts extend the quotation operation to a family of
operations from processes to names. Given a context, $M$, we can
define a \emph{nominal context}, $\quotep{M}$ by $\quotep{M}[P] :=
\quotep{M[P]}$. To foreshadow what is to come we observe that these
operations enjoy a duality with processes very much like the duality
between vectors and maps from vectors to scalars.

Further, because the calculus is essentially higher-order, we have a
correspondence between contexts and processes. More specifically,
given a name $x$ and a context $M$ we can construct $M^{*}_{x}$ such
that 

\begin{mathpar}
  M^{*}_{x} | \lift{x}{P} \red M[P]
\end{mathpar}

namely,

\begin{mathpar}
  M^{*}_{x} := x?(u).M[\dropn{u}]
\end{mathpar}

The dependence of $M^{*}_{x}$ on a name makes it an abstraction, 

\begin{mathpar}
  M^{*} := (x)x?(u).M[\dropn{u}]
\end{mathpar}

\subsection{Additional notation}

It will sometimes be convenient to denote the process a name
quotes. We already have the notation $x = \quotep{P}$, but it will be
convenient to introduce an alternate notation, $\procn{x}$, when we
want to emphasize the connection to the use of the name. Note that, by
virtue of name equivalence, $\quotep{\procn{x}} \nameeq x$; so, the
notation is consistent with previous definitions.

Further, because names have structure it is possible to effect
substitutions on the basis of that structure. This means we need to
upgrade our notation for substitutions, which we accomplish by
adapting comprehension notation. Thus,

\begin{mathpar}
  P\{ y / x : x \in S \}
\end{mathpar}

is interpreted to mean the process derived from P by replacing (in a
capture-avoiding manner) each occurrence of $x$ in $S$ by $y$. For example,

\begin{mathpar}
  P\{ \quotep{\procn{x}|\procn{x}} / x : x \in \freenames{P} \}
\end{mathpar}

will replace each (occurrence) of a free name $x$ in $P$ by
$\quotep{\procn{x}|\procn{x}}$.

Also, we will avail ourselves of the notation $x^{L}$ and $x^{R}$ to
denote injections of a name into disjoint copies of the name
space. There are numerous ways to accomplish this. One example can be
found in \cite{MeredithR05}. This notation overloads to vectors of
names: $\vec{x}^{\pi} := (x_{i}^{\pi} \; : \; 0 \leq i < |\vec{x}| )$ where $\pi \in \{L,R\}$.

We also use $P^{\Box} := P|\Box$.

In \cite{MeredithR05} an interpretation of the new operator is
given. It turns out that there are several possible interpretations
all enjoying the requisite algebraic properties of the operator (see
\cite{milner91polyadicpi}). We will therefore make liberal use of
$(\nu\; \vec{x})P$.

% subsection the_syntax_and_semantics_of_the_notation_system (end)   

\input{qm2pi.qmops} 

\input{qm2pi.sterngerlach} 

\input{qm2pi.metric} 

% section concurrent_process_calculi (end)

%\input{qm2pi.proofsketch}

% section proof sketch (end)

%\input{qm2pi.slviaknots} 

% section spatial logic via knots (end)

\input{qm2pi.conclusion}

% section conclusion (end)

%\input{qm2pi.dtcodes} 

% section wiring algorithm (end)

\input{qm2pi.ack} 

% section acknowledgments (end)

\newpage


\bibliographystyle{plain}   
\bibliography{../../biblios/main.bib}

\input{qm2pi.rhodetails}

\end{document}



\end{document}

 

% section wiring algorithm (end)

\documentclass[12pt]{llncs}
%\documentclass{jktr}

\usepackage[pdftex]{hyperref}                   
\usepackage {listings}
\usepackage {mathpartir}
\usepackage{bcprules}
%\usepackage{listings}
                       
\usepackage{graphicx} 
%\usepackage[margins=2.5cm,nohead,nofoot]{geometry}
%\usepackage{geometry}
\usepackage{amsfonts}
\usepackage{amstext}
\usepackage{latexsym}
\usepackage{amssymb}
\usepackage{color}


%\include{myPreamble}
\documentclass[12pt]{llncs}
%\documentclass{jktr}

\usepackage[pdftex]{hyperref}                   
\usepackage {listings}
\usepackage {mathpartir}
\usepackage{bcprules}
%\usepackage{listings}
                       
\usepackage{graphicx} 
%\usepackage[margins=2.5cm,nohead,nofoot]{geometry}
%\usepackage{geometry}
\usepackage{amsfonts}
\usepackage{amstext}
\usepackage{latexsym}
\usepackage{amssymb}
\usepackage{color}


%\include{myPreamble}
\include{qm2pi.local} 

%\ifpdf
%\usepackage[pdftex]{graphicx}
%\else
%\usepackage{graphicx}
%\fi

 % \ifpdf
%  \usepackage{pdfsync}
%  \if


%\title{Brief Article}
%\author{David F. Snyder}
%\author{L.G. Meredith}

%\address{Dept. of Math., Texas State University--San Marcos, San Marcos, TX 78666}
       
\pagestyle{empty}


\begin{document}

\lstset{language=[Objective]Caml,frame=shadowbox}

\input{qm2pi.front}

% section front matter (end)

\input{qm2pi.intro} 
 
% section introduction (end)

% \input{qm2pi.knotations} 

% section notation (end)

\input{qm2pi.process.calculi} 

% section concurrent_process_calculi_and_spatial_logics_ (end)
    
%\input{qm2pi.knots2pi} 

%\input{qm2pi.trefoil} 

%\input{qm2pi.mainthm} 

% subsection basic_interpretation (end)

%\input{qm2pi.rho.presentation} 
\subsection{The syntax and semantics of the notation system}\label{sub:the_syntax_and_semantics_of_the_notation_system} % (fold)

We now summarize a technical presentation of the calculus that
embodies our theory of dynamics. The typical presentation of such a
calculus follows the style of giving generators and relations on
them. The grammar, below, describing term constructors, freely
generates the set of processes, $\Proc$. This set is then quotiented
by a relation known as structural congruence and it is over this set
that the notion of dynamics is expressed. This presentation is
essentially that of \cite{MeredithR05} with the addition of
polyadicity and summation. For readability we have relegated some of
the technical subtleties to an appendix.

\subsubsection{Process grammar}\label{subsub:process_grammar}

\begin{mathpar}
  \inferrule* [lab=synchronization] {} {{M} \bc \pzero \;|\; x?F \;|\; x!C }
  \and
  \inferrule* [lab=abstraction] {} {{F} \bc (x)P}
  \and
  \inferrule* [lab=concretion] {} {{C} \bc \langle Q \rangle}
  \and
  \inferrule* [lab=process] {} {{P,Q} \bc M \;| \;P|Q \;|\; @{x}}
  \and
  \inferrule* [lab=name] {} {{x} \bc \quotep{P}}
\end{mathpar} 

Note that $\vec{x}$ (resp. $\vec{P}$) denotes a vector of names
(resp. processes) of length $|\vec{x}|$ (resp. $|\vec{P}|$). We adopt
the following useful abbreviations.

\begin{mathpar}
   x?(\vec{y}).P := x.(\vec{y})P \and  x\clift{\vec{P}} := x.\clift{\vec{P}}
   \and x!(y) := \lift{x}{\dropn{y}}
   \and \Pi_{i=0}^{n-1}P_i := P_0 | \ldots | P_{n-1}
\end{mathpar}

\subsubsection{Structural congruence}

\paragraph{Free and bound names and alpha-equivalence.} At the
core of structural equivalence is alpha-equivalence which identifies
process that are the same up to a change of variable. Formally, we
recognize the distinction between free and bound names. The free names
of a process, $\freenames{P}$, may be calculated recursively as
follows:

\begin{mathpar}
\freenames{\pzero} := \emptyset
  \and \\
  \freenames{x?(y).P} := \{ x \} \cup (\freenames{P} \setminus \{ y \})
  \and 
  \freenames{x!\langle P \rangle} := \{ x \} \cup \{ P \} 
  \and \\
  \freenames{P|Q} := \freenames{P} \cup \freenames{Q}
  \and \\
  \freenames{@{x}} := \{ x \}
\end{mathpar}

$\pi$
$\quotep{\pi}$

$\freenames{-} : \pi \to \mathcal{P}(\quotep{\pi})$

\begin{eqnarray*}
  \freenames{\pzero} & := & \emptyset \\
  \freenames{x?(y).P} & := & \{ x \} \cup (\freenames{P} \setminus \{ y \}) \\
  \freenames{x!\langle P \rangle} & := & \{ x \} \cup \{ P \} \\
  \freenames{P|Q} & := & \freenames{P} \cup \freenames{Q} \\
  \freenames{\dropn{x}} & := & \{ x \}
\end{eqnarray*}

The bound names of a process, $\boundnames{P}$, are those names occurring in $P$
that are not free. For example, in $x?(y).0$, the name $x$ is free, while $y$ is bound.

\begin{mathpar}
  \inferrule* [lab=monoidal-laws] {} { P|Q \equiv Q|P \and P|0 \equiv P \and P|(Q|R) \equiv (P|Q)|R }
\end{mathpar}

\begin{mathpar}
  \inferrule* [lab=alpha-equivalence] {} { (x)P \equiv (y)P\{y/x\} \and y \not\in \freenames{P} }
\end{mathpar}

\begin{definition}
Then two processes, $P,Q$, are alpha-equivalent if $P = Q\{\vec{y}/\vec{x}\}$ for
some $\vec{x} \in \boundnames{Q},\vec{y} \in \boundnames{P}$, where $Q\{\vec{y}/\vec{x}\}$
denotes the capture-avoiding substitution of $\vec{y}$ for $\vec{x}$ in $Q$.
\end{definition}

\begin{definition}
  The {\em structural congruence} \cite{SangiorgiWalker} , $\equiv$,
  between processes is the least congruence containing
  alpha-equivalence, satisfying the abelian monoid laws
  (associativity, commutativity and $\pzero$ as identity) for parallel
  composition $|$ and for summation $+$.
\end{definition}

\subsection{Name equivalence}

We take name equivalence, written $\nameeq$, to be the smallest
equivalence relation generated by the following rules.

\begin{mathpar}
\inferrule*[lab=Quote-drop]
{ }
{ \quotep{@{x}} \nameeq x }

\inferrule*[lab=Struct-equiv]
{ P \scong Q }
{ \quotep{P} \nameeq \quotep{Q} }
\end{mathpar}

The astute reader will have noticed that the mutual recursion of names
and processes imposes a mutual recursion on alpha-equivalence and
structural equivalence via name-equivalence. Fortunately, all of this
works out pleasantly and we may calculate in the natural way, free of
concern. The reader interested in the details is referred to the
appendix \ref{appendix:rho_details}.

\subsection{Substitution}

We use $\Proc$ for the set of processes, $\QProc$ for the set of
names, and $\id{\{}\vec{y} / \vec{x} \id{\}}$ to denote partial maps,
$s : \QProc \rightarrow \QProc$. A map, $s$ lifts, uniquely, to a map
on process terms, $\widehat{s} : \Proc \rightarrow \Proc$ by the
following equations.

\begin{mathpar}
  (0) \psubstp{Q}{P} := 0 \\
  (R \juxtap S) \psubstp{Q}{P}
  :=    
  (R)\psubstp{Q}{P} \juxtap (S) \psubstp{Q}{P} \\
  (x?(y).R) \psubstp{Q}{P}    
  :=    
  (x)\substp{Q}{P} (z)\concat( (R \psubstn{z}{y}) \psubstp{Q}{P} ) \\
  (\lift{x}{R}) \psubstp{Q}{P}  
  :=
  \lift{(x)\substp{Q}{P}}{ R \psubstp{Q}{P} } \\
%   (\dropn{x})  \psubstp{Q}{P}       
%   := 
%   \left\{ 
%     \begin{array}{ccc} 
%       \dropn{\quotep{Q}} & & x \nameeq \quotep{P} \\
%       \dropn{x} & & otherwise \\
%     \end{array}
%   \right. 
  (\dropn{x})  \psubstp{Q}{P}       
  := 
  \left\{ 
    \begin{array}{ccc} 
      Q & & x \nameeq \quotep{P} \\
      \dropn{x} & & otherwise \\
    \end{array}
  \right.
\end{mathpar}
 

where

\begin{eqnarray}
  (x)\id{\{} \lpquote Q \rpquote / \lpquote P \rpquote \id{\}}            = 
  \left\{ 
    \begin{array}{ccc}
      \lpquote Q \rpquote & & x \nameeq \lpquote P \rpquote \\
      x & & otherwise \\
    \end{array}
  \right. \nonumber
\end{eqnarray}

and $z$ is chosen distinct from $\quotep{P}$, $\quotep{Q}$, the free
names in $Q$, and all the names in $R$. Our $\alpha$-equivalence will
be built in the standard way from this substitution.

\begin{remark}\label{rem:no_self_referential_names}
  One consequence of these definitions is that $\forall P. \quotep{P}
  \not\in \freenames{P}$.
\end{remark}

\subsection{ Dynamic quote: an example }

Anticipating something of what's to come, consider applying the
substitution, $\widehat{\id{\{}u / z \id{\}}}$, to the following pair
of processes, $\lift{w}{y!(z)}$ and $w[ \lpquote y!(z) \rpquote ]$.

\begin{eqnarray}
	\lift{w}{y!(z)}\widehat{\id{\{}u / z \id{\}}}
		& = &
		\lift{w}{y!(u)} \nonumber\\
	w[ \lpquote y!(z) \rpquote ] \widehat{ \id{\{}u / z \id{\}} }
		& = &
		w[ \lpquote y!(z) \rpquote ] \nonumber
\end{eqnarray}

Because the body of the process between quotes is impervious to
substitution, we get radically different answers. In fact, by
examining the first process in an input context,
e.g. $x?(z).\lift{w}{y!(z)}$, we see that the process under the lift
operator may be shaped by prefixed inputs binding a name inside it. In
this sense, the lift operator will be seen as a way to dynamically
construct processes before reifying them as names.

Finally equipped with these standard features we can present the
dynamics of the calculus.

\subsubsection{Operational semantics} 

Finally, we introduce the computational dynamics. What marks these
algebras as distinct from other more traditionally studied algebraic
structures, e.g. vector spaces or polynomial rings, is the manner in
which dynamics is captured. In traditional structures, dynamics is typically
expressed through morphisms between such structures, as in linear maps
between vector spaces or morphisms between rings. In algebras
associated with the semantics of computation, the dynamics is
expressed as part of the algebraic structure itself, through a
reduction reduction relation typically denoted by $\red$. Below, we
give a recursive presentation of this relation for the calculus used
in the encoding.

$\red \subseteq \pi \times \pi$
$\red : \pi \to \mathcal{P}(\pi)$

\begin{mathpar}
  \inferrule* [lab=Comm] { \textsf{match}( x_{src}, x_{trgt} ) } { x_{trgt}?(y)P \; | \; x_{src}!\langle {Q} \rangle \red P\{\quotep{Q}/y}\} }
  \and \\
  \inferrule* [lab=Par] {{P} \red {P}'} {{{P} | {Q}} \red {{P}' | {Q}}}
  \and
  \inferrule* [lab=Equiv]{{{P} \scong {P}'} \andalso {{P}' \red {Q}'} \andalso {{Q}' \scong {Q}}}{{P} \red {Q}}
\end{mathpar}

\begin{eqnarray*}
  match_{\equiv} (\quotep{P},\quotep{Q}) & := & P \equiv Q \\
  match_{\dagger}(\quotep{P},\quotep{Q}) & := & \forall R. P|Q \red^{*} R => R \red^{*} 0 \\
  match_{K}(\quotep{P},\quotep{Q}) & := & K \mbox{ for some context } K
\end{eqnarray*}

$u?(x)P | u!\langle Q \rangle \red P\{\quotep{Q}/x\}$

%We write $\wred$ for $\red^*$, and $P\red$ if $\exists Q $ such that $ P \red Q$.
We write $P\red$ if $\exists Q $ such that $ P \red Q$ and $P\not\red$, otherwise.

\section{Replication}

As mentioned before, it is known that replication (and hence
recursion) can be implemented in a higher-order process algebra
\cite{SangiorgiWalker}. As our first example of calculation with the
machinery thus far presented we give the construction explicitly in
the {\rhoc}.

\begin{eqnarray}
	D_{x} & := & \prefix{x}{y}{(\binpar{\outputp{x}{y}}{@{y}})} \nonumber\\
	\bangp_{x}{P} & := & \binpar{{x}!\langle{\binpar{D_{x}}{P}}\rangle}{D_{x}} \nonumber
\end{eqnarray}

\begin{eqnarray}
	\bangp_{x}{P} & & \nonumber\\
	=
	& {x}!\langle{(\prefix{x}{y}{(\outputp{x}{y} | @{y})) | P}}\rangle 
	      | \prefix{x}{y}{(\outputp{x}{y} | @{y})} & \nonumber\\
	\red
	& (\outputp{x}{y} | @{y})\substn{\quotep{(\prefix{x}{y}{(@{y} | \outputp{x}{y})) | P}}}{y} & \nonumber\\
	=
	& \outputp{x}{\quotep{(\prefix{x}{y}{(\outputp{x}{y} | @{y})) | P}}}
	  | {(\prefix{x}{y}{(\outputp{x}{y} | @{y})) | P}} & \nonumber\\
	\red
	& \ldots & \nonumber\\
	\red^*
	& P | P | \ldots & \nonumber
\end{eqnarray}

Of course, this encoding, as an implementation, runs away, unfolding
$\bangp{P}$ eagerly. A lazier and more implementable replication
operator, restricted to input-guarded processes, may be obtained as follows.

\begin{eqnarray}
\bangp{\prefix{u}{v}{P}} 
	:= 
	\binpar{\lift{x}{\prefix{u}{v}{(\binpar{D(x)}{P})}}}{D(x)} \nonumber
\end{eqnarray}

\begin{remark}
  Note that the lazier definition still does not deal with summation
  or mixed summation (i.e. sums over input and output). The reader is
  invited to construct definitions of replication that deal with these
  features. 

  Further, the definitions are parameterized in a name, $x$. Can you,
  gentle reader, make a definition that eliminates this parameter and
  guarantees no accidental interaction between the replication
  machinery and the process being replicated -- i.e. no accidental
  sharing of names used by the process to get its work done and the
  name(s) used by the replication to effect copying. This latter
  revision of the definition of replication is crucial to obtaining
  the expected identity $!!P \sim !P$.
\end{remark}

\begin{remark}\label{rem:paradoxical_combinator}
  The reader familiar with the lambda calculus will have noticed the
  similarity between $D$ and the paradoxical combinator.

  [Ed. note: the existence of this seems to suggest we have to be more
  restrictive on the set of processes and names we admit if we are to
  support no-cloning.]
\end{remark}

\subsubsection{Bisimulation}

The computational dynamics gives rise to another kind of equivalence,
the equivalence of computational behavior. As previously mentioned
this is typically captured \emph{via} some form of bisimulation.

% The notion we use in this paper is weak barbed bisimulation
% \cite{milner91polyadicpi}.

The notion we use in this paper is derived from weak barbed
bisimulation \cite{milner91polyadicpi}. 

\begin{definition}
An \emph{observation relation}, $\downarrow_{\mathcal N}$, over a set
of names, $\mathcal N$, is the smallest relation satisfying the rules
below.

\infrule[Out-barb]{y \in {\mathcal N}, \; x \nameeq y}
		  {\outputp{x}{v} \downarrow_{\mathcal N} x}
\infrule[Par-barb]{\mbox{$P\downarrow_{\mathcal N} x$ or $Q\downarrow_{\mathcal N} x$}}
		  {\binpar{P}{Q} \downarrow_{\mathcal N} x}

We write $P \Downarrow_{\mathcal N} x$ if there is $Q$ such that 
$P \wred Q$ and $Q \downarrow_{\mathcal N} x$.
\end{definition}

\begin{definition}
%\label{def.bbisim}
An  ${\mathcal N}$-\emph{barbed bisimulation} over a set of names, ${\mathcal N}$, is a symmetric binary relation 
${\mathcal S}_{\mathcal N}$ between agents such that $P\rel{S}_{\mathcal N}Q$ implies:
\begin{enumerate}
\item If $P \red P'$ then $Q \wred Q'$ and $P'\rel{S}_{\mathcal N} Q'$.
\item If $P\downarrow_{\mathcal N} x$, then $Q\Downarrow_{\mathcal N} x$.
\end{enumerate}
$P$ is ${\mathcal N}$-barbed bisimilar to $Q$, written
$P \wbbisim_{\mathcal N} Q$, if $P \rel{S}_{\mathcal N} Q$ for some ${\mathcal N}$-barbed bisimulation ${\mathcal S}_{\mathcal N}$.
\end{definition}

$\mathcal{R} \subseteq \pi \times \pi$

$P \mathcal{R} Q => \forall P'. P \red P' \Rightarrow \exists Q'. Q \red Q', P' \mathcal{R} Q'$

$P \vdash x \Rightarrow Q \vdash x$

\begin{mathpar}
  \inferrule*[lab=Out-barb]{x \nameeq y}{{y}!\langle{Q}\rangle \vdash x}
  \and
  \inferrule*[lab=Par-barb]{\mbox{$P\vdash x$ or $Q\vdash x$}}{\binpar{P}{Q} \vdash x}
\end{mathpar}

\subsubsection{Contexts}

One of the principle advantages of computational calculi like the
$\pi$-calculus is a well-defined notion of context,
contextual-equivalence and a correlation between
contextual-equivalence and notions of bisimulation. The notion of
context allows the decomposition of a process into (sub-)process and
its syntactic environment, its context. Thus, a context may be
thought of as a process with a ``hole'' (written $\Box$) in it. The
application of a context $M$ to a process $P$, written $M[P]$, is
tantamount to filling the hole in $M$ with $P$. In this paper we do
not need the full weight of this theory, but do make use of the notion
of context in the proof the main theorem. 

\begin{mathpar}
  \inferrule* [lab=summation] {} {{M_{M},M_{N}} \bc \Box \;|\; x.M_{A} \;|\; M_{M}+M_{N}}
  \and
  \inferrule* [lab=agent] {} {{M_{A}} \bc (\vec{x})M_{P} \;| \; \clift{P_0,\ldots,M_{P},\ldots,P_N}}
  \and \\
  \inferrule* [lab=process] {} {{M_{P}} \bc M_{N} \;| \;P|M_{P} }
\end{mathpar} 

\begin{mathpar}
  \inferrule* [lab=sychronization] {} {M_{N} \bc \Box \;|\; x?M_{F} \;|\; x!M_{C}}
  \and
  \inferrule* [lab=abstraction] {} {{M_{F}} \bc (x)M_{P} }
  \and
  \inferrule* [lab=concretion] {} {{M_{C}} \bc \langle M_{P} \rangle }
  \and \\
  \inferrule* [lab=process] {} {{M_{P}} \bc M_{N} \;| \;P|M_{P} }
\end{mathpar}

\begin{definition}[contextual application] Given a context $M$, and
  process $P$, we define the \emph{contextual application}, $M[P] :=
  M\{P/\Box\}$. That is, the contextual application of M to P is the
  substitution of $P$ for $\Box$ in $M$.
\end{definition}

$\meaningof{-} : L \to \mathcal{P}(\pi)$

\begin{mathpar}
  \inferrule* [lab=collection] {} {\meaningof{true} = \pi, \and \meaningof{~E} = \pi \setminus \meaningof{E}, \and \meaningof{E_{1} \& E_{2}} = \meaningof{E_{1}} \cap \meaningof{E_{2}}}
\end{mathpar}

\begin{mathpar}
  \inferrule* [lab=structure] {} {\meaningof{0} = \{ P \in \pi | P \equiv 0 \}, \and \\ \meaningof{E_1 | E_2} = \{ P \in \pi | P \equiv P_{1} | P_{2}, P_{1} \in \meaningof{E_{1}}, P_{2} \in \meaningof{E_2}\} }
\end{mathpar}

\begin{mathpar}
 \inferrule* [lab=behavior] {} {\meaningof{\langle a?b \rangle E} = \{ P \in \pi | P \equiv Q | u?(y)P', \\ \and \\\\ \and \\ \;\;\; u \in \meaningof{a}, \forall z.P'\{z/y\} \in \meaningof{E\{z/b\}}\}, \and \\ \meaningof{a!E} = \{ P \in \pi | P \equiv Q | x!\langle P' \rangle, x \in \meaningof{a} P' \in \meaningof{E}\} }
\end{mathpar}

\begin{mathpar}
 \inferrule* [lab=nominal] {} {\meaningof{\quotep{E}} = \{ \quotep{P} \in \quotep{\pi} | P \in \meaningof{E} \}, \and \meaningof{\quotep{P}} = \{ \quotep{Q} \in \quotep{\pi} | P \equiv Q \} \and \\ \meaningof{@\quotep{E}} = \{ P \in \pi | P \equiv @x, x \in \meaningof{E} \}}
\end{mathpar}

\begin{eqnarray*}
  \\
  \meaningof{-} : TS \to ST
\end{eqnarray*}

\begin{eqnarray*}
  \\
  L : TS \to ST
\end{eqnarray*}

\begin{eqnarray*}
  \\
  P \models E \iff P \in \meaningof{E}
\end{eqnarray*}

\begin{eqnarray*}
  P \approx_{L} Q \iff \forall E \in L. P \models E \iff Q \models E
\end{eqnarray*}

\begin{eqnarray*}
  P \approx_{K} Q
\end{eqnarray*}

\begin{eqnarray*}
  P \approx Q
\end{eqnarray*}

$\approx_{K} = \approx = \approx_{L}$

\subsubsection{Contextual duality}

Note that contexts extend the quotation operation to a family of
operations from processes to names. Given a context, $M$, we can
define a \emph{nominal context}, $\quotep{M}$ by $\quotep{M}[P] :=
\quotep{M[P]}$. To foreshadow what is to come we observe that these
operations enjoy a duality with processes very much like the duality
between vectors and maps from vectors to scalars.

Further, because the calculus is essentially higher-order, we have a
correspondence between contexts and processes. More specifically,
given a name $x$ and a context $M$ we can construct $M^{*}_{x}$ such
that 

\begin{mathpar}
  M^{*}_{x} | \lift{x}{P} \red M[P]
\end{mathpar}

namely,

\begin{mathpar}
  M^{*}_{x} := x?(u).M[\dropn{u}]
\end{mathpar}

The dependence of $M^{*}_{x}$ on a name makes it an abstraction, 

\begin{mathpar}
  M^{*} := (x)x?(u).M[\dropn{u}]
\end{mathpar}

\subsection{Additional notation}

It will sometimes be convenient to denote the process a name
quotes. We already have the notation $x = \quotep{P}$, but it will be
convenient to introduce an alternate notation, $\procn{x}$, when we
want to emphasize the connection to the use of the name. Note that, by
virtue of name equivalence, $\quotep{\procn{x}} \nameeq x$; so, the
notation is consistent with previous definitions.

Further, because names have structure it is possible to effect
substitutions on the basis of that structure. This means we need to
upgrade our notation for substitutions, which we accomplish by
adapting comprehension notation. Thus,

\begin{mathpar}
  P\{ y / x : x \in S \}
\end{mathpar}

is interpreted to mean the process derived from P by replacing (in a
capture-avoiding manner) each occurrence of $x$ in $S$ by $y$. For example,

\begin{mathpar}
  P\{ \quotep{\procn{x}|\procn{x}} / x : x \in \freenames{P} \}
\end{mathpar}

will replace each (occurrence) of a free name $x$ in $P$ by
$\quotep{\procn{x}|\procn{x}}$.

Also, we will avail ourselves of the notation $x^{L}$ and $x^{R}$ to
denote injections of a name into disjoint copies of the name
space. There are numerous ways to accomplish this. One example can be
found in \cite{MeredithR05}. This notation overloads to vectors of
names: $\vec{x}^{\pi} := (x_{i}^{\pi} \; : \; 0 \leq i < |\vec{x}| )$ where $\pi \in \{L,R\}$.

We also use $P^{\Box} := P|\Box$.

In \cite{MeredithR05} an interpretation of the new operator is
given. It turns out that there are several possible interpretations
all enjoying the requisite algebraic properties of the operator (see
\cite{milner91polyadicpi}). We will therefore make liberal use of
$(\nu\; \vec{x})P$.

% subsection the_syntax_and_semantics_of_the_notation_system (end)   

\input{qm2pi.qmops} 

\input{qm2pi.sterngerlach} 

\input{qm2pi.metric} 

% section concurrent_process_calculi (end)

%\input{qm2pi.proofsketch}

% section proof sketch (end)

%\input{qm2pi.slviaknots} 

% section spatial logic via knots (end)

\input{qm2pi.conclusion}

% section conclusion (end)

%\input{qm2pi.dtcodes} 

% section wiring algorithm (end)

\input{qm2pi.ack} 

% section acknowledgments (end)

\newpage


\bibliographystyle{plain}   
\bibliography{../../biblios/main.bib}

\input{qm2pi.rhodetails}

\end{document}

 

%\ifpdf
%\usepackage[pdftex]{graphicx}
%\else
%\usepackage{graphicx}
%\fi

 % \ifpdf
%  \usepackage{pdfsync}
%  \if


%\title{Brief Article}
%\author{David F. Snyder}
%\author{L.G. Meredith}

%\address{Dept. of Math., Texas State University--San Marcos, San Marcos, TX 78666}
       
\pagestyle{empty}


\begin{document}

\lstset{language=[Objective]Caml,frame=shadowbox}

\documentclass[12pt]{llncs}
%\documentclass{jktr}

\usepackage[pdftex]{hyperref}                   
\usepackage {listings}
\usepackage {mathpartir}
\usepackage{bcprules}
%\usepackage{listings}
                       
\usepackage{graphicx} 
%\usepackage[margins=2.5cm,nohead,nofoot]{geometry}
%\usepackage{geometry}
\usepackage{amsfonts}
\usepackage{amstext}
\usepackage{latexsym}
\usepackage{amssymb}
\usepackage{color}


%\include{myPreamble}
\include{qm2pi.local} 

%\ifpdf
%\usepackage[pdftex]{graphicx}
%\else
%\usepackage{graphicx}
%\fi

 % \ifpdf
%  \usepackage{pdfsync}
%  \if


%\title{Brief Article}
%\author{David F. Snyder}
%\author{L.G. Meredith}

%\address{Dept. of Math., Texas State University--San Marcos, San Marcos, TX 78666}
       
\pagestyle{empty}


\begin{document}

\lstset{language=[Objective]Caml,frame=shadowbox}

\input{qm2pi.front}

% section front matter (end)

\input{qm2pi.intro} 
 
% section introduction (end)

% \input{qm2pi.knotations} 

% section notation (end)

\input{qm2pi.process.calculi} 

% section concurrent_process_calculi_and_spatial_logics_ (end)
    
%\input{qm2pi.knots2pi} 

%\input{qm2pi.trefoil} 

%\input{qm2pi.mainthm} 

% subsection basic_interpretation (end)

%\input{qm2pi.rho.presentation} 
\subsection{The syntax and semantics of the notation system}\label{sub:the_syntax_and_semantics_of_the_notation_system} % (fold)

We now summarize a technical presentation of the calculus that
embodies our theory of dynamics. The typical presentation of such a
calculus follows the style of giving generators and relations on
them. The grammar, below, describing term constructors, freely
generates the set of processes, $\Proc$. This set is then quotiented
by a relation known as structural congruence and it is over this set
that the notion of dynamics is expressed. This presentation is
essentially that of \cite{MeredithR05} with the addition of
polyadicity and summation. For readability we have relegated some of
the technical subtleties to an appendix.

\subsubsection{Process grammar}\label{subsub:process_grammar}

\begin{mathpar}
  \inferrule* [lab=synchronization] {} {{M} \bc \pzero \;|\; x?F \;|\; x!C }
  \and
  \inferrule* [lab=abstraction] {} {{F} \bc (x)P}
  \and
  \inferrule* [lab=concretion] {} {{C} \bc \langle Q \rangle}
  \and
  \inferrule* [lab=process] {} {{P,Q} \bc M \;| \;P|Q \;|\; @{x}}
  \and
  \inferrule* [lab=name] {} {{x} \bc \quotep{P}}
\end{mathpar} 

Note that $\vec{x}$ (resp. $\vec{P}$) denotes a vector of names
(resp. processes) of length $|\vec{x}|$ (resp. $|\vec{P}|$). We adopt
the following useful abbreviations.

\begin{mathpar}
   x?(\vec{y}).P := x.(\vec{y})P \and  x\clift{\vec{P}} := x.\clift{\vec{P}}
   \and x!(y) := \lift{x}{\dropn{y}}
   \and \Pi_{i=0}^{n-1}P_i := P_0 | \ldots | P_{n-1}
\end{mathpar}

\subsubsection{Structural congruence}

\paragraph{Free and bound names and alpha-equivalence.} At the
core of structural equivalence is alpha-equivalence which identifies
process that are the same up to a change of variable. Formally, we
recognize the distinction between free and bound names. The free names
of a process, $\freenames{P}$, may be calculated recursively as
follows:

\begin{mathpar}
\freenames{\pzero} := \emptyset
  \and \\
  \freenames{x?(y).P} := \{ x \} \cup (\freenames{P} \setminus \{ y \})
  \and 
  \freenames{x!\langle P \rangle} := \{ x \} \cup \{ P \} 
  \and \\
  \freenames{P|Q} := \freenames{P} \cup \freenames{Q}
  \and \\
  \freenames{@{x}} := \{ x \}
\end{mathpar}

$\pi$
$\quotep{\pi}$

$\freenames{-} : \pi \to \mathcal{P}(\quotep{\pi})$

\begin{eqnarray*}
  \freenames{\pzero} & := & \emptyset \\
  \freenames{x?(y).P} & := & \{ x \} \cup (\freenames{P} \setminus \{ y \}) \\
  \freenames{x!\langle P \rangle} & := & \{ x \} \cup \{ P \} \\
  \freenames{P|Q} & := & \freenames{P} \cup \freenames{Q} \\
  \freenames{\dropn{x}} & := & \{ x \}
\end{eqnarray*}

The bound names of a process, $\boundnames{P}$, are those names occurring in $P$
that are not free. For example, in $x?(y).0$, the name $x$ is free, while $y$ is bound.

\begin{mathpar}
  \inferrule* [lab=monoidal-laws] {} { P|Q \equiv Q|P \and P|0 \equiv P \and P|(Q|R) \equiv (P|Q)|R }
\end{mathpar}

\begin{mathpar}
  \inferrule* [lab=alpha-equivalence] {} { (x)P \equiv (y)P\{y/x\} \and y \not\in \freenames{P} }
\end{mathpar}

\begin{definition}
Then two processes, $P,Q$, are alpha-equivalent if $P = Q\{\vec{y}/\vec{x}\}$ for
some $\vec{x} \in \boundnames{Q},\vec{y} \in \boundnames{P}$, where $Q\{\vec{y}/\vec{x}\}$
denotes the capture-avoiding substitution of $\vec{y}$ for $\vec{x}$ in $Q$.
\end{definition}

\begin{definition}
  The {\em structural congruence} \cite{SangiorgiWalker} , $\equiv$,
  between processes is the least congruence containing
  alpha-equivalence, satisfying the abelian monoid laws
  (associativity, commutativity and $\pzero$ as identity) for parallel
  composition $|$ and for summation $+$.
\end{definition}

\subsection{Name equivalence}

We take name equivalence, written $\nameeq$, to be the smallest
equivalence relation generated by the following rules.

\begin{mathpar}
\inferrule*[lab=Quote-drop]
{ }
{ \quotep{@{x}} \nameeq x }

\inferrule*[lab=Struct-equiv]
{ P \scong Q }
{ \quotep{P} \nameeq \quotep{Q} }
\end{mathpar}

The astute reader will have noticed that the mutual recursion of names
and processes imposes a mutual recursion on alpha-equivalence and
structural equivalence via name-equivalence. Fortunately, all of this
works out pleasantly and we may calculate in the natural way, free of
concern. The reader interested in the details is referred to the
appendix \ref{appendix:rho_details}.

\subsection{Substitution}

We use $\Proc$ for the set of processes, $\QProc$ for the set of
names, and $\id{\{}\vec{y} / \vec{x} \id{\}}$ to denote partial maps,
$s : \QProc \rightarrow \QProc$. A map, $s$ lifts, uniquely, to a map
on process terms, $\widehat{s} : \Proc \rightarrow \Proc$ by the
following equations.

\begin{mathpar}
  (0) \psubstp{Q}{P} := 0 \\
  (R \juxtap S) \psubstp{Q}{P}
  :=    
  (R)\psubstp{Q}{P} \juxtap (S) \psubstp{Q}{P} \\
  (x?(y).R) \psubstp{Q}{P}    
  :=    
  (x)\substp{Q}{P} (z)\concat( (R \psubstn{z}{y}) \psubstp{Q}{P} ) \\
  (\lift{x}{R}) \psubstp{Q}{P}  
  :=
  \lift{(x)\substp{Q}{P}}{ R \psubstp{Q}{P} } \\
%   (\dropn{x})  \psubstp{Q}{P}       
%   := 
%   \left\{ 
%     \begin{array}{ccc} 
%       \dropn{\quotep{Q}} & & x \nameeq \quotep{P} \\
%       \dropn{x} & & otherwise \\
%     \end{array}
%   \right. 
  (\dropn{x})  \psubstp{Q}{P}       
  := 
  \left\{ 
    \begin{array}{ccc} 
      Q & & x \nameeq \quotep{P} \\
      \dropn{x} & & otherwise \\
    \end{array}
  \right.
\end{mathpar}
 

where

\begin{eqnarray}
  (x)\id{\{} \lpquote Q \rpquote / \lpquote P \rpquote \id{\}}            = 
  \left\{ 
    \begin{array}{ccc}
      \lpquote Q \rpquote & & x \nameeq \lpquote P \rpquote \\
      x & & otherwise \\
    \end{array}
  \right. \nonumber
\end{eqnarray}

and $z$ is chosen distinct from $\quotep{P}$, $\quotep{Q}$, the free
names in $Q$, and all the names in $R$. Our $\alpha$-equivalence will
be built in the standard way from this substitution.

\begin{remark}\label{rem:no_self_referential_names}
  One consequence of these definitions is that $\forall P. \quotep{P}
  \not\in \freenames{P}$.
\end{remark}

\subsection{ Dynamic quote: an example }

Anticipating something of what's to come, consider applying the
substitution, $\widehat{\id{\{}u / z \id{\}}}$, to the following pair
of processes, $\lift{w}{y!(z)}$ and $w[ \lpquote y!(z) \rpquote ]$.

\begin{eqnarray}
	\lift{w}{y!(z)}\widehat{\id{\{}u / z \id{\}}}
		& = &
		\lift{w}{y!(u)} \nonumber\\
	w[ \lpquote y!(z) \rpquote ] \widehat{ \id{\{}u / z \id{\}} }
		& = &
		w[ \lpquote y!(z) \rpquote ] \nonumber
\end{eqnarray}

Because the body of the process between quotes is impervious to
substitution, we get radically different answers. In fact, by
examining the first process in an input context,
e.g. $x?(z).\lift{w}{y!(z)}$, we see that the process under the lift
operator may be shaped by prefixed inputs binding a name inside it. In
this sense, the lift operator will be seen as a way to dynamically
construct processes before reifying them as names.

Finally equipped with these standard features we can present the
dynamics of the calculus.

\subsubsection{Operational semantics} 

Finally, we introduce the computational dynamics. What marks these
algebras as distinct from other more traditionally studied algebraic
structures, e.g. vector spaces or polynomial rings, is the manner in
which dynamics is captured. In traditional structures, dynamics is typically
expressed through morphisms between such structures, as in linear maps
between vector spaces or morphisms between rings. In algebras
associated with the semantics of computation, the dynamics is
expressed as part of the algebraic structure itself, through a
reduction reduction relation typically denoted by $\red$. Below, we
give a recursive presentation of this relation for the calculus used
in the encoding.

$\red \subseteq \pi \times \pi$
$\red : \pi \to \mathcal{P}(\pi)$

\begin{mathpar}
  \inferrule* [lab=Comm] { \textsf{match}( x_{src}, x_{trgt} ) } { x_{trgt}?(y)P \; | \; x_{src}!\langle {Q} \rangle \red P\{\quotep{Q}/y}\} }
  \and \\
  \inferrule* [lab=Par] {{P} \red {P}'} {{{P} | {Q}} \red {{P}' | {Q}}}
  \and
  \inferrule* [lab=Equiv]{{{P} \scong {P}'} \andalso {{P}' \red {Q}'} \andalso {{Q}' \scong {Q}}}{{P} \red {Q}}
\end{mathpar}

\begin{eqnarray*}
  match_{\equiv} (\quotep{P},\quotep{Q}) & := & P \equiv Q \\
  match_{\dagger}(\quotep{P},\quotep{Q}) & := & \forall R. P|Q \red^{*} R => R \red^{*} 0 \\
  match_{K}(\quotep{P},\quotep{Q}) & := & K \mbox{ for some context } K
\end{eqnarray*}

$u?(x)P | u!\langle Q \rangle \red P\{\quotep{Q}/x\}$

%We write $\wred$ for $\red^*$, and $P\red$ if $\exists Q $ such that $ P \red Q$.
We write $P\red$ if $\exists Q $ such that $ P \red Q$ and $P\not\red$, otherwise.

\section{Replication}

As mentioned before, it is known that replication (and hence
recursion) can be implemented in a higher-order process algebra
\cite{SangiorgiWalker}. As our first example of calculation with the
machinery thus far presented we give the construction explicitly in
the {\rhoc}.

\begin{eqnarray}
	D_{x} & := & \prefix{x}{y}{(\binpar{\outputp{x}{y}}{@{y}})} \nonumber\\
	\bangp_{x}{P} & := & \binpar{{x}!\langle{\binpar{D_{x}}{P}}\rangle}{D_{x}} \nonumber
\end{eqnarray}

\begin{eqnarray}
	\bangp_{x}{P} & & \nonumber\\
	=
	& {x}!\langle{(\prefix{x}{y}{(\outputp{x}{y} | @{y})) | P}}\rangle 
	      | \prefix{x}{y}{(\outputp{x}{y} | @{y})} & \nonumber\\
	\red
	& (\outputp{x}{y} | @{y})\substn{\quotep{(\prefix{x}{y}{(@{y} | \outputp{x}{y})) | P}}}{y} & \nonumber\\
	=
	& \outputp{x}{\quotep{(\prefix{x}{y}{(\outputp{x}{y} | @{y})) | P}}}
	  | {(\prefix{x}{y}{(\outputp{x}{y} | @{y})) | P}} & \nonumber\\
	\red
	& \ldots & \nonumber\\
	\red^*
	& P | P | \ldots & \nonumber
\end{eqnarray}

Of course, this encoding, as an implementation, runs away, unfolding
$\bangp{P}$ eagerly. A lazier and more implementable replication
operator, restricted to input-guarded processes, may be obtained as follows.

\begin{eqnarray}
\bangp{\prefix{u}{v}{P}} 
	:= 
	\binpar{\lift{x}{\prefix{u}{v}{(\binpar{D(x)}{P})}}}{D(x)} \nonumber
\end{eqnarray}

\begin{remark}
  Note that the lazier definition still does not deal with summation
  or mixed summation (i.e. sums over input and output). The reader is
  invited to construct definitions of replication that deal with these
  features. 

  Further, the definitions are parameterized in a name, $x$. Can you,
  gentle reader, make a definition that eliminates this parameter and
  guarantees no accidental interaction between the replication
  machinery and the process being replicated -- i.e. no accidental
  sharing of names used by the process to get its work done and the
  name(s) used by the replication to effect copying. This latter
  revision of the definition of replication is crucial to obtaining
  the expected identity $!!P \sim !P$.
\end{remark}

\begin{remark}\label{rem:paradoxical_combinator}
  The reader familiar with the lambda calculus will have noticed the
  similarity between $D$ and the paradoxical combinator.

  [Ed. note: the existence of this seems to suggest we have to be more
  restrictive on the set of processes and names we admit if we are to
  support no-cloning.]
\end{remark}

\subsubsection{Bisimulation}

The computational dynamics gives rise to another kind of equivalence,
the equivalence of computational behavior. As previously mentioned
this is typically captured \emph{via} some form of bisimulation.

% The notion we use in this paper is weak barbed bisimulation
% \cite{milner91polyadicpi}.

The notion we use in this paper is derived from weak barbed
bisimulation \cite{milner91polyadicpi}. 

\begin{definition}
An \emph{observation relation}, $\downarrow_{\mathcal N}$, over a set
of names, $\mathcal N$, is the smallest relation satisfying the rules
below.

\infrule[Out-barb]{y \in {\mathcal N}, \; x \nameeq y}
		  {\outputp{x}{v} \downarrow_{\mathcal N} x}
\infrule[Par-barb]{\mbox{$P\downarrow_{\mathcal N} x$ or $Q\downarrow_{\mathcal N} x$}}
		  {\binpar{P}{Q} \downarrow_{\mathcal N} x}

We write $P \Downarrow_{\mathcal N} x$ if there is $Q$ such that 
$P \wred Q$ and $Q \downarrow_{\mathcal N} x$.
\end{definition}

\begin{definition}
%\label{def.bbisim}
An  ${\mathcal N}$-\emph{barbed bisimulation} over a set of names, ${\mathcal N}$, is a symmetric binary relation 
${\mathcal S}_{\mathcal N}$ between agents such that $P\rel{S}_{\mathcal N}Q$ implies:
\begin{enumerate}
\item If $P \red P'$ then $Q \wred Q'$ and $P'\rel{S}_{\mathcal N} Q'$.
\item If $P\downarrow_{\mathcal N} x$, then $Q\Downarrow_{\mathcal N} x$.
\end{enumerate}
$P$ is ${\mathcal N}$-barbed bisimilar to $Q$, written
$P \wbbisim_{\mathcal N} Q$, if $P \rel{S}_{\mathcal N} Q$ for some ${\mathcal N}$-barbed bisimulation ${\mathcal S}_{\mathcal N}$.
\end{definition}

$\mathcal{R} \subseteq \pi \times \pi$

$P \mathcal{R} Q => \forall P'. P \red P' \Rightarrow \exists Q'. Q \red Q', P' \mathcal{R} Q'$

$P \vdash x \Rightarrow Q \vdash x$

\begin{mathpar}
  \inferrule*[lab=Out-barb]{x \nameeq y}{{y}!\langle{Q}\rangle \vdash x}
  \and
  \inferrule*[lab=Par-barb]{\mbox{$P\vdash x$ or $Q\vdash x$}}{\binpar{P}{Q} \vdash x}
\end{mathpar}

\subsubsection{Contexts}

One of the principle advantages of computational calculi like the
$\pi$-calculus is a well-defined notion of context,
contextual-equivalence and a correlation between
contextual-equivalence and notions of bisimulation. The notion of
context allows the decomposition of a process into (sub-)process and
its syntactic environment, its context. Thus, a context may be
thought of as a process with a ``hole'' (written $\Box$) in it. The
application of a context $M$ to a process $P$, written $M[P]$, is
tantamount to filling the hole in $M$ with $P$. In this paper we do
not need the full weight of this theory, but do make use of the notion
of context in the proof the main theorem. 

\begin{mathpar}
  \inferrule* [lab=summation] {} {{M_{M},M_{N}} \bc \Box \;|\; x.M_{A} \;|\; M_{M}+M_{N}}
  \and
  \inferrule* [lab=agent] {} {{M_{A}} \bc (\vec{x})M_{P} \;| \; \clift{P_0,\ldots,M_{P},\ldots,P_N}}
  \and \\
  \inferrule* [lab=process] {} {{M_{P}} \bc M_{N} \;| \;P|M_{P} }
\end{mathpar} 

\begin{mathpar}
  \inferrule* [lab=sychronization] {} {M_{N} \bc \Box \;|\; x?M_{F} \;|\; x!M_{C}}
  \and
  \inferrule* [lab=abstraction] {} {{M_{F}} \bc (x)M_{P} }
  \and
  \inferrule* [lab=concretion] {} {{M_{C}} \bc \langle M_{P} \rangle }
  \and \\
  \inferrule* [lab=process] {} {{M_{P}} \bc M_{N} \;| \;P|M_{P} }
\end{mathpar}

\begin{definition}[contextual application] Given a context $M$, and
  process $P$, we define the \emph{contextual application}, $M[P] :=
  M\{P/\Box\}$. That is, the contextual application of M to P is the
  substitution of $P$ for $\Box$ in $M$.
\end{definition}

$\meaningof{-} : L \to \mathcal{P}(\pi)$

\begin{mathpar}
  \inferrule* [lab=collection] {} {\meaningof{true} = \pi, \and \meaningof{~E} = \pi \setminus \meaningof{E}, \and \meaningof{E_{1} \& E_{2}} = \meaningof{E_{1}} \cap \meaningof{E_{2}}}
\end{mathpar}

\begin{mathpar}
  \inferrule* [lab=structure] {} {\meaningof{0} = \{ P \in \pi | P \equiv 0 \}, \and \\ \meaningof{E_1 | E_2} = \{ P \in \pi | P \equiv P_{1} | P_{2}, P_{1} \in \meaningof{E_{1}}, P_{2} \in \meaningof{E_2}\} }
\end{mathpar}

\begin{mathpar}
 \inferrule* [lab=behavior] {} {\meaningof{\langle a?b \rangle E} = \{ P \in \pi | P \equiv Q | u?(y)P', \\ \and \\\\ \and \\ \;\;\; u \in \meaningof{a}, \forall z.P'\{z/y\} \in \meaningof{E\{z/b\}}\}, \and \\ \meaningof{a!E} = \{ P \in \pi | P \equiv Q | x!\langle P' \rangle, x \in \meaningof{a} P' \in \meaningof{E}\} }
\end{mathpar}

\begin{mathpar}
 \inferrule* [lab=nominal] {} {\meaningof{\quotep{E}} = \{ \quotep{P} \in \quotep{\pi} | P \in \meaningof{E} \}, \and \meaningof{\quotep{P}} = \{ \quotep{Q} \in \quotep{\pi} | P \equiv Q \} \and \\ \meaningof{@\quotep{E}} = \{ P \in \pi | P \equiv @x, x \in \meaningof{E} \}}
\end{mathpar}

\begin{eqnarray*}
  \\
  \meaningof{-} : TS \to ST
\end{eqnarray*}

\begin{eqnarray*}
  \\
  L : TS \to ST
\end{eqnarray*}

\begin{eqnarray*}
  \\
  P \models E \iff P \in \meaningof{E}
\end{eqnarray*}

\begin{eqnarray*}
  P \approx_{L} Q \iff \forall E \in L. P \models E \iff Q \models E
\end{eqnarray*}

\begin{eqnarray*}
  P \approx_{K} Q
\end{eqnarray*}

\begin{eqnarray*}
  P \approx Q
\end{eqnarray*}

$\approx_{K} = \approx = \approx_{L}$

\subsubsection{Contextual duality}

Note that contexts extend the quotation operation to a family of
operations from processes to names. Given a context, $M$, we can
define a \emph{nominal context}, $\quotep{M}$ by $\quotep{M}[P] :=
\quotep{M[P]}$. To foreshadow what is to come we observe that these
operations enjoy a duality with processes very much like the duality
between vectors and maps from vectors to scalars.

Further, because the calculus is essentially higher-order, we have a
correspondence between contexts and processes. More specifically,
given a name $x$ and a context $M$ we can construct $M^{*}_{x}$ such
that 

\begin{mathpar}
  M^{*}_{x} | \lift{x}{P} \red M[P]
\end{mathpar}

namely,

\begin{mathpar}
  M^{*}_{x} := x?(u).M[\dropn{u}]
\end{mathpar}

The dependence of $M^{*}_{x}$ on a name makes it an abstraction, 

\begin{mathpar}
  M^{*} := (x)x?(u).M[\dropn{u}]
\end{mathpar}

\subsection{Additional notation}

It will sometimes be convenient to denote the process a name
quotes. We already have the notation $x = \quotep{P}$, but it will be
convenient to introduce an alternate notation, $\procn{x}$, when we
want to emphasize the connection to the use of the name. Note that, by
virtue of name equivalence, $\quotep{\procn{x}} \nameeq x$; so, the
notation is consistent with previous definitions.

Further, because names have structure it is possible to effect
substitutions on the basis of that structure. This means we need to
upgrade our notation for substitutions, which we accomplish by
adapting comprehension notation. Thus,

\begin{mathpar}
  P\{ y / x : x \in S \}
\end{mathpar}

is interpreted to mean the process derived from P by replacing (in a
capture-avoiding manner) each occurrence of $x$ in $S$ by $y$. For example,

\begin{mathpar}
  P\{ \quotep{\procn{x}|\procn{x}} / x : x \in \freenames{P} \}
\end{mathpar}

will replace each (occurrence) of a free name $x$ in $P$ by
$\quotep{\procn{x}|\procn{x}}$.

Also, we will avail ourselves of the notation $x^{L}$ and $x^{R}$ to
denote injections of a name into disjoint copies of the name
space. There are numerous ways to accomplish this. One example can be
found in \cite{MeredithR05}. This notation overloads to vectors of
names: $\vec{x}^{\pi} := (x_{i}^{\pi} \; : \; 0 \leq i < |\vec{x}| )$ where $\pi \in \{L,R\}$.

We also use $P^{\Box} := P|\Box$.

In \cite{MeredithR05} an interpretation of the new operator is
given. It turns out that there are several possible interpretations
all enjoying the requisite algebraic properties of the operator (see
\cite{milner91polyadicpi}). We will therefore make liberal use of
$(\nu\; \vec{x})P$.

% subsection the_syntax_and_semantics_of_the_notation_system (end)   

\input{qm2pi.qmops} 

\input{qm2pi.sterngerlach} 

\input{qm2pi.metric} 

% section concurrent_process_calculi (end)

%\input{qm2pi.proofsketch}

% section proof sketch (end)

%\input{qm2pi.slviaknots} 

% section spatial logic via knots (end)

\input{qm2pi.conclusion}

% section conclusion (end)

%\input{qm2pi.dtcodes} 

% section wiring algorithm (end)

\input{qm2pi.ack} 

% section acknowledgments (end)

\newpage


\bibliographystyle{plain}   
\bibliography{../../biblios/main.bib}

\input{qm2pi.rhodetails}

\end{document}



% section front matter (end)

\section{Introduction}\label{sec:introduction} % (fold)
In this draft of the material i am going to have to dispense with the
usual writing conventions adopted in papers on these topics. i'm going
to have adopt whatever tone i need at the time i'm writing up the
calculations. Sometimes this may be very conversational; others it may
be the barest mathematical grunts; others still it may be that i have
lifted text from one of my other papers because the exposition of some
point was better said there. i hope that my readers are not unduly put
out by this decision. i'm not doing this to flout convention or be
rebellious. i find these calculations very technically challenging. To
keep everything going technically, something has to give; i have to
let go of some cognitive burden. So, the academic writing style --
with all of its trade-offs in terms of facilitating technical
communication -- is what i'm letting go of. Perhaps subsequent drafts
can be tightened and polished, but for now, i'm going to speak as if
we were sitting together in a coffee shop with a laptop, wifi and a
pad of paper and a pencil.

So, here's what i have to say. We -- you and i, comfortably ensconced
in our coffee shop and well-equipped with our tools -- can realize and
carry out the calculations of quantum mechanics over a very different
formal theory of dynamics, a formal theory of dynamics that
corresponds to a theory of concurrent computation with
\emph{reflection}. It has the advantage that the underlying theory is
already `quantized', but supports analogues all of the continuuous
operations. Strikingly, this underlying theory has recently been
connected with a notion of metric that we can show, by calculating
together, coincides with the metric induced by the inner product.

There are a lot of reasons why you might be interested in seeing
calculations of this form. Here's why i'm interested. For the past
several centuries there has been no competitor to the ``Newtonian''
account of dynamics. As a result the predominant share of accounts of
dynamical systems and situations have had to be formulated in terms of
the Newtonian machinery. i view this as an intellectually dangerous
position to occupy. Everything, despite it's intrinsic shape, turns
into a nail to be hit with this hammer. Recently, however, the theory
of computation has matured to the point where we have candidates for
theories of dynamics that offer very different perspective on
reasoning about dynamical systems and situations. Testing these
candidates against very successful accounts of dynamical situations,
like quantum mechanics, is going to give us some sense of how mature
they are and some measure of the quality of these accounts of
dynamics.

\subsection{Summary of contributions and outline of paper}

So, we're going to develop an interpretation of the operations of
quantum mechanics normally interpreted by Hilbert spaces and
operators. We're going to do this over a theory of computation. Note
that this is very different than the usual quantum computation program
which develops notions of computation over quantum mechanics. Rather,
we are developing a story that aligns with Wheeler's slogan: It from
Bit. To do this we will first provide an account of the theory of
computation at play here. Then we will dive into a calculation-driven
interpretation of the operations of quantum mechanics.

The reason we take this approach is that -- until very recently --
there hasn't been an axiomatic account of quantum mechanics. As a
result there has been no sharp delineation of the mathematical theory
supporting interpretation of the physical theory and the physical
theory, itself. So, ambient features of the maths are free to be
exploited (or supressed) without a real accounting of their physical
relevance. There is no sharp statement ``here's the physical theory''
qua \emph{theory} and ``here's the mathematical interpretation''
enabling a judgment of how faithful the interpretation is -- apart
from experimental observation. When there is an axiomatic account we
can judge how well a given mathematical formalism supports an
interpretation of the axioms, independent of
experimentation. Likewise, we can judge how well we have captured our
physical evidence and experience with our axiomatics, independent of
any specific mathematical implementation, with accidental detail that
may or may not have physical significance. 

In lieu of a fully fleshed out and vetted axiomatic account of quantum
mechanics, interpreting the operational notions in service of modeling
physical systems will have to suffice. In other words, we are not in
the business of providing a model of Hilbert spaces and operators. We
are in the business of providing a model of quantum mechanics because
we are motivated by testing our notions of dynamics against physical
theory; and, the predictive calculations of the physical theory must
serve as the best formulation -- shy of a fully fleshed out axiomatic
account -- of the physical theory itself (as they have for scientific
theories since time immemorial). Put another way, despite a
whole-hearted commitment to an It-from-Bit ontology, we are firmly
aligned with the shut-up-and-calculate camp as the best way to obtain
results either from the physical perspective or as a quality assurance
measure of our fledgling theory of dynamics.

In detail, we present a reflective process calculus. Then we develop
intuitive correspondences between the notions available in this
calculus and the usual physical notions supporting quantum mechanical
calculations. Thus, 

\begin{table}[htp]
  \center{
    \fbox{
      \begin{tabular}{c|c}
        quantum mechanics & process calculus \\
        \hline
        scalar & name \\
        state vector & process \\
        dual & contextual duals \\
        matrix & formal sums of process-context-dual pairs \\
        orthogonality & process annihilation \\
        inner product & execution-formula + quoting
      \end{tabular}
    }
  }
  \caption{QM - process calculi correspondences}
\end{table}

Then we tighten up these intuitions to operational definitions. We
employ the Dirac notation as the best proxy we can find for an
abstract syntax of the quantum mechanical notions. The definitions we
develop put us in contact with equational constraints coming from the
theory that we demonstrate the definitions and calculations satisfy.

This puts us in a position to shut up and calculate for the
Stern-Gerlach experimental set up, showing how these predictive
calculations become calculations on processes in our theory of a
reflective process calculus.

Penultimately, we demonstrate that the notion of metric coming from
the inner product coincides with the notion of metric available from
the theory of bisimulation. This demonstration gives us the right to
think of space as arising from behavior. Finally, we consider where we
might go from the new vantage point we have obtained.

% section introduction (end) 
 
% section introduction (end)

% \documentclass[12pt]{llncs}
%\documentclass{jktr}

\usepackage[pdftex]{hyperref}                   
\usepackage {listings}
\usepackage {mathpartir}
\usepackage{bcprules}
%\usepackage{listings}
                       
\usepackage{graphicx} 
%\usepackage[margins=2.5cm,nohead,nofoot]{geometry}
%\usepackage{geometry}
\usepackage{amsfonts}
\usepackage{amstext}
\usepackage{latexsym}
\usepackage{amssymb}
\usepackage{color}


%\include{myPreamble}
\include{qm2pi.local} 

%\ifpdf
%\usepackage[pdftex]{graphicx}
%\else
%\usepackage{graphicx}
%\fi

 % \ifpdf
%  \usepackage{pdfsync}
%  \if


%\title{Brief Article}
%\author{David F. Snyder}
%\author{L.G. Meredith}

%\address{Dept. of Math., Texas State University--San Marcos, San Marcos, TX 78666}
       
\pagestyle{empty}


\begin{document}

\lstset{language=[Objective]Caml,frame=shadowbox}

\input{qm2pi.front}

% section front matter (end)

\input{qm2pi.intro} 
 
% section introduction (end)

% \input{qm2pi.knotations} 

% section notation (end)

\input{qm2pi.process.calculi} 

% section concurrent_process_calculi_and_spatial_logics_ (end)
    
%\input{qm2pi.knots2pi} 

%\input{qm2pi.trefoil} 

%\input{qm2pi.mainthm} 

% subsection basic_interpretation (end)

%\input{qm2pi.rho.presentation} 
\subsection{The syntax and semantics of the notation system}\label{sub:the_syntax_and_semantics_of_the_notation_system} % (fold)

We now summarize a technical presentation of the calculus that
embodies our theory of dynamics. The typical presentation of such a
calculus follows the style of giving generators and relations on
them. The grammar, below, describing term constructors, freely
generates the set of processes, $\Proc$. This set is then quotiented
by a relation known as structural congruence and it is over this set
that the notion of dynamics is expressed. This presentation is
essentially that of \cite{MeredithR05} with the addition of
polyadicity and summation. For readability we have relegated some of
the technical subtleties to an appendix.

\subsubsection{Process grammar}\label{subsub:process_grammar}

\begin{mathpar}
  \inferrule* [lab=synchronization] {} {{M} \bc \pzero \;|\; x?F \;|\; x!C }
  \and
  \inferrule* [lab=abstraction] {} {{F} \bc (x)P}
  \and
  \inferrule* [lab=concretion] {} {{C} \bc \langle Q \rangle}
  \and
  \inferrule* [lab=process] {} {{P,Q} \bc M \;| \;P|Q \;|\; @{x}}
  \and
  \inferrule* [lab=name] {} {{x} \bc \quotep{P}}
\end{mathpar} 

Note that $\vec{x}$ (resp. $\vec{P}$) denotes a vector of names
(resp. processes) of length $|\vec{x}|$ (resp. $|\vec{P}|$). We adopt
the following useful abbreviations.

\begin{mathpar}
   x?(\vec{y}).P := x.(\vec{y})P \and  x\clift{\vec{P}} := x.\clift{\vec{P}}
   \and x!(y) := \lift{x}{\dropn{y}}
   \and \Pi_{i=0}^{n-1}P_i := P_0 | \ldots | P_{n-1}
\end{mathpar}

\subsubsection{Structural congruence}

\paragraph{Free and bound names and alpha-equivalence.} At the
core of structural equivalence is alpha-equivalence which identifies
process that are the same up to a change of variable. Formally, we
recognize the distinction between free and bound names. The free names
of a process, $\freenames{P}$, may be calculated recursively as
follows:

\begin{mathpar}
\freenames{\pzero} := \emptyset
  \and \\
  \freenames{x?(y).P} := \{ x \} \cup (\freenames{P} \setminus \{ y \})
  \and 
  \freenames{x!\langle P \rangle} := \{ x \} \cup \{ P \} 
  \and \\
  \freenames{P|Q} := \freenames{P} \cup \freenames{Q}
  \and \\
  \freenames{@{x}} := \{ x \}
\end{mathpar}

$\pi$
$\quotep{\pi}$

$\freenames{-} : \pi \to \mathcal{P}(\quotep{\pi})$

\begin{eqnarray*}
  \freenames{\pzero} & := & \emptyset \\
  \freenames{x?(y).P} & := & \{ x \} \cup (\freenames{P} \setminus \{ y \}) \\
  \freenames{x!\langle P \rangle} & := & \{ x \} \cup \{ P \} \\
  \freenames{P|Q} & := & \freenames{P} \cup \freenames{Q} \\
  \freenames{\dropn{x}} & := & \{ x \}
\end{eqnarray*}

The bound names of a process, $\boundnames{P}$, are those names occurring in $P$
that are not free. For example, in $x?(y).0$, the name $x$ is free, while $y$ is bound.

\begin{mathpar}
  \inferrule* [lab=monoidal-laws] {} { P|Q \equiv Q|P \and P|0 \equiv P \and P|(Q|R) \equiv (P|Q)|R }
\end{mathpar}

\begin{mathpar}
  \inferrule* [lab=alpha-equivalence] {} { (x)P \equiv (y)P\{y/x\} \and y \not\in \freenames{P} }
\end{mathpar}

\begin{definition}
Then two processes, $P,Q$, are alpha-equivalent if $P = Q\{\vec{y}/\vec{x}\}$ for
some $\vec{x} \in \boundnames{Q},\vec{y} \in \boundnames{P}$, where $Q\{\vec{y}/\vec{x}\}$
denotes the capture-avoiding substitution of $\vec{y}$ for $\vec{x}$ in $Q$.
\end{definition}

\begin{definition}
  The {\em structural congruence} \cite{SangiorgiWalker} , $\equiv$,
  between processes is the least congruence containing
  alpha-equivalence, satisfying the abelian monoid laws
  (associativity, commutativity and $\pzero$ as identity) for parallel
  composition $|$ and for summation $+$.
\end{definition}

\subsection{Name equivalence}

We take name equivalence, written $\nameeq$, to be the smallest
equivalence relation generated by the following rules.

\begin{mathpar}
\inferrule*[lab=Quote-drop]
{ }
{ \quotep{@{x}} \nameeq x }

\inferrule*[lab=Struct-equiv]
{ P \scong Q }
{ \quotep{P} \nameeq \quotep{Q} }
\end{mathpar}

The astute reader will have noticed that the mutual recursion of names
and processes imposes a mutual recursion on alpha-equivalence and
structural equivalence via name-equivalence. Fortunately, all of this
works out pleasantly and we may calculate in the natural way, free of
concern. The reader interested in the details is referred to the
appendix \ref{appendix:rho_details}.

\subsection{Substitution}

We use $\Proc$ for the set of processes, $\QProc$ for the set of
names, and $\id{\{}\vec{y} / \vec{x} \id{\}}$ to denote partial maps,
$s : \QProc \rightarrow \QProc$. A map, $s$ lifts, uniquely, to a map
on process terms, $\widehat{s} : \Proc \rightarrow \Proc$ by the
following equations.

\begin{mathpar}
  (0) \psubstp{Q}{P} := 0 \\
  (R \juxtap S) \psubstp{Q}{P}
  :=    
  (R)\psubstp{Q}{P} \juxtap (S) \psubstp{Q}{P} \\
  (x?(y).R) \psubstp{Q}{P}    
  :=    
  (x)\substp{Q}{P} (z)\concat( (R \psubstn{z}{y}) \psubstp{Q}{P} ) \\
  (\lift{x}{R}) \psubstp{Q}{P}  
  :=
  \lift{(x)\substp{Q}{P}}{ R \psubstp{Q}{P} } \\
%   (\dropn{x})  \psubstp{Q}{P}       
%   := 
%   \left\{ 
%     \begin{array}{ccc} 
%       \dropn{\quotep{Q}} & & x \nameeq \quotep{P} \\
%       \dropn{x} & & otherwise \\
%     \end{array}
%   \right. 
  (\dropn{x})  \psubstp{Q}{P}       
  := 
  \left\{ 
    \begin{array}{ccc} 
      Q & & x \nameeq \quotep{P} \\
      \dropn{x} & & otherwise \\
    \end{array}
  \right.
\end{mathpar}
 

where

\begin{eqnarray}
  (x)\id{\{} \lpquote Q \rpquote / \lpquote P \rpquote \id{\}}            = 
  \left\{ 
    \begin{array}{ccc}
      \lpquote Q \rpquote & & x \nameeq \lpquote P \rpquote \\
      x & & otherwise \\
    \end{array}
  \right. \nonumber
\end{eqnarray}

and $z$ is chosen distinct from $\quotep{P}$, $\quotep{Q}$, the free
names in $Q$, and all the names in $R$. Our $\alpha$-equivalence will
be built in the standard way from this substitution.

\begin{remark}\label{rem:no_self_referential_names}
  One consequence of these definitions is that $\forall P. \quotep{P}
  \not\in \freenames{P}$.
\end{remark}

\subsection{ Dynamic quote: an example }

Anticipating something of what's to come, consider applying the
substitution, $\widehat{\id{\{}u / z \id{\}}}$, to the following pair
of processes, $\lift{w}{y!(z)}$ and $w[ \lpquote y!(z) \rpquote ]$.

\begin{eqnarray}
	\lift{w}{y!(z)}\widehat{\id{\{}u / z \id{\}}}
		& = &
		\lift{w}{y!(u)} \nonumber\\
	w[ \lpquote y!(z) \rpquote ] \widehat{ \id{\{}u / z \id{\}} }
		& = &
		w[ \lpquote y!(z) \rpquote ] \nonumber
\end{eqnarray}

Because the body of the process between quotes is impervious to
substitution, we get radically different answers. In fact, by
examining the first process in an input context,
e.g. $x?(z).\lift{w}{y!(z)}$, we see that the process under the lift
operator may be shaped by prefixed inputs binding a name inside it. In
this sense, the lift operator will be seen as a way to dynamically
construct processes before reifying them as names.

Finally equipped with these standard features we can present the
dynamics of the calculus.

\subsubsection{Operational semantics} 

Finally, we introduce the computational dynamics. What marks these
algebras as distinct from other more traditionally studied algebraic
structures, e.g. vector spaces or polynomial rings, is the manner in
which dynamics is captured. In traditional structures, dynamics is typically
expressed through morphisms between such structures, as in linear maps
between vector spaces or morphisms between rings. In algebras
associated with the semantics of computation, the dynamics is
expressed as part of the algebraic structure itself, through a
reduction reduction relation typically denoted by $\red$. Below, we
give a recursive presentation of this relation for the calculus used
in the encoding.

$\red \subseteq \pi \times \pi$
$\red : \pi \to \mathcal{P}(\pi)$

\begin{mathpar}
  \inferrule* [lab=Comm] { \textsf{match}( x_{src}, x_{trgt} ) } { x_{trgt}?(y)P \; | \; x_{src}!\langle {Q} \rangle \red P\{\quotep{Q}/y}\} }
  \and \\
  \inferrule* [lab=Par] {{P} \red {P}'} {{{P} | {Q}} \red {{P}' | {Q}}}
  \and
  \inferrule* [lab=Equiv]{{{P} \scong {P}'} \andalso {{P}' \red {Q}'} \andalso {{Q}' \scong {Q}}}{{P} \red {Q}}
\end{mathpar}

\begin{eqnarray*}
  match_{\equiv} (\quotep{P},\quotep{Q}) & := & P \equiv Q \\
  match_{\dagger}(\quotep{P},\quotep{Q}) & := & \forall R. P|Q \red^{*} R => R \red^{*} 0 \\
  match_{K}(\quotep{P},\quotep{Q}) & := & K \mbox{ for some context } K
\end{eqnarray*}

$u?(x)P | u!\langle Q \rangle \red P\{\quotep{Q}/x\}$

%We write $\wred$ for $\red^*$, and $P\red$ if $\exists Q $ such that $ P \red Q$.
We write $P\red$ if $\exists Q $ such that $ P \red Q$ and $P\not\red$, otherwise.

\section{Replication}

As mentioned before, it is known that replication (and hence
recursion) can be implemented in a higher-order process algebra
\cite{SangiorgiWalker}. As our first example of calculation with the
machinery thus far presented we give the construction explicitly in
the {\rhoc}.

\begin{eqnarray}
	D_{x} & := & \prefix{x}{y}{(\binpar{\outputp{x}{y}}{@{y}})} \nonumber\\
	\bangp_{x}{P} & := & \binpar{{x}!\langle{\binpar{D_{x}}{P}}\rangle}{D_{x}} \nonumber
\end{eqnarray}

\begin{eqnarray}
	\bangp_{x}{P} & & \nonumber\\
	=
	& {x}!\langle{(\prefix{x}{y}{(\outputp{x}{y} | @{y})) | P}}\rangle 
	      | \prefix{x}{y}{(\outputp{x}{y} | @{y})} & \nonumber\\
	\red
	& (\outputp{x}{y} | @{y})\substn{\quotep{(\prefix{x}{y}{(@{y} | \outputp{x}{y})) | P}}}{y} & \nonumber\\
	=
	& \outputp{x}{\quotep{(\prefix{x}{y}{(\outputp{x}{y} | @{y})) | P}}}
	  | {(\prefix{x}{y}{(\outputp{x}{y} | @{y})) | P}} & \nonumber\\
	\red
	& \ldots & \nonumber\\
	\red^*
	& P | P | \ldots & \nonumber
\end{eqnarray}

Of course, this encoding, as an implementation, runs away, unfolding
$\bangp{P}$ eagerly. A lazier and more implementable replication
operator, restricted to input-guarded processes, may be obtained as follows.

\begin{eqnarray}
\bangp{\prefix{u}{v}{P}} 
	:= 
	\binpar{\lift{x}{\prefix{u}{v}{(\binpar{D(x)}{P})}}}{D(x)} \nonumber
\end{eqnarray}

\begin{remark}
  Note that the lazier definition still does not deal with summation
  or mixed summation (i.e. sums over input and output). The reader is
  invited to construct definitions of replication that deal with these
  features. 

  Further, the definitions are parameterized in a name, $x$. Can you,
  gentle reader, make a definition that eliminates this parameter and
  guarantees no accidental interaction between the replication
  machinery and the process being replicated -- i.e. no accidental
  sharing of names used by the process to get its work done and the
  name(s) used by the replication to effect copying. This latter
  revision of the definition of replication is crucial to obtaining
  the expected identity $!!P \sim !P$.
\end{remark}

\begin{remark}\label{rem:paradoxical_combinator}
  The reader familiar with the lambda calculus will have noticed the
  similarity between $D$ and the paradoxical combinator.

  [Ed. note: the existence of this seems to suggest we have to be more
  restrictive on the set of processes and names we admit if we are to
  support no-cloning.]
\end{remark}

\subsubsection{Bisimulation}

The computational dynamics gives rise to another kind of equivalence,
the equivalence of computational behavior. As previously mentioned
this is typically captured \emph{via} some form of bisimulation.

% The notion we use in this paper is weak barbed bisimulation
% \cite{milner91polyadicpi}.

The notion we use in this paper is derived from weak barbed
bisimulation \cite{milner91polyadicpi}. 

\begin{definition}
An \emph{observation relation}, $\downarrow_{\mathcal N}$, over a set
of names, $\mathcal N$, is the smallest relation satisfying the rules
below.

\infrule[Out-barb]{y \in {\mathcal N}, \; x \nameeq y}
		  {\outputp{x}{v} \downarrow_{\mathcal N} x}
\infrule[Par-barb]{\mbox{$P\downarrow_{\mathcal N} x$ or $Q\downarrow_{\mathcal N} x$}}
		  {\binpar{P}{Q} \downarrow_{\mathcal N} x}

We write $P \Downarrow_{\mathcal N} x$ if there is $Q$ such that 
$P \wred Q$ and $Q \downarrow_{\mathcal N} x$.
\end{definition}

\begin{definition}
%\label{def.bbisim}
An  ${\mathcal N}$-\emph{barbed bisimulation} over a set of names, ${\mathcal N}$, is a symmetric binary relation 
${\mathcal S}_{\mathcal N}$ between agents such that $P\rel{S}_{\mathcal N}Q$ implies:
\begin{enumerate}
\item If $P \red P'$ then $Q \wred Q'$ and $P'\rel{S}_{\mathcal N} Q'$.
\item If $P\downarrow_{\mathcal N} x$, then $Q\Downarrow_{\mathcal N} x$.
\end{enumerate}
$P$ is ${\mathcal N}$-barbed bisimilar to $Q$, written
$P \wbbisim_{\mathcal N} Q$, if $P \rel{S}_{\mathcal N} Q$ for some ${\mathcal N}$-barbed bisimulation ${\mathcal S}_{\mathcal N}$.
\end{definition}

$\mathcal{R} \subseteq \pi \times \pi$

$P \mathcal{R} Q => \forall P'. P \red P' \Rightarrow \exists Q'. Q \red Q', P' \mathcal{R} Q'$

$P \vdash x \Rightarrow Q \vdash x$

\begin{mathpar}
  \inferrule*[lab=Out-barb]{x \nameeq y}{{y}!\langle{Q}\rangle \vdash x}
  \and
  \inferrule*[lab=Par-barb]{\mbox{$P\vdash x$ or $Q\vdash x$}}{\binpar{P}{Q} \vdash x}
\end{mathpar}

\subsubsection{Contexts}

One of the principle advantages of computational calculi like the
$\pi$-calculus is a well-defined notion of context,
contextual-equivalence and a correlation between
contextual-equivalence and notions of bisimulation. The notion of
context allows the decomposition of a process into (sub-)process and
its syntactic environment, its context. Thus, a context may be
thought of as a process with a ``hole'' (written $\Box$) in it. The
application of a context $M$ to a process $P$, written $M[P]$, is
tantamount to filling the hole in $M$ with $P$. In this paper we do
not need the full weight of this theory, but do make use of the notion
of context in the proof the main theorem. 

\begin{mathpar}
  \inferrule* [lab=summation] {} {{M_{M},M_{N}} \bc \Box \;|\; x.M_{A} \;|\; M_{M}+M_{N}}
  \and
  \inferrule* [lab=agent] {} {{M_{A}} \bc (\vec{x})M_{P} \;| \; \clift{P_0,\ldots,M_{P},\ldots,P_N}}
  \and \\
  \inferrule* [lab=process] {} {{M_{P}} \bc M_{N} \;| \;P|M_{P} }
\end{mathpar} 

\begin{mathpar}
  \inferrule* [lab=sychronization] {} {M_{N} \bc \Box \;|\; x?M_{F} \;|\; x!M_{C}}
  \and
  \inferrule* [lab=abstraction] {} {{M_{F}} \bc (x)M_{P} }
  \and
  \inferrule* [lab=concretion] {} {{M_{C}} \bc \langle M_{P} \rangle }
  \and \\
  \inferrule* [lab=process] {} {{M_{P}} \bc M_{N} \;| \;P|M_{P} }
\end{mathpar}

\begin{definition}[contextual application] Given a context $M$, and
  process $P$, we define the \emph{contextual application}, $M[P] :=
  M\{P/\Box\}$. That is, the contextual application of M to P is the
  substitution of $P$ for $\Box$ in $M$.
\end{definition}

$\meaningof{-} : L \to \mathcal{P}(\pi)$

\begin{mathpar}
  \inferrule* [lab=collection] {} {\meaningof{true} = \pi, \and \meaningof{~E} = \pi \setminus \meaningof{E}, \and \meaningof{E_{1} \& E_{2}} = \meaningof{E_{1}} \cap \meaningof{E_{2}}}
\end{mathpar}

\begin{mathpar}
  \inferrule* [lab=structure] {} {\meaningof{0} = \{ P \in \pi | P \equiv 0 \}, \and \\ \meaningof{E_1 | E_2} = \{ P \in \pi | P \equiv P_{1} | P_{2}, P_{1} \in \meaningof{E_{1}}, P_{2} \in \meaningof{E_2}\} }
\end{mathpar}

\begin{mathpar}
 \inferrule* [lab=behavior] {} {\meaningof{\langle a?b \rangle E} = \{ P \in \pi | P \equiv Q | u?(y)P', \\ \and \\\\ \and \\ \;\;\; u \in \meaningof{a}, \forall z.P'\{z/y\} \in \meaningof{E\{z/b\}}\}, \and \\ \meaningof{a!E} = \{ P \in \pi | P \equiv Q | x!\langle P' \rangle, x \in \meaningof{a} P' \in \meaningof{E}\} }
\end{mathpar}

\begin{mathpar}
 \inferrule* [lab=nominal] {} {\meaningof{\quotep{E}} = \{ \quotep{P} \in \quotep{\pi} | P \in \meaningof{E} \}, \and \meaningof{\quotep{P}} = \{ \quotep{Q} \in \quotep{\pi} | P \equiv Q \} \and \\ \meaningof{@\quotep{E}} = \{ P \in \pi | P \equiv @x, x \in \meaningof{E} \}}
\end{mathpar}

\begin{eqnarray*}
  \\
  \meaningof{-} : TS \to ST
\end{eqnarray*}

\begin{eqnarray*}
  \\
  L : TS \to ST
\end{eqnarray*}

\begin{eqnarray*}
  \\
  P \models E \iff P \in \meaningof{E}
\end{eqnarray*}

\begin{eqnarray*}
  P \approx_{L} Q \iff \forall E \in L. P \models E \iff Q \models E
\end{eqnarray*}

\begin{eqnarray*}
  P \approx_{K} Q
\end{eqnarray*}

\begin{eqnarray*}
  P \approx Q
\end{eqnarray*}

$\approx_{K} = \approx = \approx_{L}$

\subsubsection{Contextual duality}

Note that contexts extend the quotation operation to a family of
operations from processes to names. Given a context, $M$, we can
define a \emph{nominal context}, $\quotep{M}$ by $\quotep{M}[P] :=
\quotep{M[P]}$. To foreshadow what is to come we observe that these
operations enjoy a duality with processes very much like the duality
between vectors and maps from vectors to scalars.

Further, because the calculus is essentially higher-order, we have a
correspondence between contexts and processes. More specifically,
given a name $x$ and a context $M$ we can construct $M^{*}_{x}$ such
that 

\begin{mathpar}
  M^{*}_{x} | \lift{x}{P} \red M[P]
\end{mathpar}

namely,

\begin{mathpar}
  M^{*}_{x} := x?(u).M[\dropn{u}]
\end{mathpar}

The dependence of $M^{*}_{x}$ on a name makes it an abstraction, 

\begin{mathpar}
  M^{*} := (x)x?(u).M[\dropn{u}]
\end{mathpar}

\subsection{Additional notation}

It will sometimes be convenient to denote the process a name
quotes. We already have the notation $x = \quotep{P}$, but it will be
convenient to introduce an alternate notation, $\procn{x}$, when we
want to emphasize the connection to the use of the name. Note that, by
virtue of name equivalence, $\quotep{\procn{x}} \nameeq x$; so, the
notation is consistent with previous definitions.

Further, because names have structure it is possible to effect
substitutions on the basis of that structure. This means we need to
upgrade our notation for substitutions, which we accomplish by
adapting comprehension notation. Thus,

\begin{mathpar}
  P\{ y / x : x \in S \}
\end{mathpar}

is interpreted to mean the process derived from P by replacing (in a
capture-avoiding manner) each occurrence of $x$ in $S$ by $y$. For example,

\begin{mathpar}
  P\{ \quotep{\procn{x}|\procn{x}} / x : x \in \freenames{P} \}
\end{mathpar}

will replace each (occurrence) of a free name $x$ in $P$ by
$\quotep{\procn{x}|\procn{x}}$.

Also, we will avail ourselves of the notation $x^{L}$ and $x^{R}$ to
denote injections of a name into disjoint copies of the name
space. There are numerous ways to accomplish this. One example can be
found in \cite{MeredithR05}. This notation overloads to vectors of
names: $\vec{x}^{\pi} := (x_{i}^{\pi} \; : \; 0 \leq i < |\vec{x}| )$ where $\pi \in \{L,R\}$.

We also use $P^{\Box} := P|\Box$.

In \cite{MeredithR05} an interpretation of the new operator is
given. It turns out that there are several possible interpretations
all enjoying the requisite algebraic properties of the operator (see
\cite{milner91polyadicpi}). We will therefore make liberal use of
$(\nu\; \vec{x})P$.

% subsection the_syntax_and_semantics_of_the_notation_system (end)   

\input{qm2pi.qmops} 

\input{qm2pi.sterngerlach} 

\input{qm2pi.metric} 

% section concurrent_process_calculi (end)

%\input{qm2pi.proofsketch}

% section proof sketch (end)

%\input{qm2pi.slviaknots} 

% section spatial logic via knots (end)

\input{qm2pi.conclusion}

% section conclusion (end)

%\input{qm2pi.dtcodes} 

% section wiring algorithm (end)

\input{qm2pi.ack} 

% section acknowledgments (end)

\newpage


\bibliographystyle{plain}   
\bibliography{../../biblios/main.bib}

\input{qm2pi.rhodetails}

\end{document}

 

% section notation (end)

\input{qm2pi.process.calculi} 

% section concurrent_process_calculi_and_spatial_logics_ (end)
    
%\documentclass[12pt]{llncs}
%\documentclass{jktr}

\usepackage[pdftex]{hyperref}                   
\usepackage {listings}
\usepackage {mathpartir}
\usepackage{bcprules}
%\usepackage{listings}
                       
\usepackage{graphicx} 
%\usepackage[margins=2.5cm,nohead,nofoot]{geometry}
%\usepackage{geometry}
\usepackage{amsfonts}
\usepackage{amstext}
\usepackage{latexsym}
\usepackage{amssymb}
\usepackage{color}


%\include{myPreamble}
\include{qm2pi.local} 

%\ifpdf
%\usepackage[pdftex]{graphicx}
%\else
%\usepackage{graphicx}
%\fi

 % \ifpdf
%  \usepackage{pdfsync}
%  \if


%\title{Brief Article}
%\author{David F. Snyder}
%\author{L.G. Meredith}

%\address{Dept. of Math., Texas State University--San Marcos, San Marcos, TX 78666}
       
\pagestyle{empty}


\begin{document}

\lstset{language=[Objective]Caml,frame=shadowbox}

\input{qm2pi.front}

% section front matter (end)

\input{qm2pi.intro} 
 
% section introduction (end)

% \input{qm2pi.knotations} 

% section notation (end)

\input{qm2pi.process.calculi} 

% section concurrent_process_calculi_and_spatial_logics_ (end)
    
%\input{qm2pi.knots2pi} 

%\input{qm2pi.trefoil} 

%\input{qm2pi.mainthm} 

% subsection basic_interpretation (end)

%\input{qm2pi.rho.presentation} 
\subsection{The syntax and semantics of the notation system}\label{sub:the_syntax_and_semantics_of_the_notation_system} % (fold)

We now summarize a technical presentation of the calculus that
embodies our theory of dynamics. The typical presentation of such a
calculus follows the style of giving generators and relations on
them. The grammar, below, describing term constructors, freely
generates the set of processes, $\Proc$. This set is then quotiented
by a relation known as structural congruence and it is over this set
that the notion of dynamics is expressed. This presentation is
essentially that of \cite{MeredithR05} with the addition of
polyadicity and summation. For readability we have relegated some of
the technical subtleties to an appendix.

\subsubsection{Process grammar}\label{subsub:process_grammar}

\begin{mathpar}
  \inferrule* [lab=synchronization] {} {{M} \bc \pzero \;|\; x?F \;|\; x!C }
  \and
  \inferrule* [lab=abstraction] {} {{F} \bc (x)P}
  \and
  \inferrule* [lab=concretion] {} {{C} \bc \langle Q \rangle}
  \and
  \inferrule* [lab=process] {} {{P,Q} \bc M \;| \;P|Q \;|\; @{x}}
  \and
  \inferrule* [lab=name] {} {{x} \bc \quotep{P}}
\end{mathpar} 

Note that $\vec{x}$ (resp. $\vec{P}$) denotes a vector of names
(resp. processes) of length $|\vec{x}|$ (resp. $|\vec{P}|$). We adopt
the following useful abbreviations.

\begin{mathpar}
   x?(\vec{y}).P := x.(\vec{y})P \and  x\clift{\vec{P}} := x.\clift{\vec{P}}
   \and x!(y) := \lift{x}{\dropn{y}}
   \and \Pi_{i=0}^{n-1}P_i := P_0 | \ldots | P_{n-1}
\end{mathpar}

\subsubsection{Structural congruence}

\paragraph{Free and bound names and alpha-equivalence.} At the
core of structural equivalence is alpha-equivalence which identifies
process that are the same up to a change of variable. Formally, we
recognize the distinction between free and bound names. The free names
of a process, $\freenames{P}$, may be calculated recursively as
follows:

\begin{mathpar}
\freenames{\pzero} := \emptyset
  \and \\
  \freenames{x?(y).P} := \{ x \} \cup (\freenames{P} \setminus \{ y \})
  \and 
  \freenames{x!\langle P \rangle} := \{ x \} \cup \{ P \} 
  \and \\
  \freenames{P|Q} := \freenames{P} \cup \freenames{Q}
  \and \\
  \freenames{@{x}} := \{ x \}
\end{mathpar}

$\pi$
$\quotep{\pi}$

$\freenames{-} : \pi \to \mathcal{P}(\quotep{\pi})$

\begin{eqnarray*}
  \freenames{\pzero} & := & \emptyset \\
  \freenames{x?(y).P} & := & \{ x \} \cup (\freenames{P} \setminus \{ y \}) \\
  \freenames{x!\langle P \rangle} & := & \{ x \} \cup \{ P \} \\
  \freenames{P|Q} & := & \freenames{P} \cup \freenames{Q} \\
  \freenames{\dropn{x}} & := & \{ x \}
\end{eqnarray*}

The bound names of a process, $\boundnames{P}$, are those names occurring in $P$
that are not free. For example, in $x?(y).0$, the name $x$ is free, while $y$ is bound.

\begin{mathpar}
  \inferrule* [lab=monoidal-laws] {} { P|Q \equiv Q|P \and P|0 \equiv P \and P|(Q|R) \equiv (P|Q)|R }
\end{mathpar}

\begin{mathpar}
  \inferrule* [lab=alpha-equivalence] {} { (x)P \equiv (y)P\{y/x\} \and y \not\in \freenames{P} }
\end{mathpar}

\begin{definition}
Then two processes, $P,Q$, are alpha-equivalent if $P = Q\{\vec{y}/\vec{x}\}$ for
some $\vec{x} \in \boundnames{Q},\vec{y} \in \boundnames{P}$, where $Q\{\vec{y}/\vec{x}\}$
denotes the capture-avoiding substitution of $\vec{y}$ for $\vec{x}$ in $Q$.
\end{definition}

\begin{definition}
  The {\em structural congruence} \cite{SangiorgiWalker} , $\equiv$,
  between processes is the least congruence containing
  alpha-equivalence, satisfying the abelian monoid laws
  (associativity, commutativity and $\pzero$ as identity) for parallel
  composition $|$ and for summation $+$.
\end{definition}

\subsection{Name equivalence}

We take name equivalence, written $\nameeq$, to be the smallest
equivalence relation generated by the following rules.

\begin{mathpar}
\inferrule*[lab=Quote-drop]
{ }
{ \quotep{@{x}} \nameeq x }

\inferrule*[lab=Struct-equiv]
{ P \scong Q }
{ \quotep{P} \nameeq \quotep{Q} }
\end{mathpar}

The astute reader will have noticed that the mutual recursion of names
and processes imposes a mutual recursion on alpha-equivalence and
structural equivalence via name-equivalence. Fortunately, all of this
works out pleasantly and we may calculate in the natural way, free of
concern. The reader interested in the details is referred to the
appendix \ref{appendix:rho_details}.

\subsection{Substitution}

We use $\Proc$ for the set of processes, $\QProc$ for the set of
names, and $\id{\{}\vec{y} / \vec{x} \id{\}}$ to denote partial maps,
$s : \QProc \rightarrow \QProc$. A map, $s$ lifts, uniquely, to a map
on process terms, $\widehat{s} : \Proc \rightarrow \Proc$ by the
following equations.

\begin{mathpar}
  (0) \psubstp{Q}{P} := 0 \\
  (R \juxtap S) \psubstp{Q}{P}
  :=    
  (R)\psubstp{Q}{P} \juxtap (S) \psubstp{Q}{P} \\
  (x?(y).R) \psubstp{Q}{P}    
  :=    
  (x)\substp{Q}{P} (z)\concat( (R \psubstn{z}{y}) \psubstp{Q}{P} ) \\
  (\lift{x}{R}) \psubstp{Q}{P}  
  :=
  \lift{(x)\substp{Q}{P}}{ R \psubstp{Q}{P} } \\
%   (\dropn{x})  \psubstp{Q}{P}       
%   := 
%   \left\{ 
%     \begin{array}{ccc} 
%       \dropn{\quotep{Q}} & & x \nameeq \quotep{P} \\
%       \dropn{x} & & otherwise \\
%     \end{array}
%   \right. 
  (\dropn{x})  \psubstp{Q}{P}       
  := 
  \left\{ 
    \begin{array}{ccc} 
      Q & & x \nameeq \quotep{P} \\
      \dropn{x} & & otherwise \\
    \end{array}
  \right.
\end{mathpar}
 

where

\begin{eqnarray}
  (x)\id{\{} \lpquote Q \rpquote / \lpquote P \rpquote \id{\}}            = 
  \left\{ 
    \begin{array}{ccc}
      \lpquote Q \rpquote & & x \nameeq \lpquote P \rpquote \\
      x & & otherwise \\
    \end{array}
  \right. \nonumber
\end{eqnarray}

and $z$ is chosen distinct from $\quotep{P}$, $\quotep{Q}$, the free
names in $Q$, and all the names in $R$. Our $\alpha$-equivalence will
be built in the standard way from this substitution.

\begin{remark}\label{rem:no_self_referential_names}
  One consequence of these definitions is that $\forall P. \quotep{P}
  \not\in \freenames{P}$.
\end{remark}

\subsection{ Dynamic quote: an example }

Anticipating something of what's to come, consider applying the
substitution, $\widehat{\id{\{}u / z \id{\}}}$, to the following pair
of processes, $\lift{w}{y!(z)}$ and $w[ \lpquote y!(z) \rpquote ]$.

\begin{eqnarray}
	\lift{w}{y!(z)}\widehat{\id{\{}u / z \id{\}}}
		& = &
		\lift{w}{y!(u)} \nonumber\\
	w[ \lpquote y!(z) \rpquote ] \widehat{ \id{\{}u / z \id{\}} }
		& = &
		w[ \lpquote y!(z) \rpquote ] \nonumber
\end{eqnarray}

Because the body of the process between quotes is impervious to
substitution, we get radically different answers. In fact, by
examining the first process in an input context,
e.g. $x?(z).\lift{w}{y!(z)}$, we see that the process under the lift
operator may be shaped by prefixed inputs binding a name inside it. In
this sense, the lift operator will be seen as a way to dynamically
construct processes before reifying them as names.

Finally equipped with these standard features we can present the
dynamics of the calculus.

\subsubsection{Operational semantics} 

Finally, we introduce the computational dynamics. What marks these
algebras as distinct from other more traditionally studied algebraic
structures, e.g. vector spaces or polynomial rings, is the manner in
which dynamics is captured. In traditional structures, dynamics is typically
expressed through morphisms between such structures, as in linear maps
between vector spaces or morphisms between rings. In algebras
associated with the semantics of computation, the dynamics is
expressed as part of the algebraic structure itself, through a
reduction reduction relation typically denoted by $\red$. Below, we
give a recursive presentation of this relation for the calculus used
in the encoding.

$\red \subseteq \pi \times \pi$
$\red : \pi \to \mathcal{P}(\pi)$

\begin{mathpar}
  \inferrule* [lab=Comm] { \textsf{match}( x_{src}, x_{trgt} ) } { x_{trgt}?(y)P \; | \; x_{src}!\langle {Q} \rangle \red P\{\quotep{Q}/y}\} }
  \and \\
  \inferrule* [lab=Par] {{P} \red {P}'} {{{P} | {Q}} \red {{P}' | {Q}}}
  \and
  \inferrule* [lab=Equiv]{{{P} \scong {P}'} \andalso {{P}' \red {Q}'} \andalso {{Q}' \scong {Q}}}{{P} \red {Q}}
\end{mathpar}

\begin{eqnarray*}
  match_{\equiv} (\quotep{P},\quotep{Q}) & := & P \equiv Q \\
  match_{\dagger}(\quotep{P},\quotep{Q}) & := & \forall R. P|Q \red^{*} R => R \red^{*} 0 \\
  match_{K}(\quotep{P},\quotep{Q}) & := & K \mbox{ for some context } K
\end{eqnarray*}

$u?(x)P | u!\langle Q \rangle \red P\{\quotep{Q}/x\}$

%We write $\wred$ for $\red^*$, and $P\red$ if $\exists Q $ such that $ P \red Q$.
We write $P\red$ if $\exists Q $ such that $ P \red Q$ and $P\not\red$, otherwise.

\section{Replication}

As mentioned before, it is known that replication (and hence
recursion) can be implemented in a higher-order process algebra
\cite{SangiorgiWalker}. As our first example of calculation with the
machinery thus far presented we give the construction explicitly in
the {\rhoc}.

\begin{eqnarray}
	D_{x} & := & \prefix{x}{y}{(\binpar{\outputp{x}{y}}{@{y}})} \nonumber\\
	\bangp_{x}{P} & := & \binpar{{x}!\langle{\binpar{D_{x}}{P}}\rangle}{D_{x}} \nonumber
\end{eqnarray}

\begin{eqnarray}
	\bangp_{x}{P} & & \nonumber\\
	=
	& {x}!\langle{(\prefix{x}{y}{(\outputp{x}{y} | @{y})) | P}}\rangle 
	      | \prefix{x}{y}{(\outputp{x}{y} | @{y})} & \nonumber\\
	\red
	& (\outputp{x}{y} | @{y})\substn{\quotep{(\prefix{x}{y}{(@{y} | \outputp{x}{y})) | P}}}{y} & \nonumber\\
	=
	& \outputp{x}{\quotep{(\prefix{x}{y}{(\outputp{x}{y} | @{y})) | P}}}
	  | {(\prefix{x}{y}{(\outputp{x}{y} | @{y})) | P}} & \nonumber\\
	\red
	& \ldots & \nonumber\\
	\red^*
	& P | P | \ldots & \nonumber
\end{eqnarray}

Of course, this encoding, as an implementation, runs away, unfolding
$\bangp{P}$ eagerly. A lazier and more implementable replication
operator, restricted to input-guarded processes, may be obtained as follows.

\begin{eqnarray}
\bangp{\prefix{u}{v}{P}} 
	:= 
	\binpar{\lift{x}{\prefix{u}{v}{(\binpar{D(x)}{P})}}}{D(x)} \nonumber
\end{eqnarray}

\begin{remark}
  Note that the lazier definition still does not deal with summation
  or mixed summation (i.e. sums over input and output). The reader is
  invited to construct definitions of replication that deal with these
  features. 

  Further, the definitions are parameterized in a name, $x$. Can you,
  gentle reader, make a definition that eliminates this parameter and
  guarantees no accidental interaction between the replication
  machinery and the process being replicated -- i.e. no accidental
  sharing of names used by the process to get its work done and the
  name(s) used by the replication to effect copying. This latter
  revision of the definition of replication is crucial to obtaining
  the expected identity $!!P \sim !P$.
\end{remark}

\begin{remark}\label{rem:paradoxical_combinator}
  The reader familiar with the lambda calculus will have noticed the
  similarity between $D$ and the paradoxical combinator.

  [Ed. note: the existence of this seems to suggest we have to be more
  restrictive on the set of processes and names we admit if we are to
  support no-cloning.]
\end{remark}

\subsubsection{Bisimulation}

The computational dynamics gives rise to another kind of equivalence,
the equivalence of computational behavior. As previously mentioned
this is typically captured \emph{via} some form of bisimulation.

% The notion we use in this paper is weak barbed bisimulation
% \cite{milner91polyadicpi}.

The notion we use in this paper is derived from weak barbed
bisimulation \cite{milner91polyadicpi}. 

\begin{definition}
An \emph{observation relation}, $\downarrow_{\mathcal N}$, over a set
of names, $\mathcal N$, is the smallest relation satisfying the rules
below.

\infrule[Out-barb]{y \in {\mathcal N}, \; x \nameeq y}
		  {\outputp{x}{v} \downarrow_{\mathcal N} x}
\infrule[Par-barb]{\mbox{$P\downarrow_{\mathcal N} x$ or $Q\downarrow_{\mathcal N} x$}}
		  {\binpar{P}{Q} \downarrow_{\mathcal N} x}

We write $P \Downarrow_{\mathcal N} x$ if there is $Q$ such that 
$P \wred Q$ and $Q \downarrow_{\mathcal N} x$.
\end{definition}

\begin{definition}
%\label{def.bbisim}
An  ${\mathcal N}$-\emph{barbed bisimulation} over a set of names, ${\mathcal N}$, is a symmetric binary relation 
${\mathcal S}_{\mathcal N}$ between agents such that $P\rel{S}_{\mathcal N}Q$ implies:
\begin{enumerate}
\item If $P \red P'$ then $Q \wred Q'$ and $P'\rel{S}_{\mathcal N} Q'$.
\item If $P\downarrow_{\mathcal N} x$, then $Q\Downarrow_{\mathcal N} x$.
\end{enumerate}
$P$ is ${\mathcal N}$-barbed bisimilar to $Q$, written
$P \wbbisim_{\mathcal N} Q$, if $P \rel{S}_{\mathcal N} Q$ for some ${\mathcal N}$-barbed bisimulation ${\mathcal S}_{\mathcal N}$.
\end{definition}

$\mathcal{R} \subseteq \pi \times \pi$

$P \mathcal{R} Q => \forall P'. P \red P' \Rightarrow \exists Q'. Q \red Q', P' \mathcal{R} Q'$

$P \vdash x \Rightarrow Q \vdash x$

\begin{mathpar}
  \inferrule*[lab=Out-barb]{x \nameeq y}{{y}!\langle{Q}\rangle \vdash x}
  \and
  \inferrule*[lab=Par-barb]{\mbox{$P\vdash x$ or $Q\vdash x$}}{\binpar{P}{Q} \vdash x}
\end{mathpar}

\subsubsection{Contexts}

One of the principle advantages of computational calculi like the
$\pi$-calculus is a well-defined notion of context,
contextual-equivalence and a correlation between
contextual-equivalence and notions of bisimulation. The notion of
context allows the decomposition of a process into (sub-)process and
its syntactic environment, its context. Thus, a context may be
thought of as a process with a ``hole'' (written $\Box$) in it. The
application of a context $M$ to a process $P$, written $M[P]$, is
tantamount to filling the hole in $M$ with $P$. In this paper we do
not need the full weight of this theory, but do make use of the notion
of context in the proof the main theorem. 

\begin{mathpar}
  \inferrule* [lab=summation] {} {{M_{M},M_{N}} \bc \Box \;|\; x.M_{A} \;|\; M_{M}+M_{N}}
  \and
  \inferrule* [lab=agent] {} {{M_{A}} \bc (\vec{x})M_{P} \;| \; \clift{P_0,\ldots,M_{P},\ldots,P_N}}
  \and \\
  \inferrule* [lab=process] {} {{M_{P}} \bc M_{N} \;| \;P|M_{P} }
\end{mathpar} 

\begin{mathpar}
  \inferrule* [lab=sychronization] {} {M_{N} \bc \Box \;|\; x?M_{F} \;|\; x!M_{C}}
  \and
  \inferrule* [lab=abstraction] {} {{M_{F}} \bc (x)M_{P} }
  \and
  \inferrule* [lab=concretion] {} {{M_{C}} \bc \langle M_{P} \rangle }
  \and \\
  \inferrule* [lab=process] {} {{M_{P}} \bc M_{N} \;| \;P|M_{P} }
\end{mathpar}

\begin{definition}[contextual application] Given a context $M$, and
  process $P$, we define the \emph{contextual application}, $M[P] :=
  M\{P/\Box\}$. That is, the contextual application of M to P is the
  substitution of $P$ for $\Box$ in $M$.
\end{definition}

$\meaningof{-} : L \to \mathcal{P}(\pi)$

\begin{mathpar}
  \inferrule* [lab=collection] {} {\meaningof{true} = \pi, \and \meaningof{~E} = \pi \setminus \meaningof{E}, \and \meaningof{E_{1} \& E_{2}} = \meaningof{E_{1}} \cap \meaningof{E_{2}}}
\end{mathpar}

\begin{mathpar}
  \inferrule* [lab=structure] {} {\meaningof{0} = \{ P \in \pi | P \equiv 0 \}, \and \\ \meaningof{E_1 | E_2} = \{ P \in \pi | P \equiv P_{1} | P_{2}, P_{1} \in \meaningof{E_{1}}, P_{2} \in \meaningof{E_2}\} }
\end{mathpar}

\begin{mathpar}
 \inferrule* [lab=behavior] {} {\meaningof{\langle a?b \rangle E} = \{ P \in \pi | P \equiv Q | u?(y)P', \\ \and \\\\ \and \\ \;\;\; u \in \meaningof{a}, \forall z.P'\{z/y\} \in \meaningof{E\{z/b\}}\}, \and \\ \meaningof{a!E} = \{ P \in \pi | P \equiv Q | x!\langle P' \rangle, x \in \meaningof{a} P' \in \meaningof{E}\} }
\end{mathpar}

\begin{mathpar}
 \inferrule* [lab=nominal] {} {\meaningof{\quotep{E}} = \{ \quotep{P} \in \quotep{\pi} | P \in \meaningof{E} \}, \and \meaningof{\quotep{P}} = \{ \quotep{Q} \in \quotep{\pi} | P \equiv Q \} \and \\ \meaningof{@\quotep{E}} = \{ P \in \pi | P \equiv @x, x \in \meaningof{E} \}}
\end{mathpar}

\begin{eqnarray*}
  \\
  \meaningof{-} : TS \to ST
\end{eqnarray*}

\begin{eqnarray*}
  \\
  L : TS \to ST
\end{eqnarray*}

\begin{eqnarray*}
  \\
  P \models E \iff P \in \meaningof{E}
\end{eqnarray*}

\begin{eqnarray*}
  P \approx_{L} Q \iff \forall E \in L. P \models E \iff Q \models E
\end{eqnarray*}

\begin{eqnarray*}
  P \approx_{K} Q
\end{eqnarray*}

\begin{eqnarray*}
  P \approx Q
\end{eqnarray*}

$\approx_{K} = \approx = \approx_{L}$

\subsubsection{Contextual duality}

Note that contexts extend the quotation operation to a family of
operations from processes to names. Given a context, $M$, we can
define a \emph{nominal context}, $\quotep{M}$ by $\quotep{M}[P] :=
\quotep{M[P]}$. To foreshadow what is to come we observe that these
operations enjoy a duality with processes very much like the duality
between vectors and maps from vectors to scalars.

Further, because the calculus is essentially higher-order, we have a
correspondence between contexts and processes. More specifically,
given a name $x$ and a context $M$ we can construct $M^{*}_{x}$ such
that 

\begin{mathpar}
  M^{*}_{x} | \lift{x}{P} \red M[P]
\end{mathpar}

namely,

\begin{mathpar}
  M^{*}_{x} := x?(u).M[\dropn{u}]
\end{mathpar}

The dependence of $M^{*}_{x}$ on a name makes it an abstraction, 

\begin{mathpar}
  M^{*} := (x)x?(u).M[\dropn{u}]
\end{mathpar}

\subsection{Additional notation}

It will sometimes be convenient to denote the process a name
quotes. We already have the notation $x = \quotep{P}$, but it will be
convenient to introduce an alternate notation, $\procn{x}$, when we
want to emphasize the connection to the use of the name. Note that, by
virtue of name equivalence, $\quotep{\procn{x}} \nameeq x$; so, the
notation is consistent with previous definitions.

Further, because names have structure it is possible to effect
substitutions on the basis of that structure. This means we need to
upgrade our notation for substitutions, which we accomplish by
adapting comprehension notation. Thus,

\begin{mathpar}
  P\{ y / x : x \in S \}
\end{mathpar}

is interpreted to mean the process derived from P by replacing (in a
capture-avoiding manner) each occurrence of $x$ in $S$ by $y$. For example,

\begin{mathpar}
  P\{ \quotep{\procn{x}|\procn{x}} / x : x \in \freenames{P} \}
\end{mathpar}

will replace each (occurrence) of a free name $x$ in $P$ by
$\quotep{\procn{x}|\procn{x}}$.

Also, we will avail ourselves of the notation $x^{L}$ and $x^{R}$ to
denote injections of a name into disjoint copies of the name
space. There are numerous ways to accomplish this. One example can be
found in \cite{MeredithR05}. This notation overloads to vectors of
names: $\vec{x}^{\pi} := (x_{i}^{\pi} \; : \; 0 \leq i < |\vec{x}| )$ where $\pi \in \{L,R\}$.

We also use $P^{\Box} := P|\Box$.

In \cite{MeredithR05} an interpretation of the new operator is
given. It turns out that there are several possible interpretations
all enjoying the requisite algebraic properties of the operator (see
\cite{milner91polyadicpi}). We will therefore make liberal use of
$(\nu\; \vec{x})P$.

% subsection the_syntax_and_semantics_of_the_notation_system (end)   

\input{qm2pi.qmops} 

\input{qm2pi.sterngerlach} 

\input{qm2pi.metric} 

% section concurrent_process_calculi (end)

%\input{qm2pi.proofsketch}

% section proof sketch (end)

%\input{qm2pi.slviaknots} 

% section spatial logic via knots (end)

\input{qm2pi.conclusion}

% section conclusion (end)

%\input{qm2pi.dtcodes} 

% section wiring algorithm (end)

\input{qm2pi.ack} 

% section acknowledgments (end)

\newpage


\bibliographystyle{plain}   
\bibliography{../../biblios/main.bib}

\input{qm2pi.rhodetails}

\end{document}

 

%\documentclass[12pt]{llncs}
%\documentclass{jktr}

\usepackage[pdftex]{hyperref}                   
\usepackage {listings}
\usepackage {mathpartir}
\usepackage{bcprules}
%\usepackage{listings}
                       
\usepackage{graphicx} 
%\usepackage[margins=2.5cm,nohead,nofoot]{geometry}
%\usepackage{geometry}
\usepackage{amsfonts}
\usepackage{amstext}
\usepackage{latexsym}
\usepackage{amssymb}
\usepackage{color}


%\include{myPreamble}
\include{qm2pi.local} 

%\ifpdf
%\usepackage[pdftex]{graphicx}
%\else
%\usepackage{graphicx}
%\fi

 % \ifpdf
%  \usepackage{pdfsync}
%  \if


%\title{Brief Article}
%\author{David F. Snyder}
%\author{L.G. Meredith}

%\address{Dept. of Math., Texas State University--San Marcos, San Marcos, TX 78666}
       
\pagestyle{empty}


\begin{document}

\lstset{language=[Objective]Caml,frame=shadowbox}

\input{qm2pi.front}

% section front matter (end)

\input{qm2pi.intro} 
 
% section introduction (end)

% \input{qm2pi.knotations} 

% section notation (end)

\input{qm2pi.process.calculi} 

% section concurrent_process_calculi_and_spatial_logics_ (end)
    
%\input{qm2pi.knots2pi} 

%\input{qm2pi.trefoil} 

%\input{qm2pi.mainthm} 

% subsection basic_interpretation (end)

%\input{qm2pi.rho.presentation} 
\subsection{The syntax and semantics of the notation system}\label{sub:the_syntax_and_semantics_of_the_notation_system} % (fold)

We now summarize a technical presentation of the calculus that
embodies our theory of dynamics. The typical presentation of such a
calculus follows the style of giving generators and relations on
them. The grammar, below, describing term constructors, freely
generates the set of processes, $\Proc$. This set is then quotiented
by a relation known as structural congruence and it is over this set
that the notion of dynamics is expressed. This presentation is
essentially that of \cite{MeredithR05} with the addition of
polyadicity and summation. For readability we have relegated some of
the technical subtleties to an appendix.

\subsubsection{Process grammar}\label{subsub:process_grammar}

\begin{mathpar}
  \inferrule* [lab=synchronization] {} {{M} \bc \pzero \;|\; x?F \;|\; x!C }
  \and
  \inferrule* [lab=abstraction] {} {{F} \bc (x)P}
  \and
  \inferrule* [lab=concretion] {} {{C} \bc \langle Q \rangle}
  \and
  \inferrule* [lab=process] {} {{P,Q} \bc M \;| \;P|Q \;|\; @{x}}
  \and
  \inferrule* [lab=name] {} {{x} \bc \quotep{P}}
\end{mathpar} 

Note that $\vec{x}$ (resp. $\vec{P}$) denotes a vector of names
(resp. processes) of length $|\vec{x}|$ (resp. $|\vec{P}|$). We adopt
the following useful abbreviations.

\begin{mathpar}
   x?(\vec{y}).P := x.(\vec{y})P \and  x\clift{\vec{P}} := x.\clift{\vec{P}}
   \and x!(y) := \lift{x}{\dropn{y}}
   \and \Pi_{i=0}^{n-1}P_i := P_0 | \ldots | P_{n-1}
\end{mathpar}

\subsubsection{Structural congruence}

\paragraph{Free and bound names and alpha-equivalence.} At the
core of structural equivalence is alpha-equivalence which identifies
process that are the same up to a change of variable. Formally, we
recognize the distinction between free and bound names. The free names
of a process, $\freenames{P}$, may be calculated recursively as
follows:

\begin{mathpar}
\freenames{\pzero} := \emptyset
  \and \\
  \freenames{x?(y).P} := \{ x \} \cup (\freenames{P} \setminus \{ y \})
  \and 
  \freenames{x!\langle P \rangle} := \{ x \} \cup \{ P \} 
  \and \\
  \freenames{P|Q} := \freenames{P} \cup \freenames{Q}
  \and \\
  \freenames{@{x}} := \{ x \}
\end{mathpar}

$\pi$
$\quotep{\pi}$

$\freenames{-} : \pi \to \mathcal{P}(\quotep{\pi})$

\begin{eqnarray*}
  \freenames{\pzero} & := & \emptyset \\
  \freenames{x?(y).P} & := & \{ x \} \cup (\freenames{P} \setminus \{ y \}) \\
  \freenames{x!\langle P \rangle} & := & \{ x \} \cup \{ P \} \\
  \freenames{P|Q} & := & \freenames{P} \cup \freenames{Q} \\
  \freenames{\dropn{x}} & := & \{ x \}
\end{eqnarray*}

The bound names of a process, $\boundnames{P}$, are those names occurring in $P$
that are not free. For example, in $x?(y).0$, the name $x$ is free, while $y$ is bound.

\begin{mathpar}
  \inferrule* [lab=monoidal-laws] {} { P|Q \equiv Q|P \and P|0 \equiv P \and P|(Q|R) \equiv (P|Q)|R }
\end{mathpar}

\begin{mathpar}
  \inferrule* [lab=alpha-equivalence] {} { (x)P \equiv (y)P\{y/x\} \and y \not\in \freenames{P} }
\end{mathpar}

\begin{definition}
Then two processes, $P,Q$, are alpha-equivalent if $P = Q\{\vec{y}/\vec{x}\}$ for
some $\vec{x} \in \boundnames{Q},\vec{y} \in \boundnames{P}$, where $Q\{\vec{y}/\vec{x}\}$
denotes the capture-avoiding substitution of $\vec{y}$ for $\vec{x}$ in $Q$.
\end{definition}

\begin{definition}
  The {\em structural congruence} \cite{SangiorgiWalker} , $\equiv$,
  between processes is the least congruence containing
  alpha-equivalence, satisfying the abelian monoid laws
  (associativity, commutativity and $\pzero$ as identity) for parallel
  composition $|$ and for summation $+$.
\end{definition}

\subsection{Name equivalence}

We take name equivalence, written $\nameeq$, to be the smallest
equivalence relation generated by the following rules.

\begin{mathpar}
\inferrule*[lab=Quote-drop]
{ }
{ \quotep{@{x}} \nameeq x }

\inferrule*[lab=Struct-equiv]
{ P \scong Q }
{ \quotep{P} \nameeq \quotep{Q} }
\end{mathpar}

The astute reader will have noticed that the mutual recursion of names
and processes imposes a mutual recursion on alpha-equivalence and
structural equivalence via name-equivalence. Fortunately, all of this
works out pleasantly and we may calculate in the natural way, free of
concern. The reader interested in the details is referred to the
appendix \ref{appendix:rho_details}.

\subsection{Substitution}

We use $\Proc$ for the set of processes, $\QProc$ for the set of
names, and $\id{\{}\vec{y} / \vec{x} \id{\}}$ to denote partial maps,
$s : \QProc \rightarrow \QProc$. A map, $s$ lifts, uniquely, to a map
on process terms, $\widehat{s} : \Proc \rightarrow \Proc$ by the
following equations.

\begin{mathpar}
  (0) \psubstp{Q}{P} := 0 \\
  (R \juxtap S) \psubstp{Q}{P}
  :=    
  (R)\psubstp{Q}{P} \juxtap (S) \psubstp{Q}{P} \\
  (x?(y).R) \psubstp{Q}{P}    
  :=    
  (x)\substp{Q}{P} (z)\concat( (R \psubstn{z}{y}) \psubstp{Q}{P} ) \\
  (\lift{x}{R}) \psubstp{Q}{P}  
  :=
  \lift{(x)\substp{Q}{P}}{ R \psubstp{Q}{P} } \\
%   (\dropn{x})  \psubstp{Q}{P}       
%   := 
%   \left\{ 
%     \begin{array}{ccc} 
%       \dropn{\quotep{Q}} & & x \nameeq \quotep{P} \\
%       \dropn{x} & & otherwise \\
%     \end{array}
%   \right. 
  (\dropn{x})  \psubstp{Q}{P}       
  := 
  \left\{ 
    \begin{array}{ccc} 
      Q & & x \nameeq \quotep{P} \\
      \dropn{x} & & otherwise \\
    \end{array}
  \right.
\end{mathpar}
 

where

\begin{eqnarray}
  (x)\id{\{} \lpquote Q \rpquote / \lpquote P \rpquote \id{\}}            = 
  \left\{ 
    \begin{array}{ccc}
      \lpquote Q \rpquote & & x \nameeq \lpquote P \rpquote \\
      x & & otherwise \\
    \end{array}
  \right. \nonumber
\end{eqnarray}

and $z$ is chosen distinct from $\quotep{P}$, $\quotep{Q}$, the free
names in $Q$, and all the names in $R$. Our $\alpha$-equivalence will
be built in the standard way from this substitution.

\begin{remark}\label{rem:no_self_referential_names}
  One consequence of these definitions is that $\forall P. \quotep{P}
  \not\in \freenames{P}$.
\end{remark}

\subsection{ Dynamic quote: an example }

Anticipating something of what's to come, consider applying the
substitution, $\widehat{\id{\{}u / z \id{\}}}$, to the following pair
of processes, $\lift{w}{y!(z)}$ and $w[ \lpquote y!(z) \rpquote ]$.

\begin{eqnarray}
	\lift{w}{y!(z)}\widehat{\id{\{}u / z \id{\}}}
		& = &
		\lift{w}{y!(u)} \nonumber\\
	w[ \lpquote y!(z) \rpquote ] \widehat{ \id{\{}u / z \id{\}} }
		& = &
		w[ \lpquote y!(z) \rpquote ] \nonumber
\end{eqnarray}

Because the body of the process between quotes is impervious to
substitution, we get radically different answers. In fact, by
examining the first process in an input context,
e.g. $x?(z).\lift{w}{y!(z)}$, we see that the process under the lift
operator may be shaped by prefixed inputs binding a name inside it. In
this sense, the lift operator will be seen as a way to dynamically
construct processes before reifying them as names.

Finally equipped with these standard features we can present the
dynamics of the calculus.

\subsubsection{Operational semantics} 

Finally, we introduce the computational dynamics. What marks these
algebras as distinct from other more traditionally studied algebraic
structures, e.g. vector spaces or polynomial rings, is the manner in
which dynamics is captured. In traditional structures, dynamics is typically
expressed through morphisms between such structures, as in linear maps
between vector spaces or morphisms between rings. In algebras
associated with the semantics of computation, the dynamics is
expressed as part of the algebraic structure itself, through a
reduction reduction relation typically denoted by $\red$. Below, we
give a recursive presentation of this relation for the calculus used
in the encoding.

$\red \subseteq \pi \times \pi$
$\red : \pi \to \mathcal{P}(\pi)$

\begin{mathpar}
  \inferrule* [lab=Comm] { \textsf{match}( x_{src}, x_{trgt} ) } { x_{trgt}?(y)P \; | \; x_{src}!\langle {Q} \rangle \red P\{\quotep{Q}/y}\} }
  \and \\
  \inferrule* [lab=Par] {{P} \red {P}'} {{{P} | {Q}} \red {{P}' | {Q}}}
  \and
  \inferrule* [lab=Equiv]{{{P} \scong {P}'} \andalso {{P}' \red {Q}'} \andalso {{Q}' \scong {Q}}}{{P} \red {Q}}
\end{mathpar}

\begin{eqnarray*}
  match_{\equiv} (\quotep{P},\quotep{Q}) & := & P \equiv Q \\
  match_{\dagger}(\quotep{P},\quotep{Q}) & := & \forall R. P|Q \red^{*} R => R \red^{*} 0 \\
  match_{K}(\quotep{P},\quotep{Q}) & := & K \mbox{ for some context } K
\end{eqnarray*}

$u?(x)P | u!\langle Q \rangle \red P\{\quotep{Q}/x\}$

%We write $\wred$ for $\red^*$, and $P\red$ if $\exists Q $ such that $ P \red Q$.
We write $P\red$ if $\exists Q $ such that $ P \red Q$ and $P\not\red$, otherwise.

\section{Replication}

As mentioned before, it is known that replication (and hence
recursion) can be implemented in a higher-order process algebra
\cite{SangiorgiWalker}. As our first example of calculation with the
machinery thus far presented we give the construction explicitly in
the {\rhoc}.

\begin{eqnarray}
	D_{x} & := & \prefix{x}{y}{(\binpar{\outputp{x}{y}}{@{y}})} \nonumber\\
	\bangp_{x}{P} & := & \binpar{{x}!\langle{\binpar{D_{x}}{P}}\rangle}{D_{x}} \nonumber
\end{eqnarray}

\begin{eqnarray}
	\bangp_{x}{P} & & \nonumber\\
	=
	& {x}!\langle{(\prefix{x}{y}{(\outputp{x}{y} | @{y})) | P}}\rangle 
	      | \prefix{x}{y}{(\outputp{x}{y} | @{y})} & \nonumber\\
	\red
	& (\outputp{x}{y} | @{y})\substn{\quotep{(\prefix{x}{y}{(@{y} | \outputp{x}{y})) | P}}}{y} & \nonumber\\
	=
	& \outputp{x}{\quotep{(\prefix{x}{y}{(\outputp{x}{y} | @{y})) | P}}}
	  | {(\prefix{x}{y}{(\outputp{x}{y} | @{y})) | P}} & \nonumber\\
	\red
	& \ldots & \nonumber\\
	\red^*
	& P | P | \ldots & \nonumber
\end{eqnarray}

Of course, this encoding, as an implementation, runs away, unfolding
$\bangp{P}$ eagerly. A lazier and more implementable replication
operator, restricted to input-guarded processes, may be obtained as follows.

\begin{eqnarray}
\bangp{\prefix{u}{v}{P}} 
	:= 
	\binpar{\lift{x}{\prefix{u}{v}{(\binpar{D(x)}{P})}}}{D(x)} \nonumber
\end{eqnarray}

\begin{remark}
  Note that the lazier definition still does not deal with summation
  or mixed summation (i.e. sums over input and output). The reader is
  invited to construct definitions of replication that deal with these
  features. 

  Further, the definitions are parameterized in a name, $x$. Can you,
  gentle reader, make a definition that eliminates this parameter and
  guarantees no accidental interaction between the replication
  machinery and the process being replicated -- i.e. no accidental
  sharing of names used by the process to get its work done and the
  name(s) used by the replication to effect copying. This latter
  revision of the definition of replication is crucial to obtaining
  the expected identity $!!P \sim !P$.
\end{remark}

\begin{remark}\label{rem:paradoxical_combinator}
  The reader familiar with the lambda calculus will have noticed the
  similarity between $D$ and the paradoxical combinator.

  [Ed. note: the existence of this seems to suggest we have to be more
  restrictive on the set of processes and names we admit if we are to
  support no-cloning.]
\end{remark}

\subsubsection{Bisimulation}

The computational dynamics gives rise to another kind of equivalence,
the equivalence of computational behavior. As previously mentioned
this is typically captured \emph{via} some form of bisimulation.

% The notion we use in this paper is weak barbed bisimulation
% \cite{milner91polyadicpi}.

The notion we use in this paper is derived from weak barbed
bisimulation \cite{milner91polyadicpi}. 

\begin{definition}
An \emph{observation relation}, $\downarrow_{\mathcal N}$, over a set
of names, $\mathcal N$, is the smallest relation satisfying the rules
below.

\infrule[Out-barb]{y \in {\mathcal N}, \; x \nameeq y}
		  {\outputp{x}{v} \downarrow_{\mathcal N} x}
\infrule[Par-barb]{\mbox{$P\downarrow_{\mathcal N} x$ or $Q\downarrow_{\mathcal N} x$}}
		  {\binpar{P}{Q} \downarrow_{\mathcal N} x}

We write $P \Downarrow_{\mathcal N} x$ if there is $Q$ such that 
$P \wred Q$ and $Q \downarrow_{\mathcal N} x$.
\end{definition}

\begin{definition}
%\label{def.bbisim}
An  ${\mathcal N}$-\emph{barbed bisimulation} over a set of names, ${\mathcal N}$, is a symmetric binary relation 
${\mathcal S}_{\mathcal N}$ between agents such that $P\rel{S}_{\mathcal N}Q$ implies:
\begin{enumerate}
\item If $P \red P'$ then $Q \wred Q'$ and $P'\rel{S}_{\mathcal N} Q'$.
\item If $P\downarrow_{\mathcal N} x$, then $Q\Downarrow_{\mathcal N} x$.
\end{enumerate}
$P$ is ${\mathcal N}$-barbed bisimilar to $Q$, written
$P \wbbisim_{\mathcal N} Q$, if $P \rel{S}_{\mathcal N} Q$ for some ${\mathcal N}$-barbed bisimulation ${\mathcal S}_{\mathcal N}$.
\end{definition}

$\mathcal{R} \subseteq \pi \times \pi$

$P \mathcal{R} Q => \forall P'. P \red P' \Rightarrow \exists Q'. Q \red Q', P' \mathcal{R} Q'$

$P \vdash x \Rightarrow Q \vdash x$

\begin{mathpar}
  \inferrule*[lab=Out-barb]{x \nameeq y}{{y}!\langle{Q}\rangle \vdash x}
  \and
  \inferrule*[lab=Par-barb]{\mbox{$P\vdash x$ or $Q\vdash x$}}{\binpar{P}{Q} \vdash x}
\end{mathpar}

\subsubsection{Contexts}

One of the principle advantages of computational calculi like the
$\pi$-calculus is a well-defined notion of context,
contextual-equivalence and a correlation between
contextual-equivalence and notions of bisimulation. The notion of
context allows the decomposition of a process into (sub-)process and
its syntactic environment, its context. Thus, a context may be
thought of as a process with a ``hole'' (written $\Box$) in it. The
application of a context $M$ to a process $P$, written $M[P]$, is
tantamount to filling the hole in $M$ with $P$. In this paper we do
not need the full weight of this theory, but do make use of the notion
of context in the proof the main theorem. 

\begin{mathpar}
  \inferrule* [lab=summation] {} {{M_{M},M_{N}} \bc \Box \;|\; x.M_{A} \;|\; M_{M}+M_{N}}
  \and
  \inferrule* [lab=agent] {} {{M_{A}} \bc (\vec{x})M_{P} \;| \; \clift{P_0,\ldots,M_{P},\ldots,P_N}}
  \and \\
  \inferrule* [lab=process] {} {{M_{P}} \bc M_{N} \;| \;P|M_{P} }
\end{mathpar} 

\begin{mathpar}
  \inferrule* [lab=sychronization] {} {M_{N} \bc \Box \;|\; x?M_{F} \;|\; x!M_{C}}
  \and
  \inferrule* [lab=abstraction] {} {{M_{F}} \bc (x)M_{P} }
  \and
  \inferrule* [lab=concretion] {} {{M_{C}} \bc \langle M_{P} \rangle }
  \and \\
  \inferrule* [lab=process] {} {{M_{P}} \bc M_{N} \;| \;P|M_{P} }
\end{mathpar}

\begin{definition}[contextual application] Given a context $M$, and
  process $P$, we define the \emph{contextual application}, $M[P] :=
  M\{P/\Box\}$. That is, the contextual application of M to P is the
  substitution of $P$ for $\Box$ in $M$.
\end{definition}

$\meaningof{-} : L \to \mathcal{P}(\pi)$

\begin{mathpar}
  \inferrule* [lab=collection] {} {\meaningof{true} = \pi, \and \meaningof{~E} = \pi \setminus \meaningof{E}, \and \meaningof{E_{1} \& E_{2}} = \meaningof{E_{1}} \cap \meaningof{E_{2}}}
\end{mathpar}

\begin{mathpar}
  \inferrule* [lab=structure] {} {\meaningof{0} = \{ P \in \pi | P \equiv 0 \}, \and \\ \meaningof{E_1 | E_2} = \{ P \in \pi | P \equiv P_{1} | P_{2}, P_{1} \in \meaningof{E_{1}}, P_{2} \in \meaningof{E_2}\} }
\end{mathpar}

\begin{mathpar}
 \inferrule* [lab=behavior] {} {\meaningof{\langle a?b \rangle E} = \{ P \in \pi | P \equiv Q | u?(y)P', \\ \and \\\\ \and \\ \;\;\; u \in \meaningof{a}, \forall z.P'\{z/y\} \in \meaningof{E\{z/b\}}\}, \and \\ \meaningof{a!E} = \{ P \in \pi | P \equiv Q | x!\langle P' \rangle, x \in \meaningof{a} P' \in \meaningof{E}\} }
\end{mathpar}

\begin{mathpar}
 \inferrule* [lab=nominal] {} {\meaningof{\quotep{E}} = \{ \quotep{P} \in \quotep{\pi} | P \in \meaningof{E} \}, \and \meaningof{\quotep{P}} = \{ \quotep{Q} \in \quotep{\pi} | P \equiv Q \} \and \\ \meaningof{@\quotep{E}} = \{ P \in \pi | P \equiv @x, x \in \meaningof{E} \}}
\end{mathpar}

\begin{eqnarray*}
  \\
  \meaningof{-} : TS \to ST
\end{eqnarray*}

\begin{eqnarray*}
  \\
  L : TS \to ST
\end{eqnarray*}

\begin{eqnarray*}
  \\
  P \models E \iff P \in \meaningof{E}
\end{eqnarray*}

\begin{eqnarray*}
  P \approx_{L} Q \iff \forall E \in L. P \models E \iff Q \models E
\end{eqnarray*}

\begin{eqnarray*}
  P \approx_{K} Q
\end{eqnarray*}

\begin{eqnarray*}
  P \approx Q
\end{eqnarray*}

$\approx_{K} = \approx = \approx_{L}$

\subsubsection{Contextual duality}

Note that contexts extend the quotation operation to a family of
operations from processes to names. Given a context, $M$, we can
define a \emph{nominal context}, $\quotep{M}$ by $\quotep{M}[P] :=
\quotep{M[P]}$. To foreshadow what is to come we observe that these
operations enjoy a duality with processes very much like the duality
between vectors and maps from vectors to scalars.

Further, because the calculus is essentially higher-order, we have a
correspondence between contexts and processes. More specifically,
given a name $x$ and a context $M$ we can construct $M^{*}_{x}$ such
that 

\begin{mathpar}
  M^{*}_{x} | \lift{x}{P} \red M[P]
\end{mathpar}

namely,

\begin{mathpar}
  M^{*}_{x} := x?(u).M[\dropn{u}]
\end{mathpar}

The dependence of $M^{*}_{x}$ on a name makes it an abstraction, 

\begin{mathpar}
  M^{*} := (x)x?(u).M[\dropn{u}]
\end{mathpar}

\subsection{Additional notation}

It will sometimes be convenient to denote the process a name
quotes. We already have the notation $x = \quotep{P}$, but it will be
convenient to introduce an alternate notation, $\procn{x}$, when we
want to emphasize the connection to the use of the name. Note that, by
virtue of name equivalence, $\quotep{\procn{x}} \nameeq x$; so, the
notation is consistent with previous definitions.

Further, because names have structure it is possible to effect
substitutions on the basis of that structure. This means we need to
upgrade our notation for substitutions, which we accomplish by
adapting comprehension notation. Thus,

\begin{mathpar}
  P\{ y / x : x \in S \}
\end{mathpar}

is interpreted to mean the process derived from P by replacing (in a
capture-avoiding manner) each occurrence of $x$ in $S$ by $y$. For example,

\begin{mathpar}
  P\{ \quotep{\procn{x}|\procn{x}} / x : x \in \freenames{P} \}
\end{mathpar}

will replace each (occurrence) of a free name $x$ in $P$ by
$\quotep{\procn{x}|\procn{x}}$.

Also, we will avail ourselves of the notation $x^{L}$ and $x^{R}$ to
denote injections of a name into disjoint copies of the name
space. There are numerous ways to accomplish this. One example can be
found in \cite{MeredithR05}. This notation overloads to vectors of
names: $\vec{x}^{\pi} := (x_{i}^{\pi} \; : \; 0 \leq i < |\vec{x}| )$ where $\pi \in \{L,R\}$.

We also use $P^{\Box} := P|\Box$.

In \cite{MeredithR05} an interpretation of the new operator is
given. It turns out that there are several possible interpretations
all enjoying the requisite algebraic properties of the operator (see
\cite{milner91polyadicpi}). We will therefore make liberal use of
$(\nu\; \vec{x})P$.

% subsection the_syntax_and_semantics_of_the_notation_system (end)   

\input{qm2pi.qmops} 

\input{qm2pi.sterngerlach} 

\input{qm2pi.metric} 

% section concurrent_process_calculi (end)

%\input{qm2pi.proofsketch}

% section proof sketch (end)

%\input{qm2pi.slviaknots} 

% section spatial logic via knots (end)

\input{qm2pi.conclusion}

% section conclusion (end)

%\input{qm2pi.dtcodes} 

% section wiring algorithm (end)

\input{qm2pi.ack} 

% section acknowledgments (end)

\newpage


\bibliographystyle{plain}   
\bibliography{../../biblios/main.bib}

\input{qm2pi.rhodetails}

\end{document}

 

%\documentclass[12pt]{llncs}
%\documentclass{jktr}

\usepackage[pdftex]{hyperref}                   
\usepackage {listings}
\usepackage {mathpartir}
\usepackage{bcprules}
%\usepackage{listings}
                       
\usepackage{graphicx} 
%\usepackage[margins=2.5cm,nohead,nofoot]{geometry}
%\usepackage{geometry}
\usepackage{amsfonts}
\usepackage{amstext}
\usepackage{latexsym}
\usepackage{amssymb}
\usepackage{color}


%\include{myPreamble}
\include{qm2pi.local} 

%\ifpdf
%\usepackage[pdftex]{graphicx}
%\else
%\usepackage{graphicx}
%\fi

 % \ifpdf
%  \usepackage{pdfsync}
%  \if


%\title{Brief Article}
%\author{David F. Snyder}
%\author{L.G. Meredith}

%\address{Dept. of Math., Texas State University--San Marcos, San Marcos, TX 78666}
       
\pagestyle{empty}


\begin{document}

\lstset{language=[Objective]Caml,frame=shadowbox}

\input{qm2pi.front}

% section front matter (end)

\input{qm2pi.intro} 
 
% section introduction (end)

% \input{qm2pi.knotations} 

% section notation (end)

\input{qm2pi.process.calculi} 

% section concurrent_process_calculi_and_spatial_logics_ (end)
    
%\input{qm2pi.knots2pi} 

%\input{qm2pi.trefoil} 

%\input{qm2pi.mainthm} 

% subsection basic_interpretation (end)

%\input{qm2pi.rho.presentation} 
\subsection{The syntax and semantics of the notation system}\label{sub:the_syntax_and_semantics_of_the_notation_system} % (fold)

We now summarize a technical presentation of the calculus that
embodies our theory of dynamics. The typical presentation of such a
calculus follows the style of giving generators and relations on
them. The grammar, below, describing term constructors, freely
generates the set of processes, $\Proc$. This set is then quotiented
by a relation known as structural congruence and it is over this set
that the notion of dynamics is expressed. This presentation is
essentially that of \cite{MeredithR05} with the addition of
polyadicity and summation. For readability we have relegated some of
the technical subtleties to an appendix.

\subsubsection{Process grammar}\label{subsub:process_grammar}

\begin{mathpar}
  \inferrule* [lab=synchronization] {} {{M} \bc \pzero \;|\; x?F \;|\; x!C }
  \and
  \inferrule* [lab=abstraction] {} {{F} \bc (x)P}
  \and
  \inferrule* [lab=concretion] {} {{C} \bc \langle Q \rangle}
  \and
  \inferrule* [lab=process] {} {{P,Q} \bc M \;| \;P|Q \;|\; @{x}}
  \and
  \inferrule* [lab=name] {} {{x} \bc \quotep{P}}
\end{mathpar} 

Note that $\vec{x}$ (resp. $\vec{P}$) denotes a vector of names
(resp. processes) of length $|\vec{x}|$ (resp. $|\vec{P}|$). We adopt
the following useful abbreviations.

\begin{mathpar}
   x?(\vec{y}).P := x.(\vec{y})P \and  x\clift{\vec{P}} := x.\clift{\vec{P}}
   \and x!(y) := \lift{x}{\dropn{y}}
   \and \Pi_{i=0}^{n-1}P_i := P_0 | \ldots | P_{n-1}
\end{mathpar}

\subsubsection{Structural congruence}

\paragraph{Free and bound names and alpha-equivalence.} At the
core of structural equivalence is alpha-equivalence which identifies
process that are the same up to a change of variable. Formally, we
recognize the distinction between free and bound names. The free names
of a process, $\freenames{P}$, may be calculated recursively as
follows:

\begin{mathpar}
\freenames{\pzero} := \emptyset
  \and \\
  \freenames{x?(y).P} := \{ x \} \cup (\freenames{P} \setminus \{ y \})
  \and 
  \freenames{x!\langle P \rangle} := \{ x \} \cup \{ P \} 
  \and \\
  \freenames{P|Q} := \freenames{P} \cup \freenames{Q}
  \and \\
  \freenames{@{x}} := \{ x \}
\end{mathpar}

$\pi$
$\quotep{\pi}$

$\freenames{-} : \pi \to \mathcal{P}(\quotep{\pi})$

\begin{eqnarray*}
  \freenames{\pzero} & := & \emptyset \\
  \freenames{x?(y).P} & := & \{ x \} \cup (\freenames{P} \setminus \{ y \}) \\
  \freenames{x!\langle P \rangle} & := & \{ x \} \cup \{ P \} \\
  \freenames{P|Q} & := & \freenames{P} \cup \freenames{Q} \\
  \freenames{\dropn{x}} & := & \{ x \}
\end{eqnarray*}

The bound names of a process, $\boundnames{P}$, are those names occurring in $P$
that are not free. For example, in $x?(y).0$, the name $x$ is free, while $y$ is bound.

\begin{mathpar}
  \inferrule* [lab=monoidal-laws] {} { P|Q \equiv Q|P \and P|0 \equiv P \and P|(Q|R) \equiv (P|Q)|R }
\end{mathpar}

\begin{mathpar}
  \inferrule* [lab=alpha-equivalence] {} { (x)P \equiv (y)P\{y/x\} \and y \not\in \freenames{P} }
\end{mathpar}

\begin{definition}
Then two processes, $P,Q$, are alpha-equivalent if $P = Q\{\vec{y}/\vec{x}\}$ for
some $\vec{x} \in \boundnames{Q},\vec{y} \in \boundnames{P}$, where $Q\{\vec{y}/\vec{x}\}$
denotes the capture-avoiding substitution of $\vec{y}$ for $\vec{x}$ in $Q$.
\end{definition}

\begin{definition}
  The {\em structural congruence} \cite{SangiorgiWalker} , $\equiv$,
  between processes is the least congruence containing
  alpha-equivalence, satisfying the abelian monoid laws
  (associativity, commutativity and $\pzero$ as identity) for parallel
  composition $|$ and for summation $+$.
\end{definition}

\subsection{Name equivalence}

We take name equivalence, written $\nameeq$, to be the smallest
equivalence relation generated by the following rules.

\begin{mathpar}
\inferrule*[lab=Quote-drop]
{ }
{ \quotep{@{x}} \nameeq x }

\inferrule*[lab=Struct-equiv]
{ P \scong Q }
{ \quotep{P} \nameeq \quotep{Q} }
\end{mathpar}

The astute reader will have noticed that the mutual recursion of names
and processes imposes a mutual recursion on alpha-equivalence and
structural equivalence via name-equivalence. Fortunately, all of this
works out pleasantly and we may calculate in the natural way, free of
concern. The reader interested in the details is referred to the
appendix \ref{appendix:rho_details}.

\subsection{Substitution}

We use $\Proc$ for the set of processes, $\QProc$ for the set of
names, and $\id{\{}\vec{y} / \vec{x} \id{\}}$ to denote partial maps,
$s : \QProc \rightarrow \QProc$. A map, $s$ lifts, uniquely, to a map
on process terms, $\widehat{s} : \Proc \rightarrow \Proc$ by the
following equations.

\begin{mathpar}
  (0) \psubstp{Q}{P} := 0 \\
  (R \juxtap S) \psubstp{Q}{P}
  :=    
  (R)\psubstp{Q}{P} \juxtap (S) \psubstp{Q}{P} \\
  (x?(y).R) \psubstp{Q}{P}    
  :=    
  (x)\substp{Q}{P} (z)\concat( (R \psubstn{z}{y}) \psubstp{Q}{P} ) \\
  (\lift{x}{R}) \psubstp{Q}{P}  
  :=
  \lift{(x)\substp{Q}{P}}{ R \psubstp{Q}{P} } \\
%   (\dropn{x})  \psubstp{Q}{P}       
%   := 
%   \left\{ 
%     \begin{array}{ccc} 
%       \dropn{\quotep{Q}} & & x \nameeq \quotep{P} \\
%       \dropn{x} & & otherwise \\
%     \end{array}
%   \right. 
  (\dropn{x})  \psubstp{Q}{P}       
  := 
  \left\{ 
    \begin{array}{ccc} 
      Q & & x \nameeq \quotep{P} \\
      \dropn{x} & & otherwise \\
    \end{array}
  \right.
\end{mathpar}
 

where

\begin{eqnarray}
  (x)\id{\{} \lpquote Q \rpquote / \lpquote P \rpquote \id{\}}            = 
  \left\{ 
    \begin{array}{ccc}
      \lpquote Q \rpquote & & x \nameeq \lpquote P \rpquote \\
      x & & otherwise \\
    \end{array}
  \right. \nonumber
\end{eqnarray}

and $z$ is chosen distinct from $\quotep{P}$, $\quotep{Q}$, the free
names in $Q$, and all the names in $R$. Our $\alpha$-equivalence will
be built in the standard way from this substitution.

\begin{remark}\label{rem:no_self_referential_names}
  One consequence of these definitions is that $\forall P. \quotep{P}
  \not\in \freenames{P}$.
\end{remark}

\subsection{ Dynamic quote: an example }

Anticipating something of what's to come, consider applying the
substitution, $\widehat{\id{\{}u / z \id{\}}}$, to the following pair
of processes, $\lift{w}{y!(z)}$ and $w[ \lpquote y!(z) \rpquote ]$.

\begin{eqnarray}
	\lift{w}{y!(z)}\widehat{\id{\{}u / z \id{\}}}
		& = &
		\lift{w}{y!(u)} \nonumber\\
	w[ \lpquote y!(z) \rpquote ] \widehat{ \id{\{}u / z \id{\}} }
		& = &
		w[ \lpquote y!(z) \rpquote ] \nonumber
\end{eqnarray}

Because the body of the process between quotes is impervious to
substitution, we get radically different answers. In fact, by
examining the first process in an input context,
e.g. $x?(z).\lift{w}{y!(z)}$, we see that the process under the lift
operator may be shaped by prefixed inputs binding a name inside it. In
this sense, the lift operator will be seen as a way to dynamically
construct processes before reifying them as names.

Finally equipped with these standard features we can present the
dynamics of the calculus.

\subsubsection{Operational semantics} 

Finally, we introduce the computational dynamics. What marks these
algebras as distinct from other more traditionally studied algebraic
structures, e.g. vector spaces or polynomial rings, is the manner in
which dynamics is captured. In traditional structures, dynamics is typically
expressed through morphisms between such structures, as in linear maps
between vector spaces or morphisms between rings. In algebras
associated with the semantics of computation, the dynamics is
expressed as part of the algebraic structure itself, through a
reduction reduction relation typically denoted by $\red$. Below, we
give a recursive presentation of this relation for the calculus used
in the encoding.

$\red \subseteq \pi \times \pi$
$\red : \pi \to \mathcal{P}(\pi)$

\begin{mathpar}
  \inferrule* [lab=Comm] { \textsf{match}( x_{src}, x_{trgt} ) } { x_{trgt}?(y)P \; | \; x_{src}!\langle {Q} \rangle \red P\{\quotep{Q}/y}\} }
  \and \\
  \inferrule* [lab=Par] {{P} \red {P}'} {{{P} | {Q}} \red {{P}' | {Q}}}
  \and
  \inferrule* [lab=Equiv]{{{P} \scong {P}'} \andalso {{P}' \red {Q}'} \andalso {{Q}' \scong {Q}}}{{P} \red {Q}}
\end{mathpar}

\begin{eqnarray*}
  match_{\equiv} (\quotep{P},\quotep{Q}) & := & P \equiv Q \\
  match_{\dagger}(\quotep{P},\quotep{Q}) & := & \forall R. P|Q \red^{*} R => R \red^{*} 0 \\
  match_{K}(\quotep{P},\quotep{Q}) & := & K \mbox{ for some context } K
\end{eqnarray*}

$u?(x)P | u!\langle Q \rangle \red P\{\quotep{Q}/x\}$

%We write $\wred$ for $\red^*$, and $P\red$ if $\exists Q $ such that $ P \red Q$.
We write $P\red$ if $\exists Q $ such that $ P \red Q$ and $P\not\red$, otherwise.

\section{Replication}

As mentioned before, it is known that replication (and hence
recursion) can be implemented in a higher-order process algebra
\cite{SangiorgiWalker}. As our first example of calculation with the
machinery thus far presented we give the construction explicitly in
the {\rhoc}.

\begin{eqnarray}
	D_{x} & := & \prefix{x}{y}{(\binpar{\outputp{x}{y}}{@{y}})} \nonumber\\
	\bangp_{x}{P} & := & \binpar{{x}!\langle{\binpar{D_{x}}{P}}\rangle}{D_{x}} \nonumber
\end{eqnarray}

\begin{eqnarray}
	\bangp_{x}{P} & & \nonumber\\
	=
	& {x}!\langle{(\prefix{x}{y}{(\outputp{x}{y} | @{y})) | P}}\rangle 
	      | \prefix{x}{y}{(\outputp{x}{y} | @{y})} & \nonumber\\
	\red
	& (\outputp{x}{y} | @{y})\substn{\quotep{(\prefix{x}{y}{(@{y} | \outputp{x}{y})) | P}}}{y} & \nonumber\\
	=
	& \outputp{x}{\quotep{(\prefix{x}{y}{(\outputp{x}{y} | @{y})) | P}}}
	  | {(\prefix{x}{y}{(\outputp{x}{y} | @{y})) | P}} & \nonumber\\
	\red
	& \ldots & \nonumber\\
	\red^*
	& P | P | \ldots & \nonumber
\end{eqnarray}

Of course, this encoding, as an implementation, runs away, unfolding
$\bangp{P}$ eagerly. A lazier and more implementable replication
operator, restricted to input-guarded processes, may be obtained as follows.

\begin{eqnarray}
\bangp{\prefix{u}{v}{P}} 
	:= 
	\binpar{\lift{x}{\prefix{u}{v}{(\binpar{D(x)}{P})}}}{D(x)} \nonumber
\end{eqnarray}

\begin{remark}
  Note that the lazier definition still does not deal with summation
  or mixed summation (i.e. sums over input and output). The reader is
  invited to construct definitions of replication that deal with these
  features. 

  Further, the definitions are parameterized in a name, $x$. Can you,
  gentle reader, make a definition that eliminates this parameter and
  guarantees no accidental interaction between the replication
  machinery and the process being replicated -- i.e. no accidental
  sharing of names used by the process to get its work done and the
  name(s) used by the replication to effect copying. This latter
  revision of the definition of replication is crucial to obtaining
  the expected identity $!!P \sim !P$.
\end{remark}

\begin{remark}\label{rem:paradoxical_combinator}
  The reader familiar with the lambda calculus will have noticed the
  similarity between $D$ and the paradoxical combinator.

  [Ed. note: the existence of this seems to suggest we have to be more
  restrictive on the set of processes and names we admit if we are to
  support no-cloning.]
\end{remark}

\subsubsection{Bisimulation}

The computational dynamics gives rise to another kind of equivalence,
the equivalence of computational behavior. As previously mentioned
this is typically captured \emph{via} some form of bisimulation.

% The notion we use in this paper is weak barbed bisimulation
% \cite{milner91polyadicpi}.

The notion we use in this paper is derived from weak barbed
bisimulation \cite{milner91polyadicpi}. 

\begin{definition}
An \emph{observation relation}, $\downarrow_{\mathcal N}$, over a set
of names, $\mathcal N$, is the smallest relation satisfying the rules
below.

\infrule[Out-barb]{y \in {\mathcal N}, \; x \nameeq y}
		  {\outputp{x}{v} \downarrow_{\mathcal N} x}
\infrule[Par-barb]{\mbox{$P\downarrow_{\mathcal N} x$ or $Q\downarrow_{\mathcal N} x$}}
		  {\binpar{P}{Q} \downarrow_{\mathcal N} x}

We write $P \Downarrow_{\mathcal N} x$ if there is $Q$ such that 
$P \wred Q$ and $Q \downarrow_{\mathcal N} x$.
\end{definition}

\begin{definition}
%\label{def.bbisim}
An  ${\mathcal N}$-\emph{barbed bisimulation} over a set of names, ${\mathcal N}$, is a symmetric binary relation 
${\mathcal S}_{\mathcal N}$ between agents such that $P\rel{S}_{\mathcal N}Q$ implies:
\begin{enumerate}
\item If $P \red P'$ then $Q \wred Q'$ and $P'\rel{S}_{\mathcal N} Q'$.
\item If $P\downarrow_{\mathcal N} x$, then $Q\Downarrow_{\mathcal N} x$.
\end{enumerate}
$P$ is ${\mathcal N}$-barbed bisimilar to $Q$, written
$P \wbbisim_{\mathcal N} Q$, if $P \rel{S}_{\mathcal N} Q$ for some ${\mathcal N}$-barbed bisimulation ${\mathcal S}_{\mathcal N}$.
\end{definition}

$\mathcal{R} \subseteq \pi \times \pi$

$P \mathcal{R} Q => \forall P'. P \red P' \Rightarrow \exists Q'. Q \red Q', P' \mathcal{R} Q'$

$P \vdash x \Rightarrow Q \vdash x$

\begin{mathpar}
  \inferrule*[lab=Out-barb]{x \nameeq y}{{y}!\langle{Q}\rangle \vdash x}
  \and
  \inferrule*[lab=Par-barb]{\mbox{$P\vdash x$ or $Q\vdash x$}}{\binpar{P}{Q} \vdash x}
\end{mathpar}

\subsubsection{Contexts}

One of the principle advantages of computational calculi like the
$\pi$-calculus is a well-defined notion of context,
contextual-equivalence and a correlation between
contextual-equivalence and notions of bisimulation. The notion of
context allows the decomposition of a process into (sub-)process and
its syntactic environment, its context. Thus, a context may be
thought of as a process with a ``hole'' (written $\Box$) in it. The
application of a context $M$ to a process $P$, written $M[P]$, is
tantamount to filling the hole in $M$ with $P$. In this paper we do
not need the full weight of this theory, but do make use of the notion
of context in the proof the main theorem. 

\begin{mathpar}
  \inferrule* [lab=summation] {} {{M_{M},M_{N}} \bc \Box \;|\; x.M_{A} \;|\; M_{M}+M_{N}}
  \and
  \inferrule* [lab=agent] {} {{M_{A}} \bc (\vec{x})M_{P} \;| \; \clift{P_0,\ldots,M_{P},\ldots,P_N}}
  \and \\
  \inferrule* [lab=process] {} {{M_{P}} \bc M_{N} \;| \;P|M_{P} }
\end{mathpar} 

\begin{mathpar}
  \inferrule* [lab=sychronization] {} {M_{N} \bc \Box \;|\; x?M_{F} \;|\; x!M_{C}}
  \and
  \inferrule* [lab=abstraction] {} {{M_{F}} \bc (x)M_{P} }
  \and
  \inferrule* [lab=concretion] {} {{M_{C}} \bc \langle M_{P} \rangle }
  \and \\
  \inferrule* [lab=process] {} {{M_{P}} \bc M_{N} \;| \;P|M_{P} }
\end{mathpar}

\begin{definition}[contextual application] Given a context $M$, and
  process $P$, we define the \emph{contextual application}, $M[P] :=
  M\{P/\Box\}$. That is, the contextual application of M to P is the
  substitution of $P$ for $\Box$ in $M$.
\end{definition}

$\meaningof{-} : L \to \mathcal{P}(\pi)$

\begin{mathpar}
  \inferrule* [lab=collection] {} {\meaningof{true} = \pi, \and \meaningof{~E} = \pi \setminus \meaningof{E}, \and \meaningof{E_{1} \& E_{2}} = \meaningof{E_{1}} \cap \meaningof{E_{2}}}
\end{mathpar}

\begin{mathpar}
  \inferrule* [lab=structure] {} {\meaningof{0} = \{ P \in \pi | P \equiv 0 \}, \and \\ \meaningof{E_1 | E_2} = \{ P \in \pi | P \equiv P_{1} | P_{2}, P_{1} \in \meaningof{E_{1}}, P_{2} \in \meaningof{E_2}\} }
\end{mathpar}

\begin{mathpar}
 \inferrule* [lab=behavior] {} {\meaningof{\langle a?b \rangle E} = \{ P \in \pi | P \equiv Q | u?(y)P', \\ \and \\\\ \and \\ \;\;\; u \in \meaningof{a}, \forall z.P'\{z/y\} \in \meaningof{E\{z/b\}}\}, \and \\ \meaningof{a!E} = \{ P \in \pi | P \equiv Q | x!\langle P' \rangle, x \in \meaningof{a} P' \in \meaningof{E}\} }
\end{mathpar}

\begin{mathpar}
 \inferrule* [lab=nominal] {} {\meaningof{\quotep{E}} = \{ \quotep{P} \in \quotep{\pi} | P \in \meaningof{E} \}, \and \meaningof{\quotep{P}} = \{ \quotep{Q} \in \quotep{\pi} | P \equiv Q \} \and \\ \meaningof{@\quotep{E}} = \{ P \in \pi | P \equiv @x, x \in \meaningof{E} \}}
\end{mathpar}

\begin{eqnarray*}
  \\
  \meaningof{-} : TS \to ST
\end{eqnarray*}

\begin{eqnarray*}
  \\
  L : TS \to ST
\end{eqnarray*}

\begin{eqnarray*}
  \\
  P \models E \iff P \in \meaningof{E}
\end{eqnarray*}

\begin{eqnarray*}
  P \approx_{L} Q \iff \forall E \in L. P \models E \iff Q \models E
\end{eqnarray*}

\begin{eqnarray*}
  P \approx_{K} Q
\end{eqnarray*}

\begin{eqnarray*}
  P \approx Q
\end{eqnarray*}

$\approx_{K} = \approx = \approx_{L}$

\subsubsection{Contextual duality}

Note that contexts extend the quotation operation to a family of
operations from processes to names. Given a context, $M$, we can
define a \emph{nominal context}, $\quotep{M}$ by $\quotep{M}[P] :=
\quotep{M[P]}$. To foreshadow what is to come we observe that these
operations enjoy a duality with processes very much like the duality
between vectors and maps from vectors to scalars.

Further, because the calculus is essentially higher-order, we have a
correspondence between contexts and processes. More specifically,
given a name $x$ and a context $M$ we can construct $M^{*}_{x}$ such
that 

\begin{mathpar}
  M^{*}_{x} | \lift{x}{P} \red M[P]
\end{mathpar}

namely,

\begin{mathpar}
  M^{*}_{x} := x?(u).M[\dropn{u}]
\end{mathpar}

The dependence of $M^{*}_{x}$ on a name makes it an abstraction, 

\begin{mathpar}
  M^{*} := (x)x?(u).M[\dropn{u}]
\end{mathpar}

\subsection{Additional notation}

It will sometimes be convenient to denote the process a name
quotes. We already have the notation $x = \quotep{P}$, but it will be
convenient to introduce an alternate notation, $\procn{x}$, when we
want to emphasize the connection to the use of the name. Note that, by
virtue of name equivalence, $\quotep{\procn{x}} \nameeq x$; so, the
notation is consistent with previous definitions.

Further, because names have structure it is possible to effect
substitutions on the basis of that structure. This means we need to
upgrade our notation for substitutions, which we accomplish by
adapting comprehension notation. Thus,

\begin{mathpar}
  P\{ y / x : x \in S \}
\end{mathpar}

is interpreted to mean the process derived from P by replacing (in a
capture-avoiding manner) each occurrence of $x$ in $S$ by $y$. For example,

\begin{mathpar}
  P\{ \quotep{\procn{x}|\procn{x}} / x : x \in \freenames{P} \}
\end{mathpar}

will replace each (occurrence) of a free name $x$ in $P$ by
$\quotep{\procn{x}|\procn{x}}$.

Also, we will avail ourselves of the notation $x^{L}$ and $x^{R}$ to
denote injections of a name into disjoint copies of the name
space. There are numerous ways to accomplish this. One example can be
found in \cite{MeredithR05}. This notation overloads to vectors of
names: $\vec{x}^{\pi} := (x_{i}^{\pi} \; : \; 0 \leq i < |\vec{x}| )$ where $\pi \in \{L,R\}$.

We also use $P^{\Box} := P|\Box$.

In \cite{MeredithR05} an interpretation of the new operator is
given. It turns out that there are several possible interpretations
all enjoying the requisite algebraic properties of the operator (see
\cite{milner91polyadicpi}). We will therefore make liberal use of
$(\nu\; \vec{x})P$.

% subsection the_syntax_and_semantics_of_the_notation_system (end)   

\input{qm2pi.qmops} 

\input{qm2pi.sterngerlach} 

\input{qm2pi.metric} 

% section concurrent_process_calculi (end)

%\input{qm2pi.proofsketch}

% section proof sketch (end)

%\input{qm2pi.slviaknots} 

% section spatial logic via knots (end)

\input{qm2pi.conclusion}

% section conclusion (end)

%\input{qm2pi.dtcodes} 

% section wiring algorithm (end)

\input{qm2pi.ack} 

% section acknowledgments (end)

\newpage


\bibliographystyle{plain}   
\bibliography{../../biblios/main.bib}

\input{qm2pi.rhodetails}

\end{document}

 

% subsection basic_interpretation (end)

%\input{qm2pi.rho.presentation} 
\subsection{The syntax and semantics of the notation system}\label{sub:the_syntax_and_semantics_of_the_notation_system} % (fold)

We now summarize a technical presentation of the calculus that
embodies our theory of dynamics. The typical presentation of such a
calculus follows the style of giving generators and relations on
them. The grammar, below, describing term constructors, freely
generates the set of processes, $\Proc$. This set is then quotiented
by a relation known as structural congruence and it is over this set
that the notion of dynamics is expressed. This presentation is
essentially that of \cite{MeredithR05} with the addition of
polyadicity and summation. For readability we have relegated some of
the technical subtleties to an appendix.

\subsubsection{Process grammar}\label{subsub:process_grammar}

\begin{mathpar}
  \inferrule* [lab=synchronization] {} {{M} \bc \pzero \;|\; x?F \;|\; x!C }
  \and
  \inferrule* [lab=abstraction] {} {{F} \bc (x)P}
  \and
  \inferrule* [lab=concretion] {} {{C} \bc \langle Q \rangle}
  \and
  \inferrule* [lab=process] {} {{P,Q} \bc M \;| \;P|Q \;|\; @{x}}
  \and
  \inferrule* [lab=name] {} {{x} \bc \quotep{P}}
\end{mathpar} 

Note that $\vec{x}$ (resp. $\vec{P}$) denotes a vector of names
(resp. processes) of length $|\vec{x}|$ (resp. $|\vec{P}|$). We adopt
the following useful abbreviations.

\begin{mathpar}
   x?(\vec{y}).P := x.(\vec{y})P \and  x\clift{\vec{P}} := x.\clift{\vec{P}}
   \and x!(y) := \lift{x}{\dropn{y}}
   \and \Pi_{i=0}^{n-1}P_i := P_0 | \ldots | P_{n-1}
\end{mathpar}

\subsubsection{Structural congruence}

\paragraph{Free and bound names and alpha-equivalence.} At the
core of structural equivalence is alpha-equivalence which identifies
process that are the same up to a change of variable. Formally, we
recognize the distinction between free and bound names. The free names
of a process, $\freenames{P}$, may be calculated recursively as
follows:

\begin{mathpar}
\freenames{\pzero} := \emptyset
  \and \\
  \freenames{x?(y).P} := \{ x \} \cup (\freenames{P} \setminus \{ y \})
  \and 
  \freenames{x!\langle P \rangle} := \{ x \} \cup \{ P \} 
  \and \\
  \freenames{P|Q} := \freenames{P} \cup \freenames{Q}
  \and \\
  \freenames{@{x}} := \{ x \}
\end{mathpar}

$\pi$
$\quotep{\pi}$

$\freenames{-} : \pi \to \mathcal{P}(\quotep{\pi})$

\begin{eqnarray*}
  \freenames{\pzero} & := & \emptyset \\
  \freenames{x?(y).P} & := & \{ x \} \cup (\freenames{P} \setminus \{ y \}) \\
  \freenames{x!\langle P \rangle} & := & \{ x \} \cup \{ P \} \\
  \freenames{P|Q} & := & \freenames{P} \cup \freenames{Q} \\
  \freenames{\dropn{x}} & := & \{ x \}
\end{eqnarray*}

The bound names of a process, $\boundnames{P}$, are those names occurring in $P$
that are not free. For example, in $x?(y).0$, the name $x$ is free, while $y$ is bound.

\begin{mathpar}
  \inferrule* [lab=monoidal-laws] {} { P|Q \equiv Q|P \and P|0 \equiv P \and P|(Q|R) \equiv (P|Q)|R }
\end{mathpar}

\begin{mathpar}
  \inferrule* [lab=alpha-equivalence] {} { (x)P \equiv (y)P\{y/x\} \and y \not\in \freenames{P} }
\end{mathpar}

\begin{definition}
Then two processes, $P,Q$, are alpha-equivalent if $P = Q\{\vec{y}/\vec{x}\}$ for
some $\vec{x} \in \boundnames{Q},\vec{y} \in \boundnames{P}$, where $Q\{\vec{y}/\vec{x}\}$
denotes the capture-avoiding substitution of $\vec{y}$ for $\vec{x}$ in $Q$.
\end{definition}

\begin{definition}
  The {\em structural congruence} \cite{SangiorgiWalker} , $\equiv$,
  between processes is the least congruence containing
  alpha-equivalence, satisfying the abelian monoid laws
  (associativity, commutativity and $\pzero$ as identity) for parallel
  composition $|$ and for summation $+$.
\end{definition}

\subsection{Name equivalence}

We take name equivalence, written $\nameeq$, to be the smallest
equivalence relation generated by the following rules.

\begin{mathpar}
\inferrule*[lab=Quote-drop]
{ }
{ \quotep{@{x}} \nameeq x }

\inferrule*[lab=Struct-equiv]
{ P \scong Q }
{ \quotep{P} \nameeq \quotep{Q} }
\end{mathpar}

The astute reader will have noticed that the mutual recursion of names
and processes imposes a mutual recursion on alpha-equivalence and
structural equivalence via name-equivalence. Fortunately, all of this
works out pleasantly and we may calculate in the natural way, free of
concern. The reader interested in the details is referred to the
appendix \ref{appendix:rho_details}.

\subsection{Substitution}

We use $\Proc$ for the set of processes, $\QProc$ for the set of
names, and $\id{\{}\vec{y} / \vec{x} \id{\}}$ to denote partial maps,
$s : \QProc \rightarrow \QProc$. A map, $s$ lifts, uniquely, to a map
on process terms, $\widehat{s} : \Proc \rightarrow \Proc$ by the
following equations.

\begin{mathpar}
  (0) \psubstp{Q}{P} := 0 \\
  (R \juxtap S) \psubstp{Q}{P}
  :=    
  (R)\psubstp{Q}{P} \juxtap (S) \psubstp{Q}{P} \\
  (x?(y).R) \psubstp{Q}{P}    
  :=    
  (x)\substp{Q}{P} (z)\concat( (R \psubstn{z}{y}) \psubstp{Q}{P} ) \\
  (\lift{x}{R}) \psubstp{Q}{P}  
  :=
  \lift{(x)\substp{Q}{P}}{ R \psubstp{Q}{P} } \\
%   (\dropn{x})  \psubstp{Q}{P}       
%   := 
%   \left\{ 
%     \begin{array}{ccc} 
%       \dropn{\quotep{Q}} & & x \nameeq \quotep{P} \\
%       \dropn{x} & & otherwise \\
%     \end{array}
%   \right. 
  (\dropn{x})  \psubstp{Q}{P}       
  := 
  \left\{ 
    \begin{array}{ccc} 
      Q & & x \nameeq \quotep{P} \\
      \dropn{x} & & otherwise \\
    \end{array}
  \right.
\end{mathpar}
 

where

\begin{eqnarray}
  (x)\id{\{} \lpquote Q \rpquote / \lpquote P \rpquote \id{\}}            = 
  \left\{ 
    \begin{array}{ccc}
      \lpquote Q \rpquote & & x \nameeq \lpquote P \rpquote \\
      x & & otherwise \\
    \end{array}
  \right. \nonumber
\end{eqnarray}

and $z$ is chosen distinct from $\quotep{P}$, $\quotep{Q}$, the free
names in $Q$, and all the names in $R$. Our $\alpha$-equivalence will
be built in the standard way from this substitution.

\begin{remark}\label{rem:no_self_referential_names}
  One consequence of these definitions is that $\forall P. \quotep{P}
  \not\in \freenames{P}$.
\end{remark}

\subsection{ Dynamic quote: an example }

Anticipating something of what's to come, consider applying the
substitution, $\widehat{\id{\{}u / z \id{\}}}$, to the following pair
of processes, $\lift{w}{y!(z)}$ and $w[ \lpquote y!(z) \rpquote ]$.

\begin{eqnarray}
	\lift{w}{y!(z)}\widehat{\id{\{}u / z \id{\}}}
		& = &
		\lift{w}{y!(u)} \nonumber\\
	w[ \lpquote y!(z) \rpquote ] \widehat{ \id{\{}u / z \id{\}} }
		& = &
		w[ \lpquote y!(z) \rpquote ] \nonumber
\end{eqnarray}

Because the body of the process between quotes is impervious to
substitution, we get radically different answers. In fact, by
examining the first process in an input context,
e.g. $x?(z).\lift{w}{y!(z)}$, we see that the process under the lift
operator may be shaped by prefixed inputs binding a name inside it. In
this sense, the lift operator will be seen as a way to dynamically
construct processes before reifying them as names.

Finally equipped with these standard features we can present the
dynamics of the calculus.

\subsubsection{Operational semantics} 

Finally, we introduce the computational dynamics. What marks these
algebras as distinct from other more traditionally studied algebraic
structures, e.g. vector spaces or polynomial rings, is the manner in
which dynamics is captured. In traditional structures, dynamics is typically
expressed through morphisms between such structures, as in linear maps
between vector spaces or morphisms between rings. In algebras
associated with the semantics of computation, the dynamics is
expressed as part of the algebraic structure itself, through a
reduction reduction relation typically denoted by $\red$. Below, we
give a recursive presentation of this relation for the calculus used
in the encoding.

$\red \subseteq \pi \times \pi$
$\red : \pi \to \mathcal{P}(\pi)$

\begin{mathpar}
  \inferrule* [lab=Comm] { \textsf{match}( x_{src}, x_{trgt} ) } { x_{trgt}?(y)P \; | \; x_{src}!\langle {Q} \rangle \red P\{\quotep{Q}/y}\} }
  \and \\
  \inferrule* [lab=Par] {{P} \red {P}'} {{{P} | {Q}} \red {{P}' | {Q}}}
  \and
  \inferrule* [lab=Equiv]{{{P} \scong {P}'} \andalso {{P}' \red {Q}'} \andalso {{Q}' \scong {Q}}}{{P} \red {Q}}
\end{mathpar}

\begin{eqnarray*}
  match_{\equiv} (\quotep{P},\quotep{Q}) & := & P \equiv Q \\
  match_{\dagger}(\quotep{P},\quotep{Q}) & := & \forall R. P|Q \red^{*} R => R \red^{*} 0 \\
  match_{K}(\quotep{P},\quotep{Q}) & := & K \mbox{ for some context } K
\end{eqnarray*}

$u?(x)P | u!\langle Q \rangle \red P\{\quotep{Q}/x\}$

%We write $\wred$ for $\red^*$, and $P\red$ if $\exists Q $ such that $ P \red Q$.
We write $P\red$ if $\exists Q $ such that $ P \red Q$ and $P\not\red$, otherwise.

\section{Replication}

As mentioned before, it is known that replication (and hence
recursion) can be implemented in a higher-order process algebra
\cite{SangiorgiWalker}. As our first example of calculation with the
machinery thus far presented we give the construction explicitly in
the {\rhoc}.

\begin{eqnarray}
	D_{x} & := & \prefix{x}{y}{(\binpar{\outputp{x}{y}}{@{y}})} \nonumber\\
	\bangp_{x}{P} & := & \binpar{{x}!\langle{\binpar{D_{x}}{P}}\rangle}{D_{x}} \nonumber
\end{eqnarray}

\begin{eqnarray}
	\bangp_{x}{P} & & \nonumber\\
	=
	& {x}!\langle{(\prefix{x}{y}{(\outputp{x}{y} | @{y})) | P}}\rangle 
	      | \prefix{x}{y}{(\outputp{x}{y} | @{y})} & \nonumber\\
	\red
	& (\outputp{x}{y} | @{y})\substn{\quotep{(\prefix{x}{y}{(@{y} | \outputp{x}{y})) | P}}}{y} & \nonumber\\
	=
	& \outputp{x}{\quotep{(\prefix{x}{y}{(\outputp{x}{y} | @{y})) | P}}}
	  | {(\prefix{x}{y}{(\outputp{x}{y} | @{y})) | P}} & \nonumber\\
	\red
	& \ldots & \nonumber\\
	\red^*
	& P | P | \ldots & \nonumber
\end{eqnarray}

Of course, this encoding, as an implementation, runs away, unfolding
$\bangp{P}$ eagerly. A lazier and more implementable replication
operator, restricted to input-guarded processes, may be obtained as follows.

\begin{eqnarray}
\bangp{\prefix{u}{v}{P}} 
	:= 
	\binpar{\lift{x}{\prefix{u}{v}{(\binpar{D(x)}{P})}}}{D(x)} \nonumber
\end{eqnarray}

\begin{remark}
  Note that the lazier definition still does not deal with summation
  or mixed summation (i.e. sums over input and output). The reader is
  invited to construct definitions of replication that deal with these
  features. 

  Further, the definitions are parameterized in a name, $x$. Can you,
  gentle reader, make a definition that eliminates this parameter and
  guarantees no accidental interaction between the replication
  machinery and the process being replicated -- i.e. no accidental
  sharing of names used by the process to get its work done and the
  name(s) used by the replication to effect copying. This latter
  revision of the definition of replication is crucial to obtaining
  the expected identity $!!P \sim !P$.
\end{remark}

\begin{remark}\label{rem:paradoxical_combinator}
  The reader familiar with the lambda calculus will have noticed the
  similarity between $D$ and the paradoxical combinator.

  [Ed. note: the existence of this seems to suggest we have to be more
  restrictive on the set of processes and names we admit if we are to
  support no-cloning.]
\end{remark}

\subsubsection{Bisimulation}

The computational dynamics gives rise to another kind of equivalence,
the equivalence of computational behavior. As previously mentioned
this is typically captured \emph{via} some form of bisimulation.

% The notion we use in this paper is weak barbed bisimulation
% \cite{milner91polyadicpi}.

The notion we use in this paper is derived from weak barbed
bisimulation \cite{milner91polyadicpi}. 

\begin{definition}
An \emph{observation relation}, $\downarrow_{\mathcal N}$, over a set
of names, $\mathcal N$, is the smallest relation satisfying the rules
below.

\infrule[Out-barb]{y \in {\mathcal N}, \; x \nameeq y}
		  {\outputp{x}{v} \downarrow_{\mathcal N} x}
\infrule[Par-barb]{\mbox{$P\downarrow_{\mathcal N} x$ or $Q\downarrow_{\mathcal N} x$}}
		  {\binpar{P}{Q} \downarrow_{\mathcal N} x}

We write $P \Downarrow_{\mathcal N} x$ if there is $Q$ such that 
$P \wred Q$ and $Q \downarrow_{\mathcal N} x$.
\end{definition}

\begin{definition}
%\label{def.bbisim}
An  ${\mathcal N}$-\emph{barbed bisimulation} over a set of names, ${\mathcal N}$, is a symmetric binary relation 
${\mathcal S}_{\mathcal N}$ between agents such that $P\rel{S}_{\mathcal N}Q$ implies:
\begin{enumerate}
\item If $P \red P'$ then $Q \wred Q'$ and $P'\rel{S}_{\mathcal N} Q'$.
\item If $P\downarrow_{\mathcal N} x$, then $Q\Downarrow_{\mathcal N} x$.
\end{enumerate}
$P$ is ${\mathcal N}$-barbed bisimilar to $Q$, written
$P \wbbisim_{\mathcal N} Q$, if $P \rel{S}_{\mathcal N} Q$ for some ${\mathcal N}$-barbed bisimulation ${\mathcal S}_{\mathcal N}$.
\end{definition}

$\mathcal{R} \subseteq \pi \times \pi$

$P \mathcal{R} Q => \forall P'. P \red P' \Rightarrow \exists Q'. Q \red Q', P' \mathcal{R} Q'$

$P \vdash x \Rightarrow Q \vdash x$

\begin{mathpar}
  \inferrule*[lab=Out-barb]{x \nameeq y}{{y}!\langle{Q}\rangle \vdash x}
  \and
  \inferrule*[lab=Par-barb]{\mbox{$P\vdash x$ or $Q\vdash x$}}{\binpar{P}{Q} \vdash x}
\end{mathpar}

\subsubsection{Contexts}

One of the principle advantages of computational calculi like the
$\pi$-calculus is a well-defined notion of context,
contextual-equivalence and a correlation between
contextual-equivalence and notions of bisimulation. The notion of
context allows the decomposition of a process into (sub-)process and
its syntactic environment, its context. Thus, a context may be
thought of as a process with a ``hole'' (written $\Box$) in it. The
application of a context $M$ to a process $P$, written $M[P]$, is
tantamount to filling the hole in $M$ with $P$. In this paper we do
not need the full weight of this theory, but do make use of the notion
of context in the proof the main theorem. 

\begin{mathpar}
  \inferrule* [lab=summation] {} {{M_{M},M_{N}} \bc \Box \;|\; x.M_{A} \;|\; M_{M}+M_{N}}
  \and
  \inferrule* [lab=agent] {} {{M_{A}} \bc (\vec{x})M_{P} \;| \; \clift{P_0,\ldots,M_{P},\ldots,P_N}}
  \and \\
  \inferrule* [lab=process] {} {{M_{P}} \bc M_{N} \;| \;P|M_{P} }
\end{mathpar} 

\begin{mathpar}
  \inferrule* [lab=sychronization] {} {M_{N} \bc \Box \;|\; x?M_{F} \;|\; x!M_{C}}
  \and
  \inferrule* [lab=abstraction] {} {{M_{F}} \bc (x)M_{P} }
  \and
  \inferrule* [lab=concretion] {} {{M_{C}} \bc \langle M_{P} \rangle }
  \and \\
  \inferrule* [lab=process] {} {{M_{P}} \bc M_{N} \;| \;P|M_{P} }
\end{mathpar}

\begin{definition}[contextual application] Given a context $M$, and
  process $P$, we define the \emph{contextual application}, $M[P] :=
  M\{P/\Box\}$. That is, the contextual application of M to P is the
  substitution of $P$ for $\Box$ in $M$.
\end{definition}

$\meaningof{-} : L \to \mathcal{P}(\pi)$

\begin{mathpar}
  \inferrule* [lab=collection] {} {\meaningof{true} = \pi, \and \meaningof{~E} = \pi \setminus \meaningof{E}, \and \meaningof{E_{1} \& E_{2}} = \meaningof{E_{1}} \cap \meaningof{E_{2}}}
\end{mathpar}

\begin{mathpar}
  \inferrule* [lab=structure] {} {\meaningof{0} = \{ P \in \pi | P \equiv 0 \}, \and \\ \meaningof{E_1 | E_2} = \{ P \in \pi | P \equiv P_{1} | P_{2}, P_{1} \in \meaningof{E_{1}}, P_{2} \in \meaningof{E_2}\} }
\end{mathpar}

\begin{mathpar}
 \inferrule* [lab=behavior] {} {\meaningof{\langle a?b \rangle E} = \{ P \in \pi | P \equiv Q | u?(y)P', \\ \and \\\\ \and \\ \;\;\; u \in \meaningof{a}, \forall z.P'\{z/y\} \in \meaningof{E\{z/b\}}\}, \and \\ \meaningof{a!E} = \{ P \in \pi | P \equiv Q | x!\langle P' \rangle, x \in \meaningof{a} P' \in \meaningof{E}\} }
\end{mathpar}

\begin{mathpar}
 \inferrule* [lab=nominal] {} {\meaningof{\quotep{E}} = \{ \quotep{P} \in \quotep{\pi} | P \in \meaningof{E} \}, \and \meaningof{\quotep{P}} = \{ \quotep{Q} \in \quotep{\pi} | P \equiv Q \} \and \\ \meaningof{@\quotep{E}} = \{ P \in \pi | P \equiv @x, x \in \meaningof{E} \}}
\end{mathpar}

\begin{eqnarray*}
  \\
  \meaningof{-} : TS \to ST
\end{eqnarray*}

\begin{eqnarray*}
  \\
  L : TS \to ST
\end{eqnarray*}

\begin{eqnarray*}
  \\
  P \models E \iff P \in \meaningof{E}
\end{eqnarray*}

\begin{eqnarray*}
  P \approx_{L} Q \iff \forall E \in L. P \models E \iff Q \models E
\end{eqnarray*}

\begin{eqnarray*}
  P \approx_{K} Q
\end{eqnarray*}

\begin{eqnarray*}
  P \approx Q
\end{eqnarray*}

$\approx_{K} = \approx = \approx_{L}$

\subsubsection{Contextual duality}

Note that contexts extend the quotation operation to a family of
operations from processes to names. Given a context, $M$, we can
define a \emph{nominal context}, $\quotep{M}$ by $\quotep{M}[P] :=
\quotep{M[P]}$. To foreshadow what is to come we observe that these
operations enjoy a duality with processes very much like the duality
between vectors and maps from vectors to scalars.

Further, because the calculus is essentially higher-order, we have a
correspondence between contexts and processes. More specifically,
given a name $x$ and a context $M$ we can construct $M^{*}_{x}$ such
that 

\begin{mathpar}
  M^{*}_{x} | \lift{x}{P} \red M[P]
\end{mathpar}

namely,

\begin{mathpar}
  M^{*}_{x} := x?(u).M[\dropn{u}]
\end{mathpar}

The dependence of $M^{*}_{x}$ on a name makes it an abstraction, 

\begin{mathpar}
  M^{*} := (x)x?(u).M[\dropn{u}]
\end{mathpar}

\subsection{Additional notation}

It will sometimes be convenient to denote the process a name
quotes. We already have the notation $x = \quotep{P}$, but it will be
convenient to introduce an alternate notation, $\procn{x}$, when we
want to emphasize the connection to the use of the name. Note that, by
virtue of name equivalence, $\quotep{\procn{x}} \nameeq x$; so, the
notation is consistent with previous definitions.

Further, because names have structure it is possible to effect
substitutions on the basis of that structure. This means we need to
upgrade our notation for substitutions, which we accomplish by
adapting comprehension notation. Thus,

\begin{mathpar}
  P\{ y / x : x \in S \}
\end{mathpar}

is interpreted to mean the process derived from P by replacing (in a
capture-avoiding manner) each occurrence of $x$ in $S$ by $y$. For example,

\begin{mathpar}
  P\{ \quotep{\procn{x}|\procn{x}} / x : x \in \freenames{P} \}
\end{mathpar}

will replace each (occurrence) of a free name $x$ in $P$ by
$\quotep{\procn{x}|\procn{x}}$.

Also, we will avail ourselves of the notation $x^{L}$ and $x^{R}$ to
denote injections of a name into disjoint copies of the name
space. There are numerous ways to accomplish this. One example can be
found in \cite{MeredithR05}. This notation overloads to vectors of
names: $\vec{x}^{\pi} := (x_{i}^{\pi} \; : \; 0 \leq i < |\vec{x}| )$ where $\pi \in \{L,R\}$.

We also use $P^{\Box} := P|\Box$.

In \cite{MeredithR05} an interpretation of the new operator is
given. It turns out that there are several possible interpretations
all enjoying the requisite algebraic properties of the operator (see
\cite{milner91polyadicpi}). We will therefore make liberal use of
$(\nu\; \vec{x})P$.

% subsection the_syntax_and_semantics_of_the_notation_system (end)   

\section{Interpretation of QM}
\subsection{Supporting definitions}
\subsubsection{Multiplication}
\begin{mathpar}
  \quotep{Q} \cdot \quotep{R} := \quotep{Q|R}
  \and \\
  \quotep{Q} \cdot P := P\{ \quotep{Q|R} / \quotep{R} : \quotep{R} \in \freenames{P} \}
\end{mathpar}

\paragraph{Discussion}
The first line needs little explanation. The second line says that
each free name of the process is replaced with the multiplication of
that name by the scalar. Multiplication of a scalar (name) by a state
(process) results in a process all the names of which have been `moved
over' by parallel composition with the process the scalar
quotes. There is a subtlety that the bound names have to be
manipulated so that multiplied names aren't accidentally
captured. There are many ways to achieve this.

\begin{remark}\label{rem:multiplication_identities}
  The reader is invited to verify that for all $x,y,z \in \QProc$ and $P \in \Proc$
  \begin{mathpar}
    x \cdot \quotep{0} \equiv x 
    \and
    x \cdot y \equiv y \cdot x
    \and
    x \cdot (y \cdot z) \equiv (x \cdot y) \cdot z
    \and \\
    \quotep{0} \cdot P \equiv P
    \and \\
    x \cdot (y \cdot P) \equiv (x \cdot y) \cdot P
    \and \\
    x \cdot (P|Q) \equiv (x \cdot P) | (x \cdot Q)
    \and \\    
  \end{mathpar}
\end{remark}

\subsubsection{Tensor product}

We define a tensor product on processes by structural induction.

\paragraph{Tensor of sums} First note that all summations, including
$\pzero$ and sequence, can be written $\Sigma_{i} x_{i}.A_{i} +
\Sigma_{j} x_{j}.C_{j}$, where we have grouped input-guarded processes
together and output-guarded processes together.

Thus, we can define the tensor product of two summations, $N_{1}\otimes N_{2}$, where

\begin{mathpar}
  N_{1} := \Sigma_{i} x_{i}.A_{i} + \Sigma_{j} x_{j}.C_{j}
  \and
  N_{2} := \Sigma_{i'} y_{i'}.B_{i'} + \Sigma_{j'} y_{j'}.D_{j'} 
\end{mathpar}

as follows.

\begin{mathpar}
  \Sigma_{i} x_{i}.A_{i} + \Sigma_{j} x_{j}.C_{j} \otimes \Sigma_{i'}
  y_{i'}.B_{i'} + \Sigma_{j'} y_{j'}.D_{j'} 
  \and \\
  := \; \Sigma_{i} \Sigma_{i'} \quotep{\stackrel{\vee}{x_{i}}| \stackrel{\vee}{y_{i'}}}.(A_{i}\otimes B_{i'}) \; | \; \Sigma_{i'} \Sigma_{i} \quotep{\stackrel{\vee}{y_{i'}}|\stackrel{\vee}{x_{i}}}.(B_{i'}\otimes A_{i})
  \and
  \;\; | \;\; \Sigma_{j} \Sigma_{j'} \quotep{\stackrel{\vee}{x_{j}}|\stackrel{\vee}{y_{j'}}}.(A_{j}\otimes B_{j'}) \; | \; \Sigma_{j'} \Sigma_{j} \quotep{\stackrel{\vee}{y_{j'}}|\stackrel{\vee}{x_{j}}}.(B_{j'}\otimes A_{j})
\end{mathpar}

\begin{remark}
  Do we need to $x^{L}$ and $y^{R}$ for this construction as well?
\end{remark}

\paragraph{Tensor of parallel compositions} Next, we distribute tensor
over par.

\begin{mathpar}
  P_{1}|P_{2} \otimes Q_{1}|Q_{2} := (P_{1} \otimes Q_{1}) | (P_{1}
  \otimes Q_{2}) | (P_{2} \otimes Q_{1}) | (P_{2} \otimes Q_{2})
\end{mathpar}

\paragraph{Tensor with dropped names} We treat tensor of a
process with a dropped name as parallel composition.

\begin{mathpar}
  P \otimes \dropn{x} := P | \dropn{x}
\end{mathpar}

\paragraph{Tensor of agents}

Finally, we need to define tensor on agents. Note that the definition
of tensor on normal products only tensors inputs with inputs and
outputs with outputs. Thus, we only have to define the operation on
``homogeneous'' pairings.

\begin{mathpar}
  (\vec{x})P \otimes (\vec{y})Q
  \and \\
  := (x_{0}^{L}|y_{0}^{R},\ldots,x_{0}^{L}|y_{n}^{R},\ldots,x_{m}^{L}|y_{0}^{R},\ldots,x_{m}^{L}|y_{n}^R)(P\{ \vec{x}^{L}/\vec{x}\} \otimes Q \{ \vec{y}^{R}/\vec{y}\})
  \and \\
  \clift{\vec{P}} \otimes \clift{\vec{Q}}
  \and \\
  := \clift{P_{0}\otimes Q_{0},\ldots,P_{0}\otimes Q_{n},\ldots,P_{m}\otimes Q_{0},\ldots,P_{m}\otimes Q_{n}}
\end{mathpar}

\begin{remark}
  Observe that arities of tensored abstractions matches arities of
  tensored concretions if the original arities matched. Note also that
  the length of the arities corresponds to the increase in dimension
  we see in ordinary vector space tensor product.
\end{remark}

\begin{remark}
  Operationally, this definition distributes the tensor down to
  components ``linked'' by summation. Tensor over summation is
  intriguing in that it mixes names. Moreover, as a consequence of the
  way it mixes names we have the identities for all $x \in \QProc$ and
  $P,Q \in \Proc$

  \begin{mathpar}
    (x \cdot P) \otimes Q \equiv x \cdot (P \otimes Q) \equiv P \otimes (x \cdot Q)
    \and
    P \otimes \pzero \equiv P
  \end{mathpar}

  that the reader is invited to verify.
\end{remark}

\subsubsection{Annihilation}
\begin{mathpar}
  P^{\perp} := \{ Q | \forall R. P|Q \red^{*} R \Rightarrow R \red^{*} \pzero \}
  \and \\
  P^{\underline{\perp}} := \Sigma_{Q \in P^{\perp}} \quotep{Q}?(y).(\dropn{y}|Q) | \Sigma_{Q \in P^{\perp}} \quotep{Q}\clift{\Box}
\end{mathpar}

\paragraph{Discussion} The reader will note that $P^{\perp}$ is a
\emph{set} of processes, while $P^{\underline{\perp}}$ is a
\emph{context}. We call the set $P^{\perp}$ the \emph{annihilators} of
$P$. The parallel composition of a process in the annihilators of $P$
with $P$ will result in a process, the state space of which has all
paths eventually leading to $\pzero$. Execution may endure loops; but
under reasonable conditions of fairness (naturally guaranteed under
most notions of bisimulation) such a composite process cannot get
stuck in such a loop and will, eventually pop out and terminate.

The context $P^{\underline{\perp}}$ is ready and willing to ``take the
$P$ out of'' the process to which it is applied. It will effectively
transmit the code of the process to which it is applied to one of the
annihilators and run the process against it.

\subsubsection{Evaluation}
We fix $M$ a domain of fully abstract interpretation with an equality
coincident with bisimulation. We take $\meaningof{\cdot} : \Proc \to
M$ to be the map interpreting processes and $\nmeaningof{\cdot} : \M
\to Proc$ to be the map running the other way. Then we define

\begin{mathpar}
  \int P := \nmeaningof{\meaningof{P}}
\end{mathpar}

\paragraph{Discussion}
There are many fully abstract interpretations of Milner's
$\pi$-calculus. Any of them can be used as a basis for interpreting
the reflective calculus here. Equipped with such a domain it is
largely a matter of grinding through to check that the Yoneda
construction for the normalization-by-evaluation program can be
extended to this setting.

\begin{remark}
  The reader is invited to verify that $\int (P^{\underline{\perp}}[P]) = 0$.
\end{remark}

\subsection{Quantum mechanics}

Table \ref{tbl:core_qm_op_defns} gives the core operational definitions

\begin{table}[htp]\label{tbl:core_qm_op_defns}
  \center{
    \fbox{
      \begin{tabular}{c|c}
        quantum mechanics & process calculus \\
        \hline
        scalar & $x := \quotep{P}$ \\
        state vector & $\state{P} := P$ \\
        dual & $\state{P}^{*} := \event{P^{\underline{\perp}}} := \quotep{P^{\underline{\perp}}}[-]$ \\
        matrix & $ \Sigma_{\alpha} \state{P_{\alpha}}x_{\alpha}\event{Q_{\alpha}}$ \\
        vector addition & $\state{P} + \state{Q} := \state{P | Q}$ \\
        tensor product & $\state{P} \otimes \state{Q} := \state{P \otimes Q}$ \\
        inner product & $\innerprod{P}{Q} := \quotep{\int P^{\underline{\perp}}[Q]}$ \\
      \end{tabular}
    }
  }
  \caption{QM - operational definitions}
\end{table}

where

\begin{mathpar}
  \prmatrix{P}{Q} := \fprmatrix{P}{\quotep{\pzero}}{Q}
  \and
  \fprmatrix{P}{x}{Q} := (\state{P},x,\event{Q})
  \and
  (\fprmatrix{P}{x}{Q})(\state{R}) := x \cdot \innerprod{Q}{R} \cdot \state{P}
  \and
  (\fprmatrix{P}{x}{Q})(\event{R}) := x \cdot \innerprod{R}{P} \cdot \event{Q}
\end{mathpar}

\paragraph{Discussion}
As promised: vectors (aka states) are represented as processes; duals
as contextual duals; inner product definition should be compared with
standard inner product definition for ....

\begin{remark}
  Assuming $\int (P^{\underline{\perp}}[P]) = 0$, the reader is
  invited to verify that $(\fprmatrix{P}{x}{P})(\state{P}) = x \cdot \state{P}$.
\end{remark}

\begin{remark}
  The reader is invited to verify that $\innerprod{P}{Q}$ could
  equally well have been written $\quotep{\int \stackrel{\vee}{x}}$
  where $x = \event{P^{\underline{\perp}}}(Q)$.

  One of the motivations for this remark is that there is another way
  to factor these operations. We could package up evaluation in the dual:

  \begin{mathpar}
    \state{P}^{*} := \event{\int P^{\underline{\perp}}} := \quotep{\int P^{\underline{\perp}}}[-]
  \end{mathpar}

  and then have inner product defined by
  
  \begin{mathpar}
    \innerprod{P}{Q} := \event{P}(Q)
  \end{mathpar}

  Hopefully, experience with the calculations will provide guidance on
  the best factoring.
\end{remark}

\begin{remark}
  Assuming $\int (P^{\underline{\perp}}[P]) = 0$, the reader is
  invited to verify that $\forall P,Q. (\prmatrix{0}{Q})(\state{0}) =
  \state{0}$ and dually $(\prmatrix{P}{0})(\event{0}) = \event{0}$.
\end{remark}

\begin{remark}
  i'm a little worried that i don't (yet) have proper support for
  complex conjugacy. But, the observation above may give us a
  clue. According to Abramsky, it must be the case that the scalars
  are iso to the homset of the identity for the tensor -- which the
  observation above characterizes. 

  For now, we will simply bookmark the notion with $\overline{x}$.
\end{remark}

\subsubsection{Adjointness}

We need to give a definition of $(\cdot)^{\dagger}$ for matrices. The
obvious candidate definition is
\begin{mathpar}
(\Sigma_{\alpha}\fprmatrix{P_{\alpha}}{x_{\alpha}}{Q_{\alpha}})^{\dagger}
= \Sigma_{\alpha}\fprmatrix{(Q_{\alpha}^{\underline{\perp}})^{*}}{\overline{x}_{\alpha}}{P_{\alpha}^{\underline{\perp}}} 
\end{mathpar}

But, $(Q_{\alpha}^{\underline{\perp}})^{*}$ requires a name along
which to communicate the process to achieve the context application.

\subsubsection{Basis for a basis}
If processes label states and ``addition'' of states (a.k.a. vector
addition) is interpreted as parallel composition, what corresponds to
notions of linear independence and basis? Here, we recall that Yoshida
has developed a set of \emph{combinators} for an asynchronous verison
of Milner's $\pi$-calculus. These are a finite set of processes such
any process can be expressed as parallel composition of these
combinators together with liberal uses of the new operator and
replication. We can simply give a translation of these into the
present calculus and have reasonable expectation that the property
carries over. That is, that the resultant set allows to express all
processes via parallel composition. Note, however, that there is no
new operator or replication in this calculus. As a result, we expect
that the corresponding set is actually infinite. That is, we expect
that the space is actually infinite dimensional.

\begin{remark}
  The attentive reader may be a bit concerned. Certainly, the
  collection $S$, $K$ and $I$ is a finite set of
  combinators. Shouldn't we expect to see a finite set of combinators
  for an effectively equivalent system? i am very sympathetic to this
  critique and feel it warrants full attention. On the other hand, i
  also have in mind the following analogy. The natural numbers, as a
  monoid under addition, has exactly $1$ generator, while the natural
  numbers, as a monoid under multiplication, has countably many
  generators (the primes). We observe that the application of the
  lambda calculus is much less resource sensitive than the parallel
  composition of the $\pi$-calculus. Could it be the case that we have
  an analogy of the form
  
  \begin{mathpar}
    m + n : MN :: m*n : M|N
  \end{mathpar}

  giving a similar blow up in the set of ``primes''?  This is such a
  wonderful thought that, even if it's not true, i think it's worth
  writing down.
\end{remark}
 

\documentclass[12pt]{llncs}
%\documentclass{jktr}

\usepackage[pdftex]{hyperref}                   
\usepackage {listings}
\usepackage {mathpartir}
\usepackage{bcprules}
%\usepackage{listings}
                       
\usepackage{graphicx} 
%\usepackage[margins=2.5cm,nohead,nofoot]{geometry}
%\usepackage{geometry}
\usepackage{amsfonts}
\usepackage{amstext}
\usepackage{latexsym}
\usepackage{amssymb}
\usepackage{color}


%\include{myPreamble}
\include{qm2pi.local} 

%\ifpdf
%\usepackage[pdftex]{graphicx}
%\else
%\usepackage{graphicx}
%\fi

 % \ifpdf
%  \usepackage{pdfsync}
%  \if


%\title{Brief Article}
%\author{David F. Snyder}
%\author{L.G. Meredith}

%\address{Dept. of Math., Texas State University--San Marcos, San Marcos, TX 78666}
       
\pagestyle{empty}


\begin{document}

\lstset{language=[Objective]Caml,frame=shadowbox}

\input{qm2pi.front}

% section front matter (end)

\input{qm2pi.intro} 
 
% section introduction (end)

% \input{qm2pi.knotations} 

% section notation (end)

\input{qm2pi.process.calculi} 

% section concurrent_process_calculi_and_spatial_logics_ (end)
    
%\input{qm2pi.knots2pi} 

%\input{qm2pi.trefoil} 

%\input{qm2pi.mainthm} 

% subsection basic_interpretation (end)

%\input{qm2pi.rho.presentation} 
\subsection{The syntax and semantics of the notation system}\label{sub:the_syntax_and_semantics_of_the_notation_system} % (fold)

We now summarize a technical presentation of the calculus that
embodies our theory of dynamics. The typical presentation of such a
calculus follows the style of giving generators and relations on
them. The grammar, below, describing term constructors, freely
generates the set of processes, $\Proc$. This set is then quotiented
by a relation known as structural congruence and it is over this set
that the notion of dynamics is expressed. This presentation is
essentially that of \cite{MeredithR05} with the addition of
polyadicity and summation. For readability we have relegated some of
the technical subtleties to an appendix.

\subsubsection{Process grammar}\label{subsub:process_grammar}

\begin{mathpar}
  \inferrule* [lab=synchronization] {} {{M} \bc \pzero \;|\; x?F \;|\; x!C }
  \and
  \inferrule* [lab=abstraction] {} {{F} \bc (x)P}
  \and
  \inferrule* [lab=concretion] {} {{C} \bc \langle Q \rangle}
  \and
  \inferrule* [lab=process] {} {{P,Q} \bc M \;| \;P|Q \;|\; @{x}}
  \and
  \inferrule* [lab=name] {} {{x} \bc \quotep{P}}
\end{mathpar} 

Note that $\vec{x}$ (resp. $\vec{P}$) denotes a vector of names
(resp. processes) of length $|\vec{x}|$ (resp. $|\vec{P}|$). We adopt
the following useful abbreviations.

\begin{mathpar}
   x?(\vec{y}).P := x.(\vec{y})P \and  x\clift{\vec{P}} := x.\clift{\vec{P}}
   \and x!(y) := \lift{x}{\dropn{y}}
   \and \Pi_{i=0}^{n-1}P_i := P_0 | \ldots | P_{n-1}
\end{mathpar}

\subsubsection{Structural congruence}

\paragraph{Free and bound names and alpha-equivalence.} At the
core of structural equivalence is alpha-equivalence which identifies
process that are the same up to a change of variable. Formally, we
recognize the distinction between free and bound names. The free names
of a process, $\freenames{P}$, may be calculated recursively as
follows:

\begin{mathpar}
\freenames{\pzero} := \emptyset
  \and \\
  \freenames{x?(y).P} := \{ x \} \cup (\freenames{P} \setminus \{ y \})
  \and 
  \freenames{x!\langle P \rangle} := \{ x \} \cup \{ P \} 
  \and \\
  \freenames{P|Q} := \freenames{P} \cup \freenames{Q}
  \and \\
  \freenames{@{x}} := \{ x \}
\end{mathpar}

$\pi$
$\quotep{\pi}$

$\freenames{-} : \pi \to \mathcal{P}(\quotep{\pi})$

\begin{eqnarray*}
  \freenames{\pzero} & := & \emptyset \\
  \freenames{x?(y).P} & := & \{ x \} \cup (\freenames{P} \setminus \{ y \}) \\
  \freenames{x!\langle P \rangle} & := & \{ x \} \cup \{ P \} \\
  \freenames{P|Q} & := & \freenames{P} \cup \freenames{Q} \\
  \freenames{\dropn{x}} & := & \{ x \}
\end{eqnarray*}

The bound names of a process, $\boundnames{P}$, are those names occurring in $P$
that are not free. For example, in $x?(y).0$, the name $x$ is free, while $y$ is bound.

\begin{mathpar}
  \inferrule* [lab=monoidal-laws] {} { P|Q \equiv Q|P \and P|0 \equiv P \and P|(Q|R) \equiv (P|Q)|R }
\end{mathpar}

\begin{mathpar}
  \inferrule* [lab=alpha-equivalence] {} { (x)P \equiv (y)P\{y/x\} \and y \not\in \freenames{P} }
\end{mathpar}

\begin{definition}
Then two processes, $P,Q$, are alpha-equivalent if $P = Q\{\vec{y}/\vec{x}\}$ for
some $\vec{x} \in \boundnames{Q},\vec{y} \in \boundnames{P}$, where $Q\{\vec{y}/\vec{x}\}$
denotes the capture-avoiding substitution of $\vec{y}$ for $\vec{x}$ in $Q$.
\end{definition}

\begin{definition}
  The {\em structural congruence} \cite{SangiorgiWalker} , $\equiv$,
  between processes is the least congruence containing
  alpha-equivalence, satisfying the abelian monoid laws
  (associativity, commutativity and $\pzero$ as identity) for parallel
  composition $|$ and for summation $+$.
\end{definition}

\subsection{Name equivalence}

We take name equivalence, written $\nameeq$, to be the smallest
equivalence relation generated by the following rules.

\begin{mathpar}
\inferrule*[lab=Quote-drop]
{ }
{ \quotep{@{x}} \nameeq x }

\inferrule*[lab=Struct-equiv]
{ P \scong Q }
{ \quotep{P} \nameeq \quotep{Q} }
\end{mathpar}

The astute reader will have noticed that the mutual recursion of names
and processes imposes a mutual recursion on alpha-equivalence and
structural equivalence via name-equivalence. Fortunately, all of this
works out pleasantly and we may calculate in the natural way, free of
concern. The reader interested in the details is referred to the
appendix \ref{appendix:rho_details}.

\subsection{Substitution}

We use $\Proc$ for the set of processes, $\QProc$ for the set of
names, and $\id{\{}\vec{y} / \vec{x} \id{\}}$ to denote partial maps,
$s : \QProc \rightarrow \QProc$. A map, $s$ lifts, uniquely, to a map
on process terms, $\widehat{s} : \Proc \rightarrow \Proc$ by the
following equations.

\begin{mathpar}
  (0) \psubstp{Q}{P} := 0 \\
  (R \juxtap S) \psubstp{Q}{P}
  :=    
  (R)\psubstp{Q}{P} \juxtap (S) \psubstp{Q}{P} \\
  (x?(y).R) \psubstp{Q}{P}    
  :=    
  (x)\substp{Q}{P} (z)\concat( (R \psubstn{z}{y}) \psubstp{Q}{P} ) \\
  (\lift{x}{R}) \psubstp{Q}{P}  
  :=
  \lift{(x)\substp{Q}{P}}{ R \psubstp{Q}{P} } \\
%   (\dropn{x})  \psubstp{Q}{P}       
%   := 
%   \left\{ 
%     \begin{array}{ccc} 
%       \dropn{\quotep{Q}} & & x \nameeq \quotep{P} \\
%       \dropn{x} & & otherwise \\
%     \end{array}
%   \right. 
  (\dropn{x})  \psubstp{Q}{P}       
  := 
  \left\{ 
    \begin{array}{ccc} 
      Q & & x \nameeq \quotep{P} \\
      \dropn{x} & & otherwise \\
    \end{array}
  \right.
\end{mathpar}
 

where

\begin{eqnarray}
  (x)\id{\{} \lpquote Q \rpquote / \lpquote P \rpquote \id{\}}            = 
  \left\{ 
    \begin{array}{ccc}
      \lpquote Q \rpquote & & x \nameeq \lpquote P \rpquote \\
      x & & otherwise \\
    \end{array}
  \right. \nonumber
\end{eqnarray}

and $z$ is chosen distinct from $\quotep{P}$, $\quotep{Q}$, the free
names in $Q$, and all the names in $R$. Our $\alpha$-equivalence will
be built in the standard way from this substitution.

\begin{remark}\label{rem:no_self_referential_names}
  One consequence of these definitions is that $\forall P. \quotep{P}
  \not\in \freenames{P}$.
\end{remark}

\subsection{ Dynamic quote: an example }

Anticipating something of what's to come, consider applying the
substitution, $\widehat{\id{\{}u / z \id{\}}}$, to the following pair
of processes, $\lift{w}{y!(z)}$ and $w[ \lpquote y!(z) \rpquote ]$.

\begin{eqnarray}
	\lift{w}{y!(z)}\widehat{\id{\{}u / z \id{\}}}
		& = &
		\lift{w}{y!(u)} \nonumber\\
	w[ \lpquote y!(z) \rpquote ] \widehat{ \id{\{}u / z \id{\}} }
		& = &
		w[ \lpquote y!(z) \rpquote ] \nonumber
\end{eqnarray}

Because the body of the process between quotes is impervious to
substitution, we get radically different answers. In fact, by
examining the first process in an input context,
e.g. $x?(z).\lift{w}{y!(z)}$, we see that the process under the lift
operator may be shaped by prefixed inputs binding a name inside it. In
this sense, the lift operator will be seen as a way to dynamically
construct processes before reifying them as names.

Finally equipped with these standard features we can present the
dynamics of the calculus.

\subsubsection{Operational semantics} 

Finally, we introduce the computational dynamics. What marks these
algebras as distinct from other more traditionally studied algebraic
structures, e.g. vector spaces or polynomial rings, is the manner in
which dynamics is captured. In traditional structures, dynamics is typically
expressed through morphisms between such structures, as in linear maps
between vector spaces or morphisms between rings. In algebras
associated with the semantics of computation, the dynamics is
expressed as part of the algebraic structure itself, through a
reduction reduction relation typically denoted by $\red$. Below, we
give a recursive presentation of this relation for the calculus used
in the encoding.

$\red \subseteq \pi \times \pi$
$\red : \pi \to \mathcal{P}(\pi)$

\begin{mathpar}
  \inferrule* [lab=Comm] { \textsf{match}( x_{src}, x_{trgt} ) } { x_{trgt}?(y)P \; | \; x_{src}!\langle {Q} \rangle \red P\{\quotep{Q}/y}\} }
  \and \\
  \inferrule* [lab=Par] {{P} \red {P}'} {{{P} | {Q}} \red {{P}' | {Q}}}
  \and
  \inferrule* [lab=Equiv]{{{P} \scong {P}'} \andalso {{P}' \red {Q}'} \andalso {{Q}' \scong {Q}}}{{P} \red {Q}}
\end{mathpar}

\begin{eqnarray*}
  match_{\equiv} (\quotep{P},\quotep{Q}) & := & P \equiv Q \\
  match_{\dagger}(\quotep{P},\quotep{Q}) & := & \forall R. P|Q \red^{*} R => R \red^{*} 0 \\
  match_{K}(\quotep{P},\quotep{Q}) & := & K \mbox{ for some context } K
\end{eqnarray*}

$u?(x)P | u!\langle Q \rangle \red P\{\quotep{Q}/x\}$

%We write $\wred$ for $\red^*$, and $P\red$ if $\exists Q $ such that $ P \red Q$.
We write $P\red$ if $\exists Q $ such that $ P \red Q$ and $P\not\red$, otherwise.

\section{Replication}

As mentioned before, it is known that replication (and hence
recursion) can be implemented in a higher-order process algebra
\cite{SangiorgiWalker}. As our first example of calculation with the
machinery thus far presented we give the construction explicitly in
the {\rhoc}.

\begin{eqnarray}
	D_{x} & := & \prefix{x}{y}{(\binpar{\outputp{x}{y}}{@{y}})} \nonumber\\
	\bangp_{x}{P} & := & \binpar{{x}!\langle{\binpar{D_{x}}{P}}\rangle}{D_{x}} \nonumber
\end{eqnarray}

\begin{eqnarray}
	\bangp_{x}{P} & & \nonumber\\
	=
	& {x}!\langle{(\prefix{x}{y}{(\outputp{x}{y} | @{y})) | P}}\rangle 
	      | \prefix{x}{y}{(\outputp{x}{y} | @{y})} & \nonumber\\
	\red
	& (\outputp{x}{y} | @{y})\substn{\quotep{(\prefix{x}{y}{(@{y} | \outputp{x}{y})) | P}}}{y} & \nonumber\\
	=
	& \outputp{x}{\quotep{(\prefix{x}{y}{(\outputp{x}{y} | @{y})) | P}}}
	  | {(\prefix{x}{y}{(\outputp{x}{y} | @{y})) | P}} & \nonumber\\
	\red
	& \ldots & \nonumber\\
	\red^*
	& P | P | \ldots & \nonumber
\end{eqnarray}

Of course, this encoding, as an implementation, runs away, unfolding
$\bangp{P}$ eagerly. A lazier and more implementable replication
operator, restricted to input-guarded processes, may be obtained as follows.

\begin{eqnarray}
\bangp{\prefix{u}{v}{P}} 
	:= 
	\binpar{\lift{x}{\prefix{u}{v}{(\binpar{D(x)}{P})}}}{D(x)} \nonumber
\end{eqnarray}

\begin{remark}
  Note that the lazier definition still does not deal with summation
  or mixed summation (i.e. sums over input and output). The reader is
  invited to construct definitions of replication that deal with these
  features. 

  Further, the definitions are parameterized in a name, $x$. Can you,
  gentle reader, make a definition that eliminates this parameter and
  guarantees no accidental interaction between the replication
  machinery and the process being replicated -- i.e. no accidental
  sharing of names used by the process to get its work done and the
  name(s) used by the replication to effect copying. This latter
  revision of the definition of replication is crucial to obtaining
  the expected identity $!!P \sim !P$.
\end{remark}

\begin{remark}\label{rem:paradoxical_combinator}
  The reader familiar with the lambda calculus will have noticed the
  similarity between $D$ and the paradoxical combinator.

  [Ed. note: the existence of this seems to suggest we have to be more
  restrictive on the set of processes and names we admit if we are to
  support no-cloning.]
\end{remark}

\subsubsection{Bisimulation}

The computational dynamics gives rise to another kind of equivalence,
the equivalence of computational behavior. As previously mentioned
this is typically captured \emph{via} some form of bisimulation.

% The notion we use in this paper is weak barbed bisimulation
% \cite{milner91polyadicpi}.

The notion we use in this paper is derived from weak barbed
bisimulation \cite{milner91polyadicpi}. 

\begin{definition}
An \emph{observation relation}, $\downarrow_{\mathcal N}$, over a set
of names, $\mathcal N$, is the smallest relation satisfying the rules
below.

\infrule[Out-barb]{y \in {\mathcal N}, \; x \nameeq y}
		  {\outputp{x}{v} \downarrow_{\mathcal N} x}
\infrule[Par-barb]{\mbox{$P\downarrow_{\mathcal N} x$ or $Q\downarrow_{\mathcal N} x$}}
		  {\binpar{P}{Q} \downarrow_{\mathcal N} x}

We write $P \Downarrow_{\mathcal N} x$ if there is $Q$ such that 
$P \wred Q$ and $Q \downarrow_{\mathcal N} x$.
\end{definition}

\begin{definition}
%\label{def.bbisim}
An  ${\mathcal N}$-\emph{barbed bisimulation} over a set of names, ${\mathcal N}$, is a symmetric binary relation 
${\mathcal S}_{\mathcal N}$ between agents such that $P\rel{S}_{\mathcal N}Q$ implies:
\begin{enumerate}
\item If $P \red P'$ then $Q \wred Q'$ and $P'\rel{S}_{\mathcal N} Q'$.
\item If $P\downarrow_{\mathcal N} x$, then $Q\Downarrow_{\mathcal N} x$.
\end{enumerate}
$P$ is ${\mathcal N}$-barbed bisimilar to $Q$, written
$P \wbbisim_{\mathcal N} Q$, if $P \rel{S}_{\mathcal N} Q$ for some ${\mathcal N}$-barbed bisimulation ${\mathcal S}_{\mathcal N}$.
\end{definition}

$\mathcal{R} \subseteq \pi \times \pi$

$P \mathcal{R} Q => \forall P'. P \red P' \Rightarrow \exists Q'. Q \red Q', P' \mathcal{R} Q'$

$P \vdash x \Rightarrow Q \vdash x$

\begin{mathpar}
  \inferrule*[lab=Out-barb]{x \nameeq y}{{y}!\langle{Q}\rangle \vdash x}
  \and
  \inferrule*[lab=Par-barb]{\mbox{$P\vdash x$ or $Q\vdash x$}}{\binpar{P}{Q} \vdash x}
\end{mathpar}

\subsubsection{Contexts}

One of the principle advantages of computational calculi like the
$\pi$-calculus is a well-defined notion of context,
contextual-equivalence and a correlation between
contextual-equivalence and notions of bisimulation. The notion of
context allows the decomposition of a process into (sub-)process and
its syntactic environment, its context. Thus, a context may be
thought of as a process with a ``hole'' (written $\Box$) in it. The
application of a context $M$ to a process $P$, written $M[P]$, is
tantamount to filling the hole in $M$ with $P$. In this paper we do
not need the full weight of this theory, but do make use of the notion
of context in the proof the main theorem. 

\begin{mathpar}
  \inferrule* [lab=summation] {} {{M_{M},M_{N}} \bc \Box \;|\; x.M_{A} \;|\; M_{M}+M_{N}}
  \and
  \inferrule* [lab=agent] {} {{M_{A}} \bc (\vec{x})M_{P} \;| \; \clift{P_0,\ldots,M_{P},\ldots,P_N}}
  \and \\
  \inferrule* [lab=process] {} {{M_{P}} \bc M_{N} \;| \;P|M_{P} }
\end{mathpar} 

\begin{mathpar}
  \inferrule* [lab=sychronization] {} {M_{N} \bc \Box \;|\; x?M_{F} \;|\; x!M_{C}}
  \and
  \inferrule* [lab=abstraction] {} {{M_{F}} \bc (x)M_{P} }
  \and
  \inferrule* [lab=concretion] {} {{M_{C}} \bc \langle M_{P} \rangle }
  \and \\
  \inferrule* [lab=process] {} {{M_{P}} \bc M_{N} \;| \;P|M_{P} }
\end{mathpar}

\begin{definition}[contextual application] Given a context $M$, and
  process $P$, we define the \emph{contextual application}, $M[P] :=
  M\{P/\Box\}$. That is, the contextual application of M to P is the
  substitution of $P$ for $\Box$ in $M$.
\end{definition}

$\meaningof{-} : L \to \mathcal{P}(\pi)$

\begin{mathpar}
  \inferrule* [lab=collection] {} {\meaningof{true} = \pi, \and \meaningof{~E} = \pi \setminus \meaningof{E}, \and \meaningof{E_{1} \& E_{2}} = \meaningof{E_{1}} \cap \meaningof{E_{2}}}
\end{mathpar}

\begin{mathpar}
  \inferrule* [lab=structure] {} {\meaningof{0} = \{ P \in \pi | P \equiv 0 \}, \and \\ \meaningof{E_1 | E_2} = \{ P \in \pi | P \equiv P_{1} | P_{2}, P_{1} \in \meaningof{E_{1}}, P_{2} \in \meaningof{E_2}\} }
\end{mathpar}

\begin{mathpar}
 \inferrule* [lab=behavior] {} {\meaningof{\langle a?b \rangle E} = \{ P \in \pi | P \equiv Q | u?(y)P', \\ \and \\\\ \and \\ \;\;\; u \in \meaningof{a}, \forall z.P'\{z/y\} \in \meaningof{E\{z/b\}}\}, \and \\ \meaningof{a!E} = \{ P \in \pi | P \equiv Q | x!\langle P' \rangle, x \in \meaningof{a} P' \in \meaningof{E}\} }
\end{mathpar}

\begin{mathpar}
 \inferrule* [lab=nominal] {} {\meaningof{\quotep{E}} = \{ \quotep{P} \in \quotep{\pi} | P \in \meaningof{E} \}, \and \meaningof{\quotep{P}} = \{ \quotep{Q} \in \quotep{\pi} | P \equiv Q \} \and \\ \meaningof{@\quotep{E}} = \{ P \in \pi | P \equiv @x, x \in \meaningof{E} \}}
\end{mathpar}

\begin{eqnarray*}
  \\
  \meaningof{-} : TS \to ST
\end{eqnarray*}

\begin{eqnarray*}
  \\
  L : TS \to ST
\end{eqnarray*}

\begin{eqnarray*}
  \\
  P \models E \iff P \in \meaningof{E}
\end{eqnarray*}

\begin{eqnarray*}
  P \approx_{L} Q \iff \forall E \in L. P \models E \iff Q \models E
\end{eqnarray*}

\begin{eqnarray*}
  P \approx_{K} Q
\end{eqnarray*}

\begin{eqnarray*}
  P \approx Q
\end{eqnarray*}

$\approx_{K} = \approx = \approx_{L}$

\subsubsection{Contextual duality}

Note that contexts extend the quotation operation to a family of
operations from processes to names. Given a context, $M$, we can
define a \emph{nominal context}, $\quotep{M}$ by $\quotep{M}[P] :=
\quotep{M[P]}$. To foreshadow what is to come we observe that these
operations enjoy a duality with processes very much like the duality
between vectors and maps from vectors to scalars.

Further, because the calculus is essentially higher-order, we have a
correspondence between contexts and processes. More specifically,
given a name $x$ and a context $M$ we can construct $M^{*}_{x}$ such
that 

\begin{mathpar}
  M^{*}_{x} | \lift{x}{P} \red M[P]
\end{mathpar}

namely,

\begin{mathpar}
  M^{*}_{x} := x?(u).M[\dropn{u}]
\end{mathpar}

The dependence of $M^{*}_{x}$ on a name makes it an abstraction, 

\begin{mathpar}
  M^{*} := (x)x?(u).M[\dropn{u}]
\end{mathpar}

\subsection{Additional notation}

It will sometimes be convenient to denote the process a name
quotes. We already have the notation $x = \quotep{P}$, but it will be
convenient to introduce an alternate notation, $\procn{x}$, when we
want to emphasize the connection to the use of the name. Note that, by
virtue of name equivalence, $\quotep{\procn{x}} \nameeq x$; so, the
notation is consistent with previous definitions.

Further, because names have structure it is possible to effect
substitutions on the basis of that structure. This means we need to
upgrade our notation for substitutions, which we accomplish by
adapting comprehension notation. Thus,

\begin{mathpar}
  P\{ y / x : x \in S \}
\end{mathpar}

is interpreted to mean the process derived from P by replacing (in a
capture-avoiding manner) each occurrence of $x$ in $S$ by $y$. For example,

\begin{mathpar}
  P\{ \quotep{\procn{x}|\procn{x}} / x : x \in \freenames{P} \}
\end{mathpar}

will replace each (occurrence) of a free name $x$ in $P$ by
$\quotep{\procn{x}|\procn{x}}$.

Also, we will avail ourselves of the notation $x^{L}$ and $x^{R}$ to
denote injections of a name into disjoint copies of the name
space. There are numerous ways to accomplish this. One example can be
found in \cite{MeredithR05}. This notation overloads to vectors of
names: $\vec{x}^{\pi} := (x_{i}^{\pi} \; : \; 0 \leq i < |\vec{x}| )$ where $\pi \in \{L,R\}$.

We also use $P^{\Box} := P|\Box$.

In \cite{MeredithR05} an interpretation of the new operator is
given. It turns out that there are several possible interpretations
all enjoying the requisite algebraic properties of the operator (see
\cite{milner91polyadicpi}). We will therefore make liberal use of
$(\nu\; \vec{x})P$.

% subsection the_syntax_and_semantics_of_the_notation_system (end)   

\input{qm2pi.qmops} 

\input{qm2pi.sterngerlach} 

\input{qm2pi.metric} 

% section concurrent_process_calculi (end)

%\input{qm2pi.proofsketch}

% section proof sketch (end)

%\input{qm2pi.slviaknots} 

% section spatial logic via knots (end)

\input{qm2pi.conclusion}

% section conclusion (end)

%\input{qm2pi.dtcodes} 

% section wiring algorithm (end)

\input{qm2pi.ack} 

% section acknowledgments (end)

\newpage


\bibliographystyle{plain}   
\bibliography{../../biblios/main.bib}

\input{qm2pi.rhodetails}

\end{document}

 

\documentclass[12pt]{llncs}
%\documentclass{jktr}

\usepackage[pdftex]{hyperref}                   
\usepackage {listings}
\usepackage {mathpartir}
\usepackage{bcprules}
%\usepackage{listings}
                       
\usepackage{graphicx} 
%\usepackage[margins=2.5cm,nohead,nofoot]{geometry}
%\usepackage{geometry}
\usepackage{amsfonts}
\usepackage{amstext}
\usepackage{latexsym}
\usepackage{amssymb}
\usepackage{color}


%\include{myPreamble}
\include{qm2pi.local} 

%\ifpdf
%\usepackage[pdftex]{graphicx}
%\else
%\usepackage{graphicx}
%\fi

 % \ifpdf
%  \usepackage{pdfsync}
%  \if


%\title{Brief Article}
%\author{David F. Snyder}
%\author{L.G. Meredith}

%\address{Dept. of Math., Texas State University--San Marcos, San Marcos, TX 78666}
       
\pagestyle{empty}


\begin{document}

\lstset{language=[Objective]Caml,frame=shadowbox}

\input{qm2pi.front}

% section front matter (end)

\input{qm2pi.intro} 
 
% section introduction (end)

% \input{qm2pi.knotations} 

% section notation (end)

\input{qm2pi.process.calculi} 

% section concurrent_process_calculi_and_spatial_logics_ (end)
    
%\input{qm2pi.knots2pi} 

%\input{qm2pi.trefoil} 

%\input{qm2pi.mainthm} 

% subsection basic_interpretation (end)

%\input{qm2pi.rho.presentation} 
\subsection{The syntax and semantics of the notation system}\label{sub:the_syntax_and_semantics_of_the_notation_system} % (fold)

We now summarize a technical presentation of the calculus that
embodies our theory of dynamics. The typical presentation of such a
calculus follows the style of giving generators and relations on
them. The grammar, below, describing term constructors, freely
generates the set of processes, $\Proc$. This set is then quotiented
by a relation known as structural congruence and it is over this set
that the notion of dynamics is expressed. This presentation is
essentially that of \cite{MeredithR05} with the addition of
polyadicity and summation. For readability we have relegated some of
the technical subtleties to an appendix.

\subsubsection{Process grammar}\label{subsub:process_grammar}

\begin{mathpar}
  \inferrule* [lab=synchronization] {} {{M} \bc \pzero \;|\; x?F \;|\; x!C }
  \and
  \inferrule* [lab=abstraction] {} {{F} \bc (x)P}
  \and
  \inferrule* [lab=concretion] {} {{C} \bc \langle Q \rangle}
  \and
  \inferrule* [lab=process] {} {{P,Q} \bc M \;| \;P|Q \;|\; @{x}}
  \and
  \inferrule* [lab=name] {} {{x} \bc \quotep{P}}
\end{mathpar} 

Note that $\vec{x}$ (resp. $\vec{P}$) denotes a vector of names
(resp. processes) of length $|\vec{x}|$ (resp. $|\vec{P}|$). We adopt
the following useful abbreviations.

\begin{mathpar}
   x?(\vec{y}).P := x.(\vec{y})P \and  x\clift{\vec{P}} := x.\clift{\vec{P}}
   \and x!(y) := \lift{x}{\dropn{y}}
   \and \Pi_{i=0}^{n-1}P_i := P_0 | \ldots | P_{n-1}
\end{mathpar}

\subsubsection{Structural congruence}

\paragraph{Free and bound names and alpha-equivalence.} At the
core of structural equivalence is alpha-equivalence which identifies
process that are the same up to a change of variable. Formally, we
recognize the distinction between free and bound names. The free names
of a process, $\freenames{P}$, may be calculated recursively as
follows:

\begin{mathpar}
\freenames{\pzero} := \emptyset
  \and \\
  \freenames{x?(y).P} := \{ x \} \cup (\freenames{P} \setminus \{ y \})
  \and 
  \freenames{x!\langle P \rangle} := \{ x \} \cup \{ P \} 
  \and \\
  \freenames{P|Q} := \freenames{P} \cup \freenames{Q}
  \and \\
  \freenames{@{x}} := \{ x \}
\end{mathpar}

$\pi$
$\quotep{\pi}$

$\freenames{-} : \pi \to \mathcal{P}(\quotep{\pi})$

\begin{eqnarray*}
  \freenames{\pzero} & := & \emptyset \\
  \freenames{x?(y).P} & := & \{ x \} \cup (\freenames{P} \setminus \{ y \}) \\
  \freenames{x!\langle P \rangle} & := & \{ x \} \cup \{ P \} \\
  \freenames{P|Q} & := & \freenames{P} \cup \freenames{Q} \\
  \freenames{\dropn{x}} & := & \{ x \}
\end{eqnarray*}

The bound names of a process, $\boundnames{P}$, are those names occurring in $P$
that are not free. For example, in $x?(y).0$, the name $x$ is free, while $y$ is bound.

\begin{mathpar}
  \inferrule* [lab=monoidal-laws] {} { P|Q \equiv Q|P \and P|0 \equiv P \and P|(Q|R) \equiv (P|Q)|R }
\end{mathpar}

\begin{mathpar}
  \inferrule* [lab=alpha-equivalence] {} { (x)P \equiv (y)P\{y/x\} \and y \not\in \freenames{P} }
\end{mathpar}

\begin{definition}
Then two processes, $P,Q$, are alpha-equivalent if $P = Q\{\vec{y}/\vec{x}\}$ for
some $\vec{x} \in \boundnames{Q},\vec{y} \in \boundnames{P}$, where $Q\{\vec{y}/\vec{x}\}$
denotes the capture-avoiding substitution of $\vec{y}$ for $\vec{x}$ in $Q$.
\end{definition}

\begin{definition}
  The {\em structural congruence} \cite{SangiorgiWalker} , $\equiv$,
  between processes is the least congruence containing
  alpha-equivalence, satisfying the abelian monoid laws
  (associativity, commutativity and $\pzero$ as identity) for parallel
  composition $|$ and for summation $+$.
\end{definition}

\subsection{Name equivalence}

We take name equivalence, written $\nameeq$, to be the smallest
equivalence relation generated by the following rules.

\begin{mathpar}
\inferrule*[lab=Quote-drop]
{ }
{ \quotep{@{x}} \nameeq x }

\inferrule*[lab=Struct-equiv]
{ P \scong Q }
{ \quotep{P} \nameeq \quotep{Q} }
\end{mathpar}

The astute reader will have noticed that the mutual recursion of names
and processes imposes a mutual recursion on alpha-equivalence and
structural equivalence via name-equivalence. Fortunately, all of this
works out pleasantly and we may calculate in the natural way, free of
concern. The reader interested in the details is referred to the
appendix \ref{appendix:rho_details}.

\subsection{Substitution}

We use $\Proc$ for the set of processes, $\QProc$ for the set of
names, and $\id{\{}\vec{y} / \vec{x} \id{\}}$ to denote partial maps,
$s : \QProc \rightarrow \QProc$. A map, $s$ lifts, uniquely, to a map
on process terms, $\widehat{s} : \Proc \rightarrow \Proc$ by the
following equations.

\begin{mathpar}
  (0) \psubstp{Q}{P} := 0 \\
  (R \juxtap S) \psubstp{Q}{P}
  :=    
  (R)\psubstp{Q}{P} \juxtap (S) \psubstp{Q}{P} \\
  (x?(y).R) \psubstp{Q}{P}    
  :=    
  (x)\substp{Q}{P} (z)\concat( (R \psubstn{z}{y}) \psubstp{Q}{P} ) \\
  (\lift{x}{R}) \psubstp{Q}{P}  
  :=
  \lift{(x)\substp{Q}{P}}{ R \psubstp{Q}{P} } \\
%   (\dropn{x})  \psubstp{Q}{P}       
%   := 
%   \left\{ 
%     \begin{array}{ccc} 
%       \dropn{\quotep{Q}} & & x \nameeq \quotep{P} \\
%       \dropn{x} & & otherwise \\
%     \end{array}
%   \right. 
  (\dropn{x})  \psubstp{Q}{P}       
  := 
  \left\{ 
    \begin{array}{ccc} 
      Q & & x \nameeq \quotep{P} \\
      \dropn{x} & & otherwise \\
    \end{array}
  \right.
\end{mathpar}
 

where

\begin{eqnarray}
  (x)\id{\{} \lpquote Q \rpquote / \lpquote P \rpquote \id{\}}            = 
  \left\{ 
    \begin{array}{ccc}
      \lpquote Q \rpquote & & x \nameeq \lpquote P \rpquote \\
      x & & otherwise \\
    \end{array}
  \right. \nonumber
\end{eqnarray}

and $z$ is chosen distinct from $\quotep{P}$, $\quotep{Q}$, the free
names in $Q$, and all the names in $R$. Our $\alpha$-equivalence will
be built in the standard way from this substitution.

\begin{remark}\label{rem:no_self_referential_names}
  One consequence of these definitions is that $\forall P. \quotep{P}
  \not\in \freenames{P}$.
\end{remark}

\subsection{ Dynamic quote: an example }

Anticipating something of what's to come, consider applying the
substitution, $\widehat{\id{\{}u / z \id{\}}}$, to the following pair
of processes, $\lift{w}{y!(z)}$ and $w[ \lpquote y!(z) \rpquote ]$.

\begin{eqnarray}
	\lift{w}{y!(z)}\widehat{\id{\{}u / z \id{\}}}
		& = &
		\lift{w}{y!(u)} \nonumber\\
	w[ \lpquote y!(z) \rpquote ] \widehat{ \id{\{}u / z \id{\}} }
		& = &
		w[ \lpquote y!(z) \rpquote ] \nonumber
\end{eqnarray}

Because the body of the process between quotes is impervious to
substitution, we get radically different answers. In fact, by
examining the first process in an input context,
e.g. $x?(z).\lift{w}{y!(z)}$, we see that the process under the lift
operator may be shaped by prefixed inputs binding a name inside it. In
this sense, the lift operator will be seen as a way to dynamically
construct processes before reifying them as names.

Finally equipped with these standard features we can present the
dynamics of the calculus.

\subsubsection{Operational semantics} 

Finally, we introduce the computational dynamics. What marks these
algebras as distinct from other more traditionally studied algebraic
structures, e.g. vector spaces or polynomial rings, is the manner in
which dynamics is captured. In traditional structures, dynamics is typically
expressed through morphisms between such structures, as in linear maps
between vector spaces or morphisms between rings. In algebras
associated with the semantics of computation, the dynamics is
expressed as part of the algebraic structure itself, through a
reduction reduction relation typically denoted by $\red$. Below, we
give a recursive presentation of this relation for the calculus used
in the encoding.

$\red \subseteq \pi \times \pi$
$\red : \pi \to \mathcal{P}(\pi)$

\begin{mathpar}
  \inferrule* [lab=Comm] { \textsf{match}( x_{src}, x_{trgt} ) } { x_{trgt}?(y)P \; | \; x_{src}!\langle {Q} \rangle \red P\{\quotep{Q}/y}\} }
  \and \\
  \inferrule* [lab=Par] {{P} \red {P}'} {{{P} | {Q}} \red {{P}' | {Q}}}
  \and
  \inferrule* [lab=Equiv]{{{P} \scong {P}'} \andalso {{P}' \red {Q}'} \andalso {{Q}' \scong {Q}}}{{P} \red {Q}}
\end{mathpar}

\begin{eqnarray*}
  match_{\equiv} (\quotep{P},\quotep{Q}) & := & P \equiv Q \\
  match_{\dagger}(\quotep{P},\quotep{Q}) & := & \forall R. P|Q \red^{*} R => R \red^{*} 0 \\
  match_{K}(\quotep{P},\quotep{Q}) & := & K \mbox{ for some context } K
\end{eqnarray*}

$u?(x)P | u!\langle Q \rangle \red P\{\quotep{Q}/x\}$

%We write $\wred$ for $\red^*$, and $P\red$ if $\exists Q $ such that $ P \red Q$.
We write $P\red$ if $\exists Q $ such that $ P \red Q$ and $P\not\red$, otherwise.

\section{Replication}

As mentioned before, it is known that replication (and hence
recursion) can be implemented in a higher-order process algebra
\cite{SangiorgiWalker}. As our first example of calculation with the
machinery thus far presented we give the construction explicitly in
the {\rhoc}.

\begin{eqnarray}
	D_{x} & := & \prefix{x}{y}{(\binpar{\outputp{x}{y}}{@{y}})} \nonumber\\
	\bangp_{x}{P} & := & \binpar{{x}!\langle{\binpar{D_{x}}{P}}\rangle}{D_{x}} \nonumber
\end{eqnarray}

\begin{eqnarray}
	\bangp_{x}{P} & & \nonumber\\
	=
	& {x}!\langle{(\prefix{x}{y}{(\outputp{x}{y} | @{y})) | P}}\rangle 
	      | \prefix{x}{y}{(\outputp{x}{y} | @{y})} & \nonumber\\
	\red
	& (\outputp{x}{y} | @{y})\substn{\quotep{(\prefix{x}{y}{(@{y} | \outputp{x}{y})) | P}}}{y} & \nonumber\\
	=
	& \outputp{x}{\quotep{(\prefix{x}{y}{(\outputp{x}{y} | @{y})) | P}}}
	  | {(\prefix{x}{y}{(\outputp{x}{y} | @{y})) | P}} & \nonumber\\
	\red
	& \ldots & \nonumber\\
	\red^*
	& P | P | \ldots & \nonumber
\end{eqnarray}

Of course, this encoding, as an implementation, runs away, unfolding
$\bangp{P}$ eagerly. A lazier and more implementable replication
operator, restricted to input-guarded processes, may be obtained as follows.

\begin{eqnarray}
\bangp{\prefix{u}{v}{P}} 
	:= 
	\binpar{\lift{x}{\prefix{u}{v}{(\binpar{D(x)}{P})}}}{D(x)} \nonumber
\end{eqnarray}

\begin{remark}
  Note that the lazier definition still does not deal with summation
  or mixed summation (i.e. sums over input and output). The reader is
  invited to construct definitions of replication that deal with these
  features. 

  Further, the definitions are parameterized in a name, $x$. Can you,
  gentle reader, make a definition that eliminates this parameter and
  guarantees no accidental interaction between the replication
  machinery and the process being replicated -- i.e. no accidental
  sharing of names used by the process to get its work done and the
  name(s) used by the replication to effect copying. This latter
  revision of the definition of replication is crucial to obtaining
  the expected identity $!!P \sim !P$.
\end{remark}

\begin{remark}\label{rem:paradoxical_combinator}
  The reader familiar with the lambda calculus will have noticed the
  similarity between $D$ and the paradoxical combinator.

  [Ed. note: the existence of this seems to suggest we have to be more
  restrictive on the set of processes and names we admit if we are to
  support no-cloning.]
\end{remark}

\subsubsection{Bisimulation}

The computational dynamics gives rise to another kind of equivalence,
the equivalence of computational behavior. As previously mentioned
this is typically captured \emph{via} some form of bisimulation.

% The notion we use in this paper is weak barbed bisimulation
% \cite{milner91polyadicpi}.

The notion we use in this paper is derived from weak barbed
bisimulation \cite{milner91polyadicpi}. 

\begin{definition}
An \emph{observation relation}, $\downarrow_{\mathcal N}$, over a set
of names, $\mathcal N$, is the smallest relation satisfying the rules
below.

\infrule[Out-barb]{y \in {\mathcal N}, \; x \nameeq y}
		  {\outputp{x}{v} \downarrow_{\mathcal N} x}
\infrule[Par-barb]{\mbox{$P\downarrow_{\mathcal N} x$ or $Q\downarrow_{\mathcal N} x$}}
		  {\binpar{P}{Q} \downarrow_{\mathcal N} x}

We write $P \Downarrow_{\mathcal N} x$ if there is $Q$ such that 
$P \wred Q$ and $Q \downarrow_{\mathcal N} x$.
\end{definition}

\begin{definition}
%\label{def.bbisim}
An  ${\mathcal N}$-\emph{barbed bisimulation} over a set of names, ${\mathcal N}$, is a symmetric binary relation 
${\mathcal S}_{\mathcal N}$ between agents such that $P\rel{S}_{\mathcal N}Q$ implies:
\begin{enumerate}
\item If $P \red P'$ then $Q \wred Q'$ and $P'\rel{S}_{\mathcal N} Q'$.
\item If $P\downarrow_{\mathcal N} x$, then $Q\Downarrow_{\mathcal N} x$.
\end{enumerate}
$P$ is ${\mathcal N}$-barbed bisimilar to $Q$, written
$P \wbbisim_{\mathcal N} Q$, if $P \rel{S}_{\mathcal N} Q$ for some ${\mathcal N}$-barbed bisimulation ${\mathcal S}_{\mathcal N}$.
\end{definition}

$\mathcal{R} \subseteq \pi \times \pi$

$P \mathcal{R} Q => \forall P'. P \red P' \Rightarrow \exists Q'. Q \red Q', P' \mathcal{R} Q'$

$P \vdash x \Rightarrow Q \vdash x$

\begin{mathpar}
  \inferrule*[lab=Out-barb]{x \nameeq y}{{y}!\langle{Q}\rangle \vdash x}
  \and
  \inferrule*[lab=Par-barb]{\mbox{$P\vdash x$ or $Q\vdash x$}}{\binpar{P}{Q} \vdash x}
\end{mathpar}

\subsubsection{Contexts}

One of the principle advantages of computational calculi like the
$\pi$-calculus is a well-defined notion of context,
contextual-equivalence and a correlation between
contextual-equivalence and notions of bisimulation. The notion of
context allows the decomposition of a process into (sub-)process and
its syntactic environment, its context. Thus, a context may be
thought of as a process with a ``hole'' (written $\Box$) in it. The
application of a context $M$ to a process $P$, written $M[P]$, is
tantamount to filling the hole in $M$ with $P$. In this paper we do
not need the full weight of this theory, but do make use of the notion
of context in the proof the main theorem. 

\begin{mathpar}
  \inferrule* [lab=summation] {} {{M_{M},M_{N}} \bc \Box \;|\; x.M_{A} \;|\; M_{M}+M_{N}}
  \and
  \inferrule* [lab=agent] {} {{M_{A}} \bc (\vec{x})M_{P} \;| \; \clift{P_0,\ldots,M_{P},\ldots,P_N}}
  \and \\
  \inferrule* [lab=process] {} {{M_{P}} \bc M_{N} \;| \;P|M_{P} }
\end{mathpar} 

\begin{mathpar}
  \inferrule* [lab=sychronization] {} {M_{N} \bc \Box \;|\; x?M_{F} \;|\; x!M_{C}}
  \and
  \inferrule* [lab=abstraction] {} {{M_{F}} \bc (x)M_{P} }
  \and
  \inferrule* [lab=concretion] {} {{M_{C}} \bc \langle M_{P} \rangle }
  \and \\
  \inferrule* [lab=process] {} {{M_{P}} \bc M_{N} \;| \;P|M_{P} }
\end{mathpar}

\begin{definition}[contextual application] Given a context $M$, and
  process $P$, we define the \emph{contextual application}, $M[P] :=
  M\{P/\Box\}$. That is, the contextual application of M to P is the
  substitution of $P$ for $\Box$ in $M$.
\end{definition}

$\meaningof{-} : L \to \mathcal{P}(\pi)$

\begin{mathpar}
  \inferrule* [lab=collection] {} {\meaningof{true} = \pi, \and \meaningof{~E} = \pi \setminus \meaningof{E}, \and \meaningof{E_{1} \& E_{2}} = \meaningof{E_{1}} \cap \meaningof{E_{2}}}
\end{mathpar}

\begin{mathpar}
  \inferrule* [lab=structure] {} {\meaningof{0} = \{ P \in \pi | P \equiv 0 \}, \and \\ \meaningof{E_1 | E_2} = \{ P \in \pi | P \equiv P_{1} | P_{2}, P_{1} \in \meaningof{E_{1}}, P_{2} \in \meaningof{E_2}\} }
\end{mathpar}

\begin{mathpar}
 \inferrule* [lab=behavior] {} {\meaningof{\langle a?b \rangle E} = \{ P \in \pi | P \equiv Q | u?(y)P', \\ \and \\\\ \and \\ \;\;\; u \in \meaningof{a}, \forall z.P'\{z/y\} \in \meaningof{E\{z/b\}}\}, \and \\ \meaningof{a!E} = \{ P \in \pi | P \equiv Q | x!\langle P' \rangle, x \in \meaningof{a} P' \in \meaningof{E}\} }
\end{mathpar}

\begin{mathpar}
 \inferrule* [lab=nominal] {} {\meaningof{\quotep{E}} = \{ \quotep{P} \in \quotep{\pi} | P \in \meaningof{E} \}, \and \meaningof{\quotep{P}} = \{ \quotep{Q} \in \quotep{\pi} | P \equiv Q \} \and \\ \meaningof{@\quotep{E}} = \{ P \in \pi | P \equiv @x, x \in \meaningof{E} \}}
\end{mathpar}

\begin{eqnarray*}
  \\
  \meaningof{-} : TS \to ST
\end{eqnarray*}

\begin{eqnarray*}
  \\
  L : TS \to ST
\end{eqnarray*}

\begin{eqnarray*}
  \\
  P \models E \iff P \in \meaningof{E}
\end{eqnarray*}

\begin{eqnarray*}
  P \approx_{L} Q \iff \forall E \in L. P \models E \iff Q \models E
\end{eqnarray*}

\begin{eqnarray*}
  P \approx_{K} Q
\end{eqnarray*}

\begin{eqnarray*}
  P \approx Q
\end{eqnarray*}

$\approx_{K} = \approx = \approx_{L}$

\subsubsection{Contextual duality}

Note that contexts extend the quotation operation to a family of
operations from processes to names. Given a context, $M$, we can
define a \emph{nominal context}, $\quotep{M}$ by $\quotep{M}[P] :=
\quotep{M[P]}$. To foreshadow what is to come we observe that these
operations enjoy a duality with processes very much like the duality
between vectors and maps from vectors to scalars.

Further, because the calculus is essentially higher-order, we have a
correspondence between contexts and processes. More specifically,
given a name $x$ and a context $M$ we can construct $M^{*}_{x}$ such
that 

\begin{mathpar}
  M^{*}_{x} | \lift{x}{P} \red M[P]
\end{mathpar}

namely,

\begin{mathpar}
  M^{*}_{x} := x?(u).M[\dropn{u}]
\end{mathpar}

The dependence of $M^{*}_{x}$ on a name makes it an abstraction, 

\begin{mathpar}
  M^{*} := (x)x?(u).M[\dropn{u}]
\end{mathpar}

\subsection{Additional notation}

It will sometimes be convenient to denote the process a name
quotes. We already have the notation $x = \quotep{P}$, but it will be
convenient to introduce an alternate notation, $\procn{x}$, when we
want to emphasize the connection to the use of the name. Note that, by
virtue of name equivalence, $\quotep{\procn{x}} \nameeq x$; so, the
notation is consistent with previous definitions.

Further, because names have structure it is possible to effect
substitutions on the basis of that structure. This means we need to
upgrade our notation for substitutions, which we accomplish by
adapting comprehension notation. Thus,

\begin{mathpar}
  P\{ y / x : x \in S \}
\end{mathpar}

is interpreted to mean the process derived from P by replacing (in a
capture-avoiding manner) each occurrence of $x$ in $S$ by $y$. For example,

\begin{mathpar}
  P\{ \quotep{\procn{x}|\procn{x}} / x : x \in \freenames{P} \}
\end{mathpar}

will replace each (occurrence) of a free name $x$ in $P$ by
$\quotep{\procn{x}|\procn{x}}$.

Also, we will avail ourselves of the notation $x^{L}$ and $x^{R}$ to
denote injections of a name into disjoint copies of the name
space. There are numerous ways to accomplish this. One example can be
found in \cite{MeredithR05}. This notation overloads to vectors of
names: $\vec{x}^{\pi} := (x_{i}^{\pi} \; : \; 0 \leq i < |\vec{x}| )$ where $\pi \in \{L,R\}$.

We also use $P^{\Box} := P|\Box$.

In \cite{MeredithR05} an interpretation of the new operator is
given. It turns out that there are several possible interpretations
all enjoying the requisite algebraic properties of the operator (see
\cite{milner91polyadicpi}). We will therefore make liberal use of
$(\nu\; \vec{x})P$.

% subsection the_syntax_and_semantics_of_the_notation_system (end)   

\input{qm2pi.qmops} 

\input{qm2pi.sterngerlach} 

\input{qm2pi.metric} 

% section concurrent_process_calculi (end)

%\input{qm2pi.proofsketch}

% section proof sketch (end)

%\input{qm2pi.slviaknots} 

% section spatial logic via knots (end)

\input{qm2pi.conclusion}

% section conclusion (end)

%\input{qm2pi.dtcodes} 

% section wiring algorithm (end)

\input{qm2pi.ack} 

% section acknowledgments (end)

\newpage


\bibliographystyle{plain}   
\bibliography{../../biblios/main.bib}

\input{qm2pi.rhodetails}

\end{document}

 

% section concurrent_process_calculi (end)

%\documentclass[12pt]{llncs}
%\documentclass{jktr}

\usepackage[pdftex]{hyperref}                   
\usepackage {listings}
\usepackage {mathpartir}
\usepackage{bcprules}
%\usepackage{listings}
                       
\usepackage{graphicx} 
%\usepackage[margins=2.5cm,nohead,nofoot]{geometry}
%\usepackage{geometry}
\usepackage{amsfonts}
\usepackage{amstext}
\usepackage{latexsym}
\usepackage{amssymb}
\usepackage{color}


%\include{myPreamble}
\include{qm2pi.local} 

%\ifpdf
%\usepackage[pdftex]{graphicx}
%\else
%\usepackage{graphicx}
%\fi

 % \ifpdf
%  \usepackage{pdfsync}
%  \if


%\title{Brief Article}
%\author{David F. Snyder}
%\author{L.G. Meredith}

%\address{Dept. of Math., Texas State University--San Marcos, San Marcos, TX 78666}
       
\pagestyle{empty}


\begin{document}

\lstset{language=[Objective]Caml,frame=shadowbox}

\input{qm2pi.front}

% section front matter (end)

\input{qm2pi.intro} 
 
% section introduction (end)

% \input{qm2pi.knotations} 

% section notation (end)

\input{qm2pi.process.calculi} 

% section concurrent_process_calculi_and_spatial_logics_ (end)
    
%\input{qm2pi.knots2pi} 

%\input{qm2pi.trefoil} 

%\input{qm2pi.mainthm} 

% subsection basic_interpretation (end)

%\input{qm2pi.rho.presentation} 
\subsection{The syntax and semantics of the notation system}\label{sub:the_syntax_and_semantics_of_the_notation_system} % (fold)

We now summarize a technical presentation of the calculus that
embodies our theory of dynamics. The typical presentation of such a
calculus follows the style of giving generators and relations on
them. The grammar, below, describing term constructors, freely
generates the set of processes, $\Proc$. This set is then quotiented
by a relation known as structural congruence and it is over this set
that the notion of dynamics is expressed. This presentation is
essentially that of \cite{MeredithR05} with the addition of
polyadicity and summation. For readability we have relegated some of
the technical subtleties to an appendix.

\subsubsection{Process grammar}\label{subsub:process_grammar}

\begin{mathpar}
  \inferrule* [lab=synchronization] {} {{M} \bc \pzero \;|\; x?F \;|\; x!C }
  \and
  \inferrule* [lab=abstraction] {} {{F} \bc (x)P}
  \and
  \inferrule* [lab=concretion] {} {{C} \bc \langle Q \rangle}
  \and
  \inferrule* [lab=process] {} {{P,Q} \bc M \;| \;P|Q \;|\; @{x}}
  \and
  \inferrule* [lab=name] {} {{x} \bc \quotep{P}}
\end{mathpar} 

Note that $\vec{x}$ (resp. $\vec{P}$) denotes a vector of names
(resp. processes) of length $|\vec{x}|$ (resp. $|\vec{P}|$). We adopt
the following useful abbreviations.

\begin{mathpar}
   x?(\vec{y}).P := x.(\vec{y})P \and  x\clift{\vec{P}} := x.\clift{\vec{P}}
   \and x!(y) := \lift{x}{\dropn{y}}
   \and \Pi_{i=0}^{n-1}P_i := P_0 | \ldots | P_{n-1}
\end{mathpar}

\subsubsection{Structural congruence}

\paragraph{Free and bound names and alpha-equivalence.} At the
core of structural equivalence is alpha-equivalence which identifies
process that are the same up to a change of variable. Formally, we
recognize the distinction between free and bound names. The free names
of a process, $\freenames{P}$, may be calculated recursively as
follows:

\begin{mathpar}
\freenames{\pzero} := \emptyset
  \and \\
  \freenames{x?(y).P} := \{ x \} \cup (\freenames{P} \setminus \{ y \})
  \and 
  \freenames{x!\langle P \rangle} := \{ x \} \cup \{ P \} 
  \and \\
  \freenames{P|Q} := \freenames{P} \cup \freenames{Q}
  \and \\
  \freenames{@{x}} := \{ x \}
\end{mathpar}

$\pi$
$\quotep{\pi}$

$\freenames{-} : \pi \to \mathcal{P}(\quotep{\pi})$

\begin{eqnarray*}
  \freenames{\pzero} & := & \emptyset \\
  \freenames{x?(y).P} & := & \{ x \} \cup (\freenames{P} \setminus \{ y \}) \\
  \freenames{x!\langle P \rangle} & := & \{ x \} \cup \{ P \} \\
  \freenames{P|Q} & := & \freenames{P} \cup \freenames{Q} \\
  \freenames{\dropn{x}} & := & \{ x \}
\end{eqnarray*}

The bound names of a process, $\boundnames{P}$, are those names occurring in $P$
that are not free. For example, in $x?(y).0$, the name $x$ is free, while $y$ is bound.

\begin{mathpar}
  \inferrule* [lab=monoidal-laws] {} { P|Q \equiv Q|P \and P|0 \equiv P \and P|(Q|R) \equiv (P|Q)|R }
\end{mathpar}

\begin{mathpar}
  \inferrule* [lab=alpha-equivalence] {} { (x)P \equiv (y)P\{y/x\} \and y \not\in \freenames{P} }
\end{mathpar}

\begin{definition}
Then two processes, $P,Q$, are alpha-equivalent if $P = Q\{\vec{y}/\vec{x}\}$ for
some $\vec{x} \in \boundnames{Q},\vec{y} \in \boundnames{P}$, where $Q\{\vec{y}/\vec{x}\}$
denotes the capture-avoiding substitution of $\vec{y}$ for $\vec{x}$ in $Q$.
\end{definition}

\begin{definition}
  The {\em structural congruence} \cite{SangiorgiWalker} , $\equiv$,
  between processes is the least congruence containing
  alpha-equivalence, satisfying the abelian monoid laws
  (associativity, commutativity and $\pzero$ as identity) for parallel
  composition $|$ and for summation $+$.
\end{definition}

\subsection{Name equivalence}

We take name equivalence, written $\nameeq$, to be the smallest
equivalence relation generated by the following rules.

\begin{mathpar}
\inferrule*[lab=Quote-drop]
{ }
{ \quotep{@{x}} \nameeq x }

\inferrule*[lab=Struct-equiv]
{ P \scong Q }
{ \quotep{P} \nameeq \quotep{Q} }
\end{mathpar}

The astute reader will have noticed that the mutual recursion of names
and processes imposes a mutual recursion on alpha-equivalence and
structural equivalence via name-equivalence. Fortunately, all of this
works out pleasantly and we may calculate in the natural way, free of
concern. The reader interested in the details is referred to the
appendix \ref{appendix:rho_details}.

\subsection{Substitution}

We use $\Proc$ for the set of processes, $\QProc$ for the set of
names, and $\id{\{}\vec{y} / \vec{x} \id{\}}$ to denote partial maps,
$s : \QProc \rightarrow \QProc$. A map, $s$ lifts, uniquely, to a map
on process terms, $\widehat{s} : \Proc \rightarrow \Proc$ by the
following equations.

\begin{mathpar}
  (0) \psubstp{Q}{P} := 0 \\
  (R \juxtap S) \psubstp{Q}{P}
  :=    
  (R)\psubstp{Q}{P} \juxtap (S) \psubstp{Q}{P} \\
  (x?(y).R) \psubstp{Q}{P}    
  :=    
  (x)\substp{Q}{P} (z)\concat( (R \psubstn{z}{y}) \psubstp{Q}{P} ) \\
  (\lift{x}{R}) \psubstp{Q}{P}  
  :=
  \lift{(x)\substp{Q}{P}}{ R \psubstp{Q}{P} } \\
%   (\dropn{x})  \psubstp{Q}{P}       
%   := 
%   \left\{ 
%     \begin{array}{ccc} 
%       \dropn{\quotep{Q}} & & x \nameeq \quotep{P} \\
%       \dropn{x} & & otherwise \\
%     \end{array}
%   \right. 
  (\dropn{x})  \psubstp{Q}{P}       
  := 
  \left\{ 
    \begin{array}{ccc} 
      Q & & x \nameeq \quotep{P} \\
      \dropn{x} & & otherwise \\
    \end{array}
  \right.
\end{mathpar}
 

where

\begin{eqnarray}
  (x)\id{\{} \lpquote Q \rpquote / \lpquote P \rpquote \id{\}}            = 
  \left\{ 
    \begin{array}{ccc}
      \lpquote Q \rpquote & & x \nameeq \lpquote P \rpquote \\
      x & & otherwise \\
    \end{array}
  \right. \nonumber
\end{eqnarray}

and $z$ is chosen distinct from $\quotep{P}$, $\quotep{Q}$, the free
names in $Q$, and all the names in $R$. Our $\alpha$-equivalence will
be built in the standard way from this substitution.

\begin{remark}\label{rem:no_self_referential_names}
  One consequence of these definitions is that $\forall P. \quotep{P}
  \not\in \freenames{P}$.
\end{remark}

\subsection{ Dynamic quote: an example }

Anticipating something of what's to come, consider applying the
substitution, $\widehat{\id{\{}u / z \id{\}}}$, to the following pair
of processes, $\lift{w}{y!(z)}$ and $w[ \lpquote y!(z) \rpquote ]$.

\begin{eqnarray}
	\lift{w}{y!(z)}\widehat{\id{\{}u / z \id{\}}}
		& = &
		\lift{w}{y!(u)} \nonumber\\
	w[ \lpquote y!(z) \rpquote ] \widehat{ \id{\{}u / z \id{\}} }
		& = &
		w[ \lpquote y!(z) \rpquote ] \nonumber
\end{eqnarray}

Because the body of the process between quotes is impervious to
substitution, we get radically different answers. In fact, by
examining the first process in an input context,
e.g. $x?(z).\lift{w}{y!(z)}$, we see that the process under the lift
operator may be shaped by prefixed inputs binding a name inside it. In
this sense, the lift operator will be seen as a way to dynamically
construct processes before reifying them as names.

Finally equipped with these standard features we can present the
dynamics of the calculus.

\subsubsection{Operational semantics} 

Finally, we introduce the computational dynamics. What marks these
algebras as distinct from other more traditionally studied algebraic
structures, e.g. vector spaces or polynomial rings, is the manner in
which dynamics is captured. In traditional structures, dynamics is typically
expressed through morphisms between such structures, as in linear maps
between vector spaces or morphisms between rings. In algebras
associated with the semantics of computation, the dynamics is
expressed as part of the algebraic structure itself, through a
reduction reduction relation typically denoted by $\red$. Below, we
give a recursive presentation of this relation for the calculus used
in the encoding.

$\red \subseteq \pi \times \pi$
$\red : \pi \to \mathcal{P}(\pi)$

\begin{mathpar}
  \inferrule* [lab=Comm] { \textsf{match}( x_{src}, x_{trgt} ) } { x_{trgt}?(y)P \; | \; x_{src}!\langle {Q} \rangle \red P\{\quotep{Q}/y}\} }
  \and \\
  \inferrule* [lab=Par] {{P} \red {P}'} {{{P} | {Q}} \red {{P}' | {Q}}}
  \and
  \inferrule* [lab=Equiv]{{{P} \scong {P}'} \andalso {{P}' \red {Q}'} \andalso {{Q}' \scong {Q}}}{{P} \red {Q}}
\end{mathpar}

\begin{eqnarray*}
  match_{\equiv} (\quotep{P},\quotep{Q}) & := & P \equiv Q \\
  match_{\dagger}(\quotep{P},\quotep{Q}) & := & \forall R. P|Q \red^{*} R => R \red^{*} 0 \\
  match_{K}(\quotep{P},\quotep{Q}) & := & K \mbox{ for some context } K
\end{eqnarray*}

$u?(x)P | u!\langle Q \rangle \red P\{\quotep{Q}/x\}$

%We write $\wred$ for $\red^*$, and $P\red$ if $\exists Q $ such that $ P \red Q$.
We write $P\red$ if $\exists Q $ such that $ P \red Q$ and $P\not\red$, otherwise.

\section{Replication}

As mentioned before, it is known that replication (and hence
recursion) can be implemented in a higher-order process algebra
\cite{SangiorgiWalker}. As our first example of calculation with the
machinery thus far presented we give the construction explicitly in
the {\rhoc}.

\begin{eqnarray}
	D_{x} & := & \prefix{x}{y}{(\binpar{\outputp{x}{y}}{@{y}})} \nonumber\\
	\bangp_{x}{P} & := & \binpar{{x}!\langle{\binpar{D_{x}}{P}}\rangle}{D_{x}} \nonumber
\end{eqnarray}

\begin{eqnarray}
	\bangp_{x}{P} & & \nonumber\\
	=
	& {x}!\langle{(\prefix{x}{y}{(\outputp{x}{y} | @{y})) | P}}\rangle 
	      | \prefix{x}{y}{(\outputp{x}{y} | @{y})} & \nonumber\\
	\red
	& (\outputp{x}{y} | @{y})\substn{\quotep{(\prefix{x}{y}{(@{y} | \outputp{x}{y})) | P}}}{y} & \nonumber\\
	=
	& \outputp{x}{\quotep{(\prefix{x}{y}{(\outputp{x}{y} | @{y})) | P}}}
	  | {(\prefix{x}{y}{(\outputp{x}{y} | @{y})) | P}} & \nonumber\\
	\red
	& \ldots & \nonumber\\
	\red^*
	& P | P | \ldots & \nonumber
\end{eqnarray}

Of course, this encoding, as an implementation, runs away, unfolding
$\bangp{P}$ eagerly. A lazier and more implementable replication
operator, restricted to input-guarded processes, may be obtained as follows.

\begin{eqnarray}
\bangp{\prefix{u}{v}{P}} 
	:= 
	\binpar{\lift{x}{\prefix{u}{v}{(\binpar{D(x)}{P})}}}{D(x)} \nonumber
\end{eqnarray}

\begin{remark}
  Note that the lazier definition still does not deal with summation
  or mixed summation (i.e. sums over input and output). The reader is
  invited to construct definitions of replication that deal with these
  features. 

  Further, the definitions are parameterized in a name, $x$. Can you,
  gentle reader, make a definition that eliminates this parameter and
  guarantees no accidental interaction between the replication
  machinery and the process being replicated -- i.e. no accidental
  sharing of names used by the process to get its work done and the
  name(s) used by the replication to effect copying. This latter
  revision of the definition of replication is crucial to obtaining
  the expected identity $!!P \sim !P$.
\end{remark}

\begin{remark}\label{rem:paradoxical_combinator}
  The reader familiar with the lambda calculus will have noticed the
  similarity between $D$ and the paradoxical combinator.

  [Ed. note: the existence of this seems to suggest we have to be more
  restrictive on the set of processes and names we admit if we are to
  support no-cloning.]
\end{remark}

\subsubsection{Bisimulation}

The computational dynamics gives rise to another kind of equivalence,
the equivalence of computational behavior. As previously mentioned
this is typically captured \emph{via} some form of bisimulation.

% The notion we use in this paper is weak barbed bisimulation
% \cite{milner91polyadicpi}.

The notion we use in this paper is derived from weak barbed
bisimulation \cite{milner91polyadicpi}. 

\begin{definition}
An \emph{observation relation}, $\downarrow_{\mathcal N}$, over a set
of names, $\mathcal N$, is the smallest relation satisfying the rules
below.

\infrule[Out-barb]{y \in {\mathcal N}, \; x \nameeq y}
		  {\outputp{x}{v} \downarrow_{\mathcal N} x}
\infrule[Par-barb]{\mbox{$P\downarrow_{\mathcal N} x$ or $Q\downarrow_{\mathcal N} x$}}
		  {\binpar{P}{Q} \downarrow_{\mathcal N} x}

We write $P \Downarrow_{\mathcal N} x$ if there is $Q$ such that 
$P \wred Q$ and $Q \downarrow_{\mathcal N} x$.
\end{definition}

\begin{definition}
%\label{def.bbisim}
An  ${\mathcal N}$-\emph{barbed bisimulation} over a set of names, ${\mathcal N}$, is a symmetric binary relation 
${\mathcal S}_{\mathcal N}$ between agents such that $P\rel{S}_{\mathcal N}Q$ implies:
\begin{enumerate}
\item If $P \red P'$ then $Q \wred Q'$ and $P'\rel{S}_{\mathcal N} Q'$.
\item If $P\downarrow_{\mathcal N} x$, then $Q\Downarrow_{\mathcal N} x$.
\end{enumerate}
$P$ is ${\mathcal N}$-barbed bisimilar to $Q$, written
$P \wbbisim_{\mathcal N} Q$, if $P \rel{S}_{\mathcal N} Q$ for some ${\mathcal N}$-barbed bisimulation ${\mathcal S}_{\mathcal N}$.
\end{definition}

$\mathcal{R} \subseteq \pi \times \pi$

$P \mathcal{R} Q => \forall P'. P \red P' \Rightarrow \exists Q'. Q \red Q', P' \mathcal{R} Q'$

$P \vdash x \Rightarrow Q \vdash x$

\begin{mathpar}
  \inferrule*[lab=Out-barb]{x \nameeq y}{{y}!\langle{Q}\rangle \vdash x}
  \and
  \inferrule*[lab=Par-barb]{\mbox{$P\vdash x$ or $Q\vdash x$}}{\binpar{P}{Q} \vdash x}
\end{mathpar}

\subsubsection{Contexts}

One of the principle advantages of computational calculi like the
$\pi$-calculus is a well-defined notion of context,
contextual-equivalence and a correlation between
contextual-equivalence and notions of bisimulation. The notion of
context allows the decomposition of a process into (sub-)process and
its syntactic environment, its context. Thus, a context may be
thought of as a process with a ``hole'' (written $\Box$) in it. The
application of a context $M$ to a process $P$, written $M[P]$, is
tantamount to filling the hole in $M$ with $P$. In this paper we do
not need the full weight of this theory, but do make use of the notion
of context in the proof the main theorem. 

\begin{mathpar}
  \inferrule* [lab=summation] {} {{M_{M},M_{N}} \bc \Box \;|\; x.M_{A} \;|\; M_{M}+M_{N}}
  \and
  \inferrule* [lab=agent] {} {{M_{A}} \bc (\vec{x})M_{P} \;| \; \clift{P_0,\ldots,M_{P},\ldots,P_N}}
  \and \\
  \inferrule* [lab=process] {} {{M_{P}} \bc M_{N} \;| \;P|M_{P} }
\end{mathpar} 

\begin{mathpar}
  \inferrule* [lab=sychronization] {} {M_{N} \bc \Box \;|\; x?M_{F} \;|\; x!M_{C}}
  \and
  \inferrule* [lab=abstraction] {} {{M_{F}} \bc (x)M_{P} }
  \and
  \inferrule* [lab=concretion] {} {{M_{C}} \bc \langle M_{P} \rangle }
  \and \\
  \inferrule* [lab=process] {} {{M_{P}} \bc M_{N} \;| \;P|M_{P} }
\end{mathpar}

\begin{definition}[contextual application] Given a context $M$, and
  process $P$, we define the \emph{contextual application}, $M[P] :=
  M\{P/\Box\}$. That is, the contextual application of M to P is the
  substitution of $P$ for $\Box$ in $M$.
\end{definition}

$\meaningof{-} : L \to \mathcal{P}(\pi)$

\begin{mathpar}
  \inferrule* [lab=collection] {} {\meaningof{true} = \pi, \and \meaningof{~E} = \pi \setminus \meaningof{E}, \and \meaningof{E_{1} \& E_{2}} = \meaningof{E_{1}} \cap \meaningof{E_{2}}}
\end{mathpar}

\begin{mathpar}
  \inferrule* [lab=structure] {} {\meaningof{0} = \{ P \in \pi | P \equiv 0 \}, \and \\ \meaningof{E_1 | E_2} = \{ P \in \pi | P \equiv P_{1} | P_{2}, P_{1} \in \meaningof{E_{1}}, P_{2} \in \meaningof{E_2}\} }
\end{mathpar}

\begin{mathpar}
 \inferrule* [lab=behavior] {} {\meaningof{\langle a?b \rangle E} = \{ P \in \pi | P \equiv Q | u?(y)P', \\ \and \\\\ \and \\ \;\;\; u \in \meaningof{a}, \forall z.P'\{z/y\} \in \meaningof{E\{z/b\}}\}, \and \\ \meaningof{a!E} = \{ P \in \pi | P \equiv Q | x!\langle P' \rangle, x \in \meaningof{a} P' \in \meaningof{E}\} }
\end{mathpar}

\begin{mathpar}
 \inferrule* [lab=nominal] {} {\meaningof{\quotep{E}} = \{ \quotep{P} \in \quotep{\pi} | P \in \meaningof{E} \}, \and \meaningof{\quotep{P}} = \{ \quotep{Q} \in \quotep{\pi} | P \equiv Q \} \and \\ \meaningof{@\quotep{E}} = \{ P \in \pi | P \equiv @x, x \in \meaningof{E} \}}
\end{mathpar}

\begin{eqnarray*}
  \\
  \meaningof{-} : TS \to ST
\end{eqnarray*}

\begin{eqnarray*}
  \\
  L : TS \to ST
\end{eqnarray*}

\begin{eqnarray*}
  \\
  P \models E \iff P \in \meaningof{E}
\end{eqnarray*}

\begin{eqnarray*}
  P \approx_{L} Q \iff \forall E \in L. P \models E \iff Q \models E
\end{eqnarray*}

\begin{eqnarray*}
  P \approx_{K} Q
\end{eqnarray*}

\begin{eqnarray*}
  P \approx Q
\end{eqnarray*}

$\approx_{K} = \approx = \approx_{L}$

\subsubsection{Contextual duality}

Note that contexts extend the quotation operation to a family of
operations from processes to names. Given a context, $M$, we can
define a \emph{nominal context}, $\quotep{M}$ by $\quotep{M}[P] :=
\quotep{M[P]}$. To foreshadow what is to come we observe that these
operations enjoy a duality with processes very much like the duality
between vectors and maps from vectors to scalars.

Further, because the calculus is essentially higher-order, we have a
correspondence between contexts and processes. More specifically,
given a name $x$ and a context $M$ we can construct $M^{*}_{x}$ such
that 

\begin{mathpar}
  M^{*}_{x} | \lift{x}{P} \red M[P]
\end{mathpar}

namely,

\begin{mathpar}
  M^{*}_{x} := x?(u).M[\dropn{u}]
\end{mathpar}

The dependence of $M^{*}_{x}$ on a name makes it an abstraction, 

\begin{mathpar}
  M^{*} := (x)x?(u).M[\dropn{u}]
\end{mathpar}

\subsection{Additional notation}

It will sometimes be convenient to denote the process a name
quotes. We already have the notation $x = \quotep{P}$, but it will be
convenient to introduce an alternate notation, $\procn{x}$, when we
want to emphasize the connection to the use of the name. Note that, by
virtue of name equivalence, $\quotep{\procn{x}} \nameeq x$; so, the
notation is consistent with previous definitions.

Further, because names have structure it is possible to effect
substitutions on the basis of that structure. This means we need to
upgrade our notation for substitutions, which we accomplish by
adapting comprehension notation. Thus,

\begin{mathpar}
  P\{ y / x : x \in S \}
\end{mathpar}

is interpreted to mean the process derived from P by replacing (in a
capture-avoiding manner) each occurrence of $x$ in $S$ by $y$. For example,

\begin{mathpar}
  P\{ \quotep{\procn{x}|\procn{x}} / x : x \in \freenames{P} \}
\end{mathpar}

will replace each (occurrence) of a free name $x$ in $P$ by
$\quotep{\procn{x}|\procn{x}}$.

Also, we will avail ourselves of the notation $x^{L}$ and $x^{R}$ to
denote injections of a name into disjoint copies of the name
space. There are numerous ways to accomplish this. One example can be
found in \cite{MeredithR05}. This notation overloads to vectors of
names: $\vec{x}^{\pi} := (x_{i}^{\pi} \; : \; 0 \leq i < |\vec{x}| )$ where $\pi \in \{L,R\}$.

We also use $P^{\Box} := P|\Box$.

In \cite{MeredithR05} an interpretation of the new operator is
given. It turns out that there are several possible interpretations
all enjoying the requisite algebraic properties of the operator (see
\cite{milner91polyadicpi}). We will therefore make liberal use of
$(\nu\; \vec{x})P$.

% subsection the_syntax_and_semantics_of_the_notation_system (end)   

\input{qm2pi.qmops} 

\input{qm2pi.sterngerlach} 

\input{qm2pi.metric} 

% section concurrent_process_calculi (end)

%\input{qm2pi.proofsketch}

% section proof sketch (end)

%\input{qm2pi.slviaknots} 

% section spatial logic via knots (end)

\input{qm2pi.conclusion}

% section conclusion (end)

%\input{qm2pi.dtcodes} 

% section wiring algorithm (end)

\input{qm2pi.ack} 

% section acknowledgments (end)

\newpage


\bibliographystyle{plain}   
\bibliography{../../biblios/main.bib}

\input{qm2pi.rhodetails}

\end{document}



% section proof sketch (end)

%\section{Unlikely characters: spatial logic for
  knots}\label{sub:characteristic_formulae} % (fold)

Associated to the mobile process calculi are a family of logics known
as the Hennessy-Milner logics. These logics typically enjoy a
semantics interpreting formulae as sets of processes that when
factored through the encoding outlined above allows an identification
of classes of knots with logical formulae. In the context of this
encoding the sub-family known as the spatial logics \cite{CairesC03}
\cite{CairesC04} \cite{Caires04} are of particular interest providing
several important features for expressing and reasoning about
properties (i.e. classes) of knots. We hint here at how this may be done.

%\begin{description}
%\item [structural connectives] 
\subsubsection{Structural connectives} The spatial logics enjoy
structural connectives corresponding, at the logical level, to the
parallel composition ($P | Q$) and new name ($(\nu \; x)P$)
connectives for processes. As illustrated in the examples below, these
connectives are extremely expressive given the shape of our encoding.
%\item [decideable satisfaction]

\subsubsection{Decideable satisfaction}
In \cite{Caires04} the satisfaction relation is shown to be decideable
for a rich class of processes. It further turns out that the image of
the our encoding is a proper subset of that class. This result
provides the basis for an algorithm by which to search for knots
enjoying a given property.
%\item [characteristic formulae]

\subsubsection{Characteristic formulae}
In the same paper \cite{Caires04} , Caires presents a means of calculating
characteristic formulae, selecting equivalence classes of processes
up to a pre--specified depth limit on the support set of names. Composed with our
encoding, this characteristic formula can be used to select
characteristic formulae for knots.
%\end{description}

\subsubsection{Spatial logic formulae}

The grammar below (segmented for comprehension) summarizes the syntax
of spatial logic formulae. We employ illustrative examples in the
sequel to provide an intuitive understanding of their meaning
referring the reader to \cite{Caires04} for a more detailed explication
of the semantics.

\begin{mathpar}
  \inferrule* [lab=boolean] {} {{A,B} \bc T \;|\; \neg A \;|\; A \wedge B \;|\; \eta = \eta'}
  \and
  \inferrule* [lab=spatial] {} {|\; \pzero \;|\; A | B \;|\; x \text{\textregistered} A \;|\; \forall x . A \;|\;  H x . A}
  \and
  \inferrule* [lab=behavioral] {} {|\; \alpha . A}
  \and 
  \inferrule* [lab=recursion] {} {|\; X(\vec{u}) \;|\; \mu X(\vec{u}) . A}
  \and
  \inferrule* [lab=action] {} {\alpha \bc \langle x?(\vec{y}) \rangle \;|\; \langle x!(\vec{y}) \rangle \;|\; \langle \tau \rangle}
  \and 
  \inferrule* [lab=name] {} {\eta \bc x \;|\; \tau}
\end{mathpar} 

% subsection characteristic_formulae (end)   	 

\subsection{Example formulae}\label{sub:example_formulae_} % (fold)

\subsubsection{Crossing as formula.}
% 
% \begin{align*}
%   \frac{d}{dx} \sin x &= \cos x 
%   & \frac{d}{dx} e^x &= e^x \\
%   \frac{d}{dx} \cos x &= - \sin x 
%   & \frac{d}{dx} \log x &= \frac{1}{x} \\
% \end{align*} 

\begin{align*}
 \mu C(x_{0},x_{1},y_{0},y_{1},u).&(\langle x_{0}?(z) \rangle(\langle u! \rangle\langle y_{1}!z \rangle C(x_{0},x_{1},y_{0},y_{1},u)) & \\
  & \wedge \langle y_{1}?(z) \rangle (\langle u! \rangle \langle x_{0}!z \rangle C(x_{0},x_{1},y_{0},y_{1},u)) & \\
  & \wedge \langle x_{1}?(z) \rangle (\langle u? \rangle \langle y_{0}!z \rangle C(x_{0},x_{1},y_{0},y_{1},u)) & \\
  & \wedge \langle y_{0}?(z) \rangle (\langle u? \rangle \langle x_{1}!z \rangle C(x_{0},x_{1},y_{0},y_{1},u))) &
\end{align*}

The lexicographical similarity between the shape of this formulae and
the shape of definition of the process representing a crossing reveals
the intuitive meaning of this formulae. It describes the capabilities
of a process that has the right to represent a crossing. For example
it picks out processes that may perform an input on the port $x_0$ in
its initial menu of capabilities. What differentiates the formula
from the process, however, is that the crossing process is the
smallest candidate to satisfy the formula. Infinitely many other
processes -- with internal behavior hidden behind this interface, so
to speak -- also satisfy this formula. Even this simple formula,
then, can be seen to open a new view onto knots, providing a
computational interpretation of \emph{virtual} knots.

Note that this formula is derived by hand. A similar formula can be
derived by employing Caires' calculation of characteristic formula
\cite{Caires04} to the process representing a crossing. In light of
this discussion, we let
$\meaningof{C}_{\phi}(x0,x1,y0,y1,u)$ denote a formula specifying the
dynamics we wish to capture of a crossing. To guarantee we preserve
the shape of the interface and minimal semantics we demand that
$\meaningof{C}_{\phi}(x0,x1,y0,y1,u) \Rightarrow
\textbf{C}(x0,x1,y0,y1,u)$ where $\textbf{C}(x0,x1,y0,y1,u)$ denotes
the formula above.
                            
\subsubsection{Crossing number constraints.}
The moral content of the context lemma (Lemma \ref{context}) is that the notion of
``locality'' in the Reidemeister moves is effectively captured by the
parallel composition operator of the process calculus. This intuition
extends through the logic. Given a formula,
$\meaningof{C}_{\phi}(x0,x1,y0,y1,u)$, we can use the structural
connectives to specify constraints on crossing numbers, such as at
least $n$ crossings, or exactly $n$ crossings.
\begin{mathpar}
  \inferrule* [lab=at-least-n] {} { K^{\geq n}_{\phi}(\vec{xs},\vec{ys}) := \Pi_{i=0}^{n-1} Hu . \meaningof{C}_{\phi}(xs_i,ys_i,u) | T }
  \and 
  \inferrule* [lab=exactly-n] {} { K^{= n}_{\phi}(\vec{xs},\vec{ys}) := \Pi_{i=0}^{n-1} Hu . \meaningof{C}_{\phi}(xs_i,ys_i,u) | \neg (\forall x_0,y_0,x_1,y_1,u . \meaningof{C}_{\phi}(x_0,y_0,x_1,y_1,u) | T) }
\end{mathpar}

To round out this section, recall that the encoding of an $n$-crossing
knot decomposes into a parallel composition of $n$ \emph{copies} of a
crossing process together with a wiring harness. To specify different
knot classes with the same crossing number amounts to specifying
logical constraints on the wiring harness. In the interest of space,
we defer examples to a forthcoming paper. Suffice it to say that both
the conditions ``alternating knot'' and ``contains the tangle
corresponding to 5/3'' are expressible. For example, it is possible to
calculate the characteristic formula of a process corresponding to the
tangle 5/3 and conjoin it into the classifying formula via the
composition connective of the logic.

Finally, we wish to observe that it is entirely within reason to
contemplate a more domain-specific version of spatial logic tailored
to the shape of processes in the image of the encoding. Such a
domain-specific logic would have a better claim to the title formal
language of knot properties.

% subsection example_formulae_ (end)

% section knots_as_processes (end) 

% section spatial logic via knots (end)

\section{Conclusions and future work}

\paragraph{Testing physical space}
You, gentle reader, may wonder why of all the theorems to be proved
given this set up we pick the one above. In some sense it's hardly
central to quantum mechanics. We see it as central in the sense that
it firmly establishes a notion of physical space arising from a notion
of the equivalence of behavior. Relating bisimulation to a metric is a
big step forward, but one is faced with interpreting the relationship
of that metric space to something more physical. Quantum mechanical
notions of ``physical'' space are still far from intuitive, but by
relating this idea of distance as testing to calculations that predict
physical circumstances we are making a not insignificant step forward
toward an understanding of the physical space we inhabit as
essentially dynamic.

\paragraph{Effectivity and simulation}
One of the observations we have yet to make is that the entire program
spelled out here is effective. We have built various interpreters for
the reflective calculus at work in this interpretation. In principle,
then, we can simulate quantum mechanics on a computer. The place where
the simulation may lose fidelity is the infinitely branching summation
for the annihilator.

In this connection i also want to point out that the evaluation style
calculation of the inner product puts the non-determinism of the
summation right at the heart of measurement. This suggests that
Milner's original reduction-based formulation of the dynamics of his
calculi in terms of sums was not just notationally suggestive of a
notion of measure-and-continue but captured some significant part of
the physics.

\paragraph{Quantum continuations}
In light of this last observation i want to point out that the
predominant account of quantum mechanics is missing a key aspect of a
truly compositional story of the physical situation. In a real lab,
when a measurement is made the observation can be made to feed into
another device that then makes another measurement conditioned on the
results of the first. This means that after the superposition was
collapsed the entire experimental set up remained in
superposition. While QM offers a means of writing this down it doesn't
quite line up well with the well-trodden formulation of computation
and continuation that we see so succinctly expressed in Milner's
calculi. This suggests that there might be advantages to this account
of dynamics waiting to be explored.

\paragraph{Quantum logic}
In this connection, we also note that by virtue of having the
Hennessy-Milner construction, we can pull the construction through the
interpretation of QM. This gives us a natural candidate for a quantum
logic that enjoys an extremely tight connection with it's domain of
interpretation, making the construction much less ad hoc (rather it is
the image of functor!).

\paragraph{Quantum probabiity}
i have questions about the basis of the interpretation of inner
product as probability amplitude. In particular, using which
axiomatization of probability theory does the notion of probability
amplitude earn the right to be so dubbed? In other words, where is the
proof that the operation for calculating a probability amplitude (and
then squaring) satisfies the axioms of what it means to calculate a
probability? Even if such a proof exists (i have yet to find it in the
literature), i wonder if it might not be possible to turn things on
their heads. Can we view the calculation of the probability amplitude
as an axiomatization of probability? If so, then the definition we
give for calculating probability amplitude may provide the basis for
an \emph{effective} theory of probability.

\paragraph{Quantum vs ``biological'' information}
Finally, i want to conclude with a more philosophical observation. At
a recent workshop in which QM was a predominant topic i noticed
something about quantum information. The speaker was giving a riveting
discussion of axiomatic QM and showing how properties of ``no
cloning'' and ``no deleting'' emerged as consequences of the
axiomatization. Theorems of this form are necessary to give us a sense
of confidence that our axioms characterize the physical theory. What
struck me, though, was that if quantum information is neither erasable
nor replicable it is markedly different from \emph{life}. Two of the
things we know about life is that

\begin{itemize}
  \item it ends;
  \item to gain some measure of persistence, to transcend it's
    finitude it is imminently copyable.
\end{itemize}

Both of these qualities are summarized succinctly in the aphorism: all
flesh is grass. For me these two kinds of ``information'' -- call them
quantum and biological -- are end points on a spectrum of strategies
for persistence. At one end, we have those curious entities that enjoy
uniqueness and permanence; at the other, we have those who in the face
of a certain end and an uncertain present make a go of passing
something on. To me one of the more remarkable aspects of the latter
strategy is that in the presence of noise (and certain features of
copying) we get a kind of dynamism, a chance for improvement against a
given persistent condition.

% subsection other_calculi_other_bisimulations_and_geometry_as_behavior (end)




% section conclusion (end)

%\documentclass[12pt]{llncs}
%\documentclass{jktr}

\usepackage[pdftex]{hyperref}                   
\usepackage {listings}
\usepackage {mathpartir}
\usepackage{bcprules}
%\usepackage{listings}
                       
\usepackage{graphicx} 
%\usepackage[margins=2.5cm,nohead,nofoot]{geometry}
%\usepackage{geometry}
\usepackage{amsfonts}
\usepackage{amstext}
\usepackage{latexsym}
\usepackage{amssymb}
\usepackage{color}


%\include{myPreamble}
\include{qm2pi.local} 

%\ifpdf
%\usepackage[pdftex]{graphicx}
%\else
%\usepackage{graphicx}
%\fi

 % \ifpdf
%  \usepackage{pdfsync}
%  \if


%\title{Brief Article}
%\author{David F. Snyder}
%\author{L.G. Meredith}

%\address{Dept. of Math., Texas State University--San Marcos, San Marcos, TX 78666}
       
\pagestyle{empty}


\begin{document}

\lstset{language=[Objective]Caml,frame=shadowbox}

\input{qm2pi.front}

% section front matter (end)

\input{qm2pi.intro} 
 
% section introduction (end)

% \input{qm2pi.knotations} 

% section notation (end)

\input{qm2pi.process.calculi} 

% section concurrent_process_calculi_and_spatial_logics_ (end)
    
%\input{qm2pi.knots2pi} 

%\input{qm2pi.trefoil} 

%\input{qm2pi.mainthm} 

% subsection basic_interpretation (end)

%\input{qm2pi.rho.presentation} 
\subsection{The syntax and semantics of the notation system}\label{sub:the_syntax_and_semantics_of_the_notation_system} % (fold)

We now summarize a technical presentation of the calculus that
embodies our theory of dynamics. The typical presentation of such a
calculus follows the style of giving generators and relations on
them. The grammar, below, describing term constructors, freely
generates the set of processes, $\Proc$. This set is then quotiented
by a relation known as structural congruence and it is over this set
that the notion of dynamics is expressed. This presentation is
essentially that of \cite{MeredithR05} with the addition of
polyadicity and summation. For readability we have relegated some of
the technical subtleties to an appendix.

\subsubsection{Process grammar}\label{subsub:process_grammar}

\begin{mathpar}
  \inferrule* [lab=synchronization] {} {{M} \bc \pzero \;|\; x?F \;|\; x!C }
  \and
  \inferrule* [lab=abstraction] {} {{F} \bc (x)P}
  \and
  \inferrule* [lab=concretion] {} {{C} \bc \langle Q \rangle}
  \and
  \inferrule* [lab=process] {} {{P,Q} \bc M \;| \;P|Q \;|\; @{x}}
  \and
  \inferrule* [lab=name] {} {{x} \bc \quotep{P}}
\end{mathpar} 

Note that $\vec{x}$ (resp. $\vec{P}$) denotes a vector of names
(resp. processes) of length $|\vec{x}|$ (resp. $|\vec{P}|$). We adopt
the following useful abbreviations.

\begin{mathpar}
   x?(\vec{y}).P := x.(\vec{y})P \and  x\clift{\vec{P}} := x.\clift{\vec{P}}
   \and x!(y) := \lift{x}{\dropn{y}}
   \and \Pi_{i=0}^{n-1}P_i := P_0 | \ldots | P_{n-1}
\end{mathpar}

\subsubsection{Structural congruence}

\paragraph{Free and bound names and alpha-equivalence.} At the
core of structural equivalence is alpha-equivalence which identifies
process that are the same up to a change of variable. Formally, we
recognize the distinction between free and bound names. The free names
of a process, $\freenames{P}$, may be calculated recursively as
follows:

\begin{mathpar}
\freenames{\pzero} := \emptyset
  \and \\
  \freenames{x?(y).P} := \{ x \} \cup (\freenames{P} \setminus \{ y \})
  \and 
  \freenames{x!\langle P \rangle} := \{ x \} \cup \{ P \} 
  \and \\
  \freenames{P|Q} := \freenames{P} \cup \freenames{Q}
  \and \\
  \freenames{@{x}} := \{ x \}
\end{mathpar}

$\pi$
$\quotep{\pi}$

$\freenames{-} : \pi \to \mathcal{P}(\quotep{\pi})$

\begin{eqnarray*}
  \freenames{\pzero} & := & \emptyset \\
  \freenames{x?(y).P} & := & \{ x \} \cup (\freenames{P} \setminus \{ y \}) \\
  \freenames{x!\langle P \rangle} & := & \{ x \} \cup \{ P \} \\
  \freenames{P|Q} & := & \freenames{P} \cup \freenames{Q} \\
  \freenames{\dropn{x}} & := & \{ x \}
\end{eqnarray*}

The bound names of a process, $\boundnames{P}$, are those names occurring in $P$
that are not free. For example, in $x?(y).0$, the name $x$ is free, while $y$ is bound.

\begin{mathpar}
  \inferrule* [lab=monoidal-laws] {} { P|Q \equiv Q|P \and P|0 \equiv P \and P|(Q|R) \equiv (P|Q)|R }
\end{mathpar}

\begin{mathpar}
  \inferrule* [lab=alpha-equivalence] {} { (x)P \equiv (y)P\{y/x\} \and y \not\in \freenames{P} }
\end{mathpar}

\begin{definition}
Then two processes, $P,Q$, are alpha-equivalent if $P = Q\{\vec{y}/\vec{x}\}$ for
some $\vec{x} \in \boundnames{Q},\vec{y} \in \boundnames{P}$, where $Q\{\vec{y}/\vec{x}\}$
denotes the capture-avoiding substitution of $\vec{y}$ for $\vec{x}$ in $Q$.
\end{definition}

\begin{definition}
  The {\em structural congruence} \cite{SangiorgiWalker} , $\equiv$,
  between processes is the least congruence containing
  alpha-equivalence, satisfying the abelian monoid laws
  (associativity, commutativity and $\pzero$ as identity) for parallel
  composition $|$ and for summation $+$.
\end{definition}

\subsection{Name equivalence}

We take name equivalence, written $\nameeq$, to be the smallest
equivalence relation generated by the following rules.

\begin{mathpar}
\inferrule*[lab=Quote-drop]
{ }
{ \quotep{@{x}} \nameeq x }

\inferrule*[lab=Struct-equiv]
{ P \scong Q }
{ \quotep{P} \nameeq \quotep{Q} }
\end{mathpar}

The astute reader will have noticed that the mutual recursion of names
and processes imposes a mutual recursion on alpha-equivalence and
structural equivalence via name-equivalence. Fortunately, all of this
works out pleasantly and we may calculate in the natural way, free of
concern. The reader interested in the details is referred to the
appendix \ref{appendix:rho_details}.

\subsection{Substitution}

We use $\Proc$ for the set of processes, $\QProc$ for the set of
names, and $\id{\{}\vec{y} / \vec{x} \id{\}}$ to denote partial maps,
$s : \QProc \rightarrow \QProc$. A map, $s$ lifts, uniquely, to a map
on process terms, $\widehat{s} : \Proc \rightarrow \Proc$ by the
following equations.

\begin{mathpar}
  (0) \psubstp{Q}{P} := 0 \\
  (R \juxtap S) \psubstp{Q}{P}
  :=    
  (R)\psubstp{Q}{P} \juxtap (S) \psubstp{Q}{P} \\
  (x?(y).R) \psubstp{Q}{P}    
  :=    
  (x)\substp{Q}{P} (z)\concat( (R \psubstn{z}{y}) \psubstp{Q}{P} ) \\
  (\lift{x}{R}) \psubstp{Q}{P}  
  :=
  \lift{(x)\substp{Q}{P}}{ R \psubstp{Q}{P} } \\
%   (\dropn{x})  \psubstp{Q}{P}       
%   := 
%   \left\{ 
%     \begin{array}{ccc} 
%       \dropn{\quotep{Q}} & & x \nameeq \quotep{P} \\
%       \dropn{x} & & otherwise \\
%     \end{array}
%   \right. 
  (\dropn{x})  \psubstp{Q}{P}       
  := 
  \left\{ 
    \begin{array}{ccc} 
      Q & & x \nameeq \quotep{P} \\
      \dropn{x} & & otherwise \\
    \end{array}
  \right.
\end{mathpar}
 

where

\begin{eqnarray}
  (x)\id{\{} \lpquote Q \rpquote / \lpquote P \rpquote \id{\}}            = 
  \left\{ 
    \begin{array}{ccc}
      \lpquote Q \rpquote & & x \nameeq \lpquote P \rpquote \\
      x & & otherwise \\
    \end{array}
  \right. \nonumber
\end{eqnarray}

and $z$ is chosen distinct from $\quotep{P}$, $\quotep{Q}$, the free
names in $Q$, and all the names in $R$. Our $\alpha$-equivalence will
be built in the standard way from this substitution.

\begin{remark}\label{rem:no_self_referential_names}
  One consequence of these definitions is that $\forall P. \quotep{P}
  \not\in \freenames{P}$.
\end{remark}

\subsection{ Dynamic quote: an example }

Anticipating something of what's to come, consider applying the
substitution, $\widehat{\id{\{}u / z \id{\}}}$, to the following pair
of processes, $\lift{w}{y!(z)}$ and $w[ \lpquote y!(z) \rpquote ]$.

\begin{eqnarray}
	\lift{w}{y!(z)}\widehat{\id{\{}u / z \id{\}}}
		& = &
		\lift{w}{y!(u)} \nonumber\\
	w[ \lpquote y!(z) \rpquote ] \widehat{ \id{\{}u / z \id{\}} }
		& = &
		w[ \lpquote y!(z) \rpquote ] \nonumber
\end{eqnarray}

Because the body of the process between quotes is impervious to
substitution, we get radically different answers. In fact, by
examining the first process in an input context,
e.g. $x?(z).\lift{w}{y!(z)}$, we see that the process under the lift
operator may be shaped by prefixed inputs binding a name inside it. In
this sense, the lift operator will be seen as a way to dynamically
construct processes before reifying them as names.

Finally equipped with these standard features we can present the
dynamics of the calculus.

\subsubsection{Operational semantics} 

Finally, we introduce the computational dynamics. What marks these
algebras as distinct from other more traditionally studied algebraic
structures, e.g. vector spaces or polynomial rings, is the manner in
which dynamics is captured. In traditional structures, dynamics is typically
expressed through morphisms between such structures, as in linear maps
between vector spaces or morphisms between rings. In algebras
associated with the semantics of computation, the dynamics is
expressed as part of the algebraic structure itself, through a
reduction reduction relation typically denoted by $\red$. Below, we
give a recursive presentation of this relation for the calculus used
in the encoding.

$\red \subseteq \pi \times \pi$
$\red : \pi \to \mathcal{P}(\pi)$

\begin{mathpar}
  \inferrule* [lab=Comm] { \textsf{match}( x_{src}, x_{trgt} ) } { x_{trgt}?(y)P \; | \; x_{src}!\langle {Q} \rangle \red P\{\quotep{Q}/y}\} }
  \and \\
  \inferrule* [lab=Par] {{P} \red {P}'} {{{P} | {Q}} \red {{P}' | {Q}}}
  \and
  \inferrule* [lab=Equiv]{{{P} \scong {P}'} \andalso {{P}' \red {Q}'} \andalso {{Q}' \scong {Q}}}{{P} \red {Q}}
\end{mathpar}

\begin{eqnarray*}
  match_{\equiv} (\quotep{P},\quotep{Q}) & := & P \equiv Q \\
  match_{\dagger}(\quotep{P},\quotep{Q}) & := & \forall R. P|Q \red^{*} R => R \red^{*} 0 \\
  match_{K}(\quotep{P},\quotep{Q}) & := & K \mbox{ for some context } K
\end{eqnarray*}

$u?(x)P | u!\langle Q \rangle \red P\{\quotep{Q}/x\}$

%We write $\wred$ for $\red^*$, and $P\red$ if $\exists Q $ such that $ P \red Q$.
We write $P\red$ if $\exists Q $ such that $ P \red Q$ and $P\not\red$, otherwise.

\section{Replication}

As mentioned before, it is known that replication (and hence
recursion) can be implemented in a higher-order process algebra
\cite{SangiorgiWalker}. As our first example of calculation with the
machinery thus far presented we give the construction explicitly in
the {\rhoc}.

\begin{eqnarray}
	D_{x} & := & \prefix{x}{y}{(\binpar{\outputp{x}{y}}{@{y}})} \nonumber\\
	\bangp_{x}{P} & := & \binpar{{x}!\langle{\binpar{D_{x}}{P}}\rangle}{D_{x}} \nonumber
\end{eqnarray}

\begin{eqnarray}
	\bangp_{x}{P} & & \nonumber\\
	=
	& {x}!\langle{(\prefix{x}{y}{(\outputp{x}{y} | @{y})) | P}}\rangle 
	      | \prefix{x}{y}{(\outputp{x}{y} | @{y})} & \nonumber\\
	\red
	& (\outputp{x}{y} | @{y})\substn{\quotep{(\prefix{x}{y}{(@{y} | \outputp{x}{y})) | P}}}{y} & \nonumber\\
	=
	& \outputp{x}{\quotep{(\prefix{x}{y}{(\outputp{x}{y} | @{y})) | P}}}
	  | {(\prefix{x}{y}{(\outputp{x}{y} | @{y})) | P}} & \nonumber\\
	\red
	& \ldots & \nonumber\\
	\red^*
	& P | P | \ldots & \nonumber
\end{eqnarray}

Of course, this encoding, as an implementation, runs away, unfolding
$\bangp{P}$ eagerly. A lazier and more implementable replication
operator, restricted to input-guarded processes, may be obtained as follows.

\begin{eqnarray}
\bangp{\prefix{u}{v}{P}} 
	:= 
	\binpar{\lift{x}{\prefix{u}{v}{(\binpar{D(x)}{P})}}}{D(x)} \nonumber
\end{eqnarray}

\begin{remark}
  Note that the lazier definition still does not deal with summation
  or mixed summation (i.e. sums over input and output). The reader is
  invited to construct definitions of replication that deal with these
  features. 

  Further, the definitions are parameterized in a name, $x$. Can you,
  gentle reader, make a definition that eliminates this parameter and
  guarantees no accidental interaction between the replication
  machinery and the process being replicated -- i.e. no accidental
  sharing of names used by the process to get its work done and the
  name(s) used by the replication to effect copying. This latter
  revision of the definition of replication is crucial to obtaining
  the expected identity $!!P \sim !P$.
\end{remark}

\begin{remark}\label{rem:paradoxical_combinator}
  The reader familiar with the lambda calculus will have noticed the
  similarity between $D$ and the paradoxical combinator.

  [Ed. note: the existence of this seems to suggest we have to be more
  restrictive on the set of processes and names we admit if we are to
  support no-cloning.]
\end{remark}

\subsubsection{Bisimulation}

The computational dynamics gives rise to another kind of equivalence,
the equivalence of computational behavior. As previously mentioned
this is typically captured \emph{via} some form of bisimulation.

% The notion we use in this paper is weak barbed bisimulation
% \cite{milner91polyadicpi}.

The notion we use in this paper is derived from weak barbed
bisimulation \cite{milner91polyadicpi}. 

\begin{definition}
An \emph{observation relation}, $\downarrow_{\mathcal N}$, over a set
of names, $\mathcal N$, is the smallest relation satisfying the rules
below.

\infrule[Out-barb]{y \in {\mathcal N}, \; x \nameeq y}
		  {\outputp{x}{v} \downarrow_{\mathcal N} x}
\infrule[Par-barb]{\mbox{$P\downarrow_{\mathcal N} x$ or $Q\downarrow_{\mathcal N} x$}}
		  {\binpar{P}{Q} \downarrow_{\mathcal N} x}

We write $P \Downarrow_{\mathcal N} x$ if there is $Q$ such that 
$P \wred Q$ and $Q \downarrow_{\mathcal N} x$.
\end{definition}

\begin{definition}
%\label{def.bbisim}
An  ${\mathcal N}$-\emph{barbed bisimulation} over a set of names, ${\mathcal N}$, is a symmetric binary relation 
${\mathcal S}_{\mathcal N}$ between agents such that $P\rel{S}_{\mathcal N}Q$ implies:
\begin{enumerate}
\item If $P \red P'$ then $Q \wred Q'$ and $P'\rel{S}_{\mathcal N} Q'$.
\item If $P\downarrow_{\mathcal N} x$, then $Q\Downarrow_{\mathcal N} x$.
\end{enumerate}
$P$ is ${\mathcal N}$-barbed bisimilar to $Q$, written
$P \wbbisim_{\mathcal N} Q$, if $P \rel{S}_{\mathcal N} Q$ for some ${\mathcal N}$-barbed bisimulation ${\mathcal S}_{\mathcal N}$.
\end{definition}

$\mathcal{R} \subseteq \pi \times \pi$

$P \mathcal{R} Q => \forall P'. P \red P' \Rightarrow \exists Q'. Q \red Q', P' \mathcal{R} Q'$

$P \vdash x \Rightarrow Q \vdash x$

\begin{mathpar}
  \inferrule*[lab=Out-barb]{x \nameeq y}{{y}!\langle{Q}\rangle \vdash x}
  \and
  \inferrule*[lab=Par-barb]{\mbox{$P\vdash x$ or $Q\vdash x$}}{\binpar{P}{Q} \vdash x}
\end{mathpar}

\subsubsection{Contexts}

One of the principle advantages of computational calculi like the
$\pi$-calculus is a well-defined notion of context,
contextual-equivalence and a correlation between
contextual-equivalence and notions of bisimulation. The notion of
context allows the decomposition of a process into (sub-)process and
its syntactic environment, its context. Thus, a context may be
thought of as a process with a ``hole'' (written $\Box$) in it. The
application of a context $M$ to a process $P$, written $M[P]$, is
tantamount to filling the hole in $M$ with $P$. In this paper we do
not need the full weight of this theory, but do make use of the notion
of context in the proof the main theorem. 

\begin{mathpar}
  \inferrule* [lab=summation] {} {{M_{M},M_{N}} \bc \Box \;|\; x.M_{A} \;|\; M_{M}+M_{N}}
  \and
  \inferrule* [lab=agent] {} {{M_{A}} \bc (\vec{x})M_{P} \;| \; \clift{P_0,\ldots,M_{P},\ldots,P_N}}
  \and \\
  \inferrule* [lab=process] {} {{M_{P}} \bc M_{N} \;| \;P|M_{P} }
\end{mathpar} 

\begin{mathpar}
  \inferrule* [lab=sychronization] {} {M_{N} \bc \Box \;|\; x?M_{F} \;|\; x!M_{C}}
  \and
  \inferrule* [lab=abstraction] {} {{M_{F}} \bc (x)M_{P} }
  \and
  \inferrule* [lab=concretion] {} {{M_{C}} \bc \langle M_{P} \rangle }
  \and \\
  \inferrule* [lab=process] {} {{M_{P}} \bc M_{N} \;| \;P|M_{P} }
\end{mathpar}

\begin{definition}[contextual application] Given a context $M$, and
  process $P$, we define the \emph{contextual application}, $M[P] :=
  M\{P/\Box\}$. That is, the contextual application of M to P is the
  substitution of $P$ for $\Box$ in $M$.
\end{definition}

$\meaningof{-} : L \to \mathcal{P}(\pi)$

\begin{mathpar}
  \inferrule* [lab=collection] {} {\meaningof{true} = \pi, \and \meaningof{~E} = \pi \setminus \meaningof{E}, \and \meaningof{E_{1} \& E_{2}} = \meaningof{E_{1}} \cap \meaningof{E_{2}}}
\end{mathpar}

\begin{mathpar}
  \inferrule* [lab=structure] {} {\meaningof{0} = \{ P \in \pi | P \equiv 0 \}, \and \\ \meaningof{E_1 | E_2} = \{ P \in \pi | P \equiv P_{1} | P_{2}, P_{1} \in \meaningof{E_{1}}, P_{2} \in \meaningof{E_2}\} }
\end{mathpar}

\begin{mathpar}
 \inferrule* [lab=behavior] {} {\meaningof{\langle a?b \rangle E} = \{ P \in \pi | P \equiv Q | u?(y)P', \\ \and \\\\ \and \\ \;\;\; u \in \meaningof{a}, \forall z.P'\{z/y\} \in \meaningof{E\{z/b\}}\}, \and \\ \meaningof{a!E} = \{ P \in \pi | P \equiv Q | x!\langle P' \rangle, x \in \meaningof{a} P' \in \meaningof{E}\} }
\end{mathpar}

\begin{mathpar}
 \inferrule* [lab=nominal] {} {\meaningof{\quotep{E}} = \{ \quotep{P} \in \quotep{\pi} | P \in \meaningof{E} \}, \and \meaningof{\quotep{P}} = \{ \quotep{Q} \in \quotep{\pi} | P \equiv Q \} \and \\ \meaningof{@\quotep{E}} = \{ P \in \pi | P \equiv @x, x \in \meaningof{E} \}}
\end{mathpar}

\begin{eqnarray*}
  \\
  \meaningof{-} : TS \to ST
\end{eqnarray*}

\begin{eqnarray*}
  \\
  L : TS \to ST
\end{eqnarray*}

\begin{eqnarray*}
  \\
  P \models E \iff P \in \meaningof{E}
\end{eqnarray*}

\begin{eqnarray*}
  P \approx_{L} Q \iff \forall E \in L. P \models E \iff Q \models E
\end{eqnarray*}

\begin{eqnarray*}
  P \approx_{K} Q
\end{eqnarray*}

\begin{eqnarray*}
  P \approx Q
\end{eqnarray*}

$\approx_{K} = \approx = \approx_{L}$

\subsubsection{Contextual duality}

Note that contexts extend the quotation operation to a family of
operations from processes to names. Given a context, $M$, we can
define a \emph{nominal context}, $\quotep{M}$ by $\quotep{M}[P] :=
\quotep{M[P]}$. To foreshadow what is to come we observe that these
operations enjoy a duality with processes very much like the duality
between vectors and maps from vectors to scalars.

Further, because the calculus is essentially higher-order, we have a
correspondence between contexts and processes. More specifically,
given a name $x$ and a context $M$ we can construct $M^{*}_{x}$ such
that 

\begin{mathpar}
  M^{*}_{x} | \lift{x}{P} \red M[P]
\end{mathpar}

namely,

\begin{mathpar}
  M^{*}_{x} := x?(u).M[\dropn{u}]
\end{mathpar}

The dependence of $M^{*}_{x}$ on a name makes it an abstraction, 

\begin{mathpar}
  M^{*} := (x)x?(u).M[\dropn{u}]
\end{mathpar}

\subsection{Additional notation}

It will sometimes be convenient to denote the process a name
quotes. We already have the notation $x = \quotep{P}$, but it will be
convenient to introduce an alternate notation, $\procn{x}$, when we
want to emphasize the connection to the use of the name. Note that, by
virtue of name equivalence, $\quotep{\procn{x}} \nameeq x$; so, the
notation is consistent with previous definitions.

Further, because names have structure it is possible to effect
substitutions on the basis of that structure. This means we need to
upgrade our notation for substitutions, which we accomplish by
adapting comprehension notation. Thus,

\begin{mathpar}
  P\{ y / x : x \in S \}
\end{mathpar}

is interpreted to mean the process derived from P by replacing (in a
capture-avoiding manner) each occurrence of $x$ in $S$ by $y$. For example,

\begin{mathpar}
  P\{ \quotep{\procn{x}|\procn{x}} / x : x \in \freenames{P} \}
\end{mathpar}

will replace each (occurrence) of a free name $x$ in $P$ by
$\quotep{\procn{x}|\procn{x}}$.

Also, we will avail ourselves of the notation $x^{L}$ and $x^{R}$ to
denote injections of a name into disjoint copies of the name
space. There are numerous ways to accomplish this. One example can be
found in \cite{MeredithR05}. This notation overloads to vectors of
names: $\vec{x}^{\pi} := (x_{i}^{\pi} \; : \; 0 \leq i < |\vec{x}| )$ where $\pi \in \{L,R\}$.

We also use $P^{\Box} := P|\Box$.

In \cite{MeredithR05} an interpretation of the new operator is
given. It turns out that there are several possible interpretations
all enjoying the requisite algebraic properties of the operator (see
\cite{milner91polyadicpi}). We will therefore make liberal use of
$(\nu\; \vec{x})P$.

% subsection the_syntax_and_semantics_of_the_notation_system (end)   

\input{qm2pi.qmops} 

\input{qm2pi.sterngerlach} 

\input{qm2pi.metric} 

% section concurrent_process_calculi (end)

%\input{qm2pi.proofsketch}

% section proof sketch (end)

%\input{qm2pi.slviaknots} 

% section spatial logic via knots (end)

\input{qm2pi.conclusion}

% section conclusion (end)

%\input{qm2pi.dtcodes} 

% section wiring algorithm (end)

\input{qm2pi.ack} 

% section acknowledgments (end)

\newpage


\bibliographystyle{plain}   
\bibliography{../../biblios/main.bib}

\input{qm2pi.rhodetails}

\end{document}

 

% section wiring algorithm (end)

\documentclass[12pt]{llncs}
%\documentclass{jktr}

\usepackage[pdftex]{hyperref}                   
\usepackage {listings}
\usepackage {mathpartir}
\usepackage{bcprules}
%\usepackage{listings}
                       
\usepackage{graphicx} 
%\usepackage[margins=2.5cm,nohead,nofoot]{geometry}
%\usepackage{geometry}
\usepackage{amsfonts}
\usepackage{amstext}
\usepackage{latexsym}
\usepackage{amssymb}
\usepackage{color}


%\include{myPreamble}
\include{qm2pi.local} 

%\ifpdf
%\usepackage[pdftex]{graphicx}
%\else
%\usepackage{graphicx}
%\fi

 % \ifpdf
%  \usepackage{pdfsync}
%  \if


%\title{Brief Article}
%\author{David F. Snyder}
%\author{L.G. Meredith}

%\address{Dept. of Math., Texas State University--San Marcos, San Marcos, TX 78666}
       
\pagestyle{empty}


\begin{document}

\lstset{language=[Objective]Caml,frame=shadowbox}

\input{qm2pi.front}

% section front matter (end)

\input{qm2pi.intro} 
 
% section introduction (end)

% \input{qm2pi.knotations} 

% section notation (end)

\input{qm2pi.process.calculi} 

% section concurrent_process_calculi_and_spatial_logics_ (end)
    
%\input{qm2pi.knots2pi} 

%\input{qm2pi.trefoil} 

%\input{qm2pi.mainthm} 

% subsection basic_interpretation (end)

%\input{qm2pi.rho.presentation} 
\subsection{The syntax and semantics of the notation system}\label{sub:the_syntax_and_semantics_of_the_notation_system} % (fold)

We now summarize a technical presentation of the calculus that
embodies our theory of dynamics. The typical presentation of such a
calculus follows the style of giving generators and relations on
them. The grammar, below, describing term constructors, freely
generates the set of processes, $\Proc$. This set is then quotiented
by a relation known as structural congruence and it is over this set
that the notion of dynamics is expressed. This presentation is
essentially that of \cite{MeredithR05} with the addition of
polyadicity and summation. For readability we have relegated some of
the technical subtleties to an appendix.

\subsubsection{Process grammar}\label{subsub:process_grammar}

\begin{mathpar}
  \inferrule* [lab=synchronization] {} {{M} \bc \pzero \;|\; x?F \;|\; x!C }
  \and
  \inferrule* [lab=abstraction] {} {{F} \bc (x)P}
  \and
  \inferrule* [lab=concretion] {} {{C} \bc \langle Q \rangle}
  \and
  \inferrule* [lab=process] {} {{P,Q} \bc M \;| \;P|Q \;|\; @{x}}
  \and
  \inferrule* [lab=name] {} {{x} \bc \quotep{P}}
\end{mathpar} 

Note that $\vec{x}$ (resp. $\vec{P}$) denotes a vector of names
(resp. processes) of length $|\vec{x}|$ (resp. $|\vec{P}|$). We adopt
the following useful abbreviations.

\begin{mathpar}
   x?(\vec{y}).P := x.(\vec{y})P \and  x\clift{\vec{P}} := x.\clift{\vec{P}}
   \and x!(y) := \lift{x}{\dropn{y}}
   \and \Pi_{i=0}^{n-1}P_i := P_0 | \ldots | P_{n-1}
\end{mathpar}

\subsubsection{Structural congruence}

\paragraph{Free and bound names and alpha-equivalence.} At the
core of structural equivalence is alpha-equivalence which identifies
process that are the same up to a change of variable. Formally, we
recognize the distinction between free and bound names. The free names
of a process, $\freenames{P}$, may be calculated recursively as
follows:

\begin{mathpar}
\freenames{\pzero} := \emptyset
  \and \\
  \freenames{x?(y).P} := \{ x \} \cup (\freenames{P} \setminus \{ y \})
  \and 
  \freenames{x!\langle P \rangle} := \{ x \} \cup \{ P \} 
  \and \\
  \freenames{P|Q} := \freenames{P} \cup \freenames{Q}
  \and \\
  \freenames{@{x}} := \{ x \}
\end{mathpar}

$\pi$
$\quotep{\pi}$

$\freenames{-} : \pi \to \mathcal{P}(\quotep{\pi})$

\begin{eqnarray*}
  \freenames{\pzero} & := & \emptyset \\
  \freenames{x?(y).P} & := & \{ x \} \cup (\freenames{P} \setminus \{ y \}) \\
  \freenames{x!\langle P \rangle} & := & \{ x \} \cup \{ P \} \\
  \freenames{P|Q} & := & \freenames{P} \cup \freenames{Q} \\
  \freenames{\dropn{x}} & := & \{ x \}
\end{eqnarray*}

The bound names of a process, $\boundnames{P}$, are those names occurring in $P$
that are not free. For example, in $x?(y).0$, the name $x$ is free, while $y$ is bound.

\begin{mathpar}
  \inferrule* [lab=monoidal-laws] {} { P|Q \equiv Q|P \and P|0 \equiv P \and P|(Q|R) \equiv (P|Q)|R }
\end{mathpar}

\begin{mathpar}
  \inferrule* [lab=alpha-equivalence] {} { (x)P \equiv (y)P\{y/x\} \and y \not\in \freenames{P} }
\end{mathpar}

\begin{definition}
Then two processes, $P,Q$, are alpha-equivalent if $P = Q\{\vec{y}/\vec{x}\}$ for
some $\vec{x} \in \boundnames{Q},\vec{y} \in \boundnames{P}$, where $Q\{\vec{y}/\vec{x}\}$
denotes the capture-avoiding substitution of $\vec{y}$ for $\vec{x}$ in $Q$.
\end{definition}

\begin{definition}
  The {\em structural congruence} \cite{SangiorgiWalker} , $\equiv$,
  between processes is the least congruence containing
  alpha-equivalence, satisfying the abelian monoid laws
  (associativity, commutativity and $\pzero$ as identity) for parallel
  composition $|$ and for summation $+$.
\end{definition}

\subsection{Name equivalence}

We take name equivalence, written $\nameeq$, to be the smallest
equivalence relation generated by the following rules.

\begin{mathpar}
\inferrule*[lab=Quote-drop]
{ }
{ \quotep{@{x}} \nameeq x }

\inferrule*[lab=Struct-equiv]
{ P \scong Q }
{ \quotep{P} \nameeq \quotep{Q} }
\end{mathpar}

The astute reader will have noticed that the mutual recursion of names
and processes imposes a mutual recursion on alpha-equivalence and
structural equivalence via name-equivalence. Fortunately, all of this
works out pleasantly and we may calculate in the natural way, free of
concern. The reader interested in the details is referred to the
appendix \ref{appendix:rho_details}.

\subsection{Substitution}

We use $\Proc$ for the set of processes, $\QProc$ for the set of
names, and $\id{\{}\vec{y} / \vec{x} \id{\}}$ to denote partial maps,
$s : \QProc \rightarrow \QProc$. A map, $s$ lifts, uniquely, to a map
on process terms, $\widehat{s} : \Proc \rightarrow \Proc$ by the
following equations.

\begin{mathpar}
  (0) \psubstp{Q}{P} := 0 \\
  (R \juxtap S) \psubstp{Q}{P}
  :=    
  (R)\psubstp{Q}{P} \juxtap (S) \psubstp{Q}{P} \\
  (x?(y).R) \psubstp{Q}{P}    
  :=    
  (x)\substp{Q}{P} (z)\concat( (R \psubstn{z}{y}) \psubstp{Q}{P} ) \\
  (\lift{x}{R}) \psubstp{Q}{P}  
  :=
  \lift{(x)\substp{Q}{P}}{ R \psubstp{Q}{P} } \\
%   (\dropn{x})  \psubstp{Q}{P}       
%   := 
%   \left\{ 
%     \begin{array}{ccc} 
%       \dropn{\quotep{Q}} & & x \nameeq \quotep{P} \\
%       \dropn{x} & & otherwise \\
%     \end{array}
%   \right. 
  (\dropn{x})  \psubstp{Q}{P}       
  := 
  \left\{ 
    \begin{array}{ccc} 
      Q & & x \nameeq \quotep{P} \\
      \dropn{x} & & otherwise \\
    \end{array}
  \right.
\end{mathpar}
 

where

\begin{eqnarray}
  (x)\id{\{} \lpquote Q \rpquote / \lpquote P \rpquote \id{\}}            = 
  \left\{ 
    \begin{array}{ccc}
      \lpquote Q \rpquote & & x \nameeq \lpquote P \rpquote \\
      x & & otherwise \\
    \end{array}
  \right. \nonumber
\end{eqnarray}

and $z$ is chosen distinct from $\quotep{P}$, $\quotep{Q}$, the free
names in $Q$, and all the names in $R$. Our $\alpha$-equivalence will
be built in the standard way from this substitution.

\begin{remark}\label{rem:no_self_referential_names}
  One consequence of these definitions is that $\forall P. \quotep{P}
  \not\in \freenames{P}$.
\end{remark}

\subsection{ Dynamic quote: an example }

Anticipating something of what's to come, consider applying the
substitution, $\widehat{\id{\{}u / z \id{\}}}$, to the following pair
of processes, $\lift{w}{y!(z)}$ and $w[ \lpquote y!(z) \rpquote ]$.

\begin{eqnarray}
	\lift{w}{y!(z)}\widehat{\id{\{}u / z \id{\}}}
		& = &
		\lift{w}{y!(u)} \nonumber\\
	w[ \lpquote y!(z) \rpquote ] \widehat{ \id{\{}u / z \id{\}} }
		& = &
		w[ \lpquote y!(z) \rpquote ] \nonumber
\end{eqnarray}

Because the body of the process between quotes is impervious to
substitution, we get radically different answers. In fact, by
examining the first process in an input context,
e.g. $x?(z).\lift{w}{y!(z)}$, we see that the process under the lift
operator may be shaped by prefixed inputs binding a name inside it. In
this sense, the lift operator will be seen as a way to dynamically
construct processes before reifying them as names.

Finally equipped with these standard features we can present the
dynamics of the calculus.

\subsubsection{Operational semantics} 

Finally, we introduce the computational dynamics. What marks these
algebras as distinct from other more traditionally studied algebraic
structures, e.g. vector spaces or polynomial rings, is the manner in
which dynamics is captured. In traditional structures, dynamics is typically
expressed through morphisms between such structures, as in linear maps
between vector spaces or morphisms between rings. In algebras
associated with the semantics of computation, the dynamics is
expressed as part of the algebraic structure itself, through a
reduction reduction relation typically denoted by $\red$. Below, we
give a recursive presentation of this relation for the calculus used
in the encoding.

$\red \subseteq \pi \times \pi$
$\red : \pi \to \mathcal{P}(\pi)$

\begin{mathpar}
  \inferrule* [lab=Comm] { \textsf{match}( x_{src}, x_{trgt} ) } { x_{trgt}?(y)P \; | \; x_{src}!\langle {Q} \rangle \red P\{\quotep{Q}/y}\} }
  \and \\
  \inferrule* [lab=Par] {{P} \red {P}'} {{{P} | {Q}} \red {{P}' | {Q}}}
  \and
  \inferrule* [lab=Equiv]{{{P} \scong {P}'} \andalso {{P}' \red {Q}'} \andalso {{Q}' \scong {Q}}}{{P} \red {Q}}
\end{mathpar}

\begin{eqnarray*}
  match_{\equiv} (\quotep{P},\quotep{Q}) & := & P \equiv Q \\
  match_{\dagger}(\quotep{P},\quotep{Q}) & := & \forall R. P|Q \red^{*} R => R \red^{*} 0 \\
  match_{K}(\quotep{P},\quotep{Q}) & := & K \mbox{ for some context } K
\end{eqnarray*}

$u?(x)P | u!\langle Q \rangle \red P\{\quotep{Q}/x\}$

%We write $\wred$ for $\red^*$, and $P\red$ if $\exists Q $ such that $ P \red Q$.
We write $P\red$ if $\exists Q $ such that $ P \red Q$ and $P\not\red$, otherwise.

\section{Replication}

As mentioned before, it is known that replication (and hence
recursion) can be implemented in a higher-order process algebra
\cite{SangiorgiWalker}. As our first example of calculation with the
machinery thus far presented we give the construction explicitly in
the {\rhoc}.

\begin{eqnarray}
	D_{x} & := & \prefix{x}{y}{(\binpar{\outputp{x}{y}}{@{y}})} \nonumber\\
	\bangp_{x}{P} & := & \binpar{{x}!\langle{\binpar{D_{x}}{P}}\rangle}{D_{x}} \nonumber
\end{eqnarray}

\begin{eqnarray}
	\bangp_{x}{P} & & \nonumber\\
	=
	& {x}!\langle{(\prefix{x}{y}{(\outputp{x}{y} | @{y})) | P}}\rangle 
	      | \prefix{x}{y}{(\outputp{x}{y} | @{y})} & \nonumber\\
	\red
	& (\outputp{x}{y} | @{y})\substn{\quotep{(\prefix{x}{y}{(@{y} | \outputp{x}{y})) | P}}}{y} & \nonumber\\
	=
	& \outputp{x}{\quotep{(\prefix{x}{y}{(\outputp{x}{y} | @{y})) | P}}}
	  | {(\prefix{x}{y}{(\outputp{x}{y} | @{y})) | P}} & \nonumber\\
	\red
	& \ldots & \nonumber\\
	\red^*
	& P | P | \ldots & \nonumber
\end{eqnarray}

Of course, this encoding, as an implementation, runs away, unfolding
$\bangp{P}$ eagerly. A lazier and more implementable replication
operator, restricted to input-guarded processes, may be obtained as follows.

\begin{eqnarray}
\bangp{\prefix{u}{v}{P}} 
	:= 
	\binpar{\lift{x}{\prefix{u}{v}{(\binpar{D(x)}{P})}}}{D(x)} \nonumber
\end{eqnarray}

\begin{remark}
  Note that the lazier definition still does not deal with summation
  or mixed summation (i.e. sums over input and output). The reader is
  invited to construct definitions of replication that deal with these
  features. 

  Further, the definitions are parameterized in a name, $x$. Can you,
  gentle reader, make a definition that eliminates this parameter and
  guarantees no accidental interaction between the replication
  machinery and the process being replicated -- i.e. no accidental
  sharing of names used by the process to get its work done and the
  name(s) used by the replication to effect copying. This latter
  revision of the definition of replication is crucial to obtaining
  the expected identity $!!P \sim !P$.
\end{remark}

\begin{remark}\label{rem:paradoxical_combinator}
  The reader familiar with the lambda calculus will have noticed the
  similarity between $D$ and the paradoxical combinator.

  [Ed. note: the existence of this seems to suggest we have to be more
  restrictive on the set of processes and names we admit if we are to
  support no-cloning.]
\end{remark}

\subsubsection{Bisimulation}

The computational dynamics gives rise to another kind of equivalence,
the equivalence of computational behavior. As previously mentioned
this is typically captured \emph{via} some form of bisimulation.

% The notion we use in this paper is weak barbed bisimulation
% \cite{milner91polyadicpi}.

The notion we use in this paper is derived from weak barbed
bisimulation \cite{milner91polyadicpi}. 

\begin{definition}
An \emph{observation relation}, $\downarrow_{\mathcal N}$, over a set
of names, $\mathcal N$, is the smallest relation satisfying the rules
below.

\infrule[Out-barb]{y \in {\mathcal N}, \; x \nameeq y}
		  {\outputp{x}{v} \downarrow_{\mathcal N} x}
\infrule[Par-barb]{\mbox{$P\downarrow_{\mathcal N} x$ or $Q\downarrow_{\mathcal N} x$}}
		  {\binpar{P}{Q} \downarrow_{\mathcal N} x}

We write $P \Downarrow_{\mathcal N} x$ if there is $Q$ such that 
$P \wred Q$ and $Q \downarrow_{\mathcal N} x$.
\end{definition}

\begin{definition}
%\label{def.bbisim}
An  ${\mathcal N}$-\emph{barbed bisimulation} over a set of names, ${\mathcal N}$, is a symmetric binary relation 
${\mathcal S}_{\mathcal N}$ between agents such that $P\rel{S}_{\mathcal N}Q$ implies:
\begin{enumerate}
\item If $P \red P'$ then $Q \wred Q'$ and $P'\rel{S}_{\mathcal N} Q'$.
\item If $P\downarrow_{\mathcal N} x$, then $Q\Downarrow_{\mathcal N} x$.
\end{enumerate}
$P$ is ${\mathcal N}$-barbed bisimilar to $Q$, written
$P \wbbisim_{\mathcal N} Q$, if $P \rel{S}_{\mathcal N} Q$ for some ${\mathcal N}$-barbed bisimulation ${\mathcal S}_{\mathcal N}$.
\end{definition}

$\mathcal{R} \subseteq \pi \times \pi$

$P \mathcal{R} Q => \forall P'. P \red P' \Rightarrow \exists Q'. Q \red Q', P' \mathcal{R} Q'$

$P \vdash x \Rightarrow Q \vdash x$

\begin{mathpar}
  \inferrule*[lab=Out-barb]{x \nameeq y}{{y}!\langle{Q}\rangle \vdash x}
  \and
  \inferrule*[lab=Par-barb]{\mbox{$P\vdash x$ or $Q\vdash x$}}{\binpar{P}{Q} \vdash x}
\end{mathpar}

\subsubsection{Contexts}

One of the principle advantages of computational calculi like the
$\pi$-calculus is a well-defined notion of context,
contextual-equivalence and a correlation between
contextual-equivalence and notions of bisimulation. The notion of
context allows the decomposition of a process into (sub-)process and
its syntactic environment, its context. Thus, a context may be
thought of as a process with a ``hole'' (written $\Box$) in it. The
application of a context $M$ to a process $P$, written $M[P]$, is
tantamount to filling the hole in $M$ with $P$. In this paper we do
not need the full weight of this theory, but do make use of the notion
of context in the proof the main theorem. 

\begin{mathpar}
  \inferrule* [lab=summation] {} {{M_{M},M_{N}} \bc \Box \;|\; x.M_{A} \;|\; M_{M}+M_{N}}
  \and
  \inferrule* [lab=agent] {} {{M_{A}} \bc (\vec{x})M_{P} \;| \; \clift{P_0,\ldots,M_{P},\ldots,P_N}}
  \and \\
  \inferrule* [lab=process] {} {{M_{P}} \bc M_{N} \;| \;P|M_{P} }
\end{mathpar} 

\begin{mathpar}
  \inferrule* [lab=sychronization] {} {M_{N} \bc \Box \;|\; x?M_{F} \;|\; x!M_{C}}
  \and
  \inferrule* [lab=abstraction] {} {{M_{F}} \bc (x)M_{P} }
  \and
  \inferrule* [lab=concretion] {} {{M_{C}} \bc \langle M_{P} \rangle }
  \and \\
  \inferrule* [lab=process] {} {{M_{P}} \bc M_{N} \;| \;P|M_{P} }
\end{mathpar}

\begin{definition}[contextual application] Given a context $M$, and
  process $P$, we define the \emph{contextual application}, $M[P] :=
  M\{P/\Box\}$. That is, the contextual application of M to P is the
  substitution of $P$ for $\Box$ in $M$.
\end{definition}

$\meaningof{-} : L \to \mathcal{P}(\pi)$

\begin{mathpar}
  \inferrule* [lab=collection] {} {\meaningof{true} = \pi, \and \meaningof{~E} = \pi \setminus \meaningof{E}, \and \meaningof{E_{1} \& E_{2}} = \meaningof{E_{1}} \cap \meaningof{E_{2}}}
\end{mathpar}

\begin{mathpar}
  \inferrule* [lab=structure] {} {\meaningof{0} = \{ P \in \pi | P \equiv 0 \}, \and \\ \meaningof{E_1 | E_2} = \{ P \in \pi | P \equiv P_{1} | P_{2}, P_{1} \in \meaningof{E_{1}}, P_{2} \in \meaningof{E_2}\} }
\end{mathpar}

\begin{mathpar}
 \inferrule* [lab=behavior] {} {\meaningof{\langle a?b \rangle E} = \{ P \in \pi | P \equiv Q | u?(y)P', \\ \and \\\\ \and \\ \;\;\; u \in \meaningof{a}, \forall z.P'\{z/y\} \in \meaningof{E\{z/b\}}\}, \and \\ \meaningof{a!E} = \{ P \in \pi | P \equiv Q | x!\langle P' \rangle, x \in \meaningof{a} P' \in \meaningof{E}\} }
\end{mathpar}

\begin{mathpar}
 \inferrule* [lab=nominal] {} {\meaningof{\quotep{E}} = \{ \quotep{P} \in \quotep{\pi} | P \in \meaningof{E} \}, \and \meaningof{\quotep{P}} = \{ \quotep{Q} \in \quotep{\pi} | P \equiv Q \} \and \\ \meaningof{@\quotep{E}} = \{ P \in \pi | P \equiv @x, x \in \meaningof{E} \}}
\end{mathpar}

\begin{eqnarray*}
  \\
  \meaningof{-} : TS \to ST
\end{eqnarray*}

\begin{eqnarray*}
  \\
  L : TS \to ST
\end{eqnarray*}

\begin{eqnarray*}
  \\
  P \models E \iff P \in \meaningof{E}
\end{eqnarray*}

\begin{eqnarray*}
  P \approx_{L} Q \iff \forall E \in L. P \models E \iff Q \models E
\end{eqnarray*}

\begin{eqnarray*}
  P \approx_{K} Q
\end{eqnarray*}

\begin{eqnarray*}
  P \approx Q
\end{eqnarray*}

$\approx_{K} = \approx = \approx_{L}$

\subsubsection{Contextual duality}

Note that contexts extend the quotation operation to a family of
operations from processes to names. Given a context, $M$, we can
define a \emph{nominal context}, $\quotep{M}$ by $\quotep{M}[P] :=
\quotep{M[P]}$. To foreshadow what is to come we observe that these
operations enjoy a duality with processes very much like the duality
between vectors and maps from vectors to scalars.

Further, because the calculus is essentially higher-order, we have a
correspondence between contexts and processes. More specifically,
given a name $x$ and a context $M$ we can construct $M^{*}_{x}$ such
that 

\begin{mathpar}
  M^{*}_{x} | \lift{x}{P} \red M[P]
\end{mathpar}

namely,

\begin{mathpar}
  M^{*}_{x} := x?(u).M[\dropn{u}]
\end{mathpar}

The dependence of $M^{*}_{x}$ on a name makes it an abstraction, 

\begin{mathpar}
  M^{*} := (x)x?(u).M[\dropn{u}]
\end{mathpar}

\subsection{Additional notation}

It will sometimes be convenient to denote the process a name
quotes. We already have the notation $x = \quotep{P}$, but it will be
convenient to introduce an alternate notation, $\procn{x}$, when we
want to emphasize the connection to the use of the name. Note that, by
virtue of name equivalence, $\quotep{\procn{x}} \nameeq x$; so, the
notation is consistent with previous definitions.

Further, because names have structure it is possible to effect
substitutions on the basis of that structure. This means we need to
upgrade our notation for substitutions, which we accomplish by
adapting comprehension notation. Thus,

\begin{mathpar}
  P\{ y / x : x \in S \}
\end{mathpar}

is interpreted to mean the process derived from P by replacing (in a
capture-avoiding manner) each occurrence of $x$ in $S$ by $y$. For example,

\begin{mathpar}
  P\{ \quotep{\procn{x}|\procn{x}} / x : x \in \freenames{P} \}
\end{mathpar}

will replace each (occurrence) of a free name $x$ in $P$ by
$\quotep{\procn{x}|\procn{x}}$.

Also, we will avail ourselves of the notation $x^{L}$ and $x^{R}$ to
denote injections of a name into disjoint copies of the name
space. There are numerous ways to accomplish this. One example can be
found in \cite{MeredithR05}. This notation overloads to vectors of
names: $\vec{x}^{\pi} := (x_{i}^{\pi} \; : \; 0 \leq i < |\vec{x}| )$ where $\pi \in \{L,R\}$.

We also use $P^{\Box} := P|\Box$.

In \cite{MeredithR05} an interpretation of the new operator is
given. It turns out that there are several possible interpretations
all enjoying the requisite algebraic properties of the operator (see
\cite{milner91polyadicpi}). We will therefore make liberal use of
$(\nu\; \vec{x})P$.

% subsection the_syntax_and_semantics_of_the_notation_system (end)   

\input{qm2pi.qmops} 

\input{qm2pi.sterngerlach} 

\input{qm2pi.metric} 

% section concurrent_process_calculi (end)

%\input{qm2pi.proofsketch}

% section proof sketch (end)

%\input{qm2pi.slviaknots} 

% section spatial logic via knots (end)

\input{qm2pi.conclusion}

% section conclusion (end)

%\input{qm2pi.dtcodes} 

% section wiring algorithm (end)

\input{qm2pi.ack} 

% section acknowledgments (end)

\newpage


\bibliographystyle{plain}   
\bibliography{../../biblios/main.bib}

\input{qm2pi.rhodetails}

\end{document}

 

% section acknowledgments (end)

\newpage


\bibliographystyle{plain}   
\bibliography{../../biblios/main.bib}

\documentclass[12pt]{llncs}
%\documentclass{jktr}

\usepackage[pdftex]{hyperref}                   
\usepackage {listings}
\usepackage {mathpartir}
\usepackage{bcprules}
%\usepackage{listings}
                       
\usepackage{graphicx} 
%\usepackage[margins=2.5cm,nohead,nofoot]{geometry}
%\usepackage{geometry}
\usepackage{amsfonts}
\usepackage{amstext}
\usepackage{latexsym}
\usepackage{amssymb}
\usepackage{color}


%\include{myPreamble}
\include{qm2pi.local} 

%\ifpdf
%\usepackage[pdftex]{graphicx}
%\else
%\usepackage{graphicx}
%\fi

 % \ifpdf
%  \usepackage{pdfsync}
%  \if


%\title{Brief Article}
%\author{David F. Snyder}
%\author{L.G. Meredith}

%\address{Dept. of Math., Texas State University--San Marcos, San Marcos, TX 78666}
       
\pagestyle{empty}


\begin{document}

\lstset{language=[Objective]Caml,frame=shadowbox}

\input{qm2pi.front}

% section front matter (end)

\input{qm2pi.intro} 
 
% section introduction (end)

% \input{qm2pi.knotations} 

% section notation (end)

\input{qm2pi.process.calculi} 

% section concurrent_process_calculi_and_spatial_logics_ (end)
    
%\input{qm2pi.knots2pi} 

%\input{qm2pi.trefoil} 

%\input{qm2pi.mainthm} 

% subsection basic_interpretation (end)

%\input{qm2pi.rho.presentation} 
\subsection{The syntax and semantics of the notation system}\label{sub:the_syntax_and_semantics_of_the_notation_system} % (fold)

We now summarize a technical presentation of the calculus that
embodies our theory of dynamics. The typical presentation of such a
calculus follows the style of giving generators and relations on
them. The grammar, below, describing term constructors, freely
generates the set of processes, $\Proc$. This set is then quotiented
by a relation known as structural congruence and it is over this set
that the notion of dynamics is expressed. This presentation is
essentially that of \cite{MeredithR05} with the addition of
polyadicity and summation. For readability we have relegated some of
the technical subtleties to an appendix.

\subsubsection{Process grammar}\label{subsub:process_grammar}

\begin{mathpar}
  \inferrule* [lab=synchronization] {} {{M} \bc \pzero \;|\; x?F \;|\; x!C }
  \and
  \inferrule* [lab=abstraction] {} {{F} \bc (x)P}
  \and
  \inferrule* [lab=concretion] {} {{C} \bc \langle Q \rangle}
  \and
  \inferrule* [lab=process] {} {{P,Q} \bc M \;| \;P|Q \;|\; @{x}}
  \and
  \inferrule* [lab=name] {} {{x} \bc \quotep{P}}
\end{mathpar} 

Note that $\vec{x}$ (resp. $\vec{P}$) denotes a vector of names
(resp. processes) of length $|\vec{x}|$ (resp. $|\vec{P}|$). We adopt
the following useful abbreviations.

\begin{mathpar}
   x?(\vec{y}).P := x.(\vec{y})P \and  x\clift{\vec{P}} := x.\clift{\vec{P}}
   \and x!(y) := \lift{x}{\dropn{y}}
   \and \Pi_{i=0}^{n-1}P_i := P_0 | \ldots | P_{n-1}
\end{mathpar}

\subsubsection{Structural congruence}

\paragraph{Free and bound names and alpha-equivalence.} At the
core of structural equivalence is alpha-equivalence which identifies
process that are the same up to a change of variable. Formally, we
recognize the distinction between free and bound names. The free names
of a process, $\freenames{P}$, may be calculated recursively as
follows:

\begin{mathpar}
\freenames{\pzero} := \emptyset
  \and \\
  \freenames{x?(y).P} := \{ x \} \cup (\freenames{P} \setminus \{ y \})
  \and 
  \freenames{x!\langle P \rangle} := \{ x \} \cup \{ P \} 
  \and \\
  \freenames{P|Q} := \freenames{P} \cup \freenames{Q}
  \and \\
  \freenames{@{x}} := \{ x \}
\end{mathpar}

$\pi$
$\quotep{\pi}$

$\freenames{-} : \pi \to \mathcal{P}(\quotep{\pi})$

\begin{eqnarray*}
  \freenames{\pzero} & := & \emptyset \\
  \freenames{x?(y).P} & := & \{ x \} \cup (\freenames{P} \setminus \{ y \}) \\
  \freenames{x!\langle P \rangle} & := & \{ x \} \cup \{ P \} \\
  \freenames{P|Q} & := & \freenames{P} \cup \freenames{Q} \\
  \freenames{\dropn{x}} & := & \{ x \}
\end{eqnarray*}

The bound names of a process, $\boundnames{P}$, are those names occurring in $P$
that are not free. For example, in $x?(y).0$, the name $x$ is free, while $y$ is bound.

\begin{mathpar}
  \inferrule* [lab=monoidal-laws] {} { P|Q \equiv Q|P \and P|0 \equiv P \and P|(Q|R) \equiv (P|Q)|R }
\end{mathpar}

\begin{mathpar}
  \inferrule* [lab=alpha-equivalence] {} { (x)P \equiv (y)P\{y/x\} \and y \not\in \freenames{P} }
\end{mathpar}

\begin{definition}
Then two processes, $P,Q$, are alpha-equivalent if $P = Q\{\vec{y}/\vec{x}\}$ for
some $\vec{x} \in \boundnames{Q},\vec{y} \in \boundnames{P}$, where $Q\{\vec{y}/\vec{x}\}$
denotes the capture-avoiding substitution of $\vec{y}$ for $\vec{x}$ in $Q$.
\end{definition}

\begin{definition}
  The {\em structural congruence} \cite{SangiorgiWalker} , $\equiv$,
  between processes is the least congruence containing
  alpha-equivalence, satisfying the abelian monoid laws
  (associativity, commutativity and $\pzero$ as identity) for parallel
  composition $|$ and for summation $+$.
\end{definition}

\subsection{Name equivalence}

We take name equivalence, written $\nameeq$, to be the smallest
equivalence relation generated by the following rules.

\begin{mathpar}
\inferrule*[lab=Quote-drop]
{ }
{ \quotep{@{x}} \nameeq x }

\inferrule*[lab=Struct-equiv]
{ P \scong Q }
{ \quotep{P} \nameeq \quotep{Q} }
\end{mathpar}

The astute reader will have noticed that the mutual recursion of names
and processes imposes a mutual recursion on alpha-equivalence and
structural equivalence via name-equivalence. Fortunately, all of this
works out pleasantly and we may calculate in the natural way, free of
concern. The reader interested in the details is referred to the
appendix \ref{appendix:rho_details}.

\subsection{Substitution}

We use $\Proc$ for the set of processes, $\QProc$ for the set of
names, and $\id{\{}\vec{y} / \vec{x} \id{\}}$ to denote partial maps,
$s : \QProc \rightarrow \QProc$. A map, $s$ lifts, uniquely, to a map
on process terms, $\widehat{s} : \Proc \rightarrow \Proc$ by the
following equations.

\begin{mathpar}
  (0) \psubstp{Q}{P} := 0 \\
  (R \juxtap S) \psubstp{Q}{P}
  :=    
  (R)\psubstp{Q}{P} \juxtap (S) \psubstp{Q}{P} \\
  (x?(y).R) \psubstp{Q}{P}    
  :=    
  (x)\substp{Q}{P} (z)\concat( (R \psubstn{z}{y}) \psubstp{Q}{P} ) \\
  (\lift{x}{R}) \psubstp{Q}{P}  
  :=
  \lift{(x)\substp{Q}{P}}{ R \psubstp{Q}{P} } \\
%   (\dropn{x})  \psubstp{Q}{P}       
%   := 
%   \left\{ 
%     \begin{array}{ccc} 
%       \dropn{\quotep{Q}} & & x \nameeq \quotep{P} \\
%       \dropn{x} & & otherwise \\
%     \end{array}
%   \right. 
  (\dropn{x})  \psubstp{Q}{P}       
  := 
  \left\{ 
    \begin{array}{ccc} 
      Q & & x \nameeq \quotep{P} \\
      \dropn{x} & & otherwise \\
    \end{array}
  \right.
\end{mathpar}
 

where

\begin{eqnarray}
  (x)\id{\{} \lpquote Q \rpquote / \lpquote P \rpquote \id{\}}            = 
  \left\{ 
    \begin{array}{ccc}
      \lpquote Q \rpquote & & x \nameeq \lpquote P \rpquote \\
      x & & otherwise \\
    \end{array}
  \right. \nonumber
\end{eqnarray}

and $z$ is chosen distinct from $\quotep{P}$, $\quotep{Q}$, the free
names in $Q$, and all the names in $R$. Our $\alpha$-equivalence will
be built in the standard way from this substitution.

\begin{remark}\label{rem:no_self_referential_names}
  One consequence of these definitions is that $\forall P. \quotep{P}
  \not\in \freenames{P}$.
\end{remark}

\subsection{ Dynamic quote: an example }

Anticipating something of what's to come, consider applying the
substitution, $\widehat{\id{\{}u / z \id{\}}}$, to the following pair
of processes, $\lift{w}{y!(z)}$ and $w[ \lpquote y!(z) \rpquote ]$.

\begin{eqnarray}
	\lift{w}{y!(z)}\widehat{\id{\{}u / z \id{\}}}
		& = &
		\lift{w}{y!(u)} \nonumber\\
	w[ \lpquote y!(z) \rpquote ] \widehat{ \id{\{}u / z \id{\}} }
		& = &
		w[ \lpquote y!(z) \rpquote ] \nonumber
\end{eqnarray}

Because the body of the process between quotes is impervious to
substitution, we get radically different answers. In fact, by
examining the first process in an input context,
e.g. $x?(z).\lift{w}{y!(z)}$, we see that the process under the lift
operator may be shaped by prefixed inputs binding a name inside it. In
this sense, the lift operator will be seen as a way to dynamically
construct processes before reifying them as names.

Finally equipped with these standard features we can present the
dynamics of the calculus.

\subsubsection{Operational semantics} 

Finally, we introduce the computational dynamics. What marks these
algebras as distinct from other more traditionally studied algebraic
structures, e.g. vector spaces or polynomial rings, is the manner in
which dynamics is captured. In traditional structures, dynamics is typically
expressed through morphisms between such structures, as in linear maps
between vector spaces or morphisms between rings. In algebras
associated with the semantics of computation, the dynamics is
expressed as part of the algebraic structure itself, through a
reduction reduction relation typically denoted by $\red$. Below, we
give a recursive presentation of this relation for the calculus used
in the encoding.

$\red \subseteq \pi \times \pi$
$\red : \pi \to \mathcal{P}(\pi)$

\begin{mathpar}
  \inferrule* [lab=Comm] { \textsf{match}( x_{src}, x_{trgt} ) } { x_{trgt}?(y)P \; | \; x_{src}!\langle {Q} \rangle \red P\{\quotep{Q}/y}\} }
  \and \\
  \inferrule* [lab=Par] {{P} \red {P}'} {{{P} | {Q}} \red {{P}' | {Q}}}
  \and
  \inferrule* [lab=Equiv]{{{P} \scong {P}'} \andalso {{P}' \red {Q}'} \andalso {{Q}' \scong {Q}}}{{P} \red {Q}}
\end{mathpar}

\begin{eqnarray*}
  match_{\equiv} (\quotep{P},\quotep{Q}) & := & P \equiv Q \\
  match_{\dagger}(\quotep{P},\quotep{Q}) & := & \forall R. P|Q \red^{*} R => R \red^{*} 0 \\
  match_{K}(\quotep{P},\quotep{Q}) & := & K \mbox{ for some context } K
\end{eqnarray*}

$u?(x)P | u!\langle Q \rangle \red P\{\quotep{Q}/x\}$

%We write $\wred$ for $\red^*$, and $P\red$ if $\exists Q $ such that $ P \red Q$.
We write $P\red$ if $\exists Q $ such that $ P \red Q$ and $P\not\red$, otherwise.

\section{Replication}

As mentioned before, it is known that replication (and hence
recursion) can be implemented in a higher-order process algebra
\cite{SangiorgiWalker}. As our first example of calculation with the
machinery thus far presented we give the construction explicitly in
the {\rhoc}.

\begin{eqnarray}
	D_{x} & := & \prefix{x}{y}{(\binpar{\outputp{x}{y}}{@{y}})} \nonumber\\
	\bangp_{x}{P} & := & \binpar{{x}!\langle{\binpar{D_{x}}{P}}\rangle}{D_{x}} \nonumber
\end{eqnarray}

\begin{eqnarray}
	\bangp_{x}{P} & & \nonumber\\
	=
	& {x}!\langle{(\prefix{x}{y}{(\outputp{x}{y} | @{y})) | P}}\rangle 
	      | \prefix{x}{y}{(\outputp{x}{y} | @{y})} & \nonumber\\
	\red
	& (\outputp{x}{y} | @{y})\substn{\quotep{(\prefix{x}{y}{(@{y} | \outputp{x}{y})) | P}}}{y} & \nonumber\\
	=
	& \outputp{x}{\quotep{(\prefix{x}{y}{(\outputp{x}{y} | @{y})) | P}}}
	  | {(\prefix{x}{y}{(\outputp{x}{y} | @{y})) | P}} & \nonumber\\
	\red
	& \ldots & \nonumber\\
	\red^*
	& P | P | \ldots & \nonumber
\end{eqnarray}

Of course, this encoding, as an implementation, runs away, unfolding
$\bangp{P}$ eagerly. A lazier and more implementable replication
operator, restricted to input-guarded processes, may be obtained as follows.

\begin{eqnarray}
\bangp{\prefix{u}{v}{P}} 
	:= 
	\binpar{\lift{x}{\prefix{u}{v}{(\binpar{D(x)}{P})}}}{D(x)} \nonumber
\end{eqnarray}

\begin{remark}
  Note that the lazier definition still does not deal with summation
  or mixed summation (i.e. sums over input and output). The reader is
  invited to construct definitions of replication that deal with these
  features. 

  Further, the definitions are parameterized in a name, $x$. Can you,
  gentle reader, make a definition that eliminates this parameter and
  guarantees no accidental interaction between the replication
  machinery and the process being replicated -- i.e. no accidental
  sharing of names used by the process to get its work done and the
  name(s) used by the replication to effect copying. This latter
  revision of the definition of replication is crucial to obtaining
  the expected identity $!!P \sim !P$.
\end{remark}

\begin{remark}\label{rem:paradoxical_combinator}
  The reader familiar with the lambda calculus will have noticed the
  similarity between $D$ and the paradoxical combinator.

  [Ed. note: the existence of this seems to suggest we have to be more
  restrictive on the set of processes and names we admit if we are to
  support no-cloning.]
\end{remark}

\subsubsection{Bisimulation}

The computational dynamics gives rise to another kind of equivalence,
the equivalence of computational behavior. As previously mentioned
this is typically captured \emph{via} some form of bisimulation.

% The notion we use in this paper is weak barbed bisimulation
% \cite{milner91polyadicpi}.

The notion we use in this paper is derived from weak barbed
bisimulation \cite{milner91polyadicpi}. 

\begin{definition}
An \emph{observation relation}, $\downarrow_{\mathcal N}$, over a set
of names, $\mathcal N$, is the smallest relation satisfying the rules
below.

\infrule[Out-barb]{y \in {\mathcal N}, \; x \nameeq y}
		  {\outputp{x}{v} \downarrow_{\mathcal N} x}
\infrule[Par-barb]{\mbox{$P\downarrow_{\mathcal N} x$ or $Q\downarrow_{\mathcal N} x$}}
		  {\binpar{P}{Q} \downarrow_{\mathcal N} x}

We write $P \Downarrow_{\mathcal N} x$ if there is $Q$ such that 
$P \wred Q$ and $Q \downarrow_{\mathcal N} x$.
\end{definition}

\begin{definition}
%\label{def.bbisim}
An  ${\mathcal N}$-\emph{barbed bisimulation} over a set of names, ${\mathcal N}$, is a symmetric binary relation 
${\mathcal S}_{\mathcal N}$ between agents such that $P\rel{S}_{\mathcal N}Q$ implies:
\begin{enumerate}
\item If $P \red P'$ then $Q \wred Q'$ and $P'\rel{S}_{\mathcal N} Q'$.
\item If $P\downarrow_{\mathcal N} x$, then $Q\Downarrow_{\mathcal N} x$.
\end{enumerate}
$P$ is ${\mathcal N}$-barbed bisimilar to $Q$, written
$P \wbbisim_{\mathcal N} Q$, if $P \rel{S}_{\mathcal N} Q$ for some ${\mathcal N}$-barbed bisimulation ${\mathcal S}_{\mathcal N}$.
\end{definition}

$\mathcal{R} \subseteq \pi \times \pi$

$P \mathcal{R} Q => \forall P'. P \red P' \Rightarrow \exists Q'. Q \red Q', P' \mathcal{R} Q'$

$P \vdash x \Rightarrow Q \vdash x$

\begin{mathpar}
  \inferrule*[lab=Out-barb]{x \nameeq y}{{y}!\langle{Q}\rangle \vdash x}
  \and
  \inferrule*[lab=Par-barb]{\mbox{$P\vdash x$ or $Q\vdash x$}}{\binpar{P}{Q} \vdash x}
\end{mathpar}

\subsubsection{Contexts}

One of the principle advantages of computational calculi like the
$\pi$-calculus is a well-defined notion of context,
contextual-equivalence and a correlation between
contextual-equivalence and notions of bisimulation. The notion of
context allows the decomposition of a process into (sub-)process and
its syntactic environment, its context. Thus, a context may be
thought of as a process with a ``hole'' (written $\Box$) in it. The
application of a context $M$ to a process $P$, written $M[P]$, is
tantamount to filling the hole in $M$ with $P$. In this paper we do
not need the full weight of this theory, but do make use of the notion
of context in the proof the main theorem. 

\begin{mathpar}
  \inferrule* [lab=summation] {} {{M_{M},M_{N}} \bc \Box \;|\; x.M_{A} \;|\; M_{M}+M_{N}}
  \and
  \inferrule* [lab=agent] {} {{M_{A}} \bc (\vec{x})M_{P} \;| \; \clift{P_0,\ldots,M_{P},\ldots,P_N}}
  \and \\
  \inferrule* [lab=process] {} {{M_{P}} \bc M_{N} \;| \;P|M_{P} }
\end{mathpar} 

\begin{mathpar}
  \inferrule* [lab=sychronization] {} {M_{N} \bc \Box \;|\; x?M_{F} \;|\; x!M_{C}}
  \and
  \inferrule* [lab=abstraction] {} {{M_{F}} \bc (x)M_{P} }
  \and
  \inferrule* [lab=concretion] {} {{M_{C}} \bc \langle M_{P} \rangle }
  \and \\
  \inferrule* [lab=process] {} {{M_{P}} \bc M_{N} \;| \;P|M_{P} }
\end{mathpar}

\begin{definition}[contextual application] Given a context $M$, and
  process $P$, we define the \emph{contextual application}, $M[P] :=
  M\{P/\Box\}$. That is, the contextual application of M to P is the
  substitution of $P$ for $\Box$ in $M$.
\end{definition}

$\meaningof{-} : L \to \mathcal{P}(\pi)$

\begin{mathpar}
  \inferrule* [lab=collection] {} {\meaningof{true} = \pi, \and \meaningof{~E} = \pi \setminus \meaningof{E}, \and \meaningof{E_{1} \& E_{2}} = \meaningof{E_{1}} \cap \meaningof{E_{2}}}
\end{mathpar}

\begin{mathpar}
  \inferrule* [lab=structure] {} {\meaningof{0} = \{ P \in \pi | P \equiv 0 \}, \and \\ \meaningof{E_1 | E_2} = \{ P \in \pi | P \equiv P_{1} | P_{2}, P_{1} \in \meaningof{E_{1}}, P_{2} \in \meaningof{E_2}\} }
\end{mathpar}

\begin{mathpar}
 \inferrule* [lab=behavior] {} {\meaningof{\langle a?b \rangle E} = \{ P \in \pi | P \equiv Q | u?(y)P', \\ \and \\\\ \and \\ \;\;\; u \in \meaningof{a}, \forall z.P'\{z/y\} \in \meaningof{E\{z/b\}}\}, \and \\ \meaningof{a!E} = \{ P \in \pi | P \equiv Q | x!\langle P' \rangle, x \in \meaningof{a} P' \in \meaningof{E}\} }
\end{mathpar}

\begin{mathpar}
 \inferrule* [lab=nominal] {} {\meaningof{\quotep{E}} = \{ \quotep{P} \in \quotep{\pi} | P \in \meaningof{E} \}, \and \meaningof{\quotep{P}} = \{ \quotep{Q} \in \quotep{\pi} | P \equiv Q \} \and \\ \meaningof{@\quotep{E}} = \{ P \in \pi | P \equiv @x, x \in \meaningof{E} \}}
\end{mathpar}

\begin{eqnarray*}
  \\
  \meaningof{-} : TS \to ST
\end{eqnarray*}

\begin{eqnarray*}
  \\
  L : TS \to ST
\end{eqnarray*}

\begin{eqnarray*}
  \\
  P \models E \iff P \in \meaningof{E}
\end{eqnarray*}

\begin{eqnarray*}
  P \approx_{L} Q \iff \forall E \in L. P \models E \iff Q \models E
\end{eqnarray*}

\begin{eqnarray*}
  P \approx_{K} Q
\end{eqnarray*}

\begin{eqnarray*}
  P \approx Q
\end{eqnarray*}

$\approx_{K} = \approx = \approx_{L}$

\subsubsection{Contextual duality}

Note that contexts extend the quotation operation to a family of
operations from processes to names. Given a context, $M$, we can
define a \emph{nominal context}, $\quotep{M}$ by $\quotep{M}[P] :=
\quotep{M[P]}$. To foreshadow what is to come we observe that these
operations enjoy a duality with processes very much like the duality
between vectors and maps from vectors to scalars.

Further, because the calculus is essentially higher-order, we have a
correspondence between contexts and processes. More specifically,
given a name $x$ and a context $M$ we can construct $M^{*}_{x}$ such
that 

\begin{mathpar}
  M^{*}_{x} | \lift{x}{P} \red M[P]
\end{mathpar}

namely,

\begin{mathpar}
  M^{*}_{x} := x?(u).M[\dropn{u}]
\end{mathpar}

The dependence of $M^{*}_{x}$ on a name makes it an abstraction, 

\begin{mathpar}
  M^{*} := (x)x?(u).M[\dropn{u}]
\end{mathpar}

\subsection{Additional notation}

It will sometimes be convenient to denote the process a name
quotes. We already have the notation $x = \quotep{P}$, but it will be
convenient to introduce an alternate notation, $\procn{x}$, when we
want to emphasize the connection to the use of the name. Note that, by
virtue of name equivalence, $\quotep{\procn{x}} \nameeq x$; so, the
notation is consistent with previous definitions.

Further, because names have structure it is possible to effect
substitutions on the basis of that structure. This means we need to
upgrade our notation for substitutions, which we accomplish by
adapting comprehension notation. Thus,

\begin{mathpar}
  P\{ y / x : x \in S \}
\end{mathpar}

is interpreted to mean the process derived from P by replacing (in a
capture-avoiding manner) each occurrence of $x$ in $S$ by $y$. For example,

\begin{mathpar}
  P\{ \quotep{\procn{x}|\procn{x}} / x : x \in \freenames{P} \}
\end{mathpar}

will replace each (occurrence) of a free name $x$ in $P$ by
$\quotep{\procn{x}|\procn{x}}$.

Also, we will avail ourselves of the notation $x^{L}$ and $x^{R}$ to
denote injections of a name into disjoint copies of the name
space. There are numerous ways to accomplish this. One example can be
found in \cite{MeredithR05}. This notation overloads to vectors of
names: $\vec{x}^{\pi} := (x_{i}^{\pi} \; : \; 0 \leq i < |\vec{x}| )$ where $\pi \in \{L,R\}$.

We also use $P^{\Box} := P|\Box$.

In \cite{MeredithR05} an interpretation of the new operator is
given. It turns out that there are several possible interpretations
all enjoying the requisite algebraic properties of the operator (see
\cite{milner91polyadicpi}). We will therefore make liberal use of
$(\nu\; \vec{x})P$.

% subsection the_syntax_and_semantics_of_the_notation_system (end)   

\input{qm2pi.qmops} 

\input{qm2pi.sterngerlach} 

\input{qm2pi.metric} 

% section concurrent_process_calculi (end)

%\input{qm2pi.proofsketch}

% section proof sketch (end)

%\input{qm2pi.slviaknots} 

% section spatial logic via knots (end)

\input{qm2pi.conclusion}

% section conclusion (end)

%\input{qm2pi.dtcodes} 

% section wiring algorithm (end)

\input{qm2pi.ack} 

% section acknowledgments (end)

\newpage


\bibliographystyle{plain}   
\bibliography{../../biblios/main.bib}

\input{qm2pi.rhodetails}

\end{document}



\end{document}

 

% section acknowledgments (end)

\newpage


\bibliographystyle{plain}   
\bibliography{../../biblios/main.bib}

\documentclass[12pt]{llncs}
%\documentclass{jktr}

\usepackage[pdftex]{hyperref}                   
\usepackage {listings}
\usepackage {mathpartir}
\usepackage{bcprules}
%\usepackage{listings}
                       
\usepackage{graphicx} 
%\usepackage[margins=2.5cm,nohead,nofoot]{geometry}
%\usepackage{geometry}
\usepackage{amsfonts}
\usepackage{amstext}
\usepackage{latexsym}
\usepackage{amssymb}
\usepackage{color}


%\include{myPreamble}
\documentclass[12pt]{llncs}
%\documentclass{jktr}

\usepackage[pdftex]{hyperref}                   
\usepackage {listings}
\usepackage {mathpartir}
\usepackage{bcprules}
%\usepackage{listings}
                       
\usepackage{graphicx} 
%\usepackage[margins=2.5cm,nohead,nofoot]{geometry}
%\usepackage{geometry}
\usepackage{amsfonts}
\usepackage{amstext}
\usepackage{latexsym}
\usepackage{amssymb}
\usepackage{color}


%\include{myPreamble}
\include{qm2pi.local} 

%\ifpdf
%\usepackage[pdftex]{graphicx}
%\else
%\usepackage{graphicx}
%\fi

 % \ifpdf
%  \usepackage{pdfsync}
%  \if


%\title{Brief Article}
%\author{David F. Snyder}
%\author{L.G. Meredith}

%\address{Dept. of Math., Texas State University--San Marcos, San Marcos, TX 78666}
       
\pagestyle{empty}


\begin{document}

\lstset{language=[Objective]Caml,frame=shadowbox}

\input{qm2pi.front}

% section front matter (end)

\input{qm2pi.intro} 
 
% section introduction (end)

% \input{qm2pi.knotations} 

% section notation (end)

\input{qm2pi.process.calculi} 

% section concurrent_process_calculi_and_spatial_logics_ (end)
    
%\input{qm2pi.knots2pi} 

%\input{qm2pi.trefoil} 

%\input{qm2pi.mainthm} 

% subsection basic_interpretation (end)

%\input{qm2pi.rho.presentation} 
\subsection{The syntax and semantics of the notation system}\label{sub:the_syntax_and_semantics_of_the_notation_system} % (fold)

We now summarize a technical presentation of the calculus that
embodies our theory of dynamics. The typical presentation of such a
calculus follows the style of giving generators and relations on
them. The grammar, below, describing term constructors, freely
generates the set of processes, $\Proc$. This set is then quotiented
by a relation known as structural congruence and it is over this set
that the notion of dynamics is expressed. This presentation is
essentially that of \cite{MeredithR05} with the addition of
polyadicity and summation. For readability we have relegated some of
the technical subtleties to an appendix.

\subsubsection{Process grammar}\label{subsub:process_grammar}

\begin{mathpar}
  \inferrule* [lab=synchronization] {} {{M} \bc \pzero \;|\; x?F \;|\; x!C }
  \and
  \inferrule* [lab=abstraction] {} {{F} \bc (x)P}
  \and
  \inferrule* [lab=concretion] {} {{C} \bc \langle Q \rangle}
  \and
  \inferrule* [lab=process] {} {{P,Q} \bc M \;| \;P|Q \;|\; @{x}}
  \and
  \inferrule* [lab=name] {} {{x} \bc \quotep{P}}
\end{mathpar} 

Note that $\vec{x}$ (resp. $\vec{P}$) denotes a vector of names
(resp. processes) of length $|\vec{x}|$ (resp. $|\vec{P}|$). We adopt
the following useful abbreviations.

\begin{mathpar}
   x?(\vec{y}).P := x.(\vec{y})P \and  x\clift{\vec{P}} := x.\clift{\vec{P}}
   \and x!(y) := \lift{x}{\dropn{y}}
   \and \Pi_{i=0}^{n-1}P_i := P_0 | \ldots | P_{n-1}
\end{mathpar}

\subsubsection{Structural congruence}

\paragraph{Free and bound names and alpha-equivalence.} At the
core of structural equivalence is alpha-equivalence which identifies
process that are the same up to a change of variable. Formally, we
recognize the distinction between free and bound names. The free names
of a process, $\freenames{P}$, may be calculated recursively as
follows:

\begin{mathpar}
\freenames{\pzero} := \emptyset
  \and \\
  \freenames{x?(y).P} := \{ x \} \cup (\freenames{P} \setminus \{ y \})
  \and 
  \freenames{x!\langle P \rangle} := \{ x \} \cup \{ P \} 
  \and \\
  \freenames{P|Q} := \freenames{P} \cup \freenames{Q}
  \and \\
  \freenames{@{x}} := \{ x \}
\end{mathpar}

$\pi$
$\quotep{\pi}$

$\freenames{-} : \pi \to \mathcal{P}(\quotep{\pi})$

\begin{eqnarray*}
  \freenames{\pzero} & := & \emptyset \\
  \freenames{x?(y).P} & := & \{ x \} \cup (\freenames{P} \setminus \{ y \}) \\
  \freenames{x!\langle P \rangle} & := & \{ x \} \cup \{ P \} \\
  \freenames{P|Q} & := & \freenames{P} \cup \freenames{Q} \\
  \freenames{\dropn{x}} & := & \{ x \}
\end{eqnarray*}

The bound names of a process, $\boundnames{P}$, are those names occurring in $P$
that are not free. For example, in $x?(y).0$, the name $x$ is free, while $y$ is bound.

\begin{mathpar}
  \inferrule* [lab=monoidal-laws] {} { P|Q \equiv Q|P \and P|0 \equiv P \and P|(Q|R) \equiv (P|Q)|R }
\end{mathpar}

\begin{mathpar}
  \inferrule* [lab=alpha-equivalence] {} { (x)P \equiv (y)P\{y/x\} \and y \not\in \freenames{P} }
\end{mathpar}

\begin{definition}
Then two processes, $P,Q$, are alpha-equivalent if $P = Q\{\vec{y}/\vec{x}\}$ for
some $\vec{x} \in \boundnames{Q},\vec{y} \in \boundnames{P}$, where $Q\{\vec{y}/\vec{x}\}$
denotes the capture-avoiding substitution of $\vec{y}$ for $\vec{x}$ in $Q$.
\end{definition}

\begin{definition}
  The {\em structural congruence} \cite{SangiorgiWalker} , $\equiv$,
  between processes is the least congruence containing
  alpha-equivalence, satisfying the abelian monoid laws
  (associativity, commutativity and $\pzero$ as identity) for parallel
  composition $|$ and for summation $+$.
\end{definition}

\subsection{Name equivalence}

We take name equivalence, written $\nameeq$, to be the smallest
equivalence relation generated by the following rules.

\begin{mathpar}
\inferrule*[lab=Quote-drop]
{ }
{ \quotep{@{x}} \nameeq x }

\inferrule*[lab=Struct-equiv]
{ P \scong Q }
{ \quotep{P} \nameeq \quotep{Q} }
\end{mathpar}

The astute reader will have noticed that the mutual recursion of names
and processes imposes a mutual recursion on alpha-equivalence and
structural equivalence via name-equivalence. Fortunately, all of this
works out pleasantly and we may calculate in the natural way, free of
concern. The reader interested in the details is referred to the
appendix \ref{appendix:rho_details}.

\subsection{Substitution}

We use $\Proc$ for the set of processes, $\QProc$ for the set of
names, and $\id{\{}\vec{y} / \vec{x} \id{\}}$ to denote partial maps,
$s : \QProc \rightarrow \QProc$. A map, $s$ lifts, uniquely, to a map
on process terms, $\widehat{s} : \Proc \rightarrow \Proc$ by the
following equations.

\begin{mathpar}
  (0) \psubstp{Q}{P} := 0 \\
  (R \juxtap S) \psubstp{Q}{P}
  :=    
  (R)\psubstp{Q}{P} \juxtap (S) \psubstp{Q}{P} \\
  (x?(y).R) \psubstp{Q}{P}    
  :=    
  (x)\substp{Q}{P} (z)\concat( (R \psubstn{z}{y}) \psubstp{Q}{P} ) \\
  (\lift{x}{R}) \psubstp{Q}{P}  
  :=
  \lift{(x)\substp{Q}{P}}{ R \psubstp{Q}{P} } \\
%   (\dropn{x})  \psubstp{Q}{P}       
%   := 
%   \left\{ 
%     \begin{array}{ccc} 
%       \dropn{\quotep{Q}} & & x \nameeq \quotep{P} \\
%       \dropn{x} & & otherwise \\
%     \end{array}
%   \right. 
  (\dropn{x})  \psubstp{Q}{P}       
  := 
  \left\{ 
    \begin{array}{ccc} 
      Q & & x \nameeq \quotep{P} \\
      \dropn{x} & & otherwise \\
    \end{array}
  \right.
\end{mathpar}
 

where

\begin{eqnarray}
  (x)\id{\{} \lpquote Q \rpquote / \lpquote P \rpquote \id{\}}            = 
  \left\{ 
    \begin{array}{ccc}
      \lpquote Q \rpquote & & x \nameeq \lpquote P \rpquote \\
      x & & otherwise \\
    \end{array}
  \right. \nonumber
\end{eqnarray}

and $z$ is chosen distinct from $\quotep{P}$, $\quotep{Q}$, the free
names in $Q$, and all the names in $R$. Our $\alpha$-equivalence will
be built in the standard way from this substitution.

\begin{remark}\label{rem:no_self_referential_names}
  One consequence of these definitions is that $\forall P. \quotep{P}
  \not\in \freenames{P}$.
\end{remark}

\subsection{ Dynamic quote: an example }

Anticipating something of what's to come, consider applying the
substitution, $\widehat{\id{\{}u / z \id{\}}}$, to the following pair
of processes, $\lift{w}{y!(z)}$ and $w[ \lpquote y!(z) \rpquote ]$.

\begin{eqnarray}
	\lift{w}{y!(z)}\widehat{\id{\{}u / z \id{\}}}
		& = &
		\lift{w}{y!(u)} \nonumber\\
	w[ \lpquote y!(z) \rpquote ] \widehat{ \id{\{}u / z \id{\}} }
		& = &
		w[ \lpquote y!(z) \rpquote ] \nonumber
\end{eqnarray}

Because the body of the process between quotes is impervious to
substitution, we get radically different answers. In fact, by
examining the first process in an input context,
e.g. $x?(z).\lift{w}{y!(z)}$, we see that the process under the lift
operator may be shaped by prefixed inputs binding a name inside it. In
this sense, the lift operator will be seen as a way to dynamically
construct processes before reifying them as names.

Finally equipped with these standard features we can present the
dynamics of the calculus.

\subsubsection{Operational semantics} 

Finally, we introduce the computational dynamics. What marks these
algebras as distinct from other more traditionally studied algebraic
structures, e.g. vector spaces or polynomial rings, is the manner in
which dynamics is captured. In traditional structures, dynamics is typically
expressed through morphisms between such structures, as in linear maps
between vector spaces or morphisms between rings. In algebras
associated with the semantics of computation, the dynamics is
expressed as part of the algebraic structure itself, through a
reduction reduction relation typically denoted by $\red$. Below, we
give a recursive presentation of this relation for the calculus used
in the encoding.

$\red \subseteq \pi \times \pi$
$\red : \pi \to \mathcal{P}(\pi)$

\begin{mathpar}
  \inferrule* [lab=Comm] { \textsf{match}( x_{src}, x_{trgt} ) } { x_{trgt}?(y)P \; | \; x_{src}!\langle {Q} \rangle \red P\{\quotep{Q}/y}\} }
  \and \\
  \inferrule* [lab=Par] {{P} \red {P}'} {{{P} | {Q}} \red {{P}' | {Q}}}
  \and
  \inferrule* [lab=Equiv]{{{P} \scong {P}'} \andalso {{P}' \red {Q}'} \andalso {{Q}' \scong {Q}}}{{P} \red {Q}}
\end{mathpar}

\begin{eqnarray*}
  match_{\equiv} (\quotep{P},\quotep{Q}) & := & P \equiv Q \\
  match_{\dagger}(\quotep{P},\quotep{Q}) & := & \forall R. P|Q \red^{*} R => R \red^{*} 0 \\
  match_{K}(\quotep{P},\quotep{Q}) & := & K \mbox{ for some context } K
\end{eqnarray*}

$u?(x)P | u!\langle Q \rangle \red P\{\quotep{Q}/x\}$

%We write $\wred$ for $\red^*$, and $P\red$ if $\exists Q $ such that $ P \red Q$.
We write $P\red$ if $\exists Q $ such that $ P \red Q$ and $P\not\red$, otherwise.

\section{Replication}

As mentioned before, it is known that replication (and hence
recursion) can be implemented in a higher-order process algebra
\cite{SangiorgiWalker}. As our first example of calculation with the
machinery thus far presented we give the construction explicitly in
the {\rhoc}.

\begin{eqnarray}
	D_{x} & := & \prefix{x}{y}{(\binpar{\outputp{x}{y}}{@{y}})} \nonumber\\
	\bangp_{x}{P} & := & \binpar{{x}!\langle{\binpar{D_{x}}{P}}\rangle}{D_{x}} \nonumber
\end{eqnarray}

\begin{eqnarray}
	\bangp_{x}{P} & & \nonumber\\
	=
	& {x}!\langle{(\prefix{x}{y}{(\outputp{x}{y} | @{y})) | P}}\rangle 
	      | \prefix{x}{y}{(\outputp{x}{y} | @{y})} & \nonumber\\
	\red
	& (\outputp{x}{y} | @{y})\substn{\quotep{(\prefix{x}{y}{(@{y} | \outputp{x}{y})) | P}}}{y} & \nonumber\\
	=
	& \outputp{x}{\quotep{(\prefix{x}{y}{(\outputp{x}{y} | @{y})) | P}}}
	  | {(\prefix{x}{y}{(\outputp{x}{y} | @{y})) | P}} & \nonumber\\
	\red
	& \ldots & \nonumber\\
	\red^*
	& P | P | \ldots & \nonumber
\end{eqnarray}

Of course, this encoding, as an implementation, runs away, unfolding
$\bangp{P}$ eagerly. A lazier and more implementable replication
operator, restricted to input-guarded processes, may be obtained as follows.

\begin{eqnarray}
\bangp{\prefix{u}{v}{P}} 
	:= 
	\binpar{\lift{x}{\prefix{u}{v}{(\binpar{D(x)}{P})}}}{D(x)} \nonumber
\end{eqnarray}

\begin{remark}
  Note that the lazier definition still does not deal with summation
  or mixed summation (i.e. sums over input and output). The reader is
  invited to construct definitions of replication that deal with these
  features. 

  Further, the definitions are parameterized in a name, $x$. Can you,
  gentle reader, make a definition that eliminates this parameter and
  guarantees no accidental interaction between the replication
  machinery and the process being replicated -- i.e. no accidental
  sharing of names used by the process to get its work done and the
  name(s) used by the replication to effect copying. This latter
  revision of the definition of replication is crucial to obtaining
  the expected identity $!!P \sim !P$.
\end{remark}

\begin{remark}\label{rem:paradoxical_combinator}
  The reader familiar with the lambda calculus will have noticed the
  similarity between $D$ and the paradoxical combinator.

  [Ed. note: the existence of this seems to suggest we have to be more
  restrictive on the set of processes and names we admit if we are to
  support no-cloning.]
\end{remark}

\subsubsection{Bisimulation}

The computational dynamics gives rise to another kind of equivalence,
the equivalence of computational behavior. As previously mentioned
this is typically captured \emph{via} some form of bisimulation.

% The notion we use in this paper is weak barbed bisimulation
% \cite{milner91polyadicpi}.

The notion we use in this paper is derived from weak barbed
bisimulation \cite{milner91polyadicpi}. 

\begin{definition}
An \emph{observation relation}, $\downarrow_{\mathcal N}$, over a set
of names, $\mathcal N$, is the smallest relation satisfying the rules
below.

\infrule[Out-barb]{y \in {\mathcal N}, \; x \nameeq y}
		  {\outputp{x}{v} \downarrow_{\mathcal N} x}
\infrule[Par-barb]{\mbox{$P\downarrow_{\mathcal N} x$ or $Q\downarrow_{\mathcal N} x$}}
		  {\binpar{P}{Q} \downarrow_{\mathcal N} x}

We write $P \Downarrow_{\mathcal N} x$ if there is $Q$ such that 
$P \wred Q$ and $Q \downarrow_{\mathcal N} x$.
\end{definition}

\begin{definition}
%\label{def.bbisim}
An  ${\mathcal N}$-\emph{barbed bisimulation} over a set of names, ${\mathcal N}$, is a symmetric binary relation 
${\mathcal S}_{\mathcal N}$ between agents such that $P\rel{S}_{\mathcal N}Q$ implies:
\begin{enumerate}
\item If $P \red P'$ then $Q \wred Q'$ and $P'\rel{S}_{\mathcal N} Q'$.
\item If $P\downarrow_{\mathcal N} x$, then $Q\Downarrow_{\mathcal N} x$.
\end{enumerate}
$P$ is ${\mathcal N}$-barbed bisimilar to $Q$, written
$P \wbbisim_{\mathcal N} Q$, if $P \rel{S}_{\mathcal N} Q$ for some ${\mathcal N}$-barbed bisimulation ${\mathcal S}_{\mathcal N}$.
\end{definition}

$\mathcal{R} \subseteq \pi \times \pi$

$P \mathcal{R} Q => \forall P'. P \red P' \Rightarrow \exists Q'. Q \red Q', P' \mathcal{R} Q'$

$P \vdash x \Rightarrow Q \vdash x$

\begin{mathpar}
  \inferrule*[lab=Out-barb]{x \nameeq y}{{y}!\langle{Q}\rangle \vdash x}
  \and
  \inferrule*[lab=Par-barb]{\mbox{$P\vdash x$ or $Q\vdash x$}}{\binpar{P}{Q} \vdash x}
\end{mathpar}

\subsubsection{Contexts}

One of the principle advantages of computational calculi like the
$\pi$-calculus is a well-defined notion of context,
contextual-equivalence and a correlation between
contextual-equivalence and notions of bisimulation. The notion of
context allows the decomposition of a process into (sub-)process and
its syntactic environment, its context. Thus, a context may be
thought of as a process with a ``hole'' (written $\Box$) in it. The
application of a context $M$ to a process $P$, written $M[P]$, is
tantamount to filling the hole in $M$ with $P$. In this paper we do
not need the full weight of this theory, but do make use of the notion
of context in the proof the main theorem. 

\begin{mathpar}
  \inferrule* [lab=summation] {} {{M_{M},M_{N}} \bc \Box \;|\; x.M_{A} \;|\; M_{M}+M_{N}}
  \and
  \inferrule* [lab=agent] {} {{M_{A}} \bc (\vec{x})M_{P} \;| \; \clift{P_0,\ldots,M_{P},\ldots,P_N}}
  \and \\
  \inferrule* [lab=process] {} {{M_{P}} \bc M_{N} \;| \;P|M_{P} }
\end{mathpar} 

\begin{mathpar}
  \inferrule* [lab=sychronization] {} {M_{N} \bc \Box \;|\; x?M_{F} \;|\; x!M_{C}}
  \and
  \inferrule* [lab=abstraction] {} {{M_{F}} \bc (x)M_{P} }
  \and
  \inferrule* [lab=concretion] {} {{M_{C}} \bc \langle M_{P} \rangle }
  \and \\
  \inferrule* [lab=process] {} {{M_{P}} \bc M_{N} \;| \;P|M_{P} }
\end{mathpar}

\begin{definition}[contextual application] Given a context $M$, and
  process $P$, we define the \emph{contextual application}, $M[P] :=
  M\{P/\Box\}$. That is, the contextual application of M to P is the
  substitution of $P$ for $\Box$ in $M$.
\end{definition}

$\meaningof{-} : L \to \mathcal{P}(\pi)$

\begin{mathpar}
  \inferrule* [lab=collection] {} {\meaningof{true} = \pi, \and \meaningof{~E} = \pi \setminus \meaningof{E}, \and \meaningof{E_{1} \& E_{2}} = \meaningof{E_{1}} \cap \meaningof{E_{2}}}
\end{mathpar}

\begin{mathpar}
  \inferrule* [lab=structure] {} {\meaningof{0} = \{ P \in \pi | P \equiv 0 \}, \and \\ \meaningof{E_1 | E_2} = \{ P \in \pi | P \equiv P_{1} | P_{2}, P_{1} \in \meaningof{E_{1}}, P_{2} \in \meaningof{E_2}\} }
\end{mathpar}

\begin{mathpar}
 \inferrule* [lab=behavior] {} {\meaningof{\langle a?b \rangle E} = \{ P \in \pi | P \equiv Q | u?(y)P', \\ \and \\\\ \and \\ \;\;\; u \in \meaningof{a}, \forall z.P'\{z/y\} \in \meaningof{E\{z/b\}}\}, \and \\ \meaningof{a!E} = \{ P \in \pi | P \equiv Q | x!\langle P' \rangle, x \in \meaningof{a} P' \in \meaningof{E}\} }
\end{mathpar}

\begin{mathpar}
 \inferrule* [lab=nominal] {} {\meaningof{\quotep{E}} = \{ \quotep{P} \in \quotep{\pi} | P \in \meaningof{E} \}, \and \meaningof{\quotep{P}} = \{ \quotep{Q} \in \quotep{\pi} | P \equiv Q \} \and \\ \meaningof{@\quotep{E}} = \{ P \in \pi | P \equiv @x, x \in \meaningof{E} \}}
\end{mathpar}

\begin{eqnarray*}
  \\
  \meaningof{-} : TS \to ST
\end{eqnarray*}

\begin{eqnarray*}
  \\
  L : TS \to ST
\end{eqnarray*}

\begin{eqnarray*}
  \\
  P \models E \iff P \in \meaningof{E}
\end{eqnarray*}

\begin{eqnarray*}
  P \approx_{L} Q \iff \forall E \in L. P \models E \iff Q \models E
\end{eqnarray*}

\begin{eqnarray*}
  P \approx_{K} Q
\end{eqnarray*}

\begin{eqnarray*}
  P \approx Q
\end{eqnarray*}

$\approx_{K} = \approx = \approx_{L}$

\subsubsection{Contextual duality}

Note that contexts extend the quotation operation to a family of
operations from processes to names. Given a context, $M$, we can
define a \emph{nominal context}, $\quotep{M}$ by $\quotep{M}[P] :=
\quotep{M[P]}$. To foreshadow what is to come we observe that these
operations enjoy a duality with processes very much like the duality
between vectors and maps from vectors to scalars.

Further, because the calculus is essentially higher-order, we have a
correspondence between contexts and processes. More specifically,
given a name $x$ and a context $M$ we can construct $M^{*}_{x}$ such
that 

\begin{mathpar}
  M^{*}_{x} | \lift{x}{P} \red M[P]
\end{mathpar}

namely,

\begin{mathpar}
  M^{*}_{x} := x?(u).M[\dropn{u}]
\end{mathpar}

The dependence of $M^{*}_{x}$ on a name makes it an abstraction, 

\begin{mathpar}
  M^{*} := (x)x?(u).M[\dropn{u}]
\end{mathpar}

\subsection{Additional notation}

It will sometimes be convenient to denote the process a name
quotes. We already have the notation $x = \quotep{P}$, but it will be
convenient to introduce an alternate notation, $\procn{x}$, when we
want to emphasize the connection to the use of the name. Note that, by
virtue of name equivalence, $\quotep{\procn{x}} \nameeq x$; so, the
notation is consistent with previous definitions.

Further, because names have structure it is possible to effect
substitutions on the basis of that structure. This means we need to
upgrade our notation for substitutions, which we accomplish by
adapting comprehension notation. Thus,

\begin{mathpar}
  P\{ y / x : x \in S \}
\end{mathpar}

is interpreted to mean the process derived from P by replacing (in a
capture-avoiding manner) each occurrence of $x$ in $S$ by $y$. For example,

\begin{mathpar}
  P\{ \quotep{\procn{x}|\procn{x}} / x : x \in \freenames{P} \}
\end{mathpar}

will replace each (occurrence) of a free name $x$ in $P$ by
$\quotep{\procn{x}|\procn{x}}$.

Also, we will avail ourselves of the notation $x^{L}$ and $x^{R}$ to
denote injections of a name into disjoint copies of the name
space. There are numerous ways to accomplish this. One example can be
found in \cite{MeredithR05}. This notation overloads to vectors of
names: $\vec{x}^{\pi} := (x_{i}^{\pi} \; : \; 0 \leq i < |\vec{x}| )$ where $\pi \in \{L,R\}$.

We also use $P^{\Box} := P|\Box$.

In \cite{MeredithR05} an interpretation of the new operator is
given. It turns out that there are several possible interpretations
all enjoying the requisite algebraic properties of the operator (see
\cite{milner91polyadicpi}). We will therefore make liberal use of
$(\nu\; \vec{x})P$.

% subsection the_syntax_and_semantics_of_the_notation_system (end)   

\input{qm2pi.qmops} 

\input{qm2pi.sterngerlach} 

\input{qm2pi.metric} 

% section concurrent_process_calculi (end)

%\input{qm2pi.proofsketch}

% section proof sketch (end)

%\input{qm2pi.slviaknots} 

% section spatial logic via knots (end)

\input{qm2pi.conclusion}

% section conclusion (end)

%\input{qm2pi.dtcodes} 

% section wiring algorithm (end)

\input{qm2pi.ack} 

% section acknowledgments (end)

\newpage


\bibliographystyle{plain}   
\bibliography{../../biblios/main.bib}

\input{qm2pi.rhodetails}

\end{document}

 

%\ifpdf
%\usepackage[pdftex]{graphicx}
%\else
%\usepackage{graphicx}
%\fi

 % \ifpdf
%  \usepackage{pdfsync}
%  \if


%\title{Brief Article}
%\author{David F. Snyder}
%\author{L.G. Meredith}

%\address{Dept. of Math., Texas State University--San Marcos, San Marcos, TX 78666}
       
\pagestyle{empty}


\begin{document}

\lstset{language=[Objective]Caml,frame=shadowbox}

\documentclass[12pt]{llncs}
%\documentclass{jktr}

\usepackage[pdftex]{hyperref}                   
\usepackage {listings}
\usepackage {mathpartir}
\usepackage{bcprules}
%\usepackage{listings}
                       
\usepackage{graphicx} 
%\usepackage[margins=2.5cm,nohead,nofoot]{geometry}
%\usepackage{geometry}
\usepackage{amsfonts}
\usepackage{amstext}
\usepackage{latexsym}
\usepackage{amssymb}
\usepackage{color}


%\include{myPreamble}
\include{qm2pi.local} 

%\ifpdf
%\usepackage[pdftex]{graphicx}
%\else
%\usepackage{graphicx}
%\fi

 % \ifpdf
%  \usepackage{pdfsync}
%  \if


%\title{Brief Article}
%\author{David F. Snyder}
%\author{L.G. Meredith}

%\address{Dept. of Math., Texas State University--San Marcos, San Marcos, TX 78666}
       
\pagestyle{empty}


\begin{document}

\lstset{language=[Objective]Caml,frame=shadowbox}

\input{qm2pi.front}

% section front matter (end)

\input{qm2pi.intro} 
 
% section introduction (end)

% \input{qm2pi.knotations} 

% section notation (end)

\input{qm2pi.process.calculi} 

% section concurrent_process_calculi_and_spatial_logics_ (end)
    
%\input{qm2pi.knots2pi} 

%\input{qm2pi.trefoil} 

%\input{qm2pi.mainthm} 

% subsection basic_interpretation (end)

%\input{qm2pi.rho.presentation} 
\subsection{The syntax and semantics of the notation system}\label{sub:the_syntax_and_semantics_of_the_notation_system} % (fold)

We now summarize a technical presentation of the calculus that
embodies our theory of dynamics. The typical presentation of such a
calculus follows the style of giving generators and relations on
them. The grammar, below, describing term constructors, freely
generates the set of processes, $\Proc$. This set is then quotiented
by a relation known as structural congruence and it is over this set
that the notion of dynamics is expressed. This presentation is
essentially that of \cite{MeredithR05} with the addition of
polyadicity and summation. For readability we have relegated some of
the technical subtleties to an appendix.

\subsubsection{Process grammar}\label{subsub:process_grammar}

\begin{mathpar}
  \inferrule* [lab=synchronization] {} {{M} \bc \pzero \;|\; x?F \;|\; x!C }
  \and
  \inferrule* [lab=abstraction] {} {{F} \bc (x)P}
  \and
  \inferrule* [lab=concretion] {} {{C} \bc \langle Q \rangle}
  \and
  \inferrule* [lab=process] {} {{P,Q} \bc M \;| \;P|Q \;|\; @{x}}
  \and
  \inferrule* [lab=name] {} {{x} \bc \quotep{P}}
\end{mathpar} 

Note that $\vec{x}$ (resp. $\vec{P}$) denotes a vector of names
(resp. processes) of length $|\vec{x}|$ (resp. $|\vec{P}|$). We adopt
the following useful abbreviations.

\begin{mathpar}
   x?(\vec{y}).P := x.(\vec{y})P \and  x\clift{\vec{P}} := x.\clift{\vec{P}}
   \and x!(y) := \lift{x}{\dropn{y}}
   \and \Pi_{i=0}^{n-1}P_i := P_0 | \ldots | P_{n-1}
\end{mathpar}

\subsubsection{Structural congruence}

\paragraph{Free and bound names and alpha-equivalence.} At the
core of structural equivalence is alpha-equivalence which identifies
process that are the same up to a change of variable. Formally, we
recognize the distinction between free and bound names. The free names
of a process, $\freenames{P}$, may be calculated recursively as
follows:

\begin{mathpar}
\freenames{\pzero} := \emptyset
  \and \\
  \freenames{x?(y).P} := \{ x \} \cup (\freenames{P} \setminus \{ y \})
  \and 
  \freenames{x!\langle P \rangle} := \{ x \} \cup \{ P \} 
  \and \\
  \freenames{P|Q} := \freenames{P} \cup \freenames{Q}
  \and \\
  \freenames{@{x}} := \{ x \}
\end{mathpar}

$\pi$
$\quotep{\pi}$

$\freenames{-} : \pi \to \mathcal{P}(\quotep{\pi})$

\begin{eqnarray*}
  \freenames{\pzero} & := & \emptyset \\
  \freenames{x?(y).P} & := & \{ x \} \cup (\freenames{P} \setminus \{ y \}) \\
  \freenames{x!\langle P \rangle} & := & \{ x \} \cup \{ P \} \\
  \freenames{P|Q} & := & \freenames{P} \cup \freenames{Q} \\
  \freenames{\dropn{x}} & := & \{ x \}
\end{eqnarray*}

The bound names of a process, $\boundnames{P}$, are those names occurring in $P$
that are not free. For example, in $x?(y).0$, the name $x$ is free, while $y$ is bound.

\begin{mathpar}
  \inferrule* [lab=monoidal-laws] {} { P|Q \equiv Q|P \and P|0 \equiv P \and P|(Q|R) \equiv (P|Q)|R }
\end{mathpar}

\begin{mathpar}
  \inferrule* [lab=alpha-equivalence] {} { (x)P \equiv (y)P\{y/x\} \and y \not\in \freenames{P} }
\end{mathpar}

\begin{definition}
Then two processes, $P,Q$, are alpha-equivalent if $P = Q\{\vec{y}/\vec{x}\}$ for
some $\vec{x} \in \boundnames{Q},\vec{y} \in \boundnames{P}$, where $Q\{\vec{y}/\vec{x}\}$
denotes the capture-avoiding substitution of $\vec{y}$ for $\vec{x}$ in $Q$.
\end{definition}

\begin{definition}
  The {\em structural congruence} \cite{SangiorgiWalker} , $\equiv$,
  between processes is the least congruence containing
  alpha-equivalence, satisfying the abelian monoid laws
  (associativity, commutativity and $\pzero$ as identity) for parallel
  composition $|$ and for summation $+$.
\end{definition}

\subsection{Name equivalence}

We take name equivalence, written $\nameeq$, to be the smallest
equivalence relation generated by the following rules.

\begin{mathpar}
\inferrule*[lab=Quote-drop]
{ }
{ \quotep{@{x}} \nameeq x }

\inferrule*[lab=Struct-equiv]
{ P \scong Q }
{ \quotep{P} \nameeq \quotep{Q} }
\end{mathpar}

The astute reader will have noticed that the mutual recursion of names
and processes imposes a mutual recursion on alpha-equivalence and
structural equivalence via name-equivalence. Fortunately, all of this
works out pleasantly and we may calculate in the natural way, free of
concern. The reader interested in the details is referred to the
appendix \ref{appendix:rho_details}.

\subsection{Substitution}

We use $\Proc$ for the set of processes, $\QProc$ for the set of
names, and $\id{\{}\vec{y} / \vec{x} \id{\}}$ to denote partial maps,
$s : \QProc \rightarrow \QProc$. A map, $s$ lifts, uniquely, to a map
on process terms, $\widehat{s} : \Proc \rightarrow \Proc$ by the
following equations.

\begin{mathpar}
  (0) \psubstp{Q}{P} := 0 \\
  (R \juxtap S) \psubstp{Q}{P}
  :=    
  (R)\psubstp{Q}{P} \juxtap (S) \psubstp{Q}{P} \\
  (x?(y).R) \psubstp{Q}{P}    
  :=    
  (x)\substp{Q}{P} (z)\concat( (R \psubstn{z}{y}) \psubstp{Q}{P} ) \\
  (\lift{x}{R}) \psubstp{Q}{P}  
  :=
  \lift{(x)\substp{Q}{P}}{ R \psubstp{Q}{P} } \\
%   (\dropn{x})  \psubstp{Q}{P}       
%   := 
%   \left\{ 
%     \begin{array}{ccc} 
%       \dropn{\quotep{Q}} & & x \nameeq \quotep{P} \\
%       \dropn{x} & & otherwise \\
%     \end{array}
%   \right. 
  (\dropn{x})  \psubstp{Q}{P}       
  := 
  \left\{ 
    \begin{array}{ccc} 
      Q & & x \nameeq \quotep{P} \\
      \dropn{x} & & otherwise \\
    \end{array}
  \right.
\end{mathpar}
 

where

\begin{eqnarray}
  (x)\id{\{} \lpquote Q \rpquote / \lpquote P \rpquote \id{\}}            = 
  \left\{ 
    \begin{array}{ccc}
      \lpquote Q \rpquote & & x \nameeq \lpquote P \rpquote \\
      x & & otherwise \\
    \end{array}
  \right. \nonumber
\end{eqnarray}

and $z$ is chosen distinct from $\quotep{P}$, $\quotep{Q}$, the free
names in $Q$, and all the names in $R$. Our $\alpha$-equivalence will
be built in the standard way from this substitution.

\begin{remark}\label{rem:no_self_referential_names}
  One consequence of these definitions is that $\forall P. \quotep{P}
  \not\in \freenames{P}$.
\end{remark}

\subsection{ Dynamic quote: an example }

Anticipating something of what's to come, consider applying the
substitution, $\widehat{\id{\{}u / z \id{\}}}$, to the following pair
of processes, $\lift{w}{y!(z)}$ and $w[ \lpquote y!(z) \rpquote ]$.

\begin{eqnarray}
	\lift{w}{y!(z)}\widehat{\id{\{}u / z \id{\}}}
		& = &
		\lift{w}{y!(u)} \nonumber\\
	w[ \lpquote y!(z) \rpquote ] \widehat{ \id{\{}u / z \id{\}} }
		& = &
		w[ \lpquote y!(z) \rpquote ] \nonumber
\end{eqnarray}

Because the body of the process between quotes is impervious to
substitution, we get radically different answers. In fact, by
examining the first process in an input context,
e.g. $x?(z).\lift{w}{y!(z)}$, we see that the process under the lift
operator may be shaped by prefixed inputs binding a name inside it. In
this sense, the lift operator will be seen as a way to dynamically
construct processes before reifying them as names.

Finally equipped with these standard features we can present the
dynamics of the calculus.

\subsubsection{Operational semantics} 

Finally, we introduce the computational dynamics. What marks these
algebras as distinct from other more traditionally studied algebraic
structures, e.g. vector spaces or polynomial rings, is the manner in
which dynamics is captured. In traditional structures, dynamics is typically
expressed through morphisms between such structures, as in linear maps
between vector spaces or morphisms between rings. In algebras
associated with the semantics of computation, the dynamics is
expressed as part of the algebraic structure itself, through a
reduction reduction relation typically denoted by $\red$. Below, we
give a recursive presentation of this relation for the calculus used
in the encoding.

$\red \subseteq \pi \times \pi$
$\red : \pi \to \mathcal{P}(\pi)$

\begin{mathpar}
  \inferrule* [lab=Comm] { \textsf{match}( x_{src}, x_{trgt} ) } { x_{trgt}?(y)P \; | \; x_{src}!\langle {Q} \rangle \red P\{\quotep{Q}/y}\} }
  \and \\
  \inferrule* [lab=Par] {{P} \red {P}'} {{{P} | {Q}} \red {{P}' | {Q}}}
  \and
  \inferrule* [lab=Equiv]{{{P} \scong {P}'} \andalso {{P}' \red {Q}'} \andalso {{Q}' \scong {Q}}}{{P} \red {Q}}
\end{mathpar}

\begin{eqnarray*}
  match_{\equiv} (\quotep{P},\quotep{Q}) & := & P \equiv Q \\
  match_{\dagger}(\quotep{P},\quotep{Q}) & := & \forall R. P|Q \red^{*} R => R \red^{*} 0 \\
  match_{K}(\quotep{P},\quotep{Q}) & := & K \mbox{ for some context } K
\end{eqnarray*}

$u?(x)P | u!\langle Q \rangle \red P\{\quotep{Q}/x\}$

%We write $\wred$ for $\red^*$, and $P\red$ if $\exists Q $ such that $ P \red Q$.
We write $P\red$ if $\exists Q $ such that $ P \red Q$ and $P\not\red$, otherwise.

\section{Replication}

As mentioned before, it is known that replication (and hence
recursion) can be implemented in a higher-order process algebra
\cite{SangiorgiWalker}. As our first example of calculation with the
machinery thus far presented we give the construction explicitly in
the {\rhoc}.

\begin{eqnarray}
	D_{x} & := & \prefix{x}{y}{(\binpar{\outputp{x}{y}}{@{y}})} \nonumber\\
	\bangp_{x}{P} & := & \binpar{{x}!\langle{\binpar{D_{x}}{P}}\rangle}{D_{x}} \nonumber
\end{eqnarray}

\begin{eqnarray}
	\bangp_{x}{P} & & \nonumber\\
	=
	& {x}!\langle{(\prefix{x}{y}{(\outputp{x}{y} | @{y})) | P}}\rangle 
	      | \prefix{x}{y}{(\outputp{x}{y} | @{y})} & \nonumber\\
	\red
	& (\outputp{x}{y} | @{y})\substn{\quotep{(\prefix{x}{y}{(@{y} | \outputp{x}{y})) | P}}}{y} & \nonumber\\
	=
	& \outputp{x}{\quotep{(\prefix{x}{y}{(\outputp{x}{y} | @{y})) | P}}}
	  | {(\prefix{x}{y}{(\outputp{x}{y} | @{y})) | P}} & \nonumber\\
	\red
	& \ldots & \nonumber\\
	\red^*
	& P | P | \ldots & \nonumber
\end{eqnarray}

Of course, this encoding, as an implementation, runs away, unfolding
$\bangp{P}$ eagerly. A lazier and more implementable replication
operator, restricted to input-guarded processes, may be obtained as follows.

\begin{eqnarray}
\bangp{\prefix{u}{v}{P}} 
	:= 
	\binpar{\lift{x}{\prefix{u}{v}{(\binpar{D(x)}{P})}}}{D(x)} \nonumber
\end{eqnarray}

\begin{remark}
  Note that the lazier definition still does not deal with summation
  or mixed summation (i.e. sums over input and output). The reader is
  invited to construct definitions of replication that deal with these
  features. 

  Further, the definitions are parameterized in a name, $x$. Can you,
  gentle reader, make a definition that eliminates this parameter and
  guarantees no accidental interaction between the replication
  machinery and the process being replicated -- i.e. no accidental
  sharing of names used by the process to get its work done and the
  name(s) used by the replication to effect copying. This latter
  revision of the definition of replication is crucial to obtaining
  the expected identity $!!P \sim !P$.
\end{remark}

\begin{remark}\label{rem:paradoxical_combinator}
  The reader familiar with the lambda calculus will have noticed the
  similarity between $D$ and the paradoxical combinator.

  [Ed. note: the existence of this seems to suggest we have to be more
  restrictive on the set of processes and names we admit if we are to
  support no-cloning.]
\end{remark}

\subsubsection{Bisimulation}

The computational dynamics gives rise to another kind of equivalence,
the equivalence of computational behavior. As previously mentioned
this is typically captured \emph{via} some form of bisimulation.

% The notion we use in this paper is weak barbed bisimulation
% \cite{milner91polyadicpi}.

The notion we use in this paper is derived from weak barbed
bisimulation \cite{milner91polyadicpi}. 

\begin{definition}
An \emph{observation relation}, $\downarrow_{\mathcal N}$, over a set
of names, $\mathcal N$, is the smallest relation satisfying the rules
below.

\infrule[Out-barb]{y \in {\mathcal N}, \; x \nameeq y}
		  {\outputp{x}{v} \downarrow_{\mathcal N} x}
\infrule[Par-barb]{\mbox{$P\downarrow_{\mathcal N} x$ or $Q\downarrow_{\mathcal N} x$}}
		  {\binpar{P}{Q} \downarrow_{\mathcal N} x}

We write $P \Downarrow_{\mathcal N} x$ if there is $Q$ such that 
$P \wred Q$ and $Q \downarrow_{\mathcal N} x$.
\end{definition}

\begin{definition}
%\label{def.bbisim}
An  ${\mathcal N}$-\emph{barbed bisimulation} over a set of names, ${\mathcal N}$, is a symmetric binary relation 
${\mathcal S}_{\mathcal N}$ between agents such that $P\rel{S}_{\mathcal N}Q$ implies:
\begin{enumerate}
\item If $P \red P'$ then $Q \wred Q'$ and $P'\rel{S}_{\mathcal N} Q'$.
\item If $P\downarrow_{\mathcal N} x$, then $Q\Downarrow_{\mathcal N} x$.
\end{enumerate}
$P$ is ${\mathcal N}$-barbed bisimilar to $Q$, written
$P \wbbisim_{\mathcal N} Q$, if $P \rel{S}_{\mathcal N} Q$ for some ${\mathcal N}$-barbed bisimulation ${\mathcal S}_{\mathcal N}$.
\end{definition}

$\mathcal{R} \subseteq \pi \times \pi$

$P \mathcal{R} Q => \forall P'. P \red P' \Rightarrow \exists Q'. Q \red Q', P' \mathcal{R} Q'$

$P \vdash x \Rightarrow Q \vdash x$

\begin{mathpar}
  \inferrule*[lab=Out-barb]{x \nameeq y}{{y}!\langle{Q}\rangle \vdash x}
  \and
  \inferrule*[lab=Par-barb]{\mbox{$P\vdash x$ or $Q\vdash x$}}{\binpar{P}{Q} \vdash x}
\end{mathpar}

\subsubsection{Contexts}

One of the principle advantages of computational calculi like the
$\pi$-calculus is a well-defined notion of context,
contextual-equivalence and a correlation between
contextual-equivalence and notions of bisimulation. The notion of
context allows the decomposition of a process into (sub-)process and
its syntactic environment, its context. Thus, a context may be
thought of as a process with a ``hole'' (written $\Box$) in it. The
application of a context $M$ to a process $P$, written $M[P]$, is
tantamount to filling the hole in $M$ with $P$. In this paper we do
not need the full weight of this theory, but do make use of the notion
of context in the proof the main theorem. 

\begin{mathpar}
  \inferrule* [lab=summation] {} {{M_{M},M_{N}} \bc \Box \;|\; x.M_{A} \;|\; M_{M}+M_{N}}
  \and
  \inferrule* [lab=agent] {} {{M_{A}} \bc (\vec{x})M_{P} \;| \; \clift{P_0,\ldots,M_{P},\ldots,P_N}}
  \and \\
  \inferrule* [lab=process] {} {{M_{P}} \bc M_{N} \;| \;P|M_{P} }
\end{mathpar} 

\begin{mathpar}
  \inferrule* [lab=sychronization] {} {M_{N} \bc \Box \;|\; x?M_{F} \;|\; x!M_{C}}
  \and
  \inferrule* [lab=abstraction] {} {{M_{F}} \bc (x)M_{P} }
  \and
  \inferrule* [lab=concretion] {} {{M_{C}} \bc \langle M_{P} \rangle }
  \and \\
  \inferrule* [lab=process] {} {{M_{P}} \bc M_{N} \;| \;P|M_{P} }
\end{mathpar}

\begin{definition}[contextual application] Given a context $M$, and
  process $P$, we define the \emph{contextual application}, $M[P] :=
  M\{P/\Box\}$. That is, the contextual application of M to P is the
  substitution of $P$ for $\Box$ in $M$.
\end{definition}

$\meaningof{-} : L \to \mathcal{P}(\pi)$

\begin{mathpar}
  \inferrule* [lab=collection] {} {\meaningof{true} = \pi, \and \meaningof{~E} = \pi \setminus \meaningof{E}, \and \meaningof{E_{1} \& E_{2}} = \meaningof{E_{1}} \cap \meaningof{E_{2}}}
\end{mathpar}

\begin{mathpar}
  \inferrule* [lab=structure] {} {\meaningof{0} = \{ P \in \pi | P \equiv 0 \}, \and \\ \meaningof{E_1 | E_2} = \{ P \in \pi | P \equiv P_{1} | P_{2}, P_{1} \in \meaningof{E_{1}}, P_{2} \in \meaningof{E_2}\} }
\end{mathpar}

\begin{mathpar}
 \inferrule* [lab=behavior] {} {\meaningof{\langle a?b \rangle E} = \{ P \in \pi | P \equiv Q | u?(y)P', \\ \and \\\\ \and \\ \;\;\; u \in \meaningof{a}, \forall z.P'\{z/y\} \in \meaningof{E\{z/b\}}\}, \and \\ \meaningof{a!E} = \{ P \in \pi | P \equiv Q | x!\langle P' \rangle, x \in \meaningof{a} P' \in \meaningof{E}\} }
\end{mathpar}

\begin{mathpar}
 \inferrule* [lab=nominal] {} {\meaningof{\quotep{E}} = \{ \quotep{P} \in \quotep{\pi} | P \in \meaningof{E} \}, \and \meaningof{\quotep{P}} = \{ \quotep{Q} \in \quotep{\pi} | P \equiv Q \} \and \\ \meaningof{@\quotep{E}} = \{ P \in \pi | P \equiv @x, x \in \meaningof{E} \}}
\end{mathpar}

\begin{eqnarray*}
  \\
  \meaningof{-} : TS \to ST
\end{eqnarray*}

\begin{eqnarray*}
  \\
  L : TS \to ST
\end{eqnarray*}

\begin{eqnarray*}
  \\
  P \models E \iff P \in \meaningof{E}
\end{eqnarray*}

\begin{eqnarray*}
  P \approx_{L} Q \iff \forall E \in L. P \models E \iff Q \models E
\end{eqnarray*}

\begin{eqnarray*}
  P \approx_{K} Q
\end{eqnarray*}

\begin{eqnarray*}
  P \approx Q
\end{eqnarray*}

$\approx_{K} = \approx = \approx_{L}$

\subsubsection{Contextual duality}

Note that contexts extend the quotation operation to a family of
operations from processes to names. Given a context, $M$, we can
define a \emph{nominal context}, $\quotep{M}$ by $\quotep{M}[P] :=
\quotep{M[P]}$. To foreshadow what is to come we observe that these
operations enjoy a duality with processes very much like the duality
between vectors and maps from vectors to scalars.

Further, because the calculus is essentially higher-order, we have a
correspondence between contexts and processes. More specifically,
given a name $x$ and a context $M$ we can construct $M^{*}_{x}$ such
that 

\begin{mathpar}
  M^{*}_{x} | \lift{x}{P} \red M[P]
\end{mathpar}

namely,

\begin{mathpar}
  M^{*}_{x} := x?(u).M[\dropn{u}]
\end{mathpar}

The dependence of $M^{*}_{x}$ on a name makes it an abstraction, 

\begin{mathpar}
  M^{*} := (x)x?(u).M[\dropn{u}]
\end{mathpar}

\subsection{Additional notation}

It will sometimes be convenient to denote the process a name
quotes. We already have the notation $x = \quotep{P}$, but it will be
convenient to introduce an alternate notation, $\procn{x}$, when we
want to emphasize the connection to the use of the name. Note that, by
virtue of name equivalence, $\quotep{\procn{x}} \nameeq x$; so, the
notation is consistent with previous definitions.

Further, because names have structure it is possible to effect
substitutions on the basis of that structure. This means we need to
upgrade our notation for substitutions, which we accomplish by
adapting comprehension notation. Thus,

\begin{mathpar}
  P\{ y / x : x \in S \}
\end{mathpar}

is interpreted to mean the process derived from P by replacing (in a
capture-avoiding manner) each occurrence of $x$ in $S$ by $y$. For example,

\begin{mathpar}
  P\{ \quotep{\procn{x}|\procn{x}} / x : x \in \freenames{P} \}
\end{mathpar}

will replace each (occurrence) of a free name $x$ in $P$ by
$\quotep{\procn{x}|\procn{x}}$.

Also, we will avail ourselves of the notation $x^{L}$ and $x^{R}$ to
denote injections of a name into disjoint copies of the name
space. There are numerous ways to accomplish this. One example can be
found in \cite{MeredithR05}. This notation overloads to vectors of
names: $\vec{x}^{\pi} := (x_{i}^{\pi} \; : \; 0 \leq i < |\vec{x}| )$ where $\pi \in \{L,R\}$.

We also use $P^{\Box} := P|\Box$.

In \cite{MeredithR05} an interpretation of the new operator is
given. It turns out that there are several possible interpretations
all enjoying the requisite algebraic properties of the operator (see
\cite{milner91polyadicpi}). We will therefore make liberal use of
$(\nu\; \vec{x})P$.

% subsection the_syntax_and_semantics_of_the_notation_system (end)   

\input{qm2pi.qmops} 

\input{qm2pi.sterngerlach} 

\input{qm2pi.metric} 

% section concurrent_process_calculi (end)

%\input{qm2pi.proofsketch}

% section proof sketch (end)

%\input{qm2pi.slviaknots} 

% section spatial logic via knots (end)

\input{qm2pi.conclusion}

% section conclusion (end)

%\input{qm2pi.dtcodes} 

% section wiring algorithm (end)

\input{qm2pi.ack} 

% section acknowledgments (end)

\newpage


\bibliographystyle{plain}   
\bibliography{../../biblios/main.bib}

\input{qm2pi.rhodetails}

\end{document}



% section front matter (end)

\section{Introduction}\label{sec:introduction} % (fold)
In this draft of the material i am going to have to dispense with the
usual writing conventions adopted in papers on these topics. i'm going
to have adopt whatever tone i need at the time i'm writing up the
calculations. Sometimes this may be very conversational; others it may
be the barest mathematical grunts; others still it may be that i have
lifted text from one of my other papers because the exposition of some
point was better said there. i hope that my readers are not unduly put
out by this decision. i'm not doing this to flout convention or be
rebellious. i find these calculations very technically challenging. To
keep everything going technically, something has to give; i have to
let go of some cognitive burden. So, the academic writing style --
with all of its trade-offs in terms of facilitating technical
communication -- is what i'm letting go of. Perhaps subsequent drafts
can be tightened and polished, but for now, i'm going to speak as if
we were sitting together in a coffee shop with a laptop, wifi and a
pad of paper and a pencil.

So, here's what i have to say. We -- you and i, comfortably ensconced
in our coffee shop and well-equipped with our tools -- can realize and
carry out the calculations of quantum mechanics over a very different
formal theory of dynamics, a formal theory of dynamics that
corresponds to a theory of concurrent computation with
\emph{reflection}. It has the advantage that the underlying theory is
already `quantized', but supports analogues all of the continuuous
operations. Strikingly, this underlying theory has recently been
connected with a notion of metric that we can show, by calculating
together, coincides with the metric induced by the inner product.

There are a lot of reasons why you might be interested in seeing
calculations of this form. Here's why i'm interested. For the past
several centuries there has been no competitor to the ``Newtonian''
account of dynamics. As a result the predominant share of accounts of
dynamical systems and situations have had to be formulated in terms of
the Newtonian machinery. i view this as an intellectually dangerous
position to occupy. Everything, despite it's intrinsic shape, turns
into a nail to be hit with this hammer. Recently, however, the theory
of computation has matured to the point where we have candidates for
theories of dynamics that offer very different perspective on
reasoning about dynamical systems and situations. Testing these
candidates against very successful accounts of dynamical situations,
like quantum mechanics, is going to give us some sense of how mature
they are and some measure of the quality of these accounts of
dynamics.

\subsection{Summary of contributions and outline of paper}

So, we're going to develop an interpretation of the operations of
quantum mechanics normally interpreted by Hilbert spaces and
operators. We're going to do this over a theory of computation. Note
that this is very different than the usual quantum computation program
which develops notions of computation over quantum mechanics. Rather,
we are developing a story that aligns with Wheeler's slogan: It from
Bit. To do this we will first provide an account of the theory of
computation at play here. Then we will dive into a calculation-driven
interpretation of the operations of quantum mechanics.

The reason we take this approach is that -- until very recently --
there hasn't been an axiomatic account of quantum mechanics. As a
result there has been no sharp delineation of the mathematical theory
supporting interpretation of the physical theory and the physical
theory, itself. So, ambient features of the maths are free to be
exploited (or supressed) without a real accounting of their physical
relevance. There is no sharp statement ``here's the physical theory''
qua \emph{theory} and ``here's the mathematical interpretation''
enabling a judgment of how faithful the interpretation is -- apart
from experimental observation. When there is an axiomatic account we
can judge how well a given mathematical formalism supports an
interpretation of the axioms, independent of
experimentation. Likewise, we can judge how well we have captured our
physical evidence and experience with our axiomatics, independent of
any specific mathematical implementation, with accidental detail that
may or may not have physical significance. 

In lieu of a fully fleshed out and vetted axiomatic account of quantum
mechanics, interpreting the operational notions in service of modeling
physical systems will have to suffice. In other words, we are not in
the business of providing a model of Hilbert spaces and operators. We
are in the business of providing a model of quantum mechanics because
we are motivated by testing our notions of dynamics against physical
theory; and, the predictive calculations of the physical theory must
serve as the best formulation -- shy of a fully fleshed out axiomatic
account -- of the physical theory itself (as they have for scientific
theories since time immemorial). Put another way, despite a
whole-hearted commitment to an It-from-Bit ontology, we are firmly
aligned with the shut-up-and-calculate camp as the best way to obtain
results either from the physical perspective or as a quality assurance
measure of our fledgling theory of dynamics.

In detail, we present a reflective process calculus. Then we develop
intuitive correspondences between the notions available in this
calculus and the usual physical notions supporting quantum mechanical
calculations. Thus, 

\begin{table}[htp]
  \center{
    \fbox{
      \begin{tabular}{c|c}
        quantum mechanics & process calculus \\
        \hline
        scalar & name \\
        state vector & process \\
        dual & contextual duals \\
        matrix & formal sums of process-context-dual pairs \\
        orthogonality & process annihilation \\
        inner product & execution-formula + quoting
      \end{tabular}
    }
  }
  \caption{QM - process calculi correspondences}
\end{table}

Then we tighten up these intuitions to operational definitions. We
employ the Dirac notation as the best proxy we can find for an
abstract syntax of the quantum mechanical notions. The definitions we
develop put us in contact with equational constraints coming from the
theory that we demonstrate the definitions and calculations satisfy.

This puts us in a position to shut up and calculate for the
Stern-Gerlach experimental set up, showing how these predictive
calculations become calculations on processes in our theory of a
reflective process calculus.

Penultimately, we demonstrate that the notion of metric coming from
the inner product coincides with the notion of metric available from
the theory of bisimulation. This demonstration gives us the right to
think of space as arising from behavior. Finally, we consider where we
might go from the new vantage point we have obtained.

% section introduction (end) 
 
% section introduction (end)

% \documentclass[12pt]{llncs}
%\documentclass{jktr}

\usepackage[pdftex]{hyperref}                   
\usepackage {listings}
\usepackage {mathpartir}
\usepackage{bcprules}
%\usepackage{listings}
                       
\usepackage{graphicx} 
%\usepackage[margins=2.5cm,nohead,nofoot]{geometry}
%\usepackage{geometry}
\usepackage{amsfonts}
\usepackage{amstext}
\usepackage{latexsym}
\usepackage{amssymb}
\usepackage{color}


%\include{myPreamble}
\include{qm2pi.local} 

%\ifpdf
%\usepackage[pdftex]{graphicx}
%\else
%\usepackage{graphicx}
%\fi

 % \ifpdf
%  \usepackage{pdfsync}
%  \if


%\title{Brief Article}
%\author{David F. Snyder}
%\author{L.G. Meredith}

%\address{Dept. of Math., Texas State University--San Marcos, San Marcos, TX 78666}
       
\pagestyle{empty}


\begin{document}

\lstset{language=[Objective]Caml,frame=shadowbox}

\input{qm2pi.front}

% section front matter (end)

\input{qm2pi.intro} 
 
% section introduction (end)

% \input{qm2pi.knotations} 

% section notation (end)

\input{qm2pi.process.calculi} 

% section concurrent_process_calculi_and_spatial_logics_ (end)
    
%\input{qm2pi.knots2pi} 

%\input{qm2pi.trefoil} 

%\input{qm2pi.mainthm} 

% subsection basic_interpretation (end)

%\input{qm2pi.rho.presentation} 
\subsection{The syntax and semantics of the notation system}\label{sub:the_syntax_and_semantics_of_the_notation_system} % (fold)

We now summarize a technical presentation of the calculus that
embodies our theory of dynamics. The typical presentation of such a
calculus follows the style of giving generators and relations on
them. The grammar, below, describing term constructors, freely
generates the set of processes, $\Proc$. This set is then quotiented
by a relation known as structural congruence and it is over this set
that the notion of dynamics is expressed. This presentation is
essentially that of \cite{MeredithR05} with the addition of
polyadicity and summation. For readability we have relegated some of
the technical subtleties to an appendix.

\subsubsection{Process grammar}\label{subsub:process_grammar}

\begin{mathpar}
  \inferrule* [lab=synchronization] {} {{M} \bc \pzero \;|\; x?F \;|\; x!C }
  \and
  \inferrule* [lab=abstraction] {} {{F} \bc (x)P}
  \and
  \inferrule* [lab=concretion] {} {{C} \bc \langle Q \rangle}
  \and
  \inferrule* [lab=process] {} {{P,Q} \bc M \;| \;P|Q \;|\; @{x}}
  \and
  \inferrule* [lab=name] {} {{x} \bc \quotep{P}}
\end{mathpar} 

Note that $\vec{x}$ (resp. $\vec{P}$) denotes a vector of names
(resp. processes) of length $|\vec{x}|$ (resp. $|\vec{P}|$). We adopt
the following useful abbreviations.

\begin{mathpar}
   x?(\vec{y}).P := x.(\vec{y})P \and  x\clift{\vec{P}} := x.\clift{\vec{P}}
   \and x!(y) := \lift{x}{\dropn{y}}
   \and \Pi_{i=0}^{n-1}P_i := P_0 | \ldots | P_{n-1}
\end{mathpar}

\subsubsection{Structural congruence}

\paragraph{Free and bound names and alpha-equivalence.} At the
core of structural equivalence is alpha-equivalence which identifies
process that are the same up to a change of variable. Formally, we
recognize the distinction between free and bound names. The free names
of a process, $\freenames{P}$, may be calculated recursively as
follows:

\begin{mathpar}
\freenames{\pzero} := \emptyset
  \and \\
  \freenames{x?(y).P} := \{ x \} \cup (\freenames{P} \setminus \{ y \})
  \and 
  \freenames{x!\langle P \rangle} := \{ x \} \cup \{ P \} 
  \and \\
  \freenames{P|Q} := \freenames{P} \cup \freenames{Q}
  \and \\
  \freenames{@{x}} := \{ x \}
\end{mathpar}

$\pi$
$\quotep{\pi}$

$\freenames{-} : \pi \to \mathcal{P}(\quotep{\pi})$

\begin{eqnarray*}
  \freenames{\pzero} & := & \emptyset \\
  \freenames{x?(y).P} & := & \{ x \} \cup (\freenames{P} \setminus \{ y \}) \\
  \freenames{x!\langle P \rangle} & := & \{ x \} \cup \{ P \} \\
  \freenames{P|Q} & := & \freenames{P} \cup \freenames{Q} \\
  \freenames{\dropn{x}} & := & \{ x \}
\end{eqnarray*}

The bound names of a process, $\boundnames{P}$, are those names occurring in $P$
that are not free. For example, in $x?(y).0$, the name $x$ is free, while $y$ is bound.

\begin{mathpar}
  \inferrule* [lab=monoidal-laws] {} { P|Q \equiv Q|P \and P|0 \equiv P \and P|(Q|R) \equiv (P|Q)|R }
\end{mathpar}

\begin{mathpar}
  \inferrule* [lab=alpha-equivalence] {} { (x)P \equiv (y)P\{y/x\} \and y \not\in \freenames{P} }
\end{mathpar}

\begin{definition}
Then two processes, $P,Q$, are alpha-equivalent if $P = Q\{\vec{y}/\vec{x}\}$ for
some $\vec{x} \in \boundnames{Q},\vec{y} \in \boundnames{P}$, where $Q\{\vec{y}/\vec{x}\}$
denotes the capture-avoiding substitution of $\vec{y}$ for $\vec{x}$ in $Q$.
\end{definition}

\begin{definition}
  The {\em structural congruence} \cite{SangiorgiWalker} , $\equiv$,
  between processes is the least congruence containing
  alpha-equivalence, satisfying the abelian monoid laws
  (associativity, commutativity and $\pzero$ as identity) for parallel
  composition $|$ and for summation $+$.
\end{definition}

\subsection{Name equivalence}

We take name equivalence, written $\nameeq$, to be the smallest
equivalence relation generated by the following rules.

\begin{mathpar}
\inferrule*[lab=Quote-drop]
{ }
{ \quotep{@{x}} \nameeq x }

\inferrule*[lab=Struct-equiv]
{ P \scong Q }
{ \quotep{P} \nameeq \quotep{Q} }
\end{mathpar}

The astute reader will have noticed that the mutual recursion of names
and processes imposes a mutual recursion on alpha-equivalence and
structural equivalence via name-equivalence. Fortunately, all of this
works out pleasantly and we may calculate in the natural way, free of
concern. The reader interested in the details is referred to the
appendix \ref{appendix:rho_details}.

\subsection{Substitution}

We use $\Proc$ for the set of processes, $\QProc$ for the set of
names, and $\id{\{}\vec{y} / \vec{x} \id{\}}$ to denote partial maps,
$s : \QProc \rightarrow \QProc$. A map, $s$ lifts, uniquely, to a map
on process terms, $\widehat{s} : \Proc \rightarrow \Proc$ by the
following equations.

\begin{mathpar}
  (0) \psubstp{Q}{P} := 0 \\
  (R \juxtap S) \psubstp{Q}{P}
  :=    
  (R)\psubstp{Q}{P} \juxtap (S) \psubstp{Q}{P} \\
  (x?(y).R) \psubstp{Q}{P}    
  :=    
  (x)\substp{Q}{P} (z)\concat( (R \psubstn{z}{y}) \psubstp{Q}{P} ) \\
  (\lift{x}{R}) \psubstp{Q}{P}  
  :=
  \lift{(x)\substp{Q}{P}}{ R \psubstp{Q}{P} } \\
%   (\dropn{x})  \psubstp{Q}{P}       
%   := 
%   \left\{ 
%     \begin{array}{ccc} 
%       \dropn{\quotep{Q}} & & x \nameeq \quotep{P} \\
%       \dropn{x} & & otherwise \\
%     \end{array}
%   \right. 
  (\dropn{x})  \psubstp{Q}{P}       
  := 
  \left\{ 
    \begin{array}{ccc} 
      Q & & x \nameeq \quotep{P} \\
      \dropn{x} & & otherwise \\
    \end{array}
  \right.
\end{mathpar}
 

where

\begin{eqnarray}
  (x)\id{\{} \lpquote Q \rpquote / \lpquote P \rpquote \id{\}}            = 
  \left\{ 
    \begin{array}{ccc}
      \lpquote Q \rpquote & & x \nameeq \lpquote P \rpquote \\
      x & & otherwise \\
    \end{array}
  \right. \nonumber
\end{eqnarray}

and $z$ is chosen distinct from $\quotep{P}$, $\quotep{Q}$, the free
names in $Q$, and all the names in $R$. Our $\alpha$-equivalence will
be built in the standard way from this substitution.

\begin{remark}\label{rem:no_self_referential_names}
  One consequence of these definitions is that $\forall P. \quotep{P}
  \not\in \freenames{P}$.
\end{remark}

\subsection{ Dynamic quote: an example }

Anticipating something of what's to come, consider applying the
substitution, $\widehat{\id{\{}u / z \id{\}}}$, to the following pair
of processes, $\lift{w}{y!(z)}$ and $w[ \lpquote y!(z) \rpquote ]$.

\begin{eqnarray}
	\lift{w}{y!(z)}\widehat{\id{\{}u / z \id{\}}}
		& = &
		\lift{w}{y!(u)} \nonumber\\
	w[ \lpquote y!(z) \rpquote ] \widehat{ \id{\{}u / z \id{\}} }
		& = &
		w[ \lpquote y!(z) \rpquote ] \nonumber
\end{eqnarray}

Because the body of the process between quotes is impervious to
substitution, we get radically different answers. In fact, by
examining the first process in an input context,
e.g. $x?(z).\lift{w}{y!(z)}$, we see that the process under the lift
operator may be shaped by prefixed inputs binding a name inside it. In
this sense, the lift operator will be seen as a way to dynamically
construct processes before reifying them as names.

Finally equipped with these standard features we can present the
dynamics of the calculus.

\subsubsection{Operational semantics} 

Finally, we introduce the computational dynamics. What marks these
algebras as distinct from other more traditionally studied algebraic
structures, e.g. vector spaces or polynomial rings, is the manner in
which dynamics is captured. In traditional structures, dynamics is typically
expressed through morphisms between such structures, as in linear maps
between vector spaces or morphisms between rings. In algebras
associated with the semantics of computation, the dynamics is
expressed as part of the algebraic structure itself, through a
reduction reduction relation typically denoted by $\red$. Below, we
give a recursive presentation of this relation for the calculus used
in the encoding.

$\red \subseteq \pi \times \pi$
$\red : \pi \to \mathcal{P}(\pi)$

\begin{mathpar}
  \inferrule* [lab=Comm] { \textsf{match}( x_{src}, x_{trgt} ) } { x_{trgt}?(y)P \; | \; x_{src}!\langle {Q} \rangle \red P\{\quotep{Q}/y}\} }
  \and \\
  \inferrule* [lab=Par] {{P} \red {P}'} {{{P} | {Q}} \red {{P}' | {Q}}}
  \and
  \inferrule* [lab=Equiv]{{{P} \scong {P}'} \andalso {{P}' \red {Q}'} \andalso {{Q}' \scong {Q}}}{{P} \red {Q}}
\end{mathpar}

\begin{eqnarray*}
  match_{\equiv} (\quotep{P},\quotep{Q}) & := & P \equiv Q \\
  match_{\dagger}(\quotep{P},\quotep{Q}) & := & \forall R. P|Q \red^{*} R => R \red^{*} 0 \\
  match_{K}(\quotep{P},\quotep{Q}) & := & K \mbox{ for some context } K
\end{eqnarray*}

$u?(x)P | u!\langle Q \rangle \red P\{\quotep{Q}/x\}$

%We write $\wred$ for $\red^*$, and $P\red$ if $\exists Q $ such that $ P \red Q$.
We write $P\red$ if $\exists Q $ such that $ P \red Q$ and $P\not\red$, otherwise.

\section{Replication}

As mentioned before, it is known that replication (and hence
recursion) can be implemented in a higher-order process algebra
\cite{SangiorgiWalker}. As our first example of calculation with the
machinery thus far presented we give the construction explicitly in
the {\rhoc}.

\begin{eqnarray}
	D_{x} & := & \prefix{x}{y}{(\binpar{\outputp{x}{y}}{@{y}})} \nonumber\\
	\bangp_{x}{P} & := & \binpar{{x}!\langle{\binpar{D_{x}}{P}}\rangle}{D_{x}} \nonumber
\end{eqnarray}

\begin{eqnarray}
	\bangp_{x}{P} & & \nonumber\\
	=
	& {x}!\langle{(\prefix{x}{y}{(\outputp{x}{y} | @{y})) | P}}\rangle 
	      | \prefix{x}{y}{(\outputp{x}{y} | @{y})} & \nonumber\\
	\red
	& (\outputp{x}{y} | @{y})\substn{\quotep{(\prefix{x}{y}{(@{y} | \outputp{x}{y})) | P}}}{y} & \nonumber\\
	=
	& \outputp{x}{\quotep{(\prefix{x}{y}{(\outputp{x}{y} | @{y})) | P}}}
	  | {(\prefix{x}{y}{(\outputp{x}{y} | @{y})) | P}} & \nonumber\\
	\red
	& \ldots & \nonumber\\
	\red^*
	& P | P | \ldots & \nonumber
\end{eqnarray}

Of course, this encoding, as an implementation, runs away, unfolding
$\bangp{P}$ eagerly. A lazier and more implementable replication
operator, restricted to input-guarded processes, may be obtained as follows.

\begin{eqnarray}
\bangp{\prefix{u}{v}{P}} 
	:= 
	\binpar{\lift{x}{\prefix{u}{v}{(\binpar{D(x)}{P})}}}{D(x)} \nonumber
\end{eqnarray}

\begin{remark}
  Note that the lazier definition still does not deal with summation
  or mixed summation (i.e. sums over input and output). The reader is
  invited to construct definitions of replication that deal with these
  features. 

  Further, the definitions are parameterized in a name, $x$. Can you,
  gentle reader, make a definition that eliminates this parameter and
  guarantees no accidental interaction between the replication
  machinery and the process being replicated -- i.e. no accidental
  sharing of names used by the process to get its work done and the
  name(s) used by the replication to effect copying. This latter
  revision of the definition of replication is crucial to obtaining
  the expected identity $!!P \sim !P$.
\end{remark}

\begin{remark}\label{rem:paradoxical_combinator}
  The reader familiar with the lambda calculus will have noticed the
  similarity between $D$ and the paradoxical combinator.

  [Ed. note: the existence of this seems to suggest we have to be more
  restrictive on the set of processes and names we admit if we are to
  support no-cloning.]
\end{remark}

\subsubsection{Bisimulation}

The computational dynamics gives rise to another kind of equivalence,
the equivalence of computational behavior. As previously mentioned
this is typically captured \emph{via} some form of bisimulation.

% The notion we use in this paper is weak barbed bisimulation
% \cite{milner91polyadicpi}.

The notion we use in this paper is derived from weak barbed
bisimulation \cite{milner91polyadicpi}. 

\begin{definition}
An \emph{observation relation}, $\downarrow_{\mathcal N}$, over a set
of names, $\mathcal N$, is the smallest relation satisfying the rules
below.

\infrule[Out-barb]{y \in {\mathcal N}, \; x \nameeq y}
		  {\outputp{x}{v} \downarrow_{\mathcal N} x}
\infrule[Par-barb]{\mbox{$P\downarrow_{\mathcal N} x$ or $Q\downarrow_{\mathcal N} x$}}
		  {\binpar{P}{Q} \downarrow_{\mathcal N} x}

We write $P \Downarrow_{\mathcal N} x$ if there is $Q$ such that 
$P \wred Q$ and $Q \downarrow_{\mathcal N} x$.
\end{definition}

\begin{definition}
%\label{def.bbisim}
An  ${\mathcal N}$-\emph{barbed bisimulation} over a set of names, ${\mathcal N}$, is a symmetric binary relation 
${\mathcal S}_{\mathcal N}$ between agents such that $P\rel{S}_{\mathcal N}Q$ implies:
\begin{enumerate}
\item If $P \red P'$ then $Q \wred Q'$ and $P'\rel{S}_{\mathcal N} Q'$.
\item If $P\downarrow_{\mathcal N} x$, then $Q\Downarrow_{\mathcal N} x$.
\end{enumerate}
$P$ is ${\mathcal N}$-barbed bisimilar to $Q$, written
$P \wbbisim_{\mathcal N} Q$, if $P \rel{S}_{\mathcal N} Q$ for some ${\mathcal N}$-barbed bisimulation ${\mathcal S}_{\mathcal N}$.
\end{definition}

$\mathcal{R} \subseteq \pi \times \pi$

$P \mathcal{R} Q => \forall P'. P \red P' \Rightarrow \exists Q'. Q \red Q', P' \mathcal{R} Q'$

$P \vdash x \Rightarrow Q \vdash x$

\begin{mathpar}
  \inferrule*[lab=Out-barb]{x \nameeq y}{{y}!\langle{Q}\rangle \vdash x}
  \and
  \inferrule*[lab=Par-barb]{\mbox{$P\vdash x$ or $Q\vdash x$}}{\binpar{P}{Q} \vdash x}
\end{mathpar}

\subsubsection{Contexts}

One of the principle advantages of computational calculi like the
$\pi$-calculus is a well-defined notion of context,
contextual-equivalence and a correlation between
contextual-equivalence and notions of bisimulation. The notion of
context allows the decomposition of a process into (sub-)process and
its syntactic environment, its context. Thus, a context may be
thought of as a process with a ``hole'' (written $\Box$) in it. The
application of a context $M$ to a process $P$, written $M[P]$, is
tantamount to filling the hole in $M$ with $P$. In this paper we do
not need the full weight of this theory, but do make use of the notion
of context in the proof the main theorem. 

\begin{mathpar}
  \inferrule* [lab=summation] {} {{M_{M},M_{N}} \bc \Box \;|\; x.M_{A} \;|\; M_{M}+M_{N}}
  \and
  \inferrule* [lab=agent] {} {{M_{A}} \bc (\vec{x})M_{P} \;| \; \clift{P_0,\ldots,M_{P},\ldots,P_N}}
  \and \\
  \inferrule* [lab=process] {} {{M_{P}} \bc M_{N} \;| \;P|M_{P} }
\end{mathpar} 

\begin{mathpar}
  \inferrule* [lab=sychronization] {} {M_{N} \bc \Box \;|\; x?M_{F} \;|\; x!M_{C}}
  \and
  \inferrule* [lab=abstraction] {} {{M_{F}} \bc (x)M_{P} }
  \and
  \inferrule* [lab=concretion] {} {{M_{C}} \bc \langle M_{P} \rangle }
  \and \\
  \inferrule* [lab=process] {} {{M_{P}} \bc M_{N} \;| \;P|M_{P} }
\end{mathpar}

\begin{definition}[contextual application] Given a context $M$, and
  process $P$, we define the \emph{contextual application}, $M[P] :=
  M\{P/\Box\}$. That is, the contextual application of M to P is the
  substitution of $P$ for $\Box$ in $M$.
\end{definition}

$\meaningof{-} : L \to \mathcal{P}(\pi)$

\begin{mathpar}
  \inferrule* [lab=collection] {} {\meaningof{true} = \pi, \and \meaningof{~E} = \pi \setminus \meaningof{E}, \and \meaningof{E_{1} \& E_{2}} = \meaningof{E_{1}} \cap \meaningof{E_{2}}}
\end{mathpar}

\begin{mathpar}
  \inferrule* [lab=structure] {} {\meaningof{0} = \{ P \in \pi | P \equiv 0 \}, \and \\ \meaningof{E_1 | E_2} = \{ P \in \pi | P \equiv P_{1} | P_{2}, P_{1} \in \meaningof{E_{1}}, P_{2} \in \meaningof{E_2}\} }
\end{mathpar}

\begin{mathpar}
 \inferrule* [lab=behavior] {} {\meaningof{\langle a?b \rangle E} = \{ P \in \pi | P \equiv Q | u?(y)P', \\ \and \\\\ \and \\ \;\;\; u \in \meaningof{a}, \forall z.P'\{z/y\} \in \meaningof{E\{z/b\}}\}, \and \\ \meaningof{a!E} = \{ P \in \pi | P \equiv Q | x!\langle P' \rangle, x \in \meaningof{a} P' \in \meaningof{E}\} }
\end{mathpar}

\begin{mathpar}
 \inferrule* [lab=nominal] {} {\meaningof{\quotep{E}} = \{ \quotep{P} \in \quotep{\pi} | P \in \meaningof{E} \}, \and \meaningof{\quotep{P}} = \{ \quotep{Q} \in \quotep{\pi} | P \equiv Q \} \and \\ \meaningof{@\quotep{E}} = \{ P \in \pi | P \equiv @x, x \in \meaningof{E} \}}
\end{mathpar}

\begin{eqnarray*}
  \\
  \meaningof{-} : TS \to ST
\end{eqnarray*}

\begin{eqnarray*}
  \\
  L : TS \to ST
\end{eqnarray*}

\begin{eqnarray*}
  \\
  P \models E \iff P \in \meaningof{E}
\end{eqnarray*}

\begin{eqnarray*}
  P \approx_{L} Q \iff \forall E \in L. P \models E \iff Q \models E
\end{eqnarray*}

\begin{eqnarray*}
  P \approx_{K} Q
\end{eqnarray*}

\begin{eqnarray*}
  P \approx Q
\end{eqnarray*}

$\approx_{K} = \approx = \approx_{L}$

\subsubsection{Contextual duality}

Note that contexts extend the quotation operation to a family of
operations from processes to names. Given a context, $M$, we can
define a \emph{nominal context}, $\quotep{M}$ by $\quotep{M}[P] :=
\quotep{M[P]}$. To foreshadow what is to come we observe that these
operations enjoy a duality with processes very much like the duality
between vectors and maps from vectors to scalars.

Further, because the calculus is essentially higher-order, we have a
correspondence between contexts and processes. More specifically,
given a name $x$ and a context $M$ we can construct $M^{*}_{x}$ such
that 

\begin{mathpar}
  M^{*}_{x} | \lift{x}{P} \red M[P]
\end{mathpar}

namely,

\begin{mathpar}
  M^{*}_{x} := x?(u).M[\dropn{u}]
\end{mathpar}

The dependence of $M^{*}_{x}$ on a name makes it an abstraction, 

\begin{mathpar}
  M^{*} := (x)x?(u).M[\dropn{u}]
\end{mathpar}

\subsection{Additional notation}

It will sometimes be convenient to denote the process a name
quotes. We already have the notation $x = \quotep{P}$, but it will be
convenient to introduce an alternate notation, $\procn{x}$, when we
want to emphasize the connection to the use of the name. Note that, by
virtue of name equivalence, $\quotep{\procn{x}} \nameeq x$; so, the
notation is consistent with previous definitions.

Further, because names have structure it is possible to effect
substitutions on the basis of that structure. This means we need to
upgrade our notation for substitutions, which we accomplish by
adapting comprehension notation. Thus,

\begin{mathpar}
  P\{ y / x : x \in S \}
\end{mathpar}

is interpreted to mean the process derived from P by replacing (in a
capture-avoiding manner) each occurrence of $x$ in $S$ by $y$. For example,

\begin{mathpar}
  P\{ \quotep{\procn{x}|\procn{x}} / x : x \in \freenames{P} \}
\end{mathpar}

will replace each (occurrence) of a free name $x$ in $P$ by
$\quotep{\procn{x}|\procn{x}}$.

Also, we will avail ourselves of the notation $x^{L}$ and $x^{R}$ to
denote injections of a name into disjoint copies of the name
space. There are numerous ways to accomplish this. One example can be
found in \cite{MeredithR05}. This notation overloads to vectors of
names: $\vec{x}^{\pi} := (x_{i}^{\pi} \; : \; 0 \leq i < |\vec{x}| )$ where $\pi \in \{L,R\}$.

We also use $P^{\Box} := P|\Box$.

In \cite{MeredithR05} an interpretation of the new operator is
given. It turns out that there are several possible interpretations
all enjoying the requisite algebraic properties of the operator (see
\cite{milner91polyadicpi}). We will therefore make liberal use of
$(\nu\; \vec{x})P$.

% subsection the_syntax_and_semantics_of_the_notation_system (end)   

\input{qm2pi.qmops} 

\input{qm2pi.sterngerlach} 

\input{qm2pi.metric} 

% section concurrent_process_calculi (end)

%\input{qm2pi.proofsketch}

% section proof sketch (end)

%\input{qm2pi.slviaknots} 

% section spatial logic via knots (end)

\input{qm2pi.conclusion}

% section conclusion (end)

%\input{qm2pi.dtcodes} 

% section wiring algorithm (end)

\input{qm2pi.ack} 

% section acknowledgments (end)

\newpage


\bibliographystyle{plain}   
\bibliography{../../biblios/main.bib}

\input{qm2pi.rhodetails}

\end{document}

 

% section notation (end)

\input{qm2pi.process.calculi} 

% section concurrent_process_calculi_and_spatial_logics_ (end)
    
%\documentclass[12pt]{llncs}
%\documentclass{jktr}

\usepackage[pdftex]{hyperref}                   
\usepackage {listings}
\usepackage {mathpartir}
\usepackage{bcprules}
%\usepackage{listings}
                       
\usepackage{graphicx} 
%\usepackage[margins=2.5cm,nohead,nofoot]{geometry}
%\usepackage{geometry}
\usepackage{amsfonts}
\usepackage{amstext}
\usepackage{latexsym}
\usepackage{amssymb}
\usepackage{color}


%\include{myPreamble}
\include{qm2pi.local} 

%\ifpdf
%\usepackage[pdftex]{graphicx}
%\else
%\usepackage{graphicx}
%\fi

 % \ifpdf
%  \usepackage{pdfsync}
%  \if


%\title{Brief Article}
%\author{David F. Snyder}
%\author{L.G. Meredith}

%\address{Dept. of Math., Texas State University--San Marcos, San Marcos, TX 78666}
       
\pagestyle{empty}


\begin{document}

\lstset{language=[Objective]Caml,frame=shadowbox}

\input{qm2pi.front}

% section front matter (end)

\input{qm2pi.intro} 
 
% section introduction (end)

% \input{qm2pi.knotations} 

% section notation (end)

\input{qm2pi.process.calculi} 

% section concurrent_process_calculi_and_spatial_logics_ (end)
    
%\input{qm2pi.knots2pi} 

%\input{qm2pi.trefoil} 

%\input{qm2pi.mainthm} 

% subsection basic_interpretation (end)

%\input{qm2pi.rho.presentation} 
\subsection{The syntax and semantics of the notation system}\label{sub:the_syntax_and_semantics_of_the_notation_system} % (fold)

We now summarize a technical presentation of the calculus that
embodies our theory of dynamics. The typical presentation of such a
calculus follows the style of giving generators and relations on
them. The grammar, below, describing term constructors, freely
generates the set of processes, $\Proc$. This set is then quotiented
by a relation known as structural congruence and it is over this set
that the notion of dynamics is expressed. This presentation is
essentially that of \cite{MeredithR05} with the addition of
polyadicity and summation. For readability we have relegated some of
the technical subtleties to an appendix.

\subsubsection{Process grammar}\label{subsub:process_grammar}

\begin{mathpar}
  \inferrule* [lab=synchronization] {} {{M} \bc \pzero \;|\; x?F \;|\; x!C }
  \and
  \inferrule* [lab=abstraction] {} {{F} \bc (x)P}
  \and
  \inferrule* [lab=concretion] {} {{C} \bc \langle Q \rangle}
  \and
  \inferrule* [lab=process] {} {{P,Q} \bc M \;| \;P|Q \;|\; @{x}}
  \and
  \inferrule* [lab=name] {} {{x} \bc \quotep{P}}
\end{mathpar} 

Note that $\vec{x}$ (resp. $\vec{P}$) denotes a vector of names
(resp. processes) of length $|\vec{x}|$ (resp. $|\vec{P}|$). We adopt
the following useful abbreviations.

\begin{mathpar}
   x?(\vec{y}).P := x.(\vec{y})P \and  x\clift{\vec{P}} := x.\clift{\vec{P}}
   \and x!(y) := \lift{x}{\dropn{y}}
   \and \Pi_{i=0}^{n-1}P_i := P_0 | \ldots | P_{n-1}
\end{mathpar}

\subsubsection{Structural congruence}

\paragraph{Free and bound names and alpha-equivalence.} At the
core of structural equivalence is alpha-equivalence which identifies
process that are the same up to a change of variable. Formally, we
recognize the distinction between free and bound names. The free names
of a process, $\freenames{P}$, may be calculated recursively as
follows:

\begin{mathpar}
\freenames{\pzero} := \emptyset
  \and \\
  \freenames{x?(y).P} := \{ x \} \cup (\freenames{P} \setminus \{ y \})
  \and 
  \freenames{x!\langle P \rangle} := \{ x \} \cup \{ P \} 
  \and \\
  \freenames{P|Q} := \freenames{P} \cup \freenames{Q}
  \and \\
  \freenames{@{x}} := \{ x \}
\end{mathpar}

$\pi$
$\quotep{\pi}$

$\freenames{-} : \pi \to \mathcal{P}(\quotep{\pi})$

\begin{eqnarray*}
  \freenames{\pzero} & := & \emptyset \\
  \freenames{x?(y).P} & := & \{ x \} \cup (\freenames{P} \setminus \{ y \}) \\
  \freenames{x!\langle P \rangle} & := & \{ x \} \cup \{ P \} \\
  \freenames{P|Q} & := & \freenames{P} \cup \freenames{Q} \\
  \freenames{\dropn{x}} & := & \{ x \}
\end{eqnarray*}

The bound names of a process, $\boundnames{P}$, are those names occurring in $P$
that are not free. For example, in $x?(y).0$, the name $x$ is free, while $y$ is bound.

\begin{mathpar}
  \inferrule* [lab=monoidal-laws] {} { P|Q \equiv Q|P \and P|0 \equiv P \and P|(Q|R) \equiv (P|Q)|R }
\end{mathpar}

\begin{mathpar}
  \inferrule* [lab=alpha-equivalence] {} { (x)P \equiv (y)P\{y/x\} \and y \not\in \freenames{P} }
\end{mathpar}

\begin{definition}
Then two processes, $P,Q$, are alpha-equivalent if $P = Q\{\vec{y}/\vec{x}\}$ for
some $\vec{x} \in \boundnames{Q},\vec{y} \in \boundnames{P}$, where $Q\{\vec{y}/\vec{x}\}$
denotes the capture-avoiding substitution of $\vec{y}$ for $\vec{x}$ in $Q$.
\end{definition}

\begin{definition}
  The {\em structural congruence} \cite{SangiorgiWalker} , $\equiv$,
  between processes is the least congruence containing
  alpha-equivalence, satisfying the abelian monoid laws
  (associativity, commutativity and $\pzero$ as identity) for parallel
  composition $|$ and for summation $+$.
\end{definition}

\subsection{Name equivalence}

We take name equivalence, written $\nameeq$, to be the smallest
equivalence relation generated by the following rules.

\begin{mathpar}
\inferrule*[lab=Quote-drop]
{ }
{ \quotep{@{x}} \nameeq x }

\inferrule*[lab=Struct-equiv]
{ P \scong Q }
{ \quotep{P} \nameeq \quotep{Q} }
\end{mathpar}

The astute reader will have noticed that the mutual recursion of names
and processes imposes a mutual recursion on alpha-equivalence and
structural equivalence via name-equivalence. Fortunately, all of this
works out pleasantly and we may calculate in the natural way, free of
concern. The reader interested in the details is referred to the
appendix \ref{appendix:rho_details}.

\subsection{Substitution}

We use $\Proc$ for the set of processes, $\QProc$ for the set of
names, and $\id{\{}\vec{y} / \vec{x} \id{\}}$ to denote partial maps,
$s : \QProc \rightarrow \QProc$. A map, $s$ lifts, uniquely, to a map
on process terms, $\widehat{s} : \Proc \rightarrow \Proc$ by the
following equations.

\begin{mathpar}
  (0) \psubstp{Q}{P} := 0 \\
  (R \juxtap S) \psubstp{Q}{P}
  :=    
  (R)\psubstp{Q}{P} \juxtap (S) \psubstp{Q}{P} \\
  (x?(y).R) \psubstp{Q}{P}    
  :=    
  (x)\substp{Q}{P} (z)\concat( (R \psubstn{z}{y}) \psubstp{Q}{P} ) \\
  (\lift{x}{R}) \psubstp{Q}{P}  
  :=
  \lift{(x)\substp{Q}{P}}{ R \psubstp{Q}{P} } \\
%   (\dropn{x})  \psubstp{Q}{P}       
%   := 
%   \left\{ 
%     \begin{array}{ccc} 
%       \dropn{\quotep{Q}} & & x \nameeq \quotep{P} \\
%       \dropn{x} & & otherwise \\
%     \end{array}
%   \right. 
  (\dropn{x})  \psubstp{Q}{P}       
  := 
  \left\{ 
    \begin{array}{ccc} 
      Q & & x \nameeq \quotep{P} \\
      \dropn{x} & & otherwise \\
    \end{array}
  \right.
\end{mathpar}
 

where

\begin{eqnarray}
  (x)\id{\{} \lpquote Q \rpquote / \lpquote P \rpquote \id{\}}            = 
  \left\{ 
    \begin{array}{ccc}
      \lpquote Q \rpquote & & x \nameeq \lpquote P \rpquote \\
      x & & otherwise \\
    \end{array}
  \right. \nonumber
\end{eqnarray}

and $z$ is chosen distinct from $\quotep{P}$, $\quotep{Q}$, the free
names in $Q$, and all the names in $R$. Our $\alpha$-equivalence will
be built in the standard way from this substitution.

\begin{remark}\label{rem:no_self_referential_names}
  One consequence of these definitions is that $\forall P. \quotep{P}
  \not\in \freenames{P}$.
\end{remark}

\subsection{ Dynamic quote: an example }

Anticipating something of what's to come, consider applying the
substitution, $\widehat{\id{\{}u / z \id{\}}}$, to the following pair
of processes, $\lift{w}{y!(z)}$ and $w[ \lpquote y!(z) \rpquote ]$.

\begin{eqnarray}
	\lift{w}{y!(z)}\widehat{\id{\{}u / z \id{\}}}
		& = &
		\lift{w}{y!(u)} \nonumber\\
	w[ \lpquote y!(z) \rpquote ] \widehat{ \id{\{}u / z \id{\}} }
		& = &
		w[ \lpquote y!(z) \rpquote ] \nonumber
\end{eqnarray}

Because the body of the process between quotes is impervious to
substitution, we get radically different answers. In fact, by
examining the first process in an input context,
e.g. $x?(z).\lift{w}{y!(z)}$, we see that the process under the lift
operator may be shaped by prefixed inputs binding a name inside it. In
this sense, the lift operator will be seen as a way to dynamically
construct processes before reifying them as names.

Finally equipped with these standard features we can present the
dynamics of the calculus.

\subsubsection{Operational semantics} 

Finally, we introduce the computational dynamics. What marks these
algebras as distinct from other more traditionally studied algebraic
structures, e.g. vector spaces or polynomial rings, is the manner in
which dynamics is captured. In traditional structures, dynamics is typically
expressed through morphisms between such structures, as in linear maps
between vector spaces or morphisms between rings. In algebras
associated with the semantics of computation, the dynamics is
expressed as part of the algebraic structure itself, through a
reduction reduction relation typically denoted by $\red$. Below, we
give a recursive presentation of this relation for the calculus used
in the encoding.

$\red \subseteq \pi \times \pi$
$\red : \pi \to \mathcal{P}(\pi)$

\begin{mathpar}
  \inferrule* [lab=Comm] { \textsf{match}( x_{src}, x_{trgt} ) } { x_{trgt}?(y)P \; | \; x_{src}!\langle {Q} \rangle \red P\{\quotep{Q}/y}\} }
  \and \\
  \inferrule* [lab=Par] {{P} \red {P}'} {{{P} | {Q}} \red {{P}' | {Q}}}
  \and
  \inferrule* [lab=Equiv]{{{P} \scong {P}'} \andalso {{P}' \red {Q}'} \andalso {{Q}' \scong {Q}}}{{P} \red {Q}}
\end{mathpar}

\begin{eqnarray*}
  match_{\equiv} (\quotep{P},\quotep{Q}) & := & P \equiv Q \\
  match_{\dagger}(\quotep{P},\quotep{Q}) & := & \forall R. P|Q \red^{*} R => R \red^{*} 0 \\
  match_{K}(\quotep{P},\quotep{Q}) & := & K \mbox{ for some context } K
\end{eqnarray*}

$u?(x)P | u!\langle Q \rangle \red P\{\quotep{Q}/x\}$

%We write $\wred$ for $\red^*$, and $P\red$ if $\exists Q $ such that $ P \red Q$.
We write $P\red$ if $\exists Q $ such that $ P \red Q$ and $P\not\red$, otherwise.

\section{Replication}

As mentioned before, it is known that replication (and hence
recursion) can be implemented in a higher-order process algebra
\cite{SangiorgiWalker}. As our first example of calculation with the
machinery thus far presented we give the construction explicitly in
the {\rhoc}.

\begin{eqnarray}
	D_{x} & := & \prefix{x}{y}{(\binpar{\outputp{x}{y}}{@{y}})} \nonumber\\
	\bangp_{x}{P} & := & \binpar{{x}!\langle{\binpar{D_{x}}{P}}\rangle}{D_{x}} \nonumber
\end{eqnarray}

\begin{eqnarray}
	\bangp_{x}{P} & & \nonumber\\
	=
	& {x}!\langle{(\prefix{x}{y}{(\outputp{x}{y} | @{y})) | P}}\rangle 
	      | \prefix{x}{y}{(\outputp{x}{y} | @{y})} & \nonumber\\
	\red
	& (\outputp{x}{y} | @{y})\substn{\quotep{(\prefix{x}{y}{(@{y} | \outputp{x}{y})) | P}}}{y} & \nonumber\\
	=
	& \outputp{x}{\quotep{(\prefix{x}{y}{(\outputp{x}{y} | @{y})) | P}}}
	  | {(\prefix{x}{y}{(\outputp{x}{y} | @{y})) | P}} & \nonumber\\
	\red
	& \ldots & \nonumber\\
	\red^*
	& P | P | \ldots & \nonumber
\end{eqnarray}

Of course, this encoding, as an implementation, runs away, unfolding
$\bangp{P}$ eagerly. A lazier and more implementable replication
operator, restricted to input-guarded processes, may be obtained as follows.

\begin{eqnarray}
\bangp{\prefix{u}{v}{P}} 
	:= 
	\binpar{\lift{x}{\prefix{u}{v}{(\binpar{D(x)}{P})}}}{D(x)} \nonumber
\end{eqnarray}

\begin{remark}
  Note that the lazier definition still does not deal with summation
  or mixed summation (i.e. sums over input and output). The reader is
  invited to construct definitions of replication that deal with these
  features. 

  Further, the definitions are parameterized in a name, $x$. Can you,
  gentle reader, make a definition that eliminates this parameter and
  guarantees no accidental interaction between the replication
  machinery and the process being replicated -- i.e. no accidental
  sharing of names used by the process to get its work done and the
  name(s) used by the replication to effect copying. This latter
  revision of the definition of replication is crucial to obtaining
  the expected identity $!!P \sim !P$.
\end{remark}

\begin{remark}\label{rem:paradoxical_combinator}
  The reader familiar with the lambda calculus will have noticed the
  similarity between $D$ and the paradoxical combinator.

  [Ed. note: the existence of this seems to suggest we have to be more
  restrictive on the set of processes and names we admit if we are to
  support no-cloning.]
\end{remark}

\subsubsection{Bisimulation}

The computational dynamics gives rise to another kind of equivalence,
the equivalence of computational behavior. As previously mentioned
this is typically captured \emph{via} some form of bisimulation.

% The notion we use in this paper is weak barbed bisimulation
% \cite{milner91polyadicpi}.

The notion we use in this paper is derived from weak barbed
bisimulation \cite{milner91polyadicpi}. 

\begin{definition}
An \emph{observation relation}, $\downarrow_{\mathcal N}$, over a set
of names, $\mathcal N$, is the smallest relation satisfying the rules
below.

\infrule[Out-barb]{y \in {\mathcal N}, \; x \nameeq y}
		  {\outputp{x}{v} \downarrow_{\mathcal N} x}
\infrule[Par-barb]{\mbox{$P\downarrow_{\mathcal N} x$ or $Q\downarrow_{\mathcal N} x$}}
		  {\binpar{P}{Q} \downarrow_{\mathcal N} x}

We write $P \Downarrow_{\mathcal N} x$ if there is $Q$ such that 
$P \wred Q$ and $Q \downarrow_{\mathcal N} x$.
\end{definition}

\begin{definition}
%\label{def.bbisim}
An  ${\mathcal N}$-\emph{barbed bisimulation} over a set of names, ${\mathcal N}$, is a symmetric binary relation 
${\mathcal S}_{\mathcal N}$ between agents such that $P\rel{S}_{\mathcal N}Q$ implies:
\begin{enumerate}
\item If $P \red P'$ then $Q \wred Q'$ and $P'\rel{S}_{\mathcal N} Q'$.
\item If $P\downarrow_{\mathcal N} x$, then $Q\Downarrow_{\mathcal N} x$.
\end{enumerate}
$P$ is ${\mathcal N}$-barbed bisimilar to $Q$, written
$P \wbbisim_{\mathcal N} Q$, if $P \rel{S}_{\mathcal N} Q$ for some ${\mathcal N}$-barbed bisimulation ${\mathcal S}_{\mathcal N}$.
\end{definition}

$\mathcal{R} \subseteq \pi \times \pi$

$P \mathcal{R} Q => \forall P'. P \red P' \Rightarrow \exists Q'. Q \red Q', P' \mathcal{R} Q'$

$P \vdash x \Rightarrow Q \vdash x$

\begin{mathpar}
  \inferrule*[lab=Out-barb]{x \nameeq y}{{y}!\langle{Q}\rangle \vdash x}
  \and
  \inferrule*[lab=Par-barb]{\mbox{$P\vdash x$ or $Q\vdash x$}}{\binpar{P}{Q} \vdash x}
\end{mathpar}

\subsubsection{Contexts}

One of the principle advantages of computational calculi like the
$\pi$-calculus is a well-defined notion of context,
contextual-equivalence and a correlation between
contextual-equivalence and notions of bisimulation. The notion of
context allows the decomposition of a process into (sub-)process and
its syntactic environment, its context. Thus, a context may be
thought of as a process with a ``hole'' (written $\Box$) in it. The
application of a context $M$ to a process $P$, written $M[P]$, is
tantamount to filling the hole in $M$ with $P$. In this paper we do
not need the full weight of this theory, but do make use of the notion
of context in the proof the main theorem. 

\begin{mathpar}
  \inferrule* [lab=summation] {} {{M_{M},M_{N}} \bc \Box \;|\; x.M_{A} \;|\; M_{M}+M_{N}}
  \and
  \inferrule* [lab=agent] {} {{M_{A}} \bc (\vec{x})M_{P} \;| \; \clift{P_0,\ldots,M_{P},\ldots,P_N}}
  \and \\
  \inferrule* [lab=process] {} {{M_{P}} \bc M_{N} \;| \;P|M_{P} }
\end{mathpar} 

\begin{mathpar}
  \inferrule* [lab=sychronization] {} {M_{N} \bc \Box \;|\; x?M_{F} \;|\; x!M_{C}}
  \and
  \inferrule* [lab=abstraction] {} {{M_{F}} \bc (x)M_{P} }
  \and
  \inferrule* [lab=concretion] {} {{M_{C}} \bc \langle M_{P} \rangle }
  \and \\
  \inferrule* [lab=process] {} {{M_{P}} \bc M_{N} \;| \;P|M_{P} }
\end{mathpar}

\begin{definition}[contextual application] Given a context $M$, and
  process $P$, we define the \emph{contextual application}, $M[P] :=
  M\{P/\Box\}$. That is, the contextual application of M to P is the
  substitution of $P$ for $\Box$ in $M$.
\end{definition}

$\meaningof{-} : L \to \mathcal{P}(\pi)$

\begin{mathpar}
  \inferrule* [lab=collection] {} {\meaningof{true} = \pi, \and \meaningof{~E} = \pi \setminus \meaningof{E}, \and \meaningof{E_{1} \& E_{2}} = \meaningof{E_{1}} \cap \meaningof{E_{2}}}
\end{mathpar}

\begin{mathpar}
  \inferrule* [lab=structure] {} {\meaningof{0} = \{ P \in \pi | P \equiv 0 \}, \and \\ \meaningof{E_1 | E_2} = \{ P \in \pi | P \equiv P_{1} | P_{2}, P_{1} \in \meaningof{E_{1}}, P_{2} \in \meaningof{E_2}\} }
\end{mathpar}

\begin{mathpar}
 \inferrule* [lab=behavior] {} {\meaningof{\langle a?b \rangle E} = \{ P \in \pi | P \equiv Q | u?(y)P', \\ \and \\\\ \and \\ \;\;\; u \in \meaningof{a}, \forall z.P'\{z/y\} \in \meaningof{E\{z/b\}}\}, \and \\ \meaningof{a!E} = \{ P \in \pi | P \equiv Q | x!\langle P' \rangle, x \in \meaningof{a} P' \in \meaningof{E}\} }
\end{mathpar}

\begin{mathpar}
 \inferrule* [lab=nominal] {} {\meaningof{\quotep{E}} = \{ \quotep{P} \in \quotep{\pi} | P \in \meaningof{E} \}, \and \meaningof{\quotep{P}} = \{ \quotep{Q} \in \quotep{\pi} | P \equiv Q \} \and \\ \meaningof{@\quotep{E}} = \{ P \in \pi | P \equiv @x, x \in \meaningof{E} \}}
\end{mathpar}

\begin{eqnarray*}
  \\
  \meaningof{-} : TS \to ST
\end{eqnarray*}

\begin{eqnarray*}
  \\
  L : TS \to ST
\end{eqnarray*}

\begin{eqnarray*}
  \\
  P \models E \iff P \in \meaningof{E}
\end{eqnarray*}

\begin{eqnarray*}
  P \approx_{L} Q \iff \forall E \in L. P \models E \iff Q \models E
\end{eqnarray*}

\begin{eqnarray*}
  P \approx_{K} Q
\end{eqnarray*}

\begin{eqnarray*}
  P \approx Q
\end{eqnarray*}

$\approx_{K} = \approx = \approx_{L}$

\subsubsection{Contextual duality}

Note that contexts extend the quotation operation to a family of
operations from processes to names. Given a context, $M$, we can
define a \emph{nominal context}, $\quotep{M}$ by $\quotep{M}[P] :=
\quotep{M[P]}$. To foreshadow what is to come we observe that these
operations enjoy a duality with processes very much like the duality
between vectors and maps from vectors to scalars.

Further, because the calculus is essentially higher-order, we have a
correspondence between contexts and processes. More specifically,
given a name $x$ and a context $M$ we can construct $M^{*}_{x}$ such
that 

\begin{mathpar}
  M^{*}_{x} | \lift{x}{P} \red M[P]
\end{mathpar}

namely,

\begin{mathpar}
  M^{*}_{x} := x?(u).M[\dropn{u}]
\end{mathpar}

The dependence of $M^{*}_{x}$ on a name makes it an abstraction, 

\begin{mathpar}
  M^{*} := (x)x?(u).M[\dropn{u}]
\end{mathpar}

\subsection{Additional notation}

It will sometimes be convenient to denote the process a name
quotes. We already have the notation $x = \quotep{P}$, but it will be
convenient to introduce an alternate notation, $\procn{x}$, when we
want to emphasize the connection to the use of the name. Note that, by
virtue of name equivalence, $\quotep{\procn{x}} \nameeq x$; so, the
notation is consistent with previous definitions.

Further, because names have structure it is possible to effect
substitutions on the basis of that structure. This means we need to
upgrade our notation for substitutions, which we accomplish by
adapting comprehension notation. Thus,

\begin{mathpar}
  P\{ y / x : x \in S \}
\end{mathpar}

is interpreted to mean the process derived from P by replacing (in a
capture-avoiding manner) each occurrence of $x$ in $S$ by $y$. For example,

\begin{mathpar}
  P\{ \quotep{\procn{x}|\procn{x}} / x : x \in \freenames{P} \}
\end{mathpar}

will replace each (occurrence) of a free name $x$ in $P$ by
$\quotep{\procn{x}|\procn{x}}$.

Also, we will avail ourselves of the notation $x^{L}$ and $x^{R}$ to
denote injections of a name into disjoint copies of the name
space. There are numerous ways to accomplish this. One example can be
found in \cite{MeredithR05}. This notation overloads to vectors of
names: $\vec{x}^{\pi} := (x_{i}^{\pi} \; : \; 0 \leq i < |\vec{x}| )$ where $\pi \in \{L,R\}$.

We also use $P^{\Box} := P|\Box$.

In \cite{MeredithR05} an interpretation of the new operator is
given. It turns out that there are several possible interpretations
all enjoying the requisite algebraic properties of the operator (see
\cite{milner91polyadicpi}). We will therefore make liberal use of
$(\nu\; \vec{x})P$.

% subsection the_syntax_and_semantics_of_the_notation_system (end)   

\input{qm2pi.qmops} 

\input{qm2pi.sterngerlach} 

\input{qm2pi.metric} 

% section concurrent_process_calculi (end)

%\input{qm2pi.proofsketch}

% section proof sketch (end)

%\input{qm2pi.slviaknots} 

% section spatial logic via knots (end)

\input{qm2pi.conclusion}

% section conclusion (end)

%\input{qm2pi.dtcodes} 

% section wiring algorithm (end)

\input{qm2pi.ack} 

% section acknowledgments (end)

\newpage


\bibliographystyle{plain}   
\bibliography{../../biblios/main.bib}

\input{qm2pi.rhodetails}

\end{document}

 

%\documentclass[12pt]{llncs}
%\documentclass{jktr}

\usepackage[pdftex]{hyperref}                   
\usepackage {listings}
\usepackage {mathpartir}
\usepackage{bcprules}
%\usepackage{listings}
                       
\usepackage{graphicx} 
%\usepackage[margins=2.5cm,nohead,nofoot]{geometry}
%\usepackage{geometry}
\usepackage{amsfonts}
\usepackage{amstext}
\usepackage{latexsym}
\usepackage{amssymb}
\usepackage{color}


%\include{myPreamble}
\include{qm2pi.local} 

%\ifpdf
%\usepackage[pdftex]{graphicx}
%\else
%\usepackage{graphicx}
%\fi

 % \ifpdf
%  \usepackage{pdfsync}
%  \if


%\title{Brief Article}
%\author{David F. Snyder}
%\author{L.G. Meredith}

%\address{Dept. of Math., Texas State University--San Marcos, San Marcos, TX 78666}
       
\pagestyle{empty}


\begin{document}

\lstset{language=[Objective]Caml,frame=shadowbox}

\input{qm2pi.front}

% section front matter (end)

\input{qm2pi.intro} 
 
% section introduction (end)

% \input{qm2pi.knotations} 

% section notation (end)

\input{qm2pi.process.calculi} 

% section concurrent_process_calculi_and_spatial_logics_ (end)
    
%\input{qm2pi.knots2pi} 

%\input{qm2pi.trefoil} 

%\input{qm2pi.mainthm} 

% subsection basic_interpretation (end)

%\input{qm2pi.rho.presentation} 
\subsection{The syntax and semantics of the notation system}\label{sub:the_syntax_and_semantics_of_the_notation_system} % (fold)

We now summarize a technical presentation of the calculus that
embodies our theory of dynamics. The typical presentation of such a
calculus follows the style of giving generators and relations on
them. The grammar, below, describing term constructors, freely
generates the set of processes, $\Proc$. This set is then quotiented
by a relation known as structural congruence and it is over this set
that the notion of dynamics is expressed. This presentation is
essentially that of \cite{MeredithR05} with the addition of
polyadicity and summation. For readability we have relegated some of
the technical subtleties to an appendix.

\subsubsection{Process grammar}\label{subsub:process_grammar}

\begin{mathpar}
  \inferrule* [lab=synchronization] {} {{M} \bc \pzero \;|\; x?F \;|\; x!C }
  \and
  \inferrule* [lab=abstraction] {} {{F} \bc (x)P}
  \and
  \inferrule* [lab=concretion] {} {{C} \bc \langle Q \rangle}
  \and
  \inferrule* [lab=process] {} {{P,Q} \bc M \;| \;P|Q \;|\; @{x}}
  \and
  \inferrule* [lab=name] {} {{x} \bc \quotep{P}}
\end{mathpar} 

Note that $\vec{x}$ (resp. $\vec{P}$) denotes a vector of names
(resp. processes) of length $|\vec{x}|$ (resp. $|\vec{P}|$). We adopt
the following useful abbreviations.

\begin{mathpar}
   x?(\vec{y}).P := x.(\vec{y})P \and  x\clift{\vec{P}} := x.\clift{\vec{P}}
   \and x!(y) := \lift{x}{\dropn{y}}
   \and \Pi_{i=0}^{n-1}P_i := P_0 | \ldots | P_{n-1}
\end{mathpar}

\subsubsection{Structural congruence}

\paragraph{Free and bound names and alpha-equivalence.} At the
core of structural equivalence is alpha-equivalence which identifies
process that are the same up to a change of variable. Formally, we
recognize the distinction between free and bound names. The free names
of a process, $\freenames{P}$, may be calculated recursively as
follows:

\begin{mathpar}
\freenames{\pzero} := \emptyset
  \and \\
  \freenames{x?(y).P} := \{ x \} \cup (\freenames{P} \setminus \{ y \})
  \and 
  \freenames{x!\langle P \rangle} := \{ x \} \cup \{ P \} 
  \and \\
  \freenames{P|Q} := \freenames{P} \cup \freenames{Q}
  \and \\
  \freenames{@{x}} := \{ x \}
\end{mathpar}

$\pi$
$\quotep{\pi}$

$\freenames{-} : \pi \to \mathcal{P}(\quotep{\pi})$

\begin{eqnarray*}
  \freenames{\pzero} & := & \emptyset \\
  \freenames{x?(y).P} & := & \{ x \} \cup (\freenames{P} \setminus \{ y \}) \\
  \freenames{x!\langle P \rangle} & := & \{ x \} \cup \{ P \} \\
  \freenames{P|Q} & := & \freenames{P} \cup \freenames{Q} \\
  \freenames{\dropn{x}} & := & \{ x \}
\end{eqnarray*}

The bound names of a process, $\boundnames{P}$, are those names occurring in $P$
that are not free. For example, in $x?(y).0$, the name $x$ is free, while $y$ is bound.

\begin{mathpar}
  \inferrule* [lab=monoidal-laws] {} { P|Q \equiv Q|P \and P|0 \equiv P \and P|(Q|R) \equiv (P|Q)|R }
\end{mathpar}

\begin{mathpar}
  \inferrule* [lab=alpha-equivalence] {} { (x)P \equiv (y)P\{y/x\} \and y \not\in \freenames{P} }
\end{mathpar}

\begin{definition}
Then two processes, $P,Q$, are alpha-equivalent if $P = Q\{\vec{y}/\vec{x}\}$ for
some $\vec{x} \in \boundnames{Q},\vec{y} \in \boundnames{P}$, where $Q\{\vec{y}/\vec{x}\}$
denotes the capture-avoiding substitution of $\vec{y}$ for $\vec{x}$ in $Q$.
\end{definition}

\begin{definition}
  The {\em structural congruence} \cite{SangiorgiWalker} , $\equiv$,
  between processes is the least congruence containing
  alpha-equivalence, satisfying the abelian monoid laws
  (associativity, commutativity and $\pzero$ as identity) for parallel
  composition $|$ and for summation $+$.
\end{definition}

\subsection{Name equivalence}

We take name equivalence, written $\nameeq$, to be the smallest
equivalence relation generated by the following rules.

\begin{mathpar}
\inferrule*[lab=Quote-drop]
{ }
{ \quotep{@{x}} \nameeq x }

\inferrule*[lab=Struct-equiv]
{ P \scong Q }
{ \quotep{P} \nameeq \quotep{Q} }
\end{mathpar}

The astute reader will have noticed that the mutual recursion of names
and processes imposes a mutual recursion on alpha-equivalence and
structural equivalence via name-equivalence. Fortunately, all of this
works out pleasantly and we may calculate in the natural way, free of
concern. The reader interested in the details is referred to the
appendix \ref{appendix:rho_details}.

\subsection{Substitution}

We use $\Proc$ for the set of processes, $\QProc$ for the set of
names, and $\id{\{}\vec{y} / \vec{x} \id{\}}$ to denote partial maps,
$s : \QProc \rightarrow \QProc$. A map, $s$ lifts, uniquely, to a map
on process terms, $\widehat{s} : \Proc \rightarrow \Proc$ by the
following equations.

\begin{mathpar}
  (0) \psubstp{Q}{P} := 0 \\
  (R \juxtap S) \psubstp{Q}{P}
  :=    
  (R)\psubstp{Q}{P} \juxtap (S) \psubstp{Q}{P} \\
  (x?(y).R) \psubstp{Q}{P}    
  :=    
  (x)\substp{Q}{P} (z)\concat( (R \psubstn{z}{y}) \psubstp{Q}{P} ) \\
  (\lift{x}{R}) \psubstp{Q}{P}  
  :=
  \lift{(x)\substp{Q}{P}}{ R \psubstp{Q}{P} } \\
%   (\dropn{x})  \psubstp{Q}{P}       
%   := 
%   \left\{ 
%     \begin{array}{ccc} 
%       \dropn{\quotep{Q}} & & x \nameeq \quotep{P} \\
%       \dropn{x} & & otherwise \\
%     \end{array}
%   \right. 
  (\dropn{x})  \psubstp{Q}{P}       
  := 
  \left\{ 
    \begin{array}{ccc} 
      Q & & x \nameeq \quotep{P} \\
      \dropn{x} & & otherwise \\
    \end{array}
  \right.
\end{mathpar}
 

where

\begin{eqnarray}
  (x)\id{\{} \lpquote Q \rpquote / \lpquote P \rpquote \id{\}}            = 
  \left\{ 
    \begin{array}{ccc}
      \lpquote Q \rpquote & & x \nameeq \lpquote P \rpquote \\
      x & & otherwise \\
    \end{array}
  \right. \nonumber
\end{eqnarray}

and $z$ is chosen distinct from $\quotep{P}$, $\quotep{Q}$, the free
names in $Q$, and all the names in $R$. Our $\alpha$-equivalence will
be built in the standard way from this substitution.

\begin{remark}\label{rem:no_self_referential_names}
  One consequence of these definitions is that $\forall P. \quotep{P}
  \not\in \freenames{P}$.
\end{remark}

\subsection{ Dynamic quote: an example }

Anticipating something of what's to come, consider applying the
substitution, $\widehat{\id{\{}u / z \id{\}}}$, to the following pair
of processes, $\lift{w}{y!(z)}$ and $w[ \lpquote y!(z) \rpquote ]$.

\begin{eqnarray}
	\lift{w}{y!(z)}\widehat{\id{\{}u / z \id{\}}}
		& = &
		\lift{w}{y!(u)} \nonumber\\
	w[ \lpquote y!(z) \rpquote ] \widehat{ \id{\{}u / z \id{\}} }
		& = &
		w[ \lpquote y!(z) \rpquote ] \nonumber
\end{eqnarray}

Because the body of the process between quotes is impervious to
substitution, we get radically different answers. In fact, by
examining the first process in an input context,
e.g. $x?(z).\lift{w}{y!(z)}$, we see that the process under the lift
operator may be shaped by prefixed inputs binding a name inside it. In
this sense, the lift operator will be seen as a way to dynamically
construct processes before reifying them as names.

Finally equipped with these standard features we can present the
dynamics of the calculus.

\subsubsection{Operational semantics} 

Finally, we introduce the computational dynamics. What marks these
algebras as distinct from other more traditionally studied algebraic
structures, e.g. vector spaces or polynomial rings, is the manner in
which dynamics is captured. In traditional structures, dynamics is typically
expressed through morphisms between such structures, as in linear maps
between vector spaces or morphisms between rings. In algebras
associated with the semantics of computation, the dynamics is
expressed as part of the algebraic structure itself, through a
reduction reduction relation typically denoted by $\red$. Below, we
give a recursive presentation of this relation for the calculus used
in the encoding.

$\red \subseteq \pi \times \pi$
$\red : \pi \to \mathcal{P}(\pi)$

\begin{mathpar}
  \inferrule* [lab=Comm] { \textsf{match}( x_{src}, x_{trgt} ) } { x_{trgt}?(y)P \; | \; x_{src}!\langle {Q} \rangle \red P\{\quotep{Q}/y}\} }
  \and \\
  \inferrule* [lab=Par] {{P} \red {P}'} {{{P} | {Q}} \red {{P}' | {Q}}}
  \and
  \inferrule* [lab=Equiv]{{{P} \scong {P}'} \andalso {{P}' \red {Q}'} \andalso {{Q}' \scong {Q}}}{{P} \red {Q}}
\end{mathpar}

\begin{eqnarray*}
  match_{\equiv} (\quotep{P},\quotep{Q}) & := & P \equiv Q \\
  match_{\dagger}(\quotep{P},\quotep{Q}) & := & \forall R. P|Q \red^{*} R => R \red^{*} 0 \\
  match_{K}(\quotep{P},\quotep{Q}) & := & K \mbox{ for some context } K
\end{eqnarray*}

$u?(x)P | u!\langle Q \rangle \red P\{\quotep{Q}/x\}$

%We write $\wred$ for $\red^*$, and $P\red$ if $\exists Q $ such that $ P \red Q$.
We write $P\red$ if $\exists Q $ such that $ P \red Q$ and $P\not\red$, otherwise.

\section{Replication}

As mentioned before, it is known that replication (and hence
recursion) can be implemented in a higher-order process algebra
\cite{SangiorgiWalker}. As our first example of calculation with the
machinery thus far presented we give the construction explicitly in
the {\rhoc}.

\begin{eqnarray}
	D_{x} & := & \prefix{x}{y}{(\binpar{\outputp{x}{y}}{@{y}})} \nonumber\\
	\bangp_{x}{P} & := & \binpar{{x}!\langle{\binpar{D_{x}}{P}}\rangle}{D_{x}} \nonumber
\end{eqnarray}

\begin{eqnarray}
	\bangp_{x}{P} & & \nonumber\\
	=
	& {x}!\langle{(\prefix{x}{y}{(\outputp{x}{y} | @{y})) | P}}\rangle 
	      | \prefix{x}{y}{(\outputp{x}{y} | @{y})} & \nonumber\\
	\red
	& (\outputp{x}{y} | @{y})\substn{\quotep{(\prefix{x}{y}{(@{y} | \outputp{x}{y})) | P}}}{y} & \nonumber\\
	=
	& \outputp{x}{\quotep{(\prefix{x}{y}{(\outputp{x}{y} | @{y})) | P}}}
	  | {(\prefix{x}{y}{(\outputp{x}{y} | @{y})) | P}} & \nonumber\\
	\red
	& \ldots & \nonumber\\
	\red^*
	& P | P | \ldots & \nonumber
\end{eqnarray}

Of course, this encoding, as an implementation, runs away, unfolding
$\bangp{P}$ eagerly. A lazier and more implementable replication
operator, restricted to input-guarded processes, may be obtained as follows.

\begin{eqnarray}
\bangp{\prefix{u}{v}{P}} 
	:= 
	\binpar{\lift{x}{\prefix{u}{v}{(\binpar{D(x)}{P})}}}{D(x)} \nonumber
\end{eqnarray}

\begin{remark}
  Note that the lazier definition still does not deal with summation
  or mixed summation (i.e. sums over input and output). The reader is
  invited to construct definitions of replication that deal with these
  features. 

  Further, the definitions are parameterized in a name, $x$. Can you,
  gentle reader, make a definition that eliminates this parameter and
  guarantees no accidental interaction between the replication
  machinery and the process being replicated -- i.e. no accidental
  sharing of names used by the process to get its work done and the
  name(s) used by the replication to effect copying. This latter
  revision of the definition of replication is crucial to obtaining
  the expected identity $!!P \sim !P$.
\end{remark}

\begin{remark}\label{rem:paradoxical_combinator}
  The reader familiar with the lambda calculus will have noticed the
  similarity between $D$ and the paradoxical combinator.

  [Ed. note: the existence of this seems to suggest we have to be more
  restrictive on the set of processes and names we admit if we are to
  support no-cloning.]
\end{remark}

\subsubsection{Bisimulation}

The computational dynamics gives rise to another kind of equivalence,
the equivalence of computational behavior. As previously mentioned
this is typically captured \emph{via} some form of bisimulation.

% The notion we use in this paper is weak barbed bisimulation
% \cite{milner91polyadicpi}.

The notion we use in this paper is derived from weak barbed
bisimulation \cite{milner91polyadicpi}. 

\begin{definition}
An \emph{observation relation}, $\downarrow_{\mathcal N}$, over a set
of names, $\mathcal N$, is the smallest relation satisfying the rules
below.

\infrule[Out-barb]{y \in {\mathcal N}, \; x \nameeq y}
		  {\outputp{x}{v} \downarrow_{\mathcal N} x}
\infrule[Par-barb]{\mbox{$P\downarrow_{\mathcal N} x$ or $Q\downarrow_{\mathcal N} x$}}
		  {\binpar{P}{Q} \downarrow_{\mathcal N} x}

We write $P \Downarrow_{\mathcal N} x$ if there is $Q$ such that 
$P \wred Q$ and $Q \downarrow_{\mathcal N} x$.
\end{definition}

\begin{definition}
%\label{def.bbisim}
An  ${\mathcal N}$-\emph{barbed bisimulation} over a set of names, ${\mathcal N}$, is a symmetric binary relation 
${\mathcal S}_{\mathcal N}$ between agents such that $P\rel{S}_{\mathcal N}Q$ implies:
\begin{enumerate}
\item If $P \red P'$ then $Q \wred Q'$ and $P'\rel{S}_{\mathcal N} Q'$.
\item If $P\downarrow_{\mathcal N} x$, then $Q\Downarrow_{\mathcal N} x$.
\end{enumerate}
$P$ is ${\mathcal N}$-barbed bisimilar to $Q$, written
$P \wbbisim_{\mathcal N} Q$, if $P \rel{S}_{\mathcal N} Q$ for some ${\mathcal N}$-barbed bisimulation ${\mathcal S}_{\mathcal N}$.
\end{definition}

$\mathcal{R} \subseteq \pi \times \pi$

$P \mathcal{R} Q => \forall P'. P \red P' \Rightarrow \exists Q'. Q \red Q', P' \mathcal{R} Q'$

$P \vdash x \Rightarrow Q \vdash x$

\begin{mathpar}
  \inferrule*[lab=Out-barb]{x \nameeq y}{{y}!\langle{Q}\rangle \vdash x}
  \and
  \inferrule*[lab=Par-barb]{\mbox{$P\vdash x$ or $Q\vdash x$}}{\binpar{P}{Q} \vdash x}
\end{mathpar}

\subsubsection{Contexts}

One of the principle advantages of computational calculi like the
$\pi$-calculus is a well-defined notion of context,
contextual-equivalence and a correlation between
contextual-equivalence and notions of bisimulation. The notion of
context allows the decomposition of a process into (sub-)process and
its syntactic environment, its context. Thus, a context may be
thought of as a process with a ``hole'' (written $\Box$) in it. The
application of a context $M$ to a process $P$, written $M[P]$, is
tantamount to filling the hole in $M$ with $P$. In this paper we do
not need the full weight of this theory, but do make use of the notion
of context in the proof the main theorem. 

\begin{mathpar}
  \inferrule* [lab=summation] {} {{M_{M},M_{N}} \bc \Box \;|\; x.M_{A} \;|\; M_{M}+M_{N}}
  \and
  \inferrule* [lab=agent] {} {{M_{A}} \bc (\vec{x})M_{P} \;| \; \clift{P_0,\ldots,M_{P},\ldots,P_N}}
  \and \\
  \inferrule* [lab=process] {} {{M_{P}} \bc M_{N} \;| \;P|M_{P} }
\end{mathpar} 

\begin{mathpar}
  \inferrule* [lab=sychronization] {} {M_{N} \bc \Box \;|\; x?M_{F} \;|\; x!M_{C}}
  \and
  \inferrule* [lab=abstraction] {} {{M_{F}} \bc (x)M_{P} }
  \and
  \inferrule* [lab=concretion] {} {{M_{C}} \bc \langle M_{P} \rangle }
  \and \\
  \inferrule* [lab=process] {} {{M_{P}} \bc M_{N} \;| \;P|M_{P} }
\end{mathpar}

\begin{definition}[contextual application] Given a context $M$, and
  process $P$, we define the \emph{contextual application}, $M[P] :=
  M\{P/\Box\}$. That is, the contextual application of M to P is the
  substitution of $P$ for $\Box$ in $M$.
\end{definition}

$\meaningof{-} : L \to \mathcal{P}(\pi)$

\begin{mathpar}
  \inferrule* [lab=collection] {} {\meaningof{true} = \pi, \and \meaningof{~E} = \pi \setminus \meaningof{E}, \and \meaningof{E_{1} \& E_{2}} = \meaningof{E_{1}} \cap \meaningof{E_{2}}}
\end{mathpar}

\begin{mathpar}
  \inferrule* [lab=structure] {} {\meaningof{0} = \{ P \in \pi | P \equiv 0 \}, \and \\ \meaningof{E_1 | E_2} = \{ P \in \pi | P \equiv P_{1} | P_{2}, P_{1} \in \meaningof{E_{1}}, P_{2} \in \meaningof{E_2}\} }
\end{mathpar}

\begin{mathpar}
 \inferrule* [lab=behavior] {} {\meaningof{\langle a?b \rangle E} = \{ P \in \pi | P \equiv Q | u?(y)P', \\ \and \\\\ \and \\ \;\;\; u \in \meaningof{a}, \forall z.P'\{z/y\} \in \meaningof{E\{z/b\}}\}, \and \\ \meaningof{a!E} = \{ P \in \pi | P \equiv Q | x!\langle P' \rangle, x \in \meaningof{a} P' \in \meaningof{E}\} }
\end{mathpar}

\begin{mathpar}
 \inferrule* [lab=nominal] {} {\meaningof{\quotep{E}} = \{ \quotep{P} \in \quotep{\pi} | P \in \meaningof{E} \}, \and \meaningof{\quotep{P}} = \{ \quotep{Q} \in \quotep{\pi} | P \equiv Q \} \and \\ \meaningof{@\quotep{E}} = \{ P \in \pi | P \equiv @x, x \in \meaningof{E} \}}
\end{mathpar}

\begin{eqnarray*}
  \\
  \meaningof{-} : TS \to ST
\end{eqnarray*}

\begin{eqnarray*}
  \\
  L : TS \to ST
\end{eqnarray*}

\begin{eqnarray*}
  \\
  P \models E \iff P \in \meaningof{E}
\end{eqnarray*}

\begin{eqnarray*}
  P \approx_{L} Q \iff \forall E \in L. P \models E \iff Q \models E
\end{eqnarray*}

\begin{eqnarray*}
  P \approx_{K} Q
\end{eqnarray*}

\begin{eqnarray*}
  P \approx Q
\end{eqnarray*}

$\approx_{K} = \approx = \approx_{L}$

\subsubsection{Contextual duality}

Note that contexts extend the quotation operation to a family of
operations from processes to names. Given a context, $M$, we can
define a \emph{nominal context}, $\quotep{M}$ by $\quotep{M}[P] :=
\quotep{M[P]}$. To foreshadow what is to come we observe that these
operations enjoy a duality with processes very much like the duality
between vectors and maps from vectors to scalars.

Further, because the calculus is essentially higher-order, we have a
correspondence between contexts and processes. More specifically,
given a name $x$ and a context $M$ we can construct $M^{*}_{x}$ such
that 

\begin{mathpar}
  M^{*}_{x} | \lift{x}{P} \red M[P]
\end{mathpar}

namely,

\begin{mathpar}
  M^{*}_{x} := x?(u).M[\dropn{u}]
\end{mathpar}

The dependence of $M^{*}_{x}$ on a name makes it an abstraction, 

\begin{mathpar}
  M^{*} := (x)x?(u).M[\dropn{u}]
\end{mathpar}

\subsection{Additional notation}

It will sometimes be convenient to denote the process a name
quotes. We already have the notation $x = \quotep{P}$, but it will be
convenient to introduce an alternate notation, $\procn{x}$, when we
want to emphasize the connection to the use of the name. Note that, by
virtue of name equivalence, $\quotep{\procn{x}} \nameeq x$; so, the
notation is consistent with previous definitions.

Further, because names have structure it is possible to effect
substitutions on the basis of that structure. This means we need to
upgrade our notation for substitutions, which we accomplish by
adapting comprehension notation. Thus,

\begin{mathpar}
  P\{ y / x : x \in S \}
\end{mathpar}

is interpreted to mean the process derived from P by replacing (in a
capture-avoiding manner) each occurrence of $x$ in $S$ by $y$. For example,

\begin{mathpar}
  P\{ \quotep{\procn{x}|\procn{x}} / x : x \in \freenames{P} \}
\end{mathpar}

will replace each (occurrence) of a free name $x$ in $P$ by
$\quotep{\procn{x}|\procn{x}}$.

Also, we will avail ourselves of the notation $x^{L}$ and $x^{R}$ to
denote injections of a name into disjoint copies of the name
space. There are numerous ways to accomplish this. One example can be
found in \cite{MeredithR05}. This notation overloads to vectors of
names: $\vec{x}^{\pi} := (x_{i}^{\pi} \; : \; 0 \leq i < |\vec{x}| )$ where $\pi \in \{L,R\}$.

We also use $P^{\Box} := P|\Box$.

In \cite{MeredithR05} an interpretation of the new operator is
given. It turns out that there are several possible interpretations
all enjoying the requisite algebraic properties of the operator (see
\cite{milner91polyadicpi}). We will therefore make liberal use of
$(\nu\; \vec{x})P$.

% subsection the_syntax_and_semantics_of_the_notation_system (end)   

\input{qm2pi.qmops} 

\input{qm2pi.sterngerlach} 

\input{qm2pi.metric} 

% section concurrent_process_calculi (end)

%\input{qm2pi.proofsketch}

% section proof sketch (end)

%\input{qm2pi.slviaknots} 

% section spatial logic via knots (end)

\input{qm2pi.conclusion}

% section conclusion (end)

%\input{qm2pi.dtcodes} 

% section wiring algorithm (end)

\input{qm2pi.ack} 

% section acknowledgments (end)

\newpage


\bibliographystyle{plain}   
\bibliography{../../biblios/main.bib}

\input{qm2pi.rhodetails}

\end{document}

 

%\documentclass[12pt]{llncs}
%\documentclass{jktr}

\usepackage[pdftex]{hyperref}                   
\usepackage {listings}
\usepackage {mathpartir}
\usepackage{bcprules}
%\usepackage{listings}
                       
\usepackage{graphicx} 
%\usepackage[margins=2.5cm,nohead,nofoot]{geometry}
%\usepackage{geometry}
\usepackage{amsfonts}
\usepackage{amstext}
\usepackage{latexsym}
\usepackage{amssymb}
\usepackage{color}


%\include{myPreamble}
\include{qm2pi.local} 

%\ifpdf
%\usepackage[pdftex]{graphicx}
%\else
%\usepackage{graphicx}
%\fi

 % \ifpdf
%  \usepackage{pdfsync}
%  \if


%\title{Brief Article}
%\author{David F. Snyder}
%\author{L.G. Meredith}

%\address{Dept. of Math., Texas State University--San Marcos, San Marcos, TX 78666}
       
\pagestyle{empty}


\begin{document}

\lstset{language=[Objective]Caml,frame=shadowbox}

\input{qm2pi.front}

% section front matter (end)

\input{qm2pi.intro} 
 
% section introduction (end)

% \input{qm2pi.knotations} 

% section notation (end)

\input{qm2pi.process.calculi} 

% section concurrent_process_calculi_and_spatial_logics_ (end)
    
%\input{qm2pi.knots2pi} 

%\input{qm2pi.trefoil} 

%\input{qm2pi.mainthm} 

% subsection basic_interpretation (end)

%\input{qm2pi.rho.presentation} 
\subsection{The syntax and semantics of the notation system}\label{sub:the_syntax_and_semantics_of_the_notation_system} % (fold)

We now summarize a technical presentation of the calculus that
embodies our theory of dynamics. The typical presentation of such a
calculus follows the style of giving generators and relations on
them. The grammar, below, describing term constructors, freely
generates the set of processes, $\Proc$. This set is then quotiented
by a relation known as structural congruence and it is over this set
that the notion of dynamics is expressed. This presentation is
essentially that of \cite{MeredithR05} with the addition of
polyadicity and summation. For readability we have relegated some of
the technical subtleties to an appendix.

\subsubsection{Process grammar}\label{subsub:process_grammar}

\begin{mathpar}
  \inferrule* [lab=synchronization] {} {{M} \bc \pzero \;|\; x?F \;|\; x!C }
  \and
  \inferrule* [lab=abstraction] {} {{F} \bc (x)P}
  \and
  \inferrule* [lab=concretion] {} {{C} \bc \langle Q \rangle}
  \and
  \inferrule* [lab=process] {} {{P,Q} \bc M \;| \;P|Q \;|\; @{x}}
  \and
  \inferrule* [lab=name] {} {{x} \bc \quotep{P}}
\end{mathpar} 

Note that $\vec{x}$ (resp. $\vec{P}$) denotes a vector of names
(resp. processes) of length $|\vec{x}|$ (resp. $|\vec{P}|$). We adopt
the following useful abbreviations.

\begin{mathpar}
   x?(\vec{y}).P := x.(\vec{y})P \and  x\clift{\vec{P}} := x.\clift{\vec{P}}
   \and x!(y) := \lift{x}{\dropn{y}}
   \and \Pi_{i=0}^{n-1}P_i := P_0 | \ldots | P_{n-1}
\end{mathpar}

\subsubsection{Structural congruence}

\paragraph{Free and bound names and alpha-equivalence.} At the
core of structural equivalence is alpha-equivalence which identifies
process that are the same up to a change of variable. Formally, we
recognize the distinction between free and bound names. The free names
of a process, $\freenames{P}$, may be calculated recursively as
follows:

\begin{mathpar}
\freenames{\pzero} := \emptyset
  \and \\
  \freenames{x?(y).P} := \{ x \} \cup (\freenames{P} \setminus \{ y \})
  \and 
  \freenames{x!\langle P \rangle} := \{ x \} \cup \{ P \} 
  \and \\
  \freenames{P|Q} := \freenames{P} \cup \freenames{Q}
  \and \\
  \freenames{@{x}} := \{ x \}
\end{mathpar}

$\pi$
$\quotep{\pi}$

$\freenames{-} : \pi \to \mathcal{P}(\quotep{\pi})$

\begin{eqnarray*}
  \freenames{\pzero} & := & \emptyset \\
  \freenames{x?(y).P} & := & \{ x \} \cup (\freenames{P} \setminus \{ y \}) \\
  \freenames{x!\langle P \rangle} & := & \{ x \} \cup \{ P \} \\
  \freenames{P|Q} & := & \freenames{P} \cup \freenames{Q} \\
  \freenames{\dropn{x}} & := & \{ x \}
\end{eqnarray*}

The bound names of a process, $\boundnames{P}$, are those names occurring in $P$
that are not free. For example, in $x?(y).0$, the name $x$ is free, while $y$ is bound.

\begin{mathpar}
  \inferrule* [lab=monoidal-laws] {} { P|Q \equiv Q|P \and P|0 \equiv P \and P|(Q|R) \equiv (P|Q)|R }
\end{mathpar}

\begin{mathpar}
  \inferrule* [lab=alpha-equivalence] {} { (x)P \equiv (y)P\{y/x\} \and y \not\in \freenames{P} }
\end{mathpar}

\begin{definition}
Then two processes, $P,Q$, are alpha-equivalent if $P = Q\{\vec{y}/\vec{x}\}$ for
some $\vec{x} \in \boundnames{Q},\vec{y} \in \boundnames{P}$, where $Q\{\vec{y}/\vec{x}\}$
denotes the capture-avoiding substitution of $\vec{y}$ for $\vec{x}$ in $Q$.
\end{definition}

\begin{definition}
  The {\em structural congruence} \cite{SangiorgiWalker} , $\equiv$,
  between processes is the least congruence containing
  alpha-equivalence, satisfying the abelian monoid laws
  (associativity, commutativity and $\pzero$ as identity) for parallel
  composition $|$ and for summation $+$.
\end{definition}

\subsection{Name equivalence}

We take name equivalence, written $\nameeq$, to be the smallest
equivalence relation generated by the following rules.

\begin{mathpar}
\inferrule*[lab=Quote-drop]
{ }
{ \quotep{@{x}} \nameeq x }

\inferrule*[lab=Struct-equiv]
{ P \scong Q }
{ \quotep{P} \nameeq \quotep{Q} }
\end{mathpar}

The astute reader will have noticed that the mutual recursion of names
and processes imposes a mutual recursion on alpha-equivalence and
structural equivalence via name-equivalence. Fortunately, all of this
works out pleasantly and we may calculate in the natural way, free of
concern. The reader interested in the details is referred to the
appendix \ref{appendix:rho_details}.

\subsection{Substitution}

We use $\Proc$ for the set of processes, $\QProc$ for the set of
names, and $\id{\{}\vec{y} / \vec{x} \id{\}}$ to denote partial maps,
$s : \QProc \rightarrow \QProc$. A map, $s$ lifts, uniquely, to a map
on process terms, $\widehat{s} : \Proc \rightarrow \Proc$ by the
following equations.

\begin{mathpar}
  (0) \psubstp{Q}{P} := 0 \\
  (R \juxtap S) \psubstp{Q}{P}
  :=    
  (R)\psubstp{Q}{P} \juxtap (S) \psubstp{Q}{P} \\
  (x?(y).R) \psubstp{Q}{P}    
  :=    
  (x)\substp{Q}{P} (z)\concat( (R \psubstn{z}{y}) \psubstp{Q}{P} ) \\
  (\lift{x}{R}) \psubstp{Q}{P}  
  :=
  \lift{(x)\substp{Q}{P}}{ R \psubstp{Q}{P} } \\
%   (\dropn{x})  \psubstp{Q}{P}       
%   := 
%   \left\{ 
%     \begin{array}{ccc} 
%       \dropn{\quotep{Q}} & & x \nameeq \quotep{P} \\
%       \dropn{x} & & otherwise \\
%     \end{array}
%   \right. 
  (\dropn{x})  \psubstp{Q}{P}       
  := 
  \left\{ 
    \begin{array}{ccc} 
      Q & & x \nameeq \quotep{P} \\
      \dropn{x} & & otherwise \\
    \end{array}
  \right.
\end{mathpar}
 

where

\begin{eqnarray}
  (x)\id{\{} \lpquote Q \rpquote / \lpquote P \rpquote \id{\}}            = 
  \left\{ 
    \begin{array}{ccc}
      \lpquote Q \rpquote & & x \nameeq \lpquote P \rpquote \\
      x & & otherwise \\
    \end{array}
  \right. \nonumber
\end{eqnarray}

and $z$ is chosen distinct from $\quotep{P}$, $\quotep{Q}$, the free
names in $Q$, and all the names in $R$. Our $\alpha$-equivalence will
be built in the standard way from this substitution.

\begin{remark}\label{rem:no_self_referential_names}
  One consequence of these definitions is that $\forall P. \quotep{P}
  \not\in \freenames{P}$.
\end{remark}

\subsection{ Dynamic quote: an example }

Anticipating something of what's to come, consider applying the
substitution, $\widehat{\id{\{}u / z \id{\}}}$, to the following pair
of processes, $\lift{w}{y!(z)}$ and $w[ \lpquote y!(z) \rpquote ]$.

\begin{eqnarray}
	\lift{w}{y!(z)}\widehat{\id{\{}u / z \id{\}}}
		& = &
		\lift{w}{y!(u)} \nonumber\\
	w[ \lpquote y!(z) \rpquote ] \widehat{ \id{\{}u / z \id{\}} }
		& = &
		w[ \lpquote y!(z) \rpquote ] \nonumber
\end{eqnarray}

Because the body of the process between quotes is impervious to
substitution, we get radically different answers. In fact, by
examining the first process in an input context,
e.g. $x?(z).\lift{w}{y!(z)}$, we see that the process under the lift
operator may be shaped by prefixed inputs binding a name inside it. In
this sense, the lift operator will be seen as a way to dynamically
construct processes before reifying them as names.

Finally equipped with these standard features we can present the
dynamics of the calculus.

\subsubsection{Operational semantics} 

Finally, we introduce the computational dynamics. What marks these
algebras as distinct from other more traditionally studied algebraic
structures, e.g. vector spaces or polynomial rings, is the manner in
which dynamics is captured. In traditional structures, dynamics is typically
expressed through morphisms between such structures, as in linear maps
between vector spaces or morphisms between rings. In algebras
associated with the semantics of computation, the dynamics is
expressed as part of the algebraic structure itself, through a
reduction reduction relation typically denoted by $\red$. Below, we
give a recursive presentation of this relation for the calculus used
in the encoding.

$\red \subseteq \pi \times \pi$
$\red : \pi \to \mathcal{P}(\pi)$

\begin{mathpar}
  \inferrule* [lab=Comm] { \textsf{match}( x_{src}, x_{trgt} ) } { x_{trgt}?(y)P \; | \; x_{src}!\langle {Q} \rangle \red P\{\quotep{Q}/y}\} }
  \and \\
  \inferrule* [lab=Par] {{P} \red {P}'} {{{P} | {Q}} \red {{P}' | {Q}}}
  \and
  \inferrule* [lab=Equiv]{{{P} \scong {P}'} \andalso {{P}' \red {Q}'} \andalso {{Q}' \scong {Q}}}{{P} \red {Q}}
\end{mathpar}

\begin{eqnarray*}
  match_{\equiv} (\quotep{P},\quotep{Q}) & := & P \equiv Q \\
  match_{\dagger}(\quotep{P},\quotep{Q}) & := & \forall R. P|Q \red^{*} R => R \red^{*} 0 \\
  match_{K}(\quotep{P},\quotep{Q}) & := & K \mbox{ for some context } K
\end{eqnarray*}

$u?(x)P | u!\langle Q \rangle \red P\{\quotep{Q}/x\}$

%We write $\wred$ for $\red^*$, and $P\red$ if $\exists Q $ such that $ P \red Q$.
We write $P\red$ if $\exists Q $ such that $ P \red Q$ and $P\not\red$, otherwise.

\section{Replication}

As mentioned before, it is known that replication (and hence
recursion) can be implemented in a higher-order process algebra
\cite{SangiorgiWalker}. As our first example of calculation with the
machinery thus far presented we give the construction explicitly in
the {\rhoc}.

\begin{eqnarray}
	D_{x} & := & \prefix{x}{y}{(\binpar{\outputp{x}{y}}{@{y}})} \nonumber\\
	\bangp_{x}{P} & := & \binpar{{x}!\langle{\binpar{D_{x}}{P}}\rangle}{D_{x}} \nonumber
\end{eqnarray}

\begin{eqnarray}
	\bangp_{x}{P} & & \nonumber\\
	=
	& {x}!\langle{(\prefix{x}{y}{(\outputp{x}{y} | @{y})) | P}}\rangle 
	      | \prefix{x}{y}{(\outputp{x}{y} | @{y})} & \nonumber\\
	\red
	& (\outputp{x}{y} | @{y})\substn{\quotep{(\prefix{x}{y}{(@{y} | \outputp{x}{y})) | P}}}{y} & \nonumber\\
	=
	& \outputp{x}{\quotep{(\prefix{x}{y}{(\outputp{x}{y} | @{y})) | P}}}
	  | {(\prefix{x}{y}{(\outputp{x}{y} | @{y})) | P}} & \nonumber\\
	\red
	& \ldots & \nonumber\\
	\red^*
	& P | P | \ldots & \nonumber
\end{eqnarray}

Of course, this encoding, as an implementation, runs away, unfolding
$\bangp{P}$ eagerly. A lazier and more implementable replication
operator, restricted to input-guarded processes, may be obtained as follows.

\begin{eqnarray}
\bangp{\prefix{u}{v}{P}} 
	:= 
	\binpar{\lift{x}{\prefix{u}{v}{(\binpar{D(x)}{P})}}}{D(x)} \nonumber
\end{eqnarray}

\begin{remark}
  Note that the lazier definition still does not deal with summation
  or mixed summation (i.e. sums over input and output). The reader is
  invited to construct definitions of replication that deal with these
  features. 

  Further, the definitions are parameterized in a name, $x$. Can you,
  gentle reader, make a definition that eliminates this parameter and
  guarantees no accidental interaction between the replication
  machinery and the process being replicated -- i.e. no accidental
  sharing of names used by the process to get its work done and the
  name(s) used by the replication to effect copying. This latter
  revision of the definition of replication is crucial to obtaining
  the expected identity $!!P \sim !P$.
\end{remark}

\begin{remark}\label{rem:paradoxical_combinator}
  The reader familiar with the lambda calculus will have noticed the
  similarity between $D$ and the paradoxical combinator.

  [Ed. note: the existence of this seems to suggest we have to be more
  restrictive on the set of processes and names we admit if we are to
  support no-cloning.]
\end{remark}

\subsubsection{Bisimulation}

The computational dynamics gives rise to another kind of equivalence,
the equivalence of computational behavior. As previously mentioned
this is typically captured \emph{via} some form of bisimulation.

% The notion we use in this paper is weak barbed bisimulation
% \cite{milner91polyadicpi}.

The notion we use in this paper is derived from weak barbed
bisimulation \cite{milner91polyadicpi}. 

\begin{definition}
An \emph{observation relation}, $\downarrow_{\mathcal N}$, over a set
of names, $\mathcal N$, is the smallest relation satisfying the rules
below.

\infrule[Out-barb]{y \in {\mathcal N}, \; x \nameeq y}
		  {\outputp{x}{v} \downarrow_{\mathcal N} x}
\infrule[Par-barb]{\mbox{$P\downarrow_{\mathcal N} x$ or $Q\downarrow_{\mathcal N} x$}}
		  {\binpar{P}{Q} \downarrow_{\mathcal N} x}

We write $P \Downarrow_{\mathcal N} x$ if there is $Q$ such that 
$P \wred Q$ and $Q \downarrow_{\mathcal N} x$.
\end{definition}

\begin{definition}
%\label{def.bbisim}
An  ${\mathcal N}$-\emph{barbed bisimulation} over a set of names, ${\mathcal N}$, is a symmetric binary relation 
${\mathcal S}_{\mathcal N}$ between agents such that $P\rel{S}_{\mathcal N}Q$ implies:
\begin{enumerate}
\item If $P \red P'$ then $Q \wred Q'$ and $P'\rel{S}_{\mathcal N} Q'$.
\item If $P\downarrow_{\mathcal N} x$, then $Q\Downarrow_{\mathcal N} x$.
\end{enumerate}
$P$ is ${\mathcal N}$-barbed bisimilar to $Q$, written
$P \wbbisim_{\mathcal N} Q$, if $P \rel{S}_{\mathcal N} Q$ for some ${\mathcal N}$-barbed bisimulation ${\mathcal S}_{\mathcal N}$.
\end{definition}

$\mathcal{R} \subseteq \pi \times \pi$

$P \mathcal{R} Q => \forall P'. P \red P' \Rightarrow \exists Q'. Q \red Q', P' \mathcal{R} Q'$

$P \vdash x \Rightarrow Q \vdash x$

\begin{mathpar}
  \inferrule*[lab=Out-barb]{x \nameeq y}{{y}!\langle{Q}\rangle \vdash x}
  \and
  \inferrule*[lab=Par-barb]{\mbox{$P\vdash x$ or $Q\vdash x$}}{\binpar{P}{Q} \vdash x}
\end{mathpar}

\subsubsection{Contexts}

One of the principle advantages of computational calculi like the
$\pi$-calculus is a well-defined notion of context,
contextual-equivalence and a correlation between
contextual-equivalence and notions of bisimulation. The notion of
context allows the decomposition of a process into (sub-)process and
its syntactic environment, its context. Thus, a context may be
thought of as a process with a ``hole'' (written $\Box$) in it. The
application of a context $M$ to a process $P$, written $M[P]$, is
tantamount to filling the hole in $M$ with $P$. In this paper we do
not need the full weight of this theory, but do make use of the notion
of context in the proof the main theorem. 

\begin{mathpar}
  \inferrule* [lab=summation] {} {{M_{M},M_{N}} \bc \Box \;|\; x.M_{A} \;|\; M_{M}+M_{N}}
  \and
  \inferrule* [lab=agent] {} {{M_{A}} \bc (\vec{x})M_{P} \;| \; \clift{P_0,\ldots,M_{P},\ldots,P_N}}
  \and \\
  \inferrule* [lab=process] {} {{M_{P}} \bc M_{N} \;| \;P|M_{P} }
\end{mathpar} 

\begin{mathpar}
  \inferrule* [lab=sychronization] {} {M_{N} \bc \Box \;|\; x?M_{F} \;|\; x!M_{C}}
  \and
  \inferrule* [lab=abstraction] {} {{M_{F}} \bc (x)M_{P} }
  \and
  \inferrule* [lab=concretion] {} {{M_{C}} \bc \langle M_{P} \rangle }
  \and \\
  \inferrule* [lab=process] {} {{M_{P}} \bc M_{N} \;| \;P|M_{P} }
\end{mathpar}

\begin{definition}[contextual application] Given a context $M$, and
  process $P$, we define the \emph{contextual application}, $M[P] :=
  M\{P/\Box\}$. That is, the contextual application of M to P is the
  substitution of $P$ for $\Box$ in $M$.
\end{definition}

$\meaningof{-} : L \to \mathcal{P}(\pi)$

\begin{mathpar}
  \inferrule* [lab=collection] {} {\meaningof{true} = \pi, \and \meaningof{~E} = \pi \setminus \meaningof{E}, \and \meaningof{E_{1} \& E_{2}} = \meaningof{E_{1}} \cap \meaningof{E_{2}}}
\end{mathpar}

\begin{mathpar}
  \inferrule* [lab=structure] {} {\meaningof{0} = \{ P \in \pi | P \equiv 0 \}, \and \\ \meaningof{E_1 | E_2} = \{ P \in \pi | P \equiv P_{1} | P_{2}, P_{1} \in \meaningof{E_{1}}, P_{2} \in \meaningof{E_2}\} }
\end{mathpar}

\begin{mathpar}
 \inferrule* [lab=behavior] {} {\meaningof{\langle a?b \rangle E} = \{ P \in \pi | P \equiv Q | u?(y)P', \\ \and \\\\ \and \\ \;\;\; u \in \meaningof{a}, \forall z.P'\{z/y\} \in \meaningof{E\{z/b\}}\}, \and \\ \meaningof{a!E} = \{ P \in \pi | P \equiv Q | x!\langle P' \rangle, x \in \meaningof{a} P' \in \meaningof{E}\} }
\end{mathpar}

\begin{mathpar}
 \inferrule* [lab=nominal] {} {\meaningof{\quotep{E}} = \{ \quotep{P} \in \quotep{\pi} | P \in \meaningof{E} \}, \and \meaningof{\quotep{P}} = \{ \quotep{Q} \in \quotep{\pi} | P \equiv Q \} \and \\ \meaningof{@\quotep{E}} = \{ P \in \pi | P \equiv @x, x \in \meaningof{E} \}}
\end{mathpar}

\begin{eqnarray*}
  \\
  \meaningof{-} : TS \to ST
\end{eqnarray*}

\begin{eqnarray*}
  \\
  L : TS \to ST
\end{eqnarray*}

\begin{eqnarray*}
  \\
  P \models E \iff P \in \meaningof{E}
\end{eqnarray*}

\begin{eqnarray*}
  P \approx_{L} Q \iff \forall E \in L. P \models E \iff Q \models E
\end{eqnarray*}

\begin{eqnarray*}
  P \approx_{K} Q
\end{eqnarray*}

\begin{eqnarray*}
  P \approx Q
\end{eqnarray*}

$\approx_{K} = \approx = \approx_{L}$

\subsubsection{Contextual duality}

Note that contexts extend the quotation operation to a family of
operations from processes to names. Given a context, $M$, we can
define a \emph{nominal context}, $\quotep{M}$ by $\quotep{M}[P] :=
\quotep{M[P]}$. To foreshadow what is to come we observe that these
operations enjoy a duality with processes very much like the duality
between vectors and maps from vectors to scalars.

Further, because the calculus is essentially higher-order, we have a
correspondence between contexts and processes. More specifically,
given a name $x$ and a context $M$ we can construct $M^{*}_{x}$ such
that 

\begin{mathpar}
  M^{*}_{x} | \lift{x}{P} \red M[P]
\end{mathpar}

namely,

\begin{mathpar}
  M^{*}_{x} := x?(u).M[\dropn{u}]
\end{mathpar}

The dependence of $M^{*}_{x}$ on a name makes it an abstraction, 

\begin{mathpar}
  M^{*} := (x)x?(u).M[\dropn{u}]
\end{mathpar}

\subsection{Additional notation}

It will sometimes be convenient to denote the process a name
quotes. We already have the notation $x = \quotep{P}$, but it will be
convenient to introduce an alternate notation, $\procn{x}$, when we
want to emphasize the connection to the use of the name. Note that, by
virtue of name equivalence, $\quotep{\procn{x}} \nameeq x$; so, the
notation is consistent with previous definitions.

Further, because names have structure it is possible to effect
substitutions on the basis of that structure. This means we need to
upgrade our notation for substitutions, which we accomplish by
adapting comprehension notation. Thus,

\begin{mathpar}
  P\{ y / x : x \in S \}
\end{mathpar}

is interpreted to mean the process derived from P by replacing (in a
capture-avoiding manner) each occurrence of $x$ in $S$ by $y$. For example,

\begin{mathpar}
  P\{ \quotep{\procn{x}|\procn{x}} / x : x \in \freenames{P} \}
\end{mathpar}

will replace each (occurrence) of a free name $x$ in $P$ by
$\quotep{\procn{x}|\procn{x}}$.

Also, we will avail ourselves of the notation $x^{L}$ and $x^{R}$ to
denote injections of a name into disjoint copies of the name
space. There are numerous ways to accomplish this. One example can be
found in \cite{MeredithR05}. This notation overloads to vectors of
names: $\vec{x}^{\pi} := (x_{i}^{\pi} \; : \; 0 \leq i < |\vec{x}| )$ where $\pi \in \{L,R\}$.

We also use $P^{\Box} := P|\Box$.

In \cite{MeredithR05} an interpretation of the new operator is
given. It turns out that there are several possible interpretations
all enjoying the requisite algebraic properties of the operator (see
\cite{milner91polyadicpi}). We will therefore make liberal use of
$(\nu\; \vec{x})P$.

% subsection the_syntax_and_semantics_of_the_notation_system (end)   

\input{qm2pi.qmops} 

\input{qm2pi.sterngerlach} 

\input{qm2pi.metric} 

% section concurrent_process_calculi (end)

%\input{qm2pi.proofsketch}

% section proof sketch (end)

%\input{qm2pi.slviaknots} 

% section spatial logic via knots (end)

\input{qm2pi.conclusion}

% section conclusion (end)

%\input{qm2pi.dtcodes} 

% section wiring algorithm (end)

\input{qm2pi.ack} 

% section acknowledgments (end)

\newpage


\bibliographystyle{plain}   
\bibliography{../../biblios/main.bib}

\input{qm2pi.rhodetails}

\end{document}

 

% subsection basic_interpretation (end)

%\input{qm2pi.rho.presentation} 
\subsection{The syntax and semantics of the notation system}\label{sub:the_syntax_and_semantics_of_the_notation_system} % (fold)

We now summarize a technical presentation of the calculus that
embodies our theory of dynamics. The typical presentation of such a
calculus follows the style of giving generators and relations on
them. The grammar, below, describing term constructors, freely
generates the set of processes, $\Proc$. This set is then quotiented
by a relation known as structural congruence and it is over this set
that the notion of dynamics is expressed. This presentation is
essentially that of \cite{MeredithR05} with the addition of
polyadicity and summation. For readability we have relegated some of
the technical subtleties to an appendix.

\subsubsection{Process grammar}\label{subsub:process_grammar}

\begin{mathpar}
  \inferrule* [lab=synchronization] {} {{M} \bc \pzero \;|\; x?F \;|\; x!C }
  \and
  \inferrule* [lab=abstraction] {} {{F} \bc (x)P}
  \and
  \inferrule* [lab=concretion] {} {{C} \bc \langle Q \rangle}
  \and
  \inferrule* [lab=process] {} {{P,Q} \bc M \;| \;P|Q \;|\; @{x}}
  \and
  \inferrule* [lab=name] {} {{x} \bc \quotep{P}}
\end{mathpar} 

Note that $\vec{x}$ (resp. $\vec{P}$) denotes a vector of names
(resp. processes) of length $|\vec{x}|$ (resp. $|\vec{P}|$). We adopt
the following useful abbreviations.

\begin{mathpar}
   x?(\vec{y}).P := x.(\vec{y})P \and  x\clift{\vec{P}} := x.\clift{\vec{P}}
   \and x!(y) := \lift{x}{\dropn{y}}
   \and \Pi_{i=0}^{n-1}P_i := P_0 | \ldots | P_{n-1}
\end{mathpar}

\subsubsection{Structural congruence}

\paragraph{Free and bound names and alpha-equivalence.} At the
core of structural equivalence is alpha-equivalence which identifies
process that are the same up to a change of variable. Formally, we
recognize the distinction between free and bound names. The free names
of a process, $\freenames{P}$, may be calculated recursively as
follows:

\begin{mathpar}
\freenames{\pzero} := \emptyset
  \and \\
  \freenames{x?(y).P} := \{ x \} \cup (\freenames{P} \setminus \{ y \})
  \and 
  \freenames{x!\langle P \rangle} := \{ x \} \cup \{ P \} 
  \and \\
  \freenames{P|Q} := \freenames{P} \cup \freenames{Q}
  \and \\
  \freenames{@{x}} := \{ x \}
\end{mathpar}

$\pi$
$\quotep{\pi}$

$\freenames{-} : \pi \to \mathcal{P}(\quotep{\pi})$

\begin{eqnarray*}
  \freenames{\pzero} & := & \emptyset \\
  \freenames{x?(y).P} & := & \{ x \} \cup (\freenames{P} \setminus \{ y \}) \\
  \freenames{x!\langle P \rangle} & := & \{ x \} \cup \{ P \} \\
  \freenames{P|Q} & := & \freenames{P} \cup \freenames{Q} \\
  \freenames{\dropn{x}} & := & \{ x \}
\end{eqnarray*}

The bound names of a process, $\boundnames{P}$, are those names occurring in $P$
that are not free. For example, in $x?(y).0$, the name $x$ is free, while $y$ is bound.

\begin{mathpar}
  \inferrule* [lab=monoidal-laws] {} { P|Q \equiv Q|P \and P|0 \equiv P \and P|(Q|R) \equiv (P|Q)|R }
\end{mathpar}

\begin{mathpar}
  \inferrule* [lab=alpha-equivalence] {} { (x)P \equiv (y)P\{y/x\} \and y \not\in \freenames{P} }
\end{mathpar}

\begin{definition}
Then two processes, $P,Q$, are alpha-equivalent if $P = Q\{\vec{y}/\vec{x}\}$ for
some $\vec{x} \in \boundnames{Q},\vec{y} \in \boundnames{P}$, where $Q\{\vec{y}/\vec{x}\}$
denotes the capture-avoiding substitution of $\vec{y}$ for $\vec{x}$ in $Q$.
\end{definition}

\begin{definition}
  The {\em structural congruence} \cite{SangiorgiWalker} , $\equiv$,
  between processes is the least congruence containing
  alpha-equivalence, satisfying the abelian monoid laws
  (associativity, commutativity and $\pzero$ as identity) for parallel
  composition $|$ and for summation $+$.
\end{definition}

\subsection{Name equivalence}

We take name equivalence, written $\nameeq$, to be the smallest
equivalence relation generated by the following rules.

\begin{mathpar}
\inferrule*[lab=Quote-drop]
{ }
{ \quotep{@{x}} \nameeq x }

\inferrule*[lab=Struct-equiv]
{ P \scong Q }
{ \quotep{P} \nameeq \quotep{Q} }
\end{mathpar}

The astute reader will have noticed that the mutual recursion of names
and processes imposes a mutual recursion on alpha-equivalence and
structural equivalence via name-equivalence. Fortunately, all of this
works out pleasantly and we may calculate in the natural way, free of
concern. The reader interested in the details is referred to the
appendix \ref{appendix:rho_details}.

\subsection{Substitution}

We use $\Proc$ for the set of processes, $\QProc$ for the set of
names, and $\id{\{}\vec{y} / \vec{x} \id{\}}$ to denote partial maps,
$s : \QProc \rightarrow \QProc$. A map, $s$ lifts, uniquely, to a map
on process terms, $\widehat{s} : \Proc \rightarrow \Proc$ by the
following equations.

\begin{mathpar}
  (0) \psubstp{Q}{P} := 0 \\
  (R \juxtap S) \psubstp{Q}{P}
  :=    
  (R)\psubstp{Q}{P} \juxtap (S) \psubstp{Q}{P} \\
  (x?(y).R) \psubstp{Q}{P}    
  :=    
  (x)\substp{Q}{P} (z)\concat( (R \psubstn{z}{y}) \psubstp{Q}{P} ) \\
  (\lift{x}{R}) \psubstp{Q}{P}  
  :=
  \lift{(x)\substp{Q}{P}}{ R \psubstp{Q}{P} } \\
%   (\dropn{x})  \psubstp{Q}{P}       
%   := 
%   \left\{ 
%     \begin{array}{ccc} 
%       \dropn{\quotep{Q}} & & x \nameeq \quotep{P} \\
%       \dropn{x} & & otherwise \\
%     \end{array}
%   \right. 
  (\dropn{x})  \psubstp{Q}{P}       
  := 
  \left\{ 
    \begin{array}{ccc} 
      Q & & x \nameeq \quotep{P} \\
      \dropn{x} & & otherwise \\
    \end{array}
  \right.
\end{mathpar}
 

where

\begin{eqnarray}
  (x)\id{\{} \lpquote Q \rpquote / \lpquote P \rpquote \id{\}}            = 
  \left\{ 
    \begin{array}{ccc}
      \lpquote Q \rpquote & & x \nameeq \lpquote P \rpquote \\
      x & & otherwise \\
    \end{array}
  \right. \nonumber
\end{eqnarray}

and $z$ is chosen distinct from $\quotep{P}$, $\quotep{Q}$, the free
names in $Q$, and all the names in $R$. Our $\alpha$-equivalence will
be built in the standard way from this substitution.

\begin{remark}\label{rem:no_self_referential_names}
  One consequence of these definitions is that $\forall P. \quotep{P}
  \not\in \freenames{P}$.
\end{remark}

\subsection{ Dynamic quote: an example }

Anticipating something of what's to come, consider applying the
substitution, $\widehat{\id{\{}u / z \id{\}}}$, to the following pair
of processes, $\lift{w}{y!(z)}$ and $w[ \lpquote y!(z) \rpquote ]$.

\begin{eqnarray}
	\lift{w}{y!(z)}\widehat{\id{\{}u / z \id{\}}}
		& = &
		\lift{w}{y!(u)} \nonumber\\
	w[ \lpquote y!(z) \rpquote ] \widehat{ \id{\{}u / z \id{\}} }
		& = &
		w[ \lpquote y!(z) \rpquote ] \nonumber
\end{eqnarray}

Because the body of the process between quotes is impervious to
substitution, we get radically different answers. In fact, by
examining the first process in an input context,
e.g. $x?(z).\lift{w}{y!(z)}$, we see that the process under the lift
operator may be shaped by prefixed inputs binding a name inside it. In
this sense, the lift operator will be seen as a way to dynamically
construct processes before reifying them as names.

Finally equipped with these standard features we can present the
dynamics of the calculus.

\subsubsection{Operational semantics} 

Finally, we introduce the computational dynamics. What marks these
algebras as distinct from other more traditionally studied algebraic
structures, e.g. vector spaces or polynomial rings, is the manner in
which dynamics is captured. In traditional structures, dynamics is typically
expressed through morphisms between such structures, as in linear maps
between vector spaces or morphisms between rings. In algebras
associated with the semantics of computation, the dynamics is
expressed as part of the algebraic structure itself, through a
reduction reduction relation typically denoted by $\red$. Below, we
give a recursive presentation of this relation for the calculus used
in the encoding.

$\red \subseteq \pi \times \pi$
$\red : \pi \to \mathcal{P}(\pi)$

\begin{mathpar}
  \inferrule* [lab=Comm] { \textsf{match}( x_{src}, x_{trgt} ) } { x_{trgt}?(y)P \; | \; x_{src}!\langle {Q} \rangle \red P\{\quotep{Q}/y}\} }
  \and \\
  \inferrule* [lab=Par] {{P} \red {P}'} {{{P} | {Q}} \red {{P}' | {Q}}}
  \and
  \inferrule* [lab=Equiv]{{{P} \scong {P}'} \andalso {{P}' \red {Q}'} \andalso {{Q}' \scong {Q}}}{{P} \red {Q}}
\end{mathpar}

\begin{eqnarray*}
  match_{\equiv} (\quotep{P},\quotep{Q}) & := & P \equiv Q \\
  match_{\dagger}(\quotep{P},\quotep{Q}) & := & \forall R. P|Q \red^{*} R => R \red^{*} 0 \\
  match_{K}(\quotep{P},\quotep{Q}) & := & K \mbox{ for some context } K
\end{eqnarray*}

$u?(x)P | u!\langle Q \rangle \red P\{\quotep{Q}/x\}$

%We write $\wred$ for $\red^*$, and $P\red$ if $\exists Q $ such that $ P \red Q$.
We write $P\red$ if $\exists Q $ such that $ P \red Q$ and $P\not\red$, otherwise.

\section{Replication}

As mentioned before, it is known that replication (and hence
recursion) can be implemented in a higher-order process algebra
\cite{SangiorgiWalker}. As our first example of calculation with the
machinery thus far presented we give the construction explicitly in
the {\rhoc}.

\begin{eqnarray}
	D_{x} & := & \prefix{x}{y}{(\binpar{\outputp{x}{y}}{@{y}})} \nonumber\\
	\bangp_{x}{P} & := & \binpar{{x}!\langle{\binpar{D_{x}}{P}}\rangle}{D_{x}} \nonumber
\end{eqnarray}

\begin{eqnarray}
	\bangp_{x}{P} & & \nonumber\\
	=
	& {x}!\langle{(\prefix{x}{y}{(\outputp{x}{y} | @{y})) | P}}\rangle 
	      | \prefix{x}{y}{(\outputp{x}{y} | @{y})} & \nonumber\\
	\red
	& (\outputp{x}{y} | @{y})\substn{\quotep{(\prefix{x}{y}{(@{y} | \outputp{x}{y})) | P}}}{y} & \nonumber\\
	=
	& \outputp{x}{\quotep{(\prefix{x}{y}{(\outputp{x}{y} | @{y})) | P}}}
	  | {(\prefix{x}{y}{(\outputp{x}{y} | @{y})) | P}} & \nonumber\\
	\red
	& \ldots & \nonumber\\
	\red^*
	& P | P | \ldots & \nonumber
\end{eqnarray}

Of course, this encoding, as an implementation, runs away, unfolding
$\bangp{P}$ eagerly. A lazier and more implementable replication
operator, restricted to input-guarded processes, may be obtained as follows.

\begin{eqnarray}
\bangp{\prefix{u}{v}{P}} 
	:= 
	\binpar{\lift{x}{\prefix{u}{v}{(\binpar{D(x)}{P})}}}{D(x)} \nonumber
\end{eqnarray}

\begin{remark}
  Note that the lazier definition still does not deal with summation
  or mixed summation (i.e. sums over input and output). The reader is
  invited to construct definitions of replication that deal with these
  features. 

  Further, the definitions are parameterized in a name, $x$. Can you,
  gentle reader, make a definition that eliminates this parameter and
  guarantees no accidental interaction between the replication
  machinery and the process being replicated -- i.e. no accidental
  sharing of names used by the process to get its work done and the
  name(s) used by the replication to effect copying. This latter
  revision of the definition of replication is crucial to obtaining
  the expected identity $!!P \sim !P$.
\end{remark}

\begin{remark}\label{rem:paradoxical_combinator}
  The reader familiar with the lambda calculus will have noticed the
  similarity between $D$ and the paradoxical combinator.

  [Ed. note: the existence of this seems to suggest we have to be more
  restrictive on the set of processes and names we admit if we are to
  support no-cloning.]
\end{remark}

\subsubsection{Bisimulation}

The computational dynamics gives rise to another kind of equivalence,
the equivalence of computational behavior. As previously mentioned
this is typically captured \emph{via} some form of bisimulation.

% The notion we use in this paper is weak barbed bisimulation
% \cite{milner91polyadicpi}.

The notion we use in this paper is derived from weak barbed
bisimulation \cite{milner91polyadicpi}. 

\begin{definition}
An \emph{observation relation}, $\downarrow_{\mathcal N}$, over a set
of names, $\mathcal N$, is the smallest relation satisfying the rules
below.

\infrule[Out-barb]{y \in {\mathcal N}, \; x \nameeq y}
		  {\outputp{x}{v} \downarrow_{\mathcal N} x}
\infrule[Par-barb]{\mbox{$P\downarrow_{\mathcal N} x$ or $Q\downarrow_{\mathcal N} x$}}
		  {\binpar{P}{Q} \downarrow_{\mathcal N} x}

We write $P \Downarrow_{\mathcal N} x$ if there is $Q$ such that 
$P \wred Q$ and $Q \downarrow_{\mathcal N} x$.
\end{definition}

\begin{definition}
%\label{def.bbisim}
An  ${\mathcal N}$-\emph{barbed bisimulation} over a set of names, ${\mathcal N}$, is a symmetric binary relation 
${\mathcal S}_{\mathcal N}$ between agents such that $P\rel{S}_{\mathcal N}Q$ implies:
\begin{enumerate}
\item If $P \red P'$ then $Q \wred Q'$ and $P'\rel{S}_{\mathcal N} Q'$.
\item If $P\downarrow_{\mathcal N} x$, then $Q\Downarrow_{\mathcal N} x$.
\end{enumerate}
$P$ is ${\mathcal N}$-barbed bisimilar to $Q$, written
$P \wbbisim_{\mathcal N} Q$, if $P \rel{S}_{\mathcal N} Q$ for some ${\mathcal N}$-barbed bisimulation ${\mathcal S}_{\mathcal N}$.
\end{definition}

$\mathcal{R} \subseteq \pi \times \pi$

$P \mathcal{R} Q => \forall P'. P \red P' \Rightarrow \exists Q'. Q \red Q', P' \mathcal{R} Q'$

$P \vdash x \Rightarrow Q \vdash x$

\begin{mathpar}
  \inferrule*[lab=Out-barb]{x \nameeq y}{{y}!\langle{Q}\rangle \vdash x}
  \and
  \inferrule*[lab=Par-barb]{\mbox{$P\vdash x$ or $Q\vdash x$}}{\binpar{P}{Q} \vdash x}
\end{mathpar}

\subsubsection{Contexts}

One of the principle advantages of computational calculi like the
$\pi$-calculus is a well-defined notion of context,
contextual-equivalence and a correlation between
contextual-equivalence and notions of bisimulation. The notion of
context allows the decomposition of a process into (sub-)process and
its syntactic environment, its context. Thus, a context may be
thought of as a process with a ``hole'' (written $\Box$) in it. The
application of a context $M$ to a process $P$, written $M[P]$, is
tantamount to filling the hole in $M$ with $P$. In this paper we do
not need the full weight of this theory, but do make use of the notion
of context in the proof the main theorem. 

\begin{mathpar}
  \inferrule* [lab=summation] {} {{M_{M},M_{N}} \bc \Box \;|\; x.M_{A} \;|\; M_{M}+M_{N}}
  \and
  \inferrule* [lab=agent] {} {{M_{A}} \bc (\vec{x})M_{P} \;| \; \clift{P_0,\ldots,M_{P},\ldots,P_N}}
  \and \\
  \inferrule* [lab=process] {} {{M_{P}} \bc M_{N} \;| \;P|M_{P} }
\end{mathpar} 

\begin{mathpar}
  \inferrule* [lab=sychronization] {} {M_{N} \bc \Box \;|\; x?M_{F} \;|\; x!M_{C}}
  \and
  \inferrule* [lab=abstraction] {} {{M_{F}} \bc (x)M_{P} }
  \and
  \inferrule* [lab=concretion] {} {{M_{C}} \bc \langle M_{P} \rangle }
  \and \\
  \inferrule* [lab=process] {} {{M_{P}} \bc M_{N} \;| \;P|M_{P} }
\end{mathpar}

\begin{definition}[contextual application] Given a context $M$, and
  process $P$, we define the \emph{contextual application}, $M[P] :=
  M\{P/\Box\}$. That is, the contextual application of M to P is the
  substitution of $P$ for $\Box$ in $M$.
\end{definition}

$\meaningof{-} : L \to \mathcal{P}(\pi)$

\begin{mathpar}
  \inferrule* [lab=collection] {} {\meaningof{true} = \pi, \and \meaningof{~E} = \pi \setminus \meaningof{E}, \and \meaningof{E_{1} \& E_{2}} = \meaningof{E_{1}} \cap \meaningof{E_{2}}}
\end{mathpar}

\begin{mathpar}
  \inferrule* [lab=structure] {} {\meaningof{0} = \{ P \in \pi | P \equiv 0 \}, \and \\ \meaningof{E_1 | E_2} = \{ P \in \pi | P \equiv P_{1} | P_{2}, P_{1} \in \meaningof{E_{1}}, P_{2} \in \meaningof{E_2}\} }
\end{mathpar}

\begin{mathpar}
 \inferrule* [lab=behavior] {} {\meaningof{\langle a?b \rangle E} = \{ P \in \pi | P \equiv Q | u?(y)P', \\ \and \\\\ \and \\ \;\;\; u \in \meaningof{a}, \forall z.P'\{z/y\} \in \meaningof{E\{z/b\}}\}, \and \\ \meaningof{a!E} = \{ P \in \pi | P \equiv Q | x!\langle P' \rangle, x \in \meaningof{a} P' \in \meaningof{E}\} }
\end{mathpar}

\begin{mathpar}
 \inferrule* [lab=nominal] {} {\meaningof{\quotep{E}} = \{ \quotep{P} \in \quotep{\pi} | P \in \meaningof{E} \}, \and \meaningof{\quotep{P}} = \{ \quotep{Q} \in \quotep{\pi} | P \equiv Q \} \and \\ \meaningof{@\quotep{E}} = \{ P \in \pi | P \equiv @x, x \in \meaningof{E} \}}
\end{mathpar}

\begin{eqnarray*}
  \\
  \meaningof{-} : TS \to ST
\end{eqnarray*}

\begin{eqnarray*}
  \\
  L : TS \to ST
\end{eqnarray*}

\begin{eqnarray*}
  \\
  P \models E \iff P \in \meaningof{E}
\end{eqnarray*}

\begin{eqnarray*}
  P \approx_{L} Q \iff \forall E \in L. P \models E \iff Q \models E
\end{eqnarray*}

\begin{eqnarray*}
  P \approx_{K} Q
\end{eqnarray*}

\begin{eqnarray*}
  P \approx Q
\end{eqnarray*}

$\approx_{K} = \approx = \approx_{L}$

\subsubsection{Contextual duality}

Note that contexts extend the quotation operation to a family of
operations from processes to names. Given a context, $M$, we can
define a \emph{nominal context}, $\quotep{M}$ by $\quotep{M}[P] :=
\quotep{M[P]}$. To foreshadow what is to come we observe that these
operations enjoy a duality with processes very much like the duality
between vectors and maps from vectors to scalars.

Further, because the calculus is essentially higher-order, we have a
correspondence between contexts and processes. More specifically,
given a name $x$ and a context $M$ we can construct $M^{*}_{x}$ such
that 

\begin{mathpar}
  M^{*}_{x} | \lift{x}{P} \red M[P]
\end{mathpar}

namely,

\begin{mathpar}
  M^{*}_{x} := x?(u).M[\dropn{u}]
\end{mathpar}

The dependence of $M^{*}_{x}$ on a name makes it an abstraction, 

\begin{mathpar}
  M^{*} := (x)x?(u).M[\dropn{u}]
\end{mathpar}

\subsection{Additional notation}

It will sometimes be convenient to denote the process a name
quotes. We already have the notation $x = \quotep{P}$, but it will be
convenient to introduce an alternate notation, $\procn{x}$, when we
want to emphasize the connection to the use of the name. Note that, by
virtue of name equivalence, $\quotep{\procn{x}} \nameeq x$; so, the
notation is consistent with previous definitions.

Further, because names have structure it is possible to effect
substitutions on the basis of that structure. This means we need to
upgrade our notation for substitutions, which we accomplish by
adapting comprehension notation. Thus,

\begin{mathpar}
  P\{ y / x : x \in S \}
\end{mathpar}

is interpreted to mean the process derived from P by replacing (in a
capture-avoiding manner) each occurrence of $x$ in $S$ by $y$. For example,

\begin{mathpar}
  P\{ \quotep{\procn{x}|\procn{x}} / x : x \in \freenames{P} \}
\end{mathpar}

will replace each (occurrence) of a free name $x$ in $P$ by
$\quotep{\procn{x}|\procn{x}}$.

Also, we will avail ourselves of the notation $x^{L}$ and $x^{R}$ to
denote injections of a name into disjoint copies of the name
space. There are numerous ways to accomplish this. One example can be
found in \cite{MeredithR05}. This notation overloads to vectors of
names: $\vec{x}^{\pi} := (x_{i}^{\pi} \; : \; 0 \leq i < |\vec{x}| )$ where $\pi \in \{L,R\}$.

We also use $P^{\Box} := P|\Box$.

In \cite{MeredithR05} an interpretation of the new operator is
given. It turns out that there are several possible interpretations
all enjoying the requisite algebraic properties of the operator (see
\cite{milner91polyadicpi}). We will therefore make liberal use of
$(\nu\; \vec{x})P$.

% subsection the_syntax_and_semantics_of_the_notation_system (end)   

\section{Interpretation of QM}
\subsection{Supporting definitions}
\subsubsection{Multiplication}
\begin{mathpar}
  \quotep{Q} \cdot \quotep{R} := \quotep{Q|R}
  \and \\
  \quotep{Q} \cdot P := P\{ \quotep{Q|R} / \quotep{R} : \quotep{R} \in \freenames{P} \}
\end{mathpar}

\paragraph{Discussion}
The first line needs little explanation. The second line says that
each free name of the process is replaced with the multiplication of
that name by the scalar. Multiplication of a scalar (name) by a state
(process) results in a process all the names of which have been `moved
over' by parallel composition with the process the scalar
quotes. There is a subtlety that the bound names have to be
manipulated so that multiplied names aren't accidentally
captured. There are many ways to achieve this.

\begin{remark}\label{rem:multiplication_identities}
  The reader is invited to verify that for all $x,y,z \in \QProc$ and $P \in \Proc$
  \begin{mathpar}
    x \cdot \quotep{0} \equiv x 
    \and
    x \cdot y \equiv y \cdot x
    \and
    x \cdot (y \cdot z) \equiv (x \cdot y) \cdot z
    \and \\
    \quotep{0} \cdot P \equiv P
    \and \\
    x \cdot (y \cdot P) \equiv (x \cdot y) \cdot P
    \and \\
    x \cdot (P|Q) \equiv (x \cdot P) | (x \cdot Q)
    \and \\    
  \end{mathpar}
\end{remark}

\subsubsection{Tensor product}

We define a tensor product on processes by structural induction.

\paragraph{Tensor of sums} First note that all summations, including
$\pzero$ and sequence, can be written $\Sigma_{i} x_{i}.A_{i} +
\Sigma_{j} x_{j}.C_{j}$, where we have grouped input-guarded processes
together and output-guarded processes together.

Thus, we can define the tensor product of two summations, $N_{1}\otimes N_{2}$, where

\begin{mathpar}
  N_{1} := \Sigma_{i} x_{i}.A_{i} + \Sigma_{j} x_{j}.C_{j}
  \and
  N_{2} := \Sigma_{i'} y_{i'}.B_{i'} + \Sigma_{j'} y_{j'}.D_{j'} 
\end{mathpar}

as follows.

\begin{mathpar}
  \Sigma_{i} x_{i}.A_{i} + \Sigma_{j} x_{j}.C_{j} \otimes \Sigma_{i'}
  y_{i'}.B_{i'} + \Sigma_{j'} y_{j'}.D_{j'} 
  \and \\
  := \; \Sigma_{i} \Sigma_{i'} \quotep{\stackrel{\vee}{x_{i}}| \stackrel{\vee}{y_{i'}}}.(A_{i}\otimes B_{i'}) \; | \; \Sigma_{i'} \Sigma_{i} \quotep{\stackrel{\vee}{y_{i'}}|\stackrel{\vee}{x_{i}}}.(B_{i'}\otimes A_{i})
  \and
  \;\; | \;\; \Sigma_{j} \Sigma_{j'} \quotep{\stackrel{\vee}{x_{j}}|\stackrel{\vee}{y_{j'}}}.(A_{j}\otimes B_{j'}) \; | \; \Sigma_{j'} \Sigma_{j} \quotep{\stackrel{\vee}{y_{j'}}|\stackrel{\vee}{x_{j}}}.(B_{j'}\otimes A_{j})
\end{mathpar}

\begin{remark}
  Do we need to $x^{L}$ and $y^{R}$ for this construction as well?
\end{remark}

\paragraph{Tensor of parallel compositions} Next, we distribute tensor
over par.

\begin{mathpar}
  P_{1}|P_{2} \otimes Q_{1}|Q_{2} := (P_{1} \otimes Q_{1}) | (P_{1}
  \otimes Q_{2}) | (P_{2} \otimes Q_{1}) | (P_{2} \otimes Q_{2})
\end{mathpar}

\paragraph{Tensor with dropped names} We treat tensor of a
process with a dropped name as parallel composition.

\begin{mathpar}
  P \otimes \dropn{x} := P | \dropn{x}
\end{mathpar}

\paragraph{Tensor of agents}

Finally, we need to define tensor on agents. Note that the definition
of tensor on normal products only tensors inputs with inputs and
outputs with outputs. Thus, we only have to define the operation on
``homogeneous'' pairings.

\begin{mathpar}
  (\vec{x})P \otimes (\vec{y})Q
  \and \\
  := (x_{0}^{L}|y_{0}^{R},\ldots,x_{0}^{L}|y_{n}^{R},\ldots,x_{m}^{L}|y_{0}^{R},\ldots,x_{m}^{L}|y_{n}^R)(P\{ \vec{x}^{L}/\vec{x}\} \otimes Q \{ \vec{y}^{R}/\vec{y}\})
  \and \\
  \clift{\vec{P}} \otimes \clift{\vec{Q}}
  \and \\
  := \clift{P_{0}\otimes Q_{0},\ldots,P_{0}\otimes Q_{n},\ldots,P_{m}\otimes Q_{0},\ldots,P_{m}\otimes Q_{n}}
\end{mathpar}

\begin{remark}
  Observe that arities of tensored abstractions matches arities of
  tensored concretions if the original arities matched. Note also that
  the length of the arities corresponds to the increase in dimension
  we see in ordinary vector space tensor product.
\end{remark}

\begin{remark}
  Operationally, this definition distributes the tensor down to
  components ``linked'' by summation. Tensor over summation is
  intriguing in that it mixes names. Moreover, as a consequence of the
  way it mixes names we have the identities for all $x \in \QProc$ and
  $P,Q \in \Proc$

  \begin{mathpar}
    (x \cdot P) \otimes Q \equiv x \cdot (P \otimes Q) \equiv P \otimes (x \cdot Q)
    \and
    P \otimes \pzero \equiv P
  \end{mathpar}

  that the reader is invited to verify.
\end{remark}

\subsubsection{Annihilation}
\begin{mathpar}
  P^{\perp} := \{ Q | \forall R. P|Q \red^{*} R \Rightarrow R \red^{*} \pzero \}
  \and \\
  P^{\underline{\perp}} := \Sigma_{Q \in P^{\perp}} \quotep{Q}?(y).(\dropn{y}|Q) | \Sigma_{Q \in P^{\perp}} \quotep{Q}\clift{\Box}
\end{mathpar}

\paragraph{Discussion} The reader will note that $P^{\perp}$ is a
\emph{set} of processes, while $P^{\underline{\perp}}$ is a
\emph{context}. We call the set $P^{\perp}$ the \emph{annihilators} of
$P$. The parallel composition of a process in the annihilators of $P$
with $P$ will result in a process, the state space of which has all
paths eventually leading to $\pzero$. Execution may endure loops; but
under reasonable conditions of fairness (naturally guaranteed under
most notions of bisimulation) such a composite process cannot get
stuck in such a loop and will, eventually pop out and terminate.

The context $P^{\underline{\perp}}$ is ready and willing to ``take the
$P$ out of'' the process to which it is applied. It will effectively
transmit the code of the process to which it is applied to one of the
annihilators and run the process against it.

\subsubsection{Evaluation}
We fix $M$ a domain of fully abstract interpretation with an equality
coincident with bisimulation. We take $\meaningof{\cdot} : \Proc \to
M$ to be the map interpreting processes and $\nmeaningof{\cdot} : \M
\to Proc$ to be the map running the other way. Then we define

\begin{mathpar}
  \int P := \nmeaningof{\meaningof{P}}
\end{mathpar}

\paragraph{Discussion}
There are many fully abstract interpretations of Milner's
$\pi$-calculus. Any of them can be used as a basis for interpreting
the reflective calculus here. Equipped with such a domain it is
largely a matter of grinding through to check that the Yoneda
construction for the normalization-by-evaluation program can be
extended to this setting.

\begin{remark}
  The reader is invited to verify that $\int (P^{\underline{\perp}}[P]) = 0$.
\end{remark}

\subsection{Quantum mechanics}

Table \ref{tbl:core_qm_op_defns} gives the core operational definitions

\begin{table}[htp]\label{tbl:core_qm_op_defns}
  \center{
    \fbox{
      \begin{tabular}{c|c}
        quantum mechanics & process calculus \\
        \hline
        scalar & $x := \quotep{P}$ \\
        state vector & $\state{P} := P$ \\
        dual & $\state{P}^{*} := \event{P^{\underline{\perp}}} := \quotep{P^{\underline{\perp}}}[-]$ \\
        matrix & $ \Sigma_{\alpha} \state{P_{\alpha}}x_{\alpha}\event{Q_{\alpha}}$ \\
        vector addition & $\state{P} + \state{Q} := \state{P | Q}$ \\
        tensor product & $\state{P} \otimes \state{Q} := \state{P \otimes Q}$ \\
        inner product & $\innerprod{P}{Q} := \quotep{\int P^{\underline{\perp}}[Q]}$ \\
      \end{tabular}
    }
  }
  \caption{QM - operational definitions}
\end{table}

where

\begin{mathpar}
  \prmatrix{P}{Q} := \fprmatrix{P}{\quotep{\pzero}}{Q}
  \and
  \fprmatrix{P}{x}{Q} := (\state{P},x,\event{Q})
  \and
  (\fprmatrix{P}{x}{Q})(\state{R}) := x \cdot \innerprod{Q}{R} \cdot \state{P}
  \and
  (\fprmatrix{P}{x}{Q})(\event{R}) := x \cdot \innerprod{R}{P} \cdot \event{Q}
\end{mathpar}

\paragraph{Discussion}
As promised: vectors (aka states) are represented as processes; duals
as contextual duals; inner product definition should be compared with
standard inner product definition for ....

\begin{remark}
  Assuming $\int (P^{\underline{\perp}}[P]) = 0$, the reader is
  invited to verify that $(\fprmatrix{P}{x}{P})(\state{P}) = x \cdot \state{P}$.
\end{remark}

\begin{remark}
  The reader is invited to verify that $\innerprod{P}{Q}$ could
  equally well have been written $\quotep{\int \stackrel{\vee}{x}}$
  where $x = \event{P^{\underline{\perp}}}(Q)$.

  One of the motivations for this remark is that there is another way
  to factor these operations. We could package up evaluation in the dual:

  \begin{mathpar}
    \state{P}^{*} := \event{\int P^{\underline{\perp}}} := \quotep{\int P^{\underline{\perp}}}[-]
  \end{mathpar}

  and then have inner product defined by
  
  \begin{mathpar}
    \innerprod{P}{Q} := \event{P}(Q)
  \end{mathpar}

  Hopefully, experience with the calculations will provide guidance on
  the best factoring.
\end{remark}

\begin{remark}
  Assuming $\int (P^{\underline{\perp}}[P]) = 0$, the reader is
  invited to verify that $\forall P,Q. (\prmatrix{0}{Q})(\state{0}) =
  \state{0}$ and dually $(\prmatrix{P}{0})(\event{0}) = \event{0}$.
\end{remark}

\begin{remark}
  i'm a little worried that i don't (yet) have proper support for
  complex conjugacy. But, the observation above may give us a
  clue. According to Abramsky, it must be the case that the scalars
  are iso to the homset of the identity for the tensor -- which the
  observation above characterizes. 

  For now, we will simply bookmark the notion with $\overline{x}$.
\end{remark}

\subsubsection{Adjointness}

We need to give a definition of $(\cdot)^{\dagger}$ for matrices. The
obvious candidate definition is
\begin{mathpar}
(\Sigma_{\alpha}\fprmatrix{P_{\alpha}}{x_{\alpha}}{Q_{\alpha}})^{\dagger}
= \Sigma_{\alpha}\fprmatrix{(Q_{\alpha}^{\underline{\perp}})^{*}}{\overline{x}_{\alpha}}{P_{\alpha}^{\underline{\perp}}} 
\end{mathpar}

But, $(Q_{\alpha}^{\underline{\perp}})^{*}$ requires a name along
which to communicate the process to achieve the context application.

\subsubsection{Basis for a basis}
If processes label states and ``addition'' of states (a.k.a. vector
addition) is interpreted as parallel composition, what corresponds to
notions of linear independence and basis? Here, we recall that Yoshida
has developed a set of \emph{combinators} for an asynchronous verison
of Milner's $\pi$-calculus. These are a finite set of processes such
any process can be expressed as parallel composition of these
combinators together with liberal uses of the new operator and
replication. We can simply give a translation of these into the
present calculus and have reasonable expectation that the property
carries over. That is, that the resultant set allows to express all
processes via parallel composition. Note, however, that there is no
new operator or replication in this calculus. As a result, we expect
that the corresponding set is actually infinite. That is, we expect
that the space is actually infinite dimensional.

\begin{remark}
  The attentive reader may be a bit concerned. Certainly, the
  collection $S$, $K$ and $I$ is a finite set of
  combinators. Shouldn't we expect to see a finite set of combinators
  for an effectively equivalent system? i am very sympathetic to this
  critique and feel it warrants full attention. On the other hand, i
  also have in mind the following analogy. The natural numbers, as a
  monoid under addition, has exactly $1$ generator, while the natural
  numbers, as a monoid under multiplication, has countably many
  generators (the primes). We observe that the application of the
  lambda calculus is much less resource sensitive than the parallel
  composition of the $\pi$-calculus. Could it be the case that we have
  an analogy of the form
  
  \begin{mathpar}
    m + n : MN :: m*n : M|N
  \end{mathpar}

  giving a similar blow up in the set of ``primes''?  This is such a
  wonderful thought that, even if it's not true, i think it's worth
  writing down.
\end{remark}
 

\documentclass[12pt]{llncs}
%\documentclass{jktr}

\usepackage[pdftex]{hyperref}                   
\usepackage {listings}
\usepackage {mathpartir}
\usepackage{bcprules}
%\usepackage{listings}
                       
\usepackage{graphicx} 
%\usepackage[margins=2.5cm,nohead,nofoot]{geometry}
%\usepackage{geometry}
\usepackage{amsfonts}
\usepackage{amstext}
\usepackage{latexsym}
\usepackage{amssymb}
\usepackage{color}


%\include{myPreamble}
\include{qm2pi.local} 

%\ifpdf
%\usepackage[pdftex]{graphicx}
%\else
%\usepackage{graphicx}
%\fi

 % \ifpdf
%  \usepackage{pdfsync}
%  \if


%\title{Brief Article}
%\author{David F. Snyder}
%\author{L.G. Meredith}

%\address{Dept. of Math., Texas State University--San Marcos, San Marcos, TX 78666}
       
\pagestyle{empty}


\begin{document}

\lstset{language=[Objective]Caml,frame=shadowbox}

\input{qm2pi.front}

% section front matter (end)

\input{qm2pi.intro} 
 
% section introduction (end)

% \input{qm2pi.knotations} 

% section notation (end)

\input{qm2pi.process.calculi} 

% section concurrent_process_calculi_and_spatial_logics_ (end)
    
%\input{qm2pi.knots2pi} 

%\input{qm2pi.trefoil} 

%\input{qm2pi.mainthm} 

% subsection basic_interpretation (end)

%\input{qm2pi.rho.presentation} 
\subsection{The syntax and semantics of the notation system}\label{sub:the_syntax_and_semantics_of_the_notation_system} % (fold)

We now summarize a technical presentation of the calculus that
embodies our theory of dynamics. The typical presentation of such a
calculus follows the style of giving generators and relations on
them. The grammar, below, describing term constructors, freely
generates the set of processes, $\Proc$. This set is then quotiented
by a relation known as structural congruence and it is over this set
that the notion of dynamics is expressed. This presentation is
essentially that of \cite{MeredithR05} with the addition of
polyadicity and summation. For readability we have relegated some of
the technical subtleties to an appendix.

\subsubsection{Process grammar}\label{subsub:process_grammar}

\begin{mathpar}
  \inferrule* [lab=synchronization] {} {{M} \bc \pzero \;|\; x?F \;|\; x!C }
  \and
  \inferrule* [lab=abstraction] {} {{F} \bc (x)P}
  \and
  \inferrule* [lab=concretion] {} {{C} \bc \langle Q \rangle}
  \and
  \inferrule* [lab=process] {} {{P,Q} \bc M \;| \;P|Q \;|\; @{x}}
  \and
  \inferrule* [lab=name] {} {{x} \bc \quotep{P}}
\end{mathpar} 

Note that $\vec{x}$ (resp. $\vec{P}$) denotes a vector of names
(resp. processes) of length $|\vec{x}|$ (resp. $|\vec{P}|$). We adopt
the following useful abbreviations.

\begin{mathpar}
   x?(\vec{y}).P := x.(\vec{y})P \and  x\clift{\vec{P}} := x.\clift{\vec{P}}
   \and x!(y) := \lift{x}{\dropn{y}}
   \and \Pi_{i=0}^{n-1}P_i := P_0 | \ldots | P_{n-1}
\end{mathpar}

\subsubsection{Structural congruence}

\paragraph{Free and bound names and alpha-equivalence.} At the
core of structural equivalence is alpha-equivalence which identifies
process that are the same up to a change of variable. Formally, we
recognize the distinction between free and bound names. The free names
of a process, $\freenames{P}$, may be calculated recursively as
follows:

\begin{mathpar}
\freenames{\pzero} := \emptyset
  \and \\
  \freenames{x?(y).P} := \{ x \} \cup (\freenames{P} \setminus \{ y \})
  \and 
  \freenames{x!\langle P \rangle} := \{ x \} \cup \{ P \} 
  \and \\
  \freenames{P|Q} := \freenames{P} \cup \freenames{Q}
  \and \\
  \freenames{@{x}} := \{ x \}
\end{mathpar}

$\pi$
$\quotep{\pi}$

$\freenames{-} : \pi \to \mathcal{P}(\quotep{\pi})$

\begin{eqnarray*}
  \freenames{\pzero} & := & \emptyset \\
  \freenames{x?(y).P} & := & \{ x \} \cup (\freenames{P} \setminus \{ y \}) \\
  \freenames{x!\langle P \rangle} & := & \{ x \} \cup \{ P \} \\
  \freenames{P|Q} & := & \freenames{P} \cup \freenames{Q} \\
  \freenames{\dropn{x}} & := & \{ x \}
\end{eqnarray*}

The bound names of a process, $\boundnames{P}$, are those names occurring in $P$
that are not free. For example, in $x?(y).0$, the name $x$ is free, while $y$ is bound.

\begin{mathpar}
  \inferrule* [lab=monoidal-laws] {} { P|Q \equiv Q|P \and P|0 \equiv P \and P|(Q|R) \equiv (P|Q)|R }
\end{mathpar}

\begin{mathpar}
  \inferrule* [lab=alpha-equivalence] {} { (x)P \equiv (y)P\{y/x\} \and y \not\in \freenames{P} }
\end{mathpar}

\begin{definition}
Then two processes, $P,Q$, are alpha-equivalent if $P = Q\{\vec{y}/\vec{x}\}$ for
some $\vec{x} \in \boundnames{Q},\vec{y} \in \boundnames{P}$, where $Q\{\vec{y}/\vec{x}\}$
denotes the capture-avoiding substitution of $\vec{y}$ for $\vec{x}$ in $Q$.
\end{definition}

\begin{definition}
  The {\em structural congruence} \cite{SangiorgiWalker} , $\equiv$,
  between processes is the least congruence containing
  alpha-equivalence, satisfying the abelian monoid laws
  (associativity, commutativity and $\pzero$ as identity) for parallel
  composition $|$ and for summation $+$.
\end{definition}

\subsection{Name equivalence}

We take name equivalence, written $\nameeq$, to be the smallest
equivalence relation generated by the following rules.

\begin{mathpar}
\inferrule*[lab=Quote-drop]
{ }
{ \quotep{@{x}} \nameeq x }

\inferrule*[lab=Struct-equiv]
{ P \scong Q }
{ \quotep{P} \nameeq \quotep{Q} }
\end{mathpar}

The astute reader will have noticed that the mutual recursion of names
and processes imposes a mutual recursion on alpha-equivalence and
structural equivalence via name-equivalence. Fortunately, all of this
works out pleasantly and we may calculate in the natural way, free of
concern. The reader interested in the details is referred to the
appendix \ref{appendix:rho_details}.

\subsection{Substitution}

We use $\Proc$ for the set of processes, $\QProc$ for the set of
names, and $\id{\{}\vec{y} / \vec{x} \id{\}}$ to denote partial maps,
$s : \QProc \rightarrow \QProc$. A map, $s$ lifts, uniquely, to a map
on process terms, $\widehat{s} : \Proc \rightarrow \Proc$ by the
following equations.

\begin{mathpar}
  (0) \psubstp{Q}{P} := 0 \\
  (R \juxtap S) \psubstp{Q}{P}
  :=    
  (R)\psubstp{Q}{P} \juxtap (S) \psubstp{Q}{P} \\
  (x?(y).R) \psubstp{Q}{P}    
  :=    
  (x)\substp{Q}{P} (z)\concat( (R \psubstn{z}{y}) \psubstp{Q}{P} ) \\
  (\lift{x}{R}) \psubstp{Q}{P}  
  :=
  \lift{(x)\substp{Q}{P}}{ R \psubstp{Q}{P} } \\
%   (\dropn{x})  \psubstp{Q}{P}       
%   := 
%   \left\{ 
%     \begin{array}{ccc} 
%       \dropn{\quotep{Q}} & & x \nameeq \quotep{P} \\
%       \dropn{x} & & otherwise \\
%     \end{array}
%   \right. 
  (\dropn{x})  \psubstp{Q}{P}       
  := 
  \left\{ 
    \begin{array}{ccc} 
      Q & & x \nameeq \quotep{P} \\
      \dropn{x} & & otherwise \\
    \end{array}
  \right.
\end{mathpar}
 

where

\begin{eqnarray}
  (x)\id{\{} \lpquote Q \rpquote / \lpquote P \rpquote \id{\}}            = 
  \left\{ 
    \begin{array}{ccc}
      \lpquote Q \rpquote & & x \nameeq \lpquote P \rpquote \\
      x & & otherwise \\
    \end{array}
  \right. \nonumber
\end{eqnarray}

and $z$ is chosen distinct from $\quotep{P}$, $\quotep{Q}$, the free
names in $Q$, and all the names in $R$. Our $\alpha$-equivalence will
be built in the standard way from this substitution.

\begin{remark}\label{rem:no_self_referential_names}
  One consequence of these definitions is that $\forall P. \quotep{P}
  \not\in \freenames{P}$.
\end{remark}

\subsection{ Dynamic quote: an example }

Anticipating something of what's to come, consider applying the
substitution, $\widehat{\id{\{}u / z \id{\}}}$, to the following pair
of processes, $\lift{w}{y!(z)}$ and $w[ \lpquote y!(z) \rpquote ]$.

\begin{eqnarray}
	\lift{w}{y!(z)}\widehat{\id{\{}u / z \id{\}}}
		& = &
		\lift{w}{y!(u)} \nonumber\\
	w[ \lpquote y!(z) \rpquote ] \widehat{ \id{\{}u / z \id{\}} }
		& = &
		w[ \lpquote y!(z) \rpquote ] \nonumber
\end{eqnarray}

Because the body of the process between quotes is impervious to
substitution, we get radically different answers. In fact, by
examining the first process in an input context,
e.g. $x?(z).\lift{w}{y!(z)}$, we see that the process under the lift
operator may be shaped by prefixed inputs binding a name inside it. In
this sense, the lift operator will be seen as a way to dynamically
construct processes before reifying them as names.

Finally equipped with these standard features we can present the
dynamics of the calculus.

\subsubsection{Operational semantics} 

Finally, we introduce the computational dynamics. What marks these
algebras as distinct from other more traditionally studied algebraic
structures, e.g. vector spaces or polynomial rings, is the manner in
which dynamics is captured. In traditional structures, dynamics is typically
expressed through morphisms between such structures, as in linear maps
between vector spaces or morphisms between rings. In algebras
associated with the semantics of computation, the dynamics is
expressed as part of the algebraic structure itself, through a
reduction reduction relation typically denoted by $\red$. Below, we
give a recursive presentation of this relation for the calculus used
in the encoding.

$\red \subseteq \pi \times \pi$
$\red : \pi \to \mathcal{P}(\pi)$

\begin{mathpar}
  \inferrule* [lab=Comm] { \textsf{match}( x_{src}, x_{trgt} ) } { x_{trgt}?(y)P \; | \; x_{src}!\langle {Q} \rangle \red P\{\quotep{Q}/y}\} }
  \and \\
  \inferrule* [lab=Par] {{P} \red {P}'} {{{P} | {Q}} \red {{P}' | {Q}}}
  \and
  \inferrule* [lab=Equiv]{{{P} \scong {P}'} \andalso {{P}' \red {Q}'} \andalso {{Q}' \scong {Q}}}{{P} \red {Q}}
\end{mathpar}

\begin{eqnarray*}
  match_{\equiv} (\quotep{P},\quotep{Q}) & := & P \equiv Q \\
  match_{\dagger}(\quotep{P},\quotep{Q}) & := & \forall R. P|Q \red^{*} R => R \red^{*} 0 \\
  match_{K}(\quotep{P},\quotep{Q}) & := & K \mbox{ for some context } K
\end{eqnarray*}

$u?(x)P | u!\langle Q \rangle \red P\{\quotep{Q}/x\}$

%We write $\wred$ for $\red^*$, and $P\red$ if $\exists Q $ such that $ P \red Q$.
We write $P\red$ if $\exists Q $ such that $ P \red Q$ and $P\not\red$, otherwise.

\section{Replication}

As mentioned before, it is known that replication (and hence
recursion) can be implemented in a higher-order process algebra
\cite{SangiorgiWalker}. As our first example of calculation with the
machinery thus far presented we give the construction explicitly in
the {\rhoc}.

\begin{eqnarray}
	D_{x} & := & \prefix{x}{y}{(\binpar{\outputp{x}{y}}{@{y}})} \nonumber\\
	\bangp_{x}{P} & := & \binpar{{x}!\langle{\binpar{D_{x}}{P}}\rangle}{D_{x}} \nonumber
\end{eqnarray}

\begin{eqnarray}
	\bangp_{x}{P} & & \nonumber\\
	=
	& {x}!\langle{(\prefix{x}{y}{(\outputp{x}{y} | @{y})) | P}}\rangle 
	      | \prefix{x}{y}{(\outputp{x}{y} | @{y})} & \nonumber\\
	\red
	& (\outputp{x}{y} | @{y})\substn{\quotep{(\prefix{x}{y}{(@{y} | \outputp{x}{y})) | P}}}{y} & \nonumber\\
	=
	& \outputp{x}{\quotep{(\prefix{x}{y}{(\outputp{x}{y} | @{y})) | P}}}
	  | {(\prefix{x}{y}{(\outputp{x}{y} | @{y})) | P}} & \nonumber\\
	\red
	& \ldots & \nonumber\\
	\red^*
	& P | P | \ldots & \nonumber
\end{eqnarray}

Of course, this encoding, as an implementation, runs away, unfolding
$\bangp{P}$ eagerly. A lazier and more implementable replication
operator, restricted to input-guarded processes, may be obtained as follows.

\begin{eqnarray}
\bangp{\prefix{u}{v}{P}} 
	:= 
	\binpar{\lift{x}{\prefix{u}{v}{(\binpar{D(x)}{P})}}}{D(x)} \nonumber
\end{eqnarray}

\begin{remark}
  Note that the lazier definition still does not deal with summation
  or mixed summation (i.e. sums over input and output). The reader is
  invited to construct definitions of replication that deal with these
  features. 

  Further, the definitions are parameterized in a name, $x$. Can you,
  gentle reader, make a definition that eliminates this parameter and
  guarantees no accidental interaction between the replication
  machinery and the process being replicated -- i.e. no accidental
  sharing of names used by the process to get its work done and the
  name(s) used by the replication to effect copying. This latter
  revision of the definition of replication is crucial to obtaining
  the expected identity $!!P \sim !P$.
\end{remark}

\begin{remark}\label{rem:paradoxical_combinator}
  The reader familiar with the lambda calculus will have noticed the
  similarity between $D$ and the paradoxical combinator.

  [Ed. note: the existence of this seems to suggest we have to be more
  restrictive on the set of processes and names we admit if we are to
  support no-cloning.]
\end{remark}

\subsubsection{Bisimulation}

The computational dynamics gives rise to another kind of equivalence,
the equivalence of computational behavior. As previously mentioned
this is typically captured \emph{via} some form of bisimulation.

% The notion we use in this paper is weak barbed bisimulation
% \cite{milner91polyadicpi}.

The notion we use in this paper is derived from weak barbed
bisimulation \cite{milner91polyadicpi}. 

\begin{definition}
An \emph{observation relation}, $\downarrow_{\mathcal N}$, over a set
of names, $\mathcal N$, is the smallest relation satisfying the rules
below.

\infrule[Out-barb]{y \in {\mathcal N}, \; x \nameeq y}
		  {\outputp{x}{v} \downarrow_{\mathcal N} x}
\infrule[Par-barb]{\mbox{$P\downarrow_{\mathcal N} x$ or $Q\downarrow_{\mathcal N} x$}}
		  {\binpar{P}{Q} \downarrow_{\mathcal N} x}

We write $P \Downarrow_{\mathcal N} x$ if there is $Q$ such that 
$P \wred Q$ and $Q \downarrow_{\mathcal N} x$.
\end{definition}

\begin{definition}
%\label{def.bbisim}
An  ${\mathcal N}$-\emph{barbed bisimulation} over a set of names, ${\mathcal N}$, is a symmetric binary relation 
${\mathcal S}_{\mathcal N}$ between agents such that $P\rel{S}_{\mathcal N}Q$ implies:
\begin{enumerate}
\item If $P \red P'$ then $Q \wred Q'$ and $P'\rel{S}_{\mathcal N} Q'$.
\item If $P\downarrow_{\mathcal N} x$, then $Q\Downarrow_{\mathcal N} x$.
\end{enumerate}
$P$ is ${\mathcal N}$-barbed bisimilar to $Q$, written
$P \wbbisim_{\mathcal N} Q$, if $P \rel{S}_{\mathcal N} Q$ for some ${\mathcal N}$-barbed bisimulation ${\mathcal S}_{\mathcal N}$.
\end{definition}

$\mathcal{R} \subseteq \pi \times \pi$

$P \mathcal{R} Q => \forall P'. P \red P' \Rightarrow \exists Q'. Q \red Q', P' \mathcal{R} Q'$

$P \vdash x \Rightarrow Q \vdash x$

\begin{mathpar}
  \inferrule*[lab=Out-barb]{x \nameeq y}{{y}!\langle{Q}\rangle \vdash x}
  \and
  \inferrule*[lab=Par-barb]{\mbox{$P\vdash x$ or $Q\vdash x$}}{\binpar{P}{Q} \vdash x}
\end{mathpar}

\subsubsection{Contexts}

One of the principle advantages of computational calculi like the
$\pi$-calculus is a well-defined notion of context,
contextual-equivalence and a correlation between
contextual-equivalence and notions of bisimulation. The notion of
context allows the decomposition of a process into (sub-)process and
its syntactic environment, its context. Thus, a context may be
thought of as a process with a ``hole'' (written $\Box$) in it. The
application of a context $M$ to a process $P$, written $M[P]$, is
tantamount to filling the hole in $M$ with $P$. In this paper we do
not need the full weight of this theory, but do make use of the notion
of context in the proof the main theorem. 

\begin{mathpar}
  \inferrule* [lab=summation] {} {{M_{M},M_{N}} \bc \Box \;|\; x.M_{A} \;|\; M_{M}+M_{N}}
  \and
  \inferrule* [lab=agent] {} {{M_{A}} \bc (\vec{x})M_{P} \;| \; \clift{P_0,\ldots,M_{P},\ldots,P_N}}
  \and \\
  \inferrule* [lab=process] {} {{M_{P}} \bc M_{N} \;| \;P|M_{P} }
\end{mathpar} 

\begin{mathpar}
  \inferrule* [lab=sychronization] {} {M_{N} \bc \Box \;|\; x?M_{F} \;|\; x!M_{C}}
  \and
  \inferrule* [lab=abstraction] {} {{M_{F}} \bc (x)M_{P} }
  \and
  \inferrule* [lab=concretion] {} {{M_{C}} \bc \langle M_{P} \rangle }
  \and \\
  \inferrule* [lab=process] {} {{M_{P}} \bc M_{N} \;| \;P|M_{P} }
\end{mathpar}

\begin{definition}[contextual application] Given a context $M$, and
  process $P$, we define the \emph{contextual application}, $M[P] :=
  M\{P/\Box\}$. That is, the contextual application of M to P is the
  substitution of $P$ for $\Box$ in $M$.
\end{definition}

$\meaningof{-} : L \to \mathcal{P}(\pi)$

\begin{mathpar}
  \inferrule* [lab=collection] {} {\meaningof{true} = \pi, \and \meaningof{~E} = \pi \setminus \meaningof{E}, \and \meaningof{E_{1} \& E_{2}} = \meaningof{E_{1}} \cap \meaningof{E_{2}}}
\end{mathpar}

\begin{mathpar}
  \inferrule* [lab=structure] {} {\meaningof{0} = \{ P \in \pi | P \equiv 0 \}, \and \\ \meaningof{E_1 | E_2} = \{ P \in \pi | P \equiv P_{1} | P_{2}, P_{1} \in \meaningof{E_{1}}, P_{2} \in \meaningof{E_2}\} }
\end{mathpar}

\begin{mathpar}
 \inferrule* [lab=behavior] {} {\meaningof{\langle a?b \rangle E} = \{ P \in \pi | P \equiv Q | u?(y)P', \\ \and \\\\ \and \\ \;\;\; u \in \meaningof{a}, \forall z.P'\{z/y\} \in \meaningof{E\{z/b\}}\}, \and \\ \meaningof{a!E} = \{ P \in \pi | P \equiv Q | x!\langle P' \rangle, x \in \meaningof{a} P' \in \meaningof{E}\} }
\end{mathpar}

\begin{mathpar}
 \inferrule* [lab=nominal] {} {\meaningof{\quotep{E}} = \{ \quotep{P} \in \quotep{\pi} | P \in \meaningof{E} \}, \and \meaningof{\quotep{P}} = \{ \quotep{Q} \in \quotep{\pi} | P \equiv Q \} \and \\ \meaningof{@\quotep{E}} = \{ P \in \pi | P \equiv @x, x \in \meaningof{E} \}}
\end{mathpar}

\begin{eqnarray*}
  \\
  \meaningof{-} : TS \to ST
\end{eqnarray*}

\begin{eqnarray*}
  \\
  L : TS \to ST
\end{eqnarray*}

\begin{eqnarray*}
  \\
  P \models E \iff P \in \meaningof{E}
\end{eqnarray*}

\begin{eqnarray*}
  P \approx_{L} Q \iff \forall E \in L. P \models E \iff Q \models E
\end{eqnarray*}

\begin{eqnarray*}
  P \approx_{K} Q
\end{eqnarray*}

\begin{eqnarray*}
  P \approx Q
\end{eqnarray*}

$\approx_{K} = \approx = \approx_{L}$

\subsubsection{Contextual duality}

Note that contexts extend the quotation operation to a family of
operations from processes to names. Given a context, $M$, we can
define a \emph{nominal context}, $\quotep{M}$ by $\quotep{M}[P] :=
\quotep{M[P]}$. To foreshadow what is to come we observe that these
operations enjoy a duality with processes very much like the duality
between vectors and maps from vectors to scalars.

Further, because the calculus is essentially higher-order, we have a
correspondence between contexts and processes. More specifically,
given a name $x$ and a context $M$ we can construct $M^{*}_{x}$ such
that 

\begin{mathpar}
  M^{*}_{x} | \lift{x}{P} \red M[P]
\end{mathpar}

namely,

\begin{mathpar}
  M^{*}_{x} := x?(u).M[\dropn{u}]
\end{mathpar}

The dependence of $M^{*}_{x}$ on a name makes it an abstraction, 

\begin{mathpar}
  M^{*} := (x)x?(u).M[\dropn{u}]
\end{mathpar}

\subsection{Additional notation}

It will sometimes be convenient to denote the process a name
quotes. We already have the notation $x = \quotep{P}$, but it will be
convenient to introduce an alternate notation, $\procn{x}$, when we
want to emphasize the connection to the use of the name. Note that, by
virtue of name equivalence, $\quotep{\procn{x}} \nameeq x$; so, the
notation is consistent with previous definitions.

Further, because names have structure it is possible to effect
substitutions on the basis of that structure. This means we need to
upgrade our notation for substitutions, which we accomplish by
adapting comprehension notation. Thus,

\begin{mathpar}
  P\{ y / x : x \in S \}
\end{mathpar}

is interpreted to mean the process derived from P by replacing (in a
capture-avoiding manner) each occurrence of $x$ in $S$ by $y$. For example,

\begin{mathpar}
  P\{ \quotep{\procn{x}|\procn{x}} / x : x \in \freenames{P} \}
\end{mathpar}

will replace each (occurrence) of a free name $x$ in $P$ by
$\quotep{\procn{x}|\procn{x}}$.

Also, we will avail ourselves of the notation $x^{L}$ and $x^{R}$ to
denote injections of a name into disjoint copies of the name
space. There are numerous ways to accomplish this. One example can be
found in \cite{MeredithR05}. This notation overloads to vectors of
names: $\vec{x}^{\pi} := (x_{i}^{\pi} \; : \; 0 \leq i < |\vec{x}| )$ where $\pi \in \{L,R\}$.

We also use $P^{\Box} := P|\Box$.

In \cite{MeredithR05} an interpretation of the new operator is
given. It turns out that there are several possible interpretations
all enjoying the requisite algebraic properties of the operator (see
\cite{milner91polyadicpi}). We will therefore make liberal use of
$(\nu\; \vec{x})P$.

% subsection the_syntax_and_semantics_of_the_notation_system (end)   

\input{qm2pi.qmops} 

\input{qm2pi.sterngerlach} 

\input{qm2pi.metric} 

% section concurrent_process_calculi (end)

%\input{qm2pi.proofsketch}

% section proof sketch (end)

%\input{qm2pi.slviaknots} 

% section spatial logic via knots (end)

\input{qm2pi.conclusion}

% section conclusion (end)

%\input{qm2pi.dtcodes} 

% section wiring algorithm (end)

\input{qm2pi.ack} 

% section acknowledgments (end)

\newpage


\bibliographystyle{plain}   
\bibliography{../../biblios/main.bib}

\input{qm2pi.rhodetails}

\end{document}

 

\documentclass[12pt]{llncs}
%\documentclass{jktr}

\usepackage[pdftex]{hyperref}                   
\usepackage {listings}
\usepackage {mathpartir}
\usepackage{bcprules}
%\usepackage{listings}
                       
\usepackage{graphicx} 
%\usepackage[margins=2.5cm,nohead,nofoot]{geometry}
%\usepackage{geometry}
\usepackage{amsfonts}
\usepackage{amstext}
\usepackage{latexsym}
\usepackage{amssymb}
\usepackage{color}


%\include{myPreamble}
\include{qm2pi.local} 

%\ifpdf
%\usepackage[pdftex]{graphicx}
%\else
%\usepackage{graphicx}
%\fi

 % \ifpdf
%  \usepackage{pdfsync}
%  \if


%\title{Brief Article}
%\author{David F. Snyder}
%\author{L.G. Meredith}

%\address{Dept. of Math., Texas State University--San Marcos, San Marcos, TX 78666}
       
\pagestyle{empty}


\begin{document}

\lstset{language=[Objective]Caml,frame=shadowbox}

\input{qm2pi.front}

% section front matter (end)

\input{qm2pi.intro} 
 
% section introduction (end)

% \input{qm2pi.knotations} 

% section notation (end)

\input{qm2pi.process.calculi} 

% section concurrent_process_calculi_and_spatial_logics_ (end)
    
%\input{qm2pi.knots2pi} 

%\input{qm2pi.trefoil} 

%\input{qm2pi.mainthm} 

% subsection basic_interpretation (end)

%\input{qm2pi.rho.presentation} 
\subsection{The syntax and semantics of the notation system}\label{sub:the_syntax_and_semantics_of_the_notation_system} % (fold)

We now summarize a technical presentation of the calculus that
embodies our theory of dynamics. The typical presentation of such a
calculus follows the style of giving generators and relations on
them. The grammar, below, describing term constructors, freely
generates the set of processes, $\Proc$. This set is then quotiented
by a relation known as structural congruence and it is over this set
that the notion of dynamics is expressed. This presentation is
essentially that of \cite{MeredithR05} with the addition of
polyadicity and summation. For readability we have relegated some of
the technical subtleties to an appendix.

\subsubsection{Process grammar}\label{subsub:process_grammar}

\begin{mathpar}
  \inferrule* [lab=synchronization] {} {{M} \bc \pzero \;|\; x?F \;|\; x!C }
  \and
  \inferrule* [lab=abstraction] {} {{F} \bc (x)P}
  \and
  \inferrule* [lab=concretion] {} {{C} \bc \langle Q \rangle}
  \and
  \inferrule* [lab=process] {} {{P,Q} \bc M \;| \;P|Q \;|\; @{x}}
  \and
  \inferrule* [lab=name] {} {{x} \bc \quotep{P}}
\end{mathpar} 

Note that $\vec{x}$ (resp. $\vec{P}$) denotes a vector of names
(resp. processes) of length $|\vec{x}|$ (resp. $|\vec{P}|$). We adopt
the following useful abbreviations.

\begin{mathpar}
   x?(\vec{y}).P := x.(\vec{y})P \and  x\clift{\vec{P}} := x.\clift{\vec{P}}
   \and x!(y) := \lift{x}{\dropn{y}}
   \and \Pi_{i=0}^{n-1}P_i := P_0 | \ldots | P_{n-1}
\end{mathpar}

\subsubsection{Structural congruence}

\paragraph{Free and bound names and alpha-equivalence.} At the
core of structural equivalence is alpha-equivalence which identifies
process that are the same up to a change of variable. Formally, we
recognize the distinction between free and bound names. The free names
of a process, $\freenames{P}$, may be calculated recursively as
follows:

\begin{mathpar}
\freenames{\pzero} := \emptyset
  \and \\
  \freenames{x?(y).P} := \{ x \} \cup (\freenames{P} \setminus \{ y \})
  \and 
  \freenames{x!\langle P \rangle} := \{ x \} \cup \{ P \} 
  \and \\
  \freenames{P|Q} := \freenames{P} \cup \freenames{Q}
  \and \\
  \freenames{@{x}} := \{ x \}
\end{mathpar}

$\pi$
$\quotep{\pi}$

$\freenames{-} : \pi \to \mathcal{P}(\quotep{\pi})$

\begin{eqnarray*}
  \freenames{\pzero} & := & \emptyset \\
  \freenames{x?(y).P} & := & \{ x \} \cup (\freenames{P} \setminus \{ y \}) \\
  \freenames{x!\langle P \rangle} & := & \{ x \} \cup \{ P \} \\
  \freenames{P|Q} & := & \freenames{P} \cup \freenames{Q} \\
  \freenames{\dropn{x}} & := & \{ x \}
\end{eqnarray*}

The bound names of a process, $\boundnames{P}$, are those names occurring in $P$
that are not free. For example, in $x?(y).0$, the name $x$ is free, while $y$ is bound.

\begin{mathpar}
  \inferrule* [lab=monoidal-laws] {} { P|Q \equiv Q|P \and P|0 \equiv P \and P|(Q|R) \equiv (P|Q)|R }
\end{mathpar}

\begin{mathpar}
  \inferrule* [lab=alpha-equivalence] {} { (x)P \equiv (y)P\{y/x\} \and y \not\in \freenames{P} }
\end{mathpar}

\begin{definition}
Then two processes, $P,Q$, are alpha-equivalent if $P = Q\{\vec{y}/\vec{x}\}$ for
some $\vec{x} \in \boundnames{Q},\vec{y} \in \boundnames{P}$, where $Q\{\vec{y}/\vec{x}\}$
denotes the capture-avoiding substitution of $\vec{y}$ for $\vec{x}$ in $Q$.
\end{definition}

\begin{definition}
  The {\em structural congruence} \cite{SangiorgiWalker} , $\equiv$,
  between processes is the least congruence containing
  alpha-equivalence, satisfying the abelian monoid laws
  (associativity, commutativity and $\pzero$ as identity) for parallel
  composition $|$ and for summation $+$.
\end{definition}

\subsection{Name equivalence}

We take name equivalence, written $\nameeq$, to be the smallest
equivalence relation generated by the following rules.

\begin{mathpar}
\inferrule*[lab=Quote-drop]
{ }
{ \quotep{@{x}} \nameeq x }

\inferrule*[lab=Struct-equiv]
{ P \scong Q }
{ \quotep{P} \nameeq \quotep{Q} }
\end{mathpar}

The astute reader will have noticed that the mutual recursion of names
and processes imposes a mutual recursion on alpha-equivalence and
structural equivalence via name-equivalence. Fortunately, all of this
works out pleasantly and we may calculate in the natural way, free of
concern. The reader interested in the details is referred to the
appendix \ref{appendix:rho_details}.

\subsection{Substitution}

We use $\Proc$ for the set of processes, $\QProc$ for the set of
names, and $\id{\{}\vec{y} / \vec{x} \id{\}}$ to denote partial maps,
$s : \QProc \rightarrow \QProc$. A map, $s$ lifts, uniquely, to a map
on process terms, $\widehat{s} : \Proc \rightarrow \Proc$ by the
following equations.

\begin{mathpar}
  (0) \psubstp{Q}{P} := 0 \\
  (R \juxtap S) \psubstp{Q}{P}
  :=    
  (R)\psubstp{Q}{P} \juxtap (S) \psubstp{Q}{P} \\
  (x?(y).R) \psubstp{Q}{P}    
  :=    
  (x)\substp{Q}{P} (z)\concat( (R \psubstn{z}{y}) \psubstp{Q}{P} ) \\
  (\lift{x}{R}) \psubstp{Q}{P}  
  :=
  \lift{(x)\substp{Q}{P}}{ R \psubstp{Q}{P} } \\
%   (\dropn{x})  \psubstp{Q}{P}       
%   := 
%   \left\{ 
%     \begin{array}{ccc} 
%       \dropn{\quotep{Q}} & & x \nameeq \quotep{P} \\
%       \dropn{x} & & otherwise \\
%     \end{array}
%   \right. 
  (\dropn{x})  \psubstp{Q}{P}       
  := 
  \left\{ 
    \begin{array}{ccc} 
      Q & & x \nameeq \quotep{P} \\
      \dropn{x} & & otherwise \\
    \end{array}
  \right.
\end{mathpar}
 

where

\begin{eqnarray}
  (x)\id{\{} \lpquote Q \rpquote / \lpquote P \rpquote \id{\}}            = 
  \left\{ 
    \begin{array}{ccc}
      \lpquote Q \rpquote & & x \nameeq \lpquote P \rpquote \\
      x & & otherwise \\
    \end{array}
  \right. \nonumber
\end{eqnarray}

and $z$ is chosen distinct from $\quotep{P}$, $\quotep{Q}$, the free
names in $Q$, and all the names in $R$. Our $\alpha$-equivalence will
be built in the standard way from this substitution.

\begin{remark}\label{rem:no_self_referential_names}
  One consequence of these definitions is that $\forall P. \quotep{P}
  \not\in \freenames{P}$.
\end{remark}

\subsection{ Dynamic quote: an example }

Anticipating something of what's to come, consider applying the
substitution, $\widehat{\id{\{}u / z \id{\}}}$, to the following pair
of processes, $\lift{w}{y!(z)}$ and $w[ \lpquote y!(z) \rpquote ]$.

\begin{eqnarray}
	\lift{w}{y!(z)}\widehat{\id{\{}u / z \id{\}}}
		& = &
		\lift{w}{y!(u)} \nonumber\\
	w[ \lpquote y!(z) \rpquote ] \widehat{ \id{\{}u / z \id{\}} }
		& = &
		w[ \lpquote y!(z) \rpquote ] \nonumber
\end{eqnarray}

Because the body of the process between quotes is impervious to
substitution, we get radically different answers. In fact, by
examining the first process in an input context,
e.g. $x?(z).\lift{w}{y!(z)}$, we see that the process under the lift
operator may be shaped by prefixed inputs binding a name inside it. In
this sense, the lift operator will be seen as a way to dynamically
construct processes before reifying them as names.

Finally equipped with these standard features we can present the
dynamics of the calculus.

\subsubsection{Operational semantics} 

Finally, we introduce the computational dynamics. What marks these
algebras as distinct from other more traditionally studied algebraic
structures, e.g. vector spaces or polynomial rings, is the manner in
which dynamics is captured. In traditional structures, dynamics is typically
expressed through morphisms between such structures, as in linear maps
between vector spaces or morphisms between rings. In algebras
associated with the semantics of computation, the dynamics is
expressed as part of the algebraic structure itself, through a
reduction reduction relation typically denoted by $\red$. Below, we
give a recursive presentation of this relation for the calculus used
in the encoding.

$\red \subseteq \pi \times \pi$
$\red : \pi \to \mathcal{P}(\pi)$

\begin{mathpar}
  \inferrule* [lab=Comm] { \textsf{match}( x_{src}, x_{trgt} ) } { x_{trgt}?(y)P \; | \; x_{src}!\langle {Q} \rangle \red P\{\quotep{Q}/y}\} }
  \and \\
  \inferrule* [lab=Par] {{P} \red {P}'} {{{P} | {Q}} \red {{P}' | {Q}}}
  \and
  \inferrule* [lab=Equiv]{{{P} \scong {P}'} \andalso {{P}' \red {Q}'} \andalso {{Q}' \scong {Q}}}{{P} \red {Q}}
\end{mathpar}

\begin{eqnarray*}
  match_{\equiv} (\quotep{P},\quotep{Q}) & := & P \equiv Q \\
  match_{\dagger}(\quotep{P},\quotep{Q}) & := & \forall R. P|Q \red^{*} R => R \red^{*} 0 \\
  match_{K}(\quotep{P},\quotep{Q}) & := & K \mbox{ for some context } K
\end{eqnarray*}

$u?(x)P | u!\langle Q \rangle \red P\{\quotep{Q}/x\}$

%We write $\wred$ for $\red^*$, and $P\red$ if $\exists Q $ such that $ P \red Q$.
We write $P\red$ if $\exists Q $ such that $ P \red Q$ and $P\not\red$, otherwise.

\section{Replication}

As mentioned before, it is known that replication (and hence
recursion) can be implemented in a higher-order process algebra
\cite{SangiorgiWalker}. As our first example of calculation with the
machinery thus far presented we give the construction explicitly in
the {\rhoc}.

\begin{eqnarray}
	D_{x} & := & \prefix{x}{y}{(\binpar{\outputp{x}{y}}{@{y}})} \nonumber\\
	\bangp_{x}{P} & := & \binpar{{x}!\langle{\binpar{D_{x}}{P}}\rangle}{D_{x}} \nonumber
\end{eqnarray}

\begin{eqnarray}
	\bangp_{x}{P} & & \nonumber\\
	=
	& {x}!\langle{(\prefix{x}{y}{(\outputp{x}{y} | @{y})) | P}}\rangle 
	      | \prefix{x}{y}{(\outputp{x}{y} | @{y})} & \nonumber\\
	\red
	& (\outputp{x}{y} | @{y})\substn{\quotep{(\prefix{x}{y}{(@{y} | \outputp{x}{y})) | P}}}{y} & \nonumber\\
	=
	& \outputp{x}{\quotep{(\prefix{x}{y}{(\outputp{x}{y} | @{y})) | P}}}
	  | {(\prefix{x}{y}{(\outputp{x}{y} | @{y})) | P}} & \nonumber\\
	\red
	& \ldots & \nonumber\\
	\red^*
	& P | P | \ldots & \nonumber
\end{eqnarray}

Of course, this encoding, as an implementation, runs away, unfolding
$\bangp{P}$ eagerly. A lazier and more implementable replication
operator, restricted to input-guarded processes, may be obtained as follows.

\begin{eqnarray}
\bangp{\prefix{u}{v}{P}} 
	:= 
	\binpar{\lift{x}{\prefix{u}{v}{(\binpar{D(x)}{P})}}}{D(x)} \nonumber
\end{eqnarray}

\begin{remark}
  Note that the lazier definition still does not deal with summation
  or mixed summation (i.e. sums over input and output). The reader is
  invited to construct definitions of replication that deal with these
  features. 

  Further, the definitions are parameterized in a name, $x$. Can you,
  gentle reader, make a definition that eliminates this parameter and
  guarantees no accidental interaction between the replication
  machinery and the process being replicated -- i.e. no accidental
  sharing of names used by the process to get its work done and the
  name(s) used by the replication to effect copying. This latter
  revision of the definition of replication is crucial to obtaining
  the expected identity $!!P \sim !P$.
\end{remark}

\begin{remark}\label{rem:paradoxical_combinator}
  The reader familiar with the lambda calculus will have noticed the
  similarity between $D$ and the paradoxical combinator.

  [Ed. note: the existence of this seems to suggest we have to be more
  restrictive on the set of processes and names we admit if we are to
  support no-cloning.]
\end{remark}

\subsubsection{Bisimulation}

The computational dynamics gives rise to another kind of equivalence,
the equivalence of computational behavior. As previously mentioned
this is typically captured \emph{via} some form of bisimulation.

% The notion we use in this paper is weak barbed bisimulation
% \cite{milner91polyadicpi}.

The notion we use in this paper is derived from weak barbed
bisimulation \cite{milner91polyadicpi}. 

\begin{definition}
An \emph{observation relation}, $\downarrow_{\mathcal N}$, over a set
of names, $\mathcal N$, is the smallest relation satisfying the rules
below.

\infrule[Out-barb]{y \in {\mathcal N}, \; x \nameeq y}
		  {\outputp{x}{v} \downarrow_{\mathcal N} x}
\infrule[Par-barb]{\mbox{$P\downarrow_{\mathcal N} x$ or $Q\downarrow_{\mathcal N} x$}}
		  {\binpar{P}{Q} \downarrow_{\mathcal N} x}

We write $P \Downarrow_{\mathcal N} x$ if there is $Q$ such that 
$P \wred Q$ and $Q \downarrow_{\mathcal N} x$.
\end{definition}

\begin{definition}
%\label{def.bbisim}
An  ${\mathcal N}$-\emph{barbed bisimulation} over a set of names, ${\mathcal N}$, is a symmetric binary relation 
${\mathcal S}_{\mathcal N}$ between agents such that $P\rel{S}_{\mathcal N}Q$ implies:
\begin{enumerate}
\item If $P \red P'$ then $Q \wred Q'$ and $P'\rel{S}_{\mathcal N} Q'$.
\item If $P\downarrow_{\mathcal N} x$, then $Q\Downarrow_{\mathcal N} x$.
\end{enumerate}
$P$ is ${\mathcal N}$-barbed bisimilar to $Q$, written
$P \wbbisim_{\mathcal N} Q$, if $P \rel{S}_{\mathcal N} Q$ for some ${\mathcal N}$-barbed bisimulation ${\mathcal S}_{\mathcal N}$.
\end{definition}

$\mathcal{R} \subseteq \pi \times \pi$

$P \mathcal{R} Q => \forall P'. P \red P' \Rightarrow \exists Q'. Q \red Q', P' \mathcal{R} Q'$

$P \vdash x \Rightarrow Q \vdash x$

\begin{mathpar}
  \inferrule*[lab=Out-barb]{x \nameeq y}{{y}!\langle{Q}\rangle \vdash x}
  \and
  \inferrule*[lab=Par-barb]{\mbox{$P\vdash x$ or $Q\vdash x$}}{\binpar{P}{Q} \vdash x}
\end{mathpar}

\subsubsection{Contexts}

One of the principle advantages of computational calculi like the
$\pi$-calculus is a well-defined notion of context,
contextual-equivalence and a correlation between
contextual-equivalence and notions of bisimulation. The notion of
context allows the decomposition of a process into (sub-)process and
its syntactic environment, its context. Thus, a context may be
thought of as a process with a ``hole'' (written $\Box$) in it. The
application of a context $M$ to a process $P$, written $M[P]$, is
tantamount to filling the hole in $M$ with $P$. In this paper we do
not need the full weight of this theory, but do make use of the notion
of context in the proof the main theorem. 

\begin{mathpar}
  \inferrule* [lab=summation] {} {{M_{M},M_{N}} \bc \Box \;|\; x.M_{A} \;|\; M_{M}+M_{N}}
  \and
  \inferrule* [lab=agent] {} {{M_{A}} \bc (\vec{x})M_{P} \;| \; \clift{P_0,\ldots,M_{P},\ldots,P_N}}
  \and \\
  \inferrule* [lab=process] {} {{M_{P}} \bc M_{N} \;| \;P|M_{P} }
\end{mathpar} 

\begin{mathpar}
  \inferrule* [lab=sychronization] {} {M_{N} \bc \Box \;|\; x?M_{F} \;|\; x!M_{C}}
  \and
  \inferrule* [lab=abstraction] {} {{M_{F}} \bc (x)M_{P} }
  \and
  \inferrule* [lab=concretion] {} {{M_{C}} \bc \langle M_{P} \rangle }
  \and \\
  \inferrule* [lab=process] {} {{M_{P}} \bc M_{N} \;| \;P|M_{P} }
\end{mathpar}

\begin{definition}[contextual application] Given a context $M$, and
  process $P$, we define the \emph{contextual application}, $M[P] :=
  M\{P/\Box\}$. That is, the contextual application of M to P is the
  substitution of $P$ for $\Box$ in $M$.
\end{definition}

$\meaningof{-} : L \to \mathcal{P}(\pi)$

\begin{mathpar}
  \inferrule* [lab=collection] {} {\meaningof{true} = \pi, \and \meaningof{~E} = \pi \setminus \meaningof{E}, \and \meaningof{E_{1} \& E_{2}} = \meaningof{E_{1}} \cap \meaningof{E_{2}}}
\end{mathpar}

\begin{mathpar}
  \inferrule* [lab=structure] {} {\meaningof{0} = \{ P \in \pi | P \equiv 0 \}, \and \\ \meaningof{E_1 | E_2} = \{ P \in \pi | P \equiv P_{1} | P_{2}, P_{1} \in \meaningof{E_{1}}, P_{2} \in \meaningof{E_2}\} }
\end{mathpar}

\begin{mathpar}
 \inferrule* [lab=behavior] {} {\meaningof{\langle a?b \rangle E} = \{ P \in \pi | P \equiv Q | u?(y)P', \\ \and \\\\ \and \\ \;\;\; u \in \meaningof{a}, \forall z.P'\{z/y\} \in \meaningof{E\{z/b\}}\}, \and \\ \meaningof{a!E} = \{ P \in \pi | P \equiv Q | x!\langle P' \rangle, x \in \meaningof{a} P' \in \meaningof{E}\} }
\end{mathpar}

\begin{mathpar}
 \inferrule* [lab=nominal] {} {\meaningof{\quotep{E}} = \{ \quotep{P} \in \quotep{\pi} | P \in \meaningof{E} \}, \and \meaningof{\quotep{P}} = \{ \quotep{Q} \in \quotep{\pi} | P \equiv Q \} \and \\ \meaningof{@\quotep{E}} = \{ P \in \pi | P \equiv @x, x \in \meaningof{E} \}}
\end{mathpar}

\begin{eqnarray*}
  \\
  \meaningof{-} : TS \to ST
\end{eqnarray*}

\begin{eqnarray*}
  \\
  L : TS \to ST
\end{eqnarray*}

\begin{eqnarray*}
  \\
  P \models E \iff P \in \meaningof{E}
\end{eqnarray*}

\begin{eqnarray*}
  P \approx_{L} Q \iff \forall E \in L. P \models E \iff Q \models E
\end{eqnarray*}

\begin{eqnarray*}
  P \approx_{K} Q
\end{eqnarray*}

\begin{eqnarray*}
  P \approx Q
\end{eqnarray*}

$\approx_{K} = \approx = \approx_{L}$

\subsubsection{Contextual duality}

Note that contexts extend the quotation operation to a family of
operations from processes to names. Given a context, $M$, we can
define a \emph{nominal context}, $\quotep{M}$ by $\quotep{M}[P] :=
\quotep{M[P]}$. To foreshadow what is to come we observe that these
operations enjoy a duality with processes very much like the duality
between vectors and maps from vectors to scalars.

Further, because the calculus is essentially higher-order, we have a
correspondence between contexts and processes. More specifically,
given a name $x$ and a context $M$ we can construct $M^{*}_{x}$ such
that 

\begin{mathpar}
  M^{*}_{x} | \lift{x}{P} \red M[P]
\end{mathpar}

namely,

\begin{mathpar}
  M^{*}_{x} := x?(u).M[\dropn{u}]
\end{mathpar}

The dependence of $M^{*}_{x}$ on a name makes it an abstraction, 

\begin{mathpar}
  M^{*} := (x)x?(u).M[\dropn{u}]
\end{mathpar}

\subsection{Additional notation}

It will sometimes be convenient to denote the process a name
quotes. We already have the notation $x = \quotep{P}$, but it will be
convenient to introduce an alternate notation, $\procn{x}$, when we
want to emphasize the connection to the use of the name. Note that, by
virtue of name equivalence, $\quotep{\procn{x}} \nameeq x$; so, the
notation is consistent with previous definitions.

Further, because names have structure it is possible to effect
substitutions on the basis of that structure. This means we need to
upgrade our notation for substitutions, which we accomplish by
adapting comprehension notation. Thus,

\begin{mathpar}
  P\{ y / x : x \in S \}
\end{mathpar}

is interpreted to mean the process derived from P by replacing (in a
capture-avoiding manner) each occurrence of $x$ in $S$ by $y$. For example,

\begin{mathpar}
  P\{ \quotep{\procn{x}|\procn{x}} / x : x \in \freenames{P} \}
\end{mathpar}

will replace each (occurrence) of a free name $x$ in $P$ by
$\quotep{\procn{x}|\procn{x}}$.

Also, we will avail ourselves of the notation $x^{L}$ and $x^{R}$ to
denote injections of a name into disjoint copies of the name
space. There are numerous ways to accomplish this. One example can be
found in \cite{MeredithR05}. This notation overloads to vectors of
names: $\vec{x}^{\pi} := (x_{i}^{\pi} \; : \; 0 \leq i < |\vec{x}| )$ where $\pi \in \{L,R\}$.

We also use $P^{\Box} := P|\Box$.

In \cite{MeredithR05} an interpretation of the new operator is
given. It turns out that there are several possible interpretations
all enjoying the requisite algebraic properties of the operator (see
\cite{milner91polyadicpi}). We will therefore make liberal use of
$(\nu\; \vec{x})P$.

% subsection the_syntax_and_semantics_of_the_notation_system (end)   

\input{qm2pi.qmops} 

\input{qm2pi.sterngerlach} 

\input{qm2pi.metric} 

% section concurrent_process_calculi (end)

%\input{qm2pi.proofsketch}

% section proof sketch (end)

%\input{qm2pi.slviaknots} 

% section spatial logic via knots (end)

\input{qm2pi.conclusion}

% section conclusion (end)

%\input{qm2pi.dtcodes} 

% section wiring algorithm (end)

\input{qm2pi.ack} 

% section acknowledgments (end)

\newpage


\bibliographystyle{plain}   
\bibliography{../../biblios/main.bib}

\input{qm2pi.rhodetails}

\end{document}

 

% section concurrent_process_calculi (end)

%\documentclass[12pt]{llncs}
%\documentclass{jktr}

\usepackage[pdftex]{hyperref}                   
\usepackage {listings}
\usepackage {mathpartir}
\usepackage{bcprules}
%\usepackage{listings}
                       
\usepackage{graphicx} 
%\usepackage[margins=2.5cm,nohead,nofoot]{geometry}
%\usepackage{geometry}
\usepackage{amsfonts}
\usepackage{amstext}
\usepackage{latexsym}
\usepackage{amssymb}
\usepackage{color}


%\include{myPreamble}
\include{qm2pi.local} 

%\ifpdf
%\usepackage[pdftex]{graphicx}
%\else
%\usepackage{graphicx}
%\fi

 % \ifpdf
%  \usepackage{pdfsync}
%  \if


%\title{Brief Article}
%\author{David F. Snyder}
%\author{L.G. Meredith}

%\address{Dept. of Math., Texas State University--San Marcos, San Marcos, TX 78666}
       
\pagestyle{empty}


\begin{document}

\lstset{language=[Objective]Caml,frame=shadowbox}

\input{qm2pi.front}

% section front matter (end)

\input{qm2pi.intro} 
 
% section introduction (end)

% \input{qm2pi.knotations} 

% section notation (end)

\input{qm2pi.process.calculi} 

% section concurrent_process_calculi_and_spatial_logics_ (end)
    
%\input{qm2pi.knots2pi} 

%\input{qm2pi.trefoil} 

%\input{qm2pi.mainthm} 

% subsection basic_interpretation (end)

%\input{qm2pi.rho.presentation} 
\subsection{The syntax and semantics of the notation system}\label{sub:the_syntax_and_semantics_of_the_notation_system} % (fold)

We now summarize a technical presentation of the calculus that
embodies our theory of dynamics. The typical presentation of such a
calculus follows the style of giving generators and relations on
them. The grammar, below, describing term constructors, freely
generates the set of processes, $\Proc$. This set is then quotiented
by a relation known as structural congruence and it is over this set
that the notion of dynamics is expressed. This presentation is
essentially that of \cite{MeredithR05} with the addition of
polyadicity and summation. For readability we have relegated some of
the technical subtleties to an appendix.

\subsubsection{Process grammar}\label{subsub:process_grammar}

\begin{mathpar}
  \inferrule* [lab=synchronization] {} {{M} \bc \pzero \;|\; x?F \;|\; x!C }
  \and
  \inferrule* [lab=abstraction] {} {{F} \bc (x)P}
  \and
  \inferrule* [lab=concretion] {} {{C} \bc \langle Q \rangle}
  \and
  \inferrule* [lab=process] {} {{P,Q} \bc M \;| \;P|Q \;|\; @{x}}
  \and
  \inferrule* [lab=name] {} {{x} \bc \quotep{P}}
\end{mathpar} 

Note that $\vec{x}$ (resp. $\vec{P}$) denotes a vector of names
(resp. processes) of length $|\vec{x}|$ (resp. $|\vec{P}|$). We adopt
the following useful abbreviations.

\begin{mathpar}
   x?(\vec{y}).P := x.(\vec{y})P \and  x\clift{\vec{P}} := x.\clift{\vec{P}}
   \and x!(y) := \lift{x}{\dropn{y}}
   \and \Pi_{i=0}^{n-1}P_i := P_0 | \ldots | P_{n-1}
\end{mathpar}

\subsubsection{Structural congruence}

\paragraph{Free and bound names and alpha-equivalence.} At the
core of structural equivalence is alpha-equivalence which identifies
process that are the same up to a change of variable. Formally, we
recognize the distinction between free and bound names. The free names
of a process, $\freenames{P}$, may be calculated recursively as
follows:

\begin{mathpar}
\freenames{\pzero} := \emptyset
  \and \\
  \freenames{x?(y).P} := \{ x \} \cup (\freenames{P} \setminus \{ y \})
  \and 
  \freenames{x!\langle P \rangle} := \{ x \} \cup \{ P \} 
  \and \\
  \freenames{P|Q} := \freenames{P} \cup \freenames{Q}
  \and \\
  \freenames{@{x}} := \{ x \}
\end{mathpar}

$\pi$
$\quotep{\pi}$

$\freenames{-} : \pi \to \mathcal{P}(\quotep{\pi})$

\begin{eqnarray*}
  \freenames{\pzero} & := & \emptyset \\
  \freenames{x?(y).P} & := & \{ x \} \cup (\freenames{P} \setminus \{ y \}) \\
  \freenames{x!\langle P \rangle} & := & \{ x \} \cup \{ P \} \\
  \freenames{P|Q} & := & \freenames{P} \cup \freenames{Q} \\
  \freenames{\dropn{x}} & := & \{ x \}
\end{eqnarray*}

The bound names of a process, $\boundnames{P}$, are those names occurring in $P$
that are not free. For example, in $x?(y).0$, the name $x$ is free, while $y$ is bound.

\begin{mathpar}
  \inferrule* [lab=monoidal-laws] {} { P|Q \equiv Q|P \and P|0 \equiv P \and P|(Q|R) \equiv (P|Q)|R }
\end{mathpar}

\begin{mathpar}
  \inferrule* [lab=alpha-equivalence] {} { (x)P \equiv (y)P\{y/x\} \and y \not\in \freenames{P} }
\end{mathpar}

\begin{definition}
Then two processes, $P,Q$, are alpha-equivalent if $P = Q\{\vec{y}/\vec{x}\}$ for
some $\vec{x} \in \boundnames{Q},\vec{y} \in \boundnames{P}$, where $Q\{\vec{y}/\vec{x}\}$
denotes the capture-avoiding substitution of $\vec{y}$ for $\vec{x}$ in $Q$.
\end{definition}

\begin{definition}
  The {\em structural congruence} \cite{SangiorgiWalker} , $\equiv$,
  between processes is the least congruence containing
  alpha-equivalence, satisfying the abelian monoid laws
  (associativity, commutativity and $\pzero$ as identity) for parallel
  composition $|$ and for summation $+$.
\end{definition}

\subsection{Name equivalence}

We take name equivalence, written $\nameeq$, to be the smallest
equivalence relation generated by the following rules.

\begin{mathpar}
\inferrule*[lab=Quote-drop]
{ }
{ \quotep{@{x}} \nameeq x }

\inferrule*[lab=Struct-equiv]
{ P \scong Q }
{ \quotep{P} \nameeq \quotep{Q} }
\end{mathpar}

The astute reader will have noticed that the mutual recursion of names
and processes imposes a mutual recursion on alpha-equivalence and
structural equivalence via name-equivalence. Fortunately, all of this
works out pleasantly and we may calculate in the natural way, free of
concern. The reader interested in the details is referred to the
appendix \ref{appendix:rho_details}.

\subsection{Substitution}

We use $\Proc$ for the set of processes, $\QProc$ for the set of
names, and $\id{\{}\vec{y} / \vec{x} \id{\}}$ to denote partial maps,
$s : \QProc \rightarrow \QProc$. A map, $s$ lifts, uniquely, to a map
on process terms, $\widehat{s} : \Proc \rightarrow \Proc$ by the
following equations.

\begin{mathpar}
  (0) \psubstp{Q}{P} := 0 \\
  (R \juxtap S) \psubstp{Q}{P}
  :=    
  (R)\psubstp{Q}{P} \juxtap (S) \psubstp{Q}{P} \\
  (x?(y).R) \psubstp{Q}{P}    
  :=    
  (x)\substp{Q}{P} (z)\concat( (R \psubstn{z}{y}) \psubstp{Q}{P} ) \\
  (\lift{x}{R}) \psubstp{Q}{P}  
  :=
  \lift{(x)\substp{Q}{P}}{ R \psubstp{Q}{P} } \\
%   (\dropn{x})  \psubstp{Q}{P}       
%   := 
%   \left\{ 
%     \begin{array}{ccc} 
%       \dropn{\quotep{Q}} & & x \nameeq \quotep{P} \\
%       \dropn{x} & & otherwise \\
%     \end{array}
%   \right. 
  (\dropn{x})  \psubstp{Q}{P}       
  := 
  \left\{ 
    \begin{array}{ccc} 
      Q & & x \nameeq \quotep{P} \\
      \dropn{x} & & otherwise \\
    \end{array}
  \right.
\end{mathpar}
 

where

\begin{eqnarray}
  (x)\id{\{} \lpquote Q \rpquote / \lpquote P \rpquote \id{\}}            = 
  \left\{ 
    \begin{array}{ccc}
      \lpquote Q \rpquote & & x \nameeq \lpquote P \rpquote \\
      x & & otherwise \\
    \end{array}
  \right. \nonumber
\end{eqnarray}

and $z$ is chosen distinct from $\quotep{P}$, $\quotep{Q}$, the free
names in $Q$, and all the names in $R$. Our $\alpha$-equivalence will
be built in the standard way from this substitution.

\begin{remark}\label{rem:no_self_referential_names}
  One consequence of these definitions is that $\forall P. \quotep{P}
  \not\in \freenames{P}$.
\end{remark}

\subsection{ Dynamic quote: an example }

Anticipating something of what's to come, consider applying the
substitution, $\widehat{\id{\{}u / z \id{\}}}$, to the following pair
of processes, $\lift{w}{y!(z)}$ and $w[ \lpquote y!(z) \rpquote ]$.

\begin{eqnarray}
	\lift{w}{y!(z)}\widehat{\id{\{}u / z \id{\}}}
		& = &
		\lift{w}{y!(u)} \nonumber\\
	w[ \lpquote y!(z) \rpquote ] \widehat{ \id{\{}u / z \id{\}} }
		& = &
		w[ \lpquote y!(z) \rpquote ] \nonumber
\end{eqnarray}

Because the body of the process between quotes is impervious to
substitution, we get radically different answers. In fact, by
examining the first process in an input context,
e.g. $x?(z).\lift{w}{y!(z)}$, we see that the process under the lift
operator may be shaped by prefixed inputs binding a name inside it. In
this sense, the lift operator will be seen as a way to dynamically
construct processes before reifying them as names.

Finally equipped with these standard features we can present the
dynamics of the calculus.

\subsubsection{Operational semantics} 

Finally, we introduce the computational dynamics. What marks these
algebras as distinct from other more traditionally studied algebraic
structures, e.g. vector spaces or polynomial rings, is the manner in
which dynamics is captured. In traditional structures, dynamics is typically
expressed through morphisms between such structures, as in linear maps
between vector spaces or morphisms between rings. In algebras
associated with the semantics of computation, the dynamics is
expressed as part of the algebraic structure itself, through a
reduction reduction relation typically denoted by $\red$. Below, we
give a recursive presentation of this relation for the calculus used
in the encoding.

$\red \subseteq \pi \times \pi$
$\red : \pi \to \mathcal{P}(\pi)$

\begin{mathpar}
  \inferrule* [lab=Comm] { \textsf{match}( x_{src}, x_{trgt} ) } { x_{trgt}?(y)P \; | \; x_{src}!\langle {Q} \rangle \red P\{\quotep{Q}/y}\} }
  \and \\
  \inferrule* [lab=Par] {{P} \red {P}'} {{{P} | {Q}} \red {{P}' | {Q}}}
  \and
  \inferrule* [lab=Equiv]{{{P} \scong {P}'} \andalso {{P}' \red {Q}'} \andalso {{Q}' \scong {Q}}}{{P} \red {Q}}
\end{mathpar}

\begin{eqnarray*}
  match_{\equiv} (\quotep{P},\quotep{Q}) & := & P \equiv Q \\
  match_{\dagger}(\quotep{P},\quotep{Q}) & := & \forall R. P|Q \red^{*} R => R \red^{*} 0 \\
  match_{K}(\quotep{P},\quotep{Q}) & := & K \mbox{ for some context } K
\end{eqnarray*}

$u?(x)P | u!\langle Q \rangle \red P\{\quotep{Q}/x\}$

%We write $\wred$ for $\red^*$, and $P\red$ if $\exists Q $ such that $ P \red Q$.
We write $P\red$ if $\exists Q $ such that $ P \red Q$ and $P\not\red$, otherwise.

\section{Replication}

As mentioned before, it is known that replication (and hence
recursion) can be implemented in a higher-order process algebra
\cite{SangiorgiWalker}. As our first example of calculation with the
machinery thus far presented we give the construction explicitly in
the {\rhoc}.

\begin{eqnarray}
	D_{x} & := & \prefix{x}{y}{(\binpar{\outputp{x}{y}}{@{y}})} \nonumber\\
	\bangp_{x}{P} & := & \binpar{{x}!\langle{\binpar{D_{x}}{P}}\rangle}{D_{x}} \nonumber
\end{eqnarray}

\begin{eqnarray}
	\bangp_{x}{P} & & \nonumber\\
	=
	& {x}!\langle{(\prefix{x}{y}{(\outputp{x}{y} | @{y})) | P}}\rangle 
	      | \prefix{x}{y}{(\outputp{x}{y} | @{y})} & \nonumber\\
	\red
	& (\outputp{x}{y} | @{y})\substn{\quotep{(\prefix{x}{y}{(@{y} | \outputp{x}{y})) | P}}}{y} & \nonumber\\
	=
	& \outputp{x}{\quotep{(\prefix{x}{y}{(\outputp{x}{y} | @{y})) | P}}}
	  | {(\prefix{x}{y}{(\outputp{x}{y} | @{y})) | P}} & \nonumber\\
	\red
	& \ldots & \nonumber\\
	\red^*
	& P | P | \ldots & \nonumber
\end{eqnarray}

Of course, this encoding, as an implementation, runs away, unfolding
$\bangp{P}$ eagerly. A lazier and more implementable replication
operator, restricted to input-guarded processes, may be obtained as follows.

\begin{eqnarray}
\bangp{\prefix{u}{v}{P}} 
	:= 
	\binpar{\lift{x}{\prefix{u}{v}{(\binpar{D(x)}{P})}}}{D(x)} \nonumber
\end{eqnarray}

\begin{remark}
  Note that the lazier definition still does not deal with summation
  or mixed summation (i.e. sums over input and output). The reader is
  invited to construct definitions of replication that deal with these
  features. 

  Further, the definitions are parameterized in a name, $x$. Can you,
  gentle reader, make a definition that eliminates this parameter and
  guarantees no accidental interaction between the replication
  machinery and the process being replicated -- i.e. no accidental
  sharing of names used by the process to get its work done and the
  name(s) used by the replication to effect copying. This latter
  revision of the definition of replication is crucial to obtaining
  the expected identity $!!P \sim !P$.
\end{remark}

\begin{remark}\label{rem:paradoxical_combinator}
  The reader familiar with the lambda calculus will have noticed the
  similarity between $D$ and the paradoxical combinator.

  [Ed. note: the existence of this seems to suggest we have to be more
  restrictive on the set of processes and names we admit if we are to
  support no-cloning.]
\end{remark}

\subsubsection{Bisimulation}

The computational dynamics gives rise to another kind of equivalence,
the equivalence of computational behavior. As previously mentioned
this is typically captured \emph{via} some form of bisimulation.

% The notion we use in this paper is weak barbed bisimulation
% \cite{milner91polyadicpi}.

The notion we use in this paper is derived from weak barbed
bisimulation \cite{milner91polyadicpi}. 

\begin{definition}
An \emph{observation relation}, $\downarrow_{\mathcal N}$, over a set
of names, $\mathcal N$, is the smallest relation satisfying the rules
below.

\infrule[Out-barb]{y \in {\mathcal N}, \; x \nameeq y}
		  {\outputp{x}{v} \downarrow_{\mathcal N} x}
\infrule[Par-barb]{\mbox{$P\downarrow_{\mathcal N} x$ or $Q\downarrow_{\mathcal N} x$}}
		  {\binpar{P}{Q} \downarrow_{\mathcal N} x}

We write $P \Downarrow_{\mathcal N} x$ if there is $Q$ such that 
$P \wred Q$ and $Q \downarrow_{\mathcal N} x$.
\end{definition}

\begin{definition}
%\label{def.bbisim}
An  ${\mathcal N}$-\emph{barbed bisimulation} over a set of names, ${\mathcal N}$, is a symmetric binary relation 
${\mathcal S}_{\mathcal N}$ between agents such that $P\rel{S}_{\mathcal N}Q$ implies:
\begin{enumerate}
\item If $P \red P'$ then $Q \wred Q'$ and $P'\rel{S}_{\mathcal N} Q'$.
\item If $P\downarrow_{\mathcal N} x$, then $Q\Downarrow_{\mathcal N} x$.
\end{enumerate}
$P$ is ${\mathcal N}$-barbed bisimilar to $Q$, written
$P \wbbisim_{\mathcal N} Q$, if $P \rel{S}_{\mathcal N} Q$ for some ${\mathcal N}$-barbed bisimulation ${\mathcal S}_{\mathcal N}$.
\end{definition}

$\mathcal{R} \subseteq \pi \times \pi$

$P \mathcal{R} Q => \forall P'. P \red P' \Rightarrow \exists Q'. Q \red Q', P' \mathcal{R} Q'$

$P \vdash x \Rightarrow Q \vdash x$

\begin{mathpar}
  \inferrule*[lab=Out-barb]{x \nameeq y}{{y}!\langle{Q}\rangle \vdash x}
  \and
  \inferrule*[lab=Par-barb]{\mbox{$P\vdash x$ or $Q\vdash x$}}{\binpar{P}{Q} \vdash x}
\end{mathpar}

\subsubsection{Contexts}

One of the principle advantages of computational calculi like the
$\pi$-calculus is a well-defined notion of context,
contextual-equivalence and a correlation between
contextual-equivalence and notions of bisimulation. The notion of
context allows the decomposition of a process into (sub-)process and
its syntactic environment, its context. Thus, a context may be
thought of as a process with a ``hole'' (written $\Box$) in it. The
application of a context $M$ to a process $P$, written $M[P]$, is
tantamount to filling the hole in $M$ with $P$. In this paper we do
not need the full weight of this theory, but do make use of the notion
of context in the proof the main theorem. 

\begin{mathpar}
  \inferrule* [lab=summation] {} {{M_{M},M_{N}} \bc \Box \;|\; x.M_{A} \;|\; M_{M}+M_{N}}
  \and
  \inferrule* [lab=agent] {} {{M_{A}} \bc (\vec{x})M_{P} \;| \; \clift{P_0,\ldots,M_{P},\ldots,P_N}}
  \and \\
  \inferrule* [lab=process] {} {{M_{P}} \bc M_{N} \;| \;P|M_{P} }
\end{mathpar} 

\begin{mathpar}
  \inferrule* [lab=sychronization] {} {M_{N} \bc \Box \;|\; x?M_{F} \;|\; x!M_{C}}
  \and
  \inferrule* [lab=abstraction] {} {{M_{F}} \bc (x)M_{P} }
  \and
  \inferrule* [lab=concretion] {} {{M_{C}} \bc \langle M_{P} \rangle }
  \and \\
  \inferrule* [lab=process] {} {{M_{P}} \bc M_{N} \;| \;P|M_{P} }
\end{mathpar}

\begin{definition}[contextual application] Given a context $M$, and
  process $P$, we define the \emph{contextual application}, $M[P] :=
  M\{P/\Box\}$. That is, the contextual application of M to P is the
  substitution of $P$ for $\Box$ in $M$.
\end{definition}

$\meaningof{-} : L \to \mathcal{P}(\pi)$

\begin{mathpar}
  \inferrule* [lab=collection] {} {\meaningof{true} = \pi, \and \meaningof{~E} = \pi \setminus \meaningof{E}, \and \meaningof{E_{1} \& E_{2}} = \meaningof{E_{1}} \cap \meaningof{E_{2}}}
\end{mathpar}

\begin{mathpar}
  \inferrule* [lab=structure] {} {\meaningof{0} = \{ P \in \pi | P \equiv 0 \}, \and \\ \meaningof{E_1 | E_2} = \{ P \in \pi | P \equiv P_{1} | P_{2}, P_{1} \in \meaningof{E_{1}}, P_{2} \in \meaningof{E_2}\} }
\end{mathpar}

\begin{mathpar}
 \inferrule* [lab=behavior] {} {\meaningof{\langle a?b \rangle E} = \{ P \in \pi | P \equiv Q | u?(y)P', \\ \and \\\\ \and \\ \;\;\; u \in \meaningof{a}, \forall z.P'\{z/y\} \in \meaningof{E\{z/b\}}\}, \and \\ \meaningof{a!E} = \{ P \in \pi | P \equiv Q | x!\langle P' \rangle, x \in \meaningof{a} P' \in \meaningof{E}\} }
\end{mathpar}

\begin{mathpar}
 \inferrule* [lab=nominal] {} {\meaningof{\quotep{E}} = \{ \quotep{P} \in \quotep{\pi} | P \in \meaningof{E} \}, \and \meaningof{\quotep{P}} = \{ \quotep{Q} \in \quotep{\pi} | P \equiv Q \} \and \\ \meaningof{@\quotep{E}} = \{ P \in \pi | P \equiv @x, x \in \meaningof{E} \}}
\end{mathpar}

\begin{eqnarray*}
  \\
  \meaningof{-} : TS \to ST
\end{eqnarray*}

\begin{eqnarray*}
  \\
  L : TS \to ST
\end{eqnarray*}

\begin{eqnarray*}
  \\
  P \models E \iff P \in \meaningof{E}
\end{eqnarray*}

\begin{eqnarray*}
  P \approx_{L} Q \iff \forall E \in L. P \models E \iff Q \models E
\end{eqnarray*}

\begin{eqnarray*}
  P \approx_{K} Q
\end{eqnarray*}

\begin{eqnarray*}
  P \approx Q
\end{eqnarray*}

$\approx_{K} = \approx = \approx_{L}$

\subsubsection{Contextual duality}

Note that contexts extend the quotation operation to a family of
operations from processes to names. Given a context, $M$, we can
define a \emph{nominal context}, $\quotep{M}$ by $\quotep{M}[P] :=
\quotep{M[P]}$. To foreshadow what is to come we observe that these
operations enjoy a duality with processes very much like the duality
between vectors and maps from vectors to scalars.

Further, because the calculus is essentially higher-order, we have a
correspondence between contexts and processes. More specifically,
given a name $x$ and a context $M$ we can construct $M^{*}_{x}$ such
that 

\begin{mathpar}
  M^{*}_{x} | \lift{x}{P} \red M[P]
\end{mathpar}

namely,

\begin{mathpar}
  M^{*}_{x} := x?(u).M[\dropn{u}]
\end{mathpar}

The dependence of $M^{*}_{x}$ on a name makes it an abstraction, 

\begin{mathpar}
  M^{*} := (x)x?(u).M[\dropn{u}]
\end{mathpar}

\subsection{Additional notation}

It will sometimes be convenient to denote the process a name
quotes. We already have the notation $x = \quotep{P}$, but it will be
convenient to introduce an alternate notation, $\procn{x}$, when we
want to emphasize the connection to the use of the name. Note that, by
virtue of name equivalence, $\quotep{\procn{x}} \nameeq x$; so, the
notation is consistent with previous definitions.

Further, because names have structure it is possible to effect
substitutions on the basis of that structure. This means we need to
upgrade our notation for substitutions, which we accomplish by
adapting comprehension notation. Thus,

\begin{mathpar}
  P\{ y / x : x \in S \}
\end{mathpar}

is interpreted to mean the process derived from P by replacing (in a
capture-avoiding manner) each occurrence of $x$ in $S$ by $y$. For example,

\begin{mathpar}
  P\{ \quotep{\procn{x}|\procn{x}} / x : x \in \freenames{P} \}
\end{mathpar}

will replace each (occurrence) of a free name $x$ in $P$ by
$\quotep{\procn{x}|\procn{x}}$.

Also, we will avail ourselves of the notation $x^{L}$ and $x^{R}$ to
denote injections of a name into disjoint copies of the name
space. There are numerous ways to accomplish this. One example can be
found in \cite{MeredithR05}. This notation overloads to vectors of
names: $\vec{x}^{\pi} := (x_{i}^{\pi} \; : \; 0 \leq i < |\vec{x}| )$ where $\pi \in \{L,R\}$.

We also use $P^{\Box} := P|\Box$.

In \cite{MeredithR05} an interpretation of the new operator is
given. It turns out that there are several possible interpretations
all enjoying the requisite algebraic properties of the operator (see
\cite{milner91polyadicpi}). We will therefore make liberal use of
$(\nu\; \vec{x})P$.

% subsection the_syntax_and_semantics_of_the_notation_system (end)   

\input{qm2pi.qmops} 

\input{qm2pi.sterngerlach} 

\input{qm2pi.metric} 

% section concurrent_process_calculi (end)

%\input{qm2pi.proofsketch}

% section proof sketch (end)

%\input{qm2pi.slviaknots} 

% section spatial logic via knots (end)

\input{qm2pi.conclusion}

% section conclusion (end)

%\input{qm2pi.dtcodes} 

% section wiring algorithm (end)

\input{qm2pi.ack} 

% section acknowledgments (end)

\newpage


\bibliographystyle{plain}   
\bibliography{../../biblios/main.bib}

\input{qm2pi.rhodetails}

\end{document}



% section proof sketch (end)

%\section{Unlikely characters: spatial logic for
  knots}\label{sub:characteristic_formulae} % (fold)

Associated to the mobile process calculi are a family of logics known
as the Hennessy-Milner logics. These logics typically enjoy a
semantics interpreting formulae as sets of processes that when
factored through the encoding outlined above allows an identification
of classes of knots with logical formulae. In the context of this
encoding the sub-family known as the spatial logics \cite{CairesC03}
\cite{CairesC04} \cite{Caires04} are of particular interest providing
several important features for expressing and reasoning about
properties (i.e. classes) of knots. We hint here at how this may be done.

%\begin{description}
%\item [structural connectives] 
\subsubsection{Structural connectives} The spatial logics enjoy
structural connectives corresponding, at the logical level, to the
parallel composition ($P | Q$) and new name ($(\nu \; x)P$)
connectives for processes. As illustrated in the examples below, these
connectives are extremely expressive given the shape of our encoding.
%\item [decideable satisfaction]

\subsubsection{Decideable satisfaction}
In \cite{Caires04} the satisfaction relation is shown to be decideable
for a rich class of processes. It further turns out that the image of
the our encoding is a proper subset of that class. This result
provides the basis for an algorithm by which to search for knots
enjoying a given property.
%\item [characteristic formulae]

\subsubsection{Characteristic formulae}
In the same paper \cite{Caires04} , Caires presents a means of calculating
characteristic formulae, selecting equivalence classes of processes
up to a pre--specified depth limit on the support set of names. Composed with our
encoding, this characteristic formula can be used to select
characteristic formulae for knots.
%\end{description}

\subsubsection{Spatial logic formulae}

The grammar below (segmented for comprehension) summarizes the syntax
of spatial logic formulae. We employ illustrative examples in the
sequel to provide an intuitive understanding of their meaning
referring the reader to \cite{Caires04} for a more detailed explication
of the semantics.

\begin{mathpar}
  \inferrule* [lab=boolean] {} {{A,B} \bc T \;|\; \neg A \;|\; A \wedge B \;|\; \eta = \eta'}
  \and
  \inferrule* [lab=spatial] {} {|\; \pzero \;|\; A | B \;|\; x \text{\textregistered} A \;|\; \forall x . A \;|\;  H x . A}
  \and
  \inferrule* [lab=behavioral] {} {|\; \alpha . A}
  \and 
  \inferrule* [lab=recursion] {} {|\; X(\vec{u}) \;|\; \mu X(\vec{u}) . A}
  \and
  \inferrule* [lab=action] {} {\alpha \bc \langle x?(\vec{y}) \rangle \;|\; \langle x!(\vec{y}) \rangle \;|\; \langle \tau \rangle}
  \and 
  \inferrule* [lab=name] {} {\eta \bc x \;|\; \tau}
\end{mathpar} 

% subsection characteristic_formulae (end)   	 

\subsection{Example formulae}\label{sub:example_formulae_} % (fold)

\subsubsection{Crossing as formula.}
% 
% \begin{align*}
%   \frac{d}{dx} \sin x &= \cos x 
%   & \frac{d}{dx} e^x &= e^x \\
%   \frac{d}{dx} \cos x &= - \sin x 
%   & \frac{d}{dx} \log x &= \frac{1}{x} \\
% \end{align*} 

\begin{align*}
 \mu C(x_{0},x_{1},y_{0},y_{1},u).&(\langle x_{0}?(z) \rangle(\langle u! \rangle\langle y_{1}!z \rangle C(x_{0},x_{1},y_{0},y_{1},u)) & \\
  & \wedge \langle y_{1}?(z) \rangle (\langle u! \rangle \langle x_{0}!z \rangle C(x_{0},x_{1},y_{0},y_{1},u)) & \\
  & \wedge \langle x_{1}?(z) \rangle (\langle u? \rangle \langle y_{0}!z \rangle C(x_{0},x_{1},y_{0},y_{1},u)) & \\
  & \wedge \langle y_{0}?(z) \rangle (\langle u? \rangle \langle x_{1}!z \rangle C(x_{0},x_{1},y_{0},y_{1},u))) &
\end{align*}

The lexicographical similarity between the shape of this formulae and
the shape of definition of the process representing a crossing reveals
the intuitive meaning of this formulae. It describes the capabilities
of a process that has the right to represent a crossing. For example
it picks out processes that may perform an input on the port $x_0$ in
its initial menu of capabilities. What differentiates the formula
from the process, however, is that the crossing process is the
smallest candidate to satisfy the formula. Infinitely many other
processes -- with internal behavior hidden behind this interface, so
to speak -- also satisfy this formula. Even this simple formula,
then, can be seen to open a new view onto knots, providing a
computational interpretation of \emph{virtual} knots.

Note that this formula is derived by hand. A similar formula can be
derived by employing Caires' calculation of characteristic formula
\cite{Caires04} to the process representing a crossing. In light of
this discussion, we let
$\meaningof{C}_{\phi}(x0,x1,y0,y1,u)$ denote a formula specifying the
dynamics we wish to capture of a crossing. To guarantee we preserve
the shape of the interface and minimal semantics we demand that
$\meaningof{C}_{\phi}(x0,x1,y0,y1,u) \Rightarrow
\textbf{C}(x0,x1,y0,y1,u)$ where $\textbf{C}(x0,x1,y0,y1,u)$ denotes
the formula above.
                            
\subsubsection{Crossing number constraints.}
The moral content of the context lemma (Lemma \ref{context}) is that the notion of
``locality'' in the Reidemeister moves is effectively captured by the
parallel composition operator of the process calculus. This intuition
extends through the logic. Given a formula,
$\meaningof{C}_{\phi}(x0,x1,y0,y1,u)$, we can use the structural
connectives to specify constraints on crossing numbers, such as at
least $n$ crossings, or exactly $n$ crossings.
\begin{mathpar}
  \inferrule* [lab=at-least-n] {} { K^{\geq n}_{\phi}(\vec{xs},\vec{ys}) := \Pi_{i=0}^{n-1} Hu . \meaningof{C}_{\phi}(xs_i,ys_i,u) | T }
  \and 
  \inferrule* [lab=exactly-n] {} { K^{= n}_{\phi}(\vec{xs},\vec{ys}) := \Pi_{i=0}^{n-1} Hu . \meaningof{C}_{\phi}(xs_i,ys_i,u) | \neg (\forall x_0,y_0,x_1,y_1,u . \meaningof{C}_{\phi}(x_0,y_0,x_1,y_1,u) | T) }
\end{mathpar}

To round out this section, recall that the encoding of an $n$-crossing
knot decomposes into a parallel composition of $n$ \emph{copies} of a
crossing process together with a wiring harness. To specify different
knot classes with the same crossing number amounts to specifying
logical constraints on the wiring harness. In the interest of space,
we defer examples to a forthcoming paper. Suffice it to say that both
the conditions ``alternating knot'' and ``contains the tangle
corresponding to 5/3'' are expressible. For example, it is possible to
calculate the characteristic formula of a process corresponding to the
tangle 5/3 and conjoin it into the classifying formula via the
composition connective of the logic.

Finally, we wish to observe that it is entirely within reason to
contemplate a more domain-specific version of spatial logic tailored
to the shape of processes in the image of the encoding. Such a
domain-specific logic would have a better claim to the title formal
language of knot properties.

% subsection example_formulae_ (end)

% section knots_as_processes (end) 

% section spatial logic via knots (end)

\section{Conclusions and future work}

\paragraph{Testing physical space}
You, gentle reader, may wonder why of all the theorems to be proved
given this set up we pick the one above. In some sense it's hardly
central to quantum mechanics. We see it as central in the sense that
it firmly establishes a notion of physical space arising from a notion
of the equivalence of behavior. Relating bisimulation to a metric is a
big step forward, but one is faced with interpreting the relationship
of that metric space to something more physical. Quantum mechanical
notions of ``physical'' space are still far from intuitive, but by
relating this idea of distance as testing to calculations that predict
physical circumstances we are making a not insignificant step forward
toward an understanding of the physical space we inhabit as
essentially dynamic.

\paragraph{Effectivity and simulation}
One of the observations we have yet to make is that the entire program
spelled out here is effective. We have built various interpreters for
the reflective calculus at work in this interpretation. In principle,
then, we can simulate quantum mechanics on a computer. The place where
the simulation may lose fidelity is the infinitely branching summation
for the annihilator.

In this connection i also want to point out that the evaluation style
calculation of the inner product puts the non-determinism of the
summation right at the heart of measurement. This suggests that
Milner's original reduction-based formulation of the dynamics of his
calculi in terms of sums was not just notationally suggestive of a
notion of measure-and-continue but captured some significant part of
the physics.

\paragraph{Quantum continuations}
In light of this last observation i want to point out that the
predominant account of quantum mechanics is missing a key aspect of a
truly compositional story of the physical situation. In a real lab,
when a measurement is made the observation can be made to feed into
another device that then makes another measurement conditioned on the
results of the first. This means that after the superposition was
collapsed the entire experimental set up remained in
superposition. While QM offers a means of writing this down it doesn't
quite line up well with the well-trodden formulation of computation
and continuation that we see so succinctly expressed in Milner's
calculi. This suggests that there might be advantages to this account
of dynamics waiting to be explored.

\paragraph{Quantum logic}
In this connection, we also note that by virtue of having the
Hennessy-Milner construction, we can pull the construction through the
interpretation of QM. This gives us a natural candidate for a quantum
logic that enjoys an extremely tight connection with it's domain of
interpretation, making the construction much less ad hoc (rather it is
the image of functor!).

\paragraph{Quantum probabiity}
i have questions about the basis of the interpretation of inner
product as probability amplitude. In particular, using which
axiomatization of probability theory does the notion of probability
amplitude earn the right to be so dubbed? In other words, where is the
proof that the operation for calculating a probability amplitude (and
then squaring) satisfies the axioms of what it means to calculate a
probability? Even if such a proof exists (i have yet to find it in the
literature), i wonder if it might not be possible to turn things on
their heads. Can we view the calculation of the probability amplitude
as an axiomatization of probability? If so, then the definition we
give for calculating probability amplitude may provide the basis for
an \emph{effective} theory of probability.

\paragraph{Quantum vs ``biological'' information}
Finally, i want to conclude with a more philosophical observation. At
a recent workshop in which QM was a predominant topic i noticed
something about quantum information. The speaker was giving a riveting
discussion of axiomatic QM and showing how properties of ``no
cloning'' and ``no deleting'' emerged as consequences of the
axiomatization. Theorems of this form are necessary to give us a sense
of confidence that our axioms characterize the physical theory. What
struck me, though, was that if quantum information is neither erasable
nor replicable it is markedly different from \emph{life}. Two of the
things we know about life is that

\begin{itemize}
  \item it ends;
  \item to gain some measure of persistence, to transcend it's
    finitude it is imminently copyable.
\end{itemize}

Both of these qualities are summarized succinctly in the aphorism: all
flesh is grass. For me these two kinds of ``information'' -- call them
quantum and biological -- are end points on a spectrum of strategies
for persistence. At one end, we have those curious entities that enjoy
uniqueness and permanence; at the other, we have those who in the face
of a certain end and an uncertain present make a go of passing
something on. To me one of the more remarkable aspects of the latter
strategy is that in the presence of noise (and certain features of
copying) we get a kind of dynamism, a chance for improvement against a
given persistent condition.

% subsection other_calculi_other_bisimulations_and_geometry_as_behavior (end)




% section conclusion (end)

%\documentclass[12pt]{llncs}
%\documentclass{jktr}

\usepackage[pdftex]{hyperref}                   
\usepackage {listings}
\usepackage {mathpartir}
\usepackage{bcprules}
%\usepackage{listings}
                       
\usepackage{graphicx} 
%\usepackage[margins=2.5cm,nohead,nofoot]{geometry}
%\usepackage{geometry}
\usepackage{amsfonts}
\usepackage{amstext}
\usepackage{latexsym}
\usepackage{amssymb}
\usepackage{color}


%\include{myPreamble}
\include{qm2pi.local} 

%\ifpdf
%\usepackage[pdftex]{graphicx}
%\else
%\usepackage{graphicx}
%\fi

 % \ifpdf
%  \usepackage{pdfsync}
%  \if


%\title{Brief Article}
%\author{David F. Snyder}
%\author{L.G. Meredith}

%\address{Dept. of Math., Texas State University--San Marcos, San Marcos, TX 78666}
       
\pagestyle{empty}


\begin{document}

\lstset{language=[Objective]Caml,frame=shadowbox}

\input{qm2pi.front}

% section front matter (end)

\input{qm2pi.intro} 
 
% section introduction (end)

% \input{qm2pi.knotations} 

% section notation (end)

\input{qm2pi.process.calculi} 

% section concurrent_process_calculi_and_spatial_logics_ (end)
    
%\input{qm2pi.knots2pi} 

%\input{qm2pi.trefoil} 

%\input{qm2pi.mainthm} 

% subsection basic_interpretation (end)

%\input{qm2pi.rho.presentation} 
\subsection{The syntax and semantics of the notation system}\label{sub:the_syntax_and_semantics_of_the_notation_system} % (fold)

We now summarize a technical presentation of the calculus that
embodies our theory of dynamics. The typical presentation of such a
calculus follows the style of giving generators and relations on
them. The grammar, below, describing term constructors, freely
generates the set of processes, $\Proc$. This set is then quotiented
by a relation known as structural congruence and it is over this set
that the notion of dynamics is expressed. This presentation is
essentially that of \cite{MeredithR05} with the addition of
polyadicity and summation. For readability we have relegated some of
the technical subtleties to an appendix.

\subsubsection{Process grammar}\label{subsub:process_grammar}

\begin{mathpar}
  \inferrule* [lab=synchronization] {} {{M} \bc \pzero \;|\; x?F \;|\; x!C }
  \and
  \inferrule* [lab=abstraction] {} {{F} \bc (x)P}
  \and
  \inferrule* [lab=concretion] {} {{C} \bc \langle Q \rangle}
  \and
  \inferrule* [lab=process] {} {{P,Q} \bc M \;| \;P|Q \;|\; @{x}}
  \and
  \inferrule* [lab=name] {} {{x} \bc \quotep{P}}
\end{mathpar} 

Note that $\vec{x}$ (resp. $\vec{P}$) denotes a vector of names
(resp. processes) of length $|\vec{x}|$ (resp. $|\vec{P}|$). We adopt
the following useful abbreviations.

\begin{mathpar}
   x?(\vec{y}).P := x.(\vec{y})P \and  x\clift{\vec{P}} := x.\clift{\vec{P}}
   \and x!(y) := \lift{x}{\dropn{y}}
   \and \Pi_{i=0}^{n-1}P_i := P_0 | \ldots | P_{n-1}
\end{mathpar}

\subsubsection{Structural congruence}

\paragraph{Free and bound names and alpha-equivalence.} At the
core of structural equivalence is alpha-equivalence which identifies
process that are the same up to a change of variable. Formally, we
recognize the distinction between free and bound names. The free names
of a process, $\freenames{P}$, may be calculated recursively as
follows:

\begin{mathpar}
\freenames{\pzero} := \emptyset
  \and \\
  \freenames{x?(y).P} := \{ x \} \cup (\freenames{P} \setminus \{ y \})
  \and 
  \freenames{x!\langle P \rangle} := \{ x \} \cup \{ P \} 
  \and \\
  \freenames{P|Q} := \freenames{P} \cup \freenames{Q}
  \and \\
  \freenames{@{x}} := \{ x \}
\end{mathpar}

$\pi$
$\quotep{\pi}$

$\freenames{-} : \pi \to \mathcal{P}(\quotep{\pi})$

\begin{eqnarray*}
  \freenames{\pzero} & := & \emptyset \\
  \freenames{x?(y).P} & := & \{ x \} \cup (\freenames{P} \setminus \{ y \}) \\
  \freenames{x!\langle P \rangle} & := & \{ x \} \cup \{ P \} \\
  \freenames{P|Q} & := & \freenames{P} \cup \freenames{Q} \\
  \freenames{\dropn{x}} & := & \{ x \}
\end{eqnarray*}

The bound names of a process, $\boundnames{P}$, are those names occurring in $P$
that are not free. For example, in $x?(y).0$, the name $x$ is free, while $y$ is bound.

\begin{mathpar}
  \inferrule* [lab=monoidal-laws] {} { P|Q \equiv Q|P \and P|0 \equiv P \and P|(Q|R) \equiv (P|Q)|R }
\end{mathpar}

\begin{mathpar}
  \inferrule* [lab=alpha-equivalence] {} { (x)P \equiv (y)P\{y/x\} \and y \not\in \freenames{P} }
\end{mathpar}

\begin{definition}
Then two processes, $P,Q$, are alpha-equivalent if $P = Q\{\vec{y}/\vec{x}\}$ for
some $\vec{x} \in \boundnames{Q},\vec{y} \in \boundnames{P}$, where $Q\{\vec{y}/\vec{x}\}$
denotes the capture-avoiding substitution of $\vec{y}$ for $\vec{x}$ in $Q$.
\end{definition}

\begin{definition}
  The {\em structural congruence} \cite{SangiorgiWalker} , $\equiv$,
  between processes is the least congruence containing
  alpha-equivalence, satisfying the abelian monoid laws
  (associativity, commutativity and $\pzero$ as identity) for parallel
  composition $|$ and for summation $+$.
\end{definition}

\subsection{Name equivalence}

We take name equivalence, written $\nameeq$, to be the smallest
equivalence relation generated by the following rules.

\begin{mathpar}
\inferrule*[lab=Quote-drop]
{ }
{ \quotep{@{x}} \nameeq x }

\inferrule*[lab=Struct-equiv]
{ P \scong Q }
{ \quotep{P} \nameeq \quotep{Q} }
\end{mathpar}

The astute reader will have noticed that the mutual recursion of names
and processes imposes a mutual recursion on alpha-equivalence and
structural equivalence via name-equivalence. Fortunately, all of this
works out pleasantly and we may calculate in the natural way, free of
concern. The reader interested in the details is referred to the
appendix \ref{appendix:rho_details}.

\subsection{Substitution}

We use $\Proc$ for the set of processes, $\QProc$ for the set of
names, and $\id{\{}\vec{y} / \vec{x} \id{\}}$ to denote partial maps,
$s : \QProc \rightarrow \QProc$. A map, $s$ lifts, uniquely, to a map
on process terms, $\widehat{s} : \Proc \rightarrow \Proc$ by the
following equations.

\begin{mathpar}
  (0) \psubstp{Q}{P} := 0 \\
  (R \juxtap S) \psubstp{Q}{P}
  :=    
  (R)\psubstp{Q}{P} \juxtap (S) \psubstp{Q}{P} \\
  (x?(y).R) \psubstp{Q}{P}    
  :=    
  (x)\substp{Q}{P} (z)\concat( (R \psubstn{z}{y}) \psubstp{Q}{P} ) \\
  (\lift{x}{R}) \psubstp{Q}{P}  
  :=
  \lift{(x)\substp{Q}{P}}{ R \psubstp{Q}{P} } \\
%   (\dropn{x})  \psubstp{Q}{P}       
%   := 
%   \left\{ 
%     \begin{array}{ccc} 
%       \dropn{\quotep{Q}} & & x \nameeq \quotep{P} \\
%       \dropn{x} & & otherwise \\
%     \end{array}
%   \right. 
  (\dropn{x})  \psubstp{Q}{P}       
  := 
  \left\{ 
    \begin{array}{ccc} 
      Q & & x \nameeq \quotep{P} \\
      \dropn{x} & & otherwise \\
    \end{array}
  \right.
\end{mathpar}
 

where

\begin{eqnarray}
  (x)\id{\{} \lpquote Q \rpquote / \lpquote P \rpquote \id{\}}            = 
  \left\{ 
    \begin{array}{ccc}
      \lpquote Q \rpquote & & x \nameeq \lpquote P \rpquote \\
      x & & otherwise \\
    \end{array}
  \right. \nonumber
\end{eqnarray}

and $z$ is chosen distinct from $\quotep{P}$, $\quotep{Q}$, the free
names in $Q$, and all the names in $R$. Our $\alpha$-equivalence will
be built in the standard way from this substitution.

\begin{remark}\label{rem:no_self_referential_names}
  One consequence of these definitions is that $\forall P. \quotep{P}
  \not\in \freenames{P}$.
\end{remark}

\subsection{ Dynamic quote: an example }

Anticipating something of what's to come, consider applying the
substitution, $\widehat{\id{\{}u / z \id{\}}}$, to the following pair
of processes, $\lift{w}{y!(z)}$ and $w[ \lpquote y!(z) \rpquote ]$.

\begin{eqnarray}
	\lift{w}{y!(z)}\widehat{\id{\{}u / z \id{\}}}
		& = &
		\lift{w}{y!(u)} \nonumber\\
	w[ \lpquote y!(z) \rpquote ] \widehat{ \id{\{}u / z \id{\}} }
		& = &
		w[ \lpquote y!(z) \rpquote ] \nonumber
\end{eqnarray}

Because the body of the process between quotes is impervious to
substitution, we get radically different answers. In fact, by
examining the first process in an input context,
e.g. $x?(z).\lift{w}{y!(z)}$, we see that the process under the lift
operator may be shaped by prefixed inputs binding a name inside it. In
this sense, the lift operator will be seen as a way to dynamically
construct processes before reifying them as names.

Finally equipped with these standard features we can present the
dynamics of the calculus.

\subsubsection{Operational semantics} 

Finally, we introduce the computational dynamics. What marks these
algebras as distinct from other more traditionally studied algebraic
structures, e.g. vector spaces or polynomial rings, is the manner in
which dynamics is captured. In traditional structures, dynamics is typically
expressed through morphisms between such structures, as in linear maps
between vector spaces or morphisms between rings. In algebras
associated with the semantics of computation, the dynamics is
expressed as part of the algebraic structure itself, through a
reduction reduction relation typically denoted by $\red$. Below, we
give a recursive presentation of this relation for the calculus used
in the encoding.

$\red \subseteq \pi \times \pi$
$\red : \pi \to \mathcal{P}(\pi)$

\begin{mathpar}
  \inferrule* [lab=Comm] { \textsf{match}( x_{src}, x_{trgt} ) } { x_{trgt}?(y)P \; | \; x_{src}!\langle {Q} \rangle \red P\{\quotep{Q}/y}\} }
  \and \\
  \inferrule* [lab=Par] {{P} \red {P}'} {{{P} | {Q}} \red {{P}' | {Q}}}
  \and
  \inferrule* [lab=Equiv]{{{P} \scong {P}'} \andalso {{P}' \red {Q}'} \andalso {{Q}' \scong {Q}}}{{P} \red {Q}}
\end{mathpar}

\begin{eqnarray*}
  match_{\equiv} (\quotep{P},\quotep{Q}) & := & P \equiv Q \\
  match_{\dagger}(\quotep{P},\quotep{Q}) & := & \forall R. P|Q \red^{*} R => R \red^{*} 0 \\
  match_{K}(\quotep{P},\quotep{Q}) & := & K \mbox{ for some context } K
\end{eqnarray*}

$u?(x)P | u!\langle Q \rangle \red P\{\quotep{Q}/x\}$

%We write $\wred$ for $\red^*$, and $P\red$ if $\exists Q $ such that $ P \red Q$.
We write $P\red$ if $\exists Q $ such that $ P \red Q$ and $P\not\red$, otherwise.

\section{Replication}

As mentioned before, it is known that replication (and hence
recursion) can be implemented in a higher-order process algebra
\cite{SangiorgiWalker}. As our first example of calculation with the
machinery thus far presented we give the construction explicitly in
the {\rhoc}.

\begin{eqnarray}
	D_{x} & := & \prefix{x}{y}{(\binpar{\outputp{x}{y}}{@{y}})} \nonumber\\
	\bangp_{x}{P} & := & \binpar{{x}!\langle{\binpar{D_{x}}{P}}\rangle}{D_{x}} \nonumber
\end{eqnarray}

\begin{eqnarray}
	\bangp_{x}{P} & & \nonumber\\
	=
	& {x}!\langle{(\prefix{x}{y}{(\outputp{x}{y} | @{y})) | P}}\rangle 
	      | \prefix{x}{y}{(\outputp{x}{y} | @{y})} & \nonumber\\
	\red
	& (\outputp{x}{y} | @{y})\substn{\quotep{(\prefix{x}{y}{(@{y} | \outputp{x}{y})) | P}}}{y} & \nonumber\\
	=
	& \outputp{x}{\quotep{(\prefix{x}{y}{(\outputp{x}{y} | @{y})) | P}}}
	  | {(\prefix{x}{y}{(\outputp{x}{y} | @{y})) | P}} & \nonumber\\
	\red
	& \ldots & \nonumber\\
	\red^*
	& P | P | \ldots & \nonumber
\end{eqnarray}

Of course, this encoding, as an implementation, runs away, unfolding
$\bangp{P}$ eagerly. A lazier and more implementable replication
operator, restricted to input-guarded processes, may be obtained as follows.

\begin{eqnarray}
\bangp{\prefix{u}{v}{P}} 
	:= 
	\binpar{\lift{x}{\prefix{u}{v}{(\binpar{D(x)}{P})}}}{D(x)} \nonumber
\end{eqnarray}

\begin{remark}
  Note that the lazier definition still does not deal with summation
  or mixed summation (i.e. sums over input and output). The reader is
  invited to construct definitions of replication that deal with these
  features. 

  Further, the definitions are parameterized in a name, $x$. Can you,
  gentle reader, make a definition that eliminates this parameter and
  guarantees no accidental interaction between the replication
  machinery and the process being replicated -- i.e. no accidental
  sharing of names used by the process to get its work done and the
  name(s) used by the replication to effect copying. This latter
  revision of the definition of replication is crucial to obtaining
  the expected identity $!!P \sim !P$.
\end{remark}

\begin{remark}\label{rem:paradoxical_combinator}
  The reader familiar with the lambda calculus will have noticed the
  similarity between $D$ and the paradoxical combinator.

  [Ed. note: the existence of this seems to suggest we have to be more
  restrictive on the set of processes and names we admit if we are to
  support no-cloning.]
\end{remark}

\subsubsection{Bisimulation}

The computational dynamics gives rise to another kind of equivalence,
the equivalence of computational behavior. As previously mentioned
this is typically captured \emph{via} some form of bisimulation.

% The notion we use in this paper is weak barbed bisimulation
% \cite{milner91polyadicpi}.

The notion we use in this paper is derived from weak barbed
bisimulation \cite{milner91polyadicpi}. 

\begin{definition}
An \emph{observation relation}, $\downarrow_{\mathcal N}$, over a set
of names, $\mathcal N$, is the smallest relation satisfying the rules
below.

\infrule[Out-barb]{y \in {\mathcal N}, \; x \nameeq y}
		  {\outputp{x}{v} \downarrow_{\mathcal N} x}
\infrule[Par-barb]{\mbox{$P\downarrow_{\mathcal N} x$ or $Q\downarrow_{\mathcal N} x$}}
		  {\binpar{P}{Q} \downarrow_{\mathcal N} x}

We write $P \Downarrow_{\mathcal N} x$ if there is $Q$ such that 
$P \wred Q$ and $Q \downarrow_{\mathcal N} x$.
\end{definition}

\begin{definition}
%\label{def.bbisim}
An  ${\mathcal N}$-\emph{barbed bisimulation} over a set of names, ${\mathcal N}$, is a symmetric binary relation 
${\mathcal S}_{\mathcal N}$ between agents such that $P\rel{S}_{\mathcal N}Q$ implies:
\begin{enumerate}
\item If $P \red P'$ then $Q \wred Q'$ and $P'\rel{S}_{\mathcal N} Q'$.
\item If $P\downarrow_{\mathcal N} x$, then $Q\Downarrow_{\mathcal N} x$.
\end{enumerate}
$P$ is ${\mathcal N}$-barbed bisimilar to $Q$, written
$P \wbbisim_{\mathcal N} Q$, if $P \rel{S}_{\mathcal N} Q$ for some ${\mathcal N}$-barbed bisimulation ${\mathcal S}_{\mathcal N}$.
\end{definition}

$\mathcal{R} \subseteq \pi \times \pi$

$P \mathcal{R} Q => \forall P'. P \red P' \Rightarrow \exists Q'. Q \red Q', P' \mathcal{R} Q'$

$P \vdash x \Rightarrow Q \vdash x$

\begin{mathpar}
  \inferrule*[lab=Out-barb]{x \nameeq y}{{y}!\langle{Q}\rangle \vdash x}
  \and
  \inferrule*[lab=Par-barb]{\mbox{$P\vdash x$ or $Q\vdash x$}}{\binpar{P}{Q} \vdash x}
\end{mathpar}

\subsubsection{Contexts}

One of the principle advantages of computational calculi like the
$\pi$-calculus is a well-defined notion of context,
contextual-equivalence and a correlation between
contextual-equivalence and notions of bisimulation. The notion of
context allows the decomposition of a process into (sub-)process and
its syntactic environment, its context. Thus, a context may be
thought of as a process with a ``hole'' (written $\Box$) in it. The
application of a context $M$ to a process $P$, written $M[P]$, is
tantamount to filling the hole in $M$ with $P$. In this paper we do
not need the full weight of this theory, but do make use of the notion
of context in the proof the main theorem. 

\begin{mathpar}
  \inferrule* [lab=summation] {} {{M_{M},M_{N}} \bc \Box \;|\; x.M_{A} \;|\; M_{M}+M_{N}}
  \and
  \inferrule* [lab=agent] {} {{M_{A}} \bc (\vec{x})M_{P} \;| \; \clift{P_0,\ldots,M_{P},\ldots,P_N}}
  \and \\
  \inferrule* [lab=process] {} {{M_{P}} \bc M_{N} \;| \;P|M_{P} }
\end{mathpar} 

\begin{mathpar}
  \inferrule* [lab=sychronization] {} {M_{N} \bc \Box \;|\; x?M_{F} \;|\; x!M_{C}}
  \and
  \inferrule* [lab=abstraction] {} {{M_{F}} \bc (x)M_{P} }
  \and
  \inferrule* [lab=concretion] {} {{M_{C}} \bc \langle M_{P} \rangle }
  \and \\
  \inferrule* [lab=process] {} {{M_{P}} \bc M_{N} \;| \;P|M_{P} }
\end{mathpar}

\begin{definition}[contextual application] Given a context $M$, and
  process $P$, we define the \emph{contextual application}, $M[P] :=
  M\{P/\Box\}$. That is, the contextual application of M to P is the
  substitution of $P$ for $\Box$ in $M$.
\end{definition}

$\meaningof{-} : L \to \mathcal{P}(\pi)$

\begin{mathpar}
  \inferrule* [lab=collection] {} {\meaningof{true} = \pi, \and \meaningof{~E} = \pi \setminus \meaningof{E}, \and \meaningof{E_{1} \& E_{2}} = \meaningof{E_{1}} \cap \meaningof{E_{2}}}
\end{mathpar}

\begin{mathpar}
  \inferrule* [lab=structure] {} {\meaningof{0} = \{ P \in \pi | P \equiv 0 \}, \and \\ \meaningof{E_1 | E_2} = \{ P \in \pi | P \equiv P_{1} | P_{2}, P_{1} \in \meaningof{E_{1}}, P_{2} \in \meaningof{E_2}\} }
\end{mathpar}

\begin{mathpar}
 \inferrule* [lab=behavior] {} {\meaningof{\langle a?b \rangle E} = \{ P \in \pi | P \equiv Q | u?(y)P', \\ \and \\\\ \and \\ \;\;\; u \in \meaningof{a}, \forall z.P'\{z/y\} \in \meaningof{E\{z/b\}}\}, \and \\ \meaningof{a!E} = \{ P \in \pi | P \equiv Q | x!\langle P' \rangle, x \in \meaningof{a} P' \in \meaningof{E}\} }
\end{mathpar}

\begin{mathpar}
 \inferrule* [lab=nominal] {} {\meaningof{\quotep{E}} = \{ \quotep{P} \in \quotep{\pi} | P \in \meaningof{E} \}, \and \meaningof{\quotep{P}} = \{ \quotep{Q} \in \quotep{\pi} | P \equiv Q \} \and \\ \meaningof{@\quotep{E}} = \{ P \in \pi | P \equiv @x, x \in \meaningof{E} \}}
\end{mathpar}

\begin{eqnarray*}
  \\
  \meaningof{-} : TS \to ST
\end{eqnarray*}

\begin{eqnarray*}
  \\
  L : TS \to ST
\end{eqnarray*}

\begin{eqnarray*}
  \\
  P \models E \iff P \in \meaningof{E}
\end{eqnarray*}

\begin{eqnarray*}
  P \approx_{L} Q \iff \forall E \in L. P \models E \iff Q \models E
\end{eqnarray*}

\begin{eqnarray*}
  P \approx_{K} Q
\end{eqnarray*}

\begin{eqnarray*}
  P \approx Q
\end{eqnarray*}

$\approx_{K} = \approx = \approx_{L}$

\subsubsection{Contextual duality}

Note that contexts extend the quotation operation to a family of
operations from processes to names. Given a context, $M$, we can
define a \emph{nominal context}, $\quotep{M}$ by $\quotep{M}[P] :=
\quotep{M[P]}$. To foreshadow what is to come we observe that these
operations enjoy a duality with processes very much like the duality
between vectors and maps from vectors to scalars.

Further, because the calculus is essentially higher-order, we have a
correspondence between contexts and processes. More specifically,
given a name $x$ and a context $M$ we can construct $M^{*}_{x}$ such
that 

\begin{mathpar}
  M^{*}_{x} | \lift{x}{P} \red M[P]
\end{mathpar}

namely,

\begin{mathpar}
  M^{*}_{x} := x?(u).M[\dropn{u}]
\end{mathpar}

The dependence of $M^{*}_{x}$ on a name makes it an abstraction, 

\begin{mathpar}
  M^{*} := (x)x?(u).M[\dropn{u}]
\end{mathpar}

\subsection{Additional notation}

It will sometimes be convenient to denote the process a name
quotes. We already have the notation $x = \quotep{P}$, but it will be
convenient to introduce an alternate notation, $\procn{x}$, when we
want to emphasize the connection to the use of the name. Note that, by
virtue of name equivalence, $\quotep{\procn{x}} \nameeq x$; so, the
notation is consistent with previous definitions.

Further, because names have structure it is possible to effect
substitutions on the basis of that structure. This means we need to
upgrade our notation for substitutions, which we accomplish by
adapting comprehension notation. Thus,

\begin{mathpar}
  P\{ y / x : x \in S \}
\end{mathpar}

is interpreted to mean the process derived from P by replacing (in a
capture-avoiding manner) each occurrence of $x$ in $S$ by $y$. For example,

\begin{mathpar}
  P\{ \quotep{\procn{x}|\procn{x}} / x : x \in \freenames{P} \}
\end{mathpar}

will replace each (occurrence) of a free name $x$ in $P$ by
$\quotep{\procn{x}|\procn{x}}$.

Also, we will avail ourselves of the notation $x^{L}$ and $x^{R}$ to
denote injections of a name into disjoint copies of the name
space. There are numerous ways to accomplish this. One example can be
found in \cite{MeredithR05}. This notation overloads to vectors of
names: $\vec{x}^{\pi} := (x_{i}^{\pi} \; : \; 0 \leq i < |\vec{x}| )$ where $\pi \in \{L,R\}$.

We also use $P^{\Box} := P|\Box$.

In \cite{MeredithR05} an interpretation of the new operator is
given. It turns out that there are several possible interpretations
all enjoying the requisite algebraic properties of the operator (see
\cite{milner91polyadicpi}). We will therefore make liberal use of
$(\nu\; \vec{x})P$.

% subsection the_syntax_and_semantics_of_the_notation_system (end)   

\input{qm2pi.qmops} 

\input{qm2pi.sterngerlach} 

\input{qm2pi.metric} 

% section concurrent_process_calculi (end)

%\input{qm2pi.proofsketch}

% section proof sketch (end)

%\input{qm2pi.slviaknots} 

% section spatial logic via knots (end)

\input{qm2pi.conclusion}

% section conclusion (end)

%\input{qm2pi.dtcodes} 

% section wiring algorithm (end)

\input{qm2pi.ack} 

% section acknowledgments (end)

\newpage


\bibliographystyle{plain}   
\bibliography{../../biblios/main.bib}

\input{qm2pi.rhodetails}

\end{document}

 

% section wiring algorithm (end)

\documentclass[12pt]{llncs}
%\documentclass{jktr}

\usepackage[pdftex]{hyperref}                   
\usepackage {listings}
\usepackage {mathpartir}
\usepackage{bcprules}
%\usepackage{listings}
                       
\usepackage{graphicx} 
%\usepackage[margins=2.5cm,nohead,nofoot]{geometry}
%\usepackage{geometry}
\usepackage{amsfonts}
\usepackage{amstext}
\usepackage{latexsym}
\usepackage{amssymb}
\usepackage{color}


%\include{myPreamble}
\include{qm2pi.local} 

%\ifpdf
%\usepackage[pdftex]{graphicx}
%\else
%\usepackage{graphicx}
%\fi

 % \ifpdf
%  \usepackage{pdfsync}
%  \if


%\title{Brief Article}
%\author{David F. Snyder}
%\author{L.G. Meredith}

%\address{Dept. of Math., Texas State University--San Marcos, San Marcos, TX 78666}
       
\pagestyle{empty}


\begin{document}

\lstset{language=[Objective]Caml,frame=shadowbox}

\input{qm2pi.front}

% section front matter (end)

\input{qm2pi.intro} 
 
% section introduction (end)

% \input{qm2pi.knotations} 

% section notation (end)

\input{qm2pi.process.calculi} 

% section concurrent_process_calculi_and_spatial_logics_ (end)
    
%\input{qm2pi.knots2pi} 

%\input{qm2pi.trefoil} 

%\input{qm2pi.mainthm} 

% subsection basic_interpretation (end)

%\input{qm2pi.rho.presentation} 
\subsection{The syntax and semantics of the notation system}\label{sub:the_syntax_and_semantics_of_the_notation_system} % (fold)

We now summarize a technical presentation of the calculus that
embodies our theory of dynamics. The typical presentation of such a
calculus follows the style of giving generators and relations on
them. The grammar, below, describing term constructors, freely
generates the set of processes, $\Proc$. This set is then quotiented
by a relation known as structural congruence and it is over this set
that the notion of dynamics is expressed. This presentation is
essentially that of \cite{MeredithR05} with the addition of
polyadicity and summation. For readability we have relegated some of
the technical subtleties to an appendix.

\subsubsection{Process grammar}\label{subsub:process_grammar}

\begin{mathpar}
  \inferrule* [lab=synchronization] {} {{M} \bc \pzero \;|\; x?F \;|\; x!C }
  \and
  \inferrule* [lab=abstraction] {} {{F} \bc (x)P}
  \and
  \inferrule* [lab=concretion] {} {{C} \bc \langle Q \rangle}
  \and
  \inferrule* [lab=process] {} {{P,Q} \bc M \;| \;P|Q \;|\; @{x}}
  \and
  \inferrule* [lab=name] {} {{x} \bc \quotep{P}}
\end{mathpar} 

Note that $\vec{x}$ (resp. $\vec{P}$) denotes a vector of names
(resp. processes) of length $|\vec{x}|$ (resp. $|\vec{P}|$). We adopt
the following useful abbreviations.

\begin{mathpar}
   x?(\vec{y}).P := x.(\vec{y})P \and  x\clift{\vec{P}} := x.\clift{\vec{P}}
   \and x!(y) := \lift{x}{\dropn{y}}
   \and \Pi_{i=0}^{n-1}P_i := P_0 | \ldots | P_{n-1}
\end{mathpar}

\subsubsection{Structural congruence}

\paragraph{Free and bound names and alpha-equivalence.} At the
core of structural equivalence is alpha-equivalence which identifies
process that are the same up to a change of variable. Formally, we
recognize the distinction between free and bound names. The free names
of a process, $\freenames{P}$, may be calculated recursively as
follows:

\begin{mathpar}
\freenames{\pzero} := \emptyset
  \and \\
  \freenames{x?(y).P} := \{ x \} \cup (\freenames{P} \setminus \{ y \})
  \and 
  \freenames{x!\langle P \rangle} := \{ x \} \cup \{ P \} 
  \and \\
  \freenames{P|Q} := \freenames{P} \cup \freenames{Q}
  \and \\
  \freenames{@{x}} := \{ x \}
\end{mathpar}

$\pi$
$\quotep{\pi}$

$\freenames{-} : \pi \to \mathcal{P}(\quotep{\pi})$

\begin{eqnarray*}
  \freenames{\pzero} & := & \emptyset \\
  \freenames{x?(y).P} & := & \{ x \} \cup (\freenames{P} \setminus \{ y \}) \\
  \freenames{x!\langle P \rangle} & := & \{ x \} \cup \{ P \} \\
  \freenames{P|Q} & := & \freenames{P} \cup \freenames{Q} \\
  \freenames{\dropn{x}} & := & \{ x \}
\end{eqnarray*}

The bound names of a process, $\boundnames{P}$, are those names occurring in $P$
that are not free. For example, in $x?(y).0$, the name $x$ is free, while $y$ is bound.

\begin{mathpar}
  \inferrule* [lab=monoidal-laws] {} { P|Q \equiv Q|P \and P|0 \equiv P \and P|(Q|R) \equiv (P|Q)|R }
\end{mathpar}

\begin{mathpar}
  \inferrule* [lab=alpha-equivalence] {} { (x)P \equiv (y)P\{y/x\} \and y \not\in \freenames{P} }
\end{mathpar}

\begin{definition}
Then two processes, $P,Q$, are alpha-equivalent if $P = Q\{\vec{y}/\vec{x}\}$ for
some $\vec{x} \in \boundnames{Q},\vec{y} \in \boundnames{P}$, where $Q\{\vec{y}/\vec{x}\}$
denotes the capture-avoiding substitution of $\vec{y}$ for $\vec{x}$ in $Q$.
\end{definition}

\begin{definition}
  The {\em structural congruence} \cite{SangiorgiWalker} , $\equiv$,
  between processes is the least congruence containing
  alpha-equivalence, satisfying the abelian monoid laws
  (associativity, commutativity and $\pzero$ as identity) for parallel
  composition $|$ and for summation $+$.
\end{definition}

\subsection{Name equivalence}

We take name equivalence, written $\nameeq$, to be the smallest
equivalence relation generated by the following rules.

\begin{mathpar}
\inferrule*[lab=Quote-drop]
{ }
{ \quotep{@{x}} \nameeq x }

\inferrule*[lab=Struct-equiv]
{ P \scong Q }
{ \quotep{P} \nameeq \quotep{Q} }
\end{mathpar}

The astute reader will have noticed that the mutual recursion of names
and processes imposes a mutual recursion on alpha-equivalence and
structural equivalence via name-equivalence. Fortunately, all of this
works out pleasantly and we may calculate in the natural way, free of
concern. The reader interested in the details is referred to the
appendix \ref{appendix:rho_details}.

\subsection{Substitution}

We use $\Proc$ for the set of processes, $\QProc$ for the set of
names, and $\id{\{}\vec{y} / \vec{x} \id{\}}$ to denote partial maps,
$s : \QProc \rightarrow \QProc$. A map, $s$ lifts, uniquely, to a map
on process terms, $\widehat{s} : \Proc \rightarrow \Proc$ by the
following equations.

\begin{mathpar}
  (0) \psubstp{Q}{P} := 0 \\
  (R \juxtap S) \psubstp{Q}{P}
  :=    
  (R)\psubstp{Q}{P} \juxtap (S) \psubstp{Q}{P} \\
  (x?(y).R) \psubstp{Q}{P}    
  :=    
  (x)\substp{Q}{P} (z)\concat( (R \psubstn{z}{y}) \psubstp{Q}{P} ) \\
  (\lift{x}{R}) \psubstp{Q}{P}  
  :=
  \lift{(x)\substp{Q}{P}}{ R \psubstp{Q}{P} } \\
%   (\dropn{x})  \psubstp{Q}{P}       
%   := 
%   \left\{ 
%     \begin{array}{ccc} 
%       \dropn{\quotep{Q}} & & x \nameeq \quotep{P} \\
%       \dropn{x} & & otherwise \\
%     \end{array}
%   \right. 
  (\dropn{x})  \psubstp{Q}{P}       
  := 
  \left\{ 
    \begin{array}{ccc} 
      Q & & x \nameeq \quotep{P} \\
      \dropn{x} & & otherwise \\
    \end{array}
  \right.
\end{mathpar}
 

where

\begin{eqnarray}
  (x)\id{\{} \lpquote Q \rpquote / \lpquote P \rpquote \id{\}}            = 
  \left\{ 
    \begin{array}{ccc}
      \lpquote Q \rpquote & & x \nameeq \lpquote P \rpquote \\
      x & & otherwise \\
    \end{array}
  \right. \nonumber
\end{eqnarray}

and $z$ is chosen distinct from $\quotep{P}$, $\quotep{Q}$, the free
names in $Q$, and all the names in $R$. Our $\alpha$-equivalence will
be built in the standard way from this substitution.

\begin{remark}\label{rem:no_self_referential_names}
  One consequence of these definitions is that $\forall P. \quotep{P}
  \not\in \freenames{P}$.
\end{remark}

\subsection{ Dynamic quote: an example }

Anticipating something of what's to come, consider applying the
substitution, $\widehat{\id{\{}u / z \id{\}}}$, to the following pair
of processes, $\lift{w}{y!(z)}$ and $w[ \lpquote y!(z) \rpquote ]$.

\begin{eqnarray}
	\lift{w}{y!(z)}\widehat{\id{\{}u / z \id{\}}}
		& = &
		\lift{w}{y!(u)} \nonumber\\
	w[ \lpquote y!(z) \rpquote ] \widehat{ \id{\{}u / z \id{\}} }
		& = &
		w[ \lpquote y!(z) \rpquote ] \nonumber
\end{eqnarray}

Because the body of the process between quotes is impervious to
substitution, we get radically different answers. In fact, by
examining the first process in an input context,
e.g. $x?(z).\lift{w}{y!(z)}$, we see that the process under the lift
operator may be shaped by prefixed inputs binding a name inside it. In
this sense, the lift operator will be seen as a way to dynamically
construct processes before reifying them as names.

Finally equipped with these standard features we can present the
dynamics of the calculus.

\subsubsection{Operational semantics} 

Finally, we introduce the computational dynamics. What marks these
algebras as distinct from other more traditionally studied algebraic
structures, e.g. vector spaces or polynomial rings, is the manner in
which dynamics is captured. In traditional structures, dynamics is typically
expressed through morphisms between such structures, as in linear maps
between vector spaces or morphisms between rings. In algebras
associated with the semantics of computation, the dynamics is
expressed as part of the algebraic structure itself, through a
reduction reduction relation typically denoted by $\red$. Below, we
give a recursive presentation of this relation for the calculus used
in the encoding.

$\red \subseteq \pi \times \pi$
$\red : \pi \to \mathcal{P}(\pi)$

\begin{mathpar}
  \inferrule* [lab=Comm] { \textsf{match}( x_{src}, x_{trgt} ) } { x_{trgt}?(y)P \; | \; x_{src}!\langle {Q} \rangle \red P\{\quotep{Q}/y}\} }
  \and \\
  \inferrule* [lab=Par] {{P} \red {P}'} {{{P} | {Q}} \red {{P}' | {Q}}}
  \and
  \inferrule* [lab=Equiv]{{{P} \scong {P}'} \andalso {{P}' \red {Q}'} \andalso {{Q}' \scong {Q}}}{{P} \red {Q}}
\end{mathpar}

\begin{eqnarray*}
  match_{\equiv} (\quotep{P},\quotep{Q}) & := & P \equiv Q \\
  match_{\dagger}(\quotep{P},\quotep{Q}) & := & \forall R. P|Q \red^{*} R => R \red^{*} 0 \\
  match_{K}(\quotep{P},\quotep{Q}) & := & K \mbox{ for some context } K
\end{eqnarray*}

$u?(x)P | u!\langle Q \rangle \red P\{\quotep{Q}/x\}$

%We write $\wred$ for $\red^*$, and $P\red$ if $\exists Q $ such that $ P \red Q$.
We write $P\red$ if $\exists Q $ such that $ P \red Q$ and $P\not\red$, otherwise.

\section{Replication}

As mentioned before, it is known that replication (and hence
recursion) can be implemented in a higher-order process algebra
\cite{SangiorgiWalker}. As our first example of calculation with the
machinery thus far presented we give the construction explicitly in
the {\rhoc}.

\begin{eqnarray}
	D_{x} & := & \prefix{x}{y}{(\binpar{\outputp{x}{y}}{@{y}})} \nonumber\\
	\bangp_{x}{P} & := & \binpar{{x}!\langle{\binpar{D_{x}}{P}}\rangle}{D_{x}} \nonumber
\end{eqnarray}

\begin{eqnarray}
	\bangp_{x}{P} & & \nonumber\\
	=
	& {x}!\langle{(\prefix{x}{y}{(\outputp{x}{y} | @{y})) | P}}\rangle 
	      | \prefix{x}{y}{(\outputp{x}{y} | @{y})} & \nonumber\\
	\red
	& (\outputp{x}{y} | @{y})\substn{\quotep{(\prefix{x}{y}{(@{y} | \outputp{x}{y})) | P}}}{y} & \nonumber\\
	=
	& \outputp{x}{\quotep{(\prefix{x}{y}{(\outputp{x}{y} | @{y})) | P}}}
	  | {(\prefix{x}{y}{(\outputp{x}{y} | @{y})) | P}} & \nonumber\\
	\red
	& \ldots & \nonumber\\
	\red^*
	& P | P | \ldots & \nonumber
\end{eqnarray}

Of course, this encoding, as an implementation, runs away, unfolding
$\bangp{P}$ eagerly. A lazier and more implementable replication
operator, restricted to input-guarded processes, may be obtained as follows.

\begin{eqnarray}
\bangp{\prefix{u}{v}{P}} 
	:= 
	\binpar{\lift{x}{\prefix{u}{v}{(\binpar{D(x)}{P})}}}{D(x)} \nonumber
\end{eqnarray}

\begin{remark}
  Note that the lazier definition still does not deal with summation
  or mixed summation (i.e. sums over input and output). The reader is
  invited to construct definitions of replication that deal with these
  features. 

  Further, the definitions are parameterized in a name, $x$. Can you,
  gentle reader, make a definition that eliminates this parameter and
  guarantees no accidental interaction between the replication
  machinery and the process being replicated -- i.e. no accidental
  sharing of names used by the process to get its work done and the
  name(s) used by the replication to effect copying. This latter
  revision of the definition of replication is crucial to obtaining
  the expected identity $!!P \sim !P$.
\end{remark}

\begin{remark}\label{rem:paradoxical_combinator}
  The reader familiar with the lambda calculus will have noticed the
  similarity between $D$ and the paradoxical combinator.

  [Ed. note: the existence of this seems to suggest we have to be more
  restrictive on the set of processes and names we admit if we are to
  support no-cloning.]
\end{remark}

\subsubsection{Bisimulation}

The computational dynamics gives rise to another kind of equivalence,
the equivalence of computational behavior. As previously mentioned
this is typically captured \emph{via} some form of bisimulation.

% The notion we use in this paper is weak barbed bisimulation
% \cite{milner91polyadicpi}.

The notion we use in this paper is derived from weak barbed
bisimulation \cite{milner91polyadicpi}. 

\begin{definition}
An \emph{observation relation}, $\downarrow_{\mathcal N}$, over a set
of names, $\mathcal N$, is the smallest relation satisfying the rules
below.

\infrule[Out-barb]{y \in {\mathcal N}, \; x \nameeq y}
		  {\outputp{x}{v} \downarrow_{\mathcal N} x}
\infrule[Par-barb]{\mbox{$P\downarrow_{\mathcal N} x$ or $Q\downarrow_{\mathcal N} x$}}
		  {\binpar{P}{Q} \downarrow_{\mathcal N} x}

We write $P \Downarrow_{\mathcal N} x$ if there is $Q$ such that 
$P \wred Q$ and $Q \downarrow_{\mathcal N} x$.
\end{definition}

\begin{definition}
%\label{def.bbisim}
An  ${\mathcal N}$-\emph{barbed bisimulation} over a set of names, ${\mathcal N}$, is a symmetric binary relation 
${\mathcal S}_{\mathcal N}$ between agents such that $P\rel{S}_{\mathcal N}Q$ implies:
\begin{enumerate}
\item If $P \red P'$ then $Q \wred Q'$ and $P'\rel{S}_{\mathcal N} Q'$.
\item If $P\downarrow_{\mathcal N} x$, then $Q\Downarrow_{\mathcal N} x$.
\end{enumerate}
$P$ is ${\mathcal N}$-barbed bisimilar to $Q$, written
$P \wbbisim_{\mathcal N} Q$, if $P \rel{S}_{\mathcal N} Q$ for some ${\mathcal N}$-barbed bisimulation ${\mathcal S}_{\mathcal N}$.
\end{definition}

$\mathcal{R} \subseteq \pi \times \pi$

$P \mathcal{R} Q => \forall P'. P \red P' \Rightarrow \exists Q'. Q \red Q', P' \mathcal{R} Q'$

$P \vdash x \Rightarrow Q \vdash x$

\begin{mathpar}
  \inferrule*[lab=Out-barb]{x \nameeq y}{{y}!\langle{Q}\rangle \vdash x}
  \and
  \inferrule*[lab=Par-barb]{\mbox{$P\vdash x$ or $Q\vdash x$}}{\binpar{P}{Q} \vdash x}
\end{mathpar}

\subsubsection{Contexts}

One of the principle advantages of computational calculi like the
$\pi$-calculus is a well-defined notion of context,
contextual-equivalence and a correlation between
contextual-equivalence and notions of bisimulation. The notion of
context allows the decomposition of a process into (sub-)process and
its syntactic environment, its context. Thus, a context may be
thought of as a process with a ``hole'' (written $\Box$) in it. The
application of a context $M$ to a process $P$, written $M[P]$, is
tantamount to filling the hole in $M$ with $P$. In this paper we do
not need the full weight of this theory, but do make use of the notion
of context in the proof the main theorem. 

\begin{mathpar}
  \inferrule* [lab=summation] {} {{M_{M},M_{N}} \bc \Box \;|\; x.M_{A} \;|\; M_{M}+M_{N}}
  \and
  \inferrule* [lab=agent] {} {{M_{A}} \bc (\vec{x})M_{P} \;| \; \clift{P_0,\ldots,M_{P},\ldots,P_N}}
  \and \\
  \inferrule* [lab=process] {} {{M_{P}} \bc M_{N} \;| \;P|M_{P} }
\end{mathpar} 

\begin{mathpar}
  \inferrule* [lab=sychronization] {} {M_{N} \bc \Box \;|\; x?M_{F} \;|\; x!M_{C}}
  \and
  \inferrule* [lab=abstraction] {} {{M_{F}} \bc (x)M_{P} }
  \and
  \inferrule* [lab=concretion] {} {{M_{C}} \bc \langle M_{P} \rangle }
  \and \\
  \inferrule* [lab=process] {} {{M_{P}} \bc M_{N} \;| \;P|M_{P} }
\end{mathpar}

\begin{definition}[contextual application] Given a context $M$, and
  process $P$, we define the \emph{contextual application}, $M[P] :=
  M\{P/\Box\}$. That is, the contextual application of M to P is the
  substitution of $P$ for $\Box$ in $M$.
\end{definition}

$\meaningof{-} : L \to \mathcal{P}(\pi)$

\begin{mathpar}
  \inferrule* [lab=collection] {} {\meaningof{true} = \pi, \and \meaningof{~E} = \pi \setminus \meaningof{E}, \and \meaningof{E_{1} \& E_{2}} = \meaningof{E_{1}} \cap \meaningof{E_{2}}}
\end{mathpar}

\begin{mathpar}
  \inferrule* [lab=structure] {} {\meaningof{0} = \{ P \in \pi | P \equiv 0 \}, \and \\ \meaningof{E_1 | E_2} = \{ P \in \pi | P \equiv P_{1} | P_{2}, P_{1} \in \meaningof{E_{1}}, P_{2} \in \meaningof{E_2}\} }
\end{mathpar}

\begin{mathpar}
 \inferrule* [lab=behavior] {} {\meaningof{\langle a?b \rangle E} = \{ P \in \pi | P \equiv Q | u?(y)P', \\ \and \\\\ \and \\ \;\;\; u \in \meaningof{a}, \forall z.P'\{z/y\} \in \meaningof{E\{z/b\}}\}, \and \\ \meaningof{a!E} = \{ P \in \pi | P \equiv Q | x!\langle P' \rangle, x \in \meaningof{a} P' \in \meaningof{E}\} }
\end{mathpar}

\begin{mathpar}
 \inferrule* [lab=nominal] {} {\meaningof{\quotep{E}} = \{ \quotep{P} \in \quotep{\pi} | P \in \meaningof{E} \}, \and \meaningof{\quotep{P}} = \{ \quotep{Q} \in \quotep{\pi} | P \equiv Q \} \and \\ \meaningof{@\quotep{E}} = \{ P \in \pi | P \equiv @x, x \in \meaningof{E} \}}
\end{mathpar}

\begin{eqnarray*}
  \\
  \meaningof{-} : TS \to ST
\end{eqnarray*}

\begin{eqnarray*}
  \\
  L : TS \to ST
\end{eqnarray*}

\begin{eqnarray*}
  \\
  P \models E \iff P \in \meaningof{E}
\end{eqnarray*}

\begin{eqnarray*}
  P \approx_{L} Q \iff \forall E \in L. P \models E \iff Q \models E
\end{eqnarray*}

\begin{eqnarray*}
  P \approx_{K} Q
\end{eqnarray*}

\begin{eqnarray*}
  P \approx Q
\end{eqnarray*}

$\approx_{K} = \approx = \approx_{L}$

\subsubsection{Contextual duality}

Note that contexts extend the quotation operation to a family of
operations from processes to names. Given a context, $M$, we can
define a \emph{nominal context}, $\quotep{M}$ by $\quotep{M}[P] :=
\quotep{M[P]}$. To foreshadow what is to come we observe that these
operations enjoy a duality with processes very much like the duality
between vectors and maps from vectors to scalars.

Further, because the calculus is essentially higher-order, we have a
correspondence between contexts and processes. More specifically,
given a name $x$ and a context $M$ we can construct $M^{*}_{x}$ such
that 

\begin{mathpar}
  M^{*}_{x} | \lift{x}{P} \red M[P]
\end{mathpar}

namely,

\begin{mathpar}
  M^{*}_{x} := x?(u).M[\dropn{u}]
\end{mathpar}

The dependence of $M^{*}_{x}$ on a name makes it an abstraction, 

\begin{mathpar}
  M^{*} := (x)x?(u).M[\dropn{u}]
\end{mathpar}

\subsection{Additional notation}

It will sometimes be convenient to denote the process a name
quotes. We already have the notation $x = \quotep{P}$, but it will be
convenient to introduce an alternate notation, $\procn{x}$, when we
want to emphasize the connection to the use of the name. Note that, by
virtue of name equivalence, $\quotep{\procn{x}} \nameeq x$; so, the
notation is consistent with previous definitions.

Further, because names have structure it is possible to effect
substitutions on the basis of that structure. This means we need to
upgrade our notation for substitutions, which we accomplish by
adapting comprehension notation. Thus,

\begin{mathpar}
  P\{ y / x : x \in S \}
\end{mathpar}

is interpreted to mean the process derived from P by replacing (in a
capture-avoiding manner) each occurrence of $x$ in $S$ by $y$. For example,

\begin{mathpar}
  P\{ \quotep{\procn{x}|\procn{x}} / x : x \in \freenames{P} \}
\end{mathpar}

will replace each (occurrence) of a free name $x$ in $P$ by
$\quotep{\procn{x}|\procn{x}}$.

Also, we will avail ourselves of the notation $x^{L}$ and $x^{R}$ to
denote injections of a name into disjoint copies of the name
space. There are numerous ways to accomplish this. One example can be
found in \cite{MeredithR05}. This notation overloads to vectors of
names: $\vec{x}^{\pi} := (x_{i}^{\pi} \; : \; 0 \leq i < |\vec{x}| )$ where $\pi \in \{L,R\}$.

We also use $P^{\Box} := P|\Box$.

In \cite{MeredithR05} an interpretation of the new operator is
given. It turns out that there are several possible interpretations
all enjoying the requisite algebraic properties of the operator (see
\cite{milner91polyadicpi}). We will therefore make liberal use of
$(\nu\; \vec{x})P$.

% subsection the_syntax_and_semantics_of_the_notation_system (end)   

\input{qm2pi.qmops} 

\input{qm2pi.sterngerlach} 

\input{qm2pi.metric} 

% section concurrent_process_calculi (end)

%\input{qm2pi.proofsketch}

% section proof sketch (end)

%\input{qm2pi.slviaknots} 

% section spatial logic via knots (end)

\input{qm2pi.conclusion}

% section conclusion (end)

%\input{qm2pi.dtcodes} 

% section wiring algorithm (end)

\input{qm2pi.ack} 

% section acknowledgments (end)

\newpage


\bibliographystyle{plain}   
\bibliography{../../biblios/main.bib}

\input{qm2pi.rhodetails}

\end{document}

 

% section acknowledgments (end)

\newpage


\bibliographystyle{plain}   
\bibliography{../../biblios/main.bib}

\documentclass[12pt]{llncs}
%\documentclass{jktr}

\usepackage[pdftex]{hyperref}                   
\usepackage {listings}
\usepackage {mathpartir}
\usepackage{bcprules}
%\usepackage{listings}
                       
\usepackage{graphicx} 
%\usepackage[margins=2.5cm,nohead,nofoot]{geometry}
%\usepackage{geometry}
\usepackage{amsfonts}
\usepackage{amstext}
\usepackage{latexsym}
\usepackage{amssymb}
\usepackage{color}


%\include{myPreamble}
\include{qm2pi.local} 

%\ifpdf
%\usepackage[pdftex]{graphicx}
%\else
%\usepackage{graphicx}
%\fi

 % \ifpdf
%  \usepackage{pdfsync}
%  \if


%\title{Brief Article}
%\author{David F. Snyder}
%\author{L.G. Meredith}

%\address{Dept. of Math., Texas State University--San Marcos, San Marcos, TX 78666}
       
\pagestyle{empty}


\begin{document}

\lstset{language=[Objective]Caml,frame=shadowbox}

\input{qm2pi.front}

% section front matter (end)

\input{qm2pi.intro} 
 
% section introduction (end)

% \input{qm2pi.knotations} 

% section notation (end)

\input{qm2pi.process.calculi} 

% section concurrent_process_calculi_and_spatial_logics_ (end)
    
%\input{qm2pi.knots2pi} 

%\input{qm2pi.trefoil} 

%\input{qm2pi.mainthm} 

% subsection basic_interpretation (end)

%\input{qm2pi.rho.presentation} 
\subsection{The syntax and semantics of the notation system}\label{sub:the_syntax_and_semantics_of_the_notation_system} % (fold)

We now summarize a technical presentation of the calculus that
embodies our theory of dynamics. The typical presentation of such a
calculus follows the style of giving generators and relations on
them. The grammar, below, describing term constructors, freely
generates the set of processes, $\Proc$. This set is then quotiented
by a relation known as structural congruence and it is over this set
that the notion of dynamics is expressed. This presentation is
essentially that of \cite{MeredithR05} with the addition of
polyadicity and summation. For readability we have relegated some of
the technical subtleties to an appendix.

\subsubsection{Process grammar}\label{subsub:process_grammar}

\begin{mathpar}
  \inferrule* [lab=synchronization] {} {{M} \bc \pzero \;|\; x?F \;|\; x!C }
  \and
  \inferrule* [lab=abstraction] {} {{F} \bc (x)P}
  \and
  \inferrule* [lab=concretion] {} {{C} \bc \langle Q \rangle}
  \and
  \inferrule* [lab=process] {} {{P,Q} \bc M \;| \;P|Q \;|\; @{x}}
  \and
  \inferrule* [lab=name] {} {{x} \bc \quotep{P}}
\end{mathpar} 

Note that $\vec{x}$ (resp. $\vec{P}$) denotes a vector of names
(resp. processes) of length $|\vec{x}|$ (resp. $|\vec{P}|$). We adopt
the following useful abbreviations.

\begin{mathpar}
   x?(\vec{y}).P := x.(\vec{y})P \and  x\clift{\vec{P}} := x.\clift{\vec{P}}
   \and x!(y) := \lift{x}{\dropn{y}}
   \and \Pi_{i=0}^{n-1}P_i := P_0 | \ldots | P_{n-1}
\end{mathpar}

\subsubsection{Structural congruence}

\paragraph{Free and bound names and alpha-equivalence.} At the
core of structural equivalence is alpha-equivalence which identifies
process that are the same up to a change of variable. Formally, we
recognize the distinction between free and bound names. The free names
of a process, $\freenames{P}$, may be calculated recursively as
follows:

\begin{mathpar}
\freenames{\pzero} := \emptyset
  \and \\
  \freenames{x?(y).P} := \{ x \} \cup (\freenames{P} \setminus \{ y \})
  \and 
  \freenames{x!\langle P \rangle} := \{ x \} \cup \{ P \} 
  \and \\
  \freenames{P|Q} := \freenames{P} \cup \freenames{Q}
  \and \\
  \freenames{@{x}} := \{ x \}
\end{mathpar}

$\pi$
$\quotep{\pi}$

$\freenames{-} : \pi \to \mathcal{P}(\quotep{\pi})$

\begin{eqnarray*}
  \freenames{\pzero} & := & \emptyset \\
  \freenames{x?(y).P} & := & \{ x \} \cup (\freenames{P} \setminus \{ y \}) \\
  \freenames{x!\langle P \rangle} & := & \{ x \} \cup \{ P \} \\
  \freenames{P|Q} & := & \freenames{P} \cup \freenames{Q} \\
  \freenames{\dropn{x}} & := & \{ x \}
\end{eqnarray*}

The bound names of a process, $\boundnames{P}$, are those names occurring in $P$
that are not free. For example, in $x?(y).0$, the name $x$ is free, while $y$ is bound.

\begin{mathpar}
  \inferrule* [lab=monoidal-laws] {} { P|Q \equiv Q|P \and P|0 \equiv P \and P|(Q|R) \equiv (P|Q)|R }
\end{mathpar}

\begin{mathpar}
  \inferrule* [lab=alpha-equivalence] {} { (x)P \equiv (y)P\{y/x\} \and y \not\in \freenames{P} }
\end{mathpar}

\begin{definition}
Then two processes, $P,Q$, are alpha-equivalent if $P = Q\{\vec{y}/\vec{x}\}$ for
some $\vec{x} \in \boundnames{Q},\vec{y} \in \boundnames{P}$, where $Q\{\vec{y}/\vec{x}\}$
denotes the capture-avoiding substitution of $\vec{y}$ for $\vec{x}$ in $Q$.
\end{definition}

\begin{definition}
  The {\em structural congruence} \cite{SangiorgiWalker} , $\equiv$,
  between processes is the least congruence containing
  alpha-equivalence, satisfying the abelian monoid laws
  (associativity, commutativity and $\pzero$ as identity) for parallel
  composition $|$ and for summation $+$.
\end{definition}

\subsection{Name equivalence}

We take name equivalence, written $\nameeq$, to be the smallest
equivalence relation generated by the following rules.

\begin{mathpar}
\inferrule*[lab=Quote-drop]
{ }
{ \quotep{@{x}} \nameeq x }

\inferrule*[lab=Struct-equiv]
{ P \scong Q }
{ \quotep{P} \nameeq \quotep{Q} }
\end{mathpar}

The astute reader will have noticed that the mutual recursion of names
and processes imposes a mutual recursion on alpha-equivalence and
structural equivalence via name-equivalence. Fortunately, all of this
works out pleasantly and we may calculate in the natural way, free of
concern. The reader interested in the details is referred to the
appendix \ref{appendix:rho_details}.

\subsection{Substitution}

We use $\Proc$ for the set of processes, $\QProc$ for the set of
names, and $\id{\{}\vec{y} / \vec{x} \id{\}}$ to denote partial maps,
$s : \QProc \rightarrow \QProc$. A map, $s$ lifts, uniquely, to a map
on process terms, $\widehat{s} : \Proc \rightarrow \Proc$ by the
following equations.

\begin{mathpar}
  (0) \psubstp{Q}{P} := 0 \\
  (R \juxtap S) \psubstp{Q}{P}
  :=    
  (R)\psubstp{Q}{P} \juxtap (S) \psubstp{Q}{P} \\
  (x?(y).R) \psubstp{Q}{P}    
  :=    
  (x)\substp{Q}{P} (z)\concat( (R \psubstn{z}{y}) \psubstp{Q}{P} ) \\
  (\lift{x}{R}) \psubstp{Q}{P}  
  :=
  \lift{(x)\substp{Q}{P}}{ R \psubstp{Q}{P} } \\
%   (\dropn{x})  \psubstp{Q}{P}       
%   := 
%   \left\{ 
%     \begin{array}{ccc} 
%       \dropn{\quotep{Q}} & & x \nameeq \quotep{P} \\
%       \dropn{x} & & otherwise \\
%     \end{array}
%   \right. 
  (\dropn{x})  \psubstp{Q}{P}       
  := 
  \left\{ 
    \begin{array}{ccc} 
      Q & & x \nameeq \quotep{P} \\
      \dropn{x} & & otherwise \\
    \end{array}
  \right.
\end{mathpar}
 

where

\begin{eqnarray}
  (x)\id{\{} \lpquote Q \rpquote / \lpquote P \rpquote \id{\}}            = 
  \left\{ 
    \begin{array}{ccc}
      \lpquote Q \rpquote & & x \nameeq \lpquote P \rpquote \\
      x & & otherwise \\
    \end{array}
  \right. \nonumber
\end{eqnarray}

and $z$ is chosen distinct from $\quotep{P}$, $\quotep{Q}$, the free
names in $Q$, and all the names in $R$. Our $\alpha$-equivalence will
be built in the standard way from this substitution.

\begin{remark}\label{rem:no_self_referential_names}
  One consequence of these definitions is that $\forall P. \quotep{P}
  \not\in \freenames{P}$.
\end{remark}

\subsection{ Dynamic quote: an example }

Anticipating something of what's to come, consider applying the
substitution, $\widehat{\id{\{}u / z \id{\}}}$, to the following pair
of processes, $\lift{w}{y!(z)}$ and $w[ \lpquote y!(z) \rpquote ]$.

\begin{eqnarray}
	\lift{w}{y!(z)}\widehat{\id{\{}u / z \id{\}}}
		& = &
		\lift{w}{y!(u)} \nonumber\\
	w[ \lpquote y!(z) \rpquote ] \widehat{ \id{\{}u / z \id{\}} }
		& = &
		w[ \lpquote y!(z) \rpquote ] \nonumber
\end{eqnarray}

Because the body of the process between quotes is impervious to
substitution, we get radically different answers. In fact, by
examining the first process in an input context,
e.g. $x?(z).\lift{w}{y!(z)}$, we see that the process under the lift
operator may be shaped by prefixed inputs binding a name inside it. In
this sense, the lift operator will be seen as a way to dynamically
construct processes before reifying them as names.

Finally equipped with these standard features we can present the
dynamics of the calculus.

\subsubsection{Operational semantics} 

Finally, we introduce the computational dynamics. What marks these
algebras as distinct from other more traditionally studied algebraic
structures, e.g. vector spaces or polynomial rings, is the manner in
which dynamics is captured. In traditional structures, dynamics is typically
expressed through morphisms between such structures, as in linear maps
between vector spaces or morphisms between rings. In algebras
associated with the semantics of computation, the dynamics is
expressed as part of the algebraic structure itself, through a
reduction reduction relation typically denoted by $\red$. Below, we
give a recursive presentation of this relation for the calculus used
in the encoding.

$\red \subseteq \pi \times \pi$
$\red : \pi \to \mathcal{P}(\pi)$

\begin{mathpar}
  \inferrule* [lab=Comm] { \textsf{match}( x_{src}, x_{trgt} ) } { x_{trgt}?(y)P \; | \; x_{src}!\langle {Q} \rangle \red P\{\quotep{Q}/y}\} }
  \and \\
  \inferrule* [lab=Par] {{P} \red {P}'} {{{P} | {Q}} \red {{P}' | {Q}}}
  \and
  \inferrule* [lab=Equiv]{{{P} \scong {P}'} \andalso {{P}' \red {Q}'} \andalso {{Q}' \scong {Q}}}{{P} \red {Q}}
\end{mathpar}

\begin{eqnarray*}
  match_{\equiv} (\quotep{P},\quotep{Q}) & := & P \equiv Q \\
  match_{\dagger}(\quotep{P},\quotep{Q}) & := & \forall R. P|Q \red^{*} R => R \red^{*} 0 \\
  match_{K}(\quotep{P},\quotep{Q}) & := & K \mbox{ for some context } K
\end{eqnarray*}

$u?(x)P | u!\langle Q \rangle \red P\{\quotep{Q}/x\}$

%We write $\wred$ for $\red^*$, and $P\red$ if $\exists Q $ such that $ P \red Q$.
We write $P\red$ if $\exists Q $ such that $ P \red Q$ and $P\not\red$, otherwise.

\section{Replication}

As mentioned before, it is known that replication (and hence
recursion) can be implemented in a higher-order process algebra
\cite{SangiorgiWalker}. As our first example of calculation with the
machinery thus far presented we give the construction explicitly in
the {\rhoc}.

\begin{eqnarray}
	D_{x} & := & \prefix{x}{y}{(\binpar{\outputp{x}{y}}{@{y}})} \nonumber\\
	\bangp_{x}{P} & := & \binpar{{x}!\langle{\binpar{D_{x}}{P}}\rangle}{D_{x}} \nonumber
\end{eqnarray}

\begin{eqnarray}
	\bangp_{x}{P} & & \nonumber\\
	=
	& {x}!\langle{(\prefix{x}{y}{(\outputp{x}{y} | @{y})) | P}}\rangle 
	      | \prefix{x}{y}{(\outputp{x}{y} | @{y})} & \nonumber\\
	\red
	& (\outputp{x}{y} | @{y})\substn{\quotep{(\prefix{x}{y}{(@{y} | \outputp{x}{y})) | P}}}{y} & \nonumber\\
	=
	& \outputp{x}{\quotep{(\prefix{x}{y}{(\outputp{x}{y} | @{y})) | P}}}
	  | {(\prefix{x}{y}{(\outputp{x}{y} | @{y})) | P}} & \nonumber\\
	\red
	& \ldots & \nonumber\\
	\red^*
	& P | P | \ldots & \nonumber
\end{eqnarray}

Of course, this encoding, as an implementation, runs away, unfolding
$\bangp{P}$ eagerly. A lazier and more implementable replication
operator, restricted to input-guarded processes, may be obtained as follows.

\begin{eqnarray}
\bangp{\prefix{u}{v}{P}} 
	:= 
	\binpar{\lift{x}{\prefix{u}{v}{(\binpar{D(x)}{P})}}}{D(x)} \nonumber
\end{eqnarray}

\begin{remark}
  Note that the lazier definition still does not deal with summation
  or mixed summation (i.e. sums over input and output). The reader is
  invited to construct definitions of replication that deal with these
  features. 

  Further, the definitions are parameterized in a name, $x$. Can you,
  gentle reader, make a definition that eliminates this parameter and
  guarantees no accidental interaction between the replication
  machinery and the process being replicated -- i.e. no accidental
  sharing of names used by the process to get its work done and the
  name(s) used by the replication to effect copying. This latter
  revision of the definition of replication is crucial to obtaining
  the expected identity $!!P \sim !P$.
\end{remark}

\begin{remark}\label{rem:paradoxical_combinator}
  The reader familiar with the lambda calculus will have noticed the
  similarity between $D$ and the paradoxical combinator.

  [Ed. note: the existence of this seems to suggest we have to be more
  restrictive on the set of processes and names we admit if we are to
  support no-cloning.]
\end{remark}

\subsubsection{Bisimulation}

The computational dynamics gives rise to another kind of equivalence,
the equivalence of computational behavior. As previously mentioned
this is typically captured \emph{via} some form of bisimulation.

% The notion we use in this paper is weak barbed bisimulation
% \cite{milner91polyadicpi}.

The notion we use in this paper is derived from weak barbed
bisimulation \cite{milner91polyadicpi}. 

\begin{definition}
An \emph{observation relation}, $\downarrow_{\mathcal N}$, over a set
of names, $\mathcal N$, is the smallest relation satisfying the rules
below.

\infrule[Out-barb]{y \in {\mathcal N}, \; x \nameeq y}
		  {\outputp{x}{v} \downarrow_{\mathcal N} x}
\infrule[Par-barb]{\mbox{$P\downarrow_{\mathcal N} x$ or $Q\downarrow_{\mathcal N} x$}}
		  {\binpar{P}{Q} \downarrow_{\mathcal N} x}

We write $P \Downarrow_{\mathcal N} x$ if there is $Q$ such that 
$P \wred Q$ and $Q \downarrow_{\mathcal N} x$.
\end{definition}

\begin{definition}
%\label{def.bbisim}
An  ${\mathcal N}$-\emph{barbed bisimulation} over a set of names, ${\mathcal N}$, is a symmetric binary relation 
${\mathcal S}_{\mathcal N}$ between agents such that $P\rel{S}_{\mathcal N}Q$ implies:
\begin{enumerate}
\item If $P \red P'$ then $Q \wred Q'$ and $P'\rel{S}_{\mathcal N} Q'$.
\item If $P\downarrow_{\mathcal N} x$, then $Q\Downarrow_{\mathcal N} x$.
\end{enumerate}
$P$ is ${\mathcal N}$-barbed bisimilar to $Q$, written
$P \wbbisim_{\mathcal N} Q$, if $P \rel{S}_{\mathcal N} Q$ for some ${\mathcal N}$-barbed bisimulation ${\mathcal S}_{\mathcal N}$.
\end{definition}

$\mathcal{R} \subseteq \pi \times \pi$

$P \mathcal{R} Q => \forall P'. P \red P' \Rightarrow \exists Q'. Q \red Q', P' \mathcal{R} Q'$

$P \vdash x \Rightarrow Q \vdash x$

\begin{mathpar}
  \inferrule*[lab=Out-barb]{x \nameeq y}{{y}!\langle{Q}\rangle \vdash x}
  \and
  \inferrule*[lab=Par-barb]{\mbox{$P\vdash x$ or $Q\vdash x$}}{\binpar{P}{Q} \vdash x}
\end{mathpar}

\subsubsection{Contexts}

One of the principle advantages of computational calculi like the
$\pi$-calculus is a well-defined notion of context,
contextual-equivalence and a correlation between
contextual-equivalence and notions of bisimulation. The notion of
context allows the decomposition of a process into (sub-)process and
its syntactic environment, its context. Thus, a context may be
thought of as a process with a ``hole'' (written $\Box$) in it. The
application of a context $M$ to a process $P$, written $M[P]$, is
tantamount to filling the hole in $M$ with $P$. In this paper we do
not need the full weight of this theory, but do make use of the notion
of context in the proof the main theorem. 

\begin{mathpar}
  \inferrule* [lab=summation] {} {{M_{M},M_{N}} \bc \Box \;|\; x.M_{A} \;|\; M_{M}+M_{N}}
  \and
  \inferrule* [lab=agent] {} {{M_{A}} \bc (\vec{x})M_{P} \;| \; \clift{P_0,\ldots,M_{P},\ldots,P_N}}
  \and \\
  \inferrule* [lab=process] {} {{M_{P}} \bc M_{N} \;| \;P|M_{P} }
\end{mathpar} 

\begin{mathpar}
  \inferrule* [lab=sychronization] {} {M_{N} \bc \Box \;|\; x?M_{F} \;|\; x!M_{C}}
  \and
  \inferrule* [lab=abstraction] {} {{M_{F}} \bc (x)M_{P} }
  \and
  \inferrule* [lab=concretion] {} {{M_{C}} \bc \langle M_{P} \rangle }
  \and \\
  \inferrule* [lab=process] {} {{M_{P}} \bc M_{N} \;| \;P|M_{P} }
\end{mathpar}

\begin{definition}[contextual application] Given a context $M$, and
  process $P$, we define the \emph{contextual application}, $M[P] :=
  M\{P/\Box\}$. That is, the contextual application of M to P is the
  substitution of $P$ for $\Box$ in $M$.
\end{definition}

$\meaningof{-} : L \to \mathcal{P}(\pi)$

\begin{mathpar}
  \inferrule* [lab=collection] {} {\meaningof{true} = \pi, \and \meaningof{~E} = \pi \setminus \meaningof{E}, \and \meaningof{E_{1} \& E_{2}} = \meaningof{E_{1}} \cap \meaningof{E_{2}}}
\end{mathpar}

\begin{mathpar}
  \inferrule* [lab=structure] {} {\meaningof{0} = \{ P \in \pi | P \equiv 0 \}, \and \\ \meaningof{E_1 | E_2} = \{ P \in \pi | P \equiv P_{1} | P_{2}, P_{1} \in \meaningof{E_{1}}, P_{2} \in \meaningof{E_2}\} }
\end{mathpar}

\begin{mathpar}
 \inferrule* [lab=behavior] {} {\meaningof{\langle a?b \rangle E} = \{ P \in \pi | P \equiv Q | u?(y)P', \\ \and \\\\ \and \\ \;\;\; u \in \meaningof{a}, \forall z.P'\{z/y\} \in \meaningof{E\{z/b\}}\}, \and \\ \meaningof{a!E} = \{ P \in \pi | P \equiv Q | x!\langle P' \rangle, x \in \meaningof{a} P' \in \meaningof{E}\} }
\end{mathpar}

\begin{mathpar}
 \inferrule* [lab=nominal] {} {\meaningof{\quotep{E}} = \{ \quotep{P} \in \quotep{\pi} | P \in \meaningof{E} \}, \and \meaningof{\quotep{P}} = \{ \quotep{Q} \in \quotep{\pi} | P \equiv Q \} \and \\ \meaningof{@\quotep{E}} = \{ P \in \pi | P \equiv @x, x \in \meaningof{E} \}}
\end{mathpar}

\begin{eqnarray*}
  \\
  \meaningof{-} : TS \to ST
\end{eqnarray*}

\begin{eqnarray*}
  \\
  L : TS \to ST
\end{eqnarray*}

\begin{eqnarray*}
  \\
  P \models E \iff P \in \meaningof{E}
\end{eqnarray*}

\begin{eqnarray*}
  P \approx_{L} Q \iff \forall E \in L. P \models E \iff Q \models E
\end{eqnarray*}

\begin{eqnarray*}
  P \approx_{K} Q
\end{eqnarray*}

\begin{eqnarray*}
  P \approx Q
\end{eqnarray*}

$\approx_{K} = \approx = \approx_{L}$

\subsubsection{Contextual duality}

Note that contexts extend the quotation operation to a family of
operations from processes to names. Given a context, $M$, we can
define a \emph{nominal context}, $\quotep{M}$ by $\quotep{M}[P] :=
\quotep{M[P]}$. To foreshadow what is to come we observe that these
operations enjoy a duality with processes very much like the duality
between vectors and maps from vectors to scalars.

Further, because the calculus is essentially higher-order, we have a
correspondence between contexts and processes. More specifically,
given a name $x$ and a context $M$ we can construct $M^{*}_{x}$ such
that 

\begin{mathpar}
  M^{*}_{x} | \lift{x}{P} \red M[P]
\end{mathpar}

namely,

\begin{mathpar}
  M^{*}_{x} := x?(u).M[\dropn{u}]
\end{mathpar}

The dependence of $M^{*}_{x}$ on a name makes it an abstraction, 

\begin{mathpar}
  M^{*} := (x)x?(u).M[\dropn{u}]
\end{mathpar}

\subsection{Additional notation}

It will sometimes be convenient to denote the process a name
quotes. We already have the notation $x = \quotep{P}$, but it will be
convenient to introduce an alternate notation, $\procn{x}$, when we
want to emphasize the connection to the use of the name. Note that, by
virtue of name equivalence, $\quotep{\procn{x}} \nameeq x$; so, the
notation is consistent with previous definitions.

Further, because names have structure it is possible to effect
substitutions on the basis of that structure. This means we need to
upgrade our notation for substitutions, which we accomplish by
adapting comprehension notation. Thus,

\begin{mathpar}
  P\{ y / x : x \in S \}
\end{mathpar}

is interpreted to mean the process derived from P by replacing (in a
capture-avoiding manner) each occurrence of $x$ in $S$ by $y$. For example,

\begin{mathpar}
  P\{ \quotep{\procn{x}|\procn{x}} / x : x \in \freenames{P} \}
\end{mathpar}

will replace each (occurrence) of a free name $x$ in $P$ by
$\quotep{\procn{x}|\procn{x}}$.

Also, we will avail ourselves of the notation $x^{L}$ and $x^{R}$ to
denote injections of a name into disjoint copies of the name
space. There are numerous ways to accomplish this. One example can be
found in \cite{MeredithR05}. This notation overloads to vectors of
names: $\vec{x}^{\pi} := (x_{i}^{\pi} \; : \; 0 \leq i < |\vec{x}| )$ where $\pi \in \{L,R\}$.

We also use $P^{\Box} := P|\Box$.

In \cite{MeredithR05} an interpretation of the new operator is
given. It turns out that there are several possible interpretations
all enjoying the requisite algebraic properties of the operator (see
\cite{milner91polyadicpi}). We will therefore make liberal use of
$(\nu\; \vec{x})P$.

% subsection the_syntax_and_semantics_of_the_notation_system (end)   

\input{qm2pi.qmops} 

\input{qm2pi.sterngerlach} 

\input{qm2pi.metric} 

% section concurrent_process_calculi (end)

%\input{qm2pi.proofsketch}

% section proof sketch (end)

%\input{qm2pi.slviaknots} 

% section spatial logic via knots (end)

\input{qm2pi.conclusion}

% section conclusion (end)

%\input{qm2pi.dtcodes} 

% section wiring algorithm (end)

\input{qm2pi.ack} 

% section acknowledgments (end)

\newpage


\bibliographystyle{plain}   
\bibliography{../../biblios/main.bib}

\input{qm2pi.rhodetails}

\end{document}



\end{document}



\end{document}

 

% section concurrent_process_calculi (end)

%\documentclass[12pt]{llncs}
%\documentclass{jktr}

\usepackage[pdftex]{hyperref}                   
\usepackage {listings}
\usepackage {mathpartir}
\usepackage{bcprules}
%\usepackage{listings}
                       
\usepackage{graphicx} 
%\usepackage[margins=2.5cm,nohead,nofoot]{geometry}
%\usepackage{geometry}
\usepackage{amsfonts}
\usepackage{amstext}
\usepackage{latexsym}
\usepackage{amssymb}
\usepackage{color}


%\include{myPreamble}
\documentclass[12pt]{llncs}
%\documentclass{jktr}

\usepackage[pdftex]{hyperref}                   
\usepackage {listings}
\usepackage {mathpartir}
\usepackage{bcprules}
%\usepackage{listings}
                       
\usepackage{graphicx} 
%\usepackage[margins=2.5cm,nohead,nofoot]{geometry}
%\usepackage{geometry}
\usepackage{amsfonts}
\usepackage{amstext}
\usepackage{latexsym}
\usepackage{amssymb}
\usepackage{color}


%\include{myPreamble}
\documentclass[12pt]{llncs}
%\documentclass{jktr}

\usepackage[pdftex]{hyperref}                   
\usepackage {listings}
\usepackage {mathpartir}
\usepackage{bcprules}
%\usepackage{listings}
                       
\usepackage{graphicx} 
%\usepackage[margins=2.5cm,nohead,nofoot]{geometry}
%\usepackage{geometry}
\usepackage{amsfonts}
\usepackage{amstext}
\usepackage{latexsym}
\usepackage{amssymb}
\usepackage{color}


%\include{myPreamble}
\include{qm2pi.local} 

%\ifpdf
%\usepackage[pdftex]{graphicx}
%\else
%\usepackage{graphicx}
%\fi

 % \ifpdf
%  \usepackage{pdfsync}
%  \if


%\title{Brief Article}
%\author{David F. Snyder}
%\author{L.G. Meredith}

%\address{Dept. of Math., Texas State University--San Marcos, San Marcos, TX 78666}
       
\pagestyle{empty}


\begin{document}

\lstset{language=[Objective]Caml,frame=shadowbox}

\input{qm2pi.front}

% section front matter (end)

\input{qm2pi.intro} 
 
% section introduction (end)

% \input{qm2pi.knotations} 

% section notation (end)

\input{qm2pi.process.calculi} 

% section concurrent_process_calculi_and_spatial_logics_ (end)
    
%\input{qm2pi.knots2pi} 

%\input{qm2pi.trefoil} 

%\input{qm2pi.mainthm} 

% subsection basic_interpretation (end)

%\input{qm2pi.rho.presentation} 
\subsection{The syntax and semantics of the notation system}\label{sub:the_syntax_and_semantics_of_the_notation_system} % (fold)

We now summarize a technical presentation of the calculus that
embodies our theory of dynamics. The typical presentation of such a
calculus follows the style of giving generators and relations on
them. The grammar, below, describing term constructors, freely
generates the set of processes, $\Proc$. This set is then quotiented
by a relation known as structural congruence and it is over this set
that the notion of dynamics is expressed. This presentation is
essentially that of \cite{MeredithR05} with the addition of
polyadicity and summation. For readability we have relegated some of
the technical subtleties to an appendix.

\subsubsection{Process grammar}\label{subsub:process_grammar}

\begin{mathpar}
  \inferrule* [lab=synchronization] {} {{M} \bc \pzero \;|\; x?F \;|\; x!C }
  \and
  \inferrule* [lab=abstraction] {} {{F} \bc (x)P}
  \and
  \inferrule* [lab=concretion] {} {{C} \bc \langle Q \rangle}
  \and
  \inferrule* [lab=process] {} {{P,Q} \bc M \;| \;P|Q \;|\; @{x}}
  \and
  \inferrule* [lab=name] {} {{x} \bc \quotep{P}}
\end{mathpar} 

Note that $\vec{x}$ (resp. $\vec{P}$) denotes a vector of names
(resp. processes) of length $|\vec{x}|$ (resp. $|\vec{P}|$). We adopt
the following useful abbreviations.

\begin{mathpar}
   x?(\vec{y}).P := x.(\vec{y})P \and  x\clift{\vec{P}} := x.\clift{\vec{P}}
   \and x!(y) := \lift{x}{\dropn{y}}
   \and \Pi_{i=0}^{n-1}P_i := P_0 | \ldots | P_{n-1}
\end{mathpar}

\subsubsection{Structural congruence}

\paragraph{Free and bound names and alpha-equivalence.} At the
core of structural equivalence is alpha-equivalence which identifies
process that are the same up to a change of variable. Formally, we
recognize the distinction between free and bound names. The free names
of a process, $\freenames{P}$, may be calculated recursively as
follows:

\begin{mathpar}
\freenames{\pzero} := \emptyset
  \and \\
  \freenames{x?(y).P} := \{ x \} \cup (\freenames{P} \setminus \{ y \})
  \and 
  \freenames{x!\langle P \rangle} := \{ x \} \cup \{ P \} 
  \and \\
  \freenames{P|Q} := \freenames{P} \cup \freenames{Q}
  \and \\
  \freenames{@{x}} := \{ x \}
\end{mathpar}

$\pi$
$\quotep{\pi}$

$\freenames{-} : \pi \to \mathcal{P}(\quotep{\pi})$

\begin{eqnarray*}
  \freenames{\pzero} & := & \emptyset \\
  \freenames{x?(y).P} & := & \{ x \} \cup (\freenames{P} \setminus \{ y \}) \\
  \freenames{x!\langle P \rangle} & := & \{ x \} \cup \{ P \} \\
  \freenames{P|Q} & := & \freenames{P} \cup \freenames{Q} \\
  \freenames{\dropn{x}} & := & \{ x \}
\end{eqnarray*}

The bound names of a process, $\boundnames{P}$, are those names occurring in $P$
that are not free. For example, in $x?(y).0$, the name $x$ is free, while $y$ is bound.

\begin{mathpar}
  \inferrule* [lab=monoidal-laws] {} { P|Q \equiv Q|P \and P|0 \equiv P \and P|(Q|R) \equiv (P|Q)|R }
\end{mathpar}

\begin{mathpar}
  \inferrule* [lab=alpha-equivalence] {} { (x)P \equiv (y)P\{y/x\} \and y \not\in \freenames{P} }
\end{mathpar}

\begin{definition}
Then two processes, $P,Q$, are alpha-equivalent if $P = Q\{\vec{y}/\vec{x}\}$ for
some $\vec{x} \in \boundnames{Q},\vec{y} \in \boundnames{P}$, where $Q\{\vec{y}/\vec{x}\}$
denotes the capture-avoiding substitution of $\vec{y}$ for $\vec{x}$ in $Q$.
\end{definition}

\begin{definition}
  The {\em structural congruence} \cite{SangiorgiWalker} , $\equiv$,
  between processes is the least congruence containing
  alpha-equivalence, satisfying the abelian monoid laws
  (associativity, commutativity and $\pzero$ as identity) for parallel
  composition $|$ and for summation $+$.
\end{definition}

\subsection{Name equivalence}

We take name equivalence, written $\nameeq$, to be the smallest
equivalence relation generated by the following rules.

\begin{mathpar}
\inferrule*[lab=Quote-drop]
{ }
{ \quotep{@{x}} \nameeq x }

\inferrule*[lab=Struct-equiv]
{ P \scong Q }
{ \quotep{P} \nameeq \quotep{Q} }
\end{mathpar}

The astute reader will have noticed that the mutual recursion of names
and processes imposes a mutual recursion on alpha-equivalence and
structural equivalence via name-equivalence. Fortunately, all of this
works out pleasantly and we may calculate in the natural way, free of
concern. The reader interested in the details is referred to the
appendix \ref{appendix:rho_details}.

\subsection{Substitution}

We use $\Proc$ for the set of processes, $\QProc$ for the set of
names, and $\id{\{}\vec{y} / \vec{x} \id{\}}$ to denote partial maps,
$s : \QProc \rightarrow \QProc$. A map, $s$ lifts, uniquely, to a map
on process terms, $\widehat{s} : \Proc \rightarrow \Proc$ by the
following equations.

\begin{mathpar}
  (0) \psubstp{Q}{P} := 0 \\
  (R \juxtap S) \psubstp{Q}{P}
  :=    
  (R)\psubstp{Q}{P} \juxtap (S) \psubstp{Q}{P} \\
  (x?(y).R) \psubstp{Q}{P}    
  :=    
  (x)\substp{Q}{P} (z)\concat( (R \psubstn{z}{y}) \psubstp{Q}{P} ) \\
  (\lift{x}{R}) \psubstp{Q}{P}  
  :=
  \lift{(x)\substp{Q}{P}}{ R \psubstp{Q}{P} } \\
%   (\dropn{x})  \psubstp{Q}{P}       
%   := 
%   \left\{ 
%     \begin{array}{ccc} 
%       \dropn{\quotep{Q}} & & x \nameeq \quotep{P} \\
%       \dropn{x} & & otherwise \\
%     \end{array}
%   \right. 
  (\dropn{x})  \psubstp{Q}{P}       
  := 
  \left\{ 
    \begin{array}{ccc} 
      Q & & x \nameeq \quotep{P} \\
      \dropn{x} & & otherwise \\
    \end{array}
  \right.
\end{mathpar}
 

where

\begin{eqnarray}
  (x)\id{\{} \lpquote Q \rpquote / \lpquote P \rpquote \id{\}}            = 
  \left\{ 
    \begin{array}{ccc}
      \lpquote Q \rpquote & & x \nameeq \lpquote P \rpquote \\
      x & & otherwise \\
    \end{array}
  \right. \nonumber
\end{eqnarray}

and $z$ is chosen distinct from $\quotep{P}$, $\quotep{Q}$, the free
names in $Q$, and all the names in $R$. Our $\alpha$-equivalence will
be built in the standard way from this substitution.

\begin{remark}\label{rem:no_self_referential_names}
  One consequence of these definitions is that $\forall P. \quotep{P}
  \not\in \freenames{P}$.
\end{remark}

\subsection{ Dynamic quote: an example }

Anticipating something of what's to come, consider applying the
substitution, $\widehat{\id{\{}u / z \id{\}}}$, to the following pair
of processes, $\lift{w}{y!(z)}$ and $w[ \lpquote y!(z) \rpquote ]$.

\begin{eqnarray}
	\lift{w}{y!(z)}\widehat{\id{\{}u / z \id{\}}}
		& = &
		\lift{w}{y!(u)} \nonumber\\
	w[ \lpquote y!(z) \rpquote ] \widehat{ \id{\{}u / z \id{\}} }
		& = &
		w[ \lpquote y!(z) \rpquote ] \nonumber
\end{eqnarray}

Because the body of the process between quotes is impervious to
substitution, we get radically different answers. In fact, by
examining the first process in an input context,
e.g. $x?(z).\lift{w}{y!(z)}$, we see that the process under the lift
operator may be shaped by prefixed inputs binding a name inside it. In
this sense, the lift operator will be seen as a way to dynamically
construct processes before reifying them as names.

Finally equipped with these standard features we can present the
dynamics of the calculus.

\subsubsection{Operational semantics} 

Finally, we introduce the computational dynamics. What marks these
algebras as distinct from other more traditionally studied algebraic
structures, e.g. vector spaces or polynomial rings, is the manner in
which dynamics is captured. In traditional structures, dynamics is typically
expressed through morphisms between such structures, as in linear maps
between vector spaces or morphisms between rings. In algebras
associated with the semantics of computation, the dynamics is
expressed as part of the algebraic structure itself, through a
reduction reduction relation typically denoted by $\red$. Below, we
give a recursive presentation of this relation for the calculus used
in the encoding.

$\red \subseteq \pi \times \pi$
$\red : \pi \to \mathcal{P}(\pi)$

\begin{mathpar}
  \inferrule* [lab=Comm] { \textsf{match}( x_{src}, x_{trgt} ) } { x_{trgt}?(y)P \; | \; x_{src}!\langle {Q} \rangle \red P\{\quotep{Q}/y}\} }
  \and \\
  \inferrule* [lab=Par] {{P} \red {P}'} {{{P} | {Q}} \red {{P}' | {Q}}}
  \and
  \inferrule* [lab=Equiv]{{{P} \scong {P}'} \andalso {{P}' \red {Q}'} \andalso {{Q}' \scong {Q}}}{{P} \red {Q}}
\end{mathpar}

\begin{eqnarray*}
  match_{\equiv} (\quotep{P},\quotep{Q}) & := & P \equiv Q \\
  match_{\dagger}(\quotep{P},\quotep{Q}) & := & \forall R. P|Q \red^{*} R => R \red^{*} 0 \\
  match_{K}(\quotep{P},\quotep{Q}) & := & K \mbox{ for some context } K
\end{eqnarray*}

$u?(x)P | u!\langle Q \rangle \red P\{\quotep{Q}/x\}$

%We write $\wred$ for $\red^*$, and $P\red$ if $\exists Q $ such that $ P \red Q$.
We write $P\red$ if $\exists Q $ such that $ P \red Q$ and $P\not\red$, otherwise.

\section{Replication}

As mentioned before, it is known that replication (and hence
recursion) can be implemented in a higher-order process algebra
\cite{SangiorgiWalker}. As our first example of calculation with the
machinery thus far presented we give the construction explicitly in
the {\rhoc}.

\begin{eqnarray}
	D_{x} & := & \prefix{x}{y}{(\binpar{\outputp{x}{y}}{@{y}})} \nonumber\\
	\bangp_{x}{P} & := & \binpar{{x}!\langle{\binpar{D_{x}}{P}}\rangle}{D_{x}} \nonumber
\end{eqnarray}

\begin{eqnarray}
	\bangp_{x}{P} & & \nonumber\\
	=
	& {x}!\langle{(\prefix{x}{y}{(\outputp{x}{y} | @{y})) | P}}\rangle 
	      | \prefix{x}{y}{(\outputp{x}{y} | @{y})} & \nonumber\\
	\red
	& (\outputp{x}{y} | @{y})\substn{\quotep{(\prefix{x}{y}{(@{y} | \outputp{x}{y})) | P}}}{y} & \nonumber\\
	=
	& \outputp{x}{\quotep{(\prefix{x}{y}{(\outputp{x}{y} | @{y})) | P}}}
	  | {(\prefix{x}{y}{(\outputp{x}{y} | @{y})) | P}} & \nonumber\\
	\red
	& \ldots & \nonumber\\
	\red^*
	& P | P | \ldots & \nonumber
\end{eqnarray}

Of course, this encoding, as an implementation, runs away, unfolding
$\bangp{P}$ eagerly. A lazier and more implementable replication
operator, restricted to input-guarded processes, may be obtained as follows.

\begin{eqnarray}
\bangp{\prefix{u}{v}{P}} 
	:= 
	\binpar{\lift{x}{\prefix{u}{v}{(\binpar{D(x)}{P})}}}{D(x)} \nonumber
\end{eqnarray}

\begin{remark}
  Note that the lazier definition still does not deal with summation
  or mixed summation (i.e. sums over input and output). The reader is
  invited to construct definitions of replication that deal with these
  features. 

  Further, the definitions are parameterized in a name, $x$. Can you,
  gentle reader, make a definition that eliminates this parameter and
  guarantees no accidental interaction between the replication
  machinery and the process being replicated -- i.e. no accidental
  sharing of names used by the process to get its work done and the
  name(s) used by the replication to effect copying. This latter
  revision of the definition of replication is crucial to obtaining
  the expected identity $!!P \sim !P$.
\end{remark}

\begin{remark}\label{rem:paradoxical_combinator}
  The reader familiar with the lambda calculus will have noticed the
  similarity between $D$ and the paradoxical combinator.

  [Ed. note: the existence of this seems to suggest we have to be more
  restrictive on the set of processes and names we admit if we are to
  support no-cloning.]
\end{remark}

\subsubsection{Bisimulation}

The computational dynamics gives rise to another kind of equivalence,
the equivalence of computational behavior. As previously mentioned
this is typically captured \emph{via} some form of bisimulation.

% The notion we use in this paper is weak barbed bisimulation
% \cite{milner91polyadicpi}.

The notion we use in this paper is derived from weak barbed
bisimulation \cite{milner91polyadicpi}. 

\begin{definition}
An \emph{observation relation}, $\downarrow_{\mathcal N}$, over a set
of names, $\mathcal N$, is the smallest relation satisfying the rules
below.

\infrule[Out-barb]{y \in {\mathcal N}, \; x \nameeq y}
		  {\outputp{x}{v} \downarrow_{\mathcal N} x}
\infrule[Par-barb]{\mbox{$P\downarrow_{\mathcal N} x$ or $Q\downarrow_{\mathcal N} x$}}
		  {\binpar{P}{Q} \downarrow_{\mathcal N} x}

We write $P \Downarrow_{\mathcal N} x$ if there is $Q$ such that 
$P \wred Q$ and $Q \downarrow_{\mathcal N} x$.
\end{definition}

\begin{definition}
%\label{def.bbisim}
An  ${\mathcal N}$-\emph{barbed bisimulation} over a set of names, ${\mathcal N}$, is a symmetric binary relation 
${\mathcal S}_{\mathcal N}$ between agents such that $P\rel{S}_{\mathcal N}Q$ implies:
\begin{enumerate}
\item If $P \red P'$ then $Q \wred Q'$ and $P'\rel{S}_{\mathcal N} Q'$.
\item If $P\downarrow_{\mathcal N} x$, then $Q\Downarrow_{\mathcal N} x$.
\end{enumerate}
$P$ is ${\mathcal N}$-barbed bisimilar to $Q$, written
$P \wbbisim_{\mathcal N} Q$, if $P \rel{S}_{\mathcal N} Q$ for some ${\mathcal N}$-barbed bisimulation ${\mathcal S}_{\mathcal N}$.
\end{definition}

$\mathcal{R} \subseteq \pi \times \pi$

$P \mathcal{R} Q => \forall P'. P \red P' \Rightarrow \exists Q'. Q \red Q', P' \mathcal{R} Q'$

$P \vdash x \Rightarrow Q \vdash x$

\begin{mathpar}
  \inferrule*[lab=Out-barb]{x \nameeq y}{{y}!\langle{Q}\rangle \vdash x}
  \and
  \inferrule*[lab=Par-barb]{\mbox{$P\vdash x$ or $Q\vdash x$}}{\binpar{P}{Q} \vdash x}
\end{mathpar}

\subsubsection{Contexts}

One of the principle advantages of computational calculi like the
$\pi$-calculus is a well-defined notion of context,
contextual-equivalence and a correlation between
contextual-equivalence and notions of bisimulation. The notion of
context allows the decomposition of a process into (sub-)process and
its syntactic environment, its context. Thus, a context may be
thought of as a process with a ``hole'' (written $\Box$) in it. The
application of a context $M$ to a process $P$, written $M[P]$, is
tantamount to filling the hole in $M$ with $P$. In this paper we do
not need the full weight of this theory, but do make use of the notion
of context in the proof the main theorem. 

\begin{mathpar}
  \inferrule* [lab=summation] {} {{M_{M},M_{N}} \bc \Box \;|\; x.M_{A} \;|\; M_{M}+M_{N}}
  \and
  \inferrule* [lab=agent] {} {{M_{A}} \bc (\vec{x})M_{P} \;| \; \clift{P_0,\ldots,M_{P},\ldots,P_N}}
  \and \\
  \inferrule* [lab=process] {} {{M_{P}} \bc M_{N} \;| \;P|M_{P} }
\end{mathpar} 

\begin{mathpar}
  \inferrule* [lab=sychronization] {} {M_{N} \bc \Box \;|\; x?M_{F} \;|\; x!M_{C}}
  \and
  \inferrule* [lab=abstraction] {} {{M_{F}} \bc (x)M_{P} }
  \and
  \inferrule* [lab=concretion] {} {{M_{C}} \bc \langle M_{P} \rangle }
  \and \\
  \inferrule* [lab=process] {} {{M_{P}} \bc M_{N} \;| \;P|M_{P} }
\end{mathpar}

\begin{definition}[contextual application] Given a context $M$, and
  process $P$, we define the \emph{contextual application}, $M[P] :=
  M\{P/\Box\}$. That is, the contextual application of M to P is the
  substitution of $P$ for $\Box$ in $M$.
\end{definition}

$\meaningof{-} : L \to \mathcal{P}(\pi)$

\begin{mathpar}
  \inferrule* [lab=collection] {} {\meaningof{true} = \pi, \and \meaningof{~E} = \pi \setminus \meaningof{E}, \and \meaningof{E_{1} \& E_{2}} = \meaningof{E_{1}} \cap \meaningof{E_{2}}}
\end{mathpar}

\begin{mathpar}
  \inferrule* [lab=structure] {} {\meaningof{0} = \{ P \in \pi | P \equiv 0 \}, \and \\ \meaningof{E_1 | E_2} = \{ P \in \pi | P \equiv P_{1} | P_{2}, P_{1} \in \meaningof{E_{1}}, P_{2} \in \meaningof{E_2}\} }
\end{mathpar}

\begin{mathpar}
 \inferrule* [lab=behavior] {} {\meaningof{\langle a?b \rangle E} = \{ P \in \pi | P \equiv Q | u?(y)P', \\ \and \\\\ \and \\ \;\;\; u \in \meaningof{a}, \forall z.P'\{z/y\} \in \meaningof{E\{z/b\}}\}, \and \\ \meaningof{a!E} = \{ P \in \pi | P \equiv Q | x!\langle P' \rangle, x \in \meaningof{a} P' \in \meaningof{E}\} }
\end{mathpar}

\begin{mathpar}
 \inferrule* [lab=nominal] {} {\meaningof{\quotep{E}} = \{ \quotep{P} \in \quotep{\pi} | P \in \meaningof{E} \}, \and \meaningof{\quotep{P}} = \{ \quotep{Q} \in \quotep{\pi} | P \equiv Q \} \and \\ \meaningof{@\quotep{E}} = \{ P \in \pi | P \equiv @x, x \in \meaningof{E} \}}
\end{mathpar}

\begin{eqnarray*}
  \\
  \meaningof{-} : TS \to ST
\end{eqnarray*}

\begin{eqnarray*}
  \\
  L : TS \to ST
\end{eqnarray*}

\begin{eqnarray*}
  \\
  P \models E \iff P \in \meaningof{E}
\end{eqnarray*}

\begin{eqnarray*}
  P \approx_{L} Q \iff \forall E \in L. P \models E \iff Q \models E
\end{eqnarray*}

\begin{eqnarray*}
  P \approx_{K} Q
\end{eqnarray*}

\begin{eqnarray*}
  P \approx Q
\end{eqnarray*}

$\approx_{K} = \approx = \approx_{L}$

\subsubsection{Contextual duality}

Note that contexts extend the quotation operation to a family of
operations from processes to names. Given a context, $M$, we can
define a \emph{nominal context}, $\quotep{M}$ by $\quotep{M}[P] :=
\quotep{M[P]}$. To foreshadow what is to come we observe that these
operations enjoy a duality with processes very much like the duality
between vectors and maps from vectors to scalars.

Further, because the calculus is essentially higher-order, we have a
correspondence between contexts and processes. More specifically,
given a name $x$ and a context $M$ we can construct $M^{*}_{x}$ such
that 

\begin{mathpar}
  M^{*}_{x} | \lift{x}{P} \red M[P]
\end{mathpar}

namely,

\begin{mathpar}
  M^{*}_{x} := x?(u).M[\dropn{u}]
\end{mathpar}

The dependence of $M^{*}_{x}$ on a name makes it an abstraction, 

\begin{mathpar}
  M^{*} := (x)x?(u).M[\dropn{u}]
\end{mathpar}

\subsection{Additional notation}

It will sometimes be convenient to denote the process a name
quotes. We already have the notation $x = \quotep{P}$, but it will be
convenient to introduce an alternate notation, $\procn{x}$, when we
want to emphasize the connection to the use of the name. Note that, by
virtue of name equivalence, $\quotep{\procn{x}} \nameeq x$; so, the
notation is consistent with previous definitions.

Further, because names have structure it is possible to effect
substitutions on the basis of that structure. This means we need to
upgrade our notation for substitutions, which we accomplish by
adapting comprehension notation. Thus,

\begin{mathpar}
  P\{ y / x : x \in S \}
\end{mathpar}

is interpreted to mean the process derived from P by replacing (in a
capture-avoiding manner) each occurrence of $x$ in $S$ by $y$. For example,

\begin{mathpar}
  P\{ \quotep{\procn{x}|\procn{x}} / x : x \in \freenames{P} \}
\end{mathpar}

will replace each (occurrence) of a free name $x$ in $P$ by
$\quotep{\procn{x}|\procn{x}}$.

Also, we will avail ourselves of the notation $x^{L}$ and $x^{R}$ to
denote injections of a name into disjoint copies of the name
space. There are numerous ways to accomplish this. One example can be
found in \cite{MeredithR05}. This notation overloads to vectors of
names: $\vec{x}^{\pi} := (x_{i}^{\pi} \; : \; 0 \leq i < |\vec{x}| )$ where $\pi \in \{L,R\}$.

We also use $P^{\Box} := P|\Box$.

In \cite{MeredithR05} an interpretation of the new operator is
given. It turns out that there are several possible interpretations
all enjoying the requisite algebraic properties of the operator (see
\cite{milner91polyadicpi}). We will therefore make liberal use of
$(\nu\; \vec{x})P$.

% subsection the_syntax_and_semantics_of_the_notation_system (end)   

\input{qm2pi.qmops} 

\input{qm2pi.sterngerlach} 

\input{qm2pi.metric} 

% section concurrent_process_calculi (end)

%\input{qm2pi.proofsketch}

% section proof sketch (end)

%\input{qm2pi.slviaknots} 

% section spatial logic via knots (end)

\input{qm2pi.conclusion}

% section conclusion (end)

%\input{qm2pi.dtcodes} 

% section wiring algorithm (end)

\input{qm2pi.ack} 

% section acknowledgments (end)

\newpage


\bibliographystyle{plain}   
\bibliography{../../biblios/main.bib}

\input{qm2pi.rhodetails}

\end{document}

 

%\ifpdf
%\usepackage[pdftex]{graphicx}
%\else
%\usepackage{graphicx}
%\fi

 % \ifpdf
%  \usepackage{pdfsync}
%  \if


%\title{Brief Article}
%\author{David F. Snyder}
%\author{L.G. Meredith}

%\address{Dept. of Math., Texas State University--San Marcos, San Marcos, TX 78666}
       
\pagestyle{empty}


\begin{document}

\lstset{language=[Objective]Caml,frame=shadowbox}

\documentclass[12pt]{llncs}
%\documentclass{jktr}

\usepackage[pdftex]{hyperref}                   
\usepackage {listings}
\usepackage {mathpartir}
\usepackage{bcprules}
%\usepackage{listings}
                       
\usepackage{graphicx} 
%\usepackage[margins=2.5cm,nohead,nofoot]{geometry}
%\usepackage{geometry}
\usepackage{amsfonts}
\usepackage{amstext}
\usepackage{latexsym}
\usepackage{amssymb}
\usepackage{color}


%\include{myPreamble}
\include{qm2pi.local} 

%\ifpdf
%\usepackage[pdftex]{graphicx}
%\else
%\usepackage{graphicx}
%\fi

 % \ifpdf
%  \usepackage{pdfsync}
%  \if


%\title{Brief Article}
%\author{David F. Snyder}
%\author{L.G. Meredith}

%\address{Dept. of Math., Texas State University--San Marcos, San Marcos, TX 78666}
       
\pagestyle{empty}


\begin{document}

\lstset{language=[Objective]Caml,frame=shadowbox}

\input{qm2pi.front}

% section front matter (end)

\input{qm2pi.intro} 
 
% section introduction (end)

% \input{qm2pi.knotations} 

% section notation (end)

\input{qm2pi.process.calculi} 

% section concurrent_process_calculi_and_spatial_logics_ (end)
    
%\input{qm2pi.knots2pi} 

%\input{qm2pi.trefoil} 

%\input{qm2pi.mainthm} 

% subsection basic_interpretation (end)

%\input{qm2pi.rho.presentation} 
\subsection{The syntax and semantics of the notation system}\label{sub:the_syntax_and_semantics_of_the_notation_system} % (fold)

We now summarize a technical presentation of the calculus that
embodies our theory of dynamics. The typical presentation of such a
calculus follows the style of giving generators and relations on
them. The grammar, below, describing term constructors, freely
generates the set of processes, $\Proc$. This set is then quotiented
by a relation known as structural congruence and it is over this set
that the notion of dynamics is expressed. This presentation is
essentially that of \cite{MeredithR05} with the addition of
polyadicity and summation. For readability we have relegated some of
the technical subtleties to an appendix.

\subsubsection{Process grammar}\label{subsub:process_grammar}

\begin{mathpar}
  \inferrule* [lab=synchronization] {} {{M} \bc \pzero \;|\; x?F \;|\; x!C }
  \and
  \inferrule* [lab=abstraction] {} {{F} \bc (x)P}
  \and
  \inferrule* [lab=concretion] {} {{C} \bc \langle Q \rangle}
  \and
  \inferrule* [lab=process] {} {{P,Q} \bc M \;| \;P|Q \;|\; @{x}}
  \and
  \inferrule* [lab=name] {} {{x} \bc \quotep{P}}
\end{mathpar} 

Note that $\vec{x}$ (resp. $\vec{P}$) denotes a vector of names
(resp. processes) of length $|\vec{x}|$ (resp. $|\vec{P}|$). We adopt
the following useful abbreviations.

\begin{mathpar}
   x?(\vec{y}).P := x.(\vec{y})P \and  x\clift{\vec{P}} := x.\clift{\vec{P}}
   \and x!(y) := \lift{x}{\dropn{y}}
   \and \Pi_{i=0}^{n-1}P_i := P_0 | \ldots | P_{n-1}
\end{mathpar}

\subsubsection{Structural congruence}

\paragraph{Free and bound names and alpha-equivalence.} At the
core of structural equivalence is alpha-equivalence which identifies
process that are the same up to a change of variable. Formally, we
recognize the distinction between free and bound names. The free names
of a process, $\freenames{P}$, may be calculated recursively as
follows:

\begin{mathpar}
\freenames{\pzero} := \emptyset
  \and \\
  \freenames{x?(y).P} := \{ x \} \cup (\freenames{P} \setminus \{ y \})
  \and 
  \freenames{x!\langle P \rangle} := \{ x \} \cup \{ P \} 
  \and \\
  \freenames{P|Q} := \freenames{P} \cup \freenames{Q}
  \and \\
  \freenames{@{x}} := \{ x \}
\end{mathpar}

$\pi$
$\quotep{\pi}$

$\freenames{-} : \pi \to \mathcal{P}(\quotep{\pi})$

\begin{eqnarray*}
  \freenames{\pzero} & := & \emptyset \\
  \freenames{x?(y).P} & := & \{ x \} \cup (\freenames{P} \setminus \{ y \}) \\
  \freenames{x!\langle P \rangle} & := & \{ x \} \cup \{ P \} \\
  \freenames{P|Q} & := & \freenames{P} \cup \freenames{Q} \\
  \freenames{\dropn{x}} & := & \{ x \}
\end{eqnarray*}

The bound names of a process, $\boundnames{P}$, are those names occurring in $P$
that are not free. For example, in $x?(y).0$, the name $x$ is free, while $y$ is bound.

\begin{mathpar}
  \inferrule* [lab=monoidal-laws] {} { P|Q \equiv Q|P \and P|0 \equiv P \and P|(Q|R) \equiv (P|Q)|R }
\end{mathpar}

\begin{mathpar}
  \inferrule* [lab=alpha-equivalence] {} { (x)P \equiv (y)P\{y/x\} \and y \not\in \freenames{P} }
\end{mathpar}

\begin{definition}
Then two processes, $P,Q$, are alpha-equivalent if $P = Q\{\vec{y}/\vec{x}\}$ for
some $\vec{x} \in \boundnames{Q},\vec{y} \in \boundnames{P}$, where $Q\{\vec{y}/\vec{x}\}$
denotes the capture-avoiding substitution of $\vec{y}$ for $\vec{x}$ in $Q$.
\end{definition}

\begin{definition}
  The {\em structural congruence} \cite{SangiorgiWalker} , $\equiv$,
  between processes is the least congruence containing
  alpha-equivalence, satisfying the abelian monoid laws
  (associativity, commutativity and $\pzero$ as identity) for parallel
  composition $|$ and for summation $+$.
\end{definition}

\subsection{Name equivalence}

We take name equivalence, written $\nameeq$, to be the smallest
equivalence relation generated by the following rules.

\begin{mathpar}
\inferrule*[lab=Quote-drop]
{ }
{ \quotep{@{x}} \nameeq x }

\inferrule*[lab=Struct-equiv]
{ P \scong Q }
{ \quotep{P} \nameeq \quotep{Q} }
\end{mathpar}

The astute reader will have noticed that the mutual recursion of names
and processes imposes a mutual recursion on alpha-equivalence and
structural equivalence via name-equivalence. Fortunately, all of this
works out pleasantly and we may calculate in the natural way, free of
concern. The reader interested in the details is referred to the
appendix \ref{appendix:rho_details}.

\subsection{Substitution}

We use $\Proc$ for the set of processes, $\QProc$ for the set of
names, and $\id{\{}\vec{y} / \vec{x} \id{\}}$ to denote partial maps,
$s : \QProc \rightarrow \QProc$. A map, $s$ lifts, uniquely, to a map
on process terms, $\widehat{s} : \Proc \rightarrow \Proc$ by the
following equations.

\begin{mathpar}
  (0) \psubstp{Q}{P} := 0 \\
  (R \juxtap S) \psubstp{Q}{P}
  :=    
  (R)\psubstp{Q}{P} \juxtap (S) \psubstp{Q}{P} \\
  (x?(y).R) \psubstp{Q}{P}    
  :=    
  (x)\substp{Q}{P} (z)\concat( (R \psubstn{z}{y}) \psubstp{Q}{P} ) \\
  (\lift{x}{R}) \psubstp{Q}{P}  
  :=
  \lift{(x)\substp{Q}{P}}{ R \psubstp{Q}{P} } \\
%   (\dropn{x})  \psubstp{Q}{P}       
%   := 
%   \left\{ 
%     \begin{array}{ccc} 
%       \dropn{\quotep{Q}} & & x \nameeq \quotep{P} \\
%       \dropn{x} & & otherwise \\
%     \end{array}
%   \right. 
  (\dropn{x})  \psubstp{Q}{P}       
  := 
  \left\{ 
    \begin{array}{ccc} 
      Q & & x \nameeq \quotep{P} \\
      \dropn{x} & & otherwise \\
    \end{array}
  \right.
\end{mathpar}
 

where

\begin{eqnarray}
  (x)\id{\{} \lpquote Q \rpquote / \lpquote P \rpquote \id{\}}            = 
  \left\{ 
    \begin{array}{ccc}
      \lpquote Q \rpquote & & x \nameeq \lpquote P \rpquote \\
      x & & otherwise \\
    \end{array}
  \right. \nonumber
\end{eqnarray}

and $z$ is chosen distinct from $\quotep{P}$, $\quotep{Q}$, the free
names in $Q$, and all the names in $R$. Our $\alpha$-equivalence will
be built in the standard way from this substitution.

\begin{remark}\label{rem:no_self_referential_names}
  One consequence of these definitions is that $\forall P. \quotep{P}
  \not\in \freenames{P}$.
\end{remark}

\subsection{ Dynamic quote: an example }

Anticipating something of what's to come, consider applying the
substitution, $\widehat{\id{\{}u / z \id{\}}}$, to the following pair
of processes, $\lift{w}{y!(z)}$ and $w[ \lpquote y!(z) \rpquote ]$.

\begin{eqnarray}
	\lift{w}{y!(z)}\widehat{\id{\{}u / z \id{\}}}
		& = &
		\lift{w}{y!(u)} \nonumber\\
	w[ \lpquote y!(z) \rpquote ] \widehat{ \id{\{}u / z \id{\}} }
		& = &
		w[ \lpquote y!(z) \rpquote ] \nonumber
\end{eqnarray}

Because the body of the process between quotes is impervious to
substitution, we get radically different answers. In fact, by
examining the first process in an input context,
e.g. $x?(z).\lift{w}{y!(z)}$, we see that the process under the lift
operator may be shaped by prefixed inputs binding a name inside it. In
this sense, the lift operator will be seen as a way to dynamically
construct processes before reifying them as names.

Finally equipped with these standard features we can present the
dynamics of the calculus.

\subsubsection{Operational semantics} 

Finally, we introduce the computational dynamics. What marks these
algebras as distinct from other more traditionally studied algebraic
structures, e.g. vector spaces or polynomial rings, is the manner in
which dynamics is captured. In traditional structures, dynamics is typically
expressed through morphisms between such structures, as in linear maps
between vector spaces or morphisms between rings. In algebras
associated with the semantics of computation, the dynamics is
expressed as part of the algebraic structure itself, through a
reduction reduction relation typically denoted by $\red$. Below, we
give a recursive presentation of this relation for the calculus used
in the encoding.

$\red \subseteq \pi \times \pi$
$\red : \pi \to \mathcal{P}(\pi)$

\begin{mathpar}
  \inferrule* [lab=Comm] { \textsf{match}( x_{src}, x_{trgt} ) } { x_{trgt}?(y)P \; | \; x_{src}!\langle {Q} \rangle \red P\{\quotep{Q}/y}\} }
  \and \\
  \inferrule* [lab=Par] {{P} \red {P}'} {{{P} | {Q}} \red {{P}' | {Q}}}
  \and
  \inferrule* [lab=Equiv]{{{P} \scong {P}'} \andalso {{P}' \red {Q}'} \andalso {{Q}' \scong {Q}}}{{P} \red {Q}}
\end{mathpar}

\begin{eqnarray*}
  match_{\equiv} (\quotep{P},\quotep{Q}) & := & P \equiv Q \\
  match_{\dagger}(\quotep{P},\quotep{Q}) & := & \forall R. P|Q \red^{*} R => R \red^{*} 0 \\
  match_{K}(\quotep{P},\quotep{Q}) & := & K \mbox{ for some context } K
\end{eqnarray*}

$u?(x)P | u!\langle Q \rangle \red P\{\quotep{Q}/x\}$

%We write $\wred$ for $\red^*$, and $P\red$ if $\exists Q $ such that $ P \red Q$.
We write $P\red$ if $\exists Q $ such that $ P \red Q$ and $P\not\red$, otherwise.

\section{Replication}

As mentioned before, it is known that replication (and hence
recursion) can be implemented in a higher-order process algebra
\cite{SangiorgiWalker}. As our first example of calculation with the
machinery thus far presented we give the construction explicitly in
the {\rhoc}.

\begin{eqnarray}
	D_{x} & := & \prefix{x}{y}{(\binpar{\outputp{x}{y}}{@{y}})} \nonumber\\
	\bangp_{x}{P} & := & \binpar{{x}!\langle{\binpar{D_{x}}{P}}\rangle}{D_{x}} \nonumber
\end{eqnarray}

\begin{eqnarray}
	\bangp_{x}{P} & & \nonumber\\
	=
	& {x}!\langle{(\prefix{x}{y}{(\outputp{x}{y} | @{y})) | P}}\rangle 
	      | \prefix{x}{y}{(\outputp{x}{y} | @{y})} & \nonumber\\
	\red
	& (\outputp{x}{y} | @{y})\substn{\quotep{(\prefix{x}{y}{(@{y} | \outputp{x}{y})) | P}}}{y} & \nonumber\\
	=
	& \outputp{x}{\quotep{(\prefix{x}{y}{(\outputp{x}{y} | @{y})) | P}}}
	  | {(\prefix{x}{y}{(\outputp{x}{y} | @{y})) | P}} & \nonumber\\
	\red
	& \ldots & \nonumber\\
	\red^*
	& P | P | \ldots & \nonumber
\end{eqnarray}

Of course, this encoding, as an implementation, runs away, unfolding
$\bangp{P}$ eagerly. A lazier and more implementable replication
operator, restricted to input-guarded processes, may be obtained as follows.

\begin{eqnarray}
\bangp{\prefix{u}{v}{P}} 
	:= 
	\binpar{\lift{x}{\prefix{u}{v}{(\binpar{D(x)}{P})}}}{D(x)} \nonumber
\end{eqnarray}

\begin{remark}
  Note that the lazier definition still does not deal with summation
  or mixed summation (i.e. sums over input and output). The reader is
  invited to construct definitions of replication that deal with these
  features. 

  Further, the definitions are parameterized in a name, $x$. Can you,
  gentle reader, make a definition that eliminates this parameter and
  guarantees no accidental interaction between the replication
  machinery and the process being replicated -- i.e. no accidental
  sharing of names used by the process to get its work done and the
  name(s) used by the replication to effect copying. This latter
  revision of the definition of replication is crucial to obtaining
  the expected identity $!!P \sim !P$.
\end{remark}

\begin{remark}\label{rem:paradoxical_combinator}
  The reader familiar with the lambda calculus will have noticed the
  similarity between $D$ and the paradoxical combinator.

  [Ed. note: the existence of this seems to suggest we have to be more
  restrictive on the set of processes and names we admit if we are to
  support no-cloning.]
\end{remark}

\subsubsection{Bisimulation}

The computational dynamics gives rise to another kind of equivalence,
the equivalence of computational behavior. As previously mentioned
this is typically captured \emph{via} some form of bisimulation.

% The notion we use in this paper is weak barbed bisimulation
% \cite{milner91polyadicpi}.

The notion we use in this paper is derived from weak barbed
bisimulation \cite{milner91polyadicpi}. 

\begin{definition}
An \emph{observation relation}, $\downarrow_{\mathcal N}$, over a set
of names, $\mathcal N$, is the smallest relation satisfying the rules
below.

\infrule[Out-barb]{y \in {\mathcal N}, \; x \nameeq y}
		  {\outputp{x}{v} \downarrow_{\mathcal N} x}
\infrule[Par-barb]{\mbox{$P\downarrow_{\mathcal N} x$ or $Q\downarrow_{\mathcal N} x$}}
		  {\binpar{P}{Q} \downarrow_{\mathcal N} x}

We write $P \Downarrow_{\mathcal N} x$ if there is $Q$ such that 
$P \wred Q$ and $Q \downarrow_{\mathcal N} x$.
\end{definition}

\begin{definition}
%\label{def.bbisim}
An  ${\mathcal N}$-\emph{barbed bisimulation} over a set of names, ${\mathcal N}$, is a symmetric binary relation 
${\mathcal S}_{\mathcal N}$ between agents such that $P\rel{S}_{\mathcal N}Q$ implies:
\begin{enumerate}
\item If $P \red P'$ then $Q \wred Q'$ and $P'\rel{S}_{\mathcal N} Q'$.
\item If $P\downarrow_{\mathcal N} x$, then $Q\Downarrow_{\mathcal N} x$.
\end{enumerate}
$P$ is ${\mathcal N}$-barbed bisimilar to $Q$, written
$P \wbbisim_{\mathcal N} Q$, if $P \rel{S}_{\mathcal N} Q$ for some ${\mathcal N}$-barbed bisimulation ${\mathcal S}_{\mathcal N}$.
\end{definition}

$\mathcal{R} \subseteq \pi \times \pi$

$P \mathcal{R} Q => \forall P'. P \red P' \Rightarrow \exists Q'. Q \red Q', P' \mathcal{R} Q'$

$P \vdash x \Rightarrow Q \vdash x$

\begin{mathpar}
  \inferrule*[lab=Out-barb]{x \nameeq y}{{y}!\langle{Q}\rangle \vdash x}
  \and
  \inferrule*[lab=Par-barb]{\mbox{$P\vdash x$ or $Q\vdash x$}}{\binpar{P}{Q} \vdash x}
\end{mathpar}

\subsubsection{Contexts}

One of the principle advantages of computational calculi like the
$\pi$-calculus is a well-defined notion of context,
contextual-equivalence and a correlation between
contextual-equivalence and notions of bisimulation. The notion of
context allows the decomposition of a process into (sub-)process and
its syntactic environment, its context. Thus, a context may be
thought of as a process with a ``hole'' (written $\Box$) in it. The
application of a context $M$ to a process $P$, written $M[P]$, is
tantamount to filling the hole in $M$ with $P$. In this paper we do
not need the full weight of this theory, but do make use of the notion
of context in the proof the main theorem. 

\begin{mathpar}
  \inferrule* [lab=summation] {} {{M_{M},M_{N}} \bc \Box \;|\; x.M_{A} \;|\; M_{M}+M_{N}}
  \and
  \inferrule* [lab=agent] {} {{M_{A}} \bc (\vec{x})M_{P} \;| \; \clift{P_0,\ldots,M_{P},\ldots,P_N}}
  \and \\
  \inferrule* [lab=process] {} {{M_{P}} \bc M_{N} \;| \;P|M_{P} }
\end{mathpar} 

\begin{mathpar}
  \inferrule* [lab=sychronization] {} {M_{N} \bc \Box \;|\; x?M_{F} \;|\; x!M_{C}}
  \and
  \inferrule* [lab=abstraction] {} {{M_{F}} \bc (x)M_{P} }
  \and
  \inferrule* [lab=concretion] {} {{M_{C}} \bc \langle M_{P} \rangle }
  \and \\
  \inferrule* [lab=process] {} {{M_{P}} \bc M_{N} \;| \;P|M_{P} }
\end{mathpar}

\begin{definition}[contextual application] Given a context $M$, and
  process $P$, we define the \emph{contextual application}, $M[P] :=
  M\{P/\Box\}$. That is, the contextual application of M to P is the
  substitution of $P$ for $\Box$ in $M$.
\end{definition}

$\meaningof{-} : L \to \mathcal{P}(\pi)$

\begin{mathpar}
  \inferrule* [lab=collection] {} {\meaningof{true} = \pi, \and \meaningof{~E} = \pi \setminus \meaningof{E}, \and \meaningof{E_{1} \& E_{2}} = \meaningof{E_{1}} \cap \meaningof{E_{2}}}
\end{mathpar}

\begin{mathpar}
  \inferrule* [lab=structure] {} {\meaningof{0} = \{ P \in \pi | P \equiv 0 \}, \and \\ \meaningof{E_1 | E_2} = \{ P \in \pi | P \equiv P_{1} | P_{2}, P_{1} \in \meaningof{E_{1}}, P_{2} \in \meaningof{E_2}\} }
\end{mathpar}

\begin{mathpar}
 \inferrule* [lab=behavior] {} {\meaningof{\langle a?b \rangle E} = \{ P \in \pi | P \equiv Q | u?(y)P', \\ \and \\\\ \and \\ \;\;\; u \in \meaningof{a}, \forall z.P'\{z/y\} \in \meaningof{E\{z/b\}}\}, \and \\ \meaningof{a!E} = \{ P \in \pi | P \equiv Q | x!\langle P' \rangle, x \in \meaningof{a} P' \in \meaningof{E}\} }
\end{mathpar}

\begin{mathpar}
 \inferrule* [lab=nominal] {} {\meaningof{\quotep{E}} = \{ \quotep{P} \in \quotep{\pi} | P \in \meaningof{E} \}, \and \meaningof{\quotep{P}} = \{ \quotep{Q} \in \quotep{\pi} | P \equiv Q \} \and \\ \meaningof{@\quotep{E}} = \{ P \in \pi | P \equiv @x, x \in \meaningof{E} \}}
\end{mathpar}

\begin{eqnarray*}
  \\
  \meaningof{-} : TS \to ST
\end{eqnarray*}

\begin{eqnarray*}
  \\
  L : TS \to ST
\end{eqnarray*}

\begin{eqnarray*}
  \\
  P \models E \iff P \in \meaningof{E}
\end{eqnarray*}

\begin{eqnarray*}
  P \approx_{L} Q \iff \forall E \in L. P \models E \iff Q \models E
\end{eqnarray*}

\begin{eqnarray*}
  P \approx_{K} Q
\end{eqnarray*}

\begin{eqnarray*}
  P \approx Q
\end{eqnarray*}

$\approx_{K} = \approx = \approx_{L}$

\subsubsection{Contextual duality}

Note that contexts extend the quotation operation to a family of
operations from processes to names. Given a context, $M$, we can
define a \emph{nominal context}, $\quotep{M}$ by $\quotep{M}[P] :=
\quotep{M[P]}$. To foreshadow what is to come we observe that these
operations enjoy a duality with processes very much like the duality
between vectors and maps from vectors to scalars.

Further, because the calculus is essentially higher-order, we have a
correspondence between contexts and processes. More specifically,
given a name $x$ and a context $M$ we can construct $M^{*}_{x}$ such
that 

\begin{mathpar}
  M^{*}_{x} | \lift{x}{P} \red M[P]
\end{mathpar}

namely,

\begin{mathpar}
  M^{*}_{x} := x?(u).M[\dropn{u}]
\end{mathpar}

The dependence of $M^{*}_{x}$ on a name makes it an abstraction, 

\begin{mathpar}
  M^{*} := (x)x?(u).M[\dropn{u}]
\end{mathpar}

\subsection{Additional notation}

It will sometimes be convenient to denote the process a name
quotes. We already have the notation $x = \quotep{P}$, but it will be
convenient to introduce an alternate notation, $\procn{x}$, when we
want to emphasize the connection to the use of the name. Note that, by
virtue of name equivalence, $\quotep{\procn{x}} \nameeq x$; so, the
notation is consistent with previous definitions.

Further, because names have structure it is possible to effect
substitutions on the basis of that structure. This means we need to
upgrade our notation for substitutions, which we accomplish by
adapting comprehension notation. Thus,

\begin{mathpar}
  P\{ y / x : x \in S \}
\end{mathpar}

is interpreted to mean the process derived from P by replacing (in a
capture-avoiding manner) each occurrence of $x$ in $S$ by $y$. For example,

\begin{mathpar}
  P\{ \quotep{\procn{x}|\procn{x}} / x : x \in \freenames{P} \}
\end{mathpar}

will replace each (occurrence) of a free name $x$ in $P$ by
$\quotep{\procn{x}|\procn{x}}$.

Also, we will avail ourselves of the notation $x^{L}$ and $x^{R}$ to
denote injections of a name into disjoint copies of the name
space. There are numerous ways to accomplish this. One example can be
found in \cite{MeredithR05}. This notation overloads to vectors of
names: $\vec{x}^{\pi} := (x_{i}^{\pi} \; : \; 0 \leq i < |\vec{x}| )$ where $\pi \in \{L,R\}$.

We also use $P^{\Box} := P|\Box$.

In \cite{MeredithR05} an interpretation of the new operator is
given. It turns out that there are several possible interpretations
all enjoying the requisite algebraic properties of the operator (see
\cite{milner91polyadicpi}). We will therefore make liberal use of
$(\nu\; \vec{x})P$.

% subsection the_syntax_and_semantics_of_the_notation_system (end)   

\input{qm2pi.qmops} 

\input{qm2pi.sterngerlach} 

\input{qm2pi.metric} 

% section concurrent_process_calculi (end)

%\input{qm2pi.proofsketch}

% section proof sketch (end)

%\input{qm2pi.slviaknots} 

% section spatial logic via knots (end)

\input{qm2pi.conclusion}

% section conclusion (end)

%\input{qm2pi.dtcodes} 

% section wiring algorithm (end)

\input{qm2pi.ack} 

% section acknowledgments (end)

\newpage


\bibliographystyle{plain}   
\bibliography{../../biblios/main.bib}

\input{qm2pi.rhodetails}

\end{document}



% section front matter (end)

\section{Introduction}\label{sec:introduction} % (fold)
In this draft of the material i am going to have to dispense with the
usual writing conventions adopted in papers on these topics. i'm going
to have adopt whatever tone i need at the time i'm writing up the
calculations. Sometimes this may be very conversational; others it may
be the barest mathematical grunts; others still it may be that i have
lifted text from one of my other papers because the exposition of some
point was better said there. i hope that my readers are not unduly put
out by this decision. i'm not doing this to flout convention or be
rebellious. i find these calculations very technically challenging. To
keep everything going technically, something has to give; i have to
let go of some cognitive burden. So, the academic writing style --
with all of its trade-offs in terms of facilitating technical
communication -- is what i'm letting go of. Perhaps subsequent drafts
can be tightened and polished, but for now, i'm going to speak as if
we were sitting together in a coffee shop with a laptop, wifi and a
pad of paper and a pencil.

So, here's what i have to say. We -- you and i, comfortably ensconced
in our coffee shop and well-equipped with our tools -- can realize and
carry out the calculations of quantum mechanics over a very different
formal theory of dynamics, a formal theory of dynamics that
corresponds to a theory of concurrent computation with
\emph{reflection}. It has the advantage that the underlying theory is
already `quantized', but supports analogues all of the continuuous
operations. Strikingly, this underlying theory has recently been
connected with a notion of metric that we can show, by calculating
together, coincides with the metric induced by the inner product.

There are a lot of reasons why you might be interested in seeing
calculations of this form. Here's why i'm interested. For the past
several centuries there has been no competitor to the ``Newtonian''
account of dynamics. As a result the predominant share of accounts of
dynamical systems and situations have had to be formulated in terms of
the Newtonian machinery. i view this as an intellectually dangerous
position to occupy. Everything, despite it's intrinsic shape, turns
into a nail to be hit with this hammer. Recently, however, the theory
of computation has matured to the point where we have candidates for
theories of dynamics that offer very different perspective on
reasoning about dynamical systems and situations. Testing these
candidates against very successful accounts of dynamical situations,
like quantum mechanics, is going to give us some sense of how mature
they are and some measure of the quality of these accounts of
dynamics.

\subsection{Summary of contributions and outline of paper}

So, we're going to develop an interpretation of the operations of
quantum mechanics normally interpreted by Hilbert spaces and
operators. We're going to do this over a theory of computation. Note
that this is very different than the usual quantum computation program
which develops notions of computation over quantum mechanics. Rather,
we are developing a story that aligns with Wheeler's slogan: It from
Bit. To do this we will first provide an account of the theory of
computation at play here. Then we will dive into a calculation-driven
interpretation of the operations of quantum mechanics.

The reason we take this approach is that -- until very recently --
there hasn't been an axiomatic account of quantum mechanics. As a
result there has been no sharp delineation of the mathematical theory
supporting interpretation of the physical theory and the physical
theory, itself. So, ambient features of the maths are free to be
exploited (or supressed) without a real accounting of their physical
relevance. There is no sharp statement ``here's the physical theory''
qua \emph{theory} and ``here's the mathematical interpretation''
enabling a judgment of how faithful the interpretation is -- apart
from experimental observation. When there is an axiomatic account we
can judge how well a given mathematical formalism supports an
interpretation of the axioms, independent of
experimentation. Likewise, we can judge how well we have captured our
physical evidence and experience with our axiomatics, independent of
any specific mathematical implementation, with accidental detail that
may or may not have physical significance. 

In lieu of a fully fleshed out and vetted axiomatic account of quantum
mechanics, interpreting the operational notions in service of modeling
physical systems will have to suffice. In other words, we are not in
the business of providing a model of Hilbert spaces and operators. We
are in the business of providing a model of quantum mechanics because
we are motivated by testing our notions of dynamics against physical
theory; and, the predictive calculations of the physical theory must
serve as the best formulation -- shy of a fully fleshed out axiomatic
account -- of the physical theory itself (as they have for scientific
theories since time immemorial). Put another way, despite a
whole-hearted commitment to an It-from-Bit ontology, we are firmly
aligned with the shut-up-and-calculate camp as the best way to obtain
results either from the physical perspective or as a quality assurance
measure of our fledgling theory of dynamics.

In detail, we present a reflective process calculus. Then we develop
intuitive correspondences between the notions available in this
calculus and the usual physical notions supporting quantum mechanical
calculations. Thus, 

\begin{table}[htp]
  \center{
    \fbox{
      \begin{tabular}{c|c}
        quantum mechanics & process calculus \\
        \hline
        scalar & name \\
        state vector & process \\
        dual & contextual duals \\
        matrix & formal sums of process-context-dual pairs \\
        orthogonality & process annihilation \\
        inner product & execution-formula + quoting
      \end{tabular}
    }
  }
  \caption{QM - process calculi correspondences}
\end{table}

Then we tighten up these intuitions to operational definitions. We
employ the Dirac notation as the best proxy we can find for an
abstract syntax of the quantum mechanical notions. The definitions we
develop put us in contact with equational constraints coming from the
theory that we demonstrate the definitions and calculations satisfy.

This puts us in a position to shut up and calculate for the
Stern-Gerlach experimental set up, showing how these predictive
calculations become calculations on processes in our theory of a
reflective process calculus.

Penultimately, we demonstrate that the notion of metric coming from
the inner product coincides with the notion of metric available from
the theory of bisimulation. This demonstration gives us the right to
think of space as arising from behavior. Finally, we consider where we
might go from the new vantage point we have obtained.

% section introduction (end) 
 
% section introduction (end)

% \documentclass[12pt]{llncs}
%\documentclass{jktr}

\usepackage[pdftex]{hyperref}                   
\usepackage {listings}
\usepackage {mathpartir}
\usepackage{bcprules}
%\usepackage{listings}
                       
\usepackage{graphicx} 
%\usepackage[margins=2.5cm,nohead,nofoot]{geometry}
%\usepackage{geometry}
\usepackage{amsfonts}
\usepackage{amstext}
\usepackage{latexsym}
\usepackage{amssymb}
\usepackage{color}


%\include{myPreamble}
\include{qm2pi.local} 

%\ifpdf
%\usepackage[pdftex]{graphicx}
%\else
%\usepackage{graphicx}
%\fi

 % \ifpdf
%  \usepackage{pdfsync}
%  \if


%\title{Brief Article}
%\author{David F. Snyder}
%\author{L.G. Meredith}

%\address{Dept. of Math., Texas State University--San Marcos, San Marcos, TX 78666}
       
\pagestyle{empty}


\begin{document}

\lstset{language=[Objective]Caml,frame=shadowbox}

\input{qm2pi.front}

% section front matter (end)

\input{qm2pi.intro} 
 
% section introduction (end)

% \input{qm2pi.knotations} 

% section notation (end)

\input{qm2pi.process.calculi} 

% section concurrent_process_calculi_and_spatial_logics_ (end)
    
%\input{qm2pi.knots2pi} 

%\input{qm2pi.trefoil} 

%\input{qm2pi.mainthm} 

% subsection basic_interpretation (end)

%\input{qm2pi.rho.presentation} 
\subsection{The syntax and semantics of the notation system}\label{sub:the_syntax_and_semantics_of_the_notation_system} % (fold)

We now summarize a technical presentation of the calculus that
embodies our theory of dynamics. The typical presentation of such a
calculus follows the style of giving generators and relations on
them. The grammar, below, describing term constructors, freely
generates the set of processes, $\Proc$. This set is then quotiented
by a relation known as structural congruence and it is over this set
that the notion of dynamics is expressed. This presentation is
essentially that of \cite{MeredithR05} with the addition of
polyadicity and summation. For readability we have relegated some of
the technical subtleties to an appendix.

\subsubsection{Process grammar}\label{subsub:process_grammar}

\begin{mathpar}
  \inferrule* [lab=synchronization] {} {{M} \bc \pzero \;|\; x?F \;|\; x!C }
  \and
  \inferrule* [lab=abstraction] {} {{F} \bc (x)P}
  \and
  \inferrule* [lab=concretion] {} {{C} \bc \langle Q \rangle}
  \and
  \inferrule* [lab=process] {} {{P,Q} \bc M \;| \;P|Q \;|\; @{x}}
  \and
  \inferrule* [lab=name] {} {{x} \bc \quotep{P}}
\end{mathpar} 

Note that $\vec{x}$ (resp. $\vec{P}$) denotes a vector of names
(resp. processes) of length $|\vec{x}|$ (resp. $|\vec{P}|$). We adopt
the following useful abbreviations.

\begin{mathpar}
   x?(\vec{y}).P := x.(\vec{y})P \and  x\clift{\vec{P}} := x.\clift{\vec{P}}
   \and x!(y) := \lift{x}{\dropn{y}}
   \and \Pi_{i=0}^{n-1}P_i := P_0 | \ldots | P_{n-1}
\end{mathpar}

\subsubsection{Structural congruence}

\paragraph{Free and bound names and alpha-equivalence.} At the
core of structural equivalence is alpha-equivalence which identifies
process that are the same up to a change of variable. Formally, we
recognize the distinction between free and bound names. The free names
of a process, $\freenames{P}$, may be calculated recursively as
follows:

\begin{mathpar}
\freenames{\pzero} := \emptyset
  \and \\
  \freenames{x?(y).P} := \{ x \} \cup (\freenames{P} \setminus \{ y \})
  \and 
  \freenames{x!\langle P \rangle} := \{ x \} \cup \{ P \} 
  \and \\
  \freenames{P|Q} := \freenames{P} \cup \freenames{Q}
  \and \\
  \freenames{@{x}} := \{ x \}
\end{mathpar}

$\pi$
$\quotep{\pi}$

$\freenames{-} : \pi \to \mathcal{P}(\quotep{\pi})$

\begin{eqnarray*}
  \freenames{\pzero} & := & \emptyset \\
  \freenames{x?(y).P} & := & \{ x \} \cup (\freenames{P} \setminus \{ y \}) \\
  \freenames{x!\langle P \rangle} & := & \{ x \} \cup \{ P \} \\
  \freenames{P|Q} & := & \freenames{P} \cup \freenames{Q} \\
  \freenames{\dropn{x}} & := & \{ x \}
\end{eqnarray*}

The bound names of a process, $\boundnames{P}$, are those names occurring in $P$
that are not free. For example, in $x?(y).0$, the name $x$ is free, while $y$ is bound.

\begin{mathpar}
  \inferrule* [lab=monoidal-laws] {} { P|Q \equiv Q|P \and P|0 \equiv P \and P|(Q|R) \equiv (P|Q)|R }
\end{mathpar}

\begin{mathpar}
  \inferrule* [lab=alpha-equivalence] {} { (x)P \equiv (y)P\{y/x\} \and y \not\in \freenames{P} }
\end{mathpar}

\begin{definition}
Then two processes, $P,Q$, are alpha-equivalent if $P = Q\{\vec{y}/\vec{x}\}$ for
some $\vec{x} \in \boundnames{Q},\vec{y} \in \boundnames{P}$, where $Q\{\vec{y}/\vec{x}\}$
denotes the capture-avoiding substitution of $\vec{y}$ for $\vec{x}$ in $Q$.
\end{definition}

\begin{definition}
  The {\em structural congruence} \cite{SangiorgiWalker} , $\equiv$,
  between processes is the least congruence containing
  alpha-equivalence, satisfying the abelian monoid laws
  (associativity, commutativity and $\pzero$ as identity) for parallel
  composition $|$ and for summation $+$.
\end{definition}

\subsection{Name equivalence}

We take name equivalence, written $\nameeq$, to be the smallest
equivalence relation generated by the following rules.

\begin{mathpar}
\inferrule*[lab=Quote-drop]
{ }
{ \quotep{@{x}} \nameeq x }

\inferrule*[lab=Struct-equiv]
{ P \scong Q }
{ \quotep{P} \nameeq \quotep{Q} }
\end{mathpar}

The astute reader will have noticed that the mutual recursion of names
and processes imposes a mutual recursion on alpha-equivalence and
structural equivalence via name-equivalence. Fortunately, all of this
works out pleasantly and we may calculate in the natural way, free of
concern. The reader interested in the details is referred to the
appendix \ref{appendix:rho_details}.

\subsection{Substitution}

We use $\Proc$ for the set of processes, $\QProc$ for the set of
names, and $\id{\{}\vec{y} / \vec{x} \id{\}}$ to denote partial maps,
$s : \QProc \rightarrow \QProc$. A map, $s$ lifts, uniquely, to a map
on process terms, $\widehat{s} : \Proc \rightarrow \Proc$ by the
following equations.

\begin{mathpar}
  (0) \psubstp{Q}{P} := 0 \\
  (R \juxtap S) \psubstp{Q}{P}
  :=    
  (R)\psubstp{Q}{P} \juxtap (S) \psubstp{Q}{P} \\
  (x?(y).R) \psubstp{Q}{P}    
  :=    
  (x)\substp{Q}{P} (z)\concat( (R \psubstn{z}{y}) \psubstp{Q}{P} ) \\
  (\lift{x}{R}) \psubstp{Q}{P}  
  :=
  \lift{(x)\substp{Q}{P}}{ R \psubstp{Q}{P} } \\
%   (\dropn{x})  \psubstp{Q}{P}       
%   := 
%   \left\{ 
%     \begin{array}{ccc} 
%       \dropn{\quotep{Q}} & & x \nameeq \quotep{P} \\
%       \dropn{x} & & otherwise \\
%     \end{array}
%   \right. 
  (\dropn{x})  \psubstp{Q}{P}       
  := 
  \left\{ 
    \begin{array}{ccc} 
      Q & & x \nameeq \quotep{P} \\
      \dropn{x} & & otherwise \\
    \end{array}
  \right.
\end{mathpar}
 

where

\begin{eqnarray}
  (x)\id{\{} \lpquote Q \rpquote / \lpquote P \rpquote \id{\}}            = 
  \left\{ 
    \begin{array}{ccc}
      \lpquote Q \rpquote & & x \nameeq \lpquote P \rpquote \\
      x & & otherwise \\
    \end{array}
  \right. \nonumber
\end{eqnarray}

and $z$ is chosen distinct from $\quotep{P}$, $\quotep{Q}$, the free
names in $Q$, and all the names in $R$. Our $\alpha$-equivalence will
be built in the standard way from this substitution.

\begin{remark}\label{rem:no_self_referential_names}
  One consequence of these definitions is that $\forall P. \quotep{P}
  \not\in \freenames{P}$.
\end{remark}

\subsection{ Dynamic quote: an example }

Anticipating something of what's to come, consider applying the
substitution, $\widehat{\id{\{}u / z \id{\}}}$, to the following pair
of processes, $\lift{w}{y!(z)}$ and $w[ \lpquote y!(z) \rpquote ]$.

\begin{eqnarray}
	\lift{w}{y!(z)}\widehat{\id{\{}u / z \id{\}}}
		& = &
		\lift{w}{y!(u)} \nonumber\\
	w[ \lpquote y!(z) \rpquote ] \widehat{ \id{\{}u / z \id{\}} }
		& = &
		w[ \lpquote y!(z) \rpquote ] \nonumber
\end{eqnarray}

Because the body of the process between quotes is impervious to
substitution, we get radically different answers. In fact, by
examining the first process in an input context,
e.g. $x?(z).\lift{w}{y!(z)}$, we see that the process under the lift
operator may be shaped by prefixed inputs binding a name inside it. In
this sense, the lift operator will be seen as a way to dynamically
construct processes before reifying them as names.

Finally equipped with these standard features we can present the
dynamics of the calculus.

\subsubsection{Operational semantics} 

Finally, we introduce the computational dynamics. What marks these
algebras as distinct from other more traditionally studied algebraic
structures, e.g. vector spaces or polynomial rings, is the manner in
which dynamics is captured. In traditional structures, dynamics is typically
expressed through morphisms between such structures, as in linear maps
between vector spaces or morphisms between rings. In algebras
associated with the semantics of computation, the dynamics is
expressed as part of the algebraic structure itself, through a
reduction reduction relation typically denoted by $\red$. Below, we
give a recursive presentation of this relation for the calculus used
in the encoding.

$\red \subseteq \pi \times \pi$
$\red : \pi \to \mathcal{P}(\pi)$

\begin{mathpar}
  \inferrule* [lab=Comm] { \textsf{match}( x_{src}, x_{trgt} ) } { x_{trgt}?(y)P \; | \; x_{src}!\langle {Q} \rangle \red P\{\quotep{Q}/y}\} }
  \and \\
  \inferrule* [lab=Par] {{P} \red {P}'} {{{P} | {Q}} \red {{P}' | {Q}}}
  \and
  \inferrule* [lab=Equiv]{{{P} \scong {P}'} \andalso {{P}' \red {Q}'} \andalso {{Q}' \scong {Q}}}{{P} \red {Q}}
\end{mathpar}

\begin{eqnarray*}
  match_{\equiv} (\quotep{P},\quotep{Q}) & := & P \equiv Q \\
  match_{\dagger}(\quotep{P},\quotep{Q}) & := & \forall R. P|Q \red^{*} R => R \red^{*} 0 \\
  match_{K}(\quotep{P},\quotep{Q}) & := & K \mbox{ for some context } K
\end{eqnarray*}

$u?(x)P | u!\langle Q \rangle \red P\{\quotep{Q}/x\}$

%We write $\wred$ for $\red^*$, and $P\red$ if $\exists Q $ such that $ P \red Q$.
We write $P\red$ if $\exists Q $ such that $ P \red Q$ and $P\not\red$, otherwise.

\section{Replication}

As mentioned before, it is known that replication (and hence
recursion) can be implemented in a higher-order process algebra
\cite{SangiorgiWalker}. As our first example of calculation with the
machinery thus far presented we give the construction explicitly in
the {\rhoc}.

\begin{eqnarray}
	D_{x} & := & \prefix{x}{y}{(\binpar{\outputp{x}{y}}{@{y}})} \nonumber\\
	\bangp_{x}{P} & := & \binpar{{x}!\langle{\binpar{D_{x}}{P}}\rangle}{D_{x}} \nonumber
\end{eqnarray}

\begin{eqnarray}
	\bangp_{x}{P} & & \nonumber\\
	=
	& {x}!\langle{(\prefix{x}{y}{(\outputp{x}{y} | @{y})) | P}}\rangle 
	      | \prefix{x}{y}{(\outputp{x}{y} | @{y})} & \nonumber\\
	\red
	& (\outputp{x}{y} | @{y})\substn{\quotep{(\prefix{x}{y}{(@{y} | \outputp{x}{y})) | P}}}{y} & \nonumber\\
	=
	& \outputp{x}{\quotep{(\prefix{x}{y}{(\outputp{x}{y} | @{y})) | P}}}
	  | {(\prefix{x}{y}{(\outputp{x}{y} | @{y})) | P}} & \nonumber\\
	\red
	& \ldots & \nonumber\\
	\red^*
	& P | P | \ldots & \nonumber
\end{eqnarray}

Of course, this encoding, as an implementation, runs away, unfolding
$\bangp{P}$ eagerly. A lazier and more implementable replication
operator, restricted to input-guarded processes, may be obtained as follows.

\begin{eqnarray}
\bangp{\prefix{u}{v}{P}} 
	:= 
	\binpar{\lift{x}{\prefix{u}{v}{(\binpar{D(x)}{P})}}}{D(x)} \nonumber
\end{eqnarray}

\begin{remark}
  Note that the lazier definition still does not deal with summation
  or mixed summation (i.e. sums over input and output). The reader is
  invited to construct definitions of replication that deal with these
  features. 

  Further, the definitions are parameterized in a name, $x$. Can you,
  gentle reader, make a definition that eliminates this parameter and
  guarantees no accidental interaction between the replication
  machinery and the process being replicated -- i.e. no accidental
  sharing of names used by the process to get its work done and the
  name(s) used by the replication to effect copying. This latter
  revision of the definition of replication is crucial to obtaining
  the expected identity $!!P \sim !P$.
\end{remark}

\begin{remark}\label{rem:paradoxical_combinator}
  The reader familiar with the lambda calculus will have noticed the
  similarity between $D$ and the paradoxical combinator.

  [Ed. note: the existence of this seems to suggest we have to be more
  restrictive on the set of processes and names we admit if we are to
  support no-cloning.]
\end{remark}

\subsubsection{Bisimulation}

The computational dynamics gives rise to another kind of equivalence,
the equivalence of computational behavior. As previously mentioned
this is typically captured \emph{via} some form of bisimulation.

% The notion we use in this paper is weak barbed bisimulation
% \cite{milner91polyadicpi}.

The notion we use in this paper is derived from weak barbed
bisimulation \cite{milner91polyadicpi}. 

\begin{definition}
An \emph{observation relation}, $\downarrow_{\mathcal N}$, over a set
of names, $\mathcal N$, is the smallest relation satisfying the rules
below.

\infrule[Out-barb]{y \in {\mathcal N}, \; x \nameeq y}
		  {\outputp{x}{v} \downarrow_{\mathcal N} x}
\infrule[Par-barb]{\mbox{$P\downarrow_{\mathcal N} x$ or $Q\downarrow_{\mathcal N} x$}}
		  {\binpar{P}{Q} \downarrow_{\mathcal N} x}

We write $P \Downarrow_{\mathcal N} x$ if there is $Q$ such that 
$P \wred Q$ and $Q \downarrow_{\mathcal N} x$.
\end{definition}

\begin{definition}
%\label{def.bbisim}
An  ${\mathcal N}$-\emph{barbed bisimulation} over a set of names, ${\mathcal N}$, is a symmetric binary relation 
${\mathcal S}_{\mathcal N}$ between agents such that $P\rel{S}_{\mathcal N}Q$ implies:
\begin{enumerate}
\item If $P \red P'$ then $Q \wred Q'$ and $P'\rel{S}_{\mathcal N} Q'$.
\item If $P\downarrow_{\mathcal N} x$, then $Q\Downarrow_{\mathcal N} x$.
\end{enumerate}
$P$ is ${\mathcal N}$-barbed bisimilar to $Q$, written
$P \wbbisim_{\mathcal N} Q$, if $P \rel{S}_{\mathcal N} Q$ for some ${\mathcal N}$-barbed bisimulation ${\mathcal S}_{\mathcal N}$.
\end{definition}

$\mathcal{R} \subseteq \pi \times \pi$

$P \mathcal{R} Q => \forall P'. P \red P' \Rightarrow \exists Q'. Q \red Q', P' \mathcal{R} Q'$

$P \vdash x \Rightarrow Q \vdash x$

\begin{mathpar}
  \inferrule*[lab=Out-barb]{x \nameeq y}{{y}!\langle{Q}\rangle \vdash x}
  \and
  \inferrule*[lab=Par-barb]{\mbox{$P\vdash x$ or $Q\vdash x$}}{\binpar{P}{Q} \vdash x}
\end{mathpar}

\subsubsection{Contexts}

One of the principle advantages of computational calculi like the
$\pi$-calculus is a well-defined notion of context,
contextual-equivalence and a correlation between
contextual-equivalence and notions of bisimulation. The notion of
context allows the decomposition of a process into (sub-)process and
its syntactic environment, its context. Thus, a context may be
thought of as a process with a ``hole'' (written $\Box$) in it. The
application of a context $M$ to a process $P$, written $M[P]$, is
tantamount to filling the hole in $M$ with $P$. In this paper we do
not need the full weight of this theory, but do make use of the notion
of context in the proof the main theorem. 

\begin{mathpar}
  \inferrule* [lab=summation] {} {{M_{M},M_{N}} \bc \Box \;|\; x.M_{A} \;|\; M_{M}+M_{N}}
  \and
  \inferrule* [lab=agent] {} {{M_{A}} \bc (\vec{x})M_{P} \;| \; \clift{P_0,\ldots,M_{P},\ldots,P_N}}
  \and \\
  \inferrule* [lab=process] {} {{M_{P}} \bc M_{N} \;| \;P|M_{P} }
\end{mathpar} 

\begin{mathpar}
  \inferrule* [lab=sychronization] {} {M_{N} \bc \Box \;|\; x?M_{F} \;|\; x!M_{C}}
  \and
  \inferrule* [lab=abstraction] {} {{M_{F}} \bc (x)M_{P} }
  \and
  \inferrule* [lab=concretion] {} {{M_{C}} \bc \langle M_{P} \rangle }
  \and \\
  \inferrule* [lab=process] {} {{M_{P}} \bc M_{N} \;| \;P|M_{P} }
\end{mathpar}

\begin{definition}[contextual application] Given a context $M$, and
  process $P$, we define the \emph{contextual application}, $M[P] :=
  M\{P/\Box\}$. That is, the contextual application of M to P is the
  substitution of $P$ for $\Box$ in $M$.
\end{definition}

$\meaningof{-} : L \to \mathcal{P}(\pi)$

\begin{mathpar}
  \inferrule* [lab=collection] {} {\meaningof{true} = \pi, \and \meaningof{~E} = \pi \setminus \meaningof{E}, \and \meaningof{E_{1} \& E_{2}} = \meaningof{E_{1}} \cap \meaningof{E_{2}}}
\end{mathpar}

\begin{mathpar}
  \inferrule* [lab=structure] {} {\meaningof{0} = \{ P \in \pi | P \equiv 0 \}, \and \\ \meaningof{E_1 | E_2} = \{ P \in \pi | P \equiv P_{1} | P_{2}, P_{1} \in \meaningof{E_{1}}, P_{2} \in \meaningof{E_2}\} }
\end{mathpar}

\begin{mathpar}
 \inferrule* [lab=behavior] {} {\meaningof{\langle a?b \rangle E} = \{ P \in \pi | P \equiv Q | u?(y)P', \\ \and \\\\ \and \\ \;\;\; u \in \meaningof{a}, \forall z.P'\{z/y\} \in \meaningof{E\{z/b\}}\}, \and \\ \meaningof{a!E} = \{ P \in \pi | P \equiv Q | x!\langle P' \rangle, x \in \meaningof{a} P' \in \meaningof{E}\} }
\end{mathpar}

\begin{mathpar}
 \inferrule* [lab=nominal] {} {\meaningof{\quotep{E}} = \{ \quotep{P} \in \quotep{\pi} | P \in \meaningof{E} \}, \and \meaningof{\quotep{P}} = \{ \quotep{Q} \in \quotep{\pi} | P \equiv Q \} \and \\ \meaningof{@\quotep{E}} = \{ P \in \pi | P \equiv @x, x \in \meaningof{E} \}}
\end{mathpar}

\begin{eqnarray*}
  \\
  \meaningof{-} : TS \to ST
\end{eqnarray*}

\begin{eqnarray*}
  \\
  L : TS \to ST
\end{eqnarray*}

\begin{eqnarray*}
  \\
  P \models E \iff P \in \meaningof{E}
\end{eqnarray*}

\begin{eqnarray*}
  P \approx_{L} Q \iff \forall E \in L. P \models E \iff Q \models E
\end{eqnarray*}

\begin{eqnarray*}
  P \approx_{K} Q
\end{eqnarray*}

\begin{eqnarray*}
  P \approx Q
\end{eqnarray*}

$\approx_{K} = \approx = \approx_{L}$

\subsubsection{Contextual duality}

Note that contexts extend the quotation operation to a family of
operations from processes to names. Given a context, $M$, we can
define a \emph{nominal context}, $\quotep{M}$ by $\quotep{M}[P] :=
\quotep{M[P]}$. To foreshadow what is to come we observe that these
operations enjoy a duality with processes very much like the duality
between vectors and maps from vectors to scalars.

Further, because the calculus is essentially higher-order, we have a
correspondence between contexts and processes. More specifically,
given a name $x$ and a context $M$ we can construct $M^{*}_{x}$ such
that 

\begin{mathpar}
  M^{*}_{x} | \lift{x}{P} \red M[P]
\end{mathpar}

namely,

\begin{mathpar}
  M^{*}_{x} := x?(u).M[\dropn{u}]
\end{mathpar}

The dependence of $M^{*}_{x}$ on a name makes it an abstraction, 

\begin{mathpar}
  M^{*} := (x)x?(u).M[\dropn{u}]
\end{mathpar}

\subsection{Additional notation}

It will sometimes be convenient to denote the process a name
quotes. We already have the notation $x = \quotep{P}$, but it will be
convenient to introduce an alternate notation, $\procn{x}$, when we
want to emphasize the connection to the use of the name. Note that, by
virtue of name equivalence, $\quotep{\procn{x}} \nameeq x$; so, the
notation is consistent with previous definitions.

Further, because names have structure it is possible to effect
substitutions on the basis of that structure. This means we need to
upgrade our notation for substitutions, which we accomplish by
adapting comprehension notation. Thus,

\begin{mathpar}
  P\{ y / x : x \in S \}
\end{mathpar}

is interpreted to mean the process derived from P by replacing (in a
capture-avoiding manner) each occurrence of $x$ in $S$ by $y$. For example,

\begin{mathpar}
  P\{ \quotep{\procn{x}|\procn{x}} / x : x \in \freenames{P} \}
\end{mathpar}

will replace each (occurrence) of a free name $x$ in $P$ by
$\quotep{\procn{x}|\procn{x}}$.

Also, we will avail ourselves of the notation $x^{L}$ and $x^{R}$ to
denote injections of a name into disjoint copies of the name
space. There are numerous ways to accomplish this. One example can be
found in \cite{MeredithR05}. This notation overloads to vectors of
names: $\vec{x}^{\pi} := (x_{i}^{\pi} \; : \; 0 \leq i < |\vec{x}| )$ where $\pi \in \{L,R\}$.

We also use $P^{\Box} := P|\Box$.

In \cite{MeredithR05} an interpretation of the new operator is
given. It turns out that there are several possible interpretations
all enjoying the requisite algebraic properties of the operator (see
\cite{milner91polyadicpi}). We will therefore make liberal use of
$(\nu\; \vec{x})P$.

% subsection the_syntax_and_semantics_of_the_notation_system (end)   

\input{qm2pi.qmops} 

\input{qm2pi.sterngerlach} 

\input{qm2pi.metric} 

% section concurrent_process_calculi (end)

%\input{qm2pi.proofsketch}

% section proof sketch (end)

%\input{qm2pi.slviaknots} 

% section spatial logic via knots (end)

\input{qm2pi.conclusion}

% section conclusion (end)

%\input{qm2pi.dtcodes} 

% section wiring algorithm (end)

\input{qm2pi.ack} 

% section acknowledgments (end)

\newpage


\bibliographystyle{plain}   
\bibliography{../../biblios/main.bib}

\input{qm2pi.rhodetails}

\end{document}

 

% section notation (end)

\input{qm2pi.process.calculi} 

% section concurrent_process_calculi_and_spatial_logics_ (end)
    
%\documentclass[12pt]{llncs}
%\documentclass{jktr}

\usepackage[pdftex]{hyperref}                   
\usepackage {listings}
\usepackage {mathpartir}
\usepackage{bcprules}
%\usepackage{listings}
                       
\usepackage{graphicx} 
%\usepackage[margins=2.5cm,nohead,nofoot]{geometry}
%\usepackage{geometry}
\usepackage{amsfonts}
\usepackage{amstext}
\usepackage{latexsym}
\usepackage{amssymb}
\usepackage{color}


%\include{myPreamble}
\include{qm2pi.local} 

%\ifpdf
%\usepackage[pdftex]{graphicx}
%\else
%\usepackage{graphicx}
%\fi

 % \ifpdf
%  \usepackage{pdfsync}
%  \if


%\title{Brief Article}
%\author{David F. Snyder}
%\author{L.G. Meredith}

%\address{Dept. of Math., Texas State University--San Marcos, San Marcos, TX 78666}
       
\pagestyle{empty}


\begin{document}

\lstset{language=[Objective]Caml,frame=shadowbox}

\input{qm2pi.front}

% section front matter (end)

\input{qm2pi.intro} 
 
% section introduction (end)

% \input{qm2pi.knotations} 

% section notation (end)

\input{qm2pi.process.calculi} 

% section concurrent_process_calculi_and_spatial_logics_ (end)
    
%\input{qm2pi.knots2pi} 

%\input{qm2pi.trefoil} 

%\input{qm2pi.mainthm} 

% subsection basic_interpretation (end)

%\input{qm2pi.rho.presentation} 
\subsection{The syntax and semantics of the notation system}\label{sub:the_syntax_and_semantics_of_the_notation_system} % (fold)

We now summarize a technical presentation of the calculus that
embodies our theory of dynamics. The typical presentation of such a
calculus follows the style of giving generators and relations on
them. The grammar, below, describing term constructors, freely
generates the set of processes, $\Proc$. This set is then quotiented
by a relation known as structural congruence and it is over this set
that the notion of dynamics is expressed. This presentation is
essentially that of \cite{MeredithR05} with the addition of
polyadicity and summation. For readability we have relegated some of
the technical subtleties to an appendix.

\subsubsection{Process grammar}\label{subsub:process_grammar}

\begin{mathpar}
  \inferrule* [lab=synchronization] {} {{M} \bc \pzero \;|\; x?F \;|\; x!C }
  \and
  \inferrule* [lab=abstraction] {} {{F} \bc (x)P}
  \and
  \inferrule* [lab=concretion] {} {{C} \bc \langle Q \rangle}
  \and
  \inferrule* [lab=process] {} {{P,Q} \bc M \;| \;P|Q \;|\; @{x}}
  \and
  \inferrule* [lab=name] {} {{x} \bc \quotep{P}}
\end{mathpar} 

Note that $\vec{x}$ (resp. $\vec{P}$) denotes a vector of names
(resp. processes) of length $|\vec{x}|$ (resp. $|\vec{P}|$). We adopt
the following useful abbreviations.

\begin{mathpar}
   x?(\vec{y}).P := x.(\vec{y})P \and  x\clift{\vec{P}} := x.\clift{\vec{P}}
   \and x!(y) := \lift{x}{\dropn{y}}
   \and \Pi_{i=0}^{n-1}P_i := P_0 | \ldots | P_{n-1}
\end{mathpar}

\subsubsection{Structural congruence}

\paragraph{Free and bound names and alpha-equivalence.} At the
core of structural equivalence is alpha-equivalence which identifies
process that are the same up to a change of variable. Formally, we
recognize the distinction between free and bound names. The free names
of a process, $\freenames{P}$, may be calculated recursively as
follows:

\begin{mathpar}
\freenames{\pzero} := \emptyset
  \and \\
  \freenames{x?(y).P} := \{ x \} \cup (\freenames{P} \setminus \{ y \})
  \and 
  \freenames{x!\langle P \rangle} := \{ x \} \cup \{ P \} 
  \and \\
  \freenames{P|Q} := \freenames{P} \cup \freenames{Q}
  \and \\
  \freenames{@{x}} := \{ x \}
\end{mathpar}

$\pi$
$\quotep{\pi}$

$\freenames{-} : \pi \to \mathcal{P}(\quotep{\pi})$

\begin{eqnarray*}
  \freenames{\pzero} & := & \emptyset \\
  \freenames{x?(y).P} & := & \{ x \} \cup (\freenames{P} \setminus \{ y \}) \\
  \freenames{x!\langle P \rangle} & := & \{ x \} \cup \{ P \} \\
  \freenames{P|Q} & := & \freenames{P} \cup \freenames{Q} \\
  \freenames{\dropn{x}} & := & \{ x \}
\end{eqnarray*}

The bound names of a process, $\boundnames{P}$, are those names occurring in $P$
that are not free. For example, in $x?(y).0$, the name $x$ is free, while $y$ is bound.

\begin{mathpar}
  \inferrule* [lab=monoidal-laws] {} { P|Q \equiv Q|P \and P|0 \equiv P \and P|(Q|R) \equiv (P|Q)|R }
\end{mathpar}

\begin{mathpar}
  \inferrule* [lab=alpha-equivalence] {} { (x)P \equiv (y)P\{y/x\} \and y \not\in \freenames{P} }
\end{mathpar}

\begin{definition}
Then two processes, $P,Q$, are alpha-equivalent if $P = Q\{\vec{y}/\vec{x}\}$ for
some $\vec{x} \in \boundnames{Q},\vec{y} \in \boundnames{P}$, where $Q\{\vec{y}/\vec{x}\}$
denotes the capture-avoiding substitution of $\vec{y}$ for $\vec{x}$ in $Q$.
\end{definition}

\begin{definition}
  The {\em structural congruence} \cite{SangiorgiWalker} , $\equiv$,
  between processes is the least congruence containing
  alpha-equivalence, satisfying the abelian monoid laws
  (associativity, commutativity and $\pzero$ as identity) for parallel
  composition $|$ and for summation $+$.
\end{definition}

\subsection{Name equivalence}

We take name equivalence, written $\nameeq$, to be the smallest
equivalence relation generated by the following rules.

\begin{mathpar}
\inferrule*[lab=Quote-drop]
{ }
{ \quotep{@{x}} \nameeq x }

\inferrule*[lab=Struct-equiv]
{ P \scong Q }
{ \quotep{P} \nameeq \quotep{Q} }
\end{mathpar}

The astute reader will have noticed that the mutual recursion of names
and processes imposes a mutual recursion on alpha-equivalence and
structural equivalence via name-equivalence. Fortunately, all of this
works out pleasantly and we may calculate in the natural way, free of
concern. The reader interested in the details is referred to the
appendix \ref{appendix:rho_details}.

\subsection{Substitution}

We use $\Proc$ for the set of processes, $\QProc$ for the set of
names, and $\id{\{}\vec{y} / \vec{x} \id{\}}$ to denote partial maps,
$s : \QProc \rightarrow \QProc$. A map, $s$ lifts, uniquely, to a map
on process terms, $\widehat{s} : \Proc \rightarrow \Proc$ by the
following equations.

\begin{mathpar}
  (0) \psubstp{Q}{P} := 0 \\
  (R \juxtap S) \psubstp{Q}{P}
  :=    
  (R)\psubstp{Q}{P} \juxtap (S) \psubstp{Q}{P} \\
  (x?(y).R) \psubstp{Q}{P}    
  :=    
  (x)\substp{Q}{P} (z)\concat( (R \psubstn{z}{y}) \psubstp{Q}{P} ) \\
  (\lift{x}{R}) \psubstp{Q}{P}  
  :=
  \lift{(x)\substp{Q}{P}}{ R \psubstp{Q}{P} } \\
%   (\dropn{x})  \psubstp{Q}{P}       
%   := 
%   \left\{ 
%     \begin{array}{ccc} 
%       \dropn{\quotep{Q}} & & x \nameeq \quotep{P} \\
%       \dropn{x} & & otherwise \\
%     \end{array}
%   \right. 
  (\dropn{x})  \psubstp{Q}{P}       
  := 
  \left\{ 
    \begin{array}{ccc} 
      Q & & x \nameeq \quotep{P} \\
      \dropn{x} & & otherwise \\
    \end{array}
  \right.
\end{mathpar}
 

where

\begin{eqnarray}
  (x)\id{\{} \lpquote Q \rpquote / \lpquote P \rpquote \id{\}}            = 
  \left\{ 
    \begin{array}{ccc}
      \lpquote Q \rpquote & & x \nameeq \lpquote P \rpquote \\
      x & & otherwise \\
    \end{array}
  \right. \nonumber
\end{eqnarray}

and $z$ is chosen distinct from $\quotep{P}$, $\quotep{Q}$, the free
names in $Q$, and all the names in $R$. Our $\alpha$-equivalence will
be built in the standard way from this substitution.

\begin{remark}\label{rem:no_self_referential_names}
  One consequence of these definitions is that $\forall P. \quotep{P}
  \not\in \freenames{P}$.
\end{remark}

\subsection{ Dynamic quote: an example }

Anticipating something of what's to come, consider applying the
substitution, $\widehat{\id{\{}u / z \id{\}}}$, to the following pair
of processes, $\lift{w}{y!(z)}$ and $w[ \lpquote y!(z) \rpquote ]$.

\begin{eqnarray}
	\lift{w}{y!(z)}\widehat{\id{\{}u / z \id{\}}}
		& = &
		\lift{w}{y!(u)} \nonumber\\
	w[ \lpquote y!(z) \rpquote ] \widehat{ \id{\{}u / z \id{\}} }
		& = &
		w[ \lpquote y!(z) \rpquote ] \nonumber
\end{eqnarray}

Because the body of the process between quotes is impervious to
substitution, we get radically different answers. In fact, by
examining the first process in an input context,
e.g. $x?(z).\lift{w}{y!(z)}$, we see that the process under the lift
operator may be shaped by prefixed inputs binding a name inside it. In
this sense, the lift operator will be seen as a way to dynamically
construct processes before reifying them as names.

Finally equipped with these standard features we can present the
dynamics of the calculus.

\subsubsection{Operational semantics} 

Finally, we introduce the computational dynamics. What marks these
algebras as distinct from other more traditionally studied algebraic
structures, e.g. vector spaces or polynomial rings, is the manner in
which dynamics is captured. In traditional structures, dynamics is typically
expressed through morphisms between such structures, as in linear maps
between vector spaces or morphisms between rings. In algebras
associated with the semantics of computation, the dynamics is
expressed as part of the algebraic structure itself, through a
reduction reduction relation typically denoted by $\red$. Below, we
give a recursive presentation of this relation for the calculus used
in the encoding.

$\red \subseteq \pi \times \pi$
$\red : \pi \to \mathcal{P}(\pi)$

\begin{mathpar}
  \inferrule* [lab=Comm] { \textsf{match}( x_{src}, x_{trgt} ) } { x_{trgt}?(y)P \; | \; x_{src}!\langle {Q} \rangle \red P\{\quotep{Q}/y}\} }
  \and \\
  \inferrule* [lab=Par] {{P} \red {P}'} {{{P} | {Q}} \red {{P}' | {Q}}}
  \and
  \inferrule* [lab=Equiv]{{{P} \scong {P}'} \andalso {{P}' \red {Q}'} \andalso {{Q}' \scong {Q}}}{{P} \red {Q}}
\end{mathpar}

\begin{eqnarray*}
  match_{\equiv} (\quotep{P},\quotep{Q}) & := & P \equiv Q \\
  match_{\dagger}(\quotep{P},\quotep{Q}) & := & \forall R. P|Q \red^{*} R => R \red^{*} 0 \\
  match_{K}(\quotep{P},\quotep{Q}) & := & K \mbox{ for some context } K
\end{eqnarray*}

$u?(x)P | u!\langle Q \rangle \red P\{\quotep{Q}/x\}$

%We write $\wred$ for $\red^*$, and $P\red$ if $\exists Q $ such that $ P \red Q$.
We write $P\red$ if $\exists Q $ such that $ P \red Q$ and $P\not\red$, otherwise.

\section{Replication}

As mentioned before, it is known that replication (and hence
recursion) can be implemented in a higher-order process algebra
\cite{SangiorgiWalker}. As our first example of calculation with the
machinery thus far presented we give the construction explicitly in
the {\rhoc}.

\begin{eqnarray}
	D_{x} & := & \prefix{x}{y}{(\binpar{\outputp{x}{y}}{@{y}})} \nonumber\\
	\bangp_{x}{P} & := & \binpar{{x}!\langle{\binpar{D_{x}}{P}}\rangle}{D_{x}} \nonumber
\end{eqnarray}

\begin{eqnarray}
	\bangp_{x}{P} & & \nonumber\\
	=
	& {x}!\langle{(\prefix{x}{y}{(\outputp{x}{y} | @{y})) | P}}\rangle 
	      | \prefix{x}{y}{(\outputp{x}{y} | @{y})} & \nonumber\\
	\red
	& (\outputp{x}{y} | @{y})\substn{\quotep{(\prefix{x}{y}{(@{y} | \outputp{x}{y})) | P}}}{y} & \nonumber\\
	=
	& \outputp{x}{\quotep{(\prefix{x}{y}{(\outputp{x}{y} | @{y})) | P}}}
	  | {(\prefix{x}{y}{(\outputp{x}{y} | @{y})) | P}} & \nonumber\\
	\red
	& \ldots & \nonumber\\
	\red^*
	& P | P | \ldots & \nonumber
\end{eqnarray}

Of course, this encoding, as an implementation, runs away, unfolding
$\bangp{P}$ eagerly. A lazier and more implementable replication
operator, restricted to input-guarded processes, may be obtained as follows.

\begin{eqnarray}
\bangp{\prefix{u}{v}{P}} 
	:= 
	\binpar{\lift{x}{\prefix{u}{v}{(\binpar{D(x)}{P})}}}{D(x)} \nonumber
\end{eqnarray}

\begin{remark}
  Note that the lazier definition still does not deal with summation
  or mixed summation (i.e. sums over input and output). The reader is
  invited to construct definitions of replication that deal with these
  features. 

  Further, the definitions are parameterized in a name, $x$. Can you,
  gentle reader, make a definition that eliminates this parameter and
  guarantees no accidental interaction between the replication
  machinery and the process being replicated -- i.e. no accidental
  sharing of names used by the process to get its work done and the
  name(s) used by the replication to effect copying. This latter
  revision of the definition of replication is crucial to obtaining
  the expected identity $!!P \sim !P$.
\end{remark}

\begin{remark}\label{rem:paradoxical_combinator}
  The reader familiar with the lambda calculus will have noticed the
  similarity between $D$ and the paradoxical combinator.

  [Ed. note: the existence of this seems to suggest we have to be more
  restrictive on the set of processes and names we admit if we are to
  support no-cloning.]
\end{remark}

\subsubsection{Bisimulation}

The computational dynamics gives rise to another kind of equivalence,
the equivalence of computational behavior. As previously mentioned
this is typically captured \emph{via} some form of bisimulation.

% The notion we use in this paper is weak barbed bisimulation
% \cite{milner91polyadicpi}.

The notion we use in this paper is derived from weak barbed
bisimulation \cite{milner91polyadicpi}. 

\begin{definition}
An \emph{observation relation}, $\downarrow_{\mathcal N}$, over a set
of names, $\mathcal N$, is the smallest relation satisfying the rules
below.

\infrule[Out-barb]{y \in {\mathcal N}, \; x \nameeq y}
		  {\outputp{x}{v} \downarrow_{\mathcal N} x}
\infrule[Par-barb]{\mbox{$P\downarrow_{\mathcal N} x$ or $Q\downarrow_{\mathcal N} x$}}
		  {\binpar{P}{Q} \downarrow_{\mathcal N} x}

We write $P \Downarrow_{\mathcal N} x$ if there is $Q$ such that 
$P \wred Q$ and $Q \downarrow_{\mathcal N} x$.
\end{definition}

\begin{definition}
%\label{def.bbisim}
An  ${\mathcal N}$-\emph{barbed bisimulation} over a set of names, ${\mathcal N}$, is a symmetric binary relation 
${\mathcal S}_{\mathcal N}$ between agents such that $P\rel{S}_{\mathcal N}Q$ implies:
\begin{enumerate}
\item If $P \red P'$ then $Q \wred Q'$ and $P'\rel{S}_{\mathcal N} Q'$.
\item If $P\downarrow_{\mathcal N} x$, then $Q\Downarrow_{\mathcal N} x$.
\end{enumerate}
$P$ is ${\mathcal N}$-barbed bisimilar to $Q$, written
$P \wbbisim_{\mathcal N} Q$, if $P \rel{S}_{\mathcal N} Q$ for some ${\mathcal N}$-barbed bisimulation ${\mathcal S}_{\mathcal N}$.
\end{definition}

$\mathcal{R} \subseteq \pi \times \pi$

$P \mathcal{R} Q => \forall P'. P \red P' \Rightarrow \exists Q'. Q \red Q', P' \mathcal{R} Q'$

$P \vdash x \Rightarrow Q \vdash x$

\begin{mathpar}
  \inferrule*[lab=Out-barb]{x \nameeq y}{{y}!\langle{Q}\rangle \vdash x}
  \and
  \inferrule*[lab=Par-barb]{\mbox{$P\vdash x$ or $Q\vdash x$}}{\binpar{P}{Q} \vdash x}
\end{mathpar}

\subsubsection{Contexts}

One of the principle advantages of computational calculi like the
$\pi$-calculus is a well-defined notion of context,
contextual-equivalence and a correlation between
contextual-equivalence and notions of bisimulation. The notion of
context allows the decomposition of a process into (sub-)process and
its syntactic environment, its context. Thus, a context may be
thought of as a process with a ``hole'' (written $\Box$) in it. The
application of a context $M$ to a process $P$, written $M[P]$, is
tantamount to filling the hole in $M$ with $P$. In this paper we do
not need the full weight of this theory, but do make use of the notion
of context in the proof the main theorem. 

\begin{mathpar}
  \inferrule* [lab=summation] {} {{M_{M},M_{N}} \bc \Box \;|\; x.M_{A} \;|\; M_{M}+M_{N}}
  \and
  \inferrule* [lab=agent] {} {{M_{A}} \bc (\vec{x})M_{P} \;| \; \clift{P_0,\ldots,M_{P},\ldots,P_N}}
  \and \\
  \inferrule* [lab=process] {} {{M_{P}} \bc M_{N} \;| \;P|M_{P} }
\end{mathpar} 

\begin{mathpar}
  \inferrule* [lab=sychronization] {} {M_{N} \bc \Box \;|\; x?M_{F} \;|\; x!M_{C}}
  \and
  \inferrule* [lab=abstraction] {} {{M_{F}} \bc (x)M_{P} }
  \and
  \inferrule* [lab=concretion] {} {{M_{C}} \bc \langle M_{P} \rangle }
  \and \\
  \inferrule* [lab=process] {} {{M_{P}} \bc M_{N} \;| \;P|M_{P} }
\end{mathpar}

\begin{definition}[contextual application] Given a context $M$, and
  process $P$, we define the \emph{contextual application}, $M[P] :=
  M\{P/\Box\}$. That is, the contextual application of M to P is the
  substitution of $P$ for $\Box$ in $M$.
\end{definition}

$\meaningof{-} : L \to \mathcal{P}(\pi)$

\begin{mathpar}
  \inferrule* [lab=collection] {} {\meaningof{true} = \pi, \and \meaningof{~E} = \pi \setminus \meaningof{E}, \and \meaningof{E_{1} \& E_{2}} = \meaningof{E_{1}} \cap \meaningof{E_{2}}}
\end{mathpar}

\begin{mathpar}
  \inferrule* [lab=structure] {} {\meaningof{0} = \{ P \in \pi | P \equiv 0 \}, \and \\ \meaningof{E_1 | E_2} = \{ P \in \pi | P \equiv P_{1} | P_{2}, P_{1} \in \meaningof{E_{1}}, P_{2} \in \meaningof{E_2}\} }
\end{mathpar}

\begin{mathpar}
 \inferrule* [lab=behavior] {} {\meaningof{\langle a?b \rangle E} = \{ P \in \pi | P \equiv Q | u?(y)P', \\ \and \\\\ \and \\ \;\;\; u \in \meaningof{a}, \forall z.P'\{z/y\} \in \meaningof{E\{z/b\}}\}, \and \\ \meaningof{a!E} = \{ P \in \pi | P \equiv Q | x!\langle P' \rangle, x \in \meaningof{a} P' \in \meaningof{E}\} }
\end{mathpar}

\begin{mathpar}
 \inferrule* [lab=nominal] {} {\meaningof{\quotep{E}} = \{ \quotep{P} \in \quotep{\pi} | P \in \meaningof{E} \}, \and \meaningof{\quotep{P}} = \{ \quotep{Q} \in \quotep{\pi} | P \equiv Q \} \and \\ \meaningof{@\quotep{E}} = \{ P \in \pi | P \equiv @x, x \in \meaningof{E} \}}
\end{mathpar}

\begin{eqnarray*}
  \\
  \meaningof{-} : TS \to ST
\end{eqnarray*}

\begin{eqnarray*}
  \\
  L : TS \to ST
\end{eqnarray*}

\begin{eqnarray*}
  \\
  P \models E \iff P \in \meaningof{E}
\end{eqnarray*}

\begin{eqnarray*}
  P \approx_{L} Q \iff \forall E \in L. P \models E \iff Q \models E
\end{eqnarray*}

\begin{eqnarray*}
  P \approx_{K} Q
\end{eqnarray*}

\begin{eqnarray*}
  P \approx Q
\end{eqnarray*}

$\approx_{K} = \approx = \approx_{L}$

\subsubsection{Contextual duality}

Note that contexts extend the quotation operation to a family of
operations from processes to names. Given a context, $M$, we can
define a \emph{nominal context}, $\quotep{M}$ by $\quotep{M}[P] :=
\quotep{M[P]}$. To foreshadow what is to come we observe that these
operations enjoy a duality with processes very much like the duality
between vectors and maps from vectors to scalars.

Further, because the calculus is essentially higher-order, we have a
correspondence between contexts and processes. More specifically,
given a name $x$ and a context $M$ we can construct $M^{*}_{x}$ such
that 

\begin{mathpar}
  M^{*}_{x} | \lift{x}{P} \red M[P]
\end{mathpar}

namely,

\begin{mathpar}
  M^{*}_{x} := x?(u).M[\dropn{u}]
\end{mathpar}

The dependence of $M^{*}_{x}$ on a name makes it an abstraction, 

\begin{mathpar}
  M^{*} := (x)x?(u).M[\dropn{u}]
\end{mathpar}

\subsection{Additional notation}

It will sometimes be convenient to denote the process a name
quotes. We already have the notation $x = \quotep{P}$, but it will be
convenient to introduce an alternate notation, $\procn{x}$, when we
want to emphasize the connection to the use of the name. Note that, by
virtue of name equivalence, $\quotep{\procn{x}} \nameeq x$; so, the
notation is consistent with previous definitions.

Further, because names have structure it is possible to effect
substitutions on the basis of that structure. This means we need to
upgrade our notation for substitutions, which we accomplish by
adapting comprehension notation. Thus,

\begin{mathpar}
  P\{ y / x : x \in S \}
\end{mathpar}

is interpreted to mean the process derived from P by replacing (in a
capture-avoiding manner) each occurrence of $x$ in $S$ by $y$. For example,

\begin{mathpar}
  P\{ \quotep{\procn{x}|\procn{x}} / x : x \in \freenames{P} \}
\end{mathpar}

will replace each (occurrence) of a free name $x$ in $P$ by
$\quotep{\procn{x}|\procn{x}}$.

Also, we will avail ourselves of the notation $x^{L}$ and $x^{R}$ to
denote injections of a name into disjoint copies of the name
space. There are numerous ways to accomplish this. One example can be
found in \cite{MeredithR05}. This notation overloads to vectors of
names: $\vec{x}^{\pi} := (x_{i}^{\pi} \; : \; 0 \leq i < |\vec{x}| )$ where $\pi \in \{L,R\}$.

We also use $P^{\Box} := P|\Box$.

In \cite{MeredithR05} an interpretation of the new operator is
given. It turns out that there are several possible interpretations
all enjoying the requisite algebraic properties of the operator (see
\cite{milner91polyadicpi}). We will therefore make liberal use of
$(\nu\; \vec{x})P$.

% subsection the_syntax_and_semantics_of_the_notation_system (end)   

\input{qm2pi.qmops} 

\input{qm2pi.sterngerlach} 

\input{qm2pi.metric} 

% section concurrent_process_calculi (end)

%\input{qm2pi.proofsketch}

% section proof sketch (end)

%\input{qm2pi.slviaknots} 

% section spatial logic via knots (end)

\input{qm2pi.conclusion}

% section conclusion (end)

%\input{qm2pi.dtcodes} 

% section wiring algorithm (end)

\input{qm2pi.ack} 

% section acknowledgments (end)

\newpage


\bibliographystyle{plain}   
\bibliography{../../biblios/main.bib}

\input{qm2pi.rhodetails}

\end{document}

 

%\documentclass[12pt]{llncs}
%\documentclass{jktr}

\usepackage[pdftex]{hyperref}                   
\usepackage {listings}
\usepackage {mathpartir}
\usepackage{bcprules}
%\usepackage{listings}
                       
\usepackage{graphicx} 
%\usepackage[margins=2.5cm,nohead,nofoot]{geometry}
%\usepackage{geometry}
\usepackage{amsfonts}
\usepackage{amstext}
\usepackage{latexsym}
\usepackage{amssymb}
\usepackage{color}


%\include{myPreamble}
\include{qm2pi.local} 

%\ifpdf
%\usepackage[pdftex]{graphicx}
%\else
%\usepackage{graphicx}
%\fi

 % \ifpdf
%  \usepackage{pdfsync}
%  \if


%\title{Brief Article}
%\author{David F. Snyder}
%\author{L.G. Meredith}

%\address{Dept. of Math., Texas State University--San Marcos, San Marcos, TX 78666}
       
\pagestyle{empty}


\begin{document}

\lstset{language=[Objective]Caml,frame=shadowbox}

\input{qm2pi.front}

% section front matter (end)

\input{qm2pi.intro} 
 
% section introduction (end)

% \input{qm2pi.knotations} 

% section notation (end)

\input{qm2pi.process.calculi} 

% section concurrent_process_calculi_and_spatial_logics_ (end)
    
%\input{qm2pi.knots2pi} 

%\input{qm2pi.trefoil} 

%\input{qm2pi.mainthm} 

% subsection basic_interpretation (end)

%\input{qm2pi.rho.presentation} 
\subsection{The syntax and semantics of the notation system}\label{sub:the_syntax_and_semantics_of_the_notation_system} % (fold)

We now summarize a technical presentation of the calculus that
embodies our theory of dynamics. The typical presentation of such a
calculus follows the style of giving generators and relations on
them. The grammar, below, describing term constructors, freely
generates the set of processes, $\Proc$. This set is then quotiented
by a relation known as structural congruence and it is over this set
that the notion of dynamics is expressed. This presentation is
essentially that of \cite{MeredithR05} with the addition of
polyadicity and summation. For readability we have relegated some of
the technical subtleties to an appendix.

\subsubsection{Process grammar}\label{subsub:process_grammar}

\begin{mathpar}
  \inferrule* [lab=synchronization] {} {{M} \bc \pzero \;|\; x?F \;|\; x!C }
  \and
  \inferrule* [lab=abstraction] {} {{F} \bc (x)P}
  \and
  \inferrule* [lab=concretion] {} {{C} \bc \langle Q \rangle}
  \and
  \inferrule* [lab=process] {} {{P,Q} \bc M \;| \;P|Q \;|\; @{x}}
  \and
  \inferrule* [lab=name] {} {{x} \bc \quotep{P}}
\end{mathpar} 

Note that $\vec{x}$ (resp. $\vec{P}$) denotes a vector of names
(resp. processes) of length $|\vec{x}|$ (resp. $|\vec{P}|$). We adopt
the following useful abbreviations.

\begin{mathpar}
   x?(\vec{y}).P := x.(\vec{y})P \and  x\clift{\vec{P}} := x.\clift{\vec{P}}
   \and x!(y) := \lift{x}{\dropn{y}}
   \and \Pi_{i=0}^{n-1}P_i := P_0 | \ldots | P_{n-1}
\end{mathpar}

\subsubsection{Structural congruence}

\paragraph{Free and bound names and alpha-equivalence.} At the
core of structural equivalence is alpha-equivalence which identifies
process that are the same up to a change of variable. Formally, we
recognize the distinction between free and bound names. The free names
of a process, $\freenames{P}$, may be calculated recursively as
follows:

\begin{mathpar}
\freenames{\pzero} := \emptyset
  \and \\
  \freenames{x?(y).P} := \{ x \} \cup (\freenames{P} \setminus \{ y \})
  \and 
  \freenames{x!\langle P \rangle} := \{ x \} \cup \{ P \} 
  \and \\
  \freenames{P|Q} := \freenames{P} \cup \freenames{Q}
  \and \\
  \freenames{@{x}} := \{ x \}
\end{mathpar}

$\pi$
$\quotep{\pi}$

$\freenames{-} : \pi \to \mathcal{P}(\quotep{\pi})$

\begin{eqnarray*}
  \freenames{\pzero} & := & \emptyset \\
  \freenames{x?(y).P} & := & \{ x \} \cup (\freenames{P} \setminus \{ y \}) \\
  \freenames{x!\langle P \rangle} & := & \{ x \} \cup \{ P \} \\
  \freenames{P|Q} & := & \freenames{P} \cup \freenames{Q} \\
  \freenames{\dropn{x}} & := & \{ x \}
\end{eqnarray*}

The bound names of a process, $\boundnames{P}$, are those names occurring in $P$
that are not free. For example, in $x?(y).0$, the name $x$ is free, while $y$ is bound.

\begin{mathpar}
  \inferrule* [lab=monoidal-laws] {} { P|Q \equiv Q|P \and P|0 \equiv P \and P|(Q|R) \equiv (P|Q)|R }
\end{mathpar}

\begin{mathpar}
  \inferrule* [lab=alpha-equivalence] {} { (x)P \equiv (y)P\{y/x\} \and y \not\in \freenames{P} }
\end{mathpar}

\begin{definition}
Then two processes, $P,Q$, are alpha-equivalent if $P = Q\{\vec{y}/\vec{x}\}$ for
some $\vec{x} \in \boundnames{Q},\vec{y} \in \boundnames{P}$, where $Q\{\vec{y}/\vec{x}\}$
denotes the capture-avoiding substitution of $\vec{y}$ for $\vec{x}$ in $Q$.
\end{definition}

\begin{definition}
  The {\em structural congruence} \cite{SangiorgiWalker} , $\equiv$,
  between processes is the least congruence containing
  alpha-equivalence, satisfying the abelian monoid laws
  (associativity, commutativity and $\pzero$ as identity) for parallel
  composition $|$ and for summation $+$.
\end{definition}

\subsection{Name equivalence}

We take name equivalence, written $\nameeq$, to be the smallest
equivalence relation generated by the following rules.

\begin{mathpar}
\inferrule*[lab=Quote-drop]
{ }
{ \quotep{@{x}} \nameeq x }

\inferrule*[lab=Struct-equiv]
{ P \scong Q }
{ \quotep{P} \nameeq \quotep{Q} }
\end{mathpar}

The astute reader will have noticed that the mutual recursion of names
and processes imposes a mutual recursion on alpha-equivalence and
structural equivalence via name-equivalence. Fortunately, all of this
works out pleasantly and we may calculate in the natural way, free of
concern. The reader interested in the details is referred to the
appendix \ref{appendix:rho_details}.

\subsection{Substitution}

We use $\Proc$ for the set of processes, $\QProc$ for the set of
names, and $\id{\{}\vec{y} / \vec{x} \id{\}}$ to denote partial maps,
$s : \QProc \rightarrow \QProc$. A map, $s$ lifts, uniquely, to a map
on process terms, $\widehat{s} : \Proc \rightarrow \Proc$ by the
following equations.

\begin{mathpar}
  (0) \psubstp{Q}{P} := 0 \\
  (R \juxtap S) \psubstp{Q}{P}
  :=    
  (R)\psubstp{Q}{P} \juxtap (S) \psubstp{Q}{P} \\
  (x?(y).R) \psubstp{Q}{P}    
  :=    
  (x)\substp{Q}{P} (z)\concat( (R \psubstn{z}{y}) \psubstp{Q}{P} ) \\
  (\lift{x}{R}) \psubstp{Q}{P}  
  :=
  \lift{(x)\substp{Q}{P}}{ R \psubstp{Q}{P} } \\
%   (\dropn{x})  \psubstp{Q}{P}       
%   := 
%   \left\{ 
%     \begin{array}{ccc} 
%       \dropn{\quotep{Q}} & & x \nameeq \quotep{P} \\
%       \dropn{x} & & otherwise \\
%     \end{array}
%   \right. 
  (\dropn{x})  \psubstp{Q}{P}       
  := 
  \left\{ 
    \begin{array}{ccc} 
      Q & & x \nameeq \quotep{P} \\
      \dropn{x} & & otherwise \\
    \end{array}
  \right.
\end{mathpar}
 

where

\begin{eqnarray}
  (x)\id{\{} \lpquote Q \rpquote / \lpquote P \rpquote \id{\}}            = 
  \left\{ 
    \begin{array}{ccc}
      \lpquote Q \rpquote & & x \nameeq \lpquote P \rpquote \\
      x & & otherwise \\
    \end{array}
  \right. \nonumber
\end{eqnarray}

and $z$ is chosen distinct from $\quotep{P}$, $\quotep{Q}$, the free
names in $Q$, and all the names in $R$. Our $\alpha$-equivalence will
be built in the standard way from this substitution.

\begin{remark}\label{rem:no_self_referential_names}
  One consequence of these definitions is that $\forall P. \quotep{P}
  \not\in \freenames{P}$.
\end{remark}

\subsection{ Dynamic quote: an example }

Anticipating something of what's to come, consider applying the
substitution, $\widehat{\id{\{}u / z \id{\}}}$, to the following pair
of processes, $\lift{w}{y!(z)}$ and $w[ \lpquote y!(z) \rpquote ]$.

\begin{eqnarray}
	\lift{w}{y!(z)}\widehat{\id{\{}u / z \id{\}}}
		& = &
		\lift{w}{y!(u)} \nonumber\\
	w[ \lpquote y!(z) \rpquote ] \widehat{ \id{\{}u / z \id{\}} }
		& = &
		w[ \lpquote y!(z) \rpquote ] \nonumber
\end{eqnarray}

Because the body of the process between quotes is impervious to
substitution, we get radically different answers. In fact, by
examining the first process in an input context,
e.g. $x?(z).\lift{w}{y!(z)}$, we see that the process under the lift
operator may be shaped by prefixed inputs binding a name inside it. In
this sense, the lift operator will be seen as a way to dynamically
construct processes before reifying them as names.

Finally equipped with these standard features we can present the
dynamics of the calculus.

\subsubsection{Operational semantics} 

Finally, we introduce the computational dynamics. What marks these
algebras as distinct from other more traditionally studied algebraic
structures, e.g. vector spaces or polynomial rings, is the manner in
which dynamics is captured. In traditional structures, dynamics is typically
expressed through morphisms between such structures, as in linear maps
between vector spaces or morphisms between rings. In algebras
associated with the semantics of computation, the dynamics is
expressed as part of the algebraic structure itself, through a
reduction reduction relation typically denoted by $\red$. Below, we
give a recursive presentation of this relation for the calculus used
in the encoding.

$\red \subseteq \pi \times \pi$
$\red : \pi \to \mathcal{P}(\pi)$

\begin{mathpar}
  \inferrule* [lab=Comm] { \textsf{match}( x_{src}, x_{trgt} ) } { x_{trgt}?(y)P \; | \; x_{src}!\langle {Q} \rangle \red P\{\quotep{Q}/y}\} }
  \and \\
  \inferrule* [lab=Par] {{P} \red {P}'} {{{P} | {Q}} \red {{P}' | {Q}}}
  \and
  \inferrule* [lab=Equiv]{{{P} \scong {P}'} \andalso {{P}' \red {Q}'} \andalso {{Q}' \scong {Q}}}{{P} \red {Q}}
\end{mathpar}

\begin{eqnarray*}
  match_{\equiv} (\quotep{P},\quotep{Q}) & := & P \equiv Q \\
  match_{\dagger}(\quotep{P},\quotep{Q}) & := & \forall R. P|Q \red^{*} R => R \red^{*} 0 \\
  match_{K}(\quotep{P},\quotep{Q}) & := & K \mbox{ for some context } K
\end{eqnarray*}

$u?(x)P | u!\langle Q \rangle \red P\{\quotep{Q}/x\}$

%We write $\wred$ for $\red^*$, and $P\red$ if $\exists Q $ such that $ P \red Q$.
We write $P\red$ if $\exists Q $ such that $ P \red Q$ and $P\not\red$, otherwise.

\section{Replication}

As mentioned before, it is known that replication (and hence
recursion) can be implemented in a higher-order process algebra
\cite{SangiorgiWalker}. As our first example of calculation with the
machinery thus far presented we give the construction explicitly in
the {\rhoc}.

\begin{eqnarray}
	D_{x} & := & \prefix{x}{y}{(\binpar{\outputp{x}{y}}{@{y}})} \nonumber\\
	\bangp_{x}{P} & := & \binpar{{x}!\langle{\binpar{D_{x}}{P}}\rangle}{D_{x}} \nonumber
\end{eqnarray}

\begin{eqnarray}
	\bangp_{x}{P} & & \nonumber\\
	=
	& {x}!\langle{(\prefix{x}{y}{(\outputp{x}{y} | @{y})) | P}}\rangle 
	      | \prefix{x}{y}{(\outputp{x}{y} | @{y})} & \nonumber\\
	\red
	& (\outputp{x}{y} | @{y})\substn{\quotep{(\prefix{x}{y}{(@{y} | \outputp{x}{y})) | P}}}{y} & \nonumber\\
	=
	& \outputp{x}{\quotep{(\prefix{x}{y}{(\outputp{x}{y} | @{y})) | P}}}
	  | {(\prefix{x}{y}{(\outputp{x}{y} | @{y})) | P}} & \nonumber\\
	\red
	& \ldots & \nonumber\\
	\red^*
	& P | P | \ldots & \nonumber
\end{eqnarray}

Of course, this encoding, as an implementation, runs away, unfolding
$\bangp{P}$ eagerly. A lazier and more implementable replication
operator, restricted to input-guarded processes, may be obtained as follows.

\begin{eqnarray}
\bangp{\prefix{u}{v}{P}} 
	:= 
	\binpar{\lift{x}{\prefix{u}{v}{(\binpar{D(x)}{P})}}}{D(x)} \nonumber
\end{eqnarray}

\begin{remark}
  Note that the lazier definition still does not deal with summation
  or mixed summation (i.e. sums over input and output). The reader is
  invited to construct definitions of replication that deal with these
  features. 

  Further, the definitions are parameterized in a name, $x$. Can you,
  gentle reader, make a definition that eliminates this parameter and
  guarantees no accidental interaction between the replication
  machinery and the process being replicated -- i.e. no accidental
  sharing of names used by the process to get its work done and the
  name(s) used by the replication to effect copying. This latter
  revision of the definition of replication is crucial to obtaining
  the expected identity $!!P \sim !P$.
\end{remark}

\begin{remark}\label{rem:paradoxical_combinator}
  The reader familiar with the lambda calculus will have noticed the
  similarity between $D$ and the paradoxical combinator.

  [Ed. note: the existence of this seems to suggest we have to be more
  restrictive on the set of processes and names we admit if we are to
  support no-cloning.]
\end{remark}

\subsubsection{Bisimulation}

The computational dynamics gives rise to another kind of equivalence,
the equivalence of computational behavior. As previously mentioned
this is typically captured \emph{via} some form of bisimulation.

% The notion we use in this paper is weak barbed bisimulation
% \cite{milner91polyadicpi}.

The notion we use in this paper is derived from weak barbed
bisimulation \cite{milner91polyadicpi}. 

\begin{definition}
An \emph{observation relation}, $\downarrow_{\mathcal N}$, over a set
of names, $\mathcal N$, is the smallest relation satisfying the rules
below.

\infrule[Out-barb]{y \in {\mathcal N}, \; x \nameeq y}
		  {\outputp{x}{v} \downarrow_{\mathcal N} x}
\infrule[Par-barb]{\mbox{$P\downarrow_{\mathcal N} x$ or $Q\downarrow_{\mathcal N} x$}}
		  {\binpar{P}{Q} \downarrow_{\mathcal N} x}

We write $P \Downarrow_{\mathcal N} x$ if there is $Q$ such that 
$P \wred Q$ and $Q \downarrow_{\mathcal N} x$.
\end{definition}

\begin{definition}
%\label{def.bbisim}
An  ${\mathcal N}$-\emph{barbed bisimulation} over a set of names, ${\mathcal N}$, is a symmetric binary relation 
${\mathcal S}_{\mathcal N}$ between agents such that $P\rel{S}_{\mathcal N}Q$ implies:
\begin{enumerate}
\item If $P \red P'$ then $Q \wred Q'$ and $P'\rel{S}_{\mathcal N} Q'$.
\item If $P\downarrow_{\mathcal N} x$, then $Q\Downarrow_{\mathcal N} x$.
\end{enumerate}
$P$ is ${\mathcal N}$-barbed bisimilar to $Q$, written
$P \wbbisim_{\mathcal N} Q$, if $P \rel{S}_{\mathcal N} Q$ for some ${\mathcal N}$-barbed bisimulation ${\mathcal S}_{\mathcal N}$.
\end{definition}

$\mathcal{R} \subseteq \pi \times \pi$

$P \mathcal{R} Q => \forall P'. P \red P' \Rightarrow \exists Q'. Q \red Q', P' \mathcal{R} Q'$

$P \vdash x \Rightarrow Q \vdash x$

\begin{mathpar}
  \inferrule*[lab=Out-barb]{x \nameeq y}{{y}!\langle{Q}\rangle \vdash x}
  \and
  \inferrule*[lab=Par-barb]{\mbox{$P\vdash x$ or $Q\vdash x$}}{\binpar{P}{Q} \vdash x}
\end{mathpar}

\subsubsection{Contexts}

One of the principle advantages of computational calculi like the
$\pi$-calculus is a well-defined notion of context,
contextual-equivalence and a correlation between
contextual-equivalence and notions of bisimulation. The notion of
context allows the decomposition of a process into (sub-)process and
its syntactic environment, its context. Thus, a context may be
thought of as a process with a ``hole'' (written $\Box$) in it. The
application of a context $M$ to a process $P$, written $M[P]$, is
tantamount to filling the hole in $M$ with $P$. In this paper we do
not need the full weight of this theory, but do make use of the notion
of context in the proof the main theorem. 

\begin{mathpar}
  \inferrule* [lab=summation] {} {{M_{M},M_{N}} \bc \Box \;|\; x.M_{A} \;|\; M_{M}+M_{N}}
  \and
  \inferrule* [lab=agent] {} {{M_{A}} \bc (\vec{x})M_{P} \;| \; \clift{P_0,\ldots,M_{P},\ldots,P_N}}
  \and \\
  \inferrule* [lab=process] {} {{M_{P}} \bc M_{N} \;| \;P|M_{P} }
\end{mathpar} 

\begin{mathpar}
  \inferrule* [lab=sychronization] {} {M_{N} \bc \Box \;|\; x?M_{F} \;|\; x!M_{C}}
  \and
  \inferrule* [lab=abstraction] {} {{M_{F}} \bc (x)M_{P} }
  \and
  \inferrule* [lab=concretion] {} {{M_{C}} \bc \langle M_{P} \rangle }
  \and \\
  \inferrule* [lab=process] {} {{M_{P}} \bc M_{N} \;| \;P|M_{P} }
\end{mathpar}

\begin{definition}[contextual application] Given a context $M$, and
  process $P$, we define the \emph{contextual application}, $M[P] :=
  M\{P/\Box\}$. That is, the contextual application of M to P is the
  substitution of $P$ for $\Box$ in $M$.
\end{definition}

$\meaningof{-} : L \to \mathcal{P}(\pi)$

\begin{mathpar}
  \inferrule* [lab=collection] {} {\meaningof{true} = \pi, \and \meaningof{~E} = \pi \setminus \meaningof{E}, \and \meaningof{E_{1} \& E_{2}} = \meaningof{E_{1}} \cap \meaningof{E_{2}}}
\end{mathpar}

\begin{mathpar}
  \inferrule* [lab=structure] {} {\meaningof{0} = \{ P \in \pi | P \equiv 0 \}, \and \\ \meaningof{E_1 | E_2} = \{ P \in \pi | P \equiv P_{1} | P_{2}, P_{1} \in \meaningof{E_{1}}, P_{2} \in \meaningof{E_2}\} }
\end{mathpar}

\begin{mathpar}
 \inferrule* [lab=behavior] {} {\meaningof{\langle a?b \rangle E} = \{ P \in \pi | P \equiv Q | u?(y)P', \\ \and \\\\ \and \\ \;\;\; u \in \meaningof{a}, \forall z.P'\{z/y\} \in \meaningof{E\{z/b\}}\}, \and \\ \meaningof{a!E} = \{ P \in \pi | P \equiv Q | x!\langle P' \rangle, x \in \meaningof{a} P' \in \meaningof{E}\} }
\end{mathpar}

\begin{mathpar}
 \inferrule* [lab=nominal] {} {\meaningof{\quotep{E}} = \{ \quotep{P} \in \quotep{\pi} | P \in \meaningof{E} \}, \and \meaningof{\quotep{P}} = \{ \quotep{Q} \in \quotep{\pi} | P \equiv Q \} \and \\ \meaningof{@\quotep{E}} = \{ P \in \pi | P \equiv @x, x \in \meaningof{E} \}}
\end{mathpar}

\begin{eqnarray*}
  \\
  \meaningof{-} : TS \to ST
\end{eqnarray*}

\begin{eqnarray*}
  \\
  L : TS \to ST
\end{eqnarray*}

\begin{eqnarray*}
  \\
  P \models E \iff P \in \meaningof{E}
\end{eqnarray*}

\begin{eqnarray*}
  P \approx_{L} Q \iff \forall E \in L. P \models E \iff Q \models E
\end{eqnarray*}

\begin{eqnarray*}
  P \approx_{K} Q
\end{eqnarray*}

\begin{eqnarray*}
  P \approx Q
\end{eqnarray*}

$\approx_{K} = \approx = \approx_{L}$

\subsubsection{Contextual duality}

Note that contexts extend the quotation operation to a family of
operations from processes to names. Given a context, $M$, we can
define a \emph{nominal context}, $\quotep{M}$ by $\quotep{M}[P] :=
\quotep{M[P]}$. To foreshadow what is to come we observe that these
operations enjoy a duality with processes very much like the duality
between vectors and maps from vectors to scalars.

Further, because the calculus is essentially higher-order, we have a
correspondence between contexts and processes. More specifically,
given a name $x$ and a context $M$ we can construct $M^{*}_{x}$ such
that 

\begin{mathpar}
  M^{*}_{x} | \lift{x}{P} \red M[P]
\end{mathpar}

namely,

\begin{mathpar}
  M^{*}_{x} := x?(u).M[\dropn{u}]
\end{mathpar}

The dependence of $M^{*}_{x}$ on a name makes it an abstraction, 

\begin{mathpar}
  M^{*} := (x)x?(u).M[\dropn{u}]
\end{mathpar}

\subsection{Additional notation}

It will sometimes be convenient to denote the process a name
quotes. We already have the notation $x = \quotep{P}$, but it will be
convenient to introduce an alternate notation, $\procn{x}$, when we
want to emphasize the connection to the use of the name. Note that, by
virtue of name equivalence, $\quotep{\procn{x}} \nameeq x$; so, the
notation is consistent with previous definitions.

Further, because names have structure it is possible to effect
substitutions on the basis of that structure. This means we need to
upgrade our notation for substitutions, which we accomplish by
adapting comprehension notation. Thus,

\begin{mathpar}
  P\{ y / x : x \in S \}
\end{mathpar}

is interpreted to mean the process derived from P by replacing (in a
capture-avoiding manner) each occurrence of $x$ in $S$ by $y$. For example,

\begin{mathpar}
  P\{ \quotep{\procn{x}|\procn{x}} / x : x \in \freenames{P} \}
\end{mathpar}

will replace each (occurrence) of a free name $x$ in $P$ by
$\quotep{\procn{x}|\procn{x}}$.

Also, we will avail ourselves of the notation $x^{L}$ and $x^{R}$ to
denote injections of a name into disjoint copies of the name
space. There are numerous ways to accomplish this. One example can be
found in \cite{MeredithR05}. This notation overloads to vectors of
names: $\vec{x}^{\pi} := (x_{i}^{\pi} \; : \; 0 \leq i < |\vec{x}| )$ where $\pi \in \{L,R\}$.

We also use $P^{\Box} := P|\Box$.

In \cite{MeredithR05} an interpretation of the new operator is
given. It turns out that there are several possible interpretations
all enjoying the requisite algebraic properties of the operator (see
\cite{milner91polyadicpi}). We will therefore make liberal use of
$(\nu\; \vec{x})P$.

% subsection the_syntax_and_semantics_of_the_notation_system (end)   

\input{qm2pi.qmops} 

\input{qm2pi.sterngerlach} 

\input{qm2pi.metric} 

% section concurrent_process_calculi (end)

%\input{qm2pi.proofsketch}

% section proof sketch (end)

%\input{qm2pi.slviaknots} 

% section spatial logic via knots (end)

\input{qm2pi.conclusion}

% section conclusion (end)

%\input{qm2pi.dtcodes} 

% section wiring algorithm (end)

\input{qm2pi.ack} 

% section acknowledgments (end)

\newpage


\bibliographystyle{plain}   
\bibliography{../../biblios/main.bib}

\input{qm2pi.rhodetails}

\end{document}

 

%\documentclass[12pt]{llncs}
%\documentclass{jktr}

\usepackage[pdftex]{hyperref}                   
\usepackage {listings}
\usepackage {mathpartir}
\usepackage{bcprules}
%\usepackage{listings}
                       
\usepackage{graphicx} 
%\usepackage[margins=2.5cm,nohead,nofoot]{geometry}
%\usepackage{geometry}
\usepackage{amsfonts}
\usepackage{amstext}
\usepackage{latexsym}
\usepackage{amssymb}
\usepackage{color}


%\include{myPreamble}
\include{qm2pi.local} 

%\ifpdf
%\usepackage[pdftex]{graphicx}
%\else
%\usepackage{graphicx}
%\fi

 % \ifpdf
%  \usepackage{pdfsync}
%  \if


%\title{Brief Article}
%\author{David F. Snyder}
%\author{L.G. Meredith}

%\address{Dept. of Math., Texas State University--San Marcos, San Marcos, TX 78666}
       
\pagestyle{empty}


\begin{document}

\lstset{language=[Objective]Caml,frame=shadowbox}

\input{qm2pi.front}

% section front matter (end)

\input{qm2pi.intro} 
 
% section introduction (end)

% \input{qm2pi.knotations} 

% section notation (end)

\input{qm2pi.process.calculi} 

% section concurrent_process_calculi_and_spatial_logics_ (end)
    
%\input{qm2pi.knots2pi} 

%\input{qm2pi.trefoil} 

%\input{qm2pi.mainthm} 

% subsection basic_interpretation (end)

%\input{qm2pi.rho.presentation} 
\subsection{The syntax and semantics of the notation system}\label{sub:the_syntax_and_semantics_of_the_notation_system} % (fold)

We now summarize a technical presentation of the calculus that
embodies our theory of dynamics. The typical presentation of such a
calculus follows the style of giving generators and relations on
them. The grammar, below, describing term constructors, freely
generates the set of processes, $\Proc$. This set is then quotiented
by a relation known as structural congruence and it is over this set
that the notion of dynamics is expressed. This presentation is
essentially that of \cite{MeredithR05} with the addition of
polyadicity and summation. For readability we have relegated some of
the technical subtleties to an appendix.

\subsubsection{Process grammar}\label{subsub:process_grammar}

\begin{mathpar}
  \inferrule* [lab=synchronization] {} {{M} \bc \pzero \;|\; x?F \;|\; x!C }
  \and
  \inferrule* [lab=abstraction] {} {{F} \bc (x)P}
  \and
  \inferrule* [lab=concretion] {} {{C} \bc \langle Q \rangle}
  \and
  \inferrule* [lab=process] {} {{P,Q} \bc M \;| \;P|Q \;|\; @{x}}
  \and
  \inferrule* [lab=name] {} {{x} \bc \quotep{P}}
\end{mathpar} 

Note that $\vec{x}$ (resp. $\vec{P}$) denotes a vector of names
(resp. processes) of length $|\vec{x}|$ (resp. $|\vec{P}|$). We adopt
the following useful abbreviations.

\begin{mathpar}
   x?(\vec{y}).P := x.(\vec{y})P \and  x\clift{\vec{P}} := x.\clift{\vec{P}}
   \and x!(y) := \lift{x}{\dropn{y}}
   \and \Pi_{i=0}^{n-1}P_i := P_0 | \ldots | P_{n-1}
\end{mathpar}

\subsubsection{Structural congruence}

\paragraph{Free and bound names and alpha-equivalence.} At the
core of structural equivalence is alpha-equivalence which identifies
process that are the same up to a change of variable. Formally, we
recognize the distinction between free and bound names. The free names
of a process, $\freenames{P}$, may be calculated recursively as
follows:

\begin{mathpar}
\freenames{\pzero} := \emptyset
  \and \\
  \freenames{x?(y).P} := \{ x \} \cup (\freenames{P} \setminus \{ y \})
  \and 
  \freenames{x!\langle P \rangle} := \{ x \} \cup \{ P \} 
  \and \\
  \freenames{P|Q} := \freenames{P} \cup \freenames{Q}
  \and \\
  \freenames{@{x}} := \{ x \}
\end{mathpar}

$\pi$
$\quotep{\pi}$

$\freenames{-} : \pi \to \mathcal{P}(\quotep{\pi})$

\begin{eqnarray*}
  \freenames{\pzero} & := & \emptyset \\
  \freenames{x?(y).P} & := & \{ x \} \cup (\freenames{P} \setminus \{ y \}) \\
  \freenames{x!\langle P \rangle} & := & \{ x \} \cup \{ P \} \\
  \freenames{P|Q} & := & \freenames{P} \cup \freenames{Q} \\
  \freenames{\dropn{x}} & := & \{ x \}
\end{eqnarray*}

The bound names of a process, $\boundnames{P}$, are those names occurring in $P$
that are not free. For example, in $x?(y).0$, the name $x$ is free, while $y$ is bound.

\begin{mathpar}
  \inferrule* [lab=monoidal-laws] {} { P|Q \equiv Q|P \and P|0 \equiv P \and P|(Q|R) \equiv (P|Q)|R }
\end{mathpar}

\begin{mathpar}
  \inferrule* [lab=alpha-equivalence] {} { (x)P \equiv (y)P\{y/x\} \and y \not\in \freenames{P} }
\end{mathpar}

\begin{definition}
Then two processes, $P,Q$, are alpha-equivalent if $P = Q\{\vec{y}/\vec{x}\}$ for
some $\vec{x} \in \boundnames{Q},\vec{y} \in \boundnames{P}$, where $Q\{\vec{y}/\vec{x}\}$
denotes the capture-avoiding substitution of $\vec{y}$ for $\vec{x}$ in $Q$.
\end{definition}

\begin{definition}
  The {\em structural congruence} \cite{SangiorgiWalker} , $\equiv$,
  between processes is the least congruence containing
  alpha-equivalence, satisfying the abelian monoid laws
  (associativity, commutativity and $\pzero$ as identity) for parallel
  composition $|$ and for summation $+$.
\end{definition}

\subsection{Name equivalence}

We take name equivalence, written $\nameeq$, to be the smallest
equivalence relation generated by the following rules.

\begin{mathpar}
\inferrule*[lab=Quote-drop]
{ }
{ \quotep{@{x}} \nameeq x }

\inferrule*[lab=Struct-equiv]
{ P \scong Q }
{ \quotep{P} \nameeq \quotep{Q} }
\end{mathpar}

The astute reader will have noticed that the mutual recursion of names
and processes imposes a mutual recursion on alpha-equivalence and
structural equivalence via name-equivalence. Fortunately, all of this
works out pleasantly and we may calculate in the natural way, free of
concern. The reader interested in the details is referred to the
appendix \ref{appendix:rho_details}.

\subsection{Substitution}

We use $\Proc$ for the set of processes, $\QProc$ for the set of
names, and $\id{\{}\vec{y} / \vec{x} \id{\}}$ to denote partial maps,
$s : \QProc \rightarrow \QProc$. A map, $s$ lifts, uniquely, to a map
on process terms, $\widehat{s} : \Proc \rightarrow \Proc$ by the
following equations.

\begin{mathpar}
  (0) \psubstp{Q}{P} := 0 \\
  (R \juxtap S) \psubstp{Q}{P}
  :=    
  (R)\psubstp{Q}{P} \juxtap (S) \psubstp{Q}{P} \\
  (x?(y).R) \psubstp{Q}{P}    
  :=    
  (x)\substp{Q}{P} (z)\concat( (R \psubstn{z}{y}) \psubstp{Q}{P} ) \\
  (\lift{x}{R}) \psubstp{Q}{P}  
  :=
  \lift{(x)\substp{Q}{P}}{ R \psubstp{Q}{P} } \\
%   (\dropn{x})  \psubstp{Q}{P}       
%   := 
%   \left\{ 
%     \begin{array}{ccc} 
%       \dropn{\quotep{Q}} & & x \nameeq \quotep{P} \\
%       \dropn{x} & & otherwise \\
%     \end{array}
%   \right. 
  (\dropn{x})  \psubstp{Q}{P}       
  := 
  \left\{ 
    \begin{array}{ccc} 
      Q & & x \nameeq \quotep{P} \\
      \dropn{x} & & otherwise \\
    \end{array}
  \right.
\end{mathpar}
 

where

\begin{eqnarray}
  (x)\id{\{} \lpquote Q \rpquote / \lpquote P \rpquote \id{\}}            = 
  \left\{ 
    \begin{array}{ccc}
      \lpquote Q \rpquote & & x \nameeq \lpquote P \rpquote \\
      x & & otherwise \\
    \end{array}
  \right. \nonumber
\end{eqnarray}

and $z$ is chosen distinct from $\quotep{P}$, $\quotep{Q}$, the free
names in $Q$, and all the names in $R$. Our $\alpha$-equivalence will
be built in the standard way from this substitution.

\begin{remark}\label{rem:no_self_referential_names}
  One consequence of these definitions is that $\forall P. \quotep{P}
  \not\in \freenames{P}$.
\end{remark}

\subsection{ Dynamic quote: an example }

Anticipating something of what's to come, consider applying the
substitution, $\widehat{\id{\{}u / z \id{\}}}$, to the following pair
of processes, $\lift{w}{y!(z)}$ and $w[ \lpquote y!(z) \rpquote ]$.

\begin{eqnarray}
	\lift{w}{y!(z)}\widehat{\id{\{}u / z \id{\}}}
		& = &
		\lift{w}{y!(u)} \nonumber\\
	w[ \lpquote y!(z) \rpquote ] \widehat{ \id{\{}u / z \id{\}} }
		& = &
		w[ \lpquote y!(z) \rpquote ] \nonumber
\end{eqnarray}

Because the body of the process between quotes is impervious to
substitution, we get radically different answers. In fact, by
examining the first process in an input context,
e.g. $x?(z).\lift{w}{y!(z)}$, we see that the process under the lift
operator may be shaped by prefixed inputs binding a name inside it. In
this sense, the lift operator will be seen as a way to dynamically
construct processes before reifying them as names.

Finally equipped with these standard features we can present the
dynamics of the calculus.

\subsubsection{Operational semantics} 

Finally, we introduce the computational dynamics. What marks these
algebras as distinct from other more traditionally studied algebraic
structures, e.g. vector spaces or polynomial rings, is the manner in
which dynamics is captured. In traditional structures, dynamics is typically
expressed through morphisms between such structures, as in linear maps
between vector spaces or morphisms between rings. In algebras
associated with the semantics of computation, the dynamics is
expressed as part of the algebraic structure itself, through a
reduction reduction relation typically denoted by $\red$. Below, we
give a recursive presentation of this relation for the calculus used
in the encoding.

$\red \subseteq \pi \times \pi$
$\red : \pi \to \mathcal{P}(\pi)$

\begin{mathpar}
  \inferrule* [lab=Comm] { \textsf{match}( x_{src}, x_{trgt} ) } { x_{trgt}?(y)P \; | \; x_{src}!\langle {Q} \rangle \red P\{\quotep{Q}/y}\} }
  \and \\
  \inferrule* [lab=Par] {{P} \red {P}'} {{{P} | {Q}} \red {{P}' | {Q}}}
  \and
  \inferrule* [lab=Equiv]{{{P} \scong {P}'} \andalso {{P}' \red {Q}'} \andalso {{Q}' \scong {Q}}}{{P} \red {Q}}
\end{mathpar}

\begin{eqnarray*}
  match_{\equiv} (\quotep{P},\quotep{Q}) & := & P \equiv Q \\
  match_{\dagger}(\quotep{P},\quotep{Q}) & := & \forall R. P|Q \red^{*} R => R \red^{*} 0 \\
  match_{K}(\quotep{P},\quotep{Q}) & := & K \mbox{ for some context } K
\end{eqnarray*}

$u?(x)P | u!\langle Q \rangle \red P\{\quotep{Q}/x\}$

%We write $\wred$ for $\red^*$, and $P\red$ if $\exists Q $ such that $ P \red Q$.
We write $P\red$ if $\exists Q $ such that $ P \red Q$ and $P\not\red$, otherwise.

\section{Replication}

As mentioned before, it is known that replication (and hence
recursion) can be implemented in a higher-order process algebra
\cite{SangiorgiWalker}. As our first example of calculation with the
machinery thus far presented we give the construction explicitly in
the {\rhoc}.

\begin{eqnarray}
	D_{x} & := & \prefix{x}{y}{(\binpar{\outputp{x}{y}}{@{y}})} \nonumber\\
	\bangp_{x}{P} & := & \binpar{{x}!\langle{\binpar{D_{x}}{P}}\rangle}{D_{x}} \nonumber
\end{eqnarray}

\begin{eqnarray}
	\bangp_{x}{P} & & \nonumber\\
	=
	& {x}!\langle{(\prefix{x}{y}{(\outputp{x}{y} | @{y})) | P}}\rangle 
	      | \prefix{x}{y}{(\outputp{x}{y} | @{y})} & \nonumber\\
	\red
	& (\outputp{x}{y} | @{y})\substn{\quotep{(\prefix{x}{y}{(@{y} | \outputp{x}{y})) | P}}}{y} & \nonumber\\
	=
	& \outputp{x}{\quotep{(\prefix{x}{y}{(\outputp{x}{y} | @{y})) | P}}}
	  | {(\prefix{x}{y}{(\outputp{x}{y} | @{y})) | P}} & \nonumber\\
	\red
	& \ldots & \nonumber\\
	\red^*
	& P | P | \ldots & \nonumber
\end{eqnarray}

Of course, this encoding, as an implementation, runs away, unfolding
$\bangp{P}$ eagerly. A lazier and more implementable replication
operator, restricted to input-guarded processes, may be obtained as follows.

\begin{eqnarray}
\bangp{\prefix{u}{v}{P}} 
	:= 
	\binpar{\lift{x}{\prefix{u}{v}{(\binpar{D(x)}{P})}}}{D(x)} \nonumber
\end{eqnarray}

\begin{remark}
  Note that the lazier definition still does not deal with summation
  or mixed summation (i.e. sums over input and output). The reader is
  invited to construct definitions of replication that deal with these
  features. 

  Further, the definitions are parameterized in a name, $x$. Can you,
  gentle reader, make a definition that eliminates this parameter and
  guarantees no accidental interaction between the replication
  machinery and the process being replicated -- i.e. no accidental
  sharing of names used by the process to get its work done and the
  name(s) used by the replication to effect copying. This latter
  revision of the definition of replication is crucial to obtaining
  the expected identity $!!P \sim !P$.
\end{remark}

\begin{remark}\label{rem:paradoxical_combinator}
  The reader familiar with the lambda calculus will have noticed the
  similarity between $D$ and the paradoxical combinator.

  [Ed. note: the existence of this seems to suggest we have to be more
  restrictive on the set of processes and names we admit if we are to
  support no-cloning.]
\end{remark}

\subsubsection{Bisimulation}

The computational dynamics gives rise to another kind of equivalence,
the equivalence of computational behavior. As previously mentioned
this is typically captured \emph{via} some form of bisimulation.

% The notion we use in this paper is weak barbed bisimulation
% \cite{milner91polyadicpi}.

The notion we use in this paper is derived from weak barbed
bisimulation \cite{milner91polyadicpi}. 

\begin{definition}
An \emph{observation relation}, $\downarrow_{\mathcal N}$, over a set
of names, $\mathcal N$, is the smallest relation satisfying the rules
below.

\infrule[Out-barb]{y \in {\mathcal N}, \; x \nameeq y}
		  {\outputp{x}{v} \downarrow_{\mathcal N} x}
\infrule[Par-barb]{\mbox{$P\downarrow_{\mathcal N} x$ or $Q\downarrow_{\mathcal N} x$}}
		  {\binpar{P}{Q} \downarrow_{\mathcal N} x}

We write $P \Downarrow_{\mathcal N} x$ if there is $Q$ such that 
$P \wred Q$ and $Q \downarrow_{\mathcal N} x$.
\end{definition}

\begin{definition}
%\label{def.bbisim}
An  ${\mathcal N}$-\emph{barbed bisimulation} over a set of names, ${\mathcal N}$, is a symmetric binary relation 
${\mathcal S}_{\mathcal N}$ between agents such that $P\rel{S}_{\mathcal N}Q$ implies:
\begin{enumerate}
\item If $P \red P'$ then $Q \wred Q'$ and $P'\rel{S}_{\mathcal N} Q'$.
\item If $P\downarrow_{\mathcal N} x$, then $Q\Downarrow_{\mathcal N} x$.
\end{enumerate}
$P$ is ${\mathcal N}$-barbed bisimilar to $Q$, written
$P \wbbisim_{\mathcal N} Q$, if $P \rel{S}_{\mathcal N} Q$ for some ${\mathcal N}$-barbed bisimulation ${\mathcal S}_{\mathcal N}$.
\end{definition}

$\mathcal{R} \subseteq \pi \times \pi$

$P \mathcal{R} Q => \forall P'. P \red P' \Rightarrow \exists Q'. Q \red Q', P' \mathcal{R} Q'$

$P \vdash x \Rightarrow Q \vdash x$

\begin{mathpar}
  \inferrule*[lab=Out-barb]{x \nameeq y}{{y}!\langle{Q}\rangle \vdash x}
  \and
  \inferrule*[lab=Par-barb]{\mbox{$P\vdash x$ or $Q\vdash x$}}{\binpar{P}{Q} \vdash x}
\end{mathpar}

\subsubsection{Contexts}

One of the principle advantages of computational calculi like the
$\pi$-calculus is a well-defined notion of context,
contextual-equivalence and a correlation between
contextual-equivalence and notions of bisimulation. The notion of
context allows the decomposition of a process into (sub-)process and
its syntactic environment, its context. Thus, a context may be
thought of as a process with a ``hole'' (written $\Box$) in it. The
application of a context $M$ to a process $P$, written $M[P]$, is
tantamount to filling the hole in $M$ with $P$. In this paper we do
not need the full weight of this theory, but do make use of the notion
of context in the proof the main theorem. 

\begin{mathpar}
  \inferrule* [lab=summation] {} {{M_{M},M_{N}} \bc \Box \;|\; x.M_{A} \;|\; M_{M}+M_{N}}
  \and
  \inferrule* [lab=agent] {} {{M_{A}} \bc (\vec{x})M_{P} \;| \; \clift{P_0,\ldots,M_{P},\ldots,P_N}}
  \and \\
  \inferrule* [lab=process] {} {{M_{P}} \bc M_{N} \;| \;P|M_{P} }
\end{mathpar} 

\begin{mathpar}
  \inferrule* [lab=sychronization] {} {M_{N} \bc \Box \;|\; x?M_{F} \;|\; x!M_{C}}
  \and
  \inferrule* [lab=abstraction] {} {{M_{F}} \bc (x)M_{P} }
  \and
  \inferrule* [lab=concretion] {} {{M_{C}} \bc \langle M_{P} \rangle }
  \and \\
  \inferrule* [lab=process] {} {{M_{P}} \bc M_{N} \;| \;P|M_{P} }
\end{mathpar}

\begin{definition}[contextual application] Given a context $M$, and
  process $P$, we define the \emph{contextual application}, $M[P] :=
  M\{P/\Box\}$. That is, the contextual application of M to P is the
  substitution of $P$ for $\Box$ in $M$.
\end{definition}

$\meaningof{-} : L \to \mathcal{P}(\pi)$

\begin{mathpar}
  \inferrule* [lab=collection] {} {\meaningof{true} = \pi, \and \meaningof{~E} = \pi \setminus \meaningof{E}, \and \meaningof{E_{1} \& E_{2}} = \meaningof{E_{1}} \cap \meaningof{E_{2}}}
\end{mathpar}

\begin{mathpar}
  \inferrule* [lab=structure] {} {\meaningof{0} = \{ P \in \pi | P \equiv 0 \}, \and \\ \meaningof{E_1 | E_2} = \{ P \in \pi | P \equiv P_{1} | P_{2}, P_{1} \in \meaningof{E_{1}}, P_{2} \in \meaningof{E_2}\} }
\end{mathpar}

\begin{mathpar}
 \inferrule* [lab=behavior] {} {\meaningof{\langle a?b \rangle E} = \{ P \in \pi | P \equiv Q | u?(y)P', \\ \and \\\\ \and \\ \;\;\; u \in \meaningof{a}, \forall z.P'\{z/y\} \in \meaningof{E\{z/b\}}\}, \and \\ \meaningof{a!E} = \{ P \in \pi | P \equiv Q | x!\langle P' \rangle, x \in \meaningof{a} P' \in \meaningof{E}\} }
\end{mathpar}

\begin{mathpar}
 \inferrule* [lab=nominal] {} {\meaningof{\quotep{E}} = \{ \quotep{P} \in \quotep{\pi} | P \in \meaningof{E} \}, \and \meaningof{\quotep{P}} = \{ \quotep{Q} \in \quotep{\pi} | P \equiv Q \} \and \\ \meaningof{@\quotep{E}} = \{ P \in \pi | P \equiv @x, x \in \meaningof{E} \}}
\end{mathpar}

\begin{eqnarray*}
  \\
  \meaningof{-} : TS \to ST
\end{eqnarray*}

\begin{eqnarray*}
  \\
  L : TS \to ST
\end{eqnarray*}

\begin{eqnarray*}
  \\
  P \models E \iff P \in \meaningof{E}
\end{eqnarray*}

\begin{eqnarray*}
  P \approx_{L} Q \iff \forall E \in L. P \models E \iff Q \models E
\end{eqnarray*}

\begin{eqnarray*}
  P \approx_{K} Q
\end{eqnarray*}

\begin{eqnarray*}
  P \approx Q
\end{eqnarray*}

$\approx_{K} = \approx = \approx_{L}$

\subsubsection{Contextual duality}

Note that contexts extend the quotation operation to a family of
operations from processes to names. Given a context, $M$, we can
define a \emph{nominal context}, $\quotep{M}$ by $\quotep{M}[P] :=
\quotep{M[P]}$. To foreshadow what is to come we observe that these
operations enjoy a duality with processes very much like the duality
between vectors and maps from vectors to scalars.

Further, because the calculus is essentially higher-order, we have a
correspondence between contexts and processes. More specifically,
given a name $x$ and a context $M$ we can construct $M^{*}_{x}$ such
that 

\begin{mathpar}
  M^{*}_{x} | \lift{x}{P} \red M[P]
\end{mathpar}

namely,

\begin{mathpar}
  M^{*}_{x} := x?(u).M[\dropn{u}]
\end{mathpar}

The dependence of $M^{*}_{x}$ on a name makes it an abstraction, 

\begin{mathpar}
  M^{*} := (x)x?(u).M[\dropn{u}]
\end{mathpar}

\subsection{Additional notation}

It will sometimes be convenient to denote the process a name
quotes. We already have the notation $x = \quotep{P}$, but it will be
convenient to introduce an alternate notation, $\procn{x}$, when we
want to emphasize the connection to the use of the name. Note that, by
virtue of name equivalence, $\quotep{\procn{x}} \nameeq x$; so, the
notation is consistent with previous definitions.

Further, because names have structure it is possible to effect
substitutions on the basis of that structure. This means we need to
upgrade our notation for substitutions, which we accomplish by
adapting comprehension notation. Thus,

\begin{mathpar}
  P\{ y / x : x \in S \}
\end{mathpar}

is interpreted to mean the process derived from P by replacing (in a
capture-avoiding manner) each occurrence of $x$ in $S$ by $y$. For example,

\begin{mathpar}
  P\{ \quotep{\procn{x}|\procn{x}} / x : x \in \freenames{P} \}
\end{mathpar}

will replace each (occurrence) of a free name $x$ in $P$ by
$\quotep{\procn{x}|\procn{x}}$.

Also, we will avail ourselves of the notation $x^{L}$ and $x^{R}$ to
denote injections of a name into disjoint copies of the name
space. There are numerous ways to accomplish this. One example can be
found in \cite{MeredithR05}. This notation overloads to vectors of
names: $\vec{x}^{\pi} := (x_{i}^{\pi} \; : \; 0 \leq i < |\vec{x}| )$ where $\pi \in \{L,R\}$.

We also use $P^{\Box} := P|\Box$.

In \cite{MeredithR05} an interpretation of the new operator is
given. It turns out that there are several possible interpretations
all enjoying the requisite algebraic properties of the operator (see
\cite{milner91polyadicpi}). We will therefore make liberal use of
$(\nu\; \vec{x})P$.

% subsection the_syntax_and_semantics_of_the_notation_system (end)   

\input{qm2pi.qmops} 

\input{qm2pi.sterngerlach} 

\input{qm2pi.metric} 

% section concurrent_process_calculi (end)

%\input{qm2pi.proofsketch}

% section proof sketch (end)

%\input{qm2pi.slviaknots} 

% section spatial logic via knots (end)

\input{qm2pi.conclusion}

% section conclusion (end)

%\input{qm2pi.dtcodes} 

% section wiring algorithm (end)

\input{qm2pi.ack} 

% section acknowledgments (end)

\newpage


\bibliographystyle{plain}   
\bibliography{../../biblios/main.bib}

\input{qm2pi.rhodetails}

\end{document}

 

% subsection basic_interpretation (end)

%\input{qm2pi.rho.presentation} 
\subsection{The syntax and semantics of the notation system}\label{sub:the_syntax_and_semantics_of_the_notation_system} % (fold)

We now summarize a technical presentation of the calculus that
embodies our theory of dynamics. The typical presentation of such a
calculus follows the style of giving generators and relations on
them. The grammar, below, describing term constructors, freely
generates the set of processes, $\Proc$. This set is then quotiented
by a relation known as structural congruence and it is over this set
that the notion of dynamics is expressed. This presentation is
essentially that of \cite{MeredithR05} with the addition of
polyadicity and summation. For readability we have relegated some of
the technical subtleties to an appendix.

\subsubsection{Process grammar}\label{subsub:process_grammar}

\begin{mathpar}
  \inferrule* [lab=synchronization] {} {{M} \bc \pzero \;|\; x?F \;|\; x!C }
  \and
  \inferrule* [lab=abstraction] {} {{F} \bc (x)P}
  \and
  \inferrule* [lab=concretion] {} {{C} \bc \langle Q \rangle}
  \and
  \inferrule* [lab=process] {} {{P,Q} \bc M \;| \;P|Q \;|\; @{x}}
  \and
  \inferrule* [lab=name] {} {{x} \bc \quotep{P}}
\end{mathpar} 

Note that $\vec{x}$ (resp. $\vec{P}$) denotes a vector of names
(resp. processes) of length $|\vec{x}|$ (resp. $|\vec{P}|$). We adopt
the following useful abbreviations.

\begin{mathpar}
   x?(\vec{y}).P := x.(\vec{y})P \and  x\clift{\vec{P}} := x.\clift{\vec{P}}
   \and x!(y) := \lift{x}{\dropn{y}}
   \and \Pi_{i=0}^{n-1}P_i := P_0 | \ldots | P_{n-1}
\end{mathpar}

\subsubsection{Structural congruence}

\paragraph{Free and bound names and alpha-equivalence.} At the
core of structural equivalence is alpha-equivalence which identifies
process that are the same up to a change of variable. Formally, we
recognize the distinction between free and bound names. The free names
of a process, $\freenames{P}$, may be calculated recursively as
follows:

\begin{mathpar}
\freenames{\pzero} := \emptyset
  \and \\
  \freenames{x?(y).P} := \{ x \} \cup (\freenames{P} \setminus \{ y \})
  \and 
  \freenames{x!\langle P \rangle} := \{ x \} \cup \{ P \} 
  \and \\
  \freenames{P|Q} := \freenames{P} \cup \freenames{Q}
  \and \\
  \freenames{@{x}} := \{ x \}
\end{mathpar}

$\pi$
$\quotep{\pi}$

$\freenames{-} : \pi \to \mathcal{P}(\quotep{\pi})$

\begin{eqnarray*}
  \freenames{\pzero} & := & \emptyset \\
  \freenames{x?(y).P} & := & \{ x \} \cup (\freenames{P} \setminus \{ y \}) \\
  \freenames{x!\langle P \rangle} & := & \{ x \} \cup \{ P \} \\
  \freenames{P|Q} & := & \freenames{P} \cup \freenames{Q} \\
  \freenames{\dropn{x}} & := & \{ x \}
\end{eqnarray*}

The bound names of a process, $\boundnames{P}$, are those names occurring in $P$
that are not free. For example, in $x?(y).0$, the name $x$ is free, while $y$ is bound.

\begin{mathpar}
  \inferrule* [lab=monoidal-laws] {} { P|Q \equiv Q|P \and P|0 \equiv P \and P|(Q|R) \equiv (P|Q)|R }
\end{mathpar}

\begin{mathpar}
  \inferrule* [lab=alpha-equivalence] {} { (x)P \equiv (y)P\{y/x\} \and y \not\in \freenames{P} }
\end{mathpar}

\begin{definition}
Then two processes, $P,Q$, are alpha-equivalent if $P = Q\{\vec{y}/\vec{x}\}$ for
some $\vec{x} \in \boundnames{Q},\vec{y} \in \boundnames{P}$, where $Q\{\vec{y}/\vec{x}\}$
denotes the capture-avoiding substitution of $\vec{y}$ for $\vec{x}$ in $Q$.
\end{definition}

\begin{definition}
  The {\em structural congruence} \cite{SangiorgiWalker} , $\equiv$,
  between processes is the least congruence containing
  alpha-equivalence, satisfying the abelian monoid laws
  (associativity, commutativity and $\pzero$ as identity) for parallel
  composition $|$ and for summation $+$.
\end{definition}

\subsection{Name equivalence}

We take name equivalence, written $\nameeq$, to be the smallest
equivalence relation generated by the following rules.

\begin{mathpar}
\inferrule*[lab=Quote-drop]
{ }
{ \quotep{@{x}} \nameeq x }

\inferrule*[lab=Struct-equiv]
{ P \scong Q }
{ \quotep{P} \nameeq \quotep{Q} }
\end{mathpar}

The astute reader will have noticed that the mutual recursion of names
and processes imposes a mutual recursion on alpha-equivalence and
structural equivalence via name-equivalence. Fortunately, all of this
works out pleasantly and we may calculate in the natural way, free of
concern. The reader interested in the details is referred to the
appendix \ref{appendix:rho_details}.

\subsection{Substitution}

We use $\Proc$ for the set of processes, $\QProc$ for the set of
names, and $\id{\{}\vec{y} / \vec{x} \id{\}}$ to denote partial maps,
$s : \QProc \rightarrow \QProc$. A map, $s$ lifts, uniquely, to a map
on process terms, $\widehat{s} : \Proc \rightarrow \Proc$ by the
following equations.

\begin{mathpar}
  (0) \psubstp{Q}{P} := 0 \\
  (R \juxtap S) \psubstp{Q}{P}
  :=    
  (R)\psubstp{Q}{P} \juxtap (S) \psubstp{Q}{P} \\
  (x?(y).R) \psubstp{Q}{P}    
  :=    
  (x)\substp{Q}{P} (z)\concat( (R \psubstn{z}{y}) \psubstp{Q}{P} ) \\
  (\lift{x}{R}) \psubstp{Q}{P}  
  :=
  \lift{(x)\substp{Q}{P}}{ R \psubstp{Q}{P} } \\
%   (\dropn{x})  \psubstp{Q}{P}       
%   := 
%   \left\{ 
%     \begin{array}{ccc} 
%       \dropn{\quotep{Q}} & & x \nameeq \quotep{P} \\
%       \dropn{x} & & otherwise \\
%     \end{array}
%   \right. 
  (\dropn{x})  \psubstp{Q}{P}       
  := 
  \left\{ 
    \begin{array}{ccc} 
      Q & & x \nameeq \quotep{P} \\
      \dropn{x} & & otherwise \\
    \end{array}
  \right.
\end{mathpar}
 

where

\begin{eqnarray}
  (x)\id{\{} \lpquote Q \rpquote / \lpquote P \rpquote \id{\}}            = 
  \left\{ 
    \begin{array}{ccc}
      \lpquote Q \rpquote & & x \nameeq \lpquote P \rpquote \\
      x & & otherwise \\
    \end{array}
  \right. \nonumber
\end{eqnarray}

and $z$ is chosen distinct from $\quotep{P}$, $\quotep{Q}$, the free
names in $Q$, and all the names in $R$. Our $\alpha$-equivalence will
be built in the standard way from this substitution.

\begin{remark}\label{rem:no_self_referential_names}
  One consequence of these definitions is that $\forall P. \quotep{P}
  \not\in \freenames{P}$.
\end{remark}

\subsection{ Dynamic quote: an example }

Anticipating something of what's to come, consider applying the
substitution, $\widehat{\id{\{}u / z \id{\}}}$, to the following pair
of processes, $\lift{w}{y!(z)}$ and $w[ \lpquote y!(z) \rpquote ]$.

\begin{eqnarray}
	\lift{w}{y!(z)}\widehat{\id{\{}u / z \id{\}}}
		& = &
		\lift{w}{y!(u)} \nonumber\\
	w[ \lpquote y!(z) \rpquote ] \widehat{ \id{\{}u / z \id{\}} }
		& = &
		w[ \lpquote y!(z) \rpquote ] \nonumber
\end{eqnarray}

Because the body of the process between quotes is impervious to
substitution, we get radically different answers. In fact, by
examining the first process in an input context,
e.g. $x?(z).\lift{w}{y!(z)}$, we see that the process under the lift
operator may be shaped by prefixed inputs binding a name inside it. In
this sense, the lift operator will be seen as a way to dynamically
construct processes before reifying them as names.

Finally equipped with these standard features we can present the
dynamics of the calculus.

\subsubsection{Operational semantics} 

Finally, we introduce the computational dynamics. What marks these
algebras as distinct from other more traditionally studied algebraic
structures, e.g. vector spaces or polynomial rings, is the manner in
which dynamics is captured. In traditional structures, dynamics is typically
expressed through morphisms between such structures, as in linear maps
between vector spaces or morphisms between rings. In algebras
associated with the semantics of computation, the dynamics is
expressed as part of the algebraic structure itself, through a
reduction reduction relation typically denoted by $\red$. Below, we
give a recursive presentation of this relation for the calculus used
in the encoding.

$\red \subseteq \pi \times \pi$
$\red : \pi \to \mathcal{P}(\pi)$

\begin{mathpar}
  \inferrule* [lab=Comm] { \textsf{match}( x_{src}, x_{trgt} ) } { x_{trgt}?(y)P \; | \; x_{src}!\langle {Q} \rangle \red P\{\quotep{Q}/y}\} }
  \and \\
  \inferrule* [lab=Par] {{P} \red {P}'} {{{P} | {Q}} \red {{P}' | {Q}}}
  \and
  \inferrule* [lab=Equiv]{{{P} \scong {P}'} \andalso {{P}' \red {Q}'} \andalso {{Q}' \scong {Q}}}{{P} \red {Q}}
\end{mathpar}

\begin{eqnarray*}
  match_{\equiv} (\quotep{P},\quotep{Q}) & := & P \equiv Q \\
  match_{\dagger}(\quotep{P},\quotep{Q}) & := & \forall R. P|Q \red^{*} R => R \red^{*} 0 \\
  match_{K}(\quotep{P},\quotep{Q}) & := & K \mbox{ for some context } K
\end{eqnarray*}

$u?(x)P | u!\langle Q \rangle \red P\{\quotep{Q}/x\}$

%We write $\wred$ for $\red^*$, and $P\red$ if $\exists Q $ such that $ P \red Q$.
We write $P\red$ if $\exists Q $ such that $ P \red Q$ and $P\not\red$, otherwise.

\section{Replication}

As mentioned before, it is known that replication (and hence
recursion) can be implemented in a higher-order process algebra
\cite{SangiorgiWalker}. As our first example of calculation with the
machinery thus far presented we give the construction explicitly in
the {\rhoc}.

\begin{eqnarray}
	D_{x} & := & \prefix{x}{y}{(\binpar{\outputp{x}{y}}{@{y}})} \nonumber\\
	\bangp_{x}{P} & := & \binpar{{x}!\langle{\binpar{D_{x}}{P}}\rangle}{D_{x}} \nonumber
\end{eqnarray}

\begin{eqnarray}
	\bangp_{x}{P} & & \nonumber\\
	=
	& {x}!\langle{(\prefix{x}{y}{(\outputp{x}{y} | @{y})) | P}}\rangle 
	      | \prefix{x}{y}{(\outputp{x}{y} | @{y})} & \nonumber\\
	\red
	& (\outputp{x}{y} | @{y})\substn{\quotep{(\prefix{x}{y}{(@{y} | \outputp{x}{y})) | P}}}{y} & \nonumber\\
	=
	& \outputp{x}{\quotep{(\prefix{x}{y}{(\outputp{x}{y} | @{y})) | P}}}
	  | {(\prefix{x}{y}{(\outputp{x}{y} | @{y})) | P}} & \nonumber\\
	\red
	& \ldots & \nonumber\\
	\red^*
	& P | P | \ldots & \nonumber
\end{eqnarray}

Of course, this encoding, as an implementation, runs away, unfolding
$\bangp{P}$ eagerly. A lazier and more implementable replication
operator, restricted to input-guarded processes, may be obtained as follows.

\begin{eqnarray}
\bangp{\prefix{u}{v}{P}} 
	:= 
	\binpar{\lift{x}{\prefix{u}{v}{(\binpar{D(x)}{P})}}}{D(x)} \nonumber
\end{eqnarray}

\begin{remark}
  Note that the lazier definition still does not deal with summation
  or mixed summation (i.e. sums over input and output). The reader is
  invited to construct definitions of replication that deal with these
  features. 

  Further, the definitions are parameterized in a name, $x$. Can you,
  gentle reader, make a definition that eliminates this parameter and
  guarantees no accidental interaction between the replication
  machinery and the process being replicated -- i.e. no accidental
  sharing of names used by the process to get its work done and the
  name(s) used by the replication to effect copying. This latter
  revision of the definition of replication is crucial to obtaining
  the expected identity $!!P \sim !P$.
\end{remark}

\begin{remark}\label{rem:paradoxical_combinator}
  The reader familiar with the lambda calculus will have noticed the
  similarity between $D$ and the paradoxical combinator.

  [Ed. note: the existence of this seems to suggest we have to be more
  restrictive on the set of processes and names we admit if we are to
  support no-cloning.]
\end{remark}

\subsubsection{Bisimulation}

The computational dynamics gives rise to another kind of equivalence,
the equivalence of computational behavior. As previously mentioned
this is typically captured \emph{via} some form of bisimulation.

% The notion we use in this paper is weak barbed bisimulation
% \cite{milner91polyadicpi}.

The notion we use in this paper is derived from weak barbed
bisimulation \cite{milner91polyadicpi}. 

\begin{definition}
An \emph{observation relation}, $\downarrow_{\mathcal N}$, over a set
of names, $\mathcal N$, is the smallest relation satisfying the rules
below.

\infrule[Out-barb]{y \in {\mathcal N}, \; x \nameeq y}
		  {\outputp{x}{v} \downarrow_{\mathcal N} x}
\infrule[Par-barb]{\mbox{$P\downarrow_{\mathcal N} x$ or $Q\downarrow_{\mathcal N} x$}}
		  {\binpar{P}{Q} \downarrow_{\mathcal N} x}

We write $P \Downarrow_{\mathcal N} x$ if there is $Q$ such that 
$P \wred Q$ and $Q \downarrow_{\mathcal N} x$.
\end{definition}

\begin{definition}
%\label{def.bbisim}
An  ${\mathcal N}$-\emph{barbed bisimulation} over a set of names, ${\mathcal N}$, is a symmetric binary relation 
${\mathcal S}_{\mathcal N}$ between agents such that $P\rel{S}_{\mathcal N}Q$ implies:
\begin{enumerate}
\item If $P \red P'$ then $Q \wred Q'$ and $P'\rel{S}_{\mathcal N} Q'$.
\item If $P\downarrow_{\mathcal N} x$, then $Q\Downarrow_{\mathcal N} x$.
\end{enumerate}
$P$ is ${\mathcal N}$-barbed bisimilar to $Q$, written
$P \wbbisim_{\mathcal N} Q$, if $P \rel{S}_{\mathcal N} Q$ for some ${\mathcal N}$-barbed bisimulation ${\mathcal S}_{\mathcal N}$.
\end{definition}

$\mathcal{R} \subseteq \pi \times \pi$

$P \mathcal{R} Q => \forall P'. P \red P' \Rightarrow \exists Q'. Q \red Q', P' \mathcal{R} Q'$

$P \vdash x \Rightarrow Q \vdash x$

\begin{mathpar}
  \inferrule*[lab=Out-barb]{x \nameeq y}{{y}!\langle{Q}\rangle \vdash x}
  \and
  \inferrule*[lab=Par-barb]{\mbox{$P\vdash x$ or $Q\vdash x$}}{\binpar{P}{Q} \vdash x}
\end{mathpar}

\subsubsection{Contexts}

One of the principle advantages of computational calculi like the
$\pi$-calculus is a well-defined notion of context,
contextual-equivalence and a correlation between
contextual-equivalence and notions of bisimulation. The notion of
context allows the decomposition of a process into (sub-)process and
its syntactic environment, its context. Thus, a context may be
thought of as a process with a ``hole'' (written $\Box$) in it. The
application of a context $M$ to a process $P$, written $M[P]$, is
tantamount to filling the hole in $M$ with $P$. In this paper we do
not need the full weight of this theory, but do make use of the notion
of context in the proof the main theorem. 

\begin{mathpar}
  \inferrule* [lab=summation] {} {{M_{M},M_{N}} \bc \Box \;|\; x.M_{A} \;|\; M_{M}+M_{N}}
  \and
  \inferrule* [lab=agent] {} {{M_{A}} \bc (\vec{x})M_{P} \;| \; \clift{P_0,\ldots,M_{P},\ldots,P_N}}
  \and \\
  \inferrule* [lab=process] {} {{M_{P}} \bc M_{N} \;| \;P|M_{P} }
\end{mathpar} 

\begin{mathpar}
  \inferrule* [lab=sychronization] {} {M_{N} \bc \Box \;|\; x?M_{F} \;|\; x!M_{C}}
  \and
  \inferrule* [lab=abstraction] {} {{M_{F}} \bc (x)M_{P} }
  \and
  \inferrule* [lab=concretion] {} {{M_{C}} \bc \langle M_{P} \rangle }
  \and \\
  \inferrule* [lab=process] {} {{M_{P}} \bc M_{N} \;| \;P|M_{P} }
\end{mathpar}

\begin{definition}[contextual application] Given a context $M$, and
  process $P$, we define the \emph{contextual application}, $M[P] :=
  M\{P/\Box\}$. That is, the contextual application of M to P is the
  substitution of $P$ for $\Box$ in $M$.
\end{definition}

$\meaningof{-} : L \to \mathcal{P}(\pi)$

\begin{mathpar}
  \inferrule* [lab=collection] {} {\meaningof{true} = \pi, \and \meaningof{~E} = \pi \setminus \meaningof{E}, \and \meaningof{E_{1} \& E_{2}} = \meaningof{E_{1}} \cap \meaningof{E_{2}}}
\end{mathpar}

\begin{mathpar}
  \inferrule* [lab=structure] {} {\meaningof{0} = \{ P \in \pi | P \equiv 0 \}, \and \\ \meaningof{E_1 | E_2} = \{ P \in \pi | P \equiv P_{1} | P_{2}, P_{1} \in \meaningof{E_{1}}, P_{2} \in \meaningof{E_2}\} }
\end{mathpar}

\begin{mathpar}
 \inferrule* [lab=behavior] {} {\meaningof{\langle a?b \rangle E} = \{ P \in \pi | P \equiv Q | u?(y)P', \\ \and \\\\ \and \\ \;\;\; u \in \meaningof{a}, \forall z.P'\{z/y\} \in \meaningof{E\{z/b\}}\}, \and \\ \meaningof{a!E} = \{ P \in \pi | P \equiv Q | x!\langle P' \rangle, x \in \meaningof{a} P' \in \meaningof{E}\} }
\end{mathpar}

\begin{mathpar}
 \inferrule* [lab=nominal] {} {\meaningof{\quotep{E}} = \{ \quotep{P} \in \quotep{\pi} | P \in \meaningof{E} \}, \and \meaningof{\quotep{P}} = \{ \quotep{Q} \in \quotep{\pi} | P \equiv Q \} \and \\ \meaningof{@\quotep{E}} = \{ P \in \pi | P \equiv @x, x \in \meaningof{E} \}}
\end{mathpar}

\begin{eqnarray*}
  \\
  \meaningof{-} : TS \to ST
\end{eqnarray*}

\begin{eqnarray*}
  \\
  L : TS \to ST
\end{eqnarray*}

\begin{eqnarray*}
  \\
  P \models E \iff P \in \meaningof{E}
\end{eqnarray*}

\begin{eqnarray*}
  P \approx_{L} Q \iff \forall E \in L. P \models E \iff Q \models E
\end{eqnarray*}

\begin{eqnarray*}
  P \approx_{K} Q
\end{eqnarray*}

\begin{eqnarray*}
  P \approx Q
\end{eqnarray*}

$\approx_{K} = \approx = \approx_{L}$

\subsubsection{Contextual duality}

Note that contexts extend the quotation operation to a family of
operations from processes to names. Given a context, $M$, we can
define a \emph{nominal context}, $\quotep{M}$ by $\quotep{M}[P] :=
\quotep{M[P]}$. To foreshadow what is to come we observe that these
operations enjoy a duality with processes very much like the duality
between vectors and maps from vectors to scalars.

Further, because the calculus is essentially higher-order, we have a
correspondence between contexts and processes. More specifically,
given a name $x$ and a context $M$ we can construct $M^{*}_{x}$ such
that 

\begin{mathpar}
  M^{*}_{x} | \lift{x}{P} \red M[P]
\end{mathpar}

namely,

\begin{mathpar}
  M^{*}_{x} := x?(u).M[\dropn{u}]
\end{mathpar}

The dependence of $M^{*}_{x}$ on a name makes it an abstraction, 

\begin{mathpar}
  M^{*} := (x)x?(u).M[\dropn{u}]
\end{mathpar}

\subsection{Additional notation}

It will sometimes be convenient to denote the process a name
quotes. We already have the notation $x = \quotep{P}$, but it will be
convenient to introduce an alternate notation, $\procn{x}$, when we
want to emphasize the connection to the use of the name. Note that, by
virtue of name equivalence, $\quotep{\procn{x}} \nameeq x$; so, the
notation is consistent with previous definitions.

Further, because names have structure it is possible to effect
substitutions on the basis of that structure. This means we need to
upgrade our notation for substitutions, which we accomplish by
adapting comprehension notation. Thus,

\begin{mathpar}
  P\{ y / x : x \in S \}
\end{mathpar}

is interpreted to mean the process derived from P by replacing (in a
capture-avoiding manner) each occurrence of $x$ in $S$ by $y$. For example,

\begin{mathpar}
  P\{ \quotep{\procn{x}|\procn{x}} / x : x \in \freenames{P} \}
\end{mathpar}

will replace each (occurrence) of a free name $x$ in $P$ by
$\quotep{\procn{x}|\procn{x}}$.

Also, we will avail ourselves of the notation $x^{L}$ and $x^{R}$ to
denote injections of a name into disjoint copies of the name
space. There are numerous ways to accomplish this. One example can be
found in \cite{MeredithR05}. This notation overloads to vectors of
names: $\vec{x}^{\pi} := (x_{i}^{\pi} \; : \; 0 \leq i < |\vec{x}| )$ where $\pi \in \{L,R\}$.

We also use $P^{\Box} := P|\Box$.

In \cite{MeredithR05} an interpretation of the new operator is
given. It turns out that there are several possible interpretations
all enjoying the requisite algebraic properties of the operator (see
\cite{milner91polyadicpi}). We will therefore make liberal use of
$(\nu\; \vec{x})P$.

% subsection the_syntax_and_semantics_of_the_notation_system (end)   

\section{Interpretation of QM}
\subsection{Supporting definitions}
\subsubsection{Multiplication}
\begin{mathpar}
  \quotep{Q} \cdot \quotep{R} := \quotep{Q|R}
  \and \\
  \quotep{Q} \cdot P := P\{ \quotep{Q|R} / \quotep{R} : \quotep{R} \in \freenames{P} \}
\end{mathpar}

\paragraph{Discussion}
The first line needs little explanation. The second line says that
each free name of the process is replaced with the multiplication of
that name by the scalar. Multiplication of a scalar (name) by a state
(process) results in a process all the names of which have been `moved
over' by parallel composition with the process the scalar
quotes. There is a subtlety that the bound names have to be
manipulated so that multiplied names aren't accidentally
captured. There are many ways to achieve this.

\begin{remark}\label{rem:multiplication_identities}
  The reader is invited to verify that for all $x,y,z \in \QProc$ and $P \in \Proc$
  \begin{mathpar}
    x \cdot \quotep{0} \equiv x 
    \and
    x \cdot y \equiv y \cdot x
    \and
    x \cdot (y \cdot z) \equiv (x \cdot y) \cdot z
    \and \\
    \quotep{0} \cdot P \equiv P
    \and \\
    x \cdot (y \cdot P) \equiv (x \cdot y) \cdot P
    \and \\
    x \cdot (P|Q) \equiv (x \cdot P) | (x \cdot Q)
    \and \\    
  \end{mathpar}
\end{remark}

\subsubsection{Tensor product}

We define a tensor product on processes by structural induction.

\paragraph{Tensor of sums} First note that all summations, including
$\pzero$ and sequence, can be written $\Sigma_{i} x_{i}.A_{i} +
\Sigma_{j} x_{j}.C_{j}$, where we have grouped input-guarded processes
together and output-guarded processes together.

Thus, we can define the tensor product of two summations, $N_{1}\otimes N_{2}$, where

\begin{mathpar}
  N_{1} := \Sigma_{i} x_{i}.A_{i} + \Sigma_{j} x_{j}.C_{j}
  \and
  N_{2} := \Sigma_{i'} y_{i'}.B_{i'} + \Sigma_{j'} y_{j'}.D_{j'} 
\end{mathpar}

as follows.

\begin{mathpar}
  \Sigma_{i} x_{i}.A_{i} + \Sigma_{j} x_{j}.C_{j} \otimes \Sigma_{i'}
  y_{i'}.B_{i'} + \Sigma_{j'} y_{j'}.D_{j'} 
  \and \\
  := \; \Sigma_{i} \Sigma_{i'} \quotep{\stackrel{\vee}{x_{i}}| \stackrel{\vee}{y_{i'}}}.(A_{i}\otimes B_{i'}) \; | \; \Sigma_{i'} \Sigma_{i} \quotep{\stackrel{\vee}{y_{i'}}|\stackrel{\vee}{x_{i}}}.(B_{i'}\otimes A_{i})
  \and
  \;\; | \;\; \Sigma_{j} \Sigma_{j'} \quotep{\stackrel{\vee}{x_{j}}|\stackrel{\vee}{y_{j'}}}.(A_{j}\otimes B_{j'}) \; | \; \Sigma_{j'} \Sigma_{j} \quotep{\stackrel{\vee}{y_{j'}}|\stackrel{\vee}{x_{j}}}.(B_{j'}\otimes A_{j})
\end{mathpar}

\begin{remark}
  Do we need to $x^{L}$ and $y^{R}$ for this construction as well?
\end{remark}

\paragraph{Tensor of parallel compositions} Next, we distribute tensor
over par.

\begin{mathpar}
  P_{1}|P_{2} \otimes Q_{1}|Q_{2} := (P_{1} \otimes Q_{1}) | (P_{1}
  \otimes Q_{2}) | (P_{2} \otimes Q_{1}) | (P_{2} \otimes Q_{2})
\end{mathpar}

\paragraph{Tensor with dropped names} We treat tensor of a
process with a dropped name as parallel composition.

\begin{mathpar}
  P \otimes \dropn{x} := P | \dropn{x}
\end{mathpar}

\paragraph{Tensor of agents}

Finally, we need to define tensor on agents. Note that the definition
of tensor on normal products only tensors inputs with inputs and
outputs with outputs. Thus, we only have to define the operation on
``homogeneous'' pairings.

\begin{mathpar}
  (\vec{x})P \otimes (\vec{y})Q
  \and \\
  := (x_{0}^{L}|y_{0}^{R},\ldots,x_{0}^{L}|y_{n}^{R},\ldots,x_{m}^{L}|y_{0}^{R},\ldots,x_{m}^{L}|y_{n}^R)(P\{ \vec{x}^{L}/\vec{x}\} \otimes Q \{ \vec{y}^{R}/\vec{y}\})
  \and \\
  \clift{\vec{P}} \otimes \clift{\vec{Q}}
  \and \\
  := \clift{P_{0}\otimes Q_{0},\ldots,P_{0}\otimes Q_{n},\ldots,P_{m}\otimes Q_{0},\ldots,P_{m}\otimes Q_{n}}
\end{mathpar}

\begin{remark}
  Observe that arities of tensored abstractions matches arities of
  tensored concretions if the original arities matched. Note also that
  the length of the arities corresponds to the increase in dimension
  we see in ordinary vector space tensor product.
\end{remark}

\begin{remark}
  Operationally, this definition distributes the tensor down to
  components ``linked'' by summation. Tensor over summation is
  intriguing in that it mixes names. Moreover, as a consequence of the
  way it mixes names we have the identities for all $x \in \QProc$ and
  $P,Q \in \Proc$

  \begin{mathpar}
    (x \cdot P) \otimes Q \equiv x \cdot (P \otimes Q) \equiv P \otimes (x \cdot Q)
    \and
    P \otimes \pzero \equiv P
  \end{mathpar}

  that the reader is invited to verify.
\end{remark}

\subsubsection{Annihilation}
\begin{mathpar}
  P^{\perp} := \{ Q | \forall R. P|Q \red^{*} R \Rightarrow R \red^{*} \pzero \}
  \and \\
  P^{\underline{\perp}} := \Sigma_{Q \in P^{\perp}} \quotep{Q}?(y).(\dropn{y}|Q) | \Sigma_{Q \in P^{\perp}} \quotep{Q}\clift{\Box}
\end{mathpar}

\paragraph{Discussion} The reader will note that $P^{\perp}$ is a
\emph{set} of processes, while $P^{\underline{\perp}}$ is a
\emph{context}. We call the set $P^{\perp}$ the \emph{annihilators} of
$P$. The parallel composition of a process in the annihilators of $P$
with $P$ will result in a process, the state space of which has all
paths eventually leading to $\pzero$. Execution may endure loops; but
under reasonable conditions of fairness (naturally guaranteed under
most notions of bisimulation) such a composite process cannot get
stuck in such a loop and will, eventually pop out and terminate.

The context $P^{\underline{\perp}}$ is ready and willing to ``take the
$P$ out of'' the process to which it is applied. It will effectively
transmit the code of the process to which it is applied to one of the
annihilators and run the process against it.

\subsubsection{Evaluation}
We fix $M$ a domain of fully abstract interpretation with an equality
coincident with bisimulation. We take $\meaningof{\cdot} : \Proc \to
M$ to be the map interpreting processes and $\nmeaningof{\cdot} : \M
\to Proc$ to be the map running the other way. Then we define

\begin{mathpar}
  \int P := \nmeaningof{\meaningof{P}}
\end{mathpar}

\paragraph{Discussion}
There are many fully abstract interpretations of Milner's
$\pi$-calculus. Any of them can be used as a basis for interpreting
the reflective calculus here. Equipped with such a domain it is
largely a matter of grinding through to check that the Yoneda
construction for the normalization-by-evaluation program can be
extended to this setting.

\begin{remark}
  The reader is invited to verify that $\int (P^{\underline{\perp}}[P]) = 0$.
\end{remark}

\subsection{Quantum mechanics}

Table \ref{tbl:core_qm_op_defns} gives the core operational definitions

\begin{table}[htp]\label{tbl:core_qm_op_defns}
  \center{
    \fbox{
      \begin{tabular}{c|c}
        quantum mechanics & process calculus \\
        \hline
        scalar & $x := \quotep{P}$ \\
        state vector & $\state{P} := P$ \\
        dual & $\state{P}^{*} := \event{P^{\underline{\perp}}} := \quotep{P^{\underline{\perp}}}[-]$ \\
        matrix & $ \Sigma_{\alpha} \state{P_{\alpha}}x_{\alpha}\event{Q_{\alpha}}$ \\
        vector addition & $\state{P} + \state{Q} := \state{P | Q}$ \\
        tensor product & $\state{P} \otimes \state{Q} := \state{P \otimes Q}$ \\
        inner product & $\innerprod{P}{Q} := \quotep{\int P^{\underline{\perp}}[Q]}$ \\
      \end{tabular}
    }
  }
  \caption{QM - operational definitions}
\end{table}

where

\begin{mathpar}
  \prmatrix{P}{Q} := \fprmatrix{P}{\quotep{\pzero}}{Q}
  \and
  \fprmatrix{P}{x}{Q} := (\state{P},x,\event{Q})
  \and
  (\fprmatrix{P}{x}{Q})(\state{R}) := x \cdot \innerprod{Q}{R} \cdot \state{P}
  \and
  (\fprmatrix{P}{x}{Q})(\event{R}) := x \cdot \innerprod{R}{P} \cdot \event{Q}
\end{mathpar}

\paragraph{Discussion}
As promised: vectors (aka states) are represented as processes; duals
as contextual duals; inner product definition should be compared with
standard inner product definition for ....

\begin{remark}
  Assuming $\int (P^{\underline{\perp}}[P]) = 0$, the reader is
  invited to verify that $(\fprmatrix{P}{x}{P})(\state{P}) = x \cdot \state{P}$.
\end{remark}

\begin{remark}
  The reader is invited to verify that $\innerprod{P}{Q}$ could
  equally well have been written $\quotep{\int \stackrel{\vee}{x}}$
  where $x = \event{P^{\underline{\perp}}}(Q)$.

  One of the motivations for this remark is that there is another way
  to factor these operations. We could package up evaluation in the dual:

  \begin{mathpar}
    \state{P}^{*} := \event{\int P^{\underline{\perp}}} := \quotep{\int P^{\underline{\perp}}}[-]
  \end{mathpar}

  and then have inner product defined by
  
  \begin{mathpar}
    \innerprod{P}{Q} := \event{P}(Q)
  \end{mathpar}

  Hopefully, experience with the calculations will provide guidance on
  the best factoring.
\end{remark}

\begin{remark}
  Assuming $\int (P^{\underline{\perp}}[P]) = 0$, the reader is
  invited to verify that $\forall P,Q. (\prmatrix{0}{Q})(\state{0}) =
  \state{0}$ and dually $(\prmatrix{P}{0})(\event{0}) = \event{0}$.
\end{remark}

\begin{remark}
  i'm a little worried that i don't (yet) have proper support for
  complex conjugacy. But, the observation above may give us a
  clue. According to Abramsky, it must be the case that the scalars
  are iso to the homset of the identity for the tensor -- which the
  observation above characterizes. 

  For now, we will simply bookmark the notion with $\overline{x}$.
\end{remark}

\subsubsection{Adjointness}

We need to give a definition of $(\cdot)^{\dagger}$ for matrices. The
obvious candidate definition is
\begin{mathpar}
(\Sigma_{\alpha}\fprmatrix{P_{\alpha}}{x_{\alpha}}{Q_{\alpha}})^{\dagger}
= \Sigma_{\alpha}\fprmatrix{(Q_{\alpha}^{\underline{\perp}})^{*}}{\overline{x}_{\alpha}}{P_{\alpha}^{\underline{\perp}}} 
\end{mathpar}

But, $(Q_{\alpha}^{\underline{\perp}})^{*}$ requires a name along
which to communicate the process to achieve the context application.

\subsubsection{Basis for a basis}
If processes label states and ``addition'' of states (a.k.a. vector
addition) is interpreted as parallel composition, what corresponds to
notions of linear independence and basis? Here, we recall that Yoshida
has developed a set of \emph{combinators} for an asynchronous verison
of Milner's $\pi$-calculus. These are a finite set of processes such
any process can be expressed as parallel composition of these
combinators together with liberal uses of the new operator and
replication. We can simply give a translation of these into the
present calculus and have reasonable expectation that the property
carries over. That is, that the resultant set allows to express all
processes via parallel composition. Note, however, that there is no
new operator or replication in this calculus. As a result, we expect
that the corresponding set is actually infinite. That is, we expect
that the space is actually infinite dimensional.

\begin{remark}
  The attentive reader may be a bit concerned. Certainly, the
  collection $S$, $K$ and $I$ is a finite set of
  combinators. Shouldn't we expect to see a finite set of combinators
  for an effectively equivalent system? i am very sympathetic to this
  critique and feel it warrants full attention. On the other hand, i
  also have in mind the following analogy. The natural numbers, as a
  monoid under addition, has exactly $1$ generator, while the natural
  numbers, as a monoid under multiplication, has countably many
  generators (the primes). We observe that the application of the
  lambda calculus is much less resource sensitive than the parallel
  composition of the $\pi$-calculus. Could it be the case that we have
  an analogy of the form
  
  \begin{mathpar}
    m + n : MN :: m*n : M|N
  \end{mathpar}

  giving a similar blow up in the set of ``primes''?  This is such a
  wonderful thought that, even if it's not true, i think it's worth
  writing down.
\end{remark}
 

\documentclass[12pt]{llncs}
%\documentclass{jktr}

\usepackage[pdftex]{hyperref}                   
\usepackage {listings}
\usepackage {mathpartir}
\usepackage{bcprules}
%\usepackage{listings}
                       
\usepackage{graphicx} 
%\usepackage[margins=2.5cm,nohead,nofoot]{geometry}
%\usepackage{geometry}
\usepackage{amsfonts}
\usepackage{amstext}
\usepackage{latexsym}
\usepackage{amssymb}
\usepackage{color}


%\include{myPreamble}
\include{qm2pi.local} 

%\ifpdf
%\usepackage[pdftex]{graphicx}
%\else
%\usepackage{graphicx}
%\fi

 % \ifpdf
%  \usepackage{pdfsync}
%  \if


%\title{Brief Article}
%\author{David F. Snyder}
%\author{L.G. Meredith}

%\address{Dept. of Math., Texas State University--San Marcos, San Marcos, TX 78666}
       
\pagestyle{empty}


\begin{document}

\lstset{language=[Objective]Caml,frame=shadowbox}

\input{qm2pi.front}

% section front matter (end)

\input{qm2pi.intro} 
 
% section introduction (end)

% \input{qm2pi.knotations} 

% section notation (end)

\input{qm2pi.process.calculi} 

% section concurrent_process_calculi_and_spatial_logics_ (end)
    
%\input{qm2pi.knots2pi} 

%\input{qm2pi.trefoil} 

%\input{qm2pi.mainthm} 

% subsection basic_interpretation (end)

%\input{qm2pi.rho.presentation} 
\subsection{The syntax and semantics of the notation system}\label{sub:the_syntax_and_semantics_of_the_notation_system} % (fold)

We now summarize a technical presentation of the calculus that
embodies our theory of dynamics. The typical presentation of such a
calculus follows the style of giving generators and relations on
them. The grammar, below, describing term constructors, freely
generates the set of processes, $\Proc$. This set is then quotiented
by a relation known as structural congruence and it is over this set
that the notion of dynamics is expressed. This presentation is
essentially that of \cite{MeredithR05} with the addition of
polyadicity and summation. For readability we have relegated some of
the technical subtleties to an appendix.

\subsubsection{Process grammar}\label{subsub:process_grammar}

\begin{mathpar}
  \inferrule* [lab=synchronization] {} {{M} \bc \pzero \;|\; x?F \;|\; x!C }
  \and
  \inferrule* [lab=abstraction] {} {{F} \bc (x)P}
  \and
  \inferrule* [lab=concretion] {} {{C} \bc \langle Q \rangle}
  \and
  \inferrule* [lab=process] {} {{P,Q} \bc M \;| \;P|Q \;|\; @{x}}
  \and
  \inferrule* [lab=name] {} {{x} \bc \quotep{P}}
\end{mathpar} 

Note that $\vec{x}$ (resp. $\vec{P}$) denotes a vector of names
(resp. processes) of length $|\vec{x}|$ (resp. $|\vec{P}|$). We adopt
the following useful abbreviations.

\begin{mathpar}
   x?(\vec{y}).P := x.(\vec{y})P \and  x\clift{\vec{P}} := x.\clift{\vec{P}}
   \and x!(y) := \lift{x}{\dropn{y}}
   \and \Pi_{i=0}^{n-1}P_i := P_0 | \ldots | P_{n-1}
\end{mathpar}

\subsubsection{Structural congruence}

\paragraph{Free and bound names and alpha-equivalence.} At the
core of structural equivalence is alpha-equivalence which identifies
process that are the same up to a change of variable. Formally, we
recognize the distinction between free and bound names. The free names
of a process, $\freenames{P}$, may be calculated recursively as
follows:

\begin{mathpar}
\freenames{\pzero} := \emptyset
  \and \\
  \freenames{x?(y).P} := \{ x \} \cup (\freenames{P} \setminus \{ y \})
  \and 
  \freenames{x!\langle P \rangle} := \{ x \} \cup \{ P \} 
  \and \\
  \freenames{P|Q} := \freenames{P} \cup \freenames{Q}
  \and \\
  \freenames{@{x}} := \{ x \}
\end{mathpar}

$\pi$
$\quotep{\pi}$

$\freenames{-} : \pi \to \mathcal{P}(\quotep{\pi})$

\begin{eqnarray*}
  \freenames{\pzero} & := & \emptyset \\
  \freenames{x?(y).P} & := & \{ x \} \cup (\freenames{P} \setminus \{ y \}) \\
  \freenames{x!\langle P \rangle} & := & \{ x \} \cup \{ P \} \\
  \freenames{P|Q} & := & \freenames{P} \cup \freenames{Q} \\
  \freenames{\dropn{x}} & := & \{ x \}
\end{eqnarray*}

The bound names of a process, $\boundnames{P}$, are those names occurring in $P$
that are not free. For example, in $x?(y).0$, the name $x$ is free, while $y$ is bound.

\begin{mathpar}
  \inferrule* [lab=monoidal-laws] {} { P|Q \equiv Q|P \and P|0 \equiv P \and P|(Q|R) \equiv (P|Q)|R }
\end{mathpar}

\begin{mathpar}
  \inferrule* [lab=alpha-equivalence] {} { (x)P \equiv (y)P\{y/x\} \and y \not\in \freenames{P} }
\end{mathpar}

\begin{definition}
Then two processes, $P,Q$, are alpha-equivalent if $P = Q\{\vec{y}/\vec{x}\}$ for
some $\vec{x} \in \boundnames{Q},\vec{y} \in \boundnames{P}$, where $Q\{\vec{y}/\vec{x}\}$
denotes the capture-avoiding substitution of $\vec{y}$ for $\vec{x}$ in $Q$.
\end{definition}

\begin{definition}
  The {\em structural congruence} \cite{SangiorgiWalker} , $\equiv$,
  between processes is the least congruence containing
  alpha-equivalence, satisfying the abelian monoid laws
  (associativity, commutativity and $\pzero$ as identity) for parallel
  composition $|$ and for summation $+$.
\end{definition}

\subsection{Name equivalence}

We take name equivalence, written $\nameeq$, to be the smallest
equivalence relation generated by the following rules.

\begin{mathpar}
\inferrule*[lab=Quote-drop]
{ }
{ \quotep{@{x}} \nameeq x }

\inferrule*[lab=Struct-equiv]
{ P \scong Q }
{ \quotep{P} \nameeq \quotep{Q} }
\end{mathpar}

The astute reader will have noticed that the mutual recursion of names
and processes imposes a mutual recursion on alpha-equivalence and
structural equivalence via name-equivalence. Fortunately, all of this
works out pleasantly and we may calculate in the natural way, free of
concern. The reader interested in the details is referred to the
appendix \ref{appendix:rho_details}.

\subsection{Substitution}

We use $\Proc$ for the set of processes, $\QProc$ for the set of
names, and $\id{\{}\vec{y} / \vec{x} \id{\}}$ to denote partial maps,
$s : \QProc \rightarrow \QProc$. A map, $s$ lifts, uniquely, to a map
on process terms, $\widehat{s} : \Proc \rightarrow \Proc$ by the
following equations.

\begin{mathpar}
  (0) \psubstp{Q}{P} := 0 \\
  (R \juxtap S) \psubstp{Q}{P}
  :=    
  (R)\psubstp{Q}{P} \juxtap (S) \psubstp{Q}{P} \\
  (x?(y).R) \psubstp{Q}{P}    
  :=    
  (x)\substp{Q}{P} (z)\concat( (R \psubstn{z}{y}) \psubstp{Q}{P} ) \\
  (\lift{x}{R}) \psubstp{Q}{P}  
  :=
  \lift{(x)\substp{Q}{P}}{ R \psubstp{Q}{P} } \\
%   (\dropn{x})  \psubstp{Q}{P}       
%   := 
%   \left\{ 
%     \begin{array}{ccc} 
%       \dropn{\quotep{Q}} & & x \nameeq \quotep{P} \\
%       \dropn{x} & & otherwise \\
%     \end{array}
%   \right. 
  (\dropn{x})  \psubstp{Q}{P}       
  := 
  \left\{ 
    \begin{array}{ccc} 
      Q & & x \nameeq \quotep{P} \\
      \dropn{x} & & otherwise \\
    \end{array}
  \right.
\end{mathpar}
 

where

\begin{eqnarray}
  (x)\id{\{} \lpquote Q \rpquote / \lpquote P \rpquote \id{\}}            = 
  \left\{ 
    \begin{array}{ccc}
      \lpquote Q \rpquote & & x \nameeq \lpquote P \rpquote \\
      x & & otherwise \\
    \end{array}
  \right. \nonumber
\end{eqnarray}

and $z$ is chosen distinct from $\quotep{P}$, $\quotep{Q}$, the free
names in $Q$, and all the names in $R$. Our $\alpha$-equivalence will
be built in the standard way from this substitution.

\begin{remark}\label{rem:no_self_referential_names}
  One consequence of these definitions is that $\forall P. \quotep{P}
  \not\in \freenames{P}$.
\end{remark}

\subsection{ Dynamic quote: an example }

Anticipating something of what's to come, consider applying the
substitution, $\widehat{\id{\{}u / z \id{\}}}$, to the following pair
of processes, $\lift{w}{y!(z)}$ and $w[ \lpquote y!(z) \rpquote ]$.

\begin{eqnarray}
	\lift{w}{y!(z)}\widehat{\id{\{}u / z \id{\}}}
		& = &
		\lift{w}{y!(u)} \nonumber\\
	w[ \lpquote y!(z) \rpquote ] \widehat{ \id{\{}u / z \id{\}} }
		& = &
		w[ \lpquote y!(z) \rpquote ] \nonumber
\end{eqnarray}

Because the body of the process between quotes is impervious to
substitution, we get radically different answers. In fact, by
examining the first process in an input context,
e.g. $x?(z).\lift{w}{y!(z)}$, we see that the process under the lift
operator may be shaped by prefixed inputs binding a name inside it. In
this sense, the lift operator will be seen as a way to dynamically
construct processes before reifying them as names.

Finally equipped with these standard features we can present the
dynamics of the calculus.

\subsubsection{Operational semantics} 

Finally, we introduce the computational dynamics. What marks these
algebras as distinct from other more traditionally studied algebraic
structures, e.g. vector spaces or polynomial rings, is the manner in
which dynamics is captured. In traditional structures, dynamics is typically
expressed through morphisms between such structures, as in linear maps
between vector spaces or morphisms between rings. In algebras
associated with the semantics of computation, the dynamics is
expressed as part of the algebraic structure itself, through a
reduction reduction relation typically denoted by $\red$. Below, we
give a recursive presentation of this relation for the calculus used
in the encoding.

$\red \subseteq \pi \times \pi$
$\red : \pi \to \mathcal{P}(\pi)$

\begin{mathpar}
  \inferrule* [lab=Comm] { \textsf{match}( x_{src}, x_{trgt} ) } { x_{trgt}?(y)P \; | \; x_{src}!\langle {Q} \rangle \red P\{\quotep{Q}/y}\} }
  \and \\
  \inferrule* [lab=Par] {{P} \red {P}'} {{{P} | {Q}} \red {{P}' | {Q}}}
  \and
  \inferrule* [lab=Equiv]{{{P} \scong {P}'} \andalso {{P}' \red {Q}'} \andalso {{Q}' \scong {Q}}}{{P} \red {Q}}
\end{mathpar}

\begin{eqnarray*}
  match_{\equiv} (\quotep{P},\quotep{Q}) & := & P \equiv Q \\
  match_{\dagger}(\quotep{P},\quotep{Q}) & := & \forall R. P|Q \red^{*} R => R \red^{*} 0 \\
  match_{K}(\quotep{P},\quotep{Q}) & := & K \mbox{ for some context } K
\end{eqnarray*}

$u?(x)P | u!\langle Q \rangle \red P\{\quotep{Q}/x\}$

%We write $\wred$ for $\red^*$, and $P\red$ if $\exists Q $ such that $ P \red Q$.
We write $P\red$ if $\exists Q $ such that $ P \red Q$ and $P\not\red$, otherwise.

\section{Replication}

As mentioned before, it is known that replication (and hence
recursion) can be implemented in a higher-order process algebra
\cite{SangiorgiWalker}. As our first example of calculation with the
machinery thus far presented we give the construction explicitly in
the {\rhoc}.

\begin{eqnarray}
	D_{x} & := & \prefix{x}{y}{(\binpar{\outputp{x}{y}}{@{y}})} \nonumber\\
	\bangp_{x}{P} & := & \binpar{{x}!\langle{\binpar{D_{x}}{P}}\rangle}{D_{x}} \nonumber
\end{eqnarray}

\begin{eqnarray}
	\bangp_{x}{P} & & \nonumber\\
	=
	& {x}!\langle{(\prefix{x}{y}{(\outputp{x}{y} | @{y})) | P}}\rangle 
	      | \prefix{x}{y}{(\outputp{x}{y} | @{y})} & \nonumber\\
	\red
	& (\outputp{x}{y} | @{y})\substn{\quotep{(\prefix{x}{y}{(@{y} | \outputp{x}{y})) | P}}}{y} & \nonumber\\
	=
	& \outputp{x}{\quotep{(\prefix{x}{y}{(\outputp{x}{y} | @{y})) | P}}}
	  | {(\prefix{x}{y}{(\outputp{x}{y} | @{y})) | P}} & \nonumber\\
	\red
	& \ldots & \nonumber\\
	\red^*
	& P | P | \ldots & \nonumber
\end{eqnarray}

Of course, this encoding, as an implementation, runs away, unfolding
$\bangp{P}$ eagerly. A lazier and more implementable replication
operator, restricted to input-guarded processes, may be obtained as follows.

\begin{eqnarray}
\bangp{\prefix{u}{v}{P}} 
	:= 
	\binpar{\lift{x}{\prefix{u}{v}{(\binpar{D(x)}{P})}}}{D(x)} \nonumber
\end{eqnarray}

\begin{remark}
  Note that the lazier definition still does not deal with summation
  or mixed summation (i.e. sums over input and output). The reader is
  invited to construct definitions of replication that deal with these
  features. 

  Further, the definitions are parameterized in a name, $x$. Can you,
  gentle reader, make a definition that eliminates this parameter and
  guarantees no accidental interaction between the replication
  machinery and the process being replicated -- i.e. no accidental
  sharing of names used by the process to get its work done and the
  name(s) used by the replication to effect copying. This latter
  revision of the definition of replication is crucial to obtaining
  the expected identity $!!P \sim !P$.
\end{remark}

\begin{remark}\label{rem:paradoxical_combinator}
  The reader familiar with the lambda calculus will have noticed the
  similarity between $D$ and the paradoxical combinator.

  [Ed. note: the existence of this seems to suggest we have to be more
  restrictive on the set of processes and names we admit if we are to
  support no-cloning.]
\end{remark}

\subsubsection{Bisimulation}

The computational dynamics gives rise to another kind of equivalence,
the equivalence of computational behavior. As previously mentioned
this is typically captured \emph{via} some form of bisimulation.

% The notion we use in this paper is weak barbed bisimulation
% \cite{milner91polyadicpi}.

The notion we use in this paper is derived from weak barbed
bisimulation \cite{milner91polyadicpi}. 

\begin{definition}
An \emph{observation relation}, $\downarrow_{\mathcal N}$, over a set
of names, $\mathcal N$, is the smallest relation satisfying the rules
below.

\infrule[Out-barb]{y \in {\mathcal N}, \; x \nameeq y}
		  {\outputp{x}{v} \downarrow_{\mathcal N} x}
\infrule[Par-barb]{\mbox{$P\downarrow_{\mathcal N} x$ or $Q\downarrow_{\mathcal N} x$}}
		  {\binpar{P}{Q} \downarrow_{\mathcal N} x}

We write $P \Downarrow_{\mathcal N} x$ if there is $Q$ such that 
$P \wred Q$ and $Q \downarrow_{\mathcal N} x$.
\end{definition}

\begin{definition}
%\label{def.bbisim}
An  ${\mathcal N}$-\emph{barbed bisimulation} over a set of names, ${\mathcal N}$, is a symmetric binary relation 
${\mathcal S}_{\mathcal N}$ between agents such that $P\rel{S}_{\mathcal N}Q$ implies:
\begin{enumerate}
\item If $P \red P'$ then $Q \wred Q'$ and $P'\rel{S}_{\mathcal N} Q'$.
\item If $P\downarrow_{\mathcal N} x$, then $Q\Downarrow_{\mathcal N} x$.
\end{enumerate}
$P$ is ${\mathcal N}$-barbed bisimilar to $Q$, written
$P \wbbisim_{\mathcal N} Q$, if $P \rel{S}_{\mathcal N} Q$ for some ${\mathcal N}$-barbed bisimulation ${\mathcal S}_{\mathcal N}$.
\end{definition}

$\mathcal{R} \subseteq \pi \times \pi$

$P \mathcal{R} Q => \forall P'. P \red P' \Rightarrow \exists Q'. Q \red Q', P' \mathcal{R} Q'$

$P \vdash x \Rightarrow Q \vdash x$

\begin{mathpar}
  \inferrule*[lab=Out-barb]{x \nameeq y}{{y}!\langle{Q}\rangle \vdash x}
  \and
  \inferrule*[lab=Par-barb]{\mbox{$P\vdash x$ or $Q\vdash x$}}{\binpar{P}{Q} \vdash x}
\end{mathpar}

\subsubsection{Contexts}

One of the principle advantages of computational calculi like the
$\pi$-calculus is a well-defined notion of context,
contextual-equivalence and a correlation between
contextual-equivalence and notions of bisimulation. The notion of
context allows the decomposition of a process into (sub-)process and
its syntactic environment, its context. Thus, a context may be
thought of as a process with a ``hole'' (written $\Box$) in it. The
application of a context $M$ to a process $P$, written $M[P]$, is
tantamount to filling the hole in $M$ with $P$. In this paper we do
not need the full weight of this theory, but do make use of the notion
of context in the proof the main theorem. 

\begin{mathpar}
  \inferrule* [lab=summation] {} {{M_{M},M_{N}} \bc \Box \;|\; x.M_{A} \;|\; M_{M}+M_{N}}
  \and
  \inferrule* [lab=agent] {} {{M_{A}} \bc (\vec{x})M_{P} \;| \; \clift{P_0,\ldots,M_{P},\ldots,P_N}}
  \and \\
  \inferrule* [lab=process] {} {{M_{P}} \bc M_{N} \;| \;P|M_{P} }
\end{mathpar} 

\begin{mathpar}
  \inferrule* [lab=sychronization] {} {M_{N} \bc \Box \;|\; x?M_{F} \;|\; x!M_{C}}
  \and
  \inferrule* [lab=abstraction] {} {{M_{F}} \bc (x)M_{P} }
  \and
  \inferrule* [lab=concretion] {} {{M_{C}} \bc \langle M_{P} \rangle }
  \and \\
  \inferrule* [lab=process] {} {{M_{P}} \bc M_{N} \;| \;P|M_{P} }
\end{mathpar}

\begin{definition}[contextual application] Given a context $M$, and
  process $P$, we define the \emph{contextual application}, $M[P] :=
  M\{P/\Box\}$. That is, the contextual application of M to P is the
  substitution of $P$ for $\Box$ in $M$.
\end{definition}

$\meaningof{-} : L \to \mathcal{P}(\pi)$

\begin{mathpar}
  \inferrule* [lab=collection] {} {\meaningof{true} = \pi, \and \meaningof{~E} = \pi \setminus \meaningof{E}, \and \meaningof{E_{1} \& E_{2}} = \meaningof{E_{1}} \cap \meaningof{E_{2}}}
\end{mathpar}

\begin{mathpar}
  \inferrule* [lab=structure] {} {\meaningof{0} = \{ P \in \pi | P \equiv 0 \}, \and \\ \meaningof{E_1 | E_2} = \{ P \in \pi | P \equiv P_{1} | P_{2}, P_{1} \in \meaningof{E_{1}}, P_{2} \in \meaningof{E_2}\} }
\end{mathpar}

\begin{mathpar}
 \inferrule* [lab=behavior] {} {\meaningof{\langle a?b \rangle E} = \{ P \in \pi | P \equiv Q | u?(y)P', \\ \and \\\\ \and \\ \;\;\; u \in \meaningof{a}, \forall z.P'\{z/y\} \in \meaningof{E\{z/b\}}\}, \and \\ \meaningof{a!E} = \{ P \in \pi | P \equiv Q | x!\langle P' \rangle, x \in \meaningof{a} P' \in \meaningof{E}\} }
\end{mathpar}

\begin{mathpar}
 \inferrule* [lab=nominal] {} {\meaningof{\quotep{E}} = \{ \quotep{P} \in \quotep{\pi} | P \in \meaningof{E} \}, \and \meaningof{\quotep{P}} = \{ \quotep{Q} \in \quotep{\pi} | P \equiv Q \} \and \\ \meaningof{@\quotep{E}} = \{ P \in \pi | P \equiv @x, x \in \meaningof{E} \}}
\end{mathpar}

\begin{eqnarray*}
  \\
  \meaningof{-} : TS \to ST
\end{eqnarray*}

\begin{eqnarray*}
  \\
  L : TS \to ST
\end{eqnarray*}

\begin{eqnarray*}
  \\
  P \models E \iff P \in \meaningof{E}
\end{eqnarray*}

\begin{eqnarray*}
  P \approx_{L} Q \iff \forall E \in L. P \models E \iff Q \models E
\end{eqnarray*}

\begin{eqnarray*}
  P \approx_{K} Q
\end{eqnarray*}

\begin{eqnarray*}
  P \approx Q
\end{eqnarray*}

$\approx_{K} = \approx = \approx_{L}$

\subsubsection{Contextual duality}

Note that contexts extend the quotation operation to a family of
operations from processes to names. Given a context, $M$, we can
define a \emph{nominal context}, $\quotep{M}$ by $\quotep{M}[P] :=
\quotep{M[P]}$. To foreshadow what is to come we observe that these
operations enjoy a duality with processes very much like the duality
between vectors and maps from vectors to scalars.

Further, because the calculus is essentially higher-order, we have a
correspondence between contexts and processes. More specifically,
given a name $x$ and a context $M$ we can construct $M^{*}_{x}$ such
that 

\begin{mathpar}
  M^{*}_{x} | \lift{x}{P} \red M[P]
\end{mathpar}

namely,

\begin{mathpar}
  M^{*}_{x} := x?(u).M[\dropn{u}]
\end{mathpar}

The dependence of $M^{*}_{x}$ on a name makes it an abstraction, 

\begin{mathpar}
  M^{*} := (x)x?(u).M[\dropn{u}]
\end{mathpar}

\subsection{Additional notation}

It will sometimes be convenient to denote the process a name
quotes. We already have the notation $x = \quotep{P}$, but it will be
convenient to introduce an alternate notation, $\procn{x}$, when we
want to emphasize the connection to the use of the name. Note that, by
virtue of name equivalence, $\quotep{\procn{x}} \nameeq x$; so, the
notation is consistent with previous definitions.

Further, because names have structure it is possible to effect
substitutions on the basis of that structure. This means we need to
upgrade our notation for substitutions, which we accomplish by
adapting comprehension notation. Thus,

\begin{mathpar}
  P\{ y / x : x \in S \}
\end{mathpar}

is interpreted to mean the process derived from P by replacing (in a
capture-avoiding manner) each occurrence of $x$ in $S$ by $y$. For example,

\begin{mathpar}
  P\{ \quotep{\procn{x}|\procn{x}} / x : x \in \freenames{P} \}
\end{mathpar}

will replace each (occurrence) of a free name $x$ in $P$ by
$\quotep{\procn{x}|\procn{x}}$.

Also, we will avail ourselves of the notation $x^{L}$ and $x^{R}$ to
denote injections of a name into disjoint copies of the name
space. There are numerous ways to accomplish this. One example can be
found in \cite{MeredithR05}. This notation overloads to vectors of
names: $\vec{x}^{\pi} := (x_{i}^{\pi} \; : \; 0 \leq i < |\vec{x}| )$ where $\pi \in \{L,R\}$.

We also use $P^{\Box} := P|\Box$.

In \cite{MeredithR05} an interpretation of the new operator is
given. It turns out that there are several possible interpretations
all enjoying the requisite algebraic properties of the operator (see
\cite{milner91polyadicpi}). We will therefore make liberal use of
$(\nu\; \vec{x})P$.

% subsection the_syntax_and_semantics_of_the_notation_system (end)   

\input{qm2pi.qmops} 

\input{qm2pi.sterngerlach} 

\input{qm2pi.metric} 

% section concurrent_process_calculi (end)

%\input{qm2pi.proofsketch}

% section proof sketch (end)

%\input{qm2pi.slviaknots} 

% section spatial logic via knots (end)

\input{qm2pi.conclusion}

% section conclusion (end)

%\input{qm2pi.dtcodes} 

% section wiring algorithm (end)

\input{qm2pi.ack} 

% section acknowledgments (end)

\newpage


\bibliographystyle{plain}   
\bibliography{../../biblios/main.bib}

\input{qm2pi.rhodetails}

\end{document}

 

\documentclass[12pt]{llncs}
%\documentclass{jktr}

\usepackage[pdftex]{hyperref}                   
\usepackage {listings}
\usepackage {mathpartir}
\usepackage{bcprules}
%\usepackage{listings}
                       
\usepackage{graphicx} 
%\usepackage[margins=2.5cm,nohead,nofoot]{geometry}
%\usepackage{geometry}
\usepackage{amsfonts}
\usepackage{amstext}
\usepackage{latexsym}
\usepackage{amssymb}
\usepackage{color}


%\include{myPreamble}
\include{qm2pi.local} 

%\ifpdf
%\usepackage[pdftex]{graphicx}
%\else
%\usepackage{graphicx}
%\fi

 % \ifpdf
%  \usepackage{pdfsync}
%  \if


%\title{Brief Article}
%\author{David F. Snyder}
%\author{L.G. Meredith}

%\address{Dept. of Math., Texas State University--San Marcos, San Marcos, TX 78666}
       
\pagestyle{empty}


\begin{document}

\lstset{language=[Objective]Caml,frame=shadowbox}

\input{qm2pi.front}

% section front matter (end)

\input{qm2pi.intro} 
 
% section introduction (end)

% \input{qm2pi.knotations} 

% section notation (end)

\input{qm2pi.process.calculi} 

% section concurrent_process_calculi_and_spatial_logics_ (end)
    
%\input{qm2pi.knots2pi} 

%\input{qm2pi.trefoil} 

%\input{qm2pi.mainthm} 

% subsection basic_interpretation (end)

%\input{qm2pi.rho.presentation} 
\subsection{The syntax and semantics of the notation system}\label{sub:the_syntax_and_semantics_of_the_notation_system} % (fold)

We now summarize a technical presentation of the calculus that
embodies our theory of dynamics. The typical presentation of such a
calculus follows the style of giving generators and relations on
them. The grammar, below, describing term constructors, freely
generates the set of processes, $\Proc$. This set is then quotiented
by a relation known as structural congruence and it is over this set
that the notion of dynamics is expressed. This presentation is
essentially that of \cite{MeredithR05} with the addition of
polyadicity and summation. For readability we have relegated some of
the technical subtleties to an appendix.

\subsubsection{Process grammar}\label{subsub:process_grammar}

\begin{mathpar}
  \inferrule* [lab=synchronization] {} {{M} \bc \pzero \;|\; x?F \;|\; x!C }
  \and
  \inferrule* [lab=abstraction] {} {{F} \bc (x)P}
  \and
  \inferrule* [lab=concretion] {} {{C} \bc \langle Q \rangle}
  \and
  \inferrule* [lab=process] {} {{P,Q} \bc M \;| \;P|Q \;|\; @{x}}
  \and
  \inferrule* [lab=name] {} {{x} \bc \quotep{P}}
\end{mathpar} 

Note that $\vec{x}$ (resp. $\vec{P}$) denotes a vector of names
(resp. processes) of length $|\vec{x}|$ (resp. $|\vec{P}|$). We adopt
the following useful abbreviations.

\begin{mathpar}
   x?(\vec{y}).P := x.(\vec{y})P \and  x\clift{\vec{P}} := x.\clift{\vec{P}}
   \and x!(y) := \lift{x}{\dropn{y}}
   \and \Pi_{i=0}^{n-1}P_i := P_0 | \ldots | P_{n-1}
\end{mathpar}

\subsubsection{Structural congruence}

\paragraph{Free and bound names and alpha-equivalence.} At the
core of structural equivalence is alpha-equivalence which identifies
process that are the same up to a change of variable. Formally, we
recognize the distinction between free and bound names. The free names
of a process, $\freenames{P}$, may be calculated recursively as
follows:

\begin{mathpar}
\freenames{\pzero} := \emptyset
  \and \\
  \freenames{x?(y).P} := \{ x \} \cup (\freenames{P} \setminus \{ y \})
  \and 
  \freenames{x!\langle P \rangle} := \{ x \} \cup \{ P \} 
  \and \\
  \freenames{P|Q} := \freenames{P} \cup \freenames{Q}
  \and \\
  \freenames{@{x}} := \{ x \}
\end{mathpar}

$\pi$
$\quotep{\pi}$

$\freenames{-} : \pi \to \mathcal{P}(\quotep{\pi})$

\begin{eqnarray*}
  \freenames{\pzero} & := & \emptyset \\
  \freenames{x?(y).P} & := & \{ x \} \cup (\freenames{P} \setminus \{ y \}) \\
  \freenames{x!\langle P \rangle} & := & \{ x \} \cup \{ P \} \\
  \freenames{P|Q} & := & \freenames{P} \cup \freenames{Q} \\
  \freenames{\dropn{x}} & := & \{ x \}
\end{eqnarray*}

The bound names of a process, $\boundnames{P}$, are those names occurring in $P$
that are not free. For example, in $x?(y).0$, the name $x$ is free, while $y$ is bound.

\begin{mathpar}
  \inferrule* [lab=monoidal-laws] {} { P|Q \equiv Q|P \and P|0 \equiv P \and P|(Q|R) \equiv (P|Q)|R }
\end{mathpar}

\begin{mathpar}
  \inferrule* [lab=alpha-equivalence] {} { (x)P \equiv (y)P\{y/x\} \and y \not\in \freenames{P} }
\end{mathpar}

\begin{definition}
Then two processes, $P,Q$, are alpha-equivalent if $P = Q\{\vec{y}/\vec{x}\}$ for
some $\vec{x} \in \boundnames{Q},\vec{y} \in \boundnames{P}$, where $Q\{\vec{y}/\vec{x}\}$
denotes the capture-avoiding substitution of $\vec{y}$ for $\vec{x}$ in $Q$.
\end{definition}

\begin{definition}
  The {\em structural congruence} \cite{SangiorgiWalker} , $\equiv$,
  between processes is the least congruence containing
  alpha-equivalence, satisfying the abelian monoid laws
  (associativity, commutativity and $\pzero$ as identity) for parallel
  composition $|$ and for summation $+$.
\end{definition}

\subsection{Name equivalence}

We take name equivalence, written $\nameeq$, to be the smallest
equivalence relation generated by the following rules.

\begin{mathpar}
\inferrule*[lab=Quote-drop]
{ }
{ \quotep{@{x}} \nameeq x }

\inferrule*[lab=Struct-equiv]
{ P \scong Q }
{ \quotep{P} \nameeq \quotep{Q} }
\end{mathpar}

The astute reader will have noticed that the mutual recursion of names
and processes imposes a mutual recursion on alpha-equivalence and
structural equivalence via name-equivalence. Fortunately, all of this
works out pleasantly and we may calculate in the natural way, free of
concern. The reader interested in the details is referred to the
appendix \ref{appendix:rho_details}.

\subsection{Substitution}

We use $\Proc$ for the set of processes, $\QProc$ for the set of
names, and $\id{\{}\vec{y} / \vec{x} \id{\}}$ to denote partial maps,
$s : \QProc \rightarrow \QProc$. A map, $s$ lifts, uniquely, to a map
on process terms, $\widehat{s} : \Proc \rightarrow \Proc$ by the
following equations.

\begin{mathpar}
  (0) \psubstp{Q}{P} := 0 \\
  (R \juxtap S) \psubstp{Q}{P}
  :=    
  (R)\psubstp{Q}{P} \juxtap (S) \psubstp{Q}{P} \\
  (x?(y).R) \psubstp{Q}{P}    
  :=    
  (x)\substp{Q}{P} (z)\concat( (R \psubstn{z}{y}) \psubstp{Q}{P} ) \\
  (\lift{x}{R}) \psubstp{Q}{P}  
  :=
  \lift{(x)\substp{Q}{P}}{ R \psubstp{Q}{P} } \\
%   (\dropn{x})  \psubstp{Q}{P}       
%   := 
%   \left\{ 
%     \begin{array}{ccc} 
%       \dropn{\quotep{Q}} & & x \nameeq \quotep{P} \\
%       \dropn{x} & & otherwise \\
%     \end{array}
%   \right. 
  (\dropn{x})  \psubstp{Q}{P}       
  := 
  \left\{ 
    \begin{array}{ccc} 
      Q & & x \nameeq \quotep{P} \\
      \dropn{x} & & otherwise \\
    \end{array}
  \right.
\end{mathpar}
 

where

\begin{eqnarray}
  (x)\id{\{} \lpquote Q \rpquote / \lpquote P \rpquote \id{\}}            = 
  \left\{ 
    \begin{array}{ccc}
      \lpquote Q \rpquote & & x \nameeq \lpquote P \rpquote \\
      x & & otherwise \\
    \end{array}
  \right. \nonumber
\end{eqnarray}

and $z$ is chosen distinct from $\quotep{P}$, $\quotep{Q}$, the free
names in $Q$, and all the names in $R$. Our $\alpha$-equivalence will
be built in the standard way from this substitution.

\begin{remark}\label{rem:no_self_referential_names}
  One consequence of these definitions is that $\forall P. \quotep{P}
  \not\in \freenames{P}$.
\end{remark}

\subsection{ Dynamic quote: an example }

Anticipating something of what's to come, consider applying the
substitution, $\widehat{\id{\{}u / z \id{\}}}$, to the following pair
of processes, $\lift{w}{y!(z)}$ and $w[ \lpquote y!(z) \rpquote ]$.

\begin{eqnarray}
	\lift{w}{y!(z)}\widehat{\id{\{}u / z \id{\}}}
		& = &
		\lift{w}{y!(u)} \nonumber\\
	w[ \lpquote y!(z) \rpquote ] \widehat{ \id{\{}u / z \id{\}} }
		& = &
		w[ \lpquote y!(z) \rpquote ] \nonumber
\end{eqnarray}

Because the body of the process between quotes is impervious to
substitution, we get radically different answers. In fact, by
examining the first process in an input context,
e.g. $x?(z).\lift{w}{y!(z)}$, we see that the process under the lift
operator may be shaped by prefixed inputs binding a name inside it. In
this sense, the lift operator will be seen as a way to dynamically
construct processes before reifying them as names.

Finally equipped with these standard features we can present the
dynamics of the calculus.

\subsubsection{Operational semantics} 

Finally, we introduce the computational dynamics. What marks these
algebras as distinct from other more traditionally studied algebraic
structures, e.g. vector spaces or polynomial rings, is the manner in
which dynamics is captured. In traditional structures, dynamics is typically
expressed through morphisms between such structures, as in linear maps
between vector spaces or morphisms between rings. In algebras
associated with the semantics of computation, the dynamics is
expressed as part of the algebraic structure itself, through a
reduction reduction relation typically denoted by $\red$. Below, we
give a recursive presentation of this relation for the calculus used
in the encoding.

$\red \subseteq \pi \times \pi$
$\red : \pi \to \mathcal{P}(\pi)$

\begin{mathpar}
  \inferrule* [lab=Comm] { \textsf{match}( x_{src}, x_{trgt} ) } { x_{trgt}?(y)P \; | \; x_{src}!\langle {Q} \rangle \red P\{\quotep{Q}/y}\} }
  \and \\
  \inferrule* [lab=Par] {{P} \red {P}'} {{{P} | {Q}} \red {{P}' | {Q}}}
  \and
  \inferrule* [lab=Equiv]{{{P} \scong {P}'} \andalso {{P}' \red {Q}'} \andalso {{Q}' \scong {Q}}}{{P} \red {Q}}
\end{mathpar}

\begin{eqnarray*}
  match_{\equiv} (\quotep{P},\quotep{Q}) & := & P \equiv Q \\
  match_{\dagger}(\quotep{P},\quotep{Q}) & := & \forall R. P|Q \red^{*} R => R \red^{*} 0 \\
  match_{K}(\quotep{P},\quotep{Q}) & := & K \mbox{ for some context } K
\end{eqnarray*}

$u?(x)P | u!\langle Q \rangle \red P\{\quotep{Q}/x\}$

%We write $\wred$ for $\red^*$, and $P\red$ if $\exists Q $ such that $ P \red Q$.
We write $P\red$ if $\exists Q $ such that $ P \red Q$ and $P\not\red$, otherwise.

\section{Replication}

As mentioned before, it is known that replication (and hence
recursion) can be implemented in a higher-order process algebra
\cite{SangiorgiWalker}. As our first example of calculation with the
machinery thus far presented we give the construction explicitly in
the {\rhoc}.

\begin{eqnarray}
	D_{x} & := & \prefix{x}{y}{(\binpar{\outputp{x}{y}}{@{y}})} \nonumber\\
	\bangp_{x}{P} & := & \binpar{{x}!\langle{\binpar{D_{x}}{P}}\rangle}{D_{x}} \nonumber
\end{eqnarray}

\begin{eqnarray}
	\bangp_{x}{P} & & \nonumber\\
	=
	& {x}!\langle{(\prefix{x}{y}{(\outputp{x}{y} | @{y})) | P}}\rangle 
	      | \prefix{x}{y}{(\outputp{x}{y} | @{y})} & \nonumber\\
	\red
	& (\outputp{x}{y} | @{y})\substn{\quotep{(\prefix{x}{y}{(@{y} | \outputp{x}{y})) | P}}}{y} & \nonumber\\
	=
	& \outputp{x}{\quotep{(\prefix{x}{y}{(\outputp{x}{y} | @{y})) | P}}}
	  | {(\prefix{x}{y}{(\outputp{x}{y} | @{y})) | P}} & \nonumber\\
	\red
	& \ldots & \nonumber\\
	\red^*
	& P | P | \ldots & \nonumber
\end{eqnarray}

Of course, this encoding, as an implementation, runs away, unfolding
$\bangp{P}$ eagerly. A lazier and more implementable replication
operator, restricted to input-guarded processes, may be obtained as follows.

\begin{eqnarray}
\bangp{\prefix{u}{v}{P}} 
	:= 
	\binpar{\lift{x}{\prefix{u}{v}{(\binpar{D(x)}{P})}}}{D(x)} \nonumber
\end{eqnarray}

\begin{remark}
  Note that the lazier definition still does not deal with summation
  or mixed summation (i.e. sums over input and output). The reader is
  invited to construct definitions of replication that deal with these
  features. 

  Further, the definitions are parameterized in a name, $x$. Can you,
  gentle reader, make a definition that eliminates this parameter and
  guarantees no accidental interaction between the replication
  machinery and the process being replicated -- i.e. no accidental
  sharing of names used by the process to get its work done and the
  name(s) used by the replication to effect copying. This latter
  revision of the definition of replication is crucial to obtaining
  the expected identity $!!P \sim !P$.
\end{remark}

\begin{remark}\label{rem:paradoxical_combinator}
  The reader familiar with the lambda calculus will have noticed the
  similarity between $D$ and the paradoxical combinator.

  [Ed. note: the existence of this seems to suggest we have to be more
  restrictive on the set of processes and names we admit if we are to
  support no-cloning.]
\end{remark}

\subsubsection{Bisimulation}

The computational dynamics gives rise to another kind of equivalence,
the equivalence of computational behavior. As previously mentioned
this is typically captured \emph{via} some form of bisimulation.

% The notion we use in this paper is weak barbed bisimulation
% \cite{milner91polyadicpi}.

The notion we use in this paper is derived from weak barbed
bisimulation \cite{milner91polyadicpi}. 

\begin{definition}
An \emph{observation relation}, $\downarrow_{\mathcal N}$, over a set
of names, $\mathcal N$, is the smallest relation satisfying the rules
below.

\infrule[Out-barb]{y \in {\mathcal N}, \; x \nameeq y}
		  {\outputp{x}{v} \downarrow_{\mathcal N} x}
\infrule[Par-barb]{\mbox{$P\downarrow_{\mathcal N} x$ or $Q\downarrow_{\mathcal N} x$}}
		  {\binpar{P}{Q} \downarrow_{\mathcal N} x}

We write $P \Downarrow_{\mathcal N} x$ if there is $Q$ such that 
$P \wred Q$ and $Q \downarrow_{\mathcal N} x$.
\end{definition}

\begin{definition}
%\label{def.bbisim}
An  ${\mathcal N}$-\emph{barbed bisimulation} over a set of names, ${\mathcal N}$, is a symmetric binary relation 
${\mathcal S}_{\mathcal N}$ between agents such that $P\rel{S}_{\mathcal N}Q$ implies:
\begin{enumerate}
\item If $P \red P'$ then $Q \wred Q'$ and $P'\rel{S}_{\mathcal N} Q'$.
\item If $P\downarrow_{\mathcal N} x$, then $Q\Downarrow_{\mathcal N} x$.
\end{enumerate}
$P$ is ${\mathcal N}$-barbed bisimilar to $Q$, written
$P \wbbisim_{\mathcal N} Q$, if $P \rel{S}_{\mathcal N} Q$ for some ${\mathcal N}$-barbed bisimulation ${\mathcal S}_{\mathcal N}$.
\end{definition}

$\mathcal{R} \subseteq \pi \times \pi$

$P \mathcal{R} Q => \forall P'. P \red P' \Rightarrow \exists Q'. Q \red Q', P' \mathcal{R} Q'$

$P \vdash x \Rightarrow Q \vdash x$

\begin{mathpar}
  \inferrule*[lab=Out-barb]{x \nameeq y}{{y}!\langle{Q}\rangle \vdash x}
  \and
  \inferrule*[lab=Par-barb]{\mbox{$P\vdash x$ or $Q\vdash x$}}{\binpar{P}{Q} \vdash x}
\end{mathpar}

\subsubsection{Contexts}

One of the principle advantages of computational calculi like the
$\pi$-calculus is a well-defined notion of context,
contextual-equivalence and a correlation between
contextual-equivalence and notions of bisimulation. The notion of
context allows the decomposition of a process into (sub-)process and
its syntactic environment, its context. Thus, a context may be
thought of as a process with a ``hole'' (written $\Box$) in it. The
application of a context $M$ to a process $P$, written $M[P]$, is
tantamount to filling the hole in $M$ with $P$. In this paper we do
not need the full weight of this theory, but do make use of the notion
of context in the proof the main theorem. 

\begin{mathpar}
  \inferrule* [lab=summation] {} {{M_{M},M_{N}} \bc \Box \;|\; x.M_{A} \;|\; M_{M}+M_{N}}
  \and
  \inferrule* [lab=agent] {} {{M_{A}} \bc (\vec{x})M_{P} \;| \; \clift{P_0,\ldots,M_{P},\ldots,P_N}}
  \and \\
  \inferrule* [lab=process] {} {{M_{P}} \bc M_{N} \;| \;P|M_{P} }
\end{mathpar} 

\begin{mathpar}
  \inferrule* [lab=sychronization] {} {M_{N} \bc \Box \;|\; x?M_{F} \;|\; x!M_{C}}
  \and
  \inferrule* [lab=abstraction] {} {{M_{F}} \bc (x)M_{P} }
  \and
  \inferrule* [lab=concretion] {} {{M_{C}} \bc \langle M_{P} \rangle }
  \and \\
  \inferrule* [lab=process] {} {{M_{P}} \bc M_{N} \;| \;P|M_{P} }
\end{mathpar}

\begin{definition}[contextual application] Given a context $M$, and
  process $P$, we define the \emph{contextual application}, $M[P] :=
  M\{P/\Box\}$. That is, the contextual application of M to P is the
  substitution of $P$ for $\Box$ in $M$.
\end{definition}

$\meaningof{-} : L \to \mathcal{P}(\pi)$

\begin{mathpar}
  \inferrule* [lab=collection] {} {\meaningof{true} = \pi, \and \meaningof{~E} = \pi \setminus \meaningof{E}, \and \meaningof{E_{1} \& E_{2}} = \meaningof{E_{1}} \cap \meaningof{E_{2}}}
\end{mathpar}

\begin{mathpar}
  \inferrule* [lab=structure] {} {\meaningof{0} = \{ P \in \pi | P \equiv 0 \}, \and \\ \meaningof{E_1 | E_2} = \{ P \in \pi | P \equiv P_{1} | P_{2}, P_{1} \in \meaningof{E_{1}}, P_{2} \in \meaningof{E_2}\} }
\end{mathpar}

\begin{mathpar}
 \inferrule* [lab=behavior] {} {\meaningof{\langle a?b \rangle E} = \{ P \in \pi | P \equiv Q | u?(y)P', \\ \and \\\\ \and \\ \;\;\; u \in \meaningof{a}, \forall z.P'\{z/y\} \in \meaningof{E\{z/b\}}\}, \and \\ \meaningof{a!E} = \{ P \in \pi | P \equiv Q | x!\langle P' \rangle, x \in \meaningof{a} P' \in \meaningof{E}\} }
\end{mathpar}

\begin{mathpar}
 \inferrule* [lab=nominal] {} {\meaningof{\quotep{E}} = \{ \quotep{P} \in \quotep{\pi} | P \in \meaningof{E} \}, \and \meaningof{\quotep{P}} = \{ \quotep{Q} \in \quotep{\pi} | P \equiv Q \} \and \\ \meaningof{@\quotep{E}} = \{ P \in \pi | P \equiv @x, x \in \meaningof{E} \}}
\end{mathpar}

\begin{eqnarray*}
  \\
  \meaningof{-} : TS \to ST
\end{eqnarray*}

\begin{eqnarray*}
  \\
  L : TS \to ST
\end{eqnarray*}

\begin{eqnarray*}
  \\
  P \models E \iff P \in \meaningof{E}
\end{eqnarray*}

\begin{eqnarray*}
  P \approx_{L} Q \iff \forall E \in L. P \models E \iff Q \models E
\end{eqnarray*}

\begin{eqnarray*}
  P \approx_{K} Q
\end{eqnarray*}

\begin{eqnarray*}
  P \approx Q
\end{eqnarray*}

$\approx_{K} = \approx = \approx_{L}$

\subsubsection{Contextual duality}

Note that contexts extend the quotation operation to a family of
operations from processes to names. Given a context, $M$, we can
define a \emph{nominal context}, $\quotep{M}$ by $\quotep{M}[P] :=
\quotep{M[P]}$. To foreshadow what is to come we observe that these
operations enjoy a duality with processes very much like the duality
between vectors and maps from vectors to scalars.

Further, because the calculus is essentially higher-order, we have a
correspondence between contexts and processes. More specifically,
given a name $x$ and a context $M$ we can construct $M^{*}_{x}$ such
that 

\begin{mathpar}
  M^{*}_{x} | \lift{x}{P} \red M[P]
\end{mathpar}

namely,

\begin{mathpar}
  M^{*}_{x} := x?(u).M[\dropn{u}]
\end{mathpar}

The dependence of $M^{*}_{x}$ on a name makes it an abstraction, 

\begin{mathpar}
  M^{*} := (x)x?(u).M[\dropn{u}]
\end{mathpar}

\subsection{Additional notation}

It will sometimes be convenient to denote the process a name
quotes. We already have the notation $x = \quotep{P}$, but it will be
convenient to introduce an alternate notation, $\procn{x}$, when we
want to emphasize the connection to the use of the name. Note that, by
virtue of name equivalence, $\quotep{\procn{x}} \nameeq x$; so, the
notation is consistent with previous definitions.

Further, because names have structure it is possible to effect
substitutions on the basis of that structure. This means we need to
upgrade our notation for substitutions, which we accomplish by
adapting comprehension notation. Thus,

\begin{mathpar}
  P\{ y / x : x \in S \}
\end{mathpar}

is interpreted to mean the process derived from P by replacing (in a
capture-avoiding manner) each occurrence of $x$ in $S$ by $y$. For example,

\begin{mathpar}
  P\{ \quotep{\procn{x}|\procn{x}} / x : x \in \freenames{P} \}
\end{mathpar}

will replace each (occurrence) of a free name $x$ in $P$ by
$\quotep{\procn{x}|\procn{x}}$.

Also, we will avail ourselves of the notation $x^{L}$ and $x^{R}$ to
denote injections of a name into disjoint copies of the name
space. There are numerous ways to accomplish this. One example can be
found in \cite{MeredithR05}. This notation overloads to vectors of
names: $\vec{x}^{\pi} := (x_{i}^{\pi} \; : \; 0 \leq i < |\vec{x}| )$ where $\pi \in \{L,R\}$.

We also use $P^{\Box} := P|\Box$.

In \cite{MeredithR05} an interpretation of the new operator is
given. It turns out that there are several possible interpretations
all enjoying the requisite algebraic properties of the operator (see
\cite{milner91polyadicpi}). We will therefore make liberal use of
$(\nu\; \vec{x})P$.

% subsection the_syntax_and_semantics_of_the_notation_system (end)   

\input{qm2pi.qmops} 

\input{qm2pi.sterngerlach} 

\input{qm2pi.metric} 

% section concurrent_process_calculi (end)

%\input{qm2pi.proofsketch}

% section proof sketch (end)

%\input{qm2pi.slviaknots} 

% section spatial logic via knots (end)

\input{qm2pi.conclusion}

% section conclusion (end)

%\input{qm2pi.dtcodes} 

% section wiring algorithm (end)

\input{qm2pi.ack} 

% section acknowledgments (end)

\newpage


\bibliographystyle{plain}   
\bibliography{../../biblios/main.bib}

\input{qm2pi.rhodetails}

\end{document}

 

% section concurrent_process_calculi (end)

%\documentclass[12pt]{llncs}
%\documentclass{jktr}

\usepackage[pdftex]{hyperref}                   
\usepackage {listings}
\usepackage {mathpartir}
\usepackage{bcprules}
%\usepackage{listings}
                       
\usepackage{graphicx} 
%\usepackage[margins=2.5cm,nohead,nofoot]{geometry}
%\usepackage{geometry}
\usepackage{amsfonts}
\usepackage{amstext}
\usepackage{latexsym}
\usepackage{amssymb}
\usepackage{color}


%\include{myPreamble}
\include{qm2pi.local} 

%\ifpdf
%\usepackage[pdftex]{graphicx}
%\else
%\usepackage{graphicx}
%\fi

 % \ifpdf
%  \usepackage{pdfsync}
%  \if


%\title{Brief Article}
%\author{David F. Snyder}
%\author{L.G. Meredith}

%\address{Dept. of Math., Texas State University--San Marcos, San Marcos, TX 78666}
       
\pagestyle{empty}


\begin{document}

\lstset{language=[Objective]Caml,frame=shadowbox}

\input{qm2pi.front}

% section front matter (end)

\input{qm2pi.intro} 
 
% section introduction (end)

% \input{qm2pi.knotations} 

% section notation (end)

\input{qm2pi.process.calculi} 

% section concurrent_process_calculi_and_spatial_logics_ (end)
    
%\input{qm2pi.knots2pi} 

%\input{qm2pi.trefoil} 

%\input{qm2pi.mainthm} 

% subsection basic_interpretation (end)

%\input{qm2pi.rho.presentation} 
\subsection{The syntax and semantics of the notation system}\label{sub:the_syntax_and_semantics_of_the_notation_system} % (fold)

We now summarize a technical presentation of the calculus that
embodies our theory of dynamics. The typical presentation of such a
calculus follows the style of giving generators and relations on
them. The grammar, below, describing term constructors, freely
generates the set of processes, $\Proc$. This set is then quotiented
by a relation known as structural congruence and it is over this set
that the notion of dynamics is expressed. This presentation is
essentially that of \cite{MeredithR05} with the addition of
polyadicity and summation. For readability we have relegated some of
the technical subtleties to an appendix.

\subsubsection{Process grammar}\label{subsub:process_grammar}

\begin{mathpar}
  \inferrule* [lab=synchronization] {} {{M} \bc \pzero \;|\; x?F \;|\; x!C }
  \and
  \inferrule* [lab=abstraction] {} {{F} \bc (x)P}
  \and
  \inferrule* [lab=concretion] {} {{C} \bc \langle Q \rangle}
  \and
  \inferrule* [lab=process] {} {{P,Q} \bc M \;| \;P|Q \;|\; @{x}}
  \and
  \inferrule* [lab=name] {} {{x} \bc \quotep{P}}
\end{mathpar} 

Note that $\vec{x}$ (resp. $\vec{P}$) denotes a vector of names
(resp. processes) of length $|\vec{x}|$ (resp. $|\vec{P}|$). We adopt
the following useful abbreviations.

\begin{mathpar}
   x?(\vec{y}).P := x.(\vec{y})P \and  x\clift{\vec{P}} := x.\clift{\vec{P}}
   \and x!(y) := \lift{x}{\dropn{y}}
   \and \Pi_{i=0}^{n-1}P_i := P_0 | \ldots | P_{n-1}
\end{mathpar}

\subsubsection{Structural congruence}

\paragraph{Free and bound names and alpha-equivalence.} At the
core of structural equivalence is alpha-equivalence which identifies
process that are the same up to a change of variable. Formally, we
recognize the distinction between free and bound names. The free names
of a process, $\freenames{P}$, may be calculated recursively as
follows:

\begin{mathpar}
\freenames{\pzero} := \emptyset
  \and \\
  \freenames{x?(y).P} := \{ x \} \cup (\freenames{P} \setminus \{ y \})
  \and 
  \freenames{x!\langle P \rangle} := \{ x \} \cup \{ P \} 
  \and \\
  \freenames{P|Q} := \freenames{P} \cup \freenames{Q}
  \and \\
  \freenames{@{x}} := \{ x \}
\end{mathpar}

$\pi$
$\quotep{\pi}$

$\freenames{-} : \pi \to \mathcal{P}(\quotep{\pi})$

\begin{eqnarray*}
  \freenames{\pzero} & := & \emptyset \\
  \freenames{x?(y).P} & := & \{ x \} \cup (\freenames{P} \setminus \{ y \}) \\
  \freenames{x!\langle P \rangle} & := & \{ x \} \cup \{ P \} \\
  \freenames{P|Q} & := & \freenames{P} \cup \freenames{Q} \\
  \freenames{\dropn{x}} & := & \{ x \}
\end{eqnarray*}

The bound names of a process, $\boundnames{P}$, are those names occurring in $P$
that are not free. For example, in $x?(y).0$, the name $x$ is free, while $y$ is bound.

\begin{mathpar}
  \inferrule* [lab=monoidal-laws] {} { P|Q \equiv Q|P \and P|0 \equiv P \and P|(Q|R) \equiv (P|Q)|R }
\end{mathpar}

\begin{mathpar}
  \inferrule* [lab=alpha-equivalence] {} { (x)P \equiv (y)P\{y/x\} \and y \not\in \freenames{P} }
\end{mathpar}

\begin{definition}
Then two processes, $P,Q$, are alpha-equivalent if $P = Q\{\vec{y}/\vec{x}\}$ for
some $\vec{x} \in \boundnames{Q},\vec{y} \in \boundnames{P}$, where $Q\{\vec{y}/\vec{x}\}$
denotes the capture-avoiding substitution of $\vec{y}$ for $\vec{x}$ in $Q$.
\end{definition}

\begin{definition}
  The {\em structural congruence} \cite{SangiorgiWalker} , $\equiv$,
  between processes is the least congruence containing
  alpha-equivalence, satisfying the abelian monoid laws
  (associativity, commutativity and $\pzero$ as identity) for parallel
  composition $|$ and for summation $+$.
\end{definition}

\subsection{Name equivalence}

We take name equivalence, written $\nameeq$, to be the smallest
equivalence relation generated by the following rules.

\begin{mathpar}
\inferrule*[lab=Quote-drop]
{ }
{ \quotep{@{x}} \nameeq x }

\inferrule*[lab=Struct-equiv]
{ P \scong Q }
{ \quotep{P} \nameeq \quotep{Q} }
\end{mathpar}

The astute reader will have noticed that the mutual recursion of names
and processes imposes a mutual recursion on alpha-equivalence and
structural equivalence via name-equivalence. Fortunately, all of this
works out pleasantly and we may calculate in the natural way, free of
concern. The reader interested in the details is referred to the
appendix \ref{appendix:rho_details}.

\subsection{Substitution}

We use $\Proc$ for the set of processes, $\QProc$ for the set of
names, and $\id{\{}\vec{y} / \vec{x} \id{\}}$ to denote partial maps,
$s : \QProc \rightarrow \QProc$. A map, $s$ lifts, uniquely, to a map
on process terms, $\widehat{s} : \Proc \rightarrow \Proc$ by the
following equations.

\begin{mathpar}
  (0) \psubstp{Q}{P} := 0 \\
  (R \juxtap S) \psubstp{Q}{P}
  :=    
  (R)\psubstp{Q}{P} \juxtap (S) \psubstp{Q}{P} \\
  (x?(y).R) \psubstp{Q}{P}    
  :=    
  (x)\substp{Q}{P} (z)\concat( (R \psubstn{z}{y}) \psubstp{Q}{P} ) \\
  (\lift{x}{R}) \psubstp{Q}{P}  
  :=
  \lift{(x)\substp{Q}{P}}{ R \psubstp{Q}{P} } \\
%   (\dropn{x})  \psubstp{Q}{P}       
%   := 
%   \left\{ 
%     \begin{array}{ccc} 
%       \dropn{\quotep{Q}} & & x \nameeq \quotep{P} \\
%       \dropn{x} & & otherwise \\
%     \end{array}
%   \right. 
  (\dropn{x})  \psubstp{Q}{P}       
  := 
  \left\{ 
    \begin{array}{ccc} 
      Q & & x \nameeq \quotep{P} \\
      \dropn{x} & & otherwise \\
    \end{array}
  \right.
\end{mathpar}
 

where

\begin{eqnarray}
  (x)\id{\{} \lpquote Q \rpquote / \lpquote P \rpquote \id{\}}            = 
  \left\{ 
    \begin{array}{ccc}
      \lpquote Q \rpquote & & x \nameeq \lpquote P \rpquote \\
      x & & otherwise \\
    \end{array}
  \right. \nonumber
\end{eqnarray}

and $z$ is chosen distinct from $\quotep{P}$, $\quotep{Q}$, the free
names in $Q$, and all the names in $R$. Our $\alpha$-equivalence will
be built in the standard way from this substitution.

\begin{remark}\label{rem:no_self_referential_names}
  One consequence of these definitions is that $\forall P. \quotep{P}
  \not\in \freenames{P}$.
\end{remark}

\subsection{ Dynamic quote: an example }

Anticipating something of what's to come, consider applying the
substitution, $\widehat{\id{\{}u / z \id{\}}}$, to the following pair
of processes, $\lift{w}{y!(z)}$ and $w[ \lpquote y!(z) \rpquote ]$.

\begin{eqnarray}
	\lift{w}{y!(z)}\widehat{\id{\{}u / z \id{\}}}
		& = &
		\lift{w}{y!(u)} \nonumber\\
	w[ \lpquote y!(z) \rpquote ] \widehat{ \id{\{}u / z \id{\}} }
		& = &
		w[ \lpquote y!(z) \rpquote ] \nonumber
\end{eqnarray}

Because the body of the process between quotes is impervious to
substitution, we get radically different answers. In fact, by
examining the first process in an input context,
e.g. $x?(z).\lift{w}{y!(z)}$, we see that the process under the lift
operator may be shaped by prefixed inputs binding a name inside it. In
this sense, the lift operator will be seen as a way to dynamically
construct processes before reifying them as names.

Finally equipped with these standard features we can present the
dynamics of the calculus.

\subsubsection{Operational semantics} 

Finally, we introduce the computational dynamics. What marks these
algebras as distinct from other more traditionally studied algebraic
structures, e.g. vector spaces or polynomial rings, is the manner in
which dynamics is captured. In traditional structures, dynamics is typically
expressed through morphisms between such structures, as in linear maps
between vector spaces or morphisms between rings. In algebras
associated with the semantics of computation, the dynamics is
expressed as part of the algebraic structure itself, through a
reduction reduction relation typically denoted by $\red$. Below, we
give a recursive presentation of this relation for the calculus used
in the encoding.

$\red \subseteq \pi \times \pi$
$\red : \pi \to \mathcal{P}(\pi)$

\begin{mathpar}
  \inferrule* [lab=Comm] { \textsf{match}( x_{src}, x_{trgt} ) } { x_{trgt}?(y)P \; | \; x_{src}!\langle {Q} \rangle \red P\{\quotep{Q}/y}\} }
  \and \\
  \inferrule* [lab=Par] {{P} \red {P}'} {{{P} | {Q}} \red {{P}' | {Q}}}
  \and
  \inferrule* [lab=Equiv]{{{P} \scong {P}'} \andalso {{P}' \red {Q}'} \andalso {{Q}' \scong {Q}}}{{P} \red {Q}}
\end{mathpar}

\begin{eqnarray*}
  match_{\equiv} (\quotep{P},\quotep{Q}) & := & P \equiv Q \\
  match_{\dagger}(\quotep{P},\quotep{Q}) & := & \forall R. P|Q \red^{*} R => R \red^{*} 0 \\
  match_{K}(\quotep{P},\quotep{Q}) & := & K \mbox{ for some context } K
\end{eqnarray*}

$u?(x)P | u!\langle Q \rangle \red P\{\quotep{Q}/x\}$

%We write $\wred$ for $\red^*$, and $P\red$ if $\exists Q $ such that $ P \red Q$.
We write $P\red$ if $\exists Q $ such that $ P \red Q$ and $P\not\red$, otherwise.

\section{Replication}

As mentioned before, it is known that replication (and hence
recursion) can be implemented in a higher-order process algebra
\cite{SangiorgiWalker}. As our first example of calculation with the
machinery thus far presented we give the construction explicitly in
the {\rhoc}.

\begin{eqnarray}
	D_{x} & := & \prefix{x}{y}{(\binpar{\outputp{x}{y}}{@{y}})} \nonumber\\
	\bangp_{x}{P} & := & \binpar{{x}!\langle{\binpar{D_{x}}{P}}\rangle}{D_{x}} \nonumber
\end{eqnarray}

\begin{eqnarray}
	\bangp_{x}{P} & & \nonumber\\
	=
	& {x}!\langle{(\prefix{x}{y}{(\outputp{x}{y} | @{y})) | P}}\rangle 
	      | \prefix{x}{y}{(\outputp{x}{y} | @{y})} & \nonumber\\
	\red
	& (\outputp{x}{y} | @{y})\substn{\quotep{(\prefix{x}{y}{(@{y} | \outputp{x}{y})) | P}}}{y} & \nonumber\\
	=
	& \outputp{x}{\quotep{(\prefix{x}{y}{(\outputp{x}{y} | @{y})) | P}}}
	  | {(\prefix{x}{y}{(\outputp{x}{y} | @{y})) | P}} & \nonumber\\
	\red
	& \ldots & \nonumber\\
	\red^*
	& P | P | \ldots & \nonumber
\end{eqnarray}

Of course, this encoding, as an implementation, runs away, unfolding
$\bangp{P}$ eagerly. A lazier and more implementable replication
operator, restricted to input-guarded processes, may be obtained as follows.

\begin{eqnarray}
\bangp{\prefix{u}{v}{P}} 
	:= 
	\binpar{\lift{x}{\prefix{u}{v}{(\binpar{D(x)}{P})}}}{D(x)} \nonumber
\end{eqnarray}

\begin{remark}
  Note that the lazier definition still does not deal with summation
  or mixed summation (i.e. sums over input and output). The reader is
  invited to construct definitions of replication that deal with these
  features. 

  Further, the definitions are parameterized in a name, $x$. Can you,
  gentle reader, make a definition that eliminates this parameter and
  guarantees no accidental interaction between the replication
  machinery and the process being replicated -- i.e. no accidental
  sharing of names used by the process to get its work done and the
  name(s) used by the replication to effect copying. This latter
  revision of the definition of replication is crucial to obtaining
  the expected identity $!!P \sim !P$.
\end{remark}

\begin{remark}\label{rem:paradoxical_combinator}
  The reader familiar with the lambda calculus will have noticed the
  similarity between $D$ and the paradoxical combinator.

  [Ed. note: the existence of this seems to suggest we have to be more
  restrictive on the set of processes and names we admit if we are to
  support no-cloning.]
\end{remark}

\subsubsection{Bisimulation}

The computational dynamics gives rise to another kind of equivalence,
the equivalence of computational behavior. As previously mentioned
this is typically captured \emph{via} some form of bisimulation.

% The notion we use in this paper is weak barbed bisimulation
% \cite{milner91polyadicpi}.

The notion we use in this paper is derived from weak barbed
bisimulation \cite{milner91polyadicpi}. 

\begin{definition}
An \emph{observation relation}, $\downarrow_{\mathcal N}$, over a set
of names, $\mathcal N$, is the smallest relation satisfying the rules
below.

\infrule[Out-barb]{y \in {\mathcal N}, \; x \nameeq y}
		  {\outputp{x}{v} \downarrow_{\mathcal N} x}
\infrule[Par-barb]{\mbox{$P\downarrow_{\mathcal N} x$ or $Q\downarrow_{\mathcal N} x$}}
		  {\binpar{P}{Q} \downarrow_{\mathcal N} x}

We write $P \Downarrow_{\mathcal N} x$ if there is $Q$ such that 
$P \wred Q$ and $Q \downarrow_{\mathcal N} x$.
\end{definition}

\begin{definition}
%\label{def.bbisim}
An  ${\mathcal N}$-\emph{barbed bisimulation} over a set of names, ${\mathcal N}$, is a symmetric binary relation 
${\mathcal S}_{\mathcal N}$ between agents such that $P\rel{S}_{\mathcal N}Q$ implies:
\begin{enumerate}
\item If $P \red P'$ then $Q \wred Q'$ and $P'\rel{S}_{\mathcal N} Q'$.
\item If $P\downarrow_{\mathcal N} x$, then $Q\Downarrow_{\mathcal N} x$.
\end{enumerate}
$P$ is ${\mathcal N}$-barbed bisimilar to $Q$, written
$P \wbbisim_{\mathcal N} Q$, if $P \rel{S}_{\mathcal N} Q$ for some ${\mathcal N}$-barbed bisimulation ${\mathcal S}_{\mathcal N}$.
\end{definition}

$\mathcal{R} \subseteq \pi \times \pi$

$P \mathcal{R} Q => \forall P'. P \red P' \Rightarrow \exists Q'. Q \red Q', P' \mathcal{R} Q'$

$P \vdash x \Rightarrow Q \vdash x$

\begin{mathpar}
  \inferrule*[lab=Out-barb]{x \nameeq y}{{y}!\langle{Q}\rangle \vdash x}
  \and
  \inferrule*[lab=Par-barb]{\mbox{$P\vdash x$ or $Q\vdash x$}}{\binpar{P}{Q} \vdash x}
\end{mathpar}

\subsubsection{Contexts}

One of the principle advantages of computational calculi like the
$\pi$-calculus is a well-defined notion of context,
contextual-equivalence and a correlation between
contextual-equivalence and notions of bisimulation. The notion of
context allows the decomposition of a process into (sub-)process and
its syntactic environment, its context. Thus, a context may be
thought of as a process with a ``hole'' (written $\Box$) in it. The
application of a context $M$ to a process $P$, written $M[P]$, is
tantamount to filling the hole in $M$ with $P$. In this paper we do
not need the full weight of this theory, but do make use of the notion
of context in the proof the main theorem. 

\begin{mathpar}
  \inferrule* [lab=summation] {} {{M_{M},M_{N}} \bc \Box \;|\; x.M_{A} \;|\; M_{M}+M_{N}}
  \and
  \inferrule* [lab=agent] {} {{M_{A}} \bc (\vec{x})M_{P} \;| \; \clift{P_0,\ldots,M_{P},\ldots,P_N}}
  \and \\
  \inferrule* [lab=process] {} {{M_{P}} \bc M_{N} \;| \;P|M_{P} }
\end{mathpar} 

\begin{mathpar}
  \inferrule* [lab=sychronization] {} {M_{N} \bc \Box \;|\; x?M_{F} \;|\; x!M_{C}}
  \and
  \inferrule* [lab=abstraction] {} {{M_{F}} \bc (x)M_{P} }
  \and
  \inferrule* [lab=concretion] {} {{M_{C}} \bc \langle M_{P} \rangle }
  \and \\
  \inferrule* [lab=process] {} {{M_{P}} \bc M_{N} \;| \;P|M_{P} }
\end{mathpar}

\begin{definition}[contextual application] Given a context $M$, and
  process $P$, we define the \emph{contextual application}, $M[P] :=
  M\{P/\Box\}$. That is, the contextual application of M to P is the
  substitution of $P$ for $\Box$ in $M$.
\end{definition}

$\meaningof{-} : L \to \mathcal{P}(\pi)$

\begin{mathpar}
  \inferrule* [lab=collection] {} {\meaningof{true} = \pi, \and \meaningof{~E} = \pi \setminus \meaningof{E}, \and \meaningof{E_{1} \& E_{2}} = \meaningof{E_{1}} \cap \meaningof{E_{2}}}
\end{mathpar}

\begin{mathpar}
  \inferrule* [lab=structure] {} {\meaningof{0} = \{ P \in \pi | P \equiv 0 \}, \and \\ \meaningof{E_1 | E_2} = \{ P \in \pi | P \equiv P_{1} | P_{2}, P_{1} \in \meaningof{E_{1}}, P_{2} \in \meaningof{E_2}\} }
\end{mathpar}

\begin{mathpar}
 \inferrule* [lab=behavior] {} {\meaningof{\langle a?b \rangle E} = \{ P \in \pi | P \equiv Q | u?(y)P', \\ \and \\\\ \and \\ \;\;\; u \in \meaningof{a}, \forall z.P'\{z/y\} \in \meaningof{E\{z/b\}}\}, \and \\ \meaningof{a!E} = \{ P \in \pi | P \equiv Q | x!\langle P' \rangle, x \in \meaningof{a} P' \in \meaningof{E}\} }
\end{mathpar}

\begin{mathpar}
 \inferrule* [lab=nominal] {} {\meaningof{\quotep{E}} = \{ \quotep{P} \in \quotep{\pi} | P \in \meaningof{E} \}, \and \meaningof{\quotep{P}} = \{ \quotep{Q} \in \quotep{\pi} | P \equiv Q \} \and \\ \meaningof{@\quotep{E}} = \{ P \in \pi | P \equiv @x, x \in \meaningof{E} \}}
\end{mathpar}

\begin{eqnarray*}
  \\
  \meaningof{-} : TS \to ST
\end{eqnarray*}

\begin{eqnarray*}
  \\
  L : TS \to ST
\end{eqnarray*}

\begin{eqnarray*}
  \\
  P \models E \iff P \in \meaningof{E}
\end{eqnarray*}

\begin{eqnarray*}
  P \approx_{L} Q \iff \forall E \in L. P \models E \iff Q \models E
\end{eqnarray*}

\begin{eqnarray*}
  P \approx_{K} Q
\end{eqnarray*}

\begin{eqnarray*}
  P \approx Q
\end{eqnarray*}

$\approx_{K} = \approx = \approx_{L}$

\subsubsection{Contextual duality}

Note that contexts extend the quotation operation to a family of
operations from processes to names. Given a context, $M$, we can
define a \emph{nominal context}, $\quotep{M}$ by $\quotep{M}[P] :=
\quotep{M[P]}$. To foreshadow what is to come we observe that these
operations enjoy a duality with processes very much like the duality
between vectors and maps from vectors to scalars.

Further, because the calculus is essentially higher-order, we have a
correspondence between contexts and processes. More specifically,
given a name $x$ and a context $M$ we can construct $M^{*}_{x}$ such
that 

\begin{mathpar}
  M^{*}_{x} | \lift{x}{P} \red M[P]
\end{mathpar}

namely,

\begin{mathpar}
  M^{*}_{x} := x?(u).M[\dropn{u}]
\end{mathpar}

The dependence of $M^{*}_{x}$ on a name makes it an abstraction, 

\begin{mathpar}
  M^{*} := (x)x?(u).M[\dropn{u}]
\end{mathpar}

\subsection{Additional notation}

It will sometimes be convenient to denote the process a name
quotes. We already have the notation $x = \quotep{P}$, but it will be
convenient to introduce an alternate notation, $\procn{x}$, when we
want to emphasize the connection to the use of the name. Note that, by
virtue of name equivalence, $\quotep{\procn{x}} \nameeq x$; so, the
notation is consistent with previous definitions.

Further, because names have structure it is possible to effect
substitutions on the basis of that structure. This means we need to
upgrade our notation for substitutions, which we accomplish by
adapting comprehension notation. Thus,

\begin{mathpar}
  P\{ y / x : x \in S \}
\end{mathpar}

is interpreted to mean the process derived from P by replacing (in a
capture-avoiding manner) each occurrence of $x$ in $S$ by $y$. For example,

\begin{mathpar}
  P\{ \quotep{\procn{x}|\procn{x}} / x : x \in \freenames{P} \}
\end{mathpar}

will replace each (occurrence) of a free name $x$ in $P$ by
$\quotep{\procn{x}|\procn{x}}$.

Also, we will avail ourselves of the notation $x^{L}$ and $x^{R}$ to
denote injections of a name into disjoint copies of the name
space. There are numerous ways to accomplish this. One example can be
found in \cite{MeredithR05}. This notation overloads to vectors of
names: $\vec{x}^{\pi} := (x_{i}^{\pi} \; : \; 0 \leq i < |\vec{x}| )$ where $\pi \in \{L,R\}$.

We also use $P^{\Box} := P|\Box$.

In \cite{MeredithR05} an interpretation of the new operator is
given. It turns out that there are several possible interpretations
all enjoying the requisite algebraic properties of the operator (see
\cite{milner91polyadicpi}). We will therefore make liberal use of
$(\nu\; \vec{x})P$.

% subsection the_syntax_and_semantics_of_the_notation_system (end)   

\input{qm2pi.qmops} 

\input{qm2pi.sterngerlach} 

\input{qm2pi.metric} 

% section concurrent_process_calculi (end)

%\input{qm2pi.proofsketch}

% section proof sketch (end)

%\input{qm2pi.slviaknots} 

% section spatial logic via knots (end)

\input{qm2pi.conclusion}

% section conclusion (end)

%\input{qm2pi.dtcodes} 

% section wiring algorithm (end)

\input{qm2pi.ack} 

% section acknowledgments (end)

\newpage


\bibliographystyle{plain}   
\bibliography{../../biblios/main.bib}

\input{qm2pi.rhodetails}

\end{document}



% section proof sketch (end)

%\section{Unlikely characters: spatial logic for
  knots}\label{sub:characteristic_formulae} % (fold)

Associated to the mobile process calculi are a family of logics known
as the Hennessy-Milner logics. These logics typically enjoy a
semantics interpreting formulae as sets of processes that when
factored through the encoding outlined above allows an identification
of classes of knots with logical formulae. In the context of this
encoding the sub-family known as the spatial logics \cite{CairesC03}
\cite{CairesC04} \cite{Caires04} are of particular interest providing
several important features for expressing and reasoning about
properties (i.e. classes) of knots. We hint here at how this may be done.

%\begin{description}
%\item [structural connectives] 
\subsubsection{Structural connectives} The spatial logics enjoy
structural connectives corresponding, at the logical level, to the
parallel composition ($P | Q$) and new name ($(\nu \; x)P$)
connectives for processes. As illustrated in the examples below, these
connectives are extremely expressive given the shape of our encoding.
%\item [decideable satisfaction]

\subsubsection{Decideable satisfaction}
In \cite{Caires04} the satisfaction relation is shown to be decideable
for a rich class of processes. It further turns out that the image of
the our encoding is a proper subset of that class. This result
provides the basis for an algorithm by which to search for knots
enjoying a given property.
%\item [characteristic formulae]

\subsubsection{Characteristic formulae}
In the same paper \cite{Caires04} , Caires presents a means of calculating
characteristic formulae, selecting equivalence classes of processes
up to a pre--specified depth limit on the support set of names. Composed with our
encoding, this characteristic formula can be used to select
characteristic formulae for knots.
%\end{description}

\subsubsection{Spatial logic formulae}

The grammar below (segmented for comprehension) summarizes the syntax
of spatial logic formulae. We employ illustrative examples in the
sequel to provide an intuitive understanding of their meaning
referring the reader to \cite{Caires04} for a more detailed explication
of the semantics.

\begin{mathpar}
  \inferrule* [lab=boolean] {} {{A,B} \bc T \;|\; \neg A \;|\; A \wedge B \;|\; \eta = \eta'}
  \and
  \inferrule* [lab=spatial] {} {|\; \pzero \;|\; A | B \;|\; x \text{\textregistered} A \;|\; \forall x . A \;|\;  H x . A}
  \and
  \inferrule* [lab=behavioral] {} {|\; \alpha . A}
  \and 
  \inferrule* [lab=recursion] {} {|\; X(\vec{u}) \;|\; \mu X(\vec{u}) . A}
  \and
  \inferrule* [lab=action] {} {\alpha \bc \langle x?(\vec{y}) \rangle \;|\; \langle x!(\vec{y}) \rangle \;|\; \langle \tau \rangle}
  \and 
  \inferrule* [lab=name] {} {\eta \bc x \;|\; \tau}
\end{mathpar} 

% subsection characteristic_formulae (end)   	 

\subsection{Example formulae}\label{sub:example_formulae_} % (fold)

\subsubsection{Crossing as formula.}
% 
% \begin{align*}
%   \frac{d}{dx} \sin x &= \cos x 
%   & \frac{d}{dx} e^x &= e^x \\
%   \frac{d}{dx} \cos x &= - \sin x 
%   & \frac{d}{dx} \log x &= \frac{1}{x} \\
% \end{align*} 

\begin{align*}
 \mu C(x_{0},x_{1},y_{0},y_{1},u).&(\langle x_{0}?(z) \rangle(\langle u! \rangle\langle y_{1}!z \rangle C(x_{0},x_{1},y_{0},y_{1},u)) & \\
  & \wedge \langle y_{1}?(z) \rangle (\langle u! \rangle \langle x_{0}!z \rangle C(x_{0},x_{1},y_{0},y_{1},u)) & \\
  & \wedge \langle x_{1}?(z) \rangle (\langle u? \rangle \langle y_{0}!z \rangle C(x_{0},x_{1},y_{0},y_{1},u)) & \\
  & \wedge \langle y_{0}?(z) \rangle (\langle u? \rangle \langle x_{1}!z \rangle C(x_{0},x_{1},y_{0},y_{1},u))) &
\end{align*}

The lexicographical similarity between the shape of this formulae and
the shape of definition of the process representing a crossing reveals
the intuitive meaning of this formulae. It describes the capabilities
of a process that has the right to represent a crossing. For example
it picks out processes that may perform an input on the port $x_0$ in
its initial menu of capabilities. What differentiates the formula
from the process, however, is that the crossing process is the
smallest candidate to satisfy the formula. Infinitely many other
processes -- with internal behavior hidden behind this interface, so
to speak -- also satisfy this formula. Even this simple formula,
then, can be seen to open a new view onto knots, providing a
computational interpretation of \emph{virtual} knots.

Note that this formula is derived by hand. A similar formula can be
derived by employing Caires' calculation of characteristic formula
\cite{Caires04} to the process representing a crossing. In light of
this discussion, we let
$\meaningof{C}_{\phi}(x0,x1,y0,y1,u)$ denote a formula specifying the
dynamics we wish to capture of a crossing. To guarantee we preserve
the shape of the interface and minimal semantics we demand that
$\meaningof{C}_{\phi}(x0,x1,y0,y1,u) \Rightarrow
\textbf{C}(x0,x1,y0,y1,u)$ where $\textbf{C}(x0,x1,y0,y1,u)$ denotes
the formula above.
                            
\subsubsection{Crossing number constraints.}
The moral content of the context lemma (Lemma \ref{context}) is that the notion of
``locality'' in the Reidemeister moves is effectively captured by the
parallel composition operator of the process calculus. This intuition
extends through the logic. Given a formula,
$\meaningof{C}_{\phi}(x0,x1,y0,y1,u)$, we can use the structural
connectives to specify constraints on crossing numbers, such as at
least $n$ crossings, or exactly $n$ crossings.
\begin{mathpar}
  \inferrule* [lab=at-least-n] {} { K^{\geq n}_{\phi}(\vec{xs},\vec{ys}) := \Pi_{i=0}^{n-1} Hu . \meaningof{C}_{\phi}(xs_i,ys_i,u) | T }
  \and 
  \inferrule* [lab=exactly-n] {} { K^{= n}_{\phi}(\vec{xs},\vec{ys}) := \Pi_{i=0}^{n-1} Hu . \meaningof{C}_{\phi}(xs_i,ys_i,u) | \neg (\forall x_0,y_0,x_1,y_1,u . \meaningof{C}_{\phi}(x_0,y_0,x_1,y_1,u) | T) }
\end{mathpar}

To round out this section, recall that the encoding of an $n$-crossing
knot decomposes into a parallel composition of $n$ \emph{copies} of a
crossing process together with a wiring harness. To specify different
knot classes with the same crossing number amounts to specifying
logical constraints on the wiring harness. In the interest of space,
we defer examples to a forthcoming paper. Suffice it to say that both
the conditions ``alternating knot'' and ``contains the tangle
corresponding to 5/3'' are expressible. For example, it is possible to
calculate the characteristic formula of a process corresponding to the
tangle 5/3 and conjoin it into the classifying formula via the
composition connective of the logic.

Finally, we wish to observe that it is entirely within reason to
contemplate a more domain-specific version of spatial logic tailored
to the shape of processes in the image of the encoding. Such a
domain-specific logic would have a better claim to the title formal
language of knot properties.

% subsection example_formulae_ (end)

% section knots_as_processes (end) 

% section spatial logic via knots (end)

\section{Conclusions and future work}

\paragraph{Testing physical space}
You, gentle reader, may wonder why of all the theorems to be proved
given this set up we pick the one above. In some sense it's hardly
central to quantum mechanics. We see it as central in the sense that
it firmly establishes a notion of physical space arising from a notion
of the equivalence of behavior. Relating bisimulation to a metric is a
big step forward, but one is faced with interpreting the relationship
of that metric space to something more physical. Quantum mechanical
notions of ``physical'' space are still far from intuitive, but by
relating this idea of distance as testing to calculations that predict
physical circumstances we are making a not insignificant step forward
toward an understanding of the physical space we inhabit as
essentially dynamic.

\paragraph{Effectivity and simulation}
One of the observations we have yet to make is that the entire program
spelled out here is effective. We have built various interpreters for
the reflective calculus at work in this interpretation. In principle,
then, we can simulate quantum mechanics on a computer. The place where
the simulation may lose fidelity is the infinitely branching summation
for the annihilator.

In this connection i also want to point out that the evaluation style
calculation of the inner product puts the non-determinism of the
summation right at the heart of measurement. This suggests that
Milner's original reduction-based formulation of the dynamics of his
calculi in terms of sums was not just notationally suggestive of a
notion of measure-and-continue but captured some significant part of
the physics.

\paragraph{Quantum continuations}
In light of this last observation i want to point out that the
predominant account of quantum mechanics is missing a key aspect of a
truly compositional story of the physical situation. In a real lab,
when a measurement is made the observation can be made to feed into
another device that then makes another measurement conditioned on the
results of the first. This means that after the superposition was
collapsed the entire experimental set up remained in
superposition. While QM offers a means of writing this down it doesn't
quite line up well with the well-trodden formulation of computation
and continuation that we see so succinctly expressed in Milner's
calculi. This suggests that there might be advantages to this account
of dynamics waiting to be explored.

\paragraph{Quantum logic}
In this connection, we also note that by virtue of having the
Hennessy-Milner construction, we can pull the construction through the
interpretation of QM. This gives us a natural candidate for a quantum
logic that enjoys an extremely tight connection with it's domain of
interpretation, making the construction much less ad hoc (rather it is
the image of functor!).

\paragraph{Quantum probabiity}
i have questions about the basis of the interpretation of inner
product as probability amplitude. In particular, using which
axiomatization of probability theory does the notion of probability
amplitude earn the right to be so dubbed? In other words, where is the
proof that the operation for calculating a probability amplitude (and
then squaring) satisfies the axioms of what it means to calculate a
probability? Even if such a proof exists (i have yet to find it in the
literature), i wonder if it might not be possible to turn things on
their heads. Can we view the calculation of the probability amplitude
as an axiomatization of probability? If so, then the definition we
give for calculating probability amplitude may provide the basis for
an \emph{effective} theory of probability.

\paragraph{Quantum vs ``biological'' information}
Finally, i want to conclude with a more philosophical observation. At
a recent workshop in which QM was a predominant topic i noticed
something about quantum information. The speaker was giving a riveting
discussion of axiomatic QM and showing how properties of ``no
cloning'' and ``no deleting'' emerged as consequences of the
axiomatization. Theorems of this form are necessary to give us a sense
of confidence that our axioms characterize the physical theory. What
struck me, though, was that if quantum information is neither erasable
nor replicable it is markedly different from \emph{life}. Two of the
things we know about life is that

\begin{itemize}
  \item it ends;
  \item to gain some measure of persistence, to transcend it's
    finitude it is imminently copyable.
\end{itemize}

Both of these qualities are summarized succinctly in the aphorism: all
flesh is grass. For me these two kinds of ``information'' -- call them
quantum and biological -- are end points on a spectrum of strategies
for persistence. At one end, we have those curious entities that enjoy
uniqueness and permanence; at the other, we have those who in the face
of a certain end and an uncertain present make a go of passing
something on. To me one of the more remarkable aspects of the latter
strategy is that in the presence of noise (and certain features of
copying) we get a kind of dynamism, a chance for improvement against a
given persistent condition.

% subsection other_calculi_other_bisimulations_and_geometry_as_behavior (end)




% section conclusion (end)

%\documentclass[12pt]{llncs}
%\documentclass{jktr}

\usepackage[pdftex]{hyperref}                   
\usepackage {listings}
\usepackage {mathpartir}
\usepackage{bcprules}
%\usepackage{listings}
                       
\usepackage{graphicx} 
%\usepackage[margins=2.5cm,nohead,nofoot]{geometry}
%\usepackage{geometry}
\usepackage{amsfonts}
\usepackage{amstext}
\usepackage{latexsym}
\usepackage{amssymb}
\usepackage{color}


%\include{myPreamble}
\include{qm2pi.local} 

%\ifpdf
%\usepackage[pdftex]{graphicx}
%\else
%\usepackage{graphicx}
%\fi

 % \ifpdf
%  \usepackage{pdfsync}
%  \if


%\title{Brief Article}
%\author{David F. Snyder}
%\author{L.G. Meredith}

%\address{Dept. of Math., Texas State University--San Marcos, San Marcos, TX 78666}
       
\pagestyle{empty}


\begin{document}

\lstset{language=[Objective]Caml,frame=shadowbox}

\input{qm2pi.front}

% section front matter (end)

\input{qm2pi.intro} 
 
% section introduction (end)

% \input{qm2pi.knotations} 

% section notation (end)

\input{qm2pi.process.calculi} 

% section concurrent_process_calculi_and_spatial_logics_ (end)
    
%\input{qm2pi.knots2pi} 

%\input{qm2pi.trefoil} 

%\input{qm2pi.mainthm} 

% subsection basic_interpretation (end)

%\input{qm2pi.rho.presentation} 
\subsection{The syntax and semantics of the notation system}\label{sub:the_syntax_and_semantics_of_the_notation_system} % (fold)

We now summarize a technical presentation of the calculus that
embodies our theory of dynamics. The typical presentation of such a
calculus follows the style of giving generators and relations on
them. The grammar, below, describing term constructors, freely
generates the set of processes, $\Proc$. This set is then quotiented
by a relation known as structural congruence and it is over this set
that the notion of dynamics is expressed. This presentation is
essentially that of \cite{MeredithR05} with the addition of
polyadicity and summation. For readability we have relegated some of
the technical subtleties to an appendix.

\subsubsection{Process grammar}\label{subsub:process_grammar}

\begin{mathpar}
  \inferrule* [lab=synchronization] {} {{M} \bc \pzero \;|\; x?F \;|\; x!C }
  \and
  \inferrule* [lab=abstraction] {} {{F} \bc (x)P}
  \and
  \inferrule* [lab=concretion] {} {{C} \bc \langle Q \rangle}
  \and
  \inferrule* [lab=process] {} {{P,Q} \bc M \;| \;P|Q \;|\; @{x}}
  \and
  \inferrule* [lab=name] {} {{x} \bc \quotep{P}}
\end{mathpar} 

Note that $\vec{x}$ (resp. $\vec{P}$) denotes a vector of names
(resp. processes) of length $|\vec{x}|$ (resp. $|\vec{P}|$). We adopt
the following useful abbreviations.

\begin{mathpar}
   x?(\vec{y}).P := x.(\vec{y})P \and  x\clift{\vec{P}} := x.\clift{\vec{P}}
   \and x!(y) := \lift{x}{\dropn{y}}
   \and \Pi_{i=0}^{n-1}P_i := P_0 | \ldots | P_{n-1}
\end{mathpar}

\subsubsection{Structural congruence}

\paragraph{Free and bound names and alpha-equivalence.} At the
core of structural equivalence is alpha-equivalence which identifies
process that are the same up to a change of variable. Formally, we
recognize the distinction between free and bound names. The free names
of a process, $\freenames{P}$, may be calculated recursively as
follows:

\begin{mathpar}
\freenames{\pzero} := \emptyset
  \and \\
  \freenames{x?(y).P} := \{ x \} \cup (\freenames{P} \setminus \{ y \})
  \and 
  \freenames{x!\langle P \rangle} := \{ x \} \cup \{ P \} 
  \and \\
  \freenames{P|Q} := \freenames{P} \cup \freenames{Q}
  \and \\
  \freenames{@{x}} := \{ x \}
\end{mathpar}

$\pi$
$\quotep{\pi}$

$\freenames{-} : \pi \to \mathcal{P}(\quotep{\pi})$

\begin{eqnarray*}
  \freenames{\pzero} & := & \emptyset \\
  \freenames{x?(y).P} & := & \{ x \} \cup (\freenames{P} \setminus \{ y \}) \\
  \freenames{x!\langle P \rangle} & := & \{ x \} \cup \{ P \} \\
  \freenames{P|Q} & := & \freenames{P} \cup \freenames{Q} \\
  \freenames{\dropn{x}} & := & \{ x \}
\end{eqnarray*}

The bound names of a process, $\boundnames{P}$, are those names occurring in $P$
that are not free. For example, in $x?(y).0$, the name $x$ is free, while $y$ is bound.

\begin{mathpar}
  \inferrule* [lab=monoidal-laws] {} { P|Q \equiv Q|P \and P|0 \equiv P \and P|(Q|R) \equiv (P|Q)|R }
\end{mathpar}

\begin{mathpar}
  \inferrule* [lab=alpha-equivalence] {} { (x)P \equiv (y)P\{y/x\} \and y \not\in \freenames{P} }
\end{mathpar}

\begin{definition}
Then two processes, $P,Q$, are alpha-equivalent if $P = Q\{\vec{y}/\vec{x}\}$ for
some $\vec{x} \in \boundnames{Q},\vec{y} \in \boundnames{P}$, where $Q\{\vec{y}/\vec{x}\}$
denotes the capture-avoiding substitution of $\vec{y}$ for $\vec{x}$ in $Q$.
\end{definition}

\begin{definition}
  The {\em structural congruence} \cite{SangiorgiWalker} , $\equiv$,
  between processes is the least congruence containing
  alpha-equivalence, satisfying the abelian monoid laws
  (associativity, commutativity and $\pzero$ as identity) for parallel
  composition $|$ and for summation $+$.
\end{definition}

\subsection{Name equivalence}

We take name equivalence, written $\nameeq$, to be the smallest
equivalence relation generated by the following rules.

\begin{mathpar}
\inferrule*[lab=Quote-drop]
{ }
{ \quotep{@{x}} \nameeq x }

\inferrule*[lab=Struct-equiv]
{ P \scong Q }
{ \quotep{P} \nameeq \quotep{Q} }
\end{mathpar}

The astute reader will have noticed that the mutual recursion of names
and processes imposes a mutual recursion on alpha-equivalence and
structural equivalence via name-equivalence. Fortunately, all of this
works out pleasantly and we may calculate in the natural way, free of
concern. The reader interested in the details is referred to the
appendix \ref{appendix:rho_details}.

\subsection{Substitution}

We use $\Proc$ for the set of processes, $\QProc$ for the set of
names, and $\id{\{}\vec{y} / \vec{x} \id{\}}$ to denote partial maps,
$s : \QProc \rightarrow \QProc$. A map, $s$ lifts, uniquely, to a map
on process terms, $\widehat{s} : \Proc \rightarrow \Proc$ by the
following equations.

\begin{mathpar}
  (0) \psubstp{Q}{P} := 0 \\
  (R \juxtap S) \psubstp{Q}{P}
  :=    
  (R)\psubstp{Q}{P} \juxtap (S) \psubstp{Q}{P} \\
  (x?(y).R) \psubstp{Q}{P}    
  :=    
  (x)\substp{Q}{P} (z)\concat( (R \psubstn{z}{y}) \psubstp{Q}{P} ) \\
  (\lift{x}{R}) \psubstp{Q}{P}  
  :=
  \lift{(x)\substp{Q}{P}}{ R \psubstp{Q}{P} } \\
%   (\dropn{x})  \psubstp{Q}{P}       
%   := 
%   \left\{ 
%     \begin{array}{ccc} 
%       \dropn{\quotep{Q}} & & x \nameeq \quotep{P} \\
%       \dropn{x} & & otherwise \\
%     \end{array}
%   \right. 
  (\dropn{x})  \psubstp{Q}{P}       
  := 
  \left\{ 
    \begin{array}{ccc} 
      Q & & x \nameeq \quotep{P} \\
      \dropn{x} & & otherwise \\
    \end{array}
  \right.
\end{mathpar}
 

where

\begin{eqnarray}
  (x)\id{\{} \lpquote Q \rpquote / \lpquote P \rpquote \id{\}}            = 
  \left\{ 
    \begin{array}{ccc}
      \lpquote Q \rpquote & & x \nameeq \lpquote P \rpquote \\
      x & & otherwise \\
    \end{array}
  \right. \nonumber
\end{eqnarray}

and $z$ is chosen distinct from $\quotep{P}$, $\quotep{Q}$, the free
names in $Q$, and all the names in $R$. Our $\alpha$-equivalence will
be built in the standard way from this substitution.

\begin{remark}\label{rem:no_self_referential_names}
  One consequence of these definitions is that $\forall P. \quotep{P}
  \not\in \freenames{P}$.
\end{remark}

\subsection{ Dynamic quote: an example }

Anticipating something of what's to come, consider applying the
substitution, $\widehat{\id{\{}u / z \id{\}}}$, to the following pair
of processes, $\lift{w}{y!(z)}$ and $w[ \lpquote y!(z) \rpquote ]$.

\begin{eqnarray}
	\lift{w}{y!(z)}\widehat{\id{\{}u / z \id{\}}}
		& = &
		\lift{w}{y!(u)} \nonumber\\
	w[ \lpquote y!(z) \rpquote ] \widehat{ \id{\{}u / z \id{\}} }
		& = &
		w[ \lpquote y!(z) \rpquote ] \nonumber
\end{eqnarray}

Because the body of the process between quotes is impervious to
substitution, we get radically different answers. In fact, by
examining the first process in an input context,
e.g. $x?(z).\lift{w}{y!(z)}$, we see that the process under the lift
operator may be shaped by prefixed inputs binding a name inside it. In
this sense, the lift operator will be seen as a way to dynamically
construct processes before reifying them as names.

Finally equipped with these standard features we can present the
dynamics of the calculus.

\subsubsection{Operational semantics} 

Finally, we introduce the computational dynamics. What marks these
algebras as distinct from other more traditionally studied algebraic
structures, e.g. vector spaces or polynomial rings, is the manner in
which dynamics is captured. In traditional structures, dynamics is typically
expressed through morphisms between such structures, as in linear maps
between vector spaces or morphisms between rings. In algebras
associated with the semantics of computation, the dynamics is
expressed as part of the algebraic structure itself, through a
reduction reduction relation typically denoted by $\red$. Below, we
give a recursive presentation of this relation for the calculus used
in the encoding.

$\red \subseteq \pi \times \pi$
$\red : \pi \to \mathcal{P}(\pi)$

\begin{mathpar}
  \inferrule* [lab=Comm] { \textsf{match}( x_{src}, x_{trgt} ) } { x_{trgt}?(y)P \; | \; x_{src}!\langle {Q} \rangle \red P\{\quotep{Q}/y}\} }
  \and \\
  \inferrule* [lab=Par] {{P} \red {P}'} {{{P} | {Q}} \red {{P}' | {Q}}}
  \and
  \inferrule* [lab=Equiv]{{{P} \scong {P}'} \andalso {{P}' \red {Q}'} \andalso {{Q}' \scong {Q}}}{{P} \red {Q}}
\end{mathpar}

\begin{eqnarray*}
  match_{\equiv} (\quotep{P},\quotep{Q}) & := & P \equiv Q \\
  match_{\dagger}(\quotep{P},\quotep{Q}) & := & \forall R. P|Q \red^{*} R => R \red^{*} 0 \\
  match_{K}(\quotep{P},\quotep{Q}) & := & K \mbox{ for some context } K
\end{eqnarray*}

$u?(x)P | u!\langle Q \rangle \red P\{\quotep{Q}/x\}$

%We write $\wred$ for $\red^*$, and $P\red$ if $\exists Q $ such that $ P \red Q$.
We write $P\red$ if $\exists Q $ such that $ P \red Q$ and $P\not\red$, otherwise.

\section{Replication}

As mentioned before, it is known that replication (and hence
recursion) can be implemented in a higher-order process algebra
\cite{SangiorgiWalker}. As our first example of calculation with the
machinery thus far presented we give the construction explicitly in
the {\rhoc}.

\begin{eqnarray}
	D_{x} & := & \prefix{x}{y}{(\binpar{\outputp{x}{y}}{@{y}})} \nonumber\\
	\bangp_{x}{P} & := & \binpar{{x}!\langle{\binpar{D_{x}}{P}}\rangle}{D_{x}} \nonumber
\end{eqnarray}

\begin{eqnarray}
	\bangp_{x}{P} & & \nonumber\\
	=
	& {x}!\langle{(\prefix{x}{y}{(\outputp{x}{y} | @{y})) | P}}\rangle 
	      | \prefix{x}{y}{(\outputp{x}{y} | @{y})} & \nonumber\\
	\red
	& (\outputp{x}{y} | @{y})\substn{\quotep{(\prefix{x}{y}{(@{y} | \outputp{x}{y})) | P}}}{y} & \nonumber\\
	=
	& \outputp{x}{\quotep{(\prefix{x}{y}{(\outputp{x}{y} | @{y})) | P}}}
	  | {(\prefix{x}{y}{(\outputp{x}{y} | @{y})) | P}} & \nonumber\\
	\red
	& \ldots & \nonumber\\
	\red^*
	& P | P | \ldots & \nonumber
\end{eqnarray}

Of course, this encoding, as an implementation, runs away, unfolding
$\bangp{P}$ eagerly. A lazier and more implementable replication
operator, restricted to input-guarded processes, may be obtained as follows.

\begin{eqnarray}
\bangp{\prefix{u}{v}{P}} 
	:= 
	\binpar{\lift{x}{\prefix{u}{v}{(\binpar{D(x)}{P})}}}{D(x)} \nonumber
\end{eqnarray}

\begin{remark}
  Note that the lazier definition still does not deal with summation
  or mixed summation (i.e. sums over input and output). The reader is
  invited to construct definitions of replication that deal with these
  features. 

  Further, the definitions are parameterized in a name, $x$. Can you,
  gentle reader, make a definition that eliminates this parameter and
  guarantees no accidental interaction between the replication
  machinery and the process being replicated -- i.e. no accidental
  sharing of names used by the process to get its work done and the
  name(s) used by the replication to effect copying. This latter
  revision of the definition of replication is crucial to obtaining
  the expected identity $!!P \sim !P$.
\end{remark}

\begin{remark}\label{rem:paradoxical_combinator}
  The reader familiar with the lambda calculus will have noticed the
  similarity between $D$ and the paradoxical combinator.

  [Ed. note: the existence of this seems to suggest we have to be more
  restrictive on the set of processes and names we admit if we are to
  support no-cloning.]
\end{remark}

\subsubsection{Bisimulation}

The computational dynamics gives rise to another kind of equivalence,
the equivalence of computational behavior. As previously mentioned
this is typically captured \emph{via} some form of bisimulation.

% The notion we use in this paper is weak barbed bisimulation
% \cite{milner91polyadicpi}.

The notion we use in this paper is derived from weak barbed
bisimulation \cite{milner91polyadicpi}. 

\begin{definition}
An \emph{observation relation}, $\downarrow_{\mathcal N}$, over a set
of names, $\mathcal N$, is the smallest relation satisfying the rules
below.

\infrule[Out-barb]{y \in {\mathcal N}, \; x \nameeq y}
		  {\outputp{x}{v} \downarrow_{\mathcal N} x}
\infrule[Par-barb]{\mbox{$P\downarrow_{\mathcal N} x$ or $Q\downarrow_{\mathcal N} x$}}
		  {\binpar{P}{Q} \downarrow_{\mathcal N} x}

We write $P \Downarrow_{\mathcal N} x$ if there is $Q$ such that 
$P \wred Q$ and $Q \downarrow_{\mathcal N} x$.
\end{definition}

\begin{definition}
%\label{def.bbisim}
An  ${\mathcal N}$-\emph{barbed bisimulation} over a set of names, ${\mathcal N}$, is a symmetric binary relation 
${\mathcal S}_{\mathcal N}$ between agents such that $P\rel{S}_{\mathcal N}Q$ implies:
\begin{enumerate}
\item If $P \red P'$ then $Q \wred Q'$ and $P'\rel{S}_{\mathcal N} Q'$.
\item If $P\downarrow_{\mathcal N} x$, then $Q\Downarrow_{\mathcal N} x$.
\end{enumerate}
$P$ is ${\mathcal N}$-barbed bisimilar to $Q$, written
$P \wbbisim_{\mathcal N} Q$, if $P \rel{S}_{\mathcal N} Q$ for some ${\mathcal N}$-barbed bisimulation ${\mathcal S}_{\mathcal N}$.
\end{definition}

$\mathcal{R} \subseteq \pi \times \pi$

$P \mathcal{R} Q => \forall P'. P \red P' \Rightarrow \exists Q'. Q \red Q', P' \mathcal{R} Q'$

$P \vdash x \Rightarrow Q \vdash x$

\begin{mathpar}
  \inferrule*[lab=Out-barb]{x \nameeq y}{{y}!\langle{Q}\rangle \vdash x}
  \and
  \inferrule*[lab=Par-barb]{\mbox{$P\vdash x$ or $Q\vdash x$}}{\binpar{P}{Q} \vdash x}
\end{mathpar}

\subsubsection{Contexts}

One of the principle advantages of computational calculi like the
$\pi$-calculus is a well-defined notion of context,
contextual-equivalence and a correlation between
contextual-equivalence and notions of bisimulation. The notion of
context allows the decomposition of a process into (sub-)process and
its syntactic environment, its context. Thus, a context may be
thought of as a process with a ``hole'' (written $\Box$) in it. The
application of a context $M$ to a process $P$, written $M[P]$, is
tantamount to filling the hole in $M$ with $P$. In this paper we do
not need the full weight of this theory, but do make use of the notion
of context in the proof the main theorem. 

\begin{mathpar}
  \inferrule* [lab=summation] {} {{M_{M},M_{N}} \bc \Box \;|\; x.M_{A} \;|\; M_{M}+M_{N}}
  \and
  \inferrule* [lab=agent] {} {{M_{A}} \bc (\vec{x})M_{P} \;| \; \clift{P_0,\ldots,M_{P},\ldots,P_N}}
  \and \\
  \inferrule* [lab=process] {} {{M_{P}} \bc M_{N} \;| \;P|M_{P} }
\end{mathpar} 

\begin{mathpar}
  \inferrule* [lab=sychronization] {} {M_{N} \bc \Box \;|\; x?M_{F} \;|\; x!M_{C}}
  \and
  \inferrule* [lab=abstraction] {} {{M_{F}} \bc (x)M_{P} }
  \and
  \inferrule* [lab=concretion] {} {{M_{C}} \bc \langle M_{P} \rangle }
  \and \\
  \inferrule* [lab=process] {} {{M_{P}} \bc M_{N} \;| \;P|M_{P} }
\end{mathpar}

\begin{definition}[contextual application] Given a context $M$, and
  process $P$, we define the \emph{contextual application}, $M[P] :=
  M\{P/\Box\}$. That is, the contextual application of M to P is the
  substitution of $P$ for $\Box$ in $M$.
\end{definition}

$\meaningof{-} : L \to \mathcal{P}(\pi)$

\begin{mathpar}
  \inferrule* [lab=collection] {} {\meaningof{true} = \pi, \and \meaningof{~E} = \pi \setminus \meaningof{E}, \and \meaningof{E_{1} \& E_{2}} = \meaningof{E_{1}} \cap \meaningof{E_{2}}}
\end{mathpar}

\begin{mathpar}
  \inferrule* [lab=structure] {} {\meaningof{0} = \{ P \in \pi | P \equiv 0 \}, \and \\ \meaningof{E_1 | E_2} = \{ P \in \pi | P \equiv P_{1} | P_{2}, P_{1} \in \meaningof{E_{1}}, P_{2} \in \meaningof{E_2}\} }
\end{mathpar}

\begin{mathpar}
 \inferrule* [lab=behavior] {} {\meaningof{\langle a?b \rangle E} = \{ P \in \pi | P \equiv Q | u?(y)P', \\ \and \\\\ \and \\ \;\;\; u \in \meaningof{a}, \forall z.P'\{z/y\} \in \meaningof{E\{z/b\}}\}, \and \\ \meaningof{a!E} = \{ P \in \pi | P \equiv Q | x!\langle P' \rangle, x \in \meaningof{a} P' \in \meaningof{E}\} }
\end{mathpar}

\begin{mathpar}
 \inferrule* [lab=nominal] {} {\meaningof{\quotep{E}} = \{ \quotep{P} \in \quotep{\pi} | P \in \meaningof{E} \}, \and \meaningof{\quotep{P}} = \{ \quotep{Q} \in \quotep{\pi} | P \equiv Q \} \and \\ \meaningof{@\quotep{E}} = \{ P \in \pi | P \equiv @x, x \in \meaningof{E} \}}
\end{mathpar}

\begin{eqnarray*}
  \\
  \meaningof{-} : TS \to ST
\end{eqnarray*}

\begin{eqnarray*}
  \\
  L : TS \to ST
\end{eqnarray*}

\begin{eqnarray*}
  \\
  P \models E \iff P \in \meaningof{E}
\end{eqnarray*}

\begin{eqnarray*}
  P \approx_{L} Q \iff \forall E \in L. P \models E \iff Q \models E
\end{eqnarray*}

\begin{eqnarray*}
  P \approx_{K} Q
\end{eqnarray*}

\begin{eqnarray*}
  P \approx Q
\end{eqnarray*}

$\approx_{K} = \approx = \approx_{L}$

\subsubsection{Contextual duality}

Note that contexts extend the quotation operation to a family of
operations from processes to names. Given a context, $M$, we can
define a \emph{nominal context}, $\quotep{M}$ by $\quotep{M}[P] :=
\quotep{M[P]}$. To foreshadow what is to come we observe that these
operations enjoy a duality with processes very much like the duality
between vectors and maps from vectors to scalars.

Further, because the calculus is essentially higher-order, we have a
correspondence between contexts and processes. More specifically,
given a name $x$ and a context $M$ we can construct $M^{*}_{x}$ such
that 

\begin{mathpar}
  M^{*}_{x} | \lift{x}{P} \red M[P]
\end{mathpar}

namely,

\begin{mathpar}
  M^{*}_{x} := x?(u).M[\dropn{u}]
\end{mathpar}

The dependence of $M^{*}_{x}$ on a name makes it an abstraction, 

\begin{mathpar}
  M^{*} := (x)x?(u).M[\dropn{u}]
\end{mathpar}

\subsection{Additional notation}

It will sometimes be convenient to denote the process a name
quotes. We already have the notation $x = \quotep{P}$, but it will be
convenient to introduce an alternate notation, $\procn{x}$, when we
want to emphasize the connection to the use of the name. Note that, by
virtue of name equivalence, $\quotep{\procn{x}} \nameeq x$; so, the
notation is consistent with previous definitions.

Further, because names have structure it is possible to effect
substitutions on the basis of that structure. This means we need to
upgrade our notation for substitutions, which we accomplish by
adapting comprehension notation. Thus,

\begin{mathpar}
  P\{ y / x : x \in S \}
\end{mathpar}

is interpreted to mean the process derived from P by replacing (in a
capture-avoiding manner) each occurrence of $x$ in $S$ by $y$. For example,

\begin{mathpar}
  P\{ \quotep{\procn{x}|\procn{x}} / x : x \in \freenames{P} \}
\end{mathpar}

will replace each (occurrence) of a free name $x$ in $P$ by
$\quotep{\procn{x}|\procn{x}}$.

Also, we will avail ourselves of the notation $x^{L}$ and $x^{R}$ to
denote injections of a name into disjoint copies of the name
space. There are numerous ways to accomplish this. One example can be
found in \cite{MeredithR05}. This notation overloads to vectors of
names: $\vec{x}^{\pi} := (x_{i}^{\pi} \; : \; 0 \leq i < |\vec{x}| )$ where $\pi \in \{L,R\}$.

We also use $P^{\Box} := P|\Box$.

In \cite{MeredithR05} an interpretation of the new operator is
given. It turns out that there are several possible interpretations
all enjoying the requisite algebraic properties of the operator (see
\cite{milner91polyadicpi}). We will therefore make liberal use of
$(\nu\; \vec{x})P$.

% subsection the_syntax_and_semantics_of_the_notation_system (end)   

\input{qm2pi.qmops} 

\input{qm2pi.sterngerlach} 

\input{qm2pi.metric} 

% section concurrent_process_calculi (end)

%\input{qm2pi.proofsketch}

% section proof sketch (end)

%\input{qm2pi.slviaknots} 

% section spatial logic via knots (end)

\input{qm2pi.conclusion}

% section conclusion (end)

%\input{qm2pi.dtcodes} 

% section wiring algorithm (end)

\input{qm2pi.ack} 

% section acknowledgments (end)

\newpage


\bibliographystyle{plain}   
\bibliography{../../biblios/main.bib}

\input{qm2pi.rhodetails}

\end{document}

 

% section wiring algorithm (end)

\documentclass[12pt]{llncs}
%\documentclass{jktr}

\usepackage[pdftex]{hyperref}                   
\usepackage {listings}
\usepackage {mathpartir}
\usepackage{bcprules}
%\usepackage{listings}
                       
\usepackage{graphicx} 
%\usepackage[margins=2.5cm,nohead,nofoot]{geometry}
%\usepackage{geometry}
\usepackage{amsfonts}
\usepackage{amstext}
\usepackage{latexsym}
\usepackage{amssymb}
\usepackage{color}


%\include{myPreamble}
\include{qm2pi.local} 

%\ifpdf
%\usepackage[pdftex]{graphicx}
%\else
%\usepackage{graphicx}
%\fi

 % \ifpdf
%  \usepackage{pdfsync}
%  \if


%\title{Brief Article}
%\author{David F. Snyder}
%\author{L.G. Meredith}

%\address{Dept. of Math., Texas State University--San Marcos, San Marcos, TX 78666}
       
\pagestyle{empty}


\begin{document}

\lstset{language=[Objective]Caml,frame=shadowbox}

\input{qm2pi.front}

% section front matter (end)

\input{qm2pi.intro} 
 
% section introduction (end)

% \input{qm2pi.knotations} 

% section notation (end)

\input{qm2pi.process.calculi} 

% section concurrent_process_calculi_and_spatial_logics_ (end)
    
%\input{qm2pi.knots2pi} 

%\input{qm2pi.trefoil} 

%\input{qm2pi.mainthm} 

% subsection basic_interpretation (end)

%\input{qm2pi.rho.presentation} 
\subsection{The syntax and semantics of the notation system}\label{sub:the_syntax_and_semantics_of_the_notation_system} % (fold)

We now summarize a technical presentation of the calculus that
embodies our theory of dynamics. The typical presentation of such a
calculus follows the style of giving generators and relations on
them. The grammar, below, describing term constructors, freely
generates the set of processes, $\Proc$. This set is then quotiented
by a relation known as structural congruence and it is over this set
that the notion of dynamics is expressed. This presentation is
essentially that of \cite{MeredithR05} with the addition of
polyadicity and summation. For readability we have relegated some of
the technical subtleties to an appendix.

\subsubsection{Process grammar}\label{subsub:process_grammar}

\begin{mathpar}
  \inferrule* [lab=synchronization] {} {{M} \bc \pzero \;|\; x?F \;|\; x!C }
  \and
  \inferrule* [lab=abstraction] {} {{F} \bc (x)P}
  \and
  \inferrule* [lab=concretion] {} {{C} \bc \langle Q \rangle}
  \and
  \inferrule* [lab=process] {} {{P,Q} \bc M \;| \;P|Q \;|\; @{x}}
  \and
  \inferrule* [lab=name] {} {{x} \bc \quotep{P}}
\end{mathpar} 

Note that $\vec{x}$ (resp. $\vec{P}$) denotes a vector of names
(resp. processes) of length $|\vec{x}|$ (resp. $|\vec{P}|$). We adopt
the following useful abbreviations.

\begin{mathpar}
   x?(\vec{y}).P := x.(\vec{y})P \and  x\clift{\vec{P}} := x.\clift{\vec{P}}
   \and x!(y) := \lift{x}{\dropn{y}}
   \and \Pi_{i=0}^{n-1}P_i := P_0 | \ldots | P_{n-1}
\end{mathpar}

\subsubsection{Structural congruence}

\paragraph{Free and bound names and alpha-equivalence.} At the
core of structural equivalence is alpha-equivalence which identifies
process that are the same up to a change of variable. Formally, we
recognize the distinction between free and bound names. The free names
of a process, $\freenames{P}$, may be calculated recursively as
follows:

\begin{mathpar}
\freenames{\pzero} := \emptyset
  \and \\
  \freenames{x?(y).P} := \{ x \} \cup (\freenames{P} \setminus \{ y \})
  \and 
  \freenames{x!\langle P \rangle} := \{ x \} \cup \{ P \} 
  \and \\
  \freenames{P|Q} := \freenames{P} \cup \freenames{Q}
  \and \\
  \freenames{@{x}} := \{ x \}
\end{mathpar}

$\pi$
$\quotep{\pi}$

$\freenames{-} : \pi \to \mathcal{P}(\quotep{\pi})$

\begin{eqnarray*}
  \freenames{\pzero} & := & \emptyset \\
  \freenames{x?(y).P} & := & \{ x \} \cup (\freenames{P} \setminus \{ y \}) \\
  \freenames{x!\langle P \rangle} & := & \{ x \} \cup \{ P \} \\
  \freenames{P|Q} & := & \freenames{P} \cup \freenames{Q} \\
  \freenames{\dropn{x}} & := & \{ x \}
\end{eqnarray*}

The bound names of a process, $\boundnames{P}$, are those names occurring in $P$
that are not free. For example, in $x?(y).0$, the name $x$ is free, while $y$ is bound.

\begin{mathpar}
  \inferrule* [lab=monoidal-laws] {} { P|Q \equiv Q|P \and P|0 \equiv P \and P|(Q|R) \equiv (P|Q)|R }
\end{mathpar}

\begin{mathpar}
  \inferrule* [lab=alpha-equivalence] {} { (x)P \equiv (y)P\{y/x\} \and y \not\in \freenames{P} }
\end{mathpar}

\begin{definition}
Then two processes, $P,Q$, are alpha-equivalent if $P = Q\{\vec{y}/\vec{x}\}$ for
some $\vec{x} \in \boundnames{Q},\vec{y} \in \boundnames{P}$, where $Q\{\vec{y}/\vec{x}\}$
denotes the capture-avoiding substitution of $\vec{y}$ for $\vec{x}$ in $Q$.
\end{definition}

\begin{definition}
  The {\em structural congruence} \cite{SangiorgiWalker} , $\equiv$,
  between processes is the least congruence containing
  alpha-equivalence, satisfying the abelian monoid laws
  (associativity, commutativity and $\pzero$ as identity) for parallel
  composition $|$ and for summation $+$.
\end{definition}

\subsection{Name equivalence}

We take name equivalence, written $\nameeq$, to be the smallest
equivalence relation generated by the following rules.

\begin{mathpar}
\inferrule*[lab=Quote-drop]
{ }
{ \quotep{@{x}} \nameeq x }

\inferrule*[lab=Struct-equiv]
{ P \scong Q }
{ \quotep{P} \nameeq \quotep{Q} }
\end{mathpar}

The astute reader will have noticed that the mutual recursion of names
and processes imposes a mutual recursion on alpha-equivalence and
structural equivalence via name-equivalence. Fortunately, all of this
works out pleasantly and we may calculate in the natural way, free of
concern. The reader interested in the details is referred to the
appendix \ref{appendix:rho_details}.

\subsection{Substitution}

We use $\Proc$ for the set of processes, $\QProc$ for the set of
names, and $\id{\{}\vec{y} / \vec{x} \id{\}}$ to denote partial maps,
$s : \QProc \rightarrow \QProc$. A map, $s$ lifts, uniquely, to a map
on process terms, $\widehat{s} : \Proc \rightarrow \Proc$ by the
following equations.

\begin{mathpar}
  (0) \psubstp{Q}{P} := 0 \\
  (R \juxtap S) \psubstp{Q}{P}
  :=    
  (R)\psubstp{Q}{P} \juxtap (S) \psubstp{Q}{P} \\
  (x?(y).R) \psubstp{Q}{P}    
  :=    
  (x)\substp{Q}{P} (z)\concat( (R \psubstn{z}{y}) \psubstp{Q}{P} ) \\
  (\lift{x}{R}) \psubstp{Q}{P}  
  :=
  \lift{(x)\substp{Q}{P}}{ R \psubstp{Q}{P} } \\
%   (\dropn{x})  \psubstp{Q}{P}       
%   := 
%   \left\{ 
%     \begin{array}{ccc} 
%       \dropn{\quotep{Q}} & & x \nameeq \quotep{P} \\
%       \dropn{x} & & otherwise \\
%     \end{array}
%   \right. 
  (\dropn{x})  \psubstp{Q}{P}       
  := 
  \left\{ 
    \begin{array}{ccc} 
      Q & & x \nameeq \quotep{P} \\
      \dropn{x} & & otherwise \\
    \end{array}
  \right.
\end{mathpar}
 

where

\begin{eqnarray}
  (x)\id{\{} \lpquote Q \rpquote / \lpquote P \rpquote \id{\}}            = 
  \left\{ 
    \begin{array}{ccc}
      \lpquote Q \rpquote & & x \nameeq \lpquote P \rpquote \\
      x & & otherwise \\
    \end{array}
  \right. \nonumber
\end{eqnarray}

and $z$ is chosen distinct from $\quotep{P}$, $\quotep{Q}$, the free
names in $Q$, and all the names in $R$. Our $\alpha$-equivalence will
be built in the standard way from this substitution.

\begin{remark}\label{rem:no_self_referential_names}
  One consequence of these definitions is that $\forall P. \quotep{P}
  \not\in \freenames{P}$.
\end{remark}

\subsection{ Dynamic quote: an example }

Anticipating something of what's to come, consider applying the
substitution, $\widehat{\id{\{}u / z \id{\}}}$, to the following pair
of processes, $\lift{w}{y!(z)}$ and $w[ \lpquote y!(z) \rpquote ]$.

\begin{eqnarray}
	\lift{w}{y!(z)}\widehat{\id{\{}u / z \id{\}}}
		& = &
		\lift{w}{y!(u)} \nonumber\\
	w[ \lpquote y!(z) \rpquote ] \widehat{ \id{\{}u / z \id{\}} }
		& = &
		w[ \lpquote y!(z) \rpquote ] \nonumber
\end{eqnarray}

Because the body of the process between quotes is impervious to
substitution, we get radically different answers. In fact, by
examining the first process in an input context,
e.g. $x?(z).\lift{w}{y!(z)}$, we see that the process under the lift
operator may be shaped by prefixed inputs binding a name inside it. In
this sense, the lift operator will be seen as a way to dynamically
construct processes before reifying them as names.

Finally equipped with these standard features we can present the
dynamics of the calculus.

\subsubsection{Operational semantics} 

Finally, we introduce the computational dynamics. What marks these
algebras as distinct from other more traditionally studied algebraic
structures, e.g. vector spaces or polynomial rings, is the manner in
which dynamics is captured. In traditional structures, dynamics is typically
expressed through morphisms between such structures, as in linear maps
between vector spaces or morphisms between rings. In algebras
associated with the semantics of computation, the dynamics is
expressed as part of the algebraic structure itself, through a
reduction reduction relation typically denoted by $\red$. Below, we
give a recursive presentation of this relation for the calculus used
in the encoding.

$\red \subseteq \pi \times \pi$
$\red : \pi \to \mathcal{P}(\pi)$

\begin{mathpar}
  \inferrule* [lab=Comm] { \textsf{match}( x_{src}, x_{trgt} ) } { x_{trgt}?(y)P \; | \; x_{src}!\langle {Q} \rangle \red P\{\quotep{Q}/y}\} }
  \and \\
  \inferrule* [lab=Par] {{P} \red {P}'} {{{P} | {Q}} \red {{P}' | {Q}}}
  \and
  \inferrule* [lab=Equiv]{{{P} \scong {P}'} \andalso {{P}' \red {Q}'} \andalso {{Q}' \scong {Q}}}{{P} \red {Q}}
\end{mathpar}

\begin{eqnarray*}
  match_{\equiv} (\quotep{P},\quotep{Q}) & := & P \equiv Q \\
  match_{\dagger}(\quotep{P},\quotep{Q}) & := & \forall R. P|Q \red^{*} R => R \red^{*} 0 \\
  match_{K}(\quotep{P},\quotep{Q}) & := & K \mbox{ for some context } K
\end{eqnarray*}

$u?(x)P | u!\langle Q \rangle \red P\{\quotep{Q}/x\}$

%We write $\wred$ for $\red^*$, and $P\red$ if $\exists Q $ such that $ P \red Q$.
We write $P\red$ if $\exists Q $ such that $ P \red Q$ and $P\not\red$, otherwise.

\section{Replication}

As mentioned before, it is known that replication (and hence
recursion) can be implemented in a higher-order process algebra
\cite{SangiorgiWalker}. As our first example of calculation with the
machinery thus far presented we give the construction explicitly in
the {\rhoc}.

\begin{eqnarray}
	D_{x} & := & \prefix{x}{y}{(\binpar{\outputp{x}{y}}{@{y}})} \nonumber\\
	\bangp_{x}{P} & := & \binpar{{x}!\langle{\binpar{D_{x}}{P}}\rangle}{D_{x}} \nonumber
\end{eqnarray}

\begin{eqnarray}
	\bangp_{x}{P} & & \nonumber\\
	=
	& {x}!\langle{(\prefix{x}{y}{(\outputp{x}{y} | @{y})) | P}}\rangle 
	      | \prefix{x}{y}{(\outputp{x}{y} | @{y})} & \nonumber\\
	\red
	& (\outputp{x}{y} | @{y})\substn{\quotep{(\prefix{x}{y}{(@{y} | \outputp{x}{y})) | P}}}{y} & \nonumber\\
	=
	& \outputp{x}{\quotep{(\prefix{x}{y}{(\outputp{x}{y} | @{y})) | P}}}
	  | {(\prefix{x}{y}{(\outputp{x}{y} | @{y})) | P}} & \nonumber\\
	\red
	& \ldots & \nonumber\\
	\red^*
	& P | P | \ldots & \nonumber
\end{eqnarray}

Of course, this encoding, as an implementation, runs away, unfolding
$\bangp{P}$ eagerly. A lazier and more implementable replication
operator, restricted to input-guarded processes, may be obtained as follows.

\begin{eqnarray}
\bangp{\prefix{u}{v}{P}} 
	:= 
	\binpar{\lift{x}{\prefix{u}{v}{(\binpar{D(x)}{P})}}}{D(x)} \nonumber
\end{eqnarray}

\begin{remark}
  Note that the lazier definition still does not deal with summation
  or mixed summation (i.e. sums over input and output). The reader is
  invited to construct definitions of replication that deal with these
  features. 

  Further, the definitions are parameterized in a name, $x$. Can you,
  gentle reader, make a definition that eliminates this parameter and
  guarantees no accidental interaction between the replication
  machinery and the process being replicated -- i.e. no accidental
  sharing of names used by the process to get its work done and the
  name(s) used by the replication to effect copying. This latter
  revision of the definition of replication is crucial to obtaining
  the expected identity $!!P \sim !P$.
\end{remark}

\begin{remark}\label{rem:paradoxical_combinator}
  The reader familiar with the lambda calculus will have noticed the
  similarity between $D$ and the paradoxical combinator.

  [Ed. note: the existence of this seems to suggest we have to be more
  restrictive on the set of processes and names we admit if we are to
  support no-cloning.]
\end{remark}

\subsubsection{Bisimulation}

The computational dynamics gives rise to another kind of equivalence,
the equivalence of computational behavior. As previously mentioned
this is typically captured \emph{via} some form of bisimulation.

% The notion we use in this paper is weak barbed bisimulation
% \cite{milner91polyadicpi}.

The notion we use in this paper is derived from weak barbed
bisimulation \cite{milner91polyadicpi}. 

\begin{definition}
An \emph{observation relation}, $\downarrow_{\mathcal N}$, over a set
of names, $\mathcal N$, is the smallest relation satisfying the rules
below.

\infrule[Out-barb]{y \in {\mathcal N}, \; x \nameeq y}
		  {\outputp{x}{v} \downarrow_{\mathcal N} x}
\infrule[Par-barb]{\mbox{$P\downarrow_{\mathcal N} x$ or $Q\downarrow_{\mathcal N} x$}}
		  {\binpar{P}{Q} \downarrow_{\mathcal N} x}

We write $P \Downarrow_{\mathcal N} x$ if there is $Q$ such that 
$P \wred Q$ and $Q \downarrow_{\mathcal N} x$.
\end{definition}

\begin{definition}
%\label{def.bbisim}
An  ${\mathcal N}$-\emph{barbed bisimulation} over a set of names, ${\mathcal N}$, is a symmetric binary relation 
${\mathcal S}_{\mathcal N}$ between agents such that $P\rel{S}_{\mathcal N}Q$ implies:
\begin{enumerate}
\item If $P \red P'$ then $Q \wred Q'$ and $P'\rel{S}_{\mathcal N} Q'$.
\item If $P\downarrow_{\mathcal N} x$, then $Q\Downarrow_{\mathcal N} x$.
\end{enumerate}
$P$ is ${\mathcal N}$-barbed bisimilar to $Q$, written
$P \wbbisim_{\mathcal N} Q$, if $P \rel{S}_{\mathcal N} Q$ for some ${\mathcal N}$-barbed bisimulation ${\mathcal S}_{\mathcal N}$.
\end{definition}

$\mathcal{R} \subseteq \pi \times \pi$

$P \mathcal{R} Q => \forall P'. P \red P' \Rightarrow \exists Q'. Q \red Q', P' \mathcal{R} Q'$

$P \vdash x \Rightarrow Q \vdash x$

\begin{mathpar}
  \inferrule*[lab=Out-barb]{x \nameeq y}{{y}!\langle{Q}\rangle \vdash x}
  \and
  \inferrule*[lab=Par-barb]{\mbox{$P\vdash x$ or $Q\vdash x$}}{\binpar{P}{Q} \vdash x}
\end{mathpar}

\subsubsection{Contexts}

One of the principle advantages of computational calculi like the
$\pi$-calculus is a well-defined notion of context,
contextual-equivalence and a correlation between
contextual-equivalence and notions of bisimulation. The notion of
context allows the decomposition of a process into (sub-)process and
its syntactic environment, its context. Thus, a context may be
thought of as a process with a ``hole'' (written $\Box$) in it. The
application of a context $M$ to a process $P$, written $M[P]$, is
tantamount to filling the hole in $M$ with $P$. In this paper we do
not need the full weight of this theory, but do make use of the notion
of context in the proof the main theorem. 

\begin{mathpar}
  \inferrule* [lab=summation] {} {{M_{M},M_{N}} \bc \Box \;|\; x.M_{A} \;|\; M_{M}+M_{N}}
  \and
  \inferrule* [lab=agent] {} {{M_{A}} \bc (\vec{x})M_{P} \;| \; \clift{P_0,\ldots,M_{P},\ldots,P_N}}
  \and \\
  \inferrule* [lab=process] {} {{M_{P}} \bc M_{N} \;| \;P|M_{P} }
\end{mathpar} 

\begin{mathpar}
  \inferrule* [lab=sychronization] {} {M_{N} \bc \Box \;|\; x?M_{F} \;|\; x!M_{C}}
  \and
  \inferrule* [lab=abstraction] {} {{M_{F}} \bc (x)M_{P} }
  \and
  \inferrule* [lab=concretion] {} {{M_{C}} \bc \langle M_{P} \rangle }
  \and \\
  \inferrule* [lab=process] {} {{M_{P}} \bc M_{N} \;| \;P|M_{P} }
\end{mathpar}

\begin{definition}[contextual application] Given a context $M$, and
  process $P$, we define the \emph{contextual application}, $M[P] :=
  M\{P/\Box\}$. That is, the contextual application of M to P is the
  substitution of $P$ for $\Box$ in $M$.
\end{definition}

$\meaningof{-} : L \to \mathcal{P}(\pi)$

\begin{mathpar}
  \inferrule* [lab=collection] {} {\meaningof{true} = \pi, \and \meaningof{~E} = \pi \setminus \meaningof{E}, \and \meaningof{E_{1} \& E_{2}} = \meaningof{E_{1}} \cap \meaningof{E_{2}}}
\end{mathpar}

\begin{mathpar}
  \inferrule* [lab=structure] {} {\meaningof{0} = \{ P \in \pi | P \equiv 0 \}, \and \\ \meaningof{E_1 | E_2} = \{ P \in \pi | P \equiv P_{1} | P_{2}, P_{1} \in \meaningof{E_{1}}, P_{2} \in \meaningof{E_2}\} }
\end{mathpar}

\begin{mathpar}
 \inferrule* [lab=behavior] {} {\meaningof{\langle a?b \rangle E} = \{ P \in \pi | P \equiv Q | u?(y)P', \\ \and \\\\ \and \\ \;\;\; u \in \meaningof{a}, \forall z.P'\{z/y\} \in \meaningof{E\{z/b\}}\}, \and \\ \meaningof{a!E} = \{ P \in \pi | P \equiv Q | x!\langle P' \rangle, x \in \meaningof{a} P' \in \meaningof{E}\} }
\end{mathpar}

\begin{mathpar}
 \inferrule* [lab=nominal] {} {\meaningof{\quotep{E}} = \{ \quotep{P} \in \quotep{\pi} | P \in \meaningof{E} \}, \and \meaningof{\quotep{P}} = \{ \quotep{Q} \in \quotep{\pi} | P \equiv Q \} \and \\ \meaningof{@\quotep{E}} = \{ P \in \pi | P \equiv @x, x \in \meaningof{E} \}}
\end{mathpar}

\begin{eqnarray*}
  \\
  \meaningof{-} : TS \to ST
\end{eqnarray*}

\begin{eqnarray*}
  \\
  L : TS \to ST
\end{eqnarray*}

\begin{eqnarray*}
  \\
  P \models E \iff P \in \meaningof{E}
\end{eqnarray*}

\begin{eqnarray*}
  P \approx_{L} Q \iff \forall E \in L. P \models E \iff Q \models E
\end{eqnarray*}

\begin{eqnarray*}
  P \approx_{K} Q
\end{eqnarray*}

\begin{eqnarray*}
  P \approx Q
\end{eqnarray*}

$\approx_{K} = \approx = \approx_{L}$

\subsubsection{Contextual duality}

Note that contexts extend the quotation operation to a family of
operations from processes to names. Given a context, $M$, we can
define a \emph{nominal context}, $\quotep{M}$ by $\quotep{M}[P] :=
\quotep{M[P]}$. To foreshadow what is to come we observe that these
operations enjoy a duality with processes very much like the duality
between vectors and maps from vectors to scalars.

Further, because the calculus is essentially higher-order, we have a
correspondence between contexts and processes. More specifically,
given a name $x$ and a context $M$ we can construct $M^{*}_{x}$ such
that 

\begin{mathpar}
  M^{*}_{x} | \lift{x}{P} \red M[P]
\end{mathpar}

namely,

\begin{mathpar}
  M^{*}_{x} := x?(u).M[\dropn{u}]
\end{mathpar}

The dependence of $M^{*}_{x}$ on a name makes it an abstraction, 

\begin{mathpar}
  M^{*} := (x)x?(u).M[\dropn{u}]
\end{mathpar}

\subsection{Additional notation}

It will sometimes be convenient to denote the process a name
quotes. We already have the notation $x = \quotep{P}$, but it will be
convenient to introduce an alternate notation, $\procn{x}$, when we
want to emphasize the connection to the use of the name. Note that, by
virtue of name equivalence, $\quotep{\procn{x}} \nameeq x$; so, the
notation is consistent with previous definitions.

Further, because names have structure it is possible to effect
substitutions on the basis of that structure. This means we need to
upgrade our notation for substitutions, which we accomplish by
adapting comprehension notation. Thus,

\begin{mathpar}
  P\{ y / x : x \in S \}
\end{mathpar}

is interpreted to mean the process derived from P by replacing (in a
capture-avoiding manner) each occurrence of $x$ in $S$ by $y$. For example,

\begin{mathpar}
  P\{ \quotep{\procn{x}|\procn{x}} / x : x \in \freenames{P} \}
\end{mathpar}

will replace each (occurrence) of a free name $x$ in $P$ by
$\quotep{\procn{x}|\procn{x}}$.

Also, we will avail ourselves of the notation $x^{L}$ and $x^{R}$ to
denote injections of a name into disjoint copies of the name
space. There are numerous ways to accomplish this. One example can be
found in \cite{MeredithR05}. This notation overloads to vectors of
names: $\vec{x}^{\pi} := (x_{i}^{\pi} \; : \; 0 \leq i < |\vec{x}| )$ where $\pi \in \{L,R\}$.

We also use $P^{\Box} := P|\Box$.

In \cite{MeredithR05} an interpretation of the new operator is
given. It turns out that there are several possible interpretations
all enjoying the requisite algebraic properties of the operator (see
\cite{milner91polyadicpi}). We will therefore make liberal use of
$(\nu\; \vec{x})P$.

% subsection the_syntax_and_semantics_of_the_notation_system (end)   

\input{qm2pi.qmops} 

\input{qm2pi.sterngerlach} 

\input{qm2pi.metric} 

% section concurrent_process_calculi (end)

%\input{qm2pi.proofsketch}

% section proof sketch (end)

%\input{qm2pi.slviaknots} 

% section spatial logic via knots (end)

\input{qm2pi.conclusion}

% section conclusion (end)

%\input{qm2pi.dtcodes} 

% section wiring algorithm (end)

\input{qm2pi.ack} 

% section acknowledgments (end)

\newpage


\bibliographystyle{plain}   
\bibliography{../../biblios/main.bib}

\input{qm2pi.rhodetails}

\end{document}

 

% section acknowledgments (end)

\newpage


\bibliographystyle{plain}   
\bibliography{../../biblios/main.bib}

\documentclass[12pt]{llncs}
%\documentclass{jktr}

\usepackage[pdftex]{hyperref}                   
\usepackage {listings}
\usepackage {mathpartir}
\usepackage{bcprules}
%\usepackage{listings}
                       
\usepackage{graphicx} 
%\usepackage[margins=2.5cm,nohead,nofoot]{geometry}
%\usepackage{geometry}
\usepackage{amsfonts}
\usepackage{amstext}
\usepackage{latexsym}
\usepackage{amssymb}
\usepackage{color}


%\include{myPreamble}
\include{qm2pi.local} 

%\ifpdf
%\usepackage[pdftex]{graphicx}
%\else
%\usepackage{graphicx}
%\fi

 % \ifpdf
%  \usepackage{pdfsync}
%  \if


%\title{Brief Article}
%\author{David F. Snyder}
%\author{L.G. Meredith}

%\address{Dept. of Math., Texas State University--San Marcos, San Marcos, TX 78666}
       
\pagestyle{empty}


\begin{document}

\lstset{language=[Objective]Caml,frame=shadowbox}

\input{qm2pi.front}

% section front matter (end)

\input{qm2pi.intro} 
 
% section introduction (end)

% \input{qm2pi.knotations} 

% section notation (end)

\input{qm2pi.process.calculi} 

% section concurrent_process_calculi_and_spatial_logics_ (end)
    
%\input{qm2pi.knots2pi} 

%\input{qm2pi.trefoil} 

%\input{qm2pi.mainthm} 

% subsection basic_interpretation (end)

%\input{qm2pi.rho.presentation} 
\subsection{The syntax and semantics of the notation system}\label{sub:the_syntax_and_semantics_of_the_notation_system} % (fold)

We now summarize a technical presentation of the calculus that
embodies our theory of dynamics. The typical presentation of such a
calculus follows the style of giving generators and relations on
them. The grammar, below, describing term constructors, freely
generates the set of processes, $\Proc$. This set is then quotiented
by a relation known as structural congruence and it is over this set
that the notion of dynamics is expressed. This presentation is
essentially that of \cite{MeredithR05} with the addition of
polyadicity and summation. For readability we have relegated some of
the technical subtleties to an appendix.

\subsubsection{Process grammar}\label{subsub:process_grammar}

\begin{mathpar}
  \inferrule* [lab=synchronization] {} {{M} \bc \pzero \;|\; x?F \;|\; x!C }
  \and
  \inferrule* [lab=abstraction] {} {{F} \bc (x)P}
  \and
  \inferrule* [lab=concretion] {} {{C} \bc \langle Q \rangle}
  \and
  \inferrule* [lab=process] {} {{P,Q} \bc M \;| \;P|Q \;|\; @{x}}
  \and
  \inferrule* [lab=name] {} {{x} \bc \quotep{P}}
\end{mathpar} 

Note that $\vec{x}$ (resp. $\vec{P}$) denotes a vector of names
(resp. processes) of length $|\vec{x}|$ (resp. $|\vec{P}|$). We adopt
the following useful abbreviations.

\begin{mathpar}
   x?(\vec{y}).P := x.(\vec{y})P \and  x\clift{\vec{P}} := x.\clift{\vec{P}}
   \and x!(y) := \lift{x}{\dropn{y}}
   \and \Pi_{i=0}^{n-1}P_i := P_0 | \ldots | P_{n-1}
\end{mathpar}

\subsubsection{Structural congruence}

\paragraph{Free and bound names and alpha-equivalence.} At the
core of structural equivalence is alpha-equivalence which identifies
process that are the same up to a change of variable. Formally, we
recognize the distinction between free and bound names. The free names
of a process, $\freenames{P}$, may be calculated recursively as
follows:

\begin{mathpar}
\freenames{\pzero} := \emptyset
  \and \\
  \freenames{x?(y).P} := \{ x \} \cup (\freenames{P} \setminus \{ y \})
  \and 
  \freenames{x!\langle P \rangle} := \{ x \} \cup \{ P \} 
  \and \\
  \freenames{P|Q} := \freenames{P} \cup \freenames{Q}
  \and \\
  \freenames{@{x}} := \{ x \}
\end{mathpar}

$\pi$
$\quotep{\pi}$

$\freenames{-} : \pi \to \mathcal{P}(\quotep{\pi})$

\begin{eqnarray*}
  \freenames{\pzero} & := & \emptyset \\
  \freenames{x?(y).P} & := & \{ x \} \cup (\freenames{P} \setminus \{ y \}) \\
  \freenames{x!\langle P \rangle} & := & \{ x \} \cup \{ P \} \\
  \freenames{P|Q} & := & \freenames{P} \cup \freenames{Q} \\
  \freenames{\dropn{x}} & := & \{ x \}
\end{eqnarray*}

The bound names of a process, $\boundnames{P}$, are those names occurring in $P$
that are not free. For example, in $x?(y).0$, the name $x$ is free, while $y$ is bound.

\begin{mathpar}
  \inferrule* [lab=monoidal-laws] {} { P|Q \equiv Q|P \and P|0 \equiv P \and P|(Q|R) \equiv (P|Q)|R }
\end{mathpar}

\begin{mathpar}
  \inferrule* [lab=alpha-equivalence] {} { (x)P \equiv (y)P\{y/x\} \and y \not\in \freenames{P} }
\end{mathpar}

\begin{definition}
Then two processes, $P,Q$, are alpha-equivalent if $P = Q\{\vec{y}/\vec{x}\}$ for
some $\vec{x} \in \boundnames{Q},\vec{y} \in \boundnames{P}$, where $Q\{\vec{y}/\vec{x}\}$
denotes the capture-avoiding substitution of $\vec{y}$ for $\vec{x}$ in $Q$.
\end{definition}

\begin{definition}
  The {\em structural congruence} \cite{SangiorgiWalker} , $\equiv$,
  between processes is the least congruence containing
  alpha-equivalence, satisfying the abelian monoid laws
  (associativity, commutativity and $\pzero$ as identity) for parallel
  composition $|$ and for summation $+$.
\end{definition}

\subsection{Name equivalence}

We take name equivalence, written $\nameeq$, to be the smallest
equivalence relation generated by the following rules.

\begin{mathpar}
\inferrule*[lab=Quote-drop]
{ }
{ \quotep{@{x}} \nameeq x }

\inferrule*[lab=Struct-equiv]
{ P \scong Q }
{ \quotep{P} \nameeq \quotep{Q} }
\end{mathpar}

The astute reader will have noticed that the mutual recursion of names
and processes imposes a mutual recursion on alpha-equivalence and
structural equivalence via name-equivalence. Fortunately, all of this
works out pleasantly and we may calculate in the natural way, free of
concern. The reader interested in the details is referred to the
appendix \ref{appendix:rho_details}.

\subsection{Substitution}

We use $\Proc$ for the set of processes, $\QProc$ for the set of
names, and $\id{\{}\vec{y} / \vec{x} \id{\}}$ to denote partial maps,
$s : \QProc \rightarrow \QProc$. A map, $s$ lifts, uniquely, to a map
on process terms, $\widehat{s} : \Proc \rightarrow \Proc$ by the
following equations.

\begin{mathpar}
  (0) \psubstp{Q}{P} := 0 \\
  (R \juxtap S) \psubstp{Q}{P}
  :=    
  (R)\psubstp{Q}{P} \juxtap (S) \psubstp{Q}{P} \\
  (x?(y).R) \psubstp{Q}{P}    
  :=    
  (x)\substp{Q}{P} (z)\concat( (R \psubstn{z}{y}) \psubstp{Q}{P} ) \\
  (\lift{x}{R}) \psubstp{Q}{P}  
  :=
  \lift{(x)\substp{Q}{P}}{ R \psubstp{Q}{P} } \\
%   (\dropn{x})  \psubstp{Q}{P}       
%   := 
%   \left\{ 
%     \begin{array}{ccc} 
%       \dropn{\quotep{Q}} & & x \nameeq \quotep{P} \\
%       \dropn{x} & & otherwise \\
%     \end{array}
%   \right. 
  (\dropn{x})  \psubstp{Q}{P}       
  := 
  \left\{ 
    \begin{array}{ccc} 
      Q & & x \nameeq \quotep{P} \\
      \dropn{x} & & otherwise \\
    \end{array}
  \right.
\end{mathpar}
 

where

\begin{eqnarray}
  (x)\id{\{} \lpquote Q \rpquote / \lpquote P \rpquote \id{\}}            = 
  \left\{ 
    \begin{array}{ccc}
      \lpquote Q \rpquote & & x \nameeq \lpquote P \rpquote \\
      x & & otherwise \\
    \end{array}
  \right. \nonumber
\end{eqnarray}

and $z$ is chosen distinct from $\quotep{P}$, $\quotep{Q}$, the free
names in $Q$, and all the names in $R$. Our $\alpha$-equivalence will
be built in the standard way from this substitution.

\begin{remark}\label{rem:no_self_referential_names}
  One consequence of these definitions is that $\forall P. \quotep{P}
  \not\in \freenames{P}$.
\end{remark}

\subsection{ Dynamic quote: an example }

Anticipating something of what's to come, consider applying the
substitution, $\widehat{\id{\{}u / z \id{\}}}$, to the following pair
of processes, $\lift{w}{y!(z)}$ and $w[ \lpquote y!(z) \rpquote ]$.

\begin{eqnarray}
	\lift{w}{y!(z)}\widehat{\id{\{}u / z \id{\}}}
		& = &
		\lift{w}{y!(u)} \nonumber\\
	w[ \lpquote y!(z) \rpquote ] \widehat{ \id{\{}u / z \id{\}} }
		& = &
		w[ \lpquote y!(z) \rpquote ] \nonumber
\end{eqnarray}

Because the body of the process between quotes is impervious to
substitution, we get radically different answers. In fact, by
examining the first process in an input context,
e.g. $x?(z).\lift{w}{y!(z)}$, we see that the process under the lift
operator may be shaped by prefixed inputs binding a name inside it. In
this sense, the lift operator will be seen as a way to dynamically
construct processes before reifying them as names.

Finally equipped with these standard features we can present the
dynamics of the calculus.

\subsubsection{Operational semantics} 

Finally, we introduce the computational dynamics. What marks these
algebras as distinct from other more traditionally studied algebraic
structures, e.g. vector spaces or polynomial rings, is the manner in
which dynamics is captured. In traditional structures, dynamics is typically
expressed through morphisms between such structures, as in linear maps
between vector spaces or morphisms between rings. In algebras
associated with the semantics of computation, the dynamics is
expressed as part of the algebraic structure itself, through a
reduction reduction relation typically denoted by $\red$. Below, we
give a recursive presentation of this relation for the calculus used
in the encoding.

$\red \subseteq \pi \times \pi$
$\red : \pi \to \mathcal{P}(\pi)$

\begin{mathpar}
  \inferrule* [lab=Comm] { \textsf{match}( x_{src}, x_{trgt} ) } { x_{trgt}?(y)P \; | \; x_{src}!\langle {Q} \rangle \red P\{\quotep{Q}/y}\} }
  \and \\
  \inferrule* [lab=Par] {{P} \red {P}'} {{{P} | {Q}} \red {{P}' | {Q}}}
  \and
  \inferrule* [lab=Equiv]{{{P} \scong {P}'} \andalso {{P}' \red {Q}'} \andalso {{Q}' \scong {Q}}}{{P} \red {Q}}
\end{mathpar}

\begin{eqnarray*}
  match_{\equiv} (\quotep{P},\quotep{Q}) & := & P \equiv Q \\
  match_{\dagger}(\quotep{P},\quotep{Q}) & := & \forall R. P|Q \red^{*} R => R \red^{*} 0 \\
  match_{K}(\quotep{P},\quotep{Q}) & := & K \mbox{ for some context } K
\end{eqnarray*}

$u?(x)P | u!\langle Q \rangle \red P\{\quotep{Q}/x\}$

%We write $\wred$ for $\red^*$, and $P\red$ if $\exists Q $ such that $ P \red Q$.
We write $P\red$ if $\exists Q $ such that $ P \red Q$ and $P\not\red$, otherwise.

\section{Replication}

As mentioned before, it is known that replication (and hence
recursion) can be implemented in a higher-order process algebra
\cite{SangiorgiWalker}. As our first example of calculation with the
machinery thus far presented we give the construction explicitly in
the {\rhoc}.

\begin{eqnarray}
	D_{x} & := & \prefix{x}{y}{(\binpar{\outputp{x}{y}}{@{y}})} \nonumber\\
	\bangp_{x}{P} & := & \binpar{{x}!\langle{\binpar{D_{x}}{P}}\rangle}{D_{x}} \nonumber
\end{eqnarray}

\begin{eqnarray}
	\bangp_{x}{P} & & \nonumber\\
	=
	& {x}!\langle{(\prefix{x}{y}{(\outputp{x}{y} | @{y})) | P}}\rangle 
	      | \prefix{x}{y}{(\outputp{x}{y} | @{y})} & \nonumber\\
	\red
	& (\outputp{x}{y} | @{y})\substn{\quotep{(\prefix{x}{y}{(@{y} | \outputp{x}{y})) | P}}}{y} & \nonumber\\
	=
	& \outputp{x}{\quotep{(\prefix{x}{y}{(\outputp{x}{y} | @{y})) | P}}}
	  | {(\prefix{x}{y}{(\outputp{x}{y} | @{y})) | P}} & \nonumber\\
	\red
	& \ldots & \nonumber\\
	\red^*
	& P | P | \ldots & \nonumber
\end{eqnarray}

Of course, this encoding, as an implementation, runs away, unfolding
$\bangp{P}$ eagerly. A lazier and more implementable replication
operator, restricted to input-guarded processes, may be obtained as follows.

\begin{eqnarray}
\bangp{\prefix{u}{v}{P}} 
	:= 
	\binpar{\lift{x}{\prefix{u}{v}{(\binpar{D(x)}{P})}}}{D(x)} \nonumber
\end{eqnarray}

\begin{remark}
  Note that the lazier definition still does not deal with summation
  or mixed summation (i.e. sums over input and output). The reader is
  invited to construct definitions of replication that deal with these
  features. 

  Further, the definitions are parameterized in a name, $x$. Can you,
  gentle reader, make a definition that eliminates this parameter and
  guarantees no accidental interaction between the replication
  machinery and the process being replicated -- i.e. no accidental
  sharing of names used by the process to get its work done and the
  name(s) used by the replication to effect copying. This latter
  revision of the definition of replication is crucial to obtaining
  the expected identity $!!P \sim !P$.
\end{remark}

\begin{remark}\label{rem:paradoxical_combinator}
  The reader familiar with the lambda calculus will have noticed the
  similarity between $D$ and the paradoxical combinator.

  [Ed. note: the existence of this seems to suggest we have to be more
  restrictive on the set of processes and names we admit if we are to
  support no-cloning.]
\end{remark}

\subsubsection{Bisimulation}

The computational dynamics gives rise to another kind of equivalence,
the equivalence of computational behavior. As previously mentioned
this is typically captured \emph{via} some form of bisimulation.

% The notion we use in this paper is weak barbed bisimulation
% \cite{milner91polyadicpi}.

The notion we use in this paper is derived from weak barbed
bisimulation \cite{milner91polyadicpi}. 

\begin{definition}
An \emph{observation relation}, $\downarrow_{\mathcal N}$, over a set
of names, $\mathcal N$, is the smallest relation satisfying the rules
below.

\infrule[Out-barb]{y \in {\mathcal N}, \; x \nameeq y}
		  {\outputp{x}{v} \downarrow_{\mathcal N} x}
\infrule[Par-barb]{\mbox{$P\downarrow_{\mathcal N} x$ or $Q\downarrow_{\mathcal N} x$}}
		  {\binpar{P}{Q} \downarrow_{\mathcal N} x}

We write $P \Downarrow_{\mathcal N} x$ if there is $Q$ such that 
$P \wred Q$ and $Q \downarrow_{\mathcal N} x$.
\end{definition}

\begin{definition}
%\label{def.bbisim}
An  ${\mathcal N}$-\emph{barbed bisimulation} over a set of names, ${\mathcal N}$, is a symmetric binary relation 
${\mathcal S}_{\mathcal N}$ between agents such that $P\rel{S}_{\mathcal N}Q$ implies:
\begin{enumerate}
\item If $P \red P'$ then $Q \wred Q'$ and $P'\rel{S}_{\mathcal N} Q'$.
\item If $P\downarrow_{\mathcal N} x$, then $Q\Downarrow_{\mathcal N} x$.
\end{enumerate}
$P$ is ${\mathcal N}$-barbed bisimilar to $Q$, written
$P \wbbisim_{\mathcal N} Q$, if $P \rel{S}_{\mathcal N} Q$ for some ${\mathcal N}$-barbed bisimulation ${\mathcal S}_{\mathcal N}$.
\end{definition}

$\mathcal{R} \subseteq \pi \times \pi$

$P \mathcal{R} Q => \forall P'. P \red P' \Rightarrow \exists Q'. Q \red Q', P' \mathcal{R} Q'$

$P \vdash x \Rightarrow Q \vdash x$

\begin{mathpar}
  \inferrule*[lab=Out-barb]{x \nameeq y}{{y}!\langle{Q}\rangle \vdash x}
  \and
  \inferrule*[lab=Par-barb]{\mbox{$P\vdash x$ or $Q\vdash x$}}{\binpar{P}{Q} \vdash x}
\end{mathpar}

\subsubsection{Contexts}

One of the principle advantages of computational calculi like the
$\pi$-calculus is a well-defined notion of context,
contextual-equivalence and a correlation between
contextual-equivalence and notions of bisimulation. The notion of
context allows the decomposition of a process into (sub-)process and
its syntactic environment, its context. Thus, a context may be
thought of as a process with a ``hole'' (written $\Box$) in it. The
application of a context $M$ to a process $P$, written $M[P]$, is
tantamount to filling the hole in $M$ with $P$. In this paper we do
not need the full weight of this theory, but do make use of the notion
of context in the proof the main theorem. 

\begin{mathpar}
  \inferrule* [lab=summation] {} {{M_{M},M_{N}} \bc \Box \;|\; x.M_{A} \;|\; M_{M}+M_{N}}
  \and
  \inferrule* [lab=agent] {} {{M_{A}} \bc (\vec{x})M_{P} \;| \; \clift{P_0,\ldots,M_{P},\ldots,P_N}}
  \and \\
  \inferrule* [lab=process] {} {{M_{P}} \bc M_{N} \;| \;P|M_{P} }
\end{mathpar} 

\begin{mathpar}
  \inferrule* [lab=sychronization] {} {M_{N} \bc \Box \;|\; x?M_{F} \;|\; x!M_{C}}
  \and
  \inferrule* [lab=abstraction] {} {{M_{F}} \bc (x)M_{P} }
  \and
  \inferrule* [lab=concretion] {} {{M_{C}} \bc \langle M_{P} \rangle }
  \and \\
  \inferrule* [lab=process] {} {{M_{P}} \bc M_{N} \;| \;P|M_{P} }
\end{mathpar}

\begin{definition}[contextual application] Given a context $M$, and
  process $P$, we define the \emph{contextual application}, $M[P] :=
  M\{P/\Box\}$. That is, the contextual application of M to P is the
  substitution of $P$ for $\Box$ in $M$.
\end{definition}

$\meaningof{-} : L \to \mathcal{P}(\pi)$

\begin{mathpar}
  \inferrule* [lab=collection] {} {\meaningof{true} = \pi, \and \meaningof{~E} = \pi \setminus \meaningof{E}, \and \meaningof{E_{1} \& E_{2}} = \meaningof{E_{1}} \cap \meaningof{E_{2}}}
\end{mathpar}

\begin{mathpar}
  \inferrule* [lab=structure] {} {\meaningof{0} = \{ P \in \pi | P \equiv 0 \}, \and \\ \meaningof{E_1 | E_2} = \{ P \in \pi | P \equiv P_{1} | P_{2}, P_{1} \in \meaningof{E_{1}}, P_{2} \in \meaningof{E_2}\} }
\end{mathpar}

\begin{mathpar}
 \inferrule* [lab=behavior] {} {\meaningof{\langle a?b \rangle E} = \{ P \in \pi | P \equiv Q | u?(y)P', \\ \and \\\\ \and \\ \;\;\; u \in \meaningof{a}, \forall z.P'\{z/y\} \in \meaningof{E\{z/b\}}\}, \and \\ \meaningof{a!E} = \{ P \in \pi | P \equiv Q | x!\langle P' \rangle, x \in \meaningof{a} P' \in \meaningof{E}\} }
\end{mathpar}

\begin{mathpar}
 \inferrule* [lab=nominal] {} {\meaningof{\quotep{E}} = \{ \quotep{P} \in \quotep{\pi} | P \in \meaningof{E} \}, \and \meaningof{\quotep{P}} = \{ \quotep{Q} \in \quotep{\pi} | P \equiv Q \} \and \\ \meaningof{@\quotep{E}} = \{ P \in \pi | P \equiv @x, x \in \meaningof{E} \}}
\end{mathpar}

\begin{eqnarray*}
  \\
  \meaningof{-} : TS \to ST
\end{eqnarray*}

\begin{eqnarray*}
  \\
  L : TS \to ST
\end{eqnarray*}

\begin{eqnarray*}
  \\
  P \models E \iff P \in \meaningof{E}
\end{eqnarray*}

\begin{eqnarray*}
  P \approx_{L} Q \iff \forall E \in L. P \models E \iff Q \models E
\end{eqnarray*}

\begin{eqnarray*}
  P \approx_{K} Q
\end{eqnarray*}

\begin{eqnarray*}
  P \approx Q
\end{eqnarray*}

$\approx_{K} = \approx = \approx_{L}$

\subsubsection{Contextual duality}

Note that contexts extend the quotation operation to a family of
operations from processes to names. Given a context, $M$, we can
define a \emph{nominal context}, $\quotep{M}$ by $\quotep{M}[P] :=
\quotep{M[P]}$. To foreshadow what is to come we observe that these
operations enjoy a duality with processes very much like the duality
between vectors and maps from vectors to scalars.

Further, because the calculus is essentially higher-order, we have a
correspondence between contexts and processes. More specifically,
given a name $x$ and a context $M$ we can construct $M^{*}_{x}$ such
that 

\begin{mathpar}
  M^{*}_{x} | \lift{x}{P} \red M[P]
\end{mathpar}

namely,

\begin{mathpar}
  M^{*}_{x} := x?(u).M[\dropn{u}]
\end{mathpar}

The dependence of $M^{*}_{x}$ on a name makes it an abstraction, 

\begin{mathpar}
  M^{*} := (x)x?(u).M[\dropn{u}]
\end{mathpar}

\subsection{Additional notation}

It will sometimes be convenient to denote the process a name
quotes. We already have the notation $x = \quotep{P}$, but it will be
convenient to introduce an alternate notation, $\procn{x}$, when we
want to emphasize the connection to the use of the name. Note that, by
virtue of name equivalence, $\quotep{\procn{x}} \nameeq x$; so, the
notation is consistent with previous definitions.

Further, because names have structure it is possible to effect
substitutions on the basis of that structure. This means we need to
upgrade our notation for substitutions, which we accomplish by
adapting comprehension notation. Thus,

\begin{mathpar}
  P\{ y / x : x \in S \}
\end{mathpar}

is interpreted to mean the process derived from P by replacing (in a
capture-avoiding manner) each occurrence of $x$ in $S$ by $y$. For example,

\begin{mathpar}
  P\{ \quotep{\procn{x}|\procn{x}} / x : x \in \freenames{P} \}
\end{mathpar}

will replace each (occurrence) of a free name $x$ in $P$ by
$\quotep{\procn{x}|\procn{x}}$.

Also, we will avail ourselves of the notation $x^{L}$ and $x^{R}$ to
denote injections of a name into disjoint copies of the name
space. There are numerous ways to accomplish this. One example can be
found in \cite{MeredithR05}. This notation overloads to vectors of
names: $\vec{x}^{\pi} := (x_{i}^{\pi} \; : \; 0 \leq i < |\vec{x}| )$ where $\pi \in \{L,R\}$.

We also use $P^{\Box} := P|\Box$.

In \cite{MeredithR05} an interpretation of the new operator is
given. It turns out that there are several possible interpretations
all enjoying the requisite algebraic properties of the operator (see
\cite{milner91polyadicpi}). We will therefore make liberal use of
$(\nu\; \vec{x})P$.

% subsection the_syntax_and_semantics_of_the_notation_system (end)   

\input{qm2pi.qmops} 

\input{qm2pi.sterngerlach} 

\input{qm2pi.metric} 

% section concurrent_process_calculi (end)

%\input{qm2pi.proofsketch}

% section proof sketch (end)

%\input{qm2pi.slviaknots} 

% section spatial logic via knots (end)

\input{qm2pi.conclusion}

% section conclusion (end)

%\input{qm2pi.dtcodes} 

% section wiring algorithm (end)

\input{qm2pi.ack} 

% section acknowledgments (end)

\newpage


\bibliographystyle{plain}   
\bibliography{../../biblios/main.bib}

\input{qm2pi.rhodetails}

\end{document}



\end{document}

 

%\ifpdf
%\usepackage[pdftex]{graphicx}
%\else
%\usepackage{graphicx}
%\fi

 % \ifpdf
%  \usepackage{pdfsync}
%  \if


%\title{Brief Article}
%\author{David F. Snyder}
%\author{L.G. Meredith}

%\address{Dept. of Math., Texas State University--San Marcos, San Marcos, TX 78666}
       
\pagestyle{empty}


\begin{document}

\lstset{language=[Objective]Caml,frame=shadowbox}

\documentclass[12pt]{llncs}
%\documentclass{jktr}

\usepackage[pdftex]{hyperref}                   
\usepackage {listings}
\usepackage {mathpartir}
\usepackage{bcprules}
%\usepackage{listings}
                       
\usepackage{graphicx} 
%\usepackage[margins=2.5cm,nohead,nofoot]{geometry}
%\usepackage{geometry}
\usepackage{amsfonts}
\usepackage{amstext}
\usepackage{latexsym}
\usepackage{amssymb}
\usepackage{color}


%\include{myPreamble}
\documentclass[12pt]{llncs}
%\documentclass{jktr}

\usepackage[pdftex]{hyperref}                   
\usepackage {listings}
\usepackage {mathpartir}
\usepackage{bcprules}
%\usepackage{listings}
                       
\usepackage{graphicx} 
%\usepackage[margins=2.5cm,nohead,nofoot]{geometry}
%\usepackage{geometry}
\usepackage{amsfonts}
\usepackage{amstext}
\usepackage{latexsym}
\usepackage{amssymb}
\usepackage{color}


%\include{myPreamble}
\include{qm2pi.local} 

%\ifpdf
%\usepackage[pdftex]{graphicx}
%\else
%\usepackage{graphicx}
%\fi

 % \ifpdf
%  \usepackage{pdfsync}
%  \if


%\title{Brief Article}
%\author{David F. Snyder}
%\author{L.G. Meredith}

%\address{Dept. of Math., Texas State University--San Marcos, San Marcos, TX 78666}
       
\pagestyle{empty}


\begin{document}

\lstset{language=[Objective]Caml,frame=shadowbox}

\input{qm2pi.front}

% section front matter (end)

\input{qm2pi.intro} 
 
% section introduction (end)

% \input{qm2pi.knotations} 

% section notation (end)

\input{qm2pi.process.calculi} 

% section concurrent_process_calculi_and_spatial_logics_ (end)
    
%\input{qm2pi.knots2pi} 

%\input{qm2pi.trefoil} 

%\input{qm2pi.mainthm} 

% subsection basic_interpretation (end)

%\input{qm2pi.rho.presentation} 
\subsection{The syntax and semantics of the notation system}\label{sub:the_syntax_and_semantics_of_the_notation_system} % (fold)

We now summarize a technical presentation of the calculus that
embodies our theory of dynamics. The typical presentation of such a
calculus follows the style of giving generators and relations on
them. The grammar, below, describing term constructors, freely
generates the set of processes, $\Proc$. This set is then quotiented
by a relation known as structural congruence and it is over this set
that the notion of dynamics is expressed. This presentation is
essentially that of \cite{MeredithR05} with the addition of
polyadicity and summation. For readability we have relegated some of
the technical subtleties to an appendix.

\subsubsection{Process grammar}\label{subsub:process_grammar}

\begin{mathpar}
  \inferrule* [lab=synchronization] {} {{M} \bc \pzero \;|\; x?F \;|\; x!C }
  \and
  \inferrule* [lab=abstraction] {} {{F} \bc (x)P}
  \and
  \inferrule* [lab=concretion] {} {{C} \bc \langle Q \rangle}
  \and
  \inferrule* [lab=process] {} {{P,Q} \bc M \;| \;P|Q \;|\; @{x}}
  \and
  \inferrule* [lab=name] {} {{x} \bc \quotep{P}}
\end{mathpar} 

Note that $\vec{x}$ (resp. $\vec{P}$) denotes a vector of names
(resp. processes) of length $|\vec{x}|$ (resp. $|\vec{P}|$). We adopt
the following useful abbreviations.

\begin{mathpar}
   x?(\vec{y}).P := x.(\vec{y})P \and  x\clift{\vec{P}} := x.\clift{\vec{P}}
   \and x!(y) := \lift{x}{\dropn{y}}
   \and \Pi_{i=0}^{n-1}P_i := P_0 | \ldots | P_{n-1}
\end{mathpar}

\subsubsection{Structural congruence}

\paragraph{Free and bound names and alpha-equivalence.} At the
core of structural equivalence is alpha-equivalence which identifies
process that are the same up to a change of variable. Formally, we
recognize the distinction between free and bound names. The free names
of a process, $\freenames{P}$, may be calculated recursively as
follows:

\begin{mathpar}
\freenames{\pzero} := \emptyset
  \and \\
  \freenames{x?(y).P} := \{ x \} \cup (\freenames{P} \setminus \{ y \})
  \and 
  \freenames{x!\langle P \rangle} := \{ x \} \cup \{ P \} 
  \and \\
  \freenames{P|Q} := \freenames{P} \cup \freenames{Q}
  \and \\
  \freenames{@{x}} := \{ x \}
\end{mathpar}

$\pi$
$\quotep{\pi}$

$\freenames{-} : \pi \to \mathcal{P}(\quotep{\pi})$

\begin{eqnarray*}
  \freenames{\pzero} & := & \emptyset \\
  \freenames{x?(y).P} & := & \{ x \} \cup (\freenames{P} \setminus \{ y \}) \\
  \freenames{x!\langle P \rangle} & := & \{ x \} \cup \{ P \} \\
  \freenames{P|Q} & := & \freenames{P} \cup \freenames{Q} \\
  \freenames{\dropn{x}} & := & \{ x \}
\end{eqnarray*}

The bound names of a process, $\boundnames{P}$, are those names occurring in $P$
that are not free. For example, in $x?(y).0$, the name $x$ is free, while $y$ is bound.

\begin{mathpar}
  \inferrule* [lab=monoidal-laws] {} { P|Q \equiv Q|P \and P|0 \equiv P \and P|(Q|R) \equiv (P|Q)|R }
\end{mathpar}

\begin{mathpar}
  \inferrule* [lab=alpha-equivalence] {} { (x)P \equiv (y)P\{y/x\} \and y \not\in \freenames{P} }
\end{mathpar}

\begin{definition}
Then two processes, $P,Q$, are alpha-equivalent if $P = Q\{\vec{y}/\vec{x}\}$ for
some $\vec{x} \in \boundnames{Q},\vec{y} \in \boundnames{P}$, where $Q\{\vec{y}/\vec{x}\}$
denotes the capture-avoiding substitution of $\vec{y}$ for $\vec{x}$ in $Q$.
\end{definition}

\begin{definition}
  The {\em structural congruence} \cite{SangiorgiWalker} , $\equiv$,
  between processes is the least congruence containing
  alpha-equivalence, satisfying the abelian monoid laws
  (associativity, commutativity and $\pzero$ as identity) for parallel
  composition $|$ and for summation $+$.
\end{definition}

\subsection{Name equivalence}

We take name equivalence, written $\nameeq$, to be the smallest
equivalence relation generated by the following rules.

\begin{mathpar}
\inferrule*[lab=Quote-drop]
{ }
{ \quotep{@{x}} \nameeq x }

\inferrule*[lab=Struct-equiv]
{ P \scong Q }
{ \quotep{P} \nameeq \quotep{Q} }
\end{mathpar}

The astute reader will have noticed that the mutual recursion of names
and processes imposes a mutual recursion on alpha-equivalence and
structural equivalence via name-equivalence. Fortunately, all of this
works out pleasantly and we may calculate in the natural way, free of
concern. The reader interested in the details is referred to the
appendix \ref{appendix:rho_details}.

\subsection{Substitution}

We use $\Proc$ for the set of processes, $\QProc$ for the set of
names, and $\id{\{}\vec{y} / \vec{x} \id{\}}$ to denote partial maps,
$s : \QProc \rightarrow \QProc$. A map, $s$ lifts, uniquely, to a map
on process terms, $\widehat{s} : \Proc \rightarrow \Proc$ by the
following equations.

\begin{mathpar}
  (0) \psubstp{Q}{P} := 0 \\
  (R \juxtap S) \psubstp{Q}{P}
  :=    
  (R)\psubstp{Q}{P} \juxtap (S) \psubstp{Q}{P} \\
  (x?(y).R) \psubstp{Q}{P}    
  :=    
  (x)\substp{Q}{P} (z)\concat( (R \psubstn{z}{y}) \psubstp{Q}{P} ) \\
  (\lift{x}{R}) \psubstp{Q}{P}  
  :=
  \lift{(x)\substp{Q}{P}}{ R \psubstp{Q}{P} } \\
%   (\dropn{x})  \psubstp{Q}{P}       
%   := 
%   \left\{ 
%     \begin{array}{ccc} 
%       \dropn{\quotep{Q}} & & x \nameeq \quotep{P} \\
%       \dropn{x} & & otherwise \\
%     \end{array}
%   \right. 
  (\dropn{x})  \psubstp{Q}{P}       
  := 
  \left\{ 
    \begin{array}{ccc} 
      Q & & x \nameeq \quotep{P} \\
      \dropn{x} & & otherwise \\
    \end{array}
  \right.
\end{mathpar}
 

where

\begin{eqnarray}
  (x)\id{\{} \lpquote Q \rpquote / \lpquote P \rpquote \id{\}}            = 
  \left\{ 
    \begin{array}{ccc}
      \lpquote Q \rpquote & & x \nameeq \lpquote P \rpquote \\
      x & & otherwise \\
    \end{array}
  \right. \nonumber
\end{eqnarray}

and $z$ is chosen distinct from $\quotep{P}$, $\quotep{Q}$, the free
names in $Q$, and all the names in $R$. Our $\alpha$-equivalence will
be built in the standard way from this substitution.

\begin{remark}\label{rem:no_self_referential_names}
  One consequence of these definitions is that $\forall P. \quotep{P}
  \not\in \freenames{P}$.
\end{remark}

\subsection{ Dynamic quote: an example }

Anticipating something of what's to come, consider applying the
substitution, $\widehat{\id{\{}u / z \id{\}}}$, to the following pair
of processes, $\lift{w}{y!(z)}$ and $w[ \lpquote y!(z) \rpquote ]$.

\begin{eqnarray}
	\lift{w}{y!(z)}\widehat{\id{\{}u / z \id{\}}}
		& = &
		\lift{w}{y!(u)} \nonumber\\
	w[ \lpquote y!(z) \rpquote ] \widehat{ \id{\{}u / z \id{\}} }
		& = &
		w[ \lpquote y!(z) \rpquote ] \nonumber
\end{eqnarray}

Because the body of the process between quotes is impervious to
substitution, we get radically different answers. In fact, by
examining the first process in an input context,
e.g. $x?(z).\lift{w}{y!(z)}$, we see that the process under the lift
operator may be shaped by prefixed inputs binding a name inside it. In
this sense, the lift operator will be seen as a way to dynamically
construct processes before reifying them as names.

Finally equipped with these standard features we can present the
dynamics of the calculus.

\subsubsection{Operational semantics} 

Finally, we introduce the computational dynamics. What marks these
algebras as distinct from other more traditionally studied algebraic
structures, e.g. vector spaces or polynomial rings, is the manner in
which dynamics is captured. In traditional structures, dynamics is typically
expressed through morphisms between such structures, as in linear maps
between vector spaces or morphisms between rings. In algebras
associated with the semantics of computation, the dynamics is
expressed as part of the algebraic structure itself, through a
reduction reduction relation typically denoted by $\red$. Below, we
give a recursive presentation of this relation for the calculus used
in the encoding.

$\red \subseteq \pi \times \pi$
$\red : \pi \to \mathcal{P}(\pi)$

\begin{mathpar}
  \inferrule* [lab=Comm] { \textsf{match}( x_{src}, x_{trgt} ) } { x_{trgt}?(y)P \; | \; x_{src}!\langle {Q} \rangle \red P\{\quotep{Q}/y}\} }
  \and \\
  \inferrule* [lab=Par] {{P} \red {P}'} {{{P} | {Q}} \red {{P}' | {Q}}}
  \and
  \inferrule* [lab=Equiv]{{{P} \scong {P}'} \andalso {{P}' \red {Q}'} \andalso {{Q}' \scong {Q}}}{{P} \red {Q}}
\end{mathpar}

\begin{eqnarray*}
  match_{\equiv} (\quotep{P},\quotep{Q}) & := & P \equiv Q \\
  match_{\dagger}(\quotep{P},\quotep{Q}) & := & \forall R. P|Q \red^{*} R => R \red^{*} 0 \\
  match_{K}(\quotep{P},\quotep{Q}) & := & K \mbox{ for some context } K
\end{eqnarray*}

$u?(x)P | u!\langle Q \rangle \red P\{\quotep{Q}/x\}$

%We write $\wred$ for $\red^*$, and $P\red$ if $\exists Q $ such that $ P \red Q$.
We write $P\red$ if $\exists Q $ such that $ P \red Q$ and $P\not\red$, otherwise.

\section{Replication}

As mentioned before, it is known that replication (and hence
recursion) can be implemented in a higher-order process algebra
\cite{SangiorgiWalker}. As our first example of calculation with the
machinery thus far presented we give the construction explicitly in
the {\rhoc}.

\begin{eqnarray}
	D_{x} & := & \prefix{x}{y}{(\binpar{\outputp{x}{y}}{@{y}})} \nonumber\\
	\bangp_{x}{P} & := & \binpar{{x}!\langle{\binpar{D_{x}}{P}}\rangle}{D_{x}} \nonumber
\end{eqnarray}

\begin{eqnarray}
	\bangp_{x}{P} & & \nonumber\\
	=
	& {x}!\langle{(\prefix{x}{y}{(\outputp{x}{y} | @{y})) | P}}\rangle 
	      | \prefix{x}{y}{(\outputp{x}{y} | @{y})} & \nonumber\\
	\red
	& (\outputp{x}{y} | @{y})\substn{\quotep{(\prefix{x}{y}{(@{y} | \outputp{x}{y})) | P}}}{y} & \nonumber\\
	=
	& \outputp{x}{\quotep{(\prefix{x}{y}{(\outputp{x}{y} | @{y})) | P}}}
	  | {(\prefix{x}{y}{(\outputp{x}{y} | @{y})) | P}} & \nonumber\\
	\red
	& \ldots & \nonumber\\
	\red^*
	& P | P | \ldots & \nonumber
\end{eqnarray}

Of course, this encoding, as an implementation, runs away, unfolding
$\bangp{P}$ eagerly. A lazier and more implementable replication
operator, restricted to input-guarded processes, may be obtained as follows.

\begin{eqnarray}
\bangp{\prefix{u}{v}{P}} 
	:= 
	\binpar{\lift{x}{\prefix{u}{v}{(\binpar{D(x)}{P})}}}{D(x)} \nonumber
\end{eqnarray}

\begin{remark}
  Note that the lazier definition still does not deal with summation
  or mixed summation (i.e. sums over input and output). The reader is
  invited to construct definitions of replication that deal with these
  features. 

  Further, the definitions are parameterized in a name, $x$. Can you,
  gentle reader, make a definition that eliminates this parameter and
  guarantees no accidental interaction between the replication
  machinery and the process being replicated -- i.e. no accidental
  sharing of names used by the process to get its work done and the
  name(s) used by the replication to effect copying. This latter
  revision of the definition of replication is crucial to obtaining
  the expected identity $!!P \sim !P$.
\end{remark}

\begin{remark}\label{rem:paradoxical_combinator}
  The reader familiar with the lambda calculus will have noticed the
  similarity between $D$ and the paradoxical combinator.

  [Ed. note: the existence of this seems to suggest we have to be more
  restrictive on the set of processes and names we admit if we are to
  support no-cloning.]
\end{remark}

\subsubsection{Bisimulation}

The computational dynamics gives rise to another kind of equivalence,
the equivalence of computational behavior. As previously mentioned
this is typically captured \emph{via} some form of bisimulation.

% The notion we use in this paper is weak barbed bisimulation
% \cite{milner91polyadicpi}.

The notion we use in this paper is derived from weak barbed
bisimulation \cite{milner91polyadicpi}. 

\begin{definition}
An \emph{observation relation}, $\downarrow_{\mathcal N}$, over a set
of names, $\mathcal N$, is the smallest relation satisfying the rules
below.

\infrule[Out-barb]{y \in {\mathcal N}, \; x \nameeq y}
		  {\outputp{x}{v} \downarrow_{\mathcal N} x}
\infrule[Par-barb]{\mbox{$P\downarrow_{\mathcal N} x$ or $Q\downarrow_{\mathcal N} x$}}
		  {\binpar{P}{Q} \downarrow_{\mathcal N} x}

We write $P \Downarrow_{\mathcal N} x$ if there is $Q$ such that 
$P \wred Q$ and $Q \downarrow_{\mathcal N} x$.
\end{definition}

\begin{definition}
%\label{def.bbisim}
An  ${\mathcal N}$-\emph{barbed bisimulation} over a set of names, ${\mathcal N}$, is a symmetric binary relation 
${\mathcal S}_{\mathcal N}$ between agents such that $P\rel{S}_{\mathcal N}Q$ implies:
\begin{enumerate}
\item If $P \red P'$ then $Q \wred Q'$ and $P'\rel{S}_{\mathcal N} Q'$.
\item If $P\downarrow_{\mathcal N} x$, then $Q\Downarrow_{\mathcal N} x$.
\end{enumerate}
$P$ is ${\mathcal N}$-barbed bisimilar to $Q$, written
$P \wbbisim_{\mathcal N} Q$, if $P \rel{S}_{\mathcal N} Q$ for some ${\mathcal N}$-barbed bisimulation ${\mathcal S}_{\mathcal N}$.
\end{definition}

$\mathcal{R} \subseteq \pi \times \pi$

$P \mathcal{R} Q => \forall P'. P \red P' \Rightarrow \exists Q'. Q \red Q', P' \mathcal{R} Q'$

$P \vdash x \Rightarrow Q \vdash x$

\begin{mathpar}
  \inferrule*[lab=Out-barb]{x \nameeq y}{{y}!\langle{Q}\rangle \vdash x}
  \and
  \inferrule*[lab=Par-barb]{\mbox{$P\vdash x$ or $Q\vdash x$}}{\binpar{P}{Q} \vdash x}
\end{mathpar}

\subsubsection{Contexts}

One of the principle advantages of computational calculi like the
$\pi$-calculus is a well-defined notion of context,
contextual-equivalence and a correlation between
contextual-equivalence and notions of bisimulation. The notion of
context allows the decomposition of a process into (sub-)process and
its syntactic environment, its context. Thus, a context may be
thought of as a process with a ``hole'' (written $\Box$) in it. The
application of a context $M$ to a process $P$, written $M[P]$, is
tantamount to filling the hole in $M$ with $P$. In this paper we do
not need the full weight of this theory, but do make use of the notion
of context in the proof the main theorem. 

\begin{mathpar}
  \inferrule* [lab=summation] {} {{M_{M},M_{N}} \bc \Box \;|\; x.M_{A} \;|\; M_{M}+M_{N}}
  \and
  \inferrule* [lab=agent] {} {{M_{A}} \bc (\vec{x})M_{P} \;| \; \clift{P_0,\ldots,M_{P},\ldots,P_N}}
  \and \\
  \inferrule* [lab=process] {} {{M_{P}} \bc M_{N} \;| \;P|M_{P} }
\end{mathpar} 

\begin{mathpar}
  \inferrule* [lab=sychronization] {} {M_{N} \bc \Box \;|\; x?M_{F} \;|\; x!M_{C}}
  \and
  \inferrule* [lab=abstraction] {} {{M_{F}} \bc (x)M_{P} }
  \and
  \inferrule* [lab=concretion] {} {{M_{C}} \bc \langle M_{P} \rangle }
  \and \\
  \inferrule* [lab=process] {} {{M_{P}} \bc M_{N} \;| \;P|M_{P} }
\end{mathpar}

\begin{definition}[contextual application] Given a context $M$, and
  process $P$, we define the \emph{contextual application}, $M[P] :=
  M\{P/\Box\}$. That is, the contextual application of M to P is the
  substitution of $P$ for $\Box$ in $M$.
\end{definition}

$\meaningof{-} : L \to \mathcal{P}(\pi)$

\begin{mathpar}
  \inferrule* [lab=collection] {} {\meaningof{true} = \pi, \and \meaningof{~E} = \pi \setminus \meaningof{E}, \and \meaningof{E_{1} \& E_{2}} = \meaningof{E_{1}} \cap \meaningof{E_{2}}}
\end{mathpar}

\begin{mathpar}
  \inferrule* [lab=structure] {} {\meaningof{0} = \{ P \in \pi | P \equiv 0 \}, \and \\ \meaningof{E_1 | E_2} = \{ P \in \pi | P \equiv P_{1} | P_{2}, P_{1} \in \meaningof{E_{1}}, P_{2} \in \meaningof{E_2}\} }
\end{mathpar}

\begin{mathpar}
 \inferrule* [lab=behavior] {} {\meaningof{\langle a?b \rangle E} = \{ P \in \pi | P \equiv Q | u?(y)P', \\ \and \\\\ \and \\ \;\;\; u \in \meaningof{a}, \forall z.P'\{z/y\} \in \meaningof{E\{z/b\}}\}, \and \\ \meaningof{a!E} = \{ P \in \pi | P \equiv Q | x!\langle P' \rangle, x \in \meaningof{a} P' \in \meaningof{E}\} }
\end{mathpar}

\begin{mathpar}
 \inferrule* [lab=nominal] {} {\meaningof{\quotep{E}} = \{ \quotep{P} \in \quotep{\pi} | P \in \meaningof{E} \}, \and \meaningof{\quotep{P}} = \{ \quotep{Q} \in \quotep{\pi} | P \equiv Q \} \and \\ \meaningof{@\quotep{E}} = \{ P \in \pi | P \equiv @x, x \in \meaningof{E} \}}
\end{mathpar}

\begin{eqnarray*}
  \\
  \meaningof{-} : TS \to ST
\end{eqnarray*}

\begin{eqnarray*}
  \\
  L : TS \to ST
\end{eqnarray*}

\begin{eqnarray*}
  \\
  P \models E \iff P \in \meaningof{E}
\end{eqnarray*}

\begin{eqnarray*}
  P \approx_{L} Q \iff \forall E \in L. P \models E \iff Q \models E
\end{eqnarray*}

\begin{eqnarray*}
  P \approx_{K} Q
\end{eqnarray*}

\begin{eqnarray*}
  P \approx Q
\end{eqnarray*}

$\approx_{K} = \approx = \approx_{L}$

\subsubsection{Contextual duality}

Note that contexts extend the quotation operation to a family of
operations from processes to names. Given a context, $M$, we can
define a \emph{nominal context}, $\quotep{M}$ by $\quotep{M}[P] :=
\quotep{M[P]}$. To foreshadow what is to come we observe that these
operations enjoy a duality with processes very much like the duality
between vectors and maps from vectors to scalars.

Further, because the calculus is essentially higher-order, we have a
correspondence between contexts and processes. More specifically,
given a name $x$ and a context $M$ we can construct $M^{*}_{x}$ such
that 

\begin{mathpar}
  M^{*}_{x} | \lift{x}{P} \red M[P]
\end{mathpar}

namely,

\begin{mathpar}
  M^{*}_{x} := x?(u).M[\dropn{u}]
\end{mathpar}

The dependence of $M^{*}_{x}$ on a name makes it an abstraction, 

\begin{mathpar}
  M^{*} := (x)x?(u).M[\dropn{u}]
\end{mathpar}

\subsection{Additional notation}

It will sometimes be convenient to denote the process a name
quotes. We already have the notation $x = \quotep{P}$, but it will be
convenient to introduce an alternate notation, $\procn{x}$, when we
want to emphasize the connection to the use of the name. Note that, by
virtue of name equivalence, $\quotep{\procn{x}} \nameeq x$; so, the
notation is consistent with previous definitions.

Further, because names have structure it is possible to effect
substitutions on the basis of that structure. This means we need to
upgrade our notation for substitutions, which we accomplish by
adapting comprehension notation. Thus,

\begin{mathpar}
  P\{ y / x : x \in S \}
\end{mathpar}

is interpreted to mean the process derived from P by replacing (in a
capture-avoiding manner) each occurrence of $x$ in $S$ by $y$. For example,

\begin{mathpar}
  P\{ \quotep{\procn{x}|\procn{x}} / x : x \in \freenames{P} \}
\end{mathpar}

will replace each (occurrence) of a free name $x$ in $P$ by
$\quotep{\procn{x}|\procn{x}}$.

Also, we will avail ourselves of the notation $x^{L}$ and $x^{R}$ to
denote injections of a name into disjoint copies of the name
space. There are numerous ways to accomplish this. One example can be
found in \cite{MeredithR05}. This notation overloads to vectors of
names: $\vec{x}^{\pi} := (x_{i}^{\pi} \; : \; 0 \leq i < |\vec{x}| )$ where $\pi \in \{L,R\}$.

We also use $P^{\Box} := P|\Box$.

In \cite{MeredithR05} an interpretation of the new operator is
given. It turns out that there are several possible interpretations
all enjoying the requisite algebraic properties of the operator (see
\cite{milner91polyadicpi}). We will therefore make liberal use of
$(\nu\; \vec{x})P$.

% subsection the_syntax_and_semantics_of_the_notation_system (end)   

\input{qm2pi.qmops} 

\input{qm2pi.sterngerlach} 

\input{qm2pi.metric} 

% section concurrent_process_calculi (end)

%\input{qm2pi.proofsketch}

% section proof sketch (end)

%\input{qm2pi.slviaknots} 

% section spatial logic via knots (end)

\input{qm2pi.conclusion}

% section conclusion (end)

%\input{qm2pi.dtcodes} 

% section wiring algorithm (end)

\input{qm2pi.ack} 

% section acknowledgments (end)

\newpage


\bibliographystyle{plain}   
\bibliography{../../biblios/main.bib}

\input{qm2pi.rhodetails}

\end{document}

 

%\ifpdf
%\usepackage[pdftex]{graphicx}
%\else
%\usepackage{graphicx}
%\fi

 % \ifpdf
%  \usepackage{pdfsync}
%  \if


%\title{Brief Article}
%\author{David F. Snyder}
%\author{L.G. Meredith}

%\address{Dept. of Math., Texas State University--San Marcos, San Marcos, TX 78666}
       
\pagestyle{empty}


\begin{document}

\lstset{language=[Objective]Caml,frame=shadowbox}

\documentclass[12pt]{llncs}
%\documentclass{jktr}

\usepackage[pdftex]{hyperref}                   
\usepackage {listings}
\usepackage {mathpartir}
\usepackage{bcprules}
%\usepackage{listings}
                       
\usepackage{graphicx} 
%\usepackage[margins=2.5cm,nohead,nofoot]{geometry}
%\usepackage{geometry}
\usepackage{amsfonts}
\usepackage{amstext}
\usepackage{latexsym}
\usepackage{amssymb}
\usepackage{color}


%\include{myPreamble}
\include{qm2pi.local} 

%\ifpdf
%\usepackage[pdftex]{graphicx}
%\else
%\usepackage{graphicx}
%\fi

 % \ifpdf
%  \usepackage{pdfsync}
%  \if


%\title{Brief Article}
%\author{David F. Snyder}
%\author{L.G. Meredith}

%\address{Dept. of Math., Texas State University--San Marcos, San Marcos, TX 78666}
       
\pagestyle{empty}


\begin{document}

\lstset{language=[Objective]Caml,frame=shadowbox}

\input{qm2pi.front}

% section front matter (end)

\input{qm2pi.intro} 
 
% section introduction (end)

% \input{qm2pi.knotations} 

% section notation (end)

\input{qm2pi.process.calculi} 

% section concurrent_process_calculi_and_spatial_logics_ (end)
    
%\input{qm2pi.knots2pi} 

%\input{qm2pi.trefoil} 

%\input{qm2pi.mainthm} 

% subsection basic_interpretation (end)

%\input{qm2pi.rho.presentation} 
\subsection{The syntax and semantics of the notation system}\label{sub:the_syntax_and_semantics_of_the_notation_system} % (fold)

We now summarize a technical presentation of the calculus that
embodies our theory of dynamics. The typical presentation of such a
calculus follows the style of giving generators and relations on
them. The grammar, below, describing term constructors, freely
generates the set of processes, $\Proc$. This set is then quotiented
by a relation known as structural congruence and it is over this set
that the notion of dynamics is expressed. This presentation is
essentially that of \cite{MeredithR05} with the addition of
polyadicity and summation. For readability we have relegated some of
the technical subtleties to an appendix.

\subsubsection{Process grammar}\label{subsub:process_grammar}

\begin{mathpar}
  \inferrule* [lab=synchronization] {} {{M} \bc \pzero \;|\; x?F \;|\; x!C }
  \and
  \inferrule* [lab=abstraction] {} {{F} \bc (x)P}
  \and
  \inferrule* [lab=concretion] {} {{C} \bc \langle Q \rangle}
  \and
  \inferrule* [lab=process] {} {{P,Q} \bc M \;| \;P|Q \;|\; @{x}}
  \and
  \inferrule* [lab=name] {} {{x} \bc \quotep{P}}
\end{mathpar} 

Note that $\vec{x}$ (resp. $\vec{P}$) denotes a vector of names
(resp. processes) of length $|\vec{x}|$ (resp. $|\vec{P}|$). We adopt
the following useful abbreviations.

\begin{mathpar}
   x?(\vec{y}).P := x.(\vec{y})P \and  x\clift{\vec{P}} := x.\clift{\vec{P}}
   \and x!(y) := \lift{x}{\dropn{y}}
   \and \Pi_{i=0}^{n-1}P_i := P_0 | \ldots | P_{n-1}
\end{mathpar}

\subsubsection{Structural congruence}

\paragraph{Free and bound names and alpha-equivalence.} At the
core of structural equivalence is alpha-equivalence which identifies
process that are the same up to a change of variable. Formally, we
recognize the distinction between free and bound names. The free names
of a process, $\freenames{P}$, may be calculated recursively as
follows:

\begin{mathpar}
\freenames{\pzero} := \emptyset
  \and \\
  \freenames{x?(y).P} := \{ x \} \cup (\freenames{P} \setminus \{ y \})
  \and 
  \freenames{x!\langle P \rangle} := \{ x \} \cup \{ P \} 
  \and \\
  \freenames{P|Q} := \freenames{P} \cup \freenames{Q}
  \and \\
  \freenames{@{x}} := \{ x \}
\end{mathpar}

$\pi$
$\quotep{\pi}$

$\freenames{-} : \pi \to \mathcal{P}(\quotep{\pi})$

\begin{eqnarray*}
  \freenames{\pzero} & := & \emptyset \\
  \freenames{x?(y).P} & := & \{ x \} \cup (\freenames{P} \setminus \{ y \}) \\
  \freenames{x!\langle P \rangle} & := & \{ x \} \cup \{ P \} \\
  \freenames{P|Q} & := & \freenames{P} \cup \freenames{Q} \\
  \freenames{\dropn{x}} & := & \{ x \}
\end{eqnarray*}

The bound names of a process, $\boundnames{P}$, are those names occurring in $P$
that are not free. For example, in $x?(y).0$, the name $x$ is free, while $y$ is bound.

\begin{mathpar}
  \inferrule* [lab=monoidal-laws] {} { P|Q \equiv Q|P \and P|0 \equiv P \and P|(Q|R) \equiv (P|Q)|R }
\end{mathpar}

\begin{mathpar}
  \inferrule* [lab=alpha-equivalence] {} { (x)P \equiv (y)P\{y/x\} \and y \not\in \freenames{P} }
\end{mathpar}

\begin{definition}
Then two processes, $P,Q$, are alpha-equivalent if $P = Q\{\vec{y}/\vec{x}\}$ for
some $\vec{x} \in \boundnames{Q},\vec{y} \in \boundnames{P}$, where $Q\{\vec{y}/\vec{x}\}$
denotes the capture-avoiding substitution of $\vec{y}$ for $\vec{x}$ in $Q$.
\end{definition}

\begin{definition}
  The {\em structural congruence} \cite{SangiorgiWalker} , $\equiv$,
  between processes is the least congruence containing
  alpha-equivalence, satisfying the abelian monoid laws
  (associativity, commutativity and $\pzero$ as identity) for parallel
  composition $|$ and for summation $+$.
\end{definition}

\subsection{Name equivalence}

We take name equivalence, written $\nameeq$, to be the smallest
equivalence relation generated by the following rules.

\begin{mathpar}
\inferrule*[lab=Quote-drop]
{ }
{ \quotep{@{x}} \nameeq x }

\inferrule*[lab=Struct-equiv]
{ P \scong Q }
{ \quotep{P} \nameeq \quotep{Q} }
\end{mathpar}

The astute reader will have noticed that the mutual recursion of names
and processes imposes a mutual recursion on alpha-equivalence and
structural equivalence via name-equivalence. Fortunately, all of this
works out pleasantly and we may calculate in the natural way, free of
concern. The reader interested in the details is referred to the
appendix \ref{appendix:rho_details}.

\subsection{Substitution}

We use $\Proc$ for the set of processes, $\QProc$ for the set of
names, and $\id{\{}\vec{y} / \vec{x} \id{\}}$ to denote partial maps,
$s : \QProc \rightarrow \QProc$. A map, $s$ lifts, uniquely, to a map
on process terms, $\widehat{s} : \Proc \rightarrow \Proc$ by the
following equations.

\begin{mathpar}
  (0) \psubstp{Q}{P} := 0 \\
  (R \juxtap S) \psubstp{Q}{P}
  :=    
  (R)\psubstp{Q}{P} \juxtap (S) \psubstp{Q}{P} \\
  (x?(y).R) \psubstp{Q}{P}    
  :=    
  (x)\substp{Q}{P} (z)\concat( (R \psubstn{z}{y}) \psubstp{Q}{P} ) \\
  (\lift{x}{R}) \psubstp{Q}{P}  
  :=
  \lift{(x)\substp{Q}{P}}{ R \psubstp{Q}{P} } \\
%   (\dropn{x})  \psubstp{Q}{P}       
%   := 
%   \left\{ 
%     \begin{array}{ccc} 
%       \dropn{\quotep{Q}} & & x \nameeq \quotep{P} \\
%       \dropn{x} & & otherwise \\
%     \end{array}
%   \right. 
  (\dropn{x})  \psubstp{Q}{P}       
  := 
  \left\{ 
    \begin{array}{ccc} 
      Q & & x \nameeq \quotep{P} \\
      \dropn{x} & & otherwise \\
    \end{array}
  \right.
\end{mathpar}
 

where

\begin{eqnarray}
  (x)\id{\{} \lpquote Q \rpquote / \lpquote P \rpquote \id{\}}            = 
  \left\{ 
    \begin{array}{ccc}
      \lpquote Q \rpquote & & x \nameeq \lpquote P \rpquote \\
      x & & otherwise \\
    \end{array}
  \right. \nonumber
\end{eqnarray}

and $z$ is chosen distinct from $\quotep{P}$, $\quotep{Q}$, the free
names in $Q$, and all the names in $R$. Our $\alpha$-equivalence will
be built in the standard way from this substitution.

\begin{remark}\label{rem:no_self_referential_names}
  One consequence of these definitions is that $\forall P. \quotep{P}
  \not\in \freenames{P}$.
\end{remark}

\subsection{ Dynamic quote: an example }

Anticipating something of what's to come, consider applying the
substitution, $\widehat{\id{\{}u / z \id{\}}}$, to the following pair
of processes, $\lift{w}{y!(z)}$ and $w[ \lpquote y!(z) \rpquote ]$.

\begin{eqnarray}
	\lift{w}{y!(z)}\widehat{\id{\{}u / z \id{\}}}
		& = &
		\lift{w}{y!(u)} \nonumber\\
	w[ \lpquote y!(z) \rpquote ] \widehat{ \id{\{}u / z \id{\}} }
		& = &
		w[ \lpquote y!(z) \rpquote ] \nonumber
\end{eqnarray}

Because the body of the process between quotes is impervious to
substitution, we get radically different answers. In fact, by
examining the first process in an input context,
e.g. $x?(z).\lift{w}{y!(z)}$, we see that the process under the lift
operator may be shaped by prefixed inputs binding a name inside it. In
this sense, the lift operator will be seen as a way to dynamically
construct processes before reifying them as names.

Finally equipped with these standard features we can present the
dynamics of the calculus.

\subsubsection{Operational semantics} 

Finally, we introduce the computational dynamics. What marks these
algebras as distinct from other more traditionally studied algebraic
structures, e.g. vector spaces or polynomial rings, is the manner in
which dynamics is captured. In traditional structures, dynamics is typically
expressed through morphisms between such structures, as in linear maps
between vector spaces or morphisms between rings. In algebras
associated with the semantics of computation, the dynamics is
expressed as part of the algebraic structure itself, through a
reduction reduction relation typically denoted by $\red$. Below, we
give a recursive presentation of this relation for the calculus used
in the encoding.

$\red \subseteq \pi \times \pi$
$\red : \pi \to \mathcal{P}(\pi)$

\begin{mathpar}
  \inferrule* [lab=Comm] { \textsf{match}( x_{src}, x_{trgt} ) } { x_{trgt}?(y)P \; | \; x_{src}!\langle {Q} \rangle \red P\{\quotep{Q}/y}\} }
  \and \\
  \inferrule* [lab=Par] {{P} \red {P}'} {{{P} | {Q}} \red {{P}' | {Q}}}
  \and
  \inferrule* [lab=Equiv]{{{P} \scong {P}'} \andalso {{P}' \red {Q}'} \andalso {{Q}' \scong {Q}}}{{P} \red {Q}}
\end{mathpar}

\begin{eqnarray*}
  match_{\equiv} (\quotep{P},\quotep{Q}) & := & P \equiv Q \\
  match_{\dagger}(\quotep{P},\quotep{Q}) & := & \forall R. P|Q \red^{*} R => R \red^{*} 0 \\
  match_{K}(\quotep{P},\quotep{Q}) & := & K \mbox{ for some context } K
\end{eqnarray*}

$u?(x)P | u!\langle Q \rangle \red P\{\quotep{Q}/x\}$

%We write $\wred$ for $\red^*$, and $P\red$ if $\exists Q $ such that $ P \red Q$.
We write $P\red$ if $\exists Q $ such that $ P \red Q$ and $P\not\red$, otherwise.

\section{Replication}

As mentioned before, it is known that replication (and hence
recursion) can be implemented in a higher-order process algebra
\cite{SangiorgiWalker}. As our first example of calculation with the
machinery thus far presented we give the construction explicitly in
the {\rhoc}.

\begin{eqnarray}
	D_{x} & := & \prefix{x}{y}{(\binpar{\outputp{x}{y}}{@{y}})} \nonumber\\
	\bangp_{x}{P} & := & \binpar{{x}!\langle{\binpar{D_{x}}{P}}\rangle}{D_{x}} \nonumber
\end{eqnarray}

\begin{eqnarray}
	\bangp_{x}{P} & & \nonumber\\
	=
	& {x}!\langle{(\prefix{x}{y}{(\outputp{x}{y} | @{y})) | P}}\rangle 
	      | \prefix{x}{y}{(\outputp{x}{y} | @{y})} & \nonumber\\
	\red
	& (\outputp{x}{y} | @{y})\substn{\quotep{(\prefix{x}{y}{(@{y} | \outputp{x}{y})) | P}}}{y} & \nonumber\\
	=
	& \outputp{x}{\quotep{(\prefix{x}{y}{(\outputp{x}{y} | @{y})) | P}}}
	  | {(\prefix{x}{y}{(\outputp{x}{y} | @{y})) | P}} & \nonumber\\
	\red
	& \ldots & \nonumber\\
	\red^*
	& P | P | \ldots & \nonumber
\end{eqnarray}

Of course, this encoding, as an implementation, runs away, unfolding
$\bangp{P}$ eagerly. A lazier and more implementable replication
operator, restricted to input-guarded processes, may be obtained as follows.

\begin{eqnarray}
\bangp{\prefix{u}{v}{P}} 
	:= 
	\binpar{\lift{x}{\prefix{u}{v}{(\binpar{D(x)}{P})}}}{D(x)} \nonumber
\end{eqnarray}

\begin{remark}
  Note that the lazier definition still does not deal with summation
  or mixed summation (i.e. sums over input and output). The reader is
  invited to construct definitions of replication that deal with these
  features. 

  Further, the definitions are parameterized in a name, $x$. Can you,
  gentle reader, make a definition that eliminates this parameter and
  guarantees no accidental interaction between the replication
  machinery and the process being replicated -- i.e. no accidental
  sharing of names used by the process to get its work done and the
  name(s) used by the replication to effect copying. This latter
  revision of the definition of replication is crucial to obtaining
  the expected identity $!!P \sim !P$.
\end{remark}

\begin{remark}\label{rem:paradoxical_combinator}
  The reader familiar with the lambda calculus will have noticed the
  similarity between $D$ and the paradoxical combinator.

  [Ed. note: the existence of this seems to suggest we have to be more
  restrictive on the set of processes and names we admit if we are to
  support no-cloning.]
\end{remark}

\subsubsection{Bisimulation}

The computational dynamics gives rise to another kind of equivalence,
the equivalence of computational behavior. As previously mentioned
this is typically captured \emph{via} some form of bisimulation.

% The notion we use in this paper is weak barbed bisimulation
% \cite{milner91polyadicpi}.

The notion we use in this paper is derived from weak barbed
bisimulation \cite{milner91polyadicpi}. 

\begin{definition}
An \emph{observation relation}, $\downarrow_{\mathcal N}$, over a set
of names, $\mathcal N$, is the smallest relation satisfying the rules
below.

\infrule[Out-barb]{y \in {\mathcal N}, \; x \nameeq y}
		  {\outputp{x}{v} \downarrow_{\mathcal N} x}
\infrule[Par-barb]{\mbox{$P\downarrow_{\mathcal N} x$ or $Q\downarrow_{\mathcal N} x$}}
		  {\binpar{P}{Q} \downarrow_{\mathcal N} x}

We write $P \Downarrow_{\mathcal N} x$ if there is $Q$ such that 
$P \wred Q$ and $Q \downarrow_{\mathcal N} x$.
\end{definition}

\begin{definition}
%\label{def.bbisim}
An  ${\mathcal N}$-\emph{barbed bisimulation} over a set of names, ${\mathcal N}$, is a symmetric binary relation 
${\mathcal S}_{\mathcal N}$ between agents such that $P\rel{S}_{\mathcal N}Q$ implies:
\begin{enumerate}
\item If $P \red P'$ then $Q \wred Q'$ and $P'\rel{S}_{\mathcal N} Q'$.
\item If $P\downarrow_{\mathcal N} x$, then $Q\Downarrow_{\mathcal N} x$.
\end{enumerate}
$P$ is ${\mathcal N}$-barbed bisimilar to $Q$, written
$P \wbbisim_{\mathcal N} Q$, if $P \rel{S}_{\mathcal N} Q$ for some ${\mathcal N}$-barbed bisimulation ${\mathcal S}_{\mathcal N}$.
\end{definition}

$\mathcal{R} \subseteq \pi \times \pi$

$P \mathcal{R} Q => \forall P'. P \red P' \Rightarrow \exists Q'. Q \red Q', P' \mathcal{R} Q'$

$P \vdash x \Rightarrow Q \vdash x$

\begin{mathpar}
  \inferrule*[lab=Out-barb]{x \nameeq y}{{y}!\langle{Q}\rangle \vdash x}
  \and
  \inferrule*[lab=Par-barb]{\mbox{$P\vdash x$ or $Q\vdash x$}}{\binpar{P}{Q} \vdash x}
\end{mathpar}

\subsubsection{Contexts}

One of the principle advantages of computational calculi like the
$\pi$-calculus is a well-defined notion of context,
contextual-equivalence and a correlation between
contextual-equivalence and notions of bisimulation. The notion of
context allows the decomposition of a process into (sub-)process and
its syntactic environment, its context. Thus, a context may be
thought of as a process with a ``hole'' (written $\Box$) in it. The
application of a context $M$ to a process $P$, written $M[P]$, is
tantamount to filling the hole in $M$ with $P$. In this paper we do
not need the full weight of this theory, but do make use of the notion
of context in the proof the main theorem. 

\begin{mathpar}
  \inferrule* [lab=summation] {} {{M_{M},M_{N}} \bc \Box \;|\; x.M_{A} \;|\; M_{M}+M_{N}}
  \and
  \inferrule* [lab=agent] {} {{M_{A}} \bc (\vec{x})M_{P} \;| \; \clift{P_0,\ldots,M_{P},\ldots,P_N}}
  \and \\
  \inferrule* [lab=process] {} {{M_{P}} \bc M_{N} \;| \;P|M_{P} }
\end{mathpar} 

\begin{mathpar}
  \inferrule* [lab=sychronization] {} {M_{N} \bc \Box \;|\; x?M_{F} \;|\; x!M_{C}}
  \and
  \inferrule* [lab=abstraction] {} {{M_{F}} \bc (x)M_{P} }
  \and
  \inferrule* [lab=concretion] {} {{M_{C}} \bc \langle M_{P} \rangle }
  \and \\
  \inferrule* [lab=process] {} {{M_{P}} \bc M_{N} \;| \;P|M_{P} }
\end{mathpar}

\begin{definition}[contextual application] Given a context $M$, and
  process $P$, we define the \emph{contextual application}, $M[P] :=
  M\{P/\Box\}$. That is, the contextual application of M to P is the
  substitution of $P$ for $\Box$ in $M$.
\end{definition}

$\meaningof{-} : L \to \mathcal{P}(\pi)$

\begin{mathpar}
  \inferrule* [lab=collection] {} {\meaningof{true} = \pi, \and \meaningof{~E} = \pi \setminus \meaningof{E}, \and \meaningof{E_{1} \& E_{2}} = \meaningof{E_{1}} \cap \meaningof{E_{2}}}
\end{mathpar}

\begin{mathpar}
  \inferrule* [lab=structure] {} {\meaningof{0} = \{ P \in \pi | P \equiv 0 \}, \and \\ \meaningof{E_1 | E_2} = \{ P \in \pi | P \equiv P_{1} | P_{2}, P_{1} \in \meaningof{E_{1}}, P_{2} \in \meaningof{E_2}\} }
\end{mathpar}

\begin{mathpar}
 \inferrule* [lab=behavior] {} {\meaningof{\langle a?b \rangle E} = \{ P \in \pi | P \equiv Q | u?(y)P', \\ \and \\\\ \and \\ \;\;\; u \in \meaningof{a}, \forall z.P'\{z/y\} \in \meaningof{E\{z/b\}}\}, \and \\ \meaningof{a!E} = \{ P \in \pi | P \equiv Q | x!\langle P' \rangle, x \in \meaningof{a} P' \in \meaningof{E}\} }
\end{mathpar}

\begin{mathpar}
 \inferrule* [lab=nominal] {} {\meaningof{\quotep{E}} = \{ \quotep{P} \in \quotep{\pi} | P \in \meaningof{E} \}, \and \meaningof{\quotep{P}} = \{ \quotep{Q} \in \quotep{\pi} | P \equiv Q \} \and \\ \meaningof{@\quotep{E}} = \{ P \in \pi | P \equiv @x, x \in \meaningof{E} \}}
\end{mathpar}

\begin{eqnarray*}
  \\
  \meaningof{-} : TS \to ST
\end{eqnarray*}

\begin{eqnarray*}
  \\
  L : TS \to ST
\end{eqnarray*}

\begin{eqnarray*}
  \\
  P \models E \iff P \in \meaningof{E}
\end{eqnarray*}

\begin{eqnarray*}
  P \approx_{L} Q \iff \forall E \in L. P \models E \iff Q \models E
\end{eqnarray*}

\begin{eqnarray*}
  P \approx_{K} Q
\end{eqnarray*}

\begin{eqnarray*}
  P \approx Q
\end{eqnarray*}

$\approx_{K} = \approx = \approx_{L}$

\subsubsection{Contextual duality}

Note that contexts extend the quotation operation to a family of
operations from processes to names. Given a context, $M$, we can
define a \emph{nominal context}, $\quotep{M}$ by $\quotep{M}[P] :=
\quotep{M[P]}$. To foreshadow what is to come we observe that these
operations enjoy a duality with processes very much like the duality
between vectors and maps from vectors to scalars.

Further, because the calculus is essentially higher-order, we have a
correspondence between contexts and processes. More specifically,
given a name $x$ and a context $M$ we can construct $M^{*}_{x}$ such
that 

\begin{mathpar}
  M^{*}_{x} | \lift{x}{P} \red M[P]
\end{mathpar}

namely,

\begin{mathpar}
  M^{*}_{x} := x?(u).M[\dropn{u}]
\end{mathpar}

The dependence of $M^{*}_{x}$ on a name makes it an abstraction, 

\begin{mathpar}
  M^{*} := (x)x?(u).M[\dropn{u}]
\end{mathpar}

\subsection{Additional notation}

It will sometimes be convenient to denote the process a name
quotes. We already have the notation $x = \quotep{P}$, but it will be
convenient to introduce an alternate notation, $\procn{x}$, when we
want to emphasize the connection to the use of the name. Note that, by
virtue of name equivalence, $\quotep{\procn{x}} \nameeq x$; so, the
notation is consistent with previous definitions.

Further, because names have structure it is possible to effect
substitutions on the basis of that structure. This means we need to
upgrade our notation for substitutions, which we accomplish by
adapting comprehension notation. Thus,

\begin{mathpar}
  P\{ y / x : x \in S \}
\end{mathpar}

is interpreted to mean the process derived from P by replacing (in a
capture-avoiding manner) each occurrence of $x$ in $S$ by $y$. For example,

\begin{mathpar}
  P\{ \quotep{\procn{x}|\procn{x}} / x : x \in \freenames{P} \}
\end{mathpar}

will replace each (occurrence) of a free name $x$ in $P$ by
$\quotep{\procn{x}|\procn{x}}$.

Also, we will avail ourselves of the notation $x^{L}$ and $x^{R}$ to
denote injections of a name into disjoint copies of the name
space. There are numerous ways to accomplish this. One example can be
found in \cite{MeredithR05}. This notation overloads to vectors of
names: $\vec{x}^{\pi} := (x_{i}^{\pi} \; : \; 0 \leq i < |\vec{x}| )$ where $\pi \in \{L,R\}$.

We also use $P^{\Box} := P|\Box$.

In \cite{MeredithR05} an interpretation of the new operator is
given. It turns out that there are several possible interpretations
all enjoying the requisite algebraic properties of the operator (see
\cite{milner91polyadicpi}). We will therefore make liberal use of
$(\nu\; \vec{x})P$.

% subsection the_syntax_and_semantics_of_the_notation_system (end)   

\input{qm2pi.qmops} 

\input{qm2pi.sterngerlach} 

\input{qm2pi.metric} 

% section concurrent_process_calculi (end)

%\input{qm2pi.proofsketch}

% section proof sketch (end)

%\input{qm2pi.slviaknots} 

% section spatial logic via knots (end)

\input{qm2pi.conclusion}

% section conclusion (end)

%\input{qm2pi.dtcodes} 

% section wiring algorithm (end)

\input{qm2pi.ack} 

% section acknowledgments (end)

\newpage


\bibliographystyle{plain}   
\bibliography{../../biblios/main.bib}

\input{qm2pi.rhodetails}

\end{document}



% section front matter (end)

\section{Introduction}\label{sec:introduction} % (fold)
In this draft of the material i am going to have to dispense with the
usual writing conventions adopted in papers on these topics. i'm going
to have adopt whatever tone i need at the time i'm writing up the
calculations. Sometimes this may be very conversational; others it may
be the barest mathematical grunts; others still it may be that i have
lifted text from one of my other papers because the exposition of some
point was better said there. i hope that my readers are not unduly put
out by this decision. i'm not doing this to flout convention or be
rebellious. i find these calculations very technically challenging. To
keep everything going technically, something has to give; i have to
let go of some cognitive burden. So, the academic writing style --
with all of its trade-offs in terms of facilitating technical
communication -- is what i'm letting go of. Perhaps subsequent drafts
can be tightened and polished, but for now, i'm going to speak as if
we were sitting together in a coffee shop with a laptop, wifi and a
pad of paper and a pencil.

So, here's what i have to say. We -- you and i, comfortably ensconced
in our coffee shop and well-equipped with our tools -- can realize and
carry out the calculations of quantum mechanics over a very different
formal theory of dynamics, a formal theory of dynamics that
corresponds to a theory of concurrent computation with
\emph{reflection}. It has the advantage that the underlying theory is
already `quantized', but supports analogues all of the continuuous
operations. Strikingly, this underlying theory has recently been
connected with a notion of metric that we can show, by calculating
together, coincides with the metric induced by the inner product.

There are a lot of reasons why you might be interested in seeing
calculations of this form. Here's why i'm interested. For the past
several centuries there has been no competitor to the ``Newtonian''
account of dynamics. As a result the predominant share of accounts of
dynamical systems and situations have had to be formulated in terms of
the Newtonian machinery. i view this as an intellectually dangerous
position to occupy. Everything, despite it's intrinsic shape, turns
into a nail to be hit with this hammer. Recently, however, the theory
of computation has matured to the point where we have candidates for
theories of dynamics that offer very different perspective on
reasoning about dynamical systems and situations. Testing these
candidates against very successful accounts of dynamical situations,
like quantum mechanics, is going to give us some sense of how mature
they are and some measure of the quality of these accounts of
dynamics.

\subsection{Summary of contributions and outline of paper}

So, we're going to develop an interpretation of the operations of
quantum mechanics normally interpreted by Hilbert spaces and
operators. We're going to do this over a theory of computation. Note
that this is very different than the usual quantum computation program
which develops notions of computation over quantum mechanics. Rather,
we are developing a story that aligns with Wheeler's slogan: It from
Bit. To do this we will first provide an account of the theory of
computation at play here. Then we will dive into a calculation-driven
interpretation of the operations of quantum mechanics.

The reason we take this approach is that -- until very recently --
there hasn't been an axiomatic account of quantum mechanics. As a
result there has been no sharp delineation of the mathematical theory
supporting interpretation of the physical theory and the physical
theory, itself. So, ambient features of the maths are free to be
exploited (or supressed) without a real accounting of their physical
relevance. There is no sharp statement ``here's the physical theory''
qua \emph{theory} and ``here's the mathematical interpretation''
enabling a judgment of how faithful the interpretation is -- apart
from experimental observation. When there is an axiomatic account we
can judge how well a given mathematical formalism supports an
interpretation of the axioms, independent of
experimentation. Likewise, we can judge how well we have captured our
physical evidence and experience with our axiomatics, independent of
any specific mathematical implementation, with accidental detail that
may or may not have physical significance. 

In lieu of a fully fleshed out and vetted axiomatic account of quantum
mechanics, interpreting the operational notions in service of modeling
physical systems will have to suffice. In other words, we are not in
the business of providing a model of Hilbert spaces and operators. We
are in the business of providing a model of quantum mechanics because
we are motivated by testing our notions of dynamics against physical
theory; and, the predictive calculations of the physical theory must
serve as the best formulation -- shy of a fully fleshed out axiomatic
account -- of the physical theory itself (as they have for scientific
theories since time immemorial). Put another way, despite a
whole-hearted commitment to an It-from-Bit ontology, we are firmly
aligned with the shut-up-and-calculate camp as the best way to obtain
results either from the physical perspective or as a quality assurance
measure of our fledgling theory of dynamics.

In detail, we present a reflective process calculus. Then we develop
intuitive correspondences between the notions available in this
calculus and the usual physical notions supporting quantum mechanical
calculations. Thus, 

\begin{table}[htp]
  \center{
    \fbox{
      \begin{tabular}{c|c}
        quantum mechanics & process calculus \\
        \hline
        scalar & name \\
        state vector & process \\
        dual & contextual duals \\
        matrix & formal sums of process-context-dual pairs \\
        orthogonality & process annihilation \\
        inner product & execution-formula + quoting
      \end{tabular}
    }
  }
  \caption{QM - process calculi correspondences}
\end{table}

Then we tighten up these intuitions to operational definitions. We
employ the Dirac notation as the best proxy we can find for an
abstract syntax of the quantum mechanical notions. The definitions we
develop put us in contact with equational constraints coming from the
theory that we demonstrate the definitions and calculations satisfy.

This puts us in a position to shut up and calculate for the
Stern-Gerlach experimental set up, showing how these predictive
calculations become calculations on processes in our theory of a
reflective process calculus.

Penultimately, we demonstrate that the notion of metric coming from
the inner product coincides with the notion of metric available from
the theory of bisimulation. This demonstration gives us the right to
think of space as arising from behavior. Finally, we consider where we
might go from the new vantage point we have obtained.

% section introduction (end) 
 
% section introduction (end)

% \documentclass[12pt]{llncs}
%\documentclass{jktr}

\usepackage[pdftex]{hyperref}                   
\usepackage {listings}
\usepackage {mathpartir}
\usepackage{bcprules}
%\usepackage{listings}
                       
\usepackage{graphicx} 
%\usepackage[margins=2.5cm,nohead,nofoot]{geometry}
%\usepackage{geometry}
\usepackage{amsfonts}
\usepackage{amstext}
\usepackage{latexsym}
\usepackage{amssymb}
\usepackage{color}


%\include{myPreamble}
\include{qm2pi.local} 

%\ifpdf
%\usepackage[pdftex]{graphicx}
%\else
%\usepackage{graphicx}
%\fi

 % \ifpdf
%  \usepackage{pdfsync}
%  \if


%\title{Brief Article}
%\author{David F. Snyder}
%\author{L.G. Meredith}

%\address{Dept. of Math., Texas State University--San Marcos, San Marcos, TX 78666}
       
\pagestyle{empty}


\begin{document}

\lstset{language=[Objective]Caml,frame=shadowbox}

\input{qm2pi.front}

% section front matter (end)

\input{qm2pi.intro} 
 
% section introduction (end)

% \input{qm2pi.knotations} 

% section notation (end)

\input{qm2pi.process.calculi} 

% section concurrent_process_calculi_and_spatial_logics_ (end)
    
%\input{qm2pi.knots2pi} 

%\input{qm2pi.trefoil} 

%\input{qm2pi.mainthm} 

% subsection basic_interpretation (end)

%\input{qm2pi.rho.presentation} 
\subsection{The syntax and semantics of the notation system}\label{sub:the_syntax_and_semantics_of_the_notation_system} % (fold)

We now summarize a technical presentation of the calculus that
embodies our theory of dynamics. The typical presentation of such a
calculus follows the style of giving generators and relations on
them. The grammar, below, describing term constructors, freely
generates the set of processes, $\Proc$. This set is then quotiented
by a relation known as structural congruence and it is over this set
that the notion of dynamics is expressed. This presentation is
essentially that of \cite{MeredithR05} with the addition of
polyadicity and summation. For readability we have relegated some of
the technical subtleties to an appendix.

\subsubsection{Process grammar}\label{subsub:process_grammar}

\begin{mathpar}
  \inferrule* [lab=synchronization] {} {{M} \bc \pzero \;|\; x?F \;|\; x!C }
  \and
  \inferrule* [lab=abstraction] {} {{F} \bc (x)P}
  \and
  \inferrule* [lab=concretion] {} {{C} \bc \langle Q \rangle}
  \and
  \inferrule* [lab=process] {} {{P,Q} \bc M \;| \;P|Q \;|\; @{x}}
  \and
  \inferrule* [lab=name] {} {{x} \bc \quotep{P}}
\end{mathpar} 

Note that $\vec{x}$ (resp. $\vec{P}$) denotes a vector of names
(resp. processes) of length $|\vec{x}|$ (resp. $|\vec{P}|$). We adopt
the following useful abbreviations.

\begin{mathpar}
   x?(\vec{y}).P := x.(\vec{y})P \and  x\clift{\vec{P}} := x.\clift{\vec{P}}
   \and x!(y) := \lift{x}{\dropn{y}}
   \and \Pi_{i=0}^{n-1}P_i := P_0 | \ldots | P_{n-1}
\end{mathpar}

\subsubsection{Structural congruence}

\paragraph{Free and bound names and alpha-equivalence.} At the
core of structural equivalence is alpha-equivalence which identifies
process that are the same up to a change of variable. Formally, we
recognize the distinction between free and bound names. The free names
of a process, $\freenames{P}$, may be calculated recursively as
follows:

\begin{mathpar}
\freenames{\pzero} := \emptyset
  \and \\
  \freenames{x?(y).P} := \{ x \} \cup (\freenames{P} \setminus \{ y \})
  \and 
  \freenames{x!\langle P \rangle} := \{ x \} \cup \{ P \} 
  \and \\
  \freenames{P|Q} := \freenames{P} \cup \freenames{Q}
  \and \\
  \freenames{@{x}} := \{ x \}
\end{mathpar}

$\pi$
$\quotep{\pi}$

$\freenames{-} : \pi \to \mathcal{P}(\quotep{\pi})$

\begin{eqnarray*}
  \freenames{\pzero} & := & \emptyset \\
  \freenames{x?(y).P} & := & \{ x \} \cup (\freenames{P} \setminus \{ y \}) \\
  \freenames{x!\langle P \rangle} & := & \{ x \} \cup \{ P \} \\
  \freenames{P|Q} & := & \freenames{P} \cup \freenames{Q} \\
  \freenames{\dropn{x}} & := & \{ x \}
\end{eqnarray*}

The bound names of a process, $\boundnames{P}$, are those names occurring in $P$
that are not free. For example, in $x?(y).0$, the name $x$ is free, while $y$ is bound.

\begin{mathpar}
  \inferrule* [lab=monoidal-laws] {} { P|Q \equiv Q|P \and P|0 \equiv P \and P|(Q|R) \equiv (P|Q)|R }
\end{mathpar}

\begin{mathpar}
  \inferrule* [lab=alpha-equivalence] {} { (x)P \equiv (y)P\{y/x\} \and y \not\in \freenames{P} }
\end{mathpar}

\begin{definition}
Then two processes, $P,Q$, are alpha-equivalent if $P = Q\{\vec{y}/\vec{x}\}$ for
some $\vec{x} \in \boundnames{Q},\vec{y} \in \boundnames{P}$, where $Q\{\vec{y}/\vec{x}\}$
denotes the capture-avoiding substitution of $\vec{y}$ for $\vec{x}$ in $Q$.
\end{definition}

\begin{definition}
  The {\em structural congruence} \cite{SangiorgiWalker} , $\equiv$,
  between processes is the least congruence containing
  alpha-equivalence, satisfying the abelian monoid laws
  (associativity, commutativity and $\pzero$ as identity) for parallel
  composition $|$ and for summation $+$.
\end{definition}

\subsection{Name equivalence}

We take name equivalence, written $\nameeq$, to be the smallest
equivalence relation generated by the following rules.

\begin{mathpar}
\inferrule*[lab=Quote-drop]
{ }
{ \quotep{@{x}} \nameeq x }

\inferrule*[lab=Struct-equiv]
{ P \scong Q }
{ \quotep{P} \nameeq \quotep{Q} }
\end{mathpar}

The astute reader will have noticed that the mutual recursion of names
and processes imposes a mutual recursion on alpha-equivalence and
structural equivalence via name-equivalence. Fortunately, all of this
works out pleasantly and we may calculate in the natural way, free of
concern. The reader interested in the details is referred to the
appendix \ref{appendix:rho_details}.

\subsection{Substitution}

We use $\Proc$ for the set of processes, $\QProc$ for the set of
names, and $\id{\{}\vec{y} / \vec{x} \id{\}}$ to denote partial maps,
$s : \QProc \rightarrow \QProc$. A map, $s$ lifts, uniquely, to a map
on process terms, $\widehat{s} : \Proc \rightarrow \Proc$ by the
following equations.

\begin{mathpar}
  (0) \psubstp{Q}{P} := 0 \\
  (R \juxtap S) \psubstp{Q}{P}
  :=    
  (R)\psubstp{Q}{P} \juxtap (S) \psubstp{Q}{P} \\
  (x?(y).R) \psubstp{Q}{P}    
  :=    
  (x)\substp{Q}{P} (z)\concat( (R \psubstn{z}{y}) \psubstp{Q}{P} ) \\
  (\lift{x}{R}) \psubstp{Q}{P}  
  :=
  \lift{(x)\substp{Q}{P}}{ R \psubstp{Q}{P} } \\
%   (\dropn{x})  \psubstp{Q}{P}       
%   := 
%   \left\{ 
%     \begin{array}{ccc} 
%       \dropn{\quotep{Q}} & & x \nameeq \quotep{P} \\
%       \dropn{x} & & otherwise \\
%     \end{array}
%   \right. 
  (\dropn{x})  \psubstp{Q}{P}       
  := 
  \left\{ 
    \begin{array}{ccc} 
      Q & & x \nameeq \quotep{P} \\
      \dropn{x} & & otherwise \\
    \end{array}
  \right.
\end{mathpar}
 

where

\begin{eqnarray}
  (x)\id{\{} \lpquote Q \rpquote / \lpquote P \rpquote \id{\}}            = 
  \left\{ 
    \begin{array}{ccc}
      \lpquote Q \rpquote & & x \nameeq \lpquote P \rpquote \\
      x & & otherwise \\
    \end{array}
  \right. \nonumber
\end{eqnarray}

and $z$ is chosen distinct from $\quotep{P}$, $\quotep{Q}$, the free
names in $Q$, and all the names in $R$. Our $\alpha$-equivalence will
be built in the standard way from this substitution.

\begin{remark}\label{rem:no_self_referential_names}
  One consequence of these definitions is that $\forall P. \quotep{P}
  \not\in \freenames{P}$.
\end{remark}

\subsection{ Dynamic quote: an example }

Anticipating something of what's to come, consider applying the
substitution, $\widehat{\id{\{}u / z \id{\}}}$, to the following pair
of processes, $\lift{w}{y!(z)}$ and $w[ \lpquote y!(z) \rpquote ]$.

\begin{eqnarray}
	\lift{w}{y!(z)}\widehat{\id{\{}u / z \id{\}}}
		& = &
		\lift{w}{y!(u)} \nonumber\\
	w[ \lpquote y!(z) \rpquote ] \widehat{ \id{\{}u / z \id{\}} }
		& = &
		w[ \lpquote y!(z) \rpquote ] \nonumber
\end{eqnarray}

Because the body of the process between quotes is impervious to
substitution, we get radically different answers. In fact, by
examining the first process in an input context,
e.g. $x?(z).\lift{w}{y!(z)}$, we see that the process under the lift
operator may be shaped by prefixed inputs binding a name inside it. In
this sense, the lift operator will be seen as a way to dynamically
construct processes before reifying them as names.

Finally equipped with these standard features we can present the
dynamics of the calculus.

\subsubsection{Operational semantics} 

Finally, we introduce the computational dynamics. What marks these
algebras as distinct from other more traditionally studied algebraic
structures, e.g. vector spaces or polynomial rings, is the manner in
which dynamics is captured. In traditional structures, dynamics is typically
expressed through morphisms between such structures, as in linear maps
between vector spaces or morphisms between rings. In algebras
associated with the semantics of computation, the dynamics is
expressed as part of the algebraic structure itself, through a
reduction reduction relation typically denoted by $\red$. Below, we
give a recursive presentation of this relation for the calculus used
in the encoding.

$\red \subseteq \pi \times \pi$
$\red : \pi \to \mathcal{P}(\pi)$

\begin{mathpar}
  \inferrule* [lab=Comm] { \textsf{match}( x_{src}, x_{trgt} ) } { x_{trgt}?(y)P \; | \; x_{src}!\langle {Q} \rangle \red P\{\quotep{Q}/y}\} }
  \and \\
  \inferrule* [lab=Par] {{P} \red {P}'} {{{P} | {Q}} \red {{P}' | {Q}}}
  \and
  \inferrule* [lab=Equiv]{{{P} \scong {P}'} \andalso {{P}' \red {Q}'} \andalso {{Q}' \scong {Q}}}{{P} \red {Q}}
\end{mathpar}

\begin{eqnarray*}
  match_{\equiv} (\quotep{P},\quotep{Q}) & := & P \equiv Q \\
  match_{\dagger}(\quotep{P},\quotep{Q}) & := & \forall R. P|Q \red^{*} R => R \red^{*} 0 \\
  match_{K}(\quotep{P},\quotep{Q}) & := & K \mbox{ for some context } K
\end{eqnarray*}

$u?(x)P | u!\langle Q \rangle \red P\{\quotep{Q}/x\}$

%We write $\wred$ for $\red^*$, and $P\red$ if $\exists Q $ such that $ P \red Q$.
We write $P\red$ if $\exists Q $ such that $ P \red Q$ and $P\not\red$, otherwise.

\section{Replication}

As mentioned before, it is known that replication (and hence
recursion) can be implemented in a higher-order process algebra
\cite{SangiorgiWalker}. As our first example of calculation with the
machinery thus far presented we give the construction explicitly in
the {\rhoc}.

\begin{eqnarray}
	D_{x} & := & \prefix{x}{y}{(\binpar{\outputp{x}{y}}{@{y}})} \nonumber\\
	\bangp_{x}{P} & := & \binpar{{x}!\langle{\binpar{D_{x}}{P}}\rangle}{D_{x}} \nonumber
\end{eqnarray}

\begin{eqnarray}
	\bangp_{x}{P} & & \nonumber\\
	=
	& {x}!\langle{(\prefix{x}{y}{(\outputp{x}{y} | @{y})) | P}}\rangle 
	      | \prefix{x}{y}{(\outputp{x}{y} | @{y})} & \nonumber\\
	\red
	& (\outputp{x}{y} | @{y})\substn{\quotep{(\prefix{x}{y}{(@{y} | \outputp{x}{y})) | P}}}{y} & \nonumber\\
	=
	& \outputp{x}{\quotep{(\prefix{x}{y}{(\outputp{x}{y} | @{y})) | P}}}
	  | {(\prefix{x}{y}{(\outputp{x}{y} | @{y})) | P}} & \nonumber\\
	\red
	& \ldots & \nonumber\\
	\red^*
	& P | P | \ldots & \nonumber
\end{eqnarray}

Of course, this encoding, as an implementation, runs away, unfolding
$\bangp{P}$ eagerly. A lazier and more implementable replication
operator, restricted to input-guarded processes, may be obtained as follows.

\begin{eqnarray}
\bangp{\prefix{u}{v}{P}} 
	:= 
	\binpar{\lift{x}{\prefix{u}{v}{(\binpar{D(x)}{P})}}}{D(x)} \nonumber
\end{eqnarray}

\begin{remark}
  Note that the lazier definition still does not deal with summation
  or mixed summation (i.e. sums over input and output). The reader is
  invited to construct definitions of replication that deal with these
  features. 

  Further, the definitions are parameterized in a name, $x$. Can you,
  gentle reader, make a definition that eliminates this parameter and
  guarantees no accidental interaction between the replication
  machinery and the process being replicated -- i.e. no accidental
  sharing of names used by the process to get its work done and the
  name(s) used by the replication to effect copying. This latter
  revision of the definition of replication is crucial to obtaining
  the expected identity $!!P \sim !P$.
\end{remark}

\begin{remark}\label{rem:paradoxical_combinator}
  The reader familiar with the lambda calculus will have noticed the
  similarity between $D$ and the paradoxical combinator.

  [Ed. note: the existence of this seems to suggest we have to be more
  restrictive on the set of processes and names we admit if we are to
  support no-cloning.]
\end{remark}

\subsubsection{Bisimulation}

The computational dynamics gives rise to another kind of equivalence,
the equivalence of computational behavior. As previously mentioned
this is typically captured \emph{via} some form of bisimulation.

% The notion we use in this paper is weak barbed bisimulation
% \cite{milner91polyadicpi}.

The notion we use in this paper is derived from weak barbed
bisimulation \cite{milner91polyadicpi}. 

\begin{definition}
An \emph{observation relation}, $\downarrow_{\mathcal N}$, over a set
of names, $\mathcal N$, is the smallest relation satisfying the rules
below.

\infrule[Out-barb]{y \in {\mathcal N}, \; x \nameeq y}
		  {\outputp{x}{v} \downarrow_{\mathcal N} x}
\infrule[Par-barb]{\mbox{$P\downarrow_{\mathcal N} x$ or $Q\downarrow_{\mathcal N} x$}}
		  {\binpar{P}{Q} \downarrow_{\mathcal N} x}

We write $P \Downarrow_{\mathcal N} x$ if there is $Q$ such that 
$P \wred Q$ and $Q \downarrow_{\mathcal N} x$.
\end{definition}

\begin{definition}
%\label{def.bbisim}
An  ${\mathcal N}$-\emph{barbed bisimulation} over a set of names, ${\mathcal N}$, is a symmetric binary relation 
${\mathcal S}_{\mathcal N}$ between agents such that $P\rel{S}_{\mathcal N}Q$ implies:
\begin{enumerate}
\item If $P \red P'$ then $Q \wred Q'$ and $P'\rel{S}_{\mathcal N} Q'$.
\item If $P\downarrow_{\mathcal N} x$, then $Q\Downarrow_{\mathcal N} x$.
\end{enumerate}
$P$ is ${\mathcal N}$-barbed bisimilar to $Q$, written
$P \wbbisim_{\mathcal N} Q$, if $P \rel{S}_{\mathcal N} Q$ for some ${\mathcal N}$-barbed bisimulation ${\mathcal S}_{\mathcal N}$.
\end{definition}

$\mathcal{R} \subseteq \pi \times \pi$

$P \mathcal{R} Q => \forall P'. P \red P' \Rightarrow \exists Q'. Q \red Q', P' \mathcal{R} Q'$

$P \vdash x \Rightarrow Q \vdash x$

\begin{mathpar}
  \inferrule*[lab=Out-barb]{x \nameeq y}{{y}!\langle{Q}\rangle \vdash x}
  \and
  \inferrule*[lab=Par-barb]{\mbox{$P\vdash x$ or $Q\vdash x$}}{\binpar{P}{Q} \vdash x}
\end{mathpar}

\subsubsection{Contexts}

One of the principle advantages of computational calculi like the
$\pi$-calculus is a well-defined notion of context,
contextual-equivalence and a correlation between
contextual-equivalence and notions of bisimulation. The notion of
context allows the decomposition of a process into (sub-)process and
its syntactic environment, its context. Thus, a context may be
thought of as a process with a ``hole'' (written $\Box$) in it. The
application of a context $M$ to a process $P$, written $M[P]$, is
tantamount to filling the hole in $M$ with $P$. In this paper we do
not need the full weight of this theory, but do make use of the notion
of context in the proof the main theorem. 

\begin{mathpar}
  \inferrule* [lab=summation] {} {{M_{M},M_{N}} \bc \Box \;|\; x.M_{A} \;|\; M_{M}+M_{N}}
  \and
  \inferrule* [lab=agent] {} {{M_{A}} \bc (\vec{x})M_{P} \;| \; \clift{P_0,\ldots,M_{P},\ldots,P_N}}
  \and \\
  \inferrule* [lab=process] {} {{M_{P}} \bc M_{N} \;| \;P|M_{P} }
\end{mathpar} 

\begin{mathpar}
  \inferrule* [lab=sychronization] {} {M_{N} \bc \Box \;|\; x?M_{F} \;|\; x!M_{C}}
  \and
  \inferrule* [lab=abstraction] {} {{M_{F}} \bc (x)M_{P} }
  \and
  \inferrule* [lab=concretion] {} {{M_{C}} \bc \langle M_{P} \rangle }
  \and \\
  \inferrule* [lab=process] {} {{M_{P}} \bc M_{N} \;| \;P|M_{P} }
\end{mathpar}

\begin{definition}[contextual application] Given a context $M$, and
  process $P$, we define the \emph{contextual application}, $M[P] :=
  M\{P/\Box\}$. That is, the contextual application of M to P is the
  substitution of $P$ for $\Box$ in $M$.
\end{definition}

$\meaningof{-} : L \to \mathcal{P}(\pi)$

\begin{mathpar}
  \inferrule* [lab=collection] {} {\meaningof{true} = \pi, \and \meaningof{~E} = \pi \setminus \meaningof{E}, \and \meaningof{E_{1} \& E_{2}} = \meaningof{E_{1}} \cap \meaningof{E_{2}}}
\end{mathpar}

\begin{mathpar}
  \inferrule* [lab=structure] {} {\meaningof{0} = \{ P \in \pi | P \equiv 0 \}, \and \\ \meaningof{E_1 | E_2} = \{ P \in \pi | P \equiv P_{1} | P_{2}, P_{1} \in \meaningof{E_{1}}, P_{2} \in \meaningof{E_2}\} }
\end{mathpar}

\begin{mathpar}
 \inferrule* [lab=behavior] {} {\meaningof{\langle a?b \rangle E} = \{ P \in \pi | P \equiv Q | u?(y)P', \\ \and \\\\ \and \\ \;\;\; u \in \meaningof{a}, \forall z.P'\{z/y\} \in \meaningof{E\{z/b\}}\}, \and \\ \meaningof{a!E} = \{ P \in \pi | P \equiv Q | x!\langle P' \rangle, x \in \meaningof{a} P' \in \meaningof{E}\} }
\end{mathpar}

\begin{mathpar}
 \inferrule* [lab=nominal] {} {\meaningof{\quotep{E}} = \{ \quotep{P} \in \quotep{\pi} | P \in \meaningof{E} \}, \and \meaningof{\quotep{P}} = \{ \quotep{Q} \in \quotep{\pi} | P \equiv Q \} \and \\ \meaningof{@\quotep{E}} = \{ P \in \pi | P \equiv @x, x \in \meaningof{E} \}}
\end{mathpar}

\begin{eqnarray*}
  \\
  \meaningof{-} : TS \to ST
\end{eqnarray*}

\begin{eqnarray*}
  \\
  L : TS \to ST
\end{eqnarray*}

\begin{eqnarray*}
  \\
  P \models E \iff P \in \meaningof{E}
\end{eqnarray*}

\begin{eqnarray*}
  P \approx_{L} Q \iff \forall E \in L. P \models E \iff Q \models E
\end{eqnarray*}

\begin{eqnarray*}
  P \approx_{K} Q
\end{eqnarray*}

\begin{eqnarray*}
  P \approx Q
\end{eqnarray*}

$\approx_{K} = \approx = \approx_{L}$

\subsubsection{Contextual duality}

Note that contexts extend the quotation operation to a family of
operations from processes to names. Given a context, $M$, we can
define a \emph{nominal context}, $\quotep{M}$ by $\quotep{M}[P] :=
\quotep{M[P]}$. To foreshadow what is to come we observe that these
operations enjoy a duality with processes very much like the duality
between vectors and maps from vectors to scalars.

Further, because the calculus is essentially higher-order, we have a
correspondence between contexts and processes. More specifically,
given a name $x$ and a context $M$ we can construct $M^{*}_{x}$ such
that 

\begin{mathpar}
  M^{*}_{x} | \lift{x}{P} \red M[P]
\end{mathpar}

namely,

\begin{mathpar}
  M^{*}_{x} := x?(u).M[\dropn{u}]
\end{mathpar}

The dependence of $M^{*}_{x}$ on a name makes it an abstraction, 

\begin{mathpar}
  M^{*} := (x)x?(u).M[\dropn{u}]
\end{mathpar}

\subsection{Additional notation}

It will sometimes be convenient to denote the process a name
quotes. We already have the notation $x = \quotep{P}$, but it will be
convenient to introduce an alternate notation, $\procn{x}$, when we
want to emphasize the connection to the use of the name. Note that, by
virtue of name equivalence, $\quotep{\procn{x}} \nameeq x$; so, the
notation is consistent with previous definitions.

Further, because names have structure it is possible to effect
substitutions on the basis of that structure. This means we need to
upgrade our notation for substitutions, which we accomplish by
adapting comprehension notation. Thus,

\begin{mathpar}
  P\{ y / x : x \in S \}
\end{mathpar}

is interpreted to mean the process derived from P by replacing (in a
capture-avoiding manner) each occurrence of $x$ in $S$ by $y$. For example,

\begin{mathpar}
  P\{ \quotep{\procn{x}|\procn{x}} / x : x \in \freenames{P} \}
\end{mathpar}

will replace each (occurrence) of a free name $x$ in $P$ by
$\quotep{\procn{x}|\procn{x}}$.

Also, we will avail ourselves of the notation $x^{L}$ and $x^{R}$ to
denote injections of a name into disjoint copies of the name
space. There are numerous ways to accomplish this. One example can be
found in \cite{MeredithR05}. This notation overloads to vectors of
names: $\vec{x}^{\pi} := (x_{i}^{\pi} \; : \; 0 \leq i < |\vec{x}| )$ where $\pi \in \{L,R\}$.

We also use $P^{\Box} := P|\Box$.

In \cite{MeredithR05} an interpretation of the new operator is
given. It turns out that there are several possible interpretations
all enjoying the requisite algebraic properties of the operator (see
\cite{milner91polyadicpi}). We will therefore make liberal use of
$(\nu\; \vec{x})P$.

% subsection the_syntax_and_semantics_of_the_notation_system (end)   

\input{qm2pi.qmops} 

\input{qm2pi.sterngerlach} 

\input{qm2pi.metric} 

% section concurrent_process_calculi (end)

%\input{qm2pi.proofsketch}

% section proof sketch (end)

%\input{qm2pi.slviaknots} 

% section spatial logic via knots (end)

\input{qm2pi.conclusion}

% section conclusion (end)

%\input{qm2pi.dtcodes} 

% section wiring algorithm (end)

\input{qm2pi.ack} 

% section acknowledgments (end)

\newpage


\bibliographystyle{plain}   
\bibliography{../../biblios/main.bib}

\input{qm2pi.rhodetails}

\end{document}

 

% section notation (end)

\input{qm2pi.process.calculi} 

% section concurrent_process_calculi_and_spatial_logics_ (end)
    
%\documentclass[12pt]{llncs}
%\documentclass{jktr}

\usepackage[pdftex]{hyperref}                   
\usepackage {listings}
\usepackage {mathpartir}
\usepackage{bcprules}
%\usepackage{listings}
                       
\usepackage{graphicx} 
%\usepackage[margins=2.5cm,nohead,nofoot]{geometry}
%\usepackage{geometry}
\usepackage{amsfonts}
\usepackage{amstext}
\usepackage{latexsym}
\usepackage{amssymb}
\usepackage{color}


%\include{myPreamble}
\include{qm2pi.local} 

%\ifpdf
%\usepackage[pdftex]{graphicx}
%\else
%\usepackage{graphicx}
%\fi

 % \ifpdf
%  \usepackage{pdfsync}
%  \if


%\title{Brief Article}
%\author{David F. Snyder}
%\author{L.G. Meredith}

%\address{Dept. of Math., Texas State University--San Marcos, San Marcos, TX 78666}
       
\pagestyle{empty}


\begin{document}

\lstset{language=[Objective]Caml,frame=shadowbox}

\input{qm2pi.front}

% section front matter (end)

\input{qm2pi.intro} 
 
% section introduction (end)

% \input{qm2pi.knotations} 

% section notation (end)

\input{qm2pi.process.calculi} 

% section concurrent_process_calculi_and_spatial_logics_ (end)
    
%\input{qm2pi.knots2pi} 

%\input{qm2pi.trefoil} 

%\input{qm2pi.mainthm} 

% subsection basic_interpretation (end)

%\input{qm2pi.rho.presentation} 
\subsection{The syntax and semantics of the notation system}\label{sub:the_syntax_and_semantics_of_the_notation_system} % (fold)

We now summarize a technical presentation of the calculus that
embodies our theory of dynamics. The typical presentation of such a
calculus follows the style of giving generators and relations on
them. The grammar, below, describing term constructors, freely
generates the set of processes, $\Proc$. This set is then quotiented
by a relation known as structural congruence and it is over this set
that the notion of dynamics is expressed. This presentation is
essentially that of \cite{MeredithR05} with the addition of
polyadicity and summation. For readability we have relegated some of
the technical subtleties to an appendix.

\subsubsection{Process grammar}\label{subsub:process_grammar}

\begin{mathpar}
  \inferrule* [lab=synchronization] {} {{M} \bc \pzero \;|\; x?F \;|\; x!C }
  \and
  \inferrule* [lab=abstraction] {} {{F} \bc (x)P}
  \and
  \inferrule* [lab=concretion] {} {{C} \bc \langle Q \rangle}
  \and
  \inferrule* [lab=process] {} {{P,Q} \bc M \;| \;P|Q \;|\; @{x}}
  \and
  \inferrule* [lab=name] {} {{x} \bc \quotep{P}}
\end{mathpar} 

Note that $\vec{x}$ (resp. $\vec{P}$) denotes a vector of names
(resp. processes) of length $|\vec{x}|$ (resp. $|\vec{P}|$). We adopt
the following useful abbreviations.

\begin{mathpar}
   x?(\vec{y}).P := x.(\vec{y})P \and  x\clift{\vec{P}} := x.\clift{\vec{P}}
   \and x!(y) := \lift{x}{\dropn{y}}
   \and \Pi_{i=0}^{n-1}P_i := P_0 | \ldots | P_{n-1}
\end{mathpar}

\subsubsection{Structural congruence}

\paragraph{Free and bound names and alpha-equivalence.} At the
core of structural equivalence is alpha-equivalence which identifies
process that are the same up to a change of variable. Formally, we
recognize the distinction between free and bound names. The free names
of a process, $\freenames{P}$, may be calculated recursively as
follows:

\begin{mathpar}
\freenames{\pzero} := \emptyset
  \and \\
  \freenames{x?(y).P} := \{ x \} \cup (\freenames{P} \setminus \{ y \})
  \and 
  \freenames{x!\langle P \rangle} := \{ x \} \cup \{ P \} 
  \and \\
  \freenames{P|Q} := \freenames{P} \cup \freenames{Q}
  \and \\
  \freenames{@{x}} := \{ x \}
\end{mathpar}

$\pi$
$\quotep{\pi}$

$\freenames{-} : \pi \to \mathcal{P}(\quotep{\pi})$

\begin{eqnarray*}
  \freenames{\pzero} & := & \emptyset \\
  \freenames{x?(y).P} & := & \{ x \} \cup (\freenames{P} \setminus \{ y \}) \\
  \freenames{x!\langle P \rangle} & := & \{ x \} \cup \{ P \} \\
  \freenames{P|Q} & := & \freenames{P} \cup \freenames{Q} \\
  \freenames{\dropn{x}} & := & \{ x \}
\end{eqnarray*}

The bound names of a process, $\boundnames{P}$, are those names occurring in $P$
that are not free. For example, in $x?(y).0$, the name $x$ is free, while $y$ is bound.

\begin{mathpar}
  \inferrule* [lab=monoidal-laws] {} { P|Q \equiv Q|P \and P|0 \equiv P \and P|(Q|R) \equiv (P|Q)|R }
\end{mathpar}

\begin{mathpar}
  \inferrule* [lab=alpha-equivalence] {} { (x)P \equiv (y)P\{y/x\} \and y \not\in \freenames{P} }
\end{mathpar}

\begin{definition}
Then two processes, $P,Q$, are alpha-equivalent if $P = Q\{\vec{y}/\vec{x}\}$ for
some $\vec{x} \in \boundnames{Q},\vec{y} \in \boundnames{P}$, where $Q\{\vec{y}/\vec{x}\}$
denotes the capture-avoiding substitution of $\vec{y}$ for $\vec{x}$ in $Q$.
\end{definition}

\begin{definition}
  The {\em structural congruence} \cite{SangiorgiWalker} , $\equiv$,
  between processes is the least congruence containing
  alpha-equivalence, satisfying the abelian monoid laws
  (associativity, commutativity and $\pzero$ as identity) for parallel
  composition $|$ and for summation $+$.
\end{definition}

\subsection{Name equivalence}

We take name equivalence, written $\nameeq$, to be the smallest
equivalence relation generated by the following rules.

\begin{mathpar}
\inferrule*[lab=Quote-drop]
{ }
{ \quotep{@{x}} \nameeq x }

\inferrule*[lab=Struct-equiv]
{ P \scong Q }
{ \quotep{P} \nameeq \quotep{Q} }
\end{mathpar}

The astute reader will have noticed that the mutual recursion of names
and processes imposes a mutual recursion on alpha-equivalence and
structural equivalence via name-equivalence. Fortunately, all of this
works out pleasantly and we may calculate in the natural way, free of
concern. The reader interested in the details is referred to the
appendix \ref{appendix:rho_details}.

\subsection{Substitution}

We use $\Proc$ for the set of processes, $\QProc$ for the set of
names, and $\id{\{}\vec{y} / \vec{x} \id{\}}$ to denote partial maps,
$s : \QProc \rightarrow \QProc$. A map, $s$ lifts, uniquely, to a map
on process terms, $\widehat{s} : \Proc \rightarrow \Proc$ by the
following equations.

\begin{mathpar}
  (0) \psubstp{Q}{P} := 0 \\
  (R \juxtap S) \psubstp{Q}{P}
  :=    
  (R)\psubstp{Q}{P} \juxtap (S) \psubstp{Q}{P} \\
  (x?(y).R) \psubstp{Q}{P}    
  :=    
  (x)\substp{Q}{P} (z)\concat( (R \psubstn{z}{y}) \psubstp{Q}{P} ) \\
  (\lift{x}{R}) \psubstp{Q}{P}  
  :=
  \lift{(x)\substp{Q}{P}}{ R \psubstp{Q}{P} } \\
%   (\dropn{x})  \psubstp{Q}{P}       
%   := 
%   \left\{ 
%     \begin{array}{ccc} 
%       \dropn{\quotep{Q}} & & x \nameeq \quotep{P} \\
%       \dropn{x} & & otherwise \\
%     \end{array}
%   \right. 
  (\dropn{x})  \psubstp{Q}{P}       
  := 
  \left\{ 
    \begin{array}{ccc} 
      Q & & x \nameeq \quotep{P} \\
      \dropn{x} & & otherwise \\
    \end{array}
  \right.
\end{mathpar}
 

where

\begin{eqnarray}
  (x)\id{\{} \lpquote Q \rpquote / \lpquote P \rpquote \id{\}}            = 
  \left\{ 
    \begin{array}{ccc}
      \lpquote Q \rpquote & & x \nameeq \lpquote P \rpquote \\
      x & & otherwise \\
    \end{array}
  \right. \nonumber
\end{eqnarray}

and $z$ is chosen distinct from $\quotep{P}$, $\quotep{Q}$, the free
names in $Q$, and all the names in $R$. Our $\alpha$-equivalence will
be built in the standard way from this substitution.

\begin{remark}\label{rem:no_self_referential_names}
  One consequence of these definitions is that $\forall P. \quotep{P}
  \not\in \freenames{P}$.
\end{remark}

\subsection{ Dynamic quote: an example }

Anticipating something of what's to come, consider applying the
substitution, $\widehat{\id{\{}u / z \id{\}}}$, to the following pair
of processes, $\lift{w}{y!(z)}$ and $w[ \lpquote y!(z) \rpquote ]$.

\begin{eqnarray}
	\lift{w}{y!(z)}\widehat{\id{\{}u / z \id{\}}}
		& = &
		\lift{w}{y!(u)} \nonumber\\
	w[ \lpquote y!(z) \rpquote ] \widehat{ \id{\{}u / z \id{\}} }
		& = &
		w[ \lpquote y!(z) \rpquote ] \nonumber
\end{eqnarray}

Because the body of the process between quotes is impervious to
substitution, we get radically different answers. In fact, by
examining the first process in an input context,
e.g. $x?(z).\lift{w}{y!(z)}$, we see that the process under the lift
operator may be shaped by prefixed inputs binding a name inside it. In
this sense, the lift operator will be seen as a way to dynamically
construct processes before reifying them as names.

Finally equipped with these standard features we can present the
dynamics of the calculus.

\subsubsection{Operational semantics} 

Finally, we introduce the computational dynamics. What marks these
algebras as distinct from other more traditionally studied algebraic
structures, e.g. vector spaces or polynomial rings, is the manner in
which dynamics is captured. In traditional structures, dynamics is typically
expressed through morphisms between such structures, as in linear maps
between vector spaces or morphisms between rings. In algebras
associated with the semantics of computation, the dynamics is
expressed as part of the algebraic structure itself, through a
reduction reduction relation typically denoted by $\red$. Below, we
give a recursive presentation of this relation for the calculus used
in the encoding.

$\red \subseteq \pi \times \pi$
$\red : \pi \to \mathcal{P}(\pi)$

\begin{mathpar}
  \inferrule* [lab=Comm] { \textsf{match}( x_{src}, x_{trgt} ) } { x_{trgt}?(y)P \; | \; x_{src}!\langle {Q} \rangle \red P\{\quotep{Q}/y}\} }
  \and \\
  \inferrule* [lab=Par] {{P} \red {P}'} {{{P} | {Q}} \red {{P}' | {Q}}}
  \and
  \inferrule* [lab=Equiv]{{{P} \scong {P}'} \andalso {{P}' \red {Q}'} \andalso {{Q}' \scong {Q}}}{{P} \red {Q}}
\end{mathpar}

\begin{eqnarray*}
  match_{\equiv} (\quotep{P},\quotep{Q}) & := & P \equiv Q \\
  match_{\dagger}(\quotep{P},\quotep{Q}) & := & \forall R. P|Q \red^{*} R => R \red^{*} 0 \\
  match_{K}(\quotep{P},\quotep{Q}) & := & K \mbox{ for some context } K
\end{eqnarray*}

$u?(x)P | u!\langle Q \rangle \red P\{\quotep{Q}/x\}$

%We write $\wred$ for $\red^*$, and $P\red$ if $\exists Q $ such that $ P \red Q$.
We write $P\red$ if $\exists Q $ such that $ P \red Q$ and $P\not\red$, otherwise.

\section{Replication}

As mentioned before, it is known that replication (and hence
recursion) can be implemented in a higher-order process algebra
\cite{SangiorgiWalker}. As our first example of calculation with the
machinery thus far presented we give the construction explicitly in
the {\rhoc}.

\begin{eqnarray}
	D_{x} & := & \prefix{x}{y}{(\binpar{\outputp{x}{y}}{@{y}})} \nonumber\\
	\bangp_{x}{P} & := & \binpar{{x}!\langle{\binpar{D_{x}}{P}}\rangle}{D_{x}} \nonumber
\end{eqnarray}

\begin{eqnarray}
	\bangp_{x}{P} & & \nonumber\\
	=
	& {x}!\langle{(\prefix{x}{y}{(\outputp{x}{y} | @{y})) | P}}\rangle 
	      | \prefix{x}{y}{(\outputp{x}{y} | @{y})} & \nonumber\\
	\red
	& (\outputp{x}{y} | @{y})\substn{\quotep{(\prefix{x}{y}{(@{y} | \outputp{x}{y})) | P}}}{y} & \nonumber\\
	=
	& \outputp{x}{\quotep{(\prefix{x}{y}{(\outputp{x}{y} | @{y})) | P}}}
	  | {(\prefix{x}{y}{(\outputp{x}{y} | @{y})) | P}} & \nonumber\\
	\red
	& \ldots & \nonumber\\
	\red^*
	& P | P | \ldots & \nonumber
\end{eqnarray}

Of course, this encoding, as an implementation, runs away, unfolding
$\bangp{P}$ eagerly. A lazier and more implementable replication
operator, restricted to input-guarded processes, may be obtained as follows.

\begin{eqnarray}
\bangp{\prefix{u}{v}{P}} 
	:= 
	\binpar{\lift{x}{\prefix{u}{v}{(\binpar{D(x)}{P})}}}{D(x)} \nonumber
\end{eqnarray}

\begin{remark}
  Note that the lazier definition still does not deal with summation
  or mixed summation (i.e. sums over input and output). The reader is
  invited to construct definitions of replication that deal with these
  features. 

  Further, the definitions are parameterized in a name, $x$. Can you,
  gentle reader, make a definition that eliminates this parameter and
  guarantees no accidental interaction between the replication
  machinery and the process being replicated -- i.e. no accidental
  sharing of names used by the process to get its work done and the
  name(s) used by the replication to effect copying. This latter
  revision of the definition of replication is crucial to obtaining
  the expected identity $!!P \sim !P$.
\end{remark}

\begin{remark}\label{rem:paradoxical_combinator}
  The reader familiar with the lambda calculus will have noticed the
  similarity between $D$ and the paradoxical combinator.

  [Ed. note: the existence of this seems to suggest we have to be more
  restrictive on the set of processes and names we admit if we are to
  support no-cloning.]
\end{remark}

\subsubsection{Bisimulation}

The computational dynamics gives rise to another kind of equivalence,
the equivalence of computational behavior. As previously mentioned
this is typically captured \emph{via} some form of bisimulation.

% The notion we use in this paper is weak barbed bisimulation
% \cite{milner91polyadicpi}.

The notion we use in this paper is derived from weak barbed
bisimulation \cite{milner91polyadicpi}. 

\begin{definition}
An \emph{observation relation}, $\downarrow_{\mathcal N}$, over a set
of names, $\mathcal N$, is the smallest relation satisfying the rules
below.

\infrule[Out-barb]{y \in {\mathcal N}, \; x \nameeq y}
		  {\outputp{x}{v} \downarrow_{\mathcal N} x}
\infrule[Par-barb]{\mbox{$P\downarrow_{\mathcal N} x$ or $Q\downarrow_{\mathcal N} x$}}
		  {\binpar{P}{Q} \downarrow_{\mathcal N} x}

We write $P \Downarrow_{\mathcal N} x$ if there is $Q$ such that 
$P \wred Q$ and $Q \downarrow_{\mathcal N} x$.
\end{definition}

\begin{definition}
%\label{def.bbisim}
An  ${\mathcal N}$-\emph{barbed bisimulation} over a set of names, ${\mathcal N}$, is a symmetric binary relation 
${\mathcal S}_{\mathcal N}$ between agents such that $P\rel{S}_{\mathcal N}Q$ implies:
\begin{enumerate}
\item If $P \red P'$ then $Q \wred Q'$ and $P'\rel{S}_{\mathcal N} Q'$.
\item If $P\downarrow_{\mathcal N} x$, then $Q\Downarrow_{\mathcal N} x$.
\end{enumerate}
$P$ is ${\mathcal N}$-barbed bisimilar to $Q$, written
$P \wbbisim_{\mathcal N} Q$, if $P \rel{S}_{\mathcal N} Q$ for some ${\mathcal N}$-barbed bisimulation ${\mathcal S}_{\mathcal N}$.
\end{definition}

$\mathcal{R} \subseteq \pi \times \pi$

$P \mathcal{R} Q => \forall P'. P \red P' \Rightarrow \exists Q'. Q \red Q', P' \mathcal{R} Q'$

$P \vdash x \Rightarrow Q \vdash x$

\begin{mathpar}
  \inferrule*[lab=Out-barb]{x \nameeq y}{{y}!\langle{Q}\rangle \vdash x}
  \and
  \inferrule*[lab=Par-barb]{\mbox{$P\vdash x$ or $Q\vdash x$}}{\binpar{P}{Q} \vdash x}
\end{mathpar}

\subsubsection{Contexts}

One of the principle advantages of computational calculi like the
$\pi$-calculus is a well-defined notion of context,
contextual-equivalence and a correlation between
contextual-equivalence and notions of bisimulation. The notion of
context allows the decomposition of a process into (sub-)process and
its syntactic environment, its context. Thus, a context may be
thought of as a process with a ``hole'' (written $\Box$) in it. The
application of a context $M$ to a process $P$, written $M[P]$, is
tantamount to filling the hole in $M$ with $P$. In this paper we do
not need the full weight of this theory, but do make use of the notion
of context in the proof the main theorem. 

\begin{mathpar}
  \inferrule* [lab=summation] {} {{M_{M},M_{N}} \bc \Box \;|\; x.M_{A} \;|\; M_{M}+M_{N}}
  \and
  \inferrule* [lab=agent] {} {{M_{A}} \bc (\vec{x})M_{P} \;| \; \clift{P_0,\ldots,M_{P},\ldots,P_N}}
  \and \\
  \inferrule* [lab=process] {} {{M_{P}} \bc M_{N} \;| \;P|M_{P} }
\end{mathpar} 

\begin{mathpar}
  \inferrule* [lab=sychronization] {} {M_{N} \bc \Box \;|\; x?M_{F} \;|\; x!M_{C}}
  \and
  \inferrule* [lab=abstraction] {} {{M_{F}} \bc (x)M_{P} }
  \and
  \inferrule* [lab=concretion] {} {{M_{C}} \bc \langle M_{P} \rangle }
  \and \\
  \inferrule* [lab=process] {} {{M_{P}} \bc M_{N} \;| \;P|M_{P} }
\end{mathpar}

\begin{definition}[contextual application] Given a context $M$, and
  process $P$, we define the \emph{contextual application}, $M[P] :=
  M\{P/\Box\}$. That is, the contextual application of M to P is the
  substitution of $P$ for $\Box$ in $M$.
\end{definition}

$\meaningof{-} : L \to \mathcal{P}(\pi)$

\begin{mathpar}
  \inferrule* [lab=collection] {} {\meaningof{true} = \pi, \and \meaningof{~E} = \pi \setminus \meaningof{E}, \and \meaningof{E_{1} \& E_{2}} = \meaningof{E_{1}} \cap \meaningof{E_{2}}}
\end{mathpar}

\begin{mathpar}
  \inferrule* [lab=structure] {} {\meaningof{0} = \{ P \in \pi | P \equiv 0 \}, \and \\ \meaningof{E_1 | E_2} = \{ P \in \pi | P \equiv P_{1} | P_{2}, P_{1} \in \meaningof{E_{1}}, P_{2} \in \meaningof{E_2}\} }
\end{mathpar}

\begin{mathpar}
 \inferrule* [lab=behavior] {} {\meaningof{\langle a?b \rangle E} = \{ P \in \pi | P \equiv Q | u?(y)P', \\ \and \\\\ \and \\ \;\;\; u \in \meaningof{a}, \forall z.P'\{z/y\} \in \meaningof{E\{z/b\}}\}, \and \\ \meaningof{a!E} = \{ P \in \pi | P \equiv Q | x!\langle P' \rangle, x \in \meaningof{a} P' \in \meaningof{E}\} }
\end{mathpar}

\begin{mathpar}
 \inferrule* [lab=nominal] {} {\meaningof{\quotep{E}} = \{ \quotep{P} \in \quotep{\pi} | P \in \meaningof{E} \}, \and \meaningof{\quotep{P}} = \{ \quotep{Q} \in \quotep{\pi} | P \equiv Q \} \and \\ \meaningof{@\quotep{E}} = \{ P \in \pi | P \equiv @x, x \in \meaningof{E} \}}
\end{mathpar}

\begin{eqnarray*}
  \\
  \meaningof{-} : TS \to ST
\end{eqnarray*}

\begin{eqnarray*}
  \\
  L : TS \to ST
\end{eqnarray*}

\begin{eqnarray*}
  \\
  P \models E \iff P \in \meaningof{E}
\end{eqnarray*}

\begin{eqnarray*}
  P \approx_{L} Q \iff \forall E \in L. P \models E \iff Q \models E
\end{eqnarray*}

\begin{eqnarray*}
  P \approx_{K} Q
\end{eqnarray*}

\begin{eqnarray*}
  P \approx Q
\end{eqnarray*}

$\approx_{K} = \approx = \approx_{L}$

\subsubsection{Contextual duality}

Note that contexts extend the quotation operation to a family of
operations from processes to names. Given a context, $M$, we can
define a \emph{nominal context}, $\quotep{M}$ by $\quotep{M}[P] :=
\quotep{M[P]}$. To foreshadow what is to come we observe that these
operations enjoy a duality with processes very much like the duality
between vectors and maps from vectors to scalars.

Further, because the calculus is essentially higher-order, we have a
correspondence between contexts and processes. More specifically,
given a name $x$ and a context $M$ we can construct $M^{*}_{x}$ such
that 

\begin{mathpar}
  M^{*}_{x} | \lift{x}{P} \red M[P]
\end{mathpar}

namely,

\begin{mathpar}
  M^{*}_{x} := x?(u).M[\dropn{u}]
\end{mathpar}

The dependence of $M^{*}_{x}$ on a name makes it an abstraction, 

\begin{mathpar}
  M^{*} := (x)x?(u).M[\dropn{u}]
\end{mathpar}

\subsection{Additional notation}

It will sometimes be convenient to denote the process a name
quotes. We already have the notation $x = \quotep{P}$, but it will be
convenient to introduce an alternate notation, $\procn{x}$, when we
want to emphasize the connection to the use of the name. Note that, by
virtue of name equivalence, $\quotep{\procn{x}} \nameeq x$; so, the
notation is consistent with previous definitions.

Further, because names have structure it is possible to effect
substitutions on the basis of that structure. This means we need to
upgrade our notation for substitutions, which we accomplish by
adapting comprehension notation. Thus,

\begin{mathpar}
  P\{ y / x : x \in S \}
\end{mathpar}

is interpreted to mean the process derived from P by replacing (in a
capture-avoiding manner) each occurrence of $x$ in $S$ by $y$. For example,

\begin{mathpar}
  P\{ \quotep{\procn{x}|\procn{x}} / x : x \in \freenames{P} \}
\end{mathpar}

will replace each (occurrence) of a free name $x$ in $P$ by
$\quotep{\procn{x}|\procn{x}}$.

Also, we will avail ourselves of the notation $x^{L}$ and $x^{R}$ to
denote injections of a name into disjoint copies of the name
space. There are numerous ways to accomplish this. One example can be
found in \cite{MeredithR05}. This notation overloads to vectors of
names: $\vec{x}^{\pi} := (x_{i}^{\pi} \; : \; 0 \leq i < |\vec{x}| )$ where $\pi \in \{L,R\}$.

We also use $P^{\Box} := P|\Box$.

In \cite{MeredithR05} an interpretation of the new operator is
given. It turns out that there are several possible interpretations
all enjoying the requisite algebraic properties of the operator (see
\cite{milner91polyadicpi}). We will therefore make liberal use of
$(\nu\; \vec{x})P$.

% subsection the_syntax_and_semantics_of_the_notation_system (end)   

\input{qm2pi.qmops} 

\input{qm2pi.sterngerlach} 

\input{qm2pi.metric} 

% section concurrent_process_calculi (end)

%\input{qm2pi.proofsketch}

% section proof sketch (end)

%\input{qm2pi.slviaknots} 

% section spatial logic via knots (end)

\input{qm2pi.conclusion}

% section conclusion (end)

%\input{qm2pi.dtcodes} 

% section wiring algorithm (end)

\input{qm2pi.ack} 

% section acknowledgments (end)

\newpage


\bibliographystyle{plain}   
\bibliography{../../biblios/main.bib}

\input{qm2pi.rhodetails}

\end{document}

 

%\documentclass[12pt]{llncs}
%\documentclass{jktr}

\usepackage[pdftex]{hyperref}                   
\usepackage {listings}
\usepackage {mathpartir}
\usepackage{bcprules}
%\usepackage{listings}
                       
\usepackage{graphicx} 
%\usepackage[margins=2.5cm,nohead,nofoot]{geometry}
%\usepackage{geometry}
\usepackage{amsfonts}
\usepackage{amstext}
\usepackage{latexsym}
\usepackage{amssymb}
\usepackage{color}


%\include{myPreamble}
\include{qm2pi.local} 

%\ifpdf
%\usepackage[pdftex]{graphicx}
%\else
%\usepackage{graphicx}
%\fi

 % \ifpdf
%  \usepackage{pdfsync}
%  \if


%\title{Brief Article}
%\author{David F. Snyder}
%\author{L.G. Meredith}

%\address{Dept. of Math., Texas State University--San Marcos, San Marcos, TX 78666}
       
\pagestyle{empty}


\begin{document}

\lstset{language=[Objective]Caml,frame=shadowbox}

\input{qm2pi.front}

% section front matter (end)

\input{qm2pi.intro} 
 
% section introduction (end)

% \input{qm2pi.knotations} 

% section notation (end)

\input{qm2pi.process.calculi} 

% section concurrent_process_calculi_and_spatial_logics_ (end)
    
%\input{qm2pi.knots2pi} 

%\input{qm2pi.trefoil} 

%\input{qm2pi.mainthm} 

% subsection basic_interpretation (end)

%\input{qm2pi.rho.presentation} 
\subsection{The syntax and semantics of the notation system}\label{sub:the_syntax_and_semantics_of_the_notation_system} % (fold)

We now summarize a technical presentation of the calculus that
embodies our theory of dynamics. The typical presentation of such a
calculus follows the style of giving generators and relations on
them. The grammar, below, describing term constructors, freely
generates the set of processes, $\Proc$. This set is then quotiented
by a relation known as structural congruence and it is over this set
that the notion of dynamics is expressed. This presentation is
essentially that of \cite{MeredithR05} with the addition of
polyadicity and summation. For readability we have relegated some of
the technical subtleties to an appendix.

\subsubsection{Process grammar}\label{subsub:process_grammar}

\begin{mathpar}
  \inferrule* [lab=synchronization] {} {{M} \bc \pzero \;|\; x?F \;|\; x!C }
  \and
  \inferrule* [lab=abstraction] {} {{F} \bc (x)P}
  \and
  \inferrule* [lab=concretion] {} {{C} \bc \langle Q \rangle}
  \and
  \inferrule* [lab=process] {} {{P,Q} \bc M \;| \;P|Q \;|\; @{x}}
  \and
  \inferrule* [lab=name] {} {{x} \bc \quotep{P}}
\end{mathpar} 

Note that $\vec{x}$ (resp. $\vec{P}$) denotes a vector of names
(resp. processes) of length $|\vec{x}|$ (resp. $|\vec{P}|$). We adopt
the following useful abbreviations.

\begin{mathpar}
   x?(\vec{y}).P := x.(\vec{y})P \and  x\clift{\vec{P}} := x.\clift{\vec{P}}
   \and x!(y) := \lift{x}{\dropn{y}}
   \and \Pi_{i=0}^{n-1}P_i := P_0 | \ldots | P_{n-1}
\end{mathpar}

\subsubsection{Structural congruence}

\paragraph{Free and bound names and alpha-equivalence.} At the
core of structural equivalence is alpha-equivalence which identifies
process that are the same up to a change of variable. Formally, we
recognize the distinction between free and bound names. The free names
of a process, $\freenames{P}$, may be calculated recursively as
follows:

\begin{mathpar}
\freenames{\pzero} := \emptyset
  \and \\
  \freenames{x?(y).P} := \{ x \} \cup (\freenames{P} \setminus \{ y \})
  \and 
  \freenames{x!\langle P \rangle} := \{ x \} \cup \{ P \} 
  \and \\
  \freenames{P|Q} := \freenames{P} \cup \freenames{Q}
  \and \\
  \freenames{@{x}} := \{ x \}
\end{mathpar}

$\pi$
$\quotep{\pi}$

$\freenames{-} : \pi \to \mathcal{P}(\quotep{\pi})$

\begin{eqnarray*}
  \freenames{\pzero} & := & \emptyset \\
  \freenames{x?(y).P} & := & \{ x \} \cup (\freenames{P} \setminus \{ y \}) \\
  \freenames{x!\langle P \rangle} & := & \{ x \} \cup \{ P \} \\
  \freenames{P|Q} & := & \freenames{P} \cup \freenames{Q} \\
  \freenames{\dropn{x}} & := & \{ x \}
\end{eqnarray*}

The bound names of a process, $\boundnames{P}$, are those names occurring in $P$
that are not free. For example, in $x?(y).0$, the name $x$ is free, while $y$ is bound.

\begin{mathpar}
  \inferrule* [lab=monoidal-laws] {} { P|Q \equiv Q|P \and P|0 \equiv P \and P|(Q|R) \equiv (P|Q)|R }
\end{mathpar}

\begin{mathpar}
  \inferrule* [lab=alpha-equivalence] {} { (x)P \equiv (y)P\{y/x\} \and y \not\in \freenames{P} }
\end{mathpar}

\begin{definition}
Then two processes, $P,Q$, are alpha-equivalent if $P = Q\{\vec{y}/\vec{x}\}$ for
some $\vec{x} \in \boundnames{Q},\vec{y} \in \boundnames{P}$, where $Q\{\vec{y}/\vec{x}\}$
denotes the capture-avoiding substitution of $\vec{y}$ for $\vec{x}$ in $Q$.
\end{definition}

\begin{definition}
  The {\em structural congruence} \cite{SangiorgiWalker} , $\equiv$,
  between processes is the least congruence containing
  alpha-equivalence, satisfying the abelian monoid laws
  (associativity, commutativity and $\pzero$ as identity) for parallel
  composition $|$ and for summation $+$.
\end{definition}

\subsection{Name equivalence}

We take name equivalence, written $\nameeq$, to be the smallest
equivalence relation generated by the following rules.

\begin{mathpar}
\inferrule*[lab=Quote-drop]
{ }
{ \quotep{@{x}} \nameeq x }

\inferrule*[lab=Struct-equiv]
{ P \scong Q }
{ \quotep{P} \nameeq \quotep{Q} }
\end{mathpar}

The astute reader will have noticed that the mutual recursion of names
and processes imposes a mutual recursion on alpha-equivalence and
structural equivalence via name-equivalence. Fortunately, all of this
works out pleasantly and we may calculate in the natural way, free of
concern. The reader interested in the details is referred to the
appendix \ref{appendix:rho_details}.

\subsection{Substitution}

We use $\Proc$ for the set of processes, $\QProc$ for the set of
names, and $\id{\{}\vec{y} / \vec{x} \id{\}}$ to denote partial maps,
$s : \QProc \rightarrow \QProc$. A map, $s$ lifts, uniquely, to a map
on process terms, $\widehat{s} : \Proc \rightarrow \Proc$ by the
following equations.

\begin{mathpar}
  (0) \psubstp{Q}{P} := 0 \\
  (R \juxtap S) \psubstp{Q}{P}
  :=    
  (R)\psubstp{Q}{P} \juxtap (S) \psubstp{Q}{P} \\
  (x?(y).R) \psubstp{Q}{P}    
  :=    
  (x)\substp{Q}{P} (z)\concat( (R \psubstn{z}{y}) \psubstp{Q}{P} ) \\
  (\lift{x}{R}) \psubstp{Q}{P}  
  :=
  \lift{(x)\substp{Q}{P}}{ R \psubstp{Q}{P} } \\
%   (\dropn{x})  \psubstp{Q}{P}       
%   := 
%   \left\{ 
%     \begin{array}{ccc} 
%       \dropn{\quotep{Q}} & & x \nameeq \quotep{P} \\
%       \dropn{x} & & otherwise \\
%     \end{array}
%   \right. 
  (\dropn{x})  \psubstp{Q}{P}       
  := 
  \left\{ 
    \begin{array}{ccc} 
      Q & & x \nameeq \quotep{P} \\
      \dropn{x} & & otherwise \\
    \end{array}
  \right.
\end{mathpar}
 

where

\begin{eqnarray}
  (x)\id{\{} \lpquote Q \rpquote / \lpquote P \rpquote \id{\}}            = 
  \left\{ 
    \begin{array}{ccc}
      \lpquote Q \rpquote & & x \nameeq \lpquote P \rpquote \\
      x & & otherwise \\
    \end{array}
  \right. \nonumber
\end{eqnarray}

and $z$ is chosen distinct from $\quotep{P}$, $\quotep{Q}$, the free
names in $Q$, and all the names in $R$. Our $\alpha$-equivalence will
be built in the standard way from this substitution.

\begin{remark}\label{rem:no_self_referential_names}
  One consequence of these definitions is that $\forall P. \quotep{P}
  \not\in \freenames{P}$.
\end{remark}

\subsection{ Dynamic quote: an example }

Anticipating something of what's to come, consider applying the
substitution, $\widehat{\id{\{}u / z \id{\}}}$, to the following pair
of processes, $\lift{w}{y!(z)}$ and $w[ \lpquote y!(z) \rpquote ]$.

\begin{eqnarray}
	\lift{w}{y!(z)}\widehat{\id{\{}u / z \id{\}}}
		& = &
		\lift{w}{y!(u)} \nonumber\\
	w[ \lpquote y!(z) \rpquote ] \widehat{ \id{\{}u / z \id{\}} }
		& = &
		w[ \lpquote y!(z) \rpquote ] \nonumber
\end{eqnarray}

Because the body of the process between quotes is impervious to
substitution, we get radically different answers. In fact, by
examining the first process in an input context,
e.g. $x?(z).\lift{w}{y!(z)}$, we see that the process under the lift
operator may be shaped by prefixed inputs binding a name inside it. In
this sense, the lift operator will be seen as a way to dynamically
construct processes before reifying them as names.

Finally equipped with these standard features we can present the
dynamics of the calculus.

\subsubsection{Operational semantics} 

Finally, we introduce the computational dynamics. What marks these
algebras as distinct from other more traditionally studied algebraic
structures, e.g. vector spaces or polynomial rings, is the manner in
which dynamics is captured. In traditional structures, dynamics is typically
expressed through morphisms between such structures, as in linear maps
between vector spaces or morphisms between rings. In algebras
associated with the semantics of computation, the dynamics is
expressed as part of the algebraic structure itself, through a
reduction reduction relation typically denoted by $\red$. Below, we
give a recursive presentation of this relation for the calculus used
in the encoding.

$\red \subseteq \pi \times \pi$
$\red : \pi \to \mathcal{P}(\pi)$

\begin{mathpar}
  \inferrule* [lab=Comm] { \textsf{match}( x_{src}, x_{trgt} ) } { x_{trgt}?(y)P \; | \; x_{src}!\langle {Q} \rangle \red P\{\quotep{Q}/y}\} }
  \and \\
  \inferrule* [lab=Par] {{P} \red {P}'} {{{P} | {Q}} \red {{P}' | {Q}}}
  \and
  \inferrule* [lab=Equiv]{{{P} \scong {P}'} \andalso {{P}' \red {Q}'} \andalso {{Q}' \scong {Q}}}{{P} \red {Q}}
\end{mathpar}

\begin{eqnarray*}
  match_{\equiv} (\quotep{P},\quotep{Q}) & := & P \equiv Q \\
  match_{\dagger}(\quotep{P},\quotep{Q}) & := & \forall R. P|Q \red^{*} R => R \red^{*} 0 \\
  match_{K}(\quotep{P},\quotep{Q}) & := & K \mbox{ for some context } K
\end{eqnarray*}

$u?(x)P | u!\langle Q \rangle \red P\{\quotep{Q}/x\}$

%We write $\wred$ for $\red^*$, and $P\red$ if $\exists Q $ such that $ P \red Q$.
We write $P\red$ if $\exists Q $ such that $ P \red Q$ and $P\not\red$, otherwise.

\section{Replication}

As mentioned before, it is known that replication (and hence
recursion) can be implemented in a higher-order process algebra
\cite{SangiorgiWalker}. As our first example of calculation with the
machinery thus far presented we give the construction explicitly in
the {\rhoc}.

\begin{eqnarray}
	D_{x} & := & \prefix{x}{y}{(\binpar{\outputp{x}{y}}{@{y}})} \nonumber\\
	\bangp_{x}{P} & := & \binpar{{x}!\langle{\binpar{D_{x}}{P}}\rangle}{D_{x}} \nonumber
\end{eqnarray}

\begin{eqnarray}
	\bangp_{x}{P} & & \nonumber\\
	=
	& {x}!\langle{(\prefix{x}{y}{(\outputp{x}{y} | @{y})) | P}}\rangle 
	      | \prefix{x}{y}{(\outputp{x}{y} | @{y})} & \nonumber\\
	\red
	& (\outputp{x}{y} | @{y})\substn{\quotep{(\prefix{x}{y}{(@{y} | \outputp{x}{y})) | P}}}{y} & \nonumber\\
	=
	& \outputp{x}{\quotep{(\prefix{x}{y}{(\outputp{x}{y} | @{y})) | P}}}
	  | {(\prefix{x}{y}{(\outputp{x}{y} | @{y})) | P}} & \nonumber\\
	\red
	& \ldots & \nonumber\\
	\red^*
	& P | P | \ldots & \nonumber
\end{eqnarray}

Of course, this encoding, as an implementation, runs away, unfolding
$\bangp{P}$ eagerly. A lazier and more implementable replication
operator, restricted to input-guarded processes, may be obtained as follows.

\begin{eqnarray}
\bangp{\prefix{u}{v}{P}} 
	:= 
	\binpar{\lift{x}{\prefix{u}{v}{(\binpar{D(x)}{P})}}}{D(x)} \nonumber
\end{eqnarray}

\begin{remark}
  Note that the lazier definition still does not deal with summation
  or mixed summation (i.e. sums over input and output). The reader is
  invited to construct definitions of replication that deal with these
  features. 

  Further, the definitions are parameterized in a name, $x$. Can you,
  gentle reader, make a definition that eliminates this parameter and
  guarantees no accidental interaction between the replication
  machinery and the process being replicated -- i.e. no accidental
  sharing of names used by the process to get its work done and the
  name(s) used by the replication to effect copying. This latter
  revision of the definition of replication is crucial to obtaining
  the expected identity $!!P \sim !P$.
\end{remark}

\begin{remark}\label{rem:paradoxical_combinator}
  The reader familiar with the lambda calculus will have noticed the
  similarity between $D$ and the paradoxical combinator.

  [Ed. note: the existence of this seems to suggest we have to be more
  restrictive on the set of processes and names we admit if we are to
  support no-cloning.]
\end{remark}

\subsubsection{Bisimulation}

The computational dynamics gives rise to another kind of equivalence,
the equivalence of computational behavior. As previously mentioned
this is typically captured \emph{via} some form of bisimulation.

% The notion we use in this paper is weak barbed bisimulation
% \cite{milner91polyadicpi}.

The notion we use in this paper is derived from weak barbed
bisimulation \cite{milner91polyadicpi}. 

\begin{definition}
An \emph{observation relation}, $\downarrow_{\mathcal N}$, over a set
of names, $\mathcal N$, is the smallest relation satisfying the rules
below.

\infrule[Out-barb]{y \in {\mathcal N}, \; x \nameeq y}
		  {\outputp{x}{v} \downarrow_{\mathcal N} x}
\infrule[Par-barb]{\mbox{$P\downarrow_{\mathcal N} x$ or $Q\downarrow_{\mathcal N} x$}}
		  {\binpar{P}{Q} \downarrow_{\mathcal N} x}

We write $P \Downarrow_{\mathcal N} x$ if there is $Q$ such that 
$P \wred Q$ and $Q \downarrow_{\mathcal N} x$.
\end{definition}

\begin{definition}
%\label{def.bbisim}
An  ${\mathcal N}$-\emph{barbed bisimulation} over a set of names, ${\mathcal N}$, is a symmetric binary relation 
${\mathcal S}_{\mathcal N}$ between agents such that $P\rel{S}_{\mathcal N}Q$ implies:
\begin{enumerate}
\item If $P \red P'$ then $Q \wred Q'$ and $P'\rel{S}_{\mathcal N} Q'$.
\item If $P\downarrow_{\mathcal N} x$, then $Q\Downarrow_{\mathcal N} x$.
\end{enumerate}
$P$ is ${\mathcal N}$-barbed bisimilar to $Q$, written
$P \wbbisim_{\mathcal N} Q$, if $P \rel{S}_{\mathcal N} Q$ for some ${\mathcal N}$-barbed bisimulation ${\mathcal S}_{\mathcal N}$.
\end{definition}

$\mathcal{R} \subseteq \pi \times \pi$

$P \mathcal{R} Q => \forall P'. P \red P' \Rightarrow \exists Q'. Q \red Q', P' \mathcal{R} Q'$

$P \vdash x \Rightarrow Q \vdash x$

\begin{mathpar}
  \inferrule*[lab=Out-barb]{x \nameeq y}{{y}!\langle{Q}\rangle \vdash x}
  \and
  \inferrule*[lab=Par-barb]{\mbox{$P\vdash x$ or $Q\vdash x$}}{\binpar{P}{Q} \vdash x}
\end{mathpar}

\subsubsection{Contexts}

One of the principle advantages of computational calculi like the
$\pi$-calculus is a well-defined notion of context,
contextual-equivalence and a correlation between
contextual-equivalence and notions of bisimulation. The notion of
context allows the decomposition of a process into (sub-)process and
its syntactic environment, its context. Thus, a context may be
thought of as a process with a ``hole'' (written $\Box$) in it. The
application of a context $M$ to a process $P$, written $M[P]$, is
tantamount to filling the hole in $M$ with $P$. In this paper we do
not need the full weight of this theory, but do make use of the notion
of context in the proof the main theorem. 

\begin{mathpar}
  \inferrule* [lab=summation] {} {{M_{M},M_{N}} \bc \Box \;|\; x.M_{A} \;|\; M_{M}+M_{N}}
  \and
  \inferrule* [lab=agent] {} {{M_{A}} \bc (\vec{x})M_{P} \;| \; \clift{P_0,\ldots,M_{P},\ldots,P_N}}
  \and \\
  \inferrule* [lab=process] {} {{M_{P}} \bc M_{N} \;| \;P|M_{P} }
\end{mathpar} 

\begin{mathpar}
  \inferrule* [lab=sychronization] {} {M_{N} \bc \Box \;|\; x?M_{F} \;|\; x!M_{C}}
  \and
  \inferrule* [lab=abstraction] {} {{M_{F}} \bc (x)M_{P} }
  \and
  \inferrule* [lab=concretion] {} {{M_{C}} \bc \langle M_{P} \rangle }
  \and \\
  \inferrule* [lab=process] {} {{M_{P}} \bc M_{N} \;| \;P|M_{P} }
\end{mathpar}

\begin{definition}[contextual application] Given a context $M$, and
  process $P$, we define the \emph{contextual application}, $M[P] :=
  M\{P/\Box\}$. That is, the contextual application of M to P is the
  substitution of $P$ for $\Box$ in $M$.
\end{definition}

$\meaningof{-} : L \to \mathcal{P}(\pi)$

\begin{mathpar}
  \inferrule* [lab=collection] {} {\meaningof{true} = \pi, \and \meaningof{~E} = \pi \setminus \meaningof{E}, \and \meaningof{E_{1} \& E_{2}} = \meaningof{E_{1}} \cap \meaningof{E_{2}}}
\end{mathpar}

\begin{mathpar}
  \inferrule* [lab=structure] {} {\meaningof{0} = \{ P \in \pi | P \equiv 0 \}, \and \\ \meaningof{E_1 | E_2} = \{ P \in \pi | P \equiv P_{1} | P_{2}, P_{1} \in \meaningof{E_{1}}, P_{2} \in \meaningof{E_2}\} }
\end{mathpar}

\begin{mathpar}
 \inferrule* [lab=behavior] {} {\meaningof{\langle a?b \rangle E} = \{ P \in \pi | P \equiv Q | u?(y)P', \\ \and \\\\ \and \\ \;\;\; u \in \meaningof{a}, \forall z.P'\{z/y\} \in \meaningof{E\{z/b\}}\}, \and \\ \meaningof{a!E} = \{ P \in \pi | P \equiv Q | x!\langle P' \rangle, x \in \meaningof{a} P' \in \meaningof{E}\} }
\end{mathpar}

\begin{mathpar}
 \inferrule* [lab=nominal] {} {\meaningof{\quotep{E}} = \{ \quotep{P} \in \quotep{\pi} | P \in \meaningof{E} \}, \and \meaningof{\quotep{P}} = \{ \quotep{Q} \in \quotep{\pi} | P \equiv Q \} \and \\ \meaningof{@\quotep{E}} = \{ P \in \pi | P \equiv @x, x \in \meaningof{E} \}}
\end{mathpar}

\begin{eqnarray*}
  \\
  \meaningof{-} : TS \to ST
\end{eqnarray*}

\begin{eqnarray*}
  \\
  L : TS \to ST
\end{eqnarray*}

\begin{eqnarray*}
  \\
  P \models E \iff P \in \meaningof{E}
\end{eqnarray*}

\begin{eqnarray*}
  P \approx_{L} Q \iff \forall E \in L. P \models E \iff Q \models E
\end{eqnarray*}

\begin{eqnarray*}
  P \approx_{K} Q
\end{eqnarray*}

\begin{eqnarray*}
  P \approx Q
\end{eqnarray*}

$\approx_{K} = \approx = \approx_{L}$

\subsubsection{Contextual duality}

Note that contexts extend the quotation operation to a family of
operations from processes to names. Given a context, $M$, we can
define a \emph{nominal context}, $\quotep{M}$ by $\quotep{M}[P] :=
\quotep{M[P]}$. To foreshadow what is to come we observe that these
operations enjoy a duality with processes very much like the duality
between vectors and maps from vectors to scalars.

Further, because the calculus is essentially higher-order, we have a
correspondence between contexts and processes. More specifically,
given a name $x$ and a context $M$ we can construct $M^{*}_{x}$ such
that 

\begin{mathpar}
  M^{*}_{x} | \lift{x}{P} \red M[P]
\end{mathpar}

namely,

\begin{mathpar}
  M^{*}_{x} := x?(u).M[\dropn{u}]
\end{mathpar}

The dependence of $M^{*}_{x}$ on a name makes it an abstraction, 

\begin{mathpar}
  M^{*} := (x)x?(u).M[\dropn{u}]
\end{mathpar}

\subsection{Additional notation}

It will sometimes be convenient to denote the process a name
quotes. We already have the notation $x = \quotep{P}$, but it will be
convenient to introduce an alternate notation, $\procn{x}$, when we
want to emphasize the connection to the use of the name. Note that, by
virtue of name equivalence, $\quotep{\procn{x}} \nameeq x$; so, the
notation is consistent with previous definitions.

Further, because names have structure it is possible to effect
substitutions on the basis of that structure. This means we need to
upgrade our notation for substitutions, which we accomplish by
adapting comprehension notation. Thus,

\begin{mathpar}
  P\{ y / x : x \in S \}
\end{mathpar}

is interpreted to mean the process derived from P by replacing (in a
capture-avoiding manner) each occurrence of $x$ in $S$ by $y$. For example,

\begin{mathpar}
  P\{ \quotep{\procn{x}|\procn{x}} / x : x \in \freenames{P} \}
\end{mathpar}

will replace each (occurrence) of a free name $x$ in $P$ by
$\quotep{\procn{x}|\procn{x}}$.

Also, we will avail ourselves of the notation $x^{L}$ and $x^{R}$ to
denote injections of a name into disjoint copies of the name
space. There are numerous ways to accomplish this. One example can be
found in \cite{MeredithR05}. This notation overloads to vectors of
names: $\vec{x}^{\pi} := (x_{i}^{\pi} \; : \; 0 \leq i < |\vec{x}| )$ where $\pi \in \{L,R\}$.

We also use $P^{\Box} := P|\Box$.

In \cite{MeredithR05} an interpretation of the new operator is
given. It turns out that there are several possible interpretations
all enjoying the requisite algebraic properties of the operator (see
\cite{milner91polyadicpi}). We will therefore make liberal use of
$(\nu\; \vec{x})P$.

% subsection the_syntax_and_semantics_of_the_notation_system (end)   

\input{qm2pi.qmops} 

\input{qm2pi.sterngerlach} 

\input{qm2pi.metric} 

% section concurrent_process_calculi (end)

%\input{qm2pi.proofsketch}

% section proof sketch (end)

%\input{qm2pi.slviaknots} 

% section spatial logic via knots (end)

\input{qm2pi.conclusion}

% section conclusion (end)

%\input{qm2pi.dtcodes} 

% section wiring algorithm (end)

\input{qm2pi.ack} 

% section acknowledgments (end)

\newpage


\bibliographystyle{plain}   
\bibliography{../../biblios/main.bib}

\input{qm2pi.rhodetails}

\end{document}

 

%\documentclass[12pt]{llncs}
%\documentclass{jktr}

\usepackage[pdftex]{hyperref}                   
\usepackage {listings}
\usepackage {mathpartir}
\usepackage{bcprules}
%\usepackage{listings}
                       
\usepackage{graphicx} 
%\usepackage[margins=2.5cm,nohead,nofoot]{geometry}
%\usepackage{geometry}
\usepackage{amsfonts}
\usepackage{amstext}
\usepackage{latexsym}
\usepackage{amssymb}
\usepackage{color}


%\include{myPreamble}
\include{qm2pi.local} 

%\ifpdf
%\usepackage[pdftex]{graphicx}
%\else
%\usepackage{graphicx}
%\fi

 % \ifpdf
%  \usepackage{pdfsync}
%  \if


%\title{Brief Article}
%\author{David F. Snyder}
%\author{L.G. Meredith}

%\address{Dept. of Math., Texas State University--San Marcos, San Marcos, TX 78666}
       
\pagestyle{empty}


\begin{document}

\lstset{language=[Objective]Caml,frame=shadowbox}

\input{qm2pi.front}

% section front matter (end)

\input{qm2pi.intro} 
 
% section introduction (end)

% \input{qm2pi.knotations} 

% section notation (end)

\input{qm2pi.process.calculi} 

% section concurrent_process_calculi_and_spatial_logics_ (end)
    
%\input{qm2pi.knots2pi} 

%\input{qm2pi.trefoil} 

%\input{qm2pi.mainthm} 

% subsection basic_interpretation (end)

%\input{qm2pi.rho.presentation} 
\subsection{The syntax and semantics of the notation system}\label{sub:the_syntax_and_semantics_of_the_notation_system} % (fold)

We now summarize a technical presentation of the calculus that
embodies our theory of dynamics. The typical presentation of such a
calculus follows the style of giving generators and relations on
them. The grammar, below, describing term constructors, freely
generates the set of processes, $\Proc$. This set is then quotiented
by a relation known as structural congruence and it is over this set
that the notion of dynamics is expressed. This presentation is
essentially that of \cite{MeredithR05} with the addition of
polyadicity and summation. For readability we have relegated some of
the technical subtleties to an appendix.

\subsubsection{Process grammar}\label{subsub:process_grammar}

\begin{mathpar}
  \inferrule* [lab=synchronization] {} {{M} \bc \pzero \;|\; x?F \;|\; x!C }
  \and
  \inferrule* [lab=abstraction] {} {{F} \bc (x)P}
  \and
  \inferrule* [lab=concretion] {} {{C} \bc \langle Q \rangle}
  \and
  \inferrule* [lab=process] {} {{P,Q} \bc M \;| \;P|Q \;|\; @{x}}
  \and
  \inferrule* [lab=name] {} {{x} \bc \quotep{P}}
\end{mathpar} 

Note that $\vec{x}$ (resp. $\vec{P}$) denotes a vector of names
(resp. processes) of length $|\vec{x}|$ (resp. $|\vec{P}|$). We adopt
the following useful abbreviations.

\begin{mathpar}
   x?(\vec{y}).P := x.(\vec{y})P \and  x\clift{\vec{P}} := x.\clift{\vec{P}}
   \and x!(y) := \lift{x}{\dropn{y}}
   \and \Pi_{i=0}^{n-1}P_i := P_0 | \ldots | P_{n-1}
\end{mathpar}

\subsubsection{Structural congruence}

\paragraph{Free and bound names and alpha-equivalence.} At the
core of structural equivalence is alpha-equivalence which identifies
process that are the same up to a change of variable. Formally, we
recognize the distinction between free and bound names. The free names
of a process, $\freenames{P}$, may be calculated recursively as
follows:

\begin{mathpar}
\freenames{\pzero} := \emptyset
  \and \\
  \freenames{x?(y).P} := \{ x \} \cup (\freenames{P} \setminus \{ y \})
  \and 
  \freenames{x!\langle P \rangle} := \{ x \} \cup \{ P \} 
  \and \\
  \freenames{P|Q} := \freenames{P} \cup \freenames{Q}
  \and \\
  \freenames{@{x}} := \{ x \}
\end{mathpar}

$\pi$
$\quotep{\pi}$

$\freenames{-} : \pi \to \mathcal{P}(\quotep{\pi})$

\begin{eqnarray*}
  \freenames{\pzero} & := & \emptyset \\
  \freenames{x?(y).P} & := & \{ x \} \cup (\freenames{P} \setminus \{ y \}) \\
  \freenames{x!\langle P \rangle} & := & \{ x \} \cup \{ P \} \\
  \freenames{P|Q} & := & \freenames{P} \cup \freenames{Q} \\
  \freenames{\dropn{x}} & := & \{ x \}
\end{eqnarray*}

The bound names of a process, $\boundnames{P}$, are those names occurring in $P$
that are not free. For example, in $x?(y).0$, the name $x$ is free, while $y$ is bound.

\begin{mathpar}
  \inferrule* [lab=monoidal-laws] {} { P|Q \equiv Q|P \and P|0 \equiv P \and P|(Q|R) \equiv (P|Q)|R }
\end{mathpar}

\begin{mathpar}
  \inferrule* [lab=alpha-equivalence] {} { (x)P \equiv (y)P\{y/x\} \and y \not\in \freenames{P} }
\end{mathpar}

\begin{definition}
Then two processes, $P,Q$, are alpha-equivalent if $P = Q\{\vec{y}/\vec{x}\}$ for
some $\vec{x} \in \boundnames{Q},\vec{y} \in \boundnames{P}$, where $Q\{\vec{y}/\vec{x}\}$
denotes the capture-avoiding substitution of $\vec{y}$ for $\vec{x}$ in $Q$.
\end{definition}

\begin{definition}
  The {\em structural congruence} \cite{SangiorgiWalker} , $\equiv$,
  between processes is the least congruence containing
  alpha-equivalence, satisfying the abelian monoid laws
  (associativity, commutativity and $\pzero$ as identity) for parallel
  composition $|$ and for summation $+$.
\end{definition}

\subsection{Name equivalence}

We take name equivalence, written $\nameeq$, to be the smallest
equivalence relation generated by the following rules.

\begin{mathpar}
\inferrule*[lab=Quote-drop]
{ }
{ \quotep{@{x}} \nameeq x }

\inferrule*[lab=Struct-equiv]
{ P \scong Q }
{ \quotep{P} \nameeq \quotep{Q} }
\end{mathpar}

The astute reader will have noticed that the mutual recursion of names
and processes imposes a mutual recursion on alpha-equivalence and
structural equivalence via name-equivalence. Fortunately, all of this
works out pleasantly and we may calculate in the natural way, free of
concern. The reader interested in the details is referred to the
appendix \ref{appendix:rho_details}.

\subsection{Substitution}

We use $\Proc$ for the set of processes, $\QProc$ for the set of
names, and $\id{\{}\vec{y} / \vec{x} \id{\}}$ to denote partial maps,
$s : \QProc \rightarrow \QProc$. A map, $s$ lifts, uniquely, to a map
on process terms, $\widehat{s} : \Proc \rightarrow \Proc$ by the
following equations.

\begin{mathpar}
  (0) \psubstp{Q}{P} := 0 \\
  (R \juxtap S) \psubstp{Q}{P}
  :=    
  (R)\psubstp{Q}{P} \juxtap (S) \psubstp{Q}{P} \\
  (x?(y).R) \psubstp{Q}{P}    
  :=    
  (x)\substp{Q}{P} (z)\concat( (R \psubstn{z}{y}) \psubstp{Q}{P} ) \\
  (\lift{x}{R}) \psubstp{Q}{P}  
  :=
  \lift{(x)\substp{Q}{P}}{ R \psubstp{Q}{P} } \\
%   (\dropn{x})  \psubstp{Q}{P}       
%   := 
%   \left\{ 
%     \begin{array}{ccc} 
%       \dropn{\quotep{Q}} & & x \nameeq \quotep{P} \\
%       \dropn{x} & & otherwise \\
%     \end{array}
%   \right. 
  (\dropn{x})  \psubstp{Q}{P}       
  := 
  \left\{ 
    \begin{array}{ccc} 
      Q & & x \nameeq \quotep{P} \\
      \dropn{x} & & otherwise \\
    \end{array}
  \right.
\end{mathpar}
 

where

\begin{eqnarray}
  (x)\id{\{} \lpquote Q \rpquote / \lpquote P \rpquote \id{\}}            = 
  \left\{ 
    \begin{array}{ccc}
      \lpquote Q \rpquote & & x \nameeq \lpquote P \rpquote \\
      x & & otherwise \\
    \end{array}
  \right. \nonumber
\end{eqnarray}

and $z$ is chosen distinct from $\quotep{P}$, $\quotep{Q}$, the free
names in $Q$, and all the names in $R$. Our $\alpha$-equivalence will
be built in the standard way from this substitution.

\begin{remark}\label{rem:no_self_referential_names}
  One consequence of these definitions is that $\forall P. \quotep{P}
  \not\in \freenames{P}$.
\end{remark}

\subsection{ Dynamic quote: an example }

Anticipating something of what's to come, consider applying the
substitution, $\widehat{\id{\{}u / z \id{\}}}$, to the following pair
of processes, $\lift{w}{y!(z)}$ and $w[ \lpquote y!(z) \rpquote ]$.

\begin{eqnarray}
	\lift{w}{y!(z)}\widehat{\id{\{}u / z \id{\}}}
		& = &
		\lift{w}{y!(u)} \nonumber\\
	w[ \lpquote y!(z) \rpquote ] \widehat{ \id{\{}u / z \id{\}} }
		& = &
		w[ \lpquote y!(z) \rpquote ] \nonumber
\end{eqnarray}

Because the body of the process between quotes is impervious to
substitution, we get radically different answers. In fact, by
examining the first process in an input context,
e.g. $x?(z).\lift{w}{y!(z)}$, we see that the process under the lift
operator may be shaped by prefixed inputs binding a name inside it. In
this sense, the lift operator will be seen as a way to dynamically
construct processes before reifying them as names.

Finally equipped with these standard features we can present the
dynamics of the calculus.

\subsubsection{Operational semantics} 

Finally, we introduce the computational dynamics. What marks these
algebras as distinct from other more traditionally studied algebraic
structures, e.g. vector spaces or polynomial rings, is the manner in
which dynamics is captured. In traditional structures, dynamics is typically
expressed through morphisms between such structures, as in linear maps
between vector spaces or morphisms between rings. In algebras
associated with the semantics of computation, the dynamics is
expressed as part of the algebraic structure itself, through a
reduction reduction relation typically denoted by $\red$. Below, we
give a recursive presentation of this relation for the calculus used
in the encoding.

$\red \subseteq \pi \times \pi$
$\red : \pi \to \mathcal{P}(\pi)$

\begin{mathpar}
  \inferrule* [lab=Comm] { \textsf{match}( x_{src}, x_{trgt} ) } { x_{trgt}?(y)P \; | \; x_{src}!\langle {Q} \rangle \red P\{\quotep{Q}/y}\} }
  \and \\
  \inferrule* [lab=Par] {{P} \red {P}'} {{{P} | {Q}} \red {{P}' | {Q}}}
  \and
  \inferrule* [lab=Equiv]{{{P} \scong {P}'} \andalso {{P}' \red {Q}'} \andalso {{Q}' \scong {Q}}}{{P} \red {Q}}
\end{mathpar}

\begin{eqnarray*}
  match_{\equiv} (\quotep{P},\quotep{Q}) & := & P \equiv Q \\
  match_{\dagger}(\quotep{P},\quotep{Q}) & := & \forall R. P|Q \red^{*} R => R \red^{*} 0 \\
  match_{K}(\quotep{P},\quotep{Q}) & := & K \mbox{ for some context } K
\end{eqnarray*}

$u?(x)P | u!\langle Q \rangle \red P\{\quotep{Q}/x\}$

%We write $\wred$ for $\red^*$, and $P\red$ if $\exists Q $ such that $ P \red Q$.
We write $P\red$ if $\exists Q $ such that $ P \red Q$ and $P\not\red$, otherwise.

\section{Replication}

As mentioned before, it is known that replication (and hence
recursion) can be implemented in a higher-order process algebra
\cite{SangiorgiWalker}. As our first example of calculation with the
machinery thus far presented we give the construction explicitly in
the {\rhoc}.

\begin{eqnarray}
	D_{x} & := & \prefix{x}{y}{(\binpar{\outputp{x}{y}}{@{y}})} \nonumber\\
	\bangp_{x}{P} & := & \binpar{{x}!\langle{\binpar{D_{x}}{P}}\rangle}{D_{x}} \nonumber
\end{eqnarray}

\begin{eqnarray}
	\bangp_{x}{P} & & \nonumber\\
	=
	& {x}!\langle{(\prefix{x}{y}{(\outputp{x}{y} | @{y})) | P}}\rangle 
	      | \prefix{x}{y}{(\outputp{x}{y} | @{y})} & \nonumber\\
	\red
	& (\outputp{x}{y} | @{y})\substn{\quotep{(\prefix{x}{y}{(@{y} | \outputp{x}{y})) | P}}}{y} & \nonumber\\
	=
	& \outputp{x}{\quotep{(\prefix{x}{y}{(\outputp{x}{y} | @{y})) | P}}}
	  | {(\prefix{x}{y}{(\outputp{x}{y} | @{y})) | P}} & \nonumber\\
	\red
	& \ldots & \nonumber\\
	\red^*
	& P | P | \ldots & \nonumber
\end{eqnarray}

Of course, this encoding, as an implementation, runs away, unfolding
$\bangp{P}$ eagerly. A lazier and more implementable replication
operator, restricted to input-guarded processes, may be obtained as follows.

\begin{eqnarray}
\bangp{\prefix{u}{v}{P}} 
	:= 
	\binpar{\lift{x}{\prefix{u}{v}{(\binpar{D(x)}{P})}}}{D(x)} \nonumber
\end{eqnarray}

\begin{remark}
  Note that the lazier definition still does not deal with summation
  or mixed summation (i.e. sums over input and output). The reader is
  invited to construct definitions of replication that deal with these
  features. 

  Further, the definitions are parameterized in a name, $x$. Can you,
  gentle reader, make a definition that eliminates this parameter and
  guarantees no accidental interaction between the replication
  machinery and the process being replicated -- i.e. no accidental
  sharing of names used by the process to get its work done and the
  name(s) used by the replication to effect copying. This latter
  revision of the definition of replication is crucial to obtaining
  the expected identity $!!P \sim !P$.
\end{remark}

\begin{remark}\label{rem:paradoxical_combinator}
  The reader familiar with the lambda calculus will have noticed the
  similarity between $D$ and the paradoxical combinator.

  [Ed. note: the existence of this seems to suggest we have to be more
  restrictive on the set of processes and names we admit if we are to
  support no-cloning.]
\end{remark}

\subsubsection{Bisimulation}

The computational dynamics gives rise to another kind of equivalence,
the equivalence of computational behavior. As previously mentioned
this is typically captured \emph{via} some form of bisimulation.

% The notion we use in this paper is weak barbed bisimulation
% \cite{milner91polyadicpi}.

The notion we use in this paper is derived from weak barbed
bisimulation \cite{milner91polyadicpi}. 

\begin{definition}
An \emph{observation relation}, $\downarrow_{\mathcal N}$, over a set
of names, $\mathcal N$, is the smallest relation satisfying the rules
below.

\infrule[Out-barb]{y \in {\mathcal N}, \; x \nameeq y}
		  {\outputp{x}{v} \downarrow_{\mathcal N} x}
\infrule[Par-barb]{\mbox{$P\downarrow_{\mathcal N} x$ or $Q\downarrow_{\mathcal N} x$}}
		  {\binpar{P}{Q} \downarrow_{\mathcal N} x}

We write $P \Downarrow_{\mathcal N} x$ if there is $Q$ such that 
$P \wred Q$ and $Q \downarrow_{\mathcal N} x$.
\end{definition}

\begin{definition}
%\label{def.bbisim}
An  ${\mathcal N}$-\emph{barbed bisimulation} over a set of names, ${\mathcal N}$, is a symmetric binary relation 
${\mathcal S}_{\mathcal N}$ between agents such that $P\rel{S}_{\mathcal N}Q$ implies:
\begin{enumerate}
\item If $P \red P'$ then $Q \wred Q'$ and $P'\rel{S}_{\mathcal N} Q'$.
\item If $P\downarrow_{\mathcal N} x$, then $Q\Downarrow_{\mathcal N} x$.
\end{enumerate}
$P$ is ${\mathcal N}$-barbed bisimilar to $Q$, written
$P \wbbisim_{\mathcal N} Q$, if $P \rel{S}_{\mathcal N} Q$ for some ${\mathcal N}$-barbed bisimulation ${\mathcal S}_{\mathcal N}$.
\end{definition}

$\mathcal{R} \subseteq \pi \times \pi$

$P \mathcal{R} Q => \forall P'. P \red P' \Rightarrow \exists Q'. Q \red Q', P' \mathcal{R} Q'$

$P \vdash x \Rightarrow Q \vdash x$

\begin{mathpar}
  \inferrule*[lab=Out-barb]{x \nameeq y}{{y}!\langle{Q}\rangle \vdash x}
  \and
  \inferrule*[lab=Par-barb]{\mbox{$P\vdash x$ or $Q\vdash x$}}{\binpar{P}{Q} \vdash x}
\end{mathpar}

\subsubsection{Contexts}

One of the principle advantages of computational calculi like the
$\pi$-calculus is a well-defined notion of context,
contextual-equivalence and a correlation between
contextual-equivalence and notions of bisimulation. The notion of
context allows the decomposition of a process into (sub-)process and
its syntactic environment, its context. Thus, a context may be
thought of as a process with a ``hole'' (written $\Box$) in it. The
application of a context $M$ to a process $P$, written $M[P]$, is
tantamount to filling the hole in $M$ with $P$. In this paper we do
not need the full weight of this theory, but do make use of the notion
of context in the proof the main theorem. 

\begin{mathpar}
  \inferrule* [lab=summation] {} {{M_{M},M_{N}} \bc \Box \;|\; x.M_{A} \;|\; M_{M}+M_{N}}
  \and
  \inferrule* [lab=agent] {} {{M_{A}} \bc (\vec{x})M_{P} \;| \; \clift{P_0,\ldots,M_{P},\ldots,P_N}}
  \and \\
  \inferrule* [lab=process] {} {{M_{P}} \bc M_{N} \;| \;P|M_{P} }
\end{mathpar} 

\begin{mathpar}
  \inferrule* [lab=sychronization] {} {M_{N} \bc \Box \;|\; x?M_{F} \;|\; x!M_{C}}
  \and
  \inferrule* [lab=abstraction] {} {{M_{F}} \bc (x)M_{P} }
  \and
  \inferrule* [lab=concretion] {} {{M_{C}} \bc \langle M_{P} \rangle }
  \and \\
  \inferrule* [lab=process] {} {{M_{P}} \bc M_{N} \;| \;P|M_{P} }
\end{mathpar}

\begin{definition}[contextual application] Given a context $M$, and
  process $P$, we define the \emph{contextual application}, $M[P] :=
  M\{P/\Box\}$. That is, the contextual application of M to P is the
  substitution of $P$ for $\Box$ in $M$.
\end{definition}

$\meaningof{-} : L \to \mathcal{P}(\pi)$

\begin{mathpar}
  \inferrule* [lab=collection] {} {\meaningof{true} = \pi, \and \meaningof{~E} = \pi \setminus \meaningof{E}, \and \meaningof{E_{1} \& E_{2}} = \meaningof{E_{1}} \cap \meaningof{E_{2}}}
\end{mathpar}

\begin{mathpar}
  \inferrule* [lab=structure] {} {\meaningof{0} = \{ P \in \pi | P \equiv 0 \}, \and \\ \meaningof{E_1 | E_2} = \{ P \in \pi | P \equiv P_{1} | P_{2}, P_{1} \in \meaningof{E_{1}}, P_{2} \in \meaningof{E_2}\} }
\end{mathpar}

\begin{mathpar}
 \inferrule* [lab=behavior] {} {\meaningof{\langle a?b \rangle E} = \{ P \in \pi | P \equiv Q | u?(y)P', \\ \and \\\\ \and \\ \;\;\; u \in \meaningof{a}, \forall z.P'\{z/y\} \in \meaningof{E\{z/b\}}\}, \and \\ \meaningof{a!E} = \{ P \in \pi | P \equiv Q | x!\langle P' \rangle, x \in \meaningof{a} P' \in \meaningof{E}\} }
\end{mathpar}

\begin{mathpar}
 \inferrule* [lab=nominal] {} {\meaningof{\quotep{E}} = \{ \quotep{P} \in \quotep{\pi} | P \in \meaningof{E} \}, \and \meaningof{\quotep{P}} = \{ \quotep{Q} \in \quotep{\pi} | P \equiv Q \} \and \\ \meaningof{@\quotep{E}} = \{ P \in \pi | P \equiv @x, x \in \meaningof{E} \}}
\end{mathpar}

\begin{eqnarray*}
  \\
  \meaningof{-} : TS \to ST
\end{eqnarray*}

\begin{eqnarray*}
  \\
  L : TS \to ST
\end{eqnarray*}

\begin{eqnarray*}
  \\
  P \models E \iff P \in \meaningof{E}
\end{eqnarray*}

\begin{eqnarray*}
  P \approx_{L} Q \iff \forall E \in L. P \models E \iff Q \models E
\end{eqnarray*}

\begin{eqnarray*}
  P \approx_{K} Q
\end{eqnarray*}

\begin{eqnarray*}
  P \approx Q
\end{eqnarray*}

$\approx_{K} = \approx = \approx_{L}$

\subsubsection{Contextual duality}

Note that contexts extend the quotation operation to a family of
operations from processes to names. Given a context, $M$, we can
define a \emph{nominal context}, $\quotep{M}$ by $\quotep{M}[P] :=
\quotep{M[P]}$. To foreshadow what is to come we observe that these
operations enjoy a duality with processes very much like the duality
between vectors and maps from vectors to scalars.

Further, because the calculus is essentially higher-order, we have a
correspondence between contexts and processes. More specifically,
given a name $x$ and a context $M$ we can construct $M^{*}_{x}$ such
that 

\begin{mathpar}
  M^{*}_{x} | \lift{x}{P} \red M[P]
\end{mathpar}

namely,

\begin{mathpar}
  M^{*}_{x} := x?(u).M[\dropn{u}]
\end{mathpar}

The dependence of $M^{*}_{x}$ on a name makes it an abstraction, 

\begin{mathpar}
  M^{*} := (x)x?(u).M[\dropn{u}]
\end{mathpar}

\subsection{Additional notation}

It will sometimes be convenient to denote the process a name
quotes. We already have the notation $x = \quotep{P}$, but it will be
convenient to introduce an alternate notation, $\procn{x}$, when we
want to emphasize the connection to the use of the name. Note that, by
virtue of name equivalence, $\quotep{\procn{x}} \nameeq x$; so, the
notation is consistent with previous definitions.

Further, because names have structure it is possible to effect
substitutions on the basis of that structure. This means we need to
upgrade our notation for substitutions, which we accomplish by
adapting comprehension notation. Thus,

\begin{mathpar}
  P\{ y / x : x \in S \}
\end{mathpar}

is interpreted to mean the process derived from P by replacing (in a
capture-avoiding manner) each occurrence of $x$ in $S$ by $y$. For example,

\begin{mathpar}
  P\{ \quotep{\procn{x}|\procn{x}} / x : x \in \freenames{P} \}
\end{mathpar}

will replace each (occurrence) of a free name $x$ in $P$ by
$\quotep{\procn{x}|\procn{x}}$.

Also, we will avail ourselves of the notation $x^{L}$ and $x^{R}$ to
denote injections of a name into disjoint copies of the name
space. There are numerous ways to accomplish this. One example can be
found in \cite{MeredithR05}. This notation overloads to vectors of
names: $\vec{x}^{\pi} := (x_{i}^{\pi} \; : \; 0 \leq i < |\vec{x}| )$ where $\pi \in \{L,R\}$.

We also use $P^{\Box} := P|\Box$.

In \cite{MeredithR05} an interpretation of the new operator is
given. It turns out that there are several possible interpretations
all enjoying the requisite algebraic properties of the operator (see
\cite{milner91polyadicpi}). We will therefore make liberal use of
$(\nu\; \vec{x})P$.

% subsection the_syntax_and_semantics_of_the_notation_system (end)   

\input{qm2pi.qmops} 

\input{qm2pi.sterngerlach} 

\input{qm2pi.metric} 

% section concurrent_process_calculi (end)

%\input{qm2pi.proofsketch}

% section proof sketch (end)

%\input{qm2pi.slviaknots} 

% section spatial logic via knots (end)

\input{qm2pi.conclusion}

% section conclusion (end)

%\input{qm2pi.dtcodes} 

% section wiring algorithm (end)

\input{qm2pi.ack} 

% section acknowledgments (end)

\newpage


\bibliographystyle{plain}   
\bibliography{../../biblios/main.bib}

\input{qm2pi.rhodetails}

\end{document}

 

% subsection basic_interpretation (end)

%\input{qm2pi.rho.presentation} 
\subsection{The syntax and semantics of the notation system}\label{sub:the_syntax_and_semantics_of_the_notation_system} % (fold)

We now summarize a technical presentation of the calculus that
embodies our theory of dynamics. The typical presentation of such a
calculus follows the style of giving generators and relations on
them. The grammar, below, describing term constructors, freely
generates the set of processes, $\Proc$. This set is then quotiented
by a relation known as structural congruence and it is over this set
that the notion of dynamics is expressed. This presentation is
essentially that of \cite{MeredithR05} with the addition of
polyadicity and summation. For readability we have relegated some of
the technical subtleties to an appendix.

\subsubsection{Process grammar}\label{subsub:process_grammar}

\begin{mathpar}
  \inferrule* [lab=synchronization] {} {{M} \bc \pzero \;|\; x?F \;|\; x!C }
  \and
  \inferrule* [lab=abstraction] {} {{F} \bc (x)P}
  \and
  \inferrule* [lab=concretion] {} {{C} \bc \langle Q \rangle}
  \and
  \inferrule* [lab=process] {} {{P,Q} \bc M \;| \;P|Q \;|\; @{x}}
  \and
  \inferrule* [lab=name] {} {{x} \bc \quotep{P}}
\end{mathpar} 

Note that $\vec{x}$ (resp. $\vec{P}$) denotes a vector of names
(resp. processes) of length $|\vec{x}|$ (resp. $|\vec{P}|$). We adopt
the following useful abbreviations.

\begin{mathpar}
   x?(\vec{y}).P := x.(\vec{y})P \and  x\clift{\vec{P}} := x.\clift{\vec{P}}
   \and x!(y) := \lift{x}{\dropn{y}}
   \and \Pi_{i=0}^{n-1}P_i := P_0 | \ldots | P_{n-1}
\end{mathpar}

\subsubsection{Structural congruence}

\paragraph{Free and bound names and alpha-equivalence.} At the
core of structural equivalence is alpha-equivalence which identifies
process that are the same up to a change of variable. Formally, we
recognize the distinction between free and bound names. The free names
of a process, $\freenames{P}$, may be calculated recursively as
follows:

\begin{mathpar}
\freenames{\pzero} := \emptyset
  \and \\
  \freenames{x?(y).P} := \{ x \} \cup (\freenames{P} \setminus \{ y \})
  \and 
  \freenames{x!\langle P \rangle} := \{ x \} \cup \{ P \} 
  \and \\
  \freenames{P|Q} := \freenames{P} \cup \freenames{Q}
  \and \\
  \freenames{@{x}} := \{ x \}
\end{mathpar}

$\pi$
$\quotep{\pi}$

$\freenames{-} : \pi \to \mathcal{P}(\quotep{\pi})$

\begin{eqnarray*}
  \freenames{\pzero} & := & \emptyset \\
  \freenames{x?(y).P} & := & \{ x \} \cup (\freenames{P} \setminus \{ y \}) \\
  \freenames{x!\langle P \rangle} & := & \{ x \} \cup \{ P \} \\
  \freenames{P|Q} & := & \freenames{P} \cup \freenames{Q} \\
  \freenames{\dropn{x}} & := & \{ x \}
\end{eqnarray*}

The bound names of a process, $\boundnames{P}$, are those names occurring in $P$
that are not free. For example, in $x?(y).0$, the name $x$ is free, while $y$ is bound.

\begin{mathpar}
  \inferrule* [lab=monoidal-laws] {} { P|Q \equiv Q|P \and P|0 \equiv P \and P|(Q|R) \equiv (P|Q)|R }
\end{mathpar}

\begin{mathpar}
  \inferrule* [lab=alpha-equivalence] {} { (x)P \equiv (y)P\{y/x\} \and y \not\in \freenames{P} }
\end{mathpar}

\begin{definition}
Then two processes, $P,Q$, are alpha-equivalent if $P = Q\{\vec{y}/\vec{x}\}$ for
some $\vec{x} \in \boundnames{Q},\vec{y} \in \boundnames{P}$, where $Q\{\vec{y}/\vec{x}\}$
denotes the capture-avoiding substitution of $\vec{y}$ for $\vec{x}$ in $Q$.
\end{definition}

\begin{definition}
  The {\em structural congruence} \cite{SangiorgiWalker} , $\equiv$,
  between processes is the least congruence containing
  alpha-equivalence, satisfying the abelian monoid laws
  (associativity, commutativity and $\pzero$ as identity) for parallel
  composition $|$ and for summation $+$.
\end{definition}

\subsection{Name equivalence}

We take name equivalence, written $\nameeq$, to be the smallest
equivalence relation generated by the following rules.

\begin{mathpar}
\inferrule*[lab=Quote-drop]
{ }
{ \quotep{@{x}} \nameeq x }

\inferrule*[lab=Struct-equiv]
{ P \scong Q }
{ \quotep{P} \nameeq \quotep{Q} }
\end{mathpar}

The astute reader will have noticed that the mutual recursion of names
and processes imposes a mutual recursion on alpha-equivalence and
structural equivalence via name-equivalence. Fortunately, all of this
works out pleasantly and we may calculate in the natural way, free of
concern. The reader interested in the details is referred to the
appendix \ref{appendix:rho_details}.

\subsection{Substitution}

We use $\Proc$ for the set of processes, $\QProc$ for the set of
names, and $\id{\{}\vec{y} / \vec{x} \id{\}}$ to denote partial maps,
$s : \QProc \rightarrow \QProc$. A map, $s$ lifts, uniquely, to a map
on process terms, $\widehat{s} : \Proc \rightarrow \Proc$ by the
following equations.

\begin{mathpar}
  (0) \psubstp{Q}{P} := 0 \\
  (R \juxtap S) \psubstp{Q}{P}
  :=    
  (R)\psubstp{Q}{P} \juxtap (S) \psubstp{Q}{P} \\
  (x?(y).R) \psubstp{Q}{P}    
  :=    
  (x)\substp{Q}{P} (z)\concat( (R \psubstn{z}{y}) \psubstp{Q}{P} ) \\
  (\lift{x}{R}) \psubstp{Q}{P}  
  :=
  \lift{(x)\substp{Q}{P}}{ R \psubstp{Q}{P} } \\
%   (\dropn{x})  \psubstp{Q}{P}       
%   := 
%   \left\{ 
%     \begin{array}{ccc} 
%       \dropn{\quotep{Q}} & & x \nameeq \quotep{P} \\
%       \dropn{x} & & otherwise \\
%     \end{array}
%   \right. 
  (\dropn{x})  \psubstp{Q}{P}       
  := 
  \left\{ 
    \begin{array}{ccc} 
      Q & & x \nameeq \quotep{P} \\
      \dropn{x} & & otherwise \\
    \end{array}
  \right.
\end{mathpar}
 

where

\begin{eqnarray}
  (x)\id{\{} \lpquote Q \rpquote / \lpquote P \rpquote \id{\}}            = 
  \left\{ 
    \begin{array}{ccc}
      \lpquote Q \rpquote & & x \nameeq \lpquote P \rpquote \\
      x & & otherwise \\
    \end{array}
  \right. \nonumber
\end{eqnarray}

and $z$ is chosen distinct from $\quotep{P}$, $\quotep{Q}$, the free
names in $Q$, and all the names in $R$. Our $\alpha$-equivalence will
be built in the standard way from this substitution.

\begin{remark}\label{rem:no_self_referential_names}
  One consequence of these definitions is that $\forall P. \quotep{P}
  \not\in \freenames{P}$.
\end{remark}

\subsection{ Dynamic quote: an example }

Anticipating something of what's to come, consider applying the
substitution, $\widehat{\id{\{}u / z \id{\}}}$, to the following pair
of processes, $\lift{w}{y!(z)}$ and $w[ \lpquote y!(z) \rpquote ]$.

\begin{eqnarray}
	\lift{w}{y!(z)}\widehat{\id{\{}u / z \id{\}}}
		& = &
		\lift{w}{y!(u)} \nonumber\\
	w[ \lpquote y!(z) \rpquote ] \widehat{ \id{\{}u / z \id{\}} }
		& = &
		w[ \lpquote y!(z) \rpquote ] \nonumber
\end{eqnarray}

Because the body of the process between quotes is impervious to
substitution, we get radically different answers. In fact, by
examining the first process in an input context,
e.g. $x?(z).\lift{w}{y!(z)}$, we see that the process under the lift
operator may be shaped by prefixed inputs binding a name inside it. In
this sense, the lift operator will be seen as a way to dynamically
construct processes before reifying them as names.

Finally equipped with these standard features we can present the
dynamics of the calculus.

\subsubsection{Operational semantics} 

Finally, we introduce the computational dynamics. What marks these
algebras as distinct from other more traditionally studied algebraic
structures, e.g. vector spaces or polynomial rings, is the manner in
which dynamics is captured. In traditional structures, dynamics is typically
expressed through morphisms between such structures, as in linear maps
between vector spaces or morphisms between rings. In algebras
associated with the semantics of computation, the dynamics is
expressed as part of the algebraic structure itself, through a
reduction reduction relation typically denoted by $\red$. Below, we
give a recursive presentation of this relation for the calculus used
in the encoding.

$\red \subseteq \pi \times \pi$
$\red : \pi \to \mathcal{P}(\pi)$

\begin{mathpar}
  \inferrule* [lab=Comm] { \textsf{match}( x_{src}, x_{trgt} ) } { x_{trgt}?(y)P \; | \; x_{src}!\langle {Q} \rangle \red P\{\quotep{Q}/y}\} }
  \and \\
  \inferrule* [lab=Par] {{P} \red {P}'} {{{P} | {Q}} \red {{P}' | {Q}}}
  \and
  \inferrule* [lab=Equiv]{{{P} \scong {P}'} \andalso {{P}' \red {Q}'} \andalso {{Q}' \scong {Q}}}{{P} \red {Q}}
\end{mathpar}

\begin{eqnarray*}
  match_{\equiv} (\quotep{P},\quotep{Q}) & := & P \equiv Q \\
  match_{\dagger}(\quotep{P},\quotep{Q}) & := & \forall R. P|Q \red^{*} R => R \red^{*} 0 \\
  match_{K}(\quotep{P},\quotep{Q}) & := & K \mbox{ for some context } K
\end{eqnarray*}

$u?(x)P | u!\langle Q \rangle \red P\{\quotep{Q}/x\}$

%We write $\wred$ for $\red^*$, and $P\red$ if $\exists Q $ such that $ P \red Q$.
We write $P\red$ if $\exists Q $ such that $ P \red Q$ and $P\not\red$, otherwise.

\section{Replication}

As mentioned before, it is known that replication (and hence
recursion) can be implemented in a higher-order process algebra
\cite{SangiorgiWalker}. As our first example of calculation with the
machinery thus far presented we give the construction explicitly in
the {\rhoc}.

\begin{eqnarray}
	D_{x} & := & \prefix{x}{y}{(\binpar{\outputp{x}{y}}{@{y}})} \nonumber\\
	\bangp_{x}{P} & := & \binpar{{x}!\langle{\binpar{D_{x}}{P}}\rangle}{D_{x}} \nonumber
\end{eqnarray}

\begin{eqnarray}
	\bangp_{x}{P} & & \nonumber\\
	=
	& {x}!\langle{(\prefix{x}{y}{(\outputp{x}{y} | @{y})) | P}}\rangle 
	      | \prefix{x}{y}{(\outputp{x}{y} | @{y})} & \nonumber\\
	\red
	& (\outputp{x}{y} | @{y})\substn{\quotep{(\prefix{x}{y}{(@{y} | \outputp{x}{y})) | P}}}{y} & \nonumber\\
	=
	& \outputp{x}{\quotep{(\prefix{x}{y}{(\outputp{x}{y} | @{y})) | P}}}
	  | {(\prefix{x}{y}{(\outputp{x}{y} | @{y})) | P}} & \nonumber\\
	\red
	& \ldots & \nonumber\\
	\red^*
	& P | P | \ldots & \nonumber
\end{eqnarray}

Of course, this encoding, as an implementation, runs away, unfolding
$\bangp{P}$ eagerly. A lazier and more implementable replication
operator, restricted to input-guarded processes, may be obtained as follows.

\begin{eqnarray}
\bangp{\prefix{u}{v}{P}} 
	:= 
	\binpar{\lift{x}{\prefix{u}{v}{(\binpar{D(x)}{P})}}}{D(x)} \nonumber
\end{eqnarray}

\begin{remark}
  Note that the lazier definition still does not deal with summation
  or mixed summation (i.e. sums over input and output). The reader is
  invited to construct definitions of replication that deal with these
  features. 

  Further, the definitions are parameterized in a name, $x$. Can you,
  gentle reader, make a definition that eliminates this parameter and
  guarantees no accidental interaction between the replication
  machinery and the process being replicated -- i.e. no accidental
  sharing of names used by the process to get its work done and the
  name(s) used by the replication to effect copying. This latter
  revision of the definition of replication is crucial to obtaining
  the expected identity $!!P \sim !P$.
\end{remark}

\begin{remark}\label{rem:paradoxical_combinator}
  The reader familiar with the lambda calculus will have noticed the
  similarity between $D$ and the paradoxical combinator.

  [Ed. note: the existence of this seems to suggest we have to be more
  restrictive on the set of processes and names we admit if we are to
  support no-cloning.]
\end{remark}

\subsubsection{Bisimulation}

The computational dynamics gives rise to another kind of equivalence,
the equivalence of computational behavior. As previously mentioned
this is typically captured \emph{via} some form of bisimulation.

% The notion we use in this paper is weak barbed bisimulation
% \cite{milner91polyadicpi}.

The notion we use in this paper is derived from weak barbed
bisimulation \cite{milner91polyadicpi}. 

\begin{definition}
An \emph{observation relation}, $\downarrow_{\mathcal N}$, over a set
of names, $\mathcal N$, is the smallest relation satisfying the rules
below.

\infrule[Out-barb]{y \in {\mathcal N}, \; x \nameeq y}
		  {\outputp{x}{v} \downarrow_{\mathcal N} x}
\infrule[Par-barb]{\mbox{$P\downarrow_{\mathcal N} x$ or $Q\downarrow_{\mathcal N} x$}}
		  {\binpar{P}{Q} \downarrow_{\mathcal N} x}

We write $P \Downarrow_{\mathcal N} x$ if there is $Q$ such that 
$P \wred Q$ and $Q \downarrow_{\mathcal N} x$.
\end{definition}

\begin{definition}
%\label{def.bbisim}
An  ${\mathcal N}$-\emph{barbed bisimulation} over a set of names, ${\mathcal N}$, is a symmetric binary relation 
${\mathcal S}_{\mathcal N}$ between agents such that $P\rel{S}_{\mathcal N}Q$ implies:
\begin{enumerate}
\item If $P \red P'$ then $Q \wred Q'$ and $P'\rel{S}_{\mathcal N} Q'$.
\item If $P\downarrow_{\mathcal N} x$, then $Q\Downarrow_{\mathcal N} x$.
\end{enumerate}
$P$ is ${\mathcal N}$-barbed bisimilar to $Q$, written
$P \wbbisim_{\mathcal N} Q$, if $P \rel{S}_{\mathcal N} Q$ for some ${\mathcal N}$-barbed bisimulation ${\mathcal S}_{\mathcal N}$.
\end{definition}

$\mathcal{R} \subseteq \pi \times \pi$

$P \mathcal{R} Q => \forall P'. P \red P' \Rightarrow \exists Q'. Q \red Q', P' \mathcal{R} Q'$

$P \vdash x \Rightarrow Q \vdash x$

\begin{mathpar}
  \inferrule*[lab=Out-barb]{x \nameeq y}{{y}!\langle{Q}\rangle \vdash x}
  \and
  \inferrule*[lab=Par-barb]{\mbox{$P\vdash x$ or $Q\vdash x$}}{\binpar{P}{Q} \vdash x}
\end{mathpar}

\subsubsection{Contexts}

One of the principle advantages of computational calculi like the
$\pi$-calculus is a well-defined notion of context,
contextual-equivalence and a correlation between
contextual-equivalence and notions of bisimulation. The notion of
context allows the decomposition of a process into (sub-)process and
its syntactic environment, its context. Thus, a context may be
thought of as a process with a ``hole'' (written $\Box$) in it. The
application of a context $M$ to a process $P$, written $M[P]$, is
tantamount to filling the hole in $M$ with $P$. In this paper we do
not need the full weight of this theory, but do make use of the notion
of context in the proof the main theorem. 

\begin{mathpar}
  \inferrule* [lab=summation] {} {{M_{M},M_{N}} \bc \Box \;|\; x.M_{A} \;|\; M_{M}+M_{N}}
  \and
  \inferrule* [lab=agent] {} {{M_{A}} \bc (\vec{x})M_{P} \;| \; \clift{P_0,\ldots,M_{P},\ldots,P_N}}
  \and \\
  \inferrule* [lab=process] {} {{M_{P}} \bc M_{N} \;| \;P|M_{P} }
\end{mathpar} 

\begin{mathpar}
  \inferrule* [lab=sychronization] {} {M_{N} \bc \Box \;|\; x?M_{F} \;|\; x!M_{C}}
  \and
  \inferrule* [lab=abstraction] {} {{M_{F}} \bc (x)M_{P} }
  \and
  \inferrule* [lab=concretion] {} {{M_{C}} \bc \langle M_{P} \rangle }
  \and \\
  \inferrule* [lab=process] {} {{M_{P}} \bc M_{N} \;| \;P|M_{P} }
\end{mathpar}

\begin{definition}[contextual application] Given a context $M$, and
  process $P$, we define the \emph{contextual application}, $M[P] :=
  M\{P/\Box\}$. That is, the contextual application of M to P is the
  substitution of $P$ for $\Box$ in $M$.
\end{definition}

$\meaningof{-} : L \to \mathcal{P}(\pi)$

\begin{mathpar}
  \inferrule* [lab=collection] {} {\meaningof{true} = \pi, \and \meaningof{~E} = \pi \setminus \meaningof{E}, \and \meaningof{E_{1} \& E_{2}} = \meaningof{E_{1}} \cap \meaningof{E_{2}}}
\end{mathpar}

\begin{mathpar}
  \inferrule* [lab=structure] {} {\meaningof{0} = \{ P \in \pi | P \equiv 0 \}, \and \\ \meaningof{E_1 | E_2} = \{ P \in \pi | P \equiv P_{1} | P_{2}, P_{1} \in \meaningof{E_{1}}, P_{2} \in \meaningof{E_2}\} }
\end{mathpar}

\begin{mathpar}
 \inferrule* [lab=behavior] {} {\meaningof{\langle a?b \rangle E} = \{ P \in \pi | P \equiv Q | u?(y)P', \\ \and \\\\ \and \\ \;\;\; u \in \meaningof{a}, \forall z.P'\{z/y\} \in \meaningof{E\{z/b\}}\}, \and \\ \meaningof{a!E} = \{ P \in \pi | P \equiv Q | x!\langle P' \rangle, x \in \meaningof{a} P' \in \meaningof{E}\} }
\end{mathpar}

\begin{mathpar}
 \inferrule* [lab=nominal] {} {\meaningof{\quotep{E}} = \{ \quotep{P} \in \quotep{\pi} | P \in \meaningof{E} \}, \and \meaningof{\quotep{P}} = \{ \quotep{Q} \in \quotep{\pi} | P \equiv Q \} \and \\ \meaningof{@\quotep{E}} = \{ P \in \pi | P \equiv @x, x \in \meaningof{E} \}}
\end{mathpar}

\begin{eqnarray*}
  \\
  \meaningof{-} : TS \to ST
\end{eqnarray*}

\begin{eqnarray*}
  \\
  L : TS \to ST
\end{eqnarray*}

\begin{eqnarray*}
  \\
  P \models E \iff P \in \meaningof{E}
\end{eqnarray*}

\begin{eqnarray*}
  P \approx_{L} Q \iff \forall E \in L. P \models E \iff Q \models E
\end{eqnarray*}

\begin{eqnarray*}
  P \approx_{K} Q
\end{eqnarray*}

\begin{eqnarray*}
  P \approx Q
\end{eqnarray*}

$\approx_{K} = \approx = \approx_{L}$

\subsubsection{Contextual duality}

Note that contexts extend the quotation operation to a family of
operations from processes to names. Given a context, $M$, we can
define a \emph{nominal context}, $\quotep{M}$ by $\quotep{M}[P] :=
\quotep{M[P]}$. To foreshadow what is to come we observe that these
operations enjoy a duality with processes very much like the duality
between vectors and maps from vectors to scalars.

Further, because the calculus is essentially higher-order, we have a
correspondence between contexts and processes. More specifically,
given a name $x$ and a context $M$ we can construct $M^{*}_{x}$ such
that 

\begin{mathpar}
  M^{*}_{x} | \lift{x}{P} \red M[P]
\end{mathpar}

namely,

\begin{mathpar}
  M^{*}_{x} := x?(u).M[\dropn{u}]
\end{mathpar}

The dependence of $M^{*}_{x}$ on a name makes it an abstraction, 

\begin{mathpar}
  M^{*} := (x)x?(u).M[\dropn{u}]
\end{mathpar}

\subsection{Additional notation}

It will sometimes be convenient to denote the process a name
quotes. We already have the notation $x = \quotep{P}$, but it will be
convenient to introduce an alternate notation, $\procn{x}$, when we
want to emphasize the connection to the use of the name. Note that, by
virtue of name equivalence, $\quotep{\procn{x}} \nameeq x$; so, the
notation is consistent with previous definitions.

Further, because names have structure it is possible to effect
substitutions on the basis of that structure. This means we need to
upgrade our notation for substitutions, which we accomplish by
adapting comprehension notation. Thus,

\begin{mathpar}
  P\{ y / x : x \in S \}
\end{mathpar}

is interpreted to mean the process derived from P by replacing (in a
capture-avoiding manner) each occurrence of $x$ in $S$ by $y$. For example,

\begin{mathpar}
  P\{ \quotep{\procn{x}|\procn{x}} / x : x \in \freenames{P} \}
\end{mathpar}

will replace each (occurrence) of a free name $x$ in $P$ by
$\quotep{\procn{x}|\procn{x}}$.

Also, we will avail ourselves of the notation $x^{L}$ and $x^{R}$ to
denote injections of a name into disjoint copies of the name
space. There are numerous ways to accomplish this. One example can be
found in \cite{MeredithR05}. This notation overloads to vectors of
names: $\vec{x}^{\pi} := (x_{i}^{\pi} \; : \; 0 \leq i < |\vec{x}| )$ where $\pi \in \{L,R\}$.

We also use $P^{\Box} := P|\Box$.

In \cite{MeredithR05} an interpretation of the new operator is
given. It turns out that there are several possible interpretations
all enjoying the requisite algebraic properties of the operator (see
\cite{milner91polyadicpi}). We will therefore make liberal use of
$(\nu\; \vec{x})P$.

% subsection the_syntax_and_semantics_of_the_notation_system (end)   

\section{Interpretation of QM}
\subsection{Supporting definitions}
\subsubsection{Multiplication}
\begin{mathpar}
  \quotep{Q} \cdot \quotep{R} := \quotep{Q|R}
  \and \\
  \quotep{Q} \cdot P := P\{ \quotep{Q|R} / \quotep{R} : \quotep{R} \in \freenames{P} \}
\end{mathpar}

\paragraph{Discussion}
The first line needs little explanation. The second line says that
each free name of the process is replaced with the multiplication of
that name by the scalar. Multiplication of a scalar (name) by a state
(process) results in a process all the names of which have been `moved
over' by parallel composition with the process the scalar
quotes. There is a subtlety that the bound names have to be
manipulated so that multiplied names aren't accidentally
captured. There are many ways to achieve this.

\begin{remark}\label{rem:multiplication_identities}
  The reader is invited to verify that for all $x,y,z \in \QProc$ and $P \in \Proc$
  \begin{mathpar}
    x \cdot \quotep{0} \equiv x 
    \and
    x \cdot y \equiv y \cdot x
    \and
    x \cdot (y \cdot z) \equiv (x \cdot y) \cdot z
    \and \\
    \quotep{0} \cdot P \equiv P
    \and \\
    x \cdot (y \cdot P) \equiv (x \cdot y) \cdot P
    \and \\
    x \cdot (P|Q) \equiv (x \cdot P) | (x \cdot Q)
    \and \\    
  \end{mathpar}
\end{remark}

\subsubsection{Tensor product}

We define a tensor product on processes by structural induction.

\paragraph{Tensor of sums} First note that all summations, including
$\pzero$ and sequence, can be written $\Sigma_{i} x_{i}.A_{i} +
\Sigma_{j} x_{j}.C_{j}$, where we have grouped input-guarded processes
together and output-guarded processes together.

Thus, we can define the tensor product of two summations, $N_{1}\otimes N_{2}$, where

\begin{mathpar}
  N_{1} := \Sigma_{i} x_{i}.A_{i} + \Sigma_{j} x_{j}.C_{j}
  \and
  N_{2} := \Sigma_{i'} y_{i'}.B_{i'} + \Sigma_{j'} y_{j'}.D_{j'} 
\end{mathpar}

as follows.

\begin{mathpar}
  \Sigma_{i} x_{i}.A_{i} + \Sigma_{j} x_{j}.C_{j} \otimes \Sigma_{i'}
  y_{i'}.B_{i'} + \Sigma_{j'} y_{j'}.D_{j'} 
  \and \\
  := \; \Sigma_{i} \Sigma_{i'} \quotep{\stackrel{\vee}{x_{i}}| \stackrel{\vee}{y_{i'}}}.(A_{i}\otimes B_{i'}) \; | \; \Sigma_{i'} \Sigma_{i} \quotep{\stackrel{\vee}{y_{i'}}|\stackrel{\vee}{x_{i}}}.(B_{i'}\otimes A_{i})
  \and
  \;\; | \;\; \Sigma_{j} \Sigma_{j'} \quotep{\stackrel{\vee}{x_{j}}|\stackrel{\vee}{y_{j'}}}.(A_{j}\otimes B_{j'}) \; | \; \Sigma_{j'} \Sigma_{j} \quotep{\stackrel{\vee}{y_{j'}}|\stackrel{\vee}{x_{j}}}.(B_{j'}\otimes A_{j})
\end{mathpar}

\begin{remark}
  Do we need to $x^{L}$ and $y^{R}$ for this construction as well?
\end{remark}

\paragraph{Tensor of parallel compositions} Next, we distribute tensor
over par.

\begin{mathpar}
  P_{1}|P_{2} \otimes Q_{1}|Q_{2} := (P_{1} \otimes Q_{1}) | (P_{1}
  \otimes Q_{2}) | (P_{2} \otimes Q_{1}) | (P_{2} \otimes Q_{2})
\end{mathpar}

\paragraph{Tensor with dropped names} We treat tensor of a
process with a dropped name as parallel composition.

\begin{mathpar}
  P \otimes \dropn{x} := P | \dropn{x}
\end{mathpar}

\paragraph{Tensor of agents}

Finally, we need to define tensor on agents. Note that the definition
of tensor on normal products only tensors inputs with inputs and
outputs with outputs. Thus, we only have to define the operation on
``homogeneous'' pairings.

\begin{mathpar}
  (\vec{x})P \otimes (\vec{y})Q
  \and \\
  := (x_{0}^{L}|y_{0}^{R},\ldots,x_{0}^{L}|y_{n}^{R},\ldots,x_{m}^{L}|y_{0}^{R},\ldots,x_{m}^{L}|y_{n}^R)(P\{ \vec{x}^{L}/\vec{x}\} \otimes Q \{ \vec{y}^{R}/\vec{y}\})
  \and \\
  \clift{\vec{P}} \otimes \clift{\vec{Q}}
  \and \\
  := \clift{P_{0}\otimes Q_{0},\ldots,P_{0}\otimes Q_{n},\ldots,P_{m}\otimes Q_{0},\ldots,P_{m}\otimes Q_{n}}
\end{mathpar}

\begin{remark}
  Observe that arities of tensored abstractions matches arities of
  tensored concretions if the original arities matched. Note also that
  the length of the arities corresponds to the increase in dimension
  we see in ordinary vector space tensor product.
\end{remark}

\begin{remark}
  Operationally, this definition distributes the tensor down to
  components ``linked'' by summation. Tensor over summation is
  intriguing in that it mixes names. Moreover, as a consequence of the
  way it mixes names we have the identities for all $x \in \QProc$ and
  $P,Q \in \Proc$

  \begin{mathpar}
    (x \cdot P) \otimes Q \equiv x \cdot (P \otimes Q) \equiv P \otimes (x \cdot Q)
    \and
    P \otimes \pzero \equiv P
  \end{mathpar}

  that the reader is invited to verify.
\end{remark}

\subsubsection{Annihilation}
\begin{mathpar}
  P^{\perp} := \{ Q | \forall R. P|Q \red^{*} R \Rightarrow R \red^{*} \pzero \}
  \and \\
  P^{\underline{\perp}} := \Sigma_{Q \in P^{\perp}} \quotep{Q}?(y).(\dropn{y}|Q) | \Sigma_{Q \in P^{\perp}} \quotep{Q}\clift{\Box}
\end{mathpar}

\paragraph{Discussion} The reader will note that $P^{\perp}$ is a
\emph{set} of processes, while $P^{\underline{\perp}}$ is a
\emph{context}. We call the set $P^{\perp}$ the \emph{annihilators} of
$P$. The parallel composition of a process in the annihilators of $P$
with $P$ will result in a process, the state space of which has all
paths eventually leading to $\pzero$. Execution may endure loops; but
under reasonable conditions of fairness (naturally guaranteed under
most notions of bisimulation) such a composite process cannot get
stuck in such a loop and will, eventually pop out and terminate.

The context $P^{\underline{\perp}}$ is ready and willing to ``take the
$P$ out of'' the process to which it is applied. It will effectively
transmit the code of the process to which it is applied to one of the
annihilators and run the process against it.

\subsubsection{Evaluation}
We fix $M$ a domain of fully abstract interpretation with an equality
coincident with bisimulation. We take $\meaningof{\cdot} : \Proc \to
M$ to be the map interpreting processes and $\nmeaningof{\cdot} : \M
\to Proc$ to be the map running the other way. Then we define

\begin{mathpar}
  \int P := \nmeaningof{\meaningof{P}}
\end{mathpar}

\paragraph{Discussion}
There are many fully abstract interpretations of Milner's
$\pi$-calculus. Any of them can be used as a basis for interpreting
the reflective calculus here. Equipped with such a domain it is
largely a matter of grinding through to check that the Yoneda
construction for the normalization-by-evaluation program can be
extended to this setting.

\begin{remark}
  The reader is invited to verify that $\int (P^{\underline{\perp}}[P]) = 0$.
\end{remark}

\subsection{Quantum mechanics}

Table \ref{tbl:core_qm_op_defns} gives the core operational definitions

\begin{table}[htp]\label{tbl:core_qm_op_defns}
  \center{
    \fbox{
      \begin{tabular}{c|c}
        quantum mechanics & process calculus \\
        \hline
        scalar & $x := \quotep{P}$ \\
        state vector & $\state{P} := P$ \\
        dual & $\state{P}^{*} := \event{P^{\underline{\perp}}} := \quotep{P^{\underline{\perp}}}[-]$ \\
        matrix & $ \Sigma_{\alpha} \state{P_{\alpha}}x_{\alpha}\event{Q_{\alpha}}$ \\
        vector addition & $\state{P} + \state{Q} := \state{P | Q}$ \\
        tensor product & $\state{P} \otimes \state{Q} := \state{P \otimes Q}$ \\
        inner product & $\innerprod{P}{Q} := \quotep{\int P^{\underline{\perp}}[Q]}$ \\
      \end{tabular}
    }
  }
  \caption{QM - operational definitions}
\end{table}

where

\begin{mathpar}
  \prmatrix{P}{Q} := \fprmatrix{P}{\quotep{\pzero}}{Q}
  \and
  \fprmatrix{P}{x}{Q} := (\state{P},x,\event{Q})
  \and
  (\fprmatrix{P}{x}{Q})(\state{R}) := x \cdot \innerprod{Q}{R} \cdot \state{P}
  \and
  (\fprmatrix{P}{x}{Q})(\event{R}) := x \cdot \innerprod{R}{P} \cdot \event{Q}
\end{mathpar}

\paragraph{Discussion}
As promised: vectors (aka states) are represented as processes; duals
as contextual duals; inner product definition should be compared with
standard inner product definition for ....

\begin{remark}
  Assuming $\int (P^{\underline{\perp}}[P]) = 0$, the reader is
  invited to verify that $(\fprmatrix{P}{x}{P})(\state{P}) = x \cdot \state{P}$.
\end{remark}

\begin{remark}
  The reader is invited to verify that $\innerprod{P}{Q}$ could
  equally well have been written $\quotep{\int \stackrel{\vee}{x}}$
  where $x = \event{P^{\underline{\perp}}}(Q)$.

  One of the motivations for this remark is that there is another way
  to factor these operations. We could package up evaluation in the dual:

  \begin{mathpar}
    \state{P}^{*} := \event{\int P^{\underline{\perp}}} := \quotep{\int P^{\underline{\perp}}}[-]
  \end{mathpar}

  and then have inner product defined by
  
  \begin{mathpar}
    \innerprod{P}{Q} := \event{P}(Q)
  \end{mathpar}

  Hopefully, experience with the calculations will provide guidance on
  the best factoring.
\end{remark}

\begin{remark}
  Assuming $\int (P^{\underline{\perp}}[P]) = 0$, the reader is
  invited to verify that $\forall P,Q. (\prmatrix{0}{Q})(\state{0}) =
  \state{0}$ and dually $(\prmatrix{P}{0})(\event{0}) = \event{0}$.
\end{remark}

\begin{remark}
  i'm a little worried that i don't (yet) have proper support for
  complex conjugacy. But, the observation above may give us a
  clue. According to Abramsky, it must be the case that the scalars
  are iso to the homset of the identity for the tensor -- which the
  observation above characterizes. 

  For now, we will simply bookmark the notion with $\overline{x}$.
\end{remark}

\subsubsection{Adjointness}

We need to give a definition of $(\cdot)^{\dagger}$ for matrices. The
obvious candidate definition is
\begin{mathpar}
(\Sigma_{\alpha}\fprmatrix{P_{\alpha}}{x_{\alpha}}{Q_{\alpha}})^{\dagger}
= \Sigma_{\alpha}\fprmatrix{(Q_{\alpha}^{\underline{\perp}})^{*}}{\overline{x}_{\alpha}}{P_{\alpha}^{\underline{\perp}}} 
\end{mathpar}

But, $(Q_{\alpha}^{\underline{\perp}})^{*}$ requires a name along
which to communicate the process to achieve the context application.

\subsubsection{Basis for a basis}
If processes label states and ``addition'' of states (a.k.a. vector
addition) is interpreted as parallel composition, what corresponds to
notions of linear independence and basis? Here, we recall that Yoshida
has developed a set of \emph{combinators} for an asynchronous verison
of Milner's $\pi$-calculus. These are a finite set of processes such
any process can be expressed as parallel composition of these
combinators together with liberal uses of the new operator and
replication. We can simply give a translation of these into the
present calculus and have reasonable expectation that the property
carries over. That is, that the resultant set allows to express all
processes via parallel composition. Note, however, that there is no
new operator or replication in this calculus. As a result, we expect
that the corresponding set is actually infinite. That is, we expect
that the space is actually infinite dimensional.

\begin{remark}
  The attentive reader may be a bit concerned. Certainly, the
  collection $S$, $K$ and $I$ is a finite set of
  combinators. Shouldn't we expect to see a finite set of combinators
  for an effectively equivalent system? i am very sympathetic to this
  critique and feel it warrants full attention. On the other hand, i
  also have in mind the following analogy. The natural numbers, as a
  monoid under addition, has exactly $1$ generator, while the natural
  numbers, as a monoid under multiplication, has countably many
  generators (the primes). We observe that the application of the
  lambda calculus is much less resource sensitive than the parallel
  composition of the $\pi$-calculus. Could it be the case that we have
  an analogy of the form
  
  \begin{mathpar}
    m + n : MN :: m*n : M|N
  \end{mathpar}

  giving a similar blow up in the set of ``primes''?  This is such a
  wonderful thought that, even if it's not true, i think it's worth
  writing down.
\end{remark}
 

\documentclass[12pt]{llncs}
%\documentclass{jktr}

\usepackage[pdftex]{hyperref}                   
\usepackage {listings}
\usepackage {mathpartir}
\usepackage{bcprules}
%\usepackage{listings}
                       
\usepackage{graphicx} 
%\usepackage[margins=2.5cm,nohead,nofoot]{geometry}
%\usepackage{geometry}
\usepackage{amsfonts}
\usepackage{amstext}
\usepackage{latexsym}
\usepackage{amssymb}
\usepackage{color}


%\include{myPreamble}
\include{qm2pi.local} 

%\ifpdf
%\usepackage[pdftex]{graphicx}
%\else
%\usepackage{graphicx}
%\fi

 % \ifpdf
%  \usepackage{pdfsync}
%  \if


%\title{Brief Article}
%\author{David F. Snyder}
%\author{L.G. Meredith}

%\address{Dept. of Math., Texas State University--San Marcos, San Marcos, TX 78666}
       
\pagestyle{empty}


\begin{document}

\lstset{language=[Objective]Caml,frame=shadowbox}

\input{qm2pi.front}

% section front matter (end)

\input{qm2pi.intro} 
 
% section introduction (end)

% \input{qm2pi.knotations} 

% section notation (end)

\input{qm2pi.process.calculi} 

% section concurrent_process_calculi_and_spatial_logics_ (end)
    
%\input{qm2pi.knots2pi} 

%\input{qm2pi.trefoil} 

%\input{qm2pi.mainthm} 

% subsection basic_interpretation (end)

%\input{qm2pi.rho.presentation} 
\subsection{The syntax and semantics of the notation system}\label{sub:the_syntax_and_semantics_of_the_notation_system} % (fold)

We now summarize a technical presentation of the calculus that
embodies our theory of dynamics. The typical presentation of such a
calculus follows the style of giving generators and relations on
them. The grammar, below, describing term constructors, freely
generates the set of processes, $\Proc$. This set is then quotiented
by a relation known as structural congruence and it is over this set
that the notion of dynamics is expressed. This presentation is
essentially that of \cite{MeredithR05} with the addition of
polyadicity and summation. For readability we have relegated some of
the technical subtleties to an appendix.

\subsubsection{Process grammar}\label{subsub:process_grammar}

\begin{mathpar}
  \inferrule* [lab=synchronization] {} {{M} \bc \pzero \;|\; x?F \;|\; x!C }
  \and
  \inferrule* [lab=abstraction] {} {{F} \bc (x)P}
  \and
  \inferrule* [lab=concretion] {} {{C} \bc \langle Q \rangle}
  \and
  \inferrule* [lab=process] {} {{P,Q} \bc M \;| \;P|Q \;|\; @{x}}
  \and
  \inferrule* [lab=name] {} {{x} \bc \quotep{P}}
\end{mathpar} 

Note that $\vec{x}$ (resp. $\vec{P}$) denotes a vector of names
(resp. processes) of length $|\vec{x}|$ (resp. $|\vec{P}|$). We adopt
the following useful abbreviations.

\begin{mathpar}
   x?(\vec{y}).P := x.(\vec{y})P \and  x\clift{\vec{P}} := x.\clift{\vec{P}}
   \and x!(y) := \lift{x}{\dropn{y}}
   \and \Pi_{i=0}^{n-1}P_i := P_0 | \ldots | P_{n-1}
\end{mathpar}

\subsubsection{Structural congruence}

\paragraph{Free and bound names and alpha-equivalence.} At the
core of structural equivalence is alpha-equivalence which identifies
process that are the same up to a change of variable. Formally, we
recognize the distinction between free and bound names. The free names
of a process, $\freenames{P}$, may be calculated recursively as
follows:

\begin{mathpar}
\freenames{\pzero} := \emptyset
  \and \\
  \freenames{x?(y).P} := \{ x \} \cup (\freenames{P} \setminus \{ y \})
  \and 
  \freenames{x!\langle P \rangle} := \{ x \} \cup \{ P \} 
  \and \\
  \freenames{P|Q} := \freenames{P} \cup \freenames{Q}
  \and \\
  \freenames{@{x}} := \{ x \}
\end{mathpar}

$\pi$
$\quotep{\pi}$

$\freenames{-} : \pi \to \mathcal{P}(\quotep{\pi})$

\begin{eqnarray*}
  \freenames{\pzero} & := & \emptyset \\
  \freenames{x?(y).P} & := & \{ x \} \cup (\freenames{P} \setminus \{ y \}) \\
  \freenames{x!\langle P \rangle} & := & \{ x \} \cup \{ P \} \\
  \freenames{P|Q} & := & \freenames{P} \cup \freenames{Q} \\
  \freenames{\dropn{x}} & := & \{ x \}
\end{eqnarray*}

The bound names of a process, $\boundnames{P}$, are those names occurring in $P$
that are not free. For example, in $x?(y).0$, the name $x$ is free, while $y$ is bound.

\begin{mathpar}
  \inferrule* [lab=monoidal-laws] {} { P|Q \equiv Q|P \and P|0 \equiv P \and P|(Q|R) \equiv (P|Q)|R }
\end{mathpar}

\begin{mathpar}
  \inferrule* [lab=alpha-equivalence] {} { (x)P \equiv (y)P\{y/x\} \and y \not\in \freenames{P} }
\end{mathpar}

\begin{definition}
Then two processes, $P,Q$, are alpha-equivalent if $P = Q\{\vec{y}/\vec{x}\}$ for
some $\vec{x} \in \boundnames{Q},\vec{y} \in \boundnames{P}$, where $Q\{\vec{y}/\vec{x}\}$
denotes the capture-avoiding substitution of $\vec{y}$ for $\vec{x}$ in $Q$.
\end{definition}

\begin{definition}
  The {\em structural congruence} \cite{SangiorgiWalker} , $\equiv$,
  between processes is the least congruence containing
  alpha-equivalence, satisfying the abelian monoid laws
  (associativity, commutativity and $\pzero$ as identity) for parallel
  composition $|$ and for summation $+$.
\end{definition}

\subsection{Name equivalence}

We take name equivalence, written $\nameeq$, to be the smallest
equivalence relation generated by the following rules.

\begin{mathpar}
\inferrule*[lab=Quote-drop]
{ }
{ \quotep{@{x}} \nameeq x }

\inferrule*[lab=Struct-equiv]
{ P \scong Q }
{ \quotep{P} \nameeq \quotep{Q} }
\end{mathpar}

The astute reader will have noticed that the mutual recursion of names
and processes imposes a mutual recursion on alpha-equivalence and
structural equivalence via name-equivalence. Fortunately, all of this
works out pleasantly and we may calculate in the natural way, free of
concern. The reader interested in the details is referred to the
appendix \ref{appendix:rho_details}.

\subsection{Substitution}

We use $\Proc$ for the set of processes, $\QProc$ for the set of
names, and $\id{\{}\vec{y} / \vec{x} \id{\}}$ to denote partial maps,
$s : \QProc \rightarrow \QProc$. A map, $s$ lifts, uniquely, to a map
on process terms, $\widehat{s} : \Proc \rightarrow \Proc$ by the
following equations.

\begin{mathpar}
  (0) \psubstp{Q}{P} := 0 \\
  (R \juxtap S) \psubstp{Q}{P}
  :=    
  (R)\psubstp{Q}{P} \juxtap (S) \psubstp{Q}{P} \\
  (x?(y).R) \psubstp{Q}{P}    
  :=    
  (x)\substp{Q}{P} (z)\concat( (R \psubstn{z}{y}) \psubstp{Q}{P} ) \\
  (\lift{x}{R}) \psubstp{Q}{P}  
  :=
  \lift{(x)\substp{Q}{P}}{ R \psubstp{Q}{P} } \\
%   (\dropn{x})  \psubstp{Q}{P}       
%   := 
%   \left\{ 
%     \begin{array}{ccc} 
%       \dropn{\quotep{Q}} & & x \nameeq \quotep{P} \\
%       \dropn{x} & & otherwise \\
%     \end{array}
%   \right. 
  (\dropn{x})  \psubstp{Q}{P}       
  := 
  \left\{ 
    \begin{array}{ccc} 
      Q & & x \nameeq \quotep{P} \\
      \dropn{x} & & otherwise \\
    \end{array}
  \right.
\end{mathpar}
 

where

\begin{eqnarray}
  (x)\id{\{} \lpquote Q \rpquote / \lpquote P \rpquote \id{\}}            = 
  \left\{ 
    \begin{array}{ccc}
      \lpquote Q \rpquote & & x \nameeq \lpquote P \rpquote \\
      x & & otherwise \\
    \end{array}
  \right. \nonumber
\end{eqnarray}

and $z$ is chosen distinct from $\quotep{P}$, $\quotep{Q}$, the free
names in $Q$, and all the names in $R$. Our $\alpha$-equivalence will
be built in the standard way from this substitution.

\begin{remark}\label{rem:no_self_referential_names}
  One consequence of these definitions is that $\forall P. \quotep{P}
  \not\in \freenames{P}$.
\end{remark}

\subsection{ Dynamic quote: an example }

Anticipating something of what's to come, consider applying the
substitution, $\widehat{\id{\{}u / z \id{\}}}$, to the following pair
of processes, $\lift{w}{y!(z)}$ and $w[ \lpquote y!(z) \rpquote ]$.

\begin{eqnarray}
	\lift{w}{y!(z)}\widehat{\id{\{}u / z \id{\}}}
		& = &
		\lift{w}{y!(u)} \nonumber\\
	w[ \lpquote y!(z) \rpquote ] \widehat{ \id{\{}u / z \id{\}} }
		& = &
		w[ \lpquote y!(z) \rpquote ] \nonumber
\end{eqnarray}

Because the body of the process between quotes is impervious to
substitution, we get radically different answers. In fact, by
examining the first process in an input context,
e.g. $x?(z).\lift{w}{y!(z)}$, we see that the process under the lift
operator may be shaped by prefixed inputs binding a name inside it. In
this sense, the lift operator will be seen as a way to dynamically
construct processes before reifying them as names.

Finally equipped with these standard features we can present the
dynamics of the calculus.

\subsubsection{Operational semantics} 

Finally, we introduce the computational dynamics. What marks these
algebras as distinct from other more traditionally studied algebraic
structures, e.g. vector spaces or polynomial rings, is the manner in
which dynamics is captured. In traditional structures, dynamics is typically
expressed through morphisms between such structures, as in linear maps
between vector spaces or morphisms between rings. In algebras
associated with the semantics of computation, the dynamics is
expressed as part of the algebraic structure itself, through a
reduction reduction relation typically denoted by $\red$. Below, we
give a recursive presentation of this relation for the calculus used
in the encoding.

$\red \subseteq \pi \times \pi$
$\red : \pi \to \mathcal{P}(\pi)$

\begin{mathpar}
  \inferrule* [lab=Comm] { \textsf{match}( x_{src}, x_{trgt} ) } { x_{trgt}?(y)P \; | \; x_{src}!\langle {Q} \rangle \red P\{\quotep{Q}/y}\} }
  \and \\
  \inferrule* [lab=Par] {{P} \red {P}'} {{{P} | {Q}} \red {{P}' | {Q}}}
  \and
  \inferrule* [lab=Equiv]{{{P} \scong {P}'} \andalso {{P}' \red {Q}'} \andalso {{Q}' \scong {Q}}}{{P} \red {Q}}
\end{mathpar}

\begin{eqnarray*}
  match_{\equiv} (\quotep{P},\quotep{Q}) & := & P \equiv Q \\
  match_{\dagger}(\quotep{P},\quotep{Q}) & := & \forall R. P|Q \red^{*} R => R \red^{*} 0 \\
  match_{K}(\quotep{P},\quotep{Q}) & := & K \mbox{ for some context } K
\end{eqnarray*}

$u?(x)P | u!\langle Q \rangle \red P\{\quotep{Q}/x\}$

%We write $\wred$ for $\red^*$, and $P\red$ if $\exists Q $ such that $ P \red Q$.
We write $P\red$ if $\exists Q $ such that $ P \red Q$ and $P\not\red$, otherwise.

\section{Replication}

As mentioned before, it is known that replication (and hence
recursion) can be implemented in a higher-order process algebra
\cite{SangiorgiWalker}. As our first example of calculation with the
machinery thus far presented we give the construction explicitly in
the {\rhoc}.

\begin{eqnarray}
	D_{x} & := & \prefix{x}{y}{(\binpar{\outputp{x}{y}}{@{y}})} \nonumber\\
	\bangp_{x}{P} & := & \binpar{{x}!\langle{\binpar{D_{x}}{P}}\rangle}{D_{x}} \nonumber
\end{eqnarray}

\begin{eqnarray}
	\bangp_{x}{P} & & \nonumber\\
	=
	& {x}!\langle{(\prefix{x}{y}{(\outputp{x}{y} | @{y})) | P}}\rangle 
	      | \prefix{x}{y}{(\outputp{x}{y} | @{y})} & \nonumber\\
	\red
	& (\outputp{x}{y} | @{y})\substn{\quotep{(\prefix{x}{y}{(@{y} | \outputp{x}{y})) | P}}}{y} & \nonumber\\
	=
	& \outputp{x}{\quotep{(\prefix{x}{y}{(\outputp{x}{y} | @{y})) | P}}}
	  | {(\prefix{x}{y}{(\outputp{x}{y} | @{y})) | P}} & \nonumber\\
	\red
	& \ldots & \nonumber\\
	\red^*
	& P | P | \ldots & \nonumber
\end{eqnarray}

Of course, this encoding, as an implementation, runs away, unfolding
$\bangp{P}$ eagerly. A lazier and more implementable replication
operator, restricted to input-guarded processes, may be obtained as follows.

\begin{eqnarray}
\bangp{\prefix{u}{v}{P}} 
	:= 
	\binpar{\lift{x}{\prefix{u}{v}{(\binpar{D(x)}{P})}}}{D(x)} \nonumber
\end{eqnarray}

\begin{remark}
  Note that the lazier definition still does not deal with summation
  or mixed summation (i.e. sums over input and output). The reader is
  invited to construct definitions of replication that deal with these
  features. 

  Further, the definitions are parameterized in a name, $x$. Can you,
  gentle reader, make a definition that eliminates this parameter and
  guarantees no accidental interaction between the replication
  machinery and the process being replicated -- i.e. no accidental
  sharing of names used by the process to get its work done and the
  name(s) used by the replication to effect copying. This latter
  revision of the definition of replication is crucial to obtaining
  the expected identity $!!P \sim !P$.
\end{remark}

\begin{remark}\label{rem:paradoxical_combinator}
  The reader familiar with the lambda calculus will have noticed the
  similarity between $D$ and the paradoxical combinator.

  [Ed. note: the existence of this seems to suggest we have to be more
  restrictive on the set of processes and names we admit if we are to
  support no-cloning.]
\end{remark}

\subsubsection{Bisimulation}

The computational dynamics gives rise to another kind of equivalence,
the equivalence of computational behavior. As previously mentioned
this is typically captured \emph{via} some form of bisimulation.

% The notion we use in this paper is weak barbed bisimulation
% \cite{milner91polyadicpi}.

The notion we use in this paper is derived from weak barbed
bisimulation \cite{milner91polyadicpi}. 

\begin{definition}
An \emph{observation relation}, $\downarrow_{\mathcal N}$, over a set
of names, $\mathcal N$, is the smallest relation satisfying the rules
below.

\infrule[Out-barb]{y \in {\mathcal N}, \; x \nameeq y}
		  {\outputp{x}{v} \downarrow_{\mathcal N} x}
\infrule[Par-barb]{\mbox{$P\downarrow_{\mathcal N} x$ or $Q\downarrow_{\mathcal N} x$}}
		  {\binpar{P}{Q} \downarrow_{\mathcal N} x}

We write $P \Downarrow_{\mathcal N} x$ if there is $Q$ such that 
$P \wred Q$ and $Q \downarrow_{\mathcal N} x$.
\end{definition}

\begin{definition}
%\label{def.bbisim}
An  ${\mathcal N}$-\emph{barbed bisimulation} over a set of names, ${\mathcal N}$, is a symmetric binary relation 
${\mathcal S}_{\mathcal N}$ between agents such that $P\rel{S}_{\mathcal N}Q$ implies:
\begin{enumerate}
\item If $P \red P'$ then $Q \wred Q'$ and $P'\rel{S}_{\mathcal N} Q'$.
\item If $P\downarrow_{\mathcal N} x$, then $Q\Downarrow_{\mathcal N} x$.
\end{enumerate}
$P$ is ${\mathcal N}$-barbed bisimilar to $Q$, written
$P \wbbisim_{\mathcal N} Q$, if $P \rel{S}_{\mathcal N} Q$ for some ${\mathcal N}$-barbed bisimulation ${\mathcal S}_{\mathcal N}$.
\end{definition}

$\mathcal{R} \subseteq \pi \times \pi$

$P \mathcal{R} Q => \forall P'. P \red P' \Rightarrow \exists Q'. Q \red Q', P' \mathcal{R} Q'$

$P \vdash x \Rightarrow Q \vdash x$

\begin{mathpar}
  \inferrule*[lab=Out-barb]{x \nameeq y}{{y}!\langle{Q}\rangle \vdash x}
  \and
  \inferrule*[lab=Par-barb]{\mbox{$P\vdash x$ or $Q\vdash x$}}{\binpar{P}{Q} \vdash x}
\end{mathpar}

\subsubsection{Contexts}

One of the principle advantages of computational calculi like the
$\pi$-calculus is a well-defined notion of context,
contextual-equivalence and a correlation between
contextual-equivalence and notions of bisimulation. The notion of
context allows the decomposition of a process into (sub-)process and
its syntactic environment, its context. Thus, a context may be
thought of as a process with a ``hole'' (written $\Box$) in it. The
application of a context $M$ to a process $P$, written $M[P]$, is
tantamount to filling the hole in $M$ with $P$. In this paper we do
not need the full weight of this theory, but do make use of the notion
of context in the proof the main theorem. 

\begin{mathpar}
  \inferrule* [lab=summation] {} {{M_{M},M_{N}} \bc \Box \;|\; x.M_{A} \;|\; M_{M}+M_{N}}
  \and
  \inferrule* [lab=agent] {} {{M_{A}} \bc (\vec{x})M_{P} \;| \; \clift{P_0,\ldots,M_{P},\ldots,P_N}}
  \and \\
  \inferrule* [lab=process] {} {{M_{P}} \bc M_{N} \;| \;P|M_{P} }
\end{mathpar} 

\begin{mathpar}
  \inferrule* [lab=sychronization] {} {M_{N} \bc \Box \;|\; x?M_{F} \;|\; x!M_{C}}
  \and
  \inferrule* [lab=abstraction] {} {{M_{F}} \bc (x)M_{P} }
  \and
  \inferrule* [lab=concretion] {} {{M_{C}} \bc \langle M_{P} \rangle }
  \and \\
  \inferrule* [lab=process] {} {{M_{P}} \bc M_{N} \;| \;P|M_{P} }
\end{mathpar}

\begin{definition}[contextual application] Given a context $M$, and
  process $P$, we define the \emph{contextual application}, $M[P] :=
  M\{P/\Box\}$. That is, the contextual application of M to P is the
  substitution of $P$ for $\Box$ in $M$.
\end{definition}

$\meaningof{-} : L \to \mathcal{P}(\pi)$

\begin{mathpar}
  \inferrule* [lab=collection] {} {\meaningof{true} = \pi, \and \meaningof{~E} = \pi \setminus \meaningof{E}, \and \meaningof{E_{1} \& E_{2}} = \meaningof{E_{1}} \cap \meaningof{E_{2}}}
\end{mathpar}

\begin{mathpar}
  \inferrule* [lab=structure] {} {\meaningof{0} = \{ P \in \pi | P \equiv 0 \}, \and \\ \meaningof{E_1 | E_2} = \{ P \in \pi | P \equiv P_{1} | P_{2}, P_{1} \in \meaningof{E_{1}}, P_{2} \in \meaningof{E_2}\} }
\end{mathpar}

\begin{mathpar}
 \inferrule* [lab=behavior] {} {\meaningof{\langle a?b \rangle E} = \{ P \in \pi | P \equiv Q | u?(y)P', \\ \and \\\\ \and \\ \;\;\; u \in \meaningof{a}, \forall z.P'\{z/y\} \in \meaningof{E\{z/b\}}\}, \and \\ \meaningof{a!E} = \{ P \in \pi | P \equiv Q | x!\langle P' \rangle, x \in \meaningof{a} P' \in \meaningof{E}\} }
\end{mathpar}

\begin{mathpar}
 \inferrule* [lab=nominal] {} {\meaningof{\quotep{E}} = \{ \quotep{P} \in \quotep{\pi} | P \in \meaningof{E} \}, \and \meaningof{\quotep{P}} = \{ \quotep{Q} \in \quotep{\pi} | P \equiv Q \} \and \\ \meaningof{@\quotep{E}} = \{ P \in \pi | P \equiv @x, x \in \meaningof{E} \}}
\end{mathpar}

\begin{eqnarray*}
  \\
  \meaningof{-} : TS \to ST
\end{eqnarray*}

\begin{eqnarray*}
  \\
  L : TS \to ST
\end{eqnarray*}

\begin{eqnarray*}
  \\
  P \models E \iff P \in \meaningof{E}
\end{eqnarray*}

\begin{eqnarray*}
  P \approx_{L} Q \iff \forall E \in L. P \models E \iff Q \models E
\end{eqnarray*}

\begin{eqnarray*}
  P \approx_{K} Q
\end{eqnarray*}

\begin{eqnarray*}
  P \approx Q
\end{eqnarray*}

$\approx_{K} = \approx = \approx_{L}$

\subsubsection{Contextual duality}

Note that contexts extend the quotation operation to a family of
operations from processes to names. Given a context, $M$, we can
define a \emph{nominal context}, $\quotep{M}$ by $\quotep{M}[P] :=
\quotep{M[P]}$. To foreshadow what is to come we observe that these
operations enjoy a duality with processes very much like the duality
between vectors and maps from vectors to scalars.

Further, because the calculus is essentially higher-order, we have a
correspondence between contexts and processes. More specifically,
given a name $x$ and a context $M$ we can construct $M^{*}_{x}$ such
that 

\begin{mathpar}
  M^{*}_{x} | \lift{x}{P} \red M[P]
\end{mathpar}

namely,

\begin{mathpar}
  M^{*}_{x} := x?(u).M[\dropn{u}]
\end{mathpar}

The dependence of $M^{*}_{x}$ on a name makes it an abstraction, 

\begin{mathpar}
  M^{*} := (x)x?(u).M[\dropn{u}]
\end{mathpar}

\subsection{Additional notation}

It will sometimes be convenient to denote the process a name
quotes. We already have the notation $x = \quotep{P}$, but it will be
convenient to introduce an alternate notation, $\procn{x}$, when we
want to emphasize the connection to the use of the name. Note that, by
virtue of name equivalence, $\quotep{\procn{x}} \nameeq x$; so, the
notation is consistent with previous definitions.

Further, because names have structure it is possible to effect
substitutions on the basis of that structure. This means we need to
upgrade our notation for substitutions, which we accomplish by
adapting comprehension notation. Thus,

\begin{mathpar}
  P\{ y / x : x \in S \}
\end{mathpar}

is interpreted to mean the process derived from P by replacing (in a
capture-avoiding manner) each occurrence of $x$ in $S$ by $y$. For example,

\begin{mathpar}
  P\{ \quotep{\procn{x}|\procn{x}} / x : x \in \freenames{P} \}
\end{mathpar}

will replace each (occurrence) of a free name $x$ in $P$ by
$\quotep{\procn{x}|\procn{x}}$.

Also, we will avail ourselves of the notation $x^{L}$ and $x^{R}$ to
denote injections of a name into disjoint copies of the name
space. There are numerous ways to accomplish this. One example can be
found in \cite{MeredithR05}. This notation overloads to vectors of
names: $\vec{x}^{\pi} := (x_{i}^{\pi} \; : \; 0 \leq i < |\vec{x}| )$ where $\pi \in \{L,R\}$.

We also use $P^{\Box} := P|\Box$.

In \cite{MeredithR05} an interpretation of the new operator is
given. It turns out that there are several possible interpretations
all enjoying the requisite algebraic properties of the operator (see
\cite{milner91polyadicpi}). We will therefore make liberal use of
$(\nu\; \vec{x})P$.

% subsection the_syntax_and_semantics_of_the_notation_system (end)   

\input{qm2pi.qmops} 

\input{qm2pi.sterngerlach} 

\input{qm2pi.metric} 

% section concurrent_process_calculi (end)

%\input{qm2pi.proofsketch}

% section proof sketch (end)

%\input{qm2pi.slviaknots} 

% section spatial logic via knots (end)

\input{qm2pi.conclusion}

% section conclusion (end)

%\input{qm2pi.dtcodes} 

% section wiring algorithm (end)

\input{qm2pi.ack} 

% section acknowledgments (end)

\newpage


\bibliographystyle{plain}   
\bibliography{../../biblios/main.bib}

\input{qm2pi.rhodetails}

\end{document}

 

\documentclass[12pt]{llncs}
%\documentclass{jktr}

\usepackage[pdftex]{hyperref}                   
\usepackage {listings}
\usepackage {mathpartir}
\usepackage{bcprules}
%\usepackage{listings}
                       
\usepackage{graphicx} 
%\usepackage[margins=2.5cm,nohead,nofoot]{geometry}
%\usepackage{geometry}
\usepackage{amsfonts}
\usepackage{amstext}
\usepackage{latexsym}
\usepackage{amssymb}
\usepackage{color}


%\include{myPreamble}
\include{qm2pi.local} 

%\ifpdf
%\usepackage[pdftex]{graphicx}
%\else
%\usepackage{graphicx}
%\fi

 % \ifpdf
%  \usepackage{pdfsync}
%  \if


%\title{Brief Article}
%\author{David F. Snyder}
%\author{L.G. Meredith}

%\address{Dept. of Math., Texas State University--San Marcos, San Marcos, TX 78666}
       
\pagestyle{empty}


\begin{document}

\lstset{language=[Objective]Caml,frame=shadowbox}

\input{qm2pi.front}

% section front matter (end)

\input{qm2pi.intro} 
 
% section introduction (end)

% \input{qm2pi.knotations} 

% section notation (end)

\input{qm2pi.process.calculi} 

% section concurrent_process_calculi_and_spatial_logics_ (end)
    
%\input{qm2pi.knots2pi} 

%\input{qm2pi.trefoil} 

%\input{qm2pi.mainthm} 

% subsection basic_interpretation (end)

%\input{qm2pi.rho.presentation} 
\subsection{The syntax and semantics of the notation system}\label{sub:the_syntax_and_semantics_of_the_notation_system} % (fold)

We now summarize a technical presentation of the calculus that
embodies our theory of dynamics. The typical presentation of such a
calculus follows the style of giving generators and relations on
them. The grammar, below, describing term constructors, freely
generates the set of processes, $\Proc$. This set is then quotiented
by a relation known as structural congruence and it is over this set
that the notion of dynamics is expressed. This presentation is
essentially that of \cite{MeredithR05} with the addition of
polyadicity and summation. For readability we have relegated some of
the technical subtleties to an appendix.

\subsubsection{Process grammar}\label{subsub:process_grammar}

\begin{mathpar}
  \inferrule* [lab=synchronization] {} {{M} \bc \pzero \;|\; x?F \;|\; x!C }
  \and
  \inferrule* [lab=abstraction] {} {{F} \bc (x)P}
  \and
  \inferrule* [lab=concretion] {} {{C} \bc \langle Q \rangle}
  \and
  \inferrule* [lab=process] {} {{P,Q} \bc M \;| \;P|Q \;|\; @{x}}
  \and
  \inferrule* [lab=name] {} {{x} \bc \quotep{P}}
\end{mathpar} 

Note that $\vec{x}$ (resp. $\vec{P}$) denotes a vector of names
(resp. processes) of length $|\vec{x}|$ (resp. $|\vec{P}|$). We adopt
the following useful abbreviations.

\begin{mathpar}
   x?(\vec{y}).P := x.(\vec{y})P \and  x\clift{\vec{P}} := x.\clift{\vec{P}}
   \and x!(y) := \lift{x}{\dropn{y}}
   \and \Pi_{i=0}^{n-1}P_i := P_0 | \ldots | P_{n-1}
\end{mathpar}

\subsubsection{Structural congruence}

\paragraph{Free and bound names and alpha-equivalence.} At the
core of structural equivalence is alpha-equivalence which identifies
process that are the same up to a change of variable. Formally, we
recognize the distinction between free and bound names. The free names
of a process, $\freenames{P}$, may be calculated recursively as
follows:

\begin{mathpar}
\freenames{\pzero} := \emptyset
  \and \\
  \freenames{x?(y).P} := \{ x \} \cup (\freenames{P} \setminus \{ y \})
  \and 
  \freenames{x!\langle P \rangle} := \{ x \} \cup \{ P \} 
  \and \\
  \freenames{P|Q} := \freenames{P} \cup \freenames{Q}
  \and \\
  \freenames{@{x}} := \{ x \}
\end{mathpar}

$\pi$
$\quotep{\pi}$

$\freenames{-} : \pi \to \mathcal{P}(\quotep{\pi})$

\begin{eqnarray*}
  \freenames{\pzero} & := & \emptyset \\
  \freenames{x?(y).P} & := & \{ x \} \cup (\freenames{P} \setminus \{ y \}) \\
  \freenames{x!\langle P \rangle} & := & \{ x \} \cup \{ P \} \\
  \freenames{P|Q} & := & \freenames{P} \cup \freenames{Q} \\
  \freenames{\dropn{x}} & := & \{ x \}
\end{eqnarray*}

The bound names of a process, $\boundnames{P}$, are those names occurring in $P$
that are not free. For example, in $x?(y).0$, the name $x$ is free, while $y$ is bound.

\begin{mathpar}
  \inferrule* [lab=monoidal-laws] {} { P|Q \equiv Q|P \and P|0 \equiv P \and P|(Q|R) \equiv (P|Q)|R }
\end{mathpar}

\begin{mathpar}
  \inferrule* [lab=alpha-equivalence] {} { (x)P \equiv (y)P\{y/x\} \and y \not\in \freenames{P} }
\end{mathpar}

\begin{definition}
Then two processes, $P,Q$, are alpha-equivalent if $P = Q\{\vec{y}/\vec{x}\}$ for
some $\vec{x} \in \boundnames{Q},\vec{y} \in \boundnames{P}$, where $Q\{\vec{y}/\vec{x}\}$
denotes the capture-avoiding substitution of $\vec{y}$ for $\vec{x}$ in $Q$.
\end{definition}

\begin{definition}
  The {\em structural congruence} \cite{SangiorgiWalker} , $\equiv$,
  between processes is the least congruence containing
  alpha-equivalence, satisfying the abelian monoid laws
  (associativity, commutativity and $\pzero$ as identity) for parallel
  composition $|$ and for summation $+$.
\end{definition}

\subsection{Name equivalence}

We take name equivalence, written $\nameeq$, to be the smallest
equivalence relation generated by the following rules.

\begin{mathpar}
\inferrule*[lab=Quote-drop]
{ }
{ \quotep{@{x}} \nameeq x }

\inferrule*[lab=Struct-equiv]
{ P \scong Q }
{ \quotep{P} \nameeq \quotep{Q} }
\end{mathpar}

The astute reader will have noticed that the mutual recursion of names
and processes imposes a mutual recursion on alpha-equivalence and
structural equivalence via name-equivalence. Fortunately, all of this
works out pleasantly and we may calculate in the natural way, free of
concern. The reader interested in the details is referred to the
appendix \ref{appendix:rho_details}.

\subsection{Substitution}

We use $\Proc$ for the set of processes, $\QProc$ for the set of
names, and $\id{\{}\vec{y} / \vec{x} \id{\}}$ to denote partial maps,
$s : \QProc \rightarrow \QProc$. A map, $s$ lifts, uniquely, to a map
on process terms, $\widehat{s} : \Proc \rightarrow \Proc$ by the
following equations.

\begin{mathpar}
  (0) \psubstp{Q}{P} := 0 \\
  (R \juxtap S) \psubstp{Q}{P}
  :=    
  (R)\psubstp{Q}{P} \juxtap (S) \psubstp{Q}{P} \\
  (x?(y).R) \psubstp{Q}{P}    
  :=    
  (x)\substp{Q}{P} (z)\concat( (R \psubstn{z}{y}) \psubstp{Q}{P} ) \\
  (\lift{x}{R}) \psubstp{Q}{P}  
  :=
  \lift{(x)\substp{Q}{P}}{ R \psubstp{Q}{P} } \\
%   (\dropn{x})  \psubstp{Q}{P}       
%   := 
%   \left\{ 
%     \begin{array}{ccc} 
%       \dropn{\quotep{Q}} & & x \nameeq \quotep{P} \\
%       \dropn{x} & & otherwise \\
%     \end{array}
%   \right. 
  (\dropn{x})  \psubstp{Q}{P}       
  := 
  \left\{ 
    \begin{array}{ccc} 
      Q & & x \nameeq \quotep{P} \\
      \dropn{x} & & otherwise \\
    \end{array}
  \right.
\end{mathpar}
 

where

\begin{eqnarray}
  (x)\id{\{} \lpquote Q \rpquote / \lpquote P \rpquote \id{\}}            = 
  \left\{ 
    \begin{array}{ccc}
      \lpquote Q \rpquote & & x \nameeq \lpquote P \rpquote \\
      x & & otherwise \\
    \end{array}
  \right. \nonumber
\end{eqnarray}

and $z$ is chosen distinct from $\quotep{P}$, $\quotep{Q}$, the free
names in $Q$, and all the names in $R$. Our $\alpha$-equivalence will
be built in the standard way from this substitution.

\begin{remark}\label{rem:no_self_referential_names}
  One consequence of these definitions is that $\forall P. \quotep{P}
  \not\in \freenames{P}$.
\end{remark}

\subsection{ Dynamic quote: an example }

Anticipating something of what's to come, consider applying the
substitution, $\widehat{\id{\{}u / z \id{\}}}$, to the following pair
of processes, $\lift{w}{y!(z)}$ and $w[ \lpquote y!(z) \rpquote ]$.

\begin{eqnarray}
	\lift{w}{y!(z)}\widehat{\id{\{}u / z \id{\}}}
		& = &
		\lift{w}{y!(u)} \nonumber\\
	w[ \lpquote y!(z) \rpquote ] \widehat{ \id{\{}u / z \id{\}} }
		& = &
		w[ \lpquote y!(z) \rpquote ] \nonumber
\end{eqnarray}

Because the body of the process between quotes is impervious to
substitution, we get radically different answers. In fact, by
examining the first process in an input context,
e.g. $x?(z).\lift{w}{y!(z)}$, we see that the process under the lift
operator may be shaped by prefixed inputs binding a name inside it. In
this sense, the lift operator will be seen as a way to dynamically
construct processes before reifying them as names.

Finally equipped with these standard features we can present the
dynamics of the calculus.

\subsubsection{Operational semantics} 

Finally, we introduce the computational dynamics. What marks these
algebras as distinct from other more traditionally studied algebraic
structures, e.g. vector spaces or polynomial rings, is the manner in
which dynamics is captured. In traditional structures, dynamics is typically
expressed through morphisms between such structures, as in linear maps
between vector spaces or morphisms between rings. In algebras
associated with the semantics of computation, the dynamics is
expressed as part of the algebraic structure itself, through a
reduction reduction relation typically denoted by $\red$. Below, we
give a recursive presentation of this relation for the calculus used
in the encoding.

$\red \subseteq \pi \times \pi$
$\red : \pi \to \mathcal{P}(\pi)$

\begin{mathpar}
  \inferrule* [lab=Comm] { \textsf{match}( x_{src}, x_{trgt} ) } { x_{trgt}?(y)P \; | \; x_{src}!\langle {Q} \rangle \red P\{\quotep{Q}/y}\} }
  \and \\
  \inferrule* [lab=Par] {{P} \red {P}'} {{{P} | {Q}} \red {{P}' | {Q}}}
  \and
  \inferrule* [lab=Equiv]{{{P} \scong {P}'} \andalso {{P}' \red {Q}'} \andalso {{Q}' \scong {Q}}}{{P} \red {Q}}
\end{mathpar}

\begin{eqnarray*}
  match_{\equiv} (\quotep{P},\quotep{Q}) & := & P \equiv Q \\
  match_{\dagger}(\quotep{P},\quotep{Q}) & := & \forall R. P|Q \red^{*} R => R \red^{*} 0 \\
  match_{K}(\quotep{P},\quotep{Q}) & := & K \mbox{ for some context } K
\end{eqnarray*}

$u?(x)P | u!\langle Q \rangle \red P\{\quotep{Q}/x\}$

%We write $\wred$ for $\red^*$, and $P\red$ if $\exists Q $ such that $ P \red Q$.
We write $P\red$ if $\exists Q $ such that $ P \red Q$ and $P\not\red$, otherwise.

\section{Replication}

As mentioned before, it is known that replication (and hence
recursion) can be implemented in a higher-order process algebra
\cite{SangiorgiWalker}. As our first example of calculation with the
machinery thus far presented we give the construction explicitly in
the {\rhoc}.

\begin{eqnarray}
	D_{x} & := & \prefix{x}{y}{(\binpar{\outputp{x}{y}}{@{y}})} \nonumber\\
	\bangp_{x}{P} & := & \binpar{{x}!\langle{\binpar{D_{x}}{P}}\rangle}{D_{x}} \nonumber
\end{eqnarray}

\begin{eqnarray}
	\bangp_{x}{P} & & \nonumber\\
	=
	& {x}!\langle{(\prefix{x}{y}{(\outputp{x}{y} | @{y})) | P}}\rangle 
	      | \prefix{x}{y}{(\outputp{x}{y} | @{y})} & \nonumber\\
	\red
	& (\outputp{x}{y} | @{y})\substn{\quotep{(\prefix{x}{y}{(@{y} | \outputp{x}{y})) | P}}}{y} & \nonumber\\
	=
	& \outputp{x}{\quotep{(\prefix{x}{y}{(\outputp{x}{y} | @{y})) | P}}}
	  | {(\prefix{x}{y}{(\outputp{x}{y} | @{y})) | P}} & \nonumber\\
	\red
	& \ldots & \nonumber\\
	\red^*
	& P | P | \ldots & \nonumber
\end{eqnarray}

Of course, this encoding, as an implementation, runs away, unfolding
$\bangp{P}$ eagerly. A lazier and more implementable replication
operator, restricted to input-guarded processes, may be obtained as follows.

\begin{eqnarray}
\bangp{\prefix{u}{v}{P}} 
	:= 
	\binpar{\lift{x}{\prefix{u}{v}{(\binpar{D(x)}{P})}}}{D(x)} \nonumber
\end{eqnarray}

\begin{remark}
  Note that the lazier definition still does not deal with summation
  or mixed summation (i.e. sums over input and output). The reader is
  invited to construct definitions of replication that deal with these
  features. 

  Further, the definitions are parameterized in a name, $x$. Can you,
  gentle reader, make a definition that eliminates this parameter and
  guarantees no accidental interaction between the replication
  machinery and the process being replicated -- i.e. no accidental
  sharing of names used by the process to get its work done and the
  name(s) used by the replication to effect copying. This latter
  revision of the definition of replication is crucial to obtaining
  the expected identity $!!P \sim !P$.
\end{remark}

\begin{remark}\label{rem:paradoxical_combinator}
  The reader familiar with the lambda calculus will have noticed the
  similarity between $D$ and the paradoxical combinator.

  [Ed. note: the existence of this seems to suggest we have to be more
  restrictive on the set of processes and names we admit if we are to
  support no-cloning.]
\end{remark}

\subsubsection{Bisimulation}

The computational dynamics gives rise to another kind of equivalence,
the equivalence of computational behavior. As previously mentioned
this is typically captured \emph{via} some form of bisimulation.

% The notion we use in this paper is weak barbed bisimulation
% \cite{milner91polyadicpi}.

The notion we use in this paper is derived from weak barbed
bisimulation \cite{milner91polyadicpi}. 

\begin{definition}
An \emph{observation relation}, $\downarrow_{\mathcal N}$, over a set
of names, $\mathcal N$, is the smallest relation satisfying the rules
below.

\infrule[Out-barb]{y \in {\mathcal N}, \; x \nameeq y}
		  {\outputp{x}{v} \downarrow_{\mathcal N} x}
\infrule[Par-barb]{\mbox{$P\downarrow_{\mathcal N} x$ or $Q\downarrow_{\mathcal N} x$}}
		  {\binpar{P}{Q} \downarrow_{\mathcal N} x}

We write $P \Downarrow_{\mathcal N} x$ if there is $Q$ such that 
$P \wred Q$ and $Q \downarrow_{\mathcal N} x$.
\end{definition}

\begin{definition}
%\label{def.bbisim}
An  ${\mathcal N}$-\emph{barbed bisimulation} over a set of names, ${\mathcal N}$, is a symmetric binary relation 
${\mathcal S}_{\mathcal N}$ between agents such that $P\rel{S}_{\mathcal N}Q$ implies:
\begin{enumerate}
\item If $P \red P'$ then $Q \wred Q'$ and $P'\rel{S}_{\mathcal N} Q'$.
\item If $P\downarrow_{\mathcal N} x$, then $Q\Downarrow_{\mathcal N} x$.
\end{enumerate}
$P$ is ${\mathcal N}$-barbed bisimilar to $Q$, written
$P \wbbisim_{\mathcal N} Q$, if $P \rel{S}_{\mathcal N} Q$ for some ${\mathcal N}$-barbed bisimulation ${\mathcal S}_{\mathcal N}$.
\end{definition}

$\mathcal{R} \subseteq \pi \times \pi$

$P \mathcal{R} Q => \forall P'. P \red P' \Rightarrow \exists Q'. Q \red Q', P' \mathcal{R} Q'$

$P \vdash x \Rightarrow Q \vdash x$

\begin{mathpar}
  \inferrule*[lab=Out-barb]{x \nameeq y}{{y}!\langle{Q}\rangle \vdash x}
  \and
  \inferrule*[lab=Par-barb]{\mbox{$P\vdash x$ or $Q\vdash x$}}{\binpar{P}{Q} \vdash x}
\end{mathpar}

\subsubsection{Contexts}

One of the principle advantages of computational calculi like the
$\pi$-calculus is a well-defined notion of context,
contextual-equivalence and a correlation between
contextual-equivalence and notions of bisimulation. The notion of
context allows the decomposition of a process into (sub-)process and
its syntactic environment, its context. Thus, a context may be
thought of as a process with a ``hole'' (written $\Box$) in it. The
application of a context $M$ to a process $P$, written $M[P]$, is
tantamount to filling the hole in $M$ with $P$. In this paper we do
not need the full weight of this theory, but do make use of the notion
of context in the proof the main theorem. 

\begin{mathpar}
  \inferrule* [lab=summation] {} {{M_{M},M_{N}} \bc \Box \;|\; x.M_{A} \;|\; M_{M}+M_{N}}
  \and
  \inferrule* [lab=agent] {} {{M_{A}} \bc (\vec{x})M_{P} \;| \; \clift{P_0,\ldots,M_{P},\ldots,P_N}}
  \and \\
  \inferrule* [lab=process] {} {{M_{P}} \bc M_{N} \;| \;P|M_{P} }
\end{mathpar} 

\begin{mathpar}
  \inferrule* [lab=sychronization] {} {M_{N} \bc \Box \;|\; x?M_{F} \;|\; x!M_{C}}
  \and
  \inferrule* [lab=abstraction] {} {{M_{F}} \bc (x)M_{P} }
  \and
  \inferrule* [lab=concretion] {} {{M_{C}} \bc \langle M_{P} \rangle }
  \and \\
  \inferrule* [lab=process] {} {{M_{P}} \bc M_{N} \;| \;P|M_{P} }
\end{mathpar}

\begin{definition}[contextual application] Given a context $M$, and
  process $P$, we define the \emph{contextual application}, $M[P] :=
  M\{P/\Box\}$. That is, the contextual application of M to P is the
  substitution of $P$ for $\Box$ in $M$.
\end{definition}

$\meaningof{-} : L \to \mathcal{P}(\pi)$

\begin{mathpar}
  \inferrule* [lab=collection] {} {\meaningof{true} = \pi, \and \meaningof{~E} = \pi \setminus \meaningof{E}, \and \meaningof{E_{1} \& E_{2}} = \meaningof{E_{1}} \cap \meaningof{E_{2}}}
\end{mathpar}

\begin{mathpar}
  \inferrule* [lab=structure] {} {\meaningof{0} = \{ P \in \pi | P \equiv 0 \}, \and \\ \meaningof{E_1 | E_2} = \{ P \in \pi | P \equiv P_{1} | P_{2}, P_{1} \in \meaningof{E_{1}}, P_{2} \in \meaningof{E_2}\} }
\end{mathpar}

\begin{mathpar}
 \inferrule* [lab=behavior] {} {\meaningof{\langle a?b \rangle E} = \{ P \in \pi | P \equiv Q | u?(y)P', \\ \and \\\\ \and \\ \;\;\; u \in \meaningof{a}, \forall z.P'\{z/y\} \in \meaningof{E\{z/b\}}\}, \and \\ \meaningof{a!E} = \{ P \in \pi | P \equiv Q | x!\langle P' \rangle, x \in \meaningof{a} P' \in \meaningof{E}\} }
\end{mathpar}

\begin{mathpar}
 \inferrule* [lab=nominal] {} {\meaningof{\quotep{E}} = \{ \quotep{P} \in \quotep{\pi} | P \in \meaningof{E} \}, \and \meaningof{\quotep{P}} = \{ \quotep{Q} \in \quotep{\pi} | P \equiv Q \} \and \\ \meaningof{@\quotep{E}} = \{ P \in \pi | P \equiv @x, x \in \meaningof{E} \}}
\end{mathpar}

\begin{eqnarray*}
  \\
  \meaningof{-} : TS \to ST
\end{eqnarray*}

\begin{eqnarray*}
  \\
  L : TS \to ST
\end{eqnarray*}

\begin{eqnarray*}
  \\
  P \models E \iff P \in \meaningof{E}
\end{eqnarray*}

\begin{eqnarray*}
  P \approx_{L} Q \iff \forall E \in L. P \models E \iff Q \models E
\end{eqnarray*}

\begin{eqnarray*}
  P \approx_{K} Q
\end{eqnarray*}

\begin{eqnarray*}
  P \approx Q
\end{eqnarray*}

$\approx_{K} = \approx = \approx_{L}$

\subsubsection{Contextual duality}

Note that contexts extend the quotation operation to a family of
operations from processes to names. Given a context, $M$, we can
define a \emph{nominal context}, $\quotep{M}$ by $\quotep{M}[P] :=
\quotep{M[P]}$. To foreshadow what is to come we observe that these
operations enjoy a duality with processes very much like the duality
between vectors and maps from vectors to scalars.

Further, because the calculus is essentially higher-order, we have a
correspondence between contexts and processes. More specifically,
given a name $x$ and a context $M$ we can construct $M^{*}_{x}$ such
that 

\begin{mathpar}
  M^{*}_{x} | \lift{x}{P} \red M[P]
\end{mathpar}

namely,

\begin{mathpar}
  M^{*}_{x} := x?(u).M[\dropn{u}]
\end{mathpar}

The dependence of $M^{*}_{x}$ on a name makes it an abstraction, 

\begin{mathpar}
  M^{*} := (x)x?(u).M[\dropn{u}]
\end{mathpar}

\subsection{Additional notation}

It will sometimes be convenient to denote the process a name
quotes. We already have the notation $x = \quotep{P}$, but it will be
convenient to introduce an alternate notation, $\procn{x}$, when we
want to emphasize the connection to the use of the name. Note that, by
virtue of name equivalence, $\quotep{\procn{x}} \nameeq x$; so, the
notation is consistent with previous definitions.

Further, because names have structure it is possible to effect
substitutions on the basis of that structure. This means we need to
upgrade our notation for substitutions, which we accomplish by
adapting comprehension notation. Thus,

\begin{mathpar}
  P\{ y / x : x \in S \}
\end{mathpar}

is interpreted to mean the process derived from P by replacing (in a
capture-avoiding manner) each occurrence of $x$ in $S$ by $y$. For example,

\begin{mathpar}
  P\{ \quotep{\procn{x}|\procn{x}} / x : x \in \freenames{P} \}
\end{mathpar}

will replace each (occurrence) of a free name $x$ in $P$ by
$\quotep{\procn{x}|\procn{x}}$.

Also, we will avail ourselves of the notation $x^{L}$ and $x^{R}$ to
denote injections of a name into disjoint copies of the name
space. There are numerous ways to accomplish this. One example can be
found in \cite{MeredithR05}. This notation overloads to vectors of
names: $\vec{x}^{\pi} := (x_{i}^{\pi} \; : \; 0 \leq i < |\vec{x}| )$ where $\pi \in \{L,R\}$.

We also use $P^{\Box} := P|\Box$.

In \cite{MeredithR05} an interpretation of the new operator is
given. It turns out that there are several possible interpretations
all enjoying the requisite algebraic properties of the operator (see
\cite{milner91polyadicpi}). We will therefore make liberal use of
$(\nu\; \vec{x})P$.

% subsection the_syntax_and_semantics_of_the_notation_system (end)   

\input{qm2pi.qmops} 

\input{qm2pi.sterngerlach} 

\input{qm2pi.metric} 

% section concurrent_process_calculi (end)

%\input{qm2pi.proofsketch}

% section proof sketch (end)

%\input{qm2pi.slviaknots} 

% section spatial logic via knots (end)

\input{qm2pi.conclusion}

% section conclusion (end)

%\input{qm2pi.dtcodes} 

% section wiring algorithm (end)

\input{qm2pi.ack} 

% section acknowledgments (end)

\newpage


\bibliographystyle{plain}   
\bibliography{../../biblios/main.bib}

\input{qm2pi.rhodetails}

\end{document}

 

% section concurrent_process_calculi (end)

%\documentclass[12pt]{llncs}
%\documentclass{jktr}

\usepackage[pdftex]{hyperref}                   
\usepackage {listings}
\usepackage {mathpartir}
\usepackage{bcprules}
%\usepackage{listings}
                       
\usepackage{graphicx} 
%\usepackage[margins=2.5cm,nohead,nofoot]{geometry}
%\usepackage{geometry}
\usepackage{amsfonts}
\usepackage{amstext}
\usepackage{latexsym}
\usepackage{amssymb}
\usepackage{color}


%\include{myPreamble}
\include{qm2pi.local} 

%\ifpdf
%\usepackage[pdftex]{graphicx}
%\else
%\usepackage{graphicx}
%\fi

 % \ifpdf
%  \usepackage{pdfsync}
%  \if


%\title{Brief Article}
%\author{David F. Snyder}
%\author{L.G. Meredith}

%\address{Dept. of Math., Texas State University--San Marcos, San Marcos, TX 78666}
       
\pagestyle{empty}


\begin{document}

\lstset{language=[Objective]Caml,frame=shadowbox}

\input{qm2pi.front}

% section front matter (end)

\input{qm2pi.intro} 
 
% section introduction (end)

% \input{qm2pi.knotations} 

% section notation (end)

\input{qm2pi.process.calculi} 

% section concurrent_process_calculi_and_spatial_logics_ (end)
    
%\input{qm2pi.knots2pi} 

%\input{qm2pi.trefoil} 

%\input{qm2pi.mainthm} 

% subsection basic_interpretation (end)

%\input{qm2pi.rho.presentation} 
\subsection{The syntax and semantics of the notation system}\label{sub:the_syntax_and_semantics_of_the_notation_system} % (fold)

We now summarize a technical presentation of the calculus that
embodies our theory of dynamics. The typical presentation of such a
calculus follows the style of giving generators and relations on
them. The grammar, below, describing term constructors, freely
generates the set of processes, $\Proc$. This set is then quotiented
by a relation known as structural congruence and it is over this set
that the notion of dynamics is expressed. This presentation is
essentially that of \cite{MeredithR05} with the addition of
polyadicity and summation. For readability we have relegated some of
the technical subtleties to an appendix.

\subsubsection{Process grammar}\label{subsub:process_grammar}

\begin{mathpar}
  \inferrule* [lab=synchronization] {} {{M} \bc \pzero \;|\; x?F \;|\; x!C }
  \and
  \inferrule* [lab=abstraction] {} {{F} \bc (x)P}
  \and
  \inferrule* [lab=concretion] {} {{C} \bc \langle Q \rangle}
  \and
  \inferrule* [lab=process] {} {{P,Q} \bc M \;| \;P|Q \;|\; @{x}}
  \and
  \inferrule* [lab=name] {} {{x} \bc \quotep{P}}
\end{mathpar} 

Note that $\vec{x}$ (resp. $\vec{P}$) denotes a vector of names
(resp. processes) of length $|\vec{x}|$ (resp. $|\vec{P}|$). We adopt
the following useful abbreviations.

\begin{mathpar}
   x?(\vec{y}).P := x.(\vec{y})P \and  x\clift{\vec{P}} := x.\clift{\vec{P}}
   \and x!(y) := \lift{x}{\dropn{y}}
   \and \Pi_{i=0}^{n-1}P_i := P_0 | \ldots | P_{n-1}
\end{mathpar}

\subsubsection{Structural congruence}

\paragraph{Free and bound names and alpha-equivalence.} At the
core of structural equivalence is alpha-equivalence which identifies
process that are the same up to a change of variable. Formally, we
recognize the distinction between free and bound names. The free names
of a process, $\freenames{P}$, may be calculated recursively as
follows:

\begin{mathpar}
\freenames{\pzero} := \emptyset
  \and \\
  \freenames{x?(y).P} := \{ x \} \cup (\freenames{P} \setminus \{ y \})
  \and 
  \freenames{x!\langle P \rangle} := \{ x \} \cup \{ P \} 
  \and \\
  \freenames{P|Q} := \freenames{P} \cup \freenames{Q}
  \and \\
  \freenames{@{x}} := \{ x \}
\end{mathpar}

$\pi$
$\quotep{\pi}$

$\freenames{-} : \pi \to \mathcal{P}(\quotep{\pi})$

\begin{eqnarray*}
  \freenames{\pzero} & := & \emptyset \\
  \freenames{x?(y).P} & := & \{ x \} \cup (\freenames{P} \setminus \{ y \}) \\
  \freenames{x!\langle P \rangle} & := & \{ x \} \cup \{ P \} \\
  \freenames{P|Q} & := & \freenames{P} \cup \freenames{Q} \\
  \freenames{\dropn{x}} & := & \{ x \}
\end{eqnarray*}

The bound names of a process, $\boundnames{P}$, are those names occurring in $P$
that are not free. For example, in $x?(y).0$, the name $x$ is free, while $y$ is bound.

\begin{mathpar}
  \inferrule* [lab=monoidal-laws] {} { P|Q \equiv Q|P \and P|0 \equiv P \and P|(Q|R) \equiv (P|Q)|R }
\end{mathpar}

\begin{mathpar}
  \inferrule* [lab=alpha-equivalence] {} { (x)P \equiv (y)P\{y/x\} \and y \not\in \freenames{P} }
\end{mathpar}

\begin{definition}
Then two processes, $P,Q$, are alpha-equivalent if $P = Q\{\vec{y}/\vec{x}\}$ for
some $\vec{x} \in \boundnames{Q},\vec{y} \in \boundnames{P}$, where $Q\{\vec{y}/\vec{x}\}$
denotes the capture-avoiding substitution of $\vec{y}$ for $\vec{x}$ in $Q$.
\end{definition}

\begin{definition}
  The {\em structural congruence} \cite{SangiorgiWalker} , $\equiv$,
  between processes is the least congruence containing
  alpha-equivalence, satisfying the abelian monoid laws
  (associativity, commutativity and $\pzero$ as identity) for parallel
  composition $|$ and for summation $+$.
\end{definition}

\subsection{Name equivalence}

We take name equivalence, written $\nameeq$, to be the smallest
equivalence relation generated by the following rules.

\begin{mathpar}
\inferrule*[lab=Quote-drop]
{ }
{ \quotep{@{x}} \nameeq x }

\inferrule*[lab=Struct-equiv]
{ P \scong Q }
{ \quotep{P} \nameeq \quotep{Q} }
\end{mathpar}

The astute reader will have noticed that the mutual recursion of names
and processes imposes a mutual recursion on alpha-equivalence and
structural equivalence via name-equivalence. Fortunately, all of this
works out pleasantly and we may calculate in the natural way, free of
concern. The reader interested in the details is referred to the
appendix \ref{appendix:rho_details}.

\subsection{Substitution}

We use $\Proc$ for the set of processes, $\QProc$ for the set of
names, and $\id{\{}\vec{y} / \vec{x} \id{\}}$ to denote partial maps,
$s : \QProc \rightarrow \QProc$. A map, $s$ lifts, uniquely, to a map
on process terms, $\widehat{s} : \Proc \rightarrow \Proc$ by the
following equations.

\begin{mathpar}
  (0) \psubstp{Q}{P} := 0 \\
  (R \juxtap S) \psubstp{Q}{P}
  :=    
  (R)\psubstp{Q}{P} \juxtap (S) \psubstp{Q}{P} \\
  (x?(y).R) \psubstp{Q}{P}    
  :=    
  (x)\substp{Q}{P} (z)\concat( (R \psubstn{z}{y}) \psubstp{Q}{P} ) \\
  (\lift{x}{R}) \psubstp{Q}{P}  
  :=
  \lift{(x)\substp{Q}{P}}{ R \psubstp{Q}{P} } \\
%   (\dropn{x})  \psubstp{Q}{P}       
%   := 
%   \left\{ 
%     \begin{array}{ccc} 
%       \dropn{\quotep{Q}} & & x \nameeq \quotep{P} \\
%       \dropn{x} & & otherwise \\
%     \end{array}
%   \right. 
  (\dropn{x})  \psubstp{Q}{P}       
  := 
  \left\{ 
    \begin{array}{ccc} 
      Q & & x \nameeq \quotep{P} \\
      \dropn{x} & & otherwise \\
    \end{array}
  \right.
\end{mathpar}
 

where

\begin{eqnarray}
  (x)\id{\{} \lpquote Q \rpquote / \lpquote P \rpquote \id{\}}            = 
  \left\{ 
    \begin{array}{ccc}
      \lpquote Q \rpquote & & x \nameeq \lpquote P \rpquote \\
      x & & otherwise \\
    \end{array}
  \right. \nonumber
\end{eqnarray}

and $z$ is chosen distinct from $\quotep{P}$, $\quotep{Q}$, the free
names in $Q$, and all the names in $R$. Our $\alpha$-equivalence will
be built in the standard way from this substitution.

\begin{remark}\label{rem:no_self_referential_names}
  One consequence of these definitions is that $\forall P. \quotep{P}
  \not\in \freenames{P}$.
\end{remark}

\subsection{ Dynamic quote: an example }

Anticipating something of what's to come, consider applying the
substitution, $\widehat{\id{\{}u / z \id{\}}}$, to the following pair
of processes, $\lift{w}{y!(z)}$ and $w[ \lpquote y!(z) \rpquote ]$.

\begin{eqnarray}
	\lift{w}{y!(z)}\widehat{\id{\{}u / z \id{\}}}
		& = &
		\lift{w}{y!(u)} \nonumber\\
	w[ \lpquote y!(z) \rpquote ] \widehat{ \id{\{}u / z \id{\}} }
		& = &
		w[ \lpquote y!(z) \rpquote ] \nonumber
\end{eqnarray}

Because the body of the process between quotes is impervious to
substitution, we get radically different answers. In fact, by
examining the first process in an input context,
e.g. $x?(z).\lift{w}{y!(z)}$, we see that the process under the lift
operator may be shaped by prefixed inputs binding a name inside it. In
this sense, the lift operator will be seen as a way to dynamically
construct processes before reifying them as names.

Finally equipped with these standard features we can present the
dynamics of the calculus.

\subsubsection{Operational semantics} 

Finally, we introduce the computational dynamics. What marks these
algebras as distinct from other more traditionally studied algebraic
structures, e.g. vector spaces or polynomial rings, is the manner in
which dynamics is captured. In traditional structures, dynamics is typically
expressed through morphisms between such structures, as in linear maps
between vector spaces or morphisms between rings. In algebras
associated with the semantics of computation, the dynamics is
expressed as part of the algebraic structure itself, through a
reduction reduction relation typically denoted by $\red$. Below, we
give a recursive presentation of this relation for the calculus used
in the encoding.

$\red \subseteq \pi \times \pi$
$\red : \pi \to \mathcal{P}(\pi)$

\begin{mathpar}
  \inferrule* [lab=Comm] { \textsf{match}( x_{src}, x_{trgt} ) } { x_{trgt}?(y)P \; | \; x_{src}!\langle {Q} \rangle \red P\{\quotep{Q}/y}\} }
  \and \\
  \inferrule* [lab=Par] {{P} \red {P}'} {{{P} | {Q}} \red {{P}' | {Q}}}
  \and
  \inferrule* [lab=Equiv]{{{P} \scong {P}'} \andalso {{P}' \red {Q}'} \andalso {{Q}' \scong {Q}}}{{P} \red {Q}}
\end{mathpar}

\begin{eqnarray*}
  match_{\equiv} (\quotep{P},\quotep{Q}) & := & P \equiv Q \\
  match_{\dagger}(\quotep{P},\quotep{Q}) & := & \forall R. P|Q \red^{*} R => R \red^{*} 0 \\
  match_{K}(\quotep{P},\quotep{Q}) & := & K \mbox{ for some context } K
\end{eqnarray*}

$u?(x)P | u!\langle Q \rangle \red P\{\quotep{Q}/x\}$

%We write $\wred$ for $\red^*$, and $P\red$ if $\exists Q $ such that $ P \red Q$.
We write $P\red$ if $\exists Q $ such that $ P \red Q$ and $P\not\red$, otherwise.

\section{Replication}

As mentioned before, it is known that replication (and hence
recursion) can be implemented in a higher-order process algebra
\cite{SangiorgiWalker}. As our first example of calculation with the
machinery thus far presented we give the construction explicitly in
the {\rhoc}.

\begin{eqnarray}
	D_{x} & := & \prefix{x}{y}{(\binpar{\outputp{x}{y}}{@{y}})} \nonumber\\
	\bangp_{x}{P} & := & \binpar{{x}!\langle{\binpar{D_{x}}{P}}\rangle}{D_{x}} \nonumber
\end{eqnarray}

\begin{eqnarray}
	\bangp_{x}{P} & & \nonumber\\
	=
	& {x}!\langle{(\prefix{x}{y}{(\outputp{x}{y} | @{y})) | P}}\rangle 
	      | \prefix{x}{y}{(\outputp{x}{y} | @{y})} & \nonumber\\
	\red
	& (\outputp{x}{y} | @{y})\substn{\quotep{(\prefix{x}{y}{(@{y} | \outputp{x}{y})) | P}}}{y} & \nonumber\\
	=
	& \outputp{x}{\quotep{(\prefix{x}{y}{(\outputp{x}{y} | @{y})) | P}}}
	  | {(\prefix{x}{y}{(\outputp{x}{y} | @{y})) | P}} & \nonumber\\
	\red
	& \ldots & \nonumber\\
	\red^*
	& P | P | \ldots & \nonumber
\end{eqnarray}

Of course, this encoding, as an implementation, runs away, unfolding
$\bangp{P}$ eagerly. A lazier and more implementable replication
operator, restricted to input-guarded processes, may be obtained as follows.

\begin{eqnarray}
\bangp{\prefix{u}{v}{P}} 
	:= 
	\binpar{\lift{x}{\prefix{u}{v}{(\binpar{D(x)}{P})}}}{D(x)} \nonumber
\end{eqnarray}

\begin{remark}
  Note that the lazier definition still does not deal with summation
  or mixed summation (i.e. sums over input and output). The reader is
  invited to construct definitions of replication that deal with these
  features. 

  Further, the definitions are parameterized in a name, $x$. Can you,
  gentle reader, make a definition that eliminates this parameter and
  guarantees no accidental interaction between the replication
  machinery and the process being replicated -- i.e. no accidental
  sharing of names used by the process to get its work done and the
  name(s) used by the replication to effect copying. This latter
  revision of the definition of replication is crucial to obtaining
  the expected identity $!!P \sim !P$.
\end{remark}

\begin{remark}\label{rem:paradoxical_combinator}
  The reader familiar with the lambda calculus will have noticed the
  similarity between $D$ and the paradoxical combinator.

  [Ed. note: the existence of this seems to suggest we have to be more
  restrictive on the set of processes and names we admit if we are to
  support no-cloning.]
\end{remark}

\subsubsection{Bisimulation}

The computational dynamics gives rise to another kind of equivalence,
the equivalence of computational behavior. As previously mentioned
this is typically captured \emph{via} some form of bisimulation.

% The notion we use in this paper is weak barbed bisimulation
% \cite{milner91polyadicpi}.

The notion we use in this paper is derived from weak barbed
bisimulation \cite{milner91polyadicpi}. 

\begin{definition}
An \emph{observation relation}, $\downarrow_{\mathcal N}$, over a set
of names, $\mathcal N$, is the smallest relation satisfying the rules
below.

\infrule[Out-barb]{y \in {\mathcal N}, \; x \nameeq y}
		  {\outputp{x}{v} \downarrow_{\mathcal N} x}
\infrule[Par-barb]{\mbox{$P\downarrow_{\mathcal N} x$ or $Q\downarrow_{\mathcal N} x$}}
		  {\binpar{P}{Q} \downarrow_{\mathcal N} x}

We write $P \Downarrow_{\mathcal N} x$ if there is $Q$ such that 
$P \wred Q$ and $Q \downarrow_{\mathcal N} x$.
\end{definition}

\begin{definition}
%\label{def.bbisim}
An  ${\mathcal N}$-\emph{barbed bisimulation} over a set of names, ${\mathcal N}$, is a symmetric binary relation 
${\mathcal S}_{\mathcal N}$ between agents such that $P\rel{S}_{\mathcal N}Q$ implies:
\begin{enumerate}
\item If $P \red P'$ then $Q \wred Q'$ and $P'\rel{S}_{\mathcal N} Q'$.
\item If $P\downarrow_{\mathcal N} x$, then $Q\Downarrow_{\mathcal N} x$.
\end{enumerate}
$P$ is ${\mathcal N}$-barbed bisimilar to $Q$, written
$P \wbbisim_{\mathcal N} Q$, if $P \rel{S}_{\mathcal N} Q$ for some ${\mathcal N}$-barbed bisimulation ${\mathcal S}_{\mathcal N}$.
\end{definition}

$\mathcal{R} \subseteq \pi \times \pi$

$P \mathcal{R} Q => \forall P'. P \red P' \Rightarrow \exists Q'. Q \red Q', P' \mathcal{R} Q'$

$P \vdash x \Rightarrow Q \vdash x$

\begin{mathpar}
  \inferrule*[lab=Out-barb]{x \nameeq y}{{y}!\langle{Q}\rangle \vdash x}
  \and
  \inferrule*[lab=Par-barb]{\mbox{$P\vdash x$ or $Q\vdash x$}}{\binpar{P}{Q} \vdash x}
\end{mathpar}

\subsubsection{Contexts}

One of the principle advantages of computational calculi like the
$\pi$-calculus is a well-defined notion of context,
contextual-equivalence and a correlation between
contextual-equivalence and notions of bisimulation. The notion of
context allows the decomposition of a process into (sub-)process and
its syntactic environment, its context. Thus, a context may be
thought of as a process with a ``hole'' (written $\Box$) in it. The
application of a context $M$ to a process $P$, written $M[P]$, is
tantamount to filling the hole in $M$ with $P$. In this paper we do
not need the full weight of this theory, but do make use of the notion
of context in the proof the main theorem. 

\begin{mathpar}
  \inferrule* [lab=summation] {} {{M_{M},M_{N}} \bc \Box \;|\; x.M_{A} \;|\; M_{M}+M_{N}}
  \and
  \inferrule* [lab=agent] {} {{M_{A}} \bc (\vec{x})M_{P} \;| \; \clift{P_0,\ldots,M_{P},\ldots,P_N}}
  \and \\
  \inferrule* [lab=process] {} {{M_{P}} \bc M_{N} \;| \;P|M_{P} }
\end{mathpar} 

\begin{mathpar}
  \inferrule* [lab=sychronization] {} {M_{N} \bc \Box \;|\; x?M_{F} \;|\; x!M_{C}}
  \and
  \inferrule* [lab=abstraction] {} {{M_{F}} \bc (x)M_{P} }
  \and
  \inferrule* [lab=concretion] {} {{M_{C}} \bc \langle M_{P} \rangle }
  \and \\
  \inferrule* [lab=process] {} {{M_{P}} \bc M_{N} \;| \;P|M_{P} }
\end{mathpar}

\begin{definition}[contextual application] Given a context $M$, and
  process $P$, we define the \emph{contextual application}, $M[P] :=
  M\{P/\Box\}$. That is, the contextual application of M to P is the
  substitution of $P$ for $\Box$ in $M$.
\end{definition}

$\meaningof{-} : L \to \mathcal{P}(\pi)$

\begin{mathpar}
  \inferrule* [lab=collection] {} {\meaningof{true} = \pi, \and \meaningof{~E} = \pi \setminus \meaningof{E}, \and \meaningof{E_{1} \& E_{2}} = \meaningof{E_{1}} \cap \meaningof{E_{2}}}
\end{mathpar}

\begin{mathpar}
  \inferrule* [lab=structure] {} {\meaningof{0} = \{ P \in \pi | P \equiv 0 \}, \and \\ \meaningof{E_1 | E_2} = \{ P \in \pi | P \equiv P_{1} | P_{2}, P_{1} \in \meaningof{E_{1}}, P_{2} \in \meaningof{E_2}\} }
\end{mathpar}

\begin{mathpar}
 \inferrule* [lab=behavior] {} {\meaningof{\langle a?b \rangle E} = \{ P \in \pi | P \equiv Q | u?(y)P', \\ \and \\\\ \and \\ \;\;\; u \in \meaningof{a}, \forall z.P'\{z/y\} \in \meaningof{E\{z/b\}}\}, \and \\ \meaningof{a!E} = \{ P \in \pi | P \equiv Q | x!\langle P' \rangle, x \in \meaningof{a} P' \in \meaningof{E}\} }
\end{mathpar}

\begin{mathpar}
 \inferrule* [lab=nominal] {} {\meaningof{\quotep{E}} = \{ \quotep{P} \in \quotep{\pi} | P \in \meaningof{E} \}, \and \meaningof{\quotep{P}} = \{ \quotep{Q} \in \quotep{\pi} | P \equiv Q \} \and \\ \meaningof{@\quotep{E}} = \{ P \in \pi | P \equiv @x, x \in \meaningof{E} \}}
\end{mathpar}

\begin{eqnarray*}
  \\
  \meaningof{-} : TS \to ST
\end{eqnarray*}

\begin{eqnarray*}
  \\
  L : TS \to ST
\end{eqnarray*}

\begin{eqnarray*}
  \\
  P \models E \iff P \in \meaningof{E}
\end{eqnarray*}

\begin{eqnarray*}
  P \approx_{L} Q \iff \forall E \in L. P \models E \iff Q \models E
\end{eqnarray*}

\begin{eqnarray*}
  P \approx_{K} Q
\end{eqnarray*}

\begin{eqnarray*}
  P \approx Q
\end{eqnarray*}

$\approx_{K} = \approx = \approx_{L}$

\subsubsection{Contextual duality}

Note that contexts extend the quotation operation to a family of
operations from processes to names. Given a context, $M$, we can
define a \emph{nominal context}, $\quotep{M}$ by $\quotep{M}[P] :=
\quotep{M[P]}$. To foreshadow what is to come we observe that these
operations enjoy a duality with processes very much like the duality
between vectors and maps from vectors to scalars.

Further, because the calculus is essentially higher-order, we have a
correspondence between contexts and processes. More specifically,
given a name $x$ and a context $M$ we can construct $M^{*}_{x}$ such
that 

\begin{mathpar}
  M^{*}_{x} | \lift{x}{P} \red M[P]
\end{mathpar}

namely,

\begin{mathpar}
  M^{*}_{x} := x?(u).M[\dropn{u}]
\end{mathpar}

The dependence of $M^{*}_{x}$ on a name makes it an abstraction, 

\begin{mathpar}
  M^{*} := (x)x?(u).M[\dropn{u}]
\end{mathpar}

\subsection{Additional notation}

It will sometimes be convenient to denote the process a name
quotes. We already have the notation $x = \quotep{P}$, but it will be
convenient to introduce an alternate notation, $\procn{x}$, when we
want to emphasize the connection to the use of the name. Note that, by
virtue of name equivalence, $\quotep{\procn{x}} \nameeq x$; so, the
notation is consistent with previous definitions.

Further, because names have structure it is possible to effect
substitutions on the basis of that structure. This means we need to
upgrade our notation for substitutions, which we accomplish by
adapting comprehension notation. Thus,

\begin{mathpar}
  P\{ y / x : x \in S \}
\end{mathpar}

is interpreted to mean the process derived from P by replacing (in a
capture-avoiding manner) each occurrence of $x$ in $S$ by $y$. For example,

\begin{mathpar}
  P\{ \quotep{\procn{x}|\procn{x}} / x : x \in \freenames{P} \}
\end{mathpar}

will replace each (occurrence) of a free name $x$ in $P$ by
$\quotep{\procn{x}|\procn{x}}$.

Also, we will avail ourselves of the notation $x^{L}$ and $x^{R}$ to
denote injections of a name into disjoint copies of the name
space. There are numerous ways to accomplish this. One example can be
found in \cite{MeredithR05}. This notation overloads to vectors of
names: $\vec{x}^{\pi} := (x_{i}^{\pi} \; : \; 0 \leq i < |\vec{x}| )$ where $\pi \in \{L,R\}$.

We also use $P^{\Box} := P|\Box$.

In \cite{MeredithR05} an interpretation of the new operator is
given. It turns out that there are several possible interpretations
all enjoying the requisite algebraic properties of the operator (see
\cite{milner91polyadicpi}). We will therefore make liberal use of
$(\nu\; \vec{x})P$.

% subsection the_syntax_and_semantics_of_the_notation_system (end)   

\input{qm2pi.qmops} 

\input{qm2pi.sterngerlach} 

\input{qm2pi.metric} 

% section concurrent_process_calculi (end)

%\input{qm2pi.proofsketch}

% section proof sketch (end)

%\input{qm2pi.slviaknots} 

% section spatial logic via knots (end)

\input{qm2pi.conclusion}

% section conclusion (end)

%\input{qm2pi.dtcodes} 

% section wiring algorithm (end)

\input{qm2pi.ack} 

% section acknowledgments (end)

\newpage


\bibliographystyle{plain}   
\bibliography{../../biblios/main.bib}

\input{qm2pi.rhodetails}

\end{document}



% section proof sketch (end)

%\section{Unlikely characters: spatial logic for
  knots}\label{sub:characteristic_formulae} % (fold)

Associated to the mobile process calculi are a family of logics known
as the Hennessy-Milner logics. These logics typically enjoy a
semantics interpreting formulae as sets of processes that when
factored through the encoding outlined above allows an identification
of classes of knots with logical formulae. In the context of this
encoding the sub-family known as the spatial logics \cite{CairesC03}
\cite{CairesC04} \cite{Caires04} are of particular interest providing
several important features for expressing and reasoning about
properties (i.e. classes) of knots. We hint here at how this may be done.

%\begin{description}
%\item [structural connectives] 
\subsubsection{Structural connectives} The spatial logics enjoy
structural connectives corresponding, at the logical level, to the
parallel composition ($P | Q$) and new name ($(\nu \; x)P$)
connectives for processes. As illustrated in the examples below, these
connectives are extremely expressive given the shape of our encoding.
%\item [decideable satisfaction]

\subsubsection{Decideable satisfaction}
In \cite{Caires04} the satisfaction relation is shown to be decideable
for a rich class of processes. It further turns out that the image of
the our encoding is a proper subset of that class. This result
provides the basis for an algorithm by which to search for knots
enjoying a given property.
%\item [characteristic formulae]

\subsubsection{Characteristic formulae}
In the same paper \cite{Caires04} , Caires presents a means of calculating
characteristic formulae, selecting equivalence classes of processes
up to a pre--specified depth limit on the support set of names. Composed with our
encoding, this characteristic formula can be used to select
characteristic formulae for knots.
%\end{description}

\subsubsection{Spatial logic formulae}

The grammar below (segmented for comprehension) summarizes the syntax
of spatial logic formulae. We employ illustrative examples in the
sequel to provide an intuitive understanding of their meaning
referring the reader to \cite{Caires04} for a more detailed explication
of the semantics.

\begin{mathpar}
  \inferrule* [lab=boolean] {} {{A,B} \bc T \;|\; \neg A \;|\; A \wedge B \;|\; \eta = \eta'}
  \and
  \inferrule* [lab=spatial] {} {|\; \pzero \;|\; A | B \;|\; x \text{\textregistered} A \;|\; \forall x . A \;|\;  H x . A}
  \and
  \inferrule* [lab=behavioral] {} {|\; \alpha . A}
  \and 
  \inferrule* [lab=recursion] {} {|\; X(\vec{u}) \;|\; \mu X(\vec{u}) . A}
  \and
  \inferrule* [lab=action] {} {\alpha \bc \langle x?(\vec{y}) \rangle \;|\; \langle x!(\vec{y}) \rangle \;|\; \langle \tau \rangle}
  \and 
  \inferrule* [lab=name] {} {\eta \bc x \;|\; \tau}
\end{mathpar} 

% subsection characteristic_formulae (end)   	 

\subsection{Example formulae}\label{sub:example_formulae_} % (fold)

\subsubsection{Crossing as formula.}
% 
% \begin{align*}
%   \frac{d}{dx} \sin x &= \cos x 
%   & \frac{d}{dx} e^x &= e^x \\
%   \frac{d}{dx} \cos x &= - \sin x 
%   & \frac{d}{dx} \log x &= \frac{1}{x} \\
% \end{align*} 

\begin{align*}
 \mu C(x_{0},x_{1},y_{0},y_{1},u).&(\langle x_{0}?(z) \rangle(\langle u! \rangle\langle y_{1}!z \rangle C(x_{0},x_{1},y_{0},y_{1},u)) & \\
  & \wedge \langle y_{1}?(z) \rangle (\langle u! \rangle \langle x_{0}!z \rangle C(x_{0},x_{1},y_{0},y_{1},u)) & \\
  & \wedge \langle x_{1}?(z) \rangle (\langle u? \rangle \langle y_{0}!z \rangle C(x_{0},x_{1},y_{0},y_{1},u)) & \\
  & \wedge \langle y_{0}?(z) \rangle (\langle u? \rangle \langle x_{1}!z \rangle C(x_{0},x_{1},y_{0},y_{1},u))) &
\end{align*}

The lexicographical similarity between the shape of this formulae and
the shape of definition of the process representing a crossing reveals
the intuitive meaning of this formulae. It describes the capabilities
of a process that has the right to represent a crossing. For example
it picks out processes that may perform an input on the port $x_0$ in
its initial menu of capabilities. What differentiates the formula
from the process, however, is that the crossing process is the
smallest candidate to satisfy the formula. Infinitely many other
processes -- with internal behavior hidden behind this interface, so
to speak -- also satisfy this formula. Even this simple formula,
then, can be seen to open a new view onto knots, providing a
computational interpretation of \emph{virtual} knots.

Note that this formula is derived by hand. A similar formula can be
derived by employing Caires' calculation of characteristic formula
\cite{Caires04} to the process representing a crossing. In light of
this discussion, we let
$\meaningof{C}_{\phi}(x0,x1,y0,y1,u)$ denote a formula specifying the
dynamics we wish to capture of a crossing. To guarantee we preserve
the shape of the interface and minimal semantics we demand that
$\meaningof{C}_{\phi}(x0,x1,y0,y1,u) \Rightarrow
\textbf{C}(x0,x1,y0,y1,u)$ where $\textbf{C}(x0,x1,y0,y1,u)$ denotes
the formula above.
                            
\subsubsection{Crossing number constraints.}
The moral content of the context lemma (Lemma \ref{context}) is that the notion of
``locality'' in the Reidemeister moves is effectively captured by the
parallel composition operator of the process calculus. This intuition
extends through the logic. Given a formula,
$\meaningof{C}_{\phi}(x0,x1,y0,y1,u)$, we can use the structural
connectives to specify constraints on crossing numbers, such as at
least $n$ crossings, or exactly $n$ crossings.
\begin{mathpar}
  \inferrule* [lab=at-least-n] {} { K^{\geq n}_{\phi}(\vec{xs},\vec{ys}) := \Pi_{i=0}^{n-1} Hu . \meaningof{C}_{\phi}(xs_i,ys_i,u) | T }
  \and 
  \inferrule* [lab=exactly-n] {} { K^{= n}_{\phi}(\vec{xs},\vec{ys}) := \Pi_{i=0}^{n-1} Hu . \meaningof{C}_{\phi}(xs_i,ys_i,u) | \neg (\forall x_0,y_0,x_1,y_1,u . \meaningof{C}_{\phi}(x_0,y_0,x_1,y_1,u) | T) }
\end{mathpar}

To round out this section, recall that the encoding of an $n$-crossing
knot decomposes into a parallel composition of $n$ \emph{copies} of a
crossing process together with a wiring harness. To specify different
knot classes with the same crossing number amounts to specifying
logical constraints on the wiring harness. In the interest of space,
we defer examples to a forthcoming paper. Suffice it to say that both
the conditions ``alternating knot'' and ``contains the tangle
corresponding to 5/3'' are expressible. For example, it is possible to
calculate the characteristic formula of a process corresponding to the
tangle 5/3 and conjoin it into the classifying formula via the
composition connective of the logic.

Finally, we wish to observe that it is entirely within reason to
contemplate a more domain-specific version of spatial logic tailored
to the shape of processes in the image of the encoding. Such a
domain-specific logic would have a better claim to the title formal
language of knot properties.

% subsection example_formulae_ (end)

% section knots_as_processes (end) 

% section spatial logic via knots (end)

\section{Conclusions and future work}

\paragraph{Testing physical space}
You, gentle reader, may wonder why of all the theorems to be proved
given this set up we pick the one above. In some sense it's hardly
central to quantum mechanics. We see it as central in the sense that
it firmly establishes a notion of physical space arising from a notion
of the equivalence of behavior. Relating bisimulation to a metric is a
big step forward, but one is faced with interpreting the relationship
of that metric space to something more physical. Quantum mechanical
notions of ``physical'' space are still far from intuitive, but by
relating this idea of distance as testing to calculations that predict
physical circumstances we are making a not insignificant step forward
toward an understanding of the physical space we inhabit as
essentially dynamic.

\paragraph{Effectivity and simulation}
One of the observations we have yet to make is that the entire program
spelled out here is effective. We have built various interpreters for
the reflective calculus at work in this interpretation. In principle,
then, we can simulate quantum mechanics on a computer. The place where
the simulation may lose fidelity is the infinitely branching summation
for the annihilator.

In this connection i also want to point out that the evaluation style
calculation of the inner product puts the non-determinism of the
summation right at the heart of measurement. This suggests that
Milner's original reduction-based formulation of the dynamics of his
calculi in terms of sums was not just notationally suggestive of a
notion of measure-and-continue but captured some significant part of
the physics.

\paragraph{Quantum continuations}
In light of this last observation i want to point out that the
predominant account of quantum mechanics is missing a key aspect of a
truly compositional story of the physical situation. In a real lab,
when a measurement is made the observation can be made to feed into
another device that then makes another measurement conditioned on the
results of the first. This means that after the superposition was
collapsed the entire experimental set up remained in
superposition. While QM offers a means of writing this down it doesn't
quite line up well with the well-trodden formulation of computation
and continuation that we see so succinctly expressed in Milner's
calculi. This suggests that there might be advantages to this account
of dynamics waiting to be explored.

\paragraph{Quantum logic}
In this connection, we also note that by virtue of having the
Hennessy-Milner construction, we can pull the construction through the
interpretation of QM. This gives us a natural candidate for a quantum
logic that enjoys an extremely tight connection with it's domain of
interpretation, making the construction much less ad hoc (rather it is
the image of functor!).

\paragraph{Quantum probabiity}
i have questions about the basis of the interpretation of inner
product as probability amplitude. In particular, using which
axiomatization of probability theory does the notion of probability
amplitude earn the right to be so dubbed? In other words, where is the
proof that the operation for calculating a probability amplitude (and
then squaring) satisfies the axioms of what it means to calculate a
probability? Even if such a proof exists (i have yet to find it in the
literature), i wonder if it might not be possible to turn things on
their heads. Can we view the calculation of the probability amplitude
as an axiomatization of probability? If so, then the definition we
give for calculating probability amplitude may provide the basis for
an \emph{effective} theory of probability.

\paragraph{Quantum vs ``biological'' information}
Finally, i want to conclude with a more philosophical observation. At
a recent workshop in which QM was a predominant topic i noticed
something about quantum information. The speaker was giving a riveting
discussion of axiomatic QM and showing how properties of ``no
cloning'' and ``no deleting'' emerged as consequences of the
axiomatization. Theorems of this form are necessary to give us a sense
of confidence that our axioms characterize the physical theory. What
struck me, though, was that if quantum information is neither erasable
nor replicable it is markedly different from \emph{life}. Two of the
things we know about life is that

\begin{itemize}
  \item it ends;
  \item to gain some measure of persistence, to transcend it's
    finitude it is imminently copyable.
\end{itemize}

Both of these qualities are summarized succinctly in the aphorism: all
flesh is grass. For me these two kinds of ``information'' -- call them
quantum and biological -- are end points on a spectrum of strategies
for persistence. At one end, we have those curious entities that enjoy
uniqueness and permanence; at the other, we have those who in the face
of a certain end and an uncertain present make a go of passing
something on. To me one of the more remarkable aspects of the latter
strategy is that in the presence of noise (and certain features of
copying) we get a kind of dynamism, a chance for improvement against a
given persistent condition.

% subsection other_calculi_other_bisimulations_and_geometry_as_behavior (end)




% section conclusion (end)

%\documentclass[12pt]{llncs}
%\documentclass{jktr}

\usepackage[pdftex]{hyperref}                   
\usepackage {listings}
\usepackage {mathpartir}
\usepackage{bcprules}
%\usepackage{listings}
                       
\usepackage{graphicx} 
%\usepackage[margins=2.5cm,nohead,nofoot]{geometry}
%\usepackage{geometry}
\usepackage{amsfonts}
\usepackage{amstext}
\usepackage{latexsym}
\usepackage{amssymb}
\usepackage{color}


%\include{myPreamble}
\include{qm2pi.local} 

%\ifpdf
%\usepackage[pdftex]{graphicx}
%\else
%\usepackage{graphicx}
%\fi

 % \ifpdf
%  \usepackage{pdfsync}
%  \if


%\title{Brief Article}
%\author{David F. Snyder}
%\author{L.G. Meredith}

%\address{Dept. of Math., Texas State University--San Marcos, San Marcos, TX 78666}
       
\pagestyle{empty}


\begin{document}

\lstset{language=[Objective]Caml,frame=shadowbox}

\input{qm2pi.front}

% section front matter (end)

\input{qm2pi.intro} 
 
% section introduction (end)

% \input{qm2pi.knotations} 

% section notation (end)

\input{qm2pi.process.calculi} 

% section concurrent_process_calculi_and_spatial_logics_ (end)
    
%\input{qm2pi.knots2pi} 

%\input{qm2pi.trefoil} 

%\input{qm2pi.mainthm} 

% subsection basic_interpretation (end)

%\input{qm2pi.rho.presentation} 
\subsection{The syntax and semantics of the notation system}\label{sub:the_syntax_and_semantics_of_the_notation_system} % (fold)

We now summarize a technical presentation of the calculus that
embodies our theory of dynamics. The typical presentation of such a
calculus follows the style of giving generators and relations on
them. The grammar, below, describing term constructors, freely
generates the set of processes, $\Proc$. This set is then quotiented
by a relation known as structural congruence and it is over this set
that the notion of dynamics is expressed. This presentation is
essentially that of \cite{MeredithR05} with the addition of
polyadicity and summation. For readability we have relegated some of
the technical subtleties to an appendix.

\subsubsection{Process grammar}\label{subsub:process_grammar}

\begin{mathpar}
  \inferrule* [lab=synchronization] {} {{M} \bc \pzero \;|\; x?F \;|\; x!C }
  \and
  \inferrule* [lab=abstraction] {} {{F} \bc (x)P}
  \and
  \inferrule* [lab=concretion] {} {{C} \bc \langle Q \rangle}
  \and
  \inferrule* [lab=process] {} {{P,Q} \bc M \;| \;P|Q \;|\; @{x}}
  \and
  \inferrule* [lab=name] {} {{x} \bc \quotep{P}}
\end{mathpar} 

Note that $\vec{x}$ (resp. $\vec{P}$) denotes a vector of names
(resp. processes) of length $|\vec{x}|$ (resp. $|\vec{P}|$). We adopt
the following useful abbreviations.

\begin{mathpar}
   x?(\vec{y}).P := x.(\vec{y})P \and  x\clift{\vec{P}} := x.\clift{\vec{P}}
   \and x!(y) := \lift{x}{\dropn{y}}
   \and \Pi_{i=0}^{n-1}P_i := P_0 | \ldots | P_{n-1}
\end{mathpar}

\subsubsection{Structural congruence}

\paragraph{Free and bound names and alpha-equivalence.} At the
core of structural equivalence is alpha-equivalence which identifies
process that are the same up to a change of variable. Formally, we
recognize the distinction between free and bound names. The free names
of a process, $\freenames{P}$, may be calculated recursively as
follows:

\begin{mathpar}
\freenames{\pzero} := \emptyset
  \and \\
  \freenames{x?(y).P} := \{ x \} \cup (\freenames{P} \setminus \{ y \})
  \and 
  \freenames{x!\langle P \rangle} := \{ x \} \cup \{ P \} 
  \and \\
  \freenames{P|Q} := \freenames{P} \cup \freenames{Q}
  \and \\
  \freenames{@{x}} := \{ x \}
\end{mathpar}

$\pi$
$\quotep{\pi}$

$\freenames{-} : \pi \to \mathcal{P}(\quotep{\pi})$

\begin{eqnarray*}
  \freenames{\pzero} & := & \emptyset \\
  \freenames{x?(y).P} & := & \{ x \} \cup (\freenames{P} \setminus \{ y \}) \\
  \freenames{x!\langle P \rangle} & := & \{ x \} \cup \{ P \} \\
  \freenames{P|Q} & := & \freenames{P} \cup \freenames{Q} \\
  \freenames{\dropn{x}} & := & \{ x \}
\end{eqnarray*}

The bound names of a process, $\boundnames{P}$, are those names occurring in $P$
that are not free. For example, in $x?(y).0$, the name $x$ is free, while $y$ is bound.

\begin{mathpar}
  \inferrule* [lab=monoidal-laws] {} { P|Q \equiv Q|P \and P|0 \equiv P \and P|(Q|R) \equiv (P|Q)|R }
\end{mathpar}

\begin{mathpar}
  \inferrule* [lab=alpha-equivalence] {} { (x)P \equiv (y)P\{y/x\} \and y \not\in \freenames{P} }
\end{mathpar}

\begin{definition}
Then two processes, $P,Q$, are alpha-equivalent if $P = Q\{\vec{y}/\vec{x}\}$ for
some $\vec{x} \in \boundnames{Q},\vec{y} \in \boundnames{P}$, where $Q\{\vec{y}/\vec{x}\}$
denotes the capture-avoiding substitution of $\vec{y}$ for $\vec{x}$ in $Q$.
\end{definition}

\begin{definition}
  The {\em structural congruence} \cite{SangiorgiWalker} , $\equiv$,
  between processes is the least congruence containing
  alpha-equivalence, satisfying the abelian monoid laws
  (associativity, commutativity and $\pzero$ as identity) for parallel
  composition $|$ and for summation $+$.
\end{definition}

\subsection{Name equivalence}

We take name equivalence, written $\nameeq$, to be the smallest
equivalence relation generated by the following rules.

\begin{mathpar}
\inferrule*[lab=Quote-drop]
{ }
{ \quotep{@{x}} \nameeq x }

\inferrule*[lab=Struct-equiv]
{ P \scong Q }
{ \quotep{P} \nameeq \quotep{Q} }
\end{mathpar}

The astute reader will have noticed that the mutual recursion of names
and processes imposes a mutual recursion on alpha-equivalence and
structural equivalence via name-equivalence. Fortunately, all of this
works out pleasantly and we may calculate in the natural way, free of
concern. The reader interested in the details is referred to the
appendix \ref{appendix:rho_details}.

\subsection{Substitution}

We use $\Proc$ for the set of processes, $\QProc$ for the set of
names, and $\id{\{}\vec{y} / \vec{x} \id{\}}$ to denote partial maps,
$s : \QProc \rightarrow \QProc$. A map, $s$ lifts, uniquely, to a map
on process terms, $\widehat{s} : \Proc \rightarrow \Proc$ by the
following equations.

\begin{mathpar}
  (0) \psubstp{Q}{P} := 0 \\
  (R \juxtap S) \psubstp{Q}{P}
  :=    
  (R)\psubstp{Q}{P} \juxtap (S) \psubstp{Q}{P} \\
  (x?(y).R) \psubstp{Q}{P}    
  :=    
  (x)\substp{Q}{P} (z)\concat( (R \psubstn{z}{y}) \psubstp{Q}{P} ) \\
  (\lift{x}{R}) \psubstp{Q}{P}  
  :=
  \lift{(x)\substp{Q}{P}}{ R \psubstp{Q}{P} } \\
%   (\dropn{x})  \psubstp{Q}{P}       
%   := 
%   \left\{ 
%     \begin{array}{ccc} 
%       \dropn{\quotep{Q}} & & x \nameeq \quotep{P} \\
%       \dropn{x} & & otherwise \\
%     \end{array}
%   \right. 
  (\dropn{x})  \psubstp{Q}{P}       
  := 
  \left\{ 
    \begin{array}{ccc} 
      Q & & x \nameeq \quotep{P} \\
      \dropn{x} & & otherwise \\
    \end{array}
  \right.
\end{mathpar}
 

where

\begin{eqnarray}
  (x)\id{\{} \lpquote Q \rpquote / \lpquote P \rpquote \id{\}}            = 
  \left\{ 
    \begin{array}{ccc}
      \lpquote Q \rpquote & & x \nameeq \lpquote P \rpquote \\
      x & & otherwise \\
    \end{array}
  \right. \nonumber
\end{eqnarray}

and $z$ is chosen distinct from $\quotep{P}$, $\quotep{Q}$, the free
names in $Q$, and all the names in $R$. Our $\alpha$-equivalence will
be built in the standard way from this substitution.

\begin{remark}\label{rem:no_self_referential_names}
  One consequence of these definitions is that $\forall P. \quotep{P}
  \not\in \freenames{P}$.
\end{remark}

\subsection{ Dynamic quote: an example }

Anticipating something of what's to come, consider applying the
substitution, $\widehat{\id{\{}u / z \id{\}}}$, to the following pair
of processes, $\lift{w}{y!(z)}$ and $w[ \lpquote y!(z) \rpquote ]$.

\begin{eqnarray}
	\lift{w}{y!(z)}\widehat{\id{\{}u / z \id{\}}}
		& = &
		\lift{w}{y!(u)} \nonumber\\
	w[ \lpquote y!(z) \rpquote ] \widehat{ \id{\{}u / z \id{\}} }
		& = &
		w[ \lpquote y!(z) \rpquote ] \nonumber
\end{eqnarray}

Because the body of the process between quotes is impervious to
substitution, we get radically different answers. In fact, by
examining the first process in an input context,
e.g. $x?(z).\lift{w}{y!(z)}$, we see that the process under the lift
operator may be shaped by prefixed inputs binding a name inside it. In
this sense, the lift operator will be seen as a way to dynamically
construct processes before reifying them as names.

Finally equipped with these standard features we can present the
dynamics of the calculus.

\subsubsection{Operational semantics} 

Finally, we introduce the computational dynamics. What marks these
algebras as distinct from other more traditionally studied algebraic
structures, e.g. vector spaces or polynomial rings, is the manner in
which dynamics is captured. In traditional structures, dynamics is typically
expressed through morphisms between such structures, as in linear maps
between vector spaces or morphisms between rings. In algebras
associated with the semantics of computation, the dynamics is
expressed as part of the algebraic structure itself, through a
reduction reduction relation typically denoted by $\red$. Below, we
give a recursive presentation of this relation for the calculus used
in the encoding.

$\red \subseteq \pi \times \pi$
$\red : \pi \to \mathcal{P}(\pi)$

\begin{mathpar}
  \inferrule* [lab=Comm] { \textsf{match}( x_{src}, x_{trgt} ) } { x_{trgt}?(y)P \; | \; x_{src}!\langle {Q} \rangle \red P\{\quotep{Q}/y}\} }
  \and \\
  \inferrule* [lab=Par] {{P} \red {P}'} {{{P} | {Q}} \red {{P}' | {Q}}}
  \and
  \inferrule* [lab=Equiv]{{{P} \scong {P}'} \andalso {{P}' \red {Q}'} \andalso {{Q}' \scong {Q}}}{{P} \red {Q}}
\end{mathpar}

\begin{eqnarray*}
  match_{\equiv} (\quotep{P},\quotep{Q}) & := & P \equiv Q \\
  match_{\dagger}(\quotep{P},\quotep{Q}) & := & \forall R. P|Q \red^{*} R => R \red^{*} 0 \\
  match_{K}(\quotep{P},\quotep{Q}) & := & K \mbox{ for some context } K
\end{eqnarray*}

$u?(x)P | u!\langle Q \rangle \red P\{\quotep{Q}/x\}$

%We write $\wred$ for $\red^*$, and $P\red$ if $\exists Q $ such that $ P \red Q$.
We write $P\red$ if $\exists Q $ such that $ P \red Q$ and $P\not\red$, otherwise.

\section{Replication}

As mentioned before, it is known that replication (and hence
recursion) can be implemented in a higher-order process algebra
\cite{SangiorgiWalker}. As our first example of calculation with the
machinery thus far presented we give the construction explicitly in
the {\rhoc}.

\begin{eqnarray}
	D_{x} & := & \prefix{x}{y}{(\binpar{\outputp{x}{y}}{@{y}})} \nonumber\\
	\bangp_{x}{P} & := & \binpar{{x}!\langle{\binpar{D_{x}}{P}}\rangle}{D_{x}} \nonumber
\end{eqnarray}

\begin{eqnarray}
	\bangp_{x}{P} & & \nonumber\\
	=
	& {x}!\langle{(\prefix{x}{y}{(\outputp{x}{y} | @{y})) | P}}\rangle 
	      | \prefix{x}{y}{(\outputp{x}{y} | @{y})} & \nonumber\\
	\red
	& (\outputp{x}{y} | @{y})\substn{\quotep{(\prefix{x}{y}{(@{y} | \outputp{x}{y})) | P}}}{y} & \nonumber\\
	=
	& \outputp{x}{\quotep{(\prefix{x}{y}{(\outputp{x}{y} | @{y})) | P}}}
	  | {(\prefix{x}{y}{(\outputp{x}{y} | @{y})) | P}} & \nonumber\\
	\red
	& \ldots & \nonumber\\
	\red^*
	& P | P | \ldots & \nonumber
\end{eqnarray}

Of course, this encoding, as an implementation, runs away, unfolding
$\bangp{P}$ eagerly. A lazier and more implementable replication
operator, restricted to input-guarded processes, may be obtained as follows.

\begin{eqnarray}
\bangp{\prefix{u}{v}{P}} 
	:= 
	\binpar{\lift{x}{\prefix{u}{v}{(\binpar{D(x)}{P})}}}{D(x)} \nonumber
\end{eqnarray}

\begin{remark}
  Note that the lazier definition still does not deal with summation
  or mixed summation (i.e. sums over input and output). The reader is
  invited to construct definitions of replication that deal with these
  features. 

  Further, the definitions are parameterized in a name, $x$. Can you,
  gentle reader, make a definition that eliminates this parameter and
  guarantees no accidental interaction between the replication
  machinery and the process being replicated -- i.e. no accidental
  sharing of names used by the process to get its work done and the
  name(s) used by the replication to effect copying. This latter
  revision of the definition of replication is crucial to obtaining
  the expected identity $!!P \sim !P$.
\end{remark}

\begin{remark}\label{rem:paradoxical_combinator}
  The reader familiar with the lambda calculus will have noticed the
  similarity between $D$ and the paradoxical combinator.

  [Ed. note: the existence of this seems to suggest we have to be more
  restrictive on the set of processes and names we admit if we are to
  support no-cloning.]
\end{remark}

\subsubsection{Bisimulation}

The computational dynamics gives rise to another kind of equivalence,
the equivalence of computational behavior. As previously mentioned
this is typically captured \emph{via} some form of bisimulation.

% The notion we use in this paper is weak barbed bisimulation
% \cite{milner91polyadicpi}.

The notion we use in this paper is derived from weak barbed
bisimulation \cite{milner91polyadicpi}. 

\begin{definition}
An \emph{observation relation}, $\downarrow_{\mathcal N}$, over a set
of names, $\mathcal N$, is the smallest relation satisfying the rules
below.

\infrule[Out-barb]{y \in {\mathcal N}, \; x \nameeq y}
		  {\outputp{x}{v} \downarrow_{\mathcal N} x}
\infrule[Par-barb]{\mbox{$P\downarrow_{\mathcal N} x$ or $Q\downarrow_{\mathcal N} x$}}
		  {\binpar{P}{Q} \downarrow_{\mathcal N} x}

We write $P \Downarrow_{\mathcal N} x$ if there is $Q$ such that 
$P \wred Q$ and $Q \downarrow_{\mathcal N} x$.
\end{definition}

\begin{definition}
%\label{def.bbisim}
An  ${\mathcal N}$-\emph{barbed bisimulation} over a set of names, ${\mathcal N}$, is a symmetric binary relation 
${\mathcal S}_{\mathcal N}$ between agents such that $P\rel{S}_{\mathcal N}Q$ implies:
\begin{enumerate}
\item If $P \red P'$ then $Q \wred Q'$ and $P'\rel{S}_{\mathcal N} Q'$.
\item If $P\downarrow_{\mathcal N} x$, then $Q\Downarrow_{\mathcal N} x$.
\end{enumerate}
$P$ is ${\mathcal N}$-barbed bisimilar to $Q$, written
$P \wbbisim_{\mathcal N} Q$, if $P \rel{S}_{\mathcal N} Q$ for some ${\mathcal N}$-barbed bisimulation ${\mathcal S}_{\mathcal N}$.
\end{definition}

$\mathcal{R} \subseteq \pi \times \pi$

$P \mathcal{R} Q => \forall P'. P \red P' \Rightarrow \exists Q'. Q \red Q', P' \mathcal{R} Q'$

$P \vdash x \Rightarrow Q \vdash x$

\begin{mathpar}
  \inferrule*[lab=Out-barb]{x \nameeq y}{{y}!\langle{Q}\rangle \vdash x}
  \and
  \inferrule*[lab=Par-barb]{\mbox{$P\vdash x$ or $Q\vdash x$}}{\binpar{P}{Q} \vdash x}
\end{mathpar}

\subsubsection{Contexts}

One of the principle advantages of computational calculi like the
$\pi$-calculus is a well-defined notion of context,
contextual-equivalence and a correlation between
contextual-equivalence and notions of bisimulation. The notion of
context allows the decomposition of a process into (sub-)process and
its syntactic environment, its context. Thus, a context may be
thought of as a process with a ``hole'' (written $\Box$) in it. The
application of a context $M$ to a process $P$, written $M[P]$, is
tantamount to filling the hole in $M$ with $P$. In this paper we do
not need the full weight of this theory, but do make use of the notion
of context in the proof the main theorem. 

\begin{mathpar}
  \inferrule* [lab=summation] {} {{M_{M},M_{N}} \bc \Box \;|\; x.M_{A} \;|\; M_{M}+M_{N}}
  \and
  \inferrule* [lab=agent] {} {{M_{A}} \bc (\vec{x})M_{P} \;| \; \clift{P_0,\ldots,M_{P},\ldots,P_N}}
  \and \\
  \inferrule* [lab=process] {} {{M_{P}} \bc M_{N} \;| \;P|M_{P} }
\end{mathpar} 

\begin{mathpar}
  \inferrule* [lab=sychronization] {} {M_{N} \bc \Box \;|\; x?M_{F} \;|\; x!M_{C}}
  \and
  \inferrule* [lab=abstraction] {} {{M_{F}} \bc (x)M_{P} }
  \and
  \inferrule* [lab=concretion] {} {{M_{C}} \bc \langle M_{P} \rangle }
  \and \\
  \inferrule* [lab=process] {} {{M_{P}} \bc M_{N} \;| \;P|M_{P} }
\end{mathpar}

\begin{definition}[contextual application] Given a context $M$, and
  process $P$, we define the \emph{contextual application}, $M[P] :=
  M\{P/\Box\}$. That is, the contextual application of M to P is the
  substitution of $P$ for $\Box$ in $M$.
\end{definition}

$\meaningof{-} : L \to \mathcal{P}(\pi)$

\begin{mathpar}
  \inferrule* [lab=collection] {} {\meaningof{true} = \pi, \and \meaningof{~E} = \pi \setminus \meaningof{E}, \and \meaningof{E_{1} \& E_{2}} = \meaningof{E_{1}} \cap \meaningof{E_{2}}}
\end{mathpar}

\begin{mathpar}
  \inferrule* [lab=structure] {} {\meaningof{0} = \{ P \in \pi | P \equiv 0 \}, \and \\ \meaningof{E_1 | E_2} = \{ P \in \pi | P \equiv P_{1} | P_{2}, P_{1} \in \meaningof{E_{1}}, P_{2} \in \meaningof{E_2}\} }
\end{mathpar}

\begin{mathpar}
 \inferrule* [lab=behavior] {} {\meaningof{\langle a?b \rangle E} = \{ P \in \pi | P \equiv Q | u?(y)P', \\ \and \\\\ \and \\ \;\;\; u \in \meaningof{a}, \forall z.P'\{z/y\} \in \meaningof{E\{z/b\}}\}, \and \\ \meaningof{a!E} = \{ P \in \pi | P \equiv Q | x!\langle P' \rangle, x \in \meaningof{a} P' \in \meaningof{E}\} }
\end{mathpar}

\begin{mathpar}
 \inferrule* [lab=nominal] {} {\meaningof{\quotep{E}} = \{ \quotep{P} \in \quotep{\pi} | P \in \meaningof{E} \}, \and \meaningof{\quotep{P}} = \{ \quotep{Q} \in \quotep{\pi} | P \equiv Q \} \and \\ \meaningof{@\quotep{E}} = \{ P \in \pi | P \equiv @x, x \in \meaningof{E} \}}
\end{mathpar}

\begin{eqnarray*}
  \\
  \meaningof{-} : TS \to ST
\end{eqnarray*}

\begin{eqnarray*}
  \\
  L : TS \to ST
\end{eqnarray*}

\begin{eqnarray*}
  \\
  P \models E \iff P \in \meaningof{E}
\end{eqnarray*}

\begin{eqnarray*}
  P \approx_{L} Q \iff \forall E \in L. P \models E \iff Q \models E
\end{eqnarray*}

\begin{eqnarray*}
  P \approx_{K} Q
\end{eqnarray*}

\begin{eqnarray*}
  P \approx Q
\end{eqnarray*}

$\approx_{K} = \approx = \approx_{L}$

\subsubsection{Contextual duality}

Note that contexts extend the quotation operation to a family of
operations from processes to names. Given a context, $M$, we can
define a \emph{nominal context}, $\quotep{M}$ by $\quotep{M}[P] :=
\quotep{M[P]}$. To foreshadow what is to come we observe that these
operations enjoy a duality with processes very much like the duality
between vectors and maps from vectors to scalars.

Further, because the calculus is essentially higher-order, we have a
correspondence between contexts and processes. More specifically,
given a name $x$ and a context $M$ we can construct $M^{*}_{x}$ such
that 

\begin{mathpar}
  M^{*}_{x} | \lift{x}{P} \red M[P]
\end{mathpar}

namely,

\begin{mathpar}
  M^{*}_{x} := x?(u).M[\dropn{u}]
\end{mathpar}

The dependence of $M^{*}_{x}$ on a name makes it an abstraction, 

\begin{mathpar}
  M^{*} := (x)x?(u).M[\dropn{u}]
\end{mathpar}

\subsection{Additional notation}

It will sometimes be convenient to denote the process a name
quotes. We already have the notation $x = \quotep{P}$, but it will be
convenient to introduce an alternate notation, $\procn{x}$, when we
want to emphasize the connection to the use of the name. Note that, by
virtue of name equivalence, $\quotep{\procn{x}} \nameeq x$; so, the
notation is consistent with previous definitions.

Further, because names have structure it is possible to effect
substitutions on the basis of that structure. This means we need to
upgrade our notation for substitutions, which we accomplish by
adapting comprehension notation. Thus,

\begin{mathpar}
  P\{ y / x : x \in S \}
\end{mathpar}

is interpreted to mean the process derived from P by replacing (in a
capture-avoiding manner) each occurrence of $x$ in $S$ by $y$. For example,

\begin{mathpar}
  P\{ \quotep{\procn{x}|\procn{x}} / x : x \in \freenames{P} \}
\end{mathpar}

will replace each (occurrence) of a free name $x$ in $P$ by
$\quotep{\procn{x}|\procn{x}}$.

Also, we will avail ourselves of the notation $x^{L}$ and $x^{R}$ to
denote injections of a name into disjoint copies of the name
space. There are numerous ways to accomplish this. One example can be
found in \cite{MeredithR05}. This notation overloads to vectors of
names: $\vec{x}^{\pi} := (x_{i}^{\pi} \; : \; 0 \leq i < |\vec{x}| )$ where $\pi \in \{L,R\}$.

We also use $P^{\Box} := P|\Box$.

In \cite{MeredithR05} an interpretation of the new operator is
given. It turns out that there are several possible interpretations
all enjoying the requisite algebraic properties of the operator (see
\cite{milner91polyadicpi}). We will therefore make liberal use of
$(\nu\; \vec{x})P$.

% subsection the_syntax_and_semantics_of_the_notation_system (end)   

\input{qm2pi.qmops} 

\input{qm2pi.sterngerlach} 

\input{qm2pi.metric} 

% section concurrent_process_calculi (end)

%\input{qm2pi.proofsketch}

% section proof sketch (end)

%\input{qm2pi.slviaknots} 

% section spatial logic via knots (end)

\input{qm2pi.conclusion}

% section conclusion (end)

%\input{qm2pi.dtcodes} 

% section wiring algorithm (end)

\input{qm2pi.ack} 

% section acknowledgments (end)

\newpage


\bibliographystyle{plain}   
\bibliography{../../biblios/main.bib}

\input{qm2pi.rhodetails}

\end{document}

 

% section wiring algorithm (end)

\documentclass[12pt]{llncs}
%\documentclass{jktr}

\usepackage[pdftex]{hyperref}                   
\usepackage {listings}
\usepackage {mathpartir}
\usepackage{bcprules}
%\usepackage{listings}
                       
\usepackage{graphicx} 
%\usepackage[margins=2.5cm,nohead,nofoot]{geometry}
%\usepackage{geometry}
\usepackage{amsfonts}
\usepackage{amstext}
\usepackage{latexsym}
\usepackage{amssymb}
\usepackage{color}


%\include{myPreamble}
\include{qm2pi.local} 

%\ifpdf
%\usepackage[pdftex]{graphicx}
%\else
%\usepackage{graphicx}
%\fi

 % \ifpdf
%  \usepackage{pdfsync}
%  \if


%\title{Brief Article}
%\author{David F. Snyder}
%\author{L.G. Meredith}

%\address{Dept. of Math., Texas State University--San Marcos, San Marcos, TX 78666}
       
\pagestyle{empty}


\begin{document}

\lstset{language=[Objective]Caml,frame=shadowbox}

\input{qm2pi.front}

% section front matter (end)

\input{qm2pi.intro} 
 
% section introduction (end)

% \input{qm2pi.knotations} 

% section notation (end)

\input{qm2pi.process.calculi} 

% section concurrent_process_calculi_and_spatial_logics_ (end)
    
%\input{qm2pi.knots2pi} 

%\input{qm2pi.trefoil} 

%\input{qm2pi.mainthm} 

% subsection basic_interpretation (end)

%\input{qm2pi.rho.presentation} 
\subsection{The syntax and semantics of the notation system}\label{sub:the_syntax_and_semantics_of_the_notation_system} % (fold)

We now summarize a technical presentation of the calculus that
embodies our theory of dynamics. The typical presentation of such a
calculus follows the style of giving generators and relations on
them. The grammar, below, describing term constructors, freely
generates the set of processes, $\Proc$. This set is then quotiented
by a relation known as structural congruence and it is over this set
that the notion of dynamics is expressed. This presentation is
essentially that of \cite{MeredithR05} with the addition of
polyadicity and summation. For readability we have relegated some of
the technical subtleties to an appendix.

\subsubsection{Process grammar}\label{subsub:process_grammar}

\begin{mathpar}
  \inferrule* [lab=synchronization] {} {{M} \bc \pzero \;|\; x?F \;|\; x!C }
  \and
  \inferrule* [lab=abstraction] {} {{F} \bc (x)P}
  \and
  \inferrule* [lab=concretion] {} {{C} \bc \langle Q \rangle}
  \and
  \inferrule* [lab=process] {} {{P,Q} \bc M \;| \;P|Q \;|\; @{x}}
  \and
  \inferrule* [lab=name] {} {{x} \bc \quotep{P}}
\end{mathpar} 

Note that $\vec{x}$ (resp. $\vec{P}$) denotes a vector of names
(resp. processes) of length $|\vec{x}|$ (resp. $|\vec{P}|$). We adopt
the following useful abbreviations.

\begin{mathpar}
   x?(\vec{y}).P := x.(\vec{y})P \and  x\clift{\vec{P}} := x.\clift{\vec{P}}
   \and x!(y) := \lift{x}{\dropn{y}}
   \and \Pi_{i=0}^{n-1}P_i := P_0 | \ldots | P_{n-1}
\end{mathpar}

\subsubsection{Structural congruence}

\paragraph{Free and bound names and alpha-equivalence.} At the
core of structural equivalence is alpha-equivalence which identifies
process that are the same up to a change of variable. Formally, we
recognize the distinction between free and bound names. The free names
of a process, $\freenames{P}$, may be calculated recursively as
follows:

\begin{mathpar}
\freenames{\pzero} := \emptyset
  \and \\
  \freenames{x?(y).P} := \{ x \} \cup (\freenames{P} \setminus \{ y \})
  \and 
  \freenames{x!\langle P \rangle} := \{ x \} \cup \{ P \} 
  \and \\
  \freenames{P|Q} := \freenames{P} \cup \freenames{Q}
  \and \\
  \freenames{@{x}} := \{ x \}
\end{mathpar}

$\pi$
$\quotep{\pi}$

$\freenames{-} : \pi \to \mathcal{P}(\quotep{\pi})$

\begin{eqnarray*}
  \freenames{\pzero} & := & \emptyset \\
  \freenames{x?(y).P} & := & \{ x \} \cup (\freenames{P} \setminus \{ y \}) \\
  \freenames{x!\langle P \rangle} & := & \{ x \} \cup \{ P \} \\
  \freenames{P|Q} & := & \freenames{P} \cup \freenames{Q} \\
  \freenames{\dropn{x}} & := & \{ x \}
\end{eqnarray*}

The bound names of a process, $\boundnames{P}$, are those names occurring in $P$
that are not free. For example, in $x?(y).0$, the name $x$ is free, while $y$ is bound.

\begin{mathpar}
  \inferrule* [lab=monoidal-laws] {} { P|Q \equiv Q|P \and P|0 \equiv P \and P|(Q|R) \equiv (P|Q)|R }
\end{mathpar}

\begin{mathpar}
  \inferrule* [lab=alpha-equivalence] {} { (x)P \equiv (y)P\{y/x\} \and y \not\in \freenames{P} }
\end{mathpar}

\begin{definition}
Then two processes, $P,Q$, are alpha-equivalent if $P = Q\{\vec{y}/\vec{x}\}$ for
some $\vec{x} \in \boundnames{Q},\vec{y} \in \boundnames{P}$, where $Q\{\vec{y}/\vec{x}\}$
denotes the capture-avoiding substitution of $\vec{y}$ for $\vec{x}$ in $Q$.
\end{definition}

\begin{definition}
  The {\em structural congruence} \cite{SangiorgiWalker} , $\equiv$,
  between processes is the least congruence containing
  alpha-equivalence, satisfying the abelian monoid laws
  (associativity, commutativity and $\pzero$ as identity) for parallel
  composition $|$ and for summation $+$.
\end{definition}

\subsection{Name equivalence}

We take name equivalence, written $\nameeq$, to be the smallest
equivalence relation generated by the following rules.

\begin{mathpar}
\inferrule*[lab=Quote-drop]
{ }
{ \quotep{@{x}} \nameeq x }

\inferrule*[lab=Struct-equiv]
{ P \scong Q }
{ \quotep{P} \nameeq \quotep{Q} }
\end{mathpar}

The astute reader will have noticed that the mutual recursion of names
and processes imposes a mutual recursion on alpha-equivalence and
structural equivalence via name-equivalence. Fortunately, all of this
works out pleasantly and we may calculate in the natural way, free of
concern. The reader interested in the details is referred to the
appendix \ref{appendix:rho_details}.

\subsection{Substitution}

We use $\Proc$ for the set of processes, $\QProc$ for the set of
names, and $\id{\{}\vec{y} / \vec{x} \id{\}}$ to denote partial maps,
$s : \QProc \rightarrow \QProc$. A map, $s$ lifts, uniquely, to a map
on process terms, $\widehat{s} : \Proc \rightarrow \Proc$ by the
following equations.

\begin{mathpar}
  (0) \psubstp{Q}{P} := 0 \\
  (R \juxtap S) \psubstp{Q}{P}
  :=    
  (R)\psubstp{Q}{P} \juxtap (S) \psubstp{Q}{P} \\
  (x?(y).R) \psubstp{Q}{P}    
  :=    
  (x)\substp{Q}{P} (z)\concat( (R \psubstn{z}{y}) \psubstp{Q}{P} ) \\
  (\lift{x}{R}) \psubstp{Q}{P}  
  :=
  \lift{(x)\substp{Q}{P}}{ R \psubstp{Q}{P} } \\
%   (\dropn{x})  \psubstp{Q}{P}       
%   := 
%   \left\{ 
%     \begin{array}{ccc} 
%       \dropn{\quotep{Q}} & & x \nameeq \quotep{P} \\
%       \dropn{x} & & otherwise \\
%     \end{array}
%   \right. 
  (\dropn{x})  \psubstp{Q}{P}       
  := 
  \left\{ 
    \begin{array}{ccc} 
      Q & & x \nameeq \quotep{P} \\
      \dropn{x} & & otherwise \\
    \end{array}
  \right.
\end{mathpar}
 

where

\begin{eqnarray}
  (x)\id{\{} \lpquote Q \rpquote / \lpquote P \rpquote \id{\}}            = 
  \left\{ 
    \begin{array}{ccc}
      \lpquote Q \rpquote & & x \nameeq \lpquote P \rpquote \\
      x & & otherwise \\
    \end{array}
  \right. \nonumber
\end{eqnarray}

and $z$ is chosen distinct from $\quotep{P}$, $\quotep{Q}$, the free
names in $Q$, and all the names in $R$. Our $\alpha$-equivalence will
be built in the standard way from this substitution.

\begin{remark}\label{rem:no_self_referential_names}
  One consequence of these definitions is that $\forall P. \quotep{P}
  \not\in \freenames{P}$.
\end{remark}

\subsection{ Dynamic quote: an example }

Anticipating something of what's to come, consider applying the
substitution, $\widehat{\id{\{}u / z \id{\}}}$, to the following pair
of processes, $\lift{w}{y!(z)}$ and $w[ \lpquote y!(z) \rpquote ]$.

\begin{eqnarray}
	\lift{w}{y!(z)}\widehat{\id{\{}u / z \id{\}}}
		& = &
		\lift{w}{y!(u)} \nonumber\\
	w[ \lpquote y!(z) \rpquote ] \widehat{ \id{\{}u / z \id{\}} }
		& = &
		w[ \lpquote y!(z) \rpquote ] \nonumber
\end{eqnarray}

Because the body of the process between quotes is impervious to
substitution, we get radically different answers. In fact, by
examining the first process in an input context,
e.g. $x?(z).\lift{w}{y!(z)}$, we see that the process under the lift
operator may be shaped by prefixed inputs binding a name inside it. In
this sense, the lift operator will be seen as a way to dynamically
construct processes before reifying them as names.

Finally equipped with these standard features we can present the
dynamics of the calculus.

\subsubsection{Operational semantics} 

Finally, we introduce the computational dynamics. What marks these
algebras as distinct from other more traditionally studied algebraic
structures, e.g. vector spaces or polynomial rings, is the manner in
which dynamics is captured. In traditional structures, dynamics is typically
expressed through morphisms between such structures, as in linear maps
between vector spaces or morphisms between rings. In algebras
associated with the semantics of computation, the dynamics is
expressed as part of the algebraic structure itself, through a
reduction reduction relation typically denoted by $\red$. Below, we
give a recursive presentation of this relation for the calculus used
in the encoding.

$\red \subseteq \pi \times \pi$
$\red : \pi \to \mathcal{P}(\pi)$

\begin{mathpar}
  \inferrule* [lab=Comm] { \textsf{match}( x_{src}, x_{trgt} ) } { x_{trgt}?(y)P \; | \; x_{src}!\langle {Q} \rangle \red P\{\quotep{Q}/y}\} }
  \and \\
  \inferrule* [lab=Par] {{P} \red {P}'} {{{P} | {Q}} \red {{P}' | {Q}}}
  \and
  \inferrule* [lab=Equiv]{{{P} \scong {P}'} \andalso {{P}' \red {Q}'} \andalso {{Q}' \scong {Q}}}{{P} \red {Q}}
\end{mathpar}

\begin{eqnarray*}
  match_{\equiv} (\quotep{P},\quotep{Q}) & := & P \equiv Q \\
  match_{\dagger}(\quotep{P},\quotep{Q}) & := & \forall R. P|Q \red^{*} R => R \red^{*} 0 \\
  match_{K}(\quotep{P},\quotep{Q}) & := & K \mbox{ for some context } K
\end{eqnarray*}

$u?(x)P | u!\langle Q \rangle \red P\{\quotep{Q}/x\}$

%We write $\wred$ for $\red^*$, and $P\red$ if $\exists Q $ such that $ P \red Q$.
We write $P\red$ if $\exists Q $ such that $ P \red Q$ and $P\not\red$, otherwise.

\section{Replication}

As mentioned before, it is known that replication (and hence
recursion) can be implemented in a higher-order process algebra
\cite{SangiorgiWalker}. As our first example of calculation with the
machinery thus far presented we give the construction explicitly in
the {\rhoc}.

\begin{eqnarray}
	D_{x} & := & \prefix{x}{y}{(\binpar{\outputp{x}{y}}{@{y}})} \nonumber\\
	\bangp_{x}{P} & := & \binpar{{x}!\langle{\binpar{D_{x}}{P}}\rangle}{D_{x}} \nonumber
\end{eqnarray}

\begin{eqnarray}
	\bangp_{x}{P} & & \nonumber\\
	=
	& {x}!\langle{(\prefix{x}{y}{(\outputp{x}{y} | @{y})) | P}}\rangle 
	      | \prefix{x}{y}{(\outputp{x}{y} | @{y})} & \nonumber\\
	\red
	& (\outputp{x}{y} | @{y})\substn{\quotep{(\prefix{x}{y}{(@{y} | \outputp{x}{y})) | P}}}{y} & \nonumber\\
	=
	& \outputp{x}{\quotep{(\prefix{x}{y}{(\outputp{x}{y} | @{y})) | P}}}
	  | {(\prefix{x}{y}{(\outputp{x}{y} | @{y})) | P}} & \nonumber\\
	\red
	& \ldots & \nonumber\\
	\red^*
	& P | P | \ldots & \nonumber
\end{eqnarray}

Of course, this encoding, as an implementation, runs away, unfolding
$\bangp{P}$ eagerly. A lazier and more implementable replication
operator, restricted to input-guarded processes, may be obtained as follows.

\begin{eqnarray}
\bangp{\prefix{u}{v}{P}} 
	:= 
	\binpar{\lift{x}{\prefix{u}{v}{(\binpar{D(x)}{P})}}}{D(x)} \nonumber
\end{eqnarray}

\begin{remark}
  Note that the lazier definition still does not deal with summation
  or mixed summation (i.e. sums over input and output). The reader is
  invited to construct definitions of replication that deal with these
  features. 

  Further, the definitions are parameterized in a name, $x$. Can you,
  gentle reader, make a definition that eliminates this parameter and
  guarantees no accidental interaction between the replication
  machinery and the process being replicated -- i.e. no accidental
  sharing of names used by the process to get its work done and the
  name(s) used by the replication to effect copying. This latter
  revision of the definition of replication is crucial to obtaining
  the expected identity $!!P \sim !P$.
\end{remark}

\begin{remark}\label{rem:paradoxical_combinator}
  The reader familiar with the lambda calculus will have noticed the
  similarity between $D$ and the paradoxical combinator.

  [Ed. note: the existence of this seems to suggest we have to be more
  restrictive on the set of processes and names we admit if we are to
  support no-cloning.]
\end{remark}

\subsubsection{Bisimulation}

The computational dynamics gives rise to another kind of equivalence,
the equivalence of computational behavior. As previously mentioned
this is typically captured \emph{via} some form of bisimulation.

% The notion we use in this paper is weak barbed bisimulation
% \cite{milner91polyadicpi}.

The notion we use in this paper is derived from weak barbed
bisimulation \cite{milner91polyadicpi}. 

\begin{definition}
An \emph{observation relation}, $\downarrow_{\mathcal N}$, over a set
of names, $\mathcal N$, is the smallest relation satisfying the rules
below.

\infrule[Out-barb]{y \in {\mathcal N}, \; x \nameeq y}
		  {\outputp{x}{v} \downarrow_{\mathcal N} x}
\infrule[Par-barb]{\mbox{$P\downarrow_{\mathcal N} x$ or $Q\downarrow_{\mathcal N} x$}}
		  {\binpar{P}{Q} \downarrow_{\mathcal N} x}

We write $P \Downarrow_{\mathcal N} x$ if there is $Q$ such that 
$P \wred Q$ and $Q \downarrow_{\mathcal N} x$.
\end{definition}

\begin{definition}
%\label{def.bbisim}
An  ${\mathcal N}$-\emph{barbed bisimulation} over a set of names, ${\mathcal N}$, is a symmetric binary relation 
${\mathcal S}_{\mathcal N}$ between agents such that $P\rel{S}_{\mathcal N}Q$ implies:
\begin{enumerate}
\item If $P \red P'$ then $Q \wred Q'$ and $P'\rel{S}_{\mathcal N} Q'$.
\item If $P\downarrow_{\mathcal N} x$, then $Q\Downarrow_{\mathcal N} x$.
\end{enumerate}
$P$ is ${\mathcal N}$-barbed bisimilar to $Q$, written
$P \wbbisim_{\mathcal N} Q$, if $P \rel{S}_{\mathcal N} Q$ for some ${\mathcal N}$-barbed bisimulation ${\mathcal S}_{\mathcal N}$.
\end{definition}

$\mathcal{R} \subseteq \pi \times \pi$

$P \mathcal{R} Q => \forall P'. P \red P' \Rightarrow \exists Q'. Q \red Q', P' \mathcal{R} Q'$

$P \vdash x \Rightarrow Q \vdash x$

\begin{mathpar}
  \inferrule*[lab=Out-barb]{x \nameeq y}{{y}!\langle{Q}\rangle \vdash x}
  \and
  \inferrule*[lab=Par-barb]{\mbox{$P\vdash x$ or $Q\vdash x$}}{\binpar{P}{Q} \vdash x}
\end{mathpar}

\subsubsection{Contexts}

One of the principle advantages of computational calculi like the
$\pi$-calculus is a well-defined notion of context,
contextual-equivalence and a correlation between
contextual-equivalence and notions of bisimulation. The notion of
context allows the decomposition of a process into (sub-)process and
its syntactic environment, its context. Thus, a context may be
thought of as a process with a ``hole'' (written $\Box$) in it. The
application of a context $M$ to a process $P$, written $M[P]$, is
tantamount to filling the hole in $M$ with $P$. In this paper we do
not need the full weight of this theory, but do make use of the notion
of context in the proof the main theorem. 

\begin{mathpar}
  \inferrule* [lab=summation] {} {{M_{M},M_{N}} \bc \Box \;|\; x.M_{A} \;|\; M_{M}+M_{N}}
  \and
  \inferrule* [lab=agent] {} {{M_{A}} \bc (\vec{x})M_{P} \;| \; \clift{P_0,\ldots,M_{P},\ldots,P_N}}
  \and \\
  \inferrule* [lab=process] {} {{M_{P}} \bc M_{N} \;| \;P|M_{P} }
\end{mathpar} 

\begin{mathpar}
  \inferrule* [lab=sychronization] {} {M_{N} \bc \Box \;|\; x?M_{F} \;|\; x!M_{C}}
  \and
  \inferrule* [lab=abstraction] {} {{M_{F}} \bc (x)M_{P} }
  \and
  \inferrule* [lab=concretion] {} {{M_{C}} \bc \langle M_{P} \rangle }
  \and \\
  \inferrule* [lab=process] {} {{M_{P}} \bc M_{N} \;| \;P|M_{P} }
\end{mathpar}

\begin{definition}[contextual application] Given a context $M$, and
  process $P$, we define the \emph{contextual application}, $M[P] :=
  M\{P/\Box\}$. That is, the contextual application of M to P is the
  substitution of $P$ for $\Box$ in $M$.
\end{definition}

$\meaningof{-} : L \to \mathcal{P}(\pi)$

\begin{mathpar}
  \inferrule* [lab=collection] {} {\meaningof{true} = \pi, \and \meaningof{~E} = \pi \setminus \meaningof{E}, \and \meaningof{E_{1} \& E_{2}} = \meaningof{E_{1}} \cap \meaningof{E_{2}}}
\end{mathpar}

\begin{mathpar}
  \inferrule* [lab=structure] {} {\meaningof{0} = \{ P \in \pi | P \equiv 0 \}, \and \\ \meaningof{E_1 | E_2} = \{ P \in \pi | P \equiv P_{1} | P_{2}, P_{1} \in \meaningof{E_{1}}, P_{2} \in \meaningof{E_2}\} }
\end{mathpar}

\begin{mathpar}
 \inferrule* [lab=behavior] {} {\meaningof{\langle a?b \rangle E} = \{ P \in \pi | P \equiv Q | u?(y)P', \\ \and \\\\ \and \\ \;\;\; u \in \meaningof{a}, \forall z.P'\{z/y\} \in \meaningof{E\{z/b\}}\}, \and \\ \meaningof{a!E} = \{ P \in \pi | P \equiv Q | x!\langle P' \rangle, x \in \meaningof{a} P' \in \meaningof{E}\} }
\end{mathpar}

\begin{mathpar}
 \inferrule* [lab=nominal] {} {\meaningof{\quotep{E}} = \{ \quotep{P} \in \quotep{\pi} | P \in \meaningof{E} \}, \and \meaningof{\quotep{P}} = \{ \quotep{Q} \in \quotep{\pi} | P \equiv Q \} \and \\ \meaningof{@\quotep{E}} = \{ P \in \pi | P \equiv @x, x \in \meaningof{E} \}}
\end{mathpar}

\begin{eqnarray*}
  \\
  \meaningof{-} : TS \to ST
\end{eqnarray*}

\begin{eqnarray*}
  \\
  L : TS \to ST
\end{eqnarray*}

\begin{eqnarray*}
  \\
  P \models E \iff P \in \meaningof{E}
\end{eqnarray*}

\begin{eqnarray*}
  P \approx_{L} Q \iff \forall E \in L. P \models E \iff Q \models E
\end{eqnarray*}

\begin{eqnarray*}
  P \approx_{K} Q
\end{eqnarray*}

\begin{eqnarray*}
  P \approx Q
\end{eqnarray*}

$\approx_{K} = \approx = \approx_{L}$

\subsubsection{Contextual duality}

Note that contexts extend the quotation operation to a family of
operations from processes to names. Given a context, $M$, we can
define a \emph{nominal context}, $\quotep{M}$ by $\quotep{M}[P] :=
\quotep{M[P]}$. To foreshadow what is to come we observe that these
operations enjoy a duality with processes very much like the duality
between vectors and maps from vectors to scalars.

Further, because the calculus is essentially higher-order, we have a
correspondence between contexts and processes. More specifically,
given a name $x$ and a context $M$ we can construct $M^{*}_{x}$ such
that 

\begin{mathpar}
  M^{*}_{x} | \lift{x}{P} \red M[P]
\end{mathpar}

namely,

\begin{mathpar}
  M^{*}_{x} := x?(u).M[\dropn{u}]
\end{mathpar}

The dependence of $M^{*}_{x}$ on a name makes it an abstraction, 

\begin{mathpar}
  M^{*} := (x)x?(u).M[\dropn{u}]
\end{mathpar}

\subsection{Additional notation}

It will sometimes be convenient to denote the process a name
quotes. We already have the notation $x = \quotep{P}$, but it will be
convenient to introduce an alternate notation, $\procn{x}$, when we
want to emphasize the connection to the use of the name. Note that, by
virtue of name equivalence, $\quotep{\procn{x}} \nameeq x$; so, the
notation is consistent with previous definitions.

Further, because names have structure it is possible to effect
substitutions on the basis of that structure. This means we need to
upgrade our notation for substitutions, which we accomplish by
adapting comprehension notation. Thus,

\begin{mathpar}
  P\{ y / x : x \in S \}
\end{mathpar}

is interpreted to mean the process derived from P by replacing (in a
capture-avoiding manner) each occurrence of $x$ in $S$ by $y$. For example,

\begin{mathpar}
  P\{ \quotep{\procn{x}|\procn{x}} / x : x \in \freenames{P} \}
\end{mathpar}

will replace each (occurrence) of a free name $x$ in $P$ by
$\quotep{\procn{x}|\procn{x}}$.

Also, we will avail ourselves of the notation $x^{L}$ and $x^{R}$ to
denote injections of a name into disjoint copies of the name
space. There are numerous ways to accomplish this. One example can be
found in \cite{MeredithR05}. This notation overloads to vectors of
names: $\vec{x}^{\pi} := (x_{i}^{\pi} \; : \; 0 \leq i < |\vec{x}| )$ where $\pi \in \{L,R\}$.

We also use $P^{\Box} := P|\Box$.

In \cite{MeredithR05} an interpretation of the new operator is
given. It turns out that there are several possible interpretations
all enjoying the requisite algebraic properties of the operator (see
\cite{milner91polyadicpi}). We will therefore make liberal use of
$(\nu\; \vec{x})P$.

% subsection the_syntax_and_semantics_of_the_notation_system (end)   

\input{qm2pi.qmops} 

\input{qm2pi.sterngerlach} 

\input{qm2pi.metric} 

% section concurrent_process_calculi (end)

%\input{qm2pi.proofsketch}

% section proof sketch (end)

%\input{qm2pi.slviaknots} 

% section spatial logic via knots (end)

\input{qm2pi.conclusion}

% section conclusion (end)

%\input{qm2pi.dtcodes} 

% section wiring algorithm (end)

\input{qm2pi.ack} 

% section acknowledgments (end)

\newpage


\bibliographystyle{plain}   
\bibliography{../../biblios/main.bib}

\input{qm2pi.rhodetails}

\end{document}

 

% section acknowledgments (end)

\newpage


\bibliographystyle{plain}   
\bibliography{../../biblios/main.bib}

\documentclass[12pt]{llncs}
%\documentclass{jktr}

\usepackage[pdftex]{hyperref}                   
\usepackage {listings}
\usepackage {mathpartir}
\usepackage{bcprules}
%\usepackage{listings}
                       
\usepackage{graphicx} 
%\usepackage[margins=2.5cm,nohead,nofoot]{geometry}
%\usepackage{geometry}
\usepackage{amsfonts}
\usepackage{amstext}
\usepackage{latexsym}
\usepackage{amssymb}
\usepackage{color}


%\include{myPreamble}
\include{qm2pi.local} 

%\ifpdf
%\usepackage[pdftex]{graphicx}
%\else
%\usepackage{graphicx}
%\fi

 % \ifpdf
%  \usepackage{pdfsync}
%  \if


%\title{Brief Article}
%\author{David F. Snyder}
%\author{L.G. Meredith}

%\address{Dept. of Math., Texas State University--San Marcos, San Marcos, TX 78666}
       
\pagestyle{empty}


\begin{document}

\lstset{language=[Objective]Caml,frame=shadowbox}

\input{qm2pi.front}

% section front matter (end)

\input{qm2pi.intro} 
 
% section introduction (end)

% \input{qm2pi.knotations} 

% section notation (end)

\input{qm2pi.process.calculi} 

% section concurrent_process_calculi_and_spatial_logics_ (end)
    
%\input{qm2pi.knots2pi} 

%\input{qm2pi.trefoil} 

%\input{qm2pi.mainthm} 

% subsection basic_interpretation (end)

%\input{qm2pi.rho.presentation} 
\subsection{The syntax and semantics of the notation system}\label{sub:the_syntax_and_semantics_of_the_notation_system} % (fold)

We now summarize a technical presentation of the calculus that
embodies our theory of dynamics. The typical presentation of such a
calculus follows the style of giving generators and relations on
them. The grammar, below, describing term constructors, freely
generates the set of processes, $\Proc$. This set is then quotiented
by a relation known as structural congruence and it is over this set
that the notion of dynamics is expressed. This presentation is
essentially that of \cite{MeredithR05} with the addition of
polyadicity and summation. For readability we have relegated some of
the technical subtleties to an appendix.

\subsubsection{Process grammar}\label{subsub:process_grammar}

\begin{mathpar}
  \inferrule* [lab=synchronization] {} {{M} \bc \pzero \;|\; x?F \;|\; x!C }
  \and
  \inferrule* [lab=abstraction] {} {{F} \bc (x)P}
  \and
  \inferrule* [lab=concretion] {} {{C} \bc \langle Q \rangle}
  \and
  \inferrule* [lab=process] {} {{P,Q} \bc M \;| \;P|Q \;|\; @{x}}
  \and
  \inferrule* [lab=name] {} {{x} \bc \quotep{P}}
\end{mathpar} 

Note that $\vec{x}$ (resp. $\vec{P}$) denotes a vector of names
(resp. processes) of length $|\vec{x}|$ (resp. $|\vec{P}|$). We adopt
the following useful abbreviations.

\begin{mathpar}
   x?(\vec{y}).P := x.(\vec{y})P \and  x\clift{\vec{P}} := x.\clift{\vec{P}}
   \and x!(y) := \lift{x}{\dropn{y}}
   \and \Pi_{i=0}^{n-1}P_i := P_0 | \ldots | P_{n-1}
\end{mathpar}

\subsubsection{Structural congruence}

\paragraph{Free and bound names and alpha-equivalence.} At the
core of structural equivalence is alpha-equivalence which identifies
process that are the same up to a change of variable. Formally, we
recognize the distinction between free and bound names. The free names
of a process, $\freenames{P}$, may be calculated recursively as
follows:

\begin{mathpar}
\freenames{\pzero} := \emptyset
  \and \\
  \freenames{x?(y).P} := \{ x \} \cup (\freenames{P} \setminus \{ y \})
  \and 
  \freenames{x!\langle P \rangle} := \{ x \} \cup \{ P \} 
  \and \\
  \freenames{P|Q} := \freenames{P} \cup \freenames{Q}
  \and \\
  \freenames{@{x}} := \{ x \}
\end{mathpar}

$\pi$
$\quotep{\pi}$

$\freenames{-} : \pi \to \mathcal{P}(\quotep{\pi})$

\begin{eqnarray*}
  \freenames{\pzero} & := & \emptyset \\
  \freenames{x?(y).P} & := & \{ x \} \cup (\freenames{P} \setminus \{ y \}) \\
  \freenames{x!\langle P \rangle} & := & \{ x \} \cup \{ P \} \\
  \freenames{P|Q} & := & \freenames{P} \cup \freenames{Q} \\
  \freenames{\dropn{x}} & := & \{ x \}
\end{eqnarray*}

The bound names of a process, $\boundnames{P}$, are those names occurring in $P$
that are not free. For example, in $x?(y).0$, the name $x$ is free, while $y$ is bound.

\begin{mathpar}
  \inferrule* [lab=monoidal-laws] {} { P|Q \equiv Q|P \and P|0 \equiv P \and P|(Q|R) \equiv (P|Q)|R }
\end{mathpar}

\begin{mathpar}
  \inferrule* [lab=alpha-equivalence] {} { (x)P \equiv (y)P\{y/x\} \and y \not\in \freenames{P} }
\end{mathpar}

\begin{definition}
Then two processes, $P,Q$, are alpha-equivalent if $P = Q\{\vec{y}/\vec{x}\}$ for
some $\vec{x} \in \boundnames{Q},\vec{y} \in \boundnames{P}$, where $Q\{\vec{y}/\vec{x}\}$
denotes the capture-avoiding substitution of $\vec{y}$ for $\vec{x}$ in $Q$.
\end{definition}

\begin{definition}
  The {\em structural congruence} \cite{SangiorgiWalker} , $\equiv$,
  between processes is the least congruence containing
  alpha-equivalence, satisfying the abelian monoid laws
  (associativity, commutativity and $\pzero$ as identity) for parallel
  composition $|$ and for summation $+$.
\end{definition}

\subsection{Name equivalence}

We take name equivalence, written $\nameeq$, to be the smallest
equivalence relation generated by the following rules.

\begin{mathpar}
\inferrule*[lab=Quote-drop]
{ }
{ \quotep{@{x}} \nameeq x }

\inferrule*[lab=Struct-equiv]
{ P \scong Q }
{ \quotep{P} \nameeq \quotep{Q} }
\end{mathpar}

The astute reader will have noticed that the mutual recursion of names
and processes imposes a mutual recursion on alpha-equivalence and
structural equivalence via name-equivalence. Fortunately, all of this
works out pleasantly and we may calculate in the natural way, free of
concern. The reader interested in the details is referred to the
appendix \ref{appendix:rho_details}.

\subsection{Substitution}

We use $\Proc$ for the set of processes, $\QProc$ for the set of
names, and $\id{\{}\vec{y} / \vec{x} \id{\}}$ to denote partial maps,
$s : \QProc \rightarrow \QProc$. A map, $s$ lifts, uniquely, to a map
on process terms, $\widehat{s} : \Proc \rightarrow \Proc$ by the
following equations.

\begin{mathpar}
  (0) \psubstp{Q}{P} := 0 \\
  (R \juxtap S) \psubstp{Q}{P}
  :=    
  (R)\psubstp{Q}{P} \juxtap (S) \psubstp{Q}{P} \\
  (x?(y).R) \psubstp{Q}{P}    
  :=    
  (x)\substp{Q}{P} (z)\concat( (R \psubstn{z}{y}) \psubstp{Q}{P} ) \\
  (\lift{x}{R}) \psubstp{Q}{P}  
  :=
  \lift{(x)\substp{Q}{P}}{ R \psubstp{Q}{P} } \\
%   (\dropn{x})  \psubstp{Q}{P}       
%   := 
%   \left\{ 
%     \begin{array}{ccc} 
%       \dropn{\quotep{Q}} & & x \nameeq \quotep{P} \\
%       \dropn{x} & & otherwise \\
%     \end{array}
%   \right. 
  (\dropn{x})  \psubstp{Q}{P}       
  := 
  \left\{ 
    \begin{array}{ccc} 
      Q & & x \nameeq \quotep{P} \\
      \dropn{x} & & otherwise \\
    \end{array}
  \right.
\end{mathpar}
 

where

\begin{eqnarray}
  (x)\id{\{} \lpquote Q \rpquote / \lpquote P \rpquote \id{\}}            = 
  \left\{ 
    \begin{array}{ccc}
      \lpquote Q \rpquote & & x \nameeq \lpquote P \rpquote \\
      x & & otherwise \\
    \end{array}
  \right. \nonumber
\end{eqnarray}

and $z$ is chosen distinct from $\quotep{P}$, $\quotep{Q}$, the free
names in $Q$, and all the names in $R$. Our $\alpha$-equivalence will
be built in the standard way from this substitution.

\begin{remark}\label{rem:no_self_referential_names}
  One consequence of these definitions is that $\forall P. \quotep{P}
  \not\in \freenames{P}$.
\end{remark}

\subsection{ Dynamic quote: an example }

Anticipating something of what's to come, consider applying the
substitution, $\widehat{\id{\{}u / z \id{\}}}$, to the following pair
of processes, $\lift{w}{y!(z)}$ and $w[ \lpquote y!(z) \rpquote ]$.

\begin{eqnarray}
	\lift{w}{y!(z)}\widehat{\id{\{}u / z \id{\}}}
		& = &
		\lift{w}{y!(u)} \nonumber\\
	w[ \lpquote y!(z) \rpquote ] \widehat{ \id{\{}u / z \id{\}} }
		& = &
		w[ \lpquote y!(z) \rpquote ] \nonumber
\end{eqnarray}

Because the body of the process between quotes is impervious to
substitution, we get radically different answers. In fact, by
examining the first process in an input context,
e.g. $x?(z).\lift{w}{y!(z)}$, we see that the process under the lift
operator may be shaped by prefixed inputs binding a name inside it. In
this sense, the lift operator will be seen as a way to dynamically
construct processes before reifying them as names.

Finally equipped with these standard features we can present the
dynamics of the calculus.

\subsubsection{Operational semantics} 

Finally, we introduce the computational dynamics. What marks these
algebras as distinct from other more traditionally studied algebraic
structures, e.g. vector spaces or polynomial rings, is the manner in
which dynamics is captured. In traditional structures, dynamics is typically
expressed through morphisms between such structures, as in linear maps
between vector spaces or morphisms between rings. In algebras
associated with the semantics of computation, the dynamics is
expressed as part of the algebraic structure itself, through a
reduction reduction relation typically denoted by $\red$. Below, we
give a recursive presentation of this relation for the calculus used
in the encoding.

$\red \subseteq \pi \times \pi$
$\red : \pi \to \mathcal{P}(\pi)$

\begin{mathpar}
  \inferrule* [lab=Comm] { \textsf{match}( x_{src}, x_{trgt} ) } { x_{trgt}?(y)P \; | \; x_{src}!\langle {Q} \rangle \red P\{\quotep{Q}/y}\} }
  \and \\
  \inferrule* [lab=Par] {{P} \red {P}'} {{{P} | {Q}} \red {{P}' | {Q}}}
  \and
  \inferrule* [lab=Equiv]{{{P} \scong {P}'} \andalso {{P}' \red {Q}'} \andalso {{Q}' \scong {Q}}}{{P} \red {Q}}
\end{mathpar}

\begin{eqnarray*}
  match_{\equiv} (\quotep{P},\quotep{Q}) & := & P \equiv Q \\
  match_{\dagger}(\quotep{P},\quotep{Q}) & := & \forall R. P|Q \red^{*} R => R \red^{*} 0 \\
  match_{K}(\quotep{P},\quotep{Q}) & := & K \mbox{ for some context } K
\end{eqnarray*}

$u?(x)P | u!\langle Q \rangle \red P\{\quotep{Q}/x\}$

%We write $\wred$ for $\red^*$, and $P\red$ if $\exists Q $ such that $ P \red Q$.
We write $P\red$ if $\exists Q $ such that $ P \red Q$ and $P\not\red$, otherwise.

\section{Replication}

As mentioned before, it is known that replication (and hence
recursion) can be implemented in a higher-order process algebra
\cite{SangiorgiWalker}. As our first example of calculation with the
machinery thus far presented we give the construction explicitly in
the {\rhoc}.

\begin{eqnarray}
	D_{x} & := & \prefix{x}{y}{(\binpar{\outputp{x}{y}}{@{y}})} \nonumber\\
	\bangp_{x}{P} & := & \binpar{{x}!\langle{\binpar{D_{x}}{P}}\rangle}{D_{x}} \nonumber
\end{eqnarray}

\begin{eqnarray}
	\bangp_{x}{P} & & \nonumber\\
	=
	& {x}!\langle{(\prefix{x}{y}{(\outputp{x}{y} | @{y})) | P}}\rangle 
	      | \prefix{x}{y}{(\outputp{x}{y} | @{y})} & \nonumber\\
	\red
	& (\outputp{x}{y} | @{y})\substn{\quotep{(\prefix{x}{y}{(@{y} | \outputp{x}{y})) | P}}}{y} & \nonumber\\
	=
	& \outputp{x}{\quotep{(\prefix{x}{y}{(\outputp{x}{y} | @{y})) | P}}}
	  | {(\prefix{x}{y}{(\outputp{x}{y} | @{y})) | P}} & \nonumber\\
	\red
	& \ldots & \nonumber\\
	\red^*
	& P | P | \ldots & \nonumber
\end{eqnarray}

Of course, this encoding, as an implementation, runs away, unfolding
$\bangp{P}$ eagerly. A lazier and more implementable replication
operator, restricted to input-guarded processes, may be obtained as follows.

\begin{eqnarray}
\bangp{\prefix{u}{v}{P}} 
	:= 
	\binpar{\lift{x}{\prefix{u}{v}{(\binpar{D(x)}{P})}}}{D(x)} \nonumber
\end{eqnarray}

\begin{remark}
  Note that the lazier definition still does not deal with summation
  or mixed summation (i.e. sums over input and output). The reader is
  invited to construct definitions of replication that deal with these
  features. 

  Further, the definitions are parameterized in a name, $x$. Can you,
  gentle reader, make a definition that eliminates this parameter and
  guarantees no accidental interaction between the replication
  machinery and the process being replicated -- i.e. no accidental
  sharing of names used by the process to get its work done and the
  name(s) used by the replication to effect copying. This latter
  revision of the definition of replication is crucial to obtaining
  the expected identity $!!P \sim !P$.
\end{remark}

\begin{remark}\label{rem:paradoxical_combinator}
  The reader familiar with the lambda calculus will have noticed the
  similarity between $D$ and the paradoxical combinator.

  [Ed. note: the existence of this seems to suggest we have to be more
  restrictive on the set of processes and names we admit if we are to
  support no-cloning.]
\end{remark}

\subsubsection{Bisimulation}

The computational dynamics gives rise to another kind of equivalence,
the equivalence of computational behavior. As previously mentioned
this is typically captured \emph{via} some form of bisimulation.

% The notion we use in this paper is weak barbed bisimulation
% \cite{milner91polyadicpi}.

The notion we use in this paper is derived from weak barbed
bisimulation \cite{milner91polyadicpi}. 

\begin{definition}
An \emph{observation relation}, $\downarrow_{\mathcal N}$, over a set
of names, $\mathcal N$, is the smallest relation satisfying the rules
below.

\infrule[Out-barb]{y \in {\mathcal N}, \; x \nameeq y}
		  {\outputp{x}{v} \downarrow_{\mathcal N} x}
\infrule[Par-barb]{\mbox{$P\downarrow_{\mathcal N} x$ or $Q\downarrow_{\mathcal N} x$}}
		  {\binpar{P}{Q} \downarrow_{\mathcal N} x}

We write $P \Downarrow_{\mathcal N} x$ if there is $Q$ such that 
$P \wred Q$ and $Q \downarrow_{\mathcal N} x$.
\end{definition}

\begin{definition}
%\label{def.bbisim}
An  ${\mathcal N}$-\emph{barbed bisimulation} over a set of names, ${\mathcal N}$, is a symmetric binary relation 
${\mathcal S}_{\mathcal N}$ between agents such that $P\rel{S}_{\mathcal N}Q$ implies:
\begin{enumerate}
\item If $P \red P'$ then $Q \wred Q'$ and $P'\rel{S}_{\mathcal N} Q'$.
\item If $P\downarrow_{\mathcal N} x$, then $Q\Downarrow_{\mathcal N} x$.
\end{enumerate}
$P$ is ${\mathcal N}$-barbed bisimilar to $Q$, written
$P \wbbisim_{\mathcal N} Q$, if $P \rel{S}_{\mathcal N} Q$ for some ${\mathcal N}$-barbed bisimulation ${\mathcal S}_{\mathcal N}$.
\end{definition}

$\mathcal{R} \subseteq \pi \times \pi$

$P \mathcal{R} Q => \forall P'. P \red P' \Rightarrow \exists Q'. Q \red Q', P' \mathcal{R} Q'$

$P \vdash x \Rightarrow Q \vdash x$

\begin{mathpar}
  \inferrule*[lab=Out-barb]{x \nameeq y}{{y}!\langle{Q}\rangle \vdash x}
  \and
  \inferrule*[lab=Par-barb]{\mbox{$P\vdash x$ or $Q\vdash x$}}{\binpar{P}{Q} \vdash x}
\end{mathpar}

\subsubsection{Contexts}

One of the principle advantages of computational calculi like the
$\pi$-calculus is a well-defined notion of context,
contextual-equivalence and a correlation between
contextual-equivalence and notions of bisimulation. The notion of
context allows the decomposition of a process into (sub-)process and
its syntactic environment, its context. Thus, a context may be
thought of as a process with a ``hole'' (written $\Box$) in it. The
application of a context $M$ to a process $P$, written $M[P]$, is
tantamount to filling the hole in $M$ with $P$. In this paper we do
not need the full weight of this theory, but do make use of the notion
of context in the proof the main theorem. 

\begin{mathpar}
  \inferrule* [lab=summation] {} {{M_{M},M_{N}} \bc \Box \;|\; x.M_{A} \;|\; M_{M}+M_{N}}
  \and
  \inferrule* [lab=agent] {} {{M_{A}} \bc (\vec{x})M_{P} \;| \; \clift{P_0,\ldots,M_{P},\ldots,P_N}}
  \and \\
  \inferrule* [lab=process] {} {{M_{P}} \bc M_{N} \;| \;P|M_{P} }
\end{mathpar} 

\begin{mathpar}
  \inferrule* [lab=sychronization] {} {M_{N} \bc \Box \;|\; x?M_{F} \;|\; x!M_{C}}
  \and
  \inferrule* [lab=abstraction] {} {{M_{F}} \bc (x)M_{P} }
  \and
  \inferrule* [lab=concretion] {} {{M_{C}} \bc \langle M_{P} \rangle }
  \and \\
  \inferrule* [lab=process] {} {{M_{P}} \bc M_{N} \;| \;P|M_{P} }
\end{mathpar}

\begin{definition}[contextual application] Given a context $M$, and
  process $P$, we define the \emph{contextual application}, $M[P] :=
  M\{P/\Box\}$. That is, the contextual application of M to P is the
  substitution of $P$ for $\Box$ in $M$.
\end{definition}

$\meaningof{-} : L \to \mathcal{P}(\pi)$

\begin{mathpar}
  \inferrule* [lab=collection] {} {\meaningof{true} = \pi, \and \meaningof{~E} = \pi \setminus \meaningof{E}, \and \meaningof{E_{1} \& E_{2}} = \meaningof{E_{1}} \cap \meaningof{E_{2}}}
\end{mathpar}

\begin{mathpar}
  \inferrule* [lab=structure] {} {\meaningof{0} = \{ P \in \pi | P \equiv 0 \}, \and \\ \meaningof{E_1 | E_2} = \{ P \in \pi | P \equiv P_{1} | P_{2}, P_{1} \in \meaningof{E_{1}}, P_{2} \in \meaningof{E_2}\} }
\end{mathpar}

\begin{mathpar}
 \inferrule* [lab=behavior] {} {\meaningof{\langle a?b \rangle E} = \{ P \in \pi | P \equiv Q | u?(y)P', \\ \and \\\\ \and \\ \;\;\; u \in \meaningof{a}, \forall z.P'\{z/y\} \in \meaningof{E\{z/b\}}\}, \and \\ \meaningof{a!E} = \{ P \in \pi | P \equiv Q | x!\langle P' \rangle, x \in \meaningof{a} P' \in \meaningof{E}\} }
\end{mathpar}

\begin{mathpar}
 \inferrule* [lab=nominal] {} {\meaningof{\quotep{E}} = \{ \quotep{P} \in \quotep{\pi} | P \in \meaningof{E} \}, \and \meaningof{\quotep{P}} = \{ \quotep{Q} \in \quotep{\pi} | P \equiv Q \} \and \\ \meaningof{@\quotep{E}} = \{ P \in \pi | P \equiv @x, x \in \meaningof{E} \}}
\end{mathpar}

\begin{eqnarray*}
  \\
  \meaningof{-} : TS \to ST
\end{eqnarray*}

\begin{eqnarray*}
  \\
  L : TS \to ST
\end{eqnarray*}

\begin{eqnarray*}
  \\
  P \models E \iff P \in \meaningof{E}
\end{eqnarray*}

\begin{eqnarray*}
  P \approx_{L} Q \iff \forall E \in L. P \models E \iff Q \models E
\end{eqnarray*}

\begin{eqnarray*}
  P \approx_{K} Q
\end{eqnarray*}

\begin{eqnarray*}
  P \approx Q
\end{eqnarray*}

$\approx_{K} = \approx = \approx_{L}$

\subsubsection{Contextual duality}

Note that contexts extend the quotation operation to a family of
operations from processes to names. Given a context, $M$, we can
define a \emph{nominal context}, $\quotep{M}$ by $\quotep{M}[P] :=
\quotep{M[P]}$. To foreshadow what is to come we observe that these
operations enjoy a duality with processes very much like the duality
between vectors and maps from vectors to scalars.

Further, because the calculus is essentially higher-order, we have a
correspondence between contexts and processes. More specifically,
given a name $x$ and a context $M$ we can construct $M^{*}_{x}$ such
that 

\begin{mathpar}
  M^{*}_{x} | \lift{x}{P} \red M[P]
\end{mathpar}

namely,

\begin{mathpar}
  M^{*}_{x} := x?(u).M[\dropn{u}]
\end{mathpar}

The dependence of $M^{*}_{x}$ on a name makes it an abstraction, 

\begin{mathpar}
  M^{*} := (x)x?(u).M[\dropn{u}]
\end{mathpar}

\subsection{Additional notation}

It will sometimes be convenient to denote the process a name
quotes. We already have the notation $x = \quotep{P}$, but it will be
convenient to introduce an alternate notation, $\procn{x}$, when we
want to emphasize the connection to the use of the name. Note that, by
virtue of name equivalence, $\quotep{\procn{x}} \nameeq x$; so, the
notation is consistent with previous definitions.

Further, because names have structure it is possible to effect
substitutions on the basis of that structure. This means we need to
upgrade our notation for substitutions, which we accomplish by
adapting comprehension notation. Thus,

\begin{mathpar}
  P\{ y / x : x \in S \}
\end{mathpar}

is interpreted to mean the process derived from P by replacing (in a
capture-avoiding manner) each occurrence of $x$ in $S$ by $y$. For example,

\begin{mathpar}
  P\{ \quotep{\procn{x}|\procn{x}} / x : x \in \freenames{P} \}
\end{mathpar}

will replace each (occurrence) of a free name $x$ in $P$ by
$\quotep{\procn{x}|\procn{x}}$.

Also, we will avail ourselves of the notation $x^{L}$ and $x^{R}$ to
denote injections of a name into disjoint copies of the name
space. There are numerous ways to accomplish this. One example can be
found in \cite{MeredithR05}. This notation overloads to vectors of
names: $\vec{x}^{\pi} := (x_{i}^{\pi} \; : \; 0 \leq i < |\vec{x}| )$ where $\pi \in \{L,R\}$.

We also use $P^{\Box} := P|\Box$.

In \cite{MeredithR05} an interpretation of the new operator is
given. It turns out that there are several possible interpretations
all enjoying the requisite algebraic properties of the operator (see
\cite{milner91polyadicpi}). We will therefore make liberal use of
$(\nu\; \vec{x})P$.

% subsection the_syntax_and_semantics_of_the_notation_system (end)   

\input{qm2pi.qmops} 

\input{qm2pi.sterngerlach} 

\input{qm2pi.metric} 

% section concurrent_process_calculi (end)

%\input{qm2pi.proofsketch}

% section proof sketch (end)

%\input{qm2pi.slviaknots} 

% section spatial logic via knots (end)

\input{qm2pi.conclusion}

% section conclusion (end)

%\input{qm2pi.dtcodes} 

% section wiring algorithm (end)

\input{qm2pi.ack} 

% section acknowledgments (end)

\newpage


\bibliographystyle{plain}   
\bibliography{../../biblios/main.bib}

\input{qm2pi.rhodetails}

\end{document}



\end{document}



% section front matter (end)

\section{Introduction}\label{sec:introduction} % (fold)
In this draft of the material i am going to have to dispense with the
usual writing conventions adopted in papers on these topics. i'm going
to have adopt whatever tone i need at the time i'm writing up the
calculations. Sometimes this may be very conversational; others it may
be the barest mathematical grunts; others still it may be that i have
lifted text from one of my other papers because the exposition of some
point was better said there. i hope that my readers are not unduly put
out by this decision. i'm not doing this to flout convention or be
rebellious. i find these calculations very technically challenging. To
keep everything going technically, something has to give; i have to
let go of some cognitive burden. So, the academic writing style --
with all of its trade-offs in terms of facilitating technical
communication -- is what i'm letting go of. Perhaps subsequent drafts
can be tightened and polished, but for now, i'm going to speak as if
we were sitting together in a coffee shop with a laptop, wifi and a
pad of paper and a pencil.

So, here's what i have to say. We -- you and i, comfortably ensconced
in our coffee shop and well-equipped with our tools -- can realize and
carry out the calculations of quantum mechanics over a very different
formal theory of dynamics, a formal theory of dynamics that
corresponds to a theory of concurrent computation with
\emph{reflection}. It has the advantage that the underlying theory is
already `quantized', but supports analogues all of the continuuous
operations. Strikingly, this underlying theory has recently been
connected with a notion of metric that we can show, by calculating
together, coincides with the metric induced by the inner product.

There are a lot of reasons why you might be interested in seeing
calculations of this form. Here's why i'm interested. For the past
several centuries there has been no competitor to the ``Newtonian''
account of dynamics. As a result the predominant share of accounts of
dynamical systems and situations have had to be formulated in terms of
the Newtonian machinery. i view this as an intellectually dangerous
position to occupy. Everything, despite it's intrinsic shape, turns
into a nail to be hit with this hammer. Recently, however, the theory
of computation has matured to the point where we have candidates for
theories of dynamics that offer very different perspective on
reasoning about dynamical systems and situations. Testing these
candidates against very successful accounts of dynamical situations,
like quantum mechanics, is going to give us some sense of how mature
they are and some measure of the quality of these accounts of
dynamics.

\subsection{Summary of contributions and outline of paper}

So, we're going to develop an interpretation of the operations of
quantum mechanics normally interpreted by Hilbert spaces and
operators. We're going to do this over a theory of computation. Note
that this is very different than the usual quantum computation program
which develops notions of computation over quantum mechanics. Rather,
we are developing a story that aligns with Wheeler's slogan: It from
Bit. To do this we will first provide an account of the theory of
computation at play here. Then we will dive into a calculation-driven
interpretation of the operations of quantum mechanics.

The reason we take this approach is that -- until very recently --
there hasn't been an axiomatic account of quantum mechanics. As a
result there has been no sharp delineation of the mathematical theory
supporting interpretation of the physical theory and the physical
theory, itself. So, ambient features of the maths are free to be
exploited (or supressed) without a real accounting of their physical
relevance. There is no sharp statement ``here's the physical theory''
qua \emph{theory} and ``here's the mathematical interpretation''
enabling a judgment of how faithful the interpretation is -- apart
from experimental observation. When there is an axiomatic account we
can judge how well a given mathematical formalism supports an
interpretation of the axioms, independent of
experimentation. Likewise, we can judge how well we have captured our
physical evidence and experience with our axiomatics, independent of
any specific mathematical implementation, with accidental detail that
may or may not have physical significance. 

In lieu of a fully fleshed out and vetted axiomatic account of quantum
mechanics, interpreting the operational notions in service of modeling
physical systems will have to suffice. In other words, we are not in
the business of providing a model of Hilbert spaces and operators. We
are in the business of providing a model of quantum mechanics because
we are motivated by testing our notions of dynamics against physical
theory; and, the predictive calculations of the physical theory must
serve as the best formulation -- shy of a fully fleshed out axiomatic
account -- of the physical theory itself (as they have for scientific
theories since time immemorial). Put another way, despite a
whole-hearted commitment to an It-from-Bit ontology, we are firmly
aligned with the shut-up-and-calculate camp as the best way to obtain
results either from the physical perspective or as a quality assurance
measure of our fledgling theory of dynamics.

In detail, we present a reflective process calculus. Then we develop
intuitive correspondences between the notions available in this
calculus and the usual physical notions supporting quantum mechanical
calculations. Thus, 

\begin{table}[htp]
  \center{
    \fbox{
      \begin{tabular}{c|c}
        quantum mechanics & process calculus \\
        \hline
        scalar & name \\
        state vector & process \\
        dual & contextual duals \\
        matrix & formal sums of process-context-dual pairs \\
        orthogonality & process annihilation \\
        inner product & execution-formula + quoting
      \end{tabular}
    }
  }
  \caption{QM - process calculi correspondences}
\end{table}

Then we tighten up these intuitions to operational definitions. We
employ the Dirac notation as the best proxy we can find for an
abstract syntax of the quantum mechanical notions. The definitions we
develop put us in contact with equational constraints coming from the
theory that we demonstrate the definitions and calculations satisfy.

This puts us in a position to shut up and calculate for the
Stern-Gerlach experimental set up, showing how these predictive
calculations become calculations on processes in our theory of a
reflective process calculus.

Penultimately, we demonstrate that the notion of metric coming from
the inner product coincides with the notion of metric available from
the theory of bisimulation. This demonstration gives us the right to
think of space as arising from behavior. Finally, we consider where we
might go from the new vantage point we have obtained.

% section introduction (end) 
 
% section introduction (end)

% \documentclass[12pt]{llncs}
%\documentclass{jktr}

\usepackage[pdftex]{hyperref}                   
\usepackage {listings}
\usepackage {mathpartir}
\usepackage{bcprules}
%\usepackage{listings}
                       
\usepackage{graphicx} 
%\usepackage[margins=2.5cm,nohead,nofoot]{geometry}
%\usepackage{geometry}
\usepackage{amsfonts}
\usepackage{amstext}
\usepackage{latexsym}
\usepackage{amssymb}
\usepackage{color}


%\include{myPreamble}
\documentclass[12pt]{llncs}
%\documentclass{jktr}

\usepackage[pdftex]{hyperref}                   
\usepackage {listings}
\usepackage {mathpartir}
\usepackage{bcprules}
%\usepackage{listings}
                       
\usepackage{graphicx} 
%\usepackage[margins=2.5cm,nohead,nofoot]{geometry}
%\usepackage{geometry}
\usepackage{amsfonts}
\usepackage{amstext}
\usepackage{latexsym}
\usepackage{amssymb}
\usepackage{color}


%\include{myPreamble}
\include{qm2pi.local} 

%\ifpdf
%\usepackage[pdftex]{graphicx}
%\else
%\usepackage{graphicx}
%\fi

 % \ifpdf
%  \usepackage{pdfsync}
%  \if


%\title{Brief Article}
%\author{David F. Snyder}
%\author{L.G. Meredith}

%\address{Dept. of Math., Texas State University--San Marcos, San Marcos, TX 78666}
       
\pagestyle{empty}


\begin{document}

\lstset{language=[Objective]Caml,frame=shadowbox}

\input{qm2pi.front}

% section front matter (end)

\input{qm2pi.intro} 
 
% section introduction (end)

% \input{qm2pi.knotations} 

% section notation (end)

\input{qm2pi.process.calculi} 

% section concurrent_process_calculi_and_spatial_logics_ (end)
    
%\input{qm2pi.knots2pi} 

%\input{qm2pi.trefoil} 

%\input{qm2pi.mainthm} 

% subsection basic_interpretation (end)

%\input{qm2pi.rho.presentation} 
\subsection{The syntax and semantics of the notation system}\label{sub:the_syntax_and_semantics_of_the_notation_system} % (fold)

We now summarize a technical presentation of the calculus that
embodies our theory of dynamics. The typical presentation of such a
calculus follows the style of giving generators and relations on
them. The grammar, below, describing term constructors, freely
generates the set of processes, $\Proc$. This set is then quotiented
by a relation known as structural congruence and it is over this set
that the notion of dynamics is expressed. This presentation is
essentially that of \cite{MeredithR05} with the addition of
polyadicity and summation. For readability we have relegated some of
the technical subtleties to an appendix.

\subsubsection{Process grammar}\label{subsub:process_grammar}

\begin{mathpar}
  \inferrule* [lab=synchronization] {} {{M} \bc \pzero \;|\; x?F \;|\; x!C }
  \and
  \inferrule* [lab=abstraction] {} {{F} \bc (x)P}
  \and
  \inferrule* [lab=concretion] {} {{C} \bc \langle Q \rangle}
  \and
  \inferrule* [lab=process] {} {{P,Q} \bc M \;| \;P|Q \;|\; @{x}}
  \and
  \inferrule* [lab=name] {} {{x} \bc \quotep{P}}
\end{mathpar} 

Note that $\vec{x}$ (resp. $\vec{P}$) denotes a vector of names
(resp. processes) of length $|\vec{x}|$ (resp. $|\vec{P}|$). We adopt
the following useful abbreviations.

\begin{mathpar}
   x?(\vec{y}).P := x.(\vec{y})P \and  x\clift{\vec{P}} := x.\clift{\vec{P}}
   \and x!(y) := \lift{x}{\dropn{y}}
   \and \Pi_{i=0}^{n-1}P_i := P_0 | \ldots | P_{n-1}
\end{mathpar}

\subsubsection{Structural congruence}

\paragraph{Free and bound names and alpha-equivalence.} At the
core of structural equivalence is alpha-equivalence which identifies
process that are the same up to a change of variable. Formally, we
recognize the distinction between free and bound names. The free names
of a process, $\freenames{P}$, may be calculated recursively as
follows:

\begin{mathpar}
\freenames{\pzero} := \emptyset
  \and \\
  \freenames{x?(y).P} := \{ x \} \cup (\freenames{P} \setminus \{ y \})
  \and 
  \freenames{x!\langle P \rangle} := \{ x \} \cup \{ P \} 
  \and \\
  \freenames{P|Q} := \freenames{P} \cup \freenames{Q}
  \and \\
  \freenames{@{x}} := \{ x \}
\end{mathpar}

$\pi$
$\quotep{\pi}$

$\freenames{-} : \pi \to \mathcal{P}(\quotep{\pi})$

\begin{eqnarray*}
  \freenames{\pzero} & := & \emptyset \\
  \freenames{x?(y).P} & := & \{ x \} \cup (\freenames{P} \setminus \{ y \}) \\
  \freenames{x!\langle P \rangle} & := & \{ x \} \cup \{ P \} \\
  \freenames{P|Q} & := & \freenames{P} \cup \freenames{Q} \\
  \freenames{\dropn{x}} & := & \{ x \}
\end{eqnarray*}

The bound names of a process, $\boundnames{P}$, are those names occurring in $P$
that are not free. For example, in $x?(y).0$, the name $x$ is free, while $y$ is bound.

\begin{mathpar}
  \inferrule* [lab=monoidal-laws] {} { P|Q \equiv Q|P \and P|0 \equiv P \and P|(Q|R) \equiv (P|Q)|R }
\end{mathpar}

\begin{mathpar}
  \inferrule* [lab=alpha-equivalence] {} { (x)P \equiv (y)P\{y/x\} \and y \not\in \freenames{P} }
\end{mathpar}

\begin{definition}
Then two processes, $P,Q$, are alpha-equivalent if $P = Q\{\vec{y}/\vec{x}\}$ for
some $\vec{x} \in \boundnames{Q},\vec{y} \in \boundnames{P}$, where $Q\{\vec{y}/\vec{x}\}$
denotes the capture-avoiding substitution of $\vec{y}$ for $\vec{x}$ in $Q$.
\end{definition}

\begin{definition}
  The {\em structural congruence} \cite{SangiorgiWalker} , $\equiv$,
  between processes is the least congruence containing
  alpha-equivalence, satisfying the abelian monoid laws
  (associativity, commutativity and $\pzero$ as identity) for parallel
  composition $|$ and for summation $+$.
\end{definition}

\subsection{Name equivalence}

We take name equivalence, written $\nameeq$, to be the smallest
equivalence relation generated by the following rules.

\begin{mathpar}
\inferrule*[lab=Quote-drop]
{ }
{ \quotep{@{x}} \nameeq x }

\inferrule*[lab=Struct-equiv]
{ P \scong Q }
{ \quotep{P} \nameeq \quotep{Q} }
\end{mathpar}

The astute reader will have noticed that the mutual recursion of names
and processes imposes a mutual recursion on alpha-equivalence and
structural equivalence via name-equivalence. Fortunately, all of this
works out pleasantly and we may calculate in the natural way, free of
concern. The reader interested in the details is referred to the
appendix \ref{appendix:rho_details}.

\subsection{Substitution}

We use $\Proc$ for the set of processes, $\QProc$ for the set of
names, and $\id{\{}\vec{y} / \vec{x} \id{\}}$ to denote partial maps,
$s : \QProc \rightarrow \QProc$. A map, $s$ lifts, uniquely, to a map
on process terms, $\widehat{s} : \Proc \rightarrow \Proc$ by the
following equations.

\begin{mathpar}
  (0) \psubstp{Q}{P} := 0 \\
  (R \juxtap S) \psubstp{Q}{P}
  :=    
  (R)\psubstp{Q}{P} \juxtap (S) \psubstp{Q}{P} \\
  (x?(y).R) \psubstp{Q}{P}    
  :=    
  (x)\substp{Q}{P} (z)\concat( (R \psubstn{z}{y}) \psubstp{Q}{P} ) \\
  (\lift{x}{R}) \psubstp{Q}{P}  
  :=
  \lift{(x)\substp{Q}{P}}{ R \psubstp{Q}{P} } \\
%   (\dropn{x})  \psubstp{Q}{P}       
%   := 
%   \left\{ 
%     \begin{array}{ccc} 
%       \dropn{\quotep{Q}} & & x \nameeq \quotep{P} \\
%       \dropn{x} & & otherwise \\
%     \end{array}
%   \right. 
  (\dropn{x})  \psubstp{Q}{P}       
  := 
  \left\{ 
    \begin{array}{ccc} 
      Q & & x \nameeq \quotep{P} \\
      \dropn{x} & & otherwise \\
    \end{array}
  \right.
\end{mathpar}
 

where

\begin{eqnarray}
  (x)\id{\{} \lpquote Q \rpquote / \lpquote P \rpquote \id{\}}            = 
  \left\{ 
    \begin{array}{ccc}
      \lpquote Q \rpquote & & x \nameeq \lpquote P \rpquote \\
      x & & otherwise \\
    \end{array}
  \right. \nonumber
\end{eqnarray}

and $z$ is chosen distinct from $\quotep{P}$, $\quotep{Q}$, the free
names in $Q$, and all the names in $R$. Our $\alpha$-equivalence will
be built in the standard way from this substitution.

\begin{remark}\label{rem:no_self_referential_names}
  One consequence of these definitions is that $\forall P. \quotep{P}
  \not\in \freenames{P}$.
\end{remark}

\subsection{ Dynamic quote: an example }

Anticipating something of what's to come, consider applying the
substitution, $\widehat{\id{\{}u / z \id{\}}}$, to the following pair
of processes, $\lift{w}{y!(z)}$ and $w[ \lpquote y!(z) \rpquote ]$.

\begin{eqnarray}
	\lift{w}{y!(z)}\widehat{\id{\{}u / z \id{\}}}
		& = &
		\lift{w}{y!(u)} \nonumber\\
	w[ \lpquote y!(z) \rpquote ] \widehat{ \id{\{}u / z \id{\}} }
		& = &
		w[ \lpquote y!(z) \rpquote ] \nonumber
\end{eqnarray}

Because the body of the process between quotes is impervious to
substitution, we get radically different answers. In fact, by
examining the first process in an input context,
e.g. $x?(z).\lift{w}{y!(z)}$, we see that the process under the lift
operator may be shaped by prefixed inputs binding a name inside it. In
this sense, the lift operator will be seen as a way to dynamically
construct processes before reifying them as names.

Finally equipped with these standard features we can present the
dynamics of the calculus.

\subsubsection{Operational semantics} 

Finally, we introduce the computational dynamics. What marks these
algebras as distinct from other more traditionally studied algebraic
structures, e.g. vector spaces or polynomial rings, is the manner in
which dynamics is captured. In traditional structures, dynamics is typically
expressed through morphisms between such structures, as in linear maps
between vector spaces or morphisms between rings. In algebras
associated with the semantics of computation, the dynamics is
expressed as part of the algebraic structure itself, through a
reduction reduction relation typically denoted by $\red$. Below, we
give a recursive presentation of this relation for the calculus used
in the encoding.

$\red \subseteq \pi \times \pi$
$\red : \pi \to \mathcal{P}(\pi)$

\begin{mathpar}
  \inferrule* [lab=Comm] { \textsf{match}( x_{src}, x_{trgt} ) } { x_{trgt}?(y)P \; | \; x_{src}!\langle {Q} \rangle \red P\{\quotep{Q}/y}\} }
  \and \\
  \inferrule* [lab=Par] {{P} \red {P}'} {{{P} | {Q}} \red {{P}' | {Q}}}
  \and
  \inferrule* [lab=Equiv]{{{P} \scong {P}'} \andalso {{P}' \red {Q}'} \andalso {{Q}' \scong {Q}}}{{P} \red {Q}}
\end{mathpar}

\begin{eqnarray*}
  match_{\equiv} (\quotep{P},\quotep{Q}) & := & P \equiv Q \\
  match_{\dagger}(\quotep{P},\quotep{Q}) & := & \forall R. P|Q \red^{*} R => R \red^{*} 0 \\
  match_{K}(\quotep{P},\quotep{Q}) & := & K \mbox{ for some context } K
\end{eqnarray*}

$u?(x)P | u!\langle Q \rangle \red P\{\quotep{Q}/x\}$

%We write $\wred$ for $\red^*$, and $P\red$ if $\exists Q $ such that $ P \red Q$.
We write $P\red$ if $\exists Q $ such that $ P \red Q$ and $P\not\red$, otherwise.

\section{Replication}

As mentioned before, it is known that replication (and hence
recursion) can be implemented in a higher-order process algebra
\cite{SangiorgiWalker}. As our first example of calculation with the
machinery thus far presented we give the construction explicitly in
the {\rhoc}.

\begin{eqnarray}
	D_{x} & := & \prefix{x}{y}{(\binpar{\outputp{x}{y}}{@{y}})} \nonumber\\
	\bangp_{x}{P} & := & \binpar{{x}!\langle{\binpar{D_{x}}{P}}\rangle}{D_{x}} \nonumber
\end{eqnarray}

\begin{eqnarray}
	\bangp_{x}{P} & & \nonumber\\
	=
	& {x}!\langle{(\prefix{x}{y}{(\outputp{x}{y} | @{y})) | P}}\rangle 
	      | \prefix{x}{y}{(\outputp{x}{y} | @{y})} & \nonumber\\
	\red
	& (\outputp{x}{y} | @{y})\substn{\quotep{(\prefix{x}{y}{(@{y} | \outputp{x}{y})) | P}}}{y} & \nonumber\\
	=
	& \outputp{x}{\quotep{(\prefix{x}{y}{(\outputp{x}{y} | @{y})) | P}}}
	  | {(\prefix{x}{y}{(\outputp{x}{y} | @{y})) | P}} & \nonumber\\
	\red
	& \ldots & \nonumber\\
	\red^*
	& P | P | \ldots & \nonumber
\end{eqnarray}

Of course, this encoding, as an implementation, runs away, unfolding
$\bangp{P}$ eagerly. A lazier and more implementable replication
operator, restricted to input-guarded processes, may be obtained as follows.

\begin{eqnarray}
\bangp{\prefix{u}{v}{P}} 
	:= 
	\binpar{\lift{x}{\prefix{u}{v}{(\binpar{D(x)}{P})}}}{D(x)} \nonumber
\end{eqnarray}

\begin{remark}
  Note that the lazier definition still does not deal with summation
  or mixed summation (i.e. sums over input and output). The reader is
  invited to construct definitions of replication that deal with these
  features. 

  Further, the definitions are parameterized in a name, $x$. Can you,
  gentle reader, make a definition that eliminates this parameter and
  guarantees no accidental interaction between the replication
  machinery and the process being replicated -- i.e. no accidental
  sharing of names used by the process to get its work done and the
  name(s) used by the replication to effect copying. This latter
  revision of the definition of replication is crucial to obtaining
  the expected identity $!!P \sim !P$.
\end{remark}

\begin{remark}\label{rem:paradoxical_combinator}
  The reader familiar with the lambda calculus will have noticed the
  similarity between $D$ and the paradoxical combinator.

  [Ed. note: the existence of this seems to suggest we have to be more
  restrictive on the set of processes and names we admit if we are to
  support no-cloning.]
\end{remark}

\subsubsection{Bisimulation}

The computational dynamics gives rise to another kind of equivalence,
the equivalence of computational behavior. As previously mentioned
this is typically captured \emph{via} some form of bisimulation.

% The notion we use in this paper is weak barbed bisimulation
% \cite{milner91polyadicpi}.

The notion we use in this paper is derived from weak barbed
bisimulation \cite{milner91polyadicpi}. 

\begin{definition}
An \emph{observation relation}, $\downarrow_{\mathcal N}$, over a set
of names, $\mathcal N$, is the smallest relation satisfying the rules
below.

\infrule[Out-barb]{y \in {\mathcal N}, \; x \nameeq y}
		  {\outputp{x}{v} \downarrow_{\mathcal N} x}
\infrule[Par-barb]{\mbox{$P\downarrow_{\mathcal N} x$ or $Q\downarrow_{\mathcal N} x$}}
		  {\binpar{P}{Q} \downarrow_{\mathcal N} x}

We write $P \Downarrow_{\mathcal N} x$ if there is $Q$ such that 
$P \wred Q$ and $Q \downarrow_{\mathcal N} x$.
\end{definition}

\begin{definition}
%\label{def.bbisim}
An  ${\mathcal N}$-\emph{barbed bisimulation} over a set of names, ${\mathcal N}$, is a symmetric binary relation 
${\mathcal S}_{\mathcal N}$ between agents such that $P\rel{S}_{\mathcal N}Q$ implies:
\begin{enumerate}
\item If $P \red P'$ then $Q \wred Q'$ and $P'\rel{S}_{\mathcal N} Q'$.
\item If $P\downarrow_{\mathcal N} x$, then $Q\Downarrow_{\mathcal N} x$.
\end{enumerate}
$P$ is ${\mathcal N}$-barbed bisimilar to $Q$, written
$P \wbbisim_{\mathcal N} Q$, if $P \rel{S}_{\mathcal N} Q$ for some ${\mathcal N}$-barbed bisimulation ${\mathcal S}_{\mathcal N}$.
\end{definition}

$\mathcal{R} \subseteq \pi \times \pi$

$P \mathcal{R} Q => \forall P'. P \red P' \Rightarrow \exists Q'. Q \red Q', P' \mathcal{R} Q'$

$P \vdash x \Rightarrow Q \vdash x$

\begin{mathpar}
  \inferrule*[lab=Out-barb]{x \nameeq y}{{y}!\langle{Q}\rangle \vdash x}
  \and
  \inferrule*[lab=Par-barb]{\mbox{$P\vdash x$ or $Q\vdash x$}}{\binpar{P}{Q} \vdash x}
\end{mathpar}

\subsubsection{Contexts}

One of the principle advantages of computational calculi like the
$\pi$-calculus is a well-defined notion of context,
contextual-equivalence and a correlation between
contextual-equivalence and notions of bisimulation. The notion of
context allows the decomposition of a process into (sub-)process and
its syntactic environment, its context. Thus, a context may be
thought of as a process with a ``hole'' (written $\Box$) in it. The
application of a context $M$ to a process $P$, written $M[P]$, is
tantamount to filling the hole in $M$ with $P$. In this paper we do
not need the full weight of this theory, but do make use of the notion
of context in the proof the main theorem. 

\begin{mathpar}
  \inferrule* [lab=summation] {} {{M_{M},M_{N}} \bc \Box \;|\; x.M_{A} \;|\; M_{M}+M_{N}}
  \and
  \inferrule* [lab=agent] {} {{M_{A}} \bc (\vec{x})M_{P} \;| \; \clift{P_0,\ldots,M_{P},\ldots,P_N}}
  \and \\
  \inferrule* [lab=process] {} {{M_{P}} \bc M_{N} \;| \;P|M_{P} }
\end{mathpar} 

\begin{mathpar}
  \inferrule* [lab=sychronization] {} {M_{N} \bc \Box \;|\; x?M_{F} \;|\; x!M_{C}}
  \and
  \inferrule* [lab=abstraction] {} {{M_{F}} \bc (x)M_{P} }
  \and
  \inferrule* [lab=concretion] {} {{M_{C}} \bc \langle M_{P} \rangle }
  \and \\
  \inferrule* [lab=process] {} {{M_{P}} \bc M_{N} \;| \;P|M_{P} }
\end{mathpar}

\begin{definition}[contextual application] Given a context $M$, and
  process $P$, we define the \emph{contextual application}, $M[P] :=
  M\{P/\Box\}$. That is, the contextual application of M to P is the
  substitution of $P$ for $\Box$ in $M$.
\end{definition}

$\meaningof{-} : L \to \mathcal{P}(\pi)$

\begin{mathpar}
  \inferrule* [lab=collection] {} {\meaningof{true} = \pi, \and \meaningof{~E} = \pi \setminus \meaningof{E}, \and \meaningof{E_{1} \& E_{2}} = \meaningof{E_{1}} \cap \meaningof{E_{2}}}
\end{mathpar}

\begin{mathpar}
  \inferrule* [lab=structure] {} {\meaningof{0} = \{ P \in \pi | P \equiv 0 \}, \and \\ \meaningof{E_1 | E_2} = \{ P \in \pi | P \equiv P_{1} | P_{2}, P_{1} \in \meaningof{E_{1}}, P_{2} \in \meaningof{E_2}\} }
\end{mathpar}

\begin{mathpar}
 \inferrule* [lab=behavior] {} {\meaningof{\langle a?b \rangle E} = \{ P \in \pi | P \equiv Q | u?(y)P', \\ \and \\\\ \and \\ \;\;\; u \in \meaningof{a}, \forall z.P'\{z/y\} \in \meaningof{E\{z/b\}}\}, \and \\ \meaningof{a!E} = \{ P \in \pi | P \equiv Q | x!\langle P' \rangle, x \in \meaningof{a} P' \in \meaningof{E}\} }
\end{mathpar}

\begin{mathpar}
 \inferrule* [lab=nominal] {} {\meaningof{\quotep{E}} = \{ \quotep{P} \in \quotep{\pi} | P \in \meaningof{E} \}, \and \meaningof{\quotep{P}} = \{ \quotep{Q} \in \quotep{\pi} | P \equiv Q \} \and \\ \meaningof{@\quotep{E}} = \{ P \in \pi | P \equiv @x, x \in \meaningof{E} \}}
\end{mathpar}

\begin{eqnarray*}
  \\
  \meaningof{-} : TS \to ST
\end{eqnarray*}

\begin{eqnarray*}
  \\
  L : TS \to ST
\end{eqnarray*}

\begin{eqnarray*}
  \\
  P \models E \iff P \in \meaningof{E}
\end{eqnarray*}

\begin{eqnarray*}
  P \approx_{L} Q \iff \forall E \in L. P \models E \iff Q \models E
\end{eqnarray*}

\begin{eqnarray*}
  P \approx_{K} Q
\end{eqnarray*}

\begin{eqnarray*}
  P \approx Q
\end{eqnarray*}

$\approx_{K} = \approx = \approx_{L}$

\subsubsection{Contextual duality}

Note that contexts extend the quotation operation to a family of
operations from processes to names. Given a context, $M$, we can
define a \emph{nominal context}, $\quotep{M}$ by $\quotep{M}[P] :=
\quotep{M[P]}$. To foreshadow what is to come we observe that these
operations enjoy a duality with processes very much like the duality
between vectors and maps from vectors to scalars.

Further, because the calculus is essentially higher-order, we have a
correspondence between contexts and processes. More specifically,
given a name $x$ and a context $M$ we can construct $M^{*}_{x}$ such
that 

\begin{mathpar}
  M^{*}_{x} | \lift{x}{P} \red M[P]
\end{mathpar}

namely,

\begin{mathpar}
  M^{*}_{x} := x?(u).M[\dropn{u}]
\end{mathpar}

The dependence of $M^{*}_{x}$ on a name makes it an abstraction, 

\begin{mathpar}
  M^{*} := (x)x?(u).M[\dropn{u}]
\end{mathpar}

\subsection{Additional notation}

It will sometimes be convenient to denote the process a name
quotes. We already have the notation $x = \quotep{P}$, but it will be
convenient to introduce an alternate notation, $\procn{x}$, when we
want to emphasize the connection to the use of the name. Note that, by
virtue of name equivalence, $\quotep{\procn{x}} \nameeq x$; so, the
notation is consistent with previous definitions.

Further, because names have structure it is possible to effect
substitutions on the basis of that structure. This means we need to
upgrade our notation for substitutions, which we accomplish by
adapting comprehension notation. Thus,

\begin{mathpar}
  P\{ y / x : x \in S \}
\end{mathpar}

is interpreted to mean the process derived from P by replacing (in a
capture-avoiding manner) each occurrence of $x$ in $S$ by $y$. For example,

\begin{mathpar}
  P\{ \quotep{\procn{x}|\procn{x}} / x : x \in \freenames{P} \}
\end{mathpar}

will replace each (occurrence) of a free name $x$ in $P$ by
$\quotep{\procn{x}|\procn{x}}$.

Also, we will avail ourselves of the notation $x^{L}$ and $x^{R}$ to
denote injections of a name into disjoint copies of the name
space. There are numerous ways to accomplish this. One example can be
found in \cite{MeredithR05}. This notation overloads to vectors of
names: $\vec{x}^{\pi} := (x_{i}^{\pi} \; : \; 0 \leq i < |\vec{x}| )$ where $\pi \in \{L,R\}$.

We also use $P^{\Box} := P|\Box$.

In \cite{MeredithR05} an interpretation of the new operator is
given. It turns out that there are several possible interpretations
all enjoying the requisite algebraic properties of the operator (see
\cite{milner91polyadicpi}). We will therefore make liberal use of
$(\nu\; \vec{x})P$.

% subsection the_syntax_and_semantics_of_the_notation_system (end)   

\input{qm2pi.qmops} 

\input{qm2pi.sterngerlach} 

\input{qm2pi.metric} 

% section concurrent_process_calculi (end)

%\input{qm2pi.proofsketch}

% section proof sketch (end)

%\input{qm2pi.slviaknots} 

% section spatial logic via knots (end)

\input{qm2pi.conclusion}

% section conclusion (end)

%\input{qm2pi.dtcodes} 

% section wiring algorithm (end)

\input{qm2pi.ack} 

% section acknowledgments (end)

\newpage


\bibliographystyle{plain}   
\bibliography{../../biblios/main.bib}

\input{qm2pi.rhodetails}

\end{document}

 

%\ifpdf
%\usepackage[pdftex]{graphicx}
%\else
%\usepackage{graphicx}
%\fi

 % \ifpdf
%  \usepackage{pdfsync}
%  \if


%\title{Brief Article}
%\author{David F. Snyder}
%\author{L.G. Meredith}

%\address{Dept. of Math., Texas State University--San Marcos, San Marcos, TX 78666}
       
\pagestyle{empty}


\begin{document}

\lstset{language=[Objective]Caml,frame=shadowbox}

\documentclass[12pt]{llncs}
%\documentclass{jktr}

\usepackage[pdftex]{hyperref}                   
\usepackage {listings}
\usepackage {mathpartir}
\usepackage{bcprules}
%\usepackage{listings}
                       
\usepackage{graphicx} 
%\usepackage[margins=2.5cm,nohead,nofoot]{geometry}
%\usepackage{geometry}
\usepackage{amsfonts}
\usepackage{amstext}
\usepackage{latexsym}
\usepackage{amssymb}
\usepackage{color}


%\include{myPreamble}
\include{qm2pi.local} 

%\ifpdf
%\usepackage[pdftex]{graphicx}
%\else
%\usepackage{graphicx}
%\fi

 % \ifpdf
%  \usepackage{pdfsync}
%  \if


%\title{Brief Article}
%\author{David F. Snyder}
%\author{L.G. Meredith}

%\address{Dept. of Math., Texas State University--San Marcos, San Marcos, TX 78666}
       
\pagestyle{empty}


\begin{document}

\lstset{language=[Objective]Caml,frame=shadowbox}

\input{qm2pi.front}

% section front matter (end)

\input{qm2pi.intro} 
 
% section introduction (end)

% \input{qm2pi.knotations} 

% section notation (end)

\input{qm2pi.process.calculi} 

% section concurrent_process_calculi_and_spatial_logics_ (end)
    
%\input{qm2pi.knots2pi} 

%\input{qm2pi.trefoil} 

%\input{qm2pi.mainthm} 

% subsection basic_interpretation (end)

%\input{qm2pi.rho.presentation} 
\subsection{The syntax and semantics of the notation system}\label{sub:the_syntax_and_semantics_of_the_notation_system} % (fold)

We now summarize a technical presentation of the calculus that
embodies our theory of dynamics. The typical presentation of such a
calculus follows the style of giving generators and relations on
them. The grammar, below, describing term constructors, freely
generates the set of processes, $\Proc$. This set is then quotiented
by a relation known as structural congruence and it is over this set
that the notion of dynamics is expressed. This presentation is
essentially that of \cite{MeredithR05} with the addition of
polyadicity and summation. For readability we have relegated some of
the technical subtleties to an appendix.

\subsubsection{Process grammar}\label{subsub:process_grammar}

\begin{mathpar}
  \inferrule* [lab=synchronization] {} {{M} \bc \pzero \;|\; x?F \;|\; x!C }
  \and
  \inferrule* [lab=abstraction] {} {{F} \bc (x)P}
  \and
  \inferrule* [lab=concretion] {} {{C} \bc \langle Q \rangle}
  \and
  \inferrule* [lab=process] {} {{P,Q} \bc M \;| \;P|Q \;|\; @{x}}
  \and
  \inferrule* [lab=name] {} {{x} \bc \quotep{P}}
\end{mathpar} 

Note that $\vec{x}$ (resp. $\vec{P}$) denotes a vector of names
(resp. processes) of length $|\vec{x}|$ (resp. $|\vec{P}|$). We adopt
the following useful abbreviations.

\begin{mathpar}
   x?(\vec{y}).P := x.(\vec{y})P \and  x\clift{\vec{P}} := x.\clift{\vec{P}}
   \and x!(y) := \lift{x}{\dropn{y}}
   \and \Pi_{i=0}^{n-1}P_i := P_0 | \ldots | P_{n-1}
\end{mathpar}

\subsubsection{Structural congruence}

\paragraph{Free and bound names and alpha-equivalence.} At the
core of structural equivalence is alpha-equivalence which identifies
process that are the same up to a change of variable. Formally, we
recognize the distinction between free and bound names. The free names
of a process, $\freenames{P}$, may be calculated recursively as
follows:

\begin{mathpar}
\freenames{\pzero} := \emptyset
  \and \\
  \freenames{x?(y).P} := \{ x \} \cup (\freenames{P} \setminus \{ y \})
  \and 
  \freenames{x!\langle P \rangle} := \{ x \} \cup \{ P \} 
  \and \\
  \freenames{P|Q} := \freenames{P} \cup \freenames{Q}
  \and \\
  \freenames{@{x}} := \{ x \}
\end{mathpar}

$\pi$
$\quotep{\pi}$

$\freenames{-} : \pi \to \mathcal{P}(\quotep{\pi})$

\begin{eqnarray*}
  \freenames{\pzero} & := & \emptyset \\
  \freenames{x?(y).P} & := & \{ x \} \cup (\freenames{P} \setminus \{ y \}) \\
  \freenames{x!\langle P \rangle} & := & \{ x \} \cup \{ P \} \\
  \freenames{P|Q} & := & \freenames{P} \cup \freenames{Q} \\
  \freenames{\dropn{x}} & := & \{ x \}
\end{eqnarray*}

The bound names of a process, $\boundnames{P}$, are those names occurring in $P$
that are not free. For example, in $x?(y).0$, the name $x$ is free, while $y$ is bound.

\begin{mathpar}
  \inferrule* [lab=monoidal-laws] {} { P|Q \equiv Q|P \and P|0 \equiv P \and P|(Q|R) \equiv (P|Q)|R }
\end{mathpar}

\begin{mathpar}
  \inferrule* [lab=alpha-equivalence] {} { (x)P \equiv (y)P\{y/x\} \and y \not\in \freenames{P} }
\end{mathpar}

\begin{definition}
Then two processes, $P,Q$, are alpha-equivalent if $P = Q\{\vec{y}/\vec{x}\}$ for
some $\vec{x} \in \boundnames{Q},\vec{y} \in \boundnames{P}$, where $Q\{\vec{y}/\vec{x}\}$
denotes the capture-avoiding substitution of $\vec{y}$ for $\vec{x}$ in $Q$.
\end{definition}

\begin{definition}
  The {\em structural congruence} \cite{SangiorgiWalker} , $\equiv$,
  between processes is the least congruence containing
  alpha-equivalence, satisfying the abelian monoid laws
  (associativity, commutativity and $\pzero$ as identity) for parallel
  composition $|$ and for summation $+$.
\end{definition}

\subsection{Name equivalence}

We take name equivalence, written $\nameeq$, to be the smallest
equivalence relation generated by the following rules.

\begin{mathpar}
\inferrule*[lab=Quote-drop]
{ }
{ \quotep{@{x}} \nameeq x }

\inferrule*[lab=Struct-equiv]
{ P \scong Q }
{ \quotep{P} \nameeq \quotep{Q} }
\end{mathpar}

The astute reader will have noticed that the mutual recursion of names
and processes imposes a mutual recursion on alpha-equivalence and
structural equivalence via name-equivalence. Fortunately, all of this
works out pleasantly and we may calculate in the natural way, free of
concern. The reader interested in the details is referred to the
appendix \ref{appendix:rho_details}.

\subsection{Substitution}

We use $\Proc$ for the set of processes, $\QProc$ for the set of
names, and $\id{\{}\vec{y} / \vec{x} \id{\}}$ to denote partial maps,
$s : \QProc \rightarrow \QProc$. A map, $s$ lifts, uniquely, to a map
on process terms, $\widehat{s} : \Proc \rightarrow \Proc$ by the
following equations.

\begin{mathpar}
  (0) \psubstp{Q}{P} := 0 \\
  (R \juxtap S) \psubstp{Q}{P}
  :=    
  (R)\psubstp{Q}{P} \juxtap (S) \psubstp{Q}{P} \\
  (x?(y).R) \psubstp{Q}{P}    
  :=    
  (x)\substp{Q}{P} (z)\concat( (R \psubstn{z}{y}) \psubstp{Q}{P} ) \\
  (\lift{x}{R}) \psubstp{Q}{P}  
  :=
  \lift{(x)\substp{Q}{P}}{ R \psubstp{Q}{P} } \\
%   (\dropn{x})  \psubstp{Q}{P}       
%   := 
%   \left\{ 
%     \begin{array}{ccc} 
%       \dropn{\quotep{Q}} & & x \nameeq \quotep{P} \\
%       \dropn{x} & & otherwise \\
%     \end{array}
%   \right. 
  (\dropn{x})  \psubstp{Q}{P}       
  := 
  \left\{ 
    \begin{array}{ccc} 
      Q & & x \nameeq \quotep{P} \\
      \dropn{x} & & otherwise \\
    \end{array}
  \right.
\end{mathpar}
 

where

\begin{eqnarray}
  (x)\id{\{} \lpquote Q \rpquote / \lpquote P \rpquote \id{\}}            = 
  \left\{ 
    \begin{array}{ccc}
      \lpquote Q \rpquote & & x \nameeq \lpquote P \rpquote \\
      x & & otherwise \\
    \end{array}
  \right. \nonumber
\end{eqnarray}

and $z$ is chosen distinct from $\quotep{P}$, $\quotep{Q}$, the free
names in $Q$, and all the names in $R$. Our $\alpha$-equivalence will
be built in the standard way from this substitution.

\begin{remark}\label{rem:no_self_referential_names}
  One consequence of these definitions is that $\forall P. \quotep{P}
  \not\in \freenames{P}$.
\end{remark}

\subsection{ Dynamic quote: an example }

Anticipating something of what's to come, consider applying the
substitution, $\widehat{\id{\{}u / z \id{\}}}$, to the following pair
of processes, $\lift{w}{y!(z)}$ and $w[ \lpquote y!(z) \rpquote ]$.

\begin{eqnarray}
	\lift{w}{y!(z)}\widehat{\id{\{}u / z \id{\}}}
		& = &
		\lift{w}{y!(u)} \nonumber\\
	w[ \lpquote y!(z) \rpquote ] \widehat{ \id{\{}u / z \id{\}} }
		& = &
		w[ \lpquote y!(z) \rpquote ] \nonumber
\end{eqnarray}

Because the body of the process between quotes is impervious to
substitution, we get radically different answers. In fact, by
examining the first process in an input context,
e.g. $x?(z).\lift{w}{y!(z)}$, we see that the process under the lift
operator may be shaped by prefixed inputs binding a name inside it. In
this sense, the lift operator will be seen as a way to dynamically
construct processes before reifying them as names.

Finally equipped with these standard features we can present the
dynamics of the calculus.

\subsubsection{Operational semantics} 

Finally, we introduce the computational dynamics. What marks these
algebras as distinct from other more traditionally studied algebraic
structures, e.g. vector spaces or polynomial rings, is the manner in
which dynamics is captured. In traditional structures, dynamics is typically
expressed through morphisms between such structures, as in linear maps
between vector spaces or morphisms between rings. In algebras
associated with the semantics of computation, the dynamics is
expressed as part of the algebraic structure itself, through a
reduction reduction relation typically denoted by $\red$. Below, we
give a recursive presentation of this relation for the calculus used
in the encoding.

$\red \subseteq \pi \times \pi$
$\red : \pi \to \mathcal{P}(\pi)$

\begin{mathpar}
  \inferrule* [lab=Comm] { \textsf{match}( x_{src}, x_{trgt} ) } { x_{trgt}?(y)P \; | \; x_{src}!\langle {Q} \rangle \red P\{\quotep{Q}/y}\} }
  \and \\
  \inferrule* [lab=Par] {{P} \red {P}'} {{{P} | {Q}} \red {{P}' | {Q}}}
  \and
  \inferrule* [lab=Equiv]{{{P} \scong {P}'} \andalso {{P}' \red {Q}'} \andalso {{Q}' \scong {Q}}}{{P} \red {Q}}
\end{mathpar}

\begin{eqnarray*}
  match_{\equiv} (\quotep{P},\quotep{Q}) & := & P \equiv Q \\
  match_{\dagger}(\quotep{P},\quotep{Q}) & := & \forall R. P|Q \red^{*} R => R \red^{*} 0 \\
  match_{K}(\quotep{P},\quotep{Q}) & := & K \mbox{ for some context } K
\end{eqnarray*}

$u?(x)P | u!\langle Q \rangle \red P\{\quotep{Q}/x\}$

%We write $\wred$ for $\red^*$, and $P\red$ if $\exists Q $ such that $ P \red Q$.
We write $P\red$ if $\exists Q $ such that $ P \red Q$ and $P\not\red$, otherwise.

\section{Replication}

As mentioned before, it is known that replication (and hence
recursion) can be implemented in a higher-order process algebra
\cite{SangiorgiWalker}. As our first example of calculation with the
machinery thus far presented we give the construction explicitly in
the {\rhoc}.

\begin{eqnarray}
	D_{x} & := & \prefix{x}{y}{(\binpar{\outputp{x}{y}}{@{y}})} \nonumber\\
	\bangp_{x}{P} & := & \binpar{{x}!\langle{\binpar{D_{x}}{P}}\rangle}{D_{x}} \nonumber
\end{eqnarray}

\begin{eqnarray}
	\bangp_{x}{P} & & \nonumber\\
	=
	& {x}!\langle{(\prefix{x}{y}{(\outputp{x}{y} | @{y})) | P}}\rangle 
	      | \prefix{x}{y}{(\outputp{x}{y} | @{y})} & \nonumber\\
	\red
	& (\outputp{x}{y} | @{y})\substn{\quotep{(\prefix{x}{y}{(@{y} | \outputp{x}{y})) | P}}}{y} & \nonumber\\
	=
	& \outputp{x}{\quotep{(\prefix{x}{y}{(\outputp{x}{y} | @{y})) | P}}}
	  | {(\prefix{x}{y}{(\outputp{x}{y} | @{y})) | P}} & \nonumber\\
	\red
	& \ldots & \nonumber\\
	\red^*
	& P | P | \ldots & \nonumber
\end{eqnarray}

Of course, this encoding, as an implementation, runs away, unfolding
$\bangp{P}$ eagerly. A lazier and more implementable replication
operator, restricted to input-guarded processes, may be obtained as follows.

\begin{eqnarray}
\bangp{\prefix{u}{v}{P}} 
	:= 
	\binpar{\lift{x}{\prefix{u}{v}{(\binpar{D(x)}{P})}}}{D(x)} \nonumber
\end{eqnarray}

\begin{remark}
  Note that the lazier definition still does not deal with summation
  or mixed summation (i.e. sums over input and output). The reader is
  invited to construct definitions of replication that deal with these
  features. 

  Further, the definitions are parameterized in a name, $x$. Can you,
  gentle reader, make a definition that eliminates this parameter and
  guarantees no accidental interaction between the replication
  machinery and the process being replicated -- i.e. no accidental
  sharing of names used by the process to get its work done and the
  name(s) used by the replication to effect copying. This latter
  revision of the definition of replication is crucial to obtaining
  the expected identity $!!P \sim !P$.
\end{remark}

\begin{remark}\label{rem:paradoxical_combinator}
  The reader familiar with the lambda calculus will have noticed the
  similarity between $D$ and the paradoxical combinator.

  [Ed. note: the existence of this seems to suggest we have to be more
  restrictive on the set of processes and names we admit if we are to
  support no-cloning.]
\end{remark}

\subsubsection{Bisimulation}

The computational dynamics gives rise to another kind of equivalence,
the equivalence of computational behavior. As previously mentioned
this is typically captured \emph{via} some form of bisimulation.

% The notion we use in this paper is weak barbed bisimulation
% \cite{milner91polyadicpi}.

The notion we use in this paper is derived from weak barbed
bisimulation \cite{milner91polyadicpi}. 

\begin{definition}
An \emph{observation relation}, $\downarrow_{\mathcal N}$, over a set
of names, $\mathcal N$, is the smallest relation satisfying the rules
below.

\infrule[Out-barb]{y \in {\mathcal N}, \; x \nameeq y}
		  {\outputp{x}{v} \downarrow_{\mathcal N} x}
\infrule[Par-barb]{\mbox{$P\downarrow_{\mathcal N} x$ or $Q\downarrow_{\mathcal N} x$}}
		  {\binpar{P}{Q} \downarrow_{\mathcal N} x}

We write $P \Downarrow_{\mathcal N} x$ if there is $Q$ such that 
$P \wred Q$ and $Q \downarrow_{\mathcal N} x$.
\end{definition}

\begin{definition}
%\label{def.bbisim}
An  ${\mathcal N}$-\emph{barbed bisimulation} over a set of names, ${\mathcal N}$, is a symmetric binary relation 
${\mathcal S}_{\mathcal N}$ between agents such that $P\rel{S}_{\mathcal N}Q$ implies:
\begin{enumerate}
\item If $P \red P'$ then $Q \wred Q'$ and $P'\rel{S}_{\mathcal N} Q'$.
\item If $P\downarrow_{\mathcal N} x$, then $Q\Downarrow_{\mathcal N} x$.
\end{enumerate}
$P$ is ${\mathcal N}$-barbed bisimilar to $Q$, written
$P \wbbisim_{\mathcal N} Q$, if $P \rel{S}_{\mathcal N} Q$ for some ${\mathcal N}$-barbed bisimulation ${\mathcal S}_{\mathcal N}$.
\end{definition}

$\mathcal{R} \subseteq \pi \times \pi$

$P \mathcal{R} Q => \forall P'. P \red P' \Rightarrow \exists Q'. Q \red Q', P' \mathcal{R} Q'$

$P \vdash x \Rightarrow Q \vdash x$

\begin{mathpar}
  \inferrule*[lab=Out-barb]{x \nameeq y}{{y}!\langle{Q}\rangle \vdash x}
  \and
  \inferrule*[lab=Par-barb]{\mbox{$P\vdash x$ or $Q\vdash x$}}{\binpar{P}{Q} \vdash x}
\end{mathpar}

\subsubsection{Contexts}

One of the principle advantages of computational calculi like the
$\pi$-calculus is a well-defined notion of context,
contextual-equivalence and a correlation between
contextual-equivalence and notions of bisimulation. The notion of
context allows the decomposition of a process into (sub-)process and
its syntactic environment, its context. Thus, a context may be
thought of as a process with a ``hole'' (written $\Box$) in it. The
application of a context $M$ to a process $P$, written $M[P]$, is
tantamount to filling the hole in $M$ with $P$. In this paper we do
not need the full weight of this theory, but do make use of the notion
of context in the proof the main theorem. 

\begin{mathpar}
  \inferrule* [lab=summation] {} {{M_{M},M_{N}} \bc \Box \;|\; x.M_{A} \;|\; M_{M}+M_{N}}
  \and
  \inferrule* [lab=agent] {} {{M_{A}} \bc (\vec{x})M_{P} \;| \; \clift{P_0,\ldots,M_{P},\ldots,P_N}}
  \and \\
  \inferrule* [lab=process] {} {{M_{P}} \bc M_{N} \;| \;P|M_{P} }
\end{mathpar} 

\begin{mathpar}
  \inferrule* [lab=sychronization] {} {M_{N} \bc \Box \;|\; x?M_{F} \;|\; x!M_{C}}
  \and
  \inferrule* [lab=abstraction] {} {{M_{F}} \bc (x)M_{P} }
  \and
  \inferrule* [lab=concretion] {} {{M_{C}} \bc \langle M_{P} \rangle }
  \and \\
  \inferrule* [lab=process] {} {{M_{P}} \bc M_{N} \;| \;P|M_{P} }
\end{mathpar}

\begin{definition}[contextual application] Given a context $M$, and
  process $P$, we define the \emph{contextual application}, $M[P] :=
  M\{P/\Box\}$. That is, the contextual application of M to P is the
  substitution of $P$ for $\Box$ in $M$.
\end{definition}

$\meaningof{-} : L \to \mathcal{P}(\pi)$

\begin{mathpar}
  \inferrule* [lab=collection] {} {\meaningof{true} = \pi, \and \meaningof{~E} = \pi \setminus \meaningof{E}, \and \meaningof{E_{1} \& E_{2}} = \meaningof{E_{1}} \cap \meaningof{E_{2}}}
\end{mathpar}

\begin{mathpar}
  \inferrule* [lab=structure] {} {\meaningof{0} = \{ P \in \pi | P \equiv 0 \}, \and \\ \meaningof{E_1 | E_2} = \{ P \in \pi | P \equiv P_{1} | P_{2}, P_{1} \in \meaningof{E_{1}}, P_{2} \in \meaningof{E_2}\} }
\end{mathpar}

\begin{mathpar}
 \inferrule* [lab=behavior] {} {\meaningof{\langle a?b \rangle E} = \{ P \in \pi | P \equiv Q | u?(y)P', \\ \and \\\\ \and \\ \;\;\; u \in \meaningof{a}, \forall z.P'\{z/y\} \in \meaningof{E\{z/b\}}\}, \and \\ \meaningof{a!E} = \{ P \in \pi | P \equiv Q | x!\langle P' \rangle, x \in \meaningof{a} P' \in \meaningof{E}\} }
\end{mathpar}

\begin{mathpar}
 \inferrule* [lab=nominal] {} {\meaningof{\quotep{E}} = \{ \quotep{P} \in \quotep{\pi} | P \in \meaningof{E} \}, \and \meaningof{\quotep{P}} = \{ \quotep{Q} \in \quotep{\pi} | P \equiv Q \} \and \\ \meaningof{@\quotep{E}} = \{ P \in \pi | P \equiv @x, x \in \meaningof{E} \}}
\end{mathpar}

\begin{eqnarray*}
  \\
  \meaningof{-} : TS \to ST
\end{eqnarray*}

\begin{eqnarray*}
  \\
  L : TS \to ST
\end{eqnarray*}

\begin{eqnarray*}
  \\
  P \models E \iff P \in \meaningof{E}
\end{eqnarray*}

\begin{eqnarray*}
  P \approx_{L} Q \iff \forall E \in L. P \models E \iff Q \models E
\end{eqnarray*}

\begin{eqnarray*}
  P \approx_{K} Q
\end{eqnarray*}

\begin{eqnarray*}
  P \approx Q
\end{eqnarray*}

$\approx_{K} = \approx = \approx_{L}$

\subsubsection{Contextual duality}

Note that contexts extend the quotation operation to a family of
operations from processes to names. Given a context, $M$, we can
define a \emph{nominal context}, $\quotep{M}$ by $\quotep{M}[P] :=
\quotep{M[P]}$. To foreshadow what is to come we observe that these
operations enjoy a duality with processes very much like the duality
between vectors and maps from vectors to scalars.

Further, because the calculus is essentially higher-order, we have a
correspondence between contexts and processes. More specifically,
given a name $x$ and a context $M$ we can construct $M^{*}_{x}$ such
that 

\begin{mathpar}
  M^{*}_{x} | \lift{x}{P} \red M[P]
\end{mathpar}

namely,

\begin{mathpar}
  M^{*}_{x} := x?(u).M[\dropn{u}]
\end{mathpar}

The dependence of $M^{*}_{x}$ on a name makes it an abstraction, 

\begin{mathpar}
  M^{*} := (x)x?(u).M[\dropn{u}]
\end{mathpar}

\subsection{Additional notation}

It will sometimes be convenient to denote the process a name
quotes. We already have the notation $x = \quotep{P}$, but it will be
convenient to introduce an alternate notation, $\procn{x}$, when we
want to emphasize the connection to the use of the name. Note that, by
virtue of name equivalence, $\quotep{\procn{x}} \nameeq x$; so, the
notation is consistent with previous definitions.

Further, because names have structure it is possible to effect
substitutions on the basis of that structure. This means we need to
upgrade our notation for substitutions, which we accomplish by
adapting comprehension notation. Thus,

\begin{mathpar}
  P\{ y / x : x \in S \}
\end{mathpar}

is interpreted to mean the process derived from P by replacing (in a
capture-avoiding manner) each occurrence of $x$ in $S$ by $y$. For example,

\begin{mathpar}
  P\{ \quotep{\procn{x}|\procn{x}} / x : x \in \freenames{P} \}
\end{mathpar}

will replace each (occurrence) of a free name $x$ in $P$ by
$\quotep{\procn{x}|\procn{x}}$.

Also, we will avail ourselves of the notation $x^{L}$ and $x^{R}$ to
denote injections of a name into disjoint copies of the name
space. There are numerous ways to accomplish this. One example can be
found in \cite{MeredithR05}. This notation overloads to vectors of
names: $\vec{x}^{\pi} := (x_{i}^{\pi} \; : \; 0 \leq i < |\vec{x}| )$ where $\pi \in \{L,R\}$.

We also use $P^{\Box} := P|\Box$.

In \cite{MeredithR05} an interpretation of the new operator is
given. It turns out that there are several possible interpretations
all enjoying the requisite algebraic properties of the operator (see
\cite{milner91polyadicpi}). We will therefore make liberal use of
$(\nu\; \vec{x})P$.

% subsection the_syntax_and_semantics_of_the_notation_system (end)   

\input{qm2pi.qmops} 

\input{qm2pi.sterngerlach} 

\input{qm2pi.metric} 

% section concurrent_process_calculi (end)

%\input{qm2pi.proofsketch}

% section proof sketch (end)

%\input{qm2pi.slviaknots} 

% section spatial logic via knots (end)

\input{qm2pi.conclusion}

% section conclusion (end)

%\input{qm2pi.dtcodes} 

% section wiring algorithm (end)

\input{qm2pi.ack} 

% section acknowledgments (end)

\newpage


\bibliographystyle{plain}   
\bibliography{../../biblios/main.bib}

\input{qm2pi.rhodetails}

\end{document}



% section front matter (end)

\section{Introduction}\label{sec:introduction} % (fold)
In this draft of the material i am going to have to dispense with the
usual writing conventions adopted in papers on these topics. i'm going
to have adopt whatever tone i need at the time i'm writing up the
calculations. Sometimes this may be very conversational; others it may
be the barest mathematical grunts; others still it may be that i have
lifted text from one of my other papers because the exposition of some
point was better said there. i hope that my readers are not unduly put
out by this decision. i'm not doing this to flout convention or be
rebellious. i find these calculations very technically challenging. To
keep everything going technically, something has to give; i have to
let go of some cognitive burden. So, the academic writing style --
with all of its trade-offs in terms of facilitating technical
communication -- is what i'm letting go of. Perhaps subsequent drafts
can be tightened and polished, but for now, i'm going to speak as if
we were sitting together in a coffee shop with a laptop, wifi and a
pad of paper and a pencil.

So, here's what i have to say. We -- you and i, comfortably ensconced
in our coffee shop and well-equipped with our tools -- can realize and
carry out the calculations of quantum mechanics over a very different
formal theory of dynamics, a formal theory of dynamics that
corresponds to a theory of concurrent computation with
\emph{reflection}. It has the advantage that the underlying theory is
already `quantized', but supports analogues all of the continuuous
operations. Strikingly, this underlying theory has recently been
connected with a notion of metric that we can show, by calculating
together, coincides with the metric induced by the inner product.

There are a lot of reasons why you might be interested in seeing
calculations of this form. Here's why i'm interested. For the past
several centuries there has been no competitor to the ``Newtonian''
account of dynamics. As a result the predominant share of accounts of
dynamical systems and situations have had to be formulated in terms of
the Newtonian machinery. i view this as an intellectually dangerous
position to occupy. Everything, despite it's intrinsic shape, turns
into a nail to be hit with this hammer. Recently, however, the theory
of computation has matured to the point where we have candidates for
theories of dynamics that offer very different perspective on
reasoning about dynamical systems and situations. Testing these
candidates against very successful accounts of dynamical situations,
like quantum mechanics, is going to give us some sense of how mature
they are and some measure of the quality of these accounts of
dynamics.

\subsection{Summary of contributions and outline of paper}

So, we're going to develop an interpretation of the operations of
quantum mechanics normally interpreted by Hilbert spaces and
operators. We're going to do this over a theory of computation. Note
that this is very different than the usual quantum computation program
which develops notions of computation over quantum mechanics. Rather,
we are developing a story that aligns with Wheeler's slogan: It from
Bit. To do this we will first provide an account of the theory of
computation at play here. Then we will dive into a calculation-driven
interpretation of the operations of quantum mechanics.

The reason we take this approach is that -- until very recently --
there hasn't been an axiomatic account of quantum mechanics. As a
result there has been no sharp delineation of the mathematical theory
supporting interpretation of the physical theory and the physical
theory, itself. So, ambient features of the maths are free to be
exploited (or supressed) without a real accounting of their physical
relevance. There is no sharp statement ``here's the physical theory''
qua \emph{theory} and ``here's the mathematical interpretation''
enabling a judgment of how faithful the interpretation is -- apart
from experimental observation. When there is an axiomatic account we
can judge how well a given mathematical formalism supports an
interpretation of the axioms, independent of
experimentation. Likewise, we can judge how well we have captured our
physical evidence and experience with our axiomatics, independent of
any specific mathematical implementation, with accidental detail that
may or may not have physical significance. 

In lieu of a fully fleshed out and vetted axiomatic account of quantum
mechanics, interpreting the operational notions in service of modeling
physical systems will have to suffice. In other words, we are not in
the business of providing a model of Hilbert spaces and operators. We
are in the business of providing a model of quantum mechanics because
we are motivated by testing our notions of dynamics against physical
theory; and, the predictive calculations of the physical theory must
serve as the best formulation -- shy of a fully fleshed out axiomatic
account -- of the physical theory itself (as they have for scientific
theories since time immemorial). Put another way, despite a
whole-hearted commitment to an It-from-Bit ontology, we are firmly
aligned with the shut-up-and-calculate camp as the best way to obtain
results either from the physical perspective or as a quality assurance
measure of our fledgling theory of dynamics.

In detail, we present a reflective process calculus. Then we develop
intuitive correspondences between the notions available in this
calculus and the usual physical notions supporting quantum mechanical
calculations. Thus, 

\begin{table}[htp]
  \center{
    \fbox{
      \begin{tabular}{c|c}
        quantum mechanics & process calculus \\
        \hline
        scalar & name \\
        state vector & process \\
        dual & contextual duals \\
        matrix & formal sums of process-context-dual pairs \\
        orthogonality & process annihilation \\
        inner product & execution-formula + quoting
      \end{tabular}
    }
  }
  \caption{QM - process calculi correspondences}
\end{table}

Then we tighten up these intuitions to operational definitions. We
employ the Dirac notation as the best proxy we can find for an
abstract syntax of the quantum mechanical notions. The definitions we
develop put us in contact with equational constraints coming from the
theory that we demonstrate the definitions and calculations satisfy.

This puts us in a position to shut up and calculate for the
Stern-Gerlach experimental set up, showing how these predictive
calculations become calculations on processes in our theory of a
reflective process calculus.

Penultimately, we demonstrate that the notion of metric coming from
the inner product coincides with the notion of metric available from
the theory of bisimulation. This demonstration gives us the right to
think of space as arising from behavior. Finally, we consider where we
might go from the new vantage point we have obtained.

% section introduction (end) 
 
% section introduction (end)

% \documentclass[12pt]{llncs}
%\documentclass{jktr}

\usepackage[pdftex]{hyperref}                   
\usepackage {listings}
\usepackage {mathpartir}
\usepackage{bcprules}
%\usepackage{listings}
                       
\usepackage{graphicx} 
%\usepackage[margins=2.5cm,nohead,nofoot]{geometry}
%\usepackage{geometry}
\usepackage{amsfonts}
\usepackage{amstext}
\usepackage{latexsym}
\usepackage{amssymb}
\usepackage{color}


%\include{myPreamble}
\include{qm2pi.local} 

%\ifpdf
%\usepackage[pdftex]{graphicx}
%\else
%\usepackage{graphicx}
%\fi

 % \ifpdf
%  \usepackage{pdfsync}
%  \if


%\title{Brief Article}
%\author{David F. Snyder}
%\author{L.G. Meredith}

%\address{Dept. of Math., Texas State University--San Marcos, San Marcos, TX 78666}
       
\pagestyle{empty}


\begin{document}

\lstset{language=[Objective]Caml,frame=shadowbox}

\input{qm2pi.front}

% section front matter (end)

\input{qm2pi.intro} 
 
% section introduction (end)

% \input{qm2pi.knotations} 

% section notation (end)

\input{qm2pi.process.calculi} 

% section concurrent_process_calculi_and_spatial_logics_ (end)
    
%\input{qm2pi.knots2pi} 

%\input{qm2pi.trefoil} 

%\input{qm2pi.mainthm} 

% subsection basic_interpretation (end)

%\input{qm2pi.rho.presentation} 
\subsection{The syntax and semantics of the notation system}\label{sub:the_syntax_and_semantics_of_the_notation_system} % (fold)

We now summarize a technical presentation of the calculus that
embodies our theory of dynamics. The typical presentation of such a
calculus follows the style of giving generators and relations on
them. The grammar, below, describing term constructors, freely
generates the set of processes, $\Proc$. This set is then quotiented
by a relation known as structural congruence and it is over this set
that the notion of dynamics is expressed. This presentation is
essentially that of \cite{MeredithR05} with the addition of
polyadicity and summation. For readability we have relegated some of
the technical subtleties to an appendix.

\subsubsection{Process grammar}\label{subsub:process_grammar}

\begin{mathpar}
  \inferrule* [lab=synchronization] {} {{M} \bc \pzero \;|\; x?F \;|\; x!C }
  \and
  \inferrule* [lab=abstraction] {} {{F} \bc (x)P}
  \and
  \inferrule* [lab=concretion] {} {{C} \bc \langle Q \rangle}
  \and
  \inferrule* [lab=process] {} {{P,Q} \bc M \;| \;P|Q \;|\; @{x}}
  \and
  \inferrule* [lab=name] {} {{x} \bc \quotep{P}}
\end{mathpar} 

Note that $\vec{x}$ (resp. $\vec{P}$) denotes a vector of names
(resp. processes) of length $|\vec{x}|$ (resp. $|\vec{P}|$). We adopt
the following useful abbreviations.

\begin{mathpar}
   x?(\vec{y}).P := x.(\vec{y})P \and  x\clift{\vec{P}} := x.\clift{\vec{P}}
   \and x!(y) := \lift{x}{\dropn{y}}
   \and \Pi_{i=0}^{n-1}P_i := P_0 | \ldots | P_{n-1}
\end{mathpar}

\subsubsection{Structural congruence}

\paragraph{Free and bound names and alpha-equivalence.} At the
core of structural equivalence is alpha-equivalence which identifies
process that are the same up to a change of variable. Formally, we
recognize the distinction between free and bound names. The free names
of a process, $\freenames{P}$, may be calculated recursively as
follows:

\begin{mathpar}
\freenames{\pzero} := \emptyset
  \and \\
  \freenames{x?(y).P} := \{ x \} \cup (\freenames{P} \setminus \{ y \})
  \and 
  \freenames{x!\langle P \rangle} := \{ x \} \cup \{ P \} 
  \and \\
  \freenames{P|Q} := \freenames{P} \cup \freenames{Q}
  \and \\
  \freenames{@{x}} := \{ x \}
\end{mathpar}

$\pi$
$\quotep{\pi}$

$\freenames{-} : \pi \to \mathcal{P}(\quotep{\pi})$

\begin{eqnarray*}
  \freenames{\pzero} & := & \emptyset \\
  \freenames{x?(y).P} & := & \{ x \} \cup (\freenames{P} \setminus \{ y \}) \\
  \freenames{x!\langle P \rangle} & := & \{ x \} \cup \{ P \} \\
  \freenames{P|Q} & := & \freenames{P} \cup \freenames{Q} \\
  \freenames{\dropn{x}} & := & \{ x \}
\end{eqnarray*}

The bound names of a process, $\boundnames{P}$, are those names occurring in $P$
that are not free. For example, in $x?(y).0$, the name $x$ is free, while $y$ is bound.

\begin{mathpar}
  \inferrule* [lab=monoidal-laws] {} { P|Q \equiv Q|P \and P|0 \equiv P \and P|(Q|R) \equiv (P|Q)|R }
\end{mathpar}

\begin{mathpar}
  \inferrule* [lab=alpha-equivalence] {} { (x)P \equiv (y)P\{y/x\} \and y \not\in \freenames{P} }
\end{mathpar}

\begin{definition}
Then two processes, $P,Q$, are alpha-equivalent if $P = Q\{\vec{y}/\vec{x}\}$ for
some $\vec{x} \in \boundnames{Q},\vec{y} \in \boundnames{P}$, where $Q\{\vec{y}/\vec{x}\}$
denotes the capture-avoiding substitution of $\vec{y}$ for $\vec{x}$ in $Q$.
\end{definition}

\begin{definition}
  The {\em structural congruence} \cite{SangiorgiWalker} , $\equiv$,
  between processes is the least congruence containing
  alpha-equivalence, satisfying the abelian monoid laws
  (associativity, commutativity and $\pzero$ as identity) for parallel
  composition $|$ and for summation $+$.
\end{definition}

\subsection{Name equivalence}

We take name equivalence, written $\nameeq$, to be the smallest
equivalence relation generated by the following rules.

\begin{mathpar}
\inferrule*[lab=Quote-drop]
{ }
{ \quotep{@{x}} \nameeq x }

\inferrule*[lab=Struct-equiv]
{ P \scong Q }
{ \quotep{P} \nameeq \quotep{Q} }
\end{mathpar}

The astute reader will have noticed that the mutual recursion of names
and processes imposes a mutual recursion on alpha-equivalence and
structural equivalence via name-equivalence. Fortunately, all of this
works out pleasantly and we may calculate in the natural way, free of
concern. The reader interested in the details is referred to the
appendix \ref{appendix:rho_details}.

\subsection{Substitution}

We use $\Proc$ for the set of processes, $\QProc$ for the set of
names, and $\id{\{}\vec{y} / \vec{x} \id{\}}$ to denote partial maps,
$s : \QProc \rightarrow \QProc$. A map, $s$ lifts, uniquely, to a map
on process terms, $\widehat{s} : \Proc \rightarrow \Proc$ by the
following equations.

\begin{mathpar}
  (0) \psubstp{Q}{P} := 0 \\
  (R \juxtap S) \psubstp{Q}{P}
  :=    
  (R)\psubstp{Q}{P} \juxtap (S) \psubstp{Q}{P} \\
  (x?(y).R) \psubstp{Q}{P}    
  :=    
  (x)\substp{Q}{P} (z)\concat( (R \psubstn{z}{y}) \psubstp{Q}{P} ) \\
  (\lift{x}{R}) \psubstp{Q}{P}  
  :=
  \lift{(x)\substp{Q}{P}}{ R \psubstp{Q}{P} } \\
%   (\dropn{x})  \psubstp{Q}{P}       
%   := 
%   \left\{ 
%     \begin{array}{ccc} 
%       \dropn{\quotep{Q}} & & x \nameeq \quotep{P} \\
%       \dropn{x} & & otherwise \\
%     \end{array}
%   \right. 
  (\dropn{x})  \psubstp{Q}{P}       
  := 
  \left\{ 
    \begin{array}{ccc} 
      Q & & x \nameeq \quotep{P} \\
      \dropn{x} & & otherwise \\
    \end{array}
  \right.
\end{mathpar}
 

where

\begin{eqnarray}
  (x)\id{\{} \lpquote Q \rpquote / \lpquote P \rpquote \id{\}}            = 
  \left\{ 
    \begin{array}{ccc}
      \lpquote Q \rpquote & & x \nameeq \lpquote P \rpquote \\
      x & & otherwise \\
    \end{array}
  \right. \nonumber
\end{eqnarray}

and $z$ is chosen distinct from $\quotep{P}$, $\quotep{Q}$, the free
names in $Q$, and all the names in $R$. Our $\alpha$-equivalence will
be built in the standard way from this substitution.

\begin{remark}\label{rem:no_self_referential_names}
  One consequence of these definitions is that $\forall P. \quotep{P}
  \not\in \freenames{P}$.
\end{remark}

\subsection{ Dynamic quote: an example }

Anticipating something of what's to come, consider applying the
substitution, $\widehat{\id{\{}u / z \id{\}}}$, to the following pair
of processes, $\lift{w}{y!(z)}$ and $w[ \lpquote y!(z) \rpquote ]$.

\begin{eqnarray}
	\lift{w}{y!(z)}\widehat{\id{\{}u / z \id{\}}}
		& = &
		\lift{w}{y!(u)} \nonumber\\
	w[ \lpquote y!(z) \rpquote ] \widehat{ \id{\{}u / z \id{\}} }
		& = &
		w[ \lpquote y!(z) \rpquote ] \nonumber
\end{eqnarray}

Because the body of the process between quotes is impervious to
substitution, we get radically different answers. In fact, by
examining the first process in an input context,
e.g. $x?(z).\lift{w}{y!(z)}$, we see that the process under the lift
operator may be shaped by prefixed inputs binding a name inside it. In
this sense, the lift operator will be seen as a way to dynamically
construct processes before reifying them as names.

Finally equipped with these standard features we can present the
dynamics of the calculus.

\subsubsection{Operational semantics} 

Finally, we introduce the computational dynamics. What marks these
algebras as distinct from other more traditionally studied algebraic
structures, e.g. vector spaces or polynomial rings, is the manner in
which dynamics is captured. In traditional structures, dynamics is typically
expressed through morphisms between such structures, as in linear maps
between vector spaces or morphisms between rings. In algebras
associated with the semantics of computation, the dynamics is
expressed as part of the algebraic structure itself, through a
reduction reduction relation typically denoted by $\red$. Below, we
give a recursive presentation of this relation for the calculus used
in the encoding.

$\red \subseteq \pi \times \pi$
$\red : \pi \to \mathcal{P}(\pi)$

\begin{mathpar}
  \inferrule* [lab=Comm] { \textsf{match}( x_{src}, x_{trgt} ) } { x_{trgt}?(y)P \; | \; x_{src}!\langle {Q} \rangle \red P\{\quotep{Q}/y}\} }
  \and \\
  \inferrule* [lab=Par] {{P} \red {P}'} {{{P} | {Q}} \red {{P}' | {Q}}}
  \and
  \inferrule* [lab=Equiv]{{{P} \scong {P}'} \andalso {{P}' \red {Q}'} \andalso {{Q}' \scong {Q}}}{{P} \red {Q}}
\end{mathpar}

\begin{eqnarray*}
  match_{\equiv} (\quotep{P},\quotep{Q}) & := & P \equiv Q \\
  match_{\dagger}(\quotep{P},\quotep{Q}) & := & \forall R. P|Q \red^{*} R => R \red^{*} 0 \\
  match_{K}(\quotep{P},\quotep{Q}) & := & K \mbox{ for some context } K
\end{eqnarray*}

$u?(x)P | u!\langle Q \rangle \red P\{\quotep{Q}/x\}$

%We write $\wred$ for $\red^*$, and $P\red$ if $\exists Q $ such that $ P \red Q$.
We write $P\red$ if $\exists Q $ such that $ P \red Q$ and $P\not\red$, otherwise.

\section{Replication}

As mentioned before, it is known that replication (and hence
recursion) can be implemented in a higher-order process algebra
\cite{SangiorgiWalker}. As our first example of calculation with the
machinery thus far presented we give the construction explicitly in
the {\rhoc}.

\begin{eqnarray}
	D_{x} & := & \prefix{x}{y}{(\binpar{\outputp{x}{y}}{@{y}})} \nonumber\\
	\bangp_{x}{P} & := & \binpar{{x}!\langle{\binpar{D_{x}}{P}}\rangle}{D_{x}} \nonumber
\end{eqnarray}

\begin{eqnarray}
	\bangp_{x}{P} & & \nonumber\\
	=
	& {x}!\langle{(\prefix{x}{y}{(\outputp{x}{y} | @{y})) | P}}\rangle 
	      | \prefix{x}{y}{(\outputp{x}{y} | @{y})} & \nonumber\\
	\red
	& (\outputp{x}{y} | @{y})\substn{\quotep{(\prefix{x}{y}{(@{y} | \outputp{x}{y})) | P}}}{y} & \nonumber\\
	=
	& \outputp{x}{\quotep{(\prefix{x}{y}{(\outputp{x}{y} | @{y})) | P}}}
	  | {(\prefix{x}{y}{(\outputp{x}{y} | @{y})) | P}} & \nonumber\\
	\red
	& \ldots & \nonumber\\
	\red^*
	& P | P | \ldots & \nonumber
\end{eqnarray}

Of course, this encoding, as an implementation, runs away, unfolding
$\bangp{P}$ eagerly. A lazier and more implementable replication
operator, restricted to input-guarded processes, may be obtained as follows.

\begin{eqnarray}
\bangp{\prefix{u}{v}{P}} 
	:= 
	\binpar{\lift{x}{\prefix{u}{v}{(\binpar{D(x)}{P})}}}{D(x)} \nonumber
\end{eqnarray}

\begin{remark}
  Note that the lazier definition still does not deal with summation
  or mixed summation (i.e. sums over input and output). The reader is
  invited to construct definitions of replication that deal with these
  features. 

  Further, the definitions are parameterized in a name, $x$. Can you,
  gentle reader, make a definition that eliminates this parameter and
  guarantees no accidental interaction between the replication
  machinery and the process being replicated -- i.e. no accidental
  sharing of names used by the process to get its work done and the
  name(s) used by the replication to effect copying. This latter
  revision of the definition of replication is crucial to obtaining
  the expected identity $!!P \sim !P$.
\end{remark}

\begin{remark}\label{rem:paradoxical_combinator}
  The reader familiar with the lambda calculus will have noticed the
  similarity between $D$ and the paradoxical combinator.

  [Ed. note: the existence of this seems to suggest we have to be more
  restrictive on the set of processes and names we admit if we are to
  support no-cloning.]
\end{remark}

\subsubsection{Bisimulation}

The computational dynamics gives rise to another kind of equivalence,
the equivalence of computational behavior. As previously mentioned
this is typically captured \emph{via} some form of bisimulation.

% The notion we use in this paper is weak barbed bisimulation
% \cite{milner91polyadicpi}.

The notion we use in this paper is derived from weak barbed
bisimulation \cite{milner91polyadicpi}. 

\begin{definition}
An \emph{observation relation}, $\downarrow_{\mathcal N}$, over a set
of names, $\mathcal N$, is the smallest relation satisfying the rules
below.

\infrule[Out-barb]{y \in {\mathcal N}, \; x \nameeq y}
		  {\outputp{x}{v} \downarrow_{\mathcal N} x}
\infrule[Par-barb]{\mbox{$P\downarrow_{\mathcal N} x$ or $Q\downarrow_{\mathcal N} x$}}
		  {\binpar{P}{Q} \downarrow_{\mathcal N} x}

We write $P \Downarrow_{\mathcal N} x$ if there is $Q$ such that 
$P \wred Q$ and $Q \downarrow_{\mathcal N} x$.
\end{definition}

\begin{definition}
%\label{def.bbisim}
An  ${\mathcal N}$-\emph{barbed bisimulation} over a set of names, ${\mathcal N}$, is a symmetric binary relation 
${\mathcal S}_{\mathcal N}$ between agents such that $P\rel{S}_{\mathcal N}Q$ implies:
\begin{enumerate}
\item If $P \red P'$ then $Q \wred Q'$ and $P'\rel{S}_{\mathcal N} Q'$.
\item If $P\downarrow_{\mathcal N} x$, then $Q\Downarrow_{\mathcal N} x$.
\end{enumerate}
$P$ is ${\mathcal N}$-barbed bisimilar to $Q$, written
$P \wbbisim_{\mathcal N} Q$, if $P \rel{S}_{\mathcal N} Q$ for some ${\mathcal N}$-barbed bisimulation ${\mathcal S}_{\mathcal N}$.
\end{definition}

$\mathcal{R} \subseteq \pi \times \pi$

$P \mathcal{R} Q => \forall P'. P \red P' \Rightarrow \exists Q'. Q \red Q', P' \mathcal{R} Q'$

$P \vdash x \Rightarrow Q \vdash x$

\begin{mathpar}
  \inferrule*[lab=Out-barb]{x \nameeq y}{{y}!\langle{Q}\rangle \vdash x}
  \and
  \inferrule*[lab=Par-barb]{\mbox{$P\vdash x$ or $Q\vdash x$}}{\binpar{P}{Q} \vdash x}
\end{mathpar}

\subsubsection{Contexts}

One of the principle advantages of computational calculi like the
$\pi$-calculus is a well-defined notion of context,
contextual-equivalence and a correlation between
contextual-equivalence and notions of bisimulation. The notion of
context allows the decomposition of a process into (sub-)process and
its syntactic environment, its context. Thus, a context may be
thought of as a process with a ``hole'' (written $\Box$) in it. The
application of a context $M$ to a process $P$, written $M[P]$, is
tantamount to filling the hole in $M$ with $P$. In this paper we do
not need the full weight of this theory, but do make use of the notion
of context in the proof the main theorem. 

\begin{mathpar}
  \inferrule* [lab=summation] {} {{M_{M},M_{N}} \bc \Box \;|\; x.M_{A} \;|\; M_{M}+M_{N}}
  \and
  \inferrule* [lab=agent] {} {{M_{A}} \bc (\vec{x})M_{P} \;| \; \clift{P_0,\ldots,M_{P},\ldots,P_N}}
  \and \\
  \inferrule* [lab=process] {} {{M_{P}} \bc M_{N} \;| \;P|M_{P} }
\end{mathpar} 

\begin{mathpar}
  \inferrule* [lab=sychronization] {} {M_{N} \bc \Box \;|\; x?M_{F} \;|\; x!M_{C}}
  \and
  \inferrule* [lab=abstraction] {} {{M_{F}} \bc (x)M_{P} }
  \and
  \inferrule* [lab=concretion] {} {{M_{C}} \bc \langle M_{P} \rangle }
  \and \\
  \inferrule* [lab=process] {} {{M_{P}} \bc M_{N} \;| \;P|M_{P} }
\end{mathpar}

\begin{definition}[contextual application] Given a context $M$, and
  process $P$, we define the \emph{contextual application}, $M[P] :=
  M\{P/\Box\}$. That is, the contextual application of M to P is the
  substitution of $P$ for $\Box$ in $M$.
\end{definition}

$\meaningof{-} : L \to \mathcal{P}(\pi)$

\begin{mathpar}
  \inferrule* [lab=collection] {} {\meaningof{true} = \pi, \and \meaningof{~E} = \pi \setminus \meaningof{E}, \and \meaningof{E_{1} \& E_{2}} = \meaningof{E_{1}} \cap \meaningof{E_{2}}}
\end{mathpar}

\begin{mathpar}
  \inferrule* [lab=structure] {} {\meaningof{0} = \{ P \in \pi | P \equiv 0 \}, \and \\ \meaningof{E_1 | E_2} = \{ P \in \pi | P \equiv P_{1} | P_{2}, P_{1} \in \meaningof{E_{1}}, P_{2} \in \meaningof{E_2}\} }
\end{mathpar}

\begin{mathpar}
 \inferrule* [lab=behavior] {} {\meaningof{\langle a?b \rangle E} = \{ P \in \pi | P \equiv Q | u?(y)P', \\ \and \\\\ \and \\ \;\;\; u \in \meaningof{a}, \forall z.P'\{z/y\} \in \meaningof{E\{z/b\}}\}, \and \\ \meaningof{a!E} = \{ P \in \pi | P \equiv Q | x!\langle P' \rangle, x \in \meaningof{a} P' \in \meaningof{E}\} }
\end{mathpar}

\begin{mathpar}
 \inferrule* [lab=nominal] {} {\meaningof{\quotep{E}} = \{ \quotep{P} \in \quotep{\pi} | P \in \meaningof{E} \}, \and \meaningof{\quotep{P}} = \{ \quotep{Q} \in \quotep{\pi} | P \equiv Q \} \and \\ \meaningof{@\quotep{E}} = \{ P \in \pi | P \equiv @x, x \in \meaningof{E} \}}
\end{mathpar}

\begin{eqnarray*}
  \\
  \meaningof{-} : TS \to ST
\end{eqnarray*}

\begin{eqnarray*}
  \\
  L : TS \to ST
\end{eqnarray*}

\begin{eqnarray*}
  \\
  P \models E \iff P \in \meaningof{E}
\end{eqnarray*}

\begin{eqnarray*}
  P \approx_{L} Q \iff \forall E \in L. P \models E \iff Q \models E
\end{eqnarray*}

\begin{eqnarray*}
  P \approx_{K} Q
\end{eqnarray*}

\begin{eqnarray*}
  P \approx Q
\end{eqnarray*}

$\approx_{K} = \approx = \approx_{L}$

\subsubsection{Contextual duality}

Note that contexts extend the quotation operation to a family of
operations from processes to names. Given a context, $M$, we can
define a \emph{nominal context}, $\quotep{M}$ by $\quotep{M}[P] :=
\quotep{M[P]}$. To foreshadow what is to come we observe that these
operations enjoy a duality with processes very much like the duality
between vectors and maps from vectors to scalars.

Further, because the calculus is essentially higher-order, we have a
correspondence between contexts and processes. More specifically,
given a name $x$ and a context $M$ we can construct $M^{*}_{x}$ such
that 

\begin{mathpar}
  M^{*}_{x} | \lift{x}{P} \red M[P]
\end{mathpar}

namely,

\begin{mathpar}
  M^{*}_{x} := x?(u).M[\dropn{u}]
\end{mathpar}

The dependence of $M^{*}_{x}$ on a name makes it an abstraction, 

\begin{mathpar}
  M^{*} := (x)x?(u).M[\dropn{u}]
\end{mathpar}

\subsection{Additional notation}

It will sometimes be convenient to denote the process a name
quotes. We already have the notation $x = \quotep{P}$, but it will be
convenient to introduce an alternate notation, $\procn{x}$, when we
want to emphasize the connection to the use of the name. Note that, by
virtue of name equivalence, $\quotep{\procn{x}} \nameeq x$; so, the
notation is consistent with previous definitions.

Further, because names have structure it is possible to effect
substitutions on the basis of that structure. This means we need to
upgrade our notation for substitutions, which we accomplish by
adapting comprehension notation. Thus,

\begin{mathpar}
  P\{ y / x : x \in S \}
\end{mathpar}

is interpreted to mean the process derived from P by replacing (in a
capture-avoiding manner) each occurrence of $x$ in $S$ by $y$. For example,

\begin{mathpar}
  P\{ \quotep{\procn{x}|\procn{x}} / x : x \in \freenames{P} \}
\end{mathpar}

will replace each (occurrence) of a free name $x$ in $P$ by
$\quotep{\procn{x}|\procn{x}}$.

Also, we will avail ourselves of the notation $x^{L}$ and $x^{R}$ to
denote injections of a name into disjoint copies of the name
space. There are numerous ways to accomplish this. One example can be
found in \cite{MeredithR05}. This notation overloads to vectors of
names: $\vec{x}^{\pi} := (x_{i}^{\pi} \; : \; 0 \leq i < |\vec{x}| )$ where $\pi \in \{L,R\}$.

We also use $P^{\Box} := P|\Box$.

In \cite{MeredithR05} an interpretation of the new operator is
given. It turns out that there are several possible interpretations
all enjoying the requisite algebraic properties of the operator (see
\cite{milner91polyadicpi}). We will therefore make liberal use of
$(\nu\; \vec{x})P$.

% subsection the_syntax_and_semantics_of_the_notation_system (end)   

\input{qm2pi.qmops} 

\input{qm2pi.sterngerlach} 

\input{qm2pi.metric} 

% section concurrent_process_calculi (end)

%\input{qm2pi.proofsketch}

% section proof sketch (end)

%\input{qm2pi.slviaknots} 

% section spatial logic via knots (end)

\input{qm2pi.conclusion}

% section conclusion (end)

%\input{qm2pi.dtcodes} 

% section wiring algorithm (end)

\input{qm2pi.ack} 

% section acknowledgments (end)

\newpage


\bibliographystyle{plain}   
\bibliography{../../biblios/main.bib}

\input{qm2pi.rhodetails}

\end{document}

 

% section notation (end)

\input{qm2pi.process.calculi} 

% section concurrent_process_calculi_and_spatial_logics_ (end)
    
%\documentclass[12pt]{llncs}
%\documentclass{jktr}

\usepackage[pdftex]{hyperref}                   
\usepackage {listings}
\usepackage {mathpartir}
\usepackage{bcprules}
%\usepackage{listings}
                       
\usepackage{graphicx} 
%\usepackage[margins=2.5cm,nohead,nofoot]{geometry}
%\usepackage{geometry}
\usepackage{amsfonts}
\usepackage{amstext}
\usepackage{latexsym}
\usepackage{amssymb}
\usepackage{color}


%\include{myPreamble}
\include{qm2pi.local} 

%\ifpdf
%\usepackage[pdftex]{graphicx}
%\else
%\usepackage{graphicx}
%\fi

 % \ifpdf
%  \usepackage{pdfsync}
%  \if


%\title{Brief Article}
%\author{David F. Snyder}
%\author{L.G. Meredith}

%\address{Dept. of Math., Texas State University--San Marcos, San Marcos, TX 78666}
       
\pagestyle{empty}


\begin{document}

\lstset{language=[Objective]Caml,frame=shadowbox}

\input{qm2pi.front}

% section front matter (end)

\input{qm2pi.intro} 
 
% section introduction (end)

% \input{qm2pi.knotations} 

% section notation (end)

\input{qm2pi.process.calculi} 

% section concurrent_process_calculi_and_spatial_logics_ (end)
    
%\input{qm2pi.knots2pi} 

%\input{qm2pi.trefoil} 

%\input{qm2pi.mainthm} 

% subsection basic_interpretation (end)

%\input{qm2pi.rho.presentation} 
\subsection{The syntax and semantics of the notation system}\label{sub:the_syntax_and_semantics_of_the_notation_system} % (fold)

We now summarize a technical presentation of the calculus that
embodies our theory of dynamics. The typical presentation of such a
calculus follows the style of giving generators and relations on
them. The grammar, below, describing term constructors, freely
generates the set of processes, $\Proc$. This set is then quotiented
by a relation known as structural congruence and it is over this set
that the notion of dynamics is expressed. This presentation is
essentially that of \cite{MeredithR05} with the addition of
polyadicity and summation. For readability we have relegated some of
the technical subtleties to an appendix.

\subsubsection{Process grammar}\label{subsub:process_grammar}

\begin{mathpar}
  \inferrule* [lab=synchronization] {} {{M} \bc \pzero \;|\; x?F \;|\; x!C }
  \and
  \inferrule* [lab=abstraction] {} {{F} \bc (x)P}
  \and
  \inferrule* [lab=concretion] {} {{C} \bc \langle Q \rangle}
  \and
  \inferrule* [lab=process] {} {{P,Q} \bc M \;| \;P|Q \;|\; @{x}}
  \and
  \inferrule* [lab=name] {} {{x} \bc \quotep{P}}
\end{mathpar} 

Note that $\vec{x}$ (resp. $\vec{P}$) denotes a vector of names
(resp. processes) of length $|\vec{x}|$ (resp. $|\vec{P}|$). We adopt
the following useful abbreviations.

\begin{mathpar}
   x?(\vec{y}).P := x.(\vec{y})P \and  x\clift{\vec{P}} := x.\clift{\vec{P}}
   \and x!(y) := \lift{x}{\dropn{y}}
   \and \Pi_{i=0}^{n-1}P_i := P_0 | \ldots | P_{n-1}
\end{mathpar}

\subsubsection{Structural congruence}

\paragraph{Free and bound names and alpha-equivalence.} At the
core of structural equivalence is alpha-equivalence which identifies
process that are the same up to a change of variable. Formally, we
recognize the distinction between free and bound names. The free names
of a process, $\freenames{P}$, may be calculated recursively as
follows:

\begin{mathpar}
\freenames{\pzero} := \emptyset
  \and \\
  \freenames{x?(y).P} := \{ x \} \cup (\freenames{P} \setminus \{ y \})
  \and 
  \freenames{x!\langle P \rangle} := \{ x \} \cup \{ P \} 
  \and \\
  \freenames{P|Q} := \freenames{P} \cup \freenames{Q}
  \and \\
  \freenames{@{x}} := \{ x \}
\end{mathpar}

$\pi$
$\quotep{\pi}$

$\freenames{-} : \pi \to \mathcal{P}(\quotep{\pi})$

\begin{eqnarray*}
  \freenames{\pzero} & := & \emptyset \\
  \freenames{x?(y).P} & := & \{ x \} \cup (\freenames{P} \setminus \{ y \}) \\
  \freenames{x!\langle P \rangle} & := & \{ x \} \cup \{ P \} \\
  \freenames{P|Q} & := & \freenames{P} \cup \freenames{Q} \\
  \freenames{\dropn{x}} & := & \{ x \}
\end{eqnarray*}

The bound names of a process, $\boundnames{P}$, are those names occurring in $P$
that are not free. For example, in $x?(y).0$, the name $x$ is free, while $y$ is bound.

\begin{mathpar}
  \inferrule* [lab=monoidal-laws] {} { P|Q \equiv Q|P \and P|0 \equiv P \and P|(Q|R) \equiv (P|Q)|R }
\end{mathpar}

\begin{mathpar}
  \inferrule* [lab=alpha-equivalence] {} { (x)P \equiv (y)P\{y/x\} \and y \not\in \freenames{P} }
\end{mathpar}

\begin{definition}
Then two processes, $P,Q$, are alpha-equivalent if $P = Q\{\vec{y}/\vec{x}\}$ for
some $\vec{x} \in \boundnames{Q},\vec{y} \in \boundnames{P}$, where $Q\{\vec{y}/\vec{x}\}$
denotes the capture-avoiding substitution of $\vec{y}$ for $\vec{x}$ in $Q$.
\end{definition}

\begin{definition}
  The {\em structural congruence} \cite{SangiorgiWalker} , $\equiv$,
  between processes is the least congruence containing
  alpha-equivalence, satisfying the abelian monoid laws
  (associativity, commutativity and $\pzero$ as identity) for parallel
  composition $|$ and for summation $+$.
\end{definition}

\subsection{Name equivalence}

We take name equivalence, written $\nameeq$, to be the smallest
equivalence relation generated by the following rules.

\begin{mathpar}
\inferrule*[lab=Quote-drop]
{ }
{ \quotep{@{x}} \nameeq x }

\inferrule*[lab=Struct-equiv]
{ P \scong Q }
{ \quotep{P} \nameeq \quotep{Q} }
\end{mathpar}

The astute reader will have noticed that the mutual recursion of names
and processes imposes a mutual recursion on alpha-equivalence and
structural equivalence via name-equivalence. Fortunately, all of this
works out pleasantly and we may calculate in the natural way, free of
concern. The reader interested in the details is referred to the
appendix \ref{appendix:rho_details}.

\subsection{Substitution}

We use $\Proc$ for the set of processes, $\QProc$ for the set of
names, and $\id{\{}\vec{y} / \vec{x} \id{\}}$ to denote partial maps,
$s : \QProc \rightarrow \QProc$. A map, $s$ lifts, uniquely, to a map
on process terms, $\widehat{s} : \Proc \rightarrow \Proc$ by the
following equations.

\begin{mathpar}
  (0) \psubstp{Q}{P} := 0 \\
  (R \juxtap S) \psubstp{Q}{P}
  :=    
  (R)\psubstp{Q}{P} \juxtap (S) \psubstp{Q}{P} \\
  (x?(y).R) \psubstp{Q}{P}    
  :=    
  (x)\substp{Q}{P} (z)\concat( (R \psubstn{z}{y}) \psubstp{Q}{P} ) \\
  (\lift{x}{R}) \psubstp{Q}{P}  
  :=
  \lift{(x)\substp{Q}{P}}{ R \psubstp{Q}{P} } \\
%   (\dropn{x})  \psubstp{Q}{P}       
%   := 
%   \left\{ 
%     \begin{array}{ccc} 
%       \dropn{\quotep{Q}} & & x \nameeq \quotep{P} \\
%       \dropn{x} & & otherwise \\
%     \end{array}
%   \right. 
  (\dropn{x})  \psubstp{Q}{P}       
  := 
  \left\{ 
    \begin{array}{ccc} 
      Q & & x \nameeq \quotep{P} \\
      \dropn{x} & & otherwise \\
    \end{array}
  \right.
\end{mathpar}
 

where

\begin{eqnarray}
  (x)\id{\{} \lpquote Q \rpquote / \lpquote P \rpquote \id{\}}            = 
  \left\{ 
    \begin{array}{ccc}
      \lpquote Q \rpquote & & x \nameeq \lpquote P \rpquote \\
      x & & otherwise \\
    \end{array}
  \right. \nonumber
\end{eqnarray}

and $z$ is chosen distinct from $\quotep{P}$, $\quotep{Q}$, the free
names in $Q$, and all the names in $R$. Our $\alpha$-equivalence will
be built in the standard way from this substitution.

\begin{remark}\label{rem:no_self_referential_names}
  One consequence of these definitions is that $\forall P. \quotep{P}
  \not\in \freenames{P}$.
\end{remark}

\subsection{ Dynamic quote: an example }

Anticipating something of what's to come, consider applying the
substitution, $\widehat{\id{\{}u / z \id{\}}}$, to the following pair
of processes, $\lift{w}{y!(z)}$ and $w[ \lpquote y!(z) \rpquote ]$.

\begin{eqnarray}
	\lift{w}{y!(z)}\widehat{\id{\{}u / z \id{\}}}
		& = &
		\lift{w}{y!(u)} \nonumber\\
	w[ \lpquote y!(z) \rpquote ] \widehat{ \id{\{}u / z \id{\}} }
		& = &
		w[ \lpquote y!(z) \rpquote ] \nonumber
\end{eqnarray}

Because the body of the process between quotes is impervious to
substitution, we get radically different answers. In fact, by
examining the first process in an input context,
e.g. $x?(z).\lift{w}{y!(z)}$, we see that the process under the lift
operator may be shaped by prefixed inputs binding a name inside it. In
this sense, the lift operator will be seen as a way to dynamically
construct processes before reifying them as names.

Finally equipped with these standard features we can present the
dynamics of the calculus.

\subsubsection{Operational semantics} 

Finally, we introduce the computational dynamics. What marks these
algebras as distinct from other more traditionally studied algebraic
structures, e.g. vector spaces or polynomial rings, is the manner in
which dynamics is captured. In traditional structures, dynamics is typically
expressed through morphisms between such structures, as in linear maps
between vector spaces or morphisms between rings. In algebras
associated with the semantics of computation, the dynamics is
expressed as part of the algebraic structure itself, through a
reduction reduction relation typically denoted by $\red$. Below, we
give a recursive presentation of this relation for the calculus used
in the encoding.

$\red \subseteq \pi \times \pi$
$\red : \pi \to \mathcal{P}(\pi)$

\begin{mathpar}
  \inferrule* [lab=Comm] { \textsf{match}( x_{src}, x_{trgt} ) } { x_{trgt}?(y)P \; | \; x_{src}!\langle {Q} \rangle \red P\{\quotep{Q}/y}\} }
  \and \\
  \inferrule* [lab=Par] {{P} \red {P}'} {{{P} | {Q}} \red {{P}' | {Q}}}
  \and
  \inferrule* [lab=Equiv]{{{P} \scong {P}'} \andalso {{P}' \red {Q}'} \andalso {{Q}' \scong {Q}}}{{P} \red {Q}}
\end{mathpar}

\begin{eqnarray*}
  match_{\equiv} (\quotep{P},\quotep{Q}) & := & P \equiv Q \\
  match_{\dagger}(\quotep{P},\quotep{Q}) & := & \forall R. P|Q \red^{*} R => R \red^{*} 0 \\
  match_{K}(\quotep{P},\quotep{Q}) & := & K \mbox{ for some context } K
\end{eqnarray*}

$u?(x)P | u!\langle Q \rangle \red P\{\quotep{Q}/x\}$

%We write $\wred$ for $\red^*$, and $P\red$ if $\exists Q $ such that $ P \red Q$.
We write $P\red$ if $\exists Q $ such that $ P \red Q$ and $P\not\red$, otherwise.

\section{Replication}

As mentioned before, it is known that replication (and hence
recursion) can be implemented in a higher-order process algebra
\cite{SangiorgiWalker}. As our first example of calculation with the
machinery thus far presented we give the construction explicitly in
the {\rhoc}.

\begin{eqnarray}
	D_{x} & := & \prefix{x}{y}{(\binpar{\outputp{x}{y}}{@{y}})} \nonumber\\
	\bangp_{x}{P} & := & \binpar{{x}!\langle{\binpar{D_{x}}{P}}\rangle}{D_{x}} \nonumber
\end{eqnarray}

\begin{eqnarray}
	\bangp_{x}{P} & & \nonumber\\
	=
	& {x}!\langle{(\prefix{x}{y}{(\outputp{x}{y} | @{y})) | P}}\rangle 
	      | \prefix{x}{y}{(\outputp{x}{y} | @{y})} & \nonumber\\
	\red
	& (\outputp{x}{y} | @{y})\substn{\quotep{(\prefix{x}{y}{(@{y} | \outputp{x}{y})) | P}}}{y} & \nonumber\\
	=
	& \outputp{x}{\quotep{(\prefix{x}{y}{(\outputp{x}{y} | @{y})) | P}}}
	  | {(\prefix{x}{y}{(\outputp{x}{y} | @{y})) | P}} & \nonumber\\
	\red
	& \ldots & \nonumber\\
	\red^*
	& P | P | \ldots & \nonumber
\end{eqnarray}

Of course, this encoding, as an implementation, runs away, unfolding
$\bangp{P}$ eagerly. A lazier and more implementable replication
operator, restricted to input-guarded processes, may be obtained as follows.

\begin{eqnarray}
\bangp{\prefix{u}{v}{P}} 
	:= 
	\binpar{\lift{x}{\prefix{u}{v}{(\binpar{D(x)}{P})}}}{D(x)} \nonumber
\end{eqnarray}

\begin{remark}
  Note that the lazier definition still does not deal with summation
  or mixed summation (i.e. sums over input and output). The reader is
  invited to construct definitions of replication that deal with these
  features. 

  Further, the definitions are parameterized in a name, $x$. Can you,
  gentle reader, make a definition that eliminates this parameter and
  guarantees no accidental interaction between the replication
  machinery and the process being replicated -- i.e. no accidental
  sharing of names used by the process to get its work done and the
  name(s) used by the replication to effect copying. This latter
  revision of the definition of replication is crucial to obtaining
  the expected identity $!!P \sim !P$.
\end{remark}

\begin{remark}\label{rem:paradoxical_combinator}
  The reader familiar with the lambda calculus will have noticed the
  similarity between $D$ and the paradoxical combinator.

  [Ed. note: the existence of this seems to suggest we have to be more
  restrictive on the set of processes and names we admit if we are to
  support no-cloning.]
\end{remark}

\subsubsection{Bisimulation}

The computational dynamics gives rise to another kind of equivalence,
the equivalence of computational behavior. As previously mentioned
this is typically captured \emph{via} some form of bisimulation.

% The notion we use in this paper is weak barbed bisimulation
% \cite{milner91polyadicpi}.

The notion we use in this paper is derived from weak barbed
bisimulation \cite{milner91polyadicpi}. 

\begin{definition}
An \emph{observation relation}, $\downarrow_{\mathcal N}$, over a set
of names, $\mathcal N$, is the smallest relation satisfying the rules
below.

\infrule[Out-barb]{y \in {\mathcal N}, \; x \nameeq y}
		  {\outputp{x}{v} \downarrow_{\mathcal N} x}
\infrule[Par-barb]{\mbox{$P\downarrow_{\mathcal N} x$ or $Q\downarrow_{\mathcal N} x$}}
		  {\binpar{P}{Q} \downarrow_{\mathcal N} x}

We write $P \Downarrow_{\mathcal N} x$ if there is $Q$ such that 
$P \wred Q$ and $Q \downarrow_{\mathcal N} x$.
\end{definition}

\begin{definition}
%\label{def.bbisim}
An  ${\mathcal N}$-\emph{barbed bisimulation} over a set of names, ${\mathcal N}$, is a symmetric binary relation 
${\mathcal S}_{\mathcal N}$ between agents such that $P\rel{S}_{\mathcal N}Q$ implies:
\begin{enumerate}
\item If $P \red P'$ then $Q \wred Q'$ and $P'\rel{S}_{\mathcal N} Q'$.
\item If $P\downarrow_{\mathcal N} x$, then $Q\Downarrow_{\mathcal N} x$.
\end{enumerate}
$P$ is ${\mathcal N}$-barbed bisimilar to $Q$, written
$P \wbbisim_{\mathcal N} Q$, if $P \rel{S}_{\mathcal N} Q$ for some ${\mathcal N}$-barbed bisimulation ${\mathcal S}_{\mathcal N}$.
\end{definition}

$\mathcal{R} \subseteq \pi \times \pi$

$P \mathcal{R} Q => \forall P'. P \red P' \Rightarrow \exists Q'. Q \red Q', P' \mathcal{R} Q'$

$P \vdash x \Rightarrow Q \vdash x$

\begin{mathpar}
  \inferrule*[lab=Out-barb]{x \nameeq y}{{y}!\langle{Q}\rangle \vdash x}
  \and
  \inferrule*[lab=Par-barb]{\mbox{$P\vdash x$ or $Q\vdash x$}}{\binpar{P}{Q} \vdash x}
\end{mathpar}

\subsubsection{Contexts}

One of the principle advantages of computational calculi like the
$\pi$-calculus is a well-defined notion of context,
contextual-equivalence and a correlation between
contextual-equivalence and notions of bisimulation. The notion of
context allows the decomposition of a process into (sub-)process and
its syntactic environment, its context. Thus, a context may be
thought of as a process with a ``hole'' (written $\Box$) in it. The
application of a context $M$ to a process $P$, written $M[P]$, is
tantamount to filling the hole in $M$ with $P$. In this paper we do
not need the full weight of this theory, but do make use of the notion
of context in the proof the main theorem. 

\begin{mathpar}
  \inferrule* [lab=summation] {} {{M_{M},M_{N}} \bc \Box \;|\; x.M_{A} \;|\; M_{M}+M_{N}}
  \and
  \inferrule* [lab=agent] {} {{M_{A}} \bc (\vec{x})M_{P} \;| \; \clift{P_0,\ldots,M_{P},\ldots,P_N}}
  \and \\
  \inferrule* [lab=process] {} {{M_{P}} \bc M_{N} \;| \;P|M_{P} }
\end{mathpar} 

\begin{mathpar}
  \inferrule* [lab=sychronization] {} {M_{N} \bc \Box \;|\; x?M_{F} \;|\; x!M_{C}}
  \and
  \inferrule* [lab=abstraction] {} {{M_{F}} \bc (x)M_{P} }
  \and
  \inferrule* [lab=concretion] {} {{M_{C}} \bc \langle M_{P} \rangle }
  \and \\
  \inferrule* [lab=process] {} {{M_{P}} \bc M_{N} \;| \;P|M_{P} }
\end{mathpar}

\begin{definition}[contextual application] Given a context $M$, and
  process $P$, we define the \emph{contextual application}, $M[P] :=
  M\{P/\Box\}$. That is, the contextual application of M to P is the
  substitution of $P$ for $\Box$ in $M$.
\end{definition}

$\meaningof{-} : L \to \mathcal{P}(\pi)$

\begin{mathpar}
  \inferrule* [lab=collection] {} {\meaningof{true} = \pi, \and \meaningof{~E} = \pi \setminus \meaningof{E}, \and \meaningof{E_{1} \& E_{2}} = \meaningof{E_{1}} \cap \meaningof{E_{2}}}
\end{mathpar}

\begin{mathpar}
  \inferrule* [lab=structure] {} {\meaningof{0} = \{ P \in \pi | P \equiv 0 \}, \and \\ \meaningof{E_1 | E_2} = \{ P \in \pi | P \equiv P_{1} | P_{2}, P_{1} \in \meaningof{E_{1}}, P_{2} \in \meaningof{E_2}\} }
\end{mathpar}

\begin{mathpar}
 \inferrule* [lab=behavior] {} {\meaningof{\langle a?b \rangle E} = \{ P \in \pi | P \equiv Q | u?(y)P', \\ \and \\\\ \and \\ \;\;\; u \in \meaningof{a}, \forall z.P'\{z/y\} \in \meaningof{E\{z/b\}}\}, \and \\ \meaningof{a!E} = \{ P \in \pi | P \equiv Q | x!\langle P' \rangle, x \in \meaningof{a} P' \in \meaningof{E}\} }
\end{mathpar}

\begin{mathpar}
 \inferrule* [lab=nominal] {} {\meaningof{\quotep{E}} = \{ \quotep{P} \in \quotep{\pi} | P \in \meaningof{E} \}, \and \meaningof{\quotep{P}} = \{ \quotep{Q} \in \quotep{\pi} | P \equiv Q \} \and \\ \meaningof{@\quotep{E}} = \{ P \in \pi | P \equiv @x, x \in \meaningof{E} \}}
\end{mathpar}

\begin{eqnarray*}
  \\
  \meaningof{-} : TS \to ST
\end{eqnarray*}

\begin{eqnarray*}
  \\
  L : TS \to ST
\end{eqnarray*}

\begin{eqnarray*}
  \\
  P \models E \iff P \in \meaningof{E}
\end{eqnarray*}

\begin{eqnarray*}
  P \approx_{L} Q \iff \forall E \in L. P \models E \iff Q \models E
\end{eqnarray*}

\begin{eqnarray*}
  P \approx_{K} Q
\end{eqnarray*}

\begin{eqnarray*}
  P \approx Q
\end{eqnarray*}

$\approx_{K} = \approx = \approx_{L}$

\subsubsection{Contextual duality}

Note that contexts extend the quotation operation to a family of
operations from processes to names. Given a context, $M$, we can
define a \emph{nominal context}, $\quotep{M}$ by $\quotep{M}[P] :=
\quotep{M[P]}$. To foreshadow what is to come we observe that these
operations enjoy a duality with processes very much like the duality
between vectors and maps from vectors to scalars.

Further, because the calculus is essentially higher-order, we have a
correspondence between contexts and processes. More specifically,
given a name $x$ and a context $M$ we can construct $M^{*}_{x}$ such
that 

\begin{mathpar}
  M^{*}_{x} | \lift{x}{P} \red M[P]
\end{mathpar}

namely,

\begin{mathpar}
  M^{*}_{x} := x?(u).M[\dropn{u}]
\end{mathpar}

The dependence of $M^{*}_{x}$ on a name makes it an abstraction, 

\begin{mathpar}
  M^{*} := (x)x?(u).M[\dropn{u}]
\end{mathpar}

\subsection{Additional notation}

It will sometimes be convenient to denote the process a name
quotes. We already have the notation $x = \quotep{P}$, but it will be
convenient to introduce an alternate notation, $\procn{x}$, when we
want to emphasize the connection to the use of the name. Note that, by
virtue of name equivalence, $\quotep{\procn{x}} \nameeq x$; so, the
notation is consistent with previous definitions.

Further, because names have structure it is possible to effect
substitutions on the basis of that structure. This means we need to
upgrade our notation for substitutions, which we accomplish by
adapting comprehension notation. Thus,

\begin{mathpar}
  P\{ y / x : x \in S \}
\end{mathpar}

is interpreted to mean the process derived from P by replacing (in a
capture-avoiding manner) each occurrence of $x$ in $S$ by $y$. For example,

\begin{mathpar}
  P\{ \quotep{\procn{x}|\procn{x}} / x : x \in \freenames{P} \}
\end{mathpar}

will replace each (occurrence) of a free name $x$ in $P$ by
$\quotep{\procn{x}|\procn{x}}$.

Also, we will avail ourselves of the notation $x^{L}$ and $x^{R}$ to
denote injections of a name into disjoint copies of the name
space. There are numerous ways to accomplish this. One example can be
found in \cite{MeredithR05}. This notation overloads to vectors of
names: $\vec{x}^{\pi} := (x_{i}^{\pi} \; : \; 0 \leq i < |\vec{x}| )$ where $\pi \in \{L,R\}$.

We also use $P^{\Box} := P|\Box$.

In \cite{MeredithR05} an interpretation of the new operator is
given. It turns out that there are several possible interpretations
all enjoying the requisite algebraic properties of the operator (see
\cite{milner91polyadicpi}). We will therefore make liberal use of
$(\nu\; \vec{x})P$.

% subsection the_syntax_and_semantics_of_the_notation_system (end)   

\input{qm2pi.qmops} 

\input{qm2pi.sterngerlach} 

\input{qm2pi.metric} 

% section concurrent_process_calculi (end)

%\input{qm2pi.proofsketch}

% section proof sketch (end)

%\input{qm2pi.slviaknots} 

% section spatial logic via knots (end)

\input{qm2pi.conclusion}

% section conclusion (end)

%\input{qm2pi.dtcodes} 

% section wiring algorithm (end)

\input{qm2pi.ack} 

% section acknowledgments (end)

\newpage


\bibliographystyle{plain}   
\bibliography{../../biblios/main.bib}

\input{qm2pi.rhodetails}

\end{document}

 

%\documentclass[12pt]{llncs}
%\documentclass{jktr}

\usepackage[pdftex]{hyperref}                   
\usepackage {listings}
\usepackage {mathpartir}
\usepackage{bcprules}
%\usepackage{listings}
                       
\usepackage{graphicx} 
%\usepackage[margins=2.5cm,nohead,nofoot]{geometry}
%\usepackage{geometry}
\usepackage{amsfonts}
\usepackage{amstext}
\usepackage{latexsym}
\usepackage{amssymb}
\usepackage{color}


%\include{myPreamble}
\include{qm2pi.local} 

%\ifpdf
%\usepackage[pdftex]{graphicx}
%\else
%\usepackage{graphicx}
%\fi

 % \ifpdf
%  \usepackage{pdfsync}
%  \if


%\title{Brief Article}
%\author{David F. Snyder}
%\author{L.G. Meredith}

%\address{Dept. of Math., Texas State University--San Marcos, San Marcos, TX 78666}
       
\pagestyle{empty}


\begin{document}

\lstset{language=[Objective]Caml,frame=shadowbox}

\input{qm2pi.front}

% section front matter (end)

\input{qm2pi.intro} 
 
% section introduction (end)

% \input{qm2pi.knotations} 

% section notation (end)

\input{qm2pi.process.calculi} 

% section concurrent_process_calculi_and_spatial_logics_ (end)
    
%\input{qm2pi.knots2pi} 

%\input{qm2pi.trefoil} 

%\input{qm2pi.mainthm} 

% subsection basic_interpretation (end)

%\input{qm2pi.rho.presentation} 
\subsection{The syntax and semantics of the notation system}\label{sub:the_syntax_and_semantics_of_the_notation_system} % (fold)

We now summarize a technical presentation of the calculus that
embodies our theory of dynamics. The typical presentation of such a
calculus follows the style of giving generators and relations on
them. The grammar, below, describing term constructors, freely
generates the set of processes, $\Proc$. This set is then quotiented
by a relation known as structural congruence and it is over this set
that the notion of dynamics is expressed. This presentation is
essentially that of \cite{MeredithR05} with the addition of
polyadicity and summation. For readability we have relegated some of
the technical subtleties to an appendix.

\subsubsection{Process grammar}\label{subsub:process_grammar}

\begin{mathpar}
  \inferrule* [lab=synchronization] {} {{M} \bc \pzero \;|\; x?F \;|\; x!C }
  \and
  \inferrule* [lab=abstraction] {} {{F} \bc (x)P}
  \and
  \inferrule* [lab=concretion] {} {{C} \bc \langle Q \rangle}
  \and
  \inferrule* [lab=process] {} {{P,Q} \bc M \;| \;P|Q \;|\; @{x}}
  \and
  \inferrule* [lab=name] {} {{x} \bc \quotep{P}}
\end{mathpar} 

Note that $\vec{x}$ (resp. $\vec{P}$) denotes a vector of names
(resp. processes) of length $|\vec{x}|$ (resp. $|\vec{P}|$). We adopt
the following useful abbreviations.

\begin{mathpar}
   x?(\vec{y}).P := x.(\vec{y})P \and  x\clift{\vec{P}} := x.\clift{\vec{P}}
   \and x!(y) := \lift{x}{\dropn{y}}
   \and \Pi_{i=0}^{n-1}P_i := P_0 | \ldots | P_{n-1}
\end{mathpar}

\subsubsection{Structural congruence}

\paragraph{Free and bound names and alpha-equivalence.} At the
core of structural equivalence is alpha-equivalence which identifies
process that are the same up to a change of variable. Formally, we
recognize the distinction between free and bound names. The free names
of a process, $\freenames{P}$, may be calculated recursively as
follows:

\begin{mathpar}
\freenames{\pzero} := \emptyset
  \and \\
  \freenames{x?(y).P} := \{ x \} \cup (\freenames{P} \setminus \{ y \})
  \and 
  \freenames{x!\langle P \rangle} := \{ x \} \cup \{ P \} 
  \and \\
  \freenames{P|Q} := \freenames{P} \cup \freenames{Q}
  \and \\
  \freenames{@{x}} := \{ x \}
\end{mathpar}

$\pi$
$\quotep{\pi}$

$\freenames{-} : \pi \to \mathcal{P}(\quotep{\pi})$

\begin{eqnarray*}
  \freenames{\pzero} & := & \emptyset \\
  \freenames{x?(y).P} & := & \{ x \} \cup (\freenames{P} \setminus \{ y \}) \\
  \freenames{x!\langle P \rangle} & := & \{ x \} \cup \{ P \} \\
  \freenames{P|Q} & := & \freenames{P} \cup \freenames{Q} \\
  \freenames{\dropn{x}} & := & \{ x \}
\end{eqnarray*}

The bound names of a process, $\boundnames{P}$, are those names occurring in $P$
that are not free. For example, in $x?(y).0$, the name $x$ is free, while $y$ is bound.

\begin{mathpar}
  \inferrule* [lab=monoidal-laws] {} { P|Q \equiv Q|P \and P|0 \equiv P \and P|(Q|R) \equiv (P|Q)|R }
\end{mathpar}

\begin{mathpar}
  \inferrule* [lab=alpha-equivalence] {} { (x)P \equiv (y)P\{y/x\} \and y \not\in \freenames{P} }
\end{mathpar}

\begin{definition}
Then two processes, $P,Q$, are alpha-equivalent if $P = Q\{\vec{y}/\vec{x}\}$ for
some $\vec{x} \in \boundnames{Q},\vec{y} \in \boundnames{P}$, where $Q\{\vec{y}/\vec{x}\}$
denotes the capture-avoiding substitution of $\vec{y}$ for $\vec{x}$ in $Q$.
\end{definition}

\begin{definition}
  The {\em structural congruence} \cite{SangiorgiWalker} , $\equiv$,
  between processes is the least congruence containing
  alpha-equivalence, satisfying the abelian monoid laws
  (associativity, commutativity and $\pzero$ as identity) for parallel
  composition $|$ and for summation $+$.
\end{definition}

\subsection{Name equivalence}

We take name equivalence, written $\nameeq$, to be the smallest
equivalence relation generated by the following rules.

\begin{mathpar}
\inferrule*[lab=Quote-drop]
{ }
{ \quotep{@{x}} \nameeq x }

\inferrule*[lab=Struct-equiv]
{ P \scong Q }
{ \quotep{P} \nameeq \quotep{Q} }
\end{mathpar}

The astute reader will have noticed that the mutual recursion of names
and processes imposes a mutual recursion on alpha-equivalence and
structural equivalence via name-equivalence. Fortunately, all of this
works out pleasantly and we may calculate in the natural way, free of
concern. The reader interested in the details is referred to the
appendix \ref{appendix:rho_details}.

\subsection{Substitution}

We use $\Proc$ for the set of processes, $\QProc$ for the set of
names, and $\id{\{}\vec{y} / \vec{x} \id{\}}$ to denote partial maps,
$s : \QProc \rightarrow \QProc$. A map, $s$ lifts, uniquely, to a map
on process terms, $\widehat{s} : \Proc \rightarrow \Proc$ by the
following equations.

\begin{mathpar}
  (0) \psubstp{Q}{P} := 0 \\
  (R \juxtap S) \psubstp{Q}{P}
  :=    
  (R)\psubstp{Q}{P} \juxtap (S) \psubstp{Q}{P} \\
  (x?(y).R) \psubstp{Q}{P}    
  :=    
  (x)\substp{Q}{P} (z)\concat( (R \psubstn{z}{y}) \psubstp{Q}{P} ) \\
  (\lift{x}{R}) \psubstp{Q}{P}  
  :=
  \lift{(x)\substp{Q}{P}}{ R \psubstp{Q}{P} } \\
%   (\dropn{x})  \psubstp{Q}{P}       
%   := 
%   \left\{ 
%     \begin{array}{ccc} 
%       \dropn{\quotep{Q}} & & x \nameeq \quotep{P} \\
%       \dropn{x} & & otherwise \\
%     \end{array}
%   \right. 
  (\dropn{x})  \psubstp{Q}{P}       
  := 
  \left\{ 
    \begin{array}{ccc} 
      Q & & x \nameeq \quotep{P} \\
      \dropn{x} & & otherwise \\
    \end{array}
  \right.
\end{mathpar}
 

where

\begin{eqnarray}
  (x)\id{\{} \lpquote Q \rpquote / \lpquote P \rpquote \id{\}}            = 
  \left\{ 
    \begin{array}{ccc}
      \lpquote Q \rpquote & & x \nameeq \lpquote P \rpquote \\
      x & & otherwise \\
    \end{array}
  \right. \nonumber
\end{eqnarray}

and $z$ is chosen distinct from $\quotep{P}$, $\quotep{Q}$, the free
names in $Q$, and all the names in $R$. Our $\alpha$-equivalence will
be built in the standard way from this substitution.

\begin{remark}\label{rem:no_self_referential_names}
  One consequence of these definitions is that $\forall P. \quotep{P}
  \not\in \freenames{P}$.
\end{remark}

\subsection{ Dynamic quote: an example }

Anticipating something of what's to come, consider applying the
substitution, $\widehat{\id{\{}u / z \id{\}}}$, to the following pair
of processes, $\lift{w}{y!(z)}$ and $w[ \lpquote y!(z) \rpquote ]$.

\begin{eqnarray}
	\lift{w}{y!(z)}\widehat{\id{\{}u / z \id{\}}}
		& = &
		\lift{w}{y!(u)} \nonumber\\
	w[ \lpquote y!(z) \rpquote ] \widehat{ \id{\{}u / z \id{\}} }
		& = &
		w[ \lpquote y!(z) \rpquote ] \nonumber
\end{eqnarray}

Because the body of the process between quotes is impervious to
substitution, we get radically different answers. In fact, by
examining the first process in an input context,
e.g. $x?(z).\lift{w}{y!(z)}$, we see that the process under the lift
operator may be shaped by prefixed inputs binding a name inside it. In
this sense, the lift operator will be seen as a way to dynamically
construct processes before reifying them as names.

Finally equipped with these standard features we can present the
dynamics of the calculus.

\subsubsection{Operational semantics} 

Finally, we introduce the computational dynamics. What marks these
algebras as distinct from other more traditionally studied algebraic
structures, e.g. vector spaces or polynomial rings, is the manner in
which dynamics is captured. In traditional structures, dynamics is typically
expressed through morphisms between such structures, as in linear maps
between vector spaces or morphisms between rings. In algebras
associated with the semantics of computation, the dynamics is
expressed as part of the algebraic structure itself, through a
reduction reduction relation typically denoted by $\red$. Below, we
give a recursive presentation of this relation for the calculus used
in the encoding.

$\red \subseteq \pi \times \pi$
$\red : \pi \to \mathcal{P}(\pi)$

\begin{mathpar}
  \inferrule* [lab=Comm] { \textsf{match}( x_{src}, x_{trgt} ) } { x_{trgt}?(y)P \; | \; x_{src}!\langle {Q} \rangle \red P\{\quotep{Q}/y}\} }
  \and \\
  \inferrule* [lab=Par] {{P} \red {P}'} {{{P} | {Q}} \red {{P}' | {Q}}}
  \and
  \inferrule* [lab=Equiv]{{{P} \scong {P}'} \andalso {{P}' \red {Q}'} \andalso {{Q}' \scong {Q}}}{{P} \red {Q}}
\end{mathpar}

\begin{eqnarray*}
  match_{\equiv} (\quotep{P},\quotep{Q}) & := & P \equiv Q \\
  match_{\dagger}(\quotep{P},\quotep{Q}) & := & \forall R. P|Q \red^{*} R => R \red^{*} 0 \\
  match_{K}(\quotep{P},\quotep{Q}) & := & K \mbox{ for some context } K
\end{eqnarray*}

$u?(x)P | u!\langle Q \rangle \red P\{\quotep{Q}/x\}$

%We write $\wred$ for $\red^*$, and $P\red$ if $\exists Q $ such that $ P \red Q$.
We write $P\red$ if $\exists Q $ such that $ P \red Q$ and $P\not\red$, otherwise.

\section{Replication}

As mentioned before, it is known that replication (and hence
recursion) can be implemented in a higher-order process algebra
\cite{SangiorgiWalker}. As our first example of calculation with the
machinery thus far presented we give the construction explicitly in
the {\rhoc}.

\begin{eqnarray}
	D_{x} & := & \prefix{x}{y}{(\binpar{\outputp{x}{y}}{@{y}})} \nonumber\\
	\bangp_{x}{P} & := & \binpar{{x}!\langle{\binpar{D_{x}}{P}}\rangle}{D_{x}} \nonumber
\end{eqnarray}

\begin{eqnarray}
	\bangp_{x}{P} & & \nonumber\\
	=
	& {x}!\langle{(\prefix{x}{y}{(\outputp{x}{y} | @{y})) | P}}\rangle 
	      | \prefix{x}{y}{(\outputp{x}{y} | @{y})} & \nonumber\\
	\red
	& (\outputp{x}{y} | @{y})\substn{\quotep{(\prefix{x}{y}{(@{y} | \outputp{x}{y})) | P}}}{y} & \nonumber\\
	=
	& \outputp{x}{\quotep{(\prefix{x}{y}{(\outputp{x}{y} | @{y})) | P}}}
	  | {(\prefix{x}{y}{(\outputp{x}{y} | @{y})) | P}} & \nonumber\\
	\red
	& \ldots & \nonumber\\
	\red^*
	& P | P | \ldots & \nonumber
\end{eqnarray}

Of course, this encoding, as an implementation, runs away, unfolding
$\bangp{P}$ eagerly. A lazier and more implementable replication
operator, restricted to input-guarded processes, may be obtained as follows.

\begin{eqnarray}
\bangp{\prefix{u}{v}{P}} 
	:= 
	\binpar{\lift{x}{\prefix{u}{v}{(\binpar{D(x)}{P})}}}{D(x)} \nonumber
\end{eqnarray}

\begin{remark}
  Note that the lazier definition still does not deal with summation
  or mixed summation (i.e. sums over input and output). The reader is
  invited to construct definitions of replication that deal with these
  features. 

  Further, the definitions are parameterized in a name, $x$. Can you,
  gentle reader, make a definition that eliminates this parameter and
  guarantees no accidental interaction between the replication
  machinery and the process being replicated -- i.e. no accidental
  sharing of names used by the process to get its work done and the
  name(s) used by the replication to effect copying. This latter
  revision of the definition of replication is crucial to obtaining
  the expected identity $!!P \sim !P$.
\end{remark}

\begin{remark}\label{rem:paradoxical_combinator}
  The reader familiar with the lambda calculus will have noticed the
  similarity between $D$ and the paradoxical combinator.

  [Ed. note: the existence of this seems to suggest we have to be more
  restrictive on the set of processes and names we admit if we are to
  support no-cloning.]
\end{remark}

\subsubsection{Bisimulation}

The computational dynamics gives rise to another kind of equivalence,
the equivalence of computational behavior. As previously mentioned
this is typically captured \emph{via} some form of bisimulation.

% The notion we use in this paper is weak barbed bisimulation
% \cite{milner91polyadicpi}.

The notion we use in this paper is derived from weak barbed
bisimulation \cite{milner91polyadicpi}. 

\begin{definition}
An \emph{observation relation}, $\downarrow_{\mathcal N}$, over a set
of names, $\mathcal N$, is the smallest relation satisfying the rules
below.

\infrule[Out-barb]{y \in {\mathcal N}, \; x \nameeq y}
		  {\outputp{x}{v} \downarrow_{\mathcal N} x}
\infrule[Par-barb]{\mbox{$P\downarrow_{\mathcal N} x$ or $Q\downarrow_{\mathcal N} x$}}
		  {\binpar{P}{Q} \downarrow_{\mathcal N} x}

We write $P \Downarrow_{\mathcal N} x$ if there is $Q$ such that 
$P \wred Q$ and $Q \downarrow_{\mathcal N} x$.
\end{definition}

\begin{definition}
%\label{def.bbisim}
An  ${\mathcal N}$-\emph{barbed bisimulation} over a set of names, ${\mathcal N}$, is a symmetric binary relation 
${\mathcal S}_{\mathcal N}$ between agents such that $P\rel{S}_{\mathcal N}Q$ implies:
\begin{enumerate}
\item If $P \red P'$ then $Q \wred Q'$ and $P'\rel{S}_{\mathcal N} Q'$.
\item If $P\downarrow_{\mathcal N} x$, then $Q\Downarrow_{\mathcal N} x$.
\end{enumerate}
$P$ is ${\mathcal N}$-barbed bisimilar to $Q$, written
$P \wbbisim_{\mathcal N} Q$, if $P \rel{S}_{\mathcal N} Q$ for some ${\mathcal N}$-barbed bisimulation ${\mathcal S}_{\mathcal N}$.
\end{definition}

$\mathcal{R} \subseteq \pi \times \pi$

$P \mathcal{R} Q => \forall P'. P \red P' \Rightarrow \exists Q'. Q \red Q', P' \mathcal{R} Q'$

$P \vdash x \Rightarrow Q \vdash x$

\begin{mathpar}
  \inferrule*[lab=Out-barb]{x \nameeq y}{{y}!\langle{Q}\rangle \vdash x}
  \and
  \inferrule*[lab=Par-barb]{\mbox{$P\vdash x$ or $Q\vdash x$}}{\binpar{P}{Q} \vdash x}
\end{mathpar}

\subsubsection{Contexts}

One of the principle advantages of computational calculi like the
$\pi$-calculus is a well-defined notion of context,
contextual-equivalence and a correlation between
contextual-equivalence and notions of bisimulation. The notion of
context allows the decomposition of a process into (sub-)process and
its syntactic environment, its context. Thus, a context may be
thought of as a process with a ``hole'' (written $\Box$) in it. The
application of a context $M$ to a process $P$, written $M[P]$, is
tantamount to filling the hole in $M$ with $P$. In this paper we do
not need the full weight of this theory, but do make use of the notion
of context in the proof the main theorem. 

\begin{mathpar}
  \inferrule* [lab=summation] {} {{M_{M},M_{N}} \bc \Box \;|\; x.M_{A} \;|\; M_{M}+M_{N}}
  \and
  \inferrule* [lab=agent] {} {{M_{A}} \bc (\vec{x})M_{P} \;| \; \clift{P_0,\ldots,M_{P},\ldots,P_N}}
  \and \\
  \inferrule* [lab=process] {} {{M_{P}} \bc M_{N} \;| \;P|M_{P} }
\end{mathpar} 

\begin{mathpar}
  \inferrule* [lab=sychronization] {} {M_{N} \bc \Box \;|\; x?M_{F} \;|\; x!M_{C}}
  \and
  \inferrule* [lab=abstraction] {} {{M_{F}} \bc (x)M_{P} }
  \and
  \inferrule* [lab=concretion] {} {{M_{C}} \bc \langle M_{P} \rangle }
  \and \\
  \inferrule* [lab=process] {} {{M_{P}} \bc M_{N} \;| \;P|M_{P} }
\end{mathpar}

\begin{definition}[contextual application] Given a context $M$, and
  process $P$, we define the \emph{contextual application}, $M[P] :=
  M\{P/\Box\}$. That is, the contextual application of M to P is the
  substitution of $P$ for $\Box$ in $M$.
\end{definition}

$\meaningof{-} : L \to \mathcal{P}(\pi)$

\begin{mathpar}
  \inferrule* [lab=collection] {} {\meaningof{true} = \pi, \and \meaningof{~E} = \pi \setminus \meaningof{E}, \and \meaningof{E_{1} \& E_{2}} = \meaningof{E_{1}} \cap \meaningof{E_{2}}}
\end{mathpar}

\begin{mathpar}
  \inferrule* [lab=structure] {} {\meaningof{0} = \{ P \in \pi | P \equiv 0 \}, \and \\ \meaningof{E_1 | E_2} = \{ P \in \pi | P \equiv P_{1} | P_{2}, P_{1} \in \meaningof{E_{1}}, P_{2} \in \meaningof{E_2}\} }
\end{mathpar}

\begin{mathpar}
 \inferrule* [lab=behavior] {} {\meaningof{\langle a?b \rangle E} = \{ P \in \pi | P \equiv Q | u?(y)P', \\ \and \\\\ \and \\ \;\;\; u \in \meaningof{a}, \forall z.P'\{z/y\} \in \meaningof{E\{z/b\}}\}, \and \\ \meaningof{a!E} = \{ P \in \pi | P \equiv Q | x!\langle P' \rangle, x \in \meaningof{a} P' \in \meaningof{E}\} }
\end{mathpar}

\begin{mathpar}
 \inferrule* [lab=nominal] {} {\meaningof{\quotep{E}} = \{ \quotep{P} \in \quotep{\pi} | P \in \meaningof{E} \}, \and \meaningof{\quotep{P}} = \{ \quotep{Q} \in \quotep{\pi} | P \equiv Q \} \and \\ \meaningof{@\quotep{E}} = \{ P \in \pi | P \equiv @x, x \in \meaningof{E} \}}
\end{mathpar}

\begin{eqnarray*}
  \\
  \meaningof{-} : TS \to ST
\end{eqnarray*}

\begin{eqnarray*}
  \\
  L : TS \to ST
\end{eqnarray*}

\begin{eqnarray*}
  \\
  P \models E \iff P \in \meaningof{E}
\end{eqnarray*}

\begin{eqnarray*}
  P \approx_{L} Q \iff \forall E \in L. P \models E \iff Q \models E
\end{eqnarray*}

\begin{eqnarray*}
  P \approx_{K} Q
\end{eqnarray*}

\begin{eqnarray*}
  P \approx Q
\end{eqnarray*}

$\approx_{K} = \approx = \approx_{L}$

\subsubsection{Contextual duality}

Note that contexts extend the quotation operation to a family of
operations from processes to names. Given a context, $M$, we can
define a \emph{nominal context}, $\quotep{M}$ by $\quotep{M}[P] :=
\quotep{M[P]}$. To foreshadow what is to come we observe that these
operations enjoy a duality with processes very much like the duality
between vectors and maps from vectors to scalars.

Further, because the calculus is essentially higher-order, we have a
correspondence between contexts and processes. More specifically,
given a name $x$ and a context $M$ we can construct $M^{*}_{x}$ such
that 

\begin{mathpar}
  M^{*}_{x} | \lift{x}{P} \red M[P]
\end{mathpar}

namely,

\begin{mathpar}
  M^{*}_{x} := x?(u).M[\dropn{u}]
\end{mathpar}

The dependence of $M^{*}_{x}$ on a name makes it an abstraction, 

\begin{mathpar}
  M^{*} := (x)x?(u).M[\dropn{u}]
\end{mathpar}

\subsection{Additional notation}

It will sometimes be convenient to denote the process a name
quotes. We already have the notation $x = \quotep{P}$, but it will be
convenient to introduce an alternate notation, $\procn{x}$, when we
want to emphasize the connection to the use of the name. Note that, by
virtue of name equivalence, $\quotep{\procn{x}} \nameeq x$; so, the
notation is consistent with previous definitions.

Further, because names have structure it is possible to effect
substitutions on the basis of that structure. This means we need to
upgrade our notation for substitutions, which we accomplish by
adapting comprehension notation. Thus,

\begin{mathpar}
  P\{ y / x : x \in S \}
\end{mathpar}

is interpreted to mean the process derived from P by replacing (in a
capture-avoiding manner) each occurrence of $x$ in $S$ by $y$. For example,

\begin{mathpar}
  P\{ \quotep{\procn{x}|\procn{x}} / x : x \in \freenames{P} \}
\end{mathpar}

will replace each (occurrence) of a free name $x$ in $P$ by
$\quotep{\procn{x}|\procn{x}}$.

Also, we will avail ourselves of the notation $x^{L}$ and $x^{R}$ to
denote injections of a name into disjoint copies of the name
space. There are numerous ways to accomplish this. One example can be
found in \cite{MeredithR05}. This notation overloads to vectors of
names: $\vec{x}^{\pi} := (x_{i}^{\pi} \; : \; 0 \leq i < |\vec{x}| )$ where $\pi \in \{L,R\}$.

We also use $P^{\Box} := P|\Box$.

In \cite{MeredithR05} an interpretation of the new operator is
given. It turns out that there are several possible interpretations
all enjoying the requisite algebraic properties of the operator (see
\cite{milner91polyadicpi}). We will therefore make liberal use of
$(\nu\; \vec{x})P$.

% subsection the_syntax_and_semantics_of_the_notation_system (end)   

\input{qm2pi.qmops} 

\input{qm2pi.sterngerlach} 

\input{qm2pi.metric} 

% section concurrent_process_calculi (end)

%\input{qm2pi.proofsketch}

% section proof sketch (end)

%\input{qm2pi.slviaknots} 

% section spatial logic via knots (end)

\input{qm2pi.conclusion}

% section conclusion (end)

%\input{qm2pi.dtcodes} 

% section wiring algorithm (end)

\input{qm2pi.ack} 

% section acknowledgments (end)

\newpage


\bibliographystyle{plain}   
\bibliography{../../biblios/main.bib}

\input{qm2pi.rhodetails}

\end{document}

 

%\documentclass[12pt]{llncs}
%\documentclass{jktr}

\usepackage[pdftex]{hyperref}                   
\usepackage {listings}
\usepackage {mathpartir}
\usepackage{bcprules}
%\usepackage{listings}
                       
\usepackage{graphicx} 
%\usepackage[margins=2.5cm,nohead,nofoot]{geometry}
%\usepackage{geometry}
\usepackage{amsfonts}
\usepackage{amstext}
\usepackage{latexsym}
\usepackage{amssymb}
\usepackage{color}


%\include{myPreamble}
\include{qm2pi.local} 

%\ifpdf
%\usepackage[pdftex]{graphicx}
%\else
%\usepackage{graphicx}
%\fi

 % \ifpdf
%  \usepackage{pdfsync}
%  \if


%\title{Brief Article}
%\author{David F. Snyder}
%\author{L.G. Meredith}

%\address{Dept. of Math., Texas State University--San Marcos, San Marcos, TX 78666}
       
\pagestyle{empty}


\begin{document}

\lstset{language=[Objective]Caml,frame=shadowbox}

\input{qm2pi.front}

% section front matter (end)

\input{qm2pi.intro} 
 
% section introduction (end)

% \input{qm2pi.knotations} 

% section notation (end)

\input{qm2pi.process.calculi} 

% section concurrent_process_calculi_and_spatial_logics_ (end)
    
%\input{qm2pi.knots2pi} 

%\input{qm2pi.trefoil} 

%\input{qm2pi.mainthm} 

% subsection basic_interpretation (end)

%\input{qm2pi.rho.presentation} 
\subsection{The syntax and semantics of the notation system}\label{sub:the_syntax_and_semantics_of_the_notation_system} % (fold)

We now summarize a technical presentation of the calculus that
embodies our theory of dynamics. The typical presentation of such a
calculus follows the style of giving generators and relations on
them. The grammar, below, describing term constructors, freely
generates the set of processes, $\Proc$. This set is then quotiented
by a relation known as structural congruence and it is over this set
that the notion of dynamics is expressed. This presentation is
essentially that of \cite{MeredithR05} with the addition of
polyadicity and summation. For readability we have relegated some of
the technical subtleties to an appendix.

\subsubsection{Process grammar}\label{subsub:process_grammar}

\begin{mathpar}
  \inferrule* [lab=synchronization] {} {{M} \bc \pzero \;|\; x?F \;|\; x!C }
  \and
  \inferrule* [lab=abstraction] {} {{F} \bc (x)P}
  \and
  \inferrule* [lab=concretion] {} {{C} \bc \langle Q \rangle}
  \and
  \inferrule* [lab=process] {} {{P,Q} \bc M \;| \;P|Q \;|\; @{x}}
  \and
  \inferrule* [lab=name] {} {{x} \bc \quotep{P}}
\end{mathpar} 

Note that $\vec{x}$ (resp. $\vec{P}$) denotes a vector of names
(resp. processes) of length $|\vec{x}|$ (resp. $|\vec{P}|$). We adopt
the following useful abbreviations.

\begin{mathpar}
   x?(\vec{y}).P := x.(\vec{y})P \and  x\clift{\vec{P}} := x.\clift{\vec{P}}
   \and x!(y) := \lift{x}{\dropn{y}}
   \and \Pi_{i=0}^{n-1}P_i := P_0 | \ldots | P_{n-1}
\end{mathpar}

\subsubsection{Structural congruence}

\paragraph{Free and bound names and alpha-equivalence.} At the
core of structural equivalence is alpha-equivalence which identifies
process that are the same up to a change of variable. Formally, we
recognize the distinction between free and bound names. The free names
of a process, $\freenames{P}$, may be calculated recursively as
follows:

\begin{mathpar}
\freenames{\pzero} := \emptyset
  \and \\
  \freenames{x?(y).P} := \{ x \} \cup (\freenames{P} \setminus \{ y \})
  \and 
  \freenames{x!\langle P \rangle} := \{ x \} \cup \{ P \} 
  \and \\
  \freenames{P|Q} := \freenames{P} \cup \freenames{Q}
  \and \\
  \freenames{@{x}} := \{ x \}
\end{mathpar}

$\pi$
$\quotep{\pi}$

$\freenames{-} : \pi \to \mathcal{P}(\quotep{\pi})$

\begin{eqnarray*}
  \freenames{\pzero} & := & \emptyset \\
  \freenames{x?(y).P} & := & \{ x \} \cup (\freenames{P} \setminus \{ y \}) \\
  \freenames{x!\langle P \rangle} & := & \{ x \} \cup \{ P \} \\
  \freenames{P|Q} & := & \freenames{P} \cup \freenames{Q} \\
  \freenames{\dropn{x}} & := & \{ x \}
\end{eqnarray*}

The bound names of a process, $\boundnames{P}$, are those names occurring in $P$
that are not free. For example, in $x?(y).0$, the name $x$ is free, while $y$ is bound.

\begin{mathpar}
  \inferrule* [lab=monoidal-laws] {} { P|Q \equiv Q|P \and P|0 \equiv P \and P|(Q|R) \equiv (P|Q)|R }
\end{mathpar}

\begin{mathpar}
  \inferrule* [lab=alpha-equivalence] {} { (x)P \equiv (y)P\{y/x\} \and y \not\in \freenames{P} }
\end{mathpar}

\begin{definition}
Then two processes, $P,Q$, are alpha-equivalent if $P = Q\{\vec{y}/\vec{x}\}$ for
some $\vec{x} \in \boundnames{Q},\vec{y} \in \boundnames{P}$, where $Q\{\vec{y}/\vec{x}\}$
denotes the capture-avoiding substitution of $\vec{y}$ for $\vec{x}$ in $Q$.
\end{definition}

\begin{definition}
  The {\em structural congruence} \cite{SangiorgiWalker} , $\equiv$,
  between processes is the least congruence containing
  alpha-equivalence, satisfying the abelian monoid laws
  (associativity, commutativity and $\pzero$ as identity) for parallel
  composition $|$ and for summation $+$.
\end{definition}

\subsection{Name equivalence}

We take name equivalence, written $\nameeq$, to be the smallest
equivalence relation generated by the following rules.

\begin{mathpar}
\inferrule*[lab=Quote-drop]
{ }
{ \quotep{@{x}} \nameeq x }

\inferrule*[lab=Struct-equiv]
{ P \scong Q }
{ \quotep{P} \nameeq \quotep{Q} }
\end{mathpar}

The astute reader will have noticed that the mutual recursion of names
and processes imposes a mutual recursion on alpha-equivalence and
structural equivalence via name-equivalence. Fortunately, all of this
works out pleasantly and we may calculate in the natural way, free of
concern. The reader interested in the details is referred to the
appendix \ref{appendix:rho_details}.

\subsection{Substitution}

We use $\Proc$ for the set of processes, $\QProc$ for the set of
names, and $\id{\{}\vec{y} / \vec{x} \id{\}}$ to denote partial maps,
$s : \QProc \rightarrow \QProc$. A map, $s$ lifts, uniquely, to a map
on process terms, $\widehat{s} : \Proc \rightarrow \Proc$ by the
following equations.

\begin{mathpar}
  (0) \psubstp{Q}{P} := 0 \\
  (R \juxtap S) \psubstp{Q}{P}
  :=    
  (R)\psubstp{Q}{P} \juxtap (S) \psubstp{Q}{P} \\
  (x?(y).R) \psubstp{Q}{P}    
  :=    
  (x)\substp{Q}{P} (z)\concat( (R \psubstn{z}{y}) \psubstp{Q}{P} ) \\
  (\lift{x}{R}) \psubstp{Q}{P}  
  :=
  \lift{(x)\substp{Q}{P}}{ R \psubstp{Q}{P} } \\
%   (\dropn{x})  \psubstp{Q}{P}       
%   := 
%   \left\{ 
%     \begin{array}{ccc} 
%       \dropn{\quotep{Q}} & & x \nameeq \quotep{P} \\
%       \dropn{x} & & otherwise \\
%     \end{array}
%   \right. 
  (\dropn{x})  \psubstp{Q}{P}       
  := 
  \left\{ 
    \begin{array}{ccc} 
      Q & & x \nameeq \quotep{P} \\
      \dropn{x} & & otherwise \\
    \end{array}
  \right.
\end{mathpar}
 

where

\begin{eqnarray}
  (x)\id{\{} \lpquote Q \rpquote / \lpquote P \rpquote \id{\}}            = 
  \left\{ 
    \begin{array}{ccc}
      \lpquote Q \rpquote & & x \nameeq \lpquote P \rpquote \\
      x & & otherwise \\
    \end{array}
  \right. \nonumber
\end{eqnarray}

and $z$ is chosen distinct from $\quotep{P}$, $\quotep{Q}$, the free
names in $Q$, and all the names in $R$. Our $\alpha$-equivalence will
be built in the standard way from this substitution.

\begin{remark}\label{rem:no_self_referential_names}
  One consequence of these definitions is that $\forall P. \quotep{P}
  \not\in \freenames{P}$.
\end{remark}

\subsection{ Dynamic quote: an example }

Anticipating something of what's to come, consider applying the
substitution, $\widehat{\id{\{}u / z \id{\}}}$, to the following pair
of processes, $\lift{w}{y!(z)}$ and $w[ \lpquote y!(z) \rpquote ]$.

\begin{eqnarray}
	\lift{w}{y!(z)}\widehat{\id{\{}u / z \id{\}}}
		& = &
		\lift{w}{y!(u)} \nonumber\\
	w[ \lpquote y!(z) \rpquote ] \widehat{ \id{\{}u / z \id{\}} }
		& = &
		w[ \lpquote y!(z) \rpquote ] \nonumber
\end{eqnarray}

Because the body of the process between quotes is impervious to
substitution, we get radically different answers. In fact, by
examining the first process in an input context,
e.g. $x?(z).\lift{w}{y!(z)}$, we see that the process under the lift
operator may be shaped by prefixed inputs binding a name inside it. In
this sense, the lift operator will be seen as a way to dynamically
construct processes before reifying them as names.

Finally equipped with these standard features we can present the
dynamics of the calculus.

\subsubsection{Operational semantics} 

Finally, we introduce the computational dynamics. What marks these
algebras as distinct from other more traditionally studied algebraic
structures, e.g. vector spaces or polynomial rings, is the manner in
which dynamics is captured. In traditional structures, dynamics is typically
expressed through morphisms between such structures, as in linear maps
between vector spaces or morphisms between rings. In algebras
associated with the semantics of computation, the dynamics is
expressed as part of the algebraic structure itself, through a
reduction reduction relation typically denoted by $\red$. Below, we
give a recursive presentation of this relation for the calculus used
in the encoding.

$\red \subseteq \pi \times \pi$
$\red : \pi \to \mathcal{P}(\pi)$

\begin{mathpar}
  \inferrule* [lab=Comm] { \textsf{match}( x_{src}, x_{trgt} ) } { x_{trgt}?(y)P \; | \; x_{src}!\langle {Q} \rangle \red P\{\quotep{Q}/y}\} }
  \and \\
  \inferrule* [lab=Par] {{P} \red {P}'} {{{P} | {Q}} \red {{P}' | {Q}}}
  \and
  \inferrule* [lab=Equiv]{{{P} \scong {P}'} \andalso {{P}' \red {Q}'} \andalso {{Q}' \scong {Q}}}{{P} \red {Q}}
\end{mathpar}

\begin{eqnarray*}
  match_{\equiv} (\quotep{P},\quotep{Q}) & := & P \equiv Q \\
  match_{\dagger}(\quotep{P},\quotep{Q}) & := & \forall R. P|Q \red^{*} R => R \red^{*} 0 \\
  match_{K}(\quotep{P},\quotep{Q}) & := & K \mbox{ for some context } K
\end{eqnarray*}

$u?(x)P | u!\langle Q \rangle \red P\{\quotep{Q}/x\}$

%We write $\wred$ for $\red^*$, and $P\red$ if $\exists Q $ such that $ P \red Q$.
We write $P\red$ if $\exists Q $ such that $ P \red Q$ and $P\not\red$, otherwise.

\section{Replication}

As mentioned before, it is known that replication (and hence
recursion) can be implemented in a higher-order process algebra
\cite{SangiorgiWalker}. As our first example of calculation with the
machinery thus far presented we give the construction explicitly in
the {\rhoc}.

\begin{eqnarray}
	D_{x} & := & \prefix{x}{y}{(\binpar{\outputp{x}{y}}{@{y}})} \nonumber\\
	\bangp_{x}{P} & := & \binpar{{x}!\langle{\binpar{D_{x}}{P}}\rangle}{D_{x}} \nonumber
\end{eqnarray}

\begin{eqnarray}
	\bangp_{x}{P} & & \nonumber\\
	=
	& {x}!\langle{(\prefix{x}{y}{(\outputp{x}{y} | @{y})) | P}}\rangle 
	      | \prefix{x}{y}{(\outputp{x}{y} | @{y})} & \nonumber\\
	\red
	& (\outputp{x}{y} | @{y})\substn{\quotep{(\prefix{x}{y}{(@{y} | \outputp{x}{y})) | P}}}{y} & \nonumber\\
	=
	& \outputp{x}{\quotep{(\prefix{x}{y}{(\outputp{x}{y} | @{y})) | P}}}
	  | {(\prefix{x}{y}{(\outputp{x}{y} | @{y})) | P}} & \nonumber\\
	\red
	& \ldots & \nonumber\\
	\red^*
	& P | P | \ldots & \nonumber
\end{eqnarray}

Of course, this encoding, as an implementation, runs away, unfolding
$\bangp{P}$ eagerly. A lazier and more implementable replication
operator, restricted to input-guarded processes, may be obtained as follows.

\begin{eqnarray}
\bangp{\prefix{u}{v}{P}} 
	:= 
	\binpar{\lift{x}{\prefix{u}{v}{(\binpar{D(x)}{P})}}}{D(x)} \nonumber
\end{eqnarray}

\begin{remark}
  Note that the lazier definition still does not deal with summation
  or mixed summation (i.e. sums over input and output). The reader is
  invited to construct definitions of replication that deal with these
  features. 

  Further, the definitions are parameterized in a name, $x$. Can you,
  gentle reader, make a definition that eliminates this parameter and
  guarantees no accidental interaction between the replication
  machinery and the process being replicated -- i.e. no accidental
  sharing of names used by the process to get its work done and the
  name(s) used by the replication to effect copying. This latter
  revision of the definition of replication is crucial to obtaining
  the expected identity $!!P \sim !P$.
\end{remark}

\begin{remark}\label{rem:paradoxical_combinator}
  The reader familiar with the lambda calculus will have noticed the
  similarity between $D$ and the paradoxical combinator.

  [Ed. note: the existence of this seems to suggest we have to be more
  restrictive on the set of processes and names we admit if we are to
  support no-cloning.]
\end{remark}

\subsubsection{Bisimulation}

The computational dynamics gives rise to another kind of equivalence,
the equivalence of computational behavior. As previously mentioned
this is typically captured \emph{via} some form of bisimulation.

% The notion we use in this paper is weak barbed bisimulation
% \cite{milner91polyadicpi}.

The notion we use in this paper is derived from weak barbed
bisimulation \cite{milner91polyadicpi}. 

\begin{definition}
An \emph{observation relation}, $\downarrow_{\mathcal N}$, over a set
of names, $\mathcal N$, is the smallest relation satisfying the rules
below.

\infrule[Out-barb]{y \in {\mathcal N}, \; x \nameeq y}
		  {\outputp{x}{v} \downarrow_{\mathcal N} x}
\infrule[Par-barb]{\mbox{$P\downarrow_{\mathcal N} x$ or $Q\downarrow_{\mathcal N} x$}}
		  {\binpar{P}{Q} \downarrow_{\mathcal N} x}

We write $P \Downarrow_{\mathcal N} x$ if there is $Q$ such that 
$P \wred Q$ and $Q \downarrow_{\mathcal N} x$.
\end{definition}

\begin{definition}
%\label{def.bbisim}
An  ${\mathcal N}$-\emph{barbed bisimulation} over a set of names, ${\mathcal N}$, is a symmetric binary relation 
${\mathcal S}_{\mathcal N}$ between agents such that $P\rel{S}_{\mathcal N}Q$ implies:
\begin{enumerate}
\item If $P \red P'$ then $Q \wred Q'$ and $P'\rel{S}_{\mathcal N} Q'$.
\item If $P\downarrow_{\mathcal N} x$, then $Q\Downarrow_{\mathcal N} x$.
\end{enumerate}
$P$ is ${\mathcal N}$-barbed bisimilar to $Q$, written
$P \wbbisim_{\mathcal N} Q$, if $P \rel{S}_{\mathcal N} Q$ for some ${\mathcal N}$-barbed bisimulation ${\mathcal S}_{\mathcal N}$.
\end{definition}

$\mathcal{R} \subseteq \pi \times \pi$

$P \mathcal{R} Q => \forall P'. P \red P' \Rightarrow \exists Q'. Q \red Q', P' \mathcal{R} Q'$

$P \vdash x \Rightarrow Q \vdash x$

\begin{mathpar}
  \inferrule*[lab=Out-barb]{x \nameeq y}{{y}!\langle{Q}\rangle \vdash x}
  \and
  \inferrule*[lab=Par-barb]{\mbox{$P\vdash x$ or $Q\vdash x$}}{\binpar{P}{Q} \vdash x}
\end{mathpar}

\subsubsection{Contexts}

One of the principle advantages of computational calculi like the
$\pi$-calculus is a well-defined notion of context,
contextual-equivalence and a correlation between
contextual-equivalence and notions of bisimulation. The notion of
context allows the decomposition of a process into (sub-)process and
its syntactic environment, its context. Thus, a context may be
thought of as a process with a ``hole'' (written $\Box$) in it. The
application of a context $M$ to a process $P$, written $M[P]$, is
tantamount to filling the hole in $M$ with $P$. In this paper we do
not need the full weight of this theory, but do make use of the notion
of context in the proof the main theorem. 

\begin{mathpar}
  \inferrule* [lab=summation] {} {{M_{M},M_{N}} \bc \Box \;|\; x.M_{A} \;|\; M_{M}+M_{N}}
  \and
  \inferrule* [lab=agent] {} {{M_{A}} \bc (\vec{x})M_{P} \;| \; \clift{P_0,\ldots,M_{P},\ldots,P_N}}
  \and \\
  \inferrule* [lab=process] {} {{M_{P}} \bc M_{N} \;| \;P|M_{P} }
\end{mathpar} 

\begin{mathpar}
  \inferrule* [lab=sychronization] {} {M_{N} \bc \Box \;|\; x?M_{F} \;|\; x!M_{C}}
  \and
  \inferrule* [lab=abstraction] {} {{M_{F}} \bc (x)M_{P} }
  \and
  \inferrule* [lab=concretion] {} {{M_{C}} \bc \langle M_{P} \rangle }
  \and \\
  \inferrule* [lab=process] {} {{M_{P}} \bc M_{N} \;| \;P|M_{P} }
\end{mathpar}

\begin{definition}[contextual application] Given a context $M$, and
  process $P$, we define the \emph{contextual application}, $M[P] :=
  M\{P/\Box\}$. That is, the contextual application of M to P is the
  substitution of $P$ for $\Box$ in $M$.
\end{definition}

$\meaningof{-} : L \to \mathcal{P}(\pi)$

\begin{mathpar}
  \inferrule* [lab=collection] {} {\meaningof{true} = \pi, \and \meaningof{~E} = \pi \setminus \meaningof{E}, \and \meaningof{E_{1} \& E_{2}} = \meaningof{E_{1}} \cap \meaningof{E_{2}}}
\end{mathpar}

\begin{mathpar}
  \inferrule* [lab=structure] {} {\meaningof{0} = \{ P \in \pi | P \equiv 0 \}, \and \\ \meaningof{E_1 | E_2} = \{ P \in \pi | P \equiv P_{1} | P_{2}, P_{1} \in \meaningof{E_{1}}, P_{2} \in \meaningof{E_2}\} }
\end{mathpar}

\begin{mathpar}
 \inferrule* [lab=behavior] {} {\meaningof{\langle a?b \rangle E} = \{ P \in \pi | P \equiv Q | u?(y)P', \\ \and \\\\ \and \\ \;\;\; u \in \meaningof{a}, \forall z.P'\{z/y\} \in \meaningof{E\{z/b\}}\}, \and \\ \meaningof{a!E} = \{ P \in \pi | P \equiv Q | x!\langle P' \rangle, x \in \meaningof{a} P' \in \meaningof{E}\} }
\end{mathpar}

\begin{mathpar}
 \inferrule* [lab=nominal] {} {\meaningof{\quotep{E}} = \{ \quotep{P} \in \quotep{\pi} | P \in \meaningof{E} \}, \and \meaningof{\quotep{P}} = \{ \quotep{Q} \in \quotep{\pi} | P \equiv Q \} \and \\ \meaningof{@\quotep{E}} = \{ P \in \pi | P \equiv @x, x \in \meaningof{E} \}}
\end{mathpar}

\begin{eqnarray*}
  \\
  \meaningof{-} : TS \to ST
\end{eqnarray*}

\begin{eqnarray*}
  \\
  L : TS \to ST
\end{eqnarray*}

\begin{eqnarray*}
  \\
  P \models E \iff P \in \meaningof{E}
\end{eqnarray*}

\begin{eqnarray*}
  P \approx_{L} Q \iff \forall E \in L. P \models E \iff Q \models E
\end{eqnarray*}

\begin{eqnarray*}
  P \approx_{K} Q
\end{eqnarray*}

\begin{eqnarray*}
  P \approx Q
\end{eqnarray*}

$\approx_{K} = \approx = \approx_{L}$

\subsubsection{Contextual duality}

Note that contexts extend the quotation operation to a family of
operations from processes to names. Given a context, $M$, we can
define a \emph{nominal context}, $\quotep{M}$ by $\quotep{M}[P] :=
\quotep{M[P]}$. To foreshadow what is to come we observe that these
operations enjoy a duality with processes very much like the duality
between vectors and maps from vectors to scalars.

Further, because the calculus is essentially higher-order, we have a
correspondence between contexts and processes. More specifically,
given a name $x$ and a context $M$ we can construct $M^{*}_{x}$ such
that 

\begin{mathpar}
  M^{*}_{x} | \lift{x}{P} \red M[P]
\end{mathpar}

namely,

\begin{mathpar}
  M^{*}_{x} := x?(u).M[\dropn{u}]
\end{mathpar}

The dependence of $M^{*}_{x}$ on a name makes it an abstraction, 

\begin{mathpar}
  M^{*} := (x)x?(u).M[\dropn{u}]
\end{mathpar}

\subsection{Additional notation}

It will sometimes be convenient to denote the process a name
quotes. We already have the notation $x = \quotep{P}$, but it will be
convenient to introduce an alternate notation, $\procn{x}$, when we
want to emphasize the connection to the use of the name. Note that, by
virtue of name equivalence, $\quotep{\procn{x}} \nameeq x$; so, the
notation is consistent with previous definitions.

Further, because names have structure it is possible to effect
substitutions on the basis of that structure. This means we need to
upgrade our notation for substitutions, which we accomplish by
adapting comprehension notation. Thus,

\begin{mathpar}
  P\{ y / x : x \in S \}
\end{mathpar}

is interpreted to mean the process derived from P by replacing (in a
capture-avoiding manner) each occurrence of $x$ in $S$ by $y$. For example,

\begin{mathpar}
  P\{ \quotep{\procn{x}|\procn{x}} / x : x \in \freenames{P} \}
\end{mathpar}

will replace each (occurrence) of a free name $x$ in $P$ by
$\quotep{\procn{x}|\procn{x}}$.

Also, we will avail ourselves of the notation $x^{L}$ and $x^{R}$ to
denote injections of a name into disjoint copies of the name
space. There are numerous ways to accomplish this. One example can be
found in \cite{MeredithR05}. This notation overloads to vectors of
names: $\vec{x}^{\pi} := (x_{i}^{\pi} \; : \; 0 \leq i < |\vec{x}| )$ where $\pi \in \{L,R\}$.

We also use $P^{\Box} := P|\Box$.

In \cite{MeredithR05} an interpretation of the new operator is
given. It turns out that there are several possible interpretations
all enjoying the requisite algebraic properties of the operator (see
\cite{milner91polyadicpi}). We will therefore make liberal use of
$(\nu\; \vec{x})P$.

% subsection the_syntax_and_semantics_of_the_notation_system (end)   

\input{qm2pi.qmops} 

\input{qm2pi.sterngerlach} 

\input{qm2pi.metric} 

% section concurrent_process_calculi (end)

%\input{qm2pi.proofsketch}

% section proof sketch (end)

%\input{qm2pi.slviaknots} 

% section spatial logic via knots (end)

\input{qm2pi.conclusion}

% section conclusion (end)

%\input{qm2pi.dtcodes} 

% section wiring algorithm (end)

\input{qm2pi.ack} 

% section acknowledgments (end)

\newpage


\bibliographystyle{plain}   
\bibliography{../../biblios/main.bib}

\input{qm2pi.rhodetails}

\end{document}

 

% subsection basic_interpretation (end)

%\input{qm2pi.rho.presentation} 
\subsection{The syntax and semantics of the notation system}\label{sub:the_syntax_and_semantics_of_the_notation_system} % (fold)

We now summarize a technical presentation of the calculus that
embodies our theory of dynamics. The typical presentation of such a
calculus follows the style of giving generators and relations on
them. The grammar, below, describing term constructors, freely
generates the set of processes, $\Proc$. This set is then quotiented
by a relation known as structural congruence and it is over this set
that the notion of dynamics is expressed. This presentation is
essentially that of \cite{MeredithR05} with the addition of
polyadicity and summation. For readability we have relegated some of
the technical subtleties to an appendix.

\subsubsection{Process grammar}\label{subsub:process_grammar}

\begin{mathpar}
  \inferrule* [lab=synchronization] {} {{M} \bc \pzero \;|\; x?F \;|\; x!C }
  \and
  \inferrule* [lab=abstraction] {} {{F} \bc (x)P}
  \and
  \inferrule* [lab=concretion] {} {{C} \bc \langle Q \rangle}
  \and
  \inferrule* [lab=process] {} {{P,Q} \bc M \;| \;P|Q \;|\; @{x}}
  \and
  \inferrule* [lab=name] {} {{x} \bc \quotep{P}}
\end{mathpar} 

Note that $\vec{x}$ (resp. $\vec{P}$) denotes a vector of names
(resp. processes) of length $|\vec{x}|$ (resp. $|\vec{P}|$). We adopt
the following useful abbreviations.

\begin{mathpar}
   x?(\vec{y}).P := x.(\vec{y})P \and  x\clift{\vec{P}} := x.\clift{\vec{P}}
   \and x!(y) := \lift{x}{\dropn{y}}
   \and \Pi_{i=0}^{n-1}P_i := P_0 | \ldots | P_{n-1}
\end{mathpar}

\subsubsection{Structural congruence}

\paragraph{Free and bound names and alpha-equivalence.} At the
core of structural equivalence is alpha-equivalence which identifies
process that are the same up to a change of variable. Formally, we
recognize the distinction between free and bound names. The free names
of a process, $\freenames{P}$, may be calculated recursively as
follows:

\begin{mathpar}
\freenames{\pzero} := \emptyset
  \and \\
  \freenames{x?(y).P} := \{ x \} \cup (\freenames{P} \setminus \{ y \})
  \and 
  \freenames{x!\langle P \rangle} := \{ x \} \cup \{ P \} 
  \and \\
  \freenames{P|Q} := \freenames{P} \cup \freenames{Q}
  \and \\
  \freenames{@{x}} := \{ x \}
\end{mathpar}

$\pi$
$\quotep{\pi}$

$\freenames{-} : \pi \to \mathcal{P}(\quotep{\pi})$

\begin{eqnarray*}
  \freenames{\pzero} & := & \emptyset \\
  \freenames{x?(y).P} & := & \{ x \} \cup (\freenames{P} \setminus \{ y \}) \\
  \freenames{x!\langle P \rangle} & := & \{ x \} \cup \{ P \} \\
  \freenames{P|Q} & := & \freenames{P} \cup \freenames{Q} \\
  \freenames{\dropn{x}} & := & \{ x \}
\end{eqnarray*}

The bound names of a process, $\boundnames{P}$, are those names occurring in $P$
that are not free. For example, in $x?(y).0$, the name $x$ is free, while $y$ is bound.

\begin{mathpar}
  \inferrule* [lab=monoidal-laws] {} { P|Q \equiv Q|P \and P|0 \equiv P \and P|(Q|R) \equiv (P|Q)|R }
\end{mathpar}

\begin{mathpar}
  \inferrule* [lab=alpha-equivalence] {} { (x)P \equiv (y)P\{y/x\} \and y \not\in \freenames{P} }
\end{mathpar}

\begin{definition}
Then two processes, $P,Q$, are alpha-equivalent if $P = Q\{\vec{y}/\vec{x}\}$ for
some $\vec{x} \in \boundnames{Q},\vec{y} \in \boundnames{P}$, where $Q\{\vec{y}/\vec{x}\}$
denotes the capture-avoiding substitution of $\vec{y}$ for $\vec{x}$ in $Q$.
\end{definition}

\begin{definition}
  The {\em structural congruence} \cite{SangiorgiWalker} , $\equiv$,
  between processes is the least congruence containing
  alpha-equivalence, satisfying the abelian monoid laws
  (associativity, commutativity and $\pzero$ as identity) for parallel
  composition $|$ and for summation $+$.
\end{definition}

\subsection{Name equivalence}

We take name equivalence, written $\nameeq$, to be the smallest
equivalence relation generated by the following rules.

\begin{mathpar}
\inferrule*[lab=Quote-drop]
{ }
{ \quotep{@{x}} \nameeq x }

\inferrule*[lab=Struct-equiv]
{ P \scong Q }
{ \quotep{P} \nameeq \quotep{Q} }
\end{mathpar}

The astute reader will have noticed that the mutual recursion of names
and processes imposes a mutual recursion on alpha-equivalence and
structural equivalence via name-equivalence. Fortunately, all of this
works out pleasantly and we may calculate in the natural way, free of
concern. The reader interested in the details is referred to the
appendix \ref{appendix:rho_details}.

\subsection{Substitution}

We use $\Proc$ for the set of processes, $\QProc$ for the set of
names, and $\id{\{}\vec{y} / \vec{x} \id{\}}$ to denote partial maps,
$s : \QProc \rightarrow \QProc$. A map, $s$ lifts, uniquely, to a map
on process terms, $\widehat{s} : \Proc \rightarrow \Proc$ by the
following equations.

\begin{mathpar}
  (0) \psubstp{Q}{P} := 0 \\
  (R \juxtap S) \psubstp{Q}{P}
  :=    
  (R)\psubstp{Q}{P} \juxtap (S) \psubstp{Q}{P} \\
  (x?(y).R) \psubstp{Q}{P}    
  :=    
  (x)\substp{Q}{P} (z)\concat( (R \psubstn{z}{y}) \psubstp{Q}{P} ) \\
  (\lift{x}{R}) \psubstp{Q}{P}  
  :=
  \lift{(x)\substp{Q}{P}}{ R \psubstp{Q}{P} } \\
%   (\dropn{x})  \psubstp{Q}{P}       
%   := 
%   \left\{ 
%     \begin{array}{ccc} 
%       \dropn{\quotep{Q}} & & x \nameeq \quotep{P} \\
%       \dropn{x} & & otherwise \\
%     \end{array}
%   \right. 
  (\dropn{x})  \psubstp{Q}{P}       
  := 
  \left\{ 
    \begin{array}{ccc} 
      Q & & x \nameeq \quotep{P} \\
      \dropn{x} & & otherwise \\
    \end{array}
  \right.
\end{mathpar}
 

where

\begin{eqnarray}
  (x)\id{\{} \lpquote Q \rpquote / \lpquote P \rpquote \id{\}}            = 
  \left\{ 
    \begin{array}{ccc}
      \lpquote Q \rpquote & & x \nameeq \lpquote P \rpquote \\
      x & & otherwise \\
    \end{array}
  \right. \nonumber
\end{eqnarray}

and $z$ is chosen distinct from $\quotep{P}$, $\quotep{Q}$, the free
names in $Q$, and all the names in $R$. Our $\alpha$-equivalence will
be built in the standard way from this substitution.

\begin{remark}\label{rem:no_self_referential_names}
  One consequence of these definitions is that $\forall P. \quotep{P}
  \not\in \freenames{P}$.
\end{remark}

\subsection{ Dynamic quote: an example }

Anticipating something of what's to come, consider applying the
substitution, $\widehat{\id{\{}u / z \id{\}}}$, to the following pair
of processes, $\lift{w}{y!(z)}$ and $w[ \lpquote y!(z) \rpquote ]$.

\begin{eqnarray}
	\lift{w}{y!(z)}\widehat{\id{\{}u / z \id{\}}}
		& = &
		\lift{w}{y!(u)} \nonumber\\
	w[ \lpquote y!(z) \rpquote ] \widehat{ \id{\{}u / z \id{\}} }
		& = &
		w[ \lpquote y!(z) \rpquote ] \nonumber
\end{eqnarray}

Because the body of the process between quotes is impervious to
substitution, we get radically different answers. In fact, by
examining the first process in an input context,
e.g. $x?(z).\lift{w}{y!(z)}$, we see that the process under the lift
operator may be shaped by prefixed inputs binding a name inside it. In
this sense, the lift operator will be seen as a way to dynamically
construct processes before reifying them as names.

Finally equipped with these standard features we can present the
dynamics of the calculus.

\subsubsection{Operational semantics} 

Finally, we introduce the computational dynamics. What marks these
algebras as distinct from other more traditionally studied algebraic
structures, e.g. vector spaces or polynomial rings, is the manner in
which dynamics is captured. In traditional structures, dynamics is typically
expressed through morphisms between such structures, as in linear maps
between vector spaces or morphisms between rings. In algebras
associated with the semantics of computation, the dynamics is
expressed as part of the algebraic structure itself, through a
reduction reduction relation typically denoted by $\red$. Below, we
give a recursive presentation of this relation for the calculus used
in the encoding.

$\red \subseteq \pi \times \pi$
$\red : \pi \to \mathcal{P}(\pi)$

\begin{mathpar}
  \inferrule* [lab=Comm] { \textsf{match}( x_{src}, x_{trgt} ) } { x_{trgt}?(y)P \; | \; x_{src}!\langle {Q} \rangle \red P\{\quotep{Q}/y}\} }
  \and \\
  \inferrule* [lab=Par] {{P} \red {P}'} {{{P} | {Q}} \red {{P}' | {Q}}}
  \and
  \inferrule* [lab=Equiv]{{{P} \scong {P}'} \andalso {{P}' \red {Q}'} \andalso {{Q}' \scong {Q}}}{{P} \red {Q}}
\end{mathpar}

\begin{eqnarray*}
  match_{\equiv} (\quotep{P},\quotep{Q}) & := & P \equiv Q \\
  match_{\dagger}(\quotep{P},\quotep{Q}) & := & \forall R. P|Q \red^{*} R => R \red^{*} 0 \\
  match_{K}(\quotep{P},\quotep{Q}) & := & K \mbox{ for some context } K
\end{eqnarray*}

$u?(x)P | u!\langle Q \rangle \red P\{\quotep{Q}/x\}$

%We write $\wred$ for $\red^*$, and $P\red$ if $\exists Q $ such that $ P \red Q$.
We write $P\red$ if $\exists Q $ such that $ P \red Q$ and $P\not\red$, otherwise.

\section{Replication}

As mentioned before, it is known that replication (and hence
recursion) can be implemented in a higher-order process algebra
\cite{SangiorgiWalker}. As our first example of calculation with the
machinery thus far presented we give the construction explicitly in
the {\rhoc}.

\begin{eqnarray}
	D_{x} & := & \prefix{x}{y}{(\binpar{\outputp{x}{y}}{@{y}})} \nonumber\\
	\bangp_{x}{P} & := & \binpar{{x}!\langle{\binpar{D_{x}}{P}}\rangle}{D_{x}} \nonumber
\end{eqnarray}

\begin{eqnarray}
	\bangp_{x}{P} & & \nonumber\\
	=
	& {x}!\langle{(\prefix{x}{y}{(\outputp{x}{y} | @{y})) | P}}\rangle 
	      | \prefix{x}{y}{(\outputp{x}{y} | @{y})} & \nonumber\\
	\red
	& (\outputp{x}{y} | @{y})\substn{\quotep{(\prefix{x}{y}{(@{y} | \outputp{x}{y})) | P}}}{y} & \nonumber\\
	=
	& \outputp{x}{\quotep{(\prefix{x}{y}{(\outputp{x}{y} | @{y})) | P}}}
	  | {(\prefix{x}{y}{(\outputp{x}{y} | @{y})) | P}} & \nonumber\\
	\red
	& \ldots & \nonumber\\
	\red^*
	& P | P | \ldots & \nonumber
\end{eqnarray}

Of course, this encoding, as an implementation, runs away, unfolding
$\bangp{P}$ eagerly. A lazier and more implementable replication
operator, restricted to input-guarded processes, may be obtained as follows.

\begin{eqnarray}
\bangp{\prefix{u}{v}{P}} 
	:= 
	\binpar{\lift{x}{\prefix{u}{v}{(\binpar{D(x)}{P})}}}{D(x)} \nonumber
\end{eqnarray}

\begin{remark}
  Note that the lazier definition still does not deal with summation
  or mixed summation (i.e. sums over input and output). The reader is
  invited to construct definitions of replication that deal with these
  features. 

  Further, the definitions are parameterized in a name, $x$. Can you,
  gentle reader, make a definition that eliminates this parameter and
  guarantees no accidental interaction between the replication
  machinery and the process being replicated -- i.e. no accidental
  sharing of names used by the process to get its work done and the
  name(s) used by the replication to effect copying. This latter
  revision of the definition of replication is crucial to obtaining
  the expected identity $!!P \sim !P$.
\end{remark}

\begin{remark}\label{rem:paradoxical_combinator}
  The reader familiar with the lambda calculus will have noticed the
  similarity between $D$ and the paradoxical combinator.

  [Ed. note: the existence of this seems to suggest we have to be more
  restrictive on the set of processes and names we admit if we are to
  support no-cloning.]
\end{remark}

\subsubsection{Bisimulation}

The computational dynamics gives rise to another kind of equivalence,
the equivalence of computational behavior. As previously mentioned
this is typically captured \emph{via} some form of bisimulation.

% The notion we use in this paper is weak barbed bisimulation
% \cite{milner91polyadicpi}.

The notion we use in this paper is derived from weak barbed
bisimulation \cite{milner91polyadicpi}. 

\begin{definition}
An \emph{observation relation}, $\downarrow_{\mathcal N}$, over a set
of names, $\mathcal N$, is the smallest relation satisfying the rules
below.

\infrule[Out-barb]{y \in {\mathcal N}, \; x \nameeq y}
		  {\outputp{x}{v} \downarrow_{\mathcal N} x}
\infrule[Par-barb]{\mbox{$P\downarrow_{\mathcal N} x$ or $Q\downarrow_{\mathcal N} x$}}
		  {\binpar{P}{Q} \downarrow_{\mathcal N} x}

We write $P \Downarrow_{\mathcal N} x$ if there is $Q$ such that 
$P \wred Q$ and $Q \downarrow_{\mathcal N} x$.
\end{definition}

\begin{definition}
%\label{def.bbisim}
An  ${\mathcal N}$-\emph{barbed bisimulation} over a set of names, ${\mathcal N}$, is a symmetric binary relation 
${\mathcal S}_{\mathcal N}$ between agents such that $P\rel{S}_{\mathcal N}Q$ implies:
\begin{enumerate}
\item If $P \red P'$ then $Q \wred Q'$ and $P'\rel{S}_{\mathcal N} Q'$.
\item If $P\downarrow_{\mathcal N} x$, then $Q\Downarrow_{\mathcal N} x$.
\end{enumerate}
$P$ is ${\mathcal N}$-barbed bisimilar to $Q$, written
$P \wbbisim_{\mathcal N} Q$, if $P \rel{S}_{\mathcal N} Q$ for some ${\mathcal N}$-barbed bisimulation ${\mathcal S}_{\mathcal N}$.
\end{definition}

$\mathcal{R} \subseteq \pi \times \pi$

$P \mathcal{R} Q => \forall P'. P \red P' \Rightarrow \exists Q'. Q \red Q', P' \mathcal{R} Q'$

$P \vdash x \Rightarrow Q \vdash x$

\begin{mathpar}
  \inferrule*[lab=Out-barb]{x \nameeq y}{{y}!\langle{Q}\rangle \vdash x}
  \and
  \inferrule*[lab=Par-barb]{\mbox{$P\vdash x$ or $Q\vdash x$}}{\binpar{P}{Q} \vdash x}
\end{mathpar}

\subsubsection{Contexts}

One of the principle advantages of computational calculi like the
$\pi$-calculus is a well-defined notion of context,
contextual-equivalence and a correlation between
contextual-equivalence and notions of bisimulation. The notion of
context allows the decomposition of a process into (sub-)process and
its syntactic environment, its context. Thus, a context may be
thought of as a process with a ``hole'' (written $\Box$) in it. The
application of a context $M$ to a process $P$, written $M[P]$, is
tantamount to filling the hole in $M$ with $P$. In this paper we do
not need the full weight of this theory, but do make use of the notion
of context in the proof the main theorem. 

\begin{mathpar}
  \inferrule* [lab=summation] {} {{M_{M},M_{N}} \bc \Box \;|\; x.M_{A} \;|\; M_{M}+M_{N}}
  \and
  \inferrule* [lab=agent] {} {{M_{A}} \bc (\vec{x})M_{P} \;| \; \clift{P_0,\ldots,M_{P},\ldots,P_N}}
  \and \\
  \inferrule* [lab=process] {} {{M_{P}} \bc M_{N} \;| \;P|M_{P} }
\end{mathpar} 

\begin{mathpar}
  \inferrule* [lab=sychronization] {} {M_{N} \bc \Box \;|\; x?M_{F} \;|\; x!M_{C}}
  \and
  \inferrule* [lab=abstraction] {} {{M_{F}} \bc (x)M_{P} }
  \and
  \inferrule* [lab=concretion] {} {{M_{C}} \bc \langle M_{P} \rangle }
  \and \\
  \inferrule* [lab=process] {} {{M_{P}} \bc M_{N} \;| \;P|M_{P} }
\end{mathpar}

\begin{definition}[contextual application] Given a context $M$, and
  process $P$, we define the \emph{contextual application}, $M[P] :=
  M\{P/\Box\}$. That is, the contextual application of M to P is the
  substitution of $P$ for $\Box$ in $M$.
\end{definition}

$\meaningof{-} : L \to \mathcal{P}(\pi)$

\begin{mathpar}
  \inferrule* [lab=collection] {} {\meaningof{true} = \pi, \and \meaningof{~E} = \pi \setminus \meaningof{E}, \and \meaningof{E_{1} \& E_{2}} = \meaningof{E_{1}} \cap \meaningof{E_{2}}}
\end{mathpar}

\begin{mathpar}
  \inferrule* [lab=structure] {} {\meaningof{0} = \{ P \in \pi | P \equiv 0 \}, \and \\ \meaningof{E_1 | E_2} = \{ P \in \pi | P \equiv P_{1} | P_{2}, P_{1} \in \meaningof{E_{1}}, P_{2} \in \meaningof{E_2}\} }
\end{mathpar}

\begin{mathpar}
 \inferrule* [lab=behavior] {} {\meaningof{\langle a?b \rangle E} = \{ P \in \pi | P \equiv Q | u?(y)P', \\ \and \\\\ \and \\ \;\;\; u \in \meaningof{a}, \forall z.P'\{z/y\} \in \meaningof{E\{z/b\}}\}, \and \\ \meaningof{a!E} = \{ P \in \pi | P \equiv Q | x!\langle P' \rangle, x \in \meaningof{a} P' \in \meaningof{E}\} }
\end{mathpar}

\begin{mathpar}
 \inferrule* [lab=nominal] {} {\meaningof{\quotep{E}} = \{ \quotep{P} \in \quotep{\pi} | P \in \meaningof{E} \}, \and \meaningof{\quotep{P}} = \{ \quotep{Q} \in \quotep{\pi} | P \equiv Q \} \and \\ \meaningof{@\quotep{E}} = \{ P \in \pi | P \equiv @x, x \in \meaningof{E} \}}
\end{mathpar}

\begin{eqnarray*}
  \\
  \meaningof{-} : TS \to ST
\end{eqnarray*}

\begin{eqnarray*}
  \\
  L : TS \to ST
\end{eqnarray*}

\begin{eqnarray*}
  \\
  P \models E \iff P \in \meaningof{E}
\end{eqnarray*}

\begin{eqnarray*}
  P \approx_{L} Q \iff \forall E \in L. P \models E \iff Q \models E
\end{eqnarray*}

\begin{eqnarray*}
  P \approx_{K} Q
\end{eqnarray*}

\begin{eqnarray*}
  P \approx Q
\end{eqnarray*}

$\approx_{K} = \approx = \approx_{L}$

\subsubsection{Contextual duality}

Note that contexts extend the quotation operation to a family of
operations from processes to names. Given a context, $M$, we can
define a \emph{nominal context}, $\quotep{M}$ by $\quotep{M}[P] :=
\quotep{M[P]}$. To foreshadow what is to come we observe that these
operations enjoy a duality with processes very much like the duality
between vectors and maps from vectors to scalars.

Further, because the calculus is essentially higher-order, we have a
correspondence between contexts and processes. More specifically,
given a name $x$ and a context $M$ we can construct $M^{*}_{x}$ such
that 

\begin{mathpar}
  M^{*}_{x} | \lift{x}{P} \red M[P]
\end{mathpar}

namely,

\begin{mathpar}
  M^{*}_{x} := x?(u).M[\dropn{u}]
\end{mathpar}

The dependence of $M^{*}_{x}$ on a name makes it an abstraction, 

\begin{mathpar}
  M^{*} := (x)x?(u).M[\dropn{u}]
\end{mathpar}

\subsection{Additional notation}

It will sometimes be convenient to denote the process a name
quotes. We already have the notation $x = \quotep{P}$, but it will be
convenient to introduce an alternate notation, $\procn{x}$, when we
want to emphasize the connection to the use of the name. Note that, by
virtue of name equivalence, $\quotep{\procn{x}} \nameeq x$; so, the
notation is consistent with previous definitions.

Further, because names have structure it is possible to effect
substitutions on the basis of that structure. This means we need to
upgrade our notation for substitutions, which we accomplish by
adapting comprehension notation. Thus,

\begin{mathpar}
  P\{ y / x : x \in S \}
\end{mathpar}

is interpreted to mean the process derived from P by replacing (in a
capture-avoiding manner) each occurrence of $x$ in $S$ by $y$. For example,

\begin{mathpar}
  P\{ \quotep{\procn{x}|\procn{x}} / x : x \in \freenames{P} \}
\end{mathpar}

will replace each (occurrence) of a free name $x$ in $P$ by
$\quotep{\procn{x}|\procn{x}}$.

Also, we will avail ourselves of the notation $x^{L}$ and $x^{R}$ to
denote injections of a name into disjoint copies of the name
space. There are numerous ways to accomplish this. One example can be
found in \cite{MeredithR05}. This notation overloads to vectors of
names: $\vec{x}^{\pi} := (x_{i}^{\pi} \; : \; 0 \leq i < |\vec{x}| )$ where $\pi \in \{L,R\}$.

We also use $P^{\Box} := P|\Box$.

In \cite{MeredithR05} an interpretation of the new operator is
given. It turns out that there are several possible interpretations
all enjoying the requisite algebraic properties of the operator (see
\cite{milner91polyadicpi}). We will therefore make liberal use of
$(\nu\; \vec{x})P$.

% subsection the_syntax_and_semantics_of_the_notation_system (end)   

\section{Interpretation of QM}
\subsection{Supporting definitions}
\subsubsection{Multiplication}
\begin{mathpar}
  \quotep{Q} \cdot \quotep{R} := \quotep{Q|R}
  \and \\
  \quotep{Q} \cdot P := P\{ \quotep{Q|R} / \quotep{R} : \quotep{R} \in \freenames{P} \}
\end{mathpar}

\paragraph{Discussion}
The first line needs little explanation. The second line says that
each free name of the process is replaced with the multiplication of
that name by the scalar. Multiplication of a scalar (name) by a state
(process) results in a process all the names of which have been `moved
over' by parallel composition with the process the scalar
quotes. There is a subtlety that the bound names have to be
manipulated so that multiplied names aren't accidentally
captured. There are many ways to achieve this.

\begin{remark}\label{rem:multiplication_identities}
  The reader is invited to verify that for all $x,y,z \in \QProc$ and $P \in \Proc$
  \begin{mathpar}
    x \cdot \quotep{0} \equiv x 
    \and
    x \cdot y \equiv y \cdot x
    \and
    x \cdot (y \cdot z) \equiv (x \cdot y) \cdot z
    \and \\
    \quotep{0} \cdot P \equiv P
    \and \\
    x \cdot (y \cdot P) \equiv (x \cdot y) \cdot P
    \and \\
    x \cdot (P|Q) \equiv (x \cdot P) | (x \cdot Q)
    \and \\    
  \end{mathpar}
\end{remark}

\subsubsection{Tensor product}

We define a tensor product on processes by structural induction.

\paragraph{Tensor of sums} First note that all summations, including
$\pzero$ and sequence, can be written $\Sigma_{i} x_{i}.A_{i} +
\Sigma_{j} x_{j}.C_{j}$, where we have grouped input-guarded processes
together and output-guarded processes together.

Thus, we can define the tensor product of two summations, $N_{1}\otimes N_{2}$, where

\begin{mathpar}
  N_{1} := \Sigma_{i} x_{i}.A_{i} + \Sigma_{j} x_{j}.C_{j}
  \and
  N_{2} := \Sigma_{i'} y_{i'}.B_{i'} + \Sigma_{j'} y_{j'}.D_{j'} 
\end{mathpar}

as follows.

\begin{mathpar}
  \Sigma_{i} x_{i}.A_{i} + \Sigma_{j} x_{j}.C_{j} \otimes \Sigma_{i'}
  y_{i'}.B_{i'} + \Sigma_{j'} y_{j'}.D_{j'} 
  \and \\
  := \; \Sigma_{i} \Sigma_{i'} \quotep{\stackrel{\vee}{x_{i}}| \stackrel{\vee}{y_{i'}}}.(A_{i}\otimes B_{i'}) \; | \; \Sigma_{i'} \Sigma_{i} \quotep{\stackrel{\vee}{y_{i'}}|\stackrel{\vee}{x_{i}}}.(B_{i'}\otimes A_{i})
  \and
  \;\; | \;\; \Sigma_{j} \Sigma_{j'} \quotep{\stackrel{\vee}{x_{j}}|\stackrel{\vee}{y_{j'}}}.(A_{j}\otimes B_{j'}) \; | \; \Sigma_{j'} \Sigma_{j} \quotep{\stackrel{\vee}{y_{j'}}|\stackrel{\vee}{x_{j}}}.(B_{j'}\otimes A_{j})
\end{mathpar}

\begin{remark}
  Do we need to $x^{L}$ and $y^{R}$ for this construction as well?
\end{remark}

\paragraph{Tensor of parallel compositions} Next, we distribute tensor
over par.

\begin{mathpar}
  P_{1}|P_{2} \otimes Q_{1}|Q_{2} := (P_{1} \otimes Q_{1}) | (P_{1}
  \otimes Q_{2}) | (P_{2} \otimes Q_{1}) | (P_{2} \otimes Q_{2})
\end{mathpar}

\paragraph{Tensor with dropped names} We treat tensor of a
process with a dropped name as parallel composition.

\begin{mathpar}
  P \otimes \dropn{x} := P | \dropn{x}
\end{mathpar}

\paragraph{Tensor of agents}

Finally, we need to define tensor on agents. Note that the definition
of tensor on normal products only tensors inputs with inputs and
outputs with outputs. Thus, we only have to define the operation on
``homogeneous'' pairings.

\begin{mathpar}
  (\vec{x})P \otimes (\vec{y})Q
  \and \\
  := (x_{0}^{L}|y_{0}^{R},\ldots,x_{0}^{L}|y_{n}^{R},\ldots,x_{m}^{L}|y_{0}^{R},\ldots,x_{m}^{L}|y_{n}^R)(P\{ \vec{x}^{L}/\vec{x}\} \otimes Q \{ \vec{y}^{R}/\vec{y}\})
  \and \\
  \clift{\vec{P}} \otimes \clift{\vec{Q}}
  \and \\
  := \clift{P_{0}\otimes Q_{0},\ldots,P_{0}\otimes Q_{n},\ldots,P_{m}\otimes Q_{0},\ldots,P_{m}\otimes Q_{n}}
\end{mathpar}

\begin{remark}
  Observe that arities of tensored abstractions matches arities of
  tensored concretions if the original arities matched. Note also that
  the length of the arities corresponds to the increase in dimension
  we see in ordinary vector space tensor product.
\end{remark}

\begin{remark}
  Operationally, this definition distributes the tensor down to
  components ``linked'' by summation. Tensor over summation is
  intriguing in that it mixes names. Moreover, as a consequence of the
  way it mixes names we have the identities for all $x \in \QProc$ and
  $P,Q \in \Proc$

  \begin{mathpar}
    (x \cdot P) \otimes Q \equiv x \cdot (P \otimes Q) \equiv P \otimes (x \cdot Q)
    \and
    P \otimes \pzero \equiv P
  \end{mathpar}

  that the reader is invited to verify.
\end{remark}

\subsubsection{Annihilation}
\begin{mathpar}
  P^{\perp} := \{ Q | \forall R. P|Q \red^{*} R \Rightarrow R \red^{*} \pzero \}
  \and \\
  P^{\underline{\perp}} := \Sigma_{Q \in P^{\perp}} \quotep{Q}?(y).(\dropn{y}|Q) | \Sigma_{Q \in P^{\perp}} \quotep{Q}\clift{\Box}
\end{mathpar}

\paragraph{Discussion} The reader will note that $P^{\perp}$ is a
\emph{set} of processes, while $P^{\underline{\perp}}$ is a
\emph{context}. We call the set $P^{\perp}$ the \emph{annihilators} of
$P$. The parallel composition of a process in the annihilators of $P$
with $P$ will result in a process, the state space of which has all
paths eventually leading to $\pzero$. Execution may endure loops; but
under reasonable conditions of fairness (naturally guaranteed under
most notions of bisimulation) such a composite process cannot get
stuck in such a loop and will, eventually pop out and terminate.

The context $P^{\underline{\perp}}$ is ready and willing to ``take the
$P$ out of'' the process to which it is applied. It will effectively
transmit the code of the process to which it is applied to one of the
annihilators and run the process against it.

\subsubsection{Evaluation}
We fix $M$ a domain of fully abstract interpretation with an equality
coincident with bisimulation. We take $\meaningof{\cdot} : \Proc \to
M$ to be the map interpreting processes and $\nmeaningof{\cdot} : \M
\to Proc$ to be the map running the other way. Then we define

\begin{mathpar}
  \int P := \nmeaningof{\meaningof{P}}
\end{mathpar}

\paragraph{Discussion}
There are many fully abstract interpretations of Milner's
$\pi$-calculus. Any of them can be used as a basis for interpreting
the reflective calculus here. Equipped with such a domain it is
largely a matter of grinding through to check that the Yoneda
construction for the normalization-by-evaluation program can be
extended to this setting.

\begin{remark}
  The reader is invited to verify that $\int (P^{\underline{\perp}}[P]) = 0$.
\end{remark}

\subsection{Quantum mechanics}

Table \ref{tbl:core_qm_op_defns} gives the core operational definitions

\begin{table}[htp]\label{tbl:core_qm_op_defns}
  \center{
    \fbox{
      \begin{tabular}{c|c}
        quantum mechanics & process calculus \\
        \hline
        scalar & $x := \quotep{P}$ \\
        state vector & $\state{P} := P$ \\
        dual & $\state{P}^{*} := \event{P^{\underline{\perp}}} := \quotep{P^{\underline{\perp}}}[-]$ \\
        matrix & $ \Sigma_{\alpha} \state{P_{\alpha}}x_{\alpha}\event{Q_{\alpha}}$ \\
        vector addition & $\state{P} + \state{Q} := \state{P | Q}$ \\
        tensor product & $\state{P} \otimes \state{Q} := \state{P \otimes Q}$ \\
        inner product & $\innerprod{P}{Q} := \quotep{\int P^{\underline{\perp}}[Q]}$ \\
      \end{tabular}
    }
  }
  \caption{QM - operational definitions}
\end{table}

where

\begin{mathpar}
  \prmatrix{P}{Q} := \fprmatrix{P}{\quotep{\pzero}}{Q}
  \and
  \fprmatrix{P}{x}{Q} := (\state{P},x,\event{Q})
  \and
  (\fprmatrix{P}{x}{Q})(\state{R}) := x \cdot \innerprod{Q}{R} \cdot \state{P}
  \and
  (\fprmatrix{P}{x}{Q})(\event{R}) := x \cdot \innerprod{R}{P} \cdot \event{Q}
\end{mathpar}

\paragraph{Discussion}
As promised: vectors (aka states) are represented as processes; duals
as contextual duals; inner product definition should be compared with
standard inner product definition for ....

\begin{remark}
  Assuming $\int (P^{\underline{\perp}}[P]) = 0$, the reader is
  invited to verify that $(\fprmatrix{P}{x}{P})(\state{P}) = x \cdot \state{P}$.
\end{remark}

\begin{remark}
  The reader is invited to verify that $\innerprod{P}{Q}$ could
  equally well have been written $\quotep{\int \stackrel{\vee}{x}}$
  where $x = \event{P^{\underline{\perp}}}(Q)$.

  One of the motivations for this remark is that there is another way
  to factor these operations. We could package up evaluation in the dual:

  \begin{mathpar}
    \state{P}^{*} := \event{\int P^{\underline{\perp}}} := \quotep{\int P^{\underline{\perp}}}[-]
  \end{mathpar}

  and then have inner product defined by
  
  \begin{mathpar}
    \innerprod{P}{Q} := \event{P}(Q)
  \end{mathpar}

  Hopefully, experience with the calculations will provide guidance on
  the best factoring.
\end{remark}

\begin{remark}
  Assuming $\int (P^{\underline{\perp}}[P]) = 0$, the reader is
  invited to verify that $\forall P,Q. (\prmatrix{0}{Q})(\state{0}) =
  \state{0}$ and dually $(\prmatrix{P}{0})(\event{0}) = \event{0}$.
\end{remark}

\begin{remark}
  i'm a little worried that i don't (yet) have proper support for
  complex conjugacy. But, the observation above may give us a
  clue. According to Abramsky, it must be the case that the scalars
  are iso to the homset of the identity for the tensor -- which the
  observation above characterizes. 

  For now, we will simply bookmark the notion with $\overline{x}$.
\end{remark}

\subsubsection{Adjointness}

We need to give a definition of $(\cdot)^{\dagger}$ for matrices. The
obvious candidate definition is
\begin{mathpar}
(\Sigma_{\alpha}\fprmatrix{P_{\alpha}}{x_{\alpha}}{Q_{\alpha}})^{\dagger}
= \Sigma_{\alpha}\fprmatrix{(Q_{\alpha}^{\underline{\perp}})^{*}}{\overline{x}_{\alpha}}{P_{\alpha}^{\underline{\perp}}} 
\end{mathpar}

But, $(Q_{\alpha}^{\underline{\perp}})^{*}$ requires a name along
which to communicate the process to achieve the context application.

\subsubsection{Basis for a basis}
If processes label states and ``addition'' of states (a.k.a. vector
addition) is interpreted as parallel composition, what corresponds to
notions of linear independence and basis? Here, we recall that Yoshida
has developed a set of \emph{combinators} for an asynchronous verison
of Milner's $\pi$-calculus. These are a finite set of processes such
any process can be expressed as parallel composition of these
combinators together with liberal uses of the new operator and
replication. We can simply give a translation of these into the
present calculus and have reasonable expectation that the property
carries over. That is, that the resultant set allows to express all
processes via parallel composition. Note, however, that there is no
new operator or replication in this calculus. As a result, we expect
that the corresponding set is actually infinite. That is, we expect
that the space is actually infinite dimensional.

\begin{remark}
  The attentive reader may be a bit concerned. Certainly, the
  collection $S$, $K$ and $I$ is a finite set of
  combinators. Shouldn't we expect to see a finite set of combinators
  for an effectively equivalent system? i am very sympathetic to this
  critique and feel it warrants full attention. On the other hand, i
  also have in mind the following analogy. The natural numbers, as a
  monoid under addition, has exactly $1$ generator, while the natural
  numbers, as a monoid under multiplication, has countably many
  generators (the primes). We observe that the application of the
  lambda calculus is much less resource sensitive than the parallel
  composition of the $\pi$-calculus. Could it be the case that we have
  an analogy of the form
  
  \begin{mathpar}
    m + n : MN :: m*n : M|N
  \end{mathpar}

  giving a similar blow up in the set of ``primes''?  This is such a
  wonderful thought that, even if it's not true, i think it's worth
  writing down.
\end{remark}
 

\documentclass[12pt]{llncs}
%\documentclass{jktr}

\usepackage[pdftex]{hyperref}                   
\usepackage {listings}
\usepackage {mathpartir}
\usepackage{bcprules}
%\usepackage{listings}
                       
\usepackage{graphicx} 
%\usepackage[margins=2.5cm,nohead,nofoot]{geometry}
%\usepackage{geometry}
\usepackage{amsfonts}
\usepackage{amstext}
\usepackage{latexsym}
\usepackage{amssymb}
\usepackage{color}


%\include{myPreamble}
\include{qm2pi.local} 

%\ifpdf
%\usepackage[pdftex]{graphicx}
%\else
%\usepackage{graphicx}
%\fi

 % \ifpdf
%  \usepackage{pdfsync}
%  \if


%\title{Brief Article}
%\author{David F. Snyder}
%\author{L.G. Meredith}

%\address{Dept. of Math., Texas State University--San Marcos, San Marcos, TX 78666}
       
\pagestyle{empty}


\begin{document}

\lstset{language=[Objective]Caml,frame=shadowbox}

\input{qm2pi.front}

% section front matter (end)

\input{qm2pi.intro} 
 
% section introduction (end)

% \input{qm2pi.knotations} 

% section notation (end)

\input{qm2pi.process.calculi} 

% section concurrent_process_calculi_and_spatial_logics_ (end)
    
%\input{qm2pi.knots2pi} 

%\input{qm2pi.trefoil} 

%\input{qm2pi.mainthm} 

% subsection basic_interpretation (end)

%\input{qm2pi.rho.presentation} 
\subsection{The syntax and semantics of the notation system}\label{sub:the_syntax_and_semantics_of_the_notation_system} % (fold)

We now summarize a technical presentation of the calculus that
embodies our theory of dynamics. The typical presentation of such a
calculus follows the style of giving generators and relations on
them. The grammar, below, describing term constructors, freely
generates the set of processes, $\Proc$. This set is then quotiented
by a relation known as structural congruence and it is over this set
that the notion of dynamics is expressed. This presentation is
essentially that of \cite{MeredithR05} with the addition of
polyadicity and summation. For readability we have relegated some of
the technical subtleties to an appendix.

\subsubsection{Process grammar}\label{subsub:process_grammar}

\begin{mathpar}
  \inferrule* [lab=synchronization] {} {{M} \bc \pzero \;|\; x?F \;|\; x!C }
  \and
  \inferrule* [lab=abstraction] {} {{F} \bc (x)P}
  \and
  \inferrule* [lab=concretion] {} {{C} \bc \langle Q \rangle}
  \and
  \inferrule* [lab=process] {} {{P,Q} \bc M \;| \;P|Q \;|\; @{x}}
  \and
  \inferrule* [lab=name] {} {{x} \bc \quotep{P}}
\end{mathpar} 

Note that $\vec{x}$ (resp. $\vec{P}$) denotes a vector of names
(resp. processes) of length $|\vec{x}|$ (resp. $|\vec{P}|$). We adopt
the following useful abbreviations.

\begin{mathpar}
   x?(\vec{y}).P := x.(\vec{y})P \and  x\clift{\vec{P}} := x.\clift{\vec{P}}
   \and x!(y) := \lift{x}{\dropn{y}}
   \and \Pi_{i=0}^{n-1}P_i := P_0 | \ldots | P_{n-1}
\end{mathpar}

\subsubsection{Structural congruence}

\paragraph{Free and bound names and alpha-equivalence.} At the
core of structural equivalence is alpha-equivalence which identifies
process that are the same up to a change of variable. Formally, we
recognize the distinction between free and bound names. The free names
of a process, $\freenames{P}$, may be calculated recursively as
follows:

\begin{mathpar}
\freenames{\pzero} := \emptyset
  \and \\
  \freenames{x?(y).P} := \{ x \} \cup (\freenames{P} \setminus \{ y \})
  \and 
  \freenames{x!\langle P \rangle} := \{ x \} \cup \{ P \} 
  \and \\
  \freenames{P|Q} := \freenames{P} \cup \freenames{Q}
  \and \\
  \freenames{@{x}} := \{ x \}
\end{mathpar}

$\pi$
$\quotep{\pi}$

$\freenames{-} : \pi \to \mathcal{P}(\quotep{\pi})$

\begin{eqnarray*}
  \freenames{\pzero} & := & \emptyset \\
  \freenames{x?(y).P} & := & \{ x \} \cup (\freenames{P} \setminus \{ y \}) \\
  \freenames{x!\langle P \rangle} & := & \{ x \} \cup \{ P \} \\
  \freenames{P|Q} & := & \freenames{P} \cup \freenames{Q} \\
  \freenames{\dropn{x}} & := & \{ x \}
\end{eqnarray*}

The bound names of a process, $\boundnames{P}$, are those names occurring in $P$
that are not free. For example, in $x?(y).0$, the name $x$ is free, while $y$ is bound.

\begin{mathpar}
  \inferrule* [lab=monoidal-laws] {} { P|Q \equiv Q|P \and P|0 \equiv P \and P|(Q|R) \equiv (P|Q)|R }
\end{mathpar}

\begin{mathpar}
  \inferrule* [lab=alpha-equivalence] {} { (x)P \equiv (y)P\{y/x\} \and y \not\in \freenames{P} }
\end{mathpar}

\begin{definition}
Then two processes, $P,Q$, are alpha-equivalent if $P = Q\{\vec{y}/\vec{x}\}$ for
some $\vec{x} \in \boundnames{Q},\vec{y} \in \boundnames{P}$, where $Q\{\vec{y}/\vec{x}\}$
denotes the capture-avoiding substitution of $\vec{y}$ for $\vec{x}$ in $Q$.
\end{definition}

\begin{definition}
  The {\em structural congruence} \cite{SangiorgiWalker} , $\equiv$,
  between processes is the least congruence containing
  alpha-equivalence, satisfying the abelian monoid laws
  (associativity, commutativity and $\pzero$ as identity) for parallel
  composition $|$ and for summation $+$.
\end{definition}

\subsection{Name equivalence}

We take name equivalence, written $\nameeq$, to be the smallest
equivalence relation generated by the following rules.

\begin{mathpar}
\inferrule*[lab=Quote-drop]
{ }
{ \quotep{@{x}} \nameeq x }

\inferrule*[lab=Struct-equiv]
{ P \scong Q }
{ \quotep{P} \nameeq \quotep{Q} }
\end{mathpar}

The astute reader will have noticed that the mutual recursion of names
and processes imposes a mutual recursion on alpha-equivalence and
structural equivalence via name-equivalence. Fortunately, all of this
works out pleasantly and we may calculate in the natural way, free of
concern. The reader interested in the details is referred to the
appendix \ref{appendix:rho_details}.

\subsection{Substitution}

We use $\Proc$ for the set of processes, $\QProc$ for the set of
names, and $\id{\{}\vec{y} / \vec{x} \id{\}}$ to denote partial maps,
$s : \QProc \rightarrow \QProc$. A map, $s$ lifts, uniquely, to a map
on process terms, $\widehat{s} : \Proc \rightarrow \Proc$ by the
following equations.

\begin{mathpar}
  (0) \psubstp{Q}{P} := 0 \\
  (R \juxtap S) \psubstp{Q}{P}
  :=    
  (R)\psubstp{Q}{P} \juxtap (S) \psubstp{Q}{P} \\
  (x?(y).R) \psubstp{Q}{P}    
  :=    
  (x)\substp{Q}{P} (z)\concat( (R \psubstn{z}{y}) \psubstp{Q}{P} ) \\
  (\lift{x}{R}) \psubstp{Q}{P}  
  :=
  \lift{(x)\substp{Q}{P}}{ R \psubstp{Q}{P} } \\
%   (\dropn{x})  \psubstp{Q}{P}       
%   := 
%   \left\{ 
%     \begin{array}{ccc} 
%       \dropn{\quotep{Q}} & & x \nameeq \quotep{P} \\
%       \dropn{x} & & otherwise \\
%     \end{array}
%   \right. 
  (\dropn{x})  \psubstp{Q}{P}       
  := 
  \left\{ 
    \begin{array}{ccc} 
      Q & & x \nameeq \quotep{P} \\
      \dropn{x} & & otherwise \\
    \end{array}
  \right.
\end{mathpar}
 

where

\begin{eqnarray}
  (x)\id{\{} \lpquote Q \rpquote / \lpquote P \rpquote \id{\}}            = 
  \left\{ 
    \begin{array}{ccc}
      \lpquote Q \rpquote & & x \nameeq \lpquote P \rpquote \\
      x & & otherwise \\
    \end{array}
  \right. \nonumber
\end{eqnarray}

and $z$ is chosen distinct from $\quotep{P}$, $\quotep{Q}$, the free
names in $Q$, and all the names in $R$. Our $\alpha$-equivalence will
be built in the standard way from this substitution.

\begin{remark}\label{rem:no_self_referential_names}
  One consequence of these definitions is that $\forall P. \quotep{P}
  \not\in \freenames{P}$.
\end{remark}

\subsection{ Dynamic quote: an example }

Anticipating something of what's to come, consider applying the
substitution, $\widehat{\id{\{}u / z \id{\}}}$, to the following pair
of processes, $\lift{w}{y!(z)}$ and $w[ \lpquote y!(z) \rpquote ]$.

\begin{eqnarray}
	\lift{w}{y!(z)}\widehat{\id{\{}u / z \id{\}}}
		& = &
		\lift{w}{y!(u)} \nonumber\\
	w[ \lpquote y!(z) \rpquote ] \widehat{ \id{\{}u / z \id{\}} }
		& = &
		w[ \lpquote y!(z) \rpquote ] \nonumber
\end{eqnarray}

Because the body of the process between quotes is impervious to
substitution, we get radically different answers. In fact, by
examining the first process in an input context,
e.g. $x?(z).\lift{w}{y!(z)}$, we see that the process under the lift
operator may be shaped by prefixed inputs binding a name inside it. In
this sense, the lift operator will be seen as a way to dynamically
construct processes before reifying them as names.

Finally equipped with these standard features we can present the
dynamics of the calculus.

\subsubsection{Operational semantics} 

Finally, we introduce the computational dynamics. What marks these
algebras as distinct from other more traditionally studied algebraic
structures, e.g. vector spaces or polynomial rings, is the manner in
which dynamics is captured. In traditional structures, dynamics is typically
expressed through morphisms between such structures, as in linear maps
between vector spaces or morphisms between rings. In algebras
associated with the semantics of computation, the dynamics is
expressed as part of the algebraic structure itself, through a
reduction reduction relation typically denoted by $\red$. Below, we
give a recursive presentation of this relation for the calculus used
in the encoding.

$\red \subseteq \pi \times \pi$
$\red : \pi \to \mathcal{P}(\pi)$

\begin{mathpar}
  \inferrule* [lab=Comm] { \textsf{match}( x_{src}, x_{trgt} ) } { x_{trgt}?(y)P \; | \; x_{src}!\langle {Q} \rangle \red P\{\quotep{Q}/y}\} }
  \and \\
  \inferrule* [lab=Par] {{P} \red {P}'} {{{P} | {Q}} \red {{P}' | {Q}}}
  \and
  \inferrule* [lab=Equiv]{{{P} \scong {P}'} \andalso {{P}' \red {Q}'} \andalso {{Q}' \scong {Q}}}{{P} \red {Q}}
\end{mathpar}

\begin{eqnarray*}
  match_{\equiv} (\quotep{P},\quotep{Q}) & := & P \equiv Q \\
  match_{\dagger}(\quotep{P},\quotep{Q}) & := & \forall R. P|Q \red^{*} R => R \red^{*} 0 \\
  match_{K}(\quotep{P},\quotep{Q}) & := & K \mbox{ for some context } K
\end{eqnarray*}

$u?(x)P | u!\langle Q \rangle \red P\{\quotep{Q}/x\}$

%We write $\wred$ for $\red^*$, and $P\red$ if $\exists Q $ such that $ P \red Q$.
We write $P\red$ if $\exists Q $ such that $ P \red Q$ and $P\not\red$, otherwise.

\section{Replication}

As mentioned before, it is known that replication (and hence
recursion) can be implemented in a higher-order process algebra
\cite{SangiorgiWalker}. As our first example of calculation with the
machinery thus far presented we give the construction explicitly in
the {\rhoc}.

\begin{eqnarray}
	D_{x} & := & \prefix{x}{y}{(\binpar{\outputp{x}{y}}{@{y}})} \nonumber\\
	\bangp_{x}{P} & := & \binpar{{x}!\langle{\binpar{D_{x}}{P}}\rangle}{D_{x}} \nonumber
\end{eqnarray}

\begin{eqnarray}
	\bangp_{x}{P} & & \nonumber\\
	=
	& {x}!\langle{(\prefix{x}{y}{(\outputp{x}{y} | @{y})) | P}}\rangle 
	      | \prefix{x}{y}{(\outputp{x}{y} | @{y})} & \nonumber\\
	\red
	& (\outputp{x}{y} | @{y})\substn{\quotep{(\prefix{x}{y}{(@{y} | \outputp{x}{y})) | P}}}{y} & \nonumber\\
	=
	& \outputp{x}{\quotep{(\prefix{x}{y}{(\outputp{x}{y} | @{y})) | P}}}
	  | {(\prefix{x}{y}{(\outputp{x}{y} | @{y})) | P}} & \nonumber\\
	\red
	& \ldots & \nonumber\\
	\red^*
	& P | P | \ldots & \nonumber
\end{eqnarray}

Of course, this encoding, as an implementation, runs away, unfolding
$\bangp{P}$ eagerly. A lazier and more implementable replication
operator, restricted to input-guarded processes, may be obtained as follows.

\begin{eqnarray}
\bangp{\prefix{u}{v}{P}} 
	:= 
	\binpar{\lift{x}{\prefix{u}{v}{(\binpar{D(x)}{P})}}}{D(x)} \nonumber
\end{eqnarray}

\begin{remark}
  Note that the lazier definition still does not deal with summation
  or mixed summation (i.e. sums over input and output). The reader is
  invited to construct definitions of replication that deal with these
  features. 

  Further, the definitions are parameterized in a name, $x$. Can you,
  gentle reader, make a definition that eliminates this parameter and
  guarantees no accidental interaction between the replication
  machinery and the process being replicated -- i.e. no accidental
  sharing of names used by the process to get its work done and the
  name(s) used by the replication to effect copying. This latter
  revision of the definition of replication is crucial to obtaining
  the expected identity $!!P \sim !P$.
\end{remark}

\begin{remark}\label{rem:paradoxical_combinator}
  The reader familiar with the lambda calculus will have noticed the
  similarity between $D$ and the paradoxical combinator.

  [Ed. note: the existence of this seems to suggest we have to be more
  restrictive on the set of processes and names we admit if we are to
  support no-cloning.]
\end{remark}

\subsubsection{Bisimulation}

The computational dynamics gives rise to another kind of equivalence,
the equivalence of computational behavior. As previously mentioned
this is typically captured \emph{via} some form of bisimulation.

% The notion we use in this paper is weak barbed bisimulation
% \cite{milner91polyadicpi}.

The notion we use in this paper is derived from weak barbed
bisimulation \cite{milner91polyadicpi}. 

\begin{definition}
An \emph{observation relation}, $\downarrow_{\mathcal N}$, over a set
of names, $\mathcal N$, is the smallest relation satisfying the rules
below.

\infrule[Out-barb]{y \in {\mathcal N}, \; x \nameeq y}
		  {\outputp{x}{v} \downarrow_{\mathcal N} x}
\infrule[Par-barb]{\mbox{$P\downarrow_{\mathcal N} x$ or $Q\downarrow_{\mathcal N} x$}}
		  {\binpar{P}{Q} \downarrow_{\mathcal N} x}

We write $P \Downarrow_{\mathcal N} x$ if there is $Q$ such that 
$P \wred Q$ and $Q \downarrow_{\mathcal N} x$.
\end{definition}

\begin{definition}
%\label{def.bbisim}
An  ${\mathcal N}$-\emph{barbed bisimulation} over a set of names, ${\mathcal N}$, is a symmetric binary relation 
${\mathcal S}_{\mathcal N}$ between agents such that $P\rel{S}_{\mathcal N}Q$ implies:
\begin{enumerate}
\item If $P \red P'$ then $Q \wred Q'$ and $P'\rel{S}_{\mathcal N} Q'$.
\item If $P\downarrow_{\mathcal N} x$, then $Q\Downarrow_{\mathcal N} x$.
\end{enumerate}
$P$ is ${\mathcal N}$-barbed bisimilar to $Q$, written
$P \wbbisim_{\mathcal N} Q$, if $P \rel{S}_{\mathcal N} Q$ for some ${\mathcal N}$-barbed bisimulation ${\mathcal S}_{\mathcal N}$.
\end{definition}

$\mathcal{R} \subseteq \pi \times \pi$

$P \mathcal{R} Q => \forall P'. P \red P' \Rightarrow \exists Q'. Q \red Q', P' \mathcal{R} Q'$

$P \vdash x \Rightarrow Q \vdash x$

\begin{mathpar}
  \inferrule*[lab=Out-barb]{x \nameeq y}{{y}!\langle{Q}\rangle \vdash x}
  \and
  \inferrule*[lab=Par-barb]{\mbox{$P\vdash x$ or $Q\vdash x$}}{\binpar{P}{Q} \vdash x}
\end{mathpar}

\subsubsection{Contexts}

One of the principle advantages of computational calculi like the
$\pi$-calculus is a well-defined notion of context,
contextual-equivalence and a correlation between
contextual-equivalence and notions of bisimulation. The notion of
context allows the decomposition of a process into (sub-)process and
its syntactic environment, its context. Thus, a context may be
thought of as a process with a ``hole'' (written $\Box$) in it. The
application of a context $M$ to a process $P$, written $M[P]$, is
tantamount to filling the hole in $M$ with $P$. In this paper we do
not need the full weight of this theory, but do make use of the notion
of context in the proof the main theorem. 

\begin{mathpar}
  \inferrule* [lab=summation] {} {{M_{M},M_{N}} \bc \Box \;|\; x.M_{A} \;|\; M_{M}+M_{N}}
  \and
  \inferrule* [lab=agent] {} {{M_{A}} \bc (\vec{x})M_{P} \;| \; \clift{P_0,\ldots,M_{P},\ldots,P_N}}
  \and \\
  \inferrule* [lab=process] {} {{M_{P}} \bc M_{N} \;| \;P|M_{P} }
\end{mathpar} 

\begin{mathpar}
  \inferrule* [lab=sychronization] {} {M_{N} \bc \Box \;|\; x?M_{F} \;|\; x!M_{C}}
  \and
  \inferrule* [lab=abstraction] {} {{M_{F}} \bc (x)M_{P} }
  \and
  \inferrule* [lab=concretion] {} {{M_{C}} \bc \langle M_{P} \rangle }
  \and \\
  \inferrule* [lab=process] {} {{M_{P}} \bc M_{N} \;| \;P|M_{P} }
\end{mathpar}

\begin{definition}[contextual application] Given a context $M$, and
  process $P$, we define the \emph{contextual application}, $M[P] :=
  M\{P/\Box\}$. That is, the contextual application of M to P is the
  substitution of $P$ for $\Box$ in $M$.
\end{definition}

$\meaningof{-} : L \to \mathcal{P}(\pi)$

\begin{mathpar}
  \inferrule* [lab=collection] {} {\meaningof{true} = \pi, \and \meaningof{~E} = \pi \setminus \meaningof{E}, \and \meaningof{E_{1} \& E_{2}} = \meaningof{E_{1}} \cap \meaningof{E_{2}}}
\end{mathpar}

\begin{mathpar}
  \inferrule* [lab=structure] {} {\meaningof{0} = \{ P \in \pi | P \equiv 0 \}, \and \\ \meaningof{E_1 | E_2} = \{ P \in \pi | P \equiv P_{1} | P_{2}, P_{1} \in \meaningof{E_{1}}, P_{2} \in \meaningof{E_2}\} }
\end{mathpar}

\begin{mathpar}
 \inferrule* [lab=behavior] {} {\meaningof{\langle a?b \rangle E} = \{ P \in \pi | P \equiv Q | u?(y)P', \\ \and \\\\ \and \\ \;\;\; u \in \meaningof{a}, \forall z.P'\{z/y\} \in \meaningof{E\{z/b\}}\}, \and \\ \meaningof{a!E} = \{ P \in \pi | P \equiv Q | x!\langle P' \rangle, x \in \meaningof{a} P' \in \meaningof{E}\} }
\end{mathpar}

\begin{mathpar}
 \inferrule* [lab=nominal] {} {\meaningof{\quotep{E}} = \{ \quotep{P} \in \quotep{\pi} | P \in \meaningof{E} \}, \and \meaningof{\quotep{P}} = \{ \quotep{Q} \in \quotep{\pi} | P \equiv Q \} \and \\ \meaningof{@\quotep{E}} = \{ P \in \pi | P \equiv @x, x \in \meaningof{E} \}}
\end{mathpar}

\begin{eqnarray*}
  \\
  \meaningof{-} : TS \to ST
\end{eqnarray*}

\begin{eqnarray*}
  \\
  L : TS \to ST
\end{eqnarray*}

\begin{eqnarray*}
  \\
  P \models E \iff P \in \meaningof{E}
\end{eqnarray*}

\begin{eqnarray*}
  P \approx_{L} Q \iff \forall E \in L. P \models E \iff Q \models E
\end{eqnarray*}

\begin{eqnarray*}
  P \approx_{K} Q
\end{eqnarray*}

\begin{eqnarray*}
  P \approx Q
\end{eqnarray*}

$\approx_{K} = \approx = \approx_{L}$

\subsubsection{Contextual duality}

Note that contexts extend the quotation operation to a family of
operations from processes to names. Given a context, $M$, we can
define a \emph{nominal context}, $\quotep{M}$ by $\quotep{M}[P] :=
\quotep{M[P]}$. To foreshadow what is to come we observe that these
operations enjoy a duality with processes very much like the duality
between vectors and maps from vectors to scalars.

Further, because the calculus is essentially higher-order, we have a
correspondence between contexts and processes. More specifically,
given a name $x$ and a context $M$ we can construct $M^{*}_{x}$ such
that 

\begin{mathpar}
  M^{*}_{x} | \lift{x}{P} \red M[P]
\end{mathpar}

namely,

\begin{mathpar}
  M^{*}_{x} := x?(u).M[\dropn{u}]
\end{mathpar}

The dependence of $M^{*}_{x}$ on a name makes it an abstraction, 

\begin{mathpar}
  M^{*} := (x)x?(u).M[\dropn{u}]
\end{mathpar}

\subsection{Additional notation}

It will sometimes be convenient to denote the process a name
quotes. We already have the notation $x = \quotep{P}$, but it will be
convenient to introduce an alternate notation, $\procn{x}$, when we
want to emphasize the connection to the use of the name. Note that, by
virtue of name equivalence, $\quotep{\procn{x}} \nameeq x$; so, the
notation is consistent with previous definitions.

Further, because names have structure it is possible to effect
substitutions on the basis of that structure. This means we need to
upgrade our notation for substitutions, which we accomplish by
adapting comprehension notation. Thus,

\begin{mathpar}
  P\{ y / x : x \in S \}
\end{mathpar}

is interpreted to mean the process derived from P by replacing (in a
capture-avoiding manner) each occurrence of $x$ in $S$ by $y$. For example,

\begin{mathpar}
  P\{ \quotep{\procn{x}|\procn{x}} / x : x \in \freenames{P} \}
\end{mathpar}

will replace each (occurrence) of a free name $x$ in $P$ by
$\quotep{\procn{x}|\procn{x}}$.

Also, we will avail ourselves of the notation $x^{L}$ and $x^{R}$ to
denote injections of a name into disjoint copies of the name
space. There are numerous ways to accomplish this. One example can be
found in \cite{MeredithR05}. This notation overloads to vectors of
names: $\vec{x}^{\pi} := (x_{i}^{\pi} \; : \; 0 \leq i < |\vec{x}| )$ where $\pi \in \{L,R\}$.

We also use $P^{\Box} := P|\Box$.

In \cite{MeredithR05} an interpretation of the new operator is
given. It turns out that there are several possible interpretations
all enjoying the requisite algebraic properties of the operator (see
\cite{milner91polyadicpi}). We will therefore make liberal use of
$(\nu\; \vec{x})P$.

% subsection the_syntax_and_semantics_of_the_notation_system (end)   

\input{qm2pi.qmops} 

\input{qm2pi.sterngerlach} 

\input{qm2pi.metric} 

% section concurrent_process_calculi (end)

%\input{qm2pi.proofsketch}

% section proof sketch (end)

%\input{qm2pi.slviaknots} 

% section spatial logic via knots (end)

\input{qm2pi.conclusion}

% section conclusion (end)

%\input{qm2pi.dtcodes} 

% section wiring algorithm (end)

\input{qm2pi.ack} 

% section acknowledgments (end)

\newpage


\bibliographystyle{plain}   
\bibliography{../../biblios/main.bib}

\input{qm2pi.rhodetails}

\end{document}

 

\documentclass[12pt]{llncs}
%\documentclass{jktr}

\usepackage[pdftex]{hyperref}                   
\usepackage {listings}
\usepackage {mathpartir}
\usepackage{bcprules}
%\usepackage{listings}
                       
\usepackage{graphicx} 
%\usepackage[margins=2.5cm,nohead,nofoot]{geometry}
%\usepackage{geometry}
\usepackage{amsfonts}
\usepackage{amstext}
\usepackage{latexsym}
\usepackage{amssymb}
\usepackage{color}


%\include{myPreamble}
\include{qm2pi.local} 

%\ifpdf
%\usepackage[pdftex]{graphicx}
%\else
%\usepackage{graphicx}
%\fi

 % \ifpdf
%  \usepackage{pdfsync}
%  \if


%\title{Brief Article}
%\author{David F. Snyder}
%\author{L.G. Meredith}

%\address{Dept. of Math., Texas State University--San Marcos, San Marcos, TX 78666}
       
\pagestyle{empty}


\begin{document}

\lstset{language=[Objective]Caml,frame=shadowbox}

\input{qm2pi.front}

% section front matter (end)

\input{qm2pi.intro} 
 
% section introduction (end)

% \input{qm2pi.knotations} 

% section notation (end)

\input{qm2pi.process.calculi} 

% section concurrent_process_calculi_and_spatial_logics_ (end)
    
%\input{qm2pi.knots2pi} 

%\input{qm2pi.trefoil} 

%\input{qm2pi.mainthm} 

% subsection basic_interpretation (end)

%\input{qm2pi.rho.presentation} 
\subsection{The syntax and semantics of the notation system}\label{sub:the_syntax_and_semantics_of_the_notation_system} % (fold)

We now summarize a technical presentation of the calculus that
embodies our theory of dynamics. The typical presentation of such a
calculus follows the style of giving generators and relations on
them. The grammar, below, describing term constructors, freely
generates the set of processes, $\Proc$. This set is then quotiented
by a relation known as structural congruence and it is over this set
that the notion of dynamics is expressed. This presentation is
essentially that of \cite{MeredithR05} with the addition of
polyadicity and summation. For readability we have relegated some of
the technical subtleties to an appendix.

\subsubsection{Process grammar}\label{subsub:process_grammar}

\begin{mathpar}
  \inferrule* [lab=synchronization] {} {{M} \bc \pzero \;|\; x?F \;|\; x!C }
  \and
  \inferrule* [lab=abstraction] {} {{F} \bc (x)P}
  \and
  \inferrule* [lab=concretion] {} {{C} \bc \langle Q \rangle}
  \and
  \inferrule* [lab=process] {} {{P,Q} \bc M \;| \;P|Q \;|\; @{x}}
  \and
  \inferrule* [lab=name] {} {{x} \bc \quotep{P}}
\end{mathpar} 

Note that $\vec{x}$ (resp. $\vec{P}$) denotes a vector of names
(resp. processes) of length $|\vec{x}|$ (resp. $|\vec{P}|$). We adopt
the following useful abbreviations.

\begin{mathpar}
   x?(\vec{y}).P := x.(\vec{y})P \and  x\clift{\vec{P}} := x.\clift{\vec{P}}
   \and x!(y) := \lift{x}{\dropn{y}}
   \and \Pi_{i=0}^{n-1}P_i := P_0 | \ldots | P_{n-1}
\end{mathpar}

\subsubsection{Structural congruence}

\paragraph{Free and bound names and alpha-equivalence.} At the
core of structural equivalence is alpha-equivalence which identifies
process that are the same up to a change of variable. Formally, we
recognize the distinction between free and bound names. The free names
of a process, $\freenames{P}$, may be calculated recursively as
follows:

\begin{mathpar}
\freenames{\pzero} := \emptyset
  \and \\
  \freenames{x?(y).P} := \{ x \} \cup (\freenames{P} \setminus \{ y \})
  \and 
  \freenames{x!\langle P \rangle} := \{ x \} \cup \{ P \} 
  \and \\
  \freenames{P|Q} := \freenames{P} \cup \freenames{Q}
  \and \\
  \freenames{@{x}} := \{ x \}
\end{mathpar}

$\pi$
$\quotep{\pi}$

$\freenames{-} : \pi \to \mathcal{P}(\quotep{\pi})$

\begin{eqnarray*}
  \freenames{\pzero} & := & \emptyset \\
  \freenames{x?(y).P} & := & \{ x \} \cup (\freenames{P} \setminus \{ y \}) \\
  \freenames{x!\langle P \rangle} & := & \{ x \} \cup \{ P \} \\
  \freenames{P|Q} & := & \freenames{P} \cup \freenames{Q} \\
  \freenames{\dropn{x}} & := & \{ x \}
\end{eqnarray*}

The bound names of a process, $\boundnames{P}$, are those names occurring in $P$
that are not free. For example, in $x?(y).0$, the name $x$ is free, while $y$ is bound.

\begin{mathpar}
  \inferrule* [lab=monoidal-laws] {} { P|Q \equiv Q|P \and P|0 \equiv P \and P|(Q|R) \equiv (P|Q)|R }
\end{mathpar}

\begin{mathpar}
  \inferrule* [lab=alpha-equivalence] {} { (x)P \equiv (y)P\{y/x\} \and y \not\in \freenames{P} }
\end{mathpar}

\begin{definition}
Then two processes, $P,Q$, are alpha-equivalent if $P = Q\{\vec{y}/\vec{x}\}$ for
some $\vec{x} \in \boundnames{Q},\vec{y} \in \boundnames{P}$, where $Q\{\vec{y}/\vec{x}\}$
denotes the capture-avoiding substitution of $\vec{y}$ for $\vec{x}$ in $Q$.
\end{definition}

\begin{definition}
  The {\em structural congruence} \cite{SangiorgiWalker} , $\equiv$,
  between processes is the least congruence containing
  alpha-equivalence, satisfying the abelian monoid laws
  (associativity, commutativity and $\pzero$ as identity) for parallel
  composition $|$ and for summation $+$.
\end{definition}

\subsection{Name equivalence}

We take name equivalence, written $\nameeq$, to be the smallest
equivalence relation generated by the following rules.

\begin{mathpar}
\inferrule*[lab=Quote-drop]
{ }
{ \quotep{@{x}} \nameeq x }

\inferrule*[lab=Struct-equiv]
{ P \scong Q }
{ \quotep{P} \nameeq \quotep{Q} }
\end{mathpar}

The astute reader will have noticed that the mutual recursion of names
and processes imposes a mutual recursion on alpha-equivalence and
structural equivalence via name-equivalence. Fortunately, all of this
works out pleasantly and we may calculate in the natural way, free of
concern. The reader interested in the details is referred to the
appendix \ref{appendix:rho_details}.

\subsection{Substitution}

We use $\Proc$ for the set of processes, $\QProc$ for the set of
names, and $\id{\{}\vec{y} / \vec{x} \id{\}}$ to denote partial maps,
$s : \QProc \rightarrow \QProc$. A map, $s$ lifts, uniquely, to a map
on process terms, $\widehat{s} : \Proc \rightarrow \Proc$ by the
following equations.

\begin{mathpar}
  (0) \psubstp{Q}{P} := 0 \\
  (R \juxtap S) \psubstp{Q}{P}
  :=    
  (R)\psubstp{Q}{P} \juxtap (S) \psubstp{Q}{P} \\
  (x?(y).R) \psubstp{Q}{P}    
  :=    
  (x)\substp{Q}{P} (z)\concat( (R \psubstn{z}{y}) \psubstp{Q}{P} ) \\
  (\lift{x}{R}) \psubstp{Q}{P}  
  :=
  \lift{(x)\substp{Q}{P}}{ R \psubstp{Q}{P} } \\
%   (\dropn{x})  \psubstp{Q}{P}       
%   := 
%   \left\{ 
%     \begin{array}{ccc} 
%       \dropn{\quotep{Q}} & & x \nameeq \quotep{P} \\
%       \dropn{x} & & otherwise \\
%     \end{array}
%   \right. 
  (\dropn{x})  \psubstp{Q}{P}       
  := 
  \left\{ 
    \begin{array}{ccc} 
      Q & & x \nameeq \quotep{P} \\
      \dropn{x} & & otherwise \\
    \end{array}
  \right.
\end{mathpar}
 

where

\begin{eqnarray}
  (x)\id{\{} \lpquote Q \rpquote / \lpquote P \rpquote \id{\}}            = 
  \left\{ 
    \begin{array}{ccc}
      \lpquote Q \rpquote & & x \nameeq \lpquote P \rpquote \\
      x & & otherwise \\
    \end{array}
  \right. \nonumber
\end{eqnarray}

and $z$ is chosen distinct from $\quotep{P}$, $\quotep{Q}$, the free
names in $Q$, and all the names in $R$. Our $\alpha$-equivalence will
be built in the standard way from this substitution.

\begin{remark}\label{rem:no_self_referential_names}
  One consequence of these definitions is that $\forall P. \quotep{P}
  \not\in \freenames{P}$.
\end{remark}

\subsection{ Dynamic quote: an example }

Anticipating something of what's to come, consider applying the
substitution, $\widehat{\id{\{}u / z \id{\}}}$, to the following pair
of processes, $\lift{w}{y!(z)}$ and $w[ \lpquote y!(z) \rpquote ]$.

\begin{eqnarray}
	\lift{w}{y!(z)}\widehat{\id{\{}u / z \id{\}}}
		& = &
		\lift{w}{y!(u)} \nonumber\\
	w[ \lpquote y!(z) \rpquote ] \widehat{ \id{\{}u / z \id{\}} }
		& = &
		w[ \lpquote y!(z) \rpquote ] \nonumber
\end{eqnarray}

Because the body of the process between quotes is impervious to
substitution, we get radically different answers. In fact, by
examining the first process in an input context,
e.g. $x?(z).\lift{w}{y!(z)}$, we see that the process under the lift
operator may be shaped by prefixed inputs binding a name inside it. In
this sense, the lift operator will be seen as a way to dynamically
construct processes before reifying them as names.

Finally equipped with these standard features we can present the
dynamics of the calculus.

\subsubsection{Operational semantics} 

Finally, we introduce the computational dynamics. What marks these
algebras as distinct from other more traditionally studied algebraic
structures, e.g. vector spaces or polynomial rings, is the manner in
which dynamics is captured. In traditional structures, dynamics is typically
expressed through morphisms between such structures, as in linear maps
between vector spaces or morphisms between rings. In algebras
associated with the semantics of computation, the dynamics is
expressed as part of the algebraic structure itself, through a
reduction reduction relation typically denoted by $\red$. Below, we
give a recursive presentation of this relation for the calculus used
in the encoding.

$\red \subseteq \pi \times \pi$
$\red : \pi \to \mathcal{P}(\pi)$

\begin{mathpar}
  \inferrule* [lab=Comm] { \textsf{match}( x_{src}, x_{trgt} ) } { x_{trgt}?(y)P \; | \; x_{src}!\langle {Q} \rangle \red P\{\quotep{Q}/y}\} }
  \and \\
  \inferrule* [lab=Par] {{P} \red {P}'} {{{P} | {Q}} \red {{P}' | {Q}}}
  \and
  \inferrule* [lab=Equiv]{{{P} \scong {P}'} \andalso {{P}' \red {Q}'} \andalso {{Q}' \scong {Q}}}{{P} \red {Q}}
\end{mathpar}

\begin{eqnarray*}
  match_{\equiv} (\quotep{P},\quotep{Q}) & := & P \equiv Q \\
  match_{\dagger}(\quotep{P},\quotep{Q}) & := & \forall R. P|Q \red^{*} R => R \red^{*} 0 \\
  match_{K}(\quotep{P},\quotep{Q}) & := & K \mbox{ for some context } K
\end{eqnarray*}

$u?(x)P | u!\langle Q \rangle \red P\{\quotep{Q}/x\}$

%We write $\wred$ for $\red^*$, and $P\red$ if $\exists Q $ such that $ P \red Q$.
We write $P\red$ if $\exists Q $ such that $ P \red Q$ and $P\not\red$, otherwise.

\section{Replication}

As mentioned before, it is known that replication (and hence
recursion) can be implemented in a higher-order process algebra
\cite{SangiorgiWalker}. As our first example of calculation with the
machinery thus far presented we give the construction explicitly in
the {\rhoc}.

\begin{eqnarray}
	D_{x} & := & \prefix{x}{y}{(\binpar{\outputp{x}{y}}{@{y}})} \nonumber\\
	\bangp_{x}{P} & := & \binpar{{x}!\langle{\binpar{D_{x}}{P}}\rangle}{D_{x}} \nonumber
\end{eqnarray}

\begin{eqnarray}
	\bangp_{x}{P} & & \nonumber\\
	=
	& {x}!\langle{(\prefix{x}{y}{(\outputp{x}{y} | @{y})) | P}}\rangle 
	      | \prefix{x}{y}{(\outputp{x}{y} | @{y})} & \nonumber\\
	\red
	& (\outputp{x}{y} | @{y})\substn{\quotep{(\prefix{x}{y}{(@{y} | \outputp{x}{y})) | P}}}{y} & \nonumber\\
	=
	& \outputp{x}{\quotep{(\prefix{x}{y}{(\outputp{x}{y} | @{y})) | P}}}
	  | {(\prefix{x}{y}{(\outputp{x}{y} | @{y})) | P}} & \nonumber\\
	\red
	& \ldots & \nonumber\\
	\red^*
	& P | P | \ldots & \nonumber
\end{eqnarray}

Of course, this encoding, as an implementation, runs away, unfolding
$\bangp{P}$ eagerly. A lazier and more implementable replication
operator, restricted to input-guarded processes, may be obtained as follows.

\begin{eqnarray}
\bangp{\prefix{u}{v}{P}} 
	:= 
	\binpar{\lift{x}{\prefix{u}{v}{(\binpar{D(x)}{P})}}}{D(x)} \nonumber
\end{eqnarray}

\begin{remark}
  Note that the lazier definition still does not deal with summation
  or mixed summation (i.e. sums over input and output). The reader is
  invited to construct definitions of replication that deal with these
  features. 

  Further, the definitions are parameterized in a name, $x$. Can you,
  gentle reader, make a definition that eliminates this parameter and
  guarantees no accidental interaction between the replication
  machinery and the process being replicated -- i.e. no accidental
  sharing of names used by the process to get its work done and the
  name(s) used by the replication to effect copying. This latter
  revision of the definition of replication is crucial to obtaining
  the expected identity $!!P \sim !P$.
\end{remark}

\begin{remark}\label{rem:paradoxical_combinator}
  The reader familiar with the lambda calculus will have noticed the
  similarity between $D$ and the paradoxical combinator.

  [Ed. note: the existence of this seems to suggest we have to be more
  restrictive on the set of processes and names we admit if we are to
  support no-cloning.]
\end{remark}

\subsubsection{Bisimulation}

The computational dynamics gives rise to another kind of equivalence,
the equivalence of computational behavior. As previously mentioned
this is typically captured \emph{via} some form of bisimulation.

% The notion we use in this paper is weak barbed bisimulation
% \cite{milner91polyadicpi}.

The notion we use in this paper is derived from weak barbed
bisimulation \cite{milner91polyadicpi}. 

\begin{definition}
An \emph{observation relation}, $\downarrow_{\mathcal N}$, over a set
of names, $\mathcal N$, is the smallest relation satisfying the rules
below.

\infrule[Out-barb]{y \in {\mathcal N}, \; x \nameeq y}
		  {\outputp{x}{v} \downarrow_{\mathcal N} x}
\infrule[Par-barb]{\mbox{$P\downarrow_{\mathcal N} x$ or $Q\downarrow_{\mathcal N} x$}}
		  {\binpar{P}{Q} \downarrow_{\mathcal N} x}

We write $P \Downarrow_{\mathcal N} x$ if there is $Q$ such that 
$P \wred Q$ and $Q \downarrow_{\mathcal N} x$.
\end{definition}

\begin{definition}
%\label{def.bbisim}
An  ${\mathcal N}$-\emph{barbed bisimulation} over a set of names, ${\mathcal N}$, is a symmetric binary relation 
${\mathcal S}_{\mathcal N}$ between agents such that $P\rel{S}_{\mathcal N}Q$ implies:
\begin{enumerate}
\item If $P \red P'$ then $Q \wred Q'$ and $P'\rel{S}_{\mathcal N} Q'$.
\item If $P\downarrow_{\mathcal N} x$, then $Q\Downarrow_{\mathcal N} x$.
\end{enumerate}
$P$ is ${\mathcal N}$-barbed bisimilar to $Q$, written
$P \wbbisim_{\mathcal N} Q$, if $P \rel{S}_{\mathcal N} Q$ for some ${\mathcal N}$-barbed bisimulation ${\mathcal S}_{\mathcal N}$.
\end{definition}

$\mathcal{R} \subseteq \pi \times \pi$

$P \mathcal{R} Q => \forall P'. P \red P' \Rightarrow \exists Q'. Q \red Q', P' \mathcal{R} Q'$

$P \vdash x \Rightarrow Q \vdash x$

\begin{mathpar}
  \inferrule*[lab=Out-barb]{x \nameeq y}{{y}!\langle{Q}\rangle \vdash x}
  \and
  \inferrule*[lab=Par-barb]{\mbox{$P\vdash x$ or $Q\vdash x$}}{\binpar{P}{Q} \vdash x}
\end{mathpar}

\subsubsection{Contexts}

One of the principle advantages of computational calculi like the
$\pi$-calculus is a well-defined notion of context,
contextual-equivalence and a correlation between
contextual-equivalence and notions of bisimulation. The notion of
context allows the decomposition of a process into (sub-)process and
its syntactic environment, its context. Thus, a context may be
thought of as a process with a ``hole'' (written $\Box$) in it. The
application of a context $M$ to a process $P$, written $M[P]$, is
tantamount to filling the hole in $M$ with $P$. In this paper we do
not need the full weight of this theory, but do make use of the notion
of context in the proof the main theorem. 

\begin{mathpar}
  \inferrule* [lab=summation] {} {{M_{M},M_{N}} \bc \Box \;|\; x.M_{A} \;|\; M_{M}+M_{N}}
  \and
  \inferrule* [lab=agent] {} {{M_{A}} \bc (\vec{x})M_{P} \;| \; \clift{P_0,\ldots,M_{P},\ldots,P_N}}
  \and \\
  \inferrule* [lab=process] {} {{M_{P}} \bc M_{N} \;| \;P|M_{P} }
\end{mathpar} 

\begin{mathpar}
  \inferrule* [lab=sychronization] {} {M_{N} \bc \Box \;|\; x?M_{F} \;|\; x!M_{C}}
  \and
  \inferrule* [lab=abstraction] {} {{M_{F}} \bc (x)M_{P} }
  \and
  \inferrule* [lab=concretion] {} {{M_{C}} \bc \langle M_{P} \rangle }
  \and \\
  \inferrule* [lab=process] {} {{M_{P}} \bc M_{N} \;| \;P|M_{P} }
\end{mathpar}

\begin{definition}[contextual application] Given a context $M$, and
  process $P$, we define the \emph{contextual application}, $M[P] :=
  M\{P/\Box\}$. That is, the contextual application of M to P is the
  substitution of $P$ for $\Box$ in $M$.
\end{definition}

$\meaningof{-} : L \to \mathcal{P}(\pi)$

\begin{mathpar}
  \inferrule* [lab=collection] {} {\meaningof{true} = \pi, \and \meaningof{~E} = \pi \setminus \meaningof{E}, \and \meaningof{E_{1} \& E_{2}} = \meaningof{E_{1}} \cap \meaningof{E_{2}}}
\end{mathpar}

\begin{mathpar}
  \inferrule* [lab=structure] {} {\meaningof{0} = \{ P \in \pi | P \equiv 0 \}, \and \\ \meaningof{E_1 | E_2} = \{ P \in \pi | P \equiv P_{1} | P_{2}, P_{1} \in \meaningof{E_{1}}, P_{2} \in \meaningof{E_2}\} }
\end{mathpar}

\begin{mathpar}
 \inferrule* [lab=behavior] {} {\meaningof{\langle a?b \rangle E} = \{ P \in \pi | P \equiv Q | u?(y)P', \\ \and \\\\ \and \\ \;\;\; u \in \meaningof{a}, \forall z.P'\{z/y\} \in \meaningof{E\{z/b\}}\}, \and \\ \meaningof{a!E} = \{ P \in \pi | P \equiv Q | x!\langle P' \rangle, x \in \meaningof{a} P' \in \meaningof{E}\} }
\end{mathpar}

\begin{mathpar}
 \inferrule* [lab=nominal] {} {\meaningof{\quotep{E}} = \{ \quotep{P} \in \quotep{\pi} | P \in \meaningof{E} \}, \and \meaningof{\quotep{P}} = \{ \quotep{Q} \in \quotep{\pi} | P \equiv Q \} \and \\ \meaningof{@\quotep{E}} = \{ P \in \pi | P \equiv @x, x \in \meaningof{E} \}}
\end{mathpar}

\begin{eqnarray*}
  \\
  \meaningof{-} : TS \to ST
\end{eqnarray*}

\begin{eqnarray*}
  \\
  L : TS \to ST
\end{eqnarray*}

\begin{eqnarray*}
  \\
  P \models E \iff P \in \meaningof{E}
\end{eqnarray*}

\begin{eqnarray*}
  P \approx_{L} Q \iff \forall E \in L. P \models E \iff Q \models E
\end{eqnarray*}

\begin{eqnarray*}
  P \approx_{K} Q
\end{eqnarray*}

\begin{eqnarray*}
  P \approx Q
\end{eqnarray*}

$\approx_{K} = \approx = \approx_{L}$

\subsubsection{Contextual duality}

Note that contexts extend the quotation operation to a family of
operations from processes to names. Given a context, $M$, we can
define a \emph{nominal context}, $\quotep{M}$ by $\quotep{M}[P] :=
\quotep{M[P]}$. To foreshadow what is to come we observe that these
operations enjoy a duality with processes very much like the duality
between vectors and maps from vectors to scalars.

Further, because the calculus is essentially higher-order, we have a
correspondence between contexts and processes. More specifically,
given a name $x$ and a context $M$ we can construct $M^{*}_{x}$ such
that 

\begin{mathpar}
  M^{*}_{x} | \lift{x}{P} \red M[P]
\end{mathpar}

namely,

\begin{mathpar}
  M^{*}_{x} := x?(u).M[\dropn{u}]
\end{mathpar}

The dependence of $M^{*}_{x}$ on a name makes it an abstraction, 

\begin{mathpar}
  M^{*} := (x)x?(u).M[\dropn{u}]
\end{mathpar}

\subsection{Additional notation}

It will sometimes be convenient to denote the process a name
quotes. We already have the notation $x = \quotep{P}$, but it will be
convenient to introduce an alternate notation, $\procn{x}$, when we
want to emphasize the connection to the use of the name. Note that, by
virtue of name equivalence, $\quotep{\procn{x}} \nameeq x$; so, the
notation is consistent with previous definitions.

Further, because names have structure it is possible to effect
substitutions on the basis of that structure. This means we need to
upgrade our notation for substitutions, which we accomplish by
adapting comprehension notation. Thus,

\begin{mathpar}
  P\{ y / x : x \in S \}
\end{mathpar}

is interpreted to mean the process derived from P by replacing (in a
capture-avoiding manner) each occurrence of $x$ in $S$ by $y$. For example,

\begin{mathpar}
  P\{ \quotep{\procn{x}|\procn{x}} / x : x \in \freenames{P} \}
\end{mathpar}

will replace each (occurrence) of a free name $x$ in $P$ by
$\quotep{\procn{x}|\procn{x}}$.

Also, we will avail ourselves of the notation $x^{L}$ and $x^{R}$ to
denote injections of a name into disjoint copies of the name
space. There are numerous ways to accomplish this. One example can be
found in \cite{MeredithR05}. This notation overloads to vectors of
names: $\vec{x}^{\pi} := (x_{i}^{\pi} \; : \; 0 \leq i < |\vec{x}| )$ where $\pi \in \{L,R\}$.

We also use $P^{\Box} := P|\Box$.

In \cite{MeredithR05} an interpretation of the new operator is
given. It turns out that there are several possible interpretations
all enjoying the requisite algebraic properties of the operator (see
\cite{milner91polyadicpi}). We will therefore make liberal use of
$(\nu\; \vec{x})P$.

% subsection the_syntax_and_semantics_of_the_notation_system (end)   

\input{qm2pi.qmops} 

\input{qm2pi.sterngerlach} 

\input{qm2pi.metric} 

% section concurrent_process_calculi (end)

%\input{qm2pi.proofsketch}

% section proof sketch (end)

%\input{qm2pi.slviaknots} 

% section spatial logic via knots (end)

\input{qm2pi.conclusion}

% section conclusion (end)

%\input{qm2pi.dtcodes} 

% section wiring algorithm (end)

\input{qm2pi.ack} 

% section acknowledgments (end)

\newpage


\bibliographystyle{plain}   
\bibliography{../../biblios/main.bib}

\input{qm2pi.rhodetails}

\end{document}

 

% section concurrent_process_calculi (end)

%\documentclass[12pt]{llncs}
%\documentclass{jktr}

\usepackage[pdftex]{hyperref}                   
\usepackage {listings}
\usepackage {mathpartir}
\usepackage{bcprules}
%\usepackage{listings}
                       
\usepackage{graphicx} 
%\usepackage[margins=2.5cm,nohead,nofoot]{geometry}
%\usepackage{geometry}
\usepackage{amsfonts}
\usepackage{amstext}
\usepackage{latexsym}
\usepackage{amssymb}
\usepackage{color}


%\include{myPreamble}
\include{qm2pi.local} 

%\ifpdf
%\usepackage[pdftex]{graphicx}
%\else
%\usepackage{graphicx}
%\fi

 % \ifpdf
%  \usepackage{pdfsync}
%  \if


%\title{Brief Article}
%\author{David F. Snyder}
%\author{L.G. Meredith}

%\address{Dept. of Math., Texas State University--San Marcos, San Marcos, TX 78666}
       
\pagestyle{empty}


\begin{document}

\lstset{language=[Objective]Caml,frame=shadowbox}

\input{qm2pi.front}

% section front matter (end)

\input{qm2pi.intro} 
 
% section introduction (end)

% \input{qm2pi.knotations} 

% section notation (end)

\input{qm2pi.process.calculi} 

% section concurrent_process_calculi_and_spatial_logics_ (end)
    
%\input{qm2pi.knots2pi} 

%\input{qm2pi.trefoil} 

%\input{qm2pi.mainthm} 

% subsection basic_interpretation (end)

%\input{qm2pi.rho.presentation} 
\subsection{The syntax and semantics of the notation system}\label{sub:the_syntax_and_semantics_of_the_notation_system} % (fold)

We now summarize a technical presentation of the calculus that
embodies our theory of dynamics. The typical presentation of such a
calculus follows the style of giving generators and relations on
them. The grammar, below, describing term constructors, freely
generates the set of processes, $\Proc$. This set is then quotiented
by a relation known as structural congruence and it is over this set
that the notion of dynamics is expressed. This presentation is
essentially that of \cite{MeredithR05} with the addition of
polyadicity and summation. For readability we have relegated some of
the technical subtleties to an appendix.

\subsubsection{Process grammar}\label{subsub:process_grammar}

\begin{mathpar}
  \inferrule* [lab=synchronization] {} {{M} \bc \pzero \;|\; x?F \;|\; x!C }
  \and
  \inferrule* [lab=abstraction] {} {{F} \bc (x)P}
  \and
  \inferrule* [lab=concretion] {} {{C} \bc \langle Q \rangle}
  \and
  \inferrule* [lab=process] {} {{P,Q} \bc M \;| \;P|Q \;|\; @{x}}
  \and
  \inferrule* [lab=name] {} {{x} \bc \quotep{P}}
\end{mathpar} 

Note that $\vec{x}$ (resp. $\vec{P}$) denotes a vector of names
(resp. processes) of length $|\vec{x}|$ (resp. $|\vec{P}|$). We adopt
the following useful abbreviations.

\begin{mathpar}
   x?(\vec{y}).P := x.(\vec{y})P \and  x\clift{\vec{P}} := x.\clift{\vec{P}}
   \and x!(y) := \lift{x}{\dropn{y}}
   \and \Pi_{i=0}^{n-1}P_i := P_0 | \ldots | P_{n-1}
\end{mathpar}

\subsubsection{Structural congruence}

\paragraph{Free and bound names and alpha-equivalence.} At the
core of structural equivalence is alpha-equivalence which identifies
process that are the same up to a change of variable. Formally, we
recognize the distinction between free and bound names. The free names
of a process, $\freenames{P}$, may be calculated recursively as
follows:

\begin{mathpar}
\freenames{\pzero} := \emptyset
  \and \\
  \freenames{x?(y).P} := \{ x \} \cup (\freenames{P} \setminus \{ y \})
  \and 
  \freenames{x!\langle P \rangle} := \{ x \} \cup \{ P \} 
  \and \\
  \freenames{P|Q} := \freenames{P} \cup \freenames{Q}
  \and \\
  \freenames{@{x}} := \{ x \}
\end{mathpar}

$\pi$
$\quotep{\pi}$

$\freenames{-} : \pi \to \mathcal{P}(\quotep{\pi})$

\begin{eqnarray*}
  \freenames{\pzero} & := & \emptyset \\
  \freenames{x?(y).P} & := & \{ x \} \cup (\freenames{P} \setminus \{ y \}) \\
  \freenames{x!\langle P \rangle} & := & \{ x \} \cup \{ P \} \\
  \freenames{P|Q} & := & \freenames{P} \cup \freenames{Q} \\
  \freenames{\dropn{x}} & := & \{ x \}
\end{eqnarray*}

The bound names of a process, $\boundnames{P}$, are those names occurring in $P$
that are not free. For example, in $x?(y).0$, the name $x$ is free, while $y$ is bound.

\begin{mathpar}
  \inferrule* [lab=monoidal-laws] {} { P|Q \equiv Q|P \and P|0 \equiv P \and P|(Q|R) \equiv (P|Q)|R }
\end{mathpar}

\begin{mathpar}
  \inferrule* [lab=alpha-equivalence] {} { (x)P \equiv (y)P\{y/x\} \and y \not\in \freenames{P} }
\end{mathpar}

\begin{definition}
Then two processes, $P,Q$, are alpha-equivalent if $P = Q\{\vec{y}/\vec{x}\}$ for
some $\vec{x} \in \boundnames{Q},\vec{y} \in \boundnames{P}$, where $Q\{\vec{y}/\vec{x}\}$
denotes the capture-avoiding substitution of $\vec{y}$ for $\vec{x}$ in $Q$.
\end{definition}

\begin{definition}
  The {\em structural congruence} \cite{SangiorgiWalker} , $\equiv$,
  between processes is the least congruence containing
  alpha-equivalence, satisfying the abelian monoid laws
  (associativity, commutativity and $\pzero$ as identity) for parallel
  composition $|$ and for summation $+$.
\end{definition}

\subsection{Name equivalence}

We take name equivalence, written $\nameeq$, to be the smallest
equivalence relation generated by the following rules.

\begin{mathpar}
\inferrule*[lab=Quote-drop]
{ }
{ \quotep{@{x}} \nameeq x }

\inferrule*[lab=Struct-equiv]
{ P \scong Q }
{ \quotep{P} \nameeq \quotep{Q} }
\end{mathpar}

The astute reader will have noticed that the mutual recursion of names
and processes imposes a mutual recursion on alpha-equivalence and
structural equivalence via name-equivalence. Fortunately, all of this
works out pleasantly and we may calculate in the natural way, free of
concern. The reader interested in the details is referred to the
appendix \ref{appendix:rho_details}.

\subsection{Substitution}

We use $\Proc$ for the set of processes, $\QProc$ for the set of
names, and $\id{\{}\vec{y} / \vec{x} \id{\}}$ to denote partial maps,
$s : \QProc \rightarrow \QProc$. A map, $s$ lifts, uniquely, to a map
on process terms, $\widehat{s} : \Proc \rightarrow \Proc$ by the
following equations.

\begin{mathpar}
  (0) \psubstp{Q}{P} := 0 \\
  (R \juxtap S) \psubstp{Q}{P}
  :=    
  (R)\psubstp{Q}{P} \juxtap (S) \psubstp{Q}{P} \\
  (x?(y).R) \psubstp{Q}{P}    
  :=    
  (x)\substp{Q}{P} (z)\concat( (R \psubstn{z}{y}) \psubstp{Q}{P} ) \\
  (\lift{x}{R}) \psubstp{Q}{P}  
  :=
  \lift{(x)\substp{Q}{P}}{ R \psubstp{Q}{P} } \\
%   (\dropn{x})  \psubstp{Q}{P}       
%   := 
%   \left\{ 
%     \begin{array}{ccc} 
%       \dropn{\quotep{Q}} & & x \nameeq \quotep{P} \\
%       \dropn{x} & & otherwise \\
%     \end{array}
%   \right. 
  (\dropn{x})  \psubstp{Q}{P}       
  := 
  \left\{ 
    \begin{array}{ccc} 
      Q & & x \nameeq \quotep{P} \\
      \dropn{x} & & otherwise \\
    \end{array}
  \right.
\end{mathpar}
 

where

\begin{eqnarray}
  (x)\id{\{} \lpquote Q \rpquote / \lpquote P \rpquote \id{\}}            = 
  \left\{ 
    \begin{array}{ccc}
      \lpquote Q \rpquote & & x \nameeq \lpquote P \rpquote \\
      x & & otherwise \\
    \end{array}
  \right. \nonumber
\end{eqnarray}

and $z$ is chosen distinct from $\quotep{P}$, $\quotep{Q}$, the free
names in $Q$, and all the names in $R$. Our $\alpha$-equivalence will
be built in the standard way from this substitution.

\begin{remark}\label{rem:no_self_referential_names}
  One consequence of these definitions is that $\forall P. \quotep{P}
  \not\in \freenames{P}$.
\end{remark}

\subsection{ Dynamic quote: an example }

Anticipating something of what's to come, consider applying the
substitution, $\widehat{\id{\{}u / z \id{\}}}$, to the following pair
of processes, $\lift{w}{y!(z)}$ and $w[ \lpquote y!(z) \rpquote ]$.

\begin{eqnarray}
	\lift{w}{y!(z)}\widehat{\id{\{}u / z \id{\}}}
		& = &
		\lift{w}{y!(u)} \nonumber\\
	w[ \lpquote y!(z) \rpquote ] \widehat{ \id{\{}u / z \id{\}} }
		& = &
		w[ \lpquote y!(z) \rpquote ] \nonumber
\end{eqnarray}

Because the body of the process between quotes is impervious to
substitution, we get radically different answers. In fact, by
examining the first process in an input context,
e.g. $x?(z).\lift{w}{y!(z)}$, we see that the process under the lift
operator may be shaped by prefixed inputs binding a name inside it. In
this sense, the lift operator will be seen as a way to dynamically
construct processes before reifying them as names.

Finally equipped with these standard features we can present the
dynamics of the calculus.

\subsubsection{Operational semantics} 

Finally, we introduce the computational dynamics. What marks these
algebras as distinct from other more traditionally studied algebraic
structures, e.g. vector spaces or polynomial rings, is the manner in
which dynamics is captured. In traditional structures, dynamics is typically
expressed through morphisms between such structures, as in linear maps
between vector spaces or morphisms between rings. In algebras
associated with the semantics of computation, the dynamics is
expressed as part of the algebraic structure itself, through a
reduction reduction relation typically denoted by $\red$. Below, we
give a recursive presentation of this relation for the calculus used
in the encoding.

$\red \subseteq \pi \times \pi$
$\red : \pi \to \mathcal{P}(\pi)$

\begin{mathpar}
  \inferrule* [lab=Comm] { \textsf{match}( x_{src}, x_{trgt} ) } { x_{trgt}?(y)P \; | \; x_{src}!\langle {Q} \rangle \red P\{\quotep{Q}/y}\} }
  \and \\
  \inferrule* [lab=Par] {{P} \red {P}'} {{{P} | {Q}} \red {{P}' | {Q}}}
  \and
  \inferrule* [lab=Equiv]{{{P} \scong {P}'} \andalso {{P}' \red {Q}'} \andalso {{Q}' \scong {Q}}}{{P} \red {Q}}
\end{mathpar}

\begin{eqnarray*}
  match_{\equiv} (\quotep{P},\quotep{Q}) & := & P \equiv Q \\
  match_{\dagger}(\quotep{P},\quotep{Q}) & := & \forall R. P|Q \red^{*} R => R \red^{*} 0 \\
  match_{K}(\quotep{P},\quotep{Q}) & := & K \mbox{ for some context } K
\end{eqnarray*}

$u?(x)P | u!\langle Q \rangle \red P\{\quotep{Q}/x\}$

%We write $\wred$ for $\red^*$, and $P\red$ if $\exists Q $ such that $ P \red Q$.
We write $P\red$ if $\exists Q $ such that $ P \red Q$ and $P\not\red$, otherwise.

\section{Replication}

As mentioned before, it is known that replication (and hence
recursion) can be implemented in a higher-order process algebra
\cite{SangiorgiWalker}. As our first example of calculation with the
machinery thus far presented we give the construction explicitly in
the {\rhoc}.

\begin{eqnarray}
	D_{x} & := & \prefix{x}{y}{(\binpar{\outputp{x}{y}}{@{y}})} \nonumber\\
	\bangp_{x}{P} & := & \binpar{{x}!\langle{\binpar{D_{x}}{P}}\rangle}{D_{x}} \nonumber
\end{eqnarray}

\begin{eqnarray}
	\bangp_{x}{P} & & \nonumber\\
	=
	& {x}!\langle{(\prefix{x}{y}{(\outputp{x}{y} | @{y})) | P}}\rangle 
	      | \prefix{x}{y}{(\outputp{x}{y} | @{y})} & \nonumber\\
	\red
	& (\outputp{x}{y} | @{y})\substn{\quotep{(\prefix{x}{y}{(@{y} | \outputp{x}{y})) | P}}}{y} & \nonumber\\
	=
	& \outputp{x}{\quotep{(\prefix{x}{y}{(\outputp{x}{y} | @{y})) | P}}}
	  | {(\prefix{x}{y}{(\outputp{x}{y} | @{y})) | P}} & \nonumber\\
	\red
	& \ldots & \nonumber\\
	\red^*
	& P | P | \ldots & \nonumber
\end{eqnarray}

Of course, this encoding, as an implementation, runs away, unfolding
$\bangp{P}$ eagerly. A lazier and more implementable replication
operator, restricted to input-guarded processes, may be obtained as follows.

\begin{eqnarray}
\bangp{\prefix{u}{v}{P}} 
	:= 
	\binpar{\lift{x}{\prefix{u}{v}{(\binpar{D(x)}{P})}}}{D(x)} \nonumber
\end{eqnarray}

\begin{remark}
  Note that the lazier definition still does not deal with summation
  or mixed summation (i.e. sums over input and output). The reader is
  invited to construct definitions of replication that deal with these
  features. 

  Further, the definitions are parameterized in a name, $x$. Can you,
  gentle reader, make a definition that eliminates this parameter and
  guarantees no accidental interaction between the replication
  machinery and the process being replicated -- i.e. no accidental
  sharing of names used by the process to get its work done and the
  name(s) used by the replication to effect copying. This latter
  revision of the definition of replication is crucial to obtaining
  the expected identity $!!P \sim !P$.
\end{remark}

\begin{remark}\label{rem:paradoxical_combinator}
  The reader familiar with the lambda calculus will have noticed the
  similarity between $D$ and the paradoxical combinator.

  [Ed. note: the existence of this seems to suggest we have to be more
  restrictive on the set of processes and names we admit if we are to
  support no-cloning.]
\end{remark}

\subsubsection{Bisimulation}

The computational dynamics gives rise to another kind of equivalence,
the equivalence of computational behavior. As previously mentioned
this is typically captured \emph{via} some form of bisimulation.

% The notion we use in this paper is weak barbed bisimulation
% \cite{milner91polyadicpi}.

The notion we use in this paper is derived from weak barbed
bisimulation \cite{milner91polyadicpi}. 

\begin{definition}
An \emph{observation relation}, $\downarrow_{\mathcal N}$, over a set
of names, $\mathcal N$, is the smallest relation satisfying the rules
below.

\infrule[Out-barb]{y \in {\mathcal N}, \; x \nameeq y}
		  {\outputp{x}{v} \downarrow_{\mathcal N} x}
\infrule[Par-barb]{\mbox{$P\downarrow_{\mathcal N} x$ or $Q\downarrow_{\mathcal N} x$}}
		  {\binpar{P}{Q} \downarrow_{\mathcal N} x}

We write $P \Downarrow_{\mathcal N} x$ if there is $Q$ such that 
$P \wred Q$ and $Q \downarrow_{\mathcal N} x$.
\end{definition}

\begin{definition}
%\label{def.bbisim}
An  ${\mathcal N}$-\emph{barbed bisimulation} over a set of names, ${\mathcal N}$, is a symmetric binary relation 
${\mathcal S}_{\mathcal N}$ between agents such that $P\rel{S}_{\mathcal N}Q$ implies:
\begin{enumerate}
\item If $P \red P'$ then $Q \wred Q'$ and $P'\rel{S}_{\mathcal N} Q'$.
\item If $P\downarrow_{\mathcal N} x$, then $Q\Downarrow_{\mathcal N} x$.
\end{enumerate}
$P$ is ${\mathcal N}$-barbed bisimilar to $Q$, written
$P \wbbisim_{\mathcal N} Q$, if $P \rel{S}_{\mathcal N} Q$ for some ${\mathcal N}$-barbed bisimulation ${\mathcal S}_{\mathcal N}$.
\end{definition}

$\mathcal{R} \subseteq \pi \times \pi$

$P \mathcal{R} Q => \forall P'. P \red P' \Rightarrow \exists Q'. Q \red Q', P' \mathcal{R} Q'$

$P \vdash x \Rightarrow Q \vdash x$

\begin{mathpar}
  \inferrule*[lab=Out-barb]{x \nameeq y}{{y}!\langle{Q}\rangle \vdash x}
  \and
  \inferrule*[lab=Par-barb]{\mbox{$P\vdash x$ or $Q\vdash x$}}{\binpar{P}{Q} \vdash x}
\end{mathpar}

\subsubsection{Contexts}

One of the principle advantages of computational calculi like the
$\pi$-calculus is a well-defined notion of context,
contextual-equivalence and a correlation between
contextual-equivalence and notions of bisimulation. The notion of
context allows the decomposition of a process into (sub-)process and
its syntactic environment, its context. Thus, a context may be
thought of as a process with a ``hole'' (written $\Box$) in it. The
application of a context $M$ to a process $P$, written $M[P]$, is
tantamount to filling the hole in $M$ with $P$. In this paper we do
not need the full weight of this theory, but do make use of the notion
of context in the proof the main theorem. 

\begin{mathpar}
  \inferrule* [lab=summation] {} {{M_{M},M_{N}} \bc \Box \;|\; x.M_{A} \;|\; M_{M}+M_{N}}
  \and
  \inferrule* [lab=agent] {} {{M_{A}} \bc (\vec{x})M_{P} \;| \; \clift{P_0,\ldots,M_{P},\ldots,P_N}}
  \and \\
  \inferrule* [lab=process] {} {{M_{P}} \bc M_{N} \;| \;P|M_{P} }
\end{mathpar} 

\begin{mathpar}
  \inferrule* [lab=sychronization] {} {M_{N} \bc \Box \;|\; x?M_{F} \;|\; x!M_{C}}
  \and
  \inferrule* [lab=abstraction] {} {{M_{F}} \bc (x)M_{P} }
  \and
  \inferrule* [lab=concretion] {} {{M_{C}} \bc \langle M_{P} \rangle }
  \and \\
  \inferrule* [lab=process] {} {{M_{P}} \bc M_{N} \;| \;P|M_{P} }
\end{mathpar}

\begin{definition}[contextual application] Given a context $M$, and
  process $P$, we define the \emph{contextual application}, $M[P] :=
  M\{P/\Box\}$. That is, the contextual application of M to P is the
  substitution of $P$ for $\Box$ in $M$.
\end{definition}

$\meaningof{-} : L \to \mathcal{P}(\pi)$

\begin{mathpar}
  \inferrule* [lab=collection] {} {\meaningof{true} = \pi, \and \meaningof{~E} = \pi \setminus \meaningof{E}, \and \meaningof{E_{1} \& E_{2}} = \meaningof{E_{1}} \cap \meaningof{E_{2}}}
\end{mathpar}

\begin{mathpar}
  \inferrule* [lab=structure] {} {\meaningof{0} = \{ P \in \pi | P \equiv 0 \}, \and \\ \meaningof{E_1 | E_2} = \{ P \in \pi | P \equiv P_{1} | P_{2}, P_{1} \in \meaningof{E_{1}}, P_{2} \in \meaningof{E_2}\} }
\end{mathpar}

\begin{mathpar}
 \inferrule* [lab=behavior] {} {\meaningof{\langle a?b \rangle E} = \{ P \in \pi | P \equiv Q | u?(y)P', \\ \and \\\\ \and \\ \;\;\; u \in \meaningof{a}, \forall z.P'\{z/y\} \in \meaningof{E\{z/b\}}\}, \and \\ \meaningof{a!E} = \{ P \in \pi | P \equiv Q | x!\langle P' \rangle, x \in \meaningof{a} P' \in \meaningof{E}\} }
\end{mathpar}

\begin{mathpar}
 \inferrule* [lab=nominal] {} {\meaningof{\quotep{E}} = \{ \quotep{P} \in \quotep{\pi} | P \in \meaningof{E} \}, \and \meaningof{\quotep{P}} = \{ \quotep{Q} \in \quotep{\pi} | P \equiv Q \} \and \\ \meaningof{@\quotep{E}} = \{ P \in \pi | P \equiv @x, x \in \meaningof{E} \}}
\end{mathpar}

\begin{eqnarray*}
  \\
  \meaningof{-} : TS \to ST
\end{eqnarray*}

\begin{eqnarray*}
  \\
  L : TS \to ST
\end{eqnarray*}

\begin{eqnarray*}
  \\
  P \models E \iff P \in \meaningof{E}
\end{eqnarray*}

\begin{eqnarray*}
  P \approx_{L} Q \iff \forall E \in L. P \models E \iff Q \models E
\end{eqnarray*}

\begin{eqnarray*}
  P \approx_{K} Q
\end{eqnarray*}

\begin{eqnarray*}
  P \approx Q
\end{eqnarray*}

$\approx_{K} = \approx = \approx_{L}$

\subsubsection{Contextual duality}

Note that contexts extend the quotation operation to a family of
operations from processes to names. Given a context, $M$, we can
define a \emph{nominal context}, $\quotep{M}$ by $\quotep{M}[P] :=
\quotep{M[P]}$. To foreshadow what is to come we observe that these
operations enjoy a duality with processes very much like the duality
between vectors and maps from vectors to scalars.

Further, because the calculus is essentially higher-order, we have a
correspondence between contexts and processes. More specifically,
given a name $x$ and a context $M$ we can construct $M^{*}_{x}$ such
that 

\begin{mathpar}
  M^{*}_{x} | \lift{x}{P} \red M[P]
\end{mathpar}

namely,

\begin{mathpar}
  M^{*}_{x} := x?(u).M[\dropn{u}]
\end{mathpar}

The dependence of $M^{*}_{x}$ on a name makes it an abstraction, 

\begin{mathpar}
  M^{*} := (x)x?(u).M[\dropn{u}]
\end{mathpar}

\subsection{Additional notation}

It will sometimes be convenient to denote the process a name
quotes. We already have the notation $x = \quotep{P}$, but it will be
convenient to introduce an alternate notation, $\procn{x}$, when we
want to emphasize the connection to the use of the name. Note that, by
virtue of name equivalence, $\quotep{\procn{x}} \nameeq x$; so, the
notation is consistent with previous definitions.

Further, because names have structure it is possible to effect
substitutions on the basis of that structure. This means we need to
upgrade our notation for substitutions, which we accomplish by
adapting comprehension notation. Thus,

\begin{mathpar}
  P\{ y / x : x \in S \}
\end{mathpar}

is interpreted to mean the process derived from P by replacing (in a
capture-avoiding manner) each occurrence of $x$ in $S$ by $y$. For example,

\begin{mathpar}
  P\{ \quotep{\procn{x}|\procn{x}} / x : x \in \freenames{P} \}
\end{mathpar}

will replace each (occurrence) of a free name $x$ in $P$ by
$\quotep{\procn{x}|\procn{x}}$.

Also, we will avail ourselves of the notation $x^{L}$ and $x^{R}$ to
denote injections of a name into disjoint copies of the name
space. There are numerous ways to accomplish this. One example can be
found in \cite{MeredithR05}. This notation overloads to vectors of
names: $\vec{x}^{\pi} := (x_{i}^{\pi} \; : \; 0 \leq i < |\vec{x}| )$ where $\pi \in \{L,R\}$.

We also use $P^{\Box} := P|\Box$.

In \cite{MeredithR05} an interpretation of the new operator is
given. It turns out that there are several possible interpretations
all enjoying the requisite algebraic properties of the operator (see
\cite{milner91polyadicpi}). We will therefore make liberal use of
$(\nu\; \vec{x})P$.

% subsection the_syntax_and_semantics_of_the_notation_system (end)   

\input{qm2pi.qmops} 

\input{qm2pi.sterngerlach} 

\input{qm2pi.metric} 

% section concurrent_process_calculi (end)

%\input{qm2pi.proofsketch}

% section proof sketch (end)

%\input{qm2pi.slviaknots} 

% section spatial logic via knots (end)

\input{qm2pi.conclusion}

% section conclusion (end)

%\input{qm2pi.dtcodes} 

% section wiring algorithm (end)

\input{qm2pi.ack} 

% section acknowledgments (end)

\newpage


\bibliographystyle{plain}   
\bibliography{../../biblios/main.bib}

\input{qm2pi.rhodetails}

\end{document}



% section proof sketch (end)

%\section{Unlikely characters: spatial logic for
  knots}\label{sub:characteristic_formulae} % (fold)

Associated to the mobile process calculi are a family of logics known
as the Hennessy-Milner logics. These logics typically enjoy a
semantics interpreting formulae as sets of processes that when
factored through the encoding outlined above allows an identification
of classes of knots with logical formulae. In the context of this
encoding the sub-family known as the spatial logics \cite{CairesC03}
\cite{CairesC04} \cite{Caires04} are of particular interest providing
several important features for expressing and reasoning about
properties (i.e. classes) of knots. We hint here at how this may be done.

%\begin{description}
%\item [structural connectives] 
\subsubsection{Structural connectives} The spatial logics enjoy
structural connectives corresponding, at the logical level, to the
parallel composition ($P | Q$) and new name ($(\nu \; x)P$)
connectives for processes. As illustrated in the examples below, these
connectives are extremely expressive given the shape of our encoding.
%\item [decideable satisfaction]

\subsubsection{Decideable satisfaction}
In \cite{Caires04} the satisfaction relation is shown to be decideable
for a rich class of processes. It further turns out that the image of
the our encoding is a proper subset of that class. This result
provides the basis for an algorithm by which to search for knots
enjoying a given property.
%\item [characteristic formulae]

\subsubsection{Characteristic formulae}
In the same paper \cite{Caires04} , Caires presents a means of calculating
characteristic formulae, selecting equivalence classes of processes
up to a pre--specified depth limit on the support set of names. Composed with our
encoding, this characteristic formula can be used to select
characteristic formulae for knots.
%\end{description}

\subsubsection{Spatial logic formulae}

The grammar below (segmented for comprehension) summarizes the syntax
of spatial logic formulae. We employ illustrative examples in the
sequel to provide an intuitive understanding of their meaning
referring the reader to \cite{Caires04} for a more detailed explication
of the semantics.

\begin{mathpar}
  \inferrule* [lab=boolean] {} {{A,B} \bc T \;|\; \neg A \;|\; A \wedge B \;|\; \eta = \eta'}
  \and
  \inferrule* [lab=spatial] {} {|\; \pzero \;|\; A | B \;|\; x \text{\textregistered} A \;|\; \forall x . A \;|\;  H x . A}
  \and
  \inferrule* [lab=behavioral] {} {|\; \alpha . A}
  \and 
  \inferrule* [lab=recursion] {} {|\; X(\vec{u}) \;|\; \mu X(\vec{u}) . A}
  \and
  \inferrule* [lab=action] {} {\alpha \bc \langle x?(\vec{y}) \rangle \;|\; \langle x!(\vec{y}) \rangle \;|\; \langle \tau \rangle}
  \and 
  \inferrule* [lab=name] {} {\eta \bc x \;|\; \tau}
\end{mathpar} 

% subsection characteristic_formulae (end)   	 

\subsection{Example formulae}\label{sub:example_formulae_} % (fold)

\subsubsection{Crossing as formula.}
% 
% \begin{align*}
%   \frac{d}{dx} \sin x &= \cos x 
%   & \frac{d}{dx} e^x &= e^x \\
%   \frac{d}{dx} \cos x &= - \sin x 
%   & \frac{d}{dx} \log x &= \frac{1}{x} \\
% \end{align*} 

\begin{align*}
 \mu C(x_{0},x_{1},y_{0},y_{1},u).&(\langle x_{0}?(z) \rangle(\langle u! \rangle\langle y_{1}!z \rangle C(x_{0},x_{1},y_{0},y_{1},u)) & \\
  & \wedge \langle y_{1}?(z) \rangle (\langle u! \rangle \langle x_{0}!z \rangle C(x_{0},x_{1},y_{0},y_{1},u)) & \\
  & \wedge \langle x_{1}?(z) \rangle (\langle u? \rangle \langle y_{0}!z \rangle C(x_{0},x_{1},y_{0},y_{1},u)) & \\
  & \wedge \langle y_{0}?(z) \rangle (\langle u? \rangle \langle x_{1}!z \rangle C(x_{0},x_{1},y_{0},y_{1},u))) &
\end{align*}

The lexicographical similarity between the shape of this formulae and
the shape of definition of the process representing a crossing reveals
the intuitive meaning of this formulae. It describes the capabilities
of a process that has the right to represent a crossing. For example
it picks out processes that may perform an input on the port $x_0$ in
its initial menu of capabilities. What differentiates the formula
from the process, however, is that the crossing process is the
smallest candidate to satisfy the formula. Infinitely many other
processes -- with internal behavior hidden behind this interface, so
to speak -- also satisfy this formula. Even this simple formula,
then, can be seen to open a new view onto knots, providing a
computational interpretation of \emph{virtual} knots.

Note that this formula is derived by hand. A similar formula can be
derived by employing Caires' calculation of characteristic formula
\cite{Caires04} to the process representing a crossing. In light of
this discussion, we let
$\meaningof{C}_{\phi}(x0,x1,y0,y1,u)$ denote a formula specifying the
dynamics we wish to capture of a crossing. To guarantee we preserve
the shape of the interface and minimal semantics we demand that
$\meaningof{C}_{\phi}(x0,x1,y0,y1,u) \Rightarrow
\textbf{C}(x0,x1,y0,y1,u)$ where $\textbf{C}(x0,x1,y0,y1,u)$ denotes
the formula above.
                            
\subsubsection{Crossing number constraints.}
The moral content of the context lemma (Lemma \ref{context}) is that the notion of
``locality'' in the Reidemeister moves is effectively captured by the
parallel composition operator of the process calculus. This intuition
extends through the logic. Given a formula,
$\meaningof{C}_{\phi}(x0,x1,y0,y1,u)$, we can use the structural
connectives to specify constraints on crossing numbers, such as at
least $n$ crossings, or exactly $n$ crossings.
\begin{mathpar}
  \inferrule* [lab=at-least-n] {} { K^{\geq n}_{\phi}(\vec{xs},\vec{ys}) := \Pi_{i=0}^{n-1} Hu . \meaningof{C}_{\phi}(xs_i,ys_i,u) | T }
  \and 
  \inferrule* [lab=exactly-n] {} { K^{= n}_{\phi}(\vec{xs},\vec{ys}) := \Pi_{i=0}^{n-1} Hu . \meaningof{C}_{\phi}(xs_i,ys_i,u) | \neg (\forall x_0,y_0,x_1,y_1,u . \meaningof{C}_{\phi}(x_0,y_0,x_1,y_1,u) | T) }
\end{mathpar}

To round out this section, recall that the encoding of an $n$-crossing
knot decomposes into a parallel composition of $n$ \emph{copies} of a
crossing process together with a wiring harness. To specify different
knot classes with the same crossing number amounts to specifying
logical constraints on the wiring harness. In the interest of space,
we defer examples to a forthcoming paper. Suffice it to say that both
the conditions ``alternating knot'' and ``contains the tangle
corresponding to 5/3'' are expressible. For example, it is possible to
calculate the characteristic formula of a process corresponding to the
tangle 5/3 and conjoin it into the classifying formula via the
composition connective of the logic.

Finally, we wish to observe that it is entirely within reason to
contemplate a more domain-specific version of spatial logic tailored
to the shape of processes in the image of the encoding. Such a
domain-specific logic would have a better claim to the title formal
language of knot properties.

% subsection example_formulae_ (end)

% section knots_as_processes (end) 

% section spatial logic via knots (end)

\section{Conclusions and future work}

\paragraph{Testing physical space}
You, gentle reader, may wonder why of all the theorems to be proved
given this set up we pick the one above. In some sense it's hardly
central to quantum mechanics. We see it as central in the sense that
it firmly establishes a notion of physical space arising from a notion
of the equivalence of behavior. Relating bisimulation to a metric is a
big step forward, but one is faced with interpreting the relationship
of that metric space to something more physical. Quantum mechanical
notions of ``physical'' space are still far from intuitive, but by
relating this idea of distance as testing to calculations that predict
physical circumstances we are making a not insignificant step forward
toward an understanding of the physical space we inhabit as
essentially dynamic.

\paragraph{Effectivity and simulation}
One of the observations we have yet to make is that the entire program
spelled out here is effective. We have built various interpreters for
the reflective calculus at work in this interpretation. In principle,
then, we can simulate quantum mechanics on a computer. The place where
the simulation may lose fidelity is the infinitely branching summation
for the annihilator.

In this connection i also want to point out that the evaluation style
calculation of the inner product puts the non-determinism of the
summation right at the heart of measurement. This suggests that
Milner's original reduction-based formulation of the dynamics of his
calculi in terms of sums was not just notationally suggestive of a
notion of measure-and-continue but captured some significant part of
the physics.

\paragraph{Quantum continuations}
In light of this last observation i want to point out that the
predominant account of quantum mechanics is missing a key aspect of a
truly compositional story of the physical situation. In a real lab,
when a measurement is made the observation can be made to feed into
another device that then makes another measurement conditioned on the
results of the first. This means that after the superposition was
collapsed the entire experimental set up remained in
superposition. While QM offers a means of writing this down it doesn't
quite line up well with the well-trodden formulation of computation
and continuation that we see so succinctly expressed in Milner's
calculi. This suggests that there might be advantages to this account
of dynamics waiting to be explored.

\paragraph{Quantum logic}
In this connection, we also note that by virtue of having the
Hennessy-Milner construction, we can pull the construction through the
interpretation of QM. This gives us a natural candidate for a quantum
logic that enjoys an extremely tight connection with it's domain of
interpretation, making the construction much less ad hoc (rather it is
the image of functor!).

\paragraph{Quantum probabiity}
i have questions about the basis of the interpretation of inner
product as probability amplitude. In particular, using which
axiomatization of probability theory does the notion of probability
amplitude earn the right to be so dubbed? In other words, where is the
proof that the operation for calculating a probability amplitude (and
then squaring) satisfies the axioms of what it means to calculate a
probability? Even if such a proof exists (i have yet to find it in the
literature), i wonder if it might not be possible to turn things on
their heads. Can we view the calculation of the probability amplitude
as an axiomatization of probability? If so, then the definition we
give for calculating probability amplitude may provide the basis for
an \emph{effective} theory of probability.

\paragraph{Quantum vs ``biological'' information}
Finally, i want to conclude with a more philosophical observation. At
a recent workshop in which QM was a predominant topic i noticed
something about quantum information. The speaker was giving a riveting
discussion of axiomatic QM and showing how properties of ``no
cloning'' and ``no deleting'' emerged as consequences of the
axiomatization. Theorems of this form are necessary to give us a sense
of confidence that our axioms characterize the physical theory. What
struck me, though, was that if quantum information is neither erasable
nor replicable it is markedly different from \emph{life}. Two of the
things we know about life is that

\begin{itemize}
  \item it ends;
  \item to gain some measure of persistence, to transcend it's
    finitude it is imminently copyable.
\end{itemize}

Both of these qualities are summarized succinctly in the aphorism: all
flesh is grass. For me these two kinds of ``information'' -- call them
quantum and biological -- are end points on a spectrum of strategies
for persistence. At one end, we have those curious entities that enjoy
uniqueness and permanence; at the other, we have those who in the face
of a certain end and an uncertain present make a go of passing
something on. To me one of the more remarkable aspects of the latter
strategy is that in the presence of noise (and certain features of
copying) we get a kind of dynamism, a chance for improvement against a
given persistent condition.

% subsection other_calculi_other_bisimulations_and_geometry_as_behavior (end)




% section conclusion (end)

%\documentclass[12pt]{llncs}
%\documentclass{jktr}

\usepackage[pdftex]{hyperref}                   
\usepackage {listings}
\usepackage {mathpartir}
\usepackage{bcprules}
%\usepackage{listings}
                       
\usepackage{graphicx} 
%\usepackage[margins=2.5cm,nohead,nofoot]{geometry}
%\usepackage{geometry}
\usepackage{amsfonts}
\usepackage{amstext}
\usepackage{latexsym}
\usepackage{amssymb}
\usepackage{color}


%\include{myPreamble}
\include{qm2pi.local} 

%\ifpdf
%\usepackage[pdftex]{graphicx}
%\else
%\usepackage{graphicx}
%\fi

 % \ifpdf
%  \usepackage{pdfsync}
%  \if


%\title{Brief Article}
%\author{David F. Snyder}
%\author{L.G. Meredith}

%\address{Dept. of Math., Texas State University--San Marcos, San Marcos, TX 78666}
       
\pagestyle{empty}


\begin{document}

\lstset{language=[Objective]Caml,frame=shadowbox}

\input{qm2pi.front}

% section front matter (end)

\input{qm2pi.intro} 
 
% section introduction (end)

% \input{qm2pi.knotations} 

% section notation (end)

\input{qm2pi.process.calculi} 

% section concurrent_process_calculi_and_spatial_logics_ (end)
    
%\input{qm2pi.knots2pi} 

%\input{qm2pi.trefoil} 

%\input{qm2pi.mainthm} 

% subsection basic_interpretation (end)

%\input{qm2pi.rho.presentation} 
\subsection{The syntax and semantics of the notation system}\label{sub:the_syntax_and_semantics_of_the_notation_system} % (fold)

We now summarize a technical presentation of the calculus that
embodies our theory of dynamics. The typical presentation of such a
calculus follows the style of giving generators and relations on
them. The grammar, below, describing term constructors, freely
generates the set of processes, $\Proc$. This set is then quotiented
by a relation known as structural congruence and it is over this set
that the notion of dynamics is expressed. This presentation is
essentially that of \cite{MeredithR05} with the addition of
polyadicity and summation. For readability we have relegated some of
the technical subtleties to an appendix.

\subsubsection{Process grammar}\label{subsub:process_grammar}

\begin{mathpar}
  \inferrule* [lab=synchronization] {} {{M} \bc \pzero \;|\; x?F \;|\; x!C }
  \and
  \inferrule* [lab=abstraction] {} {{F} \bc (x)P}
  \and
  \inferrule* [lab=concretion] {} {{C} \bc \langle Q \rangle}
  \and
  \inferrule* [lab=process] {} {{P,Q} \bc M \;| \;P|Q \;|\; @{x}}
  \and
  \inferrule* [lab=name] {} {{x} \bc \quotep{P}}
\end{mathpar} 

Note that $\vec{x}$ (resp. $\vec{P}$) denotes a vector of names
(resp. processes) of length $|\vec{x}|$ (resp. $|\vec{P}|$). We adopt
the following useful abbreviations.

\begin{mathpar}
   x?(\vec{y}).P := x.(\vec{y})P \and  x\clift{\vec{P}} := x.\clift{\vec{P}}
   \and x!(y) := \lift{x}{\dropn{y}}
   \and \Pi_{i=0}^{n-1}P_i := P_0 | \ldots | P_{n-1}
\end{mathpar}

\subsubsection{Structural congruence}

\paragraph{Free and bound names and alpha-equivalence.} At the
core of structural equivalence is alpha-equivalence which identifies
process that are the same up to a change of variable. Formally, we
recognize the distinction between free and bound names. The free names
of a process, $\freenames{P}$, may be calculated recursively as
follows:

\begin{mathpar}
\freenames{\pzero} := \emptyset
  \and \\
  \freenames{x?(y).P} := \{ x \} \cup (\freenames{P} \setminus \{ y \})
  \and 
  \freenames{x!\langle P \rangle} := \{ x \} \cup \{ P \} 
  \and \\
  \freenames{P|Q} := \freenames{P} \cup \freenames{Q}
  \and \\
  \freenames{@{x}} := \{ x \}
\end{mathpar}

$\pi$
$\quotep{\pi}$

$\freenames{-} : \pi \to \mathcal{P}(\quotep{\pi})$

\begin{eqnarray*}
  \freenames{\pzero} & := & \emptyset \\
  \freenames{x?(y).P} & := & \{ x \} \cup (\freenames{P} \setminus \{ y \}) \\
  \freenames{x!\langle P \rangle} & := & \{ x \} \cup \{ P \} \\
  \freenames{P|Q} & := & \freenames{P} \cup \freenames{Q} \\
  \freenames{\dropn{x}} & := & \{ x \}
\end{eqnarray*}

The bound names of a process, $\boundnames{P}$, are those names occurring in $P$
that are not free. For example, in $x?(y).0$, the name $x$ is free, while $y$ is bound.

\begin{mathpar}
  \inferrule* [lab=monoidal-laws] {} { P|Q \equiv Q|P \and P|0 \equiv P \and P|(Q|R) \equiv (P|Q)|R }
\end{mathpar}

\begin{mathpar}
  \inferrule* [lab=alpha-equivalence] {} { (x)P \equiv (y)P\{y/x\} \and y \not\in \freenames{P} }
\end{mathpar}

\begin{definition}
Then two processes, $P,Q$, are alpha-equivalent if $P = Q\{\vec{y}/\vec{x}\}$ for
some $\vec{x} \in \boundnames{Q},\vec{y} \in \boundnames{P}$, where $Q\{\vec{y}/\vec{x}\}$
denotes the capture-avoiding substitution of $\vec{y}$ for $\vec{x}$ in $Q$.
\end{definition}

\begin{definition}
  The {\em structural congruence} \cite{SangiorgiWalker} , $\equiv$,
  between processes is the least congruence containing
  alpha-equivalence, satisfying the abelian monoid laws
  (associativity, commutativity and $\pzero$ as identity) for parallel
  composition $|$ and for summation $+$.
\end{definition}

\subsection{Name equivalence}

We take name equivalence, written $\nameeq$, to be the smallest
equivalence relation generated by the following rules.

\begin{mathpar}
\inferrule*[lab=Quote-drop]
{ }
{ \quotep{@{x}} \nameeq x }

\inferrule*[lab=Struct-equiv]
{ P \scong Q }
{ \quotep{P} \nameeq \quotep{Q} }
\end{mathpar}

The astute reader will have noticed that the mutual recursion of names
and processes imposes a mutual recursion on alpha-equivalence and
structural equivalence via name-equivalence. Fortunately, all of this
works out pleasantly and we may calculate in the natural way, free of
concern. The reader interested in the details is referred to the
appendix \ref{appendix:rho_details}.

\subsection{Substitution}

We use $\Proc$ for the set of processes, $\QProc$ for the set of
names, and $\id{\{}\vec{y} / \vec{x} \id{\}}$ to denote partial maps,
$s : \QProc \rightarrow \QProc$. A map, $s$ lifts, uniquely, to a map
on process terms, $\widehat{s} : \Proc \rightarrow \Proc$ by the
following equations.

\begin{mathpar}
  (0) \psubstp{Q}{P} := 0 \\
  (R \juxtap S) \psubstp{Q}{P}
  :=    
  (R)\psubstp{Q}{P} \juxtap (S) \psubstp{Q}{P} \\
  (x?(y).R) \psubstp{Q}{P}    
  :=    
  (x)\substp{Q}{P} (z)\concat( (R \psubstn{z}{y}) \psubstp{Q}{P} ) \\
  (\lift{x}{R}) \psubstp{Q}{P}  
  :=
  \lift{(x)\substp{Q}{P}}{ R \psubstp{Q}{P} } \\
%   (\dropn{x})  \psubstp{Q}{P}       
%   := 
%   \left\{ 
%     \begin{array}{ccc} 
%       \dropn{\quotep{Q}} & & x \nameeq \quotep{P} \\
%       \dropn{x} & & otherwise \\
%     \end{array}
%   \right. 
  (\dropn{x})  \psubstp{Q}{P}       
  := 
  \left\{ 
    \begin{array}{ccc} 
      Q & & x \nameeq \quotep{P} \\
      \dropn{x} & & otherwise \\
    \end{array}
  \right.
\end{mathpar}
 

where

\begin{eqnarray}
  (x)\id{\{} \lpquote Q \rpquote / \lpquote P \rpquote \id{\}}            = 
  \left\{ 
    \begin{array}{ccc}
      \lpquote Q \rpquote & & x \nameeq \lpquote P \rpquote \\
      x & & otherwise \\
    \end{array}
  \right. \nonumber
\end{eqnarray}

and $z$ is chosen distinct from $\quotep{P}$, $\quotep{Q}$, the free
names in $Q$, and all the names in $R$. Our $\alpha$-equivalence will
be built in the standard way from this substitution.

\begin{remark}\label{rem:no_self_referential_names}
  One consequence of these definitions is that $\forall P. \quotep{P}
  \not\in \freenames{P}$.
\end{remark}

\subsection{ Dynamic quote: an example }

Anticipating something of what's to come, consider applying the
substitution, $\widehat{\id{\{}u / z \id{\}}}$, to the following pair
of processes, $\lift{w}{y!(z)}$ and $w[ \lpquote y!(z) \rpquote ]$.

\begin{eqnarray}
	\lift{w}{y!(z)}\widehat{\id{\{}u / z \id{\}}}
		& = &
		\lift{w}{y!(u)} \nonumber\\
	w[ \lpquote y!(z) \rpquote ] \widehat{ \id{\{}u / z \id{\}} }
		& = &
		w[ \lpquote y!(z) \rpquote ] \nonumber
\end{eqnarray}

Because the body of the process between quotes is impervious to
substitution, we get radically different answers. In fact, by
examining the first process in an input context,
e.g. $x?(z).\lift{w}{y!(z)}$, we see that the process under the lift
operator may be shaped by prefixed inputs binding a name inside it. In
this sense, the lift operator will be seen as a way to dynamically
construct processes before reifying them as names.

Finally equipped with these standard features we can present the
dynamics of the calculus.

\subsubsection{Operational semantics} 

Finally, we introduce the computational dynamics. What marks these
algebras as distinct from other more traditionally studied algebraic
structures, e.g. vector spaces or polynomial rings, is the manner in
which dynamics is captured. In traditional structures, dynamics is typically
expressed through morphisms between such structures, as in linear maps
between vector spaces or morphisms between rings. In algebras
associated with the semantics of computation, the dynamics is
expressed as part of the algebraic structure itself, through a
reduction reduction relation typically denoted by $\red$. Below, we
give a recursive presentation of this relation for the calculus used
in the encoding.

$\red \subseteq \pi \times \pi$
$\red : \pi \to \mathcal{P}(\pi)$

\begin{mathpar}
  \inferrule* [lab=Comm] { \textsf{match}( x_{src}, x_{trgt} ) } { x_{trgt}?(y)P \; | \; x_{src}!\langle {Q} \rangle \red P\{\quotep{Q}/y}\} }
  \and \\
  \inferrule* [lab=Par] {{P} \red {P}'} {{{P} | {Q}} \red {{P}' | {Q}}}
  \and
  \inferrule* [lab=Equiv]{{{P} \scong {P}'} \andalso {{P}' \red {Q}'} \andalso {{Q}' \scong {Q}}}{{P} \red {Q}}
\end{mathpar}

\begin{eqnarray*}
  match_{\equiv} (\quotep{P},\quotep{Q}) & := & P \equiv Q \\
  match_{\dagger}(\quotep{P},\quotep{Q}) & := & \forall R. P|Q \red^{*} R => R \red^{*} 0 \\
  match_{K}(\quotep{P},\quotep{Q}) & := & K \mbox{ for some context } K
\end{eqnarray*}

$u?(x)P | u!\langle Q \rangle \red P\{\quotep{Q}/x\}$

%We write $\wred$ for $\red^*$, and $P\red$ if $\exists Q $ such that $ P \red Q$.
We write $P\red$ if $\exists Q $ such that $ P \red Q$ and $P\not\red$, otherwise.

\section{Replication}

As mentioned before, it is known that replication (and hence
recursion) can be implemented in a higher-order process algebra
\cite{SangiorgiWalker}. As our first example of calculation with the
machinery thus far presented we give the construction explicitly in
the {\rhoc}.

\begin{eqnarray}
	D_{x} & := & \prefix{x}{y}{(\binpar{\outputp{x}{y}}{@{y}})} \nonumber\\
	\bangp_{x}{P} & := & \binpar{{x}!\langle{\binpar{D_{x}}{P}}\rangle}{D_{x}} \nonumber
\end{eqnarray}

\begin{eqnarray}
	\bangp_{x}{P} & & \nonumber\\
	=
	& {x}!\langle{(\prefix{x}{y}{(\outputp{x}{y} | @{y})) | P}}\rangle 
	      | \prefix{x}{y}{(\outputp{x}{y} | @{y})} & \nonumber\\
	\red
	& (\outputp{x}{y} | @{y})\substn{\quotep{(\prefix{x}{y}{(@{y} | \outputp{x}{y})) | P}}}{y} & \nonumber\\
	=
	& \outputp{x}{\quotep{(\prefix{x}{y}{(\outputp{x}{y} | @{y})) | P}}}
	  | {(\prefix{x}{y}{(\outputp{x}{y} | @{y})) | P}} & \nonumber\\
	\red
	& \ldots & \nonumber\\
	\red^*
	& P | P | \ldots & \nonumber
\end{eqnarray}

Of course, this encoding, as an implementation, runs away, unfolding
$\bangp{P}$ eagerly. A lazier and more implementable replication
operator, restricted to input-guarded processes, may be obtained as follows.

\begin{eqnarray}
\bangp{\prefix{u}{v}{P}} 
	:= 
	\binpar{\lift{x}{\prefix{u}{v}{(\binpar{D(x)}{P})}}}{D(x)} \nonumber
\end{eqnarray}

\begin{remark}
  Note that the lazier definition still does not deal with summation
  or mixed summation (i.e. sums over input and output). The reader is
  invited to construct definitions of replication that deal with these
  features. 

  Further, the definitions are parameterized in a name, $x$. Can you,
  gentle reader, make a definition that eliminates this parameter and
  guarantees no accidental interaction between the replication
  machinery and the process being replicated -- i.e. no accidental
  sharing of names used by the process to get its work done and the
  name(s) used by the replication to effect copying. This latter
  revision of the definition of replication is crucial to obtaining
  the expected identity $!!P \sim !P$.
\end{remark}

\begin{remark}\label{rem:paradoxical_combinator}
  The reader familiar with the lambda calculus will have noticed the
  similarity between $D$ and the paradoxical combinator.

  [Ed. note: the existence of this seems to suggest we have to be more
  restrictive on the set of processes and names we admit if we are to
  support no-cloning.]
\end{remark}

\subsubsection{Bisimulation}

The computational dynamics gives rise to another kind of equivalence,
the equivalence of computational behavior. As previously mentioned
this is typically captured \emph{via} some form of bisimulation.

% The notion we use in this paper is weak barbed bisimulation
% \cite{milner91polyadicpi}.

The notion we use in this paper is derived from weak barbed
bisimulation \cite{milner91polyadicpi}. 

\begin{definition}
An \emph{observation relation}, $\downarrow_{\mathcal N}$, over a set
of names, $\mathcal N$, is the smallest relation satisfying the rules
below.

\infrule[Out-barb]{y \in {\mathcal N}, \; x \nameeq y}
		  {\outputp{x}{v} \downarrow_{\mathcal N} x}
\infrule[Par-barb]{\mbox{$P\downarrow_{\mathcal N} x$ or $Q\downarrow_{\mathcal N} x$}}
		  {\binpar{P}{Q} \downarrow_{\mathcal N} x}

We write $P \Downarrow_{\mathcal N} x$ if there is $Q$ such that 
$P \wred Q$ and $Q \downarrow_{\mathcal N} x$.
\end{definition}

\begin{definition}
%\label{def.bbisim}
An  ${\mathcal N}$-\emph{barbed bisimulation} over a set of names, ${\mathcal N}$, is a symmetric binary relation 
${\mathcal S}_{\mathcal N}$ between agents such that $P\rel{S}_{\mathcal N}Q$ implies:
\begin{enumerate}
\item If $P \red P'$ then $Q \wred Q'$ and $P'\rel{S}_{\mathcal N} Q'$.
\item If $P\downarrow_{\mathcal N} x$, then $Q\Downarrow_{\mathcal N} x$.
\end{enumerate}
$P$ is ${\mathcal N}$-barbed bisimilar to $Q$, written
$P \wbbisim_{\mathcal N} Q$, if $P \rel{S}_{\mathcal N} Q$ for some ${\mathcal N}$-barbed bisimulation ${\mathcal S}_{\mathcal N}$.
\end{definition}

$\mathcal{R} \subseteq \pi \times \pi$

$P \mathcal{R} Q => \forall P'. P \red P' \Rightarrow \exists Q'. Q \red Q', P' \mathcal{R} Q'$

$P \vdash x \Rightarrow Q \vdash x$

\begin{mathpar}
  \inferrule*[lab=Out-barb]{x \nameeq y}{{y}!\langle{Q}\rangle \vdash x}
  \and
  \inferrule*[lab=Par-barb]{\mbox{$P\vdash x$ or $Q\vdash x$}}{\binpar{P}{Q} \vdash x}
\end{mathpar}

\subsubsection{Contexts}

One of the principle advantages of computational calculi like the
$\pi$-calculus is a well-defined notion of context,
contextual-equivalence and a correlation between
contextual-equivalence and notions of bisimulation. The notion of
context allows the decomposition of a process into (sub-)process and
its syntactic environment, its context. Thus, a context may be
thought of as a process with a ``hole'' (written $\Box$) in it. The
application of a context $M$ to a process $P$, written $M[P]$, is
tantamount to filling the hole in $M$ with $P$. In this paper we do
not need the full weight of this theory, but do make use of the notion
of context in the proof the main theorem. 

\begin{mathpar}
  \inferrule* [lab=summation] {} {{M_{M},M_{N}} \bc \Box \;|\; x.M_{A} \;|\; M_{M}+M_{N}}
  \and
  \inferrule* [lab=agent] {} {{M_{A}} \bc (\vec{x})M_{P} \;| \; \clift{P_0,\ldots,M_{P},\ldots,P_N}}
  \and \\
  \inferrule* [lab=process] {} {{M_{P}} \bc M_{N} \;| \;P|M_{P} }
\end{mathpar} 

\begin{mathpar}
  \inferrule* [lab=sychronization] {} {M_{N} \bc \Box \;|\; x?M_{F} \;|\; x!M_{C}}
  \and
  \inferrule* [lab=abstraction] {} {{M_{F}} \bc (x)M_{P} }
  \and
  \inferrule* [lab=concretion] {} {{M_{C}} \bc \langle M_{P} \rangle }
  \and \\
  \inferrule* [lab=process] {} {{M_{P}} \bc M_{N} \;| \;P|M_{P} }
\end{mathpar}

\begin{definition}[contextual application] Given a context $M$, and
  process $P$, we define the \emph{contextual application}, $M[P] :=
  M\{P/\Box\}$. That is, the contextual application of M to P is the
  substitution of $P$ for $\Box$ in $M$.
\end{definition}

$\meaningof{-} : L \to \mathcal{P}(\pi)$

\begin{mathpar}
  \inferrule* [lab=collection] {} {\meaningof{true} = \pi, \and \meaningof{~E} = \pi \setminus \meaningof{E}, \and \meaningof{E_{1} \& E_{2}} = \meaningof{E_{1}} \cap \meaningof{E_{2}}}
\end{mathpar}

\begin{mathpar}
  \inferrule* [lab=structure] {} {\meaningof{0} = \{ P \in \pi | P \equiv 0 \}, \and \\ \meaningof{E_1 | E_2} = \{ P \in \pi | P \equiv P_{1} | P_{2}, P_{1} \in \meaningof{E_{1}}, P_{2} \in \meaningof{E_2}\} }
\end{mathpar}

\begin{mathpar}
 \inferrule* [lab=behavior] {} {\meaningof{\langle a?b \rangle E} = \{ P \in \pi | P \equiv Q | u?(y)P', \\ \and \\\\ \and \\ \;\;\; u \in \meaningof{a}, \forall z.P'\{z/y\} \in \meaningof{E\{z/b\}}\}, \and \\ \meaningof{a!E} = \{ P \in \pi | P \equiv Q | x!\langle P' \rangle, x \in \meaningof{a} P' \in \meaningof{E}\} }
\end{mathpar}

\begin{mathpar}
 \inferrule* [lab=nominal] {} {\meaningof{\quotep{E}} = \{ \quotep{P} \in \quotep{\pi} | P \in \meaningof{E} \}, \and \meaningof{\quotep{P}} = \{ \quotep{Q} \in \quotep{\pi} | P \equiv Q \} \and \\ \meaningof{@\quotep{E}} = \{ P \in \pi | P \equiv @x, x \in \meaningof{E} \}}
\end{mathpar}

\begin{eqnarray*}
  \\
  \meaningof{-} : TS \to ST
\end{eqnarray*}

\begin{eqnarray*}
  \\
  L : TS \to ST
\end{eqnarray*}

\begin{eqnarray*}
  \\
  P \models E \iff P \in \meaningof{E}
\end{eqnarray*}

\begin{eqnarray*}
  P \approx_{L} Q \iff \forall E \in L. P \models E \iff Q \models E
\end{eqnarray*}

\begin{eqnarray*}
  P \approx_{K} Q
\end{eqnarray*}

\begin{eqnarray*}
  P \approx Q
\end{eqnarray*}

$\approx_{K} = \approx = \approx_{L}$

\subsubsection{Contextual duality}

Note that contexts extend the quotation operation to a family of
operations from processes to names. Given a context, $M$, we can
define a \emph{nominal context}, $\quotep{M}$ by $\quotep{M}[P] :=
\quotep{M[P]}$. To foreshadow what is to come we observe that these
operations enjoy a duality with processes very much like the duality
between vectors and maps from vectors to scalars.

Further, because the calculus is essentially higher-order, we have a
correspondence between contexts and processes. More specifically,
given a name $x$ and a context $M$ we can construct $M^{*}_{x}$ such
that 

\begin{mathpar}
  M^{*}_{x} | \lift{x}{P} \red M[P]
\end{mathpar}

namely,

\begin{mathpar}
  M^{*}_{x} := x?(u).M[\dropn{u}]
\end{mathpar}

The dependence of $M^{*}_{x}$ on a name makes it an abstraction, 

\begin{mathpar}
  M^{*} := (x)x?(u).M[\dropn{u}]
\end{mathpar}

\subsection{Additional notation}

It will sometimes be convenient to denote the process a name
quotes. We already have the notation $x = \quotep{P}$, but it will be
convenient to introduce an alternate notation, $\procn{x}$, when we
want to emphasize the connection to the use of the name. Note that, by
virtue of name equivalence, $\quotep{\procn{x}} \nameeq x$; so, the
notation is consistent with previous definitions.

Further, because names have structure it is possible to effect
substitutions on the basis of that structure. This means we need to
upgrade our notation for substitutions, which we accomplish by
adapting comprehension notation. Thus,

\begin{mathpar}
  P\{ y / x : x \in S \}
\end{mathpar}

is interpreted to mean the process derived from P by replacing (in a
capture-avoiding manner) each occurrence of $x$ in $S$ by $y$. For example,

\begin{mathpar}
  P\{ \quotep{\procn{x}|\procn{x}} / x : x \in \freenames{P} \}
\end{mathpar}

will replace each (occurrence) of a free name $x$ in $P$ by
$\quotep{\procn{x}|\procn{x}}$.

Also, we will avail ourselves of the notation $x^{L}$ and $x^{R}$ to
denote injections of a name into disjoint copies of the name
space. There are numerous ways to accomplish this. One example can be
found in \cite{MeredithR05}. This notation overloads to vectors of
names: $\vec{x}^{\pi} := (x_{i}^{\pi} \; : \; 0 \leq i < |\vec{x}| )$ where $\pi \in \{L,R\}$.

We also use $P^{\Box} := P|\Box$.

In \cite{MeredithR05} an interpretation of the new operator is
given. It turns out that there are several possible interpretations
all enjoying the requisite algebraic properties of the operator (see
\cite{milner91polyadicpi}). We will therefore make liberal use of
$(\nu\; \vec{x})P$.

% subsection the_syntax_and_semantics_of_the_notation_system (end)   

\input{qm2pi.qmops} 

\input{qm2pi.sterngerlach} 

\input{qm2pi.metric} 

% section concurrent_process_calculi (end)

%\input{qm2pi.proofsketch}

% section proof sketch (end)

%\input{qm2pi.slviaknots} 

% section spatial logic via knots (end)

\input{qm2pi.conclusion}

% section conclusion (end)

%\input{qm2pi.dtcodes} 

% section wiring algorithm (end)

\input{qm2pi.ack} 

% section acknowledgments (end)

\newpage


\bibliographystyle{plain}   
\bibliography{../../biblios/main.bib}

\input{qm2pi.rhodetails}

\end{document}

 

% section wiring algorithm (end)

\documentclass[12pt]{llncs}
%\documentclass{jktr}

\usepackage[pdftex]{hyperref}                   
\usepackage {listings}
\usepackage {mathpartir}
\usepackage{bcprules}
%\usepackage{listings}
                       
\usepackage{graphicx} 
%\usepackage[margins=2.5cm,nohead,nofoot]{geometry}
%\usepackage{geometry}
\usepackage{amsfonts}
\usepackage{amstext}
\usepackage{latexsym}
\usepackage{amssymb}
\usepackage{color}


%\include{myPreamble}
\include{qm2pi.local} 

%\ifpdf
%\usepackage[pdftex]{graphicx}
%\else
%\usepackage{graphicx}
%\fi

 % \ifpdf
%  \usepackage{pdfsync}
%  \if


%\title{Brief Article}
%\author{David F. Snyder}
%\author{L.G. Meredith}

%\address{Dept. of Math., Texas State University--San Marcos, San Marcos, TX 78666}
       
\pagestyle{empty}


\begin{document}

\lstset{language=[Objective]Caml,frame=shadowbox}

\input{qm2pi.front}

% section front matter (end)

\input{qm2pi.intro} 
 
% section introduction (end)

% \input{qm2pi.knotations} 

% section notation (end)

\input{qm2pi.process.calculi} 

% section concurrent_process_calculi_and_spatial_logics_ (end)
    
%\input{qm2pi.knots2pi} 

%\input{qm2pi.trefoil} 

%\input{qm2pi.mainthm} 

% subsection basic_interpretation (end)

%\input{qm2pi.rho.presentation} 
\subsection{The syntax and semantics of the notation system}\label{sub:the_syntax_and_semantics_of_the_notation_system} % (fold)

We now summarize a technical presentation of the calculus that
embodies our theory of dynamics. The typical presentation of such a
calculus follows the style of giving generators and relations on
them. The grammar, below, describing term constructors, freely
generates the set of processes, $\Proc$. This set is then quotiented
by a relation known as structural congruence and it is over this set
that the notion of dynamics is expressed. This presentation is
essentially that of \cite{MeredithR05} with the addition of
polyadicity and summation. For readability we have relegated some of
the technical subtleties to an appendix.

\subsubsection{Process grammar}\label{subsub:process_grammar}

\begin{mathpar}
  \inferrule* [lab=synchronization] {} {{M} \bc \pzero \;|\; x?F \;|\; x!C }
  \and
  \inferrule* [lab=abstraction] {} {{F} \bc (x)P}
  \and
  \inferrule* [lab=concretion] {} {{C} \bc \langle Q \rangle}
  \and
  \inferrule* [lab=process] {} {{P,Q} \bc M \;| \;P|Q \;|\; @{x}}
  \and
  \inferrule* [lab=name] {} {{x} \bc \quotep{P}}
\end{mathpar} 

Note that $\vec{x}$ (resp. $\vec{P}$) denotes a vector of names
(resp. processes) of length $|\vec{x}|$ (resp. $|\vec{P}|$). We adopt
the following useful abbreviations.

\begin{mathpar}
   x?(\vec{y}).P := x.(\vec{y})P \and  x\clift{\vec{P}} := x.\clift{\vec{P}}
   \and x!(y) := \lift{x}{\dropn{y}}
   \and \Pi_{i=0}^{n-1}P_i := P_0 | \ldots | P_{n-1}
\end{mathpar}

\subsubsection{Structural congruence}

\paragraph{Free and bound names and alpha-equivalence.} At the
core of structural equivalence is alpha-equivalence which identifies
process that are the same up to a change of variable. Formally, we
recognize the distinction between free and bound names. The free names
of a process, $\freenames{P}$, may be calculated recursively as
follows:

\begin{mathpar}
\freenames{\pzero} := \emptyset
  \and \\
  \freenames{x?(y).P} := \{ x \} \cup (\freenames{P} \setminus \{ y \})
  \and 
  \freenames{x!\langle P \rangle} := \{ x \} \cup \{ P \} 
  \and \\
  \freenames{P|Q} := \freenames{P} \cup \freenames{Q}
  \and \\
  \freenames{@{x}} := \{ x \}
\end{mathpar}

$\pi$
$\quotep{\pi}$

$\freenames{-} : \pi \to \mathcal{P}(\quotep{\pi})$

\begin{eqnarray*}
  \freenames{\pzero} & := & \emptyset \\
  \freenames{x?(y).P} & := & \{ x \} \cup (\freenames{P} \setminus \{ y \}) \\
  \freenames{x!\langle P \rangle} & := & \{ x \} \cup \{ P \} \\
  \freenames{P|Q} & := & \freenames{P} \cup \freenames{Q} \\
  \freenames{\dropn{x}} & := & \{ x \}
\end{eqnarray*}

The bound names of a process, $\boundnames{P}$, are those names occurring in $P$
that are not free. For example, in $x?(y).0$, the name $x$ is free, while $y$ is bound.

\begin{mathpar}
  \inferrule* [lab=monoidal-laws] {} { P|Q \equiv Q|P \and P|0 \equiv P \and P|(Q|R) \equiv (P|Q)|R }
\end{mathpar}

\begin{mathpar}
  \inferrule* [lab=alpha-equivalence] {} { (x)P \equiv (y)P\{y/x\} \and y \not\in \freenames{P} }
\end{mathpar}

\begin{definition}
Then two processes, $P,Q$, are alpha-equivalent if $P = Q\{\vec{y}/\vec{x}\}$ for
some $\vec{x} \in \boundnames{Q},\vec{y} \in \boundnames{P}$, where $Q\{\vec{y}/\vec{x}\}$
denotes the capture-avoiding substitution of $\vec{y}$ for $\vec{x}$ in $Q$.
\end{definition}

\begin{definition}
  The {\em structural congruence} \cite{SangiorgiWalker} , $\equiv$,
  between processes is the least congruence containing
  alpha-equivalence, satisfying the abelian monoid laws
  (associativity, commutativity and $\pzero$ as identity) for parallel
  composition $|$ and for summation $+$.
\end{definition}

\subsection{Name equivalence}

We take name equivalence, written $\nameeq$, to be the smallest
equivalence relation generated by the following rules.

\begin{mathpar}
\inferrule*[lab=Quote-drop]
{ }
{ \quotep{@{x}} \nameeq x }

\inferrule*[lab=Struct-equiv]
{ P \scong Q }
{ \quotep{P} \nameeq \quotep{Q} }
\end{mathpar}

The astute reader will have noticed that the mutual recursion of names
and processes imposes a mutual recursion on alpha-equivalence and
structural equivalence via name-equivalence. Fortunately, all of this
works out pleasantly and we may calculate in the natural way, free of
concern. The reader interested in the details is referred to the
appendix \ref{appendix:rho_details}.

\subsection{Substitution}

We use $\Proc$ for the set of processes, $\QProc$ for the set of
names, and $\id{\{}\vec{y} / \vec{x} \id{\}}$ to denote partial maps,
$s : \QProc \rightarrow \QProc$. A map, $s$ lifts, uniquely, to a map
on process terms, $\widehat{s} : \Proc \rightarrow \Proc$ by the
following equations.

\begin{mathpar}
  (0) \psubstp{Q}{P} := 0 \\
  (R \juxtap S) \psubstp{Q}{P}
  :=    
  (R)\psubstp{Q}{P} \juxtap (S) \psubstp{Q}{P} \\
  (x?(y).R) \psubstp{Q}{P}    
  :=    
  (x)\substp{Q}{P} (z)\concat( (R \psubstn{z}{y}) \psubstp{Q}{P} ) \\
  (\lift{x}{R}) \psubstp{Q}{P}  
  :=
  \lift{(x)\substp{Q}{P}}{ R \psubstp{Q}{P} } \\
%   (\dropn{x})  \psubstp{Q}{P}       
%   := 
%   \left\{ 
%     \begin{array}{ccc} 
%       \dropn{\quotep{Q}} & & x \nameeq \quotep{P} \\
%       \dropn{x} & & otherwise \\
%     \end{array}
%   \right. 
  (\dropn{x})  \psubstp{Q}{P}       
  := 
  \left\{ 
    \begin{array}{ccc} 
      Q & & x \nameeq \quotep{P} \\
      \dropn{x} & & otherwise \\
    \end{array}
  \right.
\end{mathpar}
 

where

\begin{eqnarray}
  (x)\id{\{} \lpquote Q \rpquote / \lpquote P \rpquote \id{\}}            = 
  \left\{ 
    \begin{array}{ccc}
      \lpquote Q \rpquote & & x \nameeq \lpquote P \rpquote \\
      x & & otherwise \\
    \end{array}
  \right. \nonumber
\end{eqnarray}

and $z$ is chosen distinct from $\quotep{P}$, $\quotep{Q}$, the free
names in $Q$, and all the names in $R$. Our $\alpha$-equivalence will
be built in the standard way from this substitution.

\begin{remark}\label{rem:no_self_referential_names}
  One consequence of these definitions is that $\forall P. \quotep{P}
  \not\in \freenames{P}$.
\end{remark}

\subsection{ Dynamic quote: an example }

Anticipating something of what's to come, consider applying the
substitution, $\widehat{\id{\{}u / z \id{\}}}$, to the following pair
of processes, $\lift{w}{y!(z)}$ and $w[ \lpquote y!(z) \rpquote ]$.

\begin{eqnarray}
	\lift{w}{y!(z)}\widehat{\id{\{}u / z \id{\}}}
		& = &
		\lift{w}{y!(u)} \nonumber\\
	w[ \lpquote y!(z) \rpquote ] \widehat{ \id{\{}u / z \id{\}} }
		& = &
		w[ \lpquote y!(z) \rpquote ] \nonumber
\end{eqnarray}

Because the body of the process between quotes is impervious to
substitution, we get radically different answers. In fact, by
examining the first process in an input context,
e.g. $x?(z).\lift{w}{y!(z)}$, we see that the process under the lift
operator may be shaped by prefixed inputs binding a name inside it. In
this sense, the lift operator will be seen as a way to dynamically
construct processes before reifying them as names.

Finally equipped with these standard features we can present the
dynamics of the calculus.

\subsubsection{Operational semantics} 

Finally, we introduce the computational dynamics. What marks these
algebras as distinct from other more traditionally studied algebraic
structures, e.g. vector spaces or polynomial rings, is the manner in
which dynamics is captured. In traditional structures, dynamics is typically
expressed through morphisms between such structures, as in linear maps
between vector spaces or morphisms between rings. In algebras
associated with the semantics of computation, the dynamics is
expressed as part of the algebraic structure itself, through a
reduction reduction relation typically denoted by $\red$. Below, we
give a recursive presentation of this relation for the calculus used
in the encoding.

$\red \subseteq \pi \times \pi$
$\red : \pi \to \mathcal{P}(\pi)$

\begin{mathpar}
  \inferrule* [lab=Comm] { \textsf{match}( x_{src}, x_{trgt} ) } { x_{trgt}?(y)P \; | \; x_{src}!\langle {Q} \rangle \red P\{\quotep{Q}/y}\} }
  \and \\
  \inferrule* [lab=Par] {{P} \red {P}'} {{{P} | {Q}} \red {{P}' | {Q}}}
  \and
  \inferrule* [lab=Equiv]{{{P} \scong {P}'} \andalso {{P}' \red {Q}'} \andalso {{Q}' \scong {Q}}}{{P} \red {Q}}
\end{mathpar}

\begin{eqnarray*}
  match_{\equiv} (\quotep{P},\quotep{Q}) & := & P \equiv Q \\
  match_{\dagger}(\quotep{P},\quotep{Q}) & := & \forall R. P|Q \red^{*} R => R \red^{*} 0 \\
  match_{K}(\quotep{P},\quotep{Q}) & := & K \mbox{ for some context } K
\end{eqnarray*}

$u?(x)P | u!\langle Q \rangle \red P\{\quotep{Q}/x\}$

%We write $\wred$ for $\red^*$, and $P\red$ if $\exists Q $ such that $ P \red Q$.
We write $P\red$ if $\exists Q $ such that $ P \red Q$ and $P\not\red$, otherwise.

\section{Replication}

As mentioned before, it is known that replication (and hence
recursion) can be implemented in a higher-order process algebra
\cite{SangiorgiWalker}. As our first example of calculation with the
machinery thus far presented we give the construction explicitly in
the {\rhoc}.

\begin{eqnarray}
	D_{x} & := & \prefix{x}{y}{(\binpar{\outputp{x}{y}}{@{y}})} \nonumber\\
	\bangp_{x}{P} & := & \binpar{{x}!\langle{\binpar{D_{x}}{P}}\rangle}{D_{x}} \nonumber
\end{eqnarray}

\begin{eqnarray}
	\bangp_{x}{P} & & \nonumber\\
	=
	& {x}!\langle{(\prefix{x}{y}{(\outputp{x}{y} | @{y})) | P}}\rangle 
	      | \prefix{x}{y}{(\outputp{x}{y} | @{y})} & \nonumber\\
	\red
	& (\outputp{x}{y} | @{y})\substn{\quotep{(\prefix{x}{y}{(@{y} | \outputp{x}{y})) | P}}}{y} & \nonumber\\
	=
	& \outputp{x}{\quotep{(\prefix{x}{y}{(\outputp{x}{y} | @{y})) | P}}}
	  | {(\prefix{x}{y}{(\outputp{x}{y} | @{y})) | P}} & \nonumber\\
	\red
	& \ldots & \nonumber\\
	\red^*
	& P | P | \ldots & \nonumber
\end{eqnarray}

Of course, this encoding, as an implementation, runs away, unfolding
$\bangp{P}$ eagerly. A lazier and more implementable replication
operator, restricted to input-guarded processes, may be obtained as follows.

\begin{eqnarray}
\bangp{\prefix{u}{v}{P}} 
	:= 
	\binpar{\lift{x}{\prefix{u}{v}{(\binpar{D(x)}{P})}}}{D(x)} \nonumber
\end{eqnarray}

\begin{remark}
  Note that the lazier definition still does not deal with summation
  or mixed summation (i.e. sums over input and output). The reader is
  invited to construct definitions of replication that deal with these
  features. 

  Further, the definitions are parameterized in a name, $x$. Can you,
  gentle reader, make a definition that eliminates this parameter and
  guarantees no accidental interaction between the replication
  machinery and the process being replicated -- i.e. no accidental
  sharing of names used by the process to get its work done and the
  name(s) used by the replication to effect copying. This latter
  revision of the definition of replication is crucial to obtaining
  the expected identity $!!P \sim !P$.
\end{remark}

\begin{remark}\label{rem:paradoxical_combinator}
  The reader familiar with the lambda calculus will have noticed the
  similarity between $D$ and the paradoxical combinator.

  [Ed. note: the existence of this seems to suggest we have to be more
  restrictive on the set of processes and names we admit if we are to
  support no-cloning.]
\end{remark}

\subsubsection{Bisimulation}

The computational dynamics gives rise to another kind of equivalence,
the equivalence of computational behavior. As previously mentioned
this is typically captured \emph{via} some form of bisimulation.

% The notion we use in this paper is weak barbed bisimulation
% \cite{milner91polyadicpi}.

The notion we use in this paper is derived from weak barbed
bisimulation \cite{milner91polyadicpi}. 

\begin{definition}
An \emph{observation relation}, $\downarrow_{\mathcal N}$, over a set
of names, $\mathcal N$, is the smallest relation satisfying the rules
below.

\infrule[Out-barb]{y \in {\mathcal N}, \; x \nameeq y}
		  {\outputp{x}{v} \downarrow_{\mathcal N} x}
\infrule[Par-barb]{\mbox{$P\downarrow_{\mathcal N} x$ or $Q\downarrow_{\mathcal N} x$}}
		  {\binpar{P}{Q} \downarrow_{\mathcal N} x}

We write $P \Downarrow_{\mathcal N} x$ if there is $Q$ such that 
$P \wred Q$ and $Q \downarrow_{\mathcal N} x$.
\end{definition}

\begin{definition}
%\label{def.bbisim}
An  ${\mathcal N}$-\emph{barbed bisimulation} over a set of names, ${\mathcal N}$, is a symmetric binary relation 
${\mathcal S}_{\mathcal N}$ between agents such that $P\rel{S}_{\mathcal N}Q$ implies:
\begin{enumerate}
\item If $P \red P'$ then $Q \wred Q'$ and $P'\rel{S}_{\mathcal N} Q'$.
\item If $P\downarrow_{\mathcal N} x$, then $Q\Downarrow_{\mathcal N} x$.
\end{enumerate}
$P$ is ${\mathcal N}$-barbed bisimilar to $Q$, written
$P \wbbisim_{\mathcal N} Q$, if $P \rel{S}_{\mathcal N} Q$ for some ${\mathcal N}$-barbed bisimulation ${\mathcal S}_{\mathcal N}$.
\end{definition}

$\mathcal{R} \subseteq \pi \times \pi$

$P \mathcal{R} Q => \forall P'. P \red P' \Rightarrow \exists Q'. Q \red Q', P' \mathcal{R} Q'$

$P \vdash x \Rightarrow Q \vdash x$

\begin{mathpar}
  \inferrule*[lab=Out-barb]{x \nameeq y}{{y}!\langle{Q}\rangle \vdash x}
  \and
  \inferrule*[lab=Par-barb]{\mbox{$P\vdash x$ or $Q\vdash x$}}{\binpar{P}{Q} \vdash x}
\end{mathpar}

\subsubsection{Contexts}

One of the principle advantages of computational calculi like the
$\pi$-calculus is a well-defined notion of context,
contextual-equivalence and a correlation between
contextual-equivalence and notions of bisimulation. The notion of
context allows the decomposition of a process into (sub-)process and
its syntactic environment, its context. Thus, a context may be
thought of as a process with a ``hole'' (written $\Box$) in it. The
application of a context $M$ to a process $P$, written $M[P]$, is
tantamount to filling the hole in $M$ with $P$. In this paper we do
not need the full weight of this theory, but do make use of the notion
of context in the proof the main theorem. 

\begin{mathpar}
  \inferrule* [lab=summation] {} {{M_{M},M_{N}} \bc \Box \;|\; x.M_{A} \;|\; M_{M}+M_{N}}
  \and
  \inferrule* [lab=agent] {} {{M_{A}} \bc (\vec{x})M_{P} \;| \; \clift{P_0,\ldots,M_{P},\ldots,P_N}}
  \and \\
  \inferrule* [lab=process] {} {{M_{P}} \bc M_{N} \;| \;P|M_{P} }
\end{mathpar} 

\begin{mathpar}
  \inferrule* [lab=sychronization] {} {M_{N} \bc \Box \;|\; x?M_{F} \;|\; x!M_{C}}
  \and
  \inferrule* [lab=abstraction] {} {{M_{F}} \bc (x)M_{P} }
  \and
  \inferrule* [lab=concretion] {} {{M_{C}} \bc \langle M_{P} \rangle }
  \and \\
  \inferrule* [lab=process] {} {{M_{P}} \bc M_{N} \;| \;P|M_{P} }
\end{mathpar}

\begin{definition}[contextual application] Given a context $M$, and
  process $P$, we define the \emph{contextual application}, $M[P] :=
  M\{P/\Box\}$. That is, the contextual application of M to P is the
  substitution of $P$ for $\Box$ in $M$.
\end{definition}

$\meaningof{-} : L \to \mathcal{P}(\pi)$

\begin{mathpar}
  \inferrule* [lab=collection] {} {\meaningof{true} = \pi, \and \meaningof{~E} = \pi \setminus \meaningof{E}, \and \meaningof{E_{1} \& E_{2}} = \meaningof{E_{1}} \cap \meaningof{E_{2}}}
\end{mathpar}

\begin{mathpar}
  \inferrule* [lab=structure] {} {\meaningof{0} = \{ P \in \pi | P \equiv 0 \}, \and \\ \meaningof{E_1 | E_2} = \{ P \in \pi | P \equiv P_{1} | P_{2}, P_{1} \in \meaningof{E_{1}}, P_{2} \in \meaningof{E_2}\} }
\end{mathpar}

\begin{mathpar}
 \inferrule* [lab=behavior] {} {\meaningof{\langle a?b \rangle E} = \{ P \in \pi | P \equiv Q | u?(y)P', \\ \and \\\\ \and \\ \;\;\; u \in \meaningof{a}, \forall z.P'\{z/y\} \in \meaningof{E\{z/b\}}\}, \and \\ \meaningof{a!E} = \{ P \in \pi | P \equiv Q | x!\langle P' \rangle, x \in \meaningof{a} P' \in \meaningof{E}\} }
\end{mathpar}

\begin{mathpar}
 \inferrule* [lab=nominal] {} {\meaningof{\quotep{E}} = \{ \quotep{P} \in \quotep{\pi} | P \in \meaningof{E} \}, \and \meaningof{\quotep{P}} = \{ \quotep{Q} \in \quotep{\pi} | P \equiv Q \} \and \\ \meaningof{@\quotep{E}} = \{ P \in \pi | P \equiv @x, x \in \meaningof{E} \}}
\end{mathpar}

\begin{eqnarray*}
  \\
  \meaningof{-} : TS \to ST
\end{eqnarray*}

\begin{eqnarray*}
  \\
  L : TS \to ST
\end{eqnarray*}

\begin{eqnarray*}
  \\
  P \models E \iff P \in \meaningof{E}
\end{eqnarray*}

\begin{eqnarray*}
  P \approx_{L} Q \iff \forall E \in L. P \models E \iff Q \models E
\end{eqnarray*}

\begin{eqnarray*}
  P \approx_{K} Q
\end{eqnarray*}

\begin{eqnarray*}
  P \approx Q
\end{eqnarray*}

$\approx_{K} = \approx = \approx_{L}$

\subsubsection{Contextual duality}

Note that contexts extend the quotation operation to a family of
operations from processes to names. Given a context, $M$, we can
define a \emph{nominal context}, $\quotep{M}$ by $\quotep{M}[P] :=
\quotep{M[P]}$. To foreshadow what is to come we observe that these
operations enjoy a duality with processes very much like the duality
between vectors and maps from vectors to scalars.

Further, because the calculus is essentially higher-order, we have a
correspondence between contexts and processes. More specifically,
given a name $x$ and a context $M$ we can construct $M^{*}_{x}$ such
that 

\begin{mathpar}
  M^{*}_{x} | \lift{x}{P} \red M[P]
\end{mathpar}

namely,

\begin{mathpar}
  M^{*}_{x} := x?(u).M[\dropn{u}]
\end{mathpar}

The dependence of $M^{*}_{x}$ on a name makes it an abstraction, 

\begin{mathpar}
  M^{*} := (x)x?(u).M[\dropn{u}]
\end{mathpar}

\subsection{Additional notation}

It will sometimes be convenient to denote the process a name
quotes. We already have the notation $x = \quotep{P}$, but it will be
convenient to introduce an alternate notation, $\procn{x}$, when we
want to emphasize the connection to the use of the name. Note that, by
virtue of name equivalence, $\quotep{\procn{x}} \nameeq x$; so, the
notation is consistent with previous definitions.

Further, because names have structure it is possible to effect
substitutions on the basis of that structure. This means we need to
upgrade our notation for substitutions, which we accomplish by
adapting comprehension notation. Thus,

\begin{mathpar}
  P\{ y / x : x \in S \}
\end{mathpar}

is interpreted to mean the process derived from P by replacing (in a
capture-avoiding manner) each occurrence of $x$ in $S$ by $y$. For example,

\begin{mathpar}
  P\{ \quotep{\procn{x}|\procn{x}} / x : x \in \freenames{P} \}
\end{mathpar}

will replace each (occurrence) of a free name $x$ in $P$ by
$\quotep{\procn{x}|\procn{x}}$.

Also, we will avail ourselves of the notation $x^{L}$ and $x^{R}$ to
denote injections of a name into disjoint copies of the name
space. There are numerous ways to accomplish this. One example can be
found in \cite{MeredithR05}. This notation overloads to vectors of
names: $\vec{x}^{\pi} := (x_{i}^{\pi} \; : \; 0 \leq i < |\vec{x}| )$ where $\pi \in \{L,R\}$.

We also use $P^{\Box} := P|\Box$.

In \cite{MeredithR05} an interpretation of the new operator is
given. It turns out that there are several possible interpretations
all enjoying the requisite algebraic properties of the operator (see
\cite{milner91polyadicpi}). We will therefore make liberal use of
$(\nu\; \vec{x})P$.

% subsection the_syntax_and_semantics_of_the_notation_system (end)   

\input{qm2pi.qmops} 

\input{qm2pi.sterngerlach} 

\input{qm2pi.metric} 

% section concurrent_process_calculi (end)

%\input{qm2pi.proofsketch}

% section proof sketch (end)

%\input{qm2pi.slviaknots} 

% section spatial logic via knots (end)

\input{qm2pi.conclusion}

% section conclusion (end)

%\input{qm2pi.dtcodes} 

% section wiring algorithm (end)

\input{qm2pi.ack} 

% section acknowledgments (end)

\newpage


\bibliographystyle{plain}   
\bibliography{../../biblios/main.bib}

\input{qm2pi.rhodetails}

\end{document}

 

% section acknowledgments (end)

\newpage


\bibliographystyle{plain}   
\bibliography{../../biblios/main.bib}

\documentclass[12pt]{llncs}
%\documentclass{jktr}

\usepackage[pdftex]{hyperref}                   
\usepackage {listings}
\usepackage {mathpartir}
\usepackage{bcprules}
%\usepackage{listings}
                       
\usepackage{graphicx} 
%\usepackage[margins=2.5cm,nohead,nofoot]{geometry}
%\usepackage{geometry}
\usepackage{amsfonts}
\usepackage{amstext}
\usepackage{latexsym}
\usepackage{amssymb}
\usepackage{color}


%\include{myPreamble}
\include{qm2pi.local} 

%\ifpdf
%\usepackage[pdftex]{graphicx}
%\else
%\usepackage{graphicx}
%\fi

 % \ifpdf
%  \usepackage{pdfsync}
%  \if


%\title{Brief Article}
%\author{David F. Snyder}
%\author{L.G. Meredith}

%\address{Dept. of Math., Texas State University--San Marcos, San Marcos, TX 78666}
       
\pagestyle{empty}


\begin{document}

\lstset{language=[Objective]Caml,frame=shadowbox}

\input{qm2pi.front}

% section front matter (end)

\input{qm2pi.intro} 
 
% section introduction (end)

% \input{qm2pi.knotations} 

% section notation (end)

\input{qm2pi.process.calculi} 

% section concurrent_process_calculi_and_spatial_logics_ (end)
    
%\input{qm2pi.knots2pi} 

%\input{qm2pi.trefoil} 

%\input{qm2pi.mainthm} 

% subsection basic_interpretation (end)

%\input{qm2pi.rho.presentation} 
\subsection{The syntax and semantics of the notation system}\label{sub:the_syntax_and_semantics_of_the_notation_system} % (fold)

We now summarize a technical presentation of the calculus that
embodies our theory of dynamics. The typical presentation of such a
calculus follows the style of giving generators and relations on
them. The grammar, below, describing term constructors, freely
generates the set of processes, $\Proc$. This set is then quotiented
by a relation known as structural congruence and it is over this set
that the notion of dynamics is expressed. This presentation is
essentially that of \cite{MeredithR05} with the addition of
polyadicity and summation. For readability we have relegated some of
the technical subtleties to an appendix.

\subsubsection{Process grammar}\label{subsub:process_grammar}

\begin{mathpar}
  \inferrule* [lab=synchronization] {} {{M} \bc \pzero \;|\; x?F \;|\; x!C }
  \and
  \inferrule* [lab=abstraction] {} {{F} \bc (x)P}
  \and
  \inferrule* [lab=concretion] {} {{C} \bc \langle Q \rangle}
  \and
  \inferrule* [lab=process] {} {{P,Q} \bc M \;| \;P|Q \;|\; @{x}}
  \and
  \inferrule* [lab=name] {} {{x} \bc \quotep{P}}
\end{mathpar} 

Note that $\vec{x}$ (resp. $\vec{P}$) denotes a vector of names
(resp. processes) of length $|\vec{x}|$ (resp. $|\vec{P}|$). We adopt
the following useful abbreviations.

\begin{mathpar}
   x?(\vec{y}).P := x.(\vec{y})P \and  x\clift{\vec{P}} := x.\clift{\vec{P}}
   \and x!(y) := \lift{x}{\dropn{y}}
   \and \Pi_{i=0}^{n-1}P_i := P_0 | \ldots | P_{n-1}
\end{mathpar}

\subsubsection{Structural congruence}

\paragraph{Free and bound names and alpha-equivalence.} At the
core of structural equivalence is alpha-equivalence which identifies
process that are the same up to a change of variable. Formally, we
recognize the distinction between free and bound names. The free names
of a process, $\freenames{P}$, may be calculated recursively as
follows:

\begin{mathpar}
\freenames{\pzero} := \emptyset
  \and \\
  \freenames{x?(y).P} := \{ x \} \cup (\freenames{P} \setminus \{ y \})
  \and 
  \freenames{x!\langle P \rangle} := \{ x \} \cup \{ P \} 
  \and \\
  \freenames{P|Q} := \freenames{P} \cup \freenames{Q}
  \and \\
  \freenames{@{x}} := \{ x \}
\end{mathpar}

$\pi$
$\quotep{\pi}$

$\freenames{-} : \pi \to \mathcal{P}(\quotep{\pi})$

\begin{eqnarray*}
  \freenames{\pzero} & := & \emptyset \\
  \freenames{x?(y).P} & := & \{ x \} \cup (\freenames{P} \setminus \{ y \}) \\
  \freenames{x!\langle P \rangle} & := & \{ x \} \cup \{ P \} \\
  \freenames{P|Q} & := & \freenames{P} \cup \freenames{Q} \\
  \freenames{\dropn{x}} & := & \{ x \}
\end{eqnarray*}

The bound names of a process, $\boundnames{P}$, are those names occurring in $P$
that are not free. For example, in $x?(y).0$, the name $x$ is free, while $y$ is bound.

\begin{mathpar}
  \inferrule* [lab=monoidal-laws] {} { P|Q \equiv Q|P \and P|0 \equiv P \and P|(Q|R) \equiv (P|Q)|R }
\end{mathpar}

\begin{mathpar}
  \inferrule* [lab=alpha-equivalence] {} { (x)P \equiv (y)P\{y/x\} \and y \not\in \freenames{P} }
\end{mathpar}

\begin{definition}
Then two processes, $P,Q$, are alpha-equivalent if $P = Q\{\vec{y}/\vec{x}\}$ for
some $\vec{x} \in \boundnames{Q},\vec{y} \in \boundnames{P}$, where $Q\{\vec{y}/\vec{x}\}$
denotes the capture-avoiding substitution of $\vec{y}$ for $\vec{x}$ in $Q$.
\end{definition}

\begin{definition}
  The {\em structural congruence} \cite{SangiorgiWalker} , $\equiv$,
  between processes is the least congruence containing
  alpha-equivalence, satisfying the abelian monoid laws
  (associativity, commutativity and $\pzero$ as identity) for parallel
  composition $|$ and for summation $+$.
\end{definition}

\subsection{Name equivalence}

We take name equivalence, written $\nameeq$, to be the smallest
equivalence relation generated by the following rules.

\begin{mathpar}
\inferrule*[lab=Quote-drop]
{ }
{ \quotep{@{x}} \nameeq x }

\inferrule*[lab=Struct-equiv]
{ P \scong Q }
{ \quotep{P} \nameeq \quotep{Q} }
\end{mathpar}

The astute reader will have noticed that the mutual recursion of names
and processes imposes a mutual recursion on alpha-equivalence and
structural equivalence via name-equivalence. Fortunately, all of this
works out pleasantly and we may calculate in the natural way, free of
concern. The reader interested in the details is referred to the
appendix \ref{appendix:rho_details}.

\subsection{Substitution}

We use $\Proc$ for the set of processes, $\QProc$ for the set of
names, and $\id{\{}\vec{y} / \vec{x} \id{\}}$ to denote partial maps,
$s : \QProc \rightarrow \QProc$. A map, $s$ lifts, uniquely, to a map
on process terms, $\widehat{s} : \Proc \rightarrow \Proc$ by the
following equations.

\begin{mathpar}
  (0) \psubstp{Q}{P} := 0 \\
  (R \juxtap S) \psubstp{Q}{P}
  :=    
  (R)\psubstp{Q}{P} \juxtap (S) \psubstp{Q}{P} \\
  (x?(y).R) \psubstp{Q}{P}    
  :=    
  (x)\substp{Q}{P} (z)\concat( (R \psubstn{z}{y}) \psubstp{Q}{P} ) \\
  (\lift{x}{R}) \psubstp{Q}{P}  
  :=
  \lift{(x)\substp{Q}{P}}{ R \psubstp{Q}{P} } \\
%   (\dropn{x})  \psubstp{Q}{P}       
%   := 
%   \left\{ 
%     \begin{array}{ccc} 
%       \dropn{\quotep{Q}} & & x \nameeq \quotep{P} \\
%       \dropn{x} & & otherwise \\
%     \end{array}
%   \right. 
  (\dropn{x})  \psubstp{Q}{P}       
  := 
  \left\{ 
    \begin{array}{ccc} 
      Q & & x \nameeq \quotep{P} \\
      \dropn{x} & & otherwise \\
    \end{array}
  \right.
\end{mathpar}
 

where

\begin{eqnarray}
  (x)\id{\{} \lpquote Q \rpquote / \lpquote P \rpquote \id{\}}            = 
  \left\{ 
    \begin{array}{ccc}
      \lpquote Q \rpquote & & x \nameeq \lpquote P \rpquote \\
      x & & otherwise \\
    \end{array}
  \right. \nonumber
\end{eqnarray}

and $z$ is chosen distinct from $\quotep{P}$, $\quotep{Q}$, the free
names in $Q$, and all the names in $R$. Our $\alpha$-equivalence will
be built in the standard way from this substitution.

\begin{remark}\label{rem:no_self_referential_names}
  One consequence of these definitions is that $\forall P. \quotep{P}
  \not\in \freenames{P}$.
\end{remark}

\subsection{ Dynamic quote: an example }

Anticipating something of what's to come, consider applying the
substitution, $\widehat{\id{\{}u / z \id{\}}}$, to the following pair
of processes, $\lift{w}{y!(z)}$ and $w[ \lpquote y!(z) \rpquote ]$.

\begin{eqnarray}
	\lift{w}{y!(z)}\widehat{\id{\{}u / z \id{\}}}
		& = &
		\lift{w}{y!(u)} \nonumber\\
	w[ \lpquote y!(z) \rpquote ] \widehat{ \id{\{}u / z \id{\}} }
		& = &
		w[ \lpquote y!(z) \rpquote ] \nonumber
\end{eqnarray}

Because the body of the process between quotes is impervious to
substitution, we get radically different answers. In fact, by
examining the first process in an input context,
e.g. $x?(z).\lift{w}{y!(z)}$, we see that the process under the lift
operator may be shaped by prefixed inputs binding a name inside it. In
this sense, the lift operator will be seen as a way to dynamically
construct processes before reifying them as names.

Finally equipped with these standard features we can present the
dynamics of the calculus.

\subsubsection{Operational semantics} 

Finally, we introduce the computational dynamics. What marks these
algebras as distinct from other more traditionally studied algebraic
structures, e.g. vector spaces or polynomial rings, is the manner in
which dynamics is captured. In traditional structures, dynamics is typically
expressed through morphisms between such structures, as in linear maps
between vector spaces or morphisms between rings. In algebras
associated with the semantics of computation, the dynamics is
expressed as part of the algebraic structure itself, through a
reduction reduction relation typically denoted by $\red$. Below, we
give a recursive presentation of this relation for the calculus used
in the encoding.

$\red \subseteq \pi \times \pi$
$\red : \pi \to \mathcal{P}(\pi)$

\begin{mathpar}
  \inferrule* [lab=Comm] { \textsf{match}( x_{src}, x_{trgt} ) } { x_{trgt}?(y)P \; | \; x_{src}!\langle {Q} \rangle \red P\{\quotep{Q}/y}\} }
  \and \\
  \inferrule* [lab=Par] {{P} \red {P}'} {{{P} | {Q}} \red {{P}' | {Q}}}
  \and
  \inferrule* [lab=Equiv]{{{P} \scong {P}'} \andalso {{P}' \red {Q}'} \andalso {{Q}' \scong {Q}}}{{P} \red {Q}}
\end{mathpar}

\begin{eqnarray*}
  match_{\equiv} (\quotep{P},\quotep{Q}) & := & P \equiv Q \\
  match_{\dagger}(\quotep{P},\quotep{Q}) & := & \forall R. P|Q \red^{*} R => R \red^{*} 0 \\
  match_{K}(\quotep{P},\quotep{Q}) & := & K \mbox{ for some context } K
\end{eqnarray*}

$u?(x)P | u!\langle Q \rangle \red P\{\quotep{Q}/x\}$

%We write $\wred$ for $\red^*$, and $P\red$ if $\exists Q $ such that $ P \red Q$.
We write $P\red$ if $\exists Q $ such that $ P \red Q$ and $P\not\red$, otherwise.

\section{Replication}

As mentioned before, it is known that replication (and hence
recursion) can be implemented in a higher-order process algebra
\cite{SangiorgiWalker}. As our first example of calculation with the
machinery thus far presented we give the construction explicitly in
the {\rhoc}.

\begin{eqnarray}
	D_{x} & := & \prefix{x}{y}{(\binpar{\outputp{x}{y}}{@{y}})} \nonumber\\
	\bangp_{x}{P} & := & \binpar{{x}!\langle{\binpar{D_{x}}{P}}\rangle}{D_{x}} \nonumber
\end{eqnarray}

\begin{eqnarray}
	\bangp_{x}{P} & & \nonumber\\
	=
	& {x}!\langle{(\prefix{x}{y}{(\outputp{x}{y} | @{y})) | P}}\rangle 
	      | \prefix{x}{y}{(\outputp{x}{y} | @{y})} & \nonumber\\
	\red
	& (\outputp{x}{y} | @{y})\substn{\quotep{(\prefix{x}{y}{(@{y} | \outputp{x}{y})) | P}}}{y} & \nonumber\\
	=
	& \outputp{x}{\quotep{(\prefix{x}{y}{(\outputp{x}{y} | @{y})) | P}}}
	  | {(\prefix{x}{y}{(\outputp{x}{y} | @{y})) | P}} & \nonumber\\
	\red
	& \ldots & \nonumber\\
	\red^*
	& P | P | \ldots & \nonumber
\end{eqnarray}

Of course, this encoding, as an implementation, runs away, unfolding
$\bangp{P}$ eagerly. A lazier and more implementable replication
operator, restricted to input-guarded processes, may be obtained as follows.

\begin{eqnarray}
\bangp{\prefix{u}{v}{P}} 
	:= 
	\binpar{\lift{x}{\prefix{u}{v}{(\binpar{D(x)}{P})}}}{D(x)} \nonumber
\end{eqnarray}

\begin{remark}
  Note that the lazier definition still does not deal with summation
  or mixed summation (i.e. sums over input and output). The reader is
  invited to construct definitions of replication that deal with these
  features. 

  Further, the definitions are parameterized in a name, $x$. Can you,
  gentle reader, make a definition that eliminates this parameter and
  guarantees no accidental interaction between the replication
  machinery and the process being replicated -- i.e. no accidental
  sharing of names used by the process to get its work done and the
  name(s) used by the replication to effect copying. This latter
  revision of the definition of replication is crucial to obtaining
  the expected identity $!!P \sim !P$.
\end{remark}

\begin{remark}\label{rem:paradoxical_combinator}
  The reader familiar with the lambda calculus will have noticed the
  similarity between $D$ and the paradoxical combinator.

  [Ed. note: the existence of this seems to suggest we have to be more
  restrictive on the set of processes and names we admit if we are to
  support no-cloning.]
\end{remark}

\subsubsection{Bisimulation}

The computational dynamics gives rise to another kind of equivalence,
the equivalence of computational behavior. As previously mentioned
this is typically captured \emph{via} some form of bisimulation.

% The notion we use in this paper is weak barbed bisimulation
% \cite{milner91polyadicpi}.

The notion we use in this paper is derived from weak barbed
bisimulation \cite{milner91polyadicpi}. 

\begin{definition}
An \emph{observation relation}, $\downarrow_{\mathcal N}$, over a set
of names, $\mathcal N$, is the smallest relation satisfying the rules
below.

\infrule[Out-barb]{y \in {\mathcal N}, \; x \nameeq y}
		  {\outputp{x}{v} \downarrow_{\mathcal N} x}
\infrule[Par-barb]{\mbox{$P\downarrow_{\mathcal N} x$ or $Q\downarrow_{\mathcal N} x$}}
		  {\binpar{P}{Q} \downarrow_{\mathcal N} x}

We write $P \Downarrow_{\mathcal N} x$ if there is $Q$ such that 
$P \wred Q$ and $Q \downarrow_{\mathcal N} x$.
\end{definition}

\begin{definition}
%\label{def.bbisim}
An  ${\mathcal N}$-\emph{barbed bisimulation} over a set of names, ${\mathcal N}$, is a symmetric binary relation 
${\mathcal S}_{\mathcal N}$ between agents such that $P\rel{S}_{\mathcal N}Q$ implies:
\begin{enumerate}
\item If $P \red P'$ then $Q \wred Q'$ and $P'\rel{S}_{\mathcal N} Q'$.
\item If $P\downarrow_{\mathcal N} x$, then $Q\Downarrow_{\mathcal N} x$.
\end{enumerate}
$P$ is ${\mathcal N}$-barbed bisimilar to $Q$, written
$P \wbbisim_{\mathcal N} Q$, if $P \rel{S}_{\mathcal N} Q$ for some ${\mathcal N}$-barbed bisimulation ${\mathcal S}_{\mathcal N}$.
\end{definition}

$\mathcal{R} \subseteq \pi \times \pi$

$P \mathcal{R} Q => \forall P'. P \red P' \Rightarrow \exists Q'. Q \red Q', P' \mathcal{R} Q'$

$P \vdash x \Rightarrow Q \vdash x$

\begin{mathpar}
  \inferrule*[lab=Out-barb]{x \nameeq y}{{y}!\langle{Q}\rangle \vdash x}
  \and
  \inferrule*[lab=Par-barb]{\mbox{$P\vdash x$ or $Q\vdash x$}}{\binpar{P}{Q} \vdash x}
\end{mathpar}

\subsubsection{Contexts}

One of the principle advantages of computational calculi like the
$\pi$-calculus is a well-defined notion of context,
contextual-equivalence and a correlation between
contextual-equivalence and notions of bisimulation. The notion of
context allows the decomposition of a process into (sub-)process and
its syntactic environment, its context. Thus, a context may be
thought of as a process with a ``hole'' (written $\Box$) in it. The
application of a context $M$ to a process $P$, written $M[P]$, is
tantamount to filling the hole in $M$ with $P$. In this paper we do
not need the full weight of this theory, but do make use of the notion
of context in the proof the main theorem. 

\begin{mathpar}
  \inferrule* [lab=summation] {} {{M_{M},M_{N}} \bc \Box \;|\; x.M_{A} \;|\; M_{M}+M_{N}}
  \and
  \inferrule* [lab=agent] {} {{M_{A}} \bc (\vec{x})M_{P} \;| \; \clift{P_0,\ldots,M_{P},\ldots,P_N}}
  \and \\
  \inferrule* [lab=process] {} {{M_{P}} \bc M_{N} \;| \;P|M_{P} }
\end{mathpar} 

\begin{mathpar}
  \inferrule* [lab=sychronization] {} {M_{N} \bc \Box \;|\; x?M_{F} \;|\; x!M_{C}}
  \and
  \inferrule* [lab=abstraction] {} {{M_{F}} \bc (x)M_{P} }
  \and
  \inferrule* [lab=concretion] {} {{M_{C}} \bc \langle M_{P} \rangle }
  \and \\
  \inferrule* [lab=process] {} {{M_{P}} \bc M_{N} \;| \;P|M_{P} }
\end{mathpar}

\begin{definition}[contextual application] Given a context $M$, and
  process $P$, we define the \emph{contextual application}, $M[P] :=
  M\{P/\Box\}$. That is, the contextual application of M to P is the
  substitution of $P$ for $\Box$ in $M$.
\end{definition}

$\meaningof{-} : L \to \mathcal{P}(\pi)$

\begin{mathpar}
  \inferrule* [lab=collection] {} {\meaningof{true} = \pi, \and \meaningof{~E} = \pi \setminus \meaningof{E}, \and \meaningof{E_{1} \& E_{2}} = \meaningof{E_{1}} \cap \meaningof{E_{2}}}
\end{mathpar}

\begin{mathpar}
  \inferrule* [lab=structure] {} {\meaningof{0} = \{ P \in \pi | P \equiv 0 \}, \and \\ \meaningof{E_1 | E_2} = \{ P \in \pi | P \equiv P_{1} | P_{2}, P_{1} \in \meaningof{E_{1}}, P_{2} \in \meaningof{E_2}\} }
\end{mathpar}

\begin{mathpar}
 \inferrule* [lab=behavior] {} {\meaningof{\langle a?b \rangle E} = \{ P \in \pi | P \equiv Q | u?(y)P', \\ \and \\\\ \and \\ \;\;\; u \in \meaningof{a}, \forall z.P'\{z/y\} \in \meaningof{E\{z/b\}}\}, \and \\ \meaningof{a!E} = \{ P \in \pi | P \equiv Q | x!\langle P' \rangle, x \in \meaningof{a} P' \in \meaningof{E}\} }
\end{mathpar}

\begin{mathpar}
 \inferrule* [lab=nominal] {} {\meaningof{\quotep{E}} = \{ \quotep{P} \in \quotep{\pi} | P \in \meaningof{E} \}, \and \meaningof{\quotep{P}} = \{ \quotep{Q} \in \quotep{\pi} | P \equiv Q \} \and \\ \meaningof{@\quotep{E}} = \{ P \in \pi | P \equiv @x, x \in \meaningof{E} \}}
\end{mathpar}

\begin{eqnarray*}
  \\
  \meaningof{-} : TS \to ST
\end{eqnarray*}

\begin{eqnarray*}
  \\
  L : TS \to ST
\end{eqnarray*}

\begin{eqnarray*}
  \\
  P \models E \iff P \in \meaningof{E}
\end{eqnarray*}

\begin{eqnarray*}
  P \approx_{L} Q \iff \forall E \in L. P \models E \iff Q \models E
\end{eqnarray*}

\begin{eqnarray*}
  P \approx_{K} Q
\end{eqnarray*}

\begin{eqnarray*}
  P \approx Q
\end{eqnarray*}

$\approx_{K} = \approx = \approx_{L}$

\subsubsection{Contextual duality}

Note that contexts extend the quotation operation to a family of
operations from processes to names. Given a context, $M$, we can
define a \emph{nominal context}, $\quotep{M}$ by $\quotep{M}[P] :=
\quotep{M[P]}$. To foreshadow what is to come we observe that these
operations enjoy a duality with processes very much like the duality
between vectors and maps from vectors to scalars.

Further, because the calculus is essentially higher-order, we have a
correspondence between contexts and processes. More specifically,
given a name $x$ and a context $M$ we can construct $M^{*}_{x}$ such
that 

\begin{mathpar}
  M^{*}_{x} | \lift{x}{P} \red M[P]
\end{mathpar}

namely,

\begin{mathpar}
  M^{*}_{x} := x?(u).M[\dropn{u}]
\end{mathpar}

The dependence of $M^{*}_{x}$ on a name makes it an abstraction, 

\begin{mathpar}
  M^{*} := (x)x?(u).M[\dropn{u}]
\end{mathpar}

\subsection{Additional notation}

It will sometimes be convenient to denote the process a name
quotes. We already have the notation $x = \quotep{P}$, but it will be
convenient to introduce an alternate notation, $\procn{x}$, when we
want to emphasize the connection to the use of the name. Note that, by
virtue of name equivalence, $\quotep{\procn{x}} \nameeq x$; so, the
notation is consistent with previous definitions.

Further, because names have structure it is possible to effect
substitutions on the basis of that structure. This means we need to
upgrade our notation for substitutions, which we accomplish by
adapting comprehension notation. Thus,

\begin{mathpar}
  P\{ y / x : x \in S \}
\end{mathpar}

is interpreted to mean the process derived from P by replacing (in a
capture-avoiding manner) each occurrence of $x$ in $S$ by $y$. For example,

\begin{mathpar}
  P\{ \quotep{\procn{x}|\procn{x}} / x : x \in \freenames{P} \}
\end{mathpar}

will replace each (occurrence) of a free name $x$ in $P$ by
$\quotep{\procn{x}|\procn{x}}$.

Also, we will avail ourselves of the notation $x^{L}$ and $x^{R}$ to
denote injections of a name into disjoint copies of the name
space. There are numerous ways to accomplish this. One example can be
found in \cite{MeredithR05}. This notation overloads to vectors of
names: $\vec{x}^{\pi} := (x_{i}^{\pi} \; : \; 0 \leq i < |\vec{x}| )$ where $\pi \in \{L,R\}$.

We also use $P^{\Box} := P|\Box$.

In \cite{MeredithR05} an interpretation of the new operator is
given. It turns out that there are several possible interpretations
all enjoying the requisite algebraic properties of the operator (see
\cite{milner91polyadicpi}). We will therefore make liberal use of
$(\nu\; \vec{x})P$.

% subsection the_syntax_and_semantics_of_the_notation_system (end)   

\input{qm2pi.qmops} 

\input{qm2pi.sterngerlach} 

\input{qm2pi.metric} 

% section concurrent_process_calculi (end)

%\input{qm2pi.proofsketch}

% section proof sketch (end)

%\input{qm2pi.slviaknots} 

% section spatial logic via knots (end)

\input{qm2pi.conclusion}

% section conclusion (end)

%\input{qm2pi.dtcodes} 

% section wiring algorithm (end)

\input{qm2pi.ack} 

% section acknowledgments (end)

\newpage


\bibliographystyle{plain}   
\bibliography{../../biblios/main.bib}

\input{qm2pi.rhodetails}

\end{document}



\end{document}

 

% section notation (end)

\input{qm2pi.process.calculi} 

% section concurrent_process_calculi_and_spatial_logics_ (end)
    
%\documentclass[12pt]{llncs}
%\documentclass{jktr}

\usepackage[pdftex]{hyperref}                   
\usepackage {listings}
\usepackage {mathpartir}
\usepackage{bcprules}
%\usepackage{listings}
                       
\usepackage{graphicx} 
%\usepackage[margins=2.5cm,nohead,nofoot]{geometry}
%\usepackage{geometry}
\usepackage{amsfonts}
\usepackage{amstext}
\usepackage{latexsym}
\usepackage{amssymb}
\usepackage{color}


%\include{myPreamble}
\documentclass[12pt]{llncs}
%\documentclass{jktr}

\usepackage[pdftex]{hyperref}                   
\usepackage {listings}
\usepackage {mathpartir}
\usepackage{bcprules}
%\usepackage{listings}
                       
\usepackage{graphicx} 
%\usepackage[margins=2.5cm,nohead,nofoot]{geometry}
%\usepackage{geometry}
\usepackage{amsfonts}
\usepackage{amstext}
\usepackage{latexsym}
\usepackage{amssymb}
\usepackage{color}


%\include{myPreamble}
\include{qm2pi.local} 

%\ifpdf
%\usepackage[pdftex]{graphicx}
%\else
%\usepackage{graphicx}
%\fi

 % \ifpdf
%  \usepackage{pdfsync}
%  \if


%\title{Brief Article}
%\author{David F. Snyder}
%\author{L.G. Meredith}

%\address{Dept. of Math., Texas State University--San Marcos, San Marcos, TX 78666}
       
\pagestyle{empty}


\begin{document}

\lstset{language=[Objective]Caml,frame=shadowbox}

\input{qm2pi.front}

% section front matter (end)

\input{qm2pi.intro} 
 
% section introduction (end)

% \input{qm2pi.knotations} 

% section notation (end)

\input{qm2pi.process.calculi} 

% section concurrent_process_calculi_and_spatial_logics_ (end)
    
%\input{qm2pi.knots2pi} 

%\input{qm2pi.trefoil} 

%\input{qm2pi.mainthm} 

% subsection basic_interpretation (end)

%\input{qm2pi.rho.presentation} 
\subsection{The syntax and semantics of the notation system}\label{sub:the_syntax_and_semantics_of_the_notation_system} % (fold)

We now summarize a technical presentation of the calculus that
embodies our theory of dynamics. The typical presentation of such a
calculus follows the style of giving generators and relations on
them. The grammar, below, describing term constructors, freely
generates the set of processes, $\Proc$. This set is then quotiented
by a relation known as structural congruence and it is over this set
that the notion of dynamics is expressed. This presentation is
essentially that of \cite{MeredithR05} with the addition of
polyadicity and summation. For readability we have relegated some of
the technical subtleties to an appendix.

\subsubsection{Process grammar}\label{subsub:process_grammar}

\begin{mathpar}
  \inferrule* [lab=synchronization] {} {{M} \bc \pzero \;|\; x?F \;|\; x!C }
  \and
  \inferrule* [lab=abstraction] {} {{F} \bc (x)P}
  \and
  \inferrule* [lab=concretion] {} {{C} \bc \langle Q \rangle}
  \and
  \inferrule* [lab=process] {} {{P,Q} \bc M \;| \;P|Q \;|\; @{x}}
  \and
  \inferrule* [lab=name] {} {{x} \bc \quotep{P}}
\end{mathpar} 

Note that $\vec{x}$ (resp. $\vec{P}$) denotes a vector of names
(resp. processes) of length $|\vec{x}|$ (resp. $|\vec{P}|$). We adopt
the following useful abbreviations.

\begin{mathpar}
   x?(\vec{y}).P := x.(\vec{y})P \and  x\clift{\vec{P}} := x.\clift{\vec{P}}
   \and x!(y) := \lift{x}{\dropn{y}}
   \and \Pi_{i=0}^{n-1}P_i := P_0 | \ldots | P_{n-1}
\end{mathpar}

\subsubsection{Structural congruence}

\paragraph{Free and bound names and alpha-equivalence.} At the
core of structural equivalence is alpha-equivalence which identifies
process that are the same up to a change of variable. Formally, we
recognize the distinction between free and bound names. The free names
of a process, $\freenames{P}$, may be calculated recursively as
follows:

\begin{mathpar}
\freenames{\pzero} := \emptyset
  \and \\
  \freenames{x?(y).P} := \{ x \} \cup (\freenames{P} \setminus \{ y \})
  \and 
  \freenames{x!\langle P \rangle} := \{ x \} \cup \{ P \} 
  \and \\
  \freenames{P|Q} := \freenames{P} \cup \freenames{Q}
  \and \\
  \freenames{@{x}} := \{ x \}
\end{mathpar}

$\pi$
$\quotep{\pi}$

$\freenames{-} : \pi \to \mathcal{P}(\quotep{\pi})$

\begin{eqnarray*}
  \freenames{\pzero} & := & \emptyset \\
  \freenames{x?(y).P} & := & \{ x \} \cup (\freenames{P} \setminus \{ y \}) \\
  \freenames{x!\langle P \rangle} & := & \{ x \} \cup \{ P \} \\
  \freenames{P|Q} & := & \freenames{P} \cup \freenames{Q} \\
  \freenames{\dropn{x}} & := & \{ x \}
\end{eqnarray*}

The bound names of a process, $\boundnames{P}$, are those names occurring in $P$
that are not free. For example, in $x?(y).0$, the name $x$ is free, while $y$ is bound.

\begin{mathpar}
  \inferrule* [lab=monoidal-laws] {} { P|Q \equiv Q|P \and P|0 \equiv P \and P|(Q|R) \equiv (P|Q)|R }
\end{mathpar}

\begin{mathpar}
  \inferrule* [lab=alpha-equivalence] {} { (x)P \equiv (y)P\{y/x\} \and y \not\in \freenames{P} }
\end{mathpar}

\begin{definition}
Then two processes, $P,Q$, are alpha-equivalent if $P = Q\{\vec{y}/\vec{x}\}$ for
some $\vec{x} \in \boundnames{Q},\vec{y} \in \boundnames{P}$, where $Q\{\vec{y}/\vec{x}\}$
denotes the capture-avoiding substitution of $\vec{y}$ for $\vec{x}$ in $Q$.
\end{definition}

\begin{definition}
  The {\em structural congruence} \cite{SangiorgiWalker} , $\equiv$,
  between processes is the least congruence containing
  alpha-equivalence, satisfying the abelian monoid laws
  (associativity, commutativity and $\pzero$ as identity) for parallel
  composition $|$ and for summation $+$.
\end{definition}

\subsection{Name equivalence}

We take name equivalence, written $\nameeq$, to be the smallest
equivalence relation generated by the following rules.

\begin{mathpar}
\inferrule*[lab=Quote-drop]
{ }
{ \quotep{@{x}} \nameeq x }

\inferrule*[lab=Struct-equiv]
{ P \scong Q }
{ \quotep{P} \nameeq \quotep{Q} }
\end{mathpar}

The astute reader will have noticed that the mutual recursion of names
and processes imposes a mutual recursion on alpha-equivalence and
structural equivalence via name-equivalence. Fortunately, all of this
works out pleasantly and we may calculate in the natural way, free of
concern. The reader interested in the details is referred to the
appendix \ref{appendix:rho_details}.

\subsection{Substitution}

We use $\Proc$ for the set of processes, $\QProc$ for the set of
names, and $\id{\{}\vec{y} / \vec{x} \id{\}}$ to denote partial maps,
$s : \QProc \rightarrow \QProc$. A map, $s$ lifts, uniquely, to a map
on process terms, $\widehat{s} : \Proc \rightarrow \Proc$ by the
following equations.

\begin{mathpar}
  (0) \psubstp{Q}{P} := 0 \\
  (R \juxtap S) \psubstp{Q}{P}
  :=    
  (R)\psubstp{Q}{P} \juxtap (S) \psubstp{Q}{P} \\
  (x?(y).R) \psubstp{Q}{P}    
  :=    
  (x)\substp{Q}{P} (z)\concat( (R \psubstn{z}{y}) \psubstp{Q}{P} ) \\
  (\lift{x}{R}) \psubstp{Q}{P}  
  :=
  \lift{(x)\substp{Q}{P}}{ R \psubstp{Q}{P} } \\
%   (\dropn{x})  \psubstp{Q}{P}       
%   := 
%   \left\{ 
%     \begin{array}{ccc} 
%       \dropn{\quotep{Q}} & & x \nameeq \quotep{P} \\
%       \dropn{x} & & otherwise \\
%     \end{array}
%   \right. 
  (\dropn{x})  \psubstp{Q}{P}       
  := 
  \left\{ 
    \begin{array}{ccc} 
      Q & & x \nameeq \quotep{P} \\
      \dropn{x} & & otherwise \\
    \end{array}
  \right.
\end{mathpar}
 

where

\begin{eqnarray}
  (x)\id{\{} \lpquote Q \rpquote / \lpquote P \rpquote \id{\}}            = 
  \left\{ 
    \begin{array}{ccc}
      \lpquote Q \rpquote & & x \nameeq \lpquote P \rpquote \\
      x & & otherwise \\
    \end{array}
  \right. \nonumber
\end{eqnarray}

and $z$ is chosen distinct from $\quotep{P}$, $\quotep{Q}$, the free
names in $Q$, and all the names in $R$. Our $\alpha$-equivalence will
be built in the standard way from this substitution.

\begin{remark}\label{rem:no_self_referential_names}
  One consequence of these definitions is that $\forall P. \quotep{P}
  \not\in \freenames{P}$.
\end{remark}

\subsection{ Dynamic quote: an example }

Anticipating something of what's to come, consider applying the
substitution, $\widehat{\id{\{}u / z \id{\}}}$, to the following pair
of processes, $\lift{w}{y!(z)}$ and $w[ \lpquote y!(z) \rpquote ]$.

\begin{eqnarray}
	\lift{w}{y!(z)}\widehat{\id{\{}u / z \id{\}}}
		& = &
		\lift{w}{y!(u)} \nonumber\\
	w[ \lpquote y!(z) \rpquote ] \widehat{ \id{\{}u / z \id{\}} }
		& = &
		w[ \lpquote y!(z) \rpquote ] \nonumber
\end{eqnarray}

Because the body of the process between quotes is impervious to
substitution, we get radically different answers. In fact, by
examining the first process in an input context,
e.g. $x?(z).\lift{w}{y!(z)}$, we see that the process under the lift
operator may be shaped by prefixed inputs binding a name inside it. In
this sense, the lift operator will be seen as a way to dynamically
construct processes before reifying them as names.

Finally equipped with these standard features we can present the
dynamics of the calculus.

\subsubsection{Operational semantics} 

Finally, we introduce the computational dynamics. What marks these
algebras as distinct from other more traditionally studied algebraic
structures, e.g. vector spaces or polynomial rings, is the manner in
which dynamics is captured. In traditional structures, dynamics is typically
expressed through morphisms between such structures, as in linear maps
between vector spaces or morphisms between rings. In algebras
associated with the semantics of computation, the dynamics is
expressed as part of the algebraic structure itself, through a
reduction reduction relation typically denoted by $\red$. Below, we
give a recursive presentation of this relation for the calculus used
in the encoding.

$\red \subseteq \pi \times \pi$
$\red : \pi \to \mathcal{P}(\pi)$

\begin{mathpar}
  \inferrule* [lab=Comm] { \textsf{match}( x_{src}, x_{trgt} ) } { x_{trgt}?(y)P \; | \; x_{src}!\langle {Q} \rangle \red P\{\quotep{Q}/y}\} }
  \and \\
  \inferrule* [lab=Par] {{P} \red {P}'} {{{P} | {Q}} \red {{P}' | {Q}}}
  \and
  \inferrule* [lab=Equiv]{{{P} \scong {P}'} \andalso {{P}' \red {Q}'} \andalso {{Q}' \scong {Q}}}{{P} \red {Q}}
\end{mathpar}

\begin{eqnarray*}
  match_{\equiv} (\quotep{P},\quotep{Q}) & := & P \equiv Q \\
  match_{\dagger}(\quotep{P},\quotep{Q}) & := & \forall R. P|Q \red^{*} R => R \red^{*} 0 \\
  match_{K}(\quotep{P},\quotep{Q}) & := & K \mbox{ for some context } K
\end{eqnarray*}

$u?(x)P | u!\langle Q \rangle \red P\{\quotep{Q}/x\}$

%We write $\wred$ for $\red^*$, and $P\red$ if $\exists Q $ such that $ P \red Q$.
We write $P\red$ if $\exists Q $ such that $ P \red Q$ and $P\not\red$, otherwise.

\section{Replication}

As mentioned before, it is known that replication (and hence
recursion) can be implemented in a higher-order process algebra
\cite{SangiorgiWalker}. As our first example of calculation with the
machinery thus far presented we give the construction explicitly in
the {\rhoc}.

\begin{eqnarray}
	D_{x} & := & \prefix{x}{y}{(\binpar{\outputp{x}{y}}{@{y}})} \nonumber\\
	\bangp_{x}{P} & := & \binpar{{x}!\langle{\binpar{D_{x}}{P}}\rangle}{D_{x}} \nonumber
\end{eqnarray}

\begin{eqnarray}
	\bangp_{x}{P} & & \nonumber\\
	=
	& {x}!\langle{(\prefix{x}{y}{(\outputp{x}{y} | @{y})) | P}}\rangle 
	      | \prefix{x}{y}{(\outputp{x}{y} | @{y})} & \nonumber\\
	\red
	& (\outputp{x}{y} | @{y})\substn{\quotep{(\prefix{x}{y}{(@{y} | \outputp{x}{y})) | P}}}{y} & \nonumber\\
	=
	& \outputp{x}{\quotep{(\prefix{x}{y}{(\outputp{x}{y} | @{y})) | P}}}
	  | {(\prefix{x}{y}{(\outputp{x}{y} | @{y})) | P}} & \nonumber\\
	\red
	& \ldots & \nonumber\\
	\red^*
	& P | P | \ldots & \nonumber
\end{eqnarray}

Of course, this encoding, as an implementation, runs away, unfolding
$\bangp{P}$ eagerly. A lazier and more implementable replication
operator, restricted to input-guarded processes, may be obtained as follows.

\begin{eqnarray}
\bangp{\prefix{u}{v}{P}} 
	:= 
	\binpar{\lift{x}{\prefix{u}{v}{(\binpar{D(x)}{P})}}}{D(x)} \nonumber
\end{eqnarray}

\begin{remark}
  Note that the lazier definition still does not deal with summation
  or mixed summation (i.e. sums over input and output). The reader is
  invited to construct definitions of replication that deal with these
  features. 

  Further, the definitions are parameterized in a name, $x$. Can you,
  gentle reader, make a definition that eliminates this parameter and
  guarantees no accidental interaction between the replication
  machinery and the process being replicated -- i.e. no accidental
  sharing of names used by the process to get its work done and the
  name(s) used by the replication to effect copying. This latter
  revision of the definition of replication is crucial to obtaining
  the expected identity $!!P \sim !P$.
\end{remark}

\begin{remark}\label{rem:paradoxical_combinator}
  The reader familiar with the lambda calculus will have noticed the
  similarity between $D$ and the paradoxical combinator.

  [Ed. note: the existence of this seems to suggest we have to be more
  restrictive on the set of processes and names we admit if we are to
  support no-cloning.]
\end{remark}

\subsubsection{Bisimulation}

The computational dynamics gives rise to another kind of equivalence,
the equivalence of computational behavior. As previously mentioned
this is typically captured \emph{via} some form of bisimulation.

% The notion we use in this paper is weak barbed bisimulation
% \cite{milner91polyadicpi}.

The notion we use in this paper is derived from weak barbed
bisimulation \cite{milner91polyadicpi}. 

\begin{definition}
An \emph{observation relation}, $\downarrow_{\mathcal N}$, over a set
of names, $\mathcal N$, is the smallest relation satisfying the rules
below.

\infrule[Out-barb]{y \in {\mathcal N}, \; x \nameeq y}
		  {\outputp{x}{v} \downarrow_{\mathcal N} x}
\infrule[Par-barb]{\mbox{$P\downarrow_{\mathcal N} x$ or $Q\downarrow_{\mathcal N} x$}}
		  {\binpar{P}{Q} \downarrow_{\mathcal N} x}

We write $P \Downarrow_{\mathcal N} x$ if there is $Q$ such that 
$P \wred Q$ and $Q \downarrow_{\mathcal N} x$.
\end{definition}

\begin{definition}
%\label{def.bbisim}
An  ${\mathcal N}$-\emph{barbed bisimulation} over a set of names, ${\mathcal N}$, is a symmetric binary relation 
${\mathcal S}_{\mathcal N}$ between agents such that $P\rel{S}_{\mathcal N}Q$ implies:
\begin{enumerate}
\item If $P \red P'$ then $Q \wred Q'$ and $P'\rel{S}_{\mathcal N} Q'$.
\item If $P\downarrow_{\mathcal N} x$, then $Q\Downarrow_{\mathcal N} x$.
\end{enumerate}
$P$ is ${\mathcal N}$-barbed bisimilar to $Q$, written
$P \wbbisim_{\mathcal N} Q$, if $P \rel{S}_{\mathcal N} Q$ for some ${\mathcal N}$-barbed bisimulation ${\mathcal S}_{\mathcal N}$.
\end{definition}

$\mathcal{R} \subseteq \pi \times \pi$

$P \mathcal{R} Q => \forall P'. P \red P' \Rightarrow \exists Q'. Q \red Q', P' \mathcal{R} Q'$

$P \vdash x \Rightarrow Q \vdash x$

\begin{mathpar}
  \inferrule*[lab=Out-barb]{x \nameeq y}{{y}!\langle{Q}\rangle \vdash x}
  \and
  \inferrule*[lab=Par-barb]{\mbox{$P\vdash x$ or $Q\vdash x$}}{\binpar{P}{Q} \vdash x}
\end{mathpar}

\subsubsection{Contexts}

One of the principle advantages of computational calculi like the
$\pi$-calculus is a well-defined notion of context,
contextual-equivalence and a correlation between
contextual-equivalence and notions of bisimulation. The notion of
context allows the decomposition of a process into (sub-)process and
its syntactic environment, its context. Thus, a context may be
thought of as a process with a ``hole'' (written $\Box$) in it. The
application of a context $M$ to a process $P$, written $M[P]$, is
tantamount to filling the hole in $M$ with $P$. In this paper we do
not need the full weight of this theory, but do make use of the notion
of context in the proof the main theorem. 

\begin{mathpar}
  \inferrule* [lab=summation] {} {{M_{M},M_{N}} \bc \Box \;|\; x.M_{A} \;|\; M_{M}+M_{N}}
  \and
  \inferrule* [lab=agent] {} {{M_{A}} \bc (\vec{x})M_{P} \;| \; \clift{P_0,\ldots,M_{P},\ldots,P_N}}
  \and \\
  \inferrule* [lab=process] {} {{M_{P}} \bc M_{N} \;| \;P|M_{P} }
\end{mathpar} 

\begin{mathpar}
  \inferrule* [lab=sychronization] {} {M_{N} \bc \Box \;|\; x?M_{F} \;|\; x!M_{C}}
  \and
  \inferrule* [lab=abstraction] {} {{M_{F}} \bc (x)M_{P} }
  \and
  \inferrule* [lab=concretion] {} {{M_{C}} \bc \langle M_{P} \rangle }
  \and \\
  \inferrule* [lab=process] {} {{M_{P}} \bc M_{N} \;| \;P|M_{P} }
\end{mathpar}

\begin{definition}[contextual application] Given a context $M$, and
  process $P$, we define the \emph{contextual application}, $M[P] :=
  M\{P/\Box\}$. That is, the contextual application of M to P is the
  substitution of $P$ for $\Box$ in $M$.
\end{definition}

$\meaningof{-} : L \to \mathcal{P}(\pi)$

\begin{mathpar}
  \inferrule* [lab=collection] {} {\meaningof{true} = \pi, \and \meaningof{~E} = \pi \setminus \meaningof{E}, \and \meaningof{E_{1} \& E_{2}} = \meaningof{E_{1}} \cap \meaningof{E_{2}}}
\end{mathpar}

\begin{mathpar}
  \inferrule* [lab=structure] {} {\meaningof{0} = \{ P \in \pi | P \equiv 0 \}, \and \\ \meaningof{E_1 | E_2} = \{ P \in \pi | P \equiv P_{1} | P_{2}, P_{1} \in \meaningof{E_{1}}, P_{2} \in \meaningof{E_2}\} }
\end{mathpar}

\begin{mathpar}
 \inferrule* [lab=behavior] {} {\meaningof{\langle a?b \rangle E} = \{ P \in \pi | P \equiv Q | u?(y)P', \\ \and \\\\ \and \\ \;\;\; u \in \meaningof{a}, \forall z.P'\{z/y\} \in \meaningof{E\{z/b\}}\}, \and \\ \meaningof{a!E} = \{ P \in \pi | P \equiv Q | x!\langle P' \rangle, x \in \meaningof{a} P' \in \meaningof{E}\} }
\end{mathpar}

\begin{mathpar}
 \inferrule* [lab=nominal] {} {\meaningof{\quotep{E}} = \{ \quotep{P} \in \quotep{\pi} | P \in \meaningof{E} \}, \and \meaningof{\quotep{P}} = \{ \quotep{Q} \in \quotep{\pi} | P \equiv Q \} \and \\ \meaningof{@\quotep{E}} = \{ P \in \pi | P \equiv @x, x \in \meaningof{E} \}}
\end{mathpar}

\begin{eqnarray*}
  \\
  \meaningof{-} : TS \to ST
\end{eqnarray*}

\begin{eqnarray*}
  \\
  L : TS \to ST
\end{eqnarray*}

\begin{eqnarray*}
  \\
  P \models E \iff P \in \meaningof{E}
\end{eqnarray*}

\begin{eqnarray*}
  P \approx_{L} Q \iff \forall E \in L. P \models E \iff Q \models E
\end{eqnarray*}

\begin{eqnarray*}
  P \approx_{K} Q
\end{eqnarray*}

\begin{eqnarray*}
  P \approx Q
\end{eqnarray*}

$\approx_{K} = \approx = \approx_{L}$

\subsubsection{Contextual duality}

Note that contexts extend the quotation operation to a family of
operations from processes to names. Given a context, $M$, we can
define a \emph{nominal context}, $\quotep{M}$ by $\quotep{M}[P] :=
\quotep{M[P]}$. To foreshadow what is to come we observe that these
operations enjoy a duality with processes very much like the duality
between vectors and maps from vectors to scalars.

Further, because the calculus is essentially higher-order, we have a
correspondence between contexts and processes. More specifically,
given a name $x$ and a context $M$ we can construct $M^{*}_{x}$ such
that 

\begin{mathpar}
  M^{*}_{x} | \lift{x}{P} \red M[P]
\end{mathpar}

namely,

\begin{mathpar}
  M^{*}_{x} := x?(u).M[\dropn{u}]
\end{mathpar}

The dependence of $M^{*}_{x}$ on a name makes it an abstraction, 

\begin{mathpar}
  M^{*} := (x)x?(u).M[\dropn{u}]
\end{mathpar}

\subsection{Additional notation}

It will sometimes be convenient to denote the process a name
quotes. We already have the notation $x = \quotep{P}$, but it will be
convenient to introduce an alternate notation, $\procn{x}$, when we
want to emphasize the connection to the use of the name. Note that, by
virtue of name equivalence, $\quotep{\procn{x}} \nameeq x$; so, the
notation is consistent with previous definitions.

Further, because names have structure it is possible to effect
substitutions on the basis of that structure. This means we need to
upgrade our notation for substitutions, which we accomplish by
adapting comprehension notation. Thus,

\begin{mathpar}
  P\{ y / x : x \in S \}
\end{mathpar}

is interpreted to mean the process derived from P by replacing (in a
capture-avoiding manner) each occurrence of $x$ in $S$ by $y$. For example,

\begin{mathpar}
  P\{ \quotep{\procn{x}|\procn{x}} / x : x \in \freenames{P} \}
\end{mathpar}

will replace each (occurrence) of a free name $x$ in $P$ by
$\quotep{\procn{x}|\procn{x}}$.

Also, we will avail ourselves of the notation $x^{L}$ and $x^{R}$ to
denote injections of a name into disjoint copies of the name
space. There are numerous ways to accomplish this. One example can be
found in \cite{MeredithR05}. This notation overloads to vectors of
names: $\vec{x}^{\pi} := (x_{i}^{\pi} \; : \; 0 \leq i < |\vec{x}| )$ where $\pi \in \{L,R\}$.

We also use $P^{\Box} := P|\Box$.

In \cite{MeredithR05} an interpretation of the new operator is
given. It turns out that there are several possible interpretations
all enjoying the requisite algebraic properties of the operator (see
\cite{milner91polyadicpi}). We will therefore make liberal use of
$(\nu\; \vec{x})P$.

% subsection the_syntax_and_semantics_of_the_notation_system (end)   

\input{qm2pi.qmops} 

\input{qm2pi.sterngerlach} 

\input{qm2pi.metric} 

% section concurrent_process_calculi (end)

%\input{qm2pi.proofsketch}

% section proof sketch (end)

%\input{qm2pi.slviaknots} 

% section spatial logic via knots (end)

\input{qm2pi.conclusion}

% section conclusion (end)

%\input{qm2pi.dtcodes} 

% section wiring algorithm (end)

\input{qm2pi.ack} 

% section acknowledgments (end)

\newpage


\bibliographystyle{plain}   
\bibliography{../../biblios/main.bib}

\input{qm2pi.rhodetails}

\end{document}

 

%\ifpdf
%\usepackage[pdftex]{graphicx}
%\else
%\usepackage{graphicx}
%\fi

 % \ifpdf
%  \usepackage{pdfsync}
%  \if


%\title{Brief Article}
%\author{David F. Snyder}
%\author{L.G. Meredith}

%\address{Dept. of Math., Texas State University--San Marcos, San Marcos, TX 78666}
       
\pagestyle{empty}


\begin{document}

\lstset{language=[Objective]Caml,frame=shadowbox}

\documentclass[12pt]{llncs}
%\documentclass{jktr}

\usepackage[pdftex]{hyperref}                   
\usepackage {listings}
\usepackage {mathpartir}
\usepackage{bcprules}
%\usepackage{listings}
                       
\usepackage{graphicx} 
%\usepackage[margins=2.5cm,nohead,nofoot]{geometry}
%\usepackage{geometry}
\usepackage{amsfonts}
\usepackage{amstext}
\usepackage{latexsym}
\usepackage{amssymb}
\usepackage{color}


%\include{myPreamble}
\include{qm2pi.local} 

%\ifpdf
%\usepackage[pdftex]{graphicx}
%\else
%\usepackage{graphicx}
%\fi

 % \ifpdf
%  \usepackage{pdfsync}
%  \if


%\title{Brief Article}
%\author{David F. Snyder}
%\author{L.G. Meredith}

%\address{Dept. of Math., Texas State University--San Marcos, San Marcos, TX 78666}
       
\pagestyle{empty}


\begin{document}

\lstset{language=[Objective]Caml,frame=shadowbox}

\input{qm2pi.front}

% section front matter (end)

\input{qm2pi.intro} 
 
% section introduction (end)

% \input{qm2pi.knotations} 

% section notation (end)

\input{qm2pi.process.calculi} 

% section concurrent_process_calculi_and_spatial_logics_ (end)
    
%\input{qm2pi.knots2pi} 

%\input{qm2pi.trefoil} 

%\input{qm2pi.mainthm} 

% subsection basic_interpretation (end)

%\input{qm2pi.rho.presentation} 
\subsection{The syntax and semantics of the notation system}\label{sub:the_syntax_and_semantics_of_the_notation_system} % (fold)

We now summarize a technical presentation of the calculus that
embodies our theory of dynamics. The typical presentation of such a
calculus follows the style of giving generators and relations on
them. The grammar, below, describing term constructors, freely
generates the set of processes, $\Proc$. This set is then quotiented
by a relation known as structural congruence and it is over this set
that the notion of dynamics is expressed. This presentation is
essentially that of \cite{MeredithR05} with the addition of
polyadicity and summation. For readability we have relegated some of
the technical subtleties to an appendix.

\subsubsection{Process grammar}\label{subsub:process_grammar}

\begin{mathpar}
  \inferrule* [lab=synchronization] {} {{M} \bc \pzero \;|\; x?F \;|\; x!C }
  \and
  \inferrule* [lab=abstraction] {} {{F} \bc (x)P}
  \and
  \inferrule* [lab=concretion] {} {{C} \bc \langle Q \rangle}
  \and
  \inferrule* [lab=process] {} {{P,Q} \bc M \;| \;P|Q \;|\; @{x}}
  \and
  \inferrule* [lab=name] {} {{x} \bc \quotep{P}}
\end{mathpar} 

Note that $\vec{x}$ (resp. $\vec{P}$) denotes a vector of names
(resp. processes) of length $|\vec{x}|$ (resp. $|\vec{P}|$). We adopt
the following useful abbreviations.

\begin{mathpar}
   x?(\vec{y}).P := x.(\vec{y})P \and  x\clift{\vec{P}} := x.\clift{\vec{P}}
   \and x!(y) := \lift{x}{\dropn{y}}
   \and \Pi_{i=0}^{n-1}P_i := P_0 | \ldots | P_{n-1}
\end{mathpar}

\subsubsection{Structural congruence}

\paragraph{Free and bound names and alpha-equivalence.} At the
core of structural equivalence is alpha-equivalence which identifies
process that are the same up to a change of variable. Formally, we
recognize the distinction between free and bound names. The free names
of a process, $\freenames{P}$, may be calculated recursively as
follows:

\begin{mathpar}
\freenames{\pzero} := \emptyset
  \and \\
  \freenames{x?(y).P} := \{ x \} \cup (\freenames{P} \setminus \{ y \})
  \and 
  \freenames{x!\langle P \rangle} := \{ x \} \cup \{ P \} 
  \and \\
  \freenames{P|Q} := \freenames{P} \cup \freenames{Q}
  \and \\
  \freenames{@{x}} := \{ x \}
\end{mathpar}

$\pi$
$\quotep{\pi}$

$\freenames{-} : \pi \to \mathcal{P}(\quotep{\pi})$

\begin{eqnarray*}
  \freenames{\pzero} & := & \emptyset \\
  \freenames{x?(y).P} & := & \{ x \} \cup (\freenames{P} \setminus \{ y \}) \\
  \freenames{x!\langle P \rangle} & := & \{ x \} \cup \{ P \} \\
  \freenames{P|Q} & := & \freenames{P} \cup \freenames{Q} \\
  \freenames{\dropn{x}} & := & \{ x \}
\end{eqnarray*}

The bound names of a process, $\boundnames{P}$, are those names occurring in $P$
that are not free. For example, in $x?(y).0$, the name $x$ is free, while $y$ is bound.

\begin{mathpar}
  \inferrule* [lab=monoidal-laws] {} { P|Q \equiv Q|P \and P|0 \equiv P \and P|(Q|R) \equiv (P|Q)|R }
\end{mathpar}

\begin{mathpar}
  \inferrule* [lab=alpha-equivalence] {} { (x)P \equiv (y)P\{y/x\} \and y \not\in \freenames{P} }
\end{mathpar}

\begin{definition}
Then two processes, $P,Q$, are alpha-equivalent if $P = Q\{\vec{y}/\vec{x}\}$ for
some $\vec{x} \in \boundnames{Q},\vec{y} \in \boundnames{P}$, where $Q\{\vec{y}/\vec{x}\}$
denotes the capture-avoiding substitution of $\vec{y}$ for $\vec{x}$ in $Q$.
\end{definition}

\begin{definition}
  The {\em structural congruence} \cite{SangiorgiWalker} , $\equiv$,
  between processes is the least congruence containing
  alpha-equivalence, satisfying the abelian monoid laws
  (associativity, commutativity and $\pzero$ as identity) for parallel
  composition $|$ and for summation $+$.
\end{definition}

\subsection{Name equivalence}

We take name equivalence, written $\nameeq$, to be the smallest
equivalence relation generated by the following rules.

\begin{mathpar}
\inferrule*[lab=Quote-drop]
{ }
{ \quotep{@{x}} \nameeq x }

\inferrule*[lab=Struct-equiv]
{ P \scong Q }
{ \quotep{P} \nameeq \quotep{Q} }
\end{mathpar}

The astute reader will have noticed that the mutual recursion of names
and processes imposes a mutual recursion on alpha-equivalence and
structural equivalence via name-equivalence. Fortunately, all of this
works out pleasantly and we may calculate in the natural way, free of
concern. The reader interested in the details is referred to the
appendix \ref{appendix:rho_details}.

\subsection{Substitution}

We use $\Proc$ for the set of processes, $\QProc$ for the set of
names, and $\id{\{}\vec{y} / \vec{x} \id{\}}$ to denote partial maps,
$s : \QProc \rightarrow \QProc$. A map, $s$ lifts, uniquely, to a map
on process terms, $\widehat{s} : \Proc \rightarrow \Proc$ by the
following equations.

\begin{mathpar}
  (0) \psubstp{Q}{P} := 0 \\
  (R \juxtap S) \psubstp{Q}{P}
  :=    
  (R)\psubstp{Q}{P} \juxtap (S) \psubstp{Q}{P} \\
  (x?(y).R) \psubstp{Q}{P}    
  :=    
  (x)\substp{Q}{P} (z)\concat( (R \psubstn{z}{y}) \psubstp{Q}{P} ) \\
  (\lift{x}{R}) \psubstp{Q}{P}  
  :=
  \lift{(x)\substp{Q}{P}}{ R \psubstp{Q}{P} } \\
%   (\dropn{x})  \psubstp{Q}{P}       
%   := 
%   \left\{ 
%     \begin{array}{ccc} 
%       \dropn{\quotep{Q}} & & x \nameeq \quotep{P} \\
%       \dropn{x} & & otherwise \\
%     \end{array}
%   \right. 
  (\dropn{x})  \psubstp{Q}{P}       
  := 
  \left\{ 
    \begin{array}{ccc} 
      Q & & x \nameeq \quotep{P} \\
      \dropn{x} & & otherwise \\
    \end{array}
  \right.
\end{mathpar}
 

where

\begin{eqnarray}
  (x)\id{\{} \lpquote Q \rpquote / \lpquote P \rpquote \id{\}}            = 
  \left\{ 
    \begin{array}{ccc}
      \lpquote Q \rpquote & & x \nameeq \lpquote P \rpquote \\
      x & & otherwise \\
    \end{array}
  \right. \nonumber
\end{eqnarray}

and $z$ is chosen distinct from $\quotep{P}$, $\quotep{Q}$, the free
names in $Q$, and all the names in $R$. Our $\alpha$-equivalence will
be built in the standard way from this substitution.

\begin{remark}\label{rem:no_self_referential_names}
  One consequence of these definitions is that $\forall P. \quotep{P}
  \not\in \freenames{P}$.
\end{remark}

\subsection{ Dynamic quote: an example }

Anticipating something of what's to come, consider applying the
substitution, $\widehat{\id{\{}u / z \id{\}}}$, to the following pair
of processes, $\lift{w}{y!(z)}$ and $w[ \lpquote y!(z) \rpquote ]$.

\begin{eqnarray}
	\lift{w}{y!(z)}\widehat{\id{\{}u / z \id{\}}}
		& = &
		\lift{w}{y!(u)} \nonumber\\
	w[ \lpquote y!(z) \rpquote ] \widehat{ \id{\{}u / z \id{\}} }
		& = &
		w[ \lpquote y!(z) \rpquote ] \nonumber
\end{eqnarray}

Because the body of the process between quotes is impervious to
substitution, we get radically different answers. In fact, by
examining the first process in an input context,
e.g. $x?(z).\lift{w}{y!(z)}$, we see that the process under the lift
operator may be shaped by prefixed inputs binding a name inside it. In
this sense, the lift operator will be seen as a way to dynamically
construct processes before reifying them as names.

Finally equipped with these standard features we can present the
dynamics of the calculus.

\subsubsection{Operational semantics} 

Finally, we introduce the computational dynamics. What marks these
algebras as distinct from other more traditionally studied algebraic
structures, e.g. vector spaces or polynomial rings, is the manner in
which dynamics is captured. In traditional structures, dynamics is typically
expressed through morphisms between such structures, as in linear maps
between vector spaces or morphisms between rings. In algebras
associated with the semantics of computation, the dynamics is
expressed as part of the algebraic structure itself, through a
reduction reduction relation typically denoted by $\red$. Below, we
give a recursive presentation of this relation for the calculus used
in the encoding.

$\red \subseteq \pi \times \pi$
$\red : \pi \to \mathcal{P}(\pi)$

\begin{mathpar}
  \inferrule* [lab=Comm] { \textsf{match}( x_{src}, x_{trgt} ) } { x_{trgt}?(y)P \; | \; x_{src}!\langle {Q} \rangle \red P\{\quotep{Q}/y}\} }
  \and \\
  \inferrule* [lab=Par] {{P} \red {P}'} {{{P} | {Q}} \red {{P}' | {Q}}}
  \and
  \inferrule* [lab=Equiv]{{{P} \scong {P}'} \andalso {{P}' \red {Q}'} \andalso {{Q}' \scong {Q}}}{{P} \red {Q}}
\end{mathpar}

\begin{eqnarray*}
  match_{\equiv} (\quotep{P},\quotep{Q}) & := & P \equiv Q \\
  match_{\dagger}(\quotep{P},\quotep{Q}) & := & \forall R. P|Q \red^{*} R => R \red^{*} 0 \\
  match_{K}(\quotep{P},\quotep{Q}) & := & K \mbox{ for some context } K
\end{eqnarray*}

$u?(x)P | u!\langle Q \rangle \red P\{\quotep{Q}/x\}$

%We write $\wred$ for $\red^*$, and $P\red$ if $\exists Q $ such that $ P \red Q$.
We write $P\red$ if $\exists Q $ such that $ P \red Q$ and $P\not\red$, otherwise.

\section{Replication}

As mentioned before, it is known that replication (and hence
recursion) can be implemented in a higher-order process algebra
\cite{SangiorgiWalker}. As our first example of calculation with the
machinery thus far presented we give the construction explicitly in
the {\rhoc}.

\begin{eqnarray}
	D_{x} & := & \prefix{x}{y}{(\binpar{\outputp{x}{y}}{@{y}})} \nonumber\\
	\bangp_{x}{P} & := & \binpar{{x}!\langle{\binpar{D_{x}}{P}}\rangle}{D_{x}} \nonumber
\end{eqnarray}

\begin{eqnarray}
	\bangp_{x}{P} & & \nonumber\\
	=
	& {x}!\langle{(\prefix{x}{y}{(\outputp{x}{y} | @{y})) | P}}\rangle 
	      | \prefix{x}{y}{(\outputp{x}{y} | @{y})} & \nonumber\\
	\red
	& (\outputp{x}{y} | @{y})\substn{\quotep{(\prefix{x}{y}{(@{y} | \outputp{x}{y})) | P}}}{y} & \nonumber\\
	=
	& \outputp{x}{\quotep{(\prefix{x}{y}{(\outputp{x}{y} | @{y})) | P}}}
	  | {(\prefix{x}{y}{(\outputp{x}{y} | @{y})) | P}} & \nonumber\\
	\red
	& \ldots & \nonumber\\
	\red^*
	& P | P | \ldots & \nonumber
\end{eqnarray}

Of course, this encoding, as an implementation, runs away, unfolding
$\bangp{P}$ eagerly. A lazier and more implementable replication
operator, restricted to input-guarded processes, may be obtained as follows.

\begin{eqnarray}
\bangp{\prefix{u}{v}{P}} 
	:= 
	\binpar{\lift{x}{\prefix{u}{v}{(\binpar{D(x)}{P})}}}{D(x)} \nonumber
\end{eqnarray}

\begin{remark}
  Note that the lazier definition still does not deal with summation
  or mixed summation (i.e. sums over input and output). The reader is
  invited to construct definitions of replication that deal with these
  features. 

  Further, the definitions are parameterized in a name, $x$. Can you,
  gentle reader, make a definition that eliminates this parameter and
  guarantees no accidental interaction between the replication
  machinery and the process being replicated -- i.e. no accidental
  sharing of names used by the process to get its work done and the
  name(s) used by the replication to effect copying. This latter
  revision of the definition of replication is crucial to obtaining
  the expected identity $!!P \sim !P$.
\end{remark}

\begin{remark}\label{rem:paradoxical_combinator}
  The reader familiar with the lambda calculus will have noticed the
  similarity between $D$ and the paradoxical combinator.

  [Ed. note: the existence of this seems to suggest we have to be more
  restrictive on the set of processes and names we admit if we are to
  support no-cloning.]
\end{remark}

\subsubsection{Bisimulation}

The computational dynamics gives rise to another kind of equivalence,
the equivalence of computational behavior. As previously mentioned
this is typically captured \emph{via} some form of bisimulation.

% The notion we use in this paper is weak barbed bisimulation
% \cite{milner91polyadicpi}.

The notion we use in this paper is derived from weak barbed
bisimulation \cite{milner91polyadicpi}. 

\begin{definition}
An \emph{observation relation}, $\downarrow_{\mathcal N}$, over a set
of names, $\mathcal N$, is the smallest relation satisfying the rules
below.

\infrule[Out-barb]{y \in {\mathcal N}, \; x \nameeq y}
		  {\outputp{x}{v} \downarrow_{\mathcal N} x}
\infrule[Par-barb]{\mbox{$P\downarrow_{\mathcal N} x$ or $Q\downarrow_{\mathcal N} x$}}
		  {\binpar{P}{Q} \downarrow_{\mathcal N} x}

We write $P \Downarrow_{\mathcal N} x$ if there is $Q$ such that 
$P \wred Q$ and $Q \downarrow_{\mathcal N} x$.
\end{definition}

\begin{definition}
%\label{def.bbisim}
An  ${\mathcal N}$-\emph{barbed bisimulation} over a set of names, ${\mathcal N}$, is a symmetric binary relation 
${\mathcal S}_{\mathcal N}$ between agents such that $P\rel{S}_{\mathcal N}Q$ implies:
\begin{enumerate}
\item If $P \red P'$ then $Q \wred Q'$ and $P'\rel{S}_{\mathcal N} Q'$.
\item If $P\downarrow_{\mathcal N} x$, then $Q\Downarrow_{\mathcal N} x$.
\end{enumerate}
$P$ is ${\mathcal N}$-barbed bisimilar to $Q$, written
$P \wbbisim_{\mathcal N} Q$, if $P \rel{S}_{\mathcal N} Q$ for some ${\mathcal N}$-barbed bisimulation ${\mathcal S}_{\mathcal N}$.
\end{definition}

$\mathcal{R} \subseteq \pi \times \pi$

$P \mathcal{R} Q => \forall P'. P \red P' \Rightarrow \exists Q'. Q \red Q', P' \mathcal{R} Q'$

$P \vdash x \Rightarrow Q \vdash x$

\begin{mathpar}
  \inferrule*[lab=Out-barb]{x \nameeq y}{{y}!\langle{Q}\rangle \vdash x}
  \and
  \inferrule*[lab=Par-barb]{\mbox{$P\vdash x$ or $Q\vdash x$}}{\binpar{P}{Q} \vdash x}
\end{mathpar}

\subsubsection{Contexts}

One of the principle advantages of computational calculi like the
$\pi$-calculus is a well-defined notion of context,
contextual-equivalence and a correlation between
contextual-equivalence and notions of bisimulation. The notion of
context allows the decomposition of a process into (sub-)process and
its syntactic environment, its context. Thus, a context may be
thought of as a process with a ``hole'' (written $\Box$) in it. The
application of a context $M$ to a process $P$, written $M[P]$, is
tantamount to filling the hole in $M$ with $P$. In this paper we do
not need the full weight of this theory, but do make use of the notion
of context in the proof the main theorem. 

\begin{mathpar}
  \inferrule* [lab=summation] {} {{M_{M},M_{N}} \bc \Box \;|\; x.M_{A} \;|\; M_{M}+M_{N}}
  \and
  \inferrule* [lab=agent] {} {{M_{A}} \bc (\vec{x})M_{P} \;| \; \clift{P_0,\ldots,M_{P},\ldots,P_N}}
  \and \\
  \inferrule* [lab=process] {} {{M_{P}} \bc M_{N} \;| \;P|M_{P} }
\end{mathpar} 

\begin{mathpar}
  \inferrule* [lab=sychronization] {} {M_{N} \bc \Box \;|\; x?M_{F} \;|\; x!M_{C}}
  \and
  \inferrule* [lab=abstraction] {} {{M_{F}} \bc (x)M_{P} }
  \and
  \inferrule* [lab=concretion] {} {{M_{C}} \bc \langle M_{P} \rangle }
  \and \\
  \inferrule* [lab=process] {} {{M_{P}} \bc M_{N} \;| \;P|M_{P} }
\end{mathpar}

\begin{definition}[contextual application] Given a context $M$, and
  process $P$, we define the \emph{contextual application}, $M[P] :=
  M\{P/\Box\}$. That is, the contextual application of M to P is the
  substitution of $P$ for $\Box$ in $M$.
\end{definition}

$\meaningof{-} : L \to \mathcal{P}(\pi)$

\begin{mathpar}
  \inferrule* [lab=collection] {} {\meaningof{true} = \pi, \and \meaningof{~E} = \pi \setminus \meaningof{E}, \and \meaningof{E_{1} \& E_{2}} = \meaningof{E_{1}} \cap \meaningof{E_{2}}}
\end{mathpar}

\begin{mathpar}
  \inferrule* [lab=structure] {} {\meaningof{0} = \{ P \in \pi | P \equiv 0 \}, \and \\ \meaningof{E_1 | E_2} = \{ P \in \pi | P \equiv P_{1} | P_{2}, P_{1} \in \meaningof{E_{1}}, P_{2} \in \meaningof{E_2}\} }
\end{mathpar}

\begin{mathpar}
 \inferrule* [lab=behavior] {} {\meaningof{\langle a?b \rangle E} = \{ P \in \pi | P \equiv Q | u?(y)P', \\ \and \\\\ \and \\ \;\;\; u \in \meaningof{a}, \forall z.P'\{z/y\} \in \meaningof{E\{z/b\}}\}, \and \\ \meaningof{a!E} = \{ P \in \pi | P \equiv Q | x!\langle P' \rangle, x \in \meaningof{a} P' \in \meaningof{E}\} }
\end{mathpar}

\begin{mathpar}
 \inferrule* [lab=nominal] {} {\meaningof{\quotep{E}} = \{ \quotep{P} \in \quotep{\pi} | P \in \meaningof{E} \}, \and \meaningof{\quotep{P}} = \{ \quotep{Q} \in \quotep{\pi} | P \equiv Q \} \and \\ \meaningof{@\quotep{E}} = \{ P \in \pi | P \equiv @x, x \in \meaningof{E} \}}
\end{mathpar}

\begin{eqnarray*}
  \\
  \meaningof{-} : TS \to ST
\end{eqnarray*}

\begin{eqnarray*}
  \\
  L : TS \to ST
\end{eqnarray*}

\begin{eqnarray*}
  \\
  P \models E \iff P \in \meaningof{E}
\end{eqnarray*}

\begin{eqnarray*}
  P \approx_{L} Q \iff \forall E \in L. P \models E \iff Q \models E
\end{eqnarray*}

\begin{eqnarray*}
  P \approx_{K} Q
\end{eqnarray*}

\begin{eqnarray*}
  P \approx Q
\end{eqnarray*}

$\approx_{K} = \approx = \approx_{L}$

\subsubsection{Contextual duality}

Note that contexts extend the quotation operation to a family of
operations from processes to names. Given a context, $M$, we can
define a \emph{nominal context}, $\quotep{M}$ by $\quotep{M}[P] :=
\quotep{M[P]}$. To foreshadow what is to come we observe that these
operations enjoy a duality with processes very much like the duality
between vectors and maps from vectors to scalars.

Further, because the calculus is essentially higher-order, we have a
correspondence between contexts and processes. More specifically,
given a name $x$ and a context $M$ we can construct $M^{*}_{x}$ such
that 

\begin{mathpar}
  M^{*}_{x} | \lift{x}{P} \red M[P]
\end{mathpar}

namely,

\begin{mathpar}
  M^{*}_{x} := x?(u).M[\dropn{u}]
\end{mathpar}

The dependence of $M^{*}_{x}$ on a name makes it an abstraction, 

\begin{mathpar}
  M^{*} := (x)x?(u).M[\dropn{u}]
\end{mathpar}

\subsection{Additional notation}

It will sometimes be convenient to denote the process a name
quotes. We already have the notation $x = \quotep{P}$, but it will be
convenient to introduce an alternate notation, $\procn{x}$, when we
want to emphasize the connection to the use of the name. Note that, by
virtue of name equivalence, $\quotep{\procn{x}} \nameeq x$; so, the
notation is consistent with previous definitions.

Further, because names have structure it is possible to effect
substitutions on the basis of that structure. This means we need to
upgrade our notation for substitutions, which we accomplish by
adapting comprehension notation. Thus,

\begin{mathpar}
  P\{ y / x : x \in S \}
\end{mathpar}

is interpreted to mean the process derived from P by replacing (in a
capture-avoiding manner) each occurrence of $x$ in $S$ by $y$. For example,

\begin{mathpar}
  P\{ \quotep{\procn{x}|\procn{x}} / x : x \in \freenames{P} \}
\end{mathpar}

will replace each (occurrence) of a free name $x$ in $P$ by
$\quotep{\procn{x}|\procn{x}}$.

Also, we will avail ourselves of the notation $x^{L}$ and $x^{R}$ to
denote injections of a name into disjoint copies of the name
space. There are numerous ways to accomplish this. One example can be
found in \cite{MeredithR05}. This notation overloads to vectors of
names: $\vec{x}^{\pi} := (x_{i}^{\pi} \; : \; 0 \leq i < |\vec{x}| )$ where $\pi \in \{L,R\}$.

We also use $P^{\Box} := P|\Box$.

In \cite{MeredithR05} an interpretation of the new operator is
given. It turns out that there are several possible interpretations
all enjoying the requisite algebraic properties of the operator (see
\cite{milner91polyadicpi}). We will therefore make liberal use of
$(\nu\; \vec{x})P$.

% subsection the_syntax_and_semantics_of_the_notation_system (end)   

\input{qm2pi.qmops} 

\input{qm2pi.sterngerlach} 

\input{qm2pi.metric} 

% section concurrent_process_calculi (end)

%\input{qm2pi.proofsketch}

% section proof sketch (end)

%\input{qm2pi.slviaknots} 

% section spatial logic via knots (end)

\input{qm2pi.conclusion}

% section conclusion (end)

%\input{qm2pi.dtcodes} 

% section wiring algorithm (end)

\input{qm2pi.ack} 

% section acknowledgments (end)

\newpage


\bibliographystyle{plain}   
\bibliography{../../biblios/main.bib}

\input{qm2pi.rhodetails}

\end{document}



% section front matter (end)

\section{Introduction}\label{sec:introduction} % (fold)
In this draft of the material i am going to have to dispense with the
usual writing conventions adopted in papers on these topics. i'm going
to have adopt whatever tone i need at the time i'm writing up the
calculations. Sometimes this may be very conversational; others it may
be the barest mathematical grunts; others still it may be that i have
lifted text from one of my other papers because the exposition of some
point was better said there. i hope that my readers are not unduly put
out by this decision. i'm not doing this to flout convention or be
rebellious. i find these calculations very technically challenging. To
keep everything going technically, something has to give; i have to
let go of some cognitive burden. So, the academic writing style --
with all of its trade-offs in terms of facilitating technical
communication -- is what i'm letting go of. Perhaps subsequent drafts
can be tightened and polished, but for now, i'm going to speak as if
we were sitting together in a coffee shop with a laptop, wifi and a
pad of paper and a pencil.

So, here's what i have to say. We -- you and i, comfortably ensconced
in our coffee shop and well-equipped with our tools -- can realize and
carry out the calculations of quantum mechanics over a very different
formal theory of dynamics, a formal theory of dynamics that
corresponds to a theory of concurrent computation with
\emph{reflection}. It has the advantage that the underlying theory is
already `quantized', but supports analogues all of the continuuous
operations. Strikingly, this underlying theory has recently been
connected with a notion of metric that we can show, by calculating
together, coincides with the metric induced by the inner product.

There are a lot of reasons why you might be interested in seeing
calculations of this form. Here's why i'm interested. For the past
several centuries there has been no competitor to the ``Newtonian''
account of dynamics. As a result the predominant share of accounts of
dynamical systems and situations have had to be formulated in terms of
the Newtonian machinery. i view this as an intellectually dangerous
position to occupy. Everything, despite it's intrinsic shape, turns
into a nail to be hit with this hammer. Recently, however, the theory
of computation has matured to the point where we have candidates for
theories of dynamics that offer very different perspective on
reasoning about dynamical systems and situations. Testing these
candidates against very successful accounts of dynamical situations,
like quantum mechanics, is going to give us some sense of how mature
they are and some measure of the quality of these accounts of
dynamics.

\subsection{Summary of contributions and outline of paper}

So, we're going to develop an interpretation of the operations of
quantum mechanics normally interpreted by Hilbert spaces and
operators. We're going to do this over a theory of computation. Note
that this is very different than the usual quantum computation program
which develops notions of computation over quantum mechanics. Rather,
we are developing a story that aligns with Wheeler's slogan: It from
Bit. To do this we will first provide an account of the theory of
computation at play here. Then we will dive into a calculation-driven
interpretation of the operations of quantum mechanics.

The reason we take this approach is that -- until very recently --
there hasn't been an axiomatic account of quantum mechanics. As a
result there has been no sharp delineation of the mathematical theory
supporting interpretation of the physical theory and the physical
theory, itself. So, ambient features of the maths are free to be
exploited (or supressed) without a real accounting of their physical
relevance. There is no sharp statement ``here's the physical theory''
qua \emph{theory} and ``here's the mathematical interpretation''
enabling a judgment of how faithful the interpretation is -- apart
from experimental observation. When there is an axiomatic account we
can judge how well a given mathematical formalism supports an
interpretation of the axioms, independent of
experimentation. Likewise, we can judge how well we have captured our
physical evidence and experience with our axiomatics, independent of
any specific mathematical implementation, with accidental detail that
may or may not have physical significance. 

In lieu of a fully fleshed out and vetted axiomatic account of quantum
mechanics, interpreting the operational notions in service of modeling
physical systems will have to suffice. In other words, we are not in
the business of providing a model of Hilbert spaces and operators. We
are in the business of providing a model of quantum mechanics because
we are motivated by testing our notions of dynamics against physical
theory; and, the predictive calculations of the physical theory must
serve as the best formulation -- shy of a fully fleshed out axiomatic
account -- of the physical theory itself (as they have for scientific
theories since time immemorial). Put another way, despite a
whole-hearted commitment to an It-from-Bit ontology, we are firmly
aligned with the shut-up-and-calculate camp as the best way to obtain
results either from the physical perspective or as a quality assurance
measure of our fledgling theory of dynamics.

In detail, we present a reflective process calculus. Then we develop
intuitive correspondences between the notions available in this
calculus and the usual physical notions supporting quantum mechanical
calculations. Thus, 

\begin{table}[htp]
  \center{
    \fbox{
      \begin{tabular}{c|c}
        quantum mechanics & process calculus \\
        \hline
        scalar & name \\
        state vector & process \\
        dual & contextual duals \\
        matrix & formal sums of process-context-dual pairs \\
        orthogonality & process annihilation \\
        inner product & execution-formula + quoting
      \end{tabular}
    }
  }
  \caption{QM - process calculi correspondences}
\end{table}

Then we tighten up these intuitions to operational definitions. We
employ the Dirac notation as the best proxy we can find for an
abstract syntax of the quantum mechanical notions. The definitions we
develop put us in contact with equational constraints coming from the
theory that we demonstrate the definitions and calculations satisfy.

This puts us in a position to shut up and calculate for the
Stern-Gerlach experimental set up, showing how these predictive
calculations become calculations on processes in our theory of a
reflective process calculus.

Penultimately, we demonstrate that the notion of metric coming from
the inner product coincides with the notion of metric available from
the theory of bisimulation. This demonstration gives us the right to
think of space as arising from behavior. Finally, we consider where we
might go from the new vantage point we have obtained.

% section introduction (end) 
 
% section introduction (end)

% \documentclass[12pt]{llncs}
%\documentclass{jktr}

\usepackage[pdftex]{hyperref}                   
\usepackage {listings}
\usepackage {mathpartir}
\usepackage{bcprules}
%\usepackage{listings}
                       
\usepackage{graphicx} 
%\usepackage[margins=2.5cm,nohead,nofoot]{geometry}
%\usepackage{geometry}
\usepackage{amsfonts}
\usepackage{amstext}
\usepackage{latexsym}
\usepackage{amssymb}
\usepackage{color}


%\include{myPreamble}
\include{qm2pi.local} 

%\ifpdf
%\usepackage[pdftex]{graphicx}
%\else
%\usepackage{graphicx}
%\fi

 % \ifpdf
%  \usepackage{pdfsync}
%  \if


%\title{Brief Article}
%\author{David F. Snyder}
%\author{L.G. Meredith}

%\address{Dept. of Math., Texas State University--San Marcos, San Marcos, TX 78666}
       
\pagestyle{empty}


\begin{document}

\lstset{language=[Objective]Caml,frame=shadowbox}

\input{qm2pi.front}

% section front matter (end)

\input{qm2pi.intro} 
 
% section introduction (end)

% \input{qm2pi.knotations} 

% section notation (end)

\input{qm2pi.process.calculi} 

% section concurrent_process_calculi_and_spatial_logics_ (end)
    
%\input{qm2pi.knots2pi} 

%\input{qm2pi.trefoil} 

%\input{qm2pi.mainthm} 

% subsection basic_interpretation (end)

%\input{qm2pi.rho.presentation} 
\subsection{The syntax and semantics of the notation system}\label{sub:the_syntax_and_semantics_of_the_notation_system} % (fold)

We now summarize a technical presentation of the calculus that
embodies our theory of dynamics. The typical presentation of such a
calculus follows the style of giving generators and relations on
them. The grammar, below, describing term constructors, freely
generates the set of processes, $\Proc$. This set is then quotiented
by a relation known as structural congruence and it is over this set
that the notion of dynamics is expressed. This presentation is
essentially that of \cite{MeredithR05} with the addition of
polyadicity and summation. For readability we have relegated some of
the technical subtleties to an appendix.

\subsubsection{Process grammar}\label{subsub:process_grammar}

\begin{mathpar}
  \inferrule* [lab=synchronization] {} {{M} \bc \pzero \;|\; x?F \;|\; x!C }
  \and
  \inferrule* [lab=abstraction] {} {{F} \bc (x)P}
  \and
  \inferrule* [lab=concretion] {} {{C} \bc \langle Q \rangle}
  \and
  \inferrule* [lab=process] {} {{P,Q} \bc M \;| \;P|Q \;|\; @{x}}
  \and
  \inferrule* [lab=name] {} {{x} \bc \quotep{P}}
\end{mathpar} 

Note that $\vec{x}$ (resp. $\vec{P}$) denotes a vector of names
(resp. processes) of length $|\vec{x}|$ (resp. $|\vec{P}|$). We adopt
the following useful abbreviations.

\begin{mathpar}
   x?(\vec{y}).P := x.(\vec{y})P \and  x\clift{\vec{P}} := x.\clift{\vec{P}}
   \and x!(y) := \lift{x}{\dropn{y}}
   \and \Pi_{i=0}^{n-1}P_i := P_0 | \ldots | P_{n-1}
\end{mathpar}

\subsubsection{Structural congruence}

\paragraph{Free and bound names and alpha-equivalence.} At the
core of structural equivalence is alpha-equivalence which identifies
process that are the same up to a change of variable. Formally, we
recognize the distinction between free and bound names. The free names
of a process, $\freenames{P}$, may be calculated recursively as
follows:

\begin{mathpar}
\freenames{\pzero} := \emptyset
  \and \\
  \freenames{x?(y).P} := \{ x \} \cup (\freenames{P} \setminus \{ y \})
  \and 
  \freenames{x!\langle P \rangle} := \{ x \} \cup \{ P \} 
  \and \\
  \freenames{P|Q} := \freenames{P} \cup \freenames{Q}
  \and \\
  \freenames{@{x}} := \{ x \}
\end{mathpar}

$\pi$
$\quotep{\pi}$

$\freenames{-} : \pi \to \mathcal{P}(\quotep{\pi})$

\begin{eqnarray*}
  \freenames{\pzero} & := & \emptyset \\
  \freenames{x?(y).P} & := & \{ x \} \cup (\freenames{P} \setminus \{ y \}) \\
  \freenames{x!\langle P \rangle} & := & \{ x \} \cup \{ P \} \\
  \freenames{P|Q} & := & \freenames{P} \cup \freenames{Q} \\
  \freenames{\dropn{x}} & := & \{ x \}
\end{eqnarray*}

The bound names of a process, $\boundnames{P}$, are those names occurring in $P$
that are not free. For example, in $x?(y).0$, the name $x$ is free, while $y$ is bound.

\begin{mathpar}
  \inferrule* [lab=monoidal-laws] {} { P|Q \equiv Q|P \and P|0 \equiv P \and P|(Q|R) \equiv (P|Q)|R }
\end{mathpar}

\begin{mathpar}
  \inferrule* [lab=alpha-equivalence] {} { (x)P \equiv (y)P\{y/x\} \and y \not\in \freenames{P} }
\end{mathpar}

\begin{definition}
Then two processes, $P,Q$, are alpha-equivalent if $P = Q\{\vec{y}/\vec{x}\}$ for
some $\vec{x} \in \boundnames{Q},\vec{y} \in \boundnames{P}$, where $Q\{\vec{y}/\vec{x}\}$
denotes the capture-avoiding substitution of $\vec{y}$ for $\vec{x}$ in $Q$.
\end{definition}

\begin{definition}
  The {\em structural congruence} \cite{SangiorgiWalker} , $\equiv$,
  between processes is the least congruence containing
  alpha-equivalence, satisfying the abelian monoid laws
  (associativity, commutativity and $\pzero$ as identity) for parallel
  composition $|$ and for summation $+$.
\end{definition}

\subsection{Name equivalence}

We take name equivalence, written $\nameeq$, to be the smallest
equivalence relation generated by the following rules.

\begin{mathpar}
\inferrule*[lab=Quote-drop]
{ }
{ \quotep{@{x}} \nameeq x }

\inferrule*[lab=Struct-equiv]
{ P \scong Q }
{ \quotep{P} \nameeq \quotep{Q} }
\end{mathpar}

The astute reader will have noticed that the mutual recursion of names
and processes imposes a mutual recursion on alpha-equivalence and
structural equivalence via name-equivalence. Fortunately, all of this
works out pleasantly and we may calculate in the natural way, free of
concern. The reader interested in the details is referred to the
appendix \ref{appendix:rho_details}.

\subsection{Substitution}

We use $\Proc$ for the set of processes, $\QProc$ for the set of
names, and $\id{\{}\vec{y} / \vec{x} \id{\}}$ to denote partial maps,
$s : \QProc \rightarrow \QProc$. A map, $s$ lifts, uniquely, to a map
on process terms, $\widehat{s} : \Proc \rightarrow \Proc$ by the
following equations.

\begin{mathpar}
  (0) \psubstp{Q}{P} := 0 \\
  (R \juxtap S) \psubstp{Q}{P}
  :=    
  (R)\psubstp{Q}{P} \juxtap (S) \psubstp{Q}{P} \\
  (x?(y).R) \psubstp{Q}{P}    
  :=    
  (x)\substp{Q}{P} (z)\concat( (R \psubstn{z}{y}) \psubstp{Q}{P} ) \\
  (\lift{x}{R}) \psubstp{Q}{P}  
  :=
  \lift{(x)\substp{Q}{P}}{ R \psubstp{Q}{P} } \\
%   (\dropn{x})  \psubstp{Q}{P}       
%   := 
%   \left\{ 
%     \begin{array}{ccc} 
%       \dropn{\quotep{Q}} & & x \nameeq \quotep{P} \\
%       \dropn{x} & & otherwise \\
%     \end{array}
%   \right. 
  (\dropn{x})  \psubstp{Q}{P}       
  := 
  \left\{ 
    \begin{array}{ccc} 
      Q & & x \nameeq \quotep{P} \\
      \dropn{x} & & otherwise \\
    \end{array}
  \right.
\end{mathpar}
 

where

\begin{eqnarray}
  (x)\id{\{} \lpquote Q \rpquote / \lpquote P \rpquote \id{\}}            = 
  \left\{ 
    \begin{array}{ccc}
      \lpquote Q \rpquote & & x \nameeq \lpquote P \rpquote \\
      x & & otherwise \\
    \end{array}
  \right. \nonumber
\end{eqnarray}

and $z$ is chosen distinct from $\quotep{P}$, $\quotep{Q}$, the free
names in $Q$, and all the names in $R$. Our $\alpha$-equivalence will
be built in the standard way from this substitution.

\begin{remark}\label{rem:no_self_referential_names}
  One consequence of these definitions is that $\forall P. \quotep{P}
  \not\in \freenames{P}$.
\end{remark}

\subsection{ Dynamic quote: an example }

Anticipating something of what's to come, consider applying the
substitution, $\widehat{\id{\{}u / z \id{\}}}$, to the following pair
of processes, $\lift{w}{y!(z)}$ and $w[ \lpquote y!(z) \rpquote ]$.

\begin{eqnarray}
	\lift{w}{y!(z)}\widehat{\id{\{}u / z \id{\}}}
		& = &
		\lift{w}{y!(u)} \nonumber\\
	w[ \lpquote y!(z) \rpquote ] \widehat{ \id{\{}u / z \id{\}} }
		& = &
		w[ \lpquote y!(z) \rpquote ] \nonumber
\end{eqnarray}

Because the body of the process between quotes is impervious to
substitution, we get radically different answers. In fact, by
examining the first process in an input context,
e.g. $x?(z).\lift{w}{y!(z)}$, we see that the process under the lift
operator may be shaped by prefixed inputs binding a name inside it. In
this sense, the lift operator will be seen as a way to dynamically
construct processes before reifying them as names.

Finally equipped with these standard features we can present the
dynamics of the calculus.

\subsubsection{Operational semantics} 

Finally, we introduce the computational dynamics. What marks these
algebras as distinct from other more traditionally studied algebraic
structures, e.g. vector spaces or polynomial rings, is the manner in
which dynamics is captured. In traditional structures, dynamics is typically
expressed through morphisms between such structures, as in linear maps
between vector spaces or morphisms between rings. In algebras
associated with the semantics of computation, the dynamics is
expressed as part of the algebraic structure itself, through a
reduction reduction relation typically denoted by $\red$. Below, we
give a recursive presentation of this relation for the calculus used
in the encoding.

$\red \subseteq \pi \times \pi$
$\red : \pi \to \mathcal{P}(\pi)$

\begin{mathpar}
  \inferrule* [lab=Comm] { \textsf{match}( x_{src}, x_{trgt} ) } { x_{trgt}?(y)P \; | \; x_{src}!\langle {Q} \rangle \red P\{\quotep{Q}/y}\} }
  \and \\
  \inferrule* [lab=Par] {{P} \red {P}'} {{{P} | {Q}} \red {{P}' | {Q}}}
  \and
  \inferrule* [lab=Equiv]{{{P} \scong {P}'} \andalso {{P}' \red {Q}'} \andalso {{Q}' \scong {Q}}}{{P} \red {Q}}
\end{mathpar}

\begin{eqnarray*}
  match_{\equiv} (\quotep{P},\quotep{Q}) & := & P \equiv Q \\
  match_{\dagger}(\quotep{P},\quotep{Q}) & := & \forall R. P|Q \red^{*} R => R \red^{*} 0 \\
  match_{K}(\quotep{P},\quotep{Q}) & := & K \mbox{ for some context } K
\end{eqnarray*}

$u?(x)P | u!\langle Q \rangle \red P\{\quotep{Q}/x\}$

%We write $\wred$ for $\red^*$, and $P\red$ if $\exists Q $ such that $ P \red Q$.
We write $P\red$ if $\exists Q $ such that $ P \red Q$ and $P\not\red$, otherwise.

\section{Replication}

As mentioned before, it is known that replication (and hence
recursion) can be implemented in a higher-order process algebra
\cite{SangiorgiWalker}. As our first example of calculation with the
machinery thus far presented we give the construction explicitly in
the {\rhoc}.

\begin{eqnarray}
	D_{x} & := & \prefix{x}{y}{(\binpar{\outputp{x}{y}}{@{y}})} \nonumber\\
	\bangp_{x}{P} & := & \binpar{{x}!\langle{\binpar{D_{x}}{P}}\rangle}{D_{x}} \nonumber
\end{eqnarray}

\begin{eqnarray}
	\bangp_{x}{P} & & \nonumber\\
	=
	& {x}!\langle{(\prefix{x}{y}{(\outputp{x}{y} | @{y})) | P}}\rangle 
	      | \prefix{x}{y}{(\outputp{x}{y} | @{y})} & \nonumber\\
	\red
	& (\outputp{x}{y} | @{y})\substn{\quotep{(\prefix{x}{y}{(@{y} | \outputp{x}{y})) | P}}}{y} & \nonumber\\
	=
	& \outputp{x}{\quotep{(\prefix{x}{y}{(\outputp{x}{y} | @{y})) | P}}}
	  | {(\prefix{x}{y}{(\outputp{x}{y} | @{y})) | P}} & \nonumber\\
	\red
	& \ldots & \nonumber\\
	\red^*
	& P | P | \ldots & \nonumber
\end{eqnarray}

Of course, this encoding, as an implementation, runs away, unfolding
$\bangp{P}$ eagerly. A lazier and more implementable replication
operator, restricted to input-guarded processes, may be obtained as follows.

\begin{eqnarray}
\bangp{\prefix{u}{v}{P}} 
	:= 
	\binpar{\lift{x}{\prefix{u}{v}{(\binpar{D(x)}{P})}}}{D(x)} \nonumber
\end{eqnarray}

\begin{remark}
  Note that the lazier definition still does not deal with summation
  or mixed summation (i.e. sums over input and output). The reader is
  invited to construct definitions of replication that deal with these
  features. 

  Further, the definitions are parameterized in a name, $x$. Can you,
  gentle reader, make a definition that eliminates this parameter and
  guarantees no accidental interaction between the replication
  machinery and the process being replicated -- i.e. no accidental
  sharing of names used by the process to get its work done and the
  name(s) used by the replication to effect copying. This latter
  revision of the definition of replication is crucial to obtaining
  the expected identity $!!P \sim !P$.
\end{remark}

\begin{remark}\label{rem:paradoxical_combinator}
  The reader familiar with the lambda calculus will have noticed the
  similarity between $D$ and the paradoxical combinator.

  [Ed. note: the existence of this seems to suggest we have to be more
  restrictive on the set of processes and names we admit if we are to
  support no-cloning.]
\end{remark}

\subsubsection{Bisimulation}

The computational dynamics gives rise to another kind of equivalence,
the equivalence of computational behavior. As previously mentioned
this is typically captured \emph{via} some form of bisimulation.

% The notion we use in this paper is weak barbed bisimulation
% \cite{milner91polyadicpi}.

The notion we use in this paper is derived from weak barbed
bisimulation \cite{milner91polyadicpi}. 

\begin{definition}
An \emph{observation relation}, $\downarrow_{\mathcal N}$, over a set
of names, $\mathcal N$, is the smallest relation satisfying the rules
below.

\infrule[Out-barb]{y \in {\mathcal N}, \; x \nameeq y}
		  {\outputp{x}{v} \downarrow_{\mathcal N} x}
\infrule[Par-barb]{\mbox{$P\downarrow_{\mathcal N} x$ or $Q\downarrow_{\mathcal N} x$}}
		  {\binpar{P}{Q} \downarrow_{\mathcal N} x}

We write $P \Downarrow_{\mathcal N} x$ if there is $Q$ such that 
$P \wred Q$ and $Q \downarrow_{\mathcal N} x$.
\end{definition}

\begin{definition}
%\label{def.bbisim}
An  ${\mathcal N}$-\emph{barbed bisimulation} over a set of names, ${\mathcal N}$, is a symmetric binary relation 
${\mathcal S}_{\mathcal N}$ between agents such that $P\rel{S}_{\mathcal N}Q$ implies:
\begin{enumerate}
\item If $P \red P'$ then $Q \wred Q'$ and $P'\rel{S}_{\mathcal N} Q'$.
\item If $P\downarrow_{\mathcal N} x$, then $Q\Downarrow_{\mathcal N} x$.
\end{enumerate}
$P$ is ${\mathcal N}$-barbed bisimilar to $Q$, written
$P \wbbisim_{\mathcal N} Q$, if $P \rel{S}_{\mathcal N} Q$ for some ${\mathcal N}$-barbed bisimulation ${\mathcal S}_{\mathcal N}$.
\end{definition}

$\mathcal{R} \subseteq \pi \times \pi$

$P \mathcal{R} Q => \forall P'. P \red P' \Rightarrow \exists Q'. Q \red Q', P' \mathcal{R} Q'$

$P \vdash x \Rightarrow Q \vdash x$

\begin{mathpar}
  \inferrule*[lab=Out-barb]{x \nameeq y}{{y}!\langle{Q}\rangle \vdash x}
  \and
  \inferrule*[lab=Par-barb]{\mbox{$P\vdash x$ or $Q\vdash x$}}{\binpar{P}{Q} \vdash x}
\end{mathpar}

\subsubsection{Contexts}

One of the principle advantages of computational calculi like the
$\pi$-calculus is a well-defined notion of context,
contextual-equivalence and a correlation between
contextual-equivalence and notions of bisimulation. The notion of
context allows the decomposition of a process into (sub-)process and
its syntactic environment, its context. Thus, a context may be
thought of as a process with a ``hole'' (written $\Box$) in it. The
application of a context $M$ to a process $P$, written $M[P]$, is
tantamount to filling the hole in $M$ with $P$. In this paper we do
not need the full weight of this theory, but do make use of the notion
of context in the proof the main theorem. 

\begin{mathpar}
  \inferrule* [lab=summation] {} {{M_{M},M_{N}} \bc \Box \;|\; x.M_{A} \;|\; M_{M}+M_{N}}
  \and
  \inferrule* [lab=agent] {} {{M_{A}} \bc (\vec{x})M_{P} \;| \; \clift{P_0,\ldots,M_{P},\ldots,P_N}}
  \and \\
  \inferrule* [lab=process] {} {{M_{P}} \bc M_{N} \;| \;P|M_{P} }
\end{mathpar} 

\begin{mathpar}
  \inferrule* [lab=sychronization] {} {M_{N} \bc \Box \;|\; x?M_{F} \;|\; x!M_{C}}
  \and
  \inferrule* [lab=abstraction] {} {{M_{F}} \bc (x)M_{P} }
  \and
  \inferrule* [lab=concretion] {} {{M_{C}} \bc \langle M_{P} \rangle }
  \and \\
  \inferrule* [lab=process] {} {{M_{P}} \bc M_{N} \;| \;P|M_{P} }
\end{mathpar}

\begin{definition}[contextual application] Given a context $M$, and
  process $P$, we define the \emph{contextual application}, $M[P] :=
  M\{P/\Box\}$. That is, the contextual application of M to P is the
  substitution of $P$ for $\Box$ in $M$.
\end{definition}

$\meaningof{-} : L \to \mathcal{P}(\pi)$

\begin{mathpar}
  \inferrule* [lab=collection] {} {\meaningof{true} = \pi, \and \meaningof{~E} = \pi \setminus \meaningof{E}, \and \meaningof{E_{1} \& E_{2}} = \meaningof{E_{1}} \cap \meaningof{E_{2}}}
\end{mathpar}

\begin{mathpar}
  \inferrule* [lab=structure] {} {\meaningof{0} = \{ P \in \pi | P \equiv 0 \}, \and \\ \meaningof{E_1 | E_2} = \{ P \in \pi | P \equiv P_{1} | P_{2}, P_{1} \in \meaningof{E_{1}}, P_{2} \in \meaningof{E_2}\} }
\end{mathpar}

\begin{mathpar}
 \inferrule* [lab=behavior] {} {\meaningof{\langle a?b \rangle E} = \{ P \in \pi | P \equiv Q | u?(y)P', \\ \and \\\\ \and \\ \;\;\; u \in \meaningof{a}, \forall z.P'\{z/y\} \in \meaningof{E\{z/b\}}\}, \and \\ \meaningof{a!E} = \{ P \in \pi | P \equiv Q | x!\langle P' \rangle, x \in \meaningof{a} P' \in \meaningof{E}\} }
\end{mathpar}

\begin{mathpar}
 \inferrule* [lab=nominal] {} {\meaningof{\quotep{E}} = \{ \quotep{P} \in \quotep{\pi} | P \in \meaningof{E} \}, \and \meaningof{\quotep{P}} = \{ \quotep{Q} \in \quotep{\pi} | P \equiv Q \} \and \\ \meaningof{@\quotep{E}} = \{ P \in \pi | P \equiv @x, x \in \meaningof{E} \}}
\end{mathpar}

\begin{eqnarray*}
  \\
  \meaningof{-} : TS \to ST
\end{eqnarray*}

\begin{eqnarray*}
  \\
  L : TS \to ST
\end{eqnarray*}

\begin{eqnarray*}
  \\
  P \models E \iff P \in \meaningof{E}
\end{eqnarray*}

\begin{eqnarray*}
  P \approx_{L} Q \iff \forall E \in L. P \models E \iff Q \models E
\end{eqnarray*}

\begin{eqnarray*}
  P \approx_{K} Q
\end{eqnarray*}

\begin{eqnarray*}
  P \approx Q
\end{eqnarray*}

$\approx_{K} = \approx = \approx_{L}$

\subsubsection{Contextual duality}

Note that contexts extend the quotation operation to a family of
operations from processes to names. Given a context, $M$, we can
define a \emph{nominal context}, $\quotep{M}$ by $\quotep{M}[P] :=
\quotep{M[P]}$. To foreshadow what is to come we observe that these
operations enjoy a duality with processes very much like the duality
between vectors and maps from vectors to scalars.

Further, because the calculus is essentially higher-order, we have a
correspondence between contexts and processes. More specifically,
given a name $x$ and a context $M$ we can construct $M^{*}_{x}$ such
that 

\begin{mathpar}
  M^{*}_{x} | \lift{x}{P} \red M[P]
\end{mathpar}

namely,

\begin{mathpar}
  M^{*}_{x} := x?(u).M[\dropn{u}]
\end{mathpar}

The dependence of $M^{*}_{x}$ on a name makes it an abstraction, 

\begin{mathpar}
  M^{*} := (x)x?(u).M[\dropn{u}]
\end{mathpar}

\subsection{Additional notation}

It will sometimes be convenient to denote the process a name
quotes. We already have the notation $x = \quotep{P}$, but it will be
convenient to introduce an alternate notation, $\procn{x}$, when we
want to emphasize the connection to the use of the name. Note that, by
virtue of name equivalence, $\quotep{\procn{x}} \nameeq x$; so, the
notation is consistent with previous definitions.

Further, because names have structure it is possible to effect
substitutions on the basis of that structure. This means we need to
upgrade our notation for substitutions, which we accomplish by
adapting comprehension notation. Thus,

\begin{mathpar}
  P\{ y / x : x \in S \}
\end{mathpar}

is interpreted to mean the process derived from P by replacing (in a
capture-avoiding manner) each occurrence of $x$ in $S$ by $y$. For example,

\begin{mathpar}
  P\{ \quotep{\procn{x}|\procn{x}} / x : x \in \freenames{P} \}
\end{mathpar}

will replace each (occurrence) of a free name $x$ in $P$ by
$\quotep{\procn{x}|\procn{x}}$.

Also, we will avail ourselves of the notation $x^{L}$ and $x^{R}$ to
denote injections of a name into disjoint copies of the name
space. There are numerous ways to accomplish this. One example can be
found in \cite{MeredithR05}. This notation overloads to vectors of
names: $\vec{x}^{\pi} := (x_{i}^{\pi} \; : \; 0 \leq i < |\vec{x}| )$ where $\pi \in \{L,R\}$.

We also use $P^{\Box} := P|\Box$.

In \cite{MeredithR05} an interpretation of the new operator is
given. It turns out that there are several possible interpretations
all enjoying the requisite algebraic properties of the operator (see
\cite{milner91polyadicpi}). We will therefore make liberal use of
$(\nu\; \vec{x})P$.

% subsection the_syntax_and_semantics_of_the_notation_system (end)   

\input{qm2pi.qmops} 

\input{qm2pi.sterngerlach} 

\input{qm2pi.metric} 

% section concurrent_process_calculi (end)

%\input{qm2pi.proofsketch}

% section proof sketch (end)

%\input{qm2pi.slviaknots} 

% section spatial logic via knots (end)

\input{qm2pi.conclusion}

% section conclusion (end)

%\input{qm2pi.dtcodes} 

% section wiring algorithm (end)

\input{qm2pi.ack} 

% section acknowledgments (end)

\newpage


\bibliographystyle{plain}   
\bibliography{../../biblios/main.bib}

\input{qm2pi.rhodetails}

\end{document}

 

% section notation (end)

\input{qm2pi.process.calculi} 

% section concurrent_process_calculi_and_spatial_logics_ (end)
    
%\documentclass[12pt]{llncs}
%\documentclass{jktr}

\usepackage[pdftex]{hyperref}                   
\usepackage {listings}
\usepackage {mathpartir}
\usepackage{bcprules}
%\usepackage{listings}
                       
\usepackage{graphicx} 
%\usepackage[margins=2.5cm,nohead,nofoot]{geometry}
%\usepackage{geometry}
\usepackage{amsfonts}
\usepackage{amstext}
\usepackage{latexsym}
\usepackage{amssymb}
\usepackage{color}


%\include{myPreamble}
\include{qm2pi.local} 

%\ifpdf
%\usepackage[pdftex]{graphicx}
%\else
%\usepackage{graphicx}
%\fi

 % \ifpdf
%  \usepackage{pdfsync}
%  \if


%\title{Brief Article}
%\author{David F. Snyder}
%\author{L.G. Meredith}

%\address{Dept. of Math., Texas State University--San Marcos, San Marcos, TX 78666}
       
\pagestyle{empty}


\begin{document}

\lstset{language=[Objective]Caml,frame=shadowbox}

\input{qm2pi.front}

% section front matter (end)

\input{qm2pi.intro} 
 
% section introduction (end)

% \input{qm2pi.knotations} 

% section notation (end)

\input{qm2pi.process.calculi} 

% section concurrent_process_calculi_and_spatial_logics_ (end)
    
%\input{qm2pi.knots2pi} 

%\input{qm2pi.trefoil} 

%\input{qm2pi.mainthm} 

% subsection basic_interpretation (end)

%\input{qm2pi.rho.presentation} 
\subsection{The syntax and semantics of the notation system}\label{sub:the_syntax_and_semantics_of_the_notation_system} % (fold)

We now summarize a technical presentation of the calculus that
embodies our theory of dynamics. The typical presentation of such a
calculus follows the style of giving generators and relations on
them. The grammar, below, describing term constructors, freely
generates the set of processes, $\Proc$. This set is then quotiented
by a relation known as structural congruence and it is over this set
that the notion of dynamics is expressed. This presentation is
essentially that of \cite{MeredithR05} with the addition of
polyadicity and summation. For readability we have relegated some of
the technical subtleties to an appendix.

\subsubsection{Process grammar}\label{subsub:process_grammar}

\begin{mathpar}
  \inferrule* [lab=synchronization] {} {{M} \bc \pzero \;|\; x?F \;|\; x!C }
  \and
  \inferrule* [lab=abstraction] {} {{F} \bc (x)P}
  \and
  \inferrule* [lab=concretion] {} {{C} \bc \langle Q \rangle}
  \and
  \inferrule* [lab=process] {} {{P,Q} \bc M \;| \;P|Q \;|\; @{x}}
  \and
  \inferrule* [lab=name] {} {{x} \bc \quotep{P}}
\end{mathpar} 

Note that $\vec{x}$ (resp. $\vec{P}$) denotes a vector of names
(resp. processes) of length $|\vec{x}|$ (resp. $|\vec{P}|$). We adopt
the following useful abbreviations.

\begin{mathpar}
   x?(\vec{y}).P := x.(\vec{y})P \and  x\clift{\vec{P}} := x.\clift{\vec{P}}
   \and x!(y) := \lift{x}{\dropn{y}}
   \and \Pi_{i=0}^{n-1}P_i := P_0 | \ldots | P_{n-1}
\end{mathpar}

\subsubsection{Structural congruence}

\paragraph{Free and bound names and alpha-equivalence.} At the
core of structural equivalence is alpha-equivalence which identifies
process that are the same up to a change of variable. Formally, we
recognize the distinction between free and bound names. The free names
of a process, $\freenames{P}$, may be calculated recursively as
follows:

\begin{mathpar}
\freenames{\pzero} := \emptyset
  \and \\
  \freenames{x?(y).P} := \{ x \} \cup (\freenames{P} \setminus \{ y \})
  \and 
  \freenames{x!\langle P \rangle} := \{ x \} \cup \{ P \} 
  \and \\
  \freenames{P|Q} := \freenames{P} \cup \freenames{Q}
  \and \\
  \freenames{@{x}} := \{ x \}
\end{mathpar}

$\pi$
$\quotep{\pi}$

$\freenames{-} : \pi \to \mathcal{P}(\quotep{\pi})$

\begin{eqnarray*}
  \freenames{\pzero} & := & \emptyset \\
  \freenames{x?(y).P} & := & \{ x \} \cup (\freenames{P} \setminus \{ y \}) \\
  \freenames{x!\langle P \rangle} & := & \{ x \} \cup \{ P \} \\
  \freenames{P|Q} & := & \freenames{P} \cup \freenames{Q} \\
  \freenames{\dropn{x}} & := & \{ x \}
\end{eqnarray*}

The bound names of a process, $\boundnames{P}$, are those names occurring in $P$
that are not free. For example, in $x?(y).0$, the name $x$ is free, while $y$ is bound.

\begin{mathpar}
  \inferrule* [lab=monoidal-laws] {} { P|Q \equiv Q|P \and P|0 \equiv P \and P|(Q|R) \equiv (P|Q)|R }
\end{mathpar}

\begin{mathpar}
  \inferrule* [lab=alpha-equivalence] {} { (x)P \equiv (y)P\{y/x\} \and y \not\in \freenames{P} }
\end{mathpar}

\begin{definition}
Then two processes, $P,Q$, are alpha-equivalent if $P = Q\{\vec{y}/\vec{x}\}$ for
some $\vec{x} \in \boundnames{Q},\vec{y} \in \boundnames{P}$, where $Q\{\vec{y}/\vec{x}\}$
denotes the capture-avoiding substitution of $\vec{y}$ for $\vec{x}$ in $Q$.
\end{definition}

\begin{definition}
  The {\em structural congruence} \cite{SangiorgiWalker} , $\equiv$,
  between processes is the least congruence containing
  alpha-equivalence, satisfying the abelian monoid laws
  (associativity, commutativity and $\pzero$ as identity) for parallel
  composition $|$ and for summation $+$.
\end{definition}

\subsection{Name equivalence}

We take name equivalence, written $\nameeq$, to be the smallest
equivalence relation generated by the following rules.

\begin{mathpar}
\inferrule*[lab=Quote-drop]
{ }
{ \quotep{@{x}} \nameeq x }

\inferrule*[lab=Struct-equiv]
{ P \scong Q }
{ \quotep{P} \nameeq \quotep{Q} }
\end{mathpar}

The astute reader will have noticed that the mutual recursion of names
and processes imposes a mutual recursion on alpha-equivalence and
structural equivalence via name-equivalence. Fortunately, all of this
works out pleasantly and we may calculate in the natural way, free of
concern. The reader interested in the details is referred to the
appendix \ref{appendix:rho_details}.

\subsection{Substitution}

We use $\Proc$ for the set of processes, $\QProc$ for the set of
names, and $\id{\{}\vec{y} / \vec{x} \id{\}}$ to denote partial maps,
$s : \QProc \rightarrow \QProc$. A map, $s$ lifts, uniquely, to a map
on process terms, $\widehat{s} : \Proc \rightarrow \Proc$ by the
following equations.

\begin{mathpar}
  (0) \psubstp{Q}{P} := 0 \\
  (R \juxtap S) \psubstp{Q}{P}
  :=    
  (R)\psubstp{Q}{P} \juxtap (S) \psubstp{Q}{P} \\
  (x?(y).R) \psubstp{Q}{P}    
  :=    
  (x)\substp{Q}{P} (z)\concat( (R \psubstn{z}{y}) \psubstp{Q}{P} ) \\
  (\lift{x}{R}) \psubstp{Q}{P}  
  :=
  \lift{(x)\substp{Q}{P}}{ R \psubstp{Q}{P} } \\
%   (\dropn{x})  \psubstp{Q}{P}       
%   := 
%   \left\{ 
%     \begin{array}{ccc} 
%       \dropn{\quotep{Q}} & & x \nameeq \quotep{P} \\
%       \dropn{x} & & otherwise \\
%     \end{array}
%   \right. 
  (\dropn{x})  \psubstp{Q}{P}       
  := 
  \left\{ 
    \begin{array}{ccc} 
      Q & & x \nameeq \quotep{P} \\
      \dropn{x} & & otherwise \\
    \end{array}
  \right.
\end{mathpar}
 

where

\begin{eqnarray}
  (x)\id{\{} \lpquote Q \rpquote / \lpquote P \rpquote \id{\}}            = 
  \left\{ 
    \begin{array}{ccc}
      \lpquote Q \rpquote & & x \nameeq \lpquote P \rpquote \\
      x & & otherwise \\
    \end{array}
  \right. \nonumber
\end{eqnarray}

and $z$ is chosen distinct from $\quotep{P}$, $\quotep{Q}$, the free
names in $Q$, and all the names in $R$. Our $\alpha$-equivalence will
be built in the standard way from this substitution.

\begin{remark}\label{rem:no_self_referential_names}
  One consequence of these definitions is that $\forall P. \quotep{P}
  \not\in \freenames{P}$.
\end{remark}

\subsection{ Dynamic quote: an example }

Anticipating something of what's to come, consider applying the
substitution, $\widehat{\id{\{}u / z \id{\}}}$, to the following pair
of processes, $\lift{w}{y!(z)}$ and $w[ \lpquote y!(z) \rpquote ]$.

\begin{eqnarray}
	\lift{w}{y!(z)}\widehat{\id{\{}u / z \id{\}}}
		& = &
		\lift{w}{y!(u)} \nonumber\\
	w[ \lpquote y!(z) \rpquote ] \widehat{ \id{\{}u / z \id{\}} }
		& = &
		w[ \lpquote y!(z) \rpquote ] \nonumber
\end{eqnarray}

Because the body of the process between quotes is impervious to
substitution, we get radically different answers. In fact, by
examining the first process in an input context,
e.g. $x?(z).\lift{w}{y!(z)}$, we see that the process under the lift
operator may be shaped by prefixed inputs binding a name inside it. In
this sense, the lift operator will be seen as a way to dynamically
construct processes before reifying them as names.

Finally equipped with these standard features we can present the
dynamics of the calculus.

\subsubsection{Operational semantics} 

Finally, we introduce the computational dynamics. What marks these
algebras as distinct from other more traditionally studied algebraic
structures, e.g. vector spaces or polynomial rings, is the manner in
which dynamics is captured. In traditional structures, dynamics is typically
expressed through morphisms between such structures, as in linear maps
between vector spaces or morphisms between rings. In algebras
associated with the semantics of computation, the dynamics is
expressed as part of the algebraic structure itself, through a
reduction reduction relation typically denoted by $\red$. Below, we
give a recursive presentation of this relation for the calculus used
in the encoding.

$\red \subseteq \pi \times \pi$
$\red : \pi \to \mathcal{P}(\pi)$

\begin{mathpar}
  \inferrule* [lab=Comm] { \textsf{match}( x_{src}, x_{trgt} ) } { x_{trgt}?(y)P \; | \; x_{src}!\langle {Q} \rangle \red P\{\quotep{Q}/y}\} }
  \and \\
  \inferrule* [lab=Par] {{P} \red {P}'} {{{P} | {Q}} \red {{P}' | {Q}}}
  \and
  \inferrule* [lab=Equiv]{{{P} \scong {P}'} \andalso {{P}' \red {Q}'} \andalso {{Q}' \scong {Q}}}{{P} \red {Q}}
\end{mathpar}

\begin{eqnarray*}
  match_{\equiv} (\quotep{P},\quotep{Q}) & := & P \equiv Q \\
  match_{\dagger}(\quotep{P},\quotep{Q}) & := & \forall R. P|Q \red^{*} R => R \red^{*} 0 \\
  match_{K}(\quotep{P},\quotep{Q}) & := & K \mbox{ for some context } K
\end{eqnarray*}

$u?(x)P | u!\langle Q \rangle \red P\{\quotep{Q}/x\}$

%We write $\wred$ for $\red^*$, and $P\red$ if $\exists Q $ such that $ P \red Q$.
We write $P\red$ if $\exists Q $ such that $ P \red Q$ and $P\not\red$, otherwise.

\section{Replication}

As mentioned before, it is known that replication (and hence
recursion) can be implemented in a higher-order process algebra
\cite{SangiorgiWalker}. As our first example of calculation with the
machinery thus far presented we give the construction explicitly in
the {\rhoc}.

\begin{eqnarray}
	D_{x} & := & \prefix{x}{y}{(\binpar{\outputp{x}{y}}{@{y}})} \nonumber\\
	\bangp_{x}{P} & := & \binpar{{x}!\langle{\binpar{D_{x}}{P}}\rangle}{D_{x}} \nonumber
\end{eqnarray}

\begin{eqnarray}
	\bangp_{x}{P} & & \nonumber\\
	=
	& {x}!\langle{(\prefix{x}{y}{(\outputp{x}{y} | @{y})) | P}}\rangle 
	      | \prefix{x}{y}{(\outputp{x}{y} | @{y})} & \nonumber\\
	\red
	& (\outputp{x}{y} | @{y})\substn{\quotep{(\prefix{x}{y}{(@{y} | \outputp{x}{y})) | P}}}{y} & \nonumber\\
	=
	& \outputp{x}{\quotep{(\prefix{x}{y}{(\outputp{x}{y} | @{y})) | P}}}
	  | {(\prefix{x}{y}{(\outputp{x}{y} | @{y})) | P}} & \nonumber\\
	\red
	& \ldots & \nonumber\\
	\red^*
	& P | P | \ldots & \nonumber
\end{eqnarray}

Of course, this encoding, as an implementation, runs away, unfolding
$\bangp{P}$ eagerly. A lazier and more implementable replication
operator, restricted to input-guarded processes, may be obtained as follows.

\begin{eqnarray}
\bangp{\prefix{u}{v}{P}} 
	:= 
	\binpar{\lift{x}{\prefix{u}{v}{(\binpar{D(x)}{P})}}}{D(x)} \nonumber
\end{eqnarray}

\begin{remark}
  Note that the lazier definition still does not deal with summation
  or mixed summation (i.e. sums over input and output). The reader is
  invited to construct definitions of replication that deal with these
  features. 

  Further, the definitions are parameterized in a name, $x$. Can you,
  gentle reader, make a definition that eliminates this parameter and
  guarantees no accidental interaction between the replication
  machinery and the process being replicated -- i.e. no accidental
  sharing of names used by the process to get its work done and the
  name(s) used by the replication to effect copying. This latter
  revision of the definition of replication is crucial to obtaining
  the expected identity $!!P \sim !P$.
\end{remark}

\begin{remark}\label{rem:paradoxical_combinator}
  The reader familiar with the lambda calculus will have noticed the
  similarity between $D$ and the paradoxical combinator.

  [Ed. note: the existence of this seems to suggest we have to be more
  restrictive on the set of processes and names we admit if we are to
  support no-cloning.]
\end{remark}

\subsubsection{Bisimulation}

The computational dynamics gives rise to another kind of equivalence,
the equivalence of computational behavior. As previously mentioned
this is typically captured \emph{via} some form of bisimulation.

% The notion we use in this paper is weak barbed bisimulation
% \cite{milner91polyadicpi}.

The notion we use in this paper is derived from weak barbed
bisimulation \cite{milner91polyadicpi}. 

\begin{definition}
An \emph{observation relation}, $\downarrow_{\mathcal N}$, over a set
of names, $\mathcal N$, is the smallest relation satisfying the rules
below.

\infrule[Out-barb]{y \in {\mathcal N}, \; x \nameeq y}
		  {\outputp{x}{v} \downarrow_{\mathcal N} x}
\infrule[Par-barb]{\mbox{$P\downarrow_{\mathcal N} x$ or $Q\downarrow_{\mathcal N} x$}}
		  {\binpar{P}{Q} \downarrow_{\mathcal N} x}

We write $P \Downarrow_{\mathcal N} x$ if there is $Q$ such that 
$P \wred Q$ and $Q \downarrow_{\mathcal N} x$.
\end{definition}

\begin{definition}
%\label{def.bbisim}
An  ${\mathcal N}$-\emph{barbed bisimulation} over a set of names, ${\mathcal N}$, is a symmetric binary relation 
${\mathcal S}_{\mathcal N}$ between agents such that $P\rel{S}_{\mathcal N}Q$ implies:
\begin{enumerate}
\item If $P \red P'$ then $Q \wred Q'$ and $P'\rel{S}_{\mathcal N} Q'$.
\item If $P\downarrow_{\mathcal N} x$, then $Q\Downarrow_{\mathcal N} x$.
\end{enumerate}
$P$ is ${\mathcal N}$-barbed bisimilar to $Q$, written
$P \wbbisim_{\mathcal N} Q$, if $P \rel{S}_{\mathcal N} Q$ for some ${\mathcal N}$-barbed bisimulation ${\mathcal S}_{\mathcal N}$.
\end{definition}

$\mathcal{R} \subseteq \pi \times \pi$

$P \mathcal{R} Q => \forall P'. P \red P' \Rightarrow \exists Q'. Q \red Q', P' \mathcal{R} Q'$

$P \vdash x \Rightarrow Q \vdash x$

\begin{mathpar}
  \inferrule*[lab=Out-barb]{x \nameeq y}{{y}!\langle{Q}\rangle \vdash x}
  \and
  \inferrule*[lab=Par-barb]{\mbox{$P\vdash x$ or $Q\vdash x$}}{\binpar{P}{Q} \vdash x}
\end{mathpar}

\subsubsection{Contexts}

One of the principle advantages of computational calculi like the
$\pi$-calculus is a well-defined notion of context,
contextual-equivalence and a correlation between
contextual-equivalence and notions of bisimulation. The notion of
context allows the decomposition of a process into (sub-)process and
its syntactic environment, its context. Thus, a context may be
thought of as a process with a ``hole'' (written $\Box$) in it. The
application of a context $M$ to a process $P$, written $M[P]$, is
tantamount to filling the hole in $M$ with $P$. In this paper we do
not need the full weight of this theory, but do make use of the notion
of context in the proof the main theorem. 

\begin{mathpar}
  \inferrule* [lab=summation] {} {{M_{M},M_{N}} \bc \Box \;|\; x.M_{A} \;|\; M_{M}+M_{N}}
  \and
  \inferrule* [lab=agent] {} {{M_{A}} \bc (\vec{x})M_{P} \;| \; \clift{P_0,\ldots,M_{P},\ldots,P_N}}
  \and \\
  \inferrule* [lab=process] {} {{M_{P}} \bc M_{N} \;| \;P|M_{P} }
\end{mathpar} 

\begin{mathpar}
  \inferrule* [lab=sychronization] {} {M_{N} \bc \Box \;|\; x?M_{F} \;|\; x!M_{C}}
  \and
  \inferrule* [lab=abstraction] {} {{M_{F}} \bc (x)M_{P} }
  \and
  \inferrule* [lab=concretion] {} {{M_{C}} \bc \langle M_{P} \rangle }
  \and \\
  \inferrule* [lab=process] {} {{M_{P}} \bc M_{N} \;| \;P|M_{P} }
\end{mathpar}

\begin{definition}[contextual application] Given a context $M$, and
  process $P$, we define the \emph{contextual application}, $M[P] :=
  M\{P/\Box\}$. That is, the contextual application of M to P is the
  substitution of $P$ for $\Box$ in $M$.
\end{definition}

$\meaningof{-} : L \to \mathcal{P}(\pi)$

\begin{mathpar}
  \inferrule* [lab=collection] {} {\meaningof{true} = \pi, \and \meaningof{~E} = \pi \setminus \meaningof{E}, \and \meaningof{E_{1} \& E_{2}} = \meaningof{E_{1}} \cap \meaningof{E_{2}}}
\end{mathpar}

\begin{mathpar}
  \inferrule* [lab=structure] {} {\meaningof{0} = \{ P \in \pi | P \equiv 0 \}, \and \\ \meaningof{E_1 | E_2} = \{ P \in \pi | P \equiv P_{1} | P_{2}, P_{1} \in \meaningof{E_{1}}, P_{2} \in \meaningof{E_2}\} }
\end{mathpar}

\begin{mathpar}
 \inferrule* [lab=behavior] {} {\meaningof{\langle a?b \rangle E} = \{ P \in \pi | P \equiv Q | u?(y)P', \\ \and \\\\ \and \\ \;\;\; u \in \meaningof{a}, \forall z.P'\{z/y\} \in \meaningof{E\{z/b\}}\}, \and \\ \meaningof{a!E} = \{ P \in \pi | P \equiv Q | x!\langle P' \rangle, x \in \meaningof{a} P' \in \meaningof{E}\} }
\end{mathpar}

\begin{mathpar}
 \inferrule* [lab=nominal] {} {\meaningof{\quotep{E}} = \{ \quotep{P} \in \quotep{\pi} | P \in \meaningof{E} \}, \and \meaningof{\quotep{P}} = \{ \quotep{Q} \in \quotep{\pi} | P \equiv Q \} \and \\ \meaningof{@\quotep{E}} = \{ P \in \pi | P \equiv @x, x \in \meaningof{E} \}}
\end{mathpar}

\begin{eqnarray*}
  \\
  \meaningof{-} : TS \to ST
\end{eqnarray*}

\begin{eqnarray*}
  \\
  L : TS \to ST
\end{eqnarray*}

\begin{eqnarray*}
  \\
  P \models E \iff P \in \meaningof{E}
\end{eqnarray*}

\begin{eqnarray*}
  P \approx_{L} Q \iff \forall E \in L. P \models E \iff Q \models E
\end{eqnarray*}

\begin{eqnarray*}
  P \approx_{K} Q
\end{eqnarray*}

\begin{eqnarray*}
  P \approx Q
\end{eqnarray*}

$\approx_{K} = \approx = \approx_{L}$

\subsubsection{Contextual duality}

Note that contexts extend the quotation operation to a family of
operations from processes to names. Given a context, $M$, we can
define a \emph{nominal context}, $\quotep{M}$ by $\quotep{M}[P] :=
\quotep{M[P]}$. To foreshadow what is to come we observe that these
operations enjoy a duality with processes very much like the duality
between vectors and maps from vectors to scalars.

Further, because the calculus is essentially higher-order, we have a
correspondence between contexts and processes. More specifically,
given a name $x$ and a context $M$ we can construct $M^{*}_{x}$ such
that 

\begin{mathpar}
  M^{*}_{x} | \lift{x}{P} \red M[P]
\end{mathpar}

namely,

\begin{mathpar}
  M^{*}_{x} := x?(u).M[\dropn{u}]
\end{mathpar}

The dependence of $M^{*}_{x}$ on a name makes it an abstraction, 

\begin{mathpar}
  M^{*} := (x)x?(u).M[\dropn{u}]
\end{mathpar}

\subsection{Additional notation}

It will sometimes be convenient to denote the process a name
quotes. We already have the notation $x = \quotep{P}$, but it will be
convenient to introduce an alternate notation, $\procn{x}$, when we
want to emphasize the connection to the use of the name. Note that, by
virtue of name equivalence, $\quotep{\procn{x}} \nameeq x$; so, the
notation is consistent with previous definitions.

Further, because names have structure it is possible to effect
substitutions on the basis of that structure. This means we need to
upgrade our notation for substitutions, which we accomplish by
adapting comprehension notation. Thus,

\begin{mathpar}
  P\{ y / x : x \in S \}
\end{mathpar}

is interpreted to mean the process derived from P by replacing (in a
capture-avoiding manner) each occurrence of $x$ in $S$ by $y$. For example,

\begin{mathpar}
  P\{ \quotep{\procn{x}|\procn{x}} / x : x \in \freenames{P} \}
\end{mathpar}

will replace each (occurrence) of a free name $x$ in $P$ by
$\quotep{\procn{x}|\procn{x}}$.

Also, we will avail ourselves of the notation $x^{L}$ and $x^{R}$ to
denote injections of a name into disjoint copies of the name
space. There are numerous ways to accomplish this. One example can be
found in \cite{MeredithR05}. This notation overloads to vectors of
names: $\vec{x}^{\pi} := (x_{i}^{\pi} \; : \; 0 \leq i < |\vec{x}| )$ where $\pi \in \{L,R\}$.

We also use $P^{\Box} := P|\Box$.

In \cite{MeredithR05} an interpretation of the new operator is
given. It turns out that there are several possible interpretations
all enjoying the requisite algebraic properties of the operator (see
\cite{milner91polyadicpi}). We will therefore make liberal use of
$(\nu\; \vec{x})P$.

% subsection the_syntax_and_semantics_of_the_notation_system (end)   

\input{qm2pi.qmops} 

\input{qm2pi.sterngerlach} 

\input{qm2pi.metric} 

% section concurrent_process_calculi (end)

%\input{qm2pi.proofsketch}

% section proof sketch (end)

%\input{qm2pi.slviaknots} 

% section spatial logic via knots (end)

\input{qm2pi.conclusion}

% section conclusion (end)

%\input{qm2pi.dtcodes} 

% section wiring algorithm (end)

\input{qm2pi.ack} 

% section acknowledgments (end)

\newpage


\bibliographystyle{plain}   
\bibliography{../../biblios/main.bib}

\input{qm2pi.rhodetails}

\end{document}

 

%\documentclass[12pt]{llncs}
%\documentclass{jktr}

\usepackage[pdftex]{hyperref}                   
\usepackage {listings}
\usepackage {mathpartir}
\usepackage{bcprules}
%\usepackage{listings}
                       
\usepackage{graphicx} 
%\usepackage[margins=2.5cm,nohead,nofoot]{geometry}
%\usepackage{geometry}
\usepackage{amsfonts}
\usepackage{amstext}
\usepackage{latexsym}
\usepackage{amssymb}
\usepackage{color}


%\include{myPreamble}
\include{qm2pi.local} 

%\ifpdf
%\usepackage[pdftex]{graphicx}
%\else
%\usepackage{graphicx}
%\fi

 % \ifpdf
%  \usepackage{pdfsync}
%  \if


%\title{Brief Article}
%\author{David F. Snyder}
%\author{L.G. Meredith}

%\address{Dept. of Math., Texas State University--San Marcos, San Marcos, TX 78666}
       
\pagestyle{empty}


\begin{document}

\lstset{language=[Objective]Caml,frame=shadowbox}

\input{qm2pi.front}

% section front matter (end)

\input{qm2pi.intro} 
 
% section introduction (end)

% \input{qm2pi.knotations} 

% section notation (end)

\input{qm2pi.process.calculi} 

% section concurrent_process_calculi_and_spatial_logics_ (end)
    
%\input{qm2pi.knots2pi} 

%\input{qm2pi.trefoil} 

%\input{qm2pi.mainthm} 

% subsection basic_interpretation (end)

%\input{qm2pi.rho.presentation} 
\subsection{The syntax and semantics of the notation system}\label{sub:the_syntax_and_semantics_of_the_notation_system} % (fold)

We now summarize a technical presentation of the calculus that
embodies our theory of dynamics. The typical presentation of such a
calculus follows the style of giving generators and relations on
them. The grammar, below, describing term constructors, freely
generates the set of processes, $\Proc$. This set is then quotiented
by a relation known as structural congruence and it is over this set
that the notion of dynamics is expressed. This presentation is
essentially that of \cite{MeredithR05} with the addition of
polyadicity and summation. For readability we have relegated some of
the technical subtleties to an appendix.

\subsubsection{Process grammar}\label{subsub:process_grammar}

\begin{mathpar}
  \inferrule* [lab=synchronization] {} {{M} \bc \pzero \;|\; x?F \;|\; x!C }
  \and
  \inferrule* [lab=abstraction] {} {{F} \bc (x)P}
  \and
  \inferrule* [lab=concretion] {} {{C} \bc \langle Q \rangle}
  \and
  \inferrule* [lab=process] {} {{P,Q} \bc M \;| \;P|Q \;|\; @{x}}
  \and
  \inferrule* [lab=name] {} {{x} \bc \quotep{P}}
\end{mathpar} 

Note that $\vec{x}$ (resp. $\vec{P}$) denotes a vector of names
(resp. processes) of length $|\vec{x}|$ (resp. $|\vec{P}|$). We adopt
the following useful abbreviations.

\begin{mathpar}
   x?(\vec{y}).P := x.(\vec{y})P \and  x\clift{\vec{P}} := x.\clift{\vec{P}}
   \and x!(y) := \lift{x}{\dropn{y}}
   \and \Pi_{i=0}^{n-1}P_i := P_0 | \ldots | P_{n-1}
\end{mathpar}

\subsubsection{Structural congruence}

\paragraph{Free and bound names and alpha-equivalence.} At the
core of structural equivalence is alpha-equivalence which identifies
process that are the same up to a change of variable. Formally, we
recognize the distinction between free and bound names. The free names
of a process, $\freenames{P}$, may be calculated recursively as
follows:

\begin{mathpar}
\freenames{\pzero} := \emptyset
  \and \\
  \freenames{x?(y).P} := \{ x \} \cup (\freenames{P} \setminus \{ y \})
  \and 
  \freenames{x!\langle P \rangle} := \{ x \} \cup \{ P \} 
  \and \\
  \freenames{P|Q} := \freenames{P} \cup \freenames{Q}
  \and \\
  \freenames{@{x}} := \{ x \}
\end{mathpar}

$\pi$
$\quotep{\pi}$

$\freenames{-} : \pi \to \mathcal{P}(\quotep{\pi})$

\begin{eqnarray*}
  \freenames{\pzero} & := & \emptyset \\
  \freenames{x?(y).P} & := & \{ x \} \cup (\freenames{P} \setminus \{ y \}) \\
  \freenames{x!\langle P \rangle} & := & \{ x \} \cup \{ P \} \\
  \freenames{P|Q} & := & \freenames{P} \cup \freenames{Q} \\
  \freenames{\dropn{x}} & := & \{ x \}
\end{eqnarray*}

The bound names of a process, $\boundnames{P}$, are those names occurring in $P$
that are not free. For example, in $x?(y).0$, the name $x$ is free, while $y$ is bound.

\begin{mathpar}
  \inferrule* [lab=monoidal-laws] {} { P|Q \equiv Q|P \and P|0 \equiv P \and P|(Q|R) \equiv (P|Q)|R }
\end{mathpar}

\begin{mathpar}
  \inferrule* [lab=alpha-equivalence] {} { (x)P \equiv (y)P\{y/x\} \and y \not\in \freenames{P} }
\end{mathpar}

\begin{definition}
Then two processes, $P,Q$, are alpha-equivalent if $P = Q\{\vec{y}/\vec{x}\}$ for
some $\vec{x} \in \boundnames{Q},\vec{y} \in \boundnames{P}$, where $Q\{\vec{y}/\vec{x}\}$
denotes the capture-avoiding substitution of $\vec{y}$ for $\vec{x}$ in $Q$.
\end{definition}

\begin{definition}
  The {\em structural congruence} \cite{SangiorgiWalker} , $\equiv$,
  between processes is the least congruence containing
  alpha-equivalence, satisfying the abelian monoid laws
  (associativity, commutativity and $\pzero$ as identity) for parallel
  composition $|$ and for summation $+$.
\end{definition}

\subsection{Name equivalence}

We take name equivalence, written $\nameeq$, to be the smallest
equivalence relation generated by the following rules.

\begin{mathpar}
\inferrule*[lab=Quote-drop]
{ }
{ \quotep{@{x}} \nameeq x }

\inferrule*[lab=Struct-equiv]
{ P \scong Q }
{ \quotep{P} \nameeq \quotep{Q} }
\end{mathpar}

The astute reader will have noticed that the mutual recursion of names
and processes imposes a mutual recursion on alpha-equivalence and
structural equivalence via name-equivalence. Fortunately, all of this
works out pleasantly and we may calculate in the natural way, free of
concern. The reader interested in the details is referred to the
appendix \ref{appendix:rho_details}.

\subsection{Substitution}

We use $\Proc$ for the set of processes, $\QProc$ for the set of
names, and $\id{\{}\vec{y} / \vec{x} \id{\}}$ to denote partial maps,
$s : \QProc \rightarrow \QProc$. A map, $s$ lifts, uniquely, to a map
on process terms, $\widehat{s} : \Proc \rightarrow \Proc$ by the
following equations.

\begin{mathpar}
  (0) \psubstp{Q}{P} := 0 \\
  (R \juxtap S) \psubstp{Q}{P}
  :=    
  (R)\psubstp{Q}{P} \juxtap (S) \psubstp{Q}{P} \\
  (x?(y).R) \psubstp{Q}{P}    
  :=    
  (x)\substp{Q}{P} (z)\concat( (R \psubstn{z}{y}) \psubstp{Q}{P} ) \\
  (\lift{x}{R}) \psubstp{Q}{P}  
  :=
  \lift{(x)\substp{Q}{P}}{ R \psubstp{Q}{P} } \\
%   (\dropn{x})  \psubstp{Q}{P}       
%   := 
%   \left\{ 
%     \begin{array}{ccc} 
%       \dropn{\quotep{Q}} & & x \nameeq \quotep{P} \\
%       \dropn{x} & & otherwise \\
%     \end{array}
%   \right. 
  (\dropn{x})  \psubstp{Q}{P}       
  := 
  \left\{ 
    \begin{array}{ccc} 
      Q & & x \nameeq \quotep{P} \\
      \dropn{x} & & otherwise \\
    \end{array}
  \right.
\end{mathpar}
 

where

\begin{eqnarray}
  (x)\id{\{} \lpquote Q \rpquote / \lpquote P \rpquote \id{\}}            = 
  \left\{ 
    \begin{array}{ccc}
      \lpquote Q \rpquote & & x \nameeq \lpquote P \rpquote \\
      x & & otherwise \\
    \end{array}
  \right. \nonumber
\end{eqnarray}

and $z$ is chosen distinct from $\quotep{P}$, $\quotep{Q}$, the free
names in $Q$, and all the names in $R$. Our $\alpha$-equivalence will
be built in the standard way from this substitution.

\begin{remark}\label{rem:no_self_referential_names}
  One consequence of these definitions is that $\forall P. \quotep{P}
  \not\in \freenames{P}$.
\end{remark}

\subsection{ Dynamic quote: an example }

Anticipating something of what's to come, consider applying the
substitution, $\widehat{\id{\{}u / z \id{\}}}$, to the following pair
of processes, $\lift{w}{y!(z)}$ and $w[ \lpquote y!(z) \rpquote ]$.

\begin{eqnarray}
	\lift{w}{y!(z)}\widehat{\id{\{}u / z \id{\}}}
		& = &
		\lift{w}{y!(u)} \nonumber\\
	w[ \lpquote y!(z) \rpquote ] \widehat{ \id{\{}u / z \id{\}} }
		& = &
		w[ \lpquote y!(z) \rpquote ] \nonumber
\end{eqnarray}

Because the body of the process between quotes is impervious to
substitution, we get radically different answers. In fact, by
examining the first process in an input context,
e.g. $x?(z).\lift{w}{y!(z)}$, we see that the process under the lift
operator may be shaped by prefixed inputs binding a name inside it. In
this sense, the lift operator will be seen as a way to dynamically
construct processes before reifying them as names.

Finally equipped with these standard features we can present the
dynamics of the calculus.

\subsubsection{Operational semantics} 

Finally, we introduce the computational dynamics. What marks these
algebras as distinct from other more traditionally studied algebraic
structures, e.g. vector spaces or polynomial rings, is the manner in
which dynamics is captured. In traditional structures, dynamics is typically
expressed through morphisms between such structures, as in linear maps
between vector spaces or morphisms between rings. In algebras
associated with the semantics of computation, the dynamics is
expressed as part of the algebraic structure itself, through a
reduction reduction relation typically denoted by $\red$. Below, we
give a recursive presentation of this relation for the calculus used
in the encoding.

$\red \subseteq \pi \times \pi$
$\red : \pi \to \mathcal{P}(\pi)$

\begin{mathpar}
  \inferrule* [lab=Comm] { \textsf{match}( x_{src}, x_{trgt} ) } { x_{trgt}?(y)P \; | \; x_{src}!\langle {Q} \rangle \red P\{\quotep{Q}/y}\} }
  \and \\
  \inferrule* [lab=Par] {{P} \red {P}'} {{{P} | {Q}} \red {{P}' | {Q}}}
  \and
  \inferrule* [lab=Equiv]{{{P} \scong {P}'} \andalso {{P}' \red {Q}'} \andalso {{Q}' \scong {Q}}}{{P} \red {Q}}
\end{mathpar}

\begin{eqnarray*}
  match_{\equiv} (\quotep{P},\quotep{Q}) & := & P \equiv Q \\
  match_{\dagger}(\quotep{P},\quotep{Q}) & := & \forall R. P|Q \red^{*} R => R \red^{*} 0 \\
  match_{K}(\quotep{P},\quotep{Q}) & := & K \mbox{ for some context } K
\end{eqnarray*}

$u?(x)P | u!\langle Q \rangle \red P\{\quotep{Q}/x\}$

%We write $\wred$ for $\red^*$, and $P\red$ if $\exists Q $ such that $ P \red Q$.
We write $P\red$ if $\exists Q $ such that $ P \red Q$ and $P\not\red$, otherwise.

\section{Replication}

As mentioned before, it is known that replication (and hence
recursion) can be implemented in a higher-order process algebra
\cite{SangiorgiWalker}. As our first example of calculation with the
machinery thus far presented we give the construction explicitly in
the {\rhoc}.

\begin{eqnarray}
	D_{x} & := & \prefix{x}{y}{(\binpar{\outputp{x}{y}}{@{y}})} \nonumber\\
	\bangp_{x}{P} & := & \binpar{{x}!\langle{\binpar{D_{x}}{P}}\rangle}{D_{x}} \nonumber
\end{eqnarray}

\begin{eqnarray}
	\bangp_{x}{P} & & \nonumber\\
	=
	& {x}!\langle{(\prefix{x}{y}{(\outputp{x}{y} | @{y})) | P}}\rangle 
	      | \prefix{x}{y}{(\outputp{x}{y} | @{y})} & \nonumber\\
	\red
	& (\outputp{x}{y} | @{y})\substn{\quotep{(\prefix{x}{y}{(@{y} | \outputp{x}{y})) | P}}}{y} & \nonumber\\
	=
	& \outputp{x}{\quotep{(\prefix{x}{y}{(\outputp{x}{y} | @{y})) | P}}}
	  | {(\prefix{x}{y}{(\outputp{x}{y} | @{y})) | P}} & \nonumber\\
	\red
	& \ldots & \nonumber\\
	\red^*
	& P | P | \ldots & \nonumber
\end{eqnarray}

Of course, this encoding, as an implementation, runs away, unfolding
$\bangp{P}$ eagerly. A lazier and more implementable replication
operator, restricted to input-guarded processes, may be obtained as follows.

\begin{eqnarray}
\bangp{\prefix{u}{v}{P}} 
	:= 
	\binpar{\lift{x}{\prefix{u}{v}{(\binpar{D(x)}{P})}}}{D(x)} \nonumber
\end{eqnarray}

\begin{remark}
  Note that the lazier definition still does not deal with summation
  or mixed summation (i.e. sums over input and output). The reader is
  invited to construct definitions of replication that deal with these
  features. 

  Further, the definitions are parameterized in a name, $x$. Can you,
  gentle reader, make a definition that eliminates this parameter and
  guarantees no accidental interaction between the replication
  machinery and the process being replicated -- i.e. no accidental
  sharing of names used by the process to get its work done and the
  name(s) used by the replication to effect copying. This latter
  revision of the definition of replication is crucial to obtaining
  the expected identity $!!P \sim !P$.
\end{remark}

\begin{remark}\label{rem:paradoxical_combinator}
  The reader familiar with the lambda calculus will have noticed the
  similarity between $D$ and the paradoxical combinator.

  [Ed. note: the existence of this seems to suggest we have to be more
  restrictive on the set of processes and names we admit if we are to
  support no-cloning.]
\end{remark}

\subsubsection{Bisimulation}

The computational dynamics gives rise to another kind of equivalence,
the equivalence of computational behavior. As previously mentioned
this is typically captured \emph{via} some form of bisimulation.

% The notion we use in this paper is weak barbed bisimulation
% \cite{milner91polyadicpi}.

The notion we use in this paper is derived from weak barbed
bisimulation \cite{milner91polyadicpi}. 

\begin{definition}
An \emph{observation relation}, $\downarrow_{\mathcal N}$, over a set
of names, $\mathcal N$, is the smallest relation satisfying the rules
below.

\infrule[Out-barb]{y \in {\mathcal N}, \; x \nameeq y}
		  {\outputp{x}{v} \downarrow_{\mathcal N} x}
\infrule[Par-barb]{\mbox{$P\downarrow_{\mathcal N} x$ or $Q\downarrow_{\mathcal N} x$}}
		  {\binpar{P}{Q} \downarrow_{\mathcal N} x}

We write $P \Downarrow_{\mathcal N} x$ if there is $Q$ such that 
$P \wred Q$ and $Q \downarrow_{\mathcal N} x$.
\end{definition}

\begin{definition}
%\label{def.bbisim}
An  ${\mathcal N}$-\emph{barbed bisimulation} over a set of names, ${\mathcal N}$, is a symmetric binary relation 
${\mathcal S}_{\mathcal N}$ between agents such that $P\rel{S}_{\mathcal N}Q$ implies:
\begin{enumerate}
\item If $P \red P'$ then $Q \wred Q'$ and $P'\rel{S}_{\mathcal N} Q'$.
\item If $P\downarrow_{\mathcal N} x$, then $Q\Downarrow_{\mathcal N} x$.
\end{enumerate}
$P$ is ${\mathcal N}$-barbed bisimilar to $Q$, written
$P \wbbisim_{\mathcal N} Q$, if $P \rel{S}_{\mathcal N} Q$ for some ${\mathcal N}$-barbed bisimulation ${\mathcal S}_{\mathcal N}$.
\end{definition}

$\mathcal{R} \subseteq \pi \times \pi$

$P \mathcal{R} Q => \forall P'. P \red P' \Rightarrow \exists Q'. Q \red Q', P' \mathcal{R} Q'$

$P \vdash x \Rightarrow Q \vdash x$

\begin{mathpar}
  \inferrule*[lab=Out-barb]{x \nameeq y}{{y}!\langle{Q}\rangle \vdash x}
  \and
  \inferrule*[lab=Par-barb]{\mbox{$P\vdash x$ or $Q\vdash x$}}{\binpar{P}{Q} \vdash x}
\end{mathpar}

\subsubsection{Contexts}

One of the principle advantages of computational calculi like the
$\pi$-calculus is a well-defined notion of context,
contextual-equivalence and a correlation between
contextual-equivalence and notions of bisimulation. The notion of
context allows the decomposition of a process into (sub-)process and
its syntactic environment, its context. Thus, a context may be
thought of as a process with a ``hole'' (written $\Box$) in it. The
application of a context $M$ to a process $P$, written $M[P]$, is
tantamount to filling the hole in $M$ with $P$. In this paper we do
not need the full weight of this theory, but do make use of the notion
of context in the proof the main theorem. 

\begin{mathpar}
  \inferrule* [lab=summation] {} {{M_{M},M_{N}} \bc \Box \;|\; x.M_{A} \;|\; M_{M}+M_{N}}
  \and
  \inferrule* [lab=agent] {} {{M_{A}} \bc (\vec{x})M_{P} \;| \; \clift{P_0,\ldots,M_{P},\ldots,P_N}}
  \and \\
  \inferrule* [lab=process] {} {{M_{P}} \bc M_{N} \;| \;P|M_{P} }
\end{mathpar} 

\begin{mathpar}
  \inferrule* [lab=sychronization] {} {M_{N} \bc \Box \;|\; x?M_{F} \;|\; x!M_{C}}
  \and
  \inferrule* [lab=abstraction] {} {{M_{F}} \bc (x)M_{P} }
  \and
  \inferrule* [lab=concretion] {} {{M_{C}} \bc \langle M_{P} \rangle }
  \and \\
  \inferrule* [lab=process] {} {{M_{P}} \bc M_{N} \;| \;P|M_{P} }
\end{mathpar}

\begin{definition}[contextual application] Given a context $M$, and
  process $P$, we define the \emph{contextual application}, $M[P] :=
  M\{P/\Box\}$. That is, the contextual application of M to P is the
  substitution of $P$ for $\Box$ in $M$.
\end{definition}

$\meaningof{-} : L \to \mathcal{P}(\pi)$

\begin{mathpar}
  \inferrule* [lab=collection] {} {\meaningof{true} = \pi, \and \meaningof{~E} = \pi \setminus \meaningof{E}, \and \meaningof{E_{1} \& E_{2}} = \meaningof{E_{1}} \cap \meaningof{E_{2}}}
\end{mathpar}

\begin{mathpar}
  \inferrule* [lab=structure] {} {\meaningof{0} = \{ P \in \pi | P \equiv 0 \}, \and \\ \meaningof{E_1 | E_2} = \{ P \in \pi | P \equiv P_{1} | P_{2}, P_{1} \in \meaningof{E_{1}}, P_{2} \in \meaningof{E_2}\} }
\end{mathpar}

\begin{mathpar}
 \inferrule* [lab=behavior] {} {\meaningof{\langle a?b \rangle E} = \{ P \in \pi | P \equiv Q | u?(y)P', \\ \and \\\\ \and \\ \;\;\; u \in \meaningof{a}, \forall z.P'\{z/y\} \in \meaningof{E\{z/b\}}\}, \and \\ \meaningof{a!E} = \{ P \in \pi | P \equiv Q | x!\langle P' \rangle, x \in \meaningof{a} P' \in \meaningof{E}\} }
\end{mathpar}

\begin{mathpar}
 \inferrule* [lab=nominal] {} {\meaningof{\quotep{E}} = \{ \quotep{P} \in \quotep{\pi} | P \in \meaningof{E} \}, \and \meaningof{\quotep{P}} = \{ \quotep{Q} \in \quotep{\pi} | P \equiv Q \} \and \\ \meaningof{@\quotep{E}} = \{ P \in \pi | P \equiv @x, x \in \meaningof{E} \}}
\end{mathpar}

\begin{eqnarray*}
  \\
  \meaningof{-} : TS \to ST
\end{eqnarray*}

\begin{eqnarray*}
  \\
  L : TS \to ST
\end{eqnarray*}

\begin{eqnarray*}
  \\
  P \models E \iff P \in \meaningof{E}
\end{eqnarray*}

\begin{eqnarray*}
  P \approx_{L} Q \iff \forall E \in L. P \models E \iff Q \models E
\end{eqnarray*}

\begin{eqnarray*}
  P \approx_{K} Q
\end{eqnarray*}

\begin{eqnarray*}
  P \approx Q
\end{eqnarray*}

$\approx_{K} = \approx = \approx_{L}$

\subsubsection{Contextual duality}

Note that contexts extend the quotation operation to a family of
operations from processes to names. Given a context, $M$, we can
define a \emph{nominal context}, $\quotep{M}$ by $\quotep{M}[P] :=
\quotep{M[P]}$. To foreshadow what is to come we observe that these
operations enjoy a duality with processes very much like the duality
between vectors and maps from vectors to scalars.

Further, because the calculus is essentially higher-order, we have a
correspondence between contexts and processes. More specifically,
given a name $x$ and a context $M$ we can construct $M^{*}_{x}$ such
that 

\begin{mathpar}
  M^{*}_{x} | \lift{x}{P} \red M[P]
\end{mathpar}

namely,

\begin{mathpar}
  M^{*}_{x} := x?(u).M[\dropn{u}]
\end{mathpar}

The dependence of $M^{*}_{x}$ on a name makes it an abstraction, 

\begin{mathpar}
  M^{*} := (x)x?(u).M[\dropn{u}]
\end{mathpar}

\subsection{Additional notation}

It will sometimes be convenient to denote the process a name
quotes. We already have the notation $x = \quotep{P}$, but it will be
convenient to introduce an alternate notation, $\procn{x}$, when we
want to emphasize the connection to the use of the name. Note that, by
virtue of name equivalence, $\quotep{\procn{x}} \nameeq x$; so, the
notation is consistent with previous definitions.

Further, because names have structure it is possible to effect
substitutions on the basis of that structure. This means we need to
upgrade our notation for substitutions, which we accomplish by
adapting comprehension notation. Thus,

\begin{mathpar}
  P\{ y / x : x \in S \}
\end{mathpar}

is interpreted to mean the process derived from P by replacing (in a
capture-avoiding manner) each occurrence of $x$ in $S$ by $y$. For example,

\begin{mathpar}
  P\{ \quotep{\procn{x}|\procn{x}} / x : x \in \freenames{P} \}
\end{mathpar}

will replace each (occurrence) of a free name $x$ in $P$ by
$\quotep{\procn{x}|\procn{x}}$.

Also, we will avail ourselves of the notation $x^{L}$ and $x^{R}$ to
denote injections of a name into disjoint copies of the name
space. There are numerous ways to accomplish this. One example can be
found in \cite{MeredithR05}. This notation overloads to vectors of
names: $\vec{x}^{\pi} := (x_{i}^{\pi} \; : \; 0 \leq i < |\vec{x}| )$ where $\pi \in \{L,R\}$.

We also use $P^{\Box} := P|\Box$.

In \cite{MeredithR05} an interpretation of the new operator is
given. It turns out that there are several possible interpretations
all enjoying the requisite algebraic properties of the operator (see
\cite{milner91polyadicpi}). We will therefore make liberal use of
$(\nu\; \vec{x})P$.

% subsection the_syntax_and_semantics_of_the_notation_system (end)   

\input{qm2pi.qmops} 

\input{qm2pi.sterngerlach} 

\input{qm2pi.metric} 

% section concurrent_process_calculi (end)

%\input{qm2pi.proofsketch}

% section proof sketch (end)

%\input{qm2pi.slviaknots} 

% section spatial logic via knots (end)

\input{qm2pi.conclusion}

% section conclusion (end)

%\input{qm2pi.dtcodes} 

% section wiring algorithm (end)

\input{qm2pi.ack} 

% section acknowledgments (end)

\newpage


\bibliographystyle{plain}   
\bibliography{../../biblios/main.bib}

\input{qm2pi.rhodetails}

\end{document}

 

%\documentclass[12pt]{llncs}
%\documentclass{jktr}

\usepackage[pdftex]{hyperref}                   
\usepackage {listings}
\usepackage {mathpartir}
\usepackage{bcprules}
%\usepackage{listings}
                       
\usepackage{graphicx} 
%\usepackage[margins=2.5cm,nohead,nofoot]{geometry}
%\usepackage{geometry}
\usepackage{amsfonts}
\usepackage{amstext}
\usepackage{latexsym}
\usepackage{amssymb}
\usepackage{color}


%\include{myPreamble}
\include{qm2pi.local} 

%\ifpdf
%\usepackage[pdftex]{graphicx}
%\else
%\usepackage{graphicx}
%\fi

 % \ifpdf
%  \usepackage{pdfsync}
%  \if


%\title{Brief Article}
%\author{David F. Snyder}
%\author{L.G. Meredith}

%\address{Dept. of Math., Texas State University--San Marcos, San Marcos, TX 78666}
       
\pagestyle{empty}


\begin{document}

\lstset{language=[Objective]Caml,frame=shadowbox}

\input{qm2pi.front}

% section front matter (end)

\input{qm2pi.intro} 
 
% section introduction (end)

% \input{qm2pi.knotations} 

% section notation (end)

\input{qm2pi.process.calculi} 

% section concurrent_process_calculi_and_spatial_logics_ (end)
    
%\input{qm2pi.knots2pi} 

%\input{qm2pi.trefoil} 

%\input{qm2pi.mainthm} 

% subsection basic_interpretation (end)

%\input{qm2pi.rho.presentation} 
\subsection{The syntax and semantics of the notation system}\label{sub:the_syntax_and_semantics_of_the_notation_system} % (fold)

We now summarize a technical presentation of the calculus that
embodies our theory of dynamics. The typical presentation of such a
calculus follows the style of giving generators and relations on
them. The grammar, below, describing term constructors, freely
generates the set of processes, $\Proc$. This set is then quotiented
by a relation known as structural congruence and it is over this set
that the notion of dynamics is expressed. This presentation is
essentially that of \cite{MeredithR05} with the addition of
polyadicity and summation. For readability we have relegated some of
the technical subtleties to an appendix.

\subsubsection{Process grammar}\label{subsub:process_grammar}

\begin{mathpar}
  \inferrule* [lab=synchronization] {} {{M} \bc \pzero \;|\; x?F \;|\; x!C }
  \and
  \inferrule* [lab=abstraction] {} {{F} \bc (x)P}
  \and
  \inferrule* [lab=concretion] {} {{C} \bc \langle Q \rangle}
  \and
  \inferrule* [lab=process] {} {{P,Q} \bc M \;| \;P|Q \;|\; @{x}}
  \and
  \inferrule* [lab=name] {} {{x} \bc \quotep{P}}
\end{mathpar} 

Note that $\vec{x}$ (resp. $\vec{P}$) denotes a vector of names
(resp. processes) of length $|\vec{x}|$ (resp. $|\vec{P}|$). We adopt
the following useful abbreviations.

\begin{mathpar}
   x?(\vec{y}).P := x.(\vec{y})P \and  x\clift{\vec{P}} := x.\clift{\vec{P}}
   \and x!(y) := \lift{x}{\dropn{y}}
   \and \Pi_{i=0}^{n-1}P_i := P_0 | \ldots | P_{n-1}
\end{mathpar}

\subsubsection{Structural congruence}

\paragraph{Free and bound names and alpha-equivalence.} At the
core of structural equivalence is alpha-equivalence which identifies
process that are the same up to a change of variable. Formally, we
recognize the distinction between free and bound names. The free names
of a process, $\freenames{P}$, may be calculated recursively as
follows:

\begin{mathpar}
\freenames{\pzero} := \emptyset
  \and \\
  \freenames{x?(y).P} := \{ x \} \cup (\freenames{P} \setminus \{ y \})
  \and 
  \freenames{x!\langle P \rangle} := \{ x \} \cup \{ P \} 
  \and \\
  \freenames{P|Q} := \freenames{P} \cup \freenames{Q}
  \and \\
  \freenames{@{x}} := \{ x \}
\end{mathpar}

$\pi$
$\quotep{\pi}$

$\freenames{-} : \pi \to \mathcal{P}(\quotep{\pi})$

\begin{eqnarray*}
  \freenames{\pzero} & := & \emptyset \\
  \freenames{x?(y).P} & := & \{ x \} \cup (\freenames{P} \setminus \{ y \}) \\
  \freenames{x!\langle P \rangle} & := & \{ x \} \cup \{ P \} \\
  \freenames{P|Q} & := & \freenames{P} \cup \freenames{Q} \\
  \freenames{\dropn{x}} & := & \{ x \}
\end{eqnarray*}

The bound names of a process, $\boundnames{P}$, are those names occurring in $P$
that are not free. For example, in $x?(y).0$, the name $x$ is free, while $y$ is bound.

\begin{mathpar}
  \inferrule* [lab=monoidal-laws] {} { P|Q \equiv Q|P \and P|0 \equiv P \and P|(Q|R) \equiv (P|Q)|R }
\end{mathpar}

\begin{mathpar}
  \inferrule* [lab=alpha-equivalence] {} { (x)P \equiv (y)P\{y/x\} \and y \not\in \freenames{P} }
\end{mathpar}

\begin{definition}
Then two processes, $P,Q$, are alpha-equivalent if $P = Q\{\vec{y}/\vec{x}\}$ for
some $\vec{x} \in \boundnames{Q},\vec{y} \in \boundnames{P}$, where $Q\{\vec{y}/\vec{x}\}$
denotes the capture-avoiding substitution of $\vec{y}$ for $\vec{x}$ in $Q$.
\end{definition}

\begin{definition}
  The {\em structural congruence} \cite{SangiorgiWalker} , $\equiv$,
  between processes is the least congruence containing
  alpha-equivalence, satisfying the abelian monoid laws
  (associativity, commutativity and $\pzero$ as identity) for parallel
  composition $|$ and for summation $+$.
\end{definition}

\subsection{Name equivalence}

We take name equivalence, written $\nameeq$, to be the smallest
equivalence relation generated by the following rules.

\begin{mathpar}
\inferrule*[lab=Quote-drop]
{ }
{ \quotep{@{x}} \nameeq x }

\inferrule*[lab=Struct-equiv]
{ P \scong Q }
{ \quotep{P} \nameeq \quotep{Q} }
\end{mathpar}

The astute reader will have noticed that the mutual recursion of names
and processes imposes a mutual recursion on alpha-equivalence and
structural equivalence via name-equivalence. Fortunately, all of this
works out pleasantly and we may calculate in the natural way, free of
concern. The reader interested in the details is referred to the
appendix \ref{appendix:rho_details}.

\subsection{Substitution}

We use $\Proc$ for the set of processes, $\QProc$ for the set of
names, and $\id{\{}\vec{y} / \vec{x} \id{\}}$ to denote partial maps,
$s : \QProc \rightarrow \QProc$. A map, $s$ lifts, uniquely, to a map
on process terms, $\widehat{s} : \Proc \rightarrow \Proc$ by the
following equations.

\begin{mathpar}
  (0) \psubstp{Q}{P} := 0 \\
  (R \juxtap S) \psubstp{Q}{P}
  :=    
  (R)\psubstp{Q}{P} \juxtap (S) \psubstp{Q}{P} \\
  (x?(y).R) \psubstp{Q}{P}    
  :=    
  (x)\substp{Q}{P} (z)\concat( (R \psubstn{z}{y}) \psubstp{Q}{P} ) \\
  (\lift{x}{R}) \psubstp{Q}{P}  
  :=
  \lift{(x)\substp{Q}{P}}{ R \psubstp{Q}{P} } \\
%   (\dropn{x})  \psubstp{Q}{P}       
%   := 
%   \left\{ 
%     \begin{array}{ccc} 
%       \dropn{\quotep{Q}} & & x \nameeq \quotep{P} \\
%       \dropn{x} & & otherwise \\
%     \end{array}
%   \right. 
  (\dropn{x})  \psubstp{Q}{P}       
  := 
  \left\{ 
    \begin{array}{ccc} 
      Q & & x \nameeq \quotep{P} \\
      \dropn{x} & & otherwise \\
    \end{array}
  \right.
\end{mathpar}
 

where

\begin{eqnarray}
  (x)\id{\{} \lpquote Q \rpquote / \lpquote P \rpquote \id{\}}            = 
  \left\{ 
    \begin{array}{ccc}
      \lpquote Q \rpquote & & x \nameeq \lpquote P \rpquote \\
      x & & otherwise \\
    \end{array}
  \right. \nonumber
\end{eqnarray}

and $z$ is chosen distinct from $\quotep{P}$, $\quotep{Q}$, the free
names in $Q$, and all the names in $R$. Our $\alpha$-equivalence will
be built in the standard way from this substitution.

\begin{remark}\label{rem:no_self_referential_names}
  One consequence of these definitions is that $\forall P. \quotep{P}
  \not\in \freenames{P}$.
\end{remark}

\subsection{ Dynamic quote: an example }

Anticipating something of what's to come, consider applying the
substitution, $\widehat{\id{\{}u / z \id{\}}}$, to the following pair
of processes, $\lift{w}{y!(z)}$ and $w[ \lpquote y!(z) \rpquote ]$.

\begin{eqnarray}
	\lift{w}{y!(z)}\widehat{\id{\{}u / z \id{\}}}
		& = &
		\lift{w}{y!(u)} \nonumber\\
	w[ \lpquote y!(z) \rpquote ] \widehat{ \id{\{}u / z \id{\}} }
		& = &
		w[ \lpquote y!(z) \rpquote ] \nonumber
\end{eqnarray}

Because the body of the process between quotes is impervious to
substitution, we get radically different answers. In fact, by
examining the first process in an input context,
e.g. $x?(z).\lift{w}{y!(z)}$, we see that the process under the lift
operator may be shaped by prefixed inputs binding a name inside it. In
this sense, the lift operator will be seen as a way to dynamically
construct processes before reifying them as names.

Finally equipped with these standard features we can present the
dynamics of the calculus.

\subsubsection{Operational semantics} 

Finally, we introduce the computational dynamics. What marks these
algebras as distinct from other more traditionally studied algebraic
structures, e.g. vector spaces or polynomial rings, is the manner in
which dynamics is captured. In traditional structures, dynamics is typically
expressed through morphisms between such structures, as in linear maps
between vector spaces or morphisms between rings. In algebras
associated with the semantics of computation, the dynamics is
expressed as part of the algebraic structure itself, through a
reduction reduction relation typically denoted by $\red$. Below, we
give a recursive presentation of this relation for the calculus used
in the encoding.

$\red \subseteq \pi \times \pi$
$\red : \pi \to \mathcal{P}(\pi)$

\begin{mathpar}
  \inferrule* [lab=Comm] { \textsf{match}( x_{src}, x_{trgt} ) } { x_{trgt}?(y)P \; | \; x_{src}!\langle {Q} \rangle \red P\{\quotep{Q}/y}\} }
  \and \\
  \inferrule* [lab=Par] {{P} \red {P}'} {{{P} | {Q}} \red {{P}' | {Q}}}
  \and
  \inferrule* [lab=Equiv]{{{P} \scong {P}'} \andalso {{P}' \red {Q}'} \andalso {{Q}' \scong {Q}}}{{P} \red {Q}}
\end{mathpar}

\begin{eqnarray*}
  match_{\equiv} (\quotep{P},\quotep{Q}) & := & P \equiv Q \\
  match_{\dagger}(\quotep{P},\quotep{Q}) & := & \forall R. P|Q \red^{*} R => R \red^{*} 0 \\
  match_{K}(\quotep{P},\quotep{Q}) & := & K \mbox{ for some context } K
\end{eqnarray*}

$u?(x)P | u!\langle Q \rangle \red P\{\quotep{Q}/x\}$

%We write $\wred$ for $\red^*$, and $P\red$ if $\exists Q $ such that $ P \red Q$.
We write $P\red$ if $\exists Q $ such that $ P \red Q$ and $P\not\red$, otherwise.

\section{Replication}

As mentioned before, it is known that replication (and hence
recursion) can be implemented in a higher-order process algebra
\cite{SangiorgiWalker}. As our first example of calculation with the
machinery thus far presented we give the construction explicitly in
the {\rhoc}.

\begin{eqnarray}
	D_{x} & := & \prefix{x}{y}{(\binpar{\outputp{x}{y}}{@{y}})} \nonumber\\
	\bangp_{x}{P} & := & \binpar{{x}!\langle{\binpar{D_{x}}{P}}\rangle}{D_{x}} \nonumber
\end{eqnarray}

\begin{eqnarray}
	\bangp_{x}{P} & & \nonumber\\
	=
	& {x}!\langle{(\prefix{x}{y}{(\outputp{x}{y} | @{y})) | P}}\rangle 
	      | \prefix{x}{y}{(\outputp{x}{y} | @{y})} & \nonumber\\
	\red
	& (\outputp{x}{y} | @{y})\substn{\quotep{(\prefix{x}{y}{(@{y} | \outputp{x}{y})) | P}}}{y} & \nonumber\\
	=
	& \outputp{x}{\quotep{(\prefix{x}{y}{(\outputp{x}{y} | @{y})) | P}}}
	  | {(\prefix{x}{y}{(\outputp{x}{y} | @{y})) | P}} & \nonumber\\
	\red
	& \ldots & \nonumber\\
	\red^*
	& P | P | \ldots & \nonumber
\end{eqnarray}

Of course, this encoding, as an implementation, runs away, unfolding
$\bangp{P}$ eagerly. A lazier and more implementable replication
operator, restricted to input-guarded processes, may be obtained as follows.

\begin{eqnarray}
\bangp{\prefix{u}{v}{P}} 
	:= 
	\binpar{\lift{x}{\prefix{u}{v}{(\binpar{D(x)}{P})}}}{D(x)} \nonumber
\end{eqnarray}

\begin{remark}
  Note that the lazier definition still does not deal with summation
  or mixed summation (i.e. sums over input and output). The reader is
  invited to construct definitions of replication that deal with these
  features. 

  Further, the definitions are parameterized in a name, $x$. Can you,
  gentle reader, make a definition that eliminates this parameter and
  guarantees no accidental interaction between the replication
  machinery and the process being replicated -- i.e. no accidental
  sharing of names used by the process to get its work done and the
  name(s) used by the replication to effect copying. This latter
  revision of the definition of replication is crucial to obtaining
  the expected identity $!!P \sim !P$.
\end{remark}

\begin{remark}\label{rem:paradoxical_combinator}
  The reader familiar with the lambda calculus will have noticed the
  similarity between $D$ and the paradoxical combinator.

  [Ed. note: the existence of this seems to suggest we have to be more
  restrictive on the set of processes and names we admit if we are to
  support no-cloning.]
\end{remark}

\subsubsection{Bisimulation}

The computational dynamics gives rise to another kind of equivalence,
the equivalence of computational behavior. As previously mentioned
this is typically captured \emph{via} some form of bisimulation.

% The notion we use in this paper is weak barbed bisimulation
% \cite{milner91polyadicpi}.

The notion we use in this paper is derived from weak barbed
bisimulation \cite{milner91polyadicpi}. 

\begin{definition}
An \emph{observation relation}, $\downarrow_{\mathcal N}$, over a set
of names, $\mathcal N$, is the smallest relation satisfying the rules
below.

\infrule[Out-barb]{y \in {\mathcal N}, \; x \nameeq y}
		  {\outputp{x}{v} \downarrow_{\mathcal N} x}
\infrule[Par-barb]{\mbox{$P\downarrow_{\mathcal N} x$ or $Q\downarrow_{\mathcal N} x$}}
		  {\binpar{P}{Q} \downarrow_{\mathcal N} x}

We write $P \Downarrow_{\mathcal N} x$ if there is $Q$ such that 
$P \wred Q$ and $Q \downarrow_{\mathcal N} x$.
\end{definition}

\begin{definition}
%\label{def.bbisim}
An  ${\mathcal N}$-\emph{barbed bisimulation} over a set of names, ${\mathcal N}$, is a symmetric binary relation 
${\mathcal S}_{\mathcal N}$ between agents such that $P\rel{S}_{\mathcal N}Q$ implies:
\begin{enumerate}
\item If $P \red P'$ then $Q \wred Q'$ and $P'\rel{S}_{\mathcal N} Q'$.
\item If $P\downarrow_{\mathcal N} x$, then $Q\Downarrow_{\mathcal N} x$.
\end{enumerate}
$P$ is ${\mathcal N}$-barbed bisimilar to $Q$, written
$P \wbbisim_{\mathcal N} Q$, if $P \rel{S}_{\mathcal N} Q$ for some ${\mathcal N}$-barbed bisimulation ${\mathcal S}_{\mathcal N}$.
\end{definition}

$\mathcal{R} \subseteq \pi \times \pi$

$P \mathcal{R} Q => \forall P'. P \red P' \Rightarrow \exists Q'. Q \red Q', P' \mathcal{R} Q'$

$P \vdash x \Rightarrow Q \vdash x$

\begin{mathpar}
  \inferrule*[lab=Out-barb]{x \nameeq y}{{y}!\langle{Q}\rangle \vdash x}
  \and
  \inferrule*[lab=Par-barb]{\mbox{$P\vdash x$ or $Q\vdash x$}}{\binpar{P}{Q} \vdash x}
\end{mathpar}

\subsubsection{Contexts}

One of the principle advantages of computational calculi like the
$\pi$-calculus is a well-defined notion of context,
contextual-equivalence and a correlation between
contextual-equivalence and notions of bisimulation. The notion of
context allows the decomposition of a process into (sub-)process and
its syntactic environment, its context. Thus, a context may be
thought of as a process with a ``hole'' (written $\Box$) in it. The
application of a context $M$ to a process $P$, written $M[P]$, is
tantamount to filling the hole in $M$ with $P$. In this paper we do
not need the full weight of this theory, but do make use of the notion
of context in the proof the main theorem. 

\begin{mathpar}
  \inferrule* [lab=summation] {} {{M_{M},M_{N}} \bc \Box \;|\; x.M_{A} \;|\; M_{M}+M_{N}}
  \and
  \inferrule* [lab=agent] {} {{M_{A}} \bc (\vec{x})M_{P} \;| \; \clift{P_0,\ldots,M_{P},\ldots,P_N}}
  \and \\
  \inferrule* [lab=process] {} {{M_{P}} \bc M_{N} \;| \;P|M_{P} }
\end{mathpar} 

\begin{mathpar}
  \inferrule* [lab=sychronization] {} {M_{N} \bc \Box \;|\; x?M_{F} \;|\; x!M_{C}}
  \and
  \inferrule* [lab=abstraction] {} {{M_{F}} \bc (x)M_{P} }
  \and
  \inferrule* [lab=concretion] {} {{M_{C}} \bc \langle M_{P} \rangle }
  \and \\
  \inferrule* [lab=process] {} {{M_{P}} \bc M_{N} \;| \;P|M_{P} }
\end{mathpar}

\begin{definition}[contextual application] Given a context $M$, and
  process $P$, we define the \emph{contextual application}, $M[P] :=
  M\{P/\Box\}$. That is, the contextual application of M to P is the
  substitution of $P$ for $\Box$ in $M$.
\end{definition}

$\meaningof{-} : L \to \mathcal{P}(\pi)$

\begin{mathpar}
  \inferrule* [lab=collection] {} {\meaningof{true} = \pi, \and \meaningof{~E} = \pi \setminus \meaningof{E}, \and \meaningof{E_{1} \& E_{2}} = \meaningof{E_{1}} \cap \meaningof{E_{2}}}
\end{mathpar}

\begin{mathpar}
  \inferrule* [lab=structure] {} {\meaningof{0} = \{ P \in \pi | P \equiv 0 \}, \and \\ \meaningof{E_1 | E_2} = \{ P \in \pi | P \equiv P_{1} | P_{2}, P_{1} \in \meaningof{E_{1}}, P_{2} \in \meaningof{E_2}\} }
\end{mathpar}

\begin{mathpar}
 \inferrule* [lab=behavior] {} {\meaningof{\langle a?b \rangle E} = \{ P \in \pi | P \equiv Q | u?(y)P', \\ \and \\\\ \and \\ \;\;\; u \in \meaningof{a}, \forall z.P'\{z/y\} \in \meaningof{E\{z/b\}}\}, \and \\ \meaningof{a!E} = \{ P \in \pi | P \equiv Q | x!\langle P' \rangle, x \in \meaningof{a} P' \in \meaningof{E}\} }
\end{mathpar}

\begin{mathpar}
 \inferrule* [lab=nominal] {} {\meaningof{\quotep{E}} = \{ \quotep{P} \in \quotep{\pi} | P \in \meaningof{E} \}, \and \meaningof{\quotep{P}} = \{ \quotep{Q} \in \quotep{\pi} | P \equiv Q \} \and \\ \meaningof{@\quotep{E}} = \{ P \in \pi | P \equiv @x, x \in \meaningof{E} \}}
\end{mathpar}

\begin{eqnarray*}
  \\
  \meaningof{-} : TS \to ST
\end{eqnarray*}

\begin{eqnarray*}
  \\
  L : TS \to ST
\end{eqnarray*}

\begin{eqnarray*}
  \\
  P \models E \iff P \in \meaningof{E}
\end{eqnarray*}

\begin{eqnarray*}
  P \approx_{L} Q \iff \forall E \in L. P \models E \iff Q \models E
\end{eqnarray*}

\begin{eqnarray*}
  P \approx_{K} Q
\end{eqnarray*}

\begin{eqnarray*}
  P \approx Q
\end{eqnarray*}

$\approx_{K} = \approx = \approx_{L}$

\subsubsection{Contextual duality}

Note that contexts extend the quotation operation to a family of
operations from processes to names. Given a context, $M$, we can
define a \emph{nominal context}, $\quotep{M}$ by $\quotep{M}[P] :=
\quotep{M[P]}$. To foreshadow what is to come we observe that these
operations enjoy a duality with processes very much like the duality
between vectors and maps from vectors to scalars.

Further, because the calculus is essentially higher-order, we have a
correspondence between contexts and processes. More specifically,
given a name $x$ and a context $M$ we can construct $M^{*}_{x}$ such
that 

\begin{mathpar}
  M^{*}_{x} | \lift{x}{P} \red M[P]
\end{mathpar}

namely,

\begin{mathpar}
  M^{*}_{x} := x?(u).M[\dropn{u}]
\end{mathpar}

The dependence of $M^{*}_{x}$ on a name makes it an abstraction, 

\begin{mathpar}
  M^{*} := (x)x?(u).M[\dropn{u}]
\end{mathpar}

\subsection{Additional notation}

It will sometimes be convenient to denote the process a name
quotes. We already have the notation $x = \quotep{P}$, but it will be
convenient to introduce an alternate notation, $\procn{x}$, when we
want to emphasize the connection to the use of the name. Note that, by
virtue of name equivalence, $\quotep{\procn{x}} \nameeq x$; so, the
notation is consistent with previous definitions.

Further, because names have structure it is possible to effect
substitutions on the basis of that structure. This means we need to
upgrade our notation for substitutions, which we accomplish by
adapting comprehension notation. Thus,

\begin{mathpar}
  P\{ y / x : x \in S \}
\end{mathpar}

is interpreted to mean the process derived from P by replacing (in a
capture-avoiding manner) each occurrence of $x$ in $S$ by $y$. For example,

\begin{mathpar}
  P\{ \quotep{\procn{x}|\procn{x}} / x : x \in \freenames{P} \}
\end{mathpar}

will replace each (occurrence) of a free name $x$ in $P$ by
$\quotep{\procn{x}|\procn{x}}$.

Also, we will avail ourselves of the notation $x^{L}$ and $x^{R}$ to
denote injections of a name into disjoint copies of the name
space. There are numerous ways to accomplish this. One example can be
found in \cite{MeredithR05}. This notation overloads to vectors of
names: $\vec{x}^{\pi} := (x_{i}^{\pi} \; : \; 0 \leq i < |\vec{x}| )$ where $\pi \in \{L,R\}$.

We also use $P^{\Box} := P|\Box$.

In \cite{MeredithR05} an interpretation of the new operator is
given. It turns out that there are several possible interpretations
all enjoying the requisite algebraic properties of the operator (see
\cite{milner91polyadicpi}). We will therefore make liberal use of
$(\nu\; \vec{x})P$.

% subsection the_syntax_and_semantics_of_the_notation_system (end)   

\input{qm2pi.qmops} 

\input{qm2pi.sterngerlach} 

\input{qm2pi.metric} 

% section concurrent_process_calculi (end)

%\input{qm2pi.proofsketch}

% section proof sketch (end)

%\input{qm2pi.slviaknots} 

% section spatial logic via knots (end)

\input{qm2pi.conclusion}

% section conclusion (end)

%\input{qm2pi.dtcodes} 

% section wiring algorithm (end)

\input{qm2pi.ack} 

% section acknowledgments (end)

\newpage


\bibliographystyle{plain}   
\bibliography{../../biblios/main.bib}

\input{qm2pi.rhodetails}

\end{document}

 

% subsection basic_interpretation (end)

%\input{qm2pi.rho.presentation} 
\subsection{The syntax and semantics of the notation system}\label{sub:the_syntax_and_semantics_of_the_notation_system} % (fold)

We now summarize a technical presentation of the calculus that
embodies our theory of dynamics. The typical presentation of such a
calculus follows the style of giving generators and relations on
them. The grammar, below, describing term constructors, freely
generates the set of processes, $\Proc$. This set is then quotiented
by a relation known as structural congruence and it is over this set
that the notion of dynamics is expressed. This presentation is
essentially that of \cite{MeredithR05} with the addition of
polyadicity and summation. For readability we have relegated some of
the technical subtleties to an appendix.

\subsubsection{Process grammar}\label{subsub:process_grammar}

\begin{mathpar}
  \inferrule* [lab=synchronization] {} {{M} \bc \pzero \;|\; x?F \;|\; x!C }
  \and
  \inferrule* [lab=abstraction] {} {{F} \bc (x)P}
  \and
  \inferrule* [lab=concretion] {} {{C} \bc \langle Q \rangle}
  \and
  \inferrule* [lab=process] {} {{P,Q} \bc M \;| \;P|Q \;|\; @{x}}
  \and
  \inferrule* [lab=name] {} {{x} \bc \quotep{P}}
\end{mathpar} 

Note that $\vec{x}$ (resp. $\vec{P}$) denotes a vector of names
(resp. processes) of length $|\vec{x}|$ (resp. $|\vec{P}|$). We adopt
the following useful abbreviations.

\begin{mathpar}
   x?(\vec{y}).P := x.(\vec{y})P \and  x\clift{\vec{P}} := x.\clift{\vec{P}}
   \and x!(y) := \lift{x}{\dropn{y}}
   \and \Pi_{i=0}^{n-1}P_i := P_0 | \ldots | P_{n-1}
\end{mathpar}

\subsubsection{Structural congruence}

\paragraph{Free and bound names and alpha-equivalence.} At the
core of structural equivalence is alpha-equivalence which identifies
process that are the same up to a change of variable. Formally, we
recognize the distinction between free and bound names. The free names
of a process, $\freenames{P}$, may be calculated recursively as
follows:

\begin{mathpar}
\freenames{\pzero} := \emptyset
  \and \\
  \freenames{x?(y).P} := \{ x \} \cup (\freenames{P} \setminus \{ y \})
  \and 
  \freenames{x!\langle P \rangle} := \{ x \} \cup \{ P \} 
  \and \\
  \freenames{P|Q} := \freenames{P} \cup \freenames{Q}
  \and \\
  \freenames{@{x}} := \{ x \}
\end{mathpar}

$\pi$
$\quotep{\pi}$

$\freenames{-} : \pi \to \mathcal{P}(\quotep{\pi})$

\begin{eqnarray*}
  \freenames{\pzero} & := & \emptyset \\
  \freenames{x?(y).P} & := & \{ x \} \cup (\freenames{P} \setminus \{ y \}) \\
  \freenames{x!\langle P \rangle} & := & \{ x \} \cup \{ P \} \\
  \freenames{P|Q} & := & \freenames{P} \cup \freenames{Q} \\
  \freenames{\dropn{x}} & := & \{ x \}
\end{eqnarray*}

The bound names of a process, $\boundnames{P}$, are those names occurring in $P$
that are not free. For example, in $x?(y).0$, the name $x$ is free, while $y$ is bound.

\begin{mathpar}
  \inferrule* [lab=monoidal-laws] {} { P|Q \equiv Q|P \and P|0 \equiv P \and P|(Q|R) \equiv (P|Q)|R }
\end{mathpar}

\begin{mathpar}
  \inferrule* [lab=alpha-equivalence] {} { (x)P \equiv (y)P\{y/x\} \and y \not\in \freenames{P} }
\end{mathpar}

\begin{definition}
Then two processes, $P,Q$, are alpha-equivalent if $P = Q\{\vec{y}/\vec{x}\}$ for
some $\vec{x} \in \boundnames{Q},\vec{y} \in \boundnames{P}$, where $Q\{\vec{y}/\vec{x}\}$
denotes the capture-avoiding substitution of $\vec{y}$ for $\vec{x}$ in $Q$.
\end{definition}

\begin{definition}
  The {\em structural congruence} \cite{SangiorgiWalker} , $\equiv$,
  between processes is the least congruence containing
  alpha-equivalence, satisfying the abelian monoid laws
  (associativity, commutativity and $\pzero$ as identity) for parallel
  composition $|$ and for summation $+$.
\end{definition}

\subsection{Name equivalence}

We take name equivalence, written $\nameeq$, to be the smallest
equivalence relation generated by the following rules.

\begin{mathpar}
\inferrule*[lab=Quote-drop]
{ }
{ \quotep{@{x}} \nameeq x }

\inferrule*[lab=Struct-equiv]
{ P \scong Q }
{ \quotep{P} \nameeq \quotep{Q} }
\end{mathpar}

The astute reader will have noticed that the mutual recursion of names
and processes imposes a mutual recursion on alpha-equivalence and
structural equivalence via name-equivalence. Fortunately, all of this
works out pleasantly and we may calculate in the natural way, free of
concern. The reader interested in the details is referred to the
appendix \ref{appendix:rho_details}.

\subsection{Substitution}

We use $\Proc$ for the set of processes, $\QProc$ for the set of
names, and $\id{\{}\vec{y} / \vec{x} \id{\}}$ to denote partial maps,
$s : \QProc \rightarrow \QProc$. A map, $s$ lifts, uniquely, to a map
on process terms, $\widehat{s} : \Proc \rightarrow \Proc$ by the
following equations.

\begin{mathpar}
  (0) \psubstp{Q}{P} := 0 \\
  (R \juxtap S) \psubstp{Q}{P}
  :=    
  (R)\psubstp{Q}{P} \juxtap (S) \psubstp{Q}{P} \\
  (x?(y).R) \psubstp{Q}{P}    
  :=    
  (x)\substp{Q}{P} (z)\concat( (R \psubstn{z}{y}) \psubstp{Q}{P} ) \\
  (\lift{x}{R}) \psubstp{Q}{P}  
  :=
  \lift{(x)\substp{Q}{P}}{ R \psubstp{Q}{P} } \\
%   (\dropn{x})  \psubstp{Q}{P}       
%   := 
%   \left\{ 
%     \begin{array}{ccc} 
%       \dropn{\quotep{Q}} & & x \nameeq \quotep{P} \\
%       \dropn{x} & & otherwise \\
%     \end{array}
%   \right. 
  (\dropn{x})  \psubstp{Q}{P}       
  := 
  \left\{ 
    \begin{array}{ccc} 
      Q & & x \nameeq \quotep{P} \\
      \dropn{x} & & otherwise \\
    \end{array}
  \right.
\end{mathpar}
 

where

\begin{eqnarray}
  (x)\id{\{} \lpquote Q \rpquote / \lpquote P \rpquote \id{\}}            = 
  \left\{ 
    \begin{array}{ccc}
      \lpquote Q \rpquote & & x \nameeq \lpquote P \rpquote \\
      x & & otherwise \\
    \end{array}
  \right. \nonumber
\end{eqnarray}

and $z$ is chosen distinct from $\quotep{P}$, $\quotep{Q}$, the free
names in $Q$, and all the names in $R$. Our $\alpha$-equivalence will
be built in the standard way from this substitution.

\begin{remark}\label{rem:no_self_referential_names}
  One consequence of these definitions is that $\forall P. \quotep{P}
  \not\in \freenames{P}$.
\end{remark}

\subsection{ Dynamic quote: an example }

Anticipating something of what's to come, consider applying the
substitution, $\widehat{\id{\{}u / z \id{\}}}$, to the following pair
of processes, $\lift{w}{y!(z)}$ and $w[ \lpquote y!(z) \rpquote ]$.

\begin{eqnarray}
	\lift{w}{y!(z)}\widehat{\id{\{}u / z \id{\}}}
		& = &
		\lift{w}{y!(u)} \nonumber\\
	w[ \lpquote y!(z) \rpquote ] \widehat{ \id{\{}u / z \id{\}} }
		& = &
		w[ \lpquote y!(z) \rpquote ] \nonumber
\end{eqnarray}

Because the body of the process between quotes is impervious to
substitution, we get radically different answers. In fact, by
examining the first process in an input context,
e.g. $x?(z).\lift{w}{y!(z)}$, we see that the process under the lift
operator may be shaped by prefixed inputs binding a name inside it. In
this sense, the lift operator will be seen as a way to dynamically
construct processes before reifying them as names.

Finally equipped with these standard features we can present the
dynamics of the calculus.

\subsubsection{Operational semantics} 

Finally, we introduce the computational dynamics. What marks these
algebras as distinct from other more traditionally studied algebraic
structures, e.g. vector spaces or polynomial rings, is the manner in
which dynamics is captured. In traditional structures, dynamics is typically
expressed through morphisms between such structures, as in linear maps
between vector spaces or morphisms between rings. In algebras
associated with the semantics of computation, the dynamics is
expressed as part of the algebraic structure itself, through a
reduction reduction relation typically denoted by $\red$. Below, we
give a recursive presentation of this relation for the calculus used
in the encoding.

$\red \subseteq \pi \times \pi$
$\red : \pi \to \mathcal{P}(\pi)$

\begin{mathpar}
  \inferrule* [lab=Comm] { \textsf{match}( x_{src}, x_{trgt} ) } { x_{trgt}?(y)P \; | \; x_{src}!\langle {Q} \rangle \red P\{\quotep{Q}/y}\} }
  \and \\
  \inferrule* [lab=Par] {{P} \red {P}'} {{{P} | {Q}} \red {{P}' | {Q}}}
  \and
  \inferrule* [lab=Equiv]{{{P} \scong {P}'} \andalso {{P}' \red {Q}'} \andalso {{Q}' \scong {Q}}}{{P} \red {Q}}
\end{mathpar}

\begin{eqnarray*}
  match_{\equiv} (\quotep{P},\quotep{Q}) & := & P \equiv Q \\
  match_{\dagger}(\quotep{P},\quotep{Q}) & := & \forall R. P|Q \red^{*} R => R \red^{*} 0 \\
  match_{K}(\quotep{P},\quotep{Q}) & := & K \mbox{ for some context } K
\end{eqnarray*}

$u?(x)P | u!\langle Q \rangle \red P\{\quotep{Q}/x\}$

%We write $\wred$ for $\red^*$, and $P\red$ if $\exists Q $ such that $ P \red Q$.
We write $P\red$ if $\exists Q $ such that $ P \red Q$ and $P\not\red$, otherwise.

\section{Replication}

As mentioned before, it is known that replication (and hence
recursion) can be implemented in a higher-order process algebra
\cite{SangiorgiWalker}. As our first example of calculation with the
machinery thus far presented we give the construction explicitly in
the {\rhoc}.

\begin{eqnarray}
	D_{x} & := & \prefix{x}{y}{(\binpar{\outputp{x}{y}}{@{y}})} \nonumber\\
	\bangp_{x}{P} & := & \binpar{{x}!\langle{\binpar{D_{x}}{P}}\rangle}{D_{x}} \nonumber
\end{eqnarray}

\begin{eqnarray}
	\bangp_{x}{P} & & \nonumber\\
	=
	& {x}!\langle{(\prefix{x}{y}{(\outputp{x}{y} | @{y})) | P}}\rangle 
	      | \prefix{x}{y}{(\outputp{x}{y} | @{y})} & \nonumber\\
	\red
	& (\outputp{x}{y} | @{y})\substn{\quotep{(\prefix{x}{y}{(@{y} | \outputp{x}{y})) | P}}}{y} & \nonumber\\
	=
	& \outputp{x}{\quotep{(\prefix{x}{y}{(\outputp{x}{y} | @{y})) | P}}}
	  | {(\prefix{x}{y}{(\outputp{x}{y} | @{y})) | P}} & \nonumber\\
	\red
	& \ldots & \nonumber\\
	\red^*
	& P | P | \ldots & \nonumber
\end{eqnarray}

Of course, this encoding, as an implementation, runs away, unfolding
$\bangp{P}$ eagerly. A lazier and more implementable replication
operator, restricted to input-guarded processes, may be obtained as follows.

\begin{eqnarray}
\bangp{\prefix{u}{v}{P}} 
	:= 
	\binpar{\lift{x}{\prefix{u}{v}{(\binpar{D(x)}{P})}}}{D(x)} \nonumber
\end{eqnarray}

\begin{remark}
  Note that the lazier definition still does not deal with summation
  or mixed summation (i.e. sums over input and output). The reader is
  invited to construct definitions of replication that deal with these
  features. 

  Further, the definitions are parameterized in a name, $x$. Can you,
  gentle reader, make a definition that eliminates this parameter and
  guarantees no accidental interaction between the replication
  machinery and the process being replicated -- i.e. no accidental
  sharing of names used by the process to get its work done and the
  name(s) used by the replication to effect copying. This latter
  revision of the definition of replication is crucial to obtaining
  the expected identity $!!P \sim !P$.
\end{remark}

\begin{remark}\label{rem:paradoxical_combinator}
  The reader familiar with the lambda calculus will have noticed the
  similarity between $D$ and the paradoxical combinator.

  [Ed. note: the existence of this seems to suggest we have to be more
  restrictive on the set of processes and names we admit if we are to
  support no-cloning.]
\end{remark}

\subsubsection{Bisimulation}

The computational dynamics gives rise to another kind of equivalence,
the equivalence of computational behavior. As previously mentioned
this is typically captured \emph{via} some form of bisimulation.

% The notion we use in this paper is weak barbed bisimulation
% \cite{milner91polyadicpi}.

The notion we use in this paper is derived from weak barbed
bisimulation \cite{milner91polyadicpi}. 

\begin{definition}
An \emph{observation relation}, $\downarrow_{\mathcal N}$, over a set
of names, $\mathcal N$, is the smallest relation satisfying the rules
below.

\infrule[Out-barb]{y \in {\mathcal N}, \; x \nameeq y}
		  {\outputp{x}{v} \downarrow_{\mathcal N} x}
\infrule[Par-barb]{\mbox{$P\downarrow_{\mathcal N} x$ or $Q\downarrow_{\mathcal N} x$}}
		  {\binpar{P}{Q} \downarrow_{\mathcal N} x}

We write $P \Downarrow_{\mathcal N} x$ if there is $Q$ such that 
$P \wred Q$ and $Q \downarrow_{\mathcal N} x$.
\end{definition}

\begin{definition}
%\label{def.bbisim}
An  ${\mathcal N}$-\emph{barbed bisimulation} over a set of names, ${\mathcal N}$, is a symmetric binary relation 
${\mathcal S}_{\mathcal N}$ between agents such that $P\rel{S}_{\mathcal N}Q$ implies:
\begin{enumerate}
\item If $P \red P'$ then $Q \wred Q'$ and $P'\rel{S}_{\mathcal N} Q'$.
\item If $P\downarrow_{\mathcal N} x$, then $Q\Downarrow_{\mathcal N} x$.
\end{enumerate}
$P$ is ${\mathcal N}$-barbed bisimilar to $Q$, written
$P \wbbisim_{\mathcal N} Q$, if $P \rel{S}_{\mathcal N} Q$ for some ${\mathcal N}$-barbed bisimulation ${\mathcal S}_{\mathcal N}$.
\end{definition}

$\mathcal{R} \subseteq \pi \times \pi$

$P \mathcal{R} Q => \forall P'. P \red P' \Rightarrow \exists Q'. Q \red Q', P' \mathcal{R} Q'$

$P \vdash x \Rightarrow Q \vdash x$

\begin{mathpar}
  \inferrule*[lab=Out-barb]{x \nameeq y}{{y}!\langle{Q}\rangle \vdash x}
  \and
  \inferrule*[lab=Par-barb]{\mbox{$P\vdash x$ or $Q\vdash x$}}{\binpar{P}{Q} \vdash x}
\end{mathpar}

\subsubsection{Contexts}

One of the principle advantages of computational calculi like the
$\pi$-calculus is a well-defined notion of context,
contextual-equivalence and a correlation between
contextual-equivalence and notions of bisimulation. The notion of
context allows the decomposition of a process into (sub-)process and
its syntactic environment, its context. Thus, a context may be
thought of as a process with a ``hole'' (written $\Box$) in it. The
application of a context $M$ to a process $P$, written $M[P]$, is
tantamount to filling the hole in $M$ with $P$. In this paper we do
not need the full weight of this theory, but do make use of the notion
of context in the proof the main theorem. 

\begin{mathpar}
  \inferrule* [lab=summation] {} {{M_{M},M_{N}} \bc \Box \;|\; x.M_{A} \;|\; M_{M}+M_{N}}
  \and
  \inferrule* [lab=agent] {} {{M_{A}} \bc (\vec{x})M_{P} \;| \; \clift{P_0,\ldots,M_{P},\ldots,P_N}}
  \and \\
  \inferrule* [lab=process] {} {{M_{P}} \bc M_{N} \;| \;P|M_{P} }
\end{mathpar} 

\begin{mathpar}
  \inferrule* [lab=sychronization] {} {M_{N} \bc \Box \;|\; x?M_{F} \;|\; x!M_{C}}
  \and
  \inferrule* [lab=abstraction] {} {{M_{F}} \bc (x)M_{P} }
  \and
  \inferrule* [lab=concretion] {} {{M_{C}} \bc \langle M_{P} \rangle }
  \and \\
  \inferrule* [lab=process] {} {{M_{P}} \bc M_{N} \;| \;P|M_{P} }
\end{mathpar}

\begin{definition}[contextual application] Given a context $M$, and
  process $P$, we define the \emph{contextual application}, $M[P] :=
  M\{P/\Box\}$. That is, the contextual application of M to P is the
  substitution of $P$ for $\Box$ in $M$.
\end{definition}

$\meaningof{-} : L \to \mathcal{P}(\pi)$

\begin{mathpar}
  \inferrule* [lab=collection] {} {\meaningof{true} = \pi, \and \meaningof{~E} = \pi \setminus \meaningof{E}, \and \meaningof{E_{1} \& E_{2}} = \meaningof{E_{1}} \cap \meaningof{E_{2}}}
\end{mathpar}

\begin{mathpar}
  \inferrule* [lab=structure] {} {\meaningof{0} = \{ P \in \pi | P \equiv 0 \}, \and \\ \meaningof{E_1 | E_2} = \{ P \in \pi | P \equiv P_{1} | P_{2}, P_{1} \in \meaningof{E_{1}}, P_{2} \in \meaningof{E_2}\} }
\end{mathpar}

\begin{mathpar}
 \inferrule* [lab=behavior] {} {\meaningof{\langle a?b \rangle E} = \{ P \in \pi | P \equiv Q | u?(y)P', \\ \and \\\\ \and \\ \;\;\; u \in \meaningof{a}, \forall z.P'\{z/y\} \in \meaningof{E\{z/b\}}\}, \and \\ \meaningof{a!E} = \{ P \in \pi | P \equiv Q | x!\langle P' \rangle, x \in \meaningof{a} P' \in \meaningof{E}\} }
\end{mathpar}

\begin{mathpar}
 \inferrule* [lab=nominal] {} {\meaningof{\quotep{E}} = \{ \quotep{P} \in \quotep{\pi} | P \in \meaningof{E} \}, \and \meaningof{\quotep{P}} = \{ \quotep{Q} \in \quotep{\pi} | P \equiv Q \} \and \\ \meaningof{@\quotep{E}} = \{ P \in \pi | P \equiv @x, x \in \meaningof{E} \}}
\end{mathpar}

\begin{eqnarray*}
  \\
  \meaningof{-} : TS \to ST
\end{eqnarray*}

\begin{eqnarray*}
  \\
  L : TS \to ST
\end{eqnarray*}

\begin{eqnarray*}
  \\
  P \models E \iff P \in \meaningof{E}
\end{eqnarray*}

\begin{eqnarray*}
  P \approx_{L} Q \iff \forall E \in L. P \models E \iff Q \models E
\end{eqnarray*}

\begin{eqnarray*}
  P \approx_{K} Q
\end{eqnarray*}

\begin{eqnarray*}
  P \approx Q
\end{eqnarray*}

$\approx_{K} = \approx = \approx_{L}$

\subsubsection{Contextual duality}

Note that contexts extend the quotation operation to a family of
operations from processes to names. Given a context, $M$, we can
define a \emph{nominal context}, $\quotep{M}$ by $\quotep{M}[P] :=
\quotep{M[P]}$. To foreshadow what is to come we observe that these
operations enjoy a duality with processes very much like the duality
between vectors and maps from vectors to scalars.

Further, because the calculus is essentially higher-order, we have a
correspondence between contexts and processes. More specifically,
given a name $x$ and a context $M$ we can construct $M^{*}_{x}$ such
that 

\begin{mathpar}
  M^{*}_{x} | \lift{x}{P} \red M[P]
\end{mathpar}

namely,

\begin{mathpar}
  M^{*}_{x} := x?(u).M[\dropn{u}]
\end{mathpar}

The dependence of $M^{*}_{x}$ on a name makes it an abstraction, 

\begin{mathpar}
  M^{*} := (x)x?(u).M[\dropn{u}]
\end{mathpar}

\subsection{Additional notation}

It will sometimes be convenient to denote the process a name
quotes. We already have the notation $x = \quotep{P}$, but it will be
convenient to introduce an alternate notation, $\procn{x}$, when we
want to emphasize the connection to the use of the name. Note that, by
virtue of name equivalence, $\quotep{\procn{x}} \nameeq x$; so, the
notation is consistent with previous definitions.

Further, because names have structure it is possible to effect
substitutions on the basis of that structure. This means we need to
upgrade our notation for substitutions, which we accomplish by
adapting comprehension notation. Thus,

\begin{mathpar}
  P\{ y / x : x \in S \}
\end{mathpar}

is interpreted to mean the process derived from P by replacing (in a
capture-avoiding manner) each occurrence of $x$ in $S$ by $y$. For example,

\begin{mathpar}
  P\{ \quotep{\procn{x}|\procn{x}} / x : x \in \freenames{P} \}
\end{mathpar}

will replace each (occurrence) of a free name $x$ in $P$ by
$\quotep{\procn{x}|\procn{x}}$.

Also, we will avail ourselves of the notation $x^{L}$ and $x^{R}$ to
denote injections of a name into disjoint copies of the name
space. There are numerous ways to accomplish this. One example can be
found in \cite{MeredithR05}. This notation overloads to vectors of
names: $\vec{x}^{\pi} := (x_{i}^{\pi} \; : \; 0 \leq i < |\vec{x}| )$ where $\pi \in \{L,R\}$.

We also use $P^{\Box} := P|\Box$.

In \cite{MeredithR05} an interpretation of the new operator is
given. It turns out that there are several possible interpretations
all enjoying the requisite algebraic properties of the operator (see
\cite{milner91polyadicpi}). We will therefore make liberal use of
$(\nu\; \vec{x})P$.

% subsection the_syntax_and_semantics_of_the_notation_system (end)   

\section{Interpretation of QM}
\subsection{Supporting definitions}
\subsubsection{Multiplication}
\begin{mathpar}
  \quotep{Q} \cdot \quotep{R} := \quotep{Q|R}
  \and \\
  \quotep{Q} \cdot P := P\{ \quotep{Q|R} / \quotep{R} : \quotep{R} \in \freenames{P} \}
\end{mathpar}

\paragraph{Discussion}
The first line needs little explanation. The second line says that
each free name of the process is replaced with the multiplication of
that name by the scalar. Multiplication of a scalar (name) by a state
(process) results in a process all the names of which have been `moved
over' by parallel composition with the process the scalar
quotes. There is a subtlety that the bound names have to be
manipulated so that multiplied names aren't accidentally
captured. There are many ways to achieve this.

\begin{remark}\label{rem:multiplication_identities}
  The reader is invited to verify that for all $x,y,z \in \QProc$ and $P \in \Proc$
  \begin{mathpar}
    x \cdot \quotep{0} \equiv x 
    \and
    x \cdot y \equiv y \cdot x
    \and
    x \cdot (y \cdot z) \equiv (x \cdot y) \cdot z
    \and \\
    \quotep{0} \cdot P \equiv P
    \and \\
    x \cdot (y \cdot P) \equiv (x \cdot y) \cdot P
    \and \\
    x \cdot (P|Q) \equiv (x \cdot P) | (x \cdot Q)
    \and \\    
  \end{mathpar}
\end{remark}

\subsubsection{Tensor product}

We define a tensor product on processes by structural induction.

\paragraph{Tensor of sums} First note that all summations, including
$\pzero$ and sequence, can be written $\Sigma_{i} x_{i}.A_{i} +
\Sigma_{j} x_{j}.C_{j}$, where we have grouped input-guarded processes
together and output-guarded processes together.

Thus, we can define the tensor product of two summations, $N_{1}\otimes N_{2}$, where

\begin{mathpar}
  N_{1} := \Sigma_{i} x_{i}.A_{i} + \Sigma_{j} x_{j}.C_{j}
  \and
  N_{2} := \Sigma_{i'} y_{i'}.B_{i'} + \Sigma_{j'} y_{j'}.D_{j'} 
\end{mathpar}

as follows.

\begin{mathpar}
  \Sigma_{i} x_{i}.A_{i} + \Sigma_{j} x_{j}.C_{j} \otimes \Sigma_{i'}
  y_{i'}.B_{i'} + \Sigma_{j'} y_{j'}.D_{j'} 
  \and \\
  := \; \Sigma_{i} \Sigma_{i'} \quotep{\stackrel{\vee}{x_{i}}| \stackrel{\vee}{y_{i'}}}.(A_{i}\otimes B_{i'}) \; | \; \Sigma_{i'} \Sigma_{i} \quotep{\stackrel{\vee}{y_{i'}}|\stackrel{\vee}{x_{i}}}.(B_{i'}\otimes A_{i})
  \and
  \;\; | \;\; \Sigma_{j} \Sigma_{j'} \quotep{\stackrel{\vee}{x_{j}}|\stackrel{\vee}{y_{j'}}}.(A_{j}\otimes B_{j'}) \; | \; \Sigma_{j'} \Sigma_{j} \quotep{\stackrel{\vee}{y_{j'}}|\stackrel{\vee}{x_{j}}}.(B_{j'}\otimes A_{j})
\end{mathpar}

\begin{remark}
  Do we need to $x^{L}$ and $y^{R}$ for this construction as well?
\end{remark}

\paragraph{Tensor of parallel compositions} Next, we distribute tensor
over par.

\begin{mathpar}
  P_{1}|P_{2} \otimes Q_{1}|Q_{2} := (P_{1} \otimes Q_{1}) | (P_{1}
  \otimes Q_{2}) | (P_{2} \otimes Q_{1}) | (P_{2} \otimes Q_{2})
\end{mathpar}

\paragraph{Tensor with dropped names} We treat tensor of a
process with a dropped name as parallel composition.

\begin{mathpar}
  P \otimes \dropn{x} := P | \dropn{x}
\end{mathpar}

\paragraph{Tensor of agents}

Finally, we need to define tensor on agents. Note that the definition
of tensor on normal products only tensors inputs with inputs and
outputs with outputs. Thus, we only have to define the operation on
``homogeneous'' pairings.

\begin{mathpar}
  (\vec{x})P \otimes (\vec{y})Q
  \and \\
  := (x_{0}^{L}|y_{0}^{R},\ldots,x_{0}^{L}|y_{n}^{R},\ldots,x_{m}^{L}|y_{0}^{R},\ldots,x_{m}^{L}|y_{n}^R)(P\{ \vec{x}^{L}/\vec{x}\} \otimes Q \{ \vec{y}^{R}/\vec{y}\})
  \and \\
  \clift{\vec{P}} \otimes \clift{\vec{Q}}
  \and \\
  := \clift{P_{0}\otimes Q_{0},\ldots,P_{0}\otimes Q_{n},\ldots,P_{m}\otimes Q_{0},\ldots,P_{m}\otimes Q_{n}}
\end{mathpar}

\begin{remark}
  Observe that arities of tensored abstractions matches arities of
  tensored concretions if the original arities matched. Note also that
  the length of the arities corresponds to the increase in dimension
  we see in ordinary vector space tensor product.
\end{remark}

\begin{remark}
  Operationally, this definition distributes the tensor down to
  components ``linked'' by summation. Tensor over summation is
  intriguing in that it mixes names. Moreover, as a consequence of the
  way it mixes names we have the identities for all $x \in \QProc$ and
  $P,Q \in \Proc$

  \begin{mathpar}
    (x \cdot P) \otimes Q \equiv x \cdot (P \otimes Q) \equiv P \otimes (x \cdot Q)
    \and
    P \otimes \pzero \equiv P
  \end{mathpar}

  that the reader is invited to verify.
\end{remark}

\subsubsection{Annihilation}
\begin{mathpar}
  P^{\perp} := \{ Q | \forall R. P|Q \red^{*} R \Rightarrow R \red^{*} \pzero \}
  \and \\
  P^{\underline{\perp}} := \Sigma_{Q \in P^{\perp}} \quotep{Q}?(y).(\dropn{y}|Q) | \Sigma_{Q \in P^{\perp}} \quotep{Q}\clift{\Box}
\end{mathpar}

\paragraph{Discussion} The reader will note that $P^{\perp}$ is a
\emph{set} of processes, while $P^{\underline{\perp}}$ is a
\emph{context}. We call the set $P^{\perp}$ the \emph{annihilators} of
$P$. The parallel composition of a process in the annihilators of $P$
with $P$ will result in a process, the state space of which has all
paths eventually leading to $\pzero$. Execution may endure loops; but
under reasonable conditions of fairness (naturally guaranteed under
most notions of bisimulation) such a composite process cannot get
stuck in such a loop and will, eventually pop out and terminate.

The context $P^{\underline{\perp}}$ is ready and willing to ``take the
$P$ out of'' the process to which it is applied. It will effectively
transmit the code of the process to which it is applied to one of the
annihilators and run the process against it.

\subsubsection{Evaluation}
We fix $M$ a domain of fully abstract interpretation with an equality
coincident with bisimulation. We take $\meaningof{\cdot} : \Proc \to
M$ to be the map interpreting processes and $\nmeaningof{\cdot} : \M
\to Proc$ to be the map running the other way. Then we define

\begin{mathpar}
  \int P := \nmeaningof{\meaningof{P}}
\end{mathpar}

\paragraph{Discussion}
There are many fully abstract interpretations of Milner's
$\pi$-calculus. Any of them can be used as a basis for interpreting
the reflective calculus here. Equipped with such a domain it is
largely a matter of grinding through to check that the Yoneda
construction for the normalization-by-evaluation program can be
extended to this setting.

\begin{remark}
  The reader is invited to verify that $\int (P^{\underline{\perp}}[P]) = 0$.
\end{remark}

\subsection{Quantum mechanics}

Table \ref{tbl:core_qm_op_defns} gives the core operational definitions

\begin{table}[htp]\label{tbl:core_qm_op_defns}
  \center{
    \fbox{
      \begin{tabular}{c|c}
        quantum mechanics & process calculus \\
        \hline
        scalar & $x := \quotep{P}$ \\
        state vector & $\state{P} := P$ \\
        dual & $\state{P}^{*} := \event{P^{\underline{\perp}}} := \quotep{P^{\underline{\perp}}}[-]$ \\
        matrix & $ \Sigma_{\alpha} \state{P_{\alpha}}x_{\alpha}\event{Q_{\alpha}}$ \\
        vector addition & $\state{P} + \state{Q} := \state{P | Q}$ \\
        tensor product & $\state{P} \otimes \state{Q} := \state{P \otimes Q}$ \\
        inner product & $\innerprod{P}{Q} := \quotep{\int P^{\underline{\perp}}[Q]}$ \\
      \end{tabular}
    }
  }
  \caption{QM - operational definitions}
\end{table}

where

\begin{mathpar}
  \prmatrix{P}{Q} := \fprmatrix{P}{\quotep{\pzero}}{Q}
  \and
  \fprmatrix{P}{x}{Q} := (\state{P},x,\event{Q})
  \and
  (\fprmatrix{P}{x}{Q})(\state{R}) := x \cdot \innerprod{Q}{R} \cdot \state{P}
  \and
  (\fprmatrix{P}{x}{Q})(\event{R}) := x \cdot \innerprod{R}{P} \cdot \event{Q}
\end{mathpar}

\paragraph{Discussion}
As promised: vectors (aka states) are represented as processes; duals
as contextual duals; inner product definition should be compared with
standard inner product definition for ....

\begin{remark}
  Assuming $\int (P^{\underline{\perp}}[P]) = 0$, the reader is
  invited to verify that $(\fprmatrix{P}{x}{P})(\state{P}) = x \cdot \state{P}$.
\end{remark}

\begin{remark}
  The reader is invited to verify that $\innerprod{P}{Q}$ could
  equally well have been written $\quotep{\int \stackrel{\vee}{x}}$
  where $x = \event{P^{\underline{\perp}}}(Q)$.

  One of the motivations for this remark is that there is another way
  to factor these operations. We could package up evaluation in the dual:

  \begin{mathpar}
    \state{P}^{*} := \event{\int P^{\underline{\perp}}} := \quotep{\int P^{\underline{\perp}}}[-]
  \end{mathpar}

  and then have inner product defined by
  
  \begin{mathpar}
    \innerprod{P}{Q} := \event{P}(Q)
  \end{mathpar}

  Hopefully, experience with the calculations will provide guidance on
  the best factoring.
\end{remark}

\begin{remark}
  Assuming $\int (P^{\underline{\perp}}[P]) = 0$, the reader is
  invited to verify that $\forall P,Q. (\prmatrix{0}{Q})(\state{0}) =
  \state{0}$ and dually $(\prmatrix{P}{0})(\event{0}) = \event{0}$.
\end{remark}

\begin{remark}
  i'm a little worried that i don't (yet) have proper support for
  complex conjugacy. But, the observation above may give us a
  clue. According to Abramsky, it must be the case that the scalars
  are iso to the homset of the identity for the tensor -- which the
  observation above characterizes. 

  For now, we will simply bookmark the notion with $\overline{x}$.
\end{remark}

\subsubsection{Adjointness}

We need to give a definition of $(\cdot)^{\dagger}$ for matrices. The
obvious candidate definition is
\begin{mathpar}
(\Sigma_{\alpha}\fprmatrix{P_{\alpha}}{x_{\alpha}}{Q_{\alpha}})^{\dagger}
= \Sigma_{\alpha}\fprmatrix{(Q_{\alpha}^{\underline{\perp}})^{*}}{\overline{x}_{\alpha}}{P_{\alpha}^{\underline{\perp}}} 
\end{mathpar}

But, $(Q_{\alpha}^{\underline{\perp}})^{*}$ requires a name along
which to communicate the process to achieve the context application.

\subsubsection{Basis for a basis}
If processes label states and ``addition'' of states (a.k.a. vector
addition) is interpreted as parallel composition, what corresponds to
notions of linear independence and basis? Here, we recall that Yoshida
has developed a set of \emph{combinators} for an asynchronous verison
of Milner's $\pi$-calculus. These are a finite set of processes such
any process can be expressed as parallel composition of these
combinators together with liberal uses of the new operator and
replication. We can simply give a translation of these into the
present calculus and have reasonable expectation that the property
carries over. That is, that the resultant set allows to express all
processes via parallel composition. Note, however, that there is no
new operator or replication in this calculus. As a result, we expect
that the corresponding set is actually infinite. That is, we expect
that the space is actually infinite dimensional.

\begin{remark}
  The attentive reader may be a bit concerned. Certainly, the
  collection $S$, $K$ and $I$ is a finite set of
  combinators. Shouldn't we expect to see a finite set of combinators
  for an effectively equivalent system? i am very sympathetic to this
  critique and feel it warrants full attention. On the other hand, i
  also have in mind the following analogy. The natural numbers, as a
  monoid under addition, has exactly $1$ generator, while the natural
  numbers, as a monoid under multiplication, has countably many
  generators (the primes). We observe that the application of the
  lambda calculus is much less resource sensitive than the parallel
  composition of the $\pi$-calculus. Could it be the case that we have
  an analogy of the form
  
  \begin{mathpar}
    m + n : MN :: m*n : M|N
  \end{mathpar}

  giving a similar blow up in the set of ``primes''?  This is such a
  wonderful thought that, even if it's not true, i think it's worth
  writing down.
\end{remark}
 

\documentclass[12pt]{llncs}
%\documentclass{jktr}

\usepackage[pdftex]{hyperref}                   
\usepackage {listings}
\usepackage {mathpartir}
\usepackage{bcprules}
%\usepackage{listings}
                       
\usepackage{graphicx} 
%\usepackage[margins=2.5cm,nohead,nofoot]{geometry}
%\usepackage{geometry}
\usepackage{amsfonts}
\usepackage{amstext}
\usepackage{latexsym}
\usepackage{amssymb}
\usepackage{color}


%\include{myPreamble}
\include{qm2pi.local} 

%\ifpdf
%\usepackage[pdftex]{graphicx}
%\else
%\usepackage{graphicx}
%\fi

 % \ifpdf
%  \usepackage{pdfsync}
%  \if


%\title{Brief Article}
%\author{David F. Snyder}
%\author{L.G. Meredith}

%\address{Dept. of Math., Texas State University--San Marcos, San Marcos, TX 78666}
       
\pagestyle{empty}


\begin{document}

\lstset{language=[Objective]Caml,frame=shadowbox}

\input{qm2pi.front}

% section front matter (end)

\input{qm2pi.intro} 
 
% section introduction (end)

% \input{qm2pi.knotations} 

% section notation (end)

\input{qm2pi.process.calculi} 

% section concurrent_process_calculi_and_spatial_logics_ (end)
    
%\input{qm2pi.knots2pi} 

%\input{qm2pi.trefoil} 

%\input{qm2pi.mainthm} 

% subsection basic_interpretation (end)

%\input{qm2pi.rho.presentation} 
\subsection{The syntax and semantics of the notation system}\label{sub:the_syntax_and_semantics_of_the_notation_system} % (fold)

We now summarize a technical presentation of the calculus that
embodies our theory of dynamics. The typical presentation of such a
calculus follows the style of giving generators and relations on
them. The grammar, below, describing term constructors, freely
generates the set of processes, $\Proc$. This set is then quotiented
by a relation known as structural congruence and it is over this set
that the notion of dynamics is expressed. This presentation is
essentially that of \cite{MeredithR05} with the addition of
polyadicity and summation. For readability we have relegated some of
the technical subtleties to an appendix.

\subsubsection{Process grammar}\label{subsub:process_grammar}

\begin{mathpar}
  \inferrule* [lab=synchronization] {} {{M} \bc \pzero \;|\; x?F \;|\; x!C }
  \and
  \inferrule* [lab=abstraction] {} {{F} \bc (x)P}
  \and
  \inferrule* [lab=concretion] {} {{C} \bc \langle Q \rangle}
  \and
  \inferrule* [lab=process] {} {{P,Q} \bc M \;| \;P|Q \;|\; @{x}}
  \and
  \inferrule* [lab=name] {} {{x} \bc \quotep{P}}
\end{mathpar} 

Note that $\vec{x}$ (resp. $\vec{P}$) denotes a vector of names
(resp. processes) of length $|\vec{x}|$ (resp. $|\vec{P}|$). We adopt
the following useful abbreviations.

\begin{mathpar}
   x?(\vec{y}).P := x.(\vec{y})P \and  x\clift{\vec{P}} := x.\clift{\vec{P}}
   \and x!(y) := \lift{x}{\dropn{y}}
   \and \Pi_{i=0}^{n-1}P_i := P_0 | \ldots | P_{n-1}
\end{mathpar}

\subsubsection{Structural congruence}

\paragraph{Free and bound names and alpha-equivalence.} At the
core of structural equivalence is alpha-equivalence which identifies
process that are the same up to a change of variable. Formally, we
recognize the distinction between free and bound names. The free names
of a process, $\freenames{P}$, may be calculated recursively as
follows:

\begin{mathpar}
\freenames{\pzero} := \emptyset
  \and \\
  \freenames{x?(y).P} := \{ x \} \cup (\freenames{P} \setminus \{ y \})
  \and 
  \freenames{x!\langle P \rangle} := \{ x \} \cup \{ P \} 
  \and \\
  \freenames{P|Q} := \freenames{P} \cup \freenames{Q}
  \and \\
  \freenames{@{x}} := \{ x \}
\end{mathpar}

$\pi$
$\quotep{\pi}$

$\freenames{-} : \pi \to \mathcal{P}(\quotep{\pi})$

\begin{eqnarray*}
  \freenames{\pzero} & := & \emptyset \\
  \freenames{x?(y).P} & := & \{ x \} \cup (\freenames{P} \setminus \{ y \}) \\
  \freenames{x!\langle P \rangle} & := & \{ x \} \cup \{ P \} \\
  \freenames{P|Q} & := & \freenames{P} \cup \freenames{Q} \\
  \freenames{\dropn{x}} & := & \{ x \}
\end{eqnarray*}

The bound names of a process, $\boundnames{P}$, are those names occurring in $P$
that are not free. For example, in $x?(y).0$, the name $x$ is free, while $y$ is bound.

\begin{mathpar}
  \inferrule* [lab=monoidal-laws] {} { P|Q \equiv Q|P \and P|0 \equiv P \and P|(Q|R) \equiv (P|Q)|R }
\end{mathpar}

\begin{mathpar}
  \inferrule* [lab=alpha-equivalence] {} { (x)P \equiv (y)P\{y/x\} \and y \not\in \freenames{P} }
\end{mathpar}

\begin{definition}
Then two processes, $P,Q$, are alpha-equivalent if $P = Q\{\vec{y}/\vec{x}\}$ for
some $\vec{x} \in \boundnames{Q},\vec{y} \in \boundnames{P}$, where $Q\{\vec{y}/\vec{x}\}$
denotes the capture-avoiding substitution of $\vec{y}$ for $\vec{x}$ in $Q$.
\end{definition}

\begin{definition}
  The {\em structural congruence} \cite{SangiorgiWalker} , $\equiv$,
  between processes is the least congruence containing
  alpha-equivalence, satisfying the abelian monoid laws
  (associativity, commutativity and $\pzero$ as identity) for parallel
  composition $|$ and for summation $+$.
\end{definition}

\subsection{Name equivalence}

We take name equivalence, written $\nameeq$, to be the smallest
equivalence relation generated by the following rules.

\begin{mathpar}
\inferrule*[lab=Quote-drop]
{ }
{ \quotep{@{x}} \nameeq x }

\inferrule*[lab=Struct-equiv]
{ P \scong Q }
{ \quotep{P} \nameeq \quotep{Q} }
\end{mathpar}

The astute reader will have noticed that the mutual recursion of names
and processes imposes a mutual recursion on alpha-equivalence and
structural equivalence via name-equivalence. Fortunately, all of this
works out pleasantly and we may calculate in the natural way, free of
concern. The reader interested in the details is referred to the
appendix \ref{appendix:rho_details}.

\subsection{Substitution}

We use $\Proc$ for the set of processes, $\QProc$ for the set of
names, and $\id{\{}\vec{y} / \vec{x} \id{\}}$ to denote partial maps,
$s : \QProc \rightarrow \QProc$. A map, $s$ lifts, uniquely, to a map
on process terms, $\widehat{s} : \Proc \rightarrow \Proc$ by the
following equations.

\begin{mathpar}
  (0) \psubstp{Q}{P} := 0 \\
  (R \juxtap S) \psubstp{Q}{P}
  :=    
  (R)\psubstp{Q}{P} \juxtap (S) \psubstp{Q}{P} \\
  (x?(y).R) \psubstp{Q}{P}    
  :=    
  (x)\substp{Q}{P} (z)\concat( (R \psubstn{z}{y}) \psubstp{Q}{P} ) \\
  (\lift{x}{R}) \psubstp{Q}{P}  
  :=
  \lift{(x)\substp{Q}{P}}{ R \psubstp{Q}{P} } \\
%   (\dropn{x})  \psubstp{Q}{P}       
%   := 
%   \left\{ 
%     \begin{array}{ccc} 
%       \dropn{\quotep{Q}} & & x \nameeq \quotep{P} \\
%       \dropn{x} & & otherwise \\
%     \end{array}
%   \right. 
  (\dropn{x})  \psubstp{Q}{P}       
  := 
  \left\{ 
    \begin{array}{ccc} 
      Q & & x \nameeq \quotep{P} \\
      \dropn{x} & & otherwise \\
    \end{array}
  \right.
\end{mathpar}
 

where

\begin{eqnarray}
  (x)\id{\{} \lpquote Q \rpquote / \lpquote P \rpquote \id{\}}            = 
  \left\{ 
    \begin{array}{ccc}
      \lpquote Q \rpquote & & x \nameeq \lpquote P \rpquote \\
      x & & otherwise \\
    \end{array}
  \right. \nonumber
\end{eqnarray}

and $z$ is chosen distinct from $\quotep{P}$, $\quotep{Q}$, the free
names in $Q$, and all the names in $R$. Our $\alpha$-equivalence will
be built in the standard way from this substitution.

\begin{remark}\label{rem:no_self_referential_names}
  One consequence of these definitions is that $\forall P. \quotep{P}
  \not\in \freenames{P}$.
\end{remark}

\subsection{ Dynamic quote: an example }

Anticipating something of what's to come, consider applying the
substitution, $\widehat{\id{\{}u / z \id{\}}}$, to the following pair
of processes, $\lift{w}{y!(z)}$ and $w[ \lpquote y!(z) \rpquote ]$.

\begin{eqnarray}
	\lift{w}{y!(z)}\widehat{\id{\{}u / z \id{\}}}
		& = &
		\lift{w}{y!(u)} \nonumber\\
	w[ \lpquote y!(z) \rpquote ] \widehat{ \id{\{}u / z \id{\}} }
		& = &
		w[ \lpquote y!(z) \rpquote ] \nonumber
\end{eqnarray}

Because the body of the process between quotes is impervious to
substitution, we get radically different answers. In fact, by
examining the first process in an input context,
e.g. $x?(z).\lift{w}{y!(z)}$, we see that the process under the lift
operator may be shaped by prefixed inputs binding a name inside it. In
this sense, the lift operator will be seen as a way to dynamically
construct processes before reifying them as names.

Finally equipped with these standard features we can present the
dynamics of the calculus.

\subsubsection{Operational semantics} 

Finally, we introduce the computational dynamics. What marks these
algebras as distinct from other more traditionally studied algebraic
structures, e.g. vector spaces or polynomial rings, is the manner in
which dynamics is captured. In traditional structures, dynamics is typically
expressed through morphisms between such structures, as in linear maps
between vector spaces or morphisms between rings. In algebras
associated with the semantics of computation, the dynamics is
expressed as part of the algebraic structure itself, through a
reduction reduction relation typically denoted by $\red$. Below, we
give a recursive presentation of this relation for the calculus used
in the encoding.

$\red \subseteq \pi \times \pi$
$\red : \pi \to \mathcal{P}(\pi)$

\begin{mathpar}
  \inferrule* [lab=Comm] { \textsf{match}( x_{src}, x_{trgt} ) } { x_{trgt}?(y)P \; | \; x_{src}!\langle {Q} \rangle \red P\{\quotep{Q}/y}\} }
  \and \\
  \inferrule* [lab=Par] {{P} \red {P}'} {{{P} | {Q}} \red {{P}' | {Q}}}
  \and
  \inferrule* [lab=Equiv]{{{P} \scong {P}'} \andalso {{P}' \red {Q}'} \andalso {{Q}' \scong {Q}}}{{P} \red {Q}}
\end{mathpar}

\begin{eqnarray*}
  match_{\equiv} (\quotep{P},\quotep{Q}) & := & P \equiv Q \\
  match_{\dagger}(\quotep{P},\quotep{Q}) & := & \forall R. P|Q \red^{*} R => R \red^{*} 0 \\
  match_{K}(\quotep{P},\quotep{Q}) & := & K \mbox{ for some context } K
\end{eqnarray*}

$u?(x)P | u!\langle Q \rangle \red P\{\quotep{Q}/x\}$

%We write $\wred$ for $\red^*$, and $P\red$ if $\exists Q $ such that $ P \red Q$.
We write $P\red$ if $\exists Q $ such that $ P \red Q$ and $P\not\red$, otherwise.

\section{Replication}

As mentioned before, it is known that replication (and hence
recursion) can be implemented in a higher-order process algebra
\cite{SangiorgiWalker}. As our first example of calculation with the
machinery thus far presented we give the construction explicitly in
the {\rhoc}.

\begin{eqnarray}
	D_{x} & := & \prefix{x}{y}{(\binpar{\outputp{x}{y}}{@{y}})} \nonumber\\
	\bangp_{x}{P} & := & \binpar{{x}!\langle{\binpar{D_{x}}{P}}\rangle}{D_{x}} \nonumber
\end{eqnarray}

\begin{eqnarray}
	\bangp_{x}{P} & & \nonumber\\
	=
	& {x}!\langle{(\prefix{x}{y}{(\outputp{x}{y} | @{y})) | P}}\rangle 
	      | \prefix{x}{y}{(\outputp{x}{y} | @{y})} & \nonumber\\
	\red
	& (\outputp{x}{y} | @{y})\substn{\quotep{(\prefix{x}{y}{(@{y} | \outputp{x}{y})) | P}}}{y} & \nonumber\\
	=
	& \outputp{x}{\quotep{(\prefix{x}{y}{(\outputp{x}{y} | @{y})) | P}}}
	  | {(\prefix{x}{y}{(\outputp{x}{y} | @{y})) | P}} & \nonumber\\
	\red
	& \ldots & \nonumber\\
	\red^*
	& P | P | \ldots & \nonumber
\end{eqnarray}

Of course, this encoding, as an implementation, runs away, unfolding
$\bangp{P}$ eagerly. A lazier and more implementable replication
operator, restricted to input-guarded processes, may be obtained as follows.

\begin{eqnarray}
\bangp{\prefix{u}{v}{P}} 
	:= 
	\binpar{\lift{x}{\prefix{u}{v}{(\binpar{D(x)}{P})}}}{D(x)} \nonumber
\end{eqnarray}

\begin{remark}
  Note that the lazier definition still does not deal with summation
  or mixed summation (i.e. sums over input and output). The reader is
  invited to construct definitions of replication that deal with these
  features. 

  Further, the definitions are parameterized in a name, $x$. Can you,
  gentle reader, make a definition that eliminates this parameter and
  guarantees no accidental interaction between the replication
  machinery and the process being replicated -- i.e. no accidental
  sharing of names used by the process to get its work done and the
  name(s) used by the replication to effect copying. This latter
  revision of the definition of replication is crucial to obtaining
  the expected identity $!!P \sim !P$.
\end{remark}

\begin{remark}\label{rem:paradoxical_combinator}
  The reader familiar with the lambda calculus will have noticed the
  similarity between $D$ and the paradoxical combinator.

  [Ed. note: the existence of this seems to suggest we have to be more
  restrictive on the set of processes and names we admit if we are to
  support no-cloning.]
\end{remark}

\subsubsection{Bisimulation}

The computational dynamics gives rise to another kind of equivalence,
the equivalence of computational behavior. As previously mentioned
this is typically captured \emph{via} some form of bisimulation.

% The notion we use in this paper is weak barbed bisimulation
% \cite{milner91polyadicpi}.

The notion we use in this paper is derived from weak barbed
bisimulation \cite{milner91polyadicpi}. 

\begin{definition}
An \emph{observation relation}, $\downarrow_{\mathcal N}$, over a set
of names, $\mathcal N$, is the smallest relation satisfying the rules
below.

\infrule[Out-barb]{y \in {\mathcal N}, \; x \nameeq y}
		  {\outputp{x}{v} \downarrow_{\mathcal N} x}
\infrule[Par-barb]{\mbox{$P\downarrow_{\mathcal N} x$ or $Q\downarrow_{\mathcal N} x$}}
		  {\binpar{P}{Q} \downarrow_{\mathcal N} x}

We write $P \Downarrow_{\mathcal N} x$ if there is $Q$ such that 
$P \wred Q$ and $Q \downarrow_{\mathcal N} x$.
\end{definition}

\begin{definition}
%\label{def.bbisim}
An  ${\mathcal N}$-\emph{barbed bisimulation} over a set of names, ${\mathcal N}$, is a symmetric binary relation 
${\mathcal S}_{\mathcal N}$ between agents such that $P\rel{S}_{\mathcal N}Q$ implies:
\begin{enumerate}
\item If $P \red P'$ then $Q \wred Q'$ and $P'\rel{S}_{\mathcal N} Q'$.
\item If $P\downarrow_{\mathcal N} x$, then $Q\Downarrow_{\mathcal N} x$.
\end{enumerate}
$P$ is ${\mathcal N}$-barbed bisimilar to $Q$, written
$P \wbbisim_{\mathcal N} Q$, if $P \rel{S}_{\mathcal N} Q$ for some ${\mathcal N}$-barbed bisimulation ${\mathcal S}_{\mathcal N}$.
\end{definition}

$\mathcal{R} \subseteq \pi \times \pi$

$P \mathcal{R} Q => \forall P'. P \red P' \Rightarrow \exists Q'. Q \red Q', P' \mathcal{R} Q'$

$P \vdash x \Rightarrow Q \vdash x$

\begin{mathpar}
  \inferrule*[lab=Out-barb]{x \nameeq y}{{y}!\langle{Q}\rangle \vdash x}
  \and
  \inferrule*[lab=Par-barb]{\mbox{$P\vdash x$ or $Q\vdash x$}}{\binpar{P}{Q} \vdash x}
\end{mathpar}

\subsubsection{Contexts}

One of the principle advantages of computational calculi like the
$\pi$-calculus is a well-defined notion of context,
contextual-equivalence and a correlation between
contextual-equivalence and notions of bisimulation. The notion of
context allows the decomposition of a process into (sub-)process and
its syntactic environment, its context. Thus, a context may be
thought of as a process with a ``hole'' (written $\Box$) in it. The
application of a context $M$ to a process $P$, written $M[P]$, is
tantamount to filling the hole in $M$ with $P$. In this paper we do
not need the full weight of this theory, but do make use of the notion
of context in the proof the main theorem. 

\begin{mathpar}
  \inferrule* [lab=summation] {} {{M_{M},M_{N}} \bc \Box \;|\; x.M_{A} \;|\; M_{M}+M_{N}}
  \and
  \inferrule* [lab=agent] {} {{M_{A}} \bc (\vec{x})M_{P} \;| \; \clift{P_0,\ldots,M_{P},\ldots,P_N}}
  \and \\
  \inferrule* [lab=process] {} {{M_{P}} \bc M_{N} \;| \;P|M_{P} }
\end{mathpar} 

\begin{mathpar}
  \inferrule* [lab=sychronization] {} {M_{N} \bc \Box \;|\; x?M_{F} \;|\; x!M_{C}}
  \and
  \inferrule* [lab=abstraction] {} {{M_{F}} \bc (x)M_{P} }
  \and
  \inferrule* [lab=concretion] {} {{M_{C}} \bc \langle M_{P} \rangle }
  \and \\
  \inferrule* [lab=process] {} {{M_{P}} \bc M_{N} \;| \;P|M_{P} }
\end{mathpar}

\begin{definition}[contextual application] Given a context $M$, and
  process $P$, we define the \emph{contextual application}, $M[P] :=
  M\{P/\Box\}$. That is, the contextual application of M to P is the
  substitution of $P$ for $\Box$ in $M$.
\end{definition}

$\meaningof{-} : L \to \mathcal{P}(\pi)$

\begin{mathpar}
  \inferrule* [lab=collection] {} {\meaningof{true} = \pi, \and \meaningof{~E} = \pi \setminus \meaningof{E}, \and \meaningof{E_{1} \& E_{2}} = \meaningof{E_{1}} \cap \meaningof{E_{2}}}
\end{mathpar}

\begin{mathpar}
  \inferrule* [lab=structure] {} {\meaningof{0} = \{ P \in \pi | P \equiv 0 \}, \and \\ \meaningof{E_1 | E_2} = \{ P \in \pi | P \equiv P_{1} | P_{2}, P_{1} \in \meaningof{E_{1}}, P_{2} \in \meaningof{E_2}\} }
\end{mathpar}

\begin{mathpar}
 \inferrule* [lab=behavior] {} {\meaningof{\langle a?b \rangle E} = \{ P \in \pi | P \equiv Q | u?(y)P', \\ \and \\\\ \and \\ \;\;\; u \in \meaningof{a}, \forall z.P'\{z/y\} \in \meaningof{E\{z/b\}}\}, \and \\ \meaningof{a!E} = \{ P \in \pi | P \equiv Q | x!\langle P' \rangle, x \in \meaningof{a} P' \in \meaningof{E}\} }
\end{mathpar}

\begin{mathpar}
 \inferrule* [lab=nominal] {} {\meaningof{\quotep{E}} = \{ \quotep{P} \in \quotep{\pi} | P \in \meaningof{E} \}, \and \meaningof{\quotep{P}} = \{ \quotep{Q} \in \quotep{\pi} | P \equiv Q \} \and \\ \meaningof{@\quotep{E}} = \{ P \in \pi | P \equiv @x, x \in \meaningof{E} \}}
\end{mathpar}

\begin{eqnarray*}
  \\
  \meaningof{-} : TS \to ST
\end{eqnarray*}

\begin{eqnarray*}
  \\
  L : TS \to ST
\end{eqnarray*}

\begin{eqnarray*}
  \\
  P \models E \iff P \in \meaningof{E}
\end{eqnarray*}

\begin{eqnarray*}
  P \approx_{L} Q \iff \forall E \in L. P \models E \iff Q \models E
\end{eqnarray*}

\begin{eqnarray*}
  P \approx_{K} Q
\end{eqnarray*}

\begin{eqnarray*}
  P \approx Q
\end{eqnarray*}

$\approx_{K} = \approx = \approx_{L}$

\subsubsection{Contextual duality}

Note that contexts extend the quotation operation to a family of
operations from processes to names. Given a context, $M$, we can
define a \emph{nominal context}, $\quotep{M}$ by $\quotep{M}[P] :=
\quotep{M[P]}$. To foreshadow what is to come we observe that these
operations enjoy a duality with processes very much like the duality
between vectors and maps from vectors to scalars.

Further, because the calculus is essentially higher-order, we have a
correspondence between contexts and processes. More specifically,
given a name $x$ and a context $M$ we can construct $M^{*}_{x}$ such
that 

\begin{mathpar}
  M^{*}_{x} | \lift{x}{P} \red M[P]
\end{mathpar}

namely,

\begin{mathpar}
  M^{*}_{x} := x?(u).M[\dropn{u}]
\end{mathpar}

The dependence of $M^{*}_{x}$ on a name makes it an abstraction, 

\begin{mathpar}
  M^{*} := (x)x?(u).M[\dropn{u}]
\end{mathpar}

\subsection{Additional notation}

It will sometimes be convenient to denote the process a name
quotes. We already have the notation $x = \quotep{P}$, but it will be
convenient to introduce an alternate notation, $\procn{x}$, when we
want to emphasize the connection to the use of the name. Note that, by
virtue of name equivalence, $\quotep{\procn{x}} \nameeq x$; so, the
notation is consistent with previous definitions.

Further, because names have structure it is possible to effect
substitutions on the basis of that structure. This means we need to
upgrade our notation for substitutions, which we accomplish by
adapting comprehension notation. Thus,

\begin{mathpar}
  P\{ y / x : x \in S \}
\end{mathpar}

is interpreted to mean the process derived from P by replacing (in a
capture-avoiding manner) each occurrence of $x$ in $S$ by $y$. For example,

\begin{mathpar}
  P\{ \quotep{\procn{x}|\procn{x}} / x : x \in \freenames{P} \}
\end{mathpar}

will replace each (occurrence) of a free name $x$ in $P$ by
$\quotep{\procn{x}|\procn{x}}$.

Also, we will avail ourselves of the notation $x^{L}$ and $x^{R}$ to
denote injections of a name into disjoint copies of the name
space. There are numerous ways to accomplish this. One example can be
found in \cite{MeredithR05}. This notation overloads to vectors of
names: $\vec{x}^{\pi} := (x_{i}^{\pi} \; : \; 0 \leq i < |\vec{x}| )$ where $\pi \in \{L,R\}$.

We also use $P^{\Box} := P|\Box$.

In \cite{MeredithR05} an interpretation of the new operator is
given. It turns out that there are several possible interpretations
all enjoying the requisite algebraic properties of the operator (see
\cite{milner91polyadicpi}). We will therefore make liberal use of
$(\nu\; \vec{x})P$.

% subsection the_syntax_and_semantics_of_the_notation_system (end)   

\input{qm2pi.qmops} 

\input{qm2pi.sterngerlach} 

\input{qm2pi.metric} 

% section concurrent_process_calculi (end)

%\input{qm2pi.proofsketch}

% section proof sketch (end)

%\input{qm2pi.slviaknots} 

% section spatial logic via knots (end)

\input{qm2pi.conclusion}

% section conclusion (end)

%\input{qm2pi.dtcodes} 

% section wiring algorithm (end)

\input{qm2pi.ack} 

% section acknowledgments (end)

\newpage


\bibliographystyle{plain}   
\bibliography{../../biblios/main.bib}

\input{qm2pi.rhodetails}

\end{document}

 

\documentclass[12pt]{llncs}
%\documentclass{jktr}

\usepackage[pdftex]{hyperref}                   
\usepackage {listings}
\usepackage {mathpartir}
\usepackage{bcprules}
%\usepackage{listings}
                       
\usepackage{graphicx} 
%\usepackage[margins=2.5cm,nohead,nofoot]{geometry}
%\usepackage{geometry}
\usepackage{amsfonts}
\usepackage{amstext}
\usepackage{latexsym}
\usepackage{amssymb}
\usepackage{color}


%\include{myPreamble}
\include{qm2pi.local} 

%\ifpdf
%\usepackage[pdftex]{graphicx}
%\else
%\usepackage{graphicx}
%\fi

 % \ifpdf
%  \usepackage{pdfsync}
%  \if


%\title{Brief Article}
%\author{David F. Snyder}
%\author{L.G. Meredith}

%\address{Dept. of Math., Texas State University--San Marcos, San Marcos, TX 78666}
       
\pagestyle{empty}


\begin{document}

\lstset{language=[Objective]Caml,frame=shadowbox}

\input{qm2pi.front}

% section front matter (end)

\input{qm2pi.intro} 
 
% section introduction (end)

% \input{qm2pi.knotations} 

% section notation (end)

\input{qm2pi.process.calculi} 

% section concurrent_process_calculi_and_spatial_logics_ (end)
    
%\input{qm2pi.knots2pi} 

%\input{qm2pi.trefoil} 

%\input{qm2pi.mainthm} 

% subsection basic_interpretation (end)

%\input{qm2pi.rho.presentation} 
\subsection{The syntax and semantics of the notation system}\label{sub:the_syntax_and_semantics_of_the_notation_system} % (fold)

We now summarize a technical presentation of the calculus that
embodies our theory of dynamics. The typical presentation of such a
calculus follows the style of giving generators and relations on
them. The grammar, below, describing term constructors, freely
generates the set of processes, $\Proc$. This set is then quotiented
by a relation known as structural congruence and it is over this set
that the notion of dynamics is expressed. This presentation is
essentially that of \cite{MeredithR05} with the addition of
polyadicity and summation. For readability we have relegated some of
the technical subtleties to an appendix.

\subsubsection{Process grammar}\label{subsub:process_grammar}

\begin{mathpar}
  \inferrule* [lab=synchronization] {} {{M} \bc \pzero \;|\; x?F \;|\; x!C }
  \and
  \inferrule* [lab=abstraction] {} {{F} \bc (x)P}
  \and
  \inferrule* [lab=concretion] {} {{C} \bc \langle Q \rangle}
  \and
  \inferrule* [lab=process] {} {{P,Q} \bc M \;| \;P|Q \;|\; @{x}}
  \and
  \inferrule* [lab=name] {} {{x} \bc \quotep{P}}
\end{mathpar} 

Note that $\vec{x}$ (resp. $\vec{P}$) denotes a vector of names
(resp. processes) of length $|\vec{x}|$ (resp. $|\vec{P}|$). We adopt
the following useful abbreviations.

\begin{mathpar}
   x?(\vec{y}).P := x.(\vec{y})P \and  x\clift{\vec{P}} := x.\clift{\vec{P}}
   \and x!(y) := \lift{x}{\dropn{y}}
   \and \Pi_{i=0}^{n-1}P_i := P_0 | \ldots | P_{n-1}
\end{mathpar}

\subsubsection{Structural congruence}

\paragraph{Free and bound names and alpha-equivalence.} At the
core of structural equivalence is alpha-equivalence which identifies
process that are the same up to a change of variable. Formally, we
recognize the distinction between free and bound names. The free names
of a process, $\freenames{P}$, may be calculated recursively as
follows:

\begin{mathpar}
\freenames{\pzero} := \emptyset
  \and \\
  \freenames{x?(y).P} := \{ x \} \cup (\freenames{P} \setminus \{ y \})
  \and 
  \freenames{x!\langle P \rangle} := \{ x \} \cup \{ P \} 
  \and \\
  \freenames{P|Q} := \freenames{P} \cup \freenames{Q}
  \and \\
  \freenames{@{x}} := \{ x \}
\end{mathpar}

$\pi$
$\quotep{\pi}$

$\freenames{-} : \pi \to \mathcal{P}(\quotep{\pi})$

\begin{eqnarray*}
  \freenames{\pzero} & := & \emptyset \\
  \freenames{x?(y).P} & := & \{ x \} \cup (\freenames{P} \setminus \{ y \}) \\
  \freenames{x!\langle P \rangle} & := & \{ x \} \cup \{ P \} \\
  \freenames{P|Q} & := & \freenames{P} \cup \freenames{Q} \\
  \freenames{\dropn{x}} & := & \{ x \}
\end{eqnarray*}

The bound names of a process, $\boundnames{P}$, are those names occurring in $P$
that are not free. For example, in $x?(y).0$, the name $x$ is free, while $y$ is bound.

\begin{mathpar}
  \inferrule* [lab=monoidal-laws] {} { P|Q \equiv Q|P \and P|0 \equiv P \and P|(Q|R) \equiv (P|Q)|R }
\end{mathpar}

\begin{mathpar}
  \inferrule* [lab=alpha-equivalence] {} { (x)P \equiv (y)P\{y/x\} \and y \not\in \freenames{P} }
\end{mathpar}

\begin{definition}
Then two processes, $P,Q$, are alpha-equivalent if $P = Q\{\vec{y}/\vec{x}\}$ for
some $\vec{x} \in \boundnames{Q},\vec{y} \in \boundnames{P}$, where $Q\{\vec{y}/\vec{x}\}$
denotes the capture-avoiding substitution of $\vec{y}$ for $\vec{x}$ in $Q$.
\end{definition}

\begin{definition}
  The {\em structural congruence} \cite{SangiorgiWalker} , $\equiv$,
  between processes is the least congruence containing
  alpha-equivalence, satisfying the abelian monoid laws
  (associativity, commutativity and $\pzero$ as identity) for parallel
  composition $|$ and for summation $+$.
\end{definition}

\subsection{Name equivalence}

We take name equivalence, written $\nameeq$, to be the smallest
equivalence relation generated by the following rules.

\begin{mathpar}
\inferrule*[lab=Quote-drop]
{ }
{ \quotep{@{x}} \nameeq x }

\inferrule*[lab=Struct-equiv]
{ P \scong Q }
{ \quotep{P} \nameeq \quotep{Q} }
\end{mathpar}

The astute reader will have noticed that the mutual recursion of names
and processes imposes a mutual recursion on alpha-equivalence and
structural equivalence via name-equivalence. Fortunately, all of this
works out pleasantly and we may calculate in the natural way, free of
concern. The reader interested in the details is referred to the
appendix \ref{appendix:rho_details}.

\subsection{Substitution}

We use $\Proc$ for the set of processes, $\QProc$ for the set of
names, and $\id{\{}\vec{y} / \vec{x} \id{\}}$ to denote partial maps,
$s : \QProc \rightarrow \QProc$. A map, $s$ lifts, uniquely, to a map
on process terms, $\widehat{s} : \Proc \rightarrow \Proc$ by the
following equations.

\begin{mathpar}
  (0) \psubstp{Q}{P} := 0 \\
  (R \juxtap S) \psubstp{Q}{P}
  :=    
  (R)\psubstp{Q}{P} \juxtap (S) \psubstp{Q}{P} \\
  (x?(y).R) \psubstp{Q}{P}    
  :=    
  (x)\substp{Q}{P} (z)\concat( (R \psubstn{z}{y}) \psubstp{Q}{P} ) \\
  (\lift{x}{R}) \psubstp{Q}{P}  
  :=
  \lift{(x)\substp{Q}{P}}{ R \psubstp{Q}{P} } \\
%   (\dropn{x})  \psubstp{Q}{P}       
%   := 
%   \left\{ 
%     \begin{array}{ccc} 
%       \dropn{\quotep{Q}} & & x \nameeq \quotep{P} \\
%       \dropn{x} & & otherwise \\
%     \end{array}
%   \right. 
  (\dropn{x})  \psubstp{Q}{P}       
  := 
  \left\{ 
    \begin{array}{ccc} 
      Q & & x \nameeq \quotep{P} \\
      \dropn{x} & & otherwise \\
    \end{array}
  \right.
\end{mathpar}
 

where

\begin{eqnarray}
  (x)\id{\{} \lpquote Q \rpquote / \lpquote P \rpquote \id{\}}            = 
  \left\{ 
    \begin{array}{ccc}
      \lpquote Q \rpquote & & x \nameeq \lpquote P \rpquote \\
      x & & otherwise \\
    \end{array}
  \right. \nonumber
\end{eqnarray}

and $z$ is chosen distinct from $\quotep{P}$, $\quotep{Q}$, the free
names in $Q$, and all the names in $R$. Our $\alpha$-equivalence will
be built in the standard way from this substitution.

\begin{remark}\label{rem:no_self_referential_names}
  One consequence of these definitions is that $\forall P. \quotep{P}
  \not\in \freenames{P}$.
\end{remark}

\subsection{ Dynamic quote: an example }

Anticipating something of what's to come, consider applying the
substitution, $\widehat{\id{\{}u / z \id{\}}}$, to the following pair
of processes, $\lift{w}{y!(z)}$ and $w[ \lpquote y!(z) \rpquote ]$.

\begin{eqnarray}
	\lift{w}{y!(z)}\widehat{\id{\{}u / z \id{\}}}
		& = &
		\lift{w}{y!(u)} \nonumber\\
	w[ \lpquote y!(z) \rpquote ] \widehat{ \id{\{}u / z \id{\}} }
		& = &
		w[ \lpquote y!(z) \rpquote ] \nonumber
\end{eqnarray}

Because the body of the process between quotes is impervious to
substitution, we get radically different answers. In fact, by
examining the first process in an input context,
e.g. $x?(z).\lift{w}{y!(z)}$, we see that the process under the lift
operator may be shaped by prefixed inputs binding a name inside it. In
this sense, the lift operator will be seen as a way to dynamically
construct processes before reifying them as names.

Finally equipped with these standard features we can present the
dynamics of the calculus.

\subsubsection{Operational semantics} 

Finally, we introduce the computational dynamics. What marks these
algebras as distinct from other more traditionally studied algebraic
structures, e.g. vector spaces or polynomial rings, is the manner in
which dynamics is captured. In traditional structures, dynamics is typically
expressed through morphisms between such structures, as in linear maps
between vector spaces or morphisms between rings. In algebras
associated with the semantics of computation, the dynamics is
expressed as part of the algebraic structure itself, through a
reduction reduction relation typically denoted by $\red$. Below, we
give a recursive presentation of this relation for the calculus used
in the encoding.

$\red \subseteq \pi \times \pi$
$\red : \pi \to \mathcal{P}(\pi)$

\begin{mathpar}
  \inferrule* [lab=Comm] { \textsf{match}( x_{src}, x_{trgt} ) } { x_{trgt}?(y)P \; | \; x_{src}!\langle {Q} \rangle \red P\{\quotep{Q}/y}\} }
  \and \\
  \inferrule* [lab=Par] {{P} \red {P}'} {{{P} | {Q}} \red {{P}' | {Q}}}
  \and
  \inferrule* [lab=Equiv]{{{P} \scong {P}'} \andalso {{P}' \red {Q}'} \andalso {{Q}' \scong {Q}}}{{P} \red {Q}}
\end{mathpar}

\begin{eqnarray*}
  match_{\equiv} (\quotep{P},\quotep{Q}) & := & P \equiv Q \\
  match_{\dagger}(\quotep{P},\quotep{Q}) & := & \forall R. P|Q \red^{*} R => R \red^{*} 0 \\
  match_{K}(\quotep{P},\quotep{Q}) & := & K \mbox{ for some context } K
\end{eqnarray*}

$u?(x)P | u!\langle Q \rangle \red P\{\quotep{Q}/x\}$

%We write $\wred$ for $\red^*$, and $P\red$ if $\exists Q $ such that $ P \red Q$.
We write $P\red$ if $\exists Q $ such that $ P \red Q$ and $P\not\red$, otherwise.

\section{Replication}

As mentioned before, it is known that replication (and hence
recursion) can be implemented in a higher-order process algebra
\cite{SangiorgiWalker}. As our first example of calculation with the
machinery thus far presented we give the construction explicitly in
the {\rhoc}.

\begin{eqnarray}
	D_{x} & := & \prefix{x}{y}{(\binpar{\outputp{x}{y}}{@{y}})} \nonumber\\
	\bangp_{x}{P} & := & \binpar{{x}!\langle{\binpar{D_{x}}{P}}\rangle}{D_{x}} \nonumber
\end{eqnarray}

\begin{eqnarray}
	\bangp_{x}{P} & & \nonumber\\
	=
	& {x}!\langle{(\prefix{x}{y}{(\outputp{x}{y} | @{y})) | P}}\rangle 
	      | \prefix{x}{y}{(\outputp{x}{y} | @{y})} & \nonumber\\
	\red
	& (\outputp{x}{y} | @{y})\substn{\quotep{(\prefix{x}{y}{(@{y} | \outputp{x}{y})) | P}}}{y} & \nonumber\\
	=
	& \outputp{x}{\quotep{(\prefix{x}{y}{(\outputp{x}{y} | @{y})) | P}}}
	  | {(\prefix{x}{y}{(\outputp{x}{y} | @{y})) | P}} & \nonumber\\
	\red
	& \ldots & \nonumber\\
	\red^*
	& P | P | \ldots & \nonumber
\end{eqnarray}

Of course, this encoding, as an implementation, runs away, unfolding
$\bangp{P}$ eagerly. A lazier and more implementable replication
operator, restricted to input-guarded processes, may be obtained as follows.

\begin{eqnarray}
\bangp{\prefix{u}{v}{P}} 
	:= 
	\binpar{\lift{x}{\prefix{u}{v}{(\binpar{D(x)}{P})}}}{D(x)} \nonumber
\end{eqnarray}

\begin{remark}
  Note that the lazier definition still does not deal with summation
  or mixed summation (i.e. sums over input and output). The reader is
  invited to construct definitions of replication that deal with these
  features. 

  Further, the definitions are parameterized in a name, $x$. Can you,
  gentle reader, make a definition that eliminates this parameter and
  guarantees no accidental interaction between the replication
  machinery and the process being replicated -- i.e. no accidental
  sharing of names used by the process to get its work done and the
  name(s) used by the replication to effect copying. This latter
  revision of the definition of replication is crucial to obtaining
  the expected identity $!!P \sim !P$.
\end{remark}

\begin{remark}\label{rem:paradoxical_combinator}
  The reader familiar with the lambda calculus will have noticed the
  similarity between $D$ and the paradoxical combinator.

  [Ed. note: the existence of this seems to suggest we have to be more
  restrictive on the set of processes and names we admit if we are to
  support no-cloning.]
\end{remark}

\subsubsection{Bisimulation}

The computational dynamics gives rise to another kind of equivalence,
the equivalence of computational behavior. As previously mentioned
this is typically captured \emph{via} some form of bisimulation.

% The notion we use in this paper is weak barbed bisimulation
% \cite{milner91polyadicpi}.

The notion we use in this paper is derived from weak barbed
bisimulation \cite{milner91polyadicpi}. 

\begin{definition}
An \emph{observation relation}, $\downarrow_{\mathcal N}$, over a set
of names, $\mathcal N$, is the smallest relation satisfying the rules
below.

\infrule[Out-barb]{y \in {\mathcal N}, \; x \nameeq y}
		  {\outputp{x}{v} \downarrow_{\mathcal N} x}
\infrule[Par-barb]{\mbox{$P\downarrow_{\mathcal N} x$ or $Q\downarrow_{\mathcal N} x$}}
		  {\binpar{P}{Q} \downarrow_{\mathcal N} x}

We write $P \Downarrow_{\mathcal N} x$ if there is $Q$ such that 
$P \wred Q$ and $Q \downarrow_{\mathcal N} x$.
\end{definition}

\begin{definition}
%\label{def.bbisim}
An  ${\mathcal N}$-\emph{barbed bisimulation} over a set of names, ${\mathcal N}$, is a symmetric binary relation 
${\mathcal S}_{\mathcal N}$ between agents such that $P\rel{S}_{\mathcal N}Q$ implies:
\begin{enumerate}
\item If $P \red P'$ then $Q \wred Q'$ and $P'\rel{S}_{\mathcal N} Q'$.
\item If $P\downarrow_{\mathcal N} x$, then $Q\Downarrow_{\mathcal N} x$.
\end{enumerate}
$P$ is ${\mathcal N}$-barbed bisimilar to $Q$, written
$P \wbbisim_{\mathcal N} Q$, if $P \rel{S}_{\mathcal N} Q$ for some ${\mathcal N}$-barbed bisimulation ${\mathcal S}_{\mathcal N}$.
\end{definition}

$\mathcal{R} \subseteq \pi \times \pi$

$P \mathcal{R} Q => \forall P'. P \red P' \Rightarrow \exists Q'. Q \red Q', P' \mathcal{R} Q'$

$P \vdash x \Rightarrow Q \vdash x$

\begin{mathpar}
  \inferrule*[lab=Out-barb]{x \nameeq y}{{y}!\langle{Q}\rangle \vdash x}
  \and
  \inferrule*[lab=Par-barb]{\mbox{$P\vdash x$ or $Q\vdash x$}}{\binpar{P}{Q} \vdash x}
\end{mathpar}

\subsubsection{Contexts}

One of the principle advantages of computational calculi like the
$\pi$-calculus is a well-defined notion of context,
contextual-equivalence and a correlation between
contextual-equivalence and notions of bisimulation. The notion of
context allows the decomposition of a process into (sub-)process and
its syntactic environment, its context. Thus, a context may be
thought of as a process with a ``hole'' (written $\Box$) in it. The
application of a context $M$ to a process $P$, written $M[P]$, is
tantamount to filling the hole in $M$ with $P$. In this paper we do
not need the full weight of this theory, but do make use of the notion
of context in the proof the main theorem. 

\begin{mathpar}
  \inferrule* [lab=summation] {} {{M_{M},M_{N}} \bc \Box \;|\; x.M_{A} \;|\; M_{M}+M_{N}}
  \and
  \inferrule* [lab=agent] {} {{M_{A}} \bc (\vec{x})M_{P} \;| \; \clift{P_0,\ldots,M_{P},\ldots,P_N}}
  \and \\
  \inferrule* [lab=process] {} {{M_{P}} \bc M_{N} \;| \;P|M_{P} }
\end{mathpar} 

\begin{mathpar}
  \inferrule* [lab=sychronization] {} {M_{N} \bc \Box \;|\; x?M_{F} \;|\; x!M_{C}}
  \and
  \inferrule* [lab=abstraction] {} {{M_{F}} \bc (x)M_{P} }
  \and
  \inferrule* [lab=concretion] {} {{M_{C}} \bc \langle M_{P} \rangle }
  \and \\
  \inferrule* [lab=process] {} {{M_{P}} \bc M_{N} \;| \;P|M_{P} }
\end{mathpar}

\begin{definition}[contextual application] Given a context $M$, and
  process $P$, we define the \emph{contextual application}, $M[P] :=
  M\{P/\Box\}$. That is, the contextual application of M to P is the
  substitution of $P$ for $\Box$ in $M$.
\end{definition}

$\meaningof{-} : L \to \mathcal{P}(\pi)$

\begin{mathpar}
  \inferrule* [lab=collection] {} {\meaningof{true} = \pi, \and \meaningof{~E} = \pi \setminus \meaningof{E}, \and \meaningof{E_{1} \& E_{2}} = \meaningof{E_{1}} \cap \meaningof{E_{2}}}
\end{mathpar}

\begin{mathpar}
  \inferrule* [lab=structure] {} {\meaningof{0} = \{ P \in \pi | P \equiv 0 \}, \and \\ \meaningof{E_1 | E_2} = \{ P \in \pi | P \equiv P_{1} | P_{2}, P_{1} \in \meaningof{E_{1}}, P_{2} \in \meaningof{E_2}\} }
\end{mathpar}

\begin{mathpar}
 \inferrule* [lab=behavior] {} {\meaningof{\langle a?b \rangle E} = \{ P \in \pi | P \equiv Q | u?(y)P', \\ \and \\\\ \and \\ \;\;\; u \in \meaningof{a}, \forall z.P'\{z/y\} \in \meaningof{E\{z/b\}}\}, \and \\ \meaningof{a!E} = \{ P \in \pi | P \equiv Q | x!\langle P' \rangle, x \in \meaningof{a} P' \in \meaningof{E}\} }
\end{mathpar}

\begin{mathpar}
 \inferrule* [lab=nominal] {} {\meaningof{\quotep{E}} = \{ \quotep{P} \in \quotep{\pi} | P \in \meaningof{E} \}, \and \meaningof{\quotep{P}} = \{ \quotep{Q} \in \quotep{\pi} | P \equiv Q \} \and \\ \meaningof{@\quotep{E}} = \{ P \in \pi | P \equiv @x, x \in \meaningof{E} \}}
\end{mathpar}

\begin{eqnarray*}
  \\
  \meaningof{-} : TS \to ST
\end{eqnarray*}

\begin{eqnarray*}
  \\
  L : TS \to ST
\end{eqnarray*}

\begin{eqnarray*}
  \\
  P \models E \iff P \in \meaningof{E}
\end{eqnarray*}

\begin{eqnarray*}
  P \approx_{L} Q \iff \forall E \in L. P \models E \iff Q \models E
\end{eqnarray*}

\begin{eqnarray*}
  P \approx_{K} Q
\end{eqnarray*}

\begin{eqnarray*}
  P \approx Q
\end{eqnarray*}

$\approx_{K} = \approx = \approx_{L}$

\subsubsection{Contextual duality}

Note that contexts extend the quotation operation to a family of
operations from processes to names. Given a context, $M$, we can
define a \emph{nominal context}, $\quotep{M}$ by $\quotep{M}[P] :=
\quotep{M[P]}$. To foreshadow what is to come we observe that these
operations enjoy a duality with processes very much like the duality
between vectors and maps from vectors to scalars.

Further, because the calculus is essentially higher-order, we have a
correspondence between contexts and processes. More specifically,
given a name $x$ and a context $M$ we can construct $M^{*}_{x}$ such
that 

\begin{mathpar}
  M^{*}_{x} | \lift{x}{P} \red M[P]
\end{mathpar}

namely,

\begin{mathpar}
  M^{*}_{x} := x?(u).M[\dropn{u}]
\end{mathpar}

The dependence of $M^{*}_{x}$ on a name makes it an abstraction, 

\begin{mathpar}
  M^{*} := (x)x?(u).M[\dropn{u}]
\end{mathpar}

\subsection{Additional notation}

It will sometimes be convenient to denote the process a name
quotes. We already have the notation $x = \quotep{P}$, but it will be
convenient to introduce an alternate notation, $\procn{x}$, when we
want to emphasize the connection to the use of the name. Note that, by
virtue of name equivalence, $\quotep{\procn{x}} \nameeq x$; so, the
notation is consistent with previous definitions.

Further, because names have structure it is possible to effect
substitutions on the basis of that structure. This means we need to
upgrade our notation for substitutions, which we accomplish by
adapting comprehension notation. Thus,

\begin{mathpar}
  P\{ y / x : x \in S \}
\end{mathpar}

is interpreted to mean the process derived from P by replacing (in a
capture-avoiding manner) each occurrence of $x$ in $S$ by $y$. For example,

\begin{mathpar}
  P\{ \quotep{\procn{x}|\procn{x}} / x : x \in \freenames{P} \}
\end{mathpar}

will replace each (occurrence) of a free name $x$ in $P$ by
$\quotep{\procn{x}|\procn{x}}$.

Also, we will avail ourselves of the notation $x^{L}$ and $x^{R}$ to
denote injections of a name into disjoint copies of the name
space. There are numerous ways to accomplish this. One example can be
found in \cite{MeredithR05}. This notation overloads to vectors of
names: $\vec{x}^{\pi} := (x_{i}^{\pi} \; : \; 0 \leq i < |\vec{x}| )$ where $\pi \in \{L,R\}$.

We also use $P^{\Box} := P|\Box$.

In \cite{MeredithR05} an interpretation of the new operator is
given. It turns out that there are several possible interpretations
all enjoying the requisite algebraic properties of the operator (see
\cite{milner91polyadicpi}). We will therefore make liberal use of
$(\nu\; \vec{x})P$.

% subsection the_syntax_and_semantics_of_the_notation_system (end)   

\input{qm2pi.qmops} 

\input{qm2pi.sterngerlach} 

\input{qm2pi.metric} 

% section concurrent_process_calculi (end)

%\input{qm2pi.proofsketch}

% section proof sketch (end)

%\input{qm2pi.slviaknots} 

% section spatial logic via knots (end)

\input{qm2pi.conclusion}

% section conclusion (end)

%\input{qm2pi.dtcodes} 

% section wiring algorithm (end)

\input{qm2pi.ack} 

% section acknowledgments (end)

\newpage


\bibliographystyle{plain}   
\bibliography{../../biblios/main.bib}

\input{qm2pi.rhodetails}

\end{document}

 

% section concurrent_process_calculi (end)

%\documentclass[12pt]{llncs}
%\documentclass{jktr}

\usepackage[pdftex]{hyperref}                   
\usepackage {listings}
\usepackage {mathpartir}
\usepackage{bcprules}
%\usepackage{listings}
                       
\usepackage{graphicx} 
%\usepackage[margins=2.5cm,nohead,nofoot]{geometry}
%\usepackage{geometry}
\usepackage{amsfonts}
\usepackage{amstext}
\usepackage{latexsym}
\usepackage{amssymb}
\usepackage{color}


%\include{myPreamble}
\include{qm2pi.local} 

%\ifpdf
%\usepackage[pdftex]{graphicx}
%\else
%\usepackage{graphicx}
%\fi

 % \ifpdf
%  \usepackage{pdfsync}
%  \if


%\title{Brief Article}
%\author{David F. Snyder}
%\author{L.G. Meredith}

%\address{Dept. of Math., Texas State University--San Marcos, San Marcos, TX 78666}
       
\pagestyle{empty}


\begin{document}

\lstset{language=[Objective]Caml,frame=shadowbox}

\input{qm2pi.front}

% section front matter (end)

\input{qm2pi.intro} 
 
% section introduction (end)

% \input{qm2pi.knotations} 

% section notation (end)

\input{qm2pi.process.calculi} 

% section concurrent_process_calculi_and_spatial_logics_ (end)
    
%\input{qm2pi.knots2pi} 

%\input{qm2pi.trefoil} 

%\input{qm2pi.mainthm} 

% subsection basic_interpretation (end)

%\input{qm2pi.rho.presentation} 
\subsection{The syntax and semantics of the notation system}\label{sub:the_syntax_and_semantics_of_the_notation_system} % (fold)

We now summarize a technical presentation of the calculus that
embodies our theory of dynamics. The typical presentation of such a
calculus follows the style of giving generators and relations on
them. The grammar, below, describing term constructors, freely
generates the set of processes, $\Proc$. This set is then quotiented
by a relation known as structural congruence and it is over this set
that the notion of dynamics is expressed. This presentation is
essentially that of \cite{MeredithR05} with the addition of
polyadicity and summation. For readability we have relegated some of
the technical subtleties to an appendix.

\subsubsection{Process grammar}\label{subsub:process_grammar}

\begin{mathpar}
  \inferrule* [lab=synchronization] {} {{M} \bc \pzero \;|\; x?F \;|\; x!C }
  \and
  \inferrule* [lab=abstraction] {} {{F} \bc (x)P}
  \and
  \inferrule* [lab=concretion] {} {{C} \bc \langle Q \rangle}
  \and
  \inferrule* [lab=process] {} {{P,Q} \bc M \;| \;P|Q \;|\; @{x}}
  \and
  \inferrule* [lab=name] {} {{x} \bc \quotep{P}}
\end{mathpar} 

Note that $\vec{x}$ (resp. $\vec{P}$) denotes a vector of names
(resp. processes) of length $|\vec{x}|$ (resp. $|\vec{P}|$). We adopt
the following useful abbreviations.

\begin{mathpar}
   x?(\vec{y}).P := x.(\vec{y})P \and  x\clift{\vec{P}} := x.\clift{\vec{P}}
   \and x!(y) := \lift{x}{\dropn{y}}
   \and \Pi_{i=0}^{n-1}P_i := P_0 | \ldots | P_{n-1}
\end{mathpar}

\subsubsection{Structural congruence}

\paragraph{Free and bound names and alpha-equivalence.} At the
core of structural equivalence is alpha-equivalence which identifies
process that are the same up to a change of variable. Formally, we
recognize the distinction between free and bound names. The free names
of a process, $\freenames{P}$, may be calculated recursively as
follows:

\begin{mathpar}
\freenames{\pzero} := \emptyset
  \and \\
  \freenames{x?(y).P} := \{ x \} \cup (\freenames{P} \setminus \{ y \})
  \and 
  \freenames{x!\langle P \rangle} := \{ x \} \cup \{ P \} 
  \and \\
  \freenames{P|Q} := \freenames{P} \cup \freenames{Q}
  \and \\
  \freenames{@{x}} := \{ x \}
\end{mathpar}

$\pi$
$\quotep{\pi}$

$\freenames{-} : \pi \to \mathcal{P}(\quotep{\pi})$

\begin{eqnarray*}
  \freenames{\pzero} & := & \emptyset \\
  \freenames{x?(y).P} & := & \{ x \} \cup (\freenames{P} \setminus \{ y \}) \\
  \freenames{x!\langle P \rangle} & := & \{ x \} \cup \{ P \} \\
  \freenames{P|Q} & := & \freenames{P} \cup \freenames{Q} \\
  \freenames{\dropn{x}} & := & \{ x \}
\end{eqnarray*}

The bound names of a process, $\boundnames{P}$, are those names occurring in $P$
that are not free. For example, in $x?(y).0$, the name $x$ is free, while $y$ is bound.

\begin{mathpar}
  \inferrule* [lab=monoidal-laws] {} { P|Q \equiv Q|P \and P|0 \equiv P \and P|(Q|R) \equiv (P|Q)|R }
\end{mathpar}

\begin{mathpar}
  \inferrule* [lab=alpha-equivalence] {} { (x)P \equiv (y)P\{y/x\} \and y \not\in \freenames{P} }
\end{mathpar}

\begin{definition}
Then two processes, $P,Q$, are alpha-equivalent if $P = Q\{\vec{y}/\vec{x}\}$ for
some $\vec{x} \in \boundnames{Q},\vec{y} \in \boundnames{P}$, where $Q\{\vec{y}/\vec{x}\}$
denotes the capture-avoiding substitution of $\vec{y}$ for $\vec{x}$ in $Q$.
\end{definition}

\begin{definition}
  The {\em structural congruence} \cite{SangiorgiWalker} , $\equiv$,
  between processes is the least congruence containing
  alpha-equivalence, satisfying the abelian monoid laws
  (associativity, commutativity and $\pzero$ as identity) for parallel
  composition $|$ and for summation $+$.
\end{definition}

\subsection{Name equivalence}

We take name equivalence, written $\nameeq$, to be the smallest
equivalence relation generated by the following rules.

\begin{mathpar}
\inferrule*[lab=Quote-drop]
{ }
{ \quotep{@{x}} \nameeq x }

\inferrule*[lab=Struct-equiv]
{ P \scong Q }
{ \quotep{P} \nameeq \quotep{Q} }
\end{mathpar}

The astute reader will have noticed that the mutual recursion of names
and processes imposes a mutual recursion on alpha-equivalence and
structural equivalence via name-equivalence. Fortunately, all of this
works out pleasantly and we may calculate in the natural way, free of
concern. The reader interested in the details is referred to the
appendix \ref{appendix:rho_details}.

\subsection{Substitution}

We use $\Proc$ for the set of processes, $\QProc$ for the set of
names, and $\id{\{}\vec{y} / \vec{x} \id{\}}$ to denote partial maps,
$s : \QProc \rightarrow \QProc$. A map, $s$ lifts, uniquely, to a map
on process terms, $\widehat{s} : \Proc \rightarrow \Proc$ by the
following equations.

\begin{mathpar}
  (0) \psubstp{Q}{P} := 0 \\
  (R \juxtap S) \psubstp{Q}{P}
  :=    
  (R)\psubstp{Q}{P} \juxtap (S) \psubstp{Q}{P} \\
  (x?(y).R) \psubstp{Q}{P}    
  :=    
  (x)\substp{Q}{P} (z)\concat( (R \psubstn{z}{y}) \psubstp{Q}{P} ) \\
  (\lift{x}{R}) \psubstp{Q}{P}  
  :=
  \lift{(x)\substp{Q}{P}}{ R \psubstp{Q}{P} } \\
%   (\dropn{x})  \psubstp{Q}{P}       
%   := 
%   \left\{ 
%     \begin{array}{ccc} 
%       \dropn{\quotep{Q}} & & x \nameeq \quotep{P} \\
%       \dropn{x} & & otherwise \\
%     \end{array}
%   \right. 
  (\dropn{x})  \psubstp{Q}{P}       
  := 
  \left\{ 
    \begin{array}{ccc} 
      Q & & x \nameeq \quotep{P} \\
      \dropn{x} & & otherwise \\
    \end{array}
  \right.
\end{mathpar}
 

where

\begin{eqnarray}
  (x)\id{\{} \lpquote Q \rpquote / \lpquote P \rpquote \id{\}}            = 
  \left\{ 
    \begin{array}{ccc}
      \lpquote Q \rpquote & & x \nameeq \lpquote P \rpquote \\
      x & & otherwise \\
    \end{array}
  \right. \nonumber
\end{eqnarray}

and $z$ is chosen distinct from $\quotep{P}$, $\quotep{Q}$, the free
names in $Q$, and all the names in $R$. Our $\alpha$-equivalence will
be built in the standard way from this substitution.

\begin{remark}\label{rem:no_self_referential_names}
  One consequence of these definitions is that $\forall P. \quotep{P}
  \not\in \freenames{P}$.
\end{remark}

\subsection{ Dynamic quote: an example }

Anticipating something of what's to come, consider applying the
substitution, $\widehat{\id{\{}u / z \id{\}}}$, to the following pair
of processes, $\lift{w}{y!(z)}$ and $w[ \lpquote y!(z) \rpquote ]$.

\begin{eqnarray}
	\lift{w}{y!(z)}\widehat{\id{\{}u / z \id{\}}}
		& = &
		\lift{w}{y!(u)} \nonumber\\
	w[ \lpquote y!(z) \rpquote ] \widehat{ \id{\{}u / z \id{\}} }
		& = &
		w[ \lpquote y!(z) \rpquote ] \nonumber
\end{eqnarray}

Because the body of the process between quotes is impervious to
substitution, we get radically different answers. In fact, by
examining the first process in an input context,
e.g. $x?(z).\lift{w}{y!(z)}$, we see that the process under the lift
operator may be shaped by prefixed inputs binding a name inside it. In
this sense, the lift operator will be seen as a way to dynamically
construct processes before reifying them as names.

Finally equipped with these standard features we can present the
dynamics of the calculus.

\subsubsection{Operational semantics} 

Finally, we introduce the computational dynamics. What marks these
algebras as distinct from other more traditionally studied algebraic
structures, e.g. vector spaces or polynomial rings, is the manner in
which dynamics is captured. In traditional structures, dynamics is typically
expressed through morphisms between such structures, as in linear maps
between vector spaces or morphisms between rings. In algebras
associated with the semantics of computation, the dynamics is
expressed as part of the algebraic structure itself, through a
reduction reduction relation typically denoted by $\red$. Below, we
give a recursive presentation of this relation for the calculus used
in the encoding.

$\red \subseteq \pi \times \pi$
$\red : \pi \to \mathcal{P}(\pi)$

\begin{mathpar}
  \inferrule* [lab=Comm] { \textsf{match}( x_{src}, x_{trgt} ) } { x_{trgt}?(y)P \; | \; x_{src}!\langle {Q} \rangle \red P\{\quotep{Q}/y}\} }
  \and \\
  \inferrule* [lab=Par] {{P} \red {P}'} {{{P} | {Q}} \red {{P}' | {Q}}}
  \and
  \inferrule* [lab=Equiv]{{{P} \scong {P}'} \andalso {{P}' \red {Q}'} \andalso {{Q}' \scong {Q}}}{{P} \red {Q}}
\end{mathpar}

\begin{eqnarray*}
  match_{\equiv} (\quotep{P},\quotep{Q}) & := & P \equiv Q \\
  match_{\dagger}(\quotep{P},\quotep{Q}) & := & \forall R. P|Q \red^{*} R => R \red^{*} 0 \\
  match_{K}(\quotep{P},\quotep{Q}) & := & K \mbox{ for some context } K
\end{eqnarray*}

$u?(x)P | u!\langle Q \rangle \red P\{\quotep{Q}/x\}$

%We write $\wred$ for $\red^*$, and $P\red$ if $\exists Q $ such that $ P \red Q$.
We write $P\red$ if $\exists Q $ such that $ P \red Q$ and $P\not\red$, otherwise.

\section{Replication}

As mentioned before, it is known that replication (and hence
recursion) can be implemented in a higher-order process algebra
\cite{SangiorgiWalker}. As our first example of calculation with the
machinery thus far presented we give the construction explicitly in
the {\rhoc}.

\begin{eqnarray}
	D_{x} & := & \prefix{x}{y}{(\binpar{\outputp{x}{y}}{@{y}})} \nonumber\\
	\bangp_{x}{P} & := & \binpar{{x}!\langle{\binpar{D_{x}}{P}}\rangle}{D_{x}} \nonumber
\end{eqnarray}

\begin{eqnarray}
	\bangp_{x}{P} & & \nonumber\\
	=
	& {x}!\langle{(\prefix{x}{y}{(\outputp{x}{y} | @{y})) | P}}\rangle 
	      | \prefix{x}{y}{(\outputp{x}{y} | @{y})} & \nonumber\\
	\red
	& (\outputp{x}{y} | @{y})\substn{\quotep{(\prefix{x}{y}{(@{y} | \outputp{x}{y})) | P}}}{y} & \nonumber\\
	=
	& \outputp{x}{\quotep{(\prefix{x}{y}{(\outputp{x}{y} | @{y})) | P}}}
	  | {(\prefix{x}{y}{(\outputp{x}{y} | @{y})) | P}} & \nonumber\\
	\red
	& \ldots & \nonumber\\
	\red^*
	& P | P | \ldots & \nonumber
\end{eqnarray}

Of course, this encoding, as an implementation, runs away, unfolding
$\bangp{P}$ eagerly. A lazier and more implementable replication
operator, restricted to input-guarded processes, may be obtained as follows.

\begin{eqnarray}
\bangp{\prefix{u}{v}{P}} 
	:= 
	\binpar{\lift{x}{\prefix{u}{v}{(\binpar{D(x)}{P})}}}{D(x)} \nonumber
\end{eqnarray}

\begin{remark}
  Note that the lazier definition still does not deal with summation
  or mixed summation (i.e. sums over input and output). The reader is
  invited to construct definitions of replication that deal with these
  features. 

  Further, the definitions are parameterized in a name, $x$. Can you,
  gentle reader, make a definition that eliminates this parameter and
  guarantees no accidental interaction between the replication
  machinery and the process being replicated -- i.e. no accidental
  sharing of names used by the process to get its work done and the
  name(s) used by the replication to effect copying. This latter
  revision of the definition of replication is crucial to obtaining
  the expected identity $!!P \sim !P$.
\end{remark}

\begin{remark}\label{rem:paradoxical_combinator}
  The reader familiar with the lambda calculus will have noticed the
  similarity between $D$ and the paradoxical combinator.

  [Ed. note: the existence of this seems to suggest we have to be more
  restrictive on the set of processes and names we admit if we are to
  support no-cloning.]
\end{remark}

\subsubsection{Bisimulation}

The computational dynamics gives rise to another kind of equivalence,
the equivalence of computational behavior. As previously mentioned
this is typically captured \emph{via} some form of bisimulation.

% The notion we use in this paper is weak barbed bisimulation
% \cite{milner91polyadicpi}.

The notion we use in this paper is derived from weak barbed
bisimulation \cite{milner91polyadicpi}. 

\begin{definition}
An \emph{observation relation}, $\downarrow_{\mathcal N}$, over a set
of names, $\mathcal N$, is the smallest relation satisfying the rules
below.

\infrule[Out-barb]{y \in {\mathcal N}, \; x \nameeq y}
		  {\outputp{x}{v} \downarrow_{\mathcal N} x}
\infrule[Par-barb]{\mbox{$P\downarrow_{\mathcal N} x$ or $Q\downarrow_{\mathcal N} x$}}
		  {\binpar{P}{Q} \downarrow_{\mathcal N} x}

We write $P \Downarrow_{\mathcal N} x$ if there is $Q$ such that 
$P \wred Q$ and $Q \downarrow_{\mathcal N} x$.
\end{definition}

\begin{definition}
%\label{def.bbisim}
An  ${\mathcal N}$-\emph{barbed bisimulation} over a set of names, ${\mathcal N}$, is a symmetric binary relation 
${\mathcal S}_{\mathcal N}$ between agents such that $P\rel{S}_{\mathcal N}Q$ implies:
\begin{enumerate}
\item If $P \red P'$ then $Q \wred Q'$ and $P'\rel{S}_{\mathcal N} Q'$.
\item If $P\downarrow_{\mathcal N} x$, then $Q\Downarrow_{\mathcal N} x$.
\end{enumerate}
$P$ is ${\mathcal N}$-barbed bisimilar to $Q$, written
$P \wbbisim_{\mathcal N} Q$, if $P \rel{S}_{\mathcal N} Q$ for some ${\mathcal N}$-barbed bisimulation ${\mathcal S}_{\mathcal N}$.
\end{definition}

$\mathcal{R} \subseteq \pi \times \pi$

$P \mathcal{R} Q => \forall P'. P \red P' \Rightarrow \exists Q'. Q \red Q', P' \mathcal{R} Q'$

$P \vdash x \Rightarrow Q \vdash x$

\begin{mathpar}
  \inferrule*[lab=Out-barb]{x \nameeq y}{{y}!\langle{Q}\rangle \vdash x}
  \and
  \inferrule*[lab=Par-barb]{\mbox{$P\vdash x$ or $Q\vdash x$}}{\binpar{P}{Q} \vdash x}
\end{mathpar}

\subsubsection{Contexts}

One of the principle advantages of computational calculi like the
$\pi$-calculus is a well-defined notion of context,
contextual-equivalence and a correlation between
contextual-equivalence and notions of bisimulation. The notion of
context allows the decomposition of a process into (sub-)process and
its syntactic environment, its context. Thus, a context may be
thought of as a process with a ``hole'' (written $\Box$) in it. The
application of a context $M$ to a process $P$, written $M[P]$, is
tantamount to filling the hole in $M$ with $P$. In this paper we do
not need the full weight of this theory, but do make use of the notion
of context in the proof the main theorem. 

\begin{mathpar}
  \inferrule* [lab=summation] {} {{M_{M},M_{N}} \bc \Box \;|\; x.M_{A} \;|\; M_{M}+M_{N}}
  \and
  \inferrule* [lab=agent] {} {{M_{A}} \bc (\vec{x})M_{P} \;| \; \clift{P_0,\ldots,M_{P},\ldots,P_N}}
  \and \\
  \inferrule* [lab=process] {} {{M_{P}} \bc M_{N} \;| \;P|M_{P} }
\end{mathpar} 

\begin{mathpar}
  \inferrule* [lab=sychronization] {} {M_{N} \bc \Box \;|\; x?M_{F} \;|\; x!M_{C}}
  \and
  \inferrule* [lab=abstraction] {} {{M_{F}} \bc (x)M_{P} }
  \and
  \inferrule* [lab=concretion] {} {{M_{C}} \bc \langle M_{P} \rangle }
  \and \\
  \inferrule* [lab=process] {} {{M_{P}} \bc M_{N} \;| \;P|M_{P} }
\end{mathpar}

\begin{definition}[contextual application] Given a context $M$, and
  process $P$, we define the \emph{contextual application}, $M[P] :=
  M\{P/\Box\}$. That is, the contextual application of M to P is the
  substitution of $P$ for $\Box$ in $M$.
\end{definition}

$\meaningof{-} : L \to \mathcal{P}(\pi)$

\begin{mathpar}
  \inferrule* [lab=collection] {} {\meaningof{true} = \pi, \and \meaningof{~E} = \pi \setminus \meaningof{E}, \and \meaningof{E_{1} \& E_{2}} = \meaningof{E_{1}} \cap \meaningof{E_{2}}}
\end{mathpar}

\begin{mathpar}
  \inferrule* [lab=structure] {} {\meaningof{0} = \{ P \in \pi | P \equiv 0 \}, \and \\ \meaningof{E_1 | E_2} = \{ P \in \pi | P \equiv P_{1} | P_{2}, P_{1} \in \meaningof{E_{1}}, P_{2} \in \meaningof{E_2}\} }
\end{mathpar}

\begin{mathpar}
 \inferrule* [lab=behavior] {} {\meaningof{\langle a?b \rangle E} = \{ P \in \pi | P \equiv Q | u?(y)P', \\ \and \\\\ \and \\ \;\;\; u \in \meaningof{a}, \forall z.P'\{z/y\} \in \meaningof{E\{z/b\}}\}, \and \\ \meaningof{a!E} = \{ P \in \pi | P \equiv Q | x!\langle P' \rangle, x \in \meaningof{a} P' \in \meaningof{E}\} }
\end{mathpar}

\begin{mathpar}
 \inferrule* [lab=nominal] {} {\meaningof{\quotep{E}} = \{ \quotep{P} \in \quotep{\pi} | P \in \meaningof{E} \}, \and \meaningof{\quotep{P}} = \{ \quotep{Q} \in \quotep{\pi} | P \equiv Q \} \and \\ \meaningof{@\quotep{E}} = \{ P \in \pi | P \equiv @x, x \in \meaningof{E} \}}
\end{mathpar}

\begin{eqnarray*}
  \\
  \meaningof{-} : TS \to ST
\end{eqnarray*}

\begin{eqnarray*}
  \\
  L : TS \to ST
\end{eqnarray*}

\begin{eqnarray*}
  \\
  P \models E \iff P \in \meaningof{E}
\end{eqnarray*}

\begin{eqnarray*}
  P \approx_{L} Q \iff \forall E \in L. P \models E \iff Q \models E
\end{eqnarray*}

\begin{eqnarray*}
  P \approx_{K} Q
\end{eqnarray*}

\begin{eqnarray*}
  P \approx Q
\end{eqnarray*}

$\approx_{K} = \approx = \approx_{L}$

\subsubsection{Contextual duality}

Note that contexts extend the quotation operation to a family of
operations from processes to names. Given a context, $M$, we can
define a \emph{nominal context}, $\quotep{M}$ by $\quotep{M}[P] :=
\quotep{M[P]}$. To foreshadow what is to come we observe that these
operations enjoy a duality with processes very much like the duality
between vectors and maps from vectors to scalars.

Further, because the calculus is essentially higher-order, we have a
correspondence between contexts and processes. More specifically,
given a name $x$ and a context $M$ we can construct $M^{*}_{x}$ such
that 

\begin{mathpar}
  M^{*}_{x} | \lift{x}{P} \red M[P]
\end{mathpar}

namely,

\begin{mathpar}
  M^{*}_{x} := x?(u).M[\dropn{u}]
\end{mathpar}

The dependence of $M^{*}_{x}$ on a name makes it an abstraction, 

\begin{mathpar}
  M^{*} := (x)x?(u).M[\dropn{u}]
\end{mathpar}

\subsection{Additional notation}

It will sometimes be convenient to denote the process a name
quotes. We already have the notation $x = \quotep{P}$, but it will be
convenient to introduce an alternate notation, $\procn{x}$, when we
want to emphasize the connection to the use of the name. Note that, by
virtue of name equivalence, $\quotep{\procn{x}} \nameeq x$; so, the
notation is consistent with previous definitions.

Further, because names have structure it is possible to effect
substitutions on the basis of that structure. This means we need to
upgrade our notation for substitutions, which we accomplish by
adapting comprehension notation. Thus,

\begin{mathpar}
  P\{ y / x : x \in S \}
\end{mathpar}

is interpreted to mean the process derived from P by replacing (in a
capture-avoiding manner) each occurrence of $x$ in $S$ by $y$. For example,

\begin{mathpar}
  P\{ \quotep{\procn{x}|\procn{x}} / x : x \in \freenames{P} \}
\end{mathpar}

will replace each (occurrence) of a free name $x$ in $P$ by
$\quotep{\procn{x}|\procn{x}}$.

Also, we will avail ourselves of the notation $x^{L}$ and $x^{R}$ to
denote injections of a name into disjoint copies of the name
space. There are numerous ways to accomplish this. One example can be
found in \cite{MeredithR05}. This notation overloads to vectors of
names: $\vec{x}^{\pi} := (x_{i}^{\pi} \; : \; 0 \leq i < |\vec{x}| )$ where $\pi \in \{L,R\}$.

We also use $P^{\Box} := P|\Box$.

In \cite{MeredithR05} an interpretation of the new operator is
given. It turns out that there are several possible interpretations
all enjoying the requisite algebraic properties of the operator (see
\cite{milner91polyadicpi}). We will therefore make liberal use of
$(\nu\; \vec{x})P$.

% subsection the_syntax_and_semantics_of_the_notation_system (end)   

\input{qm2pi.qmops} 

\input{qm2pi.sterngerlach} 

\input{qm2pi.metric} 

% section concurrent_process_calculi (end)

%\input{qm2pi.proofsketch}

% section proof sketch (end)

%\input{qm2pi.slviaknots} 

% section spatial logic via knots (end)

\input{qm2pi.conclusion}

% section conclusion (end)

%\input{qm2pi.dtcodes} 

% section wiring algorithm (end)

\input{qm2pi.ack} 

% section acknowledgments (end)

\newpage


\bibliographystyle{plain}   
\bibliography{../../biblios/main.bib}

\input{qm2pi.rhodetails}

\end{document}



% section proof sketch (end)

%\section{Unlikely characters: spatial logic for
  knots}\label{sub:characteristic_formulae} % (fold)

Associated to the mobile process calculi are a family of logics known
as the Hennessy-Milner logics. These logics typically enjoy a
semantics interpreting formulae as sets of processes that when
factored through the encoding outlined above allows an identification
of classes of knots with logical formulae. In the context of this
encoding the sub-family known as the spatial logics \cite{CairesC03}
\cite{CairesC04} \cite{Caires04} are of particular interest providing
several important features for expressing and reasoning about
properties (i.e. classes) of knots. We hint here at how this may be done.

%\begin{description}
%\item [structural connectives] 
\subsubsection{Structural connectives} The spatial logics enjoy
structural connectives corresponding, at the logical level, to the
parallel composition ($P | Q$) and new name ($(\nu \; x)P$)
connectives for processes. As illustrated in the examples below, these
connectives are extremely expressive given the shape of our encoding.
%\item [decideable satisfaction]

\subsubsection{Decideable satisfaction}
In \cite{Caires04} the satisfaction relation is shown to be decideable
for a rich class of processes. It further turns out that the image of
the our encoding is a proper subset of that class. This result
provides the basis for an algorithm by which to search for knots
enjoying a given property.
%\item [characteristic formulae]

\subsubsection{Characteristic formulae}
In the same paper \cite{Caires04} , Caires presents a means of calculating
characteristic formulae, selecting equivalence classes of processes
up to a pre--specified depth limit on the support set of names. Composed with our
encoding, this characteristic formula can be used to select
characteristic formulae for knots.
%\end{description}

\subsubsection{Spatial logic formulae}

The grammar below (segmented for comprehension) summarizes the syntax
of spatial logic formulae. We employ illustrative examples in the
sequel to provide an intuitive understanding of their meaning
referring the reader to \cite{Caires04} for a more detailed explication
of the semantics.

\begin{mathpar}
  \inferrule* [lab=boolean] {} {{A,B} \bc T \;|\; \neg A \;|\; A \wedge B \;|\; \eta = \eta'}
  \and
  \inferrule* [lab=spatial] {} {|\; \pzero \;|\; A | B \;|\; x \text{\textregistered} A \;|\; \forall x . A \;|\;  H x . A}
  \and
  \inferrule* [lab=behavioral] {} {|\; \alpha . A}
  \and 
  \inferrule* [lab=recursion] {} {|\; X(\vec{u}) \;|\; \mu X(\vec{u}) . A}
  \and
  \inferrule* [lab=action] {} {\alpha \bc \langle x?(\vec{y}) \rangle \;|\; \langle x!(\vec{y}) \rangle \;|\; \langle \tau \rangle}
  \and 
  \inferrule* [lab=name] {} {\eta \bc x \;|\; \tau}
\end{mathpar} 

% subsection characteristic_formulae (end)   	 

\subsection{Example formulae}\label{sub:example_formulae_} % (fold)

\subsubsection{Crossing as formula.}
% 
% \begin{align*}
%   \frac{d}{dx} \sin x &= \cos x 
%   & \frac{d}{dx} e^x &= e^x \\
%   \frac{d}{dx} \cos x &= - \sin x 
%   & \frac{d}{dx} \log x &= \frac{1}{x} \\
% \end{align*} 

\begin{align*}
 \mu C(x_{0},x_{1},y_{0},y_{1},u).&(\langle x_{0}?(z) \rangle(\langle u! \rangle\langle y_{1}!z \rangle C(x_{0},x_{1},y_{0},y_{1},u)) & \\
  & \wedge \langle y_{1}?(z) \rangle (\langle u! \rangle \langle x_{0}!z \rangle C(x_{0},x_{1},y_{0},y_{1},u)) & \\
  & \wedge \langle x_{1}?(z) \rangle (\langle u? \rangle \langle y_{0}!z \rangle C(x_{0},x_{1},y_{0},y_{1},u)) & \\
  & \wedge \langle y_{0}?(z) \rangle (\langle u? \rangle \langle x_{1}!z \rangle C(x_{0},x_{1},y_{0},y_{1},u))) &
\end{align*}

The lexicographical similarity between the shape of this formulae and
the shape of definition of the process representing a crossing reveals
the intuitive meaning of this formulae. It describes the capabilities
of a process that has the right to represent a crossing. For example
it picks out processes that may perform an input on the port $x_0$ in
its initial menu of capabilities. What differentiates the formula
from the process, however, is that the crossing process is the
smallest candidate to satisfy the formula. Infinitely many other
processes -- with internal behavior hidden behind this interface, so
to speak -- also satisfy this formula. Even this simple formula,
then, can be seen to open a new view onto knots, providing a
computational interpretation of \emph{virtual} knots.

Note that this formula is derived by hand. A similar formula can be
derived by employing Caires' calculation of characteristic formula
\cite{Caires04} to the process representing a crossing. In light of
this discussion, we let
$\meaningof{C}_{\phi}(x0,x1,y0,y1,u)$ denote a formula specifying the
dynamics we wish to capture of a crossing. To guarantee we preserve
the shape of the interface and minimal semantics we demand that
$\meaningof{C}_{\phi}(x0,x1,y0,y1,u) \Rightarrow
\textbf{C}(x0,x1,y0,y1,u)$ where $\textbf{C}(x0,x1,y0,y1,u)$ denotes
the formula above.
                            
\subsubsection{Crossing number constraints.}
The moral content of the context lemma (Lemma \ref{context}) is that the notion of
``locality'' in the Reidemeister moves is effectively captured by the
parallel composition operator of the process calculus. This intuition
extends through the logic. Given a formula,
$\meaningof{C}_{\phi}(x0,x1,y0,y1,u)$, we can use the structural
connectives to specify constraints on crossing numbers, such as at
least $n$ crossings, or exactly $n$ crossings.
\begin{mathpar}
  \inferrule* [lab=at-least-n] {} { K^{\geq n}_{\phi}(\vec{xs},\vec{ys}) := \Pi_{i=0}^{n-1} Hu . \meaningof{C}_{\phi}(xs_i,ys_i,u) | T }
  \and 
  \inferrule* [lab=exactly-n] {} { K^{= n}_{\phi}(\vec{xs},\vec{ys}) := \Pi_{i=0}^{n-1} Hu . \meaningof{C}_{\phi}(xs_i,ys_i,u) | \neg (\forall x_0,y_0,x_1,y_1,u . \meaningof{C}_{\phi}(x_0,y_0,x_1,y_1,u) | T) }
\end{mathpar}

To round out this section, recall that the encoding of an $n$-crossing
knot decomposes into a parallel composition of $n$ \emph{copies} of a
crossing process together with a wiring harness. To specify different
knot classes with the same crossing number amounts to specifying
logical constraints on the wiring harness. In the interest of space,
we defer examples to a forthcoming paper. Suffice it to say that both
the conditions ``alternating knot'' and ``contains the tangle
corresponding to 5/3'' are expressible. For example, it is possible to
calculate the characteristic formula of a process corresponding to the
tangle 5/3 and conjoin it into the classifying formula via the
composition connective of the logic.

Finally, we wish to observe that it is entirely within reason to
contemplate a more domain-specific version of spatial logic tailored
to the shape of processes in the image of the encoding. Such a
domain-specific logic would have a better claim to the title formal
language of knot properties.

% subsection example_formulae_ (end)

% section knots_as_processes (end) 

% section spatial logic via knots (end)

\section{Conclusions and future work}

\paragraph{Testing physical space}
You, gentle reader, may wonder why of all the theorems to be proved
given this set up we pick the one above. In some sense it's hardly
central to quantum mechanics. We see it as central in the sense that
it firmly establishes a notion of physical space arising from a notion
of the equivalence of behavior. Relating bisimulation to a metric is a
big step forward, but one is faced with interpreting the relationship
of that metric space to something more physical. Quantum mechanical
notions of ``physical'' space are still far from intuitive, but by
relating this idea of distance as testing to calculations that predict
physical circumstances we are making a not insignificant step forward
toward an understanding of the physical space we inhabit as
essentially dynamic.

\paragraph{Effectivity and simulation}
One of the observations we have yet to make is that the entire program
spelled out here is effective. We have built various interpreters for
the reflective calculus at work in this interpretation. In principle,
then, we can simulate quantum mechanics on a computer. The place where
the simulation may lose fidelity is the infinitely branching summation
for the annihilator.

In this connection i also want to point out that the evaluation style
calculation of the inner product puts the non-determinism of the
summation right at the heart of measurement. This suggests that
Milner's original reduction-based formulation of the dynamics of his
calculi in terms of sums was not just notationally suggestive of a
notion of measure-and-continue but captured some significant part of
the physics.

\paragraph{Quantum continuations}
In light of this last observation i want to point out that the
predominant account of quantum mechanics is missing a key aspect of a
truly compositional story of the physical situation. In a real lab,
when a measurement is made the observation can be made to feed into
another device that then makes another measurement conditioned on the
results of the first. This means that after the superposition was
collapsed the entire experimental set up remained in
superposition. While QM offers a means of writing this down it doesn't
quite line up well with the well-trodden formulation of computation
and continuation that we see so succinctly expressed in Milner's
calculi. This suggests that there might be advantages to this account
of dynamics waiting to be explored.

\paragraph{Quantum logic}
In this connection, we also note that by virtue of having the
Hennessy-Milner construction, we can pull the construction through the
interpretation of QM. This gives us a natural candidate for a quantum
logic that enjoys an extremely tight connection with it's domain of
interpretation, making the construction much less ad hoc (rather it is
the image of functor!).

\paragraph{Quantum probabiity}
i have questions about the basis of the interpretation of inner
product as probability amplitude. In particular, using which
axiomatization of probability theory does the notion of probability
amplitude earn the right to be so dubbed? In other words, where is the
proof that the operation for calculating a probability amplitude (and
then squaring) satisfies the axioms of what it means to calculate a
probability? Even if such a proof exists (i have yet to find it in the
literature), i wonder if it might not be possible to turn things on
their heads. Can we view the calculation of the probability amplitude
as an axiomatization of probability? If so, then the definition we
give for calculating probability amplitude may provide the basis for
an \emph{effective} theory of probability.

\paragraph{Quantum vs ``biological'' information}
Finally, i want to conclude with a more philosophical observation. At
a recent workshop in which QM was a predominant topic i noticed
something about quantum information. The speaker was giving a riveting
discussion of axiomatic QM and showing how properties of ``no
cloning'' and ``no deleting'' emerged as consequences of the
axiomatization. Theorems of this form are necessary to give us a sense
of confidence that our axioms characterize the physical theory. What
struck me, though, was that if quantum information is neither erasable
nor replicable it is markedly different from \emph{life}. Two of the
things we know about life is that

\begin{itemize}
  \item it ends;
  \item to gain some measure of persistence, to transcend it's
    finitude it is imminently copyable.
\end{itemize}

Both of these qualities are summarized succinctly in the aphorism: all
flesh is grass. For me these two kinds of ``information'' -- call them
quantum and biological -- are end points on a spectrum of strategies
for persistence. At one end, we have those curious entities that enjoy
uniqueness and permanence; at the other, we have those who in the face
of a certain end and an uncertain present make a go of passing
something on. To me one of the more remarkable aspects of the latter
strategy is that in the presence of noise (and certain features of
copying) we get a kind of dynamism, a chance for improvement against a
given persistent condition.

% subsection other_calculi_other_bisimulations_and_geometry_as_behavior (end)




% section conclusion (end)

%\documentclass[12pt]{llncs}
%\documentclass{jktr}

\usepackage[pdftex]{hyperref}                   
\usepackage {listings}
\usepackage {mathpartir}
\usepackage{bcprules}
%\usepackage{listings}
                       
\usepackage{graphicx} 
%\usepackage[margins=2.5cm,nohead,nofoot]{geometry}
%\usepackage{geometry}
\usepackage{amsfonts}
\usepackage{amstext}
\usepackage{latexsym}
\usepackage{amssymb}
\usepackage{color}


%\include{myPreamble}
\include{qm2pi.local} 

%\ifpdf
%\usepackage[pdftex]{graphicx}
%\else
%\usepackage{graphicx}
%\fi

 % \ifpdf
%  \usepackage{pdfsync}
%  \if


%\title{Brief Article}
%\author{David F. Snyder}
%\author{L.G. Meredith}

%\address{Dept. of Math., Texas State University--San Marcos, San Marcos, TX 78666}
       
\pagestyle{empty}


\begin{document}

\lstset{language=[Objective]Caml,frame=shadowbox}

\input{qm2pi.front}

% section front matter (end)

\input{qm2pi.intro} 
 
% section introduction (end)

% \input{qm2pi.knotations} 

% section notation (end)

\input{qm2pi.process.calculi} 

% section concurrent_process_calculi_and_spatial_logics_ (end)
    
%\input{qm2pi.knots2pi} 

%\input{qm2pi.trefoil} 

%\input{qm2pi.mainthm} 

% subsection basic_interpretation (end)

%\input{qm2pi.rho.presentation} 
\subsection{The syntax and semantics of the notation system}\label{sub:the_syntax_and_semantics_of_the_notation_system} % (fold)

We now summarize a technical presentation of the calculus that
embodies our theory of dynamics. The typical presentation of such a
calculus follows the style of giving generators and relations on
them. The grammar, below, describing term constructors, freely
generates the set of processes, $\Proc$. This set is then quotiented
by a relation known as structural congruence and it is over this set
that the notion of dynamics is expressed. This presentation is
essentially that of \cite{MeredithR05} with the addition of
polyadicity and summation. For readability we have relegated some of
the technical subtleties to an appendix.

\subsubsection{Process grammar}\label{subsub:process_grammar}

\begin{mathpar}
  \inferrule* [lab=synchronization] {} {{M} \bc \pzero \;|\; x?F \;|\; x!C }
  \and
  \inferrule* [lab=abstraction] {} {{F} \bc (x)P}
  \and
  \inferrule* [lab=concretion] {} {{C} \bc \langle Q \rangle}
  \and
  \inferrule* [lab=process] {} {{P,Q} \bc M \;| \;P|Q \;|\; @{x}}
  \and
  \inferrule* [lab=name] {} {{x} \bc \quotep{P}}
\end{mathpar} 

Note that $\vec{x}$ (resp. $\vec{P}$) denotes a vector of names
(resp. processes) of length $|\vec{x}|$ (resp. $|\vec{P}|$). We adopt
the following useful abbreviations.

\begin{mathpar}
   x?(\vec{y}).P := x.(\vec{y})P \and  x\clift{\vec{P}} := x.\clift{\vec{P}}
   \and x!(y) := \lift{x}{\dropn{y}}
   \and \Pi_{i=0}^{n-1}P_i := P_0 | \ldots | P_{n-1}
\end{mathpar}

\subsubsection{Structural congruence}

\paragraph{Free and bound names and alpha-equivalence.} At the
core of structural equivalence is alpha-equivalence which identifies
process that are the same up to a change of variable. Formally, we
recognize the distinction between free and bound names. The free names
of a process, $\freenames{P}$, may be calculated recursively as
follows:

\begin{mathpar}
\freenames{\pzero} := \emptyset
  \and \\
  \freenames{x?(y).P} := \{ x \} \cup (\freenames{P} \setminus \{ y \})
  \and 
  \freenames{x!\langle P \rangle} := \{ x \} \cup \{ P \} 
  \and \\
  \freenames{P|Q} := \freenames{P} \cup \freenames{Q}
  \and \\
  \freenames{@{x}} := \{ x \}
\end{mathpar}

$\pi$
$\quotep{\pi}$

$\freenames{-} : \pi \to \mathcal{P}(\quotep{\pi})$

\begin{eqnarray*}
  \freenames{\pzero} & := & \emptyset \\
  \freenames{x?(y).P} & := & \{ x \} \cup (\freenames{P} \setminus \{ y \}) \\
  \freenames{x!\langle P \rangle} & := & \{ x \} \cup \{ P \} \\
  \freenames{P|Q} & := & \freenames{P} \cup \freenames{Q} \\
  \freenames{\dropn{x}} & := & \{ x \}
\end{eqnarray*}

The bound names of a process, $\boundnames{P}$, are those names occurring in $P$
that are not free. For example, in $x?(y).0$, the name $x$ is free, while $y$ is bound.

\begin{mathpar}
  \inferrule* [lab=monoidal-laws] {} { P|Q \equiv Q|P \and P|0 \equiv P \and P|(Q|R) \equiv (P|Q)|R }
\end{mathpar}

\begin{mathpar}
  \inferrule* [lab=alpha-equivalence] {} { (x)P \equiv (y)P\{y/x\} \and y \not\in \freenames{P} }
\end{mathpar}

\begin{definition}
Then two processes, $P,Q$, are alpha-equivalent if $P = Q\{\vec{y}/\vec{x}\}$ for
some $\vec{x} \in \boundnames{Q},\vec{y} \in \boundnames{P}$, where $Q\{\vec{y}/\vec{x}\}$
denotes the capture-avoiding substitution of $\vec{y}$ for $\vec{x}$ in $Q$.
\end{definition}

\begin{definition}
  The {\em structural congruence} \cite{SangiorgiWalker} , $\equiv$,
  between processes is the least congruence containing
  alpha-equivalence, satisfying the abelian monoid laws
  (associativity, commutativity and $\pzero$ as identity) for parallel
  composition $|$ and for summation $+$.
\end{definition}

\subsection{Name equivalence}

We take name equivalence, written $\nameeq$, to be the smallest
equivalence relation generated by the following rules.

\begin{mathpar}
\inferrule*[lab=Quote-drop]
{ }
{ \quotep{@{x}} \nameeq x }

\inferrule*[lab=Struct-equiv]
{ P \scong Q }
{ \quotep{P} \nameeq \quotep{Q} }
\end{mathpar}

The astute reader will have noticed that the mutual recursion of names
and processes imposes a mutual recursion on alpha-equivalence and
structural equivalence via name-equivalence. Fortunately, all of this
works out pleasantly and we may calculate in the natural way, free of
concern. The reader interested in the details is referred to the
appendix \ref{appendix:rho_details}.

\subsection{Substitution}

We use $\Proc$ for the set of processes, $\QProc$ for the set of
names, and $\id{\{}\vec{y} / \vec{x} \id{\}}$ to denote partial maps,
$s : \QProc \rightarrow \QProc$. A map, $s$ lifts, uniquely, to a map
on process terms, $\widehat{s} : \Proc \rightarrow \Proc$ by the
following equations.

\begin{mathpar}
  (0) \psubstp{Q}{P} := 0 \\
  (R \juxtap S) \psubstp{Q}{P}
  :=    
  (R)\psubstp{Q}{P} \juxtap (S) \psubstp{Q}{P} \\
  (x?(y).R) \psubstp{Q}{P}    
  :=    
  (x)\substp{Q}{P} (z)\concat( (R \psubstn{z}{y}) \psubstp{Q}{P} ) \\
  (\lift{x}{R}) \psubstp{Q}{P}  
  :=
  \lift{(x)\substp{Q}{P}}{ R \psubstp{Q}{P} } \\
%   (\dropn{x})  \psubstp{Q}{P}       
%   := 
%   \left\{ 
%     \begin{array}{ccc} 
%       \dropn{\quotep{Q}} & & x \nameeq \quotep{P} \\
%       \dropn{x} & & otherwise \\
%     \end{array}
%   \right. 
  (\dropn{x})  \psubstp{Q}{P}       
  := 
  \left\{ 
    \begin{array}{ccc} 
      Q & & x \nameeq \quotep{P} \\
      \dropn{x} & & otherwise \\
    \end{array}
  \right.
\end{mathpar}
 

where

\begin{eqnarray}
  (x)\id{\{} \lpquote Q \rpquote / \lpquote P \rpquote \id{\}}            = 
  \left\{ 
    \begin{array}{ccc}
      \lpquote Q \rpquote & & x \nameeq \lpquote P \rpquote \\
      x & & otherwise \\
    \end{array}
  \right. \nonumber
\end{eqnarray}

and $z$ is chosen distinct from $\quotep{P}$, $\quotep{Q}$, the free
names in $Q$, and all the names in $R$. Our $\alpha$-equivalence will
be built in the standard way from this substitution.

\begin{remark}\label{rem:no_self_referential_names}
  One consequence of these definitions is that $\forall P. \quotep{P}
  \not\in \freenames{P}$.
\end{remark}

\subsection{ Dynamic quote: an example }

Anticipating something of what's to come, consider applying the
substitution, $\widehat{\id{\{}u / z \id{\}}}$, to the following pair
of processes, $\lift{w}{y!(z)}$ and $w[ \lpquote y!(z) \rpquote ]$.

\begin{eqnarray}
	\lift{w}{y!(z)}\widehat{\id{\{}u / z \id{\}}}
		& = &
		\lift{w}{y!(u)} \nonumber\\
	w[ \lpquote y!(z) \rpquote ] \widehat{ \id{\{}u / z \id{\}} }
		& = &
		w[ \lpquote y!(z) \rpquote ] \nonumber
\end{eqnarray}

Because the body of the process between quotes is impervious to
substitution, we get radically different answers. In fact, by
examining the first process in an input context,
e.g. $x?(z).\lift{w}{y!(z)}$, we see that the process under the lift
operator may be shaped by prefixed inputs binding a name inside it. In
this sense, the lift operator will be seen as a way to dynamically
construct processes before reifying them as names.

Finally equipped with these standard features we can present the
dynamics of the calculus.

\subsubsection{Operational semantics} 

Finally, we introduce the computational dynamics. What marks these
algebras as distinct from other more traditionally studied algebraic
structures, e.g. vector spaces or polynomial rings, is the manner in
which dynamics is captured. In traditional structures, dynamics is typically
expressed through morphisms between such structures, as in linear maps
between vector spaces or morphisms between rings. In algebras
associated with the semantics of computation, the dynamics is
expressed as part of the algebraic structure itself, through a
reduction reduction relation typically denoted by $\red$. Below, we
give a recursive presentation of this relation for the calculus used
in the encoding.

$\red \subseteq \pi \times \pi$
$\red : \pi \to \mathcal{P}(\pi)$

\begin{mathpar}
  \inferrule* [lab=Comm] { \textsf{match}( x_{src}, x_{trgt} ) } { x_{trgt}?(y)P \; | \; x_{src}!\langle {Q} \rangle \red P\{\quotep{Q}/y}\} }
  \and \\
  \inferrule* [lab=Par] {{P} \red {P}'} {{{P} | {Q}} \red {{P}' | {Q}}}
  \and
  \inferrule* [lab=Equiv]{{{P} \scong {P}'} \andalso {{P}' \red {Q}'} \andalso {{Q}' \scong {Q}}}{{P} \red {Q}}
\end{mathpar}

\begin{eqnarray*}
  match_{\equiv} (\quotep{P},\quotep{Q}) & := & P \equiv Q \\
  match_{\dagger}(\quotep{P},\quotep{Q}) & := & \forall R. P|Q \red^{*} R => R \red^{*} 0 \\
  match_{K}(\quotep{P},\quotep{Q}) & := & K \mbox{ for some context } K
\end{eqnarray*}

$u?(x)P | u!\langle Q \rangle \red P\{\quotep{Q}/x\}$

%We write $\wred$ for $\red^*$, and $P\red$ if $\exists Q $ such that $ P \red Q$.
We write $P\red$ if $\exists Q $ such that $ P \red Q$ and $P\not\red$, otherwise.

\section{Replication}

As mentioned before, it is known that replication (and hence
recursion) can be implemented in a higher-order process algebra
\cite{SangiorgiWalker}. As our first example of calculation with the
machinery thus far presented we give the construction explicitly in
the {\rhoc}.

\begin{eqnarray}
	D_{x} & := & \prefix{x}{y}{(\binpar{\outputp{x}{y}}{@{y}})} \nonumber\\
	\bangp_{x}{P} & := & \binpar{{x}!\langle{\binpar{D_{x}}{P}}\rangle}{D_{x}} \nonumber
\end{eqnarray}

\begin{eqnarray}
	\bangp_{x}{P} & & \nonumber\\
	=
	& {x}!\langle{(\prefix{x}{y}{(\outputp{x}{y} | @{y})) | P}}\rangle 
	      | \prefix{x}{y}{(\outputp{x}{y} | @{y})} & \nonumber\\
	\red
	& (\outputp{x}{y} | @{y})\substn{\quotep{(\prefix{x}{y}{(@{y} | \outputp{x}{y})) | P}}}{y} & \nonumber\\
	=
	& \outputp{x}{\quotep{(\prefix{x}{y}{(\outputp{x}{y} | @{y})) | P}}}
	  | {(\prefix{x}{y}{(\outputp{x}{y} | @{y})) | P}} & \nonumber\\
	\red
	& \ldots & \nonumber\\
	\red^*
	& P | P | \ldots & \nonumber
\end{eqnarray}

Of course, this encoding, as an implementation, runs away, unfolding
$\bangp{P}$ eagerly. A lazier and more implementable replication
operator, restricted to input-guarded processes, may be obtained as follows.

\begin{eqnarray}
\bangp{\prefix{u}{v}{P}} 
	:= 
	\binpar{\lift{x}{\prefix{u}{v}{(\binpar{D(x)}{P})}}}{D(x)} \nonumber
\end{eqnarray}

\begin{remark}
  Note that the lazier definition still does not deal with summation
  or mixed summation (i.e. sums over input and output). The reader is
  invited to construct definitions of replication that deal with these
  features. 

  Further, the definitions are parameterized in a name, $x$. Can you,
  gentle reader, make a definition that eliminates this parameter and
  guarantees no accidental interaction between the replication
  machinery and the process being replicated -- i.e. no accidental
  sharing of names used by the process to get its work done and the
  name(s) used by the replication to effect copying. This latter
  revision of the definition of replication is crucial to obtaining
  the expected identity $!!P \sim !P$.
\end{remark}

\begin{remark}\label{rem:paradoxical_combinator}
  The reader familiar with the lambda calculus will have noticed the
  similarity between $D$ and the paradoxical combinator.

  [Ed. note: the existence of this seems to suggest we have to be more
  restrictive on the set of processes and names we admit if we are to
  support no-cloning.]
\end{remark}

\subsubsection{Bisimulation}

The computational dynamics gives rise to another kind of equivalence,
the equivalence of computational behavior. As previously mentioned
this is typically captured \emph{via} some form of bisimulation.

% The notion we use in this paper is weak barbed bisimulation
% \cite{milner91polyadicpi}.

The notion we use in this paper is derived from weak barbed
bisimulation \cite{milner91polyadicpi}. 

\begin{definition}
An \emph{observation relation}, $\downarrow_{\mathcal N}$, over a set
of names, $\mathcal N$, is the smallest relation satisfying the rules
below.

\infrule[Out-barb]{y \in {\mathcal N}, \; x \nameeq y}
		  {\outputp{x}{v} \downarrow_{\mathcal N} x}
\infrule[Par-barb]{\mbox{$P\downarrow_{\mathcal N} x$ or $Q\downarrow_{\mathcal N} x$}}
		  {\binpar{P}{Q} \downarrow_{\mathcal N} x}

We write $P \Downarrow_{\mathcal N} x$ if there is $Q$ such that 
$P \wred Q$ and $Q \downarrow_{\mathcal N} x$.
\end{definition}

\begin{definition}
%\label{def.bbisim}
An  ${\mathcal N}$-\emph{barbed bisimulation} over a set of names, ${\mathcal N}$, is a symmetric binary relation 
${\mathcal S}_{\mathcal N}$ between agents such that $P\rel{S}_{\mathcal N}Q$ implies:
\begin{enumerate}
\item If $P \red P'$ then $Q \wred Q'$ and $P'\rel{S}_{\mathcal N} Q'$.
\item If $P\downarrow_{\mathcal N} x$, then $Q\Downarrow_{\mathcal N} x$.
\end{enumerate}
$P$ is ${\mathcal N}$-barbed bisimilar to $Q$, written
$P \wbbisim_{\mathcal N} Q$, if $P \rel{S}_{\mathcal N} Q$ for some ${\mathcal N}$-barbed bisimulation ${\mathcal S}_{\mathcal N}$.
\end{definition}

$\mathcal{R} \subseteq \pi \times \pi$

$P \mathcal{R} Q => \forall P'. P \red P' \Rightarrow \exists Q'. Q \red Q', P' \mathcal{R} Q'$

$P \vdash x \Rightarrow Q \vdash x$

\begin{mathpar}
  \inferrule*[lab=Out-barb]{x \nameeq y}{{y}!\langle{Q}\rangle \vdash x}
  \and
  \inferrule*[lab=Par-barb]{\mbox{$P\vdash x$ or $Q\vdash x$}}{\binpar{P}{Q} \vdash x}
\end{mathpar}

\subsubsection{Contexts}

One of the principle advantages of computational calculi like the
$\pi$-calculus is a well-defined notion of context,
contextual-equivalence and a correlation between
contextual-equivalence and notions of bisimulation. The notion of
context allows the decomposition of a process into (sub-)process and
its syntactic environment, its context. Thus, a context may be
thought of as a process with a ``hole'' (written $\Box$) in it. The
application of a context $M$ to a process $P$, written $M[P]$, is
tantamount to filling the hole in $M$ with $P$. In this paper we do
not need the full weight of this theory, but do make use of the notion
of context in the proof the main theorem. 

\begin{mathpar}
  \inferrule* [lab=summation] {} {{M_{M},M_{N}} \bc \Box \;|\; x.M_{A} \;|\; M_{M}+M_{N}}
  \and
  \inferrule* [lab=agent] {} {{M_{A}} \bc (\vec{x})M_{P} \;| \; \clift{P_0,\ldots,M_{P},\ldots,P_N}}
  \and \\
  \inferrule* [lab=process] {} {{M_{P}} \bc M_{N} \;| \;P|M_{P} }
\end{mathpar} 

\begin{mathpar}
  \inferrule* [lab=sychronization] {} {M_{N} \bc \Box \;|\; x?M_{F} \;|\; x!M_{C}}
  \and
  \inferrule* [lab=abstraction] {} {{M_{F}} \bc (x)M_{P} }
  \and
  \inferrule* [lab=concretion] {} {{M_{C}} \bc \langle M_{P} \rangle }
  \and \\
  \inferrule* [lab=process] {} {{M_{P}} \bc M_{N} \;| \;P|M_{P} }
\end{mathpar}

\begin{definition}[contextual application] Given a context $M$, and
  process $P$, we define the \emph{contextual application}, $M[P] :=
  M\{P/\Box\}$. That is, the contextual application of M to P is the
  substitution of $P$ for $\Box$ in $M$.
\end{definition}

$\meaningof{-} : L \to \mathcal{P}(\pi)$

\begin{mathpar}
  \inferrule* [lab=collection] {} {\meaningof{true} = \pi, \and \meaningof{~E} = \pi \setminus \meaningof{E}, \and \meaningof{E_{1} \& E_{2}} = \meaningof{E_{1}} \cap \meaningof{E_{2}}}
\end{mathpar}

\begin{mathpar}
  \inferrule* [lab=structure] {} {\meaningof{0} = \{ P \in \pi | P \equiv 0 \}, \and \\ \meaningof{E_1 | E_2} = \{ P \in \pi | P \equiv P_{1} | P_{2}, P_{1} \in \meaningof{E_{1}}, P_{2} \in \meaningof{E_2}\} }
\end{mathpar}

\begin{mathpar}
 \inferrule* [lab=behavior] {} {\meaningof{\langle a?b \rangle E} = \{ P \in \pi | P \equiv Q | u?(y)P', \\ \and \\\\ \and \\ \;\;\; u \in \meaningof{a}, \forall z.P'\{z/y\} \in \meaningof{E\{z/b\}}\}, \and \\ \meaningof{a!E} = \{ P \in \pi | P \equiv Q | x!\langle P' \rangle, x \in \meaningof{a} P' \in \meaningof{E}\} }
\end{mathpar}

\begin{mathpar}
 \inferrule* [lab=nominal] {} {\meaningof{\quotep{E}} = \{ \quotep{P} \in \quotep{\pi} | P \in \meaningof{E} \}, \and \meaningof{\quotep{P}} = \{ \quotep{Q} \in \quotep{\pi} | P \equiv Q \} \and \\ \meaningof{@\quotep{E}} = \{ P \in \pi | P \equiv @x, x \in \meaningof{E} \}}
\end{mathpar}

\begin{eqnarray*}
  \\
  \meaningof{-} : TS \to ST
\end{eqnarray*}

\begin{eqnarray*}
  \\
  L : TS \to ST
\end{eqnarray*}

\begin{eqnarray*}
  \\
  P \models E \iff P \in \meaningof{E}
\end{eqnarray*}

\begin{eqnarray*}
  P \approx_{L} Q \iff \forall E \in L. P \models E \iff Q \models E
\end{eqnarray*}

\begin{eqnarray*}
  P \approx_{K} Q
\end{eqnarray*}

\begin{eqnarray*}
  P \approx Q
\end{eqnarray*}

$\approx_{K} = \approx = \approx_{L}$

\subsubsection{Contextual duality}

Note that contexts extend the quotation operation to a family of
operations from processes to names. Given a context, $M$, we can
define a \emph{nominal context}, $\quotep{M}$ by $\quotep{M}[P] :=
\quotep{M[P]}$. To foreshadow what is to come we observe that these
operations enjoy a duality with processes very much like the duality
between vectors and maps from vectors to scalars.

Further, because the calculus is essentially higher-order, we have a
correspondence between contexts and processes. More specifically,
given a name $x$ and a context $M$ we can construct $M^{*}_{x}$ such
that 

\begin{mathpar}
  M^{*}_{x} | \lift{x}{P} \red M[P]
\end{mathpar}

namely,

\begin{mathpar}
  M^{*}_{x} := x?(u).M[\dropn{u}]
\end{mathpar}

The dependence of $M^{*}_{x}$ on a name makes it an abstraction, 

\begin{mathpar}
  M^{*} := (x)x?(u).M[\dropn{u}]
\end{mathpar}

\subsection{Additional notation}

It will sometimes be convenient to denote the process a name
quotes. We already have the notation $x = \quotep{P}$, but it will be
convenient to introduce an alternate notation, $\procn{x}$, when we
want to emphasize the connection to the use of the name. Note that, by
virtue of name equivalence, $\quotep{\procn{x}} \nameeq x$; so, the
notation is consistent with previous definitions.

Further, because names have structure it is possible to effect
substitutions on the basis of that structure. This means we need to
upgrade our notation for substitutions, which we accomplish by
adapting comprehension notation. Thus,

\begin{mathpar}
  P\{ y / x : x \in S \}
\end{mathpar}

is interpreted to mean the process derived from P by replacing (in a
capture-avoiding manner) each occurrence of $x$ in $S$ by $y$. For example,

\begin{mathpar}
  P\{ \quotep{\procn{x}|\procn{x}} / x : x \in \freenames{P} \}
\end{mathpar}

will replace each (occurrence) of a free name $x$ in $P$ by
$\quotep{\procn{x}|\procn{x}}$.

Also, we will avail ourselves of the notation $x^{L}$ and $x^{R}$ to
denote injections of a name into disjoint copies of the name
space. There are numerous ways to accomplish this. One example can be
found in \cite{MeredithR05}. This notation overloads to vectors of
names: $\vec{x}^{\pi} := (x_{i}^{\pi} \; : \; 0 \leq i < |\vec{x}| )$ where $\pi \in \{L,R\}$.

We also use $P^{\Box} := P|\Box$.

In \cite{MeredithR05} an interpretation of the new operator is
given. It turns out that there are several possible interpretations
all enjoying the requisite algebraic properties of the operator (see
\cite{milner91polyadicpi}). We will therefore make liberal use of
$(\nu\; \vec{x})P$.

% subsection the_syntax_and_semantics_of_the_notation_system (end)   

\input{qm2pi.qmops} 

\input{qm2pi.sterngerlach} 

\input{qm2pi.metric} 

% section concurrent_process_calculi (end)

%\input{qm2pi.proofsketch}

% section proof sketch (end)

%\input{qm2pi.slviaknots} 

% section spatial logic via knots (end)

\input{qm2pi.conclusion}

% section conclusion (end)

%\input{qm2pi.dtcodes} 

% section wiring algorithm (end)

\input{qm2pi.ack} 

% section acknowledgments (end)

\newpage


\bibliographystyle{plain}   
\bibliography{../../biblios/main.bib}

\input{qm2pi.rhodetails}

\end{document}

 

% section wiring algorithm (end)

\documentclass[12pt]{llncs}
%\documentclass{jktr}

\usepackage[pdftex]{hyperref}                   
\usepackage {listings}
\usepackage {mathpartir}
\usepackage{bcprules}
%\usepackage{listings}
                       
\usepackage{graphicx} 
%\usepackage[margins=2.5cm,nohead,nofoot]{geometry}
%\usepackage{geometry}
\usepackage{amsfonts}
\usepackage{amstext}
\usepackage{latexsym}
\usepackage{amssymb}
\usepackage{color}


%\include{myPreamble}
\include{qm2pi.local} 

%\ifpdf
%\usepackage[pdftex]{graphicx}
%\else
%\usepackage{graphicx}
%\fi

 % \ifpdf
%  \usepackage{pdfsync}
%  \if


%\title{Brief Article}
%\author{David F. Snyder}
%\author{L.G. Meredith}

%\address{Dept. of Math., Texas State University--San Marcos, San Marcos, TX 78666}
       
\pagestyle{empty}


\begin{document}

\lstset{language=[Objective]Caml,frame=shadowbox}

\input{qm2pi.front}

% section front matter (end)

\input{qm2pi.intro} 
 
% section introduction (end)

% \input{qm2pi.knotations} 

% section notation (end)

\input{qm2pi.process.calculi} 

% section concurrent_process_calculi_and_spatial_logics_ (end)
    
%\input{qm2pi.knots2pi} 

%\input{qm2pi.trefoil} 

%\input{qm2pi.mainthm} 

% subsection basic_interpretation (end)

%\input{qm2pi.rho.presentation} 
\subsection{The syntax and semantics of the notation system}\label{sub:the_syntax_and_semantics_of_the_notation_system} % (fold)

We now summarize a technical presentation of the calculus that
embodies our theory of dynamics. The typical presentation of such a
calculus follows the style of giving generators and relations on
them. The grammar, below, describing term constructors, freely
generates the set of processes, $\Proc$. This set is then quotiented
by a relation known as structural congruence and it is over this set
that the notion of dynamics is expressed. This presentation is
essentially that of \cite{MeredithR05} with the addition of
polyadicity and summation. For readability we have relegated some of
the technical subtleties to an appendix.

\subsubsection{Process grammar}\label{subsub:process_grammar}

\begin{mathpar}
  \inferrule* [lab=synchronization] {} {{M} \bc \pzero \;|\; x?F \;|\; x!C }
  \and
  \inferrule* [lab=abstraction] {} {{F} \bc (x)P}
  \and
  \inferrule* [lab=concretion] {} {{C} \bc \langle Q \rangle}
  \and
  \inferrule* [lab=process] {} {{P,Q} \bc M \;| \;P|Q \;|\; @{x}}
  \and
  \inferrule* [lab=name] {} {{x} \bc \quotep{P}}
\end{mathpar} 

Note that $\vec{x}$ (resp. $\vec{P}$) denotes a vector of names
(resp. processes) of length $|\vec{x}|$ (resp. $|\vec{P}|$). We adopt
the following useful abbreviations.

\begin{mathpar}
   x?(\vec{y}).P := x.(\vec{y})P \and  x\clift{\vec{P}} := x.\clift{\vec{P}}
   \and x!(y) := \lift{x}{\dropn{y}}
   \and \Pi_{i=0}^{n-1}P_i := P_0 | \ldots | P_{n-1}
\end{mathpar}

\subsubsection{Structural congruence}

\paragraph{Free and bound names and alpha-equivalence.} At the
core of structural equivalence is alpha-equivalence which identifies
process that are the same up to a change of variable. Formally, we
recognize the distinction between free and bound names. The free names
of a process, $\freenames{P}$, may be calculated recursively as
follows:

\begin{mathpar}
\freenames{\pzero} := \emptyset
  \and \\
  \freenames{x?(y).P} := \{ x \} \cup (\freenames{P} \setminus \{ y \})
  \and 
  \freenames{x!\langle P \rangle} := \{ x \} \cup \{ P \} 
  \and \\
  \freenames{P|Q} := \freenames{P} \cup \freenames{Q}
  \and \\
  \freenames{@{x}} := \{ x \}
\end{mathpar}

$\pi$
$\quotep{\pi}$

$\freenames{-} : \pi \to \mathcal{P}(\quotep{\pi})$

\begin{eqnarray*}
  \freenames{\pzero} & := & \emptyset \\
  \freenames{x?(y).P} & := & \{ x \} \cup (\freenames{P} \setminus \{ y \}) \\
  \freenames{x!\langle P \rangle} & := & \{ x \} \cup \{ P \} \\
  \freenames{P|Q} & := & \freenames{P} \cup \freenames{Q} \\
  \freenames{\dropn{x}} & := & \{ x \}
\end{eqnarray*}

The bound names of a process, $\boundnames{P}$, are those names occurring in $P$
that are not free. For example, in $x?(y).0$, the name $x$ is free, while $y$ is bound.

\begin{mathpar}
  \inferrule* [lab=monoidal-laws] {} { P|Q \equiv Q|P \and P|0 \equiv P \and P|(Q|R) \equiv (P|Q)|R }
\end{mathpar}

\begin{mathpar}
  \inferrule* [lab=alpha-equivalence] {} { (x)P \equiv (y)P\{y/x\} \and y \not\in \freenames{P} }
\end{mathpar}

\begin{definition}
Then two processes, $P,Q$, are alpha-equivalent if $P = Q\{\vec{y}/\vec{x}\}$ for
some $\vec{x} \in \boundnames{Q},\vec{y} \in \boundnames{P}$, where $Q\{\vec{y}/\vec{x}\}$
denotes the capture-avoiding substitution of $\vec{y}$ for $\vec{x}$ in $Q$.
\end{definition}

\begin{definition}
  The {\em structural congruence} \cite{SangiorgiWalker} , $\equiv$,
  between processes is the least congruence containing
  alpha-equivalence, satisfying the abelian monoid laws
  (associativity, commutativity and $\pzero$ as identity) for parallel
  composition $|$ and for summation $+$.
\end{definition}

\subsection{Name equivalence}

We take name equivalence, written $\nameeq$, to be the smallest
equivalence relation generated by the following rules.

\begin{mathpar}
\inferrule*[lab=Quote-drop]
{ }
{ \quotep{@{x}} \nameeq x }

\inferrule*[lab=Struct-equiv]
{ P \scong Q }
{ \quotep{P} \nameeq \quotep{Q} }
\end{mathpar}

The astute reader will have noticed that the mutual recursion of names
and processes imposes a mutual recursion on alpha-equivalence and
structural equivalence via name-equivalence. Fortunately, all of this
works out pleasantly and we may calculate in the natural way, free of
concern. The reader interested in the details is referred to the
appendix \ref{appendix:rho_details}.

\subsection{Substitution}

We use $\Proc$ for the set of processes, $\QProc$ for the set of
names, and $\id{\{}\vec{y} / \vec{x} \id{\}}$ to denote partial maps,
$s : \QProc \rightarrow \QProc$. A map, $s$ lifts, uniquely, to a map
on process terms, $\widehat{s} : \Proc \rightarrow \Proc$ by the
following equations.

\begin{mathpar}
  (0) \psubstp{Q}{P} := 0 \\
  (R \juxtap S) \psubstp{Q}{P}
  :=    
  (R)\psubstp{Q}{P} \juxtap (S) \psubstp{Q}{P} \\
  (x?(y).R) \psubstp{Q}{P}    
  :=    
  (x)\substp{Q}{P} (z)\concat( (R \psubstn{z}{y}) \psubstp{Q}{P} ) \\
  (\lift{x}{R}) \psubstp{Q}{P}  
  :=
  \lift{(x)\substp{Q}{P}}{ R \psubstp{Q}{P} } \\
%   (\dropn{x})  \psubstp{Q}{P}       
%   := 
%   \left\{ 
%     \begin{array}{ccc} 
%       \dropn{\quotep{Q}} & & x \nameeq \quotep{P} \\
%       \dropn{x} & & otherwise \\
%     \end{array}
%   \right. 
  (\dropn{x})  \psubstp{Q}{P}       
  := 
  \left\{ 
    \begin{array}{ccc} 
      Q & & x \nameeq \quotep{P} \\
      \dropn{x} & & otherwise \\
    \end{array}
  \right.
\end{mathpar}
 

where

\begin{eqnarray}
  (x)\id{\{} \lpquote Q \rpquote / \lpquote P \rpquote \id{\}}            = 
  \left\{ 
    \begin{array}{ccc}
      \lpquote Q \rpquote & & x \nameeq \lpquote P \rpquote \\
      x & & otherwise \\
    \end{array}
  \right. \nonumber
\end{eqnarray}

and $z$ is chosen distinct from $\quotep{P}$, $\quotep{Q}$, the free
names in $Q$, and all the names in $R$. Our $\alpha$-equivalence will
be built in the standard way from this substitution.

\begin{remark}\label{rem:no_self_referential_names}
  One consequence of these definitions is that $\forall P. \quotep{P}
  \not\in \freenames{P}$.
\end{remark}

\subsection{ Dynamic quote: an example }

Anticipating something of what's to come, consider applying the
substitution, $\widehat{\id{\{}u / z \id{\}}}$, to the following pair
of processes, $\lift{w}{y!(z)}$ and $w[ \lpquote y!(z) \rpquote ]$.

\begin{eqnarray}
	\lift{w}{y!(z)}\widehat{\id{\{}u / z \id{\}}}
		& = &
		\lift{w}{y!(u)} \nonumber\\
	w[ \lpquote y!(z) \rpquote ] \widehat{ \id{\{}u / z \id{\}} }
		& = &
		w[ \lpquote y!(z) \rpquote ] \nonumber
\end{eqnarray}

Because the body of the process between quotes is impervious to
substitution, we get radically different answers. In fact, by
examining the first process in an input context,
e.g. $x?(z).\lift{w}{y!(z)}$, we see that the process under the lift
operator may be shaped by prefixed inputs binding a name inside it. In
this sense, the lift operator will be seen as a way to dynamically
construct processes before reifying them as names.

Finally equipped with these standard features we can present the
dynamics of the calculus.

\subsubsection{Operational semantics} 

Finally, we introduce the computational dynamics. What marks these
algebras as distinct from other more traditionally studied algebraic
structures, e.g. vector spaces or polynomial rings, is the manner in
which dynamics is captured. In traditional structures, dynamics is typically
expressed through morphisms between such structures, as in linear maps
between vector spaces or morphisms between rings. In algebras
associated with the semantics of computation, the dynamics is
expressed as part of the algebraic structure itself, through a
reduction reduction relation typically denoted by $\red$. Below, we
give a recursive presentation of this relation for the calculus used
in the encoding.

$\red \subseteq \pi \times \pi$
$\red : \pi \to \mathcal{P}(\pi)$

\begin{mathpar}
  \inferrule* [lab=Comm] { \textsf{match}( x_{src}, x_{trgt} ) } { x_{trgt}?(y)P \; | \; x_{src}!\langle {Q} \rangle \red P\{\quotep{Q}/y}\} }
  \and \\
  \inferrule* [lab=Par] {{P} \red {P}'} {{{P} | {Q}} \red {{P}' | {Q}}}
  \and
  \inferrule* [lab=Equiv]{{{P} \scong {P}'} \andalso {{P}' \red {Q}'} \andalso {{Q}' \scong {Q}}}{{P} \red {Q}}
\end{mathpar}

\begin{eqnarray*}
  match_{\equiv} (\quotep{P},\quotep{Q}) & := & P \equiv Q \\
  match_{\dagger}(\quotep{P},\quotep{Q}) & := & \forall R. P|Q \red^{*} R => R \red^{*} 0 \\
  match_{K}(\quotep{P},\quotep{Q}) & := & K \mbox{ for some context } K
\end{eqnarray*}

$u?(x)P | u!\langle Q \rangle \red P\{\quotep{Q}/x\}$

%We write $\wred$ for $\red^*$, and $P\red$ if $\exists Q $ such that $ P \red Q$.
We write $P\red$ if $\exists Q $ such that $ P \red Q$ and $P\not\red$, otherwise.

\section{Replication}

As mentioned before, it is known that replication (and hence
recursion) can be implemented in a higher-order process algebra
\cite{SangiorgiWalker}. As our first example of calculation with the
machinery thus far presented we give the construction explicitly in
the {\rhoc}.

\begin{eqnarray}
	D_{x} & := & \prefix{x}{y}{(\binpar{\outputp{x}{y}}{@{y}})} \nonumber\\
	\bangp_{x}{P} & := & \binpar{{x}!\langle{\binpar{D_{x}}{P}}\rangle}{D_{x}} \nonumber
\end{eqnarray}

\begin{eqnarray}
	\bangp_{x}{P} & & \nonumber\\
	=
	& {x}!\langle{(\prefix{x}{y}{(\outputp{x}{y} | @{y})) | P}}\rangle 
	      | \prefix{x}{y}{(\outputp{x}{y} | @{y})} & \nonumber\\
	\red
	& (\outputp{x}{y} | @{y})\substn{\quotep{(\prefix{x}{y}{(@{y} | \outputp{x}{y})) | P}}}{y} & \nonumber\\
	=
	& \outputp{x}{\quotep{(\prefix{x}{y}{(\outputp{x}{y} | @{y})) | P}}}
	  | {(\prefix{x}{y}{(\outputp{x}{y} | @{y})) | P}} & \nonumber\\
	\red
	& \ldots & \nonumber\\
	\red^*
	& P | P | \ldots & \nonumber
\end{eqnarray}

Of course, this encoding, as an implementation, runs away, unfolding
$\bangp{P}$ eagerly. A lazier and more implementable replication
operator, restricted to input-guarded processes, may be obtained as follows.

\begin{eqnarray}
\bangp{\prefix{u}{v}{P}} 
	:= 
	\binpar{\lift{x}{\prefix{u}{v}{(\binpar{D(x)}{P})}}}{D(x)} \nonumber
\end{eqnarray}

\begin{remark}
  Note that the lazier definition still does not deal with summation
  or mixed summation (i.e. sums over input and output). The reader is
  invited to construct definitions of replication that deal with these
  features. 

  Further, the definitions are parameterized in a name, $x$. Can you,
  gentle reader, make a definition that eliminates this parameter and
  guarantees no accidental interaction between the replication
  machinery and the process being replicated -- i.e. no accidental
  sharing of names used by the process to get its work done and the
  name(s) used by the replication to effect copying. This latter
  revision of the definition of replication is crucial to obtaining
  the expected identity $!!P \sim !P$.
\end{remark}

\begin{remark}\label{rem:paradoxical_combinator}
  The reader familiar with the lambda calculus will have noticed the
  similarity between $D$ and the paradoxical combinator.

  [Ed. note: the existence of this seems to suggest we have to be more
  restrictive on the set of processes and names we admit if we are to
  support no-cloning.]
\end{remark}

\subsubsection{Bisimulation}

The computational dynamics gives rise to another kind of equivalence,
the equivalence of computational behavior. As previously mentioned
this is typically captured \emph{via} some form of bisimulation.

% The notion we use in this paper is weak barbed bisimulation
% \cite{milner91polyadicpi}.

The notion we use in this paper is derived from weak barbed
bisimulation \cite{milner91polyadicpi}. 

\begin{definition}
An \emph{observation relation}, $\downarrow_{\mathcal N}$, over a set
of names, $\mathcal N$, is the smallest relation satisfying the rules
below.

\infrule[Out-barb]{y \in {\mathcal N}, \; x \nameeq y}
		  {\outputp{x}{v} \downarrow_{\mathcal N} x}
\infrule[Par-barb]{\mbox{$P\downarrow_{\mathcal N} x$ or $Q\downarrow_{\mathcal N} x$}}
		  {\binpar{P}{Q} \downarrow_{\mathcal N} x}

We write $P \Downarrow_{\mathcal N} x$ if there is $Q$ such that 
$P \wred Q$ and $Q \downarrow_{\mathcal N} x$.
\end{definition}

\begin{definition}
%\label{def.bbisim}
An  ${\mathcal N}$-\emph{barbed bisimulation} over a set of names, ${\mathcal N}$, is a symmetric binary relation 
${\mathcal S}_{\mathcal N}$ between agents such that $P\rel{S}_{\mathcal N}Q$ implies:
\begin{enumerate}
\item If $P \red P'$ then $Q \wred Q'$ and $P'\rel{S}_{\mathcal N} Q'$.
\item If $P\downarrow_{\mathcal N} x$, then $Q\Downarrow_{\mathcal N} x$.
\end{enumerate}
$P$ is ${\mathcal N}$-barbed bisimilar to $Q$, written
$P \wbbisim_{\mathcal N} Q$, if $P \rel{S}_{\mathcal N} Q$ for some ${\mathcal N}$-barbed bisimulation ${\mathcal S}_{\mathcal N}$.
\end{definition}

$\mathcal{R} \subseteq \pi \times \pi$

$P \mathcal{R} Q => \forall P'. P \red P' \Rightarrow \exists Q'. Q \red Q', P' \mathcal{R} Q'$

$P \vdash x \Rightarrow Q \vdash x$

\begin{mathpar}
  \inferrule*[lab=Out-barb]{x \nameeq y}{{y}!\langle{Q}\rangle \vdash x}
  \and
  \inferrule*[lab=Par-barb]{\mbox{$P\vdash x$ or $Q\vdash x$}}{\binpar{P}{Q} \vdash x}
\end{mathpar}

\subsubsection{Contexts}

One of the principle advantages of computational calculi like the
$\pi$-calculus is a well-defined notion of context,
contextual-equivalence and a correlation between
contextual-equivalence and notions of bisimulation. The notion of
context allows the decomposition of a process into (sub-)process and
its syntactic environment, its context. Thus, a context may be
thought of as a process with a ``hole'' (written $\Box$) in it. The
application of a context $M$ to a process $P$, written $M[P]$, is
tantamount to filling the hole in $M$ with $P$. In this paper we do
not need the full weight of this theory, but do make use of the notion
of context in the proof the main theorem. 

\begin{mathpar}
  \inferrule* [lab=summation] {} {{M_{M},M_{N}} \bc \Box \;|\; x.M_{A} \;|\; M_{M}+M_{N}}
  \and
  \inferrule* [lab=agent] {} {{M_{A}} \bc (\vec{x})M_{P} \;| \; \clift{P_0,\ldots,M_{P},\ldots,P_N}}
  \and \\
  \inferrule* [lab=process] {} {{M_{P}} \bc M_{N} \;| \;P|M_{P} }
\end{mathpar} 

\begin{mathpar}
  \inferrule* [lab=sychronization] {} {M_{N} \bc \Box \;|\; x?M_{F} \;|\; x!M_{C}}
  \and
  \inferrule* [lab=abstraction] {} {{M_{F}} \bc (x)M_{P} }
  \and
  \inferrule* [lab=concretion] {} {{M_{C}} \bc \langle M_{P} \rangle }
  \and \\
  \inferrule* [lab=process] {} {{M_{P}} \bc M_{N} \;| \;P|M_{P} }
\end{mathpar}

\begin{definition}[contextual application] Given a context $M$, and
  process $P$, we define the \emph{contextual application}, $M[P] :=
  M\{P/\Box\}$. That is, the contextual application of M to P is the
  substitution of $P$ for $\Box$ in $M$.
\end{definition}

$\meaningof{-} : L \to \mathcal{P}(\pi)$

\begin{mathpar}
  \inferrule* [lab=collection] {} {\meaningof{true} = \pi, \and \meaningof{~E} = \pi \setminus \meaningof{E}, \and \meaningof{E_{1} \& E_{2}} = \meaningof{E_{1}} \cap \meaningof{E_{2}}}
\end{mathpar}

\begin{mathpar}
  \inferrule* [lab=structure] {} {\meaningof{0} = \{ P \in \pi | P \equiv 0 \}, \and \\ \meaningof{E_1 | E_2} = \{ P \in \pi | P \equiv P_{1} | P_{2}, P_{1} \in \meaningof{E_{1}}, P_{2} \in \meaningof{E_2}\} }
\end{mathpar}

\begin{mathpar}
 \inferrule* [lab=behavior] {} {\meaningof{\langle a?b \rangle E} = \{ P \in \pi | P \equiv Q | u?(y)P', \\ \and \\\\ \and \\ \;\;\; u \in \meaningof{a}, \forall z.P'\{z/y\} \in \meaningof{E\{z/b\}}\}, \and \\ \meaningof{a!E} = \{ P \in \pi | P \equiv Q | x!\langle P' \rangle, x \in \meaningof{a} P' \in \meaningof{E}\} }
\end{mathpar}

\begin{mathpar}
 \inferrule* [lab=nominal] {} {\meaningof{\quotep{E}} = \{ \quotep{P} \in \quotep{\pi} | P \in \meaningof{E} \}, \and \meaningof{\quotep{P}} = \{ \quotep{Q} \in \quotep{\pi} | P \equiv Q \} \and \\ \meaningof{@\quotep{E}} = \{ P \in \pi | P \equiv @x, x \in \meaningof{E} \}}
\end{mathpar}

\begin{eqnarray*}
  \\
  \meaningof{-} : TS \to ST
\end{eqnarray*}

\begin{eqnarray*}
  \\
  L : TS \to ST
\end{eqnarray*}

\begin{eqnarray*}
  \\
  P \models E \iff P \in \meaningof{E}
\end{eqnarray*}

\begin{eqnarray*}
  P \approx_{L} Q \iff \forall E \in L. P \models E \iff Q \models E
\end{eqnarray*}

\begin{eqnarray*}
  P \approx_{K} Q
\end{eqnarray*}

\begin{eqnarray*}
  P \approx Q
\end{eqnarray*}

$\approx_{K} = \approx = \approx_{L}$

\subsubsection{Contextual duality}

Note that contexts extend the quotation operation to a family of
operations from processes to names. Given a context, $M$, we can
define a \emph{nominal context}, $\quotep{M}$ by $\quotep{M}[P] :=
\quotep{M[P]}$. To foreshadow what is to come we observe that these
operations enjoy a duality with processes very much like the duality
between vectors and maps from vectors to scalars.

Further, because the calculus is essentially higher-order, we have a
correspondence between contexts and processes. More specifically,
given a name $x$ and a context $M$ we can construct $M^{*}_{x}$ such
that 

\begin{mathpar}
  M^{*}_{x} | \lift{x}{P} \red M[P]
\end{mathpar}

namely,

\begin{mathpar}
  M^{*}_{x} := x?(u).M[\dropn{u}]
\end{mathpar}

The dependence of $M^{*}_{x}$ on a name makes it an abstraction, 

\begin{mathpar}
  M^{*} := (x)x?(u).M[\dropn{u}]
\end{mathpar}

\subsection{Additional notation}

It will sometimes be convenient to denote the process a name
quotes. We already have the notation $x = \quotep{P}$, but it will be
convenient to introduce an alternate notation, $\procn{x}$, when we
want to emphasize the connection to the use of the name. Note that, by
virtue of name equivalence, $\quotep{\procn{x}} \nameeq x$; so, the
notation is consistent with previous definitions.

Further, because names have structure it is possible to effect
substitutions on the basis of that structure. This means we need to
upgrade our notation for substitutions, which we accomplish by
adapting comprehension notation. Thus,

\begin{mathpar}
  P\{ y / x : x \in S \}
\end{mathpar}

is interpreted to mean the process derived from P by replacing (in a
capture-avoiding manner) each occurrence of $x$ in $S$ by $y$. For example,

\begin{mathpar}
  P\{ \quotep{\procn{x}|\procn{x}} / x : x \in \freenames{P} \}
\end{mathpar}

will replace each (occurrence) of a free name $x$ in $P$ by
$\quotep{\procn{x}|\procn{x}}$.

Also, we will avail ourselves of the notation $x^{L}$ and $x^{R}$ to
denote injections of a name into disjoint copies of the name
space. There are numerous ways to accomplish this. One example can be
found in \cite{MeredithR05}. This notation overloads to vectors of
names: $\vec{x}^{\pi} := (x_{i}^{\pi} \; : \; 0 \leq i < |\vec{x}| )$ where $\pi \in \{L,R\}$.

We also use $P^{\Box} := P|\Box$.

In \cite{MeredithR05} an interpretation of the new operator is
given. It turns out that there are several possible interpretations
all enjoying the requisite algebraic properties of the operator (see
\cite{milner91polyadicpi}). We will therefore make liberal use of
$(\nu\; \vec{x})P$.

% subsection the_syntax_and_semantics_of_the_notation_system (end)   

\input{qm2pi.qmops} 

\input{qm2pi.sterngerlach} 

\input{qm2pi.metric} 

% section concurrent_process_calculi (end)

%\input{qm2pi.proofsketch}

% section proof sketch (end)

%\input{qm2pi.slviaknots} 

% section spatial logic via knots (end)

\input{qm2pi.conclusion}

% section conclusion (end)

%\input{qm2pi.dtcodes} 

% section wiring algorithm (end)

\input{qm2pi.ack} 

% section acknowledgments (end)

\newpage


\bibliographystyle{plain}   
\bibliography{../../biblios/main.bib}

\input{qm2pi.rhodetails}

\end{document}

 

% section acknowledgments (end)

\newpage


\bibliographystyle{plain}   
\bibliography{../../biblios/main.bib}

\documentclass[12pt]{llncs}
%\documentclass{jktr}

\usepackage[pdftex]{hyperref}                   
\usepackage {listings}
\usepackage {mathpartir}
\usepackage{bcprules}
%\usepackage{listings}
                       
\usepackage{graphicx} 
%\usepackage[margins=2.5cm,nohead,nofoot]{geometry}
%\usepackage{geometry}
\usepackage{amsfonts}
\usepackage{amstext}
\usepackage{latexsym}
\usepackage{amssymb}
\usepackage{color}


%\include{myPreamble}
\include{qm2pi.local} 

%\ifpdf
%\usepackage[pdftex]{graphicx}
%\else
%\usepackage{graphicx}
%\fi

 % \ifpdf
%  \usepackage{pdfsync}
%  \if


%\title{Brief Article}
%\author{David F. Snyder}
%\author{L.G. Meredith}

%\address{Dept. of Math., Texas State University--San Marcos, San Marcos, TX 78666}
       
\pagestyle{empty}


\begin{document}

\lstset{language=[Objective]Caml,frame=shadowbox}

\input{qm2pi.front}

% section front matter (end)

\input{qm2pi.intro} 
 
% section introduction (end)

% \input{qm2pi.knotations} 

% section notation (end)

\input{qm2pi.process.calculi} 

% section concurrent_process_calculi_and_spatial_logics_ (end)
    
%\input{qm2pi.knots2pi} 

%\input{qm2pi.trefoil} 

%\input{qm2pi.mainthm} 

% subsection basic_interpretation (end)

%\input{qm2pi.rho.presentation} 
\subsection{The syntax and semantics of the notation system}\label{sub:the_syntax_and_semantics_of_the_notation_system} % (fold)

We now summarize a technical presentation of the calculus that
embodies our theory of dynamics. The typical presentation of such a
calculus follows the style of giving generators and relations on
them. The grammar, below, describing term constructors, freely
generates the set of processes, $\Proc$. This set is then quotiented
by a relation known as structural congruence and it is over this set
that the notion of dynamics is expressed. This presentation is
essentially that of \cite{MeredithR05} with the addition of
polyadicity and summation. For readability we have relegated some of
the technical subtleties to an appendix.

\subsubsection{Process grammar}\label{subsub:process_grammar}

\begin{mathpar}
  \inferrule* [lab=synchronization] {} {{M} \bc \pzero \;|\; x?F \;|\; x!C }
  \and
  \inferrule* [lab=abstraction] {} {{F} \bc (x)P}
  \and
  \inferrule* [lab=concretion] {} {{C} \bc \langle Q \rangle}
  \and
  \inferrule* [lab=process] {} {{P,Q} \bc M \;| \;P|Q \;|\; @{x}}
  \and
  \inferrule* [lab=name] {} {{x} \bc \quotep{P}}
\end{mathpar} 

Note that $\vec{x}$ (resp. $\vec{P}$) denotes a vector of names
(resp. processes) of length $|\vec{x}|$ (resp. $|\vec{P}|$). We adopt
the following useful abbreviations.

\begin{mathpar}
   x?(\vec{y}).P := x.(\vec{y})P \and  x\clift{\vec{P}} := x.\clift{\vec{P}}
   \and x!(y) := \lift{x}{\dropn{y}}
   \and \Pi_{i=0}^{n-1}P_i := P_0 | \ldots | P_{n-1}
\end{mathpar}

\subsubsection{Structural congruence}

\paragraph{Free and bound names and alpha-equivalence.} At the
core of structural equivalence is alpha-equivalence which identifies
process that are the same up to a change of variable. Formally, we
recognize the distinction between free and bound names. The free names
of a process, $\freenames{P}$, may be calculated recursively as
follows:

\begin{mathpar}
\freenames{\pzero} := \emptyset
  \and \\
  \freenames{x?(y).P} := \{ x \} \cup (\freenames{P} \setminus \{ y \})
  \and 
  \freenames{x!\langle P \rangle} := \{ x \} \cup \{ P \} 
  \and \\
  \freenames{P|Q} := \freenames{P} \cup \freenames{Q}
  \and \\
  \freenames{@{x}} := \{ x \}
\end{mathpar}

$\pi$
$\quotep{\pi}$

$\freenames{-} : \pi \to \mathcal{P}(\quotep{\pi})$

\begin{eqnarray*}
  \freenames{\pzero} & := & \emptyset \\
  \freenames{x?(y).P} & := & \{ x \} \cup (\freenames{P} \setminus \{ y \}) \\
  \freenames{x!\langle P \rangle} & := & \{ x \} \cup \{ P \} \\
  \freenames{P|Q} & := & \freenames{P} \cup \freenames{Q} \\
  \freenames{\dropn{x}} & := & \{ x \}
\end{eqnarray*}

The bound names of a process, $\boundnames{P}$, are those names occurring in $P$
that are not free. For example, in $x?(y).0$, the name $x$ is free, while $y$ is bound.

\begin{mathpar}
  \inferrule* [lab=monoidal-laws] {} { P|Q \equiv Q|P \and P|0 \equiv P \and P|(Q|R) \equiv (P|Q)|R }
\end{mathpar}

\begin{mathpar}
  \inferrule* [lab=alpha-equivalence] {} { (x)P \equiv (y)P\{y/x\} \and y \not\in \freenames{P} }
\end{mathpar}

\begin{definition}
Then two processes, $P,Q$, are alpha-equivalent if $P = Q\{\vec{y}/\vec{x}\}$ for
some $\vec{x} \in \boundnames{Q},\vec{y} \in \boundnames{P}$, where $Q\{\vec{y}/\vec{x}\}$
denotes the capture-avoiding substitution of $\vec{y}$ for $\vec{x}$ in $Q$.
\end{definition}

\begin{definition}
  The {\em structural congruence} \cite{SangiorgiWalker} , $\equiv$,
  between processes is the least congruence containing
  alpha-equivalence, satisfying the abelian monoid laws
  (associativity, commutativity and $\pzero$ as identity) for parallel
  composition $|$ and for summation $+$.
\end{definition}

\subsection{Name equivalence}

We take name equivalence, written $\nameeq$, to be the smallest
equivalence relation generated by the following rules.

\begin{mathpar}
\inferrule*[lab=Quote-drop]
{ }
{ \quotep{@{x}} \nameeq x }

\inferrule*[lab=Struct-equiv]
{ P \scong Q }
{ \quotep{P} \nameeq \quotep{Q} }
\end{mathpar}

The astute reader will have noticed that the mutual recursion of names
and processes imposes a mutual recursion on alpha-equivalence and
structural equivalence via name-equivalence. Fortunately, all of this
works out pleasantly and we may calculate in the natural way, free of
concern. The reader interested in the details is referred to the
appendix \ref{appendix:rho_details}.

\subsection{Substitution}

We use $\Proc$ for the set of processes, $\QProc$ for the set of
names, and $\id{\{}\vec{y} / \vec{x} \id{\}}$ to denote partial maps,
$s : \QProc \rightarrow \QProc$. A map, $s$ lifts, uniquely, to a map
on process terms, $\widehat{s} : \Proc \rightarrow \Proc$ by the
following equations.

\begin{mathpar}
  (0) \psubstp{Q}{P} := 0 \\
  (R \juxtap S) \psubstp{Q}{P}
  :=    
  (R)\psubstp{Q}{P} \juxtap (S) \psubstp{Q}{P} \\
  (x?(y).R) \psubstp{Q}{P}    
  :=    
  (x)\substp{Q}{P} (z)\concat( (R \psubstn{z}{y}) \psubstp{Q}{P} ) \\
  (\lift{x}{R}) \psubstp{Q}{P}  
  :=
  \lift{(x)\substp{Q}{P}}{ R \psubstp{Q}{P} } \\
%   (\dropn{x})  \psubstp{Q}{P}       
%   := 
%   \left\{ 
%     \begin{array}{ccc} 
%       \dropn{\quotep{Q}} & & x \nameeq \quotep{P} \\
%       \dropn{x} & & otherwise \\
%     \end{array}
%   \right. 
  (\dropn{x})  \psubstp{Q}{P}       
  := 
  \left\{ 
    \begin{array}{ccc} 
      Q & & x \nameeq \quotep{P} \\
      \dropn{x} & & otherwise \\
    \end{array}
  \right.
\end{mathpar}
 

where

\begin{eqnarray}
  (x)\id{\{} \lpquote Q \rpquote / \lpquote P \rpquote \id{\}}            = 
  \left\{ 
    \begin{array}{ccc}
      \lpquote Q \rpquote & & x \nameeq \lpquote P \rpquote \\
      x & & otherwise \\
    \end{array}
  \right. \nonumber
\end{eqnarray}

and $z$ is chosen distinct from $\quotep{P}$, $\quotep{Q}$, the free
names in $Q$, and all the names in $R$. Our $\alpha$-equivalence will
be built in the standard way from this substitution.

\begin{remark}\label{rem:no_self_referential_names}
  One consequence of these definitions is that $\forall P. \quotep{P}
  \not\in \freenames{P}$.
\end{remark}

\subsection{ Dynamic quote: an example }

Anticipating something of what's to come, consider applying the
substitution, $\widehat{\id{\{}u / z \id{\}}}$, to the following pair
of processes, $\lift{w}{y!(z)}$ and $w[ \lpquote y!(z) \rpquote ]$.

\begin{eqnarray}
	\lift{w}{y!(z)}\widehat{\id{\{}u / z \id{\}}}
		& = &
		\lift{w}{y!(u)} \nonumber\\
	w[ \lpquote y!(z) \rpquote ] \widehat{ \id{\{}u / z \id{\}} }
		& = &
		w[ \lpquote y!(z) \rpquote ] \nonumber
\end{eqnarray}

Because the body of the process between quotes is impervious to
substitution, we get radically different answers. In fact, by
examining the first process in an input context,
e.g. $x?(z).\lift{w}{y!(z)}$, we see that the process under the lift
operator may be shaped by prefixed inputs binding a name inside it. In
this sense, the lift operator will be seen as a way to dynamically
construct processes before reifying them as names.

Finally equipped with these standard features we can present the
dynamics of the calculus.

\subsubsection{Operational semantics} 

Finally, we introduce the computational dynamics. What marks these
algebras as distinct from other more traditionally studied algebraic
structures, e.g. vector spaces or polynomial rings, is the manner in
which dynamics is captured. In traditional structures, dynamics is typically
expressed through morphisms between such structures, as in linear maps
between vector spaces or morphisms between rings. In algebras
associated with the semantics of computation, the dynamics is
expressed as part of the algebraic structure itself, through a
reduction reduction relation typically denoted by $\red$. Below, we
give a recursive presentation of this relation for the calculus used
in the encoding.

$\red \subseteq \pi \times \pi$
$\red : \pi \to \mathcal{P}(\pi)$

\begin{mathpar}
  \inferrule* [lab=Comm] { \textsf{match}( x_{src}, x_{trgt} ) } { x_{trgt}?(y)P \; | \; x_{src}!\langle {Q} \rangle \red P\{\quotep{Q}/y}\} }
  \and \\
  \inferrule* [lab=Par] {{P} \red {P}'} {{{P} | {Q}} \red {{P}' | {Q}}}
  \and
  \inferrule* [lab=Equiv]{{{P} \scong {P}'} \andalso {{P}' \red {Q}'} \andalso {{Q}' \scong {Q}}}{{P} \red {Q}}
\end{mathpar}

\begin{eqnarray*}
  match_{\equiv} (\quotep{P},\quotep{Q}) & := & P \equiv Q \\
  match_{\dagger}(\quotep{P},\quotep{Q}) & := & \forall R. P|Q \red^{*} R => R \red^{*} 0 \\
  match_{K}(\quotep{P},\quotep{Q}) & := & K \mbox{ for some context } K
\end{eqnarray*}

$u?(x)P | u!\langle Q \rangle \red P\{\quotep{Q}/x\}$

%We write $\wred$ for $\red^*$, and $P\red$ if $\exists Q $ such that $ P \red Q$.
We write $P\red$ if $\exists Q $ such that $ P \red Q$ and $P\not\red$, otherwise.

\section{Replication}

As mentioned before, it is known that replication (and hence
recursion) can be implemented in a higher-order process algebra
\cite{SangiorgiWalker}. As our first example of calculation with the
machinery thus far presented we give the construction explicitly in
the {\rhoc}.

\begin{eqnarray}
	D_{x} & := & \prefix{x}{y}{(\binpar{\outputp{x}{y}}{@{y}})} \nonumber\\
	\bangp_{x}{P} & := & \binpar{{x}!\langle{\binpar{D_{x}}{P}}\rangle}{D_{x}} \nonumber
\end{eqnarray}

\begin{eqnarray}
	\bangp_{x}{P} & & \nonumber\\
	=
	& {x}!\langle{(\prefix{x}{y}{(\outputp{x}{y} | @{y})) | P}}\rangle 
	      | \prefix{x}{y}{(\outputp{x}{y} | @{y})} & \nonumber\\
	\red
	& (\outputp{x}{y} | @{y})\substn{\quotep{(\prefix{x}{y}{(@{y} | \outputp{x}{y})) | P}}}{y} & \nonumber\\
	=
	& \outputp{x}{\quotep{(\prefix{x}{y}{(\outputp{x}{y} | @{y})) | P}}}
	  | {(\prefix{x}{y}{(\outputp{x}{y} | @{y})) | P}} & \nonumber\\
	\red
	& \ldots & \nonumber\\
	\red^*
	& P | P | \ldots & \nonumber
\end{eqnarray}

Of course, this encoding, as an implementation, runs away, unfolding
$\bangp{P}$ eagerly. A lazier and more implementable replication
operator, restricted to input-guarded processes, may be obtained as follows.

\begin{eqnarray}
\bangp{\prefix{u}{v}{P}} 
	:= 
	\binpar{\lift{x}{\prefix{u}{v}{(\binpar{D(x)}{P})}}}{D(x)} \nonumber
\end{eqnarray}

\begin{remark}
  Note that the lazier definition still does not deal with summation
  or mixed summation (i.e. sums over input and output). The reader is
  invited to construct definitions of replication that deal with these
  features. 

  Further, the definitions are parameterized in a name, $x$. Can you,
  gentle reader, make a definition that eliminates this parameter and
  guarantees no accidental interaction between the replication
  machinery and the process being replicated -- i.e. no accidental
  sharing of names used by the process to get its work done and the
  name(s) used by the replication to effect copying. This latter
  revision of the definition of replication is crucial to obtaining
  the expected identity $!!P \sim !P$.
\end{remark}

\begin{remark}\label{rem:paradoxical_combinator}
  The reader familiar with the lambda calculus will have noticed the
  similarity between $D$ and the paradoxical combinator.

  [Ed. note: the existence of this seems to suggest we have to be more
  restrictive on the set of processes and names we admit if we are to
  support no-cloning.]
\end{remark}

\subsubsection{Bisimulation}

The computational dynamics gives rise to another kind of equivalence,
the equivalence of computational behavior. As previously mentioned
this is typically captured \emph{via} some form of bisimulation.

% The notion we use in this paper is weak barbed bisimulation
% \cite{milner91polyadicpi}.

The notion we use in this paper is derived from weak barbed
bisimulation \cite{milner91polyadicpi}. 

\begin{definition}
An \emph{observation relation}, $\downarrow_{\mathcal N}$, over a set
of names, $\mathcal N$, is the smallest relation satisfying the rules
below.

\infrule[Out-barb]{y \in {\mathcal N}, \; x \nameeq y}
		  {\outputp{x}{v} \downarrow_{\mathcal N} x}
\infrule[Par-barb]{\mbox{$P\downarrow_{\mathcal N} x$ or $Q\downarrow_{\mathcal N} x$}}
		  {\binpar{P}{Q} \downarrow_{\mathcal N} x}

We write $P \Downarrow_{\mathcal N} x$ if there is $Q$ such that 
$P \wred Q$ and $Q \downarrow_{\mathcal N} x$.
\end{definition}

\begin{definition}
%\label{def.bbisim}
An  ${\mathcal N}$-\emph{barbed bisimulation} over a set of names, ${\mathcal N}$, is a symmetric binary relation 
${\mathcal S}_{\mathcal N}$ between agents such that $P\rel{S}_{\mathcal N}Q$ implies:
\begin{enumerate}
\item If $P \red P'$ then $Q \wred Q'$ and $P'\rel{S}_{\mathcal N} Q'$.
\item If $P\downarrow_{\mathcal N} x$, then $Q\Downarrow_{\mathcal N} x$.
\end{enumerate}
$P$ is ${\mathcal N}$-barbed bisimilar to $Q$, written
$P \wbbisim_{\mathcal N} Q$, if $P \rel{S}_{\mathcal N} Q$ for some ${\mathcal N}$-barbed bisimulation ${\mathcal S}_{\mathcal N}$.
\end{definition}

$\mathcal{R} \subseteq \pi \times \pi$

$P \mathcal{R} Q => \forall P'. P \red P' \Rightarrow \exists Q'. Q \red Q', P' \mathcal{R} Q'$

$P \vdash x \Rightarrow Q \vdash x$

\begin{mathpar}
  \inferrule*[lab=Out-barb]{x \nameeq y}{{y}!\langle{Q}\rangle \vdash x}
  \and
  \inferrule*[lab=Par-barb]{\mbox{$P\vdash x$ or $Q\vdash x$}}{\binpar{P}{Q} \vdash x}
\end{mathpar}

\subsubsection{Contexts}

One of the principle advantages of computational calculi like the
$\pi$-calculus is a well-defined notion of context,
contextual-equivalence and a correlation between
contextual-equivalence and notions of bisimulation. The notion of
context allows the decomposition of a process into (sub-)process and
its syntactic environment, its context. Thus, a context may be
thought of as a process with a ``hole'' (written $\Box$) in it. The
application of a context $M$ to a process $P$, written $M[P]$, is
tantamount to filling the hole in $M$ with $P$. In this paper we do
not need the full weight of this theory, but do make use of the notion
of context in the proof the main theorem. 

\begin{mathpar}
  \inferrule* [lab=summation] {} {{M_{M},M_{N}} \bc \Box \;|\; x.M_{A} \;|\; M_{M}+M_{N}}
  \and
  \inferrule* [lab=agent] {} {{M_{A}} \bc (\vec{x})M_{P} \;| \; \clift{P_0,\ldots,M_{P},\ldots,P_N}}
  \and \\
  \inferrule* [lab=process] {} {{M_{P}} \bc M_{N} \;| \;P|M_{P} }
\end{mathpar} 

\begin{mathpar}
  \inferrule* [lab=sychronization] {} {M_{N} \bc \Box \;|\; x?M_{F} \;|\; x!M_{C}}
  \and
  \inferrule* [lab=abstraction] {} {{M_{F}} \bc (x)M_{P} }
  \and
  \inferrule* [lab=concretion] {} {{M_{C}} \bc \langle M_{P} \rangle }
  \and \\
  \inferrule* [lab=process] {} {{M_{P}} \bc M_{N} \;| \;P|M_{P} }
\end{mathpar}

\begin{definition}[contextual application] Given a context $M$, and
  process $P$, we define the \emph{contextual application}, $M[P] :=
  M\{P/\Box\}$. That is, the contextual application of M to P is the
  substitution of $P$ for $\Box$ in $M$.
\end{definition}

$\meaningof{-} : L \to \mathcal{P}(\pi)$

\begin{mathpar}
  \inferrule* [lab=collection] {} {\meaningof{true} = \pi, \and \meaningof{~E} = \pi \setminus \meaningof{E}, \and \meaningof{E_{1} \& E_{2}} = \meaningof{E_{1}} \cap \meaningof{E_{2}}}
\end{mathpar}

\begin{mathpar}
  \inferrule* [lab=structure] {} {\meaningof{0} = \{ P \in \pi | P \equiv 0 \}, \and \\ \meaningof{E_1 | E_2} = \{ P \in \pi | P \equiv P_{1} | P_{2}, P_{1} \in \meaningof{E_{1}}, P_{2} \in \meaningof{E_2}\} }
\end{mathpar}

\begin{mathpar}
 \inferrule* [lab=behavior] {} {\meaningof{\langle a?b \rangle E} = \{ P \in \pi | P \equiv Q | u?(y)P', \\ \and \\\\ \and \\ \;\;\; u \in \meaningof{a}, \forall z.P'\{z/y\} \in \meaningof{E\{z/b\}}\}, \and \\ \meaningof{a!E} = \{ P \in \pi | P \equiv Q | x!\langle P' \rangle, x \in \meaningof{a} P' \in \meaningof{E}\} }
\end{mathpar}

\begin{mathpar}
 \inferrule* [lab=nominal] {} {\meaningof{\quotep{E}} = \{ \quotep{P} \in \quotep{\pi} | P \in \meaningof{E} \}, \and \meaningof{\quotep{P}} = \{ \quotep{Q} \in \quotep{\pi} | P \equiv Q \} \and \\ \meaningof{@\quotep{E}} = \{ P \in \pi | P \equiv @x, x \in \meaningof{E} \}}
\end{mathpar}

\begin{eqnarray*}
  \\
  \meaningof{-} : TS \to ST
\end{eqnarray*}

\begin{eqnarray*}
  \\
  L : TS \to ST
\end{eqnarray*}

\begin{eqnarray*}
  \\
  P \models E \iff P \in \meaningof{E}
\end{eqnarray*}

\begin{eqnarray*}
  P \approx_{L} Q \iff \forall E \in L. P \models E \iff Q \models E
\end{eqnarray*}

\begin{eqnarray*}
  P \approx_{K} Q
\end{eqnarray*}

\begin{eqnarray*}
  P \approx Q
\end{eqnarray*}

$\approx_{K} = \approx = \approx_{L}$

\subsubsection{Contextual duality}

Note that contexts extend the quotation operation to a family of
operations from processes to names. Given a context, $M$, we can
define a \emph{nominal context}, $\quotep{M}$ by $\quotep{M}[P] :=
\quotep{M[P]}$. To foreshadow what is to come we observe that these
operations enjoy a duality with processes very much like the duality
between vectors and maps from vectors to scalars.

Further, because the calculus is essentially higher-order, we have a
correspondence between contexts and processes. More specifically,
given a name $x$ and a context $M$ we can construct $M^{*}_{x}$ such
that 

\begin{mathpar}
  M^{*}_{x} | \lift{x}{P} \red M[P]
\end{mathpar}

namely,

\begin{mathpar}
  M^{*}_{x} := x?(u).M[\dropn{u}]
\end{mathpar}

The dependence of $M^{*}_{x}$ on a name makes it an abstraction, 

\begin{mathpar}
  M^{*} := (x)x?(u).M[\dropn{u}]
\end{mathpar}

\subsection{Additional notation}

It will sometimes be convenient to denote the process a name
quotes. We already have the notation $x = \quotep{P}$, but it will be
convenient to introduce an alternate notation, $\procn{x}$, when we
want to emphasize the connection to the use of the name. Note that, by
virtue of name equivalence, $\quotep{\procn{x}} \nameeq x$; so, the
notation is consistent with previous definitions.

Further, because names have structure it is possible to effect
substitutions on the basis of that structure. This means we need to
upgrade our notation for substitutions, which we accomplish by
adapting comprehension notation. Thus,

\begin{mathpar}
  P\{ y / x : x \in S \}
\end{mathpar}

is interpreted to mean the process derived from P by replacing (in a
capture-avoiding manner) each occurrence of $x$ in $S$ by $y$. For example,

\begin{mathpar}
  P\{ \quotep{\procn{x}|\procn{x}} / x : x \in \freenames{P} \}
\end{mathpar}

will replace each (occurrence) of a free name $x$ in $P$ by
$\quotep{\procn{x}|\procn{x}}$.

Also, we will avail ourselves of the notation $x^{L}$ and $x^{R}$ to
denote injections of a name into disjoint copies of the name
space. There are numerous ways to accomplish this. One example can be
found in \cite{MeredithR05}. This notation overloads to vectors of
names: $\vec{x}^{\pi} := (x_{i}^{\pi} \; : \; 0 \leq i < |\vec{x}| )$ where $\pi \in \{L,R\}$.

We also use $P^{\Box} := P|\Box$.

In \cite{MeredithR05} an interpretation of the new operator is
given. It turns out that there are several possible interpretations
all enjoying the requisite algebraic properties of the operator (see
\cite{milner91polyadicpi}). We will therefore make liberal use of
$(\nu\; \vec{x})P$.

% subsection the_syntax_and_semantics_of_the_notation_system (end)   

\input{qm2pi.qmops} 

\input{qm2pi.sterngerlach} 

\input{qm2pi.metric} 

% section concurrent_process_calculi (end)

%\input{qm2pi.proofsketch}

% section proof sketch (end)

%\input{qm2pi.slviaknots} 

% section spatial logic via knots (end)

\input{qm2pi.conclusion}

% section conclusion (end)

%\input{qm2pi.dtcodes} 

% section wiring algorithm (end)

\input{qm2pi.ack} 

% section acknowledgments (end)

\newpage


\bibliographystyle{plain}   
\bibliography{../../biblios/main.bib}

\input{qm2pi.rhodetails}

\end{document}



\end{document}

 

%\documentclass[12pt]{llncs}
%\documentclass{jktr}

\usepackage[pdftex]{hyperref}                   
\usepackage {listings}
\usepackage {mathpartir}
\usepackage{bcprules}
%\usepackage{listings}
                       
\usepackage{graphicx} 
%\usepackage[margins=2.5cm,nohead,nofoot]{geometry}
%\usepackage{geometry}
\usepackage{amsfonts}
\usepackage{amstext}
\usepackage{latexsym}
\usepackage{amssymb}
\usepackage{color}


%\include{myPreamble}
\documentclass[12pt]{llncs}
%\documentclass{jktr}

\usepackage[pdftex]{hyperref}                   
\usepackage {listings}
\usepackage {mathpartir}
\usepackage{bcprules}
%\usepackage{listings}
                       
\usepackage{graphicx} 
%\usepackage[margins=2.5cm,nohead,nofoot]{geometry}
%\usepackage{geometry}
\usepackage{amsfonts}
\usepackage{amstext}
\usepackage{latexsym}
\usepackage{amssymb}
\usepackage{color}


%\include{myPreamble}
\include{qm2pi.local} 

%\ifpdf
%\usepackage[pdftex]{graphicx}
%\else
%\usepackage{graphicx}
%\fi

 % \ifpdf
%  \usepackage{pdfsync}
%  \if


%\title{Brief Article}
%\author{David F. Snyder}
%\author{L.G. Meredith}

%\address{Dept. of Math., Texas State University--San Marcos, San Marcos, TX 78666}
       
\pagestyle{empty}


\begin{document}

\lstset{language=[Objective]Caml,frame=shadowbox}

\input{qm2pi.front}

% section front matter (end)

\input{qm2pi.intro} 
 
% section introduction (end)

% \input{qm2pi.knotations} 

% section notation (end)

\input{qm2pi.process.calculi} 

% section concurrent_process_calculi_and_spatial_logics_ (end)
    
%\input{qm2pi.knots2pi} 

%\input{qm2pi.trefoil} 

%\input{qm2pi.mainthm} 

% subsection basic_interpretation (end)

%\input{qm2pi.rho.presentation} 
\subsection{The syntax and semantics of the notation system}\label{sub:the_syntax_and_semantics_of_the_notation_system} % (fold)

We now summarize a technical presentation of the calculus that
embodies our theory of dynamics. The typical presentation of such a
calculus follows the style of giving generators and relations on
them. The grammar, below, describing term constructors, freely
generates the set of processes, $\Proc$. This set is then quotiented
by a relation known as structural congruence and it is over this set
that the notion of dynamics is expressed. This presentation is
essentially that of \cite{MeredithR05} with the addition of
polyadicity and summation. For readability we have relegated some of
the technical subtleties to an appendix.

\subsubsection{Process grammar}\label{subsub:process_grammar}

\begin{mathpar}
  \inferrule* [lab=synchronization] {} {{M} \bc \pzero \;|\; x?F \;|\; x!C }
  \and
  \inferrule* [lab=abstraction] {} {{F} \bc (x)P}
  \and
  \inferrule* [lab=concretion] {} {{C} \bc \langle Q \rangle}
  \and
  \inferrule* [lab=process] {} {{P,Q} \bc M \;| \;P|Q \;|\; @{x}}
  \and
  \inferrule* [lab=name] {} {{x} \bc \quotep{P}}
\end{mathpar} 

Note that $\vec{x}$ (resp. $\vec{P}$) denotes a vector of names
(resp. processes) of length $|\vec{x}|$ (resp. $|\vec{P}|$). We adopt
the following useful abbreviations.

\begin{mathpar}
   x?(\vec{y}).P := x.(\vec{y})P \and  x\clift{\vec{P}} := x.\clift{\vec{P}}
   \and x!(y) := \lift{x}{\dropn{y}}
   \and \Pi_{i=0}^{n-1}P_i := P_0 | \ldots | P_{n-1}
\end{mathpar}

\subsubsection{Structural congruence}

\paragraph{Free and bound names and alpha-equivalence.} At the
core of structural equivalence is alpha-equivalence which identifies
process that are the same up to a change of variable. Formally, we
recognize the distinction between free and bound names. The free names
of a process, $\freenames{P}$, may be calculated recursively as
follows:

\begin{mathpar}
\freenames{\pzero} := \emptyset
  \and \\
  \freenames{x?(y).P} := \{ x \} \cup (\freenames{P} \setminus \{ y \})
  \and 
  \freenames{x!\langle P \rangle} := \{ x \} \cup \{ P \} 
  \and \\
  \freenames{P|Q} := \freenames{P} \cup \freenames{Q}
  \and \\
  \freenames{@{x}} := \{ x \}
\end{mathpar}

$\pi$
$\quotep{\pi}$

$\freenames{-} : \pi \to \mathcal{P}(\quotep{\pi})$

\begin{eqnarray*}
  \freenames{\pzero} & := & \emptyset \\
  \freenames{x?(y).P} & := & \{ x \} \cup (\freenames{P} \setminus \{ y \}) \\
  \freenames{x!\langle P \rangle} & := & \{ x \} \cup \{ P \} \\
  \freenames{P|Q} & := & \freenames{P} \cup \freenames{Q} \\
  \freenames{\dropn{x}} & := & \{ x \}
\end{eqnarray*}

The bound names of a process, $\boundnames{P}$, are those names occurring in $P$
that are not free. For example, in $x?(y).0$, the name $x$ is free, while $y$ is bound.

\begin{mathpar}
  \inferrule* [lab=monoidal-laws] {} { P|Q \equiv Q|P \and P|0 \equiv P \and P|(Q|R) \equiv (P|Q)|R }
\end{mathpar}

\begin{mathpar}
  \inferrule* [lab=alpha-equivalence] {} { (x)P \equiv (y)P\{y/x\} \and y \not\in \freenames{P} }
\end{mathpar}

\begin{definition}
Then two processes, $P,Q$, are alpha-equivalent if $P = Q\{\vec{y}/\vec{x}\}$ for
some $\vec{x} \in \boundnames{Q},\vec{y} \in \boundnames{P}$, where $Q\{\vec{y}/\vec{x}\}$
denotes the capture-avoiding substitution of $\vec{y}$ for $\vec{x}$ in $Q$.
\end{definition}

\begin{definition}
  The {\em structural congruence} \cite{SangiorgiWalker} , $\equiv$,
  between processes is the least congruence containing
  alpha-equivalence, satisfying the abelian monoid laws
  (associativity, commutativity and $\pzero$ as identity) for parallel
  composition $|$ and for summation $+$.
\end{definition}

\subsection{Name equivalence}

We take name equivalence, written $\nameeq$, to be the smallest
equivalence relation generated by the following rules.

\begin{mathpar}
\inferrule*[lab=Quote-drop]
{ }
{ \quotep{@{x}} \nameeq x }

\inferrule*[lab=Struct-equiv]
{ P \scong Q }
{ \quotep{P} \nameeq \quotep{Q} }
\end{mathpar}

The astute reader will have noticed that the mutual recursion of names
and processes imposes a mutual recursion on alpha-equivalence and
structural equivalence via name-equivalence. Fortunately, all of this
works out pleasantly and we may calculate in the natural way, free of
concern. The reader interested in the details is referred to the
appendix \ref{appendix:rho_details}.

\subsection{Substitution}

We use $\Proc$ for the set of processes, $\QProc$ for the set of
names, and $\id{\{}\vec{y} / \vec{x} \id{\}}$ to denote partial maps,
$s : \QProc \rightarrow \QProc$. A map, $s$ lifts, uniquely, to a map
on process terms, $\widehat{s} : \Proc \rightarrow \Proc$ by the
following equations.

\begin{mathpar}
  (0) \psubstp{Q}{P} := 0 \\
  (R \juxtap S) \psubstp{Q}{P}
  :=    
  (R)\psubstp{Q}{P} \juxtap (S) \psubstp{Q}{P} \\
  (x?(y).R) \psubstp{Q}{P}    
  :=    
  (x)\substp{Q}{P} (z)\concat( (R \psubstn{z}{y}) \psubstp{Q}{P} ) \\
  (\lift{x}{R}) \psubstp{Q}{P}  
  :=
  \lift{(x)\substp{Q}{P}}{ R \psubstp{Q}{P} } \\
%   (\dropn{x})  \psubstp{Q}{P}       
%   := 
%   \left\{ 
%     \begin{array}{ccc} 
%       \dropn{\quotep{Q}} & & x \nameeq \quotep{P} \\
%       \dropn{x} & & otherwise \\
%     \end{array}
%   \right. 
  (\dropn{x})  \psubstp{Q}{P}       
  := 
  \left\{ 
    \begin{array}{ccc} 
      Q & & x \nameeq \quotep{P} \\
      \dropn{x} & & otherwise \\
    \end{array}
  \right.
\end{mathpar}
 

where

\begin{eqnarray}
  (x)\id{\{} \lpquote Q \rpquote / \lpquote P \rpquote \id{\}}            = 
  \left\{ 
    \begin{array}{ccc}
      \lpquote Q \rpquote & & x \nameeq \lpquote P \rpquote \\
      x & & otherwise \\
    \end{array}
  \right. \nonumber
\end{eqnarray}

and $z$ is chosen distinct from $\quotep{P}$, $\quotep{Q}$, the free
names in $Q$, and all the names in $R$. Our $\alpha$-equivalence will
be built in the standard way from this substitution.

\begin{remark}\label{rem:no_self_referential_names}
  One consequence of these definitions is that $\forall P. \quotep{P}
  \not\in \freenames{P}$.
\end{remark}

\subsection{ Dynamic quote: an example }

Anticipating something of what's to come, consider applying the
substitution, $\widehat{\id{\{}u / z \id{\}}}$, to the following pair
of processes, $\lift{w}{y!(z)}$ and $w[ \lpquote y!(z) \rpquote ]$.

\begin{eqnarray}
	\lift{w}{y!(z)}\widehat{\id{\{}u / z \id{\}}}
		& = &
		\lift{w}{y!(u)} \nonumber\\
	w[ \lpquote y!(z) \rpquote ] \widehat{ \id{\{}u / z \id{\}} }
		& = &
		w[ \lpquote y!(z) \rpquote ] \nonumber
\end{eqnarray}

Because the body of the process between quotes is impervious to
substitution, we get radically different answers. In fact, by
examining the first process in an input context,
e.g. $x?(z).\lift{w}{y!(z)}$, we see that the process under the lift
operator may be shaped by prefixed inputs binding a name inside it. In
this sense, the lift operator will be seen as a way to dynamically
construct processes before reifying them as names.

Finally equipped with these standard features we can present the
dynamics of the calculus.

\subsubsection{Operational semantics} 

Finally, we introduce the computational dynamics. What marks these
algebras as distinct from other more traditionally studied algebraic
structures, e.g. vector spaces or polynomial rings, is the manner in
which dynamics is captured. In traditional structures, dynamics is typically
expressed through morphisms between such structures, as in linear maps
between vector spaces or morphisms between rings. In algebras
associated with the semantics of computation, the dynamics is
expressed as part of the algebraic structure itself, through a
reduction reduction relation typically denoted by $\red$. Below, we
give a recursive presentation of this relation for the calculus used
in the encoding.

$\red \subseteq \pi \times \pi$
$\red : \pi \to \mathcal{P}(\pi)$

\begin{mathpar}
  \inferrule* [lab=Comm] { \textsf{match}( x_{src}, x_{trgt} ) } { x_{trgt}?(y)P \; | \; x_{src}!\langle {Q} \rangle \red P\{\quotep{Q}/y}\} }
  \and \\
  \inferrule* [lab=Par] {{P} \red {P}'} {{{P} | {Q}} \red {{P}' | {Q}}}
  \and
  \inferrule* [lab=Equiv]{{{P} \scong {P}'} \andalso {{P}' \red {Q}'} \andalso {{Q}' \scong {Q}}}{{P} \red {Q}}
\end{mathpar}

\begin{eqnarray*}
  match_{\equiv} (\quotep{P},\quotep{Q}) & := & P \equiv Q \\
  match_{\dagger}(\quotep{P},\quotep{Q}) & := & \forall R. P|Q \red^{*} R => R \red^{*} 0 \\
  match_{K}(\quotep{P},\quotep{Q}) & := & K \mbox{ for some context } K
\end{eqnarray*}

$u?(x)P | u!\langle Q \rangle \red P\{\quotep{Q}/x\}$

%We write $\wred$ for $\red^*$, and $P\red$ if $\exists Q $ such that $ P \red Q$.
We write $P\red$ if $\exists Q $ such that $ P \red Q$ and $P\not\red$, otherwise.

\section{Replication}

As mentioned before, it is known that replication (and hence
recursion) can be implemented in a higher-order process algebra
\cite{SangiorgiWalker}. As our first example of calculation with the
machinery thus far presented we give the construction explicitly in
the {\rhoc}.

\begin{eqnarray}
	D_{x} & := & \prefix{x}{y}{(\binpar{\outputp{x}{y}}{@{y}})} \nonumber\\
	\bangp_{x}{P} & := & \binpar{{x}!\langle{\binpar{D_{x}}{P}}\rangle}{D_{x}} \nonumber
\end{eqnarray}

\begin{eqnarray}
	\bangp_{x}{P} & & \nonumber\\
	=
	& {x}!\langle{(\prefix{x}{y}{(\outputp{x}{y} | @{y})) | P}}\rangle 
	      | \prefix{x}{y}{(\outputp{x}{y} | @{y})} & \nonumber\\
	\red
	& (\outputp{x}{y} | @{y})\substn{\quotep{(\prefix{x}{y}{(@{y} | \outputp{x}{y})) | P}}}{y} & \nonumber\\
	=
	& \outputp{x}{\quotep{(\prefix{x}{y}{(\outputp{x}{y} | @{y})) | P}}}
	  | {(\prefix{x}{y}{(\outputp{x}{y} | @{y})) | P}} & \nonumber\\
	\red
	& \ldots & \nonumber\\
	\red^*
	& P | P | \ldots & \nonumber
\end{eqnarray}

Of course, this encoding, as an implementation, runs away, unfolding
$\bangp{P}$ eagerly. A lazier and more implementable replication
operator, restricted to input-guarded processes, may be obtained as follows.

\begin{eqnarray}
\bangp{\prefix{u}{v}{P}} 
	:= 
	\binpar{\lift{x}{\prefix{u}{v}{(\binpar{D(x)}{P})}}}{D(x)} \nonumber
\end{eqnarray}

\begin{remark}
  Note that the lazier definition still does not deal with summation
  or mixed summation (i.e. sums over input and output). The reader is
  invited to construct definitions of replication that deal with these
  features. 

  Further, the definitions are parameterized in a name, $x$. Can you,
  gentle reader, make a definition that eliminates this parameter and
  guarantees no accidental interaction between the replication
  machinery and the process being replicated -- i.e. no accidental
  sharing of names used by the process to get its work done and the
  name(s) used by the replication to effect copying. This latter
  revision of the definition of replication is crucial to obtaining
  the expected identity $!!P \sim !P$.
\end{remark}

\begin{remark}\label{rem:paradoxical_combinator}
  The reader familiar with the lambda calculus will have noticed the
  similarity between $D$ and the paradoxical combinator.

  [Ed. note: the existence of this seems to suggest we have to be more
  restrictive on the set of processes and names we admit if we are to
  support no-cloning.]
\end{remark}

\subsubsection{Bisimulation}

The computational dynamics gives rise to another kind of equivalence,
the equivalence of computational behavior. As previously mentioned
this is typically captured \emph{via} some form of bisimulation.

% The notion we use in this paper is weak barbed bisimulation
% \cite{milner91polyadicpi}.

The notion we use in this paper is derived from weak barbed
bisimulation \cite{milner91polyadicpi}. 

\begin{definition}
An \emph{observation relation}, $\downarrow_{\mathcal N}$, over a set
of names, $\mathcal N$, is the smallest relation satisfying the rules
below.

\infrule[Out-barb]{y \in {\mathcal N}, \; x \nameeq y}
		  {\outputp{x}{v} \downarrow_{\mathcal N} x}
\infrule[Par-barb]{\mbox{$P\downarrow_{\mathcal N} x$ or $Q\downarrow_{\mathcal N} x$}}
		  {\binpar{P}{Q} \downarrow_{\mathcal N} x}

We write $P \Downarrow_{\mathcal N} x$ if there is $Q$ such that 
$P \wred Q$ and $Q \downarrow_{\mathcal N} x$.
\end{definition}

\begin{definition}
%\label{def.bbisim}
An  ${\mathcal N}$-\emph{barbed bisimulation} over a set of names, ${\mathcal N}$, is a symmetric binary relation 
${\mathcal S}_{\mathcal N}$ between agents such that $P\rel{S}_{\mathcal N}Q$ implies:
\begin{enumerate}
\item If $P \red P'$ then $Q \wred Q'$ and $P'\rel{S}_{\mathcal N} Q'$.
\item If $P\downarrow_{\mathcal N} x$, then $Q\Downarrow_{\mathcal N} x$.
\end{enumerate}
$P$ is ${\mathcal N}$-barbed bisimilar to $Q$, written
$P \wbbisim_{\mathcal N} Q$, if $P \rel{S}_{\mathcal N} Q$ for some ${\mathcal N}$-barbed bisimulation ${\mathcal S}_{\mathcal N}$.
\end{definition}

$\mathcal{R} \subseteq \pi \times \pi$

$P \mathcal{R} Q => \forall P'. P \red P' \Rightarrow \exists Q'. Q \red Q', P' \mathcal{R} Q'$

$P \vdash x \Rightarrow Q \vdash x$

\begin{mathpar}
  \inferrule*[lab=Out-barb]{x \nameeq y}{{y}!\langle{Q}\rangle \vdash x}
  \and
  \inferrule*[lab=Par-barb]{\mbox{$P\vdash x$ or $Q\vdash x$}}{\binpar{P}{Q} \vdash x}
\end{mathpar}

\subsubsection{Contexts}

One of the principle advantages of computational calculi like the
$\pi$-calculus is a well-defined notion of context,
contextual-equivalence and a correlation between
contextual-equivalence and notions of bisimulation. The notion of
context allows the decomposition of a process into (sub-)process and
its syntactic environment, its context. Thus, a context may be
thought of as a process with a ``hole'' (written $\Box$) in it. The
application of a context $M$ to a process $P$, written $M[P]$, is
tantamount to filling the hole in $M$ with $P$. In this paper we do
not need the full weight of this theory, but do make use of the notion
of context in the proof the main theorem. 

\begin{mathpar}
  \inferrule* [lab=summation] {} {{M_{M},M_{N}} \bc \Box \;|\; x.M_{A} \;|\; M_{M}+M_{N}}
  \and
  \inferrule* [lab=agent] {} {{M_{A}} \bc (\vec{x})M_{P} \;| \; \clift{P_0,\ldots,M_{P},\ldots,P_N}}
  \and \\
  \inferrule* [lab=process] {} {{M_{P}} \bc M_{N} \;| \;P|M_{P} }
\end{mathpar} 

\begin{mathpar}
  \inferrule* [lab=sychronization] {} {M_{N} \bc \Box \;|\; x?M_{F} \;|\; x!M_{C}}
  \and
  \inferrule* [lab=abstraction] {} {{M_{F}} \bc (x)M_{P} }
  \and
  \inferrule* [lab=concretion] {} {{M_{C}} \bc \langle M_{P} \rangle }
  \and \\
  \inferrule* [lab=process] {} {{M_{P}} \bc M_{N} \;| \;P|M_{P} }
\end{mathpar}

\begin{definition}[contextual application] Given a context $M$, and
  process $P$, we define the \emph{contextual application}, $M[P] :=
  M\{P/\Box\}$. That is, the contextual application of M to P is the
  substitution of $P$ for $\Box$ in $M$.
\end{definition}

$\meaningof{-} : L \to \mathcal{P}(\pi)$

\begin{mathpar}
  \inferrule* [lab=collection] {} {\meaningof{true} = \pi, \and \meaningof{~E} = \pi \setminus \meaningof{E}, \and \meaningof{E_{1} \& E_{2}} = \meaningof{E_{1}} \cap \meaningof{E_{2}}}
\end{mathpar}

\begin{mathpar}
  \inferrule* [lab=structure] {} {\meaningof{0} = \{ P \in \pi | P \equiv 0 \}, \and \\ \meaningof{E_1 | E_2} = \{ P \in \pi | P \equiv P_{1} | P_{2}, P_{1} \in \meaningof{E_{1}}, P_{2} \in \meaningof{E_2}\} }
\end{mathpar}

\begin{mathpar}
 \inferrule* [lab=behavior] {} {\meaningof{\langle a?b \rangle E} = \{ P \in \pi | P \equiv Q | u?(y)P', \\ \and \\\\ \and \\ \;\;\; u \in \meaningof{a}, \forall z.P'\{z/y\} \in \meaningof{E\{z/b\}}\}, \and \\ \meaningof{a!E} = \{ P \in \pi | P \equiv Q | x!\langle P' \rangle, x \in \meaningof{a} P' \in \meaningof{E}\} }
\end{mathpar}

\begin{mathpar}
 \inferrule* [lab=nominal] {} {\meaningof{\quotep{E}} = \{ \quotep{P} \in \quotep{\pi} | P \in \meaningof{E} \}, \and \meaningof{\quotep{P}} = \{ \quotep{Q} \in \quotep{\pi} | P \equiv Q \} \and \\ \meaningof{@\quotep{E}} = \{ P \in \pi | P \equiv @x, x \in \meaningof{E} \}}
\end{mathpar}

\begin{eqnarray*}
  \\
  \meaningof{-} : TS \to ST
\end{eqnarray*}

\begin{eqnarray*}
  \\
  L : TS \to ST
\end{eqnarray*}

\begin{eqnarray*}
  \\
  P \models E \iff P \in \meaningof{E}
\end{eqnarray*}

\begin{eqnarray*}
  P \approx_{L} Q \iff \forall E \in L. P \models E \iff Q \models E
\end{eqnarray*}

\begin{eqnarray*}
  P \approx_{K} Q
\end{eqnarray*}

\begin{eqnarray*}
  P \approx Q
\end{eqnarray*}

$\approx_{K} = \approx = \approx_{L}$

\subsubsection{Contextual duality}

Note that contexts extend the quotation operation to a family of
operations from processes to names. Given a context, $M$, we can
define a \emph{nominal context}, $\quotep{M}$ by $\quotep{M}[P] :=
\quotep{M[P]}$. To foreshadow what is to come we observe that these
operations enjoy a duality with processes very much like the duality
between vectors and maps from vectors to scalars.

Further, because the calculus is essentially higher-order, we have a
correspondence between contexts and processes. More specifically,
given a name $x$ and a context $M$ we can construct $M^{*}_{x}$ such
that 

\begin{mathpar}
  M^{*}_{x} | \lift{x}{P} \red M[P]
\end{mathpar}

namely,

\begin{mathpar}
  M^{*}_{x} := x?(u).M[\dropn{u}]
\end{mathpar}

The dependence of $M^{*}_{x}$ on a name makes it an abstraction, 

\begin{mathpar}
  M^{*} := (x)x?(u).M[\dropn{u}]
\end{mathpar}

\subsection{Additional notation}

It will sometimes be convenient to denote the process a name
quotes. We already have the notation $x = \quotep{P}$, but it will be
convenient to introduce an alternate notation, $\procn{x}$, when we
want to emphasize the connection to the use of the name. Note that, by
virtue of name equivalence, $\quotep{\procn{x}} \nameeq x$; so, the
notation is consistent with previous definitions.

Further, because names have structure it is possible to effect
substitutions on the basis of that structure. This means we need to
upgrade our notation for substitutions, which we accomplish by
adapting comprehension notation. Thus,

\begin{mathpar}
  P\{ y / x : x \in S \}
\end{mathpar}

is interpreted to mean the process derived from P by replacing (in a
capture-avoiding manner) each occurrence of $x$ in $S$ by $y$. For example,

\begin{mathpar}
  P\{ \quotep{\procn{x}|\procn{x}} / x : x \in \freenames{P} \}
\end{mathpar}

will replace each (occurrence) of a free name $x$ in $P$ by
$\quotep{\procn{x}|\procn{x}}$.

Also, we will avail ourselves of the notation $x^{L}$ and $x^{R}$ to
denote injections of a name into disjoint copies of the name
space. There are numerous ways to accomplish this. One example can be
found in \cite{MeredithR05}. This notation overloads to vectors of
names: $\vec{x}^{\pi} := (x_{i}^{\pi} \; : \; 0 \leq i < |\vec{x}| )$ where $\pi \in \{L,R\}$.

We also use $P^{\Box} := P|\Box$.

In \cite{MeredithR05} an interpretation of the new operator is
given. It turns out that there are several possible interpretations
all enjoying the requisite algebraic properties of the operator (see
\cite{milner91polyadicpi}). We will therefore make liberal use of
$(\nu\; \vec{x})P$.

% subsection the_syntax_and_semantics_of_the_notation_system (end)   

\input{qm2pi.qmops} 

\input{qm2pi.sterngerlach} 

\input{qm2pi.metric} 

% section concurrent_process_calculi (end)

%\input{qm2pi.proofsketch}

% section proof sketch (end)

%\input{qm2pi.slviaknots} 

% section spatial logic via knots (end)

\input{qm2pi.conclusion}

% section conclusion (end)

%\input{qm2pi.dtcodes} 

% section wiring algorithm (end)

\input{qm2pi.ack} 

% section acknowledgments (end)

\newpage


\bibliographystyle{plain}   
\bibliography{../../biblios/main.bib}

\input{qm2pi.rhodetails}

\end{document}

 

%\ifpdf
%\usepackage[pdftex]{graphicx}
%\else
%\usepackage{graphicx}
%\fi

 % \ifpdf
%  \usepackage{pdfsync}
%  \if


%\title{Brief Article}
%\author{David F. Snyder}
%\author{L.G. Meredith}

%\address{Dept. of Math., Texas State University--San Marcos, San Marcos, TX 78666}
       
\pagestyle{empty}


\begin{document}

\lstset{language=[Objective]Caml,frame=shadowbox}

\documentclass[12pt]{llncs}
%\documentclass{jktr}

\usepackage[pdftex]{hyperref}                   
\usepackage {listings}
\usepackage {mathpartir}
\usepackage{bcprules}
%\usepackage{listings}
                       
\usepackage{graphicx} 
%\usepackage[margins=2.5cm,nohead,nofoot]{geometry}
%\usepackage{geometry}
\usepackage{amsfonts}
\usepackage{amstext}
\usepackage{latexsym}
\usepackage{amssymb}
\usepackage{color}


%\include{myPreamble}
\include{qm2pi.local} 

%\ifpdf
%\usepackage[pdftex]{graphicx}
%\else
%\usepackage{graphicx}
%\fi

 % \ifpdf
%  \usepackage{pdfsync}
%  \if


%\title{Brief Article}
%\author{David F. Snyder}
%\author{L.G. Meredith}

%\address{Dept. of Math., Texas State University--San Marcos, San Marcos, TX 78666}
       
\pagestyle{empty}


\begin{document}

\lstset{language=[Objective]Caml,frame=shadowbox}

\input{qm2pi.front}

% section front matter (end)

\input{qm2pi.intro} 
 
% section introduction (end)

% \input{qm2pi.knotations} 

% section notation (end)

\input{qm2pi.process.calculi} 

% section concurrent_process_calculi_and_spatial_logics_ (end)
    
%\input{qm2pi.knots2pi} 

%\input{qm2pi.trefoil} 

%\input{qm2pi.mainthm} 

% subsection basic_interpretation (end)

%\input{qm2pi.rho.presentation} 
\subsection{The syntax and semantics of the notation system}\label{sub:the_syntax_and_semantics_of_the_notation_system} % (fold)

We now summarize a technical presentation of the calculus that
embodies our theory of dynamics. The typical presentation of such a
calculus follows the style of giving generators and relations on
them. The grammar, below, describing term constructors, freely
generates the set of processes, $\Proc$. This set is then quotiented
by a relation known as structural congruence and it is over this set
that the notion of dynamics is expressed. This presentation is
essentially that of \cite{MeredithR05} with the addition of
polyadicity and summation. For readability we have relegated some of
the technical subtleties to an appendix.

\subsubsection{Process grammar}\label{subsub:process_grammar}

\begin{mathpar}
  \inferrule* [lab=synchronization] {} {{M} \bc \pzero \;|\; x?F \;|\; x!C }
  \and
  \inferrule* [lab=abstraction] {} {{F} \bc (x)P}
  \and
  \inferrule* [lab=concretion] {} {{C} \bc \langle Q \rangle}
  \and
  \inferrule* [lab=process] {} {{P,Q} \bc M \;| \;P|Q \;|\; @{x}}
  \and
  \inferrule* [lab=name] {} {{x} \bc \quotep{P}}
\end{mathpar} 

Note that $\vec{x}$ (resp. $\vec{P}$) denotes a vector of names
(resp. processes) of length $|\vec{x}|$ (resp. $|\vec{P}|$). We adopt
the following useful abbreviations.

\begin{mathpar}
   x?(\vec{y}).P := x.(\vec{y})P \and  x\clift{\vec{P}} := x.\clift{\vec{P}}
   \and x!(y) := \lift{x}{\dropn{y}}
   \and \Pi_{i=0}^{n-1}P_i := P_0 | \ldots | P_{n-1}
\end{mathpar}

\subsubsection{Structural congruence}

\paragraph{Free and bound names and alpha-equivalence.} At the
core of structural equivalence is alpha-equivalence which identifies
process that are the same up to a change of variable. Formally, we
recognize the distinction between free and bound names. The free names
of a process, $\freenames{P}$, may be calculated recursively as
follows:

\begin{mathpar}
\freenames{\pzero} := \emptyset
  \and \\
  \freenames{x?(y).P} := \{ x \} \cup (\freenames{P} \setminus \{ y \})
  \and 
  \freenames{x!\langle P \rangle} := \{ x \} \cup \{ P \} 
  \and \\
  \freenames{P|Q} := \freenames{P} \cup \freenames{Q}
  \and \\
  \freenames{@{x}} := \{ x \}
\end{mathpar}

$\pi$
$\quotep{\pi}$

$\freenames{-} : \pi \to \mathcal{P}(\quotep{\pi})$

\begin{eqnarray*}
  \freenames{\pzero} & := & \emptyset \\
  \freenames{x?(y).P} & := & \{ x \} \cup (\freenames{P} \setminus \{ y \}) \\
  \freenames{x!\langle P \rangle} & := & \{ x \} \cup \{ P \} \\
  \freenames{P|Q} & := & \freenames{P} \cup \freenames{Q} \\
  \freenames{\dropn{x}} & := & \{ x \}
\end{eqnarray*}

The bound names of a process, $\boundnames{P}$, are those names occurring in $P$
that are not free. For example, in $x?(y).0$, the name $x$ is free, while $y$ is bound.

\begin{mathpar}
  \inferrule* [lab=monoidal-laws] {} { P|Q \equiv Q|P \and P|0 \equiv P \and P|(Q|R) \equiv (P|Q)|R }
\end{mathpar}

\begin{mathpar}
  \inferrule* [lab=alpha-equivalence] {} { (x)P \equiv (y)P\{y/x\} \and y \not\in \freenames{P} }
\end{mathpar}

\begin{definition}
Then two processes, $P,Q$, are alpha-equivalent if $P = Q\{\vec{y}/\vec{x}\}$ for
some $\vec{x} \in \boundnames{Q},\vec{y} \in \boundnames{P}$, where $Q\{\vec{y}/\vec{x}\}$
denotes the capture-avoiding substitution of $\vec{y}$ for $\vec{x}$ in $Q$.
\end{definition}

\begin{definition}
  The {\em structural congruence} \cite{SangiorgiWalker} , $\equiv$,
  between processes is the least congruence containing
  alpha-equivalence, satisfying the abelian monoid laws
  (associativity, commutativity and $\pzero$ as identity) for parallel
  composition $|$ and for summation $+$.
\end{definition}

\subsection{Name equivalence}

We take name equivalence, written $\nameeq$, to be the smallest
equivalence relation generated by the following rules.

\begin{mathpar}
\inferrule*[lab=Quote-drop]
{ }
{ \quotep{@{x}} \nameeq x }

\inferrule*[lab=Struct-equiv]
{ P \scong Q }
{ \quotep{P} \nameeq \quotep{Q} }
\end{mathpar}

The astute reader will have noticed that the mutual recursion of names
and processes imposes a mutual recursion on alpha-equivalence and
structural equivalence via name-equivalence. Fortunately, all of this
works out pleasantly and we may calculate in the natural way, free of
concern. The reader interested in the details is referred to the
appendix \ref{appendix:rho_details}.

\subsection{Substitution}

We use $\Proc$ for the set of processes, $\QProc$ for the set of
names, and $\id{\{}\vec{y} / \vec{x} \id{\}}$ to denote partial maps,
$s : \QProc \rightarrow \QProc$. A map, $s$ lifts, uniquely, to a map
on process terms, $\widehat{s} : \Proc \rightarrow \Proc$ by the
following equations.

\begin{mathpar}
  (0) \psubstp{Q}{P} := 0 \\
  (R \juxtap S) \psubstp{Q}{P}
  :=    
  (R)\psubstp{Q}{P} \juxtap (S) \psubstp{Q}{P} \\
  (x?(y).R) \psubstp{Q}{P}    
  :=    
  (x)\substp{Q}{P} (z)\concat( (R \psubstn{z}{y}) \psubstp{Q}{P} ) \\
  (\lift{x}{R}) \psubstp{Q}{P}  
  :=
  \lift{(x)\substp{Q}{P}}{ R \psubstp{Q}{P} } \\
%   (\dropn{x})  \psubstp{Q}{P}       
%   := 
%   \left\{ 
%     \begin{array}{ccc} 
%       \dropn{\quotep{Q}} & & x \nameeq \quotep{P} \\
%       \dropn{x} & & otherwise \\
%     \end{array}
%   \right. 
  (\dropn{x})  \psubstp{Q}{P}       
  := 
  \left\{ 
    \begin{array}{ccc} 
      Q & & x \nameeq \quotep{P} \\
      \dropn{x} & & otherwise \\
    \end{array}
  \right.
\end{mathpar}
 

where

\begin{eqnarray}
  (x)\id{\{} \lpquote Q \rpquote / \lpquote P \rpquote \id{\}}            = 
  \left\{ 
    \begin{array}{ccc}
      \lpquote Q \rpquote & & x \nameeq \lpquote P \rpquote \\
      x & & otherwise \\
    \end{array}
  \right. \nonumber
\end{eqnarray}

and $z$ is chosen distinct from $\quotep{P}$, $\quotep{Q}$, the free
names in $Q$, and all the names in $R$. Our $\alpha$-equivalence will
be built in the standard way from this substitution.

\begin{remark}\label{rem:no_self_referential_names}
  One consequence of these definitions is that $\forall P. \quotep{P}
  \not\in \freenames{P}$.
\end{remark}

\subsection{ Dynamic quote: an example }

Anticipating something of what's to come, consider applying the
substitution, $\widehat{\id{\{}u / z \id{\}}}$, to the following pair
of processes, $\lift{w}{y!(z)}$ and $w[ \lpquote y!(z) \rpquote ]$.

\begin{eqnarray}
	\lift{w}{y!(z)}\widehat{\id{\{}u / z \id{\}}}
		& = &
		\lift{w}{y!(u)} \nonumber\\
	w[ \lpquote y!(z) \rpquote ] \widehat{ \id{\{}u / z \id{\}} }
		& = &
		w[ \lpquote y!(z) \rpquote ] \nonumber
\end{eqnarray}

Because the body of the process between quotes is impervious to
substitution, we get radically different answers. In fact, by
examining the first process in an input context,
e.g. $x?(z).\lift{w}{y!(z)}$, we see that the process under the lift
operator may be shaped by prefixed inputs binding a name inside it. In
this sense, the lift operator will be seen as a way to dynamically
construct processes before reifying them as names.

Finally equipped with these standard features we can present the
dynamics of the calculus.

\subsubsection{Operational semantics} 

Finally, we introduce the computational dynamics. What marks these
algebras as distinct from other more traditionally studied algebraic
structures, e.g. vector spaces or polynomial rings, is the manner in
which dynamics is captured. In traditional structures, dynamics is typically
expressed through morphisms between such structures, as in linear maps
between vector spaces or morphisms between rings. In algebras
associated with the semantics of computation, the dynamics is
expressed as part of the algebraic structure itself, through a
reduction reduction relation typically denoted by $\red$. Below, we
give a recursive presentation of this relation for the calculus used
in the encoding.

$\red \subseteq \pi \times \pi$
$\red : \pi \to \mathcal{P}(\pi)$

\begin{mathpar}
  \inferrule* [lab=Comm] { \textsf{match}( x_{src}, x_{trgt} ) } { x_{trgt}?(y)P \; | \; x_{src}!\langle {Q} \rangle \red P\{\quotep{Q}/y}\} }
  \and \\
  \inferrule* [lab=Par] {{P} \red {P}'} {{{P} | {Q}} \red {{P}' | {Q}}}
  \and
  \inferrule* [lab=Equiv]{{{P} \scong {P}'} \andalso {{P}' \red {Q}'} \andalso {{Q}' \scong {Q}}}{{P} \red {Q}}
\end{mathpar}

\begin{eqnarray*}
  match_{\equiv} (\quotep{P},\quotep{Q}) & := & P \equiv Q \\
  match_{\dagger}(\quotep{P},\quotep{Q}) & := & \forall R. P|Q \red^{*} R => R \red^{*} 0 \\
  match_{K}(\quotep{P},\quotep{Q}) & := & K \mbox{ for some context } K
\end{eqnarray*}

$u?(x)P | u!\langle Q \rangle \red P\{\quotep{Q}/x\}$

%We write $\wred$ for $\red^*$, and $P\red$ if $\exists Q $ such that $ P \red Q$.
We write $P\red$ if $\exists Q $ such that $ P \red Q$ and $P\not\red$, otherwise.

\section{Replication}

As mentioned before, it is known that replication (and hence
recursion) can be implemented in a higher-order process algebra
\cite{SangiorgiWalker}. As our first example of calculation with the
machinery thus far presented we give the construction explicitly in
the {\rhoc}.

\begin{eqnarray}
	D_{x} & := & \prefix{x}{y}{(\binpar{\outputp{x}{y}}{@{y}})} \nonumber\\
	\bangp_{x}{P} & := & \binpar{{x}!\langle{\binpar{D_{x}}{P}}\rangle}{D_{x}} \nonumber
\end{eqnarray}

\begin{eqnarray}
	\bangp_{x}{P} & & \nonumber\\
	=
	& {x}!\langle{(\prefix{x}{y}{(\outputp{x}{y} | @{y})) | P}}\rangle 
	      | \prefix{x}{y}{(\outputp{x}{y} | @{y})} & \nonumber\\
	\red
	& (\outputp{x}{y} | @{y})\substn{\quotep{(\prefix{x}{y}{(@{y} | \outputp{x}{y})) | P}}}{y} & \nonumber\\
	=
	& \outputp{x}{\quotep{(\prefix{x}{y}{(\outputp{x}{y} | @{y})) | P}}}
	  | {(\prefix{x}{y}{(\outputp{x}{y} | @{y})) | P}} & \nonumber\\
	\red
	& \ldots & \nonumber\\
	\red^*
	& P | P | \ldots & \nonumber
\end{eqnarray}

Of course, this encoding, as an implementation, runs away, unfolding
$\bangp{P}$ eagerly. A lazier and more implementable replication
operator, restricted to input-guarded processes, may be obtained as follows.

\begin{eqnarray}
\bangp{\prefix{u}{v}{P}} 
	:= 
	\binpar{\lift{x}{\prefix{u}{v}{(\binpar{D(x)}{P})}}}{D(x)} \nonumber
\end{eqnarray}

\begin{remark}
  Note that the lazier definition still does not deal with summation
  or mixed summation (i.e. sums over input and output). The reader is
  invited to construct definitions of replication that deal with these
  features. 

  Further, the definitions are parameterized in a name, $x$. Can you,
  gentle reader, make a definition that eliminates this parameter and
  guarantees no accidental interaction between the replication
  machinery and the process being replicated -- i.e. no accidental
  sharing of names used by the process to get its work done and the
  name(s) used by the replication to effect copying. This latter
  revision of the definition of replication is crucial to obtaining
  the expected identity $!!P \sim !P$.
\end{remark}

\begin{remark}\label{rem:paradoxical_combinator}
  The reader familiar with the lambda calculus will have noticed the
  similarity between $D$ and the paradoxical combinator.

  [Ed. note: the existence of this seems to suggest we have to be more
  restrictive on the set of processes and names we admit if we are to
  support no-cloning.]
\end{remark}

\subsubsection{Bisimulation}

The computational dynamics gives rise to another kind of equivalence,
the equivalence of computational behavior. As previously mentioned
this is typically captured \emph{via} some form of bisimulation.

% The notion we use in this paper is weak barbed bisimulation
% \cite{milner91polyadicpi}.

The notion we use in this paper is derived from weak barbed
bisimulation \cite{milner91polyadicpi}. 

\begin{definition}
An \emph{observation relation}, $\downarrow_{\mathcal N}$, over a set
of names, $\mathcal N$, is the smallest relation satisfying the rules
below.

\infrule[Out-barb]{y \in {\mathcal N}, \; x \nameeq y}
		  {\outputp{x}{v} \downarrow_{\mathcal N} x}
\infrule[Par-barb]{\mbox{$P\downarrow_{\mathcal N} x$ or $Q\downarrow_{\mathcal N} x$}}
		  {\binpar{P}{Q} \downarrow_{\mathcal N} x}

We write $P \Downarrow_{\mathcal N} x$ if there is $Q$ such that 
$P \wred Q$ and $Q \downarrow_{\mathcal N} x$.
\end{definition}

\begin{definition}
%\label{def.bbisim}
An  ${\mathcal N}$-\emph{barbed bisimulation} over a set of names, ${\mathcal N}$, is a symmetric binary relation 
${\mathcal S}_{\mathcal N}$ between agents such that $P\rel{S}_{\mathcal N}Q$ implies:
\begin{enumerate}
\item If $P \red P'$ then $Q \wred Q'$ and $P'\rel{S}_{\mathcal N} Q'$.
\item If $P\downarrow_{\mathcal N} x$, then $Q\Downarrow_{\mathcal N} x$.
\end{enumerate}
$P$ is ${\mathcal N}$-barbed bisimilar to $Q$, written
$P \wbbisim_{\mathcal N} Q$, if $P \rel{S}_{\mathcal N} Q$ for some ${\mathcal N}$-barbed bisimulation ${\mathcal S}_{\mathcal N}$.
\end{definition}

$\mathcal{R} \subseteq \pi \times \pi$

$P \mathcal{R} Q => \forall P'. P \red P' \Rightarrow \exists Q'. Q \red Q', P' \mathcal{R} Q'$

$P \vdash x \Rightarrow Q \vdash x$

\begin{mathpar}
  \inferrule*[lab=Out-barb]{x \nameeq y}{{y}!\langle{Q}\rangle \vdash x}
  \and
  \inferrule*[lab=Par-barb]{\mbox{$P\vdash x$ or $Q\vdash x$}}{\binpar{P}{Q} \vdash x}
\end{mathpar}

\subsubsection{Contexts}

One of the principle advantages of computational calculi like the
$\pi$-calculus is a well-defined notion of context,
contextual-equivalence and a correlation between
contextual-equivalence and notions of bisimulation. The notion of
context allows the decomposition of a process into (sub-)process and
its syntactic environment, its context. Thus, a context may be
thought of as a process with a ``hole'' (written $\Box$) in it. The
application of a context $M$ to a process $P$, written $M[P]$, is
tantamount to filling the hole in $M$ with $P$. In this paper we do
not need the full weight of this theory, but do make use of the notion
of context in the proof the main theorem. 

\begin{mathpar}
  \inferrule* [lab=summation] {} {{M_{M},M_{N}} \bc \Box \;|\; x.M_{A} \;|\; M_{M}+M_{N}}
  \and
  \inferrule* [lab=agent] {} {{M_{A}} \bc (\vec{x})M_{P} \;| \; \clift{P_0,\ldots,M_{P},\ldots,P_N}}
  \and \\
  \inferrule* [lab=process] {} {{M_{P}} \bc M_{N} \;| \;P|M_{P} }
\end{mathpar} 

\begin{mathpar}
  \inferrule* [lab=sychronization] {} {M_{N} \bc \Box \;|\; x?M_{F} \;|\; x!M_{C}}
  \and
  \inferrule* [lab=abstraction] {} {{M_{F}} \bc (x)M_{P} }
  \and
  \inferrule* [lab=concretion] {} {{M_{C}} \bc \langle M_{P} \rangle }
  \and \\
  \inferrule* [lab=process] {} {{M_{P}} \bc M_{N} \;| \;P|M_{P} }
\end{mathpar}

\begin{definition}[contextual application] Given a context $M$, and
  process $P$, we define the \emph{contextual application}, $M[P] :=
  M\{P/\Box\}$. That is, the contextual application of M to P is the
  substitution of $P$ for $\Box$ in $M$.
\end{definition}

$\meaningof{-} : L \to \mathcal{P}(\pi)$

\begin{mathpar}
  \inferrule* [lab=collection] {} {\meaningof{true} = \pi, \and \meaningof{~E} = \pi \setminus \meaningof{E}, \and \meaningof{E_{1} \& E_{2}} = \meaningof{E_{1}} \cap \meaningof{E_{2}}}
\end{mathpar}

\begin{mathpar}
  \inferrule* [lab=structure] {} {\meaningof{0} = \{ P \in \pi | P \equiv 0 \}, \and \\ \meaningof{E_1 | E_2} = \{ P \in \pi | P \equiv P_{1} | P_{2}, P_{1} \in \meaningof{E_{1}}, P_{2} \in \meaningof{E_2}\} }
\end{mathpar}

\begin{mathpar}
 \inferrule* [lab=behavior] {} {\meaningof{\langle a?b \rangle E} = \{ P \in \pi | P \equiv Q | u?(y)P', \\ \and \\\\ \and \\ \;\;\; u \in \meaningof{a}, \forall z.P'\{z/y\} \in \meaningof{E\{z/b\}}\}, \and \\ \meaningof{a!E} = \{ P \in \pi | P \equiv Q | x!\langle P' \rangle, x \in \meaningof{a} P' \in \meaningof{E}\} }
\end{mathpar}

\begin{mathpar}
 \inferrule* [lab=nominal] {} {\meaningof{\quotep{E}} = \{ \quotep{P} \in \quotep{\pi} | P \in \meaningof{E} \}, \and \meaningof{\quotep{P}} = \{ \quotep{Q} \in \quotep{\pi} | P \equiv Q \} \and \\ \meaningof{@\quotep{E}} = \{ P \in \pi | P \equiv @x, x \in \meaningof{E} \}}
\end{mathpar}

\begin{eqnarray*}
  \\
  \meaningof{-} : TS \to ST
\end{eqnarray*}

\begin{eqnarray*}
  \\
  L : TS \to ST
\end{eqnarray*}

\begin{eqnarray*}
  \\
  P \models E \iff P \in \meaningof{E}
\end{eqnarray*}

\begin{eqnarray*}
  P \approx_{L} Q \iff \forall E \in L. P \models E \iff Q \models E
\end{eqnarray*}

\begin{eqnarray*}
  P \approx_{K} Q
\end{eqnarray*}

\begin{eqnarray*}
  P \approx Q
\end{eqnarray*}

$\approx_{K} = \approx = \approx_{L}$

\subsubsection{Contextual duality}

Note that contexts extend the quotation operation to a family of
operations from processes to names. Given a context, $M$, we can
define a \emph{nominal context}, $\quotep{M}$ by $\quotep{M}[P] :=
\quotep{M[P]}$. To foreshadow what is to come we observe that these
operations enjoy a duality with processes very much like the duality
between vectors and maps from vectors to scalars.

Further, because the calculus is essentially higher-order, we have a
correspondence between contexts and processes. More specifically,
given a name $x$ and a context $M$ we can construct $M^{*}_{x}$ such
that 

\begin{mathpar}
  M^{*}_{x} | \lift{x}{P} \red M[P]
\end{mathpar}

namely,

\begin{mathpar}
  M^{*}_{x} := x?(u).M[\dropn{u}]
\end{mathpar}

The dependence of $M^{*}_{x}$ on a name makes it an abstraction, 

\begin{mathpar}
  M^{*} := (x)x?(u).M[\dropn{u}]
\end{mathpar}

\subsection{Additional notation}

It will sometimes be convenient to denote the process a name
quotes. We already have the notation $x = \quotep{P}$, but it will be
convenient to introduce an alternate notation, $\procn{x}$, when we
want to emphasize the connection to the use of the name. Note that, by
virtue of name equivalence, $\quotep{\procn{x}} \nameeq x$; so, the
notation is consistent with previous definitions.

Further, because names have structure it is possible to effect
substitutions on the basis of that structure. This means we need to
upgrade our notation for substitutions, which we accomplish by
adapting comprehension notation. Thus,

\begin{mathpar}
  P\{ y / x : x \in S \}
\end{mathpar}

is interpreted to mean the process derived from P by replacing (in a
capture-avoiding manner) each occurrence of $x$ in $S$ by $y$. For example,

\begin{mathpar}
  P\{ \quotep{\procn{x}|\procn{x}} / x : x \in \freenames{P} \}
\end{mathpar}

will replace each (occurrence) of a free name $x$ in $P$ by
$\quotep{\procn{x}|\procn{x}}$.

Also, we will avail ourselves of the notation $x^{L}$ and $x^{R}$ to
denote injections of a name into disjoint copies of the name
space. There are numerous ways to accomplish this. One example can be
found in \cite{MeredithR05}. This notation overloads to vectors of
names: $\vec{x}^{\pi} := (x_{i}^{\pi} \; : \; 0 \leq i < |\vec{x}| )$ where $\pi \in \{L,R\}$.

We also use $P^{\Box} := P|\Box$.

In \cite{MeredithR05} an interpretation of the new operator is
given. It turns out that there are several possible interpretations
all enjoying the requisite algebraic properties of the operator (see
\cite{milner91polyadicpi}). We will therefore make liberal use of
$(\nu\; \vec{x})P$.

% subsection the_syntax_and_semantics_of_the_notation_system (end)   

\input{qm2pi.qmops} 

\input{qm2pi.sterngerlach} 

\input{qm2pi.metric} 

% section concurrent_process_calculi (end)

%\input{qm2pi.proofsketch}

% section proof sketch (end)

%\input{qm2pi.slviaknots} 

% section spatial logic via knots (end)

\input{qm2pi.conclusion}

% section conclusion (end)

%\input{qm2pi.dtcodes} 

% section wiring algorithm (end)

\input{qm2pi.ack} 

% section acknowledgments (end)

\newpage


\bibliographystyle{plain}   
\bibliography{../../biblios/main.bib}

\input{qm2pi.rhodetails}

\end{document}



% section front matter (end)

\section{Introduction}\label{sec:introduction} % (fold)
In this draft of the material i am going to have to dispense with the
usual writing conventions adopted in papers on these topics. i'm going
to have adopt whatever tone i need at the time i'm writing up the
calculations. Sometimes this may be very conversational; others it may
be the barest mathematical grunts; others still it may be that i have
lifted text from one of my other papers because the exposition of some
point was better said there. i hope that my readers are not unduly put
out by this decision. i'm not doing this to flout convention or be
rebellious. i find these calculations very technically challenging. To
keep everything going technically, something has to give; i have to
let go of some cognitive burden. So, the academic writing style --
with all of its trade-offs in terms of facilitating technical
communication -- is what i'm letting go of. Perhaps subsequent drafts
can be tightened and polished, but for now, i'm going to speak as if
we were sitting together in a coffee shop with a laptop, wifi and a
pad of paper and a pencil.

So, here's what i have to say. We -- you and i, comfortably ensconced
in our coffee shop and well-equipped with our tools -- can realize and
carry out the calculations of quantum mechanics over a very different
formal theory of dynamics, a formal theory of dynamics that
corresponds to a theory of concurrent computation with
\emph{reflection}. It has the advantage that the underlying theory is
already `quantized', but supports analogues all of the continuuous
operations. Strikingly, this underlying theory has recently been
connected with a notion of metric that we can show, by calculating
together, coincides with the metric induced by the inner product.

There are a lot of reasons why you might be interested in seeing
calculations of this form. Here's why i'm interested. For the past
several centuries there has been no competitor to the ``Newtonian''
account of dynamics. As a result the predominant share of accounts of
dynamical systems and situations have had to be formulated in terms of
the Newtonian machinery. i view this as an intellectually dangerous
position to occupy. Everything, despite it's intrinsic shape, turns
into a nail to be hit with this hammer. Recently, however, the theory
of computation has matured to the point where we have candidates for
theories of dynamics that offer very different perspective on
reasoning about dynamical systems and situations. Testing these
candidates against very successful accounts of dynamical situations,
like quantum mechanics, is going to give us some sense of how mature
they are and some measure of the quality of these accounts of
dynamics.

\subsection{Summary of contributions and outline of paper}

So, we're going to develop an interpretation of the operations of
quantum mechanics normally interpreted by Hilbert spaces and
operators. We're going to do this over a theory of computation. Note
that this is very different than the usual quantum computation program
which develops notions of computation over quantum mechanics. Rather,
we are developing a story that aligns with Wheeler's slogan: It from
Bit. To do this we will first provide an account of the theory of
computation at play here. Then we will dive into a calculation-driven
interpretation of the operations of quantum mechanics.

The reason we take this approach is that -- until very recently --
there hasn't been an axiomatic account of quantum mechanics. As a
result there has been no sharp delineation of the mathematical theory
supporting interpretation of the physical theory and the physical
theory, itself. So, ambient features of the maths are free to be
exploited (or supressed) without a real accounting of their physical
relevance. There is no sharp statement ``here's the physical theory''
qua \emph{theory} and ``here's the mathematical interpretation''
enabling a judgment of how faithful the interpretation is -- apart
from experimental observation. When there is an axiomatic account we
can judge how well a given mathematical formalism supports an
interpretation of the axioms, independent of
experimentation. Likewise, we can judge how well we have captured our
physical evidence and experience with our axiomatics, independent of
any specific mathematical implementation, with accidental detail that
may or may not have physical significance. 

In lieu of a fully fleshed out and vetted axiomatic account of quantum
mechanics, interpreting the operational notions in service of modeling
physical systems will have to suffice. In other words, we are not in
the business of providing a model of Hilbert spaces and operators. We
are in the business of providing a model of quantum mechanics because
we are motivated by testing our notions of dynamics against physical
theory; and, the predictive calculations of the physical theory must
serve as the best formulation -- shy of a fully fleshed out axiomatic
account -- of the physical theory itself (as they have for scientific
theories since time immemorial). Put another way, despite a
whole-hearted commitment to an It-from-Bit ontology, we are firmly
aligned with the shut-up-and-calculate camp as the best way to obtain
results either from the physical perspective or as a quality assurance
measure of our fledgling theory of dynamics.

In detail, we present a reflective process calculus. Then we develop
intuitive correspondences between the notions available in this
calculus and the usual physical notions supporting quantum mechanical
calculations. Thus, 

\begin{table}[htp]
  \center{
    \fbox{
      \begin{tabular}{c|c}
        quantum mechanics & process calculus \\
        \hline
        scalar & name \\
        state vector & process \\
        dual & contextual duals \\
        matrix & formal sums of process-context-dual pairs \\
        orthogonality & process annihilation \\
        inner product & execution-formula + quoting
      \end{tabular}
    }
  }
  \caption{QM - process calculi correspondences}
\end{table}

Then we tighten up these intuitions to operational definitions. We
employ the Dirac notation as the best proxy we can find for an
abstract syntax of the quantum mechanical notions. The definitions we
develop put us in contact with equational constraints coming from the
theory that we demonstrate the definitions and calculations satisfy.

This puts us in a position to shut up and calculate for the
Stern-Gerlach experimental set up, showing how these predictive
calculations become calculations on processes in our theory of a
reflective process calculus.

Penultimately, we demonstrate that the notion of metric coming from
the inner product coincides with the notion of metric available from
the theory of bisimulation. This demonstration gives us the right to
think of space as arising from behavior. Finally, we consider where we
might go from the new vantage point we have obtained.

% section introduction (end) 
 
% section introduction (end)

% \documentclass[12pt]{llncs}
%\documentclass{jktr}

\usepackage[pdftex]{hyperref}                   
\usepackage {listings}
\usepackage {mathpartir}
\usepackage{bcprules}
%\usepackage{listings}
                       
\usepackage{graphicx} 
%\usepackage[margins=2.5cm,nohead,nofoot]{geometry}
%\usepackage{geometry}
\usepackage{amsfonts}
\usepackage{amstext}
\usepackage{latexsym}
\usepackage{amssymb}
\usepackage{color}


%\include{myPreamble}
\include{qm2pi.local} 

%\ifpdf
%\usepackage[pdftex]{graphicx}
%\else
%\usepackage{graphicx}
%\fi

 % \ifpdf
%  \usepackage{pdfsync}
%  \if


%\title{Brief Article}
%\author{David F. Snyder}
%\author{L.G. Meredith}

%\address{Dept. of Math., Texas State University--San Marcos, San Marcos, TX 78666}
       
\pagestyle{empty}


\begin{document}

\lstset{language=[Objective]Caml,frame=shadowbox}

\input{qm2pi.front}

% section front matter (end)

\input{qm2pi.intro} 
 
% section introduction (end)

% \input{qm2pi.knotations} 

% section notation (end)

\input{qm2pi.process.calculi} 

% section concurrent_process_calculi_and_spatial_logics_ (end)
    
%\input{qm2pi.knots2pi} 

%\input{qm2pi.trefoil} 

%\input{qm2pi.mainthm} 

% subsection basic_interpretation (end)

%\input{qm2pi.rho.presentation} 
\subsection{The syntax and semantics of the notation system}\label{sub:the_syntax_and_semantics_of_the_notation_system} % (fold)

We now summarize a technical presentation of the calculus that
embodies our theory of dynamics. The typical presentation of such a
calculus follows the style of giving generators and relations on
them. The grammar, below, describing term constructors, freely
generates the set of processes, $\Proc$. This set is then quotiented
by a relation known as structural congruence and it is over this set
that the notion of dynamics is expressed. This presentation is
essentially that of \cite{MeredithR05} with the addition of
polyadicity and summation. For readability we have relegated some of
the technical subtleties to an appendix.

\subsubsection{Process grammar}\label{subsub:process_grammar}

\begin{mathpar}
  \inferrule* [lab=synchronization] {} {{M} \bc \pzero \;|\; x?F \;|\; x!C }
  \and
  \inferrule* [lab=abstraction] {} {{F} \bc (x)P}
  \and
  \inferrule* [lab=concretion] {} {{C} \bc \langle Q \rangle}
  \and
  \inferrule* [lab=process] {} {{P,Q} \bc M \;| \;P|Q \;|\; @{x}}
  \and
  \inferrule* [lab=name] {} {{x} \bc \quotep{P}}
\end{mathpar} 

Note that $\vec{x}$ (resp. $\vec{P}$) denotes a vector of names
(resp. processes) of length $|\vec{x}|$ (resp. $|\vec{P}|$). We adopt
the following useful abbreviations.

\begin{mathpar}
   x?(\vec{y}).P := x.(\vec{y})P \and  x\clift{\vec{P}} := x.\clift{\vec{P}}
   \and x!(y) := \lift{x}{\dropn{y}}
   \and \Pi_{i=0}^{n-1}P_i := P_0 | \ldots | P_{n-1}
\end{mathpar}

\subsubsection{Structural congruence}

\paragraph{Free and bound names and alpha-equivalence.} At the
core of structural equivalence is alpha-equivalence which identifies
process that are the same up to a change of variable. Formally, we
recognize the distinction between free and bound names. The free names
of a process, $\freenames{P}$, may be calculated recursively as
follows:

\begin{mathpar}
\freenames{\pzero} := \emptyset
  \and \\
  \freenames{x?(y).P} := \{ x \} \cup (\freenames{P} \setminus \{ y \})
  \and 
  \freenames{x!\langle P \rangle} := \{ x \} \cup \{ P \} 
  \and \\
  \freenames{P|Q} := \freenames{P} \cup \freenames{Q}
  \and \\
  \freenames{@{x}} := \{ x \}
\end{mathpar}

$\pi$
$\quotep{\pi}$

$\freenames{-} : \pi \to \mathcal{P}(\quotep{\pi})$

\begin{eqnarray*}
  \freenames{\pzero} & := & \emptyset \\
  \freenames{x?(y).P} & := & \{ x \} \cup (\freenames{P} \setminus \{ y \}) \\
  \freenames{x!\langle P \rangle} & := & \{ x \} \cup \{ P \} \\
  \freenames{P|Q} & := & \freenames{P} \cup \freenames{Q} \\
  \freenames{\dropn{x}} & := & \{ x \}
\end{eqnarray*}

The bound names of a process, $\boundnames{P}$, are those names occurring in $P$
that are not free. For example, in $x?(y).0$, the name $x$ is free, while $y$ is bound.

\begin{mathpar}
  \inferrule* [lab=monoidal-laws] {} { P|Q \equiv Q|P \and P|0 \equiv P \and P|(Q|R) \equiv (P|Q)|R }
\end{mathpar}

\begin{mathpar}
  \inferrule* [lab=alpha-equivalence] {} { (x)P \equiv (y)P\{y/x\} \and y \not\in \freenames{P} }
\end{mathpar}

\begin{definition}
Then two processes, $P,Q$, are alpha-equivalent if $P = Q\{\vec{y}/\vec{x}\}$ for
some $\vec{x} \in \boundnames{Q},\vec{y} \in \boundnames{P}$, where $Q\{\vec{y}/\vec{x}\}$
denotes the capture-avoiding substitution of $\vec{y}$ for $\vec{x}$ in $Q$.
\end{definition}

\begin{definition}
  The {\em structural congruence} \cite{SangiorgiWalker} , $\equiv$,
  between processes is the least congruence containing
  alpha-equivalence, satisfying the abelian monoid laws
  (associativity, commutativity and $\pzero$ as identity) for parallel
  composition $|$ and for summation $+$.
\end{definition}

\subsection{Name equivalence}

We take name equivalence, written $\nameeq$, to be the smallest
equivalence relation generated by the following rules.

\begin{mathpar}
\inferrule*[lab=Quote-drop]
{ }
{ \quotep{@{x}} \nameeq x }

\inferrule*[lab=Struct-equiv]
{ P \scong Q }
{ \quotep{P} \nameeq \quotep{Q} }
\end{mathpar}

The astute reader will have noticed that the mutual recursion of names
and processes imposes a mutual recursion on alpha-equivalence and
structural equivalence via name-equivalence. Fortunately, all of this
works out pleasantly and we may calculate in the natural way, free of
concern. The reader interested in the details is referred to the
appendix \ref{appendix:rho_details}.

\subsection{Substitution}

We use $\Proc$ for the set of processes, $\QProc$ for the set of
names, and $\id{\{}\vec{y} / \vec{x} \id{\}}$ to denote partial maps,
$s : \QProc \rightarrow \QProc$. A map, $s$ lifts, uniquely, to a map
on process terms, $\widehat{s} : \Proc \rightarrow \Proc$ by the
following equations.

\begin{mathpar}
  (0) \psubstp{Q}{P} := 0 \\
  (R \juxtap S) \psubstp{Q}{P}
  :=    
  (R)\psubstp{Q}{P} \juxtap (S) \psubstp{Q}{P} \\
  (x?(y).R) \psubstp{Q}{P}    
  :=    
  (x)\substp{Q}{P} (z)\concat( (R \psubstn{z}{y}) \psubstp{Q}{P} ) \\
  (\lift{x}{R}) \psubstp{Q}{P}  
  :=
  \lift{(x)\substp{Q}{P}}{ R \psubstp{Q}{P} } \\
%   (\dropn{x})  \psubstp{Q}{P}       
%   := 
%   \left\{ 
%     \begin{array}{ccc} 
%       \dropn{\quotep{Q}} & & x \nameeq \quotep{P} \\
%       \dropn{x} & & otherwise \\
%     \end{array}
%   \right. 
  (\dropn{x})  \psubstp{Q}{P}       
  := 
  \left\{ 
    \begin{array}{ccc} 
      Q & & x \nameeq \quotep{P} \\
      \dropn{x} & & otherwise \\
    \end{array}
  \right.
\end{mathpar}
 

where

\begin{eqnarray}
  (x)\id{\{} \lpquote Q \rpquote / \lpquote P \rpquote \id{\}}            = 
  \left\{ 
    \begin{array}{ccc}
      \lpquote Q \rpquote & & x \nameeq \lpquote P \rpquote \\
      x & & otherwise \\
    \end{array}
  \right. \nonumber
\end{eqnarray}

and $z$ is chosen distinct from $\quotep{P}$, $\quotep{Q}$, the free
names in $Q$, and all the names in $R$. Our $\alpha$-equivalence will
be built in the standard way from this substitution.

\begin{remark}\label{rem:no_self_referential_names}
  One consequence of these definitions is that $\forall P. \quotep{P}
  \not\in \freenames{P}$.
\end{remark}

\subsection{ Dynamic quote: an example }

Anticipating something of what's to come, consider applying the
substitution, $\widehat{\id{\{}u / z \id{\}}}$, to the following pair
of processes, $\lift{w}{y!(z)}$ and $w[ \lpquote y!(z) \rpquote ]$.

\begin{eqnarray}
	\lift{w}{y!(z)}\widehat{\id{\{}u / z \id{\}}}
		& = &
		\lift{w}{y!(u)} \nonumber\\
	w[ \lpquote y!(z) \rpquote ] \widehat{ \id{\{}u / z \id{\}} }
		& = &
		w[ \lpquote y!(z) \rpquote ] \nonumber
\end{eqnarray}

Because the body of the process between quotes is impervious to
substitution, we get radically different answers. In fact, by
examining the first process in an input context,
e.g. $x?(z).\lift{w}{y!(z)}$, we see that the process under the lift
operator may be shaped by prefixed inputs binding a name inside it. In
this sense, the lift operator will be seen as a way to dynamically
construct processes before reifying them as names.

Finally equipped with these standard features we can present the
dynamics of the calculus.

\subsubsection{Operational semantics} 

Finally, we introduce the computational dynamics. What marks these
algebras as distinct from other more traditionally studied algebraic
structures, e.g. vector spaces or polynomial rings, is the manner in
which dynamics is captured. In traditional structures, dynamics is typically
expressed through morphisms between such structures, as in linear maps
between vector spaces or morphisms between rings. In algebras
associated with the semantics of computation, the dynamics is
expressed as part of the algebraic structure itself, through a
reduction reduction relation typically denoted by $\red$. Below, we
give a recursive presentation of this relation for the calculus used
in the encoding.

$\red \subseteq \pi \times \pi$
$\red : \pi \to \mathcal{P}(\pi)$

\begin{mathpar}
  \inferrule* [lab=Comm] { \textsf{match}( x_{src}, x_{trgt} ) } { x_{trgt}?(y)P \; | \; x_{src}!\langle {Q} \rangle \red P\{\quotep{Q}/y}\} }
  \and \\
  \inferrule* [lab=Par] {{P} \red {P}'} {{{P} | {Q}} \red {{P}' | {Q}}}
  \and
  \inferrule* [lab=Equiv]{{{P} \scong {P}'} \andalso {{P}' \red {Q}'} \andalso {{Q}' \scong {Q}}}{{P} \red {Q}}
\end{mathpar}

\begin{eqnarray*}
  match_{\equiv} (\quotep{P},\quotep{Q}) & := & P \equiv Q \\
  match_{\dagger}(\quotep{P},\quotep{Q}) & := & \forall R. P|Q \red^{*} R => R \red^{*} 0 \\
  match_{K}(\quotep{P},\quotep{Q}) & := & K \mbox{ for some context } K
\end{eqnarray*}

$u?(x)P | u!\langle Q \rangle \red P\{\quotep{Q}/x\}$

%We write $\wred$ for $\red^*$, and $P\red$ if $\exists Q $ such that $ P \red Q$.
We write $P\red$ if $\exists Q $ such that $ P \red Q$ and $P\not\red$, otherwise.

\section{Replication}

As mentioned before, it is known that replication (and hence
recursion) can be implemented in a higher-order process algebra
\cite{SangiorgiWalker}. As our first example of calculation with the
machinery thus far presented we give the construction explicitly in
the {\rhoc}.

\begin{eqnarray}
	D_{x} & := & \prefix{x}{y}{(\binpar{\outputp{x}{y}}{@{y}})} \nonumber\\
	\bangp_{x}{P} & := & \binpar{{x}!\langle{\binpar{D_{x}}{P}}\rangle}{D_{x}} \nonumber
\end{eqnarray}

\begin{eqnarray}
	\bangp_{x}{P} & & \nonumber\\
	=
	& {x}!\langle{(\prefix{x}{y}{(\outputp{x}{y} | @{y})) | P}}\rangle 
	      | \prefix{x}{y}{(\outputp{x}{y} | @{y})} & \nonumber\\
	\red
	& (\outputp{x}{y} | @{y})\substn{\quotep{(\prefix{x}{y}{(@{y} | \outputp{x}{y})) | P}}}{y} & \nonumber\\
	=
	& \outputp{x}{\quotep{(\prefix{x}{y}{(\outputp{x}{y} | @{y})) | P}}}
	  | {(\prefix{x}{y}{(\outputp{x}{y} | @{y})) | P}} & \nonumber\\
	\red
	& \ldots & \nonumber\\
	\red^*
	& P | P | \ldots & \nonumber
\end{eqnarray}

Of course, this encoding, as an implementation, runs away, unfolding
$\bangp{P}$ eagerly. A lazier and more implementable replication
operator, restricted to input-guarded processes, may be obtained as follows.

\begin{eqnarray}
\bangp{\prefix{u}{v}{P}} 
	:= 
	\binpar{\lift{x}{\prefix{u}{v}{(\binpar{D(x)}{P})}}}{D(x)} \nonumber
\end{eqnarray}

\begin{remark}
  Note that the lazier definition still does not deal with summation
  or mixed summation (i.e. sums over input and output). The reader is
  invited to construct definitions of replication that deal with these
  features. 

  Further, the definitions are parameterized in a name, $x$. Can you,
  gentle reader, make a definition that eliminates this parameter and
  guarantees no accidental interaction between the replication
  machinery and the process being replicated -- i.e. no accidental
  sharing of names used by the process to get its work done and the
  name(s) used by the replication to effect copying. This latter
  revision of the definition of replication is crucial to obtaining
  the expected identity $!!P \sim !P$.
\end{remark}

\begin{remark}\label{rem:paradoxical_combinator}
  The reader familiar with the lambda calculus will have noticed the
  similarity between $D$ and the paradoxical combinator.

  [Ed. note: the existence of this seems to suggest we have to be more
  restrictive on the set of processes and names we admit if we are to
  support no-cloning.]
\end{remark}

\subsubsection{Bisimulation}

The computational dynamics gives rise to another kind of equivalence,
the equivalence of computational behavior. As previously mentioned
this is typically captured \emph{via} some form of bisimulation.

% The notion we use in this paper is weak barbed bisimulation
% \cite{milner91polyadicpi}.

The notion we use in this paper is derived from weak barbed
bisimulation \cite{milner91polyadicpi}. 

\begin{definition}
An \emph{observation relation}, $\downarrow_{\mathcal N}$, over a set
of names, $\mathcal N$, is the smallest relation satisfying the rules
below.

\infrule[Out-barb]{y \in {\mathcal N}, \; x \nameeq y}
		  {\outputp{x}{v} \downarrow_{\mathcal N} x}
\infrule[Par-barb]{\mbox{$P\downarrow_{\mathcal N} x$ or $Q\downarrow_{\mathcal N} x$}}
		  {\binpar{P}{Q} \downarrow_{\mathcal N} x}

We write $P \Downarrow_{\mathcal N} x$ if there is $Q$ such that 
$P \wred Q$ and $Q \downarrow_{\mathcal N} x$.
\end{definition}

\begin{definition}
%\label{def.bbisim}
An  ${\mathcal N}$-\emph{barbed bisimulation} over a set of names, ${\mathcal N}$, is a symmetric binary relation 
${\mathcal S}_{\mathcal N}$ between agents such that $P\rel{S}_{\mathcal N}Q$ implies:
\begin{enumerate}
\item If $P \red P'$ then $Q \wred Q'$ and $P'\rel{S}_{\mathcal N} Q'$.
\item If $P\downarrow_{\mathcal N} x$, then $Q\Downarrow_{\mathcal N} x$.
\end{enumerate}
$P$ is ${\mathcal N}$-barbed bisimilar to $Q$, written
$P \wbbisim_{\mathcal N} Q$, if $P \rel{S}_{\mathcal N} Q$ for some ${\mathcal N}$-barbed bisimulation ${\mathcal S}_{\mathcal N}$.
\end{definition}

$\mathcal{R} \subseteq \pi \times \pi$

$P \mathcal{R} Q => \forall P'. P \red P' \Rightarrow \exists Q'. Q \red Q', P' \mathcal{R} Q'$

$P \vdash x \Rightarrow Q \vdash x$

\begin{mathpar}
  \inferrule*[lab=Out-barb]{x \nameeq y}{{y}!\langle{Q}\rangle \vdash x}
  \and
  \inferrule*[lab=Par-barb]{\mbox{$P\vdash x$ or $Q\vdash x$}}{\binpar{P}{Q} \vdash x}
\end{mathpar}

\subsubsection{Contexts}

One of the principle advantages of computational calculi like the
$\pi$-calculus is a well-defined notion of context,
contextual-equivalence and a correlation between
contextual-equivalence and notions of bisimulation. The notion of
context allows the decomposition of a process into (sub-)process and
its syntactic environment, its context. Thus, a context may be
thought of as a process with a ``hole'' (written $\Box$) in it. The
application of a context $M$ to a process $P$, written $M[P]$, is
tantamount to filling the hole in $M$ with $P$. In this paper we do
not need the full weight of this theory, but do make use of the notion
of context in the proof the main theorem. 

\begin{mathpar}
  \inferrule* [lab=summation] {} {{M_{M},M_{N}} \bc \Box \;|\; x.M_{A} \;|\; M_{M}+M_{N}}
  \and
  \inferrule* [lab=agent] {} {{M_{A}} \bc (\vec{x})M_{P} \;| \; \clift{P_0,\ldots,M_{P},\ldots,P_N}}
  \and \\
  \inferrule* [lab=process] {} {{M_{P}} \bc M_{N} \;| \;P|M_{P} }
\end{mathpar} 

\begin{mathpar}
  \inferrule* [lab=sychronization] {} {M_{N} \bc \Box \;|\; x?M_{F} \;|\; x!M_{C}}
  \and
  \inferrule* [lab=abstraction] {} {{M_{F}} \bc (x)M_{P} }
  \and
  \inferrule* [lab=concretion] {} {{M_{C}} \bc \langle M_{P} \rangle }
  \and \\
  \inferrule* [lab=process] {} {{M_{P}} \bc M_{N} \;| \;P|M_{P} }
\end{mathpar}

\begin{definition}[contextual application] Given a context $M$, and
  process $P$, we define the \emph{contextual application}, $M[P] :=
  M\{P/\Box\}$. That is, the contextual application of M to P is the
  substitution of $P$ for $\Box$ in $M$.
\end{definition}

$\meaningof{-} : L \to \mathcal{P}(\pi)$

\begin{mathpar}
  \inferrule* [lab=collection] {} {\meaningof{true} = \pi, \and \meaningof{~E} = \pi \setminus \meaningof{E}, \and \meaningof{E_{1} \& E_{2}} = \meaningof{E_{1}} \cap \meaningof{E_{2}}}
\end{mathpar}

\begin{mathpar}
  \inferrule* [lab=structure] {} {\meaningof{0} = \{ P \in \pi | P \equiv 0 \}, \and \\ \meaningof{E_1 | E_2} = \{ P \in \pi | P \equiv P_{1} | P_{2}, P_{1} \in \meaningof{E_{1}}, P_{2} \in \meaningof{E_2}\} }
\end{mathpar}

\begin{mathpar}
 \inferrule* [lab=behavior] {} {\meaningof{\langle a?b \rangle E} = \{ P \in \pi | P \equiv Q | u?(y)P', \\ \and \\\\ \and \\ \;\;\; u \in \meaningof{a}, \forall z.P'\{z/y\} \in \meaningof{E\{z/b\}}\}, \and \\ \meaningof{a!E} = \{ P \in \pi | P \equiv Q | x!\langle P' \rangle, x \in \meaningof{a} P' \in \meaningof{E}\} }
\end{mathpar}

\begin{mathpar}
 \inferrule* [lab=nominal] {} {\meaningof{\quotep{E}} = \{ \quotep{P} \in \quotep{\pi} | P \in \meaningof{E} \}, \and \meaningof{\quotep{P}} = \{ \quotep{Q} \in \quotep{\pi} | P \equiv Q \} \and \\ \meaningof{@\quotep{E}} = \{ P \in \pi | P \equiv @x, x \in \meaningof{E} \}}
\end{mathpar}

\begin{eqnarray*}
  \\
  \meaningof{-} : TS \to ST
\end{eqnarray*}

\begin{eqnarray*}
  \\
  L : TS \to ST
\end{eqnarray*}

\begin{eqnarray*}
  \\
  P \models E \iff P \in \meaningof{E}
\end{eqnarray*}

\begin{eqnarray*}
  P \approx_{L} Q \iff \forall E \in L. P \models E \iff Q \models E
\end{eqnarray*}

\begin{eqnarray*}
  P \approx_{K} Q
\end{eqnarray*}

\begin{eqnarray*}
  P \approx Q
\end{eqnarray*}

$\approx_{K} = \approx = \approx_{L}$

\subsubsection{Contextual duality}

Note that contexts extend the quotation operation to a family of
operations from processes to names. Given a context, $M$, we can
define a \emph{nominal context}, $\quotep{M}$ by $\quotep{M}[P] :=
\quotep{M[P]}$. To foreshadow what is to come we observe that these
operations enjoy a duality with processes very much like the duality
between vectors and maps from vectors to scalars.

Further, because the calculus is essentially higher-order, we have a
correspondence between contexts and processes. More specifically,
given a name $x$ and a context $M$ we can construct $M^{*}_{x}$ such
that 

\begin{mathpar}
  M^{*}_{x} | \lift{x}{P} \red M[P]
\end{mathpar}

namely,

\begin{mathpar}
  M^{*}_{x} := x?(u).M[\dropn{u}]
\end{mathpar}

The dependence of $M^{*}_{x}$ on a name makes it an abstraction, 

\begin{mathpar}
  M^{*} := (x)x?(u).M[\dropn{u}]
\end{mathpar}

\subsection{Additional notation}

It will sometimes be convenient to denote the process a name
quotes. We already have the notation $x = \quotep{P}$, but it will be
convenient to introduce an alternate notation, $\procn{x}$, when we
want to emphasize the connection to the use of the name. Note that, by
virtue of name equivalence, $\quotep{\procn{x}} \nameeq x$; so, the
notation is consistent with previous definitions.

Further, because names have structure it is possible to effect
substitutions on the basis of that structure. This means we need to
upgrade our notation for substitutions, which we accomplish by
adapting comprehension notation. Thus,

\begin{mathpar}
  P\{ y / x : x \in S \}
\end{mathpar}

is interpreted to mean the process derived from P by replacing (in a
capture-avoiding manner) each occurrence of $x$ in $S$ by $y$. For example,

\begin{mathpar}
  P\{ \quotep{\procn{x}|\procn{x}} / x : x \in \freenames{P} \}
\end{mathpar}

will replace each (occurrence) of a free name $x$ in $P$ by
$\quotep{\procn{x}|\procn{x}}$.

Also, we will avail ourselves of the notation $x^{L}$ and $x^{R}$ to
denote injections of a name into disjoint copies of the name
space. There are numerous ways to accomplish this. One example can be
found in \cite{MeredithR05}. This notation overloads to vectors of
names: $\vec{x}^{\pi} := (x_{i}^{\pi} \; : \; 0 \leq i < |\vec{x}| )$ where $\pi \in \{L,R\}$.

We also use $P^{\Box} := P|\Box$.

In \cite{MeredithR05} an interpretation of the new operator is
given. It turns out that there are several possible interpretations
all enjoying the requisite algebraic properties of the operator (see
\cite{milner91polyadicpi}). We will therefore make liberal use of
$(\nu\; \vec{x})P$.

% subsection the_syntax_and_semantics_of_the_notation_system (end)   

\input{qm2pi.qmops} 

\input{qm2pi.sterngerlach} 

\input{qm2pi.metric} 

% section concurrent_process_calculi (end)

%\input{qm2pi.proofsketch}

% section proof sketch (end)

%\input{qm2pi.slviaknots} 

% section spatial logic via knots (end)

\input{qm2pi.conclusion}

% section conclusion (end)

%\input{qm2pi.dtcodes} 

% section wiring algorithm (end)

\input{qm2pi.ack} 

% section acknowledgments (end)

\newpage


\bibliographystyle{plain}   
\bibliography{../../biblios/main.bib}

\input{qm2pi.rhodetails}

\end{document}

 

% section notation (end)

\input{qm2pi.process.calculi} 

% section concurrent_process_calculi_and_spatial_logics_ (end)
    
%\documentclass[12pt]{llncs}
%\documentclass{jktr}

\usepackage[pdftex]{hyperref}                   
\usepackage {listings}
\usepackage {mathpartir}
\usepackage{bcprules}
%\usepackage{listings}
                       
\usepackage{graphicx} 
%\usepackage[margins=2.5cm,nohead,nofoot]{geometry}
%\usepackage{geometry}
\usepackage{amsfonts}
\usepackage{amstext}
\usepackage{latexsym}
\usepackage{amssymb}
\usepackage{color}


%\include{myPreamble}
\include{qm2pi.local} 

%\ifpdf
%\usepackage[pdftex]{graphicx}
%\else
%\usepackage{graphicx}
%\fi

 % \ifpdf
%  \usepackage{pdfsync}
%  \if


%\title{Brief Article}
%\author{David F. Snyder}
%\author{L.G. Meredith}

%\address{Dept. of Math., Texas State University--San Marcos, San Marcos, TX 78666}
       
\pagestyle{empty}


\begin{document}

\lstset{language=[Objective]Caml,frame=shadowbox}

\input{qm2pi.front}

% section front matter (end)

\input{qm2pi.intro} 
 
% section introduction (end)

% \input{qm2pi.knotations} 

% section notation (end)

\input{qm2pi.process.calculi} 

% section concurrent_process_calculi_and_spatial_logics_ (end)
    
%\input{qm2pi.knots2pi} 

%\input{qm2pi.trefoil} 

%\input{qm2pi.mainthm} 

% subsection basic_interpretation (end)

%\input{qm2pi.rho.presentation} 
\subsection{The syntax and semantics of the notation system}\label{sub:the_syntax_and_semantics_of_the_notation_system} % (fold)

We now summarize a technical presentation of the calculus that
embodies our theory of dynamics. The typical presentation of such a
calculus follows the style of giving generators and relations on
them. The grammar, below, describing term constructors, freely
generates the set of processes, $\Proc$. This set is then quotiented
by a relation known as structural congruence and it is over this set
that the notion of dynamics is expressed. This presentation is
essentially that of \cite{MeredithR05} with the addition of
polyadicity and summation. For readability we have relegated some of
the technical subtleties to an appendix.

\subsubsection{Process grammar}\label{subsub:process_grammar}

\begin{mathpar}
  \inferrule* [lab=synchronization] {} {{M} \bc \pzero \;|\; x?F \;|\; x!C }
  \and
  \inferrule* [lab=abstraction] {} {{F} \bc (x)P}
  \and
  \inferrule* [lab=concretion] {} {{C} \bc \langle Q \rangle}
  \and
  \inferrule* [lab=process] {} {{P,Q} \bc M \;| \;P|Q \;|\; @{x}}
  \and
  \inferrule* [lab=name] {} {{x} \bc \quotep{P}}
\end{mathpar} 

Note that $\vec{x}$ (resp. $\vec{P}$) denotes a vector of names
(resp. processes) of length $|\vec{x}|$ (resp. $|\vec{P}|$). We adopt
the following useful abbreviations.

\begin{mathpar}
   x?(\vec{y}).P := x.(\vec{y})P \and  x\clift{\vec{P}} := x.\clift{\vec{P}}
   \and x!(y) := \lift{x}{\dropn{y}}
   \and \Pi_{i=0}^{n-1}P_i := P_0 | \ldots | P_{n-1}
\end{mathpar}

\subsubsection{Structural congruence}

\paragraph{Free and bound names and alpha-equivalence.} At the
core of structural equivalence is alpha-equivalence which identifies
process that are the same up to a change of variable. Formally, we
recognize the distinction between free and bound names. The free names
of a process, $\freenames{P}$, may be calculated recursively as
follows:

\begin{mathpar}
\freenames{\pzero} := \emptyset
  \and \\
  \freenames{x?(y).P} := \{ x \} \cup (\freenames{P} \setminus \{ y \})
  \and 
  \freenames{x!\langle P \rangle} := \{ x \} \cup \{ P \} 
  \and \\
  \freenames{P|Q} := \freenames{P} \cup \freenames{Q}
  \and \\
  \freenames{@{x}} := \{ x \}
\end{mathpar}

$\pi$
$\quotep{\pi}$

$\freenames{-} : \pi \to \mathcal{P}(\quotep{\pi})$

\begin{eqnarray*}
  \freenames{\pzero} & := & \emptyset \\
  \freenames{x?(y).P} & := & \{ x \} \cup (\freenames{P} \setminus \{ y \}) \\
  \freenames{x!\langle P \rangle} & := & \{ x \} \cup \{ P \} \\
  \freenames{P|Q} & := & \freenames{P} \cup \freenames{Q} \\
  \freenames{\dropn{x}} & := & \{ x \}
\end{eqnarray*}

The bound names of a process, $\boundnames{P}$, are those names occurring in $P$
that are not free. For example, in $x?(y).0$, the name $x$ is free, while $y$ is bound.

\begin{mathpar}
  \inferrule* [lab=monoidal-laws] {} { P|Q \equiv Q|P \and P|0 \equiv P \and P|(Q|R) \equiv (P|Q)|R }
\end{mathpar}

\begin{mathpar}
  \inferrule* [lab=alpha-equivalence] {} { (x)P \equiv (y)P\{y/x\} \and y \not\in \freenames{P} }
\end{mathpar}

\begin{definition}
Then two processes, $P,Q$, are alpha-equivalent if $P = Q\{\vec{y}/\vec{x}\}$ for
some $\vec{x} \in \boundnames{Q},\vec{y} \in \boundnames{P}$, where $Q\{\vec{y}/\vec{x}\}$
denotes the capture-avoiding substitution of $\vec{y}$ for $\vec{x}$ in $Q$.
\end{definition}

\begin{definition}
  The {\em structural congruence} \cite{SangiorgiWalker} , $\equiv$,
  between processes is the least congruence containing
  alpha-equivalence, satisfying the abelian monoid laws
  (associativity, commutativity and $\pzero$ as identity) for parallel
  composition $|$ and for summation $+$.
\end{definition}

\subsection{Name equivalence}

We take name equivalence, written $\nameeq$, to be the smallest
equivalence relation generated by the following rules.

\begin{mathpar}
\inferrule*[lab=Quote-drop]
{ }
{ \quotep{@{x}} \nameeq x }

\inferrule*[lab=Struct-equiv]
{ P \scong Q }
{ \quotep{P} \nameeq \quotep{Q} }
\end{mathpar}

The astute reader will have noticed that the mutual recursion of names
and processes imposes a mutual recursion on alpha-equivalence and
structural equivalence via name-equivalence. Fortunately, all of this
works out pleasantly and we may calculate in the natural way, free of
concern. The reader interested in the details is referred to the
appendix \ref{appendix:rho_details}.

\subsection{Substitution}

We use $\Proc$ for the set of processes, $\QProc$ for the set of
names, and $\id{\{}\vec{y} / \vec{x} \id{\}}$ to denote partial maps,
$s : \QProc \rightarrow \QProc$. A map, $s$ lifts, uniquely, to a map
on process terms, $\widehat{s} : \Proc \rightarrow \Proc$ by the
following equations.

\begin{mathpar}
  (0) \psubstp{Q}{P} := 0 \\
  (R \juxtap S) \psubstp{Q}{P}
  :=    
  (R)\psubstp{Q}{P} \juxtap (S) \psubstp{Q}{P} \\
  (x?(y).R) \psubstp{Q}{P}    
  :=    
  (x)\substp{Q}{P} (z)\concat( (R \psubstn{z}{y}) \psubstp{Q}{P} ) \\
  (\lift{x}{R}) \psubstp{Q}{P}  
  :=
  \lift{(x)\substp{Q}{P}}{ R \psubstp{Q}{P} } \\
%   (\dropn{x})  \psubstp{Q}{P}       
%   := 
%   \left\{ 
%     \begin{array}{ccc} 
%       \dropn{\quotep{Q}} & & x \nameeq \quotep{P} \\
%       \dropn{x} & & otherwise \\
%     \end{array}
%   \right. 
  (\dropn{x})  \psubstp{Q}{P}       
  := 
  \left\{ 
    \begin{array}{ccc} 
      Q & & x \nameeq \quotep{P} \\
      \dropn{x} & & otherwise \\
    \end{array}
  \right.
\end{mathpar}
 

where

\begin{eqnarray}
  (x)\id{\{} \lpquote Q \rpquote / \lpquote P \rpquote \id{\}}            = 
  \left\{ 
    \begin{array}{ccc}
      \lpquote Q \rpquote & & x \nameeq \lpquote P \rpquote \\
      x & & otherwise \\
    \end{array}
  \right. \nonumber
\end{eqnarray}

and $z$ is chosen distinct from $\quotep{P}$, $\quotep{Q}$, the free
names in $Q$, and all the names in $R$. Our $\alpha$-equivalence will
be built in the standard way from this substitution.

\begin{remark}\label{rem:no_self_referential_names}
  One consequence of these definitions is that $\forall P. \quotep{P}
  \not\in \freenames{P}$.
\end{remark}

\subsection{ Dynamic quote: an example }

Anticipating something of what's to come, consider applying the
substitution, $\widehat{\id{\{}u / z \id{\}}}$, to the following pair
of processes, $\lift{w}{y!(z)}$ and $w[ \lpquote y!(z) \rpquote ]$.

\begin{eqnarray}
	\lift{w}{y!(z)}\widehat{\id{\{}u / z \id{\}}}
		& = &
		\lift{w}{y!(u)} \nonumber\\
	w[ \lpquote y!(z) \rpquote ] \widehat{ \id{\{}u / z \id{\}} }
		& = &
		w[ \lpquote y!(z) \rpquote ] \nonumber
\end{eqnarray}

Because the body of the process between quotes is impervious to
substitution, we get radically different answers. In fact, by
examining the first process in an input context,
e.g. $x?(z).\lift{w}{y!(z)}$, we see that the process under the lift
operator may be shaped by prefixed inputs binding a name inside it. In
this sense, the lift operator will be seen as a way to dynamically
construct processes before reifying them as names.

Finally equipped with these standard features we can present the
dynamics of the calculus.

\subsubsection{Operational semantics} 

Finally, we introduce the computational dynamics. What marks these
algebras as distinct from other more traditionally studied algebraic
structures, e.g. vector spaces or polynomial rings, is the manner in
which dynamics is captured. In traditional structures, dynamics is typically
expressed through morphisms between such structures, as in linear maps
between vector spaces or morphisms between rings. In algebras
associated with the semantics of computation, the dynamics is
expressed as part of the algebraic structure itself, through a
reduction reduction relation typically denoted by $\red$. Below, we
give a recursive presentation of this relation for the calculus used
in the encoding.

$\red \subseteq \pi \times \pi$
$\red : \pi \to \mathcal{P}(\pi)$

\begin{mathpar}
  \inferrule* [lab=Comm] { \textsf{match}( x_{src}, x_{trgt} ) } { x_{trgt}?(y)P \; | \; x_{src}!\langle {Q} \rangle \red P\{\quotep{Q}/y}\} }
  \and \\
  \inferrule* [lab=Par] {{P} \red {P}'} {{{P} | {Q}} \red {{P}' | {Q}}}
  \and
  \inferrule* [lab=Equiv]{{{P} \scong {P}'} \andalso {{P}' \red {Q}'} \andalso {{Q}' \scong {Q}}}{{P} \red {Q}}
\end{mathpar}

\begin{eqnarray*}
  match_{\equiv} (\quotep{P},\quotep{Q}) & := & P \equiv Q \\
  match_{\dagger}(\quotep{P},\quotep{Q}) & := & \forall R. P|Q \red^{*} R => R \red^{*} 0 \\
  match_{K}(\quotep{P},\quotep{Q}) & := & K \mbox{ for some context } K
\end{eqnarray*}

$u?(x)P | u!\langle Q \rangle \red P\{\quotep{Q}/x\}$

%We write $\wred$ for $\red^*$, and $P\red$ if $\exists Q $ such that $ P \red Q$.
We write $P\red$ if $\exists Q $ such that $ P \red Q$ and $P\not\red$, otherwise.

\section{Replication}

As mentioned before, it is known that replication (and hence
recursion) can be implemented in a higher-order process algebra
\cite{SangiorgiWalker}. As our first example of calculation with the
machinery thus far presented we give the construction explicitly in
the {\rhoc}.

\begin{eqnarray}
	D_{x} & := & \prefix{x}{y}{(\binpar{\outputp{x}{y}}{@{y}})} \nonumber\\
	\bangp_{x}{P} & := & \binpar{{x}!\langle{\binpar{D_{x}}{P}}\rangle}{D_{x}} \nonumber
\end{eqnarray}

\begin{eqnarray}
	\bangp_{x}{P} & & \nonumber\\
	=
	& {x}!\langle{(\prefix{x}{y}{(\outputp{x}{y} | @{y})) | P}}\rangle 
	      | \prefix{x}{y}{(\outputp{x}{y} | @{y})} & \nonumber\\
	\red
	& (\outputp{x}{y} | @{y})\substn{\quotep{(\prefix{x}{y}{(@{y} | \outputp{x}{y})) | P}}}{y} & \nonumber\\
	=
	& \outputp{x}{\quotep{(\prefix{x}{y}{(\outputp{x}{y} | @{y})) | P}}}
	  | {(\prefix{x}{y}{(\outputp{x}{y} | @{y})) | P}} & \nonumber\\
	\red
	& \ldots & \nonumber\\
	\red^*
	& P | P | \ldots & \nonumber
\end{eqnarray}

Of course, this encoding, as an implementation, runs away, unfolding
$\bangp{P}$ eagerly. A lazier and more implementable replication
operator, restricted to input-guarded processes, may be obtained as follows.

\begin{eqnarray}
\bangp{\prefix{u}{v}{P}} 
	:= 
	\binpar{\lift{x}{\prefix{u}{v}{(\binpar{D(x)}{P})}}}{D(x)} \nonumber
\end{eqnarray}

\begin{remark}
  Note that the lazier definition still does not deal with summation
  or mixed summation (i.e. sums over input and output). The reader is
  invited to construct definitions of replication that deal with these
  features. 

  Further, the definitions are parameterized in a name, $x$. Can you,
  gentle reader, make a definition that eliminates this parameter and
  guarantees no accidental interaction between the replication
  machinery and the process being replicated -- i.e. no accidental
  sharing of names used by the process to get its work done and the
  name(s) used by the replication to effect copying. This latter
  revision of the definition of replication is crucial to obtaining
  the expected identity $!!P \sim !P$.
\end{remark}

\begin{remark}\label{rem:paradoxical_combinator}
  The reader familiar with the lambda calculus will have noticed the
  similarity between $D$ and the paradoxical combinator.

  [Ed. note: the existence of this seems to suggest we have to be more
  restrictive on the set of processes and names we admit if we are to
  support no-cloning.]
\end{remark}

\subsubsection{Bisimulation}

The computational dynamics gives rise to another kind of equivalence,
the equivalence of computational behavior. As previously mentioned
this is typically captured \emph{via} some form of bisimulation.

% The notion we use in this paper is weak barbed bisimulation
% \cite{milner91polyadicpi}.

The notion we use in this paper is derived from weak barbed
bisimulation \cite{milner91polyadicpi}. 

\begin{definition}
An \emph{observation relation}, $\downarrow_{\mathcal N}$, over a set
of names, $\mathcal N$, is the smallest relation satisfying the rules
below.

\infrule[Out-barb]{y \in {\mathcal N}, \; x \nameeq y}
		  {\outputp{x}{v} \downarrow_{\mathcal N} x}
\infrule[Par-barb]{\mbox{$P\downarrow_{\mathcal N} x$ or $Q\downarrow_{\mathcal N} x$}}
		  {\binpar{P}{Q} \downarrow_{\mathcal N} x}

We write $P \Downarrow_{\mathcal N} x$ if there is $Q$ such that 
$P \wred Q$ and $Q \downarrow_{\mathcal N} x$.
\end{definition}

\begin{definition}
%\label{def.bbisim}
An  ${\mathcal N}$-\emph{barbed bisimulation} over a set of names, ${\mathcal N}$, is a symmetric binary relation 
${\mathcal S}_{\mathcal N}$ between agents such that $P\rel{S}_{\mathcal N}Q$ implies:
\begin{enumerate}
\item If $P \red P'$ then $Q \wred Q'$ and $P'\rel{S}_{\mathcal N} Q'$.
\item If $P\downarrow_{\mathcal N} x$, then $Q\Downarrow_{\mathcal N} x$.
\end{enumerate}
$P$ is ${\mathcal N}$-barbed bisimilar to $Q$, written
$P \wbbisim_{\mathcal N} Q$, if $P \rel{S}_{\mathcal N} Q$ for some ${\mathcal N}$-barbed bisimulation ${\mathcal S}_{\mathcal N}$.
\end{definition}

$\mathcal{R} \subseteq \pi \times \pi$

$P \mathcal{R} Q => \forall P'. P \red P' \Rightarrow \exists Q'. Q \red Q', P' \mathcal{R} Q'$

$P \vdash x \Rightarrow Q \vdash x$

\begin{mathpar}
  \inferrule*[lab=Out-barb]{x \nameeq y}{{y}!\langle{Q}\rangle \vdash x}
  \and
  \inferrule*[lab=Par-barb]{\mbox{$P\vdash x$ or $Q\vdash x$}}{\binpar{P}{Q} \vdash x}
\end{mathpar}

\subsubsection{Contexts}

One of the principle advantages of computational calculi like the
$\pi$-calculus is a well-defined notion of context,
contextual-equivalence and a correlation between
contextual-equivalence and notions of bisimulation. The notion of
context allows the decomposition of a process into (sub-)process and
its syntactic environment, its context. Thus, a context may be
thought of as a process with a ``hole'' (written $\Box$) in it. The
application of a context $M$ to a process $P$, written $M[P]$, is
tantamount to filling the hole in $M$ with $P$. In this paper we do
not need the full weight of this theory, but do make use of the notion
of context in the proof the main theorem. 

\begin{mathpar}
  \inferrule* [lab=summation] {} {{M_{M},M_{N}} \bc \Box \;|\; x.M_{A} \;|\; M_{M}+M_{N}}
  \and
  \inferrule* [lab=agent] {} {{M_{A}} \bc (\vec{x})M_{P} \;| \; \clift{P_0,\ldots,M_{P},\ldots,P_N}}
  \and \\
  \inferrule* [lab=process] {} {{M_{P}} \bc M_{N} \;| \;P|M_{P} }
\end{mathpar} 

\begin{mathpar}
  \inferrule* [lab=sychronization] {} {M_{N} \bc \Box \;|\; x?M_{F} \;|\; x!M_{C}}
  \and
  \inferrule* [lab=abstraction] {} {{M_{F}} \bc (x)M_{P} }
  \and
  \inferrule* [lab=concretion] {} {{M_{C}} \bc \langle M_{P} \rangle }
  \and \\
  \inferrule* [lab=process] {} {{M_{P}} \bc M_{N} \;| \;P|M_{P} }
\end{mathpar}

\begin{definition}[contextual application] Given a context $M$, and
  process $P$, we define the \emph{contextual application}, $M[P] :=
  M\{P/\Box\}$. That is, the contextual application of M to P is the
  substitution of $P$ for $\Box$ in $M$.
\end{definition}

$\meaningof{-} : L \to \mathcal{P}(\pi)$

\begin{mathpar}
  \inferrule* [lab=collection] {} {\meaningof{true} = \pi, \and \meaningof{~E} = \pi \setminus \meaningof{E}, \and \meaningof{E_{1} \& E_{2}} = \meaningof{E_{1}} \cap \meaningof{E_{2}}}
\end{mathpar}

\begin{mathpar}
  \inferrule* [lab=structure] {} {\meaningof{0} = \{ P \in \pi | P \equiv 0 \}, \and \\ \meaningof{E_1 | E_2} = \{ P \in \pi | P \equiv P_{1} | P_{2}, P_{1} \in \meaningof{E_{1}}, P_{2} \in \meaningof{E_2}\} }
\end{mathpar}

\begin{mathpar}
 \inferrule* [lab=behavior] {} {\meaningof{\langle a?b \rangle E} = \{ P \in \pi | P \equiv Q | u?(y)P', \\ \and \\\\ \and \\ \;\;\; u \in \meaningof{a}, \forall z.P'\{z/y\} \in \meaningof{E\{z/b\}}\}, \and \\ \meaningof{a!E} = \{ P \in \pi | P \equiv Q | x!\langle P' \rangle, x \in \meaningof{a} P' \in \meaningof{E}\} }
\end{mathpar}

\begin{mathpar}
 \inferrule* [lab=nominal] {} {\meaningof{\quotep{E}} = \{ \quotep{P} \in \quotep{\pi} | P \in \meaningof{E} \}, \and \meaningof{\quotep{P}} = \{ \quotep{Q} \in \quotep{\pi} | P \equiv Q \} \and \\ \meaningof{@\quotep{E}} = \{ P \in \pi | P \equiv @x, x \in \meaningof{E} \}}
\end{mathpar}

\begin{eqnarray*}
  \\
  \meaningof{-} : TS \to ST
\end{eqnarray*}

\begin{eqnarray*}
  \\
  L : TS \to ST
\end{eqnarray*}

\begin{eqnarray*}
  \\
  P \models E \iff P \in \meaningof{E}
\end{eqnarray*}

\begin{eqnarray*}
  P \approx_{L} Q \iff \forall E \in L. P \models E \iff Q \models E
\end{eqnarray*}

\begin{eqnarray*}
  P \approx_{K} Q
\end{eqnarray*}

\begin{eqnarray*}
  P \approx Q
\end{eqnarray*}

$\approx_{K} = \approx = \approx_{L}$

\subsubsection{Contextual duality}

Note that contexts extend the quotation operation to a family of
operations from processes to names. Given a context, $M$, we can
define a \emph{nominal context}, $\quotep{M}$ by $\quotep{M}[P] :=
\quotep{M[P]}$. To foreshadow what is to come we observe that these
operations enjoy a duality with processes very much like the duality
between vectors and maps from vectors to scalars.

Further, because the calculus is essentially higher-order, we have a
correspondence between contexts and processes. More specifically,
given a name $x$ and a context $M$ we can construct $M^{*}_{x}$ such
that 

\begin{mathpar}
  M^{*}_{x} | \lift{x}{P} \red M[P]
\end{mathpar}

namely,

\begin{mathpar}
  M^{*}_{x} := x?(u).M[\dropn{u}]
\end{mathpar}

The dependence of $M^{*}_{x}$ on a name makes it an abstraction, 

\begin{mathpar}
  M^{*} := (x)x?(u).M[\dropn{u}]
\end{mathpar}

\subsection{Additional notation}

It will sometimes be convenient to denote the process a name
quotes. We already have the notation $x = \quotep{P}$, but it will be
convenient to introduce an alternate notation, $\procn{x}$, when we
want to emphasize the connection to the use of the name. Note that, by
virtue of name equivalence, $\quotep{\procn{x}} \nameeq x$; so, the
notation is consistent with previous definitions.

Further, because names have structure it is possible to effect
substitutions on the basis of that structure. This means we need to
upgrade our notation for substitutions, which we accomplish by
adapting comprehension notation. Thus,

\begin{mathpar}
  P\{ y / x : x \in S \}
\end{mathpar}

is interpreted to mean the process derived from P by replacing (in a
capture-avoiding manner) each occurrence of $x$ in $S$ by $y$. For example,

\begin{mathpar}
  P\{ \quotep{\procn{x}|\procn{x}} / x : x \in \freenames{P} \}
\end{mathpar}

will replace each (occurrence) of a free name $x$ in $P$ by
$\quotep{\procn{x}|\procn{x}}$.

Also, we will avail ourselves of the notation $x^{L}$ and $x^{R}$ to
denote injections of a name into disjoint copies of the name
space. There are numerous ways to accomplish this. One example can be
found in \cite{MeredithR05}. This notation overloads to vectors of
names: $\vec{x}^{\pi} := (x_{i}^{\pi} \; : \; 0 \leq i < |\vec{x}| )$ where $\pi \in \{L,R\}$.

We also use $P^{\Box} := P|\Box$.

In \cite{MeredithR05} an interpretation of the new operator is
given. It turns out that there are several possible interpretations
all enjoying the requisite algebraic properties of the operator (see
\cite{milner91polyadicpi}). We will therefore make liberal use of
$(\nu\; \vec{x})P$.

% subsection the_syntax_and_semantics_of_the_notation_system (end)   

\input{qm2pi.qmops} 

\input{qm2pi.sterngerlach} 

\input{qm2pi.metric} 

% section concurrent_process_calculi (end)

%\input{qm2pi.proofsketch}

% section proof sketch (end)

%\input{qm2pi.slviaknots} 

% section spatial logic via knots (end)

\input{qm2pi.conclusion}

% section conclusion (end)

%\input{qm2pi.dtcodes} 

% section wiring algorithm (end)

\input{qm2pi.ack} 

% section acknowledgments (end)

\newpage


\bibliographystyle{plain}   
\bibliography{../../biblios/main.bib}

\input{qm2pi.rhodetails}

\end{document}

 

%\documentclass[12pt]{llncs}
%\documentclass{jktr}

\usepackage[pdftex]{hyperref}                   
\usepackage {listings}
\usepackage {mathpartir}
\usepackage{bcprules}
%\usepackage{listings}
                       
\usepackage{graphicx} 
%\usepackage[margins=2.5cm,nohead,nofoot]{geometry}
%\usepackage{geometry}
\usepackage{amsfonts}
\usepackage{amstext}
\usepackage{latexsym}
\usepackage{amssymb}
\usepackage{color}


%\include{myPreamble}
\include{qm2pi.local} 

%\ifpdf
%\usepackage[pdftex]{graphicx}
%\else
%\usepackage{graphicx}
%\fi

 % \ifpdf
%  \usepackage{pdfsync}
%  \if


%\title{Brief Article}
%\author{David F. Snyder}
%\author{L.G. Meredith}

%\address{Dept. of Math., Texas State University--San Marcos, San Marcos, TX 78666}
       
\pagestyle{empty}


\begin{document}

\lstset{language=[Objective]Caml,frame=shadowbox}

\input{qm2pi.front}

% section front matter (end)

\input{qm2pi.intro} 
 
% section introduction (end)

% \input{qm2pi.knotations} 

% section notation (end)

\input{qm2pi.process.calculi} 

% section concurrent_process_calculi_and_spatial_logics_ (end)
    
%\input{qm2pi.knots2pi} 

%\input{qm2pi.trefoil} 

%\input{qm2pi.mainthm} 

% subsection basic_interpretation (end)

%\input{qm2pi.rho.presentation} 
\subsection{The syntax and semantics of the notation system}\label{sub:the_syntax_and_semantics_of_the_notation_system} % (fold)

We now summarize a technical presentation of the calculus that
embodies our theory of dynamics. The typical presentation of such a
calculus follows the style of giving generators and relations on
them. The grammar, below, describing term constructors, freely
generates the set of processes, $\Proc$. This set is then quotiented
by a relation known as structural congruence and it is over this set
that the notion of dynamics is expressed. This presentation is
essentially that of \cite{MeredithR05} with the addition of
polyadicity and summation. For readability we have relegated some of
the technical subtleties to an appendix.

\subsubsection{Process grammar}\label{subsub:process_grammar}

\begin{mathpar}
  \inferrule* [lab=synchronization] {} {{M} \bc \pzero \;|\; x?F \;|\; x!C }
  \and
  \inferrule* [lab=abstraction] {} {{F} \bc (x)P}
  \and
  \inferrule* [lab=concretion] {} {{C} \bc \langle Q \rangle}
  \and
  \inferrule* [lab=process] {} {{P,Q} \bc M \;| \;P|Q \;|\; @{x}}
  \and
  \inferrule* [lab=name] {} {{x} \bc \quotep{P}}
\end{mathpar} 

Note that $\vec{x}$ (resp. $\vec{P}$) denotes a vector of names
(resp. processes) of length $|\vec{x}|$ (resp. $|\vec{P}|$). We adopt
the following useful abbreviations.

\begin{mathpar}
   x?(\vec{y}).P := x.(\vec{y})P \and  x\clift{\vec{P}} := x.\clift{\vec{P}}
   \and x!(y) := \lift{x}{\dropn{y}}
   \and \Pi_{i=0}^{n-1}P_i := P_0 | \ldots | P_{n-1}
\end{mathpar}

\subsubsection{Structural congruence}

\paragraph{Free and bound names and alpha-equivalence.} At the
core of structural equivalence is alpha-equivalence which identifies
process that are the same up to a change of variable. Formally, we
recognize the distinction between free and bound names. The free names
of a process, $\freenames{P}$, may be calculated recursively as
follows:

\begin{mathpar}
\freenames{\pzero} := \emptyset
  \and \\
  \freenames{x?(y).P} := \{ x \} \cup (\freenames{P} \setminus \{ y \})
  \and 
  \freenames{x!\langle P \rangle} := \{ x \} \cup \{ P \} 
  \and \\
  \freenames{P|Q} := \freenames{P} \cup \freenames{Q}
  \and \\
  \freenames{@{x}} := \{ x \}
\end{mathpar}

$\pi$
$\quotep{\pi}$

$\freenames{-} : \pi \to \mathcal{P}(\quotep{\pi})$

\begin{eqnarray*}
  \freenames{\pzero} & := & \emptyset \\
  \freenames{x?(y).P} & := & \{ x \} \cup (\freenames{P} \setminus \{ y \}) \\
  \freenames{x!\langle P \rangle} & := & \{ x \} \cup \{ P \} \\
  \freenames{P|Q} & := & \freenames{P} \cup \freenames{Q} \\
  \freenames{\dropn{x}} & := & \{ x \}
\end{eqnarray*}

The bound names of a process, $\boundnames{P}$, are those names occurring in $P$
that are not free. For example, in $x?(y).0$, the name $x$ is free, while $y$ is bound.

\begin{mathpar}
  \inferrule* [lab=monoidal-laws] {} { P|Q \equiv Q|P \and P|0 \equiv P \and P|(Q|R) \equiv (P|Q)|R }
\end{mathpar}

\begin{mathpar}
  \inferrule* [lab=alpha-equivalence] {} { (x)P \equiv (y)P\{y/x\} \and y \not\in \freenames{P} }
\end{mathpar}

\begin{definition}
Then two processes, $P,Q$, are alpha-equivalent if $P = Q\{\vec{y}/\vec{x}\}$ for
some $\vec{x} \in \boundnames{Q},\vec{y} \in \boundnames{P}$, where $Q\{\vec{y}/\vec{x}\}$
denotes the capture-avoiding substitution of $\vec{y}$ for $\vec{x}$ in $Q$.
\end{definition}

\begin{definition}
  The {\em structural congruence} \cite{SangiorgiWalker} , $\equiv$,
  between processes is the least congruence containing
  alpha-equivalence, satisfying the abelian monoid laws
  (associativity, commutativity and $\pzero$ as identity) for parallel
  composition $|$ and for summation $+$.
\end{definition}

\subsection{Name equivalence}

We take name equivalence, written $\nameeq$, to be the smallest
equivalence relation generated by the following rules.

\begin{mathpar}
\inferrule*[lab=Quote-drop]
{ }
{ \quotep{@{x}} \nameeq x }

\inferrule*[lab=Struct-equiv]
{ P \scong Q }
{ \quotep{P} \nameeq \quotep{Q} }
\end{mathpar}

The astute reader will have noticed that the mutual recursion of names
and processes imposes a mutual recursion on alpha-equivalence and
structural equivalence via name-equivalence. Fortunately, all of this
works out pleasantly and we may calculate in the natural way, free of
concern. The reader interested in the details is referred to the
appendix \ref{appendix:rho_details}.

\subsection{Substitution}

We use $\Proc$ for the set of processes, $\QProc$ for the set of
names, and $\id{\{}\vec{y} / \vec{x} \id{\}}$ to denote partial maps,
$s : \QProc \rightarrow \QProc$. A map, $s$ lifts, uniquely, to a map
on process terms, $\widehat{s} : \Proc \rightarrow \Proc$ by the
following equations.

\begin{mathpar}
  (0) \psubstp{Q}{P} := 0 \\
  (R \juxtap S) \psubstp{Q}{P}
  :=    
  (R)\psubstp{Q}{P} \juxtap (S) \psubstp{Q}{P} \\
  (x?(y).R) \psubstp{Q}{P}    
  :=    
  (x)\substp{Q}{P} (z)\concat( (R \psubstn{z}{y}) \psubstp{Q}{P} ) \\
  (\lift{x}{R}) \psubstp{Q}{P}  
  :=
  \lift{(x)\substp{Q}{P}}{ R \psubstp{Q}{P} } \\
%   (\dropn{x})  \psubstp{Q}{P}       
%   := 
%   \left\{ 
%     \begin{array}{ccc} 
%       \dropn{\quotep{Q}} & & x \nameeq \quotep{P} \\
%       \dropn{x} & & otherwise \\
%     \end{array}
%   \right. 
  (\dropn{x})  \psubstp{Q}{P}       
  := 
  \left\{ 
    \begin{array}{ccc} 
      Q & & x \nameeq \quotep{P} \\
      \dropn{x} & & otherwise \\
    \end{array}
  \right.
\end{mathpar}
 

where

\begin{eqnarray}
  (x)\id{\{} \lpquote Q \rpquote / \lpquote P \rpquote \id{\}}            = 
  \left\{ 
    \begin{array}{ccc}
      \lpquote Q \rpquote & & x \nameeq \lpquote P \rpquote \\
      x & & otherwise \\
    \end{array}
  \right. \nonumber
\end{eqnarray}

and $z$ is chosen distinct from $\quotep{P}$, $\quotep{Q}$, the free
names in $Q$, and all the names in $R$. Our $\alpha$-equivalence will
be built in the standard way from this substitution.

\begin{remark}\label{rem:no_self_referential_names}
  One consequence of these definitions is that $\forall P. \quotep{P}
  \not\in \freenames{P}$.
\end{remark}

\subsection{ Dynamic quote: an example }

Anticipating something of what's to come, consider applying the
substitution, $\widehat{\id{\{}u / z \id{\}}}$, to the following pair
of processes, $\lift{w}{y!(z)}$ and $w[ \lpquote y!(z) \rpquote ]$.

\begin{eqnarray}
	\lift{w}{y!(z)}\widehat{\id{\{}u / z \id{\}}}
		& = &
		\lift{w}{y!(u)} \nonumber\\
	w[ \lpquote y!(z) \rpquote ] \widehat{ \id{\{}u / z \id{\}} }
		& = &
		w[ \lpquote y!(z) \rpquote ] \nonumber
\end{eqnarray}

Because the body of the process between quotes is impervious to
substitution, we get radically different answers. In fact, by
examining the first process in an input context,
e.g. $x?(z).\lift{w}{y!(z)}$, we see that the process under the lift
operator may be shaped by prefixed inputs binding a name inside it. In
this sense, the lift operator will be seen as a way to dynamically
construct processes before reifying them as names.

Finally equipped with these standard features we can present the
dynamics of the calculus.

\subsubsection{Operational semantics} 

Finally, we introduce the computational dynamics. What marks these
algebras as distinct from other more traditionally studied algebraic
structures, e.g. vector spaces or polynomial rings, is the manner in
which dynamics is captured. In traditional structures, dynamics is typically
expressed through morphisms between such structures, as in linear maps
between vector spaces or morphisms between rings. In algebras
associated with the semantics of computation, the dynamics is
expressed as part of the algebraic structure itself, through a
reduction reduction relation typically denoted by $\red$. Below, we
give a recursive presentation of this relation for the calculus used
in the encoding.

$\red \subseteq \pi \times \pi$
$\red : \pi \to \mathcal{P}(\pi)$

\begin{mathpar}
  \inferrule* [lab=Comm] { \textsf{match}( x_{src}, x_{trgt} ) } { x_{trgt}?(y)P \; | \; x_{src}!\langle {Q} \rangle \red P\{\quotep{Q}/y}\} }
  \and \\
  \inferrule* [lab=Par] {{P} \red {P}'} {{{P} | {Q}} \red {{P}' | {Q}}}
  \and
  \inferrule* [lab=Equiv]{{{P} \scong {P}'} \andalso {{P}' \red {Q}'} \andalso {{Q}' \scong {Q}}}{{P} \red {Q}}
\end{mathpar}

\begin{eqnarray*}
  match_{\equiv} (\quotep{P},\quotep{Q}) & := & P \equiv Q \\
  match_{\dagger}(\quotep{P},\quotep{Q}) & := & \forall R. P|Q \red^{*} R => R \red^{*} 0 \\
  match_{K}(\quotep{P},\quotep{Q}) & := & K \mbox{ for some context } K
\end{eqnarray*}

$u?(x)P | u!\langle Q \rangle \red P\{\quotep{Q}/x\}$

%We write $\wred$ for $\red^*$, and $P\red$ if $\exists Q $ such that $ P \red Q$.
We write $P\red$ if $\exists Q $ such that $ P \red Q$ and $P\not\red$, otherwise.

\section{Replication}

As mentioned before, it is known that replication (and hence
recursion) can be implemented in a higher-order process algebra
\cite{SangiorgiWalker}. As our first example of calculation with the
machinery thus far presented we give the construction explicitly in
the {\rhoc}.

\begin{eqnarray}
	D_{x} & := & \prefix{x}{y}{(\binpar{\outputp{x}{y}}{@{y}})} \nonumber\\
	\bangp_{x}{P} & := & \binpar{{x}!\langle{\binpar{D_{x}}{P}}\rangle}{D_{x}} \nonumber
\end{eqnarray}

\begin{eqnarray}
	\bangp_{x}{P} & & \nonumber\\
	=
	& {x}!\langle{(\prefix{x}{y}{(\outputp{x}{y} | @{y})) | P}}\rangle 
	      | \prefix{x}{y}{(\outputp{x}{y} | @{y})} & \nonumber\\
	\red
	& (\outputp{x}{y} | @{y})\substn{\quotep{(\prefix{x}{y}{(@{y} | \outputp{x}{y})) | P}}}{y} & \nonumber\\
	=
	& \outputp{x}{\quotep{(\prefix{x}{y}{(\outputp{x}{y} | @{y})) | P}}}
	  | {(\prefix{x}{y}{(\outputp{x}{y} | @{y})) | P}} & \nonumber\\
	\red
	& \ldots & \nonumber\\
	\red^*
	& P | P | \ldots & \nonumber
\end{eqnarray}

Of course, this encoding, as an implementation, runs away, unfolding
$\bangp{P}$ eagerly. A lazier and more implementable replication
operator, restricted to input-guarded processes, may be obtained as follows.

\begin{eqnarray}
\bangp{\prefix{u}{v}{P}} 
	:= 
	\binpar{\lift{x}{\prefix{u}{v}{(\binpar{D(x)}{P})}}}{D(x)} \nonumber
\end{eqnarray}

\begin{remark}
  Note that the lazier definition still does not deal with summation
  or mixed summation (i.e. sums over input and output). The reader is
  invited to construct definitions of replication that deal with these
  features. 

  Further, the definitions are parameterized in a name, $x$. Can you,
  gentle reader, make a definition that eliminates this parameter and
  guarantees no accidental interaction between the replication
  machinery and the process being replicated -- i.e. no accidental
  sharing of names used by the process to get its work done and the
  name(s) used by the replication to effect copying. This latter
  revision of the definition of replication is crucial to obtaining
  the expected identity $!!P \sim !P$.
\end{remark}

\begin{remark}\label{rem:paradoxical_combinator}
  The reader familiar with the lambda calculus will have noticed the
  similarity between $D$ and the paradoxical combinator.

  [Ed. note: the existence of this seems to suggest we have to be more
  restrictive on the set of processes and names we admit if we are to
  support no-cloning.]
\end{remark}

\subsubsection{Bisimulation}

The computational dynamics gives rise to another kind of equivalence,
the equivalence of computational behavior. As previously mentioned
this is typically captured \emph{via} some form of bisimulation.

% The notion we use in this paper is weak barbed bisimulation
% \cite{milner91polyadicpi}.

The notion we use in this paper is derived from weak barbed
bisimulation \cite{milner91polyadicpi}. 

\begin{definition}
An \emph{observation relation}, $\downarrow_{\mathcal N}$, over a set
of names, $\mathcal N$, is the smallest relation satisfying the rules
below.

\infrule[Out-barb]{y \in {\mathcal N}, \; x \nameeq y}
		  {\outputp{x}{v} \downarrow_{\mathcal N} x}
\infrule[Par-barb]{\mbox{$P\downarrow_{\mathcal N} x$ or $Q\downarrow_{\mathcal N} x$}}
		  {\binpar{P}{Q} \downarrow_{\mathcal N} x}

We write $P \Downarrow_{\mathcal N} x$ if there is $Q$ such that 
$P \wred Q$ and $Q \downarrow_{\mathcal N} x$.
\end{definition}

\begin{definition}
%\label{def.bbisim}
An  ${\mathcal N}$-\emph{barbed bisimulation} over a set of names, ${\mathcal N}$, is a symmetric binary relation 
${\mathcal S}_{\mathcal N}$ between agents such that $P\rel{S}_{\mathcal N}Q$ implies:
\begin{enumerate}
\item If $P \red P'$ then $Q \wred Q'$ and $P'\rel{S}_{\mathcal N} Q'$.
\item If $P\downarrow_{\mathcal N} x$, then $Q\Downarrow_{\mathcal N} x$.
\end{enumerate}
$P$ is ${\mathcal N}$-barbed bisimilar to $Q$, written
$P \wbbisim_{\mathcal N} Q$, if $P \rel{S}_{\mathcal N} Q$ for some ${\mathcal N}$-barbed bisimulation ${\mathcal S}_{\mathcal N}$.
\end{definition}

$\mathcal{R} \subseteq \pi \times \pi$

$P \mathcal{R} Q => \forall P'. P \red P' \Rightarrow \exists Q'. Q \red Q', P' \mathcal{R} Q'$

$P \vdash x \Rightarrow Q \vdash x$

\begin{mathpar}
  \inferrule*[lab=Out-barb]{x \nameeq y}{{y}!\langle{Q}\rangle \vdash x}
  \and
  \inferrule*[lab=Par-barb]{\mbox{$P\vdash x$ or $Q\vdash x$}}{\binpar{P}{Q} \vdash x}
\end{mathpar}

\subsubsection{Contexts}

One of the principle advantages of computational calculi like the
$\pi$-calculus is a well-defined notion of context,
contextual-equivalence and a correlation between
contextual-equivalence and notions of bisimulation. The notion of
context allows the decomposition of a process into (sub-)process and
its syntactic environment, its context. Thus, a context may be
thought of as a process with a ``hole'' (written $\Box$) in it. The
application of a context $M$ to a process $P$, written $M[P]$, is
tantamount to filling the hole in $M$ with $P$. In this paper we do
not need the full weight of this theory, but do make use of the notion
of context in the proof the main theorem. 

\begin{mathpar}
  \inferrule* [lab=summation] {} {{M_{M},M_{N}} \bc \Box \;|\; x.M_{A} \;|\; M_{M}+M_{N}}
  \and
  \inferrule* [lab=agent] {} {{M_{A}} \bc (\vec{x})M_{P} \;| \; \clift{P_0,\ldots,M_{P},\ldots,P_N}}
  \and \\
  \inferrule* [lab=process] {} {{M_{P}} \bc M_{N} \;| \;P|M_{P} }
\end{mathpar} 

\begin{mathpar}
  \inferrule* [lab=sychronization] {} {M_{N} \bc \Box \;|\; x?M_{F} \;|\; x!M_{C}}
  \and
  \inferrule* [lab=abstraction] {} {{M_{F}} \bc (x)M_{P} }
  \and
  \inferrule* [lab=concretion] {} {{M_{C}} \bc \langle M_{P} \rangle }
  \and \\
  \inferrule* [lab=process] {} {{M_{P}} \bc M_{N} \;| \;P|M_{P} }
\end{mathpar}

\begin{definition}[contextual application] Given a context $M$, and
  process $P$, we define the \emph{contextual application}, $M[P] :=
  M\{P/\Box\}$. That is, the contextual application of M to P is the
  substitution of $P$ for $\Box$ in $M$.
\end{definition}

$\meaningof{-} : L \to \mathcal{P}(\pi)$

\begin{mathpar}
  \inferrule* [lab=collection] {} {\meaningof{true} = \pi, \and \meaningof{~E} = \pi \setminus \meaningof{E}, \and \meaningof{E_{1} \& E_{2}} = \meaningof{E_{1}} \cap \meaningof{E_{2}}}
\end{mathpar}

\begin{mathpar}
  \inferrule* [lab=structure] {} {\meaningof{0} = \{ P \in \pi | P \equiv 0 \}, \and \\ \meaningof{E_1 | E_2} = \{ P \in \pi | P \equiv P_{1} | P_{2}, P_{1} \in \meaningof{E_{1}}, P_{2} \in \meaningof{E_2}\} }
\end{mathpar}

\begin{mathpar}
 \inferrule* [lab=behavior] {} {\meaningof{\langle a?b \rangle E} = \{ P \in \pi | P \equiv Q | u?(y)P', \\ \and \\\\ \and \\ \;\;\; u \in \meaningof{a}, \forall z.P'\{z/y\} \in \meaningof{E\{z/b\}}\}, \and \\ \meaningof{a!E} = \{ P \in \pi | P \equiv Q | x!\langle P' \rangle, x \in \meaningof{a} P' \in \meaningof{E}\} }
\end{mathpar}

\begin{mathpar}
 \inferrule* [lab=nominal] {} {\meaningof{\quotep{E}} = \{ \quotep{P} \in \quotep{\pi} | P \in \meaningof{E} \}, \and \meaningof{\quotep{P}} = \{ \quotep{Q} \in \quotep{\pi} | P \equiv Q \} \and \\ \meaningof{@\quotep{E}} = \{ P \in \pi | P \equiv @x, x \in \meaningof{E} \}}
\end{mathpar}

\begin{eqnarray*}
  \\
  \meaningof{-} : TS \to ST
\end{eqnarray*}

\begin{eqnarray*}
  \\
  L : TS \to ST
\end{eqnarray*}

\begin{eqnarray*}
  \\
  P \models E \iff P \in \meaningof{E}
\end{eqnarray*}

\begin{eqnarray*}
  P \approx_{L} Q \iff \forall E \in L. P \models E \iff Q \models E
\end{eqnarray*}

\begin{eqnarray*}
  P \approx_{K} Q
\end{eqnarray*}

\begin{eqnarray*}
  P \approx Q
\end{eqnarray*}

$\approx_{K} = \approx = \approx_{L}$

\subsubsection{Contextual duality}

Note that contexts extend the quotation operation to a family of
operations from processes to names. Given a context, $M$, we can
define a \emph{nominal context}, $\quotep{M}$ by $\quotep{M}[P] :=
\quotep{M[P]}$. To foreshadow what is to come we observe that these
operations enjoy a duality with processes very much like the duality
between vectors and maps from vectors to scalars.

Further, because the calculus is essentially higher-order, we have a
correspondence between contexts and processes. More specifically,
given a name $x$ and a context $M$ we can construct $M^{*}_{x}$ such
that 

\begin{mathpar}
  M^{*}_{x} | \lift{x}{P} \red M[P]
\end{mathpar}

namely,

\begin{mathpar}
  M^{*}_{x} := x?(u).M[\dropn{u}]
\end{mathpar}

The dependence of $M^{*}_{x}$ on a name makes it an abstraction, 

\begin{mathpar}
  M^{*} := (x)x?(u).M[\dropn{u}]
\end{mathpar}

\subsection{Additional notation}

It will sometimes be convenient to denote the process a name
quotes. We already have the notation $x = \quotep{P}$, but it will be
convenient to introduce an alternate notation, $\procn{x}$, when we
want to emphasize the connection to the use of the name. Note that, by
virtue of name equivalence, $\quotep{\procn{x}} \nameeq x$; so, the
notation is consistent with previous definitions.

Further, because names have structure it is possible to effect
substitutions on the basis of that structure. This means we need to
upgrade our notation for substitutions, which we accomplish by
adapting comprehension notation. Thus,

\begin{mathpar}
  P\{ y / x : x \in S \}
\end{mathpar}

is interpreted to mean the process derived from P by replacing (in a
capture-avoiding manner) each occurrence of $x$ in $S$ by $y$. For example,

\begin{mathpar}
  P\{ \quotep{\procn{x}|\procn{x}} / x : x \in \freenames{P} \}
\end{mathpar}

will replace each (occurrence) of a free name $x$ in $P$ by
$\quotep{\procn{x}|\procn{x}}$.

Also, we will avail ourselves of the notation $x^{L}$ and $x^{R}$ to
denote injections of a name into disjoint copies of the name
space. There are numerous ways to accomplish this. One example can be
found in \cite{MeredithR05}. This notation overloads to vectors of
names: $\vec{x}^{\pi} := (x_{i}^{\pi} \; : \; 0 \leq i < |\vec{x}| )$ where $\pi \in \{L,R\}$.

We also use $P^{\Box} := P|\Box$.

In \cite{MeredithR05} an interpretation of the new operator is
given. It turns out that there are several possible interpretations
all enjoying the requisite algebraic properties of the operator (see
\cite{milner91polyadicpi}). We will therefore make liberal use of
$(\nu\; \vec{x})P$.

% subsection the_syntax_and_semantics_of_the_notation_system (end)   

\input{qm2pi.qmops} 

\input{qm2pi.sterngerlach} 

\input{qm2pi.metric} 

% section concurrent_process_calculi (end)

%\input{qm2pi.proofsketch}

% section proof sketch (end)

%\input{qm2pi.slviaknots} 

% section spatial logic via knots (end)

\input{qm2pi.conclusion}

% section conclusion (end)

%\input{qm2pi.dtcodes} 

% section wiring algorithm (end)

\input{qm2pi.ack} 

% section acknowledgments (end)

\newpage


\bibliographystyle{plain}   
\bibliography{../../biblios/main.bib}

\input{qm2pi.rhodetails}

\end{document}

 

%\documentclass[12pt]{llncs}
%\documentclass{jktr}

\usepackage[pdftex]{hyperref}                   
\usepackage {listings}
\usepackage {mathpartir}
\usepackage{bcprules}
%\usepackage{listings}
                       
\usepackage{graphicx} 
%\usepackage[margins=2.5cm,nohead,nofoot]{geometry}
%\usepackage{geometry}
\usepackage{amsfonts}
\usepackage{amstext}
\usepackage{latexsym}
\usepackage{amssymb}
\usepackage{color}


%\include{myPreamble}
\include{qm2pi.local} 

%\ifpdf
%\usepackage[pdftex]{graphicx}
%\else
%\usepackage{graphicx}
%\fi

 % \ifpdf
%  \usepackage{pdfsync}
%  \if


%\title{Brief Article}
%\author{David F. Snyder}
%\author{L.G. Meredith}

%\address{Dept. of Math., Texas State University--San Marcos, San Marcos, TX 78666}
       
\pagestyle{empty}


\begin{document}

\lstset{language=[Objective]Caml,frame=shadowbox}

\input{qm2pi.front}

% section front matter (end)

\input{qm2pi.intro} 
 
% section introduction (end)

% \input{qm2pi.knotations} 

% section notation (end)

\input{qm2pi.process.calculi} 

% section concurrent_process_calculi_and_spatial_logics_ (end)
    
%\input{qm2pi.knots2pi} 

%\input{qm2pi.trefoil} 

%\input{qm2pi.mainthm} 

% subsection basic_interpretation (end)

%\input{qm2pi.rho.presentation} 
\subsection{The syntax and semantics of the notation system}\label{sub:the_syntax_and_semantics_of_the_notation_system} % (fold)

We now summarize a technical presentation of the calculus that
embodies our theory of dynamics. The typical presentation of such a
calculus follows the style of giving generators and relations on
them. The grammar, below, describing term constructors, freely
generates the set of processes, $\Proc$. This set is then quotiented
by a relation known as structural congruence and it is over this set
that the notion of dynamics is expressed. This presentation is
essentially that of \cite{MeredithR05} with the addition of
polyadicity and summation. For readability we have relegated some of
the technical subtleties to an appendix.

\subsubsection{Process grammar}\label{subsub:process_grammar}

\begin{mathpar}
  \inferrule* [lab=synchronization] {} {{M} \bc \pzero \;|\; x?F \;|\; x!C }
  \and
  \inferrule* [lab=abstraction] {} {{F} \bc (x)P}
  \and
  \inferrule* [lab=concretion] {} {{C} \bc \langle Q \rangle}
  \and
  \inferrule* [lab=process] {} {{P,Q} \bc M \;| \;P|Q \;|\; @{x}}
  \and
  \inferrule* [lab=name] {} {{x} \bc \quotep{P}}
\end{mathpar} 

Note that $\vec{x}$ (resp. $\vec{P}$) denotes a vector of names
(resp. processes) of length $|\vec{x}|$ (resp. $|\vec{P}|$). We adopt
the following useful abbreviations.

\begin{mathpar}
   x?(\vec{y}).P := x.(\vec{y})P \and  x\clift{\vec{P}} := x.\clift{\vec{P}}
   \and x!(y) := \lift{x}{\dropn{y}}
   \and \Pi_{i=0}^{n-1}P_i := P_0 | \ldots | P_{n-1}
\end{mathpar}

\subsubsection{Structural congruence}

\paragraph{Free and bound names and alpha-equivalence.} At the
core of structural equivalence is alpha-equivalence which identifies
process that are the same up to a change of variable. Formally, we
recognize the distinction between free and bound names. The free names
of a process, $\freenames{P}$, may be calculated recursively as
follows:

\begin{mathpar}
\freenames{\pzero} := \emptyset
  \and \\
  \freenames{x?(y).P} := \{ x \} \cup (\freenames{P} \setminus \{ y \})
  \and 
  \freenames{x!\langle P \rangle} := \{ x \} \cup \{ P \} 
  \and \\
  \freenames{P|Q} := \freenames{P} \cup \freenames{Q}
  \and \\
  \freenames{@{x}} := \{ x \}
\end{mathpar}

$\pi$
$\quotep{\pi}$

$\freenames{-} : \pi \to \mathcal{P}(\quotep{\pi})$

\begin{eqnarray*}
  \freenames{\pzero} & := & \emptyset \\
  \freenames{x?(y).P} & := & \{ x \} \cup (\freenames{P} \setminus \{ y \}) \\
  \freenames{x!\langle P \rangle} & := & \{ x \} \cup \{ P \} \\
  \freenames{P|Q} & := & \freenames{P} \cup \freenames{Q} \\
  \freenames{\dropn{x}} & := & \{ x \}
\end{eqnarray*}

The bound names of a process, $\boundnames{P}$, are those names occurring in $P$
that are not free. For example, in $x?(y).0$, the name $x$ is free, while $y$ is bound.

\begin{mathpar}
  \inferrule* [lab=monoidal-laws] {} { P|Q \equiv Q|P \and P|0 \equiv P \and P|(Q|R) \equiv (P|Q)|R }
\end{mathpar}

\begin{mathpar}
  \inferrule* [lab=alpha-equivalence] {} { (x)P \equiv (y)P\{y/x\} \and y \not\in \freenames{P} }
\end{mathpar}

\begin{definition}
Then two processes, $P,Q$, are alpha-equivalent if $P = Q\{\vec{y}/\vec{x}\}$ for
some $\vec{x} \in \boundnames{Q},\vec{y} \in \boundnames{P}$, where $Q\{\vec{y}/\vec{x}\}$
denotes the capture-avoiding substitution of $\vec{y}$ for $\vec{x}$ in $Q$.
\end{definition}

\begin{definition}
  The {\em structural congruence} \cite{SangiorgiWalker} , $\equiv$,
  between processes is the least congruence containing
  alpha-equivalence, satisfying the abelian monoid laws
  (associativity, commutativity and $\pzero$ as identity) for parallel
  composition $|$ and for summation $+$.
\end{definition}

\subsection{Name equivalence}

We take name equivalence, written $\nameeq$, to be the smallest
equivalence relation generated by the following rules.

\begin{mathpar}
\inferrule*[lab=Quote-drop]
{ }
{ \quotep{@{x}} \nameeq x }

\inferrule*[lab=Struct-equiv]
{ P \scong Q }
{ \quotep{P} \nameeq \quotep{Q} }
\end{mathpar}

The astute reader will have noticed that the mutual recursion of names
and processes imposes a mutual recursion on alpha-equivalence and
structural equivalence via name-equivalence. Fortunately, all of this
works out pleasantly and we may calculate in the natural way, free of
concern. The reader interested in the details is referred to the
appendix \ref{appendix:rho_details}.

\subsection{Substitution}

We use $\Proc$ for the set of processes, $\QProc$ for the set of
names, and $\id{\{}\vec{y} / \vec{x} \id{\}}$ to denote partial maps,
$s : \QProc \rightarrow \QProc$. A map, $s$ lifts, uniquely, to a map
on process terms, $\widehat{s} : \Proc \rightarrow \Proc$ by the
following equations.

\begin{mathpar}
  (0) \psubstp{Q}{P} := 0 \\
  (R \juxtap S) \psubstp{Q}{P}
  :=    
  (R)\psubstp{Q}{P} \juxtap (S) \psubstp{Q}{P} \\
  (x?(y).R) \psubstp{Q}{P}    
  :=    
  (x)\substp{Q}{P} (z)\concat( (R \psubstn{z}{y}) \psubstp{Q}{P} ) \\
  (\lift{x}{R}) \psubstp{Q}{P}  
  :=
  \lift{(x)\substp{Q}{P}}{ R \psubstp{Q}{P} } \\
%   (\dropn{x})  \psubstp{Q}{P}       
%   := 
%   \left\{ 
%     \begin{array}{ccc} 
%       \dropn{\quotep{Q}} & & x \nameeq \quotep{P} \\
%       \dropn{x} & & otherwise \\
%     \end{array}
%   \right. 
  (\dropn{x})  \psubstp{Q}{P}       
  := 
  \left\{ 
    \begin{array}{ccc} 
      Q & & x \nameeq \quotep{P} \\
      \dropn{x} & & otherwise \\
    \end{array}
  \right.
\end{mathpar}
 

where

\begin{eqnarray}
  (x)\id{\{} \lpquote Q \rpquote / \lpquote P \rpquote \id{\}}            = 
  \left\{ 
    \begin{array}{ccc}
      \lpquote Q \rpquote & & x \nameeq \lpquote P \rpquote \\
      x & & otherwise \\
    \end{array}
  \right. \nonumber
\end{eqnarray}

and $z$ is chosen distinct from $\quotep{P}$, $\quotep{Q}$, the free
names in $Q$, and all the names in $R$. Our $\alpha$-equivalence will
be built in the standard way from this substitution.

\begin{remark}\label{rem:no_self_referential_names}
  One consequence of these definitions is that $\forall P. \quotep{P}
  \not\in \freenames{P}$.
\end{remark}

\subsection{ Dynamic quote: an example }

Anticipating something of what's to come, consider applying the
substitution, $\widehat{\id{\{}u / z \id{\}}}$, to the following pair
of processes, $\lift{w}{y!(z)}$ and $w[ \lpquote y!(z) \rpquote ]$.

\begin{eqnarray}
	\lift{w}{y!(z)}\widehat{\id{\{}u / z \id{\}}}
		& = &
		\lift{w}{y!(u)} \nonumber\\
	w[ \lpquote y!(z) \rpquote ] \widehat{ \id{\{}u / z \id{\}} }
		& = &
		w[ \lpquote y!(z) \rpquote ] \nonumber
\end{eqnarray}

Because the body of the process between quotes is impervious to
substitution, we get radically different answers. In fact, by
examining the first process in an input context,
e.g. $x?(z).\lift{w}{y!(z)}$, we see that the process under the lift
operator may be shaped by prefixed inputs binding a name inside it. In
this sense, the lift operator will be seen as a way to dynamically
construct processes before reifying them as names.

Finally equipped with these standard features we can present the
dynamics of the calculus.

\subsubsection{Operational semantics} 

Finally, we introduce the computational dynamics. What marks these
algebras as distinct from other more traditionally studied algebraic
structures, e.g. vector spaces or polynomial rings, is the manner in
which dynamics is captured. In traditional structures, dynamics is typically
expressed through morphisms between such structures, as in linear maps
between vector spaces or morphisms between rings. In algebras
associated with the semantics of computation, the dynamics is
expressed as part of the algebraic structure itself, through a
reduction reduction relation typically denoted by $\red$. Below, we
give a recursive presentation of this relation for the calculus used
in the encoding.

$\red \subseteq \pi \times \pi$
$\red : \pi \to \mathcal{P}(\pi)$

\begin{mathpar}
  \inferrule* [lab=Comm] { \textsf{match}( x_{src}, x_{trgt} ) } { x_{trgt}?(y)P \; | \; x_{src}!\langle {Q} \rangle \red P\{\quotep{Q}/y}\} }
  \and \\
  \inferrule* [lab=Par] {{P} \red {P}'} {{{P} | {Q}} \red {{P}' | {Q}}}
  \and
  \inferrule* [lab=Equiv]{{{P} \scong {P}'} \andalso {{P}' \red {Q}'} \andalso {{Q}' \scong {Q}}}{{P} \red {Q}}
\end{mathpar}

\begin{eqnarray*}
  match_{\equiv} (\quotep{P},\quotep{Q}) & := & P \equiv Q \\
  match_{\dagger}(\quotep{P},\quotep{Q}) & := & \forall R. P|Q \red^{*} R => R \red^{*} 0 \\
  match_{K}(\quotep{P},\quotep{Q}) & := & K \mbox{ for some context } K
\end{eqnarray*}

$u?(x)P | u!\langle Q \rangle \red P\{\quotep{Q}/x\}$

%We write $\wred$ for $\red^*$, and $P\red$ if $\exists Q $ such that $ P \red Q$.
We write $P\red$ if $\exists Q $ such that $ P \red Q$ and $P\not\red$, otherwise.

\section{Replication}

As mentioned before, it is known that replication (and hence
recursion) can be implemented in a higher-order process algebra
\cite{SangiorgiWalker}. As our first example of calculation with the
machinery thus far presented we give the construction explicitly in
the {\rhoc}.

\begin{eqnarray}
	D_{x} & := & \prefix{x}{y}{(\binpar{\outputp{x}{y}}{@{y}})} \nonumber\\
	\bangp_{x}{P} & := & \binpar{{x}!\langle{\binpar{D_{x}}{P}}\rangle}{D_{x}} \nonumber
\end{eqnarray}

\begin{eqnarray}
	\bangp_{x}{P} & & \nonumber\\
	=
	& {x}!\langle{(\prefix{x}{y}{(\outputp{x}{y} | @{y})) | P}}\rangle 
	      | \prefix{x}{y}{(\outputp{x}{y} | @{y})} & \nonumber\\
	\red
	& (\outputp{x}{y} | @{y})\substn{\quotep{(\prefix{x}{y}{(@{y} | \outputp{x}{y})) | P}}}{y} & \nonumber\\
	=
	& \outputp{x}{\quotep{(\prefix{x}{y}{(\outputp{x}{y} | @{y})) | P}}}
	  | {(\prefix{x}{y}{(\outputp{x}{y} | @{y})) | P}} & \nonumber\\
	\red
	& \ldots & \nonumber\\
	\red^*
	& P | P | \ldots & \nonumber
\end{eqnarray}

Of course, this encoding, as an implementation, runs away, unfolding
$\bangp{P}$ eagerly. A lazier and more implementable replication
operator, restricted to input-guarded processes, may be obtained as follows.

\begin{eqnarray}
\bangp{\prefix{u}{v}{P}} 
	:= 
	\binpar{\lift{x}{\prefix{u}{v}{(\binpar{D(x)}{P})}}}{D(x)} \nonumber
\end{eqnarray}

\begin{remark}
  Note that the lazier definition still does not deal with summation
  or mixed summation (i.e. sums over input and output). The reader is
  invited to construct definitions of replication that deal with these
  features. 

  Further, the definitions are parameterized in a name, $x$. Can you,
  gentle reader, make a definition that eliminates this parameter and
  guarantees no accidental interaction between the replication
  machinery and the process being replicated -- i.e. no accidental
  sharing of names used by the process to get its work done and the
  name(s) used by the replication to effect copying. This latter
  revision of the definition of replication is crucial to obtaining
  the expected identity $!!P \sim !P$.
\end{remark}

\begin{remark}\label{rem:paradoxical_combinator}
  The reader familiar with the lambda calculus will have noticed the
  similarity between $D$ and the paradoxical combinator.

  [Ed. note: the existence of this seems to suggest we have to be more
  restrictive on the set of processes and names we admit if we are to
  support no-cloning.]
\end{remark}

\subsubsection{Bisimulation}

The computational dynamics gives rise to another kind of equivalence,
the equivalence of computational behavior. As previously mentioned
this is typically captured \emph{via} some form of bisimulation.

% The notion we use in this paper is weak barbed bisimulation
% \cite{milner91polyadicpi}.

The notion we use in this paper is derived from weak barbed
bisimulation \cite{milner91polyadicpi}. 

\begin{definition}
An \emph{observation relation}, $\downarrow_{\mathcal N}$, over a set
of names, $\mathcal N$, is the smallest relation satisfying the rules
below.

\infrule[Out-barb]{y \in {\mathcal N}, \; x \nameeq y}
		  {\outputp{x}{v} \downarrow_{\mathcal N} x}
\infrule[Par-barb]{\mbox{$P\downarrow_{\mathcal N} x$ or $Q\downarrow_{\mathcal N} x$}}
		  {\binpar{P}{Q} \downarrow_{\mathcal N} x}

We write $P \Downarrow_{\mathcal N} x$ if there is $Q$ such that 
$P \wred Q$ and $Q \downarrow_{\mathcal N} x$.
\end{definition}

\begin{definition}
%\label{def.bbisim}
An  ${\mathcal N}$-\emph{barbed bisimulation} over a set of names, ${\mathcal N}$, is a symmetric binary relation 
${\mathcal S}_{\mathcal N}$ between agents such that $P\rel{S}_{\mathcal N}Q$ implies:
\begin{enumerate}
\item If $P \red P'$ then $Q \wred Q'$ and $P'\rel{S}_{\mathcal N} Q'$.
\item If $P\downarrow_{\mathcal N} x$, then $Q\Downarrow_{\mathcal N} x$.
\end{enumerate}
$P$ is ${\mathcal N}$-barbed bisimilar to $Q$, written
$P \wbbisim_{\mathcal N} Q$, if $P \rel{S}_{\mathcal N} Q$ for some ${\mathcal N}$-barbed bisimulation ${\mathcal S}_{\mathcal N}$.
\end{definition}

$\mathcal{R} \subseteq \pi \times \pi$

$P \mathcal{R} Q => \forall P'. P \red P' \Rightarrow \exists Q'. Q \red Q', P' \mathcal{R} Q'$

$P \vdash x \Rightarrow Q \vdash x$

\begin{mathpar}
  \inferrule*[lab=Out-barb]{x \nameeq y}{{y}!\langle{Q}\rangle \vdash x}
  \and
  \inferrule*[lab=Par-barb]{\mbox{$P\vdash x$ or $Q\vdash x$}}{\binpar{P}{Q} \vdash x}
\end{mathpar}

\subsubsection{Contexts}

One of the principle advantages of computational calculi like the
$\pi$-calculus is a well-defined notion of context,
contextual-equivalence and a correlation between
contextual-equivalence and notions of bisimulation. The notion of
context allows the decomposition of a process into (sub-)process and
its syntactic environment, its context. Thus, a context may be
thought of as a process with a ``hole'' (written $\Box$) in it. The
application of a context $M$ to a process $P$, written $M[P]$, is
tantamount to filling the hole in $M$ with $P$. In this paper we do
not need the full weight of this theory, but do make use of the notion
of context in the proof the main theorem. 

\begin{mathpar}
  \inferrule* [lab=summation] {} {{M_{M},M_{N}} \bc \Box \;|\; x.M_{A} \;|\; M_{M}+M_{N}}
  \and
  \inferrule* [lab=agent] {} {{M_{A}} \bc (\vec{x})M_{P} \;| \; \clift{P_0,\ldots,M_{P},\ldots,P_N}}
  \and \\
  \inferrule* [lab=process] {} {{M_{P}} \bc M_{N} \;| \;P|M_{P} }
\end{mathpar} 

\begin{mathpar}
  \inferrule* [lab=sychronization] {} {M_{N} \bc \Box \;|\; x?M_{F} \;|\; x!M_{C}}
  \and
  \inferrule* [lab=abstraction] {} {{M_{F}} \bc (x)M_{P} }
  \and
  \inferrule* [lab=concretion] {} {{M_{C}} \bc \langle M_{P} \rangle }
  \and \\
  \inferrule* [lab=process] {} {{M_{P}} \bc M_{N} \;| \;P|M_{P} }
\end{mathpar}

\begin{definition}[contextual application] Given a context $M$, and
  process $P$, we define the \emph{contextual application}, $M[P] :=
  M\{P/\Box\}$. That is, the contextual application of M to P is the
  substitution of $P$ for $\Box$ in $M$.
\end{definition}

$\meaningof{-} : L \to \mathcal{P}(\pi)$

\begin{mathpar}
  \inferrule* [lab=collection] {} {\meaningof{true} = \pi, \and \meaningof{~E} = \pi \setminus \meaningof{E}, \and \meaningof{E_{1} \& E_{2}} = \meaningof{E_{1}} \cap \meaningof{E_{2}}}
\end{mathpar}

\begin{mathpar}
  \inferrule* [lab=structure] {} {\meaningof{0} = \{ P \in \pi | P \equiv 0 \}, \and \\ \meaningof{E_1 | E_2} = \{ P \in \pi | P \equiv P_{1} | P_{2}, P_{1} \in \meaningof{E_{1}}, P_{2} \in \meaningof{E_2}\} }
\end{mathpar}

\begin{mathpar}
 \inferrule* [lab=behavior] {} {\meaningof{\langle a?b \rangle E} = \{ P \in \pi | P \equiv Q | u?(y)P', \\ \and \\\\ \and \\ \;\;\; u \in \meaningof{a}, \forall z.P'\{z/y\} \in \meaningof{E\{z/b\}}\}, \and \\ \meaningof{a!E} = \{ P \in \pi | P \equiv Q | x!\langle P' \rangle, x \in \meaningof{a} P' \in \meaningof{E}\} }
\end{mathpar}

\begin{mathpar}
 \inferrule* [lab=nominal] {} {\meaningof{\quotep{E}} = \{ \quotep{P} \in \quotep{\pi} | P \in \meaningof{E} \}, \and \meaningof{\quotep{P}} = \{ \quotep{Q} \in \quotep{\pi} | P \equiv Q \} \and \\ \meaningof{@\quotep{E}} = \{ P \in \pi | P \equiv @x, x \in \meaningof{E} \}}
\end{mathpar}

\begin{eqnarray*}
  \\
  \meaningof{-} : TS \to ST
\end{eqnarray*}

\begin{eqnarray*}
  \\
  L : TS \to ST
\end{eqnarray*}

\begin{eqnarray*}
  \\
  P \models E \iff P \in \meaningof{E}
\end{eqnarray*}

\begin{eqnarray*}
  P \approx_{L} Q \iff \forall E \in L. P \models E \iff Q \models E
\end{eqnarray*}

\begin{eqnarray*}
  P \approx_{K} Q
\end{eqnarray*}

\begin{eqnarray*}
  P \approx Q
\end{eqnarray*}

$\approx_{K} = \approx = \approx_{L}$

\subsubsection{Contextual duality}

Note that contexts extend the quotation operation to a family of
operations from processes to names. Given a context, $M$, we can
define a \emph{nominal context}, $\quotep{M}$ by $\quotep{M}[P] :=
\quotep{M[P]}$. To foreshadow what is to come we observe that these
operations enjoy a duality with processes very much like the duality
between vectors and maps from vectors to scalars.

Further, because the calculus is essentially higher-order, we have a
correspondence between contexts and processes. More specifically,
given a name $x$ and a context $M$ we can construct $M^{*}_{x}$ such
that 

\begin{mathpar}
  M^{*}_{x} | \lift{x}{P} \red M[P]
\end{mathpar}

namely,

\begin{mathpar}
  M^{*}_{x} := x?(u).M[\dropn{u}]
\end{mathpar}

The dependence of $M^{*}_{x}$ on a name makes it an abstraction, 

\begin{mathpar}
  M^{*} := (x)x?(u).M[\dropn{u}]
\end{mathpar}

\subsection{Additional notation}

It will sometimes be convenient to denote the process a name
quotes. We already have the notation $x = \quotep{P}$, but it will be
convenient to introduce an alternate notation, $\procn{x}$, when we
want to emphasize the connection to the use of the name. Note that, by
virtue of name equivalence, $\quotep{\procn{x}} \nameeq x$; so, the
notation is consistent with previous definitions.

Further, because names have structure it is possible to effect
substitutions on the basis of that structure. This means we need to
upgrade our notation for substitutions, which we accomplish by
adapting comprehension notation. Thus,

\begin{mathpar}
  P\{ y / x : x \in S \}
\end{mathpar}

is interpreted to mean the process derived from P by replacing (in a
capture-avoiding manner) each occurrence of $x$ in $S$ by $y$. For example,

\begin{mathpar}
  P\{ \quotep{\procn{x}|\procn{x}} / x : x \in \freenames{P} \}
\end{mathpar}

will replace each (occurrence) of a free name $x$ in $P$ by
$\quotep{\procn{x}|\procn{x}}$.

Also, we will avail ourselves of the notation $x^{L}$ and $x^{R}$ to
denote injections of a name into disjoint copies of the name
space. There are numerous ways to accomplish this. One example can be
found in \cite{MeredithR05}. This notation overloads to vectors of
names: $\vec{x}^{\pi} := (x_{i}^{\pi} \; : \; 0 \leq i < |\vec{x}| )$ where $\pi \in \{L,R\}$.

We also use $P^{\Box} := P|\Box$.

In \cite{MeredithR05} an interpretation of the new operator is
given. It turns out that there are several possible interpretations
all enjoying the requisite algebraic properties of the operator (see
\cite{milner91polyadicpi}). We will therefore make liberal use of
$(\nu\; \vec{x})P$.

% subsection the_syntax_and_semantics_of_the_notation_system (end)   

\input{qm2pi.qmops} 

\input{qm2pi.sterngerlach} 

\input{qm2pi.metric} 

% section concurrent_process_calculi (end)

%\input{qm2pi.proofsketch}

% section proof sketch (end)

%\input{qm2pi.slviaknots} 

% section spatial logic via knots (end)

\input{qm2pi.conclusion}

% section conclusion (end)

%\input{qm2pi.dtcodes} 

% section wiring algorithm (end)

\input{qm2pi.ack} 

% section acknowledgments (end)

\newpage


\bibliographystyle{plain}   
\bibliography{../../biblios/main.bib}

\input{qm2pi.rhodetails}

\end{document}

 

% subsection basic_interpretation (end)

%\input{qm2pi.rho.presentation} 
\subsection{The syntax and semantics of the notation system}\label{sub:the_syntax_and_semantics_of_the_notation_system} % (fold)

We now summarize a technical presentation of the calculus that
embodies our theory of dynamics. The typical presentation of such a
calculus follows the style of giving generators and relations on
them. The grammar, below, describing term constructors, freely
generates the set of processes, $\Proc$. This set is then quotiented
by a relation known as structural congruence and it is over this set
that the notion of dynamics is expressed. This presentation is
essentially that of \cite{MeredithR05} with the addition of
polyadicity and summation. For readability we have relegated some of
the technical subtleties to an appendix.

\subsubsection{Process grammar}\label{subsub:process_grammar}

\begin{mathpar}
  \inferrule* [lab=synchronization] {} {{M} \bc \pzero \;|\; x?F \;|\; x!C }
  \and
  \inferrule* [lab=abstraction] {} {{F} \bc (x)P}
  \and
  \inferrule* [lab=concretion] {} {{C} \bc \langle Q \rangle}
  \and
  \inferrule* [lab=process] {} {{P,Q} \bc M \;| \;P|Q \;|\; @{x}}
  \and
  \inferrule* [lab=name] {} {{x} \bc \quotep{P}}
\end{mathpar} 

Note that $\vec{x}$ (resp. $\vec{P}$) denotes a vector of names
(resp. processes) of length $|\vec{x}|$ (resp. $|\vec{P}|$). We adopt
the following useful abbreviations.

\begin{mathpar}
   x?(\vec{y}).P := x.(\vec{y})P \and  x\clift{\vec{P}} := x.\clift{\vec{P}}
   \and x!(y) := \lift{x}{\dropn{y}}
   \and \Pi_{i=0}^{n-1}P_i := P_0 | \ldots | P_{n-1}
\end{mathpar}

\subsubsection{Structural congruence}

\paragraph{Free and bound names and alpha-equivalence.} At the
core of structural equivalence is alpha-equivalence which identifies
process that are the same up to a change of variable. Formally, we
recognize the distinction between free and bound names. The free names
of a process, $\freenames{P}$, may be calculated recursively as
follows:

\begin{mathpar}
\freenames{\pzero} := \emptyset
  \and \\
  \freenames{x?(y).P} := \{ x \} \cup (\freenames{P} \setminus \{ y \})
  \and 
  \freenames{x!\langle P \rangle} := \{ x \} \cup \{ P \} 
  \and \\
  \freenames{P|Q} := \freenames{P} \cup \freenames{Q}
  \and \\
  \freenames{@{x}} := \{ x \}
\end{mathpar}

$\pi$
$\quotep{\pi}$

$\freenames{-} : \pi \to \mathcal{P}(\quotep{\pi})$

\begin{eqnarray*}
  \freenames{\pzero} & := & \emptyset \\
  \freenames{x?(y).P} & := & \{ x \} \cup (\freenames{P} \setminus \{ y \}) \\
  \freenames{x!\langle P \rangle} & := & \{ x \} \cup \{ P \} \\
  \freenames{P|Q} & := & \freenames{P} \cup \freenames{Q} \\
  \freenames{\dropn{x}} & := & \{ x \}
\end{eqnarray*}

The bound names of a process, $\boundnames{P}$, are those names occurring in $P$
that are not free. For example, in $x?(y).0$, the name $x$ is free, while $y$ is bound.

\begin{mathpar}
  \inferrule* [lab=monoidal-laws] {} { P|Q \equiv Q|P \and P|0 \equiv P \and P|(Q|R) \equiv (P|Q)|R }
\end{mathpar}

\begin{mathpar}
  \inferrule* [lab=alpha-equivalence] {} { (x)P \equiv (y)P\{y/x\} \and y \not\in \freenames{P} }
\end{mathpar}

\begin{definition}
Then two processes, $P,Q$, are alpha-equivalent if $P = Q\{\vec{y}/\vec{x}\}$ for
some $\vec{x} \in \boundnames{Q},\vec{y} \in \boundnames{P}$, where $Q\{\vec{y}/\vec{x}\}$
denotes the capture-avoiding substitution of $\vec{y}$ for $\vec{x}$ in $Q$.
\end{definition}

\begin{definition}
  The {\em structural congruence} \cite{SangiorgiWalker} , $\equiv$,
  between processes is the least congruence containing
  alpha-equivalence, satisfying the abelian monoid laws
  (associativity, commutativity and $\pzero$ as identity) for parallel
  composition $|$ and for summation $+$.
\end{definition}

\subsection{Name equivalence}

We take name equivalence, written $\nameeq$, to be the smallest
equivalence relation generated by the following rules.

\begin{mathpar}
\inferrule*[lab=Quote-drop]
{ }
{ \quotep{@{x}} \nameeq x }

\inferrule*[lab=Struct-equiv]
{ P \scong Q }
{ \quotep{P} \nameeq \quotep{Q} }
\end{mathpar}

The astute reader will have noticed that the mutual recursion of names
and processes imposes a mutual recursion on alpha-equivalence and
structural equivalence via name-equivalence. Fortunately, all of this
works out pleasantly and we may calculate in the natural way, free of
concern. The reader interested in the details is referred to the
appendix \ref{appendix:rho_details}.

\subsection{Substitution}

We use $\Proc$ for the set of processes, $\QProc$ for the set of
names, and $\id{\{}\vec{y} / \vec{x} \id{\}}$ to denote partial maps,
$s : \QProc \rightarrow \QProc$. A map, $s$ lifts, uniquely, to a map
on process terms, $\widehat{s} : \Proc \rightarrow \Proc$ by the
following equations.

\begin{mathpar}
  (0) \psubstp{Q}{P} := 0 \\
  (R \juxtap S) \psubstp{Q}{P}
  :=    
  (R)\psubstp{Q}{P} \juxtap (S) \psubstp{Q}{P} \\
  (x?(y).R) \psubstp{Q}{P}    
  :=    
  (x)\substp{Q}{P} (z)\concat( (R \psubstn{z}{y}) \psubstp{Q}{P} ) \\
  (\lift{x}{R}) \psubstp{Q}{P}  
  :=
  \lift{(x)\substp{Q}{P}}{ R \psubstp{Q}{P} } \\
%   (\dropn{x})  \psubstp{Q}{P}       
%   := 
%   \left\{ 
%     \begin{array}{ccc} 
%       \dropn{\quotep{Q}} & & x \nameeq \quotep{P} \\
%       \dropn{x} & & otherwise \\
%     \end{array}
%   \right. 
  (\dropn{x})  \psubstp{Q}{P}       
  := 
  \left\{ 
    \begin{array}{ccc} 
      Q & & x \nameeq \quotep{P} \\
      \dropn{x} & & otherwise \\
    \end{array}
  \right.
\end{mathpar}
 

where

\begin{eqnarray}
  (x)\id{\{} \lpquote Q \rpquote / \lpquote P \rpquote \id{\}}            = 
  \left\{ 
    \begin{array}{ccc}
      \lpquote Q \rpquote & & x \nameeq \lpquote P \rpquote \\
      x & & otherwise \\
    \end{array}
  \right. \nonumber
\end{eqnarray}

and $z$ is chosen distinct from $\quotep{P}$, $\quotep{Q}$, the free
names in $Q$, and all the names in $R$. Our $\alpha$-equivalence will
be built in the standard way from this substitution.

\begin{remark}\label{rem:no_self_referential_names}
  One consequence of these definitions is that $\forall P. \quotep{P}
  \not\in \freenames{P}$.
\end{remark}

\subsection{ Dynamic quote: an example }

Anticipating something of what's to come, consider applying the
substitution, $\widehat{\id{\{}u / z \id{\}}}$, to the following pair
of processes, $\lift{w}{y!(z)}$ and $w[ \lpquote y!(z) \rpquote ]$.

\begin{eqnarray}
	\lift{w}{y!(z)}\widehat{\id{\{}u / z \id{\}}}
		& = &
		\lift{w}{y!(u)} \nonumber\\
	w[ \lpquote y!(z) \rpquote ] \widehat{ \id{\{}u / z \id{\}} }
		& = &
		w[ \lpquote y!(z) \rpquote ] \nonumber
\end{eqnarray}

Because the body of the process between quotes is impervious to
substitution, we get radically different answers. In fact, by
examining the first process in an input context,
e.g. $x?(z).\lift{w}{y!(z)}$, we see that the process under the lift
operator may be shaped by prefixed inputs binding a name inside it. In
this sense, the lift operator will be seen as a way to dynamically
construct processes before reifying them as names.

Finally equipped with these standard features we can present the
dynamics of the calculus.

\subsubsection{Operational semantics} 

Finally, we introduce the computational dynamics. What marks these
algebras as distinct from other more traditionally studied algebraic
structures, e.g. vector spaces or polynomial rings, is the manner in
which dynamics is captured. In traditional structures, dynamics is typically
expressed through morphisms between such structures, as in linear maps
between vector spaces or morphisms between rings. In algebras
associated with the semantics of computation, the dynamics is
expressed as part of the algebraic structure itself, through a
reduction reduction relation typically denoted by $\red$. Below, we
give a recursive presentation of this relation for the calculus used
in the encoding.

$\red \subseteq \pi \times \pi$
$\red : \pi \to \mathcal{P}(\pi)$

\begin{mathpar}
  \inferrule* [lab=Comm] { \textsf{match}( x_{src}, x_{trgt} ) } { x_{trgt}?(y)P \; | \; x_{src}!\langle {Q} \rangle \red P\{\quotep{Q}/y}\} }
  \and \\
  \inferrule* [lab=Par] {{P} \red {P}'} {{{P} | {Q}} \red {{P}' | {Q}}}
  \and
  \inferrule* [lab=Equiv]{{{P} \scong {P}'} \andalso {{P}' \red {Q}'} \andalso {{Q}' \scong {Q}}}{{P} \red {Q}}
\end{mathpar}

\begin{eqnarray*}
  match_{\equiv} (\quotep{P},\quotep{Q}) & := & P \equiv Q \\
  match_{\dagger}(\quotep{P},\quotep{Q}) & := & \forall R. P|Q \red^{*} R => R \red^{*} 0 \\
  match_{K}(\quotep{P},\quotep{Q}) & := & K \mbox{ for some context } K
\end{eqnarray*}

$u?(x)P | u!\langle Q \rangle \red P\{\quotep{Q}/x\}$

%We write $\wred$ for $\red^*$, and $P\red$ if $\exists Q $ such that $ P \red Q$.
We write $P\red$ if $\exists Q $ such that $ P \red Q$ and $P\not\red$, otherwise.

\section{Replication}

As mentioned before, it is known that replication (and hence
recursion) can be implemented in a higher-order process algebra
\cite{SangiorgiWalker}. As our first example of calculation with the
machinery thus far presented we give the construction explicitly in
the {\rhoc}.

\begin{eqnarray}
	D_{x} & := & \prefix{x}{y}{(\binpar{\outputp{x}{y}}{@{y}})} \nonumber\\
	\bangp_{x}{P} & := & \binpar{{x}!\langle{\binpar{D_{x}}{P}}\rangle}{D_{x}} \nonumber
\end{eqnarray}

\begin{eqnarray}
	\bangp_{x}{P} & & \nonumber\\
	=
	& {x}!\langle{(\prefix{x}{y}{(\outputp{x}{y} | @{y})) | P}}\rangle 
	      | \prefix{x}{y}{(\outputp{x}{y} | @{y})} & \nonumber\\
	\red
	& (\outputp{x}{y} | @{y})\substn{\quotep{(\prefix{x}{y}{(@{y} | \outputp{x}{y})) | P}}}{y} & \nonumber\\
	=
	& \outputp{x}{\quotep{(\prefix{x}{y}{(\outputp{x}{y} | @{y})) | P}}}
	  | {(\prefix{x}{y}{(\outputp{x}{y} | @{y})) | P}} & \nonumber\\
	\red
	& \ldots & \nonumber\\
	\red^*
	& P | P | \ldots & \nonumber
\end{eqnarray}

Of course, this encoding, as an implementation, runs away, unfolding
$\bangp{P}$ eagerly. A lazier and more implementable replication
operator, restricted to input-guarded processes, may be obtained as follows.

\begin{eqnarray}
\bangp{\prefix{u}{v}{P}} 
	:= 
	\binpar{\lift{x}{\prefix{u}{v}{(\binpar{D(x)}{P})}}}{D(x)} \nonumber
\end{eqnarray}

\begin{remark}
  Note that the lazier definition still does not deal with summation
  or mixed summation (i.e. sums over input and output). The reader is
  invited to construct definitions of replication that deal with these
  features. 

  Further, the definitions are parameterized in a name, $x$. Can you,
  gentle reader, make a definition that eliminates this parameter and
  guarantees no accidental interaction between the replication
  machinery and the process being replicated -- i.e. no accidental
  sharing of names used by the process to get its work done and the
  name(s) used by the replication to effect copying. This latter
  revision of the definition of replication is crucial to obtaining
  the expected identity $!!P \sim !P$.
\end{remark}

\begin{remark}\label{rem:paradoxical_combinator}
  The reader familiar with the lambda calculus will have noticed the
  similarity between $D$ and the paradoxical combinator.

  [Ed. note: the existence of this seems to suggest we have to be more
  restrictive on the set of processes and names we admit if we are to
  support no-cloning.]
\end{remark}

\subsubsection{Bisimulation}

The computational dynamics gives rise to another kind of equivalence,
the equivalence of computational behavior. As previously mentioned
this is typically captured \emph{via} some form of bisimulation.

% The notion we use in this paper is weak barbed bisimulation
% \cite{milner91polyadicpi}.

The notion we use in this paper is derived from weak barbed
bisimulation \cite{milner91polyadicpi}. 

\begin{definition}
An \emph{observation relation}, $\downarrow_{\mathcal N}$, over a set
of names, $\mathcal N$, is the smallest relation satisfying the rules
below.

\infrule[Out-barb]{y \in {\mathcal N}, \; x \nameeq y}
		  {\outputp{x}{v} \downarrow_{\mathcal N} x}
\infrule[Par-barb]{\mbox{$P\downarrow_{\mathcal N} x$ or $Q\downarrow_{\mathcal N} x$}}
		  {\binpar{P}{Q} \downarrow_{\mathcal N} x}

We write $P \Downarrow_{\mathcal N} x$ if there is $Q$ such that 
$P \wred Q$ and $Q \downarrow_{\mathcal N} x$.
\end{definition}

\begin{definition}
%\label{def.bbisim}
An  ${\mathcal N}$-\emph{barbed bisimulation} over a set of names, ${\mathcal N}$, is a symmetric binary relation 
${\mathcal S}_{\mathcal N}$ between agents such that $P\rel{S}_{\mathcal N}Q$ implies:
\begin{enumerate}
\item If $P \red P'$ then $Q \wred Q'$ and $P'\rel{S}_{\mathcal N} Q'$.
\item If $P\downarrow_{\mathcal N} x$, then $Q\Downarrow_{\mathcal N} x$.
\end{enumerate}
$P$ is ${\mathcal N}$-barbed bisimilar to $Q$, written
$P \wbbisim_{\mathcal N} Q$, if $P \rel{S}_{\mathcal N} Q$ for some ${\mathcal N}$-barbed bisimulation ${\mathcal S}_{\mathcal N}$.
\end{definition}

$\mathcal{R} \subseteq \pi \times \pi$

$P \mathcal{R} Q => \forall P'. P \red P' \Rightarrow \exists Q'. Q \red Q', P' \mathcal{R} Q'$

$P \vdash x \Rightarrow Q \vdash x$

\begin{mathpar}
  \inferrule*[lab=Out-barb]{x \nameeq y}{{y}!\langle{Q}\rangle \vdash x}
  \and
  \inferrule*[lab=Par-barb]{\mbox{$P\vdash x$ or $Q\vdash x$}}{\binpar{P}{Q} \vdash x}
\end{mathpar}

\subsubsection{Contexts}

One of the principle advantages of computational calculi like the
$\pi$-calculus is a well-defined notion of context,
contextual-equivalence and a correlation between
contextual-equivalence and notions of bisimulation. The notion of
context allows the decomposition of a process into (sub-)process and
its syntactic environment, its context. Thus, a context may be
thought of as a process with a ``hole'' (written $\Box$) in it. The
application of a context $M$ to a process $P$, written $M[P]$, is
tantamount to filling the hole in $M$ with $P$. In this paper we do
not need the full weight of this theory, but do make use of the notion
of context in the proof the main theorem. 

\begin{mathpar}
  \inferrule* [lab=summation] {} {{M_{M},M_{N}} \bc \Box \;|\; x.M_{A} \;|\; M_{M}+M_{N}}
  \and
  \inferrule* [lab=agent] {} {{M_{A}} \bc (\vec{x})M_{P} \;| \; \clift{P_0,\ldots,M_{P},\ldots,P_N}}
  \and \\
  \inferrule* [lab=process] {} {{M_{P}} \bc M_{N} \;| \;P|M_{P} }
\end{mathpar} 

\begin{mathpar}
  \inferrule* [lab=sychronization] {} {M_{N} \bc \Box \;|\; x?M_{F} \;|\; x!M_{C}}
  \and
  \inferrule* [lab=abstraction] {} {{M_{F}} \bc (x)M_{P} }
  \and
  \inferrule* [lab=concretion] {} {{M_{C}} \bc \langle M_{P} \rangle }
  \and \\
  \inferrule* [lab=process] {} {{M_{P}} \bc M_{N} \;| \;P|M_{P} }
\end{mathpar}

\begin{definition}[contextual application] Given a context $M$, and
  process $P$, we define the \emph{contextual application}, $M[P] :=
  M\{P/\Box\}$. That is, the contextual application of M to P is the
  substitution of $P$ for $\Box$ in $M$.
\end{definition}

$\meaningof{-} : L \to \mathcal{P}(\pi)$

\begin{mathpar}
  \inferrule* [lab=collection] {} {\meaningof{true} = \pi, \and \meaningof{~E} = \pi \setminus \meaningof{E}, \and \meaningof{E_{1} \& E_{2}} = \meaningof{E_{1}} \cap \meaningof{E_{2}}}
\end{mathpar}

\begin{mathpar}
  \inferrule* [lab=structure] {} {\meaningof{0} = \{ P \in \pi | P \equiv 0 \}, \and \\ \meaningof{E_1 | E_2} = \{ P \in \pi | P \equiv P_{1} | P_{2}, P_{1} \in \meaningof{E_{1}}, P_{2} \in \meaningof{E_2}\} }
\end{mathpar}

\begin{mathpar}
 \inferrule* [lab=behavior] {} {\meaningof{\langle a?b \rangle E} = \{ P \in \pi | P \equiv Q | u?(y)P', \\ \and \\\\ \and \\ \;\;\; u \in \meaningof{a}, \forall z.P'\{z/y\} \in \meaningof{E\{z/b\}}\}, \and \\ \meaningof{a!E} = \{ P \in \pi | P \equiv Q | x!\langle P' \rangle, x \in \meaningof{a} P' \in \meaningof{E}\} }
\end{mathpar}

\begin{mathpar}
 \inferrule* [lab=nominal] {} {\meaningof{\quotep{E}} = \{ \quotep{P} \in \quotep{\pi} | P \in \meaningof{E} \}, \and \meaningof{\quotep{P}} = \{ \quotep{Q} \in \quotep{\pi} | P \equiv Q \} \and \\ \meaningof{@\quotep{E}} = \{ P \in \pi | P \equiv @x, x \in \meaningof{E} \}}
\end{mathpar}

\begin{eqnarray*}
  \\
  \meaningof{-} : TS \to ST
\end{eqnarray*}

\begin{eqnarray*}
  \\
  L : TS \to ST
\end{eqnarray*}

\begin{eqnarray*}
  \\
  P \models E \iff P \in \meaningof{E}
\end{eqnarray*}

\begin{eqnarray*}
  P \approx_{L} Q \iff \forall E \in L. P \models E \iff Q \models E
\end{eqnarray*}

\begin{eqnarray*}
  P \approx_{K} Q
\end{eqnarray*}

\begin{eqnarray*}
  P \approx Q
\end{eqnarray*}

$\approx_{K} = \approx = \approx_{L}$

\subsubsection{Contextual duality}

Note that contexts extend the quotation operation to a family of
operations from processes to names. Given a context, $M$, we can
define a \emph{nominal context}, $\quotep{M}$ by $\quotep{M}[P] :=
\quotep{M[P]}$. To foreshadow what is to come we observe that these
operations enjoy a duality with processes very much like the duality
between vectors and maps from vectors to scalars.

Further, because the calculus is essentially higher-order, we have a
correspondence between contexts and processes. More specifically,
given a name $x$ and a context $M$ we can construct $M^{*}_{x}$ such
that 

\begin{mathpar}
  M^{*}_{x} | \lift{x}{P} \red M[P]
\end{mathpar}

namely,

\begin{mathpar}
  M^{*}_{x} := x?(u).M[\dropn{u}]
\end{mathpar}

The dependence of $M^{*}_{x}$ on a name makes it an abstraction, 

\begin{mathpar}
  M^{*} := (x)x?(u).M[\dropn{u}]
\end{mathpar}

\subsection{Additional notation}

It will sometimes be convenient to denote the process a name
quotes. We already have the notation $x = \quotep{P}$, but it will be
convenient to introduce an alternate notation, $\procn{x}$, when we
want to emphasize the connection to the use of the name. Note that, by
virtue of name equivalence, $\quotep{\procn{x}} \nameeq x$; so, the
notation is consistent with previous definitions.

Further, because names have structure it is possible to effect
substitutions on the basis of that structure. This means we need to
upgrade our notation for substitutions, which we accomplish by
adapting comprehension notation. Thus,

\begin{mathpar}
  P\{ y / x : x \in S \}
\end{mathpar}

is interpreted to mean the process derived from P by replacing (in a
capture-avoiding manner) each occurrence of $x$ in $S$ by $y$. For example,

\begin{mathpar}
  P\{ \quotep{\procn{x}|\procn{x}} / x : x \in \freenames{P} \}
\end{mathpar}

will replace each (occurrence) of a free name $x$ in $P$ by
$\quotep{\procn{x}|\procn{x}}$.

Also, we will avail ourselves of the notation $x^{L}$ and $x^{R}$ to
denote injections of a name into disjoint copies of the name
space. There are numerous ways to accomplish this. One example can be
found in \cite{MeredithR05}. This notation overloads to vectors of
names: $\vec{x}^{\pi} := (x_{i}^{\pi} \; : \; 0 \leq i < |\vec{x}| )$ where $\pi \in \{L,R\}$.

We also use $P^{\Box} := P|\Box$.

In \cite{MeredithR05} an interpretation of the new operator is
given. It turns out that there are several possible interpretations
all enjoying the requisite algebraic properties of the operator (see
\cite{milner91polyadicpi}). We will therefore make liberal use of
$(\nu\; \vec{x})P$.

% subsection the_syntax_and_semantics_of_the_notation_system (end)   

\section{Interpretation of QM}
\subsection{Supporting definitions}
\subsubsection{Multiplication}
\begin{mathpar}
  \quotep{Q} \cdot \quotep{R} := \quotep{Q|R}
  \and \\
  \quotep{Q} \cdot P := P\{ \quotep{Q|R} / \quotep{R} : \quotep{R} \in \freenames{P} \}
\end{mathpar}

\paragraph{Discussion}
The first line needs little explanation. The second line says that
each free name of the process is replaced with the multiplication of
that name by the scalar. Multiplication of a scalar (name) by a state
(process) results in a process all the names of which have been `moved
over' by parallel composition with the process the scalar
quotes. There is a subtlety that the bound names have to be
manipulated so that multiplied names aren't accidentally
captured. There are many ways to achieve this.

\begin{remark}\label{rem:multiplication_identities}
  The reader is invited to verify that for all $x,y,z \in \QProc$ and $P \in \Proc$
  \begin{mathpar}
    x \cdot \quotep{0} \equiv x 
    \and
    x \cdot y \equiv y \cdot x
    \and
    x \cdot (y \cdot z) \equiv (x \cdot y) \cdot z
    \and \\
    \quotep{0} \cdot P \equiv P
    \and \\
    x \cdot (y \cdot P) \equiv (x \cdot y) \cdot P
    \and \\
    x \cdot (P|Q) \equiv (x \cdot P) | (x \cdot Q)
    \and \\    
  \end{mathpar}
\end{remark}

\subsubsection{Tensor product}

We define a tensor product on processes by structural induction.

\paragraph{Tensor of sums} First note that all summations, including
$\pzero$ and sequence, can be written $\Sigma_{i} x_{i}.A_{i} +
\Sigma_{j} x_{j}.C_{j}$, where we have grouped input-guarded processes
together and output-guarded processes together.

Thus, we can define the tensor product of two summations, $N_{1}\otimes N_{2}$, where

\begin{mathpar}
  N_{1} := \Sigma_{i} x_{i}.A_{i} + \Sigma_{j} x_{j}.C_{j}
  \and
  N_{2} := \Sigma_{i'} y_{i'}.B_{i'} + \Sigma_{j'} y_{j'}.D_{j'} 
\end{mathpar}

as follows.

\begin{mathpar}
  \Sigma_{i} x_{i}.A_{i} + \Sigma_{j} x_{j}.C_{j} \otimes \Sigma_{i'}
  y_{i'}.B_{i'} + \Sigma_{j'} y_{j'}.D_{j'} 
  \and \\
  := \; \Sigma_{i} \Sigma_{i'} \quotep{\stackrel{\vee}{x_{i}}| \stackrel{\vee}{y_{i'}}}.(A_{i}\otimes B_{i'}) \; | \; \Sigma_{i'} \Sigma_{i} \quotep{\stackrel{\vee}{y_{i'}}|\stackrel{\vee}{x_{i}}}.(B_{i'}\otimes A_{i})
  \and
  \;\; | \;\; \Sigma_{j} \Sigma_{j'} \quotep{\stackrel{\vee}{x_{j}}|\stackrel{\vee}{y_{j'}}}.(A_{j}\otimes B_{j'}) \; | \; \Sigma_{j'} \Sigma_{j} \quotep{\stackrel{\vee}{y_{j'}}|\stackrel{\vee}{x_{j}}}.(B_{j'}\otimes A_{j})
\end{mathpar}

\begin{remark}
  Do we need to $x^{L}$ and $y^{R}$ for this construction as well?
\end{remark}

\paragraph{Tensor of parallel compositions} Next, we distribute tensor
over par.

\begin{mathpar}
  P_{1}|P_{2} \otimes Q_{1}|Q_{2} := (P_{1} \otimes Q_{1}) | (P_{1}
  \otimes Q_{2}) | (P_{2} \otimes Q_{1}) | (P_{2} \otimes Q_{2})
\end{mathpar}

\paragraph{Tensor with dropped names} We treat tensor of a
process with a dropped name as parallel composition.

\begin{mathpar}
  P \otimes \dropn{x} := P | \dropn{x}
\end{mathpar}

\paragraph{Tensor of agents}

Finally, we need to define tensor on agents. Note that the definition
of tensor on normal products only tensors inputs with inputs and
outputs with outputs. Thus, we only have to define the operation on
``homogeneous'' pairings.

\begin{mathpar}
  (\vec{x})P \otimes (\vec{y})Q
  \and \\
  := (x_{0}^{L}|y_{0}^{R},\ldots,x_{0}^{L}|y_{n}^{R},\ldots,x_{m}^{L}|y_{0}^{R},\ldots,x_{m}^{L}|y_{n}^R)(P\{ \vec{x}^{L}/\vec{x}\} \otimes Q \{ \vec{y}^{R}/\vec{y}\})
  \and \\
  \clift{\vec{P}} \otimes \clift{\vec{Q}}
  \and \\
  := \clift{P_{0}\otimes Q_{0},\ldots,P_{0}\otimes Q_{n},\ldots,P_{m}\otimes Q_{0},\ldots,P_{m}\otimes Q_{n}}
\end{mathpar}

\begin{remark}
  Observe that arities of tensored abstractions matches arities of
  tensored concretions if the original arities matched. Note also that
  the length of the arities corresponds to the increase in dimension
  we see in ordinary vector space tensor product.
\end{remark}

\begin{remark}
  Operationally, this definition distributes the tensor down to
  components ``linked'' by summation. Tensor over summation is
  intriguing in that it mixes names. Moreover, as a consequence of the
  way it mixes names we have the identities for all $x \in \QProc$ and
  $P,Q \in \Proc$

  \begin{mathpar}
    (x \cdot P) \otimes Q \equiv x \cdot (P \otimes Q) \equiv P \otimes (x \cdot Q)
    \and
    P \otimes \pzero \equiv P
  \end{mathpar}

  that the reader is invited to verify.
\end{remark}

\subsubsection{Annihilation}
\begin{mathpar}
  P^{\perp} := \{ Q | \forall R. P|Q \red^{*} R \Rightarrow R \red^{*} \pzero \}
  \and \\
  P^{\underline{\perp}} := \Sigma_{Q \in P^{\perp}} \quotep{Q}?(y).(\dropn{y}|Q) | \Sigma_{Q \in P^{\perp}} \quotep{Q}\clift{\Box}
\end{mathpar}

\paragraph{Discussion} The reader will note that $P^{\perp}$ is a
\emph{set} of processes, while $P^{\underline{\perp}}$ is a
\emph{context}. We call the set $P^{\perp}$ the \emph{annihilators} of
$P$. The parallel composition of a process in the annihilators of $P$
with $P$ will result in a process, the state space of which has all
paths eventually leading to $\pzero$. Execution may endure loops; but
under reasonable conditions of fairness (naturally guaranteed under
most notions of bisimulation) such a composite process cannot get
stuck in such a loop and will, eventually pop out and terminate.

The context $P^{\underline{\perp}}$ is ready and willing to ``take the
$P$ out of'' the process to which it is applied. It will effectively
transmit the code of the process to which it is applied to one of the
annihilators and run the process against it.

\subsubsection{Evaluation}
We fix $M$ a domain of fully abstract interpretation with an equality
coincident with bisimulation. We take $\meaningof{\cdot} : \Proc \to
M$ to be the map interpreting processes and $\nmeaningof{\cdot} : \M
\to Proc$ to be the map running the other way. Then we define

\begin{mathpar}
  \int P := \nmeaningof{\meaningof{P}}
\end{mathpar}

\paragraph{Discussion}
There are many fully abstract interpretations of Milner's
$\pi$-calculus. Any of them can be used as a basis for interpreting
the reflective calculus here. Equipped with such a domain it is
largely a matter of grinding through to check that the Yoneda
construction for the normalization-by-evaluation program can be
extended to this setting.

\begin{remark}
  The reader is invited to verify that $\int (P^{\underline{\perp}}[P]) = 0$.
\end{remark}

\subsection{Quantum mechanics}

Table \ref{tbl:core_qm_op_defns} gives the core operational definitions

\begin{table}[htp]\label{tbl:core_qm_op_defns}
  \center{
    \fbox{
      \begin{tabular}{c|c}
        quantum mechanics & process calculus \\
        \hline
        scalar & $x := \quotep{P}$ \\
        state vector & $\state{P} := P$ \\
        dual & $\state{P}^{*} := \event{P^{\underline{\perp}}} := \quotep{P^{\underline{\perp}}}[-]$ \\
        matrix & $ \Sigma_{\alpha} \state{P_{\alpha}}x_{\alpha}\event{Q_{\alpha}}$ \\
        vector addition & $\state{P} + \state{Q} := \state{P | Q}$ \\
        tensor product & $\state{P} \otimes \state{Q} := \state{P \otimes Q}$ \\
        inner product & $\innerprod{P}{Q} := \quotep{\int P^{\underline{\perp}}[Q]}$ \\
      \end{tabular}
    }
  }
  \caption{QM - operational definitions}
\end{table}

where

\begin{mathpar}
  \prmatrix{P}{Q} := \fprmatrix{P}{\quotep{\pzero}}{Q}
  \and
  \fprmatrix{P}{x}{Q} := (\state{P},x,\event{Q})
  \and
  (\fprmatrix{P}{x}{Q})(\state{R}) := x \cdot \innerprod{Q}{R} \cdot \state{P}
  \and
  (\fprmatrix{P}{x}{Q})(\event{R}) := x \cdot \innerprod{R}{P} \cdot \event{Q}
\end{mathpar}

\paragraph{Discussion}
As promised: vectors (aka states) are represented as processes; duals
as contextual duals; inner product definition should be compared with
standard inner product definition for ....

\begin{remark}
  Assuming $\int (P^{\underline{\perp}}[P]) = 0$, the reader is
  invited to verify that $(\fprmatrix{P}{x}{P})(\state{P}) = x \cdot \state{P}$.
\end{remark}

\begin{remark}
  The reader is invited to verify that $\innerprod{P}{Q}$ could
  equally well have been written $\quotep{\int \stackrel{\vee}{x}}$
  where $x = \event{P^{\underline{\perp}}}(Q)$.

  One of the motivations for this remark is that there is another way
  to factor these operations. We could package up evaluation in the dual:

  \begin{mathpar}
    \state{P}^{*} := \event{\int P^{\underline{\perp}}} := \quotep{\int P^{\underline{\perp}}}[-]
  \end{mathpar}

  and then have inner product defined by
  
  \begin{mathpar}
    \innerprod{P}{Q} := \event{P}(Q)
  \end{mathpar}

  Hopefully, experience with the calculations will provide guidance on
  the best factoring.
\end{remark}

\begin{remark}
  Assuming $\int (P^{\underline{\perp}}[P]) = 0$, the reader is
  invited to verify that $\forall P,Q. (\prmatrix{0}{Q})(\state{0}) =
  \state{0}$ and dually $(\prmatrix{P}{0})(\event{0}) = \event{0}$.
\end{remark}

\begin{remark}
  i'm a little worried that i don't (yet) have proper support for
  complex conjugacy. But, the observation above may give us a
  clue. According to Abramsky, it must be the case that the scalars
  are iso to the homset of the identity for the tensor -- which the
  observation above characterizes. 

  For now, we will simply bookmark the notion with $\overline{x}$.
\end{remark}

\subsubsection{Adjointness}

We need to give a definition of $(\cdot)^{\dagger}$ for matrices. The
obvious candidate definition is
\begin{mathpar}
(\Sigma_{\alpha}\fprmatrix{P_{\alpha}}{x_{\alpha}}{Q_{\alpha}})^{\dagger}
= \Sigma_{\alpha}\fprmatrix{(Q_{\alpha}^{\underline{\perp}})^{*}}{\overline{x}_{\alpha}}{P_{\alpha}^{\underline{\perp}}} 
\end{mathpar}

But, $(Q_{\alpha}^{\underline{\perp}})^{*}$ requires a name along
which to communicate the process to achieve the context application.

\subsubsection{Basis for a basis}
If processes label states and ``addition'' of states (a.k.a. vector
addition) is interpreted as parallel composition, what corresponds to
notions of linear independence and basis? Here, we recall that Yoshida
has developed a set of \emph{combinators} for an asynchronous verison
of Milner's $\pi$-calculus. These are a finite set of processes such
any process can be expressed as parallel composition of these
combinators together with liberal uses of the new operator and
replication. We can simply give a translation of these into the
present calculus and have reasonable expectation that the property
carries over. That is, that the resultant set allows to express all
processes via parallel composition. Note, however, that there is no
new operator or replication in this calculus. As a result, we expect
that the corresponding set is actually infinite. That is, we expect
that the space is actually infinite dimensional.

\begin{remark}
  The attentive reader may be a bit concerned. Certainly, the
  collection $S$, $K$ and $I$ is a finite set of
  combinators. Shouldn't we expect to see a finite set of combinators
  for an effectively equivalent system? i am very sympathetic to this
  critique and feel it warrants full attention. On the other hand, i
  also have in mind the following analogy. The natural numbers, as a
  monoid under addition, has exactly $1$ generator, while the natural
  numbers, as a monoid under multiplication, has countably many
  generators (the primes). We observe that the application of the
  lambda calculus is much less resource sensitive than the parallel
  composition of the $\pi$-calculus. Could it be the case that we have
  an analogy of the form
  
  \begin{mathpar}
    m + n : MN :: m*n : M|N
  \end{mathpar}

  giving a similar blow up in the set of ``primes''?  This is such a
  wonderful thought that, even if it's not true, i think it's worth
  writing down.
\end{remark}
 

\documentclass[12pt]{llncs}
%\documentclass{jktr}

\usepackage[pdftex]{hyperref}                   
\usepackage {listings}
\usepackage {mathpartir}
\usepackage{bcprules}
%\usepackage{listings}
                       
\usepackage{graphicx} 
%\usepackage[margins=2.5cm,nohead,nofoot]{geometry}
%\usepackage{geometry}
\usepackage{amsfonts}
\usepackage{amstext}
\usepackage{latexsym}
\usepackage{amssymb}
\usepackage{color}


%\include{myPreamble}
\include{qm2pi.local} 

%\ifpdf
%\usepackage[pdftex]{graphicx}
%\else
%\usepackage{graphicx}
%\fi

 % \ifpdf
%  \usepackage{pdfsync}
%  \if


%\title{Brief Article}
%\author{David F. Snyder}
%\author{L.G. Meredith}

%\address{Dept. of Math., Texas State University--San Marcos, San Marcos, TX 78666}
       
\pagestyle{empty}


\begin{document}

\lstset{language=[Objective]Caml,frame=shadowbox}

\input{qm2pi.front}

% section front matter (end)

\input{qm2pi.intro} 
 
% section introduction (end)

% \input{qm2pi.knotations} 

% section notation (end)

\input{qm2pi.process.calculi} 

% section concurrent_process_calculi_and_spatial_logics_ (end)
    
%\input{qm2pi.knots2pi} 

%\input{qm2pi.trefoil} 

%\input{qm2pi.mainthm} 

% subsection basic_interpretation (end)

%\input{qm2pi.rho.presentation} 
\subsection{The syntax and semantics of the notation system}\label{sub:the_syntax_and_semantics_of_the_notation_system} % (fold)

We now summarize a technical presentation of the calculus that
embodies our theory of dynamics. The typical presentation of such a
calculus follows the style of giving generators and relations on
them. The grammar, below, describing term constructors, freely
generates the set of processes, $\Proc$. This set is then quotiented
by a relation known as structural congruence and it is over this set
that the notion of dynamics is expressed. This presentation is
essentially that of \cite{MeredithR05} with the addition of
polyadicity and summation. For readability we have relegated some of
the technical subtleties to an appendix.

\subsubsection{Process grammar}\label{subsub:process_grammar}

\begin{mathpar}
  \inferrule* [lab=synchronization] {} {{M} \bc \pzero \;|\; x?F \;|\; x!C }
  \and
  \inferrule* [lab=abstraction] {} {{F} \bc (x)P}
  \and
  \inferrule* [lab=concretion] {} {{C} \bc \langle Q \rangle}
  \and
  \inferrule* [lab=process] {} {{P,Q} \bc M \;| \;P|Q \;|\; @{x}}
  \and
  \inferrule* [lab=name] {} {{x} \bc \quotep{P}}
\end{mathpar} 

Note that $\vec{x}$ (resp. $\vec{P}$) denotes a vector of names
(resp. processes) of length $|\vec{x}|$ (resp. $|\vec{P}|$). We adopt
the following useful abbreviations.

\begin{mathpar}
   x?(\vec{y}).P := x.(\vec{y})P \and  x\clift{\vec{P}} := x.\clift{\vec{P}}
   \and x!(y) := \lift{x}{\dropn{y}}
   \and \Pi_{i=0}^{n-1}P_i := P_0 | \ldots | P_{n-1}
\end{mathpar}

\subsubsection{Structural congruence}

\paragraph{Free and bound names and alpha-equivalence.} At the
core of structural equivalence is alpha-equivalence which identifies
process that are the same up to a change of variable. Formally, we
recognize the distinction between free and bound names. The free names
of a process, $\freenames{P}$, may be calculated recursively as
follows:

\begin{mathpar}
\freenames{\pzero} := \emptyset
  \and \\
  \freenames{x?(y).P} := \{ x \} \cup (\freenames{P} \setminus \{ y \})
  \and 
  \freenames{x!\langle P \rangle} := \{ x \} \cup \{ P \} 
  \and \\
  \freenames{P|Q} := \freenames{P} \cup \freenames{Q}
  \and \\
  \freenames{@{x}} := \{ x \}
\end{mathpar}

$\pi$
$\quotep{\pi}$

$\freenames{-} : \pi \to \mathcal{P}(\quotep{\pi})$

\begin{eqnarray*}
  \freenames{\pzero} & := & \emptyset \\
  \freenames{x?(y).P} & := & \{ x \} \cup (\freenames{P} \setminus \{ y \}) \\
  \freenames{x!\langle P \rangle} & := & \{ x \} \cup \{ P \} \\
  \freenames{P|Q} & := & \freenames{P} \cup \freenames{Q} \\
  \freenames{\dropn{x}} & := & \{ x \}
\end{eqnarray*}

The bound names of a process, $\boundnames{P}$, are those names occurring in $P$
that are not free. For example, in $x?(y).0$, the name $x$ is free, while $y$ is bound.

\begin{mathpar}
  \inferrule* [lab=monoidal-laws] {} { P|Q \equiv Q|P \and P|0 \equiv P \and P|(Q|R) \equiv (P|Q)|R }
\end{mathpar}

\begin{mathpar}
  \inferrule* [lab=alpha-equivalence] {} { (x)P \equiv (y)P\{y/x\} \and y \not\in \freenames{P} }
\end{mathpar}

\begin{definition}
Then two processes, $P,Q$, are alpha-equivalent if $P = Q\{\vec{y}/\vec{x}\}$ for
some $\vec{x} \in \boundnames{Q},\vec{y} \in \boundnames{P}$, where $Q\{\vec{y}/\vec{x}\}$
denotes the capture-avoiding substitution of $\vec{y}$ for $\vec{x}$ in $Q$.
\end{definition}

\begin{definition}
  The {\em structural congruence} \cite{SangiorgiWalker} , $\equiv$,
  between processes is the least congruence containing
  alpha-equivalence, satisfying the abelian monoid laws
  (associativity, commutativity and $\pzero$ as identity) for parallel
  composition $|$ and for summation $+$.
\end{definition}

\subsection{Name equivalence}

We take name equivalence, written $\nameeq$, to be the smallest
equivalence relation generated by the following rules.

\begin{mathpar}
\inferrule*[lab=Quote-drop]
{ }
{ \quotep{@{x}} \nameeq x }

\inferrule*[lab=Struct-equiv]
{ P \scong Q }
{ \quotep{P} \nameeq \quotep{Q} }
\end{mathpar}

The astute reader will have noticed that the mutual recursion of names
and processes imposes a mutual recursion on alpha-equivalence and
structural equivalence via name-equivalence. Fortunately, all of this
works out pleasantly and we may calculate in the natural way, free of
concern. The reader interested in the details is referred to the
appendix \ref{appendix:rho_details}.

\subsection{Substitution}

We use $\Proc$ for the set of processes, $\QProc$ for the set of
names, and $\id{\{}\vec{y} / \vec{x} \id{\}}$ to denote partial maps,
$s : \QProc \rightarrow \QProc$. A map, $s$ lifts, uniquely, to a map
on process terms, $\widehat{s} : \Proc \rightarrow \Proc$ by the
following equations.

\begin{mathpar}
  (0) \psubstp{Q}{P} := 0 \\
  (R \juxtap S) \psubstp{Q}{P}
  :=    
  (R)\psubstp{Q}{P} \juxtap (S) \psubstp{Q}{P} \\
  (x?(y).R) \psubstp{Q}{P}    
  :=    
  (x)\substp{Q}{P} (z)\concat( (R \psubstn{z}{y}) \psubstp{Q}{P} ) \\
  (\lift{x}{R}) \psubstp{Q}{P}  
  :=
  \lift{(x)\substp{Q}{P}}{ R \psubstp{Q}{P} } \\
%   (\dropn{x})  \psubstp{Q}{P}       
%   := 
%   \left\{ 
%     \begin{array}{ccc} 
%       \dropn{\quotep{Q}} & & x \nameeq \quotep{P} \\
%       \dropn{x} & & otherwise \\
%     \end{array}
%   \right. 
  (\dropn{x})  \psubstp{Q}{P}       
  := 
  \left\{ 
    \begin{array}{ccc} 
      Q & & x \nameeq \quotep{P} \\
      \dropn{x} & & otherwise \\
    \end{array}
  \right.
\end{mathpar}
 

where

\begin{eqnarray}
  (x)\id{\{} \lpquote Q \rpquote / \lpquote P \rpquote \id{\}}            = 
  \left\{ 
    \begin{array}{ccc}
      \lpquote Q \rpquote & & x \nameeq \lpquote P \rpquote \\
      x & & otherwise \\
    \end{array}
  \right. \nonumber
\end{eqnarray}

and $z$ is chosen distinct from $\quotep{P}$, $\quotep{Q}$, the free
names in $Q$, and all the names in $R$. Our $\alpha$-equivalence will
be built in the standard way from this substitution.

\begin{remark}\label{rem:no_self_referential_names}
  One consequence of these definitions is that $\forall P. \quotep{P}
  \not\in \freenames{P}$.
\end{remark}

\subsection{ Dynamic quote: an example }

Anticipating something of what's to come, consider applying the
substitution, $\widehat{\id{\{}u / z \id{\}}}$, to the following pair
of processes, $\lift{w}{y!(z)}$ and $w[ \lpquote y!(z) \rpquote ]$.

\begin{eqnarray}
	\lift{w}{y!(z)}\widehat{\id{\{}u / z \id{\}}}
		& = &
		\lift{w}{y!(u)} \nonumber\\
	w[ \lpquote y!(z) \rpquote ] \widehat{ \id{\{}u / z \id{\}} }
		& = &
		w[ \lpquote y!(z) \rpquote ] \nonumber
\end{eqnarray}

Because the body of the process between quotes is impervious to
substitution, we get radically different answers. In fact, by
examining the first process in an input context,
e.g. $x?(z).\lift{w}{y!(z)}$, we see that the process under the lift
operator may be shaped by prefixed inputs binding a name inside it. In
this sense, the lift operator will be seen as a way to dynamically
construct processes before reifying them as names.

Finally equipped with these standard features we can present the
dynamics of the calculus.

\subsubsection{Operational semantics} 

Finally, we introduce the computational dynamics. What marks these
algebras as distinct from other more traditionally studied algebraic
structures, e.g. vector spaces or polynomial rings, is the manner in
which dynamics is captured. In traditional structures, dynamics is typically
expressed through morphisms between such structures, as in linear maps
between vector spaces or morphisms between rings. In algebras
associated with the semantics of computation, the dynamics is
expressed as part of the algebraic structure itself, through a
reduction reduction relation typically denoted by $\red$. Below, we
give a recursive presentation of this relation for the calculus used
in the encoding.

$\red \subseteq \pi \times \pi$
$\red : \pi \to \mathcal{P}(\pi)$

\begin{mathpar}
  \inferrule* [lab=Comm] { \textsf{match}( x_{src}, x_{trgt} ) } { x_{trgt}?(y)P \; | \; x_{src}!\langle {Q} \rangle \red P\{\quotep{Q}/y}\} }
  \and \\
  \inferrule* [lab=Par] {{P} \red {P}'} {{{P} | {Q}} \red {{P}' | {Q}}}
  \and
  \inferrule* [lab=Equiv]{{{P} \scong {P}'} \andalso {{P}' \red {Q}'} \andalso {{Q}' \scong {Q}}}{{P} \red {Q}}
\end{mathpar}

\begin{eqnarray*}
  match_{\equiv} (\quotep{P},\quotep{Q}) & := & P \equiv Q \\
  match_{\dagger}(\quotep{P},\quotep{Q}) & := & \forall R. P|Q \red^{*} R => R \red^{*} 0 \\
  match_{K}(\quotep{P},\quotep{Q}) & := & K \mbox{ for some context } K
\end{eqnarray*}

$u?(x)P | u!\langle Q \rangle \red P\{\quotep{Q}/x\}$

%We write $\wred$ for $\red^*$, and $P\red$ if $\exists Q $ such that $ P \red Q$.
We write $P\red$ if $\exists Q $ such that $ P \red Q$ and $P\not\red$, otherwise.

\section{Replication}

As mentioned before, it is known that replication (and hence
recursion) can be implemented in a higher-order process algebra
\cite{SangiorgiWalker}. As our first example of calculation with the
machinery thus far presented we give the construction explicitly in
the {\rhoc}.

\begin{eqnarray}
	D_{x} & := & \prefix{x}{y}{(\binpar{\outputp{x}{y}}{@{y}})} \nonumber\\
	\bangp_{x}{P} & := & \binpar{{x}!\langle{\binpar{D_{x}}{P}}\rangle}{D_{x}} \nonumber
\end{eqnarray}

\begin{eqnarray}
	\bangp_{x}{P} & & \nonumber\\
	=
	& {x}!\langle{(\prefix{x}{y}{(\outputp{x}{y} | @{y})) | P}}\rangle 
	      | \prefix{x}{y}{(\outputp{x}{y} | @{y})} & \nonumber\\
	\red
	& (\outputp{x}{y} | @{y})\substn{\quotep{(\prefix{x}{y}{(@{y} | \outputp{x}{y})) | P}}}{y} & \nonumber\\
	=
	& \outputp{x}{\quotep{(\prefix{x}{y}{(\outputp{x}{y} | @{y})) | P}}}
	  | {(\prefix{x}{y}{(\outputp{x}{y} | @{y})) | P}} & \nonumber\\
	\red
	& \ldots & \nonumber\\
	\red^*
	& P | P | \ldots & \nonumber
\end{eqnarray}

Of course, this encoding, as an implementation, runs away, unfolding
$\bangp{P}$ eagerly. A lazier and more implementable replication
operator, restricted to input-guarded processes, may be obtained as follows.

\begin{eqnarray}
\bangp{\prefix{u}{v}{P}} 
	:= 
	\binpar{\lift{x}{\prefix{u}{v}{(\binpar{D(x)}{P})}}}{D(x)} \nonumber
\end{eqnarray}

\begin{remark}
  Note that the lazier definition still does not deal with summation
  or mixed summation (i.e. sums over input and output). The reader is
  invited to construct definitions of replication that deal with these
  features. 

  Further, the definitions are parameterized in a name, $x$. Can you,
  gentle reader, make a definition that eliminates this parameter and
  guarantees no accidental interaction between the replication
  machinery and the process being replicated -- i.e. no accidental
  sharing of names used by the process to get its work done and the
  name(s) used by the replication to effect copying. This latter
  revision of the definition of replication is crucial to obtaining
  the expected identity $!!P \sim !P$.
\end{remark}

\begin{remark}\label{rem:paradoxical_combinator}
  The reader familiar with the lambda calculus will have noticed the
  similarity between $D$ and the paradoxical combinator.

  [Ed. note: the existence of this seems to suggest we have to be more
  restrictive on the set of processes and names we admit if we are to
  support no-cloning.]
\end{remark}

\subsubsection{Bisimulation}

The computational dynamics gives rise to another kind of equivalence,
the equivalence of computational behavior. As previously mentioned
this is typically captured \emph{via} some form of bisimulation.

% The notion we use in this paper is weak barbed bisimulation
% \cite{milner91polyadicpi}.

The notion we use in this paper is derived from weak barbed
bisimulation \cite{milner91polyadicpi}. 

\begin{definition}
An \emph{observation relation}, $\downarrow_{\mathcal N}$, over a set
of names, $\mathcal N$, is the smallest relation satisfying the rules
below.

\infrule[Out-barb]{y \in {\mathcal N}, \; x \nameeq y}
		  {\outputp{x}{v} \downarrow_{\mathcal N} x}
\infrule[Par-barb]{\mbox{$P\downarrow_{\mathcal N} x$ or $Q\downarrow_{\mathcal N} x$}}
		  {\binpar{P}{Q} \downarrow_{\mathcal N} x}

We write $P \Downarrow_{\mathcal N} x$ if there is $Q$ such that 
$P \wred Q$ and $Q \downarrow_{\mathcal N} x$.
\end{definition}

\begin{definition}
%\label{def.bbisim}
An  ${\mathcal N}$-\emph{barbed bisimulation} over a set of names, ${\mathcal N}$, is a symmetric binary relation 
${\mathcal S}_{\mathcal N}$ between agents such that $P\rel{S}_{\mathcal N}Q$ implies:
\begin{enumerate}
\item If $P \red P'$ then $Q \wred Q'$ and $P'\rel{S}_{\mathcal N} Q'$.
\item If $P\downarrow_{\mathcal N} x$, then $Q\Downarrow_{\mathcal N} x$.
\end{enumerate}
$P$ is ${\mathcal N}$-barbed bisimilar to $Q$, written
$P \wbbisim_{\mathcal N} Q$, if $P \rel{S}_{\mathcal N} Q$ for some ${\mathcal N}$-barbed bisimulation ${\mathcal S}_{\mathcal N}$.
\end{definition}

$\mathcal{R} \subseteq \pi \times \pi$

$P \mathcal{R} Q => \forall P'. P \red P' \Rightarrow \exists Q'. Q \red Q', P' \mathcal{R} Q'$

$P \vdash x \Rightarrow Q \vdash x$

\begin{mathpar}
  \inferrule*[lab=Out-barb]{x \nameeq y}{{y}!\langle{Q}\rangle \vdash x}
  \and
  \inferrule*[lab=Par-barb]{\mbox{$P\vdash x$ or $Q\vdash x$}}{\binpar{P}{Q} \vdash x}
\end{mathpar}

\subsubsection{Contexts}

One of the principle advantages of computational calculi like the
$\pi$-calculus is a well-defined notion of context,
contextual-equivalence and a correlation between
contextual-equivalence and notions of bisimulation. The notion of
context allows the decomposition of a process into (sub-)process and
its syntactic environment, its context. Thus, a context may be
thought of as a process with a ``hole'' (written $\Box$) in it. The
application of a context $M$ to a process $P$, written $M[P]$, is
tantamount to filling the hole in $M$ with $P$. In this paper we do
not need the full weight of this theory, but do make use of the notion
of context in the proof the main theorem. 

\begin{mathpar}
  \inferrule* [lab=summation] {} {{M_{M},M_{N}} \bc \Box \;|\; x.M_{A} \;|\; M_{M}+M_{N}}
  \and
  \inferrule* [lab=agent] {} {{M_{A}} \bc (\vec{x})M_{P} \;| \; \clift{P_0,\ldots,M_{P},\ldots,P_N}}
  \and \\
  \inferrule* [lab=process] {} {{M_{P}} \bc M_{N} \;| \;P|M_{P} }
\end{mathpar} 

\begin{mathpar}
  \inferrule* [lab=sychronization] {} {M_{N} \bc \Box \;|\; x?M_{F} \;|\; x!M_{C}}
  \and
  \inferrule* [lab=abstraction] {} {{M_{F}} \bc (x)M_{P} }
  \and
  \inferrule* [lab=concretion] {} {{M_{C}} \bc \langle M_{P} \rangle }
  \and \\
  \inferrule* [lab=process] {} {{M_{P}} \bc M_{N} \;| \;P|M_{P} }
\end{mathpar}

\begin{definition}[contextual application] Given a context $M$, and
  process $P$, we define the \emph{contextual application}, $M[P] :=
  M\{P/\Box\}$. That is, the contextual application of M to P is the
  substitution of $P$ for $\Box$ in $M$.
\end{definition}

$\meaningof{-} : L \to \mathcal{P}(\pi)$

\begin{mathpar}
  \inferrule* [lab=collection] {} {\meaningof{true} = \pi, \and \meaningof{~E} = \pi \setminus \meaningof{E}, \and \meaningof{E_{1} \& E_{2}} = \meaningof{E_{1}} \cap \meaningof{E_{2}}}
\end{mathpar}

\begin{mathpar}
  \inferrule* [lab=structure] {} {\meaningof{0} = \{ P \in \pi | P \equiv 0 \}, \and \\ \meaningof{E_1 | E_2} = \{ P \in \pi | P \equiv P_{1} | P_{2}, P_{1} \in \meaningof{E_{1}}, P_{2} \in \meaningof{E_2}\} }
\end{mathpar}

\begin{mathpar}
 \inferrule* [lab=behavior] {} {\meaningof{\langle a?b \rangle E} = \{ P \in \pi | P \equiv Q | u?(y)P', \\ \and \\\\ \and \\ \;\;\; u \in \meaningof{a}, \forall z.P'\{z/y\} \in \meaningof{E\{z/b\}}\}, \and \\ \meaningof{a!E} = \{ P \in \pi | P \equiv Q | x!\langle P' \rangle, x \in \meaningof{a} P' \in \meaningof{E}\} }
\end{mathpar}

\begin{mathpar}
 \inferrule* [lab=nominal] {} {\meaningof{\quotep{E}} = \{ \quotep{P} \in \quotep{\pi} | P \in \meaningof{E} \}, \and \meaningof{\quotep{P}} = \{ \quotep{Q} \in \quotep{\pi} | P \equiv Q \} \and \\ \meaningof{@\quotep{E}} = \{ P \in \pi | P \equiv @x, x \in \meaningof{E} \}}
\end{mathpar}

\begin{eqnarray*}
  \\
  \meaningof{-} : TS \to ST
\end{eqnarray*}

\begin{eqnarray*}
  \\
  L : TS \to ST
\end{eqnarray*}

\begin{eqnarray*}
  \\
  P \models E \iff P \in \meaningof{E}
\end{eqnarray*}

\begin{eqnarray*}
  P \approx_{L} Q \iff \forall E \in L. P \models E \iff Q \models E
\end{eqnarray*}

\begin{eqnarray*}
  P \approx_{K} Q
\end{eqnarray*}

\begin{eqnarray*}
  P \approx Q
\end{eqnarray*}

$\approx_{K} = \approx = \approx_{L}$

\subsubsection{Contextual duality}

Note that contexts extend the quotation operation to a family of
operations from processes to names. Given a context, $M$, we can
define a \emph{nominal context}, $\quotep{M}$ by $\quotep{M}[P] :=
\quotep{M[P]}$. To foreshadow what is to come we observe that these
operations enjoy a duality with processes very much like the duality
between vectors and maps from vectors to scalars.

Further, because the calculus is essentially higher-order, we have a
correspondence between contexts and processes. More specifically,
given a name $x$ and a context $M$ we can construct $M^{*}_{x}$ such
that 

\begin{mathpar}
  M^{*}_{x} | \lift{x}{P} \red M[P]
\end{mathpar}

namely,

\begin{mathpar}
  M^{*}_{x} := x?(u).M[\dropn{u}]
\end{mathpar}

The dependence of $M^{*}_{x}$ on a name makes it an abstraction, 

\begin{mathpar}
  M^{*} := (x)x?(u).M[\dropn{u}]
\end{mathpar}

\subsection{Additional notation}

It will sometimes be convenient to denote the process a name
quotes. We already have the notation $x = \quotep{P}$, but it will be
convenient to introduce an alternate notation, $\procn{x}$, when we
want to emphasize the connection to the use of the name. Note that, by
virtue of name equivalence, $\quotep{\procn{x}} \nameeq x$; so, the
notation is consistent with previous definitions.

Further, because names have structure it is possible to effect
substitutions on the basis of that structure. This means we need to
upgrade our notation for substitutions, which we accomplish by
adapting comprehension notation. Thus,

\begin{mathpar}
  P\{ y / x : x \in S \}
\end{mathpar}

is interpreted to mean the process derived from P by replacing (in a
capture-avoiding manner) each occurrence of $x$ in $S$ by $y$. For example,

\begin{mathpar}
  P\{ \quotep{\procn{x}|\procn{x}} / x : x \in \freenames{P} \}
\end{mathpar}

will replace each (occurrence) of a free name $x$ in $P$ by
$\quotep{\procn{x}|\procn{x}}$.

Also, we will avail ourselves of the notation $x^{L}$ and $x^{R}$ to
denote injections of a name into disjoint copies of the name
space. There are numerous ways to accomplish this. One example can be
found in \cite{MeredithR05}. This notation overloads to vectors of
names: $\vec{x}^{\pi} := (x_{i}^{\pi} \; : \; 0 \leq i < |\vec{x}| )$ where $\pi \in \{L,R\}$.

We also use $P^{\Box} := P|\Box$.

In \cite{MeredithR05} an interpretation of the new operator is
given. It turns out that there are several possible interpretations
all enjoying the requisite algebraic properties of the operator (see
\cite{milner91polyadicpi}). We will therefore make liberal use of
$(\nu\; \vec{x})P$.

% subsection the_syntax_and_semantics_of_the_notation_system (end)   

\input{qm2pi.qmops} 

\input{qm2pi.sterngerlach} 

\input{qm2pi.metric} 

% section concurrent_process_calculi (end)

%\input{qm2pi.proofsketch}

% section proof sketch (end)

%\input{qm2pi.slviaknots} 

% section spatial logic via knots (end)

\input{qm2pi.conclusion}

% section conclusion (end)

%\input{qm2pi.dtcodes} 

% section wiring algorithm (end)

\input{qm2pi.ack} 

% section acknowledgments (end)

\newpage


\bibliographystyle{plain}   
\bibliography{../../biblios/main.bib}

\input{qm2pi.rhodetails}

\end{document}

 

\documentclass[12pt]{llncs}
%\documentclass{jktr}

\usepackage[pdftex]{hyperref}                   
\usepackage {listings}
\usepackage {mathpartir}
\usepackage{bcprules}
%\usepackage{listings}
                       
\usepackage{graphicx} 
%\usepackage[margins=2.5cm,nohead,nofoot]{geometry}
%\usepackage{geometry}
\usepackage{amsfonts}
\usepackage{amstext}
\usepackage{latexsym}
\usepackage{amssymb}
\usepackage{color}


%\include{myPreamble}
\include{qm2pi.local} 

%\ifpdf
%\usepackage[pdftex]{graphicx}
%\else
%\usepackage{graphicx}
%\fi

 % \ifpdf
%  \usepackage{pdfsync}
%  \if


%\title{Brief Article}
%\author{David F. Snyder}
%\author{L.G. Meredith}

%\address{Dept. of Math., Texas State University--San Marcos, San Marcos, TX 78666}
       
\pagestyle{empty}


\begin{document}

\lstset{language=[Objective]Caml,frame=shadowbox}

\input{qm2pi.front}

% section front matter (end)

\input{qm2pi.intro} 
 
% section introduction (end)

% \input{qm2pi.knotations} 

% section notation (end)

\input{qm2pi.process.calculi} 

% section concurrent_process_calculi_and_spatial_logics_ (end)
    
%\input{qm2pi.knots2pi} 

%\input{qm2pi.trefoil} 

%\input{qm2pi.mainthm} 

% subsection basic_interpretation (end)

%\input{qm2pi.rho.presentation} 
\subsection{The syntax and semantics of the notation system}\label{sub:the_syntax_and_semantics_of_the_notation_system} % (fold)

We now summarize a technical presentation of the calculus that
embodies our theory of dynamics. The typical presentation of such a
calculus follows the style of giving generators and relations on
them. The grammar, below, describing term constructors, freely
generates the set of processes, $\Proc$. This set is then quotiented
by a relation known as structural congruence and it is over this set
that the notion of dynamics is expressed. This presentation is
essentially that of \cite{MeredithR05} with the addition of
polyadicity and summation. For readability we have relegated some of
the technical subtleties to an appendix.

\subsubsection{Process grammar}\label{subsub:process_grammar}

\begin{mathpar}
  \inferrule* [lab=synchronization] {} {{M} \bc \pzero \;|\; x?F \;|\; x!C }
  \and
  \inferrule* [lab=abstraction] {} {{F} \bc (x)P}
  \and
  \inferrule* [lab=concretion] {} {{C} \bc \langle Q \rangle}
  \and
  \inferrule* [lab=process] {} {{P,Q} \bc M \;| \;P|Q \;|\; @{x}}
  \and
  \inferrule* [lab=name] {} {{x} \bc \quotep{P}}
\end{mathpar} 

Note that $\vec{x}$ (resp. $\vec{P}$) denotes a vector of names
(resp. processes) of length $|\vec{x}|$ (resp. $|\vec{P}|$). We adopt
the following useful abbreviations.

\begin{mathpar}
   x?(\vec{y}).P := x.(\vec{y})P \and  x\clift{\vec{P}} := x.\clift{\vec{P}}
   \and x!(y) := \lift{x}{\dropn{y}}
   \and \Pi_{i=0}^{n-1}P_i := P_0 | \ldots | P_{n-1}
\end{mathpar}

\subsubsection{Structural congruence}

\paragraph{Free and bound names and alpha-equivalence.} At the
core of structural equivalence is alpha-equivalence which identifies
process that are the same up to a change of variable. Formally, we
recognize the distinction between free and bound names. The free names
of a process, $\freenames{P}$, may be calculated recursively as
follows:

\begin{mathpar}
\freenames{\pzero} := \emptyset
  \and \\
  \freenames{x?(y).P} := \{ x \} \cup (\freenames{P} \setminus \{ y \})
  \and 
  \freenames{x!\langle P \rangle} := \{ x \} \cup \{ P \} 
  \and \\
  \freenames{P|Q} := \freenames{P} \cup \freenames{Q}
  \and \\
  \freenames{@{x}} := \{ x \}
\end{mathpar}

$\pi$
$\quotep{\pi}$

$\freenames{-} : \pi \to \mathcal{P}(\quotep{\pi})$

\begin{eqnarray*}
  \freenames{\pzero} & := & \emptyset \\
  \freenames{x?(y).P} & := & \{ x \} \cup (\freenames{P} \setminus \{ y \}) \\
  \freenames{x!\langle P \rangle} & := & \{ x \} \cup \{ P \} \\
  \freenames{P|Q} & := & \freenames{P} \cup \freenames{Q} \\
  \freenames{\dropn{x}} & := & \{ x \}
\end{eqnarray*}

The bound names of a process, $\boundnames{P}$, are those names occurring in $P$
that are not free. For example, in $x?(y).0$, the name $x$ is free, while $y$ is bound.

\begin{mathpar}
  \inferrule* [lab=monoidal-laws] {} { P|Q \equiv Q|P \and P|0 \equiv P \and P|(Q|R) \equiv (P|Q)|R }
\end{mathpar}

\begin{mathpar}
  \inferrule* [lab=alpha-equivalence] {} { (x)P \equiv (y)P\{y/x\} \and y \not\in \freenames{P} }
\end{mathpar}

\begin{definition}
Then two processes, $P,Q$, are alpha-equivalent if $P = Q\{\vec{y}/\vec{x}\}$ for
some $\vec{x} \in \boundnames{Q},\vec{y} \in \boundnames{P}$, where $Q\{\vec{y}/\vec{x}\}$
denotes the capture-avoiding substitution of $\vec{y}$ for $\vec{x}$ in $Q$.
\end{definition}

\begin{definition}
  The {\em structural congruence} \cite{SangiorgiWalker} , $\equiv$,
  between processes is the least congruence containing
  alpha-equivalence, satisfying the abelian monoid laws
  (associativity, commutativity and $\pzero$ as identity) for parallel
  composition $|$ and for summation $+$.
\end{definition}

\subsection{Name equivalence}

We take name equivalence, written $\nameeq$, to be the smallest
equivalence relation generated by the following rules.

\begin{mathpar}
\inferrule*[lab=Quote-drop]
{ }
{ \quotep{@{x}} \nameeq x }

\inferrule*[lab=Struct-equiv]
{ P \scong Q }
{ \quotep{P} \nameeq \quotep{Q} }
\end{mathpar}

The astute reader will have noticed that the mutual recursion of names
and processes imposes a mutual recursion on alpha-equivalence and
structural equivalence via name-equivalence. Fortunately, all of this
works out pleasantly and we may calculate in the natural way, free of
concern. The reader interested in the details is referred to the
appendix \ref{appendix:rho_details}.

\subsection{Substitution}

We use $\Proc$ for the set of processes, $\QProc$ for the set of
names, and $\id{\{}\vec{y} / \vec{x} \id{\}}$ to denote partial maps,
$s : \QProc \rightarrow \QProc$. A map, $s$ lifts, uniquely, to a map
on process terms, $\widehat{s} : \Proc \rightarrow \Proc$ by the
following equations.

\begin{mathpar}
  (0) \psubstp{Q}{P} := 0 \\
  (R \juxtap S) \psubstp{Q}{P}
  :=    
  (R)\psubstp{Q}{P} \juxtap (S) \psubstp{Q}{P} \\
  (x?(y).R) \psubstp{Q}{P}    
  :=    
  (x)\substp{Q}{P} (z)\concat( (R \psubstn{z}{y}) \psubstp{Q}{P} ) \\
  (\lift{x}{R}) \psubstp{Q}{P}  
  :=
  \lift{(x)\substp{Q}{P}}{ R \psubstp{Q}{P} } \\
%   (\dropn{x})  \psubstp{Q}{P}       
%   := 
%   \left\{ 
%     \begin{array}{ccc} 
%       \dropn{\quotep{Q}} & & x \nameeq \quotep{P} \\
%       \dropn{x} & & otherwise \\
%     \end{array}
%   \right. 
  (\dropn{x})  \psubstp{Q}{P}       
  := 
  \left\{ 
    \begin{array}{ccc} 
      Q & & x \nameeq \quotep{P} \\
      \dropn{x} & & otherwise \\
    \end{array}
  \right.
\end{mathpar}
 

where

\begin{eqnarray}
  (x)\id{\{} \lpquote Q \rpquote / \lpquote P \rpquote \id{\}}            = 
  \left\{ 
    \begin{array}{ccc}
      \lpquote Q \rpquote & & x \nameeq \lpquote P \rpquote \\
      x & & otherwise \\
    \end{array}
  \right. \nonumber
\end{eqnarray}

and $z$ is chosen distinct from $\quotep{P}$, $\quotep{Q}$, the free
names in $Q$, and all the names in $R$. Our $\alpha$-equivalence will
be built in the standard way from this substitution.

\begin{remark}\label{rem:no_self_referential_names}
  One consequence of these definitions is that $\forall P. \quotep{P}
  \not\in \freenames{P}$.
\end{remark}

\subsection{ Dynamic quote: an example }

Anticipating something of what's to come, consider applying the
substitution, $\widehat{\id{\{}u / z \id{\}}}$, to the following pair
of processes, $\lift{w}{y!(z)}$ and $w[ \lpquote y!(z) \rpquote ]$.

\begin{eqnarray}
	\lift{w}{y!(z)}\widehat{\id{\{}u / z \id{\}}}
		& = &
		\lift{w}{y!(u)} \nonumber\\
	w[ \lpquote y!(z) \rpquote ] \widehat{ \id{\{}u / z \id{\}} }
		& = &
		w[ \lpquote y!(z) \rpquote ] \nonumber
\end{eqnarray}

Because the body of the process between quotes is impervious to
substitution, we get radically different answers. In fact, by
examining the first process in an input context,
e.g. $x?(z).\lift{w}{y!(z)}$, we see that the process under the lift
operator may be shaped by prefixed inputs binding a name inside it. In
this sense, the lift operator will be seen as a way to dynamically
construct processes before reifying them as names.

Finally equipped with these standard features we can present the
dynamics of the calculus.

\subsubsection{Operational semantics} 

Finally, we introduce the computational dynamics. What marks these
algebras as distinct from other more traditionally studied algebraic
structures, e.g. vector spaces or polynomial rings, is the manner in
which dynamics is captured. In traditional structures, dynamics is typically
expressed through morphisms between such structures, as in linear maps
between vector spaces or morphisms between rings. In algebras
associated with the semantics of computation, the dynamics is
expressed as part of the algebraic structure itself, through a
reduction reduction relation typically denoted by $\red$. Below, we
give a recursive presentation of this relation for the calculus used
in the encoding.

$\red \subseteq \pi \times \pi$
$\red : \pi \to \mathcal{P}(\pi)$

\begin{mathpar}
  \inferrule* [lab=Comm] { \textsf{match}( x_{src}, x_{trgt} ) } { x_{trgt}?(y)P \; | \; x_{src}!\langle {Q} \rangle \red P\{\quotep{Q}/y}\} }
  \and \\
  \inferrule* [lab=Par] {{P} \red {P}'} {{{P} | {Q}} \red {{P}' | {Q}}}
  \and
  \inferrule* [lab=Equiv]{{{P} \scong {P}'} \andalso {{P}' \red {Q}'} \andalso {{Q}' \scong {Q}}}{{P} \red {Q}}
\end{mathpar}

\begin{eqnarray*}
  match_{\equiv} (\quotep{P},\quotep{Q}) & := & P \equiv Q \\
  match_{\dagger}(\quotep{P},\quotep{Q}) & := & \forall R. P|Q \red^{*} R => R \red^{*} 0 \\
  match_{K}(\quotep{P},\quotep{Q}) & := & K \mbox{ for some context } K
\end{eqnarray*}

$u?(x)P | u!\langle Q \rangle \red P\{\quotep{Q}/x\}$

%We write $\wred$ for $\red^*$, and $P\red$ if $\exists Q $ such that $ P \red Q$.
We write $P\red$ if $\exists Q $ such that $ P \red Q$ and $P\not\red$, otherwise.

\section{Replication}

As mentioned before, it is known that replication (and hence
recursion) can be implemented in a higher-order process algebra
\cite{SangiorgiWalker}. As our first example of calculation with the
machinery thus far presented we give the construction explicitly in
the {\rhoc}.

\begin{eqnarray}
	D_{x} & := & \prefix{x}{y}{(\binpar{\outputp{x}{y}}{@{y}})} \nonumber\\
	\bangp_{x}{P} & := & \binpar{{x}!\langle{\binpar{D_{x}}{P}}\rangle}{D_{x}} \nonumber
\end{eqnarray}

\begin{eqnarray}
	\bangp_{x}{P} & & \nonumber\\
	=
	& {x}!\langle{(\prefix{x}{y}{(\outputp{x}{y} | @{y})) | P}}\rangle 
	      | \prefix{x}{y}{(\outputp{x}{y} | @{y})} & \nonumber\\
	\red
	& (\outputp{x}{y} | @{y})\substn{\quotep{(\prefix{x}{y}{(@{y} | \outputp{x}{y})) | P}}}{y} & \nonumber\\
	=
	& \outputp{x}{\quotep{(\prefix{x}{y}{(\outputp{x}{y} | @{y})) | P}}}
	  | {(\prefix{x}{y}{(\outputp{x}{y} | @{y})) | P}} & \nonumber\\
	\red
	& \ldots & \nonumber\\
	\red^*
	& P | P | \ldots & \nonumber
\end{eqnarray}

Of course, this encoding, as an implementation, runs away, unfolding
$\bangp{P}$ eagerly. A lazier and more implementable replication
operator, restricted to input-guarded processes, may be obtained as follows.

\begin{eqnarray}
\bangp{\prefix{u}{v}{P}} 
	:= 
	\binpar{\lift{x}{\prefix{u}{v}{(\binpar{D(x)}{P})}}}{D(x)} \nonumber
\end{eqnarray}

\begin{remark}
  Note that the lazier definition still does not deal with summation
  or mixed summation (i.e. sums over input and output). The reader is
  invited to construct definitions of replication that deal with these
  features. 

  Further, the definitions are parameterized in a name, $x$. Can you,
  gentle reader, make a definition that eliminates this parameter and
  guarantees no accidental interaction between the replication
  machinery and the process being replicated -- i.e. no accidental
  sharing of names used by the process to get its work done and the
  name(s) used by the replication to effect copying. This latter
  revision of the definition of replication is crucial to obtaining
  the expected identity $!!P \sim !P$.
\end{remark}

\begin{remark}\label{rem:paradoxical_combinator}
  The reader familiar with the lambda calculus will have noticed the
  similarity between $D$ and the paradoxical combinator.

  [Ed. note: the existence of this seems to suggest we have to be more
  restrictive on the set of processes and names we admit if we are to
  support no-cloning.]
\end{remark}

\subsubsection{Bisimulation}

The computational dynamics gives rise to another kind of equivalence,
the equivalence of computational behavior. As previously mentioned
this is typically captured \emph{via} some form of bisimulation.

% The notion we use in this paper is weak barbed bisimulation
% \cite{milner91polyadicpi}.

The notion we use in this paper is derived from weak barbed
bisimulation \cite{milner91polyadicpi}. 

\begin{definition}
An \emph{observation relation}, $\downarrow_{\mathcal N}$, over a set
of names, $\mathcal N$, is the smallest relation satisfying the rules
below.

\infrule[Out-barb]{y \in {\mathcal N}, \; x \nameeq y}
		  {\outputp{x}{v} \downarrow_{\mathcal N} x}
\infrule[Par-barb]{\mbox{$P\downarrow_{\mathcal N} x$ or $Q\downarrow_{\mathcal N} x$}}
		  {\binpar{P}{Q} \downarrow_{\mathcal N} x}

We write $P \Downarrow_{\mathcal N} x$ if there is $Q$ such that 
$P \wred Q$ and $Q \downarrow_{\mathcal N} x$.
\end{definition}

\begin{definition}
%\label{def.bbisim}
An  ${\mathcal N}$-\emph{barbed bisimulation} over a set of names, ${\mathcal N}$, is a symmetric binary relation 
${\mathcal S}_{\mathcal N}$ between agents such that $P\rel{S}_{\mathcal N}Q$ implies:
\begin{enumerate}
\item If $P \red P'$ then $Q \wred Q'$ and $P'\rel{S}_{\mathcal N} Q'$.
\item If $P\downarrow_{\mathcal N} x$, then $Q\Downarrow_{\mathcal N} x$.
\end{enumerate}
$P$ is ${\mathcal N}$-barbed bisimilar to $Q$, written
$P \wbbisim_{\mathcal N} Q$, if $P \rel{S}_{\mathcal N} Q$ for some ${\mathcal N}$-barbed bisimulation ${\mathcal S}_{\mathcal N}$.
\end{definition}

$\mathcal{R} \subseteq \pi \times \pi$

$P \mathcal{R} Q => \forall P'. P \red P' \Rightarrow \exists Q'. Q \red Q', P' \mathcal{R} Q'$

$P \vdash x \Rightarrow Q \vdash x$

\begin{mathpar}
  \inferrule*[lab=Out-barb]{x \nameeq y}{{y}!\langle{Q}\rangle \vdash x}
  \and
  \inferrule*[lab=Par-barb]{\mbox{$P\vdash x$ or $Q\vdash x$}}{\binpar{P}{Q} \vdash x}
\end{mathpar}

\subsubsection{Contexts}

One of the principle advantages of computational calculi like the
$\pi$-calculus is a well-defined notion of context,
contextual-equivalence and a correlation between
contextual-equivalence and notions of bisimulation. The notion of
context allows the decomposition of a process into (sub-)process and
its syntactic environment, its context. Thus, a context may be
thought of as a process with a ``hole'' (written $\Box$) in it. The
application of a context $M$ to a process $P$, written $M[P]$, is
tantamount to filling the hole in $M$ with $P$. In this paper we do
not need the full weight of this theory, but do make use of the notion
of context in the proof the main theorem. 

\begin{mathpar}
  \inferrule* [lab=summation] {} {{M_{M},M_{N}} \bc \Box \;|\; x.M_{A} \;|\; M_{M}+M_{N}}
  \and
  \inferrule* [lab=agent] {} {{M_{A}} \bc (\vec{x})M_{P} \;| \; \clift{P_0,\ldots,M_{P},\ldots,P_N}}
  \and \\
  \inferrule* [lab=process] {} {{M_{P}} \bc M_{N} \;| \;P|M_{P} }
\end{mathpar} 

\begin{mathpar}
  \inferrule* [lab=sychronization] {} {M_{N} \bc \Box \;|\; x?M_{F} \;|\; x!M_{C}}
  \and
  \inferrule* [lab=abstraction] {} {{M_{F}} \bc (x)M_{P} }
  \and
  \inferrule* [lab=concretion] {} {{M_{C}} \bc \langle M_{P} \rangle }
  \and \\
  \inferrule* [lab=process] {} {{M_{P}} \bc M_{N} \;| \;P|M_{P} }
\end{mathpar}

\begin{definition}[contextual application] Given a context $M$, and
  process $P$, we define the \emph{contextual application}, $M[P] :=
  M\{P/\Box\}$. That is, the contextual application of M to P is the
  substitution of $P$ for $\Box$ in $M$.
\end{definition}

$\meaningof{-} : L \to \mathcal{P}(\pi)$

\begin{mathpar}
  \inferrule* [lab=collection] {} {\meaningof{true} = \pi, \and \meaningof{~E} = \pi \setminus \meaningof{E}, \and \meaningof{E_{1} \& E_{2}} = \meaningof{E_{1}} \cap \meaningof{E_{2}}}
\end{mathpar}

\begin{mathpar}
  \inferrule* [lab=structure] {} {\meaningof{0} = \{ P \in \pi | P \equiv 0 \}, \and \\ \meaningof{E_1 | E_2} = \{ P \in \pi | P \equiv P_{1} | P_{2}, P_{1} \in \meaningof{E_{1}}, P_{2} \in \meaningof{E_2}\} }
\end{mathpar}

\begin{mathpar}
 \inferrule* [lab=behavior] {} {\meaningof{\langle a?b \rangle E} = \{ P \in \pi | P \equiv Q | u?(y)P', \\ \and \\\\ \and \\ \;\;\; u \in \meaningof{a}, \forall z.P'\{z/y\} \in \meaningof{E\{z/b\}}\}, \and \\ \meaningof{a!E} = \{ P \in \pi | P \equiv Q | x!\langle P' \rangle, x \in \meaningof{a} P' \in \meaningof{E}\} }
\end{mathpar}

\begin{mathpar}
 \inferrule* [lab=nominal] {} {\meaningof{\quotep{E}} = \{ \quotep{P} \in \quotep{\pi} | P \in \meaningof{E} \}, \and \meaningof{\quotep{P}} = \{ \quotep{Q} \in \quotep{\pi} | P \equiv Q \} \and \\ \meaningof{@\quotep{E}} = \{ P \in \pi | P \equiv @x, x \in \meaningof{E} \}}
\end{mathpar}

\begin{eqnarray*}
  \\
  \meaningof{-} : TS \to ST
\end{eqnarray*}

\begin{eqnarray*}
  \\
  L : TS \to ST
\end{eqnarray*}

\begin{eqnarray*}
  \\
  P \models E \iff P \in \meaningof{E}
\end{eqnarray*}

\begin{eqnarray*}
  P \approx_{L} Q \iff \forall E \in L. P \models E \iff Q \models E
\end{eqnarray*}

\begin{eqnarray*}
  P \approx_{K} Q
\end{eqnarray*}

\begin{eqnarray*}
  P \approx Q
\end{eqnarray*}

$\approx_{K} = \approx = \approx_{L}$

\subsubsection{Contextual duality}

Note that contexts extend the quotation operation to a family of
operations from processes to names. Given a context, $M$, we can
define a \emph{nominal context}, $\quotep{M}$ by $\quotep{M}[P] :=
\quotep{M[P]}$. To foreshadow what is to come we observe that these
operations enjoy a duality with processes very much like the duality
between vectors and maps from vectors to scalars.

Further, because the calculus is essentially higher-order, we have a
correspondence between contexts and processes. More specifically,
given a name $x$ and a context $M$ we can construct $M^{*}_{x}$ such
that 

\begin{mathpar}
  M^{*}_{x} | \lift{x}{P} \red M[P]
\end{mathpar}

namely,

\begin{mathpar}
  M^{*}_{x} := x?(u).M[\dropn{u}]
\end{mathpar}

The dependence of $M^{*}_{x}$ on a name makes it an abstraction, 

\begin{mathpar}
  M^{*} := (x)x?(u).M[\dropn{u}]
\end{mathpar}

\subsection{Additional notation}

It will sometimes be convenient to denote the process a name
quotes. We already have the notation $x = \quotep{P}$, but it will be
convenient to introduce an alternate notation, $\procn{x}$, when we
want to emphasize the connection to the use of the name. Note that, by
virtue of name equivalence, $\quotep{\procn{x}} \nameeq x$; so, the
notation is consistent with previous definitions.

Further, because names have structure it is possible to effect
substitutions on the basis of that structure. This means we need to
upgrade our notation for substitutions, which we accomplish by
adapting comprehension notation. Thus,

\begin{mathpar}
  P\{ y / x : x \in S \}
\end{mathpar}

is interpreted to mean the process derived from P by replacing (in a
capture-avoiding manner) each occurrence of $x$ in $S$ by $y$. For example,

\begin{mathpar}
  P\{ \quotep{\procn{x}|\procn{x}} / x : x \in \freenames{P} \}
\end{mathpar}

will replace each (occurrence) of a free name $x$ in $P$ by
$\quotep{\procn{x}|\procn{x}}$.

Also, we will avail ourselves of the notation $x^{L}$ and $x^{R}$ to
denote injections of a name into disjoint copies of the name
space. There are numerous ways to accomplish this. One example can be
found in \cite{MeredithR05}. This notation overloads to vectors of
names: $\vec{x}^{\pi} := (x_{i}^{\pi} \; : \; 0 \leq i < |\vec{x}| )$ where $\pi \in \{L,R\}$.

We also use $P^{\Box} := P|\Box$.

In \cite{MeredithR05} an interpretation of the new operator is
given. It turns out that there are several possible interpretations
all enjoying the requisite algebraic properties of the operator (see
\cite{milner91polyadicpi}). We will therefore make liberal use of
$(\nu\; \vec{x})P$.

% subsection the_syntax_and_semantics_of_the_notation_system (end)   

\input{qm2pi.qmops} 

\input{qm2pi.sterngerlach} 

\input{qm2pi.metric} 

% section concurrent_process_calculi (end)

%\input{qm2pi.proofsketch}

% section proof sketch (end)

%\input{qm2pi.slviaknots} 

% section spatial logic via knots (end)

\input{qm2pi.conclusion}

% section conclusion (end)

%\input{qm2pi.dtcodes} 

% section wiring algorithm (end)

\input{qm2pi.ack} 

% section acknowledgments (end)

\newpage


\bibliographystyle{plain}   
\bibliography{../../biblios/main.bib}

\input{qm2pi.rhodetails}

\end{document}

 

% section concurrent_process_calculi (end)

%\documentclass[12pt]{llncs}
%\documentclass{jktr}

\usepackage[pdftex]{hyperref}                   
\usepackage {listings}
\usepackage {mathpartir}
\usepackage{bcprules}
%\usepackage{listings}
                       
\usepackage{graphicx} 
%\usepackage[margins=2.5cm,nohead,nofoot]{geometry}
%\usepackage{geometry}
\usepackage{amsfonts}
\usepackage{amstext}
\usepackage{latexsym}
\usepackage{amssymb}
\usepackage{color}


%\include{myPreamble}
\include{qm2pi.local} 

%\ifpdf
%\usepackage[pdftex]{graphicx}
%\else
%\usepackage{graphicx}
%\fi

 % \ifpdf
%  \usepackage{pdfsync}
%  \if


%\title{Brief Article}
%\author{David F. Snyder}
%\author{L.G. Meredith}

%\address{Dept. of Math., Texas State University--San Marcos, San Marcos, TX 78666}
       
\pagestyle{empty}


\begin{document}

\lstset{language=[Objective]Caml,frame=shadowbox}

\input{qm2pi.front}

% section front matter (end)

\input{qm2pi.intro} 
 
% section introduction (end)

% \input{qm2pi.knotations} 

% section notation (end)

\input{qm2pi.process.calculi} 

% section concurrent_process_calculi_and_spatial_logics_ (end)
    
%\input{qm2pi.knots2pi} 

%\input{qm2pi.trefoil} 

%\input{qm2pi.mainthm} 

% subsection basic_interpretation (end)

%\input{qm2pi.rho.presentation} 
\subsection{The syntax and semantics of the notation system}\label{sub:the_syntax_and_semantics_of_the_notation_system} % (fold)

We now summarize a technical presentation of the calculus that
embodies our theory of dynamics. The typical presentation of such a
calculus follows the style of giving generators and relations on
them. The grammar, below, describing term constructors, freely
generates the set of processes, $\Proc$. This set is then quotiented
by a relation known as structural congruence and it is over this set
that the notion of dynamics is expressed. This presentation is
essentially that of \cite{MeredithR05} with the addition of
polyadicity and summation. For readability we have relegated some of
the technical subtleties to an appendix.

\subsubsection{Process grammar}\label{subsub:process_grammar}

\begin{mathpar}
  \inferrule* [lab=synchronization] {} {{M} \bc \pzero \;|\; x?F \;|\; x!C }
  \and
  \inferrule* [lab=abstraction] {} {{F} \bc (x)P}
  \and
  \inferrule* [lab=concretion] {} {{C} \bc \langle Q \rangle}
  \and
  \inferrule* [lab=process] {} {{P,Q} \bc M \;| \;P|Q \;|\; @{x}}
  \and
  \inferrule* [lab=name] {} {{x} \bc \quotep{P}}
\end{mathpar} 

Note that $\vec{x}$ (resp. $\vec{P}$) denotes a vector of names
(resp. processes) of length $|\vec{x}|$ (resp. $|\vec{P}|$). We adopt
the following useful abbreviations.

\begin{mathpar}
   x?(\vec{y}).P := x.(\vec{y})P \and  x\clift{\vec{P}} := x.\clift{\vec{P}}
   \and x!(y) := \lift{x}{\dropn{y}}
   \and \Pi_{i=0}^{n-1}P_i := P_0 | \ldots | P_{n-1}
\end{mathpar}

\subsubsection{Structural congruence}

\paragraph{Free and bound names and alpha-equivalence.} At the
core of structural equivalence is alpha-equivalence which identifies
process that are the same up to a change of variable. Formally, we
recognize the distinction between free and bound names. The free names
of a process, $\freenames{P}$, may be calculated recursively as
follows:

\begin{mathpar}
\freenames{\pzero} := \emptyset
  \and \\
  \freenames{x?(y).P} := \{ x \} \cup (\freenames{P} \setminus \{ y \})
  \and 
  \freenames{x!\langle P \rangle} := \{ x \} \cup \{ P \} 
  \and \\
  \freenames{P|Q} := \freenames{P} \cup \freenames{Q}
  \and \\
  \freenames{@{x}} := \{ x \}
\end{mathpar}

$\pi$
$\quotep{\pi}$

$\freenames{-} : \pi \to \mathcal{P}(\quotep{\pi})$

\begin{eqnarray*}
  \freenames{\pzero} & := & \emptyset \\
  \freenames{x?(y).P} & := & \{ x \} \cup (\freenames{P} \setminus \{ y \}) \\
  \freenames{x!\langle P \rangle} & := & \{ x \} \cup \{ P \} \\
  \freenames{P|Q} & := & \freenames{P} \cup \freenames{Q} \\
  \freenames{\dropn{x}} & := & \{ x \}
\end{eqnarray*}

The bound names of a process, $\boundnames{P}$, are those names occurring in $P$
that are not free. For example, in $x?(y).0$, the name $x$ is free, while $y$ is bound.

\begin{mathpar}
  \inferrule* [lab=monoidal-laws] {} { P|Q \equiv Q|P \and P|0 \equiv P \and P|(Q|R) \equiv (P|Q)|R }
\end{mathpar}

\begin{mathpar}
  \inferrule* [lab=alpha-equivalence] {} { (x)P \equiv (y)P\{y/x\} \and y \not\in \freenames{P} }
\end{mathpar}

\begin{definition}
Then two processes, $P,Q$, are alpha-equivalent if $P = Q\{\vec{y}/\vec{x}\}$ for
some $\vec{x} \in \boundnames{Q},\vec{y} \in \boundnames{P}$, where $Q\{\vec{y}/\vec{x}\}$
denotes the capture-avoiding substitution of $\vec{y}$ for $\vec{x}$ in $Q$.
\end{definition}

\begin{definition}
  The {\em structural congruence} \cite{SangiorgiWalker} , $\equiv$,
  between processes is the least congruence containing
  alpha-equivalence, satisfying the abelian monoid laws
  (associativity, commutativity and $\pzero$ as identity) for parallel
  composition $|$ and for summation $+$.
\end{definition}

\subsection{Name equivalence}

We take name equivalence, written $\nameeq$, to be the smallest
equivalence relation generated by the following rules.

\begin{mathpar}
\inferrule*[lab=Quote-drop]
{ }
{ \quotep{@{x}} \nameeq x }

\inferrule*[lab=Struct-equiv]
{ P \scong Q }
{ \quotep{P} \nameeq \quotep{Q} }
\end{mathpar}

The astute reader will have noticed that the mutual recursion of names
and processes imposes a mutual recursion on alpha-equivalence and
structural equivalence via name-equivalence. Fortunately, all of this
works out pleasantly and we may calculate in the natural way, free of
concern. The reader interested in the details is referred to the
appendix \ref{appendix:rho_details}.

\subsection{Substitution}

We use $\Proc$ for the set of processes, $\QProc$ for the set of
names, and $\id{\{}\vec{y} / \vec{x} \id{\}}$ to denote partial maps,
$s : \QProc \rightarrow \QProc$. A map, $s$ lifts, uniquely, to a map
on process terms, $\widehat{s} : \Proc \rightarrow \Proc$ by the
following equations.

\begin{mathpar}
  (0) \psubstp{Q}{P} := 0 \\
  (R \juxtap S) \psubstp{Q}{P}
  :=    
  (R)\psubstp{Q}{P} \juxtap (S) \psubstp{Q}{P} \\
  (x?(y).R) \psubstp{Q}{P}    
  :=    
  (x)\substp{Q}{P} (z)\concat( (R \psubstn{z}{y}) \psubstp{Q}{P} ) \\
  (\lift{x}{R}) \psubstp{Q}{P}  
  :=
  \lift{(x)\substp{Q}{P}}{ R \psubstp{Q}{P} } \\
%   (\dropn{x})  \psubstp{Q}{P}       
%   := 
%   \left\{ 
%     \begin{array}{ccc} 
%       \dropn{\quotep{Q}} & & x \nameeq \quotep{P} \\
%       \dropn{x} & & otherwise \\
%     \end{array}
%   \right. 
  (\dropn{x})  \psubstp{Q}{P}       
  := 
  \left\{ 
    \begin{array}{ccc} 
      Q & & x \nameeq \quotep{P} \\
      \dropn{x} & & otherwise \\
    \end{array}
  \right.
\end{mathpar}
 

where

\begin{eqnarray}
  (x)\id{\{} \lpquote Q \rpquote / \lpquote P \rpquote \id{\}}            = 
  \left\{ 
    \begin{array}{ccc}
      \lpquote Q \rpquote & & x \nameeq \lpquote P \rpquote \\
      x & & otherwise \\
    \end{array}
  \right. \nonumber
\end{eqnarray}

and $z$ is chosen distinct from $\quotep{P}$, $\quotep{Q}$, the free
names in $Q$, and all the names in $R$. Our $\alpha$-equivalence will
be built in the standard way from this substitution.

\begin{remark}\label{rem:no_self_referential_names}
  One consequence of these definitions is that $\forall P. \quotep{P}
  \not\in \freenames{P}$.
\end{remark}

\subsection{ Dynamic quote: an example }

Anticipating something of what's to come, consider applying the
substitution, $\widehat{\id{\{}u / z \id{\}}}$, to the following pair
of processes, $\lift{w}{y!(z)}$ and $w[ \lpquote y!(z) \rpquote ]$.

\begin{eqnarray}
	\lift{w}{y!(z)}\widehat{\id{\{}u / z \id{\}}}
		& = &
		\lift{w}{y!(u)} \nonumber\\
	w[ \lpquote y!(z) \rpquote ] \widehat{ \id{\{}u / z \id{\}} }
		& = &
		w[ \lpquote y!(z) \rpquote ] \nonumber
\end{eqnarray}

Because the body of the process between quotes is impervious to
substitution, we get radically different answers. In fact, by
examining the first process in an input context,
e.g. $x?(z).\lift{w}{y!(z)}$, we see that the process under the lift
operator may be shaped by prefixed inputs binding a name inside it. In
this sense, the lift operator will be seen as a way to dynamically
construct processes before reifying them as names.

Finally equipped with these standard features we can present the
dynamics of the calculus.

\subsubsection{Operational semantics} 

Finally, we introduce the computational dynamics. What marks these
algebras as distinct from other more traditionally studied algebraic
structures, e.g. vector spaces or polynomial rings, is the manner in
which dynamics is captured. In traditional structures, dynamics is typically
expressed through morphisms between such structures, as in linear maps
between vector spaces or morphisms between rings. In algebras
associated with the semantics of computation, the dynamics is
expressed as part of the algebraic structure itself, through a
reduction reduction relation typically denoted by $\red$. Below, we
give a recursive presentation of this relation for the calculus used
in the encoding.

$\red \subseteq \pi \times \pi$
$\red : \pi \to \mathcal{P}(\pi)$

\begin{mathpar}
  \inferrule* [lab=Comm] { \textsf{match}( x_{src}, x_{trgt} ) } { x_{trgt}?(y)P \; | \; x_{src}!\langle {Q} \rangle \red P\{\quotep{Q}/y}\} }
  \and \\
  \inferrule* [lab=Par] {{P} \red {P}'} {{{P} | {Q}} \red {{P}' | {Q}}}
  \and
  \inferrule* [lab=Equiv]{{{P} \scong {P}'} \andalso {{P}' \red {Q}'} \andalso {{Q}' \scong {Q}}}{{P} \red {Q}}
\end{mathpar}

\begin{eqnarray*}
  match_{\equiv} (\quotep{P},\quotep{Q}) & := & P \equiv Q \\
  match_{\dagger}(\quotep{P},\quotep{Q}) & := & \forall R. P|Q \red^{*} R => R \red^{*} 0 \\
  match_{K}(\quotep{P},\quotep{Q}) & := & K \mbox{ for some context } K
\end{eqnarray*}

$u?(x)P | u!\langle Q \rangle \red P\{\quotep{Q}/x\}$

%We write $\wred$ for $\red^*$, and $P\red$ if $\exists Q $ such that $ P \red Q$.
We write $P\red$ if $\exists Q $ such that $ P \red Q$ and $P\not\red$, otherwise.

\section{Replication}

As mentioned before, it is known that replication (and hence
recursion) can be implemented in a higher-order process algebra
\cite{SangiorgiWalker}. As our first example of calculation with the
machinery thus far presented we give the construction explicitly in
the {\rhoc}.

\begin{eqnarray}
	D_{x} & := & \prefix{x}{y}{(\binpar{\outputp{x}{y}}{@{y}})} \nonumber\\
	\bangp_{x}{P} & := & \binpar{{x}!\langle{\binpar{D_{x}}{P}}\rangle}{D_{x}} \nonumber
\end{eqnarray}

\begin{eqnarray}
	\bangp_{x}{P} & & \nonumber\\
	=
	& {x}!\langle{(\prefix{x}{y}{(\outputp{x}{y} | @{y})) | P}}\rangle 
	      | \prefix{x}{y}{(\outputp{x}{y} | @{y})} & \nonumber\\
	\red
	& (\outputp{x}{y} | @{y})\substn{\quotep{(\prefix{x}{y}{(@{y} | \outputp{x}{y})) | P}}}{y} & \nonumber\\
	=
	& \outputp{x}{\quotep{(\prefix{x}{y}{(\outputp{x}{y} | @{y})) | P}}}
	  | {(\prefix{x}{y}{(\outputp{x}{y} | @{y})) | P}} & \nonumber\\
	\red
	& \ldots & \nonumber\\
	\red^*
	& P | P | \ldots & \nonumber
\end{eqnarray}

Of course, this encoding, as an implementation, runs away, unfolding
$\bangp{P}$ eagerly. A lazier and more implementable replication
operator, restricted to input-guarded processes, may be obtained as follows.

\begin{eqnarray}
\bangp{\prefix{u}{v}{P}} 
	:= 
	\binpar{\lift{x}{\prefix{u}{v}{(\binpar{D(x)}{P})}}}{D(x)} \nonumber
\end{eqnarray}

\begin{remark}
  Note that the lazier definition still does not deal with summation
  or mixed summation (i.e. sums over input and output). The reader is
  invited to construct definitions of replication that deal with these
  features. 

  Further, the definitions are parameterized in a name, $x$. Can you,
  gentle reader, make a definition that eliminates this parameter and
  guarantees no accidental interaction between the replication
  machinery and the process being replicated -- i.e. no accidental
  sharing of names used by the process to get its work done and the
  name(s) used by the replication to effect copying. This latter
  revision of the definition of replication is crucial to obtaining
  the expected identity $!!P \sim !P$.
\end{remark}

\begin{remark}\label{rem:paradoxical_combinator}
  The reader familiar with the lambda calculus will have noticed the
  similarity between $D$ and the paradoxical combinator.

  [Ed. note: the existence of this seems to suggest we have to be more
  restrictive on the set of processes and names we admit if we are to
  support no-cloning.]
\end{remark}

\subsubsection{Bisimulation}

The computational dynamics gives rise to another kind of equivalence,
the equivalence of computational behavior. As previously mentioned
this is typically captured \emph{via} some form of bisimulation.

% The notion we use in this paper is weak barbed bisimulation
% \cite{milner91polyadicpi}.

The notion we use in this paper is derived from weak barbed
bisimulation \cite{milner91polyadicpi}. 

\begin{definition}
An \emph{observation relation}, $\downarrow_{\mathcal N}$, over a set
of names, $\mathcal N$, is the smallest relation satisfying the rules
below.

\infrule[Out-barb]{y \in {\mathcal N}, \; x \nameeq y}
		  {\outputp{x}{v} \downarrow_{\mathcal N} x}
\infrule[Par-barb]{\mbox{$P\downarrow_{\mathcal N} x$ or $Q\downarrow_{\mathcal N} x$}}
		  {\binpar{P}{Q} \downarrow_{\mathcal N} x}

We write $P \Downarrow_{\mathcal N} x$ if there is $Q$ such that 
$P \wred Q$ and $Q \downarrow_{\mathcal N} x$.
\end{definition}

\begin{definition}
%\label{def.bbisim}
An  ${\mathcal N}$-\emph{barbed bisimulation} over a set of names, ${\mathcal N}$, is a symmetric binary relation 
${\mathcal S}_{\mathcal N}$ between agents such that $P\rel{S}_{\mathcal N}Q$ implies:
\begin{enumerate}
\item If $P \red P'$ then $Q \wred Q'$ and $P'\rel{S}_{\mathcal N} Q'$.
\item If $P\downarrow_{\mathcal N} x$, then $Q\Downarrow_{\mathcal N} x$.
\end{enumerate}
$P$ is ${\mathcal N}$-barbed bisimilar to $Q$, written
$P \wbbisim_{\mathcal N} Q$, if $P \rel{S}_{\mathcal N} Q$ for some ${\mathcal N}$-barbed bisimulation ${\mathcal S}_{\mathcal N}$.
\end{definition}

$\mathcal{R} \subseteq \pi \times \pi$

$P \mathcal{R} Q => \forall P'. P \red P' \Rightarrow \exists Q'. Q \red Q', P' \mathcal{R} Q'$

$P \vdash x \Rightarrow Q \vdash x$

\begin{mathpar}
  \inferrule*[lab=Out-barb]{x \nameeq y}{{y}!\langle{Q}\rangle \vdash x}
  \and
  \inferrule*[lab=Par-barb]{\mbox{$P\vdash x$ or $Q\vdash x$}}{\binpar{P}{Q} \vdash x}
\end{mathpar}

\subsubsection{Contexts}

One of the principle advantages of computational calculi like the
$\pi$-calculus is a well-defined notion of context,
contextual-equivalence and a correlation between
contextual-equivalence and notions of bisimulation. The notion of
context allows the decomposition of a process into (sub-)process and
its syntactic environment, its context. Thus, a context may be
thought of as a process with a ``hole'' (written $\Box$) in it. The
application of a context $M$ to a process $P$, written $M[P]$, is
tantamount to filling the hole in $M$ with $P$. In this paper we do
not need the full weight of this theory, but do make use of the notion
of context in the proof the main theorem. 

\begin{mathpar}
  \inferrule* [lab=summation] {} {{M_{M},M_{N}} \bc \Box \;|\; x.M_{A} \;|\; M_{M}+M_{N}}
  \and
  \inferrule* [lab=agent] {} {{M_{A}} \bc (\vec{x})M_{P} \;| \; \clift{P_0,\ldots,M_{P},\ldots,P_N}}
  \and \\
  \inferrule* [lab=process] {} {{M_{P}} \bc M_{N} \;| \;P|M_{P} }
\end{mathpar} 

\begin{mathpar}
  \inferrule* [lab=sychronization] {} {M_{N} \bc \Box \;|\; x?M_{F} \;|\; x!M_{C}}
  \and
  \inferrule* [lab=abstraction] {} {{M_{F}} \bc (x)M_{P} }
  \and
  \inferrule* [lab=concretion] {} {{M_{C}} \bc \langle M_{P} \rangle }
  \and \\
  \inferrule* [lab=process] {} {{M_{P}} \bc M_{N} \;| \;P|M_{P} }
\end{mathpar}

\begin{definition}[contextual application] Given a context $M$, and
  process $P$, we define the \emph{contextual application}, $M[P] :=
  M\{P/\Box\}$. That is, the contextual application of M to P is the
  substitution of $P$ for $\Box$ in $M$.
\end{definition}

$\meaningof{-} : L \to \mathcal{P}(\pi)$

\begin{mathpar}
  \inferrule* [lab=collection] {} {\meaningof{true} = \pi, \and \meaningof{~E} = \pi \setminus \meaningof{E}, \and \meaningof{E_{1} \& E_{2}} = \meaningof{E_{1}} \cap \meaningof{E_{2}}}
\end{mathpar}

\begin{mathpar}
  \inferrule* [lab=structure] {} {\meaningof{0} = \{ P \in \pi | P \equiv 0 \}, \and \\ \meaningof{E_1 | E_2} = \{ P \in \pi | P \equiv P_{1} | P_{2}, P_{1} \in \meaningof{E_{1}}, P_{2} \in \meaningof{E_2}\} }
\end{mathpar}

\begin{mathpar}
 \inferrule* [lab=behavior] {} {\meaningof{\langle a?b \rangle E} = \{ P \in \pi | P \equiv Q | u?(y)P', \\ \and \\\\ \and \\ \;\;\; u \in \meaningof{a}, \forall z.P'\{z/y\} \in \meaningof{E\{z/b\}}\}, \and \\ \meaningof{a!E} = \{ P \in \pi | P \equiv Q | x!\langle P' \rangle, x \in \meaningof{a} P' \in \meaningof{E}\} }
\end{mathpar}

\begin{mathpar}
 \inferrule* [lab=nominal] {} {\meaningof{\quotep{E}} = \{ \quotep{P} \in \quotep{\pi} | P \in \meaningof{E} \}, \and \meaningof{\quotep{P}} = \{ \quotep{Q} \in \quotep{\pi} | P \equiv Q \} \and \\ \meaningof{@\quotep{E}} = \{ P \in \pi | P \equiv @x, x \in \meaningof{E} \}}
\end{mathpar}

\begin{eqnarray*}
  \\
  \meaningof{-} : TS \to ST
\end{eqnarray*}

\begin{eqnarray*}
  \\
  L : TS \to ST
\end{eqnarray*}

\begin{eqnarray*}
  \\
  P \models E \iff P \in \meaningof{E}
\end{eqnarray*}

\begin{eqnarray*}
  P \approx_{L} Q \iff \forall E \in L. P \models E \iff Q \models E
\end{eqnarray*}

\begin{eqnarray*}
  P \approx_{K} Q
\end{eqnarray*}

\begin{eqnarray*}
  P \approx Q
\end{eqnarray*}

$\approx_{K} = \approx = \approx_{L}$

\subsubsection{Contextual duality}

Note that contexts extend the quotation operation to a family of
operations from processes to names. Given a context, $M$, we can
define a \emph{nominal context}, $\quotep{M}$ by $\quotep{M}[P] :=
\quotep{M[P]}$. To foreshadow what is to come we observe that these
operations enjoy a duality with processes very much like the duality
between vectors and maps from vectors to scalars.

Further, because the calculus is essentially higher-order, we have a
correspondence between contexts and processes. More specifically,
given a name $x$ and a context $M$ we can construct $M^{*}_{x}$ such
that 

\begin{mathpar}
  M^{*}_{x} | \lift{x}{P} \red M[P]
\end{mathpar}

namely,

\begin{mathpar}
  M^{*}_{x} := x?(u).M[\dropn{u}]
\end{mathpar}

The dependence of $M^{*}_{x}$ on a name makes it an abstraction, 

\begin{mathpar}
  M^{*} := (x)x?(u).M[\dropn{u}]
\end{mathpar}

\subsection{Additional notation}

It will sometimes be convenient to denote the process a name
quotes. We already have the notation $x = \quotep{P}$, but it will be
convenient to introduce an alternate notation, $\procn{x}$, when we
want to emphasize the connection to the use of the name. Note that, by
virtue of name equivalence, $\quotep{\procn{x}} \nameeq x$; so, the
notation is consistent with previous definitions.

Further, because names have structure it is possible to effect
substitutions on the basis of that structure. This means we need to
upgrade our notation for substitutions, which we accomplish by
adapting comprehension notation. Thus,

\begin{mathpar}
  P\{ y / x : x \in S \}
\end{mathpar}

is interpreted to mean the process derived from P by replacing (in a
capture-avoiding manner) each occurrence of $x$ in $S$ by $y$. For example,

\begin{mathpar}
  P\{ \quotep{\procn{x}|\procn{x}} / x : x \in \freenames{P} \}
\end{mathpar}

will replace each (occurrence) of a free name $x$ in $P$ by
$\quotep{\procn{x}|\procn{x}}$.

Also, we will avail ourselves of the notation $x^{L}$ and $x^{R}$ to
denote injections of a name into disjoint copies of the name
space. There are numerous ways to accomplish this. One example can be
found in \cite{MeredithR05}. This notation overloads to vectors of
names: $\vec{x}^{\pi} := (x_{i}^{\pi} \; : \; 0 \leq i < |\vec{x}| )$ where $\pi \in \{L,R\}$.

We also use $P^{\Box} := P|\Box$.

In \cite{MeredithR05} an interpretation of the new operator is
given. It turns out that there are several possible interpretations
all enjoying the requisite algebraic properties of the operator (see
\cite{milner91polyadicpi}). We will therefore make liberal use of
$(\nu\; \vec{x})P$.

% subsection the_syntax_and_semantics_of_the_notation_system (end)   

\input{qm2pi.qmops} 

\input{qm2pi.sterngerlach} 

\input{qm2pi.metric} 

% section concurrent_process_calculi (end)

%\input{qm2pi.proofsketch}

% section proof sketch (end)

%\input{qm2pi.slviaknots} 

% section spatial logic via knots (end)

\input{qm2pi.conclusion}

% section conclusion (end)

%\input{qm2pi.dtcodes} 

% section wiring algorithm (end)

\input{qm2pi.ack} 

% section acknowledgments (end)

\newpage


\bibliographystyle{plain}   
\bibliography{../../biblios/main.bib}

\input{qm2pi.rhodetails}

\end{document}



% section proof sketch (end)

%\section{Unlikely characters: spatial logic for
  knots}\label{sub:characteristic_formulae} % (fold)

Associated to the mobile process calculi are a family of logics known
as the Hennessy-Milner logics. These logics typically enjoy a
semantics interpreting formulae as sets of processes that when
factored through the encoding outlined above allows an identification
of classes of knots with logical formulae. In the context of this
encoding the sub-family known as the spatial logics \cite{CairesC03}
\cite{CairesC04} \cite{Caires04} are of particular interest providing
several important features for expressing and reasoning about
properties (i.e. classes) of knots. We hint here at how this may be done.

%\begin{description}
%\item [structural connectives] 
\subsubsection{Structural connectives} The spatial logics enjoy
structural connectives corresponding, at the logical level, to the
parallel composition ($P | Q$) and new name ($(\nu \; x)P$)
connectives for processes. As illustrated in the examples below, these
connectives are extremely expressive given the shape of our encoding.
%\item [decideable satisfaction]

\subsubsection{Decideable satisfaction}
In \cite{Caires04} the satisfaction relation is shown to be decideable
for a rich class of processes. It further turns out that the image of
the our encoding is a proper subset of that class. This result
provides the basis for an algorithm by which to search for knots
enjoying a given property.
%\item [characteristic formulae]

\subsubsection{Characteristic formulae}
In the same paper \cite{Caires04} , Caires presents a means of calculating
characteristic formulae, selecting equivalence classes of processes
up to a pre--specified depth limit on the support set of names. Composed with our
encoding, this characteristic formula can be used to select
characteristic formulae for knots.
%\end{description}

\subsubsection{Spatial logic formulae}

The grammar below (segmented for comprehension) summarizes the syntax
of spatial logic formulae. We employ illustrative examples in the
sequel to provide an intuitive understanding of their meaning
referring the reader to \cite{Caires04} for a more detailed explication
of the semantics.

\begin{mathpar}
  \inferrule* [lab=boolean] {} {{A,B} \bc T \;|\; \neg A \;|\; A \wedge B \;|\; \eta = \eta'}
  \and
  \inferrule* [lab=spatial] {} {|\; \pzero \;|\; A | B \;|\; x \text{\textregistered} A \;|\; \forall x . A \;|\;  H x . A}
  \and
  \inferrule* [lab=behavioral] {} {|\; \alpha . A}
  \and 
  \inferrule* [lab=recursion] {} {|\; X(\vec{u}) \;|\; \mu X(\vec{u}) . A}
  \and
  \inferrule* [lab=action] {} {\alpha \bc \langle x?(\vec{y}) \rangle \;|\; \langle x!(\vec{y}) \rangle \;|\; \langle \tau \rangle}
  \and 
  \inferrule* [lab=name] {} {\eta \bc x \;|\; \tau}
\end{mathpar} 

% subsection characteristic_formulae (end)   	 

\subsection{Example formulae}\label{sub:example_formulae_} % (fold)

\subsubsection{Crossing as formula.}
% 
% \begin{align*}
%   \frac{d}{dx} \sin x &= \cos x 
%   & \frac{d}{dx} e^x &= e^x \\
%   \frac{d}{dx} \cos x &= - \sin x 
%   & \frac{d}{dx} \log x &= \frac{1}{x} \\
% \end{align*} 

\begin{align*}
 \mu C(x_{0},x_{1},y_{0},y_{1},u).&(\langle x_{0}?(z) \rangle(\langle u! \rangle\langle y_{1}!z \rangle C(x_{0},x_{1},y_{0},y_{1},u)) & \\
  & \wedge \langle y_{1}?(z) \rangle (\langle u! \rangle \langle x_{0}!z \rangle C(x_{0},x_{1},y_{0},y_{1},u)) & \\
  & \wedge \langle x_{1}?(z) \rangle (\langle u? \rangle \langle y_{0}!z \rangle C(x_{0},x_{1},y_{0},y_{1},u)) & \\
  & \wedge \langle y_{0}?(z) \rangle (\langle u? \rangle \langle x_{1}!z \rangle C(x_{0},x_{1},y_{0},y_{1},u))) &
\end{align*}

The lexicographical similarity between the shape of this formulae and
the shape of definition of the process representing a crossing reveals
the intuitive meaning of this formulae. It describes the capabilities
of a process that has the right to represent a crossing. For example
it picks out processes that may perform an input on the port $x_0$ in
its initial menu of capabilities. What differentiates the formula
from the process, however, is that the crossing process is the
smallest candidate to satisfy the formula. Infinitely many other
processes -- with internal behavior hidden behind this interface, so
to speak -- also satisfy this formula. Even this simple formula,
then, can be seen to open a new view onto knots, providing a
computational interpretation of \emph{virtual} knots.

Note that this formula is derived by hand. A similar formula can be
derived by employing Caires' calculation of characteristic formula
\cite{Caires04} to the process representing a crossing. In light of
this discussion, we let
$\meaningof{C}_{\phi}(x0,x1,y0,y1,u)$ denote a formula specifying the
dynamics we wish to capture of a crossing. To guarantee we preserve
the shape of the interface and minimal semantics we demand that
$\meaningof{C}_{\phi}(x0,x1,y0,y1,u) \Rightarrow
\textbf{C}(x0,x1,y0,y1,u)$ where $\textbf{C}(x0,x1,y0,y1,u)$ denotes
the formula above.
                            
\subsubsection{Crossing number constraints.}
The moral content of the context lemma (Lemma \ref{context}) is that the notion of
``locality'' in the Reidemeister moves is effectively captured by the
parallel composition operator of the process calculus. This intuition
extends through the logic. Given a formula,
$\meaningof{C}_{\phi}(x0,x1,y0,y1,u)$, we can use the structural
connectives to specify constraints on crossing numbers, such as at
least $n$ crossings, or exactly $n$ crossings.
\begin{mathpar}
  \inferrule* [lab=at-least-n] {} { K^{\geq n}_{\phi}(\vec{xs},\vec{ys}) := \Pi_{i=0}^{n-1} Hu . \meaningof{C}_{\phi}(xs_i,ys_i,u) | T }
  \and 
  \inferrule* [lab=exactly-n] {} { K^{= n}_{\phi}(\vec{xs},\vec{ys}) := \Pi_{i=0}^{n-1} Hu . \meaningof{C}_{\phi}(xs_i,ys_i,u) | \neg (\forall x_0,y_0,x_1,y_1,u . \meaningof{C}_{\phi}(x_0,y_0,x_1,y_1,u) | T) }
\end{mathpar}

To round out this section, recall that the encoding of an $n$-crossing
knot decomposes into a parallel composition of $n$ \emph{copies} of a
crossing process together with a wiring harness. To specify different
knot classes with the same crossing number amounts to specifying
logical constraints on the wiring harness. In the interest of space,
we defer examples to a forthcoming paper. Suffice it to say that both
the conditions ``alternating knot'' and ``contains the tangle
corresponding to 5/3'' are expressible. For example, it is possible to
calculate the characteristic formula of a process corresponding to the
tangle 5/3 and conjoin it into the classifying formula via the
composition connective of the logic.

Finally, we wish to observe that it is entirely within reason to
contemplate a more domain-specific version of spatial logic tailored
to the shape of processes in the image of the encoding. Such a
domain-specific logic would have a better claim to the title formal
language of knot properties.

% subsection example_formulae_ (end)

% section knots_as_processes (end) 

% section spatial logic via knots (end)

\section{Conclusions and future work}

\paragraph{Testing physical space}
You, gentle reader, may wonder why of all the theorems to be proved
given this set up we pick the one above. In some sense it's hardly
central to quantum mechanics. We see it as central in the sense that
it firmly establishes a notion of physical space arising from a notion
of the equivalence of behavior. Relating bisimulation to a metric is a
big step forward, but one is faced with interpreting the relationship
of that metric space to something more physical. Quantum mechanical
notions of ``physical'' space are still far from intuitive, but by
relating this idea of distance as testing to calculations that predict
physical circumstances we are making a not insignificant step forward
toward an understanding of the physical space we inhabit as
essentially dynamic.

\paragraph{Effectivity and simulation}
One of the observations we have yet to make is that the entire program
spelled out here is effective. We have built various interpreters for
the reflective calculus at work in this interpretation. In principle,
then, we can simulate quantum mechanics on a computer. The place where
the simulation may lose fidelity is the infinitely branching summation
for the annihilator.

In this connection i also want to point out that the evaluation style
calculation of the inner product puts the non-determinism of the
summation right at the heart of measurement. This suggests that
Milner's original reduction-based formulation of the dynamics of his
calculi in terms of sums was not just notationally suggestive of a
notion of measure-and-continue but captured some significant part of
the physics.

\paragraph{Quantum continuations}
In light of this last observation i want to point out that the
predominant account of quantum mechanics is missing a key aspect of a
truly compositional story of the physical situation. In a real lab,
when a measurement is made the observation can be made to feed into
another device that then makes another measurement conditioned on the
results of the first. This means that after the superposition was
collapsed the entire experimental set up remained in
superposition. While QM offers a means of writing this down it doesn't
quite line up well with the well-trodden formulation of computation
and continuation that we see so succinctly expressed in Milner's
calculi. This suggests that there might be advantages to this account
of dynamics waiting to be explored.

\paragraph{Quantum logic}
In this connection, we also note that by virtue of having the
Hennessy-Milner construction, we can pull the construction through the
interpretation of QM. This gives us a natural candidate for a quantum
logic that enjoys an extremely tight connection with it's domain of
interpretation, making the construction much less ad hoc (rather it is
the image of functor!).

\paragraph{Quantum probabiity}
i have questions about the basis of the interpretation of inner
product as probability amplitude. In particular, using which
axiomatization of probability theory does the notion of probability
amplitude earn the right to be so dubbed? In other words, where is the
proof that the operation for calculating a probability amplitude (and
then squaring) satisfies the axioms of what it means to calculate a
probability? Even if such a proof exists (i have yet to find it in the
literature), i wonder if it might not be possible to turn things on
their heads. Can we view the calculation of the probability amplitude
as an axiomatization of probability? If so, then the definition we
give for calculating probability amplitude may provide the basis for
an \emph{effective} theory of probability.

\paragraph{Quantum vs ``biological'' information}
Finally, i want to conclude with a more philosophical observation. At
a recent workshop in which QM was a predominant topic i noticed
something about quantum information. The speaker was giving a riveting
discussion of axiomatic QM and showing how properties of ``no
cloning'' and ``no deleting'' emerged as consequences of the
axiomatization. Theorems of this form are necessary to give us a sense
of confidence that our axioms characterize the physical theory. What
struck me, though, was that if quantum information is neither erasable
nor replicable it is markedly different from \emph{life}. Two of the
things we know about life is that

\begin{itemize}
  \item it ends;
  \item to gain some measure of persistence, to transcend it's
    finitude it is imminently copyable.
\end{itemize}

Both of these qualities are summarized succinctly in the aphorism: all
flesh is grass. For me these two kinds of ``information'' -- call them
quantum and biological -- are end points on a spectrum of strategies
for persistence. At one end, we have those curious entities that enjoy
uniqueness and permanence; at the other, we have those who in the face
of a certain end and an uncertain present make a go of passing
something on. To me one of the more remarkable aspects of the latter
strategy is that in the presence of noise (and certain features of
copying) we get a kind of dynamism, a chance for improvement against a
given persistent condition.

% subsection other_calculi_other_bisimulations_and_geometry_as_behavior (end)




% section conclusion (end)

%\documentclass[12pt]{llncs}
%\documentclass{jktr}

\usepackage[pdftex]{hyperref}                   
\usepackage {listings}
\usepackage {mathpartir}
\usepackage{bcprules}
%\usepackage{listings}
                       
\usepackage{graphicx} 
%\usepackage[margins=2.5cm,nohead,nofoot]{geometry}
%\usepackage{geometry}
\usepackage{amsfonts}
\usepackage{amstext}
\usepackage{latexsym}
\usepackage{amssymb}
\usepackage{color}


%\include{myPreamble}
\include{qm2pi.local} 

%\ifpdf
%\usepackage[pdftex]{graphicx}
%\else
%\usepackage{graphicx}
%\fi

 % \ifpdf
%  \usepackage{pdfsync}
%  \if


%\title{Brief Article}
%\author{David F. Snyder}
%\author{L.G. Meredith}

%\address{Dept. of Math., Texas State University--San Marcos, San Marcos, TX 78666}
       
\pagestyle{empty}


\begin{document}

\lstset{language=[Objective]Caml,frame=shadowbox}

\input{qm2pi.front}

% section front matter (end)

\input{qm2pi.intro} 
 
% section introduction (end)

% \input{qm2pi.knotations} 

% section notation (end)

\input{qm2pi.process.calculi} 

% section concurrent_process_calculi_and_spatial_logics_ (end)
    
%\input{qm2pi.knots2pi} 

%\input{qm2pi.trefoil} 

%\input{qm2pi.mainthm} 

% subsection basic_interpretation (end)

%\input{qm2pi.rho.presentation} 
\subsection{The syntax and semantics of the notation system}\label{sub:the_syntax_and_semantics_of_the_notation_system} % (fold)

We now summarize a technical presentation of the calculus that
embodies our theory of dynamics. The typical presentation of such a
calculus follows the style of giving generators and relations on
them. The grammar, below, describing term constructors, freely
generates the set of processes, $\Proc$. This set is then quotiented
by a relation known as structural congruence and it is over this set
that the notion of dynamics is expressed. This presentation is
essentially that of \cite{MeredithR05} with the addition of
polyadicity and summation. For readability we have relegated some of
the technical subtleties to an appendix.

\subsubsection{Process grammar}\label{subsub:process_grammar}

\begin{mathpar}
  \inferrule* [lab=synchronization] {} {{M} \bc \pzero \;|\; x?F \;|\; x!C }
  \and
  \inferrule* [lab=abstraction] {} {{F} \bc (x)P}
  \and
  \inferrule* [lab=concretion] {} {{C} \bc \langle Q \rangle}
  \and
  \inferrule* [lab=process] {} {{P,Q} \bc M \;| \;P|Q \;|\; @{x}}
  \and
  \inferrule* [lab=name] {} {{x} \bc \quotep{P}}
\end{mathpar} 

Note that $\vec{x}$ (resp. $\vec{P}$) denotes a vector of names
(resp. processes) of length $|\vec{x}|$ (resp. $|\vec{P}|$). We adopt
the following useful abbreviations.

\begin{mathpar}
   x?(\vec{y}).P := x.(\vec{y})P \and  x\clift{\vec{P}} := x.\clift{\vec{P}}
   \and x!(y) := \lift{x}{\dropn{y}}
   \and \Pi_{i=0}^{n-1}P_i := P_0 | \ldots | P_{n-1}
\end{mathpar}

\subsubsection{Structural congruence}

\paragraph{Free and bound names and alpha-equivalence.} At the
core of structural equivalence is alpha-equivalence which identifies
process that are the same up to a change of variable. Formally, we
recognize the distinction between free and bound names. The free names
of a process, $\freenames{P}$, may be calculated recursively as
follows:

\begin{mathpar}
\freenames{\pzero} := \emptyset
  \and \\
  \freenames{x?(y).P} := \{ x \} \cup (\freenames{P} \setminus \{ y \})
  \and 
  \freenames{x!\langle P \rangle} := \{ x \} \cup \{ P \} 
  \and \\
  \freenames{P|Q} := \freenames{P} \cup \freenames{Q}
  \and \\
  \freenames{@{x}} := \{ x \}
\end{mathpar}

$\pi$
$\quotep{\pi}$

$\freenames{-} : \pi \to \mathcal{P}(\quotep{\pi})$

\begin{eqnarray*}
  \freenames{\pzero} & := & \emptyset \\
  \freenames{x?(y).P} & := & \{ x \} \cup (\freenames{P} \setminus \{ y \}) \\
  \freenames{x!\langle P \rangle} & := & \{ x \} \cup \{ P \} \\
  \freenames{P|Q} & := & \freenames{P} \cup \freenames{Q} \\
  \freenames{\dropn{x}} & := & \{ x \}
\end{eqnarray*}

The bound names of a process, $\boundnames{P}$, are those names occurring in $P$
that are not free. For example, in $x?(y).0$, the name $x$ is free, while $y$ is bound.

\begin{mathpar}
  \inferrule* [lab=monoidal-laws] {} { P|Q \equiv Q|P \and P|0 \equiv P \and P|(Q|R) \equiv (P|Q)|R }
\end{mathpar}

\begin{mathpar}
  \inferrule* [lab=alpha-equivalence] {} { (x)P \equiv (y)P\{y/x\} \and y \not\in \freenames{P} }
\end{mathpar}

\begin{definition}
Then two processes, $P,Q$, are alpha-equivalent if $P = Q\{\vec{y}/\vec{x}\}$ for
some $\vec{x} \in \boundnames{Q},\vec{y} \in \boundnames{P}$, where $Q\{\vec{y}/\vec{x}\}$
denotes the capture-avoiding substitution of $\vec{y}$ for $\vec{x}$ in $Q$.
\end{definition}

\begin{definition}
  The {\em structural congruence} \cite{SangiorgiWalker} , $\equiv$,
  between processes is the least congruence containing
  alpha-equivalence, satisfying the abelian monoid laws
  (associativity, commutativity and $\pzero$ as identity) for parallel
  composition $|$ and for summation $+$.
\end{definition}

\subsection{Name equivalence}

We take name equivalence, written $\nameeq$, to be the smallest
equivalence relation generated by the following rules.

\begin{mathpar}
\inferrule*[lab=Quote-drop]
{ }
{ \quotep{@{x}} \nameeq x }

\inferrule*[lab=Struct-equiv]
{ P \scong Q }
{ \quotep{P} \nameeq \quotep{Q} }
\end{mathpar}

The astute reader will have noticed that the mutual recursion of names
and processes imposes a mutual recursion on alpha-equivalence and
structural equivalence via name-equivalence. Fortunately, all of this
works out pleasantly and we may calculate in the natural way, free of
concern. The reader interested in the details is referred to the
appendix \ref{appendix:rho_details}.

\subsection{Substitution}

We use $\Proc$ for the set of processes, $\QProc$ for the set of
names, and $\id{\{}\vec{y} / \vec{x} \id{\}}$ to denote partial maps,
$s : \QProc \rightarrow \QProc$. A map, $s$ lifts, uniquely, to a map
on process terms, $\widehat{s} : \Proc \rightarrow \Proc$ by the
following equations.

\begin{mathpar}
  (0) \psubstp{Q}{P} := 0 \\
  (R \juxtap S) \psubstp{Q}{P}
  :=    
  (R)\psubstp{Q}{P} \juxtap (S) \psubstp{Q}{P} \\
  (x?(y).R) \psubstp{Q}{P}    
  :=    
  (x)\substp{Q}{P} (z)\concat( (R \psubstn{z}{y}) \psubstp{Q}{P} ) \\
  (\lift{x}{R}) \psubstp{Q}{P}  
  :=
  \lift{(x)\substp{Q}{P}}{ R \psubstp{Q}{P} } \\
%   (\dropn{x})  \psubstp{Q}{P}       
%   := 
%   \left\{ 
%     \begin{array}{ccc} 
%       \dropn{\quotep{Q}} & & x \nameeq \quotep{P} \\
%       \dropn{x} & & otherwise \\
%     \end{array}
%   \right. 
  (\dropn{x})  \psubstp{Q}{P}       
  := 
  \left\{ 
    \begin{array}{ccc} 
      Q & & x \nameeq \quotep{P} \\
      \dropn{x} & & otherwise \\
    \end{array}
  \right.
\end{mathpar}
 

where

\begin{eqnarray}
  (x)\id{\{} \lpquote Q \rpquote / \lpquote P \rpquote \id{\}}            = 
  \left\{ 
    \begin{array}{ccc}
      \lpquote Q \rpquote & & x \nameeq \lpquote P \rpquote \\
      x & & otherwise \\
    \end{array}
  \right. \nonumber
\end{eqnarray}

and $z$ is chosen distinct from $\quotep{P}$, $\quotep{Q}$, the free
names in $Q$, and all the names in $R$. Our $\alpha$-equivalence will
be built in the standard way from this substitution.

\begin{remark}\label{rem:no_self_referential_names}
  One consequence of these definitions is that $\forall P. \quotep{P}
  \not\in \freenames{P}$.
\end{remark}

\subsection{ Dynamic quote: an example }

Anticipating something of what's to come, consider applying the
substitution, $\widehat{\id{\{}u / z \id{\}}}$, to the following pair
of processes, $\lift{w}{y!(z)}$ and $w[ \lpquote y!(z) \rpquote ]$.

\begin{eqnarray}
	\lift{w}{y!(z)}\widehat{\id{\{}u / z \id{\}}}
		& = &
		\lift{w}{y!(u)} \nonumber\\
	w[ \lpquote y!(z) \rpquote ] \widehat{ \id{\{}u / z \id{\}} }
		& = &
		w[ \lpquote y!(z) \rpquote ] \nonumber
\end{eqnarray}

Because the body of the process between quotes is impervious to
substitution, we get radically different answers. In fact, by
examining the first process in an input context,
e.g. $x?(z).\lift{w}{y!(z)}$, we see that the process under the lift
operator may be shaped by prefixed inputs binding a name inside it. In
this sense, the lift operator will be seen as a way to dynamically
construct processes before reifying them as names.

Finally equipped with these standard features we can present the
dynamics of the calculus.

\subsubsection{Operational semantics} 

Finally, we introduce the computational dynamics. What marks these
algebras as distinct from other more traditionally studied algebraic
structures, e.g. vector spaces or polynomial rings, is the manner in
which dynamics is captured. In traditional structures, dynamics is typically
expressed through morphisms between such structures, as in linear maps
between vector spaces or morphisms between rings. In algebras
associated with the semantics of computation, the dynamics is
expressed as part of the algebraic structure itself, through a
reduction reduction relation typically denoted by $\red$. Below, we
give a recursive presentation of this relation for the calculus used
in the encoding.

$\red \subseteq \pi \times \pi$
$\red : \pi \to \mathcal{P}(\pi)$

\begin{mathpar}
  \inferrule* [lab=Comm] { \textsf{match}( x_{src}, x_{trgt} ) } { x_{trgt}?(y)P \; | \; x_{src}!\langle {Q} \rangle \red P\{\quotep{Q}/y}\} }
  \and \\
  \inferrule* [lab=Par] {{P} \red {P}'} {{{P} | {Q}} \red {{P}' | {Q}}}
  \and
  \inferrule* [lab=Equiv]{{{P} \scong {P}'} \andalso {{P}' \red {Q}'} \andalso {{Q}' \scong {Q}}}{{P} \red {Q}}
\end{mathpar}

\begin{eqnarray*}
  match_{\equiv} (\quotep{P},\quotep{Q}) & := & P \equiv Q \\
  match_{\dagger}(\quotep{P},\quotep{Q}) & := & \forall R. P|Q \red^{*} R => R \red^{*} 0 \\
  match_{K}(\quotep{P},\quotep{Q}) & := & K \mbox{ for some context } K
\end{eqnarray*}

$u?(x)P | u!\langle Q \rangle \red P\{\quotep{Q}/x\}$

%We write $\wred$ for $\red^*$, and $P\red$ if $\exists Q $ such that $ P \red Q$.
We write $P\red$ if $\exists Q $ such that $ P \red Q$ and $P\not\red$, otherwise.

\section{Replication}

As mentioned before, it is known that replication (and hence
recursion) can be implemented in a higher-order process algebra
\cite{SangiorgiWalker}. As our first example of calculation with the
machinery thus far presented we give the construction explicitly in
the {\rhoc}.

\begin{eqnarray}
	D_{x} & := & \prefix{x}{y}{(\binpar{\outputp{x}{y}}{@{y}})} \nonumber\\
	\bangp_{x}{P} & := & \binpar{{x}!\langle{\binpar{D_{x}}{P}}\rangle}{D_{x}} \nonumber
\end{eqnarray}

\begin{eqnarray}
	\bangp_{x}{P} & & \nonumber\\
	=
	& {x}!\langle{(\prefix{x}{y}{(\outputp{x}{y} | @{y})) | P}}\rangle 
	      | \prefix{x}{y}{(\outputp{x}{y} | @{y})} & \nonumber\\
	\red
	& (\outputp{x}{y} | @{y})\substn{\quotep{(\prefix{x}{y}{(@{y} | \outputp{x}{y})) | P}}}{y} & \nonumber\\
	=
	& \outputp{x}{\quotep{(\prefix{x}{y}{(\outputp{x}{y} | @{y})) | P}}}
	  | {(\prefix{x}{y}{(\outputp{x}{y} | @{y})) | P}} & \nonumber\\
	\red
	& \ldots & \nonumber\\
	\red^*
	& P | P | \ldots & \nonumber
\end{eqnarray}

Of course, this encoding, as an implementation, runs away, unfolding
$\bangp{P}$ eagerly. A lazier and more implementable replication
operator, restricted to input-guarded processes, may be obtained as follows.

\begin{eqnarray}
\bangp{\prefix{u}{v}{P}} 
	:= 
	\binpar{\lift{x}{\prefix{u}{v}{(\binpar{D(x)}{P})}}}{D(x)} \nonumber
\end{eqnarray}

\begin{remark}
  Note that the lazier definition still does not deal with summation
  or mixed summation (i.e. sums over input and output). The reader is
  invited to construct definitions of replication that deal with these
  features. 

  Further, the definitions are parameterized in a name, $x$. Can you,
  gentle reader, make a definition that eliminates this parameter and
  guarantees no accidental interaction between the replication
  machinery and the process being replicated -- i.e. no accidental
  sharing of names used by the process to get its work done and the
  name(s) used by the replication to effect copying. This latter
  revision of the definition of replication is crucial to obtaining
  the expected identity $!!P \sim !P$.
\end{remark}

\begin{remark}\label{rem:paradoxical_combinator}
  The reader familiar with the lambda calculus will have noticed the
  similarity between $D$ and the paradoxical combinator.

  [Ed. note: the existence of this seems to suggest we have to be more
  restrictive on the set of processes and names we admit if we are to
  support no-cloning.]
\end{remark}

\subsubsection{Bisimulation}

The computational dynamics gives rise to another kind of equivalence,
the equivalence of computational behavior. As previously mentioned
this is typically captured \emph{via} some form of bisimulation.

% The notion we use in this paper is weak barbed bisimulation
% \cite{milner91polyadicpi}.

The notion we use in this paper is derived from weak barbed
bisimulation \cite{milner91polyadicpi}. 

\begin{definition}
An \emph{observation relation}, $\downarrow_{\mathcal N}$, over a set
of names, $\mathcal N$, is the smallest relation satisfying the rules
below.

\infrule[Out-barb]{y \in {\mathcal N}, \; x \nameeq y}
		  {\outputp{x}{v} \downarrow_{\mathcal N} x}
\infrule[Par-barb]{\mbox{$P\downarrow_{\mathcal N} x$ or $Q\downarrow_{\mathcal N} x$}}
		  {\binpar{P}{Q} \downarrow_{\mathcal N} x}

We write $P \Downarrow_{\mathcal N} x$ if there is $Q$ such that 
$P \wred Q$ and $Q \downarrow_{\mathcal N} x$.
\end{definition}

\begin{definition}
%\label{def.bbisim}
An  ${\mathcal N}$-\emph{barbed bisimulation} over a set of names, ${\mathcal N}$, is a symmetric binary relation 
${\mathcal S}_{\mathcal N}$ between agents such that $P\rel{S}_{\mathcal N}Q$ implies:
\begin{enumerate}
\item If $P \red P'$ then $Q \wred Q'$ and $P'\rel{S}_{\mathcal N} Q'$.
\item If $P\downarrow_{\mathcal N} x$, then $Q\Downarrow_{\mathcal N} x$.
\end{enumerate}
$P$ is ${\mathcal N}$-barbed bisimilar to $Q$, written
$P \wbbisim_{\mathcal N} Q$, if $P \rel{S}_{\mathcal N} Q$ for some ${\mathcal N}$-barbed bisimulation ${\mathcal S}_{\mathcal N}$.
\end{definition}

$\mathcal{R} \subseteq \pi \times \pi$

$P \mathcal{R} Q => \forall P'. P \red P' \Rightarrow \exists Q'. Q \red Q', P' \mathcal{R} Q'$

$P \vdash x \Rightarrow Q \vdash x$

\begin{mathpar}
  \inferrule*[lab=Out-barb]{x \nameeq y}{{y}!\langle{Q}\rangle \vdash x}
  \and
  \inferrule*[lab=Par-barb]{\mbox{$P\vdash x$ or $Q\vdash x$}}{\binpar{P}{Q} \vdash x}
\end{mathpar}

\subsubsection{Contexts}

One of the principle advantages of computational calculi like the
$\pi$-calculus is a well-defined notion of context,
contextual-equivalence and a correlation between
contextual-equivalence and notions of bisimulation. The notion of
context allows the decomposition of a process into (sub-)process and
its syntactic environment, its context. Thus, a context may be
thought of as a process with a ``hole'' (written $\Box$) in it. The
application of a context $M$ to a process $P$, written $M[P]$, is
tantamount to filling the hole in $M$ with $P$. In this paper we do
not need the full weight of this theory, but do make use of the notion
of context in the proof the main theorem. 

\begin{mathpar}
  \inferrule* [lab=summation] {} {{M_{M},M_{N}} \bc \Box \;|\; x.M_{A} \;|\; M_{M}+M_{N}}
  \and
  \inferrule* [lab=agent] {} {{M_{A}} \bc (\vec{x})M_{P} \;| \; \clift{P_0,\ldots,M_{P},\ldots,P_N}}
  \and \\
  \inferrule* [lab=process] {} {{M_{P}} \bc M_{N} \;| \;P|M_{P} }
\end{mathpar} 

\begin{mathpar}
  \inferrule* [lab=sychronization] {} {M_{N} \bc \Box \;|\; x?M_{F} \;|\; x!M_{C}}
  \and
  \inferrule* [lab=abstraction] {} {{M_{F}} \bc (x)M_{P} }
  \and
  \inferrule* [lab=concretion] {} {{M_{C}} \bc \langle M_{P} \rangle }
  \and \\
  \inferrule* [lab=process] {} {{M_{P}} \bc M_{N} \;| \;P|M_{P} }
\end{mathpar}

\begin{definition}[contextual application] Given a context $M$, and
  process $P$, we define the \emph{contextual application}, $M[P] :=
  M\{P/\Box\}$. That is, the contextual application of M to P is the
  substitution of $P$ for $\Box$ in $M$.
\end{definition}

$\meaningof{-} : L \to \mathcal{P}(\pi)$

\begin{mathpar}
  \inferrule* [lab=collection] {} {\meaningof{true} = \pi, \and \meaningof{~E} = \pi \setminus \meaningof{E}, \and \meaningof{E_{1} \& E_{2}} = \meaningof{E_{1}} \cap \meaningof{E_{2}}}
\end{mathpar}

\begin{mathpar}
  \inferrule* [lab=structure] {} {\meaningof{0} = \{ P \in \pi | P \equiv 0 \}, \and \\ \meaningof{E_1 | E_2} = \{ P \in \pi | P \equiv P_{1} | P_{2}, P_{1} \in \meaningof{E_{1}}, P_{2} \in \meaningof{E_2}\} }
\end{mathpar}

\begin{mathpar}
 \inferrule* [lab=behavior] {} {\meaningof{\langle a?b \rangle E} = \{ P \in \pi | P \equiv Q | u?(y)P', \\ \and \\\\ \and \\ \;\;\; u \in \meaningof{a}, \forall z.P'\{z/y\} \in \meaningof{E\{z/b\}}\}, \and \\ \meaningof{a!E} = \{ P \in \pi | P \equiv Q | x!\langle P' \rangle, x \in \meaningof{a} P' \in \meaningof{E}\} }
\end{mathpar}

\begin{mathpar}
 \inferrule* [lab=nominal] {} {\meaningof{\quotep{E}} = \{ \quotep{P} \in \quotep{\pi} | P \in \meaningof{E} \}, \and \meaningof{\quotep{P}} = \{ \quotep{Q} \in \quotep{\pi} | P \equiv Q \} \and \\ \meaningof{@\quotep{E}} = \{ P \in \pi | P \equiv @x, x \in \meaningof{E} \}}
\end{mathpar}

\begin{eqnarray*}
  \\
  \meaningof{-} : TS \to ST
\end{eqnarray*}

\begin{eqnarray*}
  \\
  L : TS \to ST
\end{eqnarray*}

\begin{eqnarray*}
  \\
  P \models E \iff P \in \meaningof{E}
\end{eqnarray*}

\begin{eqnarray*}
  P \approx_{L} Q \iff \forall E \in L. P \models E \iff Q \models E
\end{eqnarray*}

\begin{eqnarray*}
  P \approx_{K} Q
\end{eqnarray*}

\begin{eqnarray*}
  P \approx Q
\end{eqnarray*}

$\approx_{K} = \approx = \approx_{L}$

\subsubsection{Contextual duality}

Note that contexts extend the quotation operation to a family of
operations from processes to names. Given a context, $M$, we can
define a \emph{nominal context}, $\quotep{M}$ by $\quotep{M}[P] :=
\quotep{M[P]}$. To foreshadow what is to come we observe that these
operations enjoy a duality with processes very much like the duality
between vectors and maps from vectors to scalars.

Further, because the calculus is essentially higher-order, we have a
correspondence between contexts and processes. More specifically,
given a name $x$ and a context $M$ we can construct $M^{*}_{x}$ such
that 

\begin{mathpar}
  M^{*}_{x} | \lift{x}{P} \red M[P]
\end{mathpar}

namely,

\begin{mathpar}
  M^{*}_{x} := x?(u).M[\dropn{u}]
\end{mathpar}

The dependence of $M^{*}_{x}$ on a name makes it an abstraction, 

\begin{mathpar}
  M^{*} := (x)x?(u).M[\dropn{u}]
\end{mathpar}

\subsection{Additional notation}

It will sometimes be convenient to denote the process a name
quotes. We already have the notation $x = \quotep{P}$, but it will be
convenient to introduce an alternate notation, $\procn{x}$, when we
want to emphasize the connection to the use of the name. Note that, by
virtue of name equivalence, $\quotep{\procn{x}} \nameeq x$; so, the
notation is consistent with previous definitions.

Further, because names have structure it is possible to effect
substitutions on the basis of that structure. This means we need to
upgrade our notation for substitutions, which we accomplish by
adapting comprehension notation. Thus,

\begin{mathpar}
  P\{ y / x : x \in S \}
\end{mathpar}

is interpreted to mean the process derived from P by replacing (in a
capture-avoiding manner) each occurrence of $x$ in $S$ by $y$. For example,

\begin{mathpar}
  P\{ \quotep{\procn{x}|\procn{x}} / x : x \in \freenames{P} \}
\end{mathpar}

will replace each (occurrence) of a free name $x$ in $P$ by
$\quotep{\procn{x}|\procn{x}}$.

Also, we will avail ourselves of the notation $x^{L}$ and $x^{R}$ to
denote injections of a name into disjoint copies of the name
space. There are numerous ways to accomplish this. One example can be
found in \cite{MeredithR05}. This notation overloads to vectors of
names: $\vec{x}^{\pi} := (x_{i}^{\pi} \; : \; 0 \leq i < |\vec{x}| )$ where $\pi \in \{L,R\}$.

We also use $P^{\Box} := P|\Box$.

In \cite{MeredithR05} an interpretation of the new operator is
given. It turns out that there are several possible interpretations
all enjoying the requisite algebraic properties of the operator (see
\cite{milner91polyadicpi}). We will therefore make liberal use of
$(\nu\; \vec{x})P$.

% subsection the_syntax_and_semantics_of_the_notation_system (end)   

\input{qm2pi.qmops} 

\input{qm2pi.sterngerlach} 

\input{qm2pi.metric} 

% section concurrent_process_calculi (end)

%\input{qm2pi.proofsketch}

% section proof sketch (end)

%\input{qm2pi.slviaknots} 

% section spatial logic via knots (end)

\input{qm2pi.conclusion}

% section conclusion (end)

%\input{qm2pi.dtcodes} 

% section wiring algorithm (end)

\input{qm2pi.ack} 

% section acknowledgments (end)

\newpage


\bibliographystyle{plain}   
\bibliography{../../biblios/main.bib}

\input{qm2pi.rhodetails}

\end{document}

 

% section wiring algorithm (end)

\documentclass[12pt]{llncs}
%\documentclass{jktr}

\usepackage[pdftex]{hyperref}                   
\usepackage {listings}
\usepackage {mathpartir}
\usepackage{bcprules}
%\usepackage{listings}
                       
\usepackage{graphicx} 
%\usepackage[margins=2.5cm,nohead,nofoot]{geometry}
%\usepackage{geometry}
\usepackage{amsfonts}
\usepackage{amstext}
\usepackage{latexsym}
\usepackage{amssymb}
\usepackage{color}


%\include{myPreamble}
\include{qm2pi.local} 

%\ifpdf
%\usepackage[pdftex]{graphicx}
%\else
%\usepackage{graphicx}
%\fi

 % \ifpdf
%  \usepackage{pdfsync}
%  \if


%\title{Brief Article}
%\author{David F. Snyder}
%\author{L.G. Meredith}

%\address{Dept. of Math., Texas State University--San Marcos, San Marcos, TX 78666}
       
\pagestyle{empty}


\begin{document}

\lstset{language=[Objective]Caml,frame=shadowbox}

\input{qm2pi.front}

% section front matter (end)

\input{qm2pi.intro} 
 
% section introduction (end)

% \input{qm2pi.knotations} 

% section notation (end)

\input{qm2pi.process.calculi} 

% section concurrent_process_calculi_and_spatial_logics_ (end)
    
%\input{qm2pi.knots2pi} 

%\input{qm2pi.trefoil} 

%\input{qm2pi.mainthm} 

% subsection basic_interpretation (end)

%\input{qm2pi.rho.presentation} 
\subsection{The syntax and semantics of the notation system}\label{sub:the_syntax_and_semantics_of_the_notation_system} % (fold)

We now summarize a technical presentation of the calculus that
embodies our theory of dynamics. The typical presentation of such a
calculus follows the style of giving generators and relations on
them. The grammar, below, describing term constructors, freely
generates the set of processes, $\Proc$. This set is then quotiented
by a relation known as structural congruence and it is over this set
that the notion of dynamics is expressed. This presentation is
essentially that of \cite{MeredithR05} with the addition of
polyadicity and summation. For readability we have relegated some of
the technical subtleties to an appendix.

\subsubsection{Process grammar}\label{subsub:process_grammar}

\begin{mathpar}
  \inferrule* [lab=synchronization] {} {{M} \bc \pzero \;|\; x?F \;|\; x!C }
  \and
  \inferrule* [lab=abstraction] {} {{F} \bc (x)P}
  \and
  \inferrule* [lab=concretion] {} {{C} \bc \langle Q \rangle}
  \and
  \inferrule* [lab=process] {} {{P,Q} \bc M \;| \;P|Q \;|\; @{x}}
  \and
  \inferrule* [lab=name] {} {{x} \bc \quotep{P}}
\end{mathpar} 

Note that $\vec{x}$ (resp. $\vec{P}$) denotes a vector of names
(resp. processes) of length $|\vec{x}|$ (resp. $|\vec{P}|$). We adopt
the following useful abbreviations.

\begin{mathpar}
   x?(\vec{y}).P := x.(\vec{y})P \and  x\clift{\vec{P}} := x.\clift{\vec{P}}
   \and x!(y) := \lift{x}{\dropn{y}}
   \and \Pi_{i=0}^{n-1}P_i := P_0 | \ldots | P_{n-1}
\end{mathpar}

\subsubsection{Structural congruence}

\paragraph{Free and bound names and alpha-equivalence.} At the
core of structural equivalence is alpha-equivalence which identifies
process that are the same up to a change of variable. Formally, we
recognize the distinction between free and bound names. The free names
of a process, $\freenames{P}$, may be calculated recursively as
follows:

\begin{mathpar}
\freenames{\pzero} := \emptyset
  \and \\
  \freenames{x?(y).P} := \{ x \} \cup (\freenames{P} \setminus \{ y \})
  \and 
  \freenames{x!\langle P \rangle} := \{ x \} \cup \{ P \} 
  \and \\
  \freenames{P|Q} := \freenames{P} \cup \freenames{Q}
  \and \\
  \freenames{@{x}} := \{ x \}
\end{mathpar}

$\pi$
$\quotep{\pi}$

$\freenames{-} : \pi \to \mathcal{P}(\quotep{\pi})$

\begin{eqnarray*}
  \freenames{\pzero} & := & \emptyset \\
  \freenames{x?(y).P} & := & \{ x \} \cup (\freenames{P} \setminus \{ y \}) \\
  \freenames{x!\langle P \rangle} & := & \{ x \} \cup \{ P \} \\
  \freenames{P|Q} & := & \freenames{P} \cup \freenames{Q} \\
  \freenames{\dropn{x}} & := & \{ x \}
\end{eqnarray*}

The bound names of a process, $\boundnames{P}$, are those names occurring in $P$
that are not free. For example, in $x?(y).0$, the name $x$ is free, while $y$ is bound.

\begin{mathpar}
  \inferrule* [lab=monoidal-laws] {} { P|Q \equiv Q|P \and P|0 \equiv P \and P|(Q|R) \equiv (P|Q)|R }
\end{mathpar}

\begin{mathpar}
  \inferrule* [lab=alpha-equivalence] {} { (x)P \equiv (y)P\{y/x\} \and y \not\in \freenames{P} }
\end{mathpar}

\begin{definition}
Then two processes, $P,Q$, are alpha-equivalent if $P = Q\{\vec{y}/\vec{x}\}$ for
some $\vec{x} \in \boundnames{Q},\vec{y} \in \boundnames{P}$, where $Q\{\vec{y}/\vec{x}\}$
denotes the capture-avoiding substitution of $\vec{y}$ for $\vec{x}$ in $Q$.
\end{definition}

\begin{definition}
  The {\em structural congruence} \cite{SangiorgiWalker} , $\equiv$,
  between processes is the least congruence containing
  alpha-equivalence, satisfying the abelian monoid laws
  (associativity, commutativity and $\pzero$ as identity) for parallel
  composition $|$ and for summation $+$.
\end{definition}

\subsection{Name equivalence}

We take name equivalence, written $\nameeq$, to be the smallest
equivalence relation generated by the following rules.

\begin{mathpar}
\inferrule*[lab=Quote-drop]
{ }
{ \quotep{@{x}} \nameeq x }

\inferrule*[lab=Struct-equiv]
{ P \scong Q }
{ \quotep{P} \nameeq \quotep{Q} }
\end{mathpar}

The astute reader will have noticed that the mutual recursion of names
and processes imposes a mutual recursion on alpha-equivalence and
structural equivalence via name-equivalence. Fortunately, all of this
works out pleasantly and we may calculate in the natural way, free of
concern. The reader interested in the details is referred to the
appendix \ref{appendix:rho_details}.

\subsection{Substitution}

We use $\Proc$ for the set of processes, $\QProc$ for the set of
names, and $\id{\{}\vec{y} / \vec{x} \id{\}}$ to denote partial maps,
$s : \QProc \rightarrow \QProc$. A map, $s$ lifts, uniquely, to a map
on process terms, $\widehat{s} : \Proc \rightarrow \Proc$ by the
following equations.

\begin{mathpar}
  (0) \psubstp{Q}{P} := 0 \\
  (R \juxtap S) \psubstp{Q}{P}
  :=    
  (R)\psubstp{Q}{P} \juxtap (S) \psubstp{Q}{P} \\
  (x?(y).R) \psubstp{Q}{P}    
  :=    
  (x)\substp{Q}{P} (z)\concat( (R \psubstn{z}{y}) \psubstp{Q}{P} ) \\
  (\lift{x}{R}) \psubstp{Q}{P}  
  :=
  \lift{(x)\substp{Q}{P}}{ R \psubstp{Q}{P} } \\
%   (\dropn{x})  \psubstp{Q}{P}       
%   := 
%   \left\{ 
%     \begin{array}{ccc} 
%       \dropn{\quotep{Q}} & & x \nameeq \quotep{P} \\
%       \dropn{x} & & otherwise \\
%     \end{array}
%   \right. 
  (\dropn{x})  \psubstp{Q}{P}       
  := 
  \left\{ 
    \begin{array}{ccc} 
      Q & & x \nameeq \quotep{P} \\
      \dropn{x} & & otherwise \\
    \end{array}
  \right.
\end{mathpar}
 

where

\begin{eqnarray}
  (x)\id{\{} \lpquote Q \rpquote / \lpquote P \rpquote \id{\}}            = 
  \left\{ 
    \begin{array}{ccc}
      \lpquote Q \rpquote & & x \nameeq \lpquote P \rpquote \\
      x & & otherwise \\
    \end{array}
  \right. \nonumber
\end{eqnarray}

and $z$ is chosen distinct from $\quotep{P}$, $\quotep{Q}$, the free
names in $Q$, and all the names in $R$. Our $\alpha$-equivalence will
be built in the standard way from this substitution.

\begin{remark}\label{rem:no_self_referential_names}
  One consequence of these definitions is that $\forall P. \quotep{P}
  \not\in \freenames{P}$.
\end{remark}

\subsection{ Dynamic quote: an example }

Anticipating something of what's to come, consider applying the
substitution, $\widehat{\id{\{}u / z \id{\}}}$, to the following pair
of processes, $\lift{w}{y!(z)}$ and $w[ \lpquote y!(z) \rpquote ]$.

\begin{eqnarray}
	\lift{w}{y!(z)}\widehat{\id{\{}u / z \id{\}}}
		& = &
		\lift{w}{y!(u)} \nonumber\\
	w[ \lpquote y!(z) \rpquote ] \widehat{ \id{\{}u / z \id{\}} }
		& = &
		w[ \lpquote y!(z) \rpquote ] \nonumber
\end{eqnarray}

Because the body of the process between quotes is impervious to
substitution, we get radically different answers. In fact, by
examining the first process in an input context,
e.g. $x?(z).\lift{w}{y!(z)}$, we see that the process under the lift
operator may be shaped by prefixed inputs binding a name inside it. In
this sense, the lift operator will be seen as a way to dynamically
construct processes before reifying them as names.

Finally equipped with these standard features we can present the
dynamics of the calculus.

\subsubsection{Operational semantics} 

Finally, we introduce the computational dynamics. What marks these
algebras as distinct from other more traditionally studied algebraic
structures, e.g. vector spaces or polynomial rings, is the manner in
which dynamics is captured. In traditional structures, dynamics is typically
expressed through morphisms between such structures, as in linear maps
between vector spaces or morphisms between rings. In algebras
associated with the semantics of computation, the dynamics is
expressed as part of the algebraic structure itself, through a
reduction reduction relation typically denoted by $\red$. Below, we
give a recursive presentation of this relation for the calculus used
in the encoding.

$\red \subseteq \pi \times \pi$
$\red : \pi \to \mathcal{P}(\pi)$

\begin{mathpar}
  \inferrule* [lab=Comm] { \textsf{match}( x_{src}, x_{trgt} ) } { x_{trgt}?(y)P \; | \; x_{src}!\langle {Q} \rangle \red P\{\quotep{Q}/y}\} }
  \and \\
  \inferrule* [lab=Par] {{P} \red {P}'} {{{P} | {Q}} \red {{P}' | {Q}}}
  \and
  \inferrule* [lab=Equiv]{{{P} \scong {P}'} \andalso {{P}' \red {Q}'} \andalso {{Q}' \scong {Q}}}{{P} \red {Q}}
\end{mathpar}

\begin{eqnarray*}
  match_{\equiv} (\quotep{P},\quotep{Q}) & := & P \equiv Q \\
  match_{\dagger}(\quotep{P},\quotep{Q}) & := & \forall R. P|Q \red^{*} R => R \red^{*} 0 \\
  match_{K}(\quotep{P},\quotep{Q}) & := & K \mbox{ for some context } K
\end{eqnarray*}

$u?(x)P | u!\langle Q \rangle \red P\{\quotep{Q}/x\}$

%We write $\wred$ for $\red^*$, and $P\red$ if $\exists Q $ such that $ P \red Q$.
We write $P\red$ if $\exists Q $ such that $ P \red Q$ and $P\not\red$, otherwise.

\section{Replication}

As mentioned before, it is known that replication (and hence
recursion) can be implemented in a higher-order process algebra
\cite{SangiorgiWalker}. As our first example of calculation with the
machinery thus far presented we give the construction explicitly in
the {\rhoc}.

\begin{eqnarray}
	D_{x} & := & \prefix{x}{y}{(\binpar{\outputp{x}{y}}{@{y}})} \nonumber\\
	\bangp_{x}{P} & := & \binpar{{x}!\langle{\binpar{D_{x}}{P}}\rangle}{D_{x}} \nonumber
\end{eqnarray}

\begin{eqnarray}
	\bangp_{x}{P} & & \nonumber\\
	=
	& {x}!\langle{(\prefix{x}{y}{(\outputp{x}{y} | @{y})) | P}}\rangle 
	      | \prefix{x}{y}{(\outputp{x}{y} | @{y})} & \nonumber\\
	\red
	& (\outputp{x}{y} | @{y})\substn{\quotep{(\prefix{x}{y}{(@{y} | \outputp{x}{y})) | P}}}{y} & \nonumber\\
	=
	& \outputp{x}{\quotep{(\prefix{x}{y}{(\outputp{x}{y} | @{y})) | P}}}
	  | {(\prefix{x}{y}{(\outputp{x}{y} | @{y})) | P}} & \nonumber\\
	\red
	& \ldots & \nonumber\\
	\red^*
	& P | P | \ldots & \nonumber
\end{eqnarray}

Of course, this encoding, as an implementation, runs away, unfolding
$\bangp{P}$ eagerly. A lazier and more implementable replication
operator, restricted to input-guarded processes, may be obtained as follows.

\begin{eqnarray}
\bangp{\prefix{u}{v}{P}} 
	:= 
	\binpar{\lift{x}{\prefix{u}{v}{(\binpar{D(x)}{P})}}}{D(x)} \nonumber
\end{eqnarray}

\begin{remark}
  Note that the lazier definition still does not deal with summation
  or mixed summation (i.e. sums over input and output). The reader is
  invited to construct definitions of replication that deal with these
  features. 

  Further, the definitions are parameterized in a name, $x$. Can you,
  gentle reader, make a definition that eliminates this parameter and
  guarantees no accidental interaction between the replication
  machinery and the process being replicated -- i.e. no accidental
  sharing of names used by the process to get its work done and the
  name(s) used by the replication to effect copying. This latter
  revision of the definition of replication is crucial to obtaining
  the expected identity $!!P \sim !P$.
\end{remark}

\begin{remark}\label{rem:paradoxical_combinator}
  The reader familiar with the lambda calculus will have noticed the
  similarity between $D$ and the paradoxical combinator.

  [Ed. note: the existence of this seems to suggest we have to be more
  restrictive on the set of processes and names we admit if we are to
  support no-cloning.]
\end{remark}

\subsubsection{Bisimulation}

The computational dynamics gives rise to another kind of equivalence,
the equivalence of computational behavior. As previously mentioned
this is typically captured \emph{via} some form of bisimulation.

% The notion we use in this paper is weak barbed bisimulation
% \cite{milner91polyadicpi}.

The notion we use in this paper is derived from weak barbed
bisimulation \cite{milner91polyadicpi}. 

\begin{definition}
An \emph{observation relation}, $\downarrow_{\mathcal N}$, over a set
of names, $\mathcal N$, is the smallest relation satisfying the rules
below.

\infrule[Out-barb]{y \in {\mathcal N}, \; x \nameeq y}
		  {\outputp{x}{v} \downarrow_{\mathcal N} x}
\infrule[Par-barb]{\mbox{$P\downarrow_{\mathcal N} x$ or $Q\downarrow_{\mathcal N} x$}}
		  {\binpar{P}{Q} \downarrow_{\mathcal N} x}

We write $P \Downarrow_{\mathcal N} x$ if there is $Q$ such that 
$P \wred Q$ and $Q \downarrow_{\mathcal N} x$.
\end{definition}

\begin{definition}
%\label{def.bbisim}
An  ${\mathcal N}$-\emph{barbed bisimulation} over a set of names, ${\mathcal N}$, is a symmetric binary relation 
${\mathcal S}_{\mathcal N}$ between agents such that $P\rel{S}_{\mathcal N}Q$ implies:
\begin{enumerate}
\item If $P \red P'$ then $Q \wred Q'$ and $P'\rel{S}_{\mathcal N} Q'$.
\item If $P\downarrow_{\mathcal N} x$, then $Q\Downarrow_{\mathcal N} x$.
\end{enumerate}
$P$ is ${\mathcal N}$-barbed bisimilar to $Q$, written
$P \wbbisim_{\mathcal N} Q$, if $P \rel{S}_{\mathcal N} Q$ for some ${\mathcal N}$-barbed bisimulation ${\mathcal S}_{\mathcal N}$.
\end{definition}

$\mathcal{R} \subseteq \pi \times \pi$

$P \mathcal{R} Q => \forall P'. P \red P' \Rightarrow \exists Q'. Q \red Q', P' \mathcal{R} Q'$

$P \vdash x \Rightarrow Q \vdash x$

\begin{mathpar}
  \inferrule*[lab=Out-barb]{x \nameeq y}{{y}!\langle{Q}\rangle \vdash x}
  \and
  \inferrule*[lab=Par-barb]{\mbox{$P\vdash x$ or $Q\vdash x$}}{\binpar{P}{Q} \vdash x}
\end{mathpar}

\subsubsection{Contexts}

One of the principle advantages of computational calculi like the
$\pi$-calculus is a well-defined notion of context,
contextual-equivalence and a correlation between
contextual-equivalence and notions of bisimulation. The notion of
context allows the decomposition of a process into (sub-)process and
its syntactic environment, its context. Thus, a context may be
thought of as a process with a ``hole'' (written $\Box$) in it. The
application of a context $M$ to a process $P$, written $M[P]$, is
tantamount to filling the hole in $M$ with $P$. In this paper we do
not need the full weight of this theory, but do make use of the notion
of context in the proof the main theorem. 

\begin{mathpar}
  \inferrule* [lab=summation] {} {{M_{M},M_{N}} \bc \Box \;|\; x.M_{A} \;|\; M_{M}+M_{N}}
  \and
  \inferrule* [lab=agent] {} {{M_{A}} \bc (\vec{x})M_{P} \;| \; \clift{P_0,\ldots,M_{P},\ldots,P_N}}
  \and \\
  \inferrule* [lab=process] {} {{M_{P}} \bc M_{N} \;| \;P|M_{P} }
\end{mathpar} 

\begin{mathpar}
  \inferrule* [lab=sychronization] {} {M_{N} \bc \Box \;|\; x?M_{F} \;|\; x!M_{C}}
  \and
  \inferrule* [lab=abstraction] {} {{M_{F}} \bc (x)M_{P} }
  \and
  \inferrule* [lab=concretion] {} {{M_{C}} \bc \langle M_{P} \rangle }
  \and \\
  \inferrule* [lab=process] {} {{M_{P}} \bc M_{N} \;| \;P|M_{P} }
\end{mathpar}

\begin{definition}[contextual application] Given a context $M$, and
  process $P$, we define the \emph{contextual application}, $M[P] :=
  M\{P/\Box\}$. That is, the contextual application of M to P is the
  substitution of $P$ for $\Box$ in $M$.
\end{definition}

$\meaningof{-} : L \to \mathcal{P}(\pi)$

\begin{mathpar}
  \inferrule* [lab=collection] {} {\meaningof{true} = \pi, \and \meaningof{~E} = \pi \setminus \meaningof{E}, \and \meaningof{E_{1} \& E_{2}} = \meaningof{E_{1}} \cap \meaningof{E_{2}}}
\end{mathpar}

\begin{mathpar}
  \inferrule* [lab=structure] {} {\meaningof{0} = \{ P \in \pi | P \equiv 0 \}, \and \\ \meaningof{E_1 | E_2} = \{ P \in \pi | P \equiv P_{1} | P_{2}, P_{1} \in \meaningof{E_{1}}, P_{2} \in \meaningof{E_2}\} }
\end{mathpar}

\begin{mathpar}
 \inferrule* [lab=behavior] {} {\meaningof{\langle a?b \rangle E} = \{ P \in \pi | P \equiv Q | u?(y)P', \\ \and \\\\ \and \\ \;\;\; u \in \meaningof{a}, \forall z.P'\{z/y\} \in \meaningof{E\{z/b\}}\}, \and \\ \meaningof{a!E} = \{ P \in \pi | P \equiv Q | x!\langle P' \rangle, x \in \meaningof{a} P' \in \meaningof{E}\} }
\end{mathpar}

\begin{mathpar}
 \inferrule* [lab=nominal] {} {\meaningof{\quotep{E}} = \{ \quotep{P} \in \quotep{\pi} | P \in \meaningof{E} \}, \and \meaningof{\quotep{P}} = \{ \quotep{Q} \in \quotep{\pi} | P \equiv Q \} \and \\ \meaningof{@\quotep{E}} = \{ P \in \pi | P \equiv @x, x \in \meaningof{E} \}}
\end{mathpar}

\begin{eqnarray*}
  \\
  \meaningof{-} : TS \to ST
\end{eqnarray*}

\begin{eqnarray*}
  \\
  L : TS \to ST
\end{eqnarray*}

\begin{eqnarray*}
  \\
  P \models E \iff P \in \meaningof{E}
\end{eqnarray*}

\begin{eqnarray*}
  P \approx_{L} Q \iff \forall E \in L. P \models E \iff Q \models E
\end{eqnarray*}

\begin{eqnarray*}
  P \approx_{K} Q
\end{eqnarray*}

\begin{eqnarray*}
  P \approx Q
\end{eqnarray*}

$\approx_{K} = \approx = \approx_{L}$

\subsubsection{Contextual duality}

Note that contexts extend the quotation operation to a family of
operations from processes to names. Given a context, $M$, we can
define a \emph{nominal context}, $\quotep{M}$ by $\quotep{M}[P] :=
\quotep{M[P]}$. To foreshadow what is to come we observe that these
operations enjoy a duality with processes very much like the duality
between vectors and maps from vectors to scalars.

Further, because the calculus is essentially higher-order, we have a
correspondence between contexts and processes. More specifically,
given a name $x$ and a context $M$ we can construct $M^{*}_{x}$ such
that 

\begin{mathpar}
  M^{*}_{x} | \lift{x}{P} \red M[P]
\end{mathpar}

namely,

\begin{mathpar}
  M^{*}_{x} := x?(u).M[\dropn{u}]
\end{mathpar}

The dependence of $M^{*}_{x}$ on a name makes it an abstraction, 

\begin{mathpar}
  M^{*} := (x)x?(u).M[\dropn{u}]
\end{mathpar}

\subsection{Additional notation}

It will sometimes be convenient to denote the process a name
quotes. We already have the notation $x = \quotep{P}$, but it will be
convenient to introduce an alternate notation, $\procn{x}$, when we
want to emphasize the connection to the use of the name. Note that, by
virtue of name equivalence, $\quotep{\procn{x}} \nameeq x$; so, the
notation is consistent with previous definitions.

Further, because names have structure it is possible to effect
substitutions on the basis of that structure. This means we need to
upgrade our notation for substitutions, which we accomplish by
adapting comprehension notation. Thus,

\begin{mathpar}
  P\{ y / x : x \in S \}
\end{mathpar}

is interpreted to mean the process derived from P by replacing (in a
capture-avoiding manner) each occurrence of $x$ in $S$ by $y$. For example,

\begin{mathpar}
  P\{ \quotep{\procn{x}|\procn{x}} / x : x \in \freenames{P} \}
\end{mathpar}

will replace each (occurrence) of a free name $x$ in $P$ by
$\quotep{\procn{x}|\procn{x}}$.

Also, we will avail ourselves of the notation $x^{L}$ and $x^{R}$ to
denote injections of a name into disjoint copies of the name
space. There are numerous ways to accomplish this. One example can be
found in \cite{MeredithR05}. This notation overloads to vectors of
names: $\vec{x}^{\pi} := (x_{i}^{\pi} \; : \; 0 \leq i < |\vec{x}| )$ where $\pi \in \{L,R\}$.

We also use $P^{\Box} := P|\Box$.

In \cite{MeredithR05} an interpretation of the new operator is
given. It turns out that there are several possible interpretations
all enjoying the requisite algebraic properties of the operator (see
\cite{milner91polyadicpi}). We will therefore make liberal use of
$(\nu\; \vec{x})P$.

% subsection the_syntax_and_semantics_of_the_notation_system (end)   

\input{qm2pi.qmops} 

\input{qm2pi.sterngerlach} 

\input{qm2pi.metric} 

% section concurrent_process_calculi (end)

%\input{qm2pi.proofsketch}

% section proof sketch (end)

%\input{qm2pi.slviaknots} 

% section spatial logic via knots (end)

\input{qm2pi.conclusion}

% section conclusion (end)

%\input{qm2pi.dtcodes} 

% section wiring algorithm (end)

\input{qm2pi.ack} 

% section acknowledgments (end)

\newpage


\bibliographystyle{plain}   
\bibliography{../../biblios/main.bib}

\input{qm2pi.rhodetails}

\end{document}

 

% section acknowledgments (end)

\newpage


\bibliographystyle{plain}   
\bibliography{../../biblios/main.bib}

\documentclass[12pt]{llncs}
%\documentclass{jktr}

\usepackage[pdftex]{hyperref}                   
\usepackage {listings}
\usepackage {mathpartir}
\usepackage{bcprules}
%\usepackage{listings}
                       
\usepackage{graphicx} 
%\usepackage[margins=2.5cm,nohead,nofoot]{geometry}
%\usepackage{geometry}
\usepackage{amsfonts}
\usepackage{amstext}
\usepackage{latexsym}
\usepackage{amssymb}
\usepackage{color}


%\include{myPreamble}
\include{qm2pi.local} 

%\ifpdf
%\usepackage[pdftex]{graphicx}
%\else
%\usepackage{graphicx}
%\fi

 % \ifpdf
%  \usepackage{pdfsync}
%  \if


%\title{Brief Article}
%\author{David F. Snyder}
%\author{L.G. Meredith}

%\address{Dept. of Math., Texas State University--San Marcos, San Marcos, TX 78666}
       
\pagestyle{empty}


\begin{document}

\lstset{language=[Objective]Caml,frame=shadowbox}

\input{qm2pi.front}

% section front matter (end)

\input{qm2pi.intro} 
 
% section introduction (end)

% \input{qm2pi.knotations} 

% section notation (end)

\input{qm2pi.process.calculi} 

% section concurrent_process_calculi_and_spatial_logics_ (end)
    
%\input{qm2pi.knots2pi} 

%\input{qm2pi.trefoil} 

%\input{qm2pi.mainthm} 

% subsection basic_interpretation (end)

%\input{qm2pi.rho.presentation} 
\subsection{The syntax and semantics of the notation system}\label{sub:the_syntax_and_semantics_of_the_notation_system} % (fold)

We now summarize a technical presentation of the calculus that
embodies our theory of dynamics. The typical presentation of such a
calculus follows the style of giving generators and relations on
them. The grammar, below, describing term constructors, freely
generates the set of processes, $\Proc$. This set is then quotiented
by a relation known as structural congruence and it is over this set
that the notion of dynamics is expressed. This presentation is
essentially that of \cite{MeredithR05} with the addition of
polyadicity and summation. For readability we have relegated some of
the technical subtleties to an appendix.

\subsubsection{Process grammar}\label{subsub:process_grammar}

\begin{mathpar}
  \inferrule* [lab=synchronization] {} {{M} \bc \pzero \;|\; x?F \;|\; x!C }
  \and
  \inferrule* [lab=abstraction] {} {{F} \bc (x)P}
  \and
  \inferrule* [lab=concretion] {} {{C} \bc \langle Q \rangle}
  \and
  \inferrule* [lab=process] {} {{P,Q} \bc M \;| \;P|Q \;|\; @{x}}
  \and
  \inferrule* [lab=name] {} {{x} \bc \quotep{P}}
\end{mathpar} 

Note that $\vec{x}$ (resp. $\vec{P}$) denotes a vector of names
(resp. processes) of length $|\vec{x}|$ (resp. $|\vec{P}|$). We adopt
the following useful abbreviations.

\begin{mathpar}
   x?(\vec{y}).P := x.(\vec{y})P \and  x\clift{\vec{P}} := x.\clift{\vec{P}}
   \and x!(y) := \lift{x}{\dropn{y}}
   \and \Pi_{i=0}^{n-1}P_i := P_0 | \ldots | P_{n-1}
\end{mathpar}

\subsubsection{Structural congruence}

\paragraph{Free and bound names and alpha-equivalence.} At the
core of structural equivalence is alpha-equivalence which identifies
process that are the same up to a change of variable. Formally, we
recognize the distinction between free and bound names. The free names
of a process, $\freenames{P}$, may be calculated recursively as
follows:

\begin{mathpar}
\freenames{\pzero} := \emptyset
  \and \\
  \freenames{x?(y).P} := \{ x \} \cup (\freenames{P} \setminus \{ y \})
  \and 
  \freenames{x!\langle P \rangle} := \{ x \} \cup \{ P \} 
  \and \\
  \freenames{P|Q} := \freenames{P} \cup \freenames{Q}
  \and \\
  \freenames{@{x}} := \{ x \}
\end{mathpar}

$\pi$
$\quotep{\pi}$

$\freenames{-} : \pi \to \mathcal{P}(\quotep{\pi})$

\begin{eqnarray*}
  \freenames{\pzero} & := & \emptyset \\
  \freenames{x?(y).P} & := & \{ x \} \cup (\freenames{P} \setminus \{ y \}) \\
  \freenames{x!\langle P \rangle} & := & \{ x \} \cup \{ P \} \\
  \freenames{P|Q} & := & \freenames{P} \cup \freenames{Q} \\
  \freenames{\dropn{x}} & := & \{ x \}
\end{eqnarray*}

The bound names of a process, $\boundnames{P}$, are those names occurring in $P$
that are not free. For example, in $x?(y).0$, the name $x$ is free, while $y$ is bound.

\begin{mathpar}
  \inferrule* [lab=monoidal-laws] {} { P|Q \equiv Q|P \and P|0 \equiv P \and P|(Q|R) \equiv (P|Q)|R }
\end{mathpar}

\begin{mathpar}
  \inferrule* [lab=alpha-equivalence] {} { (x)P \equiv (y)P\{y/x\} \and y \not\in \freenames{P} }
\end{mathpar}

\begin{definition}
Then two processes, $P,Q$, are alpha-equivalent if $P = Q\{\vec{y}/\vec{x}\}$ for
some $\vec{x} \in \boundnames{Q},\vec{y} \in \boundnames{P}$, where $Q\{\vec{y}/\vec{x}\}$
denotes the capture-avoiding substitution of $\vec{y}$ for $\vec{x}$ in $Q$.
\end{definition}

\begin{definition}
  The {\em structural congruence} \cite{SangiorgiWalker} , $\equiv$,
  between processes is the least congruence containing
  alpha-equivalence, satisfying the abelian monoid laws
  (associativity, commutativity and $\pzero$ as identity) for parallel
  composition $|$ and for summation $+$.
\end{definition}

\subsection{Name equivalence}

We take name equivalence, written $\nameeq$, to be the smallest
equivalence relation generated by the following rules.

\begin{mathpar}
\inferrule*[lab=Quote-drop]
{ }
{ \quotep{@{x}} \nameeq x }

\inferrule*[lab=Struct-equiv]
{ P \scong Q }
{ \quotep{P} \nameeq \quotep{Q} }
\end{mathpar}

The astute reader will have noticed that the mutual recursion of names
and processes imposes a mutual recursion on alpha-equivalence and
structural equivalence via name-equivalence. Fortunately, all of this
works out pleasantly and we may calculate in the natural way, free of
concern. The reader interested in the details is referred to the
appendix \ref{appendix:rho_details}.

\subsection{Substitution}

We use $\Proc$ for the set of processes, $\QProc$ for the set of
names, and $\id{\{}\vec{y} / \vec{x} \id{\}}$ to denote partial maps,
$s : \QProc \rightarrow \QProc$. A map, $s$ lifts, uniquely, to a map
on process terms, $\widehat{s} : \Proc \rightarrow \Proc$ by the
following equations.

\begin{mathpar}
  (0) \psubstp{Q}{P} := 0 \\
  (R \juxtap S) \psubstp{Q}{P}
  :=    
  (R)\psubstp{Q}{P} \juxtap (S) \psubstp{Q}{P} \\
  (x?(y).R) \psubstp{Q}{P}    
  :=    
  (x)\substp{Q}{P} (z)\concat( (R \psubstn{z}{y}) \psubstp{Q}{P} ) \\
  (\lift{x}{R}) \psubstp{Q}{P}  
  :=
  \lift{(x)\substp{Q}{P}}{ R \psubstp{Q}{P} } \\
%   (\dropn{x})  \psubstp{Q}{P}       
%   := 
%   \left\{ 
%     \begin{array}{ccc} 
%       \dropn{\quotep{Q}} & & x \nameeq \quotep{P} \\
%       \dropn{x} & & otherwise \\
%     \end{array}
%   \right. 
  (\dropn{x})  \psubstp{Q}{P}       
  := 
  \left\{ 
    \begin{array}{ccc} 
      Q & & x \nameeq \quotep{P} \\
      \dropn{x} & & otherwise \\
    \end{array}
  \right.
\end{mathpar}
 

where

\begin{eqnarray}
  (x)\id{\{} \lpquote Q \rpquote / \lpquote P \rpquote \id{\}}            = 
  \left\{ 
    \begin{array}{ccc}
      \lpquote Q \rpquote & & x \nameeq \lpquote P \rpquote \\
      x & & otherwise \\
    \end{array}
  \right. \nonumber
\end{eqnarray}

and $z$ is chosen distinct from $\quotep{P}$, $\quotep{Q}$, the free
names in $Q$, and all the names in $R$. Our $\alpha$-equivalence will
be built in the standard way from this substitution.

\begin{remark}\label{rem:no_self_referential_names}
  One consequence of these definitions is that $\forall P. \quotep{P}
  \not\in \freenames{P}$.
\end{remark}

\subsection{ Dynamic quote: an example }

Anticipating something of what's to come, consider applying the
substitution, $\widehat{\id{\{}u / z \id{\}}}$, to the following pair
of processes, $\lift{w}{y!(z)}$ and $w[ \lpquote y!(z) \rpquote ]$.

\begin{eqnarray}
	\lift{w}{y!(z)}\widehat{\id{\{}u / z \id{\}}}
		& = &
		\lift{w}{y!(u)} \nonumber\\
	w[ \lpquote y!(z) \rpquote ] \widehat{ \id{\{}u / z \id{\}} }
		& = &
		w[ \lpquote y!(z) \rpquote ] \nonumber
\end{eqnarray}

Because the body of the process between quotes is impervious to
substitution, we get radically different answers. In fact, by
examining the first process in an input context,
e.g. $x?(z).\lift{w}{y!(z)}$, we see that the process under the lift
operator may be shaped by prefixed inputs binding a name inside it. In
this sense, the lift operator will be seen as a way to dynamically
construct processes before reifying them as names.

Finally equipped with these standard features we can present the
dynamics of the calculus.

\subsubsection{Operational semantics} 

Finally, we introduce the computational dynamics. What marks these
algebras as distinct from other more traditionally studied algebraic
structures, e.g. vector spaces or polynomial rings, is the manner in
which dynamics is captured. In traditional structures, dynamics is typically
expressed through morphisms between such structures, as in linear maps
between vector spaces or morphisms between rings. In algebras
associated with the semantics of computation, the dynamics is
expressed as part of the algebraic structure itself, through a
reduction reduction relation typically denoted by $\red$. Below, we
give a recursive presentation of this relation for the calculus used
in the encoding.

$\red \subseteq \pi \times \pi$
$\red : \pi \to \mathcal{P}(\pi)$

\begin{mathpar}
  \inferrule* [lab=Comm] { \textsf{match}( x_{src}, x_{trgt} ) } { x_{trgt}?(y)P \; | \; x_{src}!\langle {Q} \rangle \red P\{\quotep{Q}/y}\} }
  \and \\
  \inferrule* [lab=Par] {{P} \red {P}'} {{{P} | {Q}} \red {{P}' | {Q}}}
  \and
  \inferrule* [lab=Equiv]{{{P} \scong {P}'} \andalso {{P}' \red {Q}'} \andalso {{Q}' \scong {Q}}}{{P} \red {Q}}
\end{mathpar}

\begin{eqnarray*}
  match_{\equiv} (\quotep{P},\quotep{Q}) & := & P \equiv Q \\
  match_{\dagger}(\quotep{P},\quotep{Q}) & := & \forall R. P|Q \red^{*} R => R \red^{*} 0 \\
  match_{K}(\quotep{P},\quotep{Q}) & := & K \mbox{ for some context } K
\end{eqnarray*}

$u?(x)P | u!\langle Q \rangle \red P\{\quotep{Q}/x\}$

%We write $\wred$ for $\red^*$, and $P\red$ if $\exists Q $ such that $ P \red Q$.
We write $P\red$ if $\exists Q $ such that $ P \red Q$ and $P\not\red$, otherwise.

\section{Replication}

As mentioned before, it is known that replication (and hence
recursion) can be implemented in a higher-order process algebra
\cite{SangiorgiWalker}. As our first example of calculation with the
machinery thus far presented we give the construction explicitly in
the {\rhoc}.

\begin{eqnarray}
	D_{x} & := & \prefix{x}{y}{(\binpar{\outputp{x}{y}}{@{y}})} \nonumber\\
	\bangp_{x}{P} & := & \binpar{{x}!\langle{\binpar{D_{x}}{P}}\rangle}{D_{x}} \nonumber
\end{eqnarray}

\begin{eqnarray}
	\bangp_{x}{P} & & \nonumber\\
	=
	& {x}!\langle{(\prefix{x}{y}{(\outputp{x}{y} | @{y})) | P}}\rangle 
	      | \prefix{x}{y}{(\outputp{x}{y} | @{y})} & \nonumber\\
	\red
	& (\outputp{x}{y} | @{y})\substn{\quotep{(\prefix{x}{y}{(@{y} | \outputp{x}{y})) | P}}}{y} & \nonumber\\
	=
	& \outputp{x}{\quotep{(\prefix{x}{y}{(\outputp{x}{y} | @{y})) | P}}}
	  | {(\prefix{x}{y}{(\outputp{x}{y} | @{y})) | P}} & \nonumber\\
	\red
	& \ldots & \nonumber\\
	\red^*
	& P | P | \ldots & \nonumber
\end{eqnarray}

Of course, this encoding, as an implementation, runs away, unfolding
$\bangp{P}$ eagerly. A lazier and more implementable replication
operator, restricted to input-guarded processes, may be obtained as follows.

\begin{eqnarray}
\bangp{\prefix{u}{v}{P}} 
	:= 
	\binpar{\lift{x}{\prefix{u}{v}{(\binpar{D(x)}{P})}}}{D(x)} \nonumber
\end{eqnarray}

\begin{remark}
  Note that the lazier definition still does not deal with summation
  or mixed summation (i.e. sums over input and output). The reader is
  invited to construct definitions of replication that deal with these
  features. 

  Further, the definitions are parameterized in a name, $x$. Can you,
  gentle reader, make a definition that eliminates this parameter and
  guarantees no accidental interaction between the replication
  machinery and the process being replicated -- i.e. no accidental
  sharing of names used by the process to get its work done and the
  name(s) used by the replication to effect copying. This latter
  revision of the definition of replication is crucial to obtaining
  the expected identity $!!P \sim !P$.
\end{remark}

\begin{remark}\label{rem:paradoxical_combinator}
  The reader familiar with the lambda calculus will have noticed the
  similarity between $D$ and the paradoxical combinator.

  [Ed. note: the existence of this seems to suggest we have to be more
  restrictive on the set of processes and names we admit if we are to
  support no-cloning.]
\end{remark}

\subsubsection{Bisimulation}

The computational dynamics gives rise to another kind of equivalence,
the equivalence of computational behavior. As previously mentioned
this is typically captured \emph{via} some form of bisimulation.

% The notion we use in this paper is weak barbed bisimulation
% \cite{milner91polyadicpi}.

The notion we use in this paper is derived from weak barbed
bisimulation \cite{milner91polyadicpi}. 

\begin{definition}
An \emph{observation relation}, $\downarrow_{\mathcal N}$, over a set
of names, $\mathcal N$, is the smallest relation satisfying the rules
below.

\infrule[Out-barb]{y \in {\mathcal N}, \; x \nameeq y}
		  {\outputp{x}{v} \downarrow_{\mathcal N} x}
\infrule[Par-barb]{\mbox{$P\downarrow_{\mathcal N} x$ or $Q\downarrow_{\mathcal N} x$}}
		  {\binpar{P}{Q} \downarrow_{\mathcal N} x}

We write $P \Downarrow_{\mathcal N} x$ if there is $Q$ such that 
$P \wred Q$ and $Q \downarrow_{\mathcal N} x$.
\end{definition}

\begin{definition}
%\label{def.bbisim}
An  ${\mathcal N}$-\emph{barbed bisimulation} over a set of names, ${\mathcal N}$, is a symmetric binary relation 
${\mathcal S}_{\mathcal N}$ between agents such that $P\rel{S}_{\mathcal N}Q$ implies:
\begin{enumerate}
\item If $P \red P'$ then $Q \wred Q'$ and $P'\rel{S}_{\mathcal N} Q'$.
\item If $P\downarrow_{\mathcal N} x$, then $Q\Downarrow_{\mathcal N} x$.
\end{enumerate}
$P$ is ${\mathcal N}$-barbed bisimilar to $Q$, written
$P \wbbisim_{\mathcal N} Q$, if $P \rel{S}_{\mathcal N} Q$ for some ${\mathcal N}$-barbed bisimulation ${\mathcal S}_{\mathcal N}$.
\end{definition}

$\mathcal{R} \subseteq \pi \times \pi$

$P \mathcal{R} Q => \forall P'. P \red P' \Rightarrow \exists Q'. Q \red Q', P' \mathcal{R} Q'$

$P \vdash x \Rightarrow Q \vdash x$

\begin{mathpar}
  \inferrule*[lab=Out-barb]{x \nameeq y}{{y}!\langle{Q}\rangle \vdash x}
  \and
  \inferrule*[lab=Par-barb]{\mbox{$P\vdash x$ or $Q\vdash x$}}{\binpar{P}{Q} \vdash x}
\end{mathpar}

\subsubsection{Contexts}

One of the principle advantages of computational calculi like the
$\pi$-calculus is a well-defined notion of context,
contextual-equivalence and a correlation between
contextual-equivalence and notions of bisimulation. The notion of
context allows the decomposition of a process into (sub-)process and
its syntactic environment, its context. Thus, a context may be
thought of as a process with a ``hole'' (written $\Box$) in it. The
application of a context $M$ to a process $P$, written $M[P]$, is
tantamount to filling the hole in $M$ with $P$. In this paper we do
not need the full weight of this theory, but do make use of the notion
of context in the proof the main theorem. 

\begin{mathpar}
  \inferrule* [lab=summation] {} {{M_{M},M_{N}} \bc \Box \;|\; x.M_{A} \;|\; M_{M}+M_{N}}
  \and
  \inferrule* [lab=agent] {} {{M_{A}} \bc (\vec{x})M_{P} \;| \; \clift{P_0,\ldots,M_{P},\ldots,P_N}}
  \and \\
  \inferrule* [lab=process] {} {{M_{P}} \bc M_{N} \;| \;P|M_{P} }
\end{mathpar} 

\begin{mathpar}
  \inferrule* [lab=sychronization] {} {M_{N} \bc \Box \;|\; x?M_{F} \;|\; x!M_{C}}
  \and
  \inferrule* [lab=abstraction] {} {{M_{F}} \bc (x)M_{P} }
  \and
  \inferrule* [lab=concretion] {} {{M_{C}} \bc \langle M_{P} \rangle }
  \and \\
  \inferrule* [lab=process] {} {{M_{P}} \bc M_{N} \;| \;P|M_{P} }
\end{mathpar}

\begin{definition}[contextual application] Given a context $M$, and
  process $P$, we define the \emph{contextual application}, $M[P] :=
  M\{P/\Box\}$. That is, the contextual application of M to P is the
  substitution of $P$ for $\Box$ in $M$.
\end{definition}

$\meaningof{-} : L \to \mathcal{P}(\pi)$

\begin{mathpar}
  \inferrule* [lab=collection] {} {\meaningof{true} = \pi, \and \meaningof{~E} = \pi \setminus \meaningof{E}, \and \meaningof{E_{1} \& E_{2}} = \meaningof{E_{1}} \cap \meaningof{E_{2}}}
\end{mathpar}

\begin{mathpar}
  \inferrule* [lab=structure] {} {\meaningof{0} = \{ P \in \pi | P \equiv 0 \}, \and \\ \meaningof{E_1 | E_2} = \{ P \in \pi | P \equiv P_{1} | P_{2}, P_{1} \in \meaningof{E_{1}}, P_{2} \in \meaningof{E_2}\} }
\end{mathpar}

\begin{mathpar}
 \inferrule* [lab=behavior] {} {\meaningof{\langle a?b \rangle E} = \{ P \in \pi | P \equiv Q | u?(y)P', \\ \and \\\\ \and \\ \;\;\; u \in \meaningof{a}, \forall z.P'\{z/y\} \in \meaningof{E\{z/b\}}\}, \and \\ \meaningof{a!E} = \{ P \in \pi | P \equiv Q | x!\langle P' \rangle, x \in \meaningof{a} P' \in \meaningof{E}\} }
\end{mathpar}

\begin{mathpar}
 \inferrule* [lab=nominal] {} {\meaningof{\quotep{E}} = \{ \quotep{P} \in \quotep{\pi} | P \in \meaningof{E} \}, \and \meaningof{\quotep{P}} = \{ \quotep{Q} \in \quotep{\pi} | P \equiv Q \} \and \\ \meaningof{@\quotep{E}} = \{ P \in \pi | P \equiv @x, x \in \meaningof{E} \}}
\end{mathpar}

\begin{eqnarray*}
  \\
  \meaningof{-} : TS \to ST
\end{eqnarray*}

\begin{eqnarray*}
  \\
  L : TS \to ST
\end{eqnarray*}

\begin{eqnarray*}
  \\
  P \models E \iff P \in \meaningof{E}
\end{eqnarray*}

\begin{eqnarray*}
  P \approx_{L} Q \iff \forall E \in L. P \models E \iff Q \models E
\end{eqnarray*}

\begin{eqnarray*}
  P \approx_{K} Q
\end{eqnarray*}

\begin{eqnarray*}
  P \approx Q
\end{eqnarray*}

$\approx_{K} = \approx = \approx_{L}$

\subsubsection{Contextual duality}

Note that contexts extend the quotation operation to a family of
operations from processes to names. Given a context, $M$, we can
define a \emph{nominal context}, $\quotep{M}$ by $\quotep{M}[P] :=
\quotep{M[P]}$. To foreshadow what is to come we observe that these
operations enjoy a duality with processes very much like the duality
between vectors and maps from vectors to scalars.

Further, because the calculus is essentially higher-order, we have a
correspondence between contexts and processes. More specifically,
given a name $x$ and a context $M$ we can construct $M^{*}_{x}$ such
that 

\begin{mathpar}
  M^{*}_{x} | \lift{x}{P} \red M[P]
\end{mathpar}

namely,

\begin{mathpar}
  M^{*}_{x} := x?(u).M[\dropn{u}]
\end{mathpar}

The dependence of $M^{*}_{x}$ on a name makes it an abstraction, 

\begin{mathpar}
  M^{*} := (x)x?(u).M[\dropn{u}]
\end{mathpar}

\subsection{Additional notation}

It will sometimes be convenient to denote the process a name
quotes. We already have the notation $x = \quotep{P}$, but it will be
convenient to introduce an alternate notation, $\procn{x}$, when we
want to emphasize the connection to the use of the name. Note that, by
virtue of name equivalence, $\quotep{\procn{x}} \nameeq x$; so, the
notation is consistent with previous definitions.

Further, because names have structure it is possible to effect
substitutions on the basis of that structure. This means we need to
upgrade our notation for substitutions, which we accomplish by
adapting comprehension notation. Thus,

\begin{mathpar}
  P\{ y / x : x \in S \}
\end{mathpar}

is interpreted to mean the process derived from P by replacing (in a
capture-avoiding manner) each occurrence of $x$ in $S$ by $y$. For example,

\begin{mathpar}
  P\{ \quotep{\procn{x}|\procn{x}} / x : x \in \freenames{P} \}
\end{mathpar}

will replace each (occurrence) of a free name $x$ in $P$ by
$\quotep{\procn{x}|\procn{x}}$.

Also, we will avail ourselves of the notation $x^{L}$ and $x^{R}$ to
denote injections of a name into disjoint copies of the name
space. There are numerous ways to accomplish this. One example can be
found in \cite{MeredithR05}. This notation overloads to vectors of
names: $\vec{x}^{\pi} := (x_{i}^{\pi} \; : \; 0 \leq i < |\vec{x}| )$ where $\pi \in \{L,R\}$.

We also use $P^{\Box} := P|\Box$.

In \cite{MeredithR05} an interpretation of the new operator is
given. It turns out that there are several possible interpretations
all enjoying the requisite algebraic properties of the operator (see
\cite{milner91polyadicpi}). We will therefore make liberal use of
$(\nu\; \vec{x})P$.

% subsection the_syntax_and_semantics_of_the_notation_system (end)   

\input{qm2pi.qmops} 

\input{qm2pi.sterngerlach} 

\input{qm2pi.metric} 

% section concurrent_process_calculi (end)

%\input{qm2pi.proofsketch}

% section proof sketch (end)

%\input{qm2pi.slviaknots} 

% section spatial logic via knots (end)

\input{qm2pi.conclusion}

% section conclusion (end)

%\input{qm2pi.dtcodes} 

% section wiring algorithm (end)

\input{qm2pi.ack} 

% section acknowledgments (end)

\newpage


\bibliographystyle{plain}   
\bibliography{../../biblios/main.bib}

\input{qm2pi.rhodetails}

\end{document}



\end{document}

 

%\documentclass[12pt]{llncs}
%\documentclass{jktr}

\usepackage[pdftex]{hyperref}                   
\usepackage {listings}
\usepackage {mathpartir}
\usepackage{bcprules}
%\usepackage{listings}
                       
\usepackage{graphicx} 
%\usepackage[margins=2.5cm,nohead,nofoot]{geometry}
%\usepackage{geometry}
\usepackage{amsfonts}
\usepackage{amstext}
\usepackage{latexsym}
\usepackage{amssymb}
\usepackage{color}


%\include{myPreamble}
\documentclass[12pt]{llncs}
%\documentclass{jktr}

\usepackage[pdftex]{hyperref}                   
\usepackage {listings}
\usepackage {mathpartir}
\usepackage{bcprules}
%\usepackage{listings}
                       
\usepackage{graphicx} 
%\usepackage[margins=2.5cm,nohead,nofoot]{geometry}
%\usepackage{geometry}
\usepackage{amsfonts}
\usepackage{amstext}
\usepackage{latexsym}
\usepackage{amssymb}
\usepackage{color}


%\include{myPreamble}
\include{qm2pi.local} 

%\ifpdf
%\usepackage[pdftex]{graphicx}
%\else
%\usepackage{graphicx}
%\fi

 % \ifpdf
%  \usepackage{pdfsync}
%  \if


%\title{Brief Article}
%\author{David F. Snyder}
%\author{L.G. Meredith}

%\address{Dept. of Math., Texas State University--San Marcos, San Marcos, TX 78666}
       
\pagestyle{empty}


\begin{document}

\lstset{language=[Objective]Caml,frame=shadowbox}

\input{qm2pi.front}

% section front matter (end)

\input{qm2pi.intro} 
 
% section introduction (end)

% \input{qm2pi.knotations} 

% section notation (end)

\input{qm2pi.process.calculi} 

% section concurrent_process_calculi_and_spatial_logics_ (end)
    
%\input{qm2pi.knots2pi} 

%\input{qm2pi.trefoil} 

%\input{qm2pi.mainthm} 

% subsection basic_interpretation (end)

%\input{qm2pi.rho.presentation} 
\subsection{The syntax and semantics of the notation system}\label{sub:the_syntax_and_semantics_of_the_notation_system} % (fold)

We now summarize a technical presentation of the calculus that
embodies our theory of dynamics. The typical presentation of such a
calculus follows the style of giving generators and relations on
them. The grammar, below, describing term constructors, freely
generates the set of processes, $\Proc$. This set is then quotiented
by a relation known as structural congruence and it is over this set
that the notion of dynamics is expressed. This presentation is
essentially that of \cite{MeredithR05} with the addition of
polyadicity and summation. For readability we have relegated some of
the technical subtleties to an appendix.

\subsubsection{Process grammar}\label{subsub:process_grammar}

\begin{mathpar}
  \inferrule* [lab=synchronization] {} {{M} \bc \pzero \;|\; x?F \;|\; x!C }
  \and
  \inferrule* [lab=abstraction] {} {{F} \bc (x)P}
  \and
  \inferrule* [lab=concretion] {} {{C} \bc \langle Q \rangle}
  \and
  \inferrule* [lab=process] {} {{P,Q} \bc M \;| \;P|Q \;|\; @{x}}
  \and
  \inferrule* [lab=name] {} {{x} \bc \quotep{P}}
\end{mathpar} 

Note that $\vec{x}$ (resp. $\vec{P}$) denotes a vector of names
(resp. processes) of length $|\vec{x}|$ (resp. $|\vec{P}|$). We adopt
the following useful abbreviations.

\begin{mathpar}
   x?(\vec{y}).P := x.(\vec{y})P \and  x\clift{\vec{P}} := x.\clift{\vec{P}}
   \and x!(y) := \lift{x}{\dropn{y}}
   \and \Pi_{i=0}^{n-1}P_i := P_0 | \ldots | P_{n-1}
\end{mathpar}

\subsubsection{Structural congruence}

\paragraph{Free and bound names and alpha-equivalence.} At the
core of structural equivalence is alpha-equivalence which identifies
process that are the same up to a change of variable. Formally, we
recognize the distinction between free and bound names. The free names
of a process, $\freenames{P}$, may be calculated recursively as
follows:

\begin{mathpar}
\freenames{\pzero} := \emptyset
  \and \\
  \freenames{x?(y).P} := \{ x \} \cup (\freenames{P} \setminus \{ y \})
  \and 
  \freenames{x!\langle P \rangle} := \{ x \} \cup \{ P \} 
  \and \\
  \freenames{P|Q} := \freenames{P} \cup \freenames{Q}
  \and \\
  \freenames{@{x}} := \{ x \}
\end{mathpar}

$\pi$
$\quotep{\pi}$

$\freenames{-} : \pi \to \mathcal{P}(\quotep{\pi})$

\begin{eqnarray*}
  \freenames{\pzero} & := & \emptyset \\
  \freenames{x?(y).P} & := & \{ x \} \cup (\freenames{P} \setminus \{ y \}) \\
  \freenames{x!\langle P \rangle} & := & \{ x \} \cup \{ P \} \\
  \freenames{P|Q} & := & \freenames{P} \cup \freenames{Q} \\
  \freenames{\dropn{x}} & := & \{ x \}
\end{eqnarray*}

The bound names of a process, $\boundnames{P}$, are those names occurring in $P$
that are not free. For example, in $x?(y).0$, the name $x$ is free, while $y$ is bound.

\begin{mathpar}
  \inferrule* [lab=monoidal-laws] {} { P|Q \equiv Q|P \and P|0 \equiv P \and P|(Q|R) \equiv (P|Q)|R }
\end{mathpar}

\begin{mathpar}
  \inferrule* [lab=alpha-equivalence] {} { (x)P \equiv (y)P\{y/x\} \and y \not\in \freenames{P} }
\end{mathpar}

\begin{definition}
Then two processes, $P,Q$, are alpha-equivalent if $P = Q\{\vec{y}/\vec{x}\}$ for
some $\vec{x} \in \boundnames{Q},\vec{y} \in \boundnames{P}$, where $Q\{\vec{y}/\vec{x}\}$
denotes the capture-avoiding substitution of $\vec{y}$ for $\vec{x}$ in $Q$.
\end{definition}

\begin{definition}
  The {\em structural congruence} \cite{SangiorgiWalker} , $\equiv$,
  between processes is the least congruence containing
  alpha-equivalence, satisfying the abelian monoid laws
  (associativity, commutativity and $\pzero$ as identity) for parallel
  composition $|$ and for summation $+$.
\end{definition}

\subsection{Name equivalence}

We take name equivalence, written $\nameeq$, to be the smallest
equivalence relation generated by the following rules.

\begin{mathpar}
\inferrule*[lab=Quote-drop]
{ }
{ \quotep{@{x}} \nameeq x }

\inferrule*[lab=Struct-equiv]
{ P \scong Q }
{ \quotep{P} \nameeq \quotep{Q} }
\end{mathpar}

The astute reader will have noticed that the mutual recursion of names
and processes imposes a mutual recursion on alpha-equivalence and
structural equivalence via name-equivalence. Fortunately, all of this
works out pleasantly and we may calculate in the natural way, free of
concern. The reader interested in the details is referred to the
appendix \ref{appendix:rho_details}.

\subsection{Substitution}

We use $\Proc$ for the set of processes, $\QProc$ for the set of
names, and $\id{\{}\vec{y} / \vec{x} \id{\}}$ to denote partial maps,
$s : \QProc \rightarrow \QProc$. A map, $s$ lifts, uniquely, to a map
on process terms, $\widehat{s} : \Proc \rightarrow \Proc$ by the
following equations.

\begin{mathpar}
  (0) \psubstp{Q}{P} := 0 \\
  (R \juxtap S) \psubstp{Q}{P}
  :=    
  (R)\psubstp{Q}{P} \juxtap (S) \psubstp{Q}{P} \\
  (x?(y).R) \psubstp{Q}{P}    
  :=    
  (x)\substp{Q}{P} (z)\concat( (R \psubstn{z}{y}) \psubstp{Q}{P} ) \\
  (\lift{x}{R}) \psubstp{Q}{P}  
  :=
  \lift{(x)\substp{Q}{P}}{ R \psubstp{Q}{P} } \\
%   (\dropn{x})  \psubstp{Q}{P}       
%   := 
%   \left\{ 
%     \begin{array}{ccc} 
%       \dropn{\quotep{Q}} & & x \nameeq \quotep{P} \\
%       \dropn{x} & & otherwise \\
%     \end{array}
%   \right. 
  (\dropn{x})  \psubstp{Q}{P}       
  := 
  \left\{ 
    \begin{array}{ccc} 
      Q & & x \nameeq \quotep{P} \\
      \dropn{x} & & otherwise \\
    \end{array}
  \right.
\end{mathpar}
 

where

\begin{eqnarray}
  (x)\id{\{} \lpquote Q \rpquote / \lpquote P \rpquote \id{\}}            = 
  \left\{ 
    \begin{array}{ccc}
      \lpquote Q \rpquote & & x \nameeq \lpquote P \rpquote \\
      x & & otherwise \\
    \end{array}
  \right. \nonumber
\end{eqnarray}

and $z$ is chosen distinct from $\quotep{P}$, $\quotep{Q}$, the free
names in $Q$, and all the names in $R$. Our $\alpha$-equivalence will
be built in the standard way from this substitution.

\begin{remark}\label{rem:no_self_referential_names}
  One consequence of these definitions is that $\forall P. \quotep{P}
  \not\in \freenames{P}$.
\end{remark}

\subsection{ Dynamic quote: an example }

Anticipating something of what's to come, consider applying the
substitution, $\widehat{\id{\{}u / z \id{\}}}$, to the following pair
of processes, $\lift{w}{y!(z)}$ and $w[ \lpquote y!(z) \rpquote ]$.

\begin{eqnarray}
	\lift{w}{y!(z)}\widehat{\id{\{}u / z \id{\}}}
		& = &
		\lift{w}{y!(u)} \nonumber\\
	w[ \lpquote y!(z) \rpquote ] \widehat{ \id{\{}u / z \id{\}} }
		& = &
		w[ \lpquote y!(z) \rpquote ] \nonumber
\end{eqnarray}

Because the body of the process between quotes is impervious to
substitution, we get radically different answers. In fact, by
examining the first process in an input context,
e.g. $x?(z).\lift{w}{y!(z)}$, we see that the process under the lift
operator may be shaped by prefixed inputs binding a name inside it. In
this sense, the lift operator will be seen as a way to dynamically
construct processes before reifying them as names.

Finally equipped with these standard features we can present the
dynamics of the calculus.

\subsubsection{Operational semantics} 

Finally, we introduce the computational dynamics. What marks these
algebras as distinct from other more traditionally studied algebraic
structures, e.g. vector spaces or polynomial rings, is the manner in
which dynamics is captured. In traditional structures, dynamics is typically
expressed through morphisms between such structures, as in linear maps
between vector spaces or morphisms between rings. In algebras
associated with the semantics of computation, the dynamics is
expressed as part of the algebraic structure itself, through a
reduction reduction relation typically denoted by $\red$. Below, we
give a recursive presentation of this relation for the calculus used
in the encoding.

$\red \subseteq \pi \times \pi$
$\red : \pi \to \mathcal{P}(\pi)$

\begin{mathpar}
  \inferrule* [lab=Comm] { \textsf{match}( x_{src}, x_{trgt} ) } { x_{trgt}?(y)P \; | \; x_{src}!\langle {Q} \rangle \red P\{\quotep{Q}/y}\} }
  \and \\
  \inferrule* [lab=Par] {{P} \red {P}'} {{{P} | {Q}} \red {{P}' | {Q}}}
  \and
  \inferrule* [lab=Equiv]{{{P} \scong {P}'} \andalso {{P}' \red {Q}'} \andalso {{Q}' \scong {Q}}}{{P} \red {Q}}
\end{mathpar}

\begin{eqnarray*}
  match_{\equiv} (\quotep{P},\quotep{Q}) & := & P \equiv Q \\
  match_{\dagger}(\quotep{P},\quotep{Q}) & := & \forall R. P|Q \red^{*} R => R \red^{*} 0 \\
  match_{K}(\quotep{P},\quotep{Q}) & := & K \mbox{ for some context } K
\end{eqnarray*}

$u?(x)P | u!\langle Q \rangle \red P\{\quotep{Q}/x\}$

%We write $\wred$ for $\red^*$, and $P\red$ if $\exists Q $ such that $ P \red Q$.
We write $P\red$ if $\exists Q $ such that $ P \red Q$ and $P\not\red$, otherwise.

\section{Replication}

As mentioned before, it is known that replication (and hence
recursion) can be implemented in a higher-order process algebra
\cite{SangiorgiWalker}. As our first example of calculation with the
machinery thus far presented we give the construction explicitly in
the {\rhoc}.

\begin{eqnarray}
	D_{x} & := & \prefix{x}{y}{(\binpar{\outputp{x}{y}}{@{y}})} \nonumber\\
	\bangp_{x}{P} & := & \binpar{{x}!\langle{\binpar{D_{x}}{P}}\rangle}{D_{x}} \nonumber
\end{eqnarray}

\begin{eqnarray}
	\bangp_{x}{P} & & \nonumber\\
	=
	& {x}!\langle{(\prefix{x}{y}{(\outputp{x}{y} | @{y})) | P}}\rangle 
	      | \prefix{x}{y}{(\outputp{x}{y} | @{y})} & \nonumber\\
	\red
	& (\outputp{x}{y} | @{y})\substn{\quotep{(\prefix{x}{y}{(@{y} | \outputp{x}{y})) | P}}}{y} & \nonumber\\
	=
	& \outputp{x}{\quotep{(\prefix{x}{y}{(\outputp{x}{y} | @{y})) | P}}}
	  | {(\prefix{x}{y}{(\outputp{x}{y} | @{y})) | P}} & \nonumber\\
	\red
	& \ldots & \nonumber\\
	\red^*
	& P | P | \ldots & \nonumber
\end{eqnarray}

Of course, this encoding, as an implementation, runs away, unfolding
$\bangp{P}$ eagerly. A lazier and more implementable replication
operator, restricted to input-guarded processes, may be obtained as follows.

\begin{eqnarray}
\bangp{\prefix{u}{v}{P}} 
	:= 
	\binpar{\lift{x}{\prefix{u}{v}{(\binpar{D(x)}{P})}}}{D(x)} \nonumber
\end{eqnarray}

\begin{remark}
  Note that the lazier definition still does not deal with summation
  or mixed summation (i.e. sums over input and output). The reader is
  invited to construct definitions of replication that deal with these
  features. 

  Further, the definitions are parameterized in a name, $x$. Can you,
  gentle reader, make a definition that eliminates this parameter and
  guarantees no accidental interaction between the replication
  machinery and the process being replicated -- i.e. no accidental
  sharing of names used by the process to get its work done and the
  name(s) used by the replication to effect copying. This latter
  revision of the definition of replication is crucial to obtaining
  the expected identity $!!P \sim !P$.
\end{remark}

\begin{remark}\label{rem:paradoxical_combinator}
  The reader familiar with the lambda calculus will have noticed the
  similarity between $D$ and the paradoxical combinator.

  [Ed. note: the existence of this seems to suggest we have to be more
  restrictive on the set of processes and names we admit if we are to
  support no-cloning.]
\end{remark}

\subsubsection{Bisimulation}

The computational dynamics gives rise to another kind of equivalence,
the equivalence of computational behavior. As previously mentioned
this is typically captured \emph{via} some form of bisimulation.

% The notion we use in this paper is weak barbed bisimulation
% \cite{milner91polyadicpi}.

The notion we use in this paper is derived from weak barbed
bisimulation \cite{milner91polyadicpi}. 

\begin{definition}
An \emph{observation relation}, $\downarrow_{\mathcal N}$, over a set
of names, $\mathcal N$, is the smallest relation satisfying the rules
below.

\infrule[Out-barb]{y \in {\mathcal N}, \; x \nameeq y}
		  {\outputp{x}{v} \downarrow_{\mathcal N} x}
\infrule[Par-barb]{\mbox{$P\downarrow_{\mathcal N} x$ or $Q\downarrow_{\mathcal N} x$}}
		  {\binpar{P}{Q} \downarrow_{\mathcal N} x}

We write $P \Downarrow_{\mathcal N} x$ if there is $Q$ such that 
$P \wred Q$ and $Q \downarrow_{\mathcal N} x$.
\end{definition}

\begin{definition}
%\label{def.bbisim}
An  ${\mathcal N}$-\emph{barbed bisimulation} over a set of names, ${\mathcal N}$, is a symmetric binary relation 
${\mathcal S}_{\mathcal N}$ between agents such that $P\rel{S}_{\mathcal N}Q$ implies:
\begin{enumerate}
\item If $P \red P'$ then $Q \wred Q'$ and $P'\rel{S}_{\mathcal N} Q'$.
\item If $P\downarrow_{\mathcal N} x$, then $Q\Downarrow_{\mathcal N} x$.
\end{enumerate}
$P$ is ${\mathcal N}$-barbed bisimilar to $Q$, written
$P \wbbisim_{\mathcal N} Q$, if $P \rel{S}_{\mathcal N} Q$ for some ${\mathcal N}$-barbed bisimulation ${\mathcal S}_{\mathcal N}$.
\end{definition}

$\mathcal{R} \subseteq \pi \times \pi$

$P \mathcal{R} Q => \forall P'. P \red P' \Rightarrow \exists Q'. Q \red Q', P' \mathcal{R} Q'$

$P \vdash x \Rightarrow Q \vdash x$

\begin{mathpar}
  \inferrule*[lab=Out-barb]{x \nameeq y}{{y}!\langle{Q}\rangle \vdash x}
  \and
  \inferrule*[lab=Par-barb]{\mbox{$P\vdash x$ or $Q\vdash x$}}{\binpar{P}{Q} \vdash x}
\end{mathpar}

\subsubsection{Contexts}

One of the principle advantages of computational calculi like the
$\pi$-calculus is a well-defined notion of context,
contextual-equivalence and a correlation between
contextual-equivalence and notions of bisimulation. The notion of
context allows the decomposition of a process into (sub-)process and
its syntactic environment, its context. Thus, a context may be
thought of as a process with a ``hole'' (written $\Box$) in it. The
application of a context $M$ to a process $P$, written $M[P]$, is
tantamount to filling the hole in $M$ with $P$. In this paper we do
not need the full weight of this theory, but do make use of the notion
of context in the proof the main theorem. 

\begin{mathpar}
  \inferrule* [lab=summation] {} {{M_{M},M_{N}} \bc \Box \;|\; x.M_{A} \;|\; M_{M}+M_{N}}
  \and
  \inferrule* [lab=agent] {} {{M_{A}} \bc (\vec{x})M_{P} \;| \; \clift{P_0,\ldots,M_{P},\ldots,P_N}}
  \and \\
  \inferrule* [lab=process] {} {{M_{P}} \bc M_{N} \;| \;P|M_{P} }
\end{mathpar} 

\begin{mathpar}
  \inferrule* [lab=sychronization] {} {M_{N} \bc \Box \;|\; x?M_{F} \;|\; x!M_{C}}
  \and
  \inferrule* [lab=abstraction] {} {{M_{F}} \bc (x)M_{P} }
  \and
  \inferrule* [lab=concretion] {} {{M_{C}} \bc \langle M_{P} \rangle }
  \and \\
  \inferrule* [lab=process] {} {{M_{P}} \bc M_{N} \;| \;P|M_{P} }
\end{mathpar}

\begin{definition}[contextual application] Given a context $M$, and
  process $P$, we define the \emph{contextual application}, $M[P] :=
  M\{P/\Box\}$. That is, the contextual application of M to P is the
  substitution of $P$ for $\Box$ in $M$.
\end{definition}

$\meaningof{-} : L \to \mathcal{P}(\pi)$

\begin{mathpar}
  \inferrule* [lab=collection] {} {\meaningof{true} = \pi, \and \meaningof{~E} = \pi \setminus \meaningof{E}, \and \meaningof{E_{1} \& E_{2}} = \meaningof{E_{1}} \cap \meaningof{E_{2}}}
\end{mathpar}

\begin{mathpar}
  \inferrule* [lab=structure] {} {\meaningof{0} = \{ P \in \pi | P \equiv 0 \}, \and \\ \meaningof{E_1 | E_2} = \{ P \in \pi | P \equiv P_{1} | P_{2}, P_{1} \in \meaningof{E_{1}}, P_{2} \in \meaningof{E_2}\} }
\end{mathpar}

\begin{mathpar}
 \inferrule* [lab=behavior] {} {\meaningof{\langle a?b \rangle E} = \{ P \in \pi | P \equiv Q | u?(y)P', \\ \and \\\\ \and \\ \;\;\; u \in \meaningof{a}, \forall z.P'\{z/y\} \in \meaningof{E\{z/b\}}\}, \and \\ \meaningof{a!E} = \{ P \in \pi | P \equiv Q | x!\langle P' \rangle, x \in \meaningof{a} P' \in \meaningof{E}\} }
\end{mathpar}

\begin{mathpar}
 \inferrule* [lab=nominal] {} {\meaningof{\quotep{E}} = \{ \quotep{P} \in \quotep{\pi} | P \in \meaningof{E} \}, \and \meaningof{\quotep{P}} = \{ \quotep{Q} \in \quotep{\pi} | P \equiv Q \} \and \\ \meaningof{@\quotep{E}} = \{ P \in \pi | P \equiv @x, x \in \meaningof{E} \}}
\end{mathpar}

\begin{eqnarray*}
  \\
  \meaningof{-} : TS \to ST
\end{eqnarray*}

\begin{eqnarray*}
  \\
  L : TS \to ST
\end{eqnarray*}

\begin{eqnarray*}
  \\
  P \models E \iff P \in \meaningof{E}
\end{eqnarray*}

\begin{eqnarray*}
  P \approx_{L} Q \iff \forall E \in L. P \models E \iff Q \models E
\end{eqnarray*}

\begin{eqnarray*}
  P \approx_{K} Q
\end{eqnarray*}

\begin{eqnarray*}
  P \approx Q
\end{eqnarray*}

$\approx_{K} = \approx = \approx_{L}$

\subsubsection{Contextual duality}

Note that contexts extend the quotation operation to a family of
operations from processes to names. Given a context, $M$, we can
define a \emph{nominal context}, $\quotep{M}$ by $\quotep{M}[P] :=
\quotep{M[P]}$. To foreshadow what is to come we observe that these
operations enjoy a duality with processes very much like the duality
between vectors and maps from vectors to scalars.

Further, because the calculus is essentially higher-order, we have a
correspondence between contexts and processes. More specifically,
given a name $x$ and a context $M$ we can construct $M^{*}_{x}$ such
that 

\begin{mathpar}
  M^{*}_{x} | \lift{x}{P} \red M[P]
\end{mathpar}

namely,

\begin{mathpar}
  M^{*}_{x} := x?(u).M[\dropn{u}]
\end{mathpar}

The dependence of $M^{*}_{x}$ on a name makes it an abstraction, 

\begin{mathpar}
  M^{*} := (x)x?(u).M[\dropn{u}]
\end{mathpar}

\subsection{Additional notation}

It will sometimes be convenient to denote the process a name
quotes. We already have the notation $x = \quotep{P}$, but it will be
convenient to introduce an alternate notation, $\procn{x}$, when we
want to emphasize the connection to the use of the name. Note that, by
virtue of name equivalence, $\quotep{\procn{x}} \nameeq x$; so, the
notation is consistent with previous definitions.

Further, because names have structure it is possible to effect
substitutions on the basis of that structure. This means we need to
upgrade our notation for substitutions, which we accomplish by
adapting comprehension notation. Thus,

\begin{mathpar}
  P\{ y / x : x \in S \}
\end{mathpar}

is interpreted to mean the process derived from P by replacing (in a
capture-avoiding manner) each occurrence of $x$ in $S$ by $y$. For example,

\begin{mathpar}
  P\{ \quotep{\procn{x}|\procn{x}} / x : x \in \freenames{P} \}
\end{mathpar}

will replace each (occurrence) of a free name $x$ in $P$ by
$\quotep{\procn{x}|\procn{x}}$.

Also, we will avail ourselves of the notation $x^{L}$ and $x^{R}$ to
denote injections of a name into disjoint copies of the name
space. There are numerous ways to accomplish this. One example can be
found in \cite{MeredithR05}. This notation overloads to vectors of
names: $\vec{x}^{\pi} := (x_{i}^{\pi} \; : \; 0 \leq i < |\vec{x}| )$ where $\pi \in \{L,R\}$.

We also use $P^{\Box} := P|\Box$.

In \cite{MeredithR05} an interpretation of the new operator is
given. It turns out that there are several possible interpretations
all enjoying the requisite algebraic properties of the operator (see
\cite{milner91polyadicpi}). We will therefore make liberal use of
$(\nu\; \vec{x})P$.

% subsection the_syntax_and_semantics_of_the_notation_system (end)   

\input{qm2pi.qmops} 

\input{qm2pi.sterngerlach} 

\input{qm2pi.metric} 

% section concurrent_process_calculi (end)

%\input{qm2pi.proofsketch}

% section proof sketch (end)

%\input{qm2pi.slviaknots} 

% section spatial logic via knots (end)

\input{qm2pi.conclusion}

% section conclusion (end)

%\input{qm2pi.dtcodes} 

% section wiring algorithm (end)

\input{qm2pi.ack} 

% section acknowledgments (end)

\newpage


\bibliographystyle{plain}   
\bibliography{../../biblios/main.bib}

\input{qm2pi.rhodetails}

\end{document}

 

%\ifpdf
%\usepackage[pdftex]{graphicx}
%\else
%\usepackage{graphicx}
%\fi

 % \ifpdf
%  \usepackage{pdfsync}
%  \if


%\title{Brief Article}
%\author{David F. Snyder}
%\author{L.G. Meredith}

%\address{Dept. of Math., Texas State University--San Marcos, San Marcos, TX 78666}
       
\pagestyle{empty}


\begin{document}

\lstset{language=[Objective]Caml,frame=shadowbox}

\documentclass[12pt]{llncs}
%\documentclass{jktr}

\usepackage[pdftex]{hyperref}                   
\usepackage {listings}
\usepackage {mathpartir}
\usepackage{bcprules}
%\usepackage{listings}
                       
\usepackage{graphicx} 
%\usepackage[margins=2.5cm,nohead,nofoot]{geometry}
%\usepackage{geometry}
\usepackage{amsfonts}
\usepackage{amstext}
\usepackage{latexsym}
\usepackage{amssymb}
\usepackage{color}


%\include{myPreamble}
\include{qm2pi.local} 

%\ifpdf
%\usepackage[pdftex]{graphicx}
%\else
%\usepackage{graphicx}
%\fi

 % \ifpdf
%  \usepackage{pdfsync}
%  \if


%\title{Brief Article}
%\author{David F. Snyder}
%\author{L.G. Meredith}

%\address{Dept. of Math., Texas State University--San Marcos, San Marcos, TX 78666}
       
\pagestyle{empty}


\begin{document}

\lstset{language=[Objective]Caml,frame=shadowbox}

\input{qm2pi.front}

% section front matter (end)

\input{qm2pi.intro} 
 
% section introduction (end)

% \input{qm2pi.knotations} 

% section notation (end)

\input{qm2pi.process.calculi} 

% section concurrent_process_calculi_and_spatial_logics_ (end)
    
%\input{qm2pi.knots2pi} 

%\input{qm2pi.trefoil} 

%\input{qm2pi.mainthm} 

% subsection basic_interpretation (end)

%\input{qm2pi.rho.presentation} 
\subsection{The syntax and semantics of the notation system}\label{sub:the_syntax_and_semantics_of_the_notation_system} % (fold)

We now summarize a technical presentation of the calculus that
embodies our theory of dynamics. The typical presentation of such a
calculus follows the style of giving generators and relations on
them. The grammar, below, describing term constructors, freely
generates the set of processes, $\Proc$. This set is then quotiented
by a relation known as structural congruence and it is over this set
that the notion of dynamics is expressed. This presentation is
essentially that of \cite{MeredithR05} with the addition of
polyadicity and summation. For readability we have relegated some of
the technical subtleties to an appendix.

\subsubsection{Process grammar}\label{subsub:process_grammar}

\begin{mathpar}
  \inferrule* [lab=synchronization] {} {{M} \bc \pzero \;|\; x?F \;|\; x!C }
  \and
  \inferrule* [lab=abstraction] {} {{F} \bc (x)P}
  \and
  \inferrule* [lab=concretion] {} {{C} \bc \langle Q \rangle}
  \and
  \inferrule* [lab=process] {} {{P,Q} \bc M \;| \;P|Q \;|\; @{x}}
  \and
  \inferrule* [lab=name] {} {{x} \bc \quotep{P}}
\end{mathpar} 

Note that $\vec{x}$ (resp. $\vec{P}$) denotes a vector of names
(resp. processes) of length $|\vec{x}|$ (resp. $|\vec{P}|$). We adopt
the following useful abbreviations.

\begin{mathpar}
   x?(\vec{y}).P := x.(\vec{y})P \and  x\clift{\vec{P}} := x.\clift{\vec{P}}
   \and x!(y) := \lift{x}{\dropn{y}}
   \and \Pi_{i=0}^{n-1}P_i := P_0 | \ldots | P_{n-1}
\end{mathpar}

\subsubsection{Structural congruence}

\paragraph{Free and bound names and alpha-equivalence.} At the
core of structural equivalence is alpha-equivalence which identifies
process that are the same up to a change of variable. Formally, we
recognize the distinction between free and bound names. The free names
of a process, $\freenames{P}$, may be calculated recursively as
follows:

\begin{mathpar}
\freenames{\pzero} := \emptyset
  \and \\
  \freenames{x?(y).P} := \{ x \} \cup (\freenames{P} \setminus \{ y \})
  \and 
  \freenames{x!\langle P \rangle} := \{ x \} \cup \{ P \} 
  \and \\
  \freenames{P|Q} := \freenames{P} \cup \freenames{Q}
  \and \\
  \freenames{@{x}} := \{ x \}
\end{mathpar}

$\pi$
$\quotep{\pi}$

$\freenames{-} : \pi \to \mathcal{P}(\quotep{\pi})$

\begin{eqnarray*}
  \freenames{\pzero} & := & \emptyset \\
  \freenames{x?(y).P} & := & \{ x \} \cup (\freenames{P} \setminus \{ y \}) \\
  \freenames{x!\langle P \rangle} & := & \{ x \} \cup \{ P \} \\
  \freenames{P|Q} & := & \freenames{P} \cup \freenames{Q} \\
  \freenames{\dropn{x}} & := & \{ x \}
\end{eqnarray*}

The bound names of a process, $\boundnames{P}$, are those names occurring in $P$
that are not free. For example, in $x?(y).0$, the name $x$ is free, while $y$ is bound.

\begin{mathpar}
  \inferrule* [lab=monoidal-laws] {} { P|Q \equiv Q|P \and P|0 \equiv P \and P|(Q|R) \equiv (P|Q)|R }
\end{mathpar}

\begin{mathpar}
  \inferrule* [lab=alpha-equivalence] {} { (x)P \equiv (y)P\{y/x\} \and y \not\in \freenames{P} }
\end{mathpar}

\begin{definition}
Then two processes, $P,Q$, are alpha-equivalent if $P = Q\{\vec{y}/\vec{x}\}$ for
some $\vec{x} \in \boundnames{Q},\vec{y} \in \boundnames{P}$, where $Q\{\vec{y}/\vec{x}\}$
denotes the capture-avoiding substitution of $\vec{y}$ for $\vec{x}$ in $Q$.
\end{definition}

\begin{definition}
  The {\em structural congruence} \cite{SangiorgiWalker} , $\equiv$,
  between processes is the least congruence containing
  alpha-equivalence, satisfying the abelian monoid laws
  (associativity, commutativity and $\pzero$ as identity) for parallel
  composition $|$ and for summation $+$.
\end{definition}

\subsection{Name equivalence}

We take name equivalence, written $\nameeq$, to be the smallest
equivalence relation generated by the following rules.

\begin{mathpar}
\inferrule*[lab=Quote-drop]
{ }
{ \quotep{@{x}} \nameeq x }

\inferrule*[lab=Struct-equiv]
{ P \scong Q }
{ \quotep{P} \nameeq \quotep{Q} }
\end{mathpar}

The astute reader will have noticed that the mutual recursion of names
and processes imposes a mutual recursion on alpha-equivalence and
structural equivalence via name-equivalence. Fortunately, all of this
works out pleasantly and we may calculate in the natural way, free of
concern. The reader interested in the details is referred to the
appendix \ref{appendix:rho_details}.

\subsection{Substitution}

We use $\Proc$ for the set of processes, $\QProc$ for the set of
names, and $\id{\{}\vec{y} / \vec{x} \id{\}}$ to denote partial maps,
$s : \QProc \rightarrow \QProc$. A map, $s$ lifts, uniquely, to a map
on process terms, $\widehat{s} : \Proc \rightarrow \Proc$ by the
following equations.

\begin{mathpar}
  (0) \psubstp{Q}{P} := 0 \\
  (R \juxtap S) \psubstp{Q}{P}
  :=    
  (R)\psubstp{Q}{P} \juxtap (S) \psubstp{Q}{P} \\
  (x?(y).R) \psubstp{Q}{P}    
  :=    
  (x)\substp{Q}{P} (z)\concat( (R \psubstn{z}{y}) \psubstp{Q}{P} ) \\
  (\lift{x}{R}) \psubstp{Q}{P}  
  :=
  \lift{(x)\substp{Q}{P}}{ R \psubstp{Q}{P} } \\
%   (\dropn{x})  \psubstp{Q}{P}       
%   := 
%   \left\{ 
%     \begin{array}{ccc} 
%       \dropn{\quotep{Q}} & & x \nameeq \quotep{P} \\
%       \dropn{x} & & otherwise \\
%     \end{array}
%   \right. 
  (\dropn{x})  \psubstp{Q}{P}       
  := 
  \left\{ 
    \begin{array}{ccc} 
      Q & & x \nameeq \quotep{P} \\
      \dropn{x} & & otherwise \\
    \end{array}
  \right.
\end{mathpar}
 

where

\begin{eqnarray}
  (x)\id{\{} \lpquote Q \rpquote / \lpquote P \rpquote \id{\}}            = 
  \left\{ 
    \begin{array}{ccc}
      \lpquote Q \rpquote & & x \nameeq \lpquote P \rpquote \\
      x & & otherwise \\
    \end{array}
  \right. \nonumber
\end{eqnarray}

and $z$ is chosen distinct from $\quotep{P}$, $\quotep{Q}$, the free
names in $Q$, and all the names in $R$. Our $\alpha$-equivalence will
be built in the standard way from this substitution.

\begin{remark}\label{rem:no_self_referential_names}
  One consequence of these definitions is that $\forall P. \quotep{P}
  \not\in \freenames{P}$.
\end{remark}

\subsection{ Dynamic quote: an example }

Anticipating something of what's to come, consider applying the
substitution, $\widehat{\id{\{}u / z \id{\}}}$, to the following pair
of processes, $\lift{w}{y!(z)}$ and $w[ \lpquote y!(z) \rpquote ]$.

\begin{eqnarray}
	\lift{w}{y!(z)}\widehat{\id{\{}u / z \id{\}}}
		& = &
		\lift{w}{y!(u)} \nonumber\\
	w[ \lpquote y!(z) \rpquote ] \widehat{ \id{\{}u / z \id{\}} }
		& = &
		w[ \lpquote y!(z) \rpquote ] \nonumber
\end{eqnarray}

Because the body of the process between quotes is impervious to
substitution, we get radically different answers. In fact, by
examining the first process in an input context,
e.g. $x?(z).\lift{w}{y!(z)}$, we see that the process under the lift
operator may be shaped by prefixed inputs binding a name inside it. In
this sense, the lift operator will be seen as a way to dynamically
construct processes before reifying them as names.

Finally equipped with these standard features we can present the
dynamics of the calculus.

\subsubsection{Operational semantics} 

Finally, we introduce the computational dynamics. What marks these
algebras as distinct from other more traditionally studied algebraic
structures, e.g. vector spaces or polynomial rings, is the manner in
which dynamics is captured. In traditional structures, dynamics is typically
expressed through morphisms between such structures, as in linear maps
between vector spaces or morphisms between rings. In algebras
associated with the semantics of computation, the dynamics is
expressed as part of the algebraic structure itself, through a
reduction reduction relation typically denoted by $\red$. Below, we
give a recursive presentation of this relation for the calculus used
in the encoding.

$\red \subseteq \pi \times \pi$
$\red : \pi \to \mathcal{P}(\pi)$

\begin{mathpar}
  \inferrule* [lab=Comm] { \textsf{match}( x_{src}, x_{trgt} ) } { x_{trgt}?(y)P \; | \; x_{src}!\langle {Q} \rangle \red P\{\quotep{Q}/y}\} }
  \and \\
  \inferrule* [lab=Par] {{P} \red {P}'} {{{P} | {Q}} \red {{P}' | {Q}}}
  \and
  \inferrule* [lab=Equiv]{{{P} \scong {P}'} \andalso {{P}' \red {Q}'} \andalso {{Q}' \scong {Q}}}{{P} \red {Q}}
\end{mathpar}

\begin{eqnarray*}
  match_{\equiv} (\quotep{P},\quotep{Q}) & := & P \equiv Q \\
  match_{\dagger}(\quotep{P},\quotep{Q}) & := & \forall R. P|Q \red^{*} R => R \red^{*} 0 \\
  match_{K}(\quotep{P},\quotep{Q}) & := & K \mbox{ for some context } K
\end{eqnarray*}

$u?(x)P | u!\langle Q \rangle \red P\{\quotep{Q}/x\}$

%We write $\wred$ for $\red^*$, and $P\red$ if $\exists Q $ such that $ P \red Q$.
We write $P\red$ if $\exists Q $ such that $ P \red Q$ and $P\not\red$, otherwise.

\section{Replication}

As mentioned before, it is known that replication (and hence
recursion) can be implemented in a higher-order process algebra
\cite{SangiorgiWalker}. As our first example of calculation with the
machinery thus far presented we give the construction explicitly in
the {\rhoc}.

\begin{eqnarray}
	D_{x} & := & \prefix{x}{y}{(\binpar{\outputp{x}{y}}{@{y}})} \nonumber\\
	\bangp_{x}{P} & := & \binpar{{x}!\langle{\binpar{D_{x}}{P}}\rangle}{D_{x}} \nonumber
\end{eqnarray}

\begin{eqnarray}
	\bangp_{x}{P} & & \nonumber\\
	=
	& {x}!\langle{(\prefix{x}{y}{(\outputp{x}{y} | @{y})) | P}}\rangle 
	      | \prefix{x}{y}{(\outputp{x}{y} | @{y})} & \nonumber\\
	\red
	& (\outputp{x}{y} | @{y})\substn{\quotep{(\prefix{x}{y}{(@{y} | \outputp{x}{y})) | P}}}{y} & \nonumber\\
	=
	& \outputp{x}{\quotep{(\prefix{x}{y}{(\outputp{x}{y} | @{y})) | P}}}
	  | {(\prefix{x}{y}{(\outputp{x}{y} | @{y})) | P}} & \nonumber\\
	\red
	& \ldots & \nonumber\\
	\red^*
	& P | P | \ldots & \nonumber
\end{eqnarray}

Of course, this encoding, as an implementation, runs away, unfolding
$\bangp{P}$ eagerly. A lazier and more implementable replication
operator, restricted to input-guarded processes, may be obtained as follows.

\begin{eqnarray}
\bangp{\prefix{u}{v}{P}} 
	:= 
	\binpar{\lift{x}{\prefix{u}{v}{(\binpar{D(x)}{P})}}}{D(x)} \nonumber
\end{eqnarray}

\begin{remark}
  Note that the lazier definition still does not deal with summation
  or mixed summation (i.e. sums over input and output). The reader is
  invited to construct definitions of replication that deal with these
  features. 

  Further, the definitions are parameterized in a name, $x$. Can you,
  gentle reader, make a definition that eliminates this parameter and
  guarantees no accidental interaction between the replication
  machinery and the process being replicated -- i.e. no accidental
  sharing of names used by the process to get its work done and the
  name(s) used by the replication to effect copying. This latter
  revision of the definition of replication is crucial to obtaining
  the expected identity $!!P \sim !P$.
\end{remark}

\begin{remark}\label{rem:paradoxical_combinator}
  The reader familiar with the lambda calculus will have noticed the
  similarity between $D$ and the paradoxical combinator.

  [Ed. note: the existence of this seems to suggest we have to be more
  restrictive on the set of processes and names we admit if we are to
  support no-cloning.]
\end{remark}

\subsubsection{Bisimulation}

The computational dynamics gives rise to another kind of equivalence,
the equivalence of computational behavior. As previously mentioned
this is typically captured \emph{via} some form of bisimulation.

% The notion we use in this paper is weak barbed bisimulation
% \cite{milner91polyadicpi}.

The notion we use in this paper is derived from weak barbed
bisimulation \cite{milner91polyadicpi}. 

\begin{definition}
An \emph{observation relation}, $\downarrow_{\mathcal N}$, over a set
of names, $\mathcal N$, is the smallest relation satisfying the rules
below.

\infrule[Out-barb]{y \in {\mathcal N}, \; x \nameeq y}
		  {\outputp{x}{v} \downarrow_{\mathcal N} x}
\infrule[Par-barb]{\mbox{$P\downarrow_{\mathcal N} x$ or $Q\downarrow_{\mathcal N} x$}}
		  {\binpar{P}{Q} \downarrow_{\mathcal N} x}

We write $P \Downarrow_{\mathcal N} x$ if there is $Q$ such that 
$P \wred Q$ and $Q \downarrow_{\mathcal N} x$.
\end{definition}

\begin{definition}
%\label{def.bbisim}
An  ${\mathcal N}$-\emph{barbed bisimulation} over a set of names, ${\mathcal N}$, is a symmetric binary relation 
${\mathcal S}_{\mathcal N}$ between agents such that $P\rel{S}_{\mathcal N}Q$ implies:
\begin{enumerate}
\item If $P \red P'$ then $Q \wred Q'$ and $P'\rel{S}_{\mathcal N} Q'$.
\item If $P\downarrow_{\mathcal N} x$, then $Q\Downarrow_{\mathcal N} x$.
\end{enumerate}
$P$ is ${\mathcal N}$-barbed bisimilar to $Q$, written
$P \wbbisim_{\mathcal N} Q$, if $P \rel{S}_{\mathcal N} Q$ for some ${\mathcal N}$-barbed bisimulation ${\mathcal S}_{\mathcal N}$.
\end{definition}

$\mathcal{R} \subseteq \pi \times \pi$

$P \mathcal{R} Q => \forall P'. P \red P' \Rightarrow \exists Q'. Q \red Q', P' \mathcal{R} Q'$

$P \vdash x \Rightarrow Q \vdash x$

\begin{mathpar}
  \inferrule*[lab=Out-barb]{x \nameeq y}{{y}!\langle{Q}\rangle \vdash x}
  \and
  \inferrule*[lab=Par-barb]{\mbox{$P\vdash x$ or $Q\vdash x$}}{\binpar{P}{Q} \vdash x}
\end{mathpar}

\subsubsection{Contexts}

One of the principle advantages of computational calculi like the
$\pi$-calculus is a well-defined notion of context,
contextual-equivalence and a correlation between
contextual-equivalence and notions of bisimulation. The notion of
context allows the decomposition of a process into (sub-)process and
its syntactic environment, its context. Thus, a context may be
thought of as a process with a ``hole'' (written $\Box$) in it. The
application of a context $M$ to a process $P$, written $M[P]$, is
tantamount to filling the hole in $M$ with $P$. In this paper we do
not need the full weight of this theory, but do make use of the notion
of context in the proof the main theorem. 

\begin{mathpar}
  \inferrule* [lab=summation] {} {{M_{M},M_{N}} \bc \Box \;|\; x.M_{A} \;|\; M_{M}+M_{N}}
  \and
  \inferrule* [lab=agent] {} {{M_{A}} \bc (\vec{x})M_{P} \;| \; \clift{P_0,\ldots,M_{P},\ldots,P_N}}
  \and \\
  \inferrule* [lab=process] {} {{M_{P}} \bc M_{N} \;| \;P|M_{P} }
\end{mathpar} 

\begin{mathpar}
  \inferrule* [lab=sychronization] {} {M_{N} \bc \Box \;|\; x?M_{F} \;|\; x!M_{C}}
  \and
  \inferrule* [lab=abstraction] {} {{M_{F}} \bc (x)M_{P} }
  \and
  \inferrule* [lab=concretion] {} {{M_{C}} \bc \langle M_{P} \rangle }
  \and \\
  \inferrule* [lab=process] {} {{M_{P}} \bc M_{N} \;| \;P|M_{P} }
\end{mathpar}

\begin{definition}[contextual application] Given a context $M$, and
  process $P$, we define the \emph{contextual application}, $M[P] :=
  M\{P/\Box\}$. That is, the contextual application of M to P is the
  substitution of $P$ for $\Box$ in $M$.
\end{definition}

$\meaningof{-} : L \to \mathcal{P}(\pi)$

\begin{mathpar}
  \inferrule* [lab=collection] {} {\meaningof{true} = \pi, \and \meaningof{~E} = \pi \setminus \meaningof{E}, \and \meaningof{E_{1} \& E_{2}} = \meaningof{E_{1}} \cap \meaningof{E_{2}}}
\end{mathpar}

\begin{mathpar}
  \inferrule* [lab=structure] {} {\meaningof{0} = \{ P \in \pi | P \equiv 0 \}, \and \\ \meaningof{E_1 | E_2} = \{ P \in \pi | P \equiv P_{1} | P_{2}, P_{1} \in \meaningof{E_{1}}, P_{2} \in \meaningof{E_2}\} }
\end{mathpar}

\begin{mathpar}
 \inferrule* [lab=behavior] {} {\meaningof{\langle a?b \rangle E} = \{ P \in \pi | P \equiv Q | u?(y)P', \\ \and \\\\ \and \\ \;\;\; u \in \meaningof{a}, \forall z.P'\{z/y\} \in \meaningof{E\{z/b\}}\}, \and \\ \meaningof{a!E} = \{ P \in \pi | P \equiv Q | x!\langle P' \rangle, x \in \meaningof{a} P' \in \meaningof{E}\} }
\end{mathpar}

\begin{mathpar}
 \inferrule* [lab=nominal] {} {\meaningof{\quotep{E}} = \{ \quotep{P} \in \quotep{\pi} | P \in \meaningof{E} \}, \and \meaningof{\quotep{P}} = \{ \quotep{Q} \in \quotep{\pi} | P \equiv Q \} \and \\ \meaningof{@\quotep{E}} = \{ P \in \pi | P \equiv @x, x \in \meaningof{E} \}}
\end{mathpar}

\begin{eqnarray*}
  \\
  \meaningof{-} : TS \to ST
\end{eqnarray*}

\begin{eqnarray*}
  \\
  L : TS \to ST
\end{eqnarray*}

\begin{eqnarray*}
  \\
  P \models E \iff P \in \meaningof{E}
\end{eqnarray*}

\begin{eqnarray*}
  P \approx_{L} Q \iff \forall E \in L. P \models E \iff Q \models E
\end{eqnarray*}

\begin{eqnarray*}
  P \approx_{K} Q
\end{eqnarray*}

\begin{eqnarray*}
  P \approx Q
\end{eqnarray*}

$\approx_{K} = \approx = \approx_{L}$

\subsubsection{Contextual duality}

Note that contexts extend the quotation operation to a family of
operations from processes to names. Given a context, $M$, we can
define a \emph{nominal context}, $\quotep{M}$ by $\quotep{M}[P] :=
\quotep{M[P]}$. To foreshadow what is to come we observe that these
operations enjoy a duality with processes very much like the duality
between vectors and maps from vectors to scalars.

Further, because the calculus is essentially higher-order, we have a
correspondence between contexts and processes. More specifically,
given a name $x$ and a context $M$ we can construct $M^{*}_{x}$ such
that 

\begin{mathpar}
  M^{*}_{x} | \lift{x}{P} \red M[P]
\end{mathpar}

namely,

\begin{mathpar}
  M^{*}_{x} := x?(u).M[\dropn{u}]
\end{mathpar}

The dependence of $M^{*}_{x}$ on a name makes it an abstraction, 

\begin{mathpar}
  M^{*} := (x)x?(u).M[\dropn{u}]
\end{mathpar}

\subsection{Additional notation}

It will sometimes be convenient to denote the process a name
quotes. We already have the notation $x = \quotep{P}$, but it will be
convenient to introduce an alternate notation, $\procn{x}$, when we
want to emphasize the connection to the use of the name. Note that, by
virtue of name equivalence, $\quotep{\procn{x}} \nameeq x$; so, the
notation is consistent with previous definitions.

Further, because names have structure it is possible to effect
substitutions on the basis of that structure. This means we need to
upgrade our notation for substitutions, which we accomplish by
adapting comprehension notation. Thus,

\begin{mathpar}
  P\{ y / x : x \in S \}
\end{mathpar}

is interpreted to mean the process derived from P by replacing (in a
capture-avoiding manner) each occurrence of $x$ in $S$ by $y$. For example,

\begin{mathpar}
  P\{ \quotep{\procn{x}|\procn{x}} / x : x \in \freenames{P} \}
\end{mathpar}

will replace each (occurrence) of a free name $x$ in $P$ by
$\quotep{\procn{x}|\procn{x}}$.

Also, we will avail ourselves of the notation $x^{L}$ and $x^{R}$ to
denote injections of a name into disjoint copies of the name
space. There are numerous ways to accomplish this. One example can be
found in \cite{MeredithR05}. This notation overloads to vectors of
names: $\vec{x}^{\pi} := (x_{i}^{\pi} \; : \; 0 \leq i < |\vec{x}| )$ where $\pi \in \{L,R\}$.

We also use $P^{\Box} := P|\Box$.

In \cite{MeredithR05} an interpretation of the new operator is
given. It turns out that there are several possible interpretations
all enjoying the requisite algebraic properties of the operator (see
\cite{milner91polyadicpi}). We will therefore make liberal use of
$(\nu\; \vec{x})P$.

% subsection the_syntax_and_semantics_of_the_notation_system (end)   

\input{qm2pi.qmops} 

\input{qm2pi.sterngerlach} 

\input{qm2pi.metric} 

% section concurrent_process_calculi (end)

%\input{qm2pi.proofsketch}

% section proof sketch (end)

%\input{qm2pi.slviaknots} 

% section spatial logic via knots (end)

\input{qm2pi.conclusion}

% section conclusion (end)

%\input{qm2pi.dtcodes} 

% section wiring algorithm (end)

\input{qm2pi.ack} 

% section acknowledgments (end)

\newpage


\bibliographystyle{plain}   
\bibliography{../../biblios/main.bib}

\input{qm2pi.rhodetails}

\end{document}



% section front matter (end)

\section{Introduction}\label{sec:introduction} % (fold)
In this draft of the material i am going to have to dispense with the
usual writing conventions adopted in papers on these topics. i'm going
to have adopt whatever tone i need at the time i'm writing up the
calculations. Sometimes this may be very conversational; others it may
be the barest mathematical grunts; others still it may be that i have
lifted text from one of my other papers because the exposition of some
point was better said there. i hope that my readers are not unduly put
out by this decision. i'm not doing this to flout convention or be
rebellious. i find these calculations very technically challenging. To
keep everything going technically, something has to give; i have to
let go of some cognitive burden. So, the academic writing style --
with all of its trade-offs in terms of facilitating technical
communication -- is what i'm letting go of. Perhaps subsequent drafts
can be tightened and polished, but for now, i'm going to speak as if
we were sitting together in a coffee shop with a laptop, wifi and a
pad of paper and a pencil.

So, here's what i have to say. We -- you and i, comfortably ensconced
in our coffee shop and well-equipped with our tools -- can realize and
carry out the calculations of quantum mechanics over a very different
formal theory of dynamics, a formal theory of dynamics that
corresponds to a theory of concurrent computation with
\emph{reflection}. It has the advantage that the underlying theory is
already `quantized', but supports analogues all of the continuuous
operations. Strikingly, this underlying theory has recently been
connected with a notion of metric that we can show, by calculating
together, coincides with the metric induced by the inner product.

There are a lot of reasons why you might be interested in seeing
calculations of this form. Here's why i'm interested. For the past
several centuries there has been no competitor to the ``Newtonian''
account of dynamics. As a result the predominant share of accounts of
dynamical systems and situations have had to be formulated in terms of
the Newtonian machinery. i view this as an intellectually dangerous
position to occupy. Everything, despite it's intrinsic shape, turns
into a nail to be hit with this hammer. Recently, however, the theory
of computation has matured to the point where we have candidates for
theories of dynamics that offer very different perspective on
reasoning about dynamical systems and situations. Testing these
candidates against very successful accounts of dynamical situations,
like quantum mechanics, is going to give us some sense of how mature
they are and some measure of the quality of these accounts of
dynamics.

\subsection{Summary of contributions and outline of paper}

So, we're going to develop an interpretation of the operations of
quantum mechanics normally interpreted by Hilbert spaces and
operators. We're going to do this over a theory of computation. Note
that this is very different than the usual quantum computation program
which develops notions of computation over quantum mechanics. Rather,
we are developing a story that aligns with Wheeler's slogan: It from
Bit. To do this we will first provide an account of the theory of
computation at play here. Then we will dive into a calculation-driven
interpretation of the operations of quantum mechanics.

The reason we take this approach is that -- until very recently --
there hasn't been an axiomatic account of quantum mechanics. As a
result there has been no sharp delineation of the mathematical theory
supporting interpretation of the physical theory and the physical
theory, itself. So, ambient features of the maths are free to be
exploited (or supressed) without a real accounting of their physical
relevance. There is no sharp statement ``here's the physical theory''
qua \emph{theory} and ``here's the mathematical interpretation''
enabling a judgment of how faithful the interpretation is -- apart
from experimental observation. When there is an axiomatic account we
can judge how well a given mathematical formalism supports an
interpretation of the axioms, independent of
experimentation. Likewise, we can judge how well we have captured our
physical evidence and experience with our axiomatics, independent of
any specific mathematical implementation, with accidental detail that
may or may not have physical significance. 

In lieu of a fully fleshed out and vetted axiomatic account of quantum
mechanics, interpreting the operational notions in service of modeling
physical systems will have to suffice. In other words, we are not in
the business of providing a model of Hilbert spaces and operators. We
are in the business of providing a model of quantum mechanics because
we are motivated by testing our notions of dynamics against physical
theory; and, the predictive calculations of the physical theory must
serve as the best formulation -- shy of a fully fleshed out axiomatic
account -- of the physical theory itself (as they have for scientific
theories since time immemorial). Put another way, despite a
whole-hearted commitment to an It-from-Bit ontology, we are firmly
aligned with the shut-up-and-calculate camp as the best way to obtain
results either from the physical perspective or as a quality assurance
measure of our fledgling theory of dynamics.

In detail, we present a reflective process calculus. Then we develop
intuitive correspondences between the notions available in this
calculus and the usual physical notions supporting quantum mechanical
calculations. Thus, 

\begin{table}[htp]
  \center{
    \fbox{
      \begin{tabular}{c|c}
        quantum mechanics & process calculus \\
        \hline
        scalar & name \\
        state vector & process \\
        dual & contextual duals \\
        matrix & formal sums of process-context-dual pairs \\
        orthogonality & process annihilation \\
        inner product & execution-formula + quoting
      \end{tabular}
    }
  }
  \caption{QM - process calculi correspondences}
\end{table}

Then we tighten up these intuitions to operational definitions. We
employ the Dirac notation as the best proxy we can find for an
abstract syntax of the quantum mechanical notions. The definitions we
develop put us in contact with equational constraints coming from the
theory that we demonstrate the definitions and calculations satisfy.

This puts us in a position to shut up and calculate for the
Stern-Gerlach experimental set up, showing how these predictive
calculations become calculations on processes in our theory of a
reflective process calculus.

Penultimately, we demonstrate that the notion of metric coming from
the inner product coincides with the notion of metric available from
the theory of bisimulation. This demonstration gives us the right to
think of space as arising from behavior. Finally, we consider where we
might go from the new vantage point we have obtained.

% section introduction (end) 
 
% section introduction (end)

% \documentclass[12pt]{llncs}
%\documentclass{jktr}

\usepackage[pdftex]{hyperref}                   
\usepackage {listings}
\usepackage {mathpartir}
\usepackage{bcprules}
%\usepackage{listings}
                       
\usepackage{graphicx} 
%\usepackage[margins=2.5cm,nohead,nofoot]{geometry}
%\usepackage{geometry}
\usepackage{amsfonts}
\usepackage{amstext}
\usepackage{latexsym}
\usepackage{amssymb}
\usepackage{color}


%\include{myPreamble}
\include{qm2pi.local} 

%\ifpdf
%\usepackage[pdftex]{graphicx}
%\else
%\usepackage{graphicx}
%\fi

 % \ifpdf
%  \usepackage{pdfsync}
%  \if


%\title{Brief Article}
%\author{David F. Snyder}
%\author{L.G. Meredith}

%\address{Dept. of Math., Texas State University--San Marcos, San Marcos, TX 78666}
       
\pagestyle{empty}


\begin{document}

\lstset{language=[Objective]Caml,frame=shadowbox}

\input{qm2pi.front}

% section front matter (end)

\input{qm2pi.intro} 
 
% section introduction (end)

% \input{qm2pi.knotations} 

% section notation (end)

\input{qm2pi.process.calculi} 

% section concurrent_process_calculi_and_spatial_logics_ (end)
    
%\input{qm2pi.knots2pi} 

%\input{qm2pi.trefoil} 

%\input{qm2pi.mainthm} 

% subsection basic_interpretation (end)

%\input{qm2pi.rho.presentation} 
\subsection{The syntax and semantics of the notation system}\label{sub:the_syntax_and_semantics_of_the_notation_system} % (fold)

We now summarize a technical presentation of the calculus that
embodies our theory of dynamics. The typical presentation of such a
calculus follows the style of giving generators and relations on
them. The grammar, below, describing term constructors, freely
generates the set of processes, $\Proc$. This set is then quotiented
by a relation known as structural congruence and it is over this set
that the notion of dynamics is expressed. This presentation is
essentially that of \cite{MeredithR05} with the addition of
polyadicity and summation. For readability we have relegated some of
the technical subtleties to an appendix.

\subsubsection{Process grammar}\label{subsub:process_grammar}

\begin{mathpar}
  \inferrule* [lab=synchronization] {} {{M} \bc \pzero \;|\; x?F \;|\; x!C }
  \and
  \inferrule* [lab=abstraction] {} {{F} \bc (x)P}
  \and
  \inferrule* [lab=concretion] {} {{C} \bc \langle Q \rangle}
  \and
  \inferrule* [lab=process] {} {{P,Q} \bc M \;| \;P|Q \;|\; @{x}}
  \and
  \inferrule* [lab=name] {} {{x} \bc \quotep{P}}
\end{mathpar} 

Note that $\vec{x}$ (resp. $\vec{P}$) denotes a vector of names
(resp. processes) of length $|\vec{x}|$ (resp. $|\vec{P}|$). We adopt
the following useful abbreviations.

\begin{mathpar}
   x?(\vec{y}).P := x.(\vec{y})P \and  x\clift{\vec{P}} := x.\clift{\vec{P}}
   \and x!(y) := \lift{x}{\dropn{y}}
   \and \Pi_{i=0}^{n-1}P_i := P_0 | \ldots | P_{n-1}
\end{mathpar}

\subsubsection{Structural congruence}

\paragraph{Free and bound names and alpha-equivalence.} At the
core of structural equivalence is alpha-equivalence which identifies
process that are the same up to a change of variable. Formally, we
recognize the distinction between free and bound names. The free names
of a process, $\freenames{P}$, may be calculated recursively as
follows:

\begin{mathpar}
\freenames{\pzero} := \emptyset
  \and \\
  \freenames{x?(y).P} := \{ x \} \cup (\freenames{P} \setminus \{ y \})
  \and 
  \freenames{x!\langle P \rangle} := \{ x \} \cup \{ P \} 
  \and \\
  \freenames{P|Q} := \freenames{P} \cup \freenames{Q}
  \and \\
  \freenames{@{x}} := \{ x \}
\end{mathpar}

$\pi$
$\quotep{\pi}$

$\freenames{-} : \pi \to \mathcal{P}(\quotep{\pi})$

\begin{eqnarray*}
  \freenames{\pzero} & := & \emptyset \\
  \freenames{x?(y).P} & := & \{ x \} \cup (\freenames{P} \setminus \{ y \}) \\
  \freenames{x!\langle P \rangle} & := & \{ x \} \cup \{ P \} \\
  \freenames{P|Q} & := & \freenames{P} \cup \freenames{Q} \\
  \freenames{\dropn{x}} & := & \{ x \}
\end{eqnarray*}

The bound names of a process, $\boundnames{P}$, are those names occurring in $P$
that are not free. For example, in $x?(y).0$, the name $x$ is free, while $y$ is bound.

\begin{mathpar}
  \inferrule* [lab=monoidal-laws] {} { P|Q \equiv Q|P \and P|0 \equiv P \and P|(Q|R) \equiv (P|Q)|R }
\end{mathpar}

\begin{mathpar}
  \inferrule* [lab=alpha-equivalence] {} { (x)P \equiv (y)P\{y/x\} \and y \not\in \freenames{P} }
\end{mathpar}

\begin{definition}
Then two processes, $P,Q$, are alpha-equivalent if $P = Q\{\vec{y}/\vec{x}\}$ for
some $\vec{x} \in \boundnames{Q},\vec{y} \in \boundnames{P}$, where $Q\{\vec{y}/\vec{x}\}$
denotes the capture-avoiding substitution of $\vec{y}$ for $\vec{x}$ in $Q$.
\end{definition}

\begin{definition}
  The {\em structural congruence} \cite{SangiorgiWalker} , $\equiv$,
  between processes is the least congruence containing
  alpha-equivalence, satisfying the abelian monoid laws
  (associativity, commutativity and $\pzero$ as identity) for parallel
  composition $|$ and for summation $+$.
\end{definition}

\subsection{Name equivalence}

We take name equivalence, written $\nameeq$, to be the smallest
equivalence relation generated by the following rules.

\begin{mathpar}
\inferrule*[lab=Quote-drop]
{ }
{ \quotep{@{x}} \nameeq x }

\inferrule*[lab=Struct-equiv]
{ P \scong Q }
{ \quotep{P} \nameeq \quotep{Q} }
\end{mathpar}

The astute reader will have noticed that the mutual recursion of names
and processes imposes a mutual recursion on alpha-equivalence and
structural equivalence via name-equivalence. Fortunately, all of this
works out pleasantly and we may calculate in the natural way, free of
concern. The reader interested in the details is referred to the
appendix \ref{appendix:rho_details}.

\subsection{Substitution}

We use $\Proc$ for the set of processes, $\QProc$ for the set of
names, and $\id{\{}\vec{y} / \vec{x} \id{\}}$ to denote partial maps,
$s : \QProc \rightarrow \QProc$. A map, $s$ lifts, uniquely, to a map
on process terms, $\widehat{s} : \Proc \rightarrow \Proc$ by the
following equations.

\begin{mathpar}
  (0) \psubstp{Q}{P} := 0 \\
  (R \juxtap S) \psubstp{Q}{P}
  :=    
  (R)\psubstp{Q}{P} \juxtap (S) \psubstp{Q}{P} \\
  (x?(y).R) \psubstp{Q}{P}    
  :=    
  (x)\substp{Q}{P} (z)\concat( (R \psubstn{z}{y}) \psubstp{Q}{P} ) \\
  (\lift{x}{R}) \psubstp{Q}{P}  
  :=
  \lift{(x)\substp{Q}{P}}{ R \psubstp{Q}{P} } \\
%   (\dropn{x})  \psubstp{Q}{P}       
%   := 
%   \left\{ 
%     \begin{array}{ccc} 
%       \dropn{\quotep{Q}} & & x \nameeq \quotep{P} \\
%       \dropn{x} & & otherwise \\
%     \end{array}
%   \right. 
  (\dropn{x})  \psubstp{Q}{P}       
  := 
  \left\{ 
    \begin{array}{ccc} 
      Q & & x \nameeq \quotep{P} \\
      \dropn{x} & & otherwise \\
    \end{array}
  \right.
\end{mathpar}
 

where

\begin{eqnarray}
  (x)\id{\{} \lpquote Q \rpquote / \lpquote P \rpquote \id{\}}            = 
  \left\{ 
    \begin{array}{ccc}
      \lpquote Q \rpquote & & x \nameeq \lpquote P \rpquote \\
      x & & otherwise \\
    \end{array}
  \right. \nonumber
\end{eqnarray}

and $z$ is chosen distinct from $\quotep{P}$, $\quotep{Q}$, the free
names in $Q$, and all the names in $R$. Our $\alpha$-equivalence will
be built in the standard way from this substitution.

\begin{remark}\label{rem:no_self_referential_names}
  One consequence of these definitions is that $\forall P. \quotep{P}
  \not\in \freenames{P}$.
\end{remark}

\subsection{ Dynamic quote: an example }

Anticipating something of what's to come, consider applying the
substitution, $\widehat{\id{\{}u / z \id{\}}}$, to the following pair
of processes, $\lift{w}{y!(z)}$ and $w[ \lpquote y!(z) \rpquote ]$.

\begin{eqnarray}
	\lift{w}{y!(z)}\widehat{\id{\{}u / z \id{\}}}
		& = &
		\lift{w}{y!(u)} \nonumber\\
	w[ \lpquote y!(z) \rpquote ] \widehat{ \id{\{}u / z \id{\}} }
		& = &
		w[ \lpquote y!(z) \rpquote ] \nonumber
\end{eqnarray}

Because the body of the process between quotes is impervious to
substitution, we get radically different answers. In fact, by
examining the first process in an input context,
e.g. $x?(z).\lift{w}{y!(z)}$, we see that the process under the lift
operator may be shaped by prefixed inputs binding a name inside it. In
this sense, the lift operator will be seen as a way to dynamically
construct processes before reifying them as names.

Finally equipped with these standard features we can present the
dynamics of the calculus.

\subsubsection{Operational semantics} 

Finally, we introduce the computational dynamics. What marks these
algebras as distinct from other more traditionally studied algebraic
structures, e.g. vector spaces or polynomial rings, is the manner in
which dynamics is captured. In traditional structures, dynamics is typically
expressed through morphisms between such structures, as in linear maps
between vector spaces or morphisms between rings. In algebras
associated with the semantics of computation, the dynamics is
expressed as part of the algebraic structure itself, through a
reduction reduction relation typically denoted by $\red$. Below, we
give a recursive presentation of this relation for the calculus used
in the encoding.

$\red \subseteq \pi \times \pi$
$\red : \pi \to \mathcal{P}(\pi)$

\begin{mathpar}
  \inferrule* [lab=Comm] { \textsf{match}( x_{src}, x_{trgt} ) } { x_{trgt}?(y)P \; | \; x_{src}!\langle {Q} \rangle \red P\{\quotep{Q}/y}\} }
  \and \\
  \inferrule* [lab=Par] {{P} \red {P}'} {{{P} | {Q}} \red {{P}' | {Q}}}
  \and
  \inferrule* [lab=Equiv]{{{P} \scong {P}'} \andalso {{P}' \red {Q}'} \andalso {{Q}' \scong {Q}}}{{P} \red {Q}}
\end{mathpar}

\begin{eqnarray*}
  match_{\equiv} (\quotep{P},\quotep{Q}) & := & P \equiv Q \\
  match_{\dagger}(\quotep{P},\quotep{Q}) & := & \forall R. P|Q \red^{*} R => R \red^{*} 0 \\
  match_{K}(\quotep{P},\quotep{Q}) & := & K \mbox{ for some context } K
\end{eqnarray*}

$u?(x)P | u!\langle Q \rangle \red P\{\quotep{Q}/x\}$

%We write $\wred$ for $\red^*$, and $P\red$ if $\exists Q $ such that $ P \red Q$.
We write $P\red$ if $\exists Q $ such that $ P \red Q$ and $P\not\red$, otherwise.

\section{Replication}

As mentioned before, it is known that replication (and hence
recursion) can be implemented in a higher-order process algebra
\cite{SangiorgiWalker}. As our first example of calculation with the
machinery thus far presented we give the construction explicitly in
the {\rhoc}.

\begin{eqnarray}
	D_{x} & := & \prefix{x}{y}{(\binpar{\outputp{x}{y}}{@{y}})} \nonumber\\
	\bangp_{x}{P} & := & \binpar{{x}!\langle{\binpar{D_{x}}{P}}\rangle}{D_{x}} \nonumber
\end{eqnarray}

\begin{eqnarray}
	\bangp_{x}{P} & & \nonumber\\
	=
	& {x}!\langle{(\prefix{x}{y}{(\outputp{x}{y} | @{y})) | P}}\rangle 
	      | \prefix{x}{y}{(\outputp{x}{y} | @{y})} & \nonumber\\
	\red
	& (\outputp{x}{y} | @{y})\substn{\quotep{(\prefix{x}{y}{(@{y} | \outputp{x}{y})) | P}}}{y} & \nonumber\\
	=
	& \outputp{x}{\quotep{(\prefix{x}{y}{(\outputp{x}{y} | @{y})) | P}}}
	  | {(\prefix{x}{y}{(\outputp{x}{y} | @{y})) | P}} & \nonumber\\
	\red
	& \ldots & \nonumber\\
	\red^*
	& P | P | \ldots & \nonumber
\end{eqnarray}

Of course, this encoding, as an implementation, runs away, unfolding
$\bangp{P}$ eagerly. A lazier and more implementable replication
operator, restricted to input-guarded processes, may be obtained as follows.

\begin{eqnarray}
\bangp{\prefix{u}{v}{P}} 
	:= 
	\binpar{\lift{x}{\prefix{u}{v}{(\binpar{D(x)}{P})}}}{D(x)} \nonumber
\end{eqnarray}

\begin{remark}
  Note that the lazier definition still does not deal with summation
  or mixed summation (i.e. sums over input and output). The reader is
  invited to construct definitions of replication that deal with these
  features. 

  Further, the definitions are parameterized in a name, $x$. Can you,
  gentle reader, make a definition that eliminates this parameter and
  guarantees no accidental interaction between the replication
  machinery and the process being replicated -- i.e. no accidental
  sharing of names used by the process to get its work done and the
  name(s) used by the replication to effect copying. This latter
  revision of the definition of replication is crucial to obtaining
  the expected identity $!!P \sim !P$.
\end{remark}

\begin{remark}\label{rem:paradoxical_combinator}
  The reader familiar with the lambda calculus will have noticed the
  similarity between $D$ and the paradoxical combinator.

  [Ed. note: the existence of this seems to suggest we have to be more
  restrictive on the set of processes and names we admit if we are to
  support no-cloning.]
\end{remark}

\subsubsection{Bisimulation}

The computational dynamics gives rise to another kind of equivalence,
the equivalence of computational behavior. As previously mentioned
this is typically captured \emph{via} some form of bisimulation.

% The notion we use in this paper is weak barbed bisimulation
% \cite{milner91polyadicpi}.

The notion we use in this paper is derived from weak barbed
bisimulation \cite{milner91polyadicpi}. 

\begin{definition}
An \emph{observation relation}, $\downarrow_{\mathcal N}$, over a set
of names, $\mathcal N$, is the smallest relation satisfying the rules
below.

\infrule[Out-barb]{y \in {\mathcal N}, \; x \nameeq y}
		  {\outputp{x}{v} \downarrow_{\mathcal N} x}
\infrule[Par-barb]{\mbox{$P\downarrow_{\mathcal N} x$ or $Q\downarrow_{\mathcal N} x$}}
		  {\binpar{P}{Q} \downarrow_{\mathcal N} x}

We write $P \Downarrow_{\mathcal N} x$ if there is $Q$ such that 
$P \wred Q$ and $Q \downarrow_{\mathcal N} x$.
\end{definition}

\begin{definition}
%\label{def.bbisim}
An  ${\mathcal N}$-\emph{barbed bisimulation} over a set of names, ${\mathcal N}$, is a symmetric binary relation 
${\mathcal S}_{\mathcal N}$ between agents such that $P\rel{S}_{\mathcal N}Q$ implies:
\begin{enumerate}
\item If $P \red P'$ then $Q \wred Q'$ and $P'\rel{S}_{\mathcal N} Q'$.
\item If $P\downarrow_{\mathcal N} x$, then $Q\Downarrow_{\mathcal N} x$.
\end{enumerate}
$P$ is ${\mathcal N}$-barbed bisimilar to $Q$, written
$P \wbbisim_{\mathcal N} Q$, if $P \rel{S}_{\mathcal N} Q$ for some ${\mathcal N}$-barbed bisimulation ${\mathcal S}_{\mathcal N}$.
\end{definition}

$\mathcal{R} \subseteq \pi \times \pi$

$P \mathcal{R} Q => \forall P'. P \red P' \Rightarrow \exists Q'. Q \red Q', P' \mathcal{R} Q'$

$P \vdash x \Rightarrow Q \vdash x$

\begin{mathpar}
  \inferrule*[lab=Out-barb]{x \nameeq y}{{y}!\langle{Q}\rangle \vdash x}
  \and
  \inferrule*[lab=Par-barb]{\mbox{$P\vdash x$ or $Q\vdash x$}}{\binpar{P}{Q} \vdash x}
\end{mathpar}

\subsubsection{Contexts}

One of the principle advantages of computational calculi like the
$\pi$-calculus is a well-defined notion of context,
contextual-equivalence and a correlation between
contextual-equivalence and notions of bisimulation. The notion of
context allows the decomposition of a process into (sub-)process and
its syntactic environment, its context. Thus, a context may be
thought of as a process with a ``hole'' (written $\Box$) in it. The
application of a context $M$ to a process $P$, written $M[P]$, is
tantamount to filling the hole in $M$ with $P$. In this paper we do
not need the full weight of this theory, but do make use of the notion
of context in the proof the main theorem. 

\begin{mathpar}
  \inferrule* [lab=summation] {} {{M_{M},M_{N}} \bc \Box \;|\; x.M_{A} \;|\; M_{M}+M_{N}}
  \and
  \inferrule* [lab=agent] {} {{M_{A}} \bc (\vec{x})M_{P} \;| \; \clift{P_0,\ldots,M_{P},\ldots,P_N}}
  \and \\
  \inferrule* [lab=process] {} {{M_{P}} \bc M_{N} \;| \;P|M_{P} }
\end{mathpar} 

\begin{mathpar}
  \inferrule* [lab=sychronization] {} {M_{N} \bc \Box \;|\; x?M_{F} \;|\; x!M_{C}}
  \and
  \inferrule* [lab=abstraction] {} {{M_{F}} \bc (x)M_{P} }
  \and
  \inferrule* [lab=concretion] {} {{M_{C}} \bc \langle M_{P} \rangle }
  \and \\
  \inferrule* [lab=process] {} {{M_{P}} \bc M_{N} \;| \;P|M_{P} }
\end{mathpar}

\begin{definition}[contextual application] Given a context $M$, and
  process $P$, we define the \emph{contextual application}, $M[P] :=
  M\{P/\Box\}$. That is, the contextual application of M to P is the
  substitution of $P$ for $\Box$ in $M$.
\end{definition}

$\meaningof{-} : L \to \mathcal{P}(\pi)$

\begin{mathpar}
  \inferrule* [lab=collection] {} {\meaningof{true} = \pi, \and \meaningof{~E} = \pi \setminus \meaningof{E}, \and \meaningof{E_{1} \& E_{2}} = \meaningof{E_{1}} \cap \meaningof{E_{2}}}
\end{mathpar}

\begin{mathpar}
  \inferrule* [lab=structure] {} {\meaningof{0} = \{ P \in \pi | P \equiv 0 \}, \and \\ \meaningof{E_1 | E_2} = \{ P \in \pi | P \equiv P_{1} | P_{2}, P_{1} \in \meaningof{E_{1}}, P_{2} \in \meaningof{E_2}\} }
\end{mathpar}

\begin{mathpar}
 \inferrule* [lab=behavior] {} {\meaningof{\langle a?b \rangle E} = \{ P \in \pi | P \equiv Q | u?(y)P', \\ \and \\\\ \and \\ \;\;\; u \in \meaningof{a}, \forall z.P'\{z/y\} \in \meaningof{E\{z/b\}}\}, \and \\ \meaningof{a!E} = \{ P \in \pi | P \equiv Q | x!\langle P' \rangle, x \in \meaningof{a} P' \in \meaningof{E}\} }
\end{mathpar}

\begin{mathpar}
 \inferrule* [lab=nominal] {} {\meaningof{\quotep{E}} = \{ \quotep{P} \in \quotep{\pi} | P \in \meaningof{E} \}, \and \meaningof{\quotep{P}} = \{ \quotep{Q} \in \quotep{\pi} | P \equiv Q \} \and \\ \meaningof{@\quotep{E}} = \{ P \in \pi | P \equiv @x, x \in \meaningof{E} \}}
\end{mathpar}

\begin{eqnarray*}
  \\
  \meaningof{-} : TS \to ST
\end{eqnarray*}

\begin{eqnarray*}
  \\
  L : TS \to ST
\end{eqnarray*}

\begin{eqnarray*}
  \\
  P \models E \iff P \in \meaningof{E}
\end{eqnarray*}

\begin{eqnarray*}
  P \approx_{L} Q \iff \forall E \in L. P \models E \iff Q \models E
\end{eqnarray*}

\begin{eqnarray*}
  P \approx_{K} Q
\end{eqnarray*}

\begin{eqnarray*}
  P \approx Q
\end{eqnarray*}

$\approx_{K} = \approx = \approx_{L}$

\subsubsection{Contextual duality}

Note that contexts extend the quotation operation to a family of
operations from processes to names. Given a context, $M$, we can
define a \emph{nominal context}, $\quotep{M}$ by $\quotep{M}[P] :=
\quotep{M[P]}$. To foreshadow what is to come we observe that these
operations enjoy a duality with processes very much like the duality
between vectors and maps from vectors to scalars.

Further, because the calculus is essentially higher-order, we have a
correspondence between contexts and processes. More specifically,
given a name $x$ and a context $M$ we can construct $M^{*}_{x}$ such
that 

\begin{mathpar}
  M^{*}_{x} | \lift{x}{P} \red M[P]
\end{mathpar}

namely,

\begin{mathpar}
  M^{*}_{x} := x?(u).M[\dropn{u}]
\end{mathpar}

The dependence of $M^{*}_{x}$ on a name makes it an abstraction, 

\begin{mathpar}
  M^{*} := (x)x?(u).M[\dropn{u}]
\end{mathpar}

\subsection{Additional notation}

It will sometimes be convenient to denote the process a name
quotes. We already have the notation $x = \quotep{P}$, but it will be
convenient to introduce an alternate notation, $\procn{x}$, when we
want to emphasize the connection to the use of the name. Note that, by
virtue of name equivalence, $\quotep{\procn{x}} \nameeq x$; so, the
notation is consistent with previous definitions.

Further, because names have structure it is possible to effect
substitutions on the basis of that structure. This means we need to
upgrade our notation for substitutions, which we accomplish by
adapting comprehension notation. Thus,

\begin{mathpar}
  P\{ y / x : x \in S \}
\end{mathpar}

is interpreted to mean the process derived from P by replacing (in a
capture-avoiding manner) each occurrence of $x$ in $S$ by $y$. For example,

\begin{mathpar}
  P\{ \quotep{\procn{x}|\procn{x}} / x : x \in \freenames{P} \}
\end{mathpar}

will replace each (occurrence) of a free name $x$ in $P$ by
$\quotep{\procn{x}|\procn{x}}$.

Also, we will avail ourselves of the notation $x^{L}$ and $x^{R}$ to
denote injections of a name into disjoint copies of the name
space. There are numerous ways to accomplish this. One example can be
found in \cite{MeredithR05}. This notation overloads to vectors of
names: $\vec{x}^{\pi} := (x_{i}^{\pi} \; : \; 0 \leq i < |\vec{x}| )$ where $\pi \in \{L,R\}$.

We also use $P^{\Box} := P|\Box$.

In \cite{MeredithR05} an interpretation of the new operator is
given. It turns out that there are several possible interpretations
all enjoying the requisite algebraic properties of the operator (see
\cite{milner91polyadicpi}). We will therefore make liberal use of
$(\nu\; \vec{x})P$.

% subsection the_syntax_and_semantics_of_the_notation_system (end)   

\input{qm2pi.qmops} 

\input{qm2pi.sterngerlach} 

\input{qm2pi.metric} 

% section concurrent_process_calculi (end)

%\input{qm2pi.proofsketch}

% section proof sketch (end)

%\input{qm2pi.slviaknots} 

% section spatial logic via knots (end)

\input{qm2pi.conclusion}

% section conclusion (end)

%\input{qm2pi.dtcodes} 

% section wiring algorithm (end)

\input{qm2pi.ack} 

% section acknowledgments (end)

\newpage


\bibliographystyle{plain}   
\bibliography{../../biblios/main.bib}

\input{qm2pi.rhodetails}

\end{document}

 

% section notation (end)

\input{qm2pi.process.calculi} 

% section concurrent_process_calculi_and_spatial_logics_ (end)
    
%\documentclass[12pt]{llncs}
%\documentclass{jktr}

\usepackage[pdftex]{hyperref}                   
\usepackage {listings}
\usepackage {mathpartir}
\usepackage{bcprules}
%\usepackage{listings}
                       
\usepackage{graphicx} 
%\usepackage[margins=2.5cm,nohead,nofoot]{geometry}
%\usepackage{geometry}
\usepackage{amsfonts}
\usepackage{amstext}
\usepackage{latexsym}
\usepackage{amssymb}
\usepackage{color}


%\include{myPreamble}
\include{qm2pi.local} 

%\ifpdf
%\usepackage[pdftex]{graphicx}
%\else
%\usepackage{graphicx}
%\fi

 % \ifpdf
%  \usepackage{pdfsync}
%  \if


%\title{Brief Article}
%\author{David F. Snyder}
%\author{L.G. Meredith}

%\address{Dept. of Math., Texas State University--San Marcos, San Marcos, TX 78666}
       
\pagestyle{empty}


\begin{document}

\lstset{language=[Objective]Caml,frame=shadowbox}

\input{qm2pi.front}

% section front matter (end)

\input{qm2pi.intro} 
 
% section introduction (end)

% \input{qm2pi.knotations} 

% section notation (end)

\input{qm2pi.process.calculi} 

% section concurrent_process_calculi_and_spatial_logics_ (end)
    
%\input{qm2pi.knots2pi} 

%\input{qm2pi.trefoil} 

%\input{qm2pi.mainthm} 

% subsection basic_interpretation (end)

%\input{qm2pi.rho.presentation} 
\subsection{The syntax and semantics of the notation system}\label{sub:the_syntax_and_semantics_of_the_notation_system} % (fold)

We now summarize a technical presentation of the calculus that
embodies our theory of dynamics. The typical presentation of such a
calculus follows the style of giving generators and relations on
them. The grammar, below, describing term constructors, freely
generates the set of processes, $\Proc$. This set is then quotiented
by a relation known as structural congruence and it is over this set
that the notion of dynamics is expressed. This presentation is
essentially that of \cite{MeredithR05} with the addition of
polyadicity and summation. For readability we have relegated some of
the technical subtleties to an appendix.

\subsubsection{Process grammar}\label{subsub:process_grammar}

\begin{mathpar}
  \inferrule* [lab=synchronization] {} {{M} \bc \pzero \;|\; x?F \;|\; x!C }
  \and
  \inferrule* [lab=abstraction] {} {{F} \bc (x)P}
  \and
  \inferrule* [lab=concretion] {} {{C} \bc \langle Q \rangle}
  \and
  \inferrule* [lab=process] {} {{P,Q} \bc M \;| \;P|Q \;|\; @{x}}
  \and
  \inferrule* [lab=name] {} {{x} \bc \quotep{P}}
\end{mathpar} 

Note that $\vec{x}$ (resp. $\vec{P}$) denotes a vector of names
(resp. processes) of length $|\vec{x}|$ (resp. $|\vec{P}|$). We adopt
the following useful abbreviations.

\begin{mathpar}
   x?(\vec{y}).P := x.(\vec{y})P \and  x\clift{\vec{P}} := x.\clift{\vec{P}}
   \and x!(y) := \lift{x}{\dropn{y}}
   \and \Pi_{i=0}^{n-1}P_i := P_0 | \ldots | P_{n-1}
\end{mathpar}

\subsubsection{Structural congruence}

\paragraph{Free and bound names and alpha-equivalence.} At the
core of structural equivalence is alpha-equivalence which identifies
process that are the same up to a change of variable. Formally, we
recognize the distinction between free and bound names. The free names
of a process, $\freenames{P}$, may be calculated recursively as
follows:

\begin{mathpar}
\freenames{\pzero} := \emptyset
  \and \\
  \freenames{x?(y).P} := \{ x \} \cup (\freenames{P} \setminus \{ y \})
  \and 
  \freenames{x!\langle P \rangle} := \{ x \} \cup \{ P \} 
  \and \\
  \freenames{P|Q} := \freenames{P} \cup \freenames{Q}
  \and \\
  \freenames{@{x}} := \{ x \}
\end{mathpar}

$\pi$
$\quotep{\pi}$

$\freenames{-} : \pi \to \mathcal{P}(\quotep{\pi})$

\begin{eqnarray*}
  \freenames{\pzero} & := & \emptyset \\
  \freenames{x?(y).P} & := & \{ x \} \cup (\freenames{P} \setminus \{ y \}) \\
  \freenames{x!\langle P \rangle} & := & \{ x \} \cup \{ P \} \\
  \freenames{P|Q} & := & \freenames{P} \cup \freenames{Q} \\
  \freenames{\dropn{x}} & := & \{ x \}
\end{eqnarray*}

The bound names of a process, $\boundnames{P}$, are those names occurring in $P$
that are not free. For example, in $x?(y).0$, the name $x$ is free, while $y$ is bound.

\begin{mathpar}
  \inferrule* [lab=monoidal-laws] {} { P|Q \equiv Q|P \and P|0 \equiv P \and P|(Q|R) \equiv (P|Q)|R }
\end{mathpar}

\begin{mathpar}
  \inferrule* [lab=alpha-equivalence] {} { (x)P \equiv (y)P\{y/x\} \and y \not\in \freenames{P} }
\end{mathpar}

\begin{definition}
Then two processes, $P,Q$, are alpha-equivalent if $P = Q\{\vec{y}/\vec{x}\}$ for
some $\vec{x} \in \boundnames{Q},\vec{y} \in \boundnames{P}$, where $Q\{\vec{y}/\vec{x}\}$
denotes the capture-avoiding substitution of $\vec{y}$ for $\vec{x}$ in $Q$.
\end{definition}

\begin{definition}
  The {\em structural congruence} \cite{SangiorgiWalker} , $\equiv$,
  between processes is the least congruence containing
  alpha-equivalence, satisfying the abelian monoid laws
  (associativity, commutativity and $\pzero$ as identity) for parallel
  composition $|$ and for summation $+$.
\end{definition}

\subsection{Name equivalence}

We take name equivalence, written $\nameeq$, to be the smallest
equivalence relation generated by the following rules.

\begin{mathpar}
\inferrule*[lab=Quote-drop]
{ }
{ \quotep{@{x}} \nameeq x }

\inferrule*[lab=Struct-equiv]
{ P \scong Q }
{ \quotep{P} \nameeq \quotep{Q} }
\end{mathpar}

The astute reader will have noticed that the mutual recursion of names
and processes imposes a mutual recursion on alpha-equivalence and
structural equivalence via name-equivalence. Fortunately, all of this
works out pleasantly and we may calculate in the natural way, free of
concern. The reader interested in the details is referred to the
appendix \ref{appendix:rho_details}.

\subsection{Substitution}

We use $\Proc$ for the set of processes, $\QProc$ for the set of
names, and $\id{\{}\vec{y} / \vec{x} \id{\}}$ to denote partial maps,
$s : \QProc \rightarrow \QProc$. A map, $s$ lifts, uniquely, to a map
on process terms, $\widehat{s} : \Proc \rightarrow \Proc$ by the
following equations.

\begin{mathpar}
  (0) \psubstp{Q}{P} := 0 \\
  (R \juxtap S) \psubstp{Q}{P}
  :=    
  (R)\psubstp{Q}{P} \juxtap (S) \psubstp{Q}{P} \\
  (x?(y).R) \psubstp{Q}{P}    
  :=    
  (x)\substp{Q}{P} (z)\concat( (R \psubstn{z}{y}) \psubstp{Q}{P} ) \\
  (\lift{x}{R}) \psubstp{Q}{P}  
  :=
  \lift{(x)\substp{Q}{P}}{ R \psubstp{Q}{P} } \\
%   (\dropn{x})  \psubstp{Q}{P}       
%   := 
%   \left\{ 
%     \begin{array}{ccc} 
%       \dropn{\quotep{Q}} & & x \nameeq \quotep{P} \\
%       \dropn{x} & & otherwise \\
%     \end{array}
%   \right. 
  (\dropn{x})  \psubstp{Q}{P}       
  := 
  \left\{ 
    \begin{array}{ccc} 
      Q & & x \nameeq \quotep{P} \\
      \dropn{x} & & otherwise \\
    \end{array}
  \right.
\end{mathpar}
 

where

\begin{eqnarray}
  (x)\id{\{} \lpquote Q \rpquote / \lpquote P \rpquote \id{\}}            = 
  \left\{ 
    \begin{array}{ccc}
      \lpquote Q \rpquote & & x \nameeq \lpquote P \rpquote \\
      x & & otherwise \\
    \end{array}
  \right. \nonumber
\end{eqnarray}

and $z$ is chosen distinct from $\quotep{P}$, $\quotep{Q}$, the free
names in $Q$, and all the names in $R$. Our $\alpha$-equivalence will
be built in the standard way from this substitution.

\begin{remark}\label{rem:no_self_referential_names}
  One consequence of these definitions is that $\forall P. \quotep{P}
  \not\in \freenames{P}$.
\end{remark}

\subsection{ Dynamic quote: an example }

Anticipating something of what's to come, consider applying the
substitution, $\widehat{\id{\{}u / z \id{\}}}$, to the following pair
of processes, $\lift{w}{y!(z)}$ and $w[ \lpquote y!(z) \rpquote ]$.

\begin{eqnarray}
	\lift{w}{y!(z)}\widehat{\id{\{}u / z \id{\}}}
		& = &
		\lift{w}{y!(u)} \nonumber\\
	w[ \lpquote y!(z) \rpquote ] \widehat{ \id{\{}u / z \id{\}} }
		& = &
		w[ \lpquote y!(z) \rpquote ] \nonumber
\end{eqnarray}

Because the body of the process between quotes is impervious to
substitution, we get radically different answers. In fact, by
examining the first process in an input context,
e.g. $x?(z).\lift{w}{y!(z)}$, we see that the process under the lift
operator may be shaped by prefixed inputs binding a name inside it. In
this sense, the lift operator will be seen as a way to dynamically
construct processes before reifying them as names.

Finally equipped with these standard features we can present the
dynamics of the calculus.

\subsubsection{Operational semantics} 

Finally, we introduce the computational dynamics. What marks these
algebras as distinct from other more traditionally studied algebraic
structures, e.g. vector spaces or polynomial rings, is the manner in
which dynamics is captured. In traditional structures, dynamics is typically
expressed through morphisms between such structures, as in linear maps
between vector spaces or morphisms between rings. In algebras
associated with the semantics of computation, the dynamics is
expressed as part of the algebraic structure itself, through a
reduction reduction relation typically denoted by $\red$. Below, we
give a recursive presentation of this relation for the calculus used
in the encoding.

$\red \subseteq \pi \times \pi$
$\red : \pi \to \mathcal{P}(\pi)$

\begin{mathpar}
  \inferrule* [lab=Comm] { \textsf{match}( x_{src}, x_{trgt} ) } { x_{trgt}?(y)P \; | \; x_{src}!\langle {Q} \rangle \red P\{\quotep{Q}/y}\} }
  \and \\
  \inferrule* [lab=Par] {{P} \red {P}'} {{{P} | {Q}} \red {{P}' | {Q}}}
  \and
  \inferrule* [lab=Equiv]{{{P} \scong {P}'} \andalso {{P}' \red {Q}'} \andalso {{Q}' \scong {Q}}}{{P} \red {Q}}
\end{mathpar}

\begin{eqnarray*}
  match_{\equiv} (\quotep{P},\quotep{Q}) & := & P \equiv Q \\
  match_{\dagger}(\quotep{P},\quotep{Q}) & := & \forall R. P|Q \red^{*} R => R \red^{*} 0 \\
  match_{K}(\quotep{P},\quotep{Q}) & := & K \mbox{ for some context } K
\end{eqnarray*}

$u?(x)P | u!\langle Q \rangle \red P\{\quotep{Q}/x\}$

%We write $\wred$ for $\red^*$, and $P\red$ if $\exists Q $ such that $ P \red Q$.
We write $P\red$ if $\exists Q $ such that $ P \red Q$ and $P\not\red$, otherwise.

\section{Replication}

As mentioned before, it is known that replication (and hence
recursion) can be implemented in a higher-order process algebra
\cite{SangiorgiWalker}. As our first example of calculation with the
machinery thus far presented we give the construction explicitly in
the {\rhoc}.

\begin{eqnarray}
	D_{x} & := & \prefix{x}{y}{(\binpar{\outputp{x}{y}}{@{y}})} \nonumber\\
	\bangp_{x}{P} & := & \binpar{{x}!\langle{\binpar{D_{x}}{P}}\rangle}{D_{x}} \nonumber
\end{eqnarray}

\begin{eqnarray}
	\bangp_{x}{P} & & \nonumber\\
	=
	& {x}!\langle{(\prefix{x}{y}{(\outputp{x}{y} | @{y})) | P}}\rangle 
	      | \prefix{x}{y}{(\outputp{x}{y} | @{y})} & \nonumber\\
	\red
	& (\outputp{x}{y} | @{y})\substn{\quotep{(\prefix{x}{y}{(@{y} | \outputp{x}{y})) | P}}}{y} & \nonumber\\
	=
	& \outputp{x}{\quotep{(\prefix{x}{y}{(\outputp{x}{y} | @{y})) | P}}}
	  | {(\prefix{x}{y}{(\outputp{x}{y} | @{y})) | P}} & \nonumber\\
	\red
	& \ldots & \nonumber\\
	\red^*
	& P | P | \ldots & \nonumber
\end{eqnarray}

Of course, this encoding, as an implementation, runs away, unfolding
$\bangp{P}$ eagerly. A lazier and more implementable replication
operator, restricted to input-guarded processes, may be obtained as follows.

\begin{eqnarray}
\bangp{\prefix{u}{v}{P}} 
	:= 
	\binpar{\lift{x}{\prefix{u}{v}{(\binpar{D(x)}{P})}}}{D(x)} \nonumber
\end{eqnarray}

\begin{remark}
  Note that the lazier definition still does not deal with summation
  or mixed summation (i.e. sums over input and output). The reader is
  invited to construct definitions of replication that deal with these
  features. 

  Further, the definitions are parameterized in a name, $x$. Can you,
  gentle reader, make a definition that eliminates this parameter and
  guarantees no accidental interaction between the replication
  machinery and the process being replicated -- i.e. no accidental
  sharing of names used by the process to get its work done and the
  name(s) used by the replication to effect copying. This latter
  revision of the definition of replication is crucial to obtaining
  the expected identity $!!P \sim !P$.
\end{remark}

\begin{remark}\label{rem:paradoxical_combinator}
  The reader familiar with the lambda calculus will have noticed the
  similarity between $D$ and the paradoxical combinator.

  [Ed. note: the existence of this seems to suggest we have to be more
  restrictive on the set of processes and names we admit if we are to
  support no-cloning.]
\end{remark}

\subsubsection{Bisimulation}

The computational dynamics gives rise to another kind of equivalence,
the equivalence of computational behavior. As previously mentioned
this is typically captured \emph{via} some form of bisimulation.

% The notion we use in this paper is weak barbed bisimulation
% \cite{milner91polyadicpi}.

The notion we use in this paper is derived from weak barbed
bisimulation \cite{milner91polyadicpi}. 

\begin{definition}
An \emph{observation relation}, $\downarrow_{\mathcal N}$, over a set
of names, $\mathcal N$, is the smallest relation satisfying the rules
below.

\infrule[Out-barb]{y \in {\mathcal N}, \; x \nameeq y}
		  {\outputp{x}{v} \downarrow_{\mathcal N} x}
\infrule[Par-barb]{\mbox{$P\downarrow_{\mathcal N} x$ or $Q\downarrow_{\mathcal N} x$}}
		  {\binpar{P}{Q} \downarrow_{\mathcal N} x}

We write $P \Downarrow_{\mathcal N} x$ if there is $Q$ such that 
$P \wred Q$ and $Q \downarrow_{\mathcal N} x$.
\end{definition}

\begin{definition}
%\label{def.bbisim}
An  ${\mathcal N}$-\emph{barbed bisimulation} over a set of names, ${\mathcal N}$, is a symmetric binary relation 
${\mathcal S}_{\mathcal N}$ between agents such that $P\rel{S}_{\mathcal N}Q$ implies:
\begin{enumerate}
\item If $P \red P'$ then $Q \wred Q'$ and $P'\rel{S}_{\mathcal N} Q'$.
\item If $P\downarrow_{\mathcal N} x$, then $Q\Downarrow_{\mathcal N} x$.
\end{enumerate}
$P$ is ${\mathcal N}$-barbed bisimilar to $Q$, written
$P \wbbisim_{\mathcal N} Q$, if $P \rel{S}_{\mathcal N} Q$ for some ${\mathcal N}$-barbed bisimulation ${\mathcal S}_{\mathcal N}$.
\end{definition}

$\mathcal{R} \subseteq \pi \times \pi$

$P \mathcal{R} Q => \forall P'. P \red P' \Rightarrow \exists Q'. Q \red Q', P' \mathcal{R} Q'$

$P \vdash x \Rightarrow Q \vdash x$

\begin{mathpar}
  \inferrule*[lab=Out-barb]{x \nameeq y}{{y}!\langle{Q}\rangle \vdash x}
  \and
  \inferrule*[lab=Par-barb]{\mbox{$P\vdash x$ or $Q\vdash x$}}{\binpar{P}{Q} \vdash x}
\end{mathpar}

\subsubsection{Contexts}

One of the principle advantages of computational calculi like the
$\pi$-calculus is a well-defined notion of context,
contextual-equivalence and a correlation between
contextual-equivalence and notions of bisimulation. The notion of
context allows the decomposition of a process into (sub-)process and
its syntactic environment, its context. Thus, a context may be
thought of as a process with a ``hole'' (written $\Box$) in it. The
application of a context $M$ to a process $P$, written $M[P]$, is
tantamount to filling the hole in $M$ with $P$. In this paper we do
not need the full weight of this theory, but do make use of the notion
of context in the proof the main theorem. 

\begin{mathpar}
  \inferrule* [lab=summation] {} {{M_{M},M_{N}} \bc \Box \;|\; x.M_{A} \;|\; M_{M}+M_{N}}
  \and
  \inferrule* [lab=agent] {} {{M_{A}} \bc (\vec{x})M_{P} \;| \; \clift{P_0,\ldots,M_{P},\ldots,P_N}}
  \and \\
  \inferrule* [lab=process] {} {{M_{P}} \bc M_{N} \;| \;P|M_{P} }
\end{mathpar} 

\begin{mathpar}
  \inferrule* [lab=sychronization] {} {M_{N} \bc \Box \;|\; x?M_{F} \;|\; x!M_{C}}
  \and
  \inferrule* [lab=abstraction] {} {{M_{F}} \bc (x)M_{P} }
  \and
  \inferrule* [lab=concretion] {} {{M_{C}} \bc \langle M_{P} \rangle }
  \and \\
  \inferrule* [lab=process] {} {{M_{P}} \bc M_{N} \;| \;P|M_{P} }
\end{mathpar}

\begin{definition}[contextual application] Given a context $M$, and
  process $P$, we define the \emph{contextual application}, $M[P] :=
  M\{P/\Box\}$. That is, the contextual application of M to P is the
  substitution of $P$ for $\Box$ in $M$.
\end{definition}

$\meaningof{-} : L \to \mathcal{P}(\pi)$

\begin{mathpar}
  \inferrule* [lab=collection] {} {\meaningof{true} = \pi, \and \meaningof{~E} = \pi \setminus \meaningof{E}, \and \meaningof{E_{1} \& E_{2}} = \meaningof{E_{1}} \cap \meaningof{E_{2}}}
\end{mathpar}

\begin{mathpar}
  \inferrule* [lab=structure] {} {\meaningof{0} = \{ P \in \pi | P \equiv 0 \}, \and \\ \meaningof{E_1 | E_2} = \{ P \in \pi | P \equiv P_{1} | P_{2}, P_{1} \in \meaningof{E_{1}}, P_{2} \in \meaningof{E_2}\} }
\end{mathpar}

\begin{mathpar}
 \inferrule* [lab=behavior] {} {\meaningof{\langle a?b \rangle E} = \{ P \in \pi | P \equiv Q | u?(y)P', \\ \and \\\\ \and \\ \;\;\; u \in \meaningof{a}, \forall z.P'\{z/y\} \in \meaningof{E\{z/b\}}\}, \and \\ \meaningof{a!E} = \{ P \in \pi | P \equiv Q | x!\langle P' \rangle, x \in \meaningof{a} P' \in \meaningof{E}\} }
\end{mathpar}

\begin{mathpar}
 \inferrule* [lab=nominal] {} {\meaningof{\quotep{E}} = \{ \quotep{P} \in \quotep{\pi} | P \in \meaningof{E} \}, \and \meaningof{\quotep{P}} = \{ \quotep{Q} \in \quotep{\pi} | P \equiv Q \} \and \\ \meaningof{@\quotep{E}} = \{ P \in \pi | P \equiv @x, x \in \meaningof{E} \}}
\end{mathpar}

\begin{eqnarray*}
  \\
  \meaningof{-} : TS \to ST
\end{eqnarray*}

\begin{eqnarray*}
  \\
  L : TS \to ST
\end{eqnarray*}

\begin{eqnarray*}
  \\
  P \models E \iff P \in \meaningof{E}
\end{eqnarray*}

\begin{eqnarray*}
  P \approx_{L} Q \iff \forall E \in L. P \models E \iff Q \models E
\end{eqnarray*}

\begin{eqnarray*}
  P \approx_{K} Q
\end{eqnarray*}

\begin{eqnarray*}
  P \approx Q
\end{eqnarray*}

$\approx_{K} = \approx = \approx_{L}$

\subsubsection{Contextual duality}

Note that contexts extend the quotation operation to a family of
operations from processes to names. Given a context, $M$, we can
define a \emph{nominal context}, $\quotep{M}$ by $\quotep{M}[P] :=
\quotep{M[P]}$. To foreshadow what is to come we observe that these
operations enjoy a duality with processes very much like the duality
between vectors and maps from vectors to scalars.

Further, because the calculus is essentially higher-order, we have a
correspondence between contexts and processes. More specifically,
given a name $x$ and a context $M$ we can construct $M^{*}_{x}$ such
that 

\begin{mathpar}
  M^{*}_{x} | \lift{x}{P} \red M[P]
\end{mathpar}

namely,

\begin{mathpar}
  M^{*}_{x} := x?(u).M[\dropn{u}]
\end{mathpar}

The dependence of $M^{*}_{x}$ on a name makes it an abstraction, 

\begin{mathpar}
  M^{*} := (x)x?(u).M[\dropn{u}]
\end{mathpar}

\subsection{Additional notation}

It will sometimes be convenient to denote the process a name
quotes. We already have the notation $x = \quotep{P}$, but it will be
convenient to introduce an alternate notation, $\procn{x}$, when we
want to emphasize the connection to the use of the name. Note that, by
virtue of name equivalence, $\quotep{\procn{x}} \nameeq x$; so, the
notation is consistent with previous definitions.

Further, because names have structure it is possible to effect
substitutions on the basis of that structure. This means we need to
upgrade our notation for substitutions, which we accomplish by
adapting comprehension notation. Thus,

\begin{mathpar}
  P\{ y / x : x \in S \}
\end{mathpar}

is interpreted to mean the process derived from P by replacing (in a
capture-avoiding manner) each occurrence of $x$ in $S$ by $y$. For example,

\begin{mathpar}
  P\{ \quotep{\procn{x}|\procn{x}} / x : x \in \freenames{P} \}
\end{mathpar}

will replace each (occurrence) of a free name $x$ in $P$ by
$\quotep{\procn{x}|\procn{x}}$.

Also, we will avail ourselves of the notation $x^{L}$ and $x^{R}$ to
denote injections of a name into disjoint copies of the name
space. There are numerous ways to accomplish this. One example can be
found in \cite{MeredithR05}. This notation overloads to vectors of
names: $\vec{x}^{\pi} := (x_{i}^{\pi} \; : \; 0 \leq i < |\vec{x}| )$ where $\pi \in \{L,R\}$.

We also use $P^{\Box} := P|\Box$.

In \cite{MeredithR05} an interpretation of the new operator is
given. It turns out that there are several possible interpretations
all enjoying the requisite algebraic properties of the operator (see
\cite{milner91polyadicpi}). We will therefore make liberal use of
$(\nu\; \vec{x})P$.

% subsection the_syntax_and_semantics_of_the_notation_system (end)   

\input{qm2pi.qmops} 

\input{qm2pi.sterngerlach} 

\input{qm2pi.metric} 

% section concurrent_process_calculi (end)

%\input{qm2pi.proofsketch}

% section proof sketch (end)

%\input{qm2pi.slviaknots} 

% section spatial logic via knots (end)

\input{qm2pi.conclusion}

% section conclusion (end)

%\input{qm2pi.dtcodes} 

% section wiring algorithm (end)

\input{qm2pi.ack} 

% section acknowledgments (end)

\newpage


\bibliographystyle{plain}   
\bibliography{../../biblios/main.bib}

\input{qm2pi.rhodetails}

\end{document}

 

%\documentclass[12pt]{llncs}
%\documentclass{jktr}

\usepackage[pdftex]{hyperref}                   
\usepackage {listings}
\usepackage {mathpartir}
\usepackage{bcprules}
%\usepackage{listings}
                       
\usepackage{graphicx} 
%\usepackage[margins=2.5cm,nohead,nofoot]{geometry}
%\usepackage{geometry}
\usepackage{amsfonts}
\usepackage{amstext}
\usepackage{latexsym}
\usepackage{amssymb}
\usepackage{color}


%\include{myPreamble}
\include{qm2pi.local} 

%\ifpdf
%\usepackage[pdftex]{graphicx}
%\else
%\usepackage{graphicx}
%\fi

 % \ifpdf
%  \usepackage{pdfsync}
%  \if


%\title{Brief Article}
%\author{David F. Snyder}
%\author{L.G. Meredith}

%\address{Dept. of Math., Texas State University--San Marcos, San Marcos, TX 78666}
       
\pagestyle{empty}


\begin{document}

\lstset{language=[Objective]Caml,frame=shadowbox}

\input{qm2pi.front}

% section front matter (end)

\input{qm2pi.intro} 
 
% section introduction (end)

% \input{qm2pi.knotations} 

% section notation (end)

\input{qm2pi.process.calculi} 

% section concurrent_process_calculi_and_spatial_logics_ (end)
    
%\input{qm2pi.knots2pi} 

%\input{qm2pi.trefoil} 

%\input{qm2pi.mainthm} 

% subsection basic_interpretation (end)

%\input{qm2pi.rho.presentation} 
\subsection{The syntax and semantics of the notation system}\label{sub:the_syntax_and_semantics_of_the_notation_system} % (fold)

We now summarize a technical presentation of the calculus that
embodies our theory of dynamics. The typical presentation of such a
calculus follows the style of giving generators and relations on
them. The grammar, below, describing term constructors, freely
generates the set of processes, $\Proc$. This set is then quotiented
by a relation known as structural congruence and it is over this set
that the notion of dynamics is expressed. This presentation is
essentially that of \cite{MeredithR05} with the addition of
polyadicity and summation. For readability we have relegated some of
the technical subtleties to an appendix.

\subsubsection{Process grammar}\label{subsub:process_grammar}

\begin{mathpar}
  \inferrule* [lab=synchronization] {} {{M} \bc \pzero \;|\; x?F \;|\; x!C }
  \and
  \inferrule* [lab=abstraction] {} {{F} \bc (x)P}
  \and
  \inferrule* [lab=concretion] {} {{C} \bc \langle Q \rangle}
  \and
  \inferrule* [lab=process] {} {{P,Q} \bc M \;| \;P|Q \;|\; @{x}}
  \and
  \inferrule* [lab=name] {} {{x} \bc \quotep{P}}
\end{mathpar} 

Note that $\vec{x}$ (resp. $\vec{P}$) denotes a vector of names
(resp. processes) of length $|\vec{x}|$ (resp. $|\vec{P}|$). We adopt
the following useful abbreviations.

\begin{mathpar}
   x?(\vec{y}).P := x.(\vec{y})P \and  x\clift{\vec{P}} := x.\clift{\vec{P}}
   \and x!(y) := \lift{x}{\dropn{y}}
   \and \Pi_{i=0}^{n-1}P_i := P_0 | \ldots | P_{n-1}
\end{mathpar}

\subsubsection{Structural congruence}

\paragraph{Free and bound names and alpha-equivalence.} At the
core of structural equivalence is alpha-equivalence which identifies
process that are the same up to a change of variable. Formally, we
recognize the distinction between free and bound names. The free names
of a process, $\freenames{P}$, may be calculated recursively as
follows:

\begin{mathpar}
\freenames{\pzero} := \emptyset
  \and \\
  \freenames{x?(y).P} := \{ x \} \cup (\freenames{P} \setminus \{ y \})
  \and 
  \freenames{x!\langle P \rangle} := \{ x \} \cup \{ P \} 
  \and \\
  \freenames{P|Q} := \freenames{P} \cup \freenames{Q}
  \and \\
  \freenames{@{x}} := \{ x \}
\end{mathpar}

$\pi$
$\quotep{\pi}$

$\freenames{-} : \pi \to \mathcal{P}(\quotep{\pi})$

\begin{eqnarray*}
  \freenames{\pzero} & := & \emptyset \\
  \freenames{x?(y).P} & := & \{ x \} \cup (\freenames{P} \setminus \{ y \}) \\
  \freenames{x!\langle P \rangle} & := & \{ x \} \cup \{ P \} \\
  \freenames{P|Q} & := & \freenames{P} \cup \freenames{Q} \\
  \freenames{\dropn{x}} & := & \{ x \}
\end{eqnarray*}

The bound names of a process, $\boundnames{P}$, are those names occurring in $P$
that are not free. For example, in $x?(y).0$, the name $x$ is free, while $y$ is bound.

\begin{mathpar}
  \inferrule* [lab=monoidal-laws] {} { P|Q \equiv Q|P \and P|0 \equiv P \and P|(Q|R) \equiv (P|Q)|R }
\end{mathpar}

\begin{mathpar}
  \inferrule* [lab=alpha-equivalence] {} { (x)P \equiv (y)P\{y/x\} \and y \not\in \freenames{P} }
\end{mathpar}

\begin{definition}
Then two processes, $P,Q$, are alpha-equivalent if $P = Q\{\vec{y}/\vec{x}\}$ for
some $\vec{x} \in \boundnames{Q},\vec{y} \in \boundnames{P}$, where $Q\{\vec{y}/\vec{x}\}$
denotes the capture-avoiding substitution of $\vec{y}$ for $\vec{x}$ in $Q$.
\end{definition}

\begin{definition}
  The {\em structural congruence} \cite{SangiorgiWalker} , $\equiv$,
  between processes is the least congruence containing
  alpha-equivalence, satisfying the abelian monoid laws
  (associativity, commutativity and $\pzero$ as identity) for parallel
  composition $|$ and for summation $+$.
\end{definition}

\subsection{Name equivalence}

We take name equivalence, written $\nameeq$, to be the smallest
equivalence relation generated by the following rules.

\begin{mathpar}
\inferrule*[lab=Quote-drop]
{ }
{ \quotep{@{x}} \nameeq x }

\inferrule*[lab=Struct-equiv]
{ P \scong Q }
{ \quotep{P} \nameeq \quotep{Q} }
\end{mathpar}

The astute reader will have noticed that the mutual recursion of names
and processes imposes a mutual recursion on alpha-equivalence and
structural equivalence via name-equivalence. Fortunately, all of this
works out pleasantly and we may calculate in the natural way, free of
concern. The reader interested in the details is referred to the
appendix \ref{appendix:rho_details}.

\subsection{Substitution}

We use $\Proc$ for the set of processes, $\QProc$ for the set of
names, and $\id{\{}\vec{y} / \vec{x} \id{\}}$ to denote partial maps,
$s : \QProc \rightarrow \QProc$. A map, $s$ lifts, uniquely, to a map
on process terms, $\widehat{s} : \Proc \rightarrow \Proc$ by the
following equations.

\begin{mathpar}
  (0) \psubstp{Q}{P} := 0 \\
  (R \juxtap S) \psubstp{Q}{P}
  :=    
  (R)\psubstp{Q}{P} \juxtap (S) \psubstp{Q}{P} \\
  (x?(y).R) \psubstp{Q}{P}    
  :=    
  (x)\substp{Q}{P} (z)\concat( (R \psubstn{z}{y}) \psubstp{Q}{P} ) \\
  (\lift{x}{R}) \psubstp{Q}{P}  
  :=
  \lift{(x)\substp{Q}{P}}{ R \psubstp{Q}{P} } \\
%   (\dropn{x})  \psubstp{Q}{P}       
%   := 
%   \left\{ 
%     \begin{array}{ccc} 
%       \dropn{\quotep{Q}} & & x \nameeq \quotep{P} \\
%       \dropn{x} & & otherwise \\
%     \end{array}
%   \right. 
  (\dropn{x})  \psubstp{Q}{P}       
  := 
  \left\{ 
    \begin{array}{ccc} 
      Q & & x \nameeq \quotep{P} \\
      \dropn{x} & & otherwise \\
    \end{array}
  \right.
\end{mathpar}
 

where

\begin{eqnarray}
  (x)\id{\{} \lpquote Q \rpquote / \lpquote P \rpquote \id{\}}            = 
  \left\{ 
    \begin{array}{ccc}
      \lpquote Q \rpquote & & x \nameeq \lpquote P \rpquote \\
      x & & otherwise \\
    \end{array}
  \right. \nonumber
\end{eqnarray}

and $z$ is chosen distinct from $\quotep{P}$, $\quotep{Q}$, the free
names in $Q$, and all the names in $R$. Our $\alpha$-equivalence will
be built in the standard way from this substitution.

\begin{remark}\label{rem:no_self_referential_names}
  One consequence of these definitions is that $\forall P. \quotep{P}
  \not\in \freenames{P}$.
\end{remark}

\subsection{ Dynamic quote: an example }

Anticipating something of what's to come, consider applying the
substitution, $\widehat{\id{\{}u / z \id{\}}}$, to the following pair
of processes, $\lift{w}{y!(z)}$ and $w[ \lpquote y!(z) \rpquote ]$.

\begin{eqnarray}
	\lift{w}{y!(z)}\widehat{\id{\{}u / z \id{\}}}
		& = &
		\lift{w}{y!(u)} \nonumber\\
	w[ \lpquote y!(z) \rpquote ] \widehat{ \id{\{}u / z \id{\}} }
		& = &
		w[ \lpquote y!(z) \rpquote ] \nonumber
\end{eqnarray}

Because the body of the process between quotes is impervious to
substitution, we get radically different answers. In fact, by
examining the first process in an input context,
e.g. $x?(z).\lift{w}{y!(z)}$, we see that the process under the lift
operator may be shaped by prefixed inputs binding a name inside it. In
this sense, the lift operator will be seen as a way to dynamically
construct processes before reifying them as names.

Finally equipped with these standard features we can present the
dynamics of the calculus.

\subsubsection{Operational semantics} 

Finally, we introduce the computational dynamics. What marks these
algebras as distinct from other more traditionally studied algebraic
structures, e.g. vector spaces or polynomial rings, is the manner in
which dynamics is captured. In traditional structures, dynamics is typically
expressed through morphisms between such structures, as in linear maps
between vector spaces or morphisms between rings. In algebras
associated with the semantics of computation, the dynamics is
expressed as part of the algebraic structure itself, through a
reduction reduction relation typically denoted by $\red$. Below, we
give a recursive presentation of this relation for the calculus used
in the encoding.

$\red \subseteq \pi \times \pi$
$\red : \pi \to \mathcal{P}(\pi)$

\begin{mathpar}
  \inferrule* [lab=Comm] { \textsf{match}( x_{src}, x_{trgt} ) } { x_{trgt}?(y)P \; | \; x_{src}!\langle {Q} \rangle \red P\{\quotep{Q}/y}\} }
  \and \\
  \inferrule* [lab=Par] {{P} \red {P}'} {{{P} | {Q}} \red {{P}' | {Q}}}
  \and
  \inferrule* [lab=Equiv]{{{P} \scong {P}'} \andalso {{P}' \red {Q}'} \andalso {{Q}' \scong {Q}}}{{P} \red {Q}}
\end{mathpar}

\begin{eqnarray*}
  match_{\equiv} (\quotep{P},\quotep{Q}) & := & P \equiv Q \\
  match_{\dagger}(\quotep{P},\quotep{Q}) & := & \forall R. P|Q \red^{*} R => R \red^{*} 0 \\
  match_{K}(\quotep{P},\quotep{Q}) & := & K \mbox{ for some context } K
\end{eqnarray*}

$u?(x)P | u!\langle Q \rangle \red P\{\quotep{Q}/x\}$

%We write $\wred$ for $\red^*$, and $P\red$ if $\exists Q $ such that $ P \red Q$.
We write $P\red$ if $\exists Q $ such that $ P \red Q$ and $P\not\red$, otherwise.

\section{Replication}

As mentioned before, it is known that replication (and hence
recursion) can be implemented in a higher-order process algebra
\cite{SangiorgiWalker}. As our first example of calculation with the
machinery thus far presented we give the construction explicitly in
the {\rhoc}.

\begin{eqnarray}
	D_{x} & := & \prefix{x}{y}{(\binpar{\outputp{x}{y}}{@{y}})} \nonumber\\
	\bangp_{x}{P} & := & \binpar{{x}!\langle{\binpar{D_{x}}{P}}\rangle}{D_{x}} \nonumber
\end{eqnarray}

\begin{eqnarray}
	\bangp_{x}{P} & & \nonumber\\
	=
	& {x}!\langle{(\prefix{x}{y}{(\outputp{x}{y} | @{y})) | P}}\rangle 
	      | \prefix{x}{y}{(\outputp{x}{y} | @{y})} & \nonumber\\
	\red
	& (\outputp{x}{y} | @{y})\substn{\quotep{(\prefix{x}{y}{(@{y} | \outputp{x}{y})) | P}}}{y} & \nonumber\\
	=
	& \outputp{x}{\quotep{(\prefix{x}{y}{(\outputp{x}{y} | @{y})) | P}}}
	  | {(\prefix{x}{y}{(\outputp{x}{y} | @{y})) | P}} & \nonumber\\
	\red
	& \ldots & \nonumber\\
	\red^*
	& P | P | \ldots & \nonumber
\end{eqnarray}

Of course, this encoding, as an implementation, runs away, unfolding
$\bangp{P}$ eagerly. A lazier and more implementable replication
operator, restricted to input-guarded processes, may be obtained as follows.

\begin{eqnarray}
\bangp{\prefix{u}{v}{P}} 
	:= 
	\binpar{\lift{x}{\prefix{u}{v}{(\binpar{D(x)}{P})}}}{D(x)} \nonumber
\end{eqnarray}

\begin{remark}
  Note that the lazier definition still does not deal with summation
  or mixed summation (i.e. sums over input and output). The reader is
  invited to construct definitions of replication that deal with these
  features. 

  Further, the definitions are parameterized in a name, $x$. Can you,
  gentle reader, make a definition that eliminates this parameter and
  guarantees no accidental interaction between the replication
  machinery and the process being replicated -- i.e. no accidental
  sharing of names used by the process to get its work done and the
  name(s) used by the replication to effect copying. This latter
  revision of the definition of replication is crucial to obtaining
  the expected identity $!!P \sim !P$.
\end{remark}

\begin{remark}\label{rem:paradoxical_combinator}
  The reader familiar with the lambda calculus will have noticed the
  similarity between $D$ and the paradoxical combinator.

  [Ed. note: the existence of this seems to suggest we have to be more
  restrictive on the set of processes and names we admit if we are to
  support no-cloning.]
\end{remark}

\subsubsection{Bisimulation}

The computational dynamics gives rise to another kind of equivalence,
the equivalence of computational behavior. As previously mentioned
this is typically captured \emph{via} some form of bisimulation.

% The notion we use in this paper is weak barbed bisimulation
% \cite{milner91polyadicpi}.

The notion we use in this paper is derived from weak barbed
bisimulation \cite{milner91polyadicpi}. 

\begin{definition}
An \emph{observation relation}, $\downarrow_{\mathcal N}$, over a set
of names, $\mathcal N$, is the smallest relation satisfying the rules
below.

\infrule[Out-barb]{y \in {\mathcal N}, \; x \nameeq y}
		  {\outputp{x}{v} \downarrow_{\mathcal N} x}
\infrule[Par-barb]{\mbox{$P\downarrow_{\mathcal N} x$ or $Q\downarrow_{\mathcal N} x$}}
		  {\binpar{P}{Q} \downarrow_{\mathcal N} x}

We write $P \Downarrow_{\mathcal N} x$ if there is $Q$ such that 
$P \wred Q$ and $Q \downarrow_{\mathcal N} x$.
\end{definition}

\begin{definition}
%\label{def.bbisim}
An  ${\mathcal N}$-\emph{barbed bisimulation} over a set of names, ${\mathcal N}$, is a symmetric binary relation 
${\mathcal S}_{\mathcal N}$ between agents such that $P\rel{S}_{\mathcal N}Q$ implies:
\begin{enumerate}
\item If $P \red P'$ then $Q \wred Q'$ and $P'\rel{S}_{\mathcal N} Q'$.
\item If $P\downarrow_{\mathcal N} x$, then $Q\Downarrow_{\mathcal N} x$.
\end{enumerate}
$P$ is ${\mathcal N}$-barbed bisimilar to $Q$, written
$P \wbbisim_{\mathcal N} Q$, if $P \rel{S}_{\mathcal N} Q$ for some ${\mathcal N}$-barbed bisimulation ${\mathcal S}_{\mathcal N}$.
\end{definition}

$\mathcal{R} \subseteq \pi \times \pi$

$P \mathcal{R} Q => \forall P'. P \red P' \Rightarrow \exists Q'. Q \red Q', P' \mathcal{R} Q'$

$P \vdash x \Rightarrow Q \vdash x$

\begin{mathpar}
  \inferrule*[lab=Out-barb]{x \nameeq y}{{y}!\langle{Q}\rangle \vdash x}
  \and
  \inferrule*[lab=Par-barb]{\mbox{$P\vdash x$ or $Q\vdash x$}}{\binpar{P}{Q} \vdash x}
\end{mathpar}

\subsubsection{Contexts}

One of the principle advantages of computational calculi like the
$\pi$-calculus is a well-defined notion of context,
contextual-equivalence and a correlation between
contextual-equivalence and notions of bisimulation. The notion of
context allows the decomposition of a process into (sub-)process and
its syntactic environment, its context. Thus, a context may be
thought of as a process with a ``hole'' (written $\Box$) in it. The
application of a context $M$ to a process $P$, written $M[P]$, is
tantamount to filling the hole in $M$ with $P$. In this paper we do
not need the full weight of this theory, but do make use of the notion
of context in the proof the main theorem. 

\begin{mathpar}
  \inferrule* [lab=summation] {} {{M_{M},M_{N}} \bc \Box \;|\; x.M_{A} \;|\; M_{M}+M_{N}}
  \and
  \inferrule* [lab=agent] {} {{M_{A}} \bc (\vec{x})M_{P} \;| \; \clift{P_0,\ldots,M_{P},\ldots,P_N}}
  \and \\
  \inferrule* [lab=process] {} {{M_{P}} \bc M_{N} \;| \;P|M_{P} }
\end{mathpar} 

\begin{mathpar}
  \inferrule* [lab=sychronization] {} {M_{N} \bc \Box \;|\; x?M_{F} \;|\; x!M_{C}}
  \and
  \inferrule* [lab=abstraction] {} {{M_{F}} \bc (x)M_{P} }
  \and
  \inferrule* [lab=concretion] {} {{M_{C}} \bc \langle M_{P} \rangle }
  \and \\
  \inferrule* [lab=process] {} {{M_{P}} \bc M_{N} \;| \;P|M_{P} }
\end{mathpar}

\begin{definition}[contextual application] Given a context $M$, and
  process $P$, we define the \emph{contextual application}, $M[P] :=
  M\{P/\Box\}$. That is, the contextual application of M to P is the
  substitution of $P$ for $\Box$ in $M$.
\end{definition}

$\meaningof{-} : L \to \mathcal{P}(\pi)$

\begin{mathpar}
  \inferrule* [lab=collection] {} {\meaningof{true} = \pi, \and \meaningof{~E} = \pi \setminus \meaningof{E}, \and \meaningof{E_{1} \& E_{2}} = \meaningof{E_{1}} \cap \meaningof{E_{2}}}
\end{mathpar}

\begin{mathpar}
  \inferrule* [lab=structure] {} {\meaningof{0} = \{ P \in \pi | P \equiv 0 \}, \and \\ \meaningof{E_1 | E_2} = \{ P \in \pi | P \equiv P_{1} | P_{2}, P_{1} \in \meaningof{E_{1}}, P_{2} \in \meaningof{E_2}\} }
\end{mathpar}

\begin{mathpar}
 \inferrule* [lab=behavior] {} {\meaningof{\langle a?b \rangle E} = \{ P \in \pi | P \equiv Q | u?(y)P', \\ \and \\\\ \and \\ \;\;\; u \in \meaningof{a}, \forall z.P'\{z/y\} \in \meaningof{E\{z/b\}}\}, \and \\ \meaningof{a!E} = \{ P \in \pi | P \equiv Q | x!\langle P' \rangle, x \in \meaningof{a} P' \in \meaningof{E}\} }
\end{mathpar}

\begin{mathpar}
 \inferrule* [lab=nominal] {} {\meaningof{\quotep{E}} = \{ \quotep{P} \in \quotep{\pi} | P \in \meaningof{E} \}, \and \meaningof{\quotep{P}} = \{ \quotep{Q} \in \quotep{\pi} | P \equiv Q \} \and \\ \meaningof{@\quotep{E}} = \{ P \in \pi | P \equiv @x, x \in \meaningof{E} \}}
\end{mathpar}

\begin{eqnarray*}
  \\
  \meaningof{-} : TS \to ST
\end{eqnarray*}

\begin{eqnarray*}
  \\
  L : TS \to ST
\end{eqnarray*}

\begin{eqnarray*}
  \\
  P \models E \iff P \in \meaningof{E}
\end{eqnarray*}

\begin{eqnarray*}
  P \approx_{L} Q \iff \forall E \in L. P \models E \iff Q \models E
\end{eqnarray*}

\begin{eqnarray*}
  P \approx_{K} Q
\end{eqnarray*}

\begin{eqnarray*}
  P \approx Q
\end{eqnarray*}

$\approx_{K} = \approx = \approx_{L}$

\subsubsection{Contextual duality}

Note that contexts extend the quotation operation to a family of
operations from processes to names. Given a context, $M$, we can
define a \emph{nominal context}, $\quotep{M}$ by $\quotep{M}[P] :=
\quotep{M[P]}$. To foreshadow what is to come we observe that these
operations enjoy a duality with processes very much like the duality
between vectors and maps from vectors to scalars.

Further, because the calculus is essentially higher-order, we have a
correspondence between contexts and processes. More specifically,
given a name $x$ and a context $M$ we can construct $M^{*}_{x}$ such
that 

\begin{mathpar}
  M^{*}_{x} | \lift{x}{P} \red M[P]
\end{mathpar}

namely,

\begin{mathpar}
  M^{*}_{x} := x?(u).M[\dropn{u}]
\end{mathpar}

The dependence of $M^{*}_{x}$ on a name makes it an abstraction, 

\begin{mathpar}
  M^{*} := (x)x?(u).M[\dropn{u}]
\end{mathpar}

\subsection{Additional notation}

It will sometimes be convenient to denote the process a name
quotes. We already have the notation $x = \quotep{P}$, but it will be
convenient to introduce an alternate notation, $\procn{x}$, when we
want to emphasize the connection to the use of the name. Note that, by
virtue of name equivalence, $\quotep{\procn{x}} \nameeq x$; so, the
notation is consistent with previous definitions.

Further, because names have structure it is possible to effect
substitutions on the basis of that structure. This means we need to
upgrade our notation for substitutions, which we accomplish by
adapting comprehension notation. Thus,

\begin{mathpar}
  P\{ y / x : x \in S \}
\end{mathpar}

is interpreted to mean the process derived from P by replacing (in a
capture-avoiding manner) each occurrence of $x$ in $S$ by $y$. For example,

\begin{mathpar}
  P\{ \quotep{\procn{x}|\procn{x}} / x : x \in \freenames{P} \}
\end{mathpar}

will replace each (occurrence) of a free name $x$ in $P$ by
$\quotep{\procn{x}|\procn{x}}$.

Also, we will avail ourselves of the notation $x^{L}$ and $x^{R}$ to
denote injections of a name into disjoint copies of the name
space. There are numerous ways to accomplish this. One example can be
found in \cite{MeredithR05}. This notation overloads to vectors of
names: $\vec{x}^{\pi} := (x_{i}^{\pi} \; : \; 0 \leq i < |\vec{x}| )$ where $\pi \in \{L,R\}$.

We also use $P^{\Box} := P|\Box$.

In \cite{MeredithR05} an interpretation of the new operator is
given. It turns out that there are several possible interpretations
all enjoying the requisite algebraic properties of the operator (see
\cite{milner91polyadicpi}). We will therefore make liberal use of
$(\nu\; \vec{x})P$.

% subsection the_syntax_and_semantics_of_the_notation_system (end)   

\input{qm2pi.qmops} 

\input{qm2pi.sterngerlach} 

\input{qm2pi.metric} 

% section concurrent_process_calculi (end)

%\input{qm2pi.proofsketch}

% section proof sketch (end)

%\input{qm2pi.slviaknots} 

% section spatial logic via knots (end)

\input{qm2pi.conclusion}

% section conclusion (end)

%\input{qm2pi.dtcodes} 

% section wiring algorithm (end)

\input{qm2pi.ack} 

% section acknowledgments (end)

\newpage


\bibliographystyle{plain}   
\bibliography{../../biblios/main.bib}

\input{qm2pi.rhodetails}

\end{document}

 

%\documentclass[12pt]{llncs}
%\documentclass{jktr}

\usepackage[pdftex]{hyperref}                   
\usepackage {listings}
\usepackage {mathpartir}
\usepackage{bcprules}
%\usepackage{listings}
                       
\usepackage{graphicx} 
%\usepackage[margins=2.5cm,nohead,nofoot]{geometry}
%\usepackage{geometry}
\usepackage{amsfonts}
\usepackage{amstext}
\usepackage{latexsym}
\usepackage{amssymb}
\usepackage{color}


%\include{myPreamble}
\include{qm2pi.local} 

%\ifpdf
%\usepackage[pdftex]{graphicx}
%\else
%\usepackage{graphicx}
%\fi

 % \ifpdf
%  \usepackage{pdfsync}
%  \if


%\title{Brief Article}
%\author{David F. Snyder}
%\author{L.G. Meredith}

%\address{Dept. of Math., Texas State University--San Marcos, San Marcos, TX 78666}
       
\pagestyle{empty}


\begin{document}

\lstset{language=[Objective]Caml,frame=shadowbox}

\input{qm2pi.front}

% section front matter (end)

\input{qm2pi.intro} 
 
% section introduction (end)

% \input{qm2pi.knotations} 

% section notation (end)

\input{qm2pi.process.calculi} 

% section concurrent_process_calculi_and_spatial_logics_ (end)
    
%\input{qm2pi.knots2pi} 

%\input{qm2pi.trefoil} 

%\input{qm2pi.mainthm} 

% subsection basic_interpretation (end)

%\input{qm2pi.rho.presentation} 
\subsection{The syntax and semantics of the notation system}\label{sub:the_syntax_and_semantics_of_the_notation_system} % (fold)

We now summarize a technical presentation of the calculus that
embodies our theory of dynamics. The typical presentation of such a
calculus follows the style of giving generators and relations on
them. The grammar, below, describing term constructors, freely
generates the set of processes, $\Proc$. This set is then quotiented
by a relation known as structural congruence and it is over this set
that the notion of dynamics is expressed. This presentation is
essentially that of \cite{MeredithR05} with the addition of
polyadicity and summation. For readability we have relegated some of
the technical subtleties to an appendix.

\subsubsection{Process grammar}\label{subsub:process_grammar}

\begin{mathpar}
  \inferrule* [lab=synchronization] {} {{M} \bc \pzero \;|\; x?F \;|\; x!C }
  \and
  \inferrule* [lab=abstraction] {} {{F} \bc (x)P}
  \and
  \inferrule* [lab=concretion] {} {{C} \bc \langle Q \rangle}
  \and
  \inferrule* [lab=process] {} {{P,Q} \bc M \;| \;P|Q \;|\; @{x}}
  \and
  \inferrule* [lab=name] {} {{x} \bc \quotep{P}}
\end{mathpar} 

Note that $\vec{x}$ (resp. $\vec{P}$) denotes a vector of names
(resp. processes) of length $|\vec{x}|$ (resp. $|\vec{P}|$). We adopt
the following useful abbreviations.

\begin{mathpar}
   x?(\vec{y}).P := x.(\vec{y})P \and  x\clift{\vec{P}} := x.\clift{\vec{P}}
   \and x!(y) := \lift{x}{\dropn{y}}
   \and \Pi_{i=0}^{n-1}P_i := P_0 | \ldots | P_{n-1}
\end{mathpar}

\subsubsection{Structural congruence}

\paragraph{Free and bound names and alpha-equivalence.} At the
core of structural equivalence is alpha-equivalence which identifies
process that are the same up to a change of variable. Formally, we
recognize the distinction between free and bound names. The free names
of a process, $\freenames{P}$, may be calculated recursively as
follows:

\begin{mathpar}
\freenames{\pzero} := \emptyset
  \and \\
  \freenames{x?(y).P} := \{ x \} \cup (\freenames{P} \setminus \{ y \})
  \and 
  \freenames{x!\langle P \rangle} := \{ x \} \cup \{ P \} 
  \and \\
  \freenames{P|Q} := \freenames{P} \cup \freenames{Q}
  \and \\
  \freenames{@{x}} := \{ x \}
\end{mathpar}

$\pi$
$\quotep{\pi}$

$\freenames{-} : \pi \to \mathcal{P}(\quotep{\pi})$

\begin{eqnarray*}
  \freenames{\pzero} & := & \emptyset \\
  \freenames{x?(y).P} & := & \{ x \} \cup (\freenames{P} \setminus \{ y \}) \\
  \freenames{x!\langle P \rangle} & := & \{ x \} \cup \{ P \} \\
  \freenames{P|Q} & := & \freenames{P} \cup \freenames{Q} \\
  \freenames{\dropn{x}} & := & \{ x \}
\end{eqnarray*}

The bound names of a process, $\boundnames{P}$, are those names occurring in $P$
that are not free. For example, in $x?(y).0$, the name $x$ is free, while $y$ is bound.

\begin{mathpar}
  \inferrule* [lab=monoidal-laws] {} { P|Q \equiv Q|P \and P|0 \equiv P \and P|(Q|R) \equiv (P|Q)|R }
\end{mathpar}

\begin{mathpar}
  \inferrule* [lab=alpha-equivalence] {} { (x)P \equiv (y)P\{y/x\} \and y \not\in \freenames{P} }
\end{mathpar}

\begin{definition}
Then two processes, $P,Q$, are alpha-equivalent if $P = Q\{\vec{y}/\vec{x}\}$ for
some $\vec{x} \in \boundnames{Q},\vec{y} \in \boundnames{P}$, where $Q\{\vec{y}/\vec{x}\}$
denotes the capture-avoiding substitution of $\vec{y}$ for $\vec{x}$ in $Q$.
\end{definition}

\begin{definition}
  The {\em structural congruence} \cite{SangiorgiWalker} , $\equiv$,
  between processes is the least congruence containing
  alpha-equivalence, satisfying the abelian monoid laws
  (associativity, commutativity and $\pzero$ as identity) for parallel
  composition $|$ and for summation $+$.
\end{definition}

\subsection{Name equivalence}

We take name equivalence, written $\nameeq$, to be the smallest
equivalence relation generated by the following rules.

\begin{mathpar}
\inferrule*[lab=Quote-drop]
{ }
{ \quotep{@{x}} \nameeq x }

\inferrule*[lab=Struct-equiv]
{ P \scong Q }
{ \quotep{P} \nameeq \quotep{Q} }
\end{mathpar}

The astute reader will have noticed that the mutual recursion of names
and processes imposes a mutual recursion on alpha-equivalence and
structural equivalence via name-equivalence. Fortunately, all of this
works out pleasantly and we may calculate in the natural way, free of
concern. The reader interested in the details is referred to the
appendix \ref{appendix:rho_details}.

\subsection{Substitution}

We use $\Proc$ for the set of processes, $\QProc$ for the set of
names, and $\id{\{}\vec{y} / \vec{x} \id{\}}$ to denote partial maps,
$s : \QProc \rightarrow \QProc$. A map, $s$ lifts, uniquely, to a map
on process terms, $\widehat{s} : \Proc \rightarrow \Proc$ by the
following equations.

\begin{mathpar}
  (0) \psubstp{Q}{P} := 0 \\
  (R \juxtap S) \psubstp{Q}{P}
  :=    
  (R)\psubstp{Q}{P} \juxtap (S) \psubstp{Q}{P} \\
  (x?(y).R) \psubstp{Q}{P}    
  :=    
  (x)\substp{Q}{P} (z)\concat( (R \psubstn{z}{y}) \psubstp{Q}{P} ) \\
  (\lift{x}{R}) \psubstp{Q}{P}  
  :=
  \lift{(x)\substp{Q}{P}}{ R \psubstp{Q}{P} } \\
%   (\dropn{x})  \psubstp{Q}{P}       
%   := 
%   \left\{ 
%     \begin{array}{ccc} 
%       \dropn{\quotep{Q}} & & x \nameeq \quotep{P} \\
%       \dropn{x} & & otherwise \\
%     \end{array}
%   \right. 
  (\dropn{x})  \psubstp{Q}{P}       
  := 
  \left\{ 
    \begin{array}{ccc} 
      Q & & x \nameeq \quotep{P} \\
      \dropn{x} & & otherwise \\
    \end{array}
  \right.
\end{mathpar}
 

where

\begin{eqnarray}
  (x)\id{\{} \lpquote Q \rpquote / \lpquote P \rpquote \id{\}}            = 
  \left\{ 
    \begin{array}{ccc}
      \lpquote Q \rpquote & & x \nameeq \lpquote P \rpquote \\
      x & & otherwise \\
    \end{array}
  \right. \nonumber
\end{eqnarray}

and $z$ is chosen distinct from $\quotep{P}$, $\quotep{Q}$, the free
names in $Q$, and all the names in $R$. Our $\alpha$-equivalence will
be built in the standard way from this substitution.

\begin{remark}\label{rem:no_self_referential_names}
  One consequence of these definitions is that $\forall P. \quotep{P}
  \not\in \freenames{P}$.
\end{remark}

\subsection{ Dynamic quote: an example }

Anticipating something of what's to come, consider applying the
substitution, $\widehat{\id{\{}u / z \id{\}}}$, to the following pair
of processes, $\lift{w}{y!(z)}$ and $w[ \lpquote y!(z) \rpquote ]$.

\begin{eqnarray}
	\lift{w}{y!(z)}\widehat{\id{\{}u / z \id{\}}}
		& = &
		\lift{w}{y!(u)} \nonumber\\
	w[ \lpquote y!(z) \rpquote ] \widehat{ \id{\{}u / z \id{\}} }
		& = &
		w[ \lpquote y!(z) \rpquote ] \nonumber
\end{eqnarray}

Because the body of the process between quotes is impervious to
substitution, we get radically different answers. In fact, by
examining the first process in an input context,
e.g. $x?(z).\lift{w}{y!(z)}$, we see that the process under the lift
operator may be shaped by prefixed inputs binding a name inside it. In
this sense, the lift operator will be seen as a way to dynamically
construct processes before reifying them as names.

Finally equipped with these standard features we can present the
dynamics of the calculus.

\subsubsection{Operational semantics} 

Finally, we introduce the computational dynamics. What marks these
algebras as distinct from other more traditionally studied algebraic
structures, e.g. vector spaces or polynomial rings, is the manner in
which dynamics is captured. In traditional structures, dynamics is typically
expressed through morphisms between such structures, as in linear maps
between vector spaces or morphisms between rings. In algebras
associated with the semantics of computation, the dynamics is
expressed as part of the algebraic structure itself, through a
reduction reduction relation typically denoted by $\red$. Below, we
give a recursive presentation of this relation for the calculus used
in the encoding.

$\red \subseteq \pi \times \pi$
$\red : \pi \to \mathcal{P}(\pi)$

\begin{mathpar}
  \inferrule* [lab=Comm] { \textsf{match}( x_{src}, x_{trgt} ) } { x_{trgt}?(y)P \; | \; x_{src}!\langle {Q} \rangle \red P\{\quotep{Q}/y}\} }
  \and \\
  \inferrule* [lab=Par] {{P} \red {P}'} {{{P} | {Q}} \red {{P}' | {Q}}}
  \and
  \inferrule* [lab=Equiv]{{{P} \scong {P}'} \andalso {{P}' \red {Q}'} \andalso {{Q}' \scong {Q}}}{{P} \red {Q}}
\end{mathpar}

\begin{eqnarray*}
  match_{\equiv} (\quotep{P},\quotep{Q}) & := & P \equiv Q \\
  match_{\dagger}(\quotep{P},\quotep{Q}) & := & \forall R. P|Q \red^{*} R => R \red^{*} 0 \\
  match_{K}(\quotep{P},\quotep{Q}) & := & K \mbox{ for some context } K
\end{eqnarray*}

$u?(x)P | u!\langle Q \rangle \red P\{\quotep{Q}/x\}$

%We write $\wred$ for $\red^*$, and $P\red$ if $\exists Q $ such that $ P \red Q$.
We write $P\red$ if $\exists Q $ such that $ P \red Q$ and $P\not\red$, otherwise.

\section{Replication}

As mentioned before, it is known that replication (and hence
recursion) can be implemented in a higher-order process algebra
\cite{SangiorgiWalker}. As our first example of calculation with the
machinery thus far presented we give the construction explicitly in
the {\rhoc}.

\begin{eqnarray}
	D_{x} & := & \prefix{x}{y}{(\binpar{\outputp{x}{y}}{@{y}})} \nonumber\\
	\bangp_{x}{P} & := & \binpar{{x}!\langle{\binpar{D_{x}}{P}}\rangle}{D_{x}} \nonumber
\end{eqnarray}

\begin{eqnarray}
	\bangp_{x}{P} & & \nonumber\\
	=
	& {x}!\langle{(\prefix{x}{y}{(\outputp{x}{y} | @{y})) | P}}\rangle 
	      | \prefix{x}{y}{(\outputp{x}{y} | @{y})} & \nonumber\\
	\red
	& (\outputp{x}{y} | @{y})\substn{\quotep{(\prefix{x}{y}{(@{y} | \outputp{x}{y})) | P}}}{y} & \nonumber\\
	=
	& \outputp{x}{\quotep{(\prefix{x}{y}{(\outputp{x}{y} | @{y})) | P}}}
	  | {(\prefix{x}{y}{(\outputp{x}{y} | @{y})) | P}} & \nonumber\\
	\red
	& \ldots & \nonumber\\
	\red^*
	& P | P | \ldots & \nonumber
\end{eqnarray}

Of course, this encoding, as an implementation, runs away, unfolding
$\bangp{P}$ eagerly. A lazier and more implementable replication
operator, restricted to input-guarded processes, may be obtained as follows.

\begin{eqnarray}
\bangp{\prefix{u}{v}{P}} 
	:= 
	\binpar{\lift{x}{\prefix{u}{v}{(\binpar{D(x)}{P})}}}{D(x)} \nonumber
\end{eqnarray}

\begin{remark}
  Note that the lazier definition still does not deal with summation
  or mixed summation (i.e. sums over input and output). The reader is
  invited to construct definitions of replication that deal with these
  features. 

  Further, the definitions are parameterized in a name, $x$. Can you,
  gentle reader, make a definition that eliminates this parameter and
  guarantees no accidental interaction between the replication
  machinery and the process being replicated -- i.e. no accidental
  sharing of names used by the process to get its work done and the
  name(s) used by the replication to effect copying. This latter
  revision of the definition of replication is crucial to obtaining
  the expected identity $!!P \sim !P$.
\end{remark}

\begin{remark}\label{rem:paradoxical_combinator}
  The reader familiar with the lambda calculus will have noticed the
  similarity between $D$ and the paradoxical combinator.

  [Ed. note: the existence of this seems to suggest we have to be more
  restrictive on the set of processes and names we admit if we are to
  support no-cloning.]
\end{remark}

\subsubsection{Bisimulation}

The computational dynamics gives rise to another kind of equivalence,
the equivalence of computational behavior. As previously mentioned
this is typically captured \emph{via} some form of bisimulation.

% The notion we use in this paper is weak barbed bisimulation
% \cite{milner91polyadicpi}.

The notion we use in this paper is derived from weak barbed
bisimulation \cite{milner91polyadicpi}. 

\begin{definition}
An \emph{observation relation}, $\downarrow_{\mathcal N}$, over a set
of names, $\mathcal N$, is the smallest relation satisfying the rules
below.

\infrule[Out-barb]{y \in {\mathcal N}, \; x \nameeq y}
		  {\outputp{x}{v} \downarrow_{\mathcal N} x}
\infrule[Par-barb]{\mbox{$P\downarrow_{\mathcal N} x$ or $Q\downarrow_{\mathcal N} x$}}
		  {\binpar{P}{Q} \downarrow_{\mathcal N} x}

We write $P \Downarrow_{\mathcal N} x$ if there is $Q$ such that 
$P \wred Q$ and $Q \downarrow_{\mathcal N} x$.
\end{definition}

\begin{definition}
%\label{def.bbisim}
An  ${\mathcal N}$-\emph{barbed bisimulation} over a set of names, ${\mathcal N}$, is a symmetric binary relation 
${\mathcal S}_{\mathcal N}$ between agents such that $P\rel{S}_{\mathcal N}Q$ implies:
\begin{enumerate}
\item If $P \red P'$ then $Q \wred Q'$ and $P'\rel{S}_{\mathcal N} Q'$.
\item If $P\downarrow_{\mathcal N} x$, then $Q\Downarrow_{\mathcal N} x$.
\end{enumerate}
$P$ is ${\mathcal N}$-barbed bisimilar to $Q$, written
$P \wbbisim_{\mathcal N} Q$, if $P \rel{S}_{\mathcal N} Q$ for some ${\mathcal N}$-barbed bisimulation ${\mathcal S}_{\mathcal N}$.
\end{definition}

$\mathcal{R} \subseteq \pi \times \pi$

$P \mathcal{R} Q => \forall P'. P \red P' \Rightarrow \exists Q'. Q \red Q', P' \mathcal{R} Q'$

$P \vdash x \Rightarrow Q \vdash x$

\begin{mathpar}
  \inferrule*[lab=Out-barb]{x \nameeq y}{{y}!\langle{Q}\rangle \vdash x}
  \and
  \inferrule*[lab=Par-barb]{\mbox{$P\vdash x$ or $Q\vdash x$}}{\binpar{P}{Q} \vdash x}
\end{mathpar}

\subsubsection{Contexts}

One of the principle advantages of computational calculi like the
$\pi$-calculus is a well-defined notion of context,
contextual-equivalence and a correlation between
contextual-equivalence and notions of bisimulation. The notion of
context allows the decomposition of a process into (sub-)process and
its syntactic environment, its context. Thus, a context may be
thought of as a process with a ``hole'' (written $\Box$) in it. The
application of a context $M$ to a process $P$, written $M[P]$, is
tantamount to filling the hole in $M$ with $P$. In this paper we do
not need the full weight of this theory, but do make use of the notion
of context in the proof the main theorem. 

\begin{mathpar}
  \inferrule* [lab=summation] {} {{M_{M},M_{N}} \bc \Box \;|\; x.M_{A} \;|\; M_{M}+M_{N}}
  \and
  \inferrule* [lab=agent] {} {{M_{A}} \bc (\vec{x})M_{P} \;| \; \clift{P_0,\ldots,M_{P},\ldots,P_N}}
  \and \\
  \inferrule* [lab=process] {} {{M_{P}} \bc M_{N} \;| \;P|M_{P} }
\end{mathpar} 

\begin{mathpar}
  \inferrule* [lab=sychronization] {} {M_{N} \bc \Box \;|\; x?M_{F} \;|\; x!M_{C}}
  \and
  \inferrule* [lab=abstraction] {} {{M_{F}} \bc (x)M_{P} }
  \and
  \inferrule* [lab=concretion] {} {{M_{C}} \bc \langle M_{P} \rangle }
  \and \\
  \inferrule* [lab=process] {} {{M_{P}} \bc M_{N} \;| \;P|M_{P} }
\end{mathpar}

\begin{definition}[contextual application] Given a context $M$, and
  process $P$, we define the \emph{contextual application}, $M[P] :=
  M\{P/\Box\}$. That is, the contextual application of M to P is the
  substitution of $P$ for $\Box$ in $M$.
\end{definition}

$\meaningof{-} : L \to \mathcal{P}(\pi)$

\begin{mathpar}
  \inferrule* [lab=collection] {} {\meaningof{true} = \pi, \and \meaningof{~E} = \pi \setminus \meaningof{E}, \and \meaningof{E_{1} \& E_{2}} = \meaningof{E_{1}} \cap \meaningof{E_{2}}}
\end{mathpar}

\begin{mathpar}
  \inferrule* [lab=structure] {} {\meaningof{0} = \{ P \in \pi | P \equiv 0 \}, \and \\ \meaningof{E_1 | E_2} = \{ P \in \pi | P \equiv P_{1} | P_{2}, P_{1} \in \meaningof{E_{1}}, P_{2} \in \meaningof{E_2}\} }
\end{mathpar}

\begin{mathpar}
 \inferrule* [lab=behavior] {} {\meaningof{\langle a?b \rangle E} = \{ P \in \pi | P \equiv Q | u?(y)P', \\ \and \\\\ \and \\ \;\;\; u \in \meaningof{a}, \forall z.P'\{z/y\} \in \meaningof{E\{z/b\}}\}, \and \\ \meaningof{a!E} = \{ P \in \pi | P \equiv Q | x!\langle P' \rangle, x \in \meaningof{a} P' \in \meaningof{E}\} }
\end{mathpar}

\begin{mathpar}
 \inferrule* [lab=nominal] {} {\meaningof{\quotep{E}} = \{ \quotep{P} \in \quotep{\pi} | P \in \meaningof{E} \}, \and \meaningof{\quotep{P}} = \{ \quotep{Q} \in \quotep{\pi} | P \equiv Q \} \and \\ \meaningof{@\quotep{E}} = \{ P \in \pi | P \equiv @x, x \in \meaningof{E} \}}
\end{mathpar}

\begin{eqnarray*}
  \\
  \meaningof{-} : TS \to ST
\end{eqnarray*}

\begin{eqnarray*}
  \\
  L : TS \to ST
\end{eqnarray*}

\begin{eqnarray*}
  \\
  P \models E \iff P \in \meaningof{E}
\end{eqnarray*}

\begin{eqnarray*}
  P \approx_{L} Q \iff \forall E \in L. P \models E \iff Q \models E
\end{eqnarray*}

\begin{eqnarray*}
  P \approx_{K} Q
\end{eqnarray*}

\begin{eqnarray*}
  P \approx Q
\end{eqnarray*}

$\approx_{K} = \approx = \approx_{L}$

\subsubsection{Contextual duality}

Note that contexts extend the quotation operation to a family of
operations from processes to names. Given a context, $M$, we can
define a \emph{nominal context}, $\quotep{M}$ by $\quotep{M}[P] :=
\quotep{M[P]}$. To foreshadow what is to come we observe that these
operations enjoy a duality with processes very much like the duality
between vectors and maps from vectors to scalars.

Further, because the calculus is essentially higher-order, we have a
correspondence between contexts and processes. More specifically,
given a name $x$ and a context $M$ we can construct $M^{*}_{x}$ such
that 

\begin{mathpar}
  M^{*}_{x} | \lift{x}{P} \red M[P]
\end{mathpar}

namely,

\begin{mathpar}
  M^{*}_{x} := x?(u).M[\dropn{u}]
\end{mathpar}

The dependence of $M^{*}_{x}$ on a name makes it an abstraction, 

\begin{mathpar}
  M^{*} := (x)x?(u).M[\dropn{u}]
\end{mathpar}

\subsection{Additional notation}

It will sometimes be convenient to denote the process a name
quotes. We already have the notation $x = \quotep{P}$, but it will be
convenient to introduce an alternate notation, $\procn{x}$, when we
want to emphasize the connection to the use of the name. Note that, by
virtue of name equivalence, $\quotep{\procn{x}} \nameeq x$; so, the
notation is consistent with previous definitions.

Further, because names have structure it is possible to effect
substitutions on the basis of that structure. This means we need to
upgrade our notation for substitutions, which we accomplish by
adapting comprehension notation. Thus,

\begin{mathpar}
  P\{ y / x : x \in S \}
\end{mathpar}

is interpreted to mean the process derived from P by replacing (in a
capture-avoiding manner) each occurrence of $x$ in $S$ by $y$. For example,

\begin{mathpar}
  P\{ \quotep{\procn{x}|\procn{x}} / x : x \in \freenames{P} \}
\end{mathpar}

will replace each (occurrence) of a free name $x$ in $P$ by
$\quotep{\procn{x}|\procn{x}}$.

Also, we will avail ourselves of the notation $x^{L}$ and $x^{R}$ to
denote injections of a name into disjoint copies of the name
space. There are numerous ways to accomplish this. One example can be
found in \cite{MeredithR05}. This notation overloads to vectors of
names: $\vec{x}^{\pi} := (x_{i}^{\pi} \; : \; 0 \leq i < |\vec{x}| )$ where $\pi \in \{L,R\}$.

We also use $P^{\Box} := P|\Box$.

In \cite{MeredithR05} an interpretation of the new operator is
given. It turns out that there are several possible interpretations
all enjoying the requisite algebraic properties of the operator (see
\cite{milner91polyadicpi}). We will therefore make liberal use of
$(\nu\; \vec{x})P$.

% subsection the_syntax_and_semantics_of_the_notation_system (end)   

\input{qm2pi.qmops} 

\input{qm2pi.sterngerlach} 

\input{qm2pi.metric} 

% section concurrent_process_calculi (end)

%\input{qm2pi.proofsketch}

% section proof sketch (end)

%\input{qm2pi.slviaknots} 

% section spatial logic via knots (end)

\input{qm2pi.conclusion}

% section conclusion (end)

%\input{qm2pi.dtcodes} 

% section wiring algorithm (end)

\input{qm2pi.ack} 

% section acknowledgments (end)

\newpage


\bibliographystyle{plain}   
\bibliography{../../biblios/main.bib}

\input{qm2pi.rhodetails}

\end{document}

 

% subsection basic_interpretation (end)

%\input{qm2pi.rho.presentation} 
\subsection{The syntax and semantics of the notation system}\label{sub:the_syntax_and_semantics_of_the_notation_system} % (fold)

We now summarize a technical presentation of the calculus that
embodies our theory of dynamics. The typical presentation of such a
calculus follows the style of giving generators and relations on
them. The grammar, below, describing term constructors, freely
generates the set of processes, $\Proc$. This set is then quotiented
by a relation known as structural congruence and it is over this set
that the notion of dynamics is expressed. This presentation is
essentially that of \cite{MeredithR05} with the addition of
polyadicity and summation. For readability we have relegated some of
the technical subtleties to an appendix.

\subsubsection{Process grammar}\label{subsub:process_grammar}

\begin{mathpar}
  \inferrule* [lab=synchronization] {} {{M} \bc \pzero \;|\; x?F \;|\; x!C }
  \and
  \inferrule* [lab=abstraction] {} {{F} \bc (x)P}
  \and
  \inferrule* [lab=concretion] {} {{C} \bc \langle Q \rangle}
  \and
  \inferrule* [lab=process] {} {{P,Q} \bc M \;| \;P|Q \;|\; @{x}}
  \and
  \inferrule* [lab=name] {} {{x} \bc \quotep{P}}
\end{mathpar} 

Note that $\vec{x}$ (resp. $\vec{P}$) denotes a vector of names
(resp. processes) of length $|\vec{x}|$ (resp. $|\vec{P}|$). We adopt
the following useful abbreviations.

\begin{mathpar}
   x?(\vec{y}).P := x.(\vec{y})P \and  x\clift{\vec{P}} := x.\clift{\vec{P}}
   \and x!(y) := \lift{x}{\dropn{y}}
   \and \Pi_{i=0}^{n-1}P_i := P_0 | \ldots | P_{n-1}
\end{mathpar}

\subsubsection{Structural congruence}

\paragraph{Free and bound names and alpha-equivalence.} At the
core of structural equivalence is alpha-equivalence which identifies
process that are the same up to a change of variable. Formally, we
recognize the distinction between free and bound names. The free names
of a process, $\freenames{P}$, may be calculated recursively as
follows:

\begin{mathpar}
\freenames{\pzero} := \emptyset
  \and \\
  \freenames{x?(y).P} := \{ x \} \cup (\freenames{P} \setminus \{ y \})
  \and 
  \freenames{x!\langle P \rangle} := \{ x \} \cup \{ P \} 
  \and \\
  \freenames{P|Q} := \freenames{P} \cup \freenames{Q}
  \and \\
  \freenames{@{x}} := \{ x \}
\end{mathpar}

$\pi$
$\quotep{\pi}$

$\freenames{-} : \pi \to \mathcal{P}(\quotep{\pi})$

\begin{eqnarray*}
  \freenames{\pzero} & := & \emptyset \\
  \freenames{x?(y).P} & := & \{ x \} \cup (\freenames{P} \setminus \{ y \}) \\
  \freenames{x!\langle P \rangle} & := & \{ x \} \cup \{ P \} \\
  \freenames{P|Q} & := & \freenames{P} \cup \freenames{Q} \\
  \freenames{\dropn{x}} & := & \{ x \}
\end{eqnarray*}

The bound names of a process, $\boundnames{P}$, are those names occurring in $P$
that are not free. For example, in $x?(y).0$, the name $x$ is free, while $y$ is bound.

\begin{mathpar}
  \inferrule* [lab=monoidal-laws] {} { P|Q \equiv Q|P \and P|0 \equiv P \and P|(Q|R) \equiv (P|Q)|R }
\end{mathpar}

\begin{mathpar}
  \inferrule* [lab=alpha-equivalence] {} { (x)P \equiv (y)P\{y/x\} \and y \not\in \freenames{P} }
\end{mathpar}

\begin{definition}
Then two processes, $P,Q$, are alpha-equivalent if $P = Q\{\vec{y}/\vec{x}\}$ for
some $\vec{x} \in \boundnames{Q},\vec{y} \in \boundnames{P}$, where $Q\{\vec{y}/\vec{x}\}$
denotes the capture-avoiding substitution of $\vec{y}$ for $\vec{x}$ in $Q$.
\end{definition}

\begin{definition}
  The {\em structural congruence} \cite{SangiorgiWalker} , $\equiv$,
  between processes is the least congruence containing
  alpha-equivalence, satisfying the abelian monoid laws
  (associativity, commutativity and $\pzero$ as identity) for parallel
  composition $|$ and for summation $+$.
\end{definition}

\subsection{Name equivalence}

We take name equivalence, written $\nameeq$, to be the smallest
equivalence relation generated by the following rules.

\begin{mathpar}
\inferrule*[lab=Quote-drop]
{ }
{ \quotep{@{x}} \nameeq x }

\inferrule*[lab=Struct-equiv]
{ P \scong Q }
{ \quotep{P} \nameeq \quotep{Q} }
\end{mathpar}

The astute reader will have noticed that the mutual recursion of names
and processes imposes a mutual recursion on alpha-equivalence and
structural equivalence via name-equivalence. Fortunately, all of this
works out pleasantly and we may calculate in the natural way, free of
concern. The reader interested in the details is referred to the
appendix \ref{appendix:rho_details}.

\subsection{Substitution}

We use $\Proc$ for the set of processes, $\QProc$ for the set of
names, and $\id{\{}\vec{y} / \vec{x} \id{\}}$ to denote partial maps,
$s : \QProc \rightarrow \QProc$. A map, $s$ lifts, uniquely, to a map
on process terms, $\widehat{s} : \Proc \rightarrow \Proc$ by the
following equations.

\begin{mathpar}
  (0) \psubstp{Q}{P} := 0 \\
  (R \juxtap S) \psubstp{Q}{P}
  :=    
  (R)\psubstp{Q}{P} \juxtap (S) \psubstp{Q}{P} \\
  (x?(y).R) \psubstp{Q}{P}    
  :=    
  (x)\substp{Q}{P} (z)\concat( (R \psubstn{z}{y}) \psubstp{Q}{P} ) \\
  (\lift{x}{R}) \psubstp{Q}{P}  
  :=
  \lift{(x)\substp{Q}{P}}{ R \psubstp{Q}{P} } \\
%   (\dropn{x})  \psubstp{Q}{P}       
%   := 
%   \left\{ 
%     \begin{array}{ccc} 
%       \dropn{\quotep{Q}} & & x \nameeq \quotep{P} \\
%       \dropn{x} & & otherwise \\
%     \end{array}
%   \right. 
  (\dropn{x})  \psubstp{Q}{P}       
  := 
  \left\{ 
    \begin{array}{ccc} 
      Q & & x \nameeq \quotep{P} \\
      \dropn{x} & & otherwise \\
    \end{array}
  \right.
\end{mathpar}
 

where

\begin{eqnarray}
  (x)\id{\{} \lpquote Q \rpquote / \lpquote P \rpquote \id{\}}            = 
  \left\{ 
    \begin{array}{ccc}
      \lpquote Q \rpquote & & x \nameeq \lpquote P \rpquote \\
      x & & otherwise \\
    \end{array}
  \right. \nonumber
\end{eqnarray}

and $z$ is chosen distinct from $\quotep{P}$, $\quotep{Q}$, the free
names in $Q$, and all the names in $R$. Our $\alpha$-equivalence will
be built in the standard way from this substitution.

\begin{remark}\label{rem:no_self_referential_names}
  One consequence of these definitions is that $\forall P. \quotep{P}
  \not\in \freenames{P}$.
\end{remark}

\subsection{ Dynamic quote: an example }

Anticipating something of what's to come, consider applying the
substitution, $\widehat{\id{\{}u / z \id{\}}}$, to the following pair
of processes, $\lift{w}{y!(z)}$ and $w[ \lpquote y!(z) \rpquote ]$.

\begin{eqnarray}
	\lift{w}{y!(z)}\widehat{\id{\{}u / z \id{\}}}
		& = &
		\lift{w}{y!(u)} \nonumber\\
	w[ \lpquote y!(z) \rpquote ] \widehat{ \id{\{}u / z \id{\}} }
		& = &
		w[ \lpquote y!(z) \rpquote ] \nonumber
\end{eqnarray}

Because the body of the process between quotes is impervious to
substitution, we get radically different answers. In fact, by
examining the first process in an input context,
e.g. $x?(z).\lift{w}{y!(z)}$, we see that the process under the lift
operator may be shaped by prefixed inputs binding a name inside it. In
this sense, the lift operator will be seen as a way to dynamically
construct processes before reifying them as names.

Finally equipped with these standard features we can present the
dynamics of the calculus.

\subsubsection{Operational semantics} 

Finally, we introduce the computational dynamics. What marks these
algebras as distinct from other more traditionally studied algebraic
structures, e.g. vector spaces or polynomial rings, is the manner in
which dynamics is captured. In traditional structures, dynamics is typically
expressed through morphisms between such structures, as in linear maps
between vector spaces or morphisms between rings. In algebras
associated with the semantics of computation, the dynamics is
expressed as part of the algebraic structure itself, through a
reduction reduction relation typically denoted by $\red$. Below, we
give a recursive presentation of this relation for the calculus used
in the encoding.

$\red \subseteq \pi \times \pi$
$\red : \pi \to \mathcal{P}(\pi)$

\begin{mathpar}
  \inferrule* [lab=Comm] { \textsf{match}( x_{src}, x_{trgt} ) } { x_{trgt}?(y)P \; | \; x_{src}!\langle {Q} \rangle \red P\{\quotep{Q}/y}\} }
  \and \\
  \inferrule* [lab=Par] {{P} \red {P}'} {{{P} | {Q}} \red {{P}' | {Q}}}
  \and
  \inferrule* [lab=Equiv]{{{P} \scong {P}'} \andalso {{P}' \red {Q}'} \andalso {{Q}' \scong {Q}}}{{P} \red {Q}}
\end{mathpar}

\begin{eqnarray*}
  match_{\equiv} (\quotep{P},\quotep{Q}) & := & P \equiv Q \\
  match_{\dagger}(\quotep{P},\quotep{Q}) & := & \forall R. P|Q \red^{*} R => R \red^{*} 0 \\
  match_{K}(\quotep{P},\quotep{Q}) & := & K \mbox{ for some context } K
\end{eqnarray*}

$u?(x)P | u!\langle Q \rangle \red P\{\quotep{Q}/x\}$

%We write $\wred$ for $\red^*$, and $P\red$ if $\exists Q $ such that $ P \red Q$.
We write $P\red$ if $\exists Q $ such that $ P \red Q$ and $P\not\red$, otherwise.

\section{Replication}

As mentioned before, it is known that replication (and hence
recursion) can be implemented in a higher-order process algebra
\cite{SangiorgiWalker}. As our first example of calculation with the
machinery thus far presented we give the construction explicitly in
the {\rhoc}.

\begin{eqnarray}
	D_{x} & := & \prefix{x}{y}{(\binpar{\outputp{x}{y}}{@{y}})} \nonumber\\
	\bangp_{x}{P} & := & \binpar{{x}!\langle{\binpar{D_{x}}{P}}\rangle}{D_{x}} \nonumber
\end{eqnarray}

\begin{eqnarray}
	\bangp_{x}{P} & & \nonumber\\
	=
	& {x}!\langle{(\prefix{x}{y}{(\outputp{x}{y} | @{y})) | P}}\rangle 
	      | \prefix{x}{y}{(\outputp{x}{y} | @{y})} & \nonumber\\
	\red
	& (\outputp{x}{y} | @{y})\substn{\quotep{(\prefix{x}{y}{(@{y} | \outputp{x}{y})) | P}}}{y} & \nonumber\\
	=
	& \outputp{x}{\quotep{(\prefix{x}{y}{(\outputp{x}{y} | @{y})) | P}}}
	  | {(\prefix{x}{y}{(\outputp{x}{y} | @{y})) | P}} & \nonumber\\
	\red
	& \ldots & \nonumber\\
	\red^*
	& P | P | \ldots & \nonumber
\end{eqnarray}

Of course, this encoding, as an implementation, runs away, unfolding
$\bangp{P}$ eagerly. A lazier and more implementable replication
operator, restricted to input-guarded processes, may be obtained as follows.

\begin{eqnarray}
\bangp{\prefix{u}{v}{P}} 
	:= 
	\binpar{\lift{x}{\prefix{u}{v}{(\binpar{D(x)}{P})}}}{D(x)} \nonumber
\end{eqnarray}

\begin{remark}
  Note that the lazier definition still does not deal with summation
  or mixed summation (i.e. sums over input and output). The reader is
  invited to construct definitions of replication that deal with these
  features. 

  Further, the definitions are parameterized in a name, $x$. Can you,
  gentle reader, make a definition that eliminates this parameter and
  guarantees no accidental interaction between the replication
  machinery and the process being replicated -- i.e. no accidental
  sharing of names used by the process to get its work done and the
  name(s) used by the replication to effect copying. This latter
  revision of the definition of replication is crucial to obtaining
  the expected identity $!!P \sim !P$.
\end{remark}

\begin{remark}\label{rem:paradoxical_combinator}
  The reader familiar with the lambda calculus will have noticed the
  similarity between $D$ and the paradoxical combinator.

  [Ed. note: the existence of this seems to suggest we have to be more
  restrictive on the set of processes and names we admit if we are to
  support no-cloning.]
\end{remark}

\subsubsection{Bisimulation}

The computational dynamics gives rise to another kind of equivalence,
the equivalence of computational behavior. As previously mentioned
this is typically captured \emph{via} some form of bisimulation.

% The notion we use in this paper is weak barbed bisimulation
% \cite{milner91polyadicpi}.

The notion we use in this paper is derived from weak barbed
bisimulation \cite{milner91polyadicpi}. 

\begin{definition}
An \emph{observation relation}, $\downarrow_{\mathcal N}$, over a set
of names, $\mathcal N$, is the smallest relation satisfying the rules
below.

\infrule[Out-barb]{y \in {\mathcal N}, \; x \nameeq y}
		  {\outputp{x}{v} \downarrow_{\mathcal N} x}
\infrule[Par-barb]{\mbox{$P\downarrow_{\mathcal N} x$ or $Q\downarrow_{\mathcal N} x$}}
		  {\binpar{P}{Q} \downarrow_{\mathcal N} x}

We write $P \Downarrow_{\mathcal N} x$ if there is $Q$ such that 
$P \wred Q$ and $Q \downarrow_{\mathcal N} x$.
\end{definition}

\begin{definition}
%\label{def.bbisim}
An  ${\mathcal N}$-\emph{barbed bisimulation} over a set of names, ${\mathcal N}$, is a symmetric binary relation 
${\mathcal S}_{\mathcal N}$ between agents such that $P\rel{S}_{\mathcal N}Q$ implies:
\begin{enumerate}
\item If $P \red P'$ then $Q \wred Q'$ and $P'\rel{S}_{\mathcal N} Q'$.
\item If $P\downarrow_{\mathcal N} x$, then $Q\Downarrow_{\mathcal N} x$.
\end{enumerate}
$P$ is ${\mathcal N}$-barbed bisimilar to $Q$, written
$P \wbbisim_{\mathcal N} Q$, if $P \rel{S}_{\mathcal N} Q$ for some ${\mathcal N}$-barbed bisimulation ${\mathcal S}_{\mathcal N}$.
\end{definition}

$\mathcal{R} \subseteq \pi \times \pi$

$P \mathcal{R} Q => \forall P'. P \red P' \Rightarrow \exists Q'. Q \red Q', P' \mathcal{R} Q'$

$P \vdash x \Rightarrow Q \vdash x$

\begin{mathpar}
  \inferrule*[lab=Out-barb]{x \nameeq y}{{y}!\langle{Q}\rangle \vdash x}
  \and
  \inferrule*[lab=Par-barb]{\mbox{$P\vdash x$ or $Q\vdash x$}}{\binpar{P}{Q} \vdash x}
\end{mathpar}

\subsubsection{Contexts}

One of the principle advantages of computational calculi like the
$\pi$-calculus is a well-defined notion of context,
contextual-equivalence and a correlation between
contextual-equivalence and notions of bisimulation. The notion of
context allows the decomposition of a process into (sub-)process and
its syntactic environment, its context. Thus, a context may be
thought of as a process with a ``hole'' (written $\Box$) in it. The
application of a context $M$ to a process $P$, written $M[P]$, is
tantamount to filling the hole in $M$ with $P$. In this paper we do
not need the full weight of this theory, but do make use of the notion
of context in the proof the main theorem. 

\begin{mathpar}
  \inferrule* [lab=summation] {} {{M_{M},M_{N}} \bc \Box \;|\; x.M_{A} \;|\; M_{M}+M_{N}}
  \and
  \inferrule* [lab=agent] {} {{M_{A}} \bc (\vec{x})M_{P} \;| \; \clift{P_0,\ldots,M_{P},\ldots,P_N}}
  \and \\
  \inferrule* [lab=process] {} {{M_{P}} \bc M_{N} \;| \;P|M_{P} }
\end{mathpar} 

\begin{mathpar}
  \inferrule* [lab=sychronization] {} {M_{N} \bc \Box \;|\; x?M_{F} \;|\; x!M_{C}}
  \and
  \inferrule* [lab=abstraction] {} {{M_{F}} \bc (x)M_{P} }
  \and
  \inferrule* [lab=concretion] {} {{M_{C}} \bc \langle M_{P} \rangle }
  \and \\
  \inferrule* [lab=process] {} {{M_{P}} \bc M_{N} \;| \;P|M_{P} }
\end{mathpar}

\begin{definition}[contextual application] Given a context $M$, and
  process $P$, we define the \emph{contextual application}, $M[P] :=
  M\{P/\Box\}$. That is, the contextual application of M to P is the
  substitution of $P$ for $\Box$ in $M$.
\end{definition}

$\meaningof{-} : L \to \mathcal{P}(\pi)$

\begin{mathpar}
  \inferrule* [lab=collection] {} {\meaningof{true} = \pi, \and \meaningof{~E} = \pi \setminus \meaningof{E}, \and \meaningof{E_{1} \& E_{2}} = \meaningof{E_{1}} \cap \meaningof{E_{2}}}
\end{mathpar}

\begin{mathpar}
  \inferrule* [lab=structure] {} {\meaningof{0} = \{ P \in \pi | P \equiv 0 \}, \and \\ \meaningof{E_1 | E_2} = \{ P \in \pi | P \equiv P_{1} | P_{2}, P_{1} \in \meaningof{E_{1}}, P_{2} \in \meaningof{E_2}\} }
\end{mathpar}

\begin{mathpar}
 \inferrule* [lab=behavior] {} {\meaningof{\langle a?b \rangle E} = \{ P \in \pi | P \equiv Q | u?(y)P', \\ \and \\\\ \and \\ \;\;\; u \in \meaningof{a}, \forall z.P'\{z/y\} \in \meaningof{E\{z/b\}}\}, \and \\ \meaningof{a!E} = \{ P \in \pi | P \equiv Q | x!\langle P' \rangle, x \in \meaningof{a} P' \in \meaningof{E}\} }
\end{mathpar}

\begin{mathpar}
 \inferrule* [lab=nominal] {} {\meaningof{\quotep{E}} = \{ \quotep{P} \in \quotep{\pi} | P \in \meaningof{E} \}, \and \meaningof{\quotep{P}} = \{ \quotep{Q} \in \quotep{\pi} | P \equiv Q \} \and \\ \meaningof{@\quotep{E}} = \{ P \in \pi | P \equiv @x, x \in \meaningof{E} \}}
\end{mathpar}

\begin{eqnarray*}
  \\
  \meaningof{-} : TS \to ST
\end{eqnarray*}

\begin{eqnarray*}
  \\
  L : TS \to ST
\end{eqnarray*}

\begin{eqnarray*}
  \\
  P \models E \iff P \in \meaningof{E}
\end{eqnarray*}

\begin{eqnarray*}
  P \approx_{L} Q \iff \forall E \in L. P \models E \iff Q \models E
\end{eqnarray*}

\begin{eqnarray*}
  P \approx_{K} Q
\end{eqnarray*}

\begin{eqnarray*}
  P \approx Q
\end{eqnarray*}

$\approx_{K} = \approx = \approx_{L}$

\subsubsection{Contextual duality}

Note that contexts extend the quotation operation to a family of
operations from processes to names. Given a context, $M$, we can
define a \emph{nominal context}, $\quotep{M}$ by $\quotep{M}[P] :=
\quotep{M[P]}$. To foreshadow what is to come we observe that these
operations enjoy a duality with processes very much like the duality
between vectors and maps from vectors to scalars.

Further, because the calculus is essentially higher-order, we have a
correspondence between contexts and processes. More specifically,
given a name $x$ and a context $M$ we can construct $M^{*}_{x}$ such
that 

\begin{mathpar}
  M^{*}_{x} | \lift{x}{P} \red M[P]
\end{mathpar}

namely,

\begin{mathpar}
  M^{*}_{x} := x?(u).M[\dropn{u}]
\end{mathpar}

The dependence of $M^{*}_{x}$ on a name makes it an abstraction, 

\begin{mathpar}
  M^{*} := (x)x?(u).M[\dropn{u}]
\end{mathpar}

\subsection{Additional notation}

It will sometimes be convenient to denote the process a name
quotes. We already have the notation $x = \quotep{P}$, but it will be
convenient to introduce an alternate notation, $\procn{x}$, when we
want to emphasize the connection to the use of the name. Note that, by
virtue of name equivalence, $\quotep{\procn{x}} \nameeq x$; so, the
notation is consistent with previous definitions.

Further, because names have structure it is possible to effect
substitutions on the basis of that structure. This means we need to
upgrade our notation for substitutions, which we accomplish by
adapting comprehension notation. Thus,

\begin{mathpar}
  P\{ y / x : x \in S \}
\end{mathpar}

is interpreted to mean the process derived from P by replacing (in a
capture-avoiding manner) each occurrence of $x$ in $S$ by $y$. For example,

\begin{mathpar}
  P\{ \quotep{\procn{x}|\procn{x}} / x : x \in \freenames{P} \}
\end{mathpar}

will replace each (occurrence) of a free name $x$ in $P$ by
$\quotep{\procn{x}|\procn{x}}$.

Also, we will avail ourselves of the notation $x^{L}$ and $x^{R}$ to
denote injections of a name into disjoint copies of the name
space. There are numerous ways to accomplish this. One example can be
found in \cite{MeredithR05}. This notation overloads to vectors of
names: $\vec{x}^{\pi} := (x_{i}^{\pi} \; : \; 0 \leq i < |\vec{x}| )$ where $\pi \in \{L,R\}$.

We also use $P^{\Box} := P|\Box$.

In \cite{MeredithR05} an interpretation of the new operator is
given. It turns out that there are several possible interpretations
all enjoying the requisite algebraic properties of the operator (see
\cite{milner91polyadicpi}). We will therefore make liberal use of
$(\nu\; \vec{x})P$.

% subsection the_syntax_and_semantics_of_the_notation_system (end)   

\section{Interpretation of QM}
\subsection{Supporting definitions}
\subsubsection{Multiplication}
\begin{mathpar}
  \quotep{Q} \cdot \quotep{R} := \quotep{Q|R}
  \and \\
  \quotep{Q} \cdot P := P\{ \quotep{Q|R} / \quotep{R} : \quotep{R} \in \freenames{P} \}
\end{mathpar}

\paragraph{Discussion}
The first line needs little explanation. The second line says that
each free name of the process is replaced with the multiplication of
that name by the scalar. Multiplication of a scalar (name) by a state
(process) results in a process all the names of which have been `moved
over' by parallel composition with the process the scalar
quotes. There is a subtlety that the bound names have to be
manipulated so that multiplied names aren't accidentally
captured. There are many ways to achieve this.

\begin{remark}\label{rem:multiplication_identities}
  The reader is invited to verify that for all $x,y,z \in \QProc$ and $P \in \Proc$
  \begin{mathpar}
    x \cdot \quotep{0} \equiv x 
    \and
    x \cdot y \equiv y \cdot x
    \and
    x \cdot (y \cdot z) \equiv (x \cdot y) \cdot z
    \and \\
    \quotep{0} \cdot P \equiv P
    \and \\
    x \cdot (y \cdot P) \equiv (x \cdot y) \cdot P
    \and \\
    x \cdot (P|Q) \equiv (x \cdot P) | (x \cdot Q)
    \and \\    
  \end{mathpar}
\end{remark}

\subsubsection{Tensor product}

We define a tensor product on processes by structural induction.

\paragraph{Tensor of sums} First note that all summations, including
$\pzero$ and sequence, can be written $\Sigma_{i} x_{i}.A_{i} +
\Sigma_{j} x_{j}.C_{j}$, where we have grouped input-guarded processes
together and output-guarded processes together.

Thus, we can define the tensor product of two summations, $N_{1}\otimes N_{2}$, where

\begin{mathpar}
  N_{1} := \Sigma_{i} x_{i}.A_{i} + \Sigma_{j} x_{j}.C_{j}
  \and
  N_{2} := \Sigma_{i'} y_{i'}.B_{i'} + \Sigma_{j'} y_{j'}.D_{j'} 
\end{mathpar}

as follows.

\begin{mathpar}
  \Sigma_{i} x_{i}.A_{i} + \Sigma_{j} x_{j}.C_{j} \otimes \Sigma_{i'}
  y_{i'}.B_{i'} + \Sigma_{j'} y_{j'}.D_{j'} 
  \and \\
  := \; \Sigma_{i} \Sigma_{i'} \quotep{\stackrel{\vee}{x_{i}}| \stackrel{\vee}{y_{i'}}}.(A_{i}\otimes B_{i'}) \; | \; \Sigma_{i'} \Sigma_{i} \quotep{\stackrel{\vee}{y_{i'}}|\stackrel{\vee}{x_{i}}}.(B_{i'}\otimes A_{i})
  \and
  \;\; | \;\; \Sigma_{j} \Sigma_{j'} \quotep{\stackrel{\vee}{x_{j}}|\stackrel{\vee}{y_{j'}}}.(A_{j}\otimes B_{j'}) \; | \; \Sigma_{j'} \Sigma_{j} \quotep{\stackrel{\vee}{y_{j'}}|\stackrel{\vee}{x_{j}}}.(B_{j'}\otimes A_{j})
\end{mathpar}

\begin{remark}
  Do we need to $x^{L}$ and $y^{R}$ for this construction as well?
\end{remark}

\paragraph{Tensor of parallel compositions} Next, we distribute tensor
over par.

\begin{mathpar}
  P_{1}|P_{2} \otimes Q_{1}|Q_{2} := (P_{1} \otimes Q_{1}) | (P_{1}
  \otimes Q_{2}) | (P_{2} \otimes Q_{1}) | (P_{2} \otimes Q_{2})
\end{mathpar}

\paragraph{Tensor with dropped names} We treat tensor of a
process with a dropped name as parallel composition.

\begin{mathpar}
  P \otimes \dropn{x} := P | \dropn{x}
\end{mathpar}

\paragraph{Tensor of agents}

Finally, we need to define tensor on agents. Note that the definition
of tensor on normal products only tensors inputs with inputs and
outputs with outputs. Thus, we only have to define the operation on
``homogeneous'' pairings.

\begin{mathpar}
  (\vec{x})P \otimes (\vec{y})Q
  \and \\
  := (x_{0}^{L}|y_{0}^{R},\ldots,x_{0}^{L}|y_{n}^{R},\ldots,x_{m}^{L}|y_{0}^{R},\ldots,x_{m}^{L}|y_{n}^R)(P\{ \vec{x}^{L}/\vec{x}\} \otimes Q \{ \vec{y}^{R}/\vec{y}\})
  \and \\
  \clift{\vec{P}} \otimes \clift{\vec{Q}}
  \and \\
  := \clift{P_{0}\otimes Q_{0},\ldots,P_{0}\otimes Q_{n},\ldots,P_{m}\otimes Q_{0},\ldots,P_{m}\otimes Q_{n}}
\end{mathpar}

\begin{remark}
  Observe that arities of tensored abstractions matches arities of
  tensored concretions if the original arities matched. Note also that
  the length of the arities corresponds to the increase in dimension
  we see in ordinary vector space tensor product.
\end{remark}

\begin{remark}
  Operationally, this definition distributes the tensor down to
  components ``linked'' by summation. Tensor over summation is
  intriguing in that it mixes names. Moreover, as a consequence of the
  way it mixes names we have the identities for all $x \in \QProc$ and
  $P,Q \in \Proc$

  \begin{mathpar}
    (x \cdot P) \otimes Q \equiv x \cdot (P \otimes Q) \equiv P \otimes (x \cdot Q)
    \and
    P \otimes \pzero \equiv P
  \end{mathpar}

  that the reader is invited to verify.
\end{remark}

\subsubsection{Annihilation}
\begin{mathpar}
  P^{\perp} := \{ Q | \forall R. P|Q \red^{*} R \Rightarrow R \red^{*} \pzero \}
  \and \\
  P^{\underline{\perp}} := \Sigma_{Q \in P^{\perp}} \quotep{Q}?(y).(\dropn{y}|Q) | \Sigma_{Q \in P^{\perp}} \quotep{Q}\clift{\Box}
\end{mathpar}

\paragraph{Discussion} The reader will note that $P^{\perp}$ is a
\emph{set} of processes, while $P^{\underline{\perp}}$ is a
\emph{context}. We call the set $P^{\perp}$ the \emph{annihilators} of
$P$. The parallel composition of a process in the annihilators of $P$
with $P$ will result in a process, the state space of which has all
paths eventually leading to $\pzero$. Execution may endure loops; but
under reasonable conditions of fairness (naturally guaranteed under
most notions of bisimulation) such a composite process cannot get
stuck in such a loop and will, eventually pop out and terminate.

The context $P^{\underline{\perp}}$ is ready and willing to ``take the
$P$ out of'' the process to which it is applied. It will effectively
transmit the code of the process to which it is applied to one of the
annihilators and run the process against it.

\subsubsection{Evaluation}
We fix $M$ a domain of fully abstract interpretation with an equality
coincident with bisimulation. We take $\meaningof{\cdot} : \Proc \to
M$ to be the map interpreting processes and $\nmeaningof{\cdot} : \M
\to Proc$ to be the map running the other way. Then we define

\begin{mathpar}
  \int P := \nmeaningof{\meaningof{P}}
\end{mathpar}

\paragraph{Discussion}
There are many fully abstract interpretations of Milner's
$\pi$-calculus. Any of them can be used as a basis for interpreting
the reflective calculus here. Equipped with such a domain it is
largely a matter of grinding through to check that the Yoneda
construction for the normalization-by-evaluation program can be
extended to this setting.

\begin{remark}
  The reader is invited to verify that $\int (P^{\underline{\perp}}[P]) = 0$.
\end{remark}

\subsection{Quantum mechanics}

Table \ref{tbl:core_qm_op_defns} gives the core operational definitions

\begin{table}[htp]\label{tbl:core_qm_op_defns}
  \center{
    \fbox{
      \begin{tabular}{c|c}
        quantum mechanics & process calculus \\
        \hline
        scalar & $x := \quotep{P}$ \\
        state vector & $\state{P} := P$ \\
        dual & $\state{P}^{*} := \event{P^{\underline{\perp}}} := \quotep{P^{\underline{\perp}}}[-]$ \\
        matrix & $ \Sigma_{\alpha} \state{P_{\alpha}}x_{\alpha}\event{Q_{\alpha}}$ \\
        vector addition & $\state{P} + \state{Q} := \state{P | Q}$ \\
        tensor product & $\state{P} \otimes \state{Q} := \state{P \otimes Q}$ \\
        inner product & $\innerprod{P}{Q} := \quotep{\int P^{\underline{\perp}}[Q]}$ \\
      \end{tabular}
    }
  }
  \caption{QM - operational definitions}
\end{table}

where

\begin{mathpar}
  \prmatrix{P}{Q} := \fprmatrix{P}{\quotep{\pzero}}{Q}
  \and
  \fprmatrix{P}{x}{Q} := (\state{P},x,\event{Q})
  \and
  (\fprmatrix{P}{x}{Q})(\state{R}) := x \cdot \innerprod{Q}{R} \cdot \state{P}
  \and
  (\fprmatrix{P}{x}{Q})(\event{R}) := x \cdot \innerprod{R}{P} \cdot \event{Q}
\end{mathpar}

\paragraph{Discussion}
As promised: vectors (aka states) are represented as processes; duals
as contextual duals; inner product definition should be compared with
standard inner product definition for ....

\begin{remark}
  Assuming $\int (P^{\underline{\perp}}[P]) = 0$, the reader is
  invited to verify that $(\fprmatrix{P}{x}{P})(\state{P}) = x \cdot \state{P}$.
\end{remark}

\begin{remark}
  The reader is invited to verify that $\innerprod{P}{Q}$ could
  equally well have been written $\quotep{\int \stackrel{\vee}{x}}$
  where $x = \event{P^{\underline{\perp}}}(Q)$.

  One of the motivations for this remark is that there is another way
  to factor these operations. We could package up evaluation in the dual:

  \begin{mathpar}
    \state{P}^{*} := \event{\int P^{\underline{\perp}}} := \quotep{\int P^{\underline{\perp}}}[-]
  \end{mathpar}

  and then have inner product defined by
  
  \begin{mathpar}
    \innerprod{P}{Q} := \event{P}(Q)
  \end{mathpar}

  Hopefully, experience with the calculations will provide guidance on
  the best factoring.
\end{remark}

\begin{remark}
  Assuming $\int (P^{\underline{\perp}}[P]) = 0$, the reader is
  invited to verify that $\forall P,Q. (\prmatrix{0}{Q})(\state{0}) =
  \state{0}$ and dually $(\prmatrix{P}{0})(\event{0}) = \event{0}$.
\end{remark}

\begin{remark}
  i'm a little worried that i don't (yet) have proper support for
  complex conjugacy. But, the observation above may give us a
  clue. According to Abramsky, it must be the case that the scalars
  are iso to the homset of the identity for the tensor -- which the
  observation above characterizes. 

  For now, we will simply bookmark the notion with $\overline{x}$.
\end{remark}

\subsubsection{Adjointness}

We need to give a definition of $(\cdot)^{\dagger}$ for matrices. The
obvious candidate definition is
\begin{mathpar}
(\Sigma_{\alpha}\fprmatrix{P_{\alpha}}{x_{\alpha}}{Q_{\alpha}})^{\dagger}
= \Sigma_{\alpha}\fprmatrix{(Q_{\alpha}^{\underline{\perp}})^{*}}{\overline{x}_{\alpha}}{P_{\alpha}^{\underline{\perp}}} 
\end{mathpar}

But, $(Q_{\alpha}^{\underline{\perp}})^{*}$ requires a name along
which to communicate the process to achieve the context application.

\subsubsection{Basis for a basis}
If processes label states and ``addition'' of states (a.k.a. vector
addition) is interpreted as parallel composition, what corresponds to
notions of linear independence and basis? Here, we recall that Yoshida
has developed a set of \emph{combinators} for an asynchronous verison
of Milner's $\pi$-calculus. These are a finite set of processes such
any process can be expressed as parallel composition of these
combinators together with liberal uses of the new operator and
replication. We can simply give a translation of these into the
present calculus and have reasonable expectation that the property
carries over. That is, that the resultant set allows to express all
processes via parallel composition. Note, however, that there is no
new operator or replication in this calculus. As a result, we expect
that the corresponding set is actually infinite. That is, we expect
that the space is actually infinite dimensional.

\begin{remark}
  The attentive reader may be a bit concerned. Certainly, the
  collection $S$, $K$ and $I$ is a finite set of
  combinators. Shouldn't we expect to see a finite set of combinators
  for an effectively equivalent system? i am very sympathetic to this
  critique and feel it warrants full attention. On the other hand, i
  also have in mind the following analogy. The natural numbers, as a
  monoid under addition, has exactly $1$ generator, while the natural
  numbers, as a monoid under multiplication, has countably many
  generators (the primes). We observe that the application of the
  lambda calculus is much less resource sensitive than the parallel
  composition of the $\pi$-calculus. Could it be the case that we have
  an analogy of the form
  
  \begin{mathpar}
    m + n : MN :: m*n : M|N
  \end{mathpar}

  giving a similar blow up in the set of ``primes''?  This is such a
  wonderful thought that, even if it's not true, i think it's worth
  writing down.
\end{remark}
 

\documentclass[12pt]{llncs}
%\documentclass{jktr}

\usepackage[pdftex]{hyperref}                   
\usepackage {listings}
\usepackage {mathpartir}
\usepackage{bcprules}
%\usepackage{listings}
                       
\usepackage{graphicx} 
%\usepackage[margins=2.5cm,nohead,nofoot]{geometry}
%\usepackage{geometry}
\usepackage{amsfonts}
\usepackage{amstext}
\usepackage{latexsym}
\usepackage{amssymb}
\usepackage{color}


%\include{myPreamble}
\include{qm2pi.local} 

%\ifpdf
%\usepackage[pdftex]{graphicx}
%\else
%\usepackage{graphicx}
%\fi

 % \ifpdf
%  \usepackage{pdfsync}
%  \if


%\title{Brief Article}
%\author{David F. Snyder}
%\author{L.G. Meredith}

%\address{Dept. of Math., Texas State University--San Marcos, San Marcos, TX 78666}
       
\pagestyle{empty}


\begin{document}

\lstset{language=[Objective]Caml,frame=shadowbox}

\input{qm2pi.front}

% section front matter (end)

\input{qm2pi.intro} 
 
% section introduction (end)

% \input{qm2pi.knotations} 

% section notation (end)

\input{qm2pi.process.calculi} 

% section concurrent_process_calculi_and_spatial_logics_ (end)
    
%\input{qm2pi.knots2pi} 

%\input{qm2pi.trefoil} 

%\input{qm2pi.mainthm} 

% subsection basic_interpretation (end)

%\input{qm2pi.rho.presentation} 
\subsection{The syntax and semantics of the notation system}\label{sub:the_syntax_and_semantics_of_the_notation_system} % (fold)

We now summarize a technical presentation of the calculus that
embodies our theory of dynamics. The typical presentation of such a
calculus follows the style of giving generators and relations on
them. The grammar, below, describing term constructors, freely
generates the set of processes, $\Proc$. This set is then quotiented
by a relation known as structural congruence and it is over this set
that the notion of dynamics is expressed. This presentation is
essentially that of \cite{MeredithR05} with the addition of
polyadicity and summation. For readability we have relegated some of
the technical subtleties to an appendix.

\subsubsection{Process grammar}\label{subsub:process_grammar}

\begin{mathpar}
  \inferrule* [lab=synchronization] {} {{M} \bc \pzero \;|\; x?F \;|\; x!C }
  \and
  \inferrule* [lab=abstraction] {} {{F} \bc (x)P}
  \and
  \inferrule* [lab=concretion] {} {{C} \bc \langle Q \rangle}
  \and
  \inferrule* [lab=process] {} {{P,Q} \bc M \;| \;P|Q \;|\; @{x}}
  \and
  \inferrule* [lab=name] {} {{x} \bc \quotep{P}}
\end{mathpar} 

Note that $\vec{x}$ (resp. $\vec{P}$) denotes a vector of names
(resp. processes) of length $|\vec{x}|$ (resp. $|\vec{P}|$). We adopt
the following useful abbreviations.

\begin{mathpar}
   x?(\vec{y}).P := x.(\vec{y})P \and  x\clift{\vec{P}} := x.\clift{\vec{P}}
   \and x!(y) := \lift{x}{\dropn{y}}
   \and \Pi_{i=0}^{n-1}P_i := P_0 | \ldots | P_{n-1}
\end{mathpar}

\subsubsection{Structural congruence}

\paragraph{Free and bound names and alpha-equivalence.} At the
core of structural equivalence is alpha-equivalence which identifies
process that are the same up to a change of variable. Formally, we
recognize the distinction between free and bound names. The free names
of a process, $\freenames{P}$, may be calculated recursively as
follows:

\begin{mathpar}
\freenames{\pzero} := \emptyset
  \and \\
  \freenames{x?(y).P} := \{ x \} \cup (\freenames{P} \setminus \{ y \})
  \and 
  \freenames{x!\langle P \rangle} := \{ x \} \cup \{ P \} 
  \and \\
  \freenames{P|Q} := \freenames{P} \cup \freenames{Q}
  \and \\
  \freenames{@{x}} := \{ x \}
\end{mathpar}

$\pi$
$\quotep{\pi}$

$\freenames{-} : \pi \to \mathcal{P}(\quotep{\pi})$

\begin{eqnarray*}
  \freenames{\pzero} & := & \emptyset \\
  \freenames{x?(y).P} & := & \{ x \} \cup (\freenames{P} \setminus \{ y \}) \\
  \freenames{x!\langle P \rangle} & := & \{ x \} \cup \{ P \} \\
  \freenames{P|Q} & := & \freenames{P} \cup \freenames{Q} \\
  \freenames{\dropn{x}} & := & \{ x \}
\end{eqnarray*}

The bound names of a process, $\boundnames{P}$, are those names occurring in $P$
that are not free. For example, in $x?(y).0$, the name $x$ is free, while $y$ is bound.

\begin{mathpar}
  \inferrule* [lab=monoidal-laws] {} { P|Q \equiv Q|P \and P|0 \equiv P \and P|(Q|R) \equiv (P|Q)|R }
\end{mathpar}

\begin{mathpar}
  \inferrule* [lab=alpha-equivalence] {} { (x)P \equiv (y)P\{y/x\} \and y \not\in \freenames{P} }
\end{mathpar}

\begin{definition}
Then two processes, $P,Q$, are alpha-equivalent if $P = Q\{\vec{y}/\vec{x}\}$ for
some $\vec{x} \in \boundnames{Q},\vec{y} \in \boundnames{P}$, where $Q\{\vec{y}/\vec{x}\}$
denotes the capture-avoiding substitution of $\vec{y}$ for $\vec{x}$ in $Q$.
\end{definition}

\begin{definition}
  The {\em structural congruence} \cite{SangiorgiWalker} , $\equiv$,
  between processes is the least congruence containing
  alpha-equivalence, satisfying the abelian monoid laws
  (associativity, commutativity and $\pzero$ as identity) for parallel
  composition $|$ and for summation $+$.
\end{definition}

\subsection{Name equivalence}

We take name equivalence, written $\nameeq$, to be the smallest
equivalence relation generated by the following rules.

\begin{mathpar}
\inferrule*[lab=Quote-drop]
{ }
{ \quotep{@{x}} \nameeq x }

\inferrule*[lab=Struct-equiv]
{ P \scong Q }
{ \quotep{P} \nameeq \quotep{Q} }
\end{mathpar}

The astute reader will have noticed that the mutual recursion of names
and processes imposes a mutual recursion on alpha-equivalence and
structural equivalence via name-equivalence. Fortunately, all of this
works out pleasantly and we may calculate in the natural way, free of
concern. The reader interested in the details is referred to the
appendix \ref{appendix:rho_details}.

\subsection{Substitution}

We use $\Proc$ for the set of processes, $\QProc$ for the set of
names, and $\id{\{}\vec{y} / \vec{x} \id{\}}$ to denote partial maps,
$s : \QProc \rightarrow \QProc$. A map, $s$ lifts, uniquely, to a map
on process terms, $\widehat{s} : \Proc \rightarrow \Proc$ by the
following equations.

\begin{mathpar}
  (0) \psubstp{Q}{P} := 0 \\
  (R \juxtap S) \psubstp{Q}{P}
  :=    
  (R)\psubstp{Q}{P} \juxtap (S) \psubstp{Q}{P} \\
  (x?(y).R) \psubstp{Q}{P}    
  :=    
  (x)\substp{Q}{P} (z)\concat( (R \psubstn{z}{y}) \psubstp{Q}{P} ) \\
  (\lift{x}{R}) \psubstp{Q}{P}  
  :=
  \lift{(x)\substp{Q}{P}}{ R \psubstp{Q}{P} } \\
%   (\dropn{x})  \psubstp{Q}{P}       
%   := 
%   \left\{ 
%     \begin{array}{ccc} 
%       \dropn{\quotep{Q}} & & x \nameeq \quotep{P} \\
%       \dropn{x} & & otherwise \\
%     \end{array}
%   \right. 
  (\dropn{x})  \psubstp{Q}{P}       
  := 
  \left\{ 
    \begin{array}{ccc} 
      Q & & x \nameeq \quotep{P} \\
      \dropn{x} & & otherwise \\
    \end{array}
  \right.
\end{mathpar}
 

where

\begin{eqnarray}
  (x)\id{\{} \lpquote Q \rpquote / \lpquote P \rpquote \id{\}}            = 
  \left\{ 
    \begin{array}{ccc}
      \lpquote Q \rpquote & & x \nameeq \lpquote P \rpquote \\
      x & & otherwise \\
    \end{array}
  \right. \nonumber
\end{eqnarray}

and $z$ is chosen distinct from $\quotep{P}$, $\quotep{Q}$, the free
names in $Q$, and all the names in $R$. Our $\alpha$-equivalence will
be built in the standard way from this substitution.

\begin{remark}\label{rem:no_self_referential_names}
  One consequence of these definitions is that $\forall P. \quotep{P}
  \not\in \freenames{P}$.
\end{remark}

\subsection{ Dynamic quote: an example }

Anticipating something of what's to come, consider applying the
substitution, $\widehat{\id{\{}u / z \id{\}}}$, to the following pair
of processes, $\lift{w}{y!(z)}$ and $w[ \lpquote y!(z) \rpquote ]$.

\begin{eqnarray}
	\lift{w}{y!(z)}\widehat{\id{\{}u / z \id{\}}}
		& = &
		\lift{w}{y!(u)} \nonumber\\
	w[ \lpquote y!(z) \rpquote ] \widehat{ \id{\{}u / z \id{\}} }
		& = &
		w[ \lpquote y!(z) \rpquote ] \nonumber
\end{eqnarray}

Because the body of the process between quotes is impervious to
substitution, we get radically different answers. In fact, by
examining the first process in an input context,
e.g. $x?(z).\lift{w}{y!(z)}$, we see that the process under the lift
operator may be shaped by prefixed inputs binding a name inside it. In
this sense, the lift operator will be seen as a way to dynamically
construct processes before reifying them as names.

Finally equipped with these standard features we can present the
dynamics of the calculus.

\subsubsection{Operational semantics} 

Finally, we introduce the computational dynamics. What marks these
algebras as distinct from other more traditionally studied algebraic
structures, e.g. vector spaces or polynomial rings, is the manner in
which dynamics is captured. In traditional structures, dynamics is typically
expressed through morphisms between such structures, as in linear maps
between vector spaces or morphisms between rings. In algebras
associated with the semantics of computation, the dynamics is
expressed as part of the algebraic structure itself, through a
reduction reduction relation typically denoted by $\red$. Below, we
give a recursive presentation of this relation for the calculus used
in the encoding.

$\red \subseteq \pi \times \pi$
$\red : \pi \to \mathcal{P}(\pi)$

\begin{mathpar}
  \inferrule* [lab=Comm] { \textsf{match}( x_{src}, x_{trgt} ) } { x_{trgt}?(y)P \; | \; x_{src}!\langle {Q} \rangle \red P\{\quotep{Q}/y}\} }
  \and \\
  \inferrule* [lab=Par] {{P} \red {P}'} {{{P} | {Q}} \red {{P}' | {Q}}}
  \and
  \inferrule* [lab=Equiv]{{{P} \scong {P}'} \andalso {{P}' \red {Q}'} \andalso {{Q}' \scong {Q}}}{{P} \red {Q}}
\end{mathpar}

\begin{eqnarray*}
  match_{\equiv} (\quotep{P},\quotep{Q}) & := & P \equiv Q \\
  match_{\dagger}(\quotep{P},\quotep{Q}) & := & \forall R. P|Q \red^{*} R => R \red^{*} 0 \\
  match_{K}(\quotep{P},\quotep{Q}) & := & K \mbox{ for some context } K
\end{eqnarray*}

$u?(x)P | u!\langle Q \rangle \red P\{\quotep{Q}/x\}$

%We write $\wred$ for $\red^*$, and $P\red$ if $\exists Q $ such that $ P \red Q$.
We write $P\red$ if $\exists Q $ such that $ P \red Q$ and $P\not\red$, otherwise.

\section{Replication}

As mentioned before, it is known that replication (and hence
recursion) can be implemented in a higher-order process algebra
\cite{SangiorgiWalker}. As our first example of calculation with the
machinery thus far presented we give the construction explicitly in
the {\rhoc}.

\begin{eqnarray}
	D_{x} & := & \prefix{x}{y}{(\binpar{\outputp{x}{y}}{@{y}})} \nonumber\\
	\bangp_{x}{P} & := & \binpar{{x}!\langle{\binpar{D_{x}}{P}}\rangle}{D_{x}} \nonumber
\end{eqnarray}

\begin{eqnarray}
	\bangp_{x}{P} & & \nonumber\\
	=
	& {x}!\langle{(\prefix{x}{y}{(\outputp{x}{y} | @{y})) | P}}\rangle 
	      | \prefix{x}{y}{(\outputp{x}{y} | @{y})} & \nonumber\\
	\red
	& (\outputp{x}{y} | @{y})\substn{\quotep{(\prefix{x}{y}{(@{y} | \outputp{x}{y})) | P}}}{y} & \nonumber\\
	=
	& \outputp{x}{\quotep{(\prefix{x}{y}{(\outputp{x}{y} | @{y})) | P}}}
	  | {(\prefix{x}{y}{(\outputp{x}{y} | @{y})) | P}} & \nonumber\\
	\red
	& \ldots & \nonumber\\
	\red^*
	& P | P | \ldots & \nonumber
\end{eqnarray}

Of course, this encoding, as an implementation, runs away, unfolding
$\bangp{P}$ eagerly. A lazier and more implementable replication
operator, restricted to input-guarded processes, may be obtained as follows.

\begin{eqnarray}
\bangp{\prefix{u}{v}{P}} 
	:= 
	\binpar{\lift{x}{\prefix{u}{v}{(\binpar{D(x)}{P})}}}{D(x)} \nonumber
\end{eqnarray}

\begin{remark}
  Note that the lazier definition still does not deal with summation
  or mixed summation (i.e. sums over input and output). The reader is
  invited to construct definitions of replication that deal with these
  features. 

  Further, the definitions are parameterized in a name, $x$. Can you,
  gentle reader, make a definition that eliminates this parameter and
  guarantees no accidental interaction between the replication
  machinery and the process being replicated -- i.e. no accidental
  sharing of names used by the process to get its work done and the
  name(s) used by the replication to effect copying. This latter
  revision of the definition of replication is crucial to obtaining
  the expected identity $!!P \sim !P$.
\end{remark}

\begin{remark}\label{rem:paradoxical_combinator}
  The reader familiar with the lambda calculus will have noticed the
  similarity between $D$ and the paradoxical combinator.

  [Ed. note: the existence of this seems to suggest we have to be more
  restrictive on the set of processes and names we admit if we are to
  support no-cloning.]
\end{remark}

\subsubsection{Bisimulation}

The computational dynamics gives rise to another kind of equivalence,
the equivalence of computational behavior. As previously mentioned
this is typically captured \emph{via} some form of bisimulation.

% The notion we use in this paper is weak barbed bisimulation
% \cite{milner91polyadicpi}.

The notion we use in this paper is derived from weak barbed
bisimulation \cite{milner91polyadicpi}. 

\begin{definition}
An \emph{observation relation}, $\downarrow_{\mathcal N}$, over a set
of names, $\mathcal N$, is the smallest relation satisfying the rules
below.

\infrule[Out-barb]{y \in {\mathcal N}, \; x \nameeq y}
		  {\outputp{x}{v} \downarrow_{\mathcal N} x}
\infrule[Par-barb]{\mbox{$P\downarrow_{\mathcal N} x$ or $Q\downarrow_{\mathcal N} x$}}
		  {\binpar{P}{Q} \downarrow_{\mathcal N} x}

We write $P \Downarrow_{\mathcal N} x$ if there is $Q$ such that 
$P \wred Q$ and $Q \downarrow_{\mathcal N} x$.
\end{definition}

\begin{definition}
%\label{def.bbisim}
An  ${\mathcal N}$-\emph{barbed bisimulation} over a set of names, ${\mathcal N}$, is a symmetric binary relation 
${\mathcal S}_{\mathcal N}$ between agents such that $P\rel{S}_{\mathcal N}Q$ implies:
\begin{enumerate}
\item If $P \red P'$ then $Q \wred Q'$ and $P'\rel{S}_{\mathcal N} Q'$.
\item If $P\downarrow_{\mathcal N} x$, then $Q\Downarrow_{\mathcal N} x$.
\end{enumerate}
$P$ is ${\mathcal N}$-barbed bisimilar to $Q$, written
$P \wbbisim_{\mathcal N} Q$, if $P \rel{S}_{\mathcal N} Q$ for some ${\mathcal N}$-barbed bisimulation ${\mathcal S}_{\mathcal N}$.
\end{definition}

$\mathcal{R} \subseteq \pi \times \pi$

$P \mathcal{R} Q => \forall P'. P \red P' \Rightarrow \exists Q'. Q \red Q', P' \mathcal{R} Q'$

$P \vdash x \Rightarrow Q \vdash x$

\begin{mathpar}
  \inferrule*[lab=Out-barb]{x \nameeq y}{{y}!\langle{Q}\rangle \vdash x}
  \and
  \inferrule*[lab=Par-barb]{\mbox{$P\vdash x$ or $Q\vdash x$}}{\binpar{P}{Q} \vdash x}
\end{mathpar}

\subsubsection{Contexts}

One of the principle advantages of computational calculi like the
$\pi$-calculus is a well-defined notion of context,
contextual-equivalence and a correlation between
contextual-equivalence and notions of bisimulation. The notion of
context allows the decomposition of a process into (sub-)process and
its syntactic environment, its context. Thus, a context may be
thought of as a process with a ``hole'' (written $\Box$) in it. The
application of a context $M$ to a process $P$, written $M[P]$, is
tantamount to filling the hole in $M$ with $P$. In this paper we do
not need the full weight of this theory, but do make use of the notion
of context in the proof the main theorem. 

\begin{mathpar}
  \inferrule* [lab=summation] {} {{M_{M},M_{N}} \bc \Box \;|\; x.M_{A} \;|\; M_{M}+M_{N}}
  \and
  \inferrule* [lab=agent] {} {{M_{A}} \bc (\vec{x})M_{P} \;| \; \clift{P_0,\ldots,M_{P},\ldots,P_N}}
  \and \\
  \inferrule* [lab=process] {} {{M_{P}} \bc M_{N} \;| \;P|M_{P} }
\end{mathpar} 

\begin{mathpar}
  \inferrule* [lab=sychronization] {} {M_{N} \bc \Box \;|\; x?M_{F} \;|\; x!M_{C}}
  \and
  \inferrule* [lab=abstraction] {} {{M_{F}} \bc (x)M_{P} }
  \and
  \inferrule* [lab=concretion] {} {{M_{C}} \bc \langle M_{P} \rangle }
  \and \\
  \inferrule* [lab=process] {} {{M_{P}} \bc M_{N} \;| \;P|M_{P} }
\end{mathpar}

\begin{definition}[contextual application] Given a context $M$, and
  process $P$, we define the \emph{contextual application}, $M[P] :=
  M\{P/\Box\}$. That is, the contextual application of M to P is the
  substitution of $P$ for $\Box$ in $M$.
\end{definition}

$\meaningof{-} : L \to \mathcal{P}(\pi)$

\begin{mathpar}
  \inferrule* [lab=collection] {} {\meaningof{true} = \pi, \and \meaningof{~E} = \pi \setminus \meaningof{E}, \and \meaningof{E_{1} \& E_{2}} = \meaningof{E_{1}} \cap \meaningof{E_{2}}}
\end{mathpar}

\begin{mathpar}
  \inferrule* [lab=structure] {} {\meaningof{0} = \{ P \in \pi | P \equiv 0 \}, \and \\ \meaningof{E_1 | E_2} = \{ P \in \pi | P \equiv P_{1} | P_{2}, P_{1} \in \meaningof{E_{1}}, P_{2} \in \meaningof{E_2}\} }
\end{mathpar}

\begin{mathpar}
 \inferrule* [lab=behavior] {} {\meaningof{\langle a?b \rangle E} = \{ P \in \pi | P \equiv Q | u?(y)P', \\ \and \\\\ \and \\ \;\;\; u \in \meaningof{a}, \forall z.P'\{z/y\} \in \meaningof{E\{z/b\}}\}, \and \\ \meaningof{a!E} = \{ P \in \pi | P \equiv Q | x!\langle P' \rangle, x \in \meaningof{a} P' \in \meaningof{E}\} }
\end{mathpar}

\begin{mathpar}
 \inferrule* [lab=nominal] {} {\meaningof{\quotep{E}} = \{ \quotep{P} \in \quotep{\pi} | P \in \meaningof{E} \}, \and \meaningof{\quotep{P}} = \{ \quotep{Q} \in \quotep{\pi} | P \equiv Q \} \and \\ \meaningof{@\quotep{E}} = \{ P \in \pi | P \equiv @x, x \in \meaningof{E} \}}
\end{mathpar}

\begin{eqnarray*}
  \\
  \meaningof{-} : TS \to ST
\end{eqnarray*}

\begin{eqnarray*}
  \\
  L : TS \to ST
\end{eqnarray*}

\begin{eqnarray*}
  \\
  P \models E \iff P \in \meaningof{E}
\end{eqnarray*}

\begin{eqnarray*}
  P \approx_{L} Q \iff \forall E \in L. P \models E \iff Q \models E
\end{eqnarray*}

\begin{eqnarray*}
  P \approx_{K} Q
\end{eqnarray*}

\begin{eqnarray*}
  P \approx Q
\end{eqnarray*}

$\approx_{K} = \approx = \approx_{L}$

\subsubsection{Contextual duality}

Note that contexts extend the quotation operation to a family of
operations from processes to names. Given a context, $M$, we can
define a \emph{nominal context}, $\quotep{M}$ by $\quotep{M}[P] :=
\quotep{M[P]}$. To foreshadow what is to come we observe that these
operations enjoy a duality with processes very much like the duality
between vectors and maps from vectors to scalars.

Further, because the calculus is essentially higher-order, we have a
correspondence between contexts and processes. More specifically,
given a name $x$ and a context $M$ we can construct $M^{*}_{x}$ such
that 

\begin{mathpar}
  M^{*}_{x} | \lift{x}{P} \red M[P]
\end{mathpar}

namely,

\begin{mathpar}
  M^{*}_{x} := x?(u).M[\dropn{u}]
\end{mathpar}

The dependence of $M^{*}_{x}$ on a name makes it an abstraction, 

\begin{mathpar}
  M^{*} := (x)x?(u).M[\dropn{u}]
\end{mathpar}

\subsection{Additional notation}

It will sometimes be convenient to denote the process a name
quotes. We already have the notation $x = \quotep{P}$, but it will be
convenient to introduce an alternate notation, $\procn{x}$, when we
want to emphasize the connection to the use of the name. Note that, by
virtue of name equivalence, $\quotep{\procn{x}} \nameeq x$; so, the
notation is consistent with previous definitions.

Further, because names have structure it is possible to effect
substitutions on the basis of that structure. This means we need to
upgrade our notation for substitutions, which we accomplish by
adapting comprehension notation. Thus,

\begin{mathpar}
  P\{ y / x : x \in S \}
\end{mathpar}

is interpreted to mean the process derived from P by replacing (in a
capture-avoiding manner) each occurrence of $x$ in $S$ by $y$. For example,

\begin{mathpar}
  P\{ \quotep{\procn{x}|\procn{x}} / x : x \in \freenames{P} \}
\end{mathpar}

will replace each (occurrence) of a free name $x$ in $P$ by
$\quotep{\procn{x}|\procn{x}}$.

Also, we will avail ourselves of the notation $x^{L}$ and $x^{R}$ to
denote injections of a name into disjoint copies of the name
space. There are numerous ways to accomplish this. One example can be
found in \cite{MeredithR05}. This notation overloads to vectors of
names: $\vec{x}^{\pi} := (x_{i}^{\pi} \; : \; 0 \leq i < |\vec{x}| )$ where $\pi \in \{L,R\}$.

We also use $P^{\Box} := P|\Box$.

In \cite{MeredithR05} an interpretation of the new operator is
given. It turns out that there are several possible interpretations
all enjoying the requisite algebraic properties of the operator (see
\cite{milner91polyadicpi}). We will therefore make liberal use of
$(\nu\; \vec{x})P$.

% subsection the_syntax_and_semantics_of_the_notation_system (end)   

\input{qm2pi.qmops} 

\input{qm2pi.sterngerlach} 

\input{qm2pi.metric} 

% section concurrent_process_calculi (end)

%\input{qm2pi.proofsketch}

% section proof sketch (end)

%\input{qm2pi.slviaknots} 

% section spatial logic via knots (end)

\input{qm2pi.conclusion}

% section conclusion (end)

%\input{qm2pi.dtcodes} 

% section wiring algorithm (end)

\input{qm2pi.ack} 

% section acknowledgments (end)

\newpage


\bibliographystyle{plain}   
\bibliography{../../biblios/main.bib}

\input{qm2pi.rhodetails}

\end{document}

 

\documentclass[12pt]{llncs}
%\documentclass{jktr}

\usepackage[pdftex]{hyperref}                   
\usepackage {listings}
\usepackage {mathpartir}
\usepackage{bcprules}
%\usepackage{listings}
                       
\usepackage{graphicx} 
%\usepackage[margins=2.5cm,nohead,nofoot]{geometry}
%\usepackage{geometry}
\usepackage{amsfonts}
\usepackage{amstext}
\usepackage{latexsym}
\usepackage{amssymb}
\usepackage{color}


%\include{myPreamble}
\include{qm2pi.local} 

%\ifpdf
%\usepackage[pdftex]{graphicx}
%\else
%\usepackage{graphicx}
%\fi

 % \ifpdf
%  \usepackage{pdfsync}
%  \if


%\title{Brief Article}
%\author{David F. Snyder}
%\author{L.G. Meredith}

%\address{Dept. of Math., Texas State University--San Marcos, San Marcos, TX 78666}
       
\pagestyle{empty}


\begin{document}

\lstset{language=[Objective]Caml,frame=shadowbox}

\input{qm2pi.front}

% section front matter (end)

\input{qm2pi.intro} 
 
% section introduction (end)

% \input{qm2pi.knotations} 

% section notation (end)

\input{qm2pi.process.calculi} 

% section concurrent_process_calculi_and_spatial_logics_ (end)
    
%\input{qm2pi.knots2pi} 

%\input{qm2pi.trefoil} 

%\input{qm2pi.mainthm} 

% subsection basic_interpretation (end)

%\input{qm2pi.rho.presentation} 
\subsection{The syntax and semantics of the notation system}\label{sub:the_syntax_and_semantics_of_the_notation_system} % (fold)

We now summarize a technical presentation of the calculus that
embodies our theory of dynamics. The typical presentation of such a
calculus follows the style of giving generators and relations on
them. The grammar, below, describing term constructors, freely
generates the set of processes, $\Proc$. This set is then quotiented
by a relation known as structural congruence and it is over this set
that the notion of dynamics is expressed. This presentation is
essentially that of \cite{MeredithR05} with the addition of
polyadicity and summation. For readability we have relegated some of
the technical subtleties to an appendix.

\subsubsection{Process grammar}\label{subsub:process_grammar}

\begin{mathpar}
  \inferrule* [lab=synchronization] {} {{M} \bc \pzero \;|\; x?F \;|\; x!C }
  \and
  \inferrule* [lab=abstraction] {} {{F} \bc (x)P}
  \and
  \inferrule* [lab=concretion] {} {{C} \bc \langle Q \rangle}
  \and
  \inferrule* [lab=process] {} {{P,Q} \bc M \;| \;P|Q \;|\; @{x}}
  \and
  \inferrule* [lab=name] {} {{x} \bc \quotep{P}}
\end{mathpar} 

Note that $\vec{x}$ (resp. $\vec{P}$) denotes a vector of names
(resp. processes) of length $|\vec{x}|$ (resp. $|\vec{P}|$). We adopt
the following useful abbreviations.

\begin{mathpar}
   x?(\vec{y}).P := x.(\vec{y})P \and  x\clift{\vec{P}} := x.\clift{\vec{P}}
   \and x!(y) := \lift{x}{\dropn{y}}
   \and \Pi_{i=0}^{n-1}P_i := P_0 | \ldots | P_{n-1}
\end{mathpar}

\subsubsection{Structural congruence}

\paragraph{Free and bound names and alpha-equivalence.} At the
core of structural equivalence is alpha-equivalence which identifies
process that are the same up to a change of variable. Formally, we
recognize the distinction between free and bound names. The free names
of a process, $\freenames{P}$, may be calculated recursively as
follows:

\begin{mathpar}
\freenames{\pzero} := \emptyset
  \and \\
  \freenames{x?(y).P} := \{ x \} \cup (\freenames{P} \setminus \{ y \})
  \and 
  \freenames{x!\langle P \rangle} := \{ x \} \cup \{ P \} 
  \and \\
  \freenames{P|Q} := \freenames{P} \cup \freenames{Q}
  \and \\
  \freenames{@{x}} := \{ x \}
\end{mathpar}

$\pi$
$\quotep{\pi}$

$\freenames{-} : \pi \to \mathcal{P}(\quotep{\pi})$

\begin{eqnarray*}
  \freenames{\pzero} & := & \emptyset \\
  \freenames{x?(y).P} & := & \{ x \} \cup (\freenames{P} \setminus \{ y \}) \\
  \freenames{x!\langle P \rangle} & := & \{ x \} \cup \{ P \} \\
  \freenames{P|Q} & := & \freenames{P} \cup \freenames{Q} \\
  \freenames{\dropn{x}} & := & \{ x \}
\end{eqnarray*}

The bound names of a process, $\boundnames{P}$, are those names occurring in $P$
that are not free. For example, in $x?(y).0$, the name $x$ is free, while $y$ is bound.

\begin{mathpar}
  \inferrule* [lab=monoidal-laws] {} { P|Q \equiv Q|P \and P|0 \equiv P \and P|(Q|R) \equiv (P|Q)|R }
\end{mathpar}

\begin{mathpar}
  \inferrule* [lab=alpha-equivalence] {} { (x)P \equiv (y)P\{y/x\} \and y \not\in \freenames{P} }
\end{mathpar}

\begin{definition}
Then two processes, $P,Q$, are alpha-equivalent if $P = Q\{\vec{y}/\vec{x}\}$ for
some $\vec{x} \in \boundnames{Q},\vec{y} \in \boundnames{P}$, where $Q\{\vec{y}/\vec{x}\}$
denotes the capture-avoiding substitution of $\vec{y}$ for $\vec{x}$ in $Q$.
\end{definition}

\begin{definition}
  The {\em structural congruence} \cite{SangiorgiWalker} , $\equiv$,
  between processes is the least congruence containing
  alpha-equivalence, satisfying the abelian monoid laws
  (associativity, commutativity and $\pzero$ as identity) for parallel
  composition $|$ and for summation $+$.
\end{definition}

\subsection{Name equivalence}

We take name equivalence, written $\nameeq$, to be the smallest
equivalence relation generated by the following rules.

\begin{mathpar}
\inferrule*[lab=Quote-drop]
{ }
{ \quotep{@{x}} \nameeq x }

\inferrule*[lab=Struct-equiv]
{ P \scong Q }
{ \quotep{P} \nameeq \quotep{Q} }
\end{mathpar}

The astute reader will have noticed that the mutual recursion of names
and processes imposes a mutual recursion on alpha-equivalence and
structural equivalence via name-equivalence. Fortunately, all of this
works out pleasantly and we may calculate in the natural way, free of
concern. The reader interested in the details is referred to the
appendix \ref{appendix:rho_details}.

\subsection{Substitution}

We use $\Proc$ for the set of processes, $\QProc$ for the set of
names, and $\id{\{}\vec{y} / \vec{x} \id{\}}$ to denote partial maps,
$s : \QProc \rightarrow \QProc$. A map, $s$ lifts, uniquely, to a map
on process terms, $\widehat{s} : \Proc \rightarrow \Proc$ by the
following equations.

\begin{mathpar}
  (0) \psubstp{Q}{P} := 0 \\
  (R \juxtap S) \psubstp{Q}{P}
  :=    
  (R)\psubstp{Q}{P} \juxtap (S) \psubstp{Q}{P} \\
  (x?(y).R) \psubstp{Q}{P}    
  :=    
  (x)\substp{Q}{P} (z)\concat( (R \psubstn{z}{y}) \psubstp{Q}{P} ) \\
  (\lift{x}{R}) \psubstp{Q}{P}  
  :=
  \lift{(x)\substp{Q}{P}}{ R \psubstp{Q}{P} } \\
%   (\dropn{x})  \psubstp{Q}{P}       
%   := 
%   \left\{ 
%     \begin{array}{ccc} 
%       \dropn{\quotep{Q}} & & x \nameeq \quotep{P} \\
%       \dropn{x} & & otherwise \\
%     \end{array}
%   \right. 
  (\dropn{x})  \psubstp{Q}{P}       
  := 
  \left\{ 
    \begin{array}{ccc} 
      Q & & x \nameeq \quotep{P} \\
      \dropn{x} & & otherwise \\
    \end{array}
  \right.
\end{mathpar}
 

where

\begin{eqnarray}
  (x)\id{\{} \lpquote Q \rpquote / \lpquote P \rpquote \id{\}}            = 
  \left\{ 
    \begin{array}{ccc}
      \lpquote Q \rpquote & & x \nameeq \lpquote P \rpquote \\
      x & & otherwise \\
    \end{array}
  \right. \nonumber
\end{eqnarray}

and $z$ is chosen distinct from $\quotep{P}$, $\quotep{Q}$, the free
names in $Q$, and all the names in $R$. Our $\alpha$-equivalence will
be built in the standard way from this substitution.

\begin{remark}\label{rem:no_self_referential_names}
  One consequence of these definitions is that $\forall P. \quotep{P}
  \not\in \freenames{P}$.
\end{remark}

\subsection{ Dynamic quote: an example }

Anticipating something of what's to come, consider applying the
substitution, $\widehat{\id{\{}u / z \id{\}}}$, to the following pair
of processes, $\lift{w}{y!(z)}$ and $w[ \lpquote y!(z) \rpquote ]$.

\begin{eqnarray}
	\lift{w}{y!(z)}\widehat{\id{\{}u / z \id{\}}}
		& = &
		\lift{w}{y!(u)} \nonumber\\
	w[ \lpquote y!(z) \rpquote ] \widehat{ \id{\{}u / z \id{\}} }
		& = &
		w[ \lpquote y!(z) \rpquote ] \nonumber
\end{eqnarray}

Because the body of the process between quotes is impervious to
substitution, we get radically different answers. In fact, by
examining the first process in an input context,
e.g. $x?(z).\lift{w}{y!(z)}$, we see that the process under the lift
operator may be shaped by prefixed inputs binding a name inside it. In
this sense, the lift operator will be seen as a way to dynamically
construct processes before reifying them as names.

Finally equipped with these standard features we can present the
dynamics of the calculus.

\subsubsection{Operational semantics} 

Finally, we introduce the computational dynamics. What marks these
algebras as distinct from other more traditionally studied algebraic
structures, e.g. vector spaces or polynomial rings, is the manner in
which dynamics is captured. In traditional structures, dynamics is typically
expressed through morphisms between such structures, as in linear maps
between vector spaces or morphisms between rings. In algebras
associated with the semantics of computation, the dynamics is
expressed as part of the algebraic structure itself, through a
reduction reduction relation typically denoted by $\red$. Below, we
give a recursive presentation of this relation for the calculus used
in the encoding.

$\red \subseteq \pi \times \pi$
$\red : \pi \to \mathcal{P}(\pi)$

\begin{mathpar}
  \inferrule* [lab=Comm] { \textsf{match}( x_{src}, x_{trgt} ) } { x_{trgt}?(y)P \; | \; x_{src}!\langle {Q} \rangle \red P\{\quotep{Q}/y}\} }
  \and \\
  \inferrule* [lab=Par] {{P} \red {P}'} {{{P} | {Q}} \red {{P}' | {Q}}}
  \and
  \inferrule* [lab=Equiv]{{{P} \scong {P}'} \andalso {{P}' \red {Q}'} \andalso {{Q}' \scong {Q}}}{{P} \red {Q}}
\end{mathpar}

\begin{eqnarray*}
  match_{\equiv} (\quotep{P},\quotep{Q}) & := & P \equiv Q \\
  match_{\dagger}(\quotep{P},\quotep{Q}) & := & \forall R. P|Q \red^{*} R => R \red^{*} 0 \\
  match_{K}(\quotep{P},\quotep{Q}) & := & K \mbox{ for some context } K
\end{eqnarray*}

$u?(x)P | u!\langle Q \rangle \red P\{\quotep{Q}/x\}$

%We write $\wred$ for $\red^*$, and $P\red$ if $\exists Q $ such that $ P \red Q$.
We write $P\red$ if $\exists Q $ such that $ P \red Q$ and $P\not\red$, otherwise.

\section{Replication}

As mentioned before, it is known that replication (and hence
recursion) can be implemented in a higher-order process algebra
\cite{SangiorgiWalker}. As our first example of calculation with the
machinery thus far presented we give the construction explicitly in
the {\rhoc}.

\begin{eqnarray}
	D_{x} & := & \prefix{x}{y}{(\binpar{\outputp{x}{y}}{@{y}})} \nonumber\\
	\bangp_{x}{P} & := & \binpar{{x}!\langle{\binpar{D_{x}}{P}}\rangle}{D_{x}} \nonumber
\end{eqnarray}

\begin{eqnarray}
	\bangp_{x}{P} & & \nonumber\\
	=
	& {x}!\langle{(\prefix{x}{y}{(\outputp{x}{y} | @{y})) | P}}\rangle 
	      | \prefix{x}{y}{(\outputp{x}{y} | @{y})} & \nonumber\\
	\red
	& (\outputp{x}{y} | @{y})\substn{\quotep{(\prefix{x}{y}{(@{y} | \outputp{x}{y})) | P}}}{y} & \nonumber\\
	=
	& \outputp{x}{\quotep{(\prefix{x}{y}{(\outputp{x}{y} | @{y})) | P}}}
	  | {(\prefix{x}{y}{(\outputp{x}{y} | @{y})) | P}} & \nonumber\\
	\red
	& \ldots & \nonumber\\
	\red^*
	& P | P | \ldots & \nonumber
\end{eqnarray}

Of course, this encoding, as an implementation, runs away, unfolding
$\bangp{P}$ eagerly. A lazier and more implementable replication
operator, restricted to input-guarded processes, may be obtained as follows.

\begin{eqnarray}
\bangp{\prefix{u}{v}{P}} 
	:= 
	\binpar{\lift{x}{\prefix{u}{v}{(\binpar{D(x)}{P})}}}{D(x)} \nonumber
\end{eqnarray}

\begin{remark}
  Note that the lazier definition still does not deal with summation
  or mixed summation (i.e. sums over input and output). The reader is
  invited to construct definitions of replication that deal with these
  features. 

  Further, the definitions are parameterized in a name, $x$. Can you,
  gentle reader, make a definition that eliminates this parameter and
  guarantees no accidental interaction between the replication
  machinery and the process being replicated -- i.e. no accidental
  sharing of names used by the process to get its work done and the
  name(s) used by the replication to effect copying. This latter
  revision of the definition of replication is crucial to obtaining
  the expected identity $!!P \sim !P$.
\end{remark}

\begin{remark}\label{rem:paradoxical_combinator}
  The reader familiar with the lambda calculus will have noticed the
  similarity between $D$ and the paradoxical combinator.

  [Ed. note: the existence of this seems to suggest we have to be more
  restrictive on the set of processes and names we admit if we are to
  support no-cloning.]
\end{remark}

\subsubsection{Bisimulation}

The computational dynamics gives rise to another kind of equivalence,
the equivalence of computational behavior. As previously mentioned
this is typically captured \emph{via} some form of bisimulation.

% The notion we use in this paper is weak barbed bisimulation
% \cite{milner91polyadicpi}.

The notion we use in this paper is derived from weak barbed
bisimulation \cite{milner91polyadicpi}. 

\begin{definition}
An \emph{observation relation}, $\downarrow_{\mathcal N}$, over a set
of names, $\mathcal N$, is the smallest relation satisfying the rules
below.

\infrule[Out-barb]{y \in {\mathcal N}, \; x \nameeq y}
		  {\outputp{x}{v} \downarrow_{\mathcal N} x}
\infrule[Par-barb]{\mbox{$P\downarrow_{\mathcal N} x$ or $Q\downarrow_{\mathcal N} x$}}
		  {\binpar{P}{Q} \downarrow_{\mathcal N} x}

We write $P \Downarrow_{\mathcal N} x$ if there is $Q$ such that 
$P \wred Q$ and $Q \downarrow_{\mathcal N} x$.
\end{definition}

\begin{definition}
%\label{def.bbisim}
An  ${\mathcal N}$-\emph{barbed bisimulation} over a set of names, ${\mathcal N}$, is a symmetric binary relation 
${\mathcal S}_{\mathcal N}$ between agents such that $P\rel{S}_{\mathcal N}Q$ implies:
\begin{enumerate}
\item If $P \red P'$ then $Q \wred Q'$ and $P'\rel{S}_{\mathcal N} Q'$.
\item If $P\downarrow_{\mathcal N} x$, then $Q\Downarrow_{\mathcal N} x$.
\end{enumerate}
$P$ is ${\mathcal N}$-barbed bisimilar to $Q$, written
$P \wbbisim_{\mathcal N} Q$, if $P \rel{S}_{\mathcal N} Q$ for some ${\mathcal N}$-barbed bisimulation ${\mathcal S}_{\mathcal N}$.
\end{definition}

$\mathcal{R} \subseteq \pi \times \pi$

$P \mathcal{R} Q => \forall P'. P \red P' \Rightarrow \exists Q'. Q \red Q', P' \mathcal{R} Q'$

$P \vdash x \Rightarrow Q \vdash x$

\begin{mathpar}
  \inferrule*[lab=Out-barb]{x \nameeq y}{{y}!\langle{Q}\rangle \vdash x}
  \and
  \inferrule*[lab=Par-barb]{\mbox{$P\vdash x$ or $Q\vdash x$}}{\binpar{P}{Q} \vdash x}
\end{mathpar}

\subsubsection{Contexts}

One of the principle advantages of computational calculi like the
$\pi$-calculus is a well-defined notion of context,
contextual-equivalence and a correlation between
contextual-equivalence and notions of bisimulation. The notion of
context allows the decomposition of a process into (sub-)process and
its syntactic environment, its context. Thus, a context may be
thought of as a process with a ``hole'' (written $\Box$) in it. The
application of a context $M$ to a process $P$, written $M[P]$, is
tantamount to filling the hole in $M$ with $P$. In this paper we do
not need the full weight of this theory, but do make use of the notion
of context in the proof the main theorem. 

\begin{mathpar}
  \inferrule* [lab=summation] {} {{M_{M},M_{N}} \bc \Box \;|\; x.M_{A} \;|\; M_{M}+M_{N}}
  \and
  \inferrule* [lab=agent] {} {{M_{A}} \bc (\vec{x})M_{P} \;| \; \clift{P_0,\ldots,M_{P},\ldots,P_N}}
  \and \\
  \inferrule* [lab=process] {} {{M_{P}} \bc M_{N} \;| \;P|M_{P} }
\end{mathpar} 

\begin{mathpar}
  \inferrule* [lab=sychronization] {} {M_{N} \bc \Box \;|\; x?M_{F} \;|\; x!M_{C}}
  \and
  \inferrule* [lab=abstraction] {} {{M_{F}} \bc (x)M_{P} }
  \and
  \inferrule* [lab=concretion] {} {{M_{C}} \bc \langle M_{P} \rangle }
  \and \\
  \inferrule* [lab=process] {} {{M_{P}} \bc M_{N} \;| \;P|M_{P} }
\end{mathpar}

\begin{definition}[contextual application] Given a context $M$, and
  process $P$, we define the \emph{contextual application}, $M[P] :=
  M\{P/\Box\}$. That is, the contextual application of M to P is the
  substitution of $P$ for $\Box$ in $M$.
\end{definition}

$\meaningof{-} : L \to \mathcal{P}(\pi)$

\begin{mathpar}
  \inferrule* [lab=collection] {} {\meaningof{true} = \pi, \and \meaningof{~E} = \pi \setminus \meaningof{E}, \and \meaningof{E_{1} \& E_{2}} = \meaningof{E_{1}} \cap \meaningof{E_{2}}}
\end{mathpar}

\begin{mathpar}
  \inferrule* [lab=structure] {} {\meaningof{0} = \{ P \in \pi | P \equiv 0 \}, \and \\ \meaningof{E_1 | E_2} = \{ P \in \pi | P \equiv P_{1} | P_{2}, P_{1} \in \meaningof{E_{1}}, P_{2} \in \meaningof{E_2}\} }
\end{mathpar}

\begin{mathpar}
 \inferrule* [lab=behavior] {} {\meaningof{\langle a?b \rangle E} = \{ P \in \pi | P \equiv Q | u?(y)P', \\ \and \\\\ \and \\ \;\;\; u \in \meaningof{a}, \forall z.P'\{z/y\} \in \meaningof{E\{z/b\}}\}, \and \\ \meaningof{a!E} = \{ P \in \pi | P \equiv Q | x!\langle P' \rangle, x \in \meaningof{a} P' \in \meaningof{E}\} }
\end{mathpar}

\begin{mathpar}
 \inferrule* [lab=nominal] {} {\meaningof{\quotep{E}} = \{ \quotep{P} \in \quotep{\pi} | P \in \meaningof{E} \}, \and \meaningof{\quotep{P}} = \{ \quotep{Q} \in \quotep{\pi} | P \equiv Q \} \and \\ \meaningof{@\quotep{E}} = \{ P \in \pi | P \equiv @x, x \in \meaningof{E} \}}
\end{mathpar}

\begin{eqnarray*}
  \\
  \meaningof{-} : TS \to ST
\end{eqnarray*}

\begin{eqnarray*}
  \\
  L : TS \to ST
\end{eqnarray*}

\begin{eqnarray*}
  \\
  P \models E \iff P \in \meaningof{E}
\end{eqnarray*}

\begin{eqnarray*}
  P \approx_{L} Q \iff \forall E \in L. P \models E \iff Q \models E
\end{eqnarray*}

\begin{eqnarray*}
  P \approx_{K} Q
\end{eqnarray*}

\begin{eqnarray*}
  P \approx Q
\end{eqnarray*}

$\approx_{K} = \approx = \approx_{L}$

\subsubsection{Contextual duality}

Note that contexts extend the quotation operation to a family of
operations from processes to names. Given a context, $M$, we can
define a \emph{nominal context}, $\quotep{M}$ by $\quotep{M}[P] :=
\quotep{M[P]}$. To foreshadow what is to come we observe that these
operations enjoy a duality with processes very much like the duality
between vectors and maps from vectors to scalars.

Further, because the calculus is essentially higher-order, we have a
correspondence between contexts and processes. More specifically,
given a name $x$ and a context $M$ we can construct $M^{*}_{x}$ such
that 

\begin{mathpar}
  M^{*}_{x} | \lift{x}{P} \red M[P]
\end{mathpar}

namely,

\begin{mathpar}
  M^{*}_{x} := x?(u).M[\dropn{u}]
\end{mathpar}

The dependence of $M^{*}_{x}$ on a name makes it an abstraction, 

\begin{mathpar}
  M^{*} := (x)x?(u).M[\dropn{u}]
\end{mathpar}

\subsection{Additional notation}

It will sometimes be convenient to denote the process a name
quotes. We already have the notation $x = \quotep{P}$, but it will be
convenient to introduce an alternate notation, $\procn{x}$, when we
want to emphasize the connection to the use of the name. Note that, by
virtue of name equivalence, $\quotep{\procn{x}} \nameeq x$; so, the
notation is consistent with previous definitions.

Further, because names have structure it is possible to effect
substitutions on the basis of that structure. This means we need to
upgrade our notation for substitutions, which we accomplish by
adapting comprehension notation. Thus,

\begin{mathpar}
  P\{ y / x : x \in S \}
\end{mathpar}

is interpreted to mean the process derived from P by replacing (in a
capture-avoiding manner) each occurrence of $x$ in $S$ by $y$. For example,

\begin{mathpar}
  P\{ \quotep{\procn{x}|\procn{x}} / x : x \in \freenames{P} \}
\end{mathpar}

will replace each (occurrence) of a free name $x$ in $P$ by
$\quotep{\procn{x}|\procn{x}}$.

Also, we will avail ourselves of the notation $x^{L}$ and $x^{R}$ to
denote injections of a name into disjoint copies of the name
space. There are numerous ways to accomplish this. One example can be
found in \cite{MeredithR05}. This notation overloads to vectors of
names: $\vec{x}^{\pi} := (x_{i}^{\pi} \; : \; 0 \leq i < |\vec{x}| )$ where $\pi \in \{L,R\}$.

We also use $P^{\Box} := P|\Box$.

In \cite{MeredithR05} an interpretation of the new operator is
given. It turns out that there are several possible interpretations
all enjoying the requisite algebraic properties of the operator (see
\cite{milner91polyadicpi}). We will therefore make liberal use of
$(\nu\; \vec{x})P$.

% subsection the_syntax_and_semantics_of_the_notation_system (end)   

\input{qm2pi.qmops} 

\input{qm2pi.sterngerlach} 

\input{qm2pi.metric} 

% section concurrent_process_calculi (end)

%\input{qm2pi.proofsketch}

% section proof sketch (end)

%\input{qm2pi.slviaknots} 

% section spatial logic via knots (end)

\input{qm2pi.conclusion}

% section conclusion (end)

%\input{qm2pi.dtcodes} 

% section wiring algorithm (end)

\input{qm2pi.ack} 

% section acknowledgments (end)

\newpage


\bibliographystyle{plain}   
\bibliography{../../biblios/main.bib}

\input{qm2pi.rhodetails}

\end{document}

 

% section concurrent_process_calculi (end)

%\documentclass[12pt]{llncs}
%\documentclass{jktr}

\usepackage[pdftex]{hyperref}                   
\usepackage {listings}
\usepackage {mathpartir}
\usepackage{bcprules}
%\usepackage{listings}
                       
\usepackage{graphicx} 
%\usepackage[margins=2.5cm,nohead,nofoot]{geometry}
%\usepackage{geometry}
\usepackage{amsfonts}
\usepackage{amstext}
\usepackage{latexsym}
\usepackage{amssymb}
\usepackage{color}


%\include{myPreamble}
\include{qm2pi.local} 

%\ifpdf
%\usepackage[pdftex]{graphicx}
%\else
%\usepackage{graphicx}
%\fi

 % \ifpdf
%  \usepackage{pdfsync}
%  \if


%\title{Brief Article}
%\author{David F. Snyder}
%\author{L.G. Meredith}

%\address{Dept. of Math., Texas State University--San Marcos, San Marcos, TX 78666}
       
\pagestyle{empty}


\begin{document}

\lstset{language=[Objective]Caml,frame=shadowbox}

\input{qm2pi.front}

% section front matter (end)

\input{qm2pi.intro} 
 
% section introduction (end)

% \input{qm2pi.knotations} 

% section notation (end)

\input{qm2pi.process.calculi} 

% section concurrent_process_calculi_and_spatial_logics_ (end)
    
%\input{qm2pi.knots2pi} 

%\input{qm2pi.trefoil} 

%\input{qm2pi.mainthm} 

% subsection basic_interpretation (end)

%\input{qm2pi.rho.presentation} 
\subsection{The syntax and semantics of the notation system}\label{sub:the_syntax_and_semantics_of_the_notation_system} % (fold)

We now summarize a technical presentation of the calculus that
embodies our theory of dynamics. The typical presentation of such a
calculus follows the style of giving generators and relations on
them. The grammar, below, describing term constructors, freely
generates the set of processes, $\Proc$. This set is then quotiented
by a relation known as structural congruence and it is over this set
that the notion of dynamics is expressed. This presentation is
essentially that of \cite{MeredithR05} with the addition of
polyadicity and summation. For readability we have relegated some of
the technical subtleties to an appendix.

\subsubsection{Process grammar}\label{subsub:process_grammar}

\begin{mathpar}
  \inferrule* [lab=synchronization] {} {{M} \bc \pzero \;|\; x?F \;|\; x!C }
  \and
  \inferrule* [lab=abstraction] {} {{F} \bc (x)P}
  \and
  \inferrule* [lab=concretion] {} {{C} \bc \langle Q \rangle}
  \and
  \inferrule* [lab=process] {} {{P,Q} \bc M \;| \;P|Q \;|\; @{x}}
  \and
  \inferrule* [lab=name] {} {{x} \bc \quotep{P}}
\end{mathpar} 

Note that $\vec{x}$ (resp. $\vec{P}$) denotes a vector of names
(resp. processes) of length $|\vec{x}|$ (resp. $|\vec{P}|$). We adopt
the following useful abbreviations.

\begin{mathpar}
   x?(\vec{y}).P := x.(\vec{y})P \and  x\clift{\vec{P}} := x.\clift{\vec{P}}
   \and x!(y) := \lift{x}{\dropn{y}}
   \and \Pi_{i=0}^{n-1}P_i := P_0 | \ldots | P_{n-1}
\end{mathpar}

\subsubsection{Structural congruence}

\paragraph{Free and bound names and alpha-equivalence.} At the
core of structural equivalence is alpha-equivalence which identifies
process that are the same up to a change of variable. Formally, we
recognize the distinction between free and bound names. The free names
of a process, $\freenames{P}$, may be calculated recursively as
follows:

\begin{mathpar}
\freenames{\pzero} := \emptyset
  \and \\
  \freenames{x?(y).P} := \{ x \} \cup (\freenames{P} \setminus \{ y \})
  \and 
  \freenames{x!\langle P \rangle} := \{ x \} \cup \{ P \} 
  \and \\
  \freenames{P|Q} := \freenames{P} \cup \freenames{Q}
  \and \\
  \freenames{@{x}} := \{ x \}
\end{mathpar}

$\pi$
$\quotep{\pi}$

$\freenames{-} : \pi \to \mathcal{P}(\quotep{\pi})$

\begin{eqnarray*}
  \freenames{\pzero} & := & \emptyset \\
  \freenames{x?(y).P} & := & \{ x \} \cup (\freenames{P} \setminus \{ y \}) \\
  \freenames{x!\langle P \rangle} & := & \{ x \} \cup \{ P \} \\
  \freenames{P|Q} & := & \freenames{P} \cup \freenames{Q} \\
  \freenames{\dropn{x}} & := & \{ x \}
\end{eqnarray*}

The bound names of a process, $\boundnames{P}$, are those names occurring in $P$
that are not free. For example, in $x?(y).0$, the name $x$ is free, while $y$ is bound.

\begin{mathpar}
  \inferrule* [lab=monoidal-laws] {} { P|Q \equiv Q|P \and P|0 \equiv P \and P|(Q|R) \equiv (P|Q)|R }
\end{mathpar}

\begin{mathpar}
  \inferrule* [lab=alpha-equivalence] {} { (x)P \equiv (y)P\{y/x\} \and y \not\in \freenames{P} }
\end{mathpar}

\begin{definition}
Then two processes, $P,Q$, are alpha-equivalent if $P = Q\{\vec{y}/\vec{x}\}$ for
some $\vec{x} \in \boundnames{Q},\vec{y} \in \boundnames{P}$, where $Q\{\vec{y}/\vec{x}\}$
denotes the capture-avoiding substitution of $\vec{y}$ for $\vec{x}$ in $Q$.
\end{definition}

\begin{definition}
  The {\em structural congruence} \cite{SangiorgiWalker} , $\equiv$,
  between processes is the least congruence containing
  alpha-equivalence, satisfying the abelian monoid laws
  (associativity, commutativity and $\pzero$ as identity) for parallel
  composition $|$ and for summation $+$.
\end{definition}

\subsection{Name equivalence}

We take name equivalence, written $\nameeq$, to be the smallest
equivalence relation generated by the following rules.

\begin{mathpar}
\inferrule*[lab=Quote-drop]
{ }
{ \quotep{@{x}} \nameeq x }

\inferrule*[lab=Struct-equiv]
{ P \scong Q }
{ \quotep{P} \nameeq \quotep{Q} }
\end{mathpar}

The astute reader will have noticed that the mutual recursion of names
and processes imposes a mutual recursion on alpha-equivalence and
structural equivalence via name-equivalence. Fortunately, all of this
works out pleasantly and we may calculate in the natural way, free of
concern. The reader interested in the details is referred to the
appendix \ref{appendix:rho_details}.

\subsection{Substitution}

We use $\Proc$ for the set of processes, $\QProc$ for the set of
names, and $\id{\{}\vec{y} / \vec{x} \id{\}}$ to denote partial maps,
$s : \QProc \rightarrow \QProc$. A map, $s$ lifts, uniquely, to a map
on process terms, $\widehat{s} : \Proc \rightarrow \Proc$ by the
following equations.

\begin{mathpar}
  (0) \psubstp{Q}{P} := 0 \\
  (R \juxtap S) \psubstp{Q}{P}
  :=    
  (R)\psubstp{Q}{P} \juxtap (S) \psubstp{Q}{P} \\
  (x?(y).R) \psubstp{Q}{P}    
  :=    
  (x)\substp{Q}{P} (z)\concat( (R \psubstn{z}{y}) \psubstp{Q}{P} ) \\
  (\lift{x}{R}) \psubstp{Q}{P}  
  :=
  \lift{(x)\substp{Q}{P}}{ R \psubstp{Q}{P} } \\
%   (\dropn{x})  \psubstp{Q}{P}       
%   := 
%   \left\{ 
%     \begin{array}{ccc} 
%       \dropn{\quotep{Q}} & & x \nameeq \quotep{P} \\
%       \dropn{x} & & otherwise \\
%     \end{array}
%   \right. 
  (\dropn{x})  \psubstp{Q}{P}       
  := 
  \left\{ 
    \begin{array}{ccc} 
      Q & & x \nameeq \quotep{P} \\
      \dropn{x} & & otherwise \\
    \end{array}
  \right.
\end{mathpar}
 

where

\begin{eqnarray}
  (x)\id{\{} \lpquote Q \rpquote / \lpquote P \rpquote \id{\}}            = 
  \left\{ 
    \begin{array}{ccc}
      \lpquote Q \rpquote & & x \nameeq \lpquote P \rpquote \\
      x & & otherwise \\
    \end{array}
  \right. \nonumber
\end{eqnarray}

and $z$ is chosen distinct from $\quotep{P}$, $\quotep{Q}$, the free
names in $Q$, and all the names in $R$. Our $\alpha$-equivalence will
be built in the standard way from this substitution.

\begin{remark}\label{rem:no_self_referential_names}
  One consequence of these definitions is that $\forall P. \quotep{P}
  \not\in \freenames{P}$.
\end{remark}

\subsection{ Dynamic quote: an example }

Anticipating something of what's to come, consider applying the
substitution, $\widehat{\id{\{}u / z \id{\}}}$, to the following pair
of processes, $\lift{w}{y!(z)}$ and $w[ \lpquote y!(z) \rpquote ]$.

\begin{eqnarray}
	\lift{w}{y!(z)}\widehat{\id{\{}u / z \id{\}}}
		& = &
		\lift{w}{y!(u)} \nonumber\\
	w[ \lpquote y!(z) \rpquote ] \widehat{ \id{\{}u / z \id{\}} }
		& = &
		w[ \lpquote y!(z) \rpquote ] \nonumber
\end{eqnarray}

Because the body of the process between quotes is impervious to
substitution, we get radically different answers. In fact, by
examining the first process in an input context,
e.g. $x?(z).\lift{w}{y!(z)}$, we see that the process under the lift
operator may be shaped by prefixed inputs binding a name inside it. In
this sense, the lift operator will be seen as a way to dynamically
construct processes before reifying them as names.

Finally equipped with these standard features we can present the
dynamics of the calculus.

\subsubsection{Operational semantics} 

Finally, we introduce the computational dynamics. What marks these
algebras as distinct from other more traditionally studied algebraic
structures, e.g. vector spaces or polynomial rings, is the manner in
which dynamics is captured. In traditional structures, dynamics is typically
expressed through morphisms between such structures, as in linear maps
between vector spaces or morphisms between rings. In algebras
associated with the semantics of computation, the dynamics is
expressed as part of the algebraic structure itself, through a
reduction reduction relation typically denoted by $\red$. Below, we
give a recursive presentation of this relation for the calculus used
in the encoding.

$\red \subseteq \pi \times \pi$
$\red : \pi \to \mathcal{P}(\pi)$

\begin{mathpar}
  \inferrule* [lab=Comm] { \textsf{match}( x_{src}, x_{trgt} ) } { x_{trgt}?(y)P \; | \; x_{src}!\langle {Q} \rangle \red P\{\quotep{Q}/y}\} }
  \and \\
  \inferrule* [lab=Par] {{P} \red {P}'} {{{P} | {Q}} \red {{P}' | {Q}}}
  \and
  \inferrule* [lab=Equiv]{{{P} \scong {P}'} \andalso {{P}' \red {Q}'} \andalso {{Q}' \scong {Q}}}{{P} \red {Q}}
\end{mathpar}

\begin{eqnarray*}
  match_{\equiv} (\quotep{P},\quotep{Q}) & := & P \equiv Q \\
  match_{\dagger}(\quotep{P},\quotep{Q}) & := & \forall R. P|Q \red^{*} R => R \red^{*} 0 \\
  match_{K}(\quotep{P},\quotep{Q}) & := & K \mbox{ for some context } K
\end{eqnarray*}

$u?(x)P | u!\langle Q \rangle \red P\{\quotep{Q}/x\}$

%We write $\wred$ for $\red^*$, and $P\red$ if $\exists Q $ such that $ P \red Q$.
We write $P\red$ if $\exists Q $ such that $ P \red Q$ and $P\not\red$, otherwise.

\section{Replication}

As mentioned before, it is known that replication (and hence
recursion) can be implemented in a higher-order process algebra
\cite{SangiorgiWalker}. As our first example of calculation with the
machinery thus far presented we give the construction explicitly in
the {\rhoc}.

\begin{eqnarray}
	D_{x} & := & \prefix{x}{y}{(\binpar{\outputp{x}{y}}{@{y}})} \nonumber\\
	\bangp_{x}{P} & := & \binpar{{x}!\langle{\binpar{D_{x}}{P}}\rangle}{D_{x}} \nonumber
\end{eqnarray}

\begin{eqnarray}
	\bangp_{x}{P} & & \nonumber\\
	=
	& {x}!\langle{(\prefix{x}{y}{(\outputp{x}{y} | @{y})) | P}}\rangle 
	      | \prefix{x}{y}{(\outputp{x}{y} | @{y})} & \nonumber\\
	\red
	& (\outputp{x}{y} | @{y})\substn{\quotep{(\prefix{x}{y}{(@{y} | \outputp{x}{y})) | P}}}{y} & \nonumber\\
	=
	& \outputp{x}{\quotep{(\prefix{x}{y}{(\outputp{x}{y} | @{y})) | P}}}
	  | {(\prefix{x}{y}{(\outputp{x}{y} | @{y})) | P}} & \nonumber\\
	\red
	& \ldots & \nonumber\\
	\red^*
	& P | P | \ldots & \nonumber
\end{eqnarray}

Of course, this encoding, as an implementation, runs away, unfolding
$\bangp{P}$ eagerly. A lazier and more implementable replication
operator, restricted to input-guarded processes, may be obtained as follows.

\begin{eqnarray}
\bangp{\prefix{u}{v}{P}} 
	:= 
	\binpar{\lift{x}{\prefix{u}{v}{(\binpar{D(x)}{P})}}}{D(x)} \nonumber
\end{eqnarray}

\begin{remark}
  Note that the lazier definition still does not deal with summation
  or mixed summation (i.e. sums over input and output). The reader is
  invited to construct definitions of replication that deal with these
  features. 

  Further, the definitions are parameterized in a name, $x$. Can you,
  gentle reader, make a definition that eliminates this parameter and
  guarantees no accidental interaction between the replication
  machinery and the process being replicated -- i.e. no accidental
  sharing of names used by the process to get its work done and the
  name(s) used by the replication to effect copying. This latter
  revision of the definition of replication is crucial to obtaining
  the expected identity $!!P \sim !P$.
\end{remark}

\begin{remark}\label{rem:paradoxical_combinator}
  The reader familiar with the lambda calculus will have noticed the
  similarity between $D$ and the paradoxical combinator.

  [Ed. note: the existence of this seems to suggest we have to be more
  restrictive on the set of processes and names we admit if we are to
  support no-cloning.]
\end{remark}

\subsubsection{Bisimulation}

The computational dynamics gives rise to another kind of equivalence,
the equivalence of computational behavior. As previously mentioned
this is typically captured \emph{via} some form of bisimulation.

% The notion we use in this paper is weak barbed bisimulation
% \cite{milner91polyadicpi}.

The notion we use in this paper is derived from weak barbed
bisimulation \cite{milner91polyadicpi}. 

\begin{definition}
An \emph{observation relation}, $\downarrow_{\mathcal N}$, over a set
of names, $\mathcal N$, is the smallest relation satisfying the rules
below.

\infrule[Out-barb]{y \in {\mathcal N}, \; x \nameeq y}
		  {\outputp{x}{v} \downarrow_{\mathcal N} x}
\infrule[Par-barb]{\mbox{$P\downarrow_{\mathcal N} x$ or $Q\downarrow_{\mathcal N} x$}}
		  {\binpar{P}{Q} \downarrow_{\mathcal N} x}

We write $P \Downarrow_{\mathcal N} x$ if there is $Q$ such that 
$P \wred Q$ and $Q \downarrow_{\mathcal N} x$.
\end{definition}

\begin{definition}
%\label{def.bbisim}
An  ${\mathcal N}$-\emph{barbed bisimulation} over a set of names, ${\mathcal N}$, is a symmetric binary relation 
${\mathcal S}_{\mathcal N}$ between agents such that $P\rel{S}_{\mathcal N}Q$ implies:
\begin{enumerate}
\item If $P \red P'$ then $Q \wred Q'$ and $P'\rel{S}_{\mathcal N} Q'$.
\item If $P\downarrow_{\mathcal N} x$, then $Q\Downarrow_{\mathcal N} x$.
\end{enumerate}
$P$ is ${\mathcal N}$-barbed bisimilar to $Q$, written
$P \wbbisim_{\mathcal N} Q$, if $P \rel{S}_{\mathcal N} Q$ for some ${\mathcal N}$-barbed bisimulation ${\mathcal S}_{\mathcal N}$.
\end{definition}

$\mathcal{R} \subseteq \pi \times \pi$

$P \mathcal{R} Q => \forall P'. P \red P' \Rightarrow \exists Q'. Q \red Q', P' \mathcal{R} Q'$

$P \vdash x \Rightarrow Q \vdash x$

\begin{mathpar}
  \inferrule*[lab=Out-barb]{x \nameeq y}{{y}!\langle{Q}\rangle \vdash x}
  \and
  \inferrule*[lab=Par-barb]{\mbox{$P\vdash x$ or $Q\vdash x$}}{\binpar{P}{Q} \vdash x}
\end{mathpar}

\subsubsection{Contexts}

One of the principle advantages of computational calculi like the
$\pi$-calculus is a well-defined notion of context,
contextual-equivalence and a correlation between
contextual-equivalence and notions of bisimulation. The notion of
context allows the decomposition of a process into (sub-)process and
its syntactic environment, its context. Thus, a context may be
thought of as a process with a ``hole'' (written $\Box$) in it. The
application of a context $M$ to a process $P$, written $M[P]$, is
tantamount to filling the hole in $M$ with $P$. In this paper we do
not need the full weight of this theory, but do make use of the notion
of context in the proof the main theorem. 

\begin{mathpar}
  \inferrule* [lab=summation] {} {{M_{M},M_{N}} \bc \Box \;|\; x.M_{A} \;|\; M_{M}+M_{N}}
  \and
  \inferrule* [lab=agent] {} {{M_{A}} \bc (\vec{x})M_{P} \;| \; \clift{P_0,\ldots,M_{P},\ldots,P_N}}
  \and \\
  \inferrule* [lab=process] {} {{M_{P}} \bc M_{N} \;| \;P|M_{P} }
\end{mathpar} 

\begin{mathpar}
  \inferrule* [lab=sychronization] {} {M_{N} \bc \Box \;|\; x?M_{F} \;|\; x!M_{C}}
  \and
  \inferrule* [lab=abstraction] {} {{M_{F}} \bc (x)M_{P} }
  \and
  \inferrule* [lab=concretion] {} {{M_{C}} \bc \langle M_{P} \rangle }
  \and \\
  \inferrule* [lab=process] {} {{M_{P}} \bc M_{N} \;| \;P|M_{P} }
\end{mathpar}

\begin{definition}[contextual application] Given a context $M$, and
  process $P$, we define the \emph{contextual application}, $M[P] :=
  M\{P/\Box\}$. That is, the contextual application of M to P is the
  substitution of $P$ for $\Box$ in $M$.
\end{definition}

$\meaningof{-} : L \to \mathcal{P}(\pi)$

\begin{mathpar}
  \inferrule* [lab=collection] {} {\meaningof{true} = \pi, \and \meaningof{~E} = \pi \setminus \meaningof{E}, \and \meaningof{E_{1} \& E_{2}} = \meaningof{E_{1}} \cap \meaningof{E_{2}}}
\end{mathpar}

\begin{mathpar}
  \inferrule* [lab=structure] {} {\meaningof{0} = \{ P \in \pi | P \equiv 0 \}, \and \\ \meaningof{E_1 | E_2} = \{ P \in \pi | P \equiv P_{1} | P_{2}, P_{1} \in \meaningof{E_{1}}, P_{2} \in \meaningof{E_2}\} }
\end{mathpar}

\begin{mathpar}
 \inferrule* [lab=behavior] {} {\meaningof{\langle a?b \rangle E} = \{ P \in \pi | P \equiv Q | u?(y)P', \\ \and \\\\ \and \\ \;\;\; u \in \meaningof{a}, \forall z.P'\{z/y\} \in \meaningof{E\{z/b\}}\}, \and \\ \meaningof{a!E} = \{ P \in \pi | P \equiv Q | x!\langle P' \rangle, x \in \meaningof{a} P' \in \meaningof{E}\} }
\end{mathpar}

\begin{mathpar}
 \inferrule* [lab=nominal] {} {\meaningof{\quotep{E}} = \{ \quotep{P} \in \quotep{\pi} | P \in \meaningof{E} \}, \and \meaningof{\quotep{P}} = \{ \quotep{Q} \in \quotep{\pi} | P \equiv Q \} \and \\ \meaningof{@\quotep{E}} = \{ P \in \pi | P \equiv @x, x \in \meaningof{E} \}}
\end{mathpar}

\begin{eqnarray*}
  \\
  \meaningof{-} : TS \to ST
\end{eqnarray*}

\begin{eqnarray*}
  \\
  L : TS \to ST
\end{eqnarray*}

\begin{eqnarray*}
  \\
  P \models E \iff P \in \meaningof{E}
\end{eqnarray*}

\begin{eqnarray*}
  P \approx_{L} Q \iff \forall E \in L. P \models E \iff Q \models E
\end{eqnarray*}

\begin{eqnarray*}
  P \approx_{K} Q
\end{eqnarray*}

\begin{eqnarray*}
  P \approx Q
\end{eqnarray*}

$\approx_{K} = \approx = \approx_{L}$

\subsubsection{Contextual duality}

Note that contexts extend the quotation operation to a family of
operations from processes to names. Given a context, $M$, we can
define a \emph{nominal context}, $\quotep{M}$ by $\quotep{M}[P] :=
\quotep{M[P]}$. To foreshadow what is to come we observe that these
operations enjoy a duality with processes very much like the duality
between vectors and maps from vectors to scalars.

Further, because the calculus is essentially higher-order, we have a
correspondence between contexts and processes. More specifically,
given a name $x$ and a context $M$ we can construct $M^{*}_{x}$ such
that 

\begin{mathpar}
  M^{*}_{x} | \lift{x}{P} \red M[P]
\end{mathpar}

namely,

\begin{mathpar}
  M^{*}_{x} := x?(u).M[\dropn{u}]
\end{mathpar}

The dependence of $M^{*}_{x}$ on a name makes it an abstraction, 

\begin{mathpar}
  M^{*} := (x)x?(u).M[\dropn{u}]
\end{mathpar}

\subsection{Additional notation}

It will sometimes be convenient to denote the process a name
quotes. We already have the notation $x = \quotep{P}$, but it will be
convenient to introduce an alternate notation, $\procn{x}$, when we
want to emphasize the connection to the use of the name. Note that, by
virtue of name equivalence, $\quotep{\procn{x}} \nameeq x$; so, the
notation is consistent with previous definitions.

Further, because names have structure it is possible to effect
substitutions on the basis of that structure. This means we need to
upgrade our notation for substitutions, which we accomplish by
adapting comprehension notation. Thus,

\begin{mathpar}
  P\{ y / x : x \in S \}
\end{mathpar}

is interpreted to mean the process derived from P by replacing (in a
capture-avoiding manner) each occurrence of $x$ in $S$ by $y$. For example,

\begin{mathpar}
  P\{ \quotep{\procn{x}|\procn{x}} / x : x \in \freenames{P} \}
\end{mathpar}

will replace each (occurrence) of a free name $x$ in $P$ by
$\quotep{\procn{x}|\procn{x}}$.

Also, we will avail ourselves of the notation $x^{L}$ and $x^{R}$ to
denote injections of a name into disjoint copies of the name
space. There are numerous ways to accomplish this. One example can be
found in \cite{MeredithR05}. This notation overloads to vectors of
names: $\vec{x}^{\pi} := (x_{i}^{\pi} \; : \; 0 \leq i < |\vec{x}| )$ where $\pi \in \{L,R\}$.

We also use $P^{\Box} := P|\Box$.

In \cite{MeredithR05} an interpretation of the new operator is
given. It turns out that there are several possible interpretations
all enjoying the requisite algebraic properties of the operator (see
\cite{milner91polyadicpi}). We will therefore make liberal use of
$(\nu\; \vec{x})P$.

% subsection the_syntax_and_semantics_of_the_notation_system (end)   

\input{qm2pi.qmops} 

\input{qm2pi.sterngerlach} 

\input{qm2pi.metric} 

% section concurrent_process_calculi (end)

%\input{qm2pi.proofsketch}

% section proof sketch (end)

%\input{qm2pi.slviaknots} 

% section spatial logic via knots (end)

\input{qm2pi.conclusion}

% section conclusion (end)

%\input{qm2pi.dtcodes} 

% section wiring algorithm (end)

\input{qm2pi.ack} 

% section acknowledgments (end)

\newpage


\bibliographystyle{plain}   
\bibliography{../../biblios/main.bib}

\input{qm2pi.rhodetails}

\end{document}



% section proof sketch (end)

%\section{Unlikely characters: spatial logic for
  knots}\label{sub:characteristic_formulae} % (fold)

Associated to the mobile process calculi are a family of logics known
as the Hennessy-Milner logics. These logics typically enjoy a
semantics interpreting formulae as sets of processes that when
factored through the encoding outlined above allows an identification
of classes of knots with logical formulae. In the context of this
encoding the sub-family known as the spatial logics \cite{CairesC03}
\cite{CairesC04} \cite{Caires04} are of particular interest providing
several important features for expressing and reasoning about
properties (i.e. classes) of knots. We hint here at how this may be done.

%\begin{description}
%\item [structural connectives] 
\subsubsection{Structural connectives} The spatial logics enjoy
structural connectives corresponding, at the logical level, to the
parallel composition ($P | Q$) and new name ($(\nu \; x)P$)
connectives for processes. As illustrated in the examples below, these
connectives are extremely expressive given the shape of our encoding.
%\item [decideable satisfaction]

\subsubsection{Decideable satisfaction}
In \cite{Caires04} the satisfaction relation is shown to be decideable
for a rich class of processes. It further turns out that the image of
the our encoding is a proper subset of that class. This result
provides the basis for an algorithm by which to search for knots
enjoying a given property.
%\item [characteristic formulae]

\subsubsection{Characteristic formulae}
In the same paper \cite{Caires04} , Caires presents a means of calculating
characteristic formulae, selecting equivalence classes of processes
up to a pre--specified depth limit on the support set of names. Composed with our
encoding, this characteristic formula can be used to select
characteristic formulae for knots.
%\end{description}

\subsubsection{Spatial logic formulae}

The grammar below (segmented for comprehension) summarizes the syntax
of spatial logic formulae. We employ illustrative examples in the
sequel to provide an intuitive understanding of their meaning
referring the reader to \cite{Caires04} for a more detailed explication
of the semantics.

\begin{mathpar}
  \inferrule* [lab=boolean] {} {{A,B} \bc T \;|\; \neg A \;|\; A \wedge B \;|\; \eta = \eta'}
  \and
  \inferrule* [lab=spatial] {} {|\; \pzero \;|\; A | B \;|\; x \text{\textregistered} A \;|\; \forall x . A \;|\;  H x . A}
  \and
  \inferrule* [lab=behavioral] {} {|\; \alpha . A}
  \and 
  \inferrule* [lab=recursion] {} {|\; X(\vec{u}) \;|\; \mu X(\vec{u}) . A}
  \and
  \inferrule* [lab=action] {} {\alpha \bc \langle x?(\vec{y}) \rangle \;|\; \langle x!(\vec{y}) \rangle \;|\; \langle \tau \rangle}
  \and 
  \inferrule* [lab=name] {} {\eta \bc x \;|\; \tau}
\end{mathpar} 

% subsection characteristic_formulae (end)   	 

\subsection{Example formulae}\label{sub:example_formulae_} % (fold)

\subsubsection{Crossing as formula.}
% 
% \begin{align*}
%   \frac{d}{dx} \sin x &= \cos x 
%   & \frac{d}{dx} e^x &= e^x \\
%   \frac{d}{dx} \cos x &= - \sin x 
%   & \frac{d}{dx} \log x &= \frac{1}{x} \\
% \end{align*} 

\begin{align*}
 \mu C(x_{0},x_{1},y_{0},y_{1},u).&(\langle x_{0}?(z) \rangle(\langle u! \rangle\langle y_{1}!z \rangle C(x_{0},x_{1},y_{0},y_{1},u)) & \\
  & \wedge \langle y_{1}?(z) \rangle (\langle u! \rangle \langle x_{0}!z \rangle C(x_{0},x_{1},y_{0},y_{1},u)) & \\
  & \wedge \langle x_{1}?(z) \rangle (\langle u? \rangle \langle y_{0}!z \rangle C(x_{0},x_{1},y_{0},y_{1},u)) & \\
  & \wedge \langle y_{0}?(z) \rangle (\langle u? \rangle \langle x_{1}!z \rangle C(x_{0},x_{1},y_{0},y_{1},u))) &
\end{align*}

The lexicographical similarity between the shape of this formulae and
the shape of definition of the process representing a crossing reveals
the intuitive meaning of this formulae. It describes the capabilities
of a process that has the right to represent a crossing. For example
it picks out processes that may perform an input on the port $x_0$ in
its initial menu of capabilities. What differentiates the formula
from the process, however, is that the crossing process is the
smallest candidate to satisfy the formula. Infinitely many other
processes -- with internal behavior hidden behind this interface, so
to speak -- also satisfy this formula. Even this simple formula,
then, can be seen to open a new view onto knots, providing a
computational interpretation of \emph{virtual} knots.

Note that this formula is derived by hand. A similar formula can be
derived by employing Caires' calculation of characteristic formula
\cite{Caires04} to the process representing a crossing. In light of
this discussion, we let
$\meaningof{C}_{\phi}(x0,x1,y0,y1,u)$ denote a formula specifying the
dynamics we wish to capture of a crossing. To guarantee we preserve
the shape of the interface and minimal semantics we demand that
$\meaningof{C}_{\phi}(x0,x1,y0,y1,u) \Rightarrow
\textbf{C}(x0,x1,y0,y1,u)$ where $\textbf{C}(x0,x1,y0,y1,u)$ denotes
the formula above.
                            
\subsubsection{Crossing number constraints.}
The moral content of the context lemma (Lemma \ref{context}) is that the notion of
``locality'' in the Reidemeister moves is effectively captured by the
parallel composition operator of the process calculus. This intuition
extends through the logic. Given a formula,
$\meaningof{C}_{\phi}(x0,x1,y0,y1,u)$, we can use the structural
connectives to specify constraints on crossing numbers, such as at
least $n$ crossings, or exactly $n$ crossings.
\begin{mathpar}
  \inferrule* [lab=at-least-n] {} { K^{\geq n}_{\phi}(\vec{xs},\vec{ys}) := \Pi_{i=0}^{n-1} Hu . \meaningof{C}_{\phi}(xs_i,ys_i,u) | T }
  \and 
  \inferrule* [lab=exactly-n] {} { K^{= n}_{\phi}(\vec{xs},\vec{ys}) := \Pi_{i=0}^{n-1} Hu . \meaningof{C}_{\phi}(xs_i,ys_i,u) | \neg (\forall x_0,y_0,x_1,y_1,u . \meaningof{C}_{\phi}(x_0,y_0,x_1,y_1,u) | T) }
\end{mathpar}

To round out this section, recall that the encoding of an $n$-crossing
knot decomposes into a parallel composition of $n$ \emph{copies} of a
crossing process together with a wiring harness. To specify different
knot classes with the same crossing number amounts to specifying
logical constraints on the wiring harness. In the interest of space,
we defer examples to a forthcoming paper. Suffice it to say that both
the conditions ``alternating knot'' and ``contains the tangle
corresponding to 5/3'' are expressible. For example, it is possible to
calculate the characteristic formula of a process corresponding to the
tangle 5/3 and conjoin it into the classifying formula via the
composition connective of the logic.

Finally, we wish to observe that it is entirely within reason to
contemplate a more domain-specific version of spatial logic tailored
to the shape of processes in the image of the encoding. Such a
domain-specific logic would have a better claim to the title formal
language of knot properties.

% subsection example_formulae_ (end)

% section knots_as_processes (end) 

% section spatial logic via knots (end)

\section{Conclusions and future work}

\paragraph{Testing physical space}
You, gentle reader, may wonder why of all the theorems to be proved
given this set up we pick the one above. In some sense it's hardly
central to quantum mechanics. We see it as central in the sense that
it firmly establishes a notion of physical space arising from a notion
of the equivalence of behavior. Relating bisimulation to a metric is a
big step forward, but one is faced with interpreting the relationship
of that metric space to something more physical. Quantum mechanical
notions of ``physical'' space are still far from intuitive, but by
relating this idea of distance as testing to calculations that predict
physical circumstances we are making a not insignificant step forward
toward an understanding of the physical space we inhabit as
essentially dynamic.

\paragraph{Effectivity and simulation}
One of the observations we have yet to make is that the entire program
spelled out here is effective. We have built various interpreters for
the reflective calculus at work in this interpretation. In principle,
then, we can simulate quantum mechanics on a computer. The place where
the simulation may lose fidelity is the infinitely branching summation
for the annihilator.

In this connection i also want to point out that the evaluation style
calculation of the inner product puts the non-determinism of the
summation right at the heart of measurement. This suggests that
Milner's original reduction-based formulation of the dynamics of his
calculi in terms of sums was not just notationally suggestive of a
notion of measure-and-continue but captured some significant part of
the physics.

\paragraph{Quantum continuations}
In light of this last observation i want to point out that the
predominant account of quantum mechanics is missing a key aspect of a
truly compositional story of the physical situation. In a real lab,
when a measurement is made the observation can be made to feed into
another device that then makes another measurement conditioned on the
results of the first. This means that after the superposition was
collapsed the entire experimental set up remained in
superposition. While QM offers a means of writing this down it doesn't
quite line up well with the well-trodden formulation of computation
and continuation that we see so succinctly expressed in Milner's
calculi. This suggests that there might be advantages to this account
of dynamics waiting to be explored.

\paragraph{Quantum logic}
In this connection, we also note that by virtue of having the
Hennessy-Milner construction, we can pull the construction through the
interpretation of QM. This gives us a natural candidate for a quantum
logic that enjoys an extremely tight connection with it's domain of
interpretation, making the construction much less ad hoc (rather it is
the image of functor!).

\paragraph{Quantum probabiity}
i have questions about the basis of the interpretation of inner
product as probability amplitude. In particular, using which
axiomatization of probability theory does the notion of probability
amplitude earn the right to be so dubbed? In other words, where is the
proof that the operation for calculating a probability amplitude (and
then squaring) satisfies the axioms of what it means to calculate a
probability? Even if such a proof exists (i have yet to find it in the
literature), i wonder if it might not be possible to turn things on
their heads. Can we view the calculation of the probability amplitude
as an axiomatization of probability? If so, then the definition we
give for calculating probability amplitude may provide the basis for
an \emph{effective} theory of probability.

\paragraph{Quantum vs ``biological'' information}
Finally, i want to conclude with a more philosophical observation. At
a recent workshop in which QM was a predominant topic i noticed
something about quantum information. The speaker was giving a riveting
discussion of axiomatic QM and showing how properties of ``no
cloning'' and ``no deleting'' emerged as consequences of the
axiomatization. Theorems of this form are necessary to give us a sense
of confidence that our axioms characterize the physical theory. What
struck me, though, was that if quantum information is neither erasable
nor replicable it is markedly different from \emph{life}. Two of the
things we know about life is that

\begin{itemize}
  \item it ends;
  \item to gain some measure of persistence, to transcend it's
    finitude it is imminently copyable.
\end{itemize}

Both of these qualities are summarized succinctly in the aphorism: all
flesh is grass. For me these two kinds of ``information'' -- call them
quantum and biological -- are end points on a spectrum of strategies
for persistence. At one end, we have those curious entities that enjoy
uniqueness and permanence; at the other, we have those who in the face
of a certain end and an uncertain present make a go of passing
something on. To me one of the more remarkable aspects of the latter
strategy is that in the presence of noise (and certain features of
copying) we get a kind of dynamism, a chance for improvement against a
given persistent condition.

% subsection other_calculi_other_bisimulations_and_geometry_as_behavior (end)




% section conclusion (end)

%\documentclass[12pt]{llncs}
%\documentclass{jktr}

\usepackage[pdftex]{hyperref}                   
\usepackage {listings}
\usepackage {mathpartir}
\usepackage{bcprules}
%\usepackage{listings}
                       
\usepackage{graphicx} 
%\usepackage[margins=2.5cm,nohead,nofoot]{geometry}
%\usepackage{geometry}
\usepackage{amsfonts}
\usepackage{amstext}
\usepackage{latexsym}
\usepackage{amssymb}
\usepackage{color}


%\include{myPreamble}
\include{qm2pi.local} 

%\ifpdf
%\usepackage[pdftex]{graphicx}
%\else
%\usepackage{graphicx}
%\fi

 % \ifpdf
%  \usepackage{pdfsync}
%  \if


%\title{Brief Article}
%\author{David F. Snyder}
%\author{L.G. Meredith}

%\address{Dept. of Math., Texas State University--San Marcos, San Marcos, TX 78666}
       
\pagestyle{empty}


\begin{document}

\lstset{language=[Objective]Caml,frame=shadowbox}

\input{qm2pi.front}

% section front matter (end)

\input{qm2pi.intro} 
 
% section introduction (end)

% \input{qm2pi.knotations} 

% section notation (end)

\input{qm2pi.process.calculi} 

% section concurrent_process_calculi_and_spatial_logics_ (end)
    
%\input{qm2pi.knots2pi} 

%\input{qm2pi.trefoil} 

%\input{qm2pi.mainthm} 

% subsection basic_interpretation (end)

%\input{qm2pi.rho.presentation} 
\subsection{The syntax and semantics of the notation system}\label{sub:the_syntax_and_semantics_of_the_notation_system} % (fold)

We now summarize a technical presentation of the calculus that
embodies our theory of dynamics. The typical presentation of such a
calculus follows the style of giving generators and relations on
them. The grammar, below, describing term constructors, freely
generates the set of processes, $\Proc$. This set is then quotiented
by a relation known as structural congruence and it is over this set
that the notion of dynamics is expressed. This presentation is
essentially that of \cite{MeredithR05} with the addition of
polyadicity and summation. For readability we have relegated some of
the technical subtleties to an appendix.

\subsubsection{Process grammar}\label{subsub:process_grammar}

\begin{mathpar}
  \inferrule* [lab=synchronization] {} {{M} \bc \pzero \;|\; x?F \;|\; x!C }
  \and
  \inferrule* [lab=abstraction] {} {{F} \bc (x)P}
  \and
  \inferrule* [lab=concretion] {} {{C} \bc \langle Q \rangle}
  \and
  \inferrule* [lab=process] {} {{P,Q} \bc M \;| \;P|Q \;|\; @{x}}
  \and
  \inferrule* [lab=name] {} {{x} \bc \quotep{P}}
\end{mathpar} 

Note that $\vec{x}$ (resp. $\vec{P}$) denotes a vector of names
(resp. processes) of length $|\vec{x}|$ (resp. $|\vec{P}|$). We adopt
the following useful abbreviations.

\begin{mathpar}
   x?(\vec{y}).P := x.(\vec{y})P \and  x\clift{\vec{P}} := x.\clift{\vec{P}}
   \and x!(y) := \lift{x}{\dropn{y}}
   \and \Pi_{i=0}^{n-1}P_i := P_0 | \ldots | P_{n-1}
\end{mathpar}

\subsubsection{Structural congruence}

\paragraph{Free and bound names and alpha-equivalence.} At the
core of structural equivalence is alpha-equivalence which identifies
process that are the same up to a change of variable. Formally, we
recognize the distinction between free and bound names. The free names
of a process, $\freenames{P}$, may be calculated recursively as
follows:

\begin{mathpar}
\freenames{\pzero} := \emptyset
  \and \\
  \freenames{x?(y).P} := \{ x \} \cup (\freenames{P} \setminus \{ y \})
  \and 
  \freenames{x!\langle P \rangle} := \{ x \} \cup \{ P \} 
  \and \\
  \freenames{P|Q} := \freenames{P} \cup \freenames{Q}
  \and \\
  \freenames{@{x}} := \{ x \}
\end{mathpar}

$\pi$
$\quotep{\pi}$

$\freenames{-} : \pi \to \mathcal{P}(\quotep{\pi})$

\begin{eqnarray*}
  \freenames{\pzero} & := & \emptyset \\
  \freenames{x?(y).P} & := & \{ x \} \cup (\freenames{P} \setminus \{ y \}) \\
  \freenames{x!\langle P \rangle} & := & \{ x \} \cup \{ P \} \\
  \freenames{P|Q} & := & \freenames{P} \cup \freenames{Q} \\
  \freenames{\dropn{x}} & := & \{ x \}
\end{eqnarray*}

The bound names of a process, $\boundnames{P}$, are those names occurring in $P$
that are not free. For example, in $x?(y).0$, the name $x$ is free, while $y$ is bound.

\begin{mathpar}
  \inferrule* [lab=monoidal-laws] {} { P|Q \equiv Q|P \and P|0 \equiv P \and P|(Q|R) \equiv (P|Q)|R }
\end{mathpar}

\begin{mathpar}
  \inferrule* [lab=alpha-equivalence] {} { (x)P \equiv (y)P\{y/x\} \and y \not\in \freenames{P} }
\end{mathpar}

\begin{definition}
Then two processes, $P,Q$, are alpha-equivalent if $P = Q\{\vec{y}/\vec{x}\}$ for
some $\vec{x} \in \boundnames{Q},\vec{y} \in \boundnames{P}$, where $Q\{\vec{y}/\vec{x}\}$
denotes the capture-avoiding substitution of $\vec{y}$ for $\vec{x}$ in $Q$.
\end{definition}

\begin{definition}
  The {\em structural congruence} \cite{SangiorgiWalker} , $\equiv$,
  between processes is the least congruence containing
  alpha-equivalence, satisfying the abelian monoid laws
  (associativity, commutativity and $\pzero$ as identity) for parallel
  composition $|$ and for summation $+$.
\end{definition}

\subsection{Name equivalence}

We take name equivalence, written $\nameeq$, to be the smallest
equivalence relation generated by the following rules.

\begin{mathpar}
\inferrule*[lab=Quote-drop]
{ }
{ \quotep{@{x}} \nameeq x }

\inferrule*[lab=Struct-equiv]
{ P \scong Q }
{ \quotep{P} \nameeq \quotep{Q} }
\end{mathpar}

The astute reader will have noticed that the mutual recursion of names
and processes imposes a mutual recursion on alpha-equivalence and
structural equivalence via name-equivalence. Fortunately, all of this
works out pleasantly and we may calculate in the natural way, free of
concern. The reader interested in the details is referred to the
appendix \ref{appendix:rho_details}.

\subsection{Substitution}

We use $\Proc$ for the set of processes, $\QProc$ for the set of
names, and $\id{\{}\vec{y} / \vec{x} \id{\}}$ to denote partial maps,
$s : \QProc \rightarrow \QProc$. A map, $s$ lifts, uniquely, to a map
on process terms, $\widehat{s} : \Proc \rightarrow \Proc$ by the
following equations.

\begin{mathpar}
  (0) \psubstp{Q}{P} := 0 \\
  (R \juxtap S) \psubstp{Q}{P}
  :=    
  (R)\psubstp{Q}{P} \juxtap (S) \psubstp{Q}{P} \\
  (x?(y).R) \psubstp{Q}{P}    
  :=    
  (x)\substp{Q}{P} (z)\concat( (R \psubstn{z}{y}) \psubstp{Q}{P} ) \\
  (\lift{x}{R}) \psubstp{Q}{P}  
  :=
  \lift{(x)\substp{Q}{P}}{ R \psubstp{Q}{P} } \\
%   (\dropn{x})  \psubstp{Q}{P}       
%   := 
%   \left\{ 
%     \begin{array}{ccc} 
%       \dropn{\quotep{Q}} & & x \nameeq \quotep{P} \\
%       \dropn{x} & & otherwise \\
%     \end{array}
%   \right. 
  (\dropn{x})  \psubstp{Q}{P}       
  := 
  \left\{ 
    \begin{array}{ccc} 
      Q & & x \nameeq \quotep{P} \\
      \dropn{x} & & otherwise \\
    \end{array}
  \right.
\end{mathpar}
 

where

\begin{eqnarray}
  (x)\id{\{} \lpquote Q \rpquote / \lpquote P \rpquote \id{\}}            = 
  \left\{ 
    \begin{array}{ccc}
      \lpquote Q \rpquote & & x \nameeq \lpquote P \rpquote \\
      x & & otherwise \\
    \end{array}
  \right. \nonumber
\end{eqnarray}

and $z$ is chosen distinct from $\quotep{P}$, $\quotep{Q}$, the free
names in $Q$, and all the names in $R$. Our $\alpha$-equivalence will
be built in the standard way from this substitution.

\begin{remark}\label{rem:no_self_referential_names}
  One consequence of these definitions is that $\forall P. \quotep{P}
  \not\in \freenames{P}$.
\end{remark}

\subsection{ Dynamic quote: an example }

Anticipating something of what's to come, consider applying the
substitution, $\widehat{\id{\{}u / z \id{\}}}$, to the following pair
of processes, $\lift{w}{y!(z)}$ and $w[ \lpquote y!(z) \rpquote ]$.

\begin{eqnarray}
	\lift{w}{y!(z)}\widehat{\id{\{}u / z \id{\}}}
		& = &
		\lift{w}{y!(u)} \nonumber\\
	w[ \lpquote y!(z) \rpquote ] \widehat{ \id{\{}u / z \id{\}} }
		& = &
		w[ \lpquote y!(z) \rpquote ] \nonumber
\end{eqnarray}

Because the body of the process between quotes is impervious to
substitution, we get radically different answers. In fact, by
examining the first process in an input context,
e.g. $x?(z).\lift{w}{y!(z)}$, we see that the process under the lift
operator may be shaped by prefixed inputs binding a name inside it. In
this sense, the lift operator will be seen as a way to dynamically
construct processes before reifying them as names.

Finally equipped with these standard features we can present the
dynamics of the calculus.

\subsubsection{Operational semantics} 

Finally, we introduce the computational dynamics. What marks these
algebras as distinct from other more traditionally studied algebraic
structures, e.g. vector spaces or polynomial rings, is the manner in
which dynamics is captured. In traditional structures, dynamics is typically
expressed through morphisms between such structures, as in linear maps
between vector spaces or morphisms between rings. In algebras
associated with the semantics of computation, the dynamics is
expressed as part of the algebraic structure itself, through a
reduction reduction relation typically denoted by $\red$. Below, we
give a recursive presentation of this relation for the calculus used
in the encoding.

$\red \subseteq \pi \times \pi$
$\red : \pi \to \mathcal{P}(\pi)$

\begin{mathpar}
  \inferrule* [lab=Comm] { \textsf{match}( x_{src}, x_{trgt} ) } { x_{trgt}?(y)P \; | \; x_{src}!\langle {Q} \rangle \red P\{\quotep{Q}/y}\} }
  \and \\
  \inferrule* [lab=Par] {{P} \red {P}'} {{{P} | {Q}} \red {{P}' | {Q}}}
  \and
  \inferrule* [lab=Equiv]{{{P} \scong {P}'} \andalso {{P}' \red {Q}'} \andalso {{Q}' \scong {Q}}}{{P} \red {Q}}
\end{mathpar}

\begin{eqnarray*}
  match_{\equiv} (\quotep{P},\quotep{Q}) & := & P \equiv Q \\
  match_{\dagger}(\quotep{P},\quotep{Q}) & := & \forall R. P|Q \red^{*} R => R \red^{*} 0 \\
  match_{K}(\quotep{P},\quotep{Q}) & := & K \mbox{ for some context } K
\end{eqnarray*}

$u?(x)P | u!\langle Q \rangle \red P\{\quotep{Q}/x\}$

%We write $\wred$ for $\red^*$, and $P\red$ if $\exists Q $ such that $ P \red Q$.
We write $P\red$ if $\exists Q $ such that $ P \red Q$ and $P\not\red$, otherwise.

\section{Replication}

As mentioned before, it is known that replication (and hence
recursion) can be implemented in a higher-order process algebra
\cite{SangiorgiWalker}. As our first example of calculation with the
machinery thus far presented we give the construction explicitly in
the {\rhoc}.

\begin{eqnarray}
	D_{x} & := & \prefix{x}{y}{(\binpar{\outputp{x}{y}}{@{y}})} \nonumber\\
	\bangp_{x}{P} & := & \binpar{{x}!\langle{\binpar{D_{x}}{P}}\rangle}{D_{x}} \nonumber
\end{eqnarray}

\begin{eqnarray}
	\bangp_{x}{P} & & \nonumber\\
	=
	& {x}!\langle{(\prefix{x}{y}{(\outputp{x}{y} | @{y})) | P}}\rangle 
	      | \prefix{x}{y}{(\outputp{x}{y} | @{y})} & \nonumber\\
	\red
	& (\outputp{x}{y} | @{y})\substn{\quotep{(\prefix{x}{y}{(@{y} | \outputp{x}{y})) | P}}}{y} & \nonumber\\
	=
	& \outputp{x}{\quotep{(\prefix{x}{y}{(\outputp{x}{y} | @{y})) | P}}}
	  | {(\prefix{x}{y}{(\outputp{x}{y} | @{y})) | P}} & \nonumber\\
	\red
	& \ldots & \nonumber\\
	\red^*
	& P | P | \ldots & \nonumber
\end{eqnarray}

Of course, this encoding, as an implementation, runs away, unfolding
$\bangp{P}$ eagerly. A lazier and more implementable replication
operator, restricted to input-guarded processes, may be obtained as follows.

\begin{eqnarray}
\bangp{\prefix{u}{v}{P}} 
	:= 
	\binpar{\lift{x}{\prefix{u}{v}{(\binpar{D(x)}{P})}}}{D(x)} \nonumber
\end{eqnarray}

\begin{remark}
  Note that the lazier definition still does not deal with summation
  or mixed summation (i.e. sums over input and output). The reader is
  invited to construct definitions of replication that deal with these
  features. 

  Further, the definitions are parameterized in a name, $x$. Can you,
  gentle reader, make a definition that eliminates this parameter and
  guarantees no accidental interaction between the replication
  machinery and the process being replicated -- i.e. no accidental
  sharing of names used by the process to get its work done and the
  name(s) used by the replication to effect copying. This latter
  revision of the definition of replication is crucial to obtaining
  the expected identity $!!P \sim !P$.
\end{remark}

\begin{remark}\label{rem:paradoxical_combinator}
  The reader familiar with the lambda calculus will have noticed the
  similarity between $D$ and the paradoxical combinator.

  [Ed. note: the existence of this seems to suggest we have to be more
  restrictive on the set of processes and names we admit if we are to
  support no-cloning.]
\end{remark}

\subsubsection{Bisimulation}

The computational dynamics gives rise to another kind of equivalence,
the equivalence of computational behavior. As previously mentioned
this is typically captured \emph{via} some form of bisimulation.

% The notion we use in this paper is weak barbed bisimulation
% \cite{milner91polyadicpi}.

The notion we use in this paper is derived from weak barbed
bisimulation \cite{milner91polyadicpi}. 

\begin{definition}
An \emph{observation relation}, $\downarrow_{\mathcal N}$, over a set
of names, $\mathcal N$, is the smallest relation satisfying the rules
below.

\infrule[Out-barb]{y \in {\mathcal N}, \; x \nameeq y}
		  {\outputp{x}{v} \downarrow_{\mathcal N} x}
\infrule[Par-barb]{\mbox{$P\downarrow_{\mathcal N} x$ or $Q\downarrow_{\mathcal N} x$}}
		  {\binpar{P}{Q} \downarrow_{\mathcal N} x}

We write $P \Downarrow_{\mathcal N} x$ if there is $Q$ such that 
$P \wred Q$ and $Q \downarrow_{\mathcal N} x$.
\end{definition}

\begin{definition}
%\label{def.bbisim}
An  ${\mathcal N}$-\emph{barbed bisimulation} over a set of names, ${\mathcal N}$, is a symmetric binary relation 
${\mathcal S}_{\mathcal N}$ between agents such that $P\rel{S}_{\mathcal N}Q$ implies:
\begin{enumerate}
\item If $P \red P'$ then $Q \wred Q'$ and $P'\rel{S}_{\mathcal N} Q'$.
\item If $P\downarrow_{\mathcal N} x$, then $Q\Downarrow_{\mathcal N} x$.
\end{enumerate}
$P$ is ${\mathcal N}$-barbed bisimilar to $Q$, written
$P \wbbisim_{\mathcal N} Q$, if $P \rel{S}_{\mathcal N} Q$ for some ${\mathcal N}$-barbed bisimulation ${\mathcal S}_{\mathcal N}$.
\end{definition}

$\mathcal{R} \subseteq \pi \times \pi$

$P \mathcal{R} Q => \forall P'. P \red P' \Rightarrow \exists Q'. Q \red Q', P' \mathcal{R} Q'$

$P \vdash x \Rightarrow Q \vdash x$

\begin{mathpar}
  \inferrule*[lab=Out-barb]{x \nameeq y}{{y}!\langle{Q}\rangle \vdash x}
  \and
  \inferrule*[lab=Par-barb]{\mbox{$P\vdash x$ or $Q\vdash x$}}{\binpar{P}{Q} \vdash x}
\end{mathpar}

\subsubsection{Contexts}

One of the principle advantages of computational calculi like the
$\pi$-calculus is a well-defined notion of context,
contextual-equivalence and a correlation between
contextual-equivalence and notions of bisimulation. The notion of
context allows the decomposition of a process into (sub-)process and
its syntactic environment, its context. Thus, a context may be
thought of as a process with a ``hole'' (written $\Box$) in it. The
application of a context $M$ to a process $P$, written $M[P]$, is
tantamount to filling the hole in $M$ with $P$. In this paper we do
not need the full weight of this theory, but do make use of the notion
of context in the proof the main theorem. 

\begin{mathpar}
  \inferrule* [lab=summation] {} {{M_{M},M_{N}} \bc \Box \;|\; x.M_{A} \;|\; M_{M}+M_{N}}
  \and
  \inferrule* [lab=agent] {} {{M_{A}} \bc (\vec{x})M_{P} \;| \; \clift{P_0,\ldots,M_{P},\ldots,P_N}}
  \and \\
  \inferrule* [lab=process] {} {{M_{P}} \bc M_{N} \;| \;P|M_{P} }
\end{mathpar} 

\begin{mathpar}
  \inferrule* [lab=sychronization] {} {M_{N} \bc \Box \;|\; x?M_{F} \;|\; x!M_{C}}
  \and
  \inferrule* [lab=abstraction] {} {{M_{F}} \bc (x)M_{P} }
  \and
  \inferrule* [lab=concretion] {} {{M_{C}} \bc \langle M_{P} \rangle }
  \and \\
  \inferrule* [lab=process] {} {{M_{P}} \bc M_{N} \;| \;P|M_{P} }
\end{mathpar}

\begin{definition}[contextual application] Given a context $M$, and
  process $P$, we define the \emph{contextual application}, $M[P] :=
  M\{P/\Box\}$. That is, the contextual application of M to P is the
  substitution of $P$ for $\Box$ in $M$.
\end{definition}

$\meaningof{-} : L \to \mathcal{P}(\pi)$

\begin{mathpar}
  \inferrule* [lab=collection] {} {\meaningof{true} = \pi, \and \meaningof{~E} = \pi \setminus \meaningof{E}, \and \meaningof{E_{1} \& E_{2}} = \meaningof{E_{1}} \cap \meaningof{E_{2}}}
\end{mathpar}

\begin{mathpar}
  \inferrule* [lab=structure] {} {\meaningof{0} = \{ P \in \pi | P \equiv 0 \}, \and \\ \meaningof{E_1 | E_2} = \{ P \in \pi | P \equiv P_{1} | P_{2}, P_{1} \in \meaningof{E_{1}}, P_{2} \in \meaningof{E_2}\} }
\end{mathpar}

\begin{mathpar}
 \inferrule* [lab=behavior] {} {\meaningof{\langle a?b \rangle E} = \{ P \in \pi | P \equiv Q | u?(y)P', \\ \and \\\\ \and \\ \;\;\; u \in \meaningof{a}, \forall z.P'\{z/y\} \in \meaningof{E\{z/b\}}\}, \and \\ \meaningof{a!E} = \{ P \in \pi | P \equiv Q | x!\langle P' \rangle, x \in \meaningof{a} P' \in \meaningof{E}\} }
\end{mathpar}

\begin{mathpar}
 \inferrule* [lab=nominal] {} {\meaningof{\quotep{E}} = \{ \quotep{P} \in \quotep{\pi} | P \in \meaningof{E} \}, \and \meaningof{\quotep{P}} = \{ \quotep{Q} \in \quotep{\pi} | P \equiv Q \} \and \\ \meaningof{@\quotep{E}} = \{ P \in \pi | P \equiv @x, x \in \meaningof{E} \}}
\end{mathpar}

\begin{eqnarray*}
  \\
  \meaningof{-} : TS \to ST
\end{eqnarray*}

\begin{eqnarray*}
  \\
  L : TS \to ST
\end{eqnarray*}

\begin{eqnarray*}
  \\
  P \models E \iff P \in \meaningof{E}
\end{eqnarray*}

\begin{eqnarray*}
  P \approx_{L} Q \iff \forall E \in L. P \models E \iff Q \models E
\end{eqnarray*}

\begin{eqnarray*}
  P \approx_{K} Q
\end{eqnarray*}

\begin{eqnarray*}
  P \approx Q
\end{eqnarray*}

$\approx_{K} = \approx = \approx_{L}$

\subsubsection{Contextual duality}

Note that contexts extend the quotation operation to a family of
operations from processes to names. Given a context, $M$, we can
define a \emph{nominal context}, $\quotep{M}$ by $\quotep{M}[P] :=
\quotep{M[P]}$. To foreshadow what is to come we observe that these
operations enjoy a duality with processes very much like the duality
between vectors and maps from vectors to scalars.

Further, because the calculus is essentially higher-order, we have a
correspondence between contexts and processes. More specifically,
given a name $x$ and a context $M$ we can construct $M^{*}_{x}$ such
that 

\begin{mathpar}
  M^{*}_{x} | \lift{x}{P} \red M[P]
\end{mathpar}

namely,

\begin{mathpar}
  M^{*}_{x} := x?(u).M[\dropn{u}]
\end{mathpar}

The dependence of $M^{*}_{x}$ on a name makes it an abstraction, 

\begin{mathpar}
  M^{*} := (x)x?(u).M[\dropn{u}]
\end{mathpar}

\subsection{Additional notation}

It will sometimes be convenient to denote the process a name
quotes. We already have the notation $x = \quotep{P}$, but it will be
convenient to introduce an alternate notation, $\procn{x}$, when we
want to emphasize the connection to the use of the name. Note that, by
virtue of name equivalence, $\quotep{\procn{x}} \nameeq x$; so, the
notation is consistent with previous definitions.

Further, because names have structure it is possible to effect
substitutions on the basis of that structure. This means we need to
upgrade our notation for substitutions, which we accomplish by
adapting comprehension notation. Thus,

\begin{mathpar}
  P\{ y / x : x \in S \}
\end{mathpar}

is interpreted to mean the process derived from P by replacing (in a
capture-avoiding manner) each occurrence of $x$ in $S$ by $y$. For example,

\begin{mathpar}
  P\{ \quotep{\procn{x}|\procn{x}} / x : x \in \freenames{P} \}
\end{mathpar}

will replace each (occurrence) of a free name $x$ in $P$ by
$\quotep{\procn{x}|\procn{x}}$.

Also, we will avail ourselves of the notation $x^{L}$ and $x^{R}$ to
denote injections of a name into disjoint copies of the name
space. There are numerous ways to accomplish this. One example can be
found in \cite{MeredithR05}. This notation overloads to vectors of
names: $\vec{x}^{\pi} := (x_{i}^{\pi} \; : \; 0 \leq i < |\vec{x}| )$ where $\pi \in \{L,R\}$.

We also use $P^{\Box} := P|\Box$.

In \cite{MeredithR05} an interpretation of the new operator is
given. It turns out that there are several possible interpretations
all enjoying the requisite algebraic properties of the operator (see
\cite{milner91polyadicpi}). We will therefore make liberal use of
$(\nu\; \vec{x})P$.

% subsection the_syntax_and_semantics_of_the_notation_system (end)   

\input{qm2pi.qmops} 

\input{qm2pi.sterngerlach} 

\input{qm2pi.metric} 

% section concurrent_process_calculi (end)

%\input{qm2pi.proofsketch}

% section proof sketch (end)

%\input{qm2pi.slviaknots} 

% section spatial logic via knots (end)

\input{qm2pi.conclusion}

% section conclusion (end)

%\input{qm2pi.dtcodes} 

% section wiring algorithm (end)

\input{qm2pi.ack} 

% section acknowledgments (end)

\newpage


\bibliographystyle{plain}   
\bibliography{../../biblios/main.bib}

\input{qm2pi.rhodetails}

\end{document}

 

% section wiring algorithm (end)

\documentclass[12pt]{llncs}
%\documentclass{jktr}

\usepackage[pdftex]{hyperref}                   
\usepackage {listings}
\usepackage {mathpartir}
\usepackage{bcprules}
%\usepackage{listings}
                       
\usepackage{graphicx} 
%\usepackage[margins=2.5cm,nohead,nofoot]{geometry}
%\usepackage{geometry}
\usepackage{amsfonts}
\usepackage{amstext}
\usepackage{latexsym}
\usepackage{amssymb}
\usepackage{color}


%\include{myPreamble}
\include{qm2pi.local} 

%\ifpdf
%\usepackage[pdftex]{graphicx}
%\else
%\usepackage{graphicx}
%\fi

 % \ifpdf
%  \usepackage{pdfsync}
%  \if


%\title{Brief Article}
%\author{David F. Snyder}
%\author{L.G. Meredith}

%\address{Dept. of Math., Texas State University--San Marcos, San Marcos, TX 78666}
       
\pagestyle{empty}


\begin{document}

\lstset{language=[Objective]Caml,frame=shadowbox}

\input{qm2pi.front}

% section front matter (end)

\input{qm2pi.intro} 
 
% section introduction (end)

% \input{qm2pi.knotations} 

% section notation (end)

\input{qm2pi.process.calculi} 

% section concurrent_process_calculi_and_spatial_logics_ (end)
    
%\input{qm2pi.knots2pi} 

%\input{qm2pi.trefoil} 

%\input{qm2pi.mainthm} 

% subsection basic_interpretation (end)

%\input{qm2pi.rho.presentation} 
\subsection{The syntax and semantics of the notation system}\label{sub:the_syntax_and_semantics_of_the_notation_system} % (fold)

We now summarize a technical presentation of the calculus that
embodies our theory of dynamics. The typical presentation of such a
calculus follows the style of giving generators and relations on
them. The grammar, below, describing term constructors, freely
generates the set of processes, $\Proc$. This set is then quotiented
by a relation known as structural congruence and it is over this set
that the notion of dynamics is expressed. This presentation is
essentially that of \cite{MeredithR05} with the addition of
polyadicity and summation. For readability we have relegated some of
the technical subtleties to an appendix.

\subsubsection{Process grammar}\label{subsub:process_grammar}

\begin{mathpar}
  \inferrule* [lab=synchronization] {} {{M} \bc \pzero \;|\; x?F \;|\; x!C }
  \and
  \inferrule* [lab=abstraction] {} {{F} \bc (x)P}
  \and
  \inferrule* [lab=concretion] {} {{C} \bc \langle Q \rangle}
  \and
  \inferrule* [lab=process] {} {{P,Q} \bc M \;| \;P|Q \;|\; @{x}}
  \and
  \inferrule* [lab=name] {} {{x} \bc \quotep{P}}
\end{mathpar} 

Note that $\vec{x}$ (resp. $\vec{P}$) denotes a vector of names
(resp. processes) of length $|\vec{x}|$ (resp. $|\vec{P}|$). We adopt
the following useful abbreviations.

\begin{mathpar}
   x?(\vec{y}).P := x.(\vec{y})P \and  x\clift{\vec{P}} := x.\clift{\vec{P}}
   \and x!(y) := \lift{x}{\dropn{y}}
   \and \Pi_{i=0}^{n-1}P_i := P_0 | \ldots | P_{n-1}
\end{mathpar}

\subsubsection{Structural congruence}

\paragraph{Free and bound names and alpha-equivalence.} At the
core of structural equivalence is alpha-equivalence which identifies
process that are the same up to a change of variable. Formally, we
recognize the distinction between free and bound names. The free names
of a process, $\freenames{P}$, may be calculated recursively as
follows:

\begin{mathpar}
\freenames{\pzero} := \emptyset
  \and \\
  \freenames{x?(y).P} := \{ x \} \cup (\freenames{P} \setminus \{ y \})
  \and 
  \freenames{x!\langle P \rangle} := \{ x \} \cup \{ P \} 
  \and \\
  \freenames{P|Q} := \freenames{P} \cup \freenames{Q}
  \and \\
  \freenames{@{x}} := \{ x \}
\end{mathpar}

$\pi$
$\quotep{\pi}$

$\freenames{-} : \pi \to \mathcal{P}(\quotep{\pi})$

\begin{eqnarray*}
  \freenames{\pzero} & := & \emptyset \\
  \freenames{x?(y).P} & := & \{ x \} \cup (\freenames{P} \setminus \{ y \}) \\
  \freenames{x!\langle P \rangle} & := & \{ x \} \cup \{ P \} \\
  \freenames{P|Q} & := & \freenames{P} \cup \freenames{Q} \\
  \freenames{\dropn{x}} & := & \{ x \}
\end{eqnarray*}

The bound names of a process, $\boundnames{P}$, are those names occurring in $P$
that are not free. For example, in $x?(y).0$, the name $x$ is free, while $y$ is bound.

\begin{mathpar}
  \inferrule* [lab=monoidal-laws] {} { P|Q \equiv Q|P \and P|0 \equiv P \and P|(Q|R) \equiv (P|Q)|R }
\end{mathpar}

\begin{mathpar}
  \inferrule* [lab=alpha-equivalence] {} { (x)P \equiv (y)P\{y/x\} \and y \not\in \freenames{P} }
\end{mathpar}

\begin{definition}
Then two processes, $P,Q$, are alpha-equivalent if $P = Q\{\vec{y}/\vec{x}\}$ for
some $\vec{x} \in \boundnames{Q},\vec{y} \in \boundnames{P}$, where $Q\{\vec{y}/\vec{x}\}$
denotes the capture-avoiding substitution of $\vec{y}$ for $\vec{x}$ in $Q$.
\end{definition}

\begin{definition}
  The {\em structural congruence} \cite{SangiorgiWalker} , $\equiv$,
  between processes is the least congruence containing
  alpha-equivalence, satisfying the abelian monoid laws
  (associativity, commutativity and $\pzero$ as identity) for parallel
  composition $|$ and for summation $+$.
\end{definition}

\subsection{Name equivalence}

We take name equivalence, written $\nameeq$, to be the smallest
equivalence relation generated by the following rules.

\begin{mathpar}
\inferrule*[lab=Quote-drop]
{ }
{ \quotep{@{x}} \nameeq x }

\inferrule*[lab=Struct-equiv]
{ P \scong Q }
{ \quotep{P} \nameeq \quotep{Q} }
\end{mathpar}

The astute reader will have noticed that the mutual recursion of names
and processes imposes a mutual recursion on alpha-equivalence and
structural equivalence via name-equivalence. Fortunately, all of this
works out pleasantly and we may calculate in the natural way, free of
concern. The reader interested in the details is referred to the
appendix \ref{appendix:rho_details}.

\subsection{Substitution}

We use $\Proc$ for the set of processes, $\QProc$ for the set of
names, and $\id{\{}\vec{y} / \vec{x} \id{\}}$ to denote partial maps,
$s : \QProc \rightarrow \QProc$. A map, $s$ lifts, uniquely, to a map
on process terms, $\widehat{s} : \Proc \rightarrow \Proc$ by the
following equations.

\begin{mathpar}
  (0) \psubstp{Q}{P} := 0 \\
  (R \juxtap S) \psubstp{Q}{P}
  :=    
  (R)\psubstp{Q}{P} \juxtap (S) \psubstp{Q}{P} \\
  (x?(y).R) \psubstp{Q}{P}    
  :=    
  (x)\substp{Q}{P} (z)\concat( (R \psubstn{z}{y}) \psubstp{Q}{P} ) \\
  (\lift{x}{R}) \psubstp{Q}{P}  
  :=
  \lift{(x)\substp{Q}{P}}{ R \psubstp{Q}{P} } \\
%   (\dropn{x})  \psubstp{Q}{P}       
%   := 
%   \left\{ 
%     \begin{array}{ccc} 
%       \dropn{\quotep{Q}} & & x \nameeq \quotep{P} \\
%       \dropn{x} & & otherwise \\
%     \end{array}
%   \right. 
  (\dropn{x})  \psubstp{Q}{P}       
  := 
  \left\{ 
    \begin{array}{ccc} 
      Q & & x \nameeq \quotep{P} \\
      \dropn{x} & & otherwise \\
    \end{array}
  \right.
\end{mathpar}
 

where

\begin{eqnarray}
  (x)\id{\{} \lpquote Q \rpquote / \lpquote P \rpquote \id{\}}            = 
  \left\{ 
    \begin{array}{ccc}
      \lpquote Q \rpquote & & x \nameeq \lpquote P \rpquote \\
      x & & otherwise \\
    \end{array}
  \right. \nonumber
\end{eqnarray}

and $z$ is chosen distinct from $\quotep{P}$, $\quotep{Q}$, the free
names in $Q$, and all the names in $R$. Our $\alpha$-equivalence will
be built in the standard way from this substitution.

\begin{remark}\label{rem:no_self_referential_names}
  One consequence of these definitions is that $\forall P. \quotep{P}
  \not\in \freenames{P}$.
\end{remark}

\subsection{ Dynamic quote: an example }

Anticipating something of what's to come, consider applying the
substitution, $\widehat{\id{\{}u / z \id{\}}}$, to the following pair
of processes, $\lift{w}{y!(z)}$ and $w[ \lpquote y!(z) \rpquote ]$.

\begin{eqnarray}
	\lift{w}{y!(z)}\widehat{\id{\{}u / z \id{\}}}
		& = &
		\lift{w}{y!(u)} \nonumber\\
	w[ \lpquote y!(z) \rpquote ] \widehat{ \id{\{}u / z \id{\}} }
		& = &
		w[ \lpquote y!(z) \rpquote ] \nonumber
\end{eqnarray}

Because the body of the process between quotes is impervious to
substitution, we get radically different answers. In fact, by
examining the first process in an input context,
e.g. $x?(z).\lift{w}{y!(z)}$, we see that the process under the lift
operator may be shaped by prefixed inputs binding a name inside it. In
this sense, the lift operator will be seen as a way to dynamically
construct processes before reifying them as names.

Finally equipped with these standard features we can present the
dynamics of the calculus.

\subsubsection{Operational semantics} 

Finally, we introduce the computational dynamics. What marks these
algebras as distinct from other more traditionally studied algebraic
structures, e.g. vector spaces or polynomial rings, is the manner in
which dynamics is captured. In traditional structures, dynamics is typically
expressed through morphisms between such structures, as in linear maps
between vector spaces or morphisms between rings. In algebras
associated with the semantics of computation, the dynamics is
expressed as part of the algebraic structure itself, through a
reduction reduction relation typically denoted by $\red$. Below, we
give a recursive presentation of this relation for the calculus used
in the encoding.

$\red \subseteq \pi \times \pi$
$\red : \pi \to \mathcal{P}(\pi)$

\begin{mathpar}
  \inferrule* [lab=Comm] { \textsf{match}( x_{src}, x_{trgt} ) } { x_{trgt}?(y)P \; | \; x_{src}!\langle {Q} \rangle \red P\{\quotep{Q}/y}\} }
  \and \\
  \inferrule* [lab=Par] {{P} \red {P}'} {{{P} | {Q}} \red {{P}' | {Q}}}
  \and
  \inferrule* [lab=Equiv]{{{P} \scong {P}'} \andalso {{P}' \red {Q}'} \andalso {{Q}' \scong {Q}}}{{P} \red {Q}}
\end{mathpar}

\begin{eqnarray*}
  match_{\equiv} (\quotep{P},\quotep{Q}) & := & P \equiv Q \\
  match_{\dagger}(\quotep{P},\quotep{Q}) & := & \forall R. P|Q \red^{*} R => R \red^{*} 0 \\
  match_{K}(\quotep{P},\quotep{Q}) & := & K \mbox{ for some context } K
\end{eqnarray*}

$u?(x)P | u!\langle Q \rangle \red P\{\quotep{Q}/x\}$

%We write $\wred$ for $\red^*$, and $P\red$ if $\exists Q $ such that $ P \red Q$.
We write $P\red$ if $\exists Q $ such that $ P \red Q$ and $P\not\red$, otherwise.

\section{Replication}

As mentioned before, it is known that replication (and hence
recursion) can be implemented in a higher-order process algebra
\cite{SangiorgiWalker}. As our first example of calculation with the
machinery thus far presented we give the construction explicitly in
the {\rhoc}.

\begin{eqnarray}
	D_{x} & := & \prefix{x}{y}{(\binpar{\outputp{x}{y}}{@{y}})} \nonumber\\
	\bangp_{x}{P} & := & \binpar{{x}!\langle{\binpar{D_{x}}{P}}\rangle}{D_{x}} \nonumber
\end{eqnarray}

\begin{eqnarray}
	\bangp_{x}{P} & & \nonumber\\
	=
	& {x}!\langle{(\prefix{x}{y}{(\outputp{x}{y} | @{y})) | P}}\rangle 
	      | \prefix{x}{y}{(\outputp{x}{y} | @{y})} & \nonumber\\
	\red
	& (\outputp{x}{y} | @{y})\substn{\quotep{(\prefix{x}{y}{(@{y} | \outputp{x}{y})) | P}}}{y} & \nonumber\\
	=
	& \outputp{x}{\quotep{(\prefix{x}{y}{(\outputp{x}{y} | @{y})) | P}}}
	  | {(\prefix{x}{y}{(\outputp{x}{y} | @{y})) | P}} & \nonumber\\
	\red
	& \ldots & \nonumber\\
	\red^*
	& P | P | \ldots & \nonumber
\end{eqnarray}

Of course, this encoding, as an implementation, runs away, unfolding
$\bangp{P}$ eagerly. A lazier and more implementable replication
operator, restricted to input-guarded processes, may be obtained as follows.

\begin{eqnarray}
\bangp{\prefix{u}{v}{P}} 
	:= 
	\binpar{\lift{x}{\prefix{u}{v}{(\binpar{D(x)}{P})}}}{D(x)} \nonumber
\end{eqnarray}

\begin{remark}
  Note that the lazier definition still does not deal with summation
  or mixed summation (i.e. sums over input and output). The reader is
  invited to construct definitions of replication that deal with these
  features. 

  Further, the definitions are parameterized in a name, $x$. Can you,
  gentle reader, make a definition that eliminates this parameter and
  guarantees no accidental interaction between the replication
  machinery and the process being replicated -- i.e. no accidental
  sharing of names used by the process to get its work done and the
  name(s) used by the replication to effect copying. This latter
  revision of the definition of replication is crucial to obtaining
  the expected identity $!!P \sim !P$.
\end{remark}

\begin{remark}\label{rem:paradoxical_combinator}
  The reader familiar with the lambda calculus will have noticed the
  similarity between $D$ and the paradoxical combinator.

  [Ed. note: the existence of this seems to suggest we have to be more
  restrictive on the set of processes and names we admit if we are to
  support no-cloning.]
\end{remark}

\subsubsection{Bisimulation}

The computational dynamics gives rise to another kind of equivalence,
the equivalence of computational behavior. As previously mentioned
this is typically captured \emph{via} some form of bisimulation.

% The notion we use in this paper is weak barbed bisimulation
% \cite{milner91polyadicpi}.

The notion we use in this paper is derived from weak barbed
bisimulation \cite{milner91polyadicpi}. 

\begin{definition}
An \emph{observation relation}, $\downarrow_{\mathcal N}$, over a set
of names, $\mathcal N$, is the smallest relation satisfying the rules
below.

\infrule[Out-barb]{y \in {\mathcal N}, \; x \nameeq y}
		  {\outputp{x}{v} \downarrow_{\mathcal N} x}
\infrule[Par-barb]{\mbox{$P\downarrow_{\mathcal N} x$ or $Q\downarrow_{\mathcal N} x$}}
		  {\binpar{P}{Q} \downarrow_{\mathcal N} x}

We write $P \Downarrow_{\mathcal N} x$ if there is $Q$ such that 
$P \wred Q$ and $Q \downarrow_{\mathcal N} x$.
\end{definition}

\begin{definition}
%\label{def.bbisim}
An  ${\mathcal N}$-\emph{barbed bisimulation} over a set of names, ${\mathcal N}$, is a symmetric binary relation 
${\mathcal S}_{\mathcal N}$ between agents such that $P\rel{S}_{\mathcal N}Q$ implies:
\begin{enumerate}
\item If $P \red P'$ then $Q \wred Q'$ and $P'\rel{S}_{\mathcal N} Q'$.
\item If $P\downarrow_{\mathcal N} x$, then $Q\Downarrow_{\mathcal N} x$.
\end{enumerate}
$P$ is ${\mathcal N}$-barbed bisimilar to $Q$, written
$P \wbbisim_{\mathcal N} Q$, if $P \rel{S}_{\mathcal N} Q$ for some ${\mathcal N}$-barbed bisimulation ${\mathcal S}_{\mathcal N}$.
\end{definition}

$\mathcal{R} \subseteq \pi \times \pi$

$P \mathcal{R} Q => \forall P'. P \red P' \Rightarrow \exists Q'. Q \red Q', P' \mathcal{R} Q'$

$P \vdash x \Rightarrow Q \vdash x$

\begin{mathpar}
  \inferrule*[lab=Out-barb]{x \nameeq y}{{y}!\langle{Q}\rangle \vdash x}
  \and
  \inferrule*[lab=Par-barb]{\mbox{$P\vdash x$ or $Q\vdash x$}}{\binpar{P}{Q} \vdash x}
\end{mathpar}

\subsubsection{Contexts}

One of the principle advantages of computational calculi like the
$\pi$-calculus is a well-defined notion of context,
contextual-equivalence and a correlation between
contextual-equivalence and notions of bisimulation. The notion of
context allows the decomposition of a process into (sub-)process and
its syntactic environment, its context. Thus, a context may be
thought of as a process with a ``hole'' (written $\Box$) in it. The
application of a context $M$ to a process $P$, written $M[P]$, is
tantamount to filling the hole in $M$ with $P$. In this paper we do
not need the full weight of this theory, but do make use of the notion
of context in the proof the main theorem. 

\begin{mathpar}
  \inferrule* [lab=summation] {} {{M_{M},M_{N}} \bc \Box \;|\; x.M_{A} \;|\; M_{M}+M_{N}}
  \and
  \inferrule* [lab=agent] {} {{M_{A}} \bc (\vec{x})M_{P} \;| \; \clift{P_0,\ldots,M_{P},\ldots,P_N}}
  \and \\
  \inferrule* [lab=process] {} {{M_{P}} \bc M_{N} \;| \;P|M_{P} }
\end{mathpar} 

\begin{mathpar}
  \inferrule* [lab=sychronization] {} {M_{N} \bc \Box \;|\; x?M_{F} \;|\; x!M_{C}}
  \and
  \inferrule* [lab=abstraction] {} {{M_{F}} \bc (x)M_{P} }
  \and
  \inferrule* [lab=concretion] {} {{M_{C}} \bc \langle M_{P} \rangle }
  \and \\
  \inferrule* [lab=process] {} {{M_{P}} \bc M_{N} \;| \;P|M_{P} }
\end{mathpar}

\begin{definition}[contextual application] Given a context $M$, and
  process $P$, we define the \emph{contextual application}, $M[P] :=
  M\{P/\Box\}$. That is, the contextual application of M to P is the
  substitution of $P$ for $\Box$ in $M$.
\end{definition}

$\meaningof{-} : L \to \mathcal{P}(\pi)$

\begin{mathpar}
  \inferrule* [lab=collection] {} {\meaningof{true} = \pi, \and \meaningof{~E} = \pi \setminus \meaningof{E}, \and \meaningof{E_{1} \& E_{2}} = \meaningof{E_{1}} \cap \meaningof{E_{2}}}
\end{mathpar}

\begin{mathpar}
  \inferrule* [lab=structure] {} {\meaningof{0} = \{ P \in \pi | P \equiv 0 \}, \and \\ \meaningof{E_1 | E_2} = \{ P \in \pi | P \equiv P_{1} | P_{2}, P_{1} \in \meaningof{E_{1}}, P_{2} \in \meaningof{E_2}\} }
\end{mathpar}

\begin{mathpar}
 \inferrule* [lab=behavior] {} {\meaningof{\langle a?b \rangle E} = \{ P \in \pi | P \equiv Q | u?(y)P', \\ \and \\\\ \and \\ \;\;\; u \in \meaningof{a}, \forall z.P'\{z/y\} \in \meaningof{E\{z/b\}}\}, \and \\ \meaningof{a!E} = \{ P \in \pi | P \equiv Q | x!\langle P' \rangle, x \in \meaningof{a} P' \in \meaningof{E}\} }
\end{mathpar}

\begin{mathpar}
 \inferrule* [lab=nominal] {} {\meaningof{\quotep{E}} = \{ \quotep{P} \in \quotep{\pi} | P \in \meaningof{E} \}, \and \meaningof{\quotep{P}} = \{ \quotep{Q} \in \quotep{\pi} | P \equiv Q \} \and \\ \meaningof{@\quotep{E}} = \{ P \in \pi | P \equiv @x, x \in \meaningof{E} \}}
\end{mathpar}

\begin{eqnarray*}
  \\
  \meaningof{-} : TS \to ST
\end{eqnarray*}

\begin{eqnarray*}
  \\
  L : TS \to ST
\end{eqnarray*}

\begin{eqnarray*}
  \\
  P \models E \iff P \in \meaningof{E}
\end{eqnarray*}

\begin{eqnarray*}
  P \approx_{L} Q \iff \forall E \in L. P \models E \iff Q \models E
\end{eqnarray*}

\begin{eqnarray*}
  P \approx_{K} Q
\end{eqnarray*}

\begin{eqnarray*}
  P \approx Q
\end{eqnarray*}

$\approx_{K} = \approx = \approx_{L}$

\subsubsection{Contextual duality}

Note that contexts extend the quotation operation to a family of
operations from processes to names. Given a context, $M$, we can
define a \emph{nominal context}, $\quotep{M}$ by $\quotep{M}[P] :=
\quotep{M[P]}$. To foreshadow what is to come we observe that these
operations enjoy a duality with processes very much like the duality
between vectors and maps from vectors to scalars.

Further, because the calculus is essentially higher-order, we have a
correspondence between contexts and processes. More specifically,
given a name $x$ and a context $M$ we can construct $M^{*}_{x}$ such
that 

\begin{mathpar}
  M^{*}_{x} | \lift{x}{P} \red M[P]
\end{mathpar}

namely,

\begin{mathpar}
  M^{*}_{x} := x?(u).M[\dropn{u}]
\end{mathpar}

The dependence of $M^{*}_{x}$ on a name makes it an abstraction, 

\begin{mathpar}
  M^{*} := (x)x?(u).M[\dropn{u}]
\end{mathpar}

\subsection{Additional notation}

It will sometimes be convenient to denote the process a name
quotes. We already have the notation $x = \quotep{P}$, but it will be
convenient to introduce an alternate notation, $\procn{x}$, when we
want to emphasize the connection to the use of the name. Note that, by
virtue of name equivalence, $\quotep{\procn{x}} \nameeq x$; so, the
notation is consistent with previous definitions.

Further, because names have structure it is possible to effect
substitutions on the basis of that structure. This means we need to
upgrade our notation for substitutions, which we accomplish by
adapting comprehension notation. Thus,

\begin{mathpar}
  P\{ y / x : x \in S \}
\end{mathpar}

is interpreted to mean the process derived from P by replacing (in a
capture-avoiding manner) each occurrence of $x$ in $S$ by $y$. For example,

\begin{mathpar}
  P\{ \quotep{\procn{x}|\procn{x}} / x : x \in \freenames{P} \}
\end{mathpar}

will replace each (occurrence) of a free name $x$ in $P$ by
$\quotep{\procn{x}|\procn{x}}$.

Also, we will avail ourselves of the notation $x^{L}$ and $x^{R}$ to
denote injections of a name into disjoint copies of the name
space. There are numerous ways to accomplish this. One example can be
found in \cite{MeredithR05}. This notation overloads to vectors of
names: $\vec{x}^{\pi} := (x_{i}^{\pi} \; : \; 0 \leq i < |\vec{x}| )$ where $\pi \in \{L,R\}$.

We also use $P^{\Box} := P|\Box$.

In \cite{MeredithR05} an interpretation of the new operator is
given. It turns out that there are several possible interpretations
all enjoying the requisite algebraic properties of the operator (see
\cite{milner91polyadicpi}). We will therefore make liberal use of
$(\nu\; \vec{x})P$.

% subsection the_syntax_and_semantics_of_the_notation_system (end)   

\input{qm2pi.qmops} 

\input{qm2pi.sterngerlach} 

\input{qm2pi.metric} 

% section concurrent_process_calculi (end)

%\input{qm2pi.proofsketch}

% section proof sketch (end)

%\input{qm2pi.slviaknots} 

% section spatial logic via knots (end)

\input{qm2pi.conclusion}

% section conclusion (end)

%\input{qm2pi.dtcodes} 

% section wiring algorithm (end)

\input{qm2pi.ack} 

% section acknowledgments (end)

\newpage


\bibliographystyle{plain}   
\bibliography{../../biblios/main.bib}

\input{qm2pi.rhodetails}

\end{document}

 

% section acknowledgments (end)

\newpage


\bibliographystyle{plain}   
\bibliography{../../biblios/main.bib}

\documentclass[12pt]{llncs}
%\documentclass{jktr}

\usepackage[pdftex]{hyperref}                   
\usepackage {listings}
\usepackage {mathpartir}
\usepackage{bcprules}
%\usepackage{listings}
                       
\usepackage{graphicx} 
%\usepackage[margins=2.5cm,nohead,nofoot]{geometry}
%\usepackage{geometry}
\usepackage{amsfonts}
\usepackage{amstext}
\usepackage{latexsym}
\usepackage{amssymb}
\usepackage{color}


%\include{myPreamble}
\include{qm2pi.local} 

%\ifpdf
%\usepackage[pdftex]{graphicx}
%\else
%\usepackage{graphicx}
%\fi

 % \ifpdf
%  \usepackage{pdfsync}
%  \if


%\title{Brief Article}
%\author{David F. Snyder}
%\author{L.G. Meredith}

%\address{Dept. of Math., Texas State University--San Marcos, San Marcos, TX 78666}
       
\pagestyle{empty}


\begin{document}

\lstset{language=[Objective]Caml,frame=shadowbox}

\input{qm2pi.front}

% section front matter (end)

\input{qm2pi.intro} 
 
% section introduction (end)

% \input{qm2pi.knotations} 

% section notation (end)

\input{qm2pi.process.calculi} 

% section concurrent_process_calculi_and_spatial_logics_ (end)
    
%\input{qm2pi.knots2pi} 

%\input{qm2pi.trefoil} 

%\input{qm2pi.mainthm} 

% subsection basic_interpretation (end)

%\input{qm2pi.rho.presentation} 
\subsection{The syntax and semantics of the notation system}\label{sub:the_syntax_and_semantics_of_the_notation_system} % (fold)

We now summarize a technical presentation of the calculus that
embodies our theory of dynamics. The typical presentation of such a
calculus follows the style of giving generators and relations on
them. The grammar, below, describing term constructors, freely
generates the set of processes, $\Proc$. This set is then quotiented
by a relation known as structural congruence and it is over this set
that the notion of dynamics is expressed. This presentation is
essentially that of \cite{MeredithR05} with the addition of
polyadicity and summation. For readability we have relegated some of
the technical subtleties to an appendix.

\subsubsection{Process grammar}\label{subsub:process_grammar}

\begin{mathpar}
  \inferrule* [lab=synchronization] {} {{M} \bc \pzero \;|\; x?F \;|\; x!C }
  \and
  \inferrule* [lab=abstraction] {} {{F} \bc (x)P}
  \and
  \inferrule* [lab=concretion] {} {{C} \bc \langle Q \rangle}
  \and
  \inferrule* [lab=process] {} {{P,Q} \bc M \;| \;P|Q \;|\; @{x}}
  \and
  \inferrule* [lab=name] {} {{x} \bc \quotep{P}}
\end{mathpar} 

Note that $\vec{x}$ (resp. $\vec{P}$) denotes a vector of names
(resp. processes) of length $|\vec{x}|$ (resp. $|\vec{P}|$). We adopt
the following useful abbreviations.

\begin{mathpar}
   x?(\vec{y}).P := x.(\vec{y})P \and  x\clift{\vec{P}} := x.\clift{\vec{P}}
   \and x!(y) := \lift{x}{\dropn{y}}
   \and \Pi_{i=0}^{n-1}P_i := P_0 | \ldots | P_{n-1}
\end{mathpar}

\subsubsection{Structural congruence}

\paragraph{Free and bound names and alpha-equivalence.} At the
core of structural equivalence is alpha-equivalence which identifies
process that are the same up to a change of variable. Formally, we
recognize the distinction between free and bound names. The free names
of a process, $\freenames{P}$, may be calculated recursively as
follows:

\begin{mathpar}
\freenames{\pzero} := \emptyset
  \and \\
  \freenames{x?(y).P} := \{ x \} \cup (\freenames{P} \setminus \{ y \})
  \and 
  \freenames{x!\langle P \rangle} := \{ x \} \cup \{ P \} 
  \and \\
  \freenames{P|Q} := \freenames{P} \cup \freenames{Q}
  \and \\
  \freenames{@{x}} := \{ x \}
\end{mathpar}

$\pi$
$\quotep{\pi}$

$\freenames{-} : \pi \to \mathcal{P}(\quotep{\pi})$

\begin{eqnarray*}
  \freenames{\pzero} & := & \emptyset \\
  \freenames{x?(y).P} & := & \{ x \} \cup (\freenames{P} \setminus \{ y \}) \\
  \freenames{x!\langle P \rangle} & := & \{ x \} \cup \{ P \} \\
  \freenames{P|Q} & := & \freenames{P} \cup \freenames{Q} \\
  \freenames{\dropn{x}} & := & \{ x \}
\end{eqnarray*}

The bound names of a process, $\boundnames{P}$, are those names occurring in $P$
that are not free. For example, in $x?(y).0$, the name $x$ is free, while $y$ is bound.

\begin{mathpar}
  \inferrule* [lab=monoidal-laws] {} { P|Q \equiv Q|P \and P|0 \equiv P \and P|(Q|R) \equiv (P|Q)|R }
\end{mathpar}

\begin{mathpar}
  \inferrule* [lab=alpha-equivalence] {} { (x)P \equiv (y)P\{y/x\} \and y \not\in \freenames{P} }
\end{mathpar}

\begin{definition}
Then two processes, $P,Q$, are alpha-equivalent if $P = Q\{\vec{y}/\vec{x}\}$ for
some $\vec{x} \in \boundnames{Q},\vec{y} \in \boundnames{P}$, where $Q\{\vec{y}/\vec{x}\}$
denotes the capture-avoiding substitution of $\vec{y}$ for $\vec{x}$ in $Q$.
\end{definition}

\begin{definition}
  The {\em structural congruence} \cite{SangiorgiWalker} , $\equiv$,
  between processes is the least congruence containing
  alpha-equivalence, satisfying the abelian monoid laws
  (associativity, commutativity and $\pzero$ as identity) for parallel
  composition $|$ and for summation $+$.
\end{definition}

\subsection{Name equivalence}

We take name equivalence, written $\nameeq$, to be the smallest
equivalence relation generated by the following rules.

\begin{mathpar}
\inferrule*[lab=Quote-drop]
{ }
{ \quotep{@{x}} \nameeq x }

\inferrule*[lab=Struct-equiv]
{ P \scong Q }
{ \quotep{P} \nameeq \quotep{Q} }
\end{mathpar}

The astute reader will have noticed that the mutual recursion of names
and processes imposes a mutual recursion on alpha-equivalence and
structural equivalence via name-equivalence. Fortunately, all of this
works out pleasantly and we may calculate in the natural way, free of
concern. The reader interested in the details is referred to the
appendix \ref{appendix:rho_details}.

\subsection{Substitution}

We use $\Proc$ for the set of processes, $\QProc$ for the set of
names, and $\id{\{}\vec{y} / \vec{x} \id{\}}$ to denote partial maps,
$s : \QProc \rightarrow \QProc$. A map, $s$ lifts, uniquely, to a map
on process terms, $\widehat{s} : \Proc \rightarrow \Proc$ by the
following equations.

\begin{mathpar}
  (0) \psubstp{Q}{P} := 0 \\
  (R \juxtap S) \psubstp{Q}{P}
  :=    
  (R)\psubstp{Q}{P} \juxtap (S) \psubstp{Q}{P} \\
  (x?(y).R) \psubstp{Q}{P}    
  :=    
  (x)\substp{Q}{P} (z)\concat( (R \psubstn{z}{y}) \psubstp{Q}{P} ) \\
  (\lift{x}{R}) \psubstp{Q}{P}  
  :=
  \lift{(x)\substp{Q}{P}}{ R \psubstp{Q}{P} } \\
%   (\dropn{x})  \psubstp{Q}{P}       
%   := 
%   \left\{ 
%     \begin{array}{ccc} 
%       \dropn{\quotep{Q}} & & x \nameeq \quotep{P} \\
%       \dropn{x} & & otherwise \\
%     \end{array}
%   \right. 
  (\dropn{x})  \psubstp{Q}{P}       
  := 
  \left\{ 
    \begin{array}{ccc} 
      Q & & x \nameeq \quotep{P} \\
      \dropn{x} & & otherwise \\
    \end{array}
  \right.
\end{mathpar}
 

where

\begin{eqnarray}
  (x)\id{\{} \lpquote Q \rpquote / \lpquote P \rpquote \id{\}}            = 
  \left\{ 
    \begin{array}{ccc}
      \lpquote Q \rpquote & & x \nameeq \lpquote P \rpquote \\
      x & & otherwise \\
    \end{array}
  \right. \nonumber
\end{eqnarray}

and $z$ is chosen distinct from $\quotep{P}$, $\quotep{Q}$, the free
names in $Q$, and all the names in $R$. Our $\alpha$-equivalence will
be built in the standard way from this substitution.

\begin{remark}\label{rem:no_self_referential_names}
  One consequence of these definitions is that $\forall P. \quotep{P}
  \not\in \freenames{P}$.
\end{remark}

\subsection{ Dynamic quote: an example }

Anticipating something of what's to come, consider applying the
substitution, $\widehat{\id{\{}u / z \id{\}}}$, to the following pair
of processes, $\lift{w}{y!(z)}$ and $w[ \lpquote y!(z) \rpquote ]$.

\begin{eqnarray}
	\lift{w}{y!(z)}\widehat{\id{\{}u / z \id{\}}}
		& = &
		\lift{w}{y!(u)} \nonumber\\
	w[ \lpquote y!(z) \rpquote ] \widehat{ \id{\{}u / z \id{\}} }
		& = &
		w[ \lpquote y!(z) \rpquote ] \nonumber
\end{eqnarray}

Because the body of the process between quotes is impervious to
substitution, we get radically different answers. In fact, by
examining the first process in an input context,
e.g. $x?(z).\lift{w}{y!(z)}$, we see that the process under the lift
operator may be shaped by prefixed inputs binding a name inside it. In
this sense, the lift operator will be seen as a way to dynamically
construct processes before reifying them as names.

Finally equipped with these standard features we can present the
dynamics of the calculus.

\subsubsection{Operational semantics} 

Finally, we introduce the computational dynamics. What marks these
algebras as distinct from other more traditionally studied algebraic
structures, e.g. vector spaces or polynomial rings, is the manner in
which dynamics is captured. In traditional structures, dynamics is typically
expressed through morphisms between such structures, as in linear maps
between vector spaces or morphisms between rings. In algebras
associated with the semantics of computation, the dynamics is
expressed as part of the algebraic structure itself, through a
reduction reduction relation typically denoted by $\red$. Below, we
give a recursive presentation of this relation for the calculus used
in the encoding.

$\red \subseteq \pi \times \pi$
$\red : \pi \to \mathcal{P}(\pi)$

\begin{mathpar}
  \inferrule* [lab=Comm] { \textsf{match}( x_{src}, x_{trgt} ) } { x_{trgt}?(y)P \; | \; x_{src}!\langle {Q} \rangle \red P\{\quotep{Q}/y}\} }
  \and \\
  \inferrule* [lab=Par] {{P} \red {P}'} {{{P} | {Q}} \red {{P}' | {Q}}}
  \and
  \inferrule* [lab=Equiv]{{{P} \scong {P}'} \andalso {{P}' \red {Q}'} \andalso {{Q}' \scong {Q}}}{{P} \red {Q}}
\end{mathpar}

\begin{eqnarray*}
  match_{\equiv} (\quotep{P},\quotep{Q}) & := & P \equiv Q \\
  match_{\dagger}(\quotep{P},\quotep{Q}) & := & \forall R. P|Q \red^{*} R => R \red^{*} 0 \\
  match_{K}(\quotep{P},\quotep{Q}) & := & K \mbox{ for some context } K
\end{eqnarray*}

$u?(x)P | u!\langle Q \rangle \red P\{\quotep{Q}/x\}$

%We write $\wred$ for $\red^*$, and $P\red$ if $\exists Q $ such that $ P \red Q$.
We write $P\red$ if $\exists Q $ such that $ P \red Q$ and $P\not\red$, otherwise.

\section{Replication}

As mentioned before, it is known that replication (and hence
recursion) can be implemented in a higher-order process algebra
\cite{SangiorgiWalker}. As our first example of calculation with the
machinery thus far presented we give the construction explicitly in
the {\rhoc}.

\begin{eqnarray}
	D_{x} & := & \prefix{x}{y}{(\binpar{\outputp{x}{y}}{@{y}})} \nonumber\\
	\bangp_{x}{P} & := & \binpar{{x}!\langle{\binpar{D_{x}}{P}}\rangle}{D_{x}} \nonumber
\end{eqnarray}

\begin{eqnarray}
	\bangp_{x}{P} & & \nonumber\\
	=
	& {x}!\langle{(\prefix{x}{y}{(\outputp{x}{y} | @{y})) | P}}\rangle 
	      | \prefix{x}{y}{(\outputp{x}{y} | @{y})} & \nonumber\\
	\red
	& (\outputp{x}{y} | @{y})\substn{\quotep{(\prefix{x}{y}{(@{y} | \outputp{x}{y})) | P}}}{y} & \nonumber\\
	=
	& \outputp{x}{\quotep{(\prefix{x}{y}{(\outputp{x}{y} | @{y})) | P}}}
	  | {(\prefix{x}{y}{(\outputp{x}{y} | @{y})) | P}} & \nonumber\\
	\red
	& \ldots & \nonumber\\
	\red^*
	& P | P | \ldots & \nonumber
\end{eqnarray}

Of course, this encoding, as an implementation, runs away, unfolding
$\bangp{P}$ eagerly. A lazier and more implementable replication
operator, restricted to input-guarded processes, may be obtained as follows.

\begin{eqnarray}
\bangp{\prefix{u}{v}{P}} 
	:= 
	\binpar{\lift{x}{\prefix{u}{v}{(\binpar{D(x)}{P})}}}{D(x)} \nonumber
\end{eqnarray}

\begin{remark}
  Note that the lazier definition still does not deal with summation
  or mixed summation (i.e. sums over input and output). The reader is
  invited to construct definitions of replication that deal with these
  features. 

  Further, the definitions are parameterized in a name, $x$. Can you,
  gentle reader, make a definition that eliminates this parameter and
  guarantees no accidental interaction between the replication
  machinery and the process being replicated -- i.e. no accidental
  sharing of names used by the process to get its work done and the
  name(s) used by the replication to effect copying. This latter
  revision of the definition of replication is crucial to obtaining
  the expected identity $!!P \sim !P$.
\end{remark}

\begin{remark}\label{rem:paradoxical_combinator}
  The reader familiar with the lambda calculus will have noticed the
  similarity between $D$ and the paradoxical combinator.

  [Ed. note: the existence of this seems to suggest we have to be more
  restrictive on the set of processes and names we admit if we are to
  support no-cloning.]
\end{remark}

\subsubsection{Bisimulation}

The computational dynamics gives rise to another kind of equivalence,
the equivalence of computational behavior. As previously mentioned
this is typically captured \emph{via} some form of bisimulation.

% The notion we use in this paper is weak barbed bisimulation
% \cite{milner91polyadicpi}.

The notion we use in this paper is derived from weak barbed
bisimulation \cite{milner91polyadicpi}. 

\begin{definition}
An \emph{observation relation}, $\downarrow_{\mathcal N}$, over a set
of names, $\mathcal N$, is the smallest relation satisfying the rules
below.

\infrule[Out-barb]{y \in {\mathcal N}, \; x \nameeq y}
		  {\outputp{x}{v} \downarrow_{\mathcal N} x}
\infrule[Par-barb]{\mbox{$P\downarrow_{\mathcal N} x$ or $Q\downarrow_{\mathcal N} x$}}
		  {\binpar{P}{Q} \downarrow_{\mathcal N} x}

We write $P \Downarrow_{\mathcal N} x$ if there is $Q$ such that 
$P \wred Q$ and $Q \downarrow_{\mathcal N} x$.
\end{definition}

\begin{definition}
%\label{def.bbisim}
An  ${\mathcal N}$-\emph{barbed bisimulation} over a set of names, ${\mathcal N}$, is a symmetric binary relation 
${\mathcal S}_{\mathcal N}$ between agents such that $P\rel{S}_{\mathcal N}Q$ implies:
\begin{enumerate}
\item If $P \red P'$ then $Q \wred Q'$ and $P'\rel{S}_{\mathcal N} Q'$.
\item If $P\downarrow_{\mathcal N} x$, then $Q\Downarrow_{\mathcal N} x$.
\end{enumerate}
$P$ is ${\mathcal N}$-barbed bisimilar to $Q$, written
$P \wbbisim_{\mathcal N} Q$, if $P \rel{S}_{\mathcal N} Q$ for some ${\mathcal N}$-barbed bisimulation ${\mathcal S}_{\mathcal N}$.
\end{definition}

$\mathcal{R} \subseteq \pi \times \pi$

$P \mathcal{R} Q => \forall P'. P \red P' \Rightarrow \exists Q'. Q \red Q', P' \mathcal{R} Q'$

$P \vdash x \Rightarrow Q \vdash x$

\begin{mathpar}
  \inferrule*[lab=Out-barb]{x \nameeq y}{{y}!\langle{Q}\rangle \vdash x}
  \and
  \inferrule*[lab=Par-barb]{\mbox{$P\vdash x$ or $Q\vdash x$}}{\binpar{P}{Q} \vdash x}
\end{mathpar}

\subsubsection{Contexts}

One of the principle advantages of computational calculi like the
$\pi$-calculus is a well-defined notion of context,
contextual-equivalence and a correlation between
contextual-equivalence and notions of bisimulation. The notion of
context allows the decomposition of a process into (sub-)process and
its syntactic environment, its context. Thus, a context may be
thought of as a process with a ``hole'' (written $\Box$) in it. The
application of a context $M$ to a process $P$, written $M[P]$, is
tantamount to filling the hole in $M$ with $P$. In this paper we do
not need the full weight of this theory, but do make use of the notion
of context in the proof the main theorem. 

\begin{mathpar}
  \inferrule* [lab=summation] {} {{M_{M},M_{N}} \bc \Box \;|\; x.M_{A} \;|\; M_{M}+M_{N}}
  \and
  \inferrule* [lab=agent] {} {{M_{A}} \bc (\vec{x})M_{P} \;| \; \clift{P_0,\ldots,M_{P},\ldots,P_N}}
  \and \\
  \inferrule* [lab=process] {} {{M_{P}} \bc M_{N} \;| \;P|M_{P} }
\end{mathpar} 

\begin{mathpar}
  \inferrule* [lab=sychronization] {} {M_{N} \bc \Box \;|\; x?M_{F} \;|\; x!M_{C}}
  \and
  \inferrule* [lab=abstraction] {} {{M_{F}} \bc (x)M_{P} }
  \and
  \inferrule* [lab=concretion] {} {{M_{C}} \bc \langle M_{P} \rangle }
  \and \\
  \inferrule* [lab=process] {} {{M_{P}} \bc M_{N} \;| \;P|M_{P} }
\end{mathpar}

\begin{definition}[contextual application] Given a context $M$, and
  process $P$, we define the \emph{contextual application}, $M[P] :=
  M\{P/\Box\}$. That is, the contextual application of M to P is the
  substitution of $P$ for $\Box$ in $M$.
\end{definition}

$\meaningof{-} : L \to \mathcal{P}(\pi)$

\begin{mathpar}
  \inferrule* [lab=collection] {} {\meaningof{true} = \pi, \and \meaningof{~E} = \pi \setminus \meaningof{E}, \and \meaningof{E_{1} \& E_{2}} = \meaningof{E_{1}} \cap \meaningof{E_{2}}}
\end{mathpar}

\begin{mathpar}
  \inferrule* [lab=structure] {} {\meaningof{0} = \{ P \in \pi | P \equiv 0 \}, \and \\ \meaningof{E_1 | E_2} = \{ P \in \pi | P \equiv P_{1} | P_{2}, P_{1} \in \meaningof{E_{1}}, P_{2} \in \meaningof{E_2}\} }
\end{mathpar}

\begin{mathpar}
 \inferrule* [lab=behavior] {} {\meaningof{\langle a?b \rangle E} = \{ P \in \pi | P \equiv Q | u?(y)P', \\ \and \\\\ \and \\ \;\;\; u \in \meaningof{a}, \forall z.P'\{z/y\} \in \meaningof{E\{z/b\}}\}, \and \\ \meaningof{a!E} = \{ P \in \pi | P \equiv Q | x!\langle P' \rangle, x \in \meaningof{a} P' \in \meaningof{E}\} }
\end{mathpar}

\begin{mathpar}
 \inferrule* [lab=nominal] {} {\meaningof{\quotep{E}} = \{ \quotep{P} \in \quotep{\pi} | P \in \meaningof{E} \}, \and \meaningof{\quotep{P}} = \{ \quotep{Q} \in \quotep{\pi} | P \equiv Q \} \and \\ \meaningof{@\quotep{E}} = \{ P \in \pi | P \equiv @x, x \in \meaningof{E} \}}
\end{mathpar}

\begin{eqnarray*}
  \\
  \meaningof{-} : TS \to ST
\end{eqnarray*}

\begin{eqnarray*}
  \\
  L : TS \to ST
\end{eqnarray*}

\begin{eqnarray*}
  \\
  P \models E \iff P \in \meaningof{E}
\end{eqnarray*}

\begin{eqnarray*}
  P \approx_{L} Q \iff \forall E \in L. P \models E \iff Q \models E
\end{eqnarray*}

\begin{eqnarray*}
  P \approx_{K} Q
\end{eqnarray*}

\begin{eqnarray*}
  P \approx Q
\end{eqnarray*}

$\approx_{K} = \approx = \approx_{L}$

\subsubsection{Contextual duality}

Note that contexts extend the quotation operation to a family of
operations from processes to names. Given a context, $M$, we can
define a \emph{nominal context}, $\quotep{M}$ by $\quotep{M}[P] :=
\quotep{M[P]}$. To foreshadow what is to come we observe that these
operations enjoy a duality with processes very much like the duality
between vectors and maps from vectors to scalars.

Further, because the calculus is essentially higher-order, we have a
correspondence between contexts and processes. More specifically,
given a name $x$ and a context $M$ we can construct $M^{*}_{x}$ such
that 

\begin{mathpar}
  M^{*}_{x} | \lift{x}{P} \red M[P]
\end{mathpar}

namely,

\begin{mathpar}
  M^{*}_{x} := x?(u).M[\dropn{u}]
\end{mathpar}

The dependence of $M^{*}_{x}$ on a name makes it an abstraction, 

\begin{mathpar}
  M^{*} := (x)x?(u).M[\dropn{u}]
\end{mathpar}

\subsection{Additional notation}

It will sometimes be convenient to denote the process a name
quotes. We already have the notation $x = \quotep{P}$, but it will be
convenient to introduce an alternate notation, $\procn{x}$, when we
want to emphasize the connection to the use of the name. Note that, by
virtue of name equivalence, $\quotep{\procn{x}} \nameeq x$; so, the
notation is consistent with previous definitions.

Further, because names have structure it is possible to effect
substitutions on the basis of that structure. This means we need to
upgrade our notation for substitutions, which we accomplish by
adapting comprehension notation. Thus,

\begin{mathpar}
  P\{ y / x : x \in S \}
\end{mathpar}

is interpreted to mean the process derived from P by replacing (in a
capture-avoiding manner) each occurrence of $x$ in $S$ by $y$. For example,

\begin{mathpar}
  P\{ \quotep{\procn{x}|\procn{x}} / x : x \in \freenames{P} \}
\end{mathpar}

will replace each (occurrence) of a free name $x$ in $P$ by
$\quotep{\procn{x}|\procn{x}}$.

Also, we will avail ourselves of the notation $x^{L}$ and $x^{R}$ to
denote injections of a name into disjoint copies of the name
space. There are numerous ways to accomplish this. One example can be
found in \cite{MeredithR05}. This notation overloads to vectors of
names: $\vec{x}^{\pi} := (x_{i}^{\pi} \; : \; 0 \leq i < |\vec{x}| )$ where $\pi \in \{L,R\}$.

We also use $P^{\Box} := P|\Box$.

In \cite{MeredithR05} an interpretation of the new operator is
given. It turns out that there are several possible interpretations
all enjoying the requisite algebraic properties of the operator (see
\cite{milner91polyadicpi}). We will therefore make liberal use of
$(\nu\; \vec{x})P$.

% subsection the_syntax_and_semantics_of_the_notation_system (end)   

\input{qm2pi.qmops} 

\input{qm2pi.sterngerlach} 

\input{qm2pi.metric} 

% section concurrent_process_calculi (end)

%\input{qm2pi.proofsketch}

% section proof sketch (end)

%\input{qm2pi.slviaknots} 

% section spatial logic via knots (end)

\input{qm2pi.conclusion}

% section conclusion (end)

%\input{qm2pi.dtcodes} 

% section wiring algorithm (end)

\input{qm2pi.ack} 

% section acknowledgments (end)

\newpage


\bibliographystyle{plain}   
\bibliography{../../biblios/main.bib}

\input{qm2pi.rhodetails}

\end{document}



\end{document}

 

% subsection basic_interpretation (end)

%\input{qm2pi.rho.presentation} 
\subsection{The syntax and semantics of the notation system}\label{sub:the_syntax_and_semantics_of_the_notation_system} % (fold)

We now summarize a technical presentation of the calculus that
embodies our theory of dynamics. The typical presentation of such a
calculus follows the style of giving generators and relations on
them. The grammar, below, describing term constructors, freely
generates the set of processes, $\Proc$. This set is then quotiented
by a relation known as structural congruence and it is over this set
that the notion of dynamics is expressed. This presentation is
essentially that of \cite{MeredithR05} with the addition of
polyadicity and summation. For readability we have relegated some of
the technical subtleties to an appendix.

\subsubsection{Process grammar}\label{subsub:process_grammar}

\begin{mathpar}
  \inferrule* [lab=synchronization] {} {{M} \bc \pzero \;|\; x?F \;|\; x!C }
  \and
  \inferrule* [lab=abstraction] {} {{F} \bc (x)P}
  \and
  \inferrule* [lab=concretion] {} {{C} \bc \langle Q \rangle}
  \and
  \inferrule* [lab=process] {} {{P,Q} \bc M \;| \;P|Q \;|\; @{x}}
  \and
  \inferrule* [lab=name] {} {{x} \bc \quotep{P}}
\end{mathpar} 

Note that $\vec{x}$ (resp. $\vec{P}$) denotes a vector of names
(resp. processes) of length $|\vec{x}|$ (resp. $|\vec{P}|$). We adopt
the following useful abbreviations.

\begin{mathpar}
   x?(\vec{y}).P := x.(\vec{y})P \and  x\clift{\vec{P}} := x.\clift{\vec{P}}
   \and x!(y) := \lift{x}{\dropn{y}}
   \and \Pi_{i=0}^{n-1}P_i := P_0 | \ldots | P_{n-1}
\end{mathpar}

\subsubsection{Structural congruence}

\paragraph{Free and bound names and alpha-equivalence.} At the
core of structural equivalence is alpha-equivalence which identifies
process that are the same up to a change of variable. Formally, we
recognize the distinction between free and bound names. The free names
of a process, $\freenames{P}$, may be calculated recursively as
follows:

\begin{mathpar}
\freenames{\pzero} := \emptyset
  \and \\
  \freenames{x?(y).P} := \{ x \} \cup (\freenames{P} \setminus \{ y \})
  \and 
  \freenames{x!\langle P \rangle} := \{ x \} \cup \{ P \} 
  \and \\
  \freenames{P|Q} := \freenames{P} \cup \freenames{Q}
  \and \\
  \freenames{@{x}} := \{ x \}
\end{mathpar}

$\pi$
$\quotep{\pi}$

$\freenames{-} : \pi \to \mathcal{P}(\quotep{\pi})$

\begin{eqnarray*}
  \freenames{\pzero} & := & \emptyset \\
  \freenames{x?(y).P} & := & \{ x \} \cup (\freenames{P} \setminus \{ y \}) \\
  \freenames{x!\langle P \rangle} & := & \{ x \} \cup \{ P \} \\
  \freenames{P|Q} & := & \freenames{P} \cup \freenames{Q} \\
  \freenames{\dropn{x}} & := & \{ x \}
\end{eqnarray*}

The bound names of a process, $\boundnames{P}$, are those names occurring in $P$
that are not free. For example, in $x?(y).0$, the name $x$ is free, while $y$ is bound.

\begin{mathpar}
  \inferrule* [lab=monoidal-laws] {} { P|Q \equiv Q|P \and P|0 \equiv P \and P|(Q|R) \equiv (P|Q)|R }
\end{mathpar}

\begin{mathpar}
  \inferrule* [lab=alpha-equivalence] {} { (x)P \equiv (y)P\{y/x\} \and y \not\in \freenames{P} }
\end{mathpar}

\begin{definition}
Then two processes, $P,Q$, are alpha-equivalent if $P = Q\{\vec{y}/\vec{x}\}$ for
some $\vec{x} \in \boundnames{Q},\vec{y} \in \boundnames{P}$, where $Q\{\vec{y}/\vec{x}\}$
denotes the capture-avoiding substitution of $\vec{y}$ for $\vec{x}$ in $Q$.
\end{definition}

\begin{definition}
  The {\em structural congruence} \cite{SangiorgiWalker} , $\equiv$,
  between processes is the least congruence containing
  alpha-equivalence, satisfying the abelian monoid laws
  (associativity, commutativity and $\pzero$ as identity) for parallel
  composition $|$ and for summation $+$.
\end{definition}

\subsection{Name equivalence}

We take name equivalence, written $\nameeq$, to be the smallest
equivalence relation generated by the following rules.

\begin{mathpar}
\inferrule*[lab=Quote-drop]
{ }
{ \quotep{@{x}} \nameeq x }

\inferrule*[lab=Struct-equiv]
{ P \scong Q }
{ \quotep{P} \nameeq \quotep{Q} }
\end{mathpar}

The astute reader will have noticed that the mutual recursion of names
and processes imposes a mutual recursion on alpha-equivalence and
structural equivalence via name-equivalence. Fortunately, all of this
works out pleasantly and we may calculate in the natural way, free of
concern. The reader interested in the details is referred to the
appendix \ref{appendix:rho_details}.

\subsection{Substitution}

We use $\Proc$ for the set of processes, $\QProc$ for the set of
names, and $\id{\{}\vec{y} / \vec{x} \id{\}}$ to denote partial maps,
$s : \QProc \rightarrow \QProc$. A map, $s$ lifts, uniquely, to a map
on process terms, $\widehat{s} : \Proc \rightarrow \Proc$ by the
following equations.

\begin{mathpar}
  (0) \psubstp{Q}{P} := 0 \\
  (R \juxtap S) \psubstp{Q}{P}
  :=    
  (R)\psubstp{Q}{P} \juxtap (S) \psubstp{Q}{P} \\
  (x?(y).R) \psubstp{Q}{P}    
  :=    
  (x)\substp{Q}{P} (z)\concat( (R \psubstn{z}{y}) \psubstp{Q}{P} ) \\
  (\lift{x}{R}) \psubstp{Q}{P}  
  :=
  \lift{(x)\substp{Q}{P}}{ R \psubstp{Q}{P} } \\
%   (\dropn{x})  \psubstp{Q}{P}       
%   := 
%   \left\{ 
%     \begin{array}{ccc} 
%       \dropn{\quotep{Q}} & & x \nameeq \quotep{P} \\
%       \dropn{x} & & otherwise \\
%     \end{array}
%   \right. 
  (\dropn{x})  \psubstp{Q}{P}       
  := 
  \left\{ 
    \begin{array}{ccc} 
      Q & & x \nameeq \quotep{P} \\
      \dropn{x} & & otherwise \\
    \end{array}
  \right.
\end{mathpar}
 

where

\begin{eqnarray}
  (x)\id{\{} \lpquote Q \rpquote / \lpquote P \rpquote \id{\}}            = 
  \left\{ 
    \begin{array}{ccc}
      \lpquote Q \rpquote & & x \nameeq \lpquote P \rpquote \\
      x & & otherwise \\
    \end{array}
  \right. \nonumber
\end{eqnarray}

and $z$ is chosen distinct from $\quotep{P}$, $\quotep{Q}$, the free
names in $Q$, and all the names in $R$. Our $\alpha$-equivalence will
be built in the standard way from this substitution.

\begin{remark}\label{rem:no_self_referential_names}
  One consequence of these definitions is that $\forall P. \quotep{P}
  \not\in \freenames{P}$.
\end{remark}

\subsection{ Dynamic quote: an example }

Anticipating something of what's to come, consider applying the
substitution, $\widehat{\id{\{}u / z \id{\}}}$, to the following pair
of processes, $\lift{w}{y!(z)}$ and $w[ \lpquote y!(z) \rpquote ]$.

\begin{eqnarray}
	\lift{w}{y!(z)}\widehat{\id{\{}u / z \id{\}}}
		& = &
		\lift{w}{y!(u)} \nonumber\\
	w[ \lpquote y!(z) \rpquote ] \widehat{ \id{\{}u / z \id{\}} }
		& = &
		w[ \lpquote y!(z) \rpquote ] \nonumber
\end{eqnarray}

Because the body of the process between quotes is impervious to
substitution, we get radically different answers. In fact, by
examining the first process in an input context,
e.g. $x?(z).\lift{w}{y!(z)}$, we see that the process under the lift
operator may be shaped by prefixed inputs binding a name inside it. In
this sense, the lift operator will be seen as a way to dynamically
construct processes before reifying them as names.

Finally equipped with these standard features we can present the
dynamics of the calculus.

\subsubsection{Operational semantics} 

Finally, we introduce the computational dynamics. What marks these
algebras as distinct from other more traditionally studied algebraic
structures, e.g. vector spaces or polynomial rings, is the manner in
which dynamics is captured. In traditional structures, dynamics is typically
expressed through morphisms between such structures, as in linear maps
between vector spaces or morphisms between rings. In algebras
associated with the semantics of computation, the dynamics is
expressed as part of the algebraic structure itself, through a
reduction reduction relation typically denoted by $\red$. Below, we
give a recursive presentation of this relation for the calculus used
in the encoding.

$\red \subseteq \pi \times \pi$
$\red : \pi \to \mathcal{P}(\pi)$

\begin{mathpar}
  \inferrule* [lab=Comm] { \textsf{match}( x_{src}, x_{trgt} ) } { x_{trgt}?(y)P \; | \; x_{src}!\langle {Q} \rangle \red P\{\quotep{Q}/y}\} }
  \and \\
  \inferrule* [lab=Par] {{P} \red {P}'} {{{P} | {Q}} \red {{P}' | {Q}}}
  \and
  \inferrule* [lab=Equiv]{{{P} \scong {P}'} \andalso {{P}' \red {Q}'} \andalso {{Q}' \scong {Q}}}{{P} \red {Q}}
\end{mathpar}

\begin{eqnarray*}
  match_{\equiv} (\quotep{P},\quotep{Q}) & := & P \equiv Q \\
  match_{\dagger}(\quotep{P},\quotep{Q}) & := & \forall R. P|Q \red^{*} R => R \red^{*} 0 \\
  match_{K}(\quotep{P},\quotep{Q}) & := & K \mbox{ for some context } K
\end{eqnarray*}

$u?(x)P | u!\langle Q \rangle \red P\{\quotep{Q}/x\}$

%We write $\wred$ for $\red^*$, and $P\red$ if $\exists Q $ such that $ P \red Q$.
We write $P\red$ if $\exists Q $ such that $ P \red Q$ and $P\not\red$, otherwise.

\section{Replication}

As mentioned before, it is known that replication (and hence
recursion) can be implemented in a higher-order process algebra
\cite{SangiorgiWalker}. As our first example of calculation with the
machinery thus far presented we give the construction explicitly in
the {\rhoc}.

\begin{eqnarray}
	D_{x} & := & \prefix{x}{y}{(\binpar{\outputp{x}{y}}{@{y}})} \nonumber\\
	\bangp_{x}{P} & := & \binpar{{x}!\langle{\binpar{D_{x}}{P}}\rangle}{D_{x}} \nonumber
\end{eqnarray}

\begin{eqnarray}
	\bangp_{x}{P} & & \nonumber\\
	=
	& {x}!\langle{(\prefix{x}{y}{(\outputp{x}{y} | @{y})) | P}}\rangle 
	      | \prefix{x}{y}{(\outputp{x}{y} | @{y})} & \nonumber\\
	\red
	& (\outputp{x}{y} | @{y})\substn{\quotep{(\prefix{x}{y}{(@{y} | \outputp{x}{y})) | P}}}{y} & \nonumber\\
	=
	& \outputp{x}{\quotep{(\prefix{x}{y}{(\outputp{x}{y} | @{y})) | P}}}
	  | {(\prefix{x}{y}{(\outputp{x}{y} | @{y})) | P}} & \nonumber\\
	\red
	& \ldots & \nonumber\\
	\red^*
	& P | P | \ldots & \nonumber
\end{eqnarray}

Of course, this encoding, as an implementation, runs away, unfolding
$\bangp{P}$ eagerly. A lazier and more implementable replication
operator, restricted to input-guarded processes, may be obtained as follows.

\begin{eqnarray}
\bangp{\prefix{u}{v}{P}} 
	:= 
	\binpar{\lift{x}{\prefix{u}{v}{(\binpar{D(x)}{P})}}}{D(x)} \nonumber
\end{eqnarray}

\begin{remark}
  Note that the lazier definition still does not deal with summation
  or mixed summation (i.e. sums over input and output). The reader is
  invited to construct definitions of replication that deal with these
  features. 

  Further, the definitions are parameterized in a name, $x$. Can you,
  gentle reader, make a definition that eliminates this parameter and
  guarantees no accidental interaction between the replication
  machinery and the process being replicated -- i.e. no accidental
  sharing of names used by the process to get its work done and the
  name(s) used by the replication to effect copying. This latter
  revision of the definition of replication is crucial to obtaining
  the expected identity $!!P \sim !P$.
\end{remark}

\begin{remark}\label{rem:paradoxical_combinator}
  The reader familiar with the lambda calculus will have noticed the
  similarity between $D$ and the paradoxical combinator.

  [Ed. note: the existence of this seems to suggest we have to be more
  restrictive on the set of processes and names we admit if we are to
  support no-cloning.]
\end{remark}

\subsubsection{Bisimulation}

The computational dynamics gives rise to another kind of equivalence,
the equivalence of computational behavior. As previously mentioned
this is typically captured \emph{via} some form of bisimulation.

% The notion we use in this paper is weak barbed bisimulation
% \cite{milner91polyadicpi}.

The notion we use in this paper is derived from weak barbed
bisimulation \cite{milner91polyadicpi}. 

\begin{definition}
An \emph{observation relation}, $\downarrow_{\mathcal N}$, over a set
of names, $\mathcal N$, is the smallest relation satisfying the rules
below.

\infrule[Out-barb]{y \in {\mathcal N}, \; x \nameeq y}
		  {\outputp{x}{v} \downarrow_{\mathcal N} x}
\infrule[Par-barb]{\mbox{$P\downarrow_{\mathcal N} x$ or $Q\downarrow_{\mathcal N} x$}}
		  {\binpar{P}{Q} \downarrow_{\mathcal N} x}

We write $P \Downarrow_{\mathcal N} x$ if there is $Q$ such that 
$P \wred Q$ and $Q \downarrow_{\mathcal N} x$.
\end{definition}

\begin{definition}
%\label{def.bbisim}
An  ${\mathcal N}$-\emph{barbed bisimulation} over a set of names, ${\mathcal N}$, is a symmetric binary relation 
${\mathcal S}_{\mathcal N}$ between agents such that $P\rel{S}_{\mathcal N}Q$ implies:
\begin{enumerate}
\item If $P \red P'$ then $Q \wred Q'$ and $P'\rel{S}_{\mathcal N} Q'$.
\item If $P\downarrow_{\mathcal N} x$, then $Q\Downarrow_{\mathcal N} x$.
\end{enumerate}
$P$ is ${\mathcal N}$-barbed bisimilar to $Q$, written
$P \wbbisim_{\mathcal N} Q$, if $P \rel{S}_{\mathcal N} Q$ for some ${\mathcal N}$-barbed bisimulation ${\mathcal S}_{\mathcal N}$.
\end{definition}

$\mathcal{R} \subseteq \pi \times \pi$

$P \mathcal{R} Q => \forall P'. P \red P' \Rightarrow \exists Q'. Q \red Q', P' \mathcal{R} Q'$

$P \vdash x \Rightarrow Q \vdash x$

\begin{mathpar}
  \inferrule*[lab=Out-barb]{x \nameeq y}{{y}!\langle{Q}\rangle \vdash x}
  \and
  \inferrule*[lab=Par-barb]{\mbox{$P\vdash x$ or $Q\vdash x$}}{\binpar{P}{Q} \vdash x}
\end{mathpar}

\subsubsection{Contexts}

One of the principle advantages of computational calculi like the
$\pi$-calculus is a well-defined notion of context,
contextual-equivalence and a correlation between
contextual-equivalence and notions of bisimulation. The notion of
context allows the decomposition of a process into (sub-)process and
its syntactic environment, its context. Thus, a context may be
thought of as a process with a ``hole'' (written $\Box$) in it. The
application of a context $M$ to a process $P$, written $M[P]$, is
tantamount to filling the hole in $M$ with $P$. In this paper we do
not need the full weight of this theory, but do make use of the notion
of context in the proof the main theorem. 

\begin{mathpar}
  \inferrule* [lab=summation] {} {{M_{M},M_{N}} \bc \Box \;|\; x.M_{A} \;|\; M_{M}+M_{N}}
  \and
  \inferrule* [lab=agent] {} {{M_{A}} \bc (\vec{x})M_{P} \;| \; \clift{P_0,\ldots,M_{P},\ldots,P_N}}
  \and \\
  \inferrule* [lab=process] {} {{M_{P}} \bc M_{N} \;| \;P|M_{P} }
\end{mathpar} 

\begin{mathpar}
  \inferrule* [lab=sychronization] {} {M_{N} \bc \Box \;|\; x?M_{F} \;|\; x!M_{C}}
  \and
  \inferrule* [lab=abstraction] {} {{M_{F}} \bc (x)M_{P} }
  \and
  \inferrule* [lab=concretion] {} {{M_{C}} \bc \langle M_{P} \rangle }
  \and \\
  \inferrule* [lab=process] {} {{M_{P}} \bc M_{N} \;| \;P|M_{P} }
\end{mathpar}

\begin{definition}[contextual application] Given a context $M$, and
  process $P$, we define the \emph{contextual application}, $M[P] :=
  M\{P/\Box\}$. That is, the contextual application of M to P is the
  substitution of $P$ for $\Box$ in $M$.
\end{definition}

$\meaningof{-} : L \to \mathcal{P}(\pi)$

\begin{mathpar}
  \inferrule* [lab=collection] {} {\meaningof{true} = \pi, \and \meaningof{~E} = \pi \setminus \meaningof{E}, \and \meaningof{E_{1} \& E_{2}} = \meaningof{E_{1}} \cap \meaningof{E_{2}}}
\end{mathpar}

\begin{mathpar}
  \inferrule* [lab=structure] {} {\meaningof{0} = \{ P \in \pi | P \equiv 0 \}, \and \\ \meaningof{E_1 | E_2} = \{ P \in \pi | P \equiv P_{1} | P_{2}, P_{1} \in \meaningof{E_{1}}, P_{2} \in \meaningof{E_2}\} }
\end{mathpar}

\begin{mathpar}
 \inferrule* [lab=behavior] {} {\meaningof{\langle a?b \rangle E} = \{ P \in \pi | P \equiv Q | u?(y)P', \\ \and \\\\ \and \\ \;\;\; u \in \meaningof{a}, \forall z.P'\{z/y\} \in \meaningof{E\{z/b\}}\}, \and \\ \meaningof{a!E} = \{ P \in \pi | P \equiv Q | x!\langle P' \rangle, x \in \meaningof{a} P' \in \meaningof{E}\} }
\end{mathpar}

\begin{mathpar}
 \inferrule* [lab=nominal] {} {\meaningof{\quotep{E}} = \{ \quotep{P} \in \quotep{\pi} | P \in \meaningof{E} \}, \and \meaningof{\quotep{P}} = \{ \quotep{Q} \in \quotep{\pi} | P \equiv Q \} \and \\ \meaningof{@\quotep{E}} = \{ P \in \pi | P \equiv @x, x \in \meaningof{E} \}}
\end{mathpar}

\begin{eqnarray*}
  \\
  \meaningof{-} : TS \to ST
\end{eqnarray*}

\begin{eqnarray*}
  \\
  L : TS \to ST
\end{eqnarray*}

\begin{eqnarray*}
  \\
  P \models E \iff P \in \meaningof{E}
\end{eqnarray*}

\begin{eqnarray*}
  P \approx_{L} Q \iff \forall E \in L. P \models E \iff Q \models E
\end{eqnarray*}

\begin{eqnarray*}
  P \approx_{K} Q
\end{eqnarray*}

\begin{eqnarray*}
  P \approx Q
\end{eqnarray*}

$\approx_{K} = \approx = \approx_{L}$

\subsubsection{Contextual duality}

Note that contexts extend the quotation operation to a family of
operations from processes to names. Given a context, $M$, we can
define a \emph{nominal context}, $\quotep{M}$ by $\quotep{M}[P] :=
\quotep{M[P]}$. To foreshadow what is to come we observe that these
operations enjoy a duality with processes very much like the duality
between vectors and maps from vectors to scalars.

Further, because the calculus is essentially higher-order, we have a
correspondence between contexts and processes. More specifically,
given a name $x$ and a context $M$ we can construct $M^{*}_{x}$ such
that 

\begin{mathpar}
  M^{*}_{x} | \lift{x}{P} \red M[P]
\end{mathpar}

namely,

\begin{mathpar}
  M^{*}_{x} := x?(u).M[\dropn{u}]
\end{mathpar}

The dependence of $M^{*}_{x}$ on a name makes it an abstraction, 

\begin{mathpar}
  M^{*} := (x)x?(u).M[\dropn{u}]
\end{mathpar}

\subsection{Additional notation}

It will sometimes be convenient to denote the process a name
quotes. We already have the notation $x = \quotep{P}$, but it will be
convenient to introduce an alternate notation, $\procn{x}$, when we
want to emphasize the connection to the use of the name. Note that, by
virtue of name equivalence, $\quotep{\procn{x}} \nameeq x$; so, the
notation is consistent with previous definitions.

Further, because names have structure it is possible to effect
substitutions on the basis of that structure. This means we need to
upgrade our notation for substitutions, which we accomplish by
adapting comprehension notation. Thus,

\begin{mathpar}
  P\{ y / x : x \in S \}
\end{mathpar}

is interpreted to mean the process derived from P by replacing (in a
capture-avoiding manner) each occurrence of $x$ in $S$ by $y$. For example,

\begin{mathpar}
  P\{ \quotep{\procn{x}|\procn{x}} / x : x \in \freenames{P} \}
\end{mathpar}

will replace each (occurrence) of a free name $x$ in $P$ by
$\quotep{\procn{x}|\procn{x}}$.

Also, we will avail ourselves of the notation $x^{L}$ and $x^{R}$ to
denote injections of a name into disjoint copies of the name
space. There are numerous ways to accomplish this. One example can be
found in \cite{MeredithR05}. This notation overloads to vectors of
names: $\vec{x}^{\pi} := (x_{i}^{\pi} \; : \; 0 \leq i < |\vec{x}| )$ where $\pi \in \{L,R\}$.

We also use $P^{\Box} := P|\Box$.

In \cite{MeredithR05} an interpretation of the new operator is
given. It turns out that there are several possible interpretations
all enjoying the requisite algebraic properties of the operator (see
\cite{milner91polyadicpi}). We will therefore make liberal use of
$(\nu\; \vec{x})P$.

% subsection the_syntax_and_semantics_of_the_notation_system (end)   

\section{Interpretation of QM}
\subsection{Supporting definitions}
\subsubsection{Multiplication}
\begin{mathpar}
  \quotep{Q} \cdot \quotep{R} := \quotep{Q|R}
  \and \\
  \quotep{Q} \cdot P := P\{ \quotep{Q|R} / \quotep{R} : \quotep{R} \in \freenames{P} \}
\end{mathpar}

\paragraph{Discussion}
The first line needs little explanation. The second line says that
each free name of the process is replaced with the multiplication of
that name by the scalar. Multiplication of a scalar (name) by a state
(process) results in a process all the names of which have been `moved
over' by parallel composition with the process the scalar
quotes. There is a subtlety that the bound names have to be
manipulated so that multiplied names aren't accidentally
captured. There are many ways to achieve this.

\begin{remark}\label{rem:multiplication_identities}
  The reader is invited to verify that for all $x,y,z \in \QProc$ and $P \in \Proc$
  \begin{mathpar}
    x \cdot \quotep{0} \equiv x 
    \and
    x \cdot y \equiv y \cdot x
    \and
    x \cdot (y \cdot z) \equiv (x \cdot y) \cdot z
    \and \\
    \quotep{0} \cdot P \equiv P
    \and \\
    x \cdot (y \cdot P) \equiv (x \cdot y) \cdot P
    \and \\
    x \cdot (P|Q) \equiv (x \cdot P) | (x \cdot Q)
    \and \\    
  \end{mathpar}
\end{remark}

\subsubsection{Tensor product}

We define a tensor product on processes by structural induction.

\paragraph{Tensor of sums} First note that all summations, including
$\pzero$ and sequence, can be written $\Sigma_{i} x_{i}.A_{i} +
\Sigma_{j} x_{j}.C_{j}$, where we have grouped input-guarded processes
together and output-guarded processes together.

Thus, we can define the tensor product of two summations, $N_{1}\otimes N_{2}$, where

\begin{mathpar}
  N_{1} := \Sigma_{i} x_{i}.A_{i} + \Sigma_{j} x_{j}.C_{j}
  \and
  N_{2} := \Sigma_{i'} y_{i'}.B_{i'} + \Sigma_{j'} y_{j'}.D_{j'} 
\end{mathpar}

as follows.

\begin{mathpar}
  \Sigma_{i} x_{i}.A_{i} + \Sigma_{j} x_{j}.C_{j} \otimes \Sigma_{i'}
  y_{i'}.B_{i'} + \Sigma_{j'} y_{j'}.D_{j'} 
  \and \\
  := \; \Sigma_{i} \Sigma_{i'} \quotep{\stackrel{\vee}{x_{i}}| \stackrel{\vee}{y_{i'}}}.(A_{i}\otimes B_{i'}) \; | \; \Sigma_{i'} \Sigma_{i} \quotep{\stackrel{\vee}{y_{i'}}|\stackrel{\vee}{x_{i}}}.(B_{i'}\otimes A_{i})
  \and
  \;\; | \;\; \Sigma_{j} \Sigma_{j'} \quotep{\stackrel{\vee}{x_{j}}|\stackrel{\vee}{y_{j'}}}.(A_{j}\otimes B_{j'}) \; | \; \Sigma_{j'} \Sigma_{j} \quotep{\stackrel{\vee}{y_{j'}}|\stackrel{\vee}{x_{j}}}.(B_{j'}\otimes A_{j})
\end{mathpar}

\begin{remark}
  Do we need to $x^{L}$ and $y^{R}$ for this construction as well?
\end{remark}

\paragraph{Tensor of parallel compositions} Next, we distribute tensor
over par.

\begin{mathpar}
  P_{1}|P_{2} \otimes Q_{1}|Q_{2} := (P_{1} \otimes Q_{1}) | (P_{1}
  \otimes Q_{2}) | (P_{2} \otimes Q_{1}) | (P_{2} \otimes Q_{2})
\end{mathpar}

\paragraph{Tensor with dropped names} We treat tensor of a
process with a dropped name as parallel composition.

\begin{mathpar}
  P \otimes \dropn{x} := P | \dropn{x}
\end{mathpar}

\paragraph{Tensor of agents}

Finally, we need to define tensor on agents. Note that the definition
of tensor on normal products only tensors inputs with inputs and
outputs with outputs. Thus, we only have to define the operation on
``homogeneous'' pairings.

\begin{mathpar}
  (\vec{x})P \otimes (\vec{y})Q
  \and \\
  := (x_{0}^{L}|y_{0}^{R},\ldots,x_{0}^{L}|y_{n}^{R},\ldots,x_{m}^{L}|y_{0}^{R},\ldots,x_{m}^{L}|y_{n}^R)(P\{ \vec{x}^{L}/\vec{x}\} \otimes Q \{ \vec{y}^{R}/\vec{y}\})
  \and \\
  \clift{\vec{P}} \otimes \clift{\vec{Q}}
  \and \\
  := \clift{P_{0}\otimes Q_{0},\ldots,P_{0}\otimes Q_{n},\ldots,P_{m}\otimes Q_{0},\ldots,P_{m}\otimes Q_{n}}
\end{mathpar}

\begin{remark}
  Observe that arities of tensored abstractions matches arities of
  tensored concretions if the original arities matched. Note also that
  the length of the arities corresponds to the increase in dimension
  we see in ordinary vector space tensor product.
\end{remark}

\begin{remark}
  Operationally, this definition distributes the tensor down to
  components ``linked'' by summation. Tensor over summation is
  intriguing in that it mixes names. Moreover, as a consequence of the
  way it mixes names we have the identities for all $x \in \QProc$ and
  $P,Q \in \Proc$

  \begin{mathpar}
    (x \cdot P) \otimes Q \equiv x \cdot (P \otimes Q) \equiv P \otimes (x \cdot Q)
    \and
    P \otimes \pzero \equiv P
  \end{mathpar}

  that the reader is invited to verify.
\end{remark}

\subsubsection{Annihilation}
\begin{mathpar}
  P^{\perp} := \{ Q | \forall R. P|Q \red^{*} R \Rightarrow R \red^{*} \pzero \}
  \and \\
  P^{\underline{\perp}} := \Sigma_{Q \in P^{\perp}} \quotep{Q}?(y).(\dropn{y}|Q) | \Sigma_{Q \in P^{\perp}} \quotep{Q}\clift{\Box}
\end{mathpar}

\paragraph{Discussion} The reader will note that $P^{\perp}$ is a
\emph{set} of processes, while $P^{\underline{\perp}}$ is a
\emph{context}. We call the set $P^{\perp}$ the \emph{annihilators} of
$P$. The parallel composition of a process in the annihilators of $P$
with $P$ will result in a process, the state space of which has all
paths eventually leading to $\pzero$. Execution may endure loops; but
under reasonable conditions of fairness (naturally guaranteed under
most notions of bisimulation) such a composite process cannot get
stuck in such a loop and will, eventually pop out and terminate.

The context $P^{\underline{\perp}}$ is ready and willing to ``take the
$P$ out of'' the process to which it is applied. It will effectively
transmit the code of the process to which it is applied to one of the
annihilators and run the process against it.

\subsubsection{Evaluation}
We fix $M$ a domain of fully abstract interpretation with an equality
coincident with bisimulation. We take $\meaningof{\cdot} : \Proc \to
M$ to be the map interpreting processes and $\nmeaningof{\cdot} : \M
\to Proc$ to be the map running the other way. Then we define

\begin{mathpar}
  \int P := \nmeaningof{\meaningof{P}}
\end{mathpar}

\paragraph{Discussion}
There are many fully abstract interpretations of Milner's
$\pi$-calculus. Any of them can be used as a basis for interpreting
the reflective calculus here. Equipped with such a domain it is
largely a matter of grinding through to check that the Yoneda
construction for the normalization-by-evaluation program can be
extended to this setting.

\begin{remark}
  The reader is invited to verify that $\int (P^{\underline{\perp}}[P]) = 0$.
\end{remark}

\subsection{Quantum mechanics}

Table \ref{tbl:core_qm_op_defns} gives the core operational definitions

\begin{table}[htp]\label{tbl:core_qm_op_defns}
  \center{
    \fbox{
      \begin{tabular}{c|c}
        quantum mechanics & process calculus \\
        \hline
        scalar & $x := \quotep{P}$ \\
        state vector & $\state{P} := P$ \\
        dual & $\state{P}^{*} := \event{P^{\underline{\perp}}} := \quotep{P^{\underline{\perp}}}[-]$ \\
        matrix & $ \Sigma_{\alpha} \state{P_{\alpha}}x_{\alpha}\event{Q_{\alpha}}$ \\
        vector addition & $\state{P} + \state{Q} := \state{P | Q}$ \\
        tensor product & $\state{P} \otimes \state{Q} := \state{P \otimes Q}$ \\
        inner product & $\innerprod{P}{Q} := \quotep{\int P^{\underline{\perp}}[Q]}$ \\
      \end{tabular}
    }
  }
  \caption{QM - operational definitions}
\end{table}

where

\begin{mathpar}
  \prmatrix{P}{Q} := \fprmatrix{P}{\quotep{\pzero}}{Q}
  \and
  \fprmatrix{P}{x}{Q} := (\state{P},x,\event{Q})
  \and
  (\fprmatrix{P}{x}{Q})(\state{R}) := x \cdot \innerprod{Q}{R} \cdot \state{P}
  \and
  (\fprmatrix{P}{x}{Q})(\event{R}) := x \cdot \innerprod{R}{P} \cdot \event{Q}
\end{mathpar}

\paragraph{Discussion}
As promised: vectors (aka states) are represented as processes; duals
as contextual duals; inner product definition should be compared with
standard inner product definition for ....

\begin{remark}
  Assuming $\int (P^{\underline{\perp}}[P]) = 0$, the reader is
  invited to verify that $(\fprmatrix{P}{x}{P})(\state{P}) = x \cdot \state{P}$.
\end{remark}

\begin{remark}
  The reader is invited to verify that $\innerprod{P}{Q}$ could
  equally well have been written $\quotep{\int \stackrel{\vee}{x}}$
  where $x = \event{P^{\underline{\perp}}}(Q)$.

  One of the motivations for this remark is that there is another way
  to factor these operations. We could package up evaluation in the dual:

  \begin{mathpar}
    \state{P}^{*} := \event{\int P^{\underline{\perp}}} := \quotep{\int P^{\underline{\perp}}}[-]
  \end{mathpar}

  and then have inner product defined by
  
  \begin{mathpar}
    \innerprod{P}{Q} := \event{P}(Q)
  \end{mathpar}

  Hopefully, experience with the calculations will provide guidance on
  the best factoring.
\end{remark}

\begin{remark}
  Assuming $\int (P^{\underline{\perp}}[P]) = 0$, the reader is
  invited to verify that $\forall P,Q. (\prmatrix{0}{Q})(\state{0}) =
  \state{0}$ and dually $(\prmatrix{P}{0})(\event{0}) = \event{0}$.
\end{remark}

\begin{remark}
  i'm a little worried that i don't (yet) have proper support for
  complex conjugacy. But, the observation above may give us a
  clue. According to Abramsky, it must be the case that the scalars
  are iso to the homset of the identity for the tensor -- which the
  observation above characterizes. 

  For now, we will simply bookmark the notion with $\overline{x}$.
\end{remark}

\subsubsection{Adjointness}

We need to give a definition of $(\cdot)^{\dagger}$ for matrices. The
obvious candidate definition is
\begin{mathpar}
(\Sigma_{\alpha}\fprmatrix{P_{\alpha}}{x_{\alpha}}{Q_{\alpha}})^{\dagger}
= \Sigma_{\alpha}\fprmatrix{(Q_{\alpha}^{\underline{\perp}})^{*}}{\overline{x}_{\alpha}}{P_{\alpha}^{\underline{\perp}}} 
\end{mathpar}

But, $(Q_{\alpha}^{\underline{\perp}})^{*}$ requires a name along
which to communicate the process to achieve the context application.

\subsubsection{Basis for a basis}
If processes label states and ``addition'' of states (a.k.a. vector
addition) is interpreted as parallel composition, what corresponds to
notions of linear independence and basis? Here, we recall that Yoshida
has developed a set of \emph{combinators} for an asynchronous verison
of Milner's $\pi$-calculus. These are a finite set of processes such
any process can be expressed as parallel composition of these
combinators together with liberal uses of the new operator and
replication. We can simply give a translation of these into the
present calculus and have reasonable expectation that the property
carries over. That is, that the resultant set allows to express all
processes via parallel composition. Note, however, that there is no
new operator or replication in this calculus. As a result, we expect
that the corresponding set is actually infinite. That is, we expect
that the space is actually infinite dimensional.

\begin{remark}
  The attentive reader may be a bit concerned. Certainly, the
  collection $S$, $K$ and $I$ is a finite set of
  combinators. Shouldn't we expect to see a finite set of combinators
  for an effectively equivalent system? i am very sympathetic to this
  critique and feel it warrants full attention. On the other hand, i
  also have in mind the following analogy. The natural numbers, as a
  monoid under addition, has exactly $1$ generator, while the natural
  numbers, as a monoid under multiplication, has countably many
  generators (the primes). We observe that the application of the
  lambda calculus is much less resource sensitive than the parallel
  composition of the $\pi$-calculus. Could it be the case that we have
  an analogy of the form
  
  \begin{mathpar}
    m + n : MN :: m*n : M|N
  \end{mathpar}

  giving a similar blow up in the set of ``primes''?  This is such a
  wonderful thought that, even if it's not true, i think it's worth
  writing down.
\end{remark}
 

\documentclass[12pt]{llncs}
%\documentclass{jktr}

\usepackage[pdftex]{hyperref}                   
\usepackage {listings}
\usepackage {mathpartir}
\usepackage{bcprules}
%\usepackage{listings}
                       
\usepackage{graphicx} 
%\usepackage[margins=2.5cm,nohead,nofoot]{geometry}
%\usepackage{geometry}
\usepackage{amsfonts}
\usepackage{amstext}
\usepackage{latexsym}
\usepackage{amssymb}
\usepackage{color}


%\include{myPreamble}
\documentclass[12pt]{llncs}
%\documentclass{jktr}

\usepackage[pdftex]{hyperref}                   
\usepackage {listings}
\usepackage {mathpartir}
\usepackage{bcprules}
%\usepackage{listings}
                       
\usepackage{graphicx} 
%\usepackage[margins=2.5cm,nohead,nofoot]{geometry}
%\usepackage{geometry}
\usepackage{amsfonts}
\usepackage{amstext}
\usepackage{latexsym}
\usepackage{amssymb}
\usepackage{color}


%\include{myPreamble}
\include{qm2pi.local} 

%\ifpdf
%\usepackage[pdftex]{graphicx}
%\else
%\usepackage{graphicx}
%\fi

 % \ifpdf
%  \usepackage{pdfsync}
%  \if


%\title{Brief Article}
%\author{David F. Snyder}
%\author{L.G. Meredith}

%\address{Dept. of Math., Texas State University--San Marcos, San Marcos, TX 78666}
       
\pagestyle{empty}


\begin{document}

\lstset{language=[Objective]Caml,frame=shadowbox}

\input{qm2pi.front}

% section front matter (end)

\input{qm2pi.intro} 
 
% section introduction (end)

% \input{qm2pi.knotations} 

% section notation (end)

\input{qm2pi.process.calculi} 

% section concurrent_process_calculi_and_spatial_logics_ (end)
    
%\input{qm2pi.knots2pi} 

%\input{qm2pi.trefoil} 

%\input{qm2pi.mainthm} 

% subsection basic_interpretation (end)

%\input{qm2pi.rho.presentation} 
\subsection{The syntax and semantics of the notation system}\label{sub:the_syntax_and_semantics_of_the_notation_system} % (fold)

We now summarize a technical presentation of the calculus that
embodies our theory of dynamics. The typical presentation of such a
calculus follows the style of giving generators and relations on
them. The grammar, below, describing term constructors, freely
generates the set of processes, $\Proc$. This set is then quotiented
by a relation known as structural congruence and it is over this set
that the notion of dynamics is expressed. This presentation is
essentially that of \cite{MeredithR05} with the addition of
polyadicity and summation. For readability we have relegated some of
the technical subtleties to an appendix.

\subsubsection{Process grammar}\label{subsub:process_grammar}

\begin{mathpar}
  \inferrule* [lab=synchronization] {} {{M} \bc \pzero \;|\; x?F \;|\; x!C }
  \and
  \inferrule* [lab=abstraction] {} {{F} \bc (x)P}
  \and
  \inferrule* [lab=concretion] {} {{C} \bc \langle Q \rangle}
  \and
  \inferrule* [lab=process] {} {{P,Q} \bc M \;| \;P|Q \;|\; @{x}}
  \and
  \inferrule* [lab=name] {} {{x} \bc \quotep{P}}
\end{mathpar} 

Note that $\vec{x}$ (resp. $\vec{P}$) denotes a vector of names
(resp. processes) of length $|\vec{x}|$ (resp. $|\vec{P}|$). We adopt
the following useful abbreviations.

\begin{mathpar}
   x?(\vec{y}).P := x.(\vec{y})P \and  x\clift{\vec{P}} := x.\clift{\vec{P}}
   \and x!(y) := \lift{x}{\dropn{y}}
   \and \Pi_{i=0}^{n-1}P_i := P_0 | \ldots | P_{n-1}
\end{mathpar}

\subsubsection{Structural congruence}

\paragraph{Free and bound names and alpha-equivalence.} At the
core of structural equivalence is alpha-equivalence which identifies
process that are the same up to a change of variable. Formally, we
recognize the distinction between free and bound names. The free names
of a process, $\freenames{P}$, may be calculated recursively as
follows:

\begin{mathpar}
\freenames{\pzero} := \emptyset
  \and \\
  \freenames{x?(y).P} := \{ x \} \cup (\freenames{P} \setminus \{ y \})
  \and 
  \freenames{x!\langle P \rangle} := \{ x \} \cup \{ P \} 
  \and \\
  \freenames{P|Q} := \freenames{P} \cup \freenames{Q}
  \and \\
  \freenames{@{x}} := \{ x \}
\end{mathpar}

$\pi$
$\quotep{\pi}$

$\freenames{-} : \pi \to \mathcal{P}(\quotep{\pi})$

\begin{eqnarray*}
  \freenames{\pzero} & := & \emptyset \\
  \freenames{x?(y).P} & := & \{ x \} \cup (\freenames{P} \setminus \{ y \}) \\
  \freenames{x!\langle P \rangle} & := & \{ x \} \cup \{ P \} \\
  \freenames{P|Q} & := & \freenames{P} \cup \freenames{Q} \\
  \freenames{\dropn{x}} & := & \{ x \}
\end{eqnarray*}

The bound names of a process, $\boundnames{P}$, are those names occurring in $P$
that are not free. For example, in $x?(y).0$, the name $x$ is free, while $y$ is bound.

\begin{mathpar}
  \inferrule* [lab=monoidal-laws] {} { P|Q \equiv Q|P \and P|0 \equiv P \and P|(Q|R) \equiv (P|Q)|R }
\end{mathpar}

\begin{mathpar}
  \inferrule* [lab=alpha-equivalence] {} { (x)P \equiv (y)P\{y/x\} \and y \not\in \freenames{P} }
\end{mathpar}

\begin{definition}
Then two processes, $P,Q$, are alpha-equivalent if $P = Q\{\vec{y}/\vec{x}\}$ for
some $\vec{x} \in \boundnames{Q},\vec{y} \in \boundnames{P}$, where $Q\{\vec{y}/\vec{x}\}$
denotes the capture-avoiding substitution of $\vec{y}$ for $\vec{x}$ in $Q$.
\end{definition}

\begin{definition}
  The {\em structural congruence} \cite{SangiorgiWalker} , $\equiv$,
  between processes is the least congruence containing
  alpha-equivalence, satisfying the abelian monoid laws
  (associativity, commutativity and $\pzero$ as identity) for parallel
  composition $|$ and for summation $+$.
\end{definition}

\subsection{Name equivalence}

We take name equivalence, written $\nameeq$, to be the smallest
equivalence relation generated by the following rules.

\begin{mathpar}
\inferrule*[lab=Quote-drop]
{ }
{ \quotep{@{x}} \nameeq x }

\inferrule*[lab=Struct-equiv]
{ P \scong Q }
{ \quotep{P} \nameeq \quotep{Q} }
\end{mathpar}

The astute reader will have noticed that the mutual recursion of names
and processes imposes a mutual recursion on alpha-equivalence and
structural equivalence via name-equivalence. Fortunately, all of this
works out pleasantly and we may calculate in the natural way, free of
concern. The reader interested in the details is referred to the
appendix \ref{appendix:rho_details}.

\subsection{Substitution}

We use $\Proc$ for the set of processes, $\QProc$ for the set of
names, and $\id{\{}\vec{y} / \vec{x} \id{\}}$ to denote partial maps,
$s : \QProc \rightarrow \QProc$. A map, $s$ lifts, uniquely, to a map
on process terms, $\widehat{s} : \Proc \rightarrow \Proc$ by the
following equations.

\begin{mathpar}
  (0) \psubstp{Q}{P} := 0 \\
  (R \juxtap S) \psubstp{Q}{P}
  :=    
  (R)\psubstp{Q}{P} \juxtap (S) \psubstp{Q}{P} \\
  (x?(y).R) \psubstp{Q}{P}    
  :=    
  (x)\substp{Q}{P} (z)\concat( (R \psubstn{z}{y}) \psubstp{Q}{P} ) \\
  (\lift{x}{R}) \psubstp{Q}{P}  
  :=
  \lift{(x)\substp{Q}{P}}{ R \psubstp{Q}{P} } \\
%   (\dropn{x})  \psubstp{Q}{P}       
%   := 
%   \left\{ 
%     \begin{array}{ccc} 
%       \dropn{\quotep{Q}} & & x \nameeq \quotep{P} \\
%       \dropn{x} & & otherwise \\
%     \end{array}
%   \right. 
  (\dropn{x})  \psubstp{Q}{P}       
  := 
  \left\{ 
    \begin{array}{ccc} 
      Q & & x \nameeq \quotep{P} \\
      \dropn{x} & & otherwise \\
    \end{array}
  \right.
\end{mathpar}
 

where

\begin{eqnarray}
  (x)\id{\{} \lpquote Q \rpquote / \lpquote P \rpquote \id{\}}            = 
  \left\{ 
    \begin{array}{ccc}
      \lpquote Q \rpquote & & x \nameeq \lpquote P \rpquote \\
      x & & otherwise \\
    \end{array}
  \right. \nonumber
\end{eqnarray}

and $z$ is chosen distinct from $\quotep{P}$, $\quotep{Q}$, the free
names in $Q$, and all the names in $R$. Our $\alpha$-equivalence will
be built in the standard way from this substitution.

\begin{remark}\label{rem:no_self_referential_names}
  One consequence of these definitions is that $\forall P. \quotep{P}
  \not\in \freenames{P}$.
\end{remark}

\subsection{ Dynamic quote: an example }

Anticipating something of what's to come, consider applying the
substitution, $\widehat{\id{\{}u / z \id{\}}}$, to the following pair
of processes, $\lift{w}{y!(z)}$ and $w[ \lpquote y!(z) \rpquote ]$.

\begin{eqnarray}
	\lift{w}{y!(z)}\widehat{\id{\{}u / z \id{\}}}
		& = &
		\lift{w}{y!(u)} \nonumber\\
	w[ \lpquote y!(z) \rpquote ] \widehat{ \id{\{}u / z \id{\}} }
		& = &
		w[ \lpquote y!(z) \rpquote ] \nonumber
\end{eqnarray}

Because the body of the process between quotes is impervious to
substitution, we get radically different answers. In fact, by
examining the first process in an input context,
e.g. $x?(z).\lift{w}{y!(z)}$, we see that the process under the lift
operator may be shaped by prefixed inputs binding a name inside it. In
this sense, the lift operator will be seen as a way to dynamically
construct processes before reifying them as names.

Finally equipped with these standard features we can present the
dynamics of the calculus.

\subsubsection{Operational semantics} 

Finally, we introduce the computational dynamics. What marks these
algebras as distinct from other more traditionally studied algebraic
structures, e.g. vector spaces or polynomial rings, is the manner in
which dynamics is captured. In traditional structures, dynamics is typically
expressed through morphisms between such structures, as in linear maps
between vector spaces or morphisms between rings. In algebras
associated with the semantics of computation, the dynamics is
expressed as part of the algebraic structure itself, through a
reduction reduction relation typically denoted by $\red$. Below, we
give a recursive presentation of this relation for the calculus used
in the encoding.

$\red \subseteq \pi \times \pi$
$\red : \pi \to \mathcal{P}(\pi)$

\begin{mathpar}
  \inferrule* [lab=Comm] { \textsf{match}( x_{src}, x_{trgt} ) } { x_{trgt}?(y)P \; | \; x_{src}!\langle {Q} \rangle \red P\{\quotep{Q}/y}\} }
  \and \\
  \inferrule* [lab=Par] {{P} \red {P}'} {{{P} | {Q}} \red {{P}' | {Q}}}
  \and
  \inferrule* [lab=Equiv]{{{P} \scong {P}'} \andalso {{P}' \red {Q}'} \andalso {{Q}' \scong {Q}}}{{P} \red {Q}}
\end{mathpar}

\begin{eqnarray*}
  match_{\equiv} (\quotep{P},\quotep{Q}) & := & P \equiv Q \\
  match_{\dagger}(\quotep{P},\quotep{Q}) & := & \forall R. P|Q \red^{*} R => R \red^{*} 0 \\
  match_{K}(\quotep{P},\quotep{Q}) & := & K \mbox{ for some context } K
\end{eqnarray*}

$u?(x)P | u!\langle Q \rangle \red P\{\quotep{Q}/x\}$

%We write $\wred$ for $\red^*$, and $P\red$ if $\exists Q $ such that $ P \red Q$.
We write $P\red$ if $\exists Q $ such that $ P \red Q$ and $P\not\red$, otherwise.

\section{Replication}

As mentioned before, it is known that replication (and hence
recursion) can be implemented in a higher-order process algebra
\cite{SangiorgiWalker}. As our first example of calculation with the
machinery thus far presented we give the construction explicitly in
the {\rhoc}.

\begin{eqnarray}
	D_{x} & := & \prefix{x}{y}{(\binpar{\outputp{x}{y}}{@{y}})} \nonumber\\
	\bangp_{x}{P} & := & \binpar{{x}!\langle{\binpar{D_{x}}{P}}\rangle}{D_{x}} \nonumber
\end{eqnarray}

\begin{eqnarray}
	\bangp_{x}{P} & & \nonumber\\
	=
	& {x}!\langle{(\prefix{x}{y}{(\outputp{x}{y} | @{y})) | P}}\rangle 
	      | \prefix{x}{y}{(\outputp{x}{y} | @{y})} & \nonumber\\
	\red
	& (\outputp{x}{y} | @{y})\substn{\quotep{(\prefix{x}{y}{(@{y} | \outputp{x}{y})) | P}}}{y} & \nonumber\\
	=
	& \outputp{x}{\quotep{(\prefix{x}{y}{(\outputp{x}{y} | @{y})) | P}}}
	  | {(\prefix{x}{y}{(\outputp{x}{y} | @{y})) | P}} & \nonumber\\
	\red
	& \ldots & \nonumber\\
	\red^*
	& P | P | \ldots & \nonumber
\end{eqnarray}

Of course, this encoding, as an implementation, runs away, unfolding
$\bangp{P}$ eagerly. A lazier and more implementable replication
operator, restricted to input-guarded processes, may be obtained as follows.

\begin{eqnarray}
\bangp{\prefix{u}{v}{P}} 
	:= 
	\binpar{\lift{x}{\prefix{u}{v}{(\binpar{D(x)}{P})}}}{D(x)} \nonumber
\end{eqnarray}

\begin{remark}
  Note that the lazier definition still does not deal with summation
  or mixed summation (i.e. sums over input and output). The reader is
  invited to construct definitions of replication that deal with these
  features. 

  Further, the definitions are parameterized in a name, $x$. Can you,
  gentle reader, make a definition that eliminates this parameter and
  guarantees no accidental interaction between the replication
  machinery and the process being replicated -- i.e. no accidental
  sharing of names used by the process to get its work done and the
  name(s) used by the replication to effect copying. This latter
  revision of the definition of replication is crucial to obtaining
  the expected identity $!!P \sim !P$.
\end{remark}

\begin{remark}\label{rem:paradoxical_combinator}
  The reader familiar with the lambda calculus will have noticed the
  similarity between $D$ and the paradoxical combinator.

  [Ed. note: the existence of this seems to suggest we have to be more
  restrictive on the set of processes and names we admit if we are to
  support no-cloning.]
\end{remark}

\subsubsection{Bisimulation}

The computational dynamics gives rise to another kind of equivalence,
the equivalence of computational behavior. As previously mentioned
this is typically captured \emph{via} some form of bisimulation.

% The notion we use in this paper is weak barbed bisimulation
% \cite{milner91polyadicpi}.

The notion we use in this paper is derived from weak barbed
bisimulation \cite{milner91polyadicpi}. 

\begin{definition}
An \emph{observation relation}, $\downarrow_{\mathcal N}$, over a set
of names, $\mathcal N$, is the smallest relation satisfying the rules
below.

\infrule[Out-barb]{y \in {\mathcal N}, \; x \nameeq y}
		  {\outputp{x}{v} \downarrow_{\mathcal N} x}
\infrule[Par-barb]{\mbox{$P\downarrow_{\mathcal N} x$ or $Q\downarrow_{\mathcal N} x$}}
		  {\binpar{P}{Q} \downarrow_{\mathcal N} x}

We write $P \Downarrow_{\mathcal N} x$ if there is $Q$ such that 
$P \wred Q$ and $Q \downarrow_{\mathcal N} x$.
\end{definition}

\begin{definition}
%\label{def.bbisim}
An  ${\mathcal N}$-\emph{barbed bisimulation} over a set of names, ${\mathcal N}$, is a symmetric binary relation 
${\mathcal S}_{\mathcal N}$ between agents such that $P\rel{S}_{\mathcal N}Q$ implies:
\begin{enumerate}
\item If $P \red P'$ then $Q \wred Q'$ and $P'\rel{S}_{\mathcal N} Q'$.
\item If $P\downarrow_{\mathcal N} x$, then $Q\Downarrow_{\mathcal N} x$.
\end{enumerate}
$P$ is ${\mathcal N}$-barbed bisimilar to $Q$, written
$P \wbbisim_{\mathcal N} Q$, if $P \rel{S}_{\mathcal N} Q$ for some ${\mathcal N}$-barbed bisimulation ${\mathcal S}_{\mathcal N}$.
\end{definition}

$\mathcal{R} \subseteq \pi \times \pi$

$P \mathcal{R} Q => \forall P'. P \red P' \Rightarrow \exists Q'. Q \red Q', P' \mathcal{R} Q'$

$P \vdash x \Rightarrow Q \vdash x$

\begin{mathpar}
  \inferrule*[lab=Out-barb]{x \nameeq y}{{y}!\langle{Q}\rangle \vdash x}
  \and
  \inferrule*[lab=Par-barb]{\mbox{$P\vdash x$ or $Q\vdash x$}}{\binpar{P}{Q} \vdash x}
\end{mathpar}

\subsubsection{Contexts}

One of the principle advantages of computational calculi like the
$\pi$-calculus is a well-defined notion of context,
contextual-equivalence and a correlation between
contextual-equivalence and notions of bisimulation. The notion of
context allows the decomposition of a process into (sub-)process and
its syntactic environment, its context. Thus, a context may be
thought of as a process with a ``hole'' (written $\Box$) in it. The
application of a context $M$ to a process $P$, written $M[P]$, is
tantamount to filling the hole in $M$ with $P$. In this paper we do
not need the full weight of this theory, but do make use of the notion
of context in the proof the main theorem. 

\begin{mathpar}
  \inferrule* [lab=summation] {} {{M_{M},M_{N}} \bc \Box \;|\; x.M_{A} \;|\; M_{M}+M_{N}}
  \and
  \inferrule* [lab=agent] {} {{M_{A}} \bc (\vec{x})M_{P} \;| \; \clift{P_0,\ldots,M_{P},\ldots,P_N}}
  \and \\
  \inferrule* [lab=process] {} {{M_{P}} \bc M_{N} \;| \;P|M_{P} }
\end{mathpar} 

\begin{mathpar}
  \inferrule* [lab=sychronization] {} {M_{N} \bc \Box \;|\; x?M_{F} \;|\; x!M_{C}}
  \and
  \inferrule* [lab=abstraction] {} {{M_{F}} \bc (x)M_{P} }
  \and
  \inferrule* [lab=concretion] {} {{M_{C}} \bc \langle M_{P} \rangle }
  \and \\
  \inferrule* [lab=process] {} {{M_{P}} \bc M_{N} \;| \;P|M_{P} }
\end{mathpar}

\begin{definition}[contextual application] Given a context $M$, and
  process $P$, we define the \emph{contextual application}, $M[P] :=
  M\{P/\Box\}$. That is, the contextual application of M to P is the
  substitution of $P$ for $\Box$ in $M$.
\end{definition}

$\meaningof{-} : L \to \mathcal{P}(\pi)$

\begin{mathpar}
  \inferrule* [lab=collection] {} {\meaningof{true} = \pi, \and \meaningof{~E} = \pi \setminus \meaningof{E}, \and \meaningof{E_{1} \& E_{2}} = \meaningof{E_{1}} \cap \meaningof{E_{2}}}
\end{mathpar}

\begin{mathpar}
  \inferrule* [lab=structure] {} {\meaningof{0} = \{ P \in \pi | P \equiv 0 \}, \and \\ \meaningof{E_1 | E_2} = \{ P \in \pi | P \equiv P_{1} | P_{2}, P_{1} \in \meaningof{E_{1}}, P_{2} \in \meaningof{E_2}\} }
\end{mathpar}

\begin{mathpar}
 \inferrule* [lab=behavior] {} {\meaningof{\langle a?b \rangle E} = \{ P \in \pi | P \equiv Q | u?(y)P', \\ \and \\\\ \and \\ \;\;\; u \in \meaningof{a}, \forall z.P'\{z/y\} \in \meaningof{E\{z/b\}}\}, \and \\ \meaningof{a!E} = \{ P \in \pi | P \equiv Q | x!\langle P' \rangle, x \in \meaningof{a} P' \in \meaningof{E}\} }
\end{mathpar}

\begin{mathpar}
 \inferrule* [lab=nominal] {} {\meaningof{\quotep{E}} = \{ \quotep{P} \in \quotep{\pi} | P \in \meaningof{E} \}, \and \meaningof{\quotep{P}} = \{ \quotep{Q} \in \quotep{\pi} | P \equiv Q \} \and \\ \meaningof{@\quotep{E}} = \{ P \in \pi | P \equiv @x, x \in \meaningof{E} \}}
\end{mathpar}

\begin{eqnarray*}
  \\
  \meaningof{-} : TS \to ST
\end{eqnarray*}

\begin{eqnarray*}
  \\
  L : TS \to ST
\end{eqnarray*}

\begin{eqnarray*}
  \\
  P \models E \iff P \in \meaningof{E}
\end{eqnarray*}

\begin{eqnarray*}
  P \approx_{L} Q \iff \forall E \in L. P \models E \iff Q \models E
\end{eqnarray*}

\begin{eqnarray*}
  P \approx_{K} Q
\end{eqnarray*}

\begin{eqnarray*}
  P \approx Q
\end{eqnarray*}

$\approx_{K} = \approx = \approx_{L}$

\subsubsection{Contextual duality}

Note that contexts extend the quotation operation to a family of
operations from processes to names. Given a context, $M$, we can
define a \emph{nominal context}, $\quotep{M}$ by $\quotep{M}[P] :=
\quotep{M[P]}$. To foreshadow what is to come we observe that these
operations enjoy a duality with processes very much like the duality
between vectors and maps from vectors to scalars.

Further, because the calculus is essentially higher-order, we have a
correspondence between contexts and processes. More specifically,
given a name $x$ and a context $M$ we can construct $M^{*}_{x}$ such
that 

\begin{mathpar}
  M^{*}_{x} | \lift{x}{P} \red M[P]
\end{mathpar}

namely,

\begin{mathpar}
  M^{*}_{x} := x?(u).M[\dropn{u}]
\end{mathpar}

The dependence of $M^{*}_{x}$ on a name makes it an abstraction, 

\begin{mathpar}
  M^{*} := (x)x?(u).M[\dropn{u}]
\end{mathpar}

\subsection{Additional notation}

It will sometimes be convenient to denote the process a name
quotes. We already have the notation $x = \quotep{P}$, but it will be
convenient to introduce an alternate notation, $\procn{x}$, when we
want to emphasize the connection to the use of the name. Note that, by
virtue of name equivalence, $\quotep{\procn{x}} \nameeq x$; so, the
notation is consistent with previous definitions.

Further, because names have structure it is possible to effect
substitutions on the basis of that structure. This means we need to
upgrade our notation for substitutions, which we accomplish by
adapting comprehension notation. Thus,

\begin{mathpar}
  P\{ y / x : x \in S \}
\end{mathpar}

is interpreted to mean the process derived from P by replacing (in a
capture-avoiding manner) each occurrence of $x$ in $S$ by $y$. For example,

\begin{mathpar}
  P\{ \quotep{\procn{x}|\procn{x}} / x : x \in \freenames{P} \}
\end{mathpar}

will replace each (occurrence) of a free name $x$ in $P$ by
$\quotep{\procn{x}|\procn{x}}$.

Also, we will avail ourselves of the notation $x^{L}$ and $x^{R}$ to
denote injections of a name into disjoint copies of the name
space. There are numerous ways to accomplish this. One example can be
found in \cite{MeredithR05}. This notation overloads to vectors of
names: $\vec{x}^{\pi} := (x_{i}^{\pi} \; : \; 0 \leq i < |\vec{x}| )$ where $\pi \in \{L,R\}$.

We also use $P^{\Box} := P|\Box$.

In \cite{MeredithR05} an interpretation of the new operator is
given. It turns out that there are several possible interpretations
all enjoying the requisite algebraic properties of the operator (see
\cite{milner91polyadicpi}). We will therefore make liberal use of
$(\nu\; \vec{x})P$.

% subsection the_syntax_and_semantics_of_the_notation_system (end)   

\input{qm2pi.qmops} 

\input{qm2pi.sterngerlach} 

\input{qm2pi.metric} 

% section concurrent_process_calculi (end)

%\input{qm2pi.proofsketch}

% section proof sketch (end)

%\input{qm2pi.slviaknots} 

% section spatial logic via knots (end)

\input{qm2pi.conclusion}

% section conclusion (end)

%\input{qm2pi.dtcodes} 

% section wiring algorithm (end)

\input{qm2pi.ack} 

% section acknowledgments (end)

\newpage


\bibliographystyle{plain}   
\bibliography{../../biblios/main.bib}

\input{qm2pi.rhodetails}

\end{document}

 

%\ifpdf
%\usepackage[pdftex]{graphicx}
%\else
%\usepackage{graphicx}
%\fi

 % \ifpdf
%  \usepackage{pdfsync}
%  \if


%\title{Brief Article}
%\author{David F. Snyder}
%\author{L.G. Meredith}

%\address{Dept. of Math., Texas State University--San Marcos, San Marcos, TX 78666}
       
\pagestyle{empty}


\begin{document}

\lstset{language=[Objective]Caml,frame=shadowbox}

\documentclass[12pt]{llncs}
%\documentclass{jktr}

\usepackage[pdftex]{hyperref}                   
\usepackage {listings}
\usepackage {mathpartir}
\usepackage{bcprules}
%\usepackage{listings}
                       
\usepackage{graphicx} 
%\usepackage[margins=2.5cm,nohead,nofoot]{geometry}
%\usepackage{geometry}
\usepackage{amsfonts}
\usepackage{amstext}
\usepackage{latexsym}
\usepackage{amssymb}
\usepackage{color}


%\include{myPreamble}
\include{qm2pi.local} 

%\ifpdf
%\usepackage[pdftex]{graphicx}
%\else
%\usepackage{graphicx}
%\fi

 % \ifpdf
%  \usepackage{pdfsync}
%  \if


%\title{Brief Article}
%\author{David F. Snyder}
%\author{L.G. Meredith}

%\address{Dept. of Math., Texas State University--San Marcos, San Marcos, TX 78666}
       
\pagestyle{empty}


\begin{document}

\lstset{language=[Objective]Caml,frame=shadowbox}

\input{qm2pi.front}

% section front matter (end)

\input{qm2pi.intro} 
 
% section introduction (end)

% \input{qm2pi.knotations} 

% section notation (end)

\input{qm2pi.process.calculi} 

% section concurrent_process_calculi_and_spatial_logics_ (end)
    
%\input{qm2pi.knots2pi} 

%\input{qm2pi.trefoil} 

%\input{qm2pi.mainthm} 

% subsection basic_interpretation (end)

%\input{qm2pi.rho.presentation} 
\subsection{The syntax and semantics of the notation system}\label{sub:the_syntax_and_semantics_of_the_notation_system} % (fold)

We now summarize a technical presentation of the calculus that
embodies our theory of dynamics. The typical presentation of such a
calculus follows the style of giving generators and relations on
them. The grammar, below, describing term constructors, freely
generates the set of processes, $\Proc$. This set is then quotiented
by a relation known as structural congruence and it is over this set
that the notion of dynamics is expressed. This presentation is
essentially that of \cite{MeredithR05} with the addition of
polyadicity and summation. For readability we have relegated some of
the technical subtleties to an appendix.

\subsubsection{Process grammar}\label{subsub:process_grammar}

\begin{mathpar}
  \inferrule* [lab=synchronization] {} {{M} \bc \pzero \;|\; x?F \;|\; x!C }
  \and
  \inferrule* [lab=abstraction] {} {{F} \bc (x)P}
  \and
  \inferrule* [lab=concretion] {} {{C} \bc \langle Q \rangle}
  \and
  \inferrule* [lab=process] {} {{P,Q} \bc M \;| \;P|Q \;|\; @{x}}
  \and
  \inferrule* [lab=name] {} {{x} \bc \quotep{P}}
\end{mathpar} 

Note that $\vec{x}$ (resp. $\vec{P}$) denotes a vector of names
(resp. processes) of length $|\vec{x}|$ (resp. $|\vec{P}|$). We adopt
the following useful abbreviations.

\begin{mathpar}
   x?(\vec{y}).P := x.(\vec{y})P \and  x\clift{\vec{P}} := x.\clift{\vec{P}}
   \and x!(y) := \lift{x}{\dropn{y}}
   \and \Pi_{i=0}^{n-1}P_i := P_0 | \ldots | P_{n-1}
\end{mathpar}

\subsubsection{Structural congruence}

\paragraph{Free and bound names and alpha-equivalence.} At the
core of structural equivalence is alpha-equivalence which identifies
process that are the same up to a change of variable. Formally, we
recognize the distinction between free and bound names. The free names
of a process, $\freenames{P}$, may be calculated recursively as
follows:

\begin{mathpar}
\freenames{\pzero} := \emptyset
  \and \\
  \freenames{x?(y).P} := \{ x \} \cup (\freenames{P} \setminus \{ y \})
  \and 
  \freenames{x!\langle P \rangle} := \{ x \} \cup \{ P \} 
  \and \\
  \freenames{P|Q} := \freenames{P} \cup \freenames{Q}
  \and \\
  \freenames{@{x}} := \{ x \}
\end{mathpar}

$\pi$
$\quotep{\pi}$

$\freenames{-} : \pi \to \mathcal{P}(\quotep{\pi})$

\begin{eqnarray*}
  \freenames{\pzero} & := & \emptyset \\
  \freenames{x?(y).P} & := & \{ x \} \cup (\freenames{P} \setminus \{ y \}) \\
  \freenames{x!\langle P \rangle} & := & \{ x \} \cup \{ P \} \\
  \freenames{P|Q} & := & \freenames{P} \cup \freenames{Q} \\
  \freenames{\dropn{x}} & := & \{ x \}
\end{eqnarray*}

The bound names of a process, $\boundnames{P}$, are those names occurring in $P$
that are not free. For example, in $x?(y).0$, the name $x$ is free, while $y$ is bound.

\begin{mathpar}
  \inferrule* [lab=monoidal-laws] {} { P|Q \equiv Q|P \and P|0 \equiv P \and P|(Q|R) \equiv (P|Q)|R }
\end{mathpar}

\begin{mathpar}
  \inferrule* [lab=alpha-equivalence] {} { (x)P \equiv (y)P\{y/x\} \and y \not\in \freenames{P} }
\end{mathpar}

\begin{definition}
Then two processes, $P,Q$, are alpha-equivalent if $P = Q\{\vec{y}/\vec{x}\}$ for
some $\vec{x} \in \boundnames{Q},\vec{y} \in \boundnames{P}$, where $Q\{\vec{y}/\vec{x}\}$
denotes the capture-avoiding substitution of $\vec{y}$ for $\vec{x}$ in $Q$.
\end{definition}

\begin{definition}
  The {\em structural congruence} \cite{SangiorgiWalker} , $\equiv$,
  between processes is the least congruence containing
  alpha-equivalence, satisfying the abelian monoid laws
  (associativity, commutativity and $\pzero$ as identity) for parallel
  composition $|$ and for summation $+$.
\end{definition}

\subsection{Name equivalence}

We take name equivalence, written $\nameeq$, to be the smallest
equivalence relation generated by the following rules.

\begin{mathpar}
\inferrule*[lab=Quote-drop]
{ }
{ \quotep{@{x}} \nameeq x }

\inferrule*[lab=Struct-equiv]
{ P \scong Q }
{ \quotep{P} \nameeq \quotep{Q} }
\end{mathpar}

The astute reader will have noticed that the mutual recursion of names
and processes imposes a mutual recursion on alpha-equivalence and
structural equivalence via name-equivalence. Fortunately, all of this
works out pleasantly and we may calculate in the natural way, free of
concern. The reader interested in the details is referred to the
appendix \ref{appendix:rho_details}.

\subsection{Substitution}

We use $\Proc$ for the set of processes, $\QProc$ for the set of
names, and $\id{\{}\vec{y} / \vec{x} \id{\}}$ to denote partial maps,
$s : \QProc \rightarrow \QProc$. A map, $s$ lifts, uniquely, to a map
on process terms, $\widehat{s} : \Proc \rightarrow \Proc$ by the
following equations.

\begin{mathpar}
  (0) \psubstp{Q}{P} := 0 \\
  (R \juxtap S) \psubstp{Q}{P}
  :=    
  (R)\psubstp{Q}{P} \juxtap (S) \psubstp{Q}{P} \\
  (x?(y).R) \psubstp{Q}{P}    
  :=    
  (x)\substp{Q}{P} (z)\concat( (R \psubstn{z}{y}) \psubstp{Q}{P} ) \\
  (\lift{x}{R}) \psubstp{Q}{P}  
  :=
  \lift{(x)\substp{Q}{P}}{ R \psubstp{Q}{P} } \\
%   (\dropn{x})  \psubstp{Q}{P}       
%   := 
%   \left\{ 
%     \begin{array}{ccc} 
%       \dropn{\quotep{Q}} & & x \nameeq \quotep{P} \\
%       \dropn{x} & & otherwise \\
%     \end{array}
%   \right. 
  (\dropn{x})  \psubstp{Q}{P}       
  := 
  \left\{ 
    \begin{array}{ccc} 
      Q & & x \nameeq \quotep{P} \\
      \dropn{x} & & otherwise \\
    \end{array}
  \right.
\end{mathpar}
 

where

\begin{eqnarray}
  (x)\id{\{} \lpquote Q \rpquote / \lpquote P \rpquote \id{\}}            = 
  \left\{ 
    \begin{array}{ccc}
      \lpquote Q \rpquote & & x \nameeq \lpquote P \rpquote \\
      x & & otherwise \\
    \end{array}
  \right. \nonumber
\end{eqnarray}

and $z$ is chosen distinct from $\quotep{P}$, $\quotep{Q}$, the free
names in $Q$, and all the names in $R$. Our $\alpha$-equivalence will
be built in the standard way from this substitution.

\begin{remark}\label{rem:no_self_referential_names}
  One consequence of these definitions is that $\forall P. \quotep{P}
  \not\in \freenames{P}$.
\end{remark}

\subsection{ Dynamic quote: an example }

Anticipating something of what's to come, consider applying the
substitution, $\widehat{\id{\{}u / z \id{\}}}$, to the following pair
of processes, $\lift{w}{y!(z)}$ and $w[ \lpquote y!(z) \rpquote ]$.

\begin{eqnarray}
	\lift{w}{y!(z)}\widehat{\id{\{}u / z \id{\}}}
		& = &
		\lift{w}{y!(u)} \nonumber\\
	w[ \lpquote y!(z) \rpquote ] \widehat{ \id{\{}u / z \id{\}} }
		& = &
		w[ \lpquote y!(z) \rpquote ] \nonumber
\end{eqnarray}

Because the body of the process between quotes is impervious to
substitution, we get radically different answers. In fact, by
examining the first process in an input context,
e.g. $x?(z).\lift{w}{y!(z)}$, we see that the process under the lift
operator may be shaped by prefixed inputs binding a name inside it. In
this sense, the lift operator will be seen as a way to dynamically
construct processes before reifying them as names.

Finally equipped with these standard features we can present the
dynamics of the calculus.

\subsubsection{Operational semantics} 

Finally, we introduce the computational dynamics. What marks these
algebras as distinct from other more traditionally studied algebraic
structures, e.g. vector spaces or polynomial rings, is the manner in
which dynamics is captured. In traditional structures, dynamics is typically
expressed through morphisms between such structures, as in linear maps
between vector spaces or morphisms between rings. In algebras
associated with the semantics of computation, the dynamics is
expressed as part of the algebraic structure itself, through a
reduction reduction relation typically denoted by $\red$. Below, we
give a recursive presentation of this relation for the calculus used
in the encoding.

$\red \subseteq \pi \times \pi$
$\red : \pi \to \mathcal{P}(\pi)$

\begin{mathpar}
  \inferrule* [lab=Comm] { \textsf{match}( x_{src}, x_{trgt} ) } { x_{trgt}?(y)P \; | \; x_{src}!\langle {Q} \rangle \red P\{\quotep{Q}/y}\} }
  \and \\
  \inferrule* [lab=Par] {{P} \red {P}'} {{{P} | {Q}} \red {{P}' | {Q}}}
  \and
  \inferrule* [lab=Equiv]{{{P} \scong {P}'} \andalso {{P}' \red {Q}'} \andalso {{Q}' \scong {Q}}}{{P} \red {Q}}
\end{mathpar}

\begin{eqnarray*}
  match_{\equiv} (\quotep{P},\quotep{Q}) & := & P \equiv Q \\
  match_{\dagger}(\quotep{P},\quotep{Q}) & := & \forall R. P|Q \red^{*} R => R \red^{*} 0 \\
  match_{K}(\quotep{P},\quotep{Q}) & := & K \mbox{ for some context } K
\end{eqnarray*}

$u?(x)P | u!\langle Q \rangle \red P\{\quotep{Q}/x\}$

%We write $\wred$ for $\red^*$, and $P\red$ if $\exists Q $ such that $ P \red Q$.
We write $P\red$ if $\exists Q $ such that $ P \red Q$ and $P\not\red$, otherwise.

\section{Replication}

As mentioned before, it is known that replication (and hence
recursion) can be implemented in a higher-order process algebra
\cite{SangiorgiWalker}. As our first example of calculation with the
machinery thus far presented we give the construction explicitly in
the {\rhoc}.

\begin{eqnarray}
	D_{x} & := & \prefix{x}{y}{(\binpar{\outputp{x}{y}}{@{y}})} \nonumber\\
	\bangp_{x}{P} & := & \binpar{{x}!\langle{\binpar{D_{x}}{P}}\rangle}{D_{x}} \nonumber
\end{eqnarray}

\begin{eqnarray}
	\bangp_{x}{P} & & \nonumber\\
	=
	& {x}!\langle{(\prefix{x}{y}{(\outputp{x}{y} | @{y})) | P}}\rangle 
	      | \prefix{x}{y}{(\outputp{x}{y} | @{y})} & \nonumber\\
	\red
	& (\outputp{x}{y} | @{y})\substn{\quotep{(\prefix{x}{y}{(@{y} | \outputp{x}{y})) | P}}}{y} & \nonumber\\
	=
	& \outputp{x}{\quotep{(\prefix{x}{y}{(\outputp{x}{y} | @{y})) | P}}}
	  | {(\prefix{x}{y}{(\outputp{x}{y} | @{y})) | P}} & \nonumber\\
	\red
	& \ldots & \nonumber\\
	\red^*
	& P | P | \ldots & \nonumber
\end{eqnarray}

Of course, this encoding, as an implementation, runs away, unfolding
$\bangp{P}$ eagerly. A lazier and more implementable replication
operator, restricted to input-guarded processes, may be obtained as follows.

\begin{eqnarray}
\bangp{\prefix{u}{v}{P}} 
	:= 
	\binpar{\lift{x}{\prefix{u}{v}{(\binpar{D(x)}{P})}}}{D(x)} \nonumber
\end{eqnarray}

\begin{remark}
  Note that the lazier definition still does not deal with summation
  or mixed summation (i.e. sums over input and output). The reader is
  invited to construct definitions of replication that deal with these
  features. 

  Further, the definitions are parameterized in a name, $x$. Can you,
  gentle reader, make a definition that eliminates this parameter and
  guarantees no accidental interaction between the replication
  machinery and the process being replicated -- i.e. no accidental
  sharing of names used by the process to get its work done and the
  name(s) used by the replication to effect copying. This latter
  revision of the definition of replication is crucial to obtaining
  the expected identity $!!P \sim !P$.
\end{remark}

\begin{remark}\label{rem:paradoxical_combinator}
  The reader familiar with the lambda calculus will have noticed the
  similarity between $D$ and the paradoxical combinator.

  [Ed. note: the existence of this seems to suggest we have to be more
  restrictive on the set of processes and names we admit if we are to
  support no-cloning.]
\end{remark}

\subsubsection{Bisimulation}

The computational dynamics gives rise to another kind of equivalence,
the equivalence of computational behavior. As previously mentioned
this is typically captured \emph{via} some form of bisimulation.

% The notion we use in this paper is weak barbed bisimulation
% \cite{milner91polyadicpi}.

The notion we use in this paper is derived from weak barbed
bisimulation \cite{milner91polyadicpi}. 

\begin{definition}
An \emph{observation relation}, $\downarrow_{\mathcal N}$, over a set
of names, $\mathcal N$, is the smallest relation satisfying the rules
below.

\infrule[Out-barb]{y \in {\mathcal N}, \; x \nameeq y}
		  {\outputp{x}{v} \downarrow_{\mathcal N} x}
\infrule[Par-barb]{\mbox{$P\downarrow_{\mathcal N} x$ or $Q\downarrow_{\mathcal N} x$}}
		  {\binpar{P}{Q} \downarrow_{\mathcal N} x}

We write $P \Downarrow_{\mathcal N} x$ if there is $Q$ such that 
$P \wred Q$ and $Q \downarrow_{\mathcal N} x$.
\end{definition}

\begin{definition}
%\label{def.bbisim}
An  ${\mathcal N}$-\emph{barbed bisimulation} over a set of names, ${\mathcal N}$, is a symmetric binary relation 
${\mathcal S}_{\mathcal N}$ between agents such that $P\rel{S}_{\mathcal N}Q$ implies:
\begin{enumerate}
\item If $P \red P'$ then $Q \wred Q'$ and $P'\rel{S}_{\mathcal N} Q'$.
\item If $P\downarrow_{\mathcal N} x$, then $Q\Downarrow_{\mathcal N} x$.
\end{enumerate}
$P$ is ${\mathcal N}$-barbed bisimilar to $Q$, written
$P \wbbisim_{\mathcal N} Q$, if $P \rel{S}_{\mathcal N} Q$ for some ${\mathcal N}$-barbed bisimulation ${\mathcal S}_{\mathcal N}$.
\end{definition}

$\mathcal{R} \subseteq \pi \times \pi$

$P \mathcal{R} Q => \forall P'. P \red P' \Rightarrow \exists Q'. Q \red Q', P' \mathcal{R} Q'$

$P \vdash x \Rightarrow Q \vdash x$

\begin{mathpar}
  \inferrule*[lab=Out-barb]{x \nameeq y}{{y}!\langle{Q}\rangle \vdash x}
  \and
  \inferrule*[lab=Par-barb]{\mbox{$P\vdash x$ or $Q\vdash x$}}{\binpar{P}{Q} \vdash x}
\end{mathpar}

\subsubsection{Contexts}

One of the principle advantages of computational calculi like the
$\pi$-calculus is a well-defined notion of context,
contextual-equivalence and a correlation between
contextual-equivalence and notions of bisimulation. The notion of
context allows the decomposition of a process into (sub-)process and
its syntactic environment, its context. Thus, a context may be
thought of as a process with a ``hole'' (written $\Box$) in it. The
application of a context $M$ to a process $P$, written $M[P]$, is
tantamount to filling the hole in $M$ with $P$. In this paper we do
not need the full weight of this theory, but do make use of the notion
of context in the proof the main theorem. 

\begin{mathpar}
  \inferrule* [lab=summation] {} {{M_{M},M_{N}} \bc \Box \;|\; x.M_{A} \;|\; M_{M}+M_{N}}
  \and
  \inferrule* [lab=agent] {} {{M_{A}} \bc (\vec{x})M_{P} \;| \; \clift{P_0,\ldots,M_{P},\ldots,P_N}}
  \and \\
  \inferrule* [lab=process] {} {{M_{P}} \bc M_{N} \;| \;P|M_{P} }
\end{mathpar} 

\begin{mathpar}
  \inferrule* [lab=sychronization] {} {M_{N} \bc \Box \;|\; x?M_{F} \;|\; x!M_{C}}
  \and
  \inferrule* [lab=abstraction] {} {{M_{F}} \bc (x)M_{P} }
  \and
  \inferrule* [lab=concretion] {} {{M_{C}} \bc \langle M_{P} \rangle }
  \and \\
  \inferrule* [lab=process] {} {{M_{P}} \bc M_{N} \;| \;P|M_{P} }
\end{mathpar}

\begin{definition}[contextual application] Given a context $M$, and
  process $P$, we define the \emph{contextual application}, $M[P] :=
  M\{P/\Box\}$. That is, the contextual application of M to P is the
  substitution of $P$ for $\Box$ in $M$.
\end{definition}

$\meaningof{-} : L \to \mathcal{P}(\pi)$

\begin{mathpar}
  \inferrule* [lab=collection] {} {\meaningof{true} = \pi, \and \meaningof{~E} = \pi \setminus \meaningof{E}, \and \meaningof{E_{1} \& E_{2}} = \meaningof{E_{1}} \cap \meaningof{E_{2}}}
\end{mathpar}

\begin{mathpar}
  \inferrule* [lab=structure] {} {\meaningof{0} = \{ P \in \pi | P \equiv 0 \}, \and \\ \meaningof{E_1 | E_2} = \{ P \in \pi | P \equiv P_{1} | P_{2}, P_{1} \in \meaningof{E_{1}}, P_{2} \in \meaningof{E_2}\} }
\end{mathpar}

\begin{mathpar}
 \inferrule* [lab=behavior] {} {\meaningof{\langle a?b \rangle E} = \{ P \in \pi | P \equiv Q | u?(y)P', \\ \and \\\\ \and \\ \;\;\; u \in \meaningof{a}, \forall z.P'\{z/y\} \in \meaningof{E\{z/b\}}\}, \and \\ \meaningof{a!E} = \{ P \in \pi | P \equiv Q | x!\langle P' \rangle, x \in \meaningof{a} P' \in \meaningof{E}\} }
\end{mathpar}

\begin{mathpar}
 \inferrule* [lab=nominal] {} {\meaningof{\quotep{E}} = \{ \quotep{P} \in \quotep{\pi} | P \in \meaningof{E} \}, \and \meaningof{\quotep{P}} = \{ \quotep{Q} \in \quotep{\pi} | P \equiv Q \} \and \\ \meaningof{@\quotep{E}} = \{ P \in \pi | P \equiv @x, x \in \meaningof{E} \}}
\end{mathpar}

\begin{eqnarray*}
  \\
  \meaningof{-} : TS \to ST
\end{eqnarray*}

\begin{eqnarray*}
  \\
  L : TS \to ST
\end{eqnarray*}

\begin{eqnarray*}
  \\
  P \models E \iff P \in \meaningof{E}
\end{eqnarray*}

\begin{eqnarray*}
  P \approx_{L} Q \iff \forall E \in L. P \models E \iff Q \models E
\end{eqnarray*}

\begin{eqnarray*}
  P \approx_{K} Q
\end{eqnarray*}

\begin{eqnarray*}
  P \approx Q
\end{eqnarray*}

$\approx_{K} = \approx = \approx_{L}$

\subsubsection{Contextual duality}

Note that contexts extend the quotation operation to a family of
operations from processes to names. Given a context, $M$, we can
define a \emph{nominal context}, $\quotep{M}$ by $\quotep{M}[P] :=
\quotep{M[P]}$. To foreshadow what is to come we observe that these
operations enjoy a duality with processes very much like the duality
between vectors and maps from vectors to scalars.

Further, because the calculus is essentially higher-order, we have a
correspondence between contexts and processes. More specifically,
given a name $x$ and a context $M$ we can construct $M^{*}_{x}$ such
that 

\begin{mathpar}
  M^{*}_{x} | \lift{x}{P} \red M[P]
\end{mathpar}

namely,

\begin{mathpar}
  M^{*}_{x} := x?(u).M[\dropn{u}]
\end{mathpar}

The dependence of $M^{*}_{x}$ on a name makes it an abstraction, 

\begin{mathpar}
  M^{*} := (x)x?(u).M[\dropn{u}]
\end{mathpar}

\subsection{Additional notation}

It will sometimes be convenient to denote the process a name
quotes. We already have the notation $x = \quotep{P}$, but it will be
convenient to introduce an alternate notation, $\procn{x}$, when we
want to emphasize the connection to the use of the name. Note that, by
virtue of name equivalence, $\quotep{\procn{x}} \nameeq x$; so, the
notation is consistent with previous definitions.

Further, because names have structure it is possible to effect
substitutions on the basis of that structure. This means we need to
upgrade our notation for substitutions, which we accomplish by
adapting comprehension notation. Thus,

\begin{mathpar}
  P\{ y / x : x \in S \}
\end{mathpar}

is interpreted to mean the process derived from P by replacing (in a
capture-avoiding manner) each occurrence of $x$ in $S$ by $y$. For example,

\begin{mathpar}
  P\{ \quotep{\procn{x}|\procn{x}} / x : x \in \freenames{P} \}
\end{mathpar}

will replace each (occurrence) of a free name $x$ in $P$ by
$\quotep{\procn{x}|\procn{x}}$.

Also, we will avail ourselves of the notation $x^{L}$ and $x^{R}$ to
denote injections of a name into disjoint copies of the name
space. There are numerous ways to accomplish this. One example can be
found in \cite{MeredithR05}. This notation overloads to vectors of
names: $\vec{x}^{\pi} := (x_{i}^{\pi} \; : \; 0 \leq i < |\vec{x}| )$ where $\pi \in \{L,R\}$.

We also use $P^{\Box} := P|\Box$.

In \cite{MeredithR05} an interpretation of the new operator is
given. It turns out that there are several possible interpretations
all enjoying the requisite algebraic properties of the operator (see
\cite{milner91polyadicpi}). We will therefore make liberal use of
$(\nu\; \vec{x})P$.

% subsection the_syntax_and_semantics_of_the_notation_system (end)   

\input{qm2pi.qmops} 

\input{qm2pi.sterngerlach} 

\input{qm2pi.metric} 

% section concurrent_process_calculi (end)

%\input{qm2pi.proofsketch}

% section proof sketch (end)

%\input{qm2pi.slviaknots} 

% section spatial logic via knots (end)

\input{qm2pi.conclusion}

% section conclusion (end)

%\input{qm2pi.dtcodes} 

% section wiring algorithm (end)

\input{qm2pi.ack} 

% section acknowledgments (end)

\newpage


\bibliographystyle{plain}   
\bibliography{../../biblios/main.bib}

\input{qm2pi.rhodetails}

\end{document}



% section front matter (end)

\section{Introduction}\label{sec:introduction} % (fold)
In this draft of the material i am going to have to dispense with the
usual writing conventions adopted in papers on these topics. i'm going
to have adopt whatever tone i need at the time i'm writing up the
calculations. Sometimes this may be very conversational; others it may
be the barest mathematical grunts; others still it may be that i have
lifted text from one of my other papers because the exposition of some
point was better said there. i hope that my readers are not unduly put
out by this decision. i'm not doing this to flout convention or be
rebellious. i find these calculations very technically challenging. To
keep everything going technically, something has to give; i have to
let go of some cognitive burden. So, the academic writing style --
with all of its trade-offs in terms of facilitating technical
communication -- is what i'm letting go of. Perhaps subsequent drafts
can be tightened and polished, but for now, i'm going to speak as if
we were sitting together in a coffee shop with a laptop, wifi and a
pad of paper and a pencil.

So, here's what i have to say. We -- you and i, comfortably ensconced
in our coffee shop and well-equipped with our tools -- can realize and
carry out the calculations of quantum mechanics over a very different
formal theory of dynamics, a formal theory of dynamics that
corresponds to a theory of concurrent computation with
\emph{reflection}. It has the advantage that the underlying theory is
already `quantized', but supports analogues all of the continuuous
operations. Strikingly, this underlying theory has recently been
connected with a notion of metric that we can show, by calculating
together, coincides with the metric induced by the inner product.

There are a lot of reasons why you might be interested in seeing
calculations of this form. Here's why i'm interested. For the past
several centuries there has been no competitor to the ``Newtonian''
account of dynamics. As a result the predominant share of accounts of
dynamical systems and situations have had to be formulated in terms of
the Newtonian machinery. i view this as an intellectually dangerous
position to occupy. Everything, despite it's intrinsic shape, turns
into a nail to be hit with this hammer. Recently, however, the theory
of computation has matured to the point where we have candidates for
theories of dynamics that offer very different perspective on
reasoning about dynamical systems and situations. Testing these
candidates against very successful accounts of dynamical situations,
like quantum mechanics, is going to give us some sense of how mature
they are and some measure of the quality of these accounts of
dynamics.

\subsection{Summary of contributions and outline of paper}

So, we're going to develop an interpretation of the operations of
quantum mechanics normally interpreted by Hilbert spaces and
operators. We're going to do this over a theory of computation. Note
that this is very different than the usual quantum computation program
which develops notions of computation over quantum mechanics. Rather,
we are developing a story that aligns with Wheeler's slogan: It from
Bit. To do this we will first provide an account of the theory of
computation at play here. Then we will dive into a calculation-driven
interpretation of the operations of quantum mechanics.

The reason we take this approach is that -- until very recently --
there hasn't been an axiomatic account of quantum mechanics. As a
result there has been no sharp delineation of the mathematical theory
supporting interpretation of the physical theory and the physical
theory, itself. So, ambient features of the maths are free to be
exploited (or supressed) without a real accounting of their physical
relevance. There is no sharp statement ``here's the physical theory''
qua \emph{theory} and ``here's the mathematical interpretation''
enabling a judgment of how faithful the interpretation is -- apart
from experimental observation. When there is an axiomatic account we
can judge how well a given mathematical formalism supports an
interpretation of the axioms, independent of
experimentation. Likewise, we can judge how well we have captured our
physical evidence and experience with our axiomatics, independent of
any specific mathematical implementation, with accidental detail that
may or may not have physical significance. 

In lieu of a fully fleshed out and vetted axiomatic account of quantum
mechanics, interpreting the operational notions in service of modeling
physical systems will have to suffice. In other words, we are not in
the business of providing a model of Hilbert spaces and operators. We
are in the business of providing a model of quantum mechanics because
we are motivated by testing our notions of dynamics against physical
theory; and, the predictive calculations of the physical theory must
serve as the best formulation -- shy of a fully fleshed out axiomatic
account -- of the physical theory itself (as they have for scientific
theories since time immemorial). Put another way, despite a
whole-hearted commitment to an It-from-Bit ontology, we are firmly
aligned with the shut-up-and-calculate camp as the best way to obtain
results either from the physical perspective or as a quality assurance
measure of our fledgling theory of dynamics.

In detail, we present a reflective process calculus. Then we develop
intuitive correspondences between the notions available in this
calculus and the usual physical notions supporting quantum mechanical
calculations. Thus, 

\begin{table}[htp]
  \center{
    \fbox{
      \begin{tabular}{c|c}
        quantum mechanics & process calculus \\
        \hline
        scalar & name \\
        state vector & process \\
        dual & contextual duals \\
        matrix & formal sums of process-context-dual pairs \\
        orthogonality & process annihilation \\
        inner product & execution-formula + quoting
      \end{tabular}
    }
  }
  \caption{QM - process calculi correspondences}
\end{table}

Then we tighten up these intuitions to operational definitions. We
employ the Dirac notation as the best proxy we can find for an
abstract syntax of the quantum mechanical notions. The definitions we
develop put us in contact with equational constraints coming from the
theory that we demonstrate the definitions and calculations satisfy.

This puts us in a position to shut up and calculate for the
Stern-Gerlach experimental set up, showing how these predictive
calculations become calculations on processes in our theory of a
reflective process calculus.

Penultimately, we demonstrate that the notion of metric coming from
the inner product coincides with the notion of metric available from
the theory of bisimulation. This demonstration gives us the right to
think of space as arising from behavior. Finally, we consider where we
might go from the new vantage point we have obtained.

% section introduction (end) 
 
% section introduction (end)

% \documentclass[12pt]{llncs}
%\documentclass{jktr}

\usepackage[pdftex]{hyperref}                   
\usepackage {listings}
\usepackage {mathpartir}
\usepackage{bcprules}
%\usepackage{listings}
                       
\usepackage{graphicx} 
%\usepackage[margins=2.5cm,nohead,nofoot]{geometry}
%\usepackage{geometry}
\usepackage{amsfonts}
\usepackage{amstext}
\usepackage{latexsym}
\usepackage{amssymb}
\usepackage{color}


%\include{myPreamble}
\include{qm2pi.local} 

%\ifpdf
%\usepackage[pdftex]{graphicx}
%\else
%\usepackage{graphicx}
%\fi

 % \ifpdf
%  \usepackage{pdfsync}
%  \if


%\title{Brief Article}
%\author{David F. Snyder}
%\author{L.G. Meredith}

%\address{Dept. of Math., Texas State University--San Marcos, San Marcos, TX 78666}
       
\pagestyle{empty}


\begin{document}

\lstset{language=[Objective]Caml,frame=shadowbox}

\input{qm2pi.front}

% section front matter (end)

\input{qm2pi.intro} 
 
% section introduction (end)

% \input{qm2pi.knotations} 

% section notation (end)

\input{qm2pi.process.calculi} 

% section concurrent_process_calculi_and_spatial_logics_ (end)
    
%\input{qm2pi.knots2pi} 

%\input{qm2pi.trefoil} 

%\input{qm2pi.mainthm} 

% subsection basic_interpretation (end)

%\input{qm2pi.rho.presentation} 
\subsection{The syntax and semantics of the notation system}\label{sub:the_syntax_and_semantics_of_the_notation_system} % (fold)

We now summarize a technical presentation of the calculus that
embodies our theory of dynamics. The typical presentation of such a
calculus follows the style of giving generators and relations on
them. The grammar, below, describing term constructors, freely
generates the set of processes, $\Proc$. This set is then quotiented
by a relation known as structural congruence and it is over this set
that the notion of dynamics is expressed. This presentation is
essentially that of \cite{MeredithR05} with the addition of
polyadicity and summation. For readability we have relegated some of
the technical subtleties to an appendix.

\subsubsection{Process grammar}\label{subsub:process_grammar}

\begin{mathpar}
  \inferrule* [lab=synchronization] {} {{M} \bc \pzero \;|\; x?F \;|\; x!C }
  \and
  \inferrule* [lab=abstraction] {} {{F} \bc (x)P}
  \and
  \inferrule* [lab=concretion] {} {{C} \bc \langle Q \rangle}
  \and
  \inferrule* [lab=process] {} {{P,Q} \bc M \;| \;P|Q \;|\; @{x}}
  \and
  \inferrule* [lab=name] {} {{x} \bc \quotep{P}}
\end{mathpar} 

Note that $\vec{x}$ (resp. $\vec{P}$) denotes a vector of names
(resp. processes) of length $|\vec{x}|$ (resp. $|\vec{P}|$). We adopt
the following useful abbreviations.

\begin{mathpar}
   x?(\vec{y}).P := x.(\vec{y})P \and  x\clift{\vec{P}} := x.\clift{\vec{P}}
   \and x!(y) := \lift{x}{\dropn{y}}
   \and \Pi_{i=0}^{n-1}P_i := P_0 | \ldots | P_{n-1}
\end{mathpar}

\subsubsection{Structural congruence}

\paragraph{Free and bound names and alpha-equivalence.} At the
core of structural equivalence is alpha-equivalence which identifies
process that are the same up to a change of variable. Formally, we
recognize the distinction between free and bound names. The free names
of a process, $\freenames{P}$, may be calculated recursively as
follows:

\begin{mathpar}
\freenames{\pzero} := \emptyset
  \and \\
  \freenames{x?(y).P} := \{ x \} \cup (\freenames{P} \setminus \{ y \})
  \and 
  \freenames{x!\langle P \rangle} := \{ x \} \cup \{ P \} 
  \and \\
  \freenames{P|Q} := \freenames{P} \cup \freenames{Q}
  \and \\
  \freenames{@{x}} := \{ x \}
\end{mathpar}

$\pi$
$\quotep{\pi}$

$\freenames{-} : \pi \to \mathcal{P}(\quotep{\pi})$

\begin{eqnarray*}
  \freenames{\pzero} & := & \emptyset \\
  \freenames{x?(y).P} & := & \{ x \} \cup (\freenames{P} \setminus \{ y \}) \\
  \freenames{x!\langle P \rangle} & := & \{ x \} \cup \{ P \} \\
  \freenames{P|Q} & := & \freenames{P} \cup \freenames{Q} \\
  \freenames{\dropn{x}} & := & \{ x \}
\end{eqnarray*}

The bound names of a process, $\boundnames{P}$, are those names occurring in $P$
that are not free. For example, in $x?(y).0$, the name $x$ is free, while $y$ is bound.

\begin{mathpar}
  \inferrule* [lab=monoidal-laws] {} { P|Q \equiv Q|P \and P|0 \equiv P \and P|(Q|R) \equiv (P|Q)|R }
\end{mathpar}

\begin{mathpar}
  \inferrule* [lab=alpha-equivalence] {} { (x)P \equiv (y)P\{y/x\} \and y \not\in \freenames{P} }
\end{mathpar}

\begin{definition}
Then two processes, $P,Q$, are alpha-equivalent if $P = Q\{\vec{y}/\vec{x}\}$ for
some $\vec{x} \in \boundnames{Q},\vec{y} \in \boundnames{P}$, where $Q\{\vec{y}/\vec{x}\}$
denotes the capture-avoiding substitution of $\vec{y}$ for $\vec{x}$ in $Q$.
\end{definition}

\begin{definition}
  The {\em structural congruence} \cite{SangiorgiWalker} , $\equiv$,
  between processes is the least congruence containing
  alpha-equivalence, satisfying the abelian monoid laws
  (associativity, commutativity and $\pzero$ as identity) for parallel
  composition $|$ and for summation $+$.
\end{definition}

\subsection{Name equivalence}

We take name equivalence, written $\nameeq$, to be the smallest
equivalence relation generated by the following rules.

\begin{mathpar}
\inferrule*[lab=Quote-drop]
{ }
{ \quotep{@{x}} \nameeq x }

\inferrule*[lab=Struct-equiv]
{ P \scong Q }
{ \quotep{P} \nameeq \quotep{Q} }
\end{mathpar}

The astute reader will have noticed that the mutual recursion of names
and processes imposes a mutual recursion on alpha-equivalence and
structural equivalence via name-equivalence. Fortunately, all of this
works out pleasantly and we may calculate in the natural way, free of
concern. The reader interested in the details is referred to the
appendix \ref{appendix:rho_details}.

\subsection{Substitution}

We use $\Proc$ for the set of processes, $\QProc$ for the set of
names, and $\id{\{}\vec{y} / \vec{x} \id{\}}$ to denote partial maps,
$s : \QProc \rightarrow \QProc$. A map, $s$ lifts, uniquely, to a map
on process terms, $\widehat{s} : \Proc \rightarrow \Proc$ by the
following equations.

\begin{mathpar}
  (0) \psubstp{Q}{P} := 0 \\
  (R \juxtap S) \psubstp{Q}{P}
  :=    
  (R)\psubstp{Q}{P} \juxtap (S) \psubstp{Q}{P} \\
  (x?(y).R) \psubstp{Q}{P}    
  :=    
  (x)\substp{Q}{P} (z)\concat( (R \psubstn{z}{y}) \psubstp{Q}{P} ) \\
  (\lift{x}{R}) \psubstp{Q}{P}  
  :=
  \lift{(x)\substp{Q}{P}}{ R \psubstp{Q}{P} } \\
%   (\dropn{x})  \psubstp{Q}{P}       
%   := 
%   \left\{ 
%     \begin{array}{ccc} 
%       \dropn{\quotep{Q}} & & x \nameeq \quotep{P} \\
%       \dropn{x} & & otherwise \\
%     \end{array}
%   \right. 
  (\dropn{x})  \psubstp{Q}{P}       
  := 
  \left\{ 
    \begin{array}{ccc} 
      Q & & x \nameeq \quotep{P} \\
      \dropn{x} & & otherwise \\
    \end{array}
  \right.
\end{mathpar}
 

where

\begin{eqnarray}
  (x)\id{\{} \lpquote Q \rpquote / \lpquote P \rpquote \id{\}}            = 
  \left\{ 
    \begin{array}{ccc}
      \lpquote Q \rpquote & & x \nameeq \lpquote P \rpquote \\
      x & & otherwise \\
    \end{array}
  \right. \nonumber
\end{eqnarray}

and $z$ is chosen distinct from $\quotep{P}$, $\quotep{Q}$, the free
names in $Q$, and all the names in $R$. Our $\alpha$-equivalence will
be built in the standard way from this substitution.

\begin{remark}\label{rem:no_self_referential_names}
  One consequence of these definitions is that $\forall P. \quotep{P}
  \not\in \freenames{P}$.
\end{remark}

\subsection{ Dynamic quote: an example }

Anticipating something of what's to come, consider applying the
substitution, $\widehat{\id{\{}u / z \id{\}}}$, to the following pair
of processes, $\lift{w}{y!(z)}$ and $w[ \lpquote y!(z) \rpquote ]$.

\begin{eqnarray}
	\lift{w}{y!(z)}\widehat{\id{\{}u / z \id{\}}}
		& = &
		\lift{w}{y!(u)} \nonumber\\
	w[ \lpquote y!(z) \rpquote ] \widehat{ \id{\{}u / z \id{\}} }
		& = &
		w[ \lpquote y!(z) \rpquote ] \nonumber
\end{eqnarray}

Because the body of the process between quotes is impervious to
substitution, we get radically different answers. In fact, by
examining the first process in an input context,
e.g. $x?(z).\lift{w}{y!(z)}$, we see that the process under the lift
operator may be shaped by prefixed inputs binding a name inside it. In
this sense, the lift operator will be seen as a way to dynamically
construct processes before reifying them as names.

Finally equipped with these standard features we can present the
dynamics of the calculus.

\subsubsection{Operational semantics} 

Finally, we introduce the computational dynamics. What marks these
algebras as distinct from other more traditionally studied algebraic
structures, e.g. vector spaces or polynomial rings, is the manner in
which dynamics is captured. In traditional structures, dynamics is typically
expressed through morphisms between such structures, as in linear maps
between vector spaces or morphisms between rings. In algebras
associated with the semantics of computation, the dynamics is
expressed as part of the algebraic structure itself, through a
reduction reduction relation typically denoted by $\red$. Below, we
give a recursive presentation of this relation for the calculus used
in the encoding.

$\red \subseteq \pi \times \pi$
$\red : \pi \to \mathcal{P}(\pi)$

\begin{mathpar}
  \inferrule* [lab=Comm] { \textsf{match}( x_{src}, x_{trgt} ) } { x_{trgt}?(y)P \; | \; x_{src}!\langle {Q} \rangle \red P\{\quotep{Q}/y}\} }
  \and \\
  \inferrule* [lab=Par] {{P} \red {P}'} {{{P} | {Q}} \red {{P}' | {Q}}}
  \and
  \inferrule* [lab=Equiv]{{{P} \scong {P}'} \andalso {{P}' \red {Q}'} \andalso {{Q}' \scong {Q}}}{{P} \red {Q}}
\end{mathpar}

\begin{eqnarray*}
  match_{\equiv} (\quotep{P},\quotep{Q}) & := & P \equiv Q \\
  match_{\dagger}(\quotep{P},\quotep{Q}) & := & \forall R. P|Q \red^{*} R => R \red^{*} 0 \\
  match_{K}(\quotep{P},\quotep{Q}) & := & K \mbox{ for some context } K
\end{eqnarray*}

$u?(x)P | u!\langle Q \rangle \red P\{\quotep{Q}/x\}$

%We write $\wred$ for $\red^*$, and $P\red$ if $\exists Q $ such that $ P \red Q$.
We write $P\red$ if $\exists Q $ such that $ P \red Q$ and $P\not\red$, otherwise.

\section{Replication}

As mentioned before, it is known that replication (and hence
recursion) can be implemented in a higher-order process algebra
\cite{SangiorgiWalker}. As our first example of calculation with the
machinery thus far presented we give the construction explicitly in
the {\rhoc}.

\begin{eqnarray}
	D_{x} & := & \prefix{x}{y}{(\binpar{\outputp{x}{y}}{@{y}})} \nonumber\\
	\bangp_{x}{P} & := & \binpar{{x}!\langle{\binpar{D_{x}}{P}}\rangle}{D_{x}} \nonumber
\end{eqnarray}

\begin{eqnarray}
	\bangp_{x}{P} & & \nonumber\\
	=
	& {x}!\langle{(\prefix{x}{y}{(\outputp{x}{y} | @{y})) | P}}\rangle 
	      | \prefix{x}{y}{(\outputp{x}{y} | @{y})} & \nonumber\\
	\red
	& (\outputp{x}{y} | @{y})\substn{\quotep{(\prefix{x}{y}{(@{y} | \outputp{x}{y})) | P}}}{y} & \nonumber\\
	=
	& \outputp{x}{\quotep{(\prefix{x}{y}{(\outputp{x}{y} | @{y})) | P}}}
	  | {(\prefix{x}{y}{(\outputp{x}{y} | @{y})) | P}} & \nonumber\\
	\red
	& \ldots & \nonumber\\
	\red^*
	& P | P | \ldots & \nonumber
\end{eqnarray}

Of course, this encoding, as an implementation, runs away, unfolding
$\bangp{P}$ eagerly. A lazier and more implementable replication
operator, restricted to input-guarded processes, may be obtained as follows.

\begin{eqnarray}
\bangp{\prefix{u}{v}{P}} 
	:= 
	\binpar{\lift{x}{\prefix{u}{v}{(\binpar{D(x)}{P})}}}{D(x)} \nonumber
\end{eqnarray}

\begin{remark}
  Note that the lazier definition still does not deal with summation
  or mixed summation (i.e. sums over input and output). The reader is
  invited to construct definitions of replication that deal with these
  features. 

  Further, the definitions are parameterized in a name, $x$. Can you,
  gentle reader, make a definition that eliminates this parameter and
  guarantees no accidental interaction between the replication
  machinery and the process being replicated -- i.e. no accidental
  sharing of names used by the process to get its work done and the
  name(s) used by the replication to effect copying. This latter
  revision of the definition of replication is crucial to obtaining
  the expected identity $!!P \sim !P$.
\end{remark}

\begin{remark}\label{rem:paradoxical_combinator}
  The reader familiar with the lambda calculus will have noticed the
  similarity between $D$ and the paradoxical combinator.

  [Ed. note: the existence of this seems to suggest we have to be more
  restrictive on the set of processes and names we admit if we are to
  support no-cloning.]
\end{remark}

\subsubsection{Bisimulation}

The computational dynamics gives rise to another kind of equivalence,
the equivalence of computational behavior. As previously mentioned
this is typically captured \emph{via} some form of bisimulation.

% The notion we use in this paper is weak barbed bisimulation
% \cite{milner91polyadicpi}.

The notion we use in this paper is derived from weak barbed
bisimulation \cite{milner91polyadicpi}. 

\begin{definition}
An \emph{observation relation}, $\downarrow_{\mathcal N}$, over a set
of names, $\mathcal N$, is the smallest relation satisfying the rules
below.

\infrule[Out-barb]{y \in {\mathcal N}, \; x \nameeq y}
		  {\outputp{x}{v} \downarrow_{\mathcal N} x}
\infrule[Par-barb]{\mbox{$P\downarrow_{\mathcal N} x$ or $Q\downarrow_{\mathcal N} x$}}
		  {\binpar{P}{Q} \downarrow_{\mathcal N} x}

We write $P \Downarrow_{\mathcal N} x$ if there is $Q$ such that 
$P \wred Q$ and $Q \downarrow_{\mathcal N} x$.
\end{definition}

\begin{definition}
%\label{def.bbisim}
An  ${\mathcal N}$-\emph{barbed bisimulation} over a set of names, ${\mathcal N}$, is a symmetric binary relation 
${\mathcal S}_{\mathcal N}$ between agents such that $P\rel{S}_{\mathcal N}Q$ implies:
\begin{enumerate}
\item If $P \red P'$ then $Q \wred Q'$ and $P'\rel{S}_{\mathcal N} Q'$.
\item If $P\downarrow_{\mathcal N} x$, then $Q\Downarrow_{\mathcal N} x$.
\end{enumerate}
$P$ is ${\mathcal N}$-barbed bisimilar to $Q$, written
$P \wbbisim_{\mathcal N} Q$, if $P \rel{S}_{\mathcal N} Q$ for some ${\mathcal N}$-barbed bisimulation ${\mathcal S}_{\mathcal N}$.
\end{definition}

$\mathcal{R} \subseteq \pi \times \pi$

$P \mathcal{R} Q => \forall P'. P \red P' \Rightarrow \exists Q'. Q \red Q', P' \mathcal{R} Q'$

$P \vdash x \Rightarrow Q \vdash x$

\begin{mathpar}
  \inferrule*[lab=Out-barb]{x \nameeq y}{{y}!\langle{Q}\rangle \vdash x}
  \and
  \inferrule*[lab=Par-barb]{\mbox{$P\vdash x$ or $Q\vdash x$}}{\binpar{P}{Q} \vdash x}
\end{mathpar}

\subsubsection{Contexts}

One of the principle advantages of computational calculi like the
$\pi$-calculus is a well-defined notion of context,
contextual-equivalence and a correlation between
contextual-equivalence and notions of bisimulation. The notion of
context allows the decomposition of a process into (sub-)process and
its syntactic environment, its context. Thus, a context may be
thought of as a process with a ``hole'' (written $\Box$) in it. The
application of a context $M$ to a process $P$, written $M[P]$, is
tantamount to filling the hole in $M$ with $P$. In this paper we do
not need the full weight of this theory, but do make use of the notion
of context in the proof the main theorem. 

\begin{mathpar}
  \inferrule* [lab=summation] {} {{M_{M},M_{N}} \bc \Box \;|\; x.M_{A} \;|\; M_{M}+M_{N}}
  \and
  \inferrule* [lab=agent] {} {{M_{A}} \bc (\vec{x})M_{P} \;| \; \clift{P_0,\ldots,M_{P},\ldots,P_N}}
  \and \\
  \inferrule* [lab=process] {} {{M_{P}} \bc M_{N} \;| \;P|M_{P} }
\end{mathpar} 

\begin{mathpar}
  \inferrule* [lab=sychronization] {} {M_{N} \bc \Box \;|\; x?M_{F} \;|\; x!M_{C}}
  \and
  \inferrule* [lab=abstraction] {} {{M_{F}} \bc (x)M_{P} }
  \and
  \inferrule* [lab=concretion] {} {{M_{C}} \bc \langle M_{P} \rangle }
  \and \\
  \inferrule* [lab=process] {} {{M_{P}} \bc M_{N} \;| \;P|M_{P} }
\end{mathpar}

\begin{definition}[contextual application] Given a context $M$, and
  process $P$, we define the \emph{contextual application}, $M[P] :=
  M\{P/\Box\}$. That is, the contextual application of M to P is the
  substitution of $P$ for $\Box$ in $M$.
\end{definition}

$\meaningof{-} : L \to \mathcal{P}(\pi)$

\begin{mathpar}
  \inferrule* [lab=collection] {} {\meaningof{true} = \pi, \and \meaningof{~E} = \pi \setminus \meaningof{E}, \and \meaningof{E_{1} \& E_{2}} = \meaningof{E_{1}} \cap \meaningof{E_{2}}}
\end{mathpar}

\begin{mathpar}
  \inferrule* [lab=structure] {} {\meaningof{0} = \{ P \in \pi | P \equiv 0 \}, \and \\ \meaningof{E_1 | E_2} = \{ P \in \pi | P \equiv P_{1} | P_{2}, P_{1} \in \meaningof{E_{1}}, P_{2} \in \meaningof{E_2}\} }
\end{mathpar}

\begin{mathpar}
 \inferrule* [lab=behavior] {} {\meaningof{\langle a?b \rangle E} = \{ P \in \pi | P \equiv Q | u?(y)P', \\ \and \\\\ \and \\ \;\;\; u \in \meaningof{a}, \forall z.P'\{z/y\} \in \meaningof{E\{z/b\}}\}, \and \\ \meaningof{a!E} = \{ P \in \pi | P \equiv Q | x!\langle P' \rangle, x \in \meaningof{a} P' \in \meaningof{E}\} }
\end{mathpar}

\begin{mathpar}
 \inferrule* [lab=nominal] {} {\meaningof{\quotep{E}} = \{ \quotep{P} \in \quotep{\pi} | P \in \meaningof{E} \}, \and \meaningof{\quotep{P}} = \{ \quotep{Q} \in \quotep{\pi} | P \equiv Q \} \and \\ \meaningof{@\quotep{E}} = \{ P \in \pi | P \equiv @x, x \in \meaningof{E} \}}
\end{mathpar}

\begin{eqnarray*}
  \\
  \meaningof{-} : TS \to ST
\end{eqnarray*}

\begin{eqnarray*}
  \\
  L : TS \to ST
\end{eqnarray*}

\begin{eqnarray*}
  \\
  P \models E \iff P \in \meaningof{E}
\end{eqnarray*}

\begin{eqnarray*}
  P \approx_{L} Q \iff \forall E \in L. P \models E \iff Q \models E
\end{eqnarray*}

\begin{eqnarray*}
  P \approx_{K} Q
\end{eqnarray*}

\begin{eqnarray*}
  P \approx Q
\end{eqnarray*}

$\approx_{K} = \approx = \approx_{L}$

\subsubsection{Contextual duality}

Note that contexts extend the quotation operation to a family of
operations from processes to names. Given a context, $M$, we can
define a \emph{nominal context}, $\quotep{M}$ by $\quotep{M}[P] :=
\quotep{M[P]}$. To foreshadow what is to come we observe that these
operations enjoy a duality with processes very much like the duality
between vectors and maps from vectors to scalars.

Further, because the calculus is essentially higher-order, we have a
correspondence between contexts and processes. More specifically,
given a name $x$ and a context $M$ we can construct $M^{*}_{x}$ such
that 

\begin{mathpar}
  M^{*}_{x} | \lift{x}{P} \red M[P]
\end{mathpar}

namely,

\begin{mathpar}
  M^{*}_{x} := x?(u).M[\dropn{u}]
\end{mathpar}

The dependence of $M^{*}_{x}$ on a name makes it an abstraction, 

\begin{mathpar}
  M^{*} := (x)x?(u).M[\dropn{u}]
\end{mathpar}

\subsection{Additional notation}

It will sometimes be convenient to denote the process a name
quotes. We already have the notation $x = \quotep{P}$, but it will be
convenient to introduce an alternate notation, $\procn{x}$, when we
want to emphasize the connection to the use of the name. Note that, by
virtue of name equivalence, $\quotep{\procn{x}} \nameeq x$; so, the
notation is consistent with previous definitions.

Further, because names have structure it is possible to effect
substitutions on the basis of that structure. This means we need to
upgrade our notation for substitutions, which we accomplish by
adapting comprehension notation. Thus,

\begin{mathpar}
  P\{ y / x : x \in S \}
\end{mathpar}

is interpreted to mean the process derived from P by replacing (in a
capture-avoiding manner) each occurrence of $x$ in $S$ by $y$. For example,

\begin{mathpar}
  P\{ \quotep{\procn{x}|\procn{x}} / x : x \in \freenames{P} \}
\end{mathpar}

will replace each (occurrence) of a free name $x$ in $P$ by
$\quotep{\procn{x}|\procn{x}}$.

Also, we will avail ourselves of the notation $x^{L}$ and $x^{R}$ to
denote injections of a name into disjoint copies of the name
space. There are numerous ways to accomplish this. One example can be
found in \cite{MeredithR05}. This notation overloads to vectors of
names: $\vec{x}^{\pi} := (x_{i}^{\pi} \; : \; 0 \leq i < |\vec{x}| )$ where $\pi \in \{L,R\}$.

We also use $P^{\Box} := P|\Box$.

In \cite{MeredithR05} an interpretation of the new operator is
given. It turns out that there are several possible interpretations
all enjoying the requisite algebraic properties of the operator (see
\cite{milner91polyadicpi}). We will therefore make liberal use of
$(\nu\; \vec{x})P$.

% subsection the_syntax_and_semantics_of_the_notation_system (end)   

\input{qm2pi.qmops} 

\input{qm2pi.sterngerlach} 

\input{qm2pi.metric} 

% section concurrent_process_calculi (end)

%\input{qm2pi.proofsketch}

% section proof sketch (end)

%\input{qm2pi.slviaknots} 

% section spatial logic via knots (end)

\input{qm2pi.conclusion}

% section conclusion (end)

%\input{qm2pi.dtcodes} 

% section wiring algorithm (end)

\input{qm2pi.ack} 

% section acknowledgments (end)

\newpage


\bibliographystyle{plain}   
\bibliography{../../biblios/main.bib}

\input{qm2pi.rhodetails}

\end{document}

 

% section notation (end)

\input{qm2pi.process.calculi} 

% section concurrent_process_calculi_and_spatial_logics_ (end)
    
%\documentclass[12pt]{llncs}
%\documentclass{jktr}

\usepackage[pdftex]{hyperref}                   
\usepackage {listings}
\usepackage {mathpartir}
\usepackage{bcprules}
%\usepackage{listings}
                       
\usepackage{graphicx} 
%\usepackage[margins=2.5cm,nohead,nofoot]{geometry}
%\usepackage{geometry}
\usepackage{amsfonts}
\usepackage{amstext}
\usepackage{latexsym}
\usepackage{amssymb}
\usepackage{color}


%\include{myPreamble}
\include{qm2pi.local} 

%\ifpdf
%\usepackage[pdftex]{graphicx}
%\else
%\usepackage{graphicx}
%\fi

 % \ifpdf
%  \usepackage{pdfsync}
%  \if


%\title{Brief Article}
%\author{David F. Snyder}
%\author{L.G. Meredith}

%\address{Dept. of Math., Texas State University--San Marcos, San Marcos, TX 78666}
       
\pagestyle{empty}


\begin{document}

\lstset{language=[Objective]Caml,frame=shadowbox}

\input{qm2pi.front}

% section front matter (end)

\input{qm2pi.intro} 
 
% section introduction (end)

% \input{qm2pi.knotations} 

% section notation (end)

\input{qm2pi.process.calculi} 

% section concurrent_process_calculi_and_spatial_logics_ (end)
    
%\input{qm2pi.knots2pi} 

%\input{qm2pi.trefoil} 

%\input{qm2pi.mainthm} 

% subsection basic_interpretation (end)

%\input{qm2pi.rho.presentation} 
\subsection{The syntax and semantics of the notation system}\label{sub:the_syntax_and_semantics_of_the_notation_system} % (fold)

We now summarize a technical presentation of the calculus that
embodies our theory of dynamics. The typical presentation of such a
calculus follows the style of giving generators and relations on
them. The grammar, below, describing term constructors, freely
generates the set of processes, $\Proc$. This set is then quotiented
by a relation known as structural congruence and it is over this set
that the notion of dynamics is expressed. This presentation is
essentially that of \cite{MeredithR05} with the addition of
polyadicity and summation. For readability we have relegated some of
the technical subtleties to an appendix.

\subsubsection{Process grammar}\label{subsub:process_grammar}

\begin{mathpar}
  \inferrule* [lab=synchronization] {} {{M} \bc \pzero \;|\; x?F \;|\; x!C }
  \and
  \inferrule* [lab=abstraction] {} {{F} \bc (x)P}
  \and
  \inferrule* [lab=concretion] {} {{C} \bc \langle Q \rangle}
  \and
  \inferrule* [lab=process] {} {{P,Q} \bc M \;| \;P|Q \;|\; @{x}}
  \and
  \inferrule* [lab=name] {} {{x} \bc \quotep{P}}
\end{mathpar} 

Note that $\vec{x}$ (resp. $\vec{P}$) denotes a vector of names
(resp. processes) of length $|\vec{x}|$ (resp. $|\vec{P}|$). We adopt
the following useful abbreviations.

\begin{mathpar}
   x?(\vec{y}).P := x.(\vec{y})P \and  x\clift{\vec{P}} := x.\clift{\vec{P}}
   \and x!(y) := \lift{x}{\dropn{y}}
   \and \Pi_{i=0}^{n-1}P_i := P_0 | \ldots | P_{n-1}
\end{mathpar}

\subsubsection{Structural congruence}

\paragraph{Free and bound names and alpha-equivalence.} At the
core of structural equivalence is alpha-equivalence which identifies
process that are the same up to a change of variable. Formally, we
recognize the distinction between free and bound names. The free names
of a process, $\freenames{P}$, may be calculated recursively as
follows:

\begin{mathpar}
\freenames{\pzero} := \emptyset
  \and \\
  \freenames{x?(y).P} := \{ x \} \cup (\freenames{P} \setminus \{ y \})
  \and 
  \freenames{x!\langle P \rangle} := \{ x \} \cup \{ P \} 
  \and \\
  \freenames{P|Q} := \freenames{P} \cup \freenames{Q}
  \and \\
  \freenames{@{x}} := \{ x \}
\end{mathpar}

$\pi$
$\quotep{\pi}$

$\freenames{-} : \pi \to \mathcal{P}(\quotep{\pi})$

\begin{eqnarray*}
  \freenames{\pzero} & := & \emptyset \\
  \freenames{x?(y).P} & := & \{ x \} \cup (\freenames{P} \setminus \{ y \}) \\
  \freenames{x!\langle P \rangle} & := & \{ x \} \cup \{ P \} \\
  \freenames{P|Q} & := & \freenames{P} \cup \freenames{Q} \\
  \freenames{\dropn{x}} & := & \{ x \}
\end{eqnarray*}

The bound names of a process, $\boundnames{P}$, are those names occurring in $P$
that are not free. For example, in $x?(y).0$, the name $x$ is free, while $y$ is bound.

\begin{mathpar}
  \inferrule* [lab=monoidal-laws] {} { P|Q \equiv Q|P \and P|0 \equiv P \and P|(Q|R) \equiv (P|Q)|R }
\end{mathpar}

\begin{mathpar}
  \inferrule* [lab=alpha-equivalence] {} { (x)P \equiv (y)P\{y/x\} \and y \not\in \freenames{P} }
\end{mathpar}

\begin{definition}
Then two processes, $P,Q$, are alpha-equivalent if $P = Q\{\vec{y}/\vec{x}\}$ for
some $\vec{x} \in \boundnames{Q},\vec{y} \in \boundnames{P}$, where $Q\{\vec{y}/\vec{x}\}$
denotes the capture-avoiding substitution of $\vec{y}$ for $\vec{x}$ in $Q$.
\end{definition}

\begin{definition}
  The {\em structural congruence} \cite{SangiorgiWalker} , $\equiv$,
  between processes is the least congruence containing
  alpha-equivalence, satisfying the abelian monoid laws
  (associativity, commutativity and $\pzero$ as identity) for parallel
  composition $|$ and for summation $+$.
\end{definition}

\subsection{Name equivalence}

We take name equivalence, written $\nameeq$, to be the smallest
equivalence relation generated by the following rules.

\begin{mathpar}
\inferrule*[lab=Quote-drop]
{ }
{ \quotep{@{x}} \nameeq x }

\inferrule*[lab=Struct-equiv]
{ P \scong Q }
{ \quotep{P} \nameeq \quotep{Q} }
\end{mathpar}

The astute reader will have noticed that the mutual recursion of names
and processes imposes a mutual recursion on alpha-equivalence and
structural equivalence via name-equivalence. Fortunately, all of this
works out pleasantly and we may calculate in the natural way, free of
concern. The reader interested in the details is referred to the
appendix \ref{appendix:rho_details}.

\subsection{Substitution}

We use $\Proc$ for the set of processes, $\QProc$ for the set of
names, and $\id{\{}\vec{y} / \vec{x} \id{\}}$ to denote partial maps,
$s : \QProc \rightarrow \QProc$. A map, $s$ lifts, uniquely, to a map
on process terms, $\widehat{s} : \Proc \rightarrow \Proc$ by the
following equations.

\begin{mathpar}
  (0) \psubstp{Q}{P} := 0 \\
  (R \juxtap S) \psubstp{Q}{P}
  :=    
  (R)\psubstp{Q}{P} \juxtap (S) \psubstp{Q}{P} \\
  (x?(y).R) \psubstp{Q}{P}    
  :=    
  (x)\substp{Q}{P} (z)\concat( (R \psubstn{z}{y}) \psubstp{Q}{P} ) \\
  (\lift{x}{R}) \psubstp{Q}{P}  
  :=
  \lift{(x)\substp{Q}{P}}{ R \psubstp{Q}{P} } \\
%   (\dropn{x})  \psubstp{Q}{P}       
%   := 
%   \left\{ 
%     \begin{array}{ccc} 
%       \dropn{\quotep{Q}} & & x \nameeq \quotep{P} \\
%       \dropn{x} & & otherwise \\
%     \end{array}
%   \right. 
  (\dropn{x})  \psubstp{Q}{P}       
  := 
  \left\{ 
    \begin{array}{ccc} 
      Q & & x \nameeq \quotep{P} \\
      \dropn{x} & & otherwise \\
    \end{array}
  \right.
\end{mathpar}
 

where

\begin{eqnarray}
  (x)\id{\{} \lpquote Q \rpquote / \lpquote P \rpquote \id{\}}            = 
  \left\{ 
    \begin{array}{ccc}
      \lpquote Q \rpquote & & x \nameeq \lpquote P \rpquote \\
      x & & otherwise \\
    \end{array}
  \right. \nonumber
\end{eqnarray}

and $z$ is chosen distinct from $\quotep{P}$, $\quotep{Q}$, the free
names in $Q$, and all the names in $R$. Our $\alpha$-equivalence will
be built in the standard way from this substitution.

\begin{remark}\label{rem:no_self_referential_names}
  One consequence of these definitions is that $\forall P. \quotep{P}
  \not\in \freenames{P}$.
\end{remark}

\subsection{ Dynamic quote: an example }

Anticipating something of what's to come, consider applying the
substitution, $\widehat{\id{\{}u / z \id{\}}}$, to the following pair
of processes, $\lift{w}{y!(z)}$ and $w[ \lpquote y!(z) \rpquote ]$.

\begin{eqnarray}
	\lift{w}{y!(z)}\widehat{\id{\{}u / z \id{\}}}
		& = &
		\lift{w}{y!(u)} \nonumber\\
	w[ \lpquote y!(z) \rpquote ] \widehat{ \id{\{}u / z \id{\}} }
		& = &
		w[ \lpquote y!(z) \rpquote ] \nonumber
\end{eqnarray}

Because the body of the process between quotes is impervious to
substitution, we get radically different answers. In fact, by
examining the first process in an input context,
e.g. $x?(z).\lift{w}{y!(z)}$, we see that the process under the lift
operator may be shaped by prefixed inputs binding a name inside it. In
this sense, the lift operator will be seen as a way to dynamically
construct processes before reifying them as names.

Finally equipped with these standard features we can present the
dynamics of the calculus.

\subsubsection{Operational semantics} 

Finally, we introduce the computational dynamics. What marks these
algebras as distinct from other more traditionally studied algebraic
structures, e.g. vector spaces or polynomial rings, is the manner in
which dynamics is captured. In traditional structures, dynamics is typically
expressed through morphisms between such structures, as in linear maps
between vector spaces or morphisms between rings. In algebras
associated with the semantics of computation, the dynamics is
expressed as part of the algebraic structure itself, through a
reduction reduction relation typically denoted by $\red$. Below, we
give a recursive presentation of this relation for the calculus used
in the encoding.

$\red \subseteq \pi \times \pi$
$\red : \pi \to \mathcal{P}(\pi)$

\begin{mathpar}
  \inferrule* [lab=Comm] { \textsf{match}( x_{src}, x_{trgt} ) } { x_{trgt}?(y)P \; | \; x_{src}!\langle {Q} \rangle \red P\{\quotep{Q}/y}\} }
  \and \\
  \inferrule* [lab=Par] {{P} \red {P}'} {{{P} | {Q}} \red {{P}' | {Q}}}
  \and
  \inferrule* [lab=Equiv]{{{P} \scong {P}'} \andalso {{P}' \red {Q}'} \andalso {{Q}' \scong {Q}}}{{P} \red {Q}}
\end{mathpar}

\begin{eqnarray*}
  match_{\equiv} (\quotep{P},\quotep{Q}) & := & P \equiv Q \\
  match_{\dagger}(\quotep{P},\quotep{Q}) & := & \forall R. P|Q \red^{*} R => R \red^{*} 0 \\
  match_{K}(\quotep{P},\quotep{Q}) & := & K \mbox{ for some context } K
\end{eqnarray*}

$u?(x)P | u!\langle Q \rangle \red P\{\quotep{Q}/x\}$

%We write $\wred$ for $\red^*$, and $P\red$ if $\exists Q $ such that $ P \red Q$.
We write $P\red$ if $\exists Q $ such that $ P \red Q$ and $P\not\red$, otherwise.

\section{Replication}

As mentioned before, it is known that replication (and hence
recursion) can be implemented in a higher-order process algebra
\cite{SangiorgiWalker}. As our first example of calculation with the
machinery thus far presented we give the construction explicitly in
the {\rhoc}.

\begin{eqnarray}
	D_{x} & := & \prefix{x}{y}{(\binpar{\outputp{x}{y}}{@{y}})} \nonumber\\
	\bangp_{x}{P} & := & \binpar{{x}!\langle{\binpar{D_{x}}{P}}\rangle}{D_{x}} \nonumber
\end{eqnarray}

\begin{eqnarray}
	\bangp_{x}{P} & & \nonumber\\
	=
	& {x}!\langle{(\prefix{x}{y}{(\outputp{x}{y} | @{y})) | P}}\rangle 
	      | \prefix{x}{y}{(\outputp{x}{y} | @{y})} & \nonumber\\
	\red
	& (\outputp{x}{y} | @{y})\substn{\quotep{(\prefix{x}{y}{(@{y} | \outputp{x}{y})) | P}}}{y} & \nonumber\\
	=
	& \outputp{x}{\quotep{(\prefix{x}{y}{(\outputp{x}{y} | @{y})) | P}}}
	  | {(\prefix{x}{y}{(\outputp{x}{y} | @{y})) | P}} & \nonumber\\
	\red
	& \ldots & \nonumber\\
	\red^*
	& P | P | \ldots & \nonumber
\end{eqnarray}

Of course, this encoding, as an implementation, runs away, unfolding
$\bangp{P}$ eagerly. A lazier and more implementable replication
operator, restricted to input-guarded processes, may be obtained as follows.

\begin{eqnarray}
\bangp{\prefix{u}{v}{P}} 
	:= 
	\binpar{\lift{x}{\prefix{u}{v}{(\binpar{D(x)}{P})}}}{D(x)} \nonumber
\end{eqnarray}

\begin{remark}
  Note that the lazier definition still does not deal with summation
  or mixed summation (i.e. sums over input and output). The reader is
  invited to construct definitions of replication that deal with these
  features. 

  Further, the definitions are parameterized in a name, $x$. Can you,
  gentle reader, make a definition that eliminates this parameter and
  guarantees no accidental interaction between the replication
  machinery and the process being replicated -- i.e. no accidental
  sharing of names used by the process to get its work done and the
  name(s) used by the replication to effect copying. This latter
  revision of the definition of replication is crucial to obtaining
  the expected identity $!!P \sim !P$.
\end{remark}

\begin{remark}\label{rem:paradoxical_combinator}
  The reader familiar with the lambda calculus will have noticed the
  similarity between $D$ and the paradoxical combinator.

  [Ed. note: the existence of this seems to suggest we have to be more
  restrictive on the set of processes and names we admit if we are to
  support no-cloning.]
\end{remark}

\subsubsection{Bisimulation}

The computational dynamics gives rise to another kind of equivalence,
the equivalence of computational behavior. As previously mentioned
this is typically captured \emph{via} some form of bisimulation.

% The notion we use in this paper is weak barbed bisimulation
% \cite{milner91polyadicpi}.

The notion we use in this paper is derived from weak barbed
bisimulation \cite{milner91polyadicpi}. 

\begin{definition}
An \emph{observation relation}, $\downarrow_{\mathcal N}$, over a set
of names, $\mathcal N$, is the smallest relation satisfying the rules
below.

\infrule[Out-barb]{y \in {\mathcal N}, \; x \nameeq y}
		  {\outputp{x}{v} \downarrow_{\mathcal N} x}
\infrule[Par-barb]{\mbox{$P\downarrow_{\mathcal N} x$ or $Q\downarrow_{\mathcal N} x$}}
		  {\binpar{P}{Q} \downarrow_{\mathcal N} x}

We write $P \Downarrow_{\mathcal N} x$ if there is $Q$ such that 
$P \wred Q$ and $Q \downarrow_{\mathcal N} x$.
\end{definition}

\begin{definition}
%\label{def.bbisim}
An  ${\mathcal N}$-\emph{barbed bisimulation} over a set of names, ${\mathcal N}$, is a symmetric binary relation 
${\mathcal S}_{\mathcal N}$ between agents such that $P\rel{S}_{\mathcal N}Q$ implies:
\begin{enumerate}
\item If $P \red P'$ then $Q \wred Q'$ and $P'\rel{S}_{\mathcal N} Q'$.
\item If $P\downarrow_{\mathcal N} x$, then $Q\Downarrow_{\mathcal N} x$.
\end{enumerate}
$P$ is ${\mathcal N}$-barbed bisimilar to $Q$, written
$P \wbbisim_{\mathcal N} Q$, if $P \rel{S}_{\mathcal N} Q$ for some ${\mathcal N}$-barbed bisimulation ${\mathcal S}_{\mathcal N}$.
\end{definition}

$\mathcal{R} \subseteq \pi \times \pi$

$P \mathcal{R} Q => \forall P'. P \red P' \Rightarrow \exists Q'. Q \red Q', P' \mathcal{R} Q'$

$P \vdash x \Rightarrow Q \vdash x$

\begin{mathpar}
  \inferrule*[lab=Out-barb]{x \nameeq y}{{y}!\langle{Q}\rangle \vdash x}
  \and
  \inferrule*[lab=Par-barb]{\mbox{$P\vdash x$ or $Q\vdash x$}}{\binpar{P}{Q} \vdash x}
\end{mathpar}

\subsubsection{Contexts}

One of the principle advantages of computational calculi like the
$\pi$-calculus is a well-defined notion of context,
contextual-equivalence and a correlation between
contextual-equivalence and notions of bisimulation. The notion of
context allows the decomposition of a process into (sub-)process and
its syntactic environment, its context. Thus, a context may be
thought of as a process with a ``hole'' (written $\Box$) in it. The
application of a context $M$ to a process $P$, written $M[P]$, is
tantamount to filling the hole in $M$ with $P$. In this paper we do
not need the full weight of this theory, but do make use of the notion
of context in the proof the main theorem. 

\begin{mathpar}
  \inferrule* [lab=summation] {} {{M_{M},M_{N}} \bc \Box \;|\; x.M_{A} \;|\; M_{M}+M_{N}}
  \and
  \inferrule* [lab=agent] {} {{M_{A}} \bc (\vec{x})M_{P} \;| \; \clift{P_0,\ldots,M_{P},\ldots,P_N}}
  \and \\
  \inferrule* [lab=process] {} {{M_{P}} \bc M_{N} \;| \;P|M_{P} }
\end{mathpar} 

\begin{mathpar}
  \inferrule* [lab=sychronization] {} {M_{N} \bc \Box \;|\; x?M_{F} \;|\; x!M_{C}}
  \and
  \inferrule* [lab=abstraction] {} {{M_{F}} \bc (x)M_{P} }
  \and
  \inferrule* [lab=concretion] {} {{M_{C}} \bc \langle M_{P} \rangle }
  \and \\
  \inferrule* [lab=process] {} {{M_{P}} \bc M_{N} \;| \;P|M_{P} }
\end{mathpar}

\begin{definition}[contextual application] Given a context $M$, and
  process $P$, we define the \emph{contextual application}, $M[P] :=
  M\{P/\Box\}$. That is, the contextual application of M to P is the
  substitution of $P$ for $\Box$ in $M$.
\end{definition}

$\meaningof{-} : L \to \mathcal{P}(\pi)$

\begin{mathpar}
  \inferrule* [lab=collection] {} {\meaningof{true} = \pi, \and \meaningof{~E} = \pi \setminus \meaningof{E}, \and \meaningof{E_{1} \& E_{2}} = \meaningof{E_{1}} \cap \meaningof{E_{2}}}
\end{mathpar}

\begin{mathpar}
  \inferrule* [lab=structure] {} {\meaningof{0} = \{ P \in \pi | P \equiv 0 \}, \and \\ \meaningof{E_1 | E_2} = \{ P \in \pi | P \equiv P_{1} | P_{2}, P_{1} \in \meaningof{E_{1}}, P_{2} \in \meaningof{E_2}\} }
\end{mathpar}

\begin{mathpar}
 \inferrule* [lab=behavior] {} {\meaningof{\langle a?b \rangle E} = \{ P \in \pi | P \equiv Q | u?(y)P', \\ \and \\\\ \and \\ \;\;\; u \in \meaningof{a}, \forall z.P'\{z/y\} \in \meaningof{E\{z/b\}}\}, \and \\ \meaningof{a!E} = \{ P \in \pi | P \equiv Q | x!\langle P' \rangle, x \in \meaningof{a} P' \in \meaningof{E}\} }
\end{mathpar}

\begin{mathpar}
 \inferrule* [lab=nominal] {} {\meaningof{\quotep{E}} = \{ \quotep{P} \in \quotep{\pi} | P \in \meaningof{E} \}, \and \meaningof{\quotep{P}} = \{ \quotep{Q} \in \quotep{\pi} | P \equiv Q \} \and \\ \meaningof{@\quotep{E}} = \{ P \in \pi | P \equiv @x, x \in \meaningof{E} \}}
\end{mathpar}

\begin{eqnarray*}
  \\
  \meaningof{-} : TS \to ST
\end{eqnarray*}

\begin{eqnarray*}
  \\
  L : TS \to ST
\end{eqnarray*}

\begin{eqnarray*}
  \\
  P \models E \iff P \in \meaningof{E}
\end{eqnarray*}

\begin{eqnarray*}
  P \approx_{L} Q \iff \forall E \in L. P \models E \iff Q \models E
\end{eqnarray*}

\begin{eqnarray*}
  P \approx_{K} Q
\end{eqnarray*}

\begin{eqnarray*}
  P \approx Q
\end{eqnarray*}

$\approx_{K} = \approx = \approx_{L}$

\subsubsection{Contextual duality}

Note that contexts extend the quotation operation to a family of
operations from processes to names. Given a context, $M$, we can
define a \emph{nominal context}, $\quotep{M}$ by $\quotep{M}[P] :=
\quotep{M[P]}$. To foreshadow what is to come we observe that these
operations enjoy a duality with processes very much like the duality
between vectors and maps from vectors to scalars.

Further, because the calculus is essentially higher-order, we have a
correspondence between contexts and processes. More specifically,
given a name $x$ and a context $M$ we can construct $M^{*}_{x}$ such
that 

\begin{mathpar}
  M^{*}_{x} | \lift{x}{P} \red M[P]
\end{mathpar}

namely,

\begin{mathpar}
  M^{*}_{x} := x?(u).M[\dropn{u}]
\end{mathpar}

The dependence of $M^{*}_{x}$ on a name makes it an abstraction, 

\begin{mathpar}
  M^{*} := (x)x?(u).M[\dropn{u}]
\end{mathpar}

\subsection{Additional notation}

It will sometimes be convenient to denote the process a name
quotes. We already have the notation $x = \quotep{P}$, but it will be
convenient to introduce an alternate notation, $\procn{x}$, when we
want to emphasize the connection to the use of the name. Note that, by
virtue of name equivalence, $\quotep{\procn{x}} \nameeq x$; so, the
notation is consistent with previous definitions.

Further, because names have structure it is possible to effect
substitutions on the basis of that structure. This means we need to
upgrade our notation for substitutions, which we accomplish by
adapting comprehension notation. Thus,

\begin{mathpar}
  P\{ y / x : x \in S \}
\end{mathpar}

is interpreted to mean the process derived from P by replacing (in a
capture-avoiding manner) each occurrence of $x$ in $S$ by $y$. For example,

\begin{mathpar}
  P\{ \quotep{\procn{x}|\procn{x}} / x : x \in \freenames{P} \}
\end{mathpar}

will replace each (occurrence) of a free name $x$ in $P$ by
$\quotep{\procn{x}|\procn{x}}$.

Also, we will avail ourselves of the notation $x^{L}$ and $x^{R}$ to
denote injections of a name into disjoint copies of the name
space. There are numerous ways to accomplish this. One example can be
found in \cite{MeredithR05}. This notation overloads to vectors of
names: $\vec{x}^{\pi} := (x_{i}^{\pi} \; : \; 0 \leq i < |\vec{x}| )$ where $\pi \in \{L,R\}$.

We also use $P^{\Box} := P|\Box$.

In \cite{MeredithR05} an interpretation of the new operator is
given. It turns out that there are several possible interpretations
all enjoying the requisite algebraic properties of the operator (see
\cite{milner91polyadicpi}). We will therefore make liberal use of
$(\nu\; \vec{x})P$.

% subsection the_syntax_and_semantics_of_the_notation_system (end)   

\input{qm2pi.qmops} 

\input{qm2pi.sterngerlach} 

\input{qm2pi.metric} 

% section concurrent_process_calculi (end)

%\input{qm2pi.proofsketch}

% section proof sketch (end)

%\input{qm2pi.slviaknots} 

% section spatial logic via knots (end)

\input{qm2pi.conclusion}

% section conclusion (end)

%\input{qm2pi.dtcodes} 

% section wiring algorithm (end)

\input{qm2pi.ack} 

% section acknowledgments (end)

\newpage


\bibliographystyle{plain}   
\bibliography{../../biblios/main.bib}

\input{qm2pi.rhodetails}

\end{document}

 

%\documentclass[12pt]{llncs}
%\documentclass{jktr}

\usepackage[pdftex]{hyperref}                   
\usepackage {listings}
\usepackage {mathpartir}
\usepackage{bcprules}
%\usepackage{listings}
                       
\usepackage{graphicx} 
%\usepackage[margins=2.5cm,nohead,nofoot]{geometry}
%\usepackage{geometry}
\usepackage{amsfonts}
\usepackage{amstext}
\usepackage{latexsym}
\usepackage{amssymb}
\usepackage{color}


%\include{myPreamble}
\include{qm2pi.local} 

%\ifpdf
%\usepackage[pdftex]{graphicx}
%\else
%\usepackage{graphicx}
%\fi

 % \ifpdf
%  \usepackage{pdfsync}
%  \if


%\title{Brief Article}
%\author{David F. Snyder}
%\author{L.G. Meredith}

%\address{Dept. of Math., Texas State University--San Marcos, San Marcos, TX 78666}
       
\pagestyle{empty}


\begin{document}

\lstset{language=[Objective]Caml,frame=shadowbox}

\input{qm2pi.front}

% section front matter (end)

\input{qm2pi.intro} 
 
% section introduction (end)

% \input{qm2pi.knotations} 

% section notation (end)

\input{qm2pi.process.calculi} 

% section concurrent_process_calculi_and_spatial_logics_ (end)
    
%\input{qm2pi.knots2pi} 

%\input{qm2pi.trefoil} 

%\input{qm2pi.mainthm} 

% subsection basic_interpretation (end)

%\input{qm2pi.rho.presentation} 
\subsection{The syntax and semantics of the notation system}\label{sub:the_syntax_and_semantics_of_the_notation_system} % (fold)

We now summarize a technical presentation of the calculus that
embodies our theory of dynamics. The typical presentation of such a
calculus follows the style of giving generators and relations on
them. The grammar, below, describing term constructors, freely
generates the set of processes, $\Proc$. This set is then quotiented
by a relation known as structural congruence and it is over this set
that the notion of dynamics is expressed. This presentation is
essentially that of \cite{MeredithR05} with the addition of
polyadicity and summation. For readability we have relegated some of
the technical subtleties to an appendix.

\subsubsection{Process grammar}\label{subsub:process_grammar}

\begin{mathpar}
  \inferrule* [lab=synchronization] {} {{M} \bc \pzero \;|\; x?F \;|\; x!C }
  \and
  \inferrule* [lab=abstraction] {} {{F} \bc (x)P}
  \and
  \inferrule* [lab=concretion] {} {{C} \bc \langle Q \rangle}
  \and
  \inferrule* [lab=process] {} {{P,Q} \bc M \;| \;P|Q \;|\; @{x}}
  \and
  \inferrule* [lab=name] {} {{x} \bc \quotep{P}}
\end{mathpar} 

Note that $\vec{x}$ (resp. $\vec{P}$) denotes a vector of names
(resp. processes) of length $|\vec{x}|$ (resp. $|\vec{P}|$). We adopt
the following useful abbreviations.

\begin{mathpar}
   x?(\vec{y}).P := x.(\vec{y})P \and  x\clift{\vec{P}} := x.\clift{\vec{P}}
   \and x!(y) := \lift{x}{\dropn{y}}
   \and \Pi_{i=0}^{n-1}P_i := P_0 | \ldots | P_{n-1}
\end{mathpar}

\subsubsection{Structural congruence}

\paragraph{Free and bound names and alpha-equivalence.} At the
core of structural equivalence is alpha-equivalence which identifies
process that are the same up to a change of variable. Formally, we
recognize the distinction between free and bound names. The free names
of a process, $\freenames{P}$, may be calculated recursively as
follows:

\begin{mathpar}
\freenames{\pzero} := \emptyset
  \and \\
  \freenames{x?(y).P} := \{ x \} \cup (\freenames{P} \setminus \{ y \})
  \and 
  \freenames{x!\langle P \rangle} := \{ x \} \cup \{ P \} 
  \and \\
  \freenames{P|Q} := \freenames{P} \cup \freenames{Q}
  \and \\
  \freenames{@{x}} := \{ x \}
\end{mathpar}

$\pi$
$\quotep{\pi}$

$\freenames{-} : \pi \to \mathcal{P}(\quotep{\pi})$

\begin{eqnarray*}
  \freenames{\pzero} & := & \emptyset \\
  \freenames{x?(y).P} & := & \{ x \} \cup (\freenames{P} \setminus \{ y \}) \\
  \freenames{x!\langle P \rangle} & := & \{ x \} \cup \{ P \} \\
  \freenames{P|Q} & := & \freenames{P} \cup \freenames{Q} \\
  \freenames{\dropn{x}} & := & \{ x \}
\end{eqnarray*}

The bound names of a process, $\boundnames{P}$, are those names occurring in $P$
that are not free. For example, in $x?(y).0$, the name $x$ is free, while $y$ is bound.

\begin{mathpar}
  \inferrule* [lab=monoidal-laws] {} { P|Q \equiv Q|P \and P|0 \equiv P \and P|(Q|R) \equiv (P|Q)|R }
\end{mathpar}

\begin{mathpar}
  \inferrule* [lab=alpha-equivalence] {} { (x)P \equiv (y)P\{y/x\} \and y \not\in \freenames{P} }
\end{mathpar}

\begin{definition}
Then two processes, $P,Q$, are alpha-equivalent if $P = Q\{\vec{y}/\vec{x}\}$ for
some $\vec{x} \in \boundnames{Q},\vec{y} \in \boundnames{P}$, where $Q\{\vec{y}/\vec{x}\}$
denotes the capture-avoiding substitution of $\vec{y}$ for $\vec{x}$ in $Q$.
\end{definition}

\begin{definition}
  The {\em structural congruence} \cite{SangiorgiWalker} , $\equiv$,
  between processes is the least congruence containing
  alpha-equivalence, satisfying the abelian monoid laws
  (associativity, commutativity and $\pzero$ as identity) for parallel
  composition $|$ and for summation $+$.
\end{definition}

\subsection{Name equivalence}

We take name equivalence, written $\nameeq$, to be the smallest
equivalence relation generated by the following rules.

\begin{mathpar}
\inferrule*[lab=Quote-drop]
{ }
{ \quotep{@{x}} \nameeq x }

\inferrule*[lab=Struct-equiv]
{ P \scong Q }
{ \quotep{P} \nameeq \quotep{Q} }
\end{mathpar}

The astute reader will have noticed that the mutual recursion of names
and processes imposes a mutual recursion on alpha-equivalence and
structural equivalence via name-equivalence. Fortunately, all of this
works out pleasantly and we may calculate in the natural way, free of
concern. The reader interested in the details is referred to the
appendix \ref{appendix:rho_details}.

\subsection{Substitution}

We use $\Proc$ for the set of processes, $\QProc$ for the set of
names, and $\id{\{}\vec{y} / \vec{x} \id{\}}$ to denote partial maps,
$s : \QProc \rightarrow \QProc$. A map, $s$ lifts, uniquely, to a map
on process terms, $\widehat{s} : \Proc \rightarrow \Proc$ by the
following equations.

\begin{mathpar}
  (0) \psubstp{Q}{P} := 0 \\
  (R \juxtap S) \psubstp{Q}{P}
  :=    
  (R)\psubstp{Q}{P} \juxtap (S) \psubstp{Q}{P} \\
  (x?(y).R) \psubstp{Q}{P}    
  :=    
  (x)\substp{Q}{P} (z)\concat( (R \psubstn{z}{y}) \psubstp{Q}{P} ) \\
  (\lift{x}{R}) \psubstp{Q}{P}  
  :=
  \lift{(x)\substp{Q}{P}}{ R \psubstp{Q}{P} } \\
%   (\dropn{x})  \psubstp{Q}{P}       
%   := 
%   \left\{ 
%     \begin{array}{ccc} 
%       \dropn{\quotep{Q}} & & x \nameeq \quotep{P} \\
%       \dropn{x} & & otherwise \\
%     \end{array}
%   \right. 
  (\dropn{x})  \psubstp{Q}{P}       
  := 
  \left\{ 
    \begin{array}{ccc} 
      Q & & x \nameeq \quotep{P} \\
      \dropn{x} & & otherwise \\
    \end{array}
  \right.
\end{mathpar}
 

where

\begin{eqnarray}
  (x)\id{\{} \lpquote Q \rpquote / \lpquote P \rpquote \id{\}}            = 
  \left\{ 
    \begin{array}{ccc}
      \lpquote Q \rpquote & & x \nameeq \lpquote P \rpquote \\
      x & & otherwise \\
    \end{array}
  \right. \nonumber
\end{eqnarray}

and $z$ is chosen distinct from $\quotep{P}$, $\quotep{Q}$, the free
names in $Q$, and all the names in $R$. Our $\alpha$-equivalence will
be built in the standard way from this substitution.

\begin{remark}\label{rem:no_self_referential_names}
  One consequence of these definitions is that $\forall P. \quotep{P}
  \not\in \freenames{P}$.
\end{remark}

\subsection{ Dynamic quote: an example }

Anticipating something of what's to come, consider applying the
substitution, $\widehat{\id{\{}u / z \id{\}}}$, to the following pair
of processes, $\lift{w}{y!(z)}$ and $w[ \lpquote y!(z) \rpquote ]$.

\begin{eqnarray}
	\lift{w}{y!(z)}\widehat{\id{\{}u / z \id{\}}}
		& = &
		\lift{w}{y!(u)} \nonumber\\
	w[ \lpquote y!(z) \rpquote ] \widehat{ \id{\{}u / z \id{\}} }
		& = &
		w[ \lpquote y!(z) \rpquote ] \nonumber
\end{eqnarray}

Because the body of the process between quotes is impervious to
substitution, we get radically different answers. In fact, by
examining the first process in an input context,
e.g. $x?(z).\lift{w}{y!(z)}$, we see that the process under the lift
operator may be shaped by prefixed inputs binding a name inside it. In
this sense, the lift operator will be seen as a way to dynamically
construct processes before reifying them as names.

Finally equipped with these standard features we can present the
dynamics of the calculus.

\subsubsection{Operational semantics} 

Finally, we introduce the computational dynamics. What marks these
algebras as distinct from other more traditionally studied algebraic
structures, e.g. vector spaces or polynomial rings, is the manner in
which dynamics is captured. In traditional structures, dynamics is typically
expressed through morphisms between such structures, as in linear maps
between vector spaces or morphisms between rings. In algebras
associated with the semantics of computation, the dynamics is
expressed as part of the algebraic structure itself, through a
reduction reduction relation typically denoted by $\red$. Below, we
give a recursive presentation of this relation for the calculus used
in the encoding.

$\red \subseteq \pi \times \pi$
$\red : \pi \to \mathcal{P}(\pi)$

\begin{mathpar}
  \inferrule* [lab=Comm] { \textsf{match}( x_{src}, x_{trgt} ) } { x_{trgt}?(y)P \; | \; x_{src}!\langle {Q} \rangle \red P\{\quotep{Q}/y}\} }
  \and \\
  \inferrule* [lab=Par] {{P} \red {P}'} {{{P} | {Q}} \red {{P}' | {Q}}}
  \and
  \inferrule* [lab=Equiv]{{{P} \scong {P}'} \andalso {{P}' \red {Q}'} \andalso {{Q}' \scong {Q}}}{{P} \red {Q}}
\end{mathpar}

\begin{eqnarray*}
  match_{\equiv} (\quotep{P},\quotep{Q}) & := & P \equiv Q \\
  match_{\dagger}(\quotep{P},\quotep{Q}) & := & \forall R. P|Q \red^{*} R => R \red^{*} 0 \\
  match_{K}(\quotep{P},\quotep{Q}) & := & K \mbox{ for some context } K
\end{eqnarray*}

$u?(x)P | u!\langle Q \rangle \red P\{\quotep{Q}/x\}$

%We write $\wred$ for $\red^*$, and $P\red$ if $\exists Q $ such that $ P \red Q$.
We write $P\red$ if $\exists Q $ such that $ P \red Q$ and $P\not\red$, otherwise.

\section{Replication}

As mentioned before, it is known that replication (and hence
recursion) can be implemented in a higher-order process algebra
\cite{SangiorgiWalker}. As our first example of calculation with the
machinery thus far presented we give the construction explicitly in
the {\rhoc}.

\begin{eqnarray}
	D_{x} & := & \prefix{x}{y}{(\binpar{\outputp{x}{y}}{@{y}})} \nonumber\\
	\bangp_{x}{P} & := & \binpar{{x}!\langle{\binpar{D_{x}}{P}}\rangle}{D_{x}} \nonumber
\end{eqnarray}

\begin{eqnarray}
	\bangp_{x}{P} & & \nonumber\\
	=
	& {x}!\langle{(\prefix{x}{y}{(\outputp{x}{y} | @{y})) | P}}\rangle 
	      | \prefix{x}{y}{(\outputp{x}{y} | @{y})} & \nonumber\\
	\red
	& (\outputp{x}{y} | @{y})\substn{\quotep{(\prefix{x}{y}{(@{y} | \outputp{x}{y})) | P}}}{y} & \nonumber\\
	=
	& \outputp{x}{\quotep{(\prefix{x}{y}{(\outputp{x}{y} | @{y})) | P}}}
	  | {(\prefix{x}{y}{(\outputp{x}{y} | @{y})) | P}} & \nonumber\\
	\red
	& \ldots & \nonumber\\
	\red^*
	& P | P | \ldots & \nonumber
\end{eqnarray}

Of course, this encoding, as an implementation, runs away, unfolding
$\bangp{P}$ eagerly. A lazier and more implementable replication
operator, restricted to input-guarded processes, may be obtained as follows.

\begin{eqnarray}
\bangp{\prefix{u}{v}{P}} 
	:= 
	\binpar{\lift{x}{\prefix{u}{v}{(\binpar{D(x)}{P})}}}{D(x)} \nonumber
\end{eqnarray}

\begin{remark}
  Note that the lazier definition still does not deal with summation
  or mixed summation (i.e. sums over input and output). The reader is
  invited to construct definitions of replication that deal with these
  features. 

  Further, the definitions are parameterized in a name, $x$. Can you,
  gentle reader, make a definition that eliminates this parameter and
  guarantees no accidental interaction between the replication
  machinery and the process being replicated -- i.e. no accidental
  sharing of names used by the process to get its work done and the
  name(s) used by the replication to effect copying. This latter
  revision of the definition of replication is crucial to obtaining
  the expected identity $!!P \sim !P$.
\end{remark}

\begin{remark}\label{rem:paradoxical_combinator}
  The reader familiar with the lambda calculus will have noticed the
  similarity between $D$ and the paradoxical combinator.

  [Ed. note: the existence of this seems to suggest we have to be more
  restrictive on the set of processes and names we admit if we are to
  support no-cloning.]
\end{remark}

\subsubsection{Bisimulation}

The computational dynamics gives rise to another kind of equivalence,
the equivalence of computational behavior. As previously mentioned
this is typically captured \emph{via} some form of bisimulation.

% The notion we use in this paper is weak barbed bisimulation
% \cite{milner91polyadicpi}.

The notion we use in this paper is derived from weak barbed
bisimulation \cite{milner91polyadicpi}. 

\begin{definition}
An \emph{observation relation}, $\downarrow_{\mathcal N}$, over a set
of names, $\mathcal N$, is the smallest relation satisfying the rules
below.

\infrule[Out-barb]{y \in {\mathcal N}, \; x \nameeq y}
		  {\outputp{x}{v} \downarrow_{\mathcal N} x}
\infrule[Par-barb]{\mbox{$P\downarrow_{\mathcal N} x$ or $Q\downarrow_{\mathcal N} x$}}
		  {\binpar{P}{Q} \downarrow_{\mathcal N} x}

We write $P \Downarrow_{\mathcal N} x$ if there is $Q$ such that 
$P \wred Q$ and $Q \downarrow_{\mathcal N} x$.
\end{definition}

\begin{definition}
%\label{def.bbisim}
An  ${\mathcal N}$-\emph{barbed bisimulation} over a set of names, ${\mathcal N}$, is a symmetric binary relation 
${\mathcal S}_{\mathcal N}$ between agents such that $P\rel{S}_{\mathcal N}Q$ implies:
\begin{enumerate}
\item If $P \red P'$ then $Q \wred Q'$ and $P'\rel{S}_{\mathcal N} Q'$.
\item If $P\downarrow_{\mathcal N} x$, then $Q\Downarrow_{\mathcal N} x$.
\end{enumerate}
$P$ is ${\mathcal N}$-barbed bisimilar to $Q$, written
$P \wbbisim_{\mathcal N} Q$, if $P \rel{S}_{\mathcal N} Q$ for some ${\mathcal N}$-barbed bisimulation ${\mathcal S}_{\mathcal N}$.
\end{definition}

$\mathcal{R} \subseteq \pi \times \pi$

$P \mathcal{R} Q => \forall P'. P \red P' \Rightarrow \exists Q'. Q \red Q', P' \mathcal{R} Q'$

$P \vdash x \Rightarrow Q \vdash x$

\begin{mathpar}
  \inferrule*[lab=Out-barb]{x \nameeq y}{{y}!\langle{Q}\rangle \vdash x}
  \and
  \inferrule*[lab=Par-barb]{\mbox{$P\vdash x$ or $Q\vdash x$}}{\binpar{P}{Q} \vdash x}
\end{mathpar}

\subsubsection{Contexts}

One of the principle advantages of computational calculi like the
$\pi$-calculus is a well-defined notion of context,
contextual-equivalence and a correlation between
contextual-equivalence and notions of bisimulation. The notion of
context allows the decomposition of a process into (sub-)process and
its syntactic environment, its context. Thus, a context may be
thought of as a process with a ``hole'' (written $\Box$) in it. The
application of a context $M$ to a process $P$, written $M[P]$, is
tantamount to filling the hole in $M$ with $P$. In this paper we do
not need the full weight of this theory, but do make use of the notion
of context in the proof the main theorem. 

\begin{mathpar}
  \inferrule* [lab=summation] {} {{M_{M},M_{N}} \bc \Box \;|\; x.M_{A} \;|\; M_{M}+M_{N}}
  \and
  \inferrule* [lab=agent] {} {{M_{A}} \bc (\vec{x})M_{P} \;| \; \clift{P_0,\ldots,M_{P},\ldots,P_N}}
  \and \\
  \inferrule* [lab=process] {} {{M_{P}} \bc M_{N} \;| \;P|M_{P} }
\end{mathpar} 

\begin{mathpar}
  \inferrule* [lab=sychronization] {} {M_{N} \bc \Box \;|\; x?M_{F} \;|\; x!M_{C}}
  \and
  \inferrule* [lab=abstraction] {} {{M_{F}} \bc (x)M_{P} }
  \and
  \inferrule* [lab=concretion] {} {{M_{C}} \bc \langle M_{P} \rangle }
  \and \\
  \inferrule* [lab=process] {} {{M_{P}} \bc M_{N} \;| \;P|M_{P} }
\end{mathpar}

\begin{definition}[contextual application] Given a context $M$, and
  process $P$, we define the \emph{contextual application}, $M[P] :=
  M\{P/\Box\}$. That is, the contextual application of M to P is the
  substitution of $P$ for $\Box$ in $M$.
\end{definition}

$\meaningof{-} : L \to \mathcal{P}(\pi)$

\begin{mathpar}
  \inferrule* [lab=collection] {} {\meaningof{true} = \pi, \and \meaningof{~E} = \pi \setminus \meaningof{E}, \and \meaningof{E_{1} \& E_{2}} = \meaningof{E_{1}} \cap \meaningof{E_{2}}}
\end{mathpar}

\begin{mathpar}
  \inferrule* [lab=structure] {} {\meaningof{0} = \{ P \in \pi | P \equiv 0 \}, \and \\ \meaningof{E_1 | E_2} = \{ P \in \pi | P \equiv P_{1} | P_{2}, P_{1} \in \meaningof{E_{1}}, P_{2} \in \meaningof{E_2}\} }
\end{mathpar}

\begin{mathpar}
 \inferrule* [lab=behavior] {} {\meaningof{\langle a?b \rangle E} = \{ P \in \pi | P \equiv Q | u?(y)P', \\ \and \\\\ \and \\ \;\;\; u \in \meaningof{a}, \forall z.P'\{z/y\} \in \meaningof{E\{z/b\}}\}, \and \\ \meaningof{a!E} = \{ P \in \pi | P \equiv Q | x!\langle P' \rangle, x \in \meaningof{a} P' \in \meaningof{E}\} }
\end{mathpar}

\begin{mathpar}
 \inferrule* [lab=nominal] {} {\meaningof{\quotep{E}} = \{ \quotep{P} \in \quotep{\pi} | P \in \meaningof{E} \}, \and \meaningof{\quotep{P}} = \{ \quotep{Q} \in \quotep{\pi} | P \equiv Q \} \and \\ \meaningof{@\quotep{E}} = \{ P \in \pi | P \equiv @x, x \in \meaningof{E} \}}
\end{mathpar}

\begin{eqnarray*}
  \\
  \meaningof{-} : TS \to ST
\end{eqnarray*}

\begin{eqnarray*}
  \\
  L : TS \to ST
\end{eqnarray*}

\begin{eqnarray*}
  \\
  P \models E \iff P \in \meaningof{E}
\end{eqnarray*}

\begin{eqnarray*}
  P \approx_{L} Q \iff \forall E \in L. P \models E \iff Q \models E
\end{eqnarray*}

\begin{eqnarray*}
  P \approx_{K} Q
\end{eqnarray*}

\begin{eqnarray*}
  P \approx Q
\end{eqnarray*}

$\approx_{K} = \approx = \approx_{L}$

\subsubsection{Contextual duality}

Note that contexts extend the quotation operation to a family of
operations from processes to names. Given a context, $M$, we can
define a \emph{nominal context}, $\quotep{M}$ by $\quotep{M}[P] :=
\quotep{M[P]}$. To foreshadow what is to come we observe that these
operations enjoy a duality with processes very much like the duality
between vectors and maps from vectors to scalars.

Further, because the calculus is essentially higher-order, we have a
correspondence between contexts and processes. More specifically,
given a name $x$ and a context $M$ we can construct $M^{*}_{x}$ such
that 

\begin{mathpar}
  M^{*}_{x} | \lift{x}{P} \red M[P]
\end{mathpar}

namely,

\begin{mathpar}
  M^{*}_{x} := x?(u).M[\dropn{u}]
\end{mathpar}

The dependence of $M^{*}_{x}$ on a name makes it an abstraction, 

\begin{mathpar}
  M^{*} := (x)x?(u).M[\dropn{u}]
\end{mathpar}

\subsection{Additional notation}

It will sometimes be convenient to denote the process a name
quotes. We already have the notation $x = \quotep{P}$, but it will be
convenient to introduce an alternate notation, $\procn{x}$, when we
want to emphasize the connection to the use of the name. Note that, by
virtue of name equivalence, $\quotep{\procn{x}} \nameeq x$; so, the
notation is consistent with previous definitions.

Further, because names have structure it is possible to effect
substitutions on the basis of that structure. This means we need to
upgrade our notation for substitutions, which we accomplish by
adapting comprehension notation. Thus,

\begin{mathpar}
  P\{ y / x : x \in S \}
\end{mathpar}

is interpreted to mean the process derived from P by replacing (in a
capture-avoiding manner) each occurrence of $x$ in $S$ by $y$. For example,

\begin{mathpar}
  P\{ \quotep{\procn{x}|\procn{x}} / x : x \in \freenames{P} \}
\end{mathpar}

will replace each (occurrence) of a free name $x$ in $P$ by
$\quotep{\procn{x}|\procn{x}}$.

Also, we will avail ourselves of the notation $x^{L}$ and $x^{R}$ to
denote injections of a name into disjoint copies of the name
space. There are numerous ways to accomplish this. One example can be
found in \cite{MeredithR05}. This notation overloads to vectors of
names: $\vec{x}^{\pi} := (x_{i}^{\pi} \; : \; 0 \leq i < |\vec{x}| )$ where $\pi \in \{L,R\}$.

We also use $P^{\Box} := P|\Box$.

In \cite{MeredithR05} an interpretation of the new operator is
given. It turns out that there are several possible interpretations
all enjoying the requisite algebraic properties of the operator (see
\cite{milner91polyadicpi}). We will therefore make liberal use of
$(\nu\; \vec{x})P$.

% subsection the_syntax_and_semantics_of_the_notation_system (end)   

\input{qm2pi.qmops} 

\input{qm2pi.sterngerlach} 

\input{qm2pi.metric} 

% section concurrent_process_calculi (end)

%\input{qm2pi.proofsketch}

% section proof sketch (end)

%\input{qm2pi.slviaknots} 

% section spatial logic via knots (end)

\input{qm2pi.conclusion}

% section conclusion (end)

%\input{qm2pi.dtcodes} 

% section wiring algorithm (end)

\input{qm2pi.ack} 

% section acknowledgments (end)

\newpage


\bibliographystyle{plain}   
\bibliography{../../biblios/main.bib}

\input{qm2pi.rhodetails}

\end{document}

 

%\documentclass[12pt]{llncs}
%\documentclass{jktr}

\usepackage[pdftex]{hyperref}                   
\usepackage {listings}
\usepackage {mathpartir}
\usepackage{bcprules}
%\usepackage{listings}
                       
\usepackage{graphicx} 
%\usepackage[margins=2.5cm,nohead,nofoot]{geometry}
%\usepackage{geometry}
\usepackage{amsfonts}
\usepackage{amstext}
\usepackage{latexsym}
\usepackage{amssymb}
\usepackage{color}


%\include{myPreamble}
\include{qm2pi.local} 

%\ifpdf
%\usepackage[pdftex]{graphicx}
%\else
%\usepackage{graphicx}
%\fi

 % \ifpdf
%  \usepackage{pdfsync}
%  \if


%\title{Brief Article}
%\author{David F. Snyder}
%\author{L.G. Meredith}

%\address{Dept. of Math., Texas State University--San Marcos, San Marcos, TX 78666}
       
\pagestyle{empty}


\begin{document}

\lstset{language=[Objective]Caml,frame=shadowbox}

\input{qm2pi.front}

% section front matter (end)

\input{qm2pi.intro} 
 
% section introduction (end)

% \input{qm2pi.knotations} 

% section notation (end)

\input{qm2pi.process.calculi} 

% section concurrent_process_calculi_and_spatial_logics_ (end)
    
%\input{qm2pi.knots2pi} 

%\input{qm2pi.trefoil} 

%\input{qm2pi.mainthm} 

% subsection basic_interpretation (end)

%\input{qm2pi.rho.presentation} 
\subsection{The syntax and semantics of the notation system}\label{sub:the_syntax_and_semantics_of_the_notation_system} % (fold)

We now summarize a technical presentation of the calculus that
embodies our theory of dynamics. The typical presentation of such a
calculus follows the style of giving generators and relations on
them. The grammar, below, describing term constructors, freely
generates the set of processes, $\Proc$. This set is then quotiented
by a relation known as structural congruence and it is over this set
that the notion of dynamics is expressed. This presentation is
essentially that of \cite{MeredithR05} with the addition of
polyadicity and summation. For readability we have relegated some of
the technical subtleties to an appendix.

\subsubsection{Process grammar}\label{subsub:process_grammar}

\begin{mathpar}
  \inferrule* [lab=synchronization] {} {{M} \bc \pzero \;|\; x?F \;|\; x!C }
  \and
  \inferrule* [lab=abstraction] {} {{F} \bc (x)P}
  \and
  \inferrule* [lab=concretion] {} {{C} \bc \langle Q \rangle}
  \and
  \inferrule* [lab=process] {} {{P,Q} \bc M \;| \;P|Q \;|\; @{x}}
  \and
  \inferrule* [lab=name] {} {{x} \bc \quotep{P}}
\end{mathpar} 

Note that $\vec{x}$ (resp. $\vec{P}$) denotes a vector of names
(resp. processes) of length $|\vec{x}|$ (resp. $|\vec{P}|$). We adopt
the following useful abbreviations.

\begin{mathpar}
   x?(\vec{y}).P := x.(\vec{y})P \and  x\clift{\vec{P}} := x.\clift{\vec{P}}
   \and x!(y) := \lift{x}{\dropn{y}}
   \and \Pi_{i=0}^{n-1}P_i := P_0 | \ldots | P_{n-1}
\end{mathpar}

\subsubsection{Structural congruence}

\paragraph{Free and bound names and alpha-equivalence.} At the
core of structural equivalence is alpha-equivalence which identifies
process that are the same up to a change of variable. Formally, we
recognize the distinction between free and bound names. The free names
of a process, $\freenames{P}$, may be calculated recursively as
follows:

\begin{mathpar}
\freenames{\pzero} := \emptyset
  \and \\
  \freenames{x?(y).P} := \{ x \} \cup (\freenames{P} \setminus \{ y \})
  \and 
  \freenames{x!\langle P \rangle} := \{ x \} \cup \{ P \} 
  \and \\
  \freenames{P|Q} := \freenames{P} \cup \freenames{Q}
  \and \\
  \freenames{@{x}} := \{ x \}
\end{mathpar}

$\pi$
$\quotep{\pi}$

$\freenames{-} : \pi \to \mathcal{P}(\quotep{\pi})$

\begin{eqnarray*}
  \freenames{\pzero} & := & \emptyset \\
  \freenames{x?(y).P} & := & \{ x \} \cup (\freenames{P} \setminus \{ y \}) \\
  \freenames{x!\langle P \rangle} & := & \{ x \} \cup \{ P \} \\
  \freenames{P|Q} & := & \freenames{P} \cup \freenames{Q} \\
  \freenames{\dropn{x}} & := & \{ x \}
\end{eqnarray*}

The bound names of a process, $\boundnames{P}$, are those names occurring in $P$
that are not free. For example, in $x?(y).0$, the name $x$ is free, while $y$ is bound.

\begin{mathpar}
  \inferrule* [lab=monoidal-laws] {} { P|Q \equiv Q|P \and P|0 \equiv P \and P|(Q|R) \equiv (P|Q)|R }
\end{mathpar}

\begin{mathpar}
  \inferrule* [lab=alpha-equivalence] {} { (x)P \equiv (y)P\{y/x\} \and y \not\in \freenames{P} }
\end{mathpar}

\begin{definition}
Then two processes, $P,Q$, are alpha-equivalent if $P = Q\{\vec{y}/\vec{x}\}$ for
some $\vec{x} \in \boundnames{Q},\vec{y} \in \boundnames{P}$, where $Q\{\vec{y}/\vec{x}\}$
denotes the capture-avoiding substitution of $\vec{y}$ for $\vec{x}$ in $Q$.
\end{definition}

\begin{definition}
  The {\em structural congruence} \cite{SangiorgiWalker} , $\equiv$,
  between processes is the least congruence containing
  alpha-equivalence, satisfying the abelian monoid laws
  (associativity, commutativity and $\pzero$ as identity) for parallel
  composition $|$ and for summation $+$.
\end{definition}

\subsection{Name equivalence}

We take name equivalence, written $\nameeq$, to be the smallest
equivalence relation generated by the following rules.

\begin{mathpar}
\inferrule*[lab=Quote-drop]
{ }
{ \quotep{@{x}} \nameeq x }

\inferrule*[lab=Struct-equiv]
{ P \scong Q }
{ \quotep{P} \nameeq \quotep{Q} }
\end{mathpar}

The astute reader will have noticed that the mutual recursion of names
and processes imposes a mutual recursion on alpha-equivalence and
structural equivalence via name-equivalence. Fortunately, all of this
works out pleasantly and we may calculate in the natural way, free of
concern. The reader interested in the details is referred to the
appendix \ref{appendix:rho_details}.

\subsection{Substitution}

We use $\Proc$ for the set of processes, $\QProc$ for the set of
names, and $\id{\{}\vec{y} / \vec{x} \id{\}}$ to denote partial maps,
$s : \QProc \rightarrow \QProc$. A map, $s$ lifts, uniquely, to a map
on process terms, $\widehat{s} : \Proc \rightarrow \Proc$ by the
following equations.

\begin{mathpar}
  (0) \psubstp{Q}{P} := 0 \\
  (R \juxtap S) \psubstp{Q}{P}
  :=    
  (R)\psubstp{Q}{P} \juxtap (S) \psubstp{Q}{P} \\
  (x?(y).R) \psubstp{Q}{P}    
  :=    
  (x)\substp{Q}{P} (z)\concat( (R \psubstn{z}{y}) \psubstp{Q}{P} ) \\
  (\lift{x}{R}) \psubstp{Q}{P}  
  :=
  \lift{(x)\substp{Q}{P}}{ R \psubstp{Q}{P} } \\
%   (\dropn{x})  \psubstp{Q}{P}       
%   := 
%   \left\{ 
%     \begin{array}{ccc} 
%       \dropn{\quotep{Q}} & & x \nameeq \quotep{P} \\
%       \dropn{x} & & otherwise \\
%     \end{array}
%   \right. 
  (\dropn{x})  \psubstp{Q}{P}       
  := 
  \left\{ 
    \begin{array}{ccc} 
      Q & & x \nameeq \quotep{P} \\
      \dropn{x} & & otherwise \\
    \end{array}
  \right.
\end{mathpar}
 

where

\begin{eqnarray}
  (x)\id{\{} \lpquote Q \rpquote / \lpquote P \rpquote \id{\}}            = 
  \left\{ 
    \begin{array}{ccc}
      \lpquote Q \rpquote & & x \nameeq \lpquote P \rpquote \\
      x & & otherwise \\
    \end{array}
  \right. \nonumber
\end{eqnarray}

and $z$ is chosen distinct from $\quotep{P}$, $\quotep{Q}$, the free
names in $Q$, and all the names in $R$. Our $\alpha$-equivalence will
be built in the standard way from this substitution.

\begin{remark}\label{rem:no_self_referential_names}
  One consequence of these definitions is that $\forall P. \quotep{P}
  \not\in \freenames{P}$.
\end{remark}

\subsection{ Dynamic quote: an example }

Anticipating something of what's to come, consider applying the
substitution, $\widehat{\id{\{}u / z \id{\}}}$, to the following pair
of processes, $\lift{w}{y!(z)}$ and $w[ \lpquote y!(z) \rpquote ]$.

\begin{eqnarray}
	\lift{w}{y!(z)}\widehat{\id{\{}u / z \id{\}}}
		& = &
		\lift{w}{y!(u)} \nonumber\\
	w[ \lpquote y!(z) \rpquote ] \widehat{ \id{\{}u / z \id{\}} }
		& = &
		w[ \lpquote y!(z) \rpquote ] \nonumber
\end{eqnarray}

Because the body of the process between quotes is impervious to
substitution, we get radically different answers. In fact, by
examining the first process in an input context,
e.g. $x?(z).\lift{w}{y!(z)}$, we see that the process under the lift
operator may be shaped by prefixed inputs binding a name inside it. In
this sense, the lift operator will be seen as a way to dynamically
construct processes before reifying them as names.

Finally equipped with these standard features we can present the
dynamics of the calculus.

\subsubsection{Operational semantics} 

Finally, we introduce the computational dynamics. What marks these
algebras as distinct from other more traditionally studied algebraic
structures, e.g. vector spaces or polynomial rings, is the manner in
which dynamics is captured. In traditional structures, dynamics is typically
expressed through morphisms between such structures, as in linear maps
between vector spaces or morphisms between rings. In algebras
associated with the semantics of computation, the dynamics is
expressed as part of the algebraic structure itself, through a
reduction reduction relation typically denoted by $\red$. Below, we
give a recursive presentation of this relation for the calculus used
in the encoding.

$\red \subseteq \pi \times \pi$
$\red : \pi \to \mathcal{P}(\pi)$

\begin{mathpar}
  \inferrule* [lab=Comm] { \textsf{match}( x_{src}, x_{trgt} ) } { x_{trgt}?(y)P \; | \; x_{src}!\langle {Q} \rangle \red P\{\quotep{Q}/y}\} }
  \and \\
  \inferrule* [lab=Par] {{P} \red {P}'} {{{P} | {Q}} \red {{P}' | {Q}}}
  \and
  \inferrule* [lab=Equiv]{{{P} \scong {P}'} \andalso {{P}' \red {Q}'} \andalso {{Q}' \scong {Q}}}{{P} \red {Q}}
\end{mathpar}

\begin{eqnarray*}
  match_{\equiv} (\quotep{P},\quotep{Q}) & := & P \equiv Q \\
  match_{\dagger}(\quotep{P},\quotep{Q}) & := & \forall R. P|Q \red^{*} R => R \red^{*} 0 \\
  match_{K}(\quotep{P},\quotep{Q}) & := & K \mbox{ for some context } K
\end{eqnarray*}

$u?(x)P | u!\langle Q \rangle \red P\{\quotep{Q}/x\}$

%We write $\wred$ for $\red^*$, and $P\red$ if $\exists Q $ such that $ P \red Q$.
We write $P\red$ if $\exists Q $ such that $ P \red Q$ and $P\not\red$, otherwise.

\section{Replication}

As mentioned before, it is known that replication (and hence
recursion) can be implemented in a higher-order process algebra
\cite{SangiorgiWalker}. As our first example of calculation with the
machinery thus far presented we give the construction explicitly in
the {\rhoc}.

\begin{eqnarray}
	D_{x} & := & \prefix{x}{y}{(\binpar{\outputp{x}{y}}{@{y}})} \nonumber\\
	\bangp_{x}{P} & := & \binpar{{x}!\langle{\binpar{D_{x}}{P}}\rangle}{D_{x}} \nonumber
\end{eqnarray}

\begin{eqnarray}
	\bangp_{x}{P} & & \nonumber\\
	=
	& {x}!\langle{(\prefix{x}{y}{(\outputp{x}{y} | @{y})) | P}}\rangle 
	      | \prefix{x}{y}{(\outputp{x}{y} | @{y})} & \nonumber\\
	\red
	& (\outputp{x}{y} | @{y})\substn{\quotep{(\prefix{x}{y}{(@{y} | \outputp{x}{y})) | P}}}{y} & \nonumber\\
	=
	& \outputp{x}{\quotep{(\prefix{x}{y}{(\outputp{x}{y} | @{y})) | P}}}
	  | {(\prefix{x}{y}{(\outputp{x}{y} | @{y})) | P}} & \nonumber\\
	\red
	& \ldots & \nonumber\\
	\red^*
	& P | P | \ldots & \nonumber
\end{eqnarray}

Of course, this encoding, as an implementation, runs away, unfolding
$\bangp{P}$ eagerly. A lazier and more implementable replication
operator, restricted to input-guarded processes, may be obtained as follows.

\begin{eqnarray}
\bangp{\prefix{u}{v}{P}} 
	:= 
	\binpar{\lift{x}{\prefix{u}{v}{(\binpar{D(x)}{P})}}}{D(x)} \nonumber
\end{eqnarray}

\begin{remark}
  Note that the lazier definition still does not deal with summation
  or mixed summation (i.e. sums over input and output). The reader is
  invited to construct definitions of replication that deal with these
  features. 

  Further, the definitions are parameterized in a name, $x$. Can you,
  gentle reader, make a definition that eliminates this parameter and
  guarantees no accidental interaction between the replication
  machinery and the process being replicated -- i.e. no accidental
  sharing of names used by the process to get its work done and the
  name(s) used by the replication to effect copying. This latter
  revision of the definition of replication is crucial to obtaining
  the expected identity $!!P \sim !P$.
\end{remark}

\begin{remark}\label{rem:paradoxical_combinator}
  The reader familiar with the lambda calculus will have noticed the
  similarity between $D$ and the paradoxical combinator.

  [Ed. note: the existence of this seems to suggest we have to be more
  restrictive on the set of processes and names we admit if we are to
  support no-cloning.]
\end{remark}

\subsubsection{Bisimulation}

The computational dynamics gives rise to another kind of equivalence,
the equivalence of computational behavior. As previously mentioned
this is typically captured \emph{via} some form of bisimulation.

% The notion we use in this paper is weak barbed bisimulation
% \cite{milner91polyadicpi}.

The notion we use in this paper is derived from weak barbed
bisimulation \cite{milner91polyadicpi}. 

\begin{definition}
An \emph{observation relation}, $\downarrow_{\mathcal N}$, over a set
of names, $\mathcal N$, is the smallest relation satisfying the rules
below.

\infrule[Out-barb]{y \in {\mathcal N}, \; x \nameeq y}
		  {\outputp{x}{v} \downarrow_{\mathcal N} x}
\infrule[Par-barb]{\mbox{$P\downarrow_{\mathcal N} x$ or $Q\downarrow_{\mathcal N} x$}}
		  {\binpar{P}{Q} \downarrow_{\mathcal N} x}

We write $P \Downarrow_{\mathcal N} x$ if there is $Q$ such that 
$P \wred Q$ and $Q \downarrow_{\mathcal N} x$.
\end{definition}

\begin{definition}
%\label{def.bbisim}
An  ${\mathcal N}$-\emph{barbed bisimulation} over a set of names, ${\mathcal N}$, is a symmetric binary relation 
${\mathcal S}_{\mathcal N}$ between agents such that $P\rel{S}_{\mathcal N}Q$ implies:
\begin{enumerate}
\item If $P \red P'$ then $Q \wred Q'$ and $P'\rel{S}_{\mathcal N} Q'$.
\item If $P\downarrow_{\mathcal N} x$, then $Q\Downarrow_{\mathcal N} x$.
\end{enumerate}
$P$ is ${\mathcal N}$-barbed bisimilar to $Q$, written
$P \wbbisim_{\mathcal N} Q$, if $P \rel{S}_{\mathcal N} Q$ for some ${\mathcal N}$-barbed bisimulation ${\mathcal S}_{\mathcal N}$.
\end{definition}

$\mathcal{R} \subseteq \pi \times \pi$

$P \mathcal{R} Q => \forall P'. P \red P' \Rightarrow \exists Q'. Q \red Q', P' \mathcal{R} Q'$

$P \vdash x \Rightarrow Q \vdash x$

\begin{mathpar}
  \inferrule*[lab=Out-barb]{x \nameeq y}{{y}!\langle{Q}\rangle \vdash x}
  \and
  \inferrule*[lab=Par-barb]{\mbox{$P\vdash x$ or $Q\vdash x$}}{\binpar{P}{Q} \vdash x}
\end{mathpar}

\subsubsection{Contexts}

One of the principle advantages of computational calculi like the
$\pi$-calculus is a well-defined notion of context,
contextual-equivalence and a correlation between
contextual-equivalence and notions of bisimulation. The notion of
context allows the decomposition of a process into (sub-)process and
its syntactic environment, its context. Thus, a context may be
thought of as a process with a ``hole'' (written $\Box$) in it. The
application of a context $M$ to a process $P$, written $M[P]$, is
tantamount to filling the hole in $M$ with $P$. In this paper we do
not need the full weight of this theory, but do make use of the notion
of context in the proof the main theorem. 

\begin{mathpar}
  \inferrule* [lab=summation] {} {{M_{M},M_{N}} \bc \Box \;|\; x.M_{A} \;|\; M_{M}+M_{N}}
  \and
  \inferrule* [lab=agent] {} {{M_{A}} \bc (\vec{x})M_{P} \;| \; \clift{P_0,\ldots,M_{P},\ldots,P_N}}
  \and \\
  \inferrule* [lab=process] {} {{M_{P}} \bc M_{N} \;| \;P|M_{P} }
\end{mathpar} 

\begin{mathpar}
  \inferrule* [lab=sychronization] {} {M_{N} \bc \Box \;|\; x?M_{F} \;|\; x!M_{C}}
  \and
  \inferrule* [lab=abstraction] {} {{M_{F}} \bc (x)M_{P} }
  \and
  \inferrule* [lab=concretion] {} {{M_{C}} \bc \langle M_{P} \rangle }
  \and \\
  \inferrule* [lab=process] {} {{M_{P}} \bc M_{N} \;| \;P|M_{P} }
\end{mathpar}

\begin{definition}[contextual application] Given a context $M$, and
  process $P$, we define the \emph{contextual application}, $M[P] :=
  M\{P/\Box\}$. That is, the contextual application of M to P is the
  substitution of $P$ for $\Box$ in $M$.
\end{definition}

$\meaningof{-} : L \to \mathcal{P}(\pi)$

\begin{mathpar}
  \inferrule* [lab=collection] {} {\meaningof{true} = \pi, \and \meaningof{~E} = \pi \setminus \meaningof{E}, \and \meaningof{E_{1} \& E_{2}} = \meaningof{E_{1}} \cap \meaningof{E_{2}}}
\end{mathpar}

\begin{mathpar}
  \inferrule* [lab=structure] {} {\meaningof{0} = \{ P \in \pi | P \equiv 0 \}, \and \\ \meaningof{E_1 | E_2} = \{ P \in \pi | P \equiv P_{1} | P_{2}, P_{1} \in \meaningof{E_{1}}, P_{2} \in \meaningof{E_2}\} }
\end{mathpar}

\begin{mathpar}
 \inferrule* [lab=behavior] {} {\meaningof{\langle a?b \rangle E} = \{ P \in \pi | P \equiv Q | u?(y)P', \\ \and \\\\ \and \\ \;\;\; u \in \meaningof{a}, \forall z.P'\{z/y\} \in \meaningof{E\{z/b\}}\}, \and \\ \meaningof{a!E} = \{ P \in \pi | P \equiv Q | x!\langle P' \rangle, x \in \meaningof{a} P' \in \meaningof{E}\} }
\end{mathpar}

\begin{mathpar}
 \inferrule* [lab=nominal] {} {\meaningof{\quotep{E}} = \{ \quotep{P} \in \quotep{\pi} | P \in \meaningof{E} \}, \and \meaningof{\quotep{P}} = \{ \quotep{Q} \in \quotep{\pi} | P \equiv Q \} \and \\ \meaningof{@\quotep{E}} = \{ P \in \pi | P \equiv @x, x \in \meaningof{E} \}}
\end{mathpar}

\begin{eqnarray*}
  \\
  \meaningof{-} : TS \to ST
\end{eqnarray*}

\begin{eqnarray*}
  \\
  L : TS \to ST
\end{eqnarray*}

\begin{eqnarray*}
  \\
  P \models E \iff P \in \meaningof{E}
\end{eqnarray*}

\begin{eqnarray*}
  P \approx_{L} Q \iff \forall E \in L. P \models E \iff Q \models E
\end{eqnarray*}

\begin{eqnarray*}
  P \approx_{K} Q
\end{eqnarray*}

\begin{eqnarray*}
  P \approx Q
\end{eqnarray*}

$\approx_{K} = \approx = \approx_{L}$

\subsubsection{Contextual duality}

Note that contexts extend the quotation operation to a family of
operations from processes to names. Given a context, $M$, we can
define a \emph{nominal context}, $\quotep{M}$ by $\quotep{M}[P] :=
\quotep{M[P]}$. To foreshadow what is to come we observe that these
operations enjoy a duality with processes very much like the duality
between vectors and maps from vectors to scalars.

Further, because the calculus is essentially higher-order, we have a
correspondence between contexts and processes. More specifically,
given a name $x$ and a context $M$ we can construct $M^{*}_{x}$ such
that 

\begin{mathpar}
  M^{*}_{x} | \lift{x}{P} \red M[P]
\end{mathpar}

namely,

\begin{mathpar}
  M^{*}_{x} := x?(u).M[\dropn{u}]
\end{mathpar}

The dependence of $M^{*}_{x}$ on a name makes it an abstraction, 

\begin{mathpar}
  M^{*} := (x)x?(u).M[\dropn{u}]
\end{mathpar}

\subsection{Additional notation}

It will sometimes be convenient to denote the process a name
quotes. We already have the notation $x = \quotep{P}$, but it will be
convenient to introduce an alternate notation, $\procn{x}$, when we
want to emphasize the connection to the use of the name. Note that, by
virtue of name equivalence, $\quotep{\procn{x}} \nameeq x$; so, the
notation is consistent with previous definitions.

Further, because names have structure it is possible to effect
substitutions on the basis of that structure. This means we need to
upgrade our notation for substitutions, which we accomplish by
adapting comprehension notation. Thus,

\begin{mathpar}
  P\{ y / x : x \in S \}
\end{mathpar}

is interpreted to mean the process derived from P by replacing (in a
capture-avoiding manner) each occurrence of $x$ in $S$ by $y$. For example,

\begin{mathpar}
  P\{ \quotep{\procn{x}|\procn{x}} / x : x \in \freenames{P} \}
\end{mathpar}

will replace each (occurrence) of a free name $x$ in $P$ by
$\quotep{\procn{x}|\procn{x}}$.

Also, we will avail ourselves of the notation $x^{L}$ and $x^{R}$ to
denote injections of a name into disjoint copies of the name
space. There are numerous ways to accomplish this. One example can be
found in \cite{MeredithR05}. This notation overloads to vectors of
names: $\vec{x}^{\pi} := (x_{i}^{\pi} \; : \; 0 \leq i < |\vec{x}| )$ where $\pi \in \{L,R\}$.

We also use $P^{\Box} := P|\Box$.

In \cite{MeredithR05} an interpretation of the new operator is
given. It turns out that there are several possible interpretations
all enjoying the requisite algebraic properties of the operator (see
\cite{milner91polyadicpi}). We will therefore make liberal use of
$(\nu\; \vec{x})P$.

% subsection the_syntax_and_semantics_of_the_notation_system (end)   

\input{qm2pi.qmops} 

\input{qm2pi.sterngerlach} 

\input{qm2pi.metric} 

% section concurrent_process_calculi (end)

%\input{qm2pi.proofsketch}

% section proof sketch (end)

%\input{qm2pi.slviaknots} 

% section spatial logic via knots (end)

\input{qm2pi.conclusion}

% section conclusion (end)

%\input{qm2pi.dtcodes} 

% section wiring algorithm (end)

\input{qm2pi.ack} 

% section acknowledgments (end)

\newpage


\bibliographystyle{plain}   
\bibliography{../../biblios/main.bib}

\input{qm2pi.rhodetails}

\end{document}

 

% subsection basic_interpretation (end)

%\input{qm2pi.rho.presentation} 
\subsection{The syntax and semantics of the notation system}\label{sub:the_syntax_and_semantics_of_the_notation_system} % (fold)

We now summarize a technical presentation of the calculus that
embodies our theory of dynamics. The typical presentation of such a
calculus follows the style of giving generators and relations on
them. The grammar, below, describing term constructors, freely
generates the set of processes, $\Proc$. This set is then quotiented
by a relation known as structural congruence and it is over this set
that the notion of dynamics is expressed. This presentation is
essentially that of \cite{MeredithR05} with the addition of
polyadicity and summation. For readability we have relegated some of
the technical subtleties to an appendix.

\subsubsection{Process grammar}\label{subsub:process_grammar}

\begin{mathpar}
  \inferrule* [lab=synchronization] {} {{M} \bc \pzero \;|\; x?F \;|\; x!C }
  \and
  \inferrule* [lab=abstraction] {} {{F} \bc (x)P}
  \and
  \inferrule* [lab=concretion] {} {{C} \bc \langle Q \rangle}
  \and
  \inferrule* [lab=process] {} {{P,Q} \bc M \;| \;P|Q \;|\; @{x}}
  \and
  \inferrule* [lab=name] {} {{x} \bc \quotep{P}}
\end{mathpar} 

Note that $\vec{x}$ (resp. $\vec{P}$) denotes a vector of names
(resp. processes) of length $|\vec{x}|$ (resp. $|\vec{P}|$). We adopt
the following useful abbreviations.

\begin{mathpar}
   x?(\vec{y}).P := x.(\vec{y})P \and  x\clift{\vec{P}} := x.\clift{\vec{P}}
   \and x!(y) := \lift{x}{\dropn{y}}
   \and \Pi_{i=0}^{n-1}P_i := P_0 | \ldots | P_{n-1}
\end{mathpar}

\subsubsection{Structural congruence}

\paragraph{Free and bound names and alpha-equivalence.} At the
core of structural equivalence is alpha-equivalence which identifies
process that are the same up to a change of variable. Formally, we
recognize the distinction between free and bound names. The free names
of a process, $\freenames{P}$, may be calculated recursively as
follows:

\begin{mathpar}
\freenames{\pzero} := \emptyset
  \and \\
  \freenames{x?(y).P} := \{ x \} \cup (\freenames{P} \setminus \{ y \})
  \and 
  \freenames{x!\langle P \rangle} := \{ x \} \cup \{ P \} 
  \and \\
  \freenames{P|Q} := \freenames{P} \cup \freenames{Q}
  \and \\
  \freenames{@{x}} := \{ x \}
\end{mathpar}

$\pi$
$\quotep{\pi}$

$\freenames{-} : \pi \to \mathcal{P}(\quotep{\pi})$

\begin{eqnarray*}
  \freenames{\pzero} & := & \emptyset \\
  \freenames{x?(y).P} & := & \{ x \} \cup (\freenames{P} \setminus \{ y \}) \\
  \freenames{x!\langle P \rangle} & := & \{ x \} \cup \{ P \} \\
  \freenames{P|Q} & := & \freenames{P} \cup \freenames{Q} \\
  \freenames{\dropn{x}} & := & \{ x \}
\end{eqnarray*}

The bound names of a process, $\boundnames{P}$, are those names occurring in $P$
that are not free. For example, in $x?(y).0$, the name $x$ is free, while $y$ is bound.

\begin{mathpar}
  \inferrule* [lab=monoidal-laws] {} { P|Q \equiv Q|P \and P|0 \equiv P \and P|(Q|R) \equiv (P|Q)|R }
\end{mathpar}

\begin{mathpar}
  \inferrule* [lab=alpha-equivalence] {} { (x)P \equiv (y)P\{y/x\} \and y \not\in \freenames{P} }
\end{mathpar}

\begin{definition}
Then two processes, $P,Q$, are alpha-equivalent if $P = Q\{\vec{y}/\vec{x}\}$ for
some $\vec{x} \in \boundnames{Q},\vec{y} \in \boundnames{P}$, where $Q\{\vec{y}/\vec{x}\}$
denotes the capture-avoiding substitution of $\vec{y}$ for $\vec{x}$ in $Q$.
\end{definition}

\begin{definition}
  The {\em structural congruence} \cite{SangiorgiWalker} , $\equiv$,
  between processes is the least congruence containing
  alpha-equivalence, satisfying the abelian monoid laws
  (associativity, commutativity and $\pzero$ as identity) for parallel
  composition $|$ and for summation $+$.
\end{definition}

\subsection{Name equivalence}

We take name equivalence, written $\nameeq$, to be the smallest
equivalence relation generated by the following rules.

\begin{mathpar}
\inferrule*[lab=Quote-drop]
{ }
{ \quotep{@{x}} \nameeq x }

\inferrule*[lab=Struct-equiv]
{ P \scong Q }
{ \quotep{P} \nameeq \quotep{Q} }
\end{mathpar}

The astute reader will have noticed that the mutual recursion of names
and processes imposes a mutual recursion on alpha-equivalence and
structural equivalence via name-equivalence. Fortunately, all of this
works out pleasantly and we may calculate in the natural way, free of
concern. The reader interested in the details is referred to the
appendix \ref{appendix:rho_details}.

\subsection{Substitution}

We use $\Proc$ for the set of processes, $\QProc$ for the set of
names, and $\id{\{}\vec{y} / \vec{x} \id{\}}$ to denote partial maps,
$s : \QProc \rightarrow \QProc$. A map, $s$ lifts, uniquely, to a map
on process terms, $\widehat{s} : \Proc \rightarrow \Proc$ by the
following equations.

\begin{mathpar}
  (0) \psubstp{Q}{P} := 0 \\
  (R \juxtap S) \psubstp{Q}{P}
  :=    
  (R)\psubstp{Q}{P} \juxtap (S) \psubstp{Q}{P} \\
  (x?(y).R) \psubstp{Q}{P}    
  :=    
  (x)\substp{Q}{P} (z)\concat( (R \psubstn{z}{y}) \psubstp{Q}{P} ) \\
  (\lift{x}{R}) \psubstp{Q}{P}  
  :=
  \lift{(x)\substp{Q}{P}}{ R \psubstp{Q}{P} } \\
%   (\dropn{x})  \psubstp{Q}{P}       
%   := 
%   \left\{ 
%     \begin{array}{ccc} 
%       \dropn{\quotep{Q}} & & x \nameeq \quotep{P} \\
%       \dropn{x} & & otherwise \\
%     \end{array}
%   \right. 
  (\dropn{x})  \psubstp{Q}{P}       
  := 
  \left\{ 
    \begin{array}{ccc} 
      Q & & x \nameeq \quotep{P} \\
      \dropn{x} & & otherwise \\
    \end{array}
  \right.
\end{mathpar}
 

where

\begin{eqnarray}
  (x)\id{\{} \lpquote Q \rpquote / \lpquote P \rpquote \id{\}}            = 
  \left\{ 
    \begin{array}{ccc}
      \lpquote Q \rpquote & & x \nameeq \lpquote P \rpquote \\
      x & & otherwise \\
    \end{array}
  \right. \nonumber
\end{eqnarray}

and $z$ is chosen distinct from $\quotep{P}$, $\quotep{Q}$, the free
names in $Q$, and all the names in $R$. Our $\alpha$-equivalence will
be built in the standard way from this substitution.

\begin{remark}\label{rem:no_self_referential_names}
  One consequence of these definitions is that $\forall P. \quotep{P}
  \not\in \freenames{P}$.
\end{remark}

\subsection{ Dynamic quote: an example }

Anticipating something of what's to come, consider applying the
substitution, $\widehat{\id{\{}u / z \id{\}}}$, to the following pair
of processes, $\lift{w}{y!(z)}$ and $w[ \lpquote y!(z) \rpquote ]$.

\begin{eqnarray}
	\lift{w}{y!(z)}\widehat{\id{\{}u / z \id{\}}}
		& = &
		\lift{w}{y!(u)} \nonumber\\
	w[ \lpquote y!(z) \rpquote ] \widehat{ \id{\{}u / z \id{\}} }
		& = &
		w[ \lpquote y!(z) \rpquote ] \nonumber
\end{eqnarray}

Because the body of the process between quotes is impervious to
substitution, we get radically different answers. In fact, by
examining the first process in an input context,
e.g. $x?(z).\lift{w}{y!(z)}$, we see that the process under the lift
operator may be shaped by prefixed inputs binding a name inside it. In
this sense, the lift operator will be seen as a way to dynamically
construct processes before reifying them as names.

Finally equipped with these standard features we can present the
dynamics of the calculus.

\subsubsection{Operational semantics} 

Finally, we introduce the computational dynamics. What marks these
algebras as distinct from other more traditionally studied algebraic
structures, e.g. vector spaces or polynomial rings, is the manner in
which dynamics is captured. In traditional structures, dynamics is typically
expressed through morphisms between such structures, as in linear maps
between vector spaces or morphisms between rings. In algebras
associated with the semantics of computation, the dynamics is
expressed as part of the algebraic structure itself, through a
reduction reduction relation typically denoted by $\red$. Below, we
give a recursive presentation of this relation for the calculus used
in the encoding.

$\red \subseteq \pi \times \pi$
$\red : \pi \to \mathcal{P}(\pi)$

\begin{mathpar}
  \inferrule* [lab=Comm] { \textsf{match}( x_{src}, x_{trgt} ) } { x_{trgt}?(y)P \; | \; x_{src}!\langle {Q} \rangle \red P\{\quotep{Q}/y}\} }
  \and \\
  \inferrule* [lab=Par] {{P} \red {P}'} {{{P} | {Q}} \red {{P}' | {Q}}}
  \and
  \inferrule* [lab=Equiv]{{{P} \scong {P}'} \andalso {{P}' \red {Q}'} \andalso {{Q}' \scong {Q}}}{{P} \red {Q}}
\end{mathpar}

\begin{eqnarray*}
  match_{\equiv} (\quotep{P},\quotep{Q}) & := & P \equiv Q \\
  match_{\dagger}(\quotep{P},\quotep{Q}) & := & \forall R. P|Q \red^{*} R => R \red^{*} 0 \\
  match_{K}(\quotep{P},\quotep{Q}) & := & K \mbox{ for some context } K
\end{eqnarray*}

$u?(x)P | u!\langle Q \rangle \red P\{\quotep{Q}/x\}$

%We write $\wred$ for $\red^*$, and $P\red$ if $\exists Q $ such that $ P \red Q$.
We write $P\red$ if $\exists Q $ such that $ P \red Q$ and $P\not\red$, otherwise.

\section{Replication}

As mentioned before, it is known that replication (and hence
recursion) can be implemented in a higher-order process algebra
\cite{SangiorgiWalker}. As our first example of calculation with the
machinery thus far presented we give the construction explicitly in
the {\rhoc}.

\begin{eqnarray}
	D_{x} & := & \prefix{x}{y}{(\binpar{\outputp{x}{y}}{@{y}})} \nonumber\\
	\bangp_{x}{P} & := & \binpar{{x}!\langle{\binpar{D_{x}}{P}}\rangle}{D_{x}} \nonumber
\end{eqnarray}

\begin{eqnarray}
	\bangp_{x}{P} & & \nonumber\\
	=
	& {x}!\langle{(\prefix{x}{y}{(\outputp{x}{y} | @{y})) | P}}\rangle 
	      | \prefix{x}{y}{(\outputp{x}{y} | @{y})} & \nonumber\\
	\red
	& (\outputp{x}{y} | @{y})\substn{\quotep{(\prefix{x}{y}{(@{y} | \outputp{x}{y})) | P}}}{y} & \nonumber\\
	=
	& \outputp{x}{\quotep{(\prefix{x}{y}{(\outputp{x}{y} | @{y})) | P}}}
	  | {(\prefix{x}{y}{(\outputp{x}{y} | @{y})) | P}} & \nonumber\\
	\red
	& \ldots & \nonumber\\
	\red^*
	& P | P | \ldots & \nonumber
\end{eqnarray}

Of course, this encoding, as an implementation, runs away, unfolding
$\bangp{P}$ eagerly. A lazier and more implementable replication
operator, restricted to input-guarded processes, may be obtained as follows.

\begin{eqnarray}
\bangp{\prefix{u}{v}{P}} 
	:= 
	\binpar{\lift{x}{\prefix{u}{v}{(\binpar{D(x)}{P})}}}{D(x)} \nonumber
\end{eqnarray}

\begin{remark}
  Note that the lazier definition still does not deal with summation
  or mixed summation (i.e. sums over input and output). The reader is
  invited to construct definitions of replication that deal with these
  features. 

  Further, the definitions are parameterized in a name, $x$. Can you,
  gentle reader, make a definition that eliminates this parameter and
  guarantees no accidental interaction between the replication
  machinery and the process being replicated -- i.e. no accidental
  sharing of names used by the process to get its work done and the
  name(s) used by the replication to effect copying. This latter
  revision of the definition of replication is crucial to obtaining
  the expected identity $!!P \sim !P$.
\end{remark}

\begin{remark}\label{rem:paradoxical_combinator}
  The reader familiar with the lambda calculus will have noticed the
  similarity between $D$ and the paradoxical combinator.

  [Ed. note: the existence of this seems to suggest we have to be more
  restrictive on the set of processes and names we admit if we are to
  support no-cloning.]
\end{remark}

\subsubsection{Bisimulation}

The computational dynamics gives rise to another kind of equivalence,
the equivalence of computational behavior. As previously mentioned
this is typically captured \emph{via} some form of bisimulation.

% The notion we use in this paper is weak barbed bisimulation
% \cite{milner91polyadicpi}.

The notion we use in this paper is derived from weak barbed
bisimulation \cite{milner91polyadicpi}. 

\begin{definition}
An \emph{observation relation}, $\downarrow_{\mathcal N}$, over a set
of names, $\mathcal N$, is the smallest relation satisfying the rules
below.

\infrule[Out-barb]{y \in {\mathcal N}, \; x \nameeq y}
		  {\outputp{x}{v} \downarrow_{\mathcal N} x}
\infrule[Par-barb]{\mbox{$P\downarrow_{\mathcal N} x$ or $Q\downarrow_{\mathcal N} x$}}
		  {\binpar{P}{Q} \downarrow_{\mathcal N} x}

We write $P \Downarrow_{\mathcal N} x$ if there is $Q$ such that 
$P \wred Q$ and $Q \downarrow_{\mathcal N} x$.
\end{definition}

\begin{definition}
%\label{def.bbisim}
An  ${\mathcal N}$-\emph{barbed bisimulation} over a set of names, ${\mathcal N}$, is a symmetric binary relation 
${\mathcal S}_{\mathcal N}$ between agents such that $P\rel{S}_{\mathcal N}Q$ implies:
\begin{enumerate}
\item If $P \red P'$ then $Q \wred Q'$ and $P'\rel{S}_{\mathcal N} Q'$.
\item If $P\downarrow_{\mathcal N} x$, then $Q\Downarrow_{\mathcal N} x$.
\end{enumerate}
$P$ is ${\mathcal N}$-barbed bisimilar to $Q$, written
$P \wbbisim_{\mathcal N} Q$, if $P \rel{S}_{\mathcal N} Q$ for some ${\mathcal N}$-barbed bisimulation ${\mathcal S}_{\mathcal N}$.
\end{definition}

$\mathcal{R} \subseteq \pi \times \pi$

$P \mathcal{R} Q => \forall P'. P \red P' \Rightarrow \exists Q'. Q \red Q', P' \mathcal{R} Q'$

$P \vdash x \Rightarrow Q \vdash x$

\begin{mathpar}
  \inferrule*[lab=Out-barb]{x \nameeq y}{{y}!\langle{Q}\rangle \vdash x}
  \and
  \inferrule*[lab=Par-barb]{\mbox{$P\vdash x$ or $Q\vdash x$}}{\binpar{P}{Q} \vdash x}
\end{mathpar}

\subsubsection{Contexts}

One of the principle advantages of computational calculi like the
$\pi$-calculus is a well-defined notion of context,
contextual-equivalence and a correlation between
contextual-equivalence and notions of bisimulation. The notion of
context allows the decomposition of a process into (sub-)process and
its syntactic environment, its context. Thus, a context may be
thought of as a process with a ``hole'' (written $\Box$) in it. The
application of a context $M$ to a process $P$, written $M[P]$, is
tantamount to filling the hole in $M$ with $P$. In this paper we do
not need the full weight of this theory, but do make use of the notion
of context in the proof the main theorem. 

\begin{mathpar}
  \inferrule* [lab=summation] {} {{M_{M},M_{N}} \bc \Box \;|\; x.M_{A} \;|\; M_{M}+M_{N}}
  \and
  \inferrule* [lab=agent] {} {{M_{A}} \bc (\vec{x})M_{P} \;| \; \clift{P_0,\ldots,M_{P},\ldots,P_N}}
  \and \\
  \inferrule* [lab=process] {} {{M_{P}} \bc M_{N} \;| \;P|M_{P} }
\end{mathpar} 

\begin{mathpar}
  \inferrule* [lab=sychronization] {} {M_{N} \bc \Box \;|\; x?M_{F} \;|\; x!M_{C}}
  \and
  \inferrule* [lab=abstraction] {} {{M_{F}} \bc (x)M_{P} }
  \and
  \inferrule* [lab=concretion] {} {{M_{C}} \bc \langle M_{P} \rangle }
  \and \\
  \inferrule* [lab=process] {} {{M_{P}} \bc M_{N} \;| \;P|M_{P} }
\end{mathpar}

\begin{definition}[contextual application] Given a context $M$, and
  process $P$, we define the \emph{contextual application}, $M[P] :=
  M\{P/\Box\}$. That is, the contextual application of M to P is the
  substitution of $P$ for $\Box$ in $M$.
\end{definition}

$\meaningof{-} : L \to \mathcal{P}(\pi)$

\begin{mathpar}
  \inferrule* [lab=collection] {} {\meaningof{true} = \pi, \and \meaningof{~E} = \pi \setminus \meaningof{E}, \and \meaningof{E_{1} \& E_{2}} = \meaningof{E_{1}} \cap \meaningof{E_{2}}}
\end{mathpar}

\begin{mathpar}
  \inferrule* [lab=structure] {} {\meaningof{0} = \{ P \in \pi | P \equiv 0 \}, \and \\ \meaningof{E_1 | E_2} = \{ P \in \pi | P \equiv P_{1} | P_{2}, P_{1} \in \meaningof{E_{1}}, P_{2} \in \meaningof{E_2}\} }
\end{mathpar}

\begin{mathpar}
 \inferrule* [lab=behavior] {} {\meaningof{\langle a?b \rangle E} = \{ P \in \pi | P \equiv Q | u?(y)P', \\ \and \\\\ \and \\ \;\;\; u \in \meaningof{a}, \forall z.P'\{z/y\} \in \meaningof{E\{z/b\}}\}, \and \\ \meaningof{a!E} = \{ P \in \pi | P \equiv Q | x!\langle P' \rangle, x \in \meaningof{a} P' \in \meaningof{E}\} }
\end{mathpar}

\begin{mathpar}
 \inferrule* [lab=nominal] {} {\meaningof{\quotep{E}} = \{ \quotep{P} \in \quotep{\pi} | P \in \meaningof{E} \}, \and \meaningof{\quotep{P}} = \{ \quotep{Q} \in \quotep{\pi} | P \equiv Q \} \and \\ \meaningof{@\quotep{E}} = \{ P \in \pi | P \equiv @x, x \in \meaningof{E} \}}
\end{mathpar}

\begin{eqnarray*}
  \\
  \meaningof{-} : TS \to ST
\end{eqnarray*}

\begin{eqnarray*}
  \\
  L : TS \to ST
\end{eqnarray*}

\begin{eqnarray*}
  \\
  P \models E \iff P \in \meaningof{E}
\end{eqnarray*}

\begin{eqnarray*}
  P \approx_{L} Q \iff \forall E \in L. P \models E \iff Q \models E
\end{eqnarray*}

\begin{eqnarray*}
  P \approx_{K} Q
\end{eqnarray*}

\begin{eqnarray*}
  P \approx Q
\end{eqnarray*}

$\approx_{K} = \approx = \approx_{L}$

\subsubsection{Contextual duality}

Note that contexts extend the quotation operation to a family of
operations from processes to names. Given a context, $M$, we can
define a \emph{nominal context}, $\quotep{M}$ by $\quotep{M}[P] :=
\quotep{M[P]}$. To foreshadow what is to come we observe that these
operations enjoy a duality with processes very much like the duality
between vectors and maps from vectors to scalars.

Further, because the calculus is essentially higher-order, we have a
correspondence between contexts and processes. More specifically,
given a name $x$ and a context $M$ we can construct $M^{*}_{x}$ such
that 

\begin{mathpar}
  M^{*}_{x} | \lift{x}{P} \red M[P]
\end{mathpar}

namely,

\begin{mathpar}
  M^{*}_{x} := x?(u).M[\dropn{u}]
\end{mathpar}

The dependence of $M^{*}_{x}$ on a name makes it an abstraction, 

\begin{mathpar}
  M^{*} := (x)x?(u).M[\dropn{u}]
\end{mathpar}

\subsection{Additional notation}

It will sometimes be convenient to denote the process a name
quotes. We already have the notation $x = \quotep{P}$, but it will be
convenient to introduce an alternate notation, $\procn{x}$, when we
want to emphasize the connection to the use of the name. Note that, by
virtue of name equivalence, $\quotep{\procn{x}} \nameeq x$; so, the
notation is consistent with previous definitions.

Further, because names have structure it is possible to effect
substitutions on the basis of that structure. This means we need to
upgrade our notation for substitutions, which we accomplish by
adapting comprehension notation. Thus,

\begin{mathpar}
  P\{ y / x : x \in S \}
\end{mathpar}

is interpreted to mean the process derived from P by replacing (in a
capture-avoiding manner) each occurrence of $x$ in $S$ by $y$. For example,

\begin{mathpar}
  P\{ \quotep{\procn{x}|\procn{x}} / x : x \in \freenames{P} \}
\end{mathpar}

will replace each (occurrence) of a free name $x$ in $P$ by
$\quotep{\procn{x}|\procn{x}}$.

Also, we will avail ourselves of the notation $x^{L}$ and $x^{R}$ to
denote injections of a name into disjoint copies of the name
space. There are numerous ways to accomplish this. One example can be
found in \cite{MeredithR05}. This notation overloads to vectors of
names: $\vec{x}^{\pi} := (x_{i}^{\pi} \; : \; 0 \leq i < |\vec{x}| )$ where $\pi \in \{L,R\}$.

We also use $P^{\Box} := P|\Box$.

In \cite{MeredithR05} an interpretation of the new operator is
given. It turns out that there are several possible interpretations
all enjoying the requisite algebraic properties of the operator (see
\cite{milner91polyadicpi}). We will therefore make liberal use of
$(\nu\; \vec{x})P$.

% subsection the_syntax_and_semantics_of_the_notation_system (end)   

\section{Interpretation of QM}
\subsection{Supporting definitions}
\subsubsection{Multiplication}
\begin{mathpar}
  \quotep{Q} \cdot \quotep{R} := \quotep{Q|R}
  \and \\
  \quotep{Q} \cdot P := P\{ \quotep{Q|R} / \quotep{R} : \quotep{R} \in \freenames{P} \}
\end{mathpar}

\paragraph{Discussion}
The first line needs little explanation. The second line says that
each free name of the process is replaced with the multiplication of
that name by the scalar. Multiplication of a scalar (name) by a state
(process) results in a process all the names of which have been `moved
over' by parallel composition with the process the scalar
quotes. There is a subtlety that the bound names have to be
manipulated so that multiplied names aren't accidentally
captured. There are many ways to achieve this.

\begin{remark}\label{rem:multiplication_identities}
  The reader is invited to verify that for all $x,y,z \in \QProc$ and $P \in \Proc$
  \begin{mathpar}
    x \cdot \quotep{0} \equiv x 
    \and
    x \cdot y \equiv y \cdot x
    \and
    x \cdot (y \cdot z) \equiv (x \cdot y) \cdot z
    \and \\
    \quotep{0} \cdot P \equiv P
    \and \\
    x \cdot (y \cdot P) \equiv (x \cdot y) \cdot P
    \and \\
    x \cdot (P|Q) \equiv (x \cdot P) | (x \cdot Q)
    \and \\    
  \end{mathpar}
\end{remark}

\subsubsection{Tensor product}

We define a tensor product on processes by structural induction.

\paragraph{Tensor of sums} First note that all summations, including
$\pzero$ and sequence, can be written $\Sigma_{i} x_{i}.A_{i} +
\Sigma_{j} x_{j}.C_{j}$, where we have grouped input-guarded processes
together and output-guarded processes together.

Thus, we can define the tensor product of two summations, $N_{1}\otimes N_{2}$, where

\begin{mathpar}
  N_{1} := \Sigma_{i} x_{i}.A_{i} + \Sigma_{j} x_{j}.C_{j}
  \and
  N_{2} := \Sigma_{i'} y_{i'}.B_{i'} + \Sigma_{j'} y_{j'}.D_{j'} 
\end{mathpar}

as follows.

\begin{mathpar}
  \Sigma_{i} x_{i}.A_{i} + \Sigma_{j} x_{j}.C_{j} \otimes \Sigma_{i'}
  y_{i'}.B_{i'} + \Sigma_{j'} y_{j'}.D_{j'} 
  \and \\
  := \; \Sigma_{i} \Sigma_{i'} \quotep{\stackrel{\vee}{x_{i}}| \stackrel{\vee}{y_{i'}}}.(A_{i}\otimes B_{i'}) \; | \; \Sigma_{i'} \Sigma_{i} \quotep{\stackrel{\vee}{y_{i'}}|\stackrel{\vee}{x_{i}}}.(B_{i'}\otimes A_{i})
  \and
  \;\; | \;\; \Sigma_{j} \Sigma_{j'} \quotep{\stackrel{\vee}{x_{j}}|\stackrel{\vee}{y_{j'}}}.(A_{j}\otimes B_{j'}) \; | \; \Sigma_{j'} \Sigma_{j} \quotep{\stackrel{\vee}{y_{j'}}|\stackrel{\vee}{x_{j}}}.(B_{j'}\otimes A_{j})
\end{mathpar}

\begin{remark}
  Do we need to $x^{L}$ and $y^{R}$ for this construction as well?
\end{remark}

\paragraph{Tensor of parallel compositions} Next, we distribute tensor
over par.

\begin{mathpar}
  P_{1}|P_{2} \otimes Q_{1}|Q_{2} := (P_{1} \otimes Q_{1}) | (P_{1}
  \otimes Q_{2}) | (P_{2} \otimes Q_{1}) | (P_{2} \otimes Q_{2})
\end{mathpar}

\paragraph{Tensor with dropped names} We treat tensor of a
process with a dropped name as parallel composition.

\begin{mathpar}
  P \otimes \dropn{x} := P | \dropn{x}
\end{mathpar}

\paragraph{Tensor of agents}

Finally, we need to define tensor on agents. Note that the definition
of tensor on normal products only tensors inputs with inputs and
outputs with outputs. Thus, we only have to define the operation on
``homogeneous'' pairings.

\begin{mathpar}
  (\vec{x})P \otimes (\vec{y})Q
  \and \\
  := (x_{0}^{L}|y_{0}^{R},\ldots,x_{0}^{L}|y_{n}^{R},\ldots,x_{m}^{L}|y_{0}^{R},\ldots,x_{m}^{L}|y_{n}^R)(P\{ \vec{x}^{L}/\vec{x}\} \otimes Q \{ \vec{y}^{R}/\vec{y}\})
  \and \\
  \clift{\vec{P}} \otimes \clift{\vec{Q}}
  \and \\
  := \clift{P_{0}\otimes Q_{0},\ldots,P_{0}\otimes Q_{n},\ldots,P_{m}\otimes Q_{0},\ldots,P_{m}\otimes Q_{n}}
\end{mathpar}

\begin{remark}
  Observe that arities of tensored abstractions matches arities of
  tensored concretions if the original arities matched. Note also that
  the length of the arities corresponds to the increase in dimension
  we see in ordinary vector space tensor product.
\end{remark}

\begin{remark}
  Operationally, this definition distributes the tensor down to
  components ``linked'' by summation. Tensor over summation is
  intriguing in that it mixes names. Moreover, as a consequence of the
  way it mixes names we have the identities for all $x \in \QProc$ and
  $P,Q \in \Proc$

  \begin{mathpar}
    (x \cdot P) \otimes Q \equiv x \cdot (P \otimes Q) \equiv P \otimes (x \cdot Q)
    \and
    P \otimes \pzero \equiv P
  \end{mathpar}

  that the reader is invited to verify.
\end{remark}

\subsubsection{Annihilation}
\begin{mathpar}
  P^{\perp} := \{ Q | \forall R. P|Q \red^{*} R \Rightarrow R \red^{*} \pzero \}
  \and \\
  P^{\underline{\perp}} := \Sigma_{Q \in P^{\perp}} \quotep{Q}?(y).(\dropn{y}|Q) | \Sigma_{Q \in P^{\perp}} \quotep{Q}\clift{\Box}
\end{mathpar}

\paragraph{Discussion} The reader will note that $P^{\perp}$ is a
\emph{set} of processes, while $P^{\underline{\perp}}$ is a
\emph{context}. We call the set $P^{\perp}$ the \emph{annihilators} of
$P$. The parallel composition of a process in the annihilators of $P$
with $P$ will result in a process, the state space of which has all
paths eventually leading to $\pzero$. Execution may endure loops; but
under reasonable conditions of fairness (naturally guaranteed under
most notions of bisimulation) such a composite process cannot get
stuck in such a loop and will, eventually pop out and terminate.

The context $P^{\underline{\perp}}$ is ready and willing to ``take the
$P$ out of'' the process to which it is applied. It will effectively
transmit the code of the process to which it is applied to one of the
annihilators and run the process against it.

\subsubsection{Evaluation}
We fix $M$ a domain of fully abstract interpretation with an equality
coincident with bisimulation. We take $\meaningof{\cdot} : \Proc \to
M$ to be the map interpreting processes and $\nmeaningof{\cdot} : \M
\to Proc$ to be the map running the other way. Then we define

\begin{mathpar}
  \int P := \nmeaningof{\meaningof{P}}
\end{mathpar}

\paragraph{Discussion}
There are many fully abstract interpretations of Milner's
$\pi$-calculus. Any of them can be used as a basis for interpreting
the reflective calculus here. Equipped with such a domain it is
largely a matter of grinding through to check that the Yoneda
construction for the normalization-by-evaluation program can be
extended to this setting.

\begin{remark}
  The reader is invited to verify that $\int (P^{\underline{\perp}}[P]) = 0$.
\end{remark}

\subsection{Quantum mechanics}

Table \ref{tbl:core_qm_op_defns} gives the core operational definitions

\begin{table}[htp]\label{tbl:core_qm_op_defns}
  \center{
    \fbox{
      \begin{tabular}{c|c}
        quantum mechanics & process calculus \\
        \hline
        scalar & $x := \quotep{P}$ \\
        state vector & $\state{P} := P$ \\
        dual & $\state{P}^{*} := \event{P^{\underline{\perp}}} := \quotep{P^{\underline{\perp}}}[-]$ \\
        matrix & $ \Sigma_{\alpha} \state{P_{\alpha}}x_{\alpha}\event{Q_{\alpha}}$ \\
        vector addition & $\state{P} + \state{Q} := \state{P | Q}$ \\
        tensor product & $\state{P} \otimes \state{Q} := \state{P \otimes Q}$ \\
        inner product & $\innerprod{P}{Q} := \quotep{\int P^{\underline{\perp}}[Q]}$ \\
      \end{tabular}
    }
  }
  \caption{QM - operational definitions}
\end{table}

where

\begin{mathpar}
  \prmatrix{P}{Q} := \fprmatrix{P}{\quotep{\pzero}}{Q}
  \and
  \fprmatrix{P}{x}{Q} := (\state{P},x,\event{Q})
  \and
  (\fprmatrix{P}{x}{Q})(\state{R}) := x \cdot \innerprod{Q}{R} \cdot \state{P}
  \and
  (\fprmatrix{P}{x}{Q})(\event{R}) := x \cdot \innerprod{R}{P} \cdot \event{Q}
\end{mathpar}

\paragraph{Discussion}
As promised: vectors (aka states) are represented as processes; duals
as contextual duals; inner product definition should be compared with
standard inner product definition for ....

\begin{remark}
  Assuming $\int (P^{\underline{\perp}}[P]) = 0$, the reader is
  invited to verify that $(\fprmatrix{P}{x}{P})(\state{P}) = x \cdot \state{P}$.
\end{remark}

\begin{remark}
  The reader is invited to verify that $\innerprod{P}{Q}$ could
  equally well have been written $\quotep{\int \stackrel{\vee}{x}}$
  where $x = \event{P^{\underline{\perp}}}(Q)$.

  One of the motivations for this remark is that there is another way
  to factor these operations. We could package up evaluation in the dual:

  \begin{mathpar}
    \state{P}^{*} := \event{\int P^{\underline{\perp}}} := \quotep{\int P^{\underline{\perp}}}[-]
  \end{mathpar}

  and then have inner product defined by
  
  \begin{mathpar}
    \innerprod{P}{Q} := \event{P}(Q)
  \end{mathpar}

  Hopefully, experience with the calculations will provide guidance on
  the best factoring.
\end{remark}

\begin{remark}
  Assuming $\int (P^{\underline{\perp}}[P]) = 0$, the reader is
  invited to verify that $\forall P,Q. (\prmatrix{0}{Q})(\state{0}) =
  \state{0}$ and dually $(\prmatrix{P}{0})(\event{0}) = \event{0}$.
\end{remark}

\begin{remark}
  i'm a little worried that i don't (yet) have proper support for
  complex conjugacy. But, the observation above may give us a
  clue. According to Abramsky, it must be the case that the scalars
  are iso to the homset of the identity for the tensor -- which the
  observation above characterizes. 

  For now, we will simply bookmark the notion with $\overline{x}$.
\end{remark}

\subsubsection{Adjointness}

We need to give a definition of $(\cdot)^{\dagger}$ for matrices. The
obvious candidate definition is
\begin{mathpar}
(\Sigma_{\alpha}\fprmatrix{P_{\alpha}}{x_{\alpha}}{Q_{\alpha}})^{\dagger}
= \Sigma_{\alpha}\fprmatrix{(Q_{\alpha}^{\underline{\perp}})^{*}}{\overline{x}_{\alpha}}{P_{\alpha}^{\underline{\perp}}} 
\end{mathpar}

But, $(Q_{\alpha}^{\underline{\perp}})^{*}$ requires a name along
which to communicate the process to achieve the context application.

\subsubsection{Basis for a basis}
If processes label states and ``addition'' of states (a.k.a. vector
addition) is interpreted as parallel composition, what corresponds to
notions of linear independence and basis? Here, we recall that Yoshida
has developed a set of \emph{combinators} for an asynchronous verison
of Milner's $\pi$-calculus. These are a finite set of processes such
any process can be expressed as parallel composition of these
combinators together with liberal uses of the new operator and
replication. We can simply give a translation of these into the
present calculus and have reasonable expectation that the property
carries over. That is, that the resultant set allows to express all
processes via parallel composition. Note, however, that there is no
new operator or replication in this calculus. As a result, we expect
that the corresponding set is actually infinite. That is, we expect
that the space is actually infinite dimensional.

\begin{remark}
  The attentive reader may be a bit concerned. Certainly, the
  collection $S$, $K$ and $I$ is a finite set of
  combinators. Shouldn't we expect to see a finite set of combinators
  for an effectively equivalent system? i am very sympathetic to this
  critique and feel it warrants full attention. On the other hand, i
  also have in mind the following analogy. The natural numbers, as a
  monoid under addition, has exactly $1$ generator, while the natural
  numbers, as a monoid under multiplication, has countably many
  generators (the primes). We observe that the application of the
  lambda calculus is much less resource sensitive than the parallel
  composition of the $\pi$-calculus. Could it be the case that we have
  an analogy of the form
  
  \begin{mathpar}
    m + n : MN :: m*n : M|N
  \end{mathpar}

  giving a similar blow up in the set of ``primes''?  This is such a
  wonderful thought that, even if it's not true, i think it's worth
  writing down.
\end{remark}
 

\documentclass[12pt]{llncs}
%\documentclass{jktr}

\usepackage[pdftex]{hyperref}                   
\usepackage {listings}
\usepackage {mathpartir}
\usepackage{bcprules}
%\usepackage{listings}
                       
\usepackage{graphicx} 
%\usepackage[margins=2.5cm,nohead,nofoot]{geometry}
%\usepackage{geometry}
\usepackage{amsfonts}
\usepackage{amstext}
\usepackage{latexsym}
\usepackage{amssymb}
\usepackage{color}


%\include{myPreamble}
\include{qm2pi.local} 

%\ifpdf
%\usepackage[pdftex]{graphicx}
%\else
%\usepackage{graphicx}
%\fi

 % \ifpdf
%  \usepackage{pdfsync}
%  \if


%\title{Brief Article}
%\author{David F. Snyder}
%\author{L.G. Meredith}

%\address{Dept. of Math., Texas State University--San Marcos, San Marcos, TX 78666}
       
\pagestyle{empty}


\begin{document}

\lstset{language=[Objective]Caml,frame=shadowbox}

\input{qm2pi.front}

% section front matter (end)

\input{qm2pi.intro} 
 
% section introduction (end)

% \input{qm2pi.knotations} 

% section notation (end)

\input{qm2pi.process.calculi} 

% section concurrent_process_calculi_and_spatial_logics_ (end)
    
%\input{qm2pi.knots2pi} 

%\input{qm2pi.trefoil} 

%\input{qm2pi.mainthm} 

% subsection basic_interpretation (end)

%\input{qm2pi.rho.presentation} 
\subsection{The syntax and semantics of the notation system}\label{sub:the_syntax_and_semantics_of_the_notation_system} % (fold)

We now summarize a technical presentation of the calculus that
embodies our theory of dynamics. The typical presentation of such a
calculus follows the style of giving generators and relations on
them. The grammar, below, describing term constructors, freely
generates the set of processes, $\Proc$. This set is then quotiented
by a relation known as structural congruence and it is over this set
that the notion of dynamics is expressed. This presentation is
essentially that of \cite{MeredithR05} with the addition of
polyadicity and summation. For readability we have relegated some of
the technical subtleties to an appendix.

\subsubsection{Process grammar}\label{subsub:process_grammar}

\begin{mathpar}
  \inferrule* [lab=synchronization] {} {{M} \bc \pzero \;|\; x?F \;|\; x!C }
  \and
  \inferrule* [lab=abstraction] {} {{F} \bc (x)P}
  \and
  \inferrule* [lab=concretion] {} {{C} \bc \langle Q \rangle}
  \and
  \inferrule* [lab=process] {} {{P,Q} \bc M \;| \;P|Q \;|\; @{x}}
  \and
  \inferrule* [lab=name] {} {{x} \bc \quotep{P}}
\end{mathpar} 

Note that $\vec{x}$ (resp. $\vec{P}$) denotes a vector of names
(resp. processes) of length $|\vec{x}|$ (resp. $|\vec{P}|$). We adopt
the following useful abbreviations.

\begin{mathpar}
   x?(\vec{y}).P := x.(\vec{y})P \and  x\clift{\vec{P}} := x.\clift{\vec{P}}
   \and x!(y) := \lift{x}{\dropn{y}}
   \and \Pi_{i=0}^{n-1}P_i := P_0 | \ldots | P_{n-1}
\end{mathpar}

\subsubsection{Structural congruence}

\paragraph{Free and bound names and alpha-equivalence.} At the
core of structural equivalence is alpha-equivalence which identifies
process that are the same up to a change of variable. Formally, we
recognize the distinction between free and bound names. The free names
of a process, $\freenames{P}$, may be calculated recursively as
follows:

\begin{mathpar}
\freenames{\pzero} := \emptyset
  \and \\
  \freenames{x?(y).P} := \{ x \} \cup (\freenames{P} \setminus \{ y \})
  \and 
  \freenames{x!\langle P \rangle} := \{ x \} \cup \{ P \} 
  \and \\
  \freenames{P|Q} := \freenames{P} \cup \freenames{Q}
  \and \\
  \freenames{@{x}} := \{ x \}
\end{mathpar}

$\pi$
$\quotep{\pi}$

$\freenames{-} : \pi \to \mathcal{P}(\quotep{\pi})$

\begin{eqnarray*}
  \freenames{\pzero} & := & \emptyset \\
  \freenames{x?(y).P} & := & \{ x \} \cup (\freenames{P} \setminus \{ y \}) \\
  \freenames{x!\langle P \rangle} & := & \{ x \} \cup \{ P \} \\
  \freenames{P|Q} & := & \freenames{P} \cup \freenames{Q} \\
  \freenames{\dropn{x}} & := & \{ x \}
\end{eqnarray*}

The bound names of a process, $\boundnames{P}$, are those names occurring in $P$
that are not free. For example, in $x?(y).0$, the name $x$ is free, while $y$ is bound.

\begin{mathpar}
  \inferrule* [lab=monoidal-laws] {} { P|Q \equiv Q|P \and P|0 \equiv P \and P|(Q|R) \equiv (P|Q)|R }
\end{mathpar}

\begin{mathpar}
  \inferrule* [lab=alpha-equivalence] {} { (x)P \equiv (y)P\{y/x\} \and y \not\in \freenames{P} }
\end{mathpar}

\begin{definition}
Then two processes, $P,Q$, are alpha-equivalent if $P = Q\{\vec{y}/\vec{x}\}$ for
some $\vec{x} \in \boundnames{Q},\vec{y} \in \boundnames{P}$, where $Q\{\vec{y}/\vec{x}\}$
denotes the capture-avoiding substitution of $\vec{y}$ for $\vec{x}$ in $Q$.
\end{definition}

\begin{definition}
  The {\em structural congruence} \cite{SangiorgiWalker} , $\equiv$,
  between processes is the least congruence containing
  alpha-equivalence, satisfying the abelian monoid laws
  (associativity, commutativity and $\pzero$ as identity) for parallel
  composition $|$ and for summation $+$.
\end{definition}

\subsection{Name equivalence}

We take name equivalence, written $\nameeq$, to be the smallest
equivalence relation generated by the following rules.

\begin{mathpar}
\inferrule*[lab=Quote-drop]
{ }
{ \quotep{@{x}} \nameeq x }

\inferrule*[lab=Struct-equiv]
{ P \scong Q }
{ \quotep{P} \nameeq \quotep{Q} }
\end{mathpar}

The astute reader will have noticed that the mutual recursion of names
and processes imposes a mutual recursion on alpha-equivalence and
structural equivalence via name-equivalence. Fortunately, all of this
works out pleasantly and we may calculate in the natural way, free of
concern. The reader interested in the details is referred to the
appendix \ref{appendix:rho_details}.

\subsection{Substitution}

We use $\Proc$ for the set of processes, $\QProc$ for the set of
names, and $\id{\{}\vec{y} / \vec{x} \id{\}}$ to denote partial maps,
$s : \QProc \rightarrow \QProc$. A map, $s$ lifts, uniquely, to a map
on process terms, $\widehat{s} : \Proc \rightarrow \Proc$ by the
following equations.

\begin{mathpar}
  (0) \psubstp{Q}{P} := 0 \\
  (R \juxtap S) \psubstp{Q}{P}
  :=    
  (R)\psubstp{Q}{P} \juxtap (S) \psubstp{Q}{P} \\
  (x?(y).R) \psubstp{Q}{P}    
  :=    
  (x)\substp{Q}{P} (z)\concat( (R \psubstn{z}{y}) \psubstp{Q}{P} ) \\
  (\lift{x}{R}) \psubstp{Q}{P}  
  :=
  \lift{(x)\substp{Q}{P}}{ R \psubstp{Q}{P} } \\
%   (\dropn{x})  \psubstp{Q}{P}       
%   := 
%   \left\{ 
%     \begin{array}{ccc} 
%       \dropn{\quotep{Q}} & & x \nameeq \quotep{P} \\
%       \dropn{x} & & otherwise \\
%     \end{array}
%   \right. 
  (\dropn{x})  \psubstp{Q}{P}       
  := 
  \left\{ 
    \begin{array}{ccc} 
      Q & & x \nameeq \quotep{P} \\
      \dropn{x} & & otherwise \\
    \end{array}
  \right.
\end{mathpar}
 

where

\begin{eqnarray}
  (x)\id{\{} \lpquote Q \rpquote / \lpquote P \rpquote \id{\}}            = 
  \left\{ 
    \begin{array}{ccc}
      \lpquote Q \rpquote & & x \nameeq \lpquote P \rpquote \\
      x & & otherwise \\
    \end{array}
  \right. \nonumber
\end{eqnarray}

and $z$ is chosen distinct from $\quotep{P}$, $\quotep{Q}$, the free
names in $Q$, and all the names in $R$. Our $\alpha$-equivalence will
be built in the standard way from this substitution.

\begin{remark}\label{rem:no_self_referential_names}
  One consequence of these definitions is that $\forall P. \quotep{P}
  \not\in \freenames{P}$.
\end{remark}

\subsection{ Dynamic quote: an example }

Anticipating something of what's to come, consider applying the
substitution, $\widehat{\id{\{}u / z \id{\}}}$, to the following pair
of processes, $\lift{w}{y!(z)}$ and $w[ \lpquote y!(z) \rpquote ]$.

\begin{eqnarray}
	\lift{w}{y!(z)}\widehat{\id{\{}u / z \id{\}}}
		& = &
		\lift{w}{y!(u)} \nonumber\\
	w[ \lpquote y!(z) \rpquote ] \widehat{ \id{\{}u / z \id{\}} }
		& = &
		w[ \lpquote y!(z) \rpquote ] \nonumber
\end{eqnarray}

Because the body of the process between quotes is impervious to
substitution, we get radically different answers. In fact, by
examining the first process in an input context,
e.g. $x?(z).\lift{w}{y!(z)}$, we see that the process under the lift
operator may be shaped by prefixed inputs binding a name inside it. In
this sense, the lift operator will be seen as a way to dynamically
construct processes before reifying them as names.

Finally equipped with these standard features we can present the
dynamics of the calculus.

\subsubsection{Operational semantics} 

Finally, we introduce the computational dynamics. What marks these
algebras as distinct from other more traditionally studied algebraic
structures, e.g. vector spaces or polynomial rings, is the manner in
which dynamics is captured. In traditional structures, dynamics is typically
expressed through morphisms between such structures, as in linear maps
between vector spaces or morphisms between rings. In algebras
associated with the semantics of computation, the dynamics is
expressed as part of the algebraic structure itself, through a
reduction reduction relation typically denoted by $\red$. Below, we
give a recursive presentation of this relation for the calculus used
in the encoding.

$\red \subseteq \pi \times \pi$
$\red : \pi \to \mathcal{P}(\pi)$

\begin{mathpar}
  \inferrule* [lab=Comm] { \textsf{match}( x_{src}, x_{trgt} ) } { x_{trgt}?(y)P \; | \; x_{src}!\langle {Q} \rangle \red P\{\quotep{Q}/y}\} }
  \and \\
  \inferrule* [lab=Par] {{P} \red {P}'} {{{P} | {Q}} \red {{P}' | {Q}}}
  \and
  \inferrule* [lab=Equiv]{{{P} \scong {P}'} \andalso {{P}' \red {Q}'} \andalso {{Q}' \scong {Q}}}{{P} \red {Q}}
\end{mathpar}

\begin{eqnarray*}
  match_{\equiv} (\quotep{P},\quotep{Q}) & := & P \equiv Q \\
  match_{\dagger}(\quotep{P},\quotep{Q}) & := & \forall R. P|Q \red^{*} R => R \red^{*} 0 \\
  match_{K}(\quotep{P},\quotep{Q}) & := & K \mbox{ for some context } K
\end{eqnarray*}

$u?(x)P | u!\langle Q \rangle \red P\{\quotep{Q}/x\}$

%We write $\wred$ for $\red^*$, and $P\red$ if $\exists Q $ such that $ P \red Q$.
We write $P\red$ if $\exists Q $ such that $ P \red Q$ and $P\not\red$, otherwise.

\section{Replication}

As mentioned before, it is known that replication (and hence
recursion) can be implemented in a higher-order process algebra
\cite{SangiorgiWalker}. As our first example of calculation with the
machinery thus far presented we give the construction explicitly in
the {\rhoc}.

\begin{eqnarray}
	D_{x} & := & \prefix{x}{y}{(\binpar{\outputp{x}{y}}{@{y}})} \nonumber\\
	\bangp_{x}{P} & := & \binpar{{x}!\langle{\binpar{D_{x}}{P}}\rangle}{D_{x}} \nonumber
\end{eqnarray}

\begin{eqnarray}
	\bangp_{x}{P} & & \nonumber\\
	=
	& {x}!\langle{(\prefix{x}{y}{(\outputp{x}{y} | @{y})) | P}}\rangle 
	      | \prefix{x}{y}{(\outputp{x}{y} | @{y})} & \nonumber\\
	\red
	& (\outputp{x}{y} | @{y})\substn{\quotep{(\prefix{x}{y}{(@{y} | \outputp{x}{y})) | P}}}{y} & \nonumber\\
	=
	& \outputp{x}{\quotep{(\prefix{x}{y}{(\outputp{x}{y} | @{y})) | P}}}
	  | {(\prefix{x}{y}{(\outputp{x}{y} | @{y})) | P}} & \nonumber\\
	\red
	& \ldots & \nonumber\\
	\red^*
	& P | P | \ldots & \nonumber
\end{eqnarray}

Of course, this encoding, as an implementation, runs away, unfolding
$\bangp{P}$ eagerly. A lazier and more implementable replication
operator, restricted to input-guarded processes, may be obtained as follows.

\begin{eqnarray}
\bangp{\prefix{u}{v}{P}} 
	:= 
	\binpar{\lift{x}{\prefix{u}{v}{(\binpar{D(x)}{P})}}}{D(x)} \nonumber
\end{eqnarray}

\begin{remark}
  Note that the lazier definition still does not deal with summation
  or mixed summation (i.e. sums over input and output). The reader is
  invited to construct definitions of replication that deal with these
  features. 

  Further, the definitions are parameterized in a name, $x$. Can you,
  gentle reader, make a definition that eliminates this parameter and
  guarantees no accidental interaction between the replication
  machinery and the process being replicated -- i.e. no accidental
  sharing of names used by the process to get its work done and the
  name(s) used by the replication to effect copying. This latter
  revision of the definition of replication is crucial to obtaining
  the expected identity $!!P \sim !P$.
\end{remark}

\begin{remark}\label{rem:paradoxical_combinator}
  The reader familiar with the lambda calculus will have noticed the
  similarity between $D$ and the paradoxical combinator.

  [Ed. note: the existence of this seems to suggest we have to be more
  restrictive on the set of processes and names we admit if we are to
  support no-cloning.]
\end{remark}

\subsubsection{Bisimulation}

The computational dynamics gives rise to another kind of equivalence,
the equivalence of computational behavior. As previously mentioned
this is typically captured \emph{via} some form of bisimulation.

% The notion we use in this paper is weak barbed bisimulation
% \cite{milner91polyadicpi}.

The notion we use in this paper is derived from weak barbed
bisimulation \cite{milner91polyadicpi}. 

\begin{definition}
An \emph{observation relation}, $\downarrow_{\mathcal N}$, over a set
of names, $\mathcal N$, is the smallest relation satisfying the rules
below.

\infrule[Out-barb]{y \in {\mathcal N}, \; x \nameeq y}
		  {\outputp{x}{v} \downarrow_{\mathcal N} x}
\infrule[Par-barb]{\mbox{$P\downarrow_{\mathcal N} x$ or $Q\downarrow_{\mathcal N} x$}}
		  {\binpar{P}{Q} \downarrow_{\mathcal N} x}

We write $P \Downarrow_{\mathcal N} x$ if there is $Q$ such that 
$P \wred Q$ and $Q \downarrow_{\mathcal N} x$.
\end{definition}

\begin{definition}
%\label{def.bbisim}
An  ${\mathcal N}$-\emph{barbed bisimulation} over a set of names, ${\mathcal N}$, is a symmetric binary relation 
${\mathcal S}_{\mathcal N}$ between agents such that $P\rel{S}_{\mathcal N}Q$ implies:
\begin{enumerate}
\item If $P \red P'$ then $Q \wred Q'$ and $P'\rel{S}_{\mathcal N} Q'$.
\item If $P\downarrow_{\mathcal N} x$, then $Q\Downarrow_{\mathcal N} x$.
\end{enumerate}
$P$ is ${\mathcal N}$-barbed bisimilar to $Q$, written
$P \wbbisim_{\mathcal N} Q$, if $P \rel{S}_{\mathcal N} Q$ for some ${\mathcal N}$-barbed bisimulation ${\mathcal S}_{\mathcal N}$.
\end{definition}

$\mathcal{R} \subseteq \pi \times \pi$

$P \mathcal{R} Q => \forall P'. P \red P' \Rightarrow \exists Q'. Q \red Q', P' \mathcal{R} Q'$

$P \vdash x \Rightarrow Q \vdash x$

\begin{mathpar}
  \inferrule*[lab=Out-barb]{x \nameeq y}{{y}!\langle{Q}\rangle \vdash x}
  \and
  \inferrule*[lab=Par-barb]{\mbox{$P\vdash x$ or $Q\vdash x$}}{\binpar{P}{Q} \vdash x}
\end{mathpar}

\subsubsection{Contexts}

One of the principle advantages of computational calculi like the
$\pi$-calculus is a well-defined notion of context,
contextual-equivalence and a correlation between
contextual-equivalence and notions of bisimulation. The notion of
context allows the decomposition of a process into (sub-)process and
its syntactic environment, its context. Thus, a context may be
thought of as a process with a ``hole'' (written $\Box$) in it. The
application of a context $M$ to a process $P$, written $M[P]$, is
tantamount to filling the hole in $M$ with $P$. In this paper we do
not need the full weight of this theory, but do make use of the notion
of context in the proof the main theorem. 

\begin{mathpar}
  \inferrule* [lab=summation] {} {{M_{M},M_{N}} \bc \Box \;|\; x.M_{A} \;|\; M_{M}+M_{N}}
  \and
  \inferrule* [lab=agent] {} {{M_{A}} \bc (\vec{x})M_{P} \;| \; \clift{P_0,\ldots,M_{P},\ldots,P_N}}
  \and \\
  \inferrule* [lab=process] {} {{M_{P}} \bc M_{N} \;| \;P|M_{P} }
\end{mathpar} 

\begin{mathpar}
  \inferrule* [lab=sychronization] {} {M_{N} \bc \Box \;|\; x?M_{F} \;|\; x!M_{C}}
  \and
  \inferrule* [lab=abstraction] {} {{M_{F}} \bc (x)M_{P} }
  \and
  \inferrule* [lab=concretion] {} {{M_{C}} \bc \langle M_{P} \rangle }
  \and \\
  \inferrule* [lab=process] {} {{M_{P}} \bc M_{N} \;| \;P|M_{P} }
\end{mathpar}

\begin{definition}[contextual application] Given a context $M$, and
  process $P$, we define the \emph{contextual application}, $M[P] :=
  M\{P/\Box\}$. That is, the contextual application of M to P is the
  substitution of $P$ for $\Box$ in $M$.
\end{definition}

$\meaningof{-} : L \to \mathcal{P}(\pi)$

\begin{mathpar}
  \inferrule* [lab=collection] {} {\meaningof{true} = \pi, \and \meaningof{~E} = \pi \setminus \meaningof{E}, \and \meaningof{E_{1} \& E_{2}} = \meaningof{E_{1}} \cap \meaningof{E_{2}}}
\end{mathpar}

\begin{mathpar}
  \inferrule* [lab=structure] {} {\meaningof{0} = \{ P \in \pi | P \equiv 0 \}, \and \\ \meaningof{E_1 | E_2} = \{ P \in \pi | P \equiv P_{1} | P_{2}, P_{1} \in \meaningof{E_{1}}, P_{2} \in \meaningof{E_2}\} }
\end{mathpar}

\begin{mathpar}
 \inferrule* [lab=behavior] {} {\meaningof{\langle a?b \rangle E} = \{ P \in \pi | P \equiv Q | u?(y)P', \\ \and \\\\ \and \\ \;\;\; u \in \meaningof{a}, \forall z.P'\{z/y\} \in \meaningof{E\{z/b\}}\}, \and \\ \meaningof{a!E} = \{ P \in \pi | P \equiv Q | x!\langle P' \rangle, x \in \meaningof{a} P' \in \meaningof{E}\} }
\end{mathpar}

\begin{mathpar}
 \inferrule* [lab=nominal] {} {\meaningof{\quotep{E}} = \{ \quotep{P} \in \quotep{\pi} | P \in \meaningof{E} \}, \and \meaningof{\quotep{P}} = \{ \quotep{Q} \in \quotep{\pi} | P \equiv Q \} \and \\ \meaningof{@\quotep{E}} = \{ P \in \pi | P \equiv @x, x \in \meaningof{E} \}}
\end{mathpar}

\begin{eqnarray*}
  \\
  \meaningof{-} : TS \to ST
\end{eqnarray*}

\begin{eqnarray*}
  \\
  L : TS \to ST
\end{eqnarray*}

\begin{eqnarray*}
  \\
  P \models E \iff P \in \meaningof{E}
\end{eqnarray*}

\begin{eqnarray*}
  P \approx_{L} Q \iff \forall E \in L. P \models E \iff Q \models E
\end{eqnarray*}

\begin{eqnarray*}
  P \approx_{K} Q
\end{eqnarray*}

\begin{eqnarray*}
  P \approx Q
\end{eqnarray*}

$\approx_{K} = \approx = \approx_{L}$

\subsubsection{Contextual duality}

Note that contexts extend the quotation operation to a family of
operations from processes to names. Given a context, $M$, we can
define a \emph{nominal context}, $\quotep{M}$ by $\quotep{M}[P] :=
\quotep{M[P]}$. To foreshadow what is to come we observe that these
operations enjoy a duality with processes very much like the duality
between vectors and maps from vectors to scalars.

Further, because the calculus is essentially higher-order, we have a
correspondence between contexts and processes. More specifically,
given a name $x$ and a context $M$ we can construct $M^{*}_{x}$ such
that 

\begin{mathpar}
  M^{*}_{x} | \lift{x}{P} \red M[P]
\end{mathpar}

namely,

\begin{mathpar}
  M^{*}_{x} := x?(u).M[\dropn{u}]
\end{mathpar}

The dependence of $M^{*}_{x}$ on a name makes it an abstraction, 

\begin{mathpar}
  M^{*} := (x)x?(u).M[\dropn{u}]
\end{mathpar}

\subsection{Additional notation}

It will sometimes be convenient to denote the process a name
quotes. We already have the notation $x = \quotep{P}$, but it will be
convenient to introduce an alternate notation, $\procn{x}$, when we
want to emphasize the connection to the use of the name. Note that, by
virtue of name equivalence, $\quotep{\procn{x}} \nameeq x$; so, the
notation is consistent with previous definitions.

Further, because names have structure it is possible to effect
substitutions on the basis of that structure. This means we need to
upgrade our notation for substitutions, which we accomplish by
adapting comprehension notation. Thus,

\begin{mathpar}
  P\{ y / x : x \in S \}
\end{mathpar}

is interpreted to mean the process derived from P by replacing (in a
capture-avoiding manner) each occurrence of $x$ in $S$ by $y$. For example,

\begin{mathpar}
  P\{ \quotep{\procn{x}|\procn{x}} / x : x \in \freenames{P} \}
\end{mathpar}

will replace each (occurrence) of a free name $x$ in $P$ by
$\quotep{\procn{x}|\procn{x}}$.

Also, we will avail ourselves of the notation $x^{L}$ and $x^{R}$ to
denote injections of a name into disjoint copies of the name
space. There are numerous ways to accomplish this. One example can be
found in \cite{MeredithR05}. This notation overloads to vectors of
names: $\vec{x}^{\pi} := (x_{i}^{\pi} \; : \; 0 \leq i < |\vec{x}| )$ where $\pi \in \{L,R\}$.

We also use $P^{\Box} := P|\Box$.

In \cite{MeredithR05} an interpretation of the new operator is
given. It turns out that there are several possible interpretations
all enjoying the requisite algebraic properties of the operator (see
\cite{milner91polyadicpi}). We will therefore make liberal use of
$(\nu\; \vec{x})P$.

% subsection the_syntax_and_semantics_of_the_notation_system (end)   

\input{qm2pi.qmops} 

\input{qm2pi.sterngerlach} 

\input{qm2pi.metric} 

% section concurrent_process_calculi (end)

%\input{qm2pi.proofsketch}

% section proof sketch (end)

%\input{qm2pi.slviaknots} 

% section spatial logic via knots (end)

\input{qm2pi.conclusion}

% section conclusion (end)

%\input{qm2pi.dtcodes} 

% section wiring algorithm (end)

\input{qm2pi.ack} 

% section acknowledgments (end)

\newpage


\bibliographystyle{plain}   
\bibliography{../../biblios/main.bib}

\input{qm2pi.rhodetails}

\end{document}

 

\documentclass[12pt]{llncs}
%\documentclass{jktr}

\usepackage[pdftex]{hyperref}                   
\usepackage {listings}
\usepackage {mathpartir}
\usepackage{bcprules}
%\usepackage{listings}
                       
\usepackage{graphicx} 
%\usepackage[margins=2.5cm,nohead,nofoot]{geometry}
%\usepackage{geometry}
\usepackage{amsfonts}
\usepackage{amstext}
\usepackage{latexsym}
\usepackage{amssymb}
\usepackage{color}


%\include{myPreamble}
\include{qm2pi.local} 

%\ifpdf
%\usepackage[pdftex]{graphicx}
%\else
%\usepackage{graphicx}
%\fi

 % \ifpdf
%  \usepackage{pdfsync}
%  \if


%\title{Brief Article}
%\author{David F. Snyder}
%\author{L.G. Meredith}

%\address{Dept. of Math., Texas State University--San Marcos, San Marcos, TX 78666}
       
\pagestyle{empty}


\begin{document}

\lstset{language=[Objective]Caml,frame=shadowbox}

\input{qm2pi.front}

% section front matter (end)

\input{qm2pi.intro} 
 
% section introduction (end)

% \input{qm2pi.knotations} 

% section notation (end)

\input{qm2pi.process.calculi} 

% section concurrent_process_calculi_and_spatial_logics_ (end)
    
%\input{qm2pi.knots2pi} 

%\input{qm2pi.trefoil} 

%\input{qm2pi.mainthm} 

% subsection basic_interpretation (end)

%\input{qm2pi.rho.presentation} 
\subsection{The syntax and semantics of the notation system}\label{sub:the_syntax_and_semantics_of_the_notation_system} % (fold)

We now summarize a technical presentation of the calculus that
embodies our theory of dynamics. The typical presentation of such a
calculus follows the style of giving generators and relations on
them. The grammar, below, describing term constructors, freely
generates the set of processes, $\Proc$. This set is then quotiented
by a relation known as structural congruence and it is over this set
that the notion of dynamics is expressed. This presentation is
essentially that of \cite{MeredithR05} with the addition of
polyadicity and summation. For readability we have relegated some of
the technical subtleties to an appendix.

\subsubsection{Process grammar}\label{subsub:process_grammar}

\begin{mathpar}
  \inferrule* [lab=synchronization] {} {{M} \bc \pzero \;|\; x?F \;|\; x!C }
  \and
  \inferrule* [lab=abstraction] {} {{F} \bc (x)P}
  \and
  \inferrule* [lab=concretion] {} {{C} \bc \langle Q \rangle}
  \and
  \inferrule* [lab=process] {} {{P,Q} \bc M \;| \;P|Q \;|\; @{x}}
  \and
  \inferrule* [lab=name] {} {{x} \bc \quotep{P}}
\end{mathpar} 

Note that $\vec{x}$ (resp. $\vec{P}$) denotes a vector of names
(resp. processes) of length $|\vec{x}|$ (resp. $|\vec{P}|$). We adopt
the following useful abbreviations.

\begin{mathpar}
   x?(\vec{y}).P := x.(\vec{y})P \and  x\clift{\vec{P}} := x.\clift{\vec{P}}
   \and x!(y) := \lift{x}{\dropn{y}}
   \and \Pi_{i=0}^{n-1}P_i := P_0 | \ldots | P_{n-1}
\end{mathpar}

\subsubsection{Structural congruence}

\paragraph{Free and bound names and alpha-equivalence.} At the
core of structural equivalence is alpha-equivalence which identifies
process that are the same up to a change of variable. Formally, we
recognize the distinction between free and bound names. The free names
of a process, $\freenames{P}$, may be calculated recursively as
follows:

\begin{mathpar}
\freenames{\pzero} := \emptyset
  \and \\
  \freenames{x?(y).P} := \{ x \} \cup (\freenames{P} \setminus \{ y \})
  \and 
  \freenames{x!\langle P \rangle} := \{ x \} \cup \{ P \} 
  \and \\
  \freenames{P|Q} := \freenames{P} \cup \freenames{Q}
  \and \\
  \freenames{@{x}} := \{ x \}
\end{mathpar}

$\pi$
$\quotep{\pi}$

$\freenames{-} : \pi \to \mathcal{P}(\quotep{\pi})$

\begin{eqnarray*}
  \freenames{\pzero} & := & \emptyset \\
  \freenames{x?(y).P} & := & \{ x \} \cup (\freenames{P} \setminus \{ y \}) \\
  \freenames{x!\langle P \rangle} & := & \{ x \} \cup \{ P \} \\
  \freenames{P|Q} & := & \freenames{P} \cup \freenames{Q} \\
  \freenames{\dropn{x}} & := & \{ x \}
\end{eqnarray*}

The bound names of a process, $\boundnames{P}$, are those names occurring in $P$
that are not free. For example, in $x?(y).0$, the name $x$ is free, while $y$ is bound.

\begin{mathpar}
  \inferrule* [lab=monoidal-laws] {} { P|Q \equiv Q|P \and P|0 \equiv P \and P|(Q|R) \equiv (P|Q)|R }
\end{mathpar}

\begin{mathpar}
  \inferrule* [lab=alpha-equivalence] {} { (x)P \equiv (y)P\{y/x\} \and y \not\in \freenames{P} }
\end{mathpar}

\begin{definition}
Then two processes, $P,Q$, are alpha-equivalent if $P = Q\{\vec{y}/\vec{x}\}$ for
some $\vec{x} \in \boundnames{Q},\vec{y} \in \boundnames{P}$, where $Q\{\vec{y}/\vec{x}\}$
denotes the capture-avoiding substitution of $\vec{y}$ for $\vec{x}$ in $Q$.
\end{definition}

\begin{definition}
  The {\em structural congruence} \cite{SangiorgiWalker} , $\equiv$,
  between processes is the least congruence containing
  alpha-equivalence, satisfying the abelian monoid laws
  (associativity, commutativity and $\pzero$ as identity) for parallel
  composition $|$ and for summation $+$.
\end{definition}

\subsection{Name equivalence}

We take name equivalence, written $\nameeq$, to be the smallest
equivalence relation generated by the following rules.

\begin{mathpar}
\inferrule*[lab=Quote-drop]
{ }
{ \quotep{@{x}} \nameeq x }

\inferrule*[lab=Struct-equiv]
{ P \scong Q }
{ \quotep{P} \nameeq \quotep{Q} }
\end{mathpar}

The astute reader will have noticed that the mutual recursion of names
and processes imposes a mutual recursion on alpha-equivalence and
structural equivalence via name-equivalence. Fortunately, all of this
works out pleasantly and we may calculate in the natural way, free of
concern. The reader interested in the details is referred to the
appendix \ref{appendix:rho_details}.

\subsection{Substitution}

We use $\Proc$ for the set of processes, $\QProc$ for the set of
names, and $\id{\{}\vec{y} / \vec{x} \id{\}}$ to denote partial maps,
$s : \QProc \rightarrow \QProc$. A map, $s$ lifts, uniquely, to a map
on process terms, $\widehat{s} : \Proc \rightarrow \Proc$ by the
following equations.

\begin{mathpar}
  (0) \psubstp{Q}{P} := 0 \\
  (R \juxtap S) \psubstp{Q}{P}
  :=    
  (R)\psubstp{Q}{P} \juxtap (S) \psubstp{Q}{P} \\
  (x?(y).R) \psubstp{Q}{P}    
  :=    
  (x)\substp{Q}{P} (z)\concat( (R \psubstn{z}{y}) \psubstp{Q}{P} ) \\
  (\lift{x}{R}) \psubstp{Q}{P}  
  :=
  \lift{(x)\substp{Q}{P}}{ R \psubstp{Q}{P} } \\
%   (\dropn{x})  \psubstp{Q}{P}       
%   := 
%   \left\{ 
%     \begin{array}{ccc} 
%       \dropn{\quotep{Q}} & & x \nameeq \quotep{P} \\
%       \dropn{x} & & otherwise \\
%     \end{array}
%   \right. 
  (\dropn{x})  \psubstp{Q}{P}       
  := 
  \left\{ 
    \begin{array}{ccc} 
      Q & & x \nameeq \quotep{P} \\
      \dropn{x} & & otherwise \\
    \end{array}
  \right.
\end{mathpar}
 

where

\begin{eqnarray}
  (x)\id{\{} \lpquote Q \rpquote / \lpquote P \rpquote \id{\}}            = 
  \left\{ 
    \begin{array}{ccc}
      \lpquote Q \rpquote & & x \nameeq \lpquote P \rpquote \\
      x & & otherwise \\
    \end{array}
  \right. \nonumber
\end{eqnarray}

and $z$ is chosen distinct from $\quotep{P}$, $\quotep{Q}$, the free
names in $Q$, and all the names in $R$. Our $\alpha$-equivalence will
be built in the standard way from this substitution.

\begin{remark}\label{rem:no_self_referential_names}
  One consequence of these definitions is that $\forall P. \quotep{P}
  \not\in \freenames{P}$.
\end{remark}

\subsection{ Dynamic quote: an example }

Anticipating something of what's to come, consider applying the
substitution, $\widehat{\id{\{}u / z \id{\}}}$, to the following pair
of processes, $\lift{w}{y!(z)}$ and $w[ \lpquote y!(z) \rpquote ]$.

\begin{eqnarray}
	\lift{w}{y!(z)}\widehat{\id{\{}u / z \id{\}}}
		& = &
		\lift{w}{y!(u)} \nonumber\\
	w[ \lpquote y!(z) \rpquote ] \widehat{ \id{\{}u / z \id{\}} }
		& = &
		w[ \lpquote y!(z) \rpquote ] \nonumber
\end{eqnarray}

Because the body of the process between quotes is impervious to
substitution, we get radically different answers. In fact, by
examining the first process in an input context,
e.g. $x?(z).\lift{w}{y!(z)}$, we see that the process under the lift
operator may be shaped by prefixed inputs binding a name inside it. In
this sense, the lift operator will be seen as a way to dynamically
construct processes before reifying them as names.

Finally equipped with these standard features we can present the
dynamics of the calculus.

\subsubsection{Operational semantics} 

Finally, we introduce the computational dynamics. What marks these
algebras as distinct from other more traditionally studied algebraic
structures, e.g. vector spaces or polynomial rings, is the manner in
which dynamics is captured. In traditional structures, dynamics is typically
expressed through morphisms between such structures, as in linear maps
between vector spaces or morphisms between rings. In algebras
associated with the semantics of computation, the dynamics is
expressed as part of the algebraic structure itself, through a
reduction reduction relation typically denoted by $\red$. Below, we
give a recursive presentation of this relation for the calculus used
in the encoding.

$\red \subseteq \pi \times \pi$
$\red : \pi \to \mathcal{P}(\pi)$

\begin{mathpar}
  \inferrule* [lab=Comm] { \textsf{match}( x_{src}, x_{trgt} ) } { x_{trgt}?(y)P \; | \; x_{src}!\langle {Q} \rangle \red P\{\quotep{Q}/y}\} }
  \and \\
  \inferrule* [lab=Par] {{P} \red {P}'} {{{P} | {Q}} \red {{P}' | {Q}}}
  \and
  \inferrule* [lab=Equiv]{{{P} \scong {P}'} \andalso {{P}' \red {Q}'} \andalso {{Q}' \scong {Q}}}{{P} \red {Q}}
\end{mathpar}

\begin{eqnarray*}
  match_{\equiv} (\quotep{P},\quotep{Q}) & := & P \equiv Q \\
  match_{\dagger}(\quotep{P},\quotep{Q}) & := & \forall R. P|Q \red^{*} R => R \red^{*} 0 \\
  match_{K}(\quotep{P},\quotep{Q}) & := & K \mbox{ for some context } K
\end{eqnarray*}

$u?(x)P | u!\langle Q \rangle \red P\{\quotep{Q}/x\}$

%We write $\wred$ for $\red^*$, and $P\red$ if $\exists Q $ such that $ P \red Q$.
We write $P\red$ if $\exists Q $ such that $ P \red Q$ and $P\not\red$, otherwise.

\section{Replication}

As mentioned before, it is known that replication (and hence
recursion) can be implemented in a higher-order process algebra
\cite{SangiorgiWalker}. As our first example of calculation with the
machinery thus far presented we give the construction explicitly in
the {\rhoc}.

\begin{eqnarray}
	D_{x} & := & \prefix{x}{y}{(\binpar{\outputp{x}{y}}{@{y}})} \nonumber\\
	\bangp_{x}{P} & := & \binpar{{x}!\langle{\binpar{D_{x}}{P}}\rangle}{D_{x}} \nonumber
\end{eqnarray}

\begin{eqnarray}
	\bangp_{x}{P} & & \nonumber\\
	=
	& {x}!\langle{(\prefix{x}{y}{(\outputp{x}{y} | @{y})) | P}}\rangle 
	      | \prefix{x}{y}{(\outputp{x}{y} | @{y})} & \nonumber\\
	\red
	& (\outputp{x}{y} | @{y})\substn{\quotep{(\prefix{x}{y}{(@{y} | \outputp{x}{y})) | P}}}{y} & \nonumber\\
	=
	& \outputp{x}{\quotep{(\prefix{x}{y}{(\outputp{x}{y} | @{y})) | P}}}
	  | {(\prefix{x}{y}{(\outputp{x}{y} | @{y})) | P}} & \nonumber\\
	\red
	& \ldots & \nonumber\\
	\red^*
	& P | P | \ldots & \nonumber
\end{eqnarray}

Of course, this encoding, as an implementation, runs away, unfolding
$\bangp{P}$ eagerly. A lazier and more implementable replication
operator, restricted to input-guarded processes, may be obtained as follows.

\begin{eqnarray}
\bangp{\prefix{u}{v}{P}} 
	:= 
	\binpar{\lift{x}{\prefix{u}{v}{(\binpar{D(x)}{P})}}}{D(x)} \nonumber
\end{eqnarray}

\begin{remark}
  Note that the lazier definition still does not deal with summation
  or mixed summation (i.e. sums over input and output). The reader is
  invited to construct definitions of replication that deal with these
  features. 

  Further, the definitions are parameterized in a name, $x$. Can you,
  gentle reader, make a definition that eliminates this parameter and
  guarantees no accidental interaction between the replication
  machinery and the process being replicated -- i.e. no accidental
  sharing of names used by the process to get its work done and the
  name(s) used by the replication to effect copying. This latter
  revision of the definition of replication is crucial to obtaining
  the expected identity $!!P \sim !P$.
\end{remark}

\begin{remark}\label{rem:paradoxical_combinator}
  The reader familiar with the lambda calculus will have noticed the
  similarity between $D$ and the paradoxical combinator.

  [Ed. note: the existence of this seems to suggest we have to be more
  restrictive on the set of processes and names we admit if we are to
  support no-cloning.]
\end{remark}

\subsubsection{Bisimulation}

The computational dynamics gives rise to another kind of equivalence,
the equivalence of computational behavior. As previously mentioned
this is typically captured \emph{via} some form of bisimulation.

% The notion we use in this paper is weak barbed bisimulation
% \cite{milner91polyadicpi}.

The notion we use in this paper is derived from weak barbed
bisimulation \cite{milner91polyadicpi}. 

\begin{definition}
An \emph{observation relation}, $\downarrow_{\mathcal N}$, over a set
of names, $\mathcal N$, is the smallest relation satisfying the rules
below.

\infrule[Out-barb]{y \in {\mathcal N}, \; x \nameeq y}
		  {\outputp{x}{v} \downarrow_{\mathcal N} x}
\infrule[Par-barb]{\mbox{$P\downarrow_{\mathcal N} x$ or $Q\downarrow_{\mathcal N} x$}}
		  {\binpar{P}{Q} \downarrow_{\mathcal N} x}

We write $P \Downarrow_{\mathcal N} x$ if there is $Q$ such that 
$P \wred Q$ and $Q \downarrow_{\mathcal N} x$.
\end{definition}

\begin{definition}
%\label{def.bbisim}
An  ${\mathcal N}$-\emph{barbed bisimulation} over a set of names, ${\mathcal N}$, is a symmetric binary relation 
${\mathcal S}_{\mathcal N}$ between agents such that $P\rel{S}_{\mathcal N}Q$ implies:
\begin{enumerate}
\item If $P \red P'$ then $Q \wred Q'$ and $P'\rel{S}_{\mathcal N} Q'$.
\item If $P\downarrow_{\mathcal N} x$, then $Q\Downarrow_{\mathcal N} x$.
\end{enumerate}
$P$ is ${\mathcal N}$-barbed bisimilar to $Q$, written
$P \wbbisim_{\mathcal N} Q$, if $P \rel{S}_{\mathcal N} Q$ for some ${\mathcal N}$-barbed bisimulation ${\mathcal S}_{\mathcal N}$.
\end{definition}

$\mathcal{R} \subseteq \pi \times \pi$

$P \mathcal{R} Q => \forall P'. P \red P' \Rightarrow \exists Q'. Q \red Q', P' \mathcal{R} Q'$

$P \vdash x \Rightarrow Q \vdash x$

\begin{mathpar}
  \inferrule*[lab=Out-barb]{x \nameeq y}{{y}!\langle{Q}\rangle \vdash x}
  \and
  \inferrule*[lab=Par-barb]{\mbox{$P\vdash x$ or $Q\vdash x$}}{\binpar{P}{Q} \vdash x}
\end{mathpar}

\subsubsection{Contexts}

One of the principle advantages of computational calculi like the
$\pi$-calculus is a well-defined notion of context,
contextual-equivalence and a correlation between
contextual-equivalence and notions of bisimulation. The notion of
context allows the decomposition of a process into (sub-)process and
its syntactic environment, its context. Thus, a context may be
thought of as a process with a ``hole'' (written $\Box$) in it. The
application of a context $M$ to a process $P$, written $M[P]$, is
tantamount to filling the hole in $M$ with $P$. In this paper we do
not need the full weight of this theory, but do make use of the notion
of context in the proof the main theorem. 

\begin{mathpar}
  \inferrule* [lab=summation] {} {{M_{M},M_{N}} \bc \Box \;|\; x.M_{A} \;|\; M_{M}+M_{N}}
  \and
  \inferrule* [lab=agent] {} {{M_{A}} \bc (\vec{x})M_{P} \;| \; \clift{P_0,\ldots,M_{P},\ldots,P_N}}
  \and \\
  \inferrule* [lab=process] {} {{M_{P}} \bc M_{N} \;| \;P|M_{P} }
\end{mathpar} 

\begin{mathpar}
  \inferrule* [lab=sychronization] {} {M_{N} \bc \Box \;|\; x?M_{F} \;|\; x!M_{C}}
  \and
  \inferrule* [lab=abstraction] {} {{M_{F}} \bc (x)M_{P} }
  \and
  \inferrule* [lab=concretion] {} {{M_{C}} \bc \langle M_{P} \rangle }
  \and \\
  \inferrule* [lab=process] {} {{M_{P}} \bc M_{N} \;| \;P|M_{P} }
\end{mathpar}

\begin{definition}[contextual application] Given a context $M$, and
  process $P$, we define the \emph{contextual application}, $M[P] :=
  M\{P/\Box\}$. That is, the contextual application of M to P is the
  substitution of $P$ for $\Box$ in $M$.
\end{definition}

$\meaningof{-} : L \to \mathcal{P}(\pi)$

\begin{mathpar}
  \inferrule* [lab=collection] {} {\meaningof{true} = \pi, \and \meaningof{~E} = \pi \setminus \meaningof{E}, \and \meaningof{E_{1} \& E_{2}} = \meaningof{E_{1}} \cap \meaningof{E_{2}}}
\end{mathpar}

\begin{mathpar}
  \inferrule* [lab=structure] {} {\meaningof{0} = \{ P \in \pi | P \equiv 0 \}, \and \\ \meaningof{E_1 | E_2} = \{ P \in \pi | P \equiv P_{1} | P_{2}, P_{1} \in \meaningof{E_{1}}, P_{2} \in \meaningof{E_2}\} }
\end{mathpar}

\begin{mathpar}
 \inferrule* [lab=behavior] {} {\meaningof{\langle a?b \rangle E} = \{ P \in \pi | P \equiv Q | u?(y)P', \\ \and \\\\ \and \\ \;\;\; u \in \meaningof{a}, \forall z.P'\{z/y\} \in \meaningof{E\{z/b\}}\}, \and \\ \meaningof{a!E} = \{ P \in \pi | P \equiv Q | x!\langle P' \rangle, x \in \meaningof{a} P' \in \meaningof{E}\} }
\end{mathpar}

\begin{mathpar}
 \inferrule* [lab=nominal] {} {\meaningof{\quotep{E}} = \{ \quotep{P} \in \quotep{\pi} | P \in \meaningof{E} \}, \and \meaningof{\quotep{P}} = \{ \quotep{Q} \in \quotep{\pi} | P \equiv Q \} \and \\ \meaningof{@\quotep{E}} = \{ P \in \pi | P \equiv @x, x \in \meaningof{E} \}}
\end{mathpar}

\begin{eqnarray*}
  \\
  \meaningof{-} : TS \to ST
\end{eqnarray*}

\begin{eqnarray*}
  \\
  L : TS \to ST
\end{eqnarray*}

\begin{eqnarray*}
  \\
  P \models E \iff P \in \meaningof{E}
\end{eqnarray*}

\begin{eqnarray*}
  P \approx_{L} Q \iff \forall E \in L. P \models E \iff Q \models E
\end{eqnarray*}

\begin{eqnarray*}
  P \approx_{K} Q
\end{eqnarray*}

\begin{eqnarray*}
  P \approx Q
\end{eqnarray*}

$\approx_{K} = \approx = \approx_{L}$

\subsubsection{Contextual duality}

Note that contexts extend the quotation operation to a family of
operations from processes to names. Given a context, $M$, we can
define a \emph{nominal context}, $\quotep{M}$ by $\quotep{M}[P] :=
\quotep{M[P]}$. To foreshadow what is to come we observe that these
operations enjoy a duality with processes very much like the duality
between vectors and maps from vectors to scalars.

Further, because the calculus is essentially higher-order, we have a
correspondence between contexts and processes. More specifically,
given a name $x$ and a context $M$ we can construct $M^{*}_{x}$ such
that 

\begin{mathpar}
  M^{*}_{x} | \lift{x}{P} \red M[P]
\end{mathpar}

namely,

\begin{mathpar}
  M^{*}_{x} := x?(u).M[\dropn{u}]
\end{mathpar}

The dependence of $M^{*}_{x}$ on a name makes it an abstraction, 

\begin{mathpar}
  M^{*} := (x)x?(u).M[\dropn{u}]
\end{mathpar}

\subsection{Additional notation}

It will sometimes be convenient to denote the process a name
quotes. We already have the notation $x = \quotep{P}$, but it will be
convenient to introduce an alternate notation, $\procn{x}$, when we
want to emphasize the connection to the use of the name. Note that, by
virtue of name equivalence, $\quotep{\procn{x}} \nameeq x$; so, the
notation is consistent with previous definitions.

Further, because names have structure it is possible to effect
substitutions on the basis of that structure. This means we need to
upgrade our notation for substitutions, which we accomplish by
adapting comprehension notation. Thus,

\begin{mathpar}
  P\{ y / x : x \in S \}
\end{mathpar}

is interpreted to mean the process derived from P by replacing (in a
capture-avoiding manner) each occurrence of $x$ in $S$ by $y$. For example,

\begin{mathpar}
  P\{ \quotep{\procn{x}|\procn{x}} / x : x \in \freenames{P} \}
\end{mathpar}

will replace each (occurrence) of a free name $x$ in $P$ by
$\quotep{\procn{x}|\procn{x}}$.

Also, we will avail ourselves of the notation $x^{L}$ and $x^{R}$ to
denote injections of a name into disjoint copies of the name
space. There are numerous ways to accomplish this. One example can be
found in \cite{MeredithR05}. This notation overloads to vectors of
names: $\vec{x}^{\pi} := (x_{i}^{\pi} \; : \; 0 \leq i < |\vec{x}| )$ where $\pi \in \{L,R\}$.

We also use $P^{\Box} := P|\Box$.

In \cite{MeredithR05} an interpretation of the new operator is
given. It turns out that there are several possible interpretations
all enjoying the requisite algebraic properties of the operator (see
\cite{milner91polyadicpi}). We will therefore make liberal use of
$(\nu\; \vec{x})P$.

% subsection the_syntax_and_semantics_of_the_notation_system (end)   

\input{qm2pi.qmops} 

\input{qm2pi.sterngerlach} 

\input{qm2pi.metric} 

% section concurrent_process_calculi (end)

%\input{qm2pi.proofsketch}

% section proof sketch (end)

%\input{qm2pi.slviaknots} 

% section spatial logic via knots (end)

\input{qm2pi.conclusion}

% section conclusion (end)

%\input{qm2pi.dtcodes} 

% section wiring algorithm (end)

\input{qm2pi.ack} 

% section acknowledgments (end)

\newpage


\bibliographystyle{plain}   
\bibliography{../../biblios/main.bib}

\input{qm2pi.rhodetails}

\end{document}

 

% section concurrent_process_calculi (end)

%\documentclass[12pt]{llncs}
%\documentclass{jktr}

\usepackage[pdftex]{hyperref}                   
\usepackage {listings}
\usepackage {mathpartir}
\usepackage{bcprules}
%\usepackage{listings}
                       
\usepackage{graphicx} 
%\usepackage[margins=2.5cm,nohead,nofoot]{geometry}
%\usepackage{geometry}
\usepackage{amsfonts}
\usepackage{amstext}
\usepackage{latexsym}
\usepackage{amssymb}
\usepackage{color}


%\include{myPreamble}
\include{qm2pi.local} 

%\ifpdf
%\usepackage[pdftex]{graphicx}
%\else
%\usepackage{graphicx}
%\fi

 % \ifpdf
%  \usepackage{pdfsync}
%  \if


%\title{Brief Article}
%\author{David F. Snyder}
%\author{L.G. Meredith}

%\address{Dept. of Math., Texas State University--San Marcos, San Marcos, TX 78666}
       
\pagestyle{empty}


\begin{document}

\lstset{language=[Objective]Caml,frame=shadowbox}

\input{qm2pi.front}

% section front matter (end)

\input{qm2pi.intro} 
 
% section introduction (end)

% \input{qm2pi.knotations} 

% section notation (end)

\input{qm2pi.process.calculi} 

% section concurrent_process_calculi_and_spatial_logics_ (end)
    
%\input{qm2pi.knots2pi} 

%\input{qm2pi.trefoil} 

%\input{qm2pi.mainthm} 

% subsection basic_interpretation (end)

%\input{qm2pi.rho.presentation} 
\subsection{The syntax and semantics of the notation system}\label{sub:the_syntax_and_semantics_of_the_notation_system} % (fold)

We now summarize a technical presentation of the calculus that
embodies our theory of dynamics. The typical presentation of such a
calculus follows the style of giving generators and relations on
them. The grammar, below, describing term constructors, freely
generates the set of processes, $\Proc$. This set is then quotiented
by a relation known as structural congruence and it is over this set
that the notion of dynamics is expressed. This presentation is
essentially that of \cite{MeredithR05} with the addition of
polyadicity and summation. For readability we have relegated some of
the technical subtleties to an appendix.

\subsubsection{Process grammar}\label{subsub:process_grammar}

\begin{mathpar}
  \inferrule* [lab=synchronization] {} {{M} \bc \pzero \;|\; x?F \;|\; x!C }
  \and
  \inferrule* [lab=abstraction] {} {{F} \bc (x)P}
  \and
  \inferrule* [lab=concretion] {} {{C} \bc \langle Q \rangle}
  \and
  \inferrule* [lab=process] {} {{P,Q} \bc M \;| \;P|Q \;|\; @{x}}
  \and
  \inferrule* [lab=name] {} {{x} \bc \quotep{P}}
\end{mathpar} 

Note that $\vec{x}$ (resp. $\vec{P}$) denotes a vector of names
(resp. processes) of length $|\vec{x}|$ (resp. $|\vec{P}|$). We adopt
the following useful abbreviations.

\begin{mathpar}
   x?(\vec{y}).P := x.(\vec{y})P \and  x\clift{\vec{P}} := x.\clift{\vec{P}}
   \and x!(y) := \lift{x}{\dropn{y}}
   \and \Pi_{i=0}^{n-1}P_i := P_0 | \ldots | P_{n-1}
\end{mathpar}

\subsubsection{Structural congruence}

\paragraph{Free and bound names and alpha-equivalence.} At the
core of structural equivalence is alpha-equivalence which identifies
process that are the same up to a change of variable. Formally, we
recognize the distinction between free and bound names. The free names
of a process, $\freenames{P}$, may be calculated recursively as
follows:

\begin{mathpar}
\freenames{\pzero} := \emptyset
  \and \\
  \freenames{x?(y).P} := \{ x \} \cup (\freenames{P} \setminus \{ y \})
  \and 
  \freenames{x!\langle P \rangle} := \{ x \} \cup \{ P \} 
  \and \\
  \freenames{P|Q} := \freenames{P} \cup \freenames{Q}
  \and \\
  \freenames{@{x}} := \{ x \}
\end{mathpar}

$\pi$
$\quotep{\pi}$

$\freenames{-} : \pi \to \mathcal{P}(\quotep{\pi})$

\begin{eqnarray*}
  \freenames{\pzero} & := & \emptyset \\
  \freenames{x?(y).P} & := & \{ x \} \cup (\freenames{P} \setminus \{ y \}) \\
  \freenames{x!\langle P \rangle} & := & \{ x \} \cup \{ P \} \\
  \freenames{P|Q} & := & \freenames{P} \cup \freenames{Q} \\
  \freenames{\dropn{x}} & := & \{ x \}
\end{eqnarray*}

The bound names of a process, $\boundnames{P}$, are those names occurring in $P$
that are not free. For example, in $x?(y).0$, the name $x$ is free, while $y$ is bound.

\begin{mathpar}
  \inferrule* [lab=monoidal-laws] {} { P|Q \equiv Q|P \and P|0 \equiv P \and P|(Q|R) \equiv (P|Q)|R }
\end{mathpar}

\begin{mathpar}
  \inferrule* [lab=alpha-equivalence] {} { (x)P \equiv (y)P\{y/x\} \and y \not\in \freenames{P} }
\end{mathpar}

\begin{definition}
Then two processes, $P,Q$, are alpha-equivalent if $P = Q\{\vec{y}/\vec{x}\}$ for
some $\vec{x} \in \boundnames{Q},\vec{y} \in \boundnames{P}$, where $Q\{\vec{y}/\vec{x}\}$
denotes the capture-avoiding substitution of $\vec{y}$ for $\vec{x}$ in $Q$.
\end{definition}

\begin{definition}
  The {\em structural congruence} \cite{SangiorgiWalker} , $\equiv$,
  between processes is the least congruence containing
  alpha-equivalence, satisfying the abelian monoid laws
  (associativity, commutativity and $\pzero$ as identity) for parallel
  composition $|$ and for summation $+$.
\end{definition}

\subsection{Name equivalence}

We take name equivalence, written $\nameeq$, to be the smallest
equivalence relation generated by the following rules.

\begin{mathpar}
\inferrule*[lab=Quote-drop]
{ }
{ \quotep{@{x}} \nameeq x }

\inferrule*[lab=Struct-equiv]
{ P \scong Q }
{ \quotep{P} \nameeq \quotep{Q} }
\end{mathpar}

The astute reader will have noticed that the mutual recursion of names
and processes imposes a mutual recursion on alpha-equivalence and
structural equivalence via name-equivalence. Fortunately, all of this
works out pleasantly and we may calculate in the natural way, free of
concern. The reader interested in the details is referred to the
appendix \ref{appendix:rho_details}.

\subsection{Substitution}

We use $\Proc$ for the set of processes, $\QProc$ for the set of
names, and $\id{\{}\vec{y} / \vec{x} \id{\}}$ to denote partial maps,
$s : \QProc \rightarrow \QProc$. A map, $s$ lifts, uniquely, to a map
on process terms, $\widehat{s} : \Proc \rightarrow \Proc$ by the
following equations.

\begin{mathpar}
  (0) \psubstp{Q}{P} := 0 \\
  (R \juxtap S) \psubstp{Q}{P}
  :=    
  (R)\psubstp{Q}{P} \juxtap (S) \psubstp{Q}{P} \\
  (x?(y).R) \psubstp{Q}{P}    
  :=    
  (x)\substp{Q}{P} (z)\concat( (R \psubstn{z}{y}) \psubstp{Q}{P} ) \\
  (\lift{x}{R}) \psubstp{Q}{P}  
  :=
  \lift{(x)\substp{Q}{P}}{ R \psubstp{Q}{P} } \\
%   (\dropn{x})  \psubstp{Q}{P}       
%   := 
%   \left\{ 
%     \begin{array}{ccc} 
%       \dropn{\quotep{Q}} & & x \nameeq \quotep{P} \\
%       \dropn{x} & & otherwise \\
%     \end{array}
%   \right. 
  (\dropn{x})  \psubstp{Q}{P}       
  := 
  \left\{ 
    \begin{array}{ccc} 
      Q & & x \nameeq \quotep{P} \\
      \dropn{x} & & otherwise \\
    \end{array}
  \right.
\end{mathpar}
 

where

\begin{eqnarray}
  (x)\id{\{} \lpquote Q \rpquote / \lpquote P \rpquote \id{\}}            = 
  \left\{ 
    \begin{array}{ccc}
      \lpquote Q \rpquote & & x \nameeq \lpquote P \rpquote \\
      x & & otherwise \\
    \end{array}
  \right. \nonumber
\end{eqnarray}

and $z$ is chosen distinct from $\quotep{P}$, $\quotep{Q}$, the free
names in $Q$, and all the names in $R$. Our $\alpha$-equivalence will
be built in the standard way from this substitution.

\begin{remark}\label{rem:no_self_referential_names}
  One consequence of these definitions is that $\forall P. \quotep{P}
  \not\in \freenames{P}$.
\end{remark}

\subsection{ Dynamic quote: an example }

Anticipating something of what's to come, consider applying the
substitution, $\widehat{\id{\{}u / z \id{\}}}$, to the following pair
of processes, $\lift{w}{y!(z)}$ and $w[ \lpquote y!(z) \rpquote ]$.

\begin{eqnarray}
	\lift{w}{y!(z)}\widehat{\id{\{}u / z \id{\}}}
		& = &
		\lift{w}{y!(u)} \nonumber\\
	w[ \lpquote y!(z) \rpquote ] \widehat{ \id{\{}u / z \id{\}} }
		& = &
		w[ \lpquote y!(z) \rpquote ] \nonumber
\end{eqnarray}

Because the body of the process between quotes is impervious to
substitution, we get radically different answers. In fact, by
examining the first process in an input context,
e.g. $x?(z).\lift{w}{y!(z)}$, we see that the process under the lift
operator may be shaped by prefixed inputs binding a name inside it. In
this sense, the lift operator will be seen as a way to dynamically
construct processes before reifying them as names.

Finally equipped with these standard features we can present the
dynamics of the calculus.

\subsubsection{Operational semantics} 

Finally, we introduce the computational dynamics. What marks these
algebras as distinct from other more traditionally studied algebraic
structures, e.g. vector spaces or polynomial rings, is the manner in
which dynamics is captured. In traditional structures, dynamics is typically
expressed through morphisms between such structures, as in linear maps
between vector spaces or morphisms between rings. In algebras
associated with the semantics of computation, the dynamics is
expressed as part of the algebraic structure itself, through a
reduction reduction relation typically denoted by $\red$. Below, we
give a recursive presentation of this relation for the calculus used
in the encoding.

$\red \subseteq \pi \times \pi$
$\red : \pi \to \mathcal{P}(\pi)$

\begin{mathpar}
  \inferrule* [lab=Comm] { \textsf{match}( x_{src}, x_{trgt} ) } { x_{trgt}?(y)P \; | \; x_{src}!\langle {Q} \rangle \red P\{\quotep{Q}/y}\} }
  \and \\
  \inferrule* [lab=Par] {{P} \red {P}'} {{{P} | {Q}} \red {{P}' | {Q}}}
  \and
  \inferrule* [lab=Equiv]{{{P} \scong {P}'} \andalso {{P}' \red {Q}'} \andalso {{Q}' \scong {Q}}}{{P} \red {Q}}
\end{mathpar}

\begin{eqnarray*}
  match_{\equiv} (\quotep{P},\quotep{Q}) & := & P \equiv Q \\
  match_{\dagger}(\quotep{P},\quotep{Q}) & := & \forall R. P|Q \red^{*} R => R \red^{*} 0 \\
  match_{K}(\quotep{P},\quotep{Q}) & := & K \mbox{ for some context } K
\end{eqnarray*}

$u?(x)P | u!\langle Q \rangle \red P\{\quotep{Q}/x\}$

%We write $\wred$ for $\red^*$, and $P\red$ if $\exists Q $ such that $ P \red Q$.
We write $P\red$ if $\exists Q $ such that $ P \red Q$ and $P\not\red$, otherwise.

\section{Replication}

As mentioned before, it is known that replication (and hence
recursion) can be implemented in a higher-order process algebra
\cite{SangiorgiWalker}. As our first example of calculation with the
machinery thus far presented we give the construction explicitly in
the {\rhoc}.

\begin{eqnarray}
	D_{x} & := & \prefix{x}{y}{(\binpar{\outputp{x}{y}}{@{y}})} \nonumber\\
	\bangp_{x}{P} & := & \binpar{{x}!\langle{\binpar{D_{x}}{P}}\rangle}{D_{x}} \nonumber
\end{eqnarray}

\begin{eqnarray}
	\bangp_{x}{P} & & \nonumber\\
	=
	& {x}!\langle{(\prefix{x}{y}{(\outputp{x}{y} | @{y})) | P}}\rangle 
	      | \prefix{x}{y}{(\outputp{x}{y} | @{y})} & \nonumber\\
	\red
	& (\outputp{x}{y} | @{y})\substn{\quotep{(\prefix{x}{y}{(@{y} | \outputp{x}{y})) | P}}}{y} & \nonumber\\
	=
	& \outputp{x}{\quotep{(\prefix{x}{y}{(\outputp{x}{y} | @{y})) | P}}}
	  | {(\prefix{x}{y}{(\outputp{x}{y} | @{y})) | P}} & \nonumber\\
	\red
	& \ldots & \nonumber\\
	\red^*
	& P | P | \ldots & \nonumber
\end{eqnarray}

Of course, this encoding, as an implementation, runs away, unfolding
$\bangp{P}$ eagerly. A lazier and more implementable replication
operator, restricted to input-guarded processes, may be obtained as follows.

\begin{eqnarray}
\bangp{\prefix{u}{v}{P}} 
	:= 
	\binpar{\lift{x}{\prefix{u}{v}{(\binpar{D(x)}{P})}}}{D(x)} \nonumber
\end{eqnarray}

\begin{remark}
  Note that the lazier definition still does not deal with summation
  or mixed summation (i.e. sums over input and output). The reader is
  invited to construct definitions of replication that deal with these
  features. 

  Further, the definitions are parameterized in a name, $x$. Can you,
  gentle reader, make a definition that eliminates this parameter and
  guarantees no accidental interaction between the replication
  machinery and the process being replicated -- i.e. no accidental
  sharing of names used by the process to get its work done and the
  name(s) used by the replication to effect copying. This latter
  revision of the definition of replication is crucial to obtaining
  the expected identity $!!P \sim !P$.
\end{remark}

\begin{remark}\label{rem:paradoxical_combinator}
  The reader familiar with the lambda calculus will have noticed the
  similarity between $D$ and the paradoxical combinator.

  [Ed. note: the existence of this seems to suggest we have to be more
  restrictive on the set of processes and names we admit if we are to
  support no-cloning.]
\end{remark}

\subsubsection{Bisimulation}

The computational dynamics gives rise to another kind of equivalence,
the equivalence of computational behavior. As previously mentioned
this is typically captured \emph{via} some form of bisimulation.

% The notion we use in this paper is weak barbed bisimulation
% \cite{milner91polyadicpi}.

The notion we use in this paper is derived from weak barbed
bisimulation \cite{milner91polyadicpi}. 

\begin{definition}
An \emph{observation relation}, $\downarrow_{\mathcal N}$, over a set
of names, $\mathcal N$, is the smallest relation satisfying the rules
below.

\infrule[Out-barb]{y \in {\mathcal N}, \; x \nameeq y}
		  {\outputp{x}{v} \downarrow_{\mathcal N} x}
\infrule[Par-barb]{\mbox{$P\downarrow_{\mathcal N} x$ or $Q\downarrow_{\mathcal N} x$}}
		  {\binpar{P}{Q} \downarrow_{\mathcal N} x}

We write $P \Downarrow_{\mathcal N} x$ if there is $Q$ such that 
$P \wred Q$ and $Q \downarrow_{\mathcal N} x$.
\end{definition}

\begin{definition}
%\label{def.bbisim}
An  ${\mathcal N}$-\emph{barbed bisimulation} over a set of names, ${\mathcal N}$, is a symmetric binary relation 
${\mathcal S}_{\mathcal N}$ between agents such that $P\rel{S}_{\mathcal N}Q$ implies:
\begin{enumerate}
\item If $P \red P'$ then $Q \wred Q'$ and $P'\rel{S}_{\mathcal N} Q'$.
\item If $P\downarrow_{\mathcal N} x$, then $Q\Downarrow_{\mathcal N} x$.
\end{enumerate}
$P$ is ${\mathcal N}$-barbed bisimilar to $Q$, written
$P \wbbisim_{\mathcal N} Q$, if $P \rel{S}_{\mathcal N} Q$ for some ${\mathcal N}$-barbed bisimulation ${\mathcal S}_{\mathcal N}$.
\end{definition}

$\mathcal{R} \subseteq \pi \times \pi$

$P \mathcal{R} Q => \forall P'. P \red P' \Rightarrow \exists Q'. Q \red Q', P' \mathcal{R} Q'$

$P \vdash x \Rightarrow Q \vdash x$

\begin{mathpar}
  \inferrule*[lab=Out-barb]{x \nameeq y}{{y}!\langle{Q}\rangle \vdash x}
  \and
  \inferrule*[lab=Par-barb]{\mbox{$P\vdash x$ or $Q\vdash x$}}{\binpar{P}{Q} \vdash x}
\end{mathpar}

\subsubsection{Contexts}

One of the principle advantages of computational calculi like the
$\pi$-calculus is a well-defined notion of context,
contextual-equivalence and a correlation between
contextual-equivalence and notions of bisimulation. The notion of
context allows the decomposition of a process into (sub-)process and
its syntactic environment, its context. Thus, a context may be
thought of as a process with a ``hole'' (written $\Box$) in it. The
application of a context $M$ to a process $P$, written $M[P]$, is
tantamount to filling the hole in $M$ with $P$. In this paper we do
not need the full weight of this theory, but do make use of the notion
of context in the proof the main theorem. 

\begin{mathpar}
  \inferrule* [lab=summation] {} {{M_{M},M_{N}} \bc \Box \;|\; x.M_{A} \;|\; M_{M}+M_{N}}
  \and
  \inferrule* [lab=agent] {} {{M_{A}} \bc (\vec{x})M_{P} \;| \; \clift{P_0,\ldots,M_{P},\ldots,P_N}}
  \and \\
  \inferrule* [lab=process] {} {{M_{P}} \bc M_{N} \;| \;P|M_{P} }
\end{mathpar} 

\begin{mathpar}
  \inferrule* [lab=sychronization] {} {M_{N} \bc \Box \;|\; x?M_{F} \;|\; x!M_{C}}
  \and
  \inferrule* [lab=abstraction] {} {{M_{F}} \bc (x)M_{P} }
  \and
  \inferrule* [lab=concretion] {} {{M_{C}} \bc \langle M_{P} \rangle }
  \and \\
  \inferrule* [lab=process] {} {{M_{P}} \bc M_{N} \;| \;P|M_{P} }
\end{mathpar}

\begin{definition}[contextual application] Given a context $M$, and
  process $P$, we define the \emph{contextual application}, $M[P] :=
  M\{P/\Box\}$. That is, the contextual application of M to P is the
  substitution of $P$ for $\Box$ in $M$.
\end{definition}

$\meaningof{-} : L \to \mathcal{P}(\pi)$

\begin{mathpar}
  \inferrule* [lab=collection] {} {\meaningof{true} = \pi, \and \meaningof{~E} = \pi \setminus \meaningof{E}, \and \meaningof{E_{1} \& E_{2}} = \meaningof{E_{1}} \cap \meaningof{E_{2}}}
\end{mathpar}

\begin{mathpar}
  \inferrule* [lab=structure] {} {\meaningof{0} = \{ P \in \pi | P \equiv 0 \}, \and \\ \meaningof{E_1 | E_2} = \{ P \in \pi | P \equiv P_{1} | P_{2}, P_{1} \in \meaningof{E_{1}}, P_{2} \in \meaningof{E_2}\} }
\end{mathpar}

\begin{mathpar}
 \inferrule* [lab=behavior] {} {\meaningof{\langle a?b \rangle E} = \{ P \in \pi | P \equiv Q | u?(y)P', \\ \and \\\\ \and \\ \;\;\; u \in \meaningof{a}, \forall z.P'\{z/y\} \in \meaningof{E\{z/b\}}\}, \and \\ \meaningof{a!E} = \{ P \in \pi | P \equiv Q | x!\langle P' \rangle, x \in \meaningof{a} P' \in \meaningof{E}\} }
\end{mathpar}

\begin{mathpar}
 \inferrule* [lab=nominal] {} {\meaningof{\quotep{E}} = \{ \quotep{P} \in \quotep{\pi} | P \in \meaningof{E} \}, \and \meaningof{\quotep{P}} = \{ \quotep{Q} \in \quotep{\pi} | P \equiv Q \} \and \\ \meaningof{@\quotep{E}} = \{ P \in \pi | P \equiv @x, x \in \meaningof{E} \}}
\end{mathpar}

\begin{eqnarray*}
  \\
  \meaningof{-} : TS \to ST
\end{eqnarray*}

\begin{eqnarray*}
  \\
  L : TS \to ST
\end{eqnarray*}

\begin{eqnarray*}
  \\
  P \models E \iff P \in \meaningof{E}
\end{eqnarray*}

\begin{eqnarray*}
  P \approx_{L} Q \iff \forall E \in L. P \models E \iff Q \models E
\end{eqnarray*}

\begin{eqnarray*}
  P \approx_{K} Q
\end{eqnarray*}

\begin{eqnarray*}
  P \approx Q
\end{eqnarray*}

$\approx_{K} = \approx = \approx_{L}$

\subsubsection{Contextual duality}

Note that contexts extend the quotation operation to a family of
operations from processes to names. Given a context, $M$, we can
define a \emph{nominal context}, $\quotep{M}$ by $\quotep{M}[P] :=
\quotep{M[P]}$. To foreshadow what is to come we observe that these
operations enjoy a duality with processes very much like the duality
between vectors and maps from vectors to scalars.

Further, because the calculus is essentially higher-order, we have a
correspondence between contexts and processes. More specifically,
given a name $x$ and a context $M$ we can construct $M^{*}_{x}$ such
that 

\begin{mathpar}
  M^{*}_{x} | \lift{x}{P} \red M[P]
\end{mathpar}

namely,

\begin{mathpar}
  M^{*}_{x} := x?(u).M[\dropn{u}]
\end{mathpar}

The dependence of $M^{*}_{x}$ on a name makes it an abstraction, 

\begin{mathpar}
  M^{*} := (x)x?(u).M[\dropn{u}]
\end{mathpar}

\subsection{Additional notation}

It will sometimes be convenient to denote the process a name
quotes. We already have the notation $x = \quotep{P}$, but it will be
convenient to introduce an alternate notation, $\procn{x}$, when we
want to emphasize the connection to the use of the name. Note that, by
virtue of name equivalence, $\quotep{\procn{x}} \nameeq x$; so, the
notation is consistent with previous definitions.

Further, because names have structure it is possible to effect
substitutions on the basis of that structure. This means we need to
upgrade our notation for substitutions, which we accomplish by
adapting comprehension notation. Thus,

\begin{mathpar}
  P\{ y / x : x \in S \}
\end{mathpar}

is interpreted to mean the process derived from P by replacing (in a
capture-avoiding manner) each occurrence of $x$ in $S$ by $y$. For example,

\begin{mathpar}
  P\{ \quotep{\procn{x}|\procn{x}} / x : x \in \freenames{P} \}
\end{mathpar}

will replace each (occurrence) of a free name $x$ in $P$ by
$\quotep{\procn{x}|\procn{x}}$.

Also, we will avail ourselves of the notation $x^{L}$ and $x^{R}$ to
denote injections of a name into disjoint copies of the name
space. There are numerous ways to accomplish this. One example can be
found in \cite{MeredithR05}. This notation overloads to vectors of
names: $\vec{x}^{\pi} := (x_{i}^{\pi} \; : \; 0 \leq i < |\vec{x}| )$ where $\pi \in \{L,R\}$.

We also use $P^{\Box} := P|\Box$.

In \cite{MeredithR05} an interpretation of the new operator is
given. It turns out that there are several possible interpretations
all enjoying the requisite algebraic properties of the operator (see
\cite{milner91polyadicpi}). We will therefore make liberal use of
$(\nu\; \vec{x})P$.

% subsection the_syntax_and_semantics_of_the_notation_system (end)   

\input{qm2pi.qmops} 

\input{qm2pi.sterngerlach} 

\input{qm2pi.metric} 

% section concurrent_process_calculi (end)

%\input{qm2pi.proofsketch}

% section proof sketch (end)

%\input{qm2pi.slviaknots} 

% section spatial logic via knots (end)

\input{qm2pi.conclusion}

% section conclusion (end)

%\input{qm2pi.dtcodes} 

% section wiring algorithm (end)

\input{qm2pi.ack} 

% section acknowledgments (end)

\newpage


\bibliographystyle{plain}   
\bibliography{../../biblios/main.bib}

\input{qm2pi.rhodetails}

\end{document}



% section proof sketch (end)

%\section{Unlikely characters: spatial logic for
  knots}\label{sub:characteristic_formulae} % (fold)

Associated to the mobile process calculi are a family of logics known
as the Hennessy-Milner logics. These logics typically enjoy a
semantics interpreting formulae as sets of processes that when
factored through the encoding outlined above allows an identification
of classes of knots with logical formulae. In the context of this
encoding the sub-family known as the spatial logics \cite{CairesC03}
\cite{CairesC04} \cite{Caires04} are of particular interest providing
several important features for expressing and reasoning about
properties (i.e. classes) of knots. We hint here at how this may be done.

%\begin{description}
%\item [structural connectives] 
\subsubsection{Structural connectives} The spatial logics enjoy
structural connectives corresponding, at the logical level, to the
parallel composition ($P | Q$) and new name ($(\nu \; x)P$)
connectives for processes. As illustrated in the examples below, these
connectives are extremely expressive given the shape of our encoding.
%\item [decideable satisfaction]

\subsubsection{Decideable satisfaction}
In \cite{Caires04} the satisfaction relation is shown to be decideable
for a rich class of processes. It further turns out that the image of
the our encoding is a proper subset of that class. This result
provides the basis for an algorithm by which to search for knots
enjoying a given property.
%\item [characteristic formulae]

\subsubsection{Characteristic formulae}
In the same paper \cite{Caires04} , Caires presents a means of calculating
characteristic formulae, selecting equivalence classes of processes
up to a pre--specified depth limit on the support set of names. Composed with our
encoding, this characteristic formula can be used to select
characteristic formulae for knots.
%\end{description}

\subsubsection{Spatial logic formulae}

The grammar below (segmented for comprehension) summarizes the syntax
of spatial logic formulae. We employ illustrative examples in the
sequel to provide an intuitive understanding of their meaning
referring the reader to \cite{Caires04} for a more detailed explication
of the semantics.

\begin{mathpar}
  \inferrule* [lab=boolean] {} {{A,B} \bc T \;|\; \neg A \;|\; A \wedge B \;|\; \eta = \eta'}
  \and
  \inferrule* [lab=spatial] {} {|\; \pzero \;|\; A | B \;|\; x \text{\textregistered} A \;|\; \forall x . A \;|\;  H x . A}
  \and
  \inferrule* [lab=behavioral] {} {|\; \alpha . A}
  \and 
  \inferrule* [lab=recursion] {} {|\; X(\vec{u}) \;|\; \mu X(\vec{u}) . A}
  \and
  \inferrule* [lab=action] {} {\alpha \bc \langle x?(\vec{y}) \rangle \;|\; \langle x!(\vec{y}) \rangle \;|\; \langle \tau \rangle}
  \and 
  \inferrule* [lab=name] {} {\eta \bc x \;|\; \tau}
\end{mathpar} 

% subsection characteristic_formulae (end)   	 

\subsection{Example formulae}\label{sub:example_formulae_} % (fold)

\subsubsection{Crossing as formula.}
% 
% \begin{align*}
%   \frac{d}{dx} \sin x &= \cos x 
%   & \frac{d}{dx} e^x &= e^x \\
%   \frac{d}{dx} \cos x &= - \sin x 
%   & \frac{d}{dx} \log x &= \frac{1}{x} \\
% \end{align*} 

\begin{align*}
 \mu C(x_{0},x_{1},y_{0},y_{1},u).&(\langle x_{0}?(z) \rangle(\langle u! \rangle\langle y_{1}!z \rangle C(x_{0},x_{1},y_{0},y_{1},u)) & \\
  & \wedge \langle y_{1}?(z) \rangle (\langle u! \rangle \langle x_{0}!z \rangle C(x_{0},x_{1},y_{0},y_{1},u)) & \\
  & \wedge \langle x_{1}?(z) \rangle (\langle u? \rangle \langle y_{0}!z \rangle C(x_{0},x_{1},y_{0},y_{1},u)) & \\
  & \wedge \langle y_{0}?(z) \rangle (\langle u? \rangle \langle x_{1}!z \rangle C(x_{0},x_{1},y_{0},y_{1},u))) &
\end{align*}

The lexicographical similarity between the shape of this formulae and
the shape of definition of the process representing a crossing reveals
the intuitive meaning of this formulae. It describes the capabilities
of a process that has the right to represent a crossing. For example
it picks out processes that may perform an input on the port $x_0$ in
its initial menu of capabilities. What differentiates the formula
from the process, however, is that the crossing process is the
smallest candidate to satisfy the formula. Infinitely many other
processes -- with internal behavior hidden behind this interface, so
to speak -- also satisfy this formula. Even this simple formula,
then, can be seen to open a new view onto knots, providing a
computational interpretation of \emph{virtual} knots.

Note that this formula is derived by hand. A similar formula can be
derived by employing Caires' calculation of characteristic formula
\cite{Caires04} to the process representing a crossing. In light of
this discussion, we let
$\meaningof{C}_{\phi}(x0,x1,y0,y1,u)$ denote a formula specifying the
dynamics we wish to capture of a crossing. To guarantee we preserve
the shape of the interface and minimal semantics we demand that
$\meaningof{C}_{\phi}(x0,x1,y0,y1,u) \Rightarrow
\textbf{C}(x0,x1,y0,y1,u)$ where $\textbf{C}(x0,x1,y0,y1,u)$ denotes
the formula above.
                            
\subsubsection{Crossing number constraints.}
The moral content of the context lemma (Lemma \ref{context}) is that the notion of
``locality'' in the Reidemeister moves is effectively captured by the
parallel composition operator of the process calculus. This intuition
extends through the logic. Given a formula,
$\meaningof{C}_{\phi}(x0,x1,y0,y1,u)$, we can use the structural
connectives to specify constraints on crossing numbers, such as at
least $n$ crossings, or exactly $n$ crossings.
\begin{mathpar}
  \inferrule* [lab=at-least-n] {} { K^{\geq n}_{\phi}(\vec{xs},\vec{ys}) := \Pi_{i=0}^{n-1} Hu . \meaningof{C}_{\phi}(xs_i,ys_i,u) | T }
  \and 
  \inferrule* [lab=exactly-n] {} { K^{= n}_{\phi}(\vec{xs},\vec{ys}) := \Pi_{i=0}^{n-1} Hu . \meaningof{C}_{\phi}(xs_i,ys_i,u) | \neg (\forall x_0,y_0,x_1,y_1,u . \meaningof{C}_{\phi}(x_0,y_0,x_1,y_1,u) | T) }
\end{mathpar}

To round out this section, recall that the encoding of an $n$-crossing
knot decomposes into a parallel composition of $n$ \emph{copies} of a
crossing process together with a wiring harness. To specify different
knot classes with the same crossing number amounts to specifying
logical constraints on the wiring harness. In the interest of space,
we defer examples to a forthcoming paper. Suffice it to say that both
the conditions ``alternating knot'' and ``contains the tangle
corresponding to 5/3'' are expressible. For example, it is possible to
calculate the characteristic formula of a process corresponding to the
tangle 5/3 and conjoin it into the classifying formula via the
composition connective of the logic.

Finally, we wish to observe that it is entirely within reason to
contemplate a more domain-specific version of spatial logic tailored
to the shape of processes in the image of the encoding. Such a
domain-specific logic would have a better claim to the title formal
language of knot properties.

% subsection example_formulae_ (end)

% section knots_as_processes (end) 

% section spatial logic via knots (end)

\section{Conclusions and future work}

\paragraph{Testing physical space}
You, gentle reader, may wonder why of all the theorems to be proved
given this set up we pick the one above. In some sense it's hardly
central to quantum mechanics. We see it as central in the sense that
it firmly establishes a notion of physical space arising from a notion
of the equivalence of behavior. Relating bisimulation to a metric is a
big step forward, but one is faced with interpreting the relationship
of that metric space to something more physical. Quantum mechanical
notions of ``physical'' space are still far from intuitive, but by
relating this idea of distance as testing to calculations that predict
physical circumstances we are making a not insignificant step forward
toward an understanding of the physical space we inhabit as
essentially dynamic.

\paragraph{Effectivity and simulation}
One of the observations we have yet to make is that the entire program
spelled out here is effective. We have built various interpreters for
the reflective calculus at work in this interpretation. In principle,
then, we can simulate quantum mechanics on a computer. The place where
the simulation may lose fidelity is the infinitely branching summation
for the annihilator.

In this connection i also want to point out that the evaluation style
calculation of the inner product puts the non-determinism of the
summation right at the heart of measurement. This suggests that
Milner's original reduction-based formulation of the dynamics of his
calculi in terms of sums was not just notationally suggestive of a
notion of measure-and-continue but captured some significant part of
the physics.

\paragraph{Quantum continuations}
In light of this last observation i want to point out that the
predominant account of quantum mechanics is missing a key aspect of a
truly compositional story of the physical situation. In a real lab,
when a measurement is made the observation can be made to feed into
another device that then makes another measurement conditioned on the
results of the first. This means that after the superposition was
collapsed the entire experimental set up remained in
superposition. While QM offers a means of writing this down it doesn't
quite line up well with the well-trodden formulation of computation
and continuation that we see so succinctly expressed in Milner's
calculi. This suggests that there might be advantages to this account
of dynamics waiting to be explored.

\paragraph{Quantum logic}
In this connection, we also note that by virtue of having the
Hennessy-Milner construction, we can pull the construction through the
interpretation of QM. This gives us a natural candidate for a quantum
logic that enjoys an extremely tight connection with it's domain of
interpretation, making the construction much less ad hoc (rather it is
the image of functor!).

\paragraph{Quantum probabiity}
i have questions about the basis of the interpretation of inner
product as probability amplitude. In particular, using which
axiomatization of probability theory does the notion of probability
amplitude earn the right to be so dubbed? In other words, where is the
proof that the operation for calculating a probability amplitude (and
then squaring) satisfies the axioms of what it means to calculate a
probability? Even if such a proof exists (i have yet to find it in the
literature), i wonder if it might not be possible to turn things on
their heads. Can we view the calculation of the probability amplitude
as an axiomatization of probability? If so, then the definition we
give for calculating probability amplitude may provide the basis for
an \emph{effective} theory of probability.

\paragraph{Quantum vs ``biological'' information}
Finally, i want to conclude with a more philosophical observation. At
a recent workshop in which QM was a predominant topic i noticed
something about quantum information. The speaker was giving a riveting
discussion of axiomatic QM and showing how properties of ``no
cloning'' and ``no deleting'' emerged as consequences of the
axiomatization. Theorems of this form are necessary to give us a sense
of confidence that our axioms characterize the physical theory. What
struck me, though, was that if quantum information is neither erasable
nor replicable it is markedly different from \emph{life}. Two of the
things we know about life is that

\begin{itemize}
  \item it ends;
  \item to gain some measure of persistence, to transcend it's
    finitude it is imminently copyable.
\end{itemize}

Both of these qualities are summarized succinctly in the aphorism: all
flesh is grass. For me these two kinds of ``information'' -- call them
quantum and biological -- are end points on a spectrum of strategies
for persistence. At one end, we have those curious entities that enjoy
uniqueness and permanence; at the other, we have those who in the face
of a certain end and an uncertain present make a go of passing
something on. To me one of the more remarkable aspects of the latter
strategy is that in the presence of noise (and certain features of
copying) we get a kind of dynamism, a chance for improvement against a
given persistent condition.

% subsection other_calculi_other_bisimulations_and_geometry_as_behavior (end)




% section conclusion (end)

%\documentclass[12pt]{llncs}
%\documentclass{jktr}

\usepackage[pdftex]{hyperref}                   
\usepackage {listings}
\usepackage {mathpartir}
\usepackage{bcprules}
%\usepackage{listings}
                       
\usepackage{graphicx} 
%\usepackage[margins=2.5cm,nohead,nofoot]{geometry}
%\usepackage{geometry}
\usepackage{amsfonts}
\usepackage{amstext}
\usepackage{latexsym}
\usepackage{amssymb}
\usepackage{color}


%\include{myPreamble}
\include{qm2pi.local} 

%\ifpdf
%\usepackage[pdftex]{graphicx}
%\else
%\usepackage{graphicx}
%\fi

 % \ifpdf
%  \usepackage{pdfsync}
%  \if


%\title{Brief Article}
%\author{David F. Snyder}
%\author{L.G. Meredith}

%\address{Dept. of Math., Texas State University--San Marcos, San Marcos, TX 78666}
       
\pagestyle{empty}


\begin{document}

\lstset{language=[Objective]Caml,frame=shadowbox}

\input{qm2pi.front}

% section front matter (end)

\input{qm2pi.intro} 
 
% section introduction (end)

% \input{qm2pi.knotations} 

% section notation (end)

\input{qm2pi.process.calculi} 

% section concurrent_process_calculi_and_spatial_logics_ (end)
    
%\input{qm2pi.knots2pi} 

%\input{qm2pi.trefoil} 

%\input{qm2pi.mainthm} 

% subsection basic_interpretation (end)

%\input{qm2pi.rho.presentation} 
\subsection{The syntax and semantics of the notation system}\label{sub:the_syntax_and_semantics_of_the_notation_system} % (fold)

We now summarize a technical presentation of the calculus that
embodies our theory of dynamics. The typical presentation of such a
calculus follows the style of giving generators and relations on
them. The grammar, below, describing term constructors, freely
generates the set of processes, $\Proc$. This set is then quotiented
by a relation known as structural congruence and it is over this set
that the notion of dynamics is expressed. This presentation is
essentially that of \cite{MeredithR05} with the addition of
polyadicity and summation. For readability we have relegated some of
the technical subtleties to an appendix.

\subsubsection{Process grammar}\label{subsub:process_grammar}

\begin{mathpar}
  \inferrule* [lab=synchronization] {} {{M} \bc \pzero \;|\; x?F \;|\; x!C }
  \and
  \inferrule* [lab=abstraction] {} {{F} \bc (x)P}
  \and
  \inferrule* [lab=concretion] {} {{C} \bc \langle Q \rangle}
  \and
  \inferrule* [lab=process] {} {{P,Q} \bc M \;| \;P|Q \;|\; @{x}}
  \and
  \inferrule* [lab=name] {} {{x} \bc \quotep{P}}
\end{mathpar} 

Note that $\vec{x}$ (resp. $\vec{P}$) denotes a vector of names
(resp. processes) of length $|\vec{x}|$ (resp. $|\vec{P}|$). We adopt
the following useful abbreviations.

\begin{mathpar}
   x?(\vec{y}).P := x.(\vec{y})P \and  x\clift{\vec{P}} := x.\clift{\vec{P}}
   \and x!(y) := \lift{x}{\dropn{y}}
   \and \Pi_{i=0}^{n-1}P_i := P_0 | \ldots | P_{n-1}
\end{mathpar}

\subsubsection{Structural congruence}

\paragraph{Free and bound names and alpha-equivalence.} At the
core of structural equivalence is alpha-equivalence which identifies
process that are the same up to a change of variable. Formally, we
recognize the distinction between free and bound names. The free names
of a process, $\freenames{P}$, may be calculated recursively as
follows:

\begin{mathpar}
\freenames{\pzero} := \emptyset
  \and \\
  \freenames{x?(y).P} := \{ x \} \cup (\freenames{P} \setminus \{ y \})
  \and 
  \freenames{x!\langle P \rangle} := \{ x \} \cup \{ P \} 
  \and \\
  \freenames{P|Q} := \freenames{P} \cup \freenames{Q}
  \and \\
  \freenames{@{x}} := \{ x \}
\end{mathpar}

$\pi$
$\quotep{\pi}$

$\freenames{-} : \pi \to \mathcal{P}(\quotep{\pi})$

\begin{eqnarray*}
  \freenames{\pzero} & := & \emptyset \\
  \freenames{x?(y).P} & := & \{ x \} \cup (\freenames{P} \setminus \{ y \}) \\
  \freenames{x!\langle P \rangle} & := & \{ x \} \cup \{ P \} \\
  \freenames{P|Q} & := & \freenames{P} \cup \freenames{Q} \\
  \freenames{\dropn{x}} & := & \{ x \}
\end{eqnarray*}

The bound names of a process, $\boundnames{P}$, are those names occurring in $P$
that are not free. For example, in $x?(y).0$, the name $x$ is free, while $y$ is bound.

\begin{mathpar}
  \inferrule* [lab=monoidal-laws] {} { P|Q \equiv Q|P \and P|0 \equiv P \and P|(Q|R) \equiv (P|Q)|R }
\end{mathpar}

\begin{mathpar}
  \inferrule* [lab=alpha-equivalence] {} { (x)P \equiv (y)P\{y/x\} \and y \not\in \freenames{P} }
\end{mathpar}

\begin{definition}
Then two processes, $P,Q$, are alpha-equivalent if $P = Q\{\vec{y}/\vec{x}\}$ for
some $\vec{x} \in \boundnames{Q},\vec{y} \in \boundnames{P}$, where $Q\{\vec{y}/\vec{x}\}$
denotes the capture-avoiding substitution of $\vec{y}$ for $\vec{x}$ in $Q$.
\end{definition}

\begin{definition}
  The {\em structural congruence} \cite{SangiorgiWalker} , $\equiv$,
  between processes is the least congruence containing
  alpha-equivalence, satisfying the abelian monoid laws
  (associativity, commutativity and $\pzero$ as identity) for parallel
  composition $|$ and for summation $+$.
\end{definition}

\subsection{Name equivalence}

We take name equivalence, written $\nameeq$, to be the smallest
equivalence relation generated by the following rules.

\begin{mathpar}
\inferrule*[lab=Quote-drop]
{ }
{ \quotep{@{x}} \nameeq x }

\inferrule*[lab=Struct-equiv]
{ P \scong Q }
{ \quotep{P} \nameeq \quotep{Q} }
\end{mathpar}

The astute reader will have noticed that the mutual recursion of names
and processes imposes a mutual recursion on alpha-equivalence and
structural equivalence via name-equivalence. Fortunately, all of this
works out pleasantly and we may calculate in the natural way, free of
concern. The reader interested in the details is referred to the
appendix \ref{appendix:rho_details}.

\subsection{Substitution}

We use $\Proc$ for the set of processes, $\QProc$ for the set of
names, and $\id{\{}\vec{y} / \vec{x} \id{\}}$ to denote partial maps,
$s : \QProc \rightarrow \QProc$. A map, $s$ lifts, uniquely, to a map
on process terms, $\widehat{s} : \Proc \rightarrow \Proc$ by the
following equations.

\begin{mathpar}
  (0) \psubstp{Q}{P} := 0 \\
  (R \juxtap S) \psubstp{Q}{P}
  :=    
  (R)\psubstp{Q}{P} \juxtap (S) \psubstp{Q}{P} \\
  (x?(y).R) \psubstp{Q}{P}    
  :=    
  (x)\substp{Q}{P} (z)\concat( (R \psubstn{z}{y}) \psubstp{Q}{P} ) \\
  (\lift{x}{R}) \psubstp{Q}{P}  
  :=
  \lift{(x)\substp{Q}{P}}{ R \psubstp{Q}{P} } \\
%   (\dropn{x})  \psubstp{Q}{P}       
%   := 
%   \left\{ 
%     \begin{array}{ccc} 
%       \dropn{\quotep{Q}} & & x \nameeq \quotep{P} \\
%       \dropn{x} & & otherwise \\
%     \end{array}
%   \right. 
  (\dropn{x})  \psubstp{Q}{P}       
  := 
  \left\{ 
    \begin{array}{ccc} 
      Q & & x \nameeq \quotep{P} \\
      \dropn{x} & & otherwise \\
    \end{array}
  \right.
\end{mathpar}
 

where

\begin{eqnarray}
  (x)\id{\{} \lpquote Q \rpquote / \lpquote P \rpquote \id{\}}            = 
  \left\{ 
    \begin{array}{ccc}
      \lpquote Q \rpquote & & x \nameeq \lpquote P \rpquote \\
      x & & otherwise \\
    \end{array}
  \right. \nonumber
\end{eqnarray}

and $z$ is chosen distinct from $\quotep{P}$, $\quotep{Q}$, the free
names in $Q$, and all the names in $R$. Our $\alpha$-equivalence will
be built in the standard way from this substitution.

\begin{remark}\label{rem:no_self_referential_names}
  One consequence of these definitions is that $\forall P. \quotep{P}
  \not\in \freenames{P}$.
\end{remark}

\subsection{ Dynamic quote: an example }

Anticipating something of what's to come, consider applying the
substitution, $\widehat{\id{\{}u / z \id{\}}}$, to the following pair
of processes, $\lift{w}{y!(z)}$ and $w[ \lpquote y!(z) \rpquote ]$.

\begin{eqnarray}
	\lift{w}{y!(z)}\widehat{\id{\{}u / z \id{\}}}
		& = &
		\lift{w}{y!(u)} \nonumber\\
	w[ \lpquote y!(z) \rpquote ] \widehat{ \id{\{}u / z \id{\}} }
		& = &
		w[ \lpquote y!(z) \rpquote ] \nonumber
\end{eqnarray}

Because the body of the process between quotes is impervious to
substitution, we get radically different answers. In fact, by
examining the first process in an input context,
e.g. $x?(z).\lift{w}{y!(z)}$, we see that the process under the lift
operator may be shaped by prefixed inputs binding a name inside it. In
this sense, the lift operator will be seen as a way to dynamically
construct processes before reifying them as names.

Finally equipped with these standard features we can present the
dynamics of the calculus.

\subsubsection{Operational semantics} 

Finally, we introduce the computational dynamics. What marks these
algebras as distinct from other more traditionally studied algebraic
structures, e.g. vector spaces or polynomial rings, is the manner in
which dynamics is captured. In traditional structures, dynamics is typically
expressed through morphisms between such structures, as in linear maps
between vector spaces or morphisms between rings. In algebras
associated with the semantics of computation, the dynamics is
expressed as part of the algebraic structure itself, through a
reduction reduction relation typically denoted by $\red$. Below, we
give a recursive presentation of this relation for the calculus used
in the encoding.

$\red \subseteq \pi \times \pi$
$\red : \pi \to \mathcal{P}(\pi)$

\begin{mathpar}
  \inferrule* [lab=Comm] { \textsf{match}( x_{src}, x_{trgt} ) } { x_{trgt}?(y)P \; | \; x_{src}!\langle {Q} \rangle \red P\{\quotep{Q}/y}\} }
  \and \\
  \inferrule* [lab=Par] {{P} \red {P}'} {{{P} | {Q}} \red {{P}' | {Q}}}
  \and
  \inferrule* [lab=Equiv]{{{P} \scong {P}'} \andalso {{P}' \red {Q}'} \andalso {{Q}' \scong {Q}}}{{P} \red {Q}}
\end{mathpar}

\begin{eqnarray*}
  match_{\equiv} (\quotep{P},\quotep{Q}) & := & P \equiv Q \\
  match_{\dagger}(\quotep{P},\quotep{Q}) & := & \forall R. P|Q \red^{*} R => R \red^{*} 0 \\
  match_{K}(\quotep{P},\quotep{Q}) & := & K \mbox{ for some context } K
\end{eqnarray*}

$u?(x)P | u!\langle Q \rangle \red P\{\quotep{Q}/x\}$

%We write $\wred$ for $\red^*$, and $P\red$ if $\exists Q $ such that $ P \red Q$.
We write $P\red$ if $\exists Q $ such that $ P \red Q$ and $P\not\red$, otherwise.

\section{Replication}

As mentioned before, it is known that replication (and hence
recursion) can be implemented in a higher-order process algebra
\cite{SangiorgiWalker}. As our first example of calculation with the
machinery thus far presented we give the construction explicitly in
the {\rhoc}.

\begin{eqnarray}
	D_{x} & := & \prefix{x}{y}{(\binpar{\outputp{x}{y}}{@{y}})} \nonumber\\
	\bangp_{x}{P} & := & \binpar{{x}!\langle{\binpar{D_{x}}{P}}\rangle}{D_{x}} \nonumber
\end{eqnarray}

\begin{eqnarray}
	\bangp_{x}{P} & & \nonumber\\
	=
	& {x}!\langle{(\prefix{x}{y}{(\outputp{x}{y} | @{y})) | P}}\rangle 
	      | \prefix{x}{y}{(\outputp{x}{y} | @{y})} & \nonumber\\
	\red
	& (\outputp{x}{y} | @{y})\substn{\quotep{(\prefix{x}{y}{(@{y} | \outputp{x}{y})) | P}}}{y} & \nonumber\\
	=
	& \outputp{x}{\quotep{(\prefix{x}{y}{(\outputp{x}{y} | @{y})) | P}}}
	  | {(\prefix{x}{y}{(\outputp{x}{y} | @{y})) | P}} & \nonumber\\
	\red
	& \ldots & \nonumber\\
	\red^*
	& P | P | \ldots & \nonumber
\end{eqnarray}

Of course, this encoding, as an implementation, runs away, unfolding
$\bangp{P}$ eagerly. A lazier and more implementable replication
operator, restricted to input-guarded processes, may be obtained as follows.

\begin{eqnarray}
\bangp{\prefix{u}{v}{P}} 
	:= 
	\binpar{\lift{x}{\prefix{u}{v}{(\binpar{D(x)}{P})}}}{D(x)} \nonumber
\end{eqnarray}

\begin{remark}
  Note that the lazier definition still does not deal with summation
  or mixed summation (i.e. sums over input and output). The reader is
  invited to construct definitions of replication that deal with these
  features. 

  Further, the definitions are parameterized in a name, $x$. Can you,
  gentle reader, make a definition that eliminates this parameter and
  guarantees no accidental interaction between the replication
  machinery and the process being replicated -- i.e. no accidental
  sharing of names used by the process to get its work done and the
  name(s) used by the replication to effect copying. This latter
  revision of the definition of replication is crucial to obtaining
  the expected identity $!!P \sim !P$.
\end{remark}

\begin{remark}\label{rem:paradoxical_combinator}
  The reader familiar with the lambda calculus will have noticed the
  similarity between $D$ and the paradoxical combinator.

  [Ed. note: the existence of this seems to suggest we have to be more
  restrictive on the set of processes and names we admit if we are to
  support no-cloning.]
\end{remark}

\subsubsection{Bisimulation}

The computational dynamics gives rise to another kind of equivalence,
the equivalence of computational behavior. As previously mentioned
this is typically captured \emph{via} some form of bisimulation.

% The notion we use in this paper is weak barbed bisimulation
% \cite{milner91polyadicpi}.

The notion we use in this paper is derived from weak barbed
bisimulation \cite{milner91polyadicpi}. 

\begin{definition}
An \emph{observation relation}, $\downarrow_{\mathcal N}$, over a set
of names, $\mathcal N$, is the smallest relation satisfying the rules
below.

\infrule[Out-barb]{y \in {\mathcal N}, \; x \nameeq y}
		  {\outputp{x}{v} \downarrow_{\mathcal N} x}
\infrule[Par-barb]{\mbox{$P\downarrow_{\mathcal N} x$ or $Q\downarrow_{\mathcal N} x$}}
		  {\binpar{P}{Q} \downarrow_{\mathcal N} x}

We write $P \Downarrow_{\mathcal N} x$ if there is $Q$ such that 
$P \wred Q$ and $Q \downarrow_{\mathcal N} x$.
\end{definition}

\begin{definition}
%\label{def.bbisim}
An  ${\mathcal N}$-\emph{barbed bisimulation} over a set of names, ${\mathcal N}$, is a symmetric binary relation 
${\mathcal S}_{\mathcal N}$ between agents such that $P\rel{S}_{\mathcal N}Q$ implies:
\begin{enumerate}
\item If $P \red P'$ then $Q \wred Q'$ and $P'\rel{S}_{\mathcal N} Q'$.
\item If $P\downarrow_{\mathcal N} x$, then $Q\Downarrow_{\mathcal N} x$.
\end{enumerate}
$P$ is ${\mathcal N}$-barbed bisimilar to $Q$, written
$P \wbbisim_{\mathcal N} Q$, if $P \rel{S}_{\mathcal N} Q$ for some ${\mathcal N}$-barbed bisimulation ${\mathcal S}_{\mathcal N}$.
\end{definition}

$\mathcal{R} \subseteq \pi \times \pi$

$P \mathcal{R} Q => \forall P'. P \red P' \Rightarrow \exists Q'. Q \red Q', P' \mathcal{R} Q'$

$P \vdash x \Rightarrow Q \vdash x$

\begin{mathpar}
  \inferrule*[lab=Out-barb]{x \nameeq y}{{y}!\langle{Q}\rangle \vdash x}
  \and
  \inferrule*[lab=Par-barb]{\mbox{$P\vdash x$ or $Q\vdash x$}}{\binpar{P}{Q} \vdash x}
\end{mathpar}

\subsubsection{Contexts}

One of the principle advantages of computational calculi like the
$\pi$-calculus is a well-defined notion of context,
contextual-equivalence and a correlation between
contextual-equivalence and notions of bisimulation. The notion of
context allows the decomposition of a process into (sub-)process and
its syntactic environment, its context. Thus, a context may be
thought of as a process with a ``hole'' (written $\Box$) in it. The
application of a context $M$ to a process $P$, written $M[P]$, is
tantamount to filling the hole in $M$ with $P$. In this paper we do
not need the full weight of this theory, but do make use of the notion
of context in the proof the main theorem. 

\begin{mathpar}
  \inferrule* [lab=summation] {} {{M_{M},M_{N}} \bc \Box \;|\; x.M_{A} \;|\; M_{M}+M_{N}}
  \and
  \inferrule* [lab=agent] {} {{M_{A}} \bc (\vec{x})M_{P} \;| \; \clift{P_0,\ldots,M_{P},\ldots,P_N}}
  \and \\
  \inferrule* [lab=process] {} {{M_{P}} \bc M_{N} \;| \;P|M_{P} }
\end{mathpar} 

\begin{mathpar}
  \inferrule* [lab=sychronization] {} {M_{N} \bc \Box \;|\; x?M_{F} \;|\; x!M_{C}}
  \and
  \inferrule* [lab=abstraction] {} {{M_{F}} \bc (x)M_{P} }
  \and
  \inferrule* [lab=concretion] {} {{M_{C}} \bc \langle M_{P} \rangle }
  \and \\
  \inferrule* [lab=process] {} {{M_{P}} \bc M_{N} \;| \;P|M_{P} }
\end{mathpar}

\begin{definition}[contextual application] Given a context $M$, and
  process $P$, we define the \emph{contextual application}, $M[P] :=
  M\{P/\Box\}$. That is, the contextual application of M to P is the
  substitution of $P$ for $\Box$ in $M$.
\end{definition}

$\meaningof{-} : L \to \mathcal{P}(\pi)$

\begin{mathpar}
  \inferrule* [lab=collection] {} {\meaningof{true} = \pi, \and \meaningof{~E} = \pi \setminus \meaningof{E}, \and \meaningof{E_{1} \& E_{2}} = \meaningof{E_{1}} \cap \meaningof{E_{2}}}
\end{mathpar}

\begin{mathpar}
  \inferrule* [lab=structure] {} {\meaningof{0} = \{ P \in \pi | P \equiv 0 \}, \and \\ \meaningof{E_1 | E_2} = \{ P \in \pi | P \equiv P_{1} | P_{2}, P_{1} \in \meaningof{E_{1}}, P_{2} \in \meaningof{E_2}\} }
\end{mathpar}

\begin{mathpar}
 \inferrule* [lab=behavior] {} {\meaningof{\langle a?b \rangle E} = \{ P \in \pi | P \equiv Q | u?(y)P', \\ \and \\\\ \and \\ \;\;\; u \in \meaningof{a}, \forall z.P'\{z/y\} \in \meaningof{E\{z/b\}}\}, \and \\ \meaningof{a!E} = \{ P \in \pi | P \equiv Q | x!\langle P' \rangle, x \in \meaningof{a} P' \in \meaningof{E}\} }
\end{mathpar}

\begin{mathpar}
 \inferrule* [lab=nominal] {} {\meaningof{\quotep{E}} = \{ \quotep{P} \in \quotep{\pi} | P \in \meaningof{E} \}, \and \meaningof{\quotep{P}} = \{ \quotep{Q} \in \quotep{\pi} | P \equiv Q \} \and \\ \meaningof{@\quotep{E}} = \{ P \in \pi | P \equiv @x, x \in \meaningof{E} \}}
\end{mathpar}

\begin{eqnarray*}
  \\
  \meaningof{-} : TS \to ST
\end{eqnarray*}

\begin{eqnarray*}
  \\
  L : TS \to ST
\end{eqnarray*}

\begin{eqnarray*}
  \\
  P \models E \iff P \in \meaningof{E}
\end{eqnarray*}

\begin{eqnarray*}
  P \approx_{L} Q \iff \forall E \in L. P \models E \iff Q \models E
\end{eqnarray*}

\begin{eqnarray*}
  P \approx_{K} Q
\end{eqnarray*}

\begin{eqnarray*}
  P \approx Q
\end{eqnarray*}

$\approx_{K} = \approx = \approx_{L}$

\subsubsection{Contextual duality}

Note that contexts extend the quotation operation to a family of
operations from processes to names. Given a context, $M$, we can
define a \emph{nominal context}, $\quotep{M}$ by $\quotep{M}[P] :=
\quotep{M[P]}$. To foreshadow what is to come we observe that these
operations enjoy a duality with processes very much like the duality
between vectors and maps from vectors to scalars.

Further, because the calculus is essentially higher-order, we have a
correspondence between contexts and processes. More specifically,
given a name $x$ and a context $M$ we can construct $M^{*}_{x}$ such
that 

\begin{mathpar}
  M^{*}_{x} | \lift{x}{P} \red M[P]
\end{mathpar}

namely,

\begin{mathpar}
  M^{*}_{x} := x?(u).M[\dropn{u}]
\end{mathpar}

The dependence of $M^{*}_{x}$ on a name makes it an abstraction, 

\begin{mathpar}
  M^{*} := (x)x?(u).M[\dropn{u}]
\end{mathpar}

\subsection{Additional notation}

It will sometimes be convenient to denote the process a name
quotes. We already have the notation $x = \quotep{P}$, but it will be
convenient to introduce an alternate notation, $\procn{x}$, when we
want to emphasize the connection to the use of the name. Note that, by
virtue of name equivalence, $\quotep{\procn{x}} \nameeq x$; so, the
notation is consistent with previous definitions.

Further, because names have structure it is possible to effect
substitutions on the basis of that structure. This means we need to
upgrade our notation for substitutions, which we accomplish by
adapting comprehension notation. Thus,

\begin{mathpar}
  P\{ y / x : x \in S \}
\end{mathpar}

is interpreted to mean the process derived from P by replacing (in a
capture-avoiding manner) each occurrence of $x$ in $S$ by $y$. For example,

\begin{mathpar}
  P\{ \quotep{\procn{x}|\procn{x}} / x : x \in \freenames{P} \}
\end{mathpar}

will replace each (occurrence) of a free name $x$ in $P$ by
$\quotep{\procn{x}|\procn{x}}$.

Also, we will avail ourselves of the notation $x^{L}$ and $x^{R}$ to
denote injections of a name into disjoint copies of the name
space. There are numerous ways to accomplish this. One example can be
found in \cite{MeredithR05}. This notation overloads to vectors of
names: $\vec{x}^{\pi} := (x_{i}^{\pi} \; : \; 0 \leq i < |\vec{x}| )$ where $\pi \in \{L,R\}$.

We also use $P^{\Box} := P|\Box$.

In \cite{MeredithR05} an interpretation of the new operator is
given. It turns out that there are several possible interpretations
all enjoying the requisite algebraic properties of the operator (see
\cite{milner91polyadicpi}). We will therefore make liberal use of
$(\nu\; \vec{x})P$.

% subsection the_syntax_and_semantics_of_the_notation_system (end)   

\input{qm2pi.qmops} 

\input{qm2pi.sterngerlach} 

\input{qm2pi.metric} 

% section concurrent_process_calculi (end)

%\input{qm2pi.proofsketch}

% section proof sketch (end)

%\input{qm2pi.slviaknots} 

% section spatial logic via knots (end)

\input{qm2pi.conclusion}

% section conclusion (end)

%\input{qm2pi.dtcodes} 

% section wiring algorithm (end)

\input{qm2pi.ack} 

% section acknowledgments (end)

\newpage


\bibliographystyle{plain}   
\bibliography{../../biblios/main.bib}

\input{qm2pi.rhodetails}

\end{document}

 

% section wiring algorithm (end)

\documentclass[12pt]{llncs}
%\documentclass{jktr}

\usepackage[pdftex]{hyperref}                   
\usepackage {listings}
\usepackage {mathpartir}
\usepackage{bcprules}
%\usepackage{listings}
                       
\usepackage{graphicx} 
%\usepackage[margins=2.5cm,nohead,nofoot]{geometry}
%\usepackage{geometry}
\usepackage{amsfonts}
\usepackage{amstext}
\usepackage{latexsym}
\usepackage{amssymb}
\usepackage{color}


%\include{myPreamble}
\include{qm2pi.local} 

%\ifpdf
%\usepackage[pdftex]{graphicx}
%\else
%\usepackage{graphicx}
%\fi

 % \ifpdf
%  \usepackage{pdfsync}
%  \if


%\title{Brief Article}
%\author{David F. Snyder}
%\author{L.G. Meredith}

%\address{Dept. of Math., Texas State University--San Marcos, San Marcos, TX 78666}
       
\pagestyle{empty}


\begin{document}

\lstset{language=[Objective]Caml,frame=shadowbox}

\input{qm2pi.front}

% section front matter (end)

\input{qm2pi.intro} 
 
% section introduction (end)

% \input{qm2pi.knotations} 

% section notation (end)

\input{qm2pi.process.calculi} 

% section concurrent_process_calculi_and_spatial_logics_ (end)
    
%\input{qm2pi.knots2pi} 

%\input{qm2pi.trefoil} 

%\input{qm2pi.mainthm} 

% subsection basic_interpretation (end)

%\input{qm2pi.rho.presentation} 
\subsection{The syntax and semantics of the notation system}\label{sub:the_syntax_and_semantics_of_the_notation_system} % (fold)

We now summarize a technical presentation of the calculus that
embodies our theory of dynamics. The typical presentation of such a
calculus follows the style of giving generators and relations on
them. The grammar, below, describing term constructors, freely
generates the set of processes, $\Proc$. This set is then quotiented
by a relation known as structural congruence and it is over this set
that the notion of dynamics is expressed. This presentation is
essentially that of \cite{MeredithR05} with the addition of
polyadicity and summation. For readability we have relegated some of
the technical subtleties to an appendix.

\subsubsection{Process grammar}\label{subsub:process_grammar}

\begin{mathpar}
  \inferrule* [lab=synchronization] {} {{M} \bc \pzero \;|\; x?F \;|\; x!C }
  \and
  \inferrule* [lab=abstraction] {} {{F} \bc (x)P}
  \and
  \inferrule* [lab=concretion] {} {{C} \bc \langle Q \rangle}
  \and
  \inferrule* [lab=process] {} {{P,Q} \bc M \;| \;P|Q \;|\; @{x}}
  \and
  \inferrule* [lab=name] {} {{x} \bc \quotep{P}}
\end{mathpar} 

Note that $\vec{x}$ (resp. $\vec{P}$) denotes a vector of names
(resp. processes) of length $|\vec{x}|$ (resp. $|\vec{P}|$). We adopt
the following useful abbreviations.

\begin{mathpar}
   x?(\vec{y}).P := x.(\vec{y})P \and  x\clift{\vec{P}} := x.\clift{\vec{P}}
   \and x!(y) := \lift{x}{\dropn{y}}
   \and \Pi_{i=0}^{n-1}P_i := P_0 | \ldots | P_{n-1}
\end{mathpar}

\subsubsection{Structural congruence}

\paragraph{Free and bound names and alpha-equivalence.} At the
core of structural equivalence is alpha-equivalence which identifies
process that are the same up to a change of variable. Formally, we
recognize the distinction between free and bound names. The free names
of a process, $\freenames{P}$, may be calculated recursively as
follows:

\begin{mathpar}
\freenames{\pzero} := \emptyset
  \and \\
  \freenames{x?(y).P} := \{ x \} \cup (\freenames{P} \setminus \{ y \})
  \and 
  \freenames{x!\langle P \rangle} := \{ x \} \cup \{ P \} 
  \and \\
  \freenames{P|Q} := \freenames{P} \cup \freenames{Q}
  \and \\
  \freenames{@{x}} := \{ x \}
\end{mathpar}

$\pi$
$\quotep{\pi}$

$\freenames{-} : \pi \to \mathcal{P}(\quotep{\pi})$

\begin{eqnarray*}
  \freenames{\pzero} & := & \emptyset \\
  \freenames{x?(y).P} & := & \{ x \} \cup (\freenames{P} \setminus \{ y \}) \\
  \freenames{x!\langle P \rangle} & := & \{ x \} \cup \{ P \} \\
  \freenames{P|Q} & := & \freenames{P} \cup \freenames{Q} \\
  \freenames{\dropn{x}} & := & \{ x \}
\end{eqnarray*}

The bound names of a process, $\boundnames{P}$, are those names occurring in $P$
that are not free. For example, in $x?(y).0$, the name $x$ is free, while $y$ is bound.

\begin{mathpar}
  \inferrule* [lab=monoidal-laws] {} { P|Q \equiv Q|P \and P|0 \equiv P \and P|(Q|R) \equiv (P|Q)|R }
\end{mathpar}

\begin{mathpar}
  \inferrule* [lab=alpha-equivalence] {} { (x)P \equiv (y)P\{y/x\} \and y \not\in \freenames{P} }
\end{mathpar}

\begin{definition}
Then two processes, $P,Q$, are alpha-equivalent if $P = Q\{\vec{y}/\vec{x}\}$ for
some $\vec{x} \in \boundnames{Q},\vec{y} \in \boundnames{P}$, where $Q\{\vec{y}/\vec{x}\}$
denotes the capture-avoiding substitution of $\vec{y}$ for $\vec{x}$ in $Q$.
\end{definition}

\begin{definition}
  The {\em structural congruence} \cite{SangiorgiWalker} , $\equiv$,
  between processes is the least congruence containing
  alpha-equivalence, satisfying the abelian monoid laws
  (associativity, commutativity and $\pzero$ as identity) for parallel
  composition $|$ and for summation $+$.
\end{definition}

\subsection{Name equivalence}

We take name equivalence, written $\nameeq$, to be the smallest
equivalence relation generated by the following rules.

\begin{mathpar}
\inferrule*[lab=Quote-drop]
{ }
{ \quotep{@{x}} \nameeq x }

\inferrule*[lab=Struct-equiv]
{ P \scong Q }
{ \quotep{P} \nameeq \quotep{Q} }
\end{mathpar}

The astute reader will have noticed that the mutual recursion of names
and processes imposes a mutual recursion on alpha-equivalence and
structural equivalence via name-equivalence. Fortunately, all of this
works out pleasantly and we may calculate in the natural way, free of
concern. The reader interested in the details is referred to the
appendix \ref{appendix:rho_details}.

\subsection{Substitution}

We use $\Proc$ for the set of processes, $\QProc$ for the set of
names, and $\id{\{}\vec{y} / \vec{x} \id{\}}$ to denote partial maps,
$s : \QProc \rightarrow \QProc$. A map, $s$ lifts, uniquely, to a map
on process terms, $\widehat{s} : \Proc \rightarrow \Proc$ by the
following equations.

\begin{mathpar}
  (0) \psubstp{Q}{P} := 0 \\
  (R \juxtap S) \psubstp{Q}{P}
  :=    
  (R)\psubstp{Q}{P} \juxtap (S) \psubstp{Q}{P} \\
  (x?(y).R) \psubstp{Q}{P}    
  :=    
  (x)\substp{Q}{P} (z)\concat( (R \psubstn{z}{y}) \psubstp{Q}{P} ) \\
  (\lift{x}{R}) \psubstp{Q}{P}  
  :=
  \lift{(x)\substp{Q}{P}}{ R \psubstp{Q}{P} } \\
%   (\dropn{x})  \psubstp{Q}{P}       
%   := 
%   \left\{ 
%     \begin{array}{ccc} 
%       \dropn{\quotep{Q}} & & x \nameeq \quotep{P} \\
%       \dropn{x} & & otherwise \\
%     \end{array}
%   \right. 
  (\dropn{x})  \psubstp{Q}{P}       
  := 
  \left\{ 
    \begin{array}{ccc} 
      Q & & x \nameeq \quotep{P} \\
      \dropn{x} & & otherwise \\
    \end{array}
  \right.
\end{mathpar}
 

where

\begin{eqnarray}
  (x)\id{\{} \lpquote Q \rpquote / \lpquote P \rpquote \id{\}}            = 
  \left\{ 
    \begin{array}{ccc}
      \lpquote Q \rpquote & & x \nameeq \lpquote P \rpquote \\
      x & & otherwise \\
    \end{array}
  \right. \nonumber
\end{eqnarray}

and $z$ is chosen distinct from $\quotep{P}$, $\quotep{Q}$, the free
names in $Q$, and all the names in $R$. Our $\alpha$-equivalence will
be built in the standard way from this substitution.

\begin{remark}\label{rem:no_self_referential_names}
  One consequence of these definitions is that $\forall P. \quotep{P}
  \not\in \freenames{P}$.
\end{remark}

\subsection{ Dynamic quote: an example }

Anticipating something of what's to come, consider applying the
substitution, $\widehat{\id{\{}u / z \id{\}}}$, to the following pair
of processes, $\lift{w}{y!(z)}$ and $w[ \lpquote y!(z) \rpquote ]$.

\begin{eqnarray}
	\lift{w}{y!(z)}\widehat{\id{\{}u / z \id{\}}}
		& = &
		\lift{w}{y!(u)} \nonumber\\
	w[ \lpquote y!(z) \rpquote ] \widehat{ \id{\{}u / z \id{\}} }
		& = &
		w[ \lpquote y!(z) \rpquote ] \nonumber
\end{eqnarray}

Because the body of the process between quotes is impervious to
substitution, we get radically different answers. In fact, by
examining the first process in an input context,
e.g. $x?(z).\lift{w}{y!(z)}$, we see that the process under the lift
operator may be shaped by prefixed inputs binding a name inside it. In
this sense, the lift operator will be seen as a way to dynamically
construct processes before reifying them as names.

Finally equipped with these standard features we can present the
dynamics of the calculus.

\subsubsection{Operational semantics} 

Finally, we introduce the computational dynamics. What marks these
algebras as distinct from other more traditionally studied algebraic
structures, e.g. vector spaces or polynomial rings, is the manner in
which dynamics is captured. In traditional structures, dynamics is typically
expressed through morphisms between such structures, as in linear maps
between vector spaces or morphisms between rings. In algebras
associated with the semantics of computation, the dynamics is
expressed as part of the algebraic structure itself, through a
reduction reduction relation typically denoted by $\red$. Below, we
give a recursive presentation of this relation for the calculus used
in the encoding.

$\red \subseteq \pi \times \pi$
$\red : \pi \to \mathcal{P}(\pi)$

\begin{mathpar}
  \inferrule* [lab=Comm] { \textsf{match}( x_{src}, x_{trgt} ) } { x_{trgt}?(y)P \; | \; x_{src}!\langle {Q} \rangle \red P\{\quotep{Q}/y}\} }
  \and \\
  \inferrule* [lab=Par] {{P} \red {P}'} {{{P} | {Q}} \red {{P}' | {Q}}}
  \and
  \inferrule* [lab=Equiv]{{{P} \scong {P}'} \andalso {{P}' \red {Q}'} \andalso {{Q}' \scong {Q}}}{{P} \red {Q}}
\end{mathpar}

\begin{eqnarray*}
  match_{\equiv} (\quotep{P},\quotep{Q}) & := & P \equiv Q \\
  match_{\dagger}(\quotep{P},\quotep{Q}) & := & \forall R. P|Q \red^{*} R => R \red^{*} 0 \\
  match_{K}(\quotep{P},\quotep{Q}) & := & K \mbox{ for some context } K
\end{eqnarray*}

$u?(x)P | u!\langle Q \rangle \red P\{\quotep{Q}/x\}$

%We write $\wred$ for $\red^*$, and $P\red$ if $\exists Q $ such that $ P \red Q$.
We write $P\red$ if $\exists Q $ such that $ P \red Q$ and $P\not\red$, otherwise.

\section{Replication}

As mentioned before, it is known that replication (and hence
recursion) can be implemented in a higher-order process algebra
\cite{SangiorgiWalker}. As our first example of calculation with the
machinery thus far presented we give the construction explicitly in
the {\rhoc}.

\begin{eqnarray}
	D_{x} & := & \prefix{x}{y}{(\binpar{\outputp{x}{y}}{@{y}})} \nonumber\\
	\bangp_{x}{P} & := & \binpar{{x}!\langle{\binpar{D_{x}}{P}}\rangle}{D_{x}} \nonumber
\end{eqnarray}

\begin{eqnarray}
	\bangp_{x}{P} & & \nonumber\\
	=
	& {x}!\langle{(\prefix{x}{y}{(\outputp{x}{y} | @{y})) | P}}\rangle 
	      | \prefix{x}{y}{(\outputp{x}{y} | @{y})} & \nonumber\\
	\red
	& (\outputp{x}{y} | @{y})\substn{\quotep{(\prefix{x}{y}{(@{y} | \outputp{x}{y})) | P}}}{y} & \nonumber\\
	=
	& \outputp{x}{\quotep{(\prefix{x}{y}{(\outputp{x}{y} | @{y})) | P}}}
	  | {(\prefix{x}{y}{(\outputp{x}{y} | @{y})) | P}} & \nonumber\\
	\red
	& \ldots & \nonumber\\
	\red^*
	& P | P | \ldots & \nonumber
\end{eqnarray}

Of course, this encoding, as an implementation, runs away, unfolding
$\bangp{P}$ eagerly. A lazier and more implementable replication
operator, restricted to input-guarded processes, may be obtained as follows.

\begin{eqnarray}
\bangp{\prefix{u}{v}{P}} 
	:= 
	\binpar{\lift{x}{\prefix{u}{v}{(\binpar{D(x)}{P})}}}{D(x)} \nonumber
\end{eqnarray}

\begin{remark}
  Note that the lazier definition still does not deal with summation
  or mixed summation (i.e. sums over input and output). The reader is
  invited to construct definitions of replication that deal with these
  features. 

  Further, the definitions are parameterized in a name, $x$. Can you,
  gentle reader, make a definition that eliminates this parameter and
  guarantees no accidental interaction between the replication
  machinery and the process being replicated -- i.e. no accidental
  sharing of names used by the process to get its work done and the
  name(s) used by the replication to effect copying. This latter
  revision of the definition of replication is crucial to obtaining
  the expected identity $!!P \sim !P$.
\end{remark}

\begin{remark}\label{rem:paradoxical_combinator}
  The reader familiar with the lambda calculus will have noticed the
  similarity between $D$ and the paradoxical combinator.

  [Ed. note: the existence of this seems to suggest we have to be more
  restrictive on the set of processes and names we admit if we are to
  support no-cloning.]
\end{remark}

\subsubsection{Bisimulation}

The computational dynamics gives rise to another kind of equivalence,
the equivalence of computational behavior. As previously mentioned
this is typically captured \emph{via} some form of bisimulation.

% The notion we use in this paper is weak barbed bisimulation
% \cite{milner91polyadicpi}.

The notion we use in this paper is derived from weak barbed
bisimulation \cite{milner91polyadicpi}. 

\begin{definition}
An \emph{observation relation}, $\downarrow_{\mathcal N}$, over a set
of names, $\mathcal N$, is the smallest relation satisfying the rules
below.

\infrule[Out-barb]{y \in {\mathcal N}, \; x \nameeq y}
		  {\outputp{x}{v} \downarrow_{\mathcal N} x}
\infrule[Par-barb]{\mbox{$P\downarrow_{\mathcal N} x$ or $Q\downarrow_{\mathcal N} x$}}
		  {\binpar{P}{Q} \downarrow_{\mathcal N} x}

We write $P \Downarrow_{\mathcal N} x$ if there is $Q$ such that 
$P \wred Q$ and $Q \downarrow_{\mathcal N} x$.
\end{definition}

\begin{definition}
%\label{def.bbisim}
An  ${\mathcal N}$-\emph{barbed bisimulation} over a set of names, ${\mathcal N}$, is a symmetric binary relation 
${\mathcal S}_{\mathcal N}$ between agents such that $P\rel{S}_{\mathcal N}Q$ implies:
\begin{enumerate}
\item If $P \red P'$ then $Q \wred Q'$ and $P'\rel{S}_{\mathcal N} Q'$.
\item If $P\downarrow_{\mathcal N} x$, then $Q\Downarrow_{\mathcal N} x$.
\end{enumerate}
$P$ is ${\mathcal N}$-barbed bisimilar to $Q$, written
$P \wbbisim_{\mathcal N} Q$, if $P \rel{S}_{\mathcal N} Q$ for some ${\mathcal N}$-barbed bisimulation ${\mathcal S}_{\mathcal N}$.
\end{definition}

$\mathcal{R} \subseteq \pi \times \pi$

$P \mathcal{R} Q => \forall P'. P \red P' \Rightarrow \exists Q'. Q \red Q', P' \mathcal{R} Q'$

$P \vdash x \Rightarrow Q \vdash x$

\begin{mathpar}
  \inferrule*[lab=Out-barb]{x \nameeq y}{{y}!\langle{Q}\rangle \vdash x}
  \and
  \inferrule*[lab=Par-barb]{\mbox{$P\vdash x$ or $Q\vdash x$}}{\binpar{P}{Q} \vdash x}
\end{mathpar}

\subsubsection{Contexts}

One of the principle advantages of computational calculi like the
$\pi$-calculus is a well-defined notion of context,
contextual-equivalence and a correlation between
contextual-equivalence and notions of bisimulation. The notion of
context allows the decomposition of a process into (sub-)process and
its syntactic environment, its context. Thus, a context may be
thought of as a process with a ``hole'' (written $\Box$) in it. The
application of a context $M$ to a process $P$, written $M[P]$, is
tantamount to filling the hole in $M$ with $P$. In this paper we do
not need the full weight of this theory, but do make use of the notion
of context in the proof the main theorem. 

\begin{mathpar}
  \inferrule* [lab=summation] {} {{M_{M},M_{N}} \bc \Box \;|\; x.M_{A} \;|\; M_{M}+M_{N}}
  \and
  \inferrule* [lab=agent] {} {{M_{A}} \bc (\vec{x})M_{P} \;| \; \clift{P_0,\ldots,M_{P},\ldots,P_N}}
  \and \\
  \inferrule* [lab=process] {} {{M_{P}} \bc M_{N} \;| \;P|M_{P} }
\end{mathpar} 

\begin{mathpar}
  \inferrule* [lab=sychronization] {} {M_{N} \bc \Box \;|\; x?M_{F} \;|\; x!M_{C}}
  \and
  \inferrule* [lab=abstraction] {} {{M_{F}} \bc (x)M_{P} }
  \and
  \inferrule* [lab=concretion] {} {{M_{C}} \bc \langle M_{P} \rangle }
  \and \\
  \inferrule* [lab=process] {} {{M_{P}} \bc M_{N} \;| \;P|M_{P} }
\end{mathpar}

\begin{definition}[contextual application] Given a context $M$, and
  process $P$, we define the \emph{contextual application}, $M[P] :=
  M\{P/\Box\}$. That is, the contextual application of M to P is the
  substitution of $P$ for $\Box$ in $M$.
\end{definition}

$\meaningof{-} : L \to \mathcal{P}(\pi)$

\begin{mathpar}
  \inferrule* [lab=collection] {} {\meaningof{true} = \pi, \and \meaningof{~E} = \pi \setminus \meaningof{E}, \and \meaningof{E_{1} \& E_{2}} = \meaningof{E_{1}} \cap \meaningof{E_{2}}}
\end{mathpar}

\begin{mathpar}
  \inferrule* [lab=structure] {} {\meaningof{0} = \{ P \in \pi | P \equiv 0 \}, \and \\ \meaningof{E_1 | E_2} = \{ P \in \pi | P \equiv P_{1} | P_{2}, P_{1} \in \meaningof{E_{1}}, P_{2} \in \meaningof{E_2}\} }
\end{mathpar}

\begin{mathpar}
 \inferrule* [lab=behavior] {} {\meaningof{\langle a?b \rangle E} = \{ P \in \pi | P \equiv Q | u?(y)P', \\ \and \\\\ \and \\ \;\;\; u \in \meaningof{a}, \forall z.P'\{z/y\} \in \meaningof{E\{z/b\}}\}, \and \\ \meaningof{a!E} = \{ P \in \pi | P \equiv Q | x!\langle P' \rangle, x \in \meaningof{a} P' \in \meaningof{E}\} }
\end{mathpar}

\begin{mathpar}
 \inferrule* [lab=nominal] {} {\meaningof{\quotep{E}} = \{ \quotep{P} \in \quotep{\pi} | P \in \meaningof{E} \}, \and \meaningof{\quotep{P}} = \{ \quotep{Q} \in \quotep{\pi} | P \equiv Q \} \and \\ \meaningof{@\quotep{E}} = \{ P \in \pi | P \equiv @x, x \in \meaningof{E} \}}
\end{mathpar}

\begin{eqnarray*}
  \\
  \meaningof{-} : TS \to ST
\end{eqnarray*}

\begin{eqnarray*}
  \\
  L : TS \to ST
\end{eqnarray*}

\begin{eqnarray*}
  \\
  P \models E \iff P \in \meaningof{E}
\end{eqnarray*}

\begin{eqnarray*}
  P \approx_{L} Q \iff \forall E \in L. P \models E \iff Q \models E
\end{eqnarray*}

\begin{eqnarray*}
  P \approx_{K} Q
\end{eqnarray*}

\begin{eqnarray*}
  P \approx Q
\end{eqnarray*}

$\approx_{K} = \approx = \approx_{L}$

\subsubsection{Contextual duality}

Note that contexts extend the quotation operation to a family of
operations from processes to names. Given a context, $M$, we can
define a \emph{nominal context}, $\quotep{M}$ by $\quotep{M}[P] :=
\quotep{M[P]}$. To foreshadow what is to come we observe that these
operations enjoy a duality with processes very much like the duality
between vectors and maps from vectors to scalars.

Further, because the calculus is essentially higher-order, we have a
correspondence between contexts and processes. More specifically,
given a name $x$ and a context $M$ we can construct $M^{*}_{x}$ such
that 

\begin{mathpar}
  M^{*}_{x} | \lift{x}{P} \red M[P]
\end{mathpar}

namely,

\begin{mathpar}
  M^{*}_{x} := x?(u).M[\dropn{u}]
\end{mathpar}

The dependence of $M^{*}_{x}$ on a name makes it an abstraction, 

\begin{mathpar}
  M^{*} := (x)x?(u).M[\dropn{u}]
\end{mathpar}

\subsection{Additional notation}

It will sometimes be convenient to denote the process a name
quotes. We already have the notation $x = \quotep{P}$, but it will be
convenient to introduce an alternate notation, $\procn{x}$, when we
want to emphasize the connection to the use of the name. Note that, by
virtue of name equivalence, $\quotep{\procn{x}} \nameeq x$; so, the
notation is consistent with previous definitions.

Further, because names have structure it is possible to effect
substitutions on the basis of that structure. This means we need to
upgrade our notation for substitutions, which we accomplish by
adapting comprehension notation. Thus,

\begin{mathpar}
  P\{ y / x : x \in S \}
\end{mathpar}

is interpreted to mean the process derived from P by replacing (in a
capture-avoiding manner) each occurrence of $x$ in $S$ by $y$. For example,

\begin{mathpar}
  P\{ \quotep{\procn{x}|\procn{x}} / x : x \in \freenames{P} \}
\end{mathpar}

will replace each (occurrence) of a free name $x$ in $P$ by
$\quotep{\procn{x}|\procn{x}}$.

Also, we will avail ourselves of the notation $x^{L}$ and $x^{R}$ to
denote injections of a name into disjoint copies of the name
space. There are numerous ways to accomplish this. One example can be
found in \cite{MeredithR05}. This notation overloads to vectors of
names: $\vec{x}^{\pi} := (x_{i}^{\pi} \; : \; 0 \leq i < |\vec{x}| )$ where $\pi \in \{L,R\}$.

We also use $P^{\Box} := P|\Box$.

In \cite{MeredithR05} an interpretation of the new operator is
given. It turns out that there are several possible interpretations
all enjoying the requisite algebraic properties of the operator (see
\cite{milner91polyadicpi}). We will therefore make liberal use of
$(\nu\; \vec{x})P$.

% subsection the_syntax_and_semantics_of_the_notation_system (end)   

\input{qm2pi.qmops} 

\input{qm2pi.sterngerlach} 

\input{qm2pi.metric} 

% section concurrent_process_calculi (end)

%\input{qm2pi.proofsketch}

% section proof sketch (end)

%\input{qm2pi.slviaknots} 

% section spatial logic via knots (end)

\input{qm2pi.conclusion}

% section conclusion (end)

%\input{qm2pi.dtcodes} 

% section wiring algorithm (end)

\input{qm2pi.ack} 

% section acknowledgments (end)

\newpage


\bibliographystyle{plain}   
\bibliography{../../biblios/main.bib}

\input{qm2pi.rhodetails}

\end{document}

 

% section acknowledgments (end)

\newpage


\bibliographystyle{plain}   
\bibliography{../../biblios/main.bib}

\documentclass[12pt]{llncs}
%\documentclass{jktr}

\usepackage[pdftex]{hyperref}                   
\usepackage {listings}
\usepackage {mathpartir}
\usepackage{bcprules}
%\usepackage{listings}
                       
\usepackage{graphicx} 
%\usepackage[margins=2.5cm,nohead,nofoot]{geometry}
%\usepackage{geometry}
\usepackage{amsfonts}
\usepackage{amstext}
\usepackage{latexsym}
\usepackage{amssymb}
\usepackage{color}


%\include{myPreamble}
\include{qm2pi.local} 

%\ifpdf
%\usepackage[pdftex]{graphicx}
%\else
%\usepackage{graphicx}
%\fi

 % \ifpdf
%  \usepackage{pdfsync}
%  \if


%\title{Brief Article}
%\author{David F. Snyder}
%\author{L.G. Meredith}

%\address{Dept. of Math., Texas State University--San Marcos, San Marcos, TX 78666}
       
\pagestyle{empty}


\begin{document}

\lstset{language=[Objective]Caml,frame=shadowbox}

\input{qm2pi.front}

% section front matter (end)

\input{qm2pi.intro} 
 
% section introduction (end)

% \input{qm2pi.knotations} 

% section notation (end)

\input{qm2pi.process.calculi} 

% section concurrent_process_calculi_and_spatial_logics_ (end)
    
%\input{qm2pi.knots2pi} 

%\input{qm2pi.trefoil} 

%\input{qm2pi.mainthm} 

% subsection basic_interpretation (end)

%\input{qm2pi.rho.presentation} 
\subsection{The syntax and semantics of the notation system}\label{sub:the_syntax_and_semantics_of_the_notation_system} % (fold)

We now summarize a technical presentation of the calculus that
embodies our theory of dynamics. The typical presentation of such a
calculus follows the style of giving generators and relations on
them. The grammar, below, describing term constructors, freely
generates the set of processes, $\Proc$. This set is then quotiented
by a relation known as structural congruence and it is over this set
that the notion of dynamics is expressed. This presentation is
essentially that of \cite{MeredithR05} with the addition of
polyadicity and summation. For readability we have relegated some of
the technical subtleties to an appendix.

\subsubsection{Process grammar}\label{subsub:process_grammar}

\begin{mathpar}
  \inferrule* [lab=synchronization] {} {{M} \bc \pzero \;|\; x?F \;|\; x!C }
  \and
  \inferrule* [lab=abstraction] {} {{F} \bc (x)P}
  \and
  \inferrule* [lab=concretion] {} {{C} \bc \langle Q \rangle}
  \and
  \inferrule* [lab=process] {} {{P,Q} \bc M \;| \;P|Q \;|\; @{x}}
  \and
  \inferrule* [lab=name] {} {{x} \bc \quotep{P}}
\end{mathpar} 

Note that $\vec{x}$ (resp. $\vec{P}$) denotes a vector of names
(resp. processes) of length $|\vec{x}|$ (resp. $|\vec{P}|$). We adopt
the following useful abbreviations.

\begin{mathpar}
   x?(\vec{y}).P := x.(\vec{y})P \and  x\clift{\vec{P}} := x.\clift{\vec{P}}
   \and x!(y) := \lift{x}{\dropn{y}}
   \and \Pi_{i=0}^{n-1}P_i := P_0 | \ldots | P_{n-1}
\end{mathpar}

\subsubsection{Structural congruence}

\paragraph{Free and bound names and alpha-equivalence.} At the
core of structural equivalence is alpha-equivalence which identifies
process that are the same up to a change of variable. Formally, we
recognize the distinction between free and bound names. The free names
of a process, $\freenames{P}$, may be calculated recursively as
follows:

\begin{mathpar}
\freenames{\pzero} := \emptyset
  \and \\
  \freenames{x?(y).P} := \{ x \} \cup (\freenames{P} \setminus \{ y \})
  \and 
  \freenames{x!\langle P \rangle} := \{ x \} \cup \{ P \} 
  \and \\
  \freenames{P|Q} := \freenames{P} \cup \freenames{Q}
  \and \\
  \freenames{@{x}} := \{ x \}
\end{mathpar}

$\pi$
$\quotep{\pi}$

$\freenames{-} : \pi \to \mathcal{P}(\quotep{\pi})$

\begin{eqnarray*}
  \freenames{\pzero} & := & \emptyset \\
  \freenames{x?(y).P} & := & \{ x \} \cup (\freenames{P} \setminus \{ y \}) \\
  \freenames{x!\langle P \rangle} & := & \{ x \} \cup \{ P \} \\
  \freenames{P|Q} & := & \freenames{P} \cup \freenames{Q} \\
  \freenames{\dropn{x}} & := & \{ x \}
\end{eqnarray*}

The bound names of a process, $\boundnames{P}$, are those names occurring in $P$
that are not free. For example, in $x?(y).0$, the name $x$ is free, while $y$ is bound.

\begin{mathpar}
  \inferrule* [lab=monoidal-laws] {} { P|Q \equiv Q|P \and P|0 \equiv P \and P|(Q|R) \equiv (P|Q)|R }
\end{mathpar}

\begin{mathpar}
  \inferrule* [lab=alpha-equivalence] {} { (x)P \equiv (y)P\{y/x\} \and y \not\in \freenames{P} }
\end{mathpar}

\begin{definition}
Then two processes, $P,Q$, are alpha-equivalent if $P = Q\{\vec{y}/\vec{x}\}$ for
some $\vec{x} \in \boundnames{Q},\vec{y} \in \boundnames{P}$, where $Q\{\vec{y}/\vec{x}\}$
denotes the capture-avoiding substitution of $\vec{y}$ for $\vec{x}$ in $Q$.
\end{definition}

\begin{definition}
  The {\em structural congruence} \cite{SangiorgiWalker} , $\equiv$,
  between processes is the least congruence containing
  alpha-equivalence, satisfying the abelian monoid laws
  (associativity, commutativity and $\pzero$ as identity) for parallel
  composition $|$ and for summation $+$.
\end{definition}

\subsection{Name equivalence}

We take name equivalence, written $\nameeq$, to be the smallest
equivalence relation generated by the following rules.

\begin{mathpar}
\inferrule*[lab=Quote-drop]
{ }
{ \quotep{@{x}} \nameeq x }

\inferrule*[lab=Struct-equiv]
{ P \scong Q }
{ \quotep{P} \nameeq \quotep{Q} }
\end{mathpar}

The astute reader will have noticed that the mutual recursion of names
and processes imposes a mutual recursion on alpha-equivalence and
structural equivalence via name-equivalence. Fortunately, all of this
works out pleasantly and we may calculate in the natural way, free of
concern. The reader interested in the details is referred to the
appendix \ref{appendix:rho_details}.

\subsection{Substitution}

We use $\Proc$ for the set of processes, $\QProc$ for the set of
names, and $\id{\{}\vec{y} / \vec{x} \id{\}}$ to denote partial maps,
$s : \QProc \rightarrow \QProc$. A map, $s$ lifts, uniquely, to a map
on process terms, $\widehat{s} : \Proc \rightarrow \Proc$ by the
following equations.

\begin{mathpar}
  (0) \psubstp{Q}{P} := 0 \\
  (R \juxtap S) \psubstp{Q}{P}
  :=    
  (R)\psubstp{Q}{P} \juxtap (S) \psubstp{Q}{P} \\
  (x?(y).R) \psubstp{Q}{P}    
  :=    
  (x)\substp{Q}{P} (z)\concat( (R \psubstn{z}{y}) \psubstp{Q}{P} ) \\
  (\lift{x}{R}) \psubstp{Q}{P}  
  :=
  \lift{(x)\substp{Q}{P}}{ R \psubstp{Q}{P} } \\
%   (\dropn{x})  \psubstp{Q}{P}       
%   := 
%   \left\{ 
%     \begin{array}{ccc} 
%       \dropn{\quotep{Q}} & & x \nameeq \quotep{P} \\
%       \dropn{x} & & otherwise \\
%     \end{array}
%   \right. 
  (\dropn{x})  \psubstp{Q}{P}       
  := 
  \left\{ 
    \begin{array}{ccc} 
      Q & & x \nameeq \quotep{P} \\
      \dropn{x} & & otherwise \\
    \end{array}
  \right.
\end{mathpar}
 

where

\begin{eqnarray}
  (x)\id{\{} \lpquote Q \rpquote / \lpquote P \rpquote \id{\}}            = 
  \left\{ 
    \begin{array}{ccc}
      \lpquote Q \rpquote & & x \nameeq \lpquote P \rpquote \\
      x & & otherwise \\
    \end{array}
  \right. \nonumber
\end{eqnarray}

and $z$ is chosen distinct from $\quotep{P}$, $\quotep{Q}$, the free
names in $Q$, and all the names in $R$. Our $\alpha$-equivalence will
be built in the standard way from this substitution.

\begin{remark}\label{rem:no_self_referential_names}
  One consequence of these definitions is that $\forall P. \quotep{P}
  \not\in \freenames{P}$.
\end{remark}

\subsection{ Dynamic quote: an example }

Anticipating something of what's to come, consider applying the
substitution, $\widehat{\id{\{}u / z \id{\}}}$, to the following pair
of processes, $\lift{w}{y!(z)}$ and $w[ \lpquote y!(z) \rpquote ]$.

\begin{eqnarray}
	\lift{w}{y!(z)}\widehat{\id{\{}u / z \id{\}}}
		& = &
		\lift{w}{y!(u)} \nonumber\\
	w[ \lpquote y!(z) \rpquote ] \widehat{ \id{\{}u / z \id{\}} }
		& = &
		w[ \lpquote y!(z) \rpquote ] \nonumber
\end{eqnarray}

Because the body of the process between quotes is impervious to
substitution, we get radically different answers. In fact, by
examining the first process in an input context,
e.g. $x?(z).\lift{w}{y!(z)}$, we see that the process under the lift
operator may be shaped by prefixed inputs binding a name inside it. In
this sense, the lift operator will be seen as a way to dynamically
construct processes before reifying them as names.

Finally equipped with these standard features we can present the
dynamics of the calculus.

\subsubsection{Operational semantics} 

Finally, we introduce the computational dynamics. What marks these
algebras as distinct from other more traditionally studied algebraic
structures, e.g. vector spaces or polynomial rings, is the manner in
which dynamics is captured. In traditional structures, dynamics is typically
expressed through morphisms between such structures, as in linear maps
between vector spaces or morphisms between rings. In algebras
associated with the semantics of computation, the dynamics is
expressed as part of the algebraic structure itself, through a
reduction reduction relation typically denoted by $\red$. Below, we
give a recursive presentation of this relation for the calculus used
in the encoding.

$\red \subseteq \pi \times \pi$
$\red : \pi \to \mathcal{P}(\pi)$

\begin{mathpar}
  \inferrule* [lab=Comm] { \textsf{match}( x_{src}, x_{trgt} ) } { x_{trgt}?(y)P \; | \; x_{src}!\langle {Q} \rangle \red P\{\quotep{Q}/y}\} }
  \and \\
  \inferrule* [lab=Par] {{P} \red {P}'} {{{P} | {Q}} \red {{P}' | {Q}}}
  \and
  \inferrule* [lab=Equiv]{{{P} \scong {P}'} \andalso {{P}' \red {Q}'} \andalso {{Q}' \scong {Q}}}{{P} \red {Q}}
\end{mathpar}

\begin{eqnarray*}
  match_{\equiv} (\quotep{P},\quotep{Q}) & := & P \equiv Q \\
  match_{\dagger}(\quotep{P},\quotep{Q}) & := & \forall R. P|Q \red^{*} R => R \red^{*} 0 \\
  match_{K}(\quotep{P},\quotep{Q}) & := & K \mbox{ for some context } K
\end{eqnarray*}

$u?(x)P | u!\langle Q \rangle \red P\{\quotep{Q}/x\}$

%We write $\wred$ for $\red^*$, and $P\red$ if $\exists Q $ such that $ P \red Q$.
We write $P\red$ if $\exists Q $ such that $ P \red Q$ and $P\not\red$, otherwise.

\section{Replication}

As mentioned before, it is known that replication (and hence
recursion) can be implemented in a higher-order process algebra
\cite{SangiorgiWalker}. As our first example of calculation with the
machinery thus far presented we give the construction explicitly in
the {\rhoc}.

\begin{eqnarray}
	D_{x} & := & \prefix{x}{y}{(\binpar{\outputp{x}{y}}{@{y}})} \nonumber\\
	\bangp_{x}{P} & := & \binpar{{x}!\langle{\binpar{D_{x}}{P}}\rangle}{D_{x}} \nonumber
\end{eqnarray}

\begin{eqnarray}
	\bangp_{x}{P} & & \nonumber\\
	=
	& {x}!\langle{(\prefix{x}{y}{(\outputp{x}{y} | @{y})) | P}}\rangle 
	      | \prefix{x}{y}{(\outputp{x}{y} | @{y})} & \nonumber\\
	\red
	& (\outputp{x}{y} | @{y})\substn{\quotep{(\prefix{x}{y}{(@{y} | \outputp{x}{y})) | P}}}{y} & \nonumber\\
	=
	& \outputp{x}{\quotep{(\prefix{x}{y}{(\outputp{x}{y} | @{y})) | P}}}
	  | {(\prefix{x}{y}{(\outputp{x}{y} | @{y})) | P}} & \nonumber\\
	\red
	& \ldots & \nonumber\\
	\red^*
	& P | P | \ldots & \nonumber
\end{eqnarray}

Of course, this encoding, as an implementation, runs away, unfolding
$\bangp{P}$ eagerly. A lazier and more implementable replication
operator, restricted to input-guarded processes, may be obtained as follows.

\begin{eqnarray}
\bangp{\prefix{u}{v}{P}} 
	:= 
	\binpar{\lift{x}{\prefix{u}{v}{(\binpar{D(x)}{P})}}}{D(x)} \nonumber
\end{eqnarray}

\begin{remark}
  Note that the lazier definition still does not deal with summation
  or mixed summation (i.e. sums over input and output). The reader is
  invited to construct definitions of replication that deal with these
  features. 

  Further, the definitions are parameterized in a name, $x$. Can you,
  gentle reader, make a definition that eliminates this parameter and
  guarantees no accidental interaction between the replication
  machinery and the process being replicated -- i.e. no accidental
  sharing of names used by the process to get its work done and the
  name(s) used by the replication to effect copying. This latter
  revision of the definition of replication is crucial to obtaining
  the expected identity $!!P \sim !P$.
\end{remark}

\begin{remark}\label{rem:paradoxical_combinator}
  The reader familiar with the lambda calculus will have noticed the
  similarity between $D$ and the paradoxical combinator.

  [Ed. note: the existence of this seems to suggest we have to be more
  restrictive on the set of processes and names we admit if we are to
  support no-cloning.]
\end{remark}

\subsubsection{Bisimulation}

The computational dynamics gives rise to another kind of equivalence,
the equivalence of computational behavior. As previously mentioned
this is typically captured \emph{via} some form of bisimulation.

% The notion we use in this paper is weak barbed bisimulation
% \cite{milner91polyadicpi}.

The notion we use in this paper is derived from weak barbed
bisimulation \cite{milner91polyadicpi}. 

\begin{definition}
An \emph{observation relation}, $\downarrow_{\mathcal N}$, over a set
of names, $\mathcal N$, is the smallest relation satisfying the rules
below.

\infrule[Out-barb]{y \in {\mathcal N}, \; x \nameeq y}
		  {\outputp{x}{v} \downarrow_{\mathcal N} x}
\infrule[Par-barb]{\mbox{$P\downarrow_{\mathcal N} x$ or $Q\downarrow_{\mathcal N} x$}}
		  {\binpar{P}{Q} \downarrow_{\mathcal N} x}

We write $P \Downarrow_{\mathcal N} x$ if there is $Q$ such that 
$P \wred Q$ and $Q \downarrow_{\mathcal N} x$.
\end{definition}

\begin{definition}
%\label{def.bbisim}
An  ${\mathcal N}$-\emph{barbed bisimulation} over a set of names, ${\mathcal N}$, is a symmetric binary relation 
${\mathcal S}_{\mathcal N}$ between agents such that $P\rel{S}_{\mathcal N}Q$ implies:
\begin{enumerate}
\item If $P \red P'$ then $Q \wred Q'$ and $P'\rel{S}_{\mathcal N} Q'$.
\item If $P\downarrow_{\mathcal N} x$, then $Q\Downarrow_{\mathcal N} x$.
\end{enumerate}
$P$ is ${\mathcal N}$-barbed bisimilar to $Q$, written
$P \wbbisim_{\mathcal N} Q$, if $P \rel{S}_{\mathcal N} Q$ for some ${\mathcal N}$-barbed bisimulation ${\mathcal S}_{\mathcal N}$.
\end{definition}

$\mathcal{R} \subseteq \pi \times \pi$

$P \mathcal{R} Q => \forall P'. P \red P' \Rightarrow \exists Q'. Q \red Q', P' \mathcal{R} Q'$

$P \vdash x \Rightarrow Q \vdash x$

\begin{mathpar}
  \inferrule*[lab=Out-barb]{x \nameeq y}{{y}!\langle{Q}\rangle \vdash x}
  \and
  \inferrule*[lab=Par-barb]{\mbox{$P\vdash x$ or $Q\vdash x$}}{\binpar{P}{Q} \vdash x}
\end{mathpar}

\subsubsection{Contexts}

One of the principle advantages of computational calculi like the
$\pi$-calculus is a well-defined notion of context,
contextual-equivalence and a correlation between
contextual-equivalence and notions of bisimulation. The notion of
context allows the decomposition of a process into (sub-)process and
its syntactic environment, its context. Thus, a context may be
thought of as a process with a ``hole'' (written $\Box$) in it. The
application of a context $M$ to a process $P$, written $M[P]$, is
tantamount to filling the hole in $M$ with $P$. In this paper we do
not need the full weight of this theory, but do make use of the notion
of context in the proof the main theorem. 

\begin{mathpar}
  \inferrule* [lab=summation] {} {{M_{M},M_{N}} \bc \Box \;|\; x.M_{A} \;|\; M_{M}+M_{N}}
  \and
  \inferrule* [lab=agent] {} {{M_{A}} \bc (\vec{x})M_{P} \;| \; \clift{P_0,\ldots,M_{P},\ldots,P_N}}
  \and \\
  \inferrule* [lab=process] {} {{M_{P}} \bc M_{N} \;| \;P|M_{P} }
\end{mathpar} 

\begin{mathpar}
  \inferrule* [lab=sychronization] {} {M_{N} \bc \Box \;|\; x?M_{F} \;|\; x!M_{C}}
  \and
  \inferrule* [lab=abstraction] {} {{M_{F}} \bc (x)M_{P} }
  \and
  \inferrule* [lab=concretion] {} {{M_{C}} \bc \langle M_{P} \rangle }
  \and \\
  \inferrule* [lab=process] {} {{M_{P}} \bc M_{N} \;| \;P|M_{P} }
\end{mathpar}

\begin{definition}[contextual application] Given a context $M$, and
  process $P$, we define the \emph{contextual application}, $M[P] :=
  M\{P/\Box\}$. That is, the contextual application of M to P is the
  substitution of $P$ for $\Box$ in $M$.
\end{definition}

$\meaningof{-} : L \to \mathcal{P}(\pi)$

\begin{mathpar}
  \inferrule* [lab=collection] {} {\meaningof{true} = \pi, \and \meaningof{~E} = \pi \setminus \meaningof{E}, \and \meaningof{E_{1} \& E_{2}} = \meaningof{E_{1}} \cap \meaningof{E_{2}}}
\end{mathpar}

\begin{mathpar}
  \inferrule* [lab=structure] {} {\meaningof{0} = \{ P \in \pi | P \equiv 0 \}, \and \\ \meaningof{E_1 | E_2} = \{ P \in \pi | P \equiv P_{1} | P_{2}, P_{1} \in \meaningof{E_{1}}, P_{2} \in \meaningof{E_2}\} }
\end{mathpar}

\begin{mathpar}
 \inferrule* [lab=behavior] {} {\meaningof{\langle a?b \rangle E} = \{ P \in \pi | P \equiv Q | u?(y)P', \\ \and \\\\ \and \\ \;\;\; u \in \meaningof{a}, \forall z.P'\{z/y\} \in \meaningof{E\{z/b\}}\}, \and \\ \meaningof{a!E} = \{ P \in \pi | P \equiv Q | x!\langle P' \rangle, x \in \meaningof{a} P' \in \meaningof{E}\} }
\end{mathpar}

\begin{mathpar}
 \inferrule* [lab=nominal] {} {\meaningof{\quotep{E}} = \{ \quotep{P} \in \quotep{\pi} | P \in \meaningof{E} \}, \and \meaningof{\quotep{P}} = \{ \quotep{Q} \in \quotep{\pi} | P \equiv Q \} \and \\ \meaningof{@\quotep{E}} = \{ P \in \pi | P \equiv @x, x \in \meaningof{E} \}}
\end{mathpar}

\begin{eqnarray*}
  \\
  \meaningof{-} : TS \to ST
\end{eqnarray*}

\begin{eqnarray*}
  \\
  L : TS \to ST
\end{eqnarray*}

\begin{eqnarray*}
  \\
  P \models E \iff P \in \meaningof{E}
\end{eqnarray*}

\begin{eqnarray*}
  P \approx_{L} Q \iff \forall E \in L. P \models E \iff Q \models E
\end{eqnarray*}

\begin{eqnarray*}
  P \approx_{K} Q
\end{eqnarray*}

\begin{eqnarray*}
  P \approx Q
\end{eqnarray*}

$\approx_{K} = \approx = \approx_{L}$

\subsubsection{Contextual duality}

Note that contexts extend the quotation operation to a family of
operations from processes to names. Given a context, $M$, we can
define a \emph{nominal context}, $\quotep{M}$ by $\quotep{M}[P] :=
\quotep{M[P]}$. To foreshadow what is to come we observe that these
operations enjoy a duality with processes very much like the duality
between vectors and maps from vectors to scalars.

Further, because the calculus is essentially higher-order, we have a
correspondence between contexts and processes. More specifically,
given a name $x$ and a context $M$ we can construct $M^{*}_{x}$ such
that 

\begin{mathpar}
  M^{*}_{x} | \lift{x}{P} \red M[P]
\end{mathpar}

namely,

\begin{mathpar}
  M^{*}_{x} := x?(u).M[\dropn{u}]
\end{mathpar}

The dependence of $M^{*}_{x}$ on a name makes it an abstraction, 

\begin{mathpar}
  M^{*} := (x)x?(u).M[\dropn{u}]
\end{mathpar}

\subsection{Additional notation}

It will sometimes be convenient to denote the process a name
quotes. We already have the notation $x = \quotep{P}$, but it will be
convenient to introduce an alternate notation, $\procn{x}$, when we
want to emphasize the connection to the use of the name. Note that, by
virtue of name equivalence, $\quotep{\procn{x}} \nameeq x$; so, the
notation is consistent with previous definitions.

Further, because names have structure it is possible to effect
substitutions on the basis of that structure. This means we need to
upgrade our notation for substitutions, which we accomplish by
adapting comprehension notation. Thus,

\begin{mathpar}
  P\{ y / x : x \in S \}
\end{mathpar}

is interpreted to mean the process derived from P by replacing (in a
capture-avoiding manner) each occurrence of $x$ in $S$ by $y$. For example,

\begin{mathpar}
  P\{ \quotep{\procn{x}|\procn{x}} / x : x \in \freenames{P} \}
\end{mathpar}

will replace each (occurrence) of a free name $x$ in $P$ by
$\quotep{\procn{x}|\procn{x}}$.

Also, we will avail ourselves of the notation $x^{L}$ and $x^{R}$ to
denote injections of a name into disjoint copies of the name
space. There are numerous ways to accomplish this. One example can be
found in \cite{MeredithR05}. This notation overloads to vectors of
names: $\vec{x}^{\pi} := (x_{i}^{\pi} \; : \; 0 \leq i < |\vec{x}| )$ where $\pi \in \{L,R\}$.

We also use $P^{\Box} := P|\Box$.

In \cite{MeredithR05} an interpretation of the new operator is
given. It turns out that there are several possible interpretations
all enjoying the requisite algebraic properties of the operator (see
\cite{milner91polyadicpi}). We will therefore make liberal use of
$(\nu\; \vec{x})P$.

% subsection the_syntax_and_semantics_of_the_notation_system (end)   

\input{qm2pi.qmops} 

\input{qm2pi.sterngerlach} 

\input{qm2pi.metric} 

% section concurrent_process_calculi (end)

%\input{qm2pi.proofsketch}

% section proof sketch (end)

%\input{qm2pi.slviaknots} 

% section spatial logic via knots (end)

\input{qm2pi.conclusion}

% section conclusion (end)

%\input{qm2pi.dtcodes} 

% section wiring algorithm (end)

\input{qm2pi.ack} 

% section acknowledgments (end)

\newpage


\bibliographystyle{plain}   
\bibliography{../../biblios/main.bib}

\input{qm2pi.rhodetails}

\end{document}



\end{document}

 

\documentclass[12pt]{llncs}
%\documentclass{jktr}

\usepackage[pdftex]{hyperref}                   
\usepackage {listings}
\usepackage {mathpartir}
\usepackage{bcprules}
%\usepackage{listings}
                       
\usepackage{graphicx} 
%\usepackage[margins=2.5cm,nohead,nofoot]{geometry}
%\usepackage{geometry}
\usepackage{amsfonts}
\usepackage{amstext}
\usepackage{latexsym}
\usepackage{amssymb}
\usepackage{color}


%\include{myPreamble}
\documentclass[12pt]{llncs}
%\documentclass{jktr}

\usepackage[pdftex]{hyperref}                   
\usepackage {listings}
\usepackage {mathpartir}
\usepackage{bcprules}
%\usepackage{listings}
                       
\usepackage{graphicx} 
%\usepackage[margins=2.5cm,nohead,nofoot]{geometry}
%\usepackage{geometry}
\usepackage{amsfonts}
\usepackage{amstext}
\usepackage{latexsym}
\usepackage{amssymb}
\usepackage{color}


%\include{myPreamble}
\include{qm2pi.local} 

%\ifpdf
%\usepackage[pdftex]{graphicx}
%\else
%\usepackage{graphicx}
%\fi

 % \ifpdf
%  \usepackage{pdfsync}
%  \if


%\title{Brief Article}
%\author{David F. Snyder}
%\author{L.G. Meredith}

%\address{Dept. of Math., Texas State University--San Marcos, San Marcos, TX 78666}
       
\pagestyle{empty}


\begin{document}

\lstset{language=[Objective]Caml,frame=shadowbox}

\input{qm2pi.front}

% section front matter (end)

\input{qm2pi.intro} 
 
% section introduction (end)

% \input{qm2pi.knotations} 

% section notation (end)

\input{qm2pi.process.calculi} 

% section concurrent_process_calculi_and_spatial_logics_ (end)
    
%\input{qm2pi.knots2pi} 

%\input{qm2pi.trefoil} 

%\input{qm2pi.mainthm} 

% subsection basic_interpretation (end)

%\input{qm2pi.rho.presentation} 
\subsection{The syntax and semantics of the notation system}\label{sub:the_syntax_and_semantics_of_the_notation_system} % (fold)

We now summarize a technical presentation of the calculus that
embodies our theory of dynamics. The typical presentation of such a
calculus follows the style of giving generators and relations on
them. The grammar, below, describing term constructors, freely
generates the set of processes, $\Proc$. This set is then quotiented
by a relation known as structural congruence and it is over this set
that the notion of dynamics is expressed. This presentation is
essentially that of \cite{MeredithR05} with the addition of
polyadicity and summation. For readability we have relegated some of
the technical subtleties to an appendix.

\subsubsection{Process grammar}\label{subsub:process_grammar}

\begin{mathpar}
  \inferrule* [lab=synchronization] {} {{M} \bc \pzero \;|\; x?F \;|\; x!C }
  \and
  \inferrule* [lab=abstraction] {} {{F} \bc (x)P}
  \and
  \inferrule* [lab=concretion] {} {{C} \bc \langle Q \rangle}
  \and
  \inferrule* [lab=process] {} {{P,Q} \bc M \;| \;P|Q \;|\; @{x}}
  \and
  \inferrule* [lab=name] {} {{x} \bc \quotep{P}}
\end{mathpar} 

Note that $\vec{x}$ (resp. $\vec{P}$) denotes a vector of names
(resp. processes) of length $|\vec{x}|$ (resp. $|\vec{P}|$). We adopt
the following useful abbreviations.

\begin{mathpar}
   x?(\vec{y}).P := x.(\vec{y})P \and  x\clift{\vec{P}} := x.\clift{\vec{P}}
   \and x!(y) := \lift{x}{\dropn{y}}
   \and \Pi_{i=0}^{n-1}P_i := P_0 | \ldots | P_{n-1}
\end{mathpar}

\subsubsection{Structural congruence}

\paragraph{Free and bound names and alpha-equivalence.} At the
core of structural equivalence is alpha-equivalence which identifies
process that are the same up to a change of variable. Formally, we
recognize the distinction between free and bound names. The free names
of a process, $\freenames{P}$, may be calculated recursively as
follows:

\begin{mathpar}
\freenames{\pzero} := \emptyset
  \and \\
  \freenames{x?(y).P} := \{ x \} \cup (\freenames{P} \setminus \{ y \})
  \and 
  \freenames{x!\langle P \rangle} := \{ x \} \cup \{ P \} 
  \and \\
  \freenames{P|Q} := \freenames{P} \cup \freenames{Q}
  \and \\
  \freenames{@{x}} := \{ x \}
\end{mathpar}

$\pi$
$\quotep{\pi}$

$\freenames{-} : \pi \to \mathcal{P}(\quotep{\pi})$

\begin{eqnarray*}
  \freenames{\pzero} & := & \emptyset \\
  \freenames{x?(y).P} & := & \{ x \} \cup (\freenames{P} \setminus \{ y \}) \\
  \freenames{x!\langle P \rangle} & := & \{ x \} \cup \{ P \} \\
  \freenames{P|Q} & := & \freenames{P} \cup \freenames{Q} \\
  \freenames{\dropn{x}} & := & \{ x \}
\end{eqnarray*}

The bound names of a process, $\boundnames{P}$, are those names occurring in $P$
that are not free. For example, in $x?(y).0$, the name $x$ is free, while $y$ is bound.

\begin{mathpar}
  \inferrule* [lab=monoidal-laws] {} { P|Q \equiv Q|P \and P|0 \equiv P \and P|(Q|R) \equiv (P|Q)|R }
\end{mathpar}

\begin{mathpar}
  \inferrule* [lab=alpha-equivalence] {} { (x)P \equiv (y)P\{y/x\} \and y \not\in \freenames{P} }
\end{mathpar}

\begin{definition}
Then two processes, $P,Q$, are alpha-equivalent if $P = Q\{\vec{y}/\vec{x}\}$ for
some $\vec{x} \in \boundnames{Q},\vec{y} \in \boundnames{P}$, where $Q\{\vec{y}/\vec{x}\}$
denotes the capture-avoiding substitution of $\vec{y}$ for $\vec{x}$ in $Q$.
\end{definition}

\begin{definition}
  The {\em structural congruence} \cite{SangiorgiWalker} , $\equiv$,
  between processes is the least congruence containing
  alpha-equivalence, satisfying the abelian monoid laws
  (associativity, commutativity and $\pzero$ as identity) for parallel
  composition $|$ and for summation $+$.
\end{definition}

\subsection{Name equivalence}

We take name equivalence, written $\nameeq$, to be the smallest
equivalence relation generated by the following rules.

\begin{mathpar}
\inferrule*[lab=Quote-drop]
{ }
{ \quotep{@{x}} \nameeq x }

\inferrule*[lab=Struct-equiv]
{ P \scong Q }
{ \quotep{P} \nameeq \quotep{Q} }
\end{mathpar}

The astute reader will have noticed that the mutual recursion of names
and processes imposes a mutual recursion on alpha-equivalence and
structural equivalence via name-equivalence. Fortunately, all of this
works out pleasantly and we may calculate in the natural way, free of
concern. The reader interested in the details is referred to the
appendix \ref{appendix:rho_details}.

\subsection{Substitution}

We use $\Proc$ for the set of processes, $\QProc$ for the set of
names, and $\id{\{}\vec{y} / \vec{x} \id{\}}$ to denote partial maps,
$s : \QProc \rightarrow \QProc$. A map, $s$ lifts, uniquely, to a map
on process terms, $\widehat{s} : \Proc \rightarrow \Proc$ by the
following equations.

\begin{mathpar}
  (0) \psubstp{Q}{P} := 0 \\
  (R \juxtap S) \psubstp{Q}{P}
  :=    
  (R)\psubstp{Q}{P} \juxtap (S) \psubstp{Q}{P} \\
  (x?(y).R) \psubstp{Q}{P}    
  :=    
  (x)\substp{Q}{P} (z)\concat( (R \psubstn{z}{y}) \psubstp{Q}{P} ) \\
  (\lift{x}{R}) \psubstp{Q}{P}  
  :=
  \lift{(x)\substp{Q}{P}}{ R \psubstp{Q}{P} } \\
%   (\dropn{x})  \psubstp{Q}{P}       
%   := 
%   \left\{ 
%     \begin{array}{ccc} 
%       \dropn{\quotep{Q}} & & x \nameeq \quotep{P} \\
%       \dropn{x} & & otherwise \\
%     \end{array}
%   \right. 
  (\dropn{x})  \psubstp{Q}{P}       
  := 
  \left\{ 
    \begin{array}{ccc} 
      Q & & x \nameeq \quotep{P} \\
      \dropn{x} & & otherwise \\
    \end{array}
  \right.
\end{mathpar}
 

where

\begin{eqnarray}
  (x)\id{\{} \lpquote Q \rpquote / \lpquote P \rpquote \id{\}}            = 
  \left\{ 
    \begin{array}{ccc}
      \lpquote Q \rpquote & & x \nameeq \lpquote P \rpquote \\
      x & & otherwise \\
    \end{array}
  \right. \nonumber
\end{eqnarray}

and $z$ is chosen distinct from $\quotep{P}$, $\quotep{Q}$, the free
names in $Q$, and all the names in $R$. Our $\alpha$-equivalence will
be built in the standard way from this substitution.

\begin{remark}\label{rem:no_self_referential_names}
  One consequence of these definitions is that $\forall P. \quotep{P}
  \not\in \freenames{P}$.
\end{remark}

\subsection{ Dynamic quote: an example }

Anticipating something of what's to come, consider applying the
substitution, $\widehat{\id{\{}u / z \id{\}}}$, to the following pair
of processes, $\lift{w}{y!(z)}$ and $w[ \lpquote y!(z) \rpquote ]$.

\begin{eqnarray}
	\lift{w}{y!(z)}\widehat{\id{\{}u / z \id{\}}}
		& = &
		\lift{w}{y!(u)} \nonumber\\
	w[ \lpquote y!(z) \rpquote ] \widehat{ \id{\{}u / z \id{\}} }
		& = &
		w[ \lpquote y!(z) \rpquote ] \nonumber
\end{eqnarray}

Because the body of the process between quotes is impervious to
substitution, we get radically different answers. In fact, by
examining the first process in an input context,
e.g. $x?(z).\lift{w}{y!(z)}$, we see that the process under the lift
operator may be shaped by prefixed inputs binding a name inside it. In
this sense, the lift operator will be seen as a way to dynamically
construct processes before reifying them as names.

Finally equipped with these standard features we can present the
dynamics of the calculus.

\subsubsection{Operational semantics} 

Finally, we introduce the computational dynamics. What marks these
algebras as distinct from other more traditionally studied algebraic
structures, e.g. vector spaces or polynomial rings, is the manner in
which dynamics is captured. In traditional structures, dynamics is typically
expressed through morphisms between such structures, as in linear maps
between vector spaces or morphisms between rings. In algebras
associated with the semantics of computation, the dynamics is
expressed as part of the algebraic structure itself, through a
reduction reduction relation typically denoted by $\red$. Below, we
give a recursive presentation of this relation for the calculus used
in the encoding.

$\red \subseteq \pi \times \pi$
$\red : \pi \to \mathcal{P}(\pi)$

\begin{mathpar}
  \inferrule* [lab=Comm] { \textsf{match}( x_{src}, x_{trgt} ) } { x_{trgt}?(y)P \; | \; x_{src}!\langle {Q} \rangle \red P\{\quotep{Q}/y}\} }
  \and \\
  \inferrule* [lab=Par] {{P} \red {P}'} {{{P} | {Q}} \red {{P}' | {Q}}}
  \and
  \inferrule* [lab=Equiv]{{{P} \scong {P}'} \andalso {{P}' \red {Q}'} \andalso {{Q}' \scong {Q}}}{{P} \red {Q}}
\end{mathpar}

\begin{eqnarray*}
  match_{\equiv} (\quotep{P},\quotep{Q}) & := & P \equiv Q \\
  match_{\dagger}(\quotep{P},\quotep{Q}) & := & \forall R. P|Q \red^{*} R => R \red^{*} 0 \\
  match_{K}(\quotep{P},\quotep{Q}) & := & K \mbox{ for some context } K
\end{eqnarray*}

$u?(x)P | u!\langle Q \rangle \red P\{\quotep{Q}/x\}$

%We write $\wred$ for $\red^*$, and $P\red$ if $\exists Q $ such that $ P \red Q$.
We write $P\red$ if $\exists Q $ such that $ P \red Q$ and $P\not\red$, otherwise.

\section{Replication}

As mentioned before, it is known that replication (and hence
recursion) can be implemented in a higher-order process algebra
\cite{SangiorgiWalker}. As our first example of calculation with the
machinery thus far presented we give the construction explicitly in
the {\rhoc}.

\begin{eqnarray}
	D_{x} & := & \prefix{x}{y}{(\binpar{\outputp{x}{y}}{@{y}})} \nonumber\\
	\bangp_{x}{P} & := & \binpar{{x}!\langle{\binpar{D_{x}}{P}}\rangle}{D_{x}} \nonumber
\end{eqnarray}

\begin{eqnarray}
	\bangp_{x}{P} & & \nonumber\\
	=
	& {x}!\langle{(\prefix{x}{y}{(\outputp{x}{y} | @{y})) | P}}\rangle 
	      | \prefix{x}{y}{(\outputp{x}{y} | @{y})} & \nonumber\\
	\red
	& (\outputp{x}{y} | @{y})\substn{\quotep{(\prefix{x}{y}{(@{y} | \outputp{x}{y})) | P}}}{y} & \nonumber\\
	=
	& \outputp{x}{\quotep{(\prefix{x}{y}{(\outputp{x}{y} | @{y})) | P}}}
	  | {(\prefix{x}{y}{(\outputp{x}{y} | @{y})) | P}} & \nonumber\\
	\red
	& \ldots & \nonumber\\
	\red^*
	& P | P | \ldots & \nonumber
\end{eqnarray}

Of course, this encoding, as an implementation, runs away, unfolding
$\bangp{P}$ eagerly. A lazier and more implementable replication
operator, restricted to input-guarded processes, may be obtained as follows.

\begin{eqnarray}
\bangp{\prefix{u}{v}{P}} 
	:= 
	\binpar{\lift{x}{\prefix{u}{v}{(\binpar{D(x)}{P})}}}{D(x)} \nonumber
\end{eqnarray}

\begin{remark}
  Note that the lazier definition still does not deal with summation
  or mixed summation (i.e. sums over input and output). The reader is
  invited to construct definitions of replication that deal with these
  features. 

  Further, the definitions are parameterized in a name, $x$. Can you,
  gentle reader, make a definition that eliminates this parameter and
  guarantees no accidental interaction between the replication
  machinery and the process being replicated -- i.e. no accidental
  sharing of names used by the process to get its work done and the
  name(s) used by the replication to effect copying. This latter
  revision of the definition of replication is crucial to obtaining
  the expected identity $!!P \sim !P$.
\end{remark}

\begin{remark}\label{rem:paradoxical_combinator}
  The reader familiar with the lambda calculus will have noticed the
  similarity between $D$ and the paradoxical combinator.

  [Ed. note: the existence of this seems to suggest we have to be more
  restrictive on the set of processes and names we admit if we are to
  support no-cloning.]
\end{remark}

\subsubsection{Bisimulation}

The computational dynamics gives rise to another kind of equivalence,
the equivalence of computational behavior. As previously mentioned
this is typically captured \emph{via} some form of bisimulation.

% The notion we use in this paper is weak barbed bisimulation
% \cite{milner91polyadicpi}.

The notion we use in this paper is derived from weak barbed
bisimulation \cite{milner91polyadicpi}. 

\begin{definition}
An \emph{observation relation}, $\downarrow_{\mathcal N}$, over a set
of names, $\mathcal N$, is the smallest relation satisfying the rules
below.

\infrule[Out-barb]{y \in {\mathcal N}, \; x \nameeq y}
		  {\outputp{x}{v} \downarrow_{\mathcal N} x}
\infrule[Par-barb]{\mbox{$P\downarrow_{\mathcal N} x$ or $Q\downarrow_{\mathcal N} x$}}
		  {\binpar{P}{Q} \downarrow_{\mathcal N} x}

We write $P \Downarrow_{\mathcal N} x$ if there is $Q$ such that 
$P \wred Q$ and $Q \downarrow_{\mathcal N} x$.
\end{definition}

\begin{definition}
%\label{def.bbisim}
An  ${\mathcal N}$-\emph{barbed bisimulation} over a set of names, ${\mathcal N}$, is a symmetric binary relation 
${\mathcal S}_{\mathcal N}$ between agents such that $P\rel{S}_{\mathcal N}Q$ implies:
\begin{enumerate}
\item If $P \red P'$ then $Q \wred Q'$ and $P'\rel{S}_{\mathcal N} Q'$.
\item If $P\downarrow_{\mathcal N} x$, then $Q\Downarrow_{\mathcal N} x$.
\end{enumerate}
$P$ is ${\mathcal N}$-barbed bisimilar to $Q$, written
$P \wbbisim_{\mathcal N} Q$, if $P \rel{S}_{\mathcal N} Q$ for some ${\mathcal N}$-barbed bisimulation ${\mathcal S}_{\mathcal N}$.
\end{definition}

$\mathcal{R} \subseteq \pi \times \pi$

$P \mathcal{R} Q => \forall P'. P \red P' \Rightarrow \exists Q'. Q \red Q', P' \mathcal{R} Q'$

$P \vdash x \Rightarrow Q \vdash x$

\begin{mathpar}
  \inferrule*[lab=Out-barb]{x \nameeq y}{{y}!\langle{Q}\rangle \vdash x}
  \and
  \inferrule*[lab=Par-barb]{\mbox{$P\vdash x$ or $Q\vdash x$}}{\binpar{P}{Q} \vdash x}
\end{mathpar}

\subsubsection{Contexts}

One of the principle advantages of computational calculi like the
$\pi$-calculus is a well-defined notion of context,
contextual-equivalence and a correlation between
contextual-equivalence and notions of bisimulation. The notion of
context allows the decomposition of a process into (sub-)process and
its syntactic environment, its context. Thus, a context may be
thought of as a process with a ``hole'' (written $\Box$) in it. The
application of a context $M$ to a process $P$, written $M[P]$, is
tantamount to filling the hole in $M$ with $P$. In this paper we do
not need the full weight of this theory, but do make use of the notion
of context in the proof the main theorem. 

\begin{mathpar}
  \inferrule* [lab=summation] {} {{M_{M},M_{N}} \bc \Box \;|\; x.M_{A} \;|\; M_{M}+M_{N}}
  \and
  \inferrule* [lab=agent] {} {{M_{A}} \bc (\vec{x})M_{P} \;| \; \clift{P_0,\ldots,M_{P},\ldots,P_N}}
  \and \\
  \inferrule* [lab=process] {} {{M_{P}} \bc M_{N} \;| \;P|M_{P} }
\end{mathpar} 

\begin{mathpar}
  \inferrule* [lab=sychronization] {} {M_{N} \bc \Box \;|\; x?M_{F} \;|\; x!M_{C}}
  \and
  \inferrule* [lab=abstraction] {} {{M_{F}} \bc (x)M_{P} }
  \and
  \inferrule* [lab=concretion] {} {{M_{C}} \bc \langle M_{P} \rangle }
  \and \\
  \inferrule* [lab=process] {} {{M_{P}} \bc M_{N} \;| \;P|M_{P} }
\end{mathpar}

\begin{definition}[contextual application] Given a context $M$, and
  process $P$, we define the \emph{contextual application}, $M[P] :=
  M\{P/\Box\}$. That is, the contextual application of M to P is the
  substitution of $P$ for $\Box$ in $M$.
\end{definition}

$\meaningof{-} : L \to \mathcal{P}(\pi)$

\begin{mathpar}
  \inferrule* [lab=collection] {} {\meaningof{true} = \pi, \and \meaningof{~E} = \pi \setminus \meaningof{E}, \and \meaningof{E_{1} \& E_{2}} = \meaningof{E_{1}} \cap \meaningof{E_{2}}}
\end{mathpar}

\begin{mathpar}
  \inferrule* [lab=structure] {} {\meaningof{0} = \{ P \in \pi | P \equiv 0 \}, \and \\ \meaningof{E_1 | E_2} = \{ P \in \pi | P \equiv P_{1} | P_{2}, P_{1} \in \meaningof{E_{1}}, P_{2} \in \meaningof{E_2}\} }
\end{mathpar}

\begin{mathpar}
 \inferrule* [lab=behavior] {} {\meaningof{\langle a?b \rangle E} = \{ P \in \pi | P \equiv Q | u?(y)P', \\ \and \\\\ \and \\ \;\;\; u \in \meaningof{a}, \forall z.P'\{z/y\} \in \meaningof{E\{z/b\}}\}, \and \\ \meaningof{a!E} = \{ P \in \pi | P \equiv Q | x!\langle P' \rangle, x \in \meaningof{a} P' \in \meaningof{E}\} }
\end{mathpar}

\begin{mathpar}
 \inferrule* [lab=nominal] {} {\meaningof{\quotep{E}} = \{ \quotep{P} \in \quotep{\pi} | P \in \meaningof{E} \}, \and \meaningof{\quotep{P}} = \{ \quotep{Q} \in \quotep{\pi} | P \equiv Q \} \and \\ \meaningof{@\quotep{E}} = \{ P \in \pi | P \equiv @x, x \in \meaningof{E} \}}
\end{mathpar}

\begin{eqnarray*}
  \\
  \meaningof{-} : TS \to ST
\end{eqnarray*}

\begin{eqnarray*}
  \\
  L : TS \to ST
\end{eqnarray*}

\begin{eqnarray*}
  \\
  P \models E \iff P \in \meaningof{E}
\end{eqnarray*}

\begin{eqnarray*}
  P \approx_{L} Q \iff \forall E \in L. P \models E \iff Q \models E
\end{eqnarray*}

\begin{eqnarray*}
  P \approx_{K} Q
\end{eqnarray*}

\begin{eqnarray*}
  P \approx Q
\end{eqnarray*}

$\approx_{K} = \approx = \approx_{L}$

\subsubsection{Contextual duality}

Note that contexts extend the quotation operation to a family of
operations from processes to names. Given a context, $M$, we can
define a \emph{nominal context}, $\quotep{M}$ by $\quotep{M}[P] :=
\quotep{M[P]}$. To foreshadow what is to come we observe that these
operations enjoy a duality with processes very much like the duality
between vectors and maps from vectors to scalars.

Further, because the calculus is essentially higher-order, we have a
correspondence between contexts and processes. More specifically,
given a name $x$ and a context $M$ we can construct $M^{*}_{x}$ such
that 

\begin{mathpar}
  M^{*}_{x} | \lift{x}{P} \red M[P]
\end{mathpar}

namely,

\begin{mathpar}
  M^{*}_{x} := x?(u).M[\dropn{u}]
\end{mathpar}

The dependence of $M^{*}_{x}$ on a name makes it an abstraction, 

\begin{mathpar}
  M^{*} := (x)x?(u).M[\dropn{u}]
\end{mathpar}

\subsection{Additional notation}

It will sometimes be convenient to denote the process a name
quotes. We already have the notation $x = \quotep{P}$, but it will be
convenient to introduce an alternate notation, $\procn{x}$, when we
want to emphasize the connection to the use of the name. Note that, by
virtue of name equivalence, $\quotep{\procn{x}} \nameeq x$; so, the
notation is consistent with previous definitions.

Further, because names have structure it is possible to effect
substitutions on the basis of that structure. This means we need to
upgrade our notation for substitutions, which we accomplish by
adapting comprehension notation. Thus,

\begin{mathpar}
  P\{ y / x : x \in S \}
\end{mathpar}

is interpreted to mean the process derived from P by replacing (in a
capture-avoiding manner) each occurrence of $x$ in $S$ by $y$. For example,

\begin{mathpar}
  P\{ \quotep{\procn{x}|\procn{x}} / x : x \in \freenames{P} \}
\end{mathpar}

will replace each (occurrence) of a free name $x$ in $P$ by
$\quotep{\procn{x}|\procn{x}}$.

Also, we will avail ourselves of the notation $x^{L}$ and $x^{R}$ to
denote injections of a name into disjoint copies of the name
space. There are numerous ways to accomplish this. One example can be
found in \cite{MeredithR05}. This notation overloads to vectors of
names: $\vec{x}^{\pi} := (x_{i}^{\pi} \; : \; 0 \leq i < |\vec{x}| )$ where $\pi \in \{L,R\}$.

We also use $P^{\Box} := P|\Box$.

In \cite{MeredithR05} an interpretation of the new operator is
given. It turns out that there are several possible interpretations
all enjoying the requisite algebraic properties of the operator (see
\cite{milner91polyadicpi}). We will therefore make liberal use of
$(\nu\; \vec{x})P$.

% subsection the_syntax_and_semantics_of_the_notation_system (end)   

\input{qm2pi.qmops} 

\input{qm2pi.sterngerlach} 

\input{qm2pi.metric} 

% section concurrent_process_calculi (end)

%\input{qm2pi.proofsketch}

% section proof sketch (end)

%\input{qm2pi.slviaknots} 

% section spatial logic via knots (end)

\input{qm2pi.conclusion}

% section conclusion (end)

%\input{qm2pi.dtcodes} 

% section wiring algorithm (end)

\input{qm2pi.ack} 

% section acknowledgments (end)

\newpage


\bibliographystyle{plain}   
\bibliography{../../biblios/main.bib}

\input{qm2pi.rhodetails}

\end{document}

 

%\ifpdf
%\usepackage[pdftex]{graphicx}
%\else
%\usepackage{graphicx}
%\fi

 % \ifpdf
%  \usepackage{pdfsync}
%  \if


%\title{Brief Article}
%\author{David F. Snyder}
%\author{L.G. Meredith}

%\address{Dept. of Math., Texas State University--San Marcos, San Marcos, TX 78666}
       
\pagestyle{empty}


\begin{document}

\lstset{language=[Objective]Caml,frame=shadowbox}

\documentclass[12pt]{llncs}
%\documentclass{jktr}

\usepackage[pdftex]{hyperref}                   
\usepackage {listings}
\usepackage {mathpartir}
\usepackage{bcprules}
%\usepackage{listings}
                       
\usepackage{graphicx} 
%\usepackage[margins=2.5cm,nohead,nofoot]{geometry}
%\usepackage{geometry}
\usepackage{amsfonts}
\usepackage{amstext}
\usepackage{latexsym}
\usepackage{amssymb}
\usepackage{color}


%\include{myPreamble}
\include{qm2pi.local} 

%\ifpdf
%\usepackage[pdftex]{graphicx}
%\else
%\usepackage{graphicx}
%\fi

 % \ifpdf
%  \usepackage{pdfsync}
%  \if


%\title{Brief Article}
%\author{David F. Snyder}
%\author{L.G. Meredith}

%\address{Dept. of Math., Texas State University--San Marcos, San Marcos, TX 78666}
       
\pagestyle{empty}


\begin{document}

\lstset{language=[Objective]Caml,frame=shadowbox}

\input{qm2pi.front}

% section front matter (end)

\input{qm2pi.intro} 
 
% section introduction (end)

% \input{qm2pi.knotations} 

% section notation (end)

\input{qm2pi.process.calculi} 

% section concurrent_process_calculi_and_spatial_logics_ (end)
    
%\input{qm2pi.knots2pi} 

%\input{qm2pi.trefoil} 

%\input{qm2pi.mainthm} 

% subsection basic_interpretation (end)

%\input{qm2pi.rho.presentation} 
\subsection{The syntax and semantics of the notation system}\label{sub:the_syntax_and_semantics_of_the_notation_system} % (fold)

We now summarize a technical presentation of the calculus that
embodies our theory of dynamics. The typical presentation of such a
calculus follows the style of giving generators and relations on
them. The grammar, below, describing term constructors, freely
generates the set of processes, $\Proc$. This set is then quotiented
by a relation known as structural congruence and it is over this set
that the notion of dynamics is expressed. This presentation is
essentially that of \cite{MeredithR05} with the addition of
polyadicity and summation. For readability we have relegated some of
the technical subtleties to an appendix.

\subsubsection{Process grammar}\label{subsub:process_grammar}

\begin{mathpar}
  \inferrule* [lab=synchronization] {} {{M} \bc \pzero \;|\; x?F \;|\; x!C }
  \and
  \inferrule* [lab=abstraction] {} {{F} \bc (x)P}
  \and
  \inferrule* [lab=concretion] {} {{C} \bc \langle Q \rangle}
  \and
  \inferrule* [lab=process] {} {{P,Q} \bc M \;| \;P|Q \;|\; @{x}}
  \and
  \inferrule* [lab=name] {} {{x} \bc \quotep{P}}
\end{mathpar} 

Note that $\vec{x}$ (resp. $\vec{P}$) denotes a vector of names
(resp. processes) of length $|\vec{x}|$ (resp. $|\vec{P}|$). We adopt
the following useful abbreviations.

\begin{mathpar}
   x?(\vec{y}).P := x.(\vec{y})P \and  x\clift{\vec{P}} := x.\clift{\vec{P}}
   \and x!(y) := \lift{x}{\dropn{y}}
   \and \Pi_{i=0}^{n-1}P_i := P_0 | \ldots | P_{n-1}
\end{mathpar}

\subsubsection{Structural congruence}

\paragraph{Free and bound names and alpha-equivalence.} At the
core of structural equivalence is alpha-equivalence which identifies
process that are the same up to a change of variable. Formally, we
recognize the distinction between free and bound names. The free names
of a process, $\freenames{P}$, may be calculated recursively as
follows:

\begin{mathpar}
\freenames{\pzero} := \emptyset
  \and \\
  \freenames{x?(y).P} := \{ x \} \cup (\freenames{P} \setminus \{ y \})
  \and 
  \freenames{x!\langle P \rangle} := \{ x \} \cup \{ P \} 
  \and \\
  \freenames{P|Q} := \freenames{P} \cup \freenames{Q}
  \and \\
  \freenames{@{x}} := \{ x \}
\end{mathpar}

$\pi$
$\quotep{\pi}$

$\freenames{-} : \pi \to \mathcal{P}(\quotep{\pi})$

\begin{eqnarray*}
  \freenames{\pzero} & := & \emptyset \\
  \freenames{x?(y).P} & := & \{ x \} \cup (\freenames{P} \setminus \{ y \}) \\
  \freenames{x!\langle P \rangle} & := & \{ x \} \cup \{ P \} \\
  \freenames{P|Q} & := & \freenames{P} \cup \freenames{Q} \\
  \freenames{\dropn{x}} & := & \{ x \}
\end{eqnarray*}

The bound names of a process, $\boundnames{P}$, are those names occurring in $P$
that are not free. For example, in $x?(y).0$, the name $x$ is free, while $y$ is bound.

\begin{mathpar}
  \inferrule* [lab=monoidal-laws] {} { P|Q \equiv Q|P \and P|0 \equiv P \and P|(Q|R) \equiv (P|Q)|R }
\end{mathpar}

\begin{mathpar}
  \inferrule* [lab=alpha-equivalence] {} { (x)P \equiv (y)P\{y/x\} \and y \not\in \freenames{P} }
\end{mathpar}

\begin{definition}
Then two processes, $P,Q$, are alpha-equivalent if $P = Q\{\vec{y}/\vec{x}\}$ for
some $\vec{x} \in \boundnames{Q},\vec{y} \in \boundnames{P}$, where $Q\{\vec{y}/\vec{x}\}$
denotes the capture-avoiding substitution of $\vec{y}$ for $\vec{x}$ in $Q$.
\end{definition}

\begin{definition}
  The {\em structural congruence} \cite{SangiorgiWalker} , $\equiv$,
  between processes is the least congruence containing
  alpha-equivalence, satisfying the abelian monoid laws
  (associativity, commutativity and $\pzero$ as identity) for parallel
  composition $|$ and for summation $+$.
\end{definition}

\subsection{Name equivalence}

We take name equivalence, written $\nameeq$, to be the smallest
equivalence relation generated by the following rules.

\begin{mathpar}
\inferrule*[lab=Quote-drop]
{ }
{ \quotep{@{x}} \nameeq x }

\inferrule*[lab=Struct-equiv]
{ P \scong Q }
{ \quotep{P} \nameeq \quotep{Q} }
\end{mathpar}

The astute reader will have noticed that the mutual recursion of names
and processes imposes a mutual recursion on alpha-equivalence and
structural equivalence via name-equivalence. Fortunately, all of this
works out pleasantly and we may calculate in the natural way, free of
concern. The reader interested in the details is referred to the
appendix \ref{appendix:rho_details}.

\subsection{Substitution}

We use $\Proc$ for the set of processes, $\QProc$ for the set of
names, and $\id{\{}\vec{y} / \vec{x} \id{\}}$ to denote partial maps,
$s : \QProc \rightarrow \QProc$. A map, $s$ lifts, uniquely, to a map
on process terms, $\widehat{s} : \Proc \rightarrow \Proc$ by the
following equations.

\begin{mathpar}
  (0) \psubstp{Q}{P} := 0 \\
  (R \juxtap S) \psubstp{Q}{P}
  :=    
  (R)\psubstp{Q}{P} \juxtap (S) \psubstp{Q}{P} \\
  (x?(y).R) \psubstp{Q}{P}    
  :=    
  (x)\substp{Q}{P} (z)\concat( (R \psubstn{z}{y}) \psubstp{Q}{P} ) \\
  (\lift{x}{R}) \psubstp{Q}{P}  
  :=
  \lift{(x)\substp{Q}{P}}{ R \psubstp{Q}{P} } \\
%   (\dropn{x})  \psubstp{Q}{P}       
%   := 
%   \left\{ 
%     \begin{array}{ccc} 
%       \dropn{\quotep{Q}} & & x \nameeq \quotep{P} \\
%       \dropn{x} & & otherwise \\
%     \end{array}
%   \right. 
  (\dropn{x})  \psubstp{Q}{P}       
  := 
  \left\{ 
    \begin{array}{ccc} 
      Q & & x \nameeq \quotep{P} \\
      \dropn{x} & & otherwise \\
    \end{array}
  \right.
\end{mathpar}
 

where

\begin{eqnarray}
  (x)\id{\{} \lpquote Q \rpquote / \lpquote P \rpquote \id{\}}            = 
  \left\{ 
    \begin{array}{ccc}
      \lpquote Q \rpquote & & x \nameeq \lpquote P \rpquote \\
      x & & otherwise \\
    \end{array}
  \right. \nonumber
\end{eqnarray}

and $z$ is chosen distinct from $\quotep{P}$, $\quotep{Q}$, the free
names in $Q$, and all the names in $R$. Our $\alpha$-equivalence will
be built in the standard way from this substitution.

\begin{remark}\label{rem:no_self_referential_names}
  One consequence of these definitions is that $\forall P. \quotep{P}
  \not\in \freenames{P}$.
\end{remark}

\subsection{ Dynamic quote: an example }

Anticipating something of what's to come, consider applying the
substitution, $\widehat{\id{\{}u / z \id{\}}}$, to the following pair
of processes, $\lift{w}{y!(z)}$ and $w[ \lpquote y!(z) \rpquote ]$.

\begin{eqnarray}
	\lift{w}{y!(z)}\widehat{\id{\{}u / z \id{\}}}
		& = &
		\lift{w}{y!(u)} \nonumber\\
	w[ \lpquote y!(z) \rpquote ] \widehat{ \id{\{}u / z \id{\}} }
		& = &
		w[ \lpquote y!(z) \rpquote ] \nonumber
\end{eqnarray}

Because the body of the process between quotes is impervious to
substitution, we get radically different answers. In fact, by
examining the first process in an input context,
e.g. $x?(z).\lift{w}{y!(z)}$, we see that the process under the lift
operator may be shaped by prefixed inputs binding a name inside it. In
this sense, the lift operator will be seen as a way to dynamically
construct processes before reifying them as names.

Finally equipped with these standard features we can present the
dynamics of the calculus.

\subsubsection{Operational semantics} 

Finally, we introduce the computational dynamics. What marks these
algebras as distinct from other more traditionally studied algebraic
structures, e.g. vector spaces or polynomial rings, is the manner in
which dynamics is captured. In traditional structures, dynamics is typically
expressed through morphisms between such structures, as in linear maps
between vector spaces or morphisms between rings. In algebras
associated with the semantics of computation, the dynamics is
expressed as part of the algebraic structure itself, through a
reduction reduction relation typically denoted by $\red$. Below, we
give a recursive presentation of this relation for the calculus used
in the encoding.

$\red \subseteq \pi \times \pi$
$\red : \pi \to \mathcal{P}(\pi)$

\begin{mathpar}
  \inferrule* [lab=Comm] { \textsf{match}( x_{src}, x_{trgt} ) } { x_{trgt}?(y)P \; | \; x_{src}!\langle {Q} \rangle \red P\{\quotep{Q}/y}\} }
  \and \\
  \inferrule* [lab=Par] {{P} \red {P}'} {{{P} | {Q}} \red {{P}' | {Q}}}
  \and
  \inferrule* [lab=Equiv]{{{P} \scong {P}'} \andalso {{P}' \red {Q}'} \andalso {{Q}' \scong {Q}}}{{P} \red {Q}}
\end{mathpar}

\begin{eqnarray*}
  match_{\equiv} (\quotep{P},\quotep{Q}) & := & P \equiv Q \\
  match_{\dagger}(\quotep{P},\quotep{Q}) & := & \forall R. P|Q \red^{*} R => R \red^{*} 0 \\
  match_{K}(\quotep{P},\quotep{Q}) & := & K \mbox{ for some context } K
\end{eqnarray*}

$u?(x)P | u!\langle Q \rangle \red P\{\quotep{Q}/x\}$

%We write $\wred$ for $\red^*$, and $P\red$ if $\exists Q $ such that $ P \red Q$.
We write $P\red$ if $\exists Q $ such that $ P \red Q$ and $P\not\red$, otherwise.

\section{Replication}

As mentioned before, it is known that replication (and hence
recursion) can be implemented in a higher-order process algebra
\cite{SangiorgiWalker}. As our first example of calculation with the
machinery thus far presented we give the construction explicitly in
the {\rhoc}.

\begin{eqnarray}
	D_{x} & := & \prefix{x}{y}{(\binpar{\outputp{x}{y}}{@{y}})} \nonumber\\
	\bangp_{x}{P} & := & \binpar{{x}!\langle{\binpar{D_{x}}{P}}\rangle}{D_{x}} \nonumber
\end{eqnarray}

\begin{eqnarray}
	\bangp_{x}{P} & & \nonumber\\
	=
	& {x}!\langle{(\prefix{x}{y}{(\outputp{x}{y} | @{y})) | P}}\rangle 
	      | \prefix{x}{y}{(\outputp{x}{y} | @{y})} & \nonumber\\
	\red
	& (\outputp{x}{y} | @{y})\substn{\quotep{(\prefix{x}{y}{(@{y} | \outputp{x}{y})) | P}}}{y} & \nonumber\\
	=
	& \outputp{x}{\quotep{(\prefix{x}{y}{(\outputp{x}{y} | @{y})) | P}}}
	  | {(\prefix{x}{y}{(\outputp{x}{y} | @{y})) | P}} & \nonumber\\
	\red
	& \ldots & \nonumber\\
	\red^*
	& P | P | \ldots & \nonumber
\end{eqnarray}

Of course, this encoding, as an implementation, runs away, unfolding
$\bangp{P}$ eagerly. A lazier and more implementable replication
operator, restricted to input-guarded processes, may be obtained as follows.

\begin{eqnarray}
\bangp{\prefix{u}{v}{P}} 
	:= 
	\binpar{\lift{x}{\prefix{u}{v}{(\binpar{D(x)}{P})}}}{D(x)} \nonumber
\end{eqnarray}

\begin{remark}
  Note that the lazier definition still does not deal with summation
  or mixed summation (i.e. sums over input and output). The reader is
  invited to construct definitions of replication that deal with these
  features. 

  Further, the definitions are parameterized in a name, $x$. Can you,
  gentle reader, make a definition that eliminates this parameter and
  guarantees no accidental interaction between the replication
  machinery and the process being replicated -- i.e. no accidental
  sharing of names used by the process to get its work done and the
  name(s) used by the replication to effect copying. This latter
  revision of the definition of replication is crucial to obtaining
  the expected identity $!!P \sim !P$.
\end{remark}

\begin{remark}\label{rem:paradoxical_combinator}
  The reader familiar with the lambda calculus will have noticed the
  similarity between $D$ and the paradoxical combinator.

  [Ed. note: the existence of this seems to suggest we have to be more
  restrictive on the set of processes and names we admit if we are to
  support no-cloning.]
\end{remark}

\subsubsection{Bisimulation}

The computational dynamics gives rise to another kind of equivalence,
the equivalence of computational behavior. As previously mentioned
this is typically captured \emph{via} some form of bisimulation.

% The notion we use in this paper is weak barbed bisimulation
% \cite{milner91polyadicpi}.

The notion we use in this paper is derived from weak barbed
bisimulation \cite{milner91polyadicpi}. 

\begin{definition}
An \emph{observation relation}, $\downarrow_{\mathcal N}$, over a set
of names, $\mathcal N$, is the smallest relation satisfying the rules
below.

\infrule[Out-barb]{y \in {\mathcal N}, \; x \nameeq y}
		  {\outputp{x}{v} \downarrow_{\mathcal N} x}
\infrule[Par-barb]{\mbox{$P\downarrow_{\mathcal N} x$ or $Q\downarrow_{\mathcal N} x$}}
		  {\binpar{P}{Q} \downarrow_{\mathcal N} x}

We write $P \Downarrow_{\mathcal N} x$ if there is $Q$ such that 
$P \wred Q$ and $Q \downarrow_{\mathcal N} x$.
\end{definition}

\begin{definition}
%\label{def.bbisim}
An  ${\mathcal N}$-\emph{barbed bisimulation} over a set of names, ${\mathcal N}$, is a symmetric binary relation 
${\mathcal S}_{\mathcal N}$ between agents such that $P\rel{S}_{\mathcal N}Q$ implies:
\begin{enumerate}
\item If $P \red P'$ then $Q \wred Q'$ and $P'\rel{S}_{\mathcal N} Q'$.
\item If $P\downarrow_{\mathcal N} x$, then $Q\Downarrow_{\mathcal N} x$.
\end{enumerate}
$P$ is ${\mathcal N}$-barbed bisimilar to $Q$, written
$P \wbbisim_{\mathcal N} Q$, if $P \rel{S}_{\mathcal N} Q$ for some ${\mathcal N}$-barbed bisimulation ${\mathcal S}_{\mathcal N}$.
\end{definition}

$\mathcal{R} \subseteq \pi \times \pi$

$P \mathcal{R} Q => \forall P'. P \red P' \Rightarrow \exists Q'. Q \red Q', P' \mathcal{R} Q'$

$P \vdash x \Rightarrow Q \vdash x$

\begin{mathpar}
  \inferrule*[lab=Out-barb]{x \nameeq y}{{y}!\langle{Q}\rangle \vdash x}
  \and
  \inferrule*[lab=Par-barb]{\mbox{$P\vdash x$ or $Q\vdash x$}}{\binpar{P}{Q} \vdash x}
\end{mathpar}

\subsubsection{Contexts}

One of the principle advantages of computational calculi like the
$\pi$-calculus is a well-defined notion of context,
contextual-equivalence and a correlation between
contextual-equivalence and notions of bisimulation. The notion of
context allows the decomposition of a process into (sub-)process and
its syntactic environment, its context. Thus, a context may be
thought of as a process with a ``hole'' (written $\Box$) in it. The
application of a context $M$ to a process $P$, written $M[P]$, is
tantamount to filling the hole in $M$ with $P$. In this paper we do
not need the full weight of this theory, but do make use of the notion
of context in the proof the main theorem. 

\begin{mathpar}
  \inferrule* [lab=summation] {} {{M_{M},M_{N}} \bc \Box \;|\; x.M_{A} \;|\; M_{M}+M_{N}}
  \and
  \inferrule* [lab=agent] {} {{M_{A}} \bc (\vec{x})M_{P} \;| \; \clift{P_0,\ldots,M_{P},\ldots,P_N}}
  \and \\
  \inferrule* [lab=process] {} {{M_{P}} \bc M_{N} \;| \;P|M_{P} }
\end{mathpar} 

\begin{mathpar}
  \inferrule* [lab=sychronization] {} {M_{N} \bc \Box \;|\; x?M_{F} \;|\; x!M_{C}}
  \and
  \inferrule* [lab=abstraction] {} {{M_{F}} \bc (x)M_{P} }
  \and
  \inferrule* [lab=concretion] {} {{M_{C}} \bc \langle M_{P} \rangle }
  \and \\
  \inferrule* [lab=process] {} {{M_{P}} \bc M_{N} \;| \;P|M_{P} }
\end{mathpar}

\begin{definition}[contextual application] Given a context $M$, and
  process $P$, we define the \emph{contextual application}, $M[P] :=
  M\{P/\Box\}$. That is, the contextual application of M to P is the
  substitution of $P$ for $\Box$ in $M$.
\end{definition}

$\meaningof{-} : L \to \mathcal{P}(\pi)$

\begin{mathpar}
  \inferrule* [lab=collection] {} {\meaningof{true} = \pi, \and \meaningof{~E} = \pi \setminus \meaningof{E}, \and \meaningof{E_{1} \& E_{2}} = \meaningof{E_{1}} \cap \meaningof{E_{2}}}
\end{mathpar}

\begin{mathpar}
  \inferrule* [lab=structure] {} {\meaningof{0} = \{ P \in \pi | P \equiv 0 \}, \and \\ \meaningof{E_1 | E_2} = \{ P \in \pi | P \equiv P_{1} | P_{2}, P_{1} \in \meaningof{E_{1}}, P_{2} \in \meaningof{E_2}\} }
\end{mathpar}

\begin{mathpar}
 \inferrule* [lab=behavior] {} {\meaningof{\langle a?b \rangle E} = \{ P \in \pi | P \equiv Q | u?(y)P', \\ \and \\\\ \and \\ \;\;\; u \in \meaningof{a}, \forall z.P'\{z/y\} \in \meaningof{E\{z/b\}}\}, \and \\ \meaningof{a!E} = \{ P \in \pi | P \equiv Q | x!\langle P' \rangle, x \in \meaningof{a} P' \in \meaningof{E}\} }
\end{mathpar}

\begin{mathpar}
 \inferrule* [lab=nominal] {} {\meaningof{\quotep{E}} = \{ \quotep{P} \in \quotep{\pi} | P \in \meaningof{E} \}, \and \meaningof{\quotep{P}} = \{ \quotep{Q} \in \quotep{\pi} | P \equiv Q \} \and \\ \meaningof{@\quotep{E}} = \{ P \in \pi | P \equiv @x, x \in \meaningof{E} \}}
\end{mathpar}

\begin{eqnarray*}
  \\
  \meaningof{-} : TS \to ST
\end{eqnarray*}

\begin{eqnarray*}
  \\
  L : TS \to ST
\end{eqnarray*}

\begin{eqnarray*}
  \\
  P \models E \iff P \in \meaningof{E}
\end{eqnarray*}

\begin{eqnarray*}
  P \approx_{L} Q \iff \forall E \in L. P \models E \iff Q \models E
\end{eqnarray*}

\begin{eqnarray*}
  P \approx_{K} Q
\end{eqnarray*}

\begin{eqnarray*}
  P \approx Q
\end{eqnarray*}

$\approx_{K} = \approx = \approx_{L}$

\subsubsection{Contextual duality}

Note that contexts extend the quotation operation to a family of
operations from processes to names. Given a context, $M$, we can
define a \emph{nominal context}, $\quotep{M}$ by $\quotep{M}[P] :=
\quotep{M[P]}$. To foreshadow what is to come we observe that these
operations enjoy a duality with processes very much like the duality
between vectors and maps from vectors to scalars.

Further, because the calculus is essentially higher-order, we have a
correspondence between contexts and processes. More specifically,
given a name $x$ and a context $M$ we can construct $M^{*}_{x}$ such
that 

\begin{mathpar}
  M^{*}_{x} | \lift{x}{P} \red M[P]
\end{mathpar}

namely,

\begin{mathpar}
  M^{*}_{x} := x?(u).M[\dropn{u}]
\end{mathpar}

The dependence of $M^{*}_{x}$ on a name makes it an abstraction, 

\begin{mathpar}
  M^{*} := (x)x?(u).M[\dropn{u}]
\end{mathpar}

\subsection{Additional notation}

It will sometimes be convenient to denote the process a name
quotes. We already have the notation $x = \quotep{P}$, but it will be
convenient to introduce an alternate notation, $\procn{x}$, when we
want to emphasize the connection to the use of the name. Note that, by
virtue of name equivalence, $\quotep{\procn{x}} \nameeq x$; so, the
notation is consistent with previous definitions.

Further, because names have structure it is possible to effect
substitutions on the basis of that structure. This means we need to
upgrade our notation for substitutions, which we accomplish by
adapting comprehension notation. Thus,

\begin{mathpar}
  P\{ y / x : x \in S \}
\end{mathpar}

is interpreted to mean the process derived from P by replacing (in a
capture-avoiding manner) each occurrence of $x$ in $S$ by $y$. For example,

\begin{mathpar}
  P\{ \quotep{\procn{x}|\procn{x}} / x : x \in \freenames{P} \}
\end{mathpar}

will replace each (occurrence) of a free name $x$ in $P$ by
$\quotep{\procn{x}|\procn{x}}$.

Also, we will avail ourselves of the notation $x^{L}$ and $x^{R}$ to
denote injections of a name into disjoint copies of the name
space. There are numerous ways to accomplish this. One example can be
found in \cite{MeredithR05}. This notation overloads to vectors of
names: $\vec{x}^{\pi} := (x_{i}^{\pi} \; : \; 0 \leq i < |\vec{x}| )$ where $\pi \in \{L,R\}$.

We also use $P^{\Box} := P|\Box$.

In \cite{MeredithR05} an interpretation of the new operator is
given. It turns out that there are several possible interpretations
all enjoying the requisite algebraic properties of the operator (see
\cite{milner91polyadicpi}). We will therefore make liberal use of
$(\nu\; \vec{x})P$.

% subsection the_syntax_and_semantics_of_the_notation_system (end)   

\input{qm2pi.qmops} 

\input{qm2pi.sterngerlach} 

\input{qm2pi.metric} 

% section concurrent_process_calculi (end)

%\input{qm2pi.proofsketch}

% section proof sketch (end)

%\input{qm2pi.slviaknots} 

% section spatial logic via knots (end)

\input{qm2pi.conclusion}

% section conclusion (end)

%\input{qm2pi.dtcodes} 

% section wiring algorithm (end)

\input{qm2pi.ack} 

% section acknowledgments (end)

\newpage


\bibliographystyle{plain}   
\bibliography{../../biblios/main.bib}

\input{qm2pi.rhodetails}

\end{document}



% section front matter (end)

\section{Introduction}\label{sec:introduction} % (fold)
In this draft of the material i am going to have to dispense with the
usual writing conventions adopted in papers on these topics. i'm going
to have adopt whatever tone i need at the time i'm writing up the
calculations. Sometimes this may be very conversational; others it may
be the barest mathematical grunts; others still it may be that i have
lifted text from one of my other papers because the exposition of some
point was better said there. i hope that my readers are not unduly put
out by this decision. i'm not doing this to flout convention or be
rebellious. i find these calculations very technically challenging. To
keep everything going technically, something has to give; i have to
let go of some cognitive burden. So, the academic writing style --
with all of its trade-offs in terms of facilitating technical
communication -- is what i'm letting go of. Perhaps subsequent drafts
can be tightened and polished, but for now, i'm going to speak as if
we were sitting together in a coffee shop with a laptop, wifi and a
pad of paper and a pencil.

So, here's what i have to say. We -- you and i, comfortably ensconced
in our coffee shop and well-equipped with our tools -- can realize and
carry out the calculations of quantum mechanics over a very different
formal theory of dynamics, a formal theory of dynamics that
corresponds to a theory of concurrent computation with
\emph{reflection}. It has the advantage that the underlying theory is
already `quantized', but supports analogues all of the continuuous
operations. Strikingly, this underlying theory has recently been
connected with a notion of metric that we can show, by calculating
together, coincides with the metric induced by the inner product.

There are a lot of reasons why you might be interested in seeing
calculations of this form. Here's why i'm interested. For the past
several centuries there has been no competitor to the ``Newtonian''
account of dynamics. As a result the predominant share of accounts of
dynamical systems and situations have had to be formulated in terms of
the Newtonian machinery. i view this as an intellectually dangerous
position to occupy. Everything, despite it's intrinsic shape, turns
into a nail to be hit with this hammer. Recently, however, the theory
of computation has matured to the point where we have candidates for
theories of dynamics that offer very different perspective on
reasoning about dynamical systems and situations. Testing these
candidates against very successful accounts of dynamical situations,
like quantum mechanics, is going to give us some sense of how mature
they are and some measure of the quality of these accounts of
dynamics.

\subsection{Summary of contributions and outline of paper}

So, we're going to develop an interpretation of the operations of
quantum mechanics normally interpreted by Hilbert spaces and
operators. We're going to do this over a theory of computation. Note
that this is very different than the usual quantum computation program
which develops notions of computation over quantum mechanics. Rather,
we are developing a story that aligns with Wheeler's slogan: It from
Bit. To do this we will first provide an account of the theory of
computation at play here. Then we will dive into a calculation-driven
interpretation of the operations of quantum mechanics.

The reason we take this approach is that -- until very recently --
there hasn't been an axiomatic account of quantum mechanics. As a
result there has been no sharp delineation of the mathematical theory
supporting interpretation of the physical theory and the physical
theory, itself. So, ambient features of the maths are free to be
exploited (or supressed) without a real accounting of their physical
relevance. There is no sharp statement ``here's the physical theory''
qua \emph{theory} and ``here's the mathematical interpretation''
enabling a judgment of how faithful the interpretation is -- apart
from experimental observation. When there is an axiomatic account we
can judge how well a given mathematical formalism supports an
interpretation of the axioms, independent of
experimentation. Likewise, we can judge how well we have captured our
physical evidence and experience with our axiomatics, independent of
any specific mathematical implementation, with accidental detail that
may or may not have physical significance. 

In lieu of a fully fleshed out and vetted axiomatic account of quantum
mechanics, interpreting the operational notions in service of modeling
physical systems will have to suffice. In other words, we are not in
the business of providing a model of Hilbert spaces and operators. We
are in the business of providing a model of quantum mechanics because
we are motivated by testing our notions of dynamics against physical
theory; and, the predictive calculations of the physical theory must
serve as the best formulation -- shy of a fully fleshed out axiomatic
account -- of the physical theory itself (as they have for scientific
theories since time immemorial). Put another way, despite a
whole-hearted commitment to an It-from-Bit ontology, we are firmly
aligned with the shut-up-and-calculate camp as the best way to obtain
results either from the physical perspective or as a quality assurance
measure of our fledgling theory of dynamics.

In detail, we present a reflective process calculus. Then we develop
intuitive correspondences between the notions available in this
calculus and the usual physical notions supporting quantum mechanical
calculations. Thus, 

\begin{table}[htp]
  \center{
    \fbox{
      \begin{tabular}{c|c}
        quantum mechanics & process calculus \\
        \hline
        scalar & name \\
        state vector & process \\
        dual & contextual duals \\
        matrix & formal sums of process-context-dual pairs \\
        orthogonality & process annihilation \\
        inner product & execution-formula + quoting
      \end{tabular}
    }
  }
  \caption{QM - process calculi correspondences}
\end{table}

Then we tighten up these intuitions to operational definitions. We
employ the Dirac notation as the best proxy we can find for an
abstract syntax of the quantum mechanical notions. The definitions we
develop put us in contact with equational constraints coming from the
theory that we demonstrate the definitions and calculations satisfy.

This puts us in a position to shut up and calculate for the
Stern-Gerlach experimental set up, showing how these predictive
calculations become calculations on processes in our theory of a
reflective process calculus.

Penultimately, we demonstrate that the notion of metric coming from
the inner product coincides with the notion of metric available from
the theory of bisimulation. This demonstration gives us the right to
think of space as arising from behavior. Finally, we consider where we
might go from the new vantage point we have obtained.

% section introduction (end) 
 
% section introduction (end)

% \documentclass[12pt]{llncs}
%\documentclass{jktr}

\usepackage[pdftex]{hyperref}                   
\usepackage {listings}
\usepackage {mathpartir}
\usepackage{bcprules}
%\usepackage{listings}
                       
\usepackage{graphicx} 
%\usepackage[margins=2.5cm,nohead,nofoot]{geometry}
%\usepackage{geometry}
\usepackage{amsfonts}
\usepackage{amstext}
\usepackage{latexsym}
\usepackage{amssymb}
\usepackage{color}


%\include{myPreamble}
\include{qm2pi.local} 

%\ifpdf
%\usepackage[pdftex]{graphicx}
%\else
%\usepackage{graphicx}
%\fi

 % \ifpdf
%  \usepackage{pdfsync}
%  \if


%\title{Brief Article}
%\author{David F. Snyder}
%\author{L.G. Meredith}

%\address{Dept. of Math., Texas State University--San Marcos, San Marcos, TX 78666}
       
\pagestyle{empty}


\begin{document}

\lstset{language=[Objective]Caml,frame=shadowbox}

\input{qm2pi.front}

% section front matter (end)

\input{qm2pi.intro} 
 
% section introduction (end)

% \input{qm2pi.knotations} 

% section notation (end)

\input{qm2pi.process.calculi} 

% section concurrent_process_calculi_and_spatial_logics_ (end)
    
%\input{qm2pi.knots2pi} 

%\input{qm2pi.trefoil} 

%\input{qm2pi.mainthm} 

% subsection basic_interpretation (end)

%\input{qm2pi.rho.presentation} 
\subsection{The syntax and semantics of the notation system}\label{sub:the_syntax_and_semantics_of_the_notation_system} % (fold)

We now summarize a technical presentation of the calculus that
embodies our theory of dynamics. The typical presentation of such a
calculus follows the style of giving generators and relations on
them. The grammar, below, describing term constructors, freely
generates the set of processes, $\Proc$. This set is then quotiented
by a relation known as structural congruence and it is over this set
that the notion of dynamics is expressed. This presentation is
essentially that of \cite{MeredithR05} with the addition of
polyadicity and summation. For readability we have relegated some of
the technical subtleties to an appendix.

\subsubsection{Process grammar}\label{subsub:process_grammar}

\begin{mathpar}
  \inferrule* [lab=synchronization] {} {{M} \bc \pzero \;|\; x?F \;|\; x!C }
  \and
  \inferrule* [lab=abstraction] {} {{F} \bc (x)P}
  \and
  \inferrule* [lab=concretion] {} {{C} \bc \langle Q \rangle}
  \and
  \inferrule* [lab=process] {} {{P,Q} \bc M \;| \;P|Q \;|\; @{x}}
  \and
  \inferrule* [lab=name] {} {{x} \bc \quotep{P}}
\end{mathpar} 

Note that $\vec{x}$ (resp. $\vec{P}$) denotes a vector of names
(resp. processes) of length $|\vec{x}|$ (resp. $|\vec{P}|$). We adopt
the following useful abbreviations.

\begin{mathpar}
   x?(\vec{y}).P := x.(\vec{y})P \and  x\clift{\vec{P}} := x.\clift{\vec{P}}
   \and x!(y) := \lift{x}{\dropn{y}}
   \and \Pi_{i=0}^{n-1}P_i := P_0 | \ldots | P_{n-1}
\end{mathpar}

\subsubsection{Structural congruence}

\paragraph{Free and bound names and alpha-equivalence.} At the
core of structural equivalence is alpha-equivalence which identifies
process that are the same up to a change of variable. Formally, we
recognize the distinction between free and bound names. The free names
of a process, $\freenames{P}$, may be calculated recursively as
follows:

\begin{mathpar}
\freenames{\pzero} := \emptyset
  \and \\
  \freenames{x?(y).P} := \{ x \} \cup (\freenames{P} \setminus \{ y \})
  \and 
  \freenames{x!\langle P \rangle} := \{ x \} \cup \{ P \} 
  \and \\
  \freenames{P|Q} := \freenames{P} \cup \freenames{Q}
  \and \\
  \freenames{@{x}} := \{ x \}
\end{mathpar}

$\pi$
$\quotep{\pi}$

$\freenames{-} : \pi \to \mathcal{P}(\quotep{\pi})$

\begin{eqnarray*}
  \freenames{\pzero} & := & \emptyset \\
  \freenames{x?(y).P} & := & \{ x \} \cup (\freenames{P} \setminus \{ y \}) \\
  \freenames{x!\langle P \rangle} & := & \{ x \} \cup \{ P \} \\
  \freenames{P|Q} & := & \freenames{P} \cup \freenames{Q} \\
  \freenames{\dropn{x}} & := & \{ x \}
\end{eqnarray*}

The bound names of a process, $\boundnames{P}$, are those names occurring in $P$
that are not free. For example, in $x?(y).0$, the name $x$ is free, while $y$ is bound.

\begin{mathpar}
  \inferrule* [lab=monoidal-laws] {} { P|Q \equiv Q|P \and P|0 \equiv P \and P|(Q|R) \equiv (P|Q)|R }
\end{mathpar}

\begin{mathpar}
  \inferrule* [lab=alpha-equivalence] {} { (x)P \equiv (y)P\{y/x\} \and y \not\in \freenames{P} }
\end{mathpar}

\begin{definition}
Then two processes, $P,Q$, are alpha-equivalent if $P = Q\{\vec{y}/\vec{x}\}$ for
some $\vec{x} \in \boundnames{Q},\vec{y} \in \boundnames{P}$, where $Q\{\vec{y}/\vec{x}\}$
denotes the capture-avoiding substitution of $\vec{y}$ for $\vec{x}$ in $Q$.
\end{definition}

\begin{definition}
  The {\em structural congruence} \cite{SangiorgiWalker} , $\equiv$,
  between processes is the least congruence containing
  alpha-equivalence, satisfying the abelian monoid laws
  (associativity, commutativity and $\pzero$ as identity) for parallel
  composition $|$ and for summation $+$.
\end{definition}

\subsection{Name equivalence}

We take name equivalence, written $\nameeq$, to be the smallest
equivalence relation generated by the following rules.

\begin{mathpar}
\inferrule*[lab=Quote-drop]
{ }
{ \quotep{@{x}} \nameeq x }

\inferrule*[lab=Struct-equiv]
{ P \scong Q }
{ \quotep{P} \nameeq \quotep{Q} }
\end{mathpar}

The astute reader will have noticed that the mutual recursion of names
and processes imposes a mutual recursion on alpha-equivalence and
structural equivalence via name-equivalence. Fortunately, all of this
works out pleasantly and we may calculate in the natural way, free of
concern. The reader interested in the details is referred to the
appendix \ref{appendix:rho_details}.

\subsection{Substitution}

We use $\Proc$ for the set of processes, $\QProc$ for the set of
names, and $\id{\{}\vec{y} / \vec{x} \id{\}}$ to denote partial maps,
$s : \QProc \rightarrow \QProc$. A map, $s$ lifts, uniquely, to a map
on process terms, $\widehat{s} : \Proc \rightarrow \Proc$ by the
following equations.

\begin{mathpar}
  (0) \psubstp{Q}{P} := 0 \\
  (R \juxtap S) \psubstp{Q}{P}
  :=    
  (R)\psubstp{Q}{P} \juxtap (S) \psubstp{Q}{P} \\
  (x?(y).R) \psubstp{Q}{P}    
  :=    
  (x)\substp{Q}{P} (z)\concat( (R \psubstn{z}{y}) \psubstp{Q}{P} ) \\
  (\lift{x}{R}) \psubstp{Q}{P}  
  :=
  \lift{(x)\substp{Q}{P}}{ R \psubstp{Q}{P} } \\
%   (\dropn{x})  \psubstp{Q}{P}       
%   := 
%   \left\{ 
%     \begin{array}{ccc} 
%       \dropn{\quotep{Q}} & & x \nameeq \quotep{P} \\
%       \dropn{x} & & otherwise \\
%     \end{array}
%   \right. 
  (\dropn{x})  \psubstp{Q}{P}       
  := 
  \left\{ 
    \begin{array}{ccc} 
      Q & & x \nameeq \quotep{P} \\
      \dropn{x} & & otherwise \\
    \end{array}
  \right.
\end{mathpar}
 

where

\begin{eqnarray}
  (x)\id{\{} \lpquote Q \rpquote / \lpquote P \rpquote \id{\}}            = 
  \left\{ 
    \begin{array}{ccc}
      \lpquote Q \rpquote & & x \nameeq \lpquote P \rpquote \\
      x & & otherwise \\
    \end{array}
  \right. \nonumber
\end{eqnarray}

and $z$ is chosen distinct from $\quotep{P}$, $\quotep{Q}$, the free
names in $Q$, and all the names in $R$. Our $\alpha$-equivalence will
be built in the standard way from this substitution.

\begin{remark}\label{rem:no_self_referential_names}
  One consequence of these definitions is that $\forall P. \quotep{P}
  \not\in \freenames{P}$.
\end{remark}

\subsection{ Dynamic quote: an example }

Anticipating something of what's to come, consider applying the
substitution, $\widehat{\id{\{}u / z \id{\}}}$, to the following pair
of processes, $\lift{w}{y!(z)}$ and $w[ \lpquote y!(z) \rpquote ]$.

\begin{eqnarray}
	\lift{w}{y!(z)}\widehat{\id{\{}u / z \id{\}}}
		& = &
		\lift{w}{y!(u)} \nonumber\\
	w[ \lpquote y!(z) \rpquote ] \widehat{ \id{\{}u / z \id{\}} }
		& = &
		w[ \lpquote y!(z) \rpquote ] \nonumber
\end{eqnarray}

Because the body of the process between quotes is impervious to
substitution, we get radically different answers. In fact, by
examining the first process in an input context,
e.g. $x?(z).\lift{w}{y!(z)}$, we see that the process under the lift
operator may be shaped by prefixed inputs binding a name inside it. In
this sense, the lift operator will be seen as a way to dynamically
construct processes before reifying them as names.

Finally equipped with these standard features we can present the
dynamics of the calculus.

\subsubsection{Operational semantics} 

Finally, we introduce the computational dynamics. What marks these
algebras as distinct from other more traditionally studied algebraic
structures, e.g. vector spaces or polynomial rings, is the manner in
which dynamics is captured. In traditional structures, dynamics is typically
expressed through morphisms between such structures, as in linear maps
between vector spaces or morphisms between rings. In algebras
associated with the semantics of computation, the dynamics is
expressed as part of the algebraic structure itself, through a
reduction reduction relation typically denoted by $\red$. Below, we
give a recursive presentation of this relation for the calculus used
in the encoding.

$\red \subseteq \pi \times \pi$
$\red : \pi \to \mathcal{P}(\pi)$

\begin{mathpar}
  \inferrule* [lab=Comm] { \textsf{match}( x_{src}, x_{trgt} ) } { x_{trgt}?(y)P \; | \; x_{src}!\langle {Q} \rangle \red P\{\quotep{Q}/y}\} }
  \and \\
  \inferrule* [lab=Par] {{P} \red {P}'} {{{P} | {Q}} \red {{P}' | {Q}}}
  \and
  \inferrule* [lab=Equiv]{{{P} \scong {P}'} \andalso {{P}' \red {Q}'} \andalso {{Q}' \scong {Q}}}{{P} \red {Q}}
\end{mathpar}

\begin{eqnarray*}
  match_{\equiv} (\quotep{P},\quotep{Q}) & := & P \equiv Q \\
  match_{\dagger}(\quotep{P},\quotep{Q}) & := & \forall R. P|Q \red^{*} R => R \red^{*} 0 \\
  match_{K}(\quotep{P},\quotep{Q}) & := & K \mbox{ for some context } K
\end{eqnarray*}

$u?(x)P | u!\langle Q \rangle \red P\{\quotep{Q}/x\}$

%We write $\wred$ for $\red^*$, and $P\red$ if $\exists Q $ such that $ P \red Q$.
We write $P\red$ if $\exists Q $ such that $ P \red Q$ and $P\not\red$, otherwise.

\section{Replication}

As mentioned before, it is known that replication (and hence
recursion) can be implemented in a higher-order process algebra
\cite{SangiorgiWalker}. As our first example of calculation with the
machinery thus far presented we give the construction explicitly in
the {\rhoc}.

\begin{eqnarray}
	D_{x} & := & \prefix{x}{y}{(\binpar{\outputp{x}{y}}{@{y}})} \nonumber\\
	\bangp_{x}{P} & := & \binpar{{x}!\langle{\binpar{D_{x}}{P}}\rangle}{D_{x}} \nonumber
\end{eqnarray}

\begin{eqnarray}
	\bangp_{x}{P} & & \nonumber\\
	=
	& {x}!\langle{(\prefix{x}{y}{(\outputp{x}{y} | @{y})) | P}}\rangle 
	      | \prefix{x}{y}{(\outputp{x}{y} | @{y})} & \nonumber\\
	\red
	& (\outputp{x}{y} | @{y})\substn{\quotep{(\prefix{x}{y}{(@{y} | \outputp{x}{y})) | P}}}{y} & \nonumber\\
	=
	& \outputp{x}{\quotep{(\prefix{x}{y}{(\outputp{x}{y} | @{y})) | P}}}
	  | {(\prefix{x}{y}{(\outputp{x}{y} | @{y})) | P}} & \nonumber\\
	\red
	& \ldots & \nonumber\\
	\red^*
	& P | P | \ldots & \nonumber
\end{eqnarray}

Of course, this encoding, as an implementation, runs away, unfolding
$\bangp{P}$ eagerly. A lazier and more implementable replication
operator, restricted to input-guarded processes, may be obtained as follows.

\begin{eqnarray}
\bangp{\prefix{u}{v}{P}} 
	:= 
	\binpar{\lift{x}{\prefix{u}{v}{(\binpar{D(x)}{P})}}}{D(x)} \nonumber
\end{eqnarray}

\begin{remark}
  Note that the lazier definition still does not deal with summation
  or mixed summation (i.e. sums over input and output). The reader is
  invited to construct definitions of replication that deal with these
  features. 

  Further, the definitions are parameterized in a name, $x$. Can you,
  gentle reader, make a definition that eliminates this parameter and
  guarantees no accidental interaction between the replication
  machinery and the process being replicated -- i.e. no accidental
  sharing of names used by the process to get its work done and the
  name(s) used by the replication to effect copying. This latter
  revision of the definition of replication is crucial to obtaining
  the expected identity $!!P \sim !P$.
\end{remark}

\begin{remark}\label{rem:paradoxical_combinator}
  The reader familiar with the lambda calculus will have noticed the
  similarity between $D$ and the paradoxical combinator.

  [Ed. note: the existence of this seems to suggest we have to be more
  restrictive on the set of processes and names we admit if we are to
  support no-cloning.]
\end{remark}

\subsubsection{Bisimulation}

The computational dynamics gives rise to another kind of equivalence,
the equivalence of computational behavior. As previously mentioned
this is typically captured \emph{via} some form of bisimulation.

% The notion we use in this paper is weak barbed bisimulation
% \cite{milner91polyadicpi}.

The notion we use in this paper is derived from weak barbed
bisimulation \cite{milner91polyadicpi}. 

\begin{definition}
An \emph{observation relation}, $\downarrow_{\mathcal N}$, over a set
of names, $\mathcal N$, is the smallest relation satisfying the rules
below.

\infrule[Out-barb]{y \in {\mathcal N}, \; x \nameeq y}
		  {\outputp{x}{v} \downarrow_{\mathcal N} x}
\infrule[Par-barb]{\mbox{$P\downarrow_{\mathcal N} x$ or $Q\downarrow_{\mathcal N} x$}}
		  {\binpar{P}{Q} \downarrow_{\mathcal N} x}

We write $P \Downarrow_{\mathcal N} x$ if there is $Q$ such that 
$P \wred Q$ and $Q \downarrow_{\mathcal N} x$.
\end{definition}

\begin{definition}
%\label{def.bbisim}
An  ${\mathcal N}$-\emph{barbed bisimulation} over a set of names, ${\mathcal N}$, is a symmetric binary relation 
${\mathcal S}_{\mathcal N}$ between agents such that $P\rel{S}_{\mathcal N}Q$ implies:
\begin{enumerate}
\item If $P \red P'$ then $Q \wred Q'$ and $P'\rel{S}_{\mathcal N} Q'$.
\item If $P\downarrow_{\mathcal N} x$, then $Q\Downarrow_{\mathcal N} x$.
\end{enumerate}
$P$ is ${\mathcal N}$-barbed bisimilar to $Q$, written
$P \wbbisim_{\mathcal N} Q$, if $P \rel{S}_{\mathcal N} Q$ for some ${\mathcal N}$-barbed bisimulation ${\mathcal S}_{\mathcal N}$.
\end{definition}

$\mathcal{R} \subseteq \pi \times \pi$

$P \mathcal{R} Q => \forall P'. P \red P' \Rightarrow \exists Q'. Q \red Q', P' \mathcal{R} Q'$

$P \vdash x \Rightarrow Q \vdash x$

\begin{mathpar}
  \inferrule*[lab=Out-barb]{x \nameeq y}{{y}!\langle{Q}\rangle \vdash x}
  \and
  \inferrule*[lab=Par-barb]{\mbox{$P\vdash x$ or $Q\vdash x$}}{\binpar{P}{Q} \vdash x}
\end{mathpar}

\subsubsection{Contexts}

One of the principle advantages of computational calculi like the
$\pi$-calculus is a well-defined notion of context,
contextual-equivalence and a correlation between
contextual-equivalence and notions of bisimulation. The notion of
context allows the decomposition of a process into (sub-)process and
its syntactic environment, its context. Thus, a context may be
thought of as a process with a ``hole'' (written $\Box$) in it. The
application of a context $M$ to a process $P$, written $M[P]$, is
tantamount to filling the hole in $M$ with $P$. In this paper we do
not need the full weight of this theory, but do make use of the notion
of context in the proof the main theorem. 

\begin{mathpar}
  \inferrule* [lab=summation] {} {{M_{M},M_{N}} \bc \Box \;|\; x.M_{A} \;|\; M_{M}+M_{N}}
  \and
  \inferrule* [lab=agent] {} {{M_{A}} \bc (\vec{x})M_{P} \;| \; \clift{P_0,\ldots,M_{P},\ldots,P_N}}
  \and \\
  \inferrule* [lab=process] {} {{M_{P}} \bc M_{N} \;| \;P|M_{P} }
\end{mathpar} 

\begin{mathpar}
  \inferrule* [lab=sychronization] {} {M_{N} \bc \Box \;|\; x?M_{F} \;|\; x!M_{C}}
  \and
  \inferrule* [lab=abstraction] {} {{M_{F}} \bc (x)M_{P} }
  \and
  \inferrule* [lab=concretion] {} {{M_{C}} \bc \langle M_{P} \rangle }
  \and \\
  \inferrule* [lab=process] {} {{M_{P}} \bc M_{N} \;| \;P|M_{P} }
\end{mathpar}

\begin{definition}[contextual application] Given a context $M$, and
  process $P$, we define the \emph{contextual application}, $M[P] :=
  M\{P/\Box\}$. That is, the contextual application of M to P is the
  substitution of $P$ for $\Box$ in $M$.
\end{definition}

$\meaningof{-} : L \to \mathcal{P}(\pi)$

\begin{mathpar}
  \inferrule* [lab=collection] {} {\meaningof{true} = \pi, \and \meaningof{~E} = \pi \setminus \meaningof{E}, \and \meaningof{E_{1} \& E_{2}} = \meaningof{E_{1}} \cap \meaningof{E_{2}}}
\end{mathpar}

\begin{mathpar}
  \inferrule* [lab=structure] {} {\meaningof{0} = \{ P \in \pi | P \equiv 0 \}, \and \\ \meaningof{E_1 | E_2} = \{ P \in \pi | P \equiv P_{1} | P_{2}, P_{1} \in \meaningof{E_{1}}, P_{2} \in \meaningof{E_2}\} }
\end{mathpar}

\begin{mathpar}
 \inferrule* [lab=behavior] {} {\meaningof{\langle a?b \rangle E} = \{ P \in \pi | P \equiv Q | u?(y)P', \\ \and \\\\ \and \\ \;\;\; u \in \meaningof{a}, \forall z.P'\{z/y\} \in \meaningof{E\{z/b\}}\}, \and \\ \meaningof{a!E} = \{ P \in \pi | P \equiv Q | x!\langle P' \rangle, x \in \meaningof{a} P' \in \meaningof{E}\} }
\end{mathpar}

\begin{mathpar}
 \inferrule* [lab=nominal] {} {\meaningof{\quotep{E}} = \{ \quotep{P} \in \quotep{\pi} | P \in \meaningof{E} \}, \and \meaningof{\quotep{P}} = \{ \quotep{Q} \in \quotep{\pi} | P \equiv Q \} \and \\ \meaningof{@\quotep{E}} = \{ P \in \pi | P \equiv @x, x \in \meaningof{E} \}}
\end{mathpar}

\begin{eqnarray*}
  \\
  \meaningof{-} : TS \to ST
\end{eqnarray*}

\begin{eqnarray*}
  \\
  L : TS \to ST
\end{eqnarray*}

\begin{eqnarray*}
  \\
  P \models E \iff P \in \meaningof{E}
\end{eqnarray*}

\begin{eqnarray*}
  P \approx_{L} Q \iff \forall E \in L. P \models E \iff Q \models E
\end{eqnarray*}

\begin{eqnarray*}
  P \approx_{K} Q
\end{eqnarray*}

\begin{eqnarray*}
  P \approx Q
\end{eqnarray*}

$\approx_{K} = \approx = \approx_{L}$

\subsubsection{Contextual duality}

Note that contexts extend the quotation operation to a family of
operations from processes to names. Given a context, $M$, we can
define a \emph{nominal context}, $\quotep{M}$ by $\quotep{M}[P] :=
\quotep{M[P]}$. To foreshadow what is to come we observe that these
operations enjoy a duality with processes very much like the duality
between vectors and maps from vectors to scalars.

Further, because the calculus is essentially higher-order, we have a
correspondence between contexts and processes. More specifically,
given a name $x$ and a context $M$ we can construct $M^{*}_{x}$ such
that 

\begin{mathpar}
  M^{*}_{x} | \lift{x}{P} \red M[P]
\end{mathpar}

namely,

\begin{mathpar}
  M^{*}_{x} := x?(u).M[\dropn{u}]
\end{mathpar}

The dependence of $M^{*}_{x}$ on a name makes it an abstraction, 

\begin{mathpar}
  M^{*} := (x)x?(u).M[\dropn{u}]
\end{mathpar}

\subsection{Additional notation}

It will sometimes be convenient to denote the process a name
quotes. We already have the notation $x = \quotep{P}$, but it will be
convenient to introduce an alternate notation, $\procn{x}$, when we
want to emphasize the connection to the use of the name. Note that, by
virtue of name equivalence, $\quotep{\procn{x}} \nameeq x$; so, the
notation is consistent with previous definitions.

Further, because names have structure it is possible to effect
substitutions on the basis of that structure. This means we need to
upgrade our notation for substitutions, which we accomplish by
adapting comprehension notation. Thus,

\begin{mathpar}
  P\{ y / x : x \in S \}
\end{mathpar}

is interpreted to mean the process derived from P by replacing (in a
capture-avoiding manner) each occurrence of $x$ in $S$ by $y$. For example,

\begin{mathpar}
  P\{ \quotep{\procn{x}|\procn{x}} / x : x \in \freenames{P} \}
\end{mathpar}

will replace each (occurrence) of a free name $x$ in $P$ by
$\quotep{\procn{x}|\procn{x}}$.

Also, we will avail ourselves of the notation $x^{L}$ and $x^{R}$ to
denote injections of a name into disjoint copies of the name
space. There are numerous ways to accomplish this. One example can be
found in \cite{MeredithR05}. This notation overloads to vectors of
names: $\vec{x}^{\pi} := (x_{i}^{\pi} \; : \; 0 \leq i < |\vec{x}| )$ where $\pi \in \{L,R\}$.

We also use $P^{\Box} := P|\Box$.

In \cite{MeredithR05} an interpretation of the new operator is
given. It turns out that there are several possible interpretations
all enjoying the requisite algebraic properties of the operator (see
\cite{milner91polyadicpi}). We will therefore make liberal use of
$(\nu\; \vec{x})P$.

% subsection the_syntax_and_semantics_of_the_notation_system (end)   

\input{qm2pi.qmops} 

\input{qm2pi.sterngerlach} 

\input{qm2pi.metric} 

% section concurrent_process_calculi (end)

%\input{qm2pi.proofsketch}

% section proof sketch (end)

%\input{qm2pi.slviaknots} 

% section spatial logic via knots (end)

\input{qm2pi.conclusion}

% section conclusion (end)

%\input{qm2pi.dtcodes} 

% section wiring algorithm (end)

\input{qm2pi.ack} 

% section acknowledgments (end)

\newpage


\bibliographystyle{plain}   
\bibliography{../../biblios/main.bib}

\input{qm2pi.rhodetails}

\end{document}

 

% section notation (end)

\input{qm2pi.process.calculi} 

% section concurrent_process_calculi_and_spatial_logics_ (end)
    
%\documentclass[12pt]{llncs}
%\documentclass{jktr}

\usepackage[pdftex]{hyperref}                   
\usepackage {listings}
\usepackage {mathpartir}
\usepackage{bcprules}
%\usepackage{listings}
                       
\usepackage{graphicx} 
%\usepackage[margins=2.5cm,nohead,nofoot]{geometry}
%\usepackage{geometry}
\usepackage{amsfonts}
\usepackage{amstext}
\usepackage{latexsym}
\usepackage{amssymb}
\usepackage{color}


%\include{myPreamble}
\include{qm2pi.local} 

%\ifpdf
%\usepackage[pdftex]{graphicx}
%\else
%\usepackage{graphicx}
%\fi

 % \ifpdf
%  \usepackage{pdfsync}
%  \if


%\title{Brief Article}
%\author{David F. Snyder}
%\author{L.G. Meredith}

%\address{Dept. of Math., Texas State University--San Marcos, San Marcos, TX 78666}
       
\pagestyle{empty}


\begin{document}

\lstset{language=[Objective]Caml,frame=shadowbox}

\input{qm2pi.front}

% section front matter (end)

\input{qm2pi.intro} 
 
% section introduction (end)

% \input{qm2pi.knotations} 

% section notation (end)

\input{qm2pi.process.calculi} 

% section concurrent_process_calculi_and_spatial_logics_ (end)
    
%\input{qm2pi.knots2pi} 

%\input{qm2pi.trefoil} 

%\input{qm2pi.mainthm} 

% subsection basic_interpretation (end)

%\input{qm2pi.rho.presentation} 
\subsection{The syntax and semantics of the notation system}\label{sub:the_syntax_and_semantics_of_the_notation_system} % (fold)

We now summarize a technical presentation of the calculus that
embodies our theory of dynamics. The typical presentation of such a
calculus follows the style of giving generators and relations on
them. The grammar, below, describing term constructors, freely
generates the set of processes, $\Proc$. This set is then quotiented
by a relation known as structural congruence and it is over this set
that the notion of dynamics is expressed. This presentation is
essentially that of \cite{MeredithR05} with the addition of
polyadicity and summation. For readability we have relegated some of
the technical subtleties to an appendix.

\subsubsection{Process grammar}\label{subsub:process_grammar}

\begin{mathpar}
  \inferrule* [lab=synchronization] {} {{M} \bc \pzero \;|\; x?F \;|\; x!C }
  \and
  \inferrule* [lab=abstraction] {} {{F} \bc (x)P}
  \and
  \inferrule* [lab=concretion] {} {{C} \bc \langle Q \rangle}
  \and
  \inferrule* [lab=process] {} {{P,Q} \bc M \;| \;P|Q \;|\; @{x}}
  \and
  \inferrule* [lab=name] {} {{x} \bc \quotep{P}}
\end{mathpar} 

Note that $\vec{x}$ (resp. $\vec{P}$) denotes a vector of names
(resp. processes) of length $|\vec{x}|$ (resp. $|\vec{P}|$). We adopt
the following useful abbreviations.

\begin{mathpar}
   x?(\vec{y}).P := x.(\vec{y})P \and  x\clift{\vec{P}} := x.\clift{\vec{P}}
   \and x!(y) := \lift{x}{\dropn{y}}
   \and \Pi_{i=0}^{n-1}P_i := P_0 | \ldots | P_{n-1}
\end{mathpar}

\subsubsection{Structural congruence}

\paragraph{Free and bound names and alpha-equivalence.} At the
core of structural equivalence is alpha-equivalence which identifies
process that are the same up to a change of variable. Formally, we
recognize the distinction between free and bound names. The free names
of a process, $\freenames{P}$, may be calculated recursively as
follows:

\begin{mathpar}
\freenames{\pzero} := \emptyset
  \and \\
  \freenames{x?(y).P} := \{ x \} \cup (\freenames{P} \setminus \{ y \})
  \and 
  \freenames{x!\langle P \rangle} := \{ x \} \cup \{ P \} 
  \and \\
  \freenames{P|Q} := \freenames{P} \cup \freenames{Q}
  \and \\
  \freenames{@{x}} := \{ x \}
\end{mathpar}

$\pi$
$\quotep{\pi}$

$\freenames{-} : \pi \to \mathcal{P}(\quotep{\pi})$

\begin{eqnarray*}
  \freenames{\pzero} & := & \emptyset \\
  \freenames{x?(y).P} & := & \{ x \} \cup (\freenames{P} \setminus \{ y \}) \\
  \freenames{x!\langle P \rangle} & := & \{ x \} \cup \{ P \} \\
  \freenames{P|Q} & := & \freenames{P} \cup \freenames{Q} \\
  \freenames{\dropn{x}} & := & \{ x \}
\end{eqnarray*}

The bound names of a process, $\boundnames{P}$, are those names occurring in $P$
that are not free. For example, in $x?(y).0$, the name $x$ is free, while $y$ is bound.

\begin{mathpar}
  \inferrule* [lab=monoidal-laws] {} { P|Q \equiv Q|P \and P|0 \equiv P \and P|(Q|R) \equiv (P|Q)|R }
\end{mathpar}

\begin{mathpar}
  \inferrule* [lab=alpha-equivalence] {} { (x)P \equiv (y)P\{y/x\} \and y \not\in \freenames{P} }
\end{mathpar}

\begin{definition}
Then two processes, $P,Q$, are alpha-equivalent if $P = Q\{\vec{y}/\vec{x}\}$ for
some $\vec{x} \in \boundnames{Q},\vec{y} \in \boundnames{P}$, where $Q\{\vec{y}/\vec{x}\}$
denotes the capture-avoiding substitution of $\vec{y}$ for $\vec{x}$ in $Q$.
\end{definition}

\begin{definition}
  The {\em structural congruence} \cite{SangiorgiWalker} , $\equiv$,
  between processes is the least congruence containing
  alpha-equivalence, satisfying the abelian monoid laws
  (associativity, commutativity and $\pzero$ as identity) for parallel
  composition $|$ and for summation $+$.
\end{definition}

\subsection{Name equivalence}

We take name equivalence, written $\nameeq$, to be the smallest
equivalence relation generated by the following rules.

\begin{mathpar}
\inferrule*[lab=Quote-drop]
{ }
{ \quotep{@{x}} \nameeq x }

\inferrule*[lab=Struct-equiv]
{ P \scong Q }
{ \quotep{P} \nameeq \quotep{Q} }
\end{mathpar}

The astute reader will have noticed that the mutual recursion of names
and processes imposes a mutual recursion on alpha-equivalence and
structural equivalence via name-equivalence. Fortunately, all of this
works out pleasantly and we may calculate in the natural way, free of
concern. The reader interested in the details is referred to the
appendix \ref{appendix:rho_details}.

\subsection{Substitution}

We use $\Proc$ for the set of processes, $\QProc$ for the set of
names, and $\id{\{}\vec{y} / \vec{x} \id{\}}$ to denote partial maps,
$s : \QProc \rightarrow \QProc$. A map, $s$ lifts, uniquely, to a map
on process terms, $\widehat{s} : \Proc \rightarrow \Proc$ by the
following equations.

\begin{mathpar}
  (0) \psubstp{Q}{P} := 0 \\
  (R \juxtap S) \psubstp{Q}{P}
  :=    
  (R)\psubstp{Q}{P} \juxtap (S) \psubstp{Q}{P} \\
  (x?(y).R) \psubstp{Q}{P}    
  :=    
  (x)\substp{Q}{P} (z)\concat( (R \psubstn{z}{y}) \psubstp{Q}{P} ) \\
  (\lift{x}{R}) \psubstp{Q}{P}  
  :=
  \lift{(x)\substp{Q}{P}}{ R \psubstp{Q}{P} } \\
%   (\dropn{x})  \psubstp{Q}{P}       
%   := 
%   \left\{ 
%     \begin{array}{ccc} 
%       \dropn{\quotep{Q}} & & x \nameeq \quotep{P} \\
%       \dropn{x} & & otherwise \\
%     \end{array}
%   \right. 
  (\dropn{x})  \psubstp{Q}{P}       
  := 
  \left\{ 
    \begin{array}{ccc} 
      Q & & x \nameeq \quotep{P} \\
      \dropn{x} & & otherwise \\
    \end{array}
  \right.
\end{mathpar}
 

where

\begin{eqnarray}
  (x)\id{\{} \lpquote Q \rpquote / \lpquote P \rpquote \id{\}}            = 
  \left\{ 
    \begin{array}{ccc}
      \lpquote Q \rpquote & & x \nameeq \lpquote P \rpquote \\
      x & & otherwise \\
    \end{array}
  \right. \nonumber
\end{eqnarray}

and $z$ is chosen distinct from $\quotep{P}$, $\quotep{Q}$, the free
names in $Q$, and all the names in $R$. Our $\alpha$-equivalence will
be built in the standard way from this substitution.

\begin{remark}\label{rem:no_self_referential_names}
  One consequence of these definitions is that $\forall P. \quotep{P}
  \not\in \freenames{P}$.
\end{remark}

\subsection{ Dynamic quote: an example }

Anticipating something of what's to come, consider applying the
substitution, $\widehat{\id{\{}u / z \id{\}}}$, to the following pair
of processes, $\lift{w}{y!(z)}$ and $w[ \lpquote y!(z) \rpquote ]$.

\begin{eqnarray}
	\lift{w}{y!(z)}\widehat{\id{\{}u / z \id{\}}}
		& = &
		\lift{w}{y!(u)} \nonumber\\
	w[ \lpquote y!(z) \rpquote ] \widehat{ \id{\{}u / z \id{\}} }
		& = &
		w[ \lpquote y!(z) \rpquote ] \nonumber
\end{eqnarray}

Because the body of the process between quotes is impervious to
substitution, we get radically different answers. In fact, by
examining the first process in an input context,
e.g. $x?(z).\lift{w}{y!(z)}$, we see that the process under the lift
operator may be shaped by prefixed inputs binding a name inside it. In
this sense, the lift operator will be seen as a way to dynamically
construct processes before reifying them as names.

Finally equipped with these standard features we can present the
dynamics of the calculus.

\subsubsection{Operational semantics} 

Finally, we introduce the computational dynamics. What marks these
algebras as distinct from other more traditionally studied algebraic
structures, e.g. vector spaces or polynomial rings, is the manner in
which dynamics is captured. In traditional structures, dynamics is typically
expressed through morphisms between such structures, as in linear maps
between vector spaces or morphisms between rings. In algebras
associated with the semantics of computation, the dynamics is
expressed as part of the algebraic structure itself, through a
reduction reduction relation typically denoted by $\red$. Below, we
give a recursive presentation of this relation for the calculus used
in the encoding.

$\red \subseteq \pi \times \pi$
$\red : \pi \to \mathcal{P}(\pi)$

\begin{mathpar}
  \inferrule* [lab=Comm] { \textsf{match}( x_{src}, x_{trgt} ) } { x_{trgt}?(y)P \; | \; x_{src}!\langle {Q} \rangle \red P\{\quotep{Q}/y}\} }
  \and \\
  \inferrule* [lab=Par] {{P} \red {P}'} {{{P} | {Q}} \red {{P}' | {Q}}}
  \and
  \inferrule* [lab=Equiv]{{{P} \scong {P}'} \andalso {{P}' \red {Q}'} \andalso {{Q}' \scong {Q}}}{{P} \red {Q}}
\end{mathpar}

\begin{eqnarray*}
  match_{\equiv} (\quotep{P},\quotep{Q}) & := & P \equiv Q \\
  match_{\dagger}(\quotep{P},\quotep{Q}) & := & \forall R. P|Q \red^{*} R => R \red^{*} 0 \\
  match_{K}(\quotep{P},\quotep{Q}) & := & K \mbox{ for some context } K
\end{eqnarray*}

$u?(x)P | u!\langle Q \rangle \red P\{\quotep{Q}/x\}$

%We write $\wred$ for $\red^*$, and $P\red$ if $\exists Q $ such that $ P \red Q$.
We write $P\red$ if $\exists Q $ such that $ P \red Q$ and $P\not\red$, otherwise.

\section{Replication}

As mentioned before, it is known that replication (and hence
recursion) can be implemented in a higher-order process algebra
\cite{SangiorgiWalker}. As our first example of calculation with the
machinery thus far presented we give the construction explicitly in
the {\rhoc}.

\begin{eqnarray}
	D_{x} & := & \prefix{x}{y}{(\binpar{\outputp{x}{y}}{@{y}})} \nonumber\\
	\bangp_{x}{P} & := & \binpar{{x}!\langle{\binpar{D_{x}}{P}}\rangle}{D_{x}} \nonumber
\end{eqnarray}

\begin{eqnarray}
	\bangp_{x}{P} & & \nonumber\\
	=
	& {x}!\langle{(\prefix{x}{y}{(\outputp{x}{y} | @{y})) | P}}\rangle 
	      | \prefix{x}{y}{(\outputp{x}{y} | @{y})} & \nonumber\\
	\red
	& (\outputp{x}{y} | @{y})\substn{\quotep{(\prefix{x}{y}{(@{y} | \outputp{x}{y})) | P}}}{y} & \nonumber\\
	=
	& \outputp{x}{\quotep{(\prefix{x}{y}{(\outputp{x}{y} | @{y})) | P}}}
	  | {(\prefix{x}{y}{(\outputp{x}{y} | @{y})) | P}} & \nonumber\\
	\red
	& \ldots & \nonumber\\
	\red^*
	& P | P | \ldots & \nonumber
\end{eqnarray}

Of course, this encoding, as an implementation, runs away, unfolding
$\bangp{P}$ eagerly. A lazier and more implementable replication
operator, restricted to input-guarded processes, may be obtained as follows.

\begin{eqnarray}
\bangp{\prefix{u}{v}{P}} 
	:= 
	\binpar{\lift{x}{\prefix{u}{v}{(\binpar{D(x)}{P})}}}{D(x)} \nonumber
\end{eqnarray}

\begin{remark}
  Note that the lazier definition still does not deal with summation
  or mixed summation (i.e. sums over input and output). The reader is
  invited to construct definitions of replication that deal with these
  features. 

  Further, the definitions are parameterized in a name, $x$. Can you,
  gentle reader, make a definition that eliminates this parameter and
  guarantees no accidental interaction between the replication
  machinery and the process being replicated -- i.e. no accidental
  sharing of names used by the process to get its work done and the
  name(s) used by the replication to effect copying. This latter
  revision of the definition of replication is crucial to obtaining
  the expected identity $!!P \sim !P$.
\end{remark}

\begin{remark}\label{rem:paradoxical_combinator}
  The reader familiar with the lambda calculus will have noticed the
  similarity between $D$ and the paradoxical combinator.

  [Ed. note: the existence of this seems to suggest we have to be more
  restrictive on the set of processes and names we admit if we are to
  support no-cloning.]
\end{remark}

\subsubsection{Bisimulation}

The computational dynamics gives rise to another kind of equivalence,
the equivalence of computational behavior. As previously mentioned
this is typically captured \emph{via} some form of bisimulation.

% The notion we use in this paper is weak barbed bisimulation
% \cite{milner91polyadicpi}.

The notion we use in this paper is derived from weak barbed
bisimulation \cite{milner91polyadicpi}. 

\begin{definition}
An \emph{observation relation}, $\downarrow_{\mathcal N}$, over a set
of names, $\mathcal N$, is the smallest relation satisfying the rules
below.

\infrule[Out-barb]{y \in {\mathcal N}, \; x \nameeq y}
		  {\outputp{x}{v} \downarrow_{\mathcal N} x}
\infrule[Par-barb]{\mbox{$P\downarrow_{\mathcal N} x$ or $Q\downarrow_{\mathcal N} x$}}
		  {\binpar{P}{Q} \downarrow_{\mathcal N} x}

We write $P \Downarrow_{\mathcal N} x$ if there is $Q$ such that 
$P \wred Q$ and $Q \downarrow_{\mathcal N} x$.
\end{definition}

\begin{definition}
%\label{def.bbisim}
An  ${\mathcal N}$-\emph{barbed bisimulation} over a set of names, ${\mathcal N}$, is a symmetric binary relation 
${\mathcal S}_{\mathcal N}$ between agents such that $P\rel{S}_{\mathcal N}Q$ implies:
\begin{enumerate}
\item If $P \red P'$ then $Q \wred Q'$ and $P'\rel{S}_{\mathcal N} Q'$.
\item If $P\downarrow_{\mathcal N} x$, then $Q\Downarrow_{\mathcal N} x$.
\end{enumerate}
$P$ is ${\mathcal N}$-barbed bisimilar to $Q$, written
$P \wbbisim_{\mathcal N} Q$, if $P \rel{S}_{\mathcal N} Q$ for some ${\mathcal N}$-barbed bisimulation ${\mathcal S}_{\mathcal N}$.
\end{definition}

$\mathcal{R} \subseteq \pi \times \pi$

$P \mathcal{R} Q => \forall P'. P \red P' \Rightarrow \exists Q'. Q \red Q', P' \mathcal{R} Q'$

$P \vdash x \Rightarrow Q \vdash x$

\begin{mathpar}
  \inferrule*[lab=Out-barb]{x \nameeq y}{{y}!\langle{Q}\rangle \vdash x}
  \and
  \inferrule*[lab=Par-barb]{\mbox{$P\vdash x$ or $Q\vdash x$}}{\binpar{P}{Q} \vdash x}
\end{mathpar}

\subsubsection{Contexts}

One of the principle advantages of computational calculi like the
$\pi$-calculus is a well-defined notion of context,
contextual-equivalence and a correlation between
contextual-equivalence and notions of bisimulation. The notion of
context allows the decomposition of a process into (sub-)process and
its syntactic environment, its context. Thus, a context may be
thought of as a process with a ``hole'' (written $\Box$) in it. The
application of a context $M$ to a process $P$, written $M[P]$, is
tantamount to filling the hole in $M$ with $P$. In this paper we do
not need the full weight of this theory, but do make use of the notion
of context in the proof the main theorem. 

\begin{mathpar}
  \inferrule* [lab=summation] {} {{M_{M},M_{N}} \bc \Box \;|\; x.M_{A} \;|\; M_{M}+M_{N}}
  \and
  \inferrule* [lab=agent] {} {{M_{A}} \bc (\vec{x})M_{P} \;| \; \clift{P_0,\ldots,M_{P},\ldots,P_N}}
  \and \\
  \inferrule* [lab=process] {} {{M_{P}} \bc M_{N} \;| \;P|M_{P} }
\end{mathpar} 

\begin{mathpar}
  \inferrule* [lab=sychronization] {} {M_{N} \bc \Box \;|\; x?M_{F} \;|\; x!M_{C}}
  \and
  \inferrule* [lab=abstraction] {} {{M_{F}} \bc (x)M_{P} }
  \and
  \inferrule* [lab=concretion] {} {{M_{C}} \bc \langle M_{P} \rangle }
  \and \\
  \inferrule* [lab=process] {} {{M_{P}} \bc M_{N} \;| \;P|M_{P} }
\end{mathpar}

\begin{definition}[contextual application] Given a context $M$, and
  process $P$, we define the \emph{contextual application}, $M[P] :=
  M\{P/\Box\}$. That is, the contextual application of M to P is the
  substitution of $P$ for $\Box$ in $M$.
\end{definition}

$\meaningof{-} : L \to \mathcal{P}(\pi)$

\begin{mathpar}
  \inferrule* [lab=collection] {} {\meaningof{true} = \pi, \and \meaningof{~E} = \pi \setminus \meaningof{E}, \and \meaningof{E_{1} \& E_{2}} = \meaningof{E_{1}} \cap \meaningof{E_{2}}}
\end{mathpar}

\begin{mathpar}
  \inferrule* [lab=structure] {} {\meaningof{0} = \{ P \in \pi | P \equiv 0 \}, \and \\ \meaningof{E_1 | E_2} = \{ P \in \pi | P \equiv P_{1} | P_{2}, P_{1} \in \meaningof{E_{1}}, P_{2} \in \meaningof{E_2}\} }
\end{mathpar}

\begin{mathpar}
 \inferrule* [lab=behavior] {} {\meaningof{\langle a?b \rangle E} = \{ P \in \pi | P \equiv Q | u?(y)P', \\ \and \\\\ \and \\ \;\;\; u \in \meaningof{a}, \forall z.P'\{z/y\} \in \meaningof{E\{z/b\}}\}, \and \\ \meaningof{a!E} = \{ P \in \pi | P \equiv Q | x!\langle P' \rangle, x \in \meaningof{a} P' \in \meaningof{E}\} }
\end{mathpar}

\begin{mathpar}
 \inferrule* [lab=nominal] {} {\meaningof{\quotep{E}} = \{ \quotep{P} \in \quotep{\pi} | P \in \meaningof{E} \}, \and \meaningof{\quotep{P}} = \{ \quotep{Q} \in \quotep{\pi} | P \equiv Q \} \and \\ \meaningof{@\quotep{E}} = \{ P \in \pi | P \equiv @x, x \in \meaningof{E} \}}
\end{mathpar}

\begin{eqnarray*}
  \\
  \meaningof{-} : TS \to ST
\end{eqnarray*}

\begin{eqnarray*}
  \\
  L : TS \to ST
\end{eqnarray*}

\begin{eqnarray*}
  \\
  P \models E \iff P \in \meaningof{E}
\end{eqnarray*}

\begin{eqnarray*}
  P \approx_{L} Q \iff \forall E \in L. P \models E \iff Q \models E
\end{eqnarray*}

\begin{eqnarray*}
  P \approx_{K} Q
\end{eqnarray*}

\begin{eqnarray*}
  P \approx Q
\end{eqnarray*}

$\approx_{K} = \approx = \approx_{L}$

\subsubsection{Contextual duality}

Note that contexts extend the quotation operation to a family of
operations from processes to names. Given a context, $M$, we can
define a \emph{nominal context}, $\quotep{M}$ by $\quotep{M}[P] :=
\quotep{M[P]}$. To foreshadow what is to come we observe that these
operations enjoy a duality with processes very much like the duality
between vectors and maps from vectors to scalars.

Further, because the calculus is essentially higher-order, we have a
correspondence between contexts and processes. More specifically,
given a name $x$ and a context $M$ we can construct $M^{*}_{x}$ such
that 

\begin{mathpar}
  M^{*}_{x} | \lift{x}{P} \red M[P]
\end{mathpar}

namely,

\begin{mathpar}
  M^{*}_{x} := x?(u).M[\dropn{u}]
\end{mathpar}

The dependence of $M^{*}_{x}$ on a name makes it an abstraction, 

\begin{mathpar}
  M^{*} := (x)x?(u).M[\dropn{u}]
\end{mathpar}

\subsection{Additional notation}

It will sometimes be convenient to denote the process a name
quotes. We already have the notation $x = \quotep{P}$, but it will be
convenient to introduce an alternate notation, $\procn{x}$, when we
want to emphasize the connection to the use of the name. Note that, by
virtue of name equivalence, $\quotep{\procn{x}} \nameeq x$; so, the
notation is consistent with previous definitions.

Further, because names have structure it is possible to effect
substitutions on the basis of that structure. This means we need to
upgrade our notation for substitutions, which we accomplish by
adapting comprehension notation. Thus,

\begin{mathpar}
  P\{ y / x : x \in S \}
\end{mathpar}

is interpreted to mean the process derived from P by replacing (in a
capture-avoiding manner) each occurrence of $x$ in $S$ by $y$. For example,

\begin{mathpar}
  P\{ \quotep{\procn{x}|\procn{x}} / x : x \in \freenames{P} \}
\end{mathpar}

will replace each (occurrence) of a free name $x$ in $P$ by
$\quotep{\procn{x}|\procn{x}}$.

Also, we will avail ourselves of the notation $x^{L}$ and $x^{R}$ to
denote injections of a name into disjoint copies of the name
space. There are numerous ways to accomplish this. One example can be
found in \cite{MeredithR05}. This notation overloads to vectors of
names: $\vec{x}^{\pi} := (x_{i}^{\pi} \; : \; 0 \leq i < |\vec{x}| )$ where $\pi \in \{L,R\}$.

We also use $P^{\Box} := P|\Box$.

In \cite{MeredithR05} an interpretation of the new operator is
given. It turns out that there are several possible interpretations
all enjoying the requisite algebraic properties of the operator (see
\cite{milner91polyadicpi}). We will therefore make liberal use of
$(\nu\; \vec{x})P$.

% subsection the_syntax_and_semantics_of_the_notation_system (end)   

\input{qm2pi.qmops} 

\input{qm2pi.sterngerlach} 

\input{qm2pi.metric} 

% section concurrent_process_calculi (end)

%\input{qm2pi.proofsketch}

% section proof sketch (end)

%\input{qm2pi.slviaknots} 

% section spatial logic via knots (end)

\input{qm2pi.conclusion}

% section conclusion (end)

%\input{qm2pi.dtcodes} 

% section wiring algorithm (end)

\input{qm2pi.ack} 

% section acknowledgments (end)

\newpage


\bibliographystyle{plain}   
\bibliography{../../biblios/main.bib}

\input{qm2pi.rhodetails}

\end{document}

 

%\documentclass[12pt]{llncs}
%\documentclass{jktr}

\usepackage[pdftex]{hyperref}                   
\usepackage {listings}
\usepackage {mathpartir}
\usepackage{bcprules}
%\usepackage{listings}
                       
\usepackage{graphicx} 
%\usepackage[margins=2.5cm,nohead,nofoot]{geometry}
%\usepackage{geometry}
\usepackage{amsfonts}
\usepackage{amstext}
\usepackage{latexsym}
\usepackage{amssymb}
\usepackage{color}


%\include{myPreamble}
\include{qm2pi.local} 

%\ifpdf
%\usepackage[pdftex]{graphicx}
%\else
%\usepackage{graphicx}
%\fi

 % \ifpdf
%  \usepackage{pdfsync}
%  \if


%\title{Brief Article}
%\author{David F. Snyder}
%\author{L.G. Meredith}

%\address{Dept. of Math., Texas State University--San Marcos, San Marcos, TX 78666}
       
\pagestyle{empty}


\begin{document}

\lstset{language=[Objective]Caml,frame=shadowbox}

\input{qm2pi.front}

% section front matter (end)

\input{qm2pi.intro} 
 
% section introduction (end)

% \input{qm2pi.knotations} 

% section notation (end)

\input{qm2pi.process.calculi} 

% section concurrent_process_calculi_and_spatial_logics_ (end)
    
%\input{qm2pi.knots2pi} 

%\input{qm2pi.trefoil} 

%\input{qm2pi.mainthm} 

% subsection basic_interpretation (end)

%\input{qm2pi.rho.presentation} 
\subsection{The syntax and semantics of the notation system}\label{sub:the_syntax_and_semantics_of_the_notation_system} % (fold)

We now summarize a technical presentation of the calculus that
embodies our theory of dynamics. The typical presentation of such a
calculus follows the style of giving generators and relations on
them. The grammar, below, describing term constructors, freely
generates the set of processes, $\Proc$. This set is then quotiented
by a relation known as structural congruence and it is over this set
that the notion of dynamics is expressed. This presentation is
essentially that of \cite{MeredithR05} with the addition of
polyadicity and summation. For readability we have relegated some of
the technical subtleties to an appendix.

\subsubsection{Process grammar}\label{subsub:process_grammar}

\begin{mathpar}
  \inferrule* [lab=synchronization] {} {{M} \bc \pzero \;|\; x?F \;|\; x!C }
  \and
  \inferrule* [lab=abstraction] {} {{F} \bc (x)P}
  \and
  \inferrule* [lab=concretion] {} {{C} \bc \langle Q \rangle}
  \and
  \inferrule* [lab=process] {} {{P,Q} \bc M \;| \;P|Q \;|\; @{x}}
  \and
  \inferrule* [lab=name] {} {{x} \bc \quotep{P}}
\end{mathpar} 

Note that $\vec{x}$ (resp. $\vec{P}$) denotes a vector of names
(resp. processes) of length $|\vec{x}|$ (resp. $|\vec{P}|$). We adopt
the following useful abbreviations.

\begin{mathpar}
   x?(\vec{y}).P := x.(\vec{y})P \and  x\clift{\vec{P}} := x.\clift{\vec{P}}
   \and x!(y) := \lift{x}{\dropn{y}}
   \and \Pi_{i=0}^{n-1}P_i := P_0 | \ldots | P_{n-1}
\end{mathpar}

\subsubsection{Structural congruence}

\paragraph{Free and bound names and alpha-equivalence.} At the
core of structural equivalence is alpha-equivalence which identifies
process that are the same up to a change of variable. Formally, we
recognize the distinction between free and bound names. The free names
of a process, $\freenames{P}$, may be calculated recursively as
follows:

\begin{mathpar}
\freenames{\pzero} := \emptyset
  \and \\
  \freenames{x?(y).P} := \{ x \} \cup (\freenames{P} \setminus \{ y \})
  \and 
  \freenames{x!\langle P \rangle} := \{ x \} \cup \{ P \} 
  \and \\
  \freenames{P|Q} := \freenames{P} \cup \freenames{Q}
  \and \\
  \freenames{@{x}} := \{ x \}
\end{mathpar}

$\pi$
$\quotep{\pi}$

$\freenames{-} : \pi \to \mathcal{P}(\quotep{\pi})$

\begin{eqnarray*}
  \freenames{\pzero} & := & \emptyset \\
  \freenames{x?(y).P} & := & \{ x \} \cup (\freenames{P} \setminus \{ y \}) \\
  \freenames{x!\langle P \rangle} & := & \{ x \} \cup \{ P \} \\
  \freenames{P|Q} & := & \freenames{P} \cup \freenames{Q} \\
  \freenames{\dropn{x}} & := & \{ x \}
\end{eqnarray*}

The bound names of a process, $\boundnames{P}$, are those names occurring in $P$
that are not free. For example, in $x?(y).0$, the name $x$ is free, while $y$ is bound.

\begin{mathpar}
  \inferrule* [lab=monoidal-laws] {} { P|Q \equiv Q|P \and P|0 \equiv P \and P|(Q|R) \equiv (P|Q)|R }
\end{mathpar}

\begin{mathpar}
  \inferrule* [lab=alpha-equivalence] {} { (x)P \equiv (y)P\{y/x\} \and y \not\in \freenames{P} }
\end{mathpar}

\begin{definition}
Then two processes, $P,Q$, are alpha-equivalent if $P = Q\{\vec{y}/\vec{x}\}$ for
some $\vec{x} \in \boundnames{Q},\vec{y} \in \boundnames{P}$, where $Q\{\vec{y}/\vec{x}\}$
denotes the capture-avoiding substitution of $\vec{y}$ for $\vec{x}$ in $Q$.
\end{definition}

\begin{definition}
  The {\em structural congruence} \cite{SangiorgiWalker} , $\equiv$,
  between processes is the least congruence containing
  alpha-equivalence, satisfying the abelian monoid laws
  (associativity, commutativity and $\pzero$ as identity) for parallel
  composition $|$ and for summation $+$.
\end{definition}

\subsection{Name equivalence}

We take name equivalence, written $\nameeq$, to be the smallest
equivalence relation generated by the following rules.

\begin{mathpar}
\inferrule*[lab=Quote-drop]
{ }
{ \quotep{@{x}} \nameeq x }

\inferrule*[lab=Struct-equiv]
{ P \scong Q }
{ \quotep{P} \nameeq \quotep{Q} }
\end{mathpar}

The astute reader will have noticed that the mutual recursion of names
and processes imposes a mutual recursion on alpha-equivalence and
structural equivalence via name-equivalence. Fortunately, all of this
works out pleasantly and we may calculate in the natural way, free of
concern. The reader interested in the details is referred to the
appendix \ref{appendix:rho_details}.

\subsection{Substitution}

We use $\Proc$ for the set of processes, $\QProc$ for the set of
names, and $\id{\{}\vec{y} / \vec{x} \id{\}}$ to denote partial maps,
$s : \QProc \rightarrow \QProc$. A map, $s$ lifts, uniquely, to a map
on process terms, $\widehat{s} : \Proc \rightarrow \Proc$ by the
following equations.

\begin{mathpar}
  (0) \psubstp{Q}{P} := 0 \\
  (R \juxtap S) \psubstp{Q}{P}
  :=    
  (R)\psubstp{Q}{P} \juxtap (S) \psubstp{Q}{P} \\
  (x?(y).R) \psubstp{Q}{P}    
  :=    
  (x)\substp{Q}{P} (z)\concat( (R \psubstn{z}{y}) \psubstp{Q}{P} ) \\
  (\lift{x}{R}) \psubstp{Q}{P}  
  :=
  \lift{(x)\substp{Q}{P}}{ R \psubstp{Q}{P} } \\
%   (\dropn{x})  \psubstp{Q}{P}       
%   := 
%   \left\{ 
%     \begin{array}{ccc} 
%       \dropn{\quotep{Q}} & & x \nameeq \quotep{P} \\
%       \dropn{x} & & otherwise \\
%     \end{array}
%   \right. 
  (\dropn{x})  \psubstp{Q}{P}       
  := 
  \left\{ 
    \begin{array}{ccc} 
      Q & & x \nameeq \quotep{P} \\
      \dropn{x} & & otherwise \\
    \end{array}
  \right.
\end{mathpar}
 

where

\begin{eqnarray}
  (x)\id{\{} \lpquote Q \rpquote / \lpquote P \rpquote \id{\}}            = 
  \left\{ 
    \begin{array}{ccc}
      \lpquote Q \rpquote & & x \nameeq \lpquote P \rpquote \\
      x & & otherwise \\
    \end{array}
  \right. \nonumber
\end{eqnarray}

and $z$ is chosen distinct from $\quotep{P}$, $\quotep{Q}$, the free
names in $Q$, and all the names in $R$. Our $\alpha$-equivalence will
be built in the standard way from this substitution.

\begin{remark}\label{rem:no_self_referential_names}
  One consequence of these definitions is that $\forall P. \quotep{P}
  \not\in \freenames{P}$.
\end{remark}

\subsection{ Dynamic quote: an example }

Anticipating something of what's to come, consider applying the
substitution, $\widehat{\id{\{}u / z \id{\}}}$, to the following pair
of processes, $\lift{w}{y!(z)}$ and $w[ \lpquote y!(z) \rpquote ]$.

\begin{eqnarray}
	\lift{w}{y!(z)}\widehat{\id{\{}u / z \id{\}}}
		& = &
		\lift{w}{y!(u)} \nonumber\\
	w[ \lpquote y!(z) \rpquote ] \widehat{ \id{\{}u / z \id{\}} }
		& = &
		w[ \lpquote y!(z) \rpquote ] \nonumber
\end{eqnarray}

Because the body of the process between quotes is impervious to
substitution, we get radically different answers. In fact, by
examining the first process in an input context,
e.g. $x?(z).\lift{w}{y!(z)}$, we see that the process under the lift
operator may be shaped by prefixed inputs binding a name inside it. In
this sense, the lift operator will be seen as a way to dynamically
construct processes before reifying them as names.

Finally equipped with these standard features we can present the
dynamics of the calculus.

\subsubsection{Operational semantics} 

Finally, we introduce the computational dynamics. What marks these
algebras as distinct from other more traditionally studied algebraic
structures, e.g. vector spaces or polynomial rings, is the manner in
which dynamics is captured. In traditional structures, dynamics is typically
expressed through morphisms between such structures, as in linear maps
between vector spaces or morphisms between rings. In algebras
associated with the semantics of computation, the dynamics is
expressed as part of the algebraic structure itself, through a
reduction reduction relation typically denoted by $\red$. Below, we
give a recursive presentation of this relation for the calculus used
in the encoding.

$\red \subseteq \pi \times \pi$
$\red : \pi \to \mathcal{P}(\pi)$

\begin{mathpar}
  \inferrule* [lab=Comm] { \textsf{match}( x_{src}, x_{trgt} ) } { x_{trgt}?(y)P \; | \; x_{src}!\langle {Q} \rangle \red P\{\quotep{Q}/y}\} }
  \and \\
  \inferrule* [lab=Par] {{P} \red {P}'} {{{P} | {Q}} \red {{P}' | {Q}}}
  \and
  \inferrule* [lab=Equiv]{{{P} \scong {P}'} \andalso {{P}' \red {Q}'} \andalso {{Q}' \scong {Q}}}{{P} \red {Q}}
\end{mathpar}

\begin{eqnarray*}
  match_{\equiv} (\quotep{P},\quotep{Q}) & := & P \equiv Q \\
  match_{\dagger}(\quotep{P},\quotep{Q}) & := & \forall R. P|Q \red^{*} R => R \red^{*} 0 \\
  match_{K}(\quotep{P},\quotep{Q}) & := & K \mbox{ for some context } K
\end{eqnarray*}

$u?(x)P | u!\langle Q \rangle \red P\{\quotep{Q}/x\}$

%We write $\wred$ for $\red^*$, and $P\red$ if $\exists Q $ such that $ P \red Q$.
We write $P\red$ if $\exists Q $ such that $ P \red Q$ and $P\not\red$, otherwise.

\section{Replication}

As mentioned before, it is known that replication (and hence
recursion) can be implemented in a higher-order process algebra
\cite{SangiorgiWalker}. As our first example of calculation with the
machinery thus far presented we give the construction explicitly in
the {\rhoc}.

\begin{eqnarray}
	D_{x} & := & \prefix{x}{y}{(\binpar{\outputp{x}{y}}{@{y}})} \nonumber\\
	\bangp_{x}{P} & := & \binpar{{x}!\langle{\binpar{D_{x}}{P}}\rangle}{D_{x}} \nonumber
\end{eqnarray}

\begin{eqnarray}
	\bangp_{x}{P} & & \nonumber\\
	=
	& {x}!\langle{(\prefix{x}{y}{(\outputp{x}{y} | @{y})) | P}}\rangle 
	      | \prefix{x}{y}{(\outputp{x}{y} | @{y})} & \nonumber\\
	\red
	& (\outputp{x}{y} | @{y})\substn{\quotep{(\prefix{x}{y}{(@{y} | \outputp{x}{y})) | P}}}{y} & \nonumber\\
	=
	& \outputp{x}{\quotep{(\prefix{x}{y}{(\outputp{x}{y} | @{y})) | P}}}
	  | {(\prefix{x}{y}{(\outputp{x}{y} | @{y})) | P}} & \nonumber\\
	\red
	& \ldots & \nonumber\\
	\red^*
	& P | P | \ldots & \nonumber
\end{eqnarray}

Of course, this encoding, as an implementation, runs away, unfolding
$\bangp{P}$ eagerly. A lazier and more implementable replication
operator, restricted to input-guarded processes, may be obtained as follows.

\begin{eqnarray}
\bangp{\prefix{u}{v}{P}} 
	:= 
	\binpar{\lift{x}{\prefix{u}{v}{(\binpar{D(x)}{P})}}}{D(x)} \nonumber
\end{eqnarray}

\begin{remark}
  Note that the lazier definition still does not deal with summation
  or mixed summation (i.e. sums over input and output). The reader is
  invited to construct definitions of replication that deal with these
  features. 

  Further, the definitions are parameterized in a name, $x$. Can you,
  gentle reader, make a definition that eliminates this parameter and
  guarantees no accidental interaction between the replication
  machinery and the process being replicated -- i.e. no accidental
  sharing of names used by the process to get its work done and the
  name(s) used by the replication to effect copying. This latter
  revision of the definition of replication is crucial to obtaining
  the expected identity $!!P \sim !P$.
\end{remark}

\begin{remark}\label{rem:paradoxical_combinator}
  The reader familiar with the lambda calculus will have noticed the
  similarity between $D$ and the paradoxical combinator.

  [Ed. note: the existence of this seems to suggest we have to be more
  restrictive on the set of processes and names we admit if we are to
  support no-cloning.]
\end{remark}

\subsubsection{Bisimulation}

The computational dynamics gives rise to another kind of equivalence,
the equivalence of computational behavior. As previously mentioned
this is typically captured \emph{via} some form of bisimulation.

% The notion we use in this paper is weak barbed bisimulation
% \cite{milner91polyadicpi}.

The notion we use in this paper is derived from weak barbed
bisimulation \cite{milner91polyadicpi}. 

\begin{definition}
An \emph{observation relation}, $\downarrow_{\mathcal N}$, over a set
of names, $\mathcal N$, is the smallest relation satisfying the rules
below.

\infrule[Out-barb]{y \in {\mathcal N}, \; x \nameeq y}
		  {\outputp{x}{v} \downarrow_{\mathcal N} x}
\infrule[Par-barb]{\mbox{$P\downarrow_{\mathcal N} x$ or $Q\downarrow_{\mathcal N} x$}}
		  {\binpar{P}{Q} \downarrow_{\mathcal N} x}

We write $P \Downarrow_{\mathcal N} x$ if there is $Q$ such that 
$P \wred Q$ and $Q \downarrow_{\mathcal N} x$.
\end{definition}

\begin{definition}
%\label{def.bbisim}
An  ${\mathcal N}$-\emph{barbed bisimulation} over a set of names, ${\mathcal N}$, is a symmetric binary relation 
${\mathcal S}_{\mathcal N}$ between agents such that $P\rel{S}_{\mathcal N}Q$ implies:
\begin{enumerate}
\item If $P \red P'$ then $Q \wred Q'$ and $P'\rel{S}_{\mathcal N} Q'$.
\item If $P\downarrow_{\mathcal N} x$, then $Q\Downarrow_{\mathcal N} x$.
\end{enumerate}
$P$ is ${\mathcal N}$-barbed bisimilar to $Q$, written
$P \wbbisim_{\mathcal N} Q$, if $P \rel{S}_{\mathcal N} Q$ for some ${\mathcal N}$-barbed bisimulation ${\mathcal S}_{\mathcal N}$.
\end{definition}

$\mathcal{R} \subseteq \pi \times \pi$

$P \mathcal{R} Q => \forall P'. P \red P' \Rightarrow \exists Q'. Q \red Q', P' \mathcal{R} Q'$

$P \vdash x \Rightarrow Q \vdash x$

\begin{mathpar}
  \inferrule*[lab=Out-barb]{x \nameeq y}{{y}!\langle{Q}\rangle \vdash x}
  \and
  \inferrule*[lab=Par-barb]{\mbox{$P\vdash x$ or $Q\vdash x$}}{\binpar{P}{Q} \vdash x}
\end{mathpar}

\subsubsection{Contexts}

One of the principle advantages of computational calculi like the
$\pi$-calculus is a well-defined notion of context,
contextual-equivalence and a correlation between
contextual-equivalence and notions of bisimulation. The notion of
context allows the decomposition of a process into (sub-)process and
its syntactic environment, its context. Thus, a context may be
thought of as a process with a ``hole'' (written $\Box$) in it. The
application of a context $M$ to a process $P$, written $M[P]$, is
tantamount to filling the hole in $M$ with $P$. In this paper we do
not need the full weight of this theory, but do make use of the notion
of context in the proof the main theorem. 

\begin{mathpar}
  \inferrule* [lab=summation] {} {{M_{M},M_{N}} \bc \Box \;|\; x.M_{A} \;|\; M_{M}+M_{N}}
  \and
  \inferrule* [lab=agent] {} {{M_{A}} \bc (\vec{x})M_{P} \;| \; \clift{P_0,\ldots,M_{P},\ldots,P_N}}
  \and \\
  \inferrule* [lab=process] {} {{M_{P}} \bc M_{N} \;| \;P|M_{P} }
\end{mathpar} 

\begin{mathpar}
  \inferrule* [lab=sychronization] {} {M_{N} \bc \Box \;|\; x?M_{F} \;|\; x!M_{C}}
  \and
  \inferrule* [lab=abstraction] {} {{M_{F}} \bc (x)M_{P} }
  \and
  \inferrule* [lab=concretion] {} {{M_{C}} \bc \langle M_{P} \rangle }
  \and \\
  \inferrule* [lab=process] {} {{M_{P}} \bc M_{N} \;| \;P|M_{P} }
\end{mathpar}

\begin{definition}[contextual application] Given a context $M$, and
  process $P$, we define the \emph{contextual application}, $M[P] :=
  M\{P/\Box\}$. That is, the contextual application of M to P is the
  substitution of $P$ for $\Box$ in $M$.
\end{definition}

$\meaningof{-} : L \to \mathcal{P}(\pi)$

\begin{mathpar}
  \inferrule* [lab=collection] {} {\meaningof{true} = \pi, \and \meaningof{~E} = \pi \setminus \meaningof{E}, \and \meaningof{E_{1} \& E_{2}} = \meaningof{E_{1}} \cap \meaningof{E_{2}}}
\end{mathpar}

\begin{mathpar}
  \inferrule* [lab=structure] {} {\meaningof{0} = \{ P \in \pi | P \equiv 0 \}, \and \\ \meaningof{E_1 | E_2} = \{ P \in \pi | P \equiv P_{1} | P_{2}, P_{1} \in \meaningof{E_{1}}, P_{2} \in \meaningof{E_2}\} }
\end{mathpar}

\begin{mathpar}
 \inferrule* [lab=behavior] {} {\meaningof{\langle a?b \rangle E} = \{ P \in \pi | P \equiv Q | u?(y)P', \\ \and \\\\ \and \\ \;\;\; u \in \meaningof{a}, \forall z.P'\{z/y\} \in \meaningof{E\{z/b\}}\}, \and \\ \meaningof{a!E} = \{ P \in \pi | P \equiv Q | x!\langle P' \rangle, x \in \meaningof{a} P' \in \meaningof{E}\} }
\end{mathpar}

\begin{mathpar}
 \inferrule* [lab=nominal] {} {\meaningof{\quotep{E}} = \{ \quotep{P} \in \quotep{\pi} | P \in \meaningof{E} \}, \and \meaningof{\quotep{P}} = \{ \quotep{Q} \in \quotep{\pi} | P \equiv Q \} \and \\ \meaningof{@\quotep{E}} = \{ P \in \pi | P \equiv @x, x \in \meaningof{E} \}}
\end{mathpar}

\begin{eqnarray*}
  \\
  \meaningof{-} : TS \to ST
\end{eqnarray*}

\begin{eqnarray*}
  \\
  L : TS \to ST
\end{eqnarray*}

\begin{eqnarray*}
  \\
  P \models E \iff P \in \meaningof{E}
\end{eqnarray*}

\begin{eqnarray*}
  P \approx_{L} Q \iff \forall E \in L. P \models E \iff Q \models E
\end{eqnarray*}

\begin{eqnarray*}
  P \approx_{K} Q
\end{eqnarray*}

\begin{eqnarray*}
  P \approx Q
\end{eqnarray*}

$\approx_{K} = \approx = \approx_{L}$

\subsubsection{Contextual duality}

Note that contexts extend the quotation operation to a family of
operations from processes to names. Given a context, $M$, we can
define a \emph{nominal context}, $\quotep{M}$ by $\quotep{M}[P] :=
\quotep{M[P]}$. To foreshadow what is to come we observe that these
operations enjoy a duality with processes very much like the duality
between vectors and maps from vectors to scalars.

Further, because the calculus is essentially higher-order, we have a
correspondence between contexts and processes. More specifically,
given a name $x$ and a context $M$ we can construct $M^{*}_{x}$ such
that 

\begin{mathpar}
  M^{*}_{x} | \lift{x}{P} \red M[P]
\end{mathpar}

namely,

\begin{mathpar}
  M^{*}_{x} := x?(u).M[\dropn{u}]
\end{mathpar}

The dependence of $M^{*}_{x}$ on a name makes it an abstraction, 

\begin{mathpar}
  M^{*} := (x)x?(u).M[\dropn{u}]
\end{mathpar}

\subsection{Additional notation}

It will sometimes be convenient to denote the process a name
quotes. We already have the notation $x = \quotep{P}$, but it will be
convenient to introduce an alternate notation, $\procn{x}$, when we
want to emphasize the connection to the use of the name. Note that, by
virtue of name equivalence, $\quotep{\procn{x}} \nameeq x$; so, the
notation is consistent with previous definitions.

Further, because names have structure it is possible to effect
substitutions on the basis of that structure. This means we need to
upgrade our notation for substitutions, which we accomplish by
adapting comprehension notation. Thus,

\begin{mathpar}
  P\{ y / x : x \in S \}
\end{mathpar}

is interpreted to mean the process derived from P by replacing (in a
capture-avoiding manner) each occurrence of $x$ in $S$ by $y$. For example,

\begin{mathpar}
  P\{ \quotep{\procn{x}|\procn{x}} / x : x \in \freenames{P} \}
\end{mathpar}

will replace each (occurrence) of a free name $x$ in $P$ by
$\quotep{\procn{x}|\procn{x}}$.

Also, we will avail ourselves of the notation $x^{L}$ and $x^{R}$ to
denote injections of a name into disjoint copies of the name
space. There are numerous ways to accomplish this. One example can be
found in \cite{MeredithR05}. This notation overloads to vectors of
names: $\vec{x}^{\pi} := (x_{i}^{\pi} \; : \; 0 \leq i < |\vec{x}| )$ where $\pi \in \{L,R\}$.

We also use $P^{\Box} := P|\Box$.

In \cite{MeredithR05} an interpretation of the new operator is
given. It turns out that there are several possible interpretations
all enjoying the requisite algebraic properties of the operator (see
\cite{milner91polyadicpi}). We will therefore make liberal use of
$(\nu\; \vec{x})P$.

% subsection the_syntax_and_semantics_of_the_notation_system (end)   

\input{qm2pi.qmops} 

\input{qm2pi.sterngerlach} 

\input{qm2pi.metric} 

% section concurrent_process_calculi (end)

%\input{qm2pi.proofsketch}

% section proof sketch (end)

%\input{qm2pi.slviaknots} 

% section spatial logic via knots (end)

\input{qm2pi.conclusion}

% section conclusion (end)

%\input{qm2pi.dtcodes} 

% section wiring algorithm (end)

\input{qm2pi.ack} 

% section acknowledgments (end)

\newpage


\bibliographystyle{plain}   
\bibliography{../../biblios/main.bib}

\input{qm2pi.rhodetails}

\end{document}

 

%\documentclass[12pt]{llncs}
%\documentclass{jktr}

\usepackage[pdftex]{hyperref}                   
\usepackage {listings}
\usepackage {mathpartir}
\usepackage{bcprules}
%\usepackage{listings}
                       
\usepackage{graphicx} 
%\usepackage[margins=2.5cm,nohead,nofoot]{geometry}
%\usepackage{geometry}
\usepackage{amsfonts}
\usepackage{amstext}
\usepackage{latexsym}
\usepackage{amssymb}
\usepackage{color}


%\include{myPreamble}
\include{qm2pi.local} 

%\ifpdf
%\usepackage[pdftex]{graphicx}
%\else
%\usepackage{graphicx}
%\fi

 % \ifpdf
%  \usepackage{pdfsync}
%  \if


%\title{Brief Article}
%\author{David F. Snyder}
%\author{L.G. Meredith}

%\address{Dept. of Math., Texas State University--San Marcos, San Marcos, TX 78666}
       
\pagestyle{empty}


\begin{document}

\lstset{language=[Objective]Caml,frame=shadowbox}

\input{qm2pi.front}

% section front matter (end)

\input{qm2pi.intro} 
 
% section introduction (end)

% \input{qm2pi.knotations} 

% section notation (end)

\input{qm2pi.process.calculi} 

% section concurrent_process_calculi_and_spatial_logics_ (end)
    
%\input{qm2pi.knots2pi} 

%\input{qm2pi.trefoil} 

%\input{qm2pi.mainthm} 

% subsection basic_interpretation (end)

%\input{qm2pi.rho.presentation} 
\subsection{The syntax and semantics of the notation system}\label{sub:the_syntax_and_semantics_of_the_notation_system} % (fold)

We now summarize a technical presentation of the calculus that
embodies our theory of dynamics. The typical presentation of such a
calculus follows the style of giving generators and relations on
them. The grammar, below, describing term constructors, freely
generates the set of processes, $\Proc$. This set is then quotiented
by a relation known as structural congruence and it is over this set
that the notion of dynamics is expressed. This presentation is
essentially that of \cite{MeredithR05} with the addition of
polyadicity and summation. For readability we have relegated some of
the technical subtleties to an appendix.

\subsubsection{Process grammar}\label{subsub:process_grammar}

\begin{mathpar}
  \inferrule* [lab=synchronization] {} {{M} \bc \pzero \;|\; x?F \;|\; x!C }
  \and
  \inferrule* [lab=abstraction] {} {{F} \bc (x)P}
  \and
  \inferrule* [lab=concretion] {} {{C} \bc \langle Q \rangle}
  \and
  \inferrule* [lab=process] {} {{P,Q} \bc M \;| \;P|Q \;|\; @{x}}
  \and
  \inferrule* [lab=name] {} {{x} \bc \quotep{P}}
\end{mathpar} 

Note that $\vec{x}$ (resp. $\vec{P}$) denotes a vector of names
(resp. processes) of length $|\vec{x}|$ (resp. $|\vec{P}|$). We adopt
the following useful abbreviations.

\begin{mathpar}
   x?(\vec{y}).P := x.(\vec{y})P \and  x\clift{\vec{P}} := x.\clift{\vec{P}}
   \and x!(y) := \lift{x}{\dropn{y}}
   \and \Pi_{i=0}^{n-1}P_i := P_0 | \ldots | P_{n-1}
\end{mathpar}

\subsubsection{Structural congruence}

\paragraph{Free and bound names and alpha-equivalence.} At the
core of structural equivalence is alpha-equivalence which identifies
process that are the same up to a change of variable. Formally, we
recognize the distinction between free and bound names. The free names
of a process, $\freenames{P}$, may be calculated recursively as
follows:

\begin{mathpar}
\freenames{\pzero} := \emptyset
  \and \\
  \freenames{x?(y).P} := \{ x \} \cup (\freenames{P} \setminus \{ y \})
  \and 
  \freenames{x!\langle P \rangle} := \{ x \} \cup \{ P \} 
  \and \\
  \freenames{P|Q} := \freenames{P} \cup \freenames{Q}
  \and \\
  \freenames{@{x}} := \{ x \}
\end{mathpar}

$\pi$
$\quotep{\pi}$

$\freenames{-} : \pi \to \mathcal{P}(\quotep{\pi})$

\begin{eqnarray*}
  \freenames{\pzero} & := & \emptyset \\
  \freenames{x?(y).P} & := & \{ x \} \cup (\freenames{P} \setminus \{ y \}) \\
  \freenames{x!\langle P \rangle} & := & \{ x \} \cup \{ P \} \\
  \freenames{P|Q} & := & \freenames{P} \cup \freenames{Q} \\
  \freenames{\dropn{x}} & := & \{ x \}
\end{eqnarray*}

The bound names of a process, $\boundnames{P}$, are those names occurring in $P$
that are not free. For example, in $x?(y).0$, the name $x$ is free, while $y$ is bound.

\begin{mathpar}
  \inferrule* [lab=monoidal-laws] {} { P|Q \equiv Q|P \and P|0 \equiv P \and P|(Q|R) \equiv (P|Q)|R }
\end{mathpar}

\begin{mathpar}
  \inferrule* [lab=alpha-equivalence] {} { (x)P \equiv (y)P\{y/x\} \and y \not\in \freenames{P} }
\end{mathpar}

\begin{definition}
Then two processes, $P,Q$, are alpha-equivalent if $P = Q\{\vec{y}/\vec{x}\}$ for
some $\vec{x} \in \boundnames{Q},\vec{y} \in \boundnames{P}$, where $Q\{\vec{y}/\vec{x}\}$
denotes the capture-avoiding substitution of $\vec{y}$ for $\vec{x}$ in $Q$.
\end{definition}

\begin{definition}
  The {\em structural congruence} \cite{SangiorgiWalker} , $\equiv$,
  between processes is the least congruence containing
  alpha-equivalence, satisfying the abelian monoid laws
  (associativity, commutativity and $\pzero$ as identity) for parallel
  composition $|$ and for summation $+$.
\end{definition}

\subsection{Name equivalence}

We take name equivalence, written $\nameeq$, to be the smallest
equivalence relation generated by the following rules.

\begin{mathpar}
\inferrule*[lab=Quote-drop]
{ }
{ \quotep{@{x}} \nameeq x }

\inferrule*[lab=Struct-equiv]
{ P \scong Q }
{ \quotep{P} \nameeq \quotep{Q} }
\end{mathpar}

The astute reader will have noticed that the mutual recursion of names
and processes imposes a mutual recursion on alpha-equivalence and
structural equivalence via name-equivalence. Fortunately, all of this
works out pleasantly and we may calculate in the natural way, free of
concern. The reader interested in the details is referred to the
appendix \ref{appendix:rho_details}.

\subsection{Substitution}

We use $\Proc$ for the set of processes, $\QProc$ for the set of
names, and $\id{\{}\vec{y} / \vec{x} \id{\}}$ to denote partial maps,
$s : \QProc \rightarrow \QProc$. A map, $s$ lifts, uniquely, to a map
on process terms, $\widehat{s} : \Proc \rightarrow \Proc$ by the
following equations.

\begin{mathpar}
  (0) \psubstp{Q}{P} := 0 \\
  (R \juxtap S) \psubstp{Q}{P}
  :=    
  (R)\psubstp{Q}{P} \juxtap (S) \psubstp{Q}{P} \\
  (x?(y).R) \psubstp{Q}{P}    
  :=    
  (x)\substp{Q}{P} (z)\concat( (R \psubstn{z}{y}) \psubstp{Q}{P} ) \\
  (\lift{x}{R}) \psubstp{Q}{P}  
  :=
  \lift{(x)\substp{Q}{P}}{ R \psubstp{Q}{P} } \\
%   (\dropn{x})  \psubstp{Q}{P}       
%   := 
%   \left\{ 
%     \begin{array}{ccc} 
%       \dropn{\quotep{Q}} & & x \nameeq \quotep{P} \\
%       \dropn{x} & & otherwise \\
%     \end{array}
%   \right. 
  (\dropn{x})  \psubstp{Q}{P}       
  := 
  \left\{ 
    \begin{array}{ccc} 
      Q & & x \nameeq \quotep{P} \\
      \dropn{x} & & otherwise \\
    \end{array}
  \right.
\end{mathpar}
 

where

\begin{eqnarray}
  (x)\id{\{} \lpquote Q \rpquote / \lpquote P \rpquote \id{\}}            = 
  \left\{ 
    \begin{array}{ccc}
      \lpquote Q \rpquote & & x \nameeq \lpquote P \rpquote \\
      x & & otherwise \\
    \end{array}
  \right. \nonumber
\end{eqnarray}

and $z$ is chosen distinct from $\quotep{P}$, $\quotep{Q}$, the free
names in $Q$, and all the names in $R$. Our $\alpha$-equivalence will
be built in the standard way from this substitution.

\begin{remark}\label{rem:no_self_referential_names}
  One consequence of these definitions is that $\forall P. \quotep{P}
  \not\in \freenames{P}$.
\end{remark}

\subsection{ Dynamic quote: an example }

Anticipating something of what's to come, consider applying the
substitution, $\widehat{\id{\{}u / z \id{\}}}$, to the following pair
of processes, $\lift{w}{y!(z)}$ and $w[ \lpquote y!(z) \rpquote ]$.

\begin{eqnarray}
	\lift{w}{y!(z)}\widehat{\id{\{}u / z \id{\}}}
		& = &
		\lift{w}{y!(u)} \nonumber\\
	w[ \lpquote y!(z) \rpquote ] \widehat{ \id{\{}u / z \id{\}} }
		& = &
		w[ \lpquote y!(z) \rpquote ] \nonumber
\end{eqnarray}

Because the body of the process between quotes is impervious to
substitution, we get radically different answers. In fact, by
examining the first process in an input context,
e.g. $x?(z).\lift{w}{y!(z)}$, we see that the process under the lift
operator may be shaped by prefixed inputs binding a name inside it. In
this sense, the lift operator will be seen as a way to dynamically
construct processes before reifying them as names.

Finally equipped with these standard features we can present the
dynamics of the calculus.

\subsubsection{Operational semantics} 

Finally, we introduce the computational dynamics. What marks these
algebras as distinct from other more traditionally studied algebraic
structures, e.g. vector spaces or polynomial rings, is the manner in
which dynamics is captured. In traditional structures, dynamics is typically
expressed through morphisms between such structures, as in linear maps
between vector spaces or morphisms between rings. In algebras
associated with the semantics of computation, the dynamics is
expressed as part of the algebraic structure itself, through a
reduction reduction relation typically denoted by $\red$. Below, we
give a recursive presentation of this relation for the calculus used
in the encoding.

$\red \subseteq \pi \times \pi$
$\red : \pi \to \mathcal{P}(\pi)$

\begin{mathpar}
  \inferrule* [lab=Comm] { \textsf{match}( x_{src}, x_{trgt} ) } { x_{trgt}?(y)P \; | \; x_{src}!\langle {Q} \rangle \red P\{\quotep{Q}/y}\} }
  \and \\
  \inferrule* [lab=Par] {{P} \red {P}'} {{{P} | {Q}} \red {{P}' | {Q}}}
  \and
  \inferrule* [lab=Equiv]{{{P} \scong {P}'} \andalso {{P}' \red {Q}'} \andalso {{Q}' \scong {Q}}}{{P} \red {Q}}
\end{mathpar}

\begin{eqnarray*}
  match_{\equiv} (\quotep{P},\quotep{Q}) & := & P \equiv Q \\
  match_{\dagger}(\quotep{P},\quotep{Q}) & := & \forall R. P|Q \red^{*} R => R \red^{*} 0 \\
  match_{K}(\quotep{P},\quotep{Q}) & := & K \mbox{ for some context } K
\end{eqnarray*}

$u?(x)P | u!\langle Q \rangle \red P\{\quotep{Q}/x\}$

%We write $\wred$ for $\red^*$, and $P\red$ if $\exists Q $ such that $ P \red Q$.
We write $P\red$ if $\exists Q $ such that $ P \red Q$ and $P\not\red$, otherwise.

\section{Replication}

As mentioned before, it is known that replication (and hence
recursion) can be implemented in a higher-order process algebra
\cite{SangiorgiWalker}. As our first example of calculation with the
machinery thus far presented we give the construction explicitly in
the {\rhoc}.

\begin{eqnarray}
	D_{x} & := & \prefix{x}{y}{(\binpar{\outputp{x}{y}}{@{y}})} \nonumber\\
	\bangp_{x}{P} & := & \binpar{{x}!\langle{\binpar{D_{x}}{P}}\rangle}{D_{x}} \nonumber
\end{eqnarray}

\begin{eqnarray}
	\bangp_{x}{P} & & \nonumber\\
	=
	& {x}!\langle{(\prefix{x}{y}{(\outputp{x}{y} | @{y})) | P}}\rangle 
	      | \prefix{x}{y}{(\outputp{x}{y} | @{y})} & \nonumber\\
	\red
	& (\outputp{x}{y} | @{y})\substn{\quotep{(\prefix{x}{y}{(@{y} | \outputp{x}{y})) | P}}}{y} & \nonumber\\
	=
	& \outputp{x}{\quotep{(\prefix{x}{y}{(\outputp{x}{y} | @{y})) | P}}}
	  | {(\prefix{x}{y}{(\outputp{x}{y} | @{y})) | P}} & \nonumber\\
	\red
	& \ldots & \nonumber\\
	\red^*
	& P | P | \ldots & \nonumber
\end{eqnarray}

Of course, this encoding, as an implementation, runs away, unfolding
$\bangp{P}$ eagerly. A lazier and more implementable replication
operator, restricted to input-guarded processes, may be obtained as follows.

\begin{eqnarray}
\bangp{\prefix{u}{v}{P}} 
	:= 
	\binpar{\lift{x}{\prefix{u}{v}{(\binpar{D(x)}{P})}}}{D(x)} \nonumber
\end{eqnarray}

\begin{remark}
  Note that the lazier definition still does not deal with summation
  or mixed summation (i.e. sums over input and output). The reader is
  invited to construct definitions of replication that deal with these
  features. 

  Further, the definitions are parameterized in a name, $x$. Can you,
  gentle reader, make a definition that eliminates this parameter and
  guarantees no accidental interaction between the replication
  machinery and the process being replicated -- i.e. no accidental
  sharing of names used by the process to get its work done and the
  name(s) used by the replication to effect copying. This latter
  revision of the definition of replication is crucial to obtaining
  the expected identity $!!P \sim !P$.
\end{remark}

\begin{remark}\label{rem:paradoxical_combinator}
  The reader familiar with the lambda calculus will have noticed the
  similarity between $D$ and the paradoxical combinator.

  [Ed. note: the existence of this seems to suggest we have to be more
  restrictive on the set of processes and names we admit if we are to
  support no-cloning.]
\end{remark}

\subsubsection{Bisimulation}

The computational dynamics gives rise to another kind of equivalence,
the equivalence of computational behavior. As previously mentioned
this is typically captured \emph{via} some form of bisimulation.

% The notion we use in this paper is weak barbed bisimulation
% \cite{milner91polyadicpi}.

The notion we use in this paper is derived from weak barbed
bisimulation \cite{milner91polyadicpi}. 

\begin{definition}
An \emph{observation relation}, $\downarrow_{\mathcal N}$, over a set
of names, $\mathcal N$, is the smallest relation satisfying the rules
below.

\infrule[Out-barb]{y \in {\mathcal N}, \; x \nameeq y}
		  {\outputp{x}{v} \downarrow_{\mathcal N} x}
\infrule[Par-barb]{\mbox{$P\downarrow_{\mathcal N} x$ or $Q\downarrow_{\mathcal N} x$}}
		  {\binpar{P}{Q} \downarrow_{\mathcal N} x}

We write $P \Downarrow_{\mathcal N} x$ if there is $Q$ such that 
$P \wred Q$ and $Q \downarrow_{\mathcal N} x$.
\end{definition}

\begin{definition}
%\label{def.bbisim}
An  ${\mathcal N}$-\emph{barbed bisimulation} over a set of names, ${\mathcal N}$, is a symmetric binary relation 
${\mathcal S}_{\mathcal N}$ between agents such that $P\rel{S}_{\mathcal N}Q$ implies:
\begin{enumerate}
\item If $P \red P'$ then $Q \wred Q'$ and $P'\rel{S}_{\mathcal N} Q'$.
\item If $P\downarrow_{\mathcal N} x$, then $Q\Downarrow_{\mathcal N} x$.
\end{enumerate}
$P$ is ${\mathcal N}$-barbed bisimilar to $Q$, written
$P \wbbisim_{\mathcal N} Q$, if $P \rel{S}_{\mathcal N} Q$ for some ${\mathcal N}$-barbed bisimulation ${\mathcal S}_{\mathcal N}$.
\end{definition}

$\mathcal{R} \subseteq \pi \times \pi$

$P \mathcal{R} Q => \forall P'. P \red P' \Rightarrow \exists Q'. Q \red Q', P' \mathcal{R} Q'$

$P \vdash x \Rightarrow Q \vdash x$

\begin{mathpar}
  \inferrule*[lab=Out-barb]{x \nameeq y}{{y}!\langle{Q}\rangle \vdash x}
  \and
  \inferrule*[lab=Par-barb]{\mbox{$P\vdash x$ or $Q\vdash x$}}{\binpar{P}{Q} \vdash x}
\end{mathpar}

\subsubsection{Contexts}

One of the principle advantages of computational calculi like the
$\pi$-calculus is a well-defined notion of context,
contextual-equivalence and a correlation between
contextual-equivalence and notions of bisimulation. The notion of
context allows the decomposition of a process into (sub-)process and
its syntactic environment, its context. Thus, a context may be
thought of as a process with a ``hole'' (written $\Box$) in it. The
application of a context $M$ to a process $P$, written $M[P]$, is
tantamount to filling the hole in $M$ with $P$. In this paper we do
not need the full weight of this theory, but do make use of the notion
of context in the proof the main theorem. 

\begin{mathpar}
  \inferrule* [lab=summation] {} {{M_{M},M_{N}} \bc \Box \;|\; x.M_{A} \;|\; M_{M}+M_{N}}
  \and
  \inferrule* [lab=agent] {} {{M_{A}} \bc (\vec{x})M_{P} \;| \; \clift{P_0,\ldots,M_{P},\ldots,P_N}}
  \and \\
  \inferrule* [lab=process] {} {{M_{P}} \bc M_{N} \;| \;P|M_{P} }
\end{mathpar} 

\begin{mathpar}
  \inferrule* [lab=sychronization] {} {M_{N} \bc \Box \;|\; x?M_{F} \;|\; x!M_{C}}
  \and
  \inferrule* [lab=abstraction] {} {{M_{F}} \bc (x)M_{P} }
  \and
  \inferrule* [lab=concretion] {} {{M_{C}} \bc \langle M_{P} \rangle }
  \and \\
  \inferrule* [lab=process] {} {{M_{P}} \bc M_{N} \;| \;P|M_{P} }
\end{mathpar}

\begin{definition}[contextual application] Given a context $M$, and
  process $P$, we define the \emph{contextual application}, $M[P] :=
  M\{P/\Box\}$. That is, the contextual application of M to P is the
  substitution of $P$ for $\Box$ in $M$.
\end{definition}

$\meaningof{-} : L \to \mathcal{P}(\pi)$

\begin{mathpar}
  \inferrule* [lab=collection] {} {\meaningof{true} = \pi, \and \meaningof{~E} = \pi \setminus \meaningof{E}, \and \meaningof{E_{1} \& E_{2}} = \meaningof{E_{1}} \cap \meaningof{E_{2}}}
\end{mathpar}

\begin{mathpar}
  \inferrule* [lab=structure] {} {\meaningof{0} = \{ P \in \pi | P \equiv 0 \}, \and \\ \meaningof{E_1 | E_2} = \{ P \in \pi | P \equiv P_{1} | P_{2}, P_{1} \in \meaningof{E_{1}}, P_{2} \in \meaningof{E_2}\} }
\end{mathpar}

\begin{mathpar}
 \inferrule* [lab=behavior] {} {\meaningof{\langle a?b \rangle E} = \{ P \in \pi | P \equiv Q | u?(y)P', \\ \and \\\\ \and \\ \;\;\; u \in \meaningof{a}, \forall z.P'\{z/y\} \in \meaningof{E\{z/b\}}\}, \and \\ \meaningof{a!E} = \{ P \in \pi | P \equiv Q | x!\langle P' \rangle, x \in \meaningof{a} P' \in \meaningof{E}\} }
\end{mathpar}

\begin{mathpar}
 \inferrule* [lab=nominal] {} {\meaningof{\quotep{E}} = \{ \quotep{P} \in \quotep{\pi} | P \in \meaningof{E} \}, \and \meaningof{\quotep{P}} = \{ \quotep{Q} \in \quotep{\pi} | P \equiv Q \} \and \\ \meaningof{@\quotep{E}} = \{ P \in \pi | P \equiv @x, x \in \meaningof{E} \}}
\end{mathpar}

\begin{eqnarray*}
  \\
  \meaningof{-} : TS \to ST
\end{eqnarray*}

\begin{eqnarray*}
  \\
  L : TS \to ST
\end{eqnarray*}

\begin{eqnarray*}
  \\
  P \models E \iff P \in \meaningof{E}
\end{eqnarray*}

\begin{eqnarray*}
  P \approx_{L} Q \iff \forall E \in L. P \models E \iff Q \models E
\end{eqnarray*}

\begin{eqnarray*}
  P \approx_{K} Q
\end{eqnarray*}

\begin{eqnarray*}
  P \approx Q
\end{eqnarray*}

$\approx_{K} = \approx = \approx_{L}$

\subsubsection{Contextual duality}

Note that contexts extend the quotation operation to a family of
operations from processes to names. Given a context, $M$, we can
define a \emph{nominal context}, $\quotep{M}$ by $\quotep{M}[P] :=
\quotep{M[P]}$. To foreshadow what is to come we observe that these
operations enjoy a duality with processes very much like the duality
between vectors and maps from vectors to scalars.

Further, because the calculus is essentially higher-order, we have a
correspondence between contexts and processes. More specifically,
given a name $x$ and a context $M$ we can construct $M^{*}_{x}$ such
that 

\begin{mathpar}
  M^{*}_{x} | \lift{x}{P} \red M[P]
\end{mathpar}

namely,

\begin{mathpar}
  M^{*}_{x} := x?(u).M[\dropn{u}]
\end{mathpar}

The dependence of $M^{*}_{x}$ on a name makes it an abstraction, 

\begin{mathpar}
  M^{*} := (x)x?(u).M[\dropn{u}]
\end{mathpar}

\subsection{Additional notation}

It will sometimes be convenient to denote the process a name
quotes. We already have the notation $x = \quotep{P}$, but it will be
convenient to introduce an alternate notation, $\procn{x}$, when we
want to emphasize the connection to the use of the name. Note that, by
virtue of name equivalence, $\quotep{\procn{x}} \nameeq x$; so, the
notation is consistent with previous definitions.

Further, because names have structure it is possible to effect
substitutions on the basis of that structure. This means we need to
upgrade our notation for substitutions, which we accomplish by
adapting comprehension notation. Thus,

\begin{mathpar}
  P\{ y / x : x \in S \}
\end{mathpar}

is interpreted to mean the process derived from P by replacing (in a
capture-avoiding manner) each occurrence of $x$ in $S$ by $y$. For example,

\begin{mathpar}
  P\{ \quotep{\procn{x}|\procn{x}} / x : x \in \freenames{P} \}
\end{mathpar}

will replace each (occurrence) of a free name $x$ in $P$ by
$\quotep{\procn{x}|\procn{x}}$.

Also, we will avail ourselves of the notation $x^{L}$ and $x^{R}$ to
denote injections of a name into disjoint copies of the name
space. There are numerous ways to accomplish this. One example can be
found in \cite{MeredithR05}. This notation overloads to vectors of
names: $\vec{x}^{\pi} := (x_{i}^{\pi} \; : \; 0 \leq i < |\vec{x}| )$ where $\pi \in \{L,R\}$.

We also use $P^{\Box} := P|\Box$.

In \cite{MeredithR05} an interpretation of the new operator is
given. It turns out that there are several possible interpretations
all enjoying the requisite algebraic properties of the operator (see
\cite{milner91polyadicpi}). We will therefore make liberal use of
$(\nu\; \vec{x})P$.

% subsection the_syntax_and_semantics_of_the_notation_system (end)   

\input{qm2pi.qmops} 

\input{qm2pi.sterngerlach} 

\input{qm2pi.metric} 

% section concurrent_process_calculi (end)

%\input{qm2pi.proofsketch}

% section proof sketch (end)

%\input{qm2pi.slviaknots} 

% section spatial logic via knots (end)

\input{qm2pi.conclusion}

% section conclusion (end)

%\input{qm2pi.dtcodes} 

% section wiring algorithm (end)

\input{qm2pi.ack} 

% section acknowledgments (end)

\newpage


\bibliographystyle{plain}   
\bibliography{../../biblios/main.bib}

\input{qm2pi.rhodetails}

\end{document}

 

% subsection basic_interpretation (end)

%\input{qm2pi.rho.presentation} 
\subsection{The syntax and semantics of the notation system}\label{sub:the_syntax_and_semantics_of_the_notation_system} % (fold)

We now summarize a technical presentation of the calculus that
embodies our theory of dynamics. The typical presentation of such a
calculus follows the style of giving generators and relations on
them. The grammar, below, describing term constructors, freely
generates the set of processes, $\Proc$. This set is then quotiented
by a relation known as structural congruence and it is over this set
that the notion of dynamics is expressed. This presentation is
essentially that of \cite{MeredithR05} with the addition of
polyadicity and summation. For readability we have relegated some of
the technical subtleties to an appendix.

\subsubsection{Process grammar}\label{subsub:process_grammar}

\begin{mathpar}
  \inferrule* [lab=synchronization] {} {{M} \bc \pzero \;|\; x?F \;|\; x!C }
  \and
  \inferrule* [lab=abstraction] {} {{F} \bc (x)P}
  \and
  \inferrule* [lab=concretion] {} {{C} \bc \langle Q \rangle}
  \and
  \inferrule* [lab=process] {} {{P,Q} \bc M \;| \;P|Q \;|\; @{x}}
  \and
  \inferrule* [lab=name] {} {{x} \bc \quotep{P}}
\end{mathpar} 

Note that $\vec{x}$ (resp. $\vec{P}$) denotes a vector of names
(resp. processes) of length $|\vec{x}|$ (resp. $|\vec{P}|$). We adopt
the following useful abbreviations.

\begin{mathpar}
   x?(\vec{y}).P := x.(\vec{y})P \and  x\clift{\vec{P}} := x.\clift{\vec{P}}
   \and x!(y) := \lift{x}{\dropn{y}}
   \and \Pi_{i=0}^{n-1}P_i := P_0 | \ldots | P_{n-1}
\end{mathpar}

\subsubsection{Structural congruence}

\paragraph{Free and bound names and alpha-equivalence.} At the
core of structural equivalence is alpha-equivalence which identifies
process that are the same up to a change of variable. Formally, we
recognize the distinction between free and bound names. The free names
of a process, $\freenames{P}$, may be calculated recursively as
follows:

\begin{mathpar}
\freenames{\pzero} := \emptyset
  \and \\
  \freenames{x?(y).P} := \{ x \} \cup (\freenames{P} \setminus \{ y \})
  \and 
  \freenames{x!\langle P \rangle} := \{ x \} \cup \{ P \} 
  \and \\
  \freenames{P|Q} := \freenames{P} \cup \freenames{Q}
  \and \\
  \freenames{@{x}} := \{ x \}
\end{mathpar}

$\pi$
$\quotep{\pi}$

$\freenames{-} : \pi \to \mathcal{P}(\quotep{\pi})$

\begin{eqnarray*}
  \freenames{\pzero} & := & \emptyset \\
  \freenames{x?(y).P} & := & \{ x \} \cup (\freenames{P} \setminus \{ y \}) \\
  \freenames{x!\langle P \rangle} & := & \{ x \} \cup \{ P \} \\
  \freenames{P|Q} & := & \freenames{P} \cup \freenames{Q} \\
  \freenames{\dropn{x}} & := & \{ x \}
\end{eqnarray*}

The bound names of a process, $\boundnames{P}$, are those names occurring in $P$
that are not free. For example, in $x?(y).0$, the name $x$ is free, while $y$ is bound.

\begin{mathpar}
  \inferrule* [lab=monoidal-laws] {} { P|Q \equiv Q|P \and P|0 \equiv P \and P|(Q|R) \equiv (P|Q)|R }
\end{mathpar}

\begin{mathpar}
  \inferrule* [lab=alpha-equivalence] {} { (x)P \equiv (y)P\{y/x\} \and y \not\in \freenames{P} }
\end{mathpar}

\begin{definition}
Then two processes, $P,Q$, are alpha-equivalent if $P = Q\{\vec{y}/\vec{x}\}$ for
some $\vec{x} \in \boundnames{Q},\vec{y} \in \boundnames{P}$, where $Q\{\vec{y}/\vec{x}\}$
denotes the capture-avoiding substitution of $\vec{y}$ for $\vec{x}$ in $Q$.
\end{definition}

\begin{definition}
  The {\em structural congruence} \cite{SangiorgiWalker} , $\equiv$,
  between processes is the least congruence containing
  alpha-equivalence, satisfying the abelian monoid laws
  (associativity, commutativity and $\pzero$ as identity) for parallel
  composition $|$ and for summation $+$.
\end{definition}

\subsection{Name equivalence}

We take name equivalence, written $\nameeq$, to be the smallest
equivalence relation generated by the following rules.

\begin{mathpar}
\inferrule*[lab=Quote-drop]
{ }
{ \quotep{@{x}} \nameeq x }

\inferrule*[lab=Struct-equiv]
{ P \scong Q }
{ \quotep{P} \nameeq \quotep{Q} }
\end{mathpar}

The astute reader will have noticed that the mutual recursion of names
and processes imposes a mutual recursion on alpha-equivalence and
structural equivalence via name-equivalence. Fortunately, all of this
works out pleasantly and we may calculate in the natural way, free of
concern. The reader interested in the details is referred to the
appendix \ref{appendix:rho_details}.

\subsection{Substitution}

We use $\Proc$ for the set of processes, $\QProc$ for the set of
names, and $\id{\{}\vec{y} / \vec{x} \id{\}}$ to denote partial maps,
$s : \QProc \rightarrow \QProc$. A map, $s$ lifts, uniquely, to a map
on process terms, $\widehat{s} : \Proc \rightarrow \Proc$ by the
following equations.

\begin{mathpar}
  (0) \psubstp{Q}{P} := 0 \\
  (R \juxtap S) \psubstp{Q}{P}
  :=    
  (R)\psubstp{Q}{P} \juxtap (S) \psubstp{Q}{P} \\
  (x?(y).R) \psubstp{Q}{P}    
  :=    
  (x)\substp{Q}{P} (z)\concat( (R \psubstn{z}{y}) \psubstp{Q}{P} ) \\
  (\lift{x}{R}) \psubstp{Q}{P}  
  :=
  \lift{(x)\substp{Q}{P}}{ R \psubstp{Q}{P} } \\
%   (\dropn{x})  \psubstp{Q}{P}       
%   := 
%   \left\{ 
%     \begin{array}{ccc} 
%       \dropn{\quotep{Q}} & & x \nameeq \quotep{P} \\
%       \dropn{x} & & otherwise \\
%     \end{array}
%   \right. 
  (\dropn{x})  \psubstp{Q}{P}       
  := 
  \left\{ 
    \begin{array}{ccc} 
      Q & & x \nameeq \quotep{P} \\
      \dropn{x} & & otherwise \\
    \end{array}
  \right.
\end{mathpar}
 

where

\begin{eqnarray}
  (x)\id{\{} \lpquote Q \rpquote / \lpquote P \rpquote \id{\}}            = 
  \left\{ 
    \begin{array}{ccc}
      \lpquote Q \rpquote & & x \nameeq \lpquote P \rpquote \\
      x & & otherwise \\
    \end{array}
  \right. \nonumber
\end{eqnarray}

and $z$ is chosen distinct from $\quotep{P}$, $\quotep{Q}$, the free
names in $Q$, and all the names in $R$. Our $\alpha$-equivalence will
be built in the standard way from this substitution.

\begin{remark}\label{rem:no_self_referential_names}
  One consequence of these definitions is that $\forall P. \quotep{P}
  \not\in \freenames{P}$.
\end{remark}

\subsection{ Dynamic quote: an example }

Anticipating something of what's to come, consider applying the
substitution, $\widehat{\id{\{}u / z \id{\}}}$, to the following pair
of processes, $\lift{w}{y!(z)}$ and $w[ \lpquote y!(z) \rpquote ]$.

\begin{eqnarray}
	\lift{w}{y!(z)}\widehat{\id{\{}u / z \id{\}}}
		& = &
		\lift{w}{y!(u)} \nonumber\\
	w[ \lpquote y!(z) \rpquote ] \widehat{ \id{\{}u / z \id{\}} }
		& = &
		w[ \lpquote y!(z) \rpquote ] \nonumber
\end{eqnarray}

Because the body of the process between quotes is impervious to
substitution, we get radically different answers. In fact, by
examining the first process in an input context,
e.g. $x?(z).\lift{w}{y!(z)}$, we see that the process under the lift
operator may be shaped by prefixed inputs binding a name inside it. In
this sense, the lift operator will be seen as a way to dynamically
construct processes before reifying them as names.

Finally equipped with these standard features we can present the
dynamics of the calculus.

\subsubsection{Operational semantics} 

Finally, we introduce the computational dynamics. What marks these
algebras as distinct from other more traditionally studied algebraic
structures, e.g. vector spaces or polynomial rings, is the manner in
which dynamics is captured. In traditional structures, dynamics is typically
expressed through morphisms between such structures, as in linear maps
between vector spaces or morphisms between rings. In algebras
associated with the semantics of computation, the dynamics is
expressed as part of the algebraic structure itself, through a
reduction reduction relation typically denoted by $\red$. Below, we
give a recursive presentation of this relation for the calculus used
in the encoding.

$\red \subseteq \pi \times \pi$
$\red : \pi \to \mathcal{P}(\pi)$

\begin{mathpar}
  \inferrule* [lab=Comm] { \textsf{match}( x_{src}, x_{trgt} ) } { x_{trgt}?(y)P \; | \; x_{src}!\langle {Q} \rangle \red P\{\quotep{Q}/y}\} }
  \and \\
  \inferrule* [lab=Par] {{P} \red {P}'} {{{P} | {Q}} \red {{P}' | {Q}}}
  \and
  \inferrule* [lab=Equiv]{{{P} \scong {P}'} \andalso {{P}' \red {Q}'} \andalso {{Q}' \scong {Q}}}{{P} \red {Q}}
\end{mathpar}

\begin{eqnarray*}
  match_{\equiv} (\quotep{P},\quotep{Q}) & := & P \equiv Q \\
  match_{\dagger}(\quotep{P},\quotep{Q}) & := & \forall R. P|Q \red^{*} R => R \red^{*} 0 \\
  match_{K}(\quotep{P},\quotep{Q}) & := & K \mbox{ for some context } K
\end{eqnarray*}

$u?(x)P | u!\langle Q \rangle \red P\{\quotep{Q}/x\}$

%We write $\wred$ for $\red^*$, and $P\red$ if $\exists Q $ such that $ P \red Q$.
We write $P\red$ if $\exists Q $ such that $ P \red Q$ and $P\not\red$, otherwise.

\section{Replication}

As mentioned before, it is known that replication (and hence
recursion) can be implemented in a higher-order process algebra
\cite{SangiorgiWalker}. As our first example of calculation with the
machinery thus far presented we give the construction explicitly in
the {\rhoc}.

\begin{eqnarray}
	D_{x} & := & \prefix{x}{y}{(\binpar{\outputp{x}{y}}{@{y}})} \nonumber\\
	\bangp_{x}{P} & := & \binpar{{x}!\langle{\binpar{D_{x}}{P}}\rangle}{D_{x}} \nonumber
\end{eqnarray}

\begin{eqnarray}
	\bangp_{x}{P} & & \nonumber\\
	=
	& {x}!\langle{(\prefix{x}{y}{(\outputp{x}{y} | @{y})) | P}}\rangle 
	      | \prefix{x}{y}{(\outputp{x}{y} | @{y})} & \nonumber\\
	\red
	& (\outputp{x}{y} | @{y})\substn{\quotep{(\prefix{x}{y}{(@{y} | \outputp{x}{y})) | P}}}{y} & \nonumber\\
	=
	& \outputp{x}{\quotep{(\prefix{x}{y}{(\outputp{x}{y} | @{y})) | P}}}
	  | {(\prefix{x}{y}{(\outputp{x}{y} | @{y})) | P}} & \nonumber\\
	\red
	& \ldots & \nonumber\\
	\red^*
	& P | P | \ldots & \nonumber
\end{eqnarray}

Of course, this encoding, as an implementation, runs away, unfolding
$\bangp{P}$ eagerly. A lazier and more implementable replication
operator, restricted to input-guarded processes, may be obtained as follows.

\begin{eqnarray}
\bangp{\prefix{u}{v}{P}} 
	:= 
	\binpar{\lift{x}{\prefix{u}{v}{(\binpar{D(x)}{P})}}}{D(x)} \nonumber
\end{eqnarray}

\begin{remark}
  Note that the lazier definition still does not deal with summation
  or mixed summation (i.e. sums over input and output). The reader is
  invited to construct definitions of replication that deal with these
  features. 

  Further, the definitions are parameterized in a name, $x$. Can you,
  gentle reader, make a definition that eliminates this parameter and
  guarantees no accidental interaction between the replication
  machinery and the process being replicated -- i.e. no accidental
  sharing of names used by the process to get its work done and the
  name(s) used by the replication to effect copying. This latter
  revision of the definition of replication is crucial to obtaining
  the expected identity $!!P \sim !P$.
\end{remark}

\begin{remark}\label{rem:paradoxical_combinator}
  The reader familiar with the lambda calculus will have noticed the
  similarity between $D$ and the paradoxical combinator.

  [Ed. note: the existence of this seems to suggest we have to be more
  restrictive on the set of processes and names we admit if we are to
  support no-cloning.]
\end{remark}

\subsubsection{Bisimulation}

The computational dynamics gives rise to another kind of equivalence,
the equivalence of computational behavior. As previously mentioned
this is typically captured \emph{via} some form of bisimulation.

% The notion we use in this paper is weak barbed bisimulation
% \cite{milner91polyadicpi}.

The notion we use in this paper is derived from weak barbed
bisimulation \cite{milner91polyadicpi}. 

\begin{definition}
An \emph{observation relation}, $\downarrow_{\mathcal N}$, over a set
of names, $\mathcal N$, is the smallest relation satisfying the rules
below.

\infrule[Out-barb]{y \in {\mathcal N}, \; x \nameeq y}
		  {\outputp{x}{v} \downarrow_{\mathcal N} x}
\infrule[Par-barb]{\mbox{$P\downarrow_{\mathcal N} x$ or $Q\downarrow_{\mathcal N} x$}}
		  {\binpar{P}{Q} \downarrow_{\mathcal N} x}

We write $P \Downarrow_{\mathcal N} x$ if there is $Q$ such that 
$P \wred Q$ and $Q \downarrow_{\mathcal N} x$.
\end{definition}

\begin{definition}
%\label{def.bbisim}
An  ${\mathcal N}$-\emph{barbed bisimulation} over a set of names, ${\mathcal N}$, is a symmetric binary relation 
${\mathcal S}_{\mathcal N}$ between agents such that $P\rel{S}_{\mathcal N}Q$ implies:
\begin{enumerate}
\item If $P \red P'$ then $Q \wred Q'$ and $P'\rel{S}_{\mathcal N} Q'$.
\item If $P\downarrow_{\mathcal N} x$, then $Q\Downarrow_{\mathcal N} x$.
\end{enumerate}
$P$ is ${\mathcal N}$-barbed bisimilar to $Q$, written
$P \wbbisim_{\mathcal N} Q$, if $P \rel{S}_{\mathcal N} Q$ for some ${\mathcal N}$-barbed bisimulation ${\mathcal S}_{\mathcal N}$.
\end{definition}

$\mathcal{R} \subseteq \pi \times \pi$

$P \mathcal{R} Q => \forall P'. P \red P' \Rightarrow \exists Q'. Q \red Q', P' \mathcal{R} Q'$

$P \vdash x \Rightarrow Q \vdash x$

\begin{mathpar}
  \inferrule*[lab=Out-barb]{x \nameeq y}{{y}!\langle{Q}\rangle \vdash x}
  \and
  \inferrule*[lab=Par-barb]{\mbox{$P\vdash x$ or $Q\vdash x$}}{\binpar{P}{Q} \vdash x}
\end{mathpar}

\subsubsection{Contexts}

One of the principle advantages of computational calculi like the
$\pi$-calculus is a well-defined notion of context,
contextual-equivalence and a correlation between
contextual-equivalence and notions of bisimulation. The notion of
context allows the decomposition of a process into (sub-)process and
its syntactic environment, its context. Thus, a context may be
thought of as a process with a ``hole'' (written $\Box$) in it. The
application of a context $M$ to a process $P$, written $M[P]$, is
tantamount to filling the hole in $M$ with $P$. In this paper we do
not need the full weight of this theory, but do make use of the notion
of context in the proof the main theorem. 

\begin{mathpar}
  \inferrule* [lab=summation] {} {{M_{M},M_{N}} \bc \Box \;|\; x.M_{A} \;|\; M_{M}+M_{N}}
  \and
  \inferrule* [lab=agent] {} {{M_{A}} \bc (\vec{x})M_{P} \;| \; \clift{P_0,\ldots,M_{P},\ldots,P_N}}
  \and \\
  \inferrule* [lab=process] {} {{M_{P}} \bc M_{N} \;| \;P|M_{P} }
\end{mathpar} 

\begin{mathpar}
  \inferrule* [lab=sychronization] {} {M_{N} \bc \Box \;|\; x?M_{F} \;|\; x!M_{C}}
  \and
  \inferrule* [lab=abstraction] {} {{M_{F}} \bc (x)M_{P} }
  \and
  \inferrule* [lab=concretion] {} {{M_{C}} \bc \langle M_{P} \rangle }
  \and \\
  \inferrule* [lab=process] {} {{M_{P}} \bc M_{N} \;| \;P|M_{P} }
\end{mathpar}

\begin{definition}[contextual application] Given a context $M$, and
  process $P$, we define the \emph{contextual application}, $M[P] :=
  M\{P/\Box\}$. That is, the contextual application of M to P is the
  substitution of $P$ for $\Box$ in $M$.
\end{definition}

$\meaningof{-} : L \to \mathcal{P}(\pi)$

\begin{mathpar}
  \inferrule* [lab=collection] {} {\meaningof{true} = \pi, \and \meaningof{~E} = \pi \setminus \meaningof{E}, \and \meaningof{E_{1} \& E_{2}} = \meaningof{E_{1}} \cap \meaningof{E_{2}}}
\end{mathpar}

\begin{mathpar}
  \inferrule* [lab=structure] {} {\meaningof{0} = \{ P \in \pi | P \equiv 0 \}, \and \\ \meaningof{E_1 | E_2} = \{ P \in \pi | P \equiv P_{1} | P_{2}, P_{1} \in \meaningof{E_{1}}, P_{2} \in \meaningof{E_2}\} }
\end{mathpar}

\begin{mathpar}
 \inferrule* [lab=behavior] {} {\meaningof{\langle a?b \rangle E} = \{ P \in \pi | P \equiv Q | u?(y)P', \\ \and \\\\ \and \\ \;\;\; u \in \meaningof{a}, \forall z.P'\{z/y\} \in \meaningof{E\{z/b\}}\}, \and \\ \meaningof{a!E} = \{ P \in \pi | P \equiv Q | x!\langle P' \rangle, x \in \meaningof{a} P' \in \meaningof{E}\} }
\end{mathpar}

\begin{mathpar}
 \inferrule* [lab=nominal] {} {\meaningof{\quotep{E}} = \{ \quotep{P} \in \quotep{\pi} | P \in \meaningof{E} \}, \and \meaningof{\quotep{P}} = \{ \quotep{Q} \in \quotep{\pi} | P \equiv Q \} \and \\ \meaningof{@\quotep{E}} = \{ P \in \pi | P \equiv @x, x \in \meaningof{E} \}}
\end{mathpar}

\begin{eqnarray*}
  \\
  \meaningof{-} : TS \to ST
\end{eqnarray*}

\begin{eqnarray*}
  \\
  L : TS \to ST
\end{eqnarray*}

\begin{eqnarray*}
  \\
  P \models E \iff P \in \meaningof{E}
\end{eqnarray*}

\begin{eqnarray*}
  P \approx_{L} Q \iff \forall E \in L. P \models E \iff Q \models E
\end{eqnarray*}

\begin{eqnarray*}
  P \approx_{K} Q
\end{eqnarray*}

\begin{eqnarray*}
  P \approx Q
\end{eqnarray*}

$\approx_{K} = \approx = \approx_{L}$

\subsubsection{Contextual duality}

Note that contexts extend the quotation operation to a family of
operations from processes to names. Given a context, $M$, we can
define a \emph{nominal context}, $\quotep{M}$ by $\quotep{M}[P] :=
\quotep{M[P]}$. To foreshadow what is to come we observe that these
operations enjoy a duality with processes very much like the duality
between vectors and maps from vectors to scalars.

Further, because the calculus is essentially higher-order, we have a
correspondence between contexts and processes. More specifically,
given a name $x$ and a context $M$ we can construct $M^{*}_{x}$ such
that 

\begin{mathpar}
  M^{*}_{x} | \lift{x}{P} \red M[P]
\end{mathpar}

namely,

\begin{mathpar}
  M^{*}_{x} := x?(u).M[\dropn{u}]
\end{mathpar}

The dependence of $M^{*}_{x}$ on a name makes it an abstraction, 

\begin{mathpar}
  M^{*} := (x)x?(u).M[\dropn{u}]
\end{mathpar}

\subsection{Additional notation}

It will sometimes be convenient to denote the process a name
quotes. We already have the notation $x = \quotep{P}$, but it will be
convenient to introduce an alternate notation, $\procn{x}$, when we
want to emphasize the connection to the use of the name. Note that, by
virtue of name equivalence, $\quotep{\procn{x}} \nameeq x$; so, the
notation is consistent with previous definitions.

Further, because names have structure it is possible to effect
substitutions on the basis of that structure. This means we need to
upgrade our notation for substitutions, which we accomplish by
adapting comprehension notation. Thus,

\begin{mathpar}
  P\{ y / x : x \in S \}
\end{mathpar}

is interpreted to mean the process derived from P by replacing (in a
capture-avoiding manner) each occurrence of $x$ in $S$ by $y$. For example,

\begin{mathpar}
  P\{ \quotep{\procn{x}|\procn{x}} / x : x \in \freenames{P} \}
\end{mathpar}

will replace each (occurrence) of a free name $x$ in $P$ by
$\quotep{\procn{x}|\procn{x}}$.

Also, we will avail ourselves of the notation $x^{L}$ and $x^{R}$ to
denote injections of a name into disjoint copies of the name
space. There are numerous ways to accomplish this. One example can be
found in \cite{MeredithR05}. This notation overloads to vectors of
names: $\vec{x}^{\pi} := (x_{i}^{\pi} \; : \; 0 \leq i < |\vec{x}| )$ where $\pi \in \{L,R\}$.

We also use $P^{\Box} := P|\Box$.

In \cite{MeredithR05} an interpretation of the new operator is
given. It turns out that there are several possible interpretations
all enjoying the requisite algebraic properties of the operator (see
\cite{milner91polyadicpi}). We will therefore make liberal use of
$(\nu\; \vec{x})P$.

% subsection the_syntax_and_semantics_of_the_notation_system (end)   

\section{Interpretation of QM}
\subsection{Supporting definitions}
\subsubsection{Multiplication}
\begin{mathpar}
  \quotep{Q} \cdot \quotep{R} := \quotep{Q|R}
  \and \\
  \quotep{Q} \cdot P := P\{ \quotep{Q|R} / \quotep{R} : \quotep{R} \in \freenames{P} \}
\end{mathpar}

\paragraph{Discussion}
The first line needs little explanation. The second line says that
each free name of the process is replaced with the multiplication of
that name by the scalar. Multiplication of a scalar (name) by a state
(process) results in a process all the names of which have been `moved
over' by parallel composition with the process the scalar
quotes. There is a subtlety that the bound names have to be
manipulated so that multiplied names aren't accidentally
captured. There are many ways to achieve this.

\begin{remark}\label{rem:multiplication_identities}
  The reader is invited to verify that for all $x,y,z \in \QProc$ and $P \in \Proc$
  \begin{mathpar}
    x \cdot \quotep{0} \equiv x 
    \and
    x \cdot y \equiv y \cdot x
    \and
    x \cdot (y \cdot z) \equiv (x \cdot y) \cdot z
    \and \\
    \quotep{0} \cdot P \equiv P
    \and \\
    x \cdot (y \cdot P) \equiv (x \cdot y) \cdot P
    \and \\
    x \cdot (P|Q) \equiv (x \cdot P) | (x \cdot Q)
    \and \\    
  \end{mathpar}
\end{remark}

\subsubsection{Tensor product}

We define a tensor product on processes by structural induction.

\paragraph{Tensor of sums} First note that all summations, including
$\pzero$ and sequence, can be written $\Sigma_{i} x_{i}.A_{i} +
\Sigma_{j} x_{j}.C_{j}$, where we have grouped input-guarded processes
together and output-guarded processes together.

Thus, we can define the tensor product of two summations, $N_{1}\otimes N_{2}$, where

\begin{mathpar}
  N_{1} := \Sigma_{i} x_{i}.A_{i} + \Sigma_{j} x_{j}.C_{j}
  \and
  N_{2} := \Sigma_{i'} y_{i'}.B_{i'} + \Sigma_{j'} y_{j'}.D_{j'} 
\end{mathpar}

as follows.

\begin{mathpar}
  \Sigma_{i} x_{i}.A_{i} + \Sigma_{j} x_{j}.C_{j} \otimes \Sigma_{i'}
  y_{i'}.B_{i'} + \Sigma_{j'} y_{j'}.D_{j'} 
  \and \\
  := \; \Sigma_{i} \Sigma_{i'} \quotep{\stackrel{\vee}{x_{i}}| \stackrel{\vee}{y_{i'}}}.(A_{i}\otimes B_{i'}) \; | \; \Sigma_{i'} \Sigma_{i} \quotep{\stackrel{\vee}{y_{i'}}|\stackrel{\vee}{x_{i}}}.(B_{i'}\otimes A_{i})
  \and
  \;\; | \;\; \Sigma_{j} \Sigma_{j'} \quotep{\stackrel{\vee}{x_{j}}|\stackrel{\vee}{y_{j'}}}.(A_{j}\otimes B_{j'}) \; | \; \Sigma_{j'} \Sigma_{j} \quotep{\stackrel{\vee}{y_{j'}}|\stackrel{\vee}{x_{j}}}.(B_{j'}\otimes A_{j})
\end{mathpar}

\begin{remark}
  Do we need to $x^{L}$ and $y^{R}$ for this construction as well?
\end{remark}

\paragraph{Tensor of parallel compositions} Next, we distribute tensor
over par.

\begin{mathpar}
  P_{1}|P_{2} \otimes Q_{1}|Q_{2} := (P_{1} \otimes Q_{1}) | (P_{1}
  \otimes Q_{2}) | (P_{2} \otimes Q_{1}) | (P_{2} \otimes Q_{2})
\end{mathpar}

\paragraph{Tensor with dropped names} We treat tensor of a
process with a dropped name as parallel composition.

\begin{mathpar}
  P \otimes \dropn{x} := P | \dropn{x}
\end{mathpar}

\paragraph{Tensor of agents}

Finally, we need to define tensor on agents. Note that the definition
of tensor on normal products only tensors inputs with inputs and
outputs with outputs. Thus, we only have to define the operation on
``homogeneous'' pairings.

\begin{mathpar}
  (\vec{x})P \otimes (\vec{y})Q
  \and \\
  := (x_{0}^{L}|y_{0}^{R},\ldots,x_{0}^{L}|y_{n}^{R},\ldots,x_{m}^{L}|y_{0}^{R},\ldots,x_{m}^{L}|y_{n}^R)(P\{ \vec{x}^{L}/\vec{x}\} \otimes Q \{ \vec{y}^{R}/\vec{y}\})
  \and \\
  \clift{\vec{P}} \otimes \clift{\vec{Q}}
  \and \\
  := \clift{P_{0}\otimes Q_{0},\ldots,P_{0}\otimes Q_{n},\ldots,P_{m}\otimes Q_{0},\ldots,P_{m}\otimes Q_{n}}
\end{mathpar}

\begin{remark}
  Observe that arities of tensored abstractions matches arities of
  tensored concretions if the original arities matched. Note also that
  the length of the arities corresponds to the increase in dimension
  we see in ordinary vector space tensor product.
\end{remark}

\begin{remark}
  Operationally, this definition distributes the tensor down to
  components ``linked'' by summation. Tensor over summation is
  intriguing in that it mixes names. Moreover, as a consequence of the
  way it mixes names we have the identities for all $x \in \QProc$ and
  $P,Q \in \Proc$

  \begin{mathpar}
    (x \cdot P) \otimes Q \equiv x \cdot (P \otimes Q) \equiv P \otimes (x \cdot Q)
    \and
    P \otimes \pzero \equiv P
  \end{mathpar}

  that the reader is invited to verify.
\end{remark}

\subsubsection{Annihilation}
\begin{mathpar}
  P^{\perp} := \{ Q | \forall R. P|Q \red^{*} R \Rightarrow R \red^{*} \pzero \}
  \and \\
  P^{\underline{\perp}} := \Sigma_{Q \in P^{\perp}} \quotep{Q}?(y).(\dropn{y}|Q) | \Sigma_{Q \in P^{\perp}} \quotep{Q}\clift{\Box}
\end{mathpar}

\paragraph{Discussion} The reader will note that $P^{\perp}$ is a
\emph{set} of processes, while $P^{\underline{\perp}}$ is a
\emph{context}. We call the set $P^{\perp}$ the \emph{annihilators} of
$P$. The parallel composition of a process in the annihilators of $P$
with $P$ will result in a process, the state space of which has all
paths eventually leading to $\pzero$. Execution may endure loops; but
under reasonable conditions of fairness (naturally guaranteed under
most notions of bisimulation) such a composite process cannot get
stuck in such a loop and will, eventually pop out and terminate.

The context $P^{\underline{\perp}}$ is ready and willing to ``take the
$P$ out of'' the process to which it is applied. It will effectively
transmit the code of the process to which it is applied to one of the
annihilators and run the process against it.

\subsubsection{Evaluation}
We fix $M$ a domain of fully abstract interpretation with an equality
coincident with bisimulation. We take $\meaningof{\cdot} : \Proc \to
M$ to be the map interpreting processes and $\nmeaningof{\cdot} : \M
\to Proc$ to be the map running the other way. Then we define

\begin{mathpar}
  \int P := \nmeaningof{\meaningof{P}}
\end{mathpar}

\paragraph{Discussion}
There are many fully abstract interpretations of Milner's
$\pi$-calculus. Any of them can be used as a basis for interpreting
the reflective calculus here. Equipped with such a domain it is
largely a matter of grinding through to check that the Yoneda
construction for the normalization-by-evaluation program can be
extended to this setting.

\begin{remark}
  The reader is invited to verify that $\int (P^{\underline{\perp}}[P]) = 0$.
\end{remark}

\subsection{Quantum mechanics}

Table \ref{tbl:core_qm_op_defns} gives the core operational definitions

\begin{table}[htp]\label{tbl:core_qm_op_defns}
  \center{
    \fbox{
      \begin{tabular}{c|c}
        quantum mechanics & process calculus \\
        \hline
        scalar & $x := \quotep{P}$ \\
        state vector & $\state{P} := P$ \\
        dual & $\state{P}^{*} := \event{P^{\underline{\perp}}} := \quotep{P^{\underline{\perp}}}[-]$ \\
        matrix & $ \Sigma_{\alpha} \state{P_{\alpha}}x_{\alpha}\event{Q_{\alpha}}$ \\
        vector addition & $\state{P} + \state{Q} := \state{P | Q}$ \\
        tensor product & $\state{P} \otimes \state{Q} := \state{P \otimes Q}$ \\
        inner product & $\innerprod{P}{Q} := \quotep{\int P^{\underline{\perp}}[Q]}$ \\
      \end{tabular}
    }
  }
  \caption{QM - operational definitions}
\end{table}

where

\begin{mathpar}
  \prmatrix{P}{Q} := \fprmatrix{P}{\quotep{\pzero}}{Q}
  \and
  \fprmatrix{P}{x}{Q} := (\state{P},x,\event{Q})
  \and
  (\fprmatrix{P}{x}{Q})(\state{R}) := x \cdot \innerprod{Q}{R} \cdot \state{P}
  \and
  (\fprmatrix{P}{x}{Q})(\event{R}) := x \cdot \innerprod{R}{P} \cdot \event{Q}
\end{mathpar}

\paragraph{Discussion}
As promised: vectors (aka states) are represented as processes; duals
as contextual duals; inner product definition should be compared with
standard inner product definition for ....

\begin{remark}
  Assuming $\int (P^{\underline{\perp}}[P]) = 0$, the reader is
  invited to verify that $(\fprmatrix{P}{x}{P})(\state{P}) = x \cdot \state{P}$.
\end{remark}

\begin{remark}
  The reader is invited to verify that $\innerprod{P}{Q}$ could
  equally well have been written $\quotep{\int \stackrel{\vee}{x}}$
  where $x = \event{P^{\underline{\perp}}}(Q)$.

  One of the motivations for this remark is that there is another way
  to factor these operations. We could package up evaluation in the dual:

  \begin{mathpar}
    \state{P}^{*} := \event{\int P^{\underline{\perp}}} := \quotep{\int P^{\underline{\perp}}}[-]
  \end{mathpar}

  and then have inner product defined by
  
  \begin{mathpar}
    \innerprod{P}{Q} := \event{P}(Q)
  \end{mathpar}

  Hopefully, experience with the calculations will provide guidance on
  the best factoring.
\end{remark}

\begin{remark}
  Assuming $\int (P^{\underline{\perp}}[P]) = 0$, the reader is
  invited to verify that $\forall P,Q. (\prmatrix{0}{Q})(\state{0}) =
  \state{0}$ and dually $(\prmatrix{P}{0})(\event{0}) = \event{0}$.
\end{remark}

\begin{remark}
  i'm a little worried that i don't (yet) have proper support for
  complex conjugacy. But, the observation above may give us a
  clue. According to Abramsky, it must be the case that the scalars
  are iso to the homset of the identity for the tensor -- which the
  observation above characterizes. 

  For now, we will simply bookmark the notion with $\overline{x}$.
\end{remark}

\subsubsection{Adjointness}

We need to give a definition of $(\cdot)^{\dagger}$ for matrices. The
obvious candidate definition is
\begin{mathpar}
(\Sigma_{\alpha}\fprmatrix{P_{\alpha}}{x_{\alpha}}{Q_{\alpha}})^{\dagger}
= \Sigma_{\alpha}\fprmatrix{(Q_{\alpha}^{\underline{\perp}})^{*}}{\overline{x}_{\alpha}}{P_{\alpha}^{\underline{\perp}}} 
\end{mathpar}

But, $(Q_{\alpha}^{\underline{\perp}})^{*}$ requires a name along
which to communicate the process to achieve the context application.

\subsubsection{Basis for a basis}
If processes label states and ``addition'' of states (a.k.a. vector
addition) is interpreted as parallel composition, what corresponds to
notions of linear independence and basis? Here, we recall that Yoshida
has developed a set of \emph{combinators} for an asynchronous verison
of Milner's $\pi$-calculus. These are a finite set of processes such
any process can be expressed as parallel composition of these
combinators together with liberal uses of the new operator and
replication. We can simply give a translation of these into the
present calculus and have reasonable expectation that the property
carries over. That is, that the resultant set allows to express all
processes via parallel composition. Note, however, that there is no
new operator or replication in this calculus. As a result, we expect
that the corresponding set is actually infinite. That is, we expect
that the space is actually infinite dimensional.

\begin{remark}
  The attentive reader may be a bit concerned. Certainly, the
  collection $S$, $K$ and $I$ is a finite set of
  combinators. Shouldn't we expect to see a finite set of combinators
  for an effectively equivalent system? i am very sympathetic to this
  critique and feel it warrants full attention. On the other hand, i
  also have in mind the following analogy. The natural numbers, as a
  monoid under addition, has exactly $1$ generator, while the natural
  numbers, as a monoid under multiplication, has countably many
  generators (the primes). We observe that the application of the
  lambda calculus is much less resource sensitive than the parallel
  composition of the $\pi$-calculus. Could it be the case that we have
  an analogy of the form
  
  \begin{mathpar}
    m + n : MN :: m*n : M|N
  \end{mathpar}

  giving a similar blow up in the set of ``primes''?  This is such a
  wonderful thought that, even if it's not true, i think it's worth
  writing down.
\end{remark}
 

\documentclass[12pt]{llncs}
%\documentclass{jktr}

\usepackage[pdftex]{hyperref}                   
\usepackage {listings}
\usepackage {mathpartir}
\usepackage{bcprules}
%\usepackage{listings}
                       
\usepackage{graphicx} 
%\usepackage[margins=2.5cm,nohead,nofoot]{geometry}
%\usepackage{geometry}
\usepackage{amsfonts}
\usepackage{amstext}
\usepackage{latexsym}
\usepackage{amssymb}
\usepackage{color}


%\include{myPreamble}
\include{qm2pi.local} 

%\ifpdf
%\usepackage[pdftex]{graphicx}
%\else
%\usepackage{graphicx}
%\fi

 % \ifpdf
%  \usepackage{pdfsync}
%  \if


%\title{Brief Article}
%\author{David F. Snyder}
%\author{L.G. Meredith}

%\address{Dept. of Math., Texas State University--San Marcos, San Marcos, TX 78666}
       
\pagestyle{empty}


\begin{document}

\lstset{language=[Objective]Caml,frame=shadowbox}

\input{qm2pi.front}

% section front matter (end)

\input{qm2pi.intro} 
 
% section introduction (end)

% \input{qm2pi.knotations} 

% section notation (end)

\input{qm2pi.process.calculi} 

% section concurrent_process_calculi_and_spatial_logics_ (end)
    
%\input{qm2pi.knots2pi} 

%\input{qm2pi.trefoil} 

%\input{qm2pi.mainthm} 

% subsection basic_interpretation (end)

%\input{qm2pi.rho.presentation} 
\subsection{The syntax and semantics of the notation system}\label{sub:the_syntax_and_semantics_of_the_notation_system} % (fold)

We now summarize a technical presentation of the calculus that
embodies our theory of dynamics. The typical presentation of such a
calculus follows the style of giving generators and relations on
them. The grammar, below, describing term constructors, freely
generates the set of processes, $\Proc$. This set is then quotiented
by a relation known as structural congruence and it is over this set
that the notion of dynamics is expressed. This presentation is
essentially that of \cite{MeredithR05} with the addition of
polyadicity and summation. For readability we have relegated some of
the technical subtleties to an appendix.

\subsubsection{Process grammar}\label{subsub:process_grammar}

\begin{mathpar}
  \inferrule* [lab=synchronization] {} {{M} \bc \pzero \;|\; x?F \;|\; x!C }
  \and
  \inferrule* [lab=abstraction] {} {{F} \bc (x)P}
  \and
  \inferrule* [lab=concretion] {} {{C} \bc \langle Q \rangle}
  \and
  \inferrule* [lab=process] {} {{P,Q} \bc M \;| \;P|Q \;|\; @{x}}
  \and
  \inferrule* [lab=name] {} {{x} \bc \quotep{P}}
\end{mathpar} 

Note that $\vec{x}$ (resp. $\vec{P}$) denotes a vector of names
(resp. processes) of length $|\vec{x}|$ (resp. $|\vec{P}|$). We adopt
the following useful abbreviations.

\begin{mathpar}
   x?(\vec{y}).P := x.(\vec{y})P \and  x\clift{\vec{P}} := x.\clift{\vec{P}}
   \and x!(y) := \lift{x}{\dropn{y}}
   \and \Pi_{i=0}^{n-1}P_i := P_0 | \ldots | P_{n-1}
\end{mathpar}

\subsubsection{Structural congruence}

\paragraph{Free and bound names and alpha-equivalence.} At the
core of structural equivalence is alpha-equivalence which identifies
process that are the same up to a change of variable. Formally, we
recognize the distinction between free and bound names. The free names
of a process, $\freenames{P}$, may be calculated recursively as
follows:

\begin{mathpar}
\freenames{\pzero} := \emptyset
  \and \\
  \freenames{x?(y).P} := \{ x \} \cup (\freenames{P} \setminus \{ y \})
  \and 
  \freenames{x!\langle P \rangle} := \{ x \} \cup \{ P \} 
  \and \\
  \freenames{P|Q} := \freenames{P} \cup \freenames{Q}
  \and \\
  \freenames{@{x}} := \{ x \}
\end{mathpar}

$\pi$
$\quotep{\pi}$

$\freenames{-} : \pi \to \mathcal{P}(\quotep{\pi})$

\begin{eqnarray*}
  \freenames{\pzero} & := & \emptyset \\
  \freenames{x?(y).P} & := & \{ x \} \cup (\freenames{P} \setminus \{ y \}) \\
  \freenames{x!\langle P \rangle} & := & \{ x \} \cup \{ P \} \\
  \freenames{P|Q} & := & \freenames{P} \cup \freenames{Q} \\
  \freenames{\dropn{x}} & := & \{ x \}
\end{eqnarray*}

The bound names of a process, $\boundnames{P}$, are those names occurring in $P$
that are not free. For example, in $x?(y).0$, the name $x$ is free, while $y$ is bound.

\begin{mathpar}
  \inferrule* [lab=monoidal-laws] {} { P|Q \equiv Q|P \and P|0 \equiv P \and P|(Q|R) \equiv (P|Q)|R }
\end{mathpar}

\begin{mathpar}
  \inferrule* [lab=alpha-equivalence] {} { (x)P \equiv (y)P\{y/x\} \and y \not\in \freenames{P} }
\end{mathpar}

\begin{definition}
Then two processes, $P,Q$, are alpha-equivalent if $P = Q\{\vec{y}/\vec{x}\}$ for
some $\vec{x} \in \boundnames{Q},\vec{y} \in \boundnames{P}$, where $Q\{\vec{y}/\vec{x}\}$
denotes the capture-avoiding substitution of $\vec{y}$ for $\vec{x}$ in $Q$.
\end{definition}

\begin{definition}
  The {\em structural congruence} \cite{SangiorgiWalker} , $\equiv$,
  between processes is the least congruence containing
  alpha-equivalence, satisfying the abelian monoid laws
  (associativity, commutativity and $\pzero$ as identity) for parallel
  composition $|$ and for summation $+$.
\end{definition}

\subsection{Name equivalence}

We take name equivalence, written $\nameeq$, to be the smallest
equivalence relation generated by the following rules.

\begin{mathpar}
\inferrule*[lab=Quote-drop]
{ }
{ \quotep{@{x}} \nameeq x }

\inferrule*[lab=Struct-equiv]
{ P \scong Q }
{ \quotep{P} \nameeq \quotep{Q} }
\end{mathpar}

The astute reader will have noticed that the mutual recursion of names
and processes imposes a mutual recursion on alpha-equivalence and
structural equivalence via name-equivalence. Fortunately, all of this
works out pleasantly and we may calculate in the natural way, free of
concern. The reader interested in the details is referred to the
appendix \ref{appendix:rho_details}.

\subsection{Substitution}

We use $\Proc$ for the set of processes, $\QProc$ for the set of
names, and $\id{\{}\vec{y} / \vec{x} \id{\}}$ to denote partial maps,
$s : \QProc \rightarrow \QProc$. A map, $s$ lifts, uniquely, to a map
on process terms, $\widehat{s} : \Proc \rightarrow \Proc$ by the
following equations.

\begin{mathpar}
  (0) \psubstp{Q}{P} := 0 \\
  (R \juxtap S) \psubstp{Q}{P}
  :=    
  (R)\psubstp{Q}{P} \juxtap (S) \psubstp{Q}{P} \\
  (x?(y).R) \psubstp{Q}{P}    
  :=    
  (x)\substp{Q}{P} (z)\concat( (R \psubstn{z}{y}) \psubstp{Q}{P} ) \\
  (\lift{x}{R}) \psubstp{Q}{P}  
  :=
  \lift{(x)\substp{Q}{P}}{ R \psubstp{Q}{P} } \\
%   (\dropn{x})  \psubstp{Q}{P}       
%   := 
%   \left\{ 
%     \begin{array}{ccc} 
%       \dropn{\quotep{Q}} & & x \nameeq \quotep{P} \\
%       \dropn{x} & & otherwise \\
%     \end{array}
%   \right. 
  (\dropn{x})  \psubstp{Q}{P}       
  := 
  \left\{ 
    \begin{array}{ccc} 
      Q & & x \nameeq \quotep{P} \\
      \dropn{x} & & otherwise \\
    \end{array}
  \right.
\end{mathpar}
 

where

\begin{eqnarray}
  (x)\id{\{} \lpquote Q \rpquote / \lpquote P \rpquote \id{\}}            = 
  \left\{ 
    \begin{array}{ccc}
      \lpquote Q \rpquote & & x \nameeq \lpquote P \rpquote \\
      x & & otherwise \\
    \end{array}
  \right. \nonumber
\end{eqnarray}

and $z$ is chosen distinct from $\quotep{P}$, $\quotep{Q}$, the free
names in $Q$, and all the names in $R$. Our $\alpha$-equivalence will
be built in the standard way from this substitution.

\begin{remark}\label{rem:no_self_referential_names}
  One consequence of these definitions is that $\forall P. \quotep{P}
  \not\in \freenames{P}$.
\end{remark}

\subsection{ Dynamic quote: an example }

Anticipating something of what's to come, consider applying the
substitution, $\widehat{\id{\{}u / z \id{\}}}$, to the following pair
of processes, $\lift{w}{y!(z)}$ and $w[ \lpquote y!(z) \rpquote ]$.

\begin{eqnarray}
	\lift{w}{y!(z)}\widehat{\id{\{}u / z \id{\}}}
		& = &
		\lift{w}{y!(u)} \nonumber\\
	w[ \lpquote y!(z) \rpquote ] \widehat{ \id{\{}u / z \id{\}} }
		& = &
		w[ \lpquote y!(z) \rpquote ] \nonumber
\end{eqnarray}

Because the body of the process between quotes is impervious to
substitution, we get radically different answers. In fact, by
examining the first process in an input context,
e.g. $x?(z).\lift{w}{y!(z)}$, we see that the process under the lift
operator may be shaped by prefixed inputs binding a name inside it. In
this sense, the lift operator will be seen as a way to dynamically
construct processes before reifying them as names.

Finally equipped with these standard features we can present the
dynamics of the calculus.

\subsubsection{Operational semantics} 

Finally, we introduce the computational dynamics. What marks these
algebras as distinct from other more traditionally studied algebraic
structures, e.g. vector spaces or polynomial rings, is the manner in
which dynamics is captured. In traditional structures, dynamics is typically
expressed through morphisms between such structures, as in linear maps
between vector spaces or morphisms between rings. In algebras
associated with the semantics of computation, the dynamics is
expressed as part of the algebraic structure itself, through a
reduction reduction relation typically denoted by $\red$. Below, we
give a recursive presentation of this relation for the calculus used
in the encoding.

$\red \subseteq \pi \times \pi$
$\red : \pi \to \mathcal{P}(\pi)$

\begin{mathpar}
  \inferrule* [lab=Comm] { \textsf{match}( x_{src}, x_{trgt} ) } { x_{trgt}?(y)P \; | \; x_{src}!\langle {Q} \rangle \red P\{\quotep{Q}/y}\} }
  \and \\
  \inferrule* [lab=Par] {{P} \red {P}'} {{{P} | {Q}} \red {{P}' | {Q}}}
  \and
  \inferrule* [lab=Equiv]{{{P} \scong {P}'} \andalso {{P}' \red {Q}'} \andalso {{Q}' \scong {Q}}}{{P} \red {Q}}
\end{mathpar}

\begin{eqnarray*}
  match_{\equiv} (\quotep{P},\quotep{Q}) & := & P \equiv Q \\
  match_{\dagger}(\quotep{P},\quotep{Q}) & := & \forall R. P|Q \red^{*} R => R \red^{*} 0 \\
  match_{K}(\quotep{P},\quotep{Q}) & := & K \mbox{ for some context } K
\end{eqnarray*}

$u?(x)P | u!\langle Q \rangle \red P\{\quotep{Q}/x\}$

%We write $\wred$ for $\red^*$, and $P\red$ if $\exists Q $ such that $ P \red Q$.
We write $P\red$ if $\exists Q $ such that $ P \red Q$ and $P\not\red$, otherwise.

\section{Replication}

As mentioned before, it is known that replication (and hence
recursion) can be implemented in a higher-order process algebra
\cite{SangiorgiWalker}. As our first example of calculation with the
machinery thus far presented we give the construction explicitly in
the {\rhoc}.

\begin{eqnarray}
	D_{x} & := & \prefix{x}{y}{(\binpar{\outputp{x}{y}}{@{y}})} \nonumber\\
	\bangp_{x}{P} & := & \binpar{{x}!\langle{\binpar{D_{x}}{P}}\rangle}{D_{x}} \nonumber
\end{eqnarray}

\begin{eqnarray}
	\bangp_{x}{P} & & \nonumber\\
	=
	& {x}!\langle{(\prefix{x}{y}{(\outputp{x}{y} | @{y})) | P}}\rangle 
	      | \prefix{x}{y}{(\outputp{x}{y} | @{y})} & \nonumber\\
	\red
	& (\outputp{x}{y} | @{y})\substn{\quotep{(\prefix{x}{y}{(@{y} | \outputp{x}{y})) | P}}}{y} & \nonumber\\
	=
	& \outputp{x}{\quotep{(\prefix{x}{y}{(\outputp{x}{y} | @{y})) | P}}}
	  | {(\prefix{x}{y}{(\outputp{x}{y} | @{y})) | P}} & \nonumber\\
	\red
	& \ldots & \nonumber\\
	\red^*
	& P | P | \ldots & \nonumber
\end{eqnarray}

Of course, this encoding, as an implementation, runs away, unfolding
$\bangp{P}$ eagerly. A lazier and more implementable replication
operator, restricted to input-guarded processes, may be obtained as follows.

\begin{eqnarray}
\bangp{\prefix{u}{v}{P}} 
	:= 
	\binpar{\lift{x}{\prefix{u}{v}{(\binpar{D(x)}{P})}}}{D(x)} \nonumber
\end{eqnarray}

\begin{remark}
  Note that the lazier definition still does not deal with summation
  or mixed summation (i.e. sums over input and output). The reader is
  invited to construct definitions of replication that deal with these
  features. 

  Further, the definitions are parameterized in a name, $x$. Can you,
  gentle reader, make a definition that eliminates this parameter and
  guarantees no accidental interaction between the replication
  machinery and the process being replicated -- i.e. no accidental
  sharing of names used by the process to get its work done and the
  name(s) used by the replication to effect copying. This latter
  revision of the definition of replication is crucial to obtaining
  the expected identity $!!P \sim !P$.
\end{remark}

\begin{remark}\label{rem:paradoxical_combinator}
  The reader familiar with the lambda calculus will have noticed the
  similarity between $D$ and the paradoxical combinator.

  [Ed. note: the existence of this seems to suggest we have to be more
  restrictive on the set of processes and names we admit if we are to
  support no-cloning.]
\end{remark}

\subsubsection{Bisimulation}

The computational dynamics gives rise to another kind of equivalence,
the equivalence of computational behavior. As previously mentioned
this is typically captured \emph{via} some form of bisimulation.

% The notion we use in this paper is weak barbed bisimulation
% \cite{milner91polyadicpi}.

The notion we use in this paper is derived from weak barbed
bisimulation \cite{milner91polyadicpi}. 

\begin{definition}
An \emph{observation relation}, $\downarrow_{\mathcal N}$, over a set
of names, $\mathcal N$, is the smallest relation satisfying the rules
below.

\infrule[Out-barb]{y \in {\mathcal N}, \; x \nameeq y}
		  {\outputp{x}{v} \downarrow_{\mathcal N} x}
\infrule[Par-barb]{\mbox{$P\downarrow_{\mathcal N} x$ or $Q\downarrow_{\mathcal N} x$}}
		  {\binpar{P}{Q} \downarrow_{\mathcal N} x}

We write $P \Downarrow_{\mathcal N} x$ if there is $Q$ such that 
$P \wred Q$ and $Q \downarrow_{\mathcal N} x$.
\end{definition}

\begin{definition}
%\label{def.bbisim}
An  ${\mathcal N}$-\emph{barbed bisimulation} over a set of names, ${\mathcal N}$, is a symmetric binary relation 
${\mathcal S}_{\mathcal N}$ between agents such that $P\rel{S}_{\mathcal N}Q$ implies:
\begin{enumerate}
\item If $P \red P'$ then $Q \wred Q'$ and $P'\rel{S}_{\mathcal N} Q'$.
\item If $P\downarrow_{\mathcal N} x$, then $Q\Downarrow_{\mathcal N} x$.
\end{enumerate}
$P$ is ${\mathcal N}$-barbed bisimilar to $Q$, written
$P \wbbisim_{\mathcal N} Q$, if $P \rel{S}_{\mathcal N} Q$ for some ${\mathcal N}$-barbed bisimulation ${\mathcal S}_{\mathcal N}$.
\end{definition}

$\mathcal{R} \subseteq \pi \times \pi$

$P \mathcal{R} Q => \forall P'. P \red P' \Rightarrow \exists Q'. Q \red Q', P' \mathcal{R} Q'$

$P \vdash x \Rightarrow Q \vdash x$

\begin{mathpar}
  \inferrule*[lab=Out-barb]{x \nameeq y}{{y}!\langle{Q}\rangle \vdash x}
  \and
  \inferrule*[lab=Par-barb]{\mbox{$P\vdash x$ or $Q\vdash x$}}{\binpar{P}{Q} \vdash x}
\end{mathpar}

\subsubsection{Contexts}

One of the principle advantages of computational calculi like the
$\pi$-calculus is a well-defined notion of context,
contextual-equivalence and a correlation between
contextual-equivalence and notions of bisimulation. The notion of
context allows the decomposition of a process into (sub-)process and
its syntactic environment, its context. Thus, a context may be
thought of as a process with a ``hole'' (written $\Box$) in it. The
application of a context $M$ to a process $P$, written $M[P]$, is
tantamount to filling the hole in $M$ with $P$. In this paper we do
not need the full weight of this theory, but do make use of the notion
of context in the proof the main theorem. 

\begin{mathpar}
  \inferrule* [lab=summation] {} {{M_{M},M_{N}} \bc \Box \;|\; x.M_{A} \;|\; M_{M}+M_{N}}
  \and
  \inferrule* [lab=agent] {} {{M_{A}} \bc (\vec{x})M_{P} \;| \; \clift{P_0,\ldots,M_{P},\ldots,P_N}}
  \and \\
  \inferrule* [lab=process] {} {{M_{P}} \bc M_{N} \;| \;P|M_{P} }
\end{mathpar} 

\begin{mathpar}
  \inferrule* [lab=sychronization] {} {M_{N} \bc \Box \;|\; x?M_{F} \;|\; x!M_{C}}
  \and
  \inferrule* [lab=abstraction] {} {{M_{F}} \bc (x)M_{P} }
  \and
  \inferrule* [lab=concretion] {} {{M_{C}} \bc \langle M_{P} \rangle }
  \and \\
  \inferrule* [lab=process] {} {{M_{P}} \bc M_{N} \;| \;P|M_{P} }
\end{mathpar}

\begin{definition}[contextual application] Given a context $M$, and
  process $P$, we define the \emph{contextual application}, $M[P] :=
  M\{P/\Box\}$. That is, the contextual application of M to P is the
  substitution of $P$ for $\Box$ in $M$.
\end{definition}

$\meaningof{-} : L \to \mathcal{P}(\pi)$

\begin{mathpar}
  \inferrule* [lab=collection] {} {\meaningof{true} = \pi, \and \meaningof{~E} = \pi \setminus \meaningof{E}, \and \meaningof{E_{1} \& E_{2}} = \meaningof{E_{1}} \cap \meaningof{E_{2}}}
\end{mathpar}

\begin{mathpar}
  \inferrule* [lab=structure] {} {\meaningof{0} = \{ P \in \pi | P \equiv 0 \}, \and \\ \meaningof{E_1 | E_2} = \{ P \in \pi | P \equiv P_{1} | P_{2}, P_{1} \in \meaningof{E_{1}}, P_{2} \in \meaningof{E_2}\} }
\end{mathpar}

\begin{mathpar}
 \inferrule* [lab=behavior] {} {\meaningof{\langle a?b \rangle E} = \{ P \in \pi | P \equiv Q | u?(y)P', \\ \and \\\\ \and \\ \;\;\; u \in \meaningof{a}, \forall z.P'\{z/y\} \in \meaningof{E\{z/b\}}\}, \and \\ \meaningof{a!E} = \{ P \in \pi | P \equiv Q | x!\langle P' \rangle, x \in \meaningof{a} P' \in \meaningof{E}\} }
\end{mathpar}

\begin{mathpar}
 \inferrule* [lab=nominal] {} {\meaningof{\quotep{E}} = \{ \quotep{P} \in \quotep{\pi} | P \in \meaningof{E} \}, \and \meaningof{\quotep{P}} = \{ \quotep{Q} \in \quotep{\pi} | P \equiv Q \} \and \\ \meaningof{@\quotep{E}} = \{ P \in \pi | P \equiv @x, x \in \meaningof{E} \}}
\end{mathpar}

\begin{eqnarray*}
  \\
  \meaningof{-} : TS \to ST
\end{eqnarray*}

\begin{eqnarray*}
  \\
  L : TS \to ST
\end{eqnarray*}

\begin{eqnarray*}
  \\
  P \models E \iff P \in \meaningof{E}
\end{eqnarray*}

\begin{eqnarray*}
  P \approx_{L} Q \iff \forall E \in L. P \models E \iff Q \models E
\end{eqnarray*}

\begin{eqnarray*}
  P \approx_{K} Q
\end{eqnarray*}

\begin{eqnarray*}
  P \approx Q
\end{eqnarray*}

$\approx_{K} = \approx = \approx_{L}$

\subsubsection{Contextual duality}

Note that contexts extend the quotation operation to a family of
operations from processes to names. Given a context, $M$, we can
define a \emph{nominal context}, $\quotep{M}$ by $\quotep{M}[P] :=
\quotep{M[P]}$. To foreshadow what is to come we observe that these
operations enjoy a duality with processes very much like the duality
between vectors and maps from vectors to scalars.

Further, because the calculus is essentially higher-order, we have a
correspondence between contexts and processes. More specifically,
given a name $x$ and a context $M$ we can construct $M^{*}_{x}$ such
that 

\begin{mathpar}
  M^{*}_{x} | \lift{x}{P} \red M[P]
\end{mathpar}

namely,

\begin{mathpar}
  M^{*}_{x} := x?(u).M[\dropn{u}]
\end{mathpar}

The dependence of $M^{*}_{x}$ on a name makes it an abstraction, 

\begin{mathpar}
  M^{*} := (x)x?(u).M[\dropn{u}]
\end{mathpar}

\subsection{Additional notation}

It will sometimes be convenient to denote the process a name
quotes. We already have the notation $x = \quotep{P}$, but it will be
convenient to introduce an alternate notation, $\procn{x}$, when we
want to emphasize the connection to the use of the name. Note that, by
virtue of name equivalence, $\quotep{\procn{x}} \nameeq x$; so, the
notation is consistent with previous definitions.

Further, because names have structure it is possible to effect
substitutions on the basis of that structure. This means we need to
upgrade our notation for substitutions, which we accomplish by
adapting comprehension notation. Thus,

\begin{mathpar}
  P\{ y / x : x \in S \}
\end{mathpar}

is interpreted to mean the process derived from P by replacing (in a
capture-avoiding manner) each occurrence of $x$ in $S$ by $y$. For example,

\begin{mathpar}
  P\{ \quotep{\procn{x}|\procn{x}} / x : x \in \freenames{P} \}
\end{mathpar}

will replace each (occurrence) of a free name $x$ in $P$ by
$\quotep{\procn{x}|\procn{x}}$.

Also, we will avail ourselves of the notation $x^{L}$ and $x^{R}$ to
denote injections of a name into disjoint copies of the name
space. There are numerous ways to accomplish this. One example can be
found in \cite{MeredithR05}. This notation overloads to vectors of
names: $\vec{x}^{\pi} := (x_{i}^{\pi} \; : \; 0 \leq i < |\vec{x}| )$ where $\pi \in \{L,R\}$.

We also use $P^{\Box} := P|\Box$.

In \cite{MeredithR05} an interpretation of the new operator is
given. It turns out that there are several possible interpretations
all enjoying the requisite algebraic properties of the operator (see
\cite{milner91polyadicpi}). We will therefore make liberal use of
$(\nu\; \vec{x})P$.

% subsection the_syntax_and_semantics_of_the_notation_system (end)   

\input{qm2pi.qmops} 

\input{qm2pi.sterngerlach} 

\input{qm2pi.metric} 

% section concurrent_process_calculi (end)

%\input{qm2pi.proofsketch}

% section proof sketch (end)

%\input{qm2pi.slviaknots} 

% section spatial logic via knots (end)

\input{qm2pi.conclusion}

% section conclusion (end)

%\input{qm2pi.dtcodes} 

% section wiring algorithm (end)

\input{qm2pi.ack} 

% section acknowledgments (end)

\newpage


\bibliographystyle{plain}   
\bibliography{../../biblios/main.bib}

\input{qm2pi.rhodetails}

\end{document}

 

\documentclass[12pt]{llncs}
%\documentclass{jktr}

\usepackage[pdftex]{hyperref}                   
\usepackage {listings}
\usepackage {mathpartir}
\usepackage{bcprules}
%\usepackage{listings}
                       
\usepackage{graphicx} 
%\usepackage[margins=2.5cm,nohead,nofoot]{geometry}
%\usepackage{geometry}
\usepackage{amsfonts}
\usepackage{amstext}
\usepackage{latexsym}
\usepackage{amssymb}
\usepackage{color}


%\include{myPreamble}
\include{qm2pi.local} 

%\ifpdf
%\usepackage[pdftex]{graphicx}
%\else
%\usepackage{graphicx}
%\fi

 % \ifpdf
%  \usepackage{pdfsync}
%  \if


%\title{Brief Article}
%\author{David F. Snyder}
%\author{L.G. Meredith}

%\address{Dept. of Math., Texas State University--San Marcos, San Marcos, TX 78666}
       
\pagestyle{empty}


\begin{document}

\lstset{language=[Objective]Caml,frame=shadowbox}

\input{qm2pi.front}

% section front matter (end)

\input{qm2pi.intro} 
 
% section introduction (end)

% \input{qm2pi.knotations} 

% section notation (end)

\input{qm2pi.process.calculi} 

% section concurrent_process_calculi_and_spatial_logics_ (end)
    
%\input{qm2pi.knots2pi} 

%\input{qm2pi.trefoil} 

%\input{qm2pi.mainthm} 

% subsection basic_interpretation (end)

%\input{qm2pi.rho.presentation} 
\subsection{The syntax and semantics of the notation system}\label{sub:the_syntax_and_semantics_of_the_notation_system} % (fold)

We now summarize a technical presentation of the calculus that
embodies our theory of dynamics. The typical presentation of such a
calculus follows the style of giving generators and relations on
them. The grammar, below, describing term constructors, freely
generates the set of processes, $\Proc$. This set is then quotiented
by a relation known as structural congruence and it is over this set
that the notion of dynamics is expressed. This presentation is
essentially that of \cite{MeredithR05} with the addition of
polyadicity and summation. For readability we have relegated some of
the technical subtleties to an appendix.

\subsubsection{Process grammar}\label{subsub:process_grammar}

\begin{mathpar}
  \inferrule* [lab=synchronization] {} {{M} \bc \pzero \;|\; x?F \;|\; x!C }
  \and
  \inferrule* [lab=abstraction] {} {{F} \bc (x)P}
  \and
  \inferrule* [lab=concretion] {} {{C} \bc \langle Q \rangle}
  \and
  \inferrule* [lab=process] {} {{P,Q} \bc M \;| \;P|Q \;|\; @{x}}
  \and
  \inferrule* [lab=name] {} {{x} \bc \quotep{P}}
\end{mathpar} 

Note that $\vec{x}$ (resp. $\vec{P}$) denotes a vector of names
(resp. processes) of length $|\vec{x}|$ (resp. $|\vec{P}|$). We adopt
the following useful abbreviations.

\begin{mathpar}
   x?(\vec{y}).P := x.(\vec{y})P \and  x\clift{\vec{P}} := x.\clift{\vec{P}}
   \and x!(y) := \lift{x}{\dropn{y}}
   \and \Pi_{i=0}^{n-1}P_i := P_0 | \ldots | P_{n-1}
\end{mathpar}

\subsubsection{Structural congruence}

\paragraph{Free and bound names and alpha-equivalence.} At the
core of structural equivalence is alpha-equivalence which identifies
process that are the same up to a change of variable. Formally, we
recognize the distinction between free and bound names. The free names
of a process, $\freenames{P}$, may be calculated recursively as
follows:

\begin{mathpar}
\freenames{\pzero} := \emptyset
  \and \\
  \freenames{x?(y).P} := \{ x \} \cup (\freenames{P} \setminus \{ y \})
  \and 
  \freenames{x!\langle P \rangle} := \{ x \} \cup \{ P \} 
  \and \\
  \freenames{P|Q} := \freenames{P} \cup \freenames{Q}
  \and \\
  \freenames{@{x}} := \{ x \}
\end{mathpar}

$\pi$
$\quotep{\pi}$

$\freenames{-} : \pi \to \mathcal{P}(\quotep{\pi})$

\begin{eqnarray*}
  \freenames{\pzero} & := & \emptyset \\
  \freenames{x?(y).P} & := & \{ x \} \cup (\freenames{P} \setminus \{ y \}) \\
  \freenames{x!\langle P \rangle} & := & \{ x \} \cup \{ P \} \\
  \freenames{P|Q} & := & \freenames{P} \cup \freenames{Q} \\
  \freenames{\dropn{x}} & := & \{ x \}
\end{eqnarray*}

The bound names of a process, $\boundnames{P}$, are those names occurring in $P$
that are not free. For example, in $x?(y).0$, the name $x$ is free, while $y$ is bound.

\begin{mathpar}
  \inferrule* [lab=monoidal-laws] {} { P|Q \equiv Q|P \and P|0 \equiv P \and P|(Q|R) \equiv (P|Q)|R }
\end{mathpar}

\begin{mathpar}
  \inferrule* [lab=alpha-equivalence] {} { (x)P \equiv (y)P\{y/x\} \and y \not\in \freenames{P} }
\end{mathpar}

\begin{definition}
Then two processes, $P,Q$, are alpha-equivalent if $P = Q\{\vec{y}/\vec{x}\}$ for
some $\vec{x} \in \boundnames{Q},\vec{y} \in \boundnames{P}$, where $Q\{\vec{y}/\vec{x}\}$
denotes the capture-avoiding substitution of $\vec{y}$ for $\vec{x}$ in $Q$.
\end{definition}

\begin{definition}
  The {\em structural congruence} \cite{SangiorgiWalker} , $\equiv$,
  between processes is the least congruence containing
  alpha-equivalence, satisfying the abelian monoid laws
  (associativity, commutativity and $\pzero$ as identity) for parallel
  composition $|$ and for summation $+$.
\end{definition}

\subsection{Name equivalence}

We take name equivalence, written $\nameeq$, to be the smallest
equivalence relation generated by the following rules.

\begin{mathpar}
\inferrule*[lab=Quote-drop]
{ }
{ \quotep{@{x}} \nameeq x }

\inferrule*[lab=Struct-equiv]
{ P \scong Q }
{ \quotep{P} \nameeq \quotep{Q} }
\end{mathpar}

The astute reader will have noticed that the mutual recursion of names
and processes imposes a mutual recursion on alpha-equivalence and
structural equivalence via name-equivalence. Fortunately, all of this
works out pleasantly and we may calculate in the natural way, free of
concern. The reader interested in the details is referred to the
appendix \ref{appendix:rho_details}.

\subsection{Substitution}

We use $\Proc$ for the set of processes, $\QProc$ for the set of
names, and $\id{\{}\vec{y} / \vec{x} \id{\}}$ to denote partial maps,
$s : \QProc \rightarrow \QProc$. A map, $s$ lifts, uniquely, to a map
on process terms, $\widehat{s} : \Proc \rightarrow \Proc$ by the
following equations.

\begin{mathpar}
  (0) \psubstp{Q}{P} := 0 \\
  (R \juxtap S) \psubstp{Q}{P}
  :=    
  (R)\psubstp{Q}{P} \juxtap (S) \psubstp{Q}{P} \\
  (x?(y).R) \psubstp{Q}{P}    
  :=    
  (x)\substp{Q}{P} (z)\concat( (R \psubstn{z}{y}) \psubstp{Q}{P} ) \\
  (\lift{x}{R}) \psubstp{Q}{P}  
  :=
  \lift{(x)\substp{Q}{P}}{ R \psubstp{Q}{P} } \\
%   (\dropn{x})  \psubstp{Q}{P}       
%   := 
%   \left\{ 
%     \begin{array}{ccc} 
%       \dropn{\quotep{Q}} & & x \nameeq \quotep{P} \\
%       \dropn{x} & & otherwise \\
%     \end{array}
%   \right. 
  (\dropn{x})  \psubstp{Q}{P}       
  := 
  \left\{ 
    \begin{array}{ccc} 
      Q & & x \nameeq \quotep{P} \\
      \dropn{x} & & otherwise \\
    \end{array}
  \right.
\end{mathpar}
 

where

\begin{eqnarray}
  (x)\id{\{} \lpquote Q \rpquote / \lpquote P \rpquote \id{\}}            = 
  \left\{ 
    \begin{array}{ccc}
      \lpquote Q \rpquote & & x \nameeq \lpquote P \rpquote \\
      x & & otherwise \\
    \end{array}
  \right. \nonumber
\end{eqnarray}

and $z$ is chosen distinct from $\quotep{P}$, $\quotep{Q}$, the free
names in $Q$, and all the names in $R$. Our $\alpha$-equivalence will
be built in the standard way from this substitution.

\begin{remark}\label{rem:no_self_referential_names}
  One consequence of these definitions is that $\forall P. \quotep{P}
  \not\in \freenames{P}$.
\end{remark}

\subsection{ Dynamic quote: an example }

Anticipating something of what's to come, consider applying the
substitution, $\widehat{\id{\{}u / z \id{\}}}$, to the following pair
of processes, $\lift{w}{y!(z)}$ and $w[ \lpquote y!(z) \rpquote ]$.

\begin{eqnarray}
	\lift{w}{y!(z)}\widehat{\id{\{}u / z \id{\}}}
		& = &
		\lift{w}{y!(u)} \nonumber\\
	w[ \lpquote y!(z) \rpquote ] \widehat{ \id{\{}u / z \id{\}} }
		& = &
		w[ \lpquote y!(z) \rpquote ] \nonumber
\end{eqnarray}

Because the body of the process between quotes is impervious to
substitution, we get radically different answers. In fact, by
examining the first process in an input context,
e.g. $x?(z).\lift{w}{y!(z)}$, we see that the process under the lift
operator may be shaped by prefixed inputs binding a name inside it. In
this sense, the lift operator will be seen as a way to dynamically
construct processes before reifying them as names.

Finally equipped with these standard features we can present the
dynamics of the calculus.

\subsubsection{Operational semantics} 

Finally, we introduce the computational dynamics. What marks these
algebras as distinct from other more traditionally studied algebraic
structures, e.g. vector spaces or polynomial rings, is the manner in
which dynamics is captured. In traditional structures, dynamics is typically
expressed through morphisms between such structures, as in linear maps
between vector spaces or morphisms between rings. In algebras
associated with the semantics of computation, the dynamics is
expressed as part of the algebraic structure itself, through a
reduction reduction relation typically denoted by $\red$. Below, we
give a recursive presentation of this relation for the calculus used
in the encoding.

$\red \subseteq \pi \times \pi$
$\red : \pi \to \mathcal{P}(\pi)$

\begin{mathpar}
  \inferrule* [lab=Comm] { \textsf{match}( x_{src}, x_{trgt} ) } { x_{trgt}?(y)P \; | \; x_{src}!\langle {Q} \rangle \red P\{\quotep{Q}/y}\} }
  \and \\
  \inferrule* [lab=Par] {{P} \red {P}'} {{{P} | {Q}} \red {{P}' | {Q}}}
  \and
  \inferrule* [lab=Equiv]{{{P} \scong {P}'} \andalso {{P}' \red {Q}'} \andalso {{Q}' \scong {Q}}}{{P} \red {Q}}
\end{mathpar}

\begin{eqnarray*}
  match_{\equiv} (\quotep{P},\quotep{Q}) & := & P \equiv Q \\
  match_{\dagger}(\quotep{P},\quotep{Q}) & := & \forall R. P|Q \red^{*} R => R \red^{*} 0 \\
  match_{K}(\quotep{P},\quotep{Q}) & := & K \mbox{ for some context } K
\end{eqnarray*}

$u?(x)P | u!\langle Q \rangle \red P\{\quotep{Q}/x\}$

%We write $\wred$ for $\red^*$, and $P\red$ if $\exists Q $ such that $ P \red Q$.
We write $P\red$ if $\exists Q $ such that $ P \red Q$ and $P\not\red$, otherwise.

\section{Replication}

As mentioned before, it is known that replication (and hence
recursion) can be implemented in a higher-order process algebra
\cite{SangiorgiWalker}. As our first example of calculation with the
machinery thus far presented we give the construction explicitly in
the {\rhoc}.

\begin{eqnarray}
	D_{x} & := & \prefix{x}{y}{(\binpar{\outputp{x}{y}}{@{y}})} \nonumber\\
	\bangp_{x}{P} & := & \binpar{{x}!\langle{\binpar{D_{x}}{P}}\rangle}{D_{x}} \nonumber
\end{eqnarray}

\begin{eqnarray}
	\bangp_{x}{P} & & \nonumber\\
	=
	& {x}!\langle{(\prefix{x}{y}{(\outputp{x}{y} | @{y})) | P}}\rangle 
	      | \prefix{x}{y}{(\outputp{x}{y} | @{y})} & \nonumber\\
	\red
	& (\outputp{x}{y} | @{y})\substn{\quotep{(\prefix{x}{y}{(@{y} | \outputp{x}{y})) | P}}}{y} & \nonumber\\
	=
	& \outputp{x}{\quotep{(\prefix{x}{y}{(\outputp{x}{y} | @{y})) | P}}}
	  | {(\prefix{x}{y}{(\outputp{x}{y} | @{y})) | P}} & \nonumber\\
	\red
	& \ldots & \nonumber\\
	\red^*
	& P | P | \ldots & \nonumber
\end{eqnarray}

Of course, this encoding, as an implementation, runs away, unfolding
$\bangp{P}$ eagerly. A lazier and more implementable replication
operator, restricted to input-guarded processes, may be obtained as follows.

\begin{eqnarray}
\bangp{\prefix{u}{v}{P}} 
	:= 
	\binpar{\lift{x}{\prefix{u}{v}{(\binpar{D(x)}{P})}}}{D(x)} \nonumber
\end{eqnarray}

\begin{remark}
  Note that the lazier definition still does not deal with summation
  or mixed summation (i.e. sums over input and output). The reader is
  invited to construct definitions of replication that deal with these
  features. 

  Further, the definitions are parameterized in a name, $x$. Can you,
  gentle reader, make a definition that eliminates this parameter and
  guarantees no accidental interaction between the replication
  machinery and the process being replicated -- i.e. no accidental
  sharing of names used by the process to get its work done and the
  name(s) used by the replication to effect copying. This latter
  revision of the definition of replication is crucial to obtaining
  the expected identity $!!P \sim !P$.
\end{remark}

\begin{remark}\label{rem:paradoxical_combinator}
  The reader familiar with the lambda calculus will have noticed the
  similarity between $D$ and the paradoxical combinator.

  [Ed. note: the existence of this seems to suggest we have to be more
  restrictive on the set of processes and names we admit if we are to
  support no-cloning.]
\end{remark}

\subsubsection{Bisimulation}

The computational dynamics gives rise to another kind of equivalence,
the equivalence of computational behavior. As previously mentioned
this is typically captured \emph{via} some form of bisimulation.

% The notion we use in this paper is weak barbed bisimulation
% \cite{milner91polyadicpi}.

The notion we use in this paper is derived from weak barbed
bisimulation \cite{milner91polyadicpi}. 

\begin{definition}
An \emph{observation relation}, $\downarrow_{\mathcal N}$, over a set
of names, $\mathcal N$, is the smallest relation satisfying the rules
below.

\infrule[Out-barb]{y \in {\mathcal N}, \; x \nameeq y}
		  {\outputp{x}{v} \downarrow_{\mathcal N} x}
\infrule[Par-barb]{\mbox{$P\downarrow_{\mathcal N} x$ or $Q\downarrow_{\mathcal N} x$}}
		  {\binpar{P}{Q} \downarrow_{\mathcal N} x}

We write $P \Downarrow_{\mathcal N} x$ if there is $Q$ such that 
$P \wred Q$ and $Q \downarrow_{\mathcal N} x$.
\end{definition}

\begin{definition}
%\label{def.bbisim}
An  ${\mathcal N}$-\emph{barbed bisimulation} over a set of names, ${\mathcal N}$, is a symmetric binary relation 
${\mathcal S}_{\mathcal N}$ between agents such that $P\rel{S}_{\mathcal N}Q$ implies:
\begin{enumerate}
\item If $P \red P'$ then $Q \wred Q'$ and $P'\rel{S}_{\mathcal N} Q'$.
\item If $P\downarrow_{\mathcal N} x$, then $Q\Downarrow_{\mathcal N} x$.
\end{enumerate}
$P$ is ${\mathcal N}$-barbed bisimilar to $Q$, written
$P \wbbisim_{\mathcal N} Q$, if $P \rel{S}_{\mathcal N} Q$ for some ${\mathcal N}$-barbed bisimulation ${\mathcal S}_{\mathcal N}$.
\end{definition}

$\mathcal{R} \subseteq \pi \times \pi$

$P \mathcal{R} Q => \forall P'. P \red P' \Rightarrow \exists Q'. Q \red Q', P' \mathcal{R} Q'$

$P \vdash x \Rightarrow Q \vdash x$

\begin{mathpar}
  \inferrule*[lab=Out-barb]{x \nameeq y}{{y}!\langle{Q}\rangle \vdash x}
  \and
  \inferrule*[lab=Par-barb]{\mbox{$P\vdash x$ or $Q\vdash x$}}{\binpar{P}{Q} \vdash x}
\end{mathpar}

\subsubsection{Contexts}

One of the principle advantages of computational calculi like the
$\pi$-calculus is a well-defined notion of context,
contextual-equivalence and a correlation between
contextual-equivalence and notions of bisimulation. The notion of
context allows the decomposition of a process into (sub-)process and
its syntactic environment, its context. Thus, a context may be
thought of as a process with a ``hole'' (written $\Box$) in it. The
application of a context $M$ to a process $P$, written $M[P]$, is
tantamount to filling the hole in $M$ with $P$. In this paper we do
not need the full weight of this theory, but do make use of the notion
of context in the proof the main theorem. 

\begin{mathpar}
  \inferrule* [lab=summation] {} {{M_{M},M_{N}} \bc \Box \;|\; x.M_{A} \;|\; M_{M}+M_{N}}
  \and
  \inferrule* [lab=agent] {} {{M_{A}} \bc (\vec{x})M_{P} \;| \; \clift{P_0,\ldots,M_{P},\ldots,P_N}}
  \and \\
  \inferrule* [lab=process] {} {{M_{P}} \bc M_{N} \;| \;P|M_{P} }
\end{mathpar} 

\begin{mathpar}
  \inferrule* [lab=sychronization] {} {M_{N} \bc \Box \;|\; x?M_{F} \;|\; x!M_{C}}
  \and
  \inferrule* [lab=abstraction] {} {{M_{F}} \bc (x)M_{P} }
  \and
  \inferrule* [lab=concretion] {} {{M_{C}} \bc \langle M_{P} \rangle }
  \and \\
  \inferrule* [lab=process] {} {{M_{P}} \bc M_{N} \;| \;P|M_{P} }
\end{mathpar}

\begin{definition}[contextual application] Given a context $M$, and
  process $P$, we define the \emph{contextual application}, $M[P] :=
  M\{P/\Box\}$. That is, the contextual application of M to P is the
  substitution of $P$ for $\Box$ in $M$.
\end{definition}

$\meaningof{-} : L \to \mathcal{P}(\pi)$

\begin{mathpar}
  \inferrule* [lab=collection] {} {\meaningof{true} = \pi, \and \meaningof{~E} = \pi \setminus \meaningof{E}, \and \meaningof{E_{1} \& E_{2}} = \meaningof{E_{1}} \cap \meaningof{E_{2}}}
\end{mathpar}

\begin{mathpar}
  \inferrule* [lab=structure] {} {\meaningof{0} = \{ P \in \pi | P \equiv 0 \}, \and \\ \meaningof{E_1 | E_2} = \{ P \in \pi | P \equiv P_{1} | P_{2}, P_{1} \in \meaningof{E_{1}}, P_{2} \in \meaningof{E_2}\} }
\end{mathpar}

\begin{mathpar}
 \inferrule* [lab=behavior] {} {\meaningof{\langle a?b \rangle E} = \{ P \in \pi | P \equiv Q | u?(y)P', \\ \and \\\\ \and \\ \;\;\; u \in \meaningof{a}, \forall z.P'\{z/y\} \in \meaningof{E\{z/b\}}\}, \and \\ \meaningof{a!E} = \{ P \in \pi | P \equiv Q | x!\langle P' \rangle, x \in \meaningof{a} P' \in \meaningof{E}\} }
\end{mathpar}

\begin{mathpar}
 \inferrule* [lab=nominal] {} {\meaningof{\quotep{E}} = \{ \quotep{P} \in \quotep{\pi} | P \in \meaningof{E} \}, \and \meaningof{\quotep{P}} = \{ \quotep{Q} \in \quotep{\pi} | P \equiv Q \} \and \\ \meaningof{@\quotep{E}} = \{ P \in \pi | P \equiv @x, x \in \meaningof{E} \}}
\end{mathpar}

\begin{eqnarray*}
  \\
  \meaningof{-} : TS \to ST
\end{eqnarray*}

\begin{eqnarray*}
  \\
  L : TS \to ST
\end{eqnarray*}

\begin{eqnarray*}
  \\
  P \models E \iff P \in \meaningof{E}
\end{eqnarray*}

\begin{eqnarray*}
  P \approx_{L} Q \iff \forall E \in L. P \models E \iff Q \models E
\end{eqnarray*}

\begin{eqnarray*}
  P \approx_{K} Q
\end{eqnarray*}

\begin{eqnarray*}
  P \approx Q
\end{eqnarray*}

$\approx_{K} = \approx = \approx_{L}$

\subsubsection{Contextual duality}

Note that contexts extend the quotation operation to a family of
operations from processes to names. Given a context, $M$, we can
define a \emph{nominal context}, $\quotep{M}$ by $\quotep{M}[P] :=
\quotep{M[P]}$. To foreshadow what is to come we observe that these
operations enjoy a duality with processes very much like the duality
between vectors and maps from vectors to scalars.

Further, because the calculus is essentially higher-order, we have a
correspondence between contexts and processes. More specifically,
given a name $x$ and a context $M$ we can construct $M^{*}_{x}$ such
that 

\begin{mathpar}
  M^{*}_{x} | \lift{x}{P} \red M[P]
\end{mathpar}

namely,

\begin{mathpar}
  M^{*}_{x} := x?(u).M[\dropn{u}]
\end{mathpar}

The dependence of $M^{*}_{x}$ on a name makes it an abstraction, 

\begin{mathpar}
  M^{*} := (x)x?(u).M[\dropn{u}]
\end{mathpar}

\subsection{Additional notation}

It will sometimes be convenient to denote the process a name
quotes. We already have the notation $x = \quotep{P}$, but it will be
convenient to introduce an alternate notation, $\procn{x}$, when we
want to emphasize the connection to the use of the name. Note that, by
virtue of name equivalence, $\quotep{\procn{x}} \nameeq x$; so, the
notation is consistent with previous definitions.

Further, because names have structure it is possible to effect
substitutions on the basis of that structure. This means we need to
upgrade our notation for substitutions, which we accomplish by
adapting comprehension notation. Thus,

\begin{mathpar}
  P\{ y / x : x \in S \}
\end{mathpar}

is interpreted to mean the process derived from P by replacing (in a
capture-avoiding manner) each occurrence of $x$ in $S$ by $y$. For example,

\begin{mathpar}
  P\{ \quotep{\procn{x}|\procn{x}} / x : x \in \freenames{P} \}
\end{mathpar}

will replace each (occurrence) of a free name $x$ in $P$ by
$\quotep{\procn{x}|\procn{x}}$.

Also, we will avail ourselves of the notation $x^{L}$ and $x^{R}$ to
denote injections of a name into disjoint copies of the name
space. There are numerous ways to accomplish this. One example can be
found in \cite{MeredithR05}. This notation overloads to vectors of
names: $\vec{x}^{\pi} := (x_{i}^{\pi} \; : \; 0 \leq i < |\vec{x}| )$ where $\pi \in \{L,R\}$.

We also use $P^{\Box} := P|\Box$.

In \cite{MeredithR05} an interpretation of the new operator is
given. It turns out that there are several possible interpretations
all enjoying the requisite algebraic properties of the operator (see
\cite{milner91polyadicpi}). We will therefore make liberal use of
$(\nu\; \vec{x})P$.

% subsection the_syntax_and_semantics_of_the_notation_system (end)   

\input{qm2pi.qmops} 

\input{qm2pi.sterngerlach} 

\input{qm2pi.metric} 

% section concurrent_process_calculi (end)

%\input{qm2pi.proofsketch}

% section proof sketch (end)

%\input{qm2pi.slviaknots} 

% section spatial logic via knots (end)

\input{qm2pi.conclusion}

% section conclusion (end)

%\input{qm2pi.dtcodes} 

% section wiring algorithm (end)

\input{qm2pi.ack} 

% section acknowledgments (end)

\newpage


\bibliographystyle{plain}   
\bibliography{../../biblios/main.bib}

\input{qm2pi.rhodetails}

\end{document}

 

% section concurrent_process_calculi (end)

%\documentclass[12pt]{llncs}
%\documentclass{jktr}

\usepackage[pdftex]{hyperref}                   
\usepackage {listings}
\usepackage {mathpartir}
\usepackage{bcprules}
%\usepackage{listings}
                       
\usepackage{graphicx} 
%\usepackage[margins=2.5cm,nohead,nofoot]{geometry}
%\usepackage{geometry}
\usepackage{amsfonts}
\usepackage{amstext}
\usepackage{latexsym}
\usepackage{amssymb}
\usepackage{color}


%\include{myPreamble}
\include{qm2pi.local} 

%\ifpdf
%\usepackage[pdftex]{graphicx}
%\else
%\usepackage{graphicx}
%\fi

 % \ifpdf
%  \usepackage{pdfsync}
%  \if


%\title{Brief Article}
%\author{David F. Snyder}
%\author{L.G. Meredith}

%\address{Dept. of Math., Texas State University--San Marcos, San Marcos, TX 78666}
       
\pagestyle{empty}


\begin{document}

\lstset{language=[Objective]Caml,frame=shadowbox}

\input{qm2pi.front}

% section front matter (end)

\input{qm2pi.intro} 
 
% section introduction (end)

% \input{qm2pi.knotations} 

% section notation (end)

\input{qm2pi.process.calculi} 

% section concurrent_process_calculi_and_spatial_logics_ (end)
    
%\input{qm2pi.knots2pi} 

%\input{qm2pi.trefoil} 

%\input{qm2pi.mainthm} 

% subsection basic_interpretation (end)

%\input{qm2pi.rho.presentation} 
\subsection{The syntax and semantics of the notation system}\label{sub:the_syntax_and_semantics_of_the_notation_system} % (fold)

We now summarize a technical presentation of the calculus that
embodies our theory of dynamics. The typical presentation of such a
calculus follows the style of giving generators and relations on
them. The grammar, below, describing term constructors, freely
generates the set of processes, $\Proc$. This set is then quotiented
by a relation known as structural congruence and it is over this set
that the notion of dynamics is expressed. This presentation is
essentially that of \cite{MeredithR05} with the addition of
polyadicity and summation. For readability we have relegated some of
the technical subtleties to an appendix.

\subsubsection{Process grammar}\label{subsub:process_grammar}

\begin{mathpar}
  \inferrule* [lab=synchronization] {} {{M} \bc \pzero \;|\; x?F \;|\; x!C }
  \and
  \inferrule* [lab=abstraction] {} {{F} \bc (x)P}
  \and
  \inferrule* [lab=concretion] {} {{C} \bc \langle Q \rangle}
  \and
  \inferrule* [lab=process] {} {{P,Q} \bc M \;| \;P|Q \;|\; @{x}}
  \and
  \inferrule* [lab=name] {} {{x} \bc \quotep{P}}
\end{mathpar} 

Note that $\vec{x}$ (resp. $\vec{P}$) denotes a vector of names
(resp. processes) of length $|\vec{x}|$ (resp. $|\vec{P}|$). We adopt
the following useful abbreviations.

\begin{mathpar}
   x?(\vec{y}).P := x.(\vec{y})P \and  x\clift{\vec{P}} := x.\clift{\vec{P}}
   \and x!(y) := \lift{x}{\dropn{y}}
   \and \Pi_{i=0}^{n-1}P_i := P_0 | \ldots | P_{n-1}
\end{mathpar}

\subsubsection{Structural congruence}

\paragraph{Free and bound names and alpha-equivalence.} At the
core of structural equivalence is alpha-equivalence which identifies
process that are the same up to a change of variable. Formally, we
recognize the distinction between free and bound names. The free names
of a process, $\freenames{P}$, may be calculated recursively as
follows:

\begin{mathpar}
\freenames{\pzero} := \emptyset
  \and \\
  \freenames{x?(y).P} := \{ x \} \cup (\freenames{P} \setminus \{ y \})
  \and 
  \freenames{x!\langle P \rangle} := \{ x \} \cup \{ P \} 
  \and \\
  \freenames{P|Q} := \freenames{P} \cup \freenames{Q}
  \and \\
  \freenames{@{x}} := \{ x \}
\end{mathpar}

$\pi$
$\quotep{\pi}$

$\freenames{-} : \pi \to \mathcal{P}(\quotep{\pi})$

\begin{eqnarray*}
  \freenames{\pzero} & := & \emptyset \\
  \freenames{x?(y).P} & := & \{ x \} \cup (\freenames{P} \setminus \{ y \}) \\
  \freenames{x!\langle P \rangle} & := & \{ x \} \cup \{ P \} \\
  \freenames{P|Q} & := & \freenames{P} \cup \freenames{Q} \\
  \freenames{\dropn{x}} & := & \{ x \}
\end{eqnarray*}

The bound names of a process, $\boundnames{P}$, are those names occurring in $P$
that are not free. For example, in $x?(y).0$, the name $x$ is free, while $y$ is bound.

\begin{mathpar}
  \inferrule* [lab=monoidal-laws] {} { P|Q \equiv Q|P \and P|0 \equiv P \and P|(Q|R) \equiv (P|Q)|R }
\end{mathpar}

\begin{mathpar}
  \inferrule* [lab=alpha-equivalence] {} { (x)P \equiv (y)P\{y/x\} \and y \not\in \freenames{P} }
\end{mathpar}

\begin{definition}
Then two processes, $P,Q$, are alpha-equivalent if $P = Q\{\vec{y}/\vec{x}\}$ for
some $\vec{x} \in \boundnames{Q},\vec{y} \in \boundnames{P}$, where $Q\{\vec{y}/\vec{x}\}$
denotes the capture-avoiding substitution of $\vec{y}$ for $\vec{x}$ in $Q$.
\end{definition}

\begin{definition}
  The {\em structural congruence} \cite{SangiorgiWalker} , $\equiv$,
  between processes is the least congruence containing
  alpha-equivalence, satisfying the abelian monoid laws
  (associativity, commutativity and $\pzero$ as identity) for parallel
  composition $|$ and for summation $+$.
\end{definition}

\subsection{Name equivalence}

We take name equivalence, written $\nameeq$, to be the smallest
equivalence relation generated by the following rules.

\begin{mathpar}
\inferrule*[lab=Quote-drop]
{ }
{ \quotep{@{x}} \nameeq x }

\inferrule*[lab=Struct-equiv]
{ P \scong Q }
{ \quotep{P} \nameeq \quotep{Q} }
\end{mathpar}

The astute reader will have noticed that the mutual recursion of names
and processes imposes a mutual recursion on alpha-equivalence and
structural equivalence via name-equivalence. Fortunately, all of this
works out pleasantly and we may calculate in the natural way, free of
concern. The reader interested in the details is referred to the
appendix \ref{appendix:rho_details}.

\subsection{Substitution}

We use $\Proc$ for the set of processes, $\QProc$ for the set of
names, and $\id{\{}\vec{y} / \vec{x} \id{\}}$ to denote partial maps,
$s : \QProc \rightarrow \QProc$. A map, $s$ lifts, uniquely, to a map
on process terms, $\widehat{s} : \Proc \rightarrow \Proc$ by the
following equations.

\begin{mathpar}
  (0) \psubstp{Q}{P} := 0 \\
  (R \juxtap S) \psubstp{Q}{P}
  :=    
  (R)\psubstp{Q}{P} \juxtap (S) \psubstp{Q}{P} \\
  (x?(y).R) \psubstp{Q}{P}    
  :=    
  (x)\substp{Q}{P} (z)\concat( (R \psubstn{z}{y}) \psubstp{Q}{P} ) \\
  (\lift{x}{R}) \psubstp{Q}{P}  
  :=
  \lift{(x)\substp{Q}{P}}{ R \psubstp{Q}{P} } \\
%   (\dropn{x})  \psubstp{Q}{P}       
%   := 
%   \left\{ 
%     \begin{array}{ccc} 
%       \dropn{\quotep{Q}} & & x \nameeq \quotep{P} \\
%       \dropn{x} & & otherwise \\
%     \end{array}
%   \right. 
  (\dropn{x})  \psubstp{Q}{P}       
  := 
  \left\{ 
    \begin{array}{ccc} 
      Q & & x \nameeq \quotep{P} \\
      \dropn{x} & & otherwise \\
    \end{array}
  \right.
\end{mathpar}
 

where

\begin{eqnarray}
  (x)\id{\{} \lpquote Q \rpquote / \lpquote P \rpquote \id{\}}            = 
  \left\{ 
    \begin{array}{ccc}
      \lpquote Q \rpquote & & x \nameeq \lpquote P \rpquote \\
      x & & otherwise \\
    \end{array}
  \right. \nonumber
\end{eqnarray}

and $z$ is chosen distinct from $\quotep{P}$, $\quotep{Q}$, the free
names in $Q$, and all the names in $R$. Our $\alpha$-equivalence will
be built in the standard way from this substitution.

\begin{remark}\label{rem:no_self_referential_names}
  One consequence of these definitions is that $\forall P. \quotep{P}
  \not\in \freenames{P}$.
\end{remark}

\subsection{ Dynamic quote: an example }

Anticipating something of what's to come, consider applying the
substitution, $\widehat{\id{\{}u / z \id{\}}}$, to the following pair
of processes, $\lift{w}{y!(z)}$ and $w[ \lpquote y!(z) \rpquote ]$.

\begin{eqnarray}
	\lift{w}{y!(z)}\widehat{\id{\{}u / z \id{\}}}
		& = &
		\lift{w}{y!(u)} \nonumber\\
	w[ \lpquote y!(z) \rpquote ] \widehat{ \id{\{}u / z \id{\}} }
		& = &
		w[ \lpquote y!(z) \rpquote ] \nonumber
\end{eqnarray}

Because the body of the process between quotes is impervious to
substitution, we get radically different answers. In fact, by
examining the first process in an input context,
e.g. $x?(z).\lift{w}{y!(z)}$, we see that the process under the lift
operator may be shaped by prefixed inputs binding a name inside it. In
this sense, the lift operator will be seen as a way to dynamically
construct processes before reifying them as names.

Finally equipped with these standard features we can present the
dynamics of the calculus.

\subsubsection{Operational semantics} 

Finally, we introduce the computational dynamics. What marks these
algebras as distinct from other more traditionally studied algebraic
structures, e.g. vector spaces or polynomial rings, is the manner in
which dynamics is captured. In traditional structures, dynamics is typically
expressed through morphisms between such structures, as in linear maps
between vector spaces or morphisms between rings. In algebras
associated with the semantics of computation, the dynamics is
expressed as part of the algebraic structure itself, through a
reduction reduction relation typically denoted by $\red$. Below, we
give a recursive presentation of this relation for the calculus used
in the encoding.

$\red \subseteq \pi \times \pi$
$\red : \pi \to \mathcal{P}(\pi)$

\begin{mathpar}
  \inferrule* [lab=Comm] { \textsf{match}( x_{src}, x_{trgt} ) } { x_{trgt}?(y)P \; | \; x_{src}!\langle {Q} \rangle \red P\{\quotep{Q}/y}\} }
  \and \\
  \inferrule* [lab=Par] {{P} \red {P}'} {{{P} | {Q}} \red {{P}' | {Q}}}
  \and
  \inferrule* [lab=Equiv]{{{P} \scong {P}'} \andalso {{P}' \red {Q}'} \andalso {{Q}' \scong {Q}}}{{P} \red {Q}}
\end{mathpar}

\begin{eqnarray*}
  match_{\equiv} (\quotep{P},\quotep{Q}) & := & P \equiv Q \\
  match_{\dagger}(\quotep{P},\quotep{Q}) & := & \forall R. P|Q \red^{*} R => R \red^{*} 0 \\
  match_{K}(\quotep{P},\quotep{Q}) & := & K \mbox{ for some context } K
\end{eqnarray*}

$u?(x)P | u!\langle Q \rangle \red P\{\quotep{Q}/x\}$

%We write $\wred$ for $\red^*$, and $P\red$ if $\exists Q $ such that $ P \red Q$.
We write $P\red$ if $\exists Q $ such that $ P \red Q$ and $P\not\red$, otherwise.

\section{Replication}

As mentioned before, it is known that replication (and hence
recursion) can be implemented in a higher-order process algebra
\cite{SangiorgiWalker}. As our first example of calculation with the
machinery thus far presented we give the construction explicitly in
the {\rhoc}.

\begin{eqnarray}
	D_{x} & := & \prefix{x}{y}{(\binpar{\outputp{x}{y}}{@{y}})} \nonumber\\
	\bangp_{x}{P} & := & \binpar{{x}!\langle{\binpar{D_{x}}{P}}\rangle}{D_{x}} \nonumber
\end{eqnarray}

\begin{eqnarray}
	\bangp_{x}{P} & & \nonumber\\
	=
	& {x}!\langle{(\prefix{x}{y}{(\outputp{x}{y} | @{y})) | P}}\rangle 
	      | \prefix{x}{y}{(\outputp{x}{y} | @{y})} & \nonumber\\
	\red
	& (\outputp{x}{y} | @{y})\substn{\quotep{(\prefix{x}{y}{(@{y} | \outputp{x}{y})) | P}}}{y} & \nonumber\\
	=
	& \outputp{x}{\quotep{(\prefix{x}{y}{(\outputp{x}{y} | @{y})) | P}}}
	  | {(\prefix{x}{y}{(\outputp{x}{y} | @{y})) | P}} & \nonumber\\
	\red
	& \ldots & \nonumber\\
	\red^*
	& P | P | \ldots & \nonumber
\end{eqnarray}

Of course, this encoding, as an implementation, runs away, unfolding
$\bangp{P}$ eagerly. A lazier and more implementable replication
operator, restricted to input-guarded processes, may be obtained as follows.

\begin{eqnarray}
\bangp{\prefix{u}{v}{P}} 
	:= 
	\binpar{\lift{x}{\prefix{u}{v}{(\binpar{D(x)}{P})}}}{D(x)} \nonumber
\end{eqnarray}

\begin{remark}
  Note that the lazier definition still does not deal with summation
  or mixed summation (i.e. sums over input and output). The reader is
  invited to construct definitions of replication that deal with these
  features. 

  Further, the definitions are parameterized in a name, $x$. Can you,
  gentle reader, make a definition that eliminates this parameter and
  guarantees no accidental interaction between the replication
  machinery and the process being replicated -- i.e. no accidental
  sharing of names used by the process to get its work done and the
  name(s) used by the replication to effect copying. This latter
  revision of the definition of replication is crucial to obtaining
  the expected identity $!!P \sim !P$.
\end{remark}

\begin{remark}\label{rem:paradoxical_combinator}
  The reader familiar with the lambda calculus will have noticed the
  similarity between $D$ and the paradoxical combinator.

  [Ed. note: the existence of this seems to suggest we have to be more
  restrictive on the set of processes and names we admit if we are to
  support no-cloning.]
\end{remark}

\subsubsection{Bisimulation}

The computational dynamics gives rise to another kind of equivalence,
the equivalence of computational behavior. As previously mentioned
this is typically captured \emph{via} some form of bisimulation.

% The notion we use in this paper is weak barbed bisimulation
% \cite{milner91polyadicpi}.

The notion we use in this paper is derived from weak barbed
bisimulation \cite{milner91polyadicpi}. 

\begin{definition}
An \emph{observation relation}, $\downarrow_{\mathcal N}$, over a set
of names, $\mathcal N$, is the smallest relation satisfying the rules
below.

\infrule[Out-barb]{y \in {\mathcal N}, \; x \nameeq y}
		  {\outputp{x}{v} \downarrow_{\mathcal N} x}
\infrule[Par-barb]{\mbox{$P\downarrow_{\mathcal N} x$ or $Q\downarrow_{\mathcal N} x$}}
		  {\binpar{P}{Q} \downarrow_{\mathcal N} x}

We write $P \Downarrow_{\mathcal N} x$ if there is $Q$ such that 
$P \wred Q$ and $Q \downarrow_{\mathcal N} x$.
\end{definition}

\begin{definition}
%\label{def.bbisim}
An  ${\mathcal N}$-\emph{barbed bisimulation} over a set of names, ${\mathcal N}$, is a symmetric binary relation 
${\mathcal S}_{\mathcal N}$ between agents such that $P\rel{S}_{\mathcal N}Q$ implies:
\begin{enumerate}
\item If $P \red P'$ then $Q \wred Q'$ and $P'\rel{S}_{\mathcal N} Q'$.
\item If $P\downarrow_{\mathcal N} x$, then $Q\Downarrow_{\mathcal N} x$.
\end{enumerate}
$P$ is ${\mathcal N}$-barbed bisimilar to $Q$, written
$P \wbbisim_{\mathcal N} Q$, if $P \rel{S}_{\mathcal N} Q$ for some ${\mathcal N}$-barbed bisimulation ${\mathcal S}_{\mathcal N}$.
\end{definition}

$\mathcal{R} \subseteq \pi \times \pi$

$P \mathcal{R} Q => \forall P'. P \red P' \Rightarrow \exists Q'. Q \red Q', P' \mathcal{R} Q'$

$P \vdash x \Rightarrow Q \vdash x$

\begin{mathpar}
  \inferrule*[lab=Out-barb]{x \nameeq y}{{y}!\langle{Q}\rangle \vdash x}
  \and
  \inferrule*[lab=Par-barb]{\mbox{$P\vdash x$ or $Q\vdash x$}}{\binpar{P}{Q} \vdash x}
\end{mathpar}

\subsubsection{Contexts}

One of the principle advantages of computational calculi like the
$\pi$-calculus is a well-defined notion of context,
contextual-equivalence and a correlation between
contextual-equivalence and notions of bisimulation. The notion of
context allows the decomposition of a process into (sub-)process and
its syntactic environment, its context. Thus, a context may be
thought of as a process with a ``hole'' (written $\Box$) in it. The
application of a context $M$ to a process $P$, written $M[P]$, is
tantamount to filling the hole in $M$ with $P$. In this paper we do
not need the full weight of this theory, but do make use of the notion
of context in the proof the main theorem. 

\begin{mathpar}
  \inferrule* [lab=summation] {} {{M_{M},M_{N}} \bc \Box \;|\; x.M_{A} \;|\; M_{M}+M_{N}}
  \and
  \inferrule* [lab=agent] {} {{M_{A}} \bc (\vec{x})M_{P} \;| \; \clift{P_0,\ldots,M_{P},\ldots,P_N}}
  \and \\
  \inferrule* [lab=process] {} {{M_{P}} \bc M_{N} \;| \;P|M_{P} }
\end{mathpar} 

\begin{mathpar}
  \inferrule* [lab=sychronization] {} {M_{N} \bc \Box \;|\; x?M_{F} \;|\; x!M_{C}}
  \and
  \inferrule* [lab=abstraction] {} {{M_{F}} \bc (x)M_{P} }
  \and
  \inferrule* [lab=concretion] {} {{M_{C}} \bc \langle M_{P} \rangle }
  \and \\
  \inferrule* [lab=process] {} {{M_{P}} \bc M_{N} \;| \;P|M_{P} }
\end{mathpar}

\begin{definition}[contextual application] Given a context $M$, and
  process $P$, we define the \emph{contextual application}, $M[P] :=
  M\{P/\Box\}$. That is, the contextual application of M to P is the
  substitution of $P$ for $\Box$ in $M$.
\end{definition}

$\meaningof{-} : L \to \mathcal{P}(\pi)$

\begin{mathpar}
  \inferrule* [lab=collection] {} {\meaningof{true} = \pi, \and \meaningof{~E} = \pi \setminus \meaningof{E}, \and \meaningof{E_{1} \& E_{2}} = \meaningof{E_{1}} \cap \meaningof{E_{2}}}
\end{mathpar}

\begin{mathpar}
  \inferrule* [lab=structure] {} {\meaningof{0} = \{ P \in \pi | P \equiv 0 \}, \and \\ \meaningof{E_1 | E_2} = \{ P \in \pi | P \equiv P_{1} | P_{2}, P_{1} \in \meaningof{E_{1}}, P_{2} \in \meaningof{E_2}\} }
\end{mathpar}

\begin{mathpar}
 \inferrule* [lab=behavior] {} {\meaningof{\langle a?b \rangle E} = \{ P \in \pi | P \equiv Q | u?(y)P', \\ \and \\\\ \and \\ \;\;\; u \in \meaningof{a}, \forall z.P'\{z/y\} \in \meaningof{E\{z/b\}}\}, \and \\ \meaningof{a!E} = \{ P \in \pi | P \equiv Q | x!\langle P' \rangle, x \in \meaningof{a} P' \in \meaningof{E}\} }
\end{mathpar}

\begin{mathpar}
 \inferrule* [lab=nominal] {} {\meaningof{\quotep{E}} = \{ \quotep{P} \in \quotep{\pi} | P \in \meaningof{E} \}, \and \meaningof{\quotep{P}} = \{ \quotep{Q} \in \quotep{\pi} | P \equiv Q \} \and \\ \meaningof{@\quotep{E}} = \{ P \in \pi | P \equiv @x, x \in \meaningof{E} \}}
\end{mathpar}

\begin{eqnarray*}
  \\
  \meaningof{-} : TS \to ST
\end{eqnarray*}

\begin{eqnarray*}
  \\
  L : TS \to ST
\end{eqnarray*}

\begin{eqnarray*}
  \\
  P \models E \iff P \in \meaningof{E}
\end{eqnarray*}

\begin{eqnarray*}
  P \approx_{L} Q \iff \forall E \in L. P \models E \iff Q \models E
\end{eqnarray*}

\begin{eqnarray*}
  P \approx_{K} Q
\end{eqnarray*}

\begin{eqnarray*}
  P \approx Q
\end{eqnarray*}

$\approx_{K} = \approx = \approx_{L}$

\subsubsection{Contextual duality}

Note that contexts extend the quotation operation to a family of
operations from processes to names. Given a context, $M$, we can
define a \emph{nominal context}, $\quotep{M}$ by $\quotep{M}[P] :=
\quotep{M[P]}$. To foreshadow what is to come we observe that these
operations enjoy a duality with processes very much like the duality
between vectors and maps from vectors to scalars.

Further, because the calculus is essentially higher-order, we have a
correspondence between contexts and processes. More specifically,
given a name $x$ and a context $M$ we can construct $M^{*}_{x}$ such
that 

\begin{mathpar}
  M^{*}_{x} | \lift{x}{P} \red M[P]
\end{mathpar}

namely,

\begin{mathpar}
  M^{*}_{x} := x?(u).M[\dropn{u}]
\end{mathpar}

The dependence of $M^{*}_{x}$ on a name makes it an abstraction, 

\begin{mathpar}
  M^{*} := (x)x?(u).M[\dropn{u}]
\end{mathpar}

\subsection{Additional notation}

It will sometimes be convenient to denote the process a name
quotes. We already have the notation $x = \quotep{P}$, but it will be
convenient to introduce an alternate notation, $\procn{x}$, when we
want to emphasize the connection to the use of the name. Note that, by
virtue of name equivalence, $\quotep{\procn{x}} \nameeq x$; so, the
notation is consistent with previous definitions.

Further, because names have structure it is possible to effect
substitutions on the basis of that structure. This means we need to
upgrade our notation for substitutions, which we accomplish by
adapting comprehension notation. Thus,

\begin{mathpar}
  P\{ y / x : x \in S \}
\end{mathpar}

is interpreted to mean the process derived from P by replacing (in a
capture-avoiding manner) each occurrence of $x$ in $S$ by $y$. For example,

\begin{mathpar}
  P\{ \quotep{\procn{x}|\procn{x}} / x : x \in \freenames{P} \}
\end{mathpar}

will replace each (occurrence) of a free name $x$ in $P$ by
$\quotep{\procn{x}|\procn{x}}$.

Also, we will avail ourselves of the notation $x^{L}$ and $x^{R}$ to
denote injections of a name into disjoint copies of the name
space. There are numerous ways to accomplish this. One example can be
found in \cite{MeredithR05}. This notation overloads to vectors of
names: $\vec{x}^{\pi} := (x_{i}^{\pi} \; : \; 0 \leq i < |\vec{x}| )$ where $\pi \in \{L,R\}$.

We also use $P^{\Box} := P|\Box$.

In \cite{MeredithR05} an interpretation of the new operator is
given. It turns out that there are several possible interpretations
all enjoying the requisite algebraic properties of the operator (see
\cite{milner91polyadicpi}). We will therefore make liberal use of
$(\nu\; \vec{x})P$.

% subsection the_syntax_and_semantics_of_the_notation_system (end)   

\input{qm2pi.qmops} 

\input{qm2pi.sterngerlach} 

\input{qm2pi.metric} 

% section concurrent_process_calculi (end)

%\input{qm2pi.proofsketch}

% section proof sketch (end)

%\input{qm2pi.slviaknots} 

% section spatial logic via knots (end)

\input{qm2pi.conclusion}

% section conclusion (end)

%\input{qm2pi.dtcodes} 

% section wiring algorithm (end)

\input{qm2pi.ack} 

% section acknowledgments (end)

\newpage


\bibliographystyle{plain}   
\bibliography{../../biblios/main.bib}

\input{qm2pi.rhodetails}

\end{document}



% section proof sketch (end)

%\section{Unlikely characters: spatial logic for
  knots}\label{sub:characteristic_formulae} % (fold)

Associated to the mobile process calculi are a family of logics known
as the Hennessy-Milner logics. These logics typically enjoy a
semantics interpreting formulae as sets of processes that when
factored through the encoding outlined above allows an identification
of classes of knots with logical formulae. In the context of this
encoding the sub-family known as the spatial logics \cite{CairesC03}
\cite{CairesC04} \cite{Caires04} are of particular interest providing
several important features for expressing and reasoning about
properties (i.e. classes) of knots. We hint here at how this may be done.

%\begin{description}
%\item [structural connectives] 
\subsubsection{Structural connectives} The spatial logics enjoy
structural connectives corresponding, at the logical level, to the
parallel composition ($P | Q$) and new name ($(\nu \; x)P$)
connectives for processes. As illustrated in the examples below, these
connectives are extremely expressive given the shape of our encoding.
%\item [decideable satisfaction]

\subsubsection{Decideable satisfaction}
In \cite{Caires04} the satisfaction relation is shown to be decideable
for a rich class of processes. It further turns out that the image of
the our encoding is a proper subset of that class. This result
provides the basis for an algorithm by which to search for knots
enjoying a given property.
%\item [characteristic formulae]

\subsubsection{Characteristic formulae}
In the same paper \cite{Caires04} , Caires presents a means of calculating
characteristic formulae, selecting equivalence classes of processes
up to a pre--specified depth limit on the support set of names. Composed with our
encoding, this characteristic formula can be used to select
characteristic formulae for knots.
%\end{description}

\subsubsection{Spatial logic formulae}

The grammar below (segmented for comprehension) summarizes the syntax
of spatial logic formulae. We employ illustrative examples in the
sequel to provide an intuitive understanding of their meaning
referring the reader to \cite{Caires04} for a more detailed explication
of the semantics.

\begin{mathpar}
  \inferrule* [lab=boolean] {} {{A,B} \bc T \;|\; \neg A \;|\; A \wedge B \;|\; \eta = \eta'}
  \and
  \inferrule* [lab=spatial] {} {|\; \pzero \;|\; A | B \;|\; x \text{\textregistered} A \;|\; \forall x . A \;|\;  H x . A}
  \and
  \inferrule* [lab=behavioral] {} {|\; \alpha . A}
  \and 
  \inferrule* [lab=recursion] {} {|\; X(\vec{u}) \;|\; \mu X(\vec{u}) . A}
  \and
  \inferrule* [lab=action] {} {\alpha \bc \langle x?(\vec{y}) \rangle \;|\; \langle x!(\vec{y}) \rangle \;|\; \langle \tau \rangle}
  \and 
  \inferrule* [lab=name] {} {\eta \bc x \;|\; \tau}
\end{mathpar} 

% subsection characteristic_formulae (end)   	 

\subsection{Example formulae}\label{sub:example_formulae_} % (fold)

\subsubsection{Crossing as formula.}
% 
% \begin{align*}
%   \frac{d}{dx} \sin x &= \cos x 
%   & \frac{d}{dx} e^x &= e^x \\
%   \frac{d}{dx} \cos x &= - \sin x 
%   & \frac{d}{dx} \log x &= \frac{1}{x} \\
% \end{align*} 

\begin{align*}
 \mu C(x_{0},x_{1},y_{0},y_{1},u).&(\langle x_{0}?(z) \rangle(\langle u! \rangle\langle y_{1}!z \rangle C(x_{0},x_{1},y_{0},y_{1},u)) & \\
  & \wedge \langle y_{1}?(z) \rangle (\langle u! \rangle \langle x_{0}!z \rangle C(x_{0},x_{1},y_{0},y_{1},u)) & \\
  & \wedge \langle x_{1}?(z) \rangle (\langle u? \rangle \langle y_{0}!z \rangle C(x_{0},x_{1},y_{0},y_{1},u)) & \\
  & \wedge \langle y_{0}?(z) \rangle (\langle u? \rangle \langle x_{1}!z \rangle C(x_{0},x_{1},y_{0},y_{1},u))) &
\end{align*}

The lexicographical similarity between the shape of this formulae and
the shape of definition of the process representing a crossing reveals
the intuitive meaning of this formulae. It describes the capabilities
of a process that has the right to represent a crossing. For example
it picks out processes that may perform an input on the port $x_0$ in
its initial menu of capabilities. What differentiates the formula
from the process, however, is that the crossing process is the
smallest candidate to satisfy the formula. Infinitely many other
processes -- with internal behavior hidden behind this interface, so
to speak -- also satisfy this formula. Even this simple formula,
then, can be seen to open a new view onto knots, providing a
computational interpretation of \emph{virtual} knots.

Note that this formula is derived by hand. A similar formula can be
derived by employing Caires' calculation of characteristic formula
\cite{Caires04} to the process representing a crossing. In light of
this discussion, we let
$\meaningof{C}_{\phi}(x0,x1,y0,y1,u)$ denote a formula specifying the
dynamics we wish to capture of a crossing. To guarantee we preserve
the shape of the interface and minimal semantics we demand that
$\meaningof{C}_{\phi}(x0,x1,y0,y1,u) \Rightarrow
\textbf{C}(x0,x1,y0,y1,u)$ where $\textbf{C}(x0,x1,y0,y1,u)$ denotes
the formula above.
                            
\subsubsection{Crossing number constraints.}
The moral content of the context lemma (Lemma \ref{context}) is that the notion of
``locality'' in the Reidemeister moves is effectively captured by the
parallel composition operator of the process calculus. This intuition
extends through the logic. Given a formula,
$\meaningof{C}_{\phi}(x0,x1,y0,y1,u)$, we can use the structural
connectives to specify constraints on crossing numbers, such as at
least $n$ crossings, or exactly $n$ crossings.
\begin{mathpar}
  \inferrule* [lab=at-least-n] {} { K^{\geq n}_{\phi}(\vec{xs},\vec{ys}) := \Pi_{i=0}^{n-1} Hu . \meaningof{C}_{\phi}(xs_i,ys_i,u) | T }
  \and 
  \inferrule* [lab=exactly-n] {} { K^{= n}_{\phi}(\vec{xs},\vec{ys}) := \Pi_{i=0}^{n-1} Hu . \meaningof{C}_{\phi}(xs_i,ys_i,u) | \neg (\forall x_0,y_0,x_1,y_1,u . \meaningof{C}_{\phi}(x_0,y_0,x_1,y_1,u) | T) }
\end{mathpar}

To round out this section, recall that the encoding of an $n$-crossing
knot decomposes into a parallel composition of $n$ \emph{copies} of a
crossing process together with a wiring harness. To specify different
knot classes with the same crossing number amounts to specifying
logical constraints on the wiring harness. In the interest of space,
we defer examples to a forthcoming paper. Suffice it to say that both
the conditions ``alternating knot'' and ``contains the tangle
corresponding to 5/3'' are expressible. For example, it is possible to
calculate the characteristic formula of a process corresponding to the
tangle 5/3 and conjoin it into the classifying formula via the
composition connective of the logic.

Finally, we wish to observe that it is entirely within reason to
contemplate a more domain-specific version of spatial logic tailored
to the shape of processes in the image of the encoding. Such a
domain-specific logic would have a better claim to the title formal
language of knot properties.

% subsection example_formulae_ (end)

% section knots_as_processes (end) 

% section spatial logic via knots (end)

\section{Conclusions and future work}

\paragraph{Testing physical space}
You, gentle reader, may wonder why of all the theorems to be proved
given this set up we pick the one above. In some sense it's hardly
central to quantum mechanics. We see it as central in the sense that
it firmly establishes a notion of physical space arising from a notion
of the equivalence of behavior. Relating bisimulation to a metric is a
big step forward, but one is faced with interpreting the relationship
of that metric space to something more physical. Quantum mechanical
notions of ``physical'' space are still far from intuitive, but by
relating this idea of distance as testing to calculations that predict
physical circumstances we are making a not insignificant step forward
toward an understanding of the physical space we inhabit as
essentially dynamic.

\paragraph{Effectivity and simulation}
One of the observations we have yet to make is that the entire program
spelled out here is effective. We have built various interpreters for
the reflective calculus at work in this interpretation. In principle,
then, we can simulate quantum mechanics on a computer. The place where
the simulation may lose fidelity is the infinitely branching summation
for the annihilator.

In this connection i also want to point out that the evaluation style
calculation of the inner product puts the non-determinism of the
summation right at the heart of measurement. This suggests that
Milner's original reduction-based formulation of the dynamics of his
calculi in terms of sums was not just notationally suggestive of a
notion of measure-and-continue but captured some significant part of
the physics.

\paragraph{Quantum continuations}
In light of this last observation i want to point out that the
predominant account of quantum mechanics is missing a key aspect of a
truly compositional story of the physical situation. In a real lab,
when a measurement is made the observation can be made to feed into
another device that then makes another measurement conditioned on the
results of the first. This means that after the superposition was
collapsed the entire experimental set up remained in
superposition. While QM offers a means of writing this down it doesn't
quite line up well with the well-trodden formulation of computation
and continuation that we see so succinctly expressed in Milner's
calculi. This suggests that there might be advantages to this account
of dynamics waiting to be explored.

\paragraph{Quantum logic}
In this connection, we also note that by virtue of having the
Hennessy-Milner construction, we can pull the construction through the
interpretation of QM. This gives us a natural candidate for a quantum
logic that enjoys an extremely tight connection with it's domain of
interpretation, making the construction much less ad hoc (rather it is
the image of functor!).

\paragraph{Quantum probabiity}
i have questions about the basis of the interpretation of inner
product as probability amplitude. In particular, using which
axiomatization of probability theory does the notion of probability
amplitude earn the right to be so dubbed? In other words, where is the
proof that the operation for calculating a probability amplitude (and
then squaring) satisfies the axioms of what it means to calculate a
probability? Even if such a proof exists (i have yet to find it in the
literature), i wonder if it might not be possible to turn things on
their heads. Can we view the calculation of the probability amplitude
as an axiomatization of probability? If so, then the definition we
give for calculating probability amplitude may provide the basis for
an \emph{effective} theory of probability.

\paragraph{Quantum vs ``biological'' information}
Finally, i want to conclude with a more philosophical observation. At
a recent workshop in which QM was a predominant topic i noticed
something about quantum information. The speaker was giving a riveting
discussion of axiomatic QM and showing how properties of ``no
cloning'' and ``no deleting'' emerged as consequences of the
axiomatization. Theorems of this form are necessary to give us a sense
of confidence that our axioms characterize the physical theory. What
struck me, though, was that if quantum information is neither erasable
nor replicable it is markedly different from \emph{life}. Two of the
things we know about life is that

\begin{itemize}
  \item it ends;
  \item to gain some measure of persistence, to transcend it's
    finitude it is imminently copyable.
\end{itemize}

Both of these qualities are summarized succinctly in the aphorism: all
flesh is grass. For me these two kinds of ``information'' -- call them
quantum and biological -- are end points on a spectrum of strategies
for persistence. At one end, we have those curious entities that enjoy
uniqueness and permanence; at the other, we have those who in the face
of a certain end and an uncertain present make a go of passing
something on. To me one of the more remarkable aspects of the latter
strategy is that in the presence of noise (and certain features of
copying) we get a kind of dynamism, a chance for improvement against a
given persistent condition.

% subsection other_calculi_other_bisimulations_and_geometry_as_behavior (end)




% section conclusion (end)

%\documentclass[12pt]{llncs}
%\documentclass{jktr}

\usepackage[pdftex]{hyperref}                   
\usepackage {listings}
\usepackage {mathpartir}
\usepackage{bcprules}
%\usepackage{listings}
                       
\usepackage{graphicx} 
%\usepackage[margins=2.5cm,nohead,nofoot]{geometry}
%\usepackage{geometry}
\usepackage{amsfonts}
\usepackage{amstext}
\usepackage{latexsym}
\usepackage{amssymb}
\usepackage{color}


%\include{myPreamble}
\include{qm2pi.local} 

%\ifpdf
%\usepackage[pdftex]{graphicx}
%\else
%\usepackage{graphicx}
%\fi

 % \ifpdf
%  \usepackage{pdfsync}
%  \if


%\title{Brief Article}
%\author{David F. Snyder}
%\author{L.G. Meredith}

%\address{Dept. of Math., Texas State University--San Marcos, San Marcos, TX 78666}
       
\pagestyle{empty}


\begin{document}

\lstset{language=[Objective]Caml,frame=shadowbox}

\input{qm2pi.front}

% section front matter (end)

\input{qm2pi.intro} 
 
% section introduction (end)

% \input{qm2pi.knotations} 

% section notation (end)

\input{qm2pi.process.calculi} 

% section concurrent_process_calculi_and_spatial_logics_ (end)
    
%\input{qm2pi.knots2pi} 

%\input{qm2pi.trefoil} 

%\input{qm2pi.mainthm} 

% subsection basic_interpretation (end)

%\input{qm2pi.rho.presentation} 
\subsection{The syntax and semantics of the notation system}\label{sub:the_syntax_and_semantics_of_the_notation_system} % (fold)

We now summarize a technical presentation of the calculus that
embodies our theory of dynamics. The typical presentation of such a
calculus follows the style of giving generators and relations on
them. The grammar, below, describing term constructors, freely
generates the set of processes, $\Proc$. This set is then quotiented
by a relation known as structural congruence and it is over this set
that the notion of dynamics is expressed. This presentation is
essentially that of \cite{MeredithR05} with the addition of
polyadicity and summation. For readability we have relegated some of
the technical subtleties to an appendix.

\subsubsection{Process grammar}\label{subsub:process_grammar}

\begin{mathpar}
  \inferrule* [lab=synchronization] {} {{M} \bc \pzero \;|\; x?F \;|\; x!C }
  \and
  \inferrule* [lab=abstraction] {} {{F} \bc (x)P}
  \and
  \inferrule* [lab=concretion] {} {{C} \bc \langle Q \rangle}
  \and
  \inferrule* [lab=process] {} {{P,Q} \bc M \;| \;P|Q \;|\; @{x}}
  \and
  \inferrule* [lab=name] {} {{x} \bc \quotep{P}}
\end{mathpar} 

Note that $\vec{x}$ (resp. $\vec{P}$) denotes a vector of names
(resp. processes) of length $|\vec{x}|$ (resp. $|\vec{P}|$). We adopt
the following useful abbreviations.

\begin{mathpar}
   x?(\vec{y}).P := x.(\vec{y})P \and  x\clift{\vec{P}} := x.\clift{\vec{P}}
   \and x!(y) := \lift{x}{\dropn{y}}
   \and \Pi_{i=0}^{n-1}P_i := P_0 | \ldots | P_{n-1}
\end{mathpar}

\subsubsection{Structural congruence}

\paragraph{Free and bound names and alpha-equivalence.} At the
core of structural equivalence is alpha-equivalence which identifies
process that are the same up to a change of variable. Formally, we
recognize the distinction between free and bound names. The free names
of a process, $\freenames{P}$, may be calculated recursively as
follows:

\begin{mathpar}
\freenames{\pzero} := \emptyset
  \and \\
  \freenames{x?(y).P} := \{ x \} \cup (\freenames{P} \setminus \{ y \})
  \and 
  \freenames{x!\langle P \rangle} := \{ x \} \cup \{ P \} 
  \and \\
  \freenames{P|Q} := \freenames{P} \cup \freenames{Q}
  \and \\
  \freenames{@{x}} := \{ x \}
\end{mathpar}

$\pi$
$\quotep{\pi}$

$\freenames{-} : \pi \to \mathcal{P}(\quotep{\pi})$

\begin{eqnarray*}
  \freenames{\pzero} & := & \emptyset \\
  \freenames{x?(y).P} & := & \{ x \} \cup (\freenames{P} \setminus \{ y \}) \\
  \freenames{x!\langle P \rangle} & := & \{ x \} \cup \{ P \} \\
  \freenames{P|Q} & := & \freenames{P} \cup \freenames{Q} \\
  \freenames{\dropn{x}} & := & \{ x \}
\end{eqnarray*}

The bound names of a process, $\boundnames{P}$, are those names occurring in $P$
that are not free. For example, in $x?(y).0$, the name $x$ is free, while $y$ is bound.

\begin{mathpar}
  \inferrule* [lab=monoidal-laws] {} { P|Q \equiv Q|P \and P|0 \equiv P \and P|(Q|R) \equiv (P|Q)|R }
\end{mathpar}

\begin{mathpar}
  \inferrule* [lab=alpha-equivalence] {} { (x)P \equiv (y)P\{y/x\} \and y \not\in \freenames{P} }
\end{mathpar}

\begin{definition}
Then two processes, $P,Q$, are alpha-equivalent if $P = Q\{\vec{y}/\vec{x}\}$ for
some $\vec{x} \in \boundnames{Q},\vec{y} \in \boundnames{P}$, where $Q\{\vec{y}/\vec{x}\}$
denotes the capture-avoiding substitution of $\vec{y}$ for $\vec{x}$ in $Q$.
\end{definition}

\begin{definition}
  The {\em structural congruence} \cite{SangiorgiWalker} , $\equiv$,
  between processes is the least congruence containing
  alpha-equivalence, satisfying the abelian monoid laws
  (associativity, commutativity and $\pzero$ as identity) for parallel
  composition $|$ and for summation $+$.
\end{definition}

\subsection{Name equivalence}

We take name equivalence, written $\nameeq$, to be the smallest
equivalence relation generated by the following rules.

\begin{mathpar}
\inferrule*[lab=Quote-drop]
{ }
{ \quotep{@{x}} \nameeq x }

\inferrule*[lab=Struct-equiv]
{ P \scong Q }
{ \quotep{P} \nameeq \quotep{Q} }
\end{mathpar}

The astute reader will have noticed that the mutual recursion of names
and processes imposes a mutual recursion on alpha-equivalence and
structural equivalence via name-equivalence. Fortunately, all of this
works out pleasantly and we may calculate in the natural way, free of
concern. The reader interested in the details is referred to the
appendix \ref{appendix:rho_details}.

\subsection{Substitution}

We use $\Proc$ for the set of processes, $\QProc$ for the set of
names, and $\id{\{}\vec{y} / \vec{x} \id{\}}$ to denote partial maps,
$s : \QProc \rightarrow \QProc$. A map, $s$ lifts, uniquely, to a map
on process terms, $\widehat{s} : \Proc \rightarrow \Proc$ by the
following equations.

\begin{mathpar}
  (0) \psubstp{Q}{P} := 0 \\
  (R \juxtap S) \psubstp{Q}{P}
  :=    
  (R)\psubstp{Q}{P} \juxtap (S) \psubstp{Q}{P} \\
  (x?(y).R) \psubstp{Q}{P}    
  :=    
  (x)\substp{Q}{P} (z)\concat( (R \psubstn{z}{y}) \psubstp{Q}{P} ) \\
  (\lift{x}{R}) \psubstp{Q}{P}  
  :=
  \lift{(x)\substp{Q}{P}}{ R \psubstp{Q}{P} } \\
%   (\dropn{x})  \psubstp{Q}{P}       
%   := 
%   \left\{ 
%     \begin{array}{ccc} 
%       \dropn{\quotep{Q}} & & x \nameeq \quotep{P} \\
%       \dropn{x} & & otherwise \\
%     \end{array}
%   \right. 
  (\dropn{x})  \psubstp{Q}{P}       
  := 
  \left\{ 
    \begin{array}{ccc} 
      Q & & x \nameeq \quotep{P} \\
      \dropn{x} & & otherwise \\
    \end{array}
  \right.
\end{mathpar}
 

where

\begin{eqnarray}
  (x)\id{\{} \lpquote Q \rpquote / \lpquote P \rpquote \id{\}}            = 
  \left\{ 
    \begin{array}{ccc}
      \lpquote Q \rpquote & & x \nameeq \lpquote P \rpquote \\
      x & & otherwise \\
    \end{array}
  \right. \nonumber
\end{eqnarray}

and $z$ is chosen distinct from $\quotep{P}$, $\quotep{Q}$, the free
names in $Q$, and all the names in $R$. Our $\alpha$-equivalence will
be built in the standard way from this substitution.

\begin{remark}\label{rem:no_self_referential_names}
  One consequence of these definitions is that $\forall P. \quotep{P}
  \not\in \freenames{P}$.
\end{remark}

\subsection{ Dynamic quote: an example }

Anticipating something of what's to come, consider applying the
substitution, $\widehat{\id{\{}u / z \id{\}}}$, to the following pair
of processes, $\lift{w}{y!(z)}$ and $w[ \lpquote y!(z) \rpquote ]$.

\begin{eqnarray}
	\lift{w}{y!(z)}\widehat{\id{\{}u / z \id{\}}}
		& = &
		\lift{w}{y!(u)} \nonumber\\
	w[ \lpquote y!(z) \rpquote ] \widehat{ \id{\{}u / z \id{\}} }
		& = &
		w[ \lpquote y!(z) \rpquote ] \nonumber
\end{eqnarray}

Because the body of the process between quotes is impervious to
substitution, we get radically different answers. In fact, by
examining the first process in an input context,
e.g. $x?(z).\lift{w}{y!(z)}$, we see that the process under the lift
operator may be shaped by prefixed inputs binding a name inside it. In
this sense, the lift operator will be seen as a way to dynamically
construct processes before reifying them as names.

Finally equipped with these standard features we can present the
dynamics of the calculus.

\subsubsection{Operational semantics} 

Finally, we introduce the computational dynamics. What marks these
algebras as distinct from other more traditionally studied algebraic
structures, e.g. vector spaces or polynomial rings, is the manner in
which dynamics is captured. In traditional structures, dynamics is typically
expressed through morphisms between such structures, as in linear maps
between vector spaces or morphisms between rings. In algebras
associated with the semantics of computation, the dynamics is
expressed as part of the algebraic structure itself, through a
reduction reduction relation typically denoted by $\red$. Below, we
give a recursive presentation of this relation for the calculus used
in the encoding.

$\red \subseteq \pi \times \pi$
$\red : \pi \to \mathcal{P}(\pi)$

\begin{mathpar}
  \inferrule* [lab=Comm] { \textsf{match}( x_{src}, x_{trgt} ) } { x_{trgt}?(y)P \; | \; x_{src}!\langle {Q} \rangle \red P\{\quotep{Q}/y}\} }
  \and \\
  \inferrule* [lab=Par] {{P} \red {P}'} {{{P} | {Q}} \red {{P}' | {Q}}}
  \and
  \inferrule* [lab=Equiv]{{{P} \scong {P}'} \andalso {{P}' \red {Q}'} \andalso {{Q}' \scong {Q}}}{{P} \red {Q}}
\end{mathpar}

\begin{eqnarray*}
  match_{\equiv} (\quotep{P},\quotep{Q}) & := & P \equiv Q \\
  match_{\dagger}(\quotep{P},\quotep{Q}) & := & \forall R. P|Q \red^{*} R => R \red^{*} 0 \\
  match_{K}(\quotep{P},\quotep{Q}) & := & K \mbox{ for some context } K
\end{eqnarray*}

$u?(x)P | u!\langle Q \rangle \red P\{\quotep{Q}/x\}$

%We write $\wred$ for $\red^*$, and $P\red$ if $\exists Q $ such that $ P \red Q$.
We write $P\red$ if $\exists Q $ such that $ P \red Q$ and $P\not\red$, otherwise.

\section{Replication}

As mentioned before, it is known that replication (and hence
recursion) can be implemented in a higher-order process algebra
\cite{SangiorgiWalker}. As our first example of calculation with the
machinery thus far presented we give the construction explicitly in
the {\rhoc}.

\begin{eqnarray}
	D_{x} & := & \prefix{x}{y}{(\binpar{\outputp{x}{y}}{@{y}})} \nonumber\\
	\bangp_{x}{P} & := & \binpar{{x}!\langle{\binpar{D_{x}}{P}}\rangle}{D_{x}} \nonumber
\end{eqnarray}

\begin{eqnarray}
	\bangp_{x}{P} & & \nonumber\\
	=
	& {x}!\langle{(\prefix{x}{y}{(\outputp{x}{y} | @{y})) | P}}\rangle 
	      | \prefix{x}{y}{(\outputp{x}{y} | @{y})} & \nonumber\\
	\red
	& (\outputp{x}{y} | @{y})\substn{\quotep{(\prefix{x}{y}{(@{y} | \outputp{x}{y})) | P}}}{y} & \nonumber\\
	=
	& \outputp{x}{\quotep{(\prefix{x}{y}{(\outputp{x}{y} | @{y})) | P}}}
	  | {(\prefix{x}{y}{(\outputp{x}{y} | @{y})) | P}} & \nonumber\\
	\red
	& \ldots & \nonumber\\
	\red^*
	& P | P | \ldots & \nonumber
\end{eqnarray}

Of course, this encoding, as an implementation, runs away, unfolding
$\bangp{P}$ eagerly. A lazier and more implementable replication
operator, restricted to input-guarded processes, may be obtained as follows.

\begin{eqnarray}
\bangp{\prefix{u}{v}{P}} 
	:= 
	\binpar{\lift{x}{\prefix{u}{v}{(\binpar{D(x)}{P})}}}{D(x)} \nonumber
\end{eqnarray}

\begin{remark}
  Note that the lazier definition still does not deal with summation
  or mixed summation (i.e. sums over input and output). The reader is
  invited to construct definitions of replication that deal with these
  features. 

  Further, the definitions are parameterized in a name, $x$. Can you,
  gentle reader, make a definition that eliminates this parameter and
  guarantees no accidental interaction between the replication
  machinery and the process being replicated -- i.e. no accidental
  sharing of names used by the process to get its work done and the
  name(s) used by the replication to effect copying. This latter
  revision of the definition of replication is crucial to obtaining
  the expected identity $!!P \sim !P$.
\end{remark}

\begin{remark}\label{rem:paradoxical_combinator}
  The reader familiar with the lambda calculus will have noticed the
  similarity between $D$ and the paradoxical combinator.

  [Ed. note: the existence of this seems to suggest we have to be more
  restrictive on the set of processes and names we admit if we are to
  support no-cloning.]
\end{remark}

\subsubsection{Bisimulation}

The computational dynamics gives rise to another kind of equivalence,
the equivalence of computational behavior. As previously mentioned
this is typically captured \emph{via} some form of bisimulation.

% The notion we use in this paper is weak barbed bisimulation
% \cite{milner91polyadicpi}.

The notion we use in this paper is derived from weak barbed
bisimulation \cite{milner91polyadicpi}. 

\begin{definition}
An \emph{observation relation}, $\downarrow_{\mathcal N}$, over a set
of names, $\mathcal N$, is the smallest relation satisfying the rules
below.

\infrule[Out-barb]{y \in {\mathcal N}, \; x \nameeq y}
		  {\outputp{x}{v} \downarrow_{\mathcal N} x}
\infrule[Par-barb]{\mbox{$P\downarrow_{\mathcal N} x$ or $Q\downarrow_{\mathcal N} x$}}
		  {\binpar{P}{Q} \downarrow_{\mathcal N} x}

We write $P \Downarrow_{\mathcal N} x$ if there is $Q$ such that 
$P \wred Q$ and $Q \downarrow_{\mathcal N} x$.
\end{definition}

\begin{definition}
%\label{def.bbisim}
An  ${\mathcal N}$-\emph{barbed bisimulation} over a set of names, ${\mathcal N}$, is a symmetric binary relation 
${\mathcal S}_{\mathcal N}$ between agents such that $P\rel{S}_{\mathcal N}Q$ implies:
\begin{enumerate}
\item If $P \red P'$ then $Q \wred Q'$ and $P'\rel{S}_{\mathcal N} Q'$.
\item If $P\downarrow_{\mathcal N} x$, then $Q\Downarrow_{\mathcal N} x$.
\end{enumerate}
$P$ is ${\mathcal N}$-barbed bisimilar to $Q$, written
$P \wbbisim_{\mathcal N} Q$, if $P \rel{S}_{\mathcal N} Q$ for some ${\mathcal N}$-barbed bisimulation ${\mathcal S}_{\mathcal N}$.
\end{definition}

$\mathcal{R} \subseteq \pi \times \pi$

$P \mathcal{R} Q => \forall P'. P \red P' \Rightarrow \exists Q'. Q \red Q', P' \mathcal{R} Q'$

$P \vdash x \Rightarrow Q \vdash x$

\begin{mathpar}
  \inferrule*[lab=Out-barb]{x \nameeq y}{{y}!\langle{Q}\rangle \vdash x}
  \and
  \inferrule*[lab=Par-barb]{\mbox{$P\vdash x$ or $Q\vdash x$}}{\binpar{P}{Q} \vdash x}
\end{mathpar}

\subsubsection{Contexts}

One of the principle advantages of computational calculi like the
$\pi$-calculus is a well-defined notion of context,
contextual-equivalence and a correlation between
contextual-equivalence and notions of bisimulation. The notion of
context allows the decomposition of a process into (sub-)process and
its syntactic environment, its context. Thus, a context may be
thought of as a process with a ``hole'' (written $\Box$) in it. The
application of a context $M$ to a process $P$, written $M[P]$, is
tantamount to filling the hole in $M$ with $P$. In this paper we do
not need the full weight of this theory, but do make use of the notion
of context in the proof the main theorem. 

\begin{mathpar}
  \inferrule* [lab=summation] {} {{M_{M},M_{N}} \bc \Box \;|\; x.M_{A} \;|\; M_{M}+M_{N}}
  \and
  \inferrule* [lab=agent] {} {{M_{A}} \bc (\vec{x})M_{P} \;| \; \clift{P_0,\ldots,M_{P},\ldots,P_N}}
  \and \\
  \inferrule* [lab=process] {} {{M_{P}} \bc M_{N} \;| \;P|M_{P} }
\end{mathpar} 

\begin{mathpar}
  \inferrule* [lab=sychronization] {} {M_{N} \bc \Box \;|\; x?M_{F} \;|\; x!M_{C}}
  \and
  \inferrule* [lab=abstraction] {} {{M_{F}} \bc (x)M_{P} }
  \and
  \inferrule* [lab=concretion] {} {{M_{C}} \bc \langle M_{P} \rangle }
  \and \\
  \inferrule* [lab=process] {} {{M_{P}} \bc M_{N} \;| \;P|M_{P} }
\end{mathpar}

\begin{definition}[contextual application] Given a context $M$, and
  process $P$, we define the \emph{contextual application}, $M[P] :=
  M\{P/\Box\}$. That is, the contextual application of M to P is the
  substitution of $P$ for $\Box$ in $M$.
\end{definition}

$\meaningof{-} : L \to \mathcal{P}(\pi)$

\begin{mathpar}
  \inferrule* [lab=collection] {} {\meaningof{true} = \pi, \and \meaningof{~E} = \pi \setminus \meaningof{E}, \and \meaningof{E_{1} \& E_{2}} = \meaningof{E_{1}} \cap \meaningof{E_{2}}}
\end{mathpar}

\begin{mathpar}
  \inferrule* [lab=structure] {} {\meaningof{0} = \{ P \in \pi | P \equiv 0 \}, \and \\ \meaningof{E_1 | E_2} = \{ P \in \pi | P \equiv P_{1} | P_{2}, P_{1} \in \meaningof{E_{1}}, P_{2} \in \meaningof{E_2}\} }
\end{mathpar}

\begin{mathpar}
 \inferrule* [lab=behavior] {} {\meaningof{\langle a?b \rangle E} = \{ P \in \pi | P \equiv Q | u?(y)P', \\ \and \\\\ \and \\ \;\;\; u \in \meaningof{a}, \forall z.P'\{z/y\} \in \meaningof{E\{z/b\}}\}, \and \\ \meaningof{a!E} = \{ P \in \pi | P \equiv Q | x!\langle P' \rangle, x \in \meaningof{a} P' \in \meaningof{E}\} }
\end{mathpar}

\begin{mathpar}
 \inferrule* [lab=nominal] {} {\meaningof{\quotep{E}} = \{ \quotep{P} \in \quotep{\pi} | P \in \meaningof{E} \}, \and \meaningof{\quotep{P}} = \{ \quotep{Q} \in \quotep{\pi} | P \equiv Q \} \and \\ \meaningof{@\quotep{E}} = \{ P \in \pi | P \equiv @x, x \in \meaningof{E} \}}
\end{mathpar}

\begin{eqnarray*}
  \\
  \meaningof{-} : TS \to ST
\end{eqnarray*}

\begin{eqnarray*}
  \\
  L : TS \to ST
\end{eqnarray*}

\begin{eqnarray*}
  \\
  P \models E \iff P \in \meaningof{E}
\end{eqnarray*}

\begin{eqnarray*}
  P \approx_{L} Q \iff \forall E \in L. P \models E \iff Q \models E
\end{eqnarray*}

\begin{eqnarray*}
  P \approx_{K} Q
\end{eqnarray*}

\begin{eqnarray*}
  P \approx Q
\end{eqnarray*}

$\approx_{K} = \approx = \approx_{L}$

\subsubsection{Contextual duality}

Note that contexts extend the quotation operation to a family of
operations from processes to names. Given a context, $M$, we can
define a \emph{nominal context}, $\quotep{M}$ by $\quotep{M}[P] :=
\quotep{M[P]}$. To foreshadow what is to come we observe that these
operations enjoy a duality with processes very much like the duality
between vectors and maps from vectors to scalars.

Further, because the calculus is essentially higher-order, we have a
correspondence between contexts and processes. More specifically,
given a name $x$ and a context $M$ we can construct $M^{*}_{x}$ such
that 

\begin{mathpar}
  M^{*}_{x} | \lift{x}{P} \red M[P]
\end{mathpar}

namely,

\begin{mathpar}
  M^{*}_{x} := x?(u).M[\dropn{u}]
\end{mathpar}

The dependence of $M^{*}_{x}$ on a name makes it an abstraction, 

\begin{mathpar}
  M^{*} := (x)x?(u).M[\dropn{u}]
\end{mathpar}

\subsection{Additional notation}

It will sometimes be convenient to denote the process a name
quotes. We already have the notation $x = \quotep{P}$, but it will be
convenient to introduce an alternate notation, $\procn{x}$, when we
want to emphasize the connection to the use of the name. Note that, by
virtue of name equivalence, $\quotep{\procn{x}} \nameeq x$; so, the
notation is consistent with previous definitions.

Further, because names have structure it is possible to effect
substitutions on the basis of that structure. This means we need to
upgrade our notation for substitutions, which we accomplish by
adapting comprehension notation. Thus,

\begin{mathpar}
  P\{ y / x : x \in S \}
\end{mathpar}

is interpreted to mean the process derived from P by replacing (in a
capture-avoiding manner) each occurrence of $x$ in $S$ by $y$. For example,

\begin{mathpar}
  P\{ \quotep{\procn{x}|\procn{x}} / x : x \in \freenames{P} \}
\end{mathpar}

will replace each (occurrence) of a free name $x$ in $P$ by
$\quotep{\procn{x}|\procn{x}}$.

Also, we will avail ourselves of the notation $x^{L}$ and $x^{R}$ to
denote injections of a name into disjoint copies of the name
space. There are numerous ways to accomplish this. One example can be
found in \cite{MeredithR05}. This notation overloads to vectors of
names: $\vec{x}^{\pi} := (x_{i}^{\pi} \; : \; 0 \leq i < |\vec{x}| )$ where $\pi \in \{L,R\}$.

We also use $P^{\Box} := P|\Box$.

In \cite{MeredithR05} an interpretation of the new operator is
given. It turns out that there are several possible interpretations
all enjoying the requisite algebraic properties of the operator (see
\cite{milner91polyadicpi}). We will therefore make liberal use of
$(\nu\; \vec{x})P$.

% subsection the_syntax_and_semantics_of_the_notation_system (end)   

\input{qm2pi.qmops} 

\input{qm2pi.sterngerlach} 

\input{qm2pi.metric} 

% section concurrent_process_calculi (end)

%\input{qm2pi.proofsketch}

% section proof sketch (end)

%\input{qm2pi.slviaknots} 

% section spatial logic via knots (end)

\input{qm2pi.conclusion}

% section conclusion (end)

%\input{qm2pi.dtcodes} 

% section wiring algorithm (end)

\input{qm2pi.ack} 

% section acknowledgments (end)

\newpage


\bibliographystyle{plain}   
\bibliography{../../biblios/main.bib}

\input{qm2pi.rhodetails}

\end{document}

 

% section wiring algorithm (end)

\documentclass[12pt]{llncs}
%\documentclass{jktr}

\usepackage[pdftex]{hyperref}                   
\usepackage {listings}
\usepackage {mathpartir}
\usepackage{bcprules}
%\usepackage{listings}
                       
\usepackage{graphicx} 
%\usepackage[margins=2.5cm,nohead,nofoot]{geometry}
%\usepackage{geometry}
\usepackage{amsfonts}
\usepackage{amstext}
\usepackage{latexsym}
\usepackage{amssymb}
\usepackage{color}


%\include{myPreamble}
\include{qm2pi.local} 

%\ifpdf
%\usepackage[pdftex]{graphicx}
%\else
%\usepackage{graphicx}
%\fi

 % \ifpdf
%  \usepackage{pdfsync}
%  \if


%\title{Brief Article}
%\author{David F. Snyder}
%\author{L.G. Meredith}

%\address{Dept. of Math., Texas State University--San Marcos, San Marcos, TX 78666}
       
\pagestyle{empty}


\begin{document}

\lstset{language=[Objective]Caml,frame=shadowbox}

\input{qm2pi.front}

% section front matter (end)

\input{qm2pi.intro} 
 
% section introduction (end)

% \input{qm2pi.knotations} 

% section notation (end)

\input{qm2pi.process.calculi} 

% section concurrent_process_calculi_and_spatial_logics_ (end)
    
%\input{qm2pi.knots2pi} 

%\input{qm2pi.trefoil} 

%\input{qm2pi.mainthm} 

% subsection basic_interpretation (end)

%\input{qm2pi.rho.presentation} 
\subsection{The syntax and semantics of the notation system}\label{sub:the_syntax_and_semantics_of_the_notation_system} % (fold)

We now summarize a technical presentation of the calculus that
embodies our theory of dynamics. The typical presentation of such a
calculus follows the style of giving generators and relations on
them. The grammar, below, describing term constructors, freely
generates the set of processes, $\Proc$. This set is then quotiented
by a relation known as structural congruence and it is over this set
that the notion of dynamics is expressed. This presentation is
essentially that of \cite{MeredithR05} with the addition of
polyadicity and summation. For readability we have relegated some of
the technical subtleties to an appendix.

\subsubsection{Process grammar}\label{subsub:process_grammar}

\begin{mathpar}
  \inferrule* [lab=synchronization] {} {{M} \bc \pzero \;|\; x?F \;|\; x!C }
  \and
  \inferrule* [lab=abstraction] {} {{F} \bc (x)P}
  \and
  \inferrule* [lab=concretion] {} {{C} \bc \langle Q \rangle}
  \and
  \inferrule* [lab=process] {} {{P,Q} \bc M \;| \;P|Q \;|\; @{x}}
  \and
  \inferrule* [lab=name] {} {{x} \bc \quotep{P}}
\end{mathpar} 

Note that $\vec{x}$ (resp. $\vec{P}$) denotes a vector of names
(resp. processes) of length $|\vec{x}|$ (resp. $|\vec{P}|$). We adopt
the following useful abbreviations.

\begin{mathpar}
   x?(\vec{y}).P := x.(\vec{y})P \and  x\clift{\vec{P}} := x.\clift{\vec{P}}
   \and x!(y) := \lift{x}{\dropn{y}}
   \and \Pi_{i=0}^{n-1}P_i := P_0 | \ldots | P_{n-1}
\end{mathpar}

\subsubsection{Structural congruence}

\paragraph{Free and bound names and alpha-equivalence.} At the
core of structural equivalence is alpha-equivalence which identifies
process that are the same up to a change of variable. Formally, we
recognize the distinction between free and bound names. The free names
of a process, $\freenames{P}$, may be calculated recursively as
follows:

\begin{mathpar}
\freenames{\pzero} := \emptyset
  \and \\
  \freenames{x?(y).P} := \{ x \} \cup (\freenames{P} \setminus \{ y \})
  \and 
  \freenames{x!\langle P \rangle} := \{ x \} \cup \{ P \} 
  \and \\
  \freenames{P|Q} := \freenames{P} \cup \freenames{Q}
  \and \\
  \freenames{@{x}} := \{ x \}
\end{mathpar}

$\pi$
$\quotep{\pi}$

$\freenames{-} : \pi \to \mathcal{P}(\quotep{\pi})$

\begin{eqnarray*}
  \freenames{\pzero} & := & \emptyset \\
  \freenames{x?(y).P} & := & \{ x \} \cup (\freenames{P} \setminus \{ y \}) \\
  \freenames{x!\langle P \rangle} & := & \{ x \} \cup \{ P \} \\
  \freenames{P|Q} & := & \freenames{P} \cup \freenames{Q} \\
  \freenames{\dropn{x}} & := & \{ x \}
\end{eqnarray*}

The bound names of a process, $\boundnames{P}$, are those names occurring in $P$
that are not free. For example, in $x?(y).0$, the name $x$ is free, while $y$ is bound.

\begin{mathpar}
  \inferrule* [lab=monoidal-laws] {} { P|Q \equiv Q|P \and P|0 \equiv P \and P|(Q|R) \equiv (P|Q)|R }
\end{mathpar}

\begin{mathpar}
  \inferrule* [lab=alpha-equivalence] {} { (x)P \equiv (y)P\{y/x\} \and y \not\in \freenames{P} }
\end{mathpar}

\begin{definition}
Then two processes, $P,Q$, are alpha-equivalent if $P = Q\{\vec{y}/\vec{x}\}$ for
some $\vec{x} \in \boundnames{Q},\vec{y} \in \boundnames{P}$, where $Q\{\vec{y}/\vec{x}\}$
denotes the capture-avoiding substitution of $\vec{y}$ for $\vec{x}$ in $Q$.
\end{definition}

\begin{definition}
  The {\em structural congruence} \cite{SangiorgiWalker} , $\equiv$,
  between processes is the least congruence containing
  alpha-equivalence, satisfying the abelian monoid laws
  (associativity, commutativity and $\pzero$ as identity) for parallel
  composition $|$ and for summation $+$.
\end{definition}

\subsection{Name equivalence}

We take name equivalence, written $\nameeq$, to be the smallest
equivalence relation generated by the following rules.

\begin{mathpar}
\inferrule*[lab=Quote-drop]
{ }
{ \quotep{@{x}} \nameeq x }

\inferrule*[lab=Struct-equiv]
{ P \scong Q }
{ \quotep{P} \nameeq \quotep{Q} }
\end{mathpar}

The astute reader will have noticed that the mutual recursion of names
and processes imposes a mutual recursion on alpha-equivalence and
structural equivalence via name-equivalence. Fortunately, all of this
works out pleasantly and we may calculate in the natural way, free of
concern. The reader interested in the details is referred to the
appendix \ref{appendix:rho_details}.

\subsection{Substitution}

We use $\Proc$ for the set of processes, $\QProc$ for the set of
names, and $\id{\{}\vec{y} / \vec{x} \id{\}}$ to denote partial maps,
$s : \QProc \rightarrow \QProc$. A map, $s$ lifts, uniquely, to a map
on process terms, $\widehat{s} : \Proc \rightarrow \Proc$ by the
following equations.

\begin{mathpar}
  (0) \psubstp{Q}{P} := 0 \\
  (R \juxtap S) \psubstp{Q}{P}
  :=    
  (R)\psubstp{Q}{P} \juxtap (S) \psubstp{Q}{P} \\
  (x?(y).R) \psubstp{Q}{P}    
  :=    
  (x)\substp{Q}{P} (z)\concat( (R \psubstn{z}{y}) \psubstp{Q}{P} ) \\
  (\lift{x}{R}) \psubstp{Q}{P}  
  :=
  \lift{(x)\substp{Q}{P}}{ R \psubstp{Q}{P} } \\
%   (\dropn{x})  \psubstp{Q}{P}       
%   := 
%   \left\{ 
%     \begin{array}{ccc} 
%       \dropn{\quotep{Q}} & & x \nameeq \quotep{P} \\
%       \dropn{x} & & otherwise \\
%     \end{array}
%   \right. 
  (\dropn{x})  \psubstp{Q}{P}       
  := 
  \left\{ 
    \begin{array}{ccc} 
      Q & & x \nameeq \quotep{P} \\
      \dropn{x} & & otherwise \\
    \end{array}
  \right.
\end{mathpar}
 

where

\begin{eqnarray}
  (x)\id{\{} \lpquote Q \rpquote / \lpquote P \rpquote \id{\}}            = 
  \left\{ 
    \begin{array}{ccc}
      \lpquote Q \rpquote & & x \nameeq \lpquote P \rpquote \\
      x & & otherwise \\
    \end{array}
  \right. \nonumber
\end{eqnarray}

and $z$ is chosen distinct from $\quotep{P}$, $\quotep{Q}$, the free
names in $Q$, and all the names in $R$. Our $\alpha$-equivalence will
be built in the standard way from this substitution.

\begin{remark}\label{rem:no_self_referential_names}
  One consequence of these definitions is that $\forall P. \quotep{P}
  \not\in \freenames{P}$.
\end{remark}

\subsection{ Dynamic quote: an example }

Anticipating something of what's to come, consider applying the
substitution, $\widehat{\id{\{}u / z \id{\}}}$, to the following pair
of processes, $\lift{w}{y!(z)}$ and $w[ \lpquote y!(z) \rpquote ]$.

\begin{eqnarray}
	\lift{w}{y!(z)}\widehat{\id{\{}u / z \id{\}}}
		& = &
		\lift{w}{y!(u)} \nonumber\\
	w[ \lpquote y!(z) \rpquote ] \widehat{ \id{\{}u / z \id{\}} }
		& = &
		w[ \lpquote y!(z) \rpquote ] \nonumber
\end{eqnarray}

Because the body of the process between quotes is impervious to
substitution, we get radically different answers. In fact, by
examining the first process in an input context,
e.g. $x?(z).\lift{w}{y!(z)}$, we see that the process under the lift
operator may be shaped by prefixed inputs binding a name inside it. In
this sense, the lift operator will be seen as a way to dynamically
construct processes before reifying them as names.

Finally equipped with these standard features we can present the
dynamics of the calculus.

\subsubsection{Operational semantics} 

Finally, we introduce the computational dynamics. What marks these
algebras as distinct from other more traditionally studied algebraic
structures, e.g. vector spaces or polynomial rings, is the manner in
which dynamics is captured. In traditional structures, dynamics is typically
expressed through morphisms between such structures, as in linear maps
between vector spaces or morphisms between rings. In algebras
associated with the semantics of computation, the dynamics is
expressed as part of the algebraic structure itself, through a
reduction reduction relation typically denoted by $\red$. Below, we
give a recursive presentation of this relation for the calculus used
in the encoding.

$\red \subseteq \pi \times \pi$
$\red : \pi \to \mathcal{P}(\pi)$

\begin{mathpar}
  \inferrule* [lab=Comm] { \textsf{match}( x_{src}, x_{trgt} ) } { x_{trgt}?(y)P \; | \; x_{src}!\langle {Q} \rangle \red P\{\quotep{Q}/y}\} }
  \and \\
  \inferrule* [lab=Par] {{P} \red {P}'} {{{P} | {Q}} \red {{P}' | {Q}}}
  \and
  \inferrule* [lab=Equiv]{{{P} \scong {P}'} \andalso {{P}' \red {Q}'} \andalso {{Q}' \scong {Q}}}{{P} \red {Q}}
\end{mathpar}

\begin{eqnarray*}
  match_{\equiv} (\quotep{P},\quotep{Q}) & := & P \equiv Q \\
  match_{\dagger}(\quotep{P},\quotep{Q}) & := & \forall R. P|Q \red^{*} R => R \red^{*} 0 \\
  match_{K}(\quotep{P},\quotep{Q}) & := & K \mbox{ for some context } K
\end{eqnarray*}

$u?(x)P | u!\langle Q \rangle \red P\{\quotep{Q}/x\}$

%We write $\wred$ for $\red^*$, and $P\red$ if $\exists Q $ such that $ P \red Q$.
We write $P\red$ if $\exists Q $ such that $ P \red Q$ and $P\not\red$, otherwise.

\section{Replication}

As mentioned before, it is known that replication (and hence
recursion) can be implemented in a higher-order process algebra
\cite{SangiorgiWalker}. As our first example of calculation with the
machinery thus far presented we give the construction explicitly in
the {\rhoc}.

\begin{eqnarray}
	D_{x} & := & \prefix{x}{y}{(\binpar{\outputp{x}{y}}{@{y}})} \nonumber\\
	\bangp_{x}{P} & := & \binpar{{x}!\langle{\binpar{D_{x}}{P}}\rangle}{D_{x}} \nonumber
\end{eqnarray}

\begin{eqnarray}
	\bangp_{x}{P} & & \nonumber\\
	=
	& {x}!\langle{(\prefix{x}{y}{(\outputp{x}{y} | @{y})) | P}}\rangle 
	      | \prefix{x}{y}{(\outputp{x}{y} | @{y})} & \nonumber\\
	\red
	& (\outputp{x}{y} | @{y})\substn{\quotep{(\prefix{x}{y}{(@{y} | \outputp{x}{y})) | P}}}{y} & \nonumber\\
	=
	& \outputp{x}{\quotep{(\prefix{x}{y}{(\outputp{x}{y} | @{y})) | P}}}
	  | {(\prefix{x}{y}{(\outputp{x}{y} | @{y})) | P}} & \nonumber\\
	\red
	& \ldots & \nonumber\\
	\red^*
	& P | P | \ldots & \nonumber
\end{eqnarray}

Of course, this encoding, as an implementation, runs away, unfolding
$\bangp{P}$ eagerly. A lazier and more implementable replication
operator, restricted to input-guarded processes, may be obtained as follows.

\begin{eqnarray}
\bangp{\prefix{u}{v}{P}} 
	:= 
	\binpar{\lift{x}{\prefix{u}{v}{(\binpar{D(x)}{P})}}}{D(x)} \nonumber
\end{eqnarray}

\begin{remark}
  Note that the lazier definition still does not deal with summation
  or mixed summation (i.e. sums over input and output). The reader is
  invited to construct definitions of replication that deal with these
  features. 

  Further, the definitions are parameterized in a name, $x$. Can you,
  gentle reader, make a definition that eliminates this parameter and
  guarantees no accidental interaction between the replication
  machinery and the process being replicated -- i.e. no accidental
  sharing of names used by the process to get its work done and the
  name(s) used by the replication to effect copying. This latter
  revision of the definition of replication is crucial to obtaining
  the expected identity $!!P \sim !P$.
\end{remark}

\begin{remark}\label{rem:paradoxical_combinator}
  The reader familiar with the lambda calculus will have noticed the
  similarity between $D$ and the paradoxical combinator.

  [Ed. note: the existence of this seems to suggest we have to be more
  restrictive on the set of processes and names we admit if we are to
  support no-cloning.]
\end{remark}

\subsubsection{Bisimulation}

The computational dynamics gives rise to another kind of equivalence,
the equivalence of computational behavior. As previously mentioned
this is typically captured \emph{via} some form of bisimulation.

% The notion we use in this paper is weak barbed bisimulation
% \cite{milner91polyadicpi}.

The notion we use in this paper is derived from weak barbed
bisimulation \cite{milner91polyadicpi}. 

\begin{definition}
An \emph{observation relation}, $\downarrow_{\mathcal N}$, over a set
of names, $\mathcal N$, is the smallest relation satisfying the rules
below.

\infrule[Out-barb]{y \in {\mathcal N}, \; x \nameeq y}
		  {\outputp{x}{v} \downarrow_{\mathcal N} x}
\infrule[Par-barb]{\mbox{$P\downarrow_{\mathcal N} x$ or $Q\downarrow_{\mathcal N} x$}}
		  {\binpar{P}{Q} \downarrow_{\mathcal N} x}

We write $P \Downarrow_{\mathcal N} x$ if there is $Q$ such that 
$P \wred Q$ and $Q \downarrow_{\mathcal N} x$.
\end{definition}

\begin{definition}
%\label{def.bbisim}
An  ${\mathcal N}$-\emph{barbed bisimulation} over a set of names, ${\mathcal N}$, is a symmetric binary relation 
${\mathcal S}_{\mathcal N}$ between agents such that $P\rel{S}_{\mathcal N}Q$ implies:
\begin{enumerate}
\item If $P \red P'$ then $Q \wred Q'$ and $P'\rel{S}_{\mathcal N} Q'$.
\item If $P\downarrow_{\mathcal N} x$, then $Q\Downarrow_{\mathcal N} x$.
\end{enumerate}
$P$ is ${\mathcal N}$-barbed bisimilar to $Q$, written
$P \wbbisim_{\mathcal N} Q$, if $P \rel{S}_{\mathcal N} Q$ for some ${\mathcal N}$-barbed bisimulation ${\mathcal S}_{\mathcal N}$.
\end{definition}

$\mathcal{R} \subseteq \pi \times \pi$

$P \mathcal{R} Q => \forall P'. P \red P' \Rightarrow \exists Q'. Q \red Q', P' \mathcal{R} Q'$

$P \vdash x \Rightarrow Q \vdash x$

\begin{mathpar}
  \inferrule*[lab=Out-barb]{x \nameeq y}{{y}!\langle{Q}\rangle \vdash x}
  \and
  \inferrule*[lab=Par-barb]{\mbox{$P\vdash x$ or $Q\vdash x$}}{\binpar{P}{Q} \vdash x}
\end{mathpar}

\subsubsection{Contexts}

One of the principle advantages of computational calculi like the
$\pi$-calculus is a well-defined notion of context,
contextual-equivalence and a correlation between
contextual-equivalence and notions of bisimulation. The notion of
context allows the decomposition of a process into (sub-)process and
its syntactic environment, its context. Thus, a context may be
thought of as a process with a ``hole'' (written $\Box$) in it. The
application of a context $M$ to a process $P$, written $M[P]$, is
tantamount to filling the hole in $M$ with $P$. In this paper we do
not need the full weight of this theory, but do make use of the notion
of context in the proof the main theorem. 

\begin{mathpar}
  \inferrule* [lab=summation] {} {{M_{M},M_{N}} \bc \Box \;|\; x.M_{A} \;|\; M_{M}+M_{N}}
  \and
  \inferrule* [lab=agent] {} {{M_{A}} \bc (\vec{x})M_{P} \;| \; \clift{P_0,\ldots,M_{P},\ldots,P_N}}
  \and \\
  \inferrule* [lab=process] {} {{M_{P}} \bc M_{N} \;| \;P|M_{P} }
\end{mathpar} 

\begin{mathpar}
  \inferrule* [lab=sychronization] {} {M_{N} \bc \Box \;|\; x?M_{F} \;|\; x!M_{C}}
  \and
  \inferrule* [lab=abstraction] {} {{M_{F}} \bc (x)M_{P} }
  \and
  \inferrule* [lab=concretion] {} {{M_{C}} \bc \langle M_{P} \rangle }
  \and \\
  \inferrule* [lab=process] {} {{M_{P}} \bc M_{N} \;| \;P|M_{P} }
\end{mathpar}

\begin{definition}[contextual application] Given a context $M$, and
  process $P$, we define the \emph{contextual application}, $M[P] :=
  M\{P/\Box\}$. That is, the contextual application of M to P is the
  substitution of $P$ for $\Box$ in $M$.
\end{definition}

$\meaningof{-} : L \to \mathcal{P}(\pi)$

\begin{mathpar}
  \inferrule* [lab=collection] {} {\meaningof{true} = \pi, \and \meaningof{~E} = \pi \setminus \meaningof{E}, \and \meaningof{E_{1} \& E_{2}} = \meaningof{E_{1}} \cap \meaningof{E_{2}}}
\end{mathpar}

\begin{mathpar}
  \inferrule* [lab=structure] {} {\meaningof{0} = \{ P \in \pi | P \equiv 0 \}, \and \\ \meaningof{E_1 | E_2} = \{ P \in \pi | P \equiv P_{1} | P_{2}, P_{1} \in \meaningof{E_{1}}, P_{2} \in \meaningof{E_2}\} }
\end{mathpar}

\begin{mathpar}
 \inferrule* [lab=behavior] {} {\meaningof{\langle a?b \rangle E} = \{ P \in \pi | P \equiv Q | u?(y)P', \\ \and \\\\ \and \\ \;\;\; u \in \meaningof{a}, \forall z.P'\{z/y\} \in \meaningof{E\{z/b\}}\}, \and \\ \meaningof{a!E} = \{ P \in \pi | P \equiv Q | x!\langle P' \rangle, x \in \meaningof{a} P' \in \meaningof{E}\} }
\end{mathpar}

\begin{mathpar}
 \inferrule* [lab=nominal] {} {\meaningof{\quotep{E}} = \{ \quotep{P} \in \quotep{\pi} | P \in \meaningof{E} \}, \and \meaningof{\quotep{P}} = \{ \quotep{Q} \in \quotep{\pi} | P \equiv Q \} \and \\ \meaningof{@\quotep{E}} = \{ P \in \pi | P \equiv @x, x \in \meaningof{E} \}}
\end{mathpar}

\begin{eqnarray*}
  \\
  \meaningof{-} : TS \to ST
\end{eqnarray*}

\begin{eqnarray*}
  \\
  L : TS \to ST
\end{eqnarray*}

\begin{eqnarray*}
  \\
  P \models E \iff P \in \meaningof{E}
\end{eqnarray*}

\begin{eqnarray*}
  P \approx_{L} Q \iff \forall E \in L. P \models E \iff Q \models E
\end{eqnarray*}

\begin{eqnarray*}
  P \approx_{K} Q
\end{eqnarray*}

\begin{eqnarray*}
  P \approx Q
\end{eqnarray*}

$\approx_{K} = \approx = \approx_{L}$

\subsubsection{Contextual duality}

Note that contexts extend the quotation operation to a family of
operations from processes to names. Given a context, $M$, we can
define a \emph{nominal context}, $\quotep{M}$ by $\quotep{M}[P] :=
\quotep{M[P]}$. To foreshadow what is to come we observe that these
operations enjoy a duality with processes very much like the duality
between vectors and maps from vectors to scalars.

Further, because the calculus is essentially higher-order, we have a
correspondence between contexts and processes. More specifically,
given a name $x$ and a context $M$ we can construct $M^{*}_{x}$ such
that 

\begin{mathpar}
  M^{*}_{x} | \lift{x}{P} \red M[P]
\end{mathpar}

namely,

\begin{mathpar}
  M^{*}_{x} := x?(u).M[\dropn{u}]
\end{mathpar}

The dependence of $M^{*}_{x}$ on a name makes it an abstraction, 

\begin{mathpar}
  M^{*} := (x)x?(u).M[\dropn{u}]
\end{mathpar}

\subsection{Additional notation}

It will sometimes be convenient to denote the process a name
quotes. We already have the notation $x = \quotep{P}$, but it will be
convenient to introduce an alternate notation, $\procn{x}$, when we
want to emphasize the connection to the use of the name. Note that, by
virtue of name equivalence, $\quotep{\procn{x}} \nameeq x$; so, the
notation is consistent with previous definitions.

Further, because names have structure it is possible to effect
substitutions on the basis of that structure. This means we need to
upgrade our notation for substitutions, which we accomplish by
adapting comprehension notation. Thus,

\begin{mathpar}
  P\{ y / x : x \in S \}
\end{mathpar}

is interpreted to mean the process derived from P by replacing (in a
capture-avoiding manner) each occurrence of $x$ in $S$ by $y$. For example,

\begin{mathpar}
  P\{ \quotep{\procn{x}|\procn{x}} / x : x \in \freenames{P} \}
\end{mathpar}

will replace each (occurrence) of a free name $x$ in $P$ by
$\quotep{\procn{x}|\procn{x}}$.

Also, we will avail ourselves of the notation $x^{L}$ and $x^{R}$ to
denote injections of a name into disjoint copies of the name
space. There are numerous ways to accomplish this. One example can be
found in \cite{MeredithR05}. This notation overloads to vectors of
names: $\vec{x}^{\pi} := (x_{i}^{\pi} \; : \; 0 \leq i < |\vec{x}| )$ where $\pi \in \{L,R\}$.

We also use $P^{\Box} := P|\Box$.

In \cite{MeredithR05} an interpretation of the new operator is
given. It turns out that there are several possible interpretations
all enjoying the requisite algebraic properties of the operator (see
\cite{milner91polyadicpi}). We will therefore make liberal use of
$(\nu\; \vec{x})P$.

% subsection the_syntax_and_semantics_of_the_notation_system (end)   

\input{qm2pi.qmops} 

\input{qm2pi.sterngerlach} 

\input{qm2pi.metric} 

% section concurrent_process_calculi (end)

%\input{qm2pi.proofsketch}

% section proof sketch (end)

%\input{qm2pi.slviaknots} 

% section spatial logic via knots (end)

\input{qm2pi.conclusion}

% section conclusion (end)

%\input{qm2pi.dtcodes} 

% section wiring algorithm (end)

\input{qm2pi.ack} 

% section acknowledgments (end)

\newpage


\bibliographystyle{plain}   
\bibliography{../../biblios/main.bib}

\input{qm2pi.rhodetails}

\end{document}

 

% section acknowledgments (end)

\newpage


\bibliographystyle{plain}   
\bibliography{../../biblios/main.bib}

\documentclass[12pt]{llncs}
%\documentclass{jktr}

\usepackage[pdftex]{hyperref}                   
\usepackage {listings}
\usepackage {mathpartir}
\usepackage{bcprules}
%\usepackage{listings}
                       
\usepackage{graphicx} 
%\usepackage[margins=2.5cm,nohead,nofoot]{geometry}
%\usepackage{geometry}
\usepackage{amsfonts}
\usepackage{amstext}
\usepackage{latexsym}
\usepackage{amssymb}
\usepackage{color}


%\include{myPreamble}
\include{qm2pi.local} 

%\ifpdf
%\usepackage[pdftex]{graphicx}
%\else
%\usepackage{graphicx}
%\fi

 % \ifpdf
%  \usepackage{pdfsync}
%  \if


%\title{Brief Article}
%\author{David F. Snyder}
%\author{L.G. Meredith}

%\address{Dept. of Math., Texas State University--San Marcos, San Marcos, TX 78666}
       
\pagestyle{empty}


\begin{document}

\lstset{language=[Objective]Caml,frame=shadowbox}

\input{qm2pi.front}

% section front matter (end)

\input{qm2pi.intro} 
 
% section introduction (end)

% \input{qm2pi.knotations} 

% section notation (end)

\input{qm2pi.process.calculi} 

% section concurrent_process_calculi_and_spatial_logics_ (end)
    
%\input{qm2pi.knots2pi} 

%\input{qm2pi.trefoil} 

%\input{qm2pi.mainthm} 

% subsection basic_interpretation (end)

%\input{qm2pi.rho.presentation} 
\subsection{The syntax and semantics of the notation system}\label{sub:the_syntax_and_semantics_of_the_notation_system} % (fold)

We now summarize a technical presentation of the calculus that
embodies our theory of dynamics. The typical presentation of such a
calculus follows the style of giving generators and relations on
them. The grammar, below, describing term constructors, freely
generates the set of processes, $\Proc$. This set is then quotiented
by a relation known as structural congruence and it is over this set
that the notion of dynamics is expressed. This presentation is
essentially that of \cite{MeredithR05} with the addition of
polyadicity and summation. For readability we have relegated some of
the technical subtleties to an appendix.

\subsubsection{Process grammar}\label{subsub:process_grammar}

\begin{mathpar}
  \inferrule* [lab=synchronization] {} {{M} \bc \pzero \;|\; x?F \;|\; x!C }
  \and
  \inferrule* [lab=abstraction] {} {{F} \bc (x)P}
  \and
  \inferrule* [lab=concretion] {} {{C} \bc \langle Q \rangle}
  \and
  \inferrule* [lab=process] {} {{P,Q} \bc M \;| \;P|Q \;|\; @{x}}
  \and
  \inferrule* [lab=name] {} {{x} \bc \quotep{P}}
\end{mathpar} 

Note that $\vec{x}$ (resp. $\vec{P}$) denotes a vector of names
(resp. processes) of length $|\vec{x}|$ (resp. $|\vec{P}|$). We adopt
the following useful abbreviations.

\begin{mathpar}
   x?(\vec{y}).P := x.(\vec{y})P \and  x\clift{\vec{P}} := x.\clift{\vec{P}}
   \and x!(y) := \lift{x}{\dropn{y}}
   \and \Pi_{i=0}^{n-1}P_i := P_0 | \ldots | P_{n-1}
\end{mathpar}

\subsubsection{Structural congruence}

\paragraph{Free and bound names and alpha-equivalence.} At the
core of structural equivalence is alpha-equivalence which identifies
process that are the same up to a change of variable. Formally, we
recognize the distinction between free and bound names. The free names
of a process, $\freenames{P}$, may be calculated recursively as
follows:

\begin{mathpar}
\freenames{\pzero} := \emptyset
  \and \\
  \freenames{x?(y).P} := \{ x \} \cup (\freenames{P} \setminus \{ y \})
  \and 
  \freenames{x!\langle P \rangle} := \{ x \} \cup \{ P \} 
  \and \\
  \freenames{P|Q} := \freenames{P} \cup \freenames{Q}
  \and \\
  \freenames{@{x}} := \{ x \}
\end{mathpar}

$\pi$
$\quotep{\pi}$

$\freenames{-} : \pi \to \mathcal{P}(\quotep{\pi})$

\begin{eqnarray*}
  \freenames{\pzero} & := & \emptyset \\
  \freenames{x?(y).P} & := & \{ x \} \cup (\freenames{P} \setminus \{ y \}) \\
  \freenames{x!\langle P \rangle} & := & \{ x \} \cup \{ P \} \\
  \freenames{P|Q} & := & \freenames{P} \cup \freenames{Q} \\
  \freenames{\dropn{x}} & := & \{ x \}
\end{eqnarray*}

The bound names of a process, $\boundnames{P}$, are those names occurring in $P$
that are not free. For example, in $x?(y).0$, the name $x$ is free, while $y$ is bound.

\begin{mathpar}
  \inferrule* [lab=monoidal-laws] {} { P|Q \equiv Q|P \and P|0 \equiv P \and P|(Q|R) \equiv (P|Q)|R }
\end{mathpar}

\begin{mathpar}
  \inferrule* [lab=alpha-equivalence] {} { (x)P \equiv (y)P\{y/x\} \and y \not\in \freenames{P} }
\end{mathpar}

\begin{definition}
Then two processes, $P,Q$, are alpha-equivalent if $P = Q\{\vec{y}/\vec{x}\}$ for
some $\vec{x} \in \boundnames{Q},\vec{y} \in \boundnames{P}$, where $Q\{\vec{y}/\vec{x}\}$
denotes the capture-avoiding substitution of $\vec{y}$ for $\vec{x}$ in $Q$.
\end{definition}

\begin{definition}
  The {\em structural congruence} \cite{SangiorgiWalker} , $\equiv$,
  between processes is the least congruence containing
  alpha-equivalence, satisfying the abelian monoid laws
  (associativity, commutativity and $\pzero$ as identity) for parallel
  composition $|$ and for summation $+$.
\end{definition}

\subsection{Name equivalence}

We take name equivalence, written $\nameeq$, to be the smallest
equivalence relation generated by the following rules.

\begin{mathpar}
\inferrule*[lab=Quote-drop]
{ }
{ \quotep{@{x}} \nameeq x }

\inferrule*[lab=Struct-equiv]
{ P \scong Q }
{ \quotep{P} \nameeq \quotep{Q} }
\end{mathpar}

The astute reader will have noticed that the mutual recursion of names
and processes imposes a mutual recursion on alpha-equivalence and
structural equivalence via name-equivalence. Fortunately, all of this
works out pleasantly and we may calculate in the natural way, free of
concern. The reader interested in the details is referred to the
appendix \ref{appendix:rho_details}.

\subsection{Substitution}

We use $\Proc$ for the set of processes, $\QProc$ for the set of
names, and $\id{\{}\vec{y} / \vec{x} \id{\}}$ to denote partial maps,
$s : \QProc \rightarrow \QProc$. A map, $s$ lifts, uniquely, to a map
on process terms, $\widehat{s} : \Proc \rightarrow \Proc$ by the
following equations.

\begin{mathpar}
  (0) \psubstp{Q}{P} := 0 \\
  (R \juxtap S) \psubstp{Q}{P}
  :=    
  (R)\psubstp{Q}{P} \juxtap (S) \psubstp{Q}{P} \\
  (x?(y).R) \psubstp{Q}{P}    
  :=    
  (x)\substp{Q}{P} (z)\concat( (R \psubstn{z}{y}) \psubstp{Q}{P} ) \\
  (\lift{x}{R}) \psubstp{Q}{P}  
  :=
  \lift{(x)\substp{Q}{P}}{ R \psubstp{Q}{P} } \\
%   (\dropn{x})  \psubstp{Q}{P}       
%   := 
%   \left\{ 
%     \begin{array}{ccc} 
%       \dropn{\quotep{Q}} & & x \nameeq \quotep{P} \\
%       \dropn{x} & & otherwise \\
%     \end{array}
%   \right. 
  (\dropn{x})  \psubstp{Q}{P}       
  := 
  \left\{ 
    \begin{array}{ccc} 
      Q & & x \nameeq \quotep{P} \\
      \dropn{x} & & otherwise \\
    \end{array}
  \right.
\end{mathpar}
 

where

\begin{eqnarray}
  (x)\id{\{} \lpquote Q \rpquote / \lpquote P \rpquote \id{\}}            = 
  \left\{ 
    \begin{array}{ccc}
      \lpquote Q \rpquote & & x \nameeq \lpquote P \rpquote \\
      x & & otherwise \\
    \end{array}
  \right. \nonumber
\end{eqnarray}

and $z$ is chosen distinct from $\quotep{P}$, $\quotep{Q}$, the free
names in $Q$, and all the names in $R$. Our $\alpha$-equivalence will
be built in the standard way from this substitution.

\begin{remark}\label{rem:no_self_referential_names}
  One consequence of these definitions is that $\forall P. \quotep{P}
  \not\in \freenames{P}$.
\end{remark}

\subsection{ Dynamic quote: an example }

Anticipating something of what's to come, consider applying the
substitution, $\widehat{\id{\{}u / z \id{\}}}$, to the following pair
of processes, $\lift{w}{y!(z)}$ and $w[ \lpquote y!(z) \rpquote ]$.

\begin{eqnarray}
	\lift{w}{y!(z)}\widehat{\id{\{}u / z \id{\}}}
		& = &
		\lift{w}{y!(u)} \nonumber\\
	w[ \lpquote y!(z) \rpquote ] \widehat{ \id{\{}u / z \id{\}} }
		& = &
		w[ \lpquote y!(z) \rpquote ] \nonumber
\end{eqnarray}

Because the body of the process between quotes is impervious to
substitution, we get radically different answers. In fact, by
examining the first process in an input context,
e.g. $x?(z).\lift{w}{y!(z)}$, we see that the process under the lift
operator may be shaped by prefixed inputs binding a name inside it. In
this sense, the lift operator will be seen as a way to dynamically
construct processes before reifying them as names.

Finally equipped with these standard features we can present the
dynamics of the calculus.

\subsubsection{Operational semantics} 

Finally, we introduce the computational dynamics. What marks these
algebras as distinct from other more traditionally studied algebraic
structures, e.g. vector spaces or polynomial rings, is the manner in
which dynamics is captured. In traditional structures, dynamics is typically
expressed through morphisms between such structures, as in linear maps
between vector spaces or morphisms between rings. In algebras
associated with the semantics of computation, the dynamics is
expressed as part of the algebraic structure itself, through a
reduction reduction relation typically denoted by $\red$. Below, we
give a recursive presentation of this relation for the calculus used
in the encoding.

$\red \subseteq \pi \times \pi$
$\red : \pi \to \mathcal{P}(\pi)$

\begin{mathpar}
  \inferrule* [lab=Comm] { \textsf{match}( x_{src}, x_{trgt} ) } { x_{trgt}?(y)P \; | \; x_{src}!\langle {Q} \rangle \red P\{\quotep{Q}/y}\} }
  \and \\
  \inferrule* [lab=Par] {{P} \red {P}'} {{{P} | {Q}} \red {{P}' | {Q}}}
  \and
  \inferrule* [lab=Equiv]{{{P} \scong {P}'} \andalso {{P}' \red {Q}'} \andalso {{Q}' \scong {Q}}}{{P} \red {Q}}
\end{mathpar}

\begin{eqnarray*}
  match_{\equiv} (\quotep{P},\quotep{Q}) & := & P \equiv Q \\
  match_{\dagger}(\quotep{P},\quotep{Q}) & := & \forall R. P|Q \red^{*} R => R \red^{*} 0 \\
  match_{K}(\quotep{P},\quotep{Q}) & := & K \mbox{ for some context } K
\end{eqnarray*}

$u?(x)P | u!\langle Q \rangle \red P\{\quotep{Q}/x\}$

%We write $\wred$ for $\red^*$, and $P\red$ if $\exists Q $ such that $ P \red Q$.
We write $P\red$ if $\exists Q $ such that $ P \red Q$ and $P\not\red$, otherwise.

\section{Replication}

As mentioned before, it is known that replication (and hence
recursion) can be implemented in a higher-order process algebra
\cite{SangiorgiWalker}. As our first example of calculation with the
machinery thus far presented we give the construction explicitly in
the {\rhoc}.

\begin{eqnarray}
	D_{x} & := & \prefix{x}{y}{(\binpar{\outputp{x}{y}}{@{y}})} \nonumber\\
	\bangp_{x}{P} & := & \binpar{{x}!\langle{\binpar{D_{x}}{P}}\rangle}{D_{x}} \nonumber
\end{eqnarray}

\begin{eqnarray}
	\bangp_{x}{P} & & \nonumber\\
	=
	& {x}!\langle{(\prefix{x}{y}{(\outputp{x}{y} | @{y})) | P}}\rangle 
	      | \prefix{x}{y}{(\outputp{x}{y} | @{y})} & \nonumber\\
	\red
	& (\outputp{x}{y} | @{y})\substn{\quotep{(\prefix{x}{y}{(@{y} | \outputp{x}{y})) | P}}}{y} & \nonumber\\
	=
	& \outputp{x}{\quotep{(\prefix{x}{y}{(\outputp{x}{y} | @{y})) | P}}}
	  | {(\prefix{x}{y}{(\outputp{x}{y} | @{y})) | P}} & \nonumber\\
	\red
	& \ldots & \nonumber\\
	\red^*
	& P | P | \ldots & \nonumber
\end{eqnarray}

Of course, this encoding, as an implementation, runs away, unfolding
$\bangp{P}$ eagerly. A lazier and more implementable replication
operator, restricted to input-guarded processes, may be obtained as follows.

\begin{eqnarray}
\bangp{\prefix{u}{v}{P}} 
	:= 
	\binpar{\lift{x}{\prefix{u}{v}{(\binpar{D(x)}{P})}}}{D(x)} \nonumber
\end{eqnarray}

\begin{remark}
  Note that the lazier definition still does not deal with summation
  or mixed summation (i.e. sums over input and output). The reader is
  invited to construct definitions of replication that deal with these
  features. 

  Further, the definitions are parameterized in a name, $x$. Can you,
  gentle reader, make a definition that eliminates this parameter and
  guarantees no accidental interaction between the replication
  machinery and the process being replicated -- i.e. no accidental
  sharing of names used by the process to get its work done and the
  name(s) used by the replication to effect copying. This latter
  revision of the definition of replication is crucial to obtaining
  the expected identity $!!P \sim !P$.
\end{remark}

\begin{remark}\label{rem:paradoxical_combinator}
  The reader familiar with the lambda calculus will have noticed the
  similarity between $D$ and the paradoxical combinator.

  [Ed. note: the existence of this seems to suggest we have to be more
  restrictive on the set of processes and names we admit if we are to
  support no-cloning.]
\end{remark}

\subsubsection{Bisimulation}

The computational dynamics gives rise to another kind of equivalence,
the equivalence of computational behavior. As previously mentioned
this is typically captured \emph{via} some form of bisimulation.

% The notion we use in this paper is weak barbed bisimulation
% \cite{milner91polyadicpi}.

The notion we use in this paper is derived from weak barbed
bisimulation \cite{milner91polyadicpi}. 

\begin{definition}
An \emph{observation relation}, $\downarrow_{\mathcal N}$, over a set
of names, $\mathcal N$, is the smallest relation satisfying the rules
below.

\infrule[Out-barb]{y \in {\mathcal N}, \; x \nameeq y}
		  {\outputp{x}{v} \downarrow_{\mathcal N} x}
\infrule[Par-barb]{\mbox{$P\downarrow_{\mathcal N} x$ or $Q\downarrow_{\mathcal N} x$}}
		  {\binpar{P}{Q} \downarrow_{\mathcal N} x}

We write $P \Downarrow_{\mathcal N} x$ if there is $Q$ such that 
$P \wred Q$ and $Q \downarrow_{\mathcal N} x$.
\end{definition}

\begin{definition}
%\label{def.bbisim}
An  ${\mathcal N}$-\emph{barbed bisimulation} over a set of names, ${\mathcal N}$, is a symmetric binary relation 
${\mathcal S}_{\mathcal N}$ between agents such that $P\rel{S}_{\mathcal N}Q$ implies:
\begin{enumerate}
\item If $P \red P'$ then $Q \wred Q'$ and $P'\rel{S}_{\mathcal N} Q'$.
\item If $P\downarrow_{\mathcal N} x$, then $Q\Downarrow_{\mathcal N} x$.
\end{enumerate}
$P$ is ${\mathcal N}$-barbed bisimilar to $Q$, written
$P \wbbisim_{\mathcal N} Q$, if $P \rel{S}_{\mathcal N} Q$ for some ${\mathcal N}$-barbed bisimulation ${\mathcal S}_{\mathcal N}$.
\end{definition}

$\mathcal{R} \subseteq \pi \times \pi$

$P \mathcal{R} Q => \forall P'. P \red P' \Rightarrow \exists Q'. Q \red Q', P' \mathcal{R} Q'$

$P \vdash x \Rightarrow Q \vdash x$

\begin{mathpar}
  \inferrule*[lab=Out-barb]{x \nameeq y}{{y}!\langle{Q}\rangle \vdash x}
  \and
  \inferrule*[lab=Par-barb]{\mbox{$P\vdash x$ or $Q\vdash x$}}{\binpar{P}{Q} \vdash x}
\end{mathpar}

\subsubsection{Contexts}

One of the principle advantages of computational calculi like the
$\pi$-calculus is a well-defined notion of context,
contextual-equivalence and a correlation between
contextual-equivalence and notions of bisimulation. The notion of
context allows the decomposition of a process into (sub-)process and
its syntactic environment, its context. Thus, a context may be
thought of as a process with a ``hole'' (written $\Box$) in it. The
application of a context $M$ to a process $P$, written $M[P]$, is
tantamount to filling the hole in $M$ with $P$. In this paper we do
not need the full weight of this theory, but do make use of the notion
of context in the proof the main theorem. 

\begin{mathpar}
  \inferrule* [lab=summation] {} {{M_{M},M_{N}} \bc \Box \;|\; x.M_{A} \;|\; M_{M}+M_{N}}
  \and
  \inferrule* [lab=agent] {} {{M_{A}} \bc (\vec{x})M_{P} \;| \; \clift{P_0,\ldots,M_{P},\ldots,P_N}}
  \and \\
  \inferrule* [lab=process] {} {{M_{P}} \bc M_{N} \;| \;P|M_{P} }
\end{mathpar} 

\begin{mathpar}
  \inferrule* [lab=sychronization] {} {M_{N} \bc \Box \;|\; x?M_{F} \;|\; x!M_{C}}
  \and
  \inferrule* [lab=abstraction] {} {{M_{F}} \bc (x)M_{P} }
  \and
  \inferrule* [lab=concretion] {} {{M_{C}} \bc \langle M_{P} \rangle }
  \and \\
  \inferrule* [lab=process] {} {{M_{P}} \bc M_{N} \;| \;P|M_{P} }
\end{mathpar}

\begin{definition}[contextual application] Given a context $M$, and
  process $P$, we define the \emph{contextual application}, $M[P] :=
  M\{P/\Box\}$. That is, the contextual application of M to P is the
  substitution of $P$ for $\Box$ in $M$.
\end{definition}

$\meaningof{-} : L \to \mathcal{P}(\pi)$

\begin{mathpar}
  \inferrule* [lab=collection] {} {\meaningof{true} = \pi, \and \meaningof{~E} = \pi \setminus \meaningof{E}, \and \meaningof{E_{1} \& E_{2}} = \meaningof{E_{1}} \cap \meaningof{E_{2}}}
\end{mathpar}

\begin{mathpar}
  \inferrule* [lab=structure] {} {\meaningof{0} = \{ P \in \pi | P \equiv 0 \}, \and \\ \meaningof{E_1 | E_2} = \{ P \in \pi | P \equiv P_{1} | P_{2}, P_{1} \in \meaningof{E_{1}}, P_{2} \in \meaningof{E_2}\} }
\end{mathpar}

\begin{mathpar}
 \inferrule* [lab=behavior] {} {\meaningof{\langle a?b \rangle E} = \{ P \in \pi | P \equiv Q | u?(y)P', \\ \and \\\\ \and \\ \;\;\; u \in \meaningof{a}, \forall z.P'\{z/y\} \in \meaningof{E\{z/b\}}\}, \and \\ \meaningof{a!E} = \{ P \in \pi | P \equiv Q | x!\langle P' \rangle, x \in \meaningof{a} P' \in \meaningof{E}\} }
\end{mathpar}

\begin{mathpar}
 \inferrule* [lab=nominal] {} {\meaningof{\quotep{E}} = \{ \quotep{P} \in \quotep{\pi} | P \in \meaningof{E} \}, \and \meaningof{\quotep{P}} = \{ \quotep{Q} \in \quotep{\pi} | P \equiv Q \} \and \\ \meaningof{@\quotep{E}} = \{ P \in \pi | P \equiv @x, x \in \meaningof{E} \}}
\end{mathpar}

\begin{eqnarray*}
  \\
  \meaningof{-} : TS \to ST
\end{eqnarray*}

\begin{eqnarray*}
  \\
  L : TS \to ST
\end{eqnarray*}

\begin{eqnarray*}
  \\
  P \models E \iff P \in \meaningof{E}
\end{eqnarray*}

\begin{eqnarray*}
  P \approx_{L} Q \iff \forall E \in L. P \models E \iff Q \models E
\end{eqnarray*}

\begin{eqnarray*}
  P \approx_{K} Q
\end{eqnarray*}

\begin{eqnarray*}
  P \approx Q
\end{eqnarray*}

$\approx_{K} = \approx = \approx_{L}$

\subsubsection{Contextual duality}

Note that contexts extend the quotation operation to a family of
operations from processes to names. Given a context, $M$, we can
define a \emph{nominal context}, $\quotep{M}$ by $\quotep{M}[P] :=
\quotep{M[P]}$. To foreshadow what is to come we observe that these
operations enjoy a duality with processes very much like the duality
between vectors and maps from vectors to scalars.

Further, because the calculus is essentially higher-order, we have a
correspondence between contexts and processes. More specifically,
given a name $x$ and a context $M$ we can construct $M^{*}_{x}$ such
that 

\begin{mathpar}
  M^{*}_{x} | \lift{x}{P} \red M[P]
\end{mathpar}

namely,

\begin{mathpar}
  M^{*}_{x} := x?(u).M[\dropn{u}]
\end{mathpar}

The dependence of $M^{*}_{x}$ on a name makes it an abstraction, 

\begin{mathpar}
  M^{*} := (x)x?(u).M[\dropn{u}]
\end{mathpar}

\subsection{Additional notation}

It will sometimes be convenient to denote the process a name
quotes. We already have the notation $x = \quotep{P}$, but it will be
convenient to introduce an alternate notation, $\procn{x}$, when we
want to emphasize the connection to the use of the name. Note that, by
virtue of name equivalence, $\quotep{\procn{x}} \nameeq x$; so, the
notation is consistent with previous definitions.

Further, because names have structure it is possible to effect
substitutions on the basis of that structure. This means we need to
upgrade our notation for substitutions, which we accomplish by
adapting comprehension notation. Thus,

\begin{mathpar}
  P\{ y / x : x \in S \}
\end{mathpar}

is interpreted to mean the process derived from P by replacing (in a
capture-avoiding manner) each occurrence of $x$ in $S$ by $y$. For example,

\begin{mathpar}
  P\{ \quotep{\procn{x}|\procn{x}} / x : x \in \freenames{P} \}
\end{mathpar}

will replace each (occurrence) of a free name $x$ in $P$ by
$\quotep{\procn{x}|\procn{x}}$.

Also, we will avail ourselves of the notation $x^{L}$ and $x^{R}$ to
denote injections of a name into disjoint copies of the name
space. There are numerous ways to accomplish this. One example can be
found in \cite{MeredithR05}. This notation overloads to vectors of
names: $\vec{x}^{\pi} := (x_{i}^{\pi} \; : \; 0 \leq i < |\vec{x}| )$ where $\pi \in \{L,R\}$.

We also use $P^{\Box} := P|\Box$.

In \cite{MeredithR05} an interpretation of the new operator is
given. It turns out that there are several possible interpretations
all enjoying the requisite algebraic properties of the operator (see
\cite{milner91polyadicpi}). We will therefore make liberal use of
$(\nu\; \vec{x})P$.

% subsection the_syntax_and_semantics_of_the_notation_system (end)   

\input{qm2pi.qmops} 

\input{qm2pi.sterngerlach} 

\input{qm2pi.metric} 

% section concurrent_process_calculi (end)

%\input{qm2pi.proofsketch}

% section proof sketch (end)

%\input{qm2pi.slviaknots} 

% section spatial logic via knots (end)

\input{qm2pi.conclusion}

% section conclusion (end)

%\input{qm2pi.dtcodes} 

% section wiring algorithm (end)

\input{qm2pi.ack} 

% section acknowledgments (end)

\newpage


\bibliographystyle{plain}   
\bibliography{../../biblios/main.bib}

\input{qm2pi.rhodetails}

\end{document}



\end{document}

 

% section concurrent_process_calculi (end)

%\documentclass[12pt]{llncs}
%\documentclass{jktr}

\usepackage[pdftex]{hyperref}                   
\usepackage {listings}
\usepackage {mathpartir}
\usepackage{bcprules}
%\usepackage{listings}
                       
\usepackage{graphicx} 
%\usepackage[margins=2.5cm,nohead,nofoot]{geometry}
%\usepackage{geometry}
\usepackage{amsfonts}
\usepackage{amstext}
\usepackage{latexsym}
\usepackage{amssymb}
\usepackage{color}


%\include{myPreamble}
\documentclass[12pt]{llncs}
%\documentclass{jktr}

\usepackage[pdftex]{hyperref}                   
\usepackage {listings}
\usepackage {mathpartir}
\usepackage{bcprules}
%\usepackage{listings}
                       
\usepackage{graphicx} 
%\usepackage[margins=2.5cm,nohead,nofoot]{geometry}
%\usepackage{geometry}
\usepackage{amsfonts}
\usepackage{amstext}
\usepackage{latexsym}
\usepackage{amssymb}
\usepackage{color}


%\include{myPreamble}
\include{qm2pi.local} 

%\ifpdf
%\usepackage[pdftex]{graphicx}
%\else
%\usepackage{graphicx}
%\fi

 % \ifpdf
%  \usepackage{pdfsync}
%  \if


%\title{Brief Article}
%\author{David F. Snyder}
%\author{L.G. Meredith}

%\address{Dept. of Math., Texas State University--San Marcos, San Marcos, TX 78666}
       
\pagestyle{empty}


\begin{document}

\lstset{language=[Objective]Caml,frame=shadowbox}

\input{qm2pi.front}

% section front matter (end)

\input{qm2pi.intro} 
 
% section introduction (end)

% \input{qm2pi.knotations} 

% section notation (end)

\input{qm2pi.process.calculi} 

% section concurrent_process_calculi_and_spatial_logics_ (end)
    
%\input{qm2pi.knots2pi} 

%\input{qm2pi.trefoil} 

%\input{qm2pi.mainthm} 

% subsection basic_interpretation (end)

%\input{qm2pi.rho.presentation} 
\subsection{The syntax and semantics of the notation system}\label{sub:the_syntax_and_semantics_of_the_notation_system} % (fold)

We now summarize a technical presentation of the calculus that
embodies our theory of dynamics. The typical presentation of such a
calculus follows the style of giving generators and relations on
them. The grammar, below, describing term constructors, freely
generates the set of processes, $\Proc$. This set is then quotiented
by a relation known as structural congruence and it is over this set
that the notion of dynamics is expressed. This presentation is
essentially that of \cite{MeredithR05} with the addition of
polyadicity and summation. For readability we have relegated some of
the technical subtleties to an appendix.

\subsubsection{Process grammar}\label{subsub:process_grammar}

\begin{mathpar}
  \inferrule* [lab=synchronization] {} {{M} \bc \pzero \;|\; x?F \;|\; x!C }
  \and
  \inferrule* [lab=abstraction] {} {{F} \bc (x)P}
  \and
  \inferrule* [lab=concretion] {} {{C} \bc \langle Q \rangle}
  \and
  \inferrule* [lab=process] {} {{P,Q} \bc M \;| \;P|Q \;|\; @{x}}
  \and
  \inferrule* [lab=name] {} {{x} \bc \quotep{P}}
\end{mathpar} 

Note that $\vec{x}$ (resp. $\vec{P}$) denotes a vector of names
(resp. processes) of length $|\vec{x}|$ (resp. $|\vec{P}|$). We adopt
the following useful abbreviations.

\begin{mathpar}
   x?(\vec{y}).P := x.(\vec{y})P \and  x\clift{\vec{P}} := x.\clift{\vec{P}}
   \and x!(y) := \lift{x}{\dropn{y}}
   \and \Pi_{i=0}^{n-1}P_i := P_0 | \ldots | P_{n-1}
\end{mathpar}

\subsubsection{Structural congruence}

\paragraph{Free and bound names and alpha-equivalence.} At the
core of structural equivalence is alpha-equivalence which identifies
process that are the same up to a change of variable. Formally, we
recognize the distinction between free and bound names. The free names
of a process, $\freenames{P}$, may be calculated recursively as
follows:

\begin{mathpar}
\freenames{\pzero} := \emptyset
  \and \\
  \freenames{x?(y).P} := \{ x \} \cup (\freenames{P} \setminus \{ y \})
  \and 
  \freenames{x!\langle P \rangle} := \{ x \} \cup \{ P \} 
  \and \\
  \freenames{P|Q} := \freenames{P} \cup \freenames{Q}
  \and \\
  \freenames{@{x}} := \{ x \}
\end{mathpar}

$\pi$
$\quotep{\pi}$

$\freenames{-} : \pi \to \mathcal{P}(\quotep{\pi})$

\begin{eqnarray*}
  \freenames{\pzero} & := & \emptyset \\
  \freenames{x?(y).P} & := & \{ x \} \cup (\freenames{P} \setminus \{ y \}) \\
  \freenames{x!\langle P \rangle} & := & \{ x \} \cup \{ P \} \\
  \freenames{P|Q} & := & \freenames{P} \cup \freenames{Q} \\
  \freenames{\dropn{x}} & := & \{ x \}
\end{eqnarray*}

The bound names of a process, $\boundnames{P}$, are those names occurring in $P$
that are not free. For example, in $x?(y).0$, the name $x$ is free, while $y$ is bound.

\begin{mathpar}
  \inferrule* [lab=monoidal-laws] {} { P|Q \equiv Q|P \and P|0 \equiv P \and P|(Q|R) \equiv (P|Q)|R }
\end{mathpar}

\begin{mathpar}
  \inferrule* [lab=alpha-equivalence] {} { (x)P \equiv (y)P\{y/x\} \and y \not\in \freenames{P} }
\end{mathpar}

\begin{definition}
Then two processes, $P,Q$, are alpha-equivalent if $P = Q\{\vec{y}/\vec{x}\}$ for
some $\vec{x} \in \boundnames{Q},\vec{y} \in \boundnames{P}$, where $Q\{\vec{y}/\vec{x}\}$
denotes the capture-avoiding substitution of $\vec{y}$ for $\vec{x}$ in $Q$.
\end{definition}

\begin{definition}
  The {\em structural congruence} \cite{SangiorgiWalker} , $\equiv$,
  between processes is the least congruence containing
  alpha-equivalence, satisfying the abelian monoid laws
  (associativity, commutativity and $\pzero$ as identity) for parallel
  composition $|$ and for summation $+$.
\end{definition}

\subsection{Name equivalence}

We take name equivalence, written $\nameeq$, to be the smallest
equivalence relation generated by the following rules.

\begin{mathpar}
\inferrule*[lab=Quote-drop]
{ }
{ \quotep{@{x}} \nameeq x }

\inferrule*[lab=Struct-equiv]
{ P \scong Q }
{ \quotep{P} \nameeq \quotep{Q} }
\end{mathpar}

The astute reader will have noticed that the mutual recursion of names
and processes imposes a mutual recursion on alpha-equivalence and
structural equivalence via name-equivalence. Fortunately, all of this
works out pleasantly and we may calculate in the natural way, free of
concern. The reader interested in the details is referred to the
appendix \ref{appendix:rho_details}.

\subsection{Substitution}

We use $\Proc$ for the set of processes, $\QProc$ for the set of
names, and $\id{\{}\vec{y} / \vec{x} \id{\}}$ to denote partial maps,
$s : \QProc \rightarrow \QProc$. A map, $s$ lifts, uniquely, to a map
on process terms, $\widehat{s} : \Proc \rightarrow \Proc$ by the
following equations.

\begin{mathpar}
  (0) \psubstp{Q}{P} := 0 \\
  (R \juxtap S) \psubstp{Q}{P}
  :=    
  (R)\psubstp{Q}{P} \juxtap (S) \psubstp{Q}{P} \\
  (x?(y).R) \psubstp{Q}{P}    
  :=    
  (x)\substp{Q}{P} (z)\concat( (R \psubstn{z}{y}) \psubstp{Q}{P} ) \\
  (\lift{x}{R}) \psubstp{Q}{P}  
  :=
  \lift{(x)\substp{Q}{P}}{ R \psubstp{Q}{P} } \\
%   (\dropn{x})  \psubstp{Q}{P}       
%   := 
%   \left\{ 
%     \begin{array}{ccc} 
%       \dropn{\quotep{Q}} & & x \nameeq \quotep{P} \\
%       \dropn{x} & & otherwise \\
%     \end{array}
%   \right. 
  (\dropn{x})  \psubstp{Q}{P}       
  := 
  \left\{ 
    \begin{array}{ccc} 
      Q & & x \nameeq \quotep{P} \\
      \dropn{x} & & otherwise \\
    \end{array}
  \right.
\end{mathpar}
 

where

\begin{eqnarray}
  (x)\id{\{} \lpquote Q \rpquote / \lpquote P \rpquote \id{\}}            = 
  \left\{ 
    \begin{array}{ccc}
      \lpquote Q \rpquote & & x \nameeq \lpquote P \rpquote \\
      x & & otherwise \\
    \end{array}
  \right. \nonumber
\end{eqnarray}

and $z$ is chosen distinct from $\quotep{P}$, $\quotep{Q}$, the free
names in $Q$, and all the names in $R$. Our $\alpha$-equivalence will
be built in the standard way from this substitution.

\begin{remark}\label{rem:no_self_referential_names}
  One consequence of these definitions is that $\forall P. \quotep{P}
  \not\in \freenames{P}$.
\end{remark}

\subsection{ Dynamic quote: an example }

Anticipating something of what's to come, consider applying the
substitution, $\widehat{\id{\{}u / z \id{\}}}$, to the following pair
of processes, $\lift{w}{y!(z)}$ and $w[ \lpquote y!(z) \rpquote ]$.

\begin{eqnarray}
	\lift{w}{y!(z)}\widehat{\id{\{}u / z \id{\}}}
		& = &
		\lift{w}{y!(u)} \nonumber\\
	w[ \lpquote y!(z) \rpquote ] \widehat{ \id{\{}u / z \id{\}} }
		& = &
		w[ \lpquote y!(z) \rpquote ] \nonumber
\end{eqnarray}

Because the body of the process between quotes is impervious to
substitution, we get radically different answers. In fact, by
examining the first process in an input context,
e.g. $x?(z).\lift{w}{y!(z)}$, we see that the process under the lift
operator may be shaped by prefixed inputs binding a name inside it. In
this sense, the lift operator will be seen as a way to dynamically
construct processes before reifying them as names.

Finally equipped with these standard features we can present the
dynamics of the calculus.

\subsubsection{Operational semantics} 

Finally, we introduce the computational dynamics. What marks these
algebras as distinct from other more traditionally studied algebraic
structures, e.g. vector spaces or polynomial rings, is the manner in
which dynamics is captured. In traditional structures, dynamics is typically
expressed through morphisms between such structures, as in linear maps
between vector spaces or morphisms between rings. In algebras
associated with the semantics of computation, the dynamics is
expressed as part of the algebraic structure itself, through a
reduction reduction relation typically denoted by $\red$. Below, we
give a recursive presentation of this relation for the calculus used
in the encoding.

$\red \subseteq \pi \times \pi$
$\red : \pi \to \mathcal{P}(\pi)$

\begin{mathpar}
  \inferrule* [lab=Comm] { \textsf{match}( x_{src}, x_{trgt} ) } { x_{trgt}?(y)P \; | \; x_{src}!\langle {Q} \rangle \red P\{\quotep{Q}/y}\} }
  \and \\
  \inferrule* [lab=Par] {{P} \red {P}'} {{{P} | {Q}} \red {{P}' | {Q}}}
  \and
  \inferrule* [lab=Equiv]{{{P} \scong {P}'} \andalso {{P}' \red {Q}'} \andalso {{Q}' \scong {Q}}}{{P} \red {Q}}
\end{mathpar}

\begin{eqnarray*}
  match_{\equiv} (\quotep{P},\quotep{Q}) & := & P \equiv Q \\
  match_{\dagger}(\quotep{P},\quotep{Q}) & := & \forall R. P|Q \red^{*} R => R \red^{*} 0 \\
  match_{K}(\quotep{P},\quotep{Q}) & := & K \mbox{ for some context } K
\end{eqnarray*}

$u?(x)P | u!\langle Q \rangle \red P\{\quotep{Q}/x\}$

%We write $\wred$ for $\red^*$, and $P\red$ if $\exists Q $ such that $ P \red Q$.
We write $P\red$ if $\exists Q $ such that $ P \red Q$ and $P\not\red$, otherwise.

\section{Replication}

As mentioned before, it is known that replication (and hence
recursion) can be implemented in a higher-order process algebra
\cite{SangiorgiWalker}. As our first example of calculation with the
machinery thus far presented we give the construction explicitly in
the {\rhoc}.

\begin{eqnarray}
	D_{x} & := & \prefix{x}{y}{(\binpar{\outputp{x}{y}}{@{y}})} \nonumber\\
	\bangp_{x}{P} & := & \binpar{{x}!\langle{\binpar{D_{x}}{P}}\rangle}{D_{x}} \nonumber
\end{eqnarray}

\begin{eqnarray}
	\bangp_{x}{P} & & \nonumber\\
	=
	& {x}!\langle{(\prefix{x}{y}{(\outputp{x}{y} | @{y})) | P}}\rangle 
	      | \prefix{x}{y}{(\outputp{x}{y} | @{y})} & \nonumber\\
	\red
	& (\outputp{x}{y} | @{y})\substn{\quotep{(\prefix{x}{y}{(@{y} | \outputp{x}{y})) | P}}}{y} & \nonumber\\
	=
	& \outputp{x}{\quotep{(\prefix{x}{y}{(\outputp{x}{y} | @{y})) | P}}}
	  | {(\prefix{x}{y}{(\outputp{x}{y} | @{y})) | P}} & \nonumber\\
	\red
	& \ldots & \nonumber\\
	\red^*
	& P | P | \ldots & \nonumber
\end{eqnarray}

Of course, this encoding, as an implementation, runs away, unfolding
$\bangp{P}$ eagerly. A lazier and more implementable replication
operator, restricted to input-guarded processes, may be obtained as follows.

\begin{eqnarray}
\bangp{\prefix{u}{v}{P}} 
	:= 
	\binpar{\lift{x}{\prefix{u}{v}{(\binpar{D(x)}{P})}}}{D(x)} \nonumber
\end{eqnarray}

\begin{remark}
  Note that the lazier definition still does not deal with summation
  or mixed summation (i.e. sums over input and output). The reader is
  invited to construct definitions of replication that deal with these
  features. 

  Further, the definitions are parameterized in a name, $x$. Can you,
  gentle reader, make a definition that eliminates this parameter and
  guarantees no accidental interaction between the replication
  machinery and the process being replicated -- i.e. no accidental
  sharing of names used by the process to get its work done and the
  name(s) used by the replication to effect copying. This latter
  revision of the definition of replication is crucial to obtaining
  the expected identity $!!P \sim !P$.
\end{remark}

\begin{remark}\label{rem:paradoxical_combinator}
  The reader familiar with the lambda calculus will have noticed the
  similarity between $D$ and the paradoxical combinator.

  [Ed. note: the existence of this seems to suggest we have to be more
  restrictive on the set of processes and names we admit if we are to
  support no-cloning.]
\end{remark}

\subsubsection{Bisimulation}

The computational dynamics gives rise to another kind of equivalence,
the equivalence of computational behavior. As previously mentioned
this is typically captured \emph{via} some form of bisimulation.

% The notion we use in this paper is weak barbed bisimulation
% \cite{milner91polyadicpi}.

The notion we use in this paper is derived from weak barbed
bisimulation \cite{milner91polyadicpi}. 

\begin{definition}
An \emph{observation relation}, $\downarrow_{\mathcal N}$, over a set
of names, $\mathcal N$, is the smallest relation satisfying the rules
below.

\infrule[Out-barb]{y \in {\mathcal N}, \; x \nameeq y}
		  {\outputp{x}{v} \downarrow_{\mathcal N} x}
\infrule[Par-barb]{\mbox{$P\downarrow_{\mathcal N} x$ or $Q\downarrow_{\mathcal N} x$}}
		  {\binpar{P}{Q} \downarrow_{\mathcal N} x}

We write $P \Downarrow_{\mathcal N} x$ if there is $Q$ such that 
$P \wred Q$ and $Q \downarrow_{\mathcal N} x$.
\end{definition}

\begin{definition}
%\label{def.bbisim}
An  ${\mathcal N}$-\emph{barbed bisimulation} over a set of names, ${\mathcal N}$, is a symmetric binary relation 
${\mathcal S}_{\mathcal N}$ between agents such that $P\rel{S}_{\mathcal N}Q$ implies:
\begin{enumerate}
\item If $P \red P'$ then $Q \wred Q'$ and $P'\rel{S}_{\mathcal N} Q'$.
\item If $P\downarrow_{\mathcal N} x$, then $Q\Downarrow_{\mathcal N} x$.
\end{enumerate}
$P$ is ${\mathcal N}$-barbed bisimilar to $Q$, written
$P \wbbisim_{\mathcal N} Q$, if $P \rel{S}_{\mathcal N} Q$ for some ${\mathcal N}$-barbed bisimulation ${\mathcal S}_{\mathcal N}$.
\end{definition}

$\mathcal{R} \subseteq \pi \times \pi$

$P \mathcal{R} Q => \forall P'. P \red P' \Rightarrow \exists Q'. Q \red Q', P' \mathcal{R} Q'$

$P \vdash x \Rightarrow Q \vdash x$

\begin{mathpar}
  \inferrule*[lab=Out-barb]{x \nameeq y}{{y}!\langle{Q}\rangle \vdash x}
  \and
  \inferrule*[lab=Par-barb]{\mbox{$P\vdash x$ or $Q\vdash x$}}{\binpar{P}{Q} \vdash x}
\end{mathpar}

\subsubsection{Contexts}

One of the principle advantages of computational calculi like the
$\pi$-calculus is a well-defined notion of context,
contextual-equivalence and a correlation between
contextual-equivalence and notions of bisimulation. The notion of
context allows the decomposition of a process into (sub-)process and
its syntactic environment, its context. Thus, a context may be
thought of as a process with a ``hole'' (written $\Box$) in it. The
application of a context $M$ to a process $P$, written $M[P]$, is
tantamount to filling the hole in $M$ with $P$. In this paper we do
not need the full weight of this theory, but do make use of the notion
of context in the proof the main theorem. 

\begin{mathpar}
  \inferrule* [lab=summation] {} {{M_{M},M_{N}} \bc \Box \;|\; x.M_{A} \;|\; M_{M}+M_{N}}
  \and
  \inferrule* [lab=agent] {} {{M_{A}} \bc (\vec{x})M_{P} \;| \; \clift{P_0,\ldots,M_{P},\ldots,P_N}}
  \and \\
  \inferrule* [lab=process] {} {{M_{P}} \bc M_{N} \;| \;P|M_{P} }
\end{mathpar} 

\begin{mathpar}
  \inferrule* [lab=sychronization] {} {M_{N} \bc \Box \;|\; x?M_{F} \;|\; x!M_{C}}
  \and
  \inferrule* [lab=abstraction] {} {{M_{F}} \bc (x)M_{P} }
  \and
  \inferrule* [lab=concretion] {} {{M_{C}} \bc \langle M_{P} \rangle }
  \and \\
  \inferrule* [lab=process] {} {{M_{P}} \bc M_{N} \;| \;P|M_{P} }
\end{mathpar}

\begin{definition}[contextual application] Given a context $M$, and
  process $P$, we define the \emph{contextual application}, $M[P] :=
  M\{P/\Box\}$. That is, the contextual application of M to P is the
  substitution of $P$ for $\Box$ in $M$.
\end{definition}

$\meaningof{-} : L \to \mathcal{P}(\pi)$

\begin{mathpar}
  \inferrule* [lab=collection] {} {\meaningof{true} = \pi, \and \meaningof{~E} = \pi \setminus \meaningof{E}, \and \meaningof{E_{1} \& E_{2}} = \meaningof{E_{1}} \cap \meaningof{E_{2}}}
\end{mathpar}

\begin{mathpar}
  \inferrule* [lab=structure] {} {\meaningof{0} = \{ P \in \pi | P \equiv 0 \}, \and \\ \meaningof{E_1 | E_2} = \{ P \in \pi | P \equiv P_{1} | P_{2}, P_{1} \in \meaningof{E_{1}}, P_{2} \in \meaningof{E_2}\} }
\end{mathpar}

\begin{mathpar}
 \inferrule* [lab=behavior] {} {\meaningof{\langle a?b \rangle E} = \{ P \in \pi | P \equiv Q | u?(y)P', \\ \and \\\\ \and \\ \;\;\; u \in \meaningof{a}, \forall z.P'\{z/y\} \in \meaningof{E\{z/b\}}\}, \and \\ \meaningof{a!E} = \{ P \in \pi | P \equiv Q | x!\langle P' \rangle, x \in \meaningof{a} P' \in \meaningof{E}\} }
\end{mathpar}

\begin{mathpar}
 \inferrule* [lab=nominal] {} {\meaningof{\quotep{E}} = \{ \quotep{P} \in \quotep{\pi} | P \in \meaningof{E} \}, \and \meaningof{\quotep{P}} = \{ \quotep{Q} \in \quotep{\pi} | P \equiv Q \} \and \\ \meaningof{@\quotep{E}} = \{ P \in \pi | P \equiv @x, x \in \meaningof{E} \}}
\end{mathpar}

\begin{eqnarray*}
  \\
  \meaningof{-} : TS \to ST
\end{eqnarray*}

\begin{eqnarray*}
  \\
  L : TS \to ST
\end{eqnarray*}

\begin{eqnarray*}
  \\
  P \models E \iff P \in \meaningof{E}
\end{eqnarray*}

\begin{eqnarray*}
  P \approx_{L} Q \iff \forall E \in L. P \models E \iff Q \models E
\end{eqnarray*}

\begin{eqnarray*}
  P \approx_{K} Q
\end{eqnarray*}

\begin{eqnarray*}
  P \approx Q
\end{eqnarray*}

$\approx_{K} = \approx = \approx_{L}$

\subsubsection{Contextual duality}

Note that contexts extend the quotation operation to a family of
operations from processes to names. Given a context, $M$, we can
define a \emph{nominal context}, $\quotep{M}$ by $\quotep{M}[P] :=
\quotep{M[P]}$. To foreshadow what is to come we observe that these
operations enjoy a duality with processes very much like the duality
between vectors and maps from vectors to scalars.

Further, because the calculus is essentially higher-order, we have a
correspondence between contexts and processes. More specifically,
given a name $x$ and a context $M$ we can construct $M^{*}_{x}$ such
that 

\begin{mathpar}
  M^{*}_{x} | \lift{x}{P} \red M[P]
\end{mathpar}

namely,

\begin{mathpar}
  M^{*}_{x} := x?(u).M[\dropn{u}]
\end{mathpar}

The dependence of $M^{*}_{x}$ on a name makes it an abstraction, 

\begin{mathpar}
  M^{*} := (x)x?(u).M[\dropn{u}]
\end{mathpar}

\subsection{Additional notation}

It will sometimes be convenient to denote the process a name
quotes. We already have the notation $x = \quotep{P}$, but it will be
convenient to introduce an alternate notation, $\procn{x}$, when we
want to emphasize the connection to the use of the name. Note that, by
virtue of name equivalence, $\quotep{\procn{x}} \nameeq x$; so, the
notation is consistent with previous definitions.

Further, because names have structure it is possible to effect
substitutions on the basis of that structure. This means we need to
upgrade our notation for substitutions, which we accomplish by
adapting comprehension notation. Thus,

\begin{mathpar}
  P\{ y / x : x \in S \}
\end{mathpar}

is interpreted to mean the process derived from P by replacing (in a
capture-avoiding manner) each occurrence of $x$ in $S$ by $y$. For example,

\begin{mathpar}
  P\{ \quotep{\procn{x}|\procn{x}} / x : x \in \freenames{P} \}
\end{mathpar}

will replace each (occurrence) of a free name $x$ in $P$ by
$\quotep{\procn{x}|\procn{x}}$.

Also, we will avail ourselves of the notation $x^{L}$ and $x^{R}$ to
denote injections of a name into disjoint copies of the name
space. There are numerous ways to accomplish this. One example can be
found in \cite{MeredithR05}. This notation overloads to vectors of
names: $\vec{x}^{\pi} := (x_{i}^{\pi} \; : \; 0 \leq i < |\vec{x}| )$ where $\pi \in \{L,R\}$.

We also use $P^{\Box} := P|\Box$.

In \cite{MeredithR05} an interpretation of the new operator is
given. It turns out that there are several possible interpretations
all enjoying the requisite algebraic properties of the operator (see
\cite{milner91polyadicpi}). We will therefore make liberal use of
$(\nu\; \vec{x})P$.

% subsection the_syntax_and_semantics_of_the_notation_system (end)   

\input{qm2pi.qmops} 

\input{qm2pi.sterngerlach} 

\input{qm2pi.metric} 

% section concurrent_process_calculi (end)

%\input{qm2pi.proofsketch}

% section proof sketch (end)

%\input{qm2pi.slviaknots} 

% section spatial logic via knots (end)

\input{qm2pi.conclusion}

% section conclusion (end)

%\input{qm2pi.dtcodes} 

% section wiring algorithm (end)

\input{qm2pi.ack} 

% section acknowledgments (end)

\newpage


\bibliographystyle{plain}   
\bibliography{../../biblios/main.bib}

\input{qm2pi.rhodetails}

\end{document}

 

%\ifpdf
%\usepackage[pdftex]{graphicx}
%\else
%\usepackage{graphicx}
%\fi

 % \ifpdf
%  \usepackage{pdfsync}
%  \if


%\title{Brief Article}
%\author{David F. Snyder}
%\author{L.G. Meredith}

%\address{Dept. of Math., Texas State University--San Marcos, San Marcos, TX 78666}
       
\pagestyle{empty}


\begin{document}

\lstset{language=[Objective]Caml,frame=shadowbox}

\documentclass[12pt]{llncs}
%\documentclass{jktr}

\usepackage[pdftex]{hyperref}                   
\usepackage {listings}
\usepackage {mathpartir}
\usepackage{bcprules}
%\usepackage{listings}
                       
\usepackage{graphicx} 
%\usepackage[margins=2.5cm,nohead,nofoot]{geometry}
%\usepackage{geometry}
\usepackage{amsfonts}
\usepackage{amstext}
\usepackage{latexsym}
\usepackage{amssymb}
\usepackage{color}


%\include{myPreamble}
\include{qm2pi.local} 

%\ifpdf
%\usepackage[pdftex]{graphicx}
%\else
%\usepackage{graphicx}
%\fi

 % \ifpdf
%  \usepackage{pdfsync}
%  \if


%\title{Brief Article}
%\author{David F. Snyder}
%\author{L.G. Meredith}

%\address{Dept. of Math., Texas State University--San Marcos, San Marcos, TX 78666}
       
\pagestyle{empty}


\begin{document}

\lstset{language=[Objective]Caml,frame=shadowbox}

\input{qm2pi.front}

% section front matter (end)

\input{qm2pi.intro} 
 
% section introduction (end)

% \input{qm2pi.knotations} 

% section notation (end)

\input{qm2pi.process.calculi} 

% section concurrent_process_calculi_and_spatial_logics_ (end)
    
%\input{qm2pi.knots2pi} 

%\input{qm2pi.trefoil} 

%\input{qm2pi.mainthm} 

% subsection basic_interpretation (end)

%\input{qm2pi.rho.presentation} 
\subsection{The syntax and semantics of the notation system}\label{sub:the_syntax_and_semantics_of_the_notation_system} % (fold)

We now summarize a technical presentation of the calculus that
embodies our theory of dynamics. The typical presentation of such a
calculus follows the style of giving generators and relations on
them. The grammar, below, describing term constructors, freely
generates the set of processes, $\Proc$. This set is then quotiented
by a relation known as structural congruence and it is over this set
that the notion of dynamics is expressed. This presentation is
essentially that of \cite{MeredithR05} with the addition of
polyadicity and summation. For readability we have relegated some of
the technical subtleties to an appendix.

\subsubsection{Process grammar}\label{subsub:process_grammar}

\begin{mathpar}
  \inferrule* [lab=synchronization] {} {{M} \bc \pzero \;|\; x?F \;|\; x!C }
  \and
  \inferrule* [lab=abstraction] {} {{F} \bc (x)P}
  \and
  \inferrule* [lab=concretion] {} {{C} \bc \langle Q \rangle}
  \and
  \inferrule* [lab=process] {} {{P,Q} \bc M \;| \;P|Q \;|\; @{x}}
  \and
  \inferrule* [lab=name] {} {{x} \bc \quotep{P}}
\end{mathpar} 

Note that $\vec{x}$ (resp. $\vec{P}$) denotes a vector of names
(resp. processes) of length $|\vec{x}|$ (resp. $|\vec{P}|$). We adopt
the following useful abbreviations.

\begin{mathpar}
   x?(\vec{y}).P := x.(\vec{y})P \and  x\clift{\vec{P}} := x.\clift{\vec{P}}
   \and x!(y) := \lift{x}{\dropn{y}}
   \and \Pi_{i=0}^{n-1}P_i := P_0 | \ldots | P_{n-1}
\end{mathpar}

\subsubsection{Structural congruence}

\paragraph{Free and bound names and alpha-equivalence.} At the
core of structural equivalence is alpha-equivalence which identifies
process that are the same up to a change of variable. Formally, we
recognize the distinction between free and bound names. The free names
of a process, $\freenames{P}$, may be calculated recursively as
follows:

\begin{mathpar}
\freenames{\pzero} := \emptyset
  \and \\
  \freenames{x?(y).P} := \{ x \} \cup (\freenames{P} \setminus \{ y \})
  \and 
  \freenames{x!\langle P \rangle} := \{ x \} \cup \{ P \} 
  \and \\
  \freenames{P|Q} := \freenames{P} \cup \freenames{Q}
  \and \\
  \freenames{@{x}} := \{ x \}
\end{mathpar}

$\pi$
$\quotep{\pi}$

$\freenames{-} : \pi \to \mathcal{P}(\quotep{\pi})$

\begin{eqnarray*}
  \freenames{\pzero} & := & \emptyset \\
  \freenames{x?(y).P} & := & \{ x \} \cup (\freenames{P} \setminus \{ y \}) \\
  \freenames{x!\langle P \rangle} & := & \{ x \} \cup \{ P \} \\
  \freenames{P|Q} & := & \freenames{P} \cup \freenames{Q} \\
  \freenames{\dropn{x}} & := & \{ x \}
\end{eqnarray*}

The bound names of a process, $\boundnames{P}$, are those names occurring in $P$
that are not free. For example, in $x?(y).0$, the name $x$ is free, while $y$ is bound.

\begin{mathpar}
  \inferrule* [lab=monoidal-laws] {} { P|Q \equiv Q|P \and P|0 \equiv P \and P|(Q|R) \equiv (P|Q)|R }
\end{mathpar}

\begin{mathpar}
  \inferrule* [lab=alpha-equivalence] {} { (x)P \equiv (y)P\{y/x\} \and y \not\in \freenames{P} }
\end{mathpar}

\begin{definition}
Then two processes, $P,Q$, are alpha-equivalent if $P = Q\{\vec{y}/\vec{x}\}$ for
some $\vec{x} \in \boundnames{Q},\vec{y} \in \boundnames{P}$, where $Q\{\vec{y}/\vec{x}\}$
denotes the capture-avoiding substitution of $\vec{y}$ for $\vec{x}$ in $Q$.
\end{definition}

\begin{definition}
  The {\em structural congruence} \cite{SangiorgiWalker} , $\equiv$,
  between processes is the least congruence containing
  alpha-equivalence, satisfying the abelian monoid laws
  (associativity, commutativity and $\pzero$ as identity) for parallel
  composition $|$ and for summation $+$.
\end{definition}

\subsection{Name equivalence}

We take name equivalence, written $\nameeq$, to be the smallest
equivalence relation generated by the following rules.

\begin{mathpar}
\inferrule*[lab=Quote-drop]
{ }
{ \quotep{@{x}} \nameeq x }

\inferrule*[lab=Struct-equiv]
{ P \scong Q }
{ \quotep{P} \nameeq \quotep{Q} }
\end{mathpar}

The astute reader will have noticed that the mutual recursion of names
and processes imposes a mutual recursion on alpha-equivalence and
structural equivalence via name-equivalence. Fortunately, all of this
works out pleasantly and we may calculate in the natural way, free of
concern. The reader interested in the details is referred to the
appendix \ref{appendix:rho_details}.

\subsection{Substitution}

We use $\Proc$ for the set of processes, $\QProc$ for the set of
names, and $\id{\{}\vec{y} / \vec{x} \id{\}}$ to denote partial maps,
$s : \QProc \rightarrow \QProc$. A map, $s$ lifts, uniquely, to a map
on process terms, $\widehat{s} : \Proc \rightarrow \Proc$ by the
following equations.

\begin{mathpar}
  (0) \psubstp{Q}{P} := 0 \\
  (R \juxtap S) \psubstp{Q}{P}
  :=    
  (R)\psubstp{Q}{P} \juxtap (S) \psubstp{Q}{P} \\
  (x?(y).R) \psubstp{Q}{P}    
  :=    
  (x)\substp{Q}{P} (z)\concat( (R \psubstn{z}{y}) \psubstp{Q}{P} ) \\
  (\lift{x}{R}) \psubstp{Q}{P}  
  :=
  \lift{(x)\substp{Q}{P}}{ R \psubstp{Q}{P} } \\
%   (\dropn{x})  \psubstp{Q}{P}       
%   := 
%   \left\{ 
%     \begin{array}{ccc} 
%       \dropn{\quotep{Q}} & & x \nameeq \quotep{P} \\
%       \dropn{x} & & otherwise \\
%     \end{array}
%   \right. 
  (\dropn{x})  \psubstp{Q}{P}       
  := 
  \left\{ 
    \begin{array}{ccc} 
      Q & & x \nameeq \quotep{P} \\
      \dropn{x} & & otherwise \\
    \end{array}
  \right.
\end{mathpar}
 

where

\begin{eqnarray}
  (x)\id{\{} \lpquote Q \rpquote / \lpquote P \rpquote \id{\}}            = 
  \left\{ 
    \begin{array}{ccc}
      \lpquote Q \rpquote & & x \nameeq \lpquote P \rpquote \\
      x & & otherwise \\
    \end{array}
  \right. \nonumber
\end{eqnarray}

and $z$ is chosen distinct from $\quotep{P}$, $\quotep{Q}$, the free
names in $Q$, and all the names in $R$. Our $\alpha$-equivalence will
be built in the standard way from this substitution.

\begin{remark}\label{rem:no_self_referential_names}
  One consequence of these definitions is that $\forall P. \quotep{P}
  \not\in \freenames{P}$.
\end{remark}

\subsection{ Dynamic quote: an example }

Anticipating something of what's to come, consider applying the
substitution, $\widehat{\id{\{}u / z \id{\}}}$, to the following pair
of processes, $\lift{w}{y!(z)}$ and $w[ \lpquote y!(z) \rpquote ]$.

\begin{eqnarray}
	\lift{w}{y!(z)}\widehat{\id{\{}u / z \id{\}}}
		& = &
		\lift{w}{y!(u)} \nonumber\\
	w[ \lpquote y!(z) \rpquote ] \widehat{ \id{\{}u / z \id{\}} }
		& = &
		w[ \lpquote y!(z) \rpquote ] \nonumber
\end{eqnarray}

Because the body of the process between quotes is impervious to
substitution, we get radically different answers. In fact, by
examining the first process in an input context,
e.g. $x?(z).\lift{w}{y!(z)}$, we see that the process under the lift
operator may be shaped by prefixed inputs binding a name inside it. In
this sense, the lift operator will be seen as a way to dynamically
construct processes before reifying them as names.

Finally equipped with these standard features we can present the
dynamics of the calculus.

\subsubsection{Operational semantics} 

Finally, we introduce the computational dynamics. What marks these
algebras as distinct from other more traditionally studied algebraic
structures, e.g. vector spaces or polynomial rings, is the manner in
which dynamics is captured. In traditional structures, dynamics is typically
expressed through morphisms between such structures, as in linear maps
between vector spaces or morphisms between rings. In algebras
associated with the semantics of computation, the dynamics is
expressed as part of the algebraic structure itself, through a
reduction reduction relation typically denoted by $\red$. Below, we
give a recursive presentation of this relation for the calculus used
in the encoding.

$\red \subseteq \pi \times \pi$
$\red : \pi \to \mathcal{P}(\pi)$

\begin{mathpar}
  \inferrule* [lab=Comm] { \textsf{match}( x_{src}, x_{trgt} ) } { x_{trgt}?(y)P \; | \; x_{src}!\langle {Q} \rangle \red P\{\quotep{Q}/y}\} }
  \and \\
  \inferrule* [lab=Par] {{P} \red {P}'} {{{P} | {Q}} \red {{P}' | {Q}}}
  \and
  \inferrule* [lab=Equiv]{{{P} \scong {P}'} \andalso {{P}' \red {Q}'} \andalso {{Q}' \scong {Q}}}{{P} \red {Q}}
\end{mathpar}

\begin{eqnarray*}
  match_{\equiv} (\quotep{P},\quotep{Q}) & := & P \equiv Q \\
  match_{\dagger}(\quotep{P},\quotep{Q}) & := & \forall R. P|Q \red^{*} R => R \red^{*} 0 \\
  match_{K}(\quotep{P},\quotep{Q}) & := & K \mbox{ for some context } K
\end{eqnarray*}

$u?(x)P | u!\langle Q \rangle \red P\{\quotep{Q}/x\}$

%We write $\wred$ for $\red^*$, and $P\red$ if $\exists Q $ such that $ P \red Q$.
We write $P\red$ if $\exists Q $ such that $ P \red Q$ and $P\not\red$, otherwise.

\section{Replication}

As mentioned before, it is known that replication (and hence
recursion) can be implemented in a higher-order process algebra
\cite{SangiorgiWalker}. As our first example of calculation with the
machinery thus far presented we give the construction explicitly in
the {\rhoc}.

\begin{eqnarray}
	D_{x} & := & \prefix{x}{y}{(\binpar{\outputp{x}{y}}{@{y}})} \nonumber\\
	\bangp_{x}{P} & := & \binpar{{x}!\langle{\binpar{D_{x}}{P}}\rangle}{D_{x}} \nonumber
\end{eqnarray}

\begin{eqnarray}
	\bangp_{x}{P} & & \nonumber\\
	=
	& {x}!\langle{(\prefix{x}{y}{(\outputp{x}{y} | @{y})) | P}}\rangle 
	      | \prefix{x}{y}{(\outputp{x}{y} | @{y})} & \nonumber\\
	\red
	& (\outputp{x}{y} | @{y})\substn{\quotep{(\prefix{x}{y}{(@{y} | \outputp{x}{y})) | P}}}{y} & \nonumber\\
	=
	& \outputp{x}{\quotep{(\prefix{x}{y}{(\outputp{x}{y} | @{y})) | P}}}
	  | {(\prefix{x}{y}{(\outputp{x}{y} | @{y})) | P}} & \nonumber\\
	\red
	& \ldots & \nonumber\\
	\red^*
	& P | P | \ldots & \nonumber
\end{eqnarray}

Of course, this encoding, as an implementation, runs away, unfolding
$\bangp{P}$ eagerly. A lazier and more implementable replication
operator, restricted to input-guarded processes, may be obtained as follows.

\begin{eqnarray}
\bangp{\prefix{u}{v}{P}} 
	:= 
	\binpar{\lift{x}{\prefix{u}{v}{(\binpar{D(x)}{P})}}}{D(x)} \nonumber
\end{eqnarray}

\begin{remark}
  Note that the lazier definition still does not deal with summation
  or mixed summation (i.e. sums over input and output). The reader is
  invited to construct definitions of replication that deal with these
  features. 

  Further, the definitions are parameterized in a name, $x$. Can you,
  gentle reader, make a definition that eliminates this parameter and
  guarantees no accidental interaction between the replication
  machinery and the process being replicated -- i.e. no accidental
  sharing of names used by the process to get its work done and the
  name(s) used by the replication to effect copying. This latter
  revision of the definition of replication is crucial to obtaining
  the expected identity $!!P \sim !P$.
\end{remark}

\begin{remark}\label{rem:paradoxical_combinator}
  The reader familiar with the lambda calculus will have noticed the
  similarity between $D$ and the paradoxical combinator.

  [Ed. note: the existence of this seems to suggest we have to be more
  restrictive on the set of processes and names we admit if we are to
  support no-cloning.]
\end{remark}

\subsubsection{Bisimulation}

The computational dynamics gives rise to another kind of equivalence,
the equivalence of computational behavior. As previously mentioned
this is typically captured \emph{via} some form of bisimulation.

% The notion we use in this paper is weak barbed bisimulation
% \cite{milner91polyadicpi}.

The notion we use in this paper is derived from weak barbed
bisimulation \cite{milner91polyadicpi}. 

\begin{definition}
An \emph{observation relation}, $\downarrow_{\mathcal N}$, over a set
of names, $\mathcal N$, is the smallest relation satisfying the rules
below.

\infrule[Out-barb]{y \in {\mathcal N}, \; x \nameeq y}
		  {\outputp{x}{v} \downarrow_{\mathcal N} x}
\infrule[Par-barb]{\mbox{$P\downarrow_{\mathcal N} x$ or $Q\downarrow_{\mathcal N} x$}}
		  {\binpar{P}{Q} \downarrow_{\mathcal N} x}

We write $P \Downarrow_{\mathcal N} x$ if there is $Q$ such that 
$P \wred Q$ and $Q \downarrow_{\mathcal N} x$.
\end{definition}

\begin{definition}
%\label{def.bbisim}
An  ${\mathcal N}$-\emph{barbed bisimulation} over a set of names, ${\mathcal N}$, is a symmetric binary relation 
${\mathcal S}_{\mathcal N}$ between agents such that $P\rel{S}_{\mathcal N}Q$ implies:
\begin{enumerate}
\item If $P \red P'$ then $Q \wred Q'$ and $P'\rel{S}_{\mathcal N} Q'$.
\item If $P\downarrow_{\mathcal N} x$, then $Q\Downarrow_{\mathcal N} x$.
\end{enumerate}
$P$ is ${\mathcal N}$-barbed bisimilar to $Q$, written
$P \wbbisim_{\mathcal N} Q$, if $P \rel{S}_{\mathcal N} Q$ for some ${\mathcal N}$-barbed bisimulation ${\mathcal S}_{\mathcal N}$.
\end{definition}

$\mathcal{R} \subseteq \pi \times \pi$

$P \mathcal{R} Q => \forall P'. P \red P' \Rightarrow \exists Q'. Q \red Q', P' \mathcal{R} Q'$

$P \vdash x \Rightarrow Q \vdash x$

\begin{mathpar}
  \inferrule*[lab=Out-barb]{x \nameeq y}{{y}!\langle{Q}\rangle \vdash x}
  \and
  \inferrule*[lab=Par-barb]{\mbox{$P\vdash x$ or $Q\vdash x$}}{\binpar{P}{Q} \vdash x}
\end{mathpar}

\subsubsection{Contexts}

One of the principle advantages of computational calculi like the
$\pi$-calculus is a well-defined notion of context,
contextual-equivalence and a correlation between
contextual-equivalence and notions of bisimulation. The notion of
context allows the decomposition of a process into (sub-)process and
its syntactic environment, its context. Thus, a context may be
thought of as a process with a ``hole'' (written $\Box$) in it. The
application of a context $M$ to a process $P$, written $M[P]$, is
tantamount to filling the hole in $M$ with $P$. In this paper we do
not need the full weight of this theory, but do make use of the notion
of context in the proof the main theorem. 

\begin{mathpar}
  \inferrule* [lab=summation] {} {{M_{M},M_{N}} \bc \Box \;|\; x.M_{A} \;|\; M_{M}+M_{N}}
  \and
  \inferrule* [lab=agent] {} {{M_{A}} \bc (\vec{x})M_{P} \;| \; \clift{P_0,\ldots,M_{P},\ldots,P_N}}
  \and \\
  \inferrule* [lab=process] {} {{M_{P}} \bc M_{N} \;| \;P|M_{P} }
\end{mathpar} 

\begin{mathpar}
  \inferrule* [lab=sychronization] {} {M_{N} \bc \Box \;|\; x?M_{F} \;|\; x!M_{C}}
  \and
  \inferrule* [lab=abstraction] {} {{M_{F}} \bc (x)M_{P} }
  \and
  \inferrule* [lab=concretion] {} {{M_{C}} \bc \langle M_{P} \rangle }
  \and \\
  \inferrule* [lab=process] {} {{M_{P}} \bc M_{N} \;| \;P|M_{P} }
\end{mathpar}

\begin{definition}[contextual application] Given a context $M$, and
  process $P$, we define the \emph{contextual application}, $M[P] :=
  M\{P/\Box\}$. That is, the contextual application of M to P is the
  substitution of $P$ for $\Box$ in $M$.
\end{definition}

$\meaningof{-} : L \to \mathcal{P}(\pi)$

\begin{mathpar}
  \inferrule* [lab=collection] {} {\meaningof{true} = \pi, \and \meaningof{~E} = \pi \setminus \meaningof{E}, \and \meaningof{E_{1} \& E_{2}} = \meaningof{E_{1}} \cap \meaningof{E_{2}}}
\end{mathpar}

\begin{mathpar}
  \inferrule* [lab=structure] {} {\meaningof{0} = \{ P \in \pi | P \equiv 0 \}, \and \\ \meaningof{E_1 | E_2} = \{ P \in \pi | P \equiv P_{1} | P_{2}, P_{1} \in \meaningof{E_{1}}, P_{2} \in \meaningof{E_2}\} }
\end{mathpar}

\begin{mathpar}
 \inferrule* [lab=behavior] {} {\meaningof{\langle a?b \rangle E} = \{ P \in \pi | P \equiv Q | u?(y)P', \\ \and \\\\ \and \\ \;\;\; u \in \meaningof{a}, \forall z.P'\{z/y\} \in \meaningof{E\{z/b\}}\}, \and \\ \meaningof{a!E} = \{ P \in \pi | P \equiv Q | x!\langle P' \rangle, x \in \meaningof{a} P' \in \meaningof{E}\} }
\end{mathpar}

\begin{mathpar}
 \inferrule* [lab=nominal] {} {\meaningof{\quotep{E}} = \{ \quotep{P} \in \quotep{\pi} | P \in \meaningof{E} \}, \and \meaningof{\quotep{P}} = \{ \quotep{Q} \in \quotep{\pi} | P \equiv Q \} \and \\ \meaningof{@\quotep{E}} = \{ P \in \pi | P \equiv @x, x \in \meaningof{E} \}}
\end{mathpar}

\begin{eqnarray*}
  \\
  \meaningof{-} : TS \to ST
\end{eqnarray*}

\begin{eqnarray*}
  \\
  L : TS \to ST
\end{eqnarray*}

\begin{eqnarray*}
  \\
  P \models E \iff P \in \meaningof{E}
\end{eqnarray*}

\begin{eqnarray*}
  P \approx_{L} Q \iff \forall E \in L. P \models E \iff Q \models E
\end{eqnarray*}

\begin{eqnarray*}
  P \approx_{K} Q
\end{eqnarray*}

\begin{eqnarray*}
  P \approx Q
\end{eqnarray*}

$\approx_{K} = \approx = \approx_{L}$

\subsubsection{Contextual duality}

Note that contexts extend the quotation operation to a family of
operations from processes to names. Given a context, $M$, we can
define a \emph{nominal context}, $\quotep{M}$ by $\quotep{M}[P] :=
\quotep{M[P]}$. To foreshadow what is to come we observe that these
operations enjoy a duality with processes very much like the duality
between vectors and maps from vectors to scalars.

Further, because the calculus is essentially higher-order, we have a
correspondence between contexts and processes. More specifically,
given a name $x$ and a context $M$ we can construct $M^{*}_{x}$ such
that 

\begin{mathpar}
  M^{*}_{x} | \lift{x}{P} \red M[P]
\end{mathpar}

namely,

\begin{mathpar}
  M^{*}_{x} := x?(u).M[\dropn{u}]
\end{mathpar}

The dependence of $M^{*}_{x}$ on a name makes it an abstraction, 

\begin{mathpar}
  M^{*} := (x)x?(u).M[\dropn{u}]
\end{mathpar}

\subsection{Additional notation}

It will sometimes be convenient to denote the process a name
quotes. We already have the notation $x = \quotep{P}$, but it will be
convenient to introduce an alternate notation, $\procn{x}$, when we
want to emphasize the connection to the use of the name. Note that, by
virtue of name equivalence, $\quotep{\procn{x}} \nameeq x$; so, the
notation is consistent with previous definitions.

Further, because names have structure it is possible to effect
substitutions on the basis of that structure. This means we need to
upgrade our notation for substitutions, which we accomplish by
adapting comprehension notation. Thus,

\begin{mathpar}
  P\{ y / x : x \in S \}
\end{mathpar}

is interpreted to mean the process derived from P by replacing (in a
capture-avoiding manner) each occurrence of $x$ in $S$ by $y$. For example,

\begin{mathpar}
  P\{ \quotep{\procn{x}|\procn{x}} / x : x \in \freenames{P} \}
\end{mathpar}

will replace each (occurrence) of a free name $x$ in $P$ by
$\quotep{\procn{x}|\procn{x}}$.

Also, we will avail ourselves of the notation $x^{L}$ and $x^{R}$ to
denote injections of a name into disjoint copies of the name
space. There are numerous ways to accomplish this. One example can be
found in \cite{MeredithR05}. This notation overloads to vectors of
names: $\vec{x}^{\pi} := (x_{i}^{\pi} \; : \; 0 \leq i < |\vec{x}| )$ where $\pi \in \{L,R\}$.

We also use $P^{\Box} := P|\Box$.

In \cite{MeredithR05} an interpretation of the new operator is
given. It turns out that there are several possible interpretations
all enjoying the requisite algebraic properties of the operator (see
\cite{milner91polyadicpi}). We will therefore make liberal use of
$(\nu\; \vec{x})P$.

% subsection the_syntax_and_semantics_of_the_notation_system (end)   

\input{qm2pi.qmops} 

\input{qm2pi.sterngerlach} 

\input{qm2pi.metric} 

% section concurrent_process_calculi (end)

%\input{qm2pi.proofsketch}

% section proof sketch (end)

%\input{qm2pi.slviaknots} 

% section spatial logic via knots (end)

\input{qm2pi.conclusion}

% section conclusion (end)

%\input{qm2pi.dtcodes} 

% section wiring algorithm (end)

\input{qm2pi.ack} 

% section acknowledgments (end)

\newpage


\bibliographystyle{plain}   
\bibliography{../../biblios/main.bib}

\input{qm2pi.rhodetails}

\end{document}



% section front matter (end)

\section{Introduction}\label{sec:introduction} % (fold)
In this draft of the material i am going to have to dispense with the
usual writing conventions adopted in papers on these topics. i'm going
to have adopt whatever tone i need at the time i'm writing up the
calculations. Sometimes this may be very conversational; others it may
be the barest mathematical grunts; others still it may be that i have
lifted text from one of my other papers because the exposition of some
point was better said there. i hope that my readers are not unduly put
out by this decision. i'm not doing this to flout convention or be
rebellious. i find these calculations very technically challenging. To
keep everything going technically, something has to give; i have to
let go of some cognitive burden. So, the academic writing style --
with all of its trade-offs in terms of facilitating technical
communication -- is what i'm letting go of. Perhaps subsequent drafts
can be tightened and polished, but for now, i'm going to speak as if
we were sitting together in a coffee shop with a laptop, wifi and a
pad of paper and a pencil.

So, here's what i have to say. We -- you and i, comfortably ensconced
in our coffee shop and well-equipped with our tools -- can realize and
carry out the calculations of quantum mechanics over a very different
formal theory of dynamics, a formal theory of dynamics that
corresponds to a theory of concurrent computation with
\emph{reflection}. It has the advantage that the underlying theory is
already `quantized', but supports analogues all of the continuuous
operations. Strikingly, this underlying theory has recently been
connected with a notion of metric that we can show, by calculating
together, coincides with the metric induced by the inner product.

There are a lot of reasons why you might be interested in seeing
calculations of this form. Here's why i'm interested. For the past
several centuries there has been no competitor to the ``Newtonian''
account of dynamics. As a result the predominant share of accounts of
dynamical systems and situations have had to be formulated in terms of
the Newtonian machinery. i view this as an intellectually dangerous
position to occupy. Everything, despite it's intrinsic shape, turns
into a nail to be hit with this hammer. Recently, however, the theory
of computation has matured to the point where we have candidates for
theories of dynamics that offer very different perspective on
reasoning about dynamical systems and situations. Testing these
candidates against very successful accounts of dynamical situations,
like quantum mechanics, is going to give us some sense of how mature
they are and some measure of the quality of these accounts of
dynamics.

\subsection{Summary of contributions and outline of paper}

So, we're going to develop an interpretation of the operations of
quantum mechanics normally interpreted by Hilbert spaces and
operators. We're going to do this over a theory of computation. Note
that this is very different than the usual quantum computation program
which develops notions of computation over quantum mechanics. Rather,
we are developing a story that aligns with Wheeler's slogan: It from
Bit. To do this we will first provide an account of the theory of
computation at play here. Then we will dive into a calculation-driven
interpretation of the operations of quantum mechanics.

The reason we take this approach is that -- until very recently --
there hasn't been an axiomatic account of quantum mechanics. As a
result there has been no sharp delineation of the mathematical theory
supporting interpretation of the physical theory and the physical
theory, itself. So, ambient features of the maths are free to be
exploited (or supressed) without a real accounting of their physical
relevance. There is no sharp statement ``here's the physical theory''
qua \emph{theory} and ``here's the mathematical interpretation''
enabling a judgment of how faithful the interpretation is -- apart
from experimental observation. When there is an axiomatic account we
can judge how well a given mathematical formalism supports an
interpretation of the axioms, independent of
experimentation. Likewise, we can judge how well we have captured our
physical evidence and experience with our axiomatics, independent of
any specific mathematical implementation, with accidental detail that
may or may not have physical significance. 

In lieu of a fully fleshed out and vetted axiomatic account of quantum
mechanics, interpreting the operational notions in service of modeling
physical systems will have to suffice. In other words, we are not in
the business of providing a model of Hilbert spaces and operators. We
are in the business of providing a model of quantum mechanics because
we are motivated by testing our notions of dynamics against physical
theory; and, the predictive calculations of the physical theory must
serve as the best formulation -- shy of a fully fleshed out axiomatic
account -- of the physical theory itself (as they have for scientific
theories since time immemorial). Put another way, despite a
whole-hearted commitment to an It-from-Bit ontology, we are firmly
aligned with the shut-up-and-calculate camp as the best way to obtain
results either from the physical perspective or as a quality assurance
measure of our fledgling theory of dynamics.

In detail, we present a reflective process calculus. Then we develop
intuitive correspondences between the notions available in this
calculus and the usual physical notions supporting quantum mechanical
calculations. Thus, 

\begin{table}[htp]
  \center{
    \fbox{
      \begin{tabular}{c|c}
        quantum mechanics & process calculus \\
        \hline
        scalar & name \\
        state vector & process \\
        dual & contextual duals \\
        matrix & formal sums of process-context-dual pairs \\
        orthogonality & process annihilation \\
        inner product & execution-formula + quoting
      \end{tabular}
    }
  }
  \caption{QM - process calculi correspondences}
\end{table}

Then we tighten up these intuitions to operational definitions. We
employ the Dirac notation as the best proxy we can find for an
abstract syntax of the quantum mechanical notions. The definitions we
develop put us in contact with equational constraints coming from the
theory that we demonstrate the definitions and calculations satisfy.

This puts us in a position to shut up and calculate for the
Stern-Gerlach experimental set up, showing how these predictive
calculations become calculations on processes in our theory of a
reflective process calculus.

Penultimately, we demonstrate that the notion of metric coming from
the inner product coincides with the notion of metric available from
the theory of bisimulation. This demonstration gives us the right to
think of space as arising from behavior. Finally, we consider where we
might go from the new vantage point we have obtained.

% section introduction (end) 
 
% section introduction (end)

% \documentclass[12pt]{llncs}
%\documentclass{jktr}

\usepackage[pdftex]{hyperref}                   
\usepackage {listings}
\usepackage {mathpartir}
\usepackage{bcprules}
%\usepackage{listings}
                       
\usepackage{graphicx} 
%\usepackage[margins=2.5cm,nohead,nofoot]{geometry}
%\usepackage{geometry}
\usepackage{amsfonts}
\usepackage{amstext}
\usepackage{latexsym}
\usepackage{amssymb}
\usepackage{color}


%\include{myPreamble}
\include{qm2pi.local} 

%\ifpdf
%\usepackage[pdftex]{graphicx}
%\else
%\usepackage{graphicx}
%\fi

 % \ifpdf
%  \usepackage{pdfsync}
%  \if


%\title{Brief Article}
%\author{David F. Snyder}
%\author{L.G. Meredith}

%\address{Dept. of Math., Texas State University--San Marcos, San Marcos, TX 78666}
       
\pagestyle{empty}


\begin{document}

\lstset{language=[Objective]Caml,frame=shadowbox}

\input{qm2pi.front}

% section front matter (end)

\input{qm2pi.intro} 
 
% section introduction (end)

% \input{qm2pi.knotations} 

% section notation (end)

\input{qm2pi.process.calculi} 

% section concurrent_process_calculi_and_spatial_logics_ (end)
    
%\input{qm2pi.knots2pi} 

%\input{qm2pi.trefoil} 

%\input{qm2pi.mainthm} 

% subsection basic_interpretation (end)

%\input{qm2pi.rho.presentation} 
\subsection{The syntax and semantics of the notation system}\label{sub:the_syntax_and_semantics_of_the_notation_system} % (fold)

We now summarize a technical presentation of the calculus that
embodies our theory of dynamics. The typical presentation of such a
calculus follows the style of giving generators and relations on
them. The grammar, below, describing term constructors, freely
generates the set of processes, $\Proc$. This set is then quotiented
by a relation known as structural congruence and it is over this set
that the notion of dynamics is expressed. This presentation is
essentially that of \cite{MeredithR05} with the addition of
polyadicity and summation. For readability we have relegated some of
the technical subtleties to an appendix.

\subsubsection{Process grammar}\label{subsub:process_grammar}

\begin{mathpar}
  \inferrule* [lab=synchronization] {} {{M} \bc \pzero \;|\; x?F \;|\; x!C }
  \and
  \inferrule* [lab=abstraction] {} {{F} \bc (x)P}
  \and
  \inferrule* [lab=concretion] {} {{C} \bc \langle Q \rangle}
  \and
  \inferrule* [lab=process] {} {{P,Q} \bc M \;| \;P|Q \;|\; @{x}}
  \and
  \inferrule* [lab=name] {} {{x} \bc \quotep{P}}
\end{mathpar} 

Note that $\vec{x}$ (resp. $\vec{P}$) denotes a vector of names
(resp. processes) of length $|\vec{x}|$ (resp. $|\vec{P}|$). We adopt
the following useful abbreviations.

\begin{mathpar}
   x?(\vec{y}).P := x.(\vec{y})P \and  x\clift{\vec{P}} := x.\clift{\vec{P}}
   \and x!(y) := \lift{x}{\dropn{y}}
   \and \Pi_{i=0}^{n-1}P_i := P_0 | \ldots | P_{n-1}
\end{mathpar}

\subsubsection{Structural congruence}

\paragraph{Free and bound names and alpha-equivalence.} At the
core of structural equivalence is alpha-equivalence which identifies
process that are the same up to a change of variable. Formally, we
recognize the distinction between free and bound names. The free names
of a process, $\freenames{P}$, may be calculated recursively as
follows:

\begin{mathpar}
\freenames{\pzero} := \emptyset
  \and \\
  \freenames{x?(y).P} := \{ x \} \cup (\freenames{P} \setminus \{ y \})
  \and 
  \freenames{x!\langle P \rangle} := \{ x \} \cup \{ P \} 
  \and \\
  \freenames{P|Q} := \freenames{P} \cup \freenames{Q}
  \and \\
  \freenames{@{x}} := \{ x \}
\end{mathpar}

$\pi$
$\quotep{\pi}$

$\freenames{-} : \pi \to \mathcal{P}(\quotep{\pi})$

\begin{eqnarray*}
  \freenames{\pzero} & := & \emptyset \\
  \freenames{x?(y).P} & := & \{ x \} \cup (\freenames{P} \setminus \{ y \}) \\
  \freenames{x!\langle P \rangle} & := & \{ x \} \cup \{ P \} \\
  \freenames{P|Q} & := & \freenames{P} \cup \freenames{Q} \\
  \freenames{\dropn{x}} & := & \{ x \}
\end{eqnarray*}

The bound names of a process, $\boundnames{P}$, are those names occurring in $P$
that are not free. For example, in $x?(y).0$, the name $x$ is free, while $y$ is bound.

\begin{mathpar}
  \inferrule* [lab=monoidal-laws] {} { P|Q \equiv Q|P \and P|0 \equiv P \and P|(Q|R) \equiv (P|Q)|R }
\end{mathpar}

\begin{mathpar}
  \inferrule* [lab=alpha-equivalence] {} { (x)P \equiv (y)P\{y/x\} \and y \not\in \freenames{P} }
\end{mathpar}

\begin{definition}
Then two processes, $P,Q$, are alpha-equivalent if $P = Q\{\vec{y}/\vec{x}\}$ for
some $\vec{x} \in \boundnames{Q},\vec{y} \in \boundnames{P}$, where $Q\{\vec{y}/\vec{x}\}$
denotes the capture-avoiding substitution of $\vec{y}$ for $\vec{x}$ in $Q$.
\end{definition}

\begin{definition}
  The {\em structural congruence} \cite{SangiorgiWalker} , $\equiv$,
  between processes is the least congruence containing
  alpha-equivalence, satisfying the abelian monoid laws
  (associativity, commutativity and $\pzero$ as identity) for parallel
  composition $|$ and for summation $+$.
\end{definition}

\subsection{Name equivalence}

We take name equivalence, written $\nameeq$, to be the smallest
equivalence relation generated by the following rules.

\begin{mathpar}
\inferrule*[lab=Quote-drop]
{ }
{ \quotep{@{x}} \nameeq x }

\inferrule*[lab=Struct-equiv]
{ P \scong Q }
{ \quotep{P} \nameeq \quotep{Q} }
\end{mathpar}

The astute reader will have noticed that the mutual recursion of names
and processes imposes a mutual recursion on alpha-equivalence and
structural equivalence via name-equivalence. Fortunately, all of this
works out pleasantly and we may calculate in the natural way, free of
concern. The reader interested in the details is referred to the
appendix \ref{appendix:rho_details}.

\subsection{Substitution}

We use $\Proc$ for the set of processes, $\QProc$ for the set of
names, and $\id{\{}\vec{y} / \vec{x} \id{\}}$ to denote partial maps,
$s : \QProc \rightarrow \QProc$. A map, $s$ lifts, uniquely, to a map
on process terms, $\widehat{s} : \Proc \rightarrow \Proc$ by the
following equations.

\begin{mathpar}
  (0) \psubstp{Q}{P} := 0 \\
  (R \juxtap S) \psubstp{Q}{P}
  :=    
  (R)\psubstp{Q}{P} \juxtap (S) \psubstp{Q}{P} \\
  (x?(y).R) \psubstp{Q}{P}    
  :=    
  (x)\substp{Q}{P} (z)\concat( (R \psubstn{z}{y}) \psubstp{Q}{P} ) \\
  (\lift{x}{R}) \psubstp{Q}{P}  
  :=
  \lift{(x)\substp{Q}{P}}{ R \psubstp{Q}{P} } \\
%   (\dropn{x})  \psubstp{Q}{P}       
%   := 
%   \left\{ 
%     \begin{array}{ccc} 
%       \dropn{\quotep{Q}} & & x \nameeq \quotep{P} \\
%       \dropn{x} & & otherwise \\
%     \end{array}
%   \right. 
  (\dropn{x})  \psubstp{Q}{P}       
  := 
  \left\{ 
    \begin{array}{ccc} 
      Q & & x \nameeq \quotep{P} \\
      \dropn{x} & & otherwise \\
    \end{array}
  \right.
\end{mathpar}
 

where

\begin{eqnarray}
  (x)\id{\{} \lpquote Q \rpquote / \lpquote P \rpquote \id{\}}            = 
  \left\{ 
    \begin{array}{ccc}
      \lpquote Q \rpquote & & x \nameeq \lpquote P \rpquote \\
      x & & otherwise \\
    \end{array}
  \right. \nonumber
\end{eqnarray}

and $z$ is chosen distinct from $\quotep{P}$, $\quotep{Q}$, the free
names in $Q$, and all the names in $R$. Our $\alpha$-equivalence will
be built in the standard way from this substitution.

\begin{remark}\label{rem:no_self_referential_names}
  One consequence of these definitions is that $\forall P. \quotep{P}
  \not\in \freenames{P}$.
\end{remark}

\subsection{ Dynamic quote: an example }

Anticipating something of what's to come, consider applying the
substitution, $\widehat{\id{\{}u / z \id{\}}}$, to the following pair
of processes, $\lift{w}{y!(z)}$ and $w[ \lpquote y!(z) \rpquote ]$.

\begin{eqnarray}
	\lift{w}{y!(z)}\widehat{\id{\{}u / z \id{\}}}
		& = &
		\lift{w}{y!(u)} \nonumber\\
	w[ \lpquote y!(z) \rpquote ] \widehat{ \id{\{}u / z \id{\}} }
		& = &
		w[ \lpquote y!(z) \rpquote ] \nonumber
\end{eqnarray}

Because the body of the process between quotes is impervious to
substitution, we get radically different answers. In fact, by
examining the first process in an input context,
e.g. $x?(z).\lift{w}{y!(z)}$, we see that the process under the lift
operator may be shaped by prefixed inputs binding a name inside it. In
this sense, the lift operator will be seen as a way to dynamically
construct processes before reifying them as names.

Finally equipped with these standard features we can present the
dynamics of the calculus.

\subsubsection{Operational semantics} 

Finally, we introduce the computational dynamics. What marks these
algebras as distinct from other more traditionally studied algebraic
structures, e.g. vector spaces or polynomial rings, is the manner in
which dynamics is captured. In traditional structures, dynamics is typically
expressed through morphisms between such structures, as in linear maps
between vector spaces or morphisms between rings. In algebras
associated with the semantics of computation, the dynamics is
expressed as part of the algebraic structure itself, through a
reduction reduction relation typically denoted by $\red$. Below, we
give a recursive presentation of this relation for the calculus used
in the encoding.

$\red \subseteq \pi \times \pi$
$\red : \pi \to \mathcal{P}(\pi)$

\begin{mathpar}
  \inferrule* [lab=Comm] { \textsf{match}( x_{src}, x_{trgt} ) } { x_{trgt}?(y)P \; | \; x_{src}!\langle {Q} \rangle \red P\{\quotep{Q}/y}\} }
  \and \\
  \inferrule* [lab=Par] {{P} \red {P}'} {{{P} | {Q}} \red {{P}' | {Q}}}
  \and
  \inferrule* [lab=Equiv]{{{P} \scong {P}'} \andalso {{P}' \red {Q}'} \andalso {{Q}' \scong {Q}}}{{P} \red {Q}}
\end{mathpar}

\begin{eqnarray*}
  match_{\equiv} (\quotep{P},\quotep{Q}) & := & P \equiv Q \\
  match_{\dagger}(\quotep{P},\quotep{Q}) & := & \forall R. P|Q \red^{*} R => R \red^{*} 0 \\
  match_{K}(\quotep{P},\quotep{Q}) & := & K \mbox{ for some context } K
\end{eqnarray*}

$u?(x)P | u!\langle Q \rangle \red P\{\quotep{Q}/x\}$

%We write $\wred$ for $\red^*$, and $P\red$ if $\exists Q $ such that $ P \red Q$.
We write $P\red$ if $\exists Q $ such that $ P \red Q$ and $P\not\red$, otherwise.

\section{Replication}

As mentioned before, it is known that replication (and hence
recursion) can be implemented in a higher-order process algebra
\cite{SangiorgiWalker}. As our first example of calculation with the
machinery thus far presented we give the construction explicitly in
the {\rhoc}.

\begin{eqnarray}
	D_{x} & := & \prefix{x}{y}{(\binpar{\outputp{x}{y}}{@{y}})} \nonumber\\
	\bangp_{x}{P} & := & \binpar{{x}!\langle{\binpar{D_{x}}{P}}\rangle}{D_{x}} \nonumber
\end{eqnarray}

\begin{eqnarray}
	\bangp_{x}{P} & & \nonumber\\
	=
	& {x}!\langle{(\prefix{x}{y}{(\outputp{x}{y} | @{y})) | P}}\rangle 
	      | \prefix{x}{y}{(\outputp{x}{y} | @{y})} & \nonumber\\
	\red
	& (\outputp{x}{y} | @{y})\substn{\quotep{(\prefix{x}{y}{(@{y} | \outputp{x}{y})) | P}}}{y} & \nonumber\\
	=
	& \outputp{x}{\quotep{(\prefix{x}{y}{(\outputp{x}{y} | @{y})) | P}}}
	  | {(\prefix{x}{y}{(\outputp{x}{y} | @{y})) | P}} & \nonumber\\
	\red
	& \ldots & \nonumber\\
	\red^*
	& P | P | \ldots & \nonumber
\end{eqnarray}

Of course, this encoding, as an implementation, runs away, unfolding
$\bangp{P}$ eagerly. A lazier and more implementable replication
operator, restricted to input-guarded processes, may be obtained as follows.

\begin{eqnarray}
\bangp{\prefix{u}{v}{P}} 
	:= 
	\binpar{\lift{x}{\prefix{u}{v}{(\binpar{D(x)}{P})}}}{D(x)} \nonumber
\end{eqnarray}

\begin{remark}
  Note that the lazier definition still does not deal with summation
  or mixed summation (i.e. sums over input and output). The reader is
  invited to construct definitions of replication that deal with these
  features. 

  Further, the definitions are parameterized in a name, $x$. Can you,
  gentle reader, make a definition that eliminates this parameter and
  guarantees no accidental interaction between the replication
  machinery and the process being replicated -- i.e. no accidental
  sharing of names used by the process to get its work done and the
  name(s) used by the replication to effect copying. This latter
  revision of the definition of replication is crucial to obtaining
  the expected identity $!!P \sim !P$.
\end{remark}

\begin{remark}\label{rem:paradoxical_combinator}
  The reader familiar with the lambda calculus will have noticed the
  similarity between $D$ and the paradoxical combinator.

  [Ed. note: the existence of this seems to suggest we have to be more
  restrictive on the set of processes and names we admit if we are to
  support no-cloning.]
\end{remark}

\subsubsection{Bisimulation}

The computational dynamics gives rise to another kind of equivalence,
the equivalence of computational behavior. As previously mentioned
this is typically captured \emph{via} some form of bisimulation.

% The notion we use in this paper is weak barbed bisimulation
% \cite{milner91polyadicpi}.

The notion we use in this paper is derived from weak barbed
bisimulation \cite{milner91polyadicpi}. 

\begin{definition}
An \emph{observation relation}, $\downarrow_{\mathcal N}$, over a set
of names, $\mathcal N$, is the smallest relation satisfying the rules
below.

\infrule[Out-barb]{y \in {\mathcal N}, \; x \nameeq y}
		  {\outputp{x}{v} \downarrow_{\mathcal N} x}
\infrule[Par-barb]{\mbox{$P\downarrow_{\mathcal N} x$ or $Q\downarrow_{\mathcal N} x$}}
		  {\binpar{P}{Q} \downarrow_{\mathcal N} x}

We write $P \Downarrow_{\mathcal N} x$ if there is $Q$ such that 
$P \wred Q$ and $Q \downarrow_{\mathcal N} x$.
\end{definition}

\begin{definition}
%\label{def.bbisim}
An  ${\mathcal N}$-\emph{barbed bisimulation} over a set of names, ${\mathcal N}$, is a symmetric binary relation 
${\mathcal S}_{\mathcal N}$ between agents such that $P\rel{S}_{\mathcal N}Q$ implies:
\begin{enumerate}
\item If $P \red P'$ then $Q \wred Q'$ and $P'\rel{S}_{\mathcal N} Q'$.
\item If $P\downarrow_{\mathcal N} x$, then $Q\Downarrow_{\mathcal N} x$.
\end{enumerate}
$P$ is ${\mathcal N}$-barbed bisimilar to $Q$, written
$P \wbbisim_{\mathcal N} Q$, if $P \rel{S}_{\mathcal N} Q$ for some ${\mathcal N}$-barbed bisimulation ${\mathcal S}_{\mathcal N}$.
\end{definition}

$\mathcal{R} \subseteq \pi \times \pi$

$P \mathcal{R} Q => \forall P'. P \red P' \Rightarrow \exists Q'. Q \red Q', P' \mathcal{R} Q'$

$P \vdash x \Rightarrow Q \vdash x$

\begin{mathpar}
  \inferrule*[lab=Out-barb]{x \nameeq y}{{y}!\langle{Q}\rangle \vdash x}
  \and
  \inferrule*[lab=Par-barb]{\mbox{$P\vdash x$ or $Q\vdash x$}}{\binpar{P}{Q} \vdash x}
\end{mathpar}

\subsubsection{Contexts}

One of the principle advantages of computational calculi like the
$\pi$-calculus is a well-defined notion of context,
contextual-equivalence and a correlation between
contextual-equivalence and notions of bisimulation. The notion of
context allows the decomposition of a process into (sub-)process and
its syntactic environment, its context. Thus, a context may be
thought of as a process with a ``hole'' (written $\Box$) in it. The
application of a context $M$ to a process $P$, written $M[P]$, is
tantamount to filling the hole in $M$ with $P$. In this paper we do
not need the full weight of this theory, but do make use of the notion
of context in the proof the main theorem. 

\begin{mathpar}
  \inferrule* [lab=summation] {} {{M_{M},M_{N}} \bc \Box \;|\; x.M_{A} \;|\; M_{M}+M_{N}}
  \and
  \inferrule* [lab=agent] {} {{M_{A}} \bc (\vec{x})M_{P} \;| \; \clift{P_0,\ldots,M_{P},\ldots,P_N}}
  \and \\
  \inferrule* [lab=process] {} {{M_{P}} \bc M_{N} \;| \;P|M_{P} }
\end{mathpar} 

\begin{mathpar}
  \inferrule* [lab=sychronization] {} {M_{N} \bc \Box \;|\; x?M_{F} \;|\; x!M_{C}}
  \and
  \inferrule* [lab=abstraction] {} {{M_{F}} \bc (x)M_{P} }
  \and
  \inferrule* [lab=concretion] {} {{M_{C}} \bc \langle M_{P} \rangle }
  \and \\
  \inferrule* [lab=process] {} {{M_{P}} \bc M_{N} \;| \;P|M_{P} }
\end{mathpar}

\begin{definition}[contextual application] Given a context $M$, and
  process $P$, we define the \emph{contextual application}, $M[P] :=
  M\{P/\Box\}$. That is, the contextual application of M to P is the
  substitution of $P$ for $\Box$ in $M$.
\end{definition}

$\meaningof{-} : L \to \mathcal{P}(\pi)$

\begin{mathpar}
  \inferrule* [lab=collection] {} {\meaningof{true} = \pi, \and \meaningof{~E} = \pi \setminus \meaningof{E}, \and \meaningof{E_{1} \& E_{2}} = \meaningof{E_{1}} \cap \meaningof{E_{2}}}
\end{mathpar}

\begin{mathpar}
  \inferrule* [lab=structure] {} {\meaningof{0} = \{ P \in \pi | P \equiv 0 \}, \and \\ \meaningof{E_1 | E_2} = \{ P \in \pi | P \equiv P_{1} | P_{2}, P_{1} \in \meaningof{E_{1}}, P_{2} \in \meaningof{E_2}\} }
\end{mathpar}

\begin{mathpar}
 \inferrule* [lab=behavior] {} {\meaningof{\langle a?b \rangle E} = \{ P \in \pi | P \equiv Q | u?(y)P', \\ \and \\\\ \and \\ \;\;\; u \in \meaningof{a}, \forall z.P'\{z/y\} \in \meaningof{E\{z/b\}}\}, \and \\ \meaningof{a!E} = \{ P \in \pi | P \equiv Q | x!\langle P' \rangle, x \in \meaningof{a} P' \in \meaningof{E}\} }
\end{mathpar}

\begin{mathpar}
 \inferrule* [lab=nominal] {} {\meaningof{\quotep{E}} = \{ \quotep{P} \in \quotep{\pi} | P \in \meaningof{E} \}, \and \meaningof{\quotep{P}} = \{ \quotep{Q} \in \quotep{\pi} | P \equiv Q \} \and \\ \meaningof{@\quotep{E}} = \{ P \in \pi | P \equiv @x, x \in \meaningof{E} \}}
\end{mathpar}

\begin{eqnarray*}
  \\
  \meaningof{-} : TS \to ST
\end{eqnarray*}

\begin{eqnarray*}
  \\
  L : TS \to ST
\end{eqnarray*}

\begin{eqnarray*}
  \\
  P \models E \iff P \in \meaningof{E}
\end{eqnarray*}

\begin{eqnarray*}
  P \approx_{L} Q \iff \forall E \in L. P \models E \iff Q \models E
\end{eqnarray*}

\begin{eqnarray*}
  P \approx_{K} Q
\end{eqnarray*}

\begin{eqnarray*}
  P \approx Q
\end{eqnarray*}

$\approx_{K} = \approx = \approx_{L}$

\subsubsection{Contextual duality}

Note that contexts extend the quotation operation to a family of
operations from processes to names. Given a context, $M$, we can
define a \emph{nominal context}, $\quotep{M}$ by $\quotep{M}[P] :=
\quotep{M[P]}$. To foreshadow what is to come we observe that these
operations enjoy a duality with processes very much like the duality
between vectors and maps from vectors to scalars.

Further, because the calculus is essentially higher-order, we have a
correspondence between contexts and processes. More specifically,
given a name $x$ and a context $M$ we can construct $M^{*}_{x}$ such
that 

\begin{mathpar}
  M^{*}_{x} | \lift{x}{P} \red M[P]
\end{mathpar}

namely,

\begin{mathpar}
  M^{*}_{x} := x?(u).M[\dropn{u}]
\end{mathpar}

The dependence of $M^{*}_{x}$ on a name makes it an abstraction, 

\begin{mathpar}
  M^{*} := (x)x?(u).M[\dropn{u}]
\end{mathpar}

\subsection{Additional notation}

It will sometimes be convenient to denote the process a name
quotes. We already have the notation $x = \quotep{P}$, but it will be
convenient to introduce an alternate notation, $\procn{x}$, when we
want to emphasize the connection to the use of the name. Note that, by
virtue of name equivalence, $\quotep{\procn{x}} \nameeq x$; so, the
notation is consistent with previous definitions.

Further, because names have structure it is possible to effect
substitutions on the basis of that structure. This means we need to
upgrade our notation for substitutions, which we accomplish by
adapting comprehension notation. Thus,

\begin{mathpar}
  P\{ y / x : x \in S \}
\end{mathpar}

is interpreted to mean the process derived from P by replacing (in a
capture-avoiding manner) each occurrence of $x$ in $S$ by $y$. For example,

\begin{mathpar}
  P\{ \quotep{\procn{x}|\procn{x}} / x : x \in \freenames{P} \}
\end{mathpar}

will replace each (occurrence) of a free name $x$ in $P$ by
$\quotep{\procn{x}|\procn{x}}$.

Also, we will avail ourselves of the notation $x^{L}$ and $x^{R}$ to
denote injections of a name into disjoint copies of the name
space. There are numerous ways to accomplish this. One example can be
found in \cite{MeredithR05}. This notation overloads to vectors of
names: $\vec{x}^{\pi} := (x_{i}^{\pi} \; : \; 0 \leq i < |\vec{x}| )$ where $\pi \in \{L,R\}$.

We also use $P^{\Box} := P|\Box$.

In \cite{MeredithR05} an interpretation of the new operator is
given. It turns out that there are several possible interpretations
all enjoying the requisite algebraic properties of the operator (see
\cite{milner91polyadicpi}). We will therefore make liberal use of
$(\nu\; \vec{x})P$.

% subsection the_syntax_and_semantics_of_the_notation_system (end)   

\input{qm2pi.qmops} 

\input{qm2pi.sterngerlach} 

\input{qm2pi.metric} 

% section concurrent_process_calculi (end)

%\input{qm2pi.proofsketch}

% section proof sketch (end)

%\input{qm2pi.slviaknots} 

% section spatial logic via knots (end)

\input{qm2pi.conclusion}

% section conclusion (end)

%\input{qm2pi.dtcodes} 

% section wiring algorithm (end)

\input{qm2pi.ack} 

% section acknowledgments (end)

\newpage


\bibliographystyle{plain}   
\bibliography{../../biblios/main.bib}

\input{qm2pi.rhodetails}

\end{document}

 

% section notation (end)

\input{qm2pi.process.calculi} 

% section concurrent_process_calculi_and_spatial_logics_ (end)
    
%\documentclass[12pt]{llncs}
%\documentclass{jktr}

\usepackage[pdftex]{hyperref}                   
\usepackage {listings}
\usepackage {mathpartir}
\usepackage{bcprules}
%\usepackage{listings}
                       
\usepackage{graphicx} 
%\usepackage[margins=2.5cm,nohead,nofoot]{geometry}
%\usepackage{geometry}
\usepackage{amsfonts}
\usepackage{amstext}
\usepackage{latexsym}
\usepackage{amssymb}
\usepackage{color}


%\include{myPreamble}
\include{qm2pi.local} 

%\ifpdf
%\usepackage[pdftex]{graphicx}
%\else
%\usepackage{graphicx}
%\fi

 % \ifpdf
%  \usepackage{pdfsync}
%  \if


%\title{Brief Article}
%\author{David F. Snyder}
%\author{L.G. Meredith}

%\address{Dept. of Math., Texas State University--San Marcos, San Marcos, TX 78666}
       
\pagestyle{empty}


\begin{document}

\lstset{language=[Objective]Caml,frame=shadowbox}

\input{qm2pi.front}

% section front matter (end)

\input{qm2pi.intro} 
 
% section introduction (end)

% \input{qm2pi.knotations} 

% section notation (end)

\input{qm2pi.process.calculi} 

% section concurrent_process_calculi_and_spatial_logics_ (end)
    
%\input{qm2pi.knots2pi} 

%\input{qm2pi.trefoil} 

%\input{qm2pi.mainthm} 

% subsection basic_interpretation (end)

%\input{qm2pi.rho.presentation} 
\subsection{The syntax and semantics of the notation system}\label{sub:the_syntax_and_semantics_of_the_notation_system} % (fold)

We now summarize a technical presentation of the calculus that
embodies our theory of dynamics. The typical presentation of such a
calculus follows the style of giving generators and relations on
them. The grammar, below, describing term constructors, freely
generates the set of processes, $\Proc$. This set is then quotiented
by a relation known as structural congruence and it is over this set
that the notion of dynamics is expressed. This presentation is
essentially that of \cite{MeredithR05} with the addition of
polyadicity and summation. For readability we have relegated some of
the technical subtleties to an appendix.

\subsubsection{Process grammar}\label{subsub:process_grammar}

\begin{mathpar}
  \inferrule* [lab=synchronization] {} {{M} \bc \pzero \;|\; x?F \;|\; x!C }
  \and
  \inferrule* [lab=abstraction] {} {{F} \bc (x)P}
  \and
  \inferrule* [lab=concretion] {} {{C} \bc \langle Q \rangle}
  \and
  \inferrule* [lab=process] {} {{P,Q} \bc M \;| \;P|Q \;|\; @{x}}
  \and
  \inferrule* [lab=name] {} {{x} \bc \quotep{P}}
\end{mathpar} 

Note that $\vec{x}$ (resp. $\vec{P}$) denotes a vector of names
(resp. processes) of length $|\vec{x}|$ (resp. $|\vec{P}|$). We adopt
the following useful abbreviations.

\begin{mathpar}
   x?(\vec{y}).P := x.(\vec{y})P \and  x\clift{\vec{P}} := x.\clift{\vec{P}}
   \and x!(y) := \lift{x}{\dropn{y}}
   \and \Pi_{i=0}^{n-1}P_i := P_0 | \ldots | P_{n-1}
\end{mathpar}

\subsubsection{Structural congruence}

\paragraph{Free and bound names and alpha-equivalence.} At the
core of structural equivalence is alpha-equivalence which identifies
process that are the same up to a change of variable. Formally, we
recognize the distinction between free and bound names. The free names
of a process, $\freenames{P}$, may be calculated recursively as
follows:

\begin{mathpar}
\freenames{\pzero} := \emptyset
  \and \\
  \freenames{x?(y).P} := \{ x \} \cup (\freenames{P} \setminus \{ y \})
  \and 
  \freenames{x!\langle P \rangle} := \{ x \} \cup \{ P \} 
  \and \\
  \freenames{P|Q} := \freenames{P} \cup \freenames{Q}
  \and \\
  \freenames{@{x}} := \{ x \}
\end{mathpar}

$\pi$
$\quotep{\pi}$

$\freenames{-} : \pi \to \mathcal{P}(\quotep{\pi})$

\begin{eqnarray*}
  \freenames{\pzero} & := & \emptyset \\
  \freenames{x?(y).P} & := & \{ x \} \cup (\freenames{P} \setminus \{ y \}) \\
  \freenames{x!\langle P \rangle} & := & \{ x \} \cup \{ P \} \\
  \freenames{P|Q} & := & \freenames{P} \cup \freenames{Q} \\
  \freenames{\dropn{x}} & := & \{ x \}
\end{eqnarray*}

The bound names of a process, $\boundnames{P}$, are those names occurring in $P$
that are not free. For example, in $x?(y).0$, the name $x$ is free, while $y$ is bound.

\begin{mathpar}
  \inferrule* [lab=monoidal-laws] {} { P|Q \equiv Q|P \and P|0 \equiv P \and P|(Q|R) \equiv (P|Q)|R }
\end{mathpar}

\begin{mathpar}
  \inferrule* [lab=alpha-equivalence] {} { (x)P \equiv (y)P\{y/x\} \and y \not\in \freenames{P} }
\end{mathpar}

\begin{definition}
Then two processes, $P,Q$, are alpha-equivalent if $P = Q\{\vec{y}/\vec{x}\}$ for
some $\vec{x} \in \boundnames{Q},\vec{y} \in \boundnames{P}$, where $Q\{\vec{y}/\vec{x}\}$
denotes the capture-avoiding substitution of $\vec{y}$ for $\vec{x}$ in $Q$.
\end{definition}

\begin{definition}
  The {\em structural congruence} \cite{SangiorgiWalker} , $\equiv$,
  between processes is the least congruence containing
  alpha-equivalence, satisfying the abelian monoid laws
  (associativity, commutativity and $\pzero$ as identity) for parallel
  composition $|$ and for summation $+$.
\end{definition}

\subsection{Name equivalence}

We take name equivalence, written $\nameeq$, to be the smallest
equivalence relation generated by the following rules.

\begin{mathpar}
\inferrule*[lab=Quote-drop]
{ }
{ \quotep{@{x}} \nameeq x }

\inferrule*[lab=Struct-equiv]
{ P \scong Q }
{ \quotep{P} \nameeq \quotep{Q} }
\end{mathpar}

The astute reader will have noticed that the mutual recursion of names
and processes imposes a mutual recursion on alpha-equivalence and
structural equivalence via name-equivalence. Fortunately, all of this
works out pleasantly and we may calculate in the natural way, free of
concern. The reader interested in the details is referred to the
appendix \ref{appendix:rho_details}.

\subsection{Substitution}

We use $\Proc$ for the set of processes, $\QProc$ for the set of
names, and $\id{\{}\vec{y} / \vec{x} \id{\}}$ to denote partial maps,
$s : \QProc \rightarrow \QProc$. A map, $s$ lifts, uniquely, to a map
on process terms, $\widehat{s} : \Proc \rightarrow \Proc$ by the
following equations.

\begin{mathpar}
  (0) \psubstp{Q}{P} := 0 \\
  (R \juxtap S) \psubstp{Q}{P}
  :=    
  (R)\psubstp{Q}{P} \juxtap (S) \psubstp{Q}{P} \\
  (x?(y).R) \psubstp{Q}{P}    
  :=    
  (x)\substp{Q}{P} (z)\concat( (R \psubstn{z}{y}) \psubstp{Q}{P} ) \\
  (\lift{x}{R}) \psubstp{Q}{P}  
  :=
  \lift{(x)\substp{Q}{P}}{ R \psubstp{Q}{P} } \\
%   (\dropn{x})  \psubstp{Q}{P}       
%   := 
%   \left\{ 
%     \begin{array}{ccc} 
%       \dropn{\quotep{Q}} & & x \nameeq \quotep{P} \\
%       \dropn{x} & & otherwise \\
%     \end{array}
%   \right. 
  (\dropn{x})  \psubstp{Q}{P}       
  := 
  \left\{ 
    \begin{array}{ccc} 
      Q & & x \nameeq \quotep{P} \\
      \dropn{x} & & otherwise \\
    \end{array}
  \right.
\end{mathpar}
 

where

\begin{eqnarray}
  (x)\id{\{} \lpquote Q \rpquote / \lpquote P \rpquote \id{\}}            = 
  \left\{ 
    \begin{array}{ccc}
      \lpquote Q \rpquote & & x \nameeq \lpquote P \rpquote \\
      x & & otherwise \\
    \end{array}
  \right. \nonumber
\end{eqnarray}

and $z$ is chosen distinct from $\quotep{P}$, $\quotep{Q}$, the free
names in $Q$, and all the names in $R$. Our $\alpha$-equivalence will
be built in the standard way from this substitution.

\begin{remark}\label{rem:no_self_referential_names}
  One consequence of these definitions is that $\forall P. \quotep{P}
  \not\in \freenames{P}$.
\end{remark}

\subsection{ Dynamic quote: an example }

Anticipating something of what's to come, consider applying the
substitution, $\widehat{\id{\{}u / z \id{\}}}$, to the following pair
of processes, $\lift{w}{y!(z)}$ and $w[ \lpquote y!(z) \rpquote ]$.

\begin{eqnarray}
	\lift{w}{y!(z)}\widehat{\id{\{}u / z \id{\}}}
		& = &
		\lift{w}{y!(u)} \nonumber\\
	w[ \lpquote y!(z) \rpquote ] \widehat{ \id{\{}u / z \id{\}} }
		& = &
		w[ \lpquote y!(z) \rpquote ] \nonumber
\end{eqnarray}

Because the body of the process between quotes is impervious to
substitution, we get radically different answers. In fact, by
examining the first process in an input context,
e.g. $x?(z).\lift{w}{y!(z)}$, we see that the process under the lift
operator may be shaped by prefixed inputs binding a name inside it. In
this sense, the lift operator will be seen as a way to dynamically
construct processes before reifying them as names.

Finally equipped with these standard features we can present the
dynamics of the calculus.

\subsubsection{Operational semantics} 

Finally, we introduce the computational dynamics. What marks these
algebras as distinct from other more traditionally studied algebraic
structures, e.g. vector spaces or polynomial rings, is the manner in
which dynamics is captured. In traditional structures, dynamics is typically
expressed through morphisms between such structures, as in linear maps
between vector spaces or morphisms between rings. In algebras
associated with the semantics of computation, the dynamics is
expressed as part of the algebraic structure itself, through a
reduction reduction relation typically denoted by $\red$. Below, we
give a recursive presentation of this relation for the calculus used
in the encoding.

$\red \subseteq \pi \times \pi$
$\red : \pi \to \mathcal{P}(\pi)$

\begin{mathpar}
  \inferrule* [lab=Comm] { \textsf{match}( x_{src}, x_{trgt} ) } { x_{trgt}?(y)P \; | \; x_{src}!\langle {Q} \rangle \red P\{\quotep{Q}/y}\} }
  \and \\
  \inferrule* [lab=Par] {{P} \red {P}'} {{{P} | {Q}} \red {{P}' | {Q}}}
  \and
  \inferrule* [lab=Equiv]{{{P} \scong {P}'} \andalso {{P}' \red {Q}'} \andalso {{Q}' \scong {Q}}}{{P} \red {Q}}
\end{mathpar}

\begin{eqnarray*}
  match_{\equiv} (\quotep{P},\quotep{Q}) & := & P \equiv Q \\
  match_{\dagger}(\quotep{P},\quotep{Q}) & := & \forall R. P|Q \red^{*} R => R \red^{*} 0 \\
  match_{K}(\quotep{P},\quotep{Q}) & := & K \mbox{ for some context } K
\end{eqnarray*}

$u?(x)P | u!\langle Q \rangle \red P\{\quotep{Q}/x\}$

%We write $\wred$ for $\red^*$, and $P\red$ if $\exists Q $ such that $ P \red Q$.
We write $P\red$ if $\exists Q $ such that $ P \red Q$ and $P\not\red$, otherwise.

\section{Replication}

As mentioned before, it is known that replication (and hence
recursion) can be implemented in a higher-order process algebra
\cite{SangiorgiWalker}. As our first example of calculation with the
machinery thus far presented we give the construction explicitly in
the {\rhoc}.

\begin{eqnarray}
	D_{x} & := & \prefix{x}{y}{(\binpar{\outputp{x}{y}}{@{y}})} \nonumber\\
	\bangp_{x}{P} & := & \binpar{{x}!\langle{\binpar{D_{x}}{P}}\rangle}{D_{x}} \nonumber
\end{eqnarray}

\begin{eqnarray}
	\bangp_{x}{P} & & \nonumber\\
	=
	& {x}!\langle{(\prefix{x}{y}{(\outputp{x}{y} | @{y})) | P}}\rangle 
	      | \prefix{x}{y}{(\outputp{x}{y} | @{y})} & \nonumber\\
	\red
	& (\outputp{x}{y} | @{y})\substn{\quotep{(\prefix{x}{y}{(@{y} | \outputp{x}{y})) | P}}}{y} & \nonumber\\
	=
	& \outputp{x}{\quotep{(\prefix{x}{y}{(\outputp{x}{y} | @{y})) | P}}}
	  | {(\prefix{x}{y}{(\outputp{x}{y} | @{y})) | P}} & \nonumber\\
	\red
	& \ldots & \nonumber\\
	\red^*
	& P | P | \ldots & \nonumber
\end{eqnarray}

Of course, this encoding, as an implementation, runs away, unfolding
$\bangp{P}$ eagerly. A lazier and more implementable replication
operator, restricted to input-guarded processes, may be obtained as follows.

\begin{eqnarray}
\bangp{\prefix{u}{v}{P}} 
	:= 
	\binpar{\lift{x}{\prefix{u}{v}{(\binpar{D(x)}{P})}}}{D(x)} \nonumber
\end{eqnarray}

\begin{remark}
  Note that the lazier definition still does not deal with summation
  or mixed summation (i.e. sums over input and output). The reader is
  invited to construct definitions of replication that deal with these
  features. 

  Further, the definitions are parameterized in a name, $x$. Can you,
  gentle reader, make a definition that eliminates this parameter and
  guarantees no accidental interaction between the replication
  machinery and the process being replicated -- i.e. no accidental
  sharing of names used by the process to get its work done and the
  name(s) used by the replication to effect copying. This latter
  revision of the definition of replication is crucial to obtaining
  the expected identity $!!P \sim !P$.
\end{remark}

\begin{remark}\label{rem:paradoxical_combinator}
  The reader familiar with the lambda calculus will have noticed the
  similarity between $D$ and the paradoxical combinator.

  [Ed. note: the existence of this seems to suggest we have to be more
  restrictive on the set of processes and names we admit if we are to
  support no-cloning.]
\end{remark}

\subsubsection{Bisimulation}

The computational dynamics gives rise to another kind of equivalence,
the equivalence of computational behavior. As previously mentioned
this is typically captured \emph{via} some form of bisimulation.

% The notion we use in this paper is weak barbed bisimulation
% \cite{milner91polyadicpi}.

The notion we use in this paper is derived from weak barbed
bisimulation \cite{milner91polyadicpi}. 

\begin{definition}
An \emph{observation relation}, $\downarrow_{\mathcal N}$, over a set
of names, $\mathcal N$, is the smallest relation satisfying the rules
below.

\infrule[Out-barb]{y \in {\mathcal N}, \; x \nameeq y}
		  {\outputp{x}{v} \downarrow_{\mathcal N} x}
\infrule[Par-barb]{\mbox{$P\downarrow_{\mathcal N} x$ or $Q\downarrow_{\mathcal N} x$}}
		  {\binpar{P}{Q} \downarrow_{\mathcal N} x}

We write $P \Downarrow_{\mathcal N} x$ if there is $Q$ such that 
$P \wred Q$ and $Q \downarrow_{\mathcal N} x$.
\end{definition}

\begin{definition}
%\label{def.bbisim}
An  ${\mathcal N}$-\emph{barbed bisimulation} over a set of names, ${\mathcal N}$, is a symmetric binary relation 
${\mathcal S}_{\mathcal N}$ between agents such that $P\rel{S}_{\mathcal N}Q$ implies:
\begin{enumerate}
\item If $P \red P'$ then $Q \wred Q'$ and $P'\rel{S}_{\mathcal N} Q'$.
\item If $P\downarrow_{\mathcal N} x$, then $Q\Downarrow_{\mathcal N} x$.
\end{enumerate}
$P$ is ${\mathcal N}$-barbed bisimilar to $Q$, written
$P \wbbisim_{\mathcal N} Q$, if $P \rel{S}_{\mathcal N} Q$ for some ${\mathcal N}$-barbed bisimulation ${\mathcal S}_{\mathcal N}$.
\end{definition}

$\mathcal{R} \subseteq \pi \times \pi$

$P \mathcal{R} Q => \forall P'. P \red P' \Rightarrow \exists Q'. Q \red Q', P' \mathcal{R} Q'$

$P \vdash x \Rightarrow Q \vdash x$

\begin{mathpar}
  \inferrule*[lab=Out-barb]{x \nameeq y}{{y}!\langle{Q}\rangle \vdash x}
  \and
  \inferrule*[lab=Par-barb]{\mbox{$P\vdash x$ or $Q\vdash x$}}{\binpar{P}{Q} \vdash x}
\end{mathpar}

\subsubsection{Contexts}

One of the principle advantages of computational calculi like the
$\pi$-calculus is a well-defined notion of context,
contextual-equivalence and a correlation between
contextual-equivalence and notions of bisimulation. The notion of
context allows the decomposition of a process into (sub-)process and
its syntactic environment, its context. Thus, a context may be
thought of as a process with a ``hole'' (written $\Box$) in it. The
application of a context $M$ to a process $P$, written $M[P]$, is
tantamount to filling the hole in $M$ with $P$. In this paper we do
not need the full weight of this theory, but do make use of the notion
of context in the proof the main theorem. 

\begin{mathpar}
  \inferrule* [lab=summation] {} {{M_{M},M_{N}} \bc \Box \;|\; x.M_{A} \;|\; M_{M}+M_{N}}
  \and
  \inferrule* [lab=agent] {} {{M_{A}} \bc (\vec{x})M_{P} \;| \; \clift{P_0,\ldots,M_{P},\ldots,P_N}}
  \and \\
  \inferrule* [lab=process] {} {{M_{P}} \bc M_{N} \;| \;P|M_{P} }
\end{mathpar} 

\begin{mathpar}
  \inferrule* [lab=sychronization] {} {M_{N} \bc \Box \;|\; x?M_{F} \;|\; x!M_{C}}
  \and
  \inferrule* [lab=abstraction] {} {{M_{F}} \bc (x)M_{P} }
  \and
  \inferrule* [lab=concretion] {} {{M_{C}} \bc \langle M_{P} \rangle }
  \and \\
  \inferrule* [lab=process] {} {{M_{P}} \bc M_{N} \;| \;P|M_{P} }
\end{mathpar}

\begin{definition}[contextual application] Given a context $M$, and
  process $P$, we define the \emph{contextual application}, $M[P] :=
  M\{P/\Box\}$. That is, the contextual application of M to P is the
  substitution of $P$ for $\Box$ in $M$.
\end{definition}

$\meaningof{-} : L \to \mathcal{P}(\pi)$

\begin{mathpar}
  \inferrule* [lab=collection] {} {\meaningof{true} = \pi, \and \meaningof{~E} = \pi \setminus \meaningof{E}, \and \meaningof{E_{1} \& E_{2}} = \meaningof{E_{1}} \cap \meaningof{E_{2}}}
\end{mathpar}

\begin{mathpar}
  \inferrule* [lab=structure] {} {\meaningof{0} = \{ P \in \pi | P \equiv 0 \}, \and \\ \meaningof{E_1 | E_2} = \{ P \in \pi | P \equiv P_{1} | P_{2}, P_{1} \in \meaningof{E_{1}}, P_{2} \in \meaningof{E_2}\} }
\end{mathpar}

\begin{mathpar}
 \inferrule* [lab=behavior] {} {\meaningof{\langle a?b \rangle E} = \{ P \in \pi | P \equiv Q | u?(y)P', \\ \and \\\\ \and \\ \;\;\; u \in \meaningof{a}, \forall z.P'\{z/y\} \in \meaningof{E\{z/b\}}\}, \and \\ \meaningof{a!E} = \{ P \in \pi | P \equiv Q | x!\langle P' \rangle, x \in \meaningof{a} P' \in \meaningof{E}\} }
\end{mathpar}

\begin{mathpar}
 \inferrule* [lab=nominal] {} {\meaningof{\quotep{E}} = \{ \quotep{P} \in \quotep{\pi} | P \in \meaningof{E} \}, \and \meaningof{\quotep{P}} = \{ \quotep{Q} \in \quotep{\pi} | P \equiv Q \} \and \\ \meaningof{@\quotep{E}} = \{ P \in \pi | P \equiv @x, x \in \meaningof{E} \}}
\end{mathpar}

\begin{eqnarray*}
  \\
  \meaningof{-} : TS \to ST
\end{eqnarray*}

\begin{eqnarray*}
  \\
  L : TS \to ST
\end{eqnarray*}

\begin{eqnarray*}
  \\
  P \models E \iff P \in \meaningof{E}
\end{eqnarray*}

\begin{eqnarray*}
  P \approx_{L} Q \iff \forall E \in L. P \models E \iff Q \models E
\end{eqnarray*}

\begin{eqnarray*}
  P \approx_{K} Q
\end{eqnarray*}

\begin{eqnarray*}
  P \approx Q
\end{eqnarray*}

$\approx_{K} = \approx = \approx_{L}$

\subsubsection{Contextual duality}

Note that contexts extend the quotation operation to a family of
operations from processes to names. Given a context, $M$, we can
define a \emph{nominal context}, $\quotep{M}$ by $\quotep{M}[P] :=
\quotep{M[P]}$. To foreshadow what is to come we observe that these
operations enjoy a duality with processes very much like the duality
between vectors and maps from vectors to scalars.

Further, because the calculus is essentially higher-order, we have a
correspondence between contexts and processes. More specifically,
given a name $x$ and a context $M$ we can construct $M^{*}_{x}$ such
that 

\begin{mathpar}
  M^{*}_{x} | \lift{x}{P} \red M[P]
\end{mathpar}

namely,

\begin{mathpar}
  M^{*}_{x} := x?(u).M[\dropn{u}]
\end{mathpar}

The dependence of $M^{*}_{x}$ on a name makes it an abstraction, 

\begin{mathpar}
  M^{*} := (x)x?(u).M[\dropn{u}]
\end{mathpar}

\subsection{Additional notation}

It will sometimes be convenient to denote the process a name
quotes. We already have the notation $x = \quotep{P}$, but it will be
convenient to introduce an alternate notation, $\procn{x}$, when we
want to emphasize the connection to the use of the name. Note that, by
virtue of name equivalence, $\quotep{\procn{x}} \nameeq x$; so, the
notation is consistent with previous definitions.

Further, because names have structure it is possible to effect
substitutions on the basis of that structure. This means we need to
upgrade our notation for substitutions, which we accomplish by
adapting comprehension notation. Thus,

\begin{mathpar}
  P\{ y / x : x \in S \}
\end{mathpar}

is interpreted to mean the process derived from P by replacing (in a
capture-avoiding manner) each occurrence of $x$ in $S$ by $y$. For example,

\begin{mathpar}
  P\{ \quotep{\procn{x}|\procn{x}} / x : x \in \freenames{P} \}
\end{mathpar}

will replace each (occurrence) of a free name $x$ in $P$ by
$\quotep{\procn{x}|\procn{x}}$.

Also, we will avail ourselves of the notation $x^{L}$ and $x^{R}$ to
denote injections of a name into disjoint copies of the name
space. There are numerous ways to accomplish this. One example can be
found in \cite{MeredithR05}. This notation overloads to vectors of
names: $\vec{x}^{\pi} := (x_{i}^{\pi} \; : \; 0 \leq i < |\vec{x}| )$ where $\pi \in \{L,R\}$.

We also use $P^{\Box} := P|\Box$.

In \cite{MeredithR05} an interpretation of the new operator is
given. It turns out that there are several possible interpretations
all enjoying the requisite algebraic properties of the operator (see
\cite{milner91polyadicpi}). We will therefore make liberal use of
$(\nu\; \vec{x})P$.

% subsection the_syntax_and_semantics_of_the_notation_system (end)   

\input{qm2pi.qmops} 

\input{qm2pi.sterngerlach} 

\input{qm2pi.metric} 

% section concurrent_process_calculi (end)

%\input{qm2pi.proofsketch}

% section proof sketch (end)

%\input{qm2pi.slviaknots} 

% section spatial logic via knots (end)

\input{qm2pi.conclusion}

% section conclusion (end)

%\input{qm2pi.dtcodes} 

% section wiring algorithm (end)

\input{qm2pi.ack} 

% section acknowledgments (end)

\newpage


\bibliographystyle{plain}   
\bibliography{../../biblios/main.bib}

\input{qm2pi.rhodetails}

\end{document}

 

%\documentclass[12pt]{llncs}
%\documentclass{jktr}

\usepackage[pdftex]{hyperref}                   
\usepackage {listings}
\usepackage {mathpartir}
\usepackage{bcprules}
%\usepackage{listings}
                       
\usepackage{graphicx} 
%\usepackage[margins=2.5cm,nohead,nofoot]{geometry}
%\usepackage{geometry}
\usepackage{amsfonts}
\usepackage{amstext}
\usepackage{latexsym}
\usepackage{amssymb}
\usepackage{color}


%\include{myPreamble}
\include{qm2pi.local} 

%\ifpdf
%\usepackage[pdftex]{graphicx}
%\else
%\usepackage{graphicx}
%\fi

 % \ifpdf
%  \usepackage{pdfsync}
%  \if


%\title{Brief Article}
%\author{David F. Snyder}
%\author{L.G. Meredith}

%\address{Dept. of Math., Texas State University--San Marcos, San Marcos, TX 78666}
       
\pagestyle{empty}


\begin{document}

\lstset{language=[Objective]Caml,frame=shadowbox}

\input{qm2pi.front}

% section front matter (end)

\input{qm2pi.intro} 
 
% section introduction (end)

% \input{qm2pi.knotations} 

% section notation (end)

\input{qm2pi.process.calculi} 

% section concurrent_process_calculi_and_spatial_logics_ (end)
    
%\input{qm2pi.knots2pi} 

%\input{qm2pi.trefoil} 

%\input{qm2pi.mainthm} 

% subsection basic_interpretation (end)

%\input{qm2pi.rho.presentation} 
\subsection{The syntax and semantics of the notation system}\label{sub:the_syntax_and_semantics_of_the_notation_system} % (fold)

We now summarize a technical presentation of the calculus that
embodies our theory of dynamics. The typical presentation of such a
calculus follows the style of giving generators and relations on
them. The grammar, below, describing term constructors, freely
generates the set of processes, $\Proc$. This set is then quotiented
by a relation known as structural congruence and it is over this set
that the notion of dynamics is expressed. This presentation is
essentially that of \cite{MeredithR05} with the addition of
polyadicity and summation. For readability we have relegated some of
the technical subtleties to an appendix.

\subsubsection{Process grammar}\label{subsub:process_grammar}

\begin{mathpar}
  \inferrule* [lab=synchronization] {} {{M} \bc \pzero \;|\; x?F \;|\; x!C }
  \and
  \inferrule* [lab=abstraction] {} {{F} \bc (x)P}
  \and
  \inferrule* [lab=concretion] {} {{C} \bc \langle Q \rangle}
  \and
  \inferrule* [lab=process] {} {{P,Q} \bc M \;| \;P|Q \;|\; @{x}}
  \and
  \inferrule* [lab=name] {} {{x} \bc \quotep{P}}
\end{mathpar} 

Note that $\vec{x}$ (resp. $\vec{P}$) denotes a vector of names
(resp. processes) of length $|\vec{x}|$ (resp. $|\vec{P}|$). We adopt
the following useful abbreviations.

\begin{mathpar}
   x?(\vec{y}).P := x.(\vec{y})P \and  x\clift{\vec{P}} := x.\clift{\vec{P}}
   \and x!(y) := \lift{x}{\dropn{y}}
   \and \Pi_{i=0}^{n-1}P_i := P_0 | \ldots | P_{n-1}
\end{mathpar}

\subsubsection{Structural congruence}

\paragraph{Free and bound names and alpha-equivalence.} At the
core of structural equivalence is alpha-equivalence which identifies
process that are the same up to a change of variable. Formally, we
recognize the distinction between free and bound names. The free names
of a process, $\freenames{P}$, may be calculated recursively as
follows:

\begin{mathpar}
\freenames{\pzero} := \emptyset
  \and \\
  \freenames{x?(y).P} := \{ x \} \cup (\freenames{P} \setminus \{ y \})
  \and 
  \freenames{x!\langle P \rangle} := \{ x \} \cup \{ P \} 
  \and \\
  \freenames{P|Q} := \freenames{P} \cup \freenames{Q}
  \and \\
  \freenames{@{x}} := \{ x \}
\end{mathpar}

$\pi$
$\quotep{\pi}$

$\freenames{-} : \pi \to \mathcal{P}(\quotep{\pi})$

\begin{eqnarray*}
  \freenames{\pzero} & := & \emptyset \\
  \freenames{x?(y).P} & := & \{ x \} \cup (\freenames{P} \setminus \{ y \}) \\
  \freenames{x!\langle P \rangle} & := & \{ x \} \cup \{ P \} \\
  \freenames{P|Q} & := & \freenames{P} \cup \freenames{Q} \\
  \freenames{\dropn{x}} & := & \{ x \}
\end{eqnarray*}

The bound names of a process, $\boundnames{P}$, are those names occurring in $P$
that are not free. For example, in $x?(y).0$, the name $x$ is free, while $y$ is bound.

\begin{mathpar}
  \inferrule* [lab=monoidal-laws] {} { P|Q \equiv Q|P \and P|0 \equiv P \and P|(Q|R) \equiv (P|Q)|R }
\end{mathpar}

\begin{mathpar}
  \inferrule* [lab=alpha-equivalence] {} { (x)P \equiv (y)P\{y/x\} \and y \not\in \freenames{P} }
\end{mathpar}

\begin{definition}
Then two processes, $P,Q$, are alpha-equivalent if $P = Q\{\vec{y}/\vec{x}\}$ for
some $\vec{x} \in \boundnames{Q},\vec{y} \in \boundnames{P}$, where $Q\{\vec{y}/\vec{x}\}$
denotes the capture-avoiding substitution of $\vec{y}$ for $\vec{x}$ in $Q$.
\end{definition}

\begin{definition}
  The {\em structural congruence} \cite{SangiorgiWalker} , $\equiv$,
  between processes is the least congruence containing
  alpha-equivalence, satisfying the abelian monoid laws
  (associativity, commutativity and $\pzero$ as identity) for parallel
  composition $|$ and for summation $+$.
\end{definition}

\subsection{Name equivalence}

We take name equivalence, written $\nameeq$, to be the smallest
equivalence relation generated by the following rules.

\begin{mathpar}
\inferrule*[lab=Quote-drop]
{ }
{ \quotep{@{x}} \nameeq x }

\inferrule*[lab=Struct-equiv]
{ P \scong Q }
{ \quotep{P} \nameeq \quotep{Q} }
\end{mathpar}

The astute reader will have noticed that the mutual recursion of names
and processes imposes a mutual recursion on alpha-equivalence and
structural equivalence via name-equivalence. Fortunately, all of this
works out pleasantly and we may calculate in the natural way, free of
concern. The reader interested in the details is referred to the
appendix \ref{appendix:rho_details}.

\subsection{Substitution}

We use $\Proc$ for the set of processes, $\QProc$ for the set of
names, and $\id{\{}\vec{y} / \vec{x} \id{\}}$ to denote partial maps,
$s : \QProc \rightarrow \QProc$. A map, $s$ lifts, uniquely, to a map
on process terms, $\widehat{s} : \Proc \rightarrow \Proc$ by the
following equations.

\begin{mathpar}
  (0) \psubstp{Q}{P} := 0 \\
  (R \juxtap S) \psubstp{Q}{P}
  :=    
  (R)\psubstp{Q}{P} \juxtap (S) \psubstp{Q}{P} \\
  (x?(y).R) \psubstp{Q}{P}    
  :=    
  (x)\substp{Q}{P} (z)\concat( (R \psubstn{z}{y}) \psubstp{Q}{P} ) \\
  (\lift{x}{R}) \psubstp{Q}{P}  
  :=
  \lift{(x)\substp{Q}{P}}{ R \psubstp{Q}{P} } \\
%   (\dropn{x})  \psubstp{Q}{P}       
%   := 
%   \left\{ 
%     \begin{array}{ccc} 
%       \dropn{\quotep{Q}} & & x \nameeq \quotep{P} \\
%       \dropn{x} & & otherwise \\
%     \end{array}
%   \right. 
  (\dropn{x})  \psubstp{Q}{P}       
  := 
  \left\{ 
    \begin{array}{ccc} 
      Q & & x \nameeq \quotep{P} \\
      \dropn{x} & & otherwise \\
    \end{array}
  \right.
\end{mathpar}
 

where

\begin{eqnarray}
  (x)\id{\{} \lpquote Q \rpquote / \lpquote P \rpquote \id{\}}            = 
  \left\{ 
    \begin{array}{ccc}
      \lpquote Q \rpquote & & x \nameeq \lpquote P \rpquote \\
      x & & otherwise \\
    \end{array}
  \right. \nonumber
\end{eqnarray}

and $z$ is chosen distinct from $\quotep{P}$, $\quotep{Q}$, the free
names in $Q$, and all the names in $R$. Our $\alpha$-equivalence will
be built in the standard way from this substitution.

\begin{remark}\label{rem:no_self_referential_names}
  One consequence of these definitions is that $\forall P. \quotep{P}
  \not\in \freenames{P}$.
\end{remark}

\subsection{ Dynamic quote: an example }

Anticipating something of what's to come, consider applying the
substitution, $\widehat{\id{\{}u / z \id{\}}}$, to the following pair
of processes, $\lift{w}{y!(z)}$ and $w[ \lpquote y!(z) \rpquote ]$.

\begin{eqnarray}
	\lift{w}{y!(z)}\widehat{\id{\{}u / z \id{\}}}
		& = &
		\lift{w}{y!(u)} \nonumber\\
	w[ \lpquote y!(z) \rpquote ] \widehat{ \id{\{}u / z \id{\}} }
		& = &
		w[ \lpquote y!(z) \rpquote ] \nonumber
\end{eqnarray}

Because the body of the process between quotes is impervious to
substitution, we get radically different answers. In fact, by
examining the first process in an input context,
e.g. $x?(z).\lift{w}{y!(z)}$, we see that the process under the lift
operator may be shaped by prefixed inputs binding a name inside it. In
this sense, the lift operator will be seen as a way to dynamically
construct processes before reifying them as names.

Finally equipped with these standard features we can present the
dynamics of the calculus.

\subsubsection{Operational semantics} 

Finally, we introduce the computational dynamics. What marks these
algebras as distinct from other more traditionally studied algebraic
structures, e.g. vector spaces or polynomial rings, is the manner in
which dynamics is captured. In traditional structures, dynamics is typically
expressed through morphisms between such structures, as in linear maps
between vector spaces or morphisms between rings. In algebras
associated with the semantics of computation, the dynamics is
expressed as part of the algebraic structure itself, through a
reduction reduction relation typically denoted by $\red$. Below, we
give a recursive presentation of this relation for the calculus used
in the encoding.

$\red \subseteq \pi \times \pi$
$\red : \pi \to \mathcal{P}(\pi)$

\begin{mathpar}
  \inferrule* [lab=Comm] { \textsf{match}( x_{src}, x_{trgt} ) } { x_{trgt}?(y)P \; | \; x_{src}!\langle {Q} \rangle \red P\{\quotep{Q}/y}\} }
  \and \\
  \inferrule* [lab=Par] {{P} \red {P}'} {{{P} | {Q}} \red {{P}' | {Q}}}
  \and
  \inferrule* [lab=Equiv]{{{P} \scong {P}'} \andalso {{P}' \red {Q}'} \andalso {{Q}' \scong {Q}}}{{P} \red {Q}}
\end{mathpar}

\begin{eqnarray*}
  match_{\equiv} (\quotep{P},\quotep{Q}) & := & P \equiv Q \\
  match_{\dagger}(\quotep{P},\quotep{Q}) & := & \forall R. P|Q \red^{*} R => R \red^{*} 0 \\
  match_{K}(\quotep{P},\quotep{Q}) & := & K \mbox{ for some context } K
\end{eqnarray*}

$u?(x)P | u!\langle Q \rangle \red P\{\quotep{Q}/x\}$

%We write $\wred$ for $\red^*$, and $P\red$ if $\exists Q $ such that $ P \red Q$.
We write $P\red$ if $\exists Q $ such that $ P \red Q$ and $P\not\red$, otherwise.

\section{Replication}

As mentioned before, it is known that replication (and hence
recursion) can be implemented in a higher-order process algebra
\cite{SangiorgiWalker}. As our first example of calculation with the
machinery thus far presented we give the construction explicitly in
the {\rhoc}.

\begin{eqnarray}
	D_{x} & := & \prefix{x}{y}{(\binpar{\outputp{x}{y}}{@{y}})} \nonumber\\
	\bangp_{x}{P} & := & \binpar{{x}!\langle{\binpar{D_{x}}{P}}\rangle}{D_{x}} \nonumber
\end{eqnarray}

\begin{eqnarray}
	\bangp_{x}{P} & & \nonumber\\
	=
	& {x}!\langle{(\prefix{x}{y}{(\outputp{x}{y} | @{y})) | P}}\rangle 
	      | \prefix{x}{y}{(\outputp{x}{y} | @{y})} & \nonumber\\
	\red
	& (\outputp{x}{y} | @{y})\substn{\quotep{(\prefix{x}{y}{(@{y} | \outputp{x}{y})) | P}}}{y} & \nonumber\\
	=
	& \outputp{x}{\quotep{(\prefix{x}{y}{(\outputp{x}{y} | @{y})) | P}}}
	  | {(\prefix{x}{y}{(\outputp{x}{y} | @{y})) | P}} & \nonumber\\
	\red
	& \ldots & \nonumber\\
	\red^*
	& P | P | \ldots & \nonumber
\end{eqnarray}

Of course, this encoding, as an implementation, runs away, unfolding
$\bangp{P}$ eagerly. A lazier and more implementable replication
operator, restricted to input-guarded processes, may be obtained as follows.

\begin{eqnarray}
\bangp{\prefix{u}{v}{P}} 
	:= 
	\binpar{\lift{x}{\prefix{u}{v}{(\binpar{D(x)}{P})}}}{D(x)} \nonumber
\end{eqnarray}

\begin{remark}
  Note that the lazier definition still does not deal with summation
  or mixed summation (i.e. sums over input and output). The reader is
  invited to construct definitions of replication that deal with these
  features. 

  Further, the definitions are parameterized in a name, $x$. Can you,
  gentle reader, make a definition that eliminates this parameter and
  guarantees no accidental interaction between the replication
  machinery and the process being replicated -- i.e. no accidental
  sharing of names used by the process to get its work done and the
  name(s) used by the replication to effect copying. This latter
  revision of the definition of replication is crucial to obtaining
  the expected identity $!!P \sim !P$.
\end{remark}

\begin{remark}\label{rem:paradoxical_combinator}
  The reader familiar with the lambda calculus will have noticed the
  similarity between $D$ and the paradoxical combinator.

  [Ed. note: the existence of this seems to suggest we have to be more
  restrictive on the set of processes and names we admit if we are to
  support no-cloning.]
\end{remark}

\subsubsection{Bisimulation}

The computational dynamics gives rise to another kind of equivalence,
the equivalence of computational behavior. As previously mentioned
this is typically captured \emph{via} some form of bisimulation.

% The notion we use in this paper is weak barbed bisimulation
% \cite{milner91polyadicpi}.

The notion we use in this paper is derived from weak barbed
bisimulation \cite{milner91polyadicpi}. 

\begin{definition}
An \emph{observation relation}, $\downarrow_{\mathcal N}$, over a set
of names, $\mathcal N$, is the smallest relation satisfying the rules
below.

\infrule[Out-barb]{y \in {\mathcal N}, \; x \nameeq y}
		  {\outputp{x}{v} \downarrow_{\mathcal N} x}
\infrule[Par-barb]{\mbox{$P\downarrow_{\mathcal N} x$ or $Q\downarrow_{\mathcal N} x$}}
		  {\binpar{P}{Q} \downarrow_{\mathcal N} x}

We write $P \Downarrow_{\mathcal N} x$ if there is $Q$ such that 
$P \wred Q$ and $Q \downarrow_{\mathcal N} x$.
\end{definition}

\begin{definition}
%\label{def.bbisim}
An  ${\mathcal N}$-\emph{barbed bisimulation} over a set of names, ${\mathcal N}$, is a symmetric binary relation 
${\mathcal S}_{\mathcal N}$ between agents such that $P\rel{S}_{\mathcal N}Q$ implies:
\begin{enumerate}
\item If $P \red P'$ then $Q \wred Q'$ and $P'\rel{S}_{\mathcal N} Q'$.
\item If $P\downarrow_{\mathcal N} x$, then $Q\Downarrow_{\mathcal N} x$.
\end{enumerate}
$P$ is ${\mathcal N}$-barbed bisimilar to $Q$, written
$P \wbbisim_{\mathcal N} Q$, if $P \rel{S}_{\mathcal N} Q$ for some ${\mathcal N}$-barbed bisimulation ${\mathcal S}_{\mathcal N}$.
\end{definition}

$\mathcal{R} \subseteq \pi \times \pi$

$P \mathcal{R} Q => \forall P'. P \red P' \Rightarrow \exists Q'. Q \red Q', P' \mathcal{R} Q'$

$P \vdash x \Rightarrow Q \vdash x$

\begin{mathpar}
  \inferrule*[lab=Out-barb]{x \nameeq y}{{y}!\langle{Q}\rangle \vdash x}
  \and
  \inferrule*[lab=Par-barb]{\mbox{$P\vdash x$ or $Q\vdash x$}}{\binpar{P}{Q} \vdash x}
\end{mathpar}

\subsubsection{Contexts}

One of the principle advantages of computational calculi like the
$\pi$-calculus is a well-defined notion of context,
contextual-equivalence and a correlation between
contextual-equivalence and notions of bisimulation. The notion of
context allows the decomposition of a process into (sub-)process and
its syntactic environment, its context. Thus, a context may be
thought of as a process with a ``hole'' (written $\Box$) in it. The
application of a context $M$ to a process $P$, written $M[P]$, is
tantamount to filling the hole in $M$ with $P$. In this paper we do
not need the full weight of this theory, but do make use of the notion
of context in the proof the main theorem. 

\begin{mathpar}
  \inferrule* [lab=summation] {} {{M_{M},M_{N}} \bc \Box \;|\; x.M_{A} \;|\; M_{M}+M_{N}}
  \and
  \inferrule* [lab=agent] {} {{M_{A}} \bc (\vec{x})M_{P} \;| \; \clift{P_0,\ldots,M_{P},\ldots,P_N}}
  \and \\
  \inferrule* [lab=process] {} {{M_{P}} \bc M_{N} \;| \;P|M_{P} }
\end{mathpar} 

\begin{mathpar}
  \inferrule* [lab=sychronization] {} {M_{N} \bc \Box \;|\; x?M_{F} \;|\; x!M_{C}}
  \and
  \inferrule* [lab=abstraction] {} {{M_{F}} \bc (x)M_{P} }
  \and
  \inferrule* [lab=concretion] {} {{M_{C}} \bc \langle M_{P} \rangle }
  \and \\
  \inferrule* [lab=process] {} {{M_{P}} \bc M_{N} \;| \;P|M_{P} }
\end{mathpar}

\begin{definition}[contextual application] Given a context $M$, and
  process $P$, we define the \emph{contextual application}, $M[P] :=
  M\{P/\Box\}$. That is, the contextual application of M to P is the
  substitution of $P$ for $\Box$ in $M$.
\end{definition}

$\meaningof{-} : L \to \mathcal{P}(\pi)$

\begin{mathpar}
  \inferrule* [lab=collection] {} {\meaningof{true} = \pi, \and \meaningof{~E} = \pi \setminus \meaningof{E}, \and \meaningof{E_{1} \& E_{2}} = \meaningof{E_{1}} \cap \meaningof{E_{2}}}
\end{mathpar}

\begin{mathpar}
  \inferrule* [lab=structure] {} {\meaningof{0} = \{ P \in \pi | P \equiv 0 \}, \and \\ \meaningof{E_1 | E_2} = \{ P \in \pi | P \equiv P_{1} | P_{2}, P_{1} \in \meaningof{E_{1}}, P_{2} \in \meaningof{E_2}\} }
\end{mathpar}

\begin{mathpar}
 \inferrule* [lab=behavior] {} {\meaningof{\langle a?b \rangle E} = \{ P \in \pi | P \equiv Q | u?(y)P', \\ \and \\\\ \and \\ \;\;\; u \in \meaningof{a}, \forall z.P'\{z/y\} \in \meaningof{E\{z/b\}}\}, \and \\ \meaningof{a!E} = \{ P \in \pi | P \equiv Q | x!\langle P' \rangle, x \in \meaningof{a} P' \in \meaningof{E}\} }
\end{mathpar}

\begin{mathpar}
 \inferrule* [lab=nominal] {} {\meaningof{\quotep{E}} = \{ \quotep{P} \in \quotep{\pi} | P \in \meaningof{E} \}, \and \meaningof{\quotep{P}} = \{ \quotep{Q} \in \quotep{\pi} | P \equiv Q \} \and \\ \meaningof{@\quotep{E}} = \{ P \in \pi | P \equiv @x, x \in \meaningof{E} \}}
\end{mathpar}

\begin{eqnarray*}
  \\
  \meaningof{-} : TS \to ST
\end{eqnarray*}

\begin{eqnarray*}
  \\
  L : TS \to ST
\end{eqnarray*}

\begin{eqnarray*}
  \\
  P \models E \iff P \in \meaningof{E}
\end{eqnarray*}

\begin{eqnarray*}
  P \approx_{L} Q \iff \forall E \in L. P \models E \iff Q \models E
\end{eqnarray*}

\begin{eqnarray*}
  P \approx_{K} Q
\end{eqnarray*}

\begin{eqnarray*}
  P \approx Q
\end{eqnarray*}

$\approx_{K} = \approx = \approx_{L}$

\subsubsection{Contextual duality}

Note that contexts extend the quotation operation to a family of
operations from processes to names. Given a context, $M$, we can
define a \emph{nominal context}, $\quotep{M}$ by $\quotep{M}[P] :=
\quotep{M[P]}$. To foreshadow what is to come we observe that these
operations enjoy a duality with processes very much like the duality
between vectors and maps from vectors to scalars.

Further, because the calculus is essentially higher-order, we have a
correspondence between contexts and processes. More specifically,
given a name $x$ and a context $M$ we can construct $M^{*}_{x}$ such
that 

\begin{mathpar}
  M^{*}_{x} | \lift{x}{P} \red M[P]
\end{mathpar}

namely,

\begin{mathpar}
  M^{*}_{x} := x?(u).M[\dropn{u}]
\end{mathpar}

The dependence of $M^{*}_{x}$ on a name makes it an abstraction, 

\begin{mathpar}
  M^{*} := (x)x?(u).M[\dropn{u}]
\end{mathpar}

\subsection{Additional notation}

It will sometimes be convenient to denote the process a name
quotes. We already have the notation $x = \quotep{P}$, but it will be
convenient to introduce an alternate notation, $\procn{x}$, when we
want to emphasize the connection to the use of the name. Note that, by
virtue of name equivalence, $\quotep{\procn{x}} \nameeq x$; so, the
notation is consistent with previous definitions.

Further, because names have structure it is possible to effect
substitutions on the basis of that structure. This means we need to
upgrade our notation for substitutions, which we accomplish by
adapting comprehension notation. Thus,

\begin{mathpar}
  P\{ y / x : x \in S \}
\end{mathpar}

is interpreted to mean the process derived from P by replacing (in a
capture-avoiding manner) each occurrence of $x$ in $S$ by $y$. For example,

\begin{mathpar}
  P\{ \quotep{\procn{x}|\procn{x}} / x : x \in \freenames{P} \}
\end{mathpar}

will replace each (occurrence) of a free name $x$ in $P$ by
$\quotep{\procn{x}|\procn{x}}$.

Also, we will avail ourselves of the notation $x^{L}$ and $x^{R}$ to
denote injections of a name into disjoint copies of the name
space. There are numerous ways to accomplish this. One example can be
found in \cite{MeredithR05}. This notation overloads to vectors of
names: $\vec{x}^{\pi} := (x_{i}^{\pi} \; : \; 0 \leq i < |\vec{x}| )$ where $\pi \in \{L,R\}$.

We also use $P^{\Box} := P|\Box$.

In \cite{MeredithR05} an interpretation of the new operator is
given. It turns out that there are several possible interpretations
all enjoying the requisite algebraic properties of the operator (see
\cite{milner91polyadicpi}). We will therefore make liberal use of
$(\nu\; \vec{x})P$.

% subsection the_syntax_and_semantics_of_the_notation_system (end)   

\input{qm2pi.qmops} 

\input{qm2pi.sterngerlach} 

\input{qm2pi.metric} 

% section concurrent_process_calculi (end)

%\input{qm2pi.proofsketch}

% section proof sketch (end)

%\input{qm2pi.slviaknots} 

% section spatial logic via knots (end)

\input{qm2pi.conclusion}

% section conclusion (end)

%\input{qm2pi.dtcodes} 

% section wiring algorithm (end)

\input{qm2pi.ack} 

% section acknowledgments (end)

\newpage


\bibliographystyle{plain}   
\bibliography{../../biblios/main.bib}

\input{qm2pi.rhodetails}

\end{document}

 

%\documentclass[12pt]{llncs}
%\documentclass{jktr}

\usepackage[pdftex]{hyperref}                   
\usepackage {listings}
\usepackage {mathpartir}
\usepackage{bcprules}
%\usepackage{listings}
                       
\usepackage{graphicx} 
%\usepackage[margins=2.5cm,nohead,nofoot]{geometry}
%\usepackage{geometry}
\usepackage{amsfonts}
\usepackage{amstext}
\usepackage{latexsym}
\usepackage{amssymb}
\usepackage{color}


%\include{myPreamble}
\include{qm2pi.local} 

%\ifpdf
%\usepackage[pdftex]{graphicx}
%\else
%\usepackage{graphicx}
%\fi

 % \ifpdf
%  \usepackage{pdfsync}
%  \if


%\title{Brief Article}
%\author{David F. Snyder}
%\author{L.G. Meredith}

%\address{Dept. of Math., Texas State University--San Marcos, San Marcos, TX 78666}
       
\pagestyle{empty}


\begin{document}

\lstset{language=[Objective]Caml,frame=shadowbox}

\input{qm2pi.front}

% section front matter (end)

\input{qm2pi.intro} 
 
% section introduction (end)

% \input{qm2pi.knotations} 

% section notation (end)

\input{qm2pi.process.calculi} 

% section concurrent_process_calculi_and_spatial_logics_ (end)
    
%\input{qm2pi.knots2pi} 

%\input{qm2pi.trefoil} 

%\input{qm2pi.mainthm} 

% subsection basic_interpretation (end)

%\input{qm2pi.rho.presentation} 
\subsection{The syntax and semantics of the notation system}\label{sub:the_syntax_and_semantics_of_the_notation_system} % (fold)

We now summarize a technical presentation of the calculus that
embodies our theory of dynamics. The typical presentation of such a
calculus follows the style of giving generators and relations on
them. The grammar, below, describing term constructors, freely
generates the set of processes, $\Proc$. This set is then quotiented
by a relation known as structural congruence and it is over this set
that the notion of dynamics is expressed. This presentation is
essentially that of \cite{MeredithR05} with the addition of
polyadicity and summation. For readability we have relegated some of
the technical subtleties to an appendix.

\subsubsection{Process grammar}\label{subsub:process_grammar}

\begin{mathpar}
  \inferrule* [lab=synchronization] {} {{M} \bc \pzero \;|\; x?F \;|\; x!C }
  \and
  \inferrule* [lab=abstraction] {} {{F} \bc (x)P}
  \and
  \inferrule* [lab=concretion] {} {{C} \bc \langle Q \rangle}
  \and
  \inferrule* [lab=process] {} {{P,Q} \bc M \;| \;P|Q \;|\; @{x}}
  \and
  \inferrule* [lab=name] {} {{x} \bc \quotep{P}}
\end{mathpar} 

Note that $\vec{x}$ (resp. $\vec{P}$) denotes a vector of names
(resp. processes) of length $|\vec{x}|$ (resp. $|\vec{P}|$). We adopt
the following useful abbreviations.

\begin{mathpar}
   x?(\vec{y}).P := x.(\vec{y})P \and  x\clift{\vec{P}} := x.\clift{\vec{P}}
   \and x!(y) := \lift{x}{\dropn{y}}
   \and \Pi_{i=0}^{n-1}P_i := P_0 | \ldots | P_{n-1}
\end{mathpar}

\subsubsection{Structural congruence}

\paragraph{Free and bound names and alpha-equivalence.} At the
core of structural equivalence is alpha-equivalence which identifies
process that are the same up to a change of variable. Formally, we
recognize the distinction between free and bound names. The free names
of a process, $\freenames{P}$, may be calculated recursively as
follows:

\begin{mathpar}
\freenames{\pzero} := \emptyset
  \and \\
  \freenames{x?(y).P} := \{ x \} \cup (\freenames{P} \setminus \{ y \})
  \and 
  \freenames{x!\langle P \rangle} := \{ x \} \cup \{ P \} 
  \and \\
  \freenames{P|Q} := \freenames{P} \cup \freenames{Q}
  \and \\
  \freenames{@{x}} := \{ x \}
\end{mathpar}

$\pi$
$\quotep{\pi}$

$\freenames{-} : \pi \to \mathcal{P}(\quotep{\pi})$

\begin{eqnarray*}
  \freenames{\pzero} & := & \emptyset \\
  \freenames{x?(y).P} & := & \{ x \} \cup (\freenames{P} \setminus \{ y \}) \\
  \freenames{x!\langle P \rangle} & := & \{ x \} \cup \{ P \} \\
  \freenames{P|Q} & := & \freenames{P} \cup \freenames{Q} \\
  \freenames{\dropn{x}} & := & \{ x \}
\end{eqnarray*}

The bound names of a process, $\boundnames{P}$, are those names occurring in $P$
that are not free. For example, in $x?(y).0$, the name $x$ is free, while $y$ is bound.

\begin{mathpar}
  \inferrule* [lab=monoidal-laws] {} { P|Q \equiv Q|P \and P|0 \equiv P \and P|(Q|R) \equiv (P|Q)|R }
\end{mathpar}

\begin{mathpar}
  \inferrule* [lab=alpha-equivalence] {} { (x)P \equiv (y)P\{y/x\} \and y \not\in \freenames{P} }
\end{mathpar}

\begin{definition}
Then two processes, $P,Q$, are alpha-equivalent if $P = Q\{\vec{y}/\vec{x}\}$ for
some $\vec{x} \in \boundnames{Q},\vec{y} \in \boundnames{P}$, where $Q\{\vec{y}/\vec{x}\}$
denotes the capture-avoiding substitution of $\vec{y}$ for $\vec{x}$ in $Q$.
\end{definition}

\begin{definition}
  The {\em structural congruence} \cite{SangiorgiWalker} , $\equiv$,
  between processes is the least congruence containing
  alpha-equivalence, satisfying the abelian monoid laws
  (associativity, commutativity and $\pzero$ as identity) for parallel
  composition $|$ and for summation $+$.
\end{definition}

\subsection{Name equivalence}

We take name equivalence, written $\nameeq$, to be the smallest
equivalence relation generated by the following rules.

\begin{mathpar}
\inferrule*[lab=Quote-drop]
{ }
{ \quotep{@{x}} \nameeq x }

\inferrule*[lab=Struct-equiv]
{ P \scong Q }
{ \quotep{P} \nameeq \quotep{Q} }
\end{mathpar}

The astute reader will have noticed that the mutual recursion of names
and processes imposes a mutual recursion on alpha-equivalence and
structural equivalence via name-equivalence. Fortunately, all of this
works out pleasantly and we may calculate in the natural way, free of
concern. The reader interested in the details is referred to the
appendix \ref{appendix:rho_details}.

\subsection{Substitution}

We use $\Proc$ for the set of processes, $\QProc$ for the set of
names, and $\id{\{}\vec{y} / \vec{x} \id{\}}$ to denote partial maps,
$s : \QProc \rightarrow \QProc$. A map, $s$ lifts, uniquely, to a map
on process terms, $\widehat{s} : \Proc \rightarrow \Proc$ by the
following equations.

\begin{mathpar}
  (0) \psubstp{Q}{P} := 0 \\
  (R \juxtap S) \psubstp{Q}{P}
  :=    
  (R)\psubstp{Q}{P} \juxtap (S) \psubstp{Q}{P} \\
  (x?(y).R) \psubstp{Q}{P}    
  :=    
  (x)\substp{Q}{P} (z)\concat( (R \psubstn{z}{y}) \psubstp{Q}{P} ) \\
  (\lift{x}{R}) \psubstp{Q}{P}  
  :=
  \lift{(x)\substp{Q}{P}}{ R \psubstp{Q}{P} } \\
%   (\dropn{x})  \psubstp{Q}{P}       
%   := 
%   \left\{ 
%     \begin{array}{ccc} 
%       \dropn{\quotep{Q}} & & x \nameeq \quotep{P} \\
%       \dropn{x} & & otherwise \\
%     \end{array}
%   \right. 
  (\dropn{x})  \psubstp{Q}{P}       
  := 
  \left\{ 
    \begin{array}{ccc} 
      Q & & x \nameeq \quotep{P} \\
      \dropn{x} & & otherwise \\
    \end{array}
  \right.
\end{mathpar}
 

where

\begin{eqnarray}
  (x)\id{\{} \lpquote Q \rpquote / \lpquote P \rpquote \id{\}}            = 
  \left\{ 
    \begin{array}{ccc}
      \lpquote Q \rpquote & & x \nameeq \lpquote P \rpquote \\
      x & & otherwise \\
    \end{array}
  \right. \nonumber
\end{eqnarray}

and $z$ is chosen distinct from $\quotep{P}$, $\quotep{Q}$, the free
names in $Q$, and all the names in $R$. Our $\alpha$-equivalence will
be built in the standard way from this substitution.

\begin{remark}\label{rem:no_self_referential_names}
  One consequence of these definitions is that $\forall P. \quotep{P}
  \not\in \freenames{P}$.
\end{remark}

\subsection{ Dynamic quote: an example }

Anticipating something of what's to come, consider applying the
substitution, $\widehat{\id{\{}u / z \id{\}}}$, to the following pair
of processes, $\lift{w}{y!(z)}$ and $w[ \lpquote y!(z) \rpquote ]$.

\begin{eqnarray}
	\lift{w}{y!(z)}\widehat{\id{\{}u / z \id{\}}}
		& = &
		\lift{w}{y!(u)} \nonumber\\
	w[ \lpquote y!(z) \rpquote ] \widehat{ \id{\{}u / z \id{\}} }
		& = &
		w[ \lpquote y!(z) \rpquote ] \nonumber
\end{eqnarray}

Because the body of the process between quotes is impervious to
substitution, we get radically different answers. In fact, by
examining the first process in an input context,
e.g. $x?(z).\lift{w}{y!(z)}$, we see that the process under the lift
operator may be shaped by prefixed inputs binding a name inside it. In
this sense, the lift operator will be seen as a way to dynamically
construct processes before reifying them as names.

Finally equipped with these standard features we can present the
dynamics of the calculus.

\subsubsection{Operational semantics} 

Finally, we introduce the computational dynamics. What marks these
algebras as distinct from other more traditionally studied algebraic
structures, e.g. vector spaces or polynomial rings, is the manner in
which dynamics is captured. In traditional structures, dynamics is typically
expressed through morphisms between such structures, as in linear maps
between vector spaces or morphisms between rings. In algebras
associated with the semantics of computation, the dynamics is
expressed as part of the algebraic structure itself, through a
reduction reduction relation typically denoted by $\red$. Below, we
give a recursive presentation of this relation for the calculus used
in the encoding.

$\red \subseteq \pi \times \pi$
$\red : \pi \to \mathcal{P}(\pi)$

\begin{mathpar}
  \inferrule* [lab=Comm] { \textsf{match}( x_{src}, x_{trgt} ) } { x_{trgt}?(y)P \; | \; x_{src}!\langle {Q} \rangle \red P\{\quotep{Q}/y}\} }
  \and \\
  \inferrule* [lab=Par] {{P} \red {P}'} {{{P} | {Q}} \red {{P}' | {Q}}}
  \and
  \inferrule* [lab=Equiv]{{{P} \scong {P}'} \andalso {{P}' \red {Q}'} \andalso {{Q}' \scong {Q}}}{{P} \red {Q}}
\end{mathpar}

\begin{eqnarray*}
  match_{\equiv} (\quotep{P},\quotep{Q}) & := & P \equiv Q \\
  match_{\dagger}(\quotep{P},\quotep{Q}) & := & \forall R. P|Q \red^{*} R => R \red^{*} 0 \\
  match_{K}(\quotep{P},\quotep{Q}) & := & K \mbox{ for some context } K
\end{eqnarray*}

$u?(x)P | u!\langle Q \rangle \red P\{\quotep{Q}/x\}$

%We write $\wred$ for $\red^*$, and $P\red$ if $\exists Q $ such that $ P \red Q$.
We write $P\red$ if $\exists Q $ such that $ P \red Q$ and $P\not\red$, otherwise.

\section{Replication}

As mentioned before, it is known that replication (and hence
recursion) can be implemented in a higher-order process algebra
\cite{SangiorgiWalker}. As our first example of calculation with the
machinery thus far presented we give the construction explicitly in
the {\rhoc}.

\begin{eqnarray}
	D_{x} & := & \prefix{x}{y}{(\binpar{\outputp{x}{y}}{@{y}})} \nonumber\\
	\bangp_{x}{P} & := & \binpar{{x}!\langle{\binpar{D_{x}}{P}}\rangle}{D_{x}} \nonumber
\end{eqnarray}

\begin{eqnarray}
	\bangp_{x}{P} & & \nonumber\\
	=
	& {x}!\langle{(\prefix{x}{y}{(\outputp{x}{y} | @{y})) | P}}\rangle 
	      | \prefix{x}{y}{(\outputp{x}{y} | @{y})} & \nonumber\\
	\red
	& (\outputp{x}{y} | @{y})\substn{\quotep{(\prefix{x}{y}{(@{y} | \outputp{x}{y})) | P}}}{y} & \nonumber\\
	=
	& \outputp{x}{\quotep{(\prefix{x}{y}{(\outputp{x}{y} | @{y})) | P}}}
	  | {(\prefix{x}{y}{(\outputp{x}{y} | @{y})) | P}} & \nonumber\\
	\red
	& \ldots & \nonumber\\
	\red^*
	& P | P | \ldots & \nonumber
\end{eqnarray}

Of course, this encoding, as an implementation, runs away, unfolding
$\bangp{P}$ eagerly. A lazier and more implementable replication
operator, restricted to input-guarded processes, may be obtained as follows.

\begin{eqnarray}
\bangp{\prefix{u}{v}{P}} 
	:= 
	\binpar{\lift{x}{\prefix{u}{v}{(\binpar{D(x)}{P})}}}{D(x)} \nonumber
\end{eqnarray}

\begin{remark}
  Note that the lazier definition still does not deal with summation
  or mixed summation (i.e. sums over input and output). The reader is
  invited to construct definitions of replication that deal with these
  features. 

  Further, the definitions are parameterized in a name, $x$. Can you,
  gentle reader, make a definition that eliminates this parameter and
  guarantees no accidental interaction between the replication
  machinery and the process being replicated -- i.e. no accidental
  sharing of names used by the process to get its work done and the
  name(s) used by the replication to effect copying. This latter
  revision of the definition of replication is crucial to obtaining
  the expected identity $!!P \sim !P$.
\end{remark}

\begin{remark}\label{rem:paradoxical_combinator}
  The reader familiar with the lambda calculus will have noticed the
  similarity between $D$ and the paradoxical combinator.

  [Ed. note: the existence of this seems to suggest we have to be more
  restrictive on the set of processes and names we admit if we are to
  support no-cloning.]
\end{remark}

\subsubsection{Bisimulation}

The computational dynamics gives rise to another kind of equivalence,
the equivalence of computational behavior. As previously mentioned
this is typically captured \emph{via} some form of bisimulation.

% The notion we use in this paper is weak barbed bisimulation
% \cite{milner91polyadicpi}.

The notion we use in this paper is derived from weak barbed
bisimulation \cite{milner91polyadicpi}. 

\begin{definition}
An \emph{observation relation}, $\downarrow_{\mathcal N}$, over a set
of names, $\mathcal N$, is the smallest relation satisfying the rules
below.

\infrule[Out-barb]{y \in {\mathcal N}, \; x \nameeq y}
		  {\outputp{x}{v} \downarrow_{\mathcal N} x}
\infrule[Par-barb]{\mbox{$P\downarrow_{\mathcal N} x$ or $Q\downarrow_{\mathcal N} x$}}
		  {\binpar{P}{Q} \downarrow_{\mathcal N} x}

We write $P \Downarrow_{\mathcal N} x$ if there is $Q$ such that 
$P \wred Q$ and $Q \downarrow_{\mathcal N} x$.
\end{definition}

\begin{definition}
%\label{def.bbisim}
An  ${\mathcal N}$-\emph{barbed bisimulation} over a set of names, ${\mathcal N}$, is a symmetric binary relation 
${\mathcal S}_{\mathcal N}$ between agents such that $P\rel{S}_{\mathcal N}Q$ implies:
\begin{enumerate}
\item If $P \red P'$ then $Q \wred Q'$ and $P'\rel{S}_{\mathcal N} Q'$.
\item If $P\downarrow_{\mathcal N} x$, then $Q\Downarrow_{\mathcal N} x$.
\end{enumerate}
$P$ is ${\mathcal N}$-barbed bisimilar to $Q$, written
$P \wbbisim_{\mathcal N} Q$, if $P \rel{S}_{\mathcal N} Q$ for some ${\mathcal N}$-barbed bisimulation ${\mathcal S}_{\mathcal N}$.
\end{definition}

$\mathcal{R} \subseteq \pi \times \pi$

$P \mathcal{R} Q => \forall P'. P \red P' \Rightarrow \exists Q'. Q \red Q', P' \mathcal{R} Q'$

$P \vdash x \Rightarrow Q \vdash x$

\begin{mathpar}
  \inferrule*[lab=Out-barb]{x \nameeq y}{{y}!\langle{Q}\rangle \vdash x}
  \and
  \inferrule*[lab=Par-barb]{\mbox{$P\vdash x$ or $Q\vdash x$}}{\binpar{P}{Q} \vdash x}
\end{mathpar}

\subsubsection{Contexts}

One of the principle advantages of computational calculi like the
$\pi$-calculus is a well-defined notion of context,
contextual-equivalence and a correlation between
contextual-equivalence and notions of bisimulation. The notion of
context allows the decomposition of a process into (sub-)process and
its syntactic environment, its context. Thus, a context may be
thought of as a process with a ``hole'' (written $\Box$) in it. The
application of a context $M$ to a process $P$, written $M[P]$, is
tantamount to filling the hole in $M$ with $P$. In this paper we do
not need the full weight of this theory, but do make use of the notion
of context in the proof the main theorem. 

\begin{mathpar}
  \inferrule* [lab=summation] {} {{M_{M},M_{N}} \bc \Box \;|\; x.M_{A} \;|\; M_{M}+M_{N}}
  \and
  \inferrule* [lab=agent] {} {{M_{A}} \bc (\vec{x})M_{P} \;| \; \clift{P_0,\ldots,M_{P},\ldots,P_N}}
  \and \\
  \inferrule* [lab=process] {} {{M_{P}} \bc M_{N} \;| \;P|M_{P} }
\end{mathpar} 

\begin{mathpar}
  \inferrule* [lab=sychronization] {} {M_{N} \bc \Box \;|\; x?M_{F} \;|\; x!M_{C}}
  \and
  \inferrule* [lab=abstraction] {} {{M_{F}} \bc (x)M_{P} }
  \and
  \inferrule* [lab=concretion] {} {{M_{C}} \bc \langle M_{P} \rangle }
  \and \\
  \inferrule* [lab=process] {} {{M_{P}} \bc M_{N} \;| \;P|M_{P} }
\end{mathpar}

\begin{definition}[contextual application] Given a context $M$, and
  process $P$, we define the \emph{contextual application}, $M[P] :=
  M\{P/\Box\}$. That is, the contextual application of M to P is the
  substitution of $P$ for $\Box$ in $M$.
\end{definition}

$\meaningof{-} : L \to \mathcal{P}(\pi)$

\begin{mathpar}
  \inferrule* [lab=collection] {} {\meaningof{true} = \pi, \and \meaningof{~E} = \pi \setminus \meaningof{E}, \and \meaningof{E_{1} \& E_{2}} = \meaningof{E_{1}} \cap \meaningof{E_{2}}}
\end{mathpar}

\begin{mathpar}
  \inferrule* [lab=structure] {} {\meaningof{0} = \{ P \in \pi | P \equiv 0 \}, \and \\ \meaningof{E_1 | E_2} = \{ P \in \pi | P \equiv P_{1} | P_{2}, P_{1} \in \meaningof{E_{1}}, P_{2} \in \meaningof{E_2}\} }
\end{mathpar}

\begin{mathpar}
 \inferrule* [lab=behavior] {} {\meaningof{\langle a?b \rangle E} = \{ P \in \pi | P \equiv Q | u?(y)P', \\ \and \\\\ \and \\ \;\;\; u \in \meaningof{a}, \forall z.P'\{z/y\} \in \meaningof{E\{z/b\}}\}, \and \\ \meaningof{a!E} = \{ P \in \pi | P \equiv Q | x!\langle P' \rangle, x \in \meaningof{a} P' \in \meaningof{E}\} }
\end{mathpar}

\begin{mathpar}
 \inferrule* [lab=nominal] {} {\meaningof{\quotep{E}} = \{ \quotep{P} \in \quotep{\pi} | P \in \meaningof{E} \}, \and \meaningof{\quotep{P}} = \{ \quotep{Q} \in \quotep{\pi} | P \equiv Q \} \and \\ \meaningof{@\quotep{E}} = \{ P \in \pi | P \equiv @x, x \in \meaningof{E} \}}
\end{mathpar}

\begin{eqnarray*}
  \\
  \meaningof{-} : TS \to ST
\end{eqnarray*}

\begin{eqnarray*}
  \\
  L : TS \to ST
\end{eqnarray*}

\begin{eqnarray*}
  \\
  P \models E \iff P \in \meaningof{E}
\end{eqnarray*}

\begin{eqnarray*}
  P \approx_{L} Q \iff \forall E \in L. P \models E \iff Q \models E
\end{eqnarray*}

\begin{eqnarray*}
  P \approx_{K} Q
\end{eqnarray*}

\begin{eqnarray*}
  P \approx Q
\end{eqnarray*}

$\approx_{K} = \approx = \approx_{L}$

\subsubsection{Contextual duality}

Note that contexts extend the quotation operation to a family of
operations from processes to names. Given a context, $M$, we can
define a \emph{nominal context}, $\quotep{M}$ by $\quotep{M}[P] :=
\quotep{M[P]}$. To foreshadow what is to come we observe that these
operations enjoy a duality with processes very much like the duality
between vectors and maps from vectors to scalars.

Further, because the calculus is essentially higher-order, we have a
correspondence between contexts and processes. More specifically,
given a name $x$ and a context $M$ we can construct $M^{*}_{x}$ such
that 

\begin{mathpar}
  M^{*}_{x} | \lift{x}{P} \red M[P]
\end{mathpar}

namely,

\begin{mathpar}
  M^{*}_{x} := x?(u).M[\dropn{u}]
\end{mathpar}

The dependence of $M^{*}_{x}$ on a name makes it an abstraction, 

\begin{mathpar}
  M^{*} := (x)x?(u).M[\dropn{u}]
\end{mathpar}

\subsection{Additional notation}

It will sometimes be convenient to denote the process a name
quotes. We already have the notation $x = \quotep{P}$, but it will be
convenient to introduce an alternate notation, $\procn{x}$, when we
want to emphasize the connection to the use of the name. Note that, by
virtue of name equivalence, $\quotep{\procn{x}} \nameeq x$; so, the
notation is consistent with previous definitions.

Further, because names have structure it is possible to effect
substitutions on the basis of that structure. This means we need to
upgrade our notation for substitutions, which we accomplish by
adapting comprehension notation. Thus,

\begin{mathpar}
  P\{ y / x : x \in S \}
\end{mathpar}

is interpreted to mean the process derived from P by replacing (in a
capture-avoiding manner) each occurrence of $x$ in $S$ by $y$. For example,

\begin{mathpar}
  P\{ \quotep{\procn{x}|\procn{x}} / x : x \in \freenames{P} \}
\end{mathpar}

will replace each (occurrence) of a free name $x$ in $P$ by
$\quotep{\procn{x}|\procn{x}}$.

Also, we will avail ourselves of the notation $x^{L}$ and $x^{R}$ to
denote injections of a name into disjoint copies of the name
space. There are numerous ways to accomplish this. One example can be
found in \cite{MeredithR05}. This notation overloads to vectors of
names: $\vec{x}^{\pi} := (x_{i}^{\pi} \; : \; 0 \leq i < |\vec{x}| )$ where $\pi \in \{L,R\}$.

We also use $P^{\Box} := P|\Box$.

In \cite{MeredithR05} an interpretation of the new operator is
given. It turns out that there are several possible interpretations
all enjoying the requisite algebraic properties of the operator (see
\cite{milner91polyadicpi}). We will therefore make liberal use of
$(\nu\; \vec{x})P$.

% subsection the_syntax_and_semantics_of_the_notation_system (end)   

\input{qm2pi.qmops} 

\input{qm2pi.sterngerlach} 

\input{qm2pi.metric} 

% section concurrent_process_calculi (end)

%\input{qm2pi.proofsketch}

% section proof sketch (end)

%\input{qm2pi.slviaknots} 

% section spatial logic via knots (end)

\input{qm2pi.conclusion}

% section conclusion (end)

%\input{qm2pi.dtcodes} 

% section wiring algorithm (end)

\input{qm2pi.ack} 

% section acknowledgments (end)

\newpage


\bibliographystyle{plain}   
\bibliography{../../biblios/main.bib}

\input{qm2pi.rhodetails}

\end{document}

 

% subsection basic_interpretation (end)

%\input{qm2pi.rho.presentation} 
\subsection{The syntax and semantics of the notation system}\label{sub:the_syntax_and_semantics_of_the_notation_system} % (fold)

We now summarize a technical presentation of the calculus that
embodies our theory of dynamics. The typical presentation of such a
calculus follows the style of giving generators and relations on
them. The grammar, below, describing term constructors, freely
generates the set of processes, $\Proc$. This set is then quotiented
by a relation known as structural congruence and it is over this set
that the notion of dynamics is expressed. This presentation is
essentially that of \cite{MeredithR05} with the addition of
polyadicity and summation. For readability we have relegated some of
the technical subtleties to an appendix.

\subsubsection{Process grammar}\label{subsub:process_grammar}

\begin{mathpar}
  \inferrule* [lab=synchronization] {} {{M} \bc \pzero \;|\; x?F \;|\; x!C }
  \and
  \inferrule* [lab=abstraction] {} {{F} \bc (x)P}
  \and
  \inferrule* [lab=concretion] {} {{C} \bc \langle Q \rangle}
  \and
  \inferrule* [lab=process] {} {{P,Q} \bc M \;| \;P|Q \;|\; @{x}}
  \and
  \inferrule* [lab=name] {} {{x} \bc \quotep{P}}
\end{mathpar} 

Note that $\vec{x}$ (resp. $\vec{P}$) denotes a vector of names
(resp. processes) of length $|\vec{x}|$ (resp. $|\vec{P}|$). We adopt
the following useful abbreviations.

\begin{mathpar}
   x?(\vec{y}).P := x.(\vec{y})P \and  x\clift{\vec{P}} := x.\clift{\vec{P}}
   \and x!(y) := \lift{x}{\dropn{y}}
   \and \Pi_{i=0}^{n-1}P_i := P_0 | \ldots | P_{n-1}
\end{mathpar}

\subsubsection{Structural congruence}

\paragraph{Free and bound names and alpha-equivalence.} At the
core of structural equivalence is alpha-equivalence which identifies
process that are the same up to a change of variable. Formally, we
recognize the distinction between free and bound names. The free names
of a process, $\freenames{P}$, may be calculated recursively as
follows:

\begin{mathpar}
\freenames{\pzero} := \emptyset
  \and \\
  \freenames{x?(y).P} := \{ x \} \cup (\freenames{P} \setminus \{ y \})
  \and 
  \freenames{x!\langle P \rangle} := \{ x \} \cup \{ P \} 
  \and \\
  \freenames{P|Q} := \freenames{P} \cup \freenames{Q}
  \and \\
  \freenames{@{x}} := \{ x \}
\end{mathpar}

$\pi$
$\quotep{\pi}$

$\freenames{-} : \pi \to \mathcal{P}(\quotep{\pi})$

\begin{eqnarray*}
  \freenames{\pzero} & := & \emptyset \\
  \freenames{x?(y).P} & := & \{ x \} \cup (\freenames{P} \setminus \{ y \}) \\
  \freenames{x!\langle P \rangle} & := & \{ x \} \cup \{ P \} \\
  \freenames{P|Q} & := & \freenames{P} \cup \freenames{Q} \\
  \freenames{\dropn{x}} & := & \{ x \}
\end{eqnarray*}

The bound names of a process, $\boundnames{P}$, are those names occurring in $P$
that are not free. For example, in $x?(y).0$, the name $x$ is free, while $y$ is bound.

\begin{mathpar}
  \inferrule* [lab=monoidal-laws] {} { P|Q \equiv Q|P \and P|0 \equiv P \and P|(Q|R) \equiv (P|Q)|R }
\end{mathpar}

\begin{mathpar}
  \inferrule* [lab=alpha-equivalence] {} { (x)P \equiv (y)P\{y/x\} \and y \not\in \freenames{P} }
\end{mathpar}

\begin{definition}
Then two processes, $P,Q$, are alpha-equivalent if $P = Q\{\vec{y}/\vec{x}\}$ for
some $\vec{x} \in \boundnames{Q},\vec{y} \in \boundnames{P}$, where $Q\{\vec{y}/\vec{x}\}$
denotes the capture-avoiding substitution of $\vec{y}$ for $\vec{x}$ in $Q$.
\end{definition}

\begin{definition}
  The {\em structural congruence} \cite{SangiorgiWalker} , $\equiv$,
  between processes is the least congruence containing
  alpha-equivalence, satisfying the abelian monoid laws
  (associativity, commutativity and $\pzero$ as identity) for parallel
  composition $|$ and for summation $+$.
\end{definition}

\subsection{Name equivalence}

We take name equivalence, written $\nameeq$, to be the smallest
equivalence relation generated by the following rules.

\begin{mathpar}
\inferrule*[lab=Quote-drop]
{ }
{ \quotep{@{x}} \nameeq x }

\inferrule*[lab=Struct-equiv]
{ P \scong Q }
{ \quotep{P} \nameeq \quotep{Q} }
\end{mathpar}

The astute reader will have noticed that the mutual recursion of names
and processes imposes a mutual recursion on alpha-equivalence and
structural equivalence via name-equivalence. Fortunately, all of this
works out pleasantly and we may calculate in the natural way, free of
concern. The reader interested in the details is referred to the
appendix \ref{appendix:rho_details}.

\subsection{Substitution}

We use $\Proc$ for the set of processes, $\QProc$ for the set of
names, and $\id{\{}\vec{y} / \vec{x} \id{\}}$ to denote partial maps,
$s : \QProc \rightarrow \QProc$. A map, $s$ lifts, uniquely, to a map
on process terms, $\widehat{s} : \Proc \rightarrow \Proc$ by the
following equations.

\begin{mathpar}
  (0) \psubstp{Q}{P} := 0 \\
  (R \juxtap S) \psubstp{Q}{P}
  :=    
  (R)\psubstp{Q}{P} \juxtap (S) \psubstp{Q}{P} \\
  (x?(y).R) \psubstp{Q}{P}    
  :=    
  (x)\substp{Q}{P} (z)\concat( (R \psubstn{z}{y}) \psubstp{Q}{P} ) \\
  (\lift{x}{R}) \psubstp{Q}{P}  
  :=
  \lift{(x)\substp{Q}{P}}{ R \psubstp{Q}{P} } \\
%   (\dropn{x})  \psubstp{Q}{P}       
%   := 
%   \left\{ 
%     \begin{array}{ccc} 
%       \dropn{\quotep{Q}} & & x \nameeq \quotep{P} \\
%       \dropn{x} & & otherwise \\
%     \end{array}
%   \right. 
  (\dropn{x})  \psubstp{Q}{P}       
  := 
  \left\{ 
    \begin{array}{ccc} 
      Q & & x \nameeq \quotep{P} \\
      \dropn{x} & & otherwise \\
    \end{array}
  \right.
\end{mathpar}
 

where

\begin{eqnarray}
  (x)\id{\{} \lpquote Q \rpquote / \lpquote P \rpquote \id{\}}            = 
  \left\{ 
    \begin{array}{ccc}
      \lpquote Q \rpquote & & x \nameeq \lpquote P \rpquote \\
      x & & otherwise \\
    \end{array}
  \right. \nonumber
\end{eqnarray}

and $z$ is chosen distinct from $\quotep{P}$, $\quotep{Q}$, the free
names in $Q$, and all the names in $R$. Our $\alpha$-equivalence will
be built in the standard way from this substitution.

\begin{remark}\label{rem:no_self_referential_names}
  One consequence of these definitions is that $\forall P. \quotep{P}
  \not\in \freenames{P}$.
\end{remark}

\subsection{ Dynamic quote: an example }

Anticipating something of what's to come, consider applying the
substitution, $\widehat{\id{\{}u / z \id{\}}}$, to the following pair
of processes, $\lift{w}{y!(z)}$ and $w[ \lpquote y!(z) \rpquote ]$.

\begin{eqnarray}
	\lift{w}{y!(z)}\widehat{\id{\{}u / z \id{\}}}
		& = &
		\lift{w}{y!(u)} \nonumber\\
	w[ \lpquote y!(z) \rpquote ] \widehat{ \id{\{}u / z \id{\}} }
		& = &
		w[ \lpquote y!(z) \rpquote ] \nonumber
\end{eqnarray}

Because the body of the process between quotes is impervious to
substitution, we get radically different answers. In fact, by
examining the first process in an input context,
e.g. $x?(z).\lift{w}{y!(z)}$, we see that the process under the lift
operator may be shaped by prefixed inputs binding a name inside it. In
this sense, the lift operator will be seen as a way to dynamically
construct processes before reifying them as names.

Finally equipped with these standard features we can present the
dynamics of the calculus.

\subsubsection{Operational semantics} 

Finally, we introduce the computational dynamics. What marks these
algebras as distinct from other more traditionally studied algebraic
structures, e.g. vector spaces or polynomial rings, is the manner in
which dynamics is captured. In traditional structures, dynamics is typically
expressed through morphisms between such structures, as in linear maps
between vector spaces or morphisms between rings. In algebras
associated with the semantics of computation, the dynamics is
expressed as part of the algebraic structure itself, through a
reduction reduction relation typically denoted by $\red$. Below, we
give a recursive presentation of this relation for the calculus used
in the encoding.

$\red \subseteq \pi \times \pi$
$\red : \pi \to \mathcal{P}(\pi)$

\begin{mathpar}
  \inferrule* [lab=Comm] { \textsf{match}( x_{src}, x_{trgt} ) } { x_{trgt}?(y)P \; | \; x_{src}!\langle {Q} \rangle \red P\{\quotep{Q}/y}\} }
  \and \\
  \inferrule* [lab=Par] {{P} \red {P}'} {{{P} | {Q}} \red {{P}' | {Q}}}
  \and
  \inferrule* [lab=Equiv]{{{P} \scong {P}'} \andalso {{P}' \red {Q}'} \andalso {{Q}' \scong {Q}}}{{P} \red {Q}}
\end{mathpar}

\begin{eqnarray*}
  match_{\equiv} (\quotep{P},\quotep{Q}) & := & P \equiv Q \\
  match_{\dagger}(\quotep{P},\quotep{Q}) & := & \forall R. P|Q \red^{*} R => R \red^{*} 0 \\
  match_{K}(\quotep{P},\quotep{Q}) & := & K \mbox{ for some context } K
\end{eqnarray*}

$u?(x)P | u!\langle Q \rangle \red P\{\quotep{Q}/x\}$

%We write $\wred$ for $\red^*$, and $P\red$ if $\exists Q $ such that $ P \red Q$.
We write $P\red$ if $\exists Q $ such that $ P \red Q$ and $P\not\red$, otherwise.

\section{Replication}

As mentioned before, it is known that replication (and hence
recursion) can be implemented in a higher-order process algebra
\cite{SangiorgiWalker}. As our first example of calculation with the
machinery thus far presented we give the construction explicitly in
the {\rhoc}.

\begin{eqnarray}
	D_{x} & := & \prefix{x}{y}{(\binpar{\outputp{x}{y}}{@{y}})} \nonumber\\
	\bangp_{x}{P} & := & \binpar{{x}!\langle{\binpar{D_{x}}{P}}\rangle}{D_{x}} \nonumber
\end{eqnarray}

\begin{eqnarray}
	\bangp_{x}{P} & & \nonumber\\
	=
	& {x}!\langle{(\prefix{x}{y}{(\outputp{x}{y} | @{y})) | P}}\rangle 
	      | \prefix{x}{y}{(\outputp{x}{y} | @{y})} & \nonumber\\
	\red
	& (\outputp{x}{y} | @{y})\substn{\quotep{(\prefix{x}{y}{(@{y} | \outputp{x}{y})) | P}}}{y} & \nonumber\\
	=
	& \outputp{x}{\quotep{(\prefix{x}{y}{(\outputp{x}{y} | @{y})) | P}}}
	  | {(\prefix{x}{y}{(\outputp{x}{y} | @{y})) | P}} & \nonumber\\
	\red
	& \ldots & \nonumber\\
	\red^*
	& P | P | \ldots & \nonumber
\end{eqnarray}

Of course, this encoding, as an implementation, runs away, unfolding
$\bangp{P}$ eagerly. A lazier and more implementable replication
operator, restricted to input-guarded processes, may be obtained as follows.

\begin{eqnarray}
\bangp{\prefix{u}{v}{P}} 
	:= 
	\binpar{\lift{x}{\prefix{u}{v}{(\binpar{D(x)}{P})}}}{D(x)} \nonumber
\end{eqnarray}

\begin{remark}
  Note that the lazier definition still does not deal with summation
  or mixed summation (i.e. sums over input and output). The reader is
  invited to construct definitions of replication that deal with these
  features. 

  Further, the definitions are parameterized in a name, $x$. Can you,
  gentle reader, make a definition that eliminates this parameter and
  guarantees no accidental interaction between the replication
  machinery and the process being replicated -- i.e. no accidental
  sharing of names used by the process to get its work done and the
  name(s) used by the replication to effect copying. This latter
  revision of the definition of replication is crucial to obtaining
  the expected identity $!!P \sim !P$.
\end{remark}

\begin{remark}\label{rem:paradoxical_combinator}
  The reader familiar with the lambda calculus will have noticed the
  similarity between $D$ and the paradoxical combinator.

  [Ed. note: the existence of this seems to suggest we have to be more
  restrictive on the set of processes and names we admit if we are to
  support no-cloning.]
\end{remark}

\subsubsection{Bisimulation}

The computational dynamics gives rise to another kind of equivalence,
the equivalence of computational behavior. As previously mentioned
this is typically captured \emph{via} some form of bisimulation.

% The notion we use in this paper is weak barbed bisimulation
% \cite{milner91polyadicpi}.

The notion we use in this paper is derived from weak barbed
bisimulation \cite{milner91polyadicpi}. 

\begin{definition}
An \emph{observation relation}, $\downarrow_{\mathcal N}$, over a set
of names, $\mathcal N$, is the smallest relation satisfying the rules
below.

\infrule[Out-barb]{y \in {\mathcal N}, \; x \nameeq y}
		  {\outputp{x}{v} \downarrow_{\mathcal N} x}
\infrule[Par-barb]{\mbox{$P\downarrow_{\mathcal N} x$ or $Q\downarrow_{\mathcal N} x$}}
		  {\binpar{P}{Q} \downarrow_{\mathcal N} x}

We write $P \Downarrow_{\mathcal N} x$ if there is $Q$ such that 
$P \wred Q$ and $Q \downarrow_{\mathcal N} x$.
\end{definition}

\begin{definition}
%\label{def.bbisim}
An  ${\mathcal N}$-\emph{barbed bisimulation} over a set of names, ${\mathcal N}$, is a symmetric binary relation 
${\mathcal S}_{\mathcal N}$ between agents such that $P\rel{S}_{\mathcal N}Q$ implies:
\begin{enumerate}
\item If $P \red P'$ then $Q \wred Q'$ and $P'\rel{S}_{\mathcal N} Q'$.
\item If $P\downarrow_{\mathcal N} x$, then $Q\Downarrow_{\mathcal N} x$.
\end{enumerate}
$P$ is ${\mathcal N}$-barbed bisimilar to $Q$, written
$P \wbbisim_{\mathcal N} Q$, if $P \rel{S}_{\mathcal N} Q$ for some ${\mathcal N}$-barbed bisimulation ${\mathcal S}_{\mathcal N}$.
\end{definition}

$\mathcal{R} \subseteq \pi \times \pi$

$P \mathcal{R} Q => \forall P'. P \red P' \Rightarrow \exists Q'. Q \red Q', P' \mathcal{R} Q'$

$P \vdash x \Rightarrow Q \vdash x$

\begin{mathpar}
  \inferrule*[lab=Out-barb]{x \nameeq y}{{y}!\langle{Q}\rangle \vdash x}
  \and
  \inferrule*[lab=Par-barb]{\mbox{$P\vdash x$ or $Q\vdash x$}}{\binpar{P}{Q} \vdash x}
\end{mathpar}

\subsubsection{Contexts}

One of the principle advantages of computational calculi like the
$\pi$-calculus is a well-defined notion of context,
contextual-equivalence and a correlation between
contextual-equivalence and notions of bisimulation. The notion of
context allows the decomposition of a process into (sub-)process and
its syntactic environment, its context. Thus, a context may be
thought of as a process with a ``hole'' (written $\Box$) in it. The
application of a context $M$ to a process $P$, written $M[P]$, is
tantamount to filling the hole in $M$ with $P$. In this paper we do
not need the full weight of this theory, but do make use of the notion
of context in the proof the main theorem. 

\begin{mathpar}
  \inferrule* [lab=summation] {} {{M_{M},M_{N}} \bc \Box \;|\; x.M_{A} \;|\; M_{M}+M_{N}}
  \and
  \inferrule* [lab=agent] {} {{M_{A}} \bc (\vec{x})M_{P} \;| \; \clift{P_0,\ldots,M_{P},\ldots,P_N}}
  \and \\
  \inferrule* [lab=process] {} {{M_{P}} \bc M_{N} \;| \;P|M_{P} }
\end{mathpar} 

\begin{mathpar}
  \inferrule* [lab=sychronization] {} {M_{N} \bc \Box \;|\; x?M_{F} \;|\; x!M_{C}}
  \and
  \inferrule* [lab=abstraction] {} {{M_{F}} \bc (x)M_{P} }
  \and
  \inferrule* [lab=concretion] {} {{M_{C}} \bc \langle M_{P} \rangle }
  \and \\
  \inferrule* [lab=process] {} {{M_{P}} \bc M_{N} \;| \;P|M_{P} }
\end{mathpar}

\begin{definition}[contextual application] Given a context $M$, and
  process $P$, we define the \emph{contextual application}, $M[P] :=
  M\{P/\Box\}$. That is, the contextual application of M to P is the
  substitution of $P$ for $\Box$ in $M$.
\end{definition}

$\meaningof{-} : L \to \mathcal{P}(\pi)$

\begin{mathpar}
  \inferrule* [lab=collection] {} {\meaningof{true} = \pi, \and \meaningof{~E} = \pi \setminus \meaningof{E}, \and \meaningof{E_{1} \& E_{2}} = \meaningof{E_{1}} \cap \meaningof{E_{2}}}
\end{mathpar}

\begin{mathpar}
  \inferrule* [lab=structure] {} {\meaningof{0} = \{ P \in \pi | P \equiv 0 \}, \and \\ \meaningof{E_1 | E_2} = \{ P \in \pi | P \equiv P_{1} | P_{2}, P_{1} \in \meaningof{E_{1}}, P_{2} \in \meaningof{E_2}\} }
\end{mathpar}

\begin{mathpar}
 \inferrule* [lab=behavior] {} {\meaningof{\langle a?b \rangle E} = \{ P \in \pi | P \equiv Q | u?(y)P', \\ \and \\\\ \and \\ \;\;\; u \in \meaningof{a}, \forall z.P'\{z/y\} \in \meaningof{E\{z/b\}}\}, \and \\ \meaningof{a!E} = \{ P \in \pi | P \equiv Q | x!\langle P' \rangle, x \in \meaningof{a} P' \in \meaningof{E}\} }
\end{mathpar}

\begin{mathpar}
 \inferrule* [lab=nominal] {} {\meaningof{\quotep{E}} = \{ \quotep{P} \in \quotep{\pi} | P \in \meaningof{E} \}, \and \meaningof{\quotep{P}} = \{ \quotep{Q} \in \quotep{\pi} | P \equiv Q \} \and \\ \meaningof{@\quotep{E}} = \{ P \in \pi | P \equiv @x, x \in \meaningof{E} \}}
\end{mathpar}

\begin{eqnarray*}
  \\
  \meaningof{-} : TS \to ST
\end{eqnarray*}

\begin{eqnarray*}
  \\
  L : TS \to ST
\end{eqnarray*}

\begin{eqnarray*}
  \\
  P \models E \iff P \in \meaningof{E}
\end{eqnarray*}

\begin{eqnarray*}
  P \approx_{L} Q \iff \forall E \in L. P \models E \iff Q \models E
\end{eqnarray*}

\begin{eqnarray*}
  P \approx_{K} Q
\end{eqnarray*}

\begin{eqnarray*}
  P \approx Q
\end{eqnarray*}

$\approx_{K} = \approx = \approx_{L}$

\subsubsection{Contextual duality}

Note that contexts extend the quotation operation to a family of
operations from processes to names. Given a context, $M$, we can
define a \emph{nominal context}, $\quotep{M}$ by $\quotep{M}[P] :=
\quotep{M[P]}$. To foreshadow what is to come we observe that these
operations enjoy a duality with processes very much like the duality
between vectors and maps from vectors to scalars.

Further, because the calculus is essentially higher-order, we have a
correspondence between contexts and processes. More specifically,
given a name $x$ and a context $M$ we can construct $M^{*}_{x}$ such
that 

\begin{mathpar}
  M^{*}_{x} | \lift{x}{P} \red M[P]
\end{mathpar}

namely,

\begin{mathpar}
  M^{*}_{x} := x?(u).M[\dropn{u}]
\end{mathpar}

The dependence of $M^{*}_{x}$ on a name makes it an abstraction, 

\begin{mathpar}
  M^{*} := (x)x?(u).M[\dropn{u}]
\end{mathpar}

\subsection{Additional notation}

It will sometimes be convenient to denote the process a name
quotes. We already have the notation $x = \quotep{P}$, but it will be
convenient to introduce an alternate notation, $\procn{x}$, when we
want to emphasize the connection to the use of the name. Note that, by
virtue of name equivalence, $\quotep{\procn{x}} \nameeq x$; so, the
notation is consistent with previous definitions.

Further, because names have structure it is possible to effect
substitutions on the basis of that structure. This means we need to
upgrade our notation for substitutions, which we accomplish by
adapting comprehension notation. Thus,

\begin{mathpar}
  P\{ y / x : x \in S \}
\end{mathpar}

is interpreted to mean the process derived from P by replacing (in a
capture-avoiding manner) each occurrence of $x$ in $S$ by $y$. For example,

\begin{mathpar}
  P\{ \quotep{\procn{x}|\procn{x}} / x : x \in \freenames{P} \}
\end{mathpar}

will replace each (occurrence) of a free name $x$ in $P$ by
$\quotep{\procn{x}|\procn{x}}$.

Also, we will avail ourselves of the notation $x^{L}$ and $x^{R}$ to
denote injections of a name into disjoint copies of the name
space. There are numerous ways to accomplish this. One example can be
found in \cite{MeredithR05}. This notation overloads to vectors of
names: $\vec{x}^{\pi} := (x_{i}^{\pi} \; : \; 0 \leq i < |\vec{x}| )$ where $\pi \in \{L,R\}$.

We also use $P^{\Box} := P|\Box$.

In \cite{MeredithR05} an interpretation of the new operator is
given. It turns out that there are several possible interpretations
all enjoying the requisite algebraic properties of the operator (see
\cite{milner91polyadicpi}). We will therefore make liberal use of
$(\nu\; \vec{x})P$.

% subsection the_syntax_and_semantics_of_the_notation_system (end)   

\section{Interpretation of QM}
\subsection{Supporting definitions}
\subsubsection{Multiplication}
\begin{mathpar}
  \quotep{Q} \cdot \quotep{R} := \quotep{Q|R}
  \and \\
  \quotep{Q} \cdot P := P\{ \quotep{Q|R} / \quotep{R} : \quotep{R} \in \freenames{P} \}
\end{mathpar}

\paragraph{Discussion}
The first line needs little explanation. The second line says that
each free name of the process is replaced with the multiplication of
that name by the scalar. Multiplication of a scalar (name) by a state
(process) results in a process all the names of which have been `moved
over' by parallel composition with the process the scalar
quotes. There is a subtlety that the bound names have to be
manipulated so that multiplied names aren't accidentally
captured. There are many ways to achieve this.

\begin{remark}\label{rem:multiplication_identities}
  The reader is invited to verify that for all $x,y,z \in \QProc$ and $P \in \Proc$
  \begin{mathpar}
    x \cdot \quotep{0} \equiv x 
    \and
    x \cdot y \equiv y \cdot x
    \and
    x \cdot (y \cdot z) \equiv (x \cdot y) \cdot z
    \and \\
    \quotep{0} \cdot P \equiv P
    \and \\
    x \cdot (y \cdot P) \equiv (x \cdot y) \cdot P
    \and \\
    x \cdot (P|Q) \equiv (x \cdot P) | (x \cdot Q)
    \and \\    
  \end{mathpar}
\end{remark}

\subsubsection{Tensor product}

We define a tensor product on processes by structural induction.

\paragraph{Tensor of sums} First note that all summations, including
$\pzero$ and sequence, can be written $\Sigma_{i} x_{i}.A_{i} +
\Sigma_{j} x_{j}.C_{j}$, where we have grouped input-guarded processes
together and output-guarded processes together.

Thus, we can define the tensor product of two summations, $N_{1}\otimes N_{2}$, where

\begin{mathpar}
  N_{1} := \Sigma_{i} x_{i}.A_{i} + \Sigma_{j} x_{j}.C_{j}
  \and
  N_{2} := \Sigma_{i'} y_{i'}.B_{i'} + \Sigma_{j'} y_{j'}.D_{j'} 
\end{mathpar}

as follows.

\begin{mathpar}
  \Sigma_{i} x_{i}.A_{i} + \Sigma_{j} x_{j}.C_{j} \otimes \Sigma_{i'}
  y_{i'}.B_{i'} + \Sigma_{j'} y_{j'}.D_{j'} 
  \and \\
  := \; \Sigma_{i} \Sigma_{i'} \quotep{\stackrel{\vee}{x_{i}}| \stackrel{\vee}{y_{i'}}}.(A_{i}\otimes B_{i'}) \; | \; \Sigma_{i'} \Sigma_{i} \quotep{\stackrel{\vee}{y_{i'}}|\stackrel{\vee}{x_{i}}}.(B_{i'}\otimes A_{i})
  \and
  \;\; | \;\; \Sigma_{j} \Sigma_{j'} \quotep{\stackrel{\vee}{x_{j}}|\stackrel{\vee}{y_{j'}}}.(A_{j}\otimes B_{j'}) \; | \; \Sigma_{j'} \Sigma_{j} \quotep{\stackrel{\vee}{y_{j'}}|\stackrel{\vee}{x_{j}}}.(B_{j'}\otimes A_{j})
\end{mathpar}

\begin{remark}
  Do we need to $x^{L}$ and $y^{R}$ for this construction as well?
\end{remark}

\paragraph{Tensor of parallel compositions} Next, we distribute tensor
over par.

\begin{mathpar}
  P_{1}|P_{2} \otimes Q_{1}|Q_{2} := (P_{1} \otimes Q_{1}) | (P_{1}
  \otimes Q_{2}) | (P_{2} \otimes Q_{1}) | (P_{2} \otimes Q_{2})
\end{mathpar}

\paragraph{Tensor with dropped names} We treat tensor of a
process with a dropped name as parallel composition.

\begin{mathpar}
  P \otimes \dropn{x} := P | \dropn{x}
\end{mathpar}

\paragraph{Tensor of agents}

Finally, we need to define tensor on agents. Note that the definition
of tensor on normal products only tensors inputs with inputs and
outputs with outputs. Thus, we only have to define the operation on
``homogeneous'' pairings.

\begin{mathpar}
  (\vec{x})P \otimes (\vec{y})Q
  \and \\
  := (x_{0}^{L}|y_{0}^{R},\ldots,x_{0}^{L}|y_{n}^{R},\ldots,x_{m}^{L}|y_{0}^{R},\ldots,x_{m}^{L}|y_{n}^R)(P\{ \vec{x}^{L}/\vec{x}\} \otimes Q \{ \vec{y}^{R}/\vec{y}\})
  \and \\
  \clift{\vec{P}} \otimes \clift{\vec{Q}}
  \and \\
  := \clift{P_{0}\otimes Q_{0},\ldots,P_{0}\otimes Q_{n},\ldots,P_{m}\otimes Q_{0},\ldots,P_{m}\otimes Q_{n}}
\end{mathpar}

\begin{remark}
  Observe that arities of tensored abstractions matches arities of
  tensored concretions if the original arities matched. Note also that
  the length of the arities corresponds to the increase in dimension
  we see in ordinary vector space tensor product.
\end{remark}

\begin{remark}
  Operationally, this definition distributes the tensor down to
  components ``linked'' by summation. Tensor over summation is
  intriguing in that it mixes names. Moreover, as a consequence of the
  way it mixes names we have the identities for all $x \in \QProc$ and
  $P,Q \in \Proc$

  \begin{mathpar}
    (x \cdot P) \otimes Q \equiv x \cdot (P \otimes Q) \equiv P \otimes (x \cdot Q)
    \and
    P \otimes \pzero \equiv P
  \end{mathpar}

  that the reader is invited to verify.
\end{remark}

\subsubsection{Annihilation}
\begin{mathpar}
  P^{\perp} := \{ Q | \forall R. P|Q \red^{*} R \Rightarrow R \red^{*} \pzero \}
  \and \\
  P^{\underline{\perp}} := \Sigma_{Q \in P^{\perp}} \quotep{Q}?(y).(\dropn{y}|Q) | \Sigma_{Q \in P^{\perp}} \quotep{Q}\clift{\Box}
\end{mathpar}

\paragraph{Discussion} The reader will note that $P^{\perp}$ is a
\emph{set} of processes, while $P^{\underline{\perp}}$ is a
\emph{context}. We call the set $P^{\perp}$ the \emph{annihilators} of
$P$. The parallel composition of a process in the annihilators of $P$
with $P$ will result in a process, the state space of which has all
paths eventually leading to $\pzero$. Execution may endure loops; but
under reasonable conditions of fairness (naturally guaranteed under
most notions of bisimulation) such a composite process cannot get
stuck in such a loop and will, eventually pop out and terminate.

The context $P^{\underline{\perp}}$ is ready and willing to ``take the
$P$ out of'' the process to which it is applied. It will effectively
transmit the code of the process to which it is applied to one of the
annihilators and run the process against it.

\subsubsection{Evaluation}
We fix $M$ a domain of fully abstract interpretation with an equality
coincident with bisimulation. We take $\meaningof{\cdot} : \Proc \to
M$ to be the map interpreting processes and $\nmeaningof{\cdot} : \M
\to Proc$ to be the map running the other way. Then we define

\begin{mathpar}
  \int P := \nmeaningof{\meaningof{P}}
\end{mathpar}

\paragraph{Discussion}
There are many fully abstract interpretations of Milner's
$\pi$-calculus. Any of them can be used as a basis for interpreting
the reflective calculus here. Equipped with such a domain it is
largely a matter of grinding through to check that the Yoneda
construction for the normalization-by-evaluation program can be
extended to this setting.

\begin{remark}
  The reader is invited to verify that $\int (P^{\underline{\perp}}[P]) = 0$.
\end{remark}

\subsection{Quantum mechanics}

Table \ref{tbl:core_qm_op_defns} gives the core operational definitions

\begin{table}[htp]\label{tbl:core_qm_op_defns}
  \center{
    \fbox{
      \begin{tabular}{c|c}
        quantum mechanics & process calculus \\
        \hline
        scalar & $x := \quotep{P}$ \\
        state vector & $\state{P} := P$ \\
        dual & $\state{P}^{*} := \event{P^{\underline{\perp}}} := \quotep{P^{\underline{\perp}}}[-]$ \\
        matrix & $ \Sigma_{\alpha} \state{P_{\alpha}}x_{\alpha}\event{Q_{\alpha}}$ \\
        vector addition & $\state{P} + \state{Q} := \state{P | Q}$ \\
        tensor product & $\state{P} \otimes \state{Q} := \state{P \otimes Q}$ \\
        inner product & $\innerprod{P}{Q} := \quotep{\int P^{\underline{\perp}}[Q]}$ \\
      \end{tabular}
    }
  }
  \caption{QM - operational definitions}
\end{table}

where

\begin{mathpar}
  \prmatrix{P}{Q} := \fprmatrix{P}{\quotep{\pzero}}{Q}
  \and
  \fprmatrix{P}{x}{Q} := (\state{P},x,\event{Q})
  \and
  (\fprmatrix{P}{x}{Q})(\state{R}) := x \cdot \innerprod{Q}{R} \cdot \state{P}
  \and
  (\fprmatrix{P}{x}{Q})(\event{R}) := x \cdot \innerprod{R}{P} \cdot \event{Q}
\end{mathpar}

\paragraph{Discussion}
As promised: vectors (aka states) are represented as processes; duals
as contextual duals; inner product definition should be compared with
standard inner product definition for ....

\begin{remark}
  Assuming $\int (P^{\underline{\perp}}[P]) = 0$, the reader is
  invited to verify that $(\fprmatrix{P}{x}{P})(\state{P}) = x \cdot \state{P}$.
\end{remark}

\begin{remark}
  The reader is invited to verify that $\innerprod{P}{Q}$ could
  equally well have been written $\quotep{\int \stackrel{\vee}{x}}$
  where $x = \event{P^{\underline{\perp}}}(Q)$.

  One of the motivations for this remark is that there is another way
  to factor these operations. We could package up evaluation in the dual:

  \begin{mathpar}
    \state{P}^{*} := \event{\int P^{\underline{\perp}}} := \quotep{\int P^{\underline{\perp}}}[-]
  \end{mathpar}

  and then have inner product defined by
  
  \begin{mathpar}
    \innerprod{P}{Q} := \event{P}(Q)
  \end{mathpar}

  Hopefully, experience with the calculations will provide guidance on
  the best factoring.
\end{remark}

\begin{remark}
  Assuming $\int (P^{\underline{\perp}}[P]) = 0$, the reader is
  invited to verify that $\forall P,Q. (\prmatrix{0}{Q})(\state{0}) =
  \state{0}$ and dually $(\prmatrix{P}{0})(\event{0}) = \event{0}$.
\end{remark}

\begin{remark}
  i'm a little worried that i don't (yet) have proper support for
  complex conjugacy. But, the observation above may give us a
  clue. According to Abramsky, it must be the case that the scalars
  are iso to the homset of the identity for the tensor -- which the
  observation above characterizes. 

  For now, we will simply bookmark the notion with $\overline{x}$.
\end{remark}

\subsubsection{Adjointness}

We need to give a definition of $(\cdot)^{\dagger}$ for matrices. The
obvious candidate definition is
\begin{mathpar}
(\Sigma_{\alpha}\fprmatrix{P_{\alpha}}{x_{\alpha}}{Q_{\alpha}})^{\dagger}
= \Sigma_{\alpha}\fprmatrix{(Q_{\alpha}^{\underline{\perp}})^{*}}{\overline{x}_{\alpha}}{P_{\alpha}^{\underline{\perp}}} 
\end{mathpar}

But, $(Q_{\alpha}^{\underline{\perp}})^{*}$ requires a name along
which to communicate the process to achieve the context application.

\subsubsection{Basis for a basis}
If processes label states and ``addition'' of states (a.k.a. vector
addition) is interpreted as parallel composition, what corresponds to
notions of linear independence and basis? Here, we recall that Yoshida
has developed a set of \emph{combinators} for an asynchronous verison
of Milner's $\pi$-calculus. These are a finite set of processes such
any process can be expressed as parallel composition of these
combinators together with liberal uses of the new operator and
replication. We can simply give a translation of these into the
present calculus and have reasonable expectation that the property
carries over. That is, that the resultant set allows to express all
processes via parallel composition. Note, however, that there is no
new operator or replication in this calculus. As a result, we expect
that the corresponding set is actually infinite. That is, we expect
that the space is actually infinite dimensional.

\begin{remark}
  The attentive reader may be a bit concerned. Certainly, the
  collection $S$, $K$ and $I$ is a finite set of
  combinators. Shouldn't we expect to see a finite set of combinators
  for an effectively equivalent system? i am very sympathetic to this
  critique and feel it warrants full attention. On the other hand, i
  also have in mind the following analogy. The natural numbers, as a
  monoid under addition, has exactly $1$ generator, while the natural
  numbers, as a monoid under multiplication, has countably many
  generators (the primes). We observe that the application of the
  lambda calculus is much less resource sensitive than the parallel
  composition of the $\pi$-calculus. Could it be the case that we have
  an analogy of the form
  
  \begin{mathpar}
    m + n : MN :: m*n : M|N
  \end{mathpar}

  giving a similar blow up in the set of ``primes''?  This is such a
  wonderful thought that, even if it's not true, i think it's worth
  writing down.
\end{remark}
 

\documentclass[12pt]{llncs}
%\documentclass{jktr}

\usepackage[pdftex]{hyperref}                   
\usepackage {listings}
\usepackage {mathpartir}
\usepackage{bcprules}
%\usepackage{listings}
                       
\usepackage{graphicx} 
%\usepackage[margins=2.5cm,nohead,nofoot]{geometry}
%\usepackage{geometry}
\usepackage{amsfonts}
\usepackage{amstext}
\usepackage{latexsym}
\usepackage{amssymb}
\usepackage{color}


%\include{myPreamble}
\include{qm2pi.local} 

%\ifpdf
%\usepackage[pdftex]{graphicx}
%\else
%\usepackage{graphicx}
%\fi

 % \ifpdf
%  \usepackage{pdfsync}
%  \if


%\title{Brief Article}
%\author{David F. Snyder}
%\author{L.G. Meredith}

%\address{Dept. of Math., Texas State University--San Marcos, San Marcos, TX 78666}
       
\pagestyle{empty}


\begin{document}

\lstset{language=[Objective]Caml,frame=shadowbox}

\input{qm2pi.front}

% section front matter (end)

\input{qm2pi.intro} 
 
% section introduction (end)

% \input{qm2pi.knotations} 

% section notation (end)

\input{qm2pi.process.calculi} 

% section concurrent_process_calculi_and_spatial_logics_ (end)
    
%\input{qm2pi.knots2pi} 

%\input{qm2pi.trefoil} 

%\input{qm2pi.mainthm} 

% subsection basic_interpretation (end)

%\input{qm2pi.rho.presentation} 
\subsection{The syntax and semantics of the notation system}\label{sub:the_syntax_and_semantics_of_the_notation_system} % (fold)

We now summarize a technical presentation of the calculus that
embodies our theory of dynamics. The typical presentation of such a
calculus follows the style of giving generators and relations on
them. The grammar, below, describing term constructors, freely
generates the set of processes, $\Proc$. This set is then quotiented
by a relation known as structural congruence and it is over this set
that the notion of dynamics is expressed. This presentation is
essentially that of \cite{MeredithR05} with the addition of
polyadicity and summation. For readability we have relegated some of
the technical subtleties to an appendix.

\subsubsection{Process grammar}\label{subsub:process_grammar}

\begin{mathpar}
  \inferrule* [lab=synchronization] {} {{M} \bc \pzero \;|\; x?F \;|\; x!C }
  \and
  \inferrule* [lab=abstraction] {} {{F} \bc (x)P}
  \and
  \inferrule* [lab=concretion] {} {{C} \bc \langle Q \rangle}
  \and
  \inferrule* [lab=process] {} {{P,Q} \bc M \;| \;P|Q \;|\; @{x}}
  \and
  \inferrule* [lab=name] {} {{x} \bc \quotep{P}}
\end{mathpar} 

Note that $\vec{x}$ (resp. $\vec{P}$) denotes a vector of names
(resp. processes) of length $|\vec{x}|$ (resp. $|\vec{P}|$). We adopt
the following useful abbreviations.

\begin{mathpar}
   x?(\vec{y}).P := x.(\vec{y})P \and  x\clift{\vec{P}} := x.\clift{\vec{P}}
   \and x!(y) := \lift{x}{\dropn{y}}
   \and \Pi_{i=0}^{n-1}P_i := P_0 | \ldots | P_{n-1}
\end{mathpar}

\subsubsection{Structural congruence}

\paragraph{Free and bound names and alpha-equivalence.} At the
core of structural equivalence is alpha-equivalence which identifies
process that are the same up to a change of variable. Formally, we
recognize the distinction between free and bound names. The free names
of a process, $\freenames{P}$, may be calculated recursively as
follows:

\begin{mathpar}
\freenames{\pzero} := \emptyset
  \and \\
  \freenames{x?(y).P} := \{ x \} \cup (\freenames{P} \setminus \{ y \})
  \and 
  \freenames{x!\langle P \rangle} := \{ x \} \cup \{ P \} 
  \and \\
  \freenames{P|Q} := \freenames{P} \cup \freenames{Q}
  \and \\
  \freenames{@{x}} := \{ x \}
\end{mathpar}

$\pi$
$\quotep{\pi}$

$\freenames{-} : \pi \to \mathcal{P}(\quotep{\pi})$

\begin{eqnarray*}
  \freenames{\pzero} & := & \emptyset \\
  \freenames{x?(y).P} & := & \{ x \} \cup (\freenames{P} \setminus \{ y \}) \\
  \freenames{x!\langle P \rangle} & := & \{ x \} \cup \{ P \} \\
  \freenames{P|Q} & := & \freenames{P} \cup \freenames{Q} \\
  \freenames{\dropn{x}} & := & \{ x \}
\end{eqnarray*}

The bound names of a process, $\boundnames{P}$, are those names occurring in $P$
that are not free. For example, in $x?(y).0$, the name $x$ is free, while $y$ is bound.

\begin{mathpar}
  \inferrule* [lab=monoidal-laws] {} { P|Q \equiv Q|P \and P|0 \equiv P \and P|(Q|R) \equiv (P|Q)|R }
\end{mathpar}

\begin{mathpar}
  \inferrule* [lab=alpha-equivalence] {} { (x)P \equiv (y)P\{y/x\} \and y \not\in \freenames{P} }
\end{mathpar}

\begin{definition}
Then two processes, $P,Q$, are alpha-equivalent if $P = Q\{\vec{y}/\vec{x}\}$ for
some $\vec{x} \in \boundnames{Q},\vec{y} \in \boundnames{P}$, where $Q\{\vec{y}/\vec{x}\}$
denotes the capture-avoiding substitution of $\vec{y}$ for $\vec{x}$ in $Q$.
\end{definition}

\begin{definition}
  The {\em structural congruence} \cite{SangiorgiWalker} , $\equiv$,
  between processes is the least congruence containing
  alpha-equivalence, satisfying the abelian monoid laws
  (associativity, commutativity and $\pzero$ as identity) for parallel
  composition $|$ and for summation $+$.
\end{definition}

\subsection{Name equivalence}

We take name equivalence, written $\nameeq$, to be the smallest
equivalence relation generated by the following rules.

\begin{mathpar}
\inferrule*[lab=Quote-drop]
{ }
{ \quotep{@{x}} \nameeq x }

\inferrule*[lab=Struct-equiv]
{ P \scong Q }
{ \quotep{P} \nameeq \quotep{Q} }
\end{mathpar}

The astute reader will have noticed that the mutual recursion of names
and processes imposes a mutual recursion on alpha-equivalence and
structural equivalence via name-equivalence. Fortunately, all of this
works out pleasantly and we may calculate in the natural way, free of
concern. The reader interested in the details is referred to the
appendix \ref{appendix:rho_details}.

\subsection{Substitution}

We use $\Proc$ for the set of processes, $\QProc$ for the set of
names, and $\id{\{}\vec{y} / \vec{x} \id{\}}$ to denote partial maps,
$s : \QProc \rightarrow \QProc$. A map, $s$ lifts, uniquely, to a map
on process terms, $\widehat{s} : \Proc \rightarrow \Proc$ by the
following equations.

\begin{mathpar}
  (0) \psubstp{Q}{P} := 0 \\
  (R \juxtap S) \psubstp{Q}{P}
  :=    
  (R)\psubstp{Q}{P} \juxtap (S) \psubstp{Q}{P} \\
  (x?(y).R) \psubstp{Q}{P}    
  :=    
  (x)\substp{Q}{P} (z)\concat( (R \psubstn{z}{y}) \psubstp{Q}{P} ) \\
  (\lift{x}{R}) \psubstp{Q}{P}  
  :=
  \lift{(x)\substp{Q}{P}}{ R \psubstp{Q}{P} } \\
%   (\dropn{x})  \psubstp{Q}{P}       
%   := 
%   \left\{ 
%     \begin{array}{ccc} 
%       \dropn{\quotep{Q}} & & x \nameeq \quotep{P} \\
%       \dropn{x} & & otherwise \\
%     \end{array}
%   \right. 
  (\dropn{x})  \psubstp{Q}{P}       
  := 
  \left\{ 
    \begin{array}{ccc} 
      Q & & x \nameeq \quotep{P} \\
      \dropn{x} & & otherwise \\
    \end{array}
  \right.
\end{mathpar}
 

where

\begin{eqnarray}
  (x)\id{\{} \lpquote Q \rpquote / \lpquote P \rpquote \id{\}}            = 
  \left\{ 
    \begin{array}{ccc}
      \lpquote Q \rpquote & & x \nameeq \lpquote P \rpquote \\
      x & & otherwise \\
    \end{array}
  \right. \nonumber
\end{eqnarray}

and $z$ is chosen distinct from $\quotep{P}$, $\quotep{Q}$, the free
names in $Q$, and all the names in $R$. Our $\alpha$-equivalence will
be built in the standard way from this substitution.

\begin{remark}\label{rem:no_self_referential_names}
  One consequence of these definitions is that $\forall P. \quotep{P}
  \not\in \freenames{P}$.
\end{remark}

\subsection{ Dynamic quote: an example }

Anticipating something of what's to come, consider applying the
substitution, $\widehat{\id{\{}u / z \id{\}}}$, to the following pair
of processes, $\lift{w}{y!(z)}$ and $w[ \lpquote y!(z) \rpquote ]$.

\begin{eqnarray}
	\lift{w}{y!(z)}\widehat{\id{\{}u / z \id{\}}}
		& = &
		\lift{w}{y!(u)} \nonumber\\
	w[ \lpquote y!(z) \rpquote ] \widehat{ \id{\{}u / z \id{\}} }
		& = &
		w[ \lpquote y!(z) \rpquote ] \nonumber
\end{eqnarray}

Because the body of the process between quotes is impervious to
substitution, we get radically different answers. In fact, by
examining the first process in an input context,
e.g. $x?(z).\lift{w}{y!(z)}$, we see that the process under the lift
operator may be shaped by prefixed inputs binding a name inside it. In
this sense, the lift operator will be seen as a way to dynamically
construct processes before reifying them as names.

Finally equipped with these standard features we can present the
dynamics of the calculus.

\subsubsection{Operational semantics} 

Finally, we introduce the computational dynamics. What marks these
algebras as distinct from other more traditionally studied algebraic
structures, e.g. vector spaces or polynomial rings, is the manner in
which dynamics is captured. In traditional structures, dynamics is typically
expressed through morphisms between such structures, as in linear maps
between vector spaces or morphisms between rings. In algebras
associated with the semantics of computation, the dynamics is
expressed as part of the algebraic structure itself, through a
reduction reduction relation typically denoted by $\red$. Below, we
give a recursive presentation of this relation for the calculus used
in the encoding.

$\red \subseteq \pi \times \pi$
$\red : \pi \to \mathcal{P}(\pi)$

\begin{mathpar}
  \inferrule* [lab=Comm] { \textsf{match}( x_{src}, x_{trgt} ) } { x_{trgt}?(y)P \; | \; x_{src}!\langle {Q} \rangle \red P\{\quotep{Q}/y}\} }
  \and \\
  \inferrule* [lab=Par] {{P} \red {P}'} {{{P} | {Q}} \red {{P}' | {Q}}}
  \and
  \inferrule* [lab=Equiv]{{{P} \scong {P}'} \andalso {{P}' \red {Q}'} \andalso {{Q}' \scong {Q}}}{{P} \red {Q}}
\end{mathpar}

\begin{eqnarray*}
  match_{\equiv} (\quotep{P},\quotep{Q}) & := & P \equiv Q \\
  match_{\dagger}(\quotep{P},\quotep{Q}) & := & \forall R. P|Q \red^{*} R => R \red^{*} 0 \\
  match_{K}(\quotep{P},\quotep{Q}) & := & K \mbox{ for some context } K
\end{eqnarray*}

$u?(x)P | u!\langle Q \rangle \red P\{\quotep{Q}/x\}$

%We write $\wred$ for $\red^*$, and $P\red$ if $\exists Q $ such that $ P \red Q$.
We write $P\red$ if $\exists Q $ such that $ P \red Q$ and $P\not\red$, otherwise.

\section{Replication}

As mentioned before, it is known that replication (and hence
recursion) can be implemented in a higher-order process algebra
\cite{SangiorgiWalker}. As our first example of calculation with the
machinery thus far presented we give the construction explicitly in
the {\rhoc}.

\begin{eqnarray}
	D_{x} & := & \prefix{x}{y}{(\binpar{\outputp{x}{y}}{@{y}})} \nonumber\\
	\bangp_{x}{P} & := & \binpar{{x}!\langle{\binpar{D_{x}}{P}}\rangle}{D_{x}} \nonumber
\end{eqnarray}

\begin{eqnarray}
	\bangp_{x}{P} & & \nonumber\\
	=
	& {x}!\langle{(\prefix{x}{y}{(\outputp{x}{y} | @{y})) | P}}\rangle 
	      | \prefix{x}{y}{(\outputp{x}{y} | @{y})} & \nonumber\\
	\red
	& (\outputp{x}{y} | @{y})\substn{\quotep{(\prefix{x}{y}{(@{y} | \outputp{x}{y})) | P}}}{y} & \nonumber\\
	=
	& \outputp{x}{\quotep{(\prefix{x}{y}{(\outputp{x}{y} | @{y})) | P}}}
	  | {(\prefix{x}{y}{(\outputp{x}{y} | @{y})) | P}} & \nonumber\\
	\red
	& \ldots & \nonumber\\
	\red^*
	& P | P | \ldots & \nonumber
\end{eqnarray}

Of course, this encoding, as an implementation, runs away, unfolding
$\bangp{P}$ eagerly. A lazier and more implementable replication
operator, restricted to input-guarded processes, may be obtained as follows.

\begin{eqnarray}
\bangp{\prefix{u}{v}{P}} 
	:= 
	\binpar{\lift{x}{\prefix{u}{v}{(\binpar{D(x)}{P})}}}{D(x)} \nonumber
\end{eqnarray}

\begin{remark}
  Note that the lazier definition still does not deal with summation
  or mixed summation (i.e. sums over input and output). The reader is
  invited to construct definitions of replication that deal with these
  features. 

  Further, the definitions are parameterized in a name, $x$. Can you,
  gentle reader, make a definition that eliminates this parameter and
  guarantees no accidental interaction between the replication
  machinery and the process being replicated -- i.e. no accidental
  sharing of names used by the process to get its work done and the
  name(s) used by the replication to effect copying. This latter
  revision of the definition of replication is crucial to obtaining
  the expected identity $!!P \sim !P$.
\end{remark}

\begin{remark}\label{rem:paradoxical_combinator}
  The reader familiar with the lambda calculus will have noticed the
  similarity between $D$ and the paradoxical combinator.

  [Ed. note: the existence of this seems to suggest we have to be more
  restrictive on the set of processes and names we admit if we are to
  support no-cloning.]
\end{remark}

\subsubsection{Bisimulation}

The computational dynamics gives rise to another kind of equivalence,
the equivalence of computational behavior. As previously mentioned
this is typically captured \emph{via} some form of bisimulation.

% The notion we use in this paper is weak barbed bisimulation
% \cite{milner91polyadicpi}.

The notion we use in this paper is derived from weak barbed
bisimulation \cite{milner91polyadicpi}. 

\begin{definition}
An \emph{observation relation}, $\downarrow_{\mathcal N}$, over a set
of names, $\mathcal N$, is the smallest relation satisfying the rules
below.

\infrule[Out-barb]{y \in {\mathcal N}, \; x \nameeq y}
		  {\outputp{x}{v} \downarrow_{\mathcal N} x}
\infrule[Par-barb]{\mbox{$P\downarrow_{\mathcal N} x$ or $Q\downarrow_{\mathcal N} x$}}
		  {\binpar{P}{Q} \downarrow_{\mathcal N} x}

We write $P \Downarrow_{\mathcal N} x$ if there is $Q$ such that 
$P \wred Q$ and $Q \downarrow_{\mathcal N} x$.
\end{definition}

\begin{definition}
%\label{def.bbisim}
An  ${\mathcal N}$-\emph{barbed bisimulation} over a set of names, ${\mathcal N}$, is a symmetric binary relation 
${\mathcal S}_{\mathcal N}$ between agents such that $P\rel{S}_{\mathcal N}Q$ implies:
\begin{enumerate}
\item If $P \red P'$ then $Q \wred Q'$ and $P'\rel{S}_{\mathcal N} Q'$.
\item If $P\downarrow_{\mathcal N} x$, then $Q\Downarrow_{\mathcal N} x$.
\end{enumerate}
$P$ is ${\mathcal N}$-barbed bisimilar to $Q$, written
$P \wbbisim_{\mathcal N} Q$, if $P \rel{S}_{\mathcal N} Q$ for some ${\mathcal N}$-barbed bisimulation ${\mathcal S}_{\mathcal N}$.
\end{definition}

$\mathcal{R} \subseteq \pi \times \pi$

$P \mathcal{R} Q => \forall P'. P \red P' \Rightarrow \exists Q'. Q \red Q', P' \mathcal{R} Q'$

$P \vdash x \Rightarrow Q \vdash x$

\begin{mathpar}
  \inferrule*[lab=Out-barb]{x \nameeq y}{{y}!\langle{Q}\rangle \vdash x}
  \and
  \inferrule*[lab=Par-barb]{\mbox{$P\vdash x$ or $Q\vdash x$}}{\binpar{P}{Q} \vdash x}
\end{mathpar}

\subsubsection{Contexts}

One of the principle advantages of computational calculi like the
$\pi$-calculus is a well-defined notion of context,
contextual-equivalence and a correlation between
contextual-equivalence and notions of bisimulation. The notion of
context allows the decomposition of a process into (sub-)process and
its syntactic environment, its context. Thus, a context may be
thought of as a process with a ``hole'' (written $\Box$) in it. The
application of a context $M$ to a process $P$, written $M[P]$, is
tantamount to filling the hole in $M$ with $P$. In this paper we do
not need the full weight of this theory, but do make use of the notion
of context in the proof the main theorem. 

\begin{mathpar}
  \inferrule* [lab=summation] {} {{M_{M},M_{N}} \bc \Box \;|\; x.M_{A} \;|\; M_{M}+M_{N}}
  \and
  \inferrule* [lab=agent] {} {{M_{A}} \bc (\vec{x})M_{P} \;| \; \clift{P_0,\ldots,M_{P},\ldots,P_N}}
  \and \\
  \inferrule* [lab=process] {} {{M_{P}} \bc M_{N} \;| \;P|M_{P} }
\end{mathpar} 

\begin{mathpar}
  \inferrule* [lab=sychronization] {} {M_{N} \bc \Box \;|\; x?M_{F} \;|\; x!M_{C}}
  \and
  \inferrule* [lab=abstraction] {} {{M_{F}} \bc (x)M_{P} }
  \and
  \inferrule* [lab=concretion] {} {{M_{C}} \bc \langle M_{P} \rangle }
  \and \\
  \inferrule* [lab=process] {} {{M_{P}} \bc M_{N} \;| \;P|M_{P} }
\end{mathpar}

\begin{definition}[contextual application] Given a context $M$, and
  process $P$, we define the \emph{contextual application}, $M[P] :=
  M\{P/\Box\}$. That is, the contextual application of M to P is the
  substitution of $P$ for $\Box$ in $M$.
\end{definition}

$\meaningof{-} : L \to \mathcal{P}(\pi)$

\begin{mathpar}
  \inferrule* [lab=collection] {} {\meaningof{true} = \pi, \and \meaningof{~E} = \pi \setminus \meaningof{E}, \and \meaningof{E_{1} \& E_{2}} = \meaningof{E_{1}} \cap \meaningof{E_{2}}}
\end{mathpar}

\begin{mathpar}
  \inferrule* [lab=structure] {} {\meaningof{0} = \{ P \in \pi | P \equiv 0 \}, \and \\ \meaningof{E_1 | E_2} = \{ P \in \pi | P \equiv P_{1} | P_{2}, P_{1} \in \meaningof{E_{1}}, P_{2} \in \meaningof{E_2}\} }
\end{mathpar}

\begin{mathpar}
 \inferrule* [lab=behavior] {} {\meaningof{\langle a?b \rangle E} = \{ P \in \pi | P \equiv Q | u?(y)P', \\ \and \\\\ \and \\ \;\;\; u \in \meaningof{a}, \forall z.P'\{z/y\} \in \meaningof{E\{z/b\}}\}, \and \\ \meaningof{a!E} = \{ P \in \pi | P \equiv Q | x!\langle P' \rangle, x \in \meaningof{a} P' \in \meaningof{E}\} }
\end{mathpar}

\begin{mathpar}
 \inferrule* [lab=nominal] {} {\meaningof{\quotep{E}} = \{ \quotep{P} \in \quotep{\pi} | P \in \meaningof{E} \}, \and \meaningof{\quotep{P}} = \{ \quotep{Q} \in \quotep{\pi} | P \equiv Q \} \and \\ \meaningof{@\quotep{E}} = \{ P \in \pi | P \equiv @x, x \in \meaningof{E} \}}
\end{mathpar}

\begin{eqnarray*}
  \\
  \meaningof{-} : TS \to ST
\end{eqnarray*}

\begin{eqnarray*}
  \\
  L : TS \to ST
\end{eqnarray*}

\begin{eqnarray*}
  \\
  P \models E \iff P \in \meaningof{E}
\end{eqnarray*}

\begin{eqnarray*}
  P \approx_{L} Q \iff \forall E \in L. P \models E \iff Q \models E
\end{eqnarray*}

\begin{eqnarray*}
  P \approx_{K} Q
\end{eqnarray*}

\begin{eqnarray*}
  P \approx Q
\end{eqnarray*}

$\approx_{K} = \approx = \approx_{L}$

\subsubsection{Contextual duality}

Note that contexts extend the quotation operation to a family of
operations from processes to names. Given a context, $M$, we can
define a \emph{nominal context}, $\quotep{M}$ by $\quotep{M}[P] :=
\quotep{M[P]}$. To foreshadow what is to come we observe that these
operations enjoy a duality with processes very much like the duality
between vectors and maps from vectors to scalars.

Further, because the calculus is essentially higher-order, we have a
correspondence between contexts and processes. More specifically,
given a name $x$ and a context $M$ we can construct $M^{*}_{x}$ such
that 

\begin{mathpar}
  M^{*}_{x} | \lift{x}{P} \red M[P]
\end{mathpar}

namely,

\begin{mathpar}
  M^{*}_{x} := x?(u).M[\dropn{u}]
\end{mathpar}

The dependence of $M^{*}_{x}$ on a name makes it an abstraction, 

\begin{mathpar}
  M^{*} := (x)x?(u).M[\dropn{u}]
\end{mathpar}

\subsection{Additional notation}

It will sometimes be convenient to denote the process a name
quotes. We already have the notation $x = \quotep{P}$, but it will be
convenient to introduce an alternate notation, $\procn{x}$, when we
want to emphasize the connection to the use of the name. Note that, by
virtue of name equivalence, $\quotep{\procn{x}} \nameeq x$; so, the
notation is consistent with previous definitions.

Further, because names have structure it is possible to effect
substitutions on the basis of that structure. This means we need to
upgrade our notation for substitutions, which we accomplish by
adapting comprehension notation. Thus,

\begin{mathpar}
  P\{ y / x : x \in S \}
\end{mathpar}

is interpreted to mean the process derived from P by replacing (in a
capture-avoiding manner) each occurrence of $x$ in $S$ by $y$. For example,

\begin{mathpar}
  P\{ \quotep{\procn{x}|\procn{x}} / x : x \in \freenames{P} \}
\end{mathpar}

will replace each (occurrence) of a free name $x$ in $P$ by
$\quotep{\procn{x}|\procn{x}}$.

Also, we will avail ourselves of the notation $x^{L}$ and $x^{R}$ to
denote injections of a name into disjoint copies of the name
space. There are numerous ways to accomplish this. One example can be
found in \cite{MeredithR05}. This notation overloads to vectors of
names: $\vec{x}^{\pi} := (x_{i}^{\pi} \; : \; 0 \leq i < |\vec{x}| )$ where $\pi \in \{L,R\}$.

We also use $P^{\Box} := P|\Box$.

In \cite{MeredithR05} an interpretation of the new operator is
given. It turns out that there are several possible interpretations
all enjoying the requisite algebraic properties of the operator (see
\cite{milner91polyadicpi}). We will therefore make liberal use of
$(\nu\; \vec{x})P$.

% subsection the_syntax_and_semantics_of_the_notation_system (end)   

\input{qm2pi.qmops} 

\input{qm2pi.sterngerlach} 

\input{qm2pi.metric} 

% section concurrent_process_calculi (end)

%\input{qm2pi.proofsketch}

% section proof sketch (end)

%\input{qm2pi.slviaknots} 

% section spatial logic via knots (end)

\input{qm2pi.conclusion}

% section conclusion (end)

%\input{qm2pi.dtcodes} 

% section wiring algorithm (end)

\input{qm2pi.ack} 

% section acknowledgments (end)

\newpage


\bibliographystyle{plain}   
\bibliography{../../biblios/main.bib}

\input{qm2pi.rhodetails}

\end{document}

 

\documentclass[12pt]{llncs}
%\documentclass{jktr}

\usepackage[pdftex]{hyperref}                   
\usepackage {listings}
\usepackage {mathpartir}
\usepackage{bcprules}
%\usepackage{listings}
                       
\usepackage{graphicx} 
%\usepackage[margins=2.5cm,nohead,nofoot]{geometry}
%\usepackage{geometry}
\usepackage{amsfonts}
\usepackage{amstext}
\usepackage{latexsym}
\usepackage{amssymb}
\usepackage{color}


%\include{myPreamble}
\include{qm2pi.local} 

%\ifpdf
%\usepackage[pdftex]{graphicx}
%\else
%\usepackage{graphicx}
%\fi

 % \ifpdf
%  \usepackage{pdfsync}
%  \if


%\title{Brief Article}
%\author{David F. Snyder}
%\author{L.G. Meredith}

%\address{Dept. of Math., Texas State University--San Marcos, San Marcos, TX 78666}
       
\pagestyle{empty}


\begin{document}

\lstset{language=[Objective]Caml,frame=shadowbox}

\input{qm2pi.front}

% section front matter (end)

\input{qm2pi.intro} 
 
% section introduction (end)

% \input{qm2pi.knotations} 

% section notation (end)

\input{qm2pi.process.calculi} 

% section concurrent_process_calculi_and_spatial_logics_ (end)
    
%\input{qm2pi.knots2pi} 

%\input{qm2pi.trefoil} 

%\input{qm2pi.mainthm} 

% subsection basic_interpretation (end)

%\input{qm2pi.rho.presentation} 
\subsection{The syntax and semantics of the notation system}\label{sub:the_syntax_and_semantics_of_the_notation_system} % (fold)

We now summarize a technical presentation of the calculus that
embodies our theory of dynamics. The typical presentation of such a
calculus follows the style of giving generators and relations on
them. The grammar, below, describing term constructors, freely
generates the set of processes, $\Proc$. This set is then quotiented
by a relation known as structural congruence and it is over this set
that the notion of dynamics is expressed. This presentation is
essentially that of \cite{MeredithR05} with the addition of
polyadicity and summation. For readability we have relegated some of
the technical subtleties to an appendix.

\subsubsection{Process grammar}\label{subsub:process_grammar}

\begin{mathpar}
  \inferrule* [lab=synchronization] {} {{M} \bc \pzero \;|\; x?F \;|\; x!C }
  \and
  \inferrule* [lab=abstraction] {} {{F} \bc (x)P}
  \and
  \inferrule* [lab=concretion] {} {{C} \bc \langle Q \rangle}
  \and
  \inferrule* [lab=process] {} {{P,Q} \bc M \;| \;P|Q \;|\; @{x}}
  \and
  \inferrule* [lab=name] {} {{x} \bc \quotep{P}}
\end{mathpar} 

Note that $\vec{x}$ (resp. $\vec{P}$) denotes a vector of names
(resp. processes) of length $|\vec{x}|$ (resp. $|\vec{P}|$). We adopt
the following useful abbreviations.

\begin{mathpar}
   x?(\vec{y}).P := x.(\vec{y})P \and  x\clift{\vec{P}} := x.\clift{\vec{P}}
   \and x!(y) := \lift{x}{\dropn{y}}
   \and \Pi_{i=0}^{n-1}P_i := P_0 | \ldots | P_{n-1}
\end{mathpar}

\subsubsection{Structural congruence}

\paragraph{Free and bound names and alpha-equivalence.} At the
core of structural equivalence is alpha-equivalence which identifies
process that are the same up to a change of variable. Formally, we
recognize the distinction between free and bound names. The free names
of a process, $\freenames{P}$, may be calculated recursively as
follows:

\begin{mathpar}
\freenames{\pzero} := \emptyset
  \and \\
  \freenames{x?(y).P} := \{ x \} \cup (\freenames{P} \setminus \{ y \})
  \and 
  \freenames{x!\langle P \rangle} := \{ x \} \cup \{ P \} 
  \and \\
  \freenames{P|Q} := \freenames{P} \cup \freenames{Q}
  \and \\
  \freenames{@{x}} := \{ x \}
\end{mathpar}

$\pi$
$\quotep{\pi}$

$\freenames{-} : \pi \to \mathcal{P}(\quotep{\pi})$

\begin{eqnarray*}
  \freenames{\pzero} & := & \emptyset \\
  \freenames{x?(y).P} & := & \{ x \} \cup (\freenames{P} \setminus \{ y \}) \\
  \freenames{x!\langle P \rangle} & := & \{ x \} \cup \{ P \} \\
  \freenames{P|Q} & := & \freenames{P} \cup \freenames{Q} \\
  \freenames{\dropn{x}} & := & \{ x \}
\end{eqnarray*}

The bound names of a process, $\boundnames{P}$, are those names occurring in $P$
that are not free. For example, in $x?(y).0$, the name $x$ is free, while $y$ is bound.

\begin{mathpar}
  \inferrule* [lab=monoidal-laws] {} { P|Q \equiv Q|P \and P|0 \equiv P \and P|(Q|R) \equiv (P|Q)|R }
\end{mathpar}

\begin{mathpar}
  \inferrule* [lab=alpha-equivalence] {} { (x)P \equiv (y)P\{y/x\} \and y \not\in \freenames{P} }
\end{mathpar}

\begin{definition}
Then two processes, $P,Q$, are alpha-equivalent if $P = Q\{\vec{y}/\vec{x}\}$ for
some $\vec{x} \in \boundnames{Q},\vec{y} \in \boundnames{P}$, where $Q\{\vec{y}/\vec{x}\}$
denotes the capture-avoiding substitution of $\vec{y}$ for $\vec{x}$ in $Q$.
\end{definition}

\begin{definition}
  The {\em structural congruence} \cite{SangiorgiWalker} , $\equiv$,
  between processes is the least congruence containing
  alpha-equivalence, satisfying the abelian monoid laws
  (associativity, commutativity and $\pzero$ as identity) for parallel
  composition $|$ and for summation $+$.
\end{definition}

\subsection{Name equivalence}

We take name equivalence, written $\nameeq$, to be the smallest
equivalence relation generated by the following rules.

\begin{mathpar}
\inferrule*[lab=Quote-drop]
{ }
{ \quotep{@{x}} \nameeq x }

\inferrule*[lab=Struct-equiv]
{ P \scong Q }
{ \quotep{P} \nameeq \quotep{Q} }
\end{mathpar}

The astute reader will have noticed that the mutual recursion of names
and processes imposes a mutual recursion on alpha-equivalence and
structural equivalence via name-equivalence. Fortunately, all of this
works out pleasantly and we may calculate in the natural way, free of
concern. The reader interested in the details is referred to the
appendix \ref{appendix:rho_details}.

\subsection{Substitution}

We use $\Proc$ for the set of processes, $\QProc$ for the set of
names, and $\id{\{}\vec{y} / \vec{x} \id{\}}$ to denote partial maps,
$s : \QProc \rightarrow \QProc$. A map, $s$ lifts, uniquely, to a map
on process terms, $\widehat{s} : \Proc \rightarrow \Proc$ by the
following equations.

\begin{mathpar}
  (0) \psubstp{Q}{P} := 0 \\
  (R \juxtap S) \psubstp{Q}{P}
  :=    
  (R)\psubstp{Q}{P} \juxtap (S) \psubstp{Q}{P} \\
  (x?(y).R) \psubstp{Q}{P}    
  :=    
  (x)\substp{Q}{P} (z)\concat( (R \psubstn{z}{y}) \psubstp{Q}{P} ) \\
  (\lift{x}{R}) \psubstp{Q}{P}  
  :=
  \lift{(x)\substp{Q}{P}}{ R \psubstp{Q}{P} } \\
%   (\dropn{x})  \psubstp{Q}{P}       
%   := 
%   \left\{ 
%     \begin{array}{ccc} 
%       \dropn{\quotep{Q}} & & x \nameeq \quotep{P} \\
%       \dropn{x} & & otherwise \\
%     \end{array}
%   \right. 
  (\dropn{x})  \psubstp{Q}{P}       
  := 
  \left\{ 
    \begin{array}{ccc} 
      Q & & x \nameeq \quotep{P} \\
      \dropn{x} & & otherwise \\
    \end{array}
  \right.
\end{mathpar}
 

where

\begin{eqnarray}
  (x)\id{\{} \lpquote Q \rpquote / \lpquote P \rpquote \id{\}}            = 
  \left\{ 
    \begin{array}{ccc}
      \lpquote Q \rpquote & & x \nameeq \lpquote P \rpquote \\
      x & & otherwise \\
    \end{array}
  \right. \nonumber
\end{eqnarray}

and $z$ is chosen distinct from $\quotep{P}$, $\quotep{Q}$, the free
names in $Q$, and all the names in $R$. Our $\alpha$-equivalence will
be built in the standard way from this substitution.

\begin{remark}\label{rem:no_self_referential_names}
  One consequence of these definitions is that $\forall P. \quotep{P}
  \not\in \freenames{P}$.
\end{remark}

\subsection{ Dynamic quote: an example }

Anticipating something of what's to come, consider applying the
substitution, $\widehat{\id{\{}u / z \id{\}}}$, to the following pair
of processes, $\lift{w}{y!(z)}$ and $w[ \lpquote y!(z) \rpquote ]$.

\begin{eqnarray}
	\lift{w}{y!(z)}\widehat{\id{\{}u / z \id{\}}}
		& = &
		\lift{w}{y!(u)} \nonumber\\
	w[ \lpquote y!(z) \rpquote ] \widehat{ \id{\{}u / z \id{\}} }
		& = &
		w[ \lpquote y!(z) \rpquote ] \nonumber
\end{eqnarray}

Because the body of the process between quotes is impervious to
substitution, we get radically different answers. In fact, by
examining the first process in an input context,
e.g. $x?(z).\lift{w}{y!(z)}$, we see that the process under the lift
operator may be shaped by prefixed inputs binding a name inside it. In
this sense, the lift operator will be seen as a way to dynamically
construct processes before reifying them as names.

Finally equipped with these standard features we can present the
dynamics of the calculus.

\subsubsection{Operational semantics} 

Finally, we introduce the computational dynamics. What marks these
algebras as distinct from other more traditionally studied algebraic
structures, e.g. vector spaces or polynomial rings, is the manner in
which dynamics is captured. In traditional structures, dynamics is typically
expressed through morphisms between such structures, as in linear maps
between vector spaces or morphisms between rings. In algebras
associated with the semantics of computation, the dynamics is
expressed as part of the algebraic structure itself, through a
reduction reduction relation typically denoted by $\red$. Below, we
give a recursive presentation of this relation for the calculus used
in the encoding.

$\red \subseteq \pi \times \pi$
$\red : \pi \to \mathcal{P}(\pi)$

\begin{mathpar}
  \inferrule* [lab=Comm] { \textsf{match}( x_{src}, x_{trgt} ) } { x_{trgt}?(y)P \; | \; x_{src}!\langle {Q} \rangle \red P\{\quotep{Q}/y}\} }
  \and \\
  \inferrule* [lab=Par] {{P} \red {P}'} {{{P} | {Q}} \red {{P}' | {Q}}}
  \and
  \inferrule* [lab=Equiv]{{{P} \scong {P}'} \andalso {{P}' \red {Q}'} \andalso {{Q}' \scong {Q}}}{{P} \red {Q}}
\end{mathpar}

\begin{eqnarray*}
  match_{\equiv} (\quotep{P},\quotep{Q}) & := & P \equiv Q \\
  match_{\dagger}(\quotep{P},\quotep{Q}) & := & \forall R. P|Q \red^{*} R => R \red^{*} 0 \\
  match_{K}(\quotep{P},\quotep{Q}) & := & K \mbox{ for some context } K
\end{eqnarray*}

$u?(x)P | u!\langle Q \rangle \red P\{\quotep{Q}/x\}$

%We write $\wred$ for $\red^*$, and $P\red$ if $\exists Q $ such that $ P \red Q$.
We write $P\red$ if $\exists Q $ such that $ P \red Q$ and $P\not\red$, otherwise.

\section{Replication}

As mentioned before, it is known that replication (and hence
recursion) can be implemented in a higher-order process algebra
\cite{SangiorgiWalker}. As our first example of calculation with the
machinery thus far presented we give the construction explicitly in
the {\rhoc}.

\begin{eqnarray}
	D_{x} & := & \prefix{x}{y}{(\binpar{\outputp{x}{y}}{@{y}})} \nonumber\\
	\bangp_{x}{P} & := & \binpar{{x}!\langle{\binpar{D_{x}}{P}}\rangle}{D_{x}} \nonumber
\end{eqnarray}

\begin{eqnarray}
	\bangp_{x}{P} & & \nonumber\\
	=
	& {x}!\langle{(\prefix{x}{y}{(\outputp{x}{y} | @{y})) | P}}\rangle 
	      | \prefix{x}{y}{(\outputp{x}{y} | @{y})} & \nonumber\\
	\red
	& (\outputp{x}{y} | @{y})\substn{\quotep{(\prefix{x}{y}{(@{y} | \outputp{x}{y})) | P}}}{y} & \nonumber\\
	=
	& \outputp{x}{\quotep{(\prefix{x}{y}{(\outputp{x}{y} | @{y})) | P}}}
	  | {(\prefix{x}{y}{(\outputp{x}{y} | @{y})) | P}} & \nonumber\\
	\red
	& \ldots & \nonumber\\
	\red^*
	& P | P | \ldots & \nonumber
\end{eqnarray}

Of course, this encoding, as an implementation, runs away, unfolding
$\bangp{P}$ eagerly. A lazier and more implementable replication
operator, restricted to input-guarded processes, may be obtained as follows.

\begin{eqnarray}
\bangp{\prefix{u}{v}{P}} 
	:= 
	\binpar{\lift{x}{\prefix{u}{v}{(\binpar{D(x)}{P})}}}{D(x)} \nonumber
\end{eqnarray}

\begin{remark}
  Note that the lazier definition still does not deal with summation
  or mixed summation (i.e. sums over input and output). The reader is
  invited to construct definitions of replication that deal with these
  features. 

  Further, the definitions are parameterized in a name, $x$. Can you,
  gentle reader, make a definition that eliminates this parameter and
  guarantees no accidental interaction between the replication
  machinery and the process being replicated -- i.e. no accidental
  sharing of names used by the process to get its work done and the
  name(s) used by the replication to effect copying. This latter
  revision of the definition of replication is crucial to obtaining
  the expected identity $!!P \sim !P$.
\end{remark}

\begin{remark}\label{rem:paradoxical_combinator}
  The reader familiar with the lambda calculus will have noticed the
  similarity between $D$ and the paradoxical combinator.

  [Ed. note: the existence of this seems to suggest we have to be more
  restrictive on the set of processes and names we admit if we are to
  support no-cloning.]
\end{remark}

\subsubsection{Bisimulation}

The computational dynamics gives rise to another kind of equivalence,
the equivalence of computational behavior. As previously mentioned
this is typically captured \emph{via} some form of bisimulation.

% The notion we use in this paper is weak barbed bisimulation
% \cite{milner91polyadicpi}.

The notion we use in this paper is derived from weak barbed
bisimulation \cite{milner91polyadicpi}. 

\begin{definition}
An \emph{observation relation}, $\downarrow_{\mathcal N}$, over a set
of names, $\mathcal N$, is the smallest relation satisfying the rules
below.

\infrule[Out-barb]{y \in {\mathcal N}, \; x \nameeq y}
		  {\outputp{x}{v} \downarrow_{\mathcal N} x}
\infrule[Par-barb]{\mbox{$P\downarrow_{\mathcal N} x$ or $Q\downarrow_{\mathcal N} x$}}
		  {\binpar{P}{Q} \downarrow_{\mathcal N} x}

We write $P \Downarrow_{\mathcal N} x$ if there is $Q$ such that 
$P \wred Q$ and $Q \downarrow_{\mathcal N} x$.
\end{definition}

\begin{definition}
%\label{def.bbisim}
An  ${\mathcal N}$-\emph{barbed bisimulation} over a set of names, ${\mathcal N}$, is a symmetric binary relation 
${\mathcal S}_{\mathcal N}$ between agents such that $P\rel{S}_{\mathcal N}Q$ implies:
\begin{enumerate}
\item If $P \red P'$ then $Q \wred Q'$ and $P'\rel{S}_{\mathcal N} Q'$.
\item If $P\downarrow_{\mathcal N} x$, then $Q\Downarrow_{\mathcal N} x$.
\end{enumerate}
$P$ is ${\mathcal N}$-barbed bisimilar to $Q$, written
$P \wbbisim_{\mathcal N} Q$, if $P \rel{S}_{\mathcal N} Q$ for some ${\mathcal N}$-barbed bisimulation ${\mathcal S}_{\mathcal N}$.
\end{definition}

$\mathcal{R} \subseteq \pi \times \pi$

$P \mathcal{R} Q => \forall P'. P \red P' \Rightarrow \exists Q'. Q \red Q', P' \mathcal{R} Q'$

$P \vdash x \Rightarrow Q \vdash x$

\begin{mathpar}
  \inferrule*[lab=Out-barb]{x \nameeq y}{{y}!\langle{Q}\rangle \vdash x}
  \and
  \inferrule*[lab=Par-barb]{\mbox{$P\vdash x$ or $Q\vdash x$}}{\binpar{P}{Q} \vdash x}
\end{mathpar}

\subsubsection{Contexts}

One of the principle advantages of computational calculi like the
$\pi$-calculus is a well-defined notion of context,
contextual-equivalence and a correlation between
contextual-equivalence and notions of bisimulation. The notion of
context allows the decomposition of a process into (sub-)process and
its syntactic environment, its context. Thus, a context may be
thought of as a process with a ``hole'' (written $\Box$) in it. The
application of a context $M$ to a process $P$, written $M[P]$, is
tantamount to filling the hole in $M$ with $P$. In this paper we do
not need the full weight of this theory, but do make use of the notion
of context in the proof the main theorem. 

\begin{mathpar}
  \inferrule* [lab=summation] {} {{M_{M},M_{N}} \bc \Box \;|\; x.M_{A} \;|\; M_{M}+M_{N}}
  \and
  \inferrule* [lab=agent] {} {{M_{A}} \bc (\vec{x})M_{P} \;| \; \clift{P_0,\ldots,M_{P},\ldots,P_N}}
  \and \\
  \inferrule* [lab=process] {} {{M_{P}} \bc M_{N} \;| \;P|M_{P} }
\end{mathpar} 

\begin{mathpar}
  \inferrule* [lab=sychronization] {} {M_{N} \bc \Box \;|\; x?M_{F} \;|\; x!M_{C}}
  \and
  \inferrule* [lab=abstraction] {} {{M_{F}} \bc (x)M_{P} }
  \and
  \inferrule* [lab=concretion] {} {{M_{C}} \bc \langle M_{P} \rangle }
  \and \\
  \inferrule* [lab=process] {} {{M_{P}} \bc M_{N} \;| \;P|M_{P} }
\end{mathpar}

\begin{definition}[contextual application] Given a context $M$, and
  process $P$, we define the \emph{contextual application}, $M[P] :=
  M\{P/\Box\}$. That is, the contextual application of M to P is the
  substitution of $P$ for $\Box$ in $M$.
\end{definition}

$\meaningof{-} : L \to \mathcal{P}(\pi)$

\begin{mathpar}
  \inferrule* [lab=collection] {} {\meaningof{true} = \pi, \and \meaningof{~E} = \pi \setminus \meaningof{E}, \and \meaningof{E_{1} \& E_{2}} = \meaningof{E_{1}} \cap \meaningof{E_{2}}}
\end{mathpar}

\begin{mathpar}
  \inferrule* [lab=structure] {} {\meaningof{0} = \{ P \in \pi | P \equiv 0 \}, \and \\ \meaningof{E_1 | E_2} = \{ P \in \pi | P \equiv P_{1} | P_{2}, P_{1} \in \meaningof{E_{1}}, P_{2} \in \meaningof{E_2}\} }
\end{mathpar}

\begin{mathpar}
 \inferrule* [lab=behavior] {} {\meaningof{\langle a?b \rangle E} = \{ P \in \pi | P \equiv Q | u?(y)P', \\ \and \\\\ \and \\ \;\;\; u \in \meaningof{a}, \forall z.P'\{z/y\} \in \meaningof{E\{z/b\}}\}, \and \\ \meaningof{a!E} = \{ P \in \pi | P \equiv Q | x!\langle P' \rangle, x \in \meaningof{a} P' \in \meaningof{E}\} }
\end{mathpar}

\begin{mathpar}
 \inferrule* [lab=nominal] {} {\meaningof{\quotep{E}} = \{ \quotep{P} \in \quotep{\pi} | P \in \meaningof{E} \}, \and \meaningof{\quotep{P}} = \{ \quotep{Q} \in \quotep{\pi} | P \equiv Q \} \and \\ \meaningof{@\quotep{E}} = \{ P \in \pi | P \equiv @x, x \in \meaningof{E} \}}
\end{mathpar}

\begin{eqnarray*}
  \\
  \meaningof{-} : TS \to ST
\end{eqnarray*}

\begin{eqnarray*}
  \\
  L : TS \to ST
\end{eqnarray*}

\begin{eqnarray*}
  \\
  P \models E \iff P \in \meaningof{E}
\end{eqnarray*}

\begin{eqnarray*}
  P \approx_{L} Q \iff \forall E \in L. P \models E \iff Q \models E
\end{eqnarray*}

\begin{eqnarray*}
  P \approx_{K} Q
\end{eqnarray*}

\begin{eqnarray*}
  P \approx Q
\end{eqnarray*}

$\approx_{K} = \approx = \approx_{L}$

\subsubsection{Contextual duality}

Note that contexts extend the quotation operation to a family of
operations from processes to names. Given a context, $M$, we can
define a \emph{nominal context}, $\quotep{M}$ by $\quotep{M}[P] :=
\quotep{M[P]}$. To foreshadow what is to come we observe that these
operations enjoy a duality with processes very much like the duality
between vectors and maps from vectors to scalars.

Further, because the calculus is essentially higher-order, we have a
correspondence between contexts and processes. More specifically,
given a name $x$ and a context $M$ we can construct $M^{*}_{x}$ such
that 

\begin{mathpar}
  M^{*}_{x} | \lift{x}{P} \red M[P]
\end{mathpar}

namely,

\begin{mathpar}
  M^{*}_{x} := x?(u).M[\dropn{u}]
\end{mathpar}

The dependence of $M^{*}_{x}$ on a name makes it an abstraction, 

\begin{mathpar}
  M^{*} := (x)x?(u).M[\dropn{u}]
\end{mathpar}

\subsection{Additional notation}

It will sometimes be convenient to denote the process a name
quotes. We already have the notation $x = \quotep{P}$, but it will be
convenient to introduce an alternate notation, $\procn{x}$, when we
want to emphasize the connection to the use of the name. Note that, by
virtue of name equivalence, $\quotep{\procn{x}} \nameeq x$; so, the
notation is consistent with previous definitions.

Further, because names have structure it is possible to effect
substitutions on the basis of that structure. This means we need to
upgrade our notation for substitutions, which we accomplish by
adapting comprehension notation. Thus,

\begin{mathpar}
  P\{ y / x : x \in S \}
\end{mathpar}

is interpreted to mean the process derived from P by replacing (in a
capture-avoiding manner) each occurrence of $x$ in $S$ by $y$. For example,

\begin{mathpar}
  P\{ \quotep{\procn{x}|\procn{x}} / x : x \in \freenames{P} \}
\end{mathpar}

will replace each (occurrence) of a free name $x$ in $P$ by
$\quotep{\procn{x}|\procn{x}}$.

Also, we will avail ourselves of the notation $x^{L}$ and $x^{R}$ to
denote injections of a name into disjoint copies of the name
space. There are numerous ways to accomplish this. One example can be
found in \cite{MeredithR05}. This notation overloads to vectors of
names: $\vec{x}^{\pi} := (x_{i}^{\pi} \; : \; 0 \leq i < |\vec{x}| )$ where $\pi \in \{L,R\}$.

We also use $P^{\Box} := P|\Box$.

In \cite{MeredithR05} an interpretation of the new operator is
given. It turns out that there are several possible interpretations
all enjoying the requisite algebraic properties of the operator (see
\cite{milner91polyadicpi}). We will therefore make liberal use of
$(\nu\; \vec{x})P$.

% subsection the_syntax_and_semantics_of_the_notation_system (end)   

\input{qm2pi.qmops} 

\input{qm2pi.sterngerlach} 

\input{qm2pi.metric} 

% section concurrent_process_calculi (end)

%\input{qm2pi.proofsketch}

% section proof sketch (end)

%\input{qm2pi.slviaknots} 

% section spatial logic via knots (end)

\input{qm2pi.conclusion}

% section conclusion (end)

%\input{qm2pi.dtcodes} 

% section wiring algorithm (end)

\input{qm2pi.ack} 

% section acknowledgments (end)

\newpage


\bibliographystyle{plain}   
\bibliography{../../biblios/main.bib}

\input{qm2pi.rhodetails}

\end{document}

 

% section concurrent_process_calculi (end)

%\documentclass[12pt]{llncs}
%\documentclass{jktr}

\usepackage[pdftex]{hyperref}                   
\usepackage {listings}
\usepackage {mathpartir}
\usepackage{bcprules}
%\usepackage{listings}
                       
\usepackage{graphicx} 
%\usepackage[margins=2.5cm,nohead,nofoot]{geometry}
%\usepackage{geometry}
\usepackage{amsfonts}
\usepackage{amstext}
\usepackage{latexsym}
\usepackage{amssymb}
\usepackage{color}


%\include{myPreamble}
\include{qm2pi.local} 

%\ifpdf
%\usepackage[pdftex]{graphicx}
%\else
%\usepackage{graphicx}
%\fi

 % \ifpdf
%  \usepackage{pdfsync}
%  \if


%\title{Brief Article}
%\author{David F. Snyder}
%\author{L.G. Meredith}

%\address{Dept. of Math., Texas State University--San Marcos, San Marcos, TX 78666}
       
\pagestyle{empty}


\begin{document}

\lstset{language=[Objective]Caml,frame=shadowbox}

\input{qm2pi.front}

% section front matter (end)

\input{qm2pi.intro} 
 
% section introduction (end)

% \input{qm2pi.knotations} 

% section notation (end)

\input{qm2pi.process.calculi} 

% section concurrent_process_calculi_and_spatial_logics_ (end)
    
%\input{qm2pi.knots2pi} 

%\input{qm2pi.trefoil} 

%\input{qm2pi.mainthm} 

% subsection basic_interpretation (end)

%\input{qm2pi.rho.presentation} 
\subsection{The syntax and semantics of the notation system}\label{sub:the_syntax_and_semantics_of_the_notation_system} % (fold)

We now summarize a technical presentation of the calculus that
embodies our theory of dynamics. The typical presentation of such a
calculus follows the style of giving generators and relations on
them. The grammar, below, describing term constructors, freely
generates the set of processes, $\Proc$. This set is then quotiented
by a relation known as structural congruence and it is over this set
that the notion of dynamics is expressed. This presentation is
essentially that of \cite{MeredithR05} with the addition of
polyadicity and summation. For readability we have relegated some of
the technical subtleties to an appendix.

\subsubsection{Process grammar}\label{subsub:process_grammar}

\begin{mathpar}
  \inferrule* [lab=synchronization] {} {{M} \bc \pzero \;|\; x?F \;|\; x!C }
  \and
  \inferrule* [lab=abstraction] {} {{F} \bc (x)P}
  \and
  \inferrule* [lab=concretion] {} {{C} \bc \langle Q \rangle}
  \and
  \inferrule* [lab=process] {} {{P,Q} \bc M \;| \;P|Q \;|\; @{x}}
  \and
  \inferrule* [lab=name] {} {{x} \bc \quotep{P}}
\end{mathpar} 

Note that $\vec{x}$ (resp. $\vec{P}$) denotes a vector of names
(resp. processes) of length $|\vec{x}|$ (resp. $|\vec{P}|$). We adopt
the following useful abbreviations.

\begin{mathpar}
   x?(\vec{y}).P := x.(\vec{y})P \and  x\clift{\vec{P}} := x.\clift{\vec{P}}
   \and x!(y) := \lift{x}{\dropn{y}}
   \and \Pi_{i=0}^{n-1}P_i := P_0 | \ldots | P_{n-1}
\end{mathpar}

\subsubsection{Structural congruence}

\paragraph{Free and bound names and alpha-equivalence.} At the
core of structural equivalence is alpha-equivalence which identifies
process that are the same up to a change of variable. Formally, we
recognize the distinction between free and bound names. The free names
of a process, $\freenames{P}$, may be calculated recursively as
follows:

\begin{mathpar}
\freenames{\pzero} := \emptyset
  \and \\
  \freenames{x?(y).P} := \{ x \} \cup (\freenames{P} \setminus \{ y \})
  \and 
  \freenames{x!\langle P \rangle} := \{ x \} \cup \{ P \} 
  \and \\
  \freenames{P|Q} := \freenames{P} \cup \freenames{Q}
  \and \\
  \freenames{@{x}} := \{ x \}
\end{mathpar}

$\pi$
$\quotep{\pi}$

$\freenames{-} : \pi \to \mathcal{P}(\quotep{\pi})$

\begin{eqnarray*}
  \freenames{\pzero} & := & \emptyset \\
  \freenames{x?(y).P} & := & \{ x \} \cup (\freenames{P} \setminus \{ y \}) \\
  \freenames{x!\langle P \rangle} & := & \{ x \} \cup \{ P \} \\
  \freenames{P|Q} & := & \freenames{P} \cup \freenames{Q} \\
  \freenames{\dropn{x}} & := & \{ x \}
\end{eqnarray*}

The bound names of a process, $\boundnames{P}$, are those names occurring in $P$
that are not free. For example, in $x?(y).0$, the name $x$ is free, while $y$ is bound.

\begin{mathpar}
  \inferrule* [lab=monoidal-laws] {} { P|Q \equiv Q|P \and P|0 \equiv P \and P|(Q|R) \equiv (P|Q)|R }
\end{mathpar}

\begin{mathpar}
  \inferrule* [lab=alpha-equivalence] {} { (x)P \equiv (y)P\{y/x\} \and y \not\in \freenames{P} }
\end{mathpar}

\begin{definition}
Then two processes, $P,Q$, are alpha-equivalent if $P = Q\{\vec{y}/\vec{x}\}$ for
some $\vec{x} \in \boundnames{Q},\vec{y} \in \boundnames{P}$, where $Q\{\vec{y}/\vec{x}\}$
denotes the capture-avoiding substitution of $\vec{y}$ for $\vec{x}$ in $Q$.
\end{definition}

\begin{definition}
  The {\em structural congruence} \cite{SangiorgiWalker} , $\equiv$,
  between processes is the least congruence containing
  alpha-equivalence, satisfying the abelian monoid laws
  (associativity, commutativity and $\pzero$ as identity) for parallel
  composition $|$ and for summation $+$.
\end{definition}

\subsection{Name equivalence}

We take name equivalence, written $\nameeq$, to be the smallest
equivalence relation generated by the following rules.

\begin{mathpar}
\inferrule*[lab=Quote-drop]
{ }
{ \quotep{@{x}} \nameeq x }

\inferrule*[lab=Struct-equiv]
{ P \scong Q }
{ \quotep{P} \nameeq \quotep{Q} }
\end{mathpar}

The astute reader will have noticed that the mutual recursion of names
and processes imposes a mutual recursion on alpha-equivalence and
structural equivalence via name-equivalence. Fortunately, all of this
works out pleasantly and we may calculate in the natural way, free of
concern. The reader interested in the details is referred to the
appendix \ref{appendix:rho_details}.

\subsection{Substitution}

We use $\Proc$ for the set of processes, $\QProc$ for the set of
names, and $\id{\{}\vec{y} / \vec{x} \id{\}}$ to denote partial maps,
$s : \QProc \rightarrow \QProc$. A map, $s$ lifts, uniquely, to a map
on process terms, $\widehat{s} : \Proc \rightarrow \Proc$ by the
following equations.

\begin{mathpar}
  (0) \psubstp{Q}{P} := 0 \\
  (R \juxtap S) \psubstp{Q}{P}
  :=    
  (R)\psubstp{Q}{P} \juxtap (S) \psubstp{Q}{P} \\
  (x?(y).R) \psubstp{Q}{P}    
  :=    
  (x)\substp{Q}{P} (z)\concat( (R \psubstn{z}{y}) \psubstp{Q}{P} ) \\
  (\lift{x}{R}) \psubstp{Q}{P}  
  :=
  \lift{(x)\substp{Q}{P}}{ R \psubstp{Q}{P} } \\
%   (\dropn{x})  \psubstp{Q}{P}       
%   := 
%   \left\{ 
%     \begin{array}{ccc} 
%       \dropn{\quotep{Q}} & & x \nameeq \quotep{P} \\
%       \dropn{x} & & otherwise \\
%     \end{array}
%   \right. 
  (\dropn{x})  \psubstp{Q}{P}       
  := 
  \left\{ 
    \begin{array}{ccc} 
      Q & & x \nameeq \quotep{P} \\
      \dropn{x} & & otherwise \\
    \end{array}
  \right.
\end{mathpar}
 

where

\begin{eqnarray}
  (x)\id{\{} \lpquote Q \rpquote / \lpquote P \rpquote \id{\}}            = 
  \left\{ 
    \begin{array}{ccc}
      \lpquote Q \rpquote & & x \nameeq \lpquote P \rpquote \\
      x & & otherwise \\
    \end{array}
  \right. \nonumber
\end{eqnarray}

and $z$ is chosen distinct from $\quotep{P}$, $\quotep{Q}$, the free
names in $Q$, and all the names in $R$. Our $\alpha$-equivalence will
be built in the standard way from this substitution.

\begin{remark}\label{rem:no_self_referential_names}
  One consequence of these definitions is that $\forall P. \quotep{P}
  \not\in \freenames{P}$.
\end{remark}

\subsection{ Dynamic quote: an example }

Anticipating something of what's to come, consider applying the
substitution, $\widehat{\id{\{}u / z \id{\}}}$, to the following pair
of processes, $\lift{w}{y!(z)}$ and $w[ \lpquote y!(z) \rpquote ]$.

\begin{eqnarray}
	\lift{w}{y!(z)}\widehat{\id{\{}u / z \id{\}}}
		& = &
		\lift{w}{y!(u)} \nonumber\\
	w[ \lpquote y!(z) \rpquote ] \widehat{ \id{\{}u / z \id{\}} }
		& = &
		w[ \lpquote y!(z) \rpquote ] \nonumber
\end{eqnarray}

Because the body of the process between quotes is impervious to
substitution, we get radically different answers. In fact, by
examining the first process in an input context,
e.g. $x?(z).\lift{w}{y!(z)}$, we see that the process under the lift
operator may be shaped by prefixed inputs binding a name inside it. In
this sense, the lift operator will be seen as a way to dynamically
construct processes before reifying them as names.

Finally equipped with these standard features we can present the
dynamics of the calculus.

\subsubsection{Operational semantics} 

Finally, we introduce the computational dynamics. What marks these
algebras as distinct from other more traditionally studied algebraic
structures, e.g. vector spaces or polynomial rings, is the manner in
which dynamics is captured. In traditional structures, dynamics is typically
expressed through morphisms between such structures, as in linear maps
between vector spaces or morphisms between rings. In algebras
associated with the semantics of computation, the dynamics is
expressed as part of the algebraic structure itself, through a
reduction reduction relation typically denoted by $\red$. Below, we
give a recursive presentation of this relation for the calculus used
in the encoding.

$\red \subseteq \pi \times \pi$
$\red : \pi \to \mathcal{P}(\pi)$

\begin{mathpar}
  \inferrule* [lab=Comm] { \textsf{match}( x_{src}, x_{trgt} ) } { x_{trgt}?(y)P \; | \; x_{src}!\langle {Q} \rangle \red P\{\quotep{Q}/y}\} }
  \and \\
  \inferrule* [lab=Par] {{P} \red {P}'} {{{P} | {Q}} \red {{P}' | {Q}}}
  \and
  \inferrule* [lab=Equiv]{{{P} \scong {P}'} \andalso {{P}' \red {Q}'} \andalso {{Q}' \scong {Q}}}{{P} \red {Q}}
\end{mathpar}

\begin{eqnarray*}
  match_{\equiv} (\quotep{P},\quotep{Q}) & := & P \equiv Q \\
  match_{\dagger}(\quotep{P},\quotep{Q}) & := & \forall R. P|Q \red^{*} R => R \red^{*} 0 \\
  match_{K}(\quotep{P},\quotep{Q}) & := & K \mbox{ for some context } K
\end{eqnarray*}

$u?(x)P | u!\langle Q \rangle \red P\{\quotep{Q}/x\}$

%We write $\wred$ for $\red^*$, and $P\red$ if $\exists Q $ such that $ P \red Q$.
We write $P\red$ if $\exists Q $ such that $ P \red Q$ and $P\not\red$, otherwise.

\section{Replication}

As mentioned before, it is known that replication (and hence
recursion) can be implemented in a higher-order process algebra
\cite{SangiorgiWalker}. As our first example of calculation with the
machinery thus far presented we give the construction explicitly in
the {\rhoc}.

\begin{eqnarray}
	D_{x} & := & \prefix{x}{y}{(\binpar{\outputp{x}{y}}{@{y}})} \nonumber\\
	\bangp_{x}{P} & := & \binpar{{x}!\langle{\binpar{D_{x}}{P}}\rangle}{D_{x}} \nonumber
\end{eqnarray}

\begin{eqnarray}
	\bangp_{x}{P} & & \nonumber\\
	=
	& {x}!\langle{(\prefix{x}{y}{(\outputp{x}{y} | @{y})) | P}}\rangle 
	      | \prefix{x}{y}{(\outputp{x}{y} | @{y})} & \nonumber\\
	\red
	& (\outputp{x}{y} | @{y})\substn{\quotep{(\prefix{x}{y}{(@{y} | \outputp{x}{y})) | P}}}{y} & \nonumber\\
	=
	& \outputp{x}{\quotep{(\prefix{x}{y}{(\outputp{x}{y} | @{y})) | P}}}
	  | {(\prefix{x}{y}{(\outputp{x}{y} | @{y})) | P}} & \nonumber\\
	\red
	& \ldots & \nonumber\\
	\red^*
	& P | P | \ldots & \nonumber
\end{eqnarray}

Of course, this encoding, as an implementation, runs away, unfolding
$\bangp{P}$ eagerly. A lazier and more implementable replication
operator, restricted to input-guarded processes, may be obtained as follows.

\begin{eqnarray}
\bangp{\prefix{u}{v}{P}} 
	:= 
	\binpar{\lift{x}{\prefix{u}{v}{(\binpar{D(x)}{P})}}}{D(x)} \nonumber
\end{eqnarray}

\begin{remark}
  Note that the lazier definition still does not deal with summation
  or mixed summation (i.e. sums over input and output). The reader is
  invited to construct definitions of replication that deal with these
  features. 

  Further, the definitions are parameterized in a name, $x$. Can you,
  gentle reader, make a definition that eliminates this parameter and
  guarantees no accidental interaction between the replication
  machinery and the process being replicated -- i.e. no accidental
  sharing of names used by the process to get its work done and the
  name(s) used by the replication to effect copying. This latter
  revision of the definition of replication is crucial to obtaining
  the expected identity $!!P \sim !P$.
\end{remark}

\begin{remark}\label{rem:paradoxical_combinator}
  The reader familiar with the lambda calculus will have noticed the
  similarity between $D$ and the paradoxical combinator.

  [Ed. note: the existence of this seems to suggest we have to be more
  restrictive on the set of processes and names we admit if we are to
  support no-cloning.]
\end{remark}

\subsubsection{Bisimulation}

The computational dynamics gives rise to another kind of equivalence,
the equivalence of computational behavior. As previously mentioned
this is typically captured \emph{via} some form of bisimulation.

% The notion we use in this paper is weak barbed bisimulation
% \cite{milner91polyadicpi}.

The notion we use in this paper is derived from weak barbed
bisimulation \cite{milner91polyadicpi}. 

\begin{definition}
An \emph{observation relation}, $\downarrow_{\mathcal N}$, over a set
of names, $\mathcal N$, is the smallest relation satisfying the rules
below.

\infrule[Out-barb]{y \in {\mathcal N}, \; x \nameeq y}
		  {\outputp{x}{v} \downarrow_{\mathcal N} x}
\infrule[Par-barb]{\mbox{$P\downarrow_{\mathcal N} x$ or $Q\downarrow_{\mathcal N} x$}}
		  {\binpar{P}{Q} \downarrow_{\mathcal N} x}

We write $P \Downarrow_{\mathcal N} x$ if there is $Q$ such that 
$P \wred Q$ and $Q \downarrow_{\mathcal N} x$.
\end{definition}

\begin{definition}
%\label{def.bbisim}
An  ${\mathcal N}$-\emph{barbed bisimulation} over a set of names, ${\mathcal N}$, is a symmetric binary relation 
${\mathcal S}_{\mathcal N}$ between agents such that $P\rel{S}_{\mathcal N}Q$ implies:
\begin{enumerate}
\item If $P \red P'$ then $Q \wred Q'$ and $P'\rel{S}_{\mathcal N} Q'$.
\item If $P\downarrow_{\mathcal N} x$, then $Q\Downarrow_{\mathcal N} x$.
\end{enumerate}
$P$ is ${\mathcal N}$-barbed bisimilar to $Q$, written
$P \wbbisim_{\mathcal N} Q$, if $P \rel{S}_{\mathcal N} Q$ for some ${\mathcal N}$-barbed bisimulation ${\mathcal S}_{\mathcal N}$.
\end{definition}

$\mathcal{R} \subseteq \pi \times \pi$

$P \mathcal{R} Q => \forall P'. P \red P' \Rightarrow \exists Q'. Q \red Q', P' \mathcal{R} Q'$

$P \vdash x \Rightarrow Q \vdash x$

\begin{mathpar}
  \inferrule*[lab=Out-barb]{x \nameeq y}{{y}!\langle{Q}\rangle \vdash x}
  \and
  \inferrule*[lab=Par-barb]{\mbox{$P\vdash x$ or $Q\vdash x$}}{\binpar{P}{Q} \vdash x}
\end{mathpar}

\subsubsection{Contexts}

One of the principle advantages of computational calculi like the
$\pi$-calculus is a well-defined notion of context,
contextual-equivalence and a correlation between
contextual-equivalence and notions of bisimulation. The notion of
context allows the decomposition of a process into (sub-)process and
its syntactic environment, its context. Thus, a context may be
thought of as a process with a ``hole'' (written $\Box$) in it. The
application of a context $M$ to a process $P$, written $M[P]$, is
tantamount to filling the hole in $M$ with $P$. In this paper we do
not need the full weight of this theory, but do make use of the notion
of context in the proof the main theorem. 

\begin{mathpar}
  \inferrule* [lab=summation] {} {{M_{M},M_{N}} \bc \Box \;|\; x.M_{A} \;|\; M_{M}+M_{N}}
  \and
  \inferrule* [lab=agent] {} {{M_{A}} \bc (\vec{x})M_{P} \;| \; \clift{P_0,\ldots,M_{P},\ldots,P_N}}
  \and \\
  \inferrule* [lab=process] {} {{M_{P}} \bc M_{N} \;| \;P|M_{P} }
\end{mathpar} 

\begin{mathpar}
  \inferrule* [lab=sychronization] {} {M_{N} \bc \Box \;|\; x?M_{F} \;|\; x!M_{C}}
  \and
  \inferrule* [lab=abstraction] {} {{M_{F}} \bc (x)M_{P} }
  \and
  \inferrule* [lab=concretion] {} {{M_{C}} \bc \langle M_{P} \rangle }
  \and \\
  \inferrule* [lab=process] {} {{M_{P}} \bc M_{N} \;| \;P|M_{P} }
\end{mathpar}

\begin{definition}[contextual application] Given a context $M$, and
  process $P$, we define the \emph{contextual application}, $M[P] :=
  M\{P/\Box\}$. That is, the contextual application of M to P is the
  substitution of $P$ for $\Box$ in $M$.
\end{definition}

$\meaningof{-} : L \to \mathcal{P}(\pi)$

\begin{mathpar}
  \inferrule* [lab=collection] {} {\meaningof{true} = \pi, \and \meaningof{~E} = \pi \setminus \meaningof{E}, \and \meaningof{E_{1} \& E_{2}} = \meaningof{E_{1}} \cap \meaningof{E_{2}}}
\end{mathpar}

\begin{mathpar}
  \inferrule* [lab=structure] {} {\meaningof{0} = \{ P \in \pi | P \equiv 0 \}, \and \\ \meaningof{E_1 | E_2} = \{ P \in \pi | P \equiv P_{1} | P_{2}, P_{1} \in \meaningof{E_{1}}, P_{2} \in \meaningof{E_2}\} }
\end{mathpar}

\begin{mathpar}
 \inferrule* [lab=behavior] {} {\meaningof{\langle a?b \rangle E} = \{ P \in \pi | P \equiv Q | u?(y)P', \\ \and \\\\ \and \\ \;\;\; u \in \meaningof{a}, \forall z.P'\{z/y\} \in \meaningof{E\{z/b\}}\}, \and \\ \meaningof{a!E} = \{ P \in \pi | P \equiv Q | x!\langle P' \rangle, x \in \meaningof{a} P' \in \meaningof{E}\} }
\end{mathpar}

\begin{mathpar}
 \inferrule* [lab=nominal] {} {\meaningof{\quotep{E}} = \{ \quotep{P} \in \quotep{\pi} | P \in \meaningof{E} \}, \and \meaningof{\quotep{P}} = \{ \quotep{Q} \in \quotep{\pi} | P \equiv Q \} \and \\ \meaningof{@\quotep{E}} = \{ P \in \pi | P \equiv @x, x \in \meaningof{E} \}}
\end{mathpar}

\begin{eqnarray*}
  \\
  \meaningof{-} : TS \to ST
\end{eqnarray*}

\begin{eqnarray*}
  \\
  L : TS \to ST
\end{eqnarray*}

\begin{eqnarray*}
  \\
  P \models E \iff P \in \meaningof{E}
\end{eqnarray*}

\begin{eqnarray*}
  P \approx_{L} Q \iff \forall E \in L. P \models E \iff Q \models E
\end{eqnarray*}

\begin{eqnarray*}
  P \approx_{K} Q
\end{eqnarray*}

\begin{eqnarray*}
  P \approx Q
\end{eqnarray*}

$\approx_{K} = \approx = \approx_{L}$

\subsubsection{Contextual duality}

Note that contexts extend the quotation operation to a family of
operations from processes to names. Given a context, $M$, we can
define a \emph{nominal context}, $\quotep{M}$ by $\quotep{M}[P] :=
\quotep{M[P]}$. To foreshadow what is to come we observe that these
operations enjoy a duality with processes very much like the duality
between vectors and maps from vectors to scalars.

Further, because the calculus is essentially higher-order, we have a
correspondence between contexts and processes. More specifically,
given a name $x$ and a context $M$ we can construct $M^{*}_{x}$ such
that 

\begin{mathpar}
  M^{*}_{x} | \lift{x}{P} \red M[P]
\end{mathpar}

namely,

\begin{mathpar}
  M^{*}_{x} := x?(u).M[\dropn{u}]
\end{mathpar}

The dependence of $M^{*}_{x}$ on a name makes it an abstraction, 

\begin{mathpar}
  M^{*} := (x)x?(u).M[\dropn{u}]
\end{mathpar}

\subsection{Additional notation}

It will sometimes be convenient to denote the process a name
quotes. We already have the notation $x = \quotep{P}$, but it will be
convenient to introduce an alternate notation, $\procn{x}$, when we
want to emphasize the connection to the use of the name. Note that, by
virtue of name equivalence, $\quotep{\procn{x}} \nameeq x$; so, the
notation is consistent with previous definitions.

Further, because names have structure it is possible to effect
substitutions on the basis of that structure. This means we need to
upgrade our notation for substitutions, which we accomplish by
adapting comprehension notation. Thus,

\begin{mathpar}
  P\{ y / x : x \in S \}
\end{mathpar}

is interpreted to mean the process derived from P by replacing (in a
capture-avoiding manner) each occurrence of $x$ in $S$ by $y$. For example,

\begin{mathpar}
  P\{ \quotep{\procn{x}|\procn{x}} / x : x \in \freenames{P} \}
\end{mathpar}

will replace each (occurrence) of a free name $x$ in $P$ by
$\quotep{\procn{x}|\procn{x}}$.

Also, we will avail ourselves of the notation $x^{L}$ and $x^{R}$ to
denote injections of a name into disjoint copies of the name
space. There are numerous ways to accomplish this. One example can be
found in \cite{MeredithR05}. This notation overloads to vectors of
names: $\vec{x}^{\pi} := (x_{i}^{\pi} \; : \; 0 \leq i < |\vec{x}| )$ where $\pi \in \{L,R\}$.

We also use $P^{\Box} := P|\Box$.

In \cite{MeredithR05} an interpretation of the new operator is
given. It turns out that there are several possible interpretations
all enjoying the requisite algebraic properties of the operator (see
\cite{milner91polyadicpi}). We will therefore make liberal use of
$(\nu\; \vec{x})P$.

% subsection the_syntax_and_semantics_of_the_notation_system (end)   

\input{qm2pi.qmops} 

\input{qm2pi.sterngerlach} 

\input{qm2pi.metric} 

% section concurrent_process_calculi (end)

%\input{qm2pi.proofsketch}

% section proof sketch (end)

%\input{qm2pi.slviaknots} 

% section spatial logic via knots (end)

\input{qm2pi.conclusion}

% section conclusion (end)

%\input{qm2pi.dtcodes} 

% section wiring algorithm (end)

\input{qm2pi.ack} 

% section acknowledgments (end)

\newpage


\bibliographystyle{plain}   
\bibliography{../../biblios/main.bib}

\input{qm2pi.rhodetails}

\end{document}



% section proof sketch (end)

%\section{Unlikely characters: spatial logic for
  knots}\label{sub:characteristic_formulae} % (fold)

Associated to the mobile process calculi are a family of logics known
as the Hennessy-Milner logics. These logics typically enjoy a
semantics interpreting formulae as sets of processes that when
factored through the encoding outlined above allows an identification
of classes of knots with logical formulae. In the context of this
encoding the sub-family known as the spatial logics \cite{CairesC03}
\cite{CairesC04} \cite{Caires04} are of particular interest providing
several important features for expressing and reasoning about
properties (i.e. classes) of knots. We hint here at how this may be done.

%\begin{description}
%\item [structural connectives] 
\subsubsection{Structural connectives} The spatial logics enjoy
structural connectives corresponding, at the logical level, to the
parallel composition ($P | Q$) and new name ($(\nu \; x)P$)
connectives for processes. As illustrated in the examples below, these
connectives are extremely expressive given the shape of our encoding.
%\item [decideable satisfaction]

\subsubsection{Decideable satisfaction}
In \cite{Caires04} the satisfaction relation is shown to be decideable
for a rich class of processes. It further turns out that the image of
the our encoding is a proper subset of that class. This result
provides the basis for an algorithm by which to search for knots
enjoying a given property.
%\item [characteristic formulae]

\subsubsection{Characteristic formulae}
In the same paper \cite{Caires04} , Caires presents a means of calculating
characteristic formulae, selecting equivalence classes of processes
up to a pre--specified depth limit on the support set of names. Composed with our
encoding, this characteristic formula can be used to select
characteristic formulae for knots.
%\end{description}

\subsubsection{Spatial logic formulae}

The grammar below (segmented for comprehension) summarizes the syntax
of spatial logic formulae. We employ illustrative examples in the
sequel to provide an intuitive understanding of their meaning
referring the reader to \cite{Caires04} for a more detailed explication
of the semantics.

\begin{mathpar}
  \inferrule* [lab=boolean] {} {{A,B} \bc T \;|\; \neg A \;|\; A \wedge B \;|\; \eta = \eta'}
  \and
  \inferrule* [lab=spatial] {} {|\; \pzero \;|\; A | B \;|\; x \text{\textregistered} A \;|\; \forall x . A \;|\;  H x . A}
  \and
  \inferrule* [lab=behavioral] {} {|\; \alpha . A}
  \and 
  \inferrule* [lab=recursion] {} {|\; X(\vec{u}) \;|\; \mu X(\vec{u}) . A}
  \and
  \inferrule* [lab=action] {} {\alpha \bc \langle x?(\vec{y}) \rangle \;|\; \langle x!(\vec{y}) \rangle \;|\; \langle \tau \rangle}
  \and 
  \inferrule* [lab=name] {} {\eta \bc x \;|\; \tau}
\end{mathpar} 

% subsection characteristic_formulae (end)   	 

\subsection{Example formulae}\label{sub:example_formulae_} % (fold)

\subsubsection{Crossing as formula.}
% 
% \begin{align*}
%   \frac{d}{dx} \sin x &= \cos x 
%   & \frac{d}{dx} e^x &= e^x \\
%   \frac{d}{dx} \cos x &= - \sin x 
%   & \frac{d}{dx} \log x &= \frac{1}{x} \\
% \end{align*} 

\begin{align*}
 \mu C(x_{0},x_{1},y_{0},y_{1},u).&(\langle x_{0}?(z) \rangle(\langle u! \rangle\langle y_{1}!z \rangle C(x_{0},x_{1},y_{0},y_{1},u)) & \\
  & \wedge \langle y_{1}?(z) \rangle (\langle u! \rangle \langle x_{0}!z \rangle C(x_{0},x_{1},y_{0},y_{1},u)) & \\
  & \wedge \langle x_{1}?(z) \rangle (\langle u? \rangle \langle y_{0}!z \rangle C(x_{0},x_{1},y_{0},y_{1},u)) & \\
  & \wedge \langle y_{0}?(z) \rangle (\langle u? \rangle \langle x_{1}!z \rangle C(x_{0},x_{1},y_{0},y_{1},u))) &
\end{align*}

The lexicographical similarity between the shape of this formulae and
the shape of definition of the process representing a crossing reveals
the intuitive meaning of this formulae. It describes the capabilities
of a process that has the right to represent a crossing. For example
it picks out processes that may perform an input on the port $x_0$ in
its initial menu of capabilities. What differentiates the formula
from the process, however, is that the crossing process is the
smallest candidate to satisfy the formula. Infinitely many other
processes -- with internal behavior hidden behind this interface, so
to speak -- also satisfy this formula. Even this simple formula,
then, can be seen to open a new view onto knots, providing a
computational interpretation of \emph{virtual} knots.

Note that this formula is derived by hand. A similar formula can be
derived by employing Caires' calculation of characteristic formula
\cite{Caires04} to the process representing a crossing. In light of
this discussion, we let
$\meaningof{C}_{\phi}(x0,x1,y0,y1,u)$ denote a formula specifying the
dynamics we wish to capture of a crossing. To guarantee we preserve
the shape of the interface and minimal semantics we demand that
$\meaningof{C}_{\phi}(x0,x1,y0,y1,u) \Rightarrow
\textbf{C}(x0,x1,y0,y1,u)$ where $\textbf{C}(x0,x1,y0,y1,u)$ denotes
the formula above.
                            
\subsubsection{Crossing number constraints.}
The moral content of the context lemma (Lemma \ref{context}) is that the notion of
``locality'' in the Reidemeister moves is effectively captured by the
parallel composition operator of the process calculus. This intuition
extends through the logic. Given a formula,
$\meaningof{C}_{\phi}(x0,x1,y0,y1,u)$, we can use the structural
connectives to specify constraints on crossing numbers, such as at
least $n$ crossings, or exactly $n$ crossings.
\begin{mathpar}
  \inferrule* [lab=at-least-n] {} { K^{\geq n}_{\phi}(\vec{xs},\vec{ys}) := \Pi_{i=0}^{n-1} Hu . \meaningof{C}_{\phi}(xs_i,ys_i,u) | T }
  \and 
  \inferrule* [lab=exactly-n] {} { K^{= n}_{\phi}(\vec{xs},\vec{ys}) := \Pi_{i=0}^{n-1} Hu . \meaningof{C}_{\phi}(xs_i,ys_i,u) | \neg (\forall x_0,y_0,x_1,y_1,u . \meaningof{C}_{\phi}(x_0,y_0,x_1,y_1,u) | T) }
\end{mathpar}

To round out this section, recall that the encoding of an $n$-crossing
knot decomposes into a parallel composition of $n$ \emph{copies} of a
crossing process together with a wiring harness. To specify different
knot classes with the same crossing number amounts to specifying
logical constraints on the wiring harness. In the interest of space,
we defer examples to a forthcoming paper. Suffice it to say that both
the conditions ``alternating knot'' and ``contains the tangle
corresponding to 5/3'' are expressible. For example, it is possible to
calculate the characteristic formula of a process corresponding to the
tangle 5/3 and conjoin it into the classifying formula via the
composition connective of the logic.

Finally, we wish to observe that it is entirely within reason to
contemplate a more domain-specific version of spatial logic tailored
to the shape of processes in the image of the encoding. Such a
domain-specific logic would have a better claim to the title formal
language of knot properties.

% subsection example_formulae_ (end)

% section knots_as_processes (end) 

% section spatial logic via knots (end)

\section{Conclusions and future work}

\paragraph{Testing physical space}
You, gentle reader, may wonder why of all the theorems to be proved
given this set up we pick the one above. In some sense it's hardly
central to quantum mechanics. We see it as central in the sense that
it firmly establishes a notion of physical space arising from a notion
of the equivalence of behavior. Relating bisimulation to a metric is a
big step forward, but one is faced with interpreting the relationship
of that metric space to something more physical. Quantum mechanical
notions of ``physical'' space are still far from intuitive, but by
relating this idea of distance as testing to calculations that predict
physical circumstances we are making a not insignificant step forward
toward an understanding of the physical space we inhabit as
essentially dynamic.

\paragraph{Effectivity and simulation}
One of the observations we have yet to make is that the entire program
spelled out here is effective. We have built various interpreters for
the reflective calculus at work in this interpretation. In principle,
then, we can simulate quantum mechanics on a computer. The place where
the simulation may lose fidelity is the infinitely branching summation
for the annihilator.

In this connection i also want to point out that the evaluation style
calculation of the inner product puts the non-determinism of the
summation right at the heart of measurement. This suggests that
Milner's original reduction-based formulation of the dynamics of his
calculi in terms of sums was not just notationally suggestive of a
notion of measure-and-continue but captured some significant part of
the physics.

\paragraph{Quantum continuations}
In light of this last observation i want to point out that the
predominant account of quantum mechanics is missing a key aspect of a
truly compositional story of the physical situation. In a real lab,
when a measurement is made the observation can be made to feed into
another device that then makes another measurement conditioned on the
results of the first. This means that after the superposition was
collapsed the entire experimental set up remained in
superposition. While QM offers a means of writing this down it doesn't
quite line up well with the well-trodden formulation of computation
and continuation that we see so succinctly expressed in Milner's
calculi. This suggests that there might be advantages to this account
of dynamics waiting to be explored.

\paragraph{Quantum logic}
In this connection, we also note that by virtue of having the
Hennessy-Milner construction, we can pull the construction through the
interpretation of QM. This gives us a natural candidate for a quantum
logic that enjoys an extremely tight connection with it's domain of
interpretation, making the construction much less ad hoc (rather it is
the image of functor!).

\paragraph{Quantum probabiity}
i have questions about the basis of the interpretation of inner
product as probability amplitude. In particular, using which
axiomatization of probability theory does the notion of probability
amplitude earn the right to be so dubbed? In other words, where is the
proof that the operation for calculating a probability amplitude (and
then squaring) satisfies the axioms of what it means to calculate a
probability? Even if such a proof exists (i have yet to find it in the
literature), i wonder if it might not be possible to turn things on
their heads. Can we view the calculation of the probability amplitude
as an axiomatization of probability? If so, then the definition we
give for calculating probability amplitude may provide the basis for
an \emph{effective} theory of probability.

\paragraph{Quantum vs ``biological'' information}
Finally, i want to conclude with a more philosophical observation. At
a recent workshop in which QM was a predominant topic i noticed
something about quantum information. The speaker was giving a riveting
discussion of axiomatic QM and showing how properties of ``no
cloning'' and ``no deleting'' emerged as consequences of the
axiomatization. Theorems of this form are necessary to give us a sense
of confidence that our axioms characterize the physical theory. What
struck me, though, was that if quantum information is neither erasable
nor replicable it is markedly different from \emph{life}. Two of the
things we know about life is that

\begin{itemize}
  \item it ends;
  \item to gain some measure of persistence, to transcend it's
    finitude it is imminently copyable.
\end{itemize}

Both of these qualities are summarized succinctly in the aphorism: all
flesh is grass. For me these two kinds of ``information'' -- call them
quantum and biological -- are end points on a spectrum of strategies
for persistence. At one end, we have those curious entities that enjoy
uniqueness and permanence; at the other, we have those who in the face
of a certain end and an uncertain present make a go of passing
something on. To me one of the more remarkable aspects of the latter
strategy is that in the presence of noise (and certain features of
copying) we get a kind of dynamism, a chance for improvement against a
given persistent condition.

% subsection other_calculi_other_bisimulations_and_geometry_as_behavior (end)




% section conclusion (end)

%\documentclass[12pt]{llncs}
%\documentclass{jktr}

\usepackage[pdftex]{hyperref}                   
\usepackage {listings}
\usepackage {mathpartir}
\usepackage{bcprules}
%\usepackage{listings}
                       
\usepackage{graphicx} 
%\usepackage[margins=2.5cm,nohead,nofoot]{geometry}
%\usepackage{geometry}
\usepackage{amsfonts}
\usepackage{amstext}
\usepackage{latexsym}
\usepackage{amssymb}
\usepackage{color}


%\include{myPreamble}
\include{qm2pi.local} 

%\ifpdf
%\usepackage[pdftex]{graphicx}
%\else
%\usepackage{graphicx}
%\fi

 % \ifpdf
%  \usepackage{pdfsync}
%  \if


%\title{Brief Article}
%\author{David F. Snyder}
%\author{L.G. Meredith}

%\address{Dept. of Math., Texas State University--San Marcos, San Marcos, TX 78666}
       
\pagestyle{empty}


\begin{document}

\lstset{language=[Objective]Caml,frame=shadowbox}

\input{qm2pi.front}

% section front matter (end)

\input{qm2pi.intro} 
 
% section introduction (end)

% \input{qm2pi.knotations} 

% section notation (end)

\input{qm2pi.process.calculi} 

% section concurrent_process_calculi_and_spatial_logics_ (end)
    
%\input{qm2pi.knots2pi} 

%\input{qm2pi.trefoil} 

%\input{qm2pi.mainthm} 

% subsection basic_interpretation (end)

%\input{qm2pi.rho.presentation} 
\subsection{The syntax and semantics of the notation system}\label{sub:the_syntax_and_semantics_of_the_notation_system} % (fold)

We now summarize a technical presentation of the calculus that
embodies our theory of dynamics. The typical presentation of such a
calculus follows the style of giving generators and relations on
them. The grammar, below, describing term constructors, freely
generates the set of processes, $\Proc$. This set is then quotiented
by a relation known as structural congruence and it is over this set
that the notion of dynamics is expressed. This presentation is
essentially that of \cite{MeredithR05} with the addition of
polyadicity and summation. For readability we have relegated some of
the technical subtleties to an appendix.

\subsubsection{Process grammar}\label{subsub:process_grammar}

\begin{mathpar}
  \inferrule* [lab=synchronization] {} {{M} \bc \pzero \;|\; x?F \;|\; x!C }
  \and
  \inferrule* [lab=abstraction] {} {{F} \bc (x)P}
  \and
  \inferrule* [lab=concretion] {} {{C} \bc \langle Q \rangle}
  \and
  \inferrule* [lab=process] {} {{P,Q} \bc M \;| \;P|Q \;|\; @{x}}
  \and
  \inferrule* [lab=name] {} {{x} \bc \quotep{P}}
\end{mathpar} 

Note that $\vec{x}$ (resp. $\vec{P}$) denotes a vector of names
(resp. processes) of length $|\vec{x}|$ (resp. $|\vec{P}|$). We adopt
the following useful abbreviations.

\begin{mathpar}
   x?(\vec{y}).P := x.(\vec{y})P \and  x\clift{\vec{P}} := x.\clift{\vec{P}}
   \and x!(y) := \lift{x}{\dropn{y}}
   \and \Pi_{i=0}^{n-1}P_i := P_0 | \ldots | P_{n-1}
\end{mathpar}

\subsubsection{Structural congruence}

\paragraph{Free and bound names and alpha-equivalence.} At the
core of structural equivalence is alpha-equivalence which identifies
process that are the same up to a change of variable. Formally, we
recognize the distinction between free and bound names. The free names
of a process, $\freenames{P}$, may be calculated recursively as
follows:

\begin{mathpar}
\freenames{\pzero} := \emptyset
  \and \\
  \freenames{x?(y).P} := \{ x \} \cup (\freenames{P} \setminus \{ y \})
  \and 
  \freenames{x!\langle P \rangle} := \{ x \} \cup \{ P \} 
  \and \\
  \freenames{P|Q} := \freenames{P} \cup \freenames{Q}
  \and \\
  \freenames{@{x}} := \{ x \}
\end{mathpar}

$\pi$
$\quotep{\pi}$

$\freenames{-} : \pi \to \mathcal{P}(\quotep{\pi})$

\begin{eqnarray*}
  \freenames{\pzero} & := & \emptyset \\
  \freenames{x?(y).P} & := & \{ x \} \cup (\freenames{P} \setminus \{ y \}) \\
  \freenames{x!\langle P \rangle} & := & \{ x \} \cup \{ P \} \\
  \freenames{P|Q} & := & \freenames{P} \cup \freenames{Q} \\
  \freenames{\dropn{x}} & := & \{ x \}
\end{eqnarray*}

The bound names of a process, $\boundnames{P}$, are those names occurring in $P$
that are not free. For example, in $x?(y).0$, the name $x$ is free, while $y$ is bound.

\begin{mathpar}
  \inferrule* [lab=monoidal-laws] {} { P|Q \equiv Q|P \and P|0 \equiv P \and P|(Q|R) \equiv (P|Q)|R }
\end{mathpar}

\begin{mathpar}
  \inferrule* [lab=alpha-equivalence] {} { (x)P \equiv (y)P\{y/x\} \and y \not\in \freenames{P} }
\end{mathpar}

\begin{definition}
Then two processes, $P,Q$, are alpha-equivalent if $P = Q\{\vec{y}/\vec{x}\}$ for
some $\vec{x} \in \boundnames{Q},\vec{y} \in \boundnames{P}$, where $Q\{\vec{y}/\vec{x}\}$
denotes the capture-avoiding substitution of $\vec{y}$ for $\vec{x}$ in $Q$.
\end{definition}

\begin{definition}
  The {\em structural congruence} \cite{SangiorgiWalker} , $\equiv$,
  between processes is the least congruence containing
  alpha-equivalence, satisfying the abelian monoid laws
  (associativity, commutativity and $\pzero$ as identity) for parallel
  composition $|$ and for summation $+$.
\end{definition}

\subsection{Name equivalence}

We take name equivalence, written $\nameeq$, to be the smallest
equivalence relation generated by the following rules.

\begin{mathpar}
\inferrule*[lab=Quote-drop]
{ }
{ \quotep{@{x}} \nameeq x }

\inferrule*[lab=Struct-equiv]
{ P \scong Q }
{ \quotep{P} \nameeq \quotep{Q} }
\end{mathpar}

The astute reader will have noticed that the mutual recursion of names
and processes imposes a mutual recursion on alpha-equivalence and
structural equivalence via name-equivalence. Fortunately, all of this
works out pleasantly and we may calculate in the natural way, free of
concern. The reader interested in the details is referred to the
appendix \ref{appendix:rho_details}.

\subsection{Substitution}

We use $\Proc$ for the set of processes, $\QProc$ for the set of
names, and $\id{\{}\vec{y} / \vec{x} \id{\}}$ to denote partial maps,
$s : \QProc \rightarrow \QProc$. A map, $s$ lifts, uniquely, to a map
on process terms, $\widehat{s} : \Proc \rightarrow \Proc$ by the
following equations.

\begin{mathpar}
  (0) \psubstp{Q}{P} := 0 \\
  (R \juxtap S) \psubstp{Q}{P}
  :=    
  (R)\psubstp{Q}{P} \juxtap (S) \psubstp{Q}{P} \\
  (x?(y).R) \psubstp{Q}{P}    
  :=    
  (x)\substp{Q}{P} (z)\concat( (R \psubstn{z}{y}) \psubstp{Q}{P} ) \\
  (\lift{x}{R}) \psubstp{Q}{P}  
  :=
  \lift{(x)\substp{Q}{P}}{ R \psubstp{Q}{P} } \\
%   (\dropn{x})  \psubstp{Q}{P}       
%   := 
%   \left\{ 
%     \begin{array}{ccc} 
%       \dropn{\quotep{Q}} & & x \nameeq \quotep{P} \\
%       \dropn{x} & & otherwise \\
%     \end{array}
%   \right. 
  (\dropn{x})  \psubstp{Q}{P}       
  := 
  \left\{ 
    \begin{array}{ccc} 
      Q & & x \nameeq \quotep{P} \\
      \dropn{x} & & otherwise \\
    \end{array}
  \right.
\end{mathpar}
 

where

\begin{eqnarray}
  (x)\id{\{} \lpquote Q \rpquote / \lpquote P \rpquote \id{\}}            = 
  \left\{ 
    \begin{array}{ccc}
      \lpquote Q \rpquote & & x \nameeq \lpquote P \rpquote \\
      x & & otherwise \\
    \end{array}
  \right. \nonumber
\end{eqnarray}

and $z$ is chosen distinct from $\quotep{P}$, $\quotep{Q}$, the free
names in $Q$, and all the names in $R$. Our $\alpha$-equivalence will
be built in the standard way from this substitution.

\begin{remark}\label{rem:no_self_referential_names}
  One consequence of these definitions is that $\forall P. \quotep{P}
  \not\in \freenames{P}$.
\end{remark}

\subsection{ Dynamic quote: an example }

Anticipating something of what's to come, consider applying the
substitution, $\widehat{\id{\{}u / z \id{\}}}$, to the following pair
of processes, $\lift{w}{y!(z)}$ and $w[ \lpquote y!(z) \rpquote ]$.

\begin{eqnarray}
	\lift{w}{y!(z)}\widehat{\id{\{}u / z \id{\}}}
		& = &
		\lift{w}{y!(u)} \nonumber\\
	w[ \lpquote y!(z) \rpquote ] \widehat{ \id{\{}u / z \id{\}} }
		& = &
		w[ \lpquote y!(z) \rpquote ] \nonumber
\end{eqnarray}

Because the body of the process between quotes is impervious to
substitution, we get radically different answers. In fact, by
examining the first process in an input context,
e.g. $x?(z).\lift{w}{y!(z)}$, we see that the process under the lift
operator may be shaped by prefixed inputs binding a name inside it. In
this sense, the lift operator will be seen as a way to dynamically
construct processes before reifying them as names.

Finally equipped with these standard features we can present the
dynamics of the calculus.

\subsubsection{Operational semantics} 

Finally, we introduce the computational dynamics. What marks these
algebras as distinct from other more traditionally studied algebraic
structures, e.g. vector spaces or polynomial rings, is the manner in
which dynamics is captured. In traditional structures, dynamics is typically
expressed through morphisms between such structures, as in linear maps
between vector spaces or morphisms between rings. In algebras
associated with the semantics of computation, the dynamics is
expressed as part of the algebraic structure itself, through a
reduction reduction relation typically denoted by $\red$. Below, we
give a recursive presentation of this relation for the calculus used
in the encoding.

$\red \subseteq \pi \times \pi$
$\red : \pi \to \mathcal{P}(\pi)$

\begin{mathpar}
  \inferrule* [lab=Comm] { \textsf{match}( x_{src}, x_{trgt} ) } { x_{trgt}?(y)P \; | \; x_{src}!\langle {Q} \rangle \red P\{\quotep{Q}/y}\} }
  \and \\
  \inferrule* [lab=Par] {{P} \red {P}'} {{{P} | {Q}} \red {{P}' | {Q}}}
  \and
  \inferrule* [lab=Equiv]{{{P} \scong {P}'} \andalso {{P}' \red {Q}'} \andalso {{Q}' \scong {Q}}}{{P} \red {Q}}
\end{mathpar}

\begin{eqnarray*}
  match_{\equiv} (\quotep{P},\quotep{Q}) & := & P \equiv Q \\
  match_{\dagger}(\quotep{P},\quotep{Q}) & := & \forall R. P|Q \red^{*} R => R \red^{*} 0 \\
  match_{K}(\quotep{P},\quotep{Q}) & := & K \mbox{ for some context } K
\end{eqnarray*}

$u?(x)P | u!\langle Q \rangle \red P\{\quotep{Q}/x\}$

%We write $\wred$ for $\red^*$, and $P\red$ if $\exists Q $ such that $ P \red Q$.
We write $P\red$ if $\exists Q $ such that $ P \red Q$ and $P\not\red$, otherwise.

\section{Replication}

As mentioned before, it is known that replication (and hence
recursion) can be implemented in a higher-order process algebra
\cite{SangiorgiWalker}. As our first example of calculation with the
machinery thus far presented we give the construction explicitly in
the {\rhoc}.

\begin{eqnarray}
	D_{x} & := & \prefix{x}{y}{(\binpar{\outputp{x}{y}}{@{y}})} \nonumber\\
	\bangp_{x}{P} & := & \binpar{{x}!\langle{\binpar{D_{x}}{P}}\rangle}{D_{x}} \nonumber
\end{eqnarray}

\begin{eqnarray}
	\bangp_{x}{P} & & \nonumber\\
	=
	& {x}!\langle{(\prefix{x}{y}{(\outputp{x}{y} | @{y})) | P}}\rangle 
	      | \prefix{x}{y}{(\outputp{x}{y} | @{y})} & \nonumber\\
	\red
	& (\outputp{x}{y} | @{y})\substn{\quotep{(\prefix{x}{y}{(@{y} | \outputp{x}{y})) | P}}}{y} & \nonumber\\
	=
	& \outputp{x}{\quotep{(\prefix{x}{y}{(\outputp{x}{y} | @{y})) | P}}}
	  | {(\prefix{x}{y}{(\outputp{x}{y} | @{y})) | P}} & \nonumber\\
	\red
	& \ldots & \nonumber\\
	\red^*
	& P | P | \ldots & \nonumber
\end{eqnarray}

Of course, this encoding, as an implementation, runs away, unfolding
$\bangp{P}$ eagerly. A lazier and more implementable replication
operator, restricted to input-guarded processes, may be obtained as follows.

\begin{eqnarray}
\bangp{\prefix{u}{v}{P}} 
	:= 
	\binpar{\lift{x}{\prefix{u}{v}{(\binpar{D(x)}{P})}}}{D(x)} \nonumber
\end{eqnarray}

\begin{remark}
  Note that the lazier definition still does not deal with summation
  or mixed summation (i.e. sums over input and output). The reader is
  invited to construct definitions of replication that deal with these
  features. 

  Further, the definitions are parameterized in a name, $x$. Can you,
  gentle reader, make a definition that eliminates this parameter and
  guarantees no accidental interaction between the replication
  machinery and the process being replicated -- i.e. no accidental
  sharing of names used by the process to get its work done and the
  name(s) used by the replication to effect copying. This latter
  revision of the definition of replication is crucial to obtaining
  the expected identity $!!P \sim !P$.
\end{remark}

\begin{remark}\label{rem:paradoxical_combinator}
  The reader familiar with the lambda calculus will have noticed the
  similarity between $D$ and the paradoxical combinator.

  [Ed. note: the existence of this seems to suggest we have to be more
  restrictive on the set of processes and names we admit if we are to
  support no-cloning.]
\end{remark}

\subsubsection{Bisimulation}

The computational dynamics gives rise to another kind of equivalence,
the equivalence of computational behavior. As previously mentioned
this is typically captured \emph{via} some form of bisimulation.

% The notion we use in this paper is weak barbed bisimulation
% \cite{milner91polyadicpi}.

The notion we use in this paper is derived from weak barbed
bisimulation \cite{milner91polyadicpi}. 

\begin{definition}
An \emph{observation relation}, $\downarrow_{\mathcal N}$, over a set
of names, $\mathcal N$, is the smallest relation satisfying the rules
below.

\infrule[Out-barb]{y \in {\mathcal N}, \; x \nameeq y}
		  {\outputp{x}{v} \downarrow_{\mathcal N} x}
\infrule[Par-barb]{\mbox{$P\downarrow_{\mathcal N} x$ or $Q\downarrow_{\mathcal N} x$}}
		  {\binpar{P}{Q} \downarrow_{\mathcal N} x}

We write $P \Downarrow_{\mathcal N} x$ if there is $Q$ such that 
$P \wred Q$ and $Q \downarrow_{\mathcal N} x$.
\end{definition}

\begin{definition}
%\label{def.bbisim}
An  ${\mathcal N}$-\emph{barbed bisimulation} over a set of names, ${\mathcal N}$, is a symmetric binary relation 
${\mathcal S}_{\mathcal N}$ between agents such that $P\rel{S}_{\mathcal N}Q$ implies:
\begin{enumerate}
\item If $P \red P'$ then $Q \wred Q'$ and $P'\rel{S}_{\mathcal N} Q'$.
\item If $P\downarrow_{\mathcal N} x$, then $Q\Downarrow_{\mathcal N} x$.
\end{enumerate}
$P$ is ${\mathcal N}$-barbed bisimilar to $Q$, written
$P \wbbisim_{\mathcal N} Q$, if $P \rel{S}_{\mathcal N} Q$ for some ${\mathcal N}$-barbed bisimulation ${\mathcal S}_{\mathcal N}$.
\end{definition}

$\mathcal{R} \subseteq \pi \times \pi$

$P \mathcal{R} Q => \forall P'. P \red P' \Rightarrow \exists Q'. Q \red Q', P' \mathcal{R} Q'$

$P \vdash x \Rightarrow Q \vdash x$

\begin{mathpar}
  \inferrule*[lab=Out-barb]{x \nameeq y}{{y}!\langle{Q}\rangle \vdash x}
  \and
  \inferrule*[lab=Par-barb]{\mbox{$P\vdash x$ or $Q\vdash x$}}{\binpar{P}{Q} \vdash x}
\end{mathpar}

\subsubsection{Contexts}

One of the principle advantages of computational calculi like the
$\pi$-calculus is a well-defined notion of context,
contextual-equivalence and a correlation between
contextual-equivalence and notions of bisimulation. The notion of
context allows the decomposition of a process into (sub-)process and
its syntactic environment, its context. Thus, a context may be
thought of as a process with a ``hole'' (written $\Box$) in it. The
application of a context $M$ to a process $P$, written $M[P]$, is
tantamount to filling the hole in $M$ with $P$. In this paper we do
not need the full weight of this theory, but do make use of the notion
of context in the proof the main theorem. 

\begin{mathpar}
  \inferrule* [lab=summation] {} {{M_{M},M_{N}} \bc \Box \;|\; x.M_{A} \;|\; M_{M}+M_{N}}
  \and
  \inferrule* [lab=agent] {} {{M_{A}} \bc (\vec{x})M_{P} \;| \; \clift{P_0,\ldots,M_{P},\ldots,P_N}}
  \and \\
  \inferrule* [lab=process] {} {{M_{P}} \bc M_{N} \;| \;P|M_{P} }
\end{mathpar} 

\begin{mathpar}
  \inferrule* [lab=sychronization] {} {M_{N} \bc \Box \;|\; x?M_{F} \;|\; x!M_{C}}
  \and
  \inferrule* [lab=abstraction] {} {{M_{F}} \bc (x)M_{P} }
  \and
  \inferrule* [lab=concretion] {} {{M_{C}} \bc \langle M_{P} \rangle }
  \and \\
  \inferrule* [lab=process] {} {{M_{P}} \bc M_{N} \;| \;P|M_{P} }
\end{mathpar}

\begin{definition}[contextual application] Given a context $M$, and
  process $P$, we define the \emph{contextual application}, $M[P] :=
  M\{P/\Box\}$. That is, the contextual application of M to P is the
  substitution of $P$ for $\Box$ in $M$.
\end{definition}

$\meaningof{-} : L \to \mathcal{P}(\pi)$

\begin{mathpar}
  \inferrule* [lab=collection] {} {\meaningof{true} = \pi, \and \meaningof{~E} = \pi \setminus \meaningof{E}, \and \meaningof{E_{1} \& E_{2}} = \meaningof{E_{1}} \cap \meaningof{E_{2}}}
\end{mathpar}

\begin{mathpar}
  \inferrule* [lab=structure] {} {\meaningof{0} = \{ P \in \pi | P \equiv 0 \}, \and \\ \meaningof{E_1 | E_2} = \{ P \in \pi | P \equiv P_{1} | P_{2}, P_{1} \in \meaningof{E_{1}}, P_{2} \in \meaningof{E_2}\} }
\end{mathpar}

\begin{mathpar}
 \inferrule* [lab=behavior] {} {\meaningof{\langle a?b \rangle E} = \{ P \in \pi | P \equiv Q | u?(y)P', \\ \and \\\\ \and \\ \;\;\; u \in \meaningof{a}, \forall z.P'\{z/y\} \in \meaningof{E\{z/b\}}\}, \and \\ \meaningof{a!E} = \{ P \in \pi | P \equiv Q | x!\langle P' \rangle, x \in \meaningof{a} P' \in \meaningof{E}\} }
\end{mathpar}

\begin{mathpar}
 \inferrule* [lab=nominal] {} {\meaningof{\quotep{E}} = \{ \quotep{P} \in \quotep{\pi} | P \in \meaningof{E} \}, \and \meaningof{\quotep{P}} = \{ \quotep{Q} \in \quotep{\pi} | P \equiv Q \} \and \\ \meaningof{@\quotep{E}} = \{ P \in \pi | P \equiv @x, x \in \meaningof{E} \}}
\end{mathpar}

\begin{eqnarray*}
  \\
  \meaningof{-} : TS \to ST
\end{eqnarray*}

\begin{eqnarray*}
  \\
  L : TS \to ST
\end{eqnarray*}

\begin{eqnarray*}
  \\
  P \models E \iff P \in \meaningof{E}
\end{eqnarray*}

\begin{eqnarray*}
  P \approx_{L} Q \iff \forall E \in L. P \models E \iff Q \models E
\end{eqnarray*}

\begin{eqnarray*}
  P \approx_{K} Q
\end{eqnarray*}

\begin{eqnarray*}
  P \approx Q
\end{eqnarray*}

$\approx_{K} = \approx = \approx_{L}$

\subsubsection{Contextual duality}

Note that contexts extend the quotation operation to a family of
operations from processes to names. Given a context, $M$, we can
define a \emph{nominal context}, $\quotep{M}$ by $\quotep{M}[P] :=
\quotep{M[P]}$. To foreshadow what is to come we observe that these
operations enjoy a duality with processes very much like the duality
between vectors and maps from vectors to scalars.

Further, because the calculus is essentially higher-order, we have a
correspondence between contexts and processes. More specifically,
given a name $x$ and a context $M$ we can construct $M^{*}_{x}$ such
that 

\begin{mathpar}
  M^{*}_{x} | \lift{x}{P} \red M[P]
\end{mathpar}

namely,

\begin{mathpar}
  M^{*}_{x} := x?(u).M[\dropn{u}]
\end{mathpar}

The dependence of $M^{*}_{x}$ on a name makes it an abstraction, 

\begin{mathpar}
  M^{*} := (x)x?(u).M[\dropn{u}]
\end{mathpar}

\subsection{Additional notation}

It will sometimes be convenient to denote the process a name
quotes. We already have the notation $x = \quotep{P}$, but it will be
convenient to introduce an alternate notation, $\procn{x}$, when we
want to emphasize the connection to the use of the name. Note that, by
virtue of name equivalence, $\quotep{\procn{x}} \nameeq x$; so, the
notation is consistent with previous definitions.

Further, because names have structure it is possible to effect
substitutions on the basis of that structure. This means we need to
upgrade our notation for substitutions, which we accomplish by
adapting comprehension notation. Thus,

\begin{mathpar}
  P\{ y / x : x \in S \}
\end{mathpar}

is interpreted to mean the process derived from P by replacing (in a
capture-avoiding manner) each occurrence of $x$ in $S$ by $y$. For example,

\begin{mathpar}
  P\{ \quotep{\procn{x}|\procn{x}} / x : x \in \freenames{P} \}
\end{mathpar}

will replace each (occurrence) of a free name $x$ in $P$ by
$\quotep{\procn{x}|\procn{x}}$.

Also, we will avail ourselves of the notation $x^{L}$ and $x^{R}$ to
denote injections of a name into disjoint copies of the name
space. There are numerous ways to accomplish this. One example can be
found in \cite{MeredithR05}. This notation overloads to vectors of
names: $\vec{x}^{\pi} := (x_{i}^{\pi} \; : \; 0 \leq i < |\vec{x}| )$ where $\pi \in \{L,R\}$.

We also use $P^{\Box} := P|\Box$.

In \cite{MeredithR05} an interpretation of the new operator is
given. It turns out that there are several possible interpretations
all enjoying the requisite algebraic properties of the operator (see
\cite{milner91polyadicpi}). We will therefore make liberal use of
$(\nu\; \vec{x})P$.

% subsection the_syntax_and_semantics_of_the_notation_system (end)   

\input{qm2pi.qmops} 

\input{qm2pi.sterngerlach} 

\input{qm2pi.metric} 

% section concurrent_process_calculi (end)

%\input{qm2pi.proofsketch}

% section proof sketch (end)

%\input{qm2pi.slviaknots} 

% section spatial logic via knots (end)

\input{qm2pi.conclusion}

% section conclusion (end)

%\input{qm2pi.dtcodes} 

% section wiring algorithm (end)

\input{qm2pi.ack} 

% section acknowledgments (end)

\newpage


\bibliographystyle{plain}   
\bibliography{../../biblios/main.bib}

\input{qm2pi.rhodetails}

\end{document}

 

% section wiring algorithm (end)

\documentclass[12pt]{llncs}
%\documentclass{jktr}

\usepackage[pdftex]{hyperref}                   
\usepackage {listings}
\usepackage {mathpartir}
\usepackage{bcprules}
%\usepackage{listings}
                       
\usepackage{graphicx} 
%\usepackage[margins=2.5cm,nohead,nofoot]{geometry}
%\usepackage{geometry}
\usepackage{amsfonts}
\usepackage{amstext}
\usepackage{latexsym}
\usepackage{amssymb}
\usepackage{color}


%\include{myPreamble}
\include{qm2pi.local} 

%\ifpdf
%\usepackage[pdftex]{graphicx}
%\else
%\usepackage{graphicx}
%\fi

 % \ifpdf
%  \usepackage{pdfsync}
%  \if


%\title{Brief Article}
%\author{David F. Snyder}
%\author{L.G. Meredith}

%\address{Dept. of Math., Texas State University--San Marcos, San Marcos, TX 78666}
       
\pagestyle{empty}


\begin{document}

\lstset{language=[Objective]Caml,frame=shadowbox}

\input{qm2pi.front}

% section front matter (end)

\input{qm2pi.intro} 
 
% section introduction (end)

% \input{qm2pi.knotations} 

% section notation (end)

\input{qm2pi.process.calculi} 

% section concurrent_process_calculi_and_spatial_logics_ (end)
    
%\input{qm2pi.knots2pi} 

%\input{qm2pi.trefoil} 

%\input{qm2pi.mainthm} 

% subsection basic_interpretation (end)

%\input{qm2pi.rho.presentation} 
\subsection{The syntax and semantics of the notation system}\label{sub:the_syntax_and_semantics_of_the_notation_system} % (fold)

We now summarize a technical presentation of the calculus that
embodies our theory of dynamics. The typical presentation of such a
calculus follows the style of giving generators and relations on
them. The grammar, below, describing term constructors, freely
generates the set of processes, $\Proc$. This set is then quotiented
by a relation known as structural congruence and it is over this set
that the notion of dynamics is expressed. This presentation is
essentially that of \cite{MeredithR05} with the addition of
polyadicity and summation. For readability we have relegated some of
the technical subtleties to an appendix.

\subsubsection{Process grammar}\label{subsub:process_grammar}

\begin{mathpar}
  \inferrule* [lab=synchronization] {} {{M} \bc \pzero \;|\; x?F \;|\; x!C }
  \and
  \inferrule* [lab=abstraction] {} {{F} \bc (x)P}
  \and
  \inferrule* [lab=concretion] {} {{C} \bc \langle Q \rangle}
  \and
  \inferrule* [lab=process] {} {{P,Q} \bc M \;| \;P|Q \;|\; @{x}}
  \and
  \inferrule* [lab=name] {} {{x} \bc \quotep{P}}
\end{mathpar} 

Note that $\vec{x}$ (resp. $\vec{P}$) denotes a vector of names
(resp. processes) of length $|\vec{x}|$ (resp. $|\vec{P}|$). We adopt
the following useful abbreviations.

\begin{mathpar}
   x?(\vec{y}).P := x.(\vec{y})P \and  x\clift{\vec{P}} := x.\clift{\vec{P}}
   \and x!(y) := \lift{x}{\dropn{y}}
   \and \Pi_{i=0}^{n-1}P_i := P_0 | \ldots | P_{n-1}
\end{mathpar}

\subsubsection{Structural congruence}

\paragraph{Free and bound names and alpha-equivalence.} At the
core of structural equivalence is alpha-equivalence which identifies
process that are the same up to a change of variable. Formally, we
recognize the distinction between free and bound names. The free names
of a process, $\freenames{P}$, may be calculated recursively as
follows:

\begin{mathpar}
\freenames{\pzero} := \emptyset
  \and \\
  \freenames{x?(y).P} := \{ x \} \cup (\freenames{P} \setminus \{ y \})
  \and 
  \freenames{x!\langle P \rangle} := \{ x \} \cup \{ P \} 
  \and \\
  \freenames{P|Q} := \freenames{P} \cup \freenames{Q}
  \and \\
  \freenames{@{x}} := \{ x \}
\end{mathpar}

$\pi$
$\quotep{\pi}$

$\freenames{-} : \pi \to \mathcal{P}(\quotep{\pi})$

\begin{eqnarray*}
  \freenames{\pzero} & := & \emptyset \\
  \freenames{x?(y).P} & := & \{ x \} \cup (\freenames{P} \setminus \{ y \}) \\
  \freenames{x!\langle P \rangle} & := & \{ x \} \cup \{ P \} \\
  \freenames{P|Q} & := & \freenames{P} \cup \freenames{Q} \\
  \freenames{\dropn{x}} & := & \{ x \}
\end{eqnarray*}

The bound names of a process, $\boundnames{P}$, are those names occurring in $P$
that are not free. For example, in $x?(y).0$, the name $x$ is free, while $y$ is bound.

\begin{mathpar}
  \inferrule* [lab=monoidal-laws] {} { P|Q \equiv Q|P \and P|0 \equiv P \and P|(Q|R) \equiv (P|Q)|R }
\end{mathpar}

\begin{mathpar}
  \inferrule* [lab=alpha-equivalence] {} { (x)P \equiv (y)P\{y/x\} \and y \not\in \freenames{P} }
\end{mathpar}

\begin{definition}
Then two processes, $P,Q$, are alpha-equivalent if $P = Q\{\vec{y}/\vec{x}\}$ for
some $\vec{x} \in \boundnames{Q},\vec{y} \in \boundnames{P}$, where $Q\{\vec{y}/\vec{x}\}$
denotes the capture-avoiding substitution of $\vec{y}$ for $\vec{x}$ in $Q$.
\end{definition}

\begin{definition}
  The {\em structural congruence} \cite{SangiorgiWalker} , $\equiv$,
  between processes is the least congruence containing
  alpha-equivalence, satisfying the abelian monoid laws
  (associativity, commutativity and $\pzero$ as identity) for parallel
  composition $|$ and for summation $+$.
\end{definition}

\subsection{Name equivalence}

We take name equivalence, written $\nameeq$, to be the smallest
equivalence relation generated by the following rules.

\begin{mathpar}
\inferrule*[lab=Quote-drop]
{ }
{ \quotep{@{x}} \nameeq x }

\inferrule*[lab=Struct-equiv]
{ P \scong Q }
{ \quotep{P} \nameeq \quotep{Q} }
\end{mathpar}

The astute reader will have noticed that the mutual recursion of names
and processes imposes a mutual recursion on alpha-equivalence and
structural equivalence via name-equivalence. Fortunately, all of this
works out pleasantly and we may calculate in the natural way, free of
concern. The reader interested in the details is referred to the
appendix \ref{appendix:rho_details}.

\subsection{Substitution}

We use $\Proc$ for the set of processes, $\QProc$ for the set of
names, and $\id{\{}\vec{y} / \vec{x} \id{\}}$ to denote partial maps,
$s : \QProc \rightarrow \QProc$. A map, $s$ lifts, uniquely, to a map
on process terms, $\widehat{s} : \Proc \rightarrow \Proc$ by the
following equations.

\begin{mathpar}
  (0) \psubstp{Q}{P} := 0 \\
  (R \juxtap S) \psubstp{Q}{P}
  :=    
  (R)\psubstp{Q}{P} \juxtap (S) \psubstp{Q}{P} \\
  (x?(y).R) \psubstp{Q}{P}    
  :=    
  (x)\substp{Q}{P} (z)\concat( (R \psubstn{z}{y}) \psubstp{Q}{P} ) \\
  (\lift{x}{R}) \psubstp{Q}{P}  
  :=
  \lift{(x)\substp{Q}{P}}{ R \psubstp{Q}{P} } \\
%   (\dropn{x})  \psubstp{Q}{P}       
%   := 
%   \left\{ 
%     \begin{array}{ccc} 
%       \dropn{\quotep{Q}} & & x \nameeq \quotep{P} \\
%       \dropn{x} & & otherwise \\
%     \end{array}
%   \right. 
  (\dropn{x})  \psubstp{Q}{P}       
  := 
  \left\{ 
    \begin{array}{ccc} 
      Q & & x \nameeq \quotep{P} \\
      \dropn{x} & & otherwise \\
    \end{array}
  \right.
\end{mathpar}
 

where

\begin{eqnarray}
  (x)\id{\{} \lpquote Q \rpquote / \lpquote P \rpquote \id{\}}            = 
  \left\{ 
    \begin{array}{ccc}
      \lpquote Q \rpquote & & x \nameeq \lpquote P \rpquote \\
      x & & otherwise \\
    \end{array}
  \right. \nonumber
\end{eqnarray}

and $z$ is chosen distinct from $\quotep{P}$, $\quotep{Q}$, the free
names in $Q$, and all the names in $R$. Our $\alpha$-equivalence will
be built in the standard way from this substitution.

\begin{remark}\label{rem:no_self_referential_names}
  One consequence of these definitions is that $\forall P. \quotep{P}
  \not\in \freenames{P}$.
\end{remark}

\subsection{ Dynamic quote: an example }

Anticipating something of what's to come, consider applying the
substitution, $\widehat{\id{\{}u / z \id{\}}}$, to the following pair
of processes, $\lift{w}{y!(z)}$ and $w[ \lpquote y!(z) \rpquote ]$.

\begin{eqnarray}
	\lift{w}{y!(z)}\widehat{\id{\{}u / z \id{\}}}
		& = &
		\lift{w}{y!(u)} \nonumber\\
	w[ \lpquote y!(z) \rpquote ] \widehat{ \id{\{}u / z \id{\}} }
		& = &
		w[ \lpquote y!(z) \rpquote ] \nonumber
\end{eqnarray}

Because the body of the process between quotes is impervious to
substitution, we get radically different answers. In fact, by
examining the first process in an input context,
e.g. $x?(z).\lift{w}{y!(z)}$, we see that the process under the lift
operator may be shaped by prefixed inputs binding a name inside it. In
this sense, the lift operator will be seen as a way to dynamically
construct processes before reifying them as names.

Finally equipped with these standard features we can present the
dynamics of the calculus.

\subsubsection{Operational semantics} 

Finally, we introduce the computational dynamics. What marks these
algebras as distinct from other more traditionally studied algebraic
structures, e.g. vector spaces or polynomial rings, is the manner in
which dynamics is captured. In traditional structures, dynamics is typically
expressed through morphisms between such structures, as in linear maps
between vector spaces or morphisms between rings. In algebras
associated with the semantics of computation, the dynamics is
expressed as part of the algebraic structure itself, through a
reduction reduction relation typically denoted by $\red$. Below, we
give a recursive presentation of this relation for the calculus used
in the encoding.

$\red \subseteq \pi \times \pi$
$\red : \pi \to \mathcal{P}(\pi)$

\begin{mathpar}
  \inferrule* [lab=Comm] { \textsf{match}( x_{src}, x_{trgt} ) } { x_{trgt}?(y)P \; | \; x_{src}!\langle {Q} \rangle \red P\{\quotep{Q}/y}\} }
  \and \\
  \inferrule* [lab=Par] {{P} \red {P}'} {{{P} | {Q}} \red {{P}' | {Q}}}
  \and
  \inferrule* [lab=Equiv]{{{P} \scong {P}'} \andalso {{P}' \red {Q}'} \andalso {{Q}' \scong {Q}}}{{P} \red {Q}}
\end{mathpar}

\begin{eqnarray*}
  match_{\equiv} (\quotep{P},\quotep{Q}) & := & P \equiv Q \\
  match_{\dagger}(\quotep{P},\quotep{Q}) & := & \forall R. P|Q \red^{*} R => R \red^{*} 0 \\
  match_{K}(\quotep{P},\quotep{Q}) & := & K \mbox{ for some context } K
\end{eqnarray*}

$u?(x)P | u!\langle Q \rangle \red P\{\quotep{Q}/x\}$

%We write $\wred$ for $\red^*$, and $P\red$ if $\exists Q $ such that $ P \red Q$.
We write $P\red$ if $\exists Q $ such that $ P \red Q$ and $P\not\red$, otherwise.

\section{Replication}

As mentioned before, it is known that replication (and hence
recursion) can be implemented in a higher-order process algebra
\cite{SangiorgiWalker}. As our first example of calculation with the
machinery thus far presented we give the construction explicitly in
the {\rhoc}.

\begin{eqnarray}
	D_{x} & := & \prefix{x}{y}{(\binpar{\outputp{x}{y}}{@{y}})} \nonumber\\
	\bangp_{x}{P} & := & \binpar{{x}!\langle{\binpar{D_{x}}{P}}\rangle}{D_{x}} \nonumber
\end{eqnarray}

\begin{eqnarray}
	\bangp_{x}{P} & & \nonumber\\
	=
	& {x}!\langle{(\prefix{x}{y}{(\outputp{x}{y} | @{y})) | P}}\rangle 
	      | \prefix{x}{y}{(\outputp{x}{y} | @{y})} & \nonumber\\
	\red
	& (\outputp{x}{y} | @{y})\substn{\quotep{(\prefix{x}{y}{(@{y} | \outputp{x}{y})) | P}}}{y} & \nonumber\\
	=
	& \outputp{x}{\quotep{(\prefix{x}{y}{(\outputp{x}{y} | @{y})) | P}}}
	  | {(\prefix{x}{y}{(\outputp{x}{y} | @{y})) | P}} & \nonumber\\
	\red
	& \ldots & \nonumber\\
	\red^*
	& P | P | \ldots & \nonumber
\end{eqnarray}

Of course, this encoding, as an implementation, runs away, unfolding
$\bangp{P}$ eagerly. A lazier and more implementable replication
operator, restricted to input-guarded processes, may be obtained as follows.

\begin{eqnarray}
\bangp{\prefix{u}{v}{P}} 
	:= 
	\binpar{\lift{x}{\prefix{u}{v}{(\binpar{D(x)}{P})}}}{D(x)} \nonumber
\end{eqnarray}

\begin{remark}
  Note that the lazier definition still does not deal with summation
  or mixed summation (i.e. sums over input and output). The reader is
  invited to construct definitions of replication that deal with these
  features. 

  Further, the definitions are parameterized in a name, $x$. Can you,
  gentle reader, make a definition that eliminates this parameter and
  guarantees no accidental interaction between the replication
  machinery and the process being replicated -- i.e. no accidental
  sharing of names used by the process to get its work done and the
  name(s) used by the replication to effect copying. This latter
  revision of the definition of replication is crucial to obtaining
  the expected identity $!!P \sim !P$.
\end{remark}

\begin{remark}\label{rem:paradoxical_combinator}
  The reader familiar with the lambda calculus will have noticed the
  similarity between $D$ and the paradoxical combinator.

  [Ed. note: the existence of this seems to suggest we have to be more
  restrictive on the set of processes and names we admit if we are to
  support no-cloning.]
\end{remark}

\subsubsection{Bisimulation}

The computational dynamics gives rise to another kind of equivalence,
the equivalence of computational behavior. As previously mentioned
this is typically captured \emph{via} some form of bisimulation.

% The notion we use in this paper is weak barbed bisimulation
% \cite{milner91polyadicpi}.

The notion we use in this paper is derived from weak barbed
bisimulation \cite{milner91polyadicpi}. 

\begin{definition}
An \emph{observation relation}, $\downarrow_{\mathcal N}$, over a set
of names, $\mathcal N$, is the smallest relation satisfying the rules
below.

\infrule[Out-barb]{y \in {\mathcal N}, \; x \nameeq y}
		  {\outputp{x}{v} \downarrow_{\mathcal N} x}
\infrule[Par-barb]{\mbox{$P\downarrow_{\mathcal N} x$ or $Q\downarrow_{\mathcal N} x$}}
		  {\binpar{P}{Q} \downarrow_{\mathcal N} x}

We write $P \Downarrow_{\mathcal N} x$ if there is $Q$ such that 
$P \wred Q$ and $Q \downarrow_{\mathcal N} x$.
\end{definition}

\begin{definition}
%\label{def.bbisim}
An  ${\mathcal N}$-\emph{barbed bisimulation} over a set of names, ${\mathcal N}$, is a symmetric binary relation 
${\mathcal S}_{\mathcal N}$ between agents such that $P\rel{S}_{\mathcal N}Q$ implies:
\begin{enumerate}
\item If $P \red P'$ then $Q \wred Q'$ and $P'\rel{S}_{\mathcal N} Q'$.
\item If $P\downarrow_{\mathcal N} x$, then $Q\Downarrow_{\mathcal N} x$.
\end{enumerate}
$P$ is ${\mathcal N}$-barbed bisimilar to $Q$, written
$P \wbbisim_{\mathcal N} Q$, if $P \rel{S}_{\mathcal N} Q$ for some ${\mathcal N}$-barbed bisimulation ${\mathcal S}_{\mathcal N}$.
\end{definition}

$\mathcal{R} \subseteq \pi \times \pi$

$P \mathcal{R} Q => \forall P'. P \red P' \Rightarrow \exists Q'. Q \red Q', P' \mathcal{R} Q'$

$P \vdash x \Rightarrow Q \vdash x$

\begin{mathpar}
  \inferrule*[lab=Out-barb]{x \nameeq y}{{y}!\langle{Q}\rangle \vdash x}
  \and
  \inferrule*[lab=Par-barb]{\mbox{$P\vdash x$ or $Q\vdash x$}}{\binpar{P}{Q} \vdash x}
\end{mathpar}

\subsubsection{Contexts}

One of the principle advantages of computational calculi like the
$\pi$-calculus is a well-defined notion of context,
contextual-equivalence and a correlation between
contextual-equivalence and notions of bisimulation. The notion of
context allows the decomposition of a process into (sub-)process and
its syntactic environment, its context. Thus, a context may be
thought of as a process with a ``hole'' (written $\Box$) in it. The
application of a context $M$ to a process $P$, written $M[P]$, is
tantamount to filling the hole in $M$ with $P$. In this paper we do
not need the full weight of this theory, but do make use of the notion
of context in the proof the main theorem. 

\begin{mathpar}
  \inferrule* [lab=summation] {} {{M_{M},M_{N}} \bc \Box \;|\; x.M_{A} \;|\; M_{M}+M_{N}}
  \and
  \inferrule* [lab=agent] {} {{M_{A}} \bc (\vec{x})M_{P} \;| \; \clift{P_0,\ldots,M_{P},\ldots,P_N}}
  \and \\
  \inferrule* [lab=process] {} {{M_{P}} \bc M_{N} \;| \;P|M_{P} }
\end{mathpar} 

\begin{mathpar}
  \inferrule* [lab=sychronization] {} {M_{N} \bc \Box \;|\; x?M_{F} \;|\; x!M_{C}}
  \and
  \inferrule* [lab=abstraction] {} {{M_{F}} \bc (x)M_{P} }
  \and
  \inferrule* [lab=concretion] {} {{M_{C}} \bc \langle M_{P} \rangle }
  \and \\
  \inferrule* [lab=process] {} {{M_{P}} \bc M_{N} \;| \;P|M_{P} }
\end{mathpar}

\begin{definition}[contextual application] Given a context $M$, and
  process $P$, we define the \emph{contextual application}, $M[P] :=
  M\{P/\Box\}$. That is, the contextual application of M to P is the
  substitution of $P$ for $\Box$ in $M$.
\end{definition}

$\meaningof{-} : L \to \mathcal{P}(\pi)$

\begin{mathpar}
  \inferrule* [lab=collection] {} {\meaningof{true} = \pi, \and \meaningof{~E} = \pi \setminus \meaningof{E}, \and \meaningof{E_{1} \& E_{2}} = \meaningof{E_{1}} \cap \meaningof{E_{2}}}
\end{mathpar}

\begin{mathpar}
  \inferrule* [lab=structure] {} {\meaningof{0} = \{ P \in \pi | P \equiv 0 \}, \and \\ \meaningof{E_1 | E_2} = \{ P \in \pi | P \equiv P_{1} | P_{2}, P_{1} \in \meaningof{E_{1}}, P_{2} \in \meaningof{E_2}\} }
\end{mathpar}

\begin{mathpar}
 \inferrule* [lab=behavior] {} {\meaningof{\langle a?b \rangle E} = \{ P \in \pi | P \equiv Q | u?(y)P', \\ \and \\\\ \and \\ \;\;\; u \in \meaningof{a}, \forall z.P'\{z/y\} \in \meaningof{E\{z/b\}}\}, \and \\ \meaningof{a!E} = \{ P \in \pi | P \equiv Q | x!\langle P' \rangle, x \in \meaningof{a} P' \in \meaningof{E}\} }
\end{mathpar}

\begin{mathpar}
 \inferrule* [lab=nominal] {} {\meaningof{\quotep{E}} = \{ \quotep{P} \in \quotep{\pi} | P \in \meaningof{E} \}, \and \meaningof{\quotep{P}} = \{ \quotep{Q} \in \quotep{\pi} | P \equiv Q \} \and \\ \meaningof{@\quotep{E}} = \{ P \in \pi | P \equiv @x, x \in \meaningof{E} \}}
\end{mathpar}

\begin{eqnarray*}
  \\
  \meaningof{-} : TS \to ST
\end{eqnarray*}

\begin{eqnarray*}
  \\
  L : TS \to ST
\end{eqnarray*}

\begin{eqnarray*}
  \\
  P \models E \iff P \in \meaningof{E}
\end{eqnarray*}

\begin{eqnarray*}
  P \approx_{L} Q \iff \forall E \in L. P \models E \iff Q \models E
\end{eqnarray*}

\begin{eqnarray*}
  P \approx_{K} Q
\end{eqnarray*}

\begin{eqnarray*}
  P \approx Q
\end{eqnarray*}

$\approx_{K} = \approx = \approx_{L}$

\subsubsection{Contextual duality}

Note that contexts extend the quotation operation to a family of
operations from processes to names. Given a context, $M$, we can
define a \emph{nominal context}, $\quotep{M}$ by $\quotep{M}[P] :=
\quotep{M[P]}$. To foreshadow what is to come we observe that these
operations enjoy a duality with processes very much like the duality
between vectors and maps from vectors to scalars.

Further, because the calculus is essentially higher-order, we have a
correspondence between contexts and processes. More specifically,
given a name $x$ and a context $M$ we can construct $M^{*}_{x}$ such
that 

\begin{mathpar}
  M^{*}_{x} | \lift{x}{P} \red M[P]
\end{mathpar}

namely,

\begin{mathpar}
  M^{*}_{x} := x?(u).M[\dropn{u}]
\end{mathpar}

The dependence of $M^{*}_{x}$ on a name makes it an abstraction, 

\begin{mathpar}
  M^{*} := (x)x?(u).M[\dropn{u}]
\end{mathpar}

\subsection{Additional notation}

It will sometimes be convenient to denote the process a name
quotes. We already have the notation $x = \quotep{P}$, but it will be
convenient to introduce an alternate notation, $\procn{x}$, when we
want to emphasize the connection to the use of the name. Note that, by
virtue of name equivalence, $\quotep{\procn{x}} \nameeq x$; so, the
notation is consistent with previous definitions.

Further, because names have structure it is possible to effect
substitutions on the basis of that structure. This means we need to
upgrade our notation for substitutions, which we accomplish by
adapting comprehension notation. Thus,

\begin{mathpar}
  P\{ y / x : x \in S \}
\end{mathpar}

is interpreted to mean the process derived from P by replacing (in a
capture-avoiding manner) each occurrence of $x$ in $S$ by $y$. For example,

\begin{mathpar}
  P\{ \quotep{\procn{x}|\procn{x}} / x : x \in \freenames{P} \}
\end{mathpar}

will replace each (occurrence) of a free name $x$ in $P$ by
$\quotep{\procn{x}|\procn{x}}$.

Also, we will avail ourselves of the notation $x^{L}$ and $x^{R}$ to
denote injections of a name into disjoint copies of the name
space. There are numerous ways to accomplish this. One example can be
found in \cite{MeredithR05}. This notation overloads to vectors of
names: $\vec{x}^{\pi} := (x_{i}^{\pi} \; : \; 0 \leq i < |\vec{x}| )$ where $\pi \in \{L,R\}$.

We also use $P^{\Box} := P|\Box$.

In \cite{MeredithR05} an interpretation of the new operator is
given. It turns out that there are several possible interpretations
all enjoying the requisite algebraic properties of the operator (see
\cite{milner91polyadicpi}). We will therefore make liberal use of
$(\nu\; \vec{x})P$.

% subsection the_syntax_and_semantics_of_the_notation_system (end)   

\input{qm2pi.qmops} 

\input{qm2pi.sterngerlach} 

\input{qm2pi.metric} 

% section concurrent_process_calculi (end)

%\input{qm2pi.proofsketch}

% section proof sketch (end)

%\input{qm2pi.slviaknots} 

% section spatial logic via knots (end)

\input{qm2pi.conclusion}

% section conclusion (end)

%\input{qm2pi.dtcodes} 

% section wiring algorithm (end)

\input{qm2pi.ack} 

% section acknowledgments (end)

\newpage


\bibliographystyle{plain}   
\bibliography{../../biblios/main.bib}

\input{qm2pi.rhodetails}

\end{document}

 

% section acknowledgments (end)

\newpage


\bibliographystyle{plain}   
\bibliography{../../biblios/main.bib}

\documentclass[12pt]{llncs}
%\documentclass{jktr}

\usepackage[pdftex]{hyperref}                   
\usepackage {listings}
\usepackage {mathpartir}
\usepackage{bcprules}
%\usepackage{listings}
                       
\usepackage{graphicx} 
%\usepackage[margins=2.5cm,nohead,nofoot]{geometry}
%\usepackage{geometry}
\usepackage{amsfonts}
\usepackage{amstext}
\usepackage{latexsym}
\usepackage{amssymb}
\usepackage{color}


%\include{myPreamble}
\include{qm2pi.local} 

%\ifpdf
%\usepackage[pdftex]{graphicx}
%\else
%\usepackage{graphicx}
%\fi

 % \ifpdf
%  \usepackage{pdfsync}
%  \if


%\title{Brief Article}
%\author{David F. Snyder}
%\author{L.G. Meredith}

%\address{Dept. of Math., Texas State University--San Marcos, San Marcos, TX 78666}
       
\pagestyle{empty}


\begin{document}

\lstset{language=[Objective]Caml,frame=shadowbox}

\input{qm2pi.front}

% section front matter (end)

\input{qm2pi.intro} 
 
% section introduction (end)

% \input{qm2pi.knotations} 

% section notation (end)

\input{qm2pi.process.calculi} 

% section concurrent_process_calculi_and_spatial_logics_ (end)
    
%\input{qm2pi.knots2pi} 

%\input{qm2pi.trefoil} 

%\input{qm2pi.mainthm} 

% subsection basic_interpretation (end)

%\input{qm2pi.rho.presentation} 
\subsection{The syntax and semantics of the notation system}\label{sub:the_syntax_and_semantics_of_the_notation_system} % (fold)

We now summarize a technical presentation of the calculus that
embodies our theory of dynamics. The typical presentation of such a
calculus follows the style of giving generators and relations on
them. The grammar, below, describing term constructors, freely
generates the set of processes, $\Proc$. This set is then quotiented
by a relation known as structural congruence and it is over this set
that the notion of dynamics is expressed. This presentation is
essentially that of \cite{MeredithR05} with the addition of
polyadicity and summation. For readability we have relegated some of
the technical subtleties to an appendix.

\subsubsection{Process grammar}\label{subsub:process_grammar}

\begin{mathpar}
  \inferrule* [lab=synchronization] {} {{M} \bc \pzero \;|\; x?F \;|\; x!C }
  \and
  \inferrule* [lab=abstraction] {} {{F} \bc (x)P}
  \and
  \inferrule* [lab=concretion] {} {{C} \bc \langle Q \rangle}
  \and
  \inferrule* [lab=process] {} {{P,Q} \bc M \;| \;P|Q \;|\; @{x}}
  \and
  \inferrule* [lab=name] {} {{x} \bc \quotep{P}}
\end{mathpar} 

Note that $\vec{x}$ (resp. $\vec{P}$) denotes a vector of names
(resp. processes) of length $|\vec{x}|$ (resp. $|\vec{P}|$). We adopt
the following useful abbreviations.

\begin{mathpar}
   x?(\vec{y}).P := x.(\vec{y})P \and  x\clift{\vec{P}} := x.\clift{\vec{P}}
   \and x!(y) := \lift{x}{\dropn{y}}
   \and \Pi_{i=0}^{n-1}P_i := P_0 | \ldots | P_{n-1}
\end{mathpar}

\subsubsection{Structural congruence}

\paragraph{Free and bound names and alpha-equivalence.} At the
core of structural equivalence is alpha-equivalence which identifies
process that are the same up to a change of variable. Formally, we
recognize the distinction between free and bound names. The free names
of a process, $\freenames{P}$, may be calculated recursively as
follows:

\begin{mathpar}
\freenames{\pzero} := \emptyset
  \and \\
  \freenames{x?(y).P} := \{ x \} \cup (\freenames{P} \setminus \{ y \})
  \and 
  \freenames{x!\langle P \rangle} := \{ x \} \cup \{ P \} 
  \and \\
  \freenames{P|Q} := \freenames{P} \cup \freenames{Q}
  \and \\
  \freenames{@{x}} := \{ x \}
\end{mathpar}

$\pi$
$\quotep{\pi}$

$\freenames{-} : \pi \to \mathcal{P}(\quotep{\pi})$

\begin{eqnarray*}
  \freenames{\pzero} & := & \emptyset \\
  \freenames{x?(y).P} & := & \{ x \} \cup (\freenames{P} \setminus \{ y \}) \\
  \freenames{x!\langle P \rangle} & := & \{ x \} \cup \{ P \} \\
  \freenames{P|Q} & := & \freenames{P} \cup \freenames{Q} \\
  \freenames{\dropn{x}} & := & \{ x \}
\end{eqnarray*}

The bound names of a process, $\boundnames{P}$, are those names occurring in $P$
that are not free. For example, in $x?(y).0$, the name $x$ is free, while $y$ is bound.

\begin{mathpar}
  \inferrule* [lab=monoidal-laws] {} { P|Q \equiv Q|P \and P|0 \equiv P \and P|(Q|R) \equiv (P|Q)|R }
\end{mathpar}

\begin{mathpar}
  \inferrule* [lab=alpha-equivalence] {} { (x)P \equiv (y)P\{y/x\} \and y \not\in \freenames{P} }
\end{mathpar}

\begin{definition}
Then two processes, $P,Q$, are alpha-equivalent if $P = Q\{\vec{y}/\vec{x}\}$ for
some $\vec{x} \in \boundnames{Q},\vec{y} \in \boundnames{P}$, where $Q\{\vec{y}/\vec{x}\}$
denotes the capture-avoiding substitution of $\vec{y}$ for $\vec{x}$ in $Q$.
\end{definition}

\begin{definition}
  The {\em structural congruence} \cite{SangiorgiWalker} , $\equiv$,
  between processes is the least congruence containing
  alpha-equivalence, satisfying the abelian monoid laws
  (associativity, commutativity and $\pzero$ as identity) for parallel
  composition $|$ and for summation $+$.
\end{definition}

\subsection{Name equivalence}

We take name equivalence, written $\nameeq$, to be the smallest
equivalence relation generated by the following rules.

\begin{mathpar}
\inferrule*[lab=Quote-drop]
{ }
{ \quotep{@{x}} \nameeq x }

\inferrule*[lab=Struct-equiv]
{ P \scong Q }
{ \quotep{P} \nameeq \quotep{Q} }
\end{mathpar}

The astute reader will have noticed that the mutual recursion of names
and processes imposes a mutual recursion on alpha-equivalence and
structural equivalence via name-equivalence. Fortunately, all of this
works out pleasantly and we may calculate in the natural way, free of
concern. The reader interested in the details is referred to the
appendix \ref{appendix:rho_details}.

\subsection{Substitution}

We use $\Proc$ for the set of processes, $\QProc$ for the set of
names, and $\id{\{}\vec{y} / \vec{x} \id{\}}$ to denote partial maps,
$s : \QProc \rightarrow \QProc$. A map, $s$ lifts, uniquely, to a map
on process terms, $\widehat{s} : \Proc \rightarrow \Proc$ by the
following equations.

\begin{mathpar}
  (0) \psubstp{Q}{P} := 0 \\
  (R \juxtap S) \psubstp{Q}{P}
  :=    
  (R)\psubstp{Q}{P} \juxtap (S) \psubstp{Q}{P} \\
  (x?(y).R) \psubstp{Q}{P}    
  :=    
  (x)\substp{Q}{P} (z)\concat( (R \psubstn{z}{y}) \psubstp{Q}{P} ) \\
  (\lift{x}{R}) \psubstp{Q}{P}  
  :=
  \lift{(x)\substp{Q}{P}}{ R \psubstp{Q}{P} } \\
%   (\dropn{x})  \psubstp{Q}{P}       
%   := 
%   \left\{ 
%     \begin{array}{ccc} 
%       \dropn{\quotep{Q}} & & x \nameeq \quotep{P} \\
%       \dropn{x} & & otherwise \\
%     \end{array}
%   \right. 
  (\dropn{x})  \psubstp{Q}{P}       
  := 
  \left\{ 
    \begin{array}{ccc} 
      Q & & x \nameeq \quotep{P} \\
      \dropn{x} & & otherwise \\
    \end{array}
  \right.
\end{mathpar}
 

where

\begin{eqnarray}
  (x)\id{\{} \lpquote Q \rpquote / \lpquote P \rpquote \id{\}}            = 
  \left\{ 
    \begin{array}{ccc}
      \lpquote Q \rpquote & & x \nameeq \lpquote P \rpquote \\
      x & & otherwise \\
    \end{array}
  \right. \nonumber
\end{eqnarray}

and $z$ is chosen distinct from $\quotep{P}$, $\quotep{Q}$, the free
names in $Q$, and all the names in $R$. Our $\alpha$-equivalence will
be built in the standard way from this substitution.

\begin{remark}\label{rem:no_self_referential_names}
  One consequence of these definitions is that $\forall P. \quotep{P}
  \not\in \freenames{P}$.
\end{remark}

\subsection{ Dynamic quote: an example }

Anticipating something of what's to come, consider applying the
substitution, $\widehat{\id{\{}u / z \id{\}}}$, to the following pair
of processes, $\lift{w}{y!(z)}$ and $w[ \lpquote y!(z) \rpquote ]$.

\begin{eqnarray}
	\lift{w}{y!(z)}\widehat{\id{\{}u / z \id{\}}}
		& = &
		\lift{w}{y!(u)} \nonumber\\
	w[ \lpquote y!(z) \rpquote ] \widehat{ \id{\{}u / z \id{\}} }
		& = &
		w[ \lpquote y!(z) \rpquote ] \nonumber
\end{eqnarray}

Because the body of the process between quotes is impervious to
substitution, we get radically different answers. In fact, by
examining the first process in an input context,
e.g. $x?(z).\lift{w}{y!(z)}$, we see that the process under the lift
operator may be shaped by prefixed inputs binding a name inside it. In
this sense, the lift operator will be seen as a way to dynamically
construct processes before reifying them as names.

Finally equipped with these standard features we can present the
dynamics of the calculus.

\subsubsection{Operational semantics} 

Finally, we introduce the computational dynamics. What marks these
algebras as distinct from other more traditionally studied algebraic
structures, e.g. vector spaces or polynomial rings, is the manner in
which dynamics is captured. In traditional structures, dynamics is typically
expressed through morphisms between such structures, as in linear maps
between vector spaces or morphisms between rings. In algebras
associated with the semantics of computation, the dynamics is
expressed as part of the algebraic structure itself, through a
reduction reduction relation typically denoted by $\red$. Below, we
give a recursive presentation of this relation for the calculus used
in the encoding.

$\red \subseteq \pi \times \pi$
$\red : \pi \to \mathcal{P}(\pi)$

\begin{mathpar}
  \inferrule* [lab=Comm] { \textsf{match}( x_{src}, x_{trgt} ) } { x_{trgt}?(y)P \; | \; x_{src}!\langle {Q} \rangle \red P\{\quotep{Q}/y}\} }
  \and \\
  \inferrule* [lab=Par] {{P} \red {P}'} {{{P} | {Q}} \red {{P}' | {Q}}}
  \and
  \inferrule* [lab=Equiv]{{{P} \scong {P}'} \andalso {{P}' \red {Q}'} \andalso {{Q}' \scong {Q}}}{{P} \red {Q}}
\end{mathpar}

\begin{eqnarray*}
  match_{\equiv} (\quotep{P},\quotep{Q}) & := & P \equiv Q \\
  match_{\dagger}(\quotep{P},\quotep{Q}) & := & \forall R. P|Q \red^{*} R => R \red^{*} 0 \\
  match_{K}(\quotep{P},\quotep{Q}) & := & K \mbox{ for some context } K
\end{eqnarray*}

$u?(x)P | u!\langle Q \rangle \red P\{\quotep{Q}/x\}$

%We write $\wred$ for $\red^*$, and $P\red$ if $\exists Q $ such that $ P \red Q$.
We write $P\red$ if $\exists Q $ such that $ P \red Q$ and $P\not\red$, otherwise.

\section{Replication}

As mentioned before, it is known that replication (and hence
recursion) can be implemented in a higher-order process algebra
\cite{SangiorgiWalker}. As our first example of calculation with the
machinery thus far presented we give the construction explicitly in
the {\rhoc}.

\begin{eqnarray}
	D_{x} & := & \prefix{x}{y}{(\binpar{\outputp{x}{y}}{@{y}})} \nonumber\\
	\bangp_{x}{P} & := & \binpar{{x}!\langle{\binpar{D_{x}}{P}}\rangle}{D_{x}} \nonumber
\end{eqnarray}

\begin{eqnarray}
	\bangp_{x}{P} & & \nonumber\\
	=
	& {x}!\langle{(\prefix{x}{y}{(\outputp{x}{y} | @{y})) | P}}\rangle 
	      | \prefix{x}{y}{(\outputp{x}{y} | @{y})} & \nonumber\\
	\red
	& (\outputp{x}{y} | @{y})\substn{\quotep{(\prefix{x}{y}{(@{y} | \outputp{x}{y})) | P}}}{y} & \nonumber\\
	=
	& \outputp{x}{\quotep{(\prefix{x}{y}{(\outputp{x}{y} | @{y})) | P}}}
	  | {(\prefix{x}{y}{(\outputp{x}{y} | @{y})) | P}} & \nonumber\\
	\red
	& \ldots & \nonumber\\
	\red^*
	& P | P | \ldots & \nonumber
\end{eqnarray}

Of course, this encoding, as an implementation, runs away, unfolding
$\bangp{P}$ eagerly. A lazier and more implementable replication
operator, restricted to input-guarded processes, may be obtained as follows.

\begin{eqnarray}
\bangp{\prefix{u}{v}{P}} 
	:= 
	\binpar{\lift{x}{\prefix{u}{v}{(\binpar{D(x)}{P})}}}{D(x)} \nonumber
\end{eqnarray}

\begin{remark}
  Note that the lazier definition still does not deal with summation
  or mixed summation (i.e. sums over input and output). The reader is
  invited to construct definitions of replication that deal with these
  features. 

  Further, the definitions are parameterized in a name, $x$. Can you,
  gentle reader, make a definition that eliminates this parameter and
  guarantees no accidental interaction between the replication
  machinery and the process being replicated -- i.e. no accidental
  sharing of names used by the process to get its work done and the
  name(s) used by the replication to effect copying. This latter
  revision of the definition of replication is crucial to obtaining
  the expected identity $!!P \sim !P$.
\end{remark}

\begin{remark}\label{rem:paradoxical_combinator}
  The reader familiar with the lambda calculus will have noticed the
  similarity between $D$ and the paradoxical combinator.

  [Ed. note: the existence of this seems to suggest we have to be more
  restrictive on the set of processes and names we admit if we are to
  support no-cloning.]
\end{remark}

\subsubsection{Bisimulation}

The computational dynamics gives rise to another kind of equivalence,
the equivalence of computational behavior. As previously mentioned
this is typically captured \emph{via} some form of bisimulation.

% The notion we use in this paper is weak barbed bisimulation
% \cite{milner91polyadicpi}.

The notion we use in this paper is derived from weak barbed
bisimulation \cite{milner91polyadicpi}. 

\begin{definition}
An \emph{observation relation}, $\downarrow_{\mathcal N}$, over a set
of names, $\mathcal N$, is the smallest relation satisfying the rules
below.

\infrule[Out-barb]{y \in {\mathcal N}, \; x \nameeq y}
		  {\outputp{x}{v} \downarrow_{\mathcal N} x}
\infrule[Par-barb]{\mbox{$P\downarrow_{\mathcal N} x$ or $Q\downarrow_{\mathcal N} x$}}
		  {\binpar{P}{Q} \downarrow_{\mathcal N} x}

We write $P \Downarrow_{\mathcal N} x$ if there is $Q$ such that 
$P \wred Q$ and $Q \downarrow_{\mathcal N} x$.
\end{definition}

\begin{definition}
%\label{def.bbisim}
An  ${\mathcal N}$-\emph{barbed bisimulation} over a set of names, ${\mathcal N}$, is a symmetric binary relation 
${\mathcal S}_{\mathcal N}$ between agents such that $P\rel{S}_{\mathcal N}Q$ implies:
\begin{enumerate}
\item If $P \red P'$ then $Q \wred Q'$ and $P'\rel{S}_{\mathcal N} Q'$.
\item If $P\downarrow_{\mathcal N} x$, then $Q\Downarrow_{\mathcal N} x$.
\end{enumerate}
$P$ is ${\mathcal N}$-barbed bisimilar to $Q$, written
$P \wbbisim_{\mathcal N} Q$, if $P \rel{S}_{\mathcal N} Q$ for some ${\mathcal N}$-barbed bisimulation ${\mathcal S}_{\mathcal N}$.
\end{definition}

$\mathcal{R} \subseteq \pi \times \pi$

$P \mathcal{R} Q => \forall P'. P \red P' \Rightarrow \exists Q'. Q \red Q', P' \mathcal{R} Q'$

$P \vdash x \Rightarrow Q \vdash x$

\begin{mathpar}
  \inferrule*[lab=Out-barb]{x \nameeq y}{{y}!\langle{Q}\rangle \vdash x}
  \and
  \inferrule*[lab=Par-barb]{\mbox{$P\vdash x$ or $Q\vdash x$}}{\binpar{P}{Q} \vdash x}
\end{mathpar}

\subsubsection{Contexts}

One of the principle advantages of computational calculi like the
$\pi$-calculus is a well-defined notion of context,
contextual-equivalence and a correlation between
contextual-equivalence and notions of bisimulation. The notion of
context allows the decomposition of a process into (sub-)process and
its syntactic environment, its context. Thus, a context may be
thought of as a process with a ``hole'' (written $\Box$) in it. The
application of a context $M$ to a process $P$, written $M[P]$, is
tantamount to filling the hole in $M$ with $P$. In this paper we do
not need the full weight of this theory, but do make use of the notion
of context in the proof the main theorem. 

\begin{mathpar}
  \inferrule* [lab=summation] {} {{M_{M},M_{N}} \bc \Box \;|\; x.M_{A} \;|\; M_{M}+M_{N}}
  \and
  \inferrule* [lab=agent] {} {{M_{A}} \bc (\vec{x})M_{P} \;| \; \clift{P_0,\ldots,M_{P},\ldots,P_N}}
  \and \\
  \inferrule* [lab=process] {} {{M_{P}} \bc M_{N} \;| \;P|M_{P} }
\end{mathpar} 

\begin{mathpar}
  \inferrule* [lab=sychronization] {} {M_{N} \bc \Box \;|\; x?M_{F} \;|\; x!M_{C}}
  \and
  \inferrule* [lab=abstraction] {} {{M_{F}} \bc (x)M_{P} }
  \and
  \inferrule* [lab=concretion] {} {{M_{C}} \bc \langle M_{P} \rangle }
  \and \\
  \inferrule* [lab=process] {} {{M_{P}} \bc M_{N} \;| \;P|M_{P} }
\end{mathpar}

\begin{definition}[contextual application] Given a context $M$, and
  process $P$, we define the \emph{contextual application}, $M[P] :=
  M\{P/\Box\}$. That is, the contextual application of M to P is the
  substitution of $P$ for $\Box$ in $M$.
\end{definition}

$\meaningof{-} : L \to \mathcal{P}(\pi)$

\begin{mathpar}
  \inferrule* [lab=collection] {} {\meaningof{true} = \pi, \and \meaningof{~E} = \pi \setminus \meaningof{E}, \and \meaningof{E_{1} \& E_{2}} = \meaningof{E_{1}} \cap \meaningof{E_{2}}}
\end{mathpar}

\begin{mathpar}
  \inferrule* [lab=structure] {} {\meaningof{0} = \{ P \in \pi | P \equiv 0 \}, \and \\ \meaningof{E_1 | E_2} = \{ P \in \pi | P \equiv P_{1} | P_{2}, P_{1} \in \meaningof{E_{1}}, P_{2} \in \meaningof{E_2}\} }
\end{mathpar}

\begin{mathpar}
 \inferrule* [lab=behavior] {} {\meaningof{\langle a?b \rangle E} = \{ P \in \pi | P \equiv Q | u?(y)P', \\ \and \\\\ \and \\ \;\;\; u \in \meaningof{a}, \forall z.P'\{z/y\} \in \meaningof{E\{z/b\}}\}, \and \\ \meaningof{a!E} = \{ P \in \pi | P \equiv Q | x!\langle P' \rangle, x \in \meaningof{a} P' \in \meaningof{E}\} }
\end{mathpar}

\begin{mathpar}
 \inferrule* [lab=nominal] {} {\meaningof{\quotep{E}} = \{ \quotep{P} \in \quotep{\pi} | P \in \meaningof{E} \}, \and \meaningof{\quotep{P}} = \{ \quotep{Q} \in \quotep{\pi} | P \equiv Q \} \and \\ \meaningof{@\quotep{E}} = \{ P \in \pi | P \equiv @x, x \in \meaningof{E} \}}
\end{mathpar}

\begin{eqnarray*}
  \\
  \meaningof{-} : TS \to ST
\end{eqnarray*}

\begin{eqnarray*}
  \\
  L : TS \to ST
\end{eqnarray*}

\begin{eqnarray*}
  \\
  P \models E \iff P \in \meaningof{E}
\end{eqnarray*}

\begin{eqnarray*}
  P \approx_{L} Q \iff \forall E \in L. P \models E \iff Q \models E
\end{eqnarray*}

\begin{eqnarray*}
  P \approx_{K} Q
\end{eqnarray*}

\begin{eqnarray*}
  P \approx Q
\end{eqnarray*}

$\approx_{K} = \approx = \approx_{L}$

\subsubsection{Contextual duality}

Note that contexts extend the quotation operation to a family of
operations from processes to names. Given a context, $M$, we can
define a \emph{nominal context}, $\quotep{M}$ by $\quotep{M}[P] :=
\quotep{M[P]}$. To foreshadow what is to come we observe that these
operations enjoy a duality with processes very much like the duality
between vectors and maps from vectors to scalars.

Further, because the calculus is essentially higher-order, we have a
correspondence between contexts and processes. More specifically,
given a name $x$ and a context $M$ we can construct $M^{*}_{x}$ such
that 

\begin{mathpar}
  M^{*}_{x} | \lift{x}{P} \red M[P]
\end{mathpar}

namely,

\begin{mathpar}
  M^{*}_{x} := x?(u).M[\dropn{u}]
\end{mathpar}

The dependence of $M^{*}_{x}$ on a name makes it an abstraction, 

\begin{mathpar}
  M^{*} := (x)x?(u).M[\dropn{u}]
\end{mathpar}

\subsection{Additional notation}

It will sometimes be convenient to denote the process a name
quotes. We already have the notation $x = \quotep{P}$, but it will be
convenient to introduce an alternate notation, $\procn{x}$, when we
want to emphasize the connection to the use of the name. Note that, by
virtue of name equivalence, $\quotep{\procn{x}} \nameeq x$; so, the
notation is consistent with previous definitions.

Further, because names have structure it is possible to effect
substitutions on the basis of that structure. This means we need to
upgrade our notation for substitutions, which we accomplish by
adapting comprehension notation. Thus,

\begin{mathpar}
  P\{ y / x : x \in S \}
\end{mathpar}

is interpreted to mean the process derived from P by replacing (in a
capture-avoiding manner) each occurrence of $x$ in $S$ by $y$. For example,

\begin{mathpar}
  P\{ \quotep{\procn{x}|\procn{x}} / x : x \in \freenames{P} \}
\end{mathpar}

will replace each (occurrence) of a free name $x$ in $P$ by
$\quotep{\procn{x}|\procn{x}}$.

Also, we will avail ourselves of the notation $x^{L}$ and $x^{R}$ to
denote injections of a name into disjoint copies of the name
space. There are numerous ways to accomplish this. One example can be
found in \cite{MeredithR05}. This notation overloads to vectors of
names: $\vec{x}^{\pi} := (x_{i}^{\pi} \; : \; 0 \leq i < |\vec{x}| )$ where $\pi \in \{L,R\}$.

We also use $P^{\Box} := P|\Box$.

In \cite{MeredithR05} an interpretation of the new operator is
given. It turns out that there are several possible interpretations
all enjoying the requisite algebraic properties of the operator (see
\cite{milner91polyadicpi}). We will therefore make liberal use of
$(\nu\; \vec{x})P$.

% subsection the_syntax_and_semantics_of_the_notation_system (end)   

\input{qm2pi.qmops} 

\input{qm2pi.sterngerlach} 

\input{qm2pi.metric} 

% section concurrent_process_calculi (end)

%\input{qm2pi.proofsketch}

% section proof sketch (end)

%\input{qm2pi.slviaknots} 

% section spatial logic via knots (end)

\input{qm2pi.conclusion}

% section conclusion (end)

%\input{qm2pi.dtcodes} 

% section wiring algorithm (end)

\input{qm2pi.ack} 

% section acknowledgments (end)

\newpage


\bibliographystyle{plain}   
\bibliography{../../biblios/main.bib}

\input{qm2pi.rhodetails}

\end{document}



\end{document}



% section proof sketch (end)

%\section{Unlikely characters: spatial logic for
  knots}\label{sub:characteristic_formulae} % (fold)

Associated to the mobile process calculi are a family of logics known
as the Hennessy-Milner logics. These logics typically enjoy a
semantics interpreting formulae as sets of processes that when
factored through the encoding outlined above allows an identification
of classes of knots with logical formulae. In the context of this
encoding the sub-family known as the spatial logics \cite{CairesC03}
\cite{CairesC04} \cite{Caires04} are of particular interest providing
several important features for expressing and reasoning about
properties (i.e. classes) of knots. We hint here at how this may be done.

%\begin{description}
%\item [structural connectives] 
\subsubsection{Structural connectives} The spatial logics enjoy
structural connectives corresponding, at the logical level, to the
parallel composition ($P | Q$) and new name ($(\nu \; x)P$)
connectives for processes. As illustrated in the examples below, these
connectives are extremely expressive given the shape of our encoding.
%\item [decideable satisfaction]

\subsubsection{Decideable satisfaction}
In \cite{Caires04} the satisfaction relation is shown to be decideable
for a rich class of processes. It further turns out that the image of
the our encoding is a proper subset of that class. This result
provides the basis for an algorithm by which to search for knots
enjoying a given property.
%\item [characteristic formulae]

\subsubsection{Characteristic formulae}
In the same paper \cite{Caires04} , Caires presents a means of calculating
characteristic formulae, selecting equivalence classes of processes
up to a pre--specified depth limit on the support set of names. Composed with our
encoding, this characteristic formula can be used to select
characteristic formulae for knots.
%\end{description}

\subsubsection{Spatial logic formulae}

The grammar below (segmented for comprehension) summarizes the syntax
of spatial logic formulae. We employ illustrative examples in the
sequel to provide an intuitive understanding of their meaning
referring the reader to \cite{Caires04} for a more detailed explication
of the semantics.

\begin{mathpar}
  \inferrule* [lab=boolean] {} {{A,B} \bc T \;|\; \neg A \;|\; A \wedge B \;|\; \eta = \eta'}
  \and
  \inferrule* [lab=spatial] {} {|\; \pzero \;|\; A | B \;|\; x \text{\textregistered} A \;|\; \forall x . A \;|\;  H x . A}
  \and
  \inferrule* [lab=behavioral] {} {|\; \alpha . A}
  \and 
  \inferrule* [lab=recursion] {} {|\; X(\vec{u}) \;|\; \mu X(\vec{u}) . A}
  \and
  \inferrule* [lab=action] {} {\alpha \bc \langle x?(\vec{y}) \rangle \;|\; \langle x!(\vec{y}) \rangle \;|\; \langle \tau \rangle}
  \and 
  \inferrule* [lab=name] {} {\eta \bc x \;|\; \tau}
\end{mathpar} 

% subsection characteristic_formulae (end)   	 

\subsection{Example formulae}\label{sub:example_formulae_} % (fold)

\subsubsection{Crossing as formula.}
% 
% \begin{align*}
%   \frac{d}{dx} \sin x &= \cos x 
%   & \frac{d}{dx} e^x &= e^x \\
%   \frac{d}{dx} \cos x &= - \sin x 
%   & \frac{d}{dx} \log x &= \frac{1}{x} \\
% \end{align*} 

\begin{align*}
 \mu C(x_{0},x_{1},y_{0},y_{1},u).&(\langle x_{0}?(z) \rangle(\langle u! \rangle\langle y_{1}!z \rangle C(x_{0},x_{1},y_{0},y_{1},u)) & \\
  & \wedge \langle y_{1}?(z) \rangle (\langle u! \rangle \langle x_{0}!z \rangle C(x_{0},x_{1},y_{0},y_{1},u)) & \\
  & \wedge \langle x_{1}?(z) \rangle (\langle u? \rangle \langle y_{0}!z \rangle C(x_{0},x_{1},y_{0},y_{1},u)) & \\
  & \wedge \langle y_{0}?(z) \rangle (\langle u? \rangle \langle x_{1}!z \rangle C(x_{0},x_{1},y_{0},y_{1},u))) &
\end{align*}

The lexicographical similarity between the shape of this formulae and
the shape of definition of the process representing a crossing reveals
the intuitive meaning of this formulae. It describes the capabilities
of a process that has the right to represent a crossing. For example
it picks out processes that may perform an input on the port $x_0$ in
its initial menu of capabilities. What differentiates the formula
from the process, however, is that the crossing process is the
smallest candidate to satisfy the formula. Infinitely many other
processes -- with internal behavior hidden behind this interface, so
to speak -- also satisfy this formula. Even this simple formula,
then, can be seen to open a new view onto knots, providing a
computational interpretation of \emph{virtual} knots.

Note that this formula is derived by hand. A similar formula can be
derived by employing Caires' calculation of characteristic formula
\cite{Caires04} to the process representing a crossing. In light of
this discussion, we let
$\meaningof{C}_{\phi}(x0,x1,y0,y1,u)$ denote a formula specifying the
dynamics we wish to capture of a crossing. To guarantee we preserve
the shape of the interface and minimal semantics we demand that
$\meaningof{C}_{\phi}(x0,x1,y0,y1,u) \Rightarrow
\textbf{C}(x0,x1,y0,y1,u)$ where $\textbf{C}(x0,x1,y0,y1,u)$ denotes
the formula above.
                            
\subsubsection{Crossing number constraints.}
The moral content of the context lemma (Lemma \ref{context}) is that the notion of
``locality'' in the Reidemeister moves is effectively captured by the
parallel composition operator of the process calculus. This intuition
extends through the logic. Given a formula,
$\meaningof{C}_{\phi}(x0,x1,y0,y1,u)$, we can use the structural
connectives to specify constraints on crossing numbers, such as at
least $n$ crossings, or exactly $n$ crossings.
\begin{mathpar}
  \inferrule* [lab=at-least-n] {} { K^{\geq n}_{\phi}(\vec{xs},\vec{ys}) := \Pi_{i=0}^{n-1} Hu . \meaningof{C}_{\phi}(xs_i,ys_i,u) | T }
  \and 
  \inferrule* [lab=exactly-n] {} { K^{= n}_{\phi}(\vec{xs},\vec{ys}) := \Pi_{i=0}^{n-1} Hu . \meaningof{C}_{\phi}(xs_i,ys_i,u) | \neg (\forall x_0,y_0,x_1,y_1,u . \meaningof{C}_{\phi}(x_0,y_0,x_1,y_1,u) | T) }
\end{mathpar}

To round out this section, recall that the encoding of an $n$-crossing
knot decomposes into a parallel composition of $n$ \emph{copies} of a
crossing process together with a wiring harness. To specify different
knot classes with the same crossing number amounts to specifying
logical constraints on the wiring harness. In the interest of space,
we defer examples to a forthcoming paper. Suffice it to say that both
the conditions ``alternating knot'' and ``contains the tangle
corresponding to 5/3'' are expressible. For example, it is possible to
calculate the characteristic formula of a process corresponding to the
tangle 5/3 and conjoin it into the classifying formula via the
composition connective of the logic.

Finally, we wish to observe that it is entirely within reason to
contemplate a more domain-specific version of spatial logic tailored
to the shape of processes in the image of the encoding. Such a
domain-specific logic would have a better claim to the title formal
language of knot properties.

% subsection example_formulae_ (end)

% section knots_as_processes (end) 

% section spatial logic via knots (end)

\section{Conclusions and future work}

\paragraph{Testing physical space}
You, gentle reader, may wonder why of all the theorems to be proved
given this set up we pick the one above. In some sense it's hardly
central to quantum mechanics. We see it as central in the sense that
it firmly establishes a notion of physical space arising from a notion
of the equivalence of behavior. Relating bisimulation to a metric is a
big step forward, but one is faced with interpreting the relationship
of that metric space to something more physical. Quantum mechanical
notions of ``physical'' space are still far from intuitive, but by
relating this idea of distance as testing to calculations that predict
physical circumstances we are making a not insignificant step forward
toward an understanding of the physical space we inhabit as
essentially dynamic.

\paragraph{Effectivity and simulation}
One of the observations we have yet to make is that the entire program
spelled out here is effective. We have built various interpreters for
the reflective calculus at work in this interpretation. In principle,
then, we can simulate quantum mechanics on a computer. The place where
the simulation may lose fidelity is the infinitely branching summation
for the annihilator.

In this connection i also want to point out that the evaluation style
calculation of the inner product puts the non-determinism of the
summation right at the heart of measurement. This suggests that
Milner's original reduction-based formulation of the dynamics of his
calculi in terms of sums was not just notationally suggestive of a
notion of measure-and-continue but captured some significant part of
the physics.

\paragraph{Quantum continuations}
In light of this last observation i want to point out that the
predominant account of quantum mechanics is missing a key aspect of a
truly compositional story of the physical situation. In a real lab,
when a measurement is made the observation can be made to feed into
another device that then makes another measurement conditioned on the
results of the first. This means that after the superposition was
collapsed the entire experimental set up remained in
superposition. While QM offers a means of writing this down it doesn't
quite line up well with the well-trodden formulation of computation
and continuation that we see so succinctly expressed in Milner's
calculi. This suggests that there might be advantages to this account
of dynamics waiting to be explored.

\paragraph{Quantum logic}
In this connection, we also note that by virtue of having the
Hennessy-Milner construction, we can pull the construction through the
interpretation of QM. This gives us a natural candidate for a quantum
logic that enjoys an extremely tight connection with it's domain of
interpretation, making the construction much less ad hoc (rather it is
the image of functor!).

\paragraph{Quantum probabiity}
i have questions about the basis of the interpretation of inner
product as probability amplitude. In particular, using which
axiomatization of probability theory does the notion of probability
amplitude earn the right to be so dubbed? In other words, where is the
proof that the operation for calculating a probability amplitude (and
then squaring) satisfies the axioms of what it means to calculate a
probability? Even if such a proof exists (i have yet to find it in the
literature), i wonder if it might not be possible to turn things on
their heads. Can we view the calculation of the probability amplitude
as an axiomatization of probability? If so, then the definition we
give for calculating probability amplitude may provide the basis for
an \emph{effective} theory of probability.

\paragraph{Quantum vs ``biological'' information}
Finally, i want to conclude with a more philosophical observation. At
a recent workshop in which QM was a predominant topic i noticed
something about quantum information. The speaker was giving a riveting
discussion of axiomatic QM and showing how properties of ``no
cloning'' and ``no deleting'' emerged as consequences of the
axiomatization. Theorems of this form are necessary to give us a sense
of confidence that our axioms characterize the physical theory. What
struck me, though, was that if quantum information is neither erasable
nor replicable it is markedly different from \emph{life}. Two of the
things we know about life is that

\begin{itemize}
  \item it ends;
  \item to gain some measure of persistence, to transcend it's
    finitude it is imminently copyable.
\end{itemize}

Both of these qualities are summarized succinctly in the aphorism: all
flesh is grass. For me these two kinds of ``information'' -- call them
quantum and biological -- are end points on a spectrum of strategies
for persistence. At one end, we have those curious entities that enjoy
uniqueness and permanence; at the other, we have those who in the face
of a certain end and an uncertain present make a go of passing
something on. To me one of the more remarkable aspects of the latter
strategy is that in the presence of noise (and certain features of
copying) we get a kind of dynamism, a chance for improvement against a
given persistent condition.

% subsection other_calculi_other_bisimulations_and_geometry_as_behavior (end)




% section conclusion (end)

%\documentclass[12pt]{llncs}
%\documentclass{jktr}

\usepackage[pdftex]{hyperref}                   
\usepackage {listings}
\usepackage {mathpartir}
\usepackage{bcprules}
%\usepackage{listings}
                       
\usepackage{graphicx} 
%\usepackage[margins=2.5cm,nohead,nofoot]{geometry}
%\usepackage{geometry}
\usepackage{amsfonts}
\usepackage{amstext}
\usepackage{latexsym}
\usepackage{amssymb}
\usepackage{color}


%\include{myPreamble}
\documentclass[12pt]{llncs}
%\documentclass{jktr}

\usepackage[pdftex]{hyperref}                   
\usepackage {listings}
\usepackage {mathpartir}
\usepackage{bcprules}
%\usepackage{listings}
                       
\usepackage{graphicx} 
%\usepackage[margins=2.5cm,nohead,nofoot]{geometry}
%\usepackage{geometry}
\usepackage{amsfonts}
\usepackage{amstext}
\usepackage{latexsym}
\usepackage{amssymb}
\usepackage{color}


%\include{myPreamble}
\include{qm2pi.local} 

%\ifpdf
%\usepackage[pdftex]{graphicx}
%\else
%\usepackage{graphicx}
%\fi

 % \ifpdf
%  \usepackage{pdfsync}
%  \if


%\title{Brief Article}
%\author{David F. Snyder}
%\author{L.G. Meredith}

%\address{Dept. of Math., Texas State University--San Marcos, San Marcos, TX 78666}
       
\pagestyle{empty}


\begin{document}

\lstset{language=[Objective]Caml,frame=shadowbox}

\input{qm2pi.front}

% section front matter (end)

\input{qm2pi.intro} 
 
% section introduction (end)

% \input{qm2pi.knotations} 

% section notation (end)

\input{qm2pi.process.calculi} 

% section concurrent_process_calculi_and_spatial_logics_ (end)
    
%\input{qm2pi.knots2pi} 

%\input{qm2pi.trefoil} 

%\input{qm2pi.mainthm} 

% subsection basic_interpretation (end)

%\input{qm2pi.rho.presentation} 
\subsection{The syntax and semantics of the notation system}\label{sub:the_syntax_and_semantics_of_the_notation_system} % (fold)

We now summarize a technical presentation of the calculus that
embodies our theory of dynamics. The typical presentation of such a
calculus follows the style of giving generators and relations on
them. The grammar, below, describing term constructors, freely
generates the set of processes, $\Proc$. This set is then quotiented
by a relation known as structural congruence and it is over this set
that the notion of dynamics is expressed. This presentation is
essentially that of \cite{MeredithR05} with the addition of
polyadicity and summation. For readability we have relegated some of
the technical subtleties to an appendix.

\subsubsection{Process grammar}\label{subsub:process_grammar}

\begin{mathpar}
  \inferrule* [lab=synchronization] {} {{M} \bc \pzero \;|\; x?F \;|\; x!C }
  \and
  \inferrule* [lab=abstraction] {} {{F} \bc (x)P}
  \and
  \inferrule* [lab=concretion] {} {{C} \bc \langle Q \rangle}
  \and
  \inferrule* [lab=process] {} {{P,Q} \bc M \;| \;P|Q \;|\; @{x}}
  \and
  \inferrule* [lab=name] {} {{x} \bc \quotep{P}}
\end{mathpar} 

Note that $\vec{x}$ (resp. $\vec{P}$) denotes a vector of names
(resp. processes) of length $|\vec{x}|$ (resp. $|\vec{P}|$). We adopt
the following useful abbreviations.

\begin{mathpar}
   x?(\vec{y}).P := x.(\vec{y})P \and  x\clift{\vec{P}} := x.\clift{\vec{P}}
   \and x!(y) := \lift{x}{\dropn{y}}
   \and \Pi_{i=0}^{n-1}P_i := P_0 | \ldots | P_{n-1}
\end{mathpar}

\subsubsection{Structural congruence}

\paragraph{Free and bound names and alpha-equivalence.} At the
core of structural equivalence is alpha-equivalence which identifies
process that are the same up to a change of variable. Formally, we
recognize the distinction between free and bound names. The free names
of a process, $\freenames{P}$, may be calculated recursively as
follows:

\begin{mathpar}
\freenames{\pzero} := \emptyset
  \and \\
  \freenames{x?(y).P} := \{ x \} \cup (\freenames{P} \setminus \{ y \})
  \and 
  \freenames{x!\langle P \rangle} := \{ x \} \cup \{ P \} 
  \and \\
  \freenames{P|Q} := \freenames{P} \cup \freenames{Q}
  \and \\
  \freenames{@{x}} := \{ x \}
\end{mathpar}

$\pi$
$\quotep{\pi}$

$\freenames{-} : \pi \to \mathcal{P}(\quotep{\pi})$

\begin{eqnarray*}
  \freenames{\pzero} & := & \emptyset \\
  \freenames{x?(y).P} & := & \{ x \} \cup (\freenames{P} \setminus \{ y \}) \\
  \freenames{x!\langle P \rangle} & := & \{ x \} \cup \{ P \} \\
  \freenames{P|Q} & := & \freenames{P} \cup \freenames{Q} \\
  \freenames{\dropn{x}} & := & \{ x \}
\end{eqnarray*}

The bound names of a process, $\boundnames{P}$, are those names occurring in $P$
that are not free. For example, in $x?(y).0$, the name $x$ is free, while $y$ is bound.

\begin{mathpar}
  \inferrule* [lab=monoidal-laws] {} { P|Q \equiv Q|P \and P|0 \equiv P \and P|(Q|R) \equiv (P|Q)|R }
\end{mathpar}

\begin{mathpar}
  \inferrule* [lab=alpha-equivalence] {} { (x)P \equiv (y)P\{y/x\} \and y \not\in \freenames{P} }
\end{mathpar}

\begin{definition}
Then two processes, $P,Q$, are alpha-equivalent if $P = Q\{\vec{y}/\vec{x}\}$ for
some $\vec{x} \in \boundnames{Q},\vec{y} \in \boundnames{P}$, where $Q\{\vec{y}/\vec{x}\}$
denotes the capture-avoiding substitution of $\vec{y}$ for $\vec{x}$ in $Q$.
\end{definition}

\begin{definition}
  The {\em structural congruence} \cite{SangiorgiWalker} , $\equiv$,
  between processes is the least congruence containing
  alpha-equivalence, satisfying the abelian monoid laws
  (associativity, commutativity and $\pzero$ as identity) for parallel
  composition $|$ and for summation $+$.
\end{definition}

\subsection{Name equivalence}

We take name equivalence, written $\nameeq$, to be the smallest
equivalence relation generated by the following rules.

\begin{mathpar}
\inferrule*[lab=Quote-drop]
{ }
{ \quotep{@{x}} \nameeq x }

\inferrule*[lab=Struct-equiv]
{ P \scong Q }
{ \quotep{P} \nameeq \quotep{Q} }
\end{mathpar}

The astute reader will have noticed that the mutual recursion of names
and processes imposes a mutual recursion on alpha-equivalence and
structural equivalence via name-equivalence. Fortunately, all of this
works out pleasantly and we may calculate in the natural way, free of
concern. The reader interested in the details is referred to the
appendix \ref{appendix:rho_details}.

\subsection{Substitution}

We use $\Proc$ for the set of processes, $\QProc$ for the set of
names, and $\id{\{}\vec{y} / \vec{x} \id{\}}$ to denote partial maps,
$s : \QProc \rightarrow \QProc$. A map, $s$ lifts, uniquely, to a map
on process terms, $\widehat{s} : \Proc \rightarrow \Proc$ by the
following equations.

\begin{mathpar}
  (0) \psubstp{Q}{P} := 0 \\
  (R \juxtap S) \psubstp{Q}{P}
  :=    
  (R)\psubstp{Q}{P} \juxtap (S) \psubstp{Q}{P} \\
  (x?(y).R) \psubstp{Q}{P}    
  :=    
  (x)\substp{Q}{P} (z)\concat( (R \psubstn{z}{y}) \psubstp{Q}{P} ) \\
  (\lift{x}{R}) \psubstp{Q}{P}  
  :=
  \lift{(x)\substp{Q}{P}}{ R \psubstp{Q}{P} } \\
%   (\dropn{x})  \psubstp{Q}{P}       
%   := 
%   \left\{ 
%     \begin{array}{ccc} 
%       \dropn{\quotep{Q}} & & x \nameeq \quotep{P} \\
%       \dropn{x} & & otherwise \\
%     \end{array}
%   \right. 
  (\dropn{x})  \psubstp{Q}{P}       
  := 
  \left\{ 
    \begin{array}{ccc} 
      Q & & x \nameeq \quotep{P} \\
      \dropn{x} & & otherwise \\
    \end{array}
  \right.
\end{mathpar}
 

where

\begin{eqnarray}
  (x)\id{\{} \lpquote Q \rpquote / \lpquote P \rpquote \id{\}}            = 
  \left\{ 
    \begin{array}{ccc}
      \lpquote Q \rpquote & & x \nameeq \lpquote P \rpquote \\
      x & & otherwise \\
    \end{array}
  \right. \nonumber
\end{eqnarray}

and $z$ is chosen distinct from $\quotep{P}$, $\quotep{Q}$, the free
names in $Q$, and all the names in $R$. Our $\alpha$-equivalence will
be built in the standard way from this substitution.

\begin{remark}\label{rem:no_self_referential_names}
  One consequence of these definitions is that $\forall P. \quotep{P}
  \not\in \freenames{P}$.
\end{remark}

\subsection{ Dynamic quote: an example }

Anticipating something of what's to come, consider applying the
substitution, $\widehat{\id{\{}u / z \id{\}}}$, to the following pair
of processes, $\lift{w}{y!(z)}$ and $w[ \lpquote y!(z) \rpquote ]$.

\begin{eqnarray}
	\lift{w}{y!(z)}\widehat{\id{\{}u / z \id{\}}}
		& = &
		\lift{w}{y!(u)} \nonumber\\
	w[ \lpquote y!(z) \rpquote ] \widehat{ \id{\{}u / z \id{\}} }
		& = &
		w[ \lpquote y!(z) \rpquote ] \nonumber
\end{eqnarray}

Because the body of the process between quotes is impervious to
substitution, we get radically different answers. In fact, by
examining the first process in an input context,
e.g. $x?(z).\lift{w}{y!(z)}$, we see that the process under the lift
operator may be shaped by prefixed inputs binding a name inside it. In
this sense, the lift operator will be seen as a way to dynamically
construct processes before reifying them as names.

Finally equipped with these standard features we can present the
dynamics of the calculus.

\subsubsection{Operational semantics} 

Finally, we introduce the computational dynamics. What marks these
algebras as distinct from other more traditionally studied algebraic
structures, e.g. vector spaces or polynomial rings, is the manner in
which dynamics is captured. In traditional structures, dynamics is typically
expressed through morphisms between such structures, as in linear maps
between vector spaces or morphisms between rings. In algebras
associated with the semantics of computation, the dynamics is
expressed as part of the algebraic structure itself, through a
reduction reduction relation typically denoted by $\red$. Below, we
give a recursive presentation of this relation for the calculus used
in the encoding.

$\red \subseteq \pi \times \pi$
$\red : \pi \to \mathcal{P}(\pi)$

\begin{mathpar}
  \inferrule* [lab=Comm] { \textsf{match}( x_{src}, x_{trgt} ) } { x_{trgt}?(y)P \; | \; x_{src}!\langle {Q} \rangle \red P\{\quotep{Q}/y}\} }
  \and \\
  \inferrule* [lab=Par] {{P} \red {P}'} {{{P} | {Q}} \red {{P}' | {Q}}}
  \and
  \inferrule* [lab=Equiv]{{{P} \scong {P}'} \andalso {{P}' \red {Q}'} \andalso {{Q}' \scong {Q}}}{{P} \red {Q}}
\end{mathpar}

\begin{eqnarray*}
  match_{\equiv} (\quotep{P},\quotep{Q}) & := & P \equiv Q \\
  match_{\dagger}(\quotep{P},\quotep{Q}) & := & \forall R. P|Q \red^{*} R => R \red^{*} 0 \\
  match_{K}(\quotep{P},\quotep{Q}) & := & K \mbox{ for some context } K
\end{eqnarray*}

$u?(x)P | u!\langle Q \rangle \red P\{\quotep{Q}/x\}$

%We write $\wred$ for $\red^*$, and $P\red$ if $\exists Q $ such that $ P \red Q$.
We write $P\red$ if $\exists Q $ such that $ P \red Q$ and $P\not\red$, otherwise.

\section{Replication}

As mentioned before, it is known that replication (and hence
recursion) can be implemented in a higher-order process algebra
\cite{SangiorgiWalker}. As our first example of calculation with the
machinery thus far presented we give the construction explicitly in
the {\rhoc}.

\begin{eqnarray}
	D_{x} & := & \prefix{x}{y}{(\binpar{\outputp{x}{y}}{@{y}})} \nonumber\\
	\bangp_{x}{P} & := & \binpar{{x}!\langle{\binpar{D_{x}}{P}}\rangle}{D_{x}} \nonumber
\end{eqnarray}

\begin{eqnarray}
	\bangp_{x}{P} & & \nonumber\\
	=
	& {x}!\langle{(\prefix{x}{y}{(\outputp{x}{y} | @{y})) | P}}\rangle 
	      | \prefix{x}{y}{(\outputp{x}{y} | @{y})} & \nonumber\\
	\red
	& (\outputp{x}{y} | @{y})\substn{\quotep{(\prefix{x}{y}{(@{y} | \outputp{x}{y})) | P}}}{y} & \nonumber\\
	=
	& \outputp{x}{\quotep{(\prefix{x}{y}{(\outputp{x}{y} | @{y})) | P}}}
	  | {(\prefix{x}{y}{(\outputp{x}{y} | @{y})) | P}} & \nonumber\\
	\red
	& \ldots & \nonumber\\
	\red^*
	& P | P | \ldots & \nonumber
\end{eqnarray}

Of course, this encoding, as an implementation, runs away, unfolding
$\bangp{P}$ eagerly. A lazier and more implementable replication
operator, restricted to input-guarded processes, may be obtained as follows.

\begin{eqnarray}
\bangp{\prefix{u}{v}{P}} 
	:= 
	\binpar{\lift{x}{\prefix{u}{v}{(\binpar{D(x)}{P})}}}{D(x)} \nonumber
\end{eqnarray}

\begin{remark}
  Note that the lazier definition still does not deal with summation
  or mixed summation (i.e. sums over input and output). The reader is
  invited to construct definitions of replication that deal with these
  features. 

  Further, the definitions are parameterized in a name, $x$. Can you,
  gentle reader, make a definition that eliminates this parameter and
  guarantees no accidental interaction between the replication
  machinery and the process being replicated -- i.e. no accidental
  sharing of names used by the process to get its work done and the
  name(s) used by the replication to effect copying. This latter
  revision of the definition of replication is crucial to obtaining
  the expected identity $!!P \sim !P$.
\end{remark}

\begin{remark}\label{rem:paradoxical_combinator}
  The reader familiar with the lambda calculus will have noticed the
  similarity between $D$ and the paradoxical combinator.

  [Ed. note: the existence of this seems to suggest we have to be more
  restrictive on the set of processes and names we admit if we are to
  support no-cloning.]
\end{remark}

\subsubsection{Bisimulation}

The computational dynamics gives rise to another kind of equivalence,
the equivalence of computational behavior. As previously mentioned
this is typically captured \emph{via} some form of bisimulation.

% The notion we use in this paper is weak barbed bisimulation
% \cite{milner91polyadicpi}.

The notion we use in this paper is derived from weak barbed
bisimulation \cite{milner91polyadicpi}. 

\begin{definition}
An \emph{observation relation}, $\downarrow_{\mathcal N}$, over a set
of names, $\mathcal N$, is the smallest relation satisfying the rules
below.

\infrule[Out-barb]{y \in {\mathcal N}, \; x \nameeq y}
		  {\outputp{x}{v} \downarrow_{\mathcal N} x}
\infrule[Par-barb]{\mbox{$P\downarrow_{\mathcal N} x$ or $Q\downarrow_{\mathcal N} x$}}
		  {\binpar{P}{Q} \downarrow_{\mathcal N} x}

We write $P \Downarrow_{\mathcal N} x$ if there is $Q$ such that 
$P \wred Q$ and $Q \downarrow_{\mathcal N} x$.
\end{definition}

\begin{definition}
%\label{def.bbisim}
An  ${\mathcal N}$-\emph{barbed bisimulation} over a set of names, ${\mathcal N}$, is a symmetric binary relation 
${\mathcal S}_{\mathcal N}$ between agents such that $P\rel{S}_{\mathcal N}Q$ implies:
\begin{enumerate}
\item If $P \red P'$ then $Q \wred Q'$ and $P'\rel{S}_{\mathcal N} Q'$.
\item If $P\downarrow_{\mathcal N} x$, then $Q\Downarrow_{\mathcal N} x$.
\end{enumerate}
$P$ is ${\mathcal N}$-barbed bisimilar to $Q$, written
$P \wbbisim_{\mathcal N} Q$, if $P \rel{S}_{\mathcal N} Q$ for some ${\mathcal N}$-barbed bisimulation ${\mathcal S}_{\mathcal N}$.
\end{definition}

$\mathcal{R} \subseteq \pi \times \pi$

$P \mathcal{R} Q => \forall P'. P \red P' \Rightarrow \exists Q'. Q \red Q', P' \mathcal{R} Q'$

$P \vdash x \Rightarrow Q \vdash x$

\begin{mathpar}
  \inferrule*[lab=Out-barb]{x \nameeq y}{{y}!\langle{Q}\rangle \vdash x}
  \and
  \inferrule*[lab=Par-barb]{\mbox{$P\vdash x$ or $Q\vdash x$}}{\binpar{P}{Q} \vdash x}
\end{mathpar}

\subsubsection{Contexts}

One of the principle advantages of computational calculi like the
$\pi$-calculus is a well-defined notion of context,
contextual-equivalence and a correlation between
contextual-equivalence and notions of bisimulation. The notion of
context allows the decomposition of a process into (sub-)process and
its syntactic environment, its context. Thus, a context may be
thought of as a process with a ``hole'' (written $\Box$) in it. The
application of a context $M$ to a process $P$, written $M[P]$, is
tantamount to filling the hole in $M$ with $P$. In this paper we do
not need the full weight of this theory, but do make use of the notion
of context in the proof the main theorem. 

\begin{mathpar}
  \inferrule* [lab=summation] {} {{M_{M},M_{N}} \bc \Box \;|\; x.M_{A} \;|\; M_{M}+M_{N}}
  \and
  \inferrule* [lab=agent] {} {{M_{A}} \bc (\vec{x})M_{P} \;| \; \clift{P_0,\ldots,M_{P},\ldots,P_N}}
  \and \\
  \inferrule* [lab=process] {} {{M_{P}} \bc M_{N} \;| \;P|M_{P} }
\end{mathpar} 

\begin{mathpar}
  \inferrule* [lab=sychronization] {} {M_{N} \bc \Box \;|\; x?M_{F} \;|\; x!M_{C}}
  \and
  \inferrule* [lab=abstraction] {} {{M_{F}} \bc (x)M_{P} }
  \and
  \inferrule* [lab=concretion] {} {{M_{C}} \bc \langle M_{P} \rangle }
  \and \\
  \inferrule* [lab=process] {} {{M_{P}} \bc M_{N} \;| \;P|M_{P} }
\end{mathpar}

\begin{definition}[contextual application] Given a context $M$, and
  process $P$, we define the \emph{contextual application}, $M[P] :=
  M\{P/\Box\}$. That is, the contextual application of M to P is the
  substitution of $P$ for $\Box$ in $M$.
\end{definition}

$\meaningof{-} : L \to \mathcal{P}(\pi)$

\begin{mathpar}
  \inferrule* [lab=collection] {} {\meaningof{true} = \pi, \and \meaningof{~E} = \pi \setminus \meaningof{E}, \and \meaningof{E_{1} \& E_{2}} = \meaningof{E_{1}} \cap \meaningof{E_{2}}}
\end{mathpar}

\begin{mathpar}
  \inferrule* [lab=structure] {} {\meaningof{0} = \{ P \in \pi | P \equiv 0 \}, \and \\ \meaningof{E_1 | E_2} = \{ P \in \pi | P \equiv P_{1} | P_{2}, P_{1} \in \meaningof{E_{1}}, P_{2} \in \meaningof{E_2}\} }
\end{mathpar}

\begin{mathpar}
 \inferrule* [lab=behavior] {} {\meaningof{\langle a?b \rangle E} = \{ P \in \pi | P \equiv Q | u?(y)P', \\ \and \\\\ \and \\ \;\;\; u \in \meaningof{a}, \forall z.P'\{z/y\} \in \meaningof{E\{z/b\}}\}, \and \\ \meaningof{a!E} = \{ P \in \pi | P \equiv Q | x!\langle P' \rangle, x \in \meaningof{a} P' \in \meaningof{E}\} }
\end{mathpar}

\begin{mathpar}
 \inferrule* [lab=nominal] {} {\meaningof{\quotep{E}} = \{ \quotep{P} \in \quotep{\pi} | P \in \meaningof{E} \}, \and \meaningof{\quotep{P}} = \{ \quotep{Q} \in \quotep{\pi} | P \equiv Q \} \and \\ \meaningof{@\quotep{E}} = \{ P \in \pi | P \equiv @x, x \in \meaningof{E} \}}
\end{mathpar}

\begin{eqnarray*}
  \\
  \meaningof{-} : TS \to ST
\end{eqnarray*}

\begin{eqnarray*}
  \\
  L : TS \to ST
\end{eqnarray*}

\begin{eqnarray*}
  \\
  P \models E \iff P \in \meaningof{E}
\end{eqnarray*}

\begin{eqnarray*}
  P \approx_{L} Q \iff \forall E \in L. P \models E \iff Q \models E
\end{eqnarray*}

\begin{eqnarray*}
  P \approx_{K} Q
\end{eqnarray*}

\begin{eqnarray*}
  P \approx Q
\end{eqnarray*}

$\approx_{K} = \approx = \approx_{L}$

\subsubsection{Contextual duality}

Note that contexts extend the quotation operation to a family of
operations from processes to names. Given a context, $M$, we can
define a \emph{nominal context}, $\quotep{M}$ by $\quotep{M}[P] :=
\quotep{M[P]}$. To foreshadow what is to come we observe that these
operations enjoy a duality with processes very much like the duality
between vectors and maps from vectors to scalars.

Further, because the calculus is essentially higher-order, we have a
correspondence between contexts and processes. More specifically,
given a name $x$ and a context $M$ we can construct $M^{*}_{x}$ such
that 

\begin{mathpar}
  M^{*}_{x} | \lift{x}{P} \red M[P]
\end{mathpar}

namely,

\begin{mathpar}
  M^{*}_{x} := x?(u).M[\dropn{u}]
\end{mathpar}

The dependence of $M^{*}_{x}$ on a name makes it an abstraction, 

\begin{mathpar}
  M^{*} := (x)x?(u).M[\dropn{u}]
\end{mathpar}

\subsection{Additional notation}

It will sometimes be convenient to denote the process a name
quotes. We already have the notation $x = \quotep{P}$, but it will be
convenient to introduce an alternate notation, $\procn{x}$, when we
want to emphasize the connection to the use of the name. Note that, by
virtue of name equivalence, $\quotep{\procn{x}} \nameeq x$; so, the
notation is consistent with previous definitions.

Further, because names have structure it is possible to effect
substitutions on the basis of that structure. This means we need to
upgrade our notation for substitutions, which we accomplish by
adapting comprehension notation. Thus,

\begin{mathpar}
  P\{ y / x : x \in S \}
\end{mathpar}

is interpreted to mean the process derived from P by replacing (in a
capture-avoiding manner) each occurrence of $x$ in $S$ by $y$. For example,

\begin{mathpar}
  P\{ \quotep{\procn{x}|\procn{x}} / x : x \in \freenames{P} \}
\end{mathpar}

will replace each (occurrence) of a free name $x$ in $P$ by
$\quotep{\procn{x}|\procn{x}}$.

Also, we will avail ourselves of the notation $x^{L}$ and $x^{R}$ to
denote injections of a name into disjoint copies of the name
space. There are numerous ways to accomplish this. One example can be
found in \cite{MeredithR05}. This notation overloads to vectors of
names: $\vec{x}^{\pi} := (x_{i}^{\pi} \; : \; 0 \leq i < |\vec{x}| )$ where $\pi \in \{L,R\}$.

We also use $P^{\Box} := P|\Box$.

In \cite{MeredithR05} an interpretation of the new operator is
given. It turns out that there are several possible interpretations
all enjoying the requisite algebraic properties of the operator (see
\cite{milner91polyadicpi}). We will therefore make liberal use of
$(\nu\; \vec{x})P$.

% subsection the_syntax_and_semantics_of_the_notation_system (end)   

\input{qm2pi.qmops} 

\input{qm2pi.sterngerlach} 

\input{qm2pi.metric} 

% section concurrent_process_calculi (end)

%\input{qm2pi.proofsketch}

% section proof sketch (end)

%\input{qm2pi.slviaknots} 

% section spatial logic via knots (end)

\input{qm2pi.conclusion}

% section conclusion (end)

%\input{qm2pi.dtcodes} 

% section wiring algorithm (end)

\input{qm2pi.ack} 

% section acknowledgments (end)

\newpage


\bibliographystyle{plain}   
\bibliography{../../biblios/main.bib}

\input{qm2pi.rhodetails}

\end{document}

 

%\ifpdf
%\usepackage[pdftex]{graphicx}
%\else
%\usepackage{graphicx}
%\fi

 % \ifpdf
%  \usepackage{pdfsync}
%  \if


%\title{Brief Article}
%\author{David F. Snyder}
%\author{L.G. Meredith}

%\address{Dept. of Math., Texas State University--San Marcos, San Marcos, TX 78666}
       
\pagestyle{empty}


\begin{document}

\lstset{language=[Objective]Caml,frame=shadowbox}

\documentclass[12pt]{llncs}
%\documentclass{jktr}

\usepackage[pdftex]{hyperref}                   
\usepackage {listings}
\usepackage {mathpartir}
\usepackage{bcprules}
%\usepackage{listings}
                       
\usepackage{graphicx} 
%\usepackage[margins=2.5cm,nohead,nofoot]{geometry}
%\usepackage{geometry}
\usepackage{amsfonts}
\usepackage{amstext}
\usepackage{latexsym}
\usepackage{amssymb}
\usepackage{color}


%\include{myPreamble}
\include{qm2pi.local} 

%\ifpdf
%\usepackage[pdftex]{graphicx}
%\else
%\usepackage{graphicx}
%\fi

 % \ifpdf
%  \usepackage{pdfsync}
%  \if


%\title{Brief Article}
%\author{David F. Snyder}
%\author{L.G. Meredith}

%\address{Dept. of Math., Texas State University--San Marcos, San Marcos, TX 78666}
       
\pagestyle{empty}


\begin{document}

\lstset{language=[Objective]Caml,frame=shadowbox}

\input{qm2pi.front}

% section front matter (end)

\input{qm2pi.intro} 
 
% section introduction (end)

% \input{qm2pi.knotations} 

% section notation (end)

\input{qm2pi.process.calculi} 

% section concurrent_process_calculi_and_spatial_logics_ (end)
    
%\input{qm2pi.knots2pi} 

%\input{qm2pi.trefoil} 

%\input{qm2pi.mainthm} 

% subsection basic_interpretation (end)

%\input{qm2pi.rho.presentation} 
\subsection{The syntax and semantics of the notation system}\label{sub:the_syntax_and_semantics_of_the_notation_system} % (fold)

We now summarize a technical presentation of the calculus that
embodies our theory of dynamics. The typical presentation of such a
calculus follows the style of giving generators and relations on
them. The grammar, below, describing term constructors, freely
generates the set of processes, $\Proc$. This set is then quotiented
by a relation known as structural congruence and it is over this set
that the notion of dynamics is expressed. This presentation is
essentially that of \cite{MeredithR05} with the addition of
polyadicity and summation. For readability we have relegated some of
the technical subtleties to an appendix.

\subsubsection{Process grammar}\label{subsub:process_grammar}

\begin{mathpar}
  \inferrule* [lab=synchronization] {} {{M} \bc \pzero \;|\; x?F \;|\; x!C }
  \and
  \inferrule* [lab=abstraction] {} {{F} \bc (x)P}
  \and
  \inferrule* [lab=concretion] {} {{C} \bc \langle Q \rangle}
  \and
  \inferrule* [lab=process] {} {{P,Q} \bc M \;| \;P|Q \;|\; @{x}}
  \and
  \inferrule* [lab=name] {} {{x} \bc \quotep{P}}
\end{mathpar} 

Note that $\vec{x}$ (resp. $\vec{P}$) denotes a vector of names
(resp. processes) of length $|\vec{x}|$ (resp. $|\vec{P}|$). We adopt
the following useful abbreviations.

\begin{mathpar}
   x?(\vec{y}).P := x.(\vec{y})P \and  x\clift{\vec{P}} := x.\clift{\vec{P}}
   \and x!(y) := \lift{x}{\dropn{y}}
   \and \Pi_{i=0}^{n-1}P_i := P_0 | \ldots | P_{n-1}
\end{mathpar}

\subsubsection{Structural congruence}

\paragraph{Free and bound names and alpha-equivalence.} At the
core of structural equivalence is alpha-equivalence which identifies
process that are the same up to a change of variable. Formally, we
recognize the distinction between free and bound names. The free names
of a process, $\freenames{P}$, may be calculated recursively as
follows:

\begin{mathpar}
\freenames{\pzero} := \emptyset
  \and \\
  \freenames{x?(y).P} := \{ x \} \cup (\freenames{P} \setminus \{ y \})
  \and 
  \freenames{x!\langle P \rangle} := \{ x \} \cup \{ P \} 
  \and \\
  \freenames{P|Q} := \freenames{P} \cup \freenames{Q}
  \and \\
  \freenames{@{x}} := \{ x \}
\end{mathpar}

$\pi$
$\quotep{\pi}$

$\freenames{-} : \pi \to \mathcal{P}(\quotep{\pi})$

\begin{eqnarray*}
  \freenames{\pzero} & := & \emptyset \\
  \freenames{x?(y).P} & := & \{ x \} \cup (\freenames{P} \setminus \{ y \}) \\
  \freenames{x!\langle P \rangle} & := & \{ x \} \cup \{ P \} \\
  \freenames{P|Q} & := & \freenames{P} \cup \freenames{Q} \\
  \freenames{\dropn{x}} & := & \{ x \}
\end{eqnarray*}

The bound names of a process, $\boundnames{P}$, are those names occurring in $P$
that are not free. For example, in $x?(y).0$, the name $x$ is free, while $y$ is bound.

\begin{mathpar}
  \inferrule* [lab=monoidal-laws] {} { P|Q \equiv Q|P \and P|0 \equiv P \and P|(Q|R) \equiv (P|Q)|R }
\end{mathpar}

\begin{mathpar}
  \inferrule* [lab=alpha-equivalence] {} { (x)P \equiv (y)P\{y/x\} \and y \not\in \freenames{P} }
\end{mathpar}

\begin{definition}
Then two processes, $P,Q$, are alpha-equivalent if $P = Q\{\vec{y}/\vec{x}\}$ for
some $\vec{x} \in \boundnames{Q},\vec{y} \in \boundnames{P}$, where $Q\{\vec{y}/\vec{x}\}$
denotes the capture-avoiding substitution of $\vec{y}$ for $\vec{x}$ in $Q$.
\end{definition}

\begin{definition}
  The {\em structural congruence} \cite{SangiorgiWalker} , $\equiv$,
  between processes is the least congruence containing
  alpha-equivalence, satisfying the abelian monoid laws
  (associativity, commutativity and $\pzero$ as identity) for parallel
  composition $|$ and for summation $+$.
\end{definition}

\subsection{Name equivalence}

We take name equivalence, written $\nameeq$, to be the smallest
equivalence relation generated by the following rules.

\begin{mathpar}
\inferrule*[lab=Quote-drop]
{ }
{ \quotep{@{x}} \nameeq x }

\inferrule*[lab=Struct-equiv]
{ P \scong Q }
{ \quotep{P} \nameeq \quotep{Q} }
\end{mathpar}

The astute reader will have noticed that the mutual recursion of names
and processes imposes a mutual recursion on alpha-equivalence and
structural equivalence via name-equivalence. Fortunately, all of this
works out pleasantly and we may calculate in the natural way, free of
concern. The reader interested in the details is referred to the
appendix \ref{appendix:rho_details}.

\subsection{Substitution}

We use $\Proc$ for the set of processes, $\QProc$ for the set of
names, and $\id{\{}\vec{y} / \vec{x} \id{\}}$ to denote partial maps,
$s : \QProc \rightarrow \QProc$. A map, $s$ lifts, uniquely, to a map
on process terms, $\widehat{s} : \Proc \rightarrow \Proc$ by the
following equations.

\begin{mathpar}
  (0) \psubstp{Q}{P} := 0 \\
  (R \juxtap S) \psubstp{Q}{P}
  :=    
  (R)\psubstp{Q}{P} \juxtap (S) \psubstp{Q}{P} \\
  (x?(y).R) \psubstp{Q}{P}    
  :=    
  (x)\substp{Q}{P} (z)\concat( (R \psubstn{z}{y}) \psubstp{Q}{P} ) \\
  (\lift{x}{R}) \psubstp{Q}{P}  
  :=
  \lift{(x)\substp{Q}{P}}{ R \psubstp{Q}{P} } \\
%   (\dropn{x})  \psubstp{Q}{P}       
%   := 
%   \left\{ 
%     \begin{array}{ccc} 
%       \dropn{\quotep{Q}} & & x \nameeq \quotep{P} \\
%       \dropn{x} & & otherwise \\
%     \end{array}
%   \right. 
  (\dropn{x})  \psubstp{Q}{P}       
  := 
  \left\{ 
    \begin{array}{ccc} 
      Q & & x \nameeq \quotep{P} \\
      \dropn{x} & & otherwise \\
    \end{array}
  \right.
\end{mathpar}
 

where

\begin{eqnarray}
  (x)\id{\{} \lpquote Q \rpquote / \lpquote P \rpquote \id{\}}            = 
  \left\{ 
    \begin{array}{ccc}
      \lpquote Q \rpquote & & x \nameeq \lpquote P \rpquote \\
      x & & otherwise \\
    \end{array}
  \right. \nonumber
\end{eqnarray}

and $z$ is chosen distinct from $\quotep{P}$, $\quotep{Q}$, the free
names in $Q$, and all the names in $R$. Our $\alpha$-equivalence will
be built in the standard way from this substitution.

\begin{remark}\label{rem:no_self_referential_names}
  One consequence of these definitions is that $\forall P. \quotep{P}
  \not\in \freenames{P}$.
\end{remark}

\subsection{ Dynamic quote: an example }

Anticipating something of what's to come, consider applying the
substitution, $\widehat{\id{\{}u / z \id{\}}}$, to the following pair
of processes, $\lift{w}{y!(z)}$ and $w[ \lpquote y!(z) \rpquote ]$.

\begin{eqnarray}
	\lift{w}{y!(z)}\widehat{\id{\{}u / z \id{\}}}
		& = &
		\lift{w}{y!(u)} \nonumber\\
	w[ \lpquote y!(z) \rpquote ] \widehat{ \id{\{}u / z \id{\}} }
		& = &
		w[ \lpquote y!(z) \rpquote ] \nonumber
\end{eqnarray}

Because the body of the process between quotes is impervious to
substitution, we get radically different answers. In fact, by
examining the first process in an input context,
e.g. $x?(z).\lift{w}{y!(z)}$, we see that the process under the lift
operator may be shaped by prefixed inputs binding a name inside it. In
this sense, the lift operator will be seen as a way to dynamically
construct processes before reifying them as names.

Finally equipped with these standard features we can present the
dynamics of the calculus.

\subsubsection{Operational semantics} 

Finally, we introduce the computational dynamics. What marks these
algebras as distinct from other more traditionally studied algebraic
structures, e.g. vector spaces or polynomial rings, is the manner in
which dynamics is captured. In traditional structures, dynamics is typically
expressed through morphisms between such structures, as in linear maps
between vector spaces or morphisms between rings. In algebras
associated with the semantics of computation, the dynamics is
expressed as part of the algebraic structure itself, through a
reduction reduction relation typically denoted by $\red$. Below, we
give a recursive presentation of this relation for the calculus used
in the encoding.

$\red \subseteq \pi \times \pi$
$\red : \pi \to \mathcal{P}(\pi)$

\begin{mathpar}
  \inferrule* [lab=Comm] { \textsf{match}( x_{src}, x_{trgt} ) } { x_{trgt}?(y)P \; | \; x_{src}!\langle {Q} \rangle \red P\{\quotep{Q}/y}\} }
  \and \\
  \inferrule* [lab=Par] {{P} \red {P}'} {{{P} | {Q}} \red {{P}' | {Q}}}
  \and
  \inferrule* [lab=Equiv]{{{P} \scong {P}'} \andalso {{P}' \red {Q}'} \andalso {{Q}' \scong {Q}}}{{P} \red {Q}}
\end{mathpar}

\begin{eqnarray*}
  match_{\equiv} (\quotep{P},\quotep{Q}) & := & P \equiv Q \\
  match_{\dagger}(\quotep{P},\quotep{Q}) & := & \forall R. P|Q \red^{*} R => R \red^{*} 0 \\
  match_{K}(\quotep{P},\quotep{Q}) & := & K \mbox{ for some context } K
\end{eqnarray*}

$u?(x)P | u!\langle Q \rangle \red P\{\quotep{Q}/x\}$

%We write $\wred$ for $\red^*$, and $P\red$ if $\exists Q $ such that $ P \red Q$.
We write $P\red$ if $\exists Q $ such that $ P \red Q$ and $P\not\red$, otherwise.

\section{Replication}

As mentioned before, it is known that replication (and hence
recursion) can be implemented in a higher-order process algebra
\cite{SangiorgiWalker}. As our first example of calculation with the
machinery thus far presented we give the construction explicitly in
the {\rhoc}.

\begin{eqnarray}
	D_{x} & := & \prefix{x}{y}{(\binpar{\outputp{x}{y}}{@{y}})} \nonumber\\
	\bangp_{x}{P} & := & \binpar{{x}!\langle{\binpar{D_{x}}{P}}\rangle}{D_{x}} \nonumber
\end{eqnarray}

\begin{eqnarray}
	\bangp_{x}{P} & & \nonumber\\
	=
	& {x}!\langle{(\prefix{x}{y}{(\outputp{x}{y} | @{y})) | P}}\rangle 
	      | \prefix{x}{y}{(\outputp{x}{y} | @{y})} & \nonumber\\
	\red
	& (\outputp{x}{y} | @{y})\substn{\quotep{(\prefix{x}{y}{(@{y} | \outputp{x}{y})) | P}}}{y} & \nonumber\\
	=
	& \outputp{x}{\quotep{(\prefix{x}{y}{(\outputp{x}{y} | @{y})) | P}}}
	  | {(\prefix{x}{y}{(\outputp{x}{y} | @{y})) | P}} & \nonumber\\
	\red
	& \ldots & \nonumber\\
	\red^*
	& P | P | \ldots & \nonumber
\end{eqnarray}

Of course, this encoding, as an implementation, runs away, unfolding
$\bangp{P}$ eagerly. A lazier and more implementable replication
operator, restricted to input-guarded processes, may be obtained as follows.

\begin{eqnarray}
\bangp{\prefix{u}{v}{P}} 
	:= 
	\binpar{\lift{x}{\prefix{u}{v}{(\binpar{D(x)}{P})}}}{D(x)} \nonumber
\end{eqnarray}

\begin{remark}
  Note that the lazier definition still does not deal with summation
  or mixed summation (i.e. sums over input and output). The reader is
  invited to construct definitions of replication that deal with these
  features. 

  Further, the definitions are parameterized in a name, $x$. Can you,
  gentle reader, make a definition that eliminates this parameter and
  guarantees no accidental interaction between the replication
  machinery and the process being replicated -- i.e. no accidental
  sharing of names used by the process to get its work done and the
  name(s) used by the replication to effect copying. This latter
  revision of the definition of replication is crucial to obtaining
  the expected identity $!!P \sim !P$.
\end{remark}

\begin{remark}\label{rem:paradoxical_combinator}
  The reader familiar with the lambda calculus will have noticed the
  similarity between $D$ and the paradoxical combinator.

  [Ed. note: the existence of this seems to suggest we have to be more
  restrictive on the set of processes and names we admit if we are to
  support no-cloning.]
\end{remark}

\subsubsection{Bisimulation}

The computational dynamics gives rise to another kind of equivalence,
the equivalence of computational behavior. As previously mentioned
this is typically captured \emph{via} some form of bisimulation.

% The notion we use in this paper is weak barbed bisimulation
% \cite{milner91polyadicpi}.

The notion we use in this paper is derived from weak barbed
bisimulation \cite{milner91polyadicpi}. 

\begin{definition}
An \emph{observation relation}, $\downarrow_{\mathcal N}$, over a set
of names, $\mathcal N$, is the smallest relation satisfying the rules
below.

\infrule[Out-barb]{y \in {\mathcal N}, \; x \nameeq y}
		  {\outputp{x}{v} \downarrow_{\mathcal N} x}
\infrule[Par-barb]{\mbox{$P\downarrow_{\mathcal N} x$ or $Q\downarrow_{\mathcal N} x$}}
		  {\binpar{P}{Q} \downarrow_{\mathcal N} x}

We write $P \Downarrow_{\mathcal N} x$ if there is $Q$ such that 
$P \wred Q$ and $Q \downarrow_{\mathcal N} x$.
\end{definition}

\begin{definition}
%\label{def.bbisim}
An  ${\mathcal N}$-\emph{barbed bisimulation} over a set of names, ${\mathcal N}$, is a symmetric binary relation 
${\mathcal S}_{\mathcal N}$ between agents such that $P\rel{S}_{\mathcal N}Q$ implies:
\begin{enumerate}
\item If $P \red P'$ then $Q \wred Q'$ and $P'\rel{S}_{\mathcal N} Q'$.
\item If $P\downarrow_{\mathcal N} x$, then $Q\Downarrow_{\mathcal N} x$.
\end{enumerate}
$P$ is ${\mathcal N}$-barbed bisimilar to $Q$, written
$P \wbbisim_{\mathcal N} Q$, if $P \rel{S}_{\mathcal N} Q$ for some ${\mathcal N}$-barbed bisimulation ${\mathcal S}_{\mathcal N}$.
\end{definition}

$\mathcal{R} \subseteq \pi \times \pi$

$P \mathcal{R} Q => \forall P'. P \red P' \Rightarrow \exists Q'. Q \red Q', P' \mathcal{R} Q'$

$P \vdash x \Rightarrow Q \vdash x$

\begin{mathpar}
  \inferrule*[lab=Out-barb]{x \nameeq y}{{y}!\langle{Q}\rangle \vdash x}
  \and
  \inferrule*[lab=Par-barb]{\mbox{$P\vdash x$ or $Q\vdash x$}}{\binpar{P}{Q} \vdash x}
\end{mathpar}

\subsubsection{Contexts}

One of the principle advantages of computational calculi like the
$\pi$-calculus is a well-defined notion of context,
contextual-equivalence and a correlation between
contextual-equivalence and notions of bisimulation. The notion of
context allows the decomposition of a process into (sub-)process and
its syntactic environment, its context. Thus, a context may be
thought of as a process with a ``hole'' (written $\Box$) in it. The
application of a context $M$ to a process $P$, written $M[P]$, is
tantamount to filling the hole in $M$ with $P$. In this paper we do
not need the full weight of this theory, but do make use of the notion
of context in the proof the main theorem. 

\begin{mathpar}
  \inferrule* [lab=summation] {} {{M_{M},M_{N}} \bc \Box \;|\; x.M_{A} \;|\; M_{M}+M_{N}}
  \and
  \inferrule* [lab=agent] {} {{M_{A}} \bc (\vec{x})M_{P} \;| \; \clift{P_0,\ldots,M_{P},\ldots,P_N}}
  \and \\
  \inferrule* [lab=process] {} {{M_{P}} \bc M_{N} \;| \;P|M_{P} }
\end{mathpar} 

\begin{mathpar}
  \inferrule* [lab=sychronization] {} {M_{N} \bc \Box \;|\; x?M_{F} \;|\; x!M_{C}}
  \and
  \inferrule* [lab=abstraction] {} {{M_{F}} \bc (x)M_{P} }
  \and
  \inferrule* [lab=concretion] {} {{M_{C}} \bc \langle M_{P} \rangle }
  \and \\
  \inferrule* [lab=process] {} {{M_{P}} \bc M_{N} \;| \;P|M_{P} }
\end{mathpar}

\begin{definition}[contextual application] Given a context $M$, and
  process $P$, we define the \emph{contextual application}, $M[P] :=
  M\{P/\Box\}$. That is, the contextual application of M to P is the
  substitution of $P$ for $\Box$ in $M$.
\end{definition}

$\meaningof{-} : L \to \mathcal{P}(\pi)$

\begin{mathpar}
  \inferrule* [lab=collection] {} {\meaningof{true} = \pi, \and \meaningof{~E} = \pi \setminus \meaningof{E}, \and \meaningof{E_{1} \& E_{2}} = \meaningof{E_{1}} \cap \meaningof{E_{2}}}
\end{mathpar}

\begin{mathpar}
  \inferrule* [lab=structure] {} {\meaningof{0} = \{ P \in \pi | P \equiv 0 \}, \and \\ \meaningof{E_1 | E_2} = \{ P \in \pi | P \equiv P_{1} | P_{2}, P_{1} \in \meaningof{E_{1}}, P_{2} \in \meaningof{E_2}\} }
\end{mathpar}

\begin{mathpar}
 \inferrule* [lab=behavior] {} {\meaningof{\langle a?b \rangle E} = \{ P \in \pi | P \equiv Q | u?(y)P', \\ \and \\\\ \and \\ \;\;\; u \in \meaningof{a}, \forall z.P'\{z/y\} \in \meaningof{E\{z/b\}}\}, \and \\ \meaningof{a!E} = \{ P \in \pi | P \equiv Q | x!\langle P' \rangle, x \in \meaningof{a} P' \in \meaningof{E}\} }
\end{mathpar}

\begin{mathpar}
 \inferrule* [lab=nominal] {} {\meaningof{\quotep{E}} = \{ \quotep{P} \in \quotep{\pi} | P \in \meaningof{E} \}, \and \meaningof{\quotep{P}} = \{ \quotep{Q} \in \quotep{\pi} | P \equiv Q \} \and \\ \meaningof{@\quotep{E}} = \{ P \in \pi | P \equiv @x, x \in \meaningof{E} \}}
\end{mathpar}

\begin{eqnarray*}
  \\
  \meaningof{-} : TS \to ST
\end{eqnarray*}

\begin{eqnarray*}
  \\
  L : TS \to ST
\end{eqnarray*}

\begin{eqnarray*}
  \\
  P \models E \iff P \in \meaningof{E}
\end{eqnarray*}

\begin{eqnarray*}
  P \approx_{L} Q \iff \forall E \in L. P \models E \iff Q \models E
\end{eqnarray*}

\begin{eqnarray*}
  P \approx_{K} Q
\end{eqnarray*}

\begin{eqnarray*}
  P \approx Q
\end{eqnarray*}

$\approx_{K} = \approx = \approx_{L}$

\subsubsection{Contextual duality}

Note that contexts extend the quotation operation to a family of
operations from processes to names. Given a context, $M$, we can
define a \emph{nominal context}, $\quotep{M}$ by $\quotep{M}[P] :=
\quotep{M[P]}$. To foreshadow what is to come we observe that these
operations enjoy a duality with processes very much like the duality
between vectors and maps from vectors to scalars.

Further, because the calculus is essentially higher-order, we have a
correspondence between contexts and processes. More specifically,
given a name $x$ and a context $M$ we can construct $M^{*}_{x}$ such
that 

\begin{mathpar}
  M^{*}_{x} | \lift{x}{P} \red M[P]
\end{mathpar}

namely,

\begin{mathpar}
  M^{*}_{x} := x?(u).M[\dropn{u}]
\end{mathpar}

The dependence of $M^{*}_{x}$ on a name makes it an abstraction, 

\begin{mathpar}
  M^{*} := (x)x?(u).M[\dropn{u}]
\end{mathpar}

\subsection{Additional notation}

It will sometimes be convenient to denote the process a name
quotes. We already have the notation $x = \quotep{P}$, but it will be
convenient to introduce an alternate notation, $\procn{x}$, when we
want to emphasize the connection to the use of the name. Note that, by
virtue of name equivalence, $\quotep{\procn{x}} \nameeq x$; so, the
notation is consistent with previous definitions.

Further, because names have structure it is possible to effect
substitutions on the basis of that structure. This means we need to
upgrade our notation for substitutions, which we accomplish by
adapting comprehension notation. Thus,

\begin{mathpar}
  P\{ y / x : x \in S \}
\end{mathpar}

is interpreted to mean the process derived from P by replacing (in a
capture-avoiding manner) each occurrence of $x$ in $S$ by $y$. For example,

\begin{mathpar}
  P\{ \quotep{\procn{x}|\procn{x}} / x : x \in \freenames{P} \}
\end{mathpar}

will replace each (occurrence) of a free name $x$ in $P$ by
$\quotep{\procn{x}|\procn{x}}$.

Also, we will avail ourselves of the notation $x^{L}$ and $x^{R}$ to
denote injections of a name into disjoint copies of the name
space. There are numerous ways to accomplish this. One example can be
found in \cite{MeredithR05}. This notation overloads to vectors of
names: $\vec{x}^{\pi} := (x_{i}^{\pi} \; : \; 0 \leq i < |\vec{x}| )$ where $\pi \in \{L,R\}$.

We also use $P^{\Box} := P|\Box$.

In \cite{MeredithR05} an interpretation of the new operator is
given. It turns out that there are several possible interpretations
all enjoying the requisite algebraic properties of the operator (see
\cite{milner91polyadicpi}). We will therefore make liberal use of
$(\nu\; \vec{x})P$.

% subsection the_syntax_and_semantics_of_the_notation_system (end)   

\input{qm2pi.qmops} 

\input{qm2pi.sterngerlach} 

\input{qm2pi.metric} 

% section concurrent_process_calculi (end)

%\input{qm2pi.proofsketch}

% section proof sketch (end)

%\input{qm2pi.slviaknots} 

% section spatial logic via knots (end)

\input{qm2pi.conclusion}

% section conclusion (end)

%\input{qm2pi.dtcodes} 

% section wiring algorithm (end)

\input{qm2pi.ack} 

% section acknowledgments (end)

\newpage


\bibliographystyle{plain}   
\bibliography{../../biblios/main.bib}

\input{qm2pi.rhodetails}

\end{document}



% section front matter (end)

\section{Introduction}\label{sec:introduction} % (fold)
In this draft of the material i am going to have to dispense with the
usual writing conventions adopted in papers on these topics. i'm going
to have adopt whatever tone i need at the time i'm writing up the
calculations. Sometimes this may be very conversational; others it may
be the barest mathematical grunts; others still it may be that i have
lifted text from one of my other papers because the exposition of some
point was better said there. i hope that my readers are not unduly put
out by this decision. i'm not doing this to flout convention or be
rebellious. i find these calculations very technically challenging. To
keep everything going technically, something has to give; i have to
let go of some cognitive burden. So, the academic writing style --
with all of its trade-offs in terms of facilitating technical
communication -- is what i'm letting go of. Perhaps subsequent drafts
can be tightened and polished, but for now, i'm going to speak as if
we were sitting together in a coffee shop with a laptop, wifi and a
pad of paper and a pencil.

So, here's what i have to say. We -- you and i, comfortably ensconced
in our coffee shop and well-equipped with our tools -- can realize and
carry out the calculations of quantum mechanics over a very different
formal theory of dynamics, a formal theory of dynamics that
corresponds to a theory of concurrent computation with
\emph{reflection}. It has the advantage that the underlying theory is
already `quantized', but supports analogues all of the continuuous
operations. Strikingly, this underlying theory has recently been
connected with a notion of metric that we can show, by calculating
together, coincides with the metric induced by the inner product.

There are a lot of reasons why you might be interested in seeing
calculations of this form. Here's why i'm interested. For the past
several centuries there has been no competitor to the ``Newtonian''
account of dynamics. As a result the predominant share of accounts of
dynamical systems and situations have had to be formulated in terms of
the Newtonian machinery. i view this as an intellectually dangerous
position to occupy. Everything, despite it's intrinsic shape, turns
into a nail to be hit with this hammer. Recently, however, the theory
of computation has matured to the point where we have candidates for
theories of dynamics that offer very different perspective on
reasoning about dynamical systems and situations. Testing these
candidates against very successful accounts of dynamical situations,
like quantum mechanics, is going to give us some sense of how mature
they are and some measure of the quality of these accounts of
dynamics.

\subsection{Summary of contributions and outline of paper}

So, we're going to develop an interpretation of the operations of
quantum mechanics normally interpreted by Hilbert spaces and
operators. We're going to do this over a theory of computation. Note
that this is very different than the usual quantum computation program
which develops notions of computation over quantum mechanics. Rather,
we are developing a story that aligns with Wheeler's slogan: It from
Bit. To do this we will first provide an account of the theory of
computation at play here. Then we will dive into a calculation-driven
interpretation of the operations of quantum mechanics.

The reason we take this approach is that -- until very recently --
there hasn't been an axiomatic account of quantum mechanics. As a
result there has been no sharp delineation of the mathematical theory
supporting interpretation of the physical theory and the physical
theory, itself. So, ambient features of the maths are free to be
exploited (or supressed) without a real accounting of their physical
relevance. There is no sharp statement ``here's the physical theory''
qua \emph{theory} and ``here's the mathematical interpretation''
enabling a judgment of how faithful the interpretation is -- apart
from experimental observation. When there is an axiomatic account we
can judge how well a given mathematical formalism supports an
interpretation of the axioms, independent of
experimentation. Likewise, we can judge how well we have captured our
physical evidence and experience with our axiomatics, independent of
any specific mathematical implementation, with accidental detail that
may or may not have physical significance. 

In lieu of a fully fleshed out and vetted axiomatic account of quantum
mechanics, interpreting the operational notions in service of modeling
physical systems will have to suffice. In other words, we are not in
the business of providing a model of Hilbert spaces and operators. We
are in the business of providing a model of quantum mechanics because
we are motivated by testing our notions of dynamics against physical
theory; and, the predictive calculations of the physical theory must
serve as the best formulation -- shy of a fully fleshed out axiomatic
account -- of the physical theory itself (as they have for scientific
theories since time immemorial). Put another way, despite a
whole-hearted commitment to an It-from-Bit ontology, we are firmly
aligned with the shut-up-and-calculate camp as the best way to obtain
results either from the physical perspective or as a quality assurance
measure of our fledgling theory of dynamics.

In detail, we present a reflective process calculus. Then we develop
intuitive correspondences between the notions available in this
calculus and the usual physical notions supporting quantum mechanical
calculations. Thus, 

\begin{table}[htp]
  \center{
    \fbox{
      \begin{tabular}{c|c}
        quantum mechanics & process calculus \\
        \hline
        scalar & name \\
        state vector & process \\
        dual & contextual duals \\
        matrix & formal sums of process-context-dual pairs \\
        orthogonality & process annihilation \\
        inner product & execution-formula + quoting
      \end{tabular}
    }
  }
  \caption{QM - process calculi correspondences}
\end{table}

Then we tighten up these intuitions to operational definitions. We
employ the Dirac notation as the best proxy we can find for an
abstract syntax of the quantum mechanical notions. The definitions we
develop put us in contact with equational constraints coming from the
theory that we demonstrate the definitions and calculations satisfy.

This puts us in a position to shut up and calculate for the
Stern-Gerlach experimental set up, showing how these predictive
calculations become calculations on processes in our theory of a
reflective process calculus.

Penultimately, we demonstrate that the notion of metric coming from
the inner product coincides with the notion of metric available from
the theory of bisimulation. This demonstration gives us the right to
think of space as arising from behavior. Finally, we consider where we
might go from the new vantage point we have obtained.

% section introduction (end) 
 
% section introduction (end)

% \documentclass[12pt]{llncs}
%\documentclass{jktr}

\usepackage[pdftex]{hyperref}                   
\usepackage {listings}
\usepackage {mathpartir}
\usepackage{bcprules}
%\usepackage{listings}
                       
\usepackage{graphicx} 
%\usepackage[margins=2.5cm,nohead,nofoot]{geometry}
%\usepackage{geometry}
\usepackage{amsfonts}
\usepackage{amstext}
\usepackage{latexsym}
\usepackage{amssymb}
\usepackage{color}


%\include{myPreamble}
\include{qm2pi.local} 

%\ifpdf
%\usepackage[pdftex]{graphicx}
%\else
%\usepackage{graphicx}
%\fi

 % \ifpdf
%  \usepackage{pdfsync}
%  \if


%\title{Brief Article}
%\author{David F. Snyder}
%\author{L.G. Meredith}

%\address{Dept. of Math., Texas State University--San Marcos, San Marcos, TX 78666}
       
\pagestyle{empty}


\begin{document}

\lstset{language=[Objective]Caml,frame=shadowbox}

\input{qm2pi.front}

% section front matter (end)

\input{qm2pi.intro} 
 
% section introduction (end)

% \input{qm2pi.knotations} 

% section notation (end)

\input{qm2pi.process.calculi} 

% section concurrent_process_calculi_and_spatial_logics_ (end)
    
%\input{qm2pi.knots2pi} 

%\input{qm2pi.trefoil} 

%\input{qm2pi.mainthm} 

% subsection basic_interpretation (end)

%\input{qm2pi.rho.presentation} 
\subsection{The syntax and semantics of the notation system}\label{sub:the_syntax_and_semantics_of_the_notation_system} % (fold)

We now summarize a technical presentation of the calculus that
embodies our theory of dynamics. The typical presentation of such a
calculus follows the style of giving generators and relations on
them. The grammar, below, describing term constructors, freely
generates the set of processes, $\Proc$. This set is then quotiented
by a relation known as structural congruence and it is over this set
that the notion of dynamics is expressed. This presentation is
essentially that of \cite{MeredithR05} with the addition of
polyadicity and summation. For readability we have relegated some of
the technical subtleties to an appendix.

\subsubsection{Process grammar}\label{subsub:process_grammar}

\begin{mathpar}
  \inferrule* [lab=synchronization] {} {{M} \bc \pzero \;|\; x?F \;|\; x!C }
  \and
  \inferrule* [lab=abstraction] {} {{F} \bc (x)P}
  \and
  \inferrule* [lab=concretion] {} {{C} \bc \langle Q \rangle}
  \and
  \inferrule* [lab=process] {} {{P,Q} \bc M \;| \;P|Q \;|\; @{x}}
  \and
  \inferrule* [lab=name] {} {{x} \bc \quotep{P}}
\end{mathpar} 

Note that $\vec{x}$ (resp. $\vec{P}$) denotes a vector of names
(resp. processes) of length $|\vec{x}|$ (resp. $|\vec{P}|$). We adopt
the following useful abbreviations.

\begin{mathpar}
   x?(\vec{y}).P := x.(\vec{y})P \and  x\clift{\vec{P}} := x.\clift{\vec{P}}
   \and x!(y) := \lift{x}{\dropn{y}}
   \and \Pi_{i=0}^{n-1}P_i := P_0 | \ldots | P_{n-1}
\end{mathpar}

\subsubsection{Structural congruence}

\paragraph{Free and bound names and alpha-equivalence.} At the
core of structural equivalence is alpha-equivalence which identifies
process that are the same up to a change of variable. Formally, we
recognize the distinction between free and bound names. The free names
of a process, $\freenames{P}$, may be calculated recursively as
follows:

\begin{mathpar}
\freenames{\pzero} := \emptyset
  \and \\
  \freenames{x?(y).P} := \{ x \} \cup (\freenames{P} \setminus \{ y \})
  \and 
  \freenames{x!\langle P \rangle} := \{ x \} \cup \{ P \} 
  \and \\
  \freenames{P|Q} := \freenames{P} \cup \freenames{Q}
  \and \\
  \freenames{@{x}} := \{ x \}
\end{mathpar}

$\pi$
$\quotep{\pi}$

$\freenames{-} : \pi \to \mathcal{P}(\quotep{\pi})$

\begin{eqnarray*}
  \freenames{\pzero} & := & \emptyset \\
  \freenames{x?(y).P} & := & \{ x \} \cup (\freenames{P} \setminus \{ y \}) \\
  \freenames{x!\langle P \rangle} & := & \{ x \} \cup \{ P \} \\
  \freenames{P|Q} & := & \freenames{P} \cup \freenames{Q} \\
  \freenames{\dropn{x}} & := & \{ x \}
\end{eqnarray*}

The bound names of a process, $\boundnames{P}$, are those names occurring in $P$
that are not free. For example, in $x?(y).0$, the name $x$ is free, while $y$ is bound.

\begin{mathpar}
  \inferrule* [lab=monoidal-laws] {} { P|Q \equiv Q|P \and P|0 \equiv P \and P|(Q|R) \equiv (P|Q)|R }
\end{mathpar}

\begin{mathpar}
  \inferrule* [lab=alpha-equivalence] {} { (x)P \equiv (y)P\{y/x\} \and y \not\in \freenames{P} }
\end{mathpar}

\begin{definition}
Then two processes, $P,Q$, are alpha-equivalent if $P = Q\{\vec{y}/\vec{x}\}$ for
some $\vec{x} \in \boundnames{Q},\vec{y} \in \boundnames{P}$, where $Q\{\vec{y}/\vec{x}\}$
denotes the capture-avoiding substitution of $\vec{y}$ for $\vec{x}$ in $Q$.
\end{definition}

\begin{definition}
  The {\em structural congruence} \cite{SangiorgiWalker} , $\equiv$,
  between processes is the least congruence containing
  alpha-equivalence, satisfying the abelian monoid laws
  (associativity, commutativity and $\pzero$ as identity) for parallel
  composition $|$ and for summation $+$.
\end{definition}

\subsection{Name equivalence}

We take name equivalence, written $\nameeq$, to be the smallest
equivalence relation generated by the following rules.

\begin{mathpar}
\inferrule*[lab=Quote-drop]
{ }
{ \quotep{@{x}} \nameeq x }

\inferrule*[lab=Struct-equiv]
{ P \scong Q }
{ \quotep{P} \nameeq \quotep{Q} }
\end{mathpar}

The astute reader will have noticed that the mutual recursion of names
and processes imposes a mutual recursion on alpha-equivalence and
structural equivalence via name-equivalence. Fortunately, all of this
works out pleasantly and we may calculate in the natural way, free of
concern. The reader interested in the details is referred to the
appendix \ref{appendix:rho_details}.

\subsection{Substitution}

We use $\Proc$ for the set of processes, $\QProc$ for the set of
names, and $\id{\{}\vec{y} / \vec{x} \id{\}}$ to denote partial maps,
$s : \QProc \rightarrow \QProc$. A map, $s$ lifts, uniquely, to a map
on process terms, $\widehat{s} : \Proc \rightarrow \Proc$ by the
following equations.

\begin{mathpar}
  (0) \psubstp{Q}{P} := 0 \\
  (R \juxtap S) \psubstp{Q}{P}
  :=    
  (R)\psubstp{Q}{P} \juxtap (S) \psubstp{Q}{P} \\
  (x?(y).R) \psubstp{Q}{P}    
  :=    
  (x)\substp{Q}{P} (z)\concat( (R \psubstn{z}{y}) \psubstp{Q}{P} ) \\
  (\lift{x}{R}) \psubstp{Q}{P}  
  :=
  \lift{(x)\substp{Q}{P}}{ R \psubstp{Q}{P} } \\
%   (\dropn{x})  \psubstp{Q}{P}       
%   := 
%   \left\{ 
%     \begin{array}{ccc} 
%       \dropn{\quotep{Q}} & & x \nameeq \quotep{P} \\
%       \dropn{x} & & otherwise \\
%     \end{array}
%   \right. 
  (\dropn{x})  \psubstp{Q}{P}       
  := 
  \left\{ 
    \begin{array}{ccc} 
      Q & & x \nameeq \quotep{P} \\
      \dropn{x} & & otherwise \\
    \end{array}
  \right.
\end{mathpar}
 

where

\begin{eqnarray}
  (x)\id{\{} \lpquote Q \rpquote / \lpquote P \rpquote \id{\}}            = 
  \left\{ 
    \begin{array}{ccc}
      \lpquote Q \rpquote & & x \nameeq \lpquote P \rpquote \\
      x & & otherwise \\
    \end{array}
  \right. \nonumber
\end{eqnarray}

and $z$ is chosen distinct from $\quotep{P}$, $\quotep{Q}$, the free
names in $Q$, and all the names in $R$. Our $\alpha$-equivalence will
be built in the standard way from this substitution.

\begin{remark}\label{rem:no_self_referential_names}
  One consequence of these definitions is that $\forall P. \quotep{P}
  \not\in \freenames{P}$.
\end{remark}

\subsection{ Dynamic quote: an example }

Anticipating something of what's to come, consider applying the
substitution, $\widehat{\id{\{}u / z \id{\}}}$, to the following pair
of processes, $\lift{w}{y!(z)}$ and $w[ \lpquote y!(z) \rpquote ]$.

\begin{eqnarray}
	\lift{w}{y!(z)}\widehat{\id{\{}u / z \id{\}}}
		& = &
		\lift{w}{y!(u)} \nonumber\\
	w[ \lpquote y!(z) \rpquote ] \widehat{ \id{\{}u / z \id{\}} }
		& = &
		w[ \lpquote y!(z) \rpquote ] \nonumber
\end{eqnarray}

Because the body of the process between quotes is impervious to
substitution, we get radically different answers. In fact, by
examining the first process in an input context,
e.g. $x?(z).\lift{w}{y!(z)}$, we see that the process under the lift
operator may be shaped by prefixed inputs binding a name inside it. In
this sense, the lift operator will be seen as a way to dynamically
construct processes before reifying them as names.

Finally equipped with these standard features we can present the
dynamics of the calculus.

\subsubsection{Operational semantics} 

Finally, we introduce the computational dynamics. What marks these
algebras as distinct from other more traditionally studied algebraic
structures, e.g. vector spaces or polynomial rings, is the manner in
which dynamics is captured. In traditional structures, dynamics is typically
expressed through morphisms between such structures, as in linear maps
between vector spaces or morphisms between rings. In algebras
associated with the semantics of computation, the dynamics is
expressed as part of the algebraic structure itself, through a
reduction reduction relation typically denoted by $\red$. Below, we
give a recursive presentation of this relation for the calculus used
in the encoding.

$\red \subseteq \pi \times \pi$
$\red : \pi \to \mathcal{P}(\pi)$

\begin{mathpar}
  \inferrule* [lab=Comm] { \textsf{match}( x_{src}, x_{trgt} ) } { x_{trgt}?(y)P \; | \; x_{src}!\langle {Q} \rangle \red P\{\quotep{Q}/y}\} }
  \and \\
  \inferrule* [lab=Par] {{P} \red {P}'} {{{P} | {Q}} \red {{P}' | {Q}}}
  \and
  \inferrule* [lab=Equiv]{{{P} \scong {P}'} \andalso {{P}' \red {Q}'} \andalso {{Q}' \scong {Q}}}{{P} \red {Q}}
\end{mathpar}

\begin{eqnarray*}
  match_{\equiv} (\quotep{P},\quotep{Q}) & := & P \equiv Q \\
  match_{\dagger}(\quotep{P},\quotep{Q}) & := & \forall R. P|Q \red^{*} R => R \red^{*} 0 \\
  match_{K}(\quotep{P},\quotep{Q}) & := & K \mbox{ for some context } K
\end{eqnarray*}

$u?(x)P | u!\langle Q \rangle \red P\{\quotep{Q}/x\}$

%We write $\wred$ for $\red^*$, and $P\red$ if $\exists Q $ such that $ P \red Q$.
We write $P\red$ if $\exists Q $ such that $ P \red Q$ and $P\not\red$, otherwise.

\section{Replication}

As mentioned before, it is known that replication (and hence
recursion) can be implemented in a higher-order process algebra
\cite{SangiorgiWalker}. As our first example of calculation with the
machinery thus far presented we give the construction explicitly in
the {\rhoc}.

\begin{eqnarray}
	D_{x} & := & \prefix{x}{y}{(\binpar{\outputp{x}{y}}{@{y}})} \nonumber\\
	\bangp_{x}{P} & := & \binpar{{x}!\langle{\binpar{D_{x}}{P}}\rangle}{D_{x}} \nonumber
\end{eqnarray}

\begin{eqnarray}
	\bangp_{x}{P} & & \nonumber\\
	=
	& {x}!\langle{(\prefix{x}{y}{(\outputp{x}{y} | @{y})) | P}}\rangle 
	      | \prefix{x}{y}{(\outputp{x}{y} | @{y})} & \nonumber\\
	\red
	& (\outputp{x}{y} | @{y})\substn{\quotep{(\prefix{x}{y}{(@{y} | \outputp{x}{y})) | P}}}{y} & \nonumber\\
	=
	& \outputp{x}{\quotep{(\prefix{x}{y}{(\outputp{x}{y} | @{y})) | P}}}
	  | {(\prefix{x}{y}{(\outputp{x}{y} | @{y})) | P}} & \nonumber\\
	\red
	& \ldots & \nonumber\\
	\red^*
	& P | P | \ldots & \nonumber
\end{eqnarray}

Of course, this encoding, as an implementation, runs away, unfolding
$\bangp{P}$ eagerly. A lazier and more implementable replication
operator, restricted to input-guarded processes, may be obtained as follows.

\begin{eqnarray}
\bangp{\prefix{u}{v}{P}} 
	:= 
	\binpar{\lift{x}{\prefix{u}{v}{(\binpar{D(x)}{P})}}}{D(x)} \nonumber
\end{eqnarray}

\begin{remark}
  Note that the lazier definition still does not deal with summation
  or mixed summation (i.e. sums over input and output). The reader is
  invited to construct definitions of replication that deal with these
  features. 

  Further, the definitions are parameterized in a name, $x$. Can you,
  gentle reader, make a definition that eliminates this parameter and
  guarantees no accidental interaction between the replication
  machinery and the process being replicated -- i.e. no accidental
  sharing of names used by the process to get its work done and the
  name(s) used by the replication to effect copying. This latter
  revision of the definition of replication is crucial to obtaining
  the expected identity $!!P \sim !P$.
\end{remark}

\begin{remark}\label{rem:paradoxical_combinator}
  The reader familiar with the lambda calculus will have noticed the
  similarity between $D$ and the paradoxical combinator.

  [Ed. note: the existence of this seems to suggest we have to be more
  restrictive on the set of processes and names we admit if we are to
  support no-cloning.]
\end{remark}

\subsubsection{Bisimulation}

The computational dynamics gives rise to another kind of equivalence,
the equivalence of computational behavior. As previously mentioned
this is typically captured \emph{via} some form of bisimulation.

% The notion we use in this paper is weak barbed bisimulation
% \cite{milner91polyadicpi}.

The notion we use in this paper is derived from weak barbed
bisimulation \cite{milner91polyadicpi}. 

\begin{definition}
An \emph{observation relation}, $\downarrow_{\mathcal N}$, over a set
of names, $\mathcal N$, is the smallest relation satisfying the rules
below.

\infrule[Out-barb]{y \in {\mathcal N}, \; x \nameeq y}
		  {\outputp{x}{v} \downarrow_{\mathcal N} x}
\infrule[Par-barb]{\mbox{$P\downarrow_{\mathcal N} x$ or $Q\downarrow_{\mathcal N} x$}}
		  {\binpar{P}{Q} \downarrow_{\mathcal N} x}

We write $P \Downarrow_{\mathcal N} x$ if there is $Q$ such that 
$P \wred Q$ and $Q \downarrow_{\mathcal N} x$.
\end{definition}

\begin{definition}
%\label{def.bbisim}
An  ${\mathcal N}$-\emph{barbed bisimulation} over a set of names, ${\mathcal N}$, is a symmetric binary relation 
${\mathcal S}_{\mathcal N}$ between agents such that $P\rel{S}_{\mathcal N}Q$ implies:
\begin{enumerate}
\item If $P \red P'$ then $Q \wred Q'$ and $P'\rel{S}_{\mathcal N} Q'$.
\item If $P\downarrow_{\mathcal N} x$, then $Q\Downarrow_{\mathcal N} x$.
\end{enumerate}
$P$ is ${\mathcal N}$-barbed bisimilar to $Q$, written
$P \wbbisim_{\mathcal N} Q$, if $P \rel{S}_{\mathcal N} Q$ for some ${\mathcal N}$-barbed bisimulation ${\mathcal S}_{\mathcal N}$.
\end{definition}

$\mathcal{R} \subseteq \pi \times \pi$

$P \mathcal{R} Q => \forall P'. P \red P' \Rightarrow \exists Q'. Q \red Q', P' \mathcal{R} Q'$

$P \vdash x \Rightarrow Q \vdash x$

\begin{mathpar}
  \inferrule*[lab=Out-barb]{x \nameeq y}{{y}!\langle{Q}\rangle \vdash x}
  \and
  \inferrule*[lab=Par-barb]{\mbox{$P\vdash x$ or $Q\vdash x$}}{\binpar{P}{Q} \vdash x}
\end{mathpar}

\subsubsection{Contexts}

One of the principle advantages of computational calculi like the
$\pi$-calculus is a well-defined notion of context,
contextual-equivalence and a correlation between
contextual-equivalence and notions of bisimulation. The notion of
context allows the decomposition of a process into (sub-)process and
its syntactic environment, its context. Thus, a context may be
thought of as a process with a ``hole'' (written $\Box$) in it. The
application of a context $M$ to a process $P$, written $M[P]$, is
tantamount to filling the hole in $M$ with $P$. In this paper we do
not need the full weight of this theory, but do make use of the notion
of context in the proof the main theorem. 

\begin{mathpar}
  \inferrule* [lab=summation] {} {{M_{M},M_{N}} \bc \Box \;|\; x.M_{A} \;|\; M_{M}+M_{N}}
  \and
  \inferrule* [lab=agent] {} {{M_{A}} \bc (\vec{x})M_{P} \;| \; \clift{P_0,\ldots,M_{P},\ldots,P_N}}
  \and \\
  \inferrule* [lab=process] {} {{M_{P}} \bc M_{N} \;| \;P|M_{P} }
\end{mathpar} 

\begin{mathpar}
  \inferrule* [lab=sychronization] {} {M_{N} \bc \Box \;|\; x?M_{F} \;|\; x!M_{C}}
  \and
  \inferrule* [lab=abstraction] {} {{M_{F}} \bc (x)M_{P} }
  \and
  \inferrule* [lab=concretion] {} {{M_{C}} \bc \langle M_{P} \rangle }
  \and \\
  \inferrule* [lab=process] {} {{M_{P}} \bc M_{N} \;| \;P|M_{P} }
\end{mathpar}

\begin{definition}[contextual application] Given a context $M$, and
  process $P$, we define the \emph{contextual application}, $M[P] :=
  M\{P/\Box\}$. That is, the contextual application of M to P is the
  substitution of $P$ for $\Box$ in $M$.
\end{definition}

$\meaningof{-} : L \to \mathcal{P}(\pi)$

\begin{mathpar}
  \inferrule* [lab=collection] {} {\meaningof{true} = \pi, \and \meaningof{~E} = \pi \setminus \meaningof{E}, \and \meaningof{E_{1} \& E_{2}} = \meaningof{E_{1}} \cap \meaningof{E_{2}}}
\end{mathpar}

\begin{mathpar}
  \inferrule* [lab=structure] {} {\meaningof{0} = \{ P \in \pi | P \equiv 0 \}, \and \\ \meaningof{E_1 | E_2} = \{ P \in \pi | P \equiv P_{1} | P_{2}, P_{1} \in \meaningof{E_{1}}, P_{2} \in \meaningof{E_2}\} }
\end{mathpar}

\begin{mathpar}
 \inferrule* [lab=behavior] {} {\meaningof{\langle a?b \rangle E} = \{ P \in \pi | P \equiv Q | u?(y)P', \\ \and \\\\ \and \\ \;\;\; u \in \meaningof{a}, \forall z.P'\{z/y\} \in \meaningof{E\{z/b\}}\}, \and \\ \meaningof{a!E} = \{ P \in \pi | P \equiv Q | x!\langle P' \rangle, x \in \meaningof{a} P' \in \meaningof{E}\} }
\end{mathpar}

\begin{mathpar}
 \inferrule* [lab=nominal] {} {\meaningof{\quotep{E}} = \{ \quotep{P} \in \quotep{\pi} | P \in \meaningof{E} \}, \and \meaningof{\quotep{P}} = \{ \quotep{Q} \in \quotep{\pi} | P \equiv Q \} \and \\ \meaningof{@\quotep{E}} = \{ P \in \pi | P \equiv @x, x \in \meaningof{E} \}}
\end{mathpar}

\begin{eqnarray*}
  \\
  \meaningof{-} : TS \to ST
\end{eqnarray*}

\begin{eqnarray*}
  \\
  L : TS \to ST
\end{eqnarray*}

\begin{eqnarray*}
  \\
  P \models E \iff P \in \meaningof{E}
\end{eqnarray*}

\begin{eqnarray*}
  P \approx_{L} Q \iff \forall E \in L. P \models E \iff Q \models E
\end{eqnarray*}

\begin{eqnarray*}
  P \approx_{K} Q
\end{eqnarray*}

\begin{eqnarray*}
  P \approx Q
\end{eqnarray*}

$\approx_{K} = \approx = \approx_{L}$

\subsubsection{Contextual duality}

Note that contexts extend the quotation operation to a family of
operations from processes to names. Given a context, $M$, we can
define a \emph{nominal context}, $\quotep{M}$ by $\quotep{M}[P] :=
\quotep{M[P]}$. To foreshadow what is to come we observe that these
operations enjoy a duality with processes very much like the duality
between vectors and maps from vectors to scalars.

Further, because the calculus is essentially higher-order, we have a
correspondence between contexts and processes. More specifically,
given a name $x$ and a context $M$ we can construct $M^{*}_{x}$ such
that 

\begin{mathpar}
  M^{*}_{x} | \lift{x}{P} \red M[P]
\end{mathpar}

namely,

\begin{mathpar}
  M^{*}_{x} := x?(u).M[\dropn{u}]
\end{mathpar}

The dependence of $M^{*}_{x}$ on a name makes it an abstraction, 

\begin{mathpar}
  M^{*} := (x)x?(u).M[\dropn{u}]
\end{mathpar}

\subsection{Additional notation}

It will sometimes be convenient to denote the process a name
quotes. We already have the notation $x = \quotep{P}$, but it will be
convenient to introduce an alternate notation, $\procn{x}$, when we
want to emphasize the connection to the use of the name. Note that, by
virtue of name equivalence, $\quotep{\procn{x}} \nameeq x$; so, the
notation is consistent with previous definitions.

Further, because names have structure it is possible to effect
substitutions on the basis of that structure. This means we need to
upgrade our notation for substitutions, which we accomplish by
adapting comprehension notation. Thus,

\begin{mathpar}
  P\{ y / x : x \in S \}
\end{mathpar}

is interpreted to mean the process derived from P by replacing (in a
capture-avoiding manner) each occurrence of $x$ in $S$ by $y$. For example,

\begin{mathpar}
  P\{ \quotep{\procn{x}|\procn{x}} / x : x \in \freenames{P} \}
\end{mathpar}

will replace each (occurrence) of a free name $x$ in $P$ by
$\quotep{\procn{x}|\procn{x}}$.

Also, we will avail ourselves of the notation $x^{L}$ and $x^{R}$ to
denote injections of a name into disjoint copies of the name
space. There are numerous ways to accomplish this. One example can be
found in \cite{MeredithR05}. This notation overloads to vectors of
names: $\vec{x}^{\pi} := (x_{i}^{\pi} \; : \; 0 \leq i < |\vec{x}| )$ where $\pi \in \{L,R\}$.

We also use $P^{\Box} := P|\Box$.

In \cite{MeredithR05} an interpretation of the new operator is
given. It turns out that there are several possible interpretations
all enjoying the requisite algebraic properties of the operator (see
\cite{milner91polyadicpi}). We will therefore make liberal use of
$(\nu\; \vec{x})P$.

% subsection the_syntax_and_semantics_of_the_notation_system (end)   

\input{qm2pi.qmops} 

\input{qm2pi.sterngerlach} 

\input{qm2pi.metric} 

% section concurrent_process_calculi (end)

%\input{qm2pi.proofsketch}

% section proof sketch (end)

%\input{qm2pi.slviaknots} 

% section spatial logic via knots (end)

\input{qm2pi.conclusion}

% section conclusion (end)

%\input{qm2pi.dtcodes} 

% section wiring algorithm (end)

\input{qm2pi.ack} 

% section acknowledgments (end)

\newpage


\bibliographystyle{plain}   
\bibliography{../../biblios/main.bib}

\input{qm2pi.rhodetails}

\end{document}

 

% section notation (end)

\input{qm2pi.process.calculi} 

% section concurrent_process_calculi_and_spatial_logics_ (end)
    
%\documentclass[12pt]{llncs}
%\documentclass{jktr}

\usepackage[pdftex]{hyperref}                   
\usepackage {listings}
\usepackage {mathpartir}
\usepackage{bcprules}
%\usepackage{listings}
                       
\usepackage{graphicx} 
%\usepackage[margins=2.5cm,nohead,nofoot]{geometry}
%\usepackage{geometry}
\usepackage{amsfonts}
\usepackage{amstext}
\usepackage{latexsym}
\usepackage{amssymb}
\usepackage{color}


%\include{myPreamble}
\include{qm2pi.local} 

%\ifpdf
%\usepackage[pdftex]{graphicx}
%\else
%\usepackage{graphicx}
%\fi

 % \ifpdf
%  \usepackage{pdfsync}
%  \if


%\title{Brief Article}
%\author{David F. Snyder}
%\author{L.G. Meredith}

%\address{Dept. of Math., Texas State University--San Marcos, San Marcos, TX 78666}
       
\pagestyle{empty}


\begin{document}

\lstset{language=[Objective]Caml,frame=shadowbox}

\input{qm2pi.front}

% section front matter (end)

\input{qm2pi.intro} 
 
% section introduction (end)

% \input{qm2pi.knotations} 

% section notation (end)

\input{qm2pi.process.calculi} 

% section concurrent_process_calculi_and_spatial_logics_ (end)
    
%\input{qm2pi.knots2pi} 

%\input{qm2pi.trefoil} 

%\input{qm2pi.mainthm} 

% subsection basic_interpretation (end)

%\input{qm2pi.rho.presentation} 
\subsection{The syntax and semantics of the notation system}\label{sub:the_syntax_and_semantics_of_the_notation_system} % (fold)

We now summarize a technical presentation of the calculus that
embodies our theory of dynamics. The typical presentation of such a
calculus follows the style of giving generators and relations on
them. The grammar, below, describing term constructors, freely
generates the set of processes, $\Proc$. This set is then quotiented
by a relation known as structural congruence and it is over this set
that the notion of dynamics is expressed. This presentation is
essentially that of \cite{MeredithR05} with the addition of
polyadicity and summation. For readability we have relegated some of
the technical subtleties to an appendix.

\subsubsection{Process grammar}\label{subsub:process_grammar}

\begin{mathpar}
  \inferrule* [lab=synchronization] {} {{M} \bc \pzero \;|\; x?F \;|\; x!C }
  \and
  \inferrule* [lab=abstraction] {} {{F} \bc (x)P}
  \and
  \inferrule* [lab=concretion] {} {{C} \bc \langle Q \rangle}
  \and
  \inferrule* [lab=process] {} {{P,Q} \bc M \;| \;P|Q \;|\; @{x}}
  \and
  \inferrule* [lab=name] {} {{x} \bc \quotep{P}}
\end{mathpar} 

Note that $\vec{x}$ (resp. $\vec{P}$) denotes a vector of names
(resp. processes) of length $|\vec{x}|$ (resp. $|\vec{P}|$). We adopt
the following useful abbreviations.

\begin{mathpar}
   x?(\vec{y}).P := x.(\vec{y})P \and  x\clift{\vec{P}} := x.\clift{\vec{P}}
   \and x!(y) := \lift{x}{\dropn{y}}
   \and \Pi_{i=0}^{n-1}P_i := P_0 | \ldots | P_{n-1}
\end{mathpar}

\subsubsection{Structural congruence}

\paragraph{Free and bound names and alpha-equivalence.} At the
core of structural equivalence is alpha-equivalence which identifies
process that are the same up to a change of variable. Formally, we
recognize the distinction between free and bound names. The free names
of a process, $\freenames{P}$, may be calculated recursively as
follows:

\begin{mathpar}
\freenames{\pzero} := \emptyset
  \and \\
  \freenames{x?(y).P} := \{ x \} \cup (\freenames{P} \setminus \{ y \})
  \and 
  \freenames{x!\langle P \rangle} := \{ x \} \cup \{ P \} 
  \and \\
  \freenames{P|Q} := \freenames{P} \cup \freenames{Q}
  \and \\
  \freenames{@{x}} := \{ x \}
\end{mathpar}

$\pi$
$\quotep{\pi}$

$\freenames{-} : \pi \to \mathcal{P}(\quotep{\pi})$

\begin{eqnarray*}
  \freenames{\pzero} & := & \emptyset \\
  \freenames{x?(y).P} & := & \{ x \} \cup (\freenames{P} \setminus \{ y \}) \\
  \freenames{x!\langle P \rangle} & := & \{ x \} \cup \{ P \} \\
  \freenames{P|Q} & := & \freenames{P} \cup \freenames{Q} \\
  \freenames{\dropn{x}} & := & \{ x \}
\end{eqnarray*}

The bound names of a process, $\boundnames{P}$, are those names occurring in $P$
that are not free. For example, in $x?(y).0$, the name $x$ is free, while $y$ is bound.

\begin{mathpar}
  \inferrule* [lab=monoidal-laws] {} { P|Q \equiv Q|P \and P|0 \equiv P \and P|(Q|R) \equiv (P|Q)|R }
\end{mathpar}

\begin{mathpar}
  \inferrule* [lab=alpha-equivalence] {} { (x)P \equiv (y)P\{y/x\} \and y \not\in \freenames{P} }
\end{mathpar}

\begin{definition}
Then two processes, $P,Q$, are alpha-equivalent if $P = Q\{\vec{y}/\vec{x}\}$ for
some $\vec{x} \in \boundnames{Q},\vec{y} \in \boundnames{P}$, where $Q\{\vec{y}/\vec{x}\}$
denotes the capture-avoiding substitution of $\vec{y}$ for $\vec{x}$ in $Q$.
\end{definition}

\begin{definition}
  The {\em structural congruence} \cite{SangiorgiWalker} , $\equiv$,
  between processes is the least congruence containing
  alpha-equivalence, satisfying the abelian monoid laws
  (associativity, commutativity and $\pzero$ as identity) for parallel
  composition $|$ and for summation $+$.
\end{definition}

\subsection{Name equivalence}

We take name equivalence, written $\nameeq$, to be the smallest
equivalence relation generated by the following rules.

\begin{mathpar}
\inferrule*[lab=Quote-drop]
{ }
{ \quotep{@{x}} \nameeq x }

\inferrule*[lab=Struct-equiv]
{ P \scong Q }
{ \quotep{P} \nameeq \quotep{Q} }
\end{mathpar}

The astute reader will have noticed that the mutual recursion of names
and processes imposes a mutual recursion on alpha-equivalence and
structural equivalence via name-equivalence. Fortunately, all of this
works out pleasantly and we may calculate in the natural way, free of
concern. The reader interested in the details is referred to the
appendix \ref{appendix:rho_details}.

\subsection{Substitution}

We use $\Proc$ for the set of processes, $\QProc$ for the set of
names, and $\id{\{}\vec{y} / \vec{x} \id{\}}$ to denote partial maps,
$s : \QProc \rightarrow \QProc$. A map, $s$ lifts, uniquely, to a map
on process terms, $\widehat{s} : \Proc \rightarrow \Proc$ by the
following equations.

\begin{mathpar}
  (0) \psubstp{Q}{P} := 0 \\
  (R \juxtap S) \psubstp{Q}{P}
  :=    
  (R)\psubstp{Q}{P} \juxtap (S) \psubstp{Q}{P} \\
  (x?(y).R) \psubstp{Q}{P}    
  :=    
  (x)\substp{Q}{P} (z)\concat( (R \psubstn{z}{y}) \psubstp{Q}{P} ) \\
  (\lift{x}{R}) \psubstp{Q}{P}  
  :=
  \lift{(x)\substp{Q}{P}}{ R \psubstp{Q}{P} } \\
%   (\dropn{x})  \psubstp{Q}{P}       
%   := 
%   \left\{ 
%     \begin{array}{ccc} 
%       \dropn{\quotep{Q}} & & x \nameeq \quotep{P} \\
%       \dropn{x} & & otherwise \\
%     \end{array}
%   \right. 
  (\dropn{x})  \psubstp{Q}{P}       
  := 
  \left\{ 
    \begin{array}{ccc} 
      Q & & x \nameeq \quotep{P} \\
      \dropn{x} & & otherwise \\
    \end{array}
  \right.
\end{mathpar}
 

where

\begin{eqnarray}
  (x)\id{\{} \lpquote Q \rpquote / \lpquote P \rpquote \id{\}}            = 
  \left\{ 
    \begin{array}{ccc}
      \lpquote Q \rpquote & & x \nameeq \lpquote P \rpquote \\
      x & & otherwise \\
    \end{array}
  \right. \nonumber
\end{eqnarray}

and $z$ is chosen distinct from $\quotep{P}$, $\quotep{Q}$, the free
names in $Q$, and all the names in $R$. Our $\alpha$-equivalence will
be built in the standard way from this substitution.

\begin{remark}\label{rem:no_self_referential_names}
  One consequence of these definitions is that $\forall P. \quotep{P}
  \not\in \freenames{P}$.
\end{remark}

\subsection{ Dynamic quote: an example }

Anticipating something of what's to come, consider applying the
substitution, $\widehat{\id{\{}u / z \id{\}}}$, to the following pair
of processes, $\lift{w}{y!(z)}$ and $w[ \lpquote y!(z) \rpquote ]$.

\begin{eqnarray}
	\lift{w}{y!(z)}\widehat{\id{\{}u / z \id{\}}}
		& = &
		\lift{w}{y!(u)} \nonumber\\
	w[ \lpquote y!(z) \rpquote ] \widehat{ \id{\{}u / z \id{\}} }
		& = &
		w[ \lpquote y!(z) \rpquote ] \nonumber
\end{eqnarray}

Because the body of the process between quotes is impervious to
substitution, we get radically different answers. In fact, by
examining the first process in an input context,
e.g. $x?(z).\lift{w}{y!(z)}$, we see that the process under the lift
operator may be shaped by prefixed inputs binding a name inside it. In
this sense, the lift operator will be seen as a way to dynamically
construct processes before reifying them as names.

Finally equipped with these standard features we can present the
dynamics of the calculus.

\subsubsection{Operational semantics} 

Finally, we introduce the computational dynamics. What marks these
algebras as distinct from other more traditionally studied algebraic
structures, e.g. vector spaces or polynomial rings, is the manner in
which dynamics is captured. In traditional structures, dynamics is typically
expressed through morphisms between such structures, as in linear maps
between vector spaces or morphisms between rings. In algebras
associated with the semantics of computation, the dynamics is
expressed as part of the algebraic structure itself, through a
reduction reduction relation typically denoted by $\red$. Below, we
give a recursive presentation of this relation for the calculus used
in the encoding.

$\red \subseteq \pi \times \pi$
$\red : \pi \to \mathcal{P}(\pi)$

\begin{mathpar}
  \inferrule* [lab=Comm] { \textsf{match}( x_{src}, x_{trgt} ) } { x_{trgt}?(y)P \; | \; x_{src}!\langle {Q} \rangle \red P\{\quotep{Q}/y}\} }
  \and \\
  \inferrule* [lab=Par] {{P} \red {P}'} {{{P} | {Q}} \red {{P}' | {Q}}}
  \and
  \inferrule* [lab=Equiv]{{{P} \scong {P}'} \andalso {{P}' \red {Q}'} \andalso {{Q}' \scong {Q}}}{{P} \red {Q}}
\end{mathpar}

\begin{eqnarray*}
  match_{\equiv} (\quotep{P},\quotep{Q}) & := & P \equiv Q \\
  match_{\dagger}(\quotep{P},\quotep{Q}) & := & \forall R. P|Q \red^{*} R => R \red^{*} 0 \\
  match_{K}(\quotep{P},\quotep{Q}) & := & K \mbox{ for some context } K
\end{eqnarray*}

$u?(x)P | u!\langle Q \rangle \red P\{\quotep{Q}/x\}$

%We write $\wred$ for $\red^*$, and $P\red$ if $\exists Q $ such that $ P \red Q$.
We write $P\red$ if $\exists Q $ such that $ P \red Q$ and $P\not\red$, otherwise.

\section{Replication}

As mentioned before, it is known that replication (and hence
recursion) can be implemented in a higher-order process algebra
\cite{SangiorgiWalker}. As our first example of calculation with the
machinery thus far presented we give the construction explicitly in
the {\rhoc}.

\begin{eqnarray}
	D_{x} & := & \prefix{x}{y}{(\binpar{\outputp{x}{y}}{@{y}})} \nonumber\\
	\bangp_{x}{P} & := & \binpar{{x}!\langle{\binpar{D_{x}}{P}}\rangle}{D_{x}} \nonumber
\end{eqnarray}

\begin{eqnarray}
	\bangp_{x}{P} & & \nonumber\\
	=
	& {x}!\langle{(\prefix{x}{y}{(\outputp{x}{y} | @{y})) | P}}\rangle 
	      | \prefix{x}{y}{(\outputp{x}{y} | @{y})} & \nonumber\\
	\red
	& (\outputp{x}{y} | @{y})\substn{\quotep{(\prefix{x}{y}{(@{y} | \outputp{x}{y})) | P}}}{y} & \nonumber\\
	=
	& \outputp{x}{\quotep{(\prefix{x}{y}{(\outputp{x}{y} | @{y})) | P}}}
	  | {(\prefix{x}{y}{(\outputp{x}{y} | @{y})) | P}} & \nonumber\\
	\red
	& \ldots & \nonumber\\
	\red^*
	& P | P | \ldots & \nonumber
\end{eqnarray}

Of course, this encoding, as an implementation, runs away, unfolding
$\bangp{P}$ eagerly. A lazier and more implementable replication
operator, restricted to input-guarded processes, may be obtained as follows.

\begin{eqnarray}
\bangp{\prefix{u}{v}{P}} 
	:= 
	\binpar{\lift{x}{\prefix{u}{v}{(\binpar{D(x)}{P})}}}{D(x)} \nonumber
\end{eqnarray}

\begin{remark}
  Note that the lazier definition still does not deal with summation
  or mixed summation (i.e. sums over input and output). The reader is
  invited to construct definitions of replication that deal with these
  features. 

  Further, the definitions are parameterized in a name, $x$. Can you,
  gentle reader, make a definition that eliminates this parameter and
  guarantees no accidental interaction between the replication
  machinery and the process being replicated -- i.e. no accidental
  sharing of names used by the process to get its work done and the
  name(s) used by the replication to effect copying. This latter
  revision of the definition of replication is crucial to obtaining
  the expected identity $!!P \sim !P$.
\end{remark}

\begin{remark}\label{rem:paradoxical_combinator}
  The reader familiar with the lambda calculus will have noticed the
  similarity between $D$ and the paradoxical combinator.

  [Ed. note: the existence of this seems to suggest we have to be more
  restrictive on the set of processes and names we admit if we are to
  support no-cloning.]
\end{remark}

\subsubsection{Bisimulation}

The computational dynamics gives rise to another kind of equivalence,
the equivalence of computational behavior. As previously mentioned
this is typically captured \emph{via} some form of bisimulation.

% The notion we use in this paper is weak barbed bisimulation
% \cite{milner91polyadicpi}.

The notion we use in this paper is derived from weak barbed
bisimulation \cite{milner91polyadicpi}. 

\begin{definition}
An \emph{observation relation}, $\downarrow_{\mathcal N}$, over a set
of names, $\mathcal N$, is the smallest relation satisfying the rules
below.

\infrule[Out-barb]{y \in {\mathcal N}, \; x \nameeq y}
		  {\outputp{x}{v} \downarrow_{\mathcal N} x}
\infrule[Par-barb]{\mbox{$P\downarrow_{\mathcal N} x$ or $Q\downarrow_{\mathcal N} x$}}
		  {\binpar{P}{Q} \downarrow_{\mathcal N} x}

We write $P \Downarrow_{\mathcal N} x$ if there is $Q$ such that 
$P \wred Q$ and $Q \downarrow_{\mathcal N} x$.
\end{definition}

\begin{definition}
%\label{def.bbisim}
An  ${\mathcal N}$-\emph{barbed bisimulation} over a set of names, ${\mathcal N}$, is a symmetric binary relation 
${\mathcal S}_{\mathcal N}$ between agents such that $P\rel{S}_{\mathcal N}Q$ implies:
\begin{enumerate}
\item If $P \red P'$ then $Q \wred Q'$ and $P'\rel{S}_{\mathcal N} Q'$.
\item If $P\downarrow_{\mathcal N} x$, then $Q\Downarrow_{\mathcal N} x$.
\end{enumerate}
$P$ is ${\mathcal N}$-barbed bisimilar to $Q$, written
$P \wbbisim_{\mathcal N} Q$, if $P \rel{S}_{\mathcal N} Q$ for some ${\mathcal N}$-barbed bisimulation ${\mathcal S}_{\mathcal N}$.
\end{definition}

$\mathcal{R} \subseteq \pi \times \pi$

$P \mathcal{R} Q => \forall P'. P \red P' \Rightarrow \exists Q'. Q \red Q', P' \mathcal{R} Q'$

$P \vdash x \Rightarrow Q \vdash x$

\begin{mathpar}
  \inferrule*[lab=Out-barb]{x \nameeq y}{{y}!\langle{Q}\rangle \vdash x}
  \and
  \inferrule*[lab=Par-barb]{\mbox{$P\vdash x$ or $Q\vdash x$}}{\binpar{P}{Q} \vdash x}
\end{mathpar}

\subsubsection{Contexts}

One of the principle advantages of computational calculi like the
$\pi$-calculus is a well-defined notion of context,
contextual-equivalence and a correlation between
contextual-equivalence and notions of bisimulation. The notion of
context allows the decomposition of a process into (sub-)process and
its syntactic environment, its context. Thus, a context may be
thought of as a process with a ``hole'' (written $\Box$) in it. The
application of a context $M$ to a process $P$, written $M[P]$, is
tantamount to filling the hole in $M$ with $P$. In this paper we do
not need the full weight of this theory, but do make use of the notion
of context in the proof the main theorem. 

\begin{mathpar}
  \inferrule* [lab=summation] {} {{M_{M},M_{N}} \bc \Box \;|\; x.M_{A} \;|\; M_{M}+M_{N}}
  \and
  \inferrule* [lab=agent] {} {{M_{A}} \bc (\vec{x})M_{P} \;| \; \clift{P_0,\ldots,M_{P},\ldots,P_N}}
  \and \\
  \inferrule* [lab=process] {} {{M_{P}} \bc M_{N} \;| \;P|M_{P} }
\end{mathpar} 

\begin{mathpar}
  \inferrule* [lab=sychronization] {} {M_{N} \bc \Box \;|\; x?M_{F} \;|\; x!M_{C}}
  \and
  \inferrule* [lab=abstraction] {} {{M_{F}} \bc (x)M_{P} }
  \and
  \inferrule* [lab=concretion] {} {{M_{C}} \bc \langle M_{P} \rangle }
  \and \\
  \inferrule* [lab=process] {} {{M_{P}} \bc M_{N} \;| \;P|M_{P} }
\end{mathpar}

\begin{definition}[contextual application] Given a context $M$, and
  process $P$, we define the \emph{contextual application}, $M[P] :=
  M\{P/\Box\}$. That is, the contextual application of M to P is the
  substitution of $P$ for $\Box$ in $M$.
\end{definition}

$\meaningof{-} : L \to \mathcal{P}(\pi)$

\begin{mathpar}
  \inferrule* [lab=collection] {} {\meaningof{true} = \pi, \and \meaningof{~E} = \pi \setminus \meaningof{E}, \and \meaningof{E_{1} \& E_{2}} = \meaningof{E_{1}} \cap \meaningof{E_{2}}}
\end{mathpar}

\begin{mathpar}
  \inferrule* [lab=structure] {} {\meaningof{0} = \{ P \in \pi | P \equiv 0 \}, \and \\ \meaningof{E_1 | E_2} = \{ P \in \pi | P \equiv P_{1} | P_{2}, P_{1} \in \meaningof{E_{1}}, P_{2} \in \meaningof{E_2}\} }
\end{mathpar}

\begin{mathpar}
 \inferrule* [lab=behavior] {} {\meaningof{\langle a?b \rangle E} = \{ P \in \pi | P \equiv Q | u?(y)P', \\ \and \\\\ \and \\ \;\;\; u \in \meaningof{a}, \forall z.P'\{z/y\} \in \meaningof{E\{z/b\}}\}, \and \\ \meaningof{a!E} = \{ P \in \pi | P \equiv Q | x!\langle P' \rangle, x \in \meaningof{a} P' \in \meaningof{E}\} }
\end{mathpar}

\begin{mathpar}
 \inferrule* [lab=nominal] {} {\meaningof{\quotep{E}} = \{ \quotep{P} \in \quotep{\pi} | P \in \meaningof{E} \}, \and \meaningof{\quotep{P}} = \{ \quotep{Q} \in \quotep{\pi} | P \equiv Q \} \and \\ \meaningof{@\quotep{E}} = \{ P \in \pi | P \equiv @x, x \in \meaningof{E} \}}
\end{mathpar}

\begin{eqnarray*}
  \\
  \meaningof{-} : TS \to ST
\end{eqnarray*}

\begin{eqnarray*}
  \\
  L : TS \to ST
\end{eqnarray*}

\begin{eqnarray*}
  \\
  P \models E \iff P \in \meaningof{E}
\end{eqnarray*}

\begin{eqnarray*}
  P \approx_{L} Q \iff \forall E \in L. P \models E \iff Q \models E
\end{eqnarray*}

\begin{eqnarray*}
  P \approx_{K} Q
\end{eqnarray*}

\begin{eqnarray*}
  P \approx Q
\end{eqnarray*}

$\approx_{K} = \approx = \approx_{L}$

\subsubsection{Contextual duality}

Note that contexts extend the quotation operation to a family of
operations from processes to names. Given a context, $M$, we can
define a \emph{nominal context}, $\quotep{M}$ by $\quotep{M}[P] :=
\quotep{M[P]}$. To foreshadow what is to come we observe that these
operations enjoy a duality with processes very much like the duality
between vectors and maps from vectors to scalars.

Further, because the calculus is essentially higher-order, we have a
correspondence between contexts and processes. More specifically,
given a name $x$ and a context $M$ we can construct $M^{*}_{x}$ such
that 

\begin{mathpar}
  M^{*}_{x} | \lift{x}{P} \red M[P]
\end{mathpar}

namely,

\begin{mathpar}
  M^{*}_{x} := x?(u).M[\dropn{u}]
\end{mathpar}

The dependence of $M^{*}_{x}$ on a name makes it an abstraction, 

\begin{mathpar}
  M^{*} := (x)x?(u).M[\dropn{u}]
\end{mathpar}

\subsection{Additional notation}

It will sometimes be convenient to denote the process a name
quotes. We already have the notation $x = \quotep{P}$, but it will be
convenient to introduce an alternate notation, $\procn{x}$, when we
want to emphasize the connection to the use of the name. Note that, by
virtue of name equivalence, $\quotep{\procn{x}} \nameeq x$; so, the
notation is consistent with previous definitions.

Further, because names have structure it is possible to effect
substitutions on the basis of that structure. This means we need to
upgrade our notation for substitutions, which we accomplish by
adapting comprehension notation. Thus,

\begin{mathpar}
  P\{ y / x : x \in S \}
\end{mathpar}

is interpreted to mean the process derived from P by replacing (in a
capture-avoiding manner) each occurrence of $x$ in $S$ by $y$. For example,

\begin{mathpar}
  P\{ \quotep{\procn{x}|\procn{x}} / x : x \in \freenames{P} \}
\end{mathpar}

will replace each (occurrence) of a free name $x$ in $P$ by
$\quotep{\procn{x}|\procn{x}}$.

Also, we will avail ourselves of the notation $x^{L}$ and $x^{R}$ to
denote injections of a name into disjoint copies of the name
space. There are numerous ways to accomplish this. One example can be
found in \cite{MeredithR05}. This notation overloads to vectors of
names: $\vec{x}^{\pi} := (x_{i}^{\pi} \; : \; 0 \leq i < |\vec{x}| )$ where $\pi \in \{L,R\}$.

We also use $P^{\Box} := P|\Box$.

In \cite{MeredithR05} an interpretation of the new operator is
given. It turns out that there are several possible interpretations
all enjoying the requisite algebraic properties of the operator (see
\cite{milner91polyadicpi}). We will therefore make liberal use of
$(\nu\; \vec{x})P$.

% subsection the_syntax_and_semantics_of_the_notation_system (end)   

\input{qm2pi.qmops} 

\input{qm2pi.sterngerlach} 

\input{qm2pi.metric} 

% section concurrent_process_calculi (end)

%\input{qm2pi.proofsketch}

% section proof sketch (end)

%\input{qm2pi.slviaknots} 

% section spatial logic via knots (end)

\input{qm2pi.conclusion}

% section conclusion (end)

%\input{qm2pi.dtcodes} 

% section wiring algorithm (end)

\input{qm2pi.ack} 

% section acknowledgments (end)

\newpage


\bibliographystyle{plain}   
\bibliography{../../biblios/main.bib}

\input{qm2pi.rhodetails}

\end{document}

 

%\documentclass[12pt]{llncs}
%\documentclass{jktr}

\usepackage[pdftex]{hyperref}                   
\usepackage {listings}
\usepackage {mathpartir}
\usepackage{bcprules}
%\usepackage{listings}
                       
\usepackage{graphicx} 
%\usepackage[margins=2.5cm,nohead,nofoot]{geometry}
%\usepackage{geometry}
\usepackage{amsfonts}
\usepackage{amstext}
\usepackage{latexsym}
\usepackage{amssymb}
\usepackage{color}


%\include{myPreamble}
\include{qm2pi.local} 

%\ifpdf
%\usepackage[pdftex]{graphicx}
%\else
%\usepackage{graphicx}
%\fi

 % \ifpdf
%  \usepackage{pdfsync}
%  \if


%\title{Brief Article}
%\author{David F. Snyder}
%\author{L.G. Meredith}

%\address{Dept. of Math., Texas State University--San Marcos, San Marcos, TX 78666}
       
\pagestyle{empty}


\begin{document}

\lstset{language=[Objective]Caml,frame=shadowbox}

\input{qm2pi.front}

% section front matter (end)

\input{qm2pi.intro} 
 
% section introduction (end)

% \input{qm2pi.knotations} 

% section notation (end)

\input{qm2pi.process.calculi} 

% section concurrent_process_calculi_and_spatial_logics_ (end)
    
%\input{qm2pi.knots2pi} 

%\input{qm2pi.trefoil} 

%\input{qm2pi.mainthm} 

% subsection basic_interpretation (end)

%\input{qm2pi.rho.presentation} 
\subsection{The syntax and semantics of the notation system}\label{sub:the_syntax_and_semantics_of_the_notation_system} % (fold)

We now summarize a technical presentation of the calculus that
embodies our theory of dynamics. The typical presentation of such a
calculus follows the style of giving generators and relations on
them. The grammar, below, describing term constructors, freely
generates the set of processes, $\Proc$. This set is then quotiented
by a relation known as structural congruence and it is over this set
that the notion of dynamics is expressed. This presentation is
essentially that of \cite{MeredithR05} with the addition of
polyadicity and summation. For readability we have relegated some of
the technical subtleties to an appendix.

\subsubsection{Process grammar}\label{subsub:process_grammar}

\begin{mathpar}
  \inferrule* [lab=synchronization] {} {{M} \bc \pzero \;|\; x?F \;|\; x!C }
  \and
  \inferrule* [lab=abstraction] {} {{F} \bc (x)P}
  \and
  \inferrule* [lab=concretion] {} {{C} \bc \langle Q \rangle}
  \and
  \inferrule* [lab=process] {} {{P,Q} \bc M \;| \;P|Q \;|\; @{x}}
  \and
  \inferrule* [lab=name] {} {{x} \bc \quotep{P}}
\end{mathpar} 

Note that $\vec{x}$ (resp. $\vec{P}$) denotes a vector of names
(resp. processes) of length $|\vec{x}|$ (resp. $|\vec{P}|$). We adopt
the following useful abbreviations.

\begin{mathpar}
   x?(\vec{y}).P := x.(\vec{y})P \and  x\clift{\vec{P}} := x.\clift{\vec{P}}
   \and x!(y) := \lift{x}{\dropn{y}}
   \and \Pi_{i=0}^{n-1}P_i := P_0 | \ldots | P_{n-1}
\end{mathpar}

\subsubsection{Structural congruence}

\paragraph{Free and bound names and alpha-equivalence.} At the
core of structural equivalence is alpha-equivalence which identifies
process that are the same up to a change of variable. Formally, we
recognize the distinction between free and bound names. The free names
of a process, $\freenames{P}$, may be calculated recursively as
follows:

\begin{mathpar}
\freenames{\pzero} := \emptyset
  \and \\
  \freenames{x?(y).P} := \{ x \} \cup (\freenames{P} \setminus \{ y \})
  \and 
  \freenames{x!\langle P \rangle} := \{ x \} \cup \{ P \} 
  \and \\
  \freenames{P|Q} := \freenames{P} \cup \freenames{Q}
  \and \\
  \freenames{@{x}} := \{ x \}
\end{mathpar}

$\pi$
$\quotep{\pi}$

$\freenames{-} : \pi \to \mathcal{P}(\quotep{\pi})$

\begin{eqnarray*}
  \freenames{\pzero} & := & \emptyset \\
  \freenames{x?(y).P} & := & \{ x \} \cup (\freenames{P} \setminus \{ y \}) \\
  \freenames{x!\langle P \rangle} & := & \{ x \} \cup \{ P \} \\
  \freenames{P|Q} & := & \freenames{P} \cup \freenames{Q} \\
  \freenames{\dropn{x}} & := & \{ x \}
\end{eqnarray*}

The bound names of a process, $\boundnames{P}$, are those names occurring in $P$
that are not free. For example, in $x?(y).0$, the name $x$ is free, while $y$ is bound.

\begin{mathpar}
  \inferrule* [lab=monoidal-laws] {} { P|Q \equiv Q|P \and P|0 \equiv P \and P|(Q|R) \equiv (P|Q)|R }
\end{mathpar}

\begin{mathpar}
  \inferrule* [lab=alpha-equivalence] {} { (x)P \equiv (y)P\{y/x\} \and y \not\in \freenames{P} }
\end{mathpar}

\begin{definition}
Then two processes, $P,Q$, are alpha-equivalent if $P = Q\{\vec{y}/\vec{x}\}$ for
some $\vec{x} \in \boundnames{Q},\vec{y} \in \boundnames{P}$, where $Q\{\vec{y}/\vec{x}\}$
denotes the capture-avoiding substitution of $\vec{y}$ for $\vec{x}$ in $Q$.
\end{definition}

\begin{definition}
  The {\em structural congruence} \cite{SangiorgiWalker} , $\equiv$,
  between processes is the least congruence containing
  alpha-equivalence, satisfying the abelian monoid laws
  (associativity, commutativity and $\pzero$ as identity) for parallel
  composition $|$ and for summation $+$.
\end{definition}

\subsection{Name equivalence}

We take name equivalence, written $\nameeq$, to be the smallest
equivalence relation generated by the following rules.

\begin{mathpar}
\inferrule*[lab=Quote-drop]
{ }
{ \quotep{@{x}} \nameeq x }

\inferrule*[lab=Struct-equiv]
{ P \scong Q }
{ \quotep{P} \nameeq \quotep{Q} }
\end{mathpar}

The astute reader will have noticed that the mutual recursion of names
and processes imposes a mutual recursion on alpha-equivalence and
structural equivalence via name-equivalence. Fortunately, all of this
works out pleasantly and we may calculate in the natural way, free of
concern. The reader interested in the details is referred to the
appendix \ref{appendix:rho_details}.

\subsection{Substitution}

We use $\Proc$ for the set of processes, $\QProc$ for the set of
names, and $\id{\{}\vec{y} / \vec{x} \id{\}}$ to denote partial maps,
$s : \QProc \rightarrow \QProc$. A map, $s$ lifts, uniquely, to a map
on process terms, $\widehat{s} : \Proc \rightarrow \Proc$ by the
following equations.

\begin{mathpar}
  (0) \psubstp{Q}{P} := 0 \\
  (R \juxtap S) \psubstp{Q}{P}
  :=    
  (R)\psubstp{Q}{P} \juxtap (S) \psubstp{Q}{P} \\
  (x?(y).R) \psubstp{Q}{P}    
  :=    
  (x)\substp{Q}{P} (z)\concat( (R \psubstn{z}{y}) \psubstp{Q}{P} ) \\
  (\lift{x}{R}) \psubstp{Q}{P}  
  :=
  \lift{(x)\substp{Q}{P}}{ R \psubstp{Q}{P} } \\
%   (\dropn{x})  \psubstp{Q}{P}       
%   := 
%   \left\{ 
%     \begin{array}{ccc} 
%       \dropn{\quotep{Q}} & & x \nameeq \quotep{P} \\
%       \dropn{x} & & otherwise \\
%     \end{array}
%   \right. 
  (\dropn{x})  \psubstp{Q}{P}       
  := 
  \left\{ 
    \begin{array}{ccc} 
      Q & & x \nameeq \quotep{P} \\
      \dropn{x} & & otherwise \\
    \end{array}
  \right.
\end{mathpar}
 

where

\begin{eqnarray}
  (x)\id{\{} \lpquote Q \rpquote / \lpquote P \rpquote \id{\}}            = 
  \left\{ 
    \begin{array}{ccc}
      \lpquote Q \rpquote & & x \nameeq \lpquote P \rpquote \\
      x & & otherwise \\
    \end{array}
  \right. \nonumber
\end{eqnarray}

and $z$ is chosen distinct from $\quotep{P}$, $\quotep{Q}$, the free
names in $Q$, and all the names in $R$. Our $\alpha$-equivalence will
be built in the standard way from this substitution.

\begin{remark}\label{rem:no_self_referential_names}
  One consequence of these definitions is that $\forall P. \quotep{P}
  \not\in \freenames{P}$.
\end{remark}

\subsection{ Dynamic quote: an example }

Anticipating something of what's to come, consider applying the
substitution, $\widehat{\id{\{}u / z \id{\}}}$, to the following pair
of processes, $\lift{w}{y!(z)}$ and $w[ \lpquote y!(z) \rpquote ]$.

\begin{eqnarray}
	\lift{w}{y!(z)}\widehat{\id{\{}u / z \id{\}}}
		& = &
		\lift{w}{y!(u)} \nonumber\\
	w[ \lpquote y!(z) \rpquote ] \widehat{ \id{\{}u / z \id{\}} }
		& = &
		w[ \lpquote y!(z) \rpquote ] \nonumber
\end{eqnarray}

Because the body of the process between quotes is impervious to
substitution, we get radically different answers. In fact, by
examining the first process in an input context,
e.g. $x?(z).\lift{w}{y!(z)}$, we see that the process under the lift
operator may be shaped by prefixed inputs binding a name inside it. In
this sense, the lift operator will be seen as a way to dynamically
construct processes before reifying them as names.

Finally equipped with these standard features we can present the
dynamics of the calculus.

\subsubsection{Operational semantics} 

Finally, we introduce the computational dynamics. What marks these
algebras as distinct from other more traditionally studied algebraic
structures, e.g. vector spaces or polynomial rings, is the manner in
which dynamics is captured. In traditional structures, dynamics is typically
expressed through morphisms between such structures, as in linear maps
between vector spaces or morphisms between rings. In algebras
associated with the semantics of computation, the dynamics is
expressed as part of the algebraic structure itself, through a
reduction reduction relation typically denoted by $\red$. Below, we
give a recursive presentation of this relation for the calculus used
in the encoding.

$\red \subseteq \pi \times \pi$
$\red : \pi \to \mathcal{P}(\pi)$

\begin{mathpar}
  \inferrule* [lab=Comm] { \textsf{match}( x_{src}, x_{trgt} ) } { x_{trgt}?(y)P \; | \; x_{src}!\langle {Q} \rangle \red P\{\quotep{Q}/y}\} }
  \and \\
  \inferrule* [lab=Par] {{P} \red {P}'} {{{P} | {Q}} \red {{P}' | {Q}}}
  \and
  \inferrule* [lab=Equiv]{{{P} \scong {P}'} \andalso {{P}' \red {Q}'} \andalso {{Q}' \scong {Q}}}{{P} \red {Q}}
\end{mathpar}

\begin{eqnarray*}
  match_{\equiv} (\quotep{P},\quotep{Q}) & := & P \equiv Q \\
  match_{\dagger}(\quotep{P},\quotep{Q}) & := & \forall R. P|Q \red^{*} R => R \red^{*} 0 \\
  match_{K}(\quotep{P},\quotep{Q}) & := & K \mbox{ for some context } K
\end{eqnarray*}

$u?(x)P | u!\langle Q \rangle \red P\{\quotep{Q}/x\}$

%We write $\wred$ for $\red^*$, and $P\red$ if $\exists Q $ such that $ P \red Q$.
We write $P\red$ if $\exists Q $ such that $ P \red Q$ and $P\not\red$, otherwise.

\section{Replication}

As mentioned before, it is known that replication (and hence
recursion) can be implemented in a higher-order process algebra
\cite{SangiorgiWalker}. As our first example of calculation with the
machinery thus far presented we give the construction explicitly in
the {\rhoc}.

\begin{eqnarray}
	D_{x} & := & \prefix{x}{y}{(\binpar{\outputp{x}{y}}{@{y}})} \nonumber\\
	\bangp_{x}{P} & := & \binpar{{x}!\langle{\binpar{D_{x}}{P}}\rangle}{D_{x}} \nonumber
\end{eqnarray}

\begin{eqnarray}
	\bangp_{x}{P} & & \nonumber\\
	=
	& {x}!\langle{(\prefix{x}{y}{(\outputp{x}{y} | @{y})) | P}}\rangle 
	      | \prefix{x}{y}{(\outputp{x}{y} | @{y})} & \nonumber\\
	\red
	& (\outputp{x}{y} | @{y})\substn{\quotep{(\prefix{x}{y}{(@{y} | \outputp{x}{y})) | P}}}{y} & \nonumber\\
	=
	& \outputp{x}{\quotep{(\prefix{x}{y}{(\outputp{x}{y} | @{y})) | P}}}
	  | {(\prefix{x}{y}{(\outputp{x}{y} | @{y})) | P}} & \nonumber\\
	\red
	& \ldots & \nonumber\\
	\red^*
	& P | P | \ldots & \nonumber
\end{eqnarray}

Of course, this encoding, as an implementation, runs away, unfolding
$\bangp{P}$ eagerly. A lazier and more implementable replication
operator, restricted to input-guarded processes, may be obtained as follows.

\begin{eqnarray}
\bangp{\prefix{u}{v}{P}} 
	:= 
	\binpar{\lift{x}{\prefix{u}{v}{(\binpar{D(x)}{P})}}}{D(x)} \nonumber
\end{eqnarray}

\begin{remark}
  Note that the lazier definition still does not deal with summation
  or mixed summation (i.e. sums over input and output). The reader is
  invited to construct definitions of replication that deal with these
  features. 

  Further, the definitions are parameterized in a name, $x$. Can you,
  gentle reader, make a definition that eliminates this parameter and
  guarantees no accidental interaction between the replication
  machinery and the process being replicated -- i.e. no accidental
  sharing of names used by the process to get its work done and the
  name(s) used by the replication to effect copying. This latter
  revision of the definition of replication is crucial to obtaining
  the expected identity $!!P \sim !P$.
\end{remark}

\begin{remark}\label{rem:paradoxical_combinator}
  The reader familiar with the lambda calculus will have noticed the
  similarity between $D$ and the paradoxical combinator.

  [Ed. note: the existence of this seems to suggest we have to be more
  restrictive on the set of processes and names we admit if we are to
  support no-cloning.]
\end{remark}

\subsubsection{Bisimulation}

The computational dynamics gives rise to another kind of equivalence,
the equivalence of computational behavior. As previously mentioned
this is typically captured \emph{via} some form of bisimulation.

% The notion we use in this paper is weak barbed bisimulation
% \cite{milner91polyadicpi}.

The notion we use in this paper is derived from weak barbed
bisimulation \cite{milner91polyadicpi}. 

\begin{definition}
An \emph{observation relation}, $\downarrow_{\mathcal N}$, over a set
of names, $\mathcal N$, is the smallest relation satisfying the rules
below.

\infrule[Out-barb]{y \in {\mathcal N}, \; x \nameeq y}
		  {\outputp{x}{v} \downarrow_{\mathcal N} x}
\infrule[Par-barb]{\mbox{$P\downarrow_{\mathcal N} x$ or $Q\downarrow_{\mathcal N} x$}}
		  {\binpar{P}{Q} \downarrow_{\mathcal N} x}

We write $P \Downarrow_{\mathcal N} x$ if there is $Q$ such that 
$P \wred Q$ and $Q \downarrow_{\mathcal N} x$.
\end{definition}

\begin{definition}
%\label{def.bbisim}
An  ${\mathcal N}$-\emph{barbed bisimulation} over a set of names, ${\mathcal N}$, is a symmetric binary relation 
${\mathcal S}_{\mathcal N}$ between agents such that $P\rel{S}_{\mathcal N}Q$ implies:
\begin{enumerate}
\item If $P \red P'$ then $Q \wred Q'$ and $P'\rel{S}_{\mathcal N} Q'$.
\item If $P\downarrow_{\mathcal N} x$, then $Q\Downarrow_{\mathcal N} x$.
\end{enumerate}
$P$ is ${\mathcal N}$-barbed bisimilar to $Q$, written
$P \wbbisim_{\mathcal N} Q$, if $P \rel{S}_{\mathcal N} Q$ for some ${\mathcal N}$-barbed bisimulation ${\mathcal S}_{\mathcal N}$.
\end{definition}

$\mathcal{R} \subseteq \pi \times \pi$

$P \mathcal{R} Q => \forall P'. P \red P' \Rightarrow \exists Q'. Q \red Q', P' \mathcal{R} Q'$

$P \vdash x \Rightarrow Q \vdash x$

\begin{mathpar}
  \inferrule*[lab=Out-barb]{x \nameeq y}{{y}!\langle{Q}\rangle \vdash x}
  \and
  \inferrule*[lab=Par-barb]{\mbox{$P\vdash x$ or $Q\vdash x$}}{\binpar{P}{Q} \vdash x}
\end{mathpar}

\subsubsection{Contexts}

One of the principle advantages of computational calculi like the
$\pi$-calculus is a well-defined notion of context,
contextual-equivalence and a correlation between
contextual-equivalence and notions of bisimulation. The notion of
context allows the decomposition of a process into (sub-)process and
its syntactic environment, its context. Thus, a context may be
thought of as a process with a ``hole'' (written $\Box$) in it. The
application of a context $M$ to a process $P$, written $M[P]$, is
tantamount to filling the hole in $M$ with $P$. In this paper we do
not need the full weight of this theory, but do make use of the notion
of context in the proof the main theorem. 

\begin{mathpar}
  \inferrule* [lab=summation] {} {{M_{M},M_{N}} \bc \Box \;|\; x.M_{A} \;|\; M_{M}+M_{N}}
  \and
  \inferrule* [lab=agent] {} {{M_{A}} \bc (\vec{x})M_{P} \;| \; \clift{P_0,\ldots,M_{P},\ldots,P_N}}
  \and \\
  \inferrule* [lab=process] {} {{M_{P}} \bc M_{N} \;| \;P|M_{P} }
\end{mathpar} 

\begin{mathpar}
  \inferrule* [lab=sychronization] {} {M_{N} \bc \Box \;|\; x?M_{F} \;|\; x!M_{C}}
  \and
  \inferrule* [lab=abstraction] {} {{M_{F}} \bc (x)M_{P} }
  \and
  \inferrule* [lab=concretion] {} {{M_{C}} \bc \langle M_{P} \rangle }
  \and \\
  \inferrule* [lab=process] {} {{M_{P}} \bc M_{N} \;| \;P|M_{P} }
\end{mathpar}

\begin{definition}[contextual application] Given a context $M$, and
  process $P$, we define the \emph{contextual application}, $M[P] :=
  M\{P/\Box\}$. That is, the contextual application of M to P is the
  substitution of $P$ for $\Box$ in $M$.
\end{definition}

$\meaningof{-} : L \to \mathcal{P}(\pi)$

\begin{mathpar}
  \inferrule* [lab=collection] {} {\meaningof{true} = \pi, \and \meaningof{~E} = \pi \setminus \meaningof{E}, \and \meaningof{E_{1} \& E_{2}} = \meaningof{E_{1}} \cap \meaningof{E_{2}}}
\end{mathpar}

\begin{mathpar}
  \inferrule* [lab=structure] {} {\meaningof{0} = \{ P \in \pi | P \equiv 0 \}, \and \\ \meaningof{E_1 | E_2} = \{ P \in \pi | P \equiv P_{1} | P_{2}, P_{1} \in \meaningof{E_{1}}, P_{2} \in \meaningof{E_2}\} }
\end{mathpar}

\begin{mathpar}
 \inferrule* [lab=behavior] {} {\meaningof{\langle a?b \rangle E} = \{ P \in \pi | P \equiv Q | u?(y)P', \\ \and \\\\ \and \\ \;\;\; u \in \meaningof{a}, \forall z.P'\{z/y\} \in \meaningof{E\{z/b\}}\}, \and \\ \meaningof{a!E} = \{ P \in \pi | P \equiv Q | x!\langle P' \rangle, x \in \meaningof{a} P' \in \meaningof{E}\} }
\end{mathpar}

\begin{mathpar}
 \inferrule* [lab=nominal] {} {\meaningof{\quotep{E}} = \{ \quotep{P} \in \quotep{\pi} | P \in \meaningof{E} \}, \and \meaningof{\quotep{P}} = \{ \quotep{Q} \in \quotep{\pi} | P \equiv Q \} \and \\ \meaningof{@\quotep{E}} = \{ P \in \pi | P \equiv @x, x \in \meaningof{E} \}}
\end{mathpar}

\begin{eqnarray*}
  \\
  \meaningof{-} : TS \to ST
\end{eqnarray*}

\begin{eqnarray*}
  \\
  L : TS \to ST
\end{eqnarray*}

\begin{eqnarray*}
  \\
  P \models E \iff P \in \meaningof{E}
\end{eqnarray*}

\begin{eqnarray*}
  P \approx_{L} Q \iff \forall E \in L. P \models E \iff Q \models E
\end{eqnarray*}

\begin{eqnarray*}
  P \approx_{K} Q
\end{eqnarray*}

\begin{eqnarray*}
  P \approx Q
\end{eqnarray*}

$\approx_{K} = \approx = \approx_{L}$

\subsubsection{Contextual duality}

Note that contexts extend the quotation operation to a family of
operations from processes to names. Given a context, $M$, we can
define a \emph{nominal context}, $\quotep{M}$ by $\quotep{M}[P] :=
\quotep{M[P]}$. To foreshadow what is to come we observe that these
operations enjoy a duality with processes very much like the duality
between vectors and maps from vectors to scalars.

Further, because the calculus is essentially higher-order, we have a
correspondence between contexts and processes. More specifically,
given a name $x$ and a context $M$ we can construct $M^{*}_{x}$ such
that 

\begin{mathpar}
  M^{*}_{x} | \lift{x}{P} \red M[P]
\end{mathpar}

namely,

\begin{mathpar}
  M^{*}_{x} := x?(u).M[\dropn{u}]
\end{mathpar}

The dependence of $M^{*}_{x}$ on a name makes it an abstraction, 

\begin{mathpar}
  M^{*} := (x)x?(u).M[\dropn{u}]
\end{mathpar}

\subsection{Additional notation}

It will sometimes be convenient to denote the process a name
quotes. We already have the notation $x = \quotep{P}$, but it will be
convenient to introduce an alternate notation, $\procn{x}$, when we
want to emphasize the connection to the use of the name. Note that, by
virtue of name equivalence, $\quotep{\procn{x}} \nameeq x$; so, the
notation is consistent with previous definitions.

Further, because names have structure it is possible to effect
substitutions on the basis of that structure. This means we need to
upgrade our notation for substitutions, which we accomplish by
adapting comprehension notation. Thus,

\begin{mathpar}
  P\{ y / x : x \in S \}
\end{mathpar}

is interpreted to mean the process derived from P by replacing (in a
capture-avoiding manner) each occurrence of $x$ in $S$ by $y$. For example,

\begin{mathpar}
  P\{ \quotep{\procn{x}|\procn{x}} / x : x \in \freenames{P} \}
\end{mathpar}

will replace each (occurrence) of a free name $x$ in $P$ by
$\quotep{\procn{x}|\procn{x}}$.

Also, we will avail ourselves of the notation $x^{L}$ and $x^{R}$ to
denote injections of a name into disjoint copies of the name
space. There are numerous ways to accomplish this. One example can be
found in \cite{MeredithR05}. This notation overloads to vectors of
names: $\vec{x}^{\pi} := (x_{i}^{\pi} \; : \; 0 \leq i < |\vec{x}| )$ where $\pi \in \{L,R\}$.

We also use $P^{\Box} := P|\Box$.

In \cite{MeredithR05} an interpretation of the new operator is
given. It turns out that there are several possible interpretations
all enjoying the requisite algebraic properties of the operator (see
\cite{milner91polyadicpi}). We will therefore make liberal use of
$(\nu\; \vec{x})P$.

% subsection the_syntax_and_semantics_of_the_notation_system (end)   

\input{qm2pi.qmops} 

\input{qm2pi.sterngerlach} 

\input{qm2pi.metric} 

% section concurrent_process_calculi (end)

%\input{qm2pi.proofsketch}

% section proof sketch (end)

%\input{qm2pi.slviaknots} 

% section spatial logic via knots (end)

\input{qm2pi.conclusion}

% section conclusion (end)

%\input{qm2pi.dtcodes} 

% section wiring algorithm (end)

\input{qm2pi.ack} 

% section acknowledgments (end)

\newpage


\bibliographystyle{plain}   
\bibliography{../../biblios/main.bib}

\input{qm2pi.rhodetails}

\end{document}

 

%\documentclass[12pt]{llncs}
%\documentclass{jktr}

\usepackage[pdftex]{hyperref}                   
\usepackage {listings}
\usepackage {mathpartir}
\usepackage{bcprules}
%\usepackage{listings}
                       
\usepackage{graphicx} 
%\usepackage[margins=2.5cm,nohead,nofoot]{geometry}
%\usepackage{geometry}
\usepackage{amsfonts}
\usepackage{amstext}
\usepackage{latexsym}
\usepackage{amssymb}
\usepackage{color}


%\include{myPreamble}
\include{qm2pi.local} 

%\ifpdf
%\usepackage[pdftex]{graphicx}
%\else
%\usepackage{graphicx}
%\fi

 % \ifpdf
%  \usepackage{pdfsync}
%  \if


%\title{Brief Article}
%\author{David F. Snyder}
%\author{L.G. Meredith}

%\address{Dept. of Math., Texas State University--San Marcos, San Marcos, TX 78666}
       
\pagestyle{empty}


\begin{document}

\lstset{language=[Objective]Caml,frame=shadowbox}

\input{qm2pi.front}

% section front matter (end)

\input{qm2pi.intro} 
 
% section introduction (end)

% \input{qm2pi.knotations} 

% section notation (end)

\input{qm2pi.process.calculi} 

% section concurrent_process_calculi_and_spatial_logics_ (end)
    
%\input{qm2pi.knots2pi} 

%\input{qm2pi.trefoil} 

%\input{qm2pi.mainthm} 

% subsection basic_interpretation (end)

%\input{qm2pi.rho.presentation} 
\subsection{The syntax and semantics of the notation system}\label{sub:the_syntax_and_semantics_of_the_notation_system} % (fold)

We now summarize a technical presentation of the calculus that
embodies our theory of dynamics. The typical presentation of such a
calculus follows the style of giving generators and relations on
them. The grammar, below, describing term constructors, freely
generates the set of processes, $\Proc$. This set is then quotiented
by a relation known as structural congruence and it is over this set
that the notion of dynamics is expressed. This presentation is
essentially that of \cite{MeredithR05} with the addition of
polyadicity and summation. For readability we have relegated some of
the technical subtleties to an appendix.

\subsubsection{Process grammar}\label{subsub:process_grammar}

\begin{mathpar}
  \inferrule* [lab=synchronization] {} {{M} \bc \pzero \;|\; x?F \;|\; x!C }
  \and
  \inferrule* [lab=abstraction] {} {{F} \bc (x)P}
  \and
  \inferrule* [lab=concretion] {} {{C} \bc \langle Q \rangle}
  \and
  \inferrule* [lab=process] {} {{P,Q} \bc M \;| \;P|Q \;|\; @{x}}
  \and
  \inferrule* [lab=name] {} {{x} \bc \quotep{P}}
\end{mathpar} 

Note that $\vec{x}$ (resp. $\vec{P}$) denotes a vector of names
(resp. processes) of length $|\vec{x}|$ (resp. $|\vec{P}|$). We adopt
the following useful abbreviations.

\begin{mathpar}
   x?(\vec{y}).P := x.(\vec{y})P \and  x\clift{\vec{P}} := x.\clift{\vec{P}}
   \and x!(y) := \lift{x}{\dropn{y}}
   \and \Pi_{i=0}^{n-1}P_i := P_0 | \ldots | P_{n-1}
\end{mathpar}

\subsubsection{Structural congruence}

\paragraph{Free and bound names and alpha-equivalence.} At the
core of structural equivalence is alpha-equivalence which identifies
process that are the same up to a change of variable. Formally, we
recognize the distinction between free and bound names. The free names
of a process, $\freenames{P}$, may be calculated recursively as
follows:

\begin{mathpar}
\freenames{\pzero} := \emptyset
  \and \\
  \freenames{x?(y).P} := \{ x \} \cup (\freenames{P} \setminus \{ y \})
  \and 
  \freenames{x!\langle P \rangle} := \{ x \} \cup \{ P \} 
  \and \\
  \freenames{P|Q} := \freenames{P} \cup \freenames{Q}
  \and \\
  \freenames{@{x}} := \{ x \}
\end{mathpar}

$\pi$
$\quotep{\pi}$

$\freenames{-} : \pi \to \mathcal{P}(\quotep{\pi})$

\begin{eqnarray*}
  \freenames{\pzero} & := & \emptyset \\
  \freenames{x?(y).P} & := & \{ x \} \cup (\freenames{P} \setminus \{ y \}) \\
  \freenames{x!\langle P \rangle} & := & \{ x \} \cup \{ P \} \\
  \freenames{P|Q} & := & \freenames{P} \cup \freenames{Q} \\
  \freenames{\dropn{x}} & := & \{ x \}
\end{eqnarray*}

The bound names of a process, $\boundnames{P}$, are those names occurring in $P$
that are not free. For example, in $x?(y).0$, the name $x$ is free, while $y$ is bound.

\begin{mathpar}
  \inferrule* [lab=monoidal-laws] {} { P|Q \equiv Q|P \and P|0 \equiv P \and P|(Q|R) \equiv (P|Q)|R }
\end{mathpar}

\begin{mathpar}
  \inferrule* [lab=alpha-equivalence] {} { (x)P \equiv (y)P\{y/x\} \and y \not\in \freenames{P} }
\end{mathpar}

\begin{definition}
Then two processes, $P,Q$, are alpha-equivalent if $P = Q\{\vec{y}/\vec{x}\}$ for
some $\vec{x} \in \boundnames{Q},\vec{y} \in \boundnames{P}$, where $Q\{\vec{y}/\vec{x}\}$
denotes the capture-avoiding substitution of $\vec{y}$ for $\vec{x}$ in $Q$.
\end{definition}

\begin{definition}
  The {\em structural congruence} \cite{SangiorgiWalker} , $\equiv$,
  between processes is the least congruence containing
  alpha-equivalence, satisfying the abelian monoid laws
  (associativity, commutativity and $\pzero$ as identity) for parallel
  composition $|$ and for summation $+$.
\end{definition}

\subsection{Name equivalence}

We take name equivalence, written $\nameeq$, to be the smallest
equivalence relation generated by the following rules.

\begin{mathpar}
\inferrule*[lab=Quote-drop]
{ }
{ \quotep{@{x}} \nameeq x }

\inferrule*[lab=Struct-equiv]
{ P \scong Q }
{ \quotep{P} \nameeq \quotep{Q} }
\end{mathpar}

The astute reader will have noticed that the mutual recursion of names
and processes imposes a mutual recursion on alpha-equivalence and
structural equivalence via name-equivalence. Fortunately, all of this
works out pleasantly and we may calculate in the natural way, free of
concern. The reader interested in the details is referred to the
appendix \ref{appendix:rho_details}.

\subsection{Substitution}

We use $\Proc$ for the set of processes, $\QProc$ for the set of
names, and $\id{\{}\vec{y} / \vec{x} \id{\}}$ to denote partial maps,
$s : \QProc \rightarrow \QProc$. A map, $s$ lifts, uniquely, to a map
on process terms, $\widehat{s} : \Proc \rightarrow \Proc$ by the
following equations.

\begin{mathpar}
  (0) \psubstp{Q}{P} := 0 \\
  (R \juxtap S) \psubstp{Q}{P}
  :=    
  (R)\psubstp{Q}{P} \juxtap (S) \psubstp{Q}{P} \\
  (x?(y).R) \psubstp{Q}{P}    
  :=    
  (x)\substp{Q}{P} (z)\concat( (R \psubstn{z}{y}) \psubstp{Q}{P} ) \\
  (\lift{x}{R}) \psubstp{Q}{P}  
  :=
  \lift{(x)\substp{Q}{P}}{ R \psubstp{Q}{P} } \\
%   (\dropn{x})  \psubstp{Q}{P}       
%   := 
%   \left\{ 
%     \begin{array}{ccc} 
%       \dropn{\quotep{Q}} & & x \nameeq \quotep{P} \\
%       \dropn{x} & & otherwise \\
%     \end{array}
%   \right. 
  (\dropn{x})  \psubstp{Q}{P}       
  := 
  \left\{ 
    \begin{array}{ccc} 
      Q & & x \nameeq \quotep{P} \\
      \dropn{x} & & otherwise \\
    \end{array}
  \right.
\end{mathpar}
 

where

\begin{eqnarray}
  (x)\id{\{} \lpquote Q \rpquote / \lpquote P \rpquote \id{\}}            = 
  \left\{ 
    \begin{array}{ccc}
      \lpquote Q \rpquote & & x \nameeq \lpquote P \rpquote \\
      x & & otherwise \\
    \end{array}
  \right. \nonumber
\end{eqnarray}

and $z$ is chosen distinct from $\quotep{P}$, $\quotep{Q}$, the free
names in $Q$, and all the names in $R$. Our $\alpha$-equivalence will
be built in the standard way from this substitution.

\begin{remark}\label{rem:no_self_referential_names}
  One consequence of these definitions is that $\forall P. \quotep{P}
  \not\in \freenames{P}$.
\end{remark}

\subsection{ Dynamic quote: an example }

Anticipating something of what's to come, consider applying the
substitution, $\widehat{\id{\{}u / z \id{\}}}$, to the following pair
of processes, $\lift{w}{y!(z)}$ and $w[ \lpquote y!(z) \rpquote ]$.

\begin{eqnarray}
	\lift{w}{y!(z)}\widehat{\id{\{}u / z \id{\}}}
		& = &
		\lift{w}{y!(u)} \nonumber\\
	w[ \lpquote y!(z) \rpquote ] \widehat{ \id{\{}u / z \id{\}} }
		& = &
		w[ \lpquote y!(z) \rpquote ] \nonumber
\end{eqnarray}

Because the body of the process between quotes is impervious to
substitution, we get radically different answers. In fact, by
examining the first process in an input context,
e.g. $x?(z).\lift{w}{y!(z)}$, we see that the process under the lift
operator may be shaped by prefixed inputs binding a name inside it. In
this sense, the lift operator will be seen as a way to dynamically
construct processes before reifying them as names.

Finally equipped with these standard features we can present the
dynamics of the calculus.

\subsubsection{Operational semantics} 

Finally, we introduce the computational dynamics. What marks these
algebras as distinct from other more traditionally studied algebraic
structures, e.g. vector spaces or polynomial rings, is the manner in
which dynamics is captured. In traditional structures, dynamics is typically
expressed through morphisms between such structures, as in linear maps
between vector spaces or morphisms between rings. In algebras
associated with the semantics of computation, the dynamics is
expressed as part of the algebraic structure itself, through a
reduction reduction relation typically denoted by $\red$. Below, we
give a recursive presentation of this relation for the calculus used
in the encoding.

$\red \subseteq \pi \times \pi$
$\red : \pi \to \mathcal{P}(\pi)$

\begin{mathpar}
  \inferrule* [lab=Comm] { \textsf{match}( x_{src}, x_{trgt} ) } { x_{trgt}?(y)P \; | \; x_{src}!\langle {Q} \rangle \red P\{\quotep{Q}/y}\} }
  \and \\
  \inferrule* [lab=Par] {{P} \red {P}'} {{{P} | {Q}} \red {{P}' | {Q}}}
  \and
  \inferrule* [lab=Equiv]{{{P} \scong {P}'} \andalso {{P}' \red {Q}'} \andalso {{Q}' \scong {Q}}}{{P} \red {Q}}
\end{mathpar}

\begin{eqnarray*}
  match_{\equiv} (\quotep{P},\quotep{Q}) & := & P \equiv Q \\
  match_{\dagger}(\quotep{P},\quotep{Q}) & := & \forall R. P|Q \red^{*} R => R \red^{*} 0 \\
  match_{K}(\quotep{P},\quotep{Q}) & := & K \mbox{ for some context } K
\end{eqnarray*}

$u?(x)P | u!\langle Q \rangle \red P\{\quotep{Q}/x\}$

%We write $\wred$ for $\red^*$, and $P\red$ if $\exists Q $ such that $ P \red Q$.
We write $P\red$ if $\exists Q $ such that $ P \red Q$ and $P\not\red$, otherwise.

\section{Replication}

As mentioned before, it is known that replication (and hence
recursion) can be implemented in a higher-order process algebra
\cite{SangiorgiWalker}. As our first example of calculation with the
machinery thus far presented we give the construction explicitly in
the {\rhoc}.

\begin{eqnarray}
	D_{x} & := & \prefix{x}{y}{(\binpar{\outputp{x}{y}}{@{y}})} \nonumber\\
	\bangp_{x}{P} & := & \binpar{{x}!\langle{\binpar{D_{x}}{P}}\rangle}{D_{x}} \nonumber
\end{eqnarray}

\begin{eqnarray}
	\bangp_{x}{P} & & \nonumber\\
	=
	& {x}!\langle{(\prefix{x}{y}{(\outputp{x}{y} | @{y})) | P}}\rangle 
	      | \prefix{x}{y}{(\outputp{x}{y} | @{y})} & \nonumber\\
	\red
	& (\outputp{x}{y} | @{y})\substn{\quotep{(\prefix{x}{y}{(@{y} | \outputp{x}{y})) | P}}}{y} & \nonumber\\
	=
	& \outputp{x}{\quotep{(\prefix{x}{y}{(\outputp{x}{y} | @{y})) | P}}}
	  | {(\prefix{x}{y}{(\outputp{x}{y} | @{y})) | P}} & \nonumber\\
	\red
	& \ldots & \nonumber\\
	\red^*
	& P | P | \ldots & \nonumber
\end{eqnarray}

Of course, this encoding, as an implementation, runs away, unfolding
$\bangp{P}$ eagerly. A lazier and more implementable replication
operator, restricted to input-guarded processes, may be obtained as follows.

\begin{eqnarray}
\bangp{\prefix{u}{v}{P}} 
	:= 
	\binpar{\lift{x}{\prefix{u}{v}{(\binpar{D(x)}{P})}}}{D(x)} \nonumber
\end{eqnarray}

\begin{remark}
  Note that the lazier definition still does not deal with summation
  or mixed summation (i.e. sums over input and output). The reader is
  invited to construct definitions of replication that deal with these
  features. 

  Further, the definitions are parameterized in a name, $x$. Can you,
  gentle reader, make a definition that eliminates this parameter and
  guarantees no accidental interaction between the replication
  machinery and the process being replicated -- i.e. no accidental
  sharing of names used by the process to get its work done and the
  name(s) used by the replication to effect copying. This latter
  revision of the definition of replication is crucial to obtaining
  the expected identity $!!P \sim !P$.
\end{remark}

\begin{remark}\label{rem:paradoxical_combinator}
  The reader familiar with the lambda calculus will have noticed the
  similarity between $D$ and the paradoxical combinator.

  [Ed. note: the existence of this seems to suggest we have to be more
  restrictive on the set of processes and names we admit if we are to
  support no-cloning.]
\end{remark}

\subsubsection{Bisimulation}

The computational dynamics gives rise to another kind of equivalence,
the equivalence of computational behavior. As previously mentioned
this is typically captured \emph{via} some form of bisimulation.

% The notion we use in this paper is weak barbed bisimulation
% \cite{milner91polyadicpi}.

The notion we use in this paper is derived from weak barbed
bisimulation \cite{milner91polyadicpi}. 

\begin{definition}
An \emph{observation relation}, $\downarrow_{\mathcal N}$, over a set
of names, $\mathcal N$, is the smallest relation satisfying the rules
below.

\infrule[Out-barb]{y \in {\mathcal N}, \; x \nameeq y}
		  {\outputp{x}{v} \downarrow_{\mathcal N} x}
\infrule[Par-barb]{\mbox{$P\downarrow_{\mathcal N} x$ or $Q\downarrow_{\mathcal N} x$}}
		  {\binpar{P}{Q} \downarrow_{\mathcal N} x}

We write $P \Downarrow_{\mathcal N} x$ if there is $Q$ such that 
$P \wred Q$ and $Q \downarrow_{\mathcal N} x$.
\end{definition}

\begin{definition}
%\label{def.bbisim}
An  ${\mathcal N}$-\emph{barbed bisimulation} over a set of names, ${\mathcal N}$, is a symmetric binary relation 
${\mathcal S}_{\mathcal N}$ between agents such that $P\rel{S}_{\mathcal N}Q$ implies:
\begin{enumerate}
\item If $P \red P'$ then $Q \wred Q'$ and $P'\rel{S}_{\mathcal N} Q'$.
\item If $P\downarrow_{\mathcal N} x$, then $Q\Downarrow_{\mathcal N} x$.
\end{enumerate}
$P$ is ${\mathcal N}$-barbed bisimilar to $Q$, written
$P \wbbisim_{\mathcal N} Q$, if $P \rel{S}_{\mathcal N} Q$ for some ${\mathcal N}$-barbed bisimulation ${\mathcal S}_{\mathcal N}$.
\end{definition}

$\mathcal{R} \subseteq \pi \times \pi$

$P \mathcal{R} Q => \forall P'. P \red P' \Rightarrow \exists Q'. Q \red Q', P' \mathcal{R} Q'$

$P \vdash x \Rightarrow Q \vdash x$

\begin{mathpar}
  \inferrule*[lab=Out-barb]{x \nameeq y}{{y}!\langle{Q}\rangle \vdash x}
  \and
  \inferrule*[lab=Par-barb]{\mbox{$P\vdash x$ or $Q\vdash x$}}{\binpar{P}{Q} \vdash x}
\end{mathpar}

\subsubsection{Contexts}

One of the principle advantages of computational calculi like the
$\pi$-calculus is a well-defined notion of context,
contextual-equivalence and a correlation between
contextual-equivalence and notions of bisimulation. The notion of
context allows the decomposition of a process into (sub-)process and
its syntactic environment, its context. Thus, a context may be
thought of as a process with a ``hole'' (written $\Box$) in it. The
application of a context $M$ to a process $P$, written $M[P]$, is
tantamount to filling the hole in $M$ with $P$. In this paper we do
not need the full weight of this theory, but do make use of the notion
of context in the proof the main theorem. 

\begin{mathpar}
  \inferrule* [lab=summation] {} {{M_{M},M_{N}} \bc \Box \;|\; x.M_{A} \;|\; M_{M}+M_{N}}
  \and
  \inferrule* [lab=agent] {} {{M_{A}} \bc (\vec{x})M_{P} \;| \; \clift{P_0,\ldots,M_{P},\ldots,P_N}}
  \and \\
  \inferrule* [lab=process] {} {{M_{P}} \bc M_{N} \;| \;P|M_{P} }
\end{mathpar} 

\begin{mathpar}
  \inferrule* [lab=sychronization] {} {M_{N} \bc \Box \;|\; x?M_{F} \;|\; x!M_{C}}
  \and
  \inferrule* [lab=abstraction] {} {{M_{F}} \bc (x)M_{P} }
  \and
  \inferrule* [lab=concretion] {} {{M_{C}} \bc \langle M_{P} \rangle }
  \and \\
  \inferrule* [lab=process] {} {{M_{P}} \bc M_{N} \;| \;P|M_{P} }
\end{mathpar}

\begin{definition}[contextual application] Given a context $M$, and
  process $P$, we define the \emph{contextual application}, $M[P] :=
  M\{P/\Box\}$. That is, the contextual application of M to P is the
  substitution of $P$ for $\Box$ in $M$.
\end{definition}

$\meaningof{-} : L \to \mathcal{P}(\pi)$

\begin{mathpar}
  \inferrule* [lab=collection] {} {\meaningof{true} = \pi, \and \meaningof{~E} = \pi \setminus \meaningof{E}, \and \meaningof{E_{1} \& E_{2}} = \meaningof{E_{1}} \cap \meaningof{E_{2}}}
\end{mathpar}

\begin{mathpar}
  \inferrule* [lab=structure] {} {\meaningof{0} = \{ P \in \pi | P \equiv 0 \}, \and \\ \meaningof{E_1 | E_2} = \{ P \in \pi | P \equiv P_{1} | P_{2}, P_{1} \in \meaningof{E_{1}}, P_{2} \in \meaningof{E_2}\} }
\end{mathpar}

\begin{mathpar}
 \inferrule* [lab=behavior] {} {\meaningof{\langle a?b \rangle E} = \{ P \in \pi | P \equiv Q | u?(y)P', \\ \and \\\\ \and \\ \;\;\; u \in \meaningof{a}, \forall z.P'\{z/y\} \in \meaningof{E\{z/b\}}\}, \and \\ \meaningof{a!E} = \{ P \in \pi | P \equiv Q | x!\langle P' \rangle, x \in \meaningof{a} P' \in \meaningof{E}\} }
\end{mathpar}

\begin{mathpar}
 \inferrule* [lab=nominal] {} {\meaningof{\quotep{E}} = \{ \quotep{P} \in \quotep{\pi} | P \in \meaningof{E} \}, \and \meaningof{\quotep{P}} = \{ \quotep{Q} \in \quotep{\pi} | P \equiv Q \} \and \\ \meaningof{@\quotep{E}} = \{ P \in \pi | P \equiv @x, x \in \meaningof{E} \}}
\end{mathpar}

\begin{eqnarray*}
  \\
  \meaningof{-} : TS \to ST
\end{eqnarray*}

\begin{eqnarray*}
  \\
  L : TS \to ST
\end{eqnarray*}

\begin{eqnarray*}
  \\
  P \models E \iff P \in \meaningof{E}
\end{eqnarray*}

\begin{eqnarray*}
  P \approx_{L} Q \iff \forall E \in L. P \models E \iff Q \models E
\end{eqnarray*}

\begin{eqnarray*}
  P \approx_{K} Q
\end{eqnarray*}

\begin{eqnarray*}
  P \approx Q
\end{eqnarray*}

$\approx_{K} = \approx = \approx_{L}$

\subsubsection{Contextual duality}

Note that contexts extend the quotation operation to a family of
operations from processes to names. Given a context, $M$, we can
define a \emph{nominal context}, $\quotep{M}$ by $\quotep{M}[P] :=
\quotep{M[P]}$. To foreshadow what is to come we observe that these
operations enjoy a duality with processes very much like the duality
between vectors and maps from vectors to scalars.

Further, because the calculus is essentially higher-order, we have a
correspondence between contexts and processes. More specifically,
given a name $x$ and a context $M$ we can construct $M^{*}_{x}$ such
that 

\begin{mathpar}
  M^{*}_{x} | \lift{x}{P} \red M[P]
\end{mathpar}

namely,

\begin{mathpar}
  M^{*}_{x} := x?(u).M[\dropn{u}]
\end{mathpar}

The dependence of $M^{*}_{x}$ on a name makes it an abstraction, 

\begin{mathpar}
  M^{*} := (x)x?(u).M[\dropn{u}]
\end{mathpar}

\subsection{Additional notation}

It will sometimes be convenient to denote the process a name
quotes. We already have the notation $x = \quotep{P}$, but it will be
convenient to introduce an alternate notation, $\procn{x}$, when we
want to emphasize the connection to the use of the name. Note that, by
virtue of name equivalence, $\quotep{\procn{x}} \nameeq x$; so, the
notation is consistent with previous definitions.

Further, because names have structure it is possible to effect
substitutions on the basis of that structure. This means we need to
upgrade our notation for substitutions, which we accomplish by
adapting comprehension notation. Thus,

\begin{mathpar}
  P\{ y / x : x \in S \}
\end{mathpar}

is interpreted to mean the process derived from P by replacing (in a
capture-avoiding manner) each occurrence of $x$ in $S$ by $y$. For example,

\begin{mathpar}
  P\{ \quotep{\procn{x}|\procn{x}} / x : x \in \freenames{P} \}
\end{mathpar}

will replace each (occurrence) of a free name $x$ in $P$ by
$\quotep{\procn{x}|\procn{x}}$.

Also, we will avail ourselves of the notation $x^{L}$ and $x^{R}$ to
denote injections of a name into disjoint copies of the name
space. There are numerous ways to accomplish this. One example can be
found in \cite{MeredithR05}. This notation overloads to vectors of
names: $\vec{x}^{\pi} := (x_{i}^{\pi} \; : \; 0 \leq i < |\vec{x}| )$ where $\pi \in \{L,R\}$.

We also use $P^{\Box} := P|\Box$.

In \cite{MeredithR05} an interpretation of the new operator is
given. It turns out that there are several possible interpretations
all enjoying the requisite algebraic properties of the operator (see
\cite{milner91polyadicpi}). We will therefore make liberal use of
$(\nu\; \vec{x})P$.

% subsection the_syntax_and_semantics_of_the_notation_system (end)   

\input{qm2pi.qmops} 

\input{qm2pi.sterngerlach} 

\input{qm2pi.metric} 

% section concurrent_process_calculi (end)

%\input{qm2pi.proofsketch}

% section proof sketch (end)

%\input{qm2pi.slviaknots} 

% section spatial logic via knots (end)

\input{qm2pi.conclusion}

% section conclusion (end)

%\input{qm2pi.dtcodes} 

% section wiring algorithm (end)

\input{qm2pi.ack} 

% section acknowledgments (end)

\newpage


\bibliographystyle{plain}   
\bibliography{../../biblios/main.bib}

\input{qm2pi.rhodetails}

\end{document}

 

% subsection basic_interpretation (end)

%\input{qm2pi.rho.presentation} 
\subsection{The syntax and semantics of the notation system}\label{sub:the_syntax_and_semantics_of_the_notation_system} % (fold)

We now summarize a technical presentation of the calculus that
embodies our theory of dynamics. The typical presentation of such a
calculus follows the style of giving generators and relations on
them. The grammar, below, describing term constructors, freely
generates the set of processes, $\Proc$. This set is then quotiented
by a relation known as structural congruence and it is over this set
that the notion of dynamics is expressed. This presentation is
essentially that of \cite{MeredithR05} with the addition of
polyadicity and summation. For readability we have relegated some of
the technical subtleties to an appendix.

\subsubsection{Process grammar}\label{subsub:process_grammar}

\begin{mathpar}
  \inferrule* [lab=synchronization] {} {{M} \bc \pzero \;|\; x?F \;|\; x!C }
  \and
  \inferrule* [lab=abstraction] {} {{F} \bc (x)P}
  \and
  \inferrule* [lab=concretion] {} {{C} \bc \langle Q \rangle}
  \and
  \inferrule* [lab=process] {} {{P,Q} \bc M \;| \;P|Q \;|\; @{x}}
  \and
  \inferrule* [lab=name] {} {{x} \bc \quotep{P}}
\end{mathpar} 

Note that $\vec{x}$ (resp. $\vec{P}$) denotes a vector of names
(resp. processes) of length $|\vec{x}|$ (resp. $|\vec{P}|$). We adopt
the following useful abbreviations.

\begin{mathpar}
   x?(\vec{y}).P := x.(\vec{y})P \and  x\clift{\vec{P}} := x.\clift{\vec{P}}
   \and x!(y) := \lift{x}{\dropn{y}}
   \and \Pi_{i=0}^{n-1}P_i := P_0 | \ldots | P_{n-1}
\end{mathpar}

\subsubsection{Structural congruence}

\paragraph{Free and bound names and alpha-equivalence.} At the
core of structural equivalence is alpha-equivalence which identifies
process that are the same up to a change of variable. Formally, we
recognize the distinction between free and bound names. The free names
of a process, $\freenames{P}$, may be calculated recursively as
follows:

\begin{mathpar}
\freenames{\pzero} := \emptyset
  \and \\
  \freenames{x?(y).P} := \{ x \} \cup (\freenames{P} \setminus \{ y \})
  \and 
  \freenames{x!\langle P \rangle} := \{ x \} \cup \{ P \} 
  \and \\
  \freenames{P|Q} := \freenames{P} \cup \freenames{Q}
  \and \\
  \freenames{@{x}} := \{ x \}
\end{mathpar}

$\pi$
$\quotep{\pi}$

$\freenames{-} : \pi \to \mathcal{P}(\quotep{\pi})$

\begin{eqnarray*}
  \freenames{\pzero} & := & \emptyset \\
  \freenames{x?(y).P} & := & \{ x \} \cup (\freenames{P} \setminus \{ y \}) \\
  \freenames{x!\langle P \rangle} & := & \{ x \} \cup \{ P \} \\
  \freenames{P|Q} & := & \freenames{P} \cup \freenames{Q} \\
  \freenames{\dropn{x}} & := & \{ x \}
\end{eqnarray*}

The bound names of a process, $\boundnames{P}$, are those names occurring in $P$
that are not free. For example, in $x?(y).0$, the name $x$ is free, while $y$ is bound.

\begin{mathpar}
  \inferrule* [lab=monoidal-laws] {} { P|Q \equiv Q|P \and P|0 \equiv P \and P|(Q|R) \equiv (P|Q)|R }
\end{mathpar}

\begin{mathpar}
  \inferrule* [lab=alpha-equivalence] {} { (x)P \equiv (y)P\{y/x\} \and y \not\in \freenames{P} }
\end{mathpar}

\begin{definition}
Then two processes, $P,Q$, are alpha-equivalent if $P = Q\{\vec{y}/\vec{x}\}$ for
some $\vec{x} \in \boundnames{Q},\vec{y} \in \boundnames{P}$, where $Q\{\vec{y}/\vec{x}\}$
denotes the capture-avoiding substitution of $\vec{y}$ for $\vec{x}$ in $Q$.
\end{definition}

\begin{definition}
  The {\em structural congruence} \cite{SangiorgiWalker} , $\equiv$,
  between processes is the least congruence containing
  alpha-equivalence, satisfying the abelian monoid laws
  (associativity, commutativity and $\pzero$ as identity) for parallel
  composition $|$ and for summation $+$.
\end{definition}

\subsection{Name equivalence}

We take name equivalence, written $\nameeq$, to be the smallest
equivalence relation generated by the following rules.

\begin{mathpar}
\inferrule*[lab=Quote-drop]
{ }
{ \quotep{@{x}} \nameeq x }

\inferrule*[lab=Struct-equiv]
{ P \scong Q }
{ \quotep{P} \nameeq \quotep{Q} }
\end{mathpar}

The astute reader will have noticed that the mutual recursion of names
and processes imposes a mutual recursion on alpha-equivalence and
structural equivalence via name-equivalence. Fortunately, all of this
works out pleasantly and we may calculate in the natural way, free of
concern. The reader interested in the details is referred to the
appendix \ref{appendix:rho_details}.

\subsection{Substitution}

We use $\Proc$ for the set of processes, $\QProc$ for the set of
names, and $\id{\{}\vec{y} / \vec{x} \id{\}}$ to denote partial maps,
$s : \QProc \rightarrow \QProc$. A map, $s$ lifts, uniquely, to a map
on process terms, $\widehat{s} : \Proc \rightarrow \Proc$ by the
following equations.

\begin{mathpar}
  (0) \psubstp{Q}{P} := 0 \\
  (R \juxtap S) \psubstp{Q}{P}
  :=    
  (R)\psubstp{Q}{P} \juxtap (S) \psubstp{Q}{P} \\
  (x?(y).R) \psubstp{Q}{P}    
  :=    
  (x)\substp{Q}{P} (z)\concat( (R \psubstn{z}{y}) \psubstp{Q}{P} ) \\
  (\lift{x}{R}) \psubstp{Q}{P}  
  :=
  \lift{(x)\substp{Q}{P}}{ R \psubstp{Q}{P} } \\
%   (\dropn{x})  \psubstp{Q}{P}       
%   := 
%   \left\{ 
%     \begin{array}{ccc} 
%       \dropn{\quotep{Q}} & & x \nameeq \quotep{P} \\
%       \dropn{x} & & otherwise \\
%     \end{array}
%   \right. 
  (\dropn{x})  \psubstp{Q}{P}       
  := 
  \left\{ 
    \begin{array}{ccc} 
      Q & & x \nameeq \quotep{P} \\
      \dropn{x} & & otherwise \\
    \end{array}
  \right.
\end{mathpar}
 

where

\begin{eqnarray}
  (x)\id{\{} \lpquote Q \rpquote / \lpquote P \rpquote \id{\}}            = 
  \left\{ 
    \begin{array}{ccc}
      \lpquote Q \rpquote & & x \nameeq \lpquote P \rpquote \\
      x & & otherwise \\
    \end{array}
  \right. \nonumber
\end{eqnarray}

and $z$ is chosen distinct from $\quotep{P}$, $\quotep{Q}$, the free
names in $Q$, and all the names in $R$. Our $\alpha$-equivalence will
be built in the standard way from this substitution.

\begin{remark}\label{rem:no_self_referential_names}
  One consequence of these definitions is that $\forall P. \quotep{P}
  \not\in \freenames{P}$.
\end{remark}

\subsection{ Dynamic quote: an example }

Anticipating something of what's to come, consider applying the
substitution, $\widehat{\id{\{}u / z \id{\}}}$, to the following pair
of processes, $\lift{w}{y!(z)}$ and $w[ \lpquote y!(z) \rpquote ]$.

\begin{eqnarray}
	\lift{w}{y!(z)}\widehat{\id{\{}u / z \id{\}}}
		& = &
		\lift{w}{y!(u)} \nonumber\\
	w[ \lpquote y!(z) \rpquote ] \widehat{ \id{\{}u / z \id{\}} }
		& = &
		w[ \lpquote y!(z) \rpquote ] \nonumber
\end{eqnarray}

Because the body of the process between quotes is impervious to
substitution, we get radically different answers. In fact, by
examining the first process in an input context,
e.g. $x?(z).\lift{w}{y!(z)}$, we see that the process under the lift
operator may be shaped by prefixed inputs binding a name inside it. In
this sense, the lift operator will be seen as a way to dynamically
construct processes before reifying them as names.

Finally equipped with these standard features we can present the
dynamics of the calculus.

\subsubsection{Operational semantics} 

Finally, we introduce the computational dynamics. What marks these
algebras as distinct from other more traditionally studied algebraic
structures, e.g. vector spaces or polynomial rings, is the manner in
which dynamics is captured. In traditional structures, dynamics is typically
expressed through morphisms between such structures, as in linear maps
between vector spaces or morphisms between rings. In algebras
associated with the semantics of computation, the dynamics is
expressed as part of the algebraic structure itself, through a
reduction reduction relation typically denoted by $\red$. Below, we
give a recursive presentation of this relation for the calculus used
in the encoding.

$\red \subseteq \pi \times \pi$
$\red : \pi \to \mathcal{P}(\pi)$

\begin{mathpar}
  \inferrule* [lab=Comm] { \textsf{match}( x_{src}, x_{trgt} ) } { x_{trgt}?(y)P \; | \; x_{src}!\langle {Q} \rangle \red P\{\quotep{Q}/y}\} }
  \and \\
  \inferrule* [lab=Par] {{P} \red {P}'} {{{P} | {Q}} \red {{P}' | {Q}}}
  \and
  \inferrule* [lab=Equiv]{{{P} \scong {P}'} \andalso {{P}' \red {Q}'} \andalso {{Q}' \scong {Q}}}{{P} \red {Q}}
\end{mathpar}

\begin{eqnarray*}
  match_{\equiv} (\quotep{P},\quotep{Q}) & := & P \equiv Q \\
  match_{\dagger}(\quotep{P},\quotep{Q}) & := & \forall R. P|Q \red^{*} R => R \red^{*} 0 \\
  match_{K}(\quotep{P},\quotep{Q}) & := & K \mbox{ for some context } K
\end{eqnarray*}

$u?(x)P | u!\langle Q \rangle \red P\{\quotep{Q}/x\}$

%We write $\wred$ for $\red^*$, and $P\red$ if $\exists Q $ such that $ P \red Q$.
We write $P\red$ if $\exists Q $ such that $ P \red Q$ and $P\not\red$, otherwise.

\section{Replication}

As mentioned before, it is known that replication (and hence
recursion) can be implemented in a higher-order process algebra
\cite{SangiorgiWalker}. As our first example of calculation with the
machinery thus far presented we give the construction explicitly in
the {\rhoc}.

\begin{eqnarray}
	D_{x} & := & \prefix{x}{y}{(\binpar{\outputp{x}{y}}{@{y}})} \nonumber\\
	\bangp_{x}{P} & := & \binpar{{x}!\langle{\binpar{D_{x}}{P}}\rangle}{D_{x}} \nonumber
\end{eqnarray}

\begin{eqnarray}
	\bangp_{x}{P} & & \nonumber\\
	=
	& {x}!\langle{(\prefix{x}{y}{(\outputp{x}{y} | @{y})) | P}}\rangle 
	      | \prefix{x}{y}{(\outputp{x}{y} | @{y})} & \nonumber\\
	\red
	& (\outputp{x}{y} | @{y})\substn{\quotep{(\prefix{x}{y}{(@{y} | \outputp{x}{y})) | P}}}{y} & \nonumber\\
	=
	& \outputp{x}{\quotep{(\prefix{x}{y}{(\outputp{x}{y} | @{y})) | P}}}
	  | {(\prefix{x}{y}{(\outputp{x}{y} | @{y})) | P}} & \nonumber\\
	\red
	& \ldots & \nonumber\\
	\red^*
	& P | P | \ldots & \nonumber
\end{eqnarray}

Of course, this encoding, as an implementation, runs away, unfolding
$\bangp{P}$ eagerly. A lazier and more implementable replication
operator, restricted to input-guarded processes, may be obtained as follows.

\begin{eqnarray}
\bangp{\prefix{u}{v}{P}} 
	:= 
	\binpar{\lift{x}{\prefix{u}{v}{(\binpar{D(x)}{P})}}}{D(x)} \nonumber
\end{eqnarray}

\begin{remark}
  Note that the lazier definition still does not deal with summation
  or mixed summation (i.e. sums over input and output). The reader is
  invited to construct definitions of replication that deal with these
  features. 

  Further, the definitions are parameterized in a name, $x$. Can you,
  gentle reader, make a definition that eliminates this parameter and
  guarantees no accidental interaction between the replication
  machinery and the process being replicated -- i.e. no accidental
  sharing of names used by the process to get its work done and the
  name(s) used by the replication to effect copying. This latter
  revision of the definition of replication is crucial to obtaining
  the expected identity $!!P \sim !P$.
\end{remark}

\begin{remark}\label{rem:paradoxical_combinator}
  The reader familiar with the lambda calculus will have noticed the
  similarity between $D$ and the paradoxical combinator.

  [Ed. note: the existence of this seems to suggest we have to be more
  restrictive on the set of processes and names we admit if we are to
  support no-cloning.]
\end{remark}

\subsubsection{Bisimulation}

The computational dynamics gives rise to another kind of equivalence,
the equivalence of computational behavior. As previously mentioned
this is typically captured \emph{via} some form of bisimulation.

% The notion we use in this paper is weak barbed bisimulation
% \cite{milner91polyadicpi}.

The notion we use in this paper is derived from weak barbed
bisimulation \cite{milner91polyadicpi}. 

\begin{definition}
An \emph{observation relation}, $\downarrow_{\mathcal N}$, over a set
of names, $\mathcal N$, is the smallest relation satisfying the rules
below.

\infrule[Out-barb]{y \in {\mathcal N}, \; x \nameeq y}
		  {\outputp{x}{v} \downarrow_{\mathcal N} x}
\infrule[Par-barb]{\mbox{$P\downarrow_{\mathcal N} x$ or $Q\downarrow_{\mathcal N} x$}}
		  {\binpar{P}{Q} \downarrow_{\mathcal N} x}

We write $P \Downarrow_{\mathcal N} x$ if there is $Q$ such that 
$P \wred Q$ and $Q \downarrow_{\mathcal N} x$.
\end{definition}

\begin{definition}
%\label{def.bbisim}
An  ${\mathcal N}$-\emph{barbed bisimulation} over a set of names, ${\mathcal N}$, is a symmetric binary relation 
${\mathcal S}_{\mathcal N}$ between agents such that $P\rel{S}_{\mathcal N}Q$ implies:
\begin{enumerate}
\item If $P \red P'$ then $Q \wred Q'$ and $P'\rel{S}_{\mathcal N} Q'$.
\item If $P\downarrow_{\mathcal N} x$, then $Q\Downarrow_{\mathcal N} x$.
\end{enumerate}
$P$ is ${\mathcal N}$-barbed bisimilar to $Q$, written
$P \wbbisim_{\mathcal N} Q$, if $P \rel{S}_{\mathcal N} Q$ for some ${\mathcal N}$-barbed bisimulation ${\mathcal S}_{\mathcal N}$.
\end{definition}

$\mathcal{R} \subseteq \pi \times \pi$

$P \mathcal{R} Q => \forall P'. P \red P' \Rightarrow \exists Q'. Q \red Q', P' \mathcal{R} Q'$

$P \vdash x \Rightarrow Q \vdash x$

\begin{mathpar}
  \inferrule*[lab=Out-barb]{x \nameeq y}{{y}!\langle{Q}\rangle \vdash x}
  \and
  \inferrule*[lab=Par-barb]{\mbox{$P\vdash x$ or $Q\vdash x$}}{\binpar{P}{Q} \vdash x}
\end{mathpar}

\subsubsection{Contexts}

One of the principle advantages of computational calculi like the
$\pi$-calculus is a well-defined notion of context,
contextual-equivalence and a correlation between
contextual-equivalence and notions of bisimulation. The notion of
context allows the decomposition of a process into (sub-)process and
its syntactic environment, its context. Thus, a context may be
thought of as a process with a ``hole'' (written $\Box$) in it. The
application of a context $M$ to a process $P$, written $M[P]$, is
tantamount to filling the hole in $M$ with $P$. In this paper we do
not need the full weight of this theory, but do make use of the notion
of context in the proof the main theorem. 

\begin{mathpar}
  \inferrule* [lab=summation] {} {{M_{M},M_{N}} \bc \Box \;|\; x.M_{A} \;|\; M_{M}+M_{N}}
  \and
  \inferrule* [lab=agent] {} {{M_{A}} \bc (\vec{x})M_{P} \;| \; \clift{P_0,\ldots,M_{P},\ldots,P_N}}
  \and \\
  \inferrule* [lab=process] {} {{M_{P}} \bc M_{N} \;| \;P|M_{P} }
\end{mathpar} 

\begin{mathpar}
  \inferrule* [lab=sychronization] {} {M_{N} \bc \Box \;|\; x?M_{F} \;|\; x!M_{C}}
  \and
  \inferrule* [lab=abstraction] {} {{M_{F}} \bc (x)M_{P} }
  \and
  \inferrule* [lab=concretion] {} {{M_{C}} \bc \langle M_{P} \rangle }
  \and \\
  \inferrule* [lab=process] {} {{M_{P}} \bc M_{N} \;| \;P|M_{P} }
\end{mathpar}

\begin{definition}[contextual application] Given a context $M$, and
  process $P$, we define the \emph{contextual application}, $M[P] :=
  M\{P/\Box\}$. That is, the contextual application of M to P is the
  substitution of $P$ for $\Box$ in $M$.
\end{definition}

$\meaningof{-} : L \to \mathcal{P}(\pi)$

\begin{mathpar}
  \inferrule* [lab=collection] {} {\meaningof{true} = \pi, \and \meaningof{~E} = \pi \setminus \meaningof{E}, \and \meaningof{E_{1} \& E_{2}} = \meaningof{E_{1}} \cap \meaningof{E_{2}}}
\end{mathpar}

\begin{mathpar}
  \inferrule* [lab=structure] {} {\meaningof{0} = \{ P \in \pi | P \equiv 0 \}, \and \\ \meaningof{E_1 | E_2} = \{ P \in \pi | P \equiv P_{1} | P_{2}, P_{1} \in \meaningof{E_{1}}, P_{2} \in \meaningof{E_2}\} }
\end{mathpar}

\begin{mathpar}
 \inferrule* [lab=behavior] {} {\meaningof{\langle a?b \rangle E} = \{ P \in \pi | P \equiv Q | u?(y)P', \\ \and \\\\ \and \\ \;\;\; u \in \meaningof{a}, \forall z.P'\{z/y\} \in \meaningof{E\{z/b\}}\}, \and \\ \meaningof{a!E} = \{ P \in \pi | P \equiv Q | x!\langle P' \rangle, x \in \meaningof{a} P' \in \meaningof{E}\} }
\end{mathpar}

\begin{mathpar}
 \inferrule* [lab=nominal] {} {\meaningof{\quotep{E}} = \{ \quotep{P} \in \quotep{\pi} | P \in \meaningof{E} \}, \and \meaningof{\quotep{P}} = \{ \quotep{Q} \in \quotep{\pi} | P \equiv Q \} \and \\ \meaningof{@\quotep{E}} = \{ P \in \pi | P \equiv @x, x \in \meaningof{E} \}}
\end{mathpar}

\begin{eqnarray*}
  \\
  \meaningof{-} : TS \to ST
\end{eqnarray*}

\begin{eqnarray*}
  \\
  L : TS \to ST
\end{eqnarray*}

\begin{eqnarray*}
  \\
  P \models E \iff P \in \meaningof{E}
\end{eqnarray*}

\begin{eqnarray*}
  P \approx_{L} Q \iff \forall E \in L. P \models E \iff Q \models E
\end{eqnarray*}

\begin{eqnarray*}
  P \approx_{K} Q
\end{eqnarray*}

\begin{eqnarray*}
  P \approx Q
\end{eqnarray*}

$\approx_{K} = \approx = \approx_{L}$

\subsubsection{Contextual duality}

Note that contexts extend the quotation operation to a family of
operations from processes to names. Given a context, $M$, we can
define a \emph{nominal context}, $\quotep{M}$ by $\quotep{M}[P] :=
\quotep{M[P]}$. To foreshadow what is to come we observe that these
operations enjoy a duality with processes very much like the duality
between vectors and maps from vectors to scalars.

Further, because the calculus is essentially higher-order, we have a
correspondence between contexts and processes. More specifically,
given a name $x$ and a context $M$ we can construct $M^{*}_{x}$ such
that 

\begin{mathpar}
  M^{*}_{x} | \lift{x}{P} \red M[P]
\end{mathpar}

namely,

\begin{mathpar}
  M^{*}_{x} := x?(u).M[\dropn{u}]
\end{mathpar}

The dependence of $M^{*}_{x}$ on a name makes it an abstraction, 

\begin{mathpar}
  M^{*} := (x)x?(u).M[\dropn{u}]
\end{mathpar}

\subsection{Additional notation}

It will sometimes be convenient to denote the process a name
quotes. We already have the notation $x = \quotep{P}$, but it will be
convenient to introduce an alternate notation, $\procn{x}$, when we
want to emphasize the connection to the use of the name. Note that, by
virtue of name equivalence, $\quotep{\procn{x}} \nameeq x$; so, the
notation is consistent with previous definitions.

Further, because names have structure it is possible to effect
substitutions on the basis of that structure. This means we need to
upgrade our notation for substitutions, which we accomplish by
adapting comprehension notation. Thus,

\begin{mathpar}
  P\{ y / x : x \in S \}
\end{mathpar}

is interpreted to mean the process derived from P by replacing (in a
capture-avoiding manner) each occurrence of $x$ in $S$ by $y$. For example,

\begin{mathpar}
  P\{ \quotep{\procn{x}|\procn{x}} / x : x \in \freenames{P} \}
\end{mathpar}

will replace each (occurrence) of a free name $x$ in $P$ by
$\quotep{\procn{x}|\procn{x}}$.

Also, we will avail ourselves of the notation $x^{L}$ and $x^{R}$ to
denote injections of a name into disjoint copies of the name
space. There are numerous ways to accomplish this. One example can be
found in \cite{MeredithR05}. This notation overloads to vectors of
names: $\vec{x}^{\pi} := (x_{i}^{\pi} \; : \; 0 \leq i < |\vec{x}| )$ where $\pi \in \{L,R\}$.

We also use $P^{\Box} := P|\Box$.

In \cite{MeredithR05} an interpretation of the new operator is
given. It turns out that there are several possible interpretations
all enjoying the requisite algebraic properties of the operator (see
\cite{milner91polyadicpi}). We will therefore make liberal use of
$(\nu\; \vec{x})P$.

% subsection the_syntax_and_semantics_of_the_notation_system (end)   

\section{Interpretation of QM}
\subsection{Supporting definitions}
\subsubsection{Multiplication}
\begin{mathpar}
  \quotep{Q} \cdot \quotep{R} := \quotep{Q|R}
  \and \\
  \quotep{Q} \cdot P := P\{ \quotep{Q|R} / \quotep{R} : \quotep{R} \in \freenames{P} \}
\end{mathpar}

\paragraph{Discussion}
The first line needs little explanation. The second line says that
each free name of the process is replaced with the multiplication of
that name by the scalar. Multiplication of a scalar (name) by a state
(process) results in a process all the names of which have been `moved
over' by parallel composition with the process the scalar
quotes. There is a subtlety that the bound names have to be
manipulated so that multiplied names aren't accidentally
captured. There are many ways to achieve this.

\begin{remark}\label{rem:multiplication_identities}
  The reader is invited to verify that for all $x,y,z \in \QProc$ and $P \in \Proc$
  \begin{mathpar}
    x \cdot \quotep{0} \equiv x 
    \and
    x \cdot y \equiv y \cdot x
    \and
    x \cdot (y \cdot z) \equiv (x \cdot y) \cdot z
    \and \\
    \quotep{0} \cdot P \equiv P
    \and \\
    x \cdot (y \cdot P) \equiv (x \cdot y) \cdot P
    \and \\
    x \cdot (P|Q) \equiv (x \cdot P) | (x \cdot Q)
    \and \\    
  \end{mathpar}
\end{remark}

\subsubsection{Tensor product}

We define a tensor product on processes by structural induction.

\paragraph{Tensor of sums} First note that all summations, including
$\pzero$ and sequence, can be written $\Sigma_{i} x_{i}.A_{i} +
\Sigma_{j} x_{j}.C_{j}$, where we have grouped input-guarded processes
together and output-guarded processes together.

Thus, we can define the tensor product of two summations, $N_{1}\otimes N_{2}$, where

\begin{mathpar}
  N_{1} := \Sigma_{i} x_{i}.A_{i} + \Sigma_{j} x_{j}.C_{j}
  \and
  N_{2} := \Sigma_{i'} y_{i'}.B_{i'} + \Sigma_{j'} y_{j'}.D_{j'} 
\end{mathpar}

as follows.

\begin{mathpar}
  \Sigma_{i} x_{i}.A_{i} + \Sigma_{j} x_{j}.C_{j} \otimes \Sigma_{i'}
  y_{i'}.B_{i'} + \Sigma_{j'} y_{j'}.D_{j'} 
  \and \\
  := \; \Sigma_{i} \Sigma_{i'} \quotep{\stackrel{\vee}{x_{i}}| \stackrel{\vee}{y_{i'}}}.(A_{i}\otimes B_{i'}) \; | \; \Sigma_{i'} \Sigma_{i} \quotep{\stackrel{\vee}{y_{i'}}|\stackrel{\vee}{x_{i}}}.(B_{i'}\otimes A_{i})
  \and
  \;\; | \;\; \Sigma_{j} \Sigma_{j'} \quotep{\stackrel{\vee}{x_{j}}|\stackrel{\vee}{y_{j'}}}.(A_{j}\otimes B_{j'}) \; | \; \Sigma_{j'} \Sigma_{j} \quotep{\stackrel{\vee}{y_{j'}}|\stackrel{\vee}{x_{j}}}.(B_{j'}\otimes A_{j})
\end{mathpar}

\begin{remark}
  Do we need to $x^{L}$ and $y^{R}$ for this construction as well?
\end{remark}

\paragraph{Tensor of parallel compositions} Next, we distribute tensor
over par.

\begin{mathpar}
  P_{1}|P_{2} \otimes Q_{1}|Q_{2} := (P_{1} \otimes Q_{1}) | (P_{1}
  \otimes Q_{2}) | (P_{2} \otimes Q_{1}) | (P_{2} \otimes Q_{2})
\end{mathpar}

\paragraph{Tensor with dropped names} We treat tensor of a
process with a dropped name as parallel composition.

\begin{mathpar}
  P \otimes \dropn{x} := P | \dropn{x}
\end{mathpar}

\paragraph{Tensor of agents}

Finally, we need to define tensor on agents. Note that the definition
of tensor on normal products only tensors inputs with inputs and
outputs with outputs. Thus, we only have to define the operation on
``homogeneous'' pairings.

\begin{mathpar}
  (\vec{x})P \otimes (\vec{y})Q
  \and \\
  := (x_{0}^{L}|y_{0}^{R},\ldots,x_{0}^{L}|y_{n}^{R},\ldots,x_{m}^{L}|y_{0}^{R},\ldots,x_{m}^{L}|y_{n}^R)(P\{ \vec{x}^{L}/\vec{x}\} \otimes Q \{ \vec{y}^{R}/\vec{y}\})
  \and \\
  \clift{\vec{P}} \otimes \clift{\vec{Q}}
  \and \\
  := \clift{P_{0}\otimes Q_{0},\ldots,P_{0}\otimes Q_{n},\ldots,P_{m}\otimes Q_{0},\ldots,P_{m}\otimes Q_{n}}
\end{mathpar}

\begin{remark}
  Observe that arities of tensored abstractions matches arities of
  tensored concretions if the original arities matched. Note also that
  the length of the arities corresponds to the increase in dimension
  we see in ordinary vector space tensor product.
\end{remark}

\begin{remark}
  Operationally, this definition distributes the tensor down to
  components ``linked'' by summation. Tensor over summation is
  intriguing in that it mixes names. Moreover, as a consequence of the
  way it mixes names we have the identities for all $x \in \QProc$ and
  $P,Q \in \Proc$

  \begin{mathpar}
    (x \cdot P) \otimes Q \equiv x \cdot (P \otimes Q) \equiv P \otimes (x \cdot Q)
    \and
    P \otimes \pzero \equiv P
  \end{mathpar}

  that the reader is invited to verify.
\end{remark}

\subsubsection{Annihilation}
\begin{mathpar}
  P^{\perp} := \{ Q | \forall R. P|Q \red^{*} R \Rightarrow R \red^{*} \pzero \}
  \and \\
  P^{\underline{\perp}} := \Sigma_{Q \in P^{\perp}} \quotep{Q}?(y).(\dropn{y}|Q) | \Sigma_{Q \in P^{\perp}} \quotep{Q}\clift{\Box}
\end{mathpar}

\paragraph{Discussion} The reader will note that $P^{\perp}$ is a
\emph{set} of processes, while $P^{\underline{\perp}}$ is a
\emph{context}. We call the set $P^{\perp}$ the \emph{annihilators} of
$P$. The parallel composition of a process in the annihilators of $P$
with $P$ will result in a process, the state space of which has all
paths eventually leading to $\pzero$. Execution may endure loops; but
under reasonable conditions of fairness (naturally guaranteed under
most notions of bisimulation) such a composite process cannot get
stuck in such a loop and will, eventually pop out and terminate.

The context $P^{\underline{\perp}}$ is ready and willing to ``take the
$P$ out of'' the process to which it is applied. It will effectively
transmit the code of the process to which it is applied to one of the
annihilators and run the process against it.

\subsubsection{Evaluation}
We fix $M$ a domain of fully abstract interpretation with an equality
coincident with bisimulation. We take $\meaningof{\cdot} : \Proc \to
M$ to be the map interpreting processes and $\nmeaningof{\cdot} : \M
\to Proc$ to be the map running the other way. Then we define

\begin{mathpar}
  \int P := \nmeaningof{\meaningof{P}}
\end{mathpar}

\paragraph{Discussion}
There are many fully abstract interpretations of Milner's
$\pi$-calculus. Any of them can be used as a basis for interpreting
the reflective calculus here. Equipped with such a domain it is
largely a matter of grinding through to check that the Yoneda
construction for the normalization-by-evaluation program can be
extended to this setting.

\begin{remark}
  The reader is invited to verify that $\int (P^{\underline{\perp}}[P]) = 0$.
\end{remark}

\subsection{Quantum mechanics}

Table \ref{tbl:core_qm_op_defns} gives the core operational definitions

\begin{table}[htp]\label{tbl:core_qm_op_defns}
  \center{
    \fbox{
      \begin{tabular}{c|c}
        quantum mechanics & process calculus \\
        \hline
        scalar & $x := \quotep{P}$ \\
        state vector & $\state{P} := P$ \\
        dual & $\state{P}^{*} := \event{P^{\underline{\perp}}} := \quotep{P^{\underline{\perp}}}[-]$ \\
        matrix & $ \Sigma_{\alpha} \state{P_{\alpha}}x_{\alpha}\event{Q_{\alpha}}$ \\
        vector addition & $\state{P} + \state{Q} := \state{P | Q}$ \\
        tensor product & $\state{P} \otimes \state{Q} := \state{P \otimes Q}$ \\
        inner product & $\innerprod{P}{Q} := \quotep{\int P^{\underline{\perp}}[Q]}$ \\
      \end{tabular}
    }
  }
  \caption{QM - operational definitions}
\end{table}

where

\begin{mathpar}
  \prmatrix{P}{Q} := \fprmatrix{P}{\quotep{\pzero}}{Q}
  \and
  \fprmatrix{P}{x}{Q} := (\state{P},x,\event{Q})
  \and
  (\fprmatrix{P}{x}{Q})(\state{R}) := x \cdot \innerprod{Q}{R} \cdot \state{P}
  \and
  (\fprmatrix{P}{x}{Q})(\event{R}) := x \cdot \innerprod{R}{P} \cdot \event{Q}
\end{mathpar}

\paragraph{Discussion}
As promised: vectors (aka states) are represented as processes; duals
as contextual duals; inner product definition should be compared with
standard inner product definition for ....

\begin{remark}
  Assuming $\int (P^{\underline{\perp}}[P]) = 0$, the reader is
  invited to verify that $(\fprmatrix{P}{x}{P})(\state{P}) = x \cdot \state{P}$.
\end{remark}

\begin{remark}
  The reader is invited to verify that $\innerprod{P}{Q}$ could
  equally well have been written $\quotep{\int \stackrel{\vee}{x}}$
  where $x = \event{P^{\underline{\perp}}}(Q)$.

  One of the motivations for this remark is that there is another way
  to factor these operations. We could package up evaluation in the dual:

  \begin{mathpar}
    \state{P}^{*} := \event{\int P^{\underline{\perp}}} := \quotep{\int P^{\underline{\perp}}}[-]
  \end{mathpar}

  and then have inner product defined by
  
  \begin{mathpar}
    \innerprod{P}{Q} := \event{P}(Q)
  \end{mathpar}

  Hopefully, experience with the calculations will provide guidance on
  the best factoring.
\end{remark}

\begin{remark}
  Assuming $\int (P^{\underline{\perp}}[P]) = 0$, the reader is
  invited to verify that $\forall P,Q. (\prmatrix{0}{Q})(\state{0}) =
  \state{0}$ and dually $(\prmatrix{P}{0})(\event{0}) = \event{0}$.
\end{remark}

\begin{remark}
  i'm a little worried that i don't (yet) have proper support for
  complex conjugacy. But, the observation above may give us a
  clue. According to Abramsky, it must be the case that the scalars
  are iso to the homset of the identity for the tensor -- which the
  observation above characterizes. 

  For now, we will simply bookmark the notion with $\overline{x}$.
\end{remark}

\subsubsection{Adjointness}

We need to give a definition of $(\cdot)^{\dagger}$ for matrices. The
obvious candidate definition is
\begin{mathpar}
(\Sigma_{\alpha}\fprmatrix{P_{\alpha}}{x_{\alpha}}{Q_{\alpha}})^{\dagger}
= \Sigma_{\alpha}\fprmatrix{(Q_{\alpha}^{\underline{\perp}})^{*}}{\overline{x}_{\alpha}}{P_{\alpha}^{\underline{\perp}}} 
\end{mathpar}

But, $(Q_{\alpha}^{\underline{\perp}})^{*}$ requires a name along
which to communicate the process to achieve the context application.

\subsubsection{Basis for a basis}
If processes label states and ``addition'' of states (a.k.a. vector
addition) is interpreted as parallel composition, what corresponds to
notions of linear independence and basis? Here, we recall that Yoshida
has developed a set of \emph{combinators} for an asynchronous verison
of Milner's $\pi$-calculus. These are a finite set of processes such
any process can be expressed as parallel composition of these
combinators together with liberal uses of the new operator and
replication. We can simply give a translation of these into the
present calculus and have reasonable expectation that the property
carries over. That is, that the resultant set allows to express all
processes via parallel composition. Note, however, that there is no
new operator or replication in this calculus. As a result, we expect
that the corresponding set is actually infinite. That is, we expect
that the space is actually infinite dimensional.

\begin{remark}
  The attentive reader may be a bit concerned. Certainly, the
  collection $S$, $K$ and $I$ is a finite set of
  combinators. Shouldn't we expect to see a finite set of combinators
  for an effectively equivalent system? i am very sympathetic to this
  critique and feel it warrants full attention. On the other hand, i
  also have in mind the following analogy. The natural numbers, as a
  monoid under addition, has exactly $1$ generator, while the natural
  numbers, as a monoid under multiplication, has countably many
  generators (the primes). We observe that the application of the
  lambda calculus is much less resource sensitive than the parallel
  composition of the $\pi$-calculus. Could it be the case that we have
  an analogy of the form
  
  \begin{mathpar}
    m + n : MN :: m*n : M|N
  \end{mathpar}

  giving a similar blow up in the set of ``primes''?  This is such a
  wonderful thought that, even if it's not true, i think it's worth
  writing down.
\end{remark}
 

\documentclass[12pt]{llncs}
%\documentclass{jktr}

\usepackage[pdftex]{hyperref}                   
\usepackage {listings}
\usepackage {mathpartir}
\usepackage{bcprules}
%\usepackage{listings}
                       
\usepackage{graphicx} 
%\usepackage[margins=2.5cm,nohead,nofoot]{geometry}
%\usepackage{geometry}
\usepackage{amsfonts}
\usepackage{amstext}
\usepackage{latexsym}
\usepackage{amssymb}
\usepackage{color}


%\include{myPreamble}
\include{qm2pi.local} 

%\ifpdf
%\usepackage[pdftex]{graphicx}
%\else
%\usepackage{graphicx}
%\fi

 % \ifpdf
%  \usepackage{pdfsync}
%  \if


%\title{Brief Article}
%\author{David F. Snyder}
%\author{L.G. Meredith}

%\address{Dept. of Math., Texas State University--San Marcos, San Marcos, TX 78666}
       
\pagestyle{empty}


\begin{document}

\lstset{language=[Objective]Caml,frame=shadowbox}

\input{qm2pi.front}

% section front matter (end)

\input{qm2pi.intro} 
 
% section introduction (end)

% \input{qm2pi.knotations} 

% section notation (end)

\input{qm2pi.process.calculi} 

% section concurrent_process_calculi_and_spatial_logics_ (end)
    
%\input{qm2pi.knots2pi} 

%\input{qm2pi.trefoil} 

%\input{qm2pi.mainthm} 

% subsection basic_interpretation (end)

%\input{qm2pi.rho.presentation} 
\subsection{The syntax and semantics of the notation system}\label{sub:the_syntax_and_semantics_of_the_notation_system} % (fold)

We now summarize a technical presentation of the calculus that
embodies our theory of dynamics. The typical presentation of such a
calculus follows the style of giving generators and relations on
them. The grammar, below, describing term constructors, freely
generates the set of processes, $\Proc$. This set is then quotiented
by a relation known as structural congruence and it is over this set
that the notion of dynamics is expressed. This presentation is
essentially that of \cite{MeredithR05} with the addition of
polyadicity and summation. For readability we have relegated some of
the technical subtleties to an appendix.

\subsubsection{Process grammar}\label{subsub:process_grammar}

\begin{mathpar}
  \inferrule* [lab=synchronization] {} {{M} \bc \pzero \;|\; x?F \;|\; x!C }
  \and
  \inferrule* [lab=abstraction] {} {{F} \bc (x)P}
  \and
  \inferrule* [lab=concretion] {} {{C} \bc \langle Q \rangle}
  \and
  \inferrule* [lab=process] {} {{P,Q} \bc M \;| \;P|Q \;|\; @{x}}
  \and
  \inferrule* [lab=name] {} {{x} \bc \quotep{P}}
\end{mathpar} 

Note that $\vec{x}$ (resp. $\vec{P}$) denotes a vector of names
(resp. processes) of length $|\vec{x}|$ (resp. $|\vec{P}|$). We adopt
the following useful abbreviations.

\begin{mathpar}
   x?(\vec{y}).P := x.(\vec{y})P \and  x\clift{\vec{P}} := x.\clift{\vec{P}}
   \and x!(y) := \lift{x}{\dropn{y}}
   \and \Pi_{i=0}^{n-1}P_i := P_0 | \ldots | P_{n-1}
\end{mathpar}

\subsubsection{Structural congruence}

\paragraph{Free and bound names and alpha-equivalence.} At the
core of structural equivalence is alpha-equivalence which identifies
process that are the same up to a change of variable. Formally, we
recognize the distinction between free and bound names. The free names
of a process, $\freenames{P}$, may be calculated recursively as
follows:

\begin{mathpar}
\freenames{\pzero} := \emptyset
  \and \\
  \freenames{x?(y).P} := \{ x \} \cup (\freenames{P} \setminus \{ y \})
  \and 
  \freenames{x!\langle P \rangle} := \{ x \} \cup \{ P \} 
  \and \\
  \freenames{P|Q} := \freenames{P} \cup \freenames{Q}
  \and \\
  \freenames{@{x}} := \{ x \}
\end{mathpar}

$\pi$
$\quotep{\pi}$

$\freenames{-} : \pi \to \mathcal{P}(\quotep{\pi})$

\begin{eqnarray*}
  \freenames{\pzero} & := & \emptyset \\
  \freenames{x?(y).P} & := & \{ x \} \cup (\freenames{P} \setminus \{ y \}) \\
  \freenames{x!\langle P \rangle} & := & \{ x \} \cup \{ P \} \\
  \freenames{P|Q} & := & \freenames{P} \cup \freenames{Q} \\
  \freenames{\dropn{x}} & := & \{ x \}
\end{eqnarray*}

The bound names of a process, $\boundnames{P}$, are those names occurring in $P$
that are not free. For example, in $x?(y).0$, the name $x$ is free, while $y$ is bound.

\begin{mathpar}
  \inferrule* [lab=monoidal-laws] {} { P|Q \equiv Q|P \and P|0 \equiv P \and P|(Q|R) \equiv (P|Q)|R }
\end{mathpar}

\begin{mathpar}
  \inferrule* [lab=alpha-equivalence] {} { (x)P \equiv (y)P\{y/x\} \and y \not\in \freenames{P} }
\end{mathpar}

\begin{definition}
Then two processes, $P,Q$, are alpha-equivalent if $P = Q\{\vec{y}/\vec{x}\}$ for
some $\vec{x} \in \boundnames{Q},\vec{y} \in \boundnames{P}$, where $Q\{\vec{y}/\vec{x}\}$
denotes the capture-avoiding substitution of $\vec{y}$ for $\vec{x}$ in $Q$.
\end{definition}

\begin{definition}
  The {\em structural congruence} \cite{SangiorgiWalker} , $\equiv$,
  between processes is the least congruence containing
  alpha-equivalence, satisfying the abelian monoid laws
  (associativity, commutativity and $\pzero$ as identity) for parallel
  composition $|$ and for summation $+$.
\end{definition}

\subsection{Name equivalence}

We take name equivalence, written $\nameeq$, to be the smallest
equivalence relation generated by the following rules.

\begin{mathpar}
\inferrule*[lab=Quote-drop]
{ }
{ \quotep{@{x}} \nameeq x }

\inferrule*[lab=Struct-equiv]
{ P \scong Q }
{ \quotep{P} \nameeq \quotep{Q} }
\end{mathpar}

The astute reader will have noticed that the mutual recursion of names
and processes imposes a mutual recursion on alpha-equivalence and
structural equivalence via name-equivalence. Fortunately, all of this
works out pleasantly and we may calculate in the natural way, free of
concern. The reader interested in the details is referred to the
appendix \ref{appendix:rho_details}.

\subsection{Substitution}

We use $\Proc$ for the set of processes, $\QProc$ for the set of
names, and $\id{\{}\vec{y} / \vec{x} \id{\}}$ to denote partial maps,
$s : \QProc \rightarrow \QProc$. A map, $s$ lifts, uniquely, to a map
on process terms, $\widehat{s} : \Proc \rightarrow \Proc$ by the
following equations.

\begin{mathpar}
  (0) \psubstp{Q}{P} := 0 \\
  (R \juxtap S) \psubstp{Q}{P}
  :=    
  (R)\psubstp{Q}{P} \juxtap (S) \psubstp{Q}{P} \\
  (x?(y).R) \psubstp{Q}{P}    
  :=    
  (x)\substp{Q}{P} (z)\concat( (R \psubstn{z}{y}) \psubstp{Q}{P} ) \\
  (\lift{x}{R}) \psubstp{Q}{P}  
  :=
  \lift{(x)\substp{Q}{P}}{ R \psubstp{Q}{P} } \\
%   (\dropn{x})  \psubstp{Q}{P}       
%   := 
%   \left\{ 
%     \begin{array}{ccc} 
%       \dropn{\quotep{Q}} & & x \nameeq \quotep{P} \\
%       \dropn{x} & & otherwise \\
%     \end{array}
%   \right. 
  (\dropn{x})  \psubstp{Q}{P}       
  := 
  \left\{ 
    \begin{array}{ccc} 
      Q & & x \nameeq \quotep{P} \\
      \dropn{x} & & otherwise \\
    \end{array}
  \right.
\end{mathpar}
 

where

\begin{eqnarray}
  (x)\id{\{} \lpquote Q \rpquote / \lpquote P \rpquote \id{\}}            = 
  \left\{ 
    \begin{array}{ccc}
      \lpquote Q \rpquote & & x \nameeq \lpquote P \rpquote \\
      x & & otherwise \\
    \end{array}
  \right. \nonumber
\end{eqnarray}

and $z$ is chosen distinct from $\quotep{P}$, $\quotep{Q}$, the free
names in $Q$, and all the names in $R$. Our $\alpha$-equivalence will
be built in the standard way from this substitution.

\begin{remark}\label{rem:no_self_referential_names}
  One consequence of these definitions is that $\forall P. \quotep{P}
  \not\in \freenames{P}$.
\end{remark}

\subsection{ Dynamic quote: an example }

Anticipating something of what's to come, consider applying the
substitution, $\widehat{\id{\{}u / z \id{\}}}$, to the following pair
of processes, $\lift{w}{y!(z)}$ and $w[ \lpquote y!(z) \rpquote ]$.

\begin{eqnarray}
	\lift{w}{y!(z)}\widehat{\id{\{}u / z \id{\}}}
		& = &
		\lift{w}{y!(u)} \nonumber\\
	w[ \lpquote y!(z) \rpquote ] \widehat{ \id{\{}u / z \id{\}} }
		& = &
		w[ \lpquote y!(z) \rpquote ] \nonumber
\end{eqnarray}

Because the body of the process between quotes is impervious to
substitution, we get radically different answers. In fact, by
examining the first process in an input context,
e.g. $x?(z).\lift{w}{y!(z)}$, we see that the process under the lift
operator may be shaped by prefixed inputs binding a name inside it. In
this sense, the lift operator will be seen as a way to dynamically
construct processes before reifying them as names.

Finally equipped with these standard features we can present the
dynamics of the calculus.

\subsubsection{Operational semantics} 

Finally, we introduce the computational dynamics. What marks these
algebras as distinct from other more traditionally studied algebraic
structures, e.g. vector spaces or polynomial rings, is the manner in
which dynamics is captured. In traditional structures, dynamics is typically
expressed through morphisms between such structures, as in linear maps
between vector spaces or morphisms between rings. In algebras
associated with the semantics of computation, the dynamics is
expressed as part of the algebraic structure itself, through a
reduction reduction relation typically denoted by $\red$. Below, we
give a recursive presentation of this relation for the calculus used
in the encoding.

$\red \subseteq \pi \times \pi$
$\red : \pi \to \mathcal{P}(\pi)$

\begin{mathpar}
  \inferrule* [lab=Comm] { \textsf{match}( x_{src}, x_{trgt} ) } { x_{trgt}?(y)P \; | \; x_{src}!\langle {Q} \rangle \red P\{\quotep{Q}/y}\} }
  \and \\
  \inferrule* [lab=Par] {{P} \red {P}'} {{{P} | {Q}} \red {{P}' | {Q}}}
  \and
  \inferrule* [lab=Equiv]{{{P} \scong {P}'} \andalso {{P}' \red {Q}'} \andalso {{Q}' \scong {Q}}}{{P} \red {Q}}
\end{mathpar}

\begin{eqnarray*}
  match_{\equiv} (\quotep{P},\quotep{Q}) & := & P \equiv Q \\
  match_{\dagger}(\quotep{P},\quotep{Q}) & := & \forall R. P|Q \red^{*} R => R \red^{*} 0 \\
  match_{K}(\quotep{P},\quotep{Q}) & := & K \mbox{ for some context } K
\end{eqnarray*}

$u?(x)P | u!\langle Q \rangle \red P\{\quotep{Q}/x\}$

%We write $\wred$ for $\red^*$, and $P\red$ if $\exists Q $ such that $ P \red Q$.
We write $P\red$ if $\exists Q $ such that $ P \red Q$ and $P\not\red$, otherwise.

\section{Replication}

As mentioned before, it is known that replication (and hence
recursion) can be implemented in a higher-order process algebra
\cite{SangiorgiWalker}. As our first example of calculation with the
machinery thus far presented we give the construction explicitly in
the {\rhoc}.

\begin{eqnarray}
	D_{x} & := & \prefix{x}{y}{(\binpar{\outputp{x}{y}}{@{y}})} \nonumber\\
	\bangp_{x}{P} & := & \binpar{{x}!\langle{\binpar{D_{x}}{P}}\rangle}{D_{x}} \nonumber
\end{eqnarray}

\begin{eqnarray}
	\bangp_{x}{P} & & \nonumber\\
	=
	& {x}!\langle{(\prefix{x}{y}{(\outputp{x}{y} | @{y})) | P}}\rangle 
	      | \prefix{x}{y}{(\outputp{x}{y} | @{y})} & \nonumber\\
	\red
	& (\outputp{x}{y} | @{y})\substn{\quotep{(\prefix{x}{y}{(@{y} | \outputp{x}{y})) | P}}}{y} & \nonumber\\
	=
	& \outputp{x}{\quotep{(\prefix{x}{y}{(\outputp{x}{y} | @{y})) | P}}}
	  | {(\prefix{x}{y}{(\outputp{x}{y} | @{y})) | P}} & \nonumber\\
	\red
	& \ldots & \nonumber\\
	\red^*
	& P | P | \ldots & \nonumber
\end{eqnarray}

Of course, this encoding, as an implementation, runs away, unfolding
$\bangp{P}$ eagerly. A lazier and more implementable replication
operator, restricted to input-guarded processes, may be obtained as follows.

\begin{eqnarray}
\bangp{\prefix{u}{v}{P}} 
	:= 
	\binpar{\lift{x}{\prefix{u}{v}{(\binpar{D(x)}{P})}}}{D(x)} \nonumber
\end{eqnarray}

\begin{remark}
  Note that the lazier definition still does not deal with summation
  or mixed summation (i.e. sums over input and output). The reader is
  invited to construct definitions of replication that deal with these
  features. 

  Further, the definitions are parameterized in a name, $x$. Can you,
  gentle reader, make a definition that eliminates this parameter and
  guarantees no accidental interaction between the replication
  machinery and the process being replicated -- i.e. no accidental
  sharing of names used by the process to get its work done and the
  name(s) used by the replication to effect copying. This latter
  revision of the definition of replication is crucial to obtaining
  the expected identity $!!P \sim !P$.
\end{remark}

\begin{remark}\label{rem:paradoxical_combinator}
  The reader familiar with the lambda calculus will have noticed the
  similarity between $D$ and the paradoxical combinator.

  [Ed. note: the existence of this seems to suggest we have to be more
  restrictive on the set of processes and names we admit if we are to
  support no-cloning.]
\end{remark}

\subsubsection{Bisimulation}

The computational dynamics gives rise to another kind of equivalence,
the equivalence of computational behavior. As previously mentioned
this is typically captured \emph{via} some form of bisimulation.

% The notion we use in this paper is weak barbed bisimulation
% \cite{milner91polyadicpi}.

The notion we use in this paper is derived from weak barbed
bisimulation \cite{milner91polyadicpi}. 

\begin{definition}
An \emph{observation relation}, $\downarrow_{\mathcal N}$, over a set
of names, $\mathcal N$, is the smallest relation satisfying the rules
below.

\infrule[Out-barb]{y \in {\mathcal N}, \; x \nameeq y}
		  {\outputp{x}{v} \downarrow_{\mathcal N} x}
\infrule[Par-barb]{\mbox{$P\downarrow_{\mathcal N} x$ or $Q\downarrow_{\mathcal N} x$}}
		  {\binpar{P}{Q} \downarrow_{\mathcal N} x}

We write $P \Downarrow_{\mathcal N} x$ if there is $Q$ such that 
$P \wred Q$ and $Q \downarrow_{\mathcal N} x$.
\end{definition}

\begin{definition}
%\label{def.bbisim}
An  ${\mathcal N}$-\emph{barbed bisimulation} over a set of names, ${\mathcal N}$, is a symmetric binary relation 
${\mathcal S}_{\mathcal N}$ between agents such that $P\rel{S}_{\mathcal N}Q$ implies:
\begin{enumerate}
\item If $P \red P'$ then $Q \wred Q'$ and $P'\rel{S}_{\mathcal N} Q'$.
\item If $P\downarrow_{\mathcal N} x$, then $Q\Downarrow_{\mathcal N} x$.
\end{enumerate}
$P$ is ${\mathcal N}$-barbed bisimilar to $Q$, written
$P \wbbisim_{\mathcal N} Q$, if $P \rel{S}_{\mathcal N} Q$ for some ${\mathcal N}$-barbed bisimulation ${\mathcal S}_{\mathcal N}$.
\end{definition}

$\mathcal{R} \subseteq \pi \times \pi$

$P \mathcal{R} Q => \forall P'. P \red P' \Rightarrow \exists Q'. Q \red Q', P' \mathcal{R} Q'$

$P \vdash x \Rightarrow Q \vdash x$

\begin{mathpar}
  \inferrule*[lab=Out-barb]{x \nameeq y}{{y}!\langle{Q}\rangle \vdash x}
  \and
  \inferrule*[lab=Par-barb]{\mbox{$P\vdash x$ or $Q\vdash x$}}{\binpar{P}{Q} \vdash x}
\end{mathpar}

\subsubsection{Contexts}

One of the principle advantages of computational calculi like the
$\pi$-calculus is a well-defined notion of context,
contextual-equivalence and a correlation between
contextual-equivalence and notions of bisimulation. The notion of
context allows the decomposition of a process into (sub-)process and
its syntactic environment, its context. Thus, a context may be
thought of as a process with a ``hole'' (written $\Box$) in it. The
application of a context $M$ to a process $P$, written $M[P]$, is
tantamount to filling the hole in $M$ with $P$. In this paper we do
not need the full weight of this theory, but do make use of the notion
of context in the proof the main theorem. 

\begin{mathpar}
  \inferrule* [lab=summation] {} {{M_{M},M_{N}} \bc \Box \;|\; x.M_{A} \;|\; M_{M}+M_{N}}
  \and
  \inferrule* [lab=agent] {} {{M_{A}} \bc (\vec{x})M_{P} \;| \; \clift{P_0,\ldots,M_{P},\ldots,P_N}}
  \and \\
  \inferrule* [lab=process] {} {{M_{P}} \bc M_{N} \;| \;P|M_{P} }
\end{mathpar} 

\begin{mathpar}
  \inferrule* [lab=sychronization] {} {M_{N} \bc \Box \;|\; x?M_{F} \;|\; x!M_{C}}
  \and
  \inferrule* [lab=abstraction] {} {{M_{F}} \bc (x)M_{P} }
  \and
  \inferrule* [lab=concretion] {} {{M_{C}} \bc \langle M_{P} \rangle }
  \and \\
  \inferrule* [lab=process] {} {{M_{P}} \bc M_{N} \;| \;P|M_{P} }
\end{mathpar}

\begin{definition}[contextual application] Given a context $M$, and
  process $P$, we define the \emph{contextual application}, $M[P] :=
  M\{P/\Box\}$. That is, the contextual application of M to P is the
  substitution of $P$ for $\Box$ in $M$.
\end{definition}

$\meaningof{-} : L \to \mathcal{P}(\pi)$

\begin{mathpar}
  \inferrule* [lab=collection] {} {\meaningof{true} = \pi, \and \meaningof{~E} = \pi \setminus \meaningof{E}, \and \meaningof{E_{1} \& E_{2}} = \meaningof{E_{1}} \cap \meaningof{E_{2}}}
\end{mathpar}

\begin{mathpar}
  \inferrule* [lab=structure] {} {\meaningof{0} = \{ P \in \pi | P \equiv 0 \}, \and \\ \meaningof{E_1 | E_2} = \{ P \in \pi | P \equiv P_{1} | P_{2}, P_{1} \in \meaningof{E_{1}}, P_{2} \in \meaningof{E_2}\} }
\end{mathpar}

\begin{mathpar}
 \inferrule* [lab=behavior] {} {\meaningof{\langle a?b \rangle E} = \{ P \in \pi | P \equiv Q | u?(y)P', \\ \and \\\\ \and \\ \;\;\; u \in \meaningof{a}, \forall z.P'\{z/y\} \in \meaningof{E\{z/b\}}\}, \and \\ \meaningof{a!E} = \{ P \in \pi | P \equiv Q | x!\langle P' \rangle, x \in \meaningof{a} P' \in \meaningof{E}\} }
\end{mathpar}

\begin{mathpar}
 \inferrule* [lab=nominal] {} {\meaningof{\quotep{E}} = \{ \quotep{P} \in \quotep{\pi} | P \in \meaningof{E} \}, \and \meaningof{\quotep{P}} = \{ \quotep{Q} \in \quotep{\pi} | P \equiv Q \} \and \\ \meaningof{@\quotep{E}} = \{ P \in \pi | P \equiv @x, x \in \meaningof{E} \}}
\end{mathpar}

\begin{eqnarray*}
  \\
  \meaningof{-} : TS \to ST
\end{eqnarray*}

\begin{eqnarray*}
  \\
  L : TS \to ST
\end{eqnarray*}

\begin{eqnarray*}
  \\
  P \models E \iff P \in \meaningof{E}
\end{eqnarray*}

\begin{eqnarray*}
  P \approx_{L} Q \iff \forall E \in L. P \models E \iff Q \models E
\end{eqnarray*}

\begin{eqnarray*}
  P \approx_{K} Q
\end{eqnarray*}

\begin{eqnarray*}
  P \approx Q
\end{eqnarray*}

$\approx_{K} = \approx = \approx_{L}$

\subsubsection{Contextual duality}

Note that contexts extend the quotation operation to a family of
operations from processes to names. Given a context, $M$, we can
define a \emph{nominal context}, $\quotep{M}$ by $\quotep{M}[P] :=
\quotep{M[P]}$. To foreshadow what is to come we observe that these
operations enjoy a duality with processes very much like the duality
between vectors and maps from vectors to scalars.

Further, because the calculus is essentially higher-order, we have a
correspondence between contexts and processes. More specifically,
given a name $x$ and a context $M$ we can construct $M^{*}_{x}$ such
that 

\begin{mathpar}
  M^{*}_{x} | \lift{x}{P} \red M[P]
\end{mathpar}

namely,

\begin{mathpar}
  M^{*}_{x} := x?(u).M[\dropn{u}]
\end{mathpar}

The dependence of $M^{*}_{x}$ on a name makes it an abstraction, 

\begin{mathpar}
  M^{*} := (x)x?(u).M[\dropn{u}]
\end{mathpar}

\subsection{Additional notation}

It will sometimes be convenient to denote the process a name
quotes. We already have the notation $x = \quotep{P}$, but it will be
convenient to introduce an alternate notation, $\procn{x}$, when we
want to emphasize the connection to the use of the name. Note that, by
virtue of name equivalence, $\quotep{\procn{x}} \nameeq x$; so, the
notation is consistent with previous definitions.

Further, because names have structure it is possible to effect
substitutions on the basis of that structure. This means we need to
upgrade our notation for substitutions, which we accomplish by
adapting comprehension notation. Thus,

\begin{mathpar}
  P\{ y / x : x \in S \}
\end{mathpar}

is interpreted to mean the process derived from P by replacing (in a
capture-avoiding manner) each occurrence of $x$ in $S$ by $y$. For example,

\begin{mathpar}
  P\{ \quotep{\procn{x}|\procn{x}} / x : x \in \freenames{P} \}
\end{mathpar}

will replace each (occurrence) of a free name $x$ in $P$ by
$\quotep{\procn{x}|\procn{x}}$.

Also, we will avail ourselves of the notation $x^{L}$ and $x^{R}$ to
denote injections of a name into disjoint copies of the name
space. There are numerous ways to accomplish this. One example can be
found in \cite{MeredithR05}. This notation overloads to vectors of
names: $\vec{x}^{\pi} := (x_{i}^{\pi} \; : \; 0 \leq i < |\vec{x}| )$ where $\pi \in \{L,R\}$.

We also use $P^{\Box} := P|\Box$.

In \cite{MeredithR05} an interpretation of the new operator is
given. It turns out that there are several possible interpretations
all enjoying the requisite algebraic properties of the operator (see
\cite{milner91polyadicpi}). We will therefore make liberal use of
$(\nu\; \vec{x})P$.

% subsection the_syntax_and_semantics_of_the_notation_system (end)   

\input{qm2pi.qmops} 

\input{qm2pi.sterngerlach} 

\input{qm2pi.metric} 

% section concurrent_process_calculi (end)

%\input{qm2pi.proofsketch}

% section proof sketch (end)

%\input{qm2pi.slviaknots} 

% section spatial logic via knots (end)

\input{qm2pi.conclusion}

% section conclusion (end)

%\input{qm2pi.dtcodes} 

% section wiring algorithm (end)

\input{qm2pi.ack} 

% section acknowledgments (end)

\newpage


\bibliographystyle{plain}   
\bibliography{../../biblios/main.bib}

\input{qm2pi.rhodetails}

\end{document}

 

\documentclass[12pt]{llncs}
%\documentclass{jktr}

\usepackage[pdftex]{hyperref}                   
\usepackage {listings}
\usepackage {mathpartir}
\usepackage{bcprules}
%\usepackage{listings}
                       
\usepackage{graphicx} 
%\usepackage[margins=2.5cm,nohead,nofoot]{geometry}
%\usepackage{geometry}
\usepackage{amsfonts}
\usepackage{amstext}
\usepackage{latexsym}
\usepackage{amssymb}
\usepackage{color}


%\include{myPreamble}
\include{qm2pi.local} 

%\ifpdf
%\usepackage[pdftex]{graphicx}
%\else
%\usepackage{graphicx}
%\fi

 % \ifpdf
%  \usepackage{pdfsync}
%  \if


%\title{Brief Article}
%\author{David F. Snyder}
%\author{L.G. Meredith}

%\address{Dept. of Math., Texas State University--San Marcos, San Marcos, TX 78666}
       
\pagestyle{empty}


\begin{document}

\lstset{language=[Objective]Caml,frame=shadowbox}

\input{qm2pi.front}

% section front matter (end)

\input{qm2pi.intro} 
 
% section introduction (end)

% \input{qm2pi.knotations} 

% section notation (end)

\input{qm2pi.process.calculi} 

% section concurrent_process_calculi_and_spatial_logics_ (end)
    
%\input{qm2pi.knots2pi} 

%\input{qm2pi.trefoil} 

%\input{qm2pi.mainthm} 

% subsection basic_interpretation (end)

%\input{qm2pi.rho.presentation} 
\subsection{The syntax and semantics of the notation system}\label{sub:the_syntax_and_semantics_of_the_notation_system} % (fold)

We now summarize a technical presentation of the calculus that
embodies our theory of dynamics. The typical presentation of such a
calculus follows the style of giving generators and relations on
them. The grammar, below, describing term constructors, freely
generates the set of processes, $\Proc$. This set is then quotiented
by a relation known as structural congruence and it is over this set
that the notion of dynamics is expressed. This presentation is
essentially that of \cite{MeredithR05} with the addition of
polyadicity and summation. For readability we have relegated some of
the technical subtleties to an appendix.

\subsubsection{Process grammar}\label{subsub:process_grammar}

\begin{mathpar}
  \inferrule* [lab=synchronization] {} {{M} \bc \pzero \;|\; x?F \;|\; x!C }
  \and
  \inferrule* [lab=abstraction] {} {{F} \bc (x)P}
  \and
  \inferrule* [lab=concretion] {} {{C} \bc \langle Q \rangle}
  \and
  \inferrule* [lab=process] {} {{P,Q} \bc M \;| \;P|Q \;|\; @{x}}
  \and
  \inferrule* [lab=name] {} {{x} \bc \quotep{P}}
\end{mathpar} 

Note that $\vec{x}$ (resp. $\vec{P}$) denotes a vector of names
(resp. processes) of length $|\vec{x}|$ (resp. $|\vec{P}|$). We adopt
the following useful abbreviations.

\begin{mathpar}
   x?(\vec{y}).P := x.(\vec{y})P \and  x\clift{\vec{P}} := x.\clift{\vec{P}}
   \and x!(y) := \lift{x}{\dropn{y}}
   \and \Pi_{i=0}^{n-1}P_i := P_0 | \ldots | P_{n-1}
\end{mathpar}

\subsubsection{Structural congruence}

\paragraph{Free and bound names and alpha-equivalence.} At the
core of structural equivalence is alpha-equivalence which identifies
process that are the same up to a change of variable. Formally, we
recognize the distinction between free and bound names. The free names
of a process, $\freenames{P}$, may be calculated recursively as
follows:

\begin{mathpar}
\freenames{\pzero} := \emptyset
  \and \\
  \freenames{x?(y).P} := \{ x \} \cup (\freenames{P} \setminus \{ y \})
  \and 
  \freenames{x!\langle P \rangle} := \{ x \} \cup \{ P \} 
  \and \\
  \freenames{P|Q} := \freenames{P} \cup \freenames{Q}
  \and \\
  \freenames{@{x}} := \{ x \}
\end{mathpar}

$\pi$
$\quotep{\pi}$

$\freenames{-} : \pi \to \mathcal{P}(\quotep{\pi})$

\begin{eqnarray*}
  \freenames{\pzero} & := & \emptyset \\
  \freenames{x?(y).P} & := & \{ x \} \cup (\freenames{P} \setminus \{ y \}) \\
  \freenames{x!\langle P \rangle} & := & \{ x \} \cup \{ P \} \\
  \freenames{P|Q} & := & \freenames{P} \cup \freenames{Q} \\
  \freenames{\dropn{x}} & := & \{ x \}
\end{eqnarray*}

The bound names of a process, $\boundnames{P}$, are those names occurring in $P$
that are not free. For example, in $x?(y).0$, the name $x$ is free, while $y$ is bound.

\begin{mathpar}
  \inferrule* [lab=monoidal-laws] {} { P|Q \equiv Q|P \and P|0 \equiv P \and P|(Q|R) \equiv (P|Q)|R }
\end{mathpar}

\begin{mathpar}
  \inferrule* [lab=alpha-equivalence] {} { (x)P \equiv (y)P\{y/x\} \and y \not\in \freenames{P} }
\end{mathpar}

\begin{definition}
Then two processes, $P,Q$, are alpha-equivalent if $P = Q\{\vec{y}/\vec{x}\}$ for
some $\vec{x} \in \boundnames{Q},\vec{y} \in \boundnames{P}$, where $Q\{\vec{y}/\vec{x}\}$
denotes the capture-avoiding substitution of $\vec{y}$ for $\vec{x}$ in $Q$.
\end{definition}

\begin{definition}
  The {\em structural congruence} \cite{SangiorgiWalker} , $\equiv$,
  between processes is the least congruence containing
  alpha-equivalence, satisfying the abelian monoid laws
  (associativity, commutativity and $\pzero$ as identity) for parallel
  composition $|$ and for summation $+$.
\end{definition}

\subsection{Name equivalence}

We take name equivalence, written $\nameeq$, to be the smallest
equivalence relation generated by the following rules.

\begin{mathpar}
\inferrule*[lab=Quote-drop]
{ }
{ \quotep{@{x}} \nameeq x }

\inferrule*[lab=Struct-equiv]
{ P \scong Q }
{ \quotep{P} \nameeq \quotep{Q} }
\end{mathpar}

The astute reader will have noticed that the mutual recursion of names
and processes imposes a mutual recursion on alpha-equivalence and
structural equivalence via name-equivalence. Fortunately, all of this
works out pleasantly and we may calculate in the natural way, free of
concern. The reader interested in the details is referred to the
appendix \ref{appendix:rho_details}.

\subsection{Substitution}

We use $\Proc$ for the set of processes, $\QProc$ for the set of
names, and $\id{\{}\vec{y} / \vec{x} \id{\}}$ to denote partial maps,
$s : \QProc \rightarrow \QProc$. A map, $s$ lifts, uniquely, to a map
on process terms, $\widehat{s} : \Proc \rightarrow \Proc$ by the
following equations.

\begin{mathpar}
  (0) \psubstp{Q}{P} := 0 \\
  (R \juxtap S) \psubstp{Q}{P}
  :=    
  (R)\psubstp{Q}{P} \juxtap (S) \psubstp{Q}{P} \\
  (x?(y).R) \psubstp{Q}{P}    
  :=    
  (x)\substp{Q}{P} (z)\concat( (R \psubstn{z}{y}) \psubstp{Q}{P} ) \\
  (\lift{x}{R}) \psubstp{Q}{P}  
  :=
  \lift{(x)\substp{Q}{P}}{ R \psubstp{Q}{P} } \\
%   (\dropn{x})  \psubstp{Q}{P}       
%   := 
%   \left\{ 
%     \begin{array}{ccc} 
%       \dropn{\quotep{Q}} & & x \nameeq \quotep{P} \\
%       \dropn{x} & & otherwise \\
%     \end{array}
%   \right. 
  (\dropn{x})  \psubstp{Q}{P}       
  := 
  \left\{ 
    \begin{array}{ccc} 
      Q & & x \nameeq \quotep{P} \\
      \dropn{x} & & otherwise \\
    \end{array}
  \right.
\end{mathpar}
 

where

\begin{eqnarray}
  (x)\id{\{} \lpquote Q \rpquote / \lpquote P \rpquote \id{\}}            = 
  \left\{ 
    \begin{array}{ccc}
      \lpquote Q \rpquote & & x \nameeq \lpquote P \rpquote \\
      x & & otherwise \\
    \end{array}
  \right. \nonumber
\end{eqnarray}

and $z$ is chosen distinct from $\quotep{P}$, $\quotep{Q}$, the free
names in $Q$, and all the names in $R$. Our $\alpha$-equivalence will
be built in the standard way from this substitution.

\begin{remark}\label{rem:no_self_referential_names}
  One consequence of these definitions is that $\forall P. \quotep{P}
  \not\in \freenames{P}$.
\end{remark}

\subsection{ Dynamic quote: an example }

Anticipating something of what's to come, consider applying the
substitution, $\widehat{\id{\{}u / z \id{\}}}$, to the following pair
of processes, $\lift{w}{y!(z)}$ and $w[ \lpquote y!(z) \rpquote ]$.

\begin{eqnarray}
	\lift{w}{y!(z)}\widehat{\id{\{}u / z \id{\}}}
		& = &
		\lift{w}{y!(u)} \nonumber\\
	w[ \lpquote y!(z) \rpquote ] \widehat{ \id{\{}u / z \id{\}} }
		& = &
		w[ \lpquote y!(z) \rpquote ] \nonumber
\end{eqnarray}

Because the body of the process between quotes is impervious to
substitution, we get radically different answers. In fact, by
examining the first process in an input context,
e.g. $x?(z).\lift{w}{y!(z)}$, we see that the process under the lift
operator may be shaped by prefixed inputs binding a name inside it. In
this sense, the lift operator will be seen as a way to dynamically
construct processes before reifying them as names.

Finally equipped with these standard features we can present the
dynamics of the calculus.

\subsubsection{Operational semantics} 

Finally, we introduce the computational dynamics. What marks these
algebras as distinct from other more traditionally studied algebraic
structures, e.g. vector spaces or polynomial rings, is the manner in
which dynamics is captured. In traditional structures, dynamics is typically
expressed through morphisms between such structures, as in linear maps
between vector spaces or morphisms between rings. In algebras
associated with the semantics of computation, the dynamics is
expressed as part of the algebraic structure itself, through a
reduction reduction relation typically denoted by $\red$. Below, we
give a recursive presentation of this relation for the calculus used
in the encoding.

$\red \subseteq \pi \times \pi$
$\red : \pi \to \mathcal{P}(\pi)$

\begin{mathpar}
  \inferrule* [lab=Comm] { \textsf{match}( x_{src}, x_{trgt} ) } { x_{trgt}?(y)P \; | \; x_{src}!\langle {Q} \rangle \red P\{\quotep{Q}/y}\} }
  \and \\
  \inferrule* [lab=Par] {{P} \red {P}'} {{{P} | {Q}} \red {{P}' | {Q}}}
  \and
  \inferrule* [lab=Equiv]{{{P} \scong {P}'} \andalso {{P}' \red {Q}'} \andalso {{Q}' \scong {Q}}}{{P} \red {Q}}
\end{mathpar}

\begin{eqnarray*}
  match_{\equiv} (\quotep{P},\quotep{Q}) & := & P \equiv Q \\
  match_{\dagger}(\quotep{P},\quotep{Q}) & := & \forall R. P|Q \red^{*} R => R \red^{*} 0 \\
  match_{K}(\quotep{P},\quotep{Q}) & := & K \mbox{ for some context } K
\end{eqnarray*}

$u?(x)P | u!\langle Q \rangle \red P\{\quotep{Q}/x\}$

%We write $\wred$ for $\red^*$, and $P\red$ if $\exists Q $ such that $ P \red Q$.
We write $P\red$ if $\exists Q $ such that $ P \red Q$ and $P\not\red$, otherwise.

\section{Replication}

As mentioned before, it is known that replication (and hence
recursion) can be implemented in a higher-order process algebra
\cite{SangiorgiWalker}. As our first example of calculation with the
machinery thus far presented we give the construction explicitly in
the {\rhoc}.

\begin{eqnarray}
	D_{x} & := & \prefix{x}{y}{(\binpar{\outputp{x}{y}}{@{y}})} \nonumber\\
	\bangp_{x}{P} & := & \binpar{{x}!\langle{\binpar{D_{x}}{P}}\rangle}{D_{x}} \nonumber
\end{eqnarray}

\begin{eqnarray}
	\bangp_{x}{P} & & \nonumber\\
	=
	& {x}!\langle{(\prefix{x}{y}{(\outputp{x}{y} | @{y})) | P}}\rangle 
	      | \prefix{x}{y}{(\outputp{x}{y} | @{y})} & \nonumber\\
	\red
	& (\outputp{x}{y} | @{y})\substn{\quotep{(\prefix{x}{y}{(@{y} | \outputp{x}{y})) | P}}}{y} & \nonumber\\
	=
	& \outputp{x}{\quotep{(\prefix{x}{y}{(\outputp{x}{y} | @{y})) | P}}}
	  | {(\prefix{x}{y}{(\outputp{x}{y} | @{y})) | P}} & \nonumber\\
	\red
	& \ldots & \nonumber\\
	\red^*
	& P | P | \ldots & \nonumber
\end{eqnarray}

Of course, this encoding, as an implementation, runs away, unfolding
$\bangp{P}$ eagerly. A lazier and more implementable replication
operator, restricted to input-guarded processes, may be obtained as follows.

\begin{eqnarray}
\bangp{\prefix{u}{v}{P}} 
	:= 
	\binpar{\lift{x}{\prefix{u}{v}{(\binpar{D(x)}{P})}}}{D(x)} \nonumber
\end{eqnarray}

\begin{remark}
  Note that the lazier definition still does not deal with summation
  or mixed summation (i.e. sums over input and output). The reader is
  invited to construct definitions of replication that deal with these
  features. 

  Further, the definitions are parameterized in a name, $x$. Can you,
  gentle reader, make a definition that eliminates this parameter and
  guarantees no accidental interaction between the replication
  machinery and the process being replicated -- i.e. no accidental
  sharing of names used by the process to get its work done and the
  name(s) used by the replication to effect copying. This latter
  revision of the definition of replication is crucial to obtaining
  the expected identity $!!P \sim !P$.
\end{remark}

\begin{remark}\label{rem:paradoxical_combinator}
  The reader familiar with the lambda calculus will have noticed the
  similarity between $D$ and the paradoxical combinator.

  [Ed. note: the existence of this seems to suggest we have to be more
  restrictive on the set of processes and names we admit if we are to
  support no-cloning.]
\end{remark}

\subsubsection{Bisimulation}

The computational dynamics gives rise to another kind of equivalence,
the equivalence of computational behavior. As previously mentioned
this is typically captured \emph{via} some form of bisimulation.

% The notion we use in this paper is weak barbed bisimulation
% \cite{milner91polyadicpi}.

The notion we use in this paper is derived from weak barbed
bisimulation \cite{milner91polyadicpi}. 

\begin{definition}
An \emph{observation relation}, $\downarrow_{\mathcal N}$, over a set
of names, $\mathcal N$, is the smallest relation satisfying the rules
below.

\infrule[Out-barb]{y \in {\mathcal N}, \; x \nameeq y}
		  {\outputp{x}{v} \downarrow_{\mathcal N} x}
\infrule[Par-barb]{\mbox{$P\downarrow_{\mathcal N} x$ or $Q\downarrow_{\mathcal N} x$}}
		  {\binpar{P}{Q} \downarrow_{\mathcal N} x}

We write $P \Downarrow_{\mathcal N} x$ if there is $Q$ such that 
$P \wred Q$ and $Q \downarrow_{\mathcal N} x$.
\end{definition}

\begin{definition}
%\label{def.bbisim}
An  ${\mathcal N}$-\emph{barbed bisimulation} over a set of names, ${\mathcal N}$, is a symmetric binary relation 
${\mathcal S}_{\mathcal N}$ between agents such that $P\rel{S}_{\mathcal N}Q$ implies:
\begin{enumerate}
\item If $P \red P'$ then $Q \wred Q'$ and $P'\rel{S}_{\mathcal N} Q'$.
\item If $P\downarrow_{\mathcal N} x$, then $Q\Downarrow_{\mathcal N} x$.
\end{enumerate}
$P$ is ${\mathcal N}$-barbed bisimilar to $Q$, written
$P \wbbisim_{\mathcal N} Q$, if $P \rel{S}_{\mathcal N} Q$ for some ${\mathcal N}$-barbed bisimulation ${\mathcal S}_{\mathcal N}$.
\end{definition}

$\mathcal{R} \subseteq \pi \times \pi$

$P \mathcal{R} Q => \forall P'. P \red P' \Rightarrow \exists Q'. Q \red Q', P' \mathcal{R} Q'$

$P \vdash x \Rightarrow Q \vdash x$

\begin{mathpar}
  \inferrule*[lab=Out-barb]{x \nameeq y}{{y}!\langle{Q}\rangle \vdash x}
  \and
  \inferrule*[lab=Par-barb]{\mbox{$P\vdash x$ or $Q\vdash x$}}{\binpar{P}{Q} \vdash x}
\end{mathpar}

\subsubsection{Contexts}

One of the principle advantages of computational calculi like the
$\pi$-calculus is a well-defined notion of context,
contextual-equivalence and a correlation between
contextual-equivalence and notions of bisimulation. The notion of
context allows the decomposition of a process into (sub-)process and
its syntactic environment, its context. Thus, a context may be
thought of as a process with a ``hole'' (written $\Box$) in it. The
application of a context $M$ to a process $P$, written $M[P]$, is
tantamount to filling the hole in $M$ with $P$. In this paper we do
not need the full weight of this theory, but do make use of the notion
of context in the proof the main theorem. 

\begin{mathpar}
  \inferrule* [lab=summation] {} {{M_{M},M_{N}} \bc \Box \;|\; x.M_{A} \;|\; M_{M}+M_{N}}
  \and
  \inferrule* [lab=agent] {} {{M_{A}} \bc (\vec{x})M_{P} \;| \; \clift{P_0,\ldots,M_{P},\ldots,P_N}}
  \and \\
  \inferrule* [lab=process] {} {{M_{P}} \bc M_{N} \;| \;P|M_{P} }
\end{mathpar} 

\begin{mathpar}
  \inferrule* [lab=sychronization] {} {M_{N} \bc \Box \;|\; x?M_{F} \;|\; x!M_{C}}
  \and
  \inferrule* [lab=abstraction] {} {{M_{F}} \bc (x)M_{P} }
  \and
  \inferrule* [lab=concretion] {} {{M_{C}} \bc \langle M_{P} \rangle }
  \and \\
  \inferrule* [lab=process] {} {{M_{P}} \bc M_{N} \;| \;P|M_{P} }
\end{mathpar}

\begin{definition}[contextual application] Given a context $M$, and
  process $P$, we define the \emph{contextual application}, $M[P] :=
  M\{P/\Box\}$. That is, the contextual application of M to P is the
  substitution of $P$ for $\Box$ in $M$.
\end{definition}

$\meaningof{-} : L \to \mathcal{P}(\pi)$

\begin{mathpar}
  \inferrule* [lab=collection] {} {\meaningof{true} = \pi, \and \meaningof{~E} = \pi \setminus \meaningof{E}, \and \meaningof{E_{1} \& E_{2}} = \meaningof{E_{1}} \cap \meaningof{E_{2}}}
\end{mathpar}

\begin{mathpar}
  \inferrule* [lab=structure] {} {\meaningof{0} = \{ P \in \pi | P \equiv 0 \}, \and \\ \meaningof{E_1 | E_2} = \{ P \in \pi | P \equiv P_{1} | P_{2}, P_{1} \in \meaningof{E_{1}}, P_{2} \in \meaningof{E_2}\} }
\end{mathpar}

\begin{mathpar}
 \inferrule* [lab=behavior] {} {\meaningof{\langle a?b \rangle E} = \{ P \in \pi | P \equiv Q | u?(y)P', \\ \and \\\\ \and \\ \;\;\; u \in \meaningof{a}, \forall z.P'\{z/y\} \in \meaningof{E\{z/b\}}\}, \and \\ \meaningof{a!E} = \{ P \in \pi | P \equiv Q | x!\langle P' \rangle, x \in \meaningof{a} P' \in \meaningof{E}\} }
\end{mathpar}

\begin{mathpar}
 \inferrule* [lab=nominal] {} {\meaningof{\quotep{E}} = \{ \quotep{P} \in \quotep{\pi} | P \in \meaningof{E} \}, \and \meaningof{\quotep{P}} = \{ \quotep{Q} \in \quotep{\pi} | P \equiv Q \} \and \\ \meaningof{@\quotep{E}} = \{ P \in \pi | P \equiv @x, x \in \meaningof{E} \}}
\end{mathpar}

\begin{eqnarray*}
  \\
  \meaningof{-} : TS \to ST
\end{eqnarray*}

\begin{eqnarray*}
  \\
  L : TS \to ST
\end{eqnarray*}

\begin{eqnarray*}
  \\
  P \models E \iff P \in \meaningof{E}
\end{eqnarray*}

\begin{eqnarray*}
  P \approx_{L} Q \iff \forall E \in L. P \models E \iff Q \models E
\end{eqnarray*}

\begin{eqnarray*}
  P \approx_{K} Q
\end{eqnarray*}

\begin{eqnarray*}
  P \approx Q
\end{eqnarray*}

$\approx_{K} = \approx = \approx_{L}$

\subsubsection{Contextual duality}

Note that contexts extend the quotation operation to a family of
operations from processes to names. Given a context, $M$, we can
define a \emph{nominal context}, $\quotep{M}$ by $\quotep{M}[P] :=
\quotep{M[P]}$. To foreshadow what is to come we observe that these
operations enjoy a duality with processes very much like the duality
between vectors and maps from vectors to scalars.

Further, because the calculus is essentially higher-order, we have a
correspondence between contexts and processes. More specifically,
given a name $x$ and a context $M$ we can construct $M^{*}_{x}$ such
that 

\begin{mathpar}
  M^{*}_{x} | \lift{x}{P} \red M[P]
\end{mathpar}

namely,

\begin{mathpar}
  M^{*}_{x} := x?(u).M[\dropn{u}]
\end{mathpar}

The dependence of $M^{*}_{x}$ on a name makes it an abstraction, 

\begin{mathpar}
  M^{*} := (x)x?(u).M[\dropn{u}]
\end{mathpar}

\subsection{Additional notation}

It will sometimes be convenient to denote the process a name
quotes. We already have the notation $x = \quotep{P}$, but it will be
convenient to introduce an alternate notation, $\procn{x}$, when we
want to emphasize the connection to the use of the name. Note that, by
virtue of name equivalence, $\quotep{\procn{x}} \nameeq x$; so, the
notation is consistent with previous definitions.

Further, because names have structure it is possible to effect
substitutions on the basis of that structure. This means we need to
upgrade our notation for substitutions, which we accomplish by
adapting comprehension notation. Thus,

\begin{mathpar}
  P\{ y / x : x \in S \}
\end{mathpar}

is interpreted to mean the process derived from P by replacing (in a
capture-avoiding manner) each occurrence of $x$ in $S$ by $y$. For example,

\begin{mathpar}
  P\{ \quotep{\procn{x}|\procn{x}} / x : x \in \freenames{P} \}
\end{mathpar}

will replace each (occurrence) of a free name $x$ in $P$ by
$\quotep{\procn{x}|\procn{x}}$.

Also, we will avail ourselves of the notation $x^{L}$ and $x^{R}$ to
denote injections of a name into disjoint copies of the name
space. There are numerous ways to accomplish this. One example can be
found in \cite{MeredithR05}. This notation overloads to vectors of
names: $\vec{x}^{\pi} := (x_{i}^{\pi} \; : \; 0 \leq i < |\vec{x}| )$ where $\pi \in \{L,R\}$.

We also use $P^{\Box} := P|\Box$.

In \cite{MeredithR05} an interpretation of the new operator is
given. It turns out that there are several possible interpretations
all enjoying the requisite algebraic properties of the operator (see
\cite{milner91polyadicpi}). We will therefore make liberal use of
$(\nu\; \vec{x})P$.

% subsection the_syntax_and_semantics_of_the_notation_system (end)   

\input{qm2pi.qmops} 

\input{qm2pi.sterngerlach} 

\input{qm2pi.metric} 

% section concurrent_process_calculi (end)

%\input{qm2pi.proofsketch}

% section proof sketch (end)

%\input{qm2pi.slviaknots} 

% section spatial logic via knots (end)

\input{qm2pi.conclusion}

% section conclusion (end)

%\input{qm2pi.dtcodes} 

% section wiring algorithm (end)

\input{qm2pi.ack} 

% section acknowledgments (end)

\newpage


\bibliographystyle{plain}   
\bibliography{../../biblios/main.bib}

\input{qm2pi.rhodetails}

\end{document}

 

% section concurrent_process_calculi (end)

%\documentclass[12pt]{llncs}
%\documentclass{jktr}

\usepackage[pdftex]{hyperref}                   
\usepackage {listings}
\usepackage {mathpartir}
\usepackage{bcprules}
%\usepackage{listings}
                       
\usepackage{graphicx} 
%\usepackage[margins=2.5cm,nohead,nofoot]{geometry}
%\usepackage{geometry}
\usepackage{amsfonts}
\usepackage{amstext}
\usepackage{latexsym}
\usepackage{amssymb}
\usepackage{color}


%\include{myPreamble}
\include{qm2pi.local} 

%\ifpdf
%\usepackage[pdftex]{graphicx}
%\else
%\usepackage{graphicx}
%\fi

 % \ifpdf
%  \usepackage{pdfsync}
%  \if


%\title{Brief Article}
%\author{David F. Snyder}
%\author{L.G. Meredith}

%\address{Dept. of Math., Texas State University--San Marcos, San Marcos, TX 78666}
       
\pagestyle{empty}


\begin{document}

\lstset{language=[Objective]Caml,frame=shadowbox}

\input{qm2pi.front}

% section front matter (end)

\input{qm2pi.intro} 
 
% section introduction (end)

% \input{qm2pi.knotations} 

% section notation (end)

\input{qm2pi.process.calculi} 

% section concurrent_process_calculi_and_spatial_logics_ (end)
    
%\input{qm2pi.knots2pi} 

%\input{qm2pi.trefoil} 

%\input{qm2pi.mainthm} 

% subsection basic_interpretation (end)

%\input{qm2pi.rho.presentation} 
\subsection{The syntax and semantics of the notation system}\label{sub:the_syntax_and_semantics_of_the_notation_system} % (fold)

We now summarize a technical presentation of the calculus that
embodies our theory of dynamics. The typical presentation of such a
calculus follows the style of giving generators and relations on
them. The grammar, below, describing term constructors, freely
generates the set of processes, $\Proc$. This set is then quotiented
by a relation known as structural congruence and it is over this set
that the notion of dynamics is expressed. This presentation is
essentially that of \cite{MeredithR05} with the addition of
polyadicity and summation. For readability we have relegated some of
the technical subtleties to an appendix.

\subsubsection{Process grammar}\label{subsub:process_grammar}

\begin{mathpar}
  \inferrule* [lab=synchronization] {} {{M} \bc \pzero \;|\; x?F \;|\; x!C }
  \and
  \inferrule* [lab=abstraction] {} {{F} \bc (x)P}
  \and
  \inferrule* [lab=concretion] {} {{C} \bc \langle Q \rangle}
  \and
  \inferrule* [lab=process] {} {{P,Q} \bc M \;| \;P|Q \;|\; @{x}}
  \and
  \inferrule* [lab=name] {} {{x} \bc \quotep{P}}
\end{mathpar} 

Note that $\vec{x}$ (resp. $\vec{P}$) denotes a vector of names
(resp. processes) of length $|\vec{x}|$ (resp. $|\vec{P}|$). We adopt
the following useful abbreviations.

\begin{mathpar}
   x?(\vec{y}).P := x.(\vec{y})P \and  x\clift{\vec{P}} := x.\clift{\vec{P}}
   \and x!(y) := \lift{x}{\dropn{y}}
   \and \Pi_{i=0}^{n-1}P_i := P_0 | \ldots | P_{n-1}
\end{mathpar}

\subsubsection{Structural congruence}

\paragraph{Free and bound names and alpha-equivalence.} At the
core of structural equivalence is alpha-equivalence which identifies
process that are the same up to a change of variable. Formally, we
recognize the distinction between free and bound names. The free names
of a process, $\freenames{P}$, may be calculated recursively as
follows:

\begin{mathpar}
\freenames{\pzero} := \emptyset
  \and \\
  \freenames{x?(y).P} := \{ x \} \cup (\freenames{P} \setminus \{ y \})
  \and 
  \freenames{x!\langle P \rangle} := \{ x \} \cup \{ P \} 
  \and \\
  \freenames{P|Q} := \freenames{P} \cup \freenames{Q}
  \and \\
  \freenames{@{x}} := \{ x \}
\end{mathpar}

$\pi$
$\quotep{\pi}$

$\freenames{-} : \pi \to \mathcal{P}(\quotep{\pi})$

\begin{eqnarray*}
  \freenames{\pzero} & := & \emptyset \\
  \freenames{x?(y).P} & := & \{ x \} \cup (\freenames{P} \setminus \{ y \}) \\
  \freenames{x!\langle P \rangle} & := & \{ x \} \cup \{ P \} \\
  \freenames{P|Q} & := & \freenames{P} \cup \freenames{Q} \\
  \freenames{\dropn{x}} & := & \{ x \}
\end{eqnarray*}

The bound names of a process, $\boundnames{P}$, are those names occurring in $P$
that are not free. For example, in $x?(y).0$, the name $x$ is free, while $y$ is bound.

\begin{mathpar}
  \inferrule* [lab=monoidal-laws] {} { P|Q \equiv Q|P \and P|0 \equiv P \and P|(Q|R) \equiv (P|Q)|R }
\end{mathpar}

\begin{mathpar}
  \inferrule* [lab=alpha-equivalence] {} { (x)P \equiv (y)P\{y/x\} \and y \not\in \freenames{P} }
\end{mathpar}

\begin{definition}
Then two processes, $P,Q$, are alpha-equivalent if $P = Q\{\vec{y}/\vec{x}\}$ for
some $\vec{x} \in \boundnames{Q},\vec{y} \in \boundnames{P}$, where $Q\{\vec{y}/\vec{x}\}$
denotes the capture-avoiding substitution of $\vec{y}$ for $\vec{x}$ in $Q$.
\end{definition}

\begin{definition}
  The {\em structural congruence} \cite{SangiorgiWalker} , $\equiv$,
  between processes is the least congruence containing
  alpha-equivalence, satisfying the abelian monoid laws
  (associativity, commutativity and $\pzero$ as identity) for parallel
  composition $|$ and for summation $+$.
\end{definition}

\subsection{Name equivalence}

We take name equivalence, written $\nameeq$, to be the smallest
equivalence relation generated by the following rules.

\begin{mathpar}
\inferrule*[lab=Quote-drop]
{ }
{ \quotep{@{x}} \nameeq x }

\inferrule*[lab=Struct-equiv]
{ P \scong Q }
{ \quotep{P} \nameeq \quotep{Q} }
\end{mathpar}

The astute reader will have noticed that the mutual recursion of names
and processes imposes a mutual recursion on alpha-equivalence and
structural equivalence via name-equivalence. Fortunately, all of this
works out pleasantly and we may calculate in the natural way, free of
concern. The reader interested in the details is referred to the
appendix \ref{appendix:rho_details}.

\subsection{Substitution}

We use $\Proc$ for the set of processes, $\QProc$ for the set of
names, and $\id{\{}\vec{y} / \vec{x} \id{\}}$ to denote partial maps,
$s : \QProc \rightarrow \QProc$. A map, $s$ lifts, uniquely, to a map
on process terms, $\widehat{s} : \Proc \rightarrow \Proc$ by the
following equations.

\begin{mathpar}
  (0) \psubstp{Q}{P} := 0 \\
  (R \juxtap S) \psubstp{Q}{P}
  :=    
  (R)\psubstp{Q}{P} \juxtap (S) \psubstp{Q}{P} \\
  (x?(y).R) \psubstp{Q}{P}    
  :=    
  (x)\substp{Q}{P} (z)\concat( (R \psubstn{z}{y}) \psubstp{Q}{P} ) \\
  (\lift{x}{R}) \psubstp{Q}{P}  
  :=
  \lift{(x)\substp{Q}{P}}{ R \psubstp{Q}{P} } \\
%   (\dropn{x})  \psubstp{Q}{P}       
%   := 
%   \left\{ 
%     \begin{array}{ccc} 
%       \dropn{\quotep{Q}} & & x \nameeq \quotep{P} \\
%       \dropn{x} & & otherwise \\
%     \end{array}
%   \right. 
  (\dropn{x})  \psubstp{Q}{P}       
  := 
  \left\{ 
    \begin{array}{ccc} 
      Q & & x \nameeq \quotep{P} \\
      \dropn{x} & & otherwise \\
    \end{array}
  \right.
\end{mathpar}
 

where

\begin{eqnarray}
  (x)\id{\{} \lpquote Q \rpquote / \lpquote P \rpquote \id{\}}            = 
  \left\{ 
    \begin{array}{ccc}
      \lpquote Q \rpquote & & x \nameeq \lpquote P \rpquote \\
      x & & otherwise \\
    \end{array}
  \right. \nonumber
\end{eqnarray}

and $z$ is chosen distinct from $\quotep{P}$, $\quotep{Q}$, the free
names in $Q$, and all the names in $R$. Our $\alpha$-equivalence will
be built in the standard way from this substitution.

\begin{remark}\label{rem:no_self_referential_names}
  One consequence of these definitions is that $\forall P. \quotep{P}
  \not\in \freenames{P}$.
\end{remark}

\subsection{ Dynamic quote: an example }

Anticipating something of what's to come, consider applying the
substitution, $\widehat{\id{\{}u / z \id{\}}}$, to the following pair
of processes, $\lift{w}{y!(z)}$ and $w[ \lpquote y!(z) \rpquote ]$.

\begin{eqnarray}
	\lift{w}{y!(z)}\widehat{\id{\{}u / z \id{\}}}
		& = &
		\lift{w}{y!(u)} \nonumber\\
	w[ \lpquote y!(z) \rpquote ] \widehat{ \id{\{}u / z \id{\}} }
		& = &
		w[ \lpquote y!(z) \rpquote ] \nonumber
\end{eqnarray}

Because the body of the process between quotes is impervious to
substitution, we get radically different answers. In fact, by
examining the first process in an input context,
e.g. $x?(z).\lift{w}{y!(z)}$, we see that the process under the lift
operator may be shaped by prefixed inputs binding a name inside it. In
this sense, the lift operator will be seen as a way to dynamically
construct processes before reifying them as names.

Finally equipped with these standard features we can present the
dynamics of the calculus.

\subsubsection{Operational semantics} 

Finally, we introduce the computational dynamics. What marks these
algebras as distinct from other more traditionally studied algebraic
structures, e.g. vector spaces or polynomial rings, is the manner in
which dynamics is captured. In traditional structures, dynamics is typically
expressed through morphisms between such structures, as in linear maps
between vector spaces or morphisms between rings. In algebras
associated with the semantics of computation, the dynamics is
expressed as part of the algebraic structure itself, through a
reduction reduction relation typically denoted by $\red$. Below, we
give a recursive presentation of this relation for the calculus used
in the encoding.

$\red \subseteq \pi \times \pi$
$\red : \pi \to \mathcal{P}(\pi)$

\begin{mathpar}
  \inferrule* [lab=Comm] { \textsf{match}( x_{src}, x_{trgt} ) } { x_{trgt}?(y)P \; | \; x_{src}!\langle {Q} \rangle \red P\{\quotep{Q}/y}\} }
  \and \\
  \inferrule* [lab=Par] {{P} \red {P}'} {{{P} | {Q}} \red {{P}' | {Q}}}
  \and
  \inferrule* [lab=Equiv]{{{P} \scong {P}'} \andalso {{P}' \red {Q}'} \andalso {{Q}' \scong {Q}}}{{P} \red {Q}}
\end{mathpar}

\begin{eqnarray*}
  match_{\equiv} (\quotep{P},\quotep{Q}) & := & P \equiv Q \\
  match_{\dagger}(\quotep{P},\quotep{Q}) & := & \forall R. P|Q \red^{*} R => R \red^{*} 0 \\
  match_{K}(\quotep{P},\quotep{Q}) & := & K \mbox{ for some context } K
\end{eqnarray*}

$u?(x)P | u!\langle Q \rangle \red P\{\quotep{Q}/x\}$

%We write $\wred$ for $\red^*$, and $P\red$ if $\exists Q $ such that $ P \red Q$.
We write $P\red$ if $\exists Q $ such that $ P \red Q$ and $P\not\red$, otherwise.

\section{Replication}

As mentioned before, it is known that replication (and hence
recursion) can be implemented in a higher-order process algebra
\cite{SangiorgiWalker}. As our first example of calculation with the
machinery thus far presented we give the construction explicitly in
the {\rhoc}.

\begin{eqnarray}
	D_{x} & := & \prefix{x}{y}{(\binpar{\outputp{x}{y}}{@{y}})} \nonumber\\
	\bangp_{x}{P} & := & \binpar{{x}!\langle{\binpar{D_{x}}{P}}\rangle}{D_{x}} \nonumber
\end{eqnarray}

\begin{eqnarray}
	\bangp_{x}{P} & & \nonumber\\
	=
	& {x}!\langle{(\prefix{x}{y}{(\outputp{x}{y} | @{y})) | P}}\rangle 
	      | \prefix{x}{y}{(\outputp{x}{y} | @{y})} & \nonumber\\
	\red
	& (\outputp{x}{y} | @{y})\substn{\quotep{(\prefix{x}{y}{(@{y} | \outputp{x}{y})) | P}}}{y} & \nonumber\\
	=
	& \outputp{x}{\quotep{(\prefix{x}{y}{(\outputp{x}{y} | @{y})) | P}}}
	  | {(\prefix{x}{y}{(\outputp{x}{y} | @{y})) | P}} & \nonumber\\
	\red
	& \ldots & \nonumber\\
	\red^*
	& P | P | \ldots & \nonumber
\end{eqnarray}

Of course, this encoding, as an implementation, runs away, unfolding
$\bangp{P}$ eagerly. A lazier and more implementable replication
operator, restricted to input-guarded processes, may be obtained as follows.

\begin{eqnarray}
\bangp{\prefix{u}{v}{P}} 
	:= 
	\binpar{\lift{x}{\prefix{u}{v}{(\binpar{D(x)}{P})}}}{D(x)} \nonumber
\end{eqnarray}

\begin{remark}
  Note that the lazier definition still does not deal with summation
  or mixed summation (i.e. sums over input and output). The reader is
  invited to construct definitions of replication that deal with these
  features. 

  Further, the definitions are parameterized in a name, $x$. Can you,
  gentle reader, make a definition that eliminates this parameter and
  guarantees no accidental interaction between the replication
  machinery and the process being replicated -- i.e. no accidental
  sharing of names used by the process to get its work done and the
  name(s) used by the replication to effect copying. This latter
  revision of the definition of replication is crucial to obtaining
  the expected identity $!!P \sim !P$.
\end{remark}

\begin{remark}\label{rem:paradoxical_combinator}
  The reader familiar with the lambda calculus will have noticed the
  similarity between $D$ and the paradoxical combinator.

  [Ed. note: the existence of this seems to suggest we have to be more
  restrictive on the set of processes and names we admit if we are to
  support no-cloning.]
\end{remark}

\subsubsection{Bisimulation}

The computational dynamics gives rise to another kind of equivalence,
the equivalence of computational behavior. As previously mentioned
this is typically captured \emph{via} some form of bisimulation.

% The notion we use in this paper is weak barbed bisimulation
% \cite{milner91polyadicpi}.

The notion we use in this paper is derived from weak barbed
bisimulation \cite{milner91polyadicpi}. 

\begin{definition}
An \emph{observation relation}, $\downarrow_{\mathcal N}$, over a set
of names, $\mathcal N$, is the smallest relation satisfying the rules
below.

\infrule[Out-barb]{y \in {\mathcal N}, \; x \nameeq y}
		  {\outputp{x}{v} \downarrow_{\mathcal N} x}
\infrule[Par-barb]{\mbox{$P\downarrow_{\mathcal N} x$ or $Q\downarrow_{\mathcal N} x$}}
		  {\binpar{P}{Q} \downarrow_{\mathcal N} x}

We write $P \Downarrow_{\mathcal N} x$ if there is $Q$ such that 
$P \wred Q$ and $Q \downarrow_{\mathcal N} x$.
\end{definition}

\begin{definition}
%\label{def.bbisim}
An  ${\mathcal N}$-\emph{barbed bisimulation} over a set of names, ${\mathcal N}$, is a symmetric binary relation 
${\mathcal S}_{\mathcal N}$ between agents such that $P\rel{S}_{\mathcal N}Q$ implies:
\begin{enumerate}
\item If $P \red P'$ then $Q \wred Q'$ and $P'\rel{S}_{\mathcal N} Q'$.
\item If $P\downarrow_{\mathcal N} x$, then $Q\Downarrow_{\mathcal N} x$.
\end{enumerate}
$P$ is ${\mathcal N}$-barbed bisimilar to $Q$, written
$P \wbbisim_{\mathcal N} Q$, if $P \rel{S}_{\mathcal N} Q$ for some ${\mathcal N}$-barbed bisimulation ${\mathcal S}_{\mathcal N}$.
\end{definition}

$\mathcal{R} \subseteq \pi \times \pi$

$P \mathcal{R} Q => \forall P'. P \red P' \Rightarrow \exists Q'. Q \red Q', P' \mathcal{R} Q'$

$P \vdash x \Rightarrow Q \vdash x$

\begin{mathpar}
  \inferrule*[lab=Out-barb]{x \nameeq y}{{y}!\langle{Q}\rangle \vdash x}
  \and
  \inferrule*[lab=Par-barb]{\mbox{$P\vdash x$ or $Q\vdash x$}}{\binpar{P}{Q} \vdash x}
\end{mathpar}

\subsubsection{Contexts}

One of the principle advantages of computational calculi like the
$\pi$-calculus is a well-defined notion of context,
contextual-equivalence and a correlation between
contextual-equivalence and notions of bisimulation. The notion of
context allows the decomposition of a process into (sub-)process and
its syntactic environment, its context. Thus, a context may be
thought of as a process with a ``hole'' (written $\Box$) in it. The
application of a context $M$ to a process $P$, written $M[P]$, is
tantamount to filling the hole in $M$ with $P$. In this paper we do
not need the full weight of this theory, but do make use of the notion
of context in the proof the main theorem. 

\begin{mathpar}
  \inferrule* [lab=summation] {} {{M_{M},M_{N}} \bc \Box \;|\; x.M_{A} \;|\; M_{M}+M_{N}}
  \and
  \inferrule* [lab=agent] {} {{M_{A}} \bc (\vec{x})M_{P} \;| \; \clift{P_0,\ldots,M_{P},\ldots,P_N}}
  \and \\
  \inferrule* [lab=process] {} {{M_{P}} \bc M_{N} \;| \;P|M_{P} }
\end{mathpar} 

\begin{mathpar}
  \inferrule* [lab=sychronization] {} {M_{N} \bc \Box \;|\; x?M_{F} \;|\; x!M_{C}}
  \and
  \inferrule* [lab=abstraction] {} {{M_{F}} \bc (x)M_{P} }
  \and
  \inferrule* [lab=concretion] {} {{M_{C}} \bc \langle M_{P} \rangle }
  \and \\
  \inferrule* [lab=process] {} {{M_{P}} \bc M_{N} \;| \;P|M_{P} }
\end{mathpar}

\begin{definition}[contextual application] Given a context $M$, and
  process $P$, we define the \emph{contextual application}, $M[P] :=
  M\{P/\Box\}$. That is, the contextual application of M to P is the
  substitution of $P$ for $\Box$ in $M$.
\end{definition}

$\meaningof{-} : L \to \mathcal{P}(\pi)$

\begin{mathpar}
  \inferrule* [lab=collection] {} {\meaningof{true} = \pi, \and \meaningof{~E} = \pi \setminus \meaningof{E}, \and \meaningof{E_{1} \& E_{2}} = \meaningof{E_{1}} \cap \meaningof{E_{2}}}
\end{mathpar}

\begin{mathpar}
  \inferrule* [lab=structure] {} {\meaningof{0} = \{ P \in \pi | P \equiv 0 \}, \and \\ \meaningof{E_1 | E_2} = \{ P \in \pi | P \equiv P_{1} | P_{2}, P_{1} \in \meaningof{E_{1}}, P_{2} \in \meaningof{E_2}\} }
\end{mathpar}

\begin{mathpar}
 \inferrule* [lab=behavior] {} {\meaningof{\langle a?b \rangle E} = \{ P \in \pi | P \equiv Q | u?(y)P', \\ \and \\\\ \and \\ \;\;\; u \in \meaningof{a}, \forall z.P'\{z/y\} \in \meaningof{E\{z/b\}}\}, \and \\ \meaningof{a!E} = \{ P \in \pi | P \equiv Q | x!\langle P' \rangle, x \in \meaningof{a} P' \in \meaningof{E}\} }
\end{mathpar}

\begin{mathpar}
 \inferrule* [lab=nominal] {} {\meaningof{\quotep{E}} = \{ \quotep{P} \in \quotep{\pi} | P \in \meaningof{E} \}, \and \meaningof{\quotep{P}} = \{ \quotep{Q} \in \quotep{\pi} | P \equiv Q \} \and \\ \meaningof{@\quotep{E}} = \{ P \in \pi | P \equiv @x, x \in \meaningof{E} \}}
\end{mathpar}

\begin{eqnarray*}
  \\
  \meaningof{-} : TS \to ST
\end{eqnarray*}

\begin{eqnarray*}
  \\
  L : TS \to ST
\end{eqnarray*}

\begin{eqnarray*}
  \\
  P \models E \iff P \in \meaningof{E}
\end{eqnarray*}

\begin{eqnarray*}
  P \approx_{L} Q \iff \forall E \in L. P \models E \iff Q \models E
\end{eqnarray*}

\begin{eqnarray*}
  P \approx_{K} Q
\end{eqnarray*}

\begin{eqnarray*}
  P \approx Q
\end{eqnarray*}

$\approx_{K} = \approx = \approx_{L}$

\subsubsection{Contextual duality}

Note that contexts extend the quotation operation to a family of
operations from processes to names. Given a context, $M$, we can
define a \emph{nominal context}, $\quotep{M}$ by $\quotep{M}[P] :=
\quotep{M[P]}$. To foreshadow what is to come we observe that these
operations enjoy a duality with processes very much like the duality
between vectors and maps from vectors to scalars.

Further, because the calculus is essentially higher-order, we have a
correspondence between contexts and processes. More specifically,
given a name $x$ and a context $M$ we can construct $M^{*}_{x}$ such
that 

\begin{mathpar}
  M^{*}_{x} | \lift{x}{P} \red M[P]
\end{mathpar}

namely,

\begin{mathpar}
  M^{*}_{x} := x?(u).M[\dropn{u}]
\end{mathpar}

The dependence of $M^{*}_{x}$ on a name makes it an abstraction, 

\begin{mathpar}
  M^{*} := (x)x?(u).M[\dropn{u}]
\end{mathpar}

\subsection{Additional notation}

It will sometimes be convenient to denote the process a name
quotes. We already have the notation $x = \quotep{P}$, but it will be
convenient to introduce an alternate notation, $\procn{x}$, when we
want to emphasize the connection to the use of the name. Note that, by
virtue of name equivalence, $\quotep{\procn{x}} \nameeq x$; so, the
notation is consistent with previous definitions.

Further, because names have structure it is possible to effect
substitutions on the basis of that structure. This means we need to
upgrade our notation for substitutions, which we accomplish by
adapting comprehension notation. Thus,

\begin{mathpar}
  P\{ y / x : x \in S \}
\end{mathpar}

is interpreted to mean the process derived from P by replacing (in a
capture-avoiding manner) each occurrence of $x$ in $S$ by $y$. For example,

\begin{mathpar}
  P\{ \quotep{\procn{x}|\procn{x}} / x : x \in \freenames{P} \}
\end{mathpar}

will replace each (occurrence) of a free name $x$ in $P$ by
$\quotep{\procn{x}|\procn{x}}$.

Also, we will avail ourselves of the notation $x^{L}$ and $x^{R}$ to
denote injections of a name into disjoint copies of the name
space. There are numerous ways to accomplish this. One example can be
found in \cite{MeredithR05}. This notation overloads to vectors of
names: $\vec{x}^{\pi} := (x_{i}^{\pi} \; : \; 0 \leq i < |\vec{x}| )$ where $\pi \in \{L,R\}$.

We also use $P^{\Box} := P|\Box$.

In \cite{MeredithR05} an interpretation of the new operator is
given. It turns out that there are several possible interpretations
all enjoying the requisite algebraic properties of the operator (see
\cite{milner91polyadicpi}). We will therefore make liberal use of
$(\nu\; \vec{x})P$.

% subsection the_syntax_and_semantics_of_the_notation_system (end)   

\input{qm2pi.qmops} 

\input{qm2pi.sterngerlach} 

\input{qm2pi.metric} 

% section concurrent_process_calculi (end)

%\input{qm2pi.proofsketch}

% section proof sketch (end)

%\input{qm2pi.slviaknots} 

% section spatial logic via knots (end)

\input{qm2pi.conclusion}

% section conclusion (end)

%\input{qm2pi.dtcodes} 

% section wiring algorithm (end)

\input{qm2pi.ack} 

% section acknowledgments (end)

\newpage


\bibliographystyle{plain}   
\bibliography{../../biblios/main.bib}

\input{qm2pi.rhodetails}

\end{document}



% section proof sketch (end)

%\section{Unlikely characters: spatial logic for
  knots}\label{sub:characteristic_formulae} % (fold)

Associated to the mobile process calculi are a family of logics known
as the Hennessy-Milner logics. These logics typically enjoy a
semantics interpreting formulae as sets of processes that when
factored through the encoding outlined above allows an identification
of classes of knots with logical formulae. In the context of this
encoding the sub-family known as the spatial logics \cite{CairesC03}
\cite{CairesC04} \cite{Caires04} are of particular interest providing
several important features for expressing and reasoning about
properties (i.e. classes) of knots. We hint here at how this may be done.

%\begin{description}
%\item [structural connectives] 
\subsubsection{Structural connectives} The spatial logics enjoy
structural connectives corresponding, at the logical level, to the
parallel composition ($P | Q$) and new name ($(\nu \; x)P$)
connectives for processes. As illustrated in the examples below, these
connectives are extremely expressive given the shape of our encoding.
%\item [decideable satisfaction]

\subsubsection{Decideable satisfaction}
In \cite{Caires04} the satisfaction relation is shown to be decideable
for a rich class of processes. It further turns out that the image of
the our encoding is a proper subset of that class. This result
provides the basis for an algorithm by which to search for knots
enjoying a given property.
%\item [characteristic formulae]

\subsubsection{Characteristic formulae}
In the same paper \cite{Caires04} , Caires presents a means of calculating
characteristic formulae, selecting equivalence classes of processes
up to a pre--specified depth limit on the support set of names. Composed with our
encoding, this characteristic formula can be used to select
characteristic formulae for knots.
%\end{description}

\subsubsection{Spatial logic formulae}

The grammar below (segmented for comprehension) summarizes the syntax
of spatial logic formulae. We employ illustrative examples in the
sequel to provide an intuitive understanding of their meaning
referring the reader to \cite{Caires04} for a more detailed explication
of the semantics.

\begin{mathpar}
  \inferrule* [lab=boolean] {} {{A,B} \bc T \;|\; \neg A \;|\; A \wedge B \;|\; \eta = \eta'}
  \and
  \inferrule* [lab=spatial] {} {|\; \pzero \;|\; A | B \;|\; x \text{\textregistered} A \;|\; \forall x . A \;|\;  H x . A}
  \and
  \inferrule* [lab=behavioral] {} {|\; \alpha . A}
  \and 
  \inferrule* [lab=recursion] {} {|\; X(\vec{u}) \;|\; \mu X(\vec{u}) . A}
  \and
  \inferrule* [lab=action] {} {\alpha \bc \langle x?(\vec{y}) \rangle \;|\; \langle x!(\vec{y}) \rangle \;|\; \langle \tau \rangle}
  \and 
  \inferrule* [lab=name] {} {\eta \bc x \;|\; \tau}
\end{mathpar} 

% subsection characteristic_formulae (end)   	 

\subsection{Example formulae}\label{sub:example_formulae_} % (fold)

\subsubsection{Crossing as formula.}
% 
% \begin{align*}
%   \frac{d}{dx} \sin x &= \cos x 
%   & \frac{d}{dx} e^x &= e^x \\
%   \frac{d}{dx} \cos x &= - \sin x 
%   & \frac{d}{dx} \log x &= \frac{1}{x} \\
% \end{align*} 

\begin{align*}
 \mu C(x_{0},x_{1},y_{0},y_{1},u).&(\langle x_{0}?(z) \rangle(\langle u! \rangle\langle y_{1}!z \rangle C(x_{0},x_{1},y_{0},y_{1},u)) & \\
  & \wedge \langle y_{1}?(z) \rangle (\langle u! \rangle \langle x_{0}!z \rangle C(x_{0},x_{1},y_{0},y_{1},u)) & \\
  & \wedge \langle x_{1}?(z) \rangle (\langle u? \rangle \langle y_{0}!z \rangle C(x_{0},x_{1},y_{0},y_{1},u)) & \\
  & \wedge \langle y_{0}?(z) \rangle (\langle u? \rangle \langle x_{1}!z \rangle C(x_{0},x_{1},y_{0},y_{1},u))) &
\end{align*}

The lexicographical similarity between the shape of this formulae and
the shape of definition of the process representing a crossing reveals
the intuitive meaning of this formulae. It describes the capabilities
of a process that has the right to represent a crossing. For example
it picks out processes that may perform an input on the port $x_0$ in
its initial menu of capabilities. What differentiates the formula
from the process, however, is that the crossing process is the
smallest candidate to satisfy the formula. Infinitely many other
processes -- with internal behavior hidden behind this interface, so
to speak -- also satisfy this formula. Even this simple formula,
then, can be seen to open a new view onto knots, providing a
computational interpretation of \emph{virtual} knots.

Note that this formula is derived by hand. A similar formula can be
derived by employing Caires' calculation of characteristic formula
\cite{Caires04} to the process representing a crossing. In light of
this discussion, we let
$\meaningof{C}_{\phi}(x0,x1,y0,y1,u)$ denote a formula specifying the
dynamics we wish to capture of a crossing. To guarantee we preserve
the shape of the interface and minimal semantics we demand that
$\meaningof{C}_{\phi}(x0,x1,y0,y1,u) \Rightarrow
\textbf{C}(x0,x1,y0,y1,u)$ where $\textbf{C}(x0,x1,y0,y1,u)$ denotes
the formula above.
                            
\subsubsection{Crossing number constraints.}
The moral content of the context lemma (Lemma \ref{context}) is that the notion of
``locality'' in the Reidemeister moves is effectively captured by the
parallel composition operator of the process calculus. This intuition
extends through the logic. Given a formula,
$\meaningof{C}_{\phi}(x0,x1,y0,y1,u)$, we can use the structural
connectives to specify constraints on crossing numbers, such as at
least $n$ crossings, or exactly $n$ crossings.
\begin{mathpar}
  \inferrule* [lab=at-least-n] {} { K^{\geq n}_{\phi}(\vec{xs},\vec{ys}) := \Pi_{i=0}^{n-1} Hu . \meaningof{C}_{\phi}(xs_i,ys_i,u) | T }
  \and 
  \inferrule* [lab=exactly-n] {} { K^{= n}_{\phi}(\vec{xs},\vec{ys}) := \Pi_{i=0}^{n-1} Hu . \meaningof{C}_{\phi}(xs_i,ys_i,u) | \neg (\forall x_0,y_0,x_1,y_1,u . \meaningof{C}_{\phi}(x_0,y_0,x_1,y_1,u) | T) }
\end{mathpar}

To round out this section, recall that the encoding of an $n$-crossing
knot decomposes into a parallel composition of $n$ \emph{copies} of a
crossing process together with a wiring harness. To specify different
knot classes with the same crossing number amounts to specifying
logical constraints on the wiring harness. In the interest of space,
we defer examples to a forthcoming paper. Suffice it to say that both
the conditions ``alternating knot'' and ``contains the tangle
corresponding to 5/3'' are expressible. For example, it is possible to
calculate the characteristic formula of a process corresponding to the
tangle 5/3 and conjoin it into the classifying formula via the
composition connective of the logic.

Finally, we wish to observe that it is entirely within reason to
contemplate a more domain-specific version of spatial logic tailored
to the shape of processes in the image of the encoding. Such a
domain-specific logic would have a better claim to the title formal
language of knot properties.

% subsection example_formulae_ (end)

% section knots_as_processes (end) 

% section spatial logic via knots (end)

\section{Conclusions and future work}

\paragraph{Testing physical space}
You, gentle reader, may wonder why of all the theorems to be proved
given this set up we pick the one above. In some sense it's hardly
central to quantum mechanics. We see it as central in the sense that
it firmly establishes a notion of physical space arising from a notion
of the equivalence of behavior. Relating bisimulation to a metric is a
big step forward, but one is faced with interpreting the relationship
of that metric space to something more physical. Quantum mechanical
notions of ``physical'' space are still far from intuitive, but by
relating this idea of distance as testing to calculations that predict
physical circumstances we are making a not insignificant step forward
toward an understanding of the physical space we inhabit as
essentially dynamic.

\paragraph{Effectivity and simulation}
One of the observations we have yet to make is that the entire program
spelled out here is effective. We have built various interpreters for
the reflective calculus at work in this interpretation. In principle,
then, we can simulate quantum mechanics on a computer. The place where
the simulation may lose fidelity is the infinitely branching summation
for the annihilator.

In this connection i also want to point out that the evaluation style
calculation of the inner product puts the non-determinism of the
summation right at the heart of measurement. This suggests that
Milner's original reduction-based formulation of the dynamics of his
calculi in terms of sums was not just notationally suggestive of a
notion of measure-and-continue but captured some significant part of
the physics.

\paragraph{Quantum continuations}
In light of this last observation i want to point out that the
predominant account of quantum mechanics is missing a key aspect of a
truly compositional story of the physical situation. In a real lab,
when a measurement is made the observation can be made to feed into
another device that then makes another measurement conditioned on the
results of the first. This means that after the superposition was
collapsed the entire experimental set up remained in
superposition. While QM offers a means of writing this down it doesn't
quite line up well with the well-trodden formulation of computation
and continuation that we see so succinctly expressed in Milner's
calculi. This suggests that there might be advantages to this account
of dynamics waiting to be explored.

\paragraph{Quantum logic}
In this connection, we also note that by virtue of having the
Hennessy-Milner construction, we can pull the construction through the
interpretation of QM. This gives us a natural candidate for a quantum
logic that enjoys an extremely tight connection with it's domain of
interpretation, making the construction much less ad hoc (rather it is
the image of functor!).

\paragraph{Quantum probabiity}
i have questions about the basis of the interpretation of inner
product as probability amplitude. In particular, using which
axiomatization of probability theory does the notion of probability
amplitude earn the right to be so dubbed? In other words, where is the
proof that the operation for calculating a probability amplitude (and
then squaring) satisfies the axioms of what it means to calculate a
probability? Even if such a proof exists (i have yet to find it in the
literature), i wonder if it might not be possible to turn things on
their heads. Can we view the calculation of the probability amplitude
as an axiomatization of probability? If so, then the definition we
give for calculating probability amplitude may provide the basis for
an \emph{effective} theory of probability.

\paragraph{Quantum vs ``biological'' information}
Finally, i want to conclude with a more philosophical observation. At
a recent workshop in which QM was a predominant topic i noticed
something about quantum information. The speaker was giving a riveting
discussion of axiomatic QM and showing how properties of ``no
cloning'' and ``no deleting'' emerged as consequences of the
axiomatization. Theorems of this form are necessary to give us a sense
of confidence that our axioms characterize the physical theory. What
struck me, though, was that if quantum information is neither erasable
nor replicable it is markedly different from \emph{life}. Two of the
things we know about life is that

\begin{itemize}
  \item it ends;
  \item to gain some measure of persistence, to transcend it's
    finitude it is imminently copyable.
\end{itemize}

Both of these qualities are summarized succinctly in the aphorism: all
flesh is grass. For me these two kinds of ``information'' -- call them
quantum and biological -- are end points on a spectrum of strategies
for persistence. At one end, we have those curious entities that enjoy
uniqueness and permanence; at the other, we have those who in the face
of a certain end and an uncertain present make a go of passing
something on. To me one of the more remarkable aspects of the latter
strategy is that in the presence of noise (and certain features of
copying) we get a kind of dynamism, a chance for improvement against a
given persistent condition.

% subsection other_calculi_other_bisimulations_and_geometry_as_behavior (end)




% section conclusion (end)

%\documentclass[12pt]{llncs}
%\documentclass{jktr}

\usepackage[pdftex]{hyperref}                   
\usepackage {listings}
\usepackage {mathpartir}
\usepackage{bcprules}
%\usepackage{listings}
                       
\usepackage{graphicx} 
%\usepackage[margins=2.5cm,nohead,nofoot]{geometry}
%\usepackage{geometry}
\usepackage{amsfonts}
\usepackage{amstext}
\usepackage{latexsym}
\usepackage{amssymb}
\usepackage{color}


%\include{myPreamble}
\include{qm2pi.local} 

%\ifpdf
%\usepackage[pdftex]{graphicx}
%\else
%\usepackage{graphicx}
%\fi

 % \ifpdf
%  \usepackage{pdfsync}
%  \if


%\title{Brief Article}
%\author{David F. Snyder}
%\author{L.G. Meredith}

%\address{Dept. of Math., Texas State University--San Marcos, San Marcos, TX 78666}
       
\pagestyle{empty}


\begin{document}

\lstset{language=[Objective]Caml,frame=shadowbox}

\input{qm2pi.front}

% section front matter (end)

\input{qm2pi.intro} 
 
% section introduction (end)

% \input{qm2pi.knotations} 

% section notation (end)

\input{qm2pi.process.calculi} 

% section concurrent_process_calculi_and_spatial_logics_ (end)
    
%\input{qm2pi.knots2pi} 

%\input{qm2pi.trefoil} 

%\input{qm2pi.mainthm} 

% subsection basic_interpretation (end)

%\input{qm2pi.rho.presentation} 
\subsection{The syntax and semantics of the notation system}\label{sub:the_syntax_and_semantics_of_the_notation_system} % (fold)

We now summarize a technical presentation of the calculus that
embodies our theory of dynamics. The typical presentation of such a
calculus follows the style of giving generators and relations on
them. The grammar, below, describing term constructors, freely
generates the set of processes, $\Proc$. This set is then quotiented
by a relation known as structural congruence and it is over this set
that the notion of dynamics is expressed. This presentation is
essentially that of \cite{MeredithR05} with the addition of
polyadicity and summation. For readability we have relegated some of
the technical subtleties to an appendix.

\subsubsection{Process grammar}\label{subsub:process_grammar}

\begin{mathpar}
  \inferrule* [lab=synchronization] {} {{M} \bc \pzero \;|\; x?F \;|\; x!C }
  \and
  \inferrule* [lab=abstraction] {} {{F} \bc (x)P}
  \and
  \inferrule* [lab=concretion] {} {{C} \bc \langle Q \rangle}
  \and
  \inferrule* [lab=process] {} {{P,Q} \bc M \;| \;P|Q \;|\; @{x}}
  \and
  \inferrule* [lab=name] {} {{x} \bc \quotep{P}}
\end{mathpar} 

Note that $\vec{x}$ (resp. $\vec{P}$) denotes a vector of names
(resp. processes) of length $|\vec{x}|$ (resp. $|\vec{P}|$). We adopt
the following useful abbreviations.

\begin{mathpar}
   x?(\vec{y}).P := x.(\vec{y})P \and  x\clift{\vec{P}} := x.\clift{\vec{P}}
   \and x!(y) := \lift{x}{\dropn{y}}
   \and \Pi_{i=0}^{n-1}P_i := P_0 | \ldots | P_{n-1}
\end{mathpar}

\subsubsection{Structural congruence}

\paragraph{Free and bound names and alpha-equivalence.} At the
core of structural equivalence is alpha-equivalence which identifies
process that are the same up to a change of variable. Formally, we
recognize the distinction between free and bound names. The free names
of a process, $\freenames{P}$, may be calculated recursively as
follows:

\begin{mathpar}
\freenames{\pzero} := \emptyset
  \and \\
  \freenames{x?(y).P} := \{ x \} \cup (\freenames{P} \setminus \{ y \})
  \and 
  \freenames{x!\langle P \rangle} := \{ x \} \cup \{ P \} 
  \and \\
  \freenames{P|Q} := \freenames{P} \cup \freenames{Q}
  \and \\
  \freenames{@{x}} := \{ x \}
\end{mathpar}

$\pi$
$\quotep{\pi}$

$\freenames{-} : \pi \to \mathcal{P}(\quotep{\pi})$

\begin{eqnarray*}
  \freenames{\pzero} & := & \emptyset \\
  \freenames{x?(y).P} & := & \{ x \} \cup (\freenames{P} \setminus \{ y \}) \\
  \freenames{x!\langle P \rangle} & := & \{ x \} \cup \{ P \} \\
  \freenames{P|Q} & := & \freenames{P} \cup \freenames{Q} \\
  \freenames{\dropn{x}} & := & \{ x \}
\end{eqnarray*}

The bound names of a process, $\boundnames{P}$, are those names occurring in $P$
that are not free. For example, in $x?(y).0$, the name $x$ is free, while $y$ is bound.

\begin{mathpar}
  \inferrule* [lab=monoidal-laws] {} { P|Q \equiv Q|P \and P|0 \equiv P \and P|(Q|R) \equiv (P|Q)|R }
\end{mathpar}

\begin{mathpar}
  \inferrule* [lab=alpha-equivalence] {} { (x)P \equiv (y)P\{y/x\} \and y \not\in \freenames{P} }
\end{mathpar}

\begin{definition}
Then two processes, $P,Q$, are alpha-equivalent if $P = Q\{\vec{y}/\vec{x}\}$ for
some $\vec{x} \in \boundnames{Q},\vec{y} \in \boundnames{P}$, where $Q\{\vec{y}/\vec{x}\}$
denotes the capture-avoiding substitution of $\vec{y}$ for $\vec{x}$ in $Q$.
\end{definition}

\begin{definition}
  The {\em structural congruence} \cite{SangiorgiWalker} , $\equiv$,
  between processes is the least congruence containing
  alpha-equivalence, satisfying the abelian monoid laws
  (associativity, commutativity and $\pzero$ as identity) for parallel
  composition $|$ and for summation $+$.
\end{definition}

\subsection{Name equivalence}

We take name equivalence, written $\nameeq$, to be the smallest
equivalence relation generated by the following rules.

\begin{mathpar}
\inferrule*[lab=Quote-drop]
{ }
{ \quotep{@{x}} \nameeq x }

\inferrule*[lab=Struct-equiv]
{ P \scong Q }
{ \quotep{P} \nameeq \quotep{Q} }
\end{mathpar}

The astute reader will have noticed that the mutual recursion of names
and processes imposes a mutual recursion on alpha-equivalence and
structural equivalence via name-equivalence. Fortunately, all of this
works out pleasantly and we may calculate in the natural way, free of
concern. The reader interested in the details is referred to the
appendix \ref{appendix:rho_details}.

\subsection{Substitution}

We use $\Proc$ for the set of processes, $\QProc$ for the set of
names, and $\id{\{}\vec{y} / \vec{x} \id{\}}$ to denote partial maps,
$s : \QProc \rightarrow \QProc$. A map, $s$ lifts, uniquely, to a map
on process terms, $\widehat{s} : \Proc \rightarrow \Proc$ by the
following equations.

\begin{mathpar}
  (0) \psubstp{Q}{P} := 0 \\
  (R \juxtap S) \psubstp{Q}{P}
  :=    
  (R)\psubstp{Q}{P} \juxtap (S) \psubstp{Q}{P} \\
  (x?(y).R) \psubstp{Q}{P}    
  :=    
  (x)\substp{Q}{P} (z)\concat( (R \psubstn{z}{y}) \psubstp{Q}{P} ) \\
  (\lift{x}{R}) \psubstp{Q}{P}  
  :=
  \lift{(x)\substp{Q}{P}}{ R \psubstp{Q}{P} } \\
%   (\dropn{x})  \psubstp{Q}{P}       
%   := 
%   \left\{ 
%     \begin{array}{ccc} 
%       \dropn{\quotep{Q}} & & x \nameeq \quotep{P} \\
%       \dropn{x} & & otherwise \\
%     \end{array}
%   \right. 
  (\dropn{x})  \psubstp{Q}{P}       
  := 
  \left\{ 
    \begin{array}{ccc} 
      Q & & x \nameeq \quotep{P} \\
      \dropn{x} & & otherwise \\
    \end{array}
  \right.
\end{mathpar}
 

where

\begin{eqnarray}
  (x)\id{\{} \lpquote Q \rpquote / \lpquote P \rpquote \id{\}}            = 
  \left\{ 
    \begin{array}{ccc}
      \lpquote Q \rpquote & & x \nameeq \lpquote P \rpquote \\
      x & & otherwise \\
    \end{array}
  \right. \nonumber
\end{eqnarray}

and $z$ is chosen distinct from $\quotep{P}$, $\quotep{Q}$, the free
names in $Q$, and all the names in $R$. Our $\alpha$-equivalence will
be built in the standard way from this substitution.

\begin{remark}\label{rem:no_self_referential_names}
  One consequence of these definitions is that $\forall P. \quotep{P}
  \not\in \freenames{P}$.
\end{remark}

\subsection{ Dynamic quote: an example }

Anticipating something of what's to come, consider applying the
substitution, $\widehat{\id{\{}u / z \id{\}}}$, to the following pair
of processes, $\lift{w}{y!(z)}$ and $w[ \lpquote y!(z) \rpquote ]$.

\begin{eqnarray}
	\lift{w}{y!(z)}\widehat{\id{\{}u / z \id{\}}}
		& = &
		\lift{w}{y!(u)} \nonumber\\
	w[ \lpquote y!(z) \rpquote ] \widehat{ \id{\{}u / z \id{\}} }
		& = &
		w[ \lpquote y!(z) \rpquote ] \nonumber
\end{eqnarray}

Because the body of the process between quotes is impervious to
substitution, we get radically different answers. In fact, by
examining the first process in an input context,
e.g. $x?(z).\lift{w}{y!(z)}$, we see that the process under the lift
operator may be shaped by prefixed inputs binding a name inside it. In
this sense, the lift operator will be seen as a way to dynamically
construct processes before reifying them as names.

Finally equipped with these standard features we can present the
dynamics of the calculus.

\subsubsection{Operational semantics} 

Finally, we introduce the computational dynamics. What marks these
algebras as distinct from other more traditionally studied algebraic
structures, e.g. vector spaces or polynomial rings, is the manner in
which dynamics is captured. In traditional structures, dynamics is typically
expressed through morphisms between such structures, as in linear maps
between vector spaces or morphisms between rings. In algebras
associated with the semantics of computation, the dynamics is
expressed as part of the algebraic structure itself, through a
reduction reduction relation typically denoted by $\red$. Below, we
give a recursive presentation of this relation for the calculus used
in the encoding.

$\red \subseteq \pi \times \pi$
$\red : \pi \to \mathcal{P}(\pi)$

\begin{mathpar}
  \inferrule* [lab=Comm] { \textsf{match}( x_{src}, x_{trgt} ) } { x_{trgt}?(y)P \; | \; x_{src}!\langle {Q} \rangle \red P\{\quotep{Q}/y}\} }
  \and \\
  \inferrule* [lab=Par] {{P} \red {P}'} {{{P} | {Q}} \red {{P}' | {Q}}}
  \and
  \inferrule* [lab=Equiv]{{{P} \scong {P}'} \andalso {{P}' \red {Q}'} \andalso {{Q}' \scong {Q}}}{{P} \red {Q}}
\end{mathpar}

\begin{eqnarray*}
  match_{\equiv} (\quotep{P},\quotep{Q}) & := & P \equiv Q \\
  match_{\dagger}(\quotep{P},\quotep{Q}) & := & \forall R. P|Q \red^{*} R => R \red^{*} 0 \\
  match_{K}(\quotep{P},\quotep{Q}) & := & K \mbox{ for some context } K
\end{eqnarray*}

$u?(x)P | u!\langle Q \rangle \red P\{\quotep{Q}/x\}$

%We write $\wred$ for $\red^*$, and $P\red$ if $\exists Q $ such that $ P \red Q$.
We write $P\red$ if $\exists Q $ such that $ P \red Q$ and $P\not\red$, otherwise.

\section{Replication}

As mentioned before, it is known that replication (and hence
recursion) can be implemented in a higher-order process algebra
\cite{SangiorgiWalker}. As our first example of calculation with the
machinery thus far presented we give the construction explicitly in
the {\rhoc}.

\begin{eqnarray}
	D_{x} & := & \prefix{x}{y}{(\binpar{\outputp{x}{y}}{@{y}})} \nonumber\\
	\bangp_{x}{P} & := & \binpar{{x}!\langle{\binpar{D_{x}}{P}}\rangle}{D_{x}} \nonumber
\end{eqnarray}

\begin{eqnarray}
	\bangp_{x}{P} & & \nonumber\\
	=
	& {x}!\langle{(\prefix{x}{y}{(\outputp{x}{y} | @{y})) | P}}\rangle 
	      | \prefix{x}{y}{(\outputp{x}{y} | @{y})} & \nonumber\\
	\red
	& (\outputp{x}{y} | @{y})\substn{\quotep{(\prefix{x}{y}{(@{y} | \outputp{x}{y})) | P}}}{y} & \nonumber\\
	=
	& \outputp{x}{\quotep{(\prefix{x}{y}{(\outputp{x}{y} | @{y})) | P}}}
	  | {(\prefix{x}{y}{(\outputp{x}{y} | @{y})) | P}} & \nonumber\\
	\red
	& \ldots & \nonumber\\
	\red^*
	& P | P | \ldots & \nonumber
\end{eqnarray}

Of course, this encoding, as an implementation, runs away, unfolding
$\bangp{P}$ eagerly. A lazier and more implementable replication
operator, restricted to input-guarded processes, may be obtained as follows.

\begin{eqnarray}
\bangp{\prefix{u}{v}{P}} 
	:= 
	\binpar{\lift{x}{\prefix{u}{v}{(\binpar{D(x)}{P})}}}{D(x)} \nonumber
\end{eqnarray}

\begin{remark}
  Note that the lazier definition still does not deal with summation
  or mixed summation (i.e. sums over input and output). The reader is
  invited to construct definitions of replication that deal with these
  features. 

  Further, the definitions are parameterized in a name, $x$. Can you,
  gentle reader, make a definition that eliminates this parameter and
  guarantees no accidental interaction between the replication
  machinery and the process being replicated -- i.e. no accidental
  sharing of names used by the process to get its work done and the
  name(s) used by the replication to effect copying. This latter
  revision of the definition of replication is crucial to obtaining
  the expected identity $!!P \sim !P$.
\end{remark}

\begin{remark}\label{rem:paradoxical_combinator}
  The reader familiar with the lambda calculus will have noticed the
  similarity between $D$ and the paradoxical combinator.

  [Ed. note: the existence of this seems to suggest we have to be more
  restrictive on the set of processes and names we admit if we are to
  support no-cloning.]
\end{remark}

\subsubsection{Bisimulation}

The computational dynamics gives rise to another kind of equivalence,
the equivalence of computational behavior. As previously mentioned
this is typically captured \emph{via} some form of bisimulation.

% The notion we use in this paper is weak barbed bisimulation
% \cite{milner91polyadicpi}.

The notion we use in this paper is derived from weak barbed
bisimulation \cite{milner91polyadicpi}. 

\begin{definition}
An \emph{observation relation}, $\downarrow_{\mathcal N}$, over a set
of names, $\mathcal N$, is the smallest relation satisfying the rules
below.

\infrule[Out-barb]{y \in {\mathcal N}, \; x \nameeq y}
		  {\outputp{x}{v} \downarrow_{\mathcal N} x}
\infrule[Par-barb]{\mbox{$P\downarrow_{\mathcal N} x$ or $Q\downarrow_{\mathcal N} x$}}
		  {\binpar{P}{Q} \downarrow_{\mathcal N} x}

We write $P \Downarrow_{\mathcal N} x$ if there is $Q$ such that 
$P \wred Q$ and $Q \downarrow_{\mathcal N} x$.
\end{definition}

\begin{definition}
%\label{def.bbisim}
An  ${\mathcal N}$-\emph{barbed bisimulation} over a set of names, ${\mathcal N}$, is a symmetric binary relation 
${\mathcal S}_{\mathcal N}$ between agents such that $P\rel{S}_{\mathcal N}Q$ implies:
\begin{enumerate}
\item If $P \red P'$ then $Q \wred Q'$ and $P'\rel{S}_{\mathcal N} Q'$.
\item If $P\downarrow_{\mathcal N} x$, then $Q\Downarrow_{\mathcal N} x$.
\end{enumerate}
$P$ is ${\mathcal N}$-barbed bisimilar to $Q$, written
$P \wbbisim_{\mathcal N} Q$, if $P \rel{S}_{\mathcal N} Q$ for some ${\mathcal N}$-barbed bisimulation ${\mathcal S}_{\mathcal N}$.
\end{definition}

$\mathcal{R} \subseteq \pi \times \pi$

$P \mathcal{R} Q => \forall P'. P \red P' \Rightarrow \exists Q'. Q \red Q', P' \mathcal{R} Q'$

$P \vdash x \Rightarrow Q \vdash x$

\begin{mathpar}
  \inferrule*[lab=Out-barb]{x \nameeq y}{{y}!\langle{Q}\rangle \vdash x}
  \and
  \inferrule*[lab=Par-barb]{\mbox{$P\vdash x$ or $Q\vdash x$}}{\binpar{P}{Q} \vdash x}
\end{mathpar}

\subsubsection{Contexts}

One of the principle advantages of computational calculi like the
$\pi$-calculus is a well-defined notion of context,
contextual-equivalence and a correlation between
contextual-equivalence and notions of bisimulation. The notion of
context allows the decomposition of a process into (sub-)process and
its syntactic environment, its context. Thus, a context may be
thought of as a process with a ``hole'' (written $\Box$) in it. The
application of a context $M$ to a process $P$, written $M[P]$, is
tantamount to filling the hole in $M$ with $P$. In this paper we do
not need the full weight of this theory, but do make use of the notion
of context in the proof the main theorem. 

\begin{mathpar}
  \inferrule* [lab=summation] {} {{M_{M},M_{N}} \bc \Box \;|\; x.M_{A} \;|\; M_{M}+M_{N}}
  \and
  \inferrule* [lab=agent] {} {{M_{A}} \bc (\vec{x})M_{P} \;| \; \clift{P_0,\ldots,M_{P},\ldots,P_N}}
  \and \\
  \inferrule* [lab=process] {} {{M_{P}} \bc M_{N} \;| \;P|M_{P} }
\end{mathpar} 

\begin{mathpar}
  \inferrule* [lab=sychronization] {} {M_{N} \bc \Box \;|\; x?M_{F} \;|\; x!M_{C}}
  \and
  \inferrule* [lab=abstraction] {} {{M_{F}} \bc (x)M_{P} }
  \and
  \inferrule* [lab=concretion] {} {{M_{C}} \bc \langle M_{P} \rangle }
  \and \\
  \inferrule* [lab=process] {} {{M_{P}} \bc M_{N} \;| \;P|M_{P} }
\end{mathpar}

\begin{definition}[contextual application] Given a context $M$, and
  process $P$, we define the \emph{contextual application}, $M[P] :=
  M\{P/\Box\}$. That is, the contextual application of M to P is the
  substitution of $P$ for $\Box$ in $M$.
\end{definition}

$\meaningof{-} : L \to \mathcal{P}(\pi)$

\begin{mathpar}
  \inferrule* [lab=collection] {} {\meaningof{true} = \pi, \and \meaningof{~E} = \pi \setminus \meaningof{E}, \and \meaningof{E_{1} \& E_{2}} = \meaningof{E_{1}} \cap \meaningof{E_{2}}}
\end{mathpar}

\begin{mathpar}
  \inferrule* [lab=structure] {} {\meaningof{0} = \{ P \in \pi | P \equiv 0 \}, \and \\ \meaningof{E_1 | E_2} = \{ P \in \pi | P \equiv P_{1} | P_{2}, P_{1} \in \meaningof{E_{1}}, P_{2} \in \meaningof{E_2}\} }
\end{mathpar}

\begin{mathpar}
 \inferrule* [lab=behavior] {} {\meaningof{\langle a?b \rangle E} = \{ P \in \pi | P \equiv Q | u?(y)P', \\ \and \\\\ \and \\ \;\;\; u \in \meaningof{a}, \forall z.P'\{z/y\} \in \meaningof{E\{z/b\}}\}, \and \\ \meaningof{a!E} = \{ P \in \pi | P \equiv Q | x!\langle P' \rangle, x \in \meaningof{a} P' \in \meaningof{E}\} }
\end{mathpar}

\begin{mathpar}
 \inferrule* [lab=nominal] {} {\meaningof{\quotep{E}} = \{ \quotep{P} \in \quotep{\pi} | P \in \meaningof{E} \}, \and \meaningof{\quotep{P}} = \{ \quotep{Q} \in \quotep{\pi} | P \equiv Q \} \and \\ \meaningof{@\quotep{E}} = \{ P \in \pi | P \equiv @x, x \in \meaningof{E} \}}
\end{mathpar}

\begin{eqnarray*}
  \\
  \meaningof{-} : TS \to ST
\end{eqnarray*}

\begin{eqnarray*}
  \\
  L : TS \to ST
\end{eqnarray*}

\begin{eqnarray*}
  \\
  P \models E \iff P \in \meaningof{E}
\end{eqnarray*}

\begin{eqnarray*}
  P \approx_{L} Q \iff \forall E \in L. P \models E \iff Q \models E
\end{eqnarray*}

\begin{eqnarray*}
  P \approx_{K} Q
\end{eqnarray*}

\begin{eqnarray*}
  P \approx Q
\end{eqnarray*}

$\approx_{K} = \approx = \approx_{L}$

\subsubsection{Contextual duality}

Note that contexts extend the quotation operation to a family of
operations from processes to names. Given a context, $M$, we can
define a \emph{nominal context}, $\quotep{M}$ by $\quotep{M}[P] :=
\quotep{M[P]}$. To foreshadow what is to come we observe that these
operations enjoy a duality with processes very much like the duality
between vectors and maps from vectors to scalars.

Further, because the calculus is essentially higher-order, we have a
correspondence between contexts and processes. More specifically,
given a name $x$ and a context $M$ we can construct $M^{*}_{x}$ such
that 

\begin{mathpar}
  M^{*}_{x} | \lift{x}{P} \red M[P]
\end{mathpar}

namely,

\begin{mathpar}
  M^{*}_{x} := x?(u).M[\dropn{u}]
\end{mathpar}

The dependence of $M^{*}_{x}$ on a name makes it an abstraction, 

\begin{mathpar}
  M^{*} := (x)x?(u).M[\dropn{u}]
\end{mathpar}

\subsection{Additional notation}

It will sometimes be convenient to denote the process a name
quotes. We already have the notation $x = \quotep{P}$, but it will be
convenient to introduce an alternate notation, $\procn{x}$, when we
want to emphasize the connection to the use of the name. Note that, by
virtue of name equivalence, $\quotep{\procn{x}} \nameeq x$; so, the
notation is consistent with previous definitions.

Further, because names have structure it is possible to effect
substitutions on the basis of that structure. This means we need to
upgrade our notation for substitutions, which we accomplish by
adapting comprehension notation. Thus,

\begin{mathpar}
  P\{ y / x : x \in S \}
\end{mathpar}

is interpreted to mean the process derived from P by replacing (in a
capture-avoiding manner) each occurrence of $x$ in $S$ by $y$. For example,

\begin{mathpar}
  P\{ \quotep{\procn{x}|\procn{x}} / x : x \in \freenames{P} \}
\end{mathpar}

will replace each (occurrence) of a free name $x$ in $P$ by
$\quotep{\procn{x}|\procn{x}}$.

Also, we will avail ourselves of the notation $x^{L}$ and $x^{R}$ to
denote injections of a name into disjoint copies of the name
space. There are numerous ways to accomplish this. One example can be
found in \cite{MeredithR05}. This notation overloads to vectors of
names: $\vec{x}^{\pi} := (x_{i}^{\pi} \; : \; 0 \leq i < |\vec{x}| )$ where $\pi \in \{L,R\}$.

We also use $P^{\Box} := P|\Box$.

In \cite{MeredithR05} an interpretation of the new operator is
given. It turns out that there are several possible interpretations
all enjoying the requisite algebraic properties of the operator (see
\cite{milner91polyadicpi}). We will therefore make liberal use of
$(\nu\; \vec{x})P$.

% subsection the_syntax_and_semantics_of_the_notation_system (end)   

\input{qm2pi.qmops} 

\input{qm2pi.sterngerlach} 

\input{qm2pi.metric} 

% section concurrent_process_calculi (end)

%\input{qm2pi.proofsketch}

% section proof sketch (end)

%\input{qm2pi.slviaknots} 

% section spatial logic via knots (end)

\input{qm2pi.conclusion}

% section conclusion (end)

%\input{qm2pi.dtcodes} 

% section wiring algorithm (end)

\input{qm2pi.ack} 

% section acknowledgments (end)

\newpage


\bibliographystyle{plain}   
\bibliography{../../biblios/main.bib}

\input{qm2pi.rhodetails}

\end{document}

 

% section wiring algorithm (end)

\documentclass[12pt]{llncs}
%\documentclass{jktr}

\usepackage[pdftex]{hyperref}                   
\usepackage {listings}
\usepackage {mathpartir}
\usepackage{bcprules}
%\usepackage{listings}
                       
\usepackage{graphicx} 
%\usepackage[margins=2.5cm,nohead,nofoot]{geometry}
%\usepackage{geometry}
\usepackage{amsfonts}
\usepackage{amstext}
\usepackage{latexsym}
\usepackage{amssymb}
\usepackage{color}


%\include{myPreamble}
\include{qm2pi.local} 

%\ifpdf
%\usepackage[pdftex]{graphicx}
%\else
%\usepackage{graphicx}
%\fi

 % \ifpdf
%  \usepackage{pdfsync}
%  \if


%\title{Brief Article}
%\author{David F. Snyder}
%\author{L.G. Meredith}

%\address{Dept. of Math., Texas State University--San Marcos, San Marcos, TX 78666}
       
\pagestyle{empty}


\begin{document}

\lstset{language=[Objective]Caml,frame=shadowbox}

\input{qm2pi.front}

% section front matter (end)

\input{qm2pi.intro} 
 
% section introduction (end)

% \input{qm2pi.knotations} 

% section notation (end)

\input{qm2pi.process.calculi} 

% section concurrent_process_calculi_and_spatial_logics_ (end)
    
%\input{qm2pi.knots2pi} 

%\input{qm2pi.trefoil} 

%\input{qm2pi.mainthm} 

% subsection basic_interpretation (end)

%\input{qm2pi.rho.presentation} 
\subsection{The syntax and semantics of the notation system}\label{sub:the_syntax_and_semantics_of_the_notation_system} % (fold)

We now summarize a technical presentation of the calculus that
embodies our theory of dynamics. The typical presentation of such a
calculus follows the style of giving generators and relations on
them. The grammar, below, describing term constructors, freely
generates the set of processes, $\Proc$. This set is then quotiented
by a relation known as structural congruence and it is over this set
that the notion of dynamics is expressed. This presentation is
essentially that of \cite{MeredithR05} with the addition of
polyadicity and summation. For readability we have relegated some of
the technical subtleties to an appendix.

\subsubsection{Process grammar}\label{subsub:process_grammar}

\begin{mathpar}
  \inferrule* [lab=synchronization] {} {{M} \bc \pzero \;|\; x?F \;|\; x!C }
  \and
  \inferrule* [lab=abstraction] {} {{F} \bc (x)P}
  \and
  \inferrule* [lab=concretion] {} {{C} \bc \langle Q \rangle}
  \and
  \inferrule* [lab=process] {} {{P,Q} \bc M \;| \;P|Q \;|\; @{x}}
  \and
  \inferrule* [lab=name] {} {{x} \bc \quotep{P}}
\end{mathpar} 

Note that $\vec{x}$ (resp. $\vec{P}$) denotes a vector of names
(resp. processes) of length $|\vec{x}|$ (resp. $|\vec{P}|$). We adopt
the following useful abbreviations.

\begin{mathpar}
   x?(\vec{y}).P := x.(\vec{y})P \and  x\clift{\vec{P}} := x.\clift{\vec{P}}
   \and x!(y) := \lift{x}{\dropn{y}}
   \and \Pi_{i=0}^{n-1}P_i := P_0 | \ldots | P_{n-1}
\end{mathpar}

\subsubsection{Structural congruence}

\paragraph{Free and bound names and alpha-equivalence.} At the
core of structural equivalence is alpha-equivalence which identifies
process that are the same up to a change of variable. Formally, we
recognize the distinction between free and bound names. The free names
of a process, $\freenames{P}$, may be calculated recursively as
follows:

\begin{mathpar}
\freenames{\pzero} := \emptyset
  \and \\
  \freenames{x?(y).P} := \{ x \} \cup (\freenames{P} \setminus \{ y \})
  \and 
  \freenames{x!\langle P \rangle} := \{ x \} \cup \{ P \} 
  \and \\
  \freenames{P|Q} := \freenames{P} \cup \freenames{Q}
  \and \\
  \freenames{@{x}} := \{ x \}
\end{mathpar}

$\pi$
$\quotep{\pi}$

$\freenames{-} : \pi \to \mathcal{P}(\quotep{\pi})$

\begin{eqnarray*}
  \freenames{\pzero} & := & \emptyset \\
  \freenames{x?(y).P} & := & \{ x \} \cup (\freenames{P} \setminus \{ y \}) \\
  \freenames{x!\langle P \rangle} & := & \{ x \} \cup \{ P \} \\
  \freenames{P|Q} & := & \freenames{P} \cup \freenames{Q} \\
  \freenames{\dropn{x}} & := & \{ x \}
\end{eqnarray*}

The bound names of a process, $\boundnames{P}$, are those names occurring in $P$
that are not free. For example, in $x?(y).0$, the name $x$ is free, while $y$ is bound.

\begin{mathpar}
  \inferrule* [lab=monoidal-laws] {} { P|Q \equiv Q|P \and P|0 \equiv P \and P|(Q|R) \equiv (P|Q)|R }
\end{mathpar}

\begin{mathpar}
  \inferrule* [lab=alpha-equivalence] {} { (x)P \equiv (y)P\{y/x\} \and y \not\in \freenames{P} }
\end{mathpar}

\begin{definition}
Then two processes, $P,Q$, are alpha-equivalent if $P = Q\{\vec{y}/\vec{x}\}$ for
some $\vec{x} \in \boundnames{Q},\vec{y} \in \boundnames{P}$, where $Q\{\vec{y}/\vec{x}\}$
denotes the capture-avoiding substitution of $\vec{y}$ for $\vec{x}$ in $Q$.
\end{definition}

\begin{definition}
  The {\em structural congruence} \cite{SangiorgiWalker} , $\equiv$,
  between processes is the least congruence containing
  alpha-equivalence, satisfying the abelian monoid laws
  (associativity, commutativity and $\pzero$ as identity) for parallel
  composition $|$ and for summation $+$.
\end{definition}

\subsection{Name equivalence}

We take name equivalence, written $\nameeq$, to be the smallest
equivalence relation generated by the following rules.

\begin{mathpar}
\inferrule*[lab=Quote-drop]
{ }
{ \quotep{@{x}} \nameeq x }

\inferrule*[lab=Struct-equiv]
{ P \scong Q }
{ \quotep{P} \nameeq \quotep{Q} }
\end{mathpar}

The astute reader will have noticed that the mutual recursion of names
and processes imposes a mutual recursion on alpha-equivalence and
structural equivalence via name-equivalence. Fortunately, all of this
works out pleasantly and we may calculate in the natural way, free of
concern. The reader interested in the details is referred to the
appendix \ref{appendix:rho_details}.

\subsection{Substitution}

We use $\Proc$ for the set of processes, $\QProc$ for the set of
names, and $\id{\{}\vec{y} / \vec{x} \id{\}}$ to denote partial maps,
$s : \QProc \rightarrow \QProc$. A map, $s$ lifts, uniquely, to a map
on process terms, $\widehat{s} : \Proc \rightarrow \Proc$ by the
following equations.

\begin{mathpar}
  (0) \psubstp{Q}{P} := 0 \\
  (R \juxtap S) \psubstp{Q}{P}
  :=    
  (R)\psubstp{Q}{P} \juxtap (S) \psubstp{Q}{P} \\
  (x?(y).R) \psubstp{Q}{P}    
  :=    
  (x)\substp{Q}{P} (z)\concat( (R \psubstn{z}{y}) \psubstp{Q}{P} ) \\
  (\lift{x}{R}) \psubstp{Q}{P}  
  :=
  \lift{(x)\substp{Q}{P}}{ R \psubstp{Q}{P} } \\
%   (\dropn{x})  \psubstp{Q}{P}       
%   := 
%   \left\{ 
%     \begin{array}{ccc} 
%       \dropn{\quotep{Q}} & & x \nameeq \quotep{P} \\
%       \dropn{x} & & otherwise \\
%     \end{array}
%   \right. 
  (\dropn{x})  \psubstp{Q}{P}       
  := 
  \left\{ 
    \begin{array}{ccc} 
      Q & & x \nameeq \quotep{P} \\
      \dropn{x} & & otherwise \\
    \end{array}
  \right.
\end{mathpar}
 

where

\begin{eqnarray}
  (x)\id{\{} \lpquote Q \rpquote / \lpquote P \rpquote \id{\}}            = 
  \left\{ 
    \begin{array}{ccc}
      \lpquote Q \rpquote & & x \nameeq \lpquote P \rpquote \\
      x & & otherwise \\
    \end{array}
  \right. \nonumber
\end{eqnarray}

and $z$ is chosen distinct from $\quotep{P}$, $\quotep{Q}$, the free
names in $Q$, and all the names in $R$. Our $\alpha$-equivalence will
be built in the standard way from this substitution.

\begin{remark}\label{rem:no_self_referential_names}
  One consequence of these definitions is that $\forall P. \quotep{P}
  \not\in \freenames{P}$.
\end{remark}

\subsection{ Dynamic quote: an example }

Anticipating something of what's to come, consider applying the
substitution, $\widehat{\id{\{}u / z \id{\}}}$, to the following pair
of processes, $\lift{w}{y!(z)}$ and $w[ \lpquote y!(z) \rpquote ]$.

\begin{eqnarray}
	\lift{w}{y!(z)}\widehat{\id{\{}u / z \id{\}}}
		& = &
		\lift{w}{y!(u)} \nonumber\\
	w[ \lpquote y!(z) \rpquote ] \widehat{ \id{\{}u / z \id{\}} }
		& = &
		w[ \lpquote y!(z) \rpquote ] \nonumber
\end{eqnarray}

Because the body of the process between quotes is impervious to
substitution, we get radically different answers. In fact, by
examining the first process in an input context,
e.g. $x?(z).\lift{w}{y!(z)}$, we see that the process under the lift
operator may be shaped by prefixed inputs binding a name inside it. In
this sense, the lift operator will be seen as a way to dynamically
construct processes before reifying them as names.

Finally equipped with these standard features we can present the
dynamics of the calculus.

\subsubsection{Operational semantics} 

Finally, we introduce the computational dynamics. What marks these
algebras as distinct from other more traditionally studied algebraic
structures, e.g. vector spaces or polynomial rings, is the manner in
which dynamics is captured. In traditional structures, dynamics is typically
expressed through morphisms between such structures, as in linear maps
between vector spaces or morphisms between rings. In algebras
associated with the semantics of computation, the dynamics is
expressed as part of the algebraic structure itself, through a
reduction reduction relation typically denoted by $\red$. Below, we
give a recursive presentation of this relation for the calculus used
in the encoding.

$\red \subseteq \pi \times \pi$
$\red : \pi \to \mathcal{P}(\pi)$

\begin{mathpar}
  \inferrule* [lab=Comm] { \textsf{match}( x_{src}, x_{trgt} ) } { x_{trgt}?(y)P \; | \; x_{src}!\langle {Q} \rangle \red P\{\quotep{Q}/y}\} }
  \and \\
  \inferrule* [lab=Par] {{P} \red {P}'} {{{P} | {Q}} \red {{P}' | {Q}}}
  \and
  \inferrule* [lab=Equiv]{{{P} \scong {P}'} \andalso {{P}' \red {Q}'} \andalso {{Q}' \scong {Q}}}{{P} \red {Q}}
\end{mathpar}

\begin{eqnarray*}
  match_{\equiv} (\quotep{P},\quotep{Q}) & := & P \equiv Q \\
  match_{\dagger}(\quotep{P},\quotep{Q}) & := & \forall R. P|Q \red^{*} R => R \red^{*} 0 \\
  match_{K}(\quotep{P},\quotep{Q}) & := & K \mbox{ for some context } K
\end{eqnarray*}

$u?(x)P | u!\langle Q \rangle \red P\{\quotep{Q}/x\}$

%We write $\wred$ for $\red^*$, and $P\red$ if $\exists Q $ such that $ P \red Q$.
We write $P\red$ if $\exists Q $ such that $ P \red Q$ and $P\not\red$, otherwise.

\section{Replication}

As mentioned before, it is known that replication (and hence
recursion) can be implemented in a higher-order process algebra
\cite{SangiorgiWalker}. As our first example of calculation with the
machinery thus far presented we give the construction explicitly in
the {\rhoc}.

\begin{eqnarray}
	D_{x} & := & \prefix{x}{y}{(\binpar{\outputp{x}{y}}{@{y}})} \nonumber\\
	\bangp_{x}{P} & := & \binpar{{x}!\langle{\binpar{D_{x}}{P}}\rangle}{D_{x}} \nonumber
\end{eqnarray}

\begin{eqnarray}
	\bangp_{x}{P} & & \nonumber\\
	=
	& {x}!\langle{(\prefix{x}{y}{(\outputp{x}{y} | @{y})) | P}}\rangle 
	      | \prefix{x}{y}{(\outputp{x}{y} | @{y})} & \nonumber\\
	\red
	& (\outputp{x}{y} | @{y})\substn{\quotep{(\prefix{x}{y}{(@{y} | \outputp{x}{y})) | P}}}{y} & \nonumber\\
	=
	& \outputp{x}{\quotep{(\prefix{x}{y}{(\outputp{x}{y} | @{y})) | P}}}
	  | {(\prefix{x}{y}{(\outputp{x}{y} | @{y})) | P}} & \nonumber\\
	\red
	& \ldots & \nonumber\\
	\red^*
	& P | P | \ldots & \nonumber
\end{eqnarray}

Of course, this encoding, as an implementation, runs away, unfolding
$\bangp{P}$ eagerly. A lazier and more implementable replication
operator, restricted to input-guarded processes, may be obtained as follows.

\begin{eqnarray}
\bangp{\prefix{u}{v}{P}} 
	:= 
	\binpar{\lift{x}{\prefix{u}{v}{(\binpar{D(x)}{P})}}}{D(x)} \nonumber
\end{eqnarray}

\begin{remark}
  Note that the lazier definition still does not deal with summation
  or mixed summation (i.e. sums over input and output). The reader is
  invited to construct definitions of replication that deal with these
  features. 

  Further, the definitions are parameterized in a name, $x$. Can you,
  gentle reader, make a definition that eliminates this parameter and
  guarantees no accidental interaction between the replication
  machinery and the process being replicated -- i.e. no accidental
  sharing of names used by the process to get its work done and the
  name(s) used by the replication to effect copying. This latter
  revision of the definition of replication is crucial to obtaining
  the expected identity $!!P \sim !P$.
\end{remark}

\begin{remark}\label{rem:paradoxical_combinator}
  The reader familiar with the lambda calculus will have noticed the
  similarity between $D$ and the paradoxical combinator.

  [Ed. note: the existence of this seems to suggest we have to be more
  restrictive on the set of processes and names we admit if we are to
  support no-cloning.]
\end{remark}

\subsubsection{Bisimulation}

The computational dynamics gives rise to another kind of equivalence,
the equivalence of computational behavior. As previously mentioned
this is typically captured \emph{via} some form of bisimulation.

% The notion we use in this paper is weak barbed bisimulation
% \cite{milner91polyadicpi}.

The notion we use in this paper is derived from weak barbed
bisimulation \cite{milner91polyadicpi}. 

\begin{definition}
An \emph{observation relation}, $\downarrow_{\mathcal N}$, over a set
of names, $\mathcal N$, is the smallest relation satisfying the rules
below.

\infrule[Out-barb]{y \in {\mathcal N}, \; x \nameeq y}
		  {\outputp{x}{v} \downarrow_{\mathcal N} x}
\infrule[Par-barb]{\mbox{$P\downarrow_{\mathcal N} x$ or $Q\downarrow_{\mathcal N} x$}}
		  {\binpar{P}{Q} \downarrow_{\mathcal N} x}

We write $P \Downarrow_{\mathcal N} x$ if there is $Q$ such that 
$P \wred Q$ and $Q \downarrow_{\mathcal N} x$.
\end{definition}

\begin{definition}
%\label{def.bbisim}
An  ${\mathcal N}$-\emph{barbed bisimulation} over a set of names, ${\mathcal N}$, is a symmetric binary relation 
${\mathcal S}_{\mathcal N}$ between agents such that $P\rel{S}_{\mathcal N}Q$ implies:
\begin{enumerate}
\item If $P \red P'$ then $Q \wred Q'$ and $P'\rel{S}_{\mathcal N} Q'$.
\item If $P\downarrow_{\mathcal N} x$, then $Q\Downarrow_{\mathcal N} x$.
\end{enumerate}
$P$ is ${\mathcal N}$-barbed bisimilar to $Q$, written
$P \wbbisim_{\mathcal N} Q$, if $P \rel{S}_{\mathcal N} Q$ for some ${\mathcal N}$-barbed bisimulation ${\mathcal S}_{\mathcal N}$.
\end{definition}

$\mathcal{R} \subseteq \pi \times \pi$

$P \mathcal{R} Q => \forall P'. P \red P' \Rightarrow \exists Q'. Q \red Q', P' \mathcal{R} Q'$

$P \vdash x \Rightarrow Q \vdash x$

\begin{mathpar}
  \inferrule*[lab=Out-barb]{x \nameeq y}{{y}!\langle{Q}\rangle \vdash x}
  \and
  \inferrule*[lab=Par-barb]{\mbox{$P\vdash x$ or $Q\vdash x$}}{\binpar{P}{Q} \vdash x}
\end{mathpar}

\subsubsection{Contexts}

One of the principle advantages of computational calculi like the
$\pi$-calculus is a well-defined notion of context,
contextual-equivalence and a correlation between
contextual-equivalence and notions of bisimulation. The notion of
context allows the decomposition of a process into (sub-)process and
its syntactic environment, its context. Thus, a context may be
thought of as a process with a ``hole'' (written $\Box$) in it. The
application of a context $M$ to a process $P$, written $M[P]$, is
tantamount to filling the hole in $M$ with $P$. In this paper we do
not need the full weight of this theory, but do make use of the notion
of context in the proof the main theorem. 

\begin{mathpar}
  \inferrule* [lab=summation] {} {{M_{M},M_{N}} \bc \Box \;|\; x.M_{A} \;|\; M_{M}+M_{N}}
  \and
  \inferrule* [lab=agent] {} {{M_{A}} \bc (\vec{x})M_{P} \;| \; \clift{P_0,\ldots,M_{P},\ldots,P_N}}
  \and \\
  \inferrule* [lab=process] {} {{M_{P}} \bc M_{N} \;| \;P|M_{P} }
\end{mathpar} 

\begin{mathpar}
  \inferrule* [lab=sychronization] {} {M_{N} \bc \Box \;|\; x?M_{F} \;|\; x!M_{C}}
  \and
  \inferrule* [lab=abstraction] {} {{M_{F}} \bc (x)M_{P} }
  \and
  \inferrule* [lab=concretion] {} {{M_{C}} \bc \langle M_{P} \rangle }
  \and \\
  \inferrule* [lab=process] {} {{M_{P}} \bc M_{N} \;| \;P|M_{P} }
\end{mathpar}

\begin{definition}[contextual application] Given a context $M$, and
  process $P$, we define the \emph{contextual application}, $M[P] :=
  M\{P/\Box\}$. That is, the contextual application of M to P is the
  substitution of $P$ for $\Box$ in $M$.
\end{definition}

$\meaningof{-} : L \to \mathcal{P}(\pi)$

\begin{mathpar}
  \inferrule* [lab=collection] {} {\meaningof{true} = \pi, \and \meaningof{~E} = \pi \setminus \meaningof{E}, \and \meaningof{E_{1} \& E_{2}} = \meaningof{E_{1}} \cap \meaningof{E_{2}}}
\end{mathpar}

\begin{mathpar}
  \inferrule* [lab=structure] {} {\meaningof{0} = \{ P \in \pi | P \equiv 0 \}, \and \\ \meaningof{E_1 | E_2} = \{ P \in \pi | P \equiv P_{1} | P_{2}, P_{1} \in \meaningof{E_{1}}, P_{2} \in \meaningof{E_2}\} }
\end{mathpar}

\begin{mathpar}
 \inferrule* [lab=behavior] {} {\meaningof{\langle a?b \rangle E} = \{ P \in \pi | P \equiv Q | u?(y)P', \\ \and \\\\ \and \\ \;\;\; u \in \meaningof{a}, \forall z.P'\{z/y\} \in \meaningof{E\{z/b\}}\}, \and \\ \meaningof{a!E} = \{ P \in \pi | P \equiv Q | x!\langle P' \rangle, x \in \meaningof{a} P' \in \meaningof{E}\} }
\end{mathpar}

\begin{mathpar}
 \inferrule* [lab=nominal] {} {\meaningof{\quotep{E}} = \{ \quotep{P} \in \quotep{\pi} | P \in \meaningof{E} \}, \and \meaningof{\quotep{P}} = \{ \quotep{Q} \in \quotep{\pi} | P \equiv Q \} \and \\ \meaningof{@\quotep{E}} = \{ P \in \pi | P \equiv @x, x \in \meaningof{E} \}}
\end{mathpar}

\begin{eqnarray*}
  \\
  \meaningof{-} : TS \to ST
\end{eqnarray*}

\begin{eqnarray*}
  \\
  L : TS \to ST
\end{eqnarray*}

\begin{eqnarray*}
  \\
  P \models E \iff P \in \meaningof{E}
\end{eqnarray*}

\begin{eqnarray*}
  P \approx_{L} Q \iff \forall E \in L. P \models E \iff Q \models E
\end{eqnarray*}

\begin{eqnarray*}
  P \approx_{K} Q
\end{eqnarray*}

\begin{eqnarray*}
  P \approx Q
\end{eqnarray*}

$\approx_{K} = \approx = \approx_{L}$

\subsubsection{Contextual duality}

Note that contexts extend the quotation operation to a family of
operations from processes to names. Given a context, $M$, we can
define a \emph{nominal context}, $\quotep{M}$ by $\quotep{M}[P] :=
\quotep{M[P]}$. To foreshadow what is to come we observe that these
operations enjoy a duality with processes very much like the duality
between vectors and maps from vectors to scalars.

Further, because the calculus is essentially higher-order, we have a
correspondence between contexts and processes. More specifically,
given a name $x$ and a context $M$ we can construct $M^{*}_{x}$ such
that 

\begin{mathpar}
  M^{*}_{x} | \lift{x}{P} \red M[P]
\end{mathpar}

namely,

\begin{mathpar}
  M^{*}_{x} := x?(u).M[\dropn{u}]
\end{mathpar}

The dependence of $M^{*}_{x}$ on a name makes it an abstraction, 

\begin{mathpar}
  M^{*} := (x)x?(u).M[\dropn{u}]
\end{mathpar}

\subsection{Additional notation}

It will sometimes be convenient to denote the process a name
quotes. We already have the notation $x = \quotep{P}$, but it will be
convenient to introduce an alternate notation, $\procn{x}$, when we
want to emphasize the connection to the use of the name. Note that, by
virtue of name equivalence, $\quotep{\procn{x}} \nameeq x$; so, the
notation is consistent with previous definitions.

Further, because names have structure it is possible to effect
substitutions on the basis of that structure. This means we need to
upgrade our notation for substitutions, which we accomplish by
adapting comprehension notation. Thus,

\begin{mathpar}
  P\{ y / x : x \in S \}
\end{mathpar}

is interpreted to mean the process derived from P by replacing (in a
capture-avoiding manner) each occurrence of $x$ in $S$ by $y$. For example,

\begin{mathpar}
  P\{ \quotep{\procn{x}|\procn{x}} / x : x \in \freenames{P} \}
\end{mathpar}

will replace each (occurrence) of a free name $x$ in $P$ by
$\quotep{\procn{x}|\procn{x}}$.

Also, we will avail ourselves of the notation $x^{L}$ and $x^{R}$ to
denote injections of a name into disjoint copies of the name
space. There are numerous ways to accomplish this. One example can be
found in \cite{MeredithR05}. This notation overloads to vectors of
names: $\vec{x}^{\pi} := (x_{i}^{\pi} \; : \; 0 \leq i < |\vec{x}| )$ where $\pi \in \{L,R\}$.

We also use $P^{\Box} := P|\Box$.

In \cite{MeredithR05} an interpretation of the new operator is
given. It turns out that there are several possible interpretations
all enjoying the requisite algebraic properties of the operator (see
\cite{milner91polyadicpi}). We will therefore make liberal use of
$(\nu\; \vec{x})P$.

% subsection the_syntax_and_semantics_of_the_notation_system (end)   

\input{qm2pi.qmops} 

\input{qm2pi.sterngerlach} 

\input{qm2pi.metric} 

% section concurrent_process_calculi (end)

%\input{qm2pi.proofsketch}

% section proof sketch (end)

%\input{qm2pi.slviaknots} 

% section spatial logic via knots (end)

\input{qm2pi.conclusion}

% section conclusion (end)

%\input{qm2pi.dtcodes} 

% section wiring algorithm (end)

\input{qm2pi.ack} 

% section acknowledgments (end)

\newpage


\bibliographystyle{plain}   
\bibliography{../../biblios/main.bib}

\input{qm2pi.rhodetails}

\end{document}

 

% section acknowledgments (end)

\newpage


\bibliographystyle{plain}   
\bibliography{../../biblios/main.bib}

\documentclass[12pt]{llncs}
%\documentclass{jktr}

\usepackage[pdftex]{hyperref}                   
\usepackage {listings}
\usepackage {mathpartir}
\usepackage{bcprules}
%\usepackage{listings}
                       
\usepackage{graphicx} 
%\usepackage[margins=2.5cm,nohead,nofoot]{geometry}
%\usepackage{geometry}
\usepackage{amsfonts}
\usepackage{amstext}
\usepackage{latexsym}
\usepackage{amssymb}
\usepackage{color}


%\include{myPreamble}
\include{qm2pi.local} 

%\ifpdf
%\usepackage[pdftex]{graphicx}
%\else
%\usepackage{graphicx}
%\fi

 % \ifpdf
%  \usepackage{pdfsync}
%  \if


%\title{Brief Article}
%\author{David F. Snyder}
%\author{L.G. Meredith}

%\address{Dept. of Math., Texas State University--San Marcos, San Marcos, TX 78666}
       
\pagestyle{empty}


\begin{document}

\lstset{language=[Objective]Caml,frame=shadowbox}

\input{qm2pi.front}

% section front matter (end)

\input{qm2pi.intro} 
 
% section introduction (end)

% \input{qm2pi.knotations} 

% section notation (end)

\input{qm2pi.process.calculi} 

% section concurrent_process_calculi_and_spatial_logics_ (end)
    
%\input{qm2pi.knots2pi} 

%\input{qm2pi.trefoil} 

%\input{qm2pi.mainthm} 

% subsection basic_interpretation (end)

%\input{qm2pi.rho.presentation} 
\subsection{The syntax and semantics of the notation system}\label{sub:the_syntax_and_semantics_of_the_notation_system} % (fold)

We now summarize a technical presentation of the calculus that
embodies our theory of dynamics. The typical presentation of such a
calculus follows the style of giving generators and relations on
them. The grammar, below, describing term constructors, freely
generates the set of processes, $\Proc$. This set is then quotiented
by a relation known as structural congruence and it is over this set
that the notion of dynamics is expressed. This presentation is
essentially that of \cite{MeredithR05} with the addition of
polyadicity and summation. For readability we have relegated some of
the technical subtleties to an appendix.

\subsubsection{Process grammar}\label{subsub:process_grammar}

\begin{mathpar}
  \inferrule* [lab=synchronization] {} {{M} \bc \pzero \;|\; x?F \;|\; x!C }
  \and
  \inferrule* [lab=abstraction] {} {{F} \bc (x)P}
  \and
  \inferrule* [lab=concretion] {} {{C} \bc \langle Q \rangle}
  \and
  \inferrule* [lab=process] {} {{P,Q} \bc M \;| \;P|Q \;|\; @{x}}
  \and
  \inferrule* [lab=name] {} {{x} \bc \quotep{P}}
\end{mathpar} 

Note that $\vec{x}$ (resp. $\vec{P}$) denotes a vector of names
(resp. processes) of length $|\vec{x}|$ (resp. $|\vec{P}|$). We adopt
the following useful abbreviations.

\begin{mathpar}
   x?(\vec{y}).P := x.(\vec{y})P \and  x\clift{\vec{P}} := x.\clift{\vec{P}}
   \and x!(y) := \lift{x}{\dropn{y}}
   \and \Pi_{i=0}^{n-1}P_i := P_0 | \ldots | P_{n-1}
\end{mathpar}

\subsubsection{Structural congruence}

\paragraph{Free and bound names and alpha-equivalence.} At the
core of structural equivalence is alpha-equivalence which identifies
process that are the same up to a change of variable. Formally, we
recognize the distinction between free and bound names. The free names
of a process, $\freenames{P}$, may be calculated recursively as
follows:

\begin{mathpar}
\freenames{\pzero} := \emptyset
  \and \\
  \freenames{x?(y).P} := \{ x \} \cup (\freenames{P} \setminus \{ y \})
  \and 
  \freenames{x!\langle P \rangle} := \{ x \} \cup \{ P \} 
  \and \\
  \freenames{P|Q} := \freenames{P} \cup \freenames{Q}
  \and \\
  \freenames{@{x}} := \{ x \}
\end{mathpar}

$\pi$
$\quotep{\pi}$

$\freenames{-} : \pi \to \mathcal{P}(\quotep{\pi})$

\begin{eqnarray*}
  \freenames{\pzero} & := & \emptyset \\
  \freenames{x?(y).P} & := & \{ x \} \cup (\freenames{P} \setminus \{ y \}) \\
  \freenames{x!\langle P \rangle} & := & \{ x \} \cup \{ P \} \\
  \freenames{P|Q} & := & \freenames{P} \cup \freenames{Q} \\
  \freenames{\dropn{x}} & := & \{ x \}
\end{eqnarray*}

The bound names of a process, $\boundnames{P}$, are those names occurring in $P$
that are not free. For example, in $x?(y).0$, the name $x$ is free, while $y$ is bound.

\begin{mathpar}
  \inferrule* [lab=monoidal-laws] {} { P|Q \equiv Q|P \and P|0 \equiv P \and P|(Q|R) \equiv (P|Q)|R }
\end{mathpar}

\begin{mathpar}
  \inferrule* [lab=alpha-equivalence] {} { (x)P \equiv (y)P\{y/x\} \and y \not\in \freenames{P} }
\end{mathpar}

\begin{definition}
Then two processes, $P,Q$, are alpha-equivalent if $P = Q\{\vec{y}/\vec{x}\}$ for
some $\vec{x} \in \boundnames{Q},\vec{y} \in \boundnames{P}$, where $Q\{\vec{y}/\vec{x}\}$
denotes the capture-avoiding substitution of $\vec{y}$ for $\vec{x}$ in $Q$.
\end{definition}

\begin{definition}
  The {\em structural congruence} \cite{SangiorgiWalker} , $\equiv$,
  between processes is the least congruence containing
  alpha-equivalence, satisfying the abelian monoid laws
  (associativity, commutativity and $\pzero$ as identity) for parallel
  composition $|$ and for summation $+$.
\end{definition}

\subsection{Name equivalence}

We take name equivalence, written $\nameeq$, to be the smallest
equivalence relation generated by the following rules.

\begin{mathpar}
\inferrule*[lab=Quote-drop]
{ }
{ \quotep{@{x}} \nameeq x }

\inferrule*[lab=Struct-equiv]
{ P \scong Q }
{ \quotep{P} \nameeq \quotep{Q} }
\end{mathpar}

The astute reader will have noticed that the mutual recursion of names
and processes imposes a mutual recursion on alpha-equivalence and
structural equivalence via name-equivalence. Fortunately, all of this
works out pleasantly and we may calculate in the natural way, free of
concern. The reader interested in the details is referred to the
appendix \ref{appendix:rho_details}.

\subsection{Substitution}

We use $\Proc$ for the set of processes, $\QProc$ for the set of
names, and $\id{\{}\vec{y} / \vec{x} \id{\}}$ to denote partial maps,
$s : \QProc \rightarrow \QProc$. A map, $s$ lifts, uniquely, to a map
on process terms, $\widehat{s} : \Proc \rightarrow \Proc$ by the
following equations.

\begin{mathpar}
  (0) \psubstp{Q}{P} := 0 \\
  (R \juxtap S) \psubstp{Q}{P}
  :=    
  (R)\psubstp{Q}{P} \juxtap (S) \psubstp{Q}{P} \\
  (x?(y).R) \psubstp{Q}{P}    
  :=    
  (x)\substp{Q}{P} (z)\concat( (R \psubstn{z}{y}) \psubstp{Q}{P} ) \\
  (\lift{x}{R}) \psubstp{Q}{P}  
  :=
  \lift{(x)\substp{Q}{P}}{ R \psubstp{Q}{P} } \\
%   (\dropn{x})  \psubstp{Q}{P}       
%   := 
%   \left\{ 
%     \begin{array}{ccc} 
%       \dropn{\quotep{Q}} & & x \nameeq \quotep{P} \\
%       \dropn{x} & & otherwise \\
%     \end{array}
%   \right. 
  (\dropn{x})  \psubstp{Q}{P}       
  := 
  \left\{ 
    \begin{array}{ccc} 
      Q & & x \nameeq \quotep{P} \\
      \dropn{x} & & otherwise \\
    \end{array}
  \right.
\end{mathpar}
 

where

\begin{eqnarray}
  (x)\id{\{} \lpquote Q \rpquote / \lpquote P \rpquote \id{\}}            = 
  \left\{ 
    \begin{array}{ccc}
      \lpquote Q \rpquote & & x \nameeq \lpquote P \rpquote \\
      x & & otherwise \\
    \end{array}
  \right. \nonumber
\end{eqnarray}

and $z$ is chosen distinct from $\quotep{P}$, $\quotep{Q}$, the free
names in $Q$, and all the names in $R$. Our $\alpha$-equivalence will
be built in the standard way from this substitution.

\begin{remark}\label{rem:no_self_referential_names}
  One consequence of these definitions is that $\forall P. \quotep{P}
  \not\in \freenames{P}$.
\end{remark}

\subsection{ Dynamic quote: an example }

Anticipating something of what's to come, consider applying the
substitution, $\widehat{\id{\{}u / z \id{\}}}$, to the following pair
of processes, $\lift{w}{y!(z)}$ and $w[ \lpquote y!(z) \rpquote ]$.

\begin{eqnarray}
	\lift{w}{y!(z)}\widehat{\id{\{}u / z \id{\}}}
		& = &
		\lift{w}{y!(u)} \nonumber\\
	w[ \lpquote y!(z) \rpquote ] \widehat{ \id{\{}u / z \id{\}} }
		& = &
		w[ \lpquote y!(z) \rpquote ] \nonumber
\end{eqnarray}

Because the body of the process between quotes is impervious to
substitution, we get radically different answers. In fact, by
examining the first process in an input context,
e.g. $x?(z).\lift{w}{y!(z)}$, we see that the process under the lift
operator may be shaped by prefixed inputs binding a name inside it. In
this sense, the lift operator will be seen as a way to dynamically
construct processes before reifying them as names.

Finally equipped with these standard features we can present the
dynamics of the calculus.

\subsubsection{Operational semantics} 

Finally, we introduce the computational dynamics. What marks these
algebras as distinct from other more traditionally studied algebraic
structures, e.g. vector spaces or polynomial rings, is the manner in
which dynamics is captured. In traditional structures, dynamics is typically
expressed through morphisms between such structures, as in linear maps
between vector spaces or morphisms between rings. In algebras
associated with the semantics of computation, the dynamics is
expressed as part of the algebraic structure itself, through a
reduction reduction relation typically denoted by $\red$. Below, we
give a recursive presentation of this relation for the calculus used
in the encoding.

$\red \subseteq \pi \times \pi$
$\red : \pi \to \mathcal{P}(\pi)$

\begin{mathpar}
  \inferrule* [lab=Comm] { \textsf{match}( x_{src}, x_{trgt} ) } { x_{trgt}?(y)P \; | \; x_{src}!\langle {Q} \rangle \red P\{\quotep{Q}/y}\} }
  \and \\
  \inferrule* [lab=Par] {{P} \red {P}'} {{{P} | {Q}} \red {{P}' | {Q}}}
  \and
  \inferrule* [lab=Equiv]{{{P} \scong {P}'} \andalso {{P}' \red {Q}'} \andalso {{Q}' \scong {Q}}}{{P} \red {Q}}
\end{mathpar}

\begin{eqnarray*}
  match_{\equiv} (\quotep{P},\quotep{Q}) & := & P \equiv Q \\
  match_{\dagger}(\quotep{P},\quotep{Q}) & := & \forall R. P|Q \red^{*} R => R \red^{*} 0 \\
  match_{K}(\quotep{P},\quotep{Q}) & := & K \mbox{ for some context } K
\end{eqnarray*}

$u?(x)P | u!\langle Q \rangle \red P\{\quotep{Q}/x\}$

%We write $\wred$ for $\red^*$, and $P\red$ if $\exists Q $ such that $ P \red Q$.
We write $P\red$ if $\exists Q $ such that $ P \red Q$ and $P\not\red$, otherwise.

\section{Replication}

As mentioned before, it is known that replication (and hence
recursion) can be implemented in a higher-order process algebra
\cite{SangiorgiWalker}. As our first example of calculation with the
machinery thus far presented we give the construction explicitly in
the {\rhoc}.

\begin{eqnarray}
	D_{x} & := & \prefix{x}{y}{(\binpar{\outputp{x}{y}}{@{y}})} \nonumber\\
	\bangp_{x}{P} & := & \binpar{{x}!\langle{\binpar{D_{x}}{P}}\rangle}{D_{x}} \nonumber
\end{eqnarray}

\begin{eqnarray}
	\bangp_{x}{P} & & \nonumber\\
	=
	& {x}!\langle{(\prefix{x}{y}{(\outputp{x}{y} | @{y})) | P}}\rangle 
	      | \prefix{x}{y}{(\outputp{x}{y} | @{y})} & \nonumber\\
	\red
	& (\outputp{x}{y} | @{y})\substn{\quotep{(\prefix{x}{y}{(@{y} | \outputp{x}{y})) | P}}}{y} & \nonumber\\
	=
	& \outputp{x}{\quotep{(\prefix{x}{y}{(\outputp{x}{y} | @{y})) | P}}}
	  | {(\prefix{x}{y}{(\outputp{x}{y} | @{y})) | P}} & \nonumber\\
	\red
	& \ldots & \nonumber\\
	\red^*
	& P | P | \ldots & \nonumber
\end{eqnarray}

Of course, this encoding, as an implementation, runs away, unfolding
$\bangp{P}$ eagerly. A lazier and more implementable replication
operator, restricted to input-guarded processes, may be obtained as follows.

\begin{eqnarray}
\bangp{\prefix{u}{v}{P}} 
	:= 
	\binpar{\lift{x}{\prefix{u}{v}{(\binpar{D(x)}{P})}}}{D(x)} \nonumber
\end{eqnarray}

\begin{remark}
  Note that the lazier definition still does not deal with summation
  or mixed summation (i.e. sums over input and output). The reader is
  invited to construct definitions of replication that deal with these
  features. 

  Further, the definitions are parameterized in a name, $x$. Can you,
  gentle reader, make a definition that eliminates this parameter and
  guarantees no accidental interaction between the replication
  machinery and the process being replicated -- i.e. no accidental
  sharing of names used by the process to get its work done and the
  name(s) used by the replication to effect copying. This latter
  revision of the definition of replication is crucial to obtaining
  the expected identity $!!P \sim !P$.
\end{remark}

\begin{remark}\label{rem:paradoxical_combinator}
  The reader familiar with the lambda calculus will have noticed the
  similarity between $D$ and the paradoxical combinator.

  [Ed. note: the existence of this seems to suggest we have to be more
  restrictive on the set of processes and names we admit if we are to
  support no-cloning.]
\end{remark}

\subsubsection{Bisimulation}

The computational dynamics gives rise to another kind of equivalence,
the equivalence of computational behavior. As previously mentioned
this is typically captured \emph{via} some form of bisimulation.

% The notion we use in this paper is weak barbed bisimulation
% \cite{milner91polyadicpi}.

The notion we use in this paper is derived from weak barbed
bisimulation \cite{milner91polyadicpi}. 

\begin{definition}
An \emph{observation relation}, $\downarrow_{\mathcal N}$, over a set
of names, $\mathcal N$, is the smallest relation satisfying the rules
below.

\infrule[Out-barb]{y \in {\mathcal N}, \; x \nameeq y}
		  {\outputp{x}{v} \downarrow_{\mathcal N} x}
\infrule[Par-barb]{\mbox{$P\downarrow_{\mathcal N} x$ or $Q\downarrow_{\mathcal N} x$}}
		  {\binpar{P}{Q} \downarrow_{\mathcal N} x}

We write $P \Downarrow_{\mathcal N} x$ if there is $Q$ such that 
$P \wred Q$ and $Q \downarrow_{\mathcal N} x$.
\end{definition}

\begin{definition}
%\label{def.bbisim}
An  ${\mathcal N}$-\emph{barbed bisimulation} over a set of names, ${\mathcal N}$, is a symmetric binary relation 
${\mathcal S}_{\mathcal N}$ between agents such that $P\rel{S}_{\mathcal N}Q$ implies:
\begin{enumerate}
\item If $P \red P'$ then $Q \wred Q'$ and $P'\rel{S}_{\mathcal N} Q'$.
\item If $P\downarrow_{\mathcal N} x$, then $Q\Downarrow_{\mathcal N} x$.
\end{enumerate}
$P$ is ${\mathcal N}$-barbed bisimilar to $Q$, written
$P \wbbisim_{\mathcal N} Q$, if $P \rel{S}_{\mathcal N} Q$ for some ${\mathcal N}$-barbed bisimulation ${\mathcal S}_{\mathcal N}$.
\end{definition}

$\mathcal{R} \subseteq \pi \times \pi$

$P \mathcal{R} Q => \forall P'. P \red P' \Rightarrow \exists Q'. Q \red Q', P' \mathcal{R} Q'$

$P \vdash x \Rightarrow Q \vdash x$

\begin{mathpar}
  \inferrule*[lab=Out-barb]{x \nameeq y}{{y}!\langle{Q}\rangle \vdash x}
  \and
  \inferrule*[lab=Par-barb]{\mbox{$P\vdash x$ or $Q\vdash x$}}{\binpar{P}{Q} \vdash x}
\end{mathpar}

\subsubsection{Contexts}

One of the principle advantages of computational calculi like the
$\pi$-calculus is a well-defined notion of context,
contextual-equivalence and a correlation between
contextual-equivalence and notions of bisimulation. The notion of
context allows the decomposition of a process into (sub-)process and
its syntactic environment, its context. Thus, a context may be
thought of as a process with a ``hole'' (written $\Box$) in it. The
application of a context $M$ to a process $P$, written $M[P]$, is
tantamount to filling the hole in $M$ with $P$. In this paper we do
not need the full weight of this theory, but do make use of the notion
of context in the proof the main theorem. 

\begin{mathpar}
  \inferrule* [lab=summation] {} {{M_{M},M_{N}} \bc \Box \;|\; x.M_{A} \;|\; M_{M}+M_{N}}
  \and
  \inferrule* [lab=agent] {} {{M_{A}} \bc (\vec{x})M_{P} \;| \; \clift{P_0,\ldots,M_{P},\ldots,P_N}}
  \and \\
  \inferrule* [lab=process] {} {{M_{P}} \bc M_{N} \;| \;P|M_{P} }
\end{mathpar} 

\begin{mathpar}
  \inferrule* [lab=sychronization] {} {M_{N} \bc \Box \;|\; x?M_{F} \;|\; x!M_{C}}
  \and
  \inferrule* [lab=abstraction] {} {{M_{F}} \bc (x)M_{P} }
  \and
  \inferrule* [lab=concretion] {} {{M_{C}} \bc \langle M_{P} \rangle }
  \and \\
  \inferrule* [lab=process] {} {{M_{P}} \bc M_{N} \;| \;P|M_{P} }
\end{mathpar}

\begin{definition}[contextual application] Given a context $M$, and
  process $P$, we define the \emph{contextual application}, $M[P] :=
  M\{P/\Box\}$. That is, the contextual application of M to P is the
  substitution of $P$ for $\Box$ in $M$.
\end{definition}

$\meaningof{-} : L \to \mathcal{P}(\pi)$

\begin{mathpar}
  \inferrule* [lab=collection] {} {\meaningof{true} = \pi, \and \meaningof{~E} = \pi \setminus \meaningof{E}, \and \meaningof{E_{1} \& E_{2}} = \meaningof{E_{1}} \cap \meaningof{E_{2}}}
\end{mathpar}

\begin{mathpar}
  \inferrule* [lab=structure] {} {\meaningof{0} = \{ P \in \pi | P \equiv 0 \}, \and \\ \meaningof{E_1 | E_2} = \{ P \in \pi | P \equiv P_{1} | P_{2}, P_{1} \in \meaningof{E_{1}}, P_{2} \in \meaningof{E_2}\} }
\end{mathpar}

\begin{mathpar}
 \inferrule* [lab=behavior] {} {\meaningof{\langle a?b \rangle E} = \{ P \in \pi | P \equiv Q | u?(y)P', \\ \and \\\\ \and \\ \;\;\; u \in \meaningof{a}, \forall z.P'\{z/y\} \in \meaningof{E\{z/b\}}\}, \and \\ \meaningof{a!E} = \{ P \in \pi | P \equiv Q | x!\langle P' \rangle, x \in \meaningof{a} P' \in \meaningof{E}\} }
\end{mathpar}

\begin{mathpar}
 \inferrule* [lab=nominal] {} {\meaningof{\quotep{E}} = \{ \quotep{P} \in \quotep{\pi} | P \in \meaningof{E} \}, \and \meaningof{\quotep{P}} = \{ \quotep{Q} \in \quotep{\pi} | P \equiv Q \} \and \\ \meaningof{@\quotep{E}} = \{ P \in \pi | P \equiv @x, x \in \meaningof{E} \}}
\end{mathpar}

\begin{eqnarray*}
  \\
  \meaningof{-} : TS \to ST
\end{eqnarray*}

\begin{eqnarray*}
  \\
  L : TS \to ST
\end{eqnarray*}

\begin{eqnarray*}
  \\
  P \models E \iff P \in \meaningof{E}
\end{eqnarray*}

\begin{eqnarray*}
  P \approx_{L} Q \iff \forall E \in L. P \models E \iff Q \models E
\end{eqnarray*}

\begin{eqnarray*}
  P \approx_{K} Q
\end{eqnarray*}

\begin{eqnarray*}
  P \approx Q
\end{eqnarray*}

$\approx_{K} = \approx = \approx_{L}$

\subsubsection{Contextual duality}

Note that contexts extend the quotation operation to a family of
operations from processes to names. Given a context, $M$, we can
define a \emph{nominal context}, $\quotep{M}$ by $\quotep{M}[P] :=
\quotep{M[P]}$. To foreshadow what is to come we observe that these
operations enjoy a duality with processes very much like the duality
between vectors and maps from vectors to scalars.

Further, because the calculus is essentially higher-order, we have a
correspondence between contexts and processes. More specifically,
given a name $x$ and a context $M$ we can construct $M^{*}_{x}$ such
that 

\begin{mathpar}
  M^{*}_{x} | \lift{x}{P} \red M[P]
\end{mathpar}

namely,

\begin{mathpar}
  M^{*}_{x} := x?(u).M[\dropn{u}]
\end{mathpar}

The dependence of $M^{*}_{x}$ on a name makes it an abstraction, 

\begin{mathpar}
  M^{*} := (x)x?(u).M[\dropn{u}]
\end{mathpar}

\subsection{Additional notation}

It will sometimes be convenient to denote the process a name
quotes. We already have the notation $x = \quotep{P}$, but it will be
convenient to introduce an alternate notation, $\procn{x}$, when we
want to emphasize the connection to the use of the name. Note that, by
virtue of name equivalence, $\quotep{\procn{x}} \nameeq x$; so, the
notation is consistent with previous definitions.

Further, because names have structure it is possible to effect
substitutions on the basis of that structure. This means we need to
upgrade our notation for substitutions, which we accomplish by
adapting comprehension notation. Thus,

\begin{mathpar}
  P\{ y / x : x \in S \}
\end{mathpar}

is interpreted to mean the process derived from P by replacing (in a
capture-avoiding manner) each occurrence of $x$ in $S$ by $y$. For example,

\begin{mathpar}
  P\{ \quotep{\procn{x}|\procn{x}} / x : x \in \freenames{P} \}
\end{mathpar}

will replace each (occurrence) of a free name $x$ in $P$ by
$\quotep{\procn{x}|\procn{x}}$.

Also, we will avail ourselves of the notation $x^{L}$ and $x^{R}$ to
denote injections of a name into disjoint copies of the name
space. There are numerous ways to accomplish this. One example can be
found in \cite{MeredithR05}. This notation overloads to vectors of
names: $\vec{x}^{\pi} := (x_{i}^{\pi} \; : \; 0 \leq i < |\vec{x}| )$ where $\pi \in \{L,R\}$.

We also use $P^{\Box} := P|\Box$.

In \cite{MeredithR05} an interpretation of the new operator is
given. It turns out that there are several possible interpretations
all enjoying the requisite algebraic properties of the operator (see
\cite{milner91polyadicpi}). We will therefore make liberal use of
$(\nu\; \vec{x})P$.

% subsection the_syntax_and_semantics_of_the_notation_system (end)   

\input{qm2pi.qmops} 

\input{qm2pi.sterngerlach} 

\input{qm2pi.metric} 

% section concurrent_process_calculi (end)

%\input{qm2pi.proofsketch}

% section proof sketch (end)

%\input{qm2pi.slviaknots} 

% section spatial logic via knots (end)

\input{qm2pi.conclusion}

% section conclusion (end)

%\input{qm2pi.dtcodes} 

% section wiring algorithm (end)

\input{qm2pi.ack} 

% section acknowledgments (end)

\newpage


\bibliographystyle{plain}   
\bibliography{../../biblios/main.bib}

\input{qm2pi.rhodetails}

\end{document}



\end{document}

 

% section wiring algorithm (end)

\documentclass[12pt]{llncs}
%\documentclass{jktr}

\usepackage[pdftex]{hyperref}                   
\usepackage {listings}
\usepackage {mathpartir}
\usepackage{bcprules}
%\usepackage{listings}
                       
\usepackage{graphicx} 
%\usepackage[margins=2.5cm,nohead,nofoot]{geometry}
%\usepackage{geometry}
\usepackage{amsfonts}
\usepackage{amstext}
\usepackage{latexsym}
\usepackage{amssymb}
\usepackage{color}


%\include{myPreamble}
\documentclass[12pt]{llncs}
%\documentclass{jktr}

\usepackage[pdftex]{hyperref}                   
\usepackage {listings}
\usepackage {mathpartir}
\usepackage{bcprules}
%\usepackage{listings}
                       
\usepackage{graphicx} 
%\usepackage[margins=2.5cm,nohead,nofoot]{geometry}
%\usepackage{geometry}
\usepackage{amsfonts}
\usepackage{amstext}
\usepackage{latexsym}
\usepackage{amssymb}
\usepackage{color}


%\include{myPreamble}
\include{qm2pi.local} 

%\ifpdf
%\usepackage[pdftex]{graphicx}
%\else
%\usepackage{graphicx}
%\fi

 % \ifpdf
%  \usepackage{pdfsync}
%  \if


%\title{Brief Article}
%\author{David F. Snyder}
%\author{L.G. Meredith}

%\address{Dept. of Math., Texas State University--San Marcos, San Marcos, TX 78666}
       
\pagestyle{empty}


\begin{document}

\lstset{language=[Objective]Caml,frame=shadowbox}

\input{qm2pi.front}

% section front matter (end)

\input{qm2pi.intro} 
 
% section introduction (end)

% \input{qm2pi.knotations} 

% section notation (end)

\input{qm2pi.process.calculi} 

% section concurrent_process_calculi_and_spatial_logics_ (end)
    
%\input{qm2pi.knots2pi} 

%\input{qm2pi.trefoil} 

%\input{qm2pi.mainthm} 

% subsection basic_interpretation (end)

%\input{qm2pi.rho.presentation} 
\subsection{The syntax and semantics of the notation system}\label{sub:the_syntax_and_semantics_of_the_notation_system} % (fold)

We now summarize a technical presentation of the calculus that
embodies our theory of dynamics. The typical presentation of such a
calculus follows the style of giving generators and relations on
them. The grammar, below, describing term constructors, freely
generates the set of processes, $\Proc$. This set is then quotiented
by a relation known as structural congruence and it is over this set
that the notion of dynamics is expressed. This presentation is
essentially that of \cite{MeredithR05} with the addition of
polyadicity and summation. For readability we have relegated some of
the technical subtleties to an appendix.

\subsubsection{Process grammar}\label{subsub:process_grammar}

\begin{mathpar}
  \inferrule* [lab=synchronization] {} {{M} \bc \pzero \;|\; x?F \;|\; x!C }
  \and
  \inferrule* [lab=abstraction] {} {{F} \bc (x)P}
  \and
  \inferrule* [lab=concretion] {} {{C} \bc \langle Q \rangle}
  \and
  \inferrule* [lab=process] {} {{P,Q} \bc M \;| \;P|Q \;|\; @{x}}
  \and
  \inferrule* [lab=name] {} {{x} \bc \quotep{P}}
\end{mathpar} 

Note that $\vec{x}$ (resp. $\vec{P}$) denotes a vector of names
(resp. processes) of length $|\vec{x}|$ (resp. $|\vec{P}|$). We adopt
the following useful abbreviations.

\begin{mathpar}
   x?(\vec{y}).P := x.(\vec{y})P \and  x\clift{\vec{P}} := x.\clift{\vec{P}}
   \and x!(y) := \lift{x}{\dropn{y}}
   \and \Pi_{i=0}^{n-1}P_i := P_0 | \ldots | P_{n-1}
\end{mathpar}

\subsubsection{Structural congruence}

\paragraph{Free and bound names and alpha-equivalence.} At the
core of structural equivalence is alpha-equivalence which identifies
process that are the same up to a change of variable. Formally, we
recognize the distinction between free and bound names. The free names
of a process, $\freenames{P}$, may be calculated recursively as
follows:

\begin{mathpar}
\freenames{\pzero} := \emptyset
  \and \\
  \freenames{x?(y).P} := \{ x \} \cup (\freenames{P} \setminus \{ y \})
  \and 
  \freenames{x!\langle P \rangle} := \{ x \} \cup \{ P \} 
  \and \\
  \freenames{P|Q} := \freenames{P} \cup \freenames{Q}
  \and \\
  \freenames{@{x}} := \{ x \}
\end{mathpar}

$\pi$
$\quotep{\pi}$

$\freenames{-} : \pi \to \mathcal{P}(\quotep{\pi})$

\begin{eqnarray*}
  \freenames{\pzero} & := & \emptyset \\
  \freenames{x?(y).P} & := & \{ x \} \cup (\freenames{P} \setminus \{ y \}) \\
  \freenames{x!\langle P \rangle} & := & \{ x \} \cup \{ P \} \\
  \freenames{P|Q} & := & \freenames{P} \cup \freenames{Q} \\
  \freenames{\dropn{x}} & := & \{ x \}
\end{eqnarray*}

The bound names of a process, $\boundnames{P}$, are those names occurring in $P$
that are not free. For example, in $x?(y).0$, the name $x$ is free, while $y$ is bound.

\begin{mathpar}
  \inferrule* [lab=monoidal-laws] {} { P|Q \equiv Q|P \and P|0 \equiv P \and P|(Q|R) \equiv (P|Q)|R }
\end{mathpar}

\begin{mathpar}
  \inferrule* [lab=alpha-equivalence] {} { (x)P \equiv (y)P\{y/x\} \and y \not\in \freenames{P} }
\end{mathpar}

\begin{definition}
Then two processes, $P,Q$, are alpha-equivalent if $P = Q\{\vec{y}/\vec{x}\}$ for
some $\vec{x} \in \boundnames{Q},\vec{y} \in \boundnames{P}$, where $Q\{\vec{y}/\vec{x}\}$
denotes the capture-avoiding substitution of $\vec{y}$ for $\vec{x}$ in $Q$.
\end{definition}

\begin{definition}
  The {\em structural congruence} \cite{SangiorgiWalker} , $\equiv$,
  between processes is the least congruence containing
  alpha-equivalence, satisfying the abelian monoid laws
  (associativity, commutativity and $\pzero$ as identity) for parallel
  composition $|$ and for summation $+$.
\end{definition}

\subsection{Name equivalence}

We take name equivalence, written $\nameeq$, to be the smallest
equivalence relation generated by the following rules.

\begin{mathpar}
\inferrule*[lab=Quote-drop]
{ }
{ \quotep{@{x}} \nameeq x }

\inferrule*[lab=Struct-equiv]
{ P \scong Q }
{ \quotep{P} \nameeq \quotep{Q} }
\end{mathpar}

The astute reader will have noticed that the mutual recursion of names
and processes imposes a mutual recursion on alpha-equivalence and
structural equivalence via name-equivalence. Fortunately, all of this
works out pleasantly and we may calculate in the natural way, free of
concern. The reader interested in the details is referred to the
appendix \ref{appendix:rho_details}.

\subsection{Substitution}

We use $\Proc$ for the set of processes, $\QProc$ for the set of
names, and $\id{\{}\vec{y} / \vec{x} \id{\}}$ to denote partial maps,
$s : \QProc \rightarrow \QProc$. A map, $s$ lifts, uniquely, to a map
on process terms, $\widehat{s} : \Proc \rightarrow \Proc$ by the
following equations.

\begin{mathpar}
  (0) \psubstp{Q}{P} := 0 \\
  (R \juxtap S) \psubstp{Q}{P}
  :=    
  (R)\psubstp{Q}{P} \juxtap (S) \psubstp{Q}{P} \\
  (x?(y).R) \psubstp{Q}{P}    
  :=    
  (x)\substp{Q}{P} (z)\concat( (R \psubstn{z}{y}) \psubstp{Q}{P} ) \\
  (\lift{x}{R}) \psubstp{Q}{P}  
  :=
  \lift{(x)\substp{Q}{P}}{ R \psubstp{Q}{P} } \\
%   (\dropn{x})  \psubstp{Q}{P}       
%   := 
%   \left\{ 
%     \begin{array}{ccc} 
%       \dropn{\quotep{Q}} & & x \nameeq \quotep{P} \\
%       \dropn{x} & & otherwise \\
%     \end{array}
%   \right. 
  (\dropn{x})  \psubstp{Q}{P}       
  := 
  \left\{ 
    \begin{array}{ccc} 
      Q & & x \nameeq \quotep{P} \\
      \dropn{x} & & otherwise \\
    \end{array}
  \right.
\end{mathpar}
 

where

\begin{eqnarray}
  (x)\id{\{} \lpquote Q \rpquote / \lpquote P \rpquote \id{\}}            = 
  \left\{ 
    \begin{array}{ccc}
      \lpquote Q \rpquote & & x \nameeq \lpquote P \rpquote \\
      x & & otherwise \\
    \end{array}
  \right. \nonumber
\end{eqnarray}

and $z$ is chosen distinct from $\quotep{P}$, $\quotep{Q}$, the free
names in $Q$, and all the names in $R$. Our $\alpha$-equivalence will
be built in the standard way from this substitution.

\begin{remark}\label{rem:no_self_referential_names}
  One consequence of these definitions is that $\forall P. \quotep{P}
  \not\in \freenames{P}$.
\end{remark}

\subsection{ Dynamic quote: an example }

Anticipating something of what's to come, consider applying the
substitution, $\widehat{\id{\{}u / z \id{\}}}$, to the following pair
of processes, $\lift{w}{y!(z)}$ and $w[ \lpquote y!(z) \rpquote ]$.

\begin{eqnarray}
	\lift{w}{y!(z)}\widehat{\id{\{}u / z \id{\}}}
		& = &
		\lift{w}{y!(u)} \nonumber\\
	w[ \lpquote y!(z) \rpquote ] \widehat{ \id{\{}u / z \id{\}} }
		& = &
		w[ \lpquote y!(z) \rpquote ] \nonumber
\end{eqnarray}

Because the body of the process between quotes is impervious to
substitution, we get radically different answers. In fact, by
examining the first process in an input context,
e.g. $x?(z).\lift{w}{y!(z)}$, we see that the process under the lift
operator may be shaped by prefixed inputs binding a name inside it. In
this sense, the lift operator will be seen as a way to dynamically
construct processes before reifying them as names.

Finally equipped with these standard features we can present the
dynamics of the calculus.

\subsubsection{Operational semantics} 

Finally, we introduce the computational dynamics. What marks these
algebras as distinct from other more traditionally studied algebraic
structures, e.g. vector spaces or polynomial rings, is the manner in
which dynamics is captured. In traditional structures, dynamics is typically
expressed through morphisms between such structures, as in linear maps
between vector spaces or morphisms between rings. In algebras
associated with the semantics of computation, the dynamics is
expressed as part of the algebraic structure itself, through a
reduction reduction relation typically denoted by $\red$. Below, we
give a recursive presentation of this relation for the calculus used
in the encoding.

$\red \subseteq \pi \times \pi$
$\red : \pi \to \mathcal{P}(\pi)$

\begin{mathpar}
  \inferrule* [lab=Comm] { \textsf{match}( x_{src}, x_{trgt} ) } { x_{trgt}?(y)P \; | \; x_{src}!\langle {Q} \rangle \red P\{\quotep{Q}/y}\} }
  \and \\
  \inferrule* [lab=Par] {{P} \red {P}'} {{{P} | {Q}} \red {{P}' | {Q}}}
  \and
  \inferrule* [lab=Equiv]{{{P} \scong {P}'} \andalso {{P}' \red {Q}'} \andalso {{Q}' \scong {Q}}}{{P} \red {Q}}
\end{mathpar}

\begin{eqnarray*}
  match_{\equiv} (\quotep{P},\quotep{Q}) & := & P \equiv Q \\
  match_{\dagger}(\quotep{P},\quotep{Q}) & := & \forall R. P|Q \red^{*} R => R \red^{*} 0 \\
  match_{K}(\quotep{P},\quotep{Q}) & := & K \mbox{ for some context } K
\end{eqnarray*}

$u?(x)P | u!\langle Q \rangle \red P\{\quotep{Q}/x\}$

%We write $\wred$ for $\red^*$, and $P\red$ if $\exists Q $ such that $ P \red Q$.
We write $P\red$ if $\exists Q $ such that $ P \red Q$ and $P\not\red$, otherwise.

\section{Replication}

As mentioned before, it is known that replication (and hence
recursion) can be implemented in a higher-order process algebra
\cite{SangiorgiWalker}. As our first example of calculation with the
machinery thus far presented we give the construction explicitly in
the {\rhoc}.

\begin{eqnarray}
	D_{x} & := & \prefix{x}{y}{(\binpar{\outputp{x}{y}}{@{y}})} \nonumber\\
	\bangp_{x}{P} & := & \binpar{{x}!\langle{\binpar{D_{x}}{P}}\rangle}{D_{x}} \nonumber
\end{eqnarray}

\begin{eqnarray}
	\bangp_{x}{P} & & \nonumber\\
	=
	& {x}!\langle{(\prefix{x}{y}{(\outputp{x}{y} | @{y})) | P}}\rangle 
	      | \prefix{x}{y}{(\outputp{x}{y} | @{y})} & \nonumber\\
	\red
	& (\outputp{x}{y} | @{y})\substn{\quotep{(\prefix{x}{y}{(@{y} | \outputp{x}{y})) | P}}}{y} & \nonumber\\
	=
	& \outputp{x}{\quotep{(\prefix{x}{y}{(\outputp{x}{y} | @{y})) | P}}}
	  | {(\prefix{x}{y}{(\outputp{x}{y} | @{y})) | P}} & \nonumber\\
	\red
	& \ldots & \nonumber\\
	\red^*
	& P | P | \ldots & \nonumber
\end{eqnarray}

Of course, this encoding, as an implementation, runs away, unfolding
$\bangp{P}$ eagerly. A lazier and more implementable replication
operator, restricted to input-guarded processes, may be obtained as follows.

\begin{eqnarray}
\bangp{\prefix{u}{v}{P}} 
	:= 
	\binpar{\lift{x}{\prefix{u}{v}{(\binpar{D(x)}{P})}}}{D(x)} \nonumber
\end{eqnarray}

\begin{remark}
  Note that the lazier definition still does not deal with summation
  or mixed summation (i.e. sums over input and output). The reader is
  invited to construct definitions of replication that deal with these
  features. 

  Further, the definitions are parameterized in a name, $x$. Can you,
  gentle reader, make a definition that eliminates this parameter and
  guarantees no accidental interaction between the replication
  machinery and the process being replicated -- i.e. no accidental
  sharing of names used by the process to get its work done and the
  name(s) used by the replication to effect copying. This latter
  revision of the definition of replication is crucial to obtaining
  the expected identity $!!P \sim !P$.
\end{remark}

\begin{remark}\label{rem:paradoxical_combinator}
  The reader familiar with the lambda calculus will have noticed the
  similarity between $D$ and the paradoxical combinator.

  [Ed. note: the existence of this seems to suggest we have to be more
  restrictive on the set of processes and names we admit if we are to
  support no-cloning.]
\end{remark}

\subsubsection{Bisimulation}

The computational dynamics gives rise to another kind of equivalence,
the equivalence of computational behavior. As previously mentioned
this is typically captured \emph{via} some form of bisimulation.

% The notion we use in this paper is weak barbed bisimulation
% \cite{milner91polyadicpi}.

The notion we use in this paper is derived from weak barbed
bisimulation \cite{milner91polyadicpi}. 

\begin{definition}
An \emph{observation relation}, $\downarrow_{\mathcal N}$, over a set
of names, $\mathcal N$, is the smallest relation satisfying the rules
below.

\infrule[Out-barb]{y \in {\mathcal N}, \; x \nameeq y}
		  {\outputp{x}{v} \downarrow_{\mathcal N} x}
\infrule[Par-barb]{\mbox{$P\downarrow_{\mathcal N} x$ or $Q\downarrow_{\mathcal N} x$}}
		  {\binpar{P}{Q} \downarrow_{\mathcal N} x}

We write $P \Downarrow_{\mathcal N} x$ if there is $Q$ such that 
$P \wred Q$ and $Q \downarrow_{\mathcal N} x$.
\end{definition}

\begin{definition}
%\label{def.bbisim}
An  ${\mathcal N}$-\emph{barbed bisimulation} over a set of names, ${\mathcal N}$, is a symmetric binary relation 
${\mathcal S}_{\mathcal N}$ between agents such that $P\rel{S}_{\mathcal N}Q$ implies:
\begin{enumerate}
\item If $P \red P'$ then $Q \wred Q'$ and $P'\rel{S}_{\mathcal N} Q'$.
\item If $P\downarrow_{\mathcal N} x$, then $Q\Downarrow_{\mathcal N} x$.
\end{enumerate}
$P$ is ${\mathcal N}$-barbed bisimilar to $Q$, written
$P \wbbisim_{\mathcal N} Q$, if $P \rel{S}_{\mathcal N} Q$ for some ${\mathcal N}$-barbed bisimulation ${\mathcal S}_{\mathcal N}$.
\end{definition}

$\mathcal{R} \subseteq \pi \times \pi$

$P \mathcal{R} Q => \forall P'. P \red P' \Rightarrow \exists Q'. Q \red Q', P' \mathcal{R} Q'$

$P \vdash x \Rightarrow Q \vdash x$

\begin{mathpar}
  \inferrule*[lab=Out-barb]{x \nameeq y}{{y}!\langle{Q}\rangle \vdash x}
  \and
  \inferrule*[lab=Par-barb]{\mbox{$P\vdash x$ or $Q\vdash x$}}{\binpar{P}{Q} \vdash x}
\end{mathpar}

\subsubsection{Contexts}

One of the principle advantages of computational calculi like the
$\pi$-calculus is a well-defined notion of context,
contextual-equivalence and a correlation between
contextual-equivalence and notions of bisimulation. The notion of
context allows the decomposition of a process into (sub-)process and
its syntactic environment, its context. Thus, a context may be
thought of as a process with a ``hole'' (written $\Box$) in it. The
application of a context $M$ to a process $P$, written $M[P]$, is
tantamount to filling the hole in $M$ with $P$. In this paper we do
not need the full weight of this theory, but do make use of the notion
of context in the proof the main theorem. 

\begin{mathpar}
  \inferrule* [lab=summation] {} {{M_{M},M_{N}} \bc \Box \;|\; x.M_{A} \;|\; M_{M}+M_{N}}
  \and
  \inferrule* [lab=agent] {} {{M_{A}} \bc (\vec{x})M_{P} \;| \; \clift{P_0,\ldots,M_{P},\ldots,P_N}}
  \and \\
  \inferrule* [lab=process] {} {{M_{P}} \bc M_{N} \;| \;P|M_{P} }
\end{mathpar} 

\begin{mathpar}
  \inferrule* [lab=sychronization] {} {M_{N} \bc \Box \;|\; x?M_{F} \;|\; x!M_{C}}
  \and
  \inferrule* [lab=abstraction] {} {{M_{F}} \bc (x)M_{P} }
  \and
  \inferrule* [lab=concretion] {} {{M_{C}} \bc \langle M_{P} \rangle }
  \and \\
  \inferrule* [lab=process] {} {{M_{P}} \bc M_{N} \;| \;P|M_{P} }
\end{mathpar}

\begin{definition}[contextual application] Given a context $M$, and
  process $P$, we define the \emph{contextual application}, $M[P] :=
  M\{P/\Box\}$. That is, the contextual application of M to P is the
  substitution of $P$ for $\Box$ in $M$.
\end{definition}

$\meaningof{-} : L \to \mathcal{P}(\pi)$

\begin{mathpar}
  \inferrule* [lab=collection] {} {\meaningof{true} = \pi, \and \meaningof{~E} = \pi \setminus \meaningof{E}, \and \meaningof{E_{1} \& E_{2}} = \meaningof{E_{1}} \cap \meaningof{E_{2}}}
\end{mathpar}

\begin{mathpar}
  \inferrule* [lab=structure] {} {\meaningof{0} = \{ P \in \pi | P \equiv 0 \}, \and \\ \meaningof{E_1 | E_2} = \{ P \in \pi | P \equiv P_{1} | P_{2}, P_{1} \in \meaningof{E_{1}}, P_{2} \in \meaningof{E_2}\} }
\end{mathpar}

\begin{mathpar}
 \inferrule* [lab=behavior] {} {\meaningof{\langle a?b \rangle E} = \{ P \in \pi | P \equiv Q | u?(y)P', \\ \and \\\\ \and \\ \;\;\; u \in \meaningof{a}, \forall z.P'\{z/y\} \in \meaningof{E\{z/b\}}\}, \and \\ \meaningof{a!E} = \{ P \in \pi | P \equiv Q | x!\langle P' \rangle, x \in \meaningof{a} P' \in \meaningof{E}\} }
\end{mathpar}

\begin{mathpar}
 \inferrule* [lab=nominal] {} {\meaningof{\quotep{E}} = \{ \quotep{P} \in \quotep{\pi} | P \in \meaningof{E} \}, \and \meaningof{\quotep{P}} = \{ \quotep{Q} \in \quotep{\pi} | P \equiv Q \} \and \\ \meaningof{@\quotep{E}} = \{ P \in \pi | P \equiv @x, x \in \meaningof{E} \}}
\end{mathpar}

\begin{eqnarray*}
  \\
  \meaningof{-} : TS \to ST
\end{eqnarray*}

\begin{eqnarray*}
  \\
  L : TS \to ST
\end{eqnarray*}

\begin{eqnarray*}
  \\
  P \models E \iff P \in \meaningof{E}
\end{eqnarray*}

\begin{eqnarray*}
  P \approx_{L} Q \iff \forall E \in L. P \models E \iff Q \models E
\end{eqnarray*}

\begin{eqnarray*}
  P \approx_{K} Q
\end{eqnarray*}

\begin{eqnarray*}
  P \approx Q
\end{eqnarray*}

$\approx_{K} = \approx = \approx_{L}$

\subsubsection{Contextual duality}

Note that contexts extend the quotation operation to a family of
operations from processes to names. Given a context, $M$, we can
define a \emph{nominal context}, $\quotep{M}$ by $\quotep{M}[P] :=
\quotep{M[P]}$. To foreshadow what is to come we observe that these
operations enjoy a duality with processes very much like the duality
between vectors and maps from vectors to scalars.

Further, because the calculus is essentially higher-order, we have a
correspondence between contexts and processes. More specifically,
given a name $x$ and a context $M$ we can construct $M^{*}_{x}$ such
that 

\begin{mathpar}
  M^{*}_{x} | \lift{x}{P} \red M[P]
\end{mathpar}

namely,

\begin{mathpar}
  M^{*}_{x} := x?(u).M[\dropn{u}]
\end{mathpar}

The dependence of $M^{*}_{x}$ on a name makes it an abstraction, 

\begin{mathpar}
  M^{*} := (x)x?(u).M[\dropn{u}]
\end{mathpar}

\subsection{Additional notation}

It will sometimes be convenient to denote the process a name
quotes. We already have the notation $x = \quotep{P}$, but it will be
convenient to introduce an alternate notation, $\procn{x}$, when we
want to emphasize the connection to the use of the name. Note that, by
virtue of name equivalence, $\quotep{\procn{x}} \nameeq x$; so, the
notation is consistent with previous definitions.

Further, because names have structure it is possible to effect
substitutions on the basis of that structure. This means we need to
upgrade our notation for substitutions, which we accomplish by
adapting comprehension notation. Thus,

\begin{mathpar}
  P\{ y / x : x \in S \}
\end{mathpar}

is interpreted to mean the process derived from P by replacing (in a
capture-avoiding manner) each occurrence of $x$ in $S$ by $y$. For example,

\begin{mathpar}
  P\{ \quotep{\procn{x}|\procn{x}} / x : x \in \freenames{P} \}
\end{mathpar}

will replace each (occurrence) of a free name $x$ in $P$ by
$\quotep{\procn{x}|\procn{x}}$.

Also, we will avail ourselves of the notation $x^{L}$ and $x^{R}$ to
denote injections of a name into disjoint copies of the name
space. There are numerous ways to accomplish this. One example can be
found in \cite{MeredithR05}. This notation overloads to vectors of
names: $\vec{x}^{\pi} := (x_{i}^{\pi} \; : \; 0 \leq i < |\vec{x}| )$ where $\pi \in \{L,R\}$.

We also use $P^{\Box} := P|\Box$.

In \cite{MeredithR05} an interpretation of the new operator is
given. It turns out that there are several possible interpretations
all enjoying the requisite algebraic properties of the operator (see
\cite{milner91polyadicpi}). We will therefore make liberal use of
$(\nu\; \vec{x})P$.

% subsection the_syntax_and_semantics_of_the_notation_system (end)   

\input{qm2pi.qmops} 

\input{qm2pi.sterngerlach} 

\input{qm2pi.metric} 

% section concurrent_process_calculi (end)

%\input{qm2pi.proofsketch}

% section proof sketch (end)

%\input{qm2pi.slviaknots} 

% section spatial logic via knots (end)

\input{qm2pi.conclusion}

% section conclusion (end)

%\input{qm2pi.dtcodes} 

% section wiring algorithm (end)

\input{qm2pi.ack} 

% section acknowledgments (end)

\newpage


\bibliographystyle{plain}   
\bibliography{../../biblios/main.bib}

\input{qm2pi.rhodetails}

\end{document}

 

%\ifpdf
%\usepackage[pdftex]{graphicx}
%\else
%\usepackage{graphicx}
%\fi

 % \ifpdf
%  \usepackage{pdfsync}
%  \if


%\title{Brief Article}
%\author{David F. Snyder}
%\author{L.G. Meredith}

%\address{Dept. of Math., Texas State University--San Marcos, San Marcos, TX 78666}
       
\pagestyle{empty}


\begin{document}

\lstset{language=[Objective]Caml,frame=shadowbox}

\documentclass[12pt]{llncs}
%\documentclass{jktr}

\usepackage[pdftex]{hyperref}                   
\usepackage {listings}
\usepackage {mathpartir}
\usepackage{bcprules}
%\usepackage{listings}
                       
\usepackage{graphicx} 
%\usepackage[margins=2.5cm,nohead,nofoot]{geometry}
%\usepackage{geometry}
\usepackage{amsfonts}
\usepackage{amstext}
\usepackage{latexsym}
\usepackage{amssymb}
\usepackage{color}


%\include{myPreamble}
\include{qm2pi.local} 

%\ifpdf
%\usepackage[pdftex]{graphicx}
%\else
%\usepackage{graphicx}
%\fi

 % \ifpdf
%  \usepackage{pdfsync}
%  \if


%\title{Brief Article}
%\author{David F. Snyder}
%\author{L.G. Meredith}

%\address{Dept. of Math., Texas State University--San Marcos, San Marcos, TX 78666}
       
\pagestyle{empty}


\begin{document}

\lstset{language=[Objective]Caml,frame=shadowbox}

\input{qm2pi.front}

% section front matter (end)

\input{qm2pi.intro} 
 
% section introduction (end)

% \input{qm2pi.knotations} 

% section notation (end)

\input{qm2pi.process.calculi} 

% section concurrent_process_calculi_and_spatial_logics_ (end)
    
%\input{qm2pi.knots2pi} 

%\input{qm2pi.trefoil} 

%\input{qm2pi.mainthm} 

% subsection basic_interpretation (end)

%\input{qm2pi.rho.presentation} 
\subsection{The syntax and semantics of the notation system}\label{sub:the_syntax_and_semantics_of_the_notation_system} % (fold)

We now summarize a technical presentation of the calculus that
embodies our theory of dynamics. The typical presentation of such a
calculus follows the style of giving generators and relations on
them. The grammar, below, describing term constructors, freely
generates the set of processes, $\Proc$. This set is then quotiented
by a relation known as structural congruence and it is over this set
that the notion of dynamics is expressed. This presentation is
essentially that of \cite{MeredithR05} with the addition of
polyadicity and summation. For readability we have relegated some of
the technical subtleties to an appendix.

\subsubsection{Process grammar}\label{subsub:process_grammar}

\begin{mathpar}
  \inferrule* [lab=synchronization] {} {{M} \bc \pzero \;|\; x?F \;|\; x!C }
  \and
  \inferrule* [lab=abstraction] {} {{F} \bc (x)P}
  \and
  \inferrule* [lab=concretion] {} {{C} \bc \langle Q \rangle}
  \and
  \inferrule* [lab=process] {} {{P,Q} \bc M \;| \;P|Q \;|\; @{x}}
  \and
  \inferrule* [lab=name] {} {{x} \bc \quotep{P}}
\end{mathpar} 

Note that $\vec{x}$ (resp. $\vec{P}$) denotes a vector of names
(resp. processes) of length $|\vec{x}|$ (resp. $|\vec{P}|$). We adopt
the following useful abbreviations.

\begin{mathpar}
   x?(\vec{y}).P := x.(\vec{y})P \and  x\clift{\vec{P}} := x.\clift{\vec{P}}
   \and x!(y) := \lift{x}{\dropn{y}}
   \and \Pi_{i=0}^{n-1}P_i := P_0 | \ldots | P_{n-1}
\end{mathpar}

\subsubsection{Structural congruence}

\paragraph{Free and bound names and alpha-equivalence.} At the
core of structural equivalence is alpha-equivalence which identifies
process that are the same up to a change of variable. Formally, we
recognize the distinction between free and bound names. The free names
of a process, $\freenames{P}$, may be calculated recursively as
follows:

\begin{mathpar}
\freenames{\pzero} := \emptyset
  \and \\
  \freenames{x?(y).P} := \{ x \} \cup (\freenames{P} \setminus \{ y \})
  \and 
  \freenames{x!\langle P \rangle} := \{ x \} \cup \{ P \} 
  \and \\
  \freenames{P|Q} := \freenames{P} \cup \freenames{Q}
  \and \\
  \freenames{@{x}} := \{ x \}
\end{mathpar}

$\pi$
$\quotep{\pi}$

$\freenames{-} : \pi \to \mathcal{P}(\quotep{\pi})$

\begin{eqnarray*}
  \freenames{\pzero} & := & \emptyset \\
  \freenames{x?(y).P} & := & \{ x \} \cup (\freenames{P} \setminus \{ y \}) \\
  \freenames{x!\langle P \rangle} & := & \{ x \} \cup \{ P \} \\
  \freenames{P|Q} & := & \freenames{P} \cup \freenames{Q} \\
  \freenames{\dropn{x}} & := & \{ x \}
\end{eqnarray*}

The bound names of a process, $\boundnames{P}$, are those names occurring in $P$
that are not free. For example, in $x?(y).0$, the name $x$ is free, while $y$ is bound.

\begin{mathpar}
  \inferrule* [lab=monoidal-laws] {} { P|Q \equiv Q|P \and P|0 \equiv P \and P|(Q|R) \equiv (P|Q)|R }
\end{mathpar}

\begin{mathpar}
  \inferrule* [lab=alpha-equivalence] {} { (x)P \equiv (y)P\{y/x\} \and y \not\in \freenames{P} }
\end{mathpar}

\begin{definition}
Then two processes, $P,Q$, are alpha-equivalent if $P = Q\{\vec{y}/\vec{x}\}$ for
some $\vec{x} \in \boundnames{Q},\vec{y} \in \boundnames{P}$, where $Q\{\vec{y}/\vec{x}\}$
denotes the capture-avoiding substitution of $\vec{y}$ for $\vec{x}$ in $Q$.
\end{definition}

\begin{definition}
  The {\em structural congruence} \cite{SangiorgiWalker} , $\equiv$,
  between processes is the least congruence containing
  alpha-equivalence, satisfying the abelian monoid laws
  (associativity, commutativity and $\pzero$ as identity) for parallel
  composition $|$ and for summation $+$.
\end{definition}

\subsection{Name equivalence}

We take name equivalence, written $\nameeq$, to be the smallest
equivalence relation generated by the following rules.

\begin{mathpar}
\inferrule*[lab=Quote-drop]
{ }
{ \quotep{@{x}} \nameeq x }

\inferrule*[lab=Struct-equiv]
{ P \scong Q }
{ \quotep{P} \nameeq \quotep{Q} }
\end{mathpar}

The astute reader will have noticed that the mutual recursion of names
and processes imposes a mutual recursion on alpha-equivalence and
structural equivalence via name-equivalence. Fortunately, all of this
works out pleasantly and we may calculate in the natural way, free of
concern. The reader interested in the details is referred to the
appendix \ref{appendix:rho_details}.

\subsection{Substitution}

We use $\Proc$ for the set of processes, $\QProc$ for the set of
names, and $\id{\{}\vec{y} / \vec{x} \id{\}}$ to denote partial maps,
$s : \QProc \rightarrow \QProc$. A map, $s$ lifts, uniquely, to a map
on process terms, $\widehat{s} : \Proc \rightarrow \Proc$ by the
following equations.

\begin{mathpar}
  (0) \psubstp{Q}{P} := 0 \\
  (R \juxtap S) \psubstp{Q}{P}
  :=    
  (R)\psubstp{Q}{P} \juxtap (S) \psubstp{Q}{P} \\
  (x?(y).R) \psubstp{Q}{P}    
  :=    
  (x)\substp{Q}{P} (z)\concat( (R \psubstn{z}{y}) \psubstp{Q}{P} ) \\
  (\lift{x}{R}) \psubstp{Q}{P}  
  :=
  \lift{(x)\substp{Q}{P}}{ R \psubstp{Q}{P} } \\
%   (\dropn{x})  \psubstp{Q}{P}       
%   := 
%   \left\{ 
%     \begin{array}{ccc} 
%       \dropn{\quotep{Q}} & & x \nameeq \quotep{P} \\
%       \dropn{x} & & otherwise \\
%     \end{array}
%   \right. 
  (\dropn{x})  \psubstp{Q}{P}       
  := 
  \left\{ 
    \begin{array}{ccc} 
      Q & & x \nameeq \quotep{P} \\
      \dropn{x} & & otherwise \\
    \end{array}
  \right.
\end{mathpar}
 

where

\begin{eqnarray}
  (x)\id{\{} \lpquote Q \rpquote / \lpquote P \rpquote \id{\}}            = 
  \left\{ 
    \begin{array}{ccc}
      \lpquote Q \rpquote & & x \nameeq \lpquote P \rpquote \\
      x & & otherwise \\
    \end{array}
  \right. \nonumber
\end{eqnarray}

and $z$ is chosen distinct from $\quotep{P}$, $\quotep{Q}$, the free
names in $Q$, and all the names in $R$. Our $\alpha$-equivalence will
be built in the standard way from this substitution.

\begin{remark}\label{rem:no_self_referential_names}
  One consequence of these definitions is that $\forall P. \quotep{P}
  \not\in \freenames{P}$.
\end{remark}

\subsection{ Dynamic quote: an example }

Anticipating something of what's to come, consider applying the
substitution, $\widehat{\id{\{}u / z \id{\}}}$, to the following pair
of processes, $\lift{w}{y!(z)}$ and $w[ \lpquote y!(z) \rpquote ]$.

\begin{eqnarray}
	\lift{w}{y!(z)}\widehat{\id{\{}u / z \id{\}}}
		& = &
		\lift{w}{y!(u)} \nonumber\\
	w[ \lpquote y!(z) \rpquote ] \widehat{ \id{\{}u / z \id{\}} }
		& = &
		w[ \lpquote y!(z) \rpquote ] \nonumber
\end{eqnarray}

Because the body of the process between quotes is impervious to
substitution, we get radically different answers. In fact, by
examining the first process in an input context,
e.g. $x?(z).\lift{w}{y!(z)}$, we see that the process under the lift
operator may be shaped by prefixed inputs binding a name inside it. In
this sense, the lift operator will be seen as a way to dynamically
construct processes before reifying them as names.

Finally equipped with these standard features we can present the
dynamics of the calculus.

\subsubsection{Operational semantics} 

Finally, we introduce the computational dynamics. What marks these
algebras as distinct from other more traditionally studied algebraic
structures, e.g. vector spaces or polynomial rings, is the manner in
which dynamics is captured. In traditional structures, dynamics is typically
expressed through morphisms between such structures, as in linear maps
between vector spaces or morphisms between rings. In algebras
associated with the semantics of computation, the dynamics is
expressed as part of the algebraic structure itself, through a
reduction reduction relation typically denoted by $\red$. Below, we
give a recursive presentation of this relation for the calculus used
in the encoding.

$\red \subseteq \pi \times \pi$
$\red : \pi \to \mathcal{P}(\pi)$

\begin{mathpar}
  \inferrule* [lab=Comm] { \textsf{match}( x_{src}, x_{trgt} ) } { x_{trgt}?(y)P \; | \; x_{src}!\langle {Q} \rangle \red P\{\quotep{Q}/y}\} }
  \and \\
  \inferrule* [lab=Par] {{P} \red {P}'} {{{P} | {Q}} \red {{P}' | {Q}}}
  \and
  \inferrule* [lab=Equiv]{{{P} \scong {P}'} \andalso {{P}' \red {Q}'} \andalso {{Q}' \scong {Q}}}{{P} \red {Q}}
\end{mathpar}

\begin{eqnarray*}
  match_{\equiv} (\quotep{P},\quotep{Q}) & := & P \equiv Q \\
  match_{\dagger}(\quotep{P},\quotep{Q}) & := & \forall R. P|Q \red^{*} R => R \red^{*} 0 \\
  match_{K}(\quotep{P},\quotep{Q}) & := & K \mbox{ for some context } K
\end{eqnarray*}

$u?(x)P | u!\langle Q \rangle \red P\{\quotep{Q}/x\}$

%We write $\wred$ for $\red^*$, and $P\red$ if $\exists Q $ such that $ P \red Q$.
We write $P\red$ if $\exists Q $ such that $ P \red Q$ and $P\not\red$, otherwise.

\section{Replication}

As mentioned before, it is known that replication (and hence
recursion) can be implemented in a higher-order process algebra
\cite{SangiorgiWalker}. As our first example of calculation with the
machinery thus far presented we give the construction explicitly in
the {\rhoc}.

\begin{eqnarray}
	D_{x} & := & \prefix{x}{y}{(\binpar{\outputp{x}{y}}{@{y}})} \nonumber\\
	\bangp_{x}{P} & := & \binpar{{x}!\langle{\binpar{D_{x}}{P}}\rangle}{D_{x}} \nonumber
\end{eqnarray}

\begin{eqnarray}
	\bangp_{x}{P} & & \nonumber\\
	=
	& {x}!\langle{(\prefix{x}{y}{(\outputp{x}{y} | @{y})) | P}}\rangle 
	      | \prefix{x}{y}{(\outputp{x}{y} | @{y})} & \nonumber\\
	\red
	& (\outputp{x}{y} | @{y})\substn{\quotep{(\prefix{x}{y}{(@{y} | \outputp{x}{y})) | P}}}{y} & \nonumber\\
	=
	& \outputp{x}{\quotep{(\prefix{x}{y}{(\outputp{x}{y} | @{y})) | P}}}
	  | {(\prefix{x}{y}{(\outputp{x}{y} | @{y})) | P}} & \nonumber\\
	\red
	& \ldots & \nonumber\\
	\red^*
	& P | P | \ldots & \nonumber
\end{eqnarray}

Of course, this encoding, as an implementation, runs away, unfolding
$\bangp{P}$ eagerly. A lazier and more implementable replication
operator, restricted to input-guarded processes, may be obtained as follows.

\begin{eqnarray}
\bangp{\prefix{u}{v}{P}} 
	:= 
	\binpar{\lift{x}{\prefix{u}{v}{(\binpar{D(x)}{P})}}}{D(x)} \nonumber
\end{eqnarray}

\begin{remark}
  Note that the lazier definition still does not deal with summation
  or mixed summation (i.e. sums over input and output). The reader is
  invited to construct definitions of replication that deal with these
  features. 

  Further, the definitions are parameterized in a name, $x$. Can you,
  gentle reader, make a definition that eliminates this parameter and
  guarantees no accidental interaction between the replication
  machinery and the process being replicated -- i.e. no accidental
  sharing of names used by the process to get its work done and the
  name(s) used by the replication to effect copying. This latter
  revision of the definition of replication is crucial to obtaining
  the expected identity $!!P \sim !P$.
\end{remark}

\begin{remark}\label{rem:paradoxical_combinator}
  The reader familiar with the lambda calculus will have noticed the
  similarity between $D$ and the paradoxical combinator.

  [Ed. note: the existence of this seems to suggest we have to be more
  restrictive on the set of processes and names we admit if we are to
  support no-cloning.]
\end{remark}

\subsubsection{Bisimulation}

The computational dynamics gives rise to another kind of equivalence,
the equivalence of computational behavior. As previously mentioned
this is typically captured \emph{via} some form of bisimulation.

% The notion we use in this paper is weak barbed bisimulation
% \cite{milner91polyadicpi}.

The notion we use in this paper is derived from weak barbed
bisimulation \cite{milner91polyadicpi}. 

\begin{definition}
An \emph{observation relation}, $\downarrow_{\mathcal N}$, over a set
of names, $\mathcal N$, is the smallest relation satisfying the rules
below.

\infrule[Out-barb]{y \in {\mathcal N}, \; x \nameeq y}
		  {\outputp{x}{v} \downarrow_{\mathcal N} x}
\infrule[Par-barb]{\mbox{$P\downarrow_{\mathcal N} x$ or $Q\downarrow_{\mathcal N} x$}}
		  {\binpar{P}{Q} \downarrow_{\mathcal N} x}

We write $P \Downarrow_{\mathcal N} x$ if there is $Q$ such that 
$P \wred Q$ and $Q \downarrow_{\mathcal N} x$.
\end{definition}

\begin{definition}
%\label{def.bbisim}
An  ${\mathcal N}$-\emph{barbed bisimulation} over a set of names, ${\mathcal N}$, is a symmetric binary relation 
${\mathcal S}_{\mathcal N}$ between agents such that $P\rel{S}_{\mathcal N}Q$ implies:
\begin{enumerate}
\item If $P \red P'$ then $Q \wred Q'$ and $P'\rel{S}_{\mathcal N} Q'$.
\item If $P\downarrow_{\mathcal N} x$, then $Q\Downarrow_{\mathcal N} x$.
\end{enumerate}
$P$ is ${\mathcal N}$-barbed bisimilar to $Q$, written
$P \wbbisim_{\mathcal N} Q$, if $P \rel{S}_{\mathcal N} Q$ for some ${\mathcal N}$-barbed bisimulation ${\mathcal S}_{\mathcal N}$.
\end{definition}

$\mathcal{R} \subseteq \pi \times \pi$

$P \mathcal{R} Q => \forall P'. P \red P' \Rightarrow \exists Q'. Q \red Q', P' \mathcal{R} Q'$

$P \vdash x \Rightarrow Q \vdash x$

\begin{mathpar}
  \inferrule*[lab=Out-barb]{x \nameeq y}{{y}!\langle{Q}\rangle \vdash x}
  \and
  \inferrule*[lab=Par-barb]{\mbox{$P\vdash x$ or $Q\vdash x$}}{\binpar{P}{Q} \vdash x}
\end{mathpar}

\subsubsection{Contexts}

One of the principle advantages of computational calculi like the
$\pi$-calculus is a well-defined notion of context,
contextual-equivalence and a correlation between
contextual-equivalence and notions of bisimulation. The notion of
context allows the decomposition of a process into (sub-)process and
its syntactic environment, its context. Thus, a context may be
thought of as a process with a ``hole'' (written $\Box$) in it. The
application of a context $M$ to a process $P$, written $M[P]$, is
tantamount to filling the hole in $M$ with $P$. In this paper we do
not need the full weight of this theory, but do make use of the notion
of context in the proof the main theorem. 

\begin{mathpar}
  \inferrule* [lab=summation] {} {{M_{M},M_{N}} \bc \Box \;|\; x.M_{A} \;|\; M_{M}+M_{N}}
  \and
  \inferrule* [lab=agent] {} {{M_{A}} \bc (\vec{x})M_{P} \;| \; \clift{P_0,\ldots,M_{P},\ldots,P_N}}
  \and \\
  \inferrule* [lab=process] {} {{M_{P}} \bc M_{N} \;| \;P|M_{P} }
\end{mathpar} 

\begin{mathpar}
  \inferrule* [lab=sychronization] {} {M_{N} \bc \Box \;|\; x?M_{F} \;|\; x!M_{C}}
  \and
  \inferrule* [lab=abstraction] {} {{M_{F}} \bc (x)M_{P} }
  \and
  \inferrule* [lab=concretion] {} {{M_{C}} \bc \langle M_{P} \rangle }
  \and \\
  \inferrule* [lab=process] {} {{M_{P}} \bc M_{N} \;| \;P|M_{P} }
\end{mathpar}

\begin{definition}[contextual application] Given a context $M$, and
  process $P$, we define the \emph{contextual application}, $M[P] :=
  M\{P/\Box\}$. That is, the contextual application of M to P is the
  substitution of $P$ for $\Box$ in $M$.
\end{definition}

$\meaningof{-} : L \to \mathcal{P}(\pi)$

\begin{mathpar}
  \inferrule* [lab=collection] {} {\meaningof{true} = \pi, \and \meaningof{~E} = \pi \setminus \meaningof{E}, \and \meaningof{E_{1} \& E_{2}} = \meaningof{E_{1}} \cap \meaningof{E_{2}}}
\end{mathpar}

\begin{mathpar}
  \inferrule* [lab=structure] {} {\meaningof{0} = \{ P \in \pi | P \equiv 0 \}, \and \\ \meaningof{E_1 | E_2} = \{ P \in \pi | P \equiv P_{1} | P_{2}, P_{1} \in \meaningof{E_{1}}, P_{2} \in \meaningof{E_2}\} }
\end{mathpar}

\begin{mathpar}
 \inferrule* [lab=behavior] {} {\meaningof{\langle a?b \rangle E} = \{ P \in \pi | P \equiv Q | u?(y)P', \\ \and \\\\ \and \\ \;\;\; u \in \meaningof{a}, \forall z.P'\{z/y\} \in \meaningof{E\{z/b\}}\}, \and \\ \meaningof{a!E} = \{ P \in \pi | P \equiv Q | x!\langle P' \rangle, x \in \meaningof{a} P' \in \meaningof{E}\} }
\end{mathpar}

\begin{mathpar}
 \inferrule* [lab=nominal] {} {\meaningof{\quotep{E}} = \{ \quotep{P} \in \quotep{\pi} | P \in \meaningof{E} \}, \and \meaningof{\quotep{P}} = \{ \quotep{Q} \in \quotep{\pi} | P \equiv Q \} \and \\ \meaningof{@\quotep{E}} = \{ P \in \pi | P \equiv @x, x \in \meaningof{E} \}}
\end{mathpar}

\begin{eqnarray*}
  \\
  \meaningof{-} : TS \to ST
\end{eqnarray*}

\begin{eqnarray*}
  \\
  L : TS \to ST
\end{eqnarray*}

\begin{eqnarray*}
  \\
  P \models E \iff P \in \meaningof{E}
\end{eqnarray*}

\begin{eqnarray*}
  P \approx_{L} Q \iff \forall E \in L. P \models E \iff Q \models E
\end{eqnarray*}

\begin{eqnarray*}
  P \approx_{K} Q
\end{eqnarray*}

\begin{eqnarray*}
  P \approx Q
\end{eqnarray*}

$\approx_{K} = \approx = \approx_{L}$

\subsubsection{Contextual duality}

Note that contexts extend the quotation operation to a family of
operations from processes to names. Given a context, $M$, we can
define a \emph{nominal context}, $\quotep{M}$ by $\quotep{M}[P] :=
\quotep{M[P]}$. To foreshadow what is to come we observe that these
operations enjoy a duality with processes very much like the duality
between vectors and maps from vectors to scalars.

Further, because the calculus is essentially higher-order, we have a
correspondence between contexts and processes. More specifically,
given a name $x$ and a context $M$ we can construct $M^{*}_{x}$ such
that 

\begin{mathpar}
  M^{*}_{x} | \lift{x}{P} \red M[P]
\end{mathpar}

namely,

\begin{mathpar}
  M^{*}_{x} := x?(u).M[\dropn{u}]
\end{mathpar}

The dependence of $M^{*}_{x}$ on a name makes it an abstraction, 

\begin{mathpar}
  M^{*} := (x)x?(u).M[\dropn{u}]
\end{mathpar}

\subsection{Additional notation}

It will sometimes be convenient to denote the process a name
quotes. We already have the notation $x = \quotep{P}$, but it will be
convenient to introduce an alternate notation, $\procn{x}$, when we
want to emphasize the connection to the use of the name. Note that, by
virtue of name equivalence, $\quotep{\procn{x}} \nameeq x$; so, the
notation is consistent with previous definitions.

Further, because names have structure it is possible to effect
substitutions on the basis of that structure. This means we need to
upgrade our notation for substitutions, which we accomplish by
adapting comprehension notation. Thus,

\begin{mathpar}
  P\{ y / x : x \in S \}
\end{mathpar}

is interpreted to mean the process derived from P by replacing (in a
capture-avoiding manner) each occurrence of $x$ in $S$ by $y$. For example,

\begin{mathpar}
  P\{ \quotep{\procn{x}|\procn{x}} / x : x \in \freenames{P} \}
\end{mathpar}

will replace each (occurrence) of a free name $x$ in $P$ by
$\quotep{\procn{x}|\procn{x}}$.

Also, we will avail ourselves of the notation $x^{L}$ and $x^{R}$ to
denote injections of a name into disjoint copies of the name
space. There are numerous ways to accomplish this. One example can be
found in \cite{MeredithR05}. This notation overloads to vectors of
names: $\vec{x}^{\pi} := (x_{i}^{\pi} \; : \; 0 \leq i < |\vec{x}| )$ where $\pi \in \{L,R\}$.

We also use $P^{\Box} := P|\Box$.

In \cite{MeredithR05} an interpretation of the new operator is
given. It turns out that there are several possible interpretations
all enjoying the requisite algebraic properties of the operator (see
\cite{milner91polyadicpi}). We will therefore make liberal use of
$(\nu\; \vec{x})P$.

% subsection the_syntax_and_semantics_of_the_notation_system (end)   

\input{qm2pi.qmops} 

\input{qm2pi.sterngerlach} 

\input{qm2pi.metric} 

% section concurrent_process_calculi (end)

%\input{qm2pi.proofsketch}

% section proof sketch (end)

%\input{qm2pi.slviaknots} 

% section spatial logic via knots (end)

\input{qm2pi.conclusion}

% section conclusion (end)

%\input{qm2pi.dtcodes} 

% section wiring algorithm (end)

\input{qm2pi.ack} 

% section acknowledgments (end)

\newpage


\bibliographystyle{plain}   
\bibliography{../../biblios/main.bib}

\input{qm2pi.rhodetails}

\end{document}



% section front matter (end)

\section{Introduction}\label{sec:introduction} % (fold)
In this draft of the material i am going to have to dispense with the
usual writing conventions adopted in papers on these topics. i'm going
to have adopt whatever tone i need at the time i'm writing up the
calculations. Sometimes this may be very conversational; others it may
be the barest mathematical grunts; others still it may be that i have
lifted text from one of my other papers because the exposition of some
point was better said there. i hope that my readers are not unduly put
out by this decision. i'm not doing this to flout convention or be
rebellious. i find these calculations very technically challenging. To
keep everything going technically, something has to give; i have to
let go of some cognitive burden. So, the academic writing style --
with all of its trade-offs in terms of facilitating technical
communication -- is what i'm letting go of. Perhaps subsequent drafts
can be tightened and polished, but for now, i'm going to speak as if
we were sitting together in a coffee shop with a laptop, wifi and a
pad of paper and a pencil.

So, here's what i have to say. We -- you and i, comfortably ensconced
in our coffee shop and well-equipped with our tools -- can realize and
carry out the calculations of quantum mechanics over a very different
formal theory of dynamics, a formal theory of dynamics that
corresponds to a theory of concurrent computation with
\emph{reflection}. It has the advantage that the underlying theory is
already `quantized', but supports analogues all of the continuuous
operations. Strikingly, this underlying theory has recently been
connected with a notion of metric that we can show, by calculating
together, coincides with the metric induced by the inner product.

There are a lot of reasons why you might be interested in seeing
calculations of this form. Here's why i'm interested. For the past
several centuries there has been no competitor to the ``Newtonian''
account of dynamics. As a result the predominant share of accounts of
dynamical systems and situations have had to be formulated in terms of
the Newtonian machinery. i view this as an intellectually dangerous
position to occupy. Everything, despite it's intrinsic shape, turns
into a nail to be hit with this hammer. Recently, however, the theory
of computation has matured to the point where we have candidates for
theories of dynamics that offer very different perspective on
reasoning about dynamical systems and situations. Testing these
candidates against very successful accounts of dynamical situations,
like quantum mechanics, is going to give us some sense of how mature
they are and some measure of the quality of these accounts of
dynamics.

\subsection{Summary of contributions and outline of paper}

So, we're going to develop an interpretation of the operations of
quantum mechanics normally interpreted by Hilbert spaces and
operators. We're going to do this over a theory of computation. Note
that this is very different than the usual quantum computation program
which develops notions of computation over quantum mechanics. Rather,
we are developing a story that aligns with Wheeler's slogan: It from
Bit. To do this we will first provide an account of the theory of
computation at play here. Then we will dive into a calculation-driven
interpretation of the operations of quantum mechanics.

The reason we take this approach is that -- until very recently --
there hasn't been an axiomatic account of quantum mechanics. As a
result there has been no sharp delineation of the mathematical theory
supporting interpretation of the physical theory and the physical
theory, itself. So, ambient features of the maths are free to be
exploited (or supressed) without a real accounting of their physical
relevance. There is no sharp statement ``here's the physical theory''
qua \emph{theory} and ``here's the mathematical interpretation''
enabling a judgment of how faithful the interpretation is -- apart
from experimental observation. When there is an axiomatic account we
can judge how well a given mathematical formalism supports an
interpretation of the axioms, independent of
experimentation. Likewise, we can judge how well we have captured our
physical evidence and experience with our axiomatics, independent of
any specific mathematical implementation, with accidental detail that
may or may not have physical significance. 

In lieu of a fully fleshed out and vetted axiomatic account of quantum
mechanics, interpreting the operational notions in service of modeling
physical systems will have to suffice. In other words, we are not in
the business of providing a model of Hilbert spaces and operators. We
are in the business of providing a model of quantum mechanics because
we are motivated by testing our notions of dynamics against physical
theory; and, the predictive calculations of the physical theory must
serve as the best formulation -- shy of a fully fleshed out axiomatic
account -- of the physical theory itself (as they have for scientific
theories since time immemorial). Put another way, despite a
whole-hearted commitment to an It-from-Bit ontology, we are firmly
aligned with the shut-up-and-calculate camp as the best way to obtain
results either from the physical perspective or as a quality assurance
measure of our fledgling theory of dynamics.

In detail, we present a reflective process calculus. Then we develop
intuitive correspondences between the notions available in this
calculus and the usual physical notions supporting quantum mechanical
calculations. Thus, 

\begin{table}[htp]
  \center{
    \fbox{
      \begin{tabular}{c|c}
        quantum mechanics & process calculus \\
        \hline
        scalar & name \\
        state vector & process \\
        dual & contextual duals \\
        matrix & formal sums of process-context-dual pairs \\
        orthogonality & process annihilation \\
        inner product & execution-formula + quoting
      \end{tabular}
    }
  }
  \caption{QM - process calculi correspondences}
\end{table}

Then we tighten up these intuitions to operational definitions. We
employ the Dirac notation as the best proxy we can find for an
abstract syntax of the quantum mechanical notions. The definitions we
develop put us in contact with equational constraints coming from the
theory that we demonstrate the definitions and calculations satisfy.

This puts us in a position to shut up and calculate for the
Stern-Gerlach experimental set up, showing how these predictive
calculations become calculations on processes in our theory of a
reflective process calculus.

Penultimately, we demonstrate that the notion of metric coming from
the inner product coincides with the notion of metric available from
the theory of bisimulation. This demonstration gives us the right to
think of space as arising from behavior. Finally, we consider where we
might go from the new vantage point we have obtained.

% section introduction (end) 
 
% section introduction (end)

% \documentclass[12pt]{llncs}
%\documentclass{jktr}

\usepackage[pdftex]{hyperref}                   
\usepackage {listings}
\usepackage {mathpartir}
\usepackage{bcprules}
%\usepackage{listings}
                       
\usepackage{graphicx} 
%\usepackage[margins=2.5cm,nohead,nofoot]{geometry}
%\usepackage{geometry}
\usepackage{amsfonts}
\usepackage{amstext}
\usepackage{latexsym}
\usepackage{amssymb}
\usepackage{color}


%\include{myPreamble}
\include{qm2pi.local} 

%\ifpdf
%\usepackage[pdftex]{graphicx}
%\else
%\usepackage{graphicx}
%\fi

 % \ifpdf
%  \usepackage{pdfsync}
%  \if


%\title{Brief Article}
%\author{David F. Snyder}
%\author{L.G. Meredith}

%\address{Dept. of Math., Texas State University--San Marcos, San Marcos, TX 78666}
       
\pagestyle{empty}


\begin{document}

\lstset{language=[Objective]Caml,frame=shadowbox}

\input{qm2pi.front}

% section front matter (end)

\input{qm2pi.intro} 
 
% section introduction (end)

% \input{qm2pi.knotations} 

% section notation (end)

\input{qm2pi.process.calculi} 

% section concurrent_process_calculi_and_spatial_logics_ (end)
    
%\input{qm2pi.knots2pi} 

%\input{qm2pi.trefoil} 

%\input{qm2pi.mainthm} 

% subsection basic_interpretation (end)

%\input{qm2pi.rho.presentation} 
\subsection{The syntax and semantics of the notation system}\label{sub:the_syntax_and_semantics_of_the_notation_system} % (fold)

We now summarize a technical presentation of the calculus that
embodies our theory of dynamics. The typical presentation of such a
calculus follows the style of giving generators and relations on
them. The grammar, below, describing term constructors, freely
generates the set of processes, $\Proc$. This set is then quotiented
by a relation known as structural congruence and it is over this set
that the notion of dynamics is expressed. This presentation is
essentially that of \cite{MeredithR05} with the addition of
polyadicity and summation. For readability we have relegated some of
the technical subtleties to an appendix.

\subsubsection{Process grammar}\label{subsub:process_grammar}

\begin{mathpar}
  \inferrule* [lab=synchronization] {} {{M} \bc \pzero \;|\; x?F \;|\; x!C }
  \and
  \inferrule* [lab=abstraction] {} {{F} \bc (x)P}
  \and
  \inferrule* [lab=concretion] {} {{C} \bc \langle Q \rangle}
  \and
  \inferrule* [lab=process] {} {{P,Q} \bc M \;| \;P|Q \;|\; @{x}}
  \and
  \inferrule* [lab=name] {} {{x} \bc \quotep{P}}
\end{mathpar} 

Note that $\vec{x}$ (resp. $\vec{P}$) denotes a vector of names
(resp. processes) of length $|\vec{x}|$ (resp. $|\vec{P}|$). We adopt
the following useful abbreviations.

\begin{mathpar}
   x?(\vec{y}).P := x.(\vec{y})P \and  x\clift{\vec{P}} := x.\clift{\vec{P}}
   \and x!(y) := \lift{x}{\dropn{y}}
   \and \Pi_{i=0}^{n-1}P_i := P_0 | \ldots | P_{n-1}
\end{mathpar}

\subsubsection{Structural congruence}

\paragraph{Free and bound names and alpha-equivalence.} At the
core of structural equivalence is alpha-equivalence which identifies
process that are the same up to a change of variable. Formally, we
recognize the distinction between free and bound names. The free names
of a process, $\freenames{P}$, may be calculated recursively as
follows:

\begin{mathpar}
\freenames{\pzero} := \emptyset
  \and \\
  \freenames{x?(y).P} := \{ x \} \cup (\freenames{P} \setminus \{ y \})
  \and 
  \freenames{x!\langle P \rangle} := \{ x \} \cup \{ P \} 
  \and \\
  \freenames{P|Q} := \freenames{P} \cup \freenames{Q}
  \and \\
  \freenames{@{x}} := \{ x \}
\end{mathpar}

$\pi$
$\quotep{\pi}$

$\freenames{-} : \pi \to \mathcal{P}(\quotep{\pi})$

\begin{eqnarray*}
  \freenames{\pzero} & := & \emptyset \\
  \freenames{x?(y).P} & := & \{ x \} \cup (\freenames{P} \setminus \{ y \}) \\
  \freenames{x!\langle P \rangle} & := & \{ x \} \cup \{ P \} \\
  \freenames{P|Q} & := & \freenames{P} \cup \freenames{Q} \\
  \freenames{\dropn{x}} & := & \{ x \}
\end{eqnarray*}

The bound names of a process, $\boundnames{P}$, are those names occurring in $P$
that are not free. For example, in $x?(y).0$, the name $x$ is free, while $y$ is bound.

\begin{mathpar}
  \inferrule* [lab=monoidal-laws] {} { P|Q \equiv Q|P \and P|0 \equiv P \and P|(Q|R) \equiv (P|Q)|R }
\end{mathpar}

\begin{mathpar}
  \inferrule* [lab=alpha-equivalence] {} { (x)P \equiv (y)P\{y/x\} \and y \not\in \freenames{P} }
\end{mathpar}

\begin{definition}
Then two processes, $P,Q$, are alpha-equivalent if $P = Q\{\vec{y}/\vec{x}\}$ for
some $\vec{x} \in \boundnames{Q},\vec{y} \in \boundnames{P}$, where $Q\{\vec{y}/\vec{x}\}$
denotes the capture-avoiding substitution of $\vec{y}$ for $\vec{x}$ in $Q$.
\end{definition}

\begin{definition}
  The {\em structural congruence} \cite{SangiorgiWalker} , $\equiv$,
  between processes is the least congruence containing
  alpha-equivalence, satisfying the abelian monoid laws
  (associativity, commutativity and $\pzero$ as identity) for parallel
  composition $|$ and for summation $+$.
\end{definition}

\subsection{Name equivalence}

We take name equivalence, written $\nameeq$, to be the smallest
equivalence relation generated by the following rules.

\begin{mathpar}
\inferrule*[lab=Quote-drop]
{ }
{ \quotep{@{x}} \nameeq x }

\inferrule*[lab=Struct-equiv]
{ P \scong Q }
{ \quotep{P} \nameeq \quotep{Q} }
\end{mathpar}

The astute reader will have noticed that the mutual recursion of names
and processes imposes a mutual recursion on alpha-equivalence and
structural equivalence via name-equivalence. Fortunately, all of this
works out pleasantly and we may calculate in the natural way, free of
concern. The reader interested in the details is referred to the
appendix \ref{appendix:rho_details}.

\subsection{Substitution}

We use $\Proc$ for the set of processes, $\QProc$ for the set of
names, and $\id{\{}\vec{y} / \vec{x} \id{\}}$ to denote partial maps,
$s : \QProc \rightarrow \QProc$. A map, $s$ lifts, uniquely, to a map
on process terms, $\widehat{s} : \Proc \rightarrow \Proc$ by the
following equations.

\begin{mathpar}
  (0) \psubstp{Q}{P} := 0 \\
  (R \juxtap S) \psubstp{Q}{P}
  :=    
  (R)\psubstp{Q}{P} \juxtap (S) \psubstp{Q}{P} \\
  (x?(y).R) \psubstp{Q}{P}    
  :=    
  (x)\substp{Q}{P} (z)\concat( (R \psubstn{z}{y}) \psubstp{Q}{P} ) \\
  (\lift{x}{R}) \psubstp{Q}{P}  
  :=
  \lift{(x)\substp{Q}{P}}{ R \psubstp{Q}{P} } \\
%   (\dropn{x})  \psubstp{Q}{P}       
%   := 
%   \left\{ 
%     \begin{array}{ccc} 
%       \dropn{\quotep{Q}} & & x \nameeq \quotep{P} \\
%       \dropn{x} & & otherwise \\
%     \end{array}
%   \right. 
  (\dropn{x})  \psubstp{Q}{P}       
  := 
  \left\{ 
    \begin{array}{ccc} 
      Q & & x \nameeq \quotep{P} \\
      \dropn{x} & & otherwise \\
    \end{array}
  \right.
\end{mathpar}
 

where

\begin{eqnarray}
  (x)\id{\{} \lpquote Q \rpquote / \lpquote P \rpquote \id{\}}            = 
  \left\{ 
    \begin{array}{ccc}
      \lpquote Q \rpquote & & x \nameeq \lpquote P \rpquote \\
      x & & otherwise \\
    \end{array}
  \right. \nonumber
\end{eqnarray}

and $z$ is chosen distinct from $\quotep{P}$, $\quotep{Q}$, the free
names in $Q$, and all the names in $R$. Our $\alpha$-equivalence will
be built in the standard way from this substitution.

\begin{remark}\label{rem:no_self_referential_names}
  One consequence of these definitions is that $\forall P. \quotep{P}
  \not\in \freenames{P}$.
\end{remark}

\subsection{ Dynamic quote: an example }

Anticipating something of what's to come, consider applying the
substitution, $\widehat{\id{\{}u / z \id{\}}}$, to the following pair
of processes, $\lift{w}{y!(z)}$ and $w[ \lpquote y!(z) \rpquote ]$.

\begin{eqnarray}
	\lift{w}{y!(z)}\widehat{\id{\{}u / z \id{\}}}
		& = &
		\lift{w}{y!(u)} \nonumber\\
	w[ \lpquote y!(z) \rpquote ] \widehat{ \id{\{}u / z \id{\}} }
		& = &
		w[ \lpquote y!(z) \rpquote ] \nonumber
\end{eqnarray}

Because the body of the process between quotes is impervious to
substitution, we get radically different answers. In fact, by
examining the first process in an input context,
e.g. $x?(z).\lift{w}{y!(z)}$, we see that the process under the lift
operator may be shaped by prefixed inputs binding a name inside it. In
this sense, the lift operator will be seen as a way to dynamically
construct processes before reifying them as names.

Finally equipped with these standard features we can present the
dynamics of the calculus.

\subsubsection{Operational semantics} 

Finally, we introduce the computational dynamics. What marks these
algebras as distinct from other more traditionally studied algebraic
structures, e.g. vector spaces or polynomial rings, is the manner in
which dynamics is captured. In traditional structures, dynamics is typically
expressed through morphisms between such structures, as in linear maps
between vector spaces or morphisms between rings. In algebras
associated with the semantics of computation, the dynamics is
expressed as part of the algebraic structure itself, through a
reduction reduction relation typically denoted by $\red$. Below, we
give a recursive presentation of this relation for the calculus used
in the encoding.

$\red \subseteq \pi \times \pi$
$\red : \pi \to \mathcal{P}(\pi)$

\begin{mathpar}
  \inferrule* [lab=Comm] { \textsf{match}( x_{src}, x_{trgt} ) } { x_{trgt}?(y)P \; | \; x_{src}!\langle {Q} \rangle \red P\{\quotep{Q}/y}\} }
  \and \\
  \inferrule* [lab=Par] {{P} \red {P}'} {{{P} | {Q}} \red {{P}' | {Q}}}
  \and
  \inferrule* [lab=Equiv]{{{P} \scong {P}'} \andalso {{P}' \red {Q}'} \andalso {{Q}' \scong {Q}}}{{P} \red {Q}}
\end{mathpar}

\begin{eqnarray*}
  match_{\equiv} (\quotep{P},\quotep{Q}) & := & P \equiv Q \\
  match_{\dagger}(\quotep{P},\quotep{Q}) & := & \forall R. P|Q \red^{*} R => R \red^{*} 0 \\
  match_{K}(\quotep{P},\quotep{Q}) & := & K \mbox{ for some context } K
\end{eqnarray*}

$u?(x)P | u!\langle Q \rangle \red P\{\quotep{Q}/x\}$

%We write $\wred$ for $\red^*$, and $P\red$ if $\exists Q $ such that $ P \red Q$.
We write $P\red$ if $\exists Q $ such that $ P \red Q$ and $P\not\red$, otherwise.

\section{Replication}

As mentioned before, it is known that replication (and hence
recursion) can be implemented in a higher-order process algebra
\cite{SangiorgiWalker}. As our first example of calculation with the
machinery thus far presented we give the construction explicitly in
the {\rhoc}.

\begin{eqnarray}
	D_{x} & := & \prefix{x}{y}{(\binpar{\outputp{x}{y}}{@{y}})} \nonumber\\
	\bangp_{x}{P} & := & \binpar{{x}!\langle{\binpar{D_{x}}{P}}\rangle}{D_{x}} \nonumber
\end{eqnarray}

\begin{eqnarray}
	\bangp_{x}{P} & & \nonumber\\
	=
	& {x}!\langle{(\prefix{x}{y}{(\outputp{x}{y} | @{y})) | P}}\rangle 
	      | \prefix{x}{y}{(\outputp{x}{y} | @{y})} & \nonumber\\
	\red
	& (\outputp{x}{y} | @{y})\substn{\quotep{(\prefix{x}{y}{(@{y} | \outputp{x}{y})) | P}}}{y} & \nonumber\\
	=
	& \outputp{x}{\quotep{(\prefix{x}{y}{(\outputp{x}{y} | @{y})) | P}}}
	  | {(\prefix{x}{y}{(\outputp{x}{y} | @{y})) | P}} & \nonumber\\
	\red
	& \ldots & \nonumber\\
	\red^*
	& P | P | \ldots & \nonumber
\end{eqnarray}

Of course, this encoding, as an implementation, runs away, unfolding
$\bangp{P}$ eagerly. A lazier and more implementable replication
operator, restricted to input-guarded processes, may be obtained as follows.

\begin{eqnarray}
\bangp{\prefix{u}{v}{P}} 
	:= 
	\binpar{\lift{x}{\prefix{u}{v}{(\binpar{D(x)}{P})}}}{D(x)} \nonumber
\end{eqnarray}

\begin{remark}
  Note that the lazier definition still does not deal with summation
  or mixed summation (i.e. sums over input and output). The reader is
  invited to construct definitions of replication that deal with these
  features. 

  Further, the definitions are parameterized in a name, $x$. Can you,
  gentle reader, make a definition that eliminates this parameter and
  guarantees no accidental interaction between the replication
  machinery and the process being replicated -- i.e. no accidental
  sharing of names used by the process to get its work done and the
  name(s) used by the replication to effect copying. This latter
  revision of the definition of replication is crucial to obtaining
  the expected identity $!!P \sim !P$.
\end{remark}

\begin{remark}\label{rem:paradoxical_combinator}
  The reader familiar with the lambda calculus will have noticed the
  similarity between $D$ and the paradoxical combinator.

  [Ed. note: the existence of this seems to suggest we have to be more
  restrictive on the set of processes and names we admit if we are to
  support no-cloning.]
\end{remark}

\subsubsection{Bisimulation}

The computational dynamics gives rise to another kind of equivalence,
the equivalence of computational behavior. As previously mentioned
this is typically captured \emph{via} some form of bisimulation.

% The notion we use in this paper is weak barbed bisimulation
% \cite{milner91polyadicpi}.

The notion we use in this paper is derived from weak barbed
bisimulation \cite{milner91polyadicpi}. 

\begin{definition}
An \emph{observation relation}, $\downarrow_{\mathcal N}$, over a set
of names, $\mathcal N$, is the smallest relation satisfying the rules
below.

\infrule[Out-barb]{y \in {\mathcal N}, \; x \nameeq y}
		  {\outputp{x}{v} \downarrow_{\mathcal N} x}
\infrule[Par-barb]{\mbox{$P\downarrow_{\mathcal N} x$ or $Q\downarrow_{\mathcal N} x$}}
		  {\binpar{P}{Q} \downarrow_{\mathcal N} x}

We write $P \Downarrow_{\mathcal N} x$ if there is $Q$ such that 
$P \wred Q$ and $Q \downarrow_{\mathcal N} x$.
\end{definition}

\begin{definition}
%\label{def.bbisim}
An  ${\mathcal N}$-\emph{barbed bisimulation} over a set of names, ${\mathcal N}$, is a symmetric binary relation 
${\mathcal S}_{\mathcal N}$ between agents such that $P\rel{S}_{\mathcal N}Q$ implies:
\begin{enumerate}
\item If $P \red P'$ then $Q \wred Q'$ and $P'\rel{S}_{\mathcal N} Q'$.
\item If $P\downarrow_{\mathcal N} x$, then $Q\Downarrow_{\mathcal N} x$.
\end{enumerate}
$P$ is ${\mathcal N}$-barbed bisimilar to $Q$, written
$P \wbbisim_{\mathcal N} Q$, if $P \rel{S}_{\mathcal N} Q$ for some ${\mathcal N}$-barbed bisimulation ${\mathcal S}_{\mathcal N}$.
\end{definition}

$\mathcal{R} \subseteq \pi \times \pi$

$P \mathcal{R} Q => \forall P'. P \red P' \Rightarrow \exists Q'. Q \red Q', P' \mathcal{R} Q'$

$P \vdash x \Rightarrow Q \vdash x$

\begin{mathpar}
  \inferrule*[lab=Out-barb]{x \nameeq y}{{y}!\langle{Q}\rangle \vdash x}
  \and
  \inferrule*[lab=Par-barb]{\mbox{$P\vdash x$ or $Q\vdash x$}}{\binpar{P}{Q} \vdash x}
\end{mathpar}

\subsubsection{Contexts}

One of the principle advantages of computational calculi like the
$\pi$-calculus is a well-defined notion of context,
contextual-equivalence and a correlation between
contextual-equivalence and notions of bisimulation. The notion of
context allows the decomposition of a process into (sub-)process and
its syntactic environment, its context. Thus, a context may be
thought of as a process with a ``hole'' (written $\Box$) in it. The
application of a context $M$ to a process $P$, written $M[P]$, is
tantamount to filling the hole in $M$ with $P$. In this paper we do
not need the full weight of this theory, but do make use of the notion
of context in the proof the main theorem. 

\begin{mathpar}
  \inferrule* [lab=summation] {} {{M_{M},M_{N}} \bc \Box \;|\; x.M_{A} \;|\; M_{M}+M_{N}}
  \and
  \inferrule* [lab=agent] {} {{M_{A}} \bc (\vec{x})M_{P} \;| \; \clift{P_0,\ldots,M_{P},\ldots,P_N}}
  \and \\
  \inferrule* [lab=process] {} {{M_{P}} \bc M_{N} \;| \;P|M_{P} }
\end{mathpar} 

\begin{mathpar}
  \inferrule* [lab=sychronization] {} {M_{N} \bc \Box \;|\; x?M_{F} \;|\; x!M_{C}}
  \and
  \inferrule* [lab=abstraction] {} {{M_{F}} \bc (x)M_{P} }
  \and
  \inferrule* [lab=concretion] {} {{M_{C}} \bc \langle M_{P} \rangle }
  \and \\
  \inferrule* [lab=process] {} {{M_{P}} \bc M_{N} \;| \;P|M_{P} }
\end{mathpar}

\begin{definition}[contextual application] Given a context $M$, and
  process $P$, we define the \emph{contextual application}, $M[P] :=
  M\{P/\Box\}$. That is, the contextual application of M to P is the
  substitution of $P$ for $\Box$ in $M$.
\end{definition}

$\meaningof{-} : L \to \mathcal{P}(\pi)$

\begin{mathpar}
  \inferrule* [lab=collection] {} {\meaningof{true} = \pi, \and \meaningof{~E} = \pi \setminus \meaningof{E}, \and \meaningof{E_{1} \& E_{2}} = \meaningof{E_{1}} \cap \meaningof{E_{2}}}
\end{mathpar}

\begin{mathpar}
  \inferrule* [lab=structure] {} {\meaningof{0} = \{ P \in \pi | P \equiv 0 \}, \and \\ \meaningof{E_1 | E_2} = \{ P \in \pi | P \equiv P_{1} | P_{2}, P_{1} \in \meaningof{E_{1}}, P_{2} \in \meaningof{E_2}\} }
\end{mathpar}

\begin{mathpar}
 \inferrule* [lab=behavior] {} {\meaningof{\langle a?b \rangle E} = \{ P \in \pi | P \equiv Q | u?(y)P', \\ \and \\\\ \and \\ \;\;\; u \in \meaningof{a}, \forall z.P'\{z/y\} \in \meaningof{E\{z/b\}}\}, \and \\ \meaningof{a!E} = \{ P \in \pi | P \equiv Q | x!\langle P' \rangle, x \in \meaningof{a} P' \in \meaningof{E}\} }
\end{mathpar}

\begin{mathpar}
 \inferrule* [lab=nominal] {} {\meaningof{\quotep{E}} = \{ \quotep{P} \in \quotep{\pi} | P \in \meaningof{E} \}, \and \meaningof{\quotep{P}} = \{ \quotep{Q} \in \quotep{\pi} | P \equiv Q \} \and \\ \meaningof{@\quotep{E}} = \{ P \in \pi | P \equiv @x, x \in \meaningof{E} \}}
\end{mathpar}

\begin{eqnarray*}
  \\
  \meaningof{-} : TS \to ST
\end{eqnarray*}

\begin{eqnarray*}
  \\
  L : TS \to ST
\end{eqnarray*}

\begin{eqnarray*}
  \\
  P \models E \iff P \in \meaningof{E}
\end{eqnarray*}

\begin{eqnarray*}
  P \approx_{L} Q \iff \forall E \in L. P \models E \iff Q \models E
\end{eqnarray*}

\begin{eqnarray*}
  P \approx_{K} Q
\end{eqnarray*}

\begin{eqnarray*}
  P \approx Q
\end{eqnarray*}

$\approx_{K} = \approx = \approx_{L}$

\subsubsection{Contextual duality}

Note that contexts extend the quotation operation to a family of
operations from processes to names. Given a context, $M$, we can
define a \emph{nominal context}, $\quotep{M}$ by $\quotep{M}[P] :=
\quotep{M[P]}$. To foreshadow what is to come we observe that these
operations enjoy a duality with processes very much like the duality
between vectors and maps from vectors to scalars.

Further, because the calculus is essentially higher-order, we have a
correspondence between contexts and processes. More specifically,
given a name $x$ and a context $M$ we can construct $M^{*}_{x}$ such
that 

\begin{mathpar}
  M^{*}_{x} | \lift{x}{P} \red M[P]
\end{mathpar}

namely,

\begin{mathpar}
  M^{*}_{x} := x?(u).M[\dropn{u}]
\end{mathpar}

The dependence of $M^{*}_{x}$ on a name makes it an abstraction, 

\begin{mathpar}
  M^{*} := (x)x?(u).M[\dropn{u}]
\end{mathpar}

\subsection{Additional notation}

It will sometimes be convenient to denote the process a name
quotes. We already have the notation $x = \quotep{P}$, but it will be
convenient to introduce an alternate notation, $\procn{x}$, when we
want to emphasize the connection to the use of the name. Note that, by
virtue of name equivalence, $\quotep{\procn{x}} \nameeq x$; so, the
notation is consistent with previous definitions.

Further, because names have structure it is possible to effect
substitutions on the basis of that structure. This means we need to
upgrade our notation for substitutions, which we accomplish by
adapting comprehension notation. Thus,

\begin{mathpar}
  P\{ y / x : x \in S \}
\end{mathpar}

is interpreted to mean the process derived from P by replacing (in a
capture-avoiding manner) each occurrence of $x$ in $S$ by $y$. For example,

\begin{mathpar}
  P\{ \quotep{\procn{x}|\procn{x}} / x : x \in \freenames{P} \}
\end{mathpar}

will replace each (occurrence) of a free name $x$ in $P$ by
$\quotep{\procn{x}|\procn{x}}$.

Also, we will avail ourselves of the notation $x^{L}$ and $x^{R}$ to
denote injections of a name into disjoint copies of the name
space. There are numerous ways to accomplish this. One example can be
found in \cite{MeredithR05}. This notation overloads to vectors of
names: $\vec{x}^{\pi} := (x_{i}^{\pi} \; : \; 0 \leq i < |\vec{x}| )$ where $\pi \in \{L,R\}$.

We also use $P^{\Box} := P|\Box$.

In \cite{MeredithR05} an interpretation of the new operator is
given. It turns out that there are several possible interpretations
all enjoying the requisite algebraic properties of the operator (see
\cite{milner91polyadicpi}). We will therefore make liberal use of
$(\nu\; \vec{x})P$.

% subsection the_syntax_and_semantics_of_the_notation_system (end)   

\input{qm2pi.qmops} 

\input{qm2pi.sterngerlach} 

\input{qm2pi.metric} 

% section concurrent_process_calculi (end)

%\input{qm2pi.proofsketch}

% section proof sketch (end)

%\input{qm2pi.slviaknots} 

% section spatial logic via knots (end)

\input{qm2pi.conclusion}

% section conclusion (end)

%\input{qm2pi.dtcodes} 

% section wiring algorithm (end)

\input{qm2pi.ack} 

% section acknowledgments (end)

\newpage


\bibliographystyle{plain}   
\bibliography{../../biblios/main.bib}

\input{qm2pi.rhodetails}

\end{document}

 

% section notation (end)

\input{qm2pi.process.calculi} 

% section concurrent_process_calculi_and_spatial_logics_ (end)
    
%\documentclass[12pt]{llncs}
%\documentclass{jktr}

\usepackage[pdftex]{hyperref}                   
\usepackage {listings}
\usepackage {mathpartir}
\usepackage{bcprules}
%\usepackage{listings}
                       
\usepackage{graphicx} 
%\usepackage[margins=2.5cm,nohead,nofoot]{geometry}
%\usepackage{geometry}
\usepackage{amsfonts}
\usepackage{amstext}
\usepackage{latexsym}
\usepackage{amssymb}
\usepackage{color}


%\include{myPreamble}
\include{qm2pi.local} 

%\ifpdf
%\usepackage[pdftex]{graphicx}
%\else
%\usepackage{graphicx}
%\fi

 % \ifpdf
%  \usepackage{pdfsync}
%  \if


%\title{Brief Article}
%\author{David F. Snyder}
%\author{L.G. Meredith}

%\address{Dept. of Math., Texas State University--San Marcos, San Marcos, TX 78666}
       
\pagestyle{empty}


\begin{document}

\lstset{language=[Objective]Caml,frame=shadowbox}

\input{qm2pi.front}

% section front matter (end)

\input{qm2pi.intro} 
 
% section introduction (end)

% \input{qm2pi.knotations} 

% section notation (end)

\input{qm2pi.process.calculi} 

% section concurrent_process_calculi_and_spatial_logics_ (end)
    
%\input{qm2pi.knots2pi} 

%\input{qm2pi.trefoil} 

%\input{qm2pi.mainthm} 

% subsection basic_interpretation (end)

%\input{qm2pi.rho.presentation} 
\subsection{The syntax and semantics of the notation system}\label{sub:the_syntax_and_semantics_of_the_notation_system} % (fold)

We now summarize a technical presentation of the calculus that
embodies our theory of dynamics. The typical presentation of such a
calculus follows the style of giving generators and relations on
them. The grammar, below, describing term constructors, freely
generates the set of processes, $\Proc$. This set is then quotiented
by a relation known as structural congruence and it is over this set
that the notion of dynamics is expressed. This presentation is
essentially that of \cite{MeredithR05} with the addition of
polyadicity and summation. For readability we have relegated some of
the technical subtleties to an appendix.

\subsubsection{Process grammar}\label{subsub:process_grammar}

\begin{mathpar}
  \inferrule* [lab=synchronization] {} {{M} \bc \pzero \;|\; x?F \;|\; x!C }
  \and
  \inferrule* [lab=abstraction] {} {{F} \bc (x)P}
  \and
  \inferrule* [lab=concretion] {} {{C} \bc \langle Q \rangle}
  \and
  \inferrule* [lab=process] {} {{P,Q} \bc M \;| \;P|Q \;|\; @{x}}
  \and
  \inferrule* [lab=name] {} {{x} \bc \quotep{P}}
\end{mathpar} 

Note that $\vec{x}$ (resp. $\vec{P}$) denotes a vector of names
(resp. processes) of length $|\vec{x}|$ (resp. $|\vec{P}|$). We adopt
the following useful abbreviations.

\begin{mathpar}
   x?(\vec{y}).P := x.(\vec{y})P \and  x\clift{\vec{P}} := x.\clift{\vec{P}}
   \and x!(y) := \lift{x}{\dropn{y}}
   \and \Pi_{i=0}^{n-1}P_i := P_0 | \ldots | P_{n-1}
\end{mathpar}

\subsubsection{Structural congruence}

\paragraph{Free and bound names and alpha-equivalence.} At the
core of structural equivalence is alpha-equivalence which identifies
process that are the same up to a change of variable. Formally, we
recognize the distinction between free and bound names. The free names
of a process, $\freenames{P}$, may be calculated recursively as
follows:

\begin{mathpar}
\freenames{\pzero} := \emptyset
  \and \\
  \freenames{x?(y).P} := \{ x \} \cup (\freenames{P} \setminus \{ y \})
  \and 
  \freenames{x!\langle P \rangle} := \{ x \} \cup \{ P \} 
  \and \\
  \freenames{P|Q} := \freenames{P} \cup \freenames{Q}
  \and \\
  \freenames{@{x}} := \{ x \}
\end{mathpar}

$\pi$
$\quotep{\pi}$

$\freenames{-} : \pi \to \mathcal{P}(\quotep{\pi})$

\begin{eqnarray*}
  \freenames{\pzero} & := & \emptyset \\
  \freenames{x?(y).P} & := & \{ x \} \cup (\freenames{P} \setminus \{ y \}) \\
  \freenames{x!\langle P \rangle} & := & \{ x \} \cup \{ P \} \\
  \freenames{P|Q} & := & \freenames{P} \cup \freenames{Q} \\
  \freenames{\dropn{x}} & := & \{ x \}
\end{eqnarray*}

The bound names of a process, $\boundnames{P}$, are those names occurring in $P$
that are not free. For example, in $x?(y).0$, the name $x$ is free, while $y$ is bound.

\begin{mathpar}
  \inferrule* [lab=monoidal-laws] {} { P|Q \equiv Q|P \and P|0 \equiv P \and P|(Q|R) \equiv (P|Q)|R }
\end{mathpar}

\begin{mathpar}
  \inferrule* [lab=alpha-equivalence] {} { (x)P \equiv (y)P\{y/x\} \and y \not\in \freenames{P} }
\end{mathpar}

\begin{definition}
Then two processes, $P,Q$, are alpha-equivalent if $P = Q\{\vec{y}/\vec{x}\}$ for
some $\vec{x} \in \boundnames{Q},\vec{y} \in \boundnames{P}$, where $Q\{\vec{y}/\vec{x}\}$
denotes the capture-avoiding substitution of $\vec{y}$ for $\vec{x}$ in $Q$.
\end{definition}

\begin{definition}
  The {\em structural congruence} \cite{SangiorgiWalker} , $\equiv$,
  between processes is the least congruence containing
  alpha-equivalence, satisfying the abelian monoid laws
  (associativity, commutativity and $\pzero$ as identity) for parallel
  composition $|$ and for summation $+$.
\end{definition}

\subsection{Name equivalence}

We take name equivalence, written $\nameeq$, to be the smallest
equivalence relation generated by the following rules.

\begin{mathpar}
\inferrule*[lab=Quote-drop]
{ }
{ \quotep{@{x}} \nameeq x }

\inferrule*[lab=Struct-equiv]
{ P \scong Q }
{ \quotep{P} \nameeq \quotep{Q} }
\end{mathpar}

The astute reader will have noticed that the mutual recursion of names
and processes imposes a mutual recursion on alpha-equivalence and
structural equivalence via name-equivalence. Fortunately, all of this
works out pleasantly and we may calculate in the natural way, free of
concern. The reader interested in the details is referred to the
appendix \ref{appendix:rho_details}.

\subsection{Substitution}

We use $\Proc$ for the set of processes, $\QProc$ for the set of
names, and $\id{\{}\vec{y} / \vec{x} \id{\}}$ to denote partial maps,
$s : \QProc \rightarrow \QProc$. A map, $s$ lifts, uniquely, to a map
on process terms, $\widehat{s} : \Proc \rightarrow \Proc$ by the
following equations.

\begin{mathpar}
  (0) \psubstp{Q}{P} := 0 \\
  (R \juxtap S) \psubstp{Q}{P}
  :=    
  (R)\psubstp{Q}{P} \juxtap (S) \psubstp{Q}{P} \\
  (x?(y).R) \psubstp{Q}{P}    
  :=    
  (x)\substp{Q}{P} (z)\concat( (R \psubstn{z}{y}) \psubstp{Q}{P} ) \\
  (\lift{x}{R}) \psubstp{Q}{P}  
  :=
  \lift{(x)\substp{Q}{P}}{ R \psubstp{Q}{P} } \\
%   (\dropn{x})  \psubstp{Q}{P}       
%   := 
%   \left\{ 
%     \begin{array}{ccc} 
%       \dropn{\quotep{Q}} & & x \nameeq \quotep{P} \\
%       \dropn{x} & & otherwise \\
%     \end{array}
%   \right. 
  (\dropn{x})  \psubstp{Q}{P}       
  := 
  \left\{ 
    \begin{array}{ccc} 
      Q & & x \nameeq \quotep{P} \\
      \dropn{x} & & otherwise \\
    \end{array}
  \right.
\end{mathpar}
 

where

\begin{eqnarray}
  (x)\id{\{} \lpquote Q \rpquote / \lpquote P \rpquote \id{\}}            = 
  \left\{ 
    \begin{array}{ccc}
      \lpquote Q \rpquote & & x \nameeq \lpquote P \rpquote \\
      x & & otherwise \\
    \end{array}
  \right. \nonumber
\end{eqnarray}

and $z$ is chosen distinct from $\quotep{P}$, $\quotep{Q}$, the free
names in $Q$, and all the names in $R$. Our $\alpha$-equivalence will
be built in the standard way from this substitution.

\begin{remark}\label{rem:no_self_referential_names}
  One consequence of these definitions is that $\forall P. \quotep{P}
  \not\in \freenames{P}$.
\end{remark}

\subsection{ Dynamic quote: an example }

Anticipating something of what's to come, consider applying the
substitution, $\widehat{\id{\{}u / z \id{\}}}$, to the following pair
of processes, $\lift{w}{y!(z)}$ and $w[ \lpquote y!(z) \rpquote ]$.

\begin{eqnarray}
	\lift{w}{y!(z)}\widehat{\id{\{}u / z \id{\}}}
		& = &
		\lift{w}{y!(u)} \nonumber\\
	w[ \lpquote y!(z) \rpquote ] \widehat{ \id{\{}u / z \id{\}} }
		& = &
		w[ \lpquote y!(z) \rpquote ] \nonumber
\end{eqnarray}

Because the body of the process between quotes is impervious to
substitution, we get radically different answers. In fact, by
examining the first process in an input context,
e.g. $x?(z).\lift{w}{y!(z)}$, we see that the process under the lift
operator may be shaped by prefixed inputs binding a name inside it. In
this sense, the lift operator will be seen as a way to dynamically
construct processes before reifying them as names.

Finally equipped with these standard features we can present the
dynamics of the calculus.

\subsubsection{Operational semantics} 

Finally, we introduce the computational dynamics. What marks these
algebras as distinct from other more traditionally studied algebraic
structures, e.g. vector spaces or polynomial rings, is the manner in
which dynamics is captured. In traditional structures, dynamics is typically
expressed through morphisms between such structures, as in linear maps
between vector spaces or morphisms between rings. In algebras
associated with the semantics of computation, the dynamics is
expressed as part of the algebraic structure itself, through a
reduction reduction relation typically denoted by $\red$. Below, we
give a recursive presentation of this relation for the calculus used
in the encoding.

$\red \subseteq \pi \times \pi$
$\red : \pi \to \mathcal{P}(\pi)$

\begin{mathpar}
  \inferrule* [lab=Comm] { \textsf{match}( x_{src}, x_{trgt} ) } { x_{trgt}?(y)P \; | \; x_{src}!\langle {Q} \rangle \red P\{\quotep{Q}/y}\} }
  \and \\
  \inferrule* [lab=Par] {{P} \red {P}'} {{{P} | {Q}} \red {{P}' | {Q}}}
  \and
  \inferrule* [lab=Equiv]{{{P} \scong {P}'} \andalso {{P}' \red {Q}'} \andalso {{Q}' \scong {Q}}}{{P} \red {Q}}
\end{mathpar}

\begin{eqnarray*}
  match_{\equiv} (\quotep{P},\quotep{Q}) & := & P \equiv Q \\
  match_{\dagger}(\quotep{P},\quotep{Q}) & := & \forall R. P|Q \red^{*} R => R \red^{*} 0 \\
  match_{K}(\quotep{P},\quotep{Q}) & := & K \mbox{ for some context } K
\end{eqnarray*}

$u?(x)P | u!\langle Q \rangle \red P\{\quotep{Q}/x\}$

%We write $\wred$ for $\red^*$, and $P\red$ if $\exists Q $ such that $ P \red Q$.
We write $P\red$ if $\exists Q $ such that $ P \red Q$ and $P\not\red$, otherwise.

\section{Replication}

As mentioned before, it is known that replication (and hence
recursion) can be implemented in a higher-order process algebra
\cite{SangiorgiWalker}. As our first example of calculation with the
machinery thus far presented we give the construction explicitly in
the {\rhoc}.

\begin{eqnarray}
	D_{x} & := & \prefix{x}{y}{(\binpar{\outputp{x}{y}}{@{y}})} \nonumber\\
	\bangp_{x}{P} & := & \binpar{{x}!\langle{\binpar{D_{x}}{P}}\rangle}{D_{x}} \nonumber
\end{eqnarray}

\begin{eqnarray}
	\bangp_{x}{P} & & \nonumber\\
	=
	& {x}!\langle{(\prefix{x}{y}{(\outputp{x}{y} | @{y})) | P}}\rangle 
	      | \prefix{x}{y}{(\outputp{x}{y} | @{y})} & \nonumber\\
	\red
	& (\outputp{x}{y} | @{y})\substn{\quotep{(\prefix{x}{y}{(@{y} | \outputp{x}{y})) | P}}}{y} & \nonumber\\
	=
	& \outputp{x}{\quotep{(\prefix{x}{y}{(\outputp{x}{y} | @{y})) | P}}}
	  | {(\prefix{x}{y}{(\outputp{x}{y} | @{y})) | P}} & \nonumber\\
	\red
	& \ldots & \nonumber\\
	\red^*
	& P | P | \ldots & \nonumber
\end{eqnarray}

Of course, this encoding, as an implementation, runs away, unfolding
$\bangp{P}$ eagerly. A lazier and more implementable replication
operator, restricted to input-guarded processes, may be obtained as follows.

\begin{eqnarray}
\bangp{\prefix{u}{v}{P}} 
	:= 
	\binpar{\lift{x}{\prefix{u}{v}{(\binpar{D(x)}{P})}}}{D(x)} \nonumber
\end{eqnarray}

\begin{remark}
  Note that the lazier definition still does not deal with summation
  or mixed summation (i.e. sums over input and output). The reader is
  invited to construct definitions of replication that deal with these
  features. 

  Further, the definitions are parameterized in a name, $x$. Can you,
  gentle reader, make a definition that eliminates this parameter and
  guarantees no accidental interaction between the replication
  machinery and the process being replicated -- i.e. no accidental
  sharing of names used by the process to get its work done and the
  name(s) used by the replication to effect copying. This latter
  revision of the definition of replication is crucial to obtaining
  the expected identity $!!P \sim !P$.
\end{remark}

\begin{remark}\label{rem:paradoxical_combinator}
  The reader familiar with the lambda calculus will have noticed the
  similarity between $D$ and the paradoxical combinator.

  [Ed. note: the existence of this seems to suggest we have to be more
  restrictive on the set of processes and names we admit if we are to
  support no-cloning.]
\end{remark}

\subsubsection{Bisimulation}

The computational dynamics gives rise to another kind of equivalence,
the equivalence of computational behavior. As previously mentioned
this is typically captured \emph{via} some form of bisimulation.

% The notion we use in this paper is weak barbed bisimulation
% \cite{milner91polyadicpi}.

The notion we use in this paper is derived from weak barbed
bisimulation \cite{milner91polyadicpi}. 

\begin{definition}
An \emph{observation relation}, $\downarrow_{\mathcal N}$, over a set
of names, $\mathcal N$, is the smallest relation satisfying the rules
below.

\infrule[Out-barb]{y \in {\mathcal N}, \; x \nameeq y}
		  {\outputp{x}{v} \downarrow_{\mathcal N} x}
\infrule[Par-barb]{\mbox{$P\downarrow_{\mathcal N} x$ or $Q\downarrow_{\mathcal N} x$}}
		  {\binpar{P}{Q} \downarrow_{\mathcal N} x}

We write $P \Downarrow_{\mathcal N} x$ if there is $Q$ such that 
$P \wred Q$ and $Q \downarrow_{\mathcal N} x$.
\end{definition}

\begin{definition}
%\label{def.bbisim}
An  ${\mathcal N}$-\emph{barbed bisimulation} over a set of names, ${\mathcal N}$, is a symmetric binary relation 
${\mathcal S}_{\mathcal N}$ between agents such that $P\rel{S}_{\mathcal N}Q$ implies:
\begin{enumerate}
\item If $P \red P'$ then $Q \wred Q'$ and $P'\rel{S}_{\mathcal N} Q'$.
\item If $P\downarrow_{\mathcal N} x$, then $Q\Downarrow_{\mathcal N} x$.
\end{enumerate}
$P$ is ${\mathcal N}$-barbed bisimilar to $Q$, written
$P \wbbisim_{\mathcal N} Q$, if $P \rel{S}_{\mathcal N} Q$ for some ${\mathcal N}$-barbed bisimulation ${\mathcal S}_{\mathcal N}$.
\end{definition}

$\mathcal{R} \subseteq \pi \times \pi$

$P \mathcal{R} Q => \forall P'. P \red P' \Rightarrow \exists Q'. Q \red Q', P' \mathcal{R} Q'$

$P \vdash x \Rightarrow Q \vdash x$

\begin{mathpar}
  \inferrule*[lab=Out-barb]{x \nameeq y}{{y}!\langle{Q}\rangle \vdash x}
  \and
  \inferrule*[lab=Par-barb]{\mbox{$P\vdash x$ or $Q\vdash x$}}{\binpar{P}{Q} \vdash x}
\end{mathpar}

\subsubsection{Contexts}

One of the principle advantages of computational calculi like the
$\pi$-calculus is a well-defined notion of context,
contextual-equivalence and a correlation between
contextual-equivalence and notions of bisimulation. The notion of
context allows the decomposition of a process into (sub-)process and
its syntactic environment, its context. Thus, a context may be
thought of as a process with a ``hole'' (written $\Box$) in it. The
application of a context $M$ to a process $P$, written $M[P]$, is
tantamount to filling the hole in $M$ with $P$. In this paper we do
not need the full weight of this theory, but do make use of the notion
of context in the proof the main theorem. 

\begin{mathpar}
  \inferrule* [lab=summation] {} {{M_{M},M_{N}} \bc \Box \;|\; x.M_{A} \;|\; M_{M}+M_{N}}
  \and
  \inferrule* [lab=agent] {} {{M_{A}} \bc (\vec{x})M_{P} \;| \; \clift{P_0,\ldots,M_{P},\ldots,P_N}}
  \and \\
  \inferrule* [lab=process] {} {{M_{P}} \bc M_{N} \;| \;P|M_{P} }
\end{mathpar} 

\begin{mathpar}
  \inferrule* [lab=sychronization] {} {M_{N} \bc \Box \;|\; x?M_{F} \;|\; x!M_{C}}
  \and
  \inferrule* [lab=abstraction] {} {{M_{F}} \bc (x)M_{P} }
  \and
  \inferrule* [lab=concretion] {} {{M_{C}} \bc \langle M_{P} \rangle }
  \and \\
  \inferrule* [lab=process] {} {{M_{P}} \bc M_{N} \;| \;P|M_{P} }
\end{mathpar}

\begin{definition}[contextual application] Given a context $M$, and
  process $P$, we define the \emph{contextual application}, $M[P] :=
  M\{P/\Box\}$. That is, the contextual application of M to P is the
  substitution of $P$ for $\Box$ in $M$.
\end{definition}

$\meaningof{-} : L \to \mathcal{P}(\pi)$

\begin{mathpar}
  \inferrule* [lab=collection] {} {\meaningof{true} = \pi, \and \meaningof{~E} = \pi \setminus \meaningof{E}, \and \meaningof{E_{1} \& E_{2}} = \meaningof{E_{1}} \cap \meaningof{E_{2}}}
\end{mathpar}

\begin{mathpar}
  \inferrule* [lab=structure] {} {\meaningof{0} = \{ P \in \pi | P \equiv 0 \}, \and \\ \meaningof{E_1 | E_2} = \{ P \in \pi | P \equiv P_{1} | P_{2}, P_{1} \in \meaningof{E_{1}}, P_{2} \in \meaningof{E_2}\} }
\end{mathpar}

\begin{mathpar}
 \inferrule* [lab=behavior] {} {\meaningof{\langle a?b \rangle E} = \{ P \in \pi | P \equiv Q | u?(y)P', \\ \and \\\\ \and \\ \;\;\; u \in \meaningof{a}, \forall z.P'\{z/y\} \in \meaningof{E\{z/b\}}\}, \and \\ \meaningof{a!E} = \{ P \in \pi | P \equiv Q | x!\langle P' \rangle, x \in \meaningof{a} P' \in \meaningof{E}\} }
\end{mathpar}

\begin{mathpar}
 \inferrule* [lab=nominal] {} {\meaningof{\quotep{E}} = \{ \quotep{P} \in \quotep{\pi} | P \in \meaningof{E} \}, \and \meaningof{\quotep{P}} = \{ \quotep{Q} \in \quotep{\pi} | P \equiv Q \} \and \\ \meaningof{@\quotep{E}} = \{ P \in \pi | P \equiv @x, x \in \meaningof{E} \}}
\end{mathpar}

\begin{eqnarray*}
  \\
  \meaningof{-} : TS \to ST
\end{eqnarray*}

\begin{eqnarray*}
  \\
  L : TS \to ST
\end{eqnarray*}

\begin{eqnarray*}
  \\
  P \models E \iff P \in \meaningof{E}
\end{eqnarray*}

\begin{eqnarray*}
  P \approx_{L} Q \iff \forall E \in L. P \models E \iff Q \models E
\end{eqnarray*}

\begin{eqnarray*}
  P \approx_{K} Q
\end{eqnarray*}

\begin{eqnarray*}
  P \approx Q
\end{eqnarray*}

$\approx_{K} = \approx = \approx_{L}$

\subsubsection{Contextual duality}

Note that contexts extend the quotation operation to a family of
operations from processes to names. Given a context, $M$, we can
define a \emph{nominal context}, $\quotep{M}$ by $\quotep{M}[P] :=
\quotep{M[P]}$. To foreshadow what is to come we observe that these
operations enjoy a duality with processes very much like the duality
between vectors and maps from vectors to scalars.

Further, because the calculus is essentially higher-order, we have a
correspondence between contexts and processes. More specifically,
given a name $x$ and a context $M$ we can construct $M^{*}_{x}$ such
that 

\begin{mathpar}
  M^{*}_{x} | \lift{x}{P} \red M[P]
\end{mathpar}

namely,

\begin{mathpar}
  M^{*}_{x} := x?(u).M[\dropn{u}]
\end{mathpar}

The dependence of $M^{*}_{x}$ on a name makes it an abstraction, 

\begin{mathpar}
  M^{*} := (x)x?(u).M[\dropn{u}]
\end{mathpar}

\subsection{Additional notation}

It will sometimes be convenient to denote the process a name
quotes. We already have the notation $x = \quotep{P}$, but it will be
convenient to introduce an alternate notation, $\procn{x}$, when we
want to emphasize the connection to the use of the name. Note that, by
virtue of name equivalence, $\quotep{\procn{x}} \nameeq x$; so, the
notation is consistent with previous definitions.

Further, because names have structure it is possible to effect
substitutions on the basis of that structure. This means we need to
upgrade our notation for substitutions, which we accomplish by
adapting comprehension notation. Thus,

\begin{mathpar}
  P\{ y / x : x \in S \}
\end{mathpar}

is interpreted to mean the process derived from P by replacing (in a
capture-avoiding manner) each occurrence of $x$ in $S$ by $y$. For example,

\begin{mathpar}
  P\{ \quotep{\procn{x}|\procn{x}} / x : x \in \freenames{P} \}
\end{mathpar}

will replace each (occurrence) of a free name $x$ in $P$ by
$\quotep{\procn{x}|\procn{x}}$.

Also, we will avail ourselves of the notation $x^{L}$ and $x^{R}$ to
denote injections of a name into disjoint copies of the name
space. There are numerous ways to accomplish this. One example can be
found in \cite{MeredithR05}. This notation overloads to vectors of
names: $\vec{x}^{\pi} := (x_{i}^{\pi} \; : \; 0 \leq i < |\vec{x}| )$ where $\pi \in \{L,R\}$.

We also use $P^{\Box} := P|\Box$.

In \cite{MeredithR05} an interpretation of the new operator is
given. It turns out that there are several possible interpretations
all enjoying the requisite algebraic properties of the operator (see
\cite{milner91polyadicpi}). We will therefore make liberal use of
$(\nu\; \vec{x})P$.

% subsection the_syntax_and_semantics_of_the_notation_system (end)   

\input{qm2pi.qmops} 

\input{qm2pi.sterngerlach} 

\input{qm2pi.metric} 

% section concurrent_process_calculi (end)

%\input{qm2pi.proofsketch}

% section proof sketch (end)

%\input{qm2pi.slviaknots} 

% section spatial logic via knots (end)

\input{qm2pi.conclusion}

% section conclusion (end)

%\input{qm2pi.dtcodes} 

% section wiring algorithm (end)

\input{qm2pi.ack} 

% section acknowledgments (end)

\newpage


\bibliographystyle{plain}   
\bibliography{../../biblios/main.bib}

\input{qm2pi.rhodetails}

\end{document}

 

%\documentclass[12pt]{llncs}
%\documentclass{jktr}

\usepackage[pdftex]{hyperref}                   
\usepackage {listings}
\usepackage {mathpartir}
\usepackage{bcprules}
%\usepackage{listings}
                       
\usepackage{graphicx} 
%\usepackage[margins=2.5cm,nohead,nofoot]{geometry}
%\usepackage{geometry}
\usepackage{amsfonts}
\usepackage{amstext}
\usepackage{latexsym}
\usepackage{amssymb}
\usepackage{color}


%\include{myPreamble}
\include{qm2pi.local} 

%\ifpdf
%\usepackage[pdftex]{graphicx}
%\else
%\usepackage{graphicx}
%\fi

 % \ifpdf
%  \usepackage{pdfsync}
%  \if


%\title{Brief Article}
%\author{David F. Snyder}
%\author{L.G. Meredith}

%\address{Dept. of Math., Texas State University--San Marcos, San Marcos, TX 78666}
       
\pagestyle{empty}


\begin{document}

\lstset{language=[Objective]Caml,frame=shadowbox}

\input{qm2pi.front}

% section front matter (end)

\input{qm2pi.intro} 
 
% section introduction (end)

% \input{qm2pi.knotations} 

% section notation (end)

\input{qm2pi.process.calculi} 

% section concurrent_process_calculi_and_spatial_logics_ (end)
    
%\input{qm2pi.knots2pi} 

%\input{qm2pi.trefoil} 

%\input{qm2pi.mainthm} 

% subsection basic_interpretation (end)

%\input{qm2pi.rho.presentation} 
\subsection{The syntax and semantics of the notation system}\label{sub:the_syntax_and_semantics_of_the_notation_system} % (fold)

We now summarize a technical presentation of the calculus that
embodies our theory of dynamics. The typical presentation of such a
calculus follows the style of giving generators and relations on
them. The grammar, below, describing term constructors, freely
generates the set of processes, $\Proc$. This set is then quotiented
by a relation known as structural congruence and it is over this set
that the notion of dynamics is expressed. This presentation is
essentially that of \cite{MeredithR05} with the addition of
polyadicity and summation. For readability we have relegated some of
the technical subtleties to an appendix.

\subsubsection{Process grammar}\label{subsub:process_grammar}

\begin{mathpar}
  \inferrule* [lab=synchronization] {} {{M} \bc \pzero \;|\; x?F \;|\; x!C }
  \and
  \inferrule* [lab=abstraction] {} {{F} \bc (x)P}
  \and
  \inferrule* [lab=concretion] {} {{C} \bc \langle Q \rangle}
  \and
  \inferrule* [lab=process] {} {{P,Q} \bc M \;| \;P|Q \;|\; @{x}}
  \and
  \inferrule* [lab=name] {} {{x} \bc \quotep{P}}
\end{mathpar} 

Note that $\vec{x}$ (resp. $\vec{P}$) denotes a vector of names
(resp. processes) of length $|\vec{x}|$ (resp. $|\vec{P}|$). We adopt
the following useful abbreviations.

\begin{mathpar}
   x?(\vec{y}).P := x.(\vec{y})P \and  x\clift{\vec{P}} := x.\clift{\vec{P}}
   \and x!(y) := \lift{x}{\dropn{y}}
   \and \Pi_{i=0}^{n-1}P_i := P_0 | \ldots | P_{n-1}
\end{mathpar}

\subsubsection{Structural congruence}

\paragraph{Free and bound names and alpha-equivalence.} At the
core of structural equivalence is alpha-equivalence which identifies
process that are the same up to a change of variable. Formally, we
recognize the distinction between free and bound names. The free names
of a process, $\freenames{P}$, may be calculated recursively as
follows:

\begin{mathpar}
\freenames{\pzero} := \emptyset
  \and \\
  \freenames{x?(y).P} := \{ x \} \cup (\freenames{P} \setminus \{ y \})
  \and 
  \freenames{x!\langle P \rangle} := \{ x \} \cup \{ P \} 
  \and \\
  \freenames{P|Q} := \freenames{P} \cup \freenames{Q}
  \and \\
  \freenames{@{x}} := \{ x \}
\end{mathpar}

$\pi$
$\quotep{\pi}$

$\freenames{-} : \pi \to \mathcal{P}(\quotep{\pi})$

\begin{eqnarray*}
  \freenames{\pzero} & := & \emptyset \\
  \freenames{x?(y).P} & := & \{ x \} \cup (\freenames{P} \setminus \{ y \}) \\
  \freenames{x!\langle P \rangle} & := & \{ x \} \cup \{ P \} \\
  \freenames{P|Q} & := & \freenames{P} \cup \freenames{Q} \\
  \freenames{\dropn{x}} & := & \{ x \}
\end{eqnarray*}

The bound names of a process, $\boundnames{P}$, are those names occurring in $P$
that are not free. For example, in $x?(y).0$, the name $x$ is free, while $y$ is bound.

\begin{mathpar}
  \inferrule* [lab=monoidal-laws] {} { P|Q \equiv Q|P \and P|0 \equiv P \and P|(Q|R) \equiv (P|Q)|R }
\end{mathpar}

\begin{mathpar}
  \inferrule* [lab=alpha-equivalence] {} { (x)P \equiv (y)P\{y/x\} \and y \not\in \freenames{P} }
\end{mathpar}

\begin{definition}
Then two processes, $P,Q$, are alpha-equivalent if $P = Q\{\vec{y}/\vec{x}\}$ for
some $\vec{x} \in \boundnames{Q},\vec{y} \in \boundnames{P}$, where $Q\{\vec{y}/\vec{x}\}$
denotes the capture-avoiding substitution of $\vec{y}$ for $\vec{x}$ in $Q$.
\end{definition}

\begin{definition}
  The {\em structural congruence} \cite{SangiorgiWalker} , $\equiv$,
  between processes is the least congruence containing
  alpha-equivalence, satisfying the abelian monoid laws
  (associativity, commutativity and $\pzero$ as identity) for parallel
  composition $|$ and for summation $+$.
\end{definition}

\subsection{Name equivalence}

We take name equivalence, written $\nameeq$, to be the smallest
equivalence relation generated by the following rules.

\begin{mathpar}
\inferrule*[lab=Quote-drop]
{ }
{ \quotep{@{x}} \nameeq x }

\inferrule*[lab=Struct-equiv]
{ P \scong Q }
{ \quotep{P} \nameeq \quotep{Q} }
\end{mathpar}

The astute reader will have noticed that the mutual recursion of names
and processes imposes a mutual recursion on alpha-equivalence and
structural equivalence via name-equivalence. Fortunately, all of this
works out pleasantly and we may calculate in the natural way, free of
concern. The reader interested in the details is referred to the
appendix \ref{appendix:rho_details}.

\subsection{Substitution}

We use $\Proc$ for the set of processes, $\QProc$ for the set of
names, and $\id{\{}\vec{y} / \vec{x} \id{\}}$ to denote partial maps,
$s : \QProc \rightarrow \QProc$. A map, $s$ lifts, uniquely, to a map
on process terms, $\widehat{s} : \Proc \rightarrow \Proc$ by the
following equations.

\begin{mathpar}
  (0) \psubstp{Q}{P} := 0 \\
  (R \juxtap S) \psubstp{Q}{P}
  :=    
  (R)\psubstp{Q}{P} \juxtap (S) \psubstp{Q}{P} \\
  (x?(y).R) \psubstp{Q}{P}    
  :=    
  (x)\substp{Q}{P} (z)\concat( (R \psubstn{z}{y}) \psubstp{Q}{P} ) \\
  (\lift{x}{R}) \psubstp{Q}{P}  
  :=
  \lift{(x)\substp{Q}{P}}{ R \psubstp{Q}{P} } \\
%   (\dropn{x})  \psubstp{Q}{P}       
%   := 
%   \left\{ 
%     \begin{array}{ccc} 
%       \dropn{\quotep{Q}} & & x \nameeq \quotep{P} \\
%       \dropn{x} & & otherwise \\
%     \end{array}
%   \right. 
  (\dropn{x})  \psubstp{Q}{P}       
  := 
  \left\{ 
    \begin{array}{ccc} 
      Q & & x \nameeq \quotep{P} \\
      \dropn{x} & & otherwise \\
    \end{array}
  \right.
\end{mathpar}
 

where

\begin{eqnarray}
  (x)\id{\{} \lpquote Q \rpquote / \lpquote P \rpquote \id{\}}            = 
  \left\{ 
    \begin{array}{ccc}
      \lpquote Q \rpquote & & x \nameeq \lpquote P \rpquote \\
      x & & otherwise \\
    \end{array}
  \right. \nonumber
\end{eqnarray}

and $z$ is chosen distinct from $\quotep{P}$, $\quotep{Q}$, the free
names in $Q$, and all the names in $R$. Our $\alpha$-equivalence will
be built in the standard way from this substitution.

\begin{remark}\label{rem:no_self_referential_names}
  One consequence of these definitions is that $\forall P. \quotep{P}
  \not\in \freenames{P}$.
\end{remark}

\subsection{ Dynamic quote: an example }

Anticipating something of what's to come, consider applying the
substitution, $\widehat{\id{\{}u / z \id{\}}}$, to the following pair
of processes, $\lift{w}{y!(z)}$ and $w[ \lpquote y!(z) \rpquote ]$.

\begin{eqnarray}
	\lift{w}{y!(z)}\widehat{\id{\{}u / z \id{\}}}
		& = &
		\lift{w}{y!(u)} \nonumber\\
	w[ \lpquote y!(z) \rpquote ] \widehat{ \id{\{}u / z \id{\}} }
		& = &
		w[ \lpquote y!(z) \rpquote ] \nonumber
\end{eqnarray}

Because the body of the process between quotes is impervious to
substitution, we get radically different answers. In fact, by
examining the first process in an input context,
e.g. $x?(z).\lift{w}{y!(z)}$, we see that the process under the lift
operator may be shaped by prefixed inputs binding a name inside it. In
this sense, the lift operator will be seen as a way to dynamically
construct processes before reifying them as names.

Finally equipped with these standard features we can present the
dynamics of the calculus.

\subsubsection{Operational semantics} 

Finally, we introduce the computational dynamics. What marks these
algebras as distinct from other more traditionally studied algebraic
structures, e.g. vector spaces or polynomial rings, is the manner in
which dynamics is captured. In traditional structures, dynamics is typically
expressed through morphisms between such structures, as in linear maps
between vector spaces or morphisms between rings. In algebras
associated with the semantics of computation, the dynamics is
expressed as part of the algebraic structure itself, through a
reduction reduction relation typically denoted by $\red$. Below, we
give a recursive presentation of this relation for the calculus used
in the encoding.

$\red \subseteq \pi \times \pi$
$\red : \pi \to \mathcal{P}(\pi)$

\begin{mathpar}
  \inferrule* [lab=Comm] { \textsf{match}( x_{src}, x_{trgt} ) } { x_{trgt}?(y)P \; | \; x_{src}!\langle {Q} \rangle \red P\{\quotep{Q}/y}\} }
  \and \\
  \inferrule* [lab=Par] {{P} \red {P}'} {{{P} | {Q}} \red {{P}' | {Q}}}
  \and
  \inferrule* [lab=Equiv]{{{P} \scong {P}'} \andalso {{P}' \red {Q}'} \andalso {{Q}' \scong {Q}}}{{P} \red {Q}}
\end{mathpar}

\begin{eqnarray*}
  match_{\equiv} (\quotep{P},\quotep{Q}) & := & P \equiv Q \\
  match_{\dagger}(\quotep{P},\quotep{Q}) & := & \forall R. P|Q \red^{*} R => R \red^{*} 0 \\
  match_{K}(\quotep{P},\quotep{Q}) & := & K \mbox{ for some context } K
\end{eqnarray*}

$u?(x)P | u!\langle Q \rangle \red P\{\quotep{Q}/x\}$

%We write $\wred$ for $\red^*$, and $P\red$ if $\exists Q $ such that $ P \red Q$.
We write $P\red$ if $\exists Q $ such that $ P \red Q$ and $P\not\red$, otherwise.

\section{Replication}

As mentioned before, it is known that replication (and hence
recursion) can be implemented in a higher-order process algebra
\cite{SangiorgiWalker}. As our first example of calculation with the
machinery thus far presented we give the construction explicitly in
the {\rhoc}.

\begin{eqnarray}
	D_{x} & := & \prefix{x}{y}{(\binpar{\outputp{x}{y}}{@{y}})} \nonumber\\
	\bangp_{x}{P} & := & \binpar{{x}!\langle{\binpar{D_{x}}{P}}\rangle}{D_{x}} \nonumber
\end{eqnarray}

\begin{eqnarray}
	\bangp_{x}{P} & & \nonumber\\
	=
	& {x}!\langle{(\prefix{x}{y}{(\outputp{x}{y} | @{y})) | P}}\rangle 
	      | \prefix{x}{y}{(\outputp{x}{y} | @{y})} & \nonumber\\
	\red
	& (\outputp{x}{y} | @{y})\substn{\quotep{(\prefix{x}{y}{(@{y} | \outputp{x}{y})) | P}}}{y} & \nonumber\\
	=
	& \outputp{x}{\quotep{(\prefix{x}{y}{(\outputp{x}{y} | @{y})) | P}}}
	  | {(\prefix{x}{y}{(\outputp{x}{y} | @{y})) | P}} & \nonumber\\
	\red
	& \ldots & \nonumber\\
	\red^*
	& P | P | \ldots & \nonumber
\end{eqnarray}

Of course, this encoding, as an implementation, runs away, unfolding
$\bangp{P}$ eagerly. A lazier and more implementable replication
operator, restricted to input-guarded processes, may be obtained as follows.

\begin{eqnarray}
\bangp{\prefix{u}{v}{P}} 
	:= 
	\binpar{\lift{x}{\prefix{u}{v}{(\binpar{D(x)}{P})}}}{D(x)} \nonumber
\end{eqnarray}

\begin{remark}
  Note that the lazier definition still does not deal with summation
  or mixed summation (i.e. sums over input and output). The reader is
  invited to construct definitions of replication that deal with these
  features. 

  Further, the definitions are parameterized in a name, $x$. Can you,
  gentle reader, make a definition that eliminates this parameter and
  guarantees no accidental interaction between the replication
  machinery and the process being replicated -- i.e. no accidental
  sharing of names used by the process to get its work done and the
  name(s) used by the replication to effect copying. This latter
  revision of the definition of replication is crucial to obtaining
  the expected identity $!!P \sim !P$.
\end{remark}

\begin{remark}\label{rem:paradoxical_combinator}
  The reader familiar with the lambda calculus will have noticed the
  similarity between $D$ and the paradoxical combinator.

  [Ed. note: the existence of this seems to suggest we have to be more
  restrictive on the set of processes and names we admit if we are to
  support no-cloning.]
\end{remark}

\subsubsection{Bisimulation}

The computational dynamics gives rise to another kind of equivalence,
the equivalence of computational behavior. As previously mentioned
this is typically captured \emph{via} some form of bisimulation.

% The notion we use in this paper is weak barbed bisimulation
% \cite{milner91polyadicpi}.

The notion we use in this paper is derived from weak barbed
bisimulation \cite{milner91polyadicpi}. 

\begin{definition}
An \emph{observation relation}, $\downarrow_{\mathcal N}$, over a set
of names, $\mathcal N$, is the smallest relation satisfying the rules
below.

\infrule[Out-barb]{y \in {\mathcal N}, \; x \nameeq y}
		  {\outputp{x}{v} \downarrow_{\mathcal N} x}
\infrule[Par-barb]{\mbox{$P\downarrow_{\mathcal N} x$ or $Q\downarrow_{\mathcal N} x$}}
		  {\binpar{P}{Q} \downarrow_{\mathcal N} x}

We write $P \Downarrow_{\mathcal N} x$ if there is $Q$ such that 
$P \wred Q$ and $Q \downarrow_{\mathcal N} x$.
\end{definition}

\begin{definition}
%\label{def.bbisim}
An  ${\mathcal N}$-\emph{barbed bisimulation} over a set of names, ${\mathcal N}$, is a symmetric binary relation 
${\mathcal S}_{\mathcal N}$ between agents such that $P\rel{S}_{\mathcal N}Q$ implies:
\begin{enumerate}
\item If $P \red P'$ then $Q \wred Q'$ and $P'\rel{S}_{\mathcal N} Q'$.
\item If $P\downarrow_{\mathcal N} x$, then $Q\Downarrow_{\mathcal N} x$.
\end{enumerate}
$P$ is ${\mathcal N}$-barbed bisimilar to $Q$, written
$P \wbbisim_{\mathcal N} Q$, if $P \rel{S}_{\mathcal N} Q$ for some ${\mathcal N}$-barbed bisimulation ${\mathcal S}_{\mathcal N}$.
\end{definition}

$\mathcal{R} \subseteq \pi \times \pi$

$P \mathcal{R} Q => \forall P'. P \red P' \Rightarrow \exists Q'. Q \red Q', P' \mathcal{R} Q'$

$P \vdash x \Rightarrow Q \vdash x$

\begin{mathpar}
  \inferrule*[lab=Out-barb]{x \nameeq y}{{y}!\langle{Q}\rangle \vdash x}
  \and
  \inferrule*[lab=Par-barb]{\mbox{$P\vdash x$ or $Q\vdash x$}}{\binpar{P}{Q} \vdash x}
\end{mathpar}

\subsubsection{Contexts}

One of the principle advantages of computational calculi like the
$\pi$-calculus is a well-defined notion of context,
contextual-equivalence and a correlation between
contextual-equivalence and notions of bisimulation. The notion of
context allows the decomposition of a process into (sub-)process and
its syntactic environment, its context. Thus, a context may be
thought of as a process with a ``hole'' (written $\Box$) in it. The
application of a context $M$ to a process $P$, written $M[P]$, is
tantamount to filling the hole in $M$ with $P$. In this paper we do
not need the full weight of this theory, but do make use of the notion
of context in the proof the main theorem. 

\begin{mathpar}
  \inferrule* [lab=summation] {} {{M_{M},M_{N}} \bc \Box \;|\; x.M_{A} \;|\; M_{M}+M_{N}}
  \and
  \inferrule* [lab=agent] {} {{M_{A}} \bc (\vec{x})M_{P} \;| \; \clift{P_0,\ldots,M_{P},\ldots,P_N}}
  \and \\
  \inferrule* [lab=process] {} {{M_{P}} \bc M_{N} \;| \;P|M_{P} }
\end{mathpar} 

\begin{mathpar}
  \inferrule* [lab=sychronization] {} {M_{N} \bc \Box \;|\; x?M_{F} \;|\; x!M_{C}}
  \and
  \inferrule* [lab=abstraction] {} {{M_{F}} \bc (x)M_{P} }
  \and
  \inferrule* [lab=concretion] {} {{M_{C}} \bc \langle M_{P} \rangle }
  \and \\
  \inferrule* [lab=process] {} {{M_{P}} \bc M_{N} \;| \;P|M_{P} }
\end{mathpar}

\begin{definition}[contextual application] Given a context $M$, and
  process $P$, we define the \emph{contextual application}, $M[P] :=
  M\{P/\Box\}$. That is, the contextual application of M to P is the
  substitution of $P$ for $\Box$ in $M$.
\end{definition}

$\meaningof{-} : L \to \mathcal{P}(\pi)$

\begin{mathpar}
  \inferrule* [lab=collection] {} {\meaningof{true} = \pi, \and \meaningof{~E} = \pi \setminus \meaningof{E}, \and \meaningof{E_{1} \& E_{2}} = \meaningof{E_{1}} \cap \meaningof{E_{2}}}
\end{mathpar}

\begin{mathpar}
  \inferrule* [lab=structure] {} {\meaningof{0} = \{ P \in \pi | P \equiv 0 \}, \and \\ \meaningof{E_1 | E_2} = \{ P \in \pi | P \equiv P_{1} | P_{2}, P_{1} \in \meaningof{E_{1}}, P_{2} \in \meaningof{E_2}\} }
\end{mathpar}

\begin{mathpar}
 \inferrule* [lab=behavior] {} {\meaningof{\langle a?b \rangle E} = \{ P \in \pi | P \equiv Q | u?(y)P', \\ \and \\\\ \and \\ \;\;\; u \in \meaningof{a}, \forall z.P'\{z/y\} \in \meaningof{E\{z/b\}}\}, \and \\ \meaningof{a!E} = \{ P \in \pi | P \equiv Q | x!\langle P' \rangle, x \in \meaningof{a} P' \in \meaningof{E}\} }
\end{mathpar}

\begin{mathpar}
 \inferrule* [lab=nominal] {} {\meaningof{\quotep{E}} = \{ \quotep{P} \in \quotep{\pi} | P \in \meaningof{E} \}, \and \meaningof{\quotep{P}} = \{ \quotep{Q} \in \quotep{\pi} | P \equiv Q \} \and \\ \meaningof{@\quotep{E}} = \{ P \in \pi | P \equiv @x, x \in \meaningof{E} \}}
\end{mathpar}

\begin{eqnarray*}
  \\
  \meaningof{-} : TS \to ST
\end{eqnarray*}

\begin{eqnarray*}
  \\
  L : TS \to ST
\end{eqnarray*}

\begin{eqnarray*}
  \\
  P \models E \iff P \in \meaningof{E}
\end{eqnarray*}

\begin{eqnarray*}
  P \approx_{L} Q \iff \forall E \in L. P \models E \iff Q \models E
\end{eqnarray*}

\begin{eqnarray*}
  P \approx_{K} Q
\end{eqnarray*}

\begin{eqnarray*}
  P \approx Q
\end{eqnarray*}

$\approx_{K} = \approx = \approx_{L}$

\subsubsection{Contextual duality}

Note that contexts extend the quotation operation to a family of
operations from processes to names. Given a context, $M$, we can
define a \emph{nominal context}, $\quotep{M}$ by $\quotep{M}[P] :=
\quotep{M[P]}$. To foreshadow what is to come we observe that these
operations enjoy a duality with processes very much like the duality
between vectors and maps from vectors to scalars.

Further, because the calculus is essentially higher-order, we have a
correspondence between contexts and processes. More specifically,
given a name $x$ and a context $M$ we can construct $M^{*}_{x}$ such
that 

\begin{mathpar}
  M^{*}_{x} | \lift{x}{P} \red M[P]
\end{mathpar}

namely,

\begin{mathpar}
  M^{*}_{x} := x?(u).M[\dropn{u}]
\end{mathpar}

The dependence of $M^{*}_{x}$ on a name makes it an abstraction, 

\begin{mathpar}
  M^{*} := (x)x?(u).M[\dropn{u}]
\end{mathpar}

\subsection{Additional notation}

It will sometimes be convenient to denote the process a name
quotes. We already have the notation $x = \quotep{P}$, but it will be
convenient to introduce an alternate notation, $\procn{x}$, when we
want to emphasize the connection to the use of the name. Note that, by
virtue of name equivalence, $\quotep{\procn{x}} \nameeq x$; so, the
notation is consistent with previous definitions.

Further, because names have structure it is possible to effect
substitutions on the basis of that structure. This means we need to
upgrade our notation for substitutions, which we accomplish by
adapting comprehension notation. Thus,

\begin{mathpar}
  P\{ y / x : x \in S \}
\end{mathpar}

is interpreted to mean the process derived from P by replacing (in a
capture-avoiding manner) each occurrence of $x$ in $S$ by $y$. For example,

\begin{mathpar}
  P\{ \quotep{\procn{x}|\procn{x}} / x : x \in \freenames{P} \}
\end{mathpar}

will replace each (occurrence) of a free name $x$ in $P$ by
$\quotep{\procn{x}|\procn{x}}$.

Also, we will avail ourselves of the notation $x^{L}$ and $x^{R}$ to
denote injections of a name into disjoint copies of the name
space. There are numerous ways to accomplish this. One example can be
found in \cite{MeredithR05}. This notation overloads to vectors of
names: $\vec{x}^{\pi} := (x_{i}^{\pi} \; : \; 0 \leq i < |\vec{x}| )$ where $\pi \in \{L,R\}$.

We also use $P^{\Box} := P|\Box$.

In \cite{MeredithR05} an interpretation of the new operator is
given. It turns out that there are several possible interpretations
all enjoying the requisite algebraic properties of the operator (see
\cite{milner91polyadicpi}). We will therefore make liberal use of
$(\nu\; \vec{x})P$.

% subsection the_syntax_and_semantics_of_the_notation_system (end)   

\input{qm2pi.qmops} 

\input{qm2pi.sterngerlach} 

\input{qm2pi.metric} 

% section concurrent_process_calculi (end)

%\input{qm2pi.proofsketch}

% section proof sketch (end)

%\input{qm2pi.slviaknots} 

% section spatial logic via knots (end)

\input{qm2pi.conclusion}

% section conclusion (end)

%\input{qm2pi.dtcodes} 

% section wiring algorithm (end)

\input{qm2pi.ack} 

% section acknowledgments (end)

\newpage


\bibliographystyle{plain}   
\bibliography{../../biblios/main.bib}

\input{qm2pi.rhodetails}

\end{document}

 

%\documentclass[12pt]{llncs}
%\documentclass{jktr}

\usepackage[pdftex]{hyperref}                   
\usepackage {listings}
\usepackage {mathpartir}
\usepackage{bcprules}
%\usepackage{listings}
                       
\usepackage{graphicx} 
%\usepackage[margins=2.5cm,nohead,nofoot]{geometry}
%\usepackage{geometry}
\usepackage{amsfonts}
\usepackage{amstext}
\usepackage{latexsym}
\usepackage{amssymb}
\usepackage{color}


%\include{myPreamble}
\include{qm2pi.local} 

%\ifpdf
%\usepackage[pdftex]{graphicx}
%\else
%\usepackage{graphicx}
%\fi

 % \ifpdf
%  \usepackage{pdfsync}
%  \if


%\title{Brief Article}
%\author{David F. Snyder}
%\author{L.G. Meredith}

%\address{Dept. of Math., Texas State University--San Marcos, San Marcos, TX 78666}
       
\pagestyle{empty}


\begin{document}

\lstset{language=[Objective]Caml,frame=shadowbox}

\input{qm2pi.front}

% section front matter (end)

\input{qm2pi.intro} 
 
% section introduction (end)

% \input{qm2pi.knotations} 

% section notation (end)

\input{qm2pi.process.calculi} 

% section concurrent_process_calculi_and_spatial_logics_ (end)
    
%\input{qm2pi.knots2pi} 

%\input{qm2pi.trefoil} 

%\input{qm2pi.mainthm} 

% subsection basic_interpretation (end)

%\input{qm2pi.rho.presentation} 
\subsection{The syntax and semantics of the notation system}\label{sub:the_syntax_and_semantics_of_the_notation_system} % (fold)

We now summarize a technical presentation of the calculus that
embodies our theory of dynamics. The typical presentation of such a
calculus follows the style of giving generators and relations on
them. The grammar, below, describing term constructors, freely
generates the set of processes, $\Proc$. This set is then quotiented
by a relation known as structural congruence and it is over this set
that the notion of dynamics is expressed. This presentation is
essentially that of \cite{MeredithR05} with the addition of
polyadicity and summation. For readability we have relegated some of
the technical subtleties to an appendix.

\subsubsection{Process grammar}\label{subsub:process_grammar}

\begin{mathpar}
  \inferrule* [lab=synchronization] {} {{M} \bc \pzero \;|\; x?F \;|\; x!C }
  \and
  \inferrule* [lab=abstraction] {} {{F} \bc (x)P}
  \and
  \inferrule* [lab=concretion] {} {{C} \bc \langle Q \rangle}
  \and
  \inferrule* [lab=process] {} {{P,Q} \bc M \;| \;P|Q \;|\; @{x}}
  \and
  \inferrule* [lab=name] {} {{x} \bc \quotep{P}}
\end{mathpar} 

Note that $\vec{x}$ (resp. $\vec{P}$) denotes a vector of names
(resp. processes) of length $|\vec{x}|$ (resp. $|\vec{P}|$). We adopt
the following useful abbreviations.

\begin{mathpar}
   x?(\vec{y}).P := x.(\vec{y})P \and  x\clift{\vec{P}} := x.\clift{\vec{P}}
   \and x!(y) := \lift{x}{\dropn{y}}
   \and \Pi_{i=0}^{n-1}P_i := P_0 | \ldots | P_{n-1}
\end{mathpar}

\subsubsection{Structural congruence}

\paragraph{Free and bound names and alpha-equivalence.} At the
core of structural equivalence is alpha-equivalence which identifies
process that are the same up to a change of variable. Formally, we
recognize the distinction between free and bound names. The free names
of a process, $\freenames{P}$, may be calculated recursively as
follows:

\begin{mathpar}
\freenames{\pzero} := \emptyset
  \and \\
  \freenames{x?(y).P} := \{ x \} \cup (\freenames{P} \setminus \{ y \})
  \and 
  \freenames{x!\langle P \rangle} := \{ x \} \cup \{ P \} 
  \and \\
  \freenames{P|Q} := \freenames{P} \cup \freenames{Q}
  \and \\
  \freenames{@{x}} := \{ x \}
\end{mathpar}

$\pi$
$\quotep{\pi}$

$\freenames{-} : \pi \to \mathcal{P}(\quotep{\pi})$

\begin{eqnarray*}
  \freenames{\pzero} & := & \emptyset \\
  \freenames{x?(y).P} & := & \{ x \} \cup (\freenames{P} \setminus \{ y \}) \\
  \freenames{x!\langle P \rangle} & := & \{ x \} \cup \{ P \} \\
  \freenames{P|Q} & := & \freenames{P} \cup \freenames{Q} \\
  \freenames{\dropn{x}} & := & \{ x \}
\end{eqnarray*}

The bound names of a process, $\boundnames{P}$, are those names occurring in $P$
that are not free. For example, in $x?(y).0$, the name $x$ is free, while $y$ is bound.

\begin{mathpar}
  \inferrule* [lab=monoidal-laws] {} { P|Q \equiv Q|P \and P|0 \equiv P \and P|(Q|R) \equiv (P|Q)|R }
\end{mathpar}

\begin{mathpar}
  \inferrule* [lab=alpha-equivalence] {} { (x)P \equiv (y)P\{y/x\} \and y \not\in \freenames{P} }
\end{mathpar}

\begin{definition}
Then two processes, $P,Q$, are alpha-equivalent if $P = Q\{\vec{y}/\vec{x}\}$ for
some $\vec{x} \in \boundnames{Q},\vec{y} \in \boundnames{P}$, where $Q\{\vec{y}/\vec{x}\}$
denotes the capture-avoiding substitution of $\vec{y}$ for $\vec{x}$ in $Q$.
\end{definition}

\begin{definition}
  The {\em structural congruence} \cite{SangiorgiWalker} , $\equiv$,
  between processes is the least congruence containing
  alpha-equivalence, satisfying the abelian monoid laws
  (associativity, commutativity and $\pzero$ as identity) for parallel
  composition $|$ and for summation $+$.
\end{definition}

\subsection{Name equivalence}

We take name equivalence, written $\nameeq$, to be the smallest
equivalence relation generated by the following rules.

\begin{mathpar}
\inferrule*[lab=Quote-drop]
{ }
{ \quotep{@{x}} \nameeq x }

\inferrule*[lab=Struct-equiv]
{ P \scong Q }
{ \quotep{P} \nameeq \quotep{Q} }
\end{mathpar}

The astute reader will have noticed that the mutual recursion of names
and processes imposes a mutual recursion on alpha-equivalence and
structural equivalence via name-equivalence. Fortunately, all of this
works out pleasantly and we may calculate in the natural way, free of
concern. The reader interested in the details is referred to the
appendix \ref{appendix:rho_details}.

\subsection{Substitution}

We use $\Proc$ for the set of processes, $\QProc$ for the set of
names, and $\id{\{}\vec{y} / \vec{x} \id{\}}$ to denote partial maps,
$s : \QProc \rightarrow \QProc$. A map, $s$ lifts, uniquely, to a map
on process terms, $\widehat{s} : \Proc \rightarrow \Proc$ by the
following equations.

\begin{mathpar}
  (0) \psubstp{Q}{P} := 0 \\
  (R \juxtap S) \psubstp{Q}{P}
  :=    
  (R)\psubstp{Q}{P} \juxtap (S) \psubstp{Q}{P} \\
  (x?(y).R) \psubstp{Q}{P}    
  :=    
  (x)\substp{Q}{P} (z)\concat( (R \psubstn{z}{y}) \psubstp{Q}{P} ) \\
  (\lift{x}{R}) \psubstp{Q}{P}  
  :=
  \lift{(x)\substp{Q}{P}}{ R \psubstp{Q}{P} } \\
%   (\dropn{x})  \psubstp{Q}{P}       
%   := 
%   \left\{ 
%     \begin{array}{ccc} 
%       \dropn{\quotep{Q}} & & x \nameeq \quotep{P} \\
%       \dropn{x} & & otherwise \\
%     \end{array}
%   \right. 
  (\dropn{x})  \psubstp{Q}{P}       
  := 
  \left\{ 
    \begin{array}{ccc} 
      Q & & x \nameeq \quotep{P} \\
      \dropn{x} & & otherwise \\
    \end{array}
  \right.
\end{mathpar}
 

where

\begin{eqnarray}
  (x)\id{\{} \lpquote Q \rpquote / \lpquote P \rpquote \id{\}}            = 
  \left\{ 
    \begin{array}{ccc}
      \lpquote Q \rpquote & & x \nameeq \lpquote P \rpquote \\
      x & & otherwise \\
    \end{array}
  \right. \nonumber
\end{eqnarray}

and $z$ is chosen distinct from $\quotep{P}$, $\quotep{Q}$, the free
names in $Q$, and all the names in $R$. Our $\alpha$-equivalence will
be built in the standard way from this substitution.

\begin{remark}\label{rem:no_self_referential_names}
  One consequence of these definitions is that $\forall P. \quotep{P}
  \not\in \freenames{P}$.
\end{remark}

\subsection{ Dynamic quote: an example }

Anticipating something of what's to come, consider applying the
substitution, $\widehat{\id{\{}u / z \id{\}}}$, to the following pair
of processes, $\lift{w}{y!(z)}$ and $w[ \lpquote y!(z) \rpquote ]$.

\begin{eqnarray}
	\lift{w}{y!(z)}\widehat{\id{\{}u / z \id{\}}}
		& = &
		\lift{w}{y!(u)} \nonumber\\
	w[ \lpquote y!(z) \rpquote ] \widehat{ \id{\{}u / z \id{\}} }
		& = &
		w[ \lpquote y!(z) \rpquote ] \nonumber
\end{eqnarray}

Because the body of the process between quotes is impervious to
substitution, we get radically different answers. In fact, by
examining the first process in an input context,
e.g. $x?(z).\lift{w}{y!(z)}$, we see that the process under the lift
operator may be shaped by prefixed inputs binding a name inside it. In
this sense, the lift operator will be seen as a way to dynamically
construct processes before reifying them as names.

Finally equipped with these standard features we can present the
dynamics of the calculus.

\subsubsection{Operational semantics} 

Finally, we introduce the computational dynamics. What marks these
algebras as distinct from other more traditionally studied algebraic
structures, e.g. vector spaces or polynomial rings, is the manner in
which dynamics is captured. In traditional structures, dynamics is typically
expressed through morphisms between such structures, as in linear maps
between vector spaces or morphisms between rings. In algebras
associated with the semantics of computation, the dynamics is
expressed as part of the algebraic structure itself, through a
reduction reduction relation typically denoted by $\red$. Below, we
give a recursive presentation of this relation for the calculus used
in the encoding.

$\red \subseteq \pi \times \pi$
$\red : \pi \to \mathcal{P}(\pi)$

\begin{mathpar}
  \inferrule* [lab=Comm] { \textsf{match}( x_{src}, x_{trgt} ) } { x_{trgt}?(y)P \; | \; x_{src}!\langle {Q} \rangle \red P\{\quotep{Q}/y}\} }
  \and \\
  \inferrule* [lab=Par] {{P} \red {P}'} {{{P} | {Q}} \red {{P}' | {Q}}}
  \and
  \inferrule* [lab=Equiv]{{{P} \scong {P}'} \andalso {{P}' \red {Q}'} \andalso {{Q}' \scong {Q}}}{{P} \red {Q}}
\end{mathpar}

\begin{eqnarray*}
  match_{\equiv} (\quotep{P},\quotep{Q}) & := & P \equiv Q \\
  match_{\dagger}(\quotep{P},\quotep{Q}) & := & \forall R. P|Q \red^{*} R => R \red^{*} 0 \\
  match_{K}(\quotep{P},\quotep{Q}) & := & K \mbox{ for some context } K
\end{eqnarray*}

$u?(x)P | u!\langle Q \rangle \red P\{\quotep{Q}/x\}$

%We write $\wred$ for $\red^*$, and $P\red$ if $\exists Q $ such that $ P \red Q$.
We write $P\red$ if $\exists Q $ such that $ P \red Q$ and $P\not\red$, otherwise.

\section{Replication}

As mentioned before, it is known that replication (and hence
recursion) can be implemented in a higher-order process algebra
\cite{SangiorgiWalker}. As our first example of calculation with the
machinery thus far presented we give the construction explicitly in
the {\rhoc}.

\begin{eqnarray}
	D_{x} & := & \prefix{x}{y}{(\binpar{\outputp{x}{y}}{@{y}})} \nonumber\\
	\bangp_{x}{P} & := & \binpar{{x}!\langle{\binpar{D_{x}}{P}}\rangle}{D_{x}} \nonumber
\end{eqnarray}

\begin{eqnarray}
	\bangp_{x}{P} & & \nonumber\\
	=
	& {x}!\langle{(\prefix{x}{y}{(\outputp{x}{y} | @{y})) | P}}\rangle 
	      | \prefix{x}{y}{(\outputp{x}{y} | @{y})} & \nonumber\\
	\red
	& (\outputp{x}{y} | @{y})\substn{\quotep{(\prefix{x}{y}{(@{y} | \outputp{x}{y})) | P}}}{y} & \nonumber\\
	=
	& \outputp{x}{\quotep{(\prefix{x}{y}{(\outputp{x}{y} | @{y})) | P}}}
	  | {(\prefix{x}{y}{(\outputp{x}{y} | @{y})) | P}} & \nonumber\\
	\red
	& \ldots & \nonumber\\
	\red^*
	& P | P | \ldots & \nonumber
\end{eqnarray}

Of course, this encoding, as an implementation, runs away, unfolding
$\bangp{P}$ eagerly. A lazier and more implementable replication
operator, restricted to input-guarded processes, may be obtained as follows.

\begin{eqnarray}
\bangp{\prefix{u}{v}{P}} 
	:= 
	\binpar{\lift{x}{\prefix{u}{v}{(\binpar{D(x)}{P})}}}{D(x)} \nonumber
\end{eqnarray}

\begin{remark}
  Note that the lazier definition still does not deal with summation
  or mixed summation (i.e. sums over input and output). The reader is
  invited to construct definitions of replication that deal with these
  features. 

  Further, the definitions are parameterized in a name, $x$. Can you,
  gentle reader, make a definition that eliminates this parameter and
  guarantees no accidental interaction between the replication
  machinery and the process being replicated -- i.e. no accidental
  sharing of names used by the process to get its work done and the
  name(s) used by the replication to effect copying. This latter
  revision of the definition of replication is crucial to obtaining
  the expected identity $!!P \sim !P$.
\end{remark}

\begin{remark}\label{rem:paradoxical_combinator}
  The reader familiar with the lambda calculus will have noticed the
  similarity between $D$ and the paradoxical combinator.

  [Ed. note: the existence of this seems to suggest we have to be more
  restrictive on the set of processes and names we admit if we are to
  support no-cloning.]
\end{remark}

\subsubsection{Bisimulation}

The computational dynamics gives rise to another kind of equivalence,
the equivalence of computational behavior. As previously mentioned
this is typically captured \emph{via} some form of bisimulation.

% The notion we use in this paper is weak barbed bisimulation
% \cite{milner91polyadicpi}.

The notion we use in this paper is derived from weak barbed
bisimulation \cite{milner91polyadicpi}. 

\begin{definition}
An \emph{observation relation}, $\downarrow_{\mathcal N}$, over a set
of names, $\mathcal N$, is the smallest relation satisfying the rules
below.

\infrule[Out-barb]{y \in {\mathcal N}, \; x \nameeq y}
		  {\outputp{x}{v} \downarrow_{\mathcal N} x}
\infrule[Par-barb]{\mbox{$P\downarrow_{\mathcal N} x$ or $Q\downarrow_{\mathcal N} x$}}
		  {\binpar{P}{Q} \downarrow_{\mathcal N} x}

We write $P \Downarrow_{\mathcal N} x$ if there is $Q$ such that 
$P \wred Q$ and $Q \downarrow_{\mathcal N} x$.
\end{definition}

\begin{definition}
%\label{def.bbisim}
An  ${\mathcal N}$-\emph{barbed bisimulation} over a set of names, ${\mathcal N}$, is a symmetric binary relation 
${\mathcal S}_{\mathcal N}$ between agents such that $P\rel{S}_{\mathcal N}Q$ implies:
\begin{enumerate}
\item If $P \red P'$ then $Q \wred Q'$ and $P'\rel{S}_{\mathcal N} Q'$.
\item If $P\downarrow_{\mathcal N} x$, then $Q\Downarrow_{\mathcal N} x$.
\end{enumerate}
$P$ is ${\mathcal N}$-barbed bisimilar to $Q$, written
$P \wbbisim_{\mathcal N} Q$, if $P \rel{S}_{\mathcal N} Q$ for some ${\mathcal N}$-barbed bisimulation ${\mathcal S}_{\mathcal N}$.
\end{definition}

$\mathcal{R} \subseteq \pi \times \pi$

$P \mathcal{R} Q => \forall P'. P \red P' \Rightarrow \exists Q'. Q \red Q', P' \mathcal{R} Q'$

$P \vdash x \Rightarrow Q \vdash x$

\begin{mathpar}
  \inferrule*[lab=Out-barb]{x \nameeq y}{{y}!\langle{Q}\rangle \vdash x}
  \and
  \inferrule*[lab=Par-barb]{\mbox{$P\vdash x$ or $Q\vdash x$}}{\binpar{P}{Q} \vdash x}
\end{mathpar}

\subsubsection{Contexts}

One of the principle advantages of computational calculi like the
$\pi$-calculus is a well-defined notion of context,
contextual-equivalence and a correlation between
contextual-equivalence and notions of bisimulation. The notion of
context allows the decomposition of a process into (sub-)process and
its syntactic environment, its context. Thus, a context may be
thought of as a process with a ``hole'' (written $\Box$) in it. The
application of a context $M$ to a process $P$, written $M[P]$, is
tantamount to filling the hole in $M$ with $P$. In this paper we do
not need the full weight of this theory, but do make use of the notion
of context in the proof the main theorem. 

\begin{mathpar}
  \inferrule* [lab=summation] {} {{M_{M},M_{N}} \bc \Box \;|\; x.M_{A} \;|\; M_{M}+M_{N}}
  \and
  \inferrule* [lab=agent] {} {{M_{A}} \bc (\vec{x})M_{P} \;| \; \clift{P_0,\ldots,M_{P},\ldots,P_N}}
  \and \\
  \inferrule* [lab=process] {} {{M_{P}} \bc M_{N} \;| \;P|M_{P} }
\end{mathpar} 

\begin{mathpar}
  \inferrule* [lab=sychronization] {} {M_{N} \bc \Box \;|\; x?M_{F} \;|\; x!M_{C}}
  \and
  \inferrule* [lab=abstraction] {} {{M_{F}} \bc (x)M_{P} }
  \and
  \inferrule* [lab=concretion] {} {{M_{C}} \bc \langle M_{P} \rangle }
  \and \\
  \inferrule* [lab=process] {} {{M_{P}} \bc M_{N} \;| \;P|M_{P} }
\end{mathpar}

\begin{definition}[contextual application] Given a context $M$, and
  process $P$, we define the \emph{contextual application}, $M[P] :=
  M\{P/\Box\}$. That is, the contextual application of M to P is the
  substitution of $P$ for $\Box$ in $M$.
\end{definition}

$\meaningof{-} : L \to \mathcal{P}(\pi)$

\begin{mathpar}
  \inferrule* [lab=collection] {} {\meaningof{true} = \pi, \and \meaningof{~E} = \pi \setminus \meaningof{E}, \and \meaningof{E_{1} \& E_{2}} = \meaningof{E_{1}} \cap \meaningof{E_{2}}}
\end{mathpar}

\begin{mathpar}
  \inferrule* [lab=structure] {} {\meaningof{0} = \{ P \in \pi | P \equiv 0 \}, \and \\ \meaningof{E_1 | E_2} = \{ P \in \pi | P \equiv P_{1} | P_{2}, P_{1} \in \meaningof{E_{1}}, P_{2} \in \meaningof{E_2}\} }
\end{mathpar}

\begin{mathpar}
 \inferrule* [lab=behavior] {} {\meaningof{\langle a?b \rangle E} = \{ P \in \pi | P \equiv Q | u?(y)P', \\ \and \\\\ \and \\ \;\;\; u \in \meaningof{a}, \forall z.P'\{z/y\} \in \meaningof{E\{z/b\}}\}, \and \\ \meaningof{a!E} = \{ P \in \pi | P \equiv Q | x!\langle P' \rangle, x \in \meaningof{a} P' \in \meaningof{E}\} }
\end{mathpar}

\begin{mathpar}
 \inferrule* [lab=nominal] {} {\meaningof{\quotep{E}} = \{ \quotep{P} \in \quotep{\pi} | P \in \meaningof{E} \}, \and \meaningof{\quotep{P}} = \{ \quotep{Q} \in \quotep{\pi} | P \equiv Q \} \and \\ \meaningof{@\quotep{E}} = \{ P \in \pi | P \equiv @x, x \in \meaningof{E} \}}
\end{mathpar}

\begin{eqnarray*}
  \\
  \meaningof{-} : TS \to ST
\end{eqnarray*}

\begin{eqnarray*}
  \\
  L : TS \to ST
\end{eqnarray*}

\begin{eqnarray*}
  \\
  P \models E \iff P \in \meaningof{E}
\end{eqnarray*}

\begin{eqnarray*}
  P \approx_{L} Q \iff \forall E \in L. P \models E \iff Q \models E
\end{eqnarray*}

\begin{eqnarray*}
  P \approx_{K} Q
\end{eqnarray*}

\begin{eqnarray*}
  P \approx Q
\end{eqnarray*}

$\approx_{K} = \approx = \approx_{L}$

\subsubsection{Contextual duality}

Note that contexts extend the quotation operation to a family of
operations from processes to names. Given a context, $M$, we can
define a \emph{nominal context}, $\quotep{M}$ by $\quotep{M}[P] :=
\quotep{M[P]}$. To foreshadow what is to come we observe that these
operations enjoy a duality with processes very much like the duality
between vectors and maps from vectors to scalars.

Further, because the calculus is essentially higher-order, we have a
correspondence between contexts and processes. More specifically,
given a name $x$ and a context $M$ we can construct $M^{*}_{x}$ such
that 

\begin{mathpar}
  M^{*}_{x} | \lift{x}{P} \red M[P]
\end{mathpar}

namely,

\begin{mathpar}
  M^{*}_{x} := x?(u).M[\dropn{u}]
\end{mathpar}

The dependence of $M^{*}_{x}$ on a name makes it an abstraction, 

\begin{mathpar}
  M^{*} := (x)x?(u).M[\dropn{u}]
\end{mathpar}

\subsection{Additional notation}

It will sometimes be convenient to denote the process a name
quotes. We already have the notation $x = \quotep{P}$, but it will be
convenient to introduce an alternate notation, $\procn{x}$, when we
want to emphasize the connection to the use of the name. Note that, by
virtue of name equivalence, $\quotep{\procn{x}} \nameeq x$; so, the
notation is consistent with previous definitions.

Further, because names have structure it is possible to effect
substitutions on the basis of that structure. This means we need to
upgrade our notation for substitutions, which we accomplish by
adapting comprehension notation. Thus,

\begin{mathpar}
  P\{ y / x : x \in S \}
\end{mathpar}

is interpreted to mean the process derived from P by replacing (in a
capture-avoiding manner) each occurrence of $x$ in $S$ by $y$. For example,

\begin{mathpar}
  P\{ \quotep{\procn{x}|\procn{x}} / x : x \in \freenames{P} \}
\end{mathpar}

will replace each (occurrence) of a free name $x$ in $P$ by
$\quotep{\procn{x}|\procn{x}}$.

Also, we will avail ourselves of the notation $x^{L}$ and $x^{R}$ to
denote injections of a name into disjoint copies of the name
space. There are numerous ways to accomplish this. One example can be
found in \cite{MeredithR05}. This notation overloads to vectors of
names: $\vec{x}^{\pi} := (x_{i}^{\pi} \; : \; 0 \leq i < |\vec{x}| )$ where $\pi \in \{L,R\}$.

We also use $P^{\Box} := P|\Box$.

In \cite{MeredithR05} an interpretation of the new operator is
given. It turns out that there are several possible interpretations
all enjoying the requisite algebraic properties of the operator (see
\cite{milner91polyadicpi}). We will therefore make liberal use of
$(\nu\; \vec{x})P$.

% subsection the_syntax_and_semantics_of_the_notation_system (end)   

\input{qm2pi.qmops} 

\input{qm2pi.sterngerlach} 

\input{qm2pi.metric} 

% section concurrent_process_calculi (end)

%\input{qm2pi.proofsketch}

% section proof sketch (end)

%\input{qm2pi.slviaknots} 

% section spatial logic via knots (end)

\input{qm2pi.conclusion}

% section conclusion (end)

%\input{qm2pi.dtcodes} 

% section wiring algorithm (end)

\input{qm2pi.ack} 

% section acknowledgments (end)

\newpage


\bibliographystyle{plain}   
\bibliography{../../biblios/main.bib}

\input{qm2pi.rhodetails}

\end{document}

 

% subsection basic_interpretation (end)

%\input{qm2pi.rho.presentation} 
\subsection{The syntax and semantics of the notation system}\label{sub:the_syntax_and_semantics_of_the_notation_system} % (fold)

We now summarize a technical presentation of the calculus that
embodies our theory of dynamics. The typical presentation of such a
calculus follows the style of giving generators and relations on
them. The grammar, below, describing term constructors, freely
generates the set of processes, $\Proc$. This set is then quotiented
by a relation known as structural congruence and it is over this set
that the notion of dynamics is expressed. This presentation is
essentially that of \cite{MeredithR05} with the addition of
polyadicity and summation. For readability we have relegated some of
the technical subtleties to an appendix.

\subsubsection{Process grammar}\label{subsub:process_grammar}

\begin{mathpar}
  \inferrule* [lab=synchronization] {} {{M} \bc \pzero \;|\; x?F \;|\; x!C }
  \and
  \inferrule* [lab=abstraction] {} {{F} \bc (x)P}
  \and
  \inferrule* [lab=concretion] {} {{C} \bc \langle Q \rangle}
  \and
  \inferrule* [lab=process] {} {{P,Q} \bc M \;| \;P|Q \;|\; @{x}}
  \and
  \inferrule* [lab=name] {} {{x} \bc \quotep{P}}
\end{mathpar} 

Note that $\vec{x}$ (resp. $\vec{P}$) denotes a vector of names
(resp. processes) of length $|\vec{x}|$ (resp. $|\vec{P}|$). We adopt
the following useful abbreviations.

\begin{mathpar}
   x?(\vec{y}).P := x.(\vec{y})P \and  x\clift{\vec{P}} := x.\clift{\vec{P}}
   \and x!(y) := \lift{x}{\dropn{y}}
   \and \Pi_{i=0}^{n-1}P_i := P_0 | \ldots | P_{n-1}
\end{mathpar}

\subsubsection{Structural congruence}

\paragraph{Free and bound names and alpha-equivalence.} At the
core of structural equivalence is alpha-equivalence which identifies
process that are the same up to a change of variable. Formally, we
recognize the distinction between free and bound names. The free names
of a process, $\freenames{P}$, may be calculated recursively as
follows:

\begin{mathpar}
\freenames{\pzero} := \emptyset
  \and \\
  \freenames{x?(y).P} := \{ x \} \cup (\freenames{P} \setminus \{ y \})
  \and 
  \freenames{x!\langle P \rangle} := \{ x \} \cup \{ P \} 
  \and \\
  \freenames{P|Q} := \freenames{P} \cup \freenames{Q}
  \and \\
  \freenames{@{x}} := \{ x \}
\end{mathpar}

$\pi$
$\quotep{\pi}$

$\freenames{-} : \pi \to \mathcal{P}(\quotep{\pi})$

\begin{eqnarray*}
  \freenames{\pzero} & := & \emptyset \\
  \freenames{x?(y).P} & := & \{ x \} \cup (\freenames{P} \setminus \{ y \}) \\
  \freenames{x!\langle P \rangle} & := & \{ x \} \cup \{ P \} \\
  \freenames{P|Q} & := & \freenames{P} \cup \freenames{Q} \\
  \freenames{\dropn{x}} & := & \{ x \}
\end{eqnarray*}

The bound names of a process, $\boundnames{P}$, are those names occurring in $P$
that are not free. For example, in $x?(y).0$, the name $x$ is free, while $y$ is bound.

\begin{mathpar}
  \inferrule* [lab=monoidal-laws] {} { P|Q \equiv Q|P \and P|0 \equiv P \and P|(Q|R) \equiv (P|Q)|R }
\end{mathpar}

\begin{mathpar}
  \inferrule* [lab=alpha-equivalence] {} { (x)P \equiv (y)P\{y/x\} \and y \not\in \freenames{P} }
\end{mathpar}

\begin{definition}
Then two processes, $P,Q$, are alpha-equivalent if $P = Q\{\vec{y}/\vec{x}\}$ for
some $\vec{x} \in \boundnames{Q},\vec{y} \in \boundnames{P}$, where $Q\{\vec{y}/\vec{x}\}$
denotes the capture-avoiding substitution of $\vec{y}$ for $\vec{x}$ in $Q$.
\end{definition}

\begin{definition}
  The {\em structural congruence} \cite{SangiorgiWalker} , $\equiv$,
  between processes is the least congruence containing
  alpha-equivalence, satisfying the abelian monoid laws
  (associativity, commutativity and $\pzero$ as identity) for parallel
  composition $|$ and for summation $+$.
\end{definition}

\subsection{Name equivalence}

We take name equivalence, written $\nameeq$, to be the smallest
equivalence relation generated by the following rules.

\begin{mathpar}
\inferrule*[lab=Quote-drop]
{ }
{ \quotep{@{x}} \nameeq x }

\inferrule*[lab=Struct-equiv]
{ P \scong Q }
{ \quotep{P} \nameeq \quotep{Q} }
\end{mathpar}

The astute reader will have noticed that the mutual recursion of names
and processes imposes a mutual recursion on alpha-equivalence and
structural equivalence via name-equivalence. Fortunately, all of this
works out pleasantly and we may calculate in the natural way, free of
concern. The reader interested in the details is referred to the
appendix \ref{appendix:rho_details}.

\subsection{Substitution}

We use $\Proc$ for the set of processes, $\QProc$ for the set of
names, and $\id{\{}\vec{y} / \vec{x} \id{\}}$ to denote partial maps,
$s : \QProc \rightarrow \QProc$. A map, $s$ lifts, uniquely, to a map
on process terms, $\widehat{s} : \Proc \rightarrow \Proc$ by the
following equations.

\begin{mathpar}
  (0) \psubstp{Q}{P} := 0 \\
  (R \juxtap S) \psubstp{Q}{P}
  :=    
  (R)\psubstp{Q}{P} \juxtap (S) \psubstp{Q}{P} \\
  (x?(y).R) \psubstp{Q}{P}    
  :=    
  (x)\substp{Q}{P} (z)\concat( (R \psubstn{z}{y}) \psubstp{Q}{P} ) \\
  (\lift{x}{R}) \psubstp{Q}{P}  
  :=
  \lift{(x)\substp{Q}{P}}{ R \psubstp{Q}{P} } \\
%   (\dropn{x})  \psubstp{Q}{P}       
%   := 
%   \left\{ 
%     \begin{array}{ccc} 
%       \dropn{\quotep{Q}} & & x \nameeq \quotep{P} \\
%       \dropn{x} & & otherwise \\
%     \end{array}
%   \right. 
  (\dropn{x})  \psubstp{Q}{P}       
  := 
  \left\{ 
    \begin{array}{ccc} 
      Q & & x \nameeq \quotep{P} \\
      \dropn{x} & & otherwise \\
    \end{array}
  \right.
\end{mathpar}
 

where

\begin{eqnarray}
  (x)\id{\{} \lpquote Q \rpquote / \lpquote P \rpquote \id{\}}            = 
  \left\{ 
    \begin{array}{ccc}
      \lpquote Q \rpquote & & x \nameeq \lpquote P \rpquote \\
      x & & otherwise \\
    \end{array}
  \right. \nonumber
\end{eqnarray}

and $z$ is chosen distinct from $\quotep{P}$, $\quotep{Q}$, the free
names in $Q$, and all the names in $R$. Our $\alpha$-equivalence will
be built in the standard way from this substitution.

\begin{remark}\label{rem:no_self_referential_names}
  One consequence of these definitions is that $\forall P. \quotep{P}
  \not\in \freenames{P}$.
\end{remark}

\subsection{ Dynamic quote: an example }

Anticipating something of what's to come, consider applying the
substitution, $\widehat{\id{\{}u / z \id{\}}}$, to the following pair
of processes, $\lift{w}{y!(z)}$ and $w[ \lpquote y!(z) \rpquote ]$.

\begin{eqnarray}
	\lift{w}{y!(z)}\widehat{\id{\{}u / z \id{\}}}
		& = &
		\lift{w}{y!(u)} \nonumber\\
	w[ \lpquote y!(z) \rpquote ] \widehat{ \id{\{}u / z \id{\}} }
		& = &
		w[ \lpquote y!(z) \rpquote ] \nonumber
\end{eqnarray}

Because the body of the process between quotes is impervious to
substitution, we get radically different answers. In fact, by
examining the first process in an input context,
e.g. $x?(z).\lift{w}{y!(z)}$, we see that the process under the lift
operator may be shaped by prefixed inputs binding a name inside it. In
this sense, the lift operator will be seen as a way to dynamically
construct processes before reifying them as names.

Finally equipped with these standard features we can present the
dynamics of the calculus.

\subsubsection{Operational semantics} 

Finally, we introduce the computational dynamics. What marks these
algebras as distinct from other more traditionally studied algebraic
structures, e.g. vector spaces or polynomial rings, is the manner in
which dynamics is captured. In traditional structures, dynamics is typically
expressed through morphisms between such structures, as in linear maps
between vector spaces or morphisms between rings. In algebras
associated with the semantics of computation, the dynamics is
expressed as part of the algebraic structure itself, through a
reduction reduction relation typically denoted by $\red$. Below, we
give a recursive presentation of this relation for the calculus used
in the encoding.

$\red \subseteq \pi \times \pi$
$\red : \pi \to \mathcal{P}(\pi)$

\begin{mathpar}
  \inferrule* [lab=Comm] { \textsf{match}( x_{src}, x_{trgt} ) } { x_{trgt}?(y)P \; | \; x_{src}!\langle {Q} \rangle \red P\{\quotep{Q}/y}\} }
  \and \\
  \inferrule* [lab=Par] {{P} \red {P}'} {{{P} | {Q}} \red {{P}' | {Q}}}
  \and
  \inferrule* [lab=Equiv]{{{P} \scong {P}'} \andalso {{P}' \red {Q}'} \andalso {{Q}' \scong {Q}}}{{P} \red {Q}}
\end{mathpar}

\begin{eqnarray*}
  match_{\equiv} (\quotep{P},\quotep{Q}) & := & P \equiv Q \\
  match_{\dagger}(\quotep{P},\quotep{Q}) & := & \forall R. P|Q \red^{*} R => R \red^{*} 0 \\
  match_{K}(\quotep{P},\quotep{Q}) & := & K \mbox{ for some context } K
\end{eqnarray*}

$u?(x)P | u!\langle Q \rangle \red P\{\quotep{Q}/x\}$

%We write $\wred$ for $\red^*$, and $P\red$ if $\exists Q $ such that $ P \red Q$.
We write $P\red$ if $\exists Q $ such that $ P \red Q$ and $P\not\red$, otherwise.

\section{Replication}

As mentioned before, it is known that replication (and hence
recursion) can be implemented in a higher-order process algebra
\cite{SangiorgiWalker}. As our first example of calculation with the
machinery thus far presented we give the construction explicitly in
the {\rhoc}.

\begin{eqnarray}
	D_{x} & := & \prefix{x}{y}{(\binpar{\outputp{x}{y}}{@{y}})} \nonumber\\
	\bangp_{x}{P} & := & \binpar{{x}!\langle{\binpar{D_{x}}{P}}\rangle}{D_{x}} \nonumber
\end{eqnarray}

\begin{eqnarray}
	\bangp_{x}{P} & & \nonumber\\
	=
	& {x}!\langle{(\prefix{x}{y}{(\outputp{x}{y} | @{y})) | P}}\rangle 
	      | \prefix{x}{y}{(\outputp{x}{y} | @{y})} & \nonumber\\
	\red
	& (\outputp{x}{y} | @{y})\substn{\quotep{(\prefix{x}{y}{(@{y} | \outputp{x}{y})) | P}}}{y} & \nonumber\\
	=
	& \outputp{x}{\quotep{(\prefix{x}{y}{(\outputp{x}{y} | @{y})) | P}}}
	  | {(\prefix{x}{y}{(\outputp{x}{y} | @{y})) | P}} & \nonumber\\
	\red
	& \ldots & \nonumber\\
	\red^*
	& P | P | \ldots & \nonumber
\end{eqnarray}

Of course, this encoding, as an implementation, runs away, unfolding
$\bangp{P}$ eagerly. A lazier and more implementable replication
operator, restricted to input-guarded processes, may be obtained as follows.

\begin{eqnarray}
\bangp{\prefix{u}{v}{P}} 
	:= 
	\binpar{\lift{x}{\prefix{u}{v}{(\binpar{D(x)}{P})}}}{D(x)} \nonumber
\end{eqnarray}

\begin{remark}
  Note that the lazier definition still does not deal with summation
  or mixed summation (i.e. sums over input and output). The reader is
  invited to construct definitions of replication that deal with these
  features. 

  Further, the definitions are parameterized in a name, $x$. Can you,
  gentle reader, make a definition that eliminates this parameter and
  guarantees no accidental interaction between the replication
  machinery and the process being replicated -- i.e. no accidental
  sharing of names used by the process to get its work done and the
  name(s) used by the replication to effect copying. This latter
  revision of the definition of replication is crucial to obtaining
  the expected identity $!!P \sim !P$.
\end{remark}

\begin{remark}\label{rem:paradoxical_combinator}
  The reader familiar with the lambda calculus will have noticed the
  similarity between $D$ and the paradoxical combinator.

  [Ed. note: the existence of this seems to suggest we have to be more
  restrictive on the set of processes and names we admit if we are to
  support no-cloning.]
\end{remark}

\subsubsection{Bisimulation}

The computational dynamics gives rise to another kind of equivalence,
the equivalence of computational behavior. As previously mentioned
this is typically captured \emph{via} some form of bisimulation.

% The notion we use in this paper is weak barbed bisimulation
% \cite{milner91polyadicpi}.

The notion we use in this paper is derived from weak barbed
bisimulation \cite{milner91polyadicpi}. 

\begin{definition}
An \emph{observation relation}, $\downarrow_{\mathcal N}$, over a set
of names, $\mathcal N$, is the smallest relation satisfying the rules
below.

\infrule[Out-barb]{y \in {\mathcal N}, \; x \nameeq y}
		  {\outputp{x}{v} \downarrow_{\mathcal N} x}
\infrule[Par-barb]{\mbox{$P\downarrow_{\mathcal N} x$ or $Q\downarrow_{\mathcal N} x$}}
		  {\binpar{P}{Q} \downarrow_{\mathcal N} x}

We write $P \Downarrow_{\mathcal N} x$ if there is $Q$ such that 
$P \wred Q$ and $Q \downarrow_{\mathcal N} x$.
\end{definition}

\begin{definition}
%\label{def.bbisim}
An  ${\mathcal N}$-\emph{barbed bisimulation} over a set of names, ${\mathcal N}$, is a symmetric binary relation 
${\mathcal S}_{\mathcal N}$ between agents such that $P\rel{S}_{\mathcal N}Q$ implies:
\begin{enumerate}
\item If $P \red P'$ then $Q \wred Q'$ and $P'\rel{S}_{\mathcal N} Q'$.
\item If $P\downarrow_{\mathcal N} x$, then $Q\Downarrow_{\mathcal N} x$.
\end{enumerate}
$P$ is ${\mathcal N}$-barbed bisimilar to $Q$, written
$P \wbbisim_{\mathcal N} Q$, if $P \rel{S}_{\mathcal N} Q$ for some ${\mathcal N}$-barbed bisimulation ${\mathcal S}_{\mathcal N}$.
\end{definition}

$\mathcal{R} \subseteq \pi \times \pi$

$P \mathcal{R} Q => \forall P'. P \red P' \Rightarrow \exists Q'. Q \red Q', P' \mathcal{R} Q'$

$P \vdash x \Rightarrow Q \vdash x$

\begin{mathpar}
  \inferrule*[lab=Out-barb]{x \nameeq y}{{y}!\langle{Q}\rangle \vdash x}
  \and
  \inferrule*[lab=Par-barb]{\mbox{$P\vdash x$ or $Q\vdash x$}}{\binpar{P}{Q} \vdash x}
\end{mathpar}

\subsubsection{Contexts}

One of the principle advantages of computational calculi like the
$\pi$-calculus is a well-defined notion of context,
contextual-equivalence and a correlation between
contextual-equivalence and notions of bisimulation. The notion of
context allows the decomposition of a process into (sub-)process and
its syntactic environment, its context. Thus, a context may be
thought of as a process with a ``hole'' (written $\Box$) in it. The
application of a context $M$ to a process $P$, written $M[P]$, is
tantamount to filling the hole in $M$ with $P$. In this paper we do
not need the full weight of this theory, but do make use of the notion
of context in the proof the main theorem. 

\begin{mathpar}
  \inferrule* [lab=summation] {} {{M_{M},M_{N}} \bc \Box \;|\; x.M_{A} \;|\; M_{M}+M_{N}}
  \and
  \inferrule* [lab=agent] {} {{M_{A}} \bc (\vec{x})M_{P} \;| \; \clift{P_0,\ldots,M_{P},\ldots,P_N}}
  \and \\
  \inferrule* [lab=process] {} {{M_{P}} \bc M_{N} \;| \;P|M_{P} }
\end{mathpar} 

\begin{mathpar}
  \inferrule* [lab=sychronization] {} {M_{N} \bc \Box \;|\; x?M_{F} \;|\; x!M_{C}}
  \and
  \inferrule* [lab=abstraction] {} {{M_{F}} \bc (x)M_{P} }
  \and
  \inferrule* [lab=concretion] {} {{M_{C}} \bc \langle M_{P} \rangle }
  \and \\
  \inferrule* [lab=process] {} {{M_{P}} \bc M_{N} \;| \;P|M_{P} }
\end{mathpar}

\begin{definition}[contextual application] Given a context $M$, and
  process $P$, we define the \emph{contextual application}, $M[P] :=
  M\{P/\Box\}$. That is, the contextual application of M to P is the
  substitution of $P$ for $\Box$ in $M$.
\end{definition}

$\meaningof{-} : L \to \mathcal{P}(\pi)$

\begin{mathpar}
  \inferrule* [lab=collection] {} {\meaningof{true} = \pi, \and \meaningof{~E} = \pi \setminus \meaningof{E}, \and \meaningof{E_{1} \& E_{2}} = \meaningof{E_{1}} \cap \meaningof{E_{2}}}
\end{mathpar}

\begin{mathpar}
  \inferrule* [lab=structure] {} {\meaningof{0} = \{ P \in \pi | P \equiv 0 \}, \and \\ \meaningof{E_1 | E_2} = \{ P \in \pi | P \equiv P_{1} | P_{2}, P_{1} \in \meaningof{E_{1}}, P_{2} \in \meaningof{E_2}\} }
\end{mathpar}

\begin{mathpar}
 \inferrule* [lab=behavior] {} {\meaningof{\langle a?b \rangle E} = \{ P \in \pi | P \equiv Q | u?(y)P', \\ \and \\\\ \and \\ \;\;\; u \in \meaningof{a}, \forall z.P'\{z/y\} \in \meaningof{E\{z/b\}}\}, \and \\ \meaningof{a!E} = \{ P \in \pi | P \equiv Q | x!\langle P' \rangle, x \in \meaningof{a} P' \in \meaningof{E}\} }
\end{mathpar}

\begin{mathpar}
 \inferrule* [lab=nominal] {} {\meaningof{\quotep{E}} = \{ \quotep{P} \in \quotep{\pi} | P \in \meaningof{E} \}, \and \meaningof{\quotep{P}} = \{ \quotep{Q} \in \quotep{\pi} | P \equiv Q \} \and \\ \meaningof{@\quotep{E}} = \{ P \in \pi | P \equiv @x, x \in \meaningof{E} \}}
\end{mathpar}

\begin{eqnarray*}
  \\
  \meaningof{-} : TS \to ST
\end{eqnarray*}

\begin{eqnarray*}
  \\
  L : TS \to ST
\end{eqnarray*}

\begin{eqnarray*}
  \\
  P \models E \iff P \in \meaningof{E}
\end{eqnarray*}

\begin{eqnarray*}
  P \approx_{L} Q \iff \forall E \in L. P \models E \iff Q \models E
\end{eqnarray*}

\begin{eqnarray*}
  P \approx_{K} Q
\end{eqnarray*}

\begin{eqnarray*}
  P \approx Q
\end{eqnarray*}

$\approx_{K} = \approx = \approx_{L}$

\subsubsection{Contextual duality}

Note that contexts extend the quotation operation to a family of
operations from processes to names. Given a context, $M$, we can
define a \emph{nominal context}, $\quotep{M}$ by $\quotep{M}[P] :=
\quotep{M[P]}$. To foreshadow what is to come we observe that these
operations enjoy a duality with processes very much like the duality
between vectors and maps from vectors to scalars.

Further, because the calculus is essentially higher-order, we have a
correspondence between contexts and processes. More specifically,
given a name $x$ and a context $M$ we can construct $M^{*}_{x}$ such
that 

\begin{mathpar}
  M^{*}_{x} | \lift{x}{P} \red M[P]
\end{mathpar}

namely,

\begin{mathpar}
  M^{*}_{x} := x?(u).M[\dropn{u}]
\end{mathpar}

The dependence of $M^{*}_{x}$ on a name makes it an abstraction, 

\begin{mathpar}
  M^{*} := (x)x?(u).M[\dropn{u}]
\end{mathpar}

\subsection{Additional notation}

It will sometimes be convenient to denote the process a name
quotes. We already have the notation $x = \quotep{P}$, but it will be
convenient to introduce an alternate notation, $\procn{x}$, when we
want to emphasize the connection to the use of the name. Note that, by
virtue of name equivalence, $\quotep{\procn{x}} \nameeq x$; so, the
notation is consistent with previous definitions.

Further, because names have structure it is possible to effect
substitutions on the basis of that structure. This means we need to
upgrade our notation for substitutions, which we accomplish by
adapting comprehension notation. Thus,

\begin{mathpar}
  P\{ y / x : x \in S \}
\end{mathpar}

is interpreted to mean the process derived from P by replacing (in a
capture-avoiding manner) each occurrence of $x$ in $S$ by $y$. For example,

\begin{mathpar}
  P\{ \quotep{\procn{x}|\procn{x}} / x : x \in \freenames{P} \}
\end{mathpar}

will replace each (occurrence) of a free name $x$ in $P$ by
$\quotep{\procn{x}|\procn{x}}$.

Also, we will avail ourselves of the notation $x^{L}$ and $x^{R}$ to
denote injections of a name into disjoint copies of the name
space. There are numerous ways to accomplish this. One example can be
found in \cite{MeredithR05}. This notation overloads to vectors of
names: $\vec{x}^{\pi} := (x_{i}^{\pi} \; : \; 0 \leq i < |\vec{x}| )$ where $\pi \in \{L,R\}$.

We also use $P^{\Box} := P|\Box$.

In \cite{MeredithR05} an interpretation of the new operator is
given. It turns out that there are several possible interpretations
all enjoying the requisite algebraic properties of the operator (see
\cite{milner91polyadicpi}). We will therefore make liberal use of
$(\nu\; \vec{x})P$.

% subsection the_syntax_and_semantics_of_the_notation_system (end)   

\section{Interpretation of QM}
\subsection{Supporting definitions}
\subsubsection{Multiplication}
\begin{mathpar}
  \quotep{Q} \cdot \quotep{R} := \quotep{Q|R}
  \and \\
  \quotep{Q} \cdot P := P\{ \quotep{Q|R} / \quotep{R} : \quotep{R} \in \freenames{P} \}
\end{mathpar}

\paragraph{Discussion}
The first line needs little explanation. The second line says that
each free name of the process is replaced with the multiplication of
that name by the scalar. Multiplication of a scalar (name) by a state
(process) results in a process all the names of which have been `moved
over' by parallel composition with the process the scalar
quotes. There is a subtlety that the bound names have to be
manipulated so that multiplied names aren't accidentally
captured. There are many ways to achieve this.

\begin{remark}\label{rem:multiplication_identities}
  The reader is invited to verify that for all $x,y,z \in \QProc$ and $P \in \Proc$
  \begin{mathpar}
    x \cdot \quotep{0} \equiv x 
    \and
    x \cdot y \equiv y \cdot x
    \and
    x \cdot (y \cdot z) \equiv (x \cdot y) \cdot z
    \and \\
    \quotep{0} \cdot P \equiv P
    \and \\
    x \cdot (y \cdot P) \equiv (x \cdot y) \cdot P
    \and \\
    x \cdot (P|Q) \equiv (x \cdot P) | (x \cdot Q)
    \and \\    
  \end{mathpar}
\end{remark}

\subsubsection{Tensor product}

We define a tensor product on processes by structural induction.

\paragraph{Tensor of sums} First note that all summations, including
$\pzero$ and sequence, can be written $\Sigma_{i} x_{i}.A_{i} +
\Sigma_{j} x_{j}.C_{j}$, where we have grouped input-guarded processes
together and output-guarded processes together.

Thus, we can define the tensor product of two summations, $N_{1}\otimes N_{2}$, where

\begin{mathpar}
  N_{1} := \Sigma_{i} x_{i}.A_{i} + \Sigma_{j} x_{j}.C_{j}
  \and
  N_{2} := \Sigma_{i'} y_{i'}.B_{i'} + \Sigma_{j'} y_{j'}.D_{j'} 
\end{mathpar}

as follows.

\begin{mathpar}
  \Sigma_{i} x_{i}.A_{i} + \Sigma_{j} x_{j}.C_{j} \otimes \Sigma_{i'}
  y_{i'}.B_{i'} + \Sigma_{j'} y_{j'}.D_{j'} 
  \and \\
  := \; \Sigma_{i} \Sigma_{i'} \quotep{\stackrel{\vee}{x_{i}}| \stackrel{\vee}{y_{i'}}}.(A_{i}\otimes B_{i'}) \; | \; \Sigma_{i'} \Sigma_{i} \quotep{\stackrel{\vee}{y_{i'}}|\stackrel{\vee}{x_{i}}}.(B_{i'}\otimes A_{i})
  \and
  \;\; | \;\; \Sigma_{j} \Sigma_{j'} \quotep{\stackrel{\vee}{x_{j}}|\stackrel{\vee}{y_{j'}}}.(A_{j}\otimes B_{j'}) \; | \; \Sigma_{j'} \Sigma_{j} \quotep{\stackrel{\vee}{y_{j'}}|\stackrel{\vee}{x_{j}}}.(B_{j'}\otimes A_{j})
\end{mathpar}

\begin{remark}
  Do we need to $x^{L}$ and $y^{R}$ for this construction as well?
\end{remark}

\paragraph{Tensor of parallel compositions} Next, we distribute tensor
over par.

\begin{mathpar}
  P_{1}|P_{2} \otimes Q_{1}|Q_{2} := (P_{1} \otimes Q_{1}) | (P_{1}
  \otimes Q_{2}) | (P_{2} \otimes Q_{1}) | (P_{2} \otimes Q_{2})
\end{mathpar}

\paragraph{Tensor with dropped names} We treat tensor of a
process with a dropped name as parallel composition.

\begin{mathpar}
  P \otimes \dropn{x} := P | \dropn{x}
\end{mathpar}

\paragraph{Tensor of agents}

Finally, we need to define tensor on agents. Note that the definition
of tensor on normal products only tensors inputs with inputs and
outputs with outputs. Thus, we only have to define the operation on
``homogeneous'' pairings.

\begin{mathpar}
  (\vec{x})P \otimes (\vec{y})Q
  \and \\
  := (x_{0}^{L}|y_{0}^{R},\ldots,x_{0}^{L}|y_{n}^{R},\ldots,x_{m}^{L}|y_{0}^{R},\ldots,x_{m}^{L}|y_{n}^R)(P\{ \vec{x}^{L}/\vec{x}\} \otimes Q \{ \vec{y}^{R}/\vec{y}\})
  \and \\
  \clift{\vec{P}} \otimes \clift{\vec{Q}}
  \and \\
  := \clift{P_{0}\otimes Q_{0},\ldots,P_{0}\otimes Q_{n},\ldots,P_{m}\otimes Q_{0},\ldots,P_{m}\otimes Q_{n}}
\end{mathpar}

\begin{remark}
  Observe that arities of tensored abstractions matches arities of
  tensored concretions if the original arities matched. Note also that
  the length of the arities corresponds to the increase in dimension
  we see in ordinary vector space tensor product.
\end{remark}

\begin{remark}
  Operationally, this definition distributes the tensor down to
  components ``linked'' by summation. Tensor over summation is
  intriguing in that it mixes names. Moreover, as a consequence of the
  way it mixes names we have the identities for all $x \in \QProc$ and
  $P,Q \in \Proc$

  \begin{mathpar}
    (x \cdot P) \otimes Q \equiv x \cdot (P \otimes Q) \equiv P \otimes (x \cdot Q)
    \and
    P \otimes \pzero \equiv P
  \end{mathpar}

  that the reader is invited to verify.
\end{remark}

\subsubsection{Annihilation}
\begin{mathpar}
  P^{\perp} := \{ Q | \forall R. P|Q \red^{*} R \Rightarrow R \red^{*} \pzero \}
  \and \\
  P^{\underline{\perp}} := \Sigma_{Q \in P^{\perp}} \quotep{Q}?(y).(\dropn{y}|Q) | \Sigma_{Q \in P^{\perp}} \quotep{Q}\clift{\Box}
\end{mathpar}

\paragraph{Discussion} The reader will note that $P^{\perp}$ is a
\emph{set} of processes, while $P^{\underline{\perp}}$ is a
\emph{context}. We call the set $P^{\perp}$ the \emph{annihilators} of
$P$. The parallel composition of a process in the annihilators of $P$
with $P$ will result in a process, the state space of which has all
paths eventually leading to $\pzero$. Execution may endure loops; but
under reasonable conditions of fairness (naturally guaranteed under
most notions of bisimulation) such a composite process cannot get
stuck in such a loop and will, eventually pop out and terminate.

The context $P^{\underline{\perp}}$ is ready and willing to ``take the
$P$ out of'' the process to which it is applied. It will effectively
transmit the code of the process to which it is applied to one of the
annihilators and run the process against it.

\subsubsection{Evaluation}
We fix $M$ a domain of fully abstract interpretation with an equality
coincident with bisimulation. We take $\meaningof{\cdot} : \Proc \to
M$ to be the map interpreting processes and $\nmeaningof{\cdot} : \M
\to Proc$ to be the map running the other way. Then we define

\begin{mathpar}
  \int P := \nmeaningof{\meaningof{P}}
\end{mathpar}

\paragraph{Discussion}
There are many fully abstract interpretations of Milner's
$\pi$-calculus. Any of them can be used as a basis for interpreting
the reflective calculus here. Equipped with such a domain it is
largely a matter of grinding through to check that the Yoneda
construction for the normalization-by-evaluation program can be
extended to this setting.

\begin{remark}
  The reader is invited to verify that $\int (P^{\underline{\perp}}[P]) = 0$.
\end{remark}

\subsection{Quantum mechanics}

Table \ref{tbl:core_qm_op_defns} gives the core operational definitions

\begin{table}[htp]\label{tbl:core_qm_op_defns}
  \center{
    \fbox{
      \begin{tabular}{c|c}
        quantum mechanics & process calculus \\
        \hline
        scalar & $x := \quotep{P}$ \\
        state vector & $\state{P} := P$ \\
        dual & $\state{P}^{*} := \event{P^{\underline{\perp}}} := \quotep{P^{\underline{\perp}}}[-]$ \\
        matrix & $ \Sigma_{\alpha} \state{P_{\alpha}}x_{\alpha}\event{Q_{\alpha}}$ \\
        vector addition & $\state{P} + \state{Q} := \state{P | Q}$ \\
        tensor product & $\state{P} \otimes \state{Q} := \state{P \otimes Q}$ \\
        inner product & $\innerprod{P}{Q} := \quotep{\int P^{\underline{\perp}}[Q]}$ \\
      \end{tabular}
    }
  }
  \caption{QM - operational definitions}
\end{table}

where

\begin{mathpar}
  \prmatrix{P}{Q} := \fprmatrix{P}{\quotep{\pzero}}{Q}
  \and
  \fprmatrix{P}{x}{Q} := (\state{P},x,\event{Q})
  \and
  (\fprmatrix{P}{x}{Q})(\state{R}) := x \cdot \innerprod{Q}{R} \cdot \state{P}
  \and
  (\fprmatrix{P}{x}{Q})(\event{R}) := x \cdot \innerprod{R}{P} \cdot \event{Q}
\end{mathpar}

\paragraph{Discussion}
As promised: vectors (aka states) are represented as processes; duals
as contextual duals; inner product definition should be compared with
standard inner product definition for ....

\begin{remark}
  Assuming $\int (P^{\underline{\perp}}[P]) = 0$, the reader is
  invited to verify that $(\fprmatrix{P}{x}{P})(\state{P}) = x \cdot \state{P}$.
\end{remark}

\begin{remark}
  The reader is invited to verify that $\innerprod{P}{Q}$ could
  equally well have been written $\quotep{\int \stackrel{\vee}{x}}$
  where $x = \event{P^{\underline{\perp}}}(Q)$.

  One of the motivations for this remark is that there is another way
  to factor these operations. We could package up evaluation in the dual:

  \begin{mathpar}
    \state{P}^{*} := \event{\int P^{\underline{\perp}}} := \quotep{\int P^{\underline{\perp}}}[-]
  \end{mathpar}

  and then have inner product defined by
  
  \begin{mathpar}
    \innerprod{P}{Q} := \event{P}(Q)
  \end{mathpar}

  Hopefully, experience with the calculations will provide guidance on
  the best factoring.
\end{remark}

\begin{remark}
  Assuming $\int (P^{\underline{\perp}}[P]) = 0$, the reader is
  invited to verify that $\forall P,Q. (\prmatrix{0}{Q})(\state{0}) =
  \state{0}$ and dually $(\prmatrix{P}{0})(\event{0}) = \event{0}$.
\end{remark}

\begin{remark}
  i'm a little worried that i don't (yet) have proper support for
  complex conjugacy. But, the observation above may give us a
  clue. According to Abramsky, it must be the case that the scalars
  are iso to the homset of the identity for the tensor -- which the
  observation above characterizes. 

  For now, we will simply bookmark the notion with $\overline{x}$.
\end{remark}

\subsubsection{Adjointness}

We need to give a definition of $(\cdot)^{\dagger}$ for matrices. The
obvious candidate definition is
\begin{mathpar}
(\Sigma_{\alpha}\fprmatrix{P_{\alpha}}{x_{\alpha}}{Q_{\alpha}})^{\dagger}
= \Sigma_{\alpha}\fprmatrix{(Q_{\alpha}^{\underline{\perp}})^{*}}{\overline{x}_{\alpha}}{P_{\alpha}^{\underline{\perp}}} 
\end{mathpar}

But, $(Q_{\alpha}^{\underline{\perp}})^{*}$ requires a name along
which to communicate the process to achieve the context application.

\subsubsection{Basis for a basis}
If processes label states and ``addition'' of states (a.k.a. vector
addition) is interpreted as parallel composition, what corresponds to
notions of linear independence and basis? Here, we recall that Yoshida
has developed a set of \emph{combinators} for an asynchronous verison
of Milner's $\pi$-calculus. These are a finite set of processes such
any process can be expressed as parallel composition of these
combinators together with liberal uses of the new operator and
replication. We can simply give a translation of these into the
present calculus and have reasonable expectation that the property
carries over. That is, that the resultant set allows to express all
processes via parallel composition. Note, however, that there is no
new operator or replication in this calculus. As a result, we expect
that the corresponding set is actually infinite. That is, we expect
that the space is actually infinite dimensional.

\begin{remark}
  The attentive reader may be a bit concerned. Certainly, the
  collection $S$, $K$ and $I$ is a finite set of
  combinators. Shouldn't we expect to see a finite set of combinators
  for an effectively equivalent system? i am very sympathetic to this
  critique and feel it warrants full attention. On the other hand, i
  also have in mind the following analogy. The natural numbers, as a
  monoid under addition, has exactly $1$ generator, while the natural
  numbers, as a monoid under multiplication, has countably many
  generators (the primes). We observe that the application of the
  lambda calculus is much less resource sensitive than the parallel
  composition of the $\pi$-calculus. Could it be the case that we have
  an analogy of the form
  
  \begin{mathpar}
    m + n : MN :: m*n : M|N
  \end{mathpar}

  giving a similar blow up in the set of ``primes''?  This is such a
  wonderful thought that, even if it's not true, i think it's worth
  writing down.
\end{remark}
 

\documentclass[12pt]{llncs}
%\documentclass{jktr}

\usepackage[pdftex]{hyperref}                   
\usepackage {listings}
\usepackage {mathpartir}
\usepackage{bcprules}
%\usepackage{listings}
                       
\usepackage{graphicx} 
%\usepackage[margins=2.5cm,nohead,nofoot]{geometry}
%\usepackage{geometry}
\usepackage{amsfonts}
\usepackage{amstext}
\usepackage{latexsym}
\usepackage{amssymb}
\usepackage{color}


%\include{myPreamble}
\include{qm2pi.local} 

%\ifpdf
%\usepackage[pdftex]{graphicx}
%\else
%\usepackage{graphicx}
%\fi

 % \ifpdf
%  \usepackage{pdfsync}
%  \if


%\title{Brief Article}
%\author{David F. Snyder}
%\author{L.G. Meredith}

%\address{Dept. of Math., Texas State University--San Marcos, San Marcos, TX 78666}
       
\pagestyle{empty}


\begin{document}

\lstset{language=[Objective]Caml,frame=shadowbox}

\input{qm2pi.front}

% section front matter (end)

\input{qm2pi.intro} 
 
% section introduction (end)

% \input{qm2pi.knotations} 

% section notation (end)

\input{qm2pi.process.calculi} 

% section concurrent_process_calculi_and_spatial_logics_ (end)
    
%\input{qm2pi.knots2pi} 

%\input{qm2pi.trefoil} 

%\input{qm2pi.mainthm} 

% subsection basic_interpretation (end)

%\input{qm2pi.rho.presentation} 
\subsection{The syntax and semantics of the notation system}\label{sub:the_syntax_and_semantics_of_the_notation_system} % (fold)

We now summarize a technical presentation of the calculus that
embodies our theory of dynamics. The typical presentation of such a
calculus follows the style of giving generators and relations on
them. The grammar, below, describing term constructors, freely
generates the set of processes, $\Proc$. This set is then quotiented
by a relation known as structural congruence and it is over this set
that the notion of dynamics is expressed. This presentation is
essentially that of \cite{MeredithR05} with the addition of
polyadicity and summation. For readability we have relegated some of
the technical subtleties to an appendix.

\subsubsection{Process grammar}\label{subsub:process_grammar}

\begin{mathpar}
  \inferrule* [lab=synchronization] {} {{M} \bc \pzero \;|\; x?F \;|\; x!C }
  \and
  \inferrule* [lab=abstraction] {} {{F} \bc (x)P}
  \and
  \inferrule* [lab=concretion] {} {{C} \bc \langle Q \rangle}
  \and
  \inferrule* [lab=process] {} {{P,Q} \bc M \;| \;P|Q \;|\; @{x}}
  \and
  \inferrule* [lab=name] {} {{x} \bc \quotep{P}}
\end{mathpar} 

Note that $\vec{x}$ (resp. $\vec{P}$) denotes a vector of names
(resp. processes) of length $|\vec{x}|$ (resp. $|\vec{P}|$). We adopt
the following useful abbreviations.

\begin{mathpar}
   x?(\vec{y}).P := x.(\vec{y})P \and  x\clift{\vec{P}} := x.\clift{\vec{P}}
   \and x!(y) := \lift{x}{\dropn{y}}
   \and \Pi_{i=0}^{n-1}P_i := P_0 | \ldots | P_{n-1}
\end{mathpar}

\subsubsection{Structural congruence}

\paragraph{Free and bound names and alpha-equivalence.} At the
core of structural equivalence is alpha-equivalence which identifies
process that are the same up to a change of variable. Formally, we
recognize the distinction between free and bound names. The free names
of a process, $\freenames{P}$, may be calculated recursively as
follows:

\begin{mathpar}
\freenames{\pzero} := \emptyset
  \and \\
  \freenames{x?(y).P} := \{ x \} \cup (\freenames{P} \setminus \{ y \})
  \and 
  \freenames{x!\langle P \rangle} := \{ x \} \cup \{ P \} 
  \and \\
  \freenames{P|Q} := \freenames{P} \cup \freenames{Q}
  \and \\
  \freenames{@{x}} := \{ x \}
\end{mathpar}

$\pi$
$\quotep{\pi}$

$\freenames{-} : \pi \to \mathcal{P}(\quotep{\pi})$

\begin{eqnarray*}
  \freenames{\pzero} & := & \emptyset \\
  \freenames{x?(y).P} & := & \{ x \} \cup (\freenames{P} \setminus \{ y \}) \\
  \freenames{x!\langle P \rangle} & := & \{ x \} \cup \{ P \} \\
  \freenames{P|Q} & := & \freenames{P} \cup \freenames{Q} \\
  \freenames{\dropn{x}} & := & \{ x \}
\end{eqnarray*}

The bound names of a process, $\boundnames{P}$, are those names occurring in $P$
that are not free. For example, in $x?(y).0$, the name $x$ is free, while $y$ is bound.

\begin{mathpar}
  \inferrule* [lab=monoidal-laws] {} { P|Q \equiv Q|P \and P|0 \equiv P \and P|(Q|R) \equiv (P|Q)|R }
\end{mathpar}

\begin{mathpar}
  \inferrule* [lab=alpha-equivalence] {} { (x)P \equiv (y)P\{y/x\} \and y \not\in \freenames{P} }
\end{mathpar}

\begin{definition}
Then two processes, $P,Q$, are alpha-equivalent if $P = Q\{\vec{y}/\vec{x}\}$ for
some $\vec{x} \in \boundnames{Q},\vec{y} \in \boundnames{P}$, where $Q\{\vec{y}/\vec{x}\}$
denotes the capture-avoiding substitution of $\vec{y}$ for $\vec{x}$ in $Q$.
\end{definition}

\begin{definition}
  The {\em structural congruence} \cite{SangiorgiWalker} , $\equiv$,
  between processes is the least congruence containing
  alpha-equivalence, satisfying the abelian monoid laws
  (associativity, commutativity and $\pzero$ as identity) for parallel
  composition $|$ and for summation $+$.
\end{definition}

\subsection{Name equivalence}

We take name equivalence, written $\nameeq$, to be the smallest
equivalence relation generated by the following rules.

\begin{mathpar}
\inferrule*[lab=Quote-drop]
{ }
{ \quotep{@{x}} \nameeq x }

\inferrule*[lab=Struct-equiv]
{ P \scong Q }
{ \quotep{P} \nameeq \quotep{Q} }
\end{mathpar}

The astute reader will have noticed that the mutual recursion of names
and processes imposes a mutual recursion on alpha-equivalence and
structural equivalence via name-equivalence. Fortunately, all of this
works out pleasantly and we may calculate in the natural way, free of
concern. The reader interested in the details is referred to the
appendix \ref{appendix:rho_details}.

\subsection{Substitution}

We use $\Proc$ for the set of processes, $\QProc$ for the set of
names, and $\id{\{}\vec{y} / \vec{x} \id{\}}$ to denote partial maps,
$s : \QProc \rightarrow \QProc$. A map, $s$ lifts, uniquely, to a map
on process terms, $\widehat{s} : \Proc \rightarrow \Proc$ by the
following equations.

\begin{mathpar}
  (0) \psubstp{Q}{P} := 0 \\
  (R \juxtap S) \psubstp{Q}{P}
  :=    
  (R)\psubstp{Q}{P} \juxtap (S) \psubstp{Q}{P} \\
  (x?(y).R) \psubstp{Q}{P}    
  :=    
  (x)\substp{Q}{P} (z)\concat( (R \psubstn{z}{y}) \psubstp{Q}{P} ) \\
  (\lift{x}{R}) \psubstp{Q}{P}  
  :=
  \lift{(x)\substp{Q}{P}}{ R \psubstp{Q}{P} } \\
%   (\dropn{x})  \psubstp{Q}{P}       
%   := 
%   \left\{ 
%     \begin{array}{ccc} 
%       \dropn{\quotep{Q}} & & x \nameeq \quotep{P} \\
%       \dropn{x} & & otherwise \\
%     \end{array}
%   \right. 
  (\dropn{x})  \psubstp{Q}{P}       
  := 
  \left\{ 
    \begin{array}{ccc} 
      Q & & x \nameeq \quotep{P} \\
      \dropn{x} & & otherwise \\
    \end{array}
  \right.
\end{mathpar}
 

where

\begin{eqnarray}
  (x)\id{\{} \lpquote Q \rpquote / \lpquote P \rpquote \id{\}}            = 
  \left\{ 
    \begin{array}{ccc}
      \lpquote Q \rpquote & & x \nameeq \lpquote P \rpquote \\
      x & & otherwise \\
    \end{array}
  \right. \nonumber
\end{eqnarray}

and $z$ is chosen distinct from $\quotep{P}$, $\quotep{Q}$, the free
names in $Q$, and all the names in $R$. Our $\alpha$-equivalence will
be built in the standard way from this substitution.

\begin{remark}\label{rem:no_self_referential_names}
  One consequence of these definitions is that $\forall P. \quotep{P}
  \not\in \freenames{P}$.
\end{remark}

\subsection{ Dynamic quote: an example }

Anticipating something of what's to come, consider applying the
substitution, $\widehat{\id{\{}u / z \id{\}}}$, to the following pair
of processes, $\lift{w}{y!(z)}$ and $w[ \lpquote y!(z) \rpquote ]$.

\begin{eqnarray}
	\lift{w}{y!(z)}\widehat{\id{\{}u / z \id{\}}}
		& = &
		\lift{w}{y!(u)} \nonumber\\
	w[ \lpquote y!(z) \rpquote ] \widehat{ \id{\{}u / z \id{\}} }
		& = &
		w[ \lpquote y!(z) \rpquote ] \nonumber
\end{eqnarray}

Because the body of the process between quotes is impervious to
substitution, we get radically different answers. In fact, by
examining the first process in an input context,
e.g. $x?(z).\lift{w}{y!(z)}$, we see that the process under the lift
operator may be shaped by prefixed inputs binding a name inside it. In
this sense, the lift operator will be seen as a way to dynamically
construct processes before reifying them as names.

Finally equipped with these standard features we can present the
dynamics of the calculus.

\subsubsection{Operational semantics} 

Finally, we introduce the computational dynamics. What marks these
algebras as distinct from other more traditionally studied algebraic
structures, e.g. vector spaces or polynomial rings, is the manner in
which dynamics is captured. In traditional structures, dynamics is typically
expressed through morphisms between such structures, as in linear maps
between vector spaces or morphisms between rings. In algebras
associated with the semantics of computation, the dynamics is
expressed as part of the algebraic structure itself, through a
reduction reduction relation typically denoted by $\red$. Below, we
give a recursive presentation of this relation for the calculus used
in the encoding.

$\red \subseteq \pi \times \pi$
$\red : \pi \to \mathcal{P}(\pi)$

\begin{mathpar}
  \inferrule* [lab=Comm] { \textsf{match}( x_{src}, x_{trgt} ) } { x_{trgt}?(y)P \; | \; x_{src}!\langle {Q} \rangle \red P\{\quotep{Q}/y}\} }
  \and \\
  \inferrule* [lab=Par] {{P} \red {P}'} {{{P} | {Q}} \red {{P}' | {Q}}}
  \and
  \inferrule* [lab=Equiv]{{{P} \scong {P}'} \andalso {{P}' \red {Q}'} \andalso {{Q}' \scong {Q}}}{{P} \red {Q}}
\end{mathpar}

\begin{eqnarray*}
  match_{\equiv} (\quotep{P},\quotep{Q}) & := & P \equiv Q \\
  match_{\dagger}(\quotep{P},\quotep{Q}) & := & \forall R. P|Q \red^{*} R => R \red^{*} 0 \\
  match_{K}(\quotep{P},\quotep{Q}) & := & K \mbox{ for some context } K
\end{eqnarray*}

$u?(x)P | u!\langle Q \rangle \red P\{\quotep{Q}/x\}$

%We write $\wred$ for $\red^*$, and $P\red$ if $\exists Q $ such that $ P \red Q$.
We write $P\red$ if $\exists Q $ such that $ P \red Q$ and $P\not\red$, otherwise.

\section{Replication}

As mentioned before, it is known that replication (and hence
recursion) can be implemented in a higher-order process algebra
\cite{SangiorgiWalker}. As our first example of calculation with the
machinery thus far presented we give the construction explicitly in
the {\rhoc}.

\begin{eqnarray}
	D_{x} & := & \prefix{x}{y}{(\binpar{\outputp{x}{y}}{@{y}})} \nonumber\\
	\bangp_{x}{P} & := & \binpar{{x}!\langle{\binpar{D_{x}}{P}}\rangle}{D_{x}} \nonumber
\end{eqnarray}

\begin{eqnarray}
	\bangp_{x}{P} & & \nonumber\\
	=
	& {x}!\langle{(\prefix{x}{y}{(\outputp{x}{y} | @{y})) | P}}\rangle 
	      | \prefix{x}{y}{(\outputp{x}{y} | @{y})} & \nonumber\\
	\red
	& (\outputp{x}{y} | @{y})\substn{\quotep{(\prefix{x}{y}{(@{y} | \outputp{x}{y})) | P}}}{y} & \nonumber\\
	=
	& \outputp{x}{\quotep{(\prefix{x}{y}{(\outputp{x}{y} | @{y})) | P}}}
	  | {(\prefix{x}{y}{(\outputp{x}{y} | @{y})) | P}} & \nonumber\\
	\red
	& \ldots & \nonumber\\
	\red^*
	& P | P | \ldots & \nonumber
\end{eqnarray}

Of course, this encoding, as an implementation, runs away, unfolding
$\bangp{P}$ eagerly. A lazier and more implementable replication
operator, restricted to input-guarded processes, may be obtained as follows.

\begin{eqnarray}
\bangp{\prefix{u}{v}{P}} 
	:= 
	\binpar{\lift{x}{\prefix{u}{v}{(\binpar{D(x)}{P})}}}{D(x)} \nonumber
\end{eqnarray}

\begin{remark}
  Note that the lazier definition still does not deal with summation
  or mixed summation (i.e. sums over input and output). The reader is
  invited to construct definitions of replication that deal with these
  features. 

  Further, the definitions are parameterized in a name, $x$. Can you,
  gentle reader, make a definition that eliminates this parameter and
  guarantees no accidental interaction between the replication
  machinery and the process being replicated -- i.e. no accidental
  sharing of names used by the process to get its work done and the
  name(s) used by the replication to effect copying. This latter
  revision of the definition of replication is crucial to obtaining
  the expected identity $!!P \sim !P$.
\end{remark}

\begin{remark}\label{rem:paradoxical_combinator}
  The reader familiar with the lambda calculus will have noticed the
  similarity between $D$ and the paradoxical combinator.

  [Ed. note: the existence of this seems to suggest we have to be more
  restrictive on the set of processes and names we admit if we are to
  support no-cloning.]
\end{remark}

\subsubsection{Bisimulation}

The computational dynamics gives rise to another kind of equivalence,
the equivalence of computational behavior. As previously mentioned
this is typically captured \emph{via} some form of bisimulation.

% The notion we use in this paper is weak barbed bisimulation
% \cite{milner91polyadicpi}.

The notion we use in this paper is derived from weak barbed
bisimulation \cite{milner91polyadicpi}. 

\begin{definition}
An \emph{observation relation}, $\downarrow_{\mathcal N}$, over a set
of names, $\mathcal N$, is the smallest relation satisfying the rules
below.

\infrule[Out-barb]{y \in {\mathcal N}, \; x \nameeq y}
		  {\outputp{x}{v} \downarrow_{\mathcal N} x}
\infrule[Par-barb]{\mbox{$P\downarrow_{\mathcal N} x$ or $Q\downarrow_{\mathcal N} x$}}
		  {\binpar{P}{Q} \downarrow_{\mathcal N} x}

We write $P \Downarrow_{\mathcal N} x$ if there is $Q$ such that 
$P \wred Q$ and $Q \downarrow_{\mathcal N} x$.
\end{definition}

\begin{definition}
%\label{def.bbisim}
An  ${\mathcal N}$-\emph{barbed bisimulation} over a set of names, ${\mathcal N}$, is a symmetric binary relation 
${\mathcal S}_{\mathcal N}$ between agents such that $P\rel{S}_{\mathcal N}Q$ implies:
\begin{enumerate}
\item If $P \red P'$ then $Q \wred Q'$ and $P'\rel{S}_{\mathcal N} Q'$.
\item If $P\downarrow_{\mathcal N} x$, then $Q\Downarrow_{\mathcal N} x$.
\end{enumerate}
$P$ is ${\mathcal N}$-barbed bisimilar to $Q$, written
$P \wbbisim_{\mathcal N} Q$, if $P \rel{S}_{\mathcal N} Q$ for some ${\mathcal N}$-barbed bisimulation ${\mathcal S}_{\mathcal N}$.
\end{definition}

$\mathcal{R} \subseteq \pi \times \pi$

$P \mathcal{R} Q => \forall P'. P \red P' \Rightarrow \exists Q'. Q \red Q', P' \mathcal{R} Q'$

$P \vdash x \Rightarrow Q \vdash x$

\begin{mathpar}
  \inferrule*[lab=Out-barb]{x \nameeq y}{{y}!\langle{Q}\rangle \vdash x}
  \and
  \inferrule*[lab=Par-barb]{\mbox{$P\vdash x$ or $Q\vdash x$}}{\binpar{P}{Q} \vdash x}
\end{mathpar}

\subsubsection{Contexts}

One of the principle advantages of computational calculi like the
$\pi$-calculus is a well-defined notion of context,
contextual-equivalence and a correlation between
contextual-equivalence and notions of bisimulation. The notion of
context allows the decomposition of a process into (sub-)process and
its syntactic environment, its context. Thus, a context may be
thought of as a process with a ``hole'' (written $\Box$) in it. The
application of a context $M$ to a process $P$, written $M[P]$, is
tantamount to filling the hole in $M$ with $P$. In this paper we do
not need the full weight of this theory, but do make use of the notion
of context in the proof the main theorem. 

\begin{mathpar}
  \inferrule* [lab=summation] {} {{M_{M},M_{N}} \bc \Box \;|\; x.M_{A} \;|\; M_{M}+M_{N}}
  \and
  \inferrule* [lab=agent] {} {{M_{A}} \bc (\vec{x})M_{P} \;| \; \clift{P_0,\ldots,M_{P},\ldots,P_N}}
  \and \\
  \inferrule* [lab=process] {} {{M_{P}} \bc M_{N} \;| \;P|M_{P} }
\end{mathpar} 

\begin{mathpar}
  \inferrule* [lab=sychronization] {} {M_{N} \bc \Box \;|\; x?M_{F} \;|\; x!M_{C}}
  \and
  \inferrule* [lab=abstraction] {} {{M_{F}} \bc (x)M_{P} }
  \and
  \inferrule* [lab=concretion] {} {{M_{C}} \bc \langle M_{P} \rangle }
  \and \\
  \inferrule* [lab=process] {} {{M_{P}} \bc M_{N} \;| \;P|M_{P} }
\end{mathpar}

\begin{definition}[contextual application] Given a context $M$, and
  process $P$, we define the \emph{contextual application}, $M[P] :=
  M\{P/\Box\}$. That is, the contextual application of M to P is the
  substitution of $P$ for $\Box$ in $M$.
\end{definition}

$\meaningof{-} : L \to \mathcal{P}(\pi)$

\begin{mathpar}
  \inferrule* [lab=collection] {} {\meaningof{true} = \pi, \and \meaningof{~E} = \pi \setminus \meaningof{E}, \and \meaningof{E_{1} \& E_{2}} = \meaningof{E_{1}} \cap \meaningof{E_{2}}}
\end{mathpar}

\begin{mathpar}
  \inferrule* [lab=structure] {} {\meaningof{0} = \{ P \in \pi | P \equiv 0 \}, \and \\ \meaningof{E_1 | E_2} = \{ P \in \pi | P \equiv P_{1} | P_{2}, P_{1} \in \meaningof{E_{1}}, P_{2} \in \meaningof{E_2}\} }
\end{mathpar}

\begin{mathpar}
 \inferrule* [lab=behavior] {} {\meaningof{\langle a?b \rangle E} = \{ P \in \pi | P \equiv Q | u?(y)P', \\ \and \\\\ \and \\ \;\;\; u \in \meaningof{a}, \forall z.P'\{z/y\} \in \meaningof{E\{z/b\}}\}, \and \\ \meaningof{a!E} = \{ P \in \pi | P \equiv Q | x!\langle P' \rangle, x \in \meaningof{a} P' \in \meaningof{E}\} }
\end{mathpar}

\begin{mathpar}
 \inferrule* [lab=nominal] {} {\meaningof{\quotep{E}} = \{ \quotep{P} \in \quotep{\pi} | P \in \meaningof{E} \}, \and \meaningof{\quotep{P}} = \{ \quotep{Q} \in \quotep{\pi} | P \equiv Q \} \and \\ \meaningof{@\quotep{E}} = \{ P \in \pi | P \equiv @x, x \in \meaningof{E} \}}
\end{mathpar}

\begin{eqnarray*}
  \\
  \meaningof{-} : TS \to ST
\end{eqnarray*}

\begin{eqnarray*}
  \\
  L : TS \to ST
\end{eqnarray*}

\begin{eqnarray*}
  \\
  P \models E \iff P \in \meaningof{E}
\end{eqnarray*}

\begin{eqnarray*}
  P \approx_{L} Q \iff \forall E \in L. P \models E \iff Q \models E
\end{eqnarray*}

\begin{eqnarray*}
  P \approx_{K} Q
\end{eqnarray*}

\begin{eqnarray*}
  P \approx Q
\end{eqnarray*}

$\approx_{K} = \approx = \approx_{L}$

\subsubsection{Contextual duality}

Note that contexts extend the quotation operation to a family of
operations from processes to names. Given a context, $M$, we can
define a \emph{nominal context}, $\quotep{M}$ by $\quotep{M}[P] :=
\quotep{M[P]}$. To foreshadow what is to come we observe that these
operations enjoy a duality with processes very much like the duality
between vectors and maps from vectors to scalars.

Further, because the calculus is essentially higher-order, we have a
correspondence between contexts and processes. More specifically,
given a name $x$ and a context $M$ we can construct $M^{*}_{x}$ such
that 

\begin{mathpar}
  M^{*}_{x} | \lift{x}{P} \red M[P]
\end{mathpar}

namely,

\begin{mathpar}
  M^{*}_{x} := x?(u).M[\dropn{u}]
\end{mathpar}

The dependence of $M^{*}_{x}$ on a name makes it an abstraction, 

\begin{mathpar}
  M^{*} := (x)x?(u).M[\dropn{u}]
\end{mathpar}

\subsection{Additional notation}

It will sometimes be convenient to denote the process a name
quotes. We already have the notation $x = \quotep{P}$, but it will be
convenient to introduce an alternate notation, $\procn{x}$, when we
want to emphasize the connection to the use of the name. Note that, by
virtue of name equivalence, $\quotep{\procn{x}} \nameeq x$; so, the
notation is consistent with previous definitions.

Further, because names have structure it is possible to effect
substitutions on the basis of that structure. This means we need to
upgrade our notation for substitutions, which we accomplish by
adapting comprehension notation. Thus,

\begin{mathpar}
  P\{ y / x : x \in S \}
\end{mathpar}

is interpreted to mean the process derived from P by replacing (in a
capture-avoiding manner) each occurrence of $x$ in $S$ by $y$. For example,

\begin{mathpar}
  P\{ \quotep{\procn{x}|\procn{x}} / x : x \in \freenames{P} \}
\end{mathpar}

will replace each (occurrence) of a free name $x$ in $P$ by
$\quotep{\procn{x}|\procn{x}}$.

Also, we will avail ourselves of the notation $x^{L}$ and $x^{R}$ to
denote injections of a name into disjoint copies of the name
space. There are numerous ways to accomplish this. One example can be
found in \cite{MeredithR05}. This notation overloads to vectors of
names: $\vec{x}^{\pi} := (x_{i}^{\pi} \; : \; 0 \leq i < |\vec{x}| )$ where $\pi \in \{L,R\}$.

We also use $P^{\Box} := P|\Box$.

In \cite{MeredithR05} an interpretation of the new operator is
given. It turns out that there are several possible interpretations
all enjoying the requisite algebraic properties of the operator (see
\cite{milner91polyadicpi}). We will therefore make liberal use of
$(\nu\; \vec{x})P$.

% subsection the_syntax_and_semantics_of_the_notation_system (end)   

\input{qm2pi.qmops} 

\input{qm2pi.sterngerlach} 

\input{qm2pi.metric} 

% section concurrent_process_calculi (end)

%\input{qm2pi.proofsketch}

% section proof sketch (end)

%\input{qm2pi.slviaknots} 

% section spatial logic via knots (end)

\input{qm2pi.conclusion}

% section conclusion (end)

%\input{qm2pi.dtcodes} 

% section wiring algorithm (end)

\input{qm2pi.ack} 

% section acknowledgments (end)

\newpage


\bibliographystyle{plain}   
\bibliography{../../biblios/main.bib}

\input{qm2pi.rhodetails}

\end{document}

 

\documentclass[12pt]{llncs}
%\documentclass{jktr}

\usepackage[pdftex]{hyperref}                   
\usepackage {listings}
\usepackage {mathpartir}
\usepackage{bcprules}
%\usepackage{listings}
                       
\usepackage{graphicx} 
%\usepackage[margins=2.5cm,nohead,nofoot]{geometry}
%\usepackage{geometry}
\usepackage{amsfonts}
\usepackage{amstext}
\usepackage{latexsym}
\usepackage{amssymb}
\usepackage{color}


%\include{myPreamble}
\include{qm2pi.local} 

%\ifpdf
%\usepackage[pdftex]{graphicx}
%\else
%\usepackage{graphicx}
%\fi

 % \ifpdf
%  \usepackage{pdfsync}
%  \if


%\title{Brief Article}
%\author{David F. Snyder}
%\author{L.G. Meredith}

%\address{Dept. of Math., Texas State University--San Marcos, San Marcos, TX 78666}
       
\pagestyle{empty}


\begin{document}

\lstset{language=[Objective]Caml,frame=shadowbox}

\input{qm2pi.front}

% section front matter (end)

\input{qm2pi.intro} 
 
% section introduction (end)

% \input{qm2pi.knotations} 

% section notation (end)

\input{qm2pi.process.calculi} 

% section concurrent_process_calculi_and_spatial_logics_ (end)
    
%\input{qm2pi.knots2pi} 

%\input{qm2pi.trefoil} 

%\input{qm2pi.mainthm} 

% subsection basic_interpretation (end)

%\input{qm2pi.rho.presentation} 
\subsection{The syntax and semantics of the notation system}\label{sub:the_syntax_and_semantics_of_the_notation_system} % (fold)

We now summarize a technical presentation of the calculus that
embodies our theory of dynamics. The typical presentation of such a
calculus follows the style of giving generators and relations on
them. The grammar, below, describing term constructors, freely
generates the set of processes, $\Proc$. This set is then quotiented
by a relation known as structural congruence and it is over this set
that the notion of dynamics is expressed. This presentation is
essentially that of \cite{MeredithR05} with the addition of
polyadicity and summation. For readability we have relegated some of
the technical subtleties to an appendix.

\subsubsection{Process grammar}\label{subsub:process_grammar}

\begin{mathpar}
  \inferrule* [lab=synchronization] {} {{M} \bc \pzero \;|\; x?F \;|\; x!C }
  \and
  \inferrule* [lab=abstraction] {} {{F} \bc (x)P}
  \and
  \inferrule* [lab=concretion] {} {{C} \bc \langle Q \rangle}
  \and
  \inferrule* [lab=process] {} {{P,Q} \bc M \;| \;P|Q \;|\; @{x}}
  \and
  \inferrule* [lab=name] {} {{x} \bc \quotep{P}}
\end{mathpar} 

Note that $\vec{x}$ (resp. $\vec{P}$) denotes a vector of names
(resp. processes) of length $|\vec{x}|$ (resp. $|\vec{P}|$). We adopt
the following useful abbreviations.

\begin{mathpar}
   x?(\vec{y}).P := x.(\vec{y})P \and  x\clift{\vec{P}} := x.\clift{\vec{P}}
   \and x!(y) := \lift{x}{\dropn{y}}
   \and \Pi_{i=0}^{n-1}P_i := P_0 | \ldots | P_{n-1}
\end{mathpar}

\subsubsection{Structural congruence}

\paragraph{Free and bound names and alpha-equivalence.} At the
core of structural equivalence is alpha-equivalence which identifies
process that are the same up to a change of variable. Formally, we
recognize the distinction between free and bound names. The free names
of a process, $\freenames{P}$, may be calculated recursively as
follows:

\begin{mathpar}
\freenames{\pzero} := \emptyset
  \and \\
  \freenames{x?(y).P} := \{ x \} \cup (\freenames{P} \setminus \{ y \})
  \and 
  \freenames{x!\langle P \rangle} := \{ x \} \cup \{ P \} 
  \and \\
  \freenames{P|Q} := \freenames{P} \cup \freenames{Q}
  \and \\
  \freenames{@{x}} := \{ x \}
\end{mathpar}

$\pi$
$\quotep{\pi}$

$\freenames{-} : \pi \to \mathcal{P}(\quotep{\pi})$

\begin{eqnarray*}
  \freenames{\pzero} & := & \emptyset \\
  \freenames{x?(y).P} & := & \{ x \} \cup (\freenames{P} \setminus \{ y \}) \\
  \freenames{x!\langle P \rangle} & := & \{ x \} \cup \{ P \} \\
  \freenames{P|Q} & := & \freenames{P} \cup \freenames{Q} \\
  \freenames{\dropn{x}} & := & \{ x \}
\end{eqnarray*}

The bound names of a process, $\boundnames{P}$, are those names occurring in $P$
that are not free. For example, in $x?(y).0$, the name $x$ is free, while $y$ is bound.

\begin{mathpar}
  \inferrule* [lab=monoidal-laws] {} { P|Q \equiv Q|P \and P|0 \equiv P \and P|(Q|R) \equiv (P|Q)|R }
\end{mathpar}

\begin{mathpar}
  \inferrule* [lab=alpha-equivalence] {} { (x)P \equiv (y)P\{y/x\} \and y \not\in \freenames{P} }
\end{mathpar}

\begin{definition}
Then two processes, $P,Q$, are alpha-equivalent if $P = Q\{\vec{y}/\vec{x}\}$ for
some $\vec{x} \in \boundnames{Q},\vec{y} \in \boundnames{P}$, where $Q\{\vec{y}/\vec{x}\}$
denotes the capture-avoiding substitution of $\vec{y}$ for $\vec{x}$ in $Q$.
\end{definition}

\begin{definition}
  The {\em structural congruence} \cite{SangiorgiWalker} , $\equiv$,
  between processes is the least congruence containing
  alpha-equivalence, satisfying the abelian monoid laws
  (associativity, commutativity and $\pzero$ as identity) for parallel
  composition $|$ and for summation $+$.
\end{definition}

\subsection{Name equivalence}

We take name equivalence, written $\nameeq$, to be the smallest
equivalence relation generated by the following rules.

\begin{mathpar}
\inferrule*[lab=Quote-drop]
{ }
{ \quotep{@{x}} \nameeq x }

\inferrule*[lab=Struct-equiv]
{ P \scong Q }
{ \quotep{P} \nameeq \quotep{Q} }
\end{mathpar}

The astute reader will have noticed that the mutual recursion of names
and processes imposes a mutual recursion on alpha-equivalence and
structural equivalence via name-equivalence. Fortunately, all of this
works out pleasantly and we may calculate in the natural way, free of
concern. The reader interested in the details is referred to the
appendix \ref{appendix:rho_details}.

\subsection{Substitution}

We use $\Proc$ for the set of processes, $\QProc$ for the set of
names, and $\id{\{}\vec{y} / \vec{x} \id{\}}$ to denote partial maps,
$s : \QProc \rightarrow \QProc$. A map, $s$ lifts, uniquely, to a map
on process terms, $\widehat{s} : \Proc \rightarrow \Proc$ by the
following equations.

\begin{mathpar}
  (0) \psubstp{Q}{P} := 0 \\
  (R \juxtap S) \psubstp{Q}{P}
  :=    
  (R)\psubstp{Q}{P} \juxtap (S) \psubstp{Q}{P} \\
  (x?(y).R) \psubstp{Q}{P}    
  :=    
  (x)\substp{Q}{P} (z)\concat( (R \psubstn{z}{y}) \psubstp{Q}{P} ) \\
  (\lift{x}{R}) \psubstp{Q}{P}  
  :=
  \lift{(x)\substp{Q}{P}}{ R \psubstp{Q}{P} } \\
%   (\dropn{x})  \psubstp{Q}{P}       
%   := 
%   \left\{ 
%     \begin{array}{ccc} 
%       \dropn{\quotep{Q}} & & x \nameeq \quotep{P} \\
%       \dropn{x} & & otherwise \\
%     \end{array}
%   \right. 
  (\dropn{x})  \psubstp{Q}{P}       
  := 
  \left\{ 
    \begin{array}{ccc} 
      Q & & x \nameeq \quotep{P} \\
      \dropn{x} & & otherwise \\
    \end{array}
  \right.
\end{mathpar}
 

where

\begin{eqnarray}
  (x)\id{\{} \lpquote Q \rpquote / \lpquote P \rpquote \id{\}}            = 
  \left\{ 
    \begin{array}{ccc}
      \lpquote Q \rpquote & & x \nameeq \lpquote P \rpquote \\
      x & & otherwise \\
    \end{array}
  \right. \nonumber
\end{eqnarray}

and $z$ is chosen distinct from $\quotep{P}$, $\quotep{Q}$, the free
names in $Q$, and all the names in $R$. Our $\alpha$-equivalence will
be built in the standard way from this substitution.

\begin{remark}\label{rem:no_self_referential_names}
  One consequence of these definitions is that $\forall P. \quotep{P}
  \not\in \freenames{P}$.
\end{remark}

\subsection{ Dynamic quote: an example }

Anticipating something of what's to come, consider applying the
substitution, $\widehat{\id{\{}u / z \id{\}}}$, to the following pair
of processes, $\lift{w}{y!(z)}$ and $w[ \lpquote y!(z) \rpquote ]$.

\begin{eqnarray}
	\lift{w}{y!(z)}\widehat{\id{\{}u / z \id{\}}}
		& = &
		\lift{w}{y!(u)} \nonumber\\
	w[ \lpquote y!(z) \rpquote ] \widehat{ \id{\{}u / z \id{\}} }
		& = &
		w[ \lpquote y!(z) \rpquote ] \nonumber
\end{eqnarray}

Because the body of the process between quotes is impervious to
substitution, we get radically different answers. In fact, by
examining the first process in an input context,
e.g. $x?(z).\lift{w}{y!(z)}$, we see that the process under the lift
operator may be shaped by prefixed inputs binding a name inside it. In
this sense, the lift operator will be seen as a way to dynamically
construct processes before reifying them as names.

Finally equipped with these standard features we can present the
dynamics of the calculus.

\subsubsection{Operational semantics} 

Finally, we introduce the computational dynamics. What marks these
algebras as distinct from other more traditionally studied algebraic
structures, e.g. vector spaces or polynomial rings, is the manner in
which dynamics is captured. In traditional structures, dynamics is typically
expressed through morphisms between such structures, as in linear maps
between vector spaces or morphisms between rings. In algebras
associated with the semantics of computation, the dynamics is
expressed as part of the algebraic structure itself, through a
reduction reduction relation typically denoted by $\red$. Below, we
give a recursive presentation of this relation for the calculus used
in the encoding.

$\red \subseteq \pi \times \pi$
$\red : \pi \to \mathcal{P}(\pi)$

\begin{mathpar}
  \inferrule* [lab=Comm] { \textsf{match}( x_{src}, x_{trgt} ) } { x_{trgt}?(y)P \; | \; x_{src}!\langle {Q} \rangle \red P\{\quotep{Q}/y}\} }
  \and \\
  \inferrule* [lab=Par] {{P} \red {P}'} {{{P} | {Q}} \red {{P}' | {Q}}}
  \and
  \inferrule* [lab=Equiv]{{{P} \scong {P}'} \andalso {{P}' \red {Q}'} \andalso {{Q}' \scong {Q}}}{{P} \red {Q}}
\end{mathpar}

\begin{eqnarray*}
  match_{\equiv} (\quotep{P},\quotep{Q}) & := & P \equiv Q \\
  match_{\dagger}(\quotep{P},\quotep{Q}) & := & \forall R. P|Q \red^{*} R => R \red^{*} 0 \\
  match_{K}(\quotep{P},\quotep{Q}) & := & K \mbox{ for some context } K
\end{eqnarray*}

$u?(x)P | u!\langle Q \rangle \red P\{\quotep{Q}/x\}$

%We write $\wred$ for $\red^*$, and $P\red$ if $\exists Q $ such that $ P \red Q$.
We write $P\red$ if $\exists Q $ such that $ P \red Q$ and $P\not\red$, otherwise.

\section{Replication}

As mentioned before, it is known that replication (and hence
recursion) can be implemented in a higher-order process algebra
\cite{SangiorgiWalker}. As our first example of calculation with the
machinery thus far presented we give the construction explicitly in
the {\rhoc}.

\begin{eqnarray}
	D_{x} & := & \prefix{x}{y}{(\binpar{\outputp{x}{y}}{@{y}})} \nonumber\\
	\bangp_{x}{P} & := & \binpar{{x}!\langle{\binpar{D_{x}}{P}}\rangle}{D_{x}} \nonumber
\end{eqnarray}

\begin{eqnarray}
	\bangp_{x}{P} & & \nonumber\\
	=
	& {x}!\langle{(\prefix{x}{y}{(\outputp{x}{y} | @{y})) | P}}\rangle 
	      | \prefix{x}{y}{(\outputp{x}{y} | @{y})} & \nonumber\\
	\red
	& (\outputp{x}{y} | @{y})\substn{\quotep{(\prefix{x}{y}{(@{y} | \outputp{x}{y})) | P}}}{y} & \nonumber\\
	=
	& \outputp{x}{\quotep{(\prefix{x}{y}{(\outputp{x}{y} | @{y})) | P}}}
	  | {(\prefix{x}{y}{(\outputp{x}{y} | @{y})) | P}} & \nonumber\\
	\red
	& \ldots & \nonumber\\
	\red^*
	& P | P | \ldots & \nonumber
\end{eqnarray}

Of course, this encoding, as an implementation, runs away, unfolding
$\bangp{P}$ eagerly. A lazier and more implementable replication
operator, restricted to input-guarded processes, may be obtained as follows.

\begin{eqnarray}
\bangp{\prefix{u}{v}{P}} 
	:= 
	\binpar{\lift{x}{\prefix{u}{v}{(\binpar{D(x)}{P})}}}{D(x)} \nonumber
\end{eqnarray}

\begin{remark}
  Note that the lazier definition still does not deal with summation
  or mixed summation (i.e. sums over input and output). The reader is
  invited to construct definitions of replication that deal with these
  features. 

  Further, the definitions are parameterized in a name, $x$. Can you,
  gentle reader, make a definition that eliminates this parameter and
  guarantees no accidental interaction between the replication
  machinery and the process being replicated -- i.e. no accidental
  sharing of names used by the process to get its work done and the
  name(s) used by the replication to effect copying. This latter
  revision of the definition of replication is crucial to obtaining
  the expected identity $!!P \sim !P$.
\end{remark}

\begin{remark}\label{rem:paradoxical_combinator}
  The reader familiar with the lambda calculus will have noticed the
  similarity between $D$ and the paradoxical combinator.

  [Ed. note: the existence of this seems to suggest we have to be more
  restrictive on the set of processes and names we admit if we are to
  support no-cloning.]
\end{remark}

\subsubsection{Bisimulation}

The computational dynamics gives rise to another kind of equivalence,
the equivalence of computational behavior. As previously mentioned
this is typically captured \emph{via} some form of bisimulation.

% The notion we use in this paper is weak barbed bisimulation
% \cite{milner91polyadicpi}.

The notion we use in this paper is derived from weak barbed
bisimulation \cite{milner91polyadicpi}. 

\begin{definition}
An \emph{observation relation}, $\downarrow_{\mathcal N}$, over a set
of names, $\mathcal N$, is the smallest relation satisfying the rules
below.

\infrule[Out-barb]{y \in {\mathcal N}, \; x \nameeq y}
		  {\outputp{x}{v} \downarrow_{\mathcal N} x}
\infrule[Par-barb]{\mbox{$P\downarrow_{\mathcal N} x$ or $Q\downarrow_{\mathcal N} x$}}
		  {\binpar{P}{Q} \downarrow_{\mathcal N} x}

We write $P \Downarrow_{\mathcal N} x$ if there is $Q$ such that 
$P \wred Q$ and $Q \downarrow_{\mathcal N} x$.
\end{definition}

\begin{definition}
%\label{def.bbisim}
An  ${\mathcal N}$-\emph{barbed bisimulation} over a set of names, ${\mathcal N}$, is a symmetric binary relation 
${\mathcal S}_{\mathcal N}$ between agents such that $P\rel{S}_{\mathcal N}Q$ implies:
\begin{enumerate}
\item If $P \red P'$ then $Q \wred Q'$ and $P'\rel{S}_{\mathcal N} Q'$.
\item If $P\downarrow_{\mathcal N} x$, then $Q\Downarrow_{\mathcal N} x$.
\end{enumerate}
$P$ is ${\mathcal N}$-barbed bisimilar to $Q$, written
$P \wbbisim_{\mathcal N} Q$, if $P \rel{S}_{\mathcal N} Q$ for some ${\mathcal N}$-barbed bisimulation ${\mathcal S}_{\mathcal N}$.
\end{definition}

$\mathcal{R} \subseteq \pi \times \pi$

$P \mathcal{R} Q => \forall P'. P \red P' \Rightarrow \exists Q'. Q \red Q', P' \mathcal{R} Q'$

$P \vdash x \Rightarrow Q \vdash x$

\begin{mathpar}
  \inferrule*[lab=Out-barb]{x \nameeq y}{{y}!\langle{Q}\rangle \vdash x}
  \and
  \inferrule*[lab=Par-barb]{\mbox{$P\vdash x$ or $Q\vdash x$}}{\binpar{P}{Q} \vdash x}
\end{mathpar}

\subsubsection{Contexts}

One of the principle advantages of computational calculi like the
$\pi$-calculus is a well-defined notion of context,
contextual-equivalence and a correlation between
contextual-equivalence and notions of bisimulation. The notion of
context allows the decomposition of a process into (sub-)process and
its syntactic environment, its context. Thus, a context may be
thought of as a process with a ``hole'' (written $\Box$) in it. The
application of a context $M$ to a process $P$, written $M[P]$, is
tantamount to filling the hole in $M$ with $P$. In this paper we do
not need the full weight of this theory, but do make use of the notion
of context in the proof the main theorem. 

\begin{mathpar}
  \inferrule* [lab=summation] {} {{M_{M},M_{N}} \bc \Box \;|\; x.M_{A} \;|\; M_{M}+M_{N}}
  \and
  \inferrule* [lab=agent] {} {{M_{A}} \bc (\vec{x})M_{P} \;| \; \clift{P_0,\ldots,M_{P},\ldots,P_N}}
  \and \\
  \inferrule* [lab=process] {} {{M_{P}} \bc M_{N} \;| \;P|M_{P} }
\end{mathpar} 

\begin{mathpar}
  \inferrule* [lab=sychronization] {} {M_{N} \bc \Box \;|\; x?M_{F} \;|\; x!M_{C}}
  \and
  \inferrule* [lab=abstraction] {} {{M_{F}} \bc (x)M_{P} }
  \and
  \inferrule* [lab=concretion] {} {{M_{C}} \bc \langle M_{P} \rangle }
  \and \\
  \inferrule* [lab=process] {} {{M_{P}} \bc M_{N} \;| \;P|M_{P} }
\end{mathpar}

\begin{definition}[contextual application] Given a context $M$, and
  process $P$, we define the \emph{contextual application}, $M[P] :=
  M\{P/\Box\}$. That is, the contextual application of M to P is the
  substitution of $P$ for $\Box$ in $M$.
\end{definition}

$\meaningof{-} : L \to \mathcal{P}(\pi)$

\begin{mathpar}
  \inferrule* [lab=collection] {} {\meaningof{true} = \pi, \and \meaningof{~E} = \pi \setminus \meaningof{E}, \and \meaningof{E_{1} \& E_{2}} = \meaningof{E_{1}} \cap \meaningof{E_{2}}}
\end{mathpar}

\begin{mathpar}
  \inferrule* [lab=structure] {} {\meaningof{0} = \{ P \in \pi | P \equiv 0 \}, \and \\ \meaningof{E_1 | E_2} = \{ P \in \pi | P \equiv P_{1} | P_{2}, P_{1} \in \meaningof{E_{1}}, P_{2} \in \meaningof{E_2}\} }
\end{mathpar}

\begin{mathpar}
 \inferrule* [lab=behavior] {} {\meaningof{\langle a?b \rangle E} = \{ P \in \pi | P \equiv Q | u?(y)P', \\ \and \\\\ \and \\ \;\;\; u \in \meaningof{a}, \forall z.P'\{z/y\} \in \meaningof{E\{z/b\}}\}, \and \\ \meaningof{a!E} = \{ P \in \pi | P \equiv Q | x!\langle P' \rangle, x \in \meaningof{a} P' \in \meaningof{E}\} }
\end{mathpar}

\begin{mathpar}
 \inferrule* [lab=nominal] {} {\meaningof{\quotep{E}} = \{ \quotep{P} \in \quotep{\pi} | P \in \meaningof{E} \}, \and \meaningof{\quotep{P}} = \{ \quotep{Q} \in \quotep{\pi} | P \equiv Q \} \and \\ \meaningof{@\quotep{E}} = \{ P \in \pi | P \equiv @x, x \in \meaningof{E} \}}
\end{mathpar}

\begin{eqnarray*}
  \\
  \meaningof{-} : TS \to ST
\end{eqnarray*}

\begin{eqnarray*}
  \\
  L : TS \to ST
\end{eqnarray*}

\begin{eqnarray*}
  \\
  P \models E \iff P \in \meaningof{E}
\end{eqnarray*}

\begin{eqnarray*}
  P \approx_{L} Q \iff \forall E \in L. P \models E \iff Q \models E
\end{eqnarray*}

\begin{eqnarray*}
  P \approx_{K} Q
\end{eqnarray*}

\begin{eqnarray*}
  P \approx Q
\end{eqnarray*}

$\approx_{K} = \approx = \approx_{L}$

\subsubsection{Contextual duality}

Note that contexts extend the quotation operation to a family of
operations from processes to names. Given a context, $M$, we can
define a \emph{nominal context}, $\quotep{M}$ by $\quotep{M}[P] :=
\quotep{M[P]}$. To foreshadow what is to come we observe that these
operations enjoy a duality with processes very much like the duality
between vectors and maps from vectors to scalars.

Further, because the calculus is essentially higher-order, we have a
correspondence between contexts and processes. More specifically,
given a name $x$ and a context $M$ we can construct $M^{*}_{x}$ such
that 

\begin{mathpar}
  M^{*}_{x} | \lift{x}{P} \red M[P]
\end{mathpar}

namely,

\begin{mathpar}
  M^{*}_{x} := x?(u).M[\dropn{u}]
\end{mathpar}

The dependence of $M^{*}_{x}$ on a name makes it an abstraction, 

\begin{mathpar}
  M^{*} := (x)x?(u).M[\dropn{u}]
\end{mathpar}

\subsection{Additional notation}

It will sometimes be convenient to denote the process a name
quotes. We already have the notation $x = \quotep{P}$, but it will be
convenient to introduce an alternate notation, $\procn{x}$, when we
want to emphasize the connection to the use of the name. Note that, by
virtue of name equivalence, $\quotep{\procn{x}} \nameeq x$; so, the
notation is consistent with previous definitions.

Further, because names have structure it is possible to effect
substitutions on the basis of that structure. This means we need to
upgrade our notation for substitutions, which we accomplish by
adapting comprehension notation. Thus,

\begin{mathpar}
  P\{ y / x : x \in S \}
\end{mathpar}

is interpreted to mean the process derived from P by replacing (in a
capture-avoiding manner) each occurrence of $x$ in $S$ by $y$. For example,

\begin{mathpar}
  P\{ \quotep{\procn{x}|\procn{x}} / x : x \in \freenames{P} \}
\end{mathpar}

will replace each (occurrence) of a free name $x$ in $P$ by
$\quotep{\procn{x}|\procn{x}}$.

Also, we will avail ourselves of the notation $x^{L}$ and $x^{R}$ to
denote injections of a name into disjoint copies of the name
space. There are numerous ways to accomplish this. One example can be
found in \cite{MeredithR05}. This notation overloads to vectors of
names: $\vec{x}^{\pi} := (x_{i}^{\pi} \; : \; 0 \leq i < |\vec{x}| )$ where $\pi \in \{L,R\}$.

We also use $P^{\Box} := P|\Box$.

In \cite{MeredithR05} an interpretation of the new operator is
given. It turns out that there are several possible interpretations
all enjoying the requisite algebraic properties of the operator (see
\cite{milner91polyadicpi}). We will therefore make liberal use of
$(\nu\; \vec{x})P$.

% subsection the_syntax_and_semantics_of_the_notation_system (end)   

\input{qm2pi.qmops} 

\input{qm2pi.sterngerlach} 

\input{qm2pi.metric} 

% section concurrent_process_calculi (end)

%\input{qm2pi.proofsketch}

% section proof sketch (end)

%\input{qm2pi.slviaknots} 

% section spatial logic via knots (end)

\input{qm2pi.conclusion}

% section conclusion (end)

%\input{qm2pi.dtcodes} 

% section wiring algorithm (end)

\input{qm2pi.ack} 

% section acknowledgments (end)

\newpage


\bibliographystyle{plain}   
\bibliography{../../biblios/main.bib}

\input{qm2pi.rhodetails}

\end{document}

 

% section concurrent_process_calculi (end)

%\documentclass[12pt]{llncs}
%\documentclass{jktr}

\usepackage[pdftex]{hyperref}                   
\usepackage {listings}
\usepackage {mathpartir}
\usepackage{bcprules}
%\usepackage{listings}
                       
\usepackage{graphicx} 
%\usepackage[margins=2.5cm,nohead,nofoot]{geometry}
%\usepackage{geometry}
\usepackage{amsfonts}
\usepackage{amstext}
\usepackage{latexsym}
\usepackage{amssymb}
\usepackage{color}


%\include{myPreamble}
\include{qm2pi.local} 

%\ifpdf
%\usepackage[pdftex]{graphicx}
%\else
%\usepackage{graphicx}
%\fi

 % \ifpdf
%  \usepackage{pdfsync}
%  \if


%\title{Brief Article}
%\author{David F. Snyder}
%\author{L.G. Meredith}

%\address{Dept. of Math., Texas State University--San Marcos, San Marcos, TX 78666}
       
\pagestyle{empty}


\begin{document}

\lstset{language=[Objective]Caml,frame=shadowbox}

\input{qm2pi.front}

% section front matter (end)

\input{qm2pi.intro} 
 
% section introduction (end)

% \input{qm2pi.knotations} 

% section notation (end)

\input{qm2pi.process.calculi} 

% section concurrent_process_calculi_and_spatial_logics_ (end)
    
%\input{qm2pi.knots2pi} 

%\input{qm2pi.trefoil} 

%\input{qm2pi.mainthm} 

% subsection basic_interpretation (end)

%\input{qm2pi.rho.presentation} 
\subsection{The syntax and semantics of the notation system}\label{sub:the_syntax_and_semantics_of_the_notation_system} % (fold)

We now summarize a technical presentation of the calculus that
embodies our theory of dynamics. The typical presentation of such a
calculus follows the style of giving generators and relations on
them. The grammar, below, describing term constructors, freely
generates the set of processes, $\Proc$. This set is then quotiented
by a relation known as structural congruence and it is over this set
that the notion of dynamics is expressed. This presentation is
essentially that of \cite{MeredithR05} with the addition of
polyadicity and summation. For readability we have relegated some of
the technical subtleties to an appendix.

\subsubsection{Process grammar}\label{subsub:process_grammar}

\begin{mathpar}
  \inferrule* [lab=synchronization] {} {{M} \bc \pzero \;|\; x?F \;|\; x!C }
  \and
  \inferrule* [lab=abstraction] {} {{F} \bc (x)P}
  \and
  \inferrule* [lab=concretion] {} {{C} \bc \langle Q \rangle}
  \and
  \inferrule* [lab=process] {} {{P,Q} \bc M \;| \;P|Q \;|\; @{x}}
  \and
  \inferrule* [lab=name] {} {{x} \bc \quotep{P}}
\end{mathpar} 

Note that $\vec{x}$ (resp. $\vec{P}$) denotes a vector of names
(resp. processes) of length $|\vec{x}|$ (resp. $|\vec{P}|$). We adopt
the following useful abbreviations.

\begin{mathpar}
   x?(\vec{y}).P := x.(\vec{y})P \and  x\clift{\vec{P}} := x.\clift{\vec{P}}
   \and x!(y) := \lift{x}{\dropn{y}}
   \and \Pi_{i=0}^{n-1}P_i := P_0 | \ldots | P_{n-1}
\end{mathpar}

\subsubsection{Structural congruence}

\paragraph{Free and bound names and alpha-equivalence.} At the
core of structural equivalence is alpha-equivalence which identifies
process that are the same up to a change of variable. Formally, we
recognize the distinction between free and bound names. The free names
of a process, $\freenames{P}$, may be calculated recursively as
follows:

\begin{mathpar}
\freenames{\pzero} := \emptyset
  \and \\
  \freenames{x?(y).P} := \{ x \} \cup (\freenames{P} \setminus \{ y \})
  \and 
  \freenames{x!\langle P \rangle} := \{ x \} \cup \{ P \} 
  \and \\
  \freenames{P|Q} := \freenames{P} \cup \freenames{Q}
  \and \\
  \freenames{@{x}} := \{ x \}
\end{mathpar}

$\pi$
$\quotep{\pi}$

$\freenames{-} : \pi \to \mathcal{P}(\quotep{\pi})$

\begin{eqnarray*}
  \freenames{\pzero} & := & \emptyset \\
  \freenames{x?(y).P} & := & \{ x \} \cup (\freenames{P} \setminus \{ y \}) \\
  \freenames{x!\langle P \rangle} & := & \{ x \} \cup \{ P \} \\
  \freenames{P|Q} & := & \freenames{P} \cup \freenames{Q} \\
  \freenames{\dropn{x}} & := & \{ x \}
\end{eqnarray*}

The bound names of a process, $\boundnames{P}$, are those names occurring in $P$
that are not free. For example, in $x?(y).0$, the name $x$ is free, while $y$ is bound.

\begin{mathpar}
  \inferrule* [lab=monoidal-laws] {} { P|Q \equiv Q|P \and P|0 \equiv P \and P|(Q|R) \equiv (P|Q)|R }
\end{mathpar}

\begin{mathpar}
  \inferrule* [lab=alpha-equivalence] {} { (x)P \equiv (y)P\{y/x\} \and y \not\in \freenames{P} }
\end{mathpar}

\begin{definition}
Then two processes, $P,Q$, are alpha-equivalent if $P = Q\{\vec{y}/\vec{x}\}$ for
some $\vec{x} \in \boundnames{Q},\vec{y} \in \boundnames{P}$, where $Q\{\vec{y}/\vec{x}\}$
denotes the capture-avoiding substitution of $\vec{y}$ for $\vec{x}$ in $Q$.
\end{definition}

\begin{definition}
  The {\em structural congruence} \cite{SangiorgiWalker} , $\equiv$,
  between processes is the least congruence containing
  alpha-equivalence, satisfying the abelian monoid laws
  (associativity, commutativity and $\pzero$ as identity) for parallel
  composition $|$ and for summation $+$.
\end{definition}

\subsection{Name equivalence}

We take name equivalence, written $\nameeq$, to be the smallest
equivalence relation generated by the following rules.

\begin{mathpar}
\inferrule*[lab=Quote-drop]
{ }
{ \quotep{@{x}} \nameeq x }

\inferrule*[lab=Struct-equiv]
{ P \scong Q }
{ \quotep{P} \nameeq \quotep{Q} }
\end{mathpar}

The astute reader will have noticed that the mutual recursion of names
and processes imposes a mutual recursion on alpha-equivalence and
structural equivalence via name-equivalence. Fortunately, all of this
works out pleasantly and we may calculate in the natural way, free of
concern. The reader interested in the details is referred to the
appendix \ref{appendix:rho_details}.

\subsection{Substitution}

We use $\Proc$ for the set of processes, $\QProc$ for the set of
names, and $\id{\{}\vec{y} / \vec{x} \id{\}}$ to denote partial maps,
$s : \QProc \rightarrow \QProc$. A map, $s$ lifts, uniquely, to a map
on process terms, $\widehat{s} : \Proc \rightarrow \Proc$ by the
following equations.

\begin{mathpar}
  (0) \psubstp{Q}{P} := 0 \\
  (R \juxtap S) \psubstp{Q}{P}
  :=    
  (R)\psubstp{Q}{P} \juxtap (S) \psubstp{Q}{P} \\
  (x?(y).R) \psubstp{Q}{P}    
  :=    
  (x)\substp{Q}{P} (z)\concat( (R \psubstn{z}{y}) \psubstp{Q}{P} ) \\
  (\lift{x}{R}) \psubstp{Q}{P}  
  :=
  \lift{(x)\substp{Q}{P}}{ R \psubstp{Q}{P} } \\
%   (\dropn{x})  \psubstp{Q}{P}       
%   := 
%   \left\{ 
%     \begin{array}{ccc} 
%       \dropn{\quotep{Q}} & & x \nameeq \quotep{P} \\
%       \dropn{x} & & otherwise \\
%     \end{array}
%   \right. 
  (\dropn{x})  \psubstp{Q}{P}       
  := 
  \left\{ 
    \begin{array}{ccc} 
      Q & & x \nameeq \quotep{P} \\
      \dropn{x} & & otherwise \\
    \end{array}
  \right.
\end{mathpar}
 

where

\begin{eqnarray}
  (x)\id{\{} \lpquote Q \rpquote / \lpquote P \rpquote \id{\}}            = 
  \left\{ 
    \begin{array}{ccc}
      \lpquote Q \rpquote & & x \nameeq \lpquote P \rpquote \\
      x & & otherwise \\
    \end{array}
  \right. \nonumber
\end{eqnarray}

and $z$ is chosen distinct from $\quotep{P}$, $\quotep{Q}$, the free
names in $Q$, and all the names in $R$. Our $\alpha$-equivalence will
be built in the standard way from this substitution.

\begin{remark}\label{rem:no_self_referential_names}
  One consequence of these definitions is that $\forall P. \quotep{P}
  \not\in \freenames{P}$.
\end{remark}

\subsection{ Dynamic quote: an example }

Anticipating something of what's to come, consider applying the
substitution, $\widehat{\id{\{}u / z \id{\}}}$, to the following pair
of processes, $\lift{w}{y!(z)}$ and $w[ \lpquote y!(z) \rpquote ]$.

\begin{eqnarray}
	\lift{w}{y!(z)}\widehat{\id{\{}u / z \id{\}}}
		& = &
		\lift{w}{y!(u)} \nonumber\\
	w[ \lpquote y!(z) \rpquote ] \widehat{ \id{\{}u / z \id{\}} }
		& = &
		w[ \lpquote y!(z) \rpquote ] \nonumber
\end{eqnarray}

Because the body of the process between quotes is impervious to
substitution, we get radically different answers. In fact, by
examining the first process in an input context,
e.g. $x?(z).\lift{w}{y!(z)}$, we see that the process under the lift
operator may be shaped by prefixed inputs binding a name inside it. In
this sense, the lift operator will be seen as a way to dynamically
construct processes before reifying them as names.

Finally equipped with these standard features we can present the
dynamics of the calculus.

\subsubsection{Operational semantics} 

Finally, we introduce the computational dynamics. What marks these
algebras as distinct from other more traditionally studied algebraic
structures, e.g. vector spaces or polynomial rings, is the manner in
which dynamics is captured. In traditional structures, dynamics is typically
expressed through morphisms between such structures, as in linear maps
between vector spaces or morphisms between rings. In algebras
associated with the semantics of computation, the dynamics is
expressed as part of the algebraic structure itself, through a
reduction reduction relation typically denoted by $\red$. Below, we
give a recursive presentation of this relation for the calculus used
in the encoding.

$\red \subseteq \pi \times \pi$
$\red : \pi \to \mathcal{P}(\pi)$

\begin{mathpar}
  \inferrule* [lab=Comm] { \textsf{match}( x_{src}, x_{trgt} ) } { x_{trgt}?(y)P \; | \; x_{src}!\langle {Q} \rangle \red P\{\quotep{Q}/y}\} }
  \and \\
  \inferrule* [lab=Par] {{P} \red {P}'} {{{P} | {Q}} \red {{P}' | {Q}}}
  \and
  \inferrule* [lab=Equiv]{{{P} \scong {P}'} \andalso {{P}' \red {Q}'} \andalso {{Q}' \scong {Q}}}{{P} \red {Q}}
\end{mathpar}

\begin{eqnarray*}
  match_{\equiv} (\quotep{P},\quotep{Q}) & := & P \equiv Q \\
  match_{\dagger}(\quotep{P},\quotep{Q}) & := & \forall R. P|Q \red^{*} R => R \red^{*} 0 \\
  match_{K}(\quotep{P},\quotep{Q}) & := & K \mbox{ for some context } K
\end{eqnarray*}

$u?(x)P | u!\langle Q \rangle \red P\{\quotep{Q}/x\}$

%We write $\wred$ for $\red^*$, and $P\red$ if $\exists Q $ such that $ P \red Q$.
We write $P\red$ if $\exists Q $ such that $ P \red Q$ and $P\not\red$, otherwise.

\section{Replication}

As mentioned before, it is known that replication (and hence
recursion) can be implemented in a higher-order process algebra
\cite{SangiorgiWalker}. As our first example of calculation with the
machinery thus far presented we give the construction explicitly in
the {\rhoc}.

\begin{eqnarray}
	D_{x} & := & \prefix{x}{y}{(\binpar{\outputp{x}{y}}{@{y}})} \nonumber\\
	\bangp_{x}{P} & := & \binpar{{x}!\langle{\binpar{D_{x}}{P}}\rangle}{D_{x}} \nonumber
\end{eqnarray}

\begin{eqnarray}
	\bangp_{x}{P} & & \nonumber\\
	=
	& {x}!\langle{(\prefix{x}{y}{(\outputp{x}{y} | @{y})) | P}}\rangle 
	      | \prefix{x}{y}{(\outputp{x}{y} | @{y})} & \nonumber\\
	\red
	& (\outputp{x}{y} | @{y})\substn{\quotep{(\prefix{x}{y}{(@{y} | \outputp{x}{y})) | P}}}{y} & \nonumber\\
	=
	& \outputp{x}{\quotep{(\prefix{x}{y}{(\outputp{x}{y} | @{y})) | P}}}
	  | {(\prefix{x}{y}{(\outputp{x}{y} | @{y})) | P}} & \nonumber\\
	\red
	& \ldots & \nonumber\\
	\red^*
	& P | P | \ldots & \nonumber
\end{eqnarray}

Of course, this encoding, as an implementation, runs away, unfolding
$\bangp{P}$ eagerly. A lazier and more implementable replication
operator, restricted to input-guarded processes, may be obtained as follows.

\begin{eqnarray}
\bangp{\prefix{u}{v}{P}} 
	:= 
	\binpar{\lift{x}{\prefix{u}{v}{(\binpar{D(x)}{P})}}}{D(x)} \nonumber
\end{eqnarray}

\begin{remark}
  Note that the lazier definition still does not deal with summation
  or mixed summation (i.e. sums over input and output). The reader is
  invited to construct definitions of replication that deal with these
  features. 

  Further, the definitions are parameterized in a name, $x$. Can you,
  gentle reader, make a definition that eliminates this parameter and
  guarantees no accidental interaction between the replication
  machinery and the process being replicated -- i.e. no accidental
  sharing of names used by the process to get its work done and the
  name(s) used by the replication to effect copying. This latter
  revision of the definition of replication is crucial to obtaining
  the expected identity $!!P \sim !P$.
\end{remark}

\begin{remark}\label{rem:paradoxical_combinator}
  The reader familiar with the lambda calculus will have noticed the
  similarity between $D$ and the paradoxical combinator.

  [Ed. note: the existence of this seems to suggest we have to be more
  restrictive on the set of processes and names we admit if we are to
  support no-cloning.]
\end{remark}

\subsubsection{Bisimulation}

The computational dynamics gives rise to another kind of equivalence,
the equivalence of computational behavior. As previously mentioned
this is typically captured \emph{via} some form of bisimulation.

% The notion we use in this paper is weak barbed bisimulation
% \cite{milner91polyadicpi}.

The notion we use in this paper is derived from weak barbed
bisimulation \cite{milner91polyadicpi}. 

\begin{definition}
An \emph{observation relation}, $\downarrow_{\mathcal N}$, over a set
of names, $\mathcal N$, is the smallest relation satisfying the rules
below.

\infrule[Out-barb]{y \in {\mathcal N}, \; x \nameeq y}
		  {\outputp{x}{v} \downarrow_{\mathcal N} x}
\infrule[Par-barb]{\mbox{$P\downarrow_{\mathcal N} x$ or $Q\downarrow_{\mathcal N} x$}}
		  {\binpar{P}{Q} \downarrow_{\mathcal N} x}

We write $P \Downarrow_{\mathcal N} x$ if there is $Q$ such that 
$P \wred Q$ and $Q \downarrow_{\mathcal N} x$.
\end{definition}

\begin{definition}
%\label{def.bbisim}
An  ${\mathcal N}$-\emph{barbed bisimulation} over a set of names, ${\mathcal N}$, is a symmetric binary relation 
${\mathcal S}_{\mathcal N}$ between agents such that $P\rel{S}_{\mathcal N}Q$ implies:
\begin{enumerate}
\item If $P \red P'$ then $Q \wred Q'$ and $P'\rel{S}_{\mathcal N} Q'$.
\item If $P\downarrow_{\mathcal N} x$, then $Q\Downarrow_{\mathcal N} x$.
\end{enumerate}
$P$ is ${\mathcal N}$-barbed bisimilar to $Q$, written
$P \wbbisim_{\mathcal N} Q$, if $P \rel{S}_{\mathcal N} Q$ for some ${\mathcal N}$-barbed bisimulation ${\mathcal S}_{\mathcal N}$.
\end{definition}

$\mathcal{R} \subseteq \pi \times \pi$

$P \mathcal{R} Q => \forall P'. P \red P' \Rightarrow \exists Q'. Q \red Q', P' \mathcal{R} Q'$

$P \vdash x \Rightarrow Q \vdash x$

\begin{mathpar}
  \inferrule*[lab=Out-barb]{x \nameeq y}{{y}!\langle{Q}\rangle \vdash x}
  \and
  \inferrule*[lab=Par-barb]{\mbox{$P\vdash x$ or $Q\vdash x$}}{\binpar{P}{Q} \vdash x}
\end{mathpar}

\subsubsection{Contexts}

One of the principle advantages of computational calculi like the
$\pi$-calculus is a well-defined notion of context,
contextual-equivalence and a correlation between
contextual-equivalence and notions of bisimulation. The notion of
context allows the decomposition of a process into (sub-)process and
its syntactic environment, its context. Thus, a context may be
thought of as a process with a ``hole'' (written $\Box$) in it. The
application of a context $M$ to a process $P$, written $M[P]$, is
tantamount to filling the hole in $M$ with $P$. In this paper we do
not need the full weight of this theory, but do make use of the notion
of context in the proof the main theorem. 

\begin{mathpar}
  \inferrule* [lab=summation] {} {{M_{M},M_{N}} \bc \Box \;|\; x.M_{A} \;|\; M_{M}+M_{N}}
  \and
  \inferrule* [lab=agent] {} {{M_{A}} \bc (\vec{x})M_{P} \;| \; \clift{P_0,\ldots,M_{P},\ldots,P_N}}
  \and \\
  \inferrule* [lab=process] {} {{M_{P}} \bc M_{N} \;| \;P|M_{P} }
\end{mathpar} 

\begin{mathpar}
  \inferrule* [lab=sychronization] {} {M_{N} \bc \Box \;|\; x?M_{F} \;|\; x!M_{C}}
  \and
  \inferrule* [lab=abstraction] {} {{M_{F}} \bc (x)M_{P} }
  \and
  \inferrule* [lab=concretion] {} {{M_{C}} \bc \langle M_{P} \rangle }
  \and \\
  \inferrule* [lab=process] {} {{M_{P}} \bc M_{N} \;| \;P|M_{P} }
\end{mathpar}

\begin{definition}[contextual application] Given a context $M$, and
  process $P$, we define the \emph{contextual application}, $M[P] :=
  M\{P/\Box\}$. That is, the contextual application of M to P is the
  substitution of $P$ for $\Box$ in $M$.
\end{definition}

$\meaningof{-} : L \to \mathcal{P}(\pi)$

\begin{mathpar}
  \inferrule* [lab=collection] {} {\meaningof{true} = \pi, \and \meaningof{~E} = \pi \setminus \meaningof{E}, \and \meaningof{E_{1} \& E_{2}} = \meaningof{E_{1}} \cap \meaningof{E_{2}}}
\end{mathpar}

\begin{mathpar}
  \inferrule* [lab=structure] {} {\meaningof{0} = \{ P \in \pi | P \equiv 0 \}, \and \\ \meaningof{E_1 | E_2} = \{ P \in \pi | P \equiv P_{1} | P_{2}, P_{1} \in \meaningof{E_{1}}, P_{2} \in \meaningof{E_2}\} }
\end{mathpar}

\begin{mathpar}
 \inferrule* [lab=behavior] {} {\meaningof{\langle a?b \rangle E} = \{ P \in \pi | P \equiv Q | u?(y)P', \\ \and \\\\ \and \\ \;\;\; u \in \meaningof{a}, \forall z.P'\{z/y\} \in \meaningof{E\{z/b\}}\}, \and \\ \meaningof{a!E} = \{ P \in \pi | P \equiv Q | x!\langle P' \rangle, x \in \meaningof{a} P' \in \meaningof{E}\} }
\end{mathpar}

\begin{mathpar}
 \inferrule* [lab=nominal] {} {\meaningof{\quotep{E}} = \{ \quotep{P} \in \quotep{\pi} | P \in \meaningof{E} \}, \and \meaningof{\quotep{P}} = \{ \quotep{Q} \in \quotep{\pi} | P \equiv Q \} \and \\ \meaningof{@\quotep{E}} = \{ P \in \pi | P \equiv @x, x \in \meaningof{E} \}}
\end{mathpar}

\begin{eqnarray*}
  \\
  \meaningof{-} : TS \to ST
\end{eqnarray*}

\begin{eqnarray*}
  \\
  L : TS \to ST
\end{eqnarray*}

\begin{eqnarray*}
  \\
  P \models E \iff P \in \meaningof{E}
\end{eqnarray*}

\begin{eqnarray*}
  P \approx_{L} Q \iff \forall E \in L. P \models E \iff Q \models E
\end{eqnarray*}

\begin{eqnarray*}
  P \approx_{K} Q
\end{eqnarray*}

\begin{eqnarray*}
  P \approx Q
\end{eqnarray*}

$\approx_{K} = \approx = \approx_{L}$

\subsubsection{Contextual duality}

Note that contexts extend the quotation operation to a family of
operations from processes to names. Given a context, $M$, we can
define a \emph{nominal context}, $\quotep{M}$ by $\quotep{M}[P] :=
\quotep{M[P]}$. To foreshadow what is to come we observe that these
operations enjoy a duality with processes very much like the duality
between vectors and maps from vectors to scalars.

Further, because the calculus is essentially higher-order, we have a
correspondence between contexts and processes. More specifically,
given a name $x$ and a context $M$ we can construct $M^{*}_{x}$ such
that 

\begin{mathpar}
  M^{*}_{x} | \lift{x}{P} \red M[P]
\end{mathpar}

namely,

\begin{mathpar}
  M^{*}_{x} := x?(u).M[\dropn{u}]
\end{mathpar}

The dependence of $M^{*}_{x}$ on a name makes it an abstraction, 

\begin{mathpar}
  M^{*} := (x)x?(u).M[\dropn{u}]
\end{mathpar}

\subsection{Additional notation}

It will sometimes be convenient to denote the process a name
quotes. We already have the notation $x = \quotep{P}$, but it will be
convenient to introduce an alternate notation, $\procn{x}$, when we
want to emphasize the connection to the use of the name. Note that, by
virtue of name equivalence, $\quotep{\procn{x}} \nameeq x$; so, the
notation is consistent with previous definitions.

Further, because names have structure it is possible to effect
substitutions on the basis of that structure. This means we need to
upgrade our notation for substitutions, which we accomplish by
adapting comprehension notation. Thus,

\begin{mathpar}
  P\{ y / x : x \in S \}
\end{mathpar}

is interpreted to mean the process derived from P by replacing (in a
capture-avoiding manner) each occurrence of $x$ in $S$ by $y$. For example,

\begin{mathpar}
  P\{ \quotep{\procn{x}|\procn{x}} / x : x \in \freenames{P} \}
\end{mathpar}

will replace each (occurrence) of a free name $x$ in $P$ by
$\quotep{\procn{x}|\procn{x}}$.

Also, we will avail ourselves of the notation $x^{L}$ and $x^{R}$ to
denote injections of a name into disjoint copies of the name
space. There are numerous ways to accomplish this. One example can be
found in \cite{MeredithR05}. This notation overloads to vectors of
names: $\vec{x}^{\pi} := (x_{i}^{\pi} \; : \; 0 \leq i < |\vec{x}| )$ where $\pi \in \{L,R\}$.

We also use $P^{\Box} := P|\Box$.

In \cite{MeredithR05} an interpretation of the new operator is
given. It turns out that there are several possible interpretations
all enjoying the requisite algebraic properties of the operator (see
\cite{milner91polyadicpi}). We will therefore make liberal use of
$(\nu\; \vec{x})P$.

% subsection the_syntax_and_semantics_of_the_notation_system (end)   

\input{qm2pi.qmops} 

\input{qm2pi.sterngerlach} 

\input{qm2pi.metric} 

% section concurrent_process_calculi (end)

%\input{qm2pi.proofsketch}

% section proof sketch (end)

%\input{qm2pi.slviaknots} 

% section spatial logic via knots (end)

\input{qm2pi.conclusion}

% section conclusion (end)

%\input{qm2pi.dtcodes} 

% section wiring algorithm (end)

\input{qm2pi.ack} 

% section acknowledgments (end)

\newpage


\bibliographystyle{plain}   
\bibliography{../../biblios/main.bib}

\input{qm2pi.rhodetails}

\end{document}



% section proof sketch (end)

%\section{Unlikely characters: spatial logic for
  knots}\label{sub:characteristic_formulae} % (fold)

Associated to the mobile process calculi are a family of logics known
as the Hennessy-Milner logics. These logics typically enjoy a
semantics interpreting formulae as sets of processes that when
factored through the encoding outlined above allows an identification
of classes of knots with logical formulae. In the context of this
encoding the sub-family known as the spatial logics \cite{CairesC03}
\cite{CairesC04} \cite{Caires04} are of particular interest providing
several important features for expressing and reasoning about
properties (i.e. classes) of knots. We hint here at how this may be done.

%\begin{description}
%\item [structural connectives] 
\subsubsection{Structural connectives} The spatial logics enjoy
structural connectives corresponding, at the logical level, to the
parallel composition ($P | Q$) and new name ($(\nu \; x)P$)
connectives for processes. As illustrated in the examples below, these
connectives are extremely expressive given the shape of our encoding.
%\item [decideable satisfaction]

\subsubsection{Decideable satisfaction}
In \cite{Caires04} the satisfaction relation is shown to be decideable
for a rich class of processes. It further turns out that the image of
the our encoding is a proper subset of that class. This result
provides the basis for an algorithm by which to search for knots
enjoying a given property.
%\item [characteristic formulae]

\subsubsection{Characteristic formulae}
In the same paper \cite{Caires04} , Caires presents a means of calculating
characteristic formulae, selecting equivalence classes of processes
up to a pre--specified depth limit on the support set of names. Composed with our
encoding, this characteristic formula can be used to select
characteristic formulae for knots.
%\end{description}

\subsubsection{Spatial logic formulae}

The grammar below (segmented for comprehension) summarizes the syntax
of spatial logic formulae. We employ illustrative examples in the
sequel to provide an intuitive understanding of their meaning
referring the reader to \cite{Caires04} for a more detailed explication
of the semantics.

\begin{mathpar}
  \inferrule* [lab=boolean] {} {{A,B} \bc T \;|\; \neg A \;|\; A \wedge B \;|\; \eta = \eta'}
  \and
  \inferrule* [lab=spatial] {} {|\; \pzero \;|\; A | B \;|\; x \text{\textregistered} A \;|\; \forall x . A \;|\;  H x . A}
  \and
  \inferrule* [lab=behavioral] {} {|\; \alpha . A}
  \and 
  \inferrule* [lab=recursion] {} {|\; X(\vec{u}) \;|\; \mu X(\vec{u}) . A}
  \and
  \inferrule* [lab=action] {} {\alpha \bc \langle x?(\vec{y}) \rangle \;|\; \langle x!(\vec{y}) \rangle \;|\; \langle \tau \rangle}
  \and 
  \inferrule* [lab=name] {} {\eta \bc x \;|\; \tau}
\end{mathpar} 

% subsection characteristic_formulae (end)   	 

\subsection{Example formulae}\label{sub:example_formulae_} % (fold)

\subsubsection{Crossing as formula.}
% 
% \begin{align*}
%   \frac{d}{dx} \sin x &= \cos x 
%   & \frac{d}{dx} e^x &= e^x \\
%   \frac{d}{dx} \cos x &= - \sin x 
%   & \frac{d}{dx} \log x &= \frac{1}{x} \\
% \end{align*} 

\begin{align*}
 \mu C(x_{0},x_{1},y_{0},y_{1},u).&(\langle x_{0}?(z) \rangle(\langle u! \rangle\langle y_{1}!z \rangle C(x_{0},x_{1},y_{0},y_{1},u)) & \\
  & \wedge \langle y_{1}?(z) \rangle (\langle u! \rangle \langle x_{0}!z \rangle C(x_{0},x_{1},y_{0},y_{1},u)) & \\
  & \wedge \langle x_{1}?(z) \rangle (\langle u? \rangle \langle y_{0}!z \rangle C(x_{0},x_{1},y_{0},y_{1},u)) & \\
  & \wedge \langle y_{0}?(z) \rangle (\langle u? \rangle \langle x_{1}!z \rangle C(x_{0},x_{1},y_{0},y_{1},u))) &
\end{align*}

The lexicographical similarity between the shape of this formulae and
the shape of definition of the process representing a crossing reveals
the intuitive meaning of this formulae. It describes the capabilities
of a process that has the right to represent a crossing. For example
it picks out processes that may perform an input on the port $x_0$ in
its initial menu of capabilities. What differentiates the formula
from the process, however, is that the crossing process is the
smallest candidate to satisfy the formula. Infinitely many other
processes -- with internal behavior hidden behind this interface, so
to speak -- also satisfy this formula. Even this simple formula,
then, can be seen to open a new view onto knots, providing a
computational interpretation of \emph{virtual} knots.

Note that this formula is derived by hand. A similar formula can be
derived by employing Caires' calculation of characteristic formula
\cite{Caires04} to the process representing a crossing. In light of
this discussion, we let
$\meaningof{C}_{\phi}(x0,x1,y0,y1,u)$ denote a formula specifying the
dynamics we wish to capture of a crossing. To guarantee we preserve
the shape of the interface and minimal semantics we demand that
$\meaningof{C}_{\phi}(x0,x1,y0,y1,u) \Rightarrow
\textbf{C}(x0,x1,y0,y1,u)$ where $\textbf{C}(x0,x1,y0,y1,u)$ denotes
the formula above.
                            
\subsubsection{Crossing number constraints.}
The moral content of the context lemma (Lemma \ref{context}) is that the notion of
``locality'' in the Reidemeister moves is effectively captured by the
parallel composition operator of the process calculus. This intuition
extends through the logic. Given a formula,
$\meaningof{C}_{\phi}(x0,x1,y0,y1,u)$, we can use the structural
connectives to specify constraints on crossing numbers, such as at
least $n$ crossings, or exactly $n$ crossings.
\begin{mathpar}
  \inferrule* [lab=at-least-n] {} { K^{\geq n}_{\phi}(\vec{xs},\vec{ys}) := \Pi_{i=0}^{n-1} Hu . \meaningof{C}_{\phi}(xs_i,ys_i,u) | T }
  \and 
  \inferrule* [lab=exactly-n] {} { K^{= n}_{\phi}(\vec{xs},\vec{ys}) := \Pi_{i=0}^{n-1} Hu . \meaningof{C}_{\phi}(xs_i,ys_i,u) | \neg (\forall x_0,y_0,x_1,y_1,u . \meaningof{C}_{\phi}(x_0,y_0,x_1,y_1,u) | T) }
\end{mathpar}

To round out this section, recall that the encoding of an $n$-crossing
knot decomposes into a parallel composition of $n$ \emph{copies} of a
crossing process together with a wiring harness. To specify different
knot classes with the same crossing number amounts to specifying
logical constraints on the wiring harness. In the interest of space,
we defer examples to a forthcoming paper. Suffice it to say that both
the conditions ``alternating knot'' and ``contains the tangle
corresponding to 5/3'' are expressible. For example, it is possible to
calculate the characteristic formula of a process corresponding to the
tangle 5/3 and conjoin it into the classifying formula via the
composition connective of the logic.

Finally, we wish to observe that it is entirely within reason to
contemplate a more domain-specific version of spatial logic tailored
to the shape of processes in the image of the encoding. Such a
domain-specific logic would have a better claim to the title formal
language of knot properties.

% subsection example_formulae_ (end)

% section knots_as_processes (end) 

% section spatial logic via knots (end)

\section{Conclusions and future work}

\paragraph{Testing physical space}
You, gentle reader, may wonder why of all the theorems to be proved
given this set up we pick the one above. In some sense it's hardly
central to quantum mechanics. We see it as central in the sense that
it firmly establishes a notion of physical space arising from a notion
of the equivalence of behavior. Relating bisimulation to a metric is a
big step forward, but one is faced with interpreting the relationship
of that metric space to something more physical. Quantum mechanical
notions of ``physical'' space are still far from intuitive, but by
relating this idea of distance as testing to calculations that predict
physical circumstances we are making a not insignificant step forward
toward an understanding of the physical space we inhabit as
essentially dynamic.

\paragraph{Effectivity and simulation}
One of the observations we have yet to make is that the entire program
spelled out here is effective. We have built various interpreters for
the reflective calculus at work in this interpretation. In principle,
then, we can simulate quantum mechanics on a computer. The place where
the simulation may lose fidelity is the infinitely branching summation
for the annihilator.

In this connection i also want to point out that the evaluation style
calculation of the inner product puts the non-determinism of the
summation right at the heart of measurement. This suggests that
Milner's original reduction-based formulation of the dynamics of his
calculi in terms of sums was not just notationally suggestive of a
notion of measure-and-continue but captured some significant part of
the physics.

\paragraph{Quantum continuations}
In light of this last observation i want to point out that the
predominant account of quantum mechanics is missing a key aspect of a
truly compositional story of the physical situation. In a real lab,
when a measurement is made the observation can be made to feed into
another device that then makes another measurement conditioned on the
results of the first. This means that after the superposition was
collapsed the entire experimental set up remained in
superposition. While QM offers a means of writing this down it doesn't
quite line up well with the well-trodden formulation of computation
and continuation that we see so succinctly expressed in Milner's
calculi. This suggests that there might be advantages to this account
of dynamics waiting to be explored.

\paragraph{Quantum logic}
In this connection, we also note that by virtue of having the
Hennessy-Milner construction, we can pull the construction through the
interpretation of QM. This gives us a natural candidate for a quantum
logic that enjoys an extremely tight connection with it's domain of
interpretation, making the construction much less ad hoc (rather it is
the image of functor!).

\paragraph{Quantum probabiity}
i have questions about the basis of the interpretation of inner
product as probability amplitude. In particular, using which
axiomatization of probability theory does the notion of probability
amplitude earn the right to be so dubbed? In other words, where is the
proof that the operation for calculating a probability amplitude (and
then squaring) satisfies the axioms of what it means to calculate a
probability? Even if such a proof exists (i have yet to find it in the
literature), i wonder if it might not be possible to turn things on
their heads. Can we view the calculation of the probability amplitude
as an axiomatization of probability? If so, then the definition we
give for calculating probability amplitude may provide the basis for
an \emph{effective} theory of probability.

\paragraph{Quantum vs ``biological'' information}
Finally, i want to conclude with a more philosophical observation. At
a recent workshop in which QM was a predominant topic i noticed
something about quantum information. The speaker was giving a riveting
discussion of axiomatic QM and showing how properties of ``no
cloning'' and ``no deleting'' emerged as consequences of the
axiomatization. Theorems of this form are necessary to give us a sense
of confidence that our axioms characterize the physical theory. What
struck me, though, was that if quantum information is neither erasable
nor replicable it is markedly different from \emph{life}. Two of the
things we know about life is that

\begin{itemize}
  \item it ends;
  \item to gain some measure of persistence, to transcend it's
    finitude it is imminently copyable.
\end{itemize}

Both of these qualities are summarized succinctly in the aphorism: all
flesh is grass. For me these two kinds of ``information'' -- call them
quantum and biological -- are end points on a spectrum of strategies
for persistence. At one end, we have those curious entities that enjoy
uniqueness and permanence; at the other, we have those who in the face
of a certain end and an uncertain present make a go of passing
something on. To me one of the more remarkable aspects of the latter
strategy is that in the presence of noise (and certain features of
copying) we get a kind of dynamism, a chance for improvement against a
given persistent condition.

% subsection other_calculi_other_bisimulations_and_geometry_as_behavior (end)




% section conclusion (end)

%\documentclass[12pt]{llncs}
%\documentclass{jktr}

\usepackage[pdftex]{hyperref}                   
\usepackage {listings}
\usepackage {mathpartir}
\usepackage{bcprules}
%\usepackage{listings}
                       
\usepackage{graphicx} 
%\usepackage[margins=2.5cm,nohead,nofoot]{geometry}
%\usepackage{geometry}
\usepackage{amsfonts}
\usepackage{amstext}
\usepackage{latexsym}
\usepackage{amssymb}
\usepackage{color}


%\include{myPreamble}
\include{qm2pi.local} 

%\ifpdf
%\usepackage[pdftex]{graphicx}
%\else
%\usepackage{graphicx}
%\fi

 % \ifpdf
%  \usepackage{pdfsync}
%  \if


%\title{Brief Article}
%\author{David F. Snyder}
%\author{L.G. Meredith}

%\address{Dept. of Math., Texas State University--San Marcos, San Marcos, TX 78666}
       
\pagestyle{empty}


\begin{document}

\lstset{language=[Objective]Caml,frame=shadowbox}

\input{qm2pi.front}

% section front matter (end)

\input{qm2pi.intro} 
 
% section introduction (end)

% \input{qm2pi.knotations} 

% section notation (end)

\input{qm2pi.process.calculi} 

% section concurrent_process_calculi_and_spatial_logics_ (end)
    
%\input{qm2pi.knots2pi} 

%\input{qm2pi.trefoil} 

%\input{qm2pi.mainthm} 

% subsection basic_interpretation (end)

%\input{qm2pi.rho.presentation} 
\subsection{The syntax and semantics of the notation system}\label{sub:the_syntax_and_semantics_of_the_notation_system} % (fold)

We now summarize a technical presentation of the calculus that
embodies our theory of dynamics. The typical presentation of such a
calculus follows the style of giving generators and relations on
them. The grammar, below, describing term constructors, freely
generates the set of processes, $\Proc$. This set is then quotiented
by a relation known as structural congruence and it is over this set
that the notion of dynamics is expressed. This presentation is
essentially that of \cite{MeredithR05} with the addition of
polyadicity and summation. For readability we have relegated some of
the technical subtleties to an appendix.

\subsubsection{Process grammar}\label{subsub:process_grammar}

\begin{mathpar}
  \inferrule* [lab=synchronization] {} {{M} \bc \pzero \;|\; x?F \;|\; x!C }
  \and
  \inferrule* [lab=abstraction] {} {{F} \bc (x)P}
  \and
  \inferrule* [lab=concretion] {} {{C} \bc \langle Q \rangle}
  \and
  \inferrule* [lab=process] {} {{P,Q} \bc M \;| \;P|Q \;|\; @{x}}
  \and
  \inferrule* [lab=name] {} {{x} \bc \quotep{P}}
\end{mathpar} 

Note that $\vec{x}$ (resp. $\vec{P}$) denotes a vector of names
(resp. processes) of length $|\vec{x}|$ (resp. $|\vec{P}|$). We adopt
the following useful abbreviations.

\begin{mathpar}
   x?(\vec{y}).P := x.(\vec{y})P \and  x\clift{\vec{P}} := x.\clift{\vec{P}}
   \and x!(y) := \lift{x}{\dropn{y}}
   \and \Pi_{i=0}^{n-1}P_i := P_0 | \ldots | P_{n-1}
\end{mathpar}

\subsubsection{Structural congruence}

\paragraph{Free and bound names and alpha-equivalence.} At the
core of structural equivalence is alpha-equivalence which identifies
process that are the same up to a change of variable. Formally, we
recognize the distinction between free and bound names. The free names
of a process, $\freenames{P}$, may be calculated recursively as
follows:

\begin{mathpar}
\freenames{\pzero} := \emptyset
  \and \\
  \freenames{x?(y).P} := \{ x \} \cup (\freenames{P} \setminus \{ y \})
  \and 
  \freenames{x!\langle P \rangle} := \{ x \} \cup \{ P \} 
  \and \\
  \freenames{P|Q} := \freenames{P} \cup \freenames{Q}
  \and \\
  \freenames{@{x}} := \{ x \}
\end{mathpar}

$\pi$
$\quotep{\pi}$

$\freenames{-} : \pi \to \mathcal{P}(\quotep{\pi})$

\begin{eqnarray*}
  \freenames{\pzero} & := & \emptyset \\
  \freenames{x?(y).P} & := & \{ x \} \cup (\freenames{P} \setminus \{ y \}) \\
  \freenames{x!\langle P \rangle} & := & \{ x \} \cup \{ P \} \\
  \freenames{P|Q} & := & \freenames{P} \cup \freenames{Q} \\
  \freenames{\dropn{x}} & := & \{ x \}
\end{eqnarray*}

The bound names of a process, $\boundnames{P}$, are those names occurring in $P$
that are not free. For example, in $x?(y).0$, the name $x$ is free, while $y$ is bound.

\begin{mathpar}
  \inferrule* [lab=monoidal-laws] {} { P|Q \equiv Q|P \and P|0 \equiv P \and P|(Q|R) \equiv (P|Q)|R }
\end{mathpar}

\begin{mathpar}
  \inferrule* [lab=alpha-equivalence] {} { (x)P \equiv (y)P\{y/x\} \and y \not\in \freenames{P} }
\end{mathpar}

\begin{definition}
Then two processes, $P,Q$, are alpha-equivalent if $P = Q\{\vec{y}/\vec{x}\}$ for
some $\vec{x} \in \boundnames{Q},\vec{y} \in \boundnames{P}$, where $Q\{\vec{y}/\vec{x}\}$
denotes the capture-avoiding substitution of $\vec{y}$ for $\vec{x}$ in $Q$.
\end{definition}

\begin{definition}
  The {\em structural congruence} \cite{SangiorgiWalker} , $\equiv$,
  between processes is the least congruence containing
  alpha-equivalence, satisfying the abelian monoid laws
  (associativity, commutativity and $\pzero$ as identity) for parallel
  composition $|$ and for summation $+$.
\end{definition}

\subsection{Name equivalence}

We take name equivalence, written $\nameeq$, to be the smallest
equivalence relation generated by the following rules.

\begin{mathpar}
\inferrule*[lab=Quote-drop]
{ }
{ \quotep{@{x}} \nameeq x }

\inferrule*[lab=Struct-equiv]
{ P \scong Q }
{ \quotep{P} \nameeq \quotep{Q} }
\end{mathpar}

The astute reader will have noticed that the mutual recursion of names
and processes imposes a mutual recursion on alpha-equivalence and
structural equivalence via name-equivalence. Fortunately, all of this
works out pleasantly and we may calculate in the natural way, free of
concern. The reader interested in the details is referred to the
appendix \ref{appendix:rho_details}.

\subsection{Substitution}

We use $\Proc$ for the set of processes, $\QProc$ for the set of
names, and $\id{\{}\vec{y} / \vec{x} \id{\}}$ to denote partial maps,
$s : \QProc \rightarrow \QProc$. A map, $s$ lifts, uniquely, to a map
on process terms, $\widehat{s} : \Proc \rightarrow \Proc$ by the
following equations.

\begin{mathpar}
  (0) \psubstp{Q}{P} := 0 \\
  (R \juxtap S) \psubstp{Q}{P}
  :=    
  (R)\psubstp{Q}{P} \juxtap (S) \psubstp{Q}{P} \\
  (x?(y).R) \psubstp{Q}{P}    
  :=    
  (x)\substp{Q}{P} (z)\concat( (R \psubstn{z}{y}) \psubstp{Q}{P} ) \\
  (\lift{x}{R}) \psubstp{Q}{P}  
  :=
  \lift{(x)\substp{Q}{P}}{ R \psubstp{Q}{P} } \\
%   (\dropn{x})  \psubstp{Q}{P}       
%   := 
%   \left\{ 
%     \begin{array}{ccc} 
%       \dropn{\quotep{Q}} & & x \nameeq \quotep{P} \\
%       \dropn{x} & & otherwise \\
%     \end{array}
%   \right. 
  (\dropn{x})  \psubstp{Q}{P}       
  := 
  \left\{ 
    \begin{array}{ccc} 
      Q & & x \nameeq \quotep{P} \\
      \dropn{x} & & otherwise \\
    \end{array}
  \right.
\end{mathpar}
 

where

\begin{eqnarray}
  (x)\id{\{} \lpquote Q \rpquote / \lpquote P \rpquote \id{\}}            = 
  \left\{ 
    \begin{array}{ccc}
      \lpquote Q \rpquote & & x \nameeq \lpquote P \rpquote \\
      x & & otherwise \\
    \end{array}
  \right. \nonumber
\end{eqnarray}

and $z$ is chosen distinct from $\quotep{P}$, $\quotep{Q}$, the free
names in $Q$, and all the names in $R$. Our $\alpha$-equivalence will
be built in the standard way from this substitution.

\begin{remark}\label{rem:no_self_referential_names}
  One consequence of these definitions is that $\forall P. \quotep{P}
  \not\in \freenames{P}$.
\end{remark}

\subsection{ Dynamic quote: an example }

Anticipating something of what's to come, consider applying the
substitution, $\widehat{\id{\{}u / z \id{\}}}$, to the following pair
of processes, $\lift{w}{y!(z)}$ and $w[ \lpquote y!(z) \rpquote ]$.

\begin{eqnarray}
	\lift{w}{y!(z)}\widehat{\id{\{}u / z \id{\}}}
		& = &
		\lift{w}{y!(u)} \nonumber\\
	w[ \lpquote y!(z) \rpquote ] \widehat{ \id{\{}u / z \id{\}} }
		& = &
		w[ \lpquote y!(z) \rpquote ] \nonumber
\end{eqnarray}

Because the body of the process between quotes is impervious to
substitution, we get radically different answers. In fact, by
examining the first process in an input context,
e.g. $x?(z).\lift{w}{y!(z)}$, we see that the process under the lift
operator may be shaped by prefixed inputs binding a name inside it. In
this sense, the lift operator will be seen as a way to dynamically
construct processes before reifying them as names.

Finally equipped with these standard features we can present the
dynamics of the calculus.

\subsubsection{Operational semantics} 

Finally, we introduce the computational dynamics. What marks these
algebras as distinct from other more traditionally studied algebraic
structures, e.g. vector spaces or polynomial rings, is the manner in
which dynamics is captured. In traditional structures, dynamics is typically
expressed through morphisms between such structures, as in linear maps
between vector spaces or morphisms between rings. In algebras
associated with the semantics of computation, the dynamics is
expressed as part of the algebraic structure itself, through a
reduction reduction relation typically denoted by $\red$. Below, we
give a recursive presentation of this relation for the calculus used
in the encoding.

$\red \subseteq \pi \times \pi$
$\red : \pi \to \mathcal{P}(\pi)$

\begin{mathpar}
  \inferrule* [lab=Comm] { \textsf{match}( x_{src}, x_{trgt} ) } { x_{trgt}?(y)P \; | \; x_{src}!\langle {Q} \rangle \red P\{\quotep{Q}/y}\} }
  \and \\
  \inferrule* [lab=Par] {{P} \red {P}'} {{{P} | {Q}} \red {{P}' | {Q}}}
  \and
  \inferrule* [lab=Equiv]{{{P} \scong {P}'} \andalso {{P}' \red {Q}'} \andalso {{Q}' \scong {Q}}}{{P} \red {Q}}
\end{mathpar}

\begin{eqnarray*}
  match_{\equiv} (\quotep{P},\quotep{Q}) & := & P \equiv Q \\
  match_{\dagger}(\quotep{P},\quotep{Q}) & := & \forall R. P|Q \red^{*} R => R \red^{*} 0 \\
  match_{K}(\quotep{P},\quotep{Q}) & := & K \mbox{ for some context } K
\end{eqnarray*}

$u?(x)P | u!\langle Q \rangle \red P\{\quotep{Q}/x\}$

%We write $\wred$ for $\red^*$, and $P\red$ if $\exists Q $ such that $ P \red Q$.
We write $P\red$ if $\exists Q $ such that $ P \red Q$ and $P\not\red$, otherwise.

\section{Replication}

As mentioned before, it is known that replication (and hence
recursion) can be implemented in a higher-order process algebra
\cite{SangiorgiWalker}. As our first example of calculation with the
machinery thus far presented we give the construction explicitly in
the {\rhoc}.

\begin{eqnarray}
	D_{x} & := & \prefix{x}{y}{(\binpar{\outputp{x}{y}}{@{y}})} \nonumber\\
	\bangp_{x}{P} & := & \binpar{{x}!\langle{\binpar{D_{x}}{P}}\rangle}{D_{x}} \nonumber
\end{eqnarray}

\begin{eqnarray}
	\bangp_{x}{P} & & \nonumber\\
	=
	& {x}!\langle{(\prefix{x}{y}{(\outputp{x}{y} | @{y})) | P}}\rangle 
	      | \prefix{x}{y}{(\outputp{x}{y} | @{y})} & \nonumber\\
	\red
	& (\outputp{x}{y} | @{y})\substn{\quotep{(\prefix{x}{y}{(@{y} | \outputp{x}{y})) | P}}}{y} & \nonumber\\
	=
	& \outputp{x}{\quotep{(\prefix{x}{y}{(\outputp{x}{y} | @{y})) | P}}}
	  | {(\prefix{x}{y}{(\outputp{x}{y} | @{y})) | P}} & \nonumber\\
	\red
	& \ldots & \nonumber\\
	\red^*
	& P | P | \ldots & \nonumber
\end{eqnarray}

Of course, this encoding, as an implementation, runs away, unfolding
$\bangp{P}$ eagerly. A lazier and more implementable replication
operator, restricted to input-guarded processes, may be obtained as follows.

\begin{eqnarray}
\bangp{\prefix{u}{v}{P}} 
	:= 
	\binpar{\lift{x}{\prefix{u}{v}{(\binpar{D(x)}{P})}}}{D(x)} \nonumber
\end{eqnarray}

\begin{remark}
  Note that the lazier definition still does not deal with summation
  or mixed summation (i.e. sums over input and output). The reader is
  invited to construct definitions of replication that deal with these
  features. 

  Further, the definitions are parameterized in a name, $x$. Can you,
  gentle reader, make a definition that eliminates this parameter and
  guarantees no accidental interaction between the replication
  machinery and the process being replicated -- i.e. no accidental
  sharing of names used by the process to get its work done and the
  name(s) used by the replication to effect copying. This latter
  revision of the definition of replication is crucial to obtaining
  the expected identity $!!P \sim !P$.
\end{remark}

\begin{remark}\label{rem:paradoxical_combinator}
  The reader familiar with the lambda calculus will have noticed the
  similarity between $D$ and the paradoxical combinator.

  [Ed. note: the existence of this seems to suggest we have to be more
  restrictive on the set of processes and names we admit if we are to
  support no-cloning.]
\end{remark}

\subsubsection{Bisimulation}

The computational dynamics gives rise to another kind of equivalence,
the equivalence of computational behavior. As previously mentioned
this is typically captured \emph{via} some form of bisimulation.

% The notion we use in this paper is weak barbed bisimulation
% \cite{milner91polyadicpi}.

The notion we use in this paper is derived from weak barbed
bisimulation \cite{milner91polyadicpi}. 

\begin{definition}
An \emph{observation relation}, $\downarrow_{\mathcal N}$, over a set
of names, $\mathcal N$, is the smallest relation satisfying the rules
below.

\infrule[Out-barb]{y \in {\mathcal N}, \; x \nameeq y}
		  {\outputp{x}{v} \downarrow_{\mathcal N} x}
\infrule[Par-barb]{\mbox{$P\downarrow_{\mathcal N} x$ or $Q\downarrow_{\mathcal N} x$}}
		  {\binpar{P}{Q} \downarrow_{\mathcal N} x}

We write $P \Downarrow_{\mathcal N} x$ if there is $Q$ such that 
$P \wred Q$ and $Q \downarrow_{\mathcal N} x$.
\end{definition}

\begin{definition}
%\label{def.bbisim}
An  ${\mathcal N}$-\emph{barbed bisimulation} over a set of names, ${\mathcal N}$, is a symmetric binary relation 
${\mathcal S}_{\mathcal N}$ between agents such that $P\rel{S}_{\mathcal N}Q$ implies:
\begin{enumerate}
\item If $P \red P'$ then $Q \wred Q'$ and $P'\rel{S}_{\mathcal N} Q'$.
\item If $P\downarrow_{\mathcal N} x$, then $Q\Downarrow_{\mathcal N} x$.
\end{enumerate}
$P$ is ${\mathcal N}$-barbed bisimilar to $Q$, written
$P \wbbisim_{\mathcal N} Q$, if $P \rel{S}_{\mathcal N} Q$ for some ${\mathcal N}$-barbed bisimulation ${\mathcal S}_{\mathcal N}$.
\end{definition}

$\mathcal{R} \subseteq \pi \times \pi$

$P \mathcal{R} Q => \forall P'. P \red P' \Rightarrow \exists Q'. Q \red Q', P' \mathcal{R} Q'$

$P \vdash x \Rightarrow Q \vdash x$

\begin{mathpar}
  \inferrule*[lab=Out-barb]{x \nameeq y}{{y}!\langle{Q}\rangle \vdash x}
  \and
  \inferrule*[lab=Par-barb]{\mbox{$P\vdash x$ or $Q\vdash x$}}{\binpar{P}{Q} \vdash x}
\end{mathpar}

\subsubsection{Contexts}

One of the principle advantages of computational calculi like the
$\pi$-calculus is a well-defined notion of context,
contextual-equivalence and a correlation between
contextual-equivalence and notions of bisimulation. The notion of
context allows the decomposition of a process into (sub-)process and
its syntactic environment, its context. Thus, a context may be
thought of as a process with a ``hole'' (written $\Box$) in it. The
application of a context $M$ to a process $P$, written $M[P]$, is
tantamount to filling the hole in $M$ with $P$. In this paper we do
not need the full weight of this theory, but do make use of the notion
of context in the proof the main theorem. 

\begin{mathpar}
  \inferrule* [lab=summation] {} {{M_{M},M_{N}} \bc \Box \;|\; x.M_{A} \;|\; M_{M}+M_{N}}
  \and
  \inferrule* [lab=agent] {} {{M_{A}} \bc (\vec{x})M_{P} \;| \; \clift{P_0,\ldots,M_{P},\ldots,P_N}}
  \and \\
  \inferrule* [lab=process] {} {{M_{P}} \bc M_{N} \;| \;P|M_{P} }
\end{mathpar} 

\begin{mathpar}
  \inferrule* [lab=sychronization] {} {M_{N} \bc \Box \;|\; x?M_{F} \;|\; x!M_{C}}
  \and
  \inferrule* [lab=abstraction] {} {{M_{F}} \bc (x)M_{P} }
  \and
  \inferrule* [lab=concretion] {} {{M_{C}} \bc \langle M_{P} \rangle }
  \and \\
  \inferrule* [lab=process] {} {{M_{P}} \bc M_{N} \;| \;P|M_{P} }
\end{mathpar}

\begin{definition}[contextual application] Given a context $M$, and
  process $P$, we define the \emph{contextual application}, $M[P] :=
  M\{P/\Box\}$. That is, the contextual application of M to P is the
  substitution of $P$ for $\Box$ in $M$.
\end{definition}

$\meaningof{-} : L \to \mathcal{P}(\pi)$

\begin{mathpar}
  \inferrule* [lab=collection] {} {\meaningof{true} = \pi, \and \meaningof{~E} = \pi \setminus \meaningof{E}, \and \meaningof{E_{1} \& E_{2}} = \meaningof{E_{1}} \cap \meaningof{E_{2}}}
\end{mathpar}

\begin{mathpar}
  \inferrule* [lab=structure] {} {\meaningof{0} = \{ P \in \pi | P \equiv 0 \}, \and \\ \meaningof{E_1 | E_2} = \{ P \in \pi | P \equiv P_{1} | P_{2}, P_{1} \in \meaningof{E_{1}}, P_{2} \in \meaningof{E_2}\} }
\end{mathpar}

\begin{mathpar}
 \inferrule* [lab=behavior] {} {\meaningof{\langle a?b \rangle E} = \{ P \in \pi | P \equiv Q | u?(y)P', \\ \and \\\\ \and \\ \;\;\; u \in \meaningof{a}, \forall z.P'\{z/y\} \in \meaningof{E\{z/b\}}\}, \and \\ \meaningof{a!E} = \{ P \in \pi | P \equiv Q | x!\langle P' \rangle, x \in \meaningof{a} P' \in \meaningof{E}\} }
\end{mathpar}

\begin{mathpar}
 \inferrule* [lab=nominal] {} {\meaningof{\quotep{E}} = \{ \quotep{P} \in \quotep{\pi} | P \in \meaningof{E} \}, \and \meaningof{\quotep{P}} = \{ \quotep{Q} \in \quotep{\pi} | P \equiv Q \} \and \\ \meaningof{@\quotep{E}} = \{ P \in \pi | P \equiv @x, x \in \meaningof{E} \}}
\end{mathpar}

\begin{eqnarray*}
  \\
  \meaningof{-} : TS \to ST
\end{eqnarray*}

\begin{eqnarray*}
  \\
  L : TS \to ST
\end{eqnarray*}

\begin{eqnarray*}
  \\
  P \models E \iff P \in \meaningof{E}
\end{eqnarray*}

\begin{eqnarray*}
  P \approx_{L} Q \iff \forall E \in L. P \models E \iff Q \models E
\end{eqnarray*}

\begin{eqnarray*}
  P \approx_{K} Q
\end{eqnarray*}

\begin{eqnarray*}
  P \approx Q
\end{eqnarray*}

$\approx_{K} = \approx = \approx_{L}$

\subsubsection{Contextual duality}

Note that contexts extend the quotation operation to a family of
operations from processes to names. Given a context, $M$, we can
define a \emph{nominal context}, $\quotep{M}$ by $\quotep{M}[P] :=
\quotep{M[P]}$. To foreshadow what is to come we observe that these
operations enjoy a duality with processes very much like the duality
between vectors and maps from vectors to scalars.

Further, because the calculus is essentially higher-order, we have a
correspondence between contexts and processes. More specifically,
given a name $x$ and a context $M$ we can construct $M^{*}_{x}$ such
that 

\begin{mathpar}
  M^{*}_{x} | \lift{x}{P} \red M[P]
\end{mathpar}

namely,

\begin{mathpar}
  M^{*}_{x} := x?(u).M[\dropn{u}]
\end{mathpar}

The dependence of $M^{*}_{x}$ on a name makes it an abstraction, 

\begin{mathpar}
  M^{*} := (x)x?(u).M[\dropn{u}]
\end{mathpar}

\subsection{Additional notation}

It will sometimes be convenient to denote the process a name
quotes. We already have the notation $x = \quotep{P}$, but it will be
convenient to introduce an alternate notation, $\procn{x}$, when we
want to emphasize the connection to the use of the name. Note that, by
virtue of name equivalence, $\quotep{\procn{x}} \nameeq x$; so, the
notation is consistent with previous definitions.

Further, because names have structure it is possible to effect
substitutions on the basis of that structure. This means we need to
upgrade our notation for substitutions, which we accomplish by
adapting comprehension notation. Thus,

\begin{mathpar}
  P\{ y / x : x \in S \}
\end{mathpar}

is interpreted to mean the process derived from P by replacing (in a
capture-avoiding manner) each occurrence of $x$ in $S$ by $y$. For example,

\begin{mathpar}
  P\{ \quotep{\procn{x}|\procn{x}} / x : x \in \freenames{P} \}
\end{mathpar}

will replace each (occurrence) of a free name $x$ in $P$ by
$\quotep{\procn{x}|\procn{x}}$.

Also, we will avail ourselves of the notation $x^{L}$ and $x^{R}$ to
denote injections of a name into disjoint copies of the name
space. There are numerous ways to accomplish this. One example can be
found in \cite{MeredithR05}. This notation overloads to vectors of
names: $\vec{x}^{\pi} := (x_{i}^{\pi} \; : \; 0 \leq i < |\vec{x}| )$ where $\pi \in \{L,R\}$.

We also use $P^{\Box} := P|\Box$.

In \cite{MeredithR05} an interpretation of the new operator is
given. It turns out that there are several possible interpretations
all enjoying the requisite algebraic properties of the operator (see
\cite{milner91polyadicpi}). We will therefore make liberal use of
$(\nu\; \vec{x})P$.

% subsection the_syntax_and_semantics_of_the_notation_system (end)   

\input{qm2pi.qmops} 

\input{qm2pi.sterngerlach} 

\input{qm2pi.metric} 

% section concurrent_process_calculi (end)

%\input{qm2pi.proofsketch}

% section proof sketch (end)

%\input{qm2pi.slviaknots} 

% section spatial logic via knots (end)

\input{qm2pi.conclusion}

% section conclusion (end)

%\input{qm2pi.dtcodes} 

% section wiring algorithm (end)

\input{qm2pi.ack} 

% section acknowledgments (end)

\newpage


\bibliographystyle{plain}   
\bibliography{../../biblios/main.bib}

\input{qm2pi.rhodetails}

\end{document}

 

% section wiring algorithm (end)

\documentclass[12pt]{llncs}
%\documentclass{jktr}

\usepackage[pdftex]{hyperref}                   
\usepackage {listings}
\usepackage {mathpartir}
\usepackage{bcprules}
%\usepackage{listings}
                       
\usepackage{graphicx} 
%\usepackage[margins=2.5cm,nohead,nofoot]{geometry}
%\usepackage{geometry}
\usepackage{amsfonts}
\usepackage{amstext}
\usepackage{latexsym}
\usepackage{amssymb}
\usepackage{color}


%\include{myPreamble}
\include{qm2pi.local} 

%\ifpdf
%\usepackage[pdftex]{graphicx}
%\else
%\usepackage{graphicx}
%\fi

 % \ifpdf
%  \usepackage{pdfsync}
%  \if


%\title{Brief Article}
%\author{David F. Snyder}
%\author{L.G. Meredith}

%\address{Dept. of Math., Texas State University--San Marcos, San Marcos, TX 78666}
       
\pagestyle{empty}


\begin{document}

\lstset{language=[Objective]Caml,frame=shadowbox}

\input{qm2pi.front}

% section front matter (end)

\input{qm2pi.intro} 
 
% section introduction (end)

% \input{qm2pi.knotations} 

% section notation (end)

\input{qm2pi.process.calculi} 

% section concurrent_process_calculi_and_spatial_logics_ (end)
    
%\input{qm2pi.knots2pi} 

%\input{qm2pi.trefoil} 

%\input{qm2pi.mainthm} 

% subsection basic_interpretation (end)

%\input{qm2pi.rho.presentation} 
\subsection{The syntax and semantics of the notation system}\label{sub:the_syntax_and_semantics_of_the_notation_system} % (fold)

We now summarize a technical presentation of the calculus that
embodies our theory of dynamics. The typical presentation of such a
calculus follows the style of giving generators and relations on
them. The grammar, below, describing term constructors, freely
generates the set of processes, $\Proc$. This set is then quotiented
by a relation known as structural congruence and it is over this set
that the notion of dynamics is expressed. This presentation is
essentially that of \cite{MeredithR05} with the addition of
polyadicity and summation. For readability we have relegated some of
the technical subtleties to an appendix.

\subsubsection{Process grammar}\label{subsub:process_grammar}

\begin{mathpar}
  \inferrule* [lab=synchronization] {} {{M} \bc \pzero \;|\; x?F \;|\; x!C }
  \and
  \inferrule* [lab=abstraction] {} {{F} \bc (x)P}
  \and
  \inferrule* [lab=concretion] {} {{C} \bc \langle Q \rangle}
  \and
  \inferrule* [lab=process] {} {{P,Q} \bc M \;| \;P|Q \;|\; @{x}}
  \and
  \inferrule* [lab=name] {} {{x} \bc \quotep{P}}
\end{mathpar} 

Note that $\vec{x}$ (resp. $\vec{P}$) denotes a vector of names
(resp. processes) of length $|\vec{x}|$ (resp. $|\vec{P}|$). We adopt
the following useful abbreviations.

\begin{mathpar}
   x?(\vec{y}).P := x.(\vec{y})P \and  x\clift{\vec{P}} := x.\clift{\vec{P}}
   \and x!(y) := \lift{x}{\dropn{y}}
   \and \Pi_{i=0}^{n-1}P_i := P_0 | \ldots | P_{n-1}
\end{mathpar}

\subsubsection{Structural congruence}

\paragraph{Free and bound names and alpha-equivalence.} At the
core of structural equivalence is alpha-equivalence which identifies
process that are the same up to a change of variable. Formally, we
recognize the distinction between free and bound names. The free names
of a process, $\freenames{P}$, may be calculated recursively as
follows:

\begin{mathpar}
\freenames{\pzero} := \emptyset
  \and \\
  \freenames{x?(y).P} := \{ x \} \cup (\freenames{P} \setminus \{ y \})
  \and 
  \freenames{x!\langle P \rangle} := \{ x \} \cup \{ P \} 
  \and \\
  \freenames{P|Q} := \freenames{P} \cup \freenames{Q}
  \and \\
  \freenames{@{x}} := \{ x \}
\end{mathpar}

$\pi$
$\quotep{\pi}$

$\freenames{-} : \pi \to \mathcal{P}(\quotep{\pi})$

\begin{eqnarray*}
  \freenames{\pzero} & := & \emptyset \\
  \freenames{x?(y).P} & := & \{ x \} \cup (\freenames{P} \setminus \{ y \}) \\
  \freenames{x!\langle P \rangle} & := & \{ x \} \cup \{ P \} \\
  \freenames{P|Q} & := & \freenames{P} \cup \freenames{Q} \\
  \freenames{\dropn{x}} & := & \{ x \}
\end{eqnarray*}

The bound names of a process, $\boundnames{P}$, are those names occurring in $P$
that are not free. For example, in $x?(y).0$, the name $x$ is free, while $y$ is bound.

\begin{mathpar}
  \inferrule* [lab=monoidal-laws] {} { P|Q \equiv Q|P \and P|0 \equiv P \and P|(Q|R) \equiv (P|Q)|R }
\end{mathpar}

\begin{mathpar}
  \inferrule* [lab=alpha-equivalence] {} { (x)P \equiv (y)P\{y/x\} \and y \not\in \freenames{P} }
\end{mathpar}

\begin{definition}
Then two processes, $P,Q$, are alpha-equivalent if $P = Q\{\vec{y}/\vec{x}\}$ for
some $\vec{x} \in \boundnames{Q},\vec{y} \in \boundnames{P}$, where $Q\{\vec{y}/\vec{x}\}$
denotes the capture-avoiding substitution of $\vec{y}$ for $\vec{x}$ in $Q$.
\end{definition}

\begin{definition}
  The {\em structural congruence} \cite{SangiorgiWalker} , $\equiv$,
  between processes is the least congruence containing
  alpha-equivalence, satisfying the abelian monoid laws
  (associativity, commutativity and $\pzero$ as identity) for parallel
  composition $|$ and for summation $+$.
\end{definition}

\subsection{Name equivalence}

We take name equivalence, written $\nameeq$, to be the smallest
equivalence relation generated by the following rules.

\begin{mathpar}
\inferrule*[lab=Quote-drop]
{ }
{ \quotep{@{x}} \nameeq x }

\inferrule*[lab=Struct-equiv]
{ P \scong Q }
{ \quotep{P} \nameeq \quotep{Q} }
\end{mathpar}

The astute reader will have noticed that the mutual recursion of names
and processes imposes a mutual recursion on alpha-equivalence and
structural equivalence via name-equivalence. Fortunately, all of this
works out pleasantly and we may calculate in the natural way, free of
concern. The reader interested in the details is referred to the
appendix \ref{appendix:rho_details}.

\subsection{Substitution}

We use $\Proc$ for the set of processes, $\QProc$ for the set of
names, and $\id{\{}\vec{y} / \vec{x} \id{\}}$ to denote partial maps,
$s : \QProc \rightarrow \QProc$. A map, $s$ lifts, uniquely, to a map
on process terms, $\widehat{s} : \Proc \rightarrow \Proc$ by the
following equations.

\begin{mathpar}
  (0) \psubstp{Q}{P} := 0 \\
  (R \juxtap S) \psubstp{Q}{P}
  :=    
  (R)\psubstp{Q}{P} \juxtap (S) \psubstp{Q}{P} \\
  (x?(y).R) \psubstp{Q}{P}    
  :=    
  (x)\substp{Q}{P} (z)\concat( (R \psubstn{z}{y}) \psubstp{Q}{P} ) \\
  (\lift{x}{R}) \psubstp{Q}{P}  
  :=
  \lift{(x)\substp{Q}{P}}{ R \psubstp{Q}{P} } \\
%   (\dropn{x})  \psubstp{Q}{P}       
%   := 
%   \left\{ 
%     \begin{array}{ccc} 
%       \dropn{\quotep{Q}} & & x \nameeq \quotep{P} \\
%       \dropn{x} & & otherwise \\
%     \end{array}
%   \right. 
  (\dropn{x})  \psubstp{Q}{P}       
  := 
  \left\{ 
    \begin{array}{ccc} 
      Q & & x \nameeq \quotep{P} \\
      \dropn{x} & & otherwise \\
    \end{array}
  \right.
\end{mathpar}
 

where

\begin{eqnarray}
  (x)\id{\{} \lpquote Q \rpquote / \lpquote P \rpquote \id{\}}            = 
  \left\{ 
    \begin{array}{ccc}
      \lpquote Q \rpquote & & x \nameeq \lpquote P \rpquote \\
      x & & otherwise \\
    \end{array}
  \right. \nonumber
\end{eqnarray}

and $z$ is chosen distinct from $\quotep{P}$, $\quotep{Q}$, the free
names in $Q$, and all the names in $R$. Our $\alpha$-equivalence will
be built in the standard way from this substitution.

\begin{remark}\label{rem:no_self_referential_names}
  One consequence of these definitions is that $\forall P. \quotep{P}
  \not\in \freenames{P}$.
\end{remark}

\subsection{ Dynamic quote: an example }

Anticipating something of what's to come, consider applying the
substitution, $\widehat{\id{\{}u / z \id{\}}}$, to the following pair
of processes, $\lift{w}{y!(z)}$ and $w[ \lpquote y!(z) \rpquote ]$.

\begin{eqnarray}
	\lift{w}{y!(z)}\widehat{\id{\{}u / z \id{\}}}
		& = &
		\lift{w}{y!(u)} \nonumber\\
	w[ \lpquote y!(z) \rpquote ] \widehat{ \id{\{}u / z \id{\}} }
		& = &
		w[ \lpquote y!(z) \rpquote ] \nonumber
\end{eqnarray}

Because the body of the process between quotes is impervious to
substitution, we get radically different answers. In fact, by
examining the first process in an input context,
e.g. $x?(z).\lift{w}{y!(z)}$, we see that the process under the lift
operator may be shaped by prefixed inputs binding a name inside it. In
this sense, the lift operator will be seen as a way to dynamically
construct processes before reifying them as names.

Finally equipped with these standard features we can present the
dynamics of the calculus.

\subsubsection{Operational semantics} 

Finally, we introduce the computational dynamics. What marks these
algebras as distinct from other more traditionally studied algebraic
structures, e.g. vector spaces or polynomial rings, is the manner in
which dynamics is captured. In traditional structures, dynamics is typically
expressed through morphisms between such structures, as in linear maps
between vector spaces or morphisms between rings. In algebras
associated with the semantics of computation, the dynamics is
expressed as part of the algebraic structure itself, through a
reduction reduction relation typically denoted by $\red$. Below, we
give a recursive presentation of this relation for the calculus used
in the encoding.

$\red \subseteq \pi \times \pi$
$\red : \pi \to \mathcal{P}(\pi)$

\begin{mathpar}
  \inferrule* [lab=Comm] { \textsf{match}( x_{src}, x_{trgt} ) } { x_{trgt}?(y)P \; | \; x_{src}!\langle {Q} \rangle \red P\{\quotep{Q}/y}\} }
  \and \\
  \inferrule* [lab=Par] {{P} \red {P}'} {{{P} | {Q}} \red {{P}' | {Q}}}
  \and
  \inferrule* [lab=Equiv]{{{P} \scong {P}'} \andalso {{P}' \red {Q}'} \andalso {{Q}' \scong {Q}}}{{P} \red {Q}}
\end{mathpar}

\begin{eqnarray*}
  match_{\equiv} (\quotep{P},\quotep{Q}) & := & P \equiv Q \\
  match_{\dagger}(\quotep{P},\quotep{Q}) & := & \forall R. P|Q \red^{*} R => R \red^{*} 0 \\
  match_{K}(\quotep{P},\quotep{Q}) & := & K \mbox{ for some context } K
\end{eqnarray*}

$u?(x)P | u!\langle Q \rangle \red P\{\quotep{Q}/x\}$

%We write $\wred$ for $\red^*$, and $P\red$ if $\exists Q $ such that $ P \red Q$.
We write $P\red$ if $\exists Q $ such that $ P \red Q$ and $P\not\red$, otherwise.

\section{Replication}

As mentioned before, it is known that replication (and hence
recursion) can be implemented in a higher-order process algebra
\cite{SangiorgiWalker}. As our first example of calculation with the
machinery thus far presented we give the construction explicitly in
the {\rhoc}.

\begin{eqnarray}
	D_{x} & := & \prefix{x}{y}{(\binpar{\outputp{x}{y}}{@{y}})} \nonumber\\
	\bangp_{x}{P} & := & \binpar{{x}!\langle{\binpar{D_{x}}{P}}\rangle}{D_{x}} \nonumber
\end{eqnarray}

\begin{eqnarray}
	\bangp_{x}{P} & & \nonumber\\
	=
	& {x}!\langle{(\prefix{x}{y}{(\outputp{x}{y} | @{y})) | P}}\rangle 
	      | \prefix{x}{y}{(\outputp{x}{y} | @{y})} & \nonumber\\
	\red
	& (\outputp{x}{y} | @{y})\substn{\quotep{(\prefix{x}{y}{(@{y} | \outputp{x}{y})) | P}}}{y} & \nonumber\\
	=
	& \outputp{x}{\quotep{(\prefix{x}{y}{(\outputp{x}{y} | @{y})) | P}}}
	  | {(\prefix{x}{y}{(\outputp{x}{y} | @{y})) | P}} & \nonumber\\
	\red
	& \ldots & \nonumber\\
	\red^*
	& P | P | \ldots & \nonumber
\end{eqnarray}

Of course, this encoding, as an implementation, runs away, unfolding
$\bangp{P}$ eagerly. A lazier and more implementable replication
operator, restricted to input-guarded processes, may be obtained as follows.

\begin{eqnarray}
\bangp{\prefix{u}{v}{P}} 
	:= 
	\binpar{\lift{x}{\prefix{u}{v}{(\binpar{D(x)}{P})}}}{D(x)} \nonumber
\end{eqnarray}

\begin{remark}
  Note that the lazier definition still does not deal with summation
  or mixed summation (i.e. sums over input and output). The reader is
  invited to construct definitions of replication that deal with these
  features. 

  Further, the definitions are parameterized in a name, $x$. Can you,
  gentle reader, make a definition that eliminates this parameter and
  guarantees no accidental interaction between the replication
  machinery and the process being replicated -- i.e. no accidental
  sharing of names used by the process to get its work done and the
  name(s) used by the replication to effect copying. This latter
  revision of the definition of replication is crucial to obtaining
  the expected identity $!!P \sim !P$.
\end{remark}

\begin{remark}\label{rem:paradoxical_combinator}
  The reader familiar with the lambda calculus will have noticed the
  similarity between $D$ and the paradoxical combinator.

  [Ed. note: the existence of this seems to suggest we have to be more
  restrictive on the set of processes and names we admit if we are to
  support no-cloning.]
\end{remark}

\subsubsection{Bisimulation}

The computational dynamics gives rise to another kind of equivalence,
the equivalence of computational behavior. As previously mentioned
this is typically captured \emph{via} some form of bisimulation.

% The notion we use in this paper is weak barbed bisimulation
% \cite{milner91polyadicpi}.

The notion we use in this paper is derived from weak barbed
bisimulation \cite{milner91polyadicpi}. 

\begin{definition}
An \emph{observation relation}, $\downarrow_{\mathcal N}$, over a set
of names, $\mathcal N$, is the smallest relation satisfying the rules
below.

\infrule[Out-barb]{y \in {\mathcal N}, \; x \nameeq y}
		  {\outputp{x}{v} \downarrow_{\mathcal N} x}
\infrule[Par-barb]{\mbox{$P\downarrow_{\mathcal N} x$ or $Q\downarrow_{\mathcal N} x$}}
		  {\binpar{P}{Q} \downarrow_{\mathcal N} x}

We write $P \Downarrow_{\mathcal N} x$ if there is $Q$ such that 
$P \wred Q$ and $Q \downarrow_{\mathcal N} x$.
\end{definition}

\begin{definition}
%\label{def.bbisim}
An  ${\mathcal N}$-\emph{barbed bisimulation} over a set of names, ${\mathcal N}$, is a symmetric binary relation 
${\mathcal S}_{\mathcal N}$ between agents such that $P\rel{S}_{\mathcal N}Q$ implies:
\begin{enumerate}
\item If $P \red P'$ then $Q \wred Q'$ and $P'\rel{S}_{\mathcal N} Q'$.
\item If $P\downarrow_{\mathcal N} x$, then $Q\Downarrow_{\mathcal N} x$.
\end{enumerate}
$P$ is ${\mathcal N}$-barbed bisimilar to $Q$, written
$P \wbbisim_{\mathcal N} Q$, if $P \rel{S}_{\mathcal N} Q$ for some ${\mathcal N}$-barbed bisimulation ${\mathcal S}_{\mathcal N}$.
\end{definition}

$\mathcal{R} \subseteq \pi \times \pi$

$P \mathcal{R} Q => \forall P'. P \red P' \Rightarrow \exists Q'. Q \red Q', P' \mathcal{R} Q'$

$P \vdash x \Rightarrow Q \vdash x$

\begin{mathpar}
  \inferrule*[lab=Out-barb]{x \nameeq y}{{y}!\langle{Q}\rangle \vdash x}
  \and
  \inferrule*[lab=Par-barb]{\mbox{$P\vdash x$ or $Q\vdash x$}}{\binpar{P}{Q} \vdash x}
\end{mathpar}

\subsubsection{Contexts}

One of the principle advantages of computational calculi like the
$\pi$-calculus is a well-defined notion of context,
contextual-equivalence and a correlation between
contextual-equivalence and notions of bisimulation. The notion of
context allows the decomposition of a process into (sub-)process and
its syntactic environment, its context. Thus, a context may be
thought of as a process with a ``hole'' (written $\Box$) in it. The
application of a context $M$ to a process $P$, written $M[P]$, is
tantamount to filling the hole in $M$ with $P$. In this paper we do
not need the full weight of this theory, but do make use of the notion
of context in the proof the main theorem. 

\begin{mathpar}
  \inferrule* [lab=summation] {} {{M_{M},M_{N}} \bc \Box \;|\; x.M_{A} \;|\; M_{M}+M_{N}}
  \and
  \inferrule* [lab=agent] {} {{M_{A}} \bc (\vec{x})M_{P} \;| \; \clift{P_0,\ldots,M_{P},\ldots,P_N}}
  \and \\
  \inferrule* [lab=process] {} {{M_{P}} \bc M_{N} \;| \;P|M_{P} }
\end{mathpar} 

\begin{mathpar}
  \inferrule* [lab=sychronization] {} {M_{N} \bc \Box \;|\; x?M_{F} \;|\; x!M_{C}}
  \and
  \inferrule* [lab=abstraction] {} {{M_{F}} \bc (x)M_{P} }
  \and
  \inferrule* [lab=concretion] {} {{M_{C}} \bc \langle M_{P} \rangle }
  \and \\
  \inferrule* [lab=process] {} {{M_{P}} \bc M_{N} \;| \;P|M_{P} }
\end{mathpar}

\begin{definition}[contextual application] Given a context $M$, and
  process $P$, we define the \emph{contextual application}, $M[P] :=
  M\{P/\Box\}$. That is, the contextual application of M to P is the
  substitution of $P$ for $\Box$ in $M$.
\end{definition}

$\meaningof{-} : L \to \mathcal{P}(\pi)$

\begin{mathpar}
  \inferrule* [lab=collection] {} {\meaningof{true} = \pi, \and \meaningof{~E} = \pi \setminus \meaningof{E}, \and \meaningof{E_{1} \& E_{2}} = \meaningof{E_{1}} \cap \meaningof{E_{2}}}
\end{mathpar}

\begin{mathpar}
  \inferrule* [lab=structure] {} {\meaningof{0} = \{ P \in \pi | P \equiv 0 \}, \and \\ \meaningof{E_1 | E_2} = \{ P \in \pi | P \equiv P_{1} | P_{2}, P_{1} \in \meaningof{E_{1}}, P_{2} \in \meaningof{E_2}\} }
\end{mathpar}

\begin{mathpar}
 \inferrule* [lab=behavior] {} {\meaningof{\langle a?b \rangle E} = \{ P \in \pi | P \equiv Q | u?(y)P', \\ \and \\\\ \and \\ \;\;\; u \in \meaningof{a}, \forall z.P'\{z/y\} \in \meaningof{E\{z/b\}}\}, \and \\ \meaningof{a!E} = \{ P \in \pi | P \equiv Q | x!\langle P' \rangle, x \in \meaningof{a} P' \in \meaningof{E}\} }
\end{mathpar}

\begin{mathpar}
 \inferrule* [lab=nominal] {} {\meaningof{\quotep{E}} = \{ \quotep{P} \in \quotep{\pi} | P \in \meaningof{E} \}, \and \meaningof{\quotep{P}} = \{ \quotep{Q} \in \quotep{\pi} | P \equiv Q \} \and \\ \meaningof{@\quotep{E}} = \{ P \in \pi | P \equiv @x, x \in \meaningof{E} \}}
\end{mathpar}

\begin{eqnarray*}
  \\
  \meaningof{-} : TS \to ST
\end{eqnarray*}

\begin{eqnarray*}
  \\
  L : TS \to ST
\end{eqnarray*}

\begin{eqnarray*}
  \\
  P \models E \iff P \in \meaningof{E}
\end{eqnarray*}

\begin{eqnarray*}
  P \approx_{L} Q \iff \forall E \in L. P \models E \iff Q \models E
\end{eqnarray*}

\begin{eqnarray*}
  P \approx_{K} Q
\end{eqnarray*}

\begin{eqnarray*}
  P \approx Q
\end{eqnarray*}

$\approx_{K} = \approx = \approx_{L}$

\subsubsection{Contextual duality}

Note that contexts extend the quotation operation to a family of
operations from processes to names. Given a context, $M$, we can
define a \emph{nominal context}, $\quotep{M}$ by $\quotep{M}[P] :=
\quotep{M[P]}$. To foreshadow what is to come we observe that these
operations enjoy a duality with processes very much like the duality
between vectors and maps from vectors to scalars.

Further, because the calculus is essentially higher-order, we have a
correspondence between contexts and processes. More specifically,
given a name $x$ and a context $M$ we can construct $M^{*}_{x}$ such
that 

\begin{mathpar}
  M^{*}_{x} | \lift{x}{P} \red M[P]
\end{mathpar}

namely,

\begin{mathpar}
  M^{*}_{x} := x?(u).M[\dropn{u}]
\end{mathpar}

The dependence of $M^{*}_{x}$ on a name makes it an abstraction, 

\begin{mathpar}
  M^{*} := (x)x?(u).M[\dropn{u}]
\end{mathpar}

\subsection{Additional notation}

It will sometimes be convenient to denote the process a name
quotes. We already have the notation $x = \quotep{P}$, but it will be
convenient to introduce an alternate notation, $\procn{x}$, when we
want to emphasize the connection to the use of the name. Note that, by
virtue of name equivalence, $\quotep{\procn{x}} \nameeq x$; so, the
notation is consistent with previous definitions.

Further, because names have structure it is possible to effect
substitutions on the basis of that structure. This means we need to
upgrade our notation for substitutions, which we accomplish by
adapting comprehension notation. Thus,

\begin{mathpar}
  P\{ y / x : x \in S \}
\end{mathpar}

is interpreted to mean the process derived from P by replacing (in a
capture-avoiding manner) each occurrence of $x$ in $S$ by $y$. For example,

\begin{mathpar}
  P\{ \quotep{\procn{x}|\procn{x}} / x : x \in \freenames{P} \}
\end{mathpar}

will replace each (occurrence) of a free name $x$ in $P$ by
$\quotep{\procn{x}|\procn{x}}$.

Also, we will avail ourselves of the notation $x^{L}$ and $x^{R}$ to
denote injections of a name into disjoint copies of the name
space. There are numerous ways to accomplish this. One example can be
found in \cite{MeredithR05}. This notation overloads to vectors of
names: $\vec{x}^{\pi} := (x_{i}^{\pi} \; : \; 0 \leq i < |\vec{x}| )$ where $\pi \in \{L,R\}$.

We also use $P^{\Box} := P|\Box$.

In \cite{MeredithR05} an interpretation of the new operator is
given. It turns out that there are several possible interpretations
all enjoying the requisite algebraic properties of the operator (see
\cite{milner91polyadicpi}). We will therefore make liberal use of
$(\nu\; \vec{x})P$.

% subsection the_syntax_and_semantics_of_the_notation_system (end)   

\input{qm2pi.qmops} 

\input{qm2pi.sterngerlach} 

\input{qm2pi.metric} 

% section concurrent_process_calculi (end)

%\input{qm2pi.proofsketch}

% section proof sketch (end)

%\input{qm2pi.slviaknots} 

% section spatial logic via knots (end)

\input{qm2pi.conclusion}

% section conclusion (end)

%\input{qm2pi.dtcodes} 

% section wiring algorithm (end)

\input{qm2pi.ack} 

% section acknowledgments (end)

\newpage


\bibliographystyle{plain}   
\bibliography{../../biblios/main.bib}

\input{qm2pi.rhodetails}

\end{document}

 

% section acknowledgments (end)

\newpage


\bibliographystyle{plain}   
\bibliography{../../biblios/main.bib}

\documentclass[12pt]{llncs}
%\documentclass{jktr}

\usepackage[pdftex]{hyperref}                   
\usepackage {listings}
\usepackage {mathpartir}
\usepackage{bcprules}
%\usepackage{listings}
                       
\usepackage{graphicx} 
%\usepackage[margins=2.5cm,nohead,nofoot]{geometry}
%\usepackage{geometry}
\usepackage{amsfonts}
\usepackage{amstext}
\usepackage{latexsym}
\usepackage{amssymb}
\usepackage{color}


%\include{myPreamble}
\include{qm2pi.local} 

%\ifpdf
%\usepackage[pdftex]{graphicx}
%\else
%\usepackage{graphicx}
%\fi

 % \ifpdf
%  \usepackage{pdfsync}
%  \if


%\title{Brief Article}
%\author{David F. Snyder}
%\author{L.G. Meredith}

%\address{Dept. of Math., Texas State University--San Marcos, San Marcos, TX 78666}
       
\pagestyle{empty}


\begin{document}

\lstset{language=[Objective]Caml,frame=shadowbox}

\input{qm2pi.front}

% section front matter (end)

\input{qm2pi.intro} 
 
% section introduction (end)

% \input{qm2pi.knotations} 

% section notation (end)

\input{qm2pi.process.calculi} 

% section concurrent_process_calculi_and_spatial_logics_ (end)
    
%\input{qm2pi.knots2pi} 

%\input{qm2pi.trefoil} 

%\input{qm2pi.mainthm} 

% subsection basic_interpretation (end)

%\input{qm2pi.rho.presentation} 
\subsection{The syntax and semantics of the notation system}\label{sub:the_syntax_and_semantics_of_the_notation_system} % (fold)

We now summarize a technical presentation of the calculus that
embodies our theory of dynamics. The typical presentation of such a
calculus follows the style of giving generators and relations on
them. The grammar, below, describing term constructors, freely
generates the set of processes, $\Proc$. This set is then quotiented
by a relation known as structural congruence and it is over this set
that the notion of dynamics is expressed. This presentation is
essentially that of \cite{MeredithR05} with the addition of
polyadicity and summation. For readability we have relegated some of
the technical subtleties to an appendix.

\subsubsection{Process grammar}\label{subsub:process_grammar}

\begin{mathpar}
  \inferrule* [lab=synchronization] {} {{M} \bc \pzero \;|\; x?F \;|\; x!C }
  \and
  \inferrule* [lab=abstraction] {} {{F} \bc (x)P}
  \and
  \inferrule* [lab=concretion] {} {{C} \bc \langle Q \rangle}
  \and
  \inferrule* [lab=process] {} {{P,Q} \bc M \;| \;P|Q \;|\; @{x}}
  \and
  \inferrule* [lab=name] {} {{x} \bc \quotep{P}}
\end{mathpar} 

Note that $\vec{x}$ (resp. $\vec{P}$) denotes a vector of names
(resp. processes) of length $|\vec{x}|$ (resp. $|\vec{P}|$). We adopt
the following useful abbreviations.

\begin{mathpar}
   x?(\vec{y}).P := x.(\vec{y})P \and  x\clift{\vec{P}} := x.\clift{\vec{P}}
   \and x!(y) := \lift{x}{\dropn{y}}
   \and \Pi_{i=0}^{n-1}P_i := P_0 | \ldots | P_{n-1}
\end{mathpar}

\subsubsection{Structural congruence}

\paragraph{Free and bound names and alpha-equivalence.} At the
core of structural equivalence is alpha-equivalence which identifies
process that are the same up to a change of variable. Formally, we
recognize the distinction between free and bound names. The free names
of a process, $\freenames{P}$, may be calculated recursively as
follows:

\begin{mathpar}
\freenames{\pzero} := \emptyset
  \and \\
  \freenames{x?(y).P} := \{ x \} \cup (\freenames{P} \setminus \{ y \})
  \and 
  \freenames{x!\langle P \rangle} := \{ x \} \cup \{ P \} 
  \and \\
  \freenames{P|Q} := \freenames{P} \cup \freenames{Q}
  \and \\
  \freenames{@{x}} := \{ x \}
\end{mathpar}

$\pi$
$\quotep{\pi}$

$\freenames{-} : \pi \to \mathcal{P}(\quotep{\pi})$

\begin{eqnarray*}
  \freenames{\pzero} & := & \emptyset \\
  \freenames{x?(y).P} & := & \{ x \} \cup (\freenames{P} \setminus \{ y \}) \\
  \freenames{x!\langle P \rangle} & := & \{ x \} \cup \{ P \} \\
  \freenames{P|Q} & := & \freenames{P} \cup \freenames{Q} \\
  \freenames{\dropn{x}} & := & \{ x \}
\end{eqnarray*}

The bound names of a process, $\boundnames{P}$, are those names occurring in $P$
that are not free. For example, in $x?(y).0$, the name $x$ is free, while $y$ is bound.

\begin{mathpar}
  \inferrule* [lab=monoidal-laws] {} { P|Q \equiv Q|P \and P|0 \equiv P \and P|(Q|R) \equiv (P|Q)|R }
\end{mathpar}

\begin{mathpar}
  \inferrule* [lab=alpha-equivalence] {} { (x)P \equiv (y)P\{y/x\} \and y \not\in \freenames{P} }
\end{mathpar}

\begin{definition}
Then two processes, $P,Q$, are alpha-equivalent if $P = Q\{\vec{y}/\vec{x}\}$ for
some $\vec{x} \in \boundnames{Q},\vec{y} \in \boundnames{P}$, where $Q\{\vec{y}/\vec{x}\}$
denotes the capture-avoiding substitution of $\vec{y}$ for $\vec{x}$ in $Q$.
\end{definition}

\begin{definition}
  The {\em structural congruence} \cite{SangiorgiWalker} , $\equiv$,
  between processes is the least congruence containing
  alpha-equivalence, satisfying the abelian monoid laws
  (associativity, commutativity and $\pzero$ as identity) for parallel
  composition $|$ and for summation $+$.
\end{definition}

\subsection{Name equivalence}

We take name equivalence, written $\nameeq$, to be the smallest
equivalence relation generated by the following rules.

\begin{mathpar}
\inferrule*[lab=Quote-drop]
{ }
{ \quotep{@{x}} \nameeq x }

\inferrule*[lab=Struct-equiv]
{ P \scong Q }
{ \quotep{P} \nameeq \quotep{Q} }
\end{mathpar}

The astute reader will have noticed that the mutual recursion of names
and processes imposes a mutual recursion on alpha-equivalence and
structural equivalence via name-equivalence. Fortunately, all of this
works out pleasantly and we may calculate in the natural way, free of
concern. The reader interested in the details is referred to the
appendix \ref{appendix:rho_details}.

\subsection{Substitution}

We use $\Proc$ for the set of processes, $\QProc$ for the set of
names, and $\id{\{}\vec{y} / \vec{x} \id{\}}$ to denote partial maps,
$s : \QProc \rightarrow \QProc$. A map, $s$ lifts, uniquely, to a map
on process terms, $\widehat{s} : \Proc \rightarrow \Proc$ by the
following equations.

\begin{mathpar}
  (0) \psubstp{Q}{P} := 0 \\
  (R \juxtap S) \psubstp{Q}{P}
  :=    
  (R)\psubstp{Q}{P} \juxtap (S) \psubstp{Q}{P} \\
  (x?(y).R) \psubstp{Q}{P}    
  :=    
  (x)\substp{Q}{P} (z)\concat( (R \psubstn{z}{y}) \psubstp{Q}{P} ) \\
  (\lift{x}{R}) \psubstp{Q}{P}  
  :=
  \lift{(x)\substp{Q}{P}}{ R \psubstp{Q}{P} } \\
%   (\dropn{x})  \psubstp{Q}{P}       
%   := 
%   \left\{ 
%     \begin{array}{ccc} 
%       \dropn{\quotep{Q}} & & x \nameeq \quotep{P} \\
%       \dropn{x} & & otherwise \\
%     \end{array}
%   \right. 
  (\dropn{x})  \psubstp{Q}{P}       
  := 
  \left\{ 
    \begin{array}{ccc} 
      Q & & x \nameeq \quotep{P} \\
      \dropn{x} & & otherwise \\
    \end{array}
  \right.
\end{mathpar}
 

where

\begin{eqnarray}
  (x)\id{\{} \lpquote Q \rpquote / \lpquote P \rpquote \id{\}}            = 
  \left\{ 
    \begin{array}{ccc}
      \lpquote Q \rpquote & & x \nameeq \lpquote P \rpquote \\
      x & & otherwise \\
    \end{array}
  \right. \nonumber
\end{eqnarray}

and $z$ is chosen distinct from $\quotep{P}$, $\quotep{Q}$, the free
names in $Q$, and all the names in $R$. Our $\alpha$-equivalence will
be built in the standard way from this substitution.

\begin{remark}\label{rem:no_self_referential_names}
  One consequence of these definitions is that $\forall P. \quotep{P}
  \not\in \freenames{P}$.
\end{remark}

\subsection{ Dynamic quote: an example }

Anticipating something of what's to come, consider applying the
substitution, $\widehat{\id{\{}u / z \id{\}}}$, to the following pair
of processes, $\lift{w}{y!(z)}$ and $w[ \lpquote y!(z) \rpquote ]$.

\begin{eqnarray}
	\lift{w}{y!(z)}\widehat{\id{\{}u / z \id{\}}}
		& = &
		\lift{w}{y!(u)} \nonumber\\
	w[ \lpquote y!(z) \rpquote ] \widehat{ \id{\{}u / z \id{\}} }
		& = &
		w[ \lpquote y!(z) \rpquote ] \nonumber
\end{eqnarray}

Because the body of the process between quotes is impervious to
substitution, we get radically different answers. In fact, by
examining the first process in an input context,
e.g. $x?(z).\lift{w}{y!(z)}$, we see that the process under the lift
operator may be shaped by prefixed inputs binding a name inside it. In
this sense, the lift operator will be seen as a way to dynamically
construct processes before reifying them as names.

Finally equipped with these standard features we can present the
dynamics of the calculus.

\subsubsection{Operational semantics} 

Finally, we introduce the computational dynamics. What marks these
algebras as distinct from other more traditionally studied algebraic
structures, e.g. vector spaces or polynomial rings, is the manner in
which dynamics is captured. In traditional structures, dynamics is typically
expressed through morphisms between such structures, as in linear maps
between vector spaces or morphisms between rings. In algebras
associated with the semantics of computation, the dynamics is
expressed as part of the algebraic structure itself, through a
reduction reduction relation typically denoted by $\red$. Below, we
give a recursive presentation of this relation for the calculus used
in the encoding.

$\red \subseteq \pi \times \pi$
$\red : \pi \to \mathcal{P}(\pi)$

\begin{mathpar}
  \inferrule* [lab=Comm] { \textsf{match}( x_{src}, x_{trgt} ) } { x_{trgt}?(y)P \; | \; x_{src}!\langle {Q} \rangle \red P\{\quotep{Q}/y}\} }
  \and \\
  \inferrule* [lab=Par] {{P} \red {P}'} {{{P} | {Q}} \red {{P}' | {Q}}}
  \and
  \inferrule* [lab=Equiv]{{{P} \scong {P}'} \andalso {{P}' \red {Q}'} \andalso {{Q}' \scong {Q}}}{{P} \red {Q}}
\end{mathpar}

\begin{eqnarray*}
  match_{\equiv} (\quotep{P},\quotep{Q}) & := & P \equiv Q \\
  match_{\dagger}(\quotep{P},\quotep{Q}) & := & \forall R. P|Q \red^{*} R => R \red^{*} 0 \\
  match_{K}(\quotep{P},\quotep{Q}) & := & K \mbox{ for some context } K
\end{eqnarray*}

$u?(x)P | u!\langle Q \rangle \red P\{\quotep{Q}/x\}$

%We write $\wred$ for $\red^*$, and $P\red$ if $\exists Q $ such that $ P \red Q$.
We write $P\red$ if $\exists Q $ such that $ P \red Q$ and $P\not\red$, otherwise.

\section{Replication}

As mentioned before, it is known that replication (and hence
recursion) can be implemented in a higher-order process algebra
\cite{SangiorgiWalker}. As our first example of calculation with the
machinery thus far presented we give the construction explicitly in
the {\rhoc}.

\begin{eqnarray}
	D_{x} & := & \prefix{x}{y}{(\binpar{\outputp{x}{y}}{@{y}})} \nonumber\\
	\bangp_{x}{P} & := & \binpar{{x}!\langle{\binpar{D_{x}}{P}}\rangle}{D_{x}} \nonumber
\end{eqnarray}

\begin{eqnarray}
	\bangp_{x}{P} & & \nonumber\\
	=
	& {x}!\langle{(\prefix{x}{y}{(\outputp{x}{y} | @{y})) | P}}\rangle 
	      | \prefix{x}{y}{(\outputp{x}{y} | @{y})} & \nonumber\\
	\red
	& (\outputp{x}{y} | @{y})\substn{\quotep{(\prefix{x}{y}{(@{y} | \outputp{x}{y})) | P}}}{y} & \nonumber\\
	=
	& \outputp{x}{\quotep{(\prefix{x}{y}{(\outputp{x}{y} | @{y})) | P}}}
	  | {(\prefix{x}{y}{(\outputp{x}{y} | @{y})) | P}} & \nonumber\\
	\red
	& \ldots & \nonumber\\
	\red^*
	& P | P | \ldots & \nonumber
\end{eqnarray}

Of course, this encoding, as an implementation, runs away, unfolding
$\bangp{P}$ eagerly. A lazier and more implementable replication
operator, restricted to input-guarded processes, may be obtained as follows.

\begin{eqnarray}
\bangp{\prefix{u}{v}{P}} 
	:= 
	\binpar{\lift{x}{\prefix{u}{v}{(\binpar{D(x)}{P})}}}{D(x)} \nonumber
\end{eqnarray}

\begin{remark}
  Note that the lazier definition still does not deal with summation
  or mixed summation (i.e. sums over input and output). The reader is
  invited to construct definitions of replication that deal with these
  features. 

  Further, the definitions are parameterized in a name, $x$. Can you,
  gentle reader, make a definition that eliminates this parameter and
  guarantees no accidental interaction between the replication
  machinery and the process being replicated -- i.e. no accidental
  sharing of names used by the process to get its work done and the
  name(s) used by the replication to effect copying. This latter
  revision of the definition of replication is crucial to obtaining
  the expected identity $!!P \sim !P$.
\end{remark}

\begin{remark}\label{rem:paradoxical_combinator}
  The reader familiar with the lambda calculus will have noticed the
  similarity between $D$ and the paradoxical combinator.

  [Ed. note: the existence of this seems to suggest we have to be more
  restrictive on the set of processes and names we admit if we are to
  support no-cloning.]
\end{remark}

\subsubsection{Bisimulation}

The computational dynamics gives rise to another kind of equivalence,
the equivalence of computational behavior. As previously mentioned
this is typically captured \emph{via} some form of bisimulation.

% The notion we use in this paper is weak barbed bisimulation
% \cite{milner91polyadicpi}.

The notion we use in this paper is derived from weak barbed
bisimulation \cite{milner91polyadicpi}. 

\begin{definition}
An \emph{observation relation}, $\downarrow_{\mathcal N}$, over a set
of names, $\mathcal N$, is the smallest relation satisfying the rules
below.

\infrule[Out-barb]{y \in {\mathcal N}, \; x \nameeq y}
		  {\outputp{x}{v} \downarrow_{\mathcal N} x}
\infrule[Par-barb]{\mbox{$P\downarrow_{\mathcal N} x$ or $Q\downarrow_{\mathcal N} x$}}
		  {\binpar{P}{Q} \downarrow_{\mathcal N} x}

We write $P \Downarrow_{\mathcal N} x$ if there is $Q$ such that 
$P \wred Q$ and $Q \downarrow_{\mathcal N} x$.
\end{definition}

\begin{definition}
%\label{def.bbisim}
An  ${\mathcal N}$-\emph{barbed bisimulation} over a set of names, ${\mathcal N}$, is a symmetric binary relation 
${\mathcal S}_{\mathcal N}$ between agents such that $P\rel{S}_{\mathcal N}Q$ implies:
\begin{enumerate}
\item If $P \red P'$ then $Q \wred Q'$ and $P'\rel{S}_{\mathcal N} Q'$.
\item If $P\downarrow_{\mathcal N} x$, then $Q\Downarrow_{\mathcal N} x$.
\end{enumerate}
$P$ is ${\mathcal N}$-barbed bisimilar to $Q$, written
$P \wbbisim_{\mathcal N} Q$, if $P \rel{S}_{\mathcal N} Q$ for some ${\mathcal N}$-barbed bisimulation ${\mathcal S}_{\mathcal N}$.
\end{definition}

$\mathcal{R} \subseteq \pi \times \pi$

$P \mathcal{R} Q => \forall P'. P \red P' \Rightarrow \exists Q'. Q \red Q', P' \mathcal{R} Q'$

$P \vdash x \Rightarrow Q \vdash x$

\begin{mathpar}
  \inferrule*[lab=Out-barb]{x \nameeq y}{{y}!\langle{Q}\rangle \vdash x}
  \and
  \inferrule*[lab=Par-barb]{\mbox{$P\vdash x$ or $Q\vdash x$}}{\binpar{P}{Q} \vdash x}
\end{mathpar}

\subsubsection{Contexts}

One of the principle advantages of computational calculi like the
$\pi$-calculus is a well-defined notion of context,
contextual-equivalence and a correlation between
contextual-equivalence and notions of bisimulation. The notion of
context allows the decomposition of a process into (sub-)process and
its syntactic environment, its context. Thus, a context may be
thought of as a process with a ``hole'' (written $\Box$) in it. The
application of a context $M$ to a process $P$, written $M[P]$, is
tantamount to filling the hole in $M$ with $P$. In this paper we do
not need the full weight of this theory, but do make use of the notion
of context in the proof the main theorem. 

\begin{mathpar}
  \inferrule* [lab=summation] {} {{M_{M},M_{N}} \bc \Box \;|\; x.M_{A} \;|\; M_{M}+M_{N}}
  \and
  \inferrule* [lab=agent] {} {{M_{A}} \bc (\vec{x})M_{P} \;| \; \clift{P_0,\ldots,M_{P},\ldots,P_N}}
  \and \\
  \inferrule* [lab=process] {} {{M_{P}} \bc M_{N} \;| \;P|M_{P} }
\end{mathpar} 

\begin{mathpar}
  \inferrule* [lab=sychronization] {} {M_{N} \bc \Box \;|\; x?M_{F} \;|\; x!M_{C}}
  \and
  \inferrule* [lab=abstraction] {} {{M_{F}} \bc (x)M_{P} }
  \and
  \inferrule* [lab=concretion] {} {{M_{C}} \bc \langle M_{P} \rangle }
  \and \\
  \inferrule* [lab=process] {} {{M_{P}} \bc M_{N} \;| \;P|M_{P} }
\end{mathpar}

\begin{definition}[contextual application] Given a context $M$, and
  process $P$, we define the \emph{contextual application}, $M[P] :=
  M\{P/\Box\}$. That is, the contextual application of M to P is the
  substitution of $P$ for $\Box$ in $M$.
\end{definition}

$\meaningof{-} : L \to \mathcal{P}(\pi)$

\begin{mathpar}
  \inferrule* [lab=collection] {} {\meaningof{true} = \pi, \and \meaningof{~E} = \pi \setminus \meaningof{E}, \and \meaningof{E_{1} \& E_{2}} = \meaningof{E_{1}} \cap \meaningof{E_{2}}}
\end{mathpar}

\begin{mathpar}
  \inferrule* [lab=structure] {} {\meaningof{0} = \{ P \in \pi | P \equiv 0 \}, \and \\ \meaningof{E_1 | E_2} = \{ P \in \pi | P \equiv P_{1} | P_{2}, P_{1} \in \meaningof{E_{1}}, P_{2} \in \meaningof{E_2}\} }
\end{mathpar}

\begin{mathpar}
 \inferrule* [lab=behavior] {} {\meaningof{\langle a?b \rangle E} = \{ P \in \pi | P \equiv Q | u?(y)P', \\ \and \\\\ \and \\ \;\;\; u \in \meaningof{a}, \forall z.P'\{z/y\} \in \meaningof{E\{z/b\}}\}, \and \\ \meaningof{a!E} = \{ P \in \pi | P \equiv Q | x!\langle P' \rangle, x \in \meaningof{a} P' \in \meaningof{E}\} }
\end{mathpar}

\begin{mathpar}
 \inferrule* [lab=nominal] {} {\meaningof{\quotep{E}} = \{ \quotep{P} \in \quotep{\pi} | P \in \meaningof{E} \}, \and \meaningof{\quotep{P}} = \{ \quotep{Q} \in \quotep{\pi} | P \equiv Q \} \and \\ \meaningof{@\quotep{E}} = \{ P \in \pi | P \equiv @x, x \in \meaningof{E} \}}
\end{mathpar}

\begin{eqnarray*}
  \\
  \meaningof{-} : TS \to ST
\end{eqnarray*}

\begin{eqnarray*}
  \\
  L : TS \to ST
\end{eqnarray*}

\begin{eqnarray*}
  \\
  P \models E \iff P \in \meaningof{E}
\end{eqnarray*}

\begin{eqnarray*}
  P \approx_{L} Q \iff \forall E \in L. P \models E \iff Q \models E
\end{eqnarray*}

\begin{eqnarray*}
  P \approx_{K} Q
\end{eqnarray*}

\begin{eqnarray*}
  P \approx Q
\end{eqnarray*}

$\approx_{K} = \approx = \approx_{L}$

\subsubsection{Contextual duality}

Note that contexts extend the quotation operation to a family of
operations from processes to names. Given a context, $M$, we can
define a \emph{nominal context}, $\quotep{M}$ by $\quotep{M}[P] :=
\quotep{M[P]}$. To foreshadow what is to come we observe that these
operations enjoy a duality with processes very much like the duality
between vectors and maps from vectors to scalars.

Further, because the calculus is essentially higher-order, we have a
correspondence between contexts and processes. More specifically,
given a name $x$ and a context $M$ we can construct $M^{*}_{x}$ such
that 

\begin{mathpar}
  M^{*}_{x} | \lift{x}{P} \red M[P]
\end{mathpar}

namely,

\begin{mathpar}
  M^{*}_{x} := x?(u).M[\dropn{u}]
\end{mathpar}

The dependence of $M^{*}_{x}$ on a name makes it an abstraction, 

\begin{mathpar}
  M^{*} := (x)x?(u).M[\dropn{u}]
\end{mathpar}

\subsection{Additional notation}

It will sometimes be convenient to denote the process a name
quotes. We already have the notation $x = \quotep{P}$, but it will be
convenient to introduce an alternate notation, $\procn{x}$, when we
want to emphasize the connection to the use of the name. Note that, by
virtue of name equivalence, $\quotep{\procn{x}} \nameeq x$; so, the
notation is consistent with previous definitions.

Further, because names have structure it is possible to effect
substitutions on the basis of that structure. This means we need to
upgrade our notation for substitutions, which we accomplish by
adapting comprehension notation. Thus,

\begin{mathpar}
  P\{ y / x : x \in S \}
\end{mathpar}

is interpreted to mean the process derived from P by replacing (in a
capture-avoiding manner) each occurrence of $x$ in $S$ by $y$. For example,

\begin{mathpar}
  P\{ \quotep{\procn{x}|\procn{x}} / x : x \in \freenames{P} \}
\end{mathpar}

will replace each (occurrence) of a free name $x$ in $P$ by
$\quotep{\procn{x}|\procn{x}}$.

Also, we will avail ourselves of the notation $x^{L}$ and $x^{R}$ to
denote injections of a name into disjoint copies of the name
space. There are numerous ways to accomplish this. One example can be
found in \cite{MeredithR05}. This notation overloads to vectors of
names: $\vec{x}^{\pi} := (x_{i}^{\pi} \; : \; 0 \leq i < |\vec{x}| )$ where $\pi \in \{L,R\}$.

We also use $P^{\Box} := P|\Box$.

In \cite{MeredithR05} an interpretation of the new operator is
given. It turns out that there are several possible interpretations
all enjoying the requisite algebraic properties of the operator (see
\cite{milner91polyadicpi}). We will therefore make liberal use of
$(\nu\; \vec{x})P$.

% subsection the_syntax_and_semantics_of_the_notation_system (end)   

\input{qm2pi.qmops} 

\input{qm2pi.sterngerlach} 

\input{qm2pi.metric} 

% section concurrent_process_calculi (end)

%\input{qm2pi.proofsketch}

% section proof sketch (end)

%\input{qm2pi.slviaknots} 

% section spatial logic via knots (end)

\input{qm2pi.conclusion}

% section conclusion (end)

%\input{qm2pi.dtcodes} 

% section wiring algorithm (end)

\input{qm2pi.ack} 

% section acknowledgments (end)

\newpage


\bibliographystyle{plain}   
\bibliography{../../biblios/main.bib}

\input{qm2pi.rhodetails}

\end{document}



\end{document}

 

% section acknowledgments (end)

\newpage


\bibliographystyle{plain}   
\bibliography{../../biblios/main.bib}

\documentclass[12pt]{llncs}
%\documentclass{jktr}

\usepackage[pdftex]{hyperref}                   
\usepackage {listings}
\usepackage {mathpartir}
\usepackage{bcprules}
%\usepackage{listings}
                       
\usepackage{graphicx} 
%\usepackage[margins=2.5cm,nohead,nofoot]{geometry}
%\usepackage{geometry}
\usepackage{amsfonts}
\usepackage{amstext}
\usepackage{latexsym}
\usepackage{amssymb}
\usepackage{color}


%\include{myPreamble}
\documentclass[12pt]{llncs}
%\documentclass{jktr}

\usepackage[pdftex]{hyperref}                   
\usepackage {listings}
\usepackage {mathpartir}
\usepackage{bcprules}
%\usepackage{listings}
                       
\usepackage{graphicx} 
%\usepackage[margins=2.5cm,nohead,nofoot]{geometry}
%\usepackage{geometry}
\usepackage{amsfonts}
\usepackage{amstext}
\usepackage{latexsym}
\usepackage{amssymb}
\usepackage{color}


%\include{myPreamble}
\include{qm2pi.local} 

%\ifpdf
%\usepackage[pdftex]{graphicx}
%\else
%\usepackage{graphicx}
%\fi

 % \ifpdf
%  \usepackage{pdfsync}
%  \if


%\title{Brief Article}
%\author{David F. Snyder}
%\author{L.G. Meredith}

%\address{Dept. of Math., Texas State University--San Marcos, San Marcos, TX 78666}
       
\pagestyle{empty}


\begin{document}

\lstset{language=[Objective]Caml,frame=shadowbox}

\input{qm2pi.front}

% section front matter (end)

\input{qm2pi.intro} 
 
% section introduction (end)

% \input{qm2pi.knotations} 

% section notation (end)

\input{qm2pi.process.calculi} 

% section concurrent_process_calculi_and_spatial_logics_ (end)
    
%\input{qm2pi.knots2pi} 

%\input{qm2pi.trefoil} 

%\input{qm2pi.mainthm} 

% subsection basic_interpretation (end)

%\input{qm2pi.rho.presentation} 
\subsection{The syntax and semantics of the notation system}\label{sub:the_syntax_and_semantics_of_the_notation_system} % (fold)

We now summarize a technical presentation of the calculus that
embodies our theory of dynamics. The typical presentation of such a
calculus follows the style of giving generators and relations on
them. The grammar, below, describing term constructors, freely
generates the set of processes, $\Proc$. This set is then quotiented
by a relation known as structural congruence and it is over this set
that the notion of dynamics is expressed. This presentation is
essentially that of \cite{MeredithR05} with the addition of
polyadicity and summation. For readability we have relegated some of
the technical subtleties to an appendix.

\subsubsection{Process grammar}\label{subsub:process_grammar}

\begin{mathpar}
  \inferrule* [lab=synchronization] {} {{M} \bc \pzero \;|\; x?F \;|\; x!C }
  \and
  \inferrule* [lab=abstraction] {} {{F} \bc (x)P}
  \and
  \inferrule* [lab=concretion] {} {{C} \bc \langle Q \rangle}
  \and
  \inferrule* [lab=process] {} {{P,Q} \bc M \;| \;P|Q \;|\; @{x}}
  \and
  \inferrule* [lab=name] {} {{x} \bc \quotep{P}}
\end{mathpar} 

Note that $\vec{x}$ (resp. $\vec{P}$) denotes a vector of names
(resp. processes) of length $|\vec{x}|$ (resp. $|\vec{P}|$). We adopt
the following useful abbreviations.

\begin{mathpar}
   x?(\vec{y}).P := x.(\vec{y})P \and  x\clift{\vec{P}} := x.\clift{\vec{P}}
   \and x!(y) := \lift{x}{\dropn{y}}
   \and \Pi_{i=0}^{n-1}P_i := P_0 | \ldots | P_{n-1}
\end{mathpar}

\subsubsection{Structural congruence}

\paragraph{Free and bound names and alpha-equivalence.} At the
core of structural equivalence is alpha-equivalence which identifies
process that are the same up to a change of variable. Formally, we
recognize the distinction between free and bound names. The free names
of a process, $\freenames{P}$, may be calculated recursively as
follows:

\begin{mathpar}
\freenames{\pzero} := \emptyset
  \and \\
  \freenames{x?(y).P} := \{ x \} \cup (\freenames{P} \setminus \{ y \})
  \and 
  \freenames{x!\langle P \rangle} := \{ x \} \cup \{ P \} 
  \and \\
  \freenames{P|Q} := \freenames{P} \cup \freenames{Q}
  \and \\
  \freenames{@{x}} := \{ x \}
\end{mathpar}

$\pi$
$\quotep{\pi}$

$\freenames{-} : \pi \to \mathcal{P}(\quotep{\pi})$

\begin{eqnarray*}
  \freenames{\pzero} & := & \emptyset \\
  \freenames{x?(y).P} & := & \{ x \} \cup (\freenames{P} \setminus \{ y \}) \\
  \freenames{x!\langle P \rangle} & := & \{ x \} \cup \{ P \} \\
  \freenames{P|Q} & := & \freenames{P} \cup \freenames{Q} \\
  \freenames{\dropn{x}} & := & \{ x \}
\end{eqnarray*}

The bound names of a process, $\boundnames{P}$, are those names occurring in $P$
that are not free. For example, in $x?(y).0$, the name $x$ is free, while $y$ is bound.

\begin{mathpar}
  \inferrule* [lab=monoidal-laws] {} { P|Q \equiv Q|P \and P|0 \equiv P \and P|(Q|R) \equiv (P|Q)|R }
\end{mathpar}

\begin{mathpar}
  \inferrule* [lab=alpha-equivalence] {} { (x)P \equiv (y)P\{y/x\} \and y \not\in \freenames{P} }
\end{mathpar}

\begin{definition}
Then two processes, $P,Q$, are alpha-equivalent if $P = Q\{\vec{y}/\vec{x}\}$ for
some $\vec{x} \in \boundnames{Q},\vec{y} \in \boundnames{P}$, where $Q\{\vec{y}/\vec{x}\}$
denotes the capture-avoiding substitution of $\vec{y}$ for $\vec{x}$ in $Q$.
\end{definition}

\begin{definition}
  The {\em structural congruence} \cite{SangiorgiWalker} , $\equiv$,
  between processes is the least congruence containing
  alpha-equivalence, satisfying the abelian monoid laws
  (associativity, commutativity and $\pzero$ as identity) for parallel
  composition $|$ and for summation $+$.
\end{definition}

\subsection{Name equivalence}

We take name equivalence, written $\nameeq$, to be the smallest
equivalence relation generated by the following rules.

\begin{mathpar}
\inferrule*[lab=Quote-drop]
{ }
{ \quotep{@{x}} \nameeq x }

\inferrule*[lab=Struct-equiv]
{ P \scong Q }
{ \quotep{P} \nameeq \quotep{Q} }
\end{mathpar}

The astute reader will have noticed that the mutual recursion of names
and processes imposes a mutual recursion on alpha-equivalence and
structural equivalence via name-equivalence. Fortunately, all of this
works out pleasantly and we may calculate in the natural way, free of
concern. The reader interested in the details is referred to the
appendix \ref{appendix:rho_details}.

\subsection{Substitution}

We use $\Proc$ for the set of processes, $\QProc$ for the set of
names, and $\id{\{}\vec{y} / \vec{x} \id{\}}$ to denote partial maps,
$s : \QProc \rightarrow \QProc$. A map, $s$ lifts, uniquely, to a map
on process terms, $\widehat{s} : \Proc \rightarrow \Proc$ by the
following equations.

\begin{mathpar}
  (0) \psubstp{Q}{P} := 0 \\
  (R \juxtap S) \psubstp{Q}{P}
  :=    
  (R)\psubstp{Q}{P} \juxtap (S) \psubstp{Q}{P} \\
  (x?(y).R) \psubstp{Q}{P}    
  :=    
  (x)\substp{Q}{P} (z)\concat( (R \psubstn{z}{y}) \psubstp{Q}{P} ) \\
  (\lift{x}{R}) \psubstp{Q}{P}  
  :=
  \lift{(x)\substp{Q}{P}}{ R \psubstp{Q}{P} } \\
%   (\dropn{x})  \psubstp{Q}{P}       
%   := 
%   \left\{ 
%     \begin{array}{ccc} 
%       \dropn{\quotep{Q}} & & x \nameeq \quotep{P} \\
%       \dropn{x} & & otherwise \\
%     \end{array}
%   \right. 
  (\dropn{x})  \psubstp{Q}{P}       
  := 
  \left\{ 
    \begin{array}{ccc} 
      Q & & x \nameeq \quotep{P} \\
      \dropn{x} & & otherwise \\
    \end{array}
  \right.
\end{mathpar}
 

where

\begin{eqnarray}
  (x)\id{\{} \lpquote Q \rpquote / \lpquote P \rpquote \id{\}}            = 
  \left\{ 
    \begin{array}{ccc}
      \lpquote Q \rpquote & & x \nameeq \lpquote P \rpquote \\
      x & & otherwise \\
    \end{array}
  \right. \nonumber
\end{eqnarray}

and $z$ is chosen distinct from $\quotep{P}$, $\quotep{Q}$, the free
names in $Q$, and all the names in $R$. Our $\alpha$-equivalence will
be built in the standard way from this substitution.

\begin{remark}\label{rem:no_self_referential_names}
  One consequence of these definitions is that $\forall P. \quotep{P}
  \not\in \freenames{P}$.
\end{remark}

\subsection{ Dynamic quote: an example }

Anticipating something of what's to come, consider applying the
substitution, $\widehat{\id{\{}u / z \id{\}}}$, to the following pair
of processes, $\lift{w}{y!(z)}$ and $w[ \lpquote y!(z) \rpquote ]$.

\begin{eqnarray}
	\lift{w}{y!(z)}\widehat{\id{\{}u / z \id{\}}}
		& = &
		\lift{w}{y!(u)} \nonumber\\
	w[ \lpquote y!(z) \rpquote ] \widehat{ \id{\{}u / z \id{\}} }
		& = &
		w[ \lpquote y!(z) \rpquote ] \nonumber
\end{eqnarray}

Because the body of the process between quotes is impervious to
substitution, we get radically different answers. In fact, by
examining the first process in an input context,
e.g. $x?(z).\lift{w}{y!(z)}$, we see that the process under the lift
operator may be shaped by prefixed inputs binding a name inside it. In
this sense, the lift operator will be seen as a way to dynamically
construct processes before reifying them as names.

Finally equipped with these standard features we can present the
dynamics of the calculus.

\subsubsection{Operational semantics} 

Finally, we introduce the computational dynamics. What marks these
algebras as distinct from other more traditionally studied algebraic
structures, e.g. vector spaces or polynomial rings, is the manner in
which dynamics is captured. In traditional structures, dynamics is typically
expressed through morphisms between such structures, as in linear maps
between vector spaces or morphisms between rings. In algebras
associated with the semantics of computation, the dynamics is
expressed as part of the algebraic structure itself, through a
reduction reduction relation typically denoted by $\red$. Below, we
give a recursive presentation of this relation for the calculus used
in the encoding.

$\red \subseteq \pi \times \pi$
$\red : \pi \to \mathcal{P}(\pi)$

\begin{mathpar}
  \inferrule* [lab=Comm] { \textsf{match}( x_{src}, x_{trgt} ) } { x_{trgt}?(y)P \; | \; x_{src}!\langle {Q} \rangle \red P\{\quotep{Q}/y}\} }
  \and \\
  \inferrule* [lab=Par] {{P} \red {P}'} {{{P} | {Q}} \red {{P}' | {Q}}}
  \and
  \inferrule* [lab=Equiv]{{{P} \scong {P}'} \andalso {{P}' \red {Q}'} \andalso {{Q}' \scong {Q}}}{{P} \red {Q}}
\end{mathpar}

\begin{eqnarray*}
  match_{\equiv} (\quotep{P},\quotep{Q}) & := & P \equiv Q \\
  match_{\dagger}(\quotep{P},\quotep{Q}) & := & \forall R. P|Q \red^{*} R => R \red^{*} 0 \\
  match_{K}(\quotep{P},\quotep{Q}) & := & K \mbox{ for some context } K
\end{eqnarray*}

$u?(x)P | u!\langle Q \rangle \red P\{\quotep{Q}/x\}$

%We write $\wred$ for $\red^*$, and $P\red$ if $\exists Q $ such that $ P \red Q$.
We write $P\red$ if $\exists Q $ such that $ P \red Q$ and $P\not\red$, otherwise.

\section{Replication}

As mentioned before, it is known that replication (and hence
recursion) can be implemented in a higher-order process algebra
\cite{SangiorgiWalker}. As our first example of calculation with the
machinery thus far presented we give the construction explicitly in
the {\rhoc}.

\begin{eqnarray}
	D_{x} & := & \prefix{x}{y}{(\binpar{\outputp{x}{y}}{@{y}})} \nonumber\\
	\bangp_{x}{P} & := & \binpar{{x}!\langle{\binpar{D_{x}}{P}}\rangle}{D_{x}} \nonumber
\end{eqnarray}

\begin{eqnarray}
	\bangp_{x}{P} & & \nonumber\\
	=
	& {x}!\langle{(\prefix{x}{y}{(\outputp{x}{y} | @{y})) | P}}\rangle 
	      | \prefix{x}{y}{(\outputp{x}{y} | @{y})} & \nonumber\\
	\red
	& (\outputp{x}{y} | @{y})\substn{\quotep{(\prefix{x}{y}{(@{y} | \outputp{x}{y})) | P}}}{y} & \nonumber\\
	=
	& \outputp{x}{\quotep{(\prefix{x}{y}{(\outputp{x}{y} | @{y})) | P}}}
	  | {(\prefix{x}{y}{(\outputp{x}{y} | @{y})) | P}} & \nonumber\\
	\red
	& \ldots & \nonumber\\
	\red^*
	& P | P | \ldots & \nonumber
\end{eqnarray}

Of course, this encoding, as an implementation, runs away, unfolding
$\bangp{P}$ eagerly. A lazier and more implementable replication
operator, restricted to input-guarded processes, may be obtained as follows.

\begin{eqnarray}
\bangp{\prefix{u}{v}{P}} 
	:= 
	\binpar{\lift{x}{\prefix{u}{v}{(\binpar{D(x)}{P})}}}{D(x)} \nonumber
\end{eqnarray}

\begin{remark}
  Note that the lazier definition still does not deal with summation
  or mixed summation (i.e. sums over input and output). The reader is
  invited to construct definitions of replication that deal with these
  features. 

  Further, the definitions are parameterized in a name, $x$. Can you,
  gentle reader, make a definition that eliminates this parameter and
  guarantees no accidental interaction between the replication
  machinery and the process being replicated -- i.e. no accidental
  sharing of names used by the process to get its work done and the
  name(s) used by the replication to effect copying. This latter
  revision of the definition of replication is crucial to obtaining
  the expected identity $!!P \sim !P$.
\end{remark}

\begin{remark}\label{rem:paradoxical_combinator}
  The reader familiar with the lambda calculus will have noticed the
  similarity between $D$ and the paradoxical combinator.

  [Ed. note: the existence of this seems to suggest we have to be more
  restrictive on the set of processes and names we admit if we are to
  support no-cloning.]
\end{remark}

\subsubsection{Bisimulation}

The computational dynamics gives rise to another kind of equivalence,
the equivalence of computational behavior. As previously mentioned
this is typically captured \emph{via} some form of bisimulation.

% The notion we use in this paper is weak barbed bisimulation
% \cite{milner91polyadicpi}.

The notion we use in this paper is derived from weak barbed
bisimulation \cite{milner91polyadicpi}. 

\begin{definition}
An \emph{observation relation}, $\downarrow_{\mathcal N}$, over a set
of names, $\mathcal N$, is the smallest relation satisfying the rules
below.

\infrule[Out-barb]{y \in {\mathcal N}, \; x \nameeq y}
		  {\outputp{x}{v} \downarrow_{\mathcal N} x}
\infrule[Par-barb]{\mbox{$P\downarrow_{\mathcal N} x$ or $Q\downarrow_{\mathcal N} x$}}
		  {\binpar{P}{Q} \downarrow_{\mathcal N} x}

We write $P \Downarrow_{\mathcal N} x$ if there is $Q$ such that 
$P \wred Q$ and $Q \downarrow_{\mathcal N} x$.
\end{definition}

\begin{definition}
%\label{def.bbisim}
An  ${\mathcal N}$-\emph{barbed bisimulation} over a set of names, ${\mathcal N}$, is a symmetric binary relation 
${\mathcal S}_{\mathcal N}$ between agents such that $P\rel{S}_{\mathcal N}Q$ implies:
\begin{enumerate}
\item If $P \red P'$ then $Q \wred Q'$ and $P'\rel{S}_{\mathcal N} Q'$.
\item If $P\downarrow_{\mathcal N} x$, then $Q\Downarrow_{\mathcal N} x$.
\end{enumerate}
$P$ is ${\mathcal N}$-barbed bisimilar to $Q$, written
$P \wbbisim_{\mathcal N} Q$, if $P \rel{S}_{\mathcal N} Q$ for some ${\mathcal N}$-barbed bisimulation ${\mathcal S}_{\mathcal N}$.
\end{definition}

$\mathcal{R} \subseteq \pi \times \pi$

$P \mathcal{R} Q => \forall P'. P \red P' \Rightarrow \exists Q'. Q \red Q', P' \mathcal{R} Q'$

$P \vdash x \Rightarrow Q \vdash x$

\begin{mathpar}
  \inferrule*[lab=Out-barb]{x \nameeq y}{{y}!\langle{Q}\rangle \vdash x}
  \and
  \inferrule*[lab=Par-barb]{\mbox{$P\vdash x$ or $Q\vdash x$}}{\binpar{P}{Q} \vdash x}
\end{mathpar}

\subsubsection{Contexts}

One of the principle advantages of computational calculi like the
$\pi$-calculus is a well-defined notion of context,
contextual-equivalence and a correlation between
contextual-equivalence and notions of bisimulation. The notion of
context allows the decomposition of a process into (sub-)process and
its syntactic environment, its context. Thus, a context may be
thought of as a process with a ``hole'' (written $\Box$) in it. The
application of a context $M$ to a process $P$, written $M[P]$, is
tantamount to filling the hole in $M$ with $P$. In this paper we do
not need the full weight of this theory, but do make use of the notion
of context in the proof the main theorem. 

\begin{mathpar}
  \inferrule* [lab=summation] {} {{M_{M},M_{N}} \bc \Box \;|\; x.M_{A} \;|\; M_{M}+M_{N}}
  \and
  \inferrule* [lab=agent] {} {{M_{A}} \bc (\vec{x})M_{P} \;| \; \clift{P_0,\ldots,M_{P},\ldots,P_N}}
  \and \\
  \inferrule* [lab=process] {} {{M_{P}} \bc M_{N} \;| \;P|M_{P} }
\end{mathpar} 

\begin{mathpar}
  \inferrule* [lab=sychronization] {} {M_{N} \bc \Box \;|\; x?M_{F} \;|\; x!M_{C}}
  \and
  \inferrule* [lab=abstraction] {} {{M_{F}} \bc (x)M_{P} }
  \and
  \inferrule* [lab=concretion] {} {{M_{C}} \bc \langle M_{P} \rangle }
  \and \\
  \inferrule* [lab=process] {} {{M_{P}} \bc M_{N} \;| \;P|M_{P} }
\end{mathpar}

\begin{definition}[contextual application] Given a context $M$, and
  process $P$, we define the \emph{contextual application}, $M[P] :=
  M\{P/\Box\}$. That is, the contextual application of M to P is the
  substitution of $P$ for $\Box$ in $M$.
\end{definition}

$\meaningof{-} : L \to \mathcal{P}(\pi)$

\begin{mathpar}
  \inferrule* [lab=collection] {} {\meaningof{true} = \pi, \and \meaningof{~E} = \pi \setminus \meaningof{E}, \and \meaningof{E_{1} \& E_{2}} = \meaningof{E_{1}} \cap \meaningof{E_{2}}}
\end{mathpar}

\begin{mathpar}
  \inferrule* [lab=structure] {} {\meaningof{0} = \{ P \in \pi | P \equiv 0 \}, \and \\ \meaningof{E_1 | E_2} = \{ P \in \pi | P \equiv P_{1} | P_{2}, P_{1} \in \meaningof{E_{1}}, P_{2} \in \meaningof{E_2}\} }
\end{mathpar}

\begin{mathpar}
 \inferrule* [lab=behavior] {} {\meaningof{\langle a?b \rangle E} = \{ P \in \pi | P \equiv Q | u?(y)P', \\ \and \\\\ \and \\ \;\;\; u \in \meaningof{a}, \forall z.P'\{z/y\} \in \meaningof{E\{z/b\}}\}, \and \\ \meaningof{a!E} = \{ P \in \pi | P \equiv Q | x!\langle P' \rangle, x \in \meaningof{a} P' \in \meaningof{E}\} }
\end{mathpar}

\begin{mathpar}
 \inferrule* [lab=nominal] {} {\meaningof{\quotep{E}} = \{ \quotep{P} \in \quotep{\pi} | P \in \meaningof{E} \}, \and \meaningof{\quotep{P}} = \{ \quotep{Q} \in \quotep{\pi} | P \equiv Q \} \and \\ \meaningof{@\quotep{E}} = \{ P \in \pi | P \equiv @x, x \in \meaningof{E} \}}
\end{mathpar}

\begin{eqnarray*}
  \\
  \meaningof{-} : TS \to ST
\end{eqnarray*}

\begin{eqnarray*}
  \\
  L : TS \to ST
\end{eqnarray*}

\begin{eqnarray*}
  \\
  P \models E \iff P \in \meaningof{E}
\end{eqnarray*}

\begin{eqnarray*}
  P \approx_{L} Q \iff \forall E \in L. P \models E \iff Q \models E
\end{eqnarray*}

\begin{eqnarray*}
  P \approx_{K} Q
\end{eqnarray*}

\begin{eqnarray*}
  P \approx Q
\end{eqnarray*}

$\approx_{K} = \approx = \approx_{L}$

\subsubsection{Contextual duality}

Note that contexts extend the quotation operation to a family of
operations from processes to names. Given a context, $M$, we can
define a \emph{nominal context}, $\quotep{M}$ by $\quotep{M}[P] :=
\quotep{M[P]}$. To foreshadow what is to come we observe that these
operations enjoy a duality with processes very much like the duality
between vectors and maps from vectors to scalars.

Further, because the calculus is essentially higher-order, we have a
correspondence between contexts and processes. More specifically,
given a name $x$ and a context $M$ we can construct $M^{*}_{x}$ such
that 

\begin{mathpar}
  M^{*}_{x} | \lift{x}{P} \red M[P]
\end{mathpar}

namely,

\begin{mathpar}
  M^{*}_{x} := x?(u).M[\dropn{u}]
\end{mathpar}

The dependence of $M^{*}_{x}$ on a name makes it an abstraction, 

\begin{mathpar}
  M^{*} := (x)x?(u).M[\dropn{u}]
\end{mathpar}

\subsection{Additional notation}

It will sometimes be convenient to denote the process a name
quotes. We already have the notation $x = \quotep{P}$, but it will be
convenient to introduce an alternate notation, $\procn{x}$, when we
want to emphasize the connection to the use of the name. Note that, by
virtue of name equivalence, $\quotep{\procn{x}} \nameeq x$; so, the
notation is consistent with previous definitions.

Further, because names have structure it is possible to effect
substitutions on the basis of that structure. This means we need to
upgrade our notation for substitutions, which we accomplish by
adapting comprehension notation. Thus,

\begin{mathpar}
  P\{ y / x : x \in S \}
\end{mathpar}

is interpreted to mean the process derived from P by replacing (in a
capture-avoiding manner) each occurrence of $x$ in $S$ by $y$. For example,

\begin{mathpar}
  P\{ \quotep{\procn{x}|\procn{x}} / x : x \in \freenames{P} \}
\end{mathpar}

will replace each (occurrence) of a free name $x$ in $P$ by
$\quotep{\procn{x}|\procn{x}}$.

Also, we will avail ourselves of the notation $x^{L}$ and $x^{R}$ to
denote injections of a name into disjoint copies of the name
space. There are numerous ways to accomplish this. One example can be
found in \cite{MeredithR05}. This notation overloads to vectors of
names: $\vec{x}^{\pi} := (x_{i}^{\pi} \; : \; 0 \leq i < |\vec{x}| )$ where $\pi \in \{L,R\}$.

We also use $P^{\Box} := P|\Box$.

In \cite{MeredithR05} an interpretation of the new operator is
given. It turns out that there are several possible interpretations
all enjoying the requisite algebraic properties of the operator (see
\cite{milner91polyadicpi}). We will therefore make liberal use of
$(\nu\; \vec{x})P$.

% subsection the_syntax_and_semantics_of_the_notation_system (end)   

\input{qm2pi.qmops} 

\input{qm2pi.sterngerlach} 

\input{qm2pi.metric} 

% section concurrent_process_calculi (end)

%\input{qm2pi.proofsketch}

% section proof sketch (end)

%\input{qm2pi.slviaknots} 

% section spatial logic via knots (end)

\input{qm2pi.conclusion}

% section conclusion (end)

%\input{qm2pi.dtcodes} 

% section wiring algorithm (end)

\input{qm2pi.ack} 

% section acknowledgments (end)

\newpage


\bibliographystyle{plain}   
\bibliography{../../biblios/main.bib}

\input{qm2pi.rhodetails}

\end{document}

 

%\ifpdf
%\usepackage[pdftex]{graphicx}
%\else
%\usepackage{graphicx}
%\fi

 % \ifpdf
%  \usepackage{pdfsync}
%  \if


%\title{Brief Article}
%\author{David F. Snyder}
%\author{L.G. Meredith}

%\address{Dept. of Math., Texas State University--San Marcos, San Marcos, TX 78666}
       
\pagestyle{empty}


\begin{document}

\lstset{language=[Objective]Caml,frame=shadowbox}

\documentclass[12pt]{llncs}
%\documentclass{jktr}

\usepackage[pdftex]{hyperref}                   
\usepackage {listings}
\usepackage {mathpartir}
\usepackage{bcprules}
%\usepackage{listings}
                       
\usepackage{graphicx} 
%\usepackage[margins=2.5cm,nohead,nofoot]{geometry}
%\usepackage{geometry}
\usepackage{amsfonts}
\usepackage{amstext}
\usepackage{latexsym}
\usepackage{amssymb}
\usepackage{color}


%\include{myPreamble}
\include{qm2pi.local} 

%\ifpdf
%\usepackage[pdftex]{graphicx}
%\else
%\usepackage{graphicx}
%\fi

 % \ifpdf
%  \usepackage{pdfsync}
%  \if


%\title{Brief Article}
%\author{David F. Snyder}
%\author{L.G. Meredith}

%\address{Dept. of Math., Texas State University--San Marcos, San Marcos, TX 78666}
       
\pagestyle{empty}


\begin{document}

\lstset{language=[Objective]Caml,frame=shadowbox}

\input{qm2pi.front}

% section front matter (end)

\input{qm2pi.intro} 
 
% section introduction (end)

% \input{qm2pi.knotations} 

% section notation (end)

\input{qm2pi.process.calculi} 

% section concurrent_process_calculi_and_spatial_logics_ (end)
    
%\input{qm2pi.knots2pi} 

%\input{qm2pi.trefoil} 

%\input{qm2pi.mainthm} 

% subsection basic_interpretation (end)

%\input{qm2pi.rho.presentation} 
\subsection{The syntax and semantics of the notation system}\label{sub:the_syntax_and_semantics_of_the_notation_system} % (fold)

We now summarize a technical presentation of the calculus that
embodies our theory of dynamics. The typical presentation of such a
calculus follows the style of giving generators and relations on
them. The grammar, below, describing term constructors, freely
generates the set of processes, $\Proc$. This set is then quotiented
by a relation known as structural congruence and it is over this set
that the notion of dynamics is expressed. This presentation is
essentially that of \cite{MeredithR05} with the addition of
polyadicity and summation. For readability we have relegated some of
the technical subtleties to an appendix.

\subsubsection{Process grammar}\label{subsub:process_grammar}

\begin{mathpar}
  \inferrule* [lab=synchronization] {} {{M} \bc \pzero \;|\; x?F \;|\; x!C }
  \and
  \inferrule* [lab=abstraction] {} {{F} \bc (x)P}
  \and
  \inferrule* [lab=concretion] {} {{C} \bc \langle Q \rangle}
  \and
  \inferrule* [lab=process] {} {{P,Q} \bc M \;| \;P|Q \;|\; @{x}}
  \and
  \inferrule* [lab=name] {} {{x} \bc \quotep{P}}
\end{mathpar} 

Note that $\vec{x}$ (resp. $\vec{P}$) denotes a vector of names
(resp. processes) of length $|\vec{x}|$ (resp. $|\vec{P}|$). We adopt
the following useful abbreviations.

\begin{mathpar}
   x?(\vec{y}).P := x.(\vec{y})P \and  x\clift{\vec{P}} := x.\clift{\vec{P}}
   \and x!(y) := \lift{x}{\dropn{y}}
   \and \Pi_{i=0}^{n-1}P_i := P_0 | \ldots | P_{n-1}
\end{mathpar}

\subsubsection{Structural congruence}

\paragraph{Free and bound names and alpha-equivalence.} At the
core of structural equivalence is alpha-equivalence which identifies
process that are the same up to a change of variable. Formally, we
recognize the distinction between free and bound names. The free names
of a process, $\freenames{P}$, may be calculated recursively as
follows:

\begin{mathpar}
\freenames{\pzero} := \emptyset
  \and \\
  \freenames{x?(y).P} := \{ x \} \cup (\freenames{P} \setminus \{ y \})
  \and 
  \freenames{x!\langle P \rangle} := \{ x \} \cup \{ P \} 
  \and \\
  \freenames{P|Q} := \freenames{P} \cup \freenames{Q}
  \and \\
  \freenames{@{x}} := \{ x \}
\end{mathpar}

$\pi$
$\quotep{\pi}$

$\freenames{-} : \pi \to \mathcal{P}(\quotep{\pi})$

\begin{eqnarray*}
  \freenames{\pzero} & := & \emptyset \\
  \freenames{x?(y).P} & := & \{ x \} \cup (\freenames{P} \setminus \{ y \}) \\
  \freenames{x!\langle P \rangle} & := & \{ x \} \cup \{ P \} \\
  \freenames{P|Q} & := & \freenames{P} \cup \freenames{Q} \\
  \freenames{\dropn{x}} & := & \{ x \}
\end{eqnarray*}

The bound names of a process, $\boundnames{P}$, are those names occurring in $P$
that are not free. For example, in $x?(y).0$, the name $x$ is free, while $y$ is bound.

\begin{mathpar}
  \inferrule* [lab=monoidal-laws] {} { P|Q \equiv Q|P \and P|0 \equiv P \and P|(Q|R) \equiv (P|Q)|R }
\end{mathpar}

\begin{mathpar}
  \inferrule* [lab=alpha-equivalence] {} { (x)P \equiv (y)P\{y/x\} \and y \not\in \freenames{P} }
\end{mathpar}

\begin{definition}
Then two processes, $P,Q$, are alpha-equivalent if $P = Q\{\vec{y}/\vec{x}\}$ for
some $\vec{x} \in \boundnames{Q},\vec{y} \in \boundnames{P}$, where $Q\{\vec{y}/\vec{x}\}$
denotes the capture-avoiding substitution of $\vec{y}$ for $\vec{x}$ in $Q$.
\end{definition}

\begin{definition}
  The {\em structural congruence} \cite{SangiorgiWalker} , $\equiv$,
  between processes is the least congruence containing
  alpha-equivalence, satisfying the abelian monoid laws
  (associativity, commutativity and $\pzero$ as identity) for parallel
  composition $|$ and for summation $+$.
\end{definition}

\subsection{Name equivalence}

We take name equivalence, written $\nameeq$, to be the smallest
equivalence relation generated by the following rules.

\begin{mathpar}
\inferrule*[lab=Quote-drop]
{ }
{ \quotep{@{x}} \nameeq x }

\inferrule*[lab=Struct-equiv]
{ P \scong Q }
{ \quotep{P} \nameeq \quotep{Q} }
\end{mathpar}

The astute reader will have noticed that the mutual recursion of names
and processes imposes a mutual recursion on alpha-equivalence and
structural equivalence via name-equivalence. Fortunately, all of this
works out pleasantly and we may calculate in the natural way, free of
concern. The reader interested in the details is referred to the
appendix \ref{appendix:rho_details}.

\subsection{Substitution}

We use $\Proc$ for the set of processes, $\QProc$ for the set of
names, and $\id{\{}\vec{y} / \vec{x} \id{\}}$ to denote partial maps,
$s : \QProc \rightarrow \QProc$. A map, $s$ lifts, uniquely, to a map
on process terms, $\widehat{s} : \Proc \rightarrow \Proc$ by the
following equations.

\begin{mathpar}
  (0) \psubstp{Q}{P} := 0 \\
  (R \juxtap S) \psubstp{Q}{P}
  :=    
  (R)\psubstp{Q}{P} \juxtap (S) \psubstp{Q}{P} \\
  (x?(y).R) \psubstp{Q}{P}    
  :=    
  (x)\substp{Q}{P} (z)\concat( (R \psubstn{z}{y}) \psubstp{Q}{P} ) \\
  (\lift{x}{R}) \psubstp{Q}{P}  
  :=
  \lift{(x)\substp{Q}{P}}{ R \psubstp{Q}{P} } \\
%   (\dropn{x})  \psubstp{Q}{P}       
%   := 
%   \left\{ 
%     \begin{array}{ccc} 
%       \dropn{\quotep{Q}} & & x \nameeq \quotep{P} \\
%       \dropn{x} & & otherwise \\
%     \end{array}
%   \right. 
  (\dropn{x})  \psubstp{Q}{P}       
  := 
  \left\{ 
    \begin{array}{ccc} 
      Q & & x \nameeq \quotep{P} \\
      \dropn{x} & & otherwise \\
    \end{array}
  \right.
\end{mathpar}
 

where

\begin{eqnarray}
  (x)\id{\{} \lpquote Q \rpquote / \lpquote P \rpquote \id{\}}            = 
  \left\{ 
    \begin{array}{ccc}
      \lpquote Q \rpquote & & x \nameeq \lpquote P \rpquote \\
      x & & otherwise \\
    \end{array}
  \right. \nonumber
\end{eqnarray}

and $z$ is chosen distinct from $\quotep{P}$, $\quotep{Q}$, the free
names in $Q$, and all the names in $R$. Our $\alpha$-equivalence will
be built in the standard way from this substitution.

\begin{remark}\label{rem:no_self_referential_names}
  One consequence of these definitions is that $\forall P. \quotep{P}
  \not\in \freenames{P}$.
\end{remark}

\subsection{ Dynamic quote: an example }

Anticipating something of what's to come, consider applying the
substitution, $\widehat{\id{\{}u / z \id{\}}}$, to the following pair
of processes, $\lift{w}{y!(z)}$ and $w[ \lpquote y!(z) \rpquote ]$.

\begin{eqnarray}
	\lift{w}{y!(z)}\widehat{\id{\{}u / z \id{\}}}
		& = &
		\lift{w}{y!(u)} \nonumber\\
	w[ \lpquote y!(z) \rpquote ] \widehat{ \id{\{}u / z \id{\}} }
		& = &
		w[ \lpquote y!(z) \rpquote ] \nonumber
\end{eqnarray}

Because the body of the process between quotes is impervious to
substitution, we get radically different answers. In fact, by
examining the first process in an input context,
e.g. $x?(z).\lift{w}{y!(z)}$, we see that the process under the lift
operator may be shaped by prefixed inputs binding a name inside it. In
this sense, the lift operator will be seen as a way to dynamically
construct processes before reifying them as names.

Finally equipped with these standard features we can present the
dynamics of the calculus.

\subsubsection{Operational semantics} 

Finally, we introduce the computational dynamics. What marks these
algebras as distinct from other more traditionally studied algebraic
structures, e.g. vector spaces or polynomial rings, is the manner in
which dynamics is captured. In traditional structures, dynamics is typically
expressed through morphisms between such structures, as in linear maps
between vector spaces or morphisms between rings. In algebras
associated with the semantics of computation, the dynamics is
expressed as part of the algebraic structure itself, through a
reduction reduction relation typically denoted by $\red$. Below, we
give a recursive presentation of this relation for the calculus used
in the encoding.

$\red \subseteq \pi \times \pi$
$\red : \pi \to \mathcal{P}(\pi)$

\begin{mathpar}
  \inferrule* [lab=Comm] { \textsf{match}( x_{src}, x_{trgt} ) } { x_{trgt}?(y)P \; | \; x_{src}!\langle {Q} \rangle \red P\{\quotep{Q}/y}\} }
  \and \\
  \inferrule* [lab=Par] {{P} \red {P}'} {{{P} | {Q}} \red {{P}' | {Q}}}
  \and
  \inferrule* [lab=Equiv]{{{P} \scong {P}'} \andalso {{P}' \red {Q}'} \andalso {{Q}' \scong {Q}}}{{P} \red {Q}}
\end{mathpar}

\begin{eqnarray*}
  match_{\equiv} (\quotep{P},\quotep{Q}) & := & P \equiv Q \\
  match_{\dagger}(\quotep{P},\quotep{Q}) & := & \forall R. P|Q \red^{*} R => R \red^{*} 0 \\
  match_{K}(\quotep{P},\quotep{Q}) & := & K \mbox{ for some context } K
\end{eqnarray*}

$u?(x)P | u!\langle Q \rangle \red P\{\quotep{Q}/x\}$

%We write $\wred$ for $\red^*$, and $P\red$ if $\exists Q $ such that $ P \red Q$.
We write $P\red$ if $\exists Q $ such that $ P \red Q$ and $P\not\red$, otherwise.

\section{Replication}

As mentioned before, it is known that replication (and hence
recursion) can be implemented in a higher-order process algebra
\cite{SangiorgiWalker}. As our first example of calculation with the
machinery thus far presented we give the construction explicitly in
the {\rhoc}.

\begin{eqnarray}
	D_{x} & := & \prefix{x}{y}{(\binpar{\outputp{x}{y}}{@{y}})} \nonumber\\
	\bangp_{x}{P} & := & \binpar{{x}!\langle{\binpar{D_{x}}{P}}\rangle}{D_{x}} \nonumber
\end{eqnarray}

\begin{eqnarray}
	\bangp_{x}{P} & & \nonumber\\
	=
	& {x}!\langle{(\prefix{x}{y}{(\outputp{x}{y} | @{y})) | P}}\rangle 
	      | \prefix{x}{y}{(\outputp{x}{y} | @{y})} & \nonumber\\
	\red
	& (\outputp{x}{y} | @{y})\substn{\quotep{(\prefix{x}{y}{(@{y} | \outputp{x}{y})) | P}}}{y} & \nonumber\\
	=
	& \outputp{x}{\quotep{(\prefix{x}{y}{(\outputp{x}{y} | @{y})) | P}}}
	  | {(\prefix{x}{y}{(\outputp{x}{y} | @{y})) | P}} & \nonumber\\
	\red
	& \ldots & \nonumber\\
	\red^*
	& P | P | \ldots & \nonumber
\end{eqnarray}

Of course, this encoding, as an implementation, runs away, unfolding
$\bangp{P}$ eagerly. A lazier and more implementable replication
operator, restricted to input-guarded processes, may be obtained as follows.

\begin{eqnarray}
\bangp{\prefix{u}{v}{P}} 
	:= 
	\binpar{\lift{x}{\prefix{u}{v}{(\binpar{D(x)}{P})}}}{D(x)} \nonumber
\end{eqnarray}

\begin{remark}
  Note that the lazier definition still does not deal with summation
  or mixed summation (i.e. sums over input and output). The reader is
  invited to construct definitions of replication that deal with these
  features. 

  Further, the definitions are parameterized in a name, $x$. Can you,
  gentle reader, make a definition that eliminates this parameter and
  guarantees no accidental interaction between the replication
  machinery and the process being replicated -- i.e. no accidental
  sharing of names used by the process to get its work done and the
  name(s) used by the replication to effect copying. This latter
  revision of the definition of replication is crucial to obtaining
  the expected identity $!!P \sim !P$.
\end{remark}

\begin{remark}\label{rem:paradoxical_combinator}
  The reader familiar with the lambda calculus will have noticed the
  similarity between $D$ and the paradoxical combinator.

  [Ed. note: the existence of this seems to suggest we have to be more
  restrictive on the set of processes and names we admit if we are to
  support no-cloning.]
\end{remark}

\subsubsection{Bisimulation}

The computational dynamics gives rise to another kind of equivalence,
the equivalence of computational behavior. As previously mentioned
this is typically captured \emph{via} some form of bisimulation.

% The notion we use in this paper is weak barbed bisimulation
% \cite{milner91polyadicpi}.

The notion we use in this paper is derived from weak barbed
bisimulation \cite{milner91polyadicpi}. 

\begin{definition}
An \emph{observation relation}, $\downarrow_{\mathcal N}$, over a set
of names, $\mathcal N$, is the smallest relation satisfying the rules
below.

\infrule[Out-barb]{y \in {\mathcal N}, \; x \nameeq y}
		  {\outputp{x}{v} \downarrow_{\mathcal N} x}
\infrule[Par-barb]{\mbox{$P\downarrow_{\mathcal N} x$ or $Q\downarrow_{\mathcal N} x$}}
		  {\binpar{P}{Q} \downarrow_{\mathcal N} x}

We write $P \Downarrow_{\mathcal N} x$ if there is $Q$ such that 
$P \wred Q$ and $Q \downarrow_{\mathcal N} x$.
\end{definition}

\begin{definition}
%\label{def.bbisim}
An  ${\mathcal N}$-\emph{barbed bisimulation} over a set of names, ${\mathcal N}$, is a symmetric binary relation 
${\mathcal S}_{\mathcal N}$ between agents such that $P\rel{S}_{\mathcal N}Q$ implies:
\begin{enumerate}
\item If $P \red P'$ then $Q \wred Q'$ and $P'\rel{S}_{\mathcal N} Q'$.
\item If $P\downarrow_{\mathcal N} x$, then $Q\Downarrow_{\mathcal N} x$.
\end{enumerate}
$P$ is ${\mathcal N}$-barbed bisimilar to $Q$, written
$P \wbbisim_{\mathcal N} Q$, if $P \rel{S}_{\mathcal N} Q$ for some ${\mathcal N}$-barbed bisimulation ${\mathcal S}_{\mathcal N}$.
\end{definition}

$\mathcal{R} \subseteq \pi \times \pi$

$P \mathcal{R} Q => \forall P'. P \red P' \Rightarrow \exists Q'. Q \red Q', P' \mathcal{R} Q'$

$P \vdash x \Rightarrow Q \vdash x$

\begin{mathpar}
  \inferrule*[lab=Out-barb]{x \nameeq y}{{y}!\langle{Q}\rangle \vdash x}
  \and
  \inferrule*[lab=Par-barb]{\mbox{$P\vdash x$ or $Q\vdash x$}}{\binpar{P}{Q} \vdash x}
\end{mathpar}

\subsubsection{Contexts}

One of the principle advantages of computational calculi like the
$\pi$-calculus is a well-defined notion of context,
contextual-equivalence and a correlation between
contextual-equivalence and notions of bisimulation. The notion of
context allows the decomposition of a process into (sub-)process and
its syntactic environment, its context. Thus, a context may be
thought of as a process with a ``hole'' (written $\Box$) in it. The
application of a context $M$ to a process $P$, written $M[P]$, is
tantamount to filling the hole in $M$ with $P$. In this paper we do
not need the full weight of this theory, but do make use of the notion
of context in the proof the main theorem. 

\begin{mathpar}
  \inferrule* [lab=summation] {} {{M_{M},M_{N}} \bc \Box \;|\; x.M_{A} \;|\; M_{M}+M_{N}}
  \and
  \inferrule* [lab=agent] {} {{M_{A}} \bc (\vec{x})M_{P} \;| \; \clift{P_0,\ldots,M_{P},\ldots,P_N}}
  \and \\
  \inferrule* [lab=process] {} {{M_{P}} \bc M_{N} \;| \;P|M_{P} }
\end{mathpar} 

\begin{mathpar}
  \inferrule* [lab=sychronization] {} {M_{N} \bc \Box \;|\; x?M_{F} \;|\; x!M_{C}}
  \and
  \inferrule* [lab=abstraction] {} {{M_{F}} \bc (x)M_{P} }
  \and
  \inferrule* [lab=concretion] {} {{M_{C}} \bc \langle M_{P} \rangle }
  \and \\
  \inferrule* [lab=process] {} {{M_{P}} \bc M_{N} \;| \;P|M_{P} }
\end{mathpar}

\begin{definition}[contextual application] Given a context $M$, and
  process $P$, we define the \emph{contextual application}, $M[P] :=
  M\{P/\Box\}$. That is, the contextual application of M to P is the
  substitution of $P$ for $\Box$ in $M$.
\end{definition}

$\meaningof{-} : L \to \mathcal{P}(\pi)$

\begin{mathpar}
  \inferrule* [lab=collection] {} {\meaningof{true} = \pi, \and \meaningof{~E} = \pi \setminus \meaningof{E}, \and \meaningof{E_{1} \& E_{2}} = \meaningof{E_{1}} \cap \meaningof{E_{2}}}
\end{mathpar}

\begin{mathpar}
  \inferrule* [lab=structure] {} {\meaningof{0} = \{ P \in \pi | P \equiv 0 \}, \and \\ \meaningof{E_1 | E_2} = \{ P \in \pi | P \equiv P_{1} | P_{2}, P_{1} \in \meaningof{E_{1}}, P_{2} \in \meaningof{E_2}\} }
\end{mathpar}

\begin{mathpar}
 \inferrule* [lab=behavior] {} {\meaningof{\langle a?b \rangle E} = \{ P \in \pi | P \equiv Q | u?(y)P', \\ \and \\\\ \and \\ \;\;\; u \in \meaningof{a}, \forall z.P'\{z/y\} \in \meaningof{E\{z/b\}}\}, \and \\ \meaningof{a!E} = \{ P \in \pi | P \equiv Q | x!\langle P' \rangle, x \in \meaningof{a} P' \in \meaningof{E}\} }
\end{mathpar}

\begin{mathpar}
 \inferrule* [lab=nominal] {} {\meaningof{\quotep{E}} = \{ \quotep{P} \in \quotep{\pi} | P \in \meaningof{E} \}, \and \meaningof{\quotep{P}} = \{ \quotep{Q} \in \quotep{\pi} | P \equiv Q \} \and \\ \meaningof{@\quotep{E}} = \{ P \in \pi | P \equiv @x, x \in \meaningof{E} \}}
\end{mathpar}

\begin{eqnarray*}
  \\
  \meaningof{-} : TS \to ST
\end{eqnarray*}

\begin{eqnarray*}
  \\
  L : TS \to ST
\end{eqnarray*}

\begin{eqnarray*}
  \\
  P \models E \iff P \in \meaningof{E}
\end{eqnarray*}

\begin{eqnarray*}
  P \approx_{L} Q \iff \forall E \in L. P \models E \iff Q \models E
\end{eqnarray*}

\begin{eqnarray*}
  P \approx_{K} Q
\end{eqnarray*}

\begin{eqnarray*}
  P \approx Q
\end{eqnarray*}

$\approx_{K} = \approx = \approx_{L}$

\subsubsection{Contextual duality}

Note that contexts extend the quotation operation to a family of
operations from processes to names. Given a context, $M$, we can
define a \emph{nominal context}, $\quotep{M}$ by $\quotep{M}[P] :=
\quotep{M[P]}$. To foreshadow what is to come we observe that these
operations enjoy a duality with processes very much like the duality
between vectors and maps from vectors to scalars.

Further, because the calculus is essentially higher-order, we have a
correspondence between contexts and processes. More specifically,
given a name $x$ and a context $M$ we can construct $M^{*}_{x}$ such
that 

\begin{mathpar}
  M^{*}_{x} | \lift{x}{P} \red M[P]
\end{mathpar}

namely,

\begin{mathpar}
  M^{*}_{x} := x?(u).M[\dropn{u}]
\end{mathpar}

The dependence of $M^{*}_{x}$ on a name makes it an abstraction, 

\begin{mathpar}
  M^{*} := (x)x?(u).M[\dropn{u}]
\end{mathpar}

\subsection{Additional notation}

It will sometimes be convenient to denote the process a name
quotes. We already have the notation $x = \quotep{P}$, but it will be
convenient to introduce an alternate notation, $\procn{x}$, when we
want to emphasize the connection to the use of the name. Note that, by
virtue of name equivalence, $\quotep{\procn{x}} \nameeq x$; so, the
notation is consistent with previous definitions.

Further, because names have structure it is possible to effect
substitutions on the basis of that structure. This means we need to
upgrade our notation for substitutions, which we accomplish by
adapting comprehension notation. Thus,

\begin{mathpar}
  P\{ y / x : x \in S \}
\end{mathpar}

is interpreted to mean the process derived from P by replacing (in a
capture-avoiding manner) each occurrence of $x$ in $S$ by $y$. For example,

\begin{mathpar}
  P\{ \quotep{\procn{x}|\procn{x}} / x : x \in \freenames{P} \}
\end{mathpar}

will replace each (occurrence) of a free name $x$ in $P$ by
$\quotep{\procn{x}|\procn{x}}$.

Also, we will avail ourselves of the notation $x^{L}$ and $x^{R}$ to
denote injections of a name into disjoint copies of the name
space. There are numerous ways to accomplish this. One example can be
found in \cite{MeredithR05}. This notation overloads to vectors of
names: $\vec{x}^{\pi} := (x_{i}^{\pi} \; : \; 0 \leq i < |\vec{x}| )$ where $\pi \in \{L,R\}$.

We also use $P^{\Box} := P|\Box$.

In \cite{MeredithR05} an interpretation of the new operator is
given. It turns out that there are several possible interpretations
all enjoying the requisite algebraic properties of the operator (see
\cite{milner91polyadicpi}). We will therefore make liberal use of
$(\nu\; \vec{x})P$.

% subsection the_syntax_and_semantics_of_the_notation_system (end)   

\input{qm2pi.qmops} 

\input{qm2pi.sterngerlach} 

\input{qm2pi.metric} 

% section concurrent_process_calculi (end)

%\input{qm2pi.proofsketch}

% section proof sketch (end)

%\input{qm2pi.slviaknots} 

% section spatial logic via knots (end)

\input{qm2pi.conclusion}

% section conclusion (end)

%\input{qm2pi.dtcodes} 

% section wiring algorithm (end)

\input{qm2pi.ack} 

% section acknowledgments (end)

\newpage


\bibliographystyle{plain}   
\bibliography{../../biblios/main.bib}

\input{qm2pi.rhodetails}

\end{document}



% section front matter (end)

\section{Introduction}\label{sec:introduction} % (fold)
In this draft of the material i am going to have to dispense with the
usual writing conventions adopted in papers on these topics. i'm going
to have adopt whatever tone i need at the time i'm writing up the
calculations. Sometimes this may be very conversational; others it may
be the barest mathematical grunts; others still it may be that i have
lifted text from one of my other papers because the exposition of some
point was better said there. i hope that my readers are not unduly put
out by this decision. i'm not doing this to flout convention or be
rebellious. i find these calculations very technically challenging. To
keep everything going technically, something has to give; i have to
let go of some cognitive burden. So, the academic writing style --
with all of its trade-offs in terms of facilitating technical
communication -- is what i'm letting go of. Perhaps subsequent drafts
can be tightened and polished, but for now, i'm going to speak as if
we were sitting together in a coffee shop with a laptop, wifi and a
pad of paper and a pencil.

So, here's what i have to say. We -- you and i, comfortably ensconced
in our coffee shop and well-equipped with our tools -- can realize and
carry out the calculations of quantum mechanics over a very different
formal theory of dynamics, a formal theory of dynamics that
corresponds to a theory of concurrent computation with
\emph{reflection}. It has the advantage that the underlying theory is
already `quantized', but supports analogues all of the continuuous
operations. Strikingly, this underlying theory has recently been
connected with a notion of metric that we can show, by calculating
together, coincides with the metric induced by the inner product.

There are a lot of reasons why you might be interested in seeing
calculations of this form. Here's why i'm interested. For the past
several centuries there has been no competitor to the ``Newtonian''
account of dynamics. As a result the predominant share of accounts of
dynamical systems and situations have had to be formulated in terms of
the Newtonian machinery. i view this as an intellectually dangerous
position to occupy. Everything, despite it's intrinsic shape, turns
into a nail to be hit with this hammer. Recently, however, the theory
of computation has matured to the point where we have candidates for
theories of dynamics that offer very different perspective on
reasoning about dynamical systems and situations. Testing these
candidates against very successful accounts of dynamical situations,
like quantum mechanics, is going to give us some sense of how mature
they are and some measure of the quality of these accounts of
dynamics.

\subsection{Summary of contributions and outline of paper}

So, we're going to develop an interpretation of the operations of
quantum mechanics normally interpreted by Hilbert spaces and
operators. We're going to do this over a theory of computation. Note
that this is very different than the usual quantum computation program
which develops notions of computation over quantum mechanics. Rather,
we are developing a story that aligns with Wheeler's slogan: It from
Bit. To do this we will first provide an account of the theory of
computation at play here. Then we will dive into a calculation-driven
interpretation of the operations of quantum mechanics.

The reason we take this approach is that -- until very recently --
there hasn't been an axiomatic account of quantum mechanics. As a
result there has been no sharp delineation of the mathematical theory
supporting interpretation of the physical theory and the physical
theory, itself. So, ambient features of the maths are free to be
exploited (or supressed) without a real accounting of their physical
relevance. There is no sharp statement ``here's the physical theory''
qua \emph{theory} and ``here's the mathematical interpretation''
enabling a judgment of how faithful the interpretation is -- apart
from experimental observation. When there is an axiomatic account we
can judge how well a given mathematical formalism supports an
interpretation of the axioms, independent of
experimentation. Likewise, we can judge how well we have captured our
physical evidence and experience with our axiomatics, independent of
any specific mathematical implementation, with accidental detail that
may or may not have physical significance. 

In lieu of a fully fleshed out and vetted axiomatic account of quantum
mechanics, interpreting the operational notions in service of modeling
physical systems will have to suffice. In other words, we are not in
the business of providing a model of Hilbert spaces and operators. We
are in the business of providing a model of quantum mechanics because
we are motivated by testing our notions of dynamics against physical
theory; and, the predictive calculations of the physical theory must
serve as the best formulation -- shy of a fully fleshed out axiomatic
account -- of the physical theory itself (as they have for scientific
theories since time immemorial). Put another way, despite a
whole-hearted commitment to an It-from-Bit ontology, we are firmly
aligned with the shut-up-and-calculate camp as the best way to obtain
results either from the physical perspective or as a quality assurance
measure of our fledgling theory of dynamics.

In detail, we present a reflective process calculus. Then we develop
intuitive correspondences between the notions available in this
calculus and the usual physical notions supporting quantum mechanical
calculations. Thus, 

\begin{table}[htp]
  \center{
    \fbox{
      \begin{tabular}{c|c}
        quantum mechanics & process calculus \\
        \hline
        scalar & name \\
        state vector & process \\
        dual & contextual duals \\
        matrix & formal sums of process-context-dual pairs \\
        orthogonality & process annihilation \\
        inner product & execution-formula + quoting
      \end{tabular}
    }
  }
  \caption{QM - process calculi correspondences}
\end{table}

Then we tighten up these intuitions to operational definitions. We
employ the Dirac notation as the best proxy we can find for an
abstract syntax of the quantum mechanical notions. The definitions we
develop put us in contact with equational constraints coming from the
theory that we demonstrate the definitions and calculations satisfy.

This puts us in a position to shut up and calculate for the
Stern-Gerlach experimental set up, showing how these predictive
calculations become calculations on processes in our theory of a
reflective process calculus.

Penultimately, we demonstrate that the notion of metric coming from
the inner product coincides with the notion of metric available from
the theory of bisimulation. This demonstration gives us the right to
think of space as arising from behavior. Finally, we consider where we
might go from the new vantage point we have obtained.

% section introduction (end) 
 
% section introduction (end)

% \documentclass[12pt]{llncs}
%\documentclass{jktr}

\usepackage[pdftex]{hyperref}                   
\usepackage {listings}
\usepackage {mathpartir}
\usepackage{bcprules}
%\usepackage{listings}
                       
\usepackage{graphicx} 
%\usepackage[margins=2.5cm,nohead,nofoot]{geometry}
%\usepackage{geometry}
\usepackage{amsfonts}
\usepackage{amstext}
\usepackage{latexsym}
\usepackage{amssymb}
\usepackage{color}


%\include{myPreamble}
\include{qm2pi.local} 

%\ifpdf
%\usepackage[pdftex]{graphicx}
%\else
%\usepackage{graphicx}
%\fi

 % \ifpdf
%  \usepackage{pdfsync}
%  \if


%\title{Brief Article}
%\author{David F. Snyder}
%\author{L.G. Meredith}

%\address{Dept. of Math., Texas State University--San Marcos, San Marcos, TX 78666}
       
\pagestyle{empty}


\begin{document}

\lstset{language=[Objective]Caml,frame=shadowbox}

\input{qm2pi.front}

% section front matter (end)

\input{qm2pi.intro} 
 
% section introduction (end)

% \input{qm2pi.knotations} 

% section notation (end)

\input{qm2pi.process.calculi} 

% section concurrent_process_calculi_and_spatial_logics_ (end)
    
%\input{qm2pi.knots2pi} 

%\input{qm2pi.trefoil} 

%\input{qm2pi.mainthm} 

% subsection basic_interpretation (end)

%\input{qm2pi.rho.presentation} 
\subsection{The syntax and semantics of the notation system}\label{sub:the_syntax_and_semantics_of_the_notation_system} % (fold)

We now summarize a technical presentation of the calculus that
embodies our theory of dynamics. The typical presentation of such a
calculus follows the style of giving generators and relations on
them. The grammar, below, describing term constructors, freely
generates the set of processes, $\Proc$. This set is then quotiented
by a relation known as structural congruence and it is over this set
that the notion of dynamics is expressed. This presentation is
essentially that of \cite{MeredithR05} with the addition of
polyadicity and summation. For readability we have relegated some of
the technical subtleties to an appendix.

\subsubsection{Process grammar}\label{subsub:process_grammar}

\begin{mathpar}
  \inferrule* [lab=synchronization] {} {{M} \bc \pzero \;|\; x?F \;|\; x!C }
  \and
  \inferrule* [lab=abstraction] {} {{F} \bc (x)P}
  \and
  \inferrule* [lab=concretion] {} {{C} \bc \langle Q \rangle}
  \and
  \inferrule* [lab=process] {} {{P,Q} \bc M \;| \;P|Q \;|\; @{x}}
  \and
  \inferrule* [lab=name] {} {{x} \bc \quotep{P}}
\end{mathpar} 

Note that $\vec{x}$ (resp. $\vec{P}$) denotes a vector of names
(resp. processes) of length $|\vec{x}|$ (resp. $|\vec{P}|$). We adopt
the following useful abbreviations.

\begin{mathpar}
   x?(\vec{y}).P := x.(\vec{y})P \and  x\clift{\vec{P}} := x.\clift{\vec{P}}
   \and x!(y) := \lift{x}{\dropn{y}}
   \and \Pi_{i=0}^{n-1}P_i := P_0 | \ldots | P_{n-1}
\end{mathpar}

\subsubsection{Structural congruence}

\paragraph{Free and bound names and alpha-equivalence.} At the
core of structural equivalence is alpha-equivalence which identifies
process that are the same up to a change of variable. Formally, we
recognize the distinction between free and bound names. The free names
of a process, $\freenames{P}$, may be calculated recursively as
follows:

\begin{mathpar}
\freenames{\pzero} := \emptyset
  \and \\
  \freenames{x?(y).P} := \{ x \} \cup (\freenames{P} \setminus \{ y \})
  \and 
  \freenames{x!\langle P \rangle} := \{ x \} \cup \{ P \} 
  \and \\
  \freenames{P|Q} := \freenames{P} \cup \freenames{Q}
  \and \\
  \freenames{@{x}} := \{ x \}
\end{mathpar}

$\pi$
$\quotep{\pi}$

$\freenames{-} : \pi \to \mathcal{P}(\quotep{\pi})$

\begin{eqnarray*}
  \freenames{\pzero} & := & \emptyset \\
  \freenames{x?(y).P} & := & \{ x \} \cup (\freenames{P} \setminus \{ y \}) \\
  \freenames{x!\langle P \rangle} & := & \{ x \} \cup \{ P \} \\
  \freenames{P|Q} & := & \freenames{P} \cup \freenames{Q} \\
  \freenames{\dropn{x}} & := & \{ x \}
\end{eqnarray*}

The bound names of a process, $\boundnames{P}$, are those names occurring in $P$
that are not free. For example, in $x?(y).0$, the name $x$ is free, while $y$ is bound.

\begin{mathpar}
  \inferrule* [lab=monoidal-laws] {} { P|Q \equiv Q|P \and P|0 \equiv P \and P|(Q|R) \equiv (P|Q)|R }
\end{mathpar}

\begin{mathpar}
  \inferrule* [lab=alpha-equivalence] {} { (x)P \equiv (y)P\{y/x\} \and y \not\in \freenames{P} }
\end{mathpar}

\begin{definition}
Then two processes, $P,Q$, are alpha-equivalent if $P = Q\{\vec{y}/\vec{x}\}$ for
some $\vec{x} \in \boundnames{Q},\vec{y} \in \boundnames{P}$, where $Q\{\vec{y}/\vec{x}\}$
denotes the capture-avoiding substitution of $\vec{y}$ for $\vec{x}$ in $Q$.
\end{definition}

\begin{definition}
  The {\em structural congruence} \cite{SangiorgiWalker} , $\equiv$,
  between processes is the least congruence containing
  alpha-equivalence, satisfying the abelian monoid laws
  (associativity, commutativity and $\pzero$ as identity) for parallel
  composition $|$ and for summation $+$.
\end{definition}

\subsection{Name equivalence}

We take name equivalence, written $\nameeq$, to be the smallest
equivalence relation generated by the following rules.

\begin{mathpar}
\inferrule*[lab=Quote-drop]
{ }
{ \quotep{@{x}} \nameeq x }

\inferrule*[lab=Struct-equiv]
{ P \scong Q }
{ \quotep{P} \nameeq \quotep{Q} }
\end{mathpar}

The astute reader will have noticed that the mutual recursion of names
and processes imposes a mutual recursion on alpha-equivalence and
structural equivalence via name-equivalence. Fortunately, all of this
works out pleasantly and we may calculate in the natural way, free of
concern. The reader interested in the details is referred to the
appendix \ref{appendix:rho_details}.

\subsection{Substitution}

We use $\Proc$ for the set of processes, $\QProc$ for the set of
names, and $\id{\{}\vec{y} / \vec{x} \id{\}}$ to denote partial maps,
$s : \QProc \rightarrow \QProc$. A map, $s$ lifts, uniquely, to a map
on process terms, $\widehat{s} : \Proc \rightarrow \Proc$ by the
following equations.

\begin{mathpar}
  (0) \psubstp{Q}{P} := 0 \\
  (R \juxtap S) \psubstp{Q}{P}
  :=    
  (R)\psubstp{Q}{P} \juxtap (S) \psubstp{Q}{P} \\
  (x?(y).R) \psubstp{Q}{P}    
  :=    
  (x)\substp{Q}{P} (z)\concat( (R \psubstn{z}{y}) \psubstp{Q}{P} ) \\
  (\lift{x}{R}) \psubstp{Q}{P}  
  :=
  \lift{(x)\substp{Q}{P}}{ R \psubstp{Q}{P} } \\
%   (\dropn{x})  \psubstp{Q}{P}       
%   := 
%   \left\{ 
%     \begin{array}{ccc} 
%       \dropn{\quotep{Q}} & & x \nameeq \quotep{P} \\
%       \dropn{x} & & otherwise \\
%     \end{array}
%   \right. 
  (\dropn{x})  \psubstp{Q}{P}       
  := 
  \left\{ 
    \begin{array}{ccc} 
      Q & & x \nameeq \quotep{P} \\
      \dropn{x} & & otherwise \\
    \end{array}
  \right.
\end{mathpar}
 

where

\begin{eqnarray}
  (x)\id{\{} \lpquote Q \rpquote / \lpquote P \rpquote \id{\}}            = 
  \left\{ 
    \begin{array}{ccc}
      \lpquote Q \rpquote & & x \nameeq \lpquote P \rpquote \\
      x & & otherwise \\
    \end{array}
  \right. \nonumber
\end{eqnarray}

and $z$ is chosen distinct from $\quotep{P}$, $\quotep{Q}$, the free
names in $Q$, and all the names in $R$. Our $\alpha$-equivalence will
be built in the standard way from this substitution.

\begin{remark}\label{rem:no_self_referential_names}
  One consequence of these definitions is that $\forall P. \quotep{P}
  \not\in \freenames{P}$.
\end{remark}

\subsection{ Dynamic quote: an example }

Anticipating something of what's to come, consider applying the
substitution, $\widehat{\id{\{}u / z \id{\}}}$, to the following pair
of processes, $\lift{w}{y!(z)}$ and $w[ \lpquote y!(z) \rpquote ]$.

\begin{eqnarray}
	\lift{w}{y!(z)}\widehat{\id{\{}u / z \id{\}}}
		& = &
		\lift{w}{y!(u)} \nonumber\\
	w[ \lpquote y!(z) \rpquote ] \widehat{ \id{\{}u / z \id{\}} }
		& = &
		w[ \lpquote y!(z) \rpquote ] \nonumber
\end{eqnarray}

Because the body of the process between quotes is impervious to
substitution, we get radically different answers. In fact, by
examining the first process in an input context,
e.g. $x?(z).\lift{w}{y!(z)}$, we see that the process under the lift
operator may be shaped by prefixed inputs binding a name inside it. In
this sense, the lift operator will be seen as a way to dynamically
construct processes before reifying them as names.

Finally equipped with these standard features we can present the
dynamics of the calculus.

\subsubsection{Operational semantics} 

Finally, we introduce the computational dynamics. What marks these
algebras as distinct from other more traditionally studied algebraic
structures, e.g. vector spaces or polynomial rings, is the manner in
which dynamics is captured. In traditional structures, dynamics is typically
expressed through morphisms between such structures, as in linear maps
between vector spaces or morphisms between rings. In algebras
associated with the semantics of computation, the dynamics is
expressed as part of the algebraic structure itself, through a
reduction reduction relation typically denoted by $\red$. Below, we
give a recursive presentation of this relation for the calculus used
in the encoding.

$\red \subseteq \pi \times \pi$
$\red : \pi \to \mathcal{P}(\pi)$

\begin{mathpar}
  \inferrule* [lab=Comm] { \textsf{match}( x_{src}, x_{trgt} ) } { x_{trgt}?(y)P \; | \; x_{src}!\langle {Q} \rangle \red P\{\quotep{Q}/y}\} }
  \and \\
  \inferrule* [lab=Par] {{P} \red {P}'} {{{P} | {Q}} \red {{P}' | {Q}}}
  \and
  \inferrule* [lab=Equiv]{{{P} \scong {P}'} \andalso {{P}' \red {Q}'} \andalso {{Q}' \scong {Q}}}{{P} \red {Q}}
\end{mathpar}

\begin{eqnarray*}
  match_{\equiv} (\quotep{P},\quotep{Q}) & := & P \equiv Q \\
  match_{\dagger}(\quotep{P},\quotep{Q}) & := & \forall R. P|Q \red^{*} R => R \red^{*} 0 \\
  match_{K}(\quotep{P},\quotep{Q}) & := & K \mbox{ for some context } K
\end{eqnarray*}

$u?(x)P | u!\langle Q \rangle \red P\{\quotep{Q}/x\}$

%We write $\wred$ for $\red^*$, and $P\red$ if $\exists Q $ such that $ P \red Q$.
We write $P\red$ if $\exists Q $ such that $ P \red Q$ and $P\not\red$, otherwise.

\section{Replication}

As mentioned before, it is known that replication (and hence
recursion) can be implemented in a higher-order process algebra
\cite{SangiorgiWalker}. As our first example of calculation with the
machinery thus far presented we give the construction explicitly in
the {\rhoc}.

\begin{eqnarray}
	D_{x} & := & \prefix{x}{y}{(\binpar{\outputp{x}{y}}{@{y}})} \nonumber\\
	\bangp_{x}{P} & := & \binpar{{x}!\langle{\binpar{D_{x}}{P}}\rangle}{D_{x}} \nonumber
\end{eqnarray}

\begin{eqnarray}
	\bangp_{x}{P} & & \nonumber\\
	=
	& {x}!\langle{(\prefix{x}{y}{(\outputp{x}{y} | @{y})) | P}}\rangle 
	      | \prefix{x}{y}{(\outputp{x}{y} | @{y})} & \nonumber\\
	\red
	& (\outputp{x}{y} | @{y})\substn{\quotep{(\prefix{x}{y}{(@{y} | \outputp{x}{y})) | P}}}{y} & \nonumber\\
	=
	& \outputp{x}{\quotep{(\prefix{x}{y}{(\outputp{x}{y} | @{y})) | P}}}
	  | {(\prefix{x}{y}{(\outputp{x}{y} | @{y})) | P}} & \nonumber\\
	\red
	& \ldots & \nonumber\\
	\red^*
	& P | P | \ldots & \nonumber
\end{eqnarray}

Of course, this encoding, as an implementation, runs away, unfolding
$\bangp{P}$ eagerly. A lazier and more implementable replication
operator, restricted to input-guarded processes, may be obtained as follows.

\begin{eqnarray}
\bangp{\prefix{u}{v}{P}} 
	:= 
	\binpar{\lift{x}{\prefix{u}{v}{(\binpar{D(x)}{P})}}}{D(x)} \nonumber
\end{eqnarray}

\begin{remark}
  Note that the lazier definition still does not deal with summation
  or mixed summation (i.e. sums over input and output). The reader is
  invited to construct definitions of replication that deal with these
  features. 

  Further, the definitions are parameterized in a name, $x$. Can you,
  gentle reader, make a definition that eliminates this parameter and
  guarantees no accidental interaction between the replication
  machinery and the process being replicated -- i.e. no accidental
  sharing of names used by the process to get its work done and the
  name(s) used by the replication to effect copying. This latter
  revision of the definition of replication is crucial to obtaining
  the expected identity $!!P \sim !P$.
\end{remark}

\begin{remark}\label{rem:paradoxical_combinator}
  The reader familiar with the lambda calculus will have noticed the
  similarity between $D$ and the paradoxical combinator.

  [Ed. note: the existence of this seems to suggest we have to be more
  restrictive on the set of processes and names we admit if we are to
  support no-cloning.]
\end{remark}

\subsubsection{Bisimulation}

The computational dynamics gives rise to another kind of equivalence,
the equivalence of computational behavior. As previously mentioned
this is typically captured \emph{via} some form of bisimulation.

% The notion we use in this paper is weak barbed bisimulation
% \cite{milner91polyadicpi}.

The notion we use in this paper is derived from weak barbed
bisimulation \cite{milner91polyadicpi}. 

\begin{definition}
An \emph{observation relation}, $\downarrow_{\mathcal N}$, over a set
of names, $\mathcal N$, is the smallest relation satisfying the rules
below.

\infrule[Out-barb]{y \in {\mathcal N}, \; x \nameeq y}
		  {\outputp{x}{v} \downarrow_{\mathcal N} x}
\infrule[Par-barb]{\mbox{$P\downarrow_{\mathcal N} x$ or $Q\downarrow_{\mathcal N} x$}}
		  {\binpar{P}{Q} \downarrow_{\mathcal N} x}

We write $P \Downarrow_{\mathcal N} x$ if there is $Q$ such that 
$P \wred Q$ and $Q \downarrow_{\mathcal N} x$.
\end{definition}

\begin{definition}
%\label{def.bbisim}
An  ${\mathcal N}$-\emph{barbed bisimulation} over a set of names, ${\mathcal N}$, is a symmetric binary relation 
${\mathcal S}_{\mathcal N}$ between agents such that $P\rel{S}_{\mathcal N}Q$ implies:
\begin{enumerate}
\item If $P \red P'$ then $Q \wred Q'$ and $P'\rel{S}_{\mathcal N} Q'$.
\item If $P\downarrow_{\mathcal N} x$, then $Q\Downarrow_{\mathcal N} x$.
\end{enumerate}
$P$ is ${\mathcal N}$-barbed bisimilar to $Q$, written
$P \wbbisim_{\mathcal N} Q$, if $P \rel{S}_{\mathcal N} Q$ for some ${\mathcal N}$-barbed bisimulation ${\mathcal S}_{\mathcal N}$.
\end{definition}

$\mathcal{R} \subseteq \pi \times \pi$

$P \mathcal{R} Q => \forall P'. P \red P' \Rightarrow \exists Q'. Q \red Q', P' \mathcal{R} Q'$

$P \vdash x \Rightarrow Q \vdash x$

\begin{mathpar}
  \inferrule*[lab=Out-barb]{x \nameeq y}{{y}!\langle{Q}\rangle \vdash x}
  \and
  \inferrule*[lab=Par-barb]{\mbox{$P\vdash x$ or $Q\vdash x$}}{\binpar{P}{Q} \vdash x}
\end{mathpar}

\subsubsection{Contexts}

One of the principle advantages of computational calculi like the
$\pi$-calculus is a well-defined notion of context,
contextual-equivalence and a correlation between
contextual-equivalence and notions of bisimulation. The notion of
context allows the decomposition of a process into (sub-)process and
its syntactic environment, its context. Thus, a context may be
thought of as a process with a ``hole'' (written $\Box$) in it. The
application of a context $M$ to a process $P$, written $M[P]$, is
tantamount to filling the hole in $M$ with $P$. In this paper we do
not need the full weight of this theory, but do make use of the notion
of context in the proof the main theorem. 

\begin{mathpar}
  \inferrule* [lab=summation] {} {{M_{M},M_{N}} \bc \Box \;|\; x.M_{A} \;|\; M_{M}+M_{N}}
  \and
  \inferrule* [lab=agent] {} {{M_{A}} \bc (\vec{x})M_{P} \;| \; \clift{P_0,\ldots,M_{P},\ldots,P_N}}
  \and \\
  \inferrule* [lab=process] {} {{M_{P}} \bc M_{N} \;| \;P|M_{P} }
\end{mathpar} 

\begin{mathpar}
  \inferrule* [lab=sychronization] {} {M_{N} \bc \Box \;|\; x?M_{F} \;|\; x!M_{C}}
  \and
  \inferrule* [lab=abstraction] {} {{M_{F}} \bc (x)M_{P} }
  \and
  \inferrule* [lab=concretion] {} {{M_{C}} \bc \langle M_{P} \rangle }
  \and \\
  \inferrule* [lab=process] {} {{M_{P}} \bc M_{N} \;| \;P|M_{P} }
\end{mathpar}

\begin{definition}[contextual application] Given a context $M$, and
  process $P$, we define the \emph{contextual application}, $M[P] :=
  M\{P/\Box\}$. That is, the contextual application of M to P is the
  substitution of $P$ for $\Box$ in $M$.
\end{definition}

$\meaningof{-} : L \to \mathcal{P}(\pi)$

\begin{mathpar}
  \inferrule* [lab=collection] {} {\meaningof{true} = \pi, \and \meaningof{~E} = \pi \setminus \meaningof{E}, \and \meaningof{E_{1} \& E_{2}} = \meaningof{E_{1}} \cap \meaningof{E_{2}}}
\end{mathpar}

\begin{mathpar}
  \inferrule* [lab=structure] {} {\meaningof{0} = \{ P \in \pi | P \equiv 0 \}, \and \\ \meaningof{E_1 | E_2} = \{ P \in \pi | P \equiv P_{1} | P_{2}, P_{1} \in \meaningof{E_{1}}, P_{2} \in \meaningof{E_2}\} }
\end{mathpar}

\begin{mathpar}
 \inferrule* [lab=behavior] {} {\meaningof{\langle a?b \rangle E} = \{ P \in \pi | P \equiv Q | u?(y)P', \\ \and \\\\ \and \\ \;\;\; u \in \meaningof{a}, \forall z.P'\{z/y\} \in \meaningof{E\{z/b\}}\}, \and \\ \meaningof{a!E} = \{ P \in \pi | P \equiv Q | x!\langle P' \rangle, x \in \meaningof{a} P' \in \meaningof{E}\} }
\end{mathpar}

\begin{mathpar}
 \inferrule* [lab=nominal] {} {\meaningof{\quotep{E}} = \{ \quotep{P} \in \quotep{\pi} | P \in \meaningof{E} \}, \and \meaningof{\quotep{P}} = \{ \quotep{Q} \in \quotep{\pi} | P \equiv Q \} \and \\ \meaningof{@\quotep{E}} = \{ P \in \pi | P \equiv @x, x \in \meaningof{E} \}}
\end{mathpar}

\begin{eqnarray*}
  \\
  \meaningof{-} : TS \to ST
\end{eqnarray*}

\begin{eqnarray*}
  \\
  L : TS \to ST
\end{eqnarray*}

\begin{eqnarray*}
  \\
  P \models E \iff P \in \meaningof{E}
\end{eqnarray*}

\begin{eqnarray*}
  P \approx_{L} Q \iff \forall E \in L. P \models E \iff Q \models E
\end{eqnarray*}

\begin{eqnarray*}
  P \approx_{K} Q
\end{eqnarray*}

\begin{eqnarray*}
  P \approx Q
\end{eqnarray*}

$\approx_{K} = \approx = \approx_{L}$

\subsubsection{Contextual duality}

Note that contexts extend the quotation operation to a family of
operations from processes to names. Given a context, $M$, we can
define a \emph{nominal context}, $\quotep{M}$ by $\quotep{M}[P] :=
\quotep{M[P]}$. To foreshadow what is to come we observe that these
operations enjoy a duality with processes very much like the duality
between vectors and maps from vectors to scalars.

Further, because the calculus is essentially higher-order, we have a
correspondence between contexts and processes. More specifically,
given a name $x$ and a context $M$ we can construct $M^{*}_{x}$ such
that 

\begin{mathpar}
  M^{*}_{x} | \lift{x}{P} \red M[P]
\end{mathpar}

namely,

\begin{mathpar}
  M^{*}_{x} := x?(u).M[\dropn{u}]
\end{mathpar}

The dependence of $M^{*}_{x}$ on a name makes it an abstraction, 

\begin{mathpar}
  M^{*} := (x)x?(u).M[\dropn{u}]
\end{mathpar}

\subsection{Additional notation}

It will sometimes be convenient to denote the process a name
quotes. We already have the notation $x = \quotep{P}$, but it will be
convenient to introduce an alternate notation, $\procn{x}$, when we
want to emphasize the connection to the use of the name. Note that, by
virtue of name equivalence, $\quotep{\procn{x}} \nameeq x$; so, the
notation is consistent with previous definitions.

Further, because names have structure it is possible to effect
substitutions on the basis of that structure. This means we need to
upgrade our notation for substitutions, which we accomplish by
adapting comprehension notation. Thus,

\begin{mathpar}
  P\{ y / x : x \in S \}
\end{mathpar}

is interpreted to mean the process derived from P by replacing (in a
capture-avoiding manner) each occurrence of $x$ in $S$ by $y$. For example,

\begin{mathpar}
  P\{ \quotep{\procn{x}|\procn{x}} / x : x \in \freenames{P} \}
\end{mathpar}

will replace each (occurrence) of a free name $x$ in $P$ by
$\quotep{\procn{x}|\procn{x}}$.

Also, we will avail ourselves of the notation $x^{L}$ and $x^{R}$ to
denote injections of a name into disjoint copies of the name
space. There are numerous ways to accomplish this. One example can be
found in \cite{MeredithR05}. This notation overloads to vectors of
names: $\vec{x}^{\pi} := (x_{i}^{\pi} \; : \; 0 \leq i < |\vec{x}| )$ where $\pi \in \{L,R\}$.

We also use $P^{\Box} := P|\Box$.

In \cite{MeredithR05} an interpretation of the new operator is
given. It turns out that there are several possible interpretations
all enjoying the requisite algebraic properties of the operator (see
\cite{milner91polyadicpi}). We will therefore make liberal use of
$(\nu\; \vec{x})P$.

% subsection the_syntax_and_semantics_of_the_notation_system (end)   

\input{qm2pi.qmops} 

\input{qm2pi.sterngerlach} 

\input{qm2pi.metric} 

% section concurrent_process_calculi (end)

%\input{qm2pi.proofsketch}

% section proof sketch (end)

%\input{qm2pi.slviaknots} 

% section spatial logic via knots (end)

\input{qm2pi.conclusion}

% section conclusion (end)

%\input{qm2pi.dtcodes} 

% section wiring algorithm (end)

\input{qm2pi.ack} 

% section acknowledgments (end)

\newpage


\bibliographystyle{plain}   
\bibliography{../../biblios/main.bib}

\input{qm2pi.rhodetails}

\end{document}

 

% section notation (end)

\input{qm2pi.process.calculi} 

% section concurrent_process_calculi_and_spatial_logics_ (end)
    
%\documentclass[12pt]{llncs}
%\documentclass{jktr}

\usepackage[pdftex]{hyperref}                   
\usepackage {listings}
\usepackage {mathpartir}
\usepackage{bcprules}
%\usepackage{listings}
                       
\usepackage{graphicx} 
%\usepackage[margins=2.5cm,nohead,nofoot]{geometry}
%\usepackage{geometry}
\usepackage{amsfonts}
\usepackage{amstext}
\usepackage{latexsym}
\usepackage{amssymb}
\usepackage{color}


%\include{myPreamble}
\include{qm2pi.local} 

%\ifpdf
%\usepackage[pdftex]{graphicx}
%\else
%\usepackage{graphicx}
%\fi

 % \ifpdf
%  \usepackage{pdfsync}
%  \if


%\title{Brief Article}
%\author{David F. Snyder}
%\author{L.G. Meredith}

%\address{Dept. of Math., Texas State University--San Marcos, San Marcos, TX 78666}
       
\pagestyle{empty}


\begin{document}

\lstset{language=[Objective]Caml,frame=shadowbox}

\input{qm2pi.front}

% section front matter (end)

\input{qm2pi.intro} 
 
% section introduction (end)

% \input{qm2pi.knotations} 

% section notation (end)

\input{qm2pi.process.calculi} 

% section concurrent_process_calculi_and_spatial_logics_ (end)
    
%\input{qm2pi.knots2pi} 

%\input{qm2pi.trefoil} 

%\input{qm2pi.mainthm} 

% subsection basic_interpretation (end)

%\input{qm2pi.rho.presentation} 
\subsection{The syntax and semantics of the notation system}\label{sub:the_syntax_and_semantics_of_the_notation_system} % (fold)

We now summarize a technical presentation of the calculus that
embodies our theory of dynamics. The typical presentation of such a
calculus follows the style of giving generators and relations on
them. The grammar, below, describing term constructors, freely
generates the set of processes, $\Proc$. This set is then quotiented
by a relation known as structural congruence and it is over this set
that the notion of dynamics is expressed. This presentation is
essentially that of \cite{MeredithR05} with the addition of
polyadicity and summation. For readability we have relegated some of
the technical subtleties to an appendix.

\subsubsection{Process grammar}\label{subsub:process_grammar}

\begin{mathpar}
  \inferrule* [lab=synchronization] {} {{M} \bc \pzero \;|\; x?F \;|\; x!C }
  \and
  \inferrule* [lab=abstraction] {} {{F} \bc (x)P}
  \and
  \inferrule* [lab=concretion] {} {{C} \bc \langle Q \rangle}
  \and
  \inferrule* [lab=process] {} {{P,Q} \bc M \;| \;P|Q \;|\; @{x}}
  \and
  \inferrule* [lab=name] {} {{x} \bc \quotep{P}}
\end{mathpar} 

Note that $\vec{x}$ (resp. $\vec{P}$) denotes a vector of names
(resp. processes) of length $|\vec{x}|$ (resp. $|\vec{P}|$). We adopt
the following useful abbreviations.

\begin{mathpar}
   x?(\vec{y}).P := x.(\vec{y})P \and  x\clift{\vec{P}} := x.\clift{\vec{P}}
   \and x!(y) := \lift{x}{\dropn{y}}
   \and \Pi_{i=0}^{n-1}P_i := P_0 | \ldots | P_{n-1}
\end{mathpar}

\subsubsection{Structural congruence}

\paragraph{Free and bound names and alpha-equivalence.} At the
core of structural equivalence is alpha-equivalence which identifies
process that are the same up to a change of variable. Formally, we
recognize the distinction between free and bound names. The free names
of a process, $\freenames{P}$, may be calculated recursively as
follows:

\begin{mathpar}
\freenames{\pzero} := \emptyset
  \and \\
  \freenames{x?(y).P} := \{ x \} \cup (\freenames{P} \setminus \{ y \})
  \and 
  \freenames{x!\langle P \rangle} := \{ x \} \cup \{ P \} 
  \and \\
  \freenames{P|Q} := \freenames{P} \cup \freenames{Q}
  \and \\
  \freenames{@{x}} := \{ x \}
\end{mathpar}

$\pi$
$\quotep{\pi}$

$\freenames{-} : \pi \to \mathcal{P}(\quotep{\pi})$

\begin{eqnarray*}
  \freenames{\pzero} & := & \emptyset \\
  \freenames{x?(y).P} & := & \{ x \} \cup (\freenames{P} \setminus \{ y \}) \\
  \freenames{x!\langle P \rangle} & := & \{ x \} \cup \{ P \} \\
  \freenames{P|Q} & := & \freenames{P} \cup \freenames{Q} \\
  \freenames{\dropn{x}} & := & \{ x \}
\end{eqnarray*}

The bound names of a process, $\boundnames{P}$, are those names occurring in $P$
that are not free. For example, in $x?(y).0$, the name $x$ is free, while $y$ is bound.

\begin{mathpar}
  \inferrule* [lab=monoidal-laws] {} { P|Q \equiv Q|P \and P|0 \equiv P \and P|(Q|R) \equiv (P|Q)|R }
\end{mathpar}

\begin{mathpar}
  \inferrule* [lab=alpha-equivalence] {} { (x)P \equiv (y)P\{y/x\} \and y \not\in \freenames{P} }
\end{mathpar}

\begin{definition}
Then two processes, $P,Q$, are alpha-equivalent if $P = Q\{\vec{y}/\vec{x}\}$ for
some $\vec{x} \in \boundnames{Q},\vec{y} \in \boundnames{P}$, where $Q\{\vec{y}/\vec{x}\}$
denotes the capture-avoiding substitution of $\vec{y}$ for $\vec{x}$ in $Q$.
\end{definition}

\begin{definition}
  The {\em structural congruence} \cite{SangiorgiWalker} , $\equiv$,
  between processes is the least congruence containing
  alpha-equivalence, satisfying the abelian monoid laws
  (associativity, commutativity and $\pzero$ as identity) for parallel
  composition $|$ and for summation $+$.
\end{definition}

\subsection{Name equivalence}

We take name equivalence, written $\nameeq$, to be the smallest
equivalence relation generated by the following rules.

\begin{mathpar}
\inferrule*[lab=Quote-drop]
{ }
{ \quotep{@{x}} \nameeq x }

\inferrule*[lab=Struct-equiv]
{ P \scong Q }
{ \quotep{P} \nameeq \quotep{Q} }
\end{mathpar}

The astute reader will have noticed that the mutual recursion of names
and processes imposes a mutual recursion on alpha-equivalence and
structural equivalence via name-equivalence. Fortunately, all of this
works out pleasantly and we may calculate in the natural way, free of
concern. The reader interested in the details is referred to the
appendix \ref{appendix:rho_details}.

\subsection{Substitution}

We use $\Proc$ for the set of processes, $\QProc$ for the set of
names, and $\id{\{}\vec{y} / \vec{x} \id{\}}$ to denote partial maps,
$s : \QProc \rightarrow \QProc$. A map, $s$ lifts, uniquely, to a map
on process terms, $\widehat{s} : \Proc \rightarrow \Proc$ by the
following equations.

\begin{mathpar}
  (0) \psubstp{Q}{P} := 0 \\
  (R \juxtap S) \psubstp{Q}{P}
  :=    
  (R)\psubstp{Q}{P} \juxtap (S) \psubstp{Q}{P} \\
  (x?(y).R) \psubstp{Q}{P}    
  :=    
  (x)\substp{Q}{P} (z)\concat( (R \psubstn{z}{y}) \psubstp{Q}{P} ) \\
  (\lift{x}{R}) \psubstp{Q}{P}  
  :=
  \lift{(x)\substp{Q}{P}}{ R \psubstp{Q}{P} } \\
%   (\dropn{x})  \psubstp{Q}{P}       
%   := 
%   \left\{ 
%     \begin{array}{ccc} 
%       \dropn{\quotep{Q}} & & x \nameeq \quotep{P} \\
%       \dropn{x} & & otherwise \\
%     \end{array}
%   \right. 
  (\dropn{x})  \psubstp{Q}{P}       
  := 
  \left\{ 
    \begin{array}{ccc} 
      Q & & x \nameeq \quotep{P} \\
      \dropn{x} & & otherwise \\
    \end{array}
  \right.
\end{mathpar}
 

where

\begin{eqnarray}
  (x)\id{\{} \lpquote Q \rpquote / \lpquote P \rpquote \id{\}}            = 
  \left\{ 
    \begin{array}{ccc}
      \lpquote Q \rpquote & & x \nameeq \lpquote P \rpquote \\
      x & & otherwise \\
    \end{array}
  \right. \nonumber
\end{eqnarray}

and $z$ is chosen distinct from $\quotep{P}$, $\quotep{Q}$, the free
names in $Q$, and all the names in $R$. Our $\alpha$-equivalence will
be built in the standard way from this substitution.

\begin{remark}\label{rem:no_self_referential_names}
  One consequence of these definitions is that $\forall P. \quotep{P}
  \not\in \freenames{P}$.
\end{remark}

\subsection{ Dynamic quote: an example }

Anticipating something of what's to come, consider applying the
substitution, $\widehat{\id{\{}u / z \id{\}}}$, to the following pair
of processes, $\lift{w}{y!(z)}$ and $w[ \lpquote y!(z) \rpquote ]$.

\begin{eqnarray}
	\lift{w}{y!(z)}\widehat{\id{\{}u / z \id{\}}}
		& = &
		\lift{w}{y!(u)} \nonumber\\
	w[ \lpquote y!(z) \rpquote ] \widehat{ \id{\{}u / z \id{\}} }
		& = &
		w[ \lpquote y!(z) \rpquote ] \nonumber
\end{eqnarray}

Because the body of the process between quotes is impervious to
substitution, we get radically different answers. In fact, by
examining the first process in an input context,
e.g. $x?(z).\lift{w}{y!(z)}$, we see that the process under the lift
operator may be shaped by prefixed inputs binding a name inside it. In
this sense, the lift operator will be seen as a way to dynamically
construct processes before reifying them as names.

Finally equipped with these standard features we can present the
dynamics of the calculus.

\subsubsection{Operational semantics} 

Finally, we introduce the computational dynamics. What marks these
algebras as distinct from other more traditionally studied algebraic
structures, e.g. vector spaces or polynomial rings, is the manner in
which dynamics is captured. In traditional structures, dynamics is typically
expressed through morphisms between such structures, as in linear maps
between vector spaces or morphisms between rings. In algebras
associated with the semantics of computation, the dynamics is
expressed as part of the algebraic structure itself, through a
reduction reduction relation typically denoted by $\red$. Below, we
give a recursive presentation of this relation for the calculus used
in the encoding.

$\red \subseteq \pi \times \pi$
$\red : \pi \to \mathcal{P}(\pi)$

\begin{mathpar}
  \inferrule* [lab=Comm] { \textsf{match}( x_{src}, x_{trgt} ) } { x_{trgt}?(y)P \; | \; x_{src}!\langle {Q} \rangle \red P\{\quotep{Q}/y}\} }
  \and \\
  \inferrule* [lab=Par] {{P} \red {P}'} {{{P} | {Q}} \red {{P}' | {Q}}}
  \and
  \inferrule* [lab=Equiv]{{{P} \scong {P}'} \andalso {{P}' \red {Q}'} \andalso {{Q}' \scong {Q}}}{{P} \red {Q}}
\end{mathpar}

\begin{eqnarray*}
  match_{\equiv} (\quotep{P},\quotep{Q}) & := & P \equiv Q \\
  match_{\dagger}(\quotep{P},\quotep{Q}) & := & \forall R. P|Q \red^{*} R => R \red^{*} 0 \\
  match_{K}(\quotep{P},\quotep{Q}) & := & K \mbox{ for some context } K
\end{eqnarray*}

$u?(x)P | u!\langle Q \rangle \red P\{\quotep{Q}/x\}$

%We write $\wred$ for $\red^*$, and $P\red$ if $\exists Q $ such that $ P \red Q$.
We write $P\red$ if $\exists Q $ such that $ P \red Q$ and $P\not\red$, otherwise.

\section{Replication}

As mentioned before, it is known that replication (and hence
recursion) can be implemented in a higher-order process algebra
\cite{SangiorgiWalker}. As our first example of calculation with the
machinery thus far presented we give the construction explicitly in
the {\rhoc}.

\begin{eqnarray}
	D_{x} & := & \prefix{x}{y}{(\binpar{\outputp{x}{y}}{@{y}})} \nonumber\\
	\bangp_{x}{P} & := & \binpar{{x}!\langle{\binpar{D_{x}}{P}}\rangle}{D_{x}} \nonumber
\end{eqnarray}

\begin{eqnarray}
	\bangp_{x}{P} & & \nonumber\\
	=
	& {x}!\langle{(\prefix{x}{y}{(\outputp{x}{y} | @{y})) | P}}\rangle 
	      | \prefix{x}{y}{(\outputp{x}{y} | @{y})} & \nonumber\\
	\red
	& (\outputp{x}{y} | @{y})\substn{\quotep{(\prefix{x}{y}{(@{y} | \outputp{x}{y})) | P}}}{y} & \nonumber\\
	=
	& \outputp{x}{\quotep{(\prefix{x}{y}{(\outputp{x}{y} | @{y})) | P}}}
	  | {(\prefix{x}{y}{(\outputp{x}{y} | @{y})) | P}} & \nonumber\\
	\red
	& \ldots & \nonumber\\
	\red^*
	& P | P | \ldots & \nonumber
\end{eqnarray}

Of course, this encoding, as an implementation, runs away, unfolding
$\bangp{P}$ eagerly. A lazier and more implementable replication
operator, restricted to input-guarded processes, may be obtained as follows.

\begin{eqnarray}
\bangp{\prefix{u}{v}{P}} 
	:= 
	\binpar{\lift{x}{\prefix{u}{v}{(\binpar{D(x)}{P})}}}{D(x)} \nonumber
\end{eqnarray}

\begin{remark}
  Note that the lazier definition still does not deal with summation
  or mixed summation (i.e. sums over input and output). The reader is
  invited to construct definitions of replication that deal with these
  features. 

  Further, the definitions are parameterized in a name, $x$. Can you,
  gentle reader, make a definition that eliminates this parameter and
  guarantees no accidental interaction between the replication
  machinery and the process being replicated -- i.e. no accidental
  sharing of names used by the process to get its work done and the
  name(s) used by the replication to effect copying. This latter
  revision of the definition of replication is crucial to obtaining
  the expected identity $!!P \sim !P$.
\end{remark}

\begin{remark}\label{rem:paradoxical_combinator}
  The reader familiar with the lambda calculus will have noticed the
  similarity between $D$ and the paradoxical combinator.

  [Ed. note: the existence of this seems to suggest we have to be more
  restrictive on the set of processes and names we admit if we are to
  support no-cloning.]
\end{remark}

\subsubsection{Bisimulation}

The computational dynamics gives rise to another kind of equivalence,
the equivalence of computational behavior. As previously mentioned
this is typically captured \emph{via} some form of bisimulation.

% The notion we use in this paper is weak barbed bisimulation
% \cite{milner91polyadicpi}.

The notion we use in this paper is derived from weak barbed
bisimulation \cite{milner91polyadicpi}. 

\begin{definition}
An \emph{observation relation}, $\downarrow_{\mathcal N}$, over a set
of names, $\mathcal N$, is the smallest relation satisfying the rules
below.

\infrule[Out-barb]{y \in {\mathcal N}, \; x \nameeq y}
		  {\outputp{x}{v} \downarrow_{\mathcal N} x}
\infrule[Par-barb]{\mbox{$P\downarrow_{\mathcal N} x$ or $Q\downarrow_{\mathcal N} x$}}
		  {\binpar{P}{Q} \downarrow_{\mathcal N} x}

We write $P \Downarrow_{\mathcal N} x$ if there is $Q$ such that 
$P \wred Q$ and $Q \downarrow_{\mathcal N} x$.
\end{definition}

\begin{definition}
%\label{def.bbisim}
An  ${\mathcal N}$-\emph{barbed bisimulation} over a set of names, ${\mathcal N}$, is a symmetric binary relation 
${\mathcal S}_{\mathcal N}$ between agents such that $P\rel{S}_{\mathcal N}Q$ implies:
\begin{enumerate}
\item If $P \red P'$ then $Q \wred Q'$ and $P'\rel{S}_{\mathcal N} Q'$.
\item If $P\downarrow_{\mathcal N} x$, then $Q\Downarrow_{\mathcal N} x$.
\end{enumerate}
$P$ is ${\mathcal N}$-barbed bisimilar to $Q$, written
$P \wbbisim_{\mathcal N} Q$, if $P \rel{S}_{\mathcal N} Q$ for some ${\mathcal N}$-barbed bisimulation ${\mathcal S}_{\mathcal N}$.
\end{definition}

$\mathcal{R} \subseteq \pi \times \pi$

$P \mathcal{R} Q => \forall P'. P \red P' \Rightarrow \exists Q'. Q \red Q', P' \mathcal{R} Q'$

$P \vdash x \Rightarrow Q \vdash x$

\begin{mathpar}
  \inferrule*[lab=Out-barb]{x \nameeq y}{{y}!\langle{Q}\rangle \vdash x}
  \and
  \inferrule*[lab=Par-barb]{\mbox{$P\vdash x$ or $Q\vdash x$}}{\binpar{P}{Q} \vdash x}
\end{mathpar}

\subsubsection{Contexts}

One of the principle advantages of computational calculi like the
$\pi$-calculus is a well-defined notion of context,
contextual-equivalence and a correlation between
contextual-equivalence and notions of bisimulation. The notion of
context allows the decomposition of a process into (sub-)process and
its syntactic environment, its context. Thus, a context may be
thought of as a process with a ``hole'' (written $\Box$) in it. The
application of a context $M$ to a process $P$, written $M[P]$, is
tantamount to filling the hole in $M$ with $P$. In this paper we do
not need the full weight of this theory, but do make use of the notion
of context in the proof the main theorem. 

\begin{mathpar}
  \inferrule* [lab=summation] {} {{M_{M},M_{N}} \bc \Box \;|\; x.M_{A} \;|\; M_{M}+M_{N}}
  \and
  \inferrule* [lab=agent] {} {{M_{A}} \bc (\vec{x})M_{P} \;| \; \clift{P_0,\ldots,M_{P},\ldots,P_N}}
  \and \\
  \inferrule* [lab=process] {} {{M_{P}} \bc M_{N} \;| \;P|M_{P} }
\end{mathpar} 

\begin{mathpar}
  \inferrule* [lab=sychronization] {} {M_{N} \bc \Box \;|\; x?M_{F} \;|\; x!M_{C}}
  \and
  \inferrule* [lab=abstraction] {} {{M_{F}} \bc (x)M_{P} }
  \and
  \inferrule* [lab=concretion] {} {{M_{C}} \bc \langle M_{P} \rangle }
  \and \\
  \inferrule* [lab=process] {} {{M_{P}} \bc M_{N} \;| \;P|M_{P} }
\end{mathpar}

\begin{definition}[contextual application] Given a context $M$, and
  process $P$, we define the \emph{contextual application}, $M[P] :=
  M\{P/\Box\}$. That is, the contextual application of M to P is the
  substitution of $P$ for $\Box$ in $M$.
\end{definition}

$\meaningof{-} : L \to \mathcal{P}(\pi)$

\begin{mathpar}
  \inferrule* [lab=collection] {} {\meaningof{true} = \pi, \and \meaningof{~E} = \pi \setminus \meaningof{E}, \and \meaningof{E_{1} \& E_{2}} = \meaningof{E_{1}} \cap \meaningof{E_{2}}}
\end{mathpar}

\begin{mathpar}
  \inferrule* [lab=structure] {} {\meaningof{0} = \{ P \in \pi | P \equiv 0 \}, \and \\ \meaningof{E_1 | E_2} = \{ P \in \pi | P \equiv P_{1} | P_{2}, P_{1} \in \meaningof{E_{1}}, P_{2} \in \meaningof{E_2}\} }
\end{mathpar}

\begin{mathpar}
 \inferrule* [lab=behavior] {} {\meaningof{\langle a?b \rangle E} = \{ P \in \pi | P \equiv Q | u?(y)P', \\ \and \\\\ \and \\ \;\;\; u \in \meaningof{a}, \forall z.P'\{z/y\} \in \meaningof{E\{z/b\}}\}, \and \\ \meaningof{a!E} = \{ P \in \pi | P \equiv Q | x!\langle P' \rangle, x \in \meaningof{a} P' \in \meaningof{E}\} }
\end{mathpar}

\begin{mathpar}
 \inferrule* [lab=nominal] {} {\meaningof{\quotep{E}} = \{ \quotep{P} \in \quotep{\pi} | P \in \meaningof{E} \}, \and \meaningof{\quotep{P}} = \{ \quotep{Q} \in \quotep{\pi} | P \equiv Q \} \and \\ \meaningof{@\quotep{E}} = \{ P \in \pi | P \equiv @x, x \in \meaningof{E} \}}
\end{mathpar}

\begin{eqnarray*}
  \\
  \meaningof{-} : TS \to ST
\end{eqnarray*}

\begin{eqnarray*}
  \\
  L : TS \to ST
\end{eqnarray*}

\begin{eqnarray*}
  \\
  P \models E \iff P \in \meaningof{E}
\end{eqnarray*}

\begin{eqnarray*}
  P \approx_{L} Q \iff \forall E \in L. P \models E \iff Q \models E
\end{eqnarray*}

\begin{eqnarray*}
  P \approx_{K} Q
\end{eqnarray*}

\begin{eqnarray*}
  P \approx Q
\end{eqnarray*}

$\approx_{K} = \approx = \approx_{L}$

\subsubsection{Contextual duality}

Note that contexts extend the quotation operation to a family of
operations from processes to names. Given a context, $M$, we can
define a \emph{nominal context}, $\quotep{M}$ by $\quotep{M}[P] :=
\quotep{M[P]}$. To foreshadow what is to come we observe that these
operations enjoy a duality with processes very much like the duality
between vectors and maps from vectors to scalars.

Further, because the calculus is essentially higher-order, we have a
correspondence between contexts and processes. More specifically,
given a name $x$ and a context $M$ we can construct $M^{*}_{x}$ such
that 

\begin{mathpar}
  M^{*}_{x} | \lift{x}{P} \red M[P]
\end{mathpar}

namely,

\begin{mathpar}
  M^{*}_{x} := x?(u).M[\dropn{u}]
\end{mathpar}

The dependence of $M^{*}_{x}$ on a name makes it an abstraction, 

\begin{mathpar}
  M^{*} := (x)x?(u).M[\dropn{u}]
\end{mathpar}

\subsection{Additional notation}

It will sometimes be convenient to denote the process a name
quotes. We already have the notation $x = \quotep{P}$, but it will be
convenient to introduce an alternate notation, $\procn{x}$, when we
want to emphasize the connection to the use of the name. Note that, by
virtue of name equivalence, $\quotep{\procn{x}} \nameeq x$; so, the
notation is consistent with previous definitions.

Further, because names have structure it is possible to effect
substitutions on the basis of that structure. This means we need to
upgrade our notation for substitutions, which we accomplish by
adapting comprehension notation. Thus,

\begin{mathpar}
  P\{ y / x : x \in S \}
\end{mathpar}

is interpreted to mean the process derived from P by replacing (in a
capture-avoiding manner) each occurrence of $x$ in $S$ by $y$. For example,

\begin{mathpar}
  P\{ \quotep{\procn{x}|\procn{x}} / x : x \in \freenames{P} \}
\end{mathpar}

will replace each (occurrence) of a free name $x$ in $P$ by
$\quotep{\procn{x}|\procn{x}}$.

Also, we will avail ourselves of the notation $x^{L}$ and $x^{R}$ to
denote injections of a name into disjoint copies of the name
space. There are numerous ways to accomplish this. One example can be
found in \cite{MeredithR05}. This notation overloads to vectors of
names: $\vec{x}^{\pi} := (x_{i}^{\pi} \; : \; 0 \leq i < |\vec{x}| )$ where $\pi \in \{L,R\}$.

We also use $P^{\Box} := P|\Box$.

In \cite{MeredithR05} an interpretation of the new operator is
given. It turns out that there are several possible interpretations
all enjoying the requisite algebraic properties of the operator (see
\cite{milner91polyadicpi}). We will therefore make liberal use of
$(\nu\; \vec{x})P$.

% subsection the_syntax_and_semantics_of_the_notation_system (end)   

\input{qm2pi.qmops} 

\input{qm2pi.sterngerlach} 

\input{qm2pi.metric} 

% section concurrent_process_calculi (end)

%\input{qm2pi.proofsketch}

% section proof sketch (end)

%\input{qm2pi.slviaknots} 

% section spatial logic via knots (end)

\input{qm2pi.conclusion}

% section conclusion (end)

%\input{qm2pi.dtcodes} 

% section wiring algorithm (end)

\input{qm2pi.ack} 

% section acknowledgments (end)

\newpage


\bibliographystyle{plain}   
\bibliography{../../biblios/main.bib}

\input{qm2pi.rhodetails}

\end{document}

 

%\documentclass[12pt]{llncs}
%\documentclass{jktr}

\usepackage[pdftex]{hyperref}                   
\usepackage {listings}
\usepackage {mathpartir}
\usepackage{bcprules}
%\usepackage{listings}
                       
\usepackage{graphicx} 
%\usepackage[margins=2.5cm,nohead,nofoot]{geometry}
%\usepackage{geometry}
\usepackage{amsfonts}
\usepackage{amstext}
\usepackage{latexsym}
\usepackage{amssymb}
\usepackage{color}


%\include{myPreamble}
\include{qm2pi.local} 

%\ifpdf
%\usepackage[pdftex]{graphicx}
%\else
%\usepackage{graphicx}
%\fi

 % \ifpdf
%  \usepackage{pdfsync}
%  \if


%\title{Brief Article}
%\author{David F. Snyder}
%\author{L.G. Meredith}

%\address{Dept. of Math., Texas State University--San Marcos, San Marcos, TX 78666}
       
\pagestyle{empty}


\begin{document}

\lstset{language=[Objective]Caml,frame=shadowbox}

\input{qm2pi.front}

% section front matter (end)

\input{qm2pi.intro} 
 
% section introduction (end)

% \input{qm2pi.knotations} 

% section notation (end)

\input{qm2pi.process.calculi} 

% section concurrent_process_calculi_and_spatial_logics_ (end)
    
%\input{qm2pi.knots2pi} 

%\input{qm2pi.trefoil} 

%\input{qm2pi.mainthm} 

% subsection basic_interpretation (end)

%\input{qm2pi.rho.presentation} 
\subsection{The syntax and semantics of the notation system}\label{sub:the_syntax_and_semantics_of_the_notation_system} % (fold)

We now summarize a technical presentation of the calculus that
embodies our theory of dynamics. The typical presentation of such a
calculus follows the style of giving generators and relations on
them. The grammar, below, describing term constructors, freely
generates the set of processes, $\Proc$. This set is then quotiented
by a relation known as structural congruence and it is over this set
that the notion of dynamics is expressed. This presentation is
essentially that of \cite{MeredithR05} with the addition of
polyadicity and summation. For readability we have relegated some of
the technical subtleties to an appendix.

\subsubsection{Process grammar}\label{subsub:process_grammar}

\begin{mathpar}
  \inferrule* [lab=synchronization] {} {{M} \bc \pzero \;|\; x?F \;|\; x!C }
  \and
  \inferrule* [lab=abstraction] {} {{F} \bc (x)P}
  \and
  \inferrule* [lab=concretion] {} {{C} \bc \langle Q \rangle}
  \and
  \inferrule* [lab=process] {} {{P,Q} \bc M \;| \;P|Q \;|\; @{x}}
  \and
  \inferrule* [lab=name] {} {{x} \bc \quotep{P}}
\end{mathpar} 

Note that $\vec{x}$ (resp. $\vec{P}$) denotes a vector of names
(resp. processes) of length $|\vec{x}|$ (resp. $|\vec{P}|$). We adopt
the following useful abbreviations.

\begin{mathpar}
   x?(\vec{y}).P := x.(\vec{y})P \and  x\clift{\vec{P}} := x.\clift{\vec{P}}
   \and x!(y) := \lift{x}{\dropn{y}}
   \and \Pi_{i=0}^{n-1}P_i := P_0 | \ldots | P_{n-1}
\end{mathpar}

\subsubsection{Structural congruence}

\paragraph{Free and bound names and alpha-equivalence.} At the
core of structural equivalence is alpha-equivalence which identifies
process that are the same up to a change of variable. Formally, we
recognize the distinction between free and bound names. The free names
of a process, $\freenames{P}$, may be calculated recursively as
follows:

\begin{mathpar}
\freenames{\pzero} := \emptyset
  \and \\
  \freenames{x?(y).P} := \{ x \} \cup (\freenames{P} \setminus \{ y \})
  \and 
  \freenames{x!\langle P \rangle} := \{ x \} \cup \{ P \} 
  \and \\
  \freenames{P|Q} := \freenames{P} \cup \freenames{Q}
  \and \\
  \freenames{@{x}} := \{ x \}
\end{mathpar}

$\pi$
$\quotep{\pi}$

$\freenames{-} : \pi \to \mathcal{P}(\quotep{\pi})$

\begin{eqnarray*}
  \freenames{\pzero} & := & \emptyset \\
  \freenames{x?(y).P} & := & \{ x \} \cup (\freenames{P} \setminus \{ y \}) \\
  \freenames{x!\langle P \rangle} & := & \{ x \} \cup \{ P \} \\
  \freenames{P|Q} & := & \freenames{P} \cup \freenames{Q} \\
  \freenames{\dropn{x}} & := & \{ x \}
\end{eqnarray*}

The bound names of a process, $\boundnames{P}$, are those names occurring in $P$
that are not free. For example, in $x?(y).0$, the name $x$ is free, while $y$ is bound.

\begin{mathpar}
  \inferrule* [lab=monoidal-laws] {} { P|Q \equiv Q|P \and P|0 \equiv P \and P|(Q|R) \equiv (P|Q)|R }
\end{mathpar}

\begin{mathpar}
  \inferrule* [lab=alpha-equivalence] {} { (x)P \equiv (y)P\{y/x\} \and y \not\in \freenames{P} }
\end{mathpar}

\begin{definition}
Then two processes, $P,Q$, are alpha-equivalent if $P = Q\{\vec{y}/\vec{x}\}$ for
some $\vec{x} \in \boundnames{Q},\vec{y} \in \boundnames{P}$, where $Q\{\vec{y}/\vec{x}\}$
denotes the capture-avoiding substitution of $\vec{y}$ for $\vec{x}$ in $Q$.
\end{definition}

\begin{definition}
  The {\em structural congruence} \cite{SangiorgiWalker} , $\equiv$,
  between processes is the least congruence containing
  alpha-equivalence, satisfying the abelian monoid laws
  (associativity, commutativity and $\pzero$ as identity) for parallel
  composition $|$ and for summation $+$.
\end{definition}

\subsection{Name equivalence}

We take name equivalence, written $\nameeq$, to be the smallest
equivalence relation generated by the following rules.

\begin{mathpar}
\inferrule*[lab=Quote-drop]
{ }
{ \quotep{@{x}} \nameeq x }

\inferrule*[lab=Struct-equiv]
{ P \scong Q }
{ \quotep{P} \nameeq \quotep{Q} }
\end{mathpar}

The astute reader will have noticed that the mutual recursion of names
and processes imposes a mutual recursion on alpha-equivalence and
structural equivalence via name-equivalence. Fortunately, all of this
works out pleasantly and we may calculate in the natural way, free of
concern. The reader interested in the details is referred to the
appendix \ref{appendix:rho_details}.

\subsection{Substitution}

We use $\Proc$ for the set of processes, $\QProc$ for the set of
names, and $\id{\{}\vec{y} / \vec{x} \id{\}}$ to denote partial maps,
$s : \QProc \rightarrow \QProc$. A map, $s$ lifts, uniquely, to a map
on process terms, $\widehat{s} : \Proc \rightarrow \Proc$ by the
following equations.

\begin{mathpar}
  (0) \psubstp{Q}{P} := 0 \\
  (R \juxtap S) \psubstp{Q}{P}
  :=    
  (R)\psubstp{Q}{P} \juxtap (S) \psubstp{Q}{P} \\
  (x?(y).R) \psubstp{Q}{P}    
  :=    
  (x)\substp{Q}{P} (z)\concat( (R \psubstn{z}{y}) \psubstp{Q}{P} ) \\
  (\lift{x}{R}) \psubstp{Q}{P}  
  :=
  \lift{(x)\substp{Q}{P}}{ R \psubstp{Q}{P} } \\
%   (\dropn{x})  \psubstp{Q}{P}       
%   := 
%   \left\{ 
%     \begin{array}{ccc} 
%       \dropn{\quotep{Q}} & & x \nameeq \quotep{P} \\
%       \dropn{x} & & otherwise \\
%     \end{array}
%   \right. 
  (\dropn{x})  \psubstp{Q}{P}       
  := 
  \left\{ 
    \begin{array}{ccc} 
      Q & & x \nameeq \quotep{P} \\
      \dropn{x} & & otherwise \\
    \end{array}
  \right.
\end{mathpar}
 

where

\begin{eqnarray}
  (x)\id{\{} \lpquote Q \rpquote / \lpquote P \rpquote \id{\}}            = 
  \left\{ 
    \begin{array}{ccc}
      \lpquote Q \rpquote & & x \nameeq \lpquote P \rpquote \\
      x & & otherwise \\
    \end{array}
  \right. \nonumber
\end{eqnarray}

and $z$ is chosen distinct from $\quotep{P}$, $\quotep{Q}$, the free
names in $Q$, and all the names in $R$. Our $\alpha$-equivalence will
be built in the standard way from this substitution.

\begin{remark}\label{rem:no_self_referential_names}
  One consequence of these definitions is that $\forall P. \quotep{P}
  \not\in \freenames{P}$.
\end{remark}

\subsection{ Dynamic quote: an example }

Anticipating something of what's to come, consider applying the
substitution, $\widehat{\id{\{}u / z \id{\}}}$, to the following pair
of processes, $\lift{w}{y!(z)}$ and $w[ \lpquote y!(z) \rpquote ]$.

\begin{eqnarray}
	\lift{w}{y!(z)}\widehat{\id{\{}u / z \id{\}}}
		& = &
		\lift{w}{y!(u)} \nonumber\\
	w[ \lpquote y!(z) \rpquote ] \widehat{ \id{\{}u / z \id{\}} }
		& = &
		w[ \lpquote y!(z) \rpquote ] \nonumber
\end{eqnarray}

Because the body of the process between quotes is impervious to
substitution, we get radically different answers. In fact, by
examining the first process in an input context,
e.g. $x?(z).\lift{w}{y!(z)}$, we see that the process under the lift
operator may be shaped by prefixed inputs binding a name inside it. In
this sense, the lift operator will be seen as a way to dynamically
construct processes before reifying them as names.

Finally equipped with these standard features we can present the
dynamics of the calculus.

\subsubsection{Operational semantics} 

Finally, we introduce the computational dynamics. What marks these
algebras as distinct from other more traditionally studied algebraic
structures, e.g. vector spaces or polynomial rings, is the manner in
which dynamics is captured. In traditional structures, dynamics is typically
expressed through morphisms between such structures, as in linear maps
between vector spaces or morphisms between rings. In algebras
associated with the semantics of computation, the dynamics is
expressed as part of the algebraic structure itself, through a
reduction reduction relation typically denoted by $\red$. Below, we
give a recursive presentation of this relation for the calculus used
in the encoding.

$\red \subseteq \pi \times \pi$
$\red : \pi \to \mathcal{P}(\pi)$

\begin{mathpar}
  \inferrule* [lab=Comm] { \textsf{match}( x_{src}, x_{trgt} ) } { x_{trgt}?(y)P \; | \; x_{src}!\langle {Q} \rangle \red P\{\quotep{Q}/y}\} }
  \and \\
  \inferrule* [lab=Par] {{P} \red {P}'} {{{P} | {Q}} \red {{P}' | {Q}}}
  \and
  \inferrule* [lab=Equiv]{{{P} \scong {P}'} \andalso {{P}' \red {Q}'} \andalso {{Q}' \scong {Q}}}{{P} \red {Q}}
\end{mathpar}

\begin{eqnarray*}
  match_{\equiv} (\quotep{P},\quotep{Q}) & := & P \equiv Q \\
  match_{\dagger}(\quotep{P},\quotep{Q}) & := & \forall R. P|Q \red^{*} R => R \red^{*} 0 \\
  match_{K}(\quotep{P},\quotep{Q}) & := & K \mbox{ for some context } K
\end{eqnarray*}

$u?(x)P | u!\langle Q \rangle \red P\{\quotep{Q}/x\}$

%We write $\wred$ for $\red^*$, and $P\red$ if $\exists Q $ such that $ P \red Q$.
We write $P\red$ if $\exists Q $ such that $ P \red Q$ and $P\not\red$, otherwise.

\section{Replication}

As mentioned before, it is known that replication (and hence
recursion) can be implemented in a higher-order process algebra
\cite{SangiorgiWalker}. As our first example of calculation with the
machinery thus far presented we give the construction explicitly in
the {\rhoc}.

\begin{eqnarray}
	D_{x} & := & \prefix{x}{y}{(\binpar{\outputp{x}{y}}{@{y}})} \nonumber\\
	\bangp_{x}{P} & := & \binpar{{x}!\langle{\binpar{D_{x}}{P}}\rangle}{D_{x}} \nonumber
\end{eqnarray}

\begin{eqnarray}
	\bangp_{x}{P} & & \nonumber\\
	=
	& {x}!\langle{(\prefix{x}{y}{(\outputp{x}{y} | @{y})) | P}}\rangle 
	      | \prefix{x}{y}{(\outputp{x}{y} | @{y})} & \nonumber\\
	\red
	& (\outputp{x}{y} | @{y})\substn{\quotep{(\prefix{x}{y}{(@{y} | \outputp{x}{y})) | P}}}{y} & \nonumber\\
	=
	& \outputp{x}{\quotep{(\prefix{x}{y}{(\outputp{x}{y} | @{y})) | P}}}
	  | {(\prefix{x}{y}{(\outputp{x}{y} | @{y})) | P}} & \nonumber\\
	\red
	& \ldots & \nonumber\\
	\red^*
	& P | P | \ldots & \nonumber
\end{eqnarray}

Of course, this encoding, as an implementation, runs away, unfolding
$\bangp{P}$ eagerly. A lazier and more implementable replication
operator, restricted to input-guarded processes, may be obtained as follows.

\begin{eqnarray}
\bangp{\prefix{u}{v}{P}} 
	:= 
	\binpar{\lift{x}{\prefix{u}{v}{(\binpar{D(x)}{P})}}}{D(x)} \nonumber
\end{eqnarray}

\begin{remark}
  Note that the lazier definition still does not deal with summation
  or mixed summation (i.e. sums over input and output). The reader is
  invited to construct definitions of replication that deal with these
  features. 

  Further, the definitions are parameterized in a name, $x$. Can you,
  gentle reader, make a definition that eliminates this parameter and
  guarantees no accidental interaction between the replication
  machinery and the process being replicated -- i.e. no accidental
  sharing of names used by the process to get its work done and the
  name(s) used by the replication to effect copying. This latter
  revision of the definition of replication is crucial to obtaining
  the expected identity $!!P \sim !P$.
\end{remark}

\begin{remark}\label{rem:paradoxical_combinator}
  The reader familiar with the lambda calculus will have noticed the
  similarity between $D$ and the paradoxical combinator.

  [Ed. note: the existence of this seems to suggest we have to be more
  restrictive on the set of processes and names we admit if we are to
  support no-cloning.]
\end{remark}

\subsubsection{Bisimulation}

The computational dynamics gives rise to another kind of equivalence,
the equivalence of computational behavior. As previously mentioned
this is typically captured \emph{via} some form of bisimulation.

% The notion we use in this paper is weak barbed bisimulation
% \cite{milner91polyadicpi}.

The notion we use in this paper is derived from weak barbed
bisimulation \cite{milner91polyadicpi}. 

\begin{definition}
An \emph{observation relation}, $\downarrow_{\mathcal N}$, over a set
of names, $\mathcal N$, is the smallest relation satisfying the rules
below.

\infrule[Out-barb]{y \in {\mathcal N}, \; x \nameeq y}
		  {\outputp{x}{v} \downarrow_{\mathcal N} x}
\infrule[Par-barb]{\mbox{$P\downarrow_{\mathcal N} x$ or $Q\downarrow_{\mathcal N} x$}}
		  {\binpar{P}{Q} \downarrow_{\mathcal N} x}

We write $P \Downarrow_{\mathcal N} x$ if there is $Q$ such that 
$P \wred Q$ and $Q \downarrow_{\mathcal N} x$.
\end{definition}

\begin{definition}
%\label{def.bbisim}
An  ${\mathcal N}$-\emph{barbed bisimulation} over a set of names, ${\mathcal N}$, is a symmetric binary relation 
${\mathcal S}_{\mathcal N}$ between agents such that $P\rel{S}_{\mathcal N}Q$ implies:
\begin{enumerate}
\item If $P \red P'$ then $Q \wred Q'$ and $P'\rel{S}_{\mathcal N} Q'$.
\item If $P\downarrow_{\mathcal N} x$, then $Q\Downarrow_{\mathcal N} x$.
\end{enumerate}
$P$ is ${\mathcal N}$-barbed bisimilar to $Q$, written
$P \wbbisim_{\mathcal N} Q$, if $P \rel{S}_{\mathcal N} Q$ for some ${\mathcal N}$-barbed bisimulation ${\mathcal S}_{\mathcal N}$.
\end{definition}

$\mathcal{R} \subseteq \pi \times \pi$

$P \mathcal{R} Q => \forall P'. P \red P' \Rightarrow \exists Q'. Q \red Q', P' \mathcal{R} Q'$

$P \vdash x \Rightarrow Q \vdash x$

\begin{mathpar}
  \inferrule*[lab=Out-barb]{x \nameeq y}{{y}!\langle{Q}\rangle \vdash x}
  \and
  \inferrule*[lab=Par-barb]{\mbox{$P\vdash x$ or $Q\vdash x$}}{\binpar{P}{Q} \vdash x}
\end{mathpar}

\subsubsection{Contexts}

One of the principle advantages of computational calculi like the
$\pi$-calculus is a well-defined notion of context,
contextual-equivalence and a correlation between
contextual-equivalence and notions of bisimulation. The notion of
context allows the decomposition of a process into (sub-)process and
its syntactic environment, its context. Thus, a context may be
thought of as a process with a ``hole'' (written $\Box$) in it. The
application of a context $M$ to a process $P$, written $M[P]$, is
tantamount to filling the hole in $M$ with $P$. In this paper we do
not need the full weight of this theory, but do make use of the notion
of context in the proof the main theorem. 

\begin{mathpar}
  \inferrule* [lab=summation] {} {{M_{M},M_{N}} \bc \Box \;|\; x.M_{A} \;|\; M_{M}+M_{N}}
  \and
  \inferrule* [lab=agent] {} {{M_{A}} \bc (\vec{x})M_{P} \;| \; \clift{P_0,\ldots,M_{P},\ldots,P_N}}
  \and \\
  \inferrule* [lab=process] {} {{M_{P}} \bc M_{N} \;| \;P|M_{P} }
\end{mathpar} 

\begin{mathpar}
  \inferrule* [lab=sychronization] {} {M_{N} \bc \Box \;|\; x?M_{F} \;|\; x!M_{C}}
  \and
  \inferrule* [lab=abstraction] {} {{M_{F}} \bc (x)M_{P} }
  \and
  \inferrule* [lab=concretion] {} {{M_{C}} \bc \langle M_{P} \rangle }
  \and \\
  \inferrule* [lab=process] {} {{M_{P}} \bc M_{N} \;| \;P|M_{P} }
\end{mathpar}

\begin{definition}[contextual application] Given a context $M$, and
  process $P$, we define the \emph{contextual application}, $M[P] :=
  M\{P/\Box\}$. That is, the contextual application of M to P is the
  substitution of $P$ for $\Box$ in $M$.
\end{definition}

$\meaningof{-} : L \to \mathcal{P}(\pi)$

\begin{mathpar}
  \inferrule* [lab=collection] {} {\meaningof{true} = \pi, \and \meaningof{~E} = \pi \setminus \meaningof{E}, \and \meaningof{E_{1} \& E_{2}} = \meaningof{E_{1}} \cap \meaningof{E_{2}}}
\end{mathpar}

\begin{mathpar}
  \inferrule* [lab=structure] {} {\meaningof{0} = \{ P \in \pi | P \equiv 0 \}, \and \\ \meaningof{E_1 | E_2} = \{ P \in \pi | P \equiv P_{1} | P_{2}, P_{1} \in \meaningof{E_{1}}, P_{2} \in \meaningof{E_2}\} }
\end{mathpar}

\begin{mathpar}
 \inferrule* [lab=behavior] {} {\meaningof{\langle a?b \rangle E} = \{ P \in \pi | P \equiv Q | u?(y)P', \\ \and \\\\ \and \\ \;\;\; u \in \meaningof{a}, \forall z.P'\{z/y\} \in \meaningof{E\{z/b\}}\}, \and \\ \meaningof{a!E} = \{ P \in \pi | P \equiv Q | x!\langle P' \rangle, x \in \meaningof{a} P' \in \meaningof{E}\} }
\end{mathpar}

\begin{mathpar}
 \inferrule* [lab=nominal] {} {\meaningof{\quotep{E}} = \{ \quotep{P} \in \quotep{\pi} | P \in \meaningof{E} \}, \and \meaningof{\quotep{P}} = \{ \quotep{Q} \in \quotep{\pi} | P \equiv Q \} \and \\ \meaningof{@\quotep{E}} = \{ P \in \pi | P \equiv @x, x \in \meaningof{E} \}}
\end{mathpar}

\begin{eqnarray*}
  \\
  \meaningof{-} : TS \to ST
\end{eqnarray*}

\begin{eqnarray*}
  \\
  L : TS \to ST
\end{eqnarray*}

\begin{eqnarray*}
  \\
  P \models E \iff P \in \meaningof{E}
\end{eqnarray*}

\begin{eqnarray*}
  P \approx_{L} Q \iff \forall E \in L. P \models E \iff Q \models E
\end{eqnarray*}

\begin{eqnarray*}
  P \approx_{K} Q
\end{eqnarray*}

\begin{eqnarray*}
  P \approx Q
\end{eqnarray*}

$\approx_{K} = \approx = \approx_{L}$

\subsubsection{Contextual duality}

Note that contexts extend the quotation operation to a family of
operations from processes to names. Given a context, $M$, we can
define a \emph{nominal context}, $\quotep{M}$ by $\quotep{M}[P] :=
\quotep{M[P]}$. To foreshadow what is to come we observe that these
operations enjoy a duality with processes very much like the duality
between vectors and maps from vectors to scalars.

Further, because the calculus is essentially higher-order, we have a
correspondence between contexts and processes. More specifically,
given a name $x$ and a context $M$ we can construct $M^{*}_{x}$ such
that 

\begin{mathpar}
  M^{*}_{x} | \lift{x}{P} \red M[P]
\end{mathpar}

namely,

\begin{mathpar}
  M^{*}_{x} := x?(u).M[\dropn{u}]
\end{mathpar}

The dependence of $M^{*}_{x}$ on a name makes it an abstraction, 

\begin{mathpar}
  M^{*} := (x)x?(u).M[\dropn{u}]
\end{mathpar}

\subsection{Additional notation}

It will sometimes be convenient to denote the process a name
quotes. We already have the notation $x = \quotep{P}$, but it will be
convenient to introduce an alternate notation, $\procn{x}$, when we
want to emphasize the connection to the use of the name. Note that, by
virtue of name equivalence, $\quotep{\procn{x}} \nameeq x$; so, the
notation is consistent with previous definitions.

Further, because names have structure it is possible to effect
substitutions on the basis of that structure. This means we need to
upgrade our notation for substitutions, which we accomplish by
adapting comprehension notation. Thus,

\begin{mathpar}
  P\{ y / x : x \in S \}
\end{mathpar}

is interpreted to mean the process derived from P by replacing (in a
capture-avoiding manner) each occurrence of $x$ in $S$ by $y$. For example,

\begin{mathpar}
  P\{ \quotep{\procn{x}|\procn{x}} / x : x \in \freenames{P} \}
\end{mathpar}

will replace each (occurrence) of a free name $x$ in $P$ by
$\quotep{\procn{x}|\procn{x}}$.

Also, we will avail ourselves of the notation $x^{L}$ and $x^{R}$ to
denote injections of a name into disjoint copies of the name
space. There are numerous ways to accomplish this. One example can be
found in \cite{MeredithR05}. This notation overloads to vectors of
names: $\vec{x}^{\pi} := (x_{i}^{\pi} \; : \; 0 \leq i < |\vec{x}| )$ where $\pi \in \{L,R\}$.

We also use $P^{\Box} := P|\Box$.

In \cite{MeredithR05} an interpretation of the new operator is
given. It turns out that there are several possible interpretations
all enjoying the requisite algebraic properties of the operator (see
\cite{milner91polyadicpi}). We will therefore make liberal use of
$(\nu\; \vec{x})P$.

% subsection the_syntax_and_semantics_of_the_notation_system (end)   

\input{qm2pi.qmops} 

\input{qm2pi.sterngerlach} 

\input{qm2pi.metric} 

% section concurrent_process_calculi (end)

%\input{qm2pi.proofsketch}

% section proof sketch (end)

%\input{qm2pi.slviaknots} 

% section spatial logic via knots (end)

\input{qm2pi.conclusion}

% section conclusion (end)

%\input{qm2pi.dtcodes} 

% section wiring algorithm (end)

\input{qm2pi.ack} 

% section acknowledgments (end)

\newpage


\bibliographystyle{plain}   
\bibliography{../../biblios/main.bib}

\input{qm2pi.rhodetails}

\end{document}

 

%\documentclass[12pt]{llncs}
%\documentclass{jktr}

\usepackage[pdftex]{hyperref}                   
\usepackage {listings}
\usepackage {mathpartir}
\usepackage{bcprules}
%\usepackage{listings}
                       
\usepackage{graphicx} 
%\usepackage[margins=2.5cm,nohead,nofoot]{geometry}
%\usepackage{geometry}
\usepackage{amsfonts}
\usepackage{amstext}
\usepackage{latexsym}
\usepackage{amssymb}
\usepackage{color}


%\include{myPreamble}
\include{qm2pi.local} 

%\ifpdf
%\usepackage[pdftex]{graphicx}
%\else
%\usepackage{graphicx}
%\fi

 % \ifpdf
%  \usepackage{pdfsync}
%  \if


%\title{Brief Article}
%\author{David F. Snyder}
%\author{L.G. Meredith}

%\address{Dept. of Math., Texas State University--San Marcos, San Marcos, TX 78666}
       
\pagestyle{empty}


\begin{document}

\lstset{language=[Objective]Caml,frame=shadowbox}

\input{qm2pi.front}

% section front matter (end)

\input{qm2pi.intro} 
 
% section introduction (end)

% \input{qm2pi.knotations} 

% section notation (end)

\input{qm2pi.process.calculi} 

% section concurrent_process_calculi_and_spatial_logics_ (end)
    
%\input{qm2pi.knots2pi} 

%\input{qm2pi.trefoil} 

%\input{qm2pi.mainthm} 

% subsection basic_interpretation (end)

%\input{qm2pi.rho.presentation} 
\subsection{The syntax and semantics of the notation system}\label{sub:the_syntax_and_semantics_of_the_notation_system} % (fold)

We now summarize a technical presentation of the calculus that
embodies our theory of dynamics. The typical presentation of such a
calculus follows the style of giving generators and relations on
them. The grammar, below, describing term constructors, freely
generates the set of processes, $\Proc$. This set is then quotiented
by a relation known as structural congruence and it is over this set
that the notion of dynamics is expressed. This presentation is
essentially that of \cite{MeredithR05} with the addition of
polyadicity and summation. For readability we have relegated some of
the technical subtleties to an appendix.

\subsubsection{Process grammar}\label{subsub:process_grammar}

\begin{mathpar}
  \inferrule* [lab=synchronization] {} {{M} \bc \pzero \;|\; x?F \;|\; x!C }
  \and
  \inferrule* [lab=abstraction] {} {{F} \bc (x)P}
  \and
  \inferrule* [lab=concretion] {} {{C} \bc \langle Q \rangle}
  \and
  \inferrule* [lab=process] {} {{P,Q} \bc M \;| \;P|Q \;|\; @{x}}
  \and
  \inferrule* [lab=name] {} {{x} \bc \quotep{P}}
\end{mathpar} 

Note that $\vec{x}$ (resp. $\vec{P}$) denotes a vector of names
(resp. processes) of length $|\vec{x}|$ (resp. $|\vec{P}|$). We adopt
the following useful abbreviations.

\begin{mathpar}
   x?(\vec{y}).P := x.(\vec{y})P \and  x\clift{\vec{P}} := x.\clift{\vec{P}}
   \and x!(y) := \lift{x}{\dropn{y}}
   \and \Pi_{i=0}^{n-1}P_i := P_0 | \ldots | P_{n-1}
\end{mathpar}

\subsubsection{Structural congruence}

\paragraph{Free and bound names and alpha-equivalence.} At the
core of structural equivalence is alpha-equivalence which identifies
process that are the same up to a change of variable. Formally, we
recognize the distinction between free and bound names. The free names
of a process, $\freenames{P}$, may be calculated recursively as
follows:

\begin{mathpar}
\freenames{\pzero} := \emptyset
  \and \\
  \freenames{x?(y).P} := \{ x \} \cup (\freenames{P} \setminus \{ y \})
  \and 
  \freenames{x!\langle P \rangle} := \{ x \} \cup \{ P \} 
  \and \\
  \freenames{P|Q} := \freenames{P} \cup \freenames{Q}
  \and \\
  \freenames{@{x}} := \{ x \}
\end{mathpar}

$\pi$
$\quotep{\pi}$

$\freenames{-} : \pi \to \mathcal{P}(\quotep{\pi})$

\begin{eqnarray*}
  \freenames{\pzero} & := & \emptyset \\
  \freenames{x?(y).P} & := & \{ x \} \cup (\freenames{P} \setminus \{ y \}) \\
  \freenames{x!\langle P \rangle} & := & \{ x \} \cup \{ P \} \\
  \freenames{P|Q} & := & \freenames{P} \cup \freenames{Q} \\
  \freenames{\dropn{x}} & := & \{ x \}
\end{eqnarray*}

The bound names of a process, $\boundnames{P}$, are those names occurring in $P$
that are not free. For example, in $x?(y).0$, the name $x$ is free, while $y$ is bound.

\begin{mathpar}
  \inferrule* [lab=monoidal-laws] {} { P|Q \equiv Q|P \and P|0 \equiv P \and P|(Q|R) \equiv (P|Q)|R }
\end{mathpar}

\begin{mathpar}
  \inferrule* [lab=alpha-equivalence] {} { (x)P \equiv (y)P\{y/x\} \and y \not\in \freenames{P} }
\end{mathpar}

\begin{definition}
Then two processes, $P,Q$, are alpha-equivalent if $P = Q\{\vec{y}/\vec{x}\}$ for
some $\vec{x} \in \boundnames{Q},\vec{y} \in \boundnames{P}$, where $Q\{\vec{y}/\vec{x}\}$
denotes the capture-avoiding substitution of $\vec{y}$ for $\vec{x}$ in $Q$.
\end{definition}

\begin{definition}
  The {\em structural congruence} \cite{SangiorgiWalker} , $\equiv$,
  between processes is the least congruence containing
  alpha-equivalence, satisfying the abelian monoid laws
  (associativity, commutativity and $\pzero$ as identity) for parallel
  composition $|$ and for summation $+$.
\end{definition}

\subsection{Name equivalence}

We take name equivalence, written $\nameeq$, to be the smallest
equivalence relation generated by the following rules.

\begin{mathpar}
\inferrule*[lab=Quote-drop]
{ }
{ \quotep{@{x}} \nameeq x }

\inferrule*[lab=Struct-equiv]
{ P \scong Q }
{ \quotep{P} \nameeq \quotep{Q} }
\end{mathpar}

The astute reader will have noticed that the mutual recursion of names
and processes imposes a mutual recursion on alpha-equivalence and
structural equivalence via name-equivalence. Fortunately, all of this
works out pleasantly and we may calculate in the natural way, free of
concern. The reader interested in the details is referred to the
appendix \ref{appendix:rho_details}.

\subsection{Substitution}

We use $\Proc$ for the set of processes, $\QProc$ for the set of
names, and $\id{\{}\vec{y} / \vec{x} \id{\}}$ to denote partial maps,
$s : \QProc \rightarrow \QProc$. A map, $s$ lifts, uniquely, to a map
on process terms, $\widehat{s} : \Proc \rightarrow \Proc$ by the
following equations.

\begin{mathpar}
  (0) \psubstp{Q}{P} := 0 \\
  (R \juxtap S) \psubstp{Q}{P}
  :=    
  (R)\psubstp{Q}{P} \juxtap (S) \psubstp{Q}{P} \\
  (x?(y).R) \psubstp{Q}{P}    
  :=    
  (x)\substp{Q}{P} (z)\concat( (R \psubstn{z}{y}) \psubstp{Q}{P} ) \\
  (\lift{x}{R}) \psubstp{Q}{P}  
  :=
  \lift{(x)\substp{Q}{P}}{ R \psubstp{Q}{P} } \\
%   (\dropn{x})  \psubstp{Q}{P}       
%   := 
%   \left\{ 
%     \begin{array}{ccc} 
%       \dropn{\quotep{Q}} & & x \nameeq \quotep{P} \\
%       \dropn{x} & & otherwise \\
%     \end{array}
%   \right. 
  (\dropn{x})  \psubstp{Q}{P}       
  := 
  \left\{ 
    \begin{array}{ccc} 
      Q & & x \nameeq \quotep{P} \\
      \dropn{x} & & otherwise \\
    \end{array}
  \right.
\end{mathpar}
 

where

\begin{eqnarray}
  (x)\id{\{} \lpquote Q \rpquote / \lpquote P \rpquote \id{\}}            = 
  \left\{ 
    \begin{array}{ccc}
      \lpquote Q \rpquote & & x \nameeq \lpquote P \rpquote \\
      x & & otherwise \\
    \end{array}
  \right. \nonumber
\end{eqnarray}

and $z$ is chosen distinct from $\quotep{P}$, $\quotep{Q}$, the free
names in $Q$, and all the names in $R$. Our $\alpha$-equivalence will
be built in the standard way from this substitution.

\begin{remark}\label{rem:no_self_referential_names}
  One consequence of these definitions is that $\forall P. \quotep{P}
  \not\in \freenames{P}$.
\end{remark}

\subsection{ Dynamic quote: an example }

Anticipating something of what's to come, consider applying the
substitution, $\widehat{\id{\{}u / z \id{\}}}$, to the following pair
of processes, $\lift{w}{y!(z)}$ and $w[ \lpquote y!(z) \rpquote ]$.

\begin{eqnarray}
	\lift{w}{y!(z)}\widehat{\id{\{}u / z \id{\}}}
		& = &
		\lift{w}{y!(u)} \nonumber\\
	w[ \lpquote y!(z) \rpquote ] \widehat{ \id{\{}u / z \id{\}} }
		& = &
		w[ \lpquote y!(z) \rpquote ] \nonumber
\end{eqnarray}

Because the body of the process between quotes is impervious to
substitution, we get radically different answers. In fact, by
examining the first process in an input context,
e.g. $x?(z).\lift{w}{y!(z)}$, we see that the process under the lift
operator may be shaped by prefixed inputs binding a name inside it. In
this sense, the lift operator will be seen as a way to dynamically
construct processes before reifying them as names.

Finally equipped with these standard features we can present the
dynamics of the calculus.

\subsubsection{Operational semantics} 

Finally, we introduce the computational dynamics. What marks these
algebras as distinct from other more traditionally studied algebraic
structures, e.g. vector spaces or polynomial rings, is the manner in
which dynamics is captured. In traditional structures, dynamics is typically
expressed through morphisms between such structures, as in linear maps
between vector spaces or morphisms between rings. In algebras
associated with the semantics of computation, the dynamics is
expressed as part of the algebraic structure itself, through a
reduction reduction relation typically denoted by $\red$. Below, we
give a recursive presentation of this relation for the calculus used
in the encoding.

$\red \subseteq \pi \times \pi$
$\red : \pi \to \mathcal{P}(\pi)$

\begin{mathpar}
  \inferrule* [lab=Comm] { \textsf{match}( x_{src}, x_{trgt} ) } { x_{trgt}?(y)P \; | \; x_{src}!\langle {Q} \rangle \red P\{\quotep{Q}/y}\} }
  \and \\
  \inferrule* [lab=Par] {{P} \red {P}'} {{{P} | {Q}} \red {{P}' | {Q}}}
  \and
  \inferrule* [lab=Equiv]{{{P} \scong {P}'} \andalso {{P}' \red {Q}'} \andalso {{Q}' \scong {Q}}}{{P} \red {Q}}
\end{mathpar}

\begin{eqnarray*}
  match_{\equiv} (\quotep{P},\quotep{Q}) & := & P \equiv Q \\
  match_{\dagger}(\quotep{P},\quotep{Q}) & := & \forall R. P|Q \red^{*} R => R \red^{*} 0 \\
  match_{K}(\quotep{P},\quotep{Q}) & := & K \mbox{ for some context } K
\end{eqnarray*}

$u?(x)P | u!\langle Q \rangle \red P\{\quotep{Q}/x\}$

%We write $\wred$ for $\red^*$, and $P\red$ if $\exists Q $ such that $ P \red Q$.
We write $P\red$ if $\exists Q $ such that $ P \red Q$ and $P\not\red$, otherwise.

\section{Replication}

As mentioned before, it is known that replication (and hence
recursion) can be implemented in a higher-order process algebra
\cite{SangiorgiWalker}. As our first example of calculation with the
machinery thus far presented we give the construction explicitly in
the {\rhoc}.

\begin{eqnarray}
	D_{x} & := & \prefix{x}{y}{(\binpar{\outputp{x}{y}}{@{y}})} \nonumber\\
	\bangp_{x}{P} & := & \binpar{{x}!\langle{\binpar{D_{x}}{P}}\rangle}{D_{x}} \nonumber
\end{eqnarray}

\begin{eqnarray}
	\bangp_{x}{P} & & \nonumber\\
	=
	& {x}!\langle{(\prefix{x}{y}{(\outputp{x}{y} | @{y})) | P}}\rangle 
	      | \prefix{x}{y}{(\outputp{x}{y} | @{y})} & \nonumber\\
	\red
	& (\outputp{x}{y} | @{y})\substn{\quotep{(\prefix{x}{y}{(@{y} | \outputp{x}{y})) | P}}}{y} & \nonumber\\
	=
	& \outputp{x}{\quotep{(\prefix{x}{y}{(\outputp{x}{y} | @{y})) | P}}}
	  | {(\prefix{x}{y}{(\outputp{x}{y} | @{y})) | P}} & \nonumber\\
	\red
	& \ldots & \nonumber\\
	\red^*
	& P | P | \ldots & \nonumber
\end{eqnarray}

Of course, this encoding, as an implementation, runs away, unfolding
$\bangp{P}$ eagerly. A lazier and more implementable replication
operator, restricted to input-guarded processes, may be obtained as follows.

\begin{eqnarray}
\bangp{\prefix{u}{v}{P}} 
	:= 
	\binpar{\lift{x}{\prefix{u}{v}{(\binpar{D(x)}{P})}}}{D(x)} \nonumber
\end{eqnarray}

\begin{remark}
  Note that the lazier definition still does not deal with summation
  or mixed summation (i.e. sums over input and output). The reader is
  invited to construct definitions of replication that deal with these
  features. 

  Further, the definitions are parameterized in a name, $x$. Can you,
  gentle reader, make a definition that eliminates this parameter and
  guarantees no accidental interaction between the replication
  machinery and the process being replicated -- i.e. no accidental
  sharing of names used by the process to get its work done and the
  name(s) used by the replication to effect copying. This latter
  revision of the definition of replication is crucial to obtaining
  the expected identity $!!P \sim !P$.
\end{remark}

\begin{remark}\label{rem:paradoxical_combinator}
  The reader familiar with the lambda calculus will have noticed the
  similarity between $D$ and the paradoxical combinator.

  [Ed. note: the existence of this seems to suggest we have to be more
  restrictive on the set of processes and names we admit if we are to
  support no-cloning.]
\end{remark}

\subsubsection{Bisimulation}

The computational dynamics gives rise to another kind of equivalence,
the equivalence of computational behavior. As previously mentioned
this is typically captured \emph{via} some form of bisimulation.

% The notion we use in this paper is weak barbed bisimulation
% \cite{milner91polyadicpi}.

The notion we use in this paper is derived from weak barbed
bisimulation \cite{milner91polyadicpi}. 

\begin{definition}
An \emph{observation relation}, $\downarrow_{\mathcal N}$, over a set
of names, $\mathcal N$, is the smallest relation satisfying the rules
below.

\infrule[Out-barb]{y \in {\mathcal N}, \; x \nameeq y}
		  {\outputp{x}{v} \downarrow_{\mathcal N} x}
\infrule[Par-barb]{\mbox{$P\downarrow_{\mathcal N} x$ or $Q\downarrow_{\mathcal N} x$}}
		  {\binpar{P}{Q} \downarrow_{\mathcal N} x}

We write $P \Downarrow_{\mathcal N} x$ if there is $Q$ such that 
$P \wred Q$ and $Q \downarrow_{\mathcal N} x$.
\end{definition}

\begin{definition}
%\label{def.bbisim}
An  ${\mathcal N}$-\emph{barbed bisimulation} over a set of names, ${\mathcal N}$, is a symmetric binary relation 
${\mathcal S}_{\mathcal N}$ between agents such that $P\rel{S}_{\mathcal N}Q$ implies:
\begin{enumerate}
\item If $P \red P'$ then $Q \wred Q'$ and $P'\rel{S}_{\mathcal N} Q'$.
\item If $P\downarrow_{\mathcal N} x$, then $Q\Downarrow_{\mathcal N} x$.
\end{enumerate}
$P$ is ${\mathcal N}$-barbed bisimilar to $Q$, written
$P \wbbisim_{\mathcal N} Q$, if $P \rel{S}_{\mathcal N} Q$ for some ${\mathcal N}$-barbed bisimulation ${\mathcal S}_{\mathcal N}$.
\end{definition}

$\mathcal{R} \subseteq \pi \times \pi$

$P \mathcal{R} Q => \forall P'. P \red P' \Rightarrow \exists Q'. Q \red Q', P' \mathcal{R} Q'$

$P \vdash x \Rightarrow Q \vdash x$

\begin{mathpar}
  \inferrule*[lab=Out-barb]{x \nameeq y}{{y}!\langle{Q}\rangle \vdash x}
  \and
  \inferrule*[lab=Par-barb]{\mbox{$P\vdash x$ or $Q\vdash x$}}{\binpar{P}{Q} \vdash x}
\end{mathpar}

\subsubsection{Contexts}

One of the principle advantages of computational calculi like the
$\pi$-calculus is a well-defined notion of context,
contextual-equivalence and a correlation between
contextual-equivalence and notions of bisimulation. The notion of
context allows the decomposition of a process into (sub-)process and
its syntactic environment, its context. Thus, a context may be
thought of as a process with a ``hole'' (written $\Box$) in it. The
application of a context $M$ to a process $P$, written $M[P]$, is
tantamount to filling the hole in $M$ with $P$. In this paper we do
not need the full weight of this theory, but do make use of the notion
of context in the proof the main theorem. 

\begin{mathpar}
  \inferrule* [lab=summation] {} {{M_{M},M_{N}} \bc \Box \;|\; x.M_{A} \;|\; M_{M}+M_{N}}
  \and
  \inferrule* [lab=agent] {} {{M_{A}} \bc (\vec{x})M_{P} \;| \; \clift{P_0,\ldots,M_{P},\ldots,P_N}}
  \and \\
  \inferrule* [lab=process] {} {{M_{P}} \bc M_{N} \;| \;P|M_{P} }
\end{mathpar} 

\begin{mathpar}
  \inferrule* [lab=sychronization] {} {M_{N} \bc \Box \;|\; x?M_{F} \;|\; x!M_{C}}
  \and
  \inferrule* [lab=abstraction] {} {{M_{F}} \bc (x)M_{P} }
  \and
  \inferrule* [lab=concretion] {} {{M_{C}} \bc \langle M_{P} \rangle }
  \and \\
  \inferrule* [lab=process] {} {{M_{P}} \bc M_{N} \;| \;P|M_{P} }
\end{mathpar}

\begin{definition}[contextual application] Given a context $M$, and
  process $P$, we define the \emph{contextual application}, $M[P] :=
  M\{P/\Box\}$. That is, the contextual application of M to P is the
  substitution of $P$ for $\Box$ in $M$.
\end{definition}

$\meaningof{-} : L \to \mathcal{P}(\pi)$

\begin{mathpar}
  \inferrule* [lab=collection] {} {\meaningof{true} = \pi, \and \meaningof{~E} = \pi \setminus \meaningof{E}, \and \meaningof{E_{1} \& E_{2}} = \meaningof{E_{1}} \cap \meaningof{E_{2}}}
\end{mathpar}

\begin{mathpar}
  \inferrule* [lab=structure] {} {\meaningof{0} = \{ P \in \pi | P \equiv 0 \}, \and \\ \meaningof{E_1 | E_2} = \{ P \in \pi | P \equiv P_{1} | P_{2}, P_{1} \in \meaningof{E_{1}}, P_{2} \in \meaningof{E_2}\} }
\end{mathpar}

\begin{mathpar}
 \inferrule* [lab=behavior] {} {\meaningof{\langle a?b \rangle E} = \{ P \in \pi | P \equiv Q | u?(y)P', \\ \and \\\\ \and \\ \;\;\; u \in \meaningof{a}, \forall z.P'\{z/y\} \in \meaningof{E\{z/b\}}\}, \and \\ \meaningof{a!E} = \{ P \in \pi | P \equiv Q | x!\langle P' \rangle, x \in \meaningof{a} P' \in \meaningof{E}\} }
\end{mathpar}

\begin{mathpar}
 \inferrule* [lab=nominal] {} {\meaningof{\quotep{E}} = \{ \quotep{P} \in \quotep{\pi} | P \in \meaningof{E} \}, \and \meaningof{\quotep{P}} = \{ \quotep{Q} \in \quotep{\pi} | P \equiv Q \} \and \\ \meaningof{@\quotep{E}} = \{ P \in \pi | P \equiv @x, x \in \meaningof{E} \}}
\end{mathpar}

\begin{eqnarray*}
  \\
  \meaningof{-} : TS \to ST
\end{eqnarray*}

\begin{eqnarray*}
  \\
  L : TS \to ST
\end{eqnarray*}

\begin{eqnarray*}
  \\
  P \models E \iff P \in \meaningof{E}
\end{eqnarray*}

\begin{eqnarray*}
  P \approx_{L} Q \iff \forall E \in L. P \models E \iff Q \models E
\end{eqnarray*}

\begin{eqnarray*}
  P \approx_{K} Q
\end{eqnarray*}

\begin{eqnarray*}
  P \approx Q
\end{eqnarray*}

$\approx_{K} = \approx = \approx_{L}$

\subsubsection{Contextual duality}

Note that contexts extend the quotation operation to a family of
operations from processes to names. Given a context, $M$, we can
define a \emph{nominal context}, $\quotep{M}$ by $\quotep{M}[P] :=
\quotep{M[P]}$. To foreshadow what is to come we observe that these
operations enjoy a duality with processes very much like the duality
between vectors and maps from vectors to scalars.

Further, because the calculus is essentially higher-order, we have a
correspondence between contexts and processes. More specifically,
given a name $x$ and a context $M$ we can construct $M^{*}_{x}$ such
that 

\begin{mathpar}
  M^{*}_{x} | \lift{x}{P} \red M[P]
\end{mathpar}

namely,

\begin{mathpar}
  M^{*}_{x} := x?(u).M[\dropn{u}]
\end{mathpar}

The dependence of $M^{*}_{x}$ on a name makes it an abstraction, 

\begin{mathpar}
  M^{*} := (x)x?(u).M[\dropn{u}]
\end{mathpar}

\subsection{Additional notation}

It will sometimes be convenient to denote the process a name
quotes. We already have the notation $x = \quotep{P}$, but it will be
convenient to introduce an alternate notation, $\procn{x}$, when we
want to emphasize the connection to the use of the name. Note that, by
virtue of name equivalence, $\quotep{\procn{x}} \nameeq x$; so, the
notation is consistent with previous definitions.

Further, because names have structure it is possible to effect
substitutions on the basis of that structure. This means we need to
upgrade our notation for substitutions, which we accomplish by
adapting comprehension notation. Thus,

\begin{mathpar}
  P\{ y / x : x \in S \}
\end{mathpar}

is interpreted to mean the process derived from P by replacing (in a
capture-avoiding manner) each occurrence of $x$ in $S$ by $y$. For example,

\begin{mathpar}
  P\{ \quotep{\procn{x}|\procn{x}} / x : x \in \freenames{P} \}
\end{mathpar}

will replace each (occurrence) of a free name $x$ in $P$ by
$\quotep{\procn{x}|\procn{x}}$.

Also, we will avail ourselves of the notation $x^{L}$ and $x^{R}$ to
denote injections of a name into disjoint copies of the name
space. There are numerous ways to accomplish this. One example can be
found in \cite{MeredithR05}. This notation overloads to vectors of
names: $\vec{x}^{\pi} := (x_{i}^{\pi} \; : \; 0 \leq i < |\vec{x}| )$ where $\pi \in \{L,R\}$.

We also use $P^{\Box} := P|\Box$.

In \cite{MeredithR05} an interpretation of the new operator is
given. It turns out that there are several possible interpretations
all enjoying the requisite algebraic properties of the operator (see
\cite{milner91polyadicpi}). We will therefore make liberal use of
$(\nu\; \vec{x})P$.

% subsection the_syntax_and_semantics_of_the_notation_system (end)   

\input{qm2pi.qmops} 

\input{qm2pi.sterngerlach} 

\input{qm2pi.metric} 

% section concurrent_process_calculi (end)

%\input{qm2pi.proofsketch}

% section proof sketch (end)

%\input{qm2pi.slviaknots} 

% section spatial logic via knots (end)

\input{qm2pi.conclusion}

% section conclusion (end)

%\input{qm2pi.dtcodes} 

% section wiring algorithm (end)

\input{qm2pi.ack} 

% section acknowledgments (end)

\newpage


\bibliographystyle{plain}   
\bibliography{../../biblios/main.bib}

\input{qm2pi.rhodetails}

\end{document}

 

% subsection basic_interpretation (end)

%\input{qm2pi.rho.presentation} 
\subsection{The syntax and semantics of the notation system}\label{sub:the_syntax_and_semantics_of_the_notation_system} % (fold)

We now summarize a technical presentation of the calculus that
embodies our theory of dynamics. The typical presentation of such a
calculus follows the style of giving generators and relations on
them. The grammar, below, describing term constructors, freely
generates the set of processes, $\Proc$. This set is then quotiented
by a relation known as structural congruence and it is over this set
that the notion of dynamics is expressed. This presentation is
essentially that of \cite{MeredithR05} with the addition of
polyadicity and summation. For readability we have relegated some of
the technical subtleties to an appendix.

\subsubsection{Process grammar}\label{subsub:process_grammar}

\begin{mathpar}
  \inferrule* [lab=synchronization] {} {{M} \bc \pzero \;|\; x?F \;|\; x!C }
  \and
  \inferrule* [lab=abstraction] {} {{F} \bc (x)P}
  \and
  \inferrule* [lab=concretion] {} {{C} \bc \langle Q \rangle}
  \and
  \inferrule* [lab=process] {} {{P,Q} \bc M \;| \;P|Q \;|\; @{x}}
  \and
  \inferrule* [lab=name] {} {{x} \bc \quotep{P}}
\end{mathpar} 

Note that $\vec{x}$ (resp. $\vec{P}$) denotes a vector of names
(resp. processes) of length $|\vec{x}|$ (resp. $|\vec{P}|$). We adopt
the following useful abbreviations.

\begin{mathpar}
   x?(\vec{y}).P := x.(\vec{y})P \and  x\clift{\vec{P}} := x.\clift{\vec{P}}
   \and x!(y) := \lift{x}{\dropn{y}}
   \and \Pi_{i=0}^{n-1}P_i := P_0 | \ldots | P_{n-1}
\end{mathpar}

\subsubsection{Structural congruence}

\paragraph{Free and bound names and alpha-equivalence.} At the
core of structural equivalence is alpha-equivalence which identifies
process that are the same up to a change of variable. Formally, we
recognize the distinction between free and bound names. The free names
of a process, $\freenames{P}$, may be calculated recursively as
follows:

\begin{mathpar}
\freenames{\pzero} := \emptyset
  \and \\
  \freenames{x?(y).P} := \{ x \} \cup (\freenames{P} \setminus \{ y \})
  \and 
  \freenames{x!\langle P \rangle} := \{ x \} \cup \{ P \} 
  \and \\
  \freenames{P|Q} := \freenames{P} \cup \freenames{Q}
  \and \\
  \freenames{@{x}} := \{ x \}
\end{mathpar}

$\pi$
$\quotep{\pi}$

$\freenames{-} : \pi \to \mathcal{P}(\quotep{\pi})$

\begin{eqnarray*}
  \freenames{\pzero} & := & \emptyset \\
  \freenames{x?(y).P} & := & \{ x \} \cup (\freenames{P} \setminus \{ y \}) \\
  \freenames{x!\langle P \rangle} & := & \{ x \} \cup \{ P \} \\
  \freenames{P|Q} & := & \freenames{P} \cup \freenames{Q} \\
  \freenames{\dropn{x}} & := & \{ x \}
\end{eqnarray*}

The bound names of a process, $\boundnames{P}$, are those names occurring in $P$
that are not free. For example, in $x?(y).0$, the name $x$ is free, while $y$ is bound.

\begin{mathpar}
  \inferrule* [lab=monoidal-laws] {} { P|Q \equiv Q|P \and P|0 \equiv P \and P|(Q|R) \equiv (P|Q)|R }
\end{mathpar}

\begin{mathpar}
  \inferrule* [lab=alpha-equivalence] {} { (x)P \equiv (y)P\{y/x\} \and y \not\in \freenames{P} }
\end{mathpar}

\begin{definition}
Then two processes, $P,Q$, are alpha-equivalent if $P = Q\{\vec{y}/\vec{x}\}$ for
some $\vec{x} \in \boundnames{Q},\vec{y} \in \boundnames{P}$, where $Q\{\vec{y}/\vec{x}\}$
denotes the capture-avoiding substitution of $\vec{y}$ for $\vec{x}$ in $Q$.
\end{definition}

\begin{definition}
  The {\em structural congruence} \cite{SangiorgiWalker} , $\equiv$,
  between processes is the least congruence containing
  alpha-equivalence, satisfying the abelian monoid laws
  (associativity, commutativity and $\pzero$ as identity) for parallel
  composition $|$ and for summation $+$.
\end{definition}

\subsection{Name equivalence}

We take name equivalence, written $\nameeq$, to be the smallest
equivalence relation generated by the following rules.

\begin{mathpar}
\inferrule*[lab=Quote-drop]
{ }
{ \quotep{@{x}} \nameeq x }

\inferrule*[lab=Struct-equiv]
{ P \scong Q }
{ \quotep{P} \nameeq \quotep{Q} }
\end{mathpar}

The astute reader will have noticed that the mutual recursion of names
and processes imposes a mutual recursion on alpha-equivalence and
structural equivalence via name-equivalence. Fortunately, all of this
works out pleasantly and we may calculate in the natural way, free of
concern. The reader interested in the details is referred to the
appendix \ref{appendix:rho_details}.

\subsection{Substitution}

We use $\Proc$ for the set of processes, $\QProc$ for the set of
names, and $\id{\{}\vec{y} / \vec{x} \id{\}}$ to denote partial maps,
$s : \QProc \rightarrow \QProc$. A map, $s$ lifts, uniquely, to a map
on process terms, $\widehat{s} : \Proc \rightarrow \Proc$ by the
following equations.

\begin{mathpar}
  (0) \psubstp{Q}{P} := 0 \\
  (R \juxtap S) \psubstp{Q}{P}
  :=    
  (R)\psubstp{Q}{P} \juxtap (S) \psubstp{Q}{P} \\
  (x?(y).R) \psubstp{Q}{P}    
  :=    
  (x)\substp{Q}{P} (z)\concat( (R \psubstn{z}{y}) \psubstp{Q}{P} ) \\
  (\lift{x}{R}) \psubstp{Q}{P}  
  :=
  \lift{(x)\substp{Q}{P}}{ R \psubstp{Q}{P} } \\
%   (\dropn{x})  \psubstp{Q}{P}       
%   := 
%   \left\{ 
%     \begin{array}{ccc} 
%       \dropn{\quotep{Q}} & & x \nameeq \quotep{P} \\
%       \dropn{x} & & otherwise \\
%     \end{array}
%   \right. 
  (\dropn{x})  \psubstp{Q}{P}       
  := 
  \left\{ 
    \begin{array}{ccc} 
      Q & & x \nameeq \quotep{P} \\
      \dropn{x} & & otherwise \\
    \end{array}
  \right.
\end{mathpar}
 

where

\begin{eqnarray}
  (x)\id{\{} \lpquote Q \rpquote / \lpquote P \rpquote \id{\}}            = 
  \left\{ 
    \begin{array}{ccc}
      \lpquote Q \rpquote & & x \nameeq \lpquote P \rpquote \\
      x & & otherwise \\
    \end{array}
  \right. \nonumber
\end{eqnarray}

and $z$ is chosen distinct from $\quotep{P}$, $\quotep{Q}$, the free
names in $Q$, and all the names in $R$. Our $\alpha$-equivalence will
be built in the standard way from this substitution.

\begin{remark}\label{rem:no_self_referential_names}
  One consequence of these definitions is that $\forall P. \quotep{P}
  \not\in \freenames{P}$.
\end{remark}

\subsection{ Dynamic quote: an example }

Anticipating something of what's to come, consider applying the
substitution, $\widehat{\id{\{}u / z \id{\}}}$, to the following pair
of processes, $\lift{w}{y!(z)}$ and $w[ \lpquote y!(z) \rpquote ]$.

\begin{eqnarray}
	\lift{w}{y!(z)}\widehat{\id{\{}u / z \id{\}}}
		& = &
		\lift{w}{y!(u)} \nonumber\\
	w[ \lpquote y!(z) \rpquote ] \widehat{ \id{\{}u / z \id{\}} }
		& = &
		w[ \lpquote y!(z) \rpquote ] \nonumber
\end{eqnarray}

Because the body of the process between quotes is impervious to
substitution, we get radically different answers. In fact, by
examining the first process in an input context,
e.g. $x?(z).\lift{w}{y!(z)}$, we see that the process under the lift
operator may be shaped by prefixed inputs binding a name inside it. In
this sense, the lift operator will be seen as a way to dynamically
construct processes before reifying them as names.

Finally equipped with these standard features we can present the
dynamics of the calculus.

\subsubsection{Operational semantics} 

Finally, we introduce the computational dynamics. What marks these
algebras as distinct from other more traditionally studied algebraic
structures, e.g. vector spaces or polynomial rings, is the manner in
which dynamics is captured. In traditional structures, dynamics is typically
expressed through morphisms between such structures, as in linear maps
between vector spaces or morphisms between rings. In algebras
associated with the semantics of computation, the dynamics is
expressed as part of the algebraic structure itself, through a
reduction reduction relation typically denoted by $\red$. Below, we
give a recursive presentation of this relation for the calculus used
in the encoding.

$\red \subseteq \pi \times \pi$
$\red : \pi \to \mathcal{P}(\pi)$

\begin{mathpar}
  \inferrule* [lab=Comm] { \textsf{match}( x_{src}, x_{trgt} ) } { x_{trgt}?(y)P \; | \; x_{src}!\langle {Q} \rangle \red P\{\quotep{Q}/y}\} }
  \and \\
  \inferrule* [lab=Par] {{P} \red {P}'} {{{P} | {Q}} \red {{P}' | {Q}}}
  \and
  \inferrule* [lab=Equiv]{{{P} \scong {P}'} \andalso {{P}' \red {Q}'} \andalso {{Q}' \scong {Q}}}{{P} \red {Q}}
\end{mathpar}

\begin{eqnarray*}
  match_{\equiv} (\quotep{P},\quotep{Q}) & := & P \equiv Q \\
  match_{\dagger}(\quotep{P},\quotep{Q}) & := & \forall R. P|Q \red^{*} R => R \red^{*} 0 \\
  match_{K}(\quotep{P},\quotep{Q}) & := & K \mbox{ for some context } K
\end{eqnarray*}

$u?(x)P | u!\langle Q \rangle \red P\{\quotep{Q}/x\}$

%We write $\wred$ for $\red^*$, and $P\red$ if $\exists Q $ such that $ P \red Q$.
We write $P\red$ if $\exists Q $ such that $ P \red Q$ and $P\not\red$, otherwise.

\section{Replication}

As mentioned before, it is known that replication (and hence
recursion) can be implemented in a higher-order process algebra
\cite{SangiorgiWalker}. As our first example of calculation with the
machinery thus far presented we give the construction explicitly in
the {\rhoc}.

\begin{eqnarray}
	D_{x} & := & \prefix{x}{y}{(\binpar{\outputp{x}{y}}{@{y}})} \nonumber\\
	\bangp_{x}{P} & := & \binpar{{x}!\langle{\binpar{D_{x}}{P}}\rangle}{D_{x}} \nonumber
\end{eqnarray}

\begin{eqnarray}
	\bangp_{x}{P} & & \nonumber\\
	=
	& {x}!\langle{(\prefix{x}{y}{(\outputp{x}{y} | @{y})) | P}}\rangle 
	      | \prefix{x}{y}{(\outputp{x}{y} | @{y})} & \nonumber\\
	\red
	& (\outputp{x}{y} | @{y})\substn{\quotep{(\prefix{x}{y}{(@{y} | \outputp{x}{y})) | P}}}{y} & \nonumber\\
	=
	& \outputp{x}{\quotep{(\prefix{x}{y}{(\outputp{x}{y} | @{y})) | P}}}
	  | {(\prefix{x}{y}{(\outputp{x}{y} | @{y})) | P}} & \nonumber\\
	\red
	& \ldots & \nonumber\\
	\red^*
	& P | P | \ldots & \nonumber
\end{eqnarray}

Of course, this encoding, as an implementation, runs away, unfolding
$\bangp{P}$ eagerly. A lazier and more implementable replication
operator, restricted to input-guarded processes, may be obtained as follows.

\begin{eqnarray}
\bangp{\prefix{u}{v}{P}} 
	:= 
	\binpar{\lift{x}{\prefix{u}{v}{(\binpar{D(x)}{P})}}}{D(x)} \nonumber
\end{eqnarray}

\begin{remark}
  Note that the lazier definition still does not deal with summation
  or mixed summation (i.e. sums over input and output). The reader is
  invited to construct definitions of replication that deal with these
  features. 

  Further, the definitions are parameterized in a name, $x$. Can you,
  gentle reader, make a definition that eliminates this parameter and
  guarantees no accidental interaction between the replication
  machinery and the process being replicated -- i.e. no accidental
  sharing of names used by the process to get its work done and the
  name(s) used by the replication to effect copying. This latter
  revision of the definition of replication is crucial to obtaining
  the expected identity $!!P \sim !P$.
\end{remark}

\begin{remark}\label{rem:paradoxical_combinator}
  The reader familiar with the lambda calculus will have noticed the
  similarity between $D$ and the paradoxical combinator.

  [Ed. note: the existence of this seems to suggest we have to be more
  restrictive on the set of processes and names we admit if we are to
  support no-cloning.]
\end{remark}

\subsubsection{Bisimulation}

The computational dynamics gives rise to another kind of equivalence,
the equivalence of computational behavior. As previously mentioned
this is typically captured \emph{via} some form of bisimulation.

% The notion we use in this paper is weak barbed bisimulation
% \cite{milner91polyadicpi}.

The notion we use in this paper is derived from weak barbed
bisimulation \cite{milner91polyadicpi}. 

\begin{definition}
An \emph{observation relation}, $\downarrow_{\mathcal N}$, over a set
of names, $\mathcal N$, is the smallest relation satisfying the rules
below.

\infrule[Out-barb]{y \in {\mathcal N}, \; x \nameeq y}
		  {\outputp{x}{v} \downarrow_{\mathcal N} x}
\infrule[Par-barb]{\mbox{$P\downarrow_{\mathcal N} x$ or $Q\downarrow_{\mathcal N} x$}}
		  {\binpar{P}{Q} \downarrow_{\mathcal N} x}

We write $P \Downarrow_{\mathcal N} x$ if there is $Q$ such that 
$P \wred Q$ and $Q \downarrow_{\mathcal N} x$.
\end{definition}

\begin{definition}
%\label{def.bbisim}
An  ${\mathcal N}$-\emph{barbed bisimulation} over a set of names, ${\mathcal N}$, is a symmetric binary relation 
${\mathcal S}_{\mathcal N}$ between agents such that $P\rel{S}_{\mathcal N}Q$ implies:
\begin{enumerate}
\item If $P \red P'$ then $Q \wred Q'$ and $P'\rel{S}_{\mathcal N} Q'$.
\item If $P\downarrow_{\mathcal N} x$, then $Q\Downarrow_{\mathcal N} x$.
\end{enumerate}
$P$ is ${\mathcal N}$-barbed bisimilar to $Q$, written
$P \wbbisim_{\mathcal N} Q$, if $P \rel{S}_{\mathcal N} Q$ for some ${\mathcal N}$-barbed bisimulation ${\mathcal S}_{\mathcal N}$.
\end{definition}

$\mathcal{R} \subseteq \pi \times \pi$

$P \mathcal{R} Q => \forall P'. P \red P' \Rightarrow \exists Q'. Q \red Q', P' \mathcal{R} Q'$

$P \vdash x \Rightarrow Q \vdash x$

\begin{mathpar}
  \inferrule*[lab=Out-barb]{x \nameeq y}{{y}!\langle{Q}\rangle \vdash x}
  \and
  \inferrule*[lab=Par-barb]{\mbox{$P\vdash x$ or $Q\vdash x$}}{\binpar{P}{Q} \vdash x}
\end{mathpar}

\subsubsection{Contexts}

One of the principle advantages of computational calculi like the
$\pi$-calculus is a well-defined notion of context,
contextual-equivalence and a correlation between
contextual-equivalence and notions of bisimulation. The notion of
context allows the decomposition of a process into (sub-)process and
its syntactic environment, its context. Thus, a context may be
thought of as a process with a ``hole'' (written $\Box$) in it. The
application of a context $M$ to a process $P$, written $M[P]$, is
tantamount to filling the hole in $M$ with $P$. In this paper we do
not need the full weight of this theory, but do make use of the notion
of context in the proof the main theorem. 

\begin{mathpar}
  \inferrule* [lab=summation] {} {{M_{M},M_{N}} \bc \Box \;|\; x.M_{A} \;|\; M_{M}+M_{N}}
  \and
  \inferrule* [lab=agent] {} {{M_{A}} \bc (\vec{x})M_{P} \;| \; \clift{P_0,\ldots,M_{P},\ldots,P_N}}
  \and \\
  \inferrule* [lab=process] {} {{M_{P}} \bc M_{N} \;| \;P|M_{P} }
\end{mathpar} 

\begin{mathpar}
  \inferrule* [lab=sychronization] {} {M_{N} \bc \Box \;|\; x?M_{F} \;|\; x!M_{C}}
  \and
  \inferrule* [lab=abstraction] {} {{M_{F}} \bc (x)M_{P} }
  \and
  \inferrule* [lab=concretion] {} {{M_{C}} \bc \langle M_{P} \rangle }
  \and \\
  \inferrule* [lab=process] {} {{M_{P}} \bc M_{N} \;| \;P|M_{P} }
\end{mathpar}

\begin{definition}[contextual application] Given a context $M$, and
  process $P$, we define the \emph{contextual application}, $M[P] :=
  M\{P/\Box\}$. That is, the contextual application of M to P is the
  substitution of $P$ for $\Box$ in $M$.
\end{definition}

$\meaningof{-} : L \to \mathcal{P}(\pi)$

\begin{mathpar}
  \inferrule* [lab=collection] {} {\meaningof{true} = \pi, \and \meaningof{~E} = \pi \setminus \meaningof{E}, \and \meaningof{E_{1} \& E_{2}} = \meaningof{E_{1}} \cap \meaningof{E_{2}}}
\end{mathpar}

\begin{mathpar}
  \inferrule* [lab=structure] {} {\meaningof{0} = \{ P \in \pi | P \equiv 0 \}, \and \\ \meaningof{E_1 | E_2} = \{ P \in \pi | P \equiv P_{1} | P_{2}, P_{1} \in \meaningof{E_{1}}, P_{2} \in \meaningof{E_2}\} }
\end{mathpar}

\begin{mathpar}
 \inferrule* [lab=behavior] {} {\meaningof{\langle a?b \rangle E} = \{ P \in \pi | P \equiv Q | u?(y)P', \\ \and \\\\ \and \\ \;\;\; u \in \meaningof{a}, \forall z.P'\{z/y\} \in \meaningof{E\{z/b\}}\}, \and \\ \meaningof{a!E} = \{ P \in \pi | P \equiv Q | x!\langle P' \rangle, x \in \meaningof{a} P' \in \meaningof{E}\} }
\end{mathpar}

\begin{mathpar}
 \inferrule* [lab=nominal] {} {\meaningof{\quotep{E}} = \{ \quotep{P} \in \quotep{\pi} | P \in \meaningof{E} \}, \and \meaningof{\quotep{P}} = \{ \quotep{Q} \in \quotep{\pi} | P \equiv Q \} \and \\ \meaningof{@\quotep{E}} = \{ P \in \pi | P \equiv @x, x \in \meaningof{E} \}}
\end{mathpar}

\begin{eqnarray*}
  \\
  \meaningof{-} : TS \to ST
\end{eqnarray*}

\begin{eqnarray*}
  \\
  L : TS \to ST
\end{eqnarray*}

\begin{eqnarray*}
  \\
  P \models E \iff P \in \meaningof{E}
\end{eqnarray*}

\begin{eqnarray*}
  P \approx_{L} Q \iff \forall E \in L. P \models E \iff Q \models E
\end{eqnarray*}

\begin{eqnarray*}
  P \approx_{K} Q
\end{eqnarray*}

\begin{eqnarray*}
  P \approx Q
\end{eqnarray*}

$\approx_{K} = \approx = \approx_{L}$

\subsubsection{Contextual duality}

Note that contexts extend the quotation operation to a family of
operations from processes to names. Given a context, $M$, we can
define a \emph{nominal context}, $\quotep{M}$ by $\quotep{M}[P] :=
\quotep{M[P]}$. To foreshadow what is to come we observe that these
operations enjoy a duality with processes very much like the duality
between vectors and maps from vectors to scalars.

Further, because the calculus is essentially higher-order, we have a
correspondence between contexts and processes. More specifically,
given a name $x$ and a context $M$ we can construct $M^{*}_{x}$ such
that 

\begin{mathpar}
  M^{*}_{x} | \lift{x}{P} \red M[P]
\end{mathpar}

namely,

\begin{mathpar}
  M^{*}_{x} := x?(u).M[\dropn{u}]
\end{mathpar}

The dependence of $M^{*}_{x}$ on a name makes it an abstraction, 

\begin{mathpar}
  M^{*} := (x)x?(u).M[\dropn{u}]
\end{mathpar}

\subsection{Additional notation}

It will sometimes be convenient to denote the process a name
quotes. We already have the notation $x = \quotep{P}$, but it will be
convenient to introduce an alternate notation, $\procn{x}$, when we
want to emphasize the connection to the use of the name. Note that, by
virtue of name equivalence, $\quotep{\procn{x}} \nameeq x$; so, the
notation is consistent with previous definitions.

Further, because names have structure it is possible to effect
substitutions on the basis of that structure. This means we need to
upgrade our notation for substitutions, which we accomplish by
adapting comprehension notation. Thus,

\begin{mathpar}
  P\{ y / x : x \in S \}
\end{mathpar}

is interpreted to mean the process derived from P by replacing (in a
capture-avoiding manner) each occurrence of $x$ in $S$ by $y$. For example,

\begin{mathpar}
  P\{ \quotep{\procn{x}|\procn{x}} / x : x \in \freenames{P} \}
\end{mathpar}

will replace each (occurrence) of a free name $x$ in $P$ by
$\quotep{\procn{x}|\procn{x}}$.

Also, we will avail ourselves of the notation $x^{L}$ and $x^{R}$ to
denote injections of a name into disjoint copies of the name
space. There are numerous ways to accomplish this. One example can be
found in \cite{MeredithR05}. This notation overloads to vectors of
names: $\vec{x}^{\pi} := (x_{i}^{\pi} \; : \; 0 \leq i < |\vec{x}| )$ where $\pi \in \{L,R\}$.

We also use $P^{\Box} := P|\Box$.

In \cite{MeredithR05} an interpretation of the new operator is
given. It turns out that there are several possible interpretations
all enjoying the requisite algebraic properties of the operator (see
\cite{milner91polyadicpi}). We will therefore make liberal use of
$(\nu\; \vec{x})P$.

% subsection the_syntax_and_semantics_of_the_notation_system (end)   

\section{Interpretation of QM}
\subsection{Supporting definitions}
\subsubsection{Multiplication}
\begin{mathpar}
  \quotep{Q} \cdot \quotep{R} := \quotep{Q|R}
  \and \\
  \quotep{Q} \cdot P := P\{ \quotep{Q|R} / \quotep{R} : \quotep{R} \in \freenames{P} \}
\end{mathpar}

\paragraph{Discussion}
The first line needs little explanation. The second line says that
each free name of the process is replaced with the multiplication of
that name by the scalar. Multiplication of a scalar (name) by a state
(process) results in a process all the names of which have been `moved
over' by parallel composition with the process the scalar
quotes. There is a subtlety that the bound names have to be
manipulated so that multiplied names aren't accidentally
captured. There are many ways to achieve this.

\begin{remark}\label{rem:multiplication_identities}
  The reader is invited to verify that for all $x,y,z \in \QProc$ and $P \in \Proc$
  \begin{mathpar}
    x \cdot \quotep{0} \equiv x 
    \and
    x \cdot y \equiv y \cdot x
    \and
    x \cdot (y \cdot z) \equiv (x \cdot y) \cdot z
    \and \\
    \quotep{0} \cdot P \equiv P
    \and \\
    x \cdot (y \cdot P) \equiv (x \cdot y) \cdot P
    \and \\
    x \cdot (P|Q) \equiv (x \cdot P) | (x \cdot Q)
    \and \\    
  \end{mathpar}
\end{remark}

\subsubsection{Tensor product}

We define a tensor product on processes by structural induction.

\paragraph{Tensor of sums} First note that all summations, including
$\pzero$ and sequence, can be written $\Sigma_{i} x_{i}.A_{i} +
\Sigma_{j} x_{j}.C_{j}$, where we have grouped input-guarded processes
together and output-guarded processes together.

Thus, we can define the tensor product of two summations, $N_{1}\otimes N_{2}$, where

\begin{mathpar}
  N_{1} := \Sigma_{i} x_{i}.A_{i} + \Sigma_{j} x_{j}.C_{j}
  \and
  N_{2} := \Sigma_{i'} y_{i'}.B_{i'} + \Sigma_{j'} y_{j'}.D_{j'} 
\end{mathpar}

as follows.

\begin{mathpar}
  \Sigma_{i} x_{i}.A_{i} + \Sigma_{j} x_{j}.C_{j} \otimes \Sigma_{i'}
  y_{i'}.B_{i'} + \Sigma_{j'} y_{j'}.D_{j'} 
  \and \\
  := \; \Sigma_{i} \Sigma_{i'} \quotep{\stackrel{\vee}{x_{i}}| \stackrel{\vee}{y_{i'}}}.(A_{i}\otimes B_{i'}) \; | \; \Sigma_{i'} \Sigma_{i} \quotep{\stackrel{\vee}{y_{i'}}|\stackrel{\vee}{x_{i}}}.(B_{i'}\otimes A_{i})
  \and
  \;\; | \;\; \Sigma_{j} \Sigma_{j'} \quotep{\stackrel{\vee}{x_{j}}|\stackrel{\vee}{y_{j'}}}.(A_{j}\otimes B_{j'}) \; | \; \Sigma_{j'} \Sigma_{j} \quotep{\stackrel{\vee}{y_{j'}}|\stackrel{\vee}{x_{j}}}.(B_{j'}\otimes A_{j})
\end{mathpar}

\begin{remark}
  Do we need to $x^{L}$ and $y^{R}$ for this construction as well?
\end{remark}

\paragraph{Tensor of parallel compositions} Next, we distribute tensor
over par.

\begin{mathpar}
  P_{1}|P_{2} \otimes Q_{1}|Q_{2} := (P_{1} \otimes Q_{1}) | (P_{1}
  \otimes Q_{2}) | (P_{2} \otimes Q_{1}) | (P_{2} \otimes Q_{2})
\end{mathpar}

\paragraph{Tensor with dropped names} We treat tensor of a
process with a dropped name as parallel composition.

\begin{mathpar}
  P \otimes \dropn{x} := P | \dropn{x}
\end{mathpar}

\paragraph{Tensor of agents}

Finally, we need to define tensor on agents. Note that the definition
of tensor on normal products only tensors inputs with inputs and
outputs with outputs. Thus, we only have to define the operation on
``homogeneous'' pairings.

\begin{mathpar}
  (\vec{x})P \otimes (\vec{y})Q
  \and \\
  := (x_{0}^{L}|y_{0}^{R},\ldots,x_{0}^{L}|y_{n}^{R},\ldots,x_{m}^{L}|y_{0}^{R},\ldots,x_{m}^{L}|y_{n}^R)(P\{ \vec{x}^{L}/\vec{x}\} \otimes Q \{ \vec{y}^{R}/\vec{y}\})
  \and \\
  \clift{\vec{P}} \otimes \clift{\vec{Q}}
  \and \\
  := \clift{P_{0}\otimes Q_{0},\ldots,P_{0}\otimes Q_{n},\ldots,P_{m}\otimes Q_{0},\ldots,P_{m}\otimes Q_{n}}
\end{mathpar}

\begin{remark}
  Observe that arities of tensored abstractions matches arities of
  tensored concretions if the original arities matched. Note also that
  the length of the arities corresponds to the increase in dimension
  we see in ordinary vector space tensor product.
\end{remark}

\begin{remark}
  Operationally, this definition distributes the tensor down to
  components ``linked'' by summation. Tensor over summation is
  intriguing in that it mixes names. Moreover, as a consequence of the
  way it mixes names we have the identities for all $x \in \QProc$ and
  $P,Q \in \Proc$

  \begin{mathpar}
    (x \cdot P) \otimes Q \equiv x \cdot (P \otimes Q) \equiv P \otimes (x \cdot Q)
    \and
    P \otimes \pzero \equiv P
  \end{mathpar}

  that the reader is invited to verify.
\end{remark}

\subsubsection{Annihilation}
\begin{mathpar}
  P^{\perp} := \{ Q | \forall R. P|Q \red^{*} R \Rightarrow R \red^{*} \pzero \}
  \and \\
  P^{\underline{\perp}} := \Sigma_{Q \in P^{\perp}} \quotep{Q}?(y).(\dropn{y}|Q) | \Sigma_{Q \in P^{\perp}} \quotep{Q}\clift{\Box}
\end{mathpar}

\paragraph{Discussion} The reader will note that $P^{\perp}$ is a
\emph{set} of processes, while $P^{\underline{\perp}}$ is a
\emph{context}. We call the set $P^{\perp}$ the \emph{annihilators} of
$P$. The parallel composition of a process in the annihilators of $P$
with $P$ will result in a process, the state space of which has all
paths eventually leading to $\pzero$. Execution may endure loops; but
under reasonable conditions of fairness (naturally guaranteed under
most notions of bisimulation) such a composite process cannot get
stuck in such a loop and will, eventually pop out and terminate.

The context $P^{\underline{\perp}}$ is ready and willing to ``take the
$P$ out of'' the process to which it is applied. It will effectively
transmit the code of the process to which it is applied to one of the
annihilators and run the process against it.

\subsubsection{Evaluation}
We fix $M$ a domain of fully abstract interpretation with an equality
coincident with bisimulation. We take $\meaningof{\cdot} : \Proc \to
M$ to be the map interpreting processes and $\nmeaningof{\cdot} : \M
\to Proc$ to be the map running the other way. Then we define

\begin{mathpar}
  \int P := \nmeaningof{\meaningof{P}}
\end{mathpar}

\paragraph{Discussion}
There are many fully abstract interpretations of Milner's
$\pi$-calculus. Any of them can be used as a basis for interpreting
the reflective calculus here. Equipped with such a domain it is
largely a matter of grinding through to check that the Yoneda
construction for the normalization-by-evaluation program can be
extended to this setting.

\begin{remark}
  The reader is invited to verify that $\int (P^{\underline{\perp}}[P]) = 0$.
\end{remark}

\subsection{Quantum mechanics}

Table \ref{tbl:core_qm_op_defns} gives the core operational definitions

\begin{table}[htp]\label{tbl:core_qm_op_defns}
  \center{
    \fbox{
      \begin{tabular}{c|c}
        quantum mechanics & process calculus \\
        \hline
        scalar & $x := \quotep{P}$ \\
        state vector & $\state{P} := P$ \\
        dual & $\state{P}^{*} := \event{P^{\underline{\perp}}} := \quotep{P^{\underline{\perp}}}[-]$ \\
        matrix & $ \Sigma_{\alpha} \state{P_{\alpha}}x_{\alpha}\event{Q_{\alpha}}$ \\
        vector addition & $\state{P} + \state{Q} := \state{P | Q}$ \\
        tensor product & $\state{P} \otimes \state{Q} := \state{P \otimes Q}$ \\
        inner product & $\innerprod{P}{Q} := \quotep{\int P^{\underline{\perp}}[Q]}$ \\
      \end{tabular}
    }
  }
  \caption{QM - operational definitions}
\end{table}

where

\begin{mathpar}
  \prmatrix{P}{Q} := \fprmatrix{P}{\quotep{\pzero}}{Q}
  \and
  \fprmatrix{P}{x}{Q} := (\state{P},x,\event{Q})
  \and
  (\fprmatrix{P}{x}{Q})(\state{R}) := x \cdot \innerprod{Q}{R} \cdot \state{P}
  \and
  (\fprmatrix{P}{x}{Q})(\event{R}) := x \cdot \innerprod{R}{P} \cdot \event{Q}
\end{mathpar}

\paragraph{Discussion}
As promised: vectors (aka states) are represented as processes; duals
as contextual duals; inner product definition should be compared with
standard inner product definition for ....

\begin{remark}
  Assuming $\int (P^{\underline{\perp}}[P]) = 0$, the reader is
  invited to verify that $(\fprmatrix{P}{x}{P})(\state{P}) = x \cdot \state{P}$.
\end{remark}

\begin{remark}
  The reader is invited to verify that $\innerprod{P}{Q}$ could
  equally well have been written $\quotep{\int \stackrel{\vee}{x}}$
  where $x = \event{P^{\underline{\perp}}}(Q)$.

  One of the motivations for this remark is that there is another way
  to factor these operations. We could package up evaluation in the dual:

  \begin{mathpar}
    \state{P}^{*} := \event{\int P^{\underline{\perp}}} := \quotep{\int P^{\underline{\perp}}}[-]
  \end{mathpar}

  and then have inner product defined by
  
  \begin{mathpar}
    \innerprod{P}{Q} := \event{P}(Q)
  \end{mathpar}

  Hopefully, experience with the calculations will provide guidance on
  the best factoring.
\end{remark}

\begin{remark}
  Assuming $\int (P^{\underline{\perp}}[P]) = 0$, the reader is
  invited to verify that $\forall P,Q. (\prmatrix{0}{Q})(\state{0}) =
  \state{0}$ and dually $(\prmatrix{P}{0})(\event{0}) = \event{0}$.
\end{remark}

\begin{remark}
  i'm a little worried that i don't (yet) have proper support for
  complex conjugacy. But, the observation above may give us a
  clue. According to Abramsky, it must be the case that the scalars
  are iso to the homset of the identity for the tensor -- which the
  observation above characterizes. 

  For now, we will simply bookmark the notion with $\overline{x}$.
\end{remark}

\subsubsection{Adjointness}

We need to give a definition of $(\cdot)^{\dagger}$ for matrices. The
obvious candidate definition is
\begin{mathpar}
(\Sigma_{\alpha}\fprmatrix{P_{\alpha}}{x_{\alpha}}{Q_{\alpha}})^{\dagger}
= \Sigma_{\alpha}\fprmatrix{(Q_{\alpha}^{\underline{\perp}})^{*}}{\overline{x}_{\alpha}}{P_{\alpha}^{\underline{\perp}}} 
\end{mathpar}

But, $(Q_{\alpha}^{\underline{\perp}})^{*}$ requires a name along
which to communicate the process to achieve the context application.

\subsubsection{Basis for a basis}
If processes label states and ``addition'' of states (a.k.a. vector
addition) is interpreted as parallel composition, what corresponds to
notions of linear independence and basis? Here, we recall that Yoshida
has developed a set of \emph{combinators} for an asynchronous verison
of Milner's $\pi$-calculus. These are a finite set of processes such
any process can be expressed as parallel composition of these
combinators together with liberal uses of the new operator and
replication. We can simply give a translation of these into the
present calculus and have reasonable expectation that the property
carries over. That is, that the resultant set allows to express all
processes via parallel composition. Note, however, that there is no
new operator or replication in this calculus. As a result, we expect
that the corresponding set is actually infinite. That is, we expect
that the space is actually infinite dimensional.

\begin{remark}
  The attentive reader may be a bit concerned. Certainly, the
  collection $S$, $K$ and $I$ is a finite set of
  combinators. Shouldn't we expect to see a finite set of combinators
  for an effectively equivalent system? i am very sympathetic to this
  critique and feel it warrants full attention. On the other hand, i
  also have in mind the following analogy. The natural numbers, as a
  monoid under addition, has exactly $1$ generator, while the natural
  numbers, as a monoid under multiplication, has countably many
  generators (the primes). We observe that the application of the
  lambda calculus is much less resource sensitive than the parallel
  composition of the $\pi$-calculus. Could it be the case that we have
  an analogy of the form
  
  \begin{mathpar}
    m + n : MN :: m*n : M|N
  \end{mathpar}

  giving a similar blow up in the set of ``primes''?  This is such a
  wonderful thought that, even if it's not true, i think it's worth
  writing down.
\end{remark}
 

\documentclass[12pt]{llncs}
%\documentclass{jktr}

\usepackage[pdftex]{hyperref}                   
\usepackage {listings}
\usepackage {mathpartir}
\usepackage{bcprules}
%\usepackage{listings}
                       
\usepackage{graphicx} 
%\usepackage[margins=2.5cm,nohead,nofoot]{geometry}
%\usepackage{geometry}
\usepackage{amsfonts}
\usepackage{amstext}
\usepackage{latexsym}
\usepackage{amssymb}
\usepackage{color}


%\include{myPreamble}
\include{qm2pi.local} 

%\ifpdf
%\usepackage[pdftex]{graphicx}
%\else
%\usepackage{graphicx}
%\fi

 % \ifpdf
%  \usepackage{pdfsync}
%  \if


%\title{Brief Article}
%\author{David F. Snyder}
%\author{L.G. Meredith}

%\address{Dept. of Math., Texas State University--San Marcos, San Marcos, TX 78666}
       
\pagestyle{empty}


\begin{document}

\lstset{language=[Objective]Caml,frame=shadowbox}

\input{qm2pi.front}

% section front matter (end)

\input{qm2pi.intro} 
 
% section introduction (end)

% \input{qm2pi.knotations} 

% section notation (end)

\input{qm2pi.process.calculi} 

% section concurrent_process_calculi_and_spatial_logics_ (end)
    
%\input{qm2pi.knots2pi} 

%\input{qm2pi.trefoil} 

%\input{qm2pi.mainthm} 

% subsection basic_interpretation (end)

%\input{qm2pi.rho.presentation} 
\subsection{The syntax and semantics of the notation system}\label{sub:the_syntax_and_semantics_of_the_notation_system} % (fold)

We now summarize a technical presentation of the calculus that
embodies our theory of dynamics. The typical presentation of such a
calculus follows the style of giving generators and relations on
them. The grammar, below, describing term constructors, freely
generates the set of processes, $\Proc$. This set is then quotiented
by a relation known as structural congruence and it is over this set
that the notion of dynamics is expressed. This presentation is
essentially that of \cite{MeredithR05} with the addition of
polyadicity and summation. For readability we have relegated some of
the technical subtleties to an appendix.

\subsubsection{Process grammar}\label{subsub:process_grammar}

\begin{mathpar}
  \inferrule* [lab=synchronization] {} {{M} \bc \pzero \;|\; x?F \;|\; x!C }
  \and
  \inferrule* [lab=abstraction] {} {{F} \bc (x)P}
  \and
  \inferrule* [lab=concretion] {} {{C} \bc \langle Q \rangle}
  \and
  \inferrule* [lab=process] {} {{P,Q} \bc M \;| \;P|Q \;|\; @{x}}
  \and
  \inferrule* [lab=name] {} {{x} \bc \quotep{P}}
\end{mathpar} 

Note that $\vec{x}$ (resp. $\vec{P}$) denotes a vector of names
(resp. processes) of length $|\vec{x}|$ (resp. $|\vec{P}|$). We adopt
the following useful abbreviations.

\begin{mathpar}
   x?(\vec{y}).P := x.(\vec{y})P \and  x\clift{\vec{P}} := x.\clift{\vec{P}}
   \and x!(y) := \lift{x}{\dropn{y}}
   \and \Pi_{i=0}^{n-1}P_i := P_0 | \ldots | P_{n-1}
\end{mathpar}

\subsubsection{Structural congruence}

\paragraph{Free and bound names and alpha-equivalence.} At the
core of structural equivalence is alpha-equivalence which identifies
process that are the same up to a change of variable. Formally, we
recognize the distinction between free and bound names. The free names
of a process, $\freenames{P}$, may be calculated recursively as
follows:

\begin{mathpar}
\freenames{\pzero} := \emptyset
  \and \\
  \freenames{x?(y).P} := \{ x \} \cup (\freenames{P} \setminus \{ y \})
  \and 
  \freenames{x!\langle P \rangle} := \{ x \} \cup \{ P \} 
  \and \\
  \freenames{P|Q} := \freenames{P} \cup \freenames{Q}
  \and \\
  \freenames{@{x}} := \{ x \}
\end{mathpar}

$\pi$
$\quotep{\pi}$

$\freenames{-} : \pi \to \mathcal{P}(\quotep{\pi})$

\begin{eqnarray*}
  \freenames{\pzero} & := & \emptyset \\
  \freenames{x?(y).P} & := & \{ x \} \cup (\freenames{P} \setminus \{ y \}) \\
  \freenames{x!\langle P \rangle} & := & \{ x \} \cup \{ P \} \\
  \freenames{P|Q} & := & \freenames{P} \cup \freenames{Q} \\
  \freenames{\dropn{x}} & := & \{ x \}
\end{eqnarray*}

The bound names of a process, $\boundnames{P}$, are those names occurring in $P$
that are not free. For example, in $x?(y).0$, the name $x$ is free, while $y$ is bound.

\begin{mathpar}
  \inferrule* [lab=monoidal-laws] {} { P|Q \equiv Q|P \and P|0 \equiv P \and P|(Q|R) \equiv (P|Q)|R }
\end{mathpar}

\begin{mathpar}
  \inferrule* [lab=alpha-equivalence] {} { (x)P \equiv (y)P\{y/x\} \and y \not\in \freenames{P} }
\end{mathpar}

\begin{definition}
Then two processes, $P,Q$, are alpha-equivalent if $P = Q\{\vec{y}/\vec{x}\}$ for
some $\vec{x} \in \boundnames{Q},\vec{y} \in \boundnames{P}$, where $Q\{\vec{y}/\vec{x}\}$
denotes the capture-avoiding substitution of $\vec{y}$ for $\vec{x}$ in $Q$.
\end{definition}

\begin{definition}
  The {\em structural congruence} \cite{SangiorgiWalker} , $\equiv$,
  between processes is the least congruence containing
  alpha-equivalence, satisfying the abelian monoid laws
  (associativity, commutativity and $\pzero$ as identity) for parallel
  composition $|$ and for summation $+$.
\end{definition}

\subsection{Name equivalence}

We take name equivalence, written $\nameeq$, to be the smallest
equivalence relation generated by the following rules.

\begin{mathpar}
\inferrule*[lab=Quote-drop]
{ }
{ \quotep{@{x}} \nameeq x }

\inferrule*[lab=Struct-equiv]
{ P \scong Q }
{ \quotep{P} \nameeq \quotep{Q} }
\end{mathpar}

The astute reader will have noticed that the mutual recursion of names
and processes imposes a mutual recursion on alpha-equivalence and
structural equivalence via name-equivalence. Fortunately, all of this
works out pleasantly and we may calculate in the natural way, free of
concern. The reader interested in the details is referred to the
appendix \ref{appendix:rho_details}.

\subsection{Substitution}

We use $\Proc$ for the set of processes, $\QProc$ for the set of
names, and $\id{\{}\vec{y} / \vec{x} \id{\}}$ to denote partial maps,
$s : \QProc \rightarrow \QProc$. A map, $s$ lifts, uniquely, to a map
on process terms, $\widehat{s} : \Proc \rightarrow \Proc$ by the
following equations.

\begin{mathpar}
  (0) \psubstp{Q}{P} := 0 \\
  (R \juxtap S) \psubstp{Q}{P}
  :=    
  (R)\psubstp{Q}{P} \juxtap (S) \psubstp{Q}{P} \\
  (x?(y).R) \psubstp{Q}{P}    
  :=    
  (x)\substp{Q}{P} (z)\concat( (R \psubstn{z}{y}) \psubstp{Q}{P} ) \\
  (\lift{x}{R}) \psubstp{Q}{P}  
  :=
  \lift{(x)\substp{Q}{P}}{ R \psubstp{Q}{P} } \\
%   (\dropn{x})  \psubstp{Q}{P}       
%   := 
%   \left\{ 
%     \begin{array}{ccc} 
%       \dropn{\quotep{Q}} & & x \nameeq \quotep{P} \\
%       \dropn{x} & & otherwise \\
%     \end{array}
%   \right. 
  (\dropn{x})  \psubstp{Q}{P}       
  := 
  \left\{ 
    \begin{array}{ccc} 
      Q & & x \nameeq \quotep{P} \\
      \dropn{x} & & otherwise \\
    \end{array}
  \right.
\end{mathpar}
 

where

\begin{eqnarray}
  (x)\id{\{} \lpquote Q \rpquote / \lpquote P \rpquote \id{\}}            = 
  \left\{ 
    \begin{array}{ccc}
      \lpquote Q \rpquote & & x \nameeq \lpquote P \rpquote \\
      x & & otherwise \\
    \end{array}
  \right. \nonumber
\end{eqnarray}

and $z$ is chosen distinct from $\quotep{P}$, $\quotep{Q}$, the free
names in $Q$, and all the names in $R$. Our $\alpha$-equivalence will
be built in the standard way from this substitution.

\begin{remark}\label{rem:no_self_referential_names}
  One consequence of these definitions is that $\forall P. \quotep{P}
  \not\in \freenames{P}$.
\end{remark}

\subsection{ Dynamic quote: an example }

Anticipating something of what's to come, consider applying the
substitution, $\widehat{\id{\{}u / z \id{\}}}$, to the following pair
of processes, $\lift{w}{y!(z)}$ and $w[ \lpquote y!(z) \rpquote ]$.

\begin{eqnarray}
	\lift{w}{y!(z)}\widehat{\id{\{}u / z \id{\}}}
		& = &
		\lift{w}{y!(u)} \nonumber\\
	w[ \lpquote y!(z) \rpquote ] \widehat{ \id{\{}u / z \id{\}} }
		& = &
		w[ \lpquote y!(z) \rpquote ] \nonumber
\end{eqnarray}

Because the body of the process between quotes is impervious to
substitution, we get radically different answers. In fact, by
examining the first process in an input context,
e.g. $x?(z).\lift{w}{y!(z)}$, we see that the process under the lift
operator may be shaped by prefixed inputs binding a name inside it. In
this sense, the lift operator will be seen as a way to dynamically
construct processes before reifying them as names.

Finally equipped with these standard features we can present the
dynamics of the calculus.

\subsubsection{Operational semantics} 

Finally, we introduce the computational dynamics. What marks these
algebras as distinct from other more traditionally studied algebraic
structures, e.g. vector spaces or polynomial rings, is the manner in
which dynamics is captured. In traditional structures, dynamics is typically
expressed through morphisms between such structures, as in linear maps
between vector spaces or morphisms between rings. In algebras
associated with the semantics of computation, the dynamics is
expressed as part of the algebraic structure itself, through a
reduction reduction relation typically denoted by $\red$. Below, we
give a recursive presentation of this relation for the calculus used
in the encoding.

$\red \subseteq \pi \times \pi$
$\red : \pi \to \mathcal{P}(\pi)$

\begin{mathpar}
  \inferrule* [lab=Comm] { \textsf{match}( x_{src}, x_{trgt} ) } { x_{trgt}?(y)P \; | \; x_{src}!\langle {Q} \rangle \red P\{\quotep{Q}/y}\} }
  \and \\
  \inferrule* [lab=Par] {{P} \red {P}'} {{{P} | {Q}} \red {{P}' | {Q}}}
  \and
  \inferrule* [lab=Equiv]{{{P} \scong {P}'} \andalso {{P}' \red {Q}'} \andalso {{Q}' \scong {Q}}}{{P} \red {Q}}
\end{mathpar}

\begin{eqnarray*}
  match_{\equiv} (\quotep{P},\quotep{Q}) & := & P \equiv Q \\
  match_{\dagger}(\quotep{P},\quotep{Q}) & := & \forall R. P|Q \red^{*} R => R \red^{*} 0 \\
  match_{K}(\quotep{P},\quotep{Q}) & := & K \mbox{ for some context } K
\end{eqnarray*}

$u?(x)P | u!\langle Q \rangle \red P\{\quotep{Q}/x\}$

%We write $\wred$ for $\red^*$, and $P\red$ if $\exists Q $ such that $ P \red Q$.
We write $P\red$ if $\exists Q $ such that $ P \red Q$ and $P\not\red$, otherwise.

\section{Replication}

As mentioned before, it is known that replication (and hence
recursion) can be implemented in a higher-order process algebra
\cite{SangiorgiWalker}. As our first example of calculation with the
machinery thus far presented we give the construction explicitly in
the {\rhoc}.

\begin{eqnarray}
	D_{x} & := & \prefix{x}{y}{(\binpar{\outputp{x}{y}}{@{y}})} \nonumber\\
	\bangp_{x}{P} & := & \binpar{{x}!\langle{\binpar{D_{x}}{P}}\rangle}{D_{x}} \nonumber
\end{eqnarray}

\begin{eqnarray}
	\bangp_{x}{P} & & \nonumber\\
	=
	& {x}!\langle{(\prefix{x}{y}{(\outputp{x}{y} | @{y})) | P}}\rangle 
	      | \prefix{x}{y}{(\outputp{x}{y} | @{y})} & \nonumber\\
	\red
	& (\outputp{x}{y} | @{y})\substn{\quotep{(\prefix{x}{y}{(@{y} | \outputp{x}{y})) | P}}}{y} & \nonumber\\
	=
	& \outputp{x}{\quotep{(\prefix{x}{y}{(\outputp{x}{y} | @{y})) | P}}}
	  | {(\prefix{x}{y}{(\outputp{x}{y} | @{y})) | P}} & \nonumber\\
	\red
	& \ldots & \nonumber\\
	\red^*
	& P | P | \ldots & \nonumber
\end{eqnarray}

Of course, this encoding, as an implementation, runs away, unfolding
$\bangp{P}$ eagerly. A lazier and more implementable replication
operator, restricted to input-guarded processes, may be obtained as follows.

\begin{eqnarray}
\bangp{\prefix{u}{v}{P}} 
	:= 
	\binpar{\lift{x}{\prefix{u}{v}{(\binpar{D(x)}{P})}}}{D(x)} \nonumber
\end{eqnarray}

\begin{remark}
  Note that the lazier definition still does not deal with summation
  or mixed summation (i.e. sums over input and output). The reader is
  invited to construct definitions of replication that deal with these
  features. 

  Further, the definitions are parameterized in a name, $x$. Can you,
  gentle reader, make a definition that eliminates this parameter and
  guarantees no accidental interaction between the replication
  machinery and the process being replicated -- i.e. no accidental
  sharing of names used by the process to get its work done and the
  name(s) used by the replication to effect copying. This latter
  revision of the definition of replication is crucial to obtaining
  the expected identity $!!P \sim !P$.
\end{remark}

\begin{remark}\label{rem:paradoxical_combinator}
  The reader familiar with the lambda calculus will have noticed the
  similarity between $D$ and the paradoxical combinator.

  [Ed. note: the existence of this seems to suggest we have to be more
  restrictive on the set of processes and names we admit if we are to
  support no-cloning.]
\end{remark}

\subsubsection{Bisimulation}

The computational dynamics gives rise to another kind of equivalence,
the equivalence of computational behavior. As previously mentioned
this is typically captured \emph{via} some form of bisimulation.

% The notion we use in this paper is weak barbed bisimulation
% \cite{milner91polyadicpi}.

The notion we use in this paper is derived from weak barbed
bisimulation \cite{milner91polyadicpi}. 

\begin{definition}
An \emph{observation relation}, $\downarrow_{\mathcal N}$, over a set
of names, $\mathcal N$, is the smallest relation satisfying the rules
below.

\infrule[Out-barb]{y \in {\mathcal N}, \; x \nameeq y}
		  {\outputp{x}{v} \downarrow_{\mathcal N} x}
\infrule[Par-barb]{\mbox{$P\downarrow_{\mathcal N} x$ or $Q\downarrow_{\mathcal N} x$}}
		  {\binpar{P}{Q} \downarrow_{\mathcal N} x}

We write $P \Downarrow_{\mathcal N} x$ if there is $Q$ such that 
$P \wred Q$ and $Q \downarrow_{\mathcal N} x$.
\end{definition}

\begin{definition}
%\label{def.bbisim}
An  ${\mathcal N}$-\emph{barbed bisimulation} over a set of names, ${\mathcal N}$, is a symmetric binary relation 
${\mathcal S}_{\mathcal N}$ between agents such that $P\rel{S}_{\mathcal N}Q$ implies:
\begin{enumerate}
\item If $P \red P'$ then $Q \wred Q'$ and $P'\rel{S}_{\mathcal N} Q'$.
\item If $P\downarrow_{\mathcal N} x$, then $Q\Downarrow_{\mathcal N} x$.
\end{enumerate}
$P$ is ${\mathcal N}$-barbed bisimilar to $Q$, written
$P \wbbisim_{\mathcal N} Q$, if $P \rel{S}_{\mathcal N} Q$ for some ${\mathcal N}$-barbed bisimulation ${\mathcal S}_{\mathcal N}$.
\end{definition}

$\mathcal{R} \subseteq \pi \times \pi$

$P \mathcal{R} Q => \forall P'. P \red P' \Rightarrow \exists Q'. Q \red Q', P' \mathcal{R} Q'$

$P \vdash x \Rightarrow Q \vdash x$

\begin{mathpar}
  \inferrule*[lab=Out-barb]{x \nameeq y}{{y}!\langle{Q}\rangle \vdash x}
  \and
  \inferrule*[lab=Par-barb]{\mbox{$P\vdash x$ or $Q\vdash x$}}{\binpar{P}{Q} \vdash x}
\end{mathpar}

\subsubsection{Contexts}

One of the principle advantages of computational calculi like the
$\pi$-calculus is a well-defined notion of context,
contextual-equivalence and a correlation between
contextual-equivalence and notions of bisimulation. The notion of
context allows the decomposition of a process into (sub-)process and
its syntactic environment, its context. Thus, a context may be
thought of as a process with a ``hole'' (written $\Box$) in it. The
application of a context $M$ to a process $P$, written $M[P]$, is
tantamount to filling the hole in $M$ with $P$. In this paper we do
not need the full weight of this theory, but do make use of the notion
of context in the proof the main theorem. 

\begin{mathpar}
  \inferrule* [lab=summation] {} {{M_{M},M_{N}} \bc \Box \;|\; x.M_{A} \;|\; M_{M}+M_{N}}
  \and
  \inferrule* [lab=agent] {} {{M_{A}} \bc (\vec{x})M_{P} \;| \; \clift{P_0,\ldots,M_{P},\ldots,P_N}}
  \and \\
  \inferrule* [lab=process] {} {{M_{P}} \bc M_{N} \;| \;P|M_{P} }
\end{mathpar} 

\begin{mathpar}
  \inferrule* [lab=sychronization] {} {M_{N} \bc \Box \;|\; x?M_{F} \;|\; x!M_{C}}
  \and
  \inferrule* [lab=abstraction] {} {{M_{F}} \bc (x)M_{P} }
  \and
  \inferrule* [lab=concretion] {} {{M_{C}} \bc \langle M_{P} \rangle }
  \and \\
  \inferrule* [lab=process] {} {{M_{P}} \bc M_{N} \;| \;P|M_{P} }
\end{mathpar}

\begin{definition}[contextual application] Given a context $M$, and
  process $P$, we define the \emph{contextual application}, $M[P] :=
  M\{P/\Box\}$. That is, the contextual application of M to P is the
  substitution of $P$ for $\Box$ in $M$.
\end{definition}

$\meaningof{-} : L \to \mathcal{P}(\pi)$

\begin{mathpar}
  \inferrule* [lab=collection] {} {\meaningof{true} = \pi, \and \meaningof{~E} = \pi \setminus \meaningof{E}, \and \meaningof{E_{1} \& E_{2}} = \meaningof{E_{1}} \cap \meaningof{E_{2}}}
\end{mathpar}

\begin{mathpar}
  \inferrule* [lab=structure] {} {\meaningof{0} = \{ P \in \pi | P \equiv 0 \}, \and \\ \meaningof{E_1 | E_2} = \{ P \in \pi | P \equiv P_{1} | P_{2}, P_{1} \in \meaningof{E_{1}}, P_{2} \in \meaningof{E_2}\} }
\end{mathpar}

\begin{mathpar}
 \inferrule* [lab=behavior] {} {\meaningof{\langle a?b \rangle E} = \{ P \in \pi | P \equiv Q | u?(y)P', \\ \and \\\\ \and \\ \;\;\; u \in \meaningof{a}, \forall z.P'\{z/y\} \in \meaningof{E\{z/b\}}\}, \and \\ \meaningof{a!E} = \{ P \in \pi | P \equiv Q | x!\langle P' \rangle, x \in \meaningof{a} P' \in \meaningof{E}\} }
\end{mathpar}

\begin{mathpar}
 \inferrule* [lab=nominal] {} {\meaningof{\quotep{E}} = \{ \quotep{P} \in \quotep{\pi} | P \in \meaningof{E} \}, \and \meaningof{\quotep{P}} = \{ \quotep{Q} \in \quotep{\pi} | P \equiv Q \} \and \\ \meaningof{@\quotep{E}} = \{ P \in \pi | P \equiv @x, x \in \meaningof{E} \}}
\end{mathpar}

\begin{eqnarray*}
  \\
  \meaningof{-} : TS \to ST
\end{eqnarray*}

\begin{eqnarray*}
  \\
  L : TS \to ST
\end{eqnarray*}

\begin{eqnarray*}
  \\
  P \models E \iff P \in \meaningof{E}
\end{eqnarray*}

\begin{eqnarray*}
  P \approx_{L} Q \iff \forall E \in L. P \models E \iff Q \models E
\end{eqnarray*}

\begin{eqnarray*}
  P \approx_{K} Q
\end{eqnarray*}

\begin{eqnarray*}
  P \approx Q
\end{eqnarray*}

$\approx_{K} = \approx = \approx_{L}$

\subsubsection{Contextual duality}

Note that contexts extend the quotation operation to a family of
operations from processes to names. Given a context, $M$, we can
define a \emph{nominal context}, $\quotep{M}$ by $\quotep{M}[P] :=
\quotep{M[P]}$. To foreshadow what is to come we observe that these
operations enjoy a duality with processes very much like the duality
between vectors and maps from vectors to scalars.

Further, because the calculus is essentially higher-order, we have a
correspondence between contexts and processes. More specifically,
given a name $x$ and a context $M$ we can construct $M^{*}_{x}$ such
that 

\begin{mathpar}
  M^{*}_{x} | \lift{x}{P} \red M[P]
\end{mathpar}

namely,

\begin{mathpar}
  M^{*}_{x} := x?(u).M[\dropn{u}]
\end{mathpar}

The dependence of $M^{*}_{x}$ on a name makes it an abstraction, 

\begin{mathpar}
  M^{*} := (x)x?(u).M[\dropn{u}]
\end{mathpar}

\subsection{Additional notation}

It will sometimes be convenient to denote the process a name
quotes. We already have the notation $x = \quotep{P}$, but it will be
convenient to introduce an alternate notation, $\procn{x}$, when we
want to emphasize the connection to the use of the name. Note that, by
virtue of name equivalence, $\quotep{\procn{x}} \nameeq x$; so, the
notation is consistent with previous definitions.

Further, because names have structure it is possible to effect
substitutions on the basis of that structure. This means we need to
upgrade our notation for substitutions, which we accomplish by
adapting comprehension notation. Thus,

\begin{mathpar}
  P\{ y / x : x \in S \}
\end{mathpar}

is interpreted to mean the process derived from P by replacing (in a
capture-avoiding manner) each occurrence of $x$ in $S$ by $y$. For example,

\begin{mathpar}
  P\{ \quotep{\procn{x}|\procn{x}} / x : x \in \freenames{P} \}
\end{mathpar}

will replace each (occurrence) of a free name $x$ in $P$ by
$\quotep{\procn{x}|\procn{x}}$.

Also, we will avail ourselves of the notation $x^{L}$ and $x^{R}$ to
denote injections of a name into disjoint copies of the name
space. There are numerous ways to accomplish this. One example can be
found in \cite{MeredithR05}. This notation overloads to vectors of
names: $\vec{x}^{\pi} := (x_{i}^{\pi} \; : \; 0 \leq i < |\vec{x}| )$ where $\pi \in \{L,R\}$.

We also use $P^{\Box} := P|\Box$.

In \cite{MeredithR05} an interpretation of the new operator is
given. It turns out that there are several possible interpretations
all enjoying the requisite algebraic properties of the operator (see
\cite{milner91polyadicpi}). We will therefore make liberal use of
$(\nu\; \vec{x})P$.

% subsection the_syntax_and_semantics_of_the_notation_system (end)   

\input{qm2pi.qmops} 

\input{qm2pi.sterngerlach} 

\input{qm2pi.metric} 

% section concurrent_process_calculi (end)

%\input{qm2pi.proofsketch}

% section proof sketch (end)

%\input{qm2pi.slviaknots} 

% section spatial logic via knots (end)

\input{qm2pi.conclusion}

% section conclusion (end)

%\input{qm2pi.dtcodes} 

% section wiring algorithm (end)

\input{qm2pi.ack} 

% section acknowledgments (end)

\newpage


\bibliographystyle{plain}   
\bibliography{../../biblios/main.bib}

\input{qm2pi.rhodetails}

\end{document}

 

\documentclass[12pt]{llncs}
%\documentclass{jktr}

\usepackage[pdftex]{hyperref}                   
\usepackage {listings}
\usepackage {mathpartir}
\usepackage{bcprules}
%\usepackage{listings}
                       
\usepackage{graphicx} 
%\usepackage[margins=2.5cm,nohead,nofoot]{geometry}
%\usepackage{geometry}
\usepackage{amsfonts}
\usepackage{amstext}
\usepackage{latexsym}
\usepackage{amssymb}
\usepackage{color}


%\include{myPreamble}
\include{qm2pi.local} 

%\ifpdf
%\usepackage[pdftex]{graphicx}
%\else
%\usepackage{graphicx}
%\fi

 % \ifpdf
%  \usepackage{pdfsync}
%  \if


%\title{Brief Article}
%\author{David F. Snyder}
%\author{L.G. Meredith}

%\address{Dept. of Math., Texas State University--San Marcos, San Marcos, TX 78666}
       
\pagestyle{empty}


\begin{document}

\lstset{language=[Objective]Caml,frame=shadowbox}

\input{qm2pi.front}

% section front matter (end)

\input{qm2pi.intro} 
 
% section introduction (end)

% \input{qm2pi.knotations} 

% section notation (end)

\input{qm2pi.process.calculi} 

% section concurrent_process_calculi_and_spatial_logics_ (end)
    
%\input{qm2pi.knots2pi} 

%\input{qm2pi.trefoil} 

%\input{qm2pi.mainthm} 

% subsection basic_interpretation (end)

%\input{qm2pi.rho.presentation} 
\subsection{The syntax and semantics of the notation system}\label{sub:the_syntax_and_semantics_of_the_notation_system} % (fold)

We now summarize a technical presentation of the calculus that
embodies our theory of dynamics. The typical presentation of such a
calculus follows the style of giving generators and relations on
them. The grammar, below, describing term constructors, freely
generates the set of processes, $\Proc$. This set is then quotiented
by a relation known as structural congruence and it is over this set
that the notion of dynamics is expressed. This presentation is
essentially that of \cite{MeredithR05} with the addition of
polyadicity and summation. For readability we have relegated some of
the technical subtleties to an appendix.

\subsubsection{Process grammar}\label{subsub:process_grammar}

\begin{mathpar}
  \inferrule* [lab=synchronization] {} {{M} \bc \pzero \;|\; x?F \;|\; x!C }
  \and
  \inferrule* [lab=abstraction] {} {{F} \bc (x)P}
  \and
  \inferrule* [lab=concretion] {} {{C} \bc \langle Q \rangle}
  \and
  \inferrule* [lab=process] {} {{P,Q} \bc M \;| \;P|Q \;|\; @{x}}
  \and
  \inferrule* [lab=name] {} {{x} \bc \quotep{P}}
\end{mathpar} 

Note that $\vec{x}$ (resp. $\vec{P}$) denotes a vector of names
(resp. processes) of length $|\vec{x}|$ (resp. $|\vec{P}|$). We adopt
the following useful abbreviations.

\begin{mathpar}
   x?(\vec{y}).P := x.(\vec{y})P \and  x\clift{\vec{P}} := x.\clift{\vec{P}}
   \and x!(y) := \lift{x}{\dropn{y}}
   \and \Pi_{i=0}^{n-1}P_i := P_0 | \ldots | P_{n-1}
\end{mathpar}

\subsubsection{Structural congruence}

\paragraph{Free and bound names and alpha-equivalence.} At the
core of structural equivalence is alpha-equivalence which identifies
process that are the same up to a change of variable. Formally, we
recognize the distinction between free and bound names. The free names
of a process, $\freenames{P}$, may be calculated recursively as
follows:

\begin{mathpar}
\freenames{\pzero} := \emptyset
  \and \\
  \freenames{x?(y).P} := \{ x \} \cup (\freenames{P} \setminus \{ y \})
  \and 
  \freenames{x!\langle P \rangle} := \{ x \} \cup \{ P \} 
  \and \\
  \freenames{P|Q} := \freenames{P} \cup \freenames{Q}
  \and \\
  \freenames{@{x}} := \{ x \}
\end{mathpar}

$\pi$
$\quotep{\pi}$

$\freenames{-} : \pi \to \mathcal{P}(\quotep{\pi})$

\begin{eqnarray*}
  \freenames{\pzero} & := & \emptyset \\
  \freenames{x?(y).P} & := & \{ x \} \cup (\freenames{P} \setminus \{ y \}) \\
  \freenames{x!\langle P \rangle} & := & \{ x \} \cup \{ P \} \\
  \freenames{P|Q} & := & \freenames{P} \cup \freenames{Q} \\
  \freenames{\dropn{x}} & := & \{ x \}
\end{eqnarray*}

The bound names of a process, $\boundnames{P}$, are those names occurring in $P$
that are not free. For example, in $x?(y).0$, the name $x$ is free, while $y$ is bound.

\begin{mathpar}
  \inferrule* [lab=monoidal-laws] {} { P|Q \equiv Q|P \and P|0 \equiv P \and P|(Q|R) \equiv (P|Q)|R }
\end{mathpar}

\begin{mathpar}
  \inferrule* [lab=alpha-equivalence] {} { (x)P \equiv (y)P\{y/x\} \and y \not\in \freenames{P} }
\end{mathpar}

\begin{definition}
Then two processes, $P,Q$, are alpha-equivalent if $P = Q\{\vec{y}/\vec{x}\}$ for
some $\vec{x} \in \boundnames{Q},\vec{y} \in \boundnames{P}$, where $Q\{\vec{y}/\vec{x}\}$
denotes the capture-avoiding substitution of $\vec{y}$ for $\vec{x}$ in $Q$.
\end{definition}

\begin{definition}
  The {\em structural congruence} \cite{SangiorgiWalker} , $\equiv$,
  between processes is the least congruence containing
  alpha-equivalence, satisfying the abelian monoid laws
  (associativity, commutativity and $\pzero$ as identity) for parallel
  composition $|$ and for summation $+$.
\end{definition}

\subsection{Name equivalence}

We take name equivalence, written $\nameeq$, to be the smallest
equivalence relation generated by the following rules.

\begin{mathpar}
\inferrule*[lab=Quote-drop]
{ }
{ \quotep{@{x}} \nameeq x }

\inferrule*[lab=Struct-equiv]
{ P \scong Q }
{ \quotep{P} \nameeq \quotep{Q} }
\end{mathpar}

The astute reader will have noticed that the mutual recursion of names
and processes imposes a mutual recursion on alpha-equivalence and
structural equivalence via name-equivalence. Fortunately, all of this
works out pleasantly and we may calculate in the natural way, free of
concern. The reader interested in the details is referred to the
appendix \ref{appendix:rho_details}.

\subsection{Substitution}

We use $\Proc$ for the set of processes, $\QProc$ for the set of
names, and $\id{\{}\vec{y} / \vec{x} \id{\}}$ to denote partial maps,
$s : \QProc \rightarrow \QProc$. A map, $s$ lifts, uniquely, to a map
on process terms, $\widehat{s} : \Proc \rightarrow \Proc$ by the
following equations.

\begin{mathpar}
  (0) \psubstp{Q}{P} := 0 \\
  (R \juxtap S) \psubstp{Q}{P}
  :=    
  (R)\psubstp{Q}{P} \juxtap (S) \psubstp{Q}{P} \\
  (x?(y).R) \psubstp{Q}{P}    
  :=    
  (x)\substp{Q}{P} (z)\concat( (R \psubstn{z}{y}) \psubstp{Q}{P} ) \\
  (\lift{x}{R}) \psubstp{Q}{P}  
  :=
  \lift{(x)\substp{Q}{P}}{ R \psubstp{Q}{P} } \\
%   (\dropn{x})  \psubstp{Q}{P}       
%   := 
%   \left\{ 
%     \begin{array}{ccc} 
%       \dropn{\quotep{Q}} & & x \nameeq \quotep{P} \\
%       \dropn{x} & & otherwise \\
%     \end{array}
%   \right. 
  (\dropn{x})  \psubstp{Q}{P}       
  := 
  \left\{ 
    \begin{array}{ccc} 
      Q & & x \nameeq \quotep{P} \\
      \dropn{x} & & otherwise \\
    \end{array}
  \right.
\end{mathpar}
 

where

\begin{eqnarray}
  (x)\id{\{} \lpquote Q \rpquote / \lpquote P \rpquote \id{\}}            = 
  \left\{ 
    \begin{array}{ccc}
      \lpquote Q \rpquote & & x \nameeq \lpquote P \rpquote \\
      x & & otherwise \\
    \end{array}
  \right. \nonumber
\end{eqnarray}

and $z$ is chosen distinct from $\quotep{P}$, $\quotep{Q}$, the free
names in $Q$, and all the names in $R$. Our $\alpha$-equivalence will
be built in the standard way from this substitution.

\begin{remark}\label{rem:no_self_referential_names}
  One consequence of these definitions is that $\forall P. \quotep{P}
  \not\in \freenames{P}$.
\end{remark}

\subsection{ Dynamic quote: an example }

Anticipating something of what's to come, consider applying the
substitution, $\widehat{\id{\{}u / z \id{\}}}$, to the following pair
of processes, $\lift{w}{y!(z)}$ and $w[ \lpquote y!(z) \rpquote ]$.

\begin{eqnarray}
	\lift{w}{y!(z)}\widehat{\id{\{}u / z \id{\}}}
		& = &
		\lift{w}{y!(u)} \nonumber\\
	w[ \lpquote y!(z) \rpquote ] \widehat{ \id{\{}u / z \id{\}} }
		& = &
		w[ \lpquote y!(z) \rpquote ] \nonumber
\end{eqnarray}

Because the body of the process between quotes is impervious to
substitution, we get radically different answers. In fact, by
examining the first process in an input context,
e.g. $x?(z).\lift{w}{y!(z)}$, we see that the process under the lift
operator may be shaped by prefixed inputs binding a name inside it. In
this sense, the lift operator will be seen as a way to dynamically
construct processes before reifying them as names.

Finally equipped with these standard features we can present the
dynamics of the calculus.

\subsubsection{Operational semantics} 

Finally, we introduce the computational dynamics. What marks these
algebras as distinct from other more traditionally studied algebraic
structures, e.g. vector spaces or polynomial rings, is the manner in
which dynamics is captured. In traditional structures, dynamics is typically
expressed through morphisms between such structures, as in linear maps
between vector spaces or morphisms between rings. In algebras
associated with the semantics of computation, the dynamics is
expressed as part of the algebraic structure itself, through a
reduction reduction relation typically denoted by $\red$. Below, we
give a recursive presentation of this relation for the calculus used
in the encoding.

$\red \subseteq \pi \times \pi$
$\red : \pi \to \mathcal{P}(\pi)$

\begin{mathpar}
  \inferrule* [lab=Comm] { \textsf{match}( x_{src}, x_{trgt} ) } { x_{trgt}?(y)P \; | \; x_{src}!\langle {Q} \rangle \red P\{\quotep{Q}/y}\} }
  \and \\
  \inferrule* [lab=Par] {{P} \red {P}'} {{{P} | {Q}} \red {{P}' | {Q}}}
  \and
  \inferrule* [lab=Equiv]{{{P} \scong {P}'} \andalso {{P}' \red {Q}'} \andalso {{Q}' \scong {Q}}}{{P} \red {Q}}
\end{mathpar}

\begin{eqnarray*}
  match_{\equiv} (\quotep{P},\quotep{Q}) & := & P \equiv Q \\
  match_{\dagger}(\quotep{P},\quotep{Q}) & := & \forall R. P|Q \red^{*} R => R \red^{*} 0 \\
  match_{K}(\quotep{P},\quotep{Q}) & := & K \mbox{ for some context } K
\end{eqnarray*}

$u?(x)P | u!\langle Q \rangle \red P\{\quotep{Q}/x\}$

%We write $\wred$ for $\red^*$, and $P\red$ if $\exists Q $ such that $ P \red Q$.
We write $P\red$ if $\exists Q $ such that $ P \red Q$ and $P\not\red$, otherwise.

\section{Replication}

As mentioned before, it is known that replication (and hence
recursion) can be implemented in a higher-order process algebra
\cite{SangiorgiWalker}. As our first example of calculation with the
machinery thus far presented we give the construction explicitly in
the {\rhoc}.

\begin{eqnarray}
	D_{x} & := & \prefix{x}{y}{(\binpar{\outputp{x}{y}}{@{y}})} \nonumber\\
	\bangp_{x}{P} & := & \binpar{{x}!\langle{\binpar{D_{x}}{P}}\rangle}{D_{x}} \nonumber
\end{eqnarray}

\begin{eqnarray}
	\bangp_{x}{P} & & \nonumber\\
	=
	& {x}!\langle{(\prefix{x}{y}{(\outputp{x}{y} | @{y})) | P}}\rangle 
	      | \prefix{x}{y}{(\outputp{x}{y} | @{y})} & \nonumber\\
	\red
	& (\outputp{x}{y} | @{y})\substn{\quotep{(\prefix{x}{y}{(@{y} | \outputp{x}{y})) | P}}}{y} & \nonumber\\
	=
	& \outputp{x}{\quotep{(\prefix{x}{y}{(\outputp{x}{y} | @{y})) | P}}}
	  | {(\prefix{x}{y}{(\outputp{x}{y} | @{y})) | P}} & \nonumber\\
	\red
	& \ldots & \nonumber\\
	\red^*
	& P | P | \ldots & \nonumber
\end{eqnarray}

Of course, this encoding, as an implementation, runs away, unfolding
$\bangp{P}$ eagerly. A lazier and more implementable replication
operator, restricted to input-guarded processes, may be obtained as follows.

\begin{eqnarray}
\bangp{\prefix{u}{v}{P}} 
	:= 
	\binpar{\lift{x}{\prefix{u}{v}{(\binpar{D(x)}{P})}}}{D(x)} \nonumber
\end{eqnarray}

\begin{remark}
  Note that the lazier definition still does not deal with summation
  or mixed summation (i.e. sums over input and output). The reader is
  invited to construct definitions of replication that deal with these
  features. 

  Further, the definitions are parameterized in a name, $x$. Can you,
  gentle reader, make a definition that eliminates this parameter and
  guarantees no accidental interaction between the replication
  machinery and the process being replicated -- i.e. no accidental
  sharing of names used by the process to get its work done and the
  name(s) used by the replication to effect copying. This latter
  revision of the definition of replication is crucial to obtaining
  the expected identity $!!P \sim !P$.
\end{remark}

\begin{remark}\label{rem:paradoxical_combinator}
  The reader familiar with the lambda calculus will have noticed the
  similarity between $D$ and the paradoxical combinator.

  [Ed. note: the existence of this seems to suggest we have to be more
  restrictive on the set of processes and names we admit if we are to
  support no-cloning.]
\end{remark}

\subsubsection{Bisimulation}

The computational dynamics gives rise to another kind of equivalence,
the equivalence of computational behavior. As previously mentioned
this is typically captured \emph{via} some form of bisimulation.

% The notion we use in this paper is weak barbed bisimulation
% \cite{milner91polyadicpi}.

The notion we use in this paper is derived from weak barbed
bisimulation \cite{milner91polyadicpi}. 

\begin{definition}
An \emph{observation relation}, $\downarrow_{\mathcal N}$, over a set
of names, $\mathcal N$, is the smallest relation satisfying the rules
below.

\infrule[Out-barb]{y \in {\mathcal N}, \; x \nameeq y}
		  {\outputp{x}{v} \downarrow_{\mathcal N} x}
\infrule[Par-barb]{\mbox{$P\downarrow_{\mathcal N} x$ or $Q\downarrow_{\mathcal N} x$}}
		  {\binpar{P}{Q} \downarrow_{\mathcal N} x}

We write $P \Downarrow_{\mathcal N} x$ if there is $Q$ such that 
$P \wred Q$ and $Q \downarrow_{\mathcal N} x$.
\end{definition}

\begin{definition}
%\label{def.bbisim}
An  ${\mathcal N}$-\emph{barbed bisimulation} over a set of names, ${\mathcal N}$, is a symmetric binary relation 
${\mathcal S}_{\mathcal N}$ between agents such that $P\rel{S}_{\mathcal N}Q$ implies:
\begin{enumerate}
\item If $P \red P'$ then $Q \wred Q'$ and $P'\rel{S}_{\mathcal N} Q'$.
\item If $P\downarrow_{\mathcal N} x$, then $Q\Downarrow_{\mathcal N} x$.
\end{enumerate}
$P$ is ${\mathcal N}$-barbed bisimilar to $Q$, written
$P \wbbisim_{\mathcal N} Q$, if $P \rel{S}_{\mathcal N} Q$ for some ${\mathcal N}$-barbed bisimulation ${\mathcal S}_{\mathcal N}$.
\end{definition}

$\mathcal{R} \subseteq \pi \times \pi$

$P \mathcal{R} Q => \forall P'. P \red P' \Rightarrow \exists Q'. Q \red Q', P' \mathcal{R} Q'$

$P \vdash x \Rightarrow Q \vdash x$

\begin{mathpar}
  \inferrule*[lab=Out-barb]{x \nameeq y}{{y}!\langle{Q}\rangle \vdash x}
  \and
  \inferrule*[lab=Par-barb]{\mbox{$P\vdash x$ or $Q\vdash x$}}{\binpar{P}{Q} \vdash x}
\end{mathpar}

\subsubsection{Contexts}

One of the principle advantages of computational calculi like the
$\pi$-calculus is a well-defined notion of context,
contextual-equivalence and a correlation between
contextual-equivalence and notions of bisimulation. The notion of
context allows the decomposition of a process into (sub-)process and
its syntactic environment, its context. Thus, a context may be
thought of as a process with a ``hole'' (written $\Box$) in it. The
application of a context $M$ to a process $P$, written $M[P]$, is
tantamount to filling the hole in $M$ with $P$. In this paper we do
not need the full weight of this theory, but do make use of the notion
of context in the proof the main theorem. 

\begin{mathpar}
  \inferrule* [lab=summation] {} {{M_{M},M_{N}} \bc \Box \;|\; x.M_{A} \;|\; M_{M}+M_{N}}
  \and
  \inferrule* [lab=agent] {} {{M_{A}} \bc (\vec{x})M_{P} \;| \; \clift{P_0,\ldots,M_{P},\ldots,P_N}}
  \and \\
  \inferrule* [lab=process] {} {{M_{P}} \bc M_{N} \;| \;P|M_{P} }
\end{mathpar} 

\begin{mathpar}
  \inferrule* [lab=sychronization] {} {M_{N} \bc \Box \;|\; x?M_{F} \;|\; x!M_{C}}
  \and
  \inferrule* [lab=abstraction] {} {{M_{F}} \bc (x)M_{P} }
  \and
  \inferrule* [lab=concretion] {} {{M_{C}} \bc \langle M_{P} \rangle }
  \and \\
  \inferrule* [lab=process] {} {{M_{P}} \bc M_{N} \;| \;P|M_{P} }
\end{mathpar}

\begin{definition}[contextual application] Given a context $M$, and
  process $P$, we define the \emph{contextual application}, $M[P] :=
  M\{P/\Box\}$. That is, the contextual application of M to P is the
  substitution of $P$ for $\Box$ in $M$.
\end{definition}

$\meaningof{-} : L \to \mathcal{P}(\pi)$

\begin{mathpar}
  \inferrule* [lab=collection] {} {\meaningof{true} = \pi, \and \meaningof{~E} = \pi \setminus \meaningof{E}, \and \meaningof{E_{1} \& E_{2}} = \meaningof{E_{1}} \cap \meaningof{E_{2}}}
\end{mathpar}

\begin{mathpar}
  \inferrule* [lab=structure] {} {\meaningof{0} = \{ P \in \pi | P \equiv 0 \}, \and \\ \meaningof{E_1 | E_2} = \{ P \in \pi | P \equiv P_{1} | P_{2}, P_{1} \in \meaningof{E_{1}}, P_{2} \in \meaningof{E_2}\} }
\end{mathpar}

\begin{mathpar}
 \inferrule* [lab=behavior] {} {\meaningof{\langle a?b \rangle E} = \{ P \in \pi | P \equiv Q | u?(y)P', \\ \and \\\\ \and \\ \;\;\; u \in \meaningof{a}, \forall z.P'\{z/y\} \in \meaningof{E\{z/b\}}\}, \and \\ \meaningof{a!E} = \{ P \in \pi | P \equiv Q | x!\langle P' \rangle, x \in \meaningof{a} P' \in \meaningof{E}\} }
\end{mathpar}

\begin{mathpar}
 \inferrule* [lab=nominal] {} {\meaningof{\quotep{E}} = \{ \quotep{P} \in \quotep{\pi} | P \in \meaningof{E} \}, \and \meaningof{\quotep{P}} = \{ \quotep{Q} \in \quotep{\pi} | P \equiv Q \} \and \\ \meaningof{@\quotep{E}} = \{ P \in \pi | P \equiv @x, x \in \meaningof{E} \}}
\end{mathpar}

\begin{eqnarray*}
  \\
  \meaningof{-} : TS \to ST
\end{eqnarray*}

\begin{eqnarray*}
  \\
  L : TS \to ST
\end{eqnarray*}

\begin{eqnarray*}
  \\
  P \models E \iff P \in \meaningof{E}
\end{eqnarray*}

\begin{eqnarray*}
  P \approx_{L} Q \iff \forall E \in L. P \models E \iff Q \models E
\end{eqnarray*}

\begin{eqnarray*}
  P \approx_{K} Q
\end{eqnarray*}

\begin{eqnarray*}
  P \approx Q
\end{eqnarray*}

$\approx_{K} = \approx = \approx_{L}$

\subsubsection{Contextual duality}

Note that contexts extend the quotation operation to a family of
operations from processes to names. Given a context, $M$, we can
define a \emph{nominal context}, $\quotep{M}$ by $\quotep{M}[P] :=
\quotep{M[P]}$. To foreshadow what is to come we observe that these
operations enjoy a duality with processes very much like the duality
between vectors and maps from vectors to scalars.

Further, because the calculus is essentially higher-order, we have a
correspondence between contexts and processes. More specifically,
given a name $x$ and a context $M$ we can construct $M^{*}_{x}$ such
that 

\begin{mathpar}
  M^{*}_{x} | \lift{x}{P} \red M[P]
\end{mathpar}

namely,

\begin{mathpar}
  M^{*}_{x} := x?(u).M[\dropn{u}]
\end{mathpar}

The dependence of $M^{*}_{x}$ on a name makes it an abstraction, 

\begin{mathpar}
  M^{*} := (x)x?(u).M[\dropn{u}]
\end{mathpar}

\subsection{Additional notation}

It will sometimes be convenient to denote the process a name
quotes. We already have the notation $x = \quotep{P}$, but it will be
convenient to introduce an alternate notation, $\procn{x}$, when we
want to emphasize the connection to the use of the name. Note that, by
virtue of name equivalence, $\quotep{\procn{x}} \nameeq x$; so, the
notation is consistent with previous definitions.

Further, because names have structure it is possible to effect
substitutions on the basis of that structure. This means we need to
upgrade our notation for substitutions, which we accomplish by
adapting comprehension notation. Thus,

\begin{mathpar}
  P\{ y / x : x \in S \}
\end{mathpar}

is interpreted to mean the process derived from P by replacing (in a
capture-avoiding manner) each occurrence of $x$ in $S$ by $y$. For example,

\begin{mathpar}
  P\{ \quotep{\procn{x}|\procn{x}} / x : x \in \freenames{P} \}
\end{mathpar}

will replace each (occurrence) of a free name $x$ in $P$ by
$\quotep{\procn{x}|\procn{x}}$.

Also, we will avail ourselves of the notation $x^{L}$ and $x^{R}$ to
denote injections of a name into disjoint copies of the name
space. There are numerous ways to accomplish this. One example can be
found in \cite{MeredithR05}. This notation overloads to vectors of
names: $\vec{x}^{\pi} := (x_{i}^{\pi} \; : \; 0 \leq i < |\vec{x}| )$ where $\pi \in \{L,R\}$.

We also use $P^{\Box} := P|\Box$.

In \cite{MeredithR05} an interpretation of the new operator is
given. It turns out that there are several possible interpretations
all enjoying the requisite algebraic properties of the operator (see
\cite{milner91polyadicpi}). We will therefore make liberal use of
$(\nu\; \vec{x})P$.

% subsection the_syntax_and_semantics_of_the_notation_system (end)   

\input{qm2pi.qmops} 

\input{qm2pi.sterngerlach} 

\input{qm2pi.metric} 

% section concurrent_process_calculi (end)

%\input{qm2pi.proofsketch}

% section proof sketch (end)

%\input{qm2pi.slviaknots} 

% section spatial logic via knots (end)

\input{qm2pi.conclusion}

% section conclusion (end)

%\input{qm2pi.dtcodes} 

% section wiring algorithm (end)

\input{qm2pi.ack} 

% section acknowledgments (end)

\newpage


\bibliographystyle{plain}   
\bibliography{../../biblios/main.bib}

\input{qm2pi.rhodetails}

\end{document}

 

% section concurrent_process_calculi (end)

%\documentclass[12pt]{llncs}
%\documentclass{jktr}

\usepackage[pdftex]{hyperref}                   
\usepackage {listings}
\usepackage {mathpartir}
\usepackage{bcprules}
%\usepackage{listings}
                       
\usepackage{graphicx} 
%\usepackage[margins=2.5cm,nohead,nofoot]{geometry}
%\usepackage{geometry}
\usepackage{amsfonts}
\usepackage{amstext}
\usepackage{latexsym}
\usepackage{amssymb}
\usepackage{color}


%\include{myPreamble}
\include{qm2pi.local} 

%\ifpdf
%\usepackage[pdftex]{graphicx}
%\else
%\usepackage{graphicx}
%\fi

 % \ifpdf
%  \usepackage{pdfsync}
%  \if


%\title{Brief Article}
%\author{David F. Snyder}
%\author{L.G. Meredith}

%\address{Dept. of Math., Texas State University--San Marcos, San Marcos, TX 78666}
       
\pagestyle{empty}


\begin{document}

\lstset{language=[Objective]Caml,frame=shadowbox}

\input{qm2pi.front}

% section front matter (end)

\input{qm2pi.intro} 
 
% section introduction (end)

% \input{qm2pi.knotations} 

% section notation (end)

\input{qm2pi.process.calculi} 

% section concurrent_process_calculi_and_spatial_logics_ (end)
    
%\input{qm2pi.knots2pi} 

%\input{qm2pi.trefoil} 

%\input{qm2pi.mainthm} 

% subsection basic_interpretation (end)

%\input{qm2pi.rho.presentation} 
\subsection{The syntax and semantics of the notation system}\label{sub:the_syntax_and_semantics_of_the_notation_system} % (fold)

We now summarize a technical presentation of the calculus that
embodies our theory of dynamics. The typical presentation of such a
calculus follows the style of giving generators and relations on
them. The grammar, below, describing term constructors, freely
generates the set of processes, $\Proc$. This set is then quotiented
by a relation known as structural congruence and it is over this set
that the notion of dynamics is expressed. This presentation is
essentially that of \cite{MeredithR05} with the addition of
polyadicity and summation. For readability we have relegated some of
the technical subtleties to an appendix.

\subsubsection{Process grammar}\label{subsub:process_grammar}

\begin{mathpar}
  \inferrule* [lab=synchronization] {} {{M} \bc \pzero \;|\; x?F \;|\; x!C }
  \and
  \inferrule* [lab=abstraction] {} {{F} \bc (x)P}
  \and
  \inferrule* [lab=concretion] {} {{C} \bc \langle Q \rangle}
  \and
  \inferrule* [lab=process] {} {{P,Q} \bc M \;| \;P|Q \;|\; @{x}}
  \and
  \inferrule* [lab=name] {} {{x} \bc \quotep{P}}
\end{mathpar} 

Note that $\vec{x}$ (resp. $\vec{P}$) denotes a vector of names
(resp. processes) of length $|\vec{x}|$ (resp. $|\vec{P}|$). We adopt
the following useful abbreviations.

\begin{mathpar}
   x?(\vec{y}).P := x.(\vec{y})P \and  x\clift{\vec{P}} := x.\clift{\vec{P}}
   \and x!(y) := \lift{x}{\dropn{y}}
   \and \Pi_{i=0}^{n-1}P_i := P_0 | \ldots | P_{n-1}
\end{mathpar}

\subsubsection{Structural congruence}

\paragraph{Free and bound names and alpha-equivalence.} At the
core of structural equivalence is alpha-equivalence which identifies
process that are the same up to a change of variable. Formally, we
recognize the distinction between free and bound names. The free names
of a process, $\freenames{P}$, may be calculated recursively as
follows:

\begin{mathpar}
\freenames{\pzero} := \emptyset
  \and \\
  \freenames{x?(y).P} := \{ x \} \cup (\freenames{P} \setminus \{ y \})
  \and 
  \freenames{x!\langle P \rangle} := \{ x \} \cup \{ P \} 
  \and \\
  \freenames{P|Q} := \freenames{P} \cup \freenames{Q}
  \and \\
  \freenames{@{x}} := \{ x \}
\end{mathpar}

$\pi$
$\quotep{\pi}$

$\freenames{-} : \pi \to \mathcal{P}(\quotep{\pi})$

\begin{eqnarray*}
  \freenames{\pzero} & := & \emptyset \\
  \freenames{x?(y).P} & := & \{ x \} \cup (\freenames{P} \setminus \{ y \}) \\
  \freenames{x!\langle P \rangle} & := & \{ x \} \cup \{ P \} \\
  \freenames{P|Q} & := & \freenames{P} \cup \freenames{Q} \\
  \freenames{\dropn{x}} & := & \{ x \}
\end{eqnarray*}

The bound names of a process, $\boundnames{P}$, are those names occurring in $P$
that are not free. For example, in $x?(y).0$, the name $x$ is free, while $y$ is bound.

\begin{mathpar}
  \inferrule* [lab=monoidal-laws] {} { P|Q \equiv Q|P \and P|0 \equiv P \and P|(Q|R) \equiv (P|Q)|R }
\end{mathpar}

\begin{mathpar}
  \inferrule* [lab=alpha-equivalence] {} { (x)P \equiv (y)P\{y/x\} \and y \not\in \freenames{P} }
\end{mathpar}

\begin{definition}
Then two processes, $P,Q$, are alpha-equivalent if $P = Q\{\vec{y}/\vec{x}\}$ for
some $\vec{x} \in \boundnames{Q},\vec{y} \in \boundnames{P}$, where $Q\{\vec{y}/\vec{x}\}$
denotes the capture-avoiding substitution of $\vec{y}$ for $\vec{x}$ in $Q$.
\end{definition}

\begin{definition}
  The {\em structural congruence} \cite{SangiorgiWalker} , $\equiv$,
  between processes is the least congruence containing
  alpha-equivalence, satisfying the abelian monoid laws
  (associativity, commutativity and $\pzero$ as identity) for parallel
  composition $|$ and for summation $+$.
\end{definition}

\subsection{Name equivalence}

We take name equivalence, written $\nameeq$, to be the smallest
equivalence relation generated by the following rules.

\begin{mathpar}
\inferrule*[lab=Quote-drop]
{ }
{ \quotep{@{x}} \nameeq x }

\inferrule*[lab=Struct-equiv]
{ P \scong Q }
{ \quotep{P} \nameeq \quotep{Q} }
\end{mathpar}

The astute reader will have noticed that the mutual recursion of names
and processes imposes a mutual recursion on alpha-equivalence and
structural equivalence via name-equivalence. Fortunately, all of this
works out pleasantly and we may calculate in the natural way, free of
concern. The reader interested in the details is referred to the
appendix \ref{appendix:rho_details}.

\subsection{Substitution}

We use $\Proc$ for the set of processes, $\QProc$ for the set of
names, and $\id{\{}\vec{y} / \vec{x} \id{\}}$ to denote partial maps,
$s : \QProc \rightarrow \QProc$. A map, $s$ lifts, uniquely, to a map
on process terms, $\widehat{s} : \Proc \rightarrow \Proc$ by the
following equations.

\begin{mathpar}
  (0) \psubstp{Q}{P} := 0 \\
  (R \juxtap S) \psubstp{Q}{P}
  :=    
  (R)\psubstp{Q}{P} \juxtap (S) \psubstp{Q}{P} \\
  (x?(y).R) \psubstp{Q}{P}    
  :=    
  (x)\substp{Q}{P} (z)\concat( (R \psubstn{z}{y}) \psubstp{Q}{P} ) \\
  (\lift{x}{R}) \psubstp{Q}{P}  
  :=
  \lift{(x)\substp{Q}{P}}{ R \psubstp{Q}{P} } \\
%   (\dropn{x})  \psubstp{Q}{P}       
%   := 
%   \left\{ 
%     \begin{array}{ccc} 
%       \dropn{\quotep{Q}} & & x \nameeq \quotep{P} \\
%       \dropn{x} & & otherwise \\
%     \end{array}
%   \right. 
  (\dropn{x})  \psubstp{Q}{P}       
  := 
  \left\{ 
    \begin{array}{ccc} 
      Q & & x \nameeq \quotep{P} \\
      \dropn{x} & & otherwise \\
    \end{array}
  \right.
\end{mathpar}
 

where

\begin{eqnarray}
  (x)\id{\{} \lpquote Q \rpquote / \lpquote P \rpquote \id{\}}            = 
  \left\{ 
    \begin{array}{ccc}
      \lpquote Q \rpquote & & x \nameeq \lpquote P \rpquote \\
      x & & otherwise \\
    \end{array}
  \right. \nonumber
\end{eqnarray}

and $z$ is chosen distinct from $\quotep{P}$, $\quotep{Q}$, the free
names in $Q$, and all the names in $R$. Our $\alpha$-equivalence will
be built in the standard way from this substitution.

\begin{remark}\label{rem:no_self_referential_names}
  One consequence of these definitions is that $\forall P. \quotep{P}
  \not\in \freenames{P}$.
\end{remark}

\subsection{ Dynamic quote: an example }

Anticipating something of what's to come, consider applying the
substitution, $\widehat{\id{\{}u / z \id{\}}}$, to the following pair
of processes, $\lift{w}{y!(z)}$ and $w[ \lpquote y!(z) \rpquote ]$.

\begin{eqnarray}
	\lift{w}{y!(z)}\widehat{\id{\{}u / z \id{\}}}
		& = &
		\lift{w}{y!(u)} \nonumber\\
	w[ \lpquote y!(z) \rpquote ] \widehat{ \id{\{}u / z \id{\}} }
		& = &
		w[ \lpquote y!(z) \rpquote ] \nonumber
\end{eqnarray}

Because the body of the process between quotes is impervious to
substitution, we get radically different answers. In fact, by
examining the first process in an input context,
e.g. $x?(z).\lift{w}{y!(z)}$, we see that the process under the lift
operator may be shaped by prefixed inputs binding a name inside it. In
this sense, the lift operator will be seen as a way to dynamically
construct processes before reifying them as names.

Finally equipped with these standard features we can present the
dynamics of the calculus.

\subsubsection{Operational semantics} 

Finally, we introduce the computational dynamics. What marks these
algebras as distinct from other more traditionally studied algebraic
structures, e.g. vector spaces or polynomial rings, is the manner in
which dynamics is captured. In traditional structures, dynamics is typically
expressed through morphisms between such structures, as in linear maps
between vector spaces or morphisms between rings. In algebras
associated with the semantics of computation, the dynamics is
expressed as part of the algebraic structure itself, through a
reduction reduction relation typically denoted by $\red$. Below, we
give a recursive presentation of this relation for the calculus used
in the encoding.

$\red \subseteq \pi \times \pi$
$\red : \pi \to \mathcal{P}(\pi)$

\begin{mathpar}
  \inferrule* [lab=Comm] { \textsf{match}( x_{src}, x_{trgt} ) } { x_{trgt}?(y)P \; | \; x_{src}!\langle {Q} \rangle \red P\{\quotep{Q}/y}\} }
  \and \\
  \inferrule* [lab=Par] {{P} \red {P}'} {{{P} | {Q}} \red {{P}' | {Q}}}
  \and
  \inferrule* [lab=Equiv]{{{P} \scong {P}'} \andalso {{P}' \red {Q}'} \andalso {{Q}' \scong {Q}}}{{P} \red {Q}}
\end{mathpar}

\begin{eqnarray*}
  match_{\equiv} (\quotep{P},\quotep{Q}) & := & P \equiv Q \\
  match_{\dagger}(\quotep{P},\quotep{Q}) & := & \forall R. P|Q \red^{*} R => R \red^{*} 0 \\
  match_{K}(\quotep{P},\quotep{Q}) & := & K \mbox{ for some context } K
\end{eqnarray*}

$u?(x)P | u!\langle Q \rangle \red P\{\quotep{Q}/x\}$

%We write $\wred$ for $\red^*$, and $P\red$ if $\exists Q $ such that $ P \red Q$.
We write $P\red$ if $\exists Q $ such that $ P \red Q$ and $P\not\red$, otherwise.

\section{Replication}

As mentioned before, it is known that replication (and hence
recursion) can be implemented in a higher-order process algebra
\cite{SangiorgiWalker}. As our first example of calculation with the
machinery thus far presented we give the construction explicitly in
the {\rhoc}.

\begin{eqnarray}
	D_{x} & := & \prefix{x}{y}{(\binpar{\outputp{x}{y}}{@{y}})} \nonumber\\
	\bangp_{x}{P} & := & \binpar{{x}!\langle{\binpar{D_{x}}{P}}\rangle}{D_{x}} \nonumber
\end{eqnarray}

\begin{eqnarray}
	\bangp_{x}{P} & & \nonumber\\
	=
	& {x}!\langle{(\prefix{x}{y}{(\outputp{x}{y} | @{y})) | P}}\rangle 
	      | \prefix{x}{y}{(\outputp{x}{y} | @{y})} & \nonumber\\
	\red
	& (\outputp{x}{y} | @{y})\substn{\quotep{(\prefix{x}{y}{(@{y} | \outputp{x}{y})) | P}}}{y} & \nonumber\\
	=
	& \outputp{x}{\quotep{(\prefix{x}{y}{(\outputp{x}{y} | @{y})) | P}}}
	  | {(\prefix{x}{y}{(\outputp{x}{y} | @{y})) | P}} & \nonumber\\
	\red
	& \ldots & \nonumber\\
	\red^*
	& P | P | \ldots & \nonumber
\end{eqnarray}

Of course, this encoding, as an implementation, runs away, unfolding
$\bangp{P}$ eagerly. A lazier and more implementable replication
operator, restricted to input-guarded processes, may be obtained as follows.

\begin{eqnarray}
\bangp{\prefix{u}{v}{P}} 
	:= 
	\binpar{\lift{x}{\prefix{u}{v}{(\binpar{D(x)}{P})}}}{D(x)} \nonumber
\end{eqnarray}

\begin{remark}
  Note that the lazier definition still does not deal with summation
  or mixed summation (i.e. sums over input and output). The reader is
  invited to construct definitions of replication that deal with these
  features. 

  Further, the definitions are parameterized in a name, $x$. Can you,
  gentle reader, make a definition that eliminates this parameter and
  guarantees no accidental interaction between the replication
  machinery and the process being replicated -- i.e. no accidental
  sharing of names used by the process to get its work done and the
  name(s) used by the replication to effect copying. This latter
  revision of the definition of replication is crucial to obtaining
  the expected identity $!!P \sim !P$.
\end{remark}

\begin{remark}\label{rem:paradoxical_combinator}
  The reader familiar with the lambda calculus will have noticed the
  similarity between $D$ and the paradoxical combinator.

  [Ed. note: the existence of this seems to suggest we have to be more
  restrictive on the set of processes and names we admit if we are to
  support no-cloning.]
\end{remark}

\subsubsection{Bisimulation}

The computational dynamics gives rise to another kind of equivalence,
the equivalence of computational behavior. As previously mentioned
this is typically captured \emph{via} some form of bisimulation.

% The notion we use in this paper is weak barbed bisimulation
% \cite{milner91polyadicpi}.

The notion we use in this paper is derived from weak barbed
bisimulation \cite{milner91polyadicpi}. 

\begin{definition}
An \emph{observation relation}, $\downarrow_{\mathcal N}$, over a set
of names, $\mathcal N$, is the smallest relation satisfying the rules
below.

\infrule[Out-barb]{y \in {\mathcal N}, \; x \nameeq y}
		  {\outputp{x}{v} \downarrow_{\mathcal N} x}
\infrule[Par-barb]{\mbox{$P\downarrow_{\mathcal N} x$ or $Q\downarrow_{\mathcal N} x$}}
		  {\binpar{P}{Q} \downarrow_{\mathcal N} x}

We write $P \Downarrow_{\mathcal N} x$ if there is $Q$ such that 
$P \wred Q$ and $Q \downarrow_{\mathcal N} x$.
\end{definition}

\begin{definition}
%\label{def.bbisim}
An  ${\mathcal N}$-\emph{barbed bisimulation} over a set of names, ${\mathcal N}$, is a symmetric binary relation 
${\mathcal S}_{\mathcal N}$ between agents such that $P\rel{S}_{\mathcal N}Q$ implies:
\begin{enumerate}
\item If $P \red P'$ then $Q \wred Q'$ and $P'\rel{S}_{\mathcal N} Q'$.
\item If $P\downarrow_{\mathcal N} x$, then $Q\Downarrow_{\mathcal N} x$.
\end{enumerate}
$P$ is ${\mathcal N}$-barbed bisimilar to $Q$, written
$P \wbbisim_{\mathcal N} Q$, if $P \rel{S}_{\mathcal N} Q$ for some ${\mathcal N}$-barbed bisimulation ${\mathcal S}_{\mathcal N}$.
\end{definition}

$\mathcal{R} \subseteq \pi \times \pi$

$P \mathcal{R} Q => \forall P'. P \red P' \Rightarrow \exists Q'. Q \red Q', P' \mathcal{R} Q'$

$P \vdash x \Rightarrow Q \vdash x$

\begin{mathpar}
  \inferrule*[lab=Out-barb]{x \nameeq y}{{y}!\langle{Q}\rangle \vdash x}
  \and
  \inferrule*[lab=Par-barb]{\mbox{$P\vdash x$ or $Q\vdash x$}}{\binpar{P}{Q} \vdash x}
\end{mathpar}

\subsubsection{Contexts}

One of the principle advantages of computational calculi like the
$\pi$-calculus is a well-defined notion of context,
contextual-equivalence and a correlation between
contextual-equivalence and notions of bisimulation. The notion of
context allows the decomposition of a process into (sub-)process and
its syntactic environment, its context. Thus, a context may be
thought of as a process with a ``hole'' (written $\Box$) in it. The
application of a context $M$ to a process $P$, written $M[P]$, is
tantamount to filling the hole in $M$ with $P$. In this paper we do
not need the full weight of this theory, but do make use of the notion
of context in the proof the main theorem. 

\begin{mathpar}
  \inferrule* [lab=summation] {} {{M_{M},M_{N}} \bc \Box \;|\; x.M_{A} \;|\; M_{M}+M_{N}}
  \and
  \inferrule* [lab=agent] {} {{M_{A}} \bc (\vec{x})M_{P} \;| \; \clift{P_0,\ldots,M_{P},\ldots,P_N}}
  \and \\
  \inferrule* [lab=process] {} {{M_{P}} \bc M_{N} \;| \;P|M_{P} }
\end{mathpar} 

\begin{mathpar}
  \inferrule* [lab=sychronization] {} {M_{N} \bc \Box \;|\; x?M_{F} \;|\; x!M_{C}}
  \and
  \inferrule* [lab=abstraction] {} {{M_{F}} \bc (x)M_{P} }
  \and
  \inferrule* [lab=concretion] {} {{M_{C}} \bc \langle M_{P} \rangle }
  \and \\
  \inferrule* [lab=process] {} {{M_{P}} \bc M_{N} \;| \;P|M_{P} }
\end{mathpar}

\begin{definition}[contextual application] Given a context $M$, and
  process $P$, we define the \emph{contextual application}, $M[P] :=
  M\{P/\Box\}$. That is, the contextual application of M to P is the
  substitution of $P$ for $\Box$ in $M$.
\end{definition}

$\meaningof{-} : L \to \mathcal{P}(\pi)$

\begin{mathpar}
  \inferrule* [lab=collection] {} {\meaningof{true} = \pi, \and \meaningof{~E} = \pi \setminus \meaningof{E}, \and \meaningof{E_{1} \& E_{2}} = \meaningof{E_{1}} \cap \meaningof{E_{2}}}
\end{mathpar}

\begin{mathpar}
  \inferrule* [lab=structure] {} {\meaningof{0} = \{ P \in \pi | P \equiv 0 \}, \and \\ \meaningof{E_1 | E_2} = \{ P \in \pi | P \equiv P_{1} | P_{2}, P_{1} \in \meaningof{E_{1}}, P_{2} \in \meaningof{E_2}\} }
\end{mathpar}

\begin{mathpar}
 \inferrule* [lab=behavior] {} {\meaningof{\langle a?b \rangle E} = \{ P \in \pi | P \equiv Q | u?(y)P', \\ \and \\\\ \and \\ \;\;\; u \in \meaningof{a}, \forall z.P'\{z/y\} \in \meaningof{E\{z/b\}}\}, \and \\ \meaningof{a!E} = \{ P \in \pi | P \equiv Q | x!\langle P' \rangle, x \in \meaningof{a} P' \in \meaningof{E}\} }
\end{mathpar}

\begin{mathpar}
 \inferrule* [lab=nominal] {} {\meaningof{\quotep{E}} = \{ \quotep{P} \in \quotep{\pi} | P \in \meaningof{E} \}, \and \meaningof{\quotep{P}} = \{ \quotep{Q} \in \quotep{\pi} | P \equiv Q \} \and \\ \meaningof{@\quotep{E}} = \{ P \in \pi | P \equiv @x, x \in \meaningof{E} \}}
\end{mathpar}

\begin{eqnarray*}
  \\
  \meaningof{-} : TS \to ST
\end{eqnarray*}

\begin{eqnarray*}
  \\
  L : TS \to ST
\end{eqnarray*}

\begin{eqnarray*}
  \\
  P \models E \iff P \in \meaningof{E}
\end{eqnarray*}

\begin{eqnarray*}
  P \approx_{L} Q \iff \forall E \in L. P \models E \iff Q \models E
\end{eqnarray*}

\begin{eqnarray*}
  P \approx_{K} Q
\end{eqnarray*}

\begin{eqnarray*}
  P \approx Q
\end{eqnarray*}

$\approx_{K} = \approx = \approx_{L}$

\subsubsection{Contextual duality}

Note that contexts extend the quotation operation to a family of
operations from processes to names. Given a context, $M$, we can
define a \emph{nominal context}, $\quotep{M}$ by $\quotep{M}[P] :=
\quotep{M[P]}$. To foreshadow what is to come we observe that these
operations enjoy a duality with processes very much like the duality
between vectors and maps from vectors to scalars.

Further, because the calculus is essentially higher-order, we have a
correspondence between contexts and processes. More specifically,
given a name $x$ and a context $M$ we can construct $M^{*}_{x}$ such
that 

\begin{mathpar}
  M^{*}_{x} | \lift{x}{P} \red M[P]
\end{mathpar}

namely,

\begin{mathpar}
  M^{*}_{x} := x?(u).M[\dropn{u}]
\end{mathpar}

The dependence of $M^{*}_{x}$ on a name makes it an abstraction, 

\begin{mathpar}
  M^{*} := (x)x?(u).M[\dropn{u}]
\end{mathpar}

\subsection{Additional notation}

It will sometimes be convenient to denote the process a name
quotes. We already have the notation $x = \quotep{P}$, but it will be
convenient to introduce an alternate notation, $\procn{x}$, when we
want to emphasize the connection to the use of the name. Note that, by
virtue of name equivalence, $\quotep{\procn{x}} \nameeq x$; so, the
notation is consistent with previous definitions.

Further, because names have structure it is possible to effect
substitutions on the basis of that structure. This means we need to
upgrade our notation for substitutions, which we accomplish by
adapting comprehension notation. Thus,

\begin{mathpar}
  P\{ y / x : x \in S \}
\end{mathpar}

is interpreted to mean the process derived from P by replacing (in a
capture-avoiding manner) each occurrence of $x$ in $S$ by $y$. For example,

\begin{mathpar}
  P\{ \quotep{\procn{x}|\procn{x}} / x : x \in \freenames{P} \}
\end{mathpar}

will replace each (occurrence) of a free name $x$ in $P$ by
$\quotep{\procn{x}|\procn{x}}$.

Also, we will avail ourselves of the notation $x^{L}$ and $x^{R}$ to
denote injections of a name into disjoint copies of the name
space. There are numerous ways to accomplish this. One example can be
found in \cite{MeredithR05}. This notation overloads to vectors of
names: $\vec{x}^{\pi} := (x_{i}^{\pi} \; : \; 0 \leq i < |\vec{x}| )$ where $\pi \in \{L,R\}$.

We also use $P^{\Box} := P|\Box$.

In \cite{MeredithR05} an interpretation of the new operator is
given. It turns out that there are several possible interpretations
all enjoying the requisite algebraic properties of the operator (see
\cite{milner91polyadicpi}). We will therefore make liberal use of
$(\nu\; \vec{x})P$.

% subsection the_syntax_and_semantics_of_the_notation_system (end)   

\input{qm2pi.qmops} 

\input{qm2pi.sterngerlach} 

\input{qm2pi.metric} 

% section concurrent_process_calculi (end)

%\input{qm2pi.proofsketch}

% section proof sketch (end)

%\input{qm2pi.slviaknots} 

% section spatial logic via knots (end)

\input{qm2pi.conclusion}

% section conclusion (end)

%\input{qm2pi.dtcodes} 

% section wiring algorithm (end)

\input{qm2pi.ack} 

% section acknowledgments (end)

\newpage


\bibliographystyle{plain}   
\bibliography{../../biblios/main.bib}

\input{qm2pi.rhodetails}

\end{document}



% section proof sketch (end)

%\section{Unlikely characters: spatial logic for
  knots}\label{sub:characteristic_formulae} % (fold)

Associated to the mobile process calculi are a family of logics known
as the Hennessy-Milner logics. These logics typically enjoy a
semantics interpreting formulae as sets of processes that when
factored through the encoding outlined above allows an identification
of classes of knots with logical formulae. In the context of this
encoding the sub-family known as the spatial logics \cite{CairesC03}
\cite{CairesC04} \cite{Caires04} are of particular interest providing
several important features for expressing and reasoning about
properties (i.e. classes) of knots. We hint here at how this may be done.

%\begin{description}
%\item [structural connectives] 
\subsubsection{Structural connectives} The spatial logics enjoy
structural connectives corresponding, at the logical level, to the
parallel composition ($P | Q$) and new name ($(\nu \; x)P$)
connectives for processes. As illustrated in the examples below, these
connectives are extremely expressive given the shape of our encoding.
%\item [decideable satisfaction]

\subsubsection{Decideable satisfaction}
In \cite{Caires04} the satisfaction relation is shown to be decideable
for a rich class of processes. It further turns out that the image of
the our encoding is a proper subset of that class. This result
provides the basis for an algorithm by which to search for knots
enjoying a given property.
%\item [characteristic formulae]

\subsubsection{Characteristic formulae}
In the same paper \cite{Caires04} , Caires presents a means of calculating
characteristic formulae, selecting equivalence classes of processes
up to a pre--specified depth limit on the support set of names. Composed with our
encoding, this characteristic formula can be used to select
characteristic formulae for knots.
%\end{description}

\subsubsection{Spatial logic formulae}

The grammar below (segmented for comprehension) summarizes the syntax
of spatial logic formulae. We employ illustrative examples in the
sequel to provide an intuitive understanding of their meaning
referring the reader to \cite{Caires04} for a more detailed explication
of the semantics.

\begin{mathpar}
  \inferrule* [lab=boolean] {} {{A,B} \bc T \;|\; \neg A \;|\; A \wedge B \;|\; \eta = \eta'}
  \and
  \inferrule* [lab=spatial] {} {|\; \pzero \;|\; A | B \;|\; x \text{\textregistered} A \;|\; \forall x . A \;|\;  H x . A}
  \and
  \inferrule* [lab=behavioral] {} {|\; \alpha . A}
  \and 
  \inferrule* [lab=recursion] {} {|\; X(\vec{u}) \;|\; \mu X(\vec{u}) . A}
  \and
  \inferrule* [lab=action] {} {\alpha \bc \langle x?(\vec{y}) \rangle \;|\; \langle x!(\vec{y}) \rangle \;|\; \langle \tau \rangle}
  \and 
  \inferrule* [lab=name] {} {\eta \bc x \;|\; \tau}
\end{mathpar} 

% subsection characteristic_formulae (end)   	 

\subsection{Example formulae}\label{sub:example_formulae_} % (fold)

\subsubsection{Crossing as formula.}
% 
% \begin{align*}
%   \frac{d}{dx} \sin x &= \cos x 
%   & \frac{d}{dx} e^x &= e^x \\
%   \frac{d}{dx} \cos x &= - \sin x 
%   & \frac{d}{dx} \log x &= \frac{1}{x} \\
% \end{align*} 

\begin{align*}
 \mu C(x_{0},x_{1},y_{0},y_{1},u).&(\langle x_{0}?(z) \rangle(\langle u! \rangle\langle y_{1}!z \rangle C(x_{0},x_{1},y_{0},y_{1},u)) & \\
  & \wedge \langle y_{1}?(z) \rangle (\langle u! \rangle \langle x_{0}!z \rangle C(x_{0},x_{1},y_{0},y_{1},u)) & \\
  & \wedge \langle x_{1}?(z) \rangle (\langle u? \rangle \langle y_{0}!z \rangle C(x_{0},x_{1},y_{0},y_{1},u)) & \\
  & \wedge \langle y_{0}?(z) \rangle (\langle u? \rangle \langle x_{1}!z \rangle C(x_{0},x_{1},y_{0},y_{1},u))) &
\end{align*}

The lexicographical similarity between the shape of this formulae and
the shape of definition of the process representing a crossing reveals
the intuitive meaning of this formulae. It describes the capabilities
of a process that has the right to represent a crossing. For example
it picks out processes that may perform an input on the port $x_0$ in
its initial menu of capabilities. What differentiates the formula
from the process, however, is that the crossing process is the
smallest candidate to satisfy the formula. Infinitely many other
processes -- with internal behavior hidden behind this interface, so
to speak -- also satisfy this formula. Even this simple formula,
then, can be seen to open a new view onto knots, providing a
computational interpretation of \emph{virtual} knots.

Note that this formula is derived by hand. A similar formula can be
derived by employing Caires' calculation of characteristic formula
\cite{Caires04} to the process representing a crossing. In light of
this discussion, we let
$\meaningof{C}_{\phi}(x0,x1,y0,y1,u)$ denote a formula specifying the
dynamics we wish to capture of a crossing. To guarantee we preserve
the shape of the interface and minimal semantics we demand that
$\meaningof{C}_{\phi}(x0,x1,y0,y1,u) \Rightarrow
\textbf{C}(x0,x1,y0,y1,u)$ where $\textbf{C}(x0,x1,y0,y1,u)$ denotes
the formula above.
                            
\subsubsection{Crossing number constraints.}
The moral content of the context lemma (Lemma \ref{context}) is that the notion of
``locality'' in the Reidemeister moves is effectively captured by the
parallel composition operator of the process calculus. This intuition
extends through the logic. Given a formula,
$\meaningof{C}_{\phi}(x0,x1,y0,y1,u)$, we can use the structural
connectives to specify constraints on crossing numbers, such as at
least $n$ crossings, or exactly $n$ crossings.
\begin{mathpar}
  \inferrule* [lab=at-least-n] {} { K^{\geq n}_{\phi}(\vec{xs},\vec{ys}) := \Pi_{i=0}^{n-1} Hu . \meaningof{C}_{\phi}(xs_i,ys_i,u) | T }
  \and 
  \inferrule* [lab=exactly-n] {} { K^{= n}_{\phi}(\vec{xs},\vec{ys}) := \Pi_{i=0}^{n-1} Hu . \meaningof{C}_{\phi}(xs_i,ys_i,u) | \neg (\forall x_0,y_0,x_1,y_1,u . \meaningof{C}_{\phi}(x_0,y_0,x_1,y_1,u) | T) }
\end{mathpar}

To round out this section, recall that the encoding of an $n$-crossing
knot decomposes into a parallel composition of $n$ \emph{copies} of a
crossing process together with a wiring harness. To specify different
knot classes with the same crossing number amounts to specifying
logical constraints on the wiring harness. In the interest of space,
we defer examples to a forthcoming paper. Suffice it to say that both
the conditions ``alternating knot'' and ``contains the tangle
corresponding to 5/3'' are expressible. For example, it is possible to
calculate the characteristic formula of a process corresponding to the
tangle 5/3 and conjoin it into the classifying formula via the
composition connective of the logic.

Finally, we wish to observe that it is entirely within reason to
contemplate a more domain-specific version of spatial logic tailored
to the shape of processes in the image of the encoding. Such a
domain-specific logic would have a better claim to the title formal
language of knot properties.

% subsection example_formulae_ (end)

% section knots_as_processes (end) 

% section spatial logic via knots (end)

\section{Conclusions and future work}

\paragraph{Testing physical space}
You, gentle reader, may wonder why of all the theorems to be proved
given this set up we pick the one above. In some sense it's hardly
central to quantum mechanics. We see it as central in the sense that
it firmly establishes a notion of physical space arising from a notion
of the equivalence of behavior. Relating bisimulation to a metric is a
big step forward, but one is faced with interpreting the relationship
of that metric space to something more physical. Quantum mechanical
notions of ``physical'' space are still far from intuitive, but by
relating this idea of distance as testing to calculations that predict
physical circumstances we are making a not insignificant step forward
toward an understanding of the physical space we inhabit as
essentially dynamic.

\paragraph{Effectivity and simulation}
One of the observations we have yet to make is that the entire program
spelled out here is effective. We have built various interpreters for
the reflective calculus at work in this interpretation. In principle,
then, we can simulate quantum mechanics on a computer. The place where
the simulation may lose fidelity is the infinitely branching summation
for the annihilator.

In this connection i also want to point out that the evaluation style
calculation of the inner product puts the non-determinism of the
summation right at the heart of measurement. This suggests that
Milner's original reduction-based formulation of the dynamics of his
calculi in terms of sums was not just notationally suggestive of a
notion of measure-and-continue but captured some significant part of
the physics.

\paragraph{Quantum continuations}
In light of this last observation i want to point out that the
predominant account of quantum mechanics is missing a key aspect of a
truly compositional story of the physical situation. In a real lab,
when a measurement is made the observation can be made to feed into
another device that then makes another measurement conditioned on the
results of the first. This means that after the superposition was
collapsed the entire experimental set up remained in
superposition. While QM offers a means of writing this down it doesn't
quite line up well with the well-trodden formulation of computation
and continuation that we see so succinctly expressed in Milner's
calculi. This suggests that there might be advantages to this account
of dynamics waiting to be explored.

\paragraph{Quantum logic}
In this connection, we also note that by virtue of having the
Hennessy-Milner construction, we can pull the construction through the
interpretation of QM. This gives us a natural candidate for a quantum
logic that enjoys an extremely tight connection with it's domain of
interpretation, making the construction much less ad hoc (rather it is
the image of functor!).

\paragraph{Quantum probabiity}
i have questions about the basis of the interpretation of inner
product as probability amplitude. In particular, using which
axiomatization of probability theory does the notion of probability
amplitude earn the right to be so dubbed? In other words, where is the
proof that the operation for calculating a probability amplitude (and
then squaring) satisfies the axioms of what it means to calculate a
probability? Even if such a proof exists (i have yet to find it in the
literature), i wonder if it might not be possible to turn things on
their heads. Can we view the calculation of the probability amplitude
as an axiomatization of probability? If so, then the definition we
give for calculating probability amplitude may provide the basis for
an \emph{effective} theory of probability.

\paragraph{Quantum vs ``biological'' information}
Finally, i want to conclude with a more philosophical observation. At
a recent workshop in which QM was a predominant topic i noticed
something about quantum information. The speaker was giving a riveting
discussion of axiomatic QM and showing how properties of ``no
cloning'' and ``no deleting'' emerged as consequences of the
axiomatization. Theorems of this form are necessary to give us a sense
of confidence that our axioms characterize the physical theory. What
struck me, though, was that if quantum information is neither erasable
nor replicable it is markedly different from \emph{life}. Two of the
things we know about life is that

\begin{itemize}
  \item it ends;
  \item to gain some measure of persistence, to transcend it's
    finitude it is imminently copyable.
\end{itemize}

Both of these qualities are summarized succinctly in the aphorism: all
flesh is grass. For me these two kinds of ``information'' -- call them
quantum and biological -- are end points on a spectrum of strategies
for persistence. At one end, we have those curious entities that enjoy
uniqueness and permanence; at the other, we have those who in the face
of a certain end and an uncertain present make a go of passing
something on. To me one of the more remarkable aspects of the latter
strategy is that in the presence of noise (and certain features of
copying) we get a kind of dynamism, a chance for improvement against a
given persistent condition.

% subsection other_calculi_other_bisimulations_and_geometry_as_behavior (end)




% section conclusion (end)

%\documentclass[12pt]{llncs}
%\documentclass{jktr}

\usepackage[pdftex]{hyperref}                   
\usepackage {listings}
\usepackage {mathpartir}
\usepackage{bcprules}
%\usepackage{listings}
                       
\usepackage{graphicx} 
%\usepackage[margins=2.5cm,nohead,nofoot]{geometry}
%\usepackage{geometry}
\usepackage{amsfonts}
\usepackage{amstext}
\usepackage{latexsym}
\usepackage{amssymb}
\usepackage{color}


%\include{myPreamble}
\include{qm2pi.local} 

%\ifpdf
%\usepackage[pdftex]{graphicx}
%\else
%\usepackage{graphicx}
%\fi

 % \ifpdf
%  \usepackage{pdfsync}
%  \if


%\title{Brief Article}
%\author{David F. Snyder}
%\author{L.G. Meredith}

%\address{Dept. of Math., Texas State University--San Marcos, San Marcos, TX 78666}
       
\pagestyle{empty}


\begin{document}

\lstset{language=[Objective]Caml,frame=shadowbox}

\input{qm2pi.front}

% section front matter (end)

\input{qm2pi.intro} 
 
% section introduction (end)

% \input{qm2pi.knotations} 

% section notation (end)

\input{qm2pi.process.calculi} 

% section concurrent_process_calculi_and_spatial_logics_ (end)
    
%\input{qm2pi.knots2pi} 

%\input{qm2pi.trefoil} 

%\input{qm2pi.mainthm} 

% subsection basic_interpretation (end)

%\input{qm2pi.rho.presentation} 
\subsection{The syntax and semantics of the notation system}\label{sub:the_syntax_and_semantics_of_the_notation_system} % (fold)

We now summarize a technical presentation of the calculus that
embodies our theory of dynamics. The typical presentation of such a
calculus follows the style of giving generators and relations on
them. The grammar, below, describing term constructors, freely
generates the set of processes, $\Proc$. This set is then quotiented
by a relation known as structural congruence and it is over this set
that the notion of dynamics is expressed. This presentation is
essentially that of \cite{MeredithR05} with the addition of
polyadicity and summation. For readability we have relegated some of
the technical subtleties to an appendix.

\subsubsection{Process grammar}\label{subsub:process_grammar}

\begin{mathpar}
  \inferrule* [lab=synchronization] {} {{M} \bc \pzero \;|\; x?F \;|\; x!C }
  \and
  \inferrule* [lab=abstraction] {} {{F} \bc (x)P}
  \and
  \inferrule* [lab=concretion] {} {{C} \bc \langle Q \rangle}
  \and
  \inferrule* [lab=process] {} {{P,Q} \bc M \;| \;P|Q \;|\; @{x}}
  \and
  \inferrule* [lab=name] {} {{x} \bc \quotep{P}}
\end{mathpar} 

Note that $\vec{x}$ (resp. $\vec{P}$) denotes a vector of names
(resp. processes) of length $|\vec{x}|$ (resp. $|\vec{P}|$). We adopt
the following useful abbreviations.

\begin{mathpar}
   x?(\vec{y}).P := x.(\vec{y})P \and  x\clift{\vec{P}} := x.\clift{\vec{P}}
   \and x!(y) := \lift{x}{\dropn{y}}
   \and \Pi_{i=0}^{n-1}P_i := P_0 | \ldots | P_{n-1}
\end{mathpar}

\subsubsection{Structural congruence}

\paragraph{Free and bound names and alpha-equivalence.} At the
core of structural equivalence is alpha-equivalence which identifies
process that are the same up to a change of variable. Formally, we
recognize the distinction between free and bound names. The free names
of a process, $\freenames{P}$, may be calculated recursively as
follows:

\begin{mathpar}
\freenames{\pzero} := \emptyset
  \and \\
  \freenames{x?(y).P} := \{ x \} \cup (\freenames{P} \setminus \{ y \})
  \and 
  \freenames{x!\langle P \rangle} := \{ x \} \cup \{ P \} 
  \and \\
  \freenames{P|Q} := \freenames{P} \cup \freenames{Q}
  \and \\
  \freenames{@{x}} := \{ x \}
\end{mathpar}

$\pi$
$\quotep{\pi}$

$\freenames{-} : \pi \to \mathcal{P}(\quotep{\pi})$

\begin{eqnarray*}
  \freenames{\pzero} & := & \emptyset \\
  \freenames{x?(y).P} & := & \{ x \} \cup (\freenames{P} \setminus \{ y \}) \\
  \freenames{x!\langle P \rangle} & := & \{ x \} \cup \{ P \} \\
  \freenames{P|Q} & := & \freenames{P} \cup \freenames{Q} \\
  \freenames{\dropn{x}} & := & \{ x \}
\end{eqnarray*}

The bound names of a process, $\boundnames{P}$, are those names occurring in $P$
that are not free. For example, in $x?(y).0$, the name $x$ is free, while $y$ is bound.

\begin{mathpar}
  \inferrule* [lab=monoidal-laws] {} { P|Q \equiv Q|P \and P|0 \equiv P \and P|(Q|R) \equiv (P|Q)|R }
\end{mathpar}

\begin{mathpar}
  \inferrule* [lab=alpha-equivalence] {} { (x)P \equiv (y)P\{y/x\} \and y \not\in \freenames{P} }
\end{mathpar}

\begin{definition}
Then two processes, $P,Q$, are alpha-equivalent if $P = Q\{\vec{y}/\vec{x}\}$ for
some $\vec{x} \in \boundnames{Q},\vec{y} \in \boundnames{P}$, where $Q\{\vec{y}/\vec{x}\}$
denotes the capture-avoiding substitution of $\vec{y}$ for $\vec{x}$ in $Q$.
\end{definition}

\begin{definition}
  The {\em structural congruence} \cite{SangiorgiWalker} , $\equiv$,
  between processes is the least congruence containing
  alpha-equivalence, satisfying the abelian monoid laws
  (associativity, commutativity and $\pzero$ as identity) for parallel
  composition $|$ and for summation $+$.
\end{definition}

\subsection{Name equivalence}

We take name equivalence, written $\nameeq$, to be the smallest
equivalence relation generated by the following rules.

\begin{mathpar}
\inferrule*[lab=Quote-drop]
{ }
{ \quotep{@{x}} \nameeq x }

\inferrule*[lab=Struct-equiv]
{ P \scong Q }
{ \quotep{P} \nameeq \quotep{Q} }
\end{mathpar}

The astute reader will have noticed that the mutual recursion of names
and processes imposes a mutual recursion on alpha-equivalence and
structural equivalence via name-equivalence. Fortunately, all of this
works out pleasantly and we may calculate in the natural way, free of
concern. The reader interested in the details is referred to the
appendix \ref{appendix:rho_details}.

\subsection{Substitution}

We use $\Proc$ for the set of processes, $\QProc$ for the set of
names, and $\id{\{}\vec{y} / \vec{x} \id{\}}$ to denote partial maps,
$s : \QProc \rightarrow \QProc$. A map, $s$ lifts, uniquely, to a map
on process terms, $\widehat{s} : \Proc \rightarrow \Proc$ by the
following equations.

\begin{mathpar}
  (0) \psubstp{Q}{P} := 0 \\
  (R \juxtap S) \psubstp{Q}{P}
  :=    
  (R)\psubstp{Q}{P} \juxtap (S) \psubstp{Q}{P} \\
  (x?(y).R) \psubstp{Q}{P}    
  :=    
  (x)\substp{Q}{P} (z)\concat( (R \psubstn{z}{y}) \psubstp{Q}{P} ) \\
  (\lift{x}{R}) \psubstp{Q}{P}  
  :=
  \lift{(x)\substp{Q}{P}}{ R \psubstp{Q}{P} } \\
%   (\dropn{x})  \psubstp{Q}{P}       
%   := 
%   \left\{ 
%     \begin{array}{ccc} 
%       \dropn{\quotep{Q}} & & x \nameeq \quotep{P} \\
%       \dropn{x} & & otherwise \\
%     \end{array}
%   \right. 
  (\dropn{x})  \psubstp{Q}{P}       
  := 
  \left\{ 
    \begin{array}{ccc} 
      Q & & x \nameeq \quotep{P} \\
      \dropn{x} & & otherwise \\
    \end{array}
  \right.
\end{mathpar}
 

where

\begin{eqnarray}
  (x)\id{\{} \lpquote Q \rpquote / \lpquote P \rpquote \id{\}}            = 
  \left\{ 
    \begin{array}{ccc}
      \lpquote Q \rpquote & & x \nameeq \lpquote P \rpquote \\
      x & & otherwise \\
    \end{array}
  \right. \nonumber
\end{eqnarray}

and $z$ is chosen distinct from $\quotep{P}$, $\quotep{Q}$, the free
names in $Q$, and all the names in $R$. Our $\alpha$-equivalence will
be built in the standard way from this substitution.

\begin{remark}\label{rem:no_self_referential_names}
  One consequence of these definitions is that $\forall P. \quotep{P}
  \not\in \freenames{P}$.
\end{remark}

\subsection{ Dynamic quote: an example }

Anticipating something of what's to come, consider applying the
substitution, $\widehat{\id{\{}u / z \id{\}}}$, to the following pair
of processes, $\lift{w}{y!(z)}$ and $w[ \lpquote y!(z) \rpquote ]$.

\begin{eqnarray}
	\lift{w}{y!(z)}\widehat{\id{\{}u / z \id{\}}}
		& = &
		\lift{w}{y!(u)} \nonumber\\
	w[ \lpquote y!(z) \rpquote ] \widehat{ \id{\{}u / z \id{\}} }
		& = &
		w[ \lpquote y!(z) \rpquote ] \nonumber
\end{eqnarray}

Because the body of the process between quotes is impervious to
substitution, we get radically different answers. In fact, by
examining the first process in an input context,
e.g. $x?(z).\lift{w}{y!(z)}$, we see that the process under the lift
operator may be shaped by prefixed inputs binding a name inside it. In
this sense, the lift operator will be seen as a way to dynamically
construct processes before reifying them as names.

Finally equipped with these standard features we can present the
dynamics of the calculus.

\subsubsection{Operational semantics} 

Finally, we introduce the computational dynamics. What marks these
algebras as distinct from other more traditionally studied algebraic
structures, e.g. vector spaces or polynomial rings, is the manner in
which dynamics is captured. In traditional structures, dynamics is typically
expressed through morphisms between such structures, as in linear maps
between vector spaces or morphisms between rings. In algebras
associated with the semantics of computation, the dynamics is
expressed as part of the algebraic structure itself, through a
reduction reduction relation typically denoted by $\red$. Below, we
give a recursive presentation of this relation for the calculus used
in the encoding.

$\red \subseteq \pi \times \pi$
$\red : \pi \to \mathcal{P}(\pi)$

\begin{mathpar}
  \inferrule* [lab=Comm] { \textsf{match}( x_{src}, x_{trgt} ) } { x_{trgt}?(y)P \; | \; x_{src}!\langle {Q} \rangle \red P\{\quotep{Q}/y}\} }
  \and \\
  \inferrule* [lab=Par] {{P} \red {P}'} {{{P} | {Q}} \red {{P}' | {Q}}}
  \and
  \inferrule* [lab=Equiv]{{{P} \scong {P}'} \andalso {{P}' \red {Q}'} \andalso {{Q}' \scong {Q}}}{{P} \red {Q}}
\end{mathpar}

\begin{eqnarray*}
  match_{\equiv} (\quotep{P},\quotep{Q}) & := & P \equiv Q \\
  match_{\dagger}(\quotep{P},\quotep{Q}) & := & \forall R. P|Q \red^{*} R => R \red^{*} 0 \\
  match_{K}(\quotep{P},\quotep{Q}) & := & K \mbox{ for some context } K
\end{eqnarray*}

$u?(x)P | u!\langle Q \rangle \red P\{\quotep{Q}/x\}$

%We write $\wred$ for $\red^*$, and $P\red$ if $\exists Q $ such that $ P \red Q$.
We write $P\red$ if $\exists Q $ such that $ P \red Q$ and $P\not\red$, otherwise.

\section{Replication}

As mentioned before, it is known that replication (and hence
recursion) can be implemented in a higher-order process algebra
\cite{SangiorgiWalker}. As our first example of calculation with the
machinery thus far presented we give the construction explicitly in
the {\rhoc}.

\begin{eqnarray}
	D_{x} & := & \prefix{x}{y}{(\binpar{\outputp{x}{y}}{@{y}})} \nonumber\\
	\bangp_{x}{P} & := & \binpar{{x}!\langle{\binpar{D_{x}}{P}}\rangle}{D_{x}} \nonumber
\end{eqnarray}

\begin{eqnarray}
	\bangp_{x}{P} & & \nonumber\\
	=
	& {x}!\langle{(\prefix{x}{y}{(\outputp{x}{y} | @{y})) | P}}\rangle 
	      | \prefix{x}{y}{(\outputp{x}{y} | @{y})} & \nonumber\\
	\red
	& (\outputp{x}{y} | @{y})\substn{\quotep{(\prefix{x}{y}{(@{y} | \outputp{x}{y})) | P}}}{y} & \nonumber\\
	=
	& \outputp{x}{\quotep{(\prefix{x}{y}{(\outputp{x}{y} | @{y})) | P}}}
	  | {(\prefix{x}{y}{(\outputp{x}{y} | @{y})) | P}} & \nonumber\\
	\red
	& \ldots & \nonumber\\
	\red^*
	& P | P | \ldots & \nonumber
\end{eqnarray}

Of course, this encoding, as an implementation, runs away, unfolding
$\bangp{P}$ eagerly. A lazier and more implementable replication
operator, restricted to input-guarded processes, may be obtained as follows.

\begin{eqnarray}
\bangp{\prefix{u}{v}{P}} 
	:= 
	\binpar{\lift{x}{\prefix{u}{v}{(\binpar{D(x)}{P})}}}{D(x)} \nonumber
\end{eqnarray}

\begin{remark}
  Note that the lazier definition still does not deal with summation
  or mixed summation (i.e. sums over input and output). The reader is
  invited to construct definitions of replication that deal with these
  features. 

  Further, the definitions are parameterized in a name, $x$. Can you,
  gentle reader, make a definition that eliminates this parameter and
  guarantees no accidental interaction between the replication
  machinery and the process being replicated -- i.e. no accidental
  sharing of names used by the process to get its work done and the
  name(s) used by the replication to effect copying. This latter
  revision of the definition of replication is crucial to obtaining
  the expected identity $!!P \sim !P$.
\end{remark}

\begin{remark}\label{rem:paradoxical_combinator}
  The reader familiar with the lambda calculus will have noticed the
  similarity between $D$ and the paradoxical combinator.

  [Ed. note: the existence of this seems to suggest we have to be more
  restrictive on the set of processes and names we admit if we are to
  support no-cloning.]
\end{remark}

\subsubsection{Bisimulation}

The computational dynamics gives rise to another kind of equivalence,
the equivalence of computational behavior. As previously mentioned
this is typically captured \emph{via} some form of bisimulation.

% The notion we use in this paper is weak barbed bisimulation
% \cite{milner91polyadicpi}.

The notion we use in this paper is derived from weak barbed
bisimulation \cite{milner91polyadicpi}. 

\begin{definition}
An \emph{observation relation}, $\downarrow_{\mathcal N}$, over a set
of names, $\mathcal N$, is the smallest relation satisfying the rules
below.

\infrule[Out-barb]{y \in {\mathcal N}, \; x \nameeq y}
		  {\outputp{x}{v} \downarrow_{\mathcal N} x}
\infrule[Par-barb]{\mbox{$P\downarrow_{\mathcal N} x$ or $Q\downarrow_{\mathcal N} x$}}
		  {\binpar{P}{Q} \downarrow_{\mathcal N} x}

We write $P \Downarrow_{\mathcal N} x$ if there is $Q$ such that 
$P \wred Q$ and $Q \downarrow_{\mathcal N} x$.
\end{definition}

\begin{definition}
%\label{def.bbisim}
An  ${\mathcal N}$-\emph{barbed bisimulation} over a set of names, ${\mathcal N}$, is a symmetric binary relation 
${\mathcal S}_{\mathcal N}$ between agents such that $P\rel{S}_{\mathcal N}Q$ implies:
\begin{enumerate}
\item If $P \red P'$ then $Q \wred Q'$ and $P'\rel{S}_{\mathcal N} Q'$.
\item If $P\downarrow_{\mathcal N} x$, then $Q\Downarrow_{\mathcal N} x$.
\end{enumerate}
$P$ is ${\mathcal N}$-barbed bisimilar to $Q$, written
$P \wbbisim_{\mathcal N} Q$, if $P \rel{S}_{\mathcal N} Q$ for some ${\mathcal N}$-barbed bisimulation ${\mathcal S}_{\mathcal N}$.
\end{definition}

$\mathcal{R} \subseteq \pi \times \pi$

$P \mathcal{R} Q => \forall P'. P \red P' \Rightarrow \exists Q'. Q \red Q', P' \mathcal{R} Q'$

$P \vdash x \Rightarrow Q \vdash x$

\begin{mathpar}
  \inferrule*[lab=Out-barb]{x \nameeq y}{{y}!\langle{Q}\rangle \vdash x}
  \and
  \inferrule*[lab=Par-barb]{\mbox{$P\vdash x$ or $Q\vdash x$}}{\binpar{P}{Q} \vdash x}
\end{mathpar}

\subsubsection{Contexts}

One of the principle advantages of computational calculi like the
$\pi$-calculus is a well-defined notion of context,
contextual-equivalence and a correlation between
contextual-equivalence and notions of bisimulation. The notion of
context allows the decomposition of a process into (sub-)process and
its syntactic environment, its context. Thus, a context may be
thought of as a process with a ``hole'' (written $\Box$) in it. The
application of a context $M$ to a process $P$, written $M[P]$, is
tantamount to filling the hole in $M$ with $P$. In this paper we do
not need the full weight of this theory, but do make use of the notion
of context in the proof the main theorem. 

\begin{mathpar}
  \inferrule* [lab=summation] {} {{M_{M},M_{N}} \bc \Box \;|\; x.M_{A} \;|\; M_{M}+M_{N}}
  \and
  \inferrule* [lab=agent] {} {{M_{A}} \bc (\vec{x})M_{P} \;| \; \clift{P_0,\ldots,M_{P},\ldots,P_N}}
  \and \\
  \inferrule* [lab=process] {} {{M_{P}} \bc M_{N} \;| \;P|M_{P} }
\end{mathpar} 

\begin{mathpar}
  \inferrule* [lab=sychronization] {} {M_{N} \bc \Box \;|\; x?M_{F} \;|\; x!M_{C}}
  \and
  \inferrule* [lab=abstraction] {} {{M_{F}} \bc (x)M_{P} }
  \and
  \inferrule* [lab=concretion] {} {{M_{C}} \bc \langle M_{P} \rangle }
  \and \\
  \inferrule* [lab=process] {} {{M_{P}} \bc M_{N} \;| \;P|M_{P} }
\end{mathpar}

\begin{definition}[contextual application] Given a context $M$, and
  process $P$, we define the \emph{contextual application}, $M[P] :=
  M\{P/\Box\}$. That is, the contextual application of M to P is the
  substitution of $P$ for $\Box$ in $M$.
\end{definition}

$\meaningof{-} : L \to \mathcal{P}(\pi)$

\begin{mathpar}
  \inferrule* [lab=collection] {} {\meaningof{true} = \pi, \and \meaningof{~E} = \pi \setminus \meaningof{E}, \and \meaningof{E_{1} \& E_{2}} = \meaningof{E_{1}} \cap \meaningof{E_{2}}}
\end{mathpar}

\begin{mathpar}
  \inferrule* [lab=structure] {} {\meaningof{0} = \{ P \in \pi | P \equiv 0 \}, \and \\ \meaningof{E_1 | E_2} = \{ P \in \pi | P \equiv P_{1} | P_{2}, P_{1} \in \meaningof{E_{1}}, P_{2} \in \meaningof{E_2}\} }
\end{mathpar}

\begin{mathpar}
 \inferrule* [lab=behavior] {} {\meaningof{\langle a?b \rangle E} = \{ P \in \pi | P \equiv Q | u?(y)P', \\ \and \\\\ \and \\ \;\;\; u \in \meaningof{a}, \forall z.P'\{z/y\} \in \meaningof{E\{z/b\}}\}, \and \\ \meaningof{a!E} = \{ P \in \pi | P \equiv Q | x!\langle P' \rangle, x \in \meaningof{a} P' \in \meaningof{E}\} }
\end{mathpar}

\begin{mathpar}
 \inferrule* [lab=nominal] {} {\meaningof{\quotep{E}} = \{ \quotep{P} \in \quotep{\pi} | P \in \meaningof{E} \}, \and \meaningof{\quotep{P}} = \{ \quotep{Q} \in \quotep{\pi} | P \equiv Q \} \and \\ \meaningof{@\quotep{E}} = \{ P \in \pi | P \equiv @x, x \in \meaningof{E} \}}
\end{mathpar}

\begin{eqnarray*}
  \\
  \meaningof{-} : TS \to ST
\end{eqnarray*}

\begin{eqnarray*}
  \\
  L : TS \to ST
\end{eqnarray*}

\begin{eqnarray*}
  \\
  P \models E \iff P \in \meaningof{E}
\end{eqnarray*}

\begin{eqnarray*}
  P \approx_{L} Q \iff \forall E \in L. P \models E \iff Q \models E
\end{eqnarray*}

\begin{eqnarray*}
  P \approx_{K} Q
\end{eqnarray*}

\begin{eqnarray*}
  P \approx Q
\end{eqnarray*}

$\approx_{K} = \approx = \approx_{L}$

\subsubsection{Contextual duality}

Note that contexts extend the quotation operation to a family of
operations from processes to names. Given a context, $M$, we can
define a \emph{nominal context}, $\quotep{M}$ by $\quotep{M}[P] :=
\quotep{M[P]}$. To foreshadow what is to come we observe that these
operations enjoy a duality with processes very much like the duality
between vectors and maps from vectors to scalars.

Further, because the calculus is essentially higher-order, we have a
correspondence between contexts and processes. More specifically,
given a name $x$ and a context $M$ we can construct $M^{*}_{x}$ such
that 

\begin{mathpar}
  M^{*}_{x} | \lift{x}{P} \red M[P]
\end{mathpar}

namely,

\begin{mathpar}
  M^{*}_{x} := x?(u).M[\dropn{u}]
\end{mathpar}

The dependence of $M^{*}_{x}$ on a name makes it an abstraction, 

\begin{mathpar}
  M^{*} := (x)x?(u).M[\dropn{u}]
\end{mathpar}

\subsection{Additional notation}

It will sometimes be convenient to denote the process a name
quotes. We already have the notation $x = \quotep{P}$, but it will be
convenient to introduce an alternate notation, $\procn{x}$, when we
want to emphasize the connection to the use of the name. Note that, by
virtue of name equivalence, $\quotep{\procn{x}} \nameeq x$; so, the
notation is consistent with previous definitions.

Further, because names have structure it is possible to effect
substitutions on the basis of that structure. This means we need to
upgrade our notation for substitutions, which we accomplish by
adapting comprehension notation. Thus,

\begin{mathpar}
  P\{ y / x : x \in S \}
\end{mathpar}

is interpreted to mean the process derived from P by replacing (in a
capture-avoiding manner) each occurrence of $x$ in $S$ by $y$. For example,

\begin{mathpar}
  P\{ \quotep{\procn{x}|\procn{x}} / x : x \in \freenames{P} \}
\end{mathpar}

will replace each (occurrence) of a free name $x$ in $P$ by
$\quotep{\procn{x}|\procn{x}}$.

Also, we will avail ourselves of the notation $x^{L}$ and $x^{R}$ to
denote injections of a name into disjoint copies of the name
space. There are numerous ways to accomplish this. One example can be
found in \cite{MeredithR05}. This notation overloads to vectors of
names: $\vec{x}^{\pi} := (x_{i}^{\pi} \; : \; 0 \leq i < |\vec{x}| )$ where $\pi \in \{L,R\}$.

We also use $P^{\Box} := P|\Box$.

In \cite{MeredithR05} an interpretation of the new operator is
given. It turns out that there are several possible interpretations
all enjoying the requisite algebraic properties of the operator (see
\cite{milner91polyadicpi}). We will therefore make liberal use of
$(\nu\; \vec{x})P$.

% subsection the_syntax_and_semantics_of_the_notation_system (end)   

\input{qm2pi.qmops} 

\input{qm2pi.sterngerlach} 

\input{qm2pi.metric} 

% section concurrent_process_calculi (end)

%\input{qm2pi.proofsketch}

% section proof sketch (end)

%\input{qm2pi.slviaknots} 

% section spatial logic via knots (end)

\input{qm2pi.conclusion}

% section conclusion (end)

%\input{qm2pi.dtcodes} 

% section wiring algorithm (end)

\input{qm2pi.ack} 

% section acknowledgments (end)

\newpage


\bibliographystyle{plain}   
\bibliography{../../biblios/main.bib}

\input{qm2pi.rhodetails}

\end{document}

 

% section wiring algorithm (end)

\documentclass[12pt]{llncs}
%\documentclass{jktr}

\usepackage[pdftex]{hyperref}                   
\usepackage {listings}
\usepackage {mathpartir}
\usepackage{bcprules}
%\usepackage{listings}
                       
\usepackage{graphicx} 
%\usepackage[margins=2.5cm,nohead,nofoot]{geometry}
%\usepackage{geometry}
\usepackage{amsfonts}
\usepackage{amstext}
\usepackage{latexsym}
\usepackage{amssymb}
\usepackage{color}


%\include{myPreamble}
\include{qm2pi.local} 

%\ifpdf
%\usepackage[pdftex]{graphicx}
%\else
%\usepackage{graphicx}
%\fi

 % \ifpdf
%  \usepackage{pdfsync}
%  \if


%\title{Brief Article}
%\author{David F. Snyder}
%\author{L.G. Meredith}

%\address{Dept. of Math., Texas State University--San Marcos, San Marcos, TX 78666}
       
\pagestyle{empty}


\begin{document}

\lstset{language=[Objective]Caml,frame=shadowbox}

\input{qm2pi.front}

% section front matter (end)

\input{qm2pi.intro} 
 
% section introduction (end)

% \input{qm2pi.knotations} 

% section notation (end)

\input{qm2pi.process.calculi} 

% section concurrent_process_calculi_and_spatial_logics_ (end)
    
%\input{qm2pi.knots2pi} 

%\input{qm2pi.trefoil} 

%\input{qm2pi.mainthm} 

% subsection basic_interpretation (end)

%\input{qm2pi.rho.presentation} 
\subsection{The syntax and semantics of the notation system}\label{sub:the_syntax_and_semantics_of_the_notation_system} % (fold)

We now summarize a technical presentation of the calculus that
embodies our theory of dynamics. The typical presentation of such a
calculus follows the style of giving generators and relations on
them. The grammar, below, describing term constructors, freely
generates the set of processes, $\Proc$. This set is then quotiented
by a relation known as structural congruence and it is over this set
that the notion of dynamics is expressed. This presentation is
essentially that of \cite{MeredithR05} with the addition of
polyadicity and summation. For readability we have relegated some of
the technical subtleties to an appendix.

\subsubsection{Process grammar}\label{subsub:process_grammar}

\begin{mathpar}
  \inferrule* [lab=synchronization] {} {{M} \bc \pzero \;|\; x?F \;|\; x!C }
  \and
  \inferrule* [lab=abstraction] {} {{F} \bc (x)P}
  \and
  \inferrule* [lab=concretion] {} {{C} \bc \langle Q \rangle}
  \and
  \inferrule* [lab=process] {} {{P,Q} \bc M \;| \;P|Q \;|\; @{x}}
  \and
  \inferrule* [lab=name] {} {{x} \bc \quotep{P}}
\end{mathpar} 

Note that $\vec{x}$ (resp. $\vec{P}$) denotes a vector of names
(resp. processes) of length $|\vec{x}|$ (resp. $|\vec{P}|$). We adopt
the following useful abbreviations.

\begin{mathpar}
   x?(\vec{y}).P := x.(\vec{y})P \and  x\clift{\vec{P}} := x.\clift{\vec{P}}
   \and x!(y) := \lift{x}{\dropn{y}}
   \and \Pi_{i=0}^{n-1}P_i := P_0 | \ldots | P_{n-1}
\end{mathpar}

\subsubsection{Structural congruence}

\paragraph{Free and bound names and alpha-equivalence.} At the
core of structural equivalence is alpha-equivalence which identifies
process that are the same up to a change of variable. Formally, we
recognize the distinction between free and bound names. The free names
of a process, $\freenames{P}$, may be calculated recursively as
follows:

\begin{mathpar}
\freenames{\pzero} := \emptyset
  \and \\
  \freenames{x?(y).P} := \{ x \} \cup (\freenames{P} \setminus \{ y \})
  \and 
  \freenames{x!\langle P \rangle} := \{ x \} \cup \{ P \} 
  \and \\
  \freenames{P|Q} := \freenames{P} \cup \freenames{Q}
  \and \\
  \freenames{@{x}} := \{ x \}
\end{mathpar}

$\pi$
$\quotep{\pi}$

$\freenames{-} : \pi \to \mathcal{P}(\quotep{\pi})$

\begin{eqnarray*}
  \freenames{\pzero} & := & \emptyset \\
  \freenames{x?(y).P} & := & \{ x \} \cup (\freenames{P} \setminus \{ y \}) \\
  \freenames{x!\langle P \rangle} & := & \{ x \} \cup \{ P \} \\
  \freenames{P|Q} & := & \freenames{P} \cup \freenames{Q} \\
  \freenames{\dropn{x}} & := & \{ x \}
\end{eqnarray*}

The bound names of a process, $\boundnames{P}$, are those names occurring in $P$
that are not free. For example, in $x?(y).0$, the name $x$ is free, while $y$ is bound.

\begin{mathpar}
  \inferrule* [lab=monoidal-laws] {} { P|Q \equiv Q|P \and P|0 \equiv P \and P|(Q|R) \equiv (P|Q)|R }
\end{mathpar}

\begin{mathpar}
  \inferrule* [lab=alpha-equivalence] {} { (x)P \equiv (y)P\{y/x\} \and y \not\in \freenames{P} }
\end{mathpar}

\begin{definition}
Then two processes, $P,Q$, are alpha-equivalent if $P = Q\{\vec{y}/\vec{x}\}$ for
some $\vec{x} \in \boundnames{Q},\vec{y} \in \boundnames{P}$, where $Q\{\vec{y}/\vec{x}\}$
denotes the capture-avoiding substitution of $\vec{y}$ for $\vec{x}$ in $Q$.
\end{definition}

\begin{definition}
  The {\em structural congruence} \cite{SangiorgiWalker} , $\equiv$,
  between processes is the least congruence containing
  alpha-equivalence, satisfying the abelian monoid laws
  (associativity, commutativity and $\pzero$ as identity) for parallel
  composition $|$ and for summation $+$.
\end{definition}

\subsection{Name equivalence}

We take name equivalence, written $\nameeq$, to be the smallest
equivalence relation generated by the following rules.

\begin{mathpar}
\inferrule*[lab=Quote-drop]
{ }
{ \quotep{@{x}} \nameeq x }

\inferrule*[lab=Struct-equiv]
{ P \scong Q }
{ \quotep{P} \nameeq \quotep{Q} }
\end{mathpar}

The astute reader will have noticed that the mutual recursion of names
and processes imposes a mutual recursion on alpha-equivalence and
structural equivalence via name-equivalence. Fortunately, all of this
works out pleasantly and we may calculate in the natural way, free of
concern. The reader interested in the details is referred to the
appendix \ref{appendix:rho_details}.

\subsection{Substitution}

We use $\Proc$ for the set of processes, $\QProc$ for the set of
names, and $\id{\{}\vec{y} / \vec{x} \id{\}}$ to denote partial maps,
$s : \QProc \rightarrow \QProc$. A map, $s$ lifts, uniquely, to a map
on process terms, $\widehat{s} : \Proc \rightarrow \Proc$ by the
following equations.

\begin{mathpar}
  (0) \psubstp{Q}{P} := 0 \\
  (R \juxtap S) \psubstp{Q}{P}
  :=    
  (R)\psubstp{Q}{P} \juxtap (S) \psubstp{Q}{P} \\
  (x?(y).R) \psubstp{Q}{P}    
  :=    
  (x)\substp{Q}{P} (z)\concat( (R \psubstn{z}{y}) \psubstp{Q}{P} ) \\
  (\lift{x}{R}) \psubstp{Q}{P}  
  :=
  \lift{(x)\substp{Q}{P}}{ R \psubstp{Q}{P} } \\
%   (\dropn{x})  \psubstp{Q}{P}       
%   := 
%   \left\{ 
%     \begin{array}{ccc} 
%       \dropn{\quotep{Q}} & & x \nameeq \quotep{P} \\
%       \dropn{x} & & otherwise \\
%     \end{array}
%   \right. 
  (\dropn{x})  \psubstp{Q}{P}       
  := 
  \left\{ 
    \begin{array}{ccc} 
      Q & & x \nameeq \quotep{P} \\
      \dropn{x} & & otherwise \\
    \end{array}
  \right.
\end{mathpar}
 

where

\begin{eqnarray}
  (x)\id{\{} \lpquote Q \rpquote / \lpquote P \rpquote \id{\}}            = 
  \left\{ 
    \begin{array}{ccc}
      \lpquote Q \rpquote & & x \nameeq \lpquote P \rpquote \\
      x & & otherwise \\
    \end{array}
  \right. \nonumber
\end{eqnarray}

and $z$ is chosen distinct from $\quotep{P}$, $\quotep{Q}$, the free
names in $Q$, and all the names in $R$. Our $\alpha$-equivalence will
be built in the standard way from this substitution.

\begin{remark}\label{rem:no_self_referential_names}
  One consequence of these definitions is that $\forall P. \quotep{P}
  \not\in \freenames{P}$.
\end{remark}

\subsection{ Dynamic quote: an example }

Anticipating something of what's to come, consider applying the
substitution, $\widehat{\id{\{}u / z \id{\}}}$, to the following pair
of processes, $\lift{w}{y!(z)}$ and $w[ \lpquote y!(z) \rpquote ]$.

\begin{eqnarray}
	\lift{w}{y!(z)}\widehat{\id{\{}u / z \id{\}}}
		& = &
		\lift{w}{y!(u)} \nonumber\\
	w[ \lpquote y!(z) \rpquote ] \widehat{ \id{\{}u / z \id{\}} }
		& = &
		w[ \lpquote y!(z) \rpquote ] \nonumber
\end{eqnarray}

Because the body of the process between quotes is impervious to
substitution, we get radically different answers. In fact, by
examining the first process in an input context,
e.g. $x?(z).\lift{w}{y!(z)}$, we see that the process under the lift
operator may be shaped by prefixed inputs binding a name inside it. In
this sense, the lift operator will be seen as a way to dynamically
construct processes before reifying them as names.

Finally equipped with these standard features we can present the
dynamics of the calculus.

\subsubsection{Operational semantics} 

Finally, we introduce the computational dynamics. What marks these
algebras as distinct from other more traditionally studied algebraic
structures, e.g. vector spaces or polynomial rings, is the manner in
which dynamics is captured. In traditional structures, dynamics is typically
expressed through morphisms between such structures, as in linear maps
between vector spaces or morphisms between rings. In algebras
associated with the semantics of computation, the dynamics is
expressed as part of the algebraic structure itself, through a
reduction reduction relation typically denoted by $\red$. Below, we
give a recursive presentation of this relation for the calculus used
in the encoding.

$\red \subseteq \pi \times \pi$
$\red : \pi \to \mathcal{P}(\pi)$

\begin{mathpar}
  \inferrule* [lab=Comm] { \textsf{match}( x_{src}, x_{trgt} ) } { x_{trgt}?(y)P \; | \; x_{src}!\langle {Q} \rangle \red P\{\quotep{Q}/y}\} }
  \and \\
  \inferrule* [lab=Par] {{P} \red {P}'} {{{P} | {Q}} \red {{P}' | {Q}}}
  \and
  \inferrule* [lab=Equiv]{{{P} \scong {P}'} \andalso {{P}' \red {Q}'} \andalso {{Q}' \scong {Q}}}{{P} \red {Q}}
\end{mathpar}

\begin{eqnarray*}
  match_{\equiv} (\quotep{P},\quotep{Q}) & := & P \equiv Q \\
  match_{\dagger}(\quotep{P},\quotep{Q}) & := & \forall R. P|Q \red^{*} R => R \red^{*} 0 \\
  match_{K}(\quotep{P},\quotep{Q}) & := & K \mbox{ for some context } K
\end{eqnarray*}

$u?(x)P | u!\langle Q \rangle \red P\{\quotep{Q}/x\}$

%We write $\wred$ for $\red^*$, and $P\red$ if $\exists Q $ such that $ P \red Q$.
We write $P\red$ if $\exists Q $ such that $ P \red Q$ and $P\not\red$, otherwise.

\section{Replication}

As mentioned before, it is known that replication (and hence
recursion) can be implemented in a higher-order process algebra
\cite{SangiorgiWalker}. As our first example of calculation with the
machinery thus far presented we give the construction explicitly in
the {\rhoc}.

\begin{eqnarray}
	D_{x} & := & \prefix{x}{y}{(\binpar{\outputp{x}{y}}{@{y}})} \nonumber\\
	\bangp_{x}{P} & := & \binpar{{x}!\langle{\binpar{D_{x}}{P}}\rangle}{D_{x}} \nonumber
\end{eqnarray}

\begin{eqnarray}
	\bangp_{x}{P} & & \nonumber\\
	=
	& {x}!\langle{(\prefix{x}{y}{(\outputp{x}{y} | @{y})) | P}}\rangle 
	      | \prefix{x}{y}{(\outputp{x}{y} | @{y})} & \nonumber\\
	\red
	& (\outputp{x}{y} | @{y})\substn{\quotep{(\prefix{x}{y}{(@{y} | \outputp{x}{y})) | P}}}{y} & \nonumber\\
	=
	& \outputp{x}{\quotep{(\prefix{x}{y}{(\outputp{x}{y} | @{y})) | P}}}
	  | {(\prefix{x}{y}{(\outputp{x}{y} | @{y})) | P}} & \nonumber\\
	\red
	& \ldots & \nonumber\\
	\red^*
	& P | P | \ldots & \nonumber
\end{eqnarray}

Of course, this encoding, as an implementation, runs away, unfolding
$\bangp{P}$ eagerly. A lazier and more implementable replication
operator, restricted to input-guarded processes, may be obtained as follows.

\begin{eqnarray}
\bangp{\prefix{u}{v}{P}} 
	:= 
	\binpar{\lift{x}{\prefix{u}{v}{(\binpar{D(x)}{P})}}}{D(x)} \nonumber
\end{eqnarray}

\begin{remark}
  Note that the lazier definition still does not deal with summation
  or mixed summation (i.e. sums over input and output). The reader is
  invited to construct definitions of replication that deal with these
  features. 

  Further, the definitions are parameterized in a name, $x$. Can you,
  gentle reader, make a definition that eliminates this parameter and
  guarantees no accidental interaction between the replication
  machinery and the process being replicated -- i.e. no accidental
  sharing of names used by the process to get its work done and the
  name(s) used by the replication to effect copying. This latter
  revision of the definition of replication is crucial to obtaining
  the expected identity $!!P \sim !P$.
\end{remark}

\begin{remark}\label{rem:paradoxical_combinator}
  The reader familiar with the lambda calculus will have noticed the
  similarity between $D$ and the paradoxical combinator.

  [Ed. note: the existence of this seems to suggest we have to be more
  restrictive on the set of processes and names we admit if we are to
  support no-cloning.]
\end{remark}

\subsubsection{Bisimulation}

The computational dynamics gives rise to another kind of equivalence,
the equivalence of computational behavior. As previously mentioned
this is typically captured \emph{via} some form of bisimulation.

% The notion we use in this paper is weak barbed bisimulation
% \cite{milner91polyadicpi}.

The notion we use in this paper is derived from weak barbed
bisimulation \cite{milner91polyadicpi}. 

\begin{definition}
An \emph{observation relation}, $\downarrow_{\mathcal N}$, over a set
of names, $\mathcal N$, is the smallest relation satisfying the rules
below.

\infrule[Out-barb]{y \in {\mathcal N}, \; x \nameeq y}
		  {\outputp{x}{v} \downarrow_{\mathcal N} x}
\infrule[Par-barb]{\mbox{$P\downarrow_{\mathcal N} x$ or $Q\downarrow_{\mathcal N} x$}}
		  {\binpar{P}{Q} \downarrow_{\mathcal N} x}

We write $P \Downarrow_{\mathcal N} x$ if there is $Q$ such that 
$P \wred Q$ and $Q \downarrow_{\mathcal N} x$.
\end{definition}

\begin{definition}
%\label{def.bbisim}
An  ${\mathcal N}$-\emph{barbed bisimulation} over a set of names, ${\mathcal N}$, is a symmetric binary relation 
${\mathcal S}_{\mathcal N}$ between agents such that $P\rel{S}_{\mathcal N}Q$ implies:
\begin{enumerate}
\item If $P \red P'$ then $Q \wred Q'$ and $P'\rel{S}_{\mathcal N} Q'$.
\item If $P\downarrow_{\mathcal N} x$, then $Q\Downarrow_{\mathcal N} x$.
\end{enumerate}
$P$ is ${\mathcal N}$-barbed bisimilar to $Q$, written
$P \wbbisim_{\mathcal N} Q$, if $P \rel{S}_{\mathcal N} Q$ for some ${\mathcal N}$-barbed bisimulation ${\mathcal S}_{\mathcal N}$.
\end{definition}

$\mathcal{R} \subseteq \pi \times \pi$

$P \mathcal{R} Q => \forall P'. P \red P' \Rightarrow \exists Q'. Q \red Q', P' \mathcal{R} Q'$

$P \vdash x \Rightarrow Q \vdash x$

\begin{mathpar}
  \inferrule*[lab=Out-barb]{x \nameeq y}{{y}!\langle{Q}\rangle \vdash x}
  \and
  \inferrule*[lab=Par-barb]{\mbox{$P\vdash x$ or $Q\vdash x$}}{\binpar{P}{Q} \vdash x}
\end{mathpar}

\subsubsection{Contexts}

One of the principle advantages of computational calculi like the
$\pi$-calculus is a well-defined notion of context,
contextual-equivalence and a correlation between
contextual-equivalence and notions of bisimulation. The notion of
context allows the decomposition of a process into (sub-)process and
its syntactic environment, its context. Thus, a context may be
thought of as a process with a ``hole'' (written $\Box$) in it. The
application of a context $M$ to a process $P$, written $M[P]$, is
tantamount to filling the hole in $M$ with $P$. In this paper we do
not need the full weight of this theory, but do make use of the notion
of context in the proof the main theorem. 

\begin{mathpar}
  \inferrule* [lab=summation] {} {{M_{M},M_{N}} \bc \Box \;|\; x.M_{A} \;|\; M_{M}+M_{N}}
  \and
  \inferrule* [lab=agent] {} {{M_{A}} \bc (\vec{x})M_{P} \;| \; \clift{P_0,\ldots,M_{P},\ldots,P_N}}
  \and \\
  \inferrule* [lab=process] {} {{M_{P}} \bc M_{N} \;| \;P|M_{P} }
\end{mathpar} 

\begin{mathpar}
  \inferrule* [lab=sychronization] {} {M_{N} \bc \Box \;|\; x?M_{F} \;|\; x!M_{C}}
  \and
  \inferrule* [lab=abstraction] {} {{M_{F}} \bc (x)M_{P} }
  \and
  \inferrule* [lab=concretion] {} {{M_{C}} \bc \langle M_{P} \rangle }
  \and \\
  \inferrule* [lab=process] {} {{M_{P}} \bc M_{N} \;| \;P|M_{P} }
\end{mathpar}

\begin{definition}[contextual application] Given a context $M$, and
  process $P$, we define the \emph{contextual application}, $M[P] :=
  M\{P/\Box\}$. That is, the contextual application of M to P is the
  substitution of $P$ for $\Box$ in $M$.
\end{definition}

$\meaningof{-} : L \to \mathcal{P}(\pi)$

\begin{mathpar}
  \inferrule* [lab=collection] {} {\meaningof{true} = \pi, \and \meaningof{~E} = \pi \setminus \meaningof{E}, \and \meaningof{E_{1} \& E_{2}} = \meaningof{E_{1}} \cap \meaningof{E_{2}}}
\end{mathpar}

\begin{mathpar}
  \inferrule* [lab=structure] {} {\meaningof{0} = \{ P \in \pi | P \equiv 0 \}, \and \\ \meaningof{E_1 | E_2} = \{ P \in \pi | P \equiv P_{1} | P_{2}, P_{1} \in \meaningof{E_{1}}, P_{2} \in \meaningof{E_2}\} }
\end{mathpar}

\begin{mathpar}
 \inferrule* [lab=behavior] {} {\meaningof{\langle a?b \rangle E} = \{ P \in \pi | P \equiv Q | u?(y)P', \\ \and \\\\ \and \\ \;\;\; u \in \meaningof{a}, \forall z.P'\{z/y\} \in \meaningof{E\{z/b\}}\}, \and \\ \meaningof{a!E} = \{ P \in \pi | P \equiv Q | x!\langle P' \rangle, x \in \meaningof{a} P' \in \meaningof{E}\} }
\end{mathpar}

\begin{mathpar}
 \inferrule* [lab=nominal] {} {\meaningof{\quotep{E}} = \{ \quotep{P} \in \quotep{\pi} | P \in \meaningof{E} \}, \and \meaningof{\quotep{P}} = \{ \quotep{Q} \in \quotep{\pi} | P \equiv Q \} \and \\ \meaningof{@\quotep{E}} = \{ P \in \pi | P \equiv @x, x \in \meaningof{E} \}}
\end{mathpar}

\begin{eqnarray*}
  \\
  \meaningof{-} : TS \to ST
\end{eqnarray*}

\begin{eqnarray*}
  \\
  L : TS \to ST
\end{eqnarray*}

\begin{eqnarray*}
  \\
  P \models E \iff P \in \meaningof{E}
\end{eqnarray*}

\begin{eqnarray*}
  P \approx_{L} Q \iff \forall E \in L. P \models E \iff Q \models E
\end{eqnarray*}

\begin{eqnarray*}
  P \approx_{K} Q
\end{eqnarray*}

\begin{eqnarray*}
  P \approx Q
\end{eqnarray*}

$\approx_{K} = \approx = \approx_{L}$

\subsubsection{Contextual duality}

Note that contexts extend the quotation operation to a family of
operations from processes to names. Given a context, $M$, we can
define a \emph{nominal context}, $\quotep{M}$ by $\quotep{M}[P] :=
\quotep{M[P]}$. To foreshadow what is to come we observe that these
operations enjoy a duality with processes very much like the duality
between vectors and maps from vectors to scalars.

Further, because the calculus is essentially higher-order, we have a
correspondence between contexts and processes. More specifically,
given a name $x$ and a context $M$ we can construct $M^{*}_{x}$ such
that 

\begin{mathpar}
  M^{*}_{x} | \lift{x}{P} \red M[P]
\end{mathpar}

namely,

\begin{mathpar}
  M^{*}_{x} := x?(u).M[\dropn{u}]
\end{mathpar}

The dependence of $M^{*}_{x}$ on a name makes it an abstraction, 

\begin{mathpar}
  M^{*} := (x)x?(u).M[\dropn{u}]
\end{mathpar}

\subsection{Additional notation}

It will sometimes be convenient to denote the process a name
quotes. We already have the notation $x = \quotep{P}$, but it will be
convenient to introduce an alternate notation, $\procn{x}$, when we
want to emphasize the connection to the use of the name. Note that, by
virtue of name equivalence, $\quotep{\procn{x}} \nameeq x$; so, the
notation is consistent with previous definitions.

Further, because names have structure it is possible to effect
substitutions on the basis of that structure. This means we need to
upgrade our notation for substitutions, which we accomplish by
adapting comprehension notation. Thus,

\begin{mathpar}
  P\{ y / x : x \in S \}
\end{mathpar}

is interpreted to mean the process derived from P by replacing (in a
capture-avoiding manner) each occurrence of $x$ in $S$ by $y$. For example,

\begin{mathpar}
  P\{ \quotep{\procn{x}|\procn{x}} / x : x \in \freenames{P} \}
\end{mathpar}

will replace each (occurrence) of a free name $x$ in $P$ by
$\quotep{\procn{x}|\procn{x}}$.

Also, we will avail ourselves of the notation $x^{L}$ and $x^{R}$ to
denote injections of a name into disjoint copies of the name
space. There are numerous ways to accomplish this. One example can be
found in \cite{MeredithR05}. This notation overloads to vectors of
names: $\vec{x}^{\pi} := (x_{i}^{\pi} \; : \; 0 \leq i < |\vec{x}| )$ where $\pi \in \{L,R\}$.

We also use $P^{\Box} := P|\Box$.

In \cite{MeredithR05} an interpretation of the new operator is
given. It turns out that there are several possible interpretations
all enjoying the requisite algebraic properties of the operator (see
\cite{milner91polyadicpi}). We will therefore make liberal use of
$(\nu\; \vec{x})P$.

% subsection the_syntax_and_semantics_of_the_notation_system (end)   

\input{qm2pi.qmops} 

\input{qm2pi.sterngerlach} 

\input{qm2pi.metric} 

% section concurrent_process_calculi (end)

%\input{qm2pi.proofsketch}

% section proof sketch (end)

%\input{qm2pi.slviaknots} 

% section spatial logic via knots (end)

\input{qm2pi.conclusion}

% section conclusion (end)

%\input{qm2pi.dtcodes} 

% section wiring algorithm (end)

\input{qm2pi.ack} 

% section acknowledgments (end)

\newpage


\bibliographystyle{plain}   
\bibliography{../../biblios/main.bib}

\input{qm2pi.rhodetails}

\end{document}

 

% section acknowledgments (end)

\newpage


\bibliographystyle{plain}   
\bibliography{../../biblios/main.bib}

\documentclass[12pt]{llncs}
%\documentclass{jktr}

\usepackage[pdftex]{hyperref}                   
\usepackage {listings}
\usepackage {mathpartir}
\usepackage{bcprules}
%\usepackage{listings}
                       
\usepackage{graphicx} 
%\usepackage[margins=2.5cm,nohead,nofoot]{geometry}
%\usepackage{geometry}
\usepackage{amsfonts}
\usepackage{amstext}
\usepackage{latexsym}
\usepackage{amssymb}
\usepackage{color}


%\include{myPreamble}
\include{qm2pi.local} 

%\ifpdf
%\usepackage[pdftex]{graphicx}
%\else
%\usepackage{graphicx}
%\fi

 % \ifpdf
%  \usepackage{pdfsync}
%  \if


%\title{Brief Article}
%\author{David F. Snyder}
%\author{L.G. Meredith}

%\address{Dept. of Math., Texas State University--San Marcos, San Marcos, TX 78666}
       
\pagestyle{empty}


\begin{document}

\lstset{language=[Objective]Caml,frame=shadowbox}

\input{qm2pi.front}

% section front matter (end)

\input{qm2pi.intro} 
 
% section introduction (end)

% \input{qm2pi.knotations} 

% section notation (end)

\input{qm2pi.process.calculi} 

% section concurrent_process_calculi_and_spatial_logics_ (end)
    
%\input{qm2pi.knots2pi} 

%\input{qm2pi.trefoil} 

%\input{qm2pi.mainthm} 

% subsection basic_interpretation (end)

%\input{qm2pi.rho.presentation} 
\subsection{The syntax and semantics of the notation system}\label{sub:the_syntax_and_semantics_of_the_notation_system} % (fold)

We now summarize a technical presentation of the calculus that
embodies our theory of dynamics. The typical presentation of such a
calculus follows the style of giving generators and relations on
them. The grammar, below, describing term constructors, freely
generates the set of processes, $\Proc$. This set is then quotiented
by a relation known as structural congruence and it is over this set
that the notion of dynamics is expressed. This presentation is
essentially that of \cite{MeredithR05} with the addition of
polyadicity and summation. For readability we have relegated some of
the technical subtleties to an appendix.

\subsubsection{Process grammar}\label{subsub:process_grammar}

\begin{mathpar}
  \inferrule* [lab=synchronization] {} {{M} \bc \pzero \;|\; x?F \;|\; x!C }
  \and
  \inferrule* [lab=abstraction] {} {{F} \bc (x)P}
  \and
  \inferrule* [lab=concretion] {} {{C} \bc \langle Q \rangle}
  \and
  \inferrule* [lab=process] {} {{P,Q} \bc M \;| \;P|Q \;|\; @{x}}
  \and
  \inferrule* [lab=name] {} {{x} \bc \quotep{P}}
\end{mathpar} 

Note that $\vec{x}$ (resp. $\vec{P}$) denotes a vector of names
(resp. processes) of length $|\vec{x}|$ (resp. $|\vec{P}|$). We adopt
the following useful abbreviations.

\begin{mathpar}
   x?(\vec{y}).P := x.(\vec{y})P \and  x\clift{\vec{P}} := x.\clift{\vec{P}}
   \and x!(y) := \lift{x}{\dropn{y}}
   \and \Pi_{i=0}^{n-1}P_i := P_0 | \ldots | P_{n-1}
\end{mathpar}

\subsubsection{Structural congruence}

\paragraph{Free and bound names and alpha-equivalence.} At the
core of structural equivalence is alpha-equivalence which identifies
process that are the same up to a change of variable. Formally, we
recognize the distinction between free and bound names. The free names
of a process, $\freenames{P}$, may be calculated recursively as
follows:

\begin{mathpar}
\freenames{\pzero} := \emptyset
  \and \\
  \freenames{x?(y).P} := \{ x \} \cup (\freenames{P} \setminus \{ y \})
  \and 
  \freenames{x!\langle P \rangle} := \{ x \} \cup \{ P \} 
  \and \\
  \freenames{P|Q} := \freenames{P} \cup \freenames{Q}
  \and \\
  \freenames{@{x}} := \{ x \}
\end{mathpar}

$\pi$
$\quotep{\pi}$

$\freenames{-} : \pi \to \mathcal{P}(\quotep{\pi})$

\begin{eqnarray*}
  \freenames{\pzero} & := & \emptyset \\
  \freenames{x?(y).P} & := & \{ x \} \cup (\freenames{P} \setminus \{ y \}) \\
  \freenames{x!\langle P \rangle} & := & \{ x \} \cup \{ P \} \\
  \freenames{P|Q} & := & \freenames{P} \cup \freenames{Q} \\
  \freenames{\dropn{x}} & := & \{ x \}
\end{eqnarray*}

The bound names of a process, $\boundnames{P}$, are those names occurring in $P$
that are not free. For example, in $x?(y).0$, the name $x$ is free, while $y$ is bound.

\begin{mathpar}
  \inferrule* [lab=monoidal-laws] {} { P|Q \equiv Q|P \and P|0 \equiv P \and P|(Q|R) \equiv (P|Q)|R }
\end{mathpar}

\begin{mathpar}
  \inferrule* [lab=alpha-equivalence] {} { (x)P \equiv (y)P\{y/x\} \and y \not\in \freenames{P} }
\end{mathpar}

\begin{definition}
Then two processes, $P,Q$, are alpha-equivalent if $P = Q\{\vec{y}/\vec{x}\}$ for
some $\vec{x} \in \boundnames{Q},\vec{y} \in \boundnames{P}$, where $Q\{\vec{y}/\vec{x}\}$
denotes the capture-avoiding substitution of $\vec{y}$ for $\vec{x}$ in $Q$.
\end{definition}

\begin{definition}
  The {\em structural congruence} \cite{SangiorgiWalker} , $\equiv$,
  between processes is the least congruence containing
  alpha-equivalence, satisfying the abelian monoid laws
  (associativity, commutativity and $\pzero$ as identity) for parallel
  composition $|$ and for summation $+$.
\end{definition}

\subsection{Name equivalence}

We take name equivalence, written $\nameeq$, to be the smallest
equivalence relation generated by the following rules.

\begin{mathpar}
\inferrule*[lab=Quote-drop]
{ }
{ \quotep{@{x}} \nameeq x }

\inferrule*[lab=Struct-equiv]
{ P \scong Q }
{ \quotep{P} \nameeq \quotep{Q} }
\end{mathpar}

The astute reader will have noticed that the mutual recursion of names
and processes imposes a mutual recursion on alpha-equivalence and
structural equivalence via name-equivalence. Fortunately, all of this
works out pleasantly and we may calculate in the natural way, free of
concern. The reader interested in the details is referred to the
appendix \ref{appendix:rho_details}.

\subsection{Substitution}

We use $\Proc$ for the set of processes, $\QProc$ for the set of
names, and $\id{\{}\vec{y} / \vec{x} \id{\}}$ to denote partial maps,
$s : \QProc \rightarrow \QProc$. A map, $s$ lifts, uniquely, to a map
on process terms, $\widehat{s} : \Proc \rightarrow \Proc$ by the
following equations.

\begin{mathpar}
  (0) \psubstp{Q}{P} := 0 \\
  (R \juxtap S) \psubstp{Q}{P}
  :=    
  (R)\psubstp{Q}{P} \juxtap (S) \psubstp{Q}{P} \\
  (x?(y).R) \psubstp{Q}{P}    
  :=    
  (x)\substp{Q}{P} (z)\concat( (R \psubstn{z}{y}) \psubstp{Q}{P} ) \\
  (\lift{x}{R}) \psubstp{Q}{P}  
  :=
  \lift{(x)\substp{Q}{P}}{ R \psubstp{Q}{P} } \\
%   (\dropn{x})  \psubstp{Q}{P}       
%   := 
%   \left\{ 
%     \begin{array}{ccc} 
%       \dropn{\quotep{Q}} & & x \nameeq \quotep{P} \\
%       \dropn{x} & & otherwise \\
%     \end{array}
%   \right. 
  (\dropn{x})  \psubstp{Q}{P}       
  := 
  \left\{ 
    \begin{array}{ccc} 
      Q & & x \nameeq \quotep{P} \\
      \dropn{x} & & otherwise \\
    \end{array}
  \right.
\end{mathpar}
 

where

\begin{eqnarray}
  (x)\id{\{} \lpquote Q \rpquote / \lpquote P \rpquote \id{\}}            = 
  \left\{ 
    \begin{array}{ccc}
      \lpquote Q \rpquote & & x \nameeq \lpquote P \rpquote \\
      x & & otherwise \\
    \end{array}
  \right. \nonumber
\end{eqnarray}

and $z$ is chosen distinct from $\quotep{P}$, $\quotep{Q}$, the free
names in $Q$, and all the names in $R$. Our $\alpha$-equivalence will
be built in the standard way from this substitution.

\begin{remark}\label{rem:no_self_referential_names}
  One consequence of these definitions is that $\forall P. \quotep{P}
  \not\in \freenames{P}$.
\end{remark}

\subsection{ Dynamic quote: an example }

Anticipating something of what's to come, consider applying the
substitution, $\widehat{\id{\{}u / z \id{\}}}$, to the following pair
of processes, $\lift{w}{y!(z)}$ and $w[ \lpquote y!(z) \rpquote ]$.

\begin{eqnarray}
	\lift{w}{y!(z)}\widehat{\id{\{}u / z \id{\}}}
		& = &
		\lift{w}{y!(u)} \nonumber\\
	w[ \lpquote y!(z) \rpquote ] \widehat{ \id{\{}u / z \id{\}} }
		& = &
		w[ \lpquote y!(z) \rpquote ] \nonumber
\end{eqnarray}

Because the body of the process between quotes is impervious to
substitution, we get radically different answers. In fact, by
examining the first process in an input context,
e.g. $x?(z).\lift{w}{y!(z)}$, we see that the process under the lift
operator may be shaped by prefixed inputs binding a name inside it. In
this sense, the lift operator will be seen as a way to dynamically
construct processes before reifying them as names.

Finally equipped with these standard features we can present the
dynamics of the calculus.

\subsubsection{Operational semantics} 

Finally, we introduce the computational dynamics. What marks these
algebras as distinct from other more traditionally studied algebraic
structures, e.g. vector spaces or polynomial rings, is the manner in
which dynamics is captured. In traditional structures, dynamics is typically
expressed through morphisms between such structures, as in linear maps
between vector spaces or morphisms between rings. In algebras
associated with the semantics of computation, the dynamics is
expressed as part of the algebraic structure itself, through a
reduction reduction relation typically denoted by $\red$. Below, we
give a recursive presentation of this relation for the calculus used
in the encoding.

$\red \subseteq \pi \times \pi$
$\red : \pi \to \mathcal{P}(\pi)$

\begin{mathpar}
  \inferrule* [lab=Comm] { \textsf{match}( x_{src}, x_{trgt} ) } { x_{trgt}?(y)P \; | \; x_{src}!\langle {Q} \rangle \red P\{\quotep{Q}/y}\} }
  \and \\
  \inferrule* [lab=Par] {{P} \red {P}'} {{{P} | {Q}} \red {{P}' | {Q}}}
  \and
  \inferrule* [lab=Equiv]{{{P} \scong {P}'} \andalso {{P}' \red {Q}'} \andalso {{Q}' \scong {Q}}}{{P} \red {Q}}
\end{mathpar}

\begin{eqnarray*}
  match_{\equiv} (\quotep{P},\quotep{Q}) & := & P \equiv Q \\
  match_{\dagger}(\quotep{P},\quotep{Q}) & := & \forall R. P|Q \red^{*} R => R \red^{*} 0 \\
  match_{K}(\quotep{P},\quotep{Q}) & := & K \mbox{ for some context } K
\end{eqnarray*}

$u?(x)P | u!\langle Q \rangle \red P\{\quotep{Q}/x\}$

%We write $\wred$ for $\red^*$, and $P\red$ if $\exists Q $ such that $ P \red Q$.
We write $P\red$ if $\exists Q $ such that $ P \red Q$ and $P\not\red$, otherwise.

\section{Replication}

As mentioned before, it is known that replication (and hence
recursion) can be implemented in a higher-order process algebra
\cite{SangiorgiWalker}. As our first example of calculation with the
machinery thus far presented we give the construction explicitly in
the {\rhoc}.

\begin{eqnarray}
	D_{x} & := & \prefix{x}{y}{(\binpar{\outputp{x}{y}}{@{y}})} \nonumber\\
	\bangp_{x}{P} & := & \binpar{{x}!\langle{\binpar{D_{x}}{P}}\rangle}{D_{x}} \nonumber
\end{eqnarray}

\begin{eqnarray}
	\bangp_{x}{P} & & \nonumber\\
	=
	& {x}!\langle{(\prefix{x}{y}{(\outputp{x}{y} | @{y})) | P}}\rangle 
	      | \prefix{x}{y}{(\outputp{x}{y} | @{y})} & \nonumber\\
	\red
	& (\outputp{x}{y} | @{y})\substn{\quotep{(\prefix{x}{y}{(@{y} | \outputp{x}{y})) | P}}}{y} & \nonumber\\
	=
	& \outputp{x}{\quotep{(\prefix{x}{y}{(\outputp{x}{y} | @{y})) | P}}}
	  | {(\prefix{x}{y}{(\outputp{x}{y} | @{y})) | P}} & \nonumber\\
	\red
	& \ldots & \nonumber\\
	\red^*
	& P | P | \ldots & \nonumber
\end{eqnarray}

Of course, this encoding, as an implementation, runs away, unfolding
$\bangp{P}$ eagerly. A lazier and more implementable replication
operator, restricted to input-guarded processes, may be obtained as follows.

\begin{eqnarray}
\bangp{\prefix{u}{v}{P}} 
	:= 
	\binpar{\lift{x}{\prefix{u}{v}{(\binpar{D(x)}{P})}}}{D(x)} \nonumber
\end{eqnarray}

\begin{remark}
  Note that the lazier definition still does not deal with summation
  or mixed summation (i.e. sums over input and output). The reader is
  invited to construct definitions of replication that deal with these
  features. 

  Further, the definitions are parameterized in a name, $x$. Can you,
  gentle reader, make a definition that eliminates this parameter and
  guarantees no accidental interaction between the replication
  machinery and the process being replicated -- i.e. no accidental
  sharing of names used by the process to get its work done and the
  name(s) used by the replication to effect copying. This latter
  revision of the definition of replication is crucial to obtaining
  the expected identity $!!P \sim !P$.
\end{remark}

\begin{remark}\label{rem:paradoxical_combinator}
  The reader familiar with the lambda calculus will have noticed the
  similarity between $D$ and the paradoxical combinator.

  [Ed. note: the existence of this seems to suggest we have to be more
  restrictive on the set of processes and names we admit if we are to
  support no-cloning.]
\end{remark}

\subsubsection{Bisimulation}

The computational dynamics gives rise to another kind of equivalence,
the equivalence of computational behavior. As previously mentioned
this is typically captured \emph{via} some form of bisimulation.

% The notion we use in this paper is weak barbed bisimulation
% \cite{milner91polyadicpi}.

The notion we use in this paper is derived from weak barbed
bisimulation \cite{milner91polyadicpi}. 

\begin{definition}
An \emph{observation relation}, $\downarrow_{\mathcal N}$, over a set
of names, $\mathcal N$, is the smallest relation satisfying the rules
below.

\infrule[Out-barb]{y \in {\mathcal N}, \; x \nameeq y}
		  {\outputp{x}{v} \downarrow_{\mathcal N} x}
\infrule[Par-barb]{\mbox{$P\downarrow_{\mathcal N} x$ or $Q\downarrow_{\mathcal N} x$}}
		  {\binpar{P}{Q} \downarrow_{\mathcal N} x}

We write $P \Downarrow_{\mathcal N} x$ if there is $Q$ such that 
$P \wred Q$ and $Q \downarrow_{\mathcal N} x$.
\end{definition}

\begin{definition}
%\label{def.bbisim}
An  ${\mathcal N}$-\emph{barbed bisimulation} over a set of names, ${\mathcal N}$, is a symmetric binary relation 
${\mathcal S}_{\mathcal N}$ between agents such that $P\rel{S}_{\mathcal N}Q$ implies:
\begin{enumerate}
\item If $P \red P'$ then $Q \wred Q'$ and $P'\rel{S}_{\mathcal N} Q'$.
\item If $P\downarrow_{\mathcal N} x$, then $Q\Downarrow_{\mathcal N} x$.
\end{enumerate}
$P$ is ${\mathcal N}$-barbed bisimilar to $Q$, written
$P \wbbisim_{\mathcal N} Q$, if $P \rel{S}_{\mathcal N} Q$ for some ${\mathcal N}$-barbed bisimulation ${\mathcal S}_{\mathcal N}$.
\end{definition}

$\mathcal{R} \subseteq \pi \times \pi$

$P \mathcal{R} Q => \forall P'. P \red P' \Rightarrow \exists Q'. Q \red Q', P' \mathcal{R} Q'$

$P \vdash x \Rightarrow Q \vdash x$

\begin{mathpar}
  \inferrule*[lab=Out-barb]{x \nameeq y}{{y}!\langle{Q}\rangle \vdash x}
  \and
  \inferrule*[lab=Par-barb]{\mbox{$P\vdash x$ or $Q\vdash x$}}{\binpar{P}{Q} \vdash x}
\end{mathpar}

\subsubsection{Contexts}

One of the principle advantages of computational calculi like the
$\pi$-calculus is a well-defined notion of context,
contextual-equivalence and a correlation between
contextual-equivalence and notions of bisimulation. The notion of
context allows the decomposition of a process into (sub-)process and
its syntactic environment, its context. Thus, a context may be
thought of as a process with a ``hole'' (written $\Box$) in it. The
application of a context $M$ to a process $P$, written $M[P]$, is
tantamount to filling the hole in $M$ with $P$. In this paper we do
not need the full weight of this theory, but do make use of the notion
of context in the proof the main theorem. 

\begin{mathpar}
  \inferrule* [lab=summation] {} {{M_{M},M_{N}} \bc \Box \;|\; x.M_{A} \;|\; M_{M}+M_{N}}
  \and
  \inferrule* [lab=agent] {} {{M_{A}} \bc (\vec{x})M_{P} \;| \; \clift{P_0,\ldots,M_{P},\ldots,P_N}}
  \and \\
  \inferrule* [lab=process] {} {{M_{P}} \bc M_{N} \;| \;P|M_{P} }
\end{mathpar} 

\begin{mathpar}
  \inferrule* [lab=sychronization] {} {M_{N} \bc \Box \;|\; x?M_{F} \;|\; x!M_{C}}
  \and
  \inferrule* [lab=abstraction] {} {{M_{F}} \bc (x)M_{P} }
  \and
  \inferrule* [lab=concretion] {} {{M_{C}} \bc \langle M_{P} \rangle }
  \and \\
  \inferrule* [lab=process] {} {{M_{P}} \bc M_{N} \;| \;P|M_{P} }
\end{mathpar}

\begin{definition}[contextual application] Given a context $M$, and
  process $P$, we define the \emph{contextual application}, $M[P] :=
  M\{P/\Box\}$. That is, the contextual application of M to P is the
  substitution of $P$ for $\Box$ in $M$.
\end{definition}

$\meaningof{-} : L \to \mathcal{P}(\pi)$

\begin{mathpar}
  \inferrule* [lab=collection] {} {\meaningof{true} = \pi, \and \meaningof{~E} = \pi \setminus \meaningof{E}, \and \meaningof{E_{1} \& E_{2}} = \meaningof{E_{1}} \cap \meaningof{E_{2}}}
\end{mathpar}

\begin{mathpar}
  \inferrule* [lab=structure] {} {\meaningof{0} = \{ P \in \pi | P \equiv 0 \}, \and \\ \meaningof{E_1 | E_2} = \{ P \in \pi | P \equiv P_{1} | P_{2}, P_{1} \in \meaningof{E_{1}}, P_{2} \in \meaningof{E_2}\} }
\end{mathpar}

\begin{mathpar}
 \inferrule* [lab=behavior] {} {\meaningof{\langle a?b \rangle E} = \{ P \in \pi | P \equiv Q | u?(y)P', \\ \and \\\\ \and \\ \;\;\; u \in \meaningof{a}, \forall z.P'\{z/y\} \in \meaningof{E\{z/b\}}\}, \and \\ \meaningof{a!E} = \{ P \in \pi | P \equiv Q | x!\langle P' \rangle, x \in \meaningof{a} P' \in \meaningof{E}\} }
\end{mathpar}

\begin{mathpar}
 \inferrule* [lab=nominal] {} {\meaningof{\quotep{E}} = \{ \quotep{P} \in \quotep{\pi} | P \in \meaningof{E} \}, \and \meaningof{\quotep{P}} = \{ \quotep{Q} \in \quotep{\pi} | P \equiv Q \} \and \\ \meaningof{@\quotep{E}} = \{ P \in \pi | P \equiv @x, x \in \meaningof{E} \}}
\end{mathpar}

\begin{eqnarray*}
  \\
  \meaningof{-} : TS \to ST
\end{eqnarray*}

\begin{eqnarray*}
  \\
  L : TS \to ST
\end{eqnarray*}

\begin{eqnarray*}
  \\
  P \models E \iff P \in \meaningof{E}
\end{eqnarray*}

\begin{eqnarray*}
  P \approx_{L} Q \iff \forall E \in L. P \models E \iff Q \models E
\end{eqnarray*}

\begin{eqnarray*}
  P \approx_{K} Q
\end{eqnarray*}

\begin{eqnarray*}
  P \approx Q
\end{eqnarray*}

$\approx_{K} = \approx = \approx_{L}$

\subsubsection{Contextual duality}

Note that contexts extend the quotation operation to a family of
operations from processes to names. Given a context, $M$, we can
define a \emph{nominal context}, $\quotep{M}$ by $\quotep{M}[P] :=
\quotep{M[P]}$. To foreshadow what is to come we observe that these
operations enjoy a duality with processes very much like the duality
between vectors and maps from vectors to scalars.

Further, because the calculus is essentially higher-order, we have a
correspondence between contexts and processes. More specifically,
given a name $x$ and a context $M$ we can construct $M^{*}_{x}$ such
that 

\begin{mathpar}
  M^{*}_{x} | \lift{x}{P} \red M[P]
\end{mathpar}

namely,

\begin{mathpar}
  M^{*}_{x} := x?(u).M[\dropn{u}]
\end{mathpar}

The dependence of $M^{*}_{x}$ on a name makes it an abstraction, 

\begin{mathpar}
  M^{*} := (x)x?(u).M[\dropn{u}]
\end{mathpar}

\subsection{Additional notation}

It will sometimes be convenient to denote the process a name
quotes. We already have the notation $x = \quotep{P}$, but it will be
convenient to introduce an alternate notation, $\procn{x}$, when we
want to emphasize the connection to the use of the name. Note that, by
virtue of name equivalence, $\quotep{\procn{x}} \nameeq x$; so, the
notation is consistent with previous definitions.

Further, because names have structure it is possible to effect
substitutions on the basis of that structure. This means we need to
upgrade our notation for substitutions, which we accomplish by
adapting comprehension notation. Thus,

\begin{mathpar}
  P\{ y / x : x \in S \}
\end{mathpar}

is interpreted to mean the process derived from P by replacing (in a
capture-avoiding manner) each occurrence of $x$ in $S$ by $y$. For example,

\begin{mathpar}
  P\{ \quotep{\procn{x}|\procn{x}} / x : x \in \freenames{P} \}
\end{mathpar}

will replace each (occurrence) of a free name $x$ in $P$ by
$\quotep{\procn{x}|\procn{x}}$.

Also, we will avail ourselves of the notation $x^{L}$ and $x^{R}$ to
denote injections of a name into disjoint copies of the name
space. There are numerous ways to accomplish this. One example can be
found in \cite{MeredithR05}. This notation overloads to vectors of
names: $\vec{x}^{\pi} := (x_{i}^{\pi} \; : \; 0 \leq i < |\vec{x}| )$ where $\pi \in \{L,R\}$.

We also use $P^{\Box} := P|\Box$.

In \cite{MeredithR05} an interpretation of the new operator is
given. It turns out that there are several possible interpretations
all enjoying the requisite algebraic properties of the operator (see
\cite{milner91polyadicpi}). We will therefore make liberal use of
$(\nu\; \vec{x})P$.

% subsection the_syntax_and_semantics_of_the_notation_system (end)   

\input{qm2pi.qmops} 

\input{qm2pi.sterngerlach} 

\input{qm2pi.metric} 

% section concurrent_process_calculi (end)

%\input{qm2pi.proofsketch}

% section proof sketch (end)

%\input{qm2pi.slviaknots} 

% section spatial logic via knots (end)

\input{qm2pi.conclusion}

% section conclusion (end)

%\input{qm2pi.dtcodes} 

% section wiring algorithm (end)

\input{qm2pi.ack} 

% section acknowledgments (end)

\newpage


\bibliographystyle{plain}   
\bibliography{../../biblios/main.bib}

\input{qm2pi.rhodetails}

\end{document}



\end{document}



\end{document}



% section proof sketch (end)

%\section{Unlikely characters: spatial logic for
  knots}\label{sub:characteristic_formulae} % (fold)

Associated to the mobile process calculi are a family of logics known
as the Hennessy-Milner logics. These logics typically enjoy a
semantics interpreting formulae as sets of processes that when
factored through the encoding outlined above allows an identification
of classes of knots with logical formulae. In the context of this
encoding the sub-family known as the spatial logics \cite{CairesC03}
\cite{CairesC04} \cite{Caires04} are of particular interest providing
several important features for expressing and reasoning about
properties (i.e. classes) of knots. We hint here at how this may be done.

%\begin{description}
%\item [structural connectives] 
\subsubsection{Structural connectives} The spatial logics enjoy
structural connectives corresponding, at the logical level, to the
parallel composition ($P | Q$) and new name ($(\nu \; x)P$)
connectives for processes. As illustrated in the examples below, these
connectives are extremely expressive given the shape of our encoding.
%\item [decideable satisfaction]

\subsubsection{Decideable satisfaction}
In \cite{Caires04} the satisfaction relation is shown to be decideable
for a rich class of processes. It further turns out that the image of
the our encoding is a proper subset of that class. This result
provides the basis for an algorithm by which to search for knots
enjoying a given property.
%\item [characteristic formulae]

\subsubsection{Characteristic formulae}
In the same paper \cite{Caires04} , Caires presents a means of calculating
characteristic formulae, selecting equivalence classes of processes
up to a pre--specified depth limit on the support set of names. Composed with our
encoding, this characteristic formula can be used to select
characteristic formulae for knots.
%\end{description}

\subsubsection{Spatial logic formulae}

The grammar below (segmented for comprehension) summarizes the syntax
of spatial logic formulae. We employ illustrative examples in the
sequel to provide an intuitive understanding of their meaning
referring the reader to \cite{Caires04} for a more detailed explication
of the semantics.

\begin{mathpar}
  \inferrule* [lab=boolean] {} {{A,B} \bc T \;|\; \neg A \;|\; A \wedge B \;|\; \eta = \eta'}
  \and
  \inferrule* [lab=spatial] {} {|\; \pzero \;|\; A | B \;|\; x \text{\textregistered} A \;|\; \forall x . A \;|\;  H x . A}
  \and
  \inferrule* [lab=behavioral] {} {|\; \alpha . A}
  \and 
  \inferrule* [lab=recursion] {} {|\; X(\vec{u}) \;|\; \mu X(\vec{u}) . A}
  \and
  \inferrule* [lab=action] {} {\alpha \bc \langle x?(\vec{y}) \rangle \;|\; \langle x!(\vec{y}) \rangle \;|\; \langle \tau \rangle}
  \and 
  \inferrule* [lab=name] {} {\eta \bc x \;|\; \tau}
\end{mathpar} 

% subsection characteristic_formulae (end)   	 

\subsection{Example formulae}\label{sub:example_formulae_} % (fold)

\subsubsection{Crossing as formula.}
% 
% \begin{align*}
%   \frac{d}{dx} \sin x &= \cos x 
%   & \frac{d}{dx} e^x &= e^x \\
%   \frac{d}{dx} \cos x &= - \sin x 
%   & \frac{d}{dx} \log x &= \frac{1}{x} \\
% \end{align*} 

\begin{align*}
 \mu C(x_{0},x_{1},y_{0},y_{1},u).&(\langle x_{0}?(z) \rangle(\langle u! \rangle\langle y_{1}!z \rangle C(x_{0},x_{1},y_{0},y_{1},u)) & \\
  & \wedge \langle y_{1}?(z) \rangle (\langle u! \rangle \langle x_{0}!z \rangle C(x_{0},x_{1},y_{0},y_{1},u)) & \\
  & \wedge \langle x_{1}?(z) \rangle (\langle u? \rangle \langle y_{0}!z \rangle C(x_{0},x_{1},y_{0},y_{1},u)) & \\
  & \wedge \langle y_{0}?(z) \rangle (\langle u? \rangle \langle x_{1}!z \rangle C(x_{0},x_{1},y_{0},y_{1},u))) &
\end{align*}

The lexicographical similarity between the shape of this formulae and
the shape of definition of the process representing a crossing reveals
the intuitive meaning of this formulae. It describes the capabilities
of a process that has the right to represent a crossing. For example
it picks out processes that may perform an input on the port $x_0$ in
its initial menu of capabilities. What differentiates the formula
from the process, however, is that the crossing process is the
smallest candidate to satisfy the formula. Infinitely many other
processes -- with internal behavior hidden behind this interface, so
to speak -- also satisfy this formula. Even this simple formula,
then, can be seen to open a new view onto knots, providing a
computational interpretation of \emph{virtual} knots.

Note that this formula is derived by hand. A similar formula can be
derived by employing Caires' calculation of characteristic formula
\cite{Caires04} to the process representing a crossing. In light of
this discussion, we let
$\meaningof{C}_{\phi}(x0,x1,y0,y1,u)$ denote a formula specifying the
dynamics we wish to capture of a crossing. To guarantee we preserve
the shape of the interface and minimal semantics we demand that
$\meaningof{C}_{\phi}(x0,x1,y0,y1,u) \Rightarrow
\textbf{C}(x0,x1,y0,y1,u)$ where $\textbf{C}(x0,x1,y0,y1,u)$ denotes
the formula above.
                            
\subsubsection{Crossing number constraints.}
The moral content of the context lemma (Lemma \ref{context}) is that the notion of
``locality'' in the Reidemeister moves is effectively captured by the
parallel composition operator of the process calculus. This intuition
extends through the logic. Given a formula,
$\meaningof{C}_{\phi}(x0,x1,y0,y1,u)$, we can use the structural
connectives to specify constraints on crossing numbers, such as at
least $n$ crossings, or exactly $n$ crossings.
\begin{mathpar}
  \inferrule* [lab=at-least-n] {} { K^{\geq n}_{\phi}(\vec{xs},\vec{ys}) := \Pi_{i=0}^{n-1} Hu . \meaningof{C}_{\phi}(xs_i,ys_i,u) | T }
  \and 
  \inferrule* [lab=exactly-n] {} { K^{= n}_{\phi}(\vec{xs},\vec{ys}) := \Pi_{i=0}^{n-1} Hu . \meaningof{C}_{\phi}(xs_i,ys_i,u) | \neg (\forall x_0,y_0,x_1,y_1,u . \meaningof{C}_{\phi}(x_0,y_0,x_1,y_1,u) | T) }
\end{mathpar}

To round out this section, recall that the encoding of an $n$-crossing
knot decomposes into a parallel composition of $n$ \emph{copies} of a
crossing process together with a wiring harness. To specify different
knot classes with the same crossing number amounts to specifying
logical constraints on the wiring harness. In the interest of space,
we defer examples to a forthcoming paper. Suffice it to say that both
the conditions ``alternating knot'' and ``contains the tangle
corresponding to 5/3'' are expressible. For example, it is possible to
calculate the characteristic formula of a process corresponding to the
tangle 5/3 and conjoin it into the classifying formula via the
composition connective of the logic.

Finally, we wish to observe that it is entirely within reason to
contemplate a more domain-specific version of spatial logic tailored
to the shape of processes in the image of the encoding. Such a
domain-specific logic would have a better claim to the title formal
language of knot properties.

% subsection example_formulae_ (end)

% section knots_as_processes (end) 

% section spatial logic via knots (end)

\section{Conclusions and future work}

\paragraph{Testing physical space}
You, gentle reader, may wonder why of all the theorems to be proved
given this set up we pick the one above. In some sense it's hardly
central to quantum mechanics. We see it as central in the sense that
it firmly establishes a notion of physical space arising from a notion
of the equivalence of behavior. Relating bisimulation to a metric is a
big step forward, but one is faced with interpreting the relationship
of that metric space to something more physical. Quantum mechanical
notions of ``physical'' space are still far from intuitive, but by
relating this idea of distance as testing to calculations that predict
physical circumstances we are making a not insignificant step forward
toward an understanding of the physical space we inhabit as
essentially dynamic.

\paragraph{Effectivity and simulation}
One of the observations we have yet to make is that the entire program
spelled out here is effective. We have built various interpreters for
the reflective calculus at work in this interpretation. In principle,
then, we can simulate quantum mechanics on a computer. The place where
the simulation may lose fidelity is the infinitely branching summation
for the annihilator.

In this connection i also want to point out that the evaluation style
calculation of the inner product puts the non-determinism of the
summation right at the heart of measurement. This suggests that
Milner's original reduction-based formulation of the dynamics of his
calculi in terms of sums was not just notationally suggestive of a
notion of measure-and-continue but captured some significant part of
the physics.

\paragraph{Quantum continuations}
In light of this last observation i want to point out that the
predominant account of quantum mechanics is missing a key aspect of a
truly compositional story of the physical situation. In a real lab,
when a measurement is made the observation can be made to feed into
another device that then makes another measurement conditioned on the
results of the first. This means that after the superposition was
collapsed the entire experimental set up remained in
superposition. While QM offers a means of writing this down it doesn't
quite line up well with the well-trodden formulation of computation
and continuation that we see so succinctly expressed in Milner's
calculi. This suggests that there might be advantages to this account
of dynamics waiting to be explored.

\paragraph{Quantum logic}
In this connection, we also note that by virtue of having the
Hennessy-Milner construction, we can pull the construction through the
interpretation of QM. This gives us a natural candidate for a quantum
logic that enjoys an extremely tight connection with it's domain of
interpretation, making the construction much less ad hoc (rather it is
the image of functor!).

\paragraph{Quantum probabiity}
i have questions about the basis of the interpretation of inner
product as probability amplitude. In particular, using which
axiomatization of probability theory does the notion of probability
amplitude earn the right to be so dubbed? In other words, where is the
proof that the operation for calculating a probability amplitude (and
then squaring) satisfies the axioms of what it means to calculate a
probability? Even if such a proof exists (i have yet to find it in the
literature), i wonder if it might not be possible to turn things on
their heads. Can we view the calculation of the probability amplitude
as an axiomatization of probability? If so, then the definition we
give for calculating probability amplitude may provide the basis for
an \emph{effective} theory of probability.

\paragraph{Quantum vs ``biological'' information}
Finally, i want to conclude with a more philosophical observation. At
a recent workshop in which QM was a predominant topic i noticed
something about quantum information. The speaker was giving a riveting
discussion of axiomatic QM and showing how properties of ``no
cloning'' and ``no deleting'' emerged as consequences of the
axiomatization. Theorems of this form are necessary to give us a sense
of confidence that our axioms characterize the physical theory. What
struck me, though, was that if quantum information is neither erasable
nor replicable it is markedly different from \emph{life}. Two of the
things we know about life is that

\begin{itemize}
  \item it ends;
  \item to gain some measure of persistence, to transcend it's
    finitude it is imminently copyable.
\end{itemize}

Both of these qualities are summarized succinctly in the aphorism: all
flesh is grass. For me these two kinds of ``information'' -- call them
quantum and biological -- are end points on a spectrum of strategies
for persistence. At one end, we have those curious entities that enjoy
uniqueness and permanence; at the other, we have those who in the face
of a certain end and an uncertain present make a go of passing
something on. To me one of the more remarkable aspects of the latter
strategy is that in the presence of noise (and certain features of
copying) we get a kind of dynamism, a chance for improvement against a
given persistent condition.

% subsection other_calculi_other_bisimulations_and_geometry_as_behavior (end)




% section conclusion (end)

%\documentclass[12pt]{llncs}
%\documentclass{jktr}

\usepackage[pdftex]{hyperref}                   
\usepackage {listings}
\usepackage {mathpartir}
\usepackage{bcprules}
%\usepackage{listings}
                       
\usepackage{graphicx} 
%\usepackage[margins=2.5cm,nohead,nofoot]{geometry}
%\usepackage{geometry}
\usepackage{amsfonts}
\usepackage{amstext}
\usepackage{latexsym}
\usepackage{amssymb}
\usepackage{color}


%\include{myPreamble}
\documentclass[12pt]{llncs}
%\documentclass{jktr}

\usepackage[pdftex]{hyperref}                   
\usepackage {listings}
\usepackage {mathpartir}
\usepackage{bcprules}
%\usepackage{listings}
                       
\usepackage{graphicx} 
%\usepackage[margins=2.5cm,nohead,nofoot]{geometry}
%\usepackage{geometry}
\usepackage{amsfonts}
\usepackage{amstext}
\usepackage{latexsym}
\usepackage{amssymb}
\usepackage{color}


%\include{myPreamble}
\documentclass[12pt]{llncs}
%\documentclass{jktr}

\usepackage[pdftex]{hyperref}                   
\usepackage {listings}
\usepackage {mathpartir}
\usepackage{bcprules}
%\usepackage{listings}
                       
\usepackage{graphicx} 
%\usepackage[margins=2.5cm,nohead,nofoot]{geometry}
%\usepackage{geometry}
\usepackage{amsfonts}
\usepackage{amstext}
\usepackage{latexsym}
\usepackage{amssymb}
\usepackage{color}


%\include{myPreamble}
\include{qm2pi.local} 

%\ifpdf
%\usepackage[pdftex]{graphicx}
%\else
%\usepackage{graphicx}
%\fi

 % \ifpdf
%  \usepackage{pdfsync}
%  \if


%\title{Brief Article}
%\author{David F. Snyder}
%\author{L.G. Meredith}

%\address{Dept. of Math., Texas State University--San Marcos, San Marcos, TX 78666}
       
\pagestyle{empty}


\begin{document}

\lstset{language=[Objective]Caml,frame=shadowbox}

\input{qm2pi.front}

% section front matter (end)

\input{qm2pi.intro} 
 
% section introduction (end)

% \input{qm2pi.knotations} 

% section notation (end)

\input{qm2pi.process.calculi} 

% section concurrent_process_calculi_and_spatial_logics_ (end)
    
%\input{qm2pi.knots2pi} 

%\input{qm2pi.trefoil} 

%\input{qm2pi.mainthm} 

% subsection basic_interpretation (end)

%\input{qm2pi.rho.presentation} 
\subsection{The syntax and semantics of the notation system}\label{sub:the_syntax_and_semantics_of_the_notation_system} % (fold)

We now summarize a technical presentation of the calculus that
embodies our theory of dynamics. The typical presentation of such a
calculus follows the style of giving generators and relations on
them. The grammar, below, describing term constructors, freely
generates the set of processes, $\Proc$. This set is then quotiented
by a relation known as structural congruence and it is over this set
that the notion of dynamics is expressed. This presentation is
essentially that of \cite{MeredithR05} with the addition of
polyadicity and summation. For readability we have relegated some of
the technical subtleties to an appendix.

\subsubsection{Process grammar}\label{subsub:process_grammar}

\begin{mathpar}
  \inferrule* [lab=synchronization] {} {{M} \bc \pzero \;|\; x?F \;|\; x!C }
  \and
  \inferrule* [lab=abstraction] {} {{F} \bc (x)P}
  \and
  \inferrule* [lab=concretion] {} {{C} \bc \langle Q \rangle}
  \and
  \inferrule* [lab=process] {} {{P,Q} \bc M \;| \;P|Q \;|\; @{x}}
  \and
  \inferrule* [lab=name] {} {{x} \bc \quotep{P}}
\end{mathpar} 

Note that $\vec{x}$ (resp. $\vec{P}$) denotes a vector of names
(resp. processes) of length $|\vec{x}|$ (resp. $|\vec{P}|$). We adopt
the following useful abbreviations.

\begin{mathpar}
   x?(\vec{y}).P := x.(\vec{y})P \and  x\clift{\vec{P}} := x.\clift{\vec{P}}
   \and x!(y) := \lift{x}{\dropn{y}}
   \and \Pi_{i=0}^{n-1}P_i := P_0 | \ldots | P_{n-1}
\end{mathpar}

\subsubsection{Structural congruence}

\paragraph{Free and bound names and alpha-equivalence.} At the
core of structural equivalence is alpha-equivalence which identifies
process that are the same up to a change of variable. Formally, we
recognize the distinction between free and bound names. The free names
of a process, $\freenames{P}$, may be calculated recursively as
follows:

\begin{mathpar}
\freenames{\pzero} := \emptyset
  \and \\
  \freenames{x?(y).P} := \{ x \} \cup (\freenames{P} \setminus \{ y \})
  \and 
  \freenames{x!\langle P \rangle} := \{ x \} \cup \{ P \} 
  \and \\
  \freenames{P|Q} := \freenames{P} \cup \freenames{Q}
  \and \\
  \freenames{@{x}} := \{ x \}
\end{mathpar}

$\pi$
$\quotep{\pi}$

$\freenames{-} : \pi \to \mathcal{P}(\quotep{\pi})$

\begin{eqnarray*}
  \freenames{\pzero} & := & \emptyset \\
  \freenames{x?(y).P} & := & \{ x \} \cup (\freenames{P} \setminus \{ y \}) \\
  \freenames{x!\langle P \rangle} & := & \{ x \} \cup \{ P \} \\
  \freenames{P|Q} & := & \freenames{P} \cup \freenames{Q} \\
  \freenames{\dropn{x}} & := & \{ x \}
\end{eqnarray*}

The bound names of a process, $\boundnames{P}$, are those names occurring in $P$
that are not free. For example, in $x?(y).0$, the name $x$ is free, while $y$ is bound.

\begin{mathpar}
  \inferrule* [lab=monoidal-laws] {} { P|Q \equiv Q|P \and P|0 \equiv P \and P|(Q|R) \equiv (P|Q)|R }
\end{mathpar}

\begin{mathpar}
  \inferrule* [lab=alpha-equivalence] {} { (x)P \equiv (y)P\{y/x\} \and y \not\in \freenames{P} }
\end{mathpar}

\begin{definition}
Then two processes, $P,Q$, are alpha-equivalent if $P = Q\{\vec{y}/\vec{x}\}$ for
some $\vec{x} \in \boundnames{Q},\vec{y} \in \boundnames{P}$, where $Q\{\vec{y}/\vec{x}\}$
denotes the capture-avoiding substitution of $\vec{y}$ for $\vec{x}$ in $Q$.
\end{definition}

\begin{definition}
  The {\em structural congruence} \cite{SangiorgiWalker} , $\equiv$,
  between processes is the least congruence containing
  alpha-equivalence, satisfying the abelian monoid laws
  (associativity, commutativity and $\pzero$ as identity) for parallel
  composition $|$ and for summation $+$.
\end{definition}

\subsection{Name equivalence}

We take name equivalence, written $\nameeq$, to be the smallest
equivalence relation generated by the following rules.

\begin{mathpar}
\inferrule*[lab=Quote-drop]
{ }
{ \quotep{@{x}} \nameeq x }

\inferrule*[lab=Struct-equiv]
{ P \scong Q }
{ \quotep{P} \nameeq \quotep{Q} }
\end{mathpar}

The astute reader will have noticed that the mutual recursion of names
and processes imposes a mutual recursion on alpha-equivalence and
structural equivalence via name-equivalence. Fortunately, all of this
works out pleasantly and we may calculate in the natural way, free of
concern. The reader interested in the details is referred to the
appendix \ref{appendix:rho_details}.

\subsection{Substitution}

We use $\Proc$ for the set of processes, $\QProc$ for the set of
names, and $\id{\{}\vec{y} / \vec{x} \id{\}}$ to denote partial maps,
$s : \QProc \rightarrow \QProc$. A map, $s$ lifts, uniquely, to a map
on process terms, $\widehat{s} : \Proc \rightarrow \Proc$ by the
following equations.

\begin{mathpar}
  (0) \psubstp{Q}{P} := 0 \\
  (R \juxtap S) \psubstp{Q}{P}
  :=    
  (R)\psubstp{Q}{P} \juxtap (S) \psubstp{Q}{P} \\
  (x?(y).R) \psubstp{Q}{P}    
  :=    
  (x)\substp{Q}{P} (z)\concat( (R \psubstn{z}{y}) \psubstp{Q}{P} ) \\
  (\lift{x}{R}) \psubstp{Q}{P}  
  :=
  \lift{(x)\substp{Q}{P}}{ R \psubstp{Q}{P} } \\
%   (\dropn{x})  \psubstp{Q}{P}       
%   := 
%   \left\{ 
%     \begin{array}{ccc} 
%       \dropn{\quotep{Q}} & & x \nameeq \quotep{P} \\
%       \dropn{x} & & otherwise \\
%     \end{array}
%   \right. 
  (\dropn{x})  \psubstp{Q}{P}       
  := 
  \left\{ 
    \begin{array}{ccc} 
      Q & & x \nameeq \quotep{P} \\
      \dropn{x} & & otherwise \\
    \end{array}
  \right.
\end{mathpar}
 

where

\begin{eqnarray}
  (x)\id{\{} \lpquote Q \rpquote / \lpquote P \rpquote \id{\}}            = 
  \left\{ 
    \begin{array}{ccc}
      \lpquote Q \rpquote & & x \nameeq \lpquote P \rpquote \\
      x & & otherwise \\
    \end{array}
  \right. \nonumber
\end{eqnarray}

and $z$ is chosen distinct from $\quotep{P}$, $\quotep{Q}$, the free
names in $Q$, and all the names in $R$. Our $\alpha$-equivalence will
be built in the standard way from this substitution.

\begin{remark}\label{rem:no_self_referential_names}
  One consequence of these definitions is that $\forall P. \quotep{P}
  \not\in \freenames{P}$.
\end{remark}

\subsection{ Dynamic quote: an example }

Anticipating something of what's to come, consider applying the
substitution, $\widehat{\id{\{}u / z \id{\}}}$, to the following pair
of processes, $\lift{w}{y!(z)}$ and $w[ \lpquote y!(z) \rpquote ]$.

\begin{eqnarray}
	\lift{w}{y!(z)}\widehat{\id{\{}u / z \id{\}}}
		& = &
		\lift{w}{y!(u)} \nonumber\\
	w[ \lpquote y!(z) \rpquote ] \widehat{ \id{\{}u / z \id{\}} }
		& = &
		w[ \lpquote y!(z) \rpquote ] \nonumber
\end{eqnarray}

Because the body of the process between quotes is impervious to
substitution, we get radically different answers. In fact, by
examining the first process in an input context,
e.g. $x?(z).\lift{w}{y!(z)}$, we see that the process under the lift
operator may be shaped by prefixed inputs binding a name inside it. In
this sense, the lift operator will be seen as a way to dynamically
construct processes before reifying them as names.

Finally equipped with these standard features we can present the
dynamics of the calculus.

\subsubsection{Operational semantics} 

Finally, we introduce the computational dynamics. What marks these
algebras as distinct from other more traditionally studied algebraic
structures, e.g. vector spaces or polynomial rings, is the manner in
which dynamics is captured. In traditional structures, dynamics is typically
expressed through morphisms between such structures, as in linear maps
between vector spaces or morphisms between rings. In algebras
associated with the semantics of computation, the dynamics is
expressed as part of the algebraic structure itself, through a
reduction reduction relation typically denoted by $\red$. Below, we
give a recursive presentation of this relation for the calculus used
in the encoding.

$\red \subseteq \pi \times \pi$
$\red : \pi \to \mathcal{P}(\pi)$

\begin{mathpar}
  \inferrule* [lab=Comm] { \textsf{match}( x_{src}, x_{trgt} ) } { x_{trgt}?(y)P \; | \; x_{src}!\langle {Q} \rangle \red P\{\quotep{Q}/y}\} }
  \and \\
  \inferrule* [lab=Par] {{P} \red {P}'} {{{P} | {Q}} \red {{P}' | {Q}}}
  \and
  \inferrule* [lab=Equiv]{{{P} \scong {P}'} \andalso {{P}' \red {Q}'} \andalso {{Q}' \scong {Q}}}{{P} \red {Q}}
\end{mathpar}

\begin{eqnarray*}
  match_{\equiv} (\quotep{P},\quotep{Q}) & := & P \equiv Q \\
  match_{\dagger}(\quotep{P},\quotep{Q}) & := & \forall R. P|Q \red^{*} R => R \red^{*} 0 \\
  match_{K}(\quotep{P},\quotep{Q}) & := & K \mbox{ for some context } K
\end{eqnarray*}

$u?(x)P | u!\langle Q \rangle \red P\{\quotep{Q}/x\}$

%We write $\wred$ for $\red^*$, and $P\red$ if $\exists Q $ such that $ P \red Q$.
We write $P\red$ if $\exists Q $ such that $ P \red Q$ and $P\not\red$, otherwise.

\section{Replication}

As mentioned before, it is known that replication (and hence
recursion) can be implemented in a higher-order process algebra
\cite{SangiorgiWalker}. As our first example of calculation with the
machinery thus far presented we give the construction explicitly in
the {\rhoc}.

\begin{eqnarray}
	D_{x} & := & \prefix{x}{y}{(\binpar{\outputp{x}{y}}{@{y}})} \nonumber\\
	\bangp_{x}{P} & := & \binpar{{x}!\langle{\binpar{D_{x}}{P}}\rangle}{D_{x}} \nonumber
\end{eqnarray}

\begin{eqnarray}
	\bangp_{x}{P} & & \nonumber\\
	=
	& {x}!\langle{(\prefix{x}{y}{(\outputp{x}{y} | @{y})) | P}}\rangle 
	      | \prefix{x}{y}{(\outputp{x}{y} | @{y})} & \nonumber\\
	\red
	& (\outputp{x}{y} | @{y})\substn{\quotep{(\prefix{x}{y}{(@{y} | \outputp{x}{y})) | P}}}{y} & \nonumber\\
	=
	& \outputp{x}{\quotep{(\prefix{x}{y}{(\outputp{x}{y} | @{y})) | P}}}
	  | {(\prefix{x}{y}{(\outputp{x}{y} | @{y})) | P}} & \nonumber\\
	\red
	& \ldots & \nonumber\\
	\red^*
	& P | P | \ldots & \nonumber
\end{eqnarray}

Of course, this encoding, as an implementation, runs away, unfolding
$\bangp{P}$ eagerly. A lazier and more implementable replication
operator, restricted to input-guarded processes, may be obtained as follows.

\begin{eqnarray}
\bangp{\prefix{u}{v}{P}} 
	:= 
	\binpar{\lift{x}{\prefix{u}{v}{(\binpar{D(x)}{P})}}}{D(x)} \nonumber
\end{eqnarray}

\begin{remark}
  Note that the lazier definition still does not deal with summation
  or mixed summation (i.e. sums over input and output). The reader is
  invited to construct definitions of replication that deal with these
  features. 

  Further, the definitions are parameterized in a name, $x$. Can you,
  gentle reader, make a definition that eliminates this parameter and
  guarantees no accidental interaction between the replication
  machinery and the process being replicated -- i.e. no accidental
  sharing of names used by the process to get its work done and the
  name(s) used by the replication to effect copying. This latter
  revision of the definition of replication is crucial to obtaining
  the expected identity $!!P \sim !P$.
\end{remark}

\begin{remark}\label{rem:paradoxical_combinator}
  The reader familiar with the lambda calculus will have noticed the
  similarity between $D$ and the paradoxical combinator.

  [Ed. note: the existence of this seems to suggest we have to be more
  restrictive on the set of processes and names we admit if we are to
  support no-cloning.]
\end{remark}

\subsubsection{Bisimulation}

The computational dynamics gives rise to another kind of equivalence,
the equivalence of computational behavior. As previously mentioned
this is typically captured \emph{via} some form of bisimulation.

% The notion we use in this paper is weak barbed bisimulation
% \cite{milner91polyadicpi}.

The notion we use in this paper is derived from weak barbed
bisimulation \cite{milner91polyadicpi}. 

\begin{definition}
An \emph{observation relation}, $\downarrow_{\mathcal N}$, over a set
of names, $\mathcal N$, is the smallest relation satisfying the rules
below.

\infrule[Out-barb]{y \in {\mathcal N}, \; x \nameeq y}
		  {\outputp{x}{v} \downarrow_{\mathcal N} x}
\infrule[Par-barb]{\mbox{$P\downarrow_{\mathcal N} x$ or $Q\downarrow_{\mathcal N} x$}}
		  {\binpar{P}{Q} \downarrow_{\mathcal N} x}

We write $P \Downarrow_{\mathcal N} x$ if there is $Q$ such that 
$P \wred Q$ and $Q \downarrow_{\mathcal N} x$.
\end{definition}

\begin{definition}
%\label{def.bbisim}
An  ${\mathcal N}$-\emph{barbed bisimulation} over a set of names, ${\mathcal N}$, is a symmetric binary relation 
${\mathcal S}_{\mathcal N}$ between agents such that $P\rel{S}_{\mathcal N}Q$ implies:
\begin{enumerate}
\item If $P \red P'$ then $Q \wred Q'$ and $P'\rel{S}_{\mathcal N} Q'$.
\item If $P\downarrow_{\mathcal N} x$, then $Q\Downarrow_{\mathcal N} x$.
\end{enumerate}
$P$ is ${\mathcal N}$-barbed bisimilar to $Q$, written
$P \wbbisim_{\mathcal N} Q$, if $P \rel{S}_{\mathcal N} Q$ for some ${\mathcal N}$-barbed bisimulation ${\mathcal S}_{\mathcal N}$.
\end{definition}

$\mathcal{R} \subseteq \pi \times \pi$

$P \mathcal{R} Q => \forall P'. P \red P' \Rightarrow \exists Q'. Q \red Q', P' \mathcal{R} Q'$

$P \vdash x \Rightarrow Q \vdash x$

\begin{mathpar}
  \inferrule*[lab=Out-barb]{x \nameeq y}{{y}!\langle{Q}\rangle \vdash x}
  \and
  \inferrule*[lab=Par-barb]{\mbox{$P\vdash x$ or $Q\vdash x$}}{\binpar{P}{Q} \vdash x}
\end{mathpar}

\subsubsection{Contexts}

One of the principle advantages of computational calculi like the
$\pi$-calculus is a well-defined notion of context,
contextual-equivalence and a correlation between
contextual-equivalence and notions of bisimulation. The notion of
context allows the decomposition of a process into (sub-)process and
its syntactic environment, its context. Thus, a context may be
thought of as a process with a ``hole'' (written $\Box$) in it. The
application of a context $M$ to a process $P$, written $M[P]$, is
tantamount to filling the hole in $M$ with $P$. In this paper we do
not need the full weight of this theory, but do make use of the notion
of context in the proof the main theorem. 

\begin{mathpar}
  \inferrule* [lab=summation] {} {{M_{M},M_{N}} \bc \Box \;|\; x.M_{A} \;|\; M_{M}+M_{N}}
  \and
  \inferrule* [lab=agent] {} {{M_{A}} \bc (\vec{x})M_{P} \;| \; \clift{P_0,\ldots,M_{P},\ldots,P_N}}
  \and \\
  \inferrule* [lab=process] {} {{M_{P}} \bc M_{N} \;| \;P|M_{P} }
\end{mathpar} 

\begin{mathpar}
  \inferrule* [lab=sychronization] {} {M_{N} \bc \Box \;|\; x?M_{F} \;|\; x!M_{C}}
  \and
  \inferrule* [lab=abstraction] {} {{M_{F}} \bc (x)M_{P} }
  \and
  \inferrule* [lab=concretion] {} {{M_{C}} \bc \langle M_{P} \rangle }
  \and \\
  \inferrule* [lab=process] {} {{M_{P}} \bc M_{N} \;| \;P|M_{P} }
\end{mathpar}

\begin{definition}[contextual application] Given a context $M$, and
  process $P$, we define the \emph{contextual application}, $M[P] :=
  M\{P/\Box\}$. That is, the contextual application of M to P is the
  substitution of $P$ for $\Box$ in $M$.
\end{definition}

$\meaningof{-} : L \to \mathcal{P}(\pi)$

\begin{mathpar}
  \inferrule* [lab=collection] {} {\meaningof{true} = \pi, \and \meaningof{~E} = \pi \setminus \meaningof{E}, \and \meaningof{E_{1} \& E_{2}} = \meaningof{E_{1}} \cap \meaningof{E_{2}}}
\end{mathpar}

\begin{mathpar}
  \inferrule* [lab=structure] {} {\meaningof{0} = \{ P \in \pi | P \equiv 0 \}, \and \\ \meaningof{E_1 | E_2} = \{ P \in \pi | P \equiv P_{1} | P_{2}, P_{1} \in \meaningof{E_{1}}, P_{2} \in \meaningof{E_2}\} }
\end{mathpar}

\begin{mathpar}
 \inferrule* [lab=behavior] {} {\meaningof{\langle a?b \rangle E} = \{ P \in \pi | P \equiv Q | u?(y)P', \\ \and \\\\ \and \\ \;\;\; u \in \meaningof{a}, \forall z.P'\{z/y\} \in \meaningof{E\{z/b\}}\}, \and \\ \meaningof{a!E} = \{ P \in \pi | P \equiv Q | x!\langle P' \rangle, x \in \meaningof{a} P' \in \meaningof{E}\} }
\end{mathpar}

\begin{mathpar}
 \inferrule* [lab=nominal] {} {\meaningof{\quotep{E}} = \{ \quotep{P} \in \quotep{\pi} | P \in \meaningof{E} \}, \and \meaningof{\quotep{P}} = \{ \quotep{Q} \in \quotep{\pi} | P \equiv Q \} \and \\ \meaningof{@\quotep{E}} = \{ P \in \pi | P \equiv @x, x \in \meaningof{E} \}}
\end{mathpar}

\begin{eqnarray*}
  \\
  \meaningof{-} : TS \to ST
\end{eqnarray*}

\begin{eqnarray*}
  \\
  L : TS \to ST
\end{eqnarray*}

\begin{eqnarray*}
  \\
  P \models E \iff P \in \meaningof{E}
\end{eqnarray*}

\begin{eqnarray*}
  P \approx_{L} Q \iff \forall E \in L. P \models E \iff Q \models E
\end{eqnarray*}

\begin{eqnarray*}
  P \approx_{K} Q
\end{eqnarray*}

\begin{eqnarray*}
  P \approx Q
\end{eqnarray*}

$\approx_{K} = \approx = \approx_{L}$

\subsubsection{Contextual duality}

Note that contexts extend the quotation operation to a family of
operations from processes to names. Given a context, $M$, we can
define a \emph{nominal context}, $\quotep{M}$ by $\quotep{M}[P] :=
\quotep{M[P]}$. To foreshadow what is to come we observe that these
operations enjoy a duality with processes very much like the duality
between vectors and maps from vectors to scalars.

Further, because the calculus is essentially higher-order, we have a
correspondence between contexts and processes. More specifically,
given a name $x$ and a context $M$ we can construct $M^{*}_{x}$ such
that 

\begin{mathpar}
  M^{*}_{x} | \lift{x}{P} \red M[P]
\end{mathpar}

namely,

\begin{mathpar}
  M^{*}_{x} := x?(u).M[\dropn{u}]
\end{mathpar}

The dependence of $M^{*}_{x}$ on a name makes it an abstraction, 

\begin{mathpar}
  M^{*} := (x)x?(u).M[\dropn{u}]
\end{mathpar}

\subsection{Additional notation}

It will sometimes be convenient to denote the process a name
quotes. We already have the notation $x = \quotep{P}$, but it will be
convenient to introduce an alternate notation, $\procn{x}$, when we
want to emphasize the connection to the use of the name. Note that, by
virtue of name equivalence, $\quotep{\procn{x}} \nameeq x$; so, the
notation is consistent with previous definitions.

Further, because names have structure it is possible to effect
substitutions on the basis of that structure. This means we need to
upgrade our notation for substitutions, which we accomplish by
adapting comprehension notation. Thus,

\begin{mathpar}
  P\{ y / x : x \in S \}
\end{mathpar}

is interpreted to mean the process derived from P by replacing (in a
capture-avoiding manner) each occurrence of $x$ in $S$ by $y$. For example,

\begin{mathpar}
  P\{ \quotep{\procn{x}|\procn{x}} / x : x \in \freenames{P} \}
\end{mathpar}

will replace each (occurrence) of a free name $x$ in $P$ by
$\quotep{\procn{x}|\procn{x}}$.

Also, we will avail ourselves of the notation $x^{L}$ and $x^{R}$ to
denote injections of a name into disjoint copies of the name
space. There are numerous ways to accomplish this. One example can be
found in \cite{MeredithR05}. This notation overloads to vectors of
names: $\vec{x}^{\pi} := (x_{i}^{\pi} \; : \; 0 \leq i < |\vec{x}| )$ where $\pi \in \{L,R\}$.

We also use $P^{\Box} := P|\Box$.

In \cite{MeredithR05} an interpretation of the new operator is
given. It turns out that there are several possible interpretations
all enjoying the requisite algebraic properties of the operator (see
\cite{milner91polyadicpi}). We will therefore make liberal use of
$(\nu\; \vec{x})P$.

% subsection the_syntax_and_semantics_of_the_notation_system (end)   

\input{qm2pi.qmops} 

\input{qm2pi.sterngerlach} 

\input{qm2pi.metric} 

% section concurrent_process_calculi (end)

%\input{qm2pi.proofsketch}

% section proof sketch (end)

%\input{qm2pi.slviaknots} 

% section spatial logic via knots (end)

\input{qm2pi.conclusion}

% section conclusion (end)

%\input{qm2pi.dtcodes} 

% section wiring algorithm (end)

\input{qm2pi.ack} 

% section acknowledgments (end)

\newpage


\bibliographystyle{plain}   
\bibliography{../../biblios/main.bib}

\input{qm2pi.rhodetails}

\end{document}

 

%\ifpdf
%\usepackage[pdftex]{graphicx}
%\else
%\usepackage{graphicx}
%\fi

 % \ifpdf
%  \usepackage{pdfsync}
%  \if


%\title{Brief Article}
%\author{David F. Snyder}
%\author{L.G. Meredith}

%\address{Dept. of Math., Texas State University--San Marcos, San Marcos, TX 78666}
       
\pagestyle{empty}


\begin{document}

\lstset{language=[Objective]Caml,frame=shadowbox}

\documentclass[12pt]{llncs}
%\documentclass{jktr}

\usepackage[pdftex]{hyperref}                   
\usepackage {listings}
\usepackage {mathpartir}
\usepackage{bcprules}
%\usepackage{listings}
                       
\usepackage{graphicx} 
%\usepackage[margins=2.5cm,nohead,nofoot]{geometry}
%\usepackage{geometry}
\usepackage{amsfonts}
\usepackage{amstext}
\usepackage{latexsym}
\usepackage{amssymb}
\usepackage{color}


%\include{myPreamble}
\include{qm2pi.local} 

%\ifpdf
%\usepackage[pdftex]{graphicx}
%\else
%\usepackage{graphicx}
%\fi

 % \ifpdf
%  \usepackage{pdfsync}
%  \if


%\title{Brief Article}
%\author{David F. Snyder}
%\author{L.G. Meredith}

%\address{Dept. of Math., Texas State University--San Marcos, San Marcos, TX 78666}
       
\pagestyle{empty}


\begin{document}

\lstset{language=[Objective]Caml,frame=shadowbox}

\input{qm2pi.front}

% section front matter (end)

\input{qm2pi.intro} 
 
% section introduction (end)

% \input{qm2pi.knotations} 

% section notation (end)

\input{qm2pi.process.calculi} 

% section concurrent_process_calculi_and_spatial_logics_ (end)
    
%\input{qm2pi.knots2pi} 

%\input{qm2pi.trefoil} 

%\input{qm2pi.mainthm} 

% subsection basic_interpretation (end)

%\input{qm2pi.rho.presentation} 
\subsection{The syntax and semantics of the notation system}\label{sub:the_syntax_and_semantics_of_the_notation_system} % (fold)

We now summarize a technical presentation of the calculus that
embodies our theory of dynamics. The typical presentation of such a
calculus follows the style of giving generators and relations on
them. The grammar, below, describing term constructors, freely
generates the set of processes, $\Proc$. This set is then quotiented
by a relation known as structural congruence and it is over this set
that the notion of dynamics is expressed. This presentation is
essentially that of \cite{MeredithR05} with the addition of
polyadicity and summation. For readability we have relegated some of
the technical subtleties to an appendix.

\subsubsection{Process grammar}\label{subsub:process_grammar}

\begin{mathpar}
  \inferrule* [lab=synchronization] {} {{M} \bc \pzero \;|\; x?F \;|\; x!C }
  \and
  \inferrule* [lab=abstraction] {} {{F} \bc (x)P}
  \and
  \inferrule* [lab=concretion] {} {{C} \bc \langle Q \rangle}
  \and
  \inferrule* [lab=process] {} {{P,Q} \bc M \;| \;P|Q \;|\; @{x}}
  \and
  \inferrule* [lab=name] {} {{x} \bc \quotep{P}}
\end{mathpar} 

Note that $\vec{x}$ (resp. $\vec{P}$) denotes a vector of names
(resp. processes) of length $|\vec{x}|$ (resp. $|\vec{P}|$). We adopt
the following useful abbreviations.

\begin{mathpar}
   x?(\vec{y}).P := x.(\vec{y})P \and  x\clift{\vec{P}} := x.\clift{\vec{P}}
   \and x!(y) := \lift{x}{\dropn{y}}
   \and \Pi_{i=0}^{n-1}P_i := P_0 | \ldots | P_{n-1}
\end{mathpar}

\subsubsection{Structural congruence}

\paragraph{Free and bound names and alpha-equivalence.} At the
core of structural equivalence is alpha-equivalence which identifies
process that are the same up to a change of variable. Formally, we
recognize the distinction between free and bound names. The free names
of a process, $\freenames{P}$, may be calculated recursively as
follows:

\begin{mathpar}
\freenames{\pzero} := \emptyset
  \and \\
  \freenames{x?(y).P} := \{ x \} \cup (\freenames{P} \setminus \{ y \})
  \and 
  \freenames{x!\langle P \rangle} := \{ x \} \cup \{ P \} 
  \and \\
  \freenames{P|Q} := \freenames{P} \cup \freenames{Q}
  \and \\
  \freenames{@{x}} := \{ x \}
\end{mathpar}

$\pi$
$\quotep{\pi}$

$\freenames{-} : \pi \to \mathcal{P}(\quotep{\pi})$

\begin{eqnarray*}
  \freenames{\pzero} & := & \emptyset \\
  \freenames{x?(y).P} & := & \{ x \} \cup (\freenames{P} \setminus \{ y \}) \\
  \freenames{x!\langle P \rangle} & := & \{ x \} \cup \{ P \} \\
  \freenames{P|Q} & := & \freenames{P} \cup \freenames{Q} \\
  \freenames{\dropn{x}} & := & \{ x \}
\end{eqnarray*}

The bound names of a process, $\boundnames{P}$, are those names occurring in $P$
that are not free. For example, in $x?(y).0$, the name $x$ is free, while $y$ is bound.

\begin{mathpar}
  \inferrule* [lab=monoidal-laws] {} { P|Q \equiv Q|P \and P|0 \equiv P \and P|(Q|R) \equiv (P|Q)|R }
\end{mathpar}

\begin{mathpar}
  \inferrule* [lab=alpha-equivalence] {} { (x)P \equiv (y)P\{y/x\} \and y \not\in \freenames{P} }
\end{mathpar}

\begin{definition}
Then two processes, $P,Q$, are alpha-equivalent if $P = Q\{\vec{y}/\vec{x}\}$ for
some $\vec{x} \in \boundnames{Q},\vec{y} \in \boundnames{P}$, where $Q\{\vec{y}/\vec{x}\}$
denotes the capture-avoiding substitution of $\vec{y}$ for $\vec{x}$ in $Q$.
\end{definition}

\begin{definition}
  The {\em structural congruence} \cite{SangiorgiWalker} , $\equiv$,
  between processes is the least congruence containing
  alpha-equivalence, satisfying the abelian monoid laws
  (associativity, commutativity and $\pzero$ as identity) for parallel
  composition $|$ and for summation $+$.
\end{definition}

\subsection{Name equivalence}

We take name equivalence, written $\nameeq$, to be the smallest
equivalence relation generated by the following rules.

\begin{mathpar}
\inferrule*[lab=Quote-drop]
{ }
{ \quotep{@{x}} \nameeq x }

\inferrule*[lab=Struct-equiv]
{ P \scong Q }
{ \quotep{P} \nameeq \quotep{Q} }
\end{mathpar}

The astute reader will have noticed that the mutual recursion of names
and processes imposes a mutual recursion on alpha-equivalence and
structural equivalence via name-equivalence. Fortunately, all of this
works out pleasantly and we may calculate in the natural way, free of
concern. The reader interested in the details is referred to the
appendix \ref{appendix:rho_details}.

\subsection{Substitution}

We use $\Proc$ for the set of processes, $\QProc$ for the set of
names, and $\id{\{}\vec{y} / \vec{x} \id{\}}$ to denote partial maps,
$s : \QProc \rightarrow \QProc$. A map, $s$ lifts, uniquely, to a map
on process terms, $\widehat{s} : \Proc \rightarrow \Proc$ by the
following equations.

\begin{mathpar}
  (0) \psubstp{Q}{P} := 0 \\
  (R \juxtap S) \psubstp{Q}{P}
  :=    
  (R)\psubstp{Q}{P} \juxtap (S) \psubstp{Q}{P} \\
  (x?(y).R) \psubstp{Q}{P}    
  :=    
  (x)\substp{Q}{P} (z)\concat( (R \psubstn{z}{y}) \psubstp{Q}{P} ) \\
  (\lift{x}{R}) \psubstp{Q}{P}  
  :=
  \lift{(x)\substp{Q}{P}}{ R \psubstp{Q}{P} } \\
%   (\dropn{x})  \psubstp{Q}{P}       
%   := 
%   \left\{ 
%     \begin{array}{ccc} 
%       \dropn{\quotep{Q}} & & x \nameeq \quotep{P} \\
%       \dropn{x} & & otherwise \\
%     \end{array}
%   \right. 
  (\dropn{x})  \psubstp{Q}{P}       
  := 
  \left\{ 
    \begin{array}{ccc} 
      Q & & x \nameeq \quotep{P} \\
      \dropn{x} & & otherwise \\
    \end{array}
  \right.
\end{mathpar}
 

where

\begin{eqnarray}
  (x)\id{\{} \lpquote Q \rpquote / \lpquote P \rpquote \id{\}}            = 
  \left\{ 
    \begin{array}{ccc}
      \lpquote Q \rpquote & & x \nameeq \lpquote P \rpquote \\
      x & & otherwise \\
    \end{array}
  \right. \nonumber
\end{eqnarray}

and $z$ is chosen distinct from $\quotep{P}$, $\quotep{Q}$, the free
names in $Q$, and all the names in $R$. Our $\alpha$-equivalence will
be built in the standard way from this substitution.

\begin{remark}\label{rem:no_self_referential_names}
  One consequence of these definitions is that $\forall P. \quotep{P}
  \not\in \freenames{P}$.
\end{remark}

\subsection{ Dynamic quote: an example }

Anticipating something of what's to come, consider applying the
substitution, $\widehat{\id{\{}u / z \id{\}}}$, to the following pair
of processes, $\lift{w}{y!(z)}$ and $w[ \lpquote y!(z) \rpquote ]$.

\begin{eqnarray}
	\lift{w}{y!(z)}\widehat{\id{\{}u / z \id{\}}}
		& = &
		\lift{w}{y!(u)} \nonumber\\
	w[ \lpquote y!(z) \rpquote ] \widehat{ \id{\{}u / z \id{\}} }
		& = &
		w[ \lpquote y!(z) \rpquote ] \nonumber
\end{eqnarray}

Because the body of the process between quotes is impervious to
substitution, we get radically different answers. In fact, by
examining the first process in an input context,
e.g. $x?(z).\lift{w}{y!(z)}$, we see that the process under the lift
operator may be shaped by prefixed inputs binding a name inside it. In
this sense, the lift operator will be seen as a way to dynamically
construct processes before reifying them as names.

Finally equipped with these standard features we can present the
dynamics of the calculus.

\subsubsection{Operational semantics} 

Finally, we introduce the computational dynamics. What marks these
algebras as distinct from other more traditionally studied algebraic
structures, e.g. vector spaces or polynomial rings, is the manner in
which dynamics is captured. In traditional structures, dynamics is typically
expressed through morphisms between such structures, as in linear maps
between vector spaces or morphisms between rings. In algebras
associated with the semantics of computation, the dynamics is
expressed as part of the algebraic structure itself, through a
reduction reduction relation typically denoted by $\red$. Below, we
give a recursive presentation of this relation for the calculus used
in the encoding.

$\red \subseteq \pi \times \pi$
$\red : \pi \to \mathcal{P}(\pi)$

\begin{mathpar}
  \inferrule* [lab=Comm] { \textsf{match}( x_{src}, x_{trgt} ) } { x_{trgt}?(y)P \; | \; x_{src}!\langle {Q} \rangle \red P\{\quotep{Q}/y}\} }
  \and \\
  \inferrule* [lab=Par] {{P} \red {P}'} {{{P} | {Q}} \red {{P}' | {Q}}}
  \and
  \inferrule* [lab=Equiv]{{{P} \scong {P}'} \andalso {{P}' \red {Q}'} \andalso {{Q}' \scong {Q}}}{{P} \red {Q}}
\end{mathpar}

\begin{eqnarray*}
  match_{\equiv} (\quotep{P},\quotep{Q}) & := & P \equiv Q \\
  match_{\dagger}(\quotep{P},\quotep{Q}) & := & \forall R. P|Q \red^{*} R => R \red^{*} 0 \\
  match_{K}(\quotep{P},\quotep{Q}) & := & K \mbox{ for some context } K
\end{eqnarray*}

$u?(x)P | u!\langle Q \rangle \red P\{\quotep{Q}/x\}$

%We write $\wred$ for $\red^*$, and $P\red$ if $\exists Q $ such that $ P \red Q$.
We write $P\red$ if $\exists Q $ such that $ P \red Q$ and $P\not\red$, otherwise.

\section{Replication}

As mentioned before, it is known that replication (and hence
recursion) can be implemented in a higher-order process algebra
\cite{SangiorgiWalker}. As our first example of calculation with the
machinery thus far presented we give the construction explicitly in
the {\rhoc}.

\begin{eqnarray}
	D_{x} & := & \prefix{x}{y}{(\binpar{\outputp{x}{y}}{@{y}})} \nonumber\\
	\bangp_{x}{P} & := & \binpar{{x}!\langle{\binpar{D_{x}}{P}}\rangle}{D_{x}} \nonumber
\end{eqnarray}

\begin{eqnarray}
	\bangp_{x}{P} & & \nonumber\\
	=
	& {x}!\langle{(\prefix{x}{y}{(\outputp{x}{y} | @{y})) | P}}\rangle 
	      | \prefix{x}{y}{(\outputp{x}{y} | @{y})} & \nonumber\\
	\red
	& (\outputp{x}{y} | @{y})\substn{\quotep{(\prefix{x}{y}{(@{y} | \outputp{x}{y})) | P}}}{y} & \nonumber\\
	=
	& \outputp{x}{\quotep{(\prefix{x}{y}{(\outputp{x}{y} | @{y})) | P}}}
	  | {(\prefix{x}{y}{(\outputp{x}{y} | @{y})) | P}} & \nonumber\\
	\red
	& \ldots & \nonumber\\
	\red^*
	& P | P | \ldots & \nonumber
\end{eqnarray}

Of course, this encoding, as an implementation, runs away, unfolding
$\bangp{P}$ eagerly. A lazier and more implementable replication
operator, restricted to input-guarded processes, may be obtained as follows.

\begin{eqnarray}
\bangp{\prefix{u}{v}{P}} 
	:= 
	\binpar{\lift{x}{\prefix{u}{v}{(\binpar{D(x)}{P})}}}{D(x)} \nonumber
\end{eqnarray}

\begin{remark}
  Note that the lazier definition still does not deal with summation
  or mixed summation (i.e. sums over input and output). The reader is
  invited to construct definitions of replication that deal with these
  features. 

  Further, the definitions are parameterized in a name, $x$. Can you,
  gentle reader, make a definition that eliminates this parameter and
  guarantees no accidental interaction between the replication
  machinery and the process being replicated -- i.e. no accidental
  sharing of names used by the process to get its work done and the
  name(s) used by the replication to effect copying. This latter
  revision of the definition of replication is crucial to obtaining
  the expected identity $!!P \sim !P$.
\end{remark}

\begin{remark}\label{rem:paradoxical_combinator}
  The reader familiar with the lambda calculus will have noticed the
  similarity between $D$ and the paradoxical combinator.

  [Ed. note: the existence of this seems to suggest we have to be more
  restrictive on the set of processes and names we admit if we are to
  support no-cloning.]
\end{remark}

\subsubsection{Bisimulation}

The computational dynamics gives rise to another kind of equivalence,
the equivalence of computational behavior. As previously mentioned
this is typically captured \emph{via} some form of bisimulation.

% The notion we use in this paper is weak barbed bisimulation
% \cite{milner91polyadicpi}.

The notion we use in this paper is derived from weak barbed
bisimulation \cite{milner91polyadicpi}. 

\begin{definition}
An \emph{observation relation}, $\downarrow_{\mathcal N}$, over a set
of names, $\mathcal N$, is the smallest relation satisfying the rules
below.

\infrule[Out-barb]{y \in {\mathcal N}, \; x \nameeq y}
		  {\outputp{x}{v} \downarrow_{\mathcal N} x}
\infrule[Par-barb]{\mbox{$P\downarrow_{\mathcal N} x$ or $Q\downarrow_{\mathcal N} x$}}
		  {\binpar{P}{Q} \downarrow_{\mathcal N} x}

We write $P \Downarrow_{\mathcal N} x$ if there is $Q$ such that 
$P \wred Q$ and $Q \downarrow_{\mathcal N} x$.
\end{definition}

\begin{definition}
%\label{def.bbisim}
An  ${\mathcal N}$-\emph{barbed bisimulation} over a set of names, ${\mathcal N}$, is a symmetric binary relation 
${\mathcal S}_{\mathcal N}$ between agents such that $P\rel{S}_{\mathcal N}Q$ implies:
\begin{enumerate}
\item If $P \red P'$ then $Q \wred Q'$ and $P'\rel{S}_{\mathcal N} Q'$.
\item If $P\downarrow_{\mathcal N} x$, then $Q\Downarrow_{\mathcal N} x$.
\end{enumerate}
$P$ is ${\mathcal N}$-barbed bisimilar to $Q$, written
$P \wbbisim_{\mathcal N} Q$, if $P \rel{S}_{\mathcal N} Q$ for some ${\mathcal N}$-barbed bisimulation ${\mathcal S}_{\mathcal N}$.
\end{definition}

$\mathcal{R} \subseteq \pi \times \pi$

$P \mathcal{R} Q => \forall P'. P \red P' \Rightarrow \exists Q'. Q \red Q', P' \mathcal{R} Q'$

$P \vdash x \Rightarrow Q \vdash x$

\begin{mathpar}
  \inferrule*[lab=Out-barb]{x \nameeq y}{{y}!\langle{Q}\rangle \vdash x}
  \and
  \inferrule*[lab=Par-barb]{\mbox{$P\vdash x$ or $Q\vdash x$}}{\binpar{P}{Q} \vdash x}
\end{mathpar}

\subsubsection{Contexts}

One of the principle advantages of computational calculi like the
$\pi$-calculus is a well-defined notion of context,
contextual-equivalence and a correlation between
contextual-equivalence and notions of bisimulation. The notion of
context allows the decomposition of a process into (sub-)process and
its syntactic environment, its context. Thus, a context may be
thought of as a process with a ``hole'' (written $\Box$) in it. The
application of a context $M$ to a process $P$, written $M[P]$, is
tantamount to filling the hole in $M$ with $P$. In this paper we do
not need the full weight of this theory, but do make use of the notion
of context in the proof the main theorem. 

\begin{mathpar}
  \inferrule* [lab=summation] {} {{M_{M},M_{N}} \bc \Box \;|\; x.M_{A} \;|\; M_{M}+M_{N}}
  \and
  \inferrule* [lab=agent] {} {{M_{A}} \bc (\vec{x})M_{P} \;| \; \clift{P_0,\ldots,M_{P},\ldots,P_N}}
  \and \\
  \inferrule* [lab=process] {} {{M_{P}} \bc M_{N} \;| \;P|M_{P} }
\end{mathpar} 

\begin{mathpar}
  \inferrule* [lab=sychronization] {} {M_{N} \bc \Box \;|\; x?M_{F} \;|\; x!M_{C}}
  \and
  \inferrule* [lab=abstraction] {} {{M_{F}} \bc (x)M_{P} }
  \and
  \inferrule* [lab=concretion] {} {{M_{C}} \bc \langle M_{P} \rangle }
  \and \\
  \inferrule* [lab=process] {} {{M_{P}} \bc M_{N} \;| \;P|M_{P} }
\end{mathpar}

\begin{definition}[contextual application] Given a context $M$, and
  process $P$, we define the \emph{contextual application}, $M[P] :=
  M\{P/\Box\}$. That is, the contextual application of M to P is the
  substitution of $P$ for $\Box$ in $M$.
\end{definition}

$\meaningof{-} : L \to \mathcal{P}(\pi)$

\begin{mathpar}
  \inferrule* [lab=collection] {} {\meaningof{true} = \pi, \and \meaningof{~E} = \pi \setminus \meaningof{E}, \and \meaningof{E_{1} \& E_{2}} = \meaningof{E_{1}} \cap \meaningof{E_{2}}}
\end{mathpar}

\begin{mathpar}
  \inferrule* [lab=structure] {} {\meaningof{0} = \{ P \in \pi | P \equiv 0 \}, \and \\ \meaningof{E_1 | E_2} = \{ P \in \pi | P \equiv P_{1} | P_{2}, P_{1} \in \meaningof{E_{1}}, P_{2} \in \meaningof{E_2}\} }
\end{mathpar}

\begin{mathpar}
 \inferrule* [lab=behavior] {} {\meaningof{\langle a?b \rangle E} = \{ P \in \pi | P \equiv Q | u?(y)P', \\ \and \\\\ \and \\ \;\;\; u \in \meaningof{a}, \forall z.P'\{z/y\} \in \meaningof{E\{z/b\}}\}, \and \\ \meaningof{a!E} = \{ P \in \pi | P \equiv Q | x!\langle P' \rangle, x \in \meaningof{a} P' \in \meaningof{E}\} }
\end{mathpar}

\begin{mathpar}
 \inferrule* [lab=nominal] {} {\meaningof{\quotep{E}} = \{ \quotep{P} \in \quotep{\pi} | P \in \meaningof{E} \}, \and \meaningof{\quotep{P}} = \{ \quotep{Q} \in \quotep{\pi} | P \equiv Q \} \and \\ \meaningof{@\quotep{E}} = \{ P \in \pi | P \equiv @x, x \in \meaningof{E} \}}
\end{mathpar}

\begin{eqnarray*}
  \\
  \meaningof{-} : TS \to ST
\end{eqnarray*}

\begin{eqnarray*}
  \\
  L : TS \to ST
\end{eqnarray*}

\begin{eqnarray*}
  \\
  P \models E \iff P \in \meaningof{E}
\end{eqnarray*}

\begin{eqnarray*}
  P \approx_{L} Q \iff \forall E \in L. P \models E \iff Q \models E
\end{eqnarray*}

\begin{eqnarray*}
  P \approx_{K} Q
\end{eqnarray*}

\begin{eqnarray*}
  P \approx Q
\end{eqnarray*}

$\approx_{K} = \approx = \approx_{L}$

\subsubsection{Contextual duality}

Note that contexts extend the quotation operation to a family of
operations from processes to names. Given a context, $M$, we can
define a \emph{nominal context}, $\quotep{M}$ by $\quotep{M}[P] :=
\quotep{M[P]}$. To foreshadow what is to come we observe that these
operations enjoy a duality with processes very much like the duality
between vectors and maps from vectors to scalars.

Further, because the calculus is essentially higher-order, we have a
correspondence between contexts and processes. More specifically,
given a name $x$ and a context $M$ we can construct $M^{*}_{x}$ such
that 

\begin{mathpar}
  M^{*}_{x} | \lift{x}{P} \red M[P]
\end{mathpar}

namely,

\begin{mathpar}
  M^{*}_{x} := x?(u).M[\dropn{u}]
\end{mathpar}

The dependence of $M^{*}_{x}$ on a name makes it an abstraction, 

\begin{mathpar}
  M^{*} := (x)x?(u).M[\dropn{u}]
\end{mathpar}

\subsection{Additional notation}

It will sometimes be convenient to denote the process a name
quotes. We already have the notation $x = \quotep{P}$, but it will be
convenient to introduce an alternate notation, $\procn{x}$, when we
want to emphasize the connection to the use of the name. Note that, by
virtue of name equivalence, $\quotep{\procn{x}} \nameeq x$; so, the
notation is consistent with previous definitions.

Further, because names have structure it is possible to effect
substitutions on the basis of that structure. This means we need to
upgrade our notation for substitutions, which we accomplish by
adapting comprehension notation. Thus,

\begin{mathpar}
  P\{ y / x : x \in S \}
\end{mathpar}

is interpreted to mean the process derived from P by replacing (in a
capture-avoiding manner) each occurrence of $x$ in $S$ by $y$. For example,

\begin{mathpar}
  P\{ \quotep{\procn{x}|\procn{x}} / x : x \in \freenames{P} \}
\end{mathpar}

will replace each (occurrence) of a free name $x$ in $P$ by
$\quotep{\procn{x}|\procn{x}}$.

Also, we will avail ourselves of the notation $x^{L}$ and $x^{R}$ to
denote injections of a name into disjoint copies of the name
space. There are numerous ways to accomplish this. One example can be
found in \cite{MeredithR05}. This notation overloads to vectors of
names: $\vec{x}^{\pi} := (x_{i}^{\pi} \; : \; 0 \leq i < |\vec{x}| )$ where $\pi \in \{L,R\}$.

We also use $P^{\Box} := P|\Box$.

In \cite{MeredithR05} an interpretation of the new operator is
given. It turns out that there are several possible interpretations
all enjoying the requisite algebraic properties of the operator (see
\cite{milner91polyadicpi}). We will therefore make liberal use of
$(\nu\; \vec{x})P$.

% subsection the_syntax_and_semantics_of_the_notation_system (end)   

\input{qm2pi.qmops} 

\input{qm2pi.sterngerlach} 

\input{qm2pi.metric} 

% section concurrent_process_calculi (end)

%\input{qm2pi.proofsketch}

% section proof sketch (end)

%\input{qm2pi.slviaknots} 

% section spatial logic via knots (end)

\input{qm2pi.conclusion}

% section conclusion (end)

%\input{qm2pi.dtcodes} 

% section wiring algorithm (end)

\input{qm2pi.ack} 

% section acknowledgments (end)

\newpage


\bibliographystyle{plain}   
\bibliography{../../biblios/main.bib}

\input{qm2pi.rhodetails}

\end{document}



% section front matter (end)

\section{Introduction}\label{sec:introduction} % (fold)
In this draft of the material i am going to have to dispense with the
usual writing conventions adopted in papers on these topics. i'm going
to have adopt whatever tone i need at the time i'm writing up the
calculations. Sometimes this may be very conversational; others it may
be the barest mathematical grunts; others still it may be that i have
lifted text from one of my other papers because the exposition of some
point was better said there. i hope that my readers are not unduly put
out by this decision. i'm not doing this to flout convention or be
rebellious. i find these calculations very technically challenging. To
keep everything going technically, something has to give; i have to
let go of some cognitive burden. So, the academic writing style --
with all of its trade-offs in terms of facilitating technical
communication -- is what i'm letting go of. Perhaps subsequent drafts
can be tightened and polished, but for now, i'm going to speak as if
we were sitting together in a coffee shop with a laptop, wifi and a
pad of paper and a pencil.

So, here's what i have to say. We -- you and i, comfortably ensconced
in our coffee shop and well-equipped with our tools -- can realize and
carry out the calculations of quantum mechanics over a very different
formal theory of dynamics, a formal theory of dynamics that
corresponds to a theory of concurrent computation with
\emph{reflection}. It has the advantage that the underlying theory is
already `quantized', but supports analogues all of the continuuous
operations. Strikingly, this underlying theory has recently been
connected with a notion of metric that we can show, by calculating
together, coincides with the metric induced by the inner product.

There are a lot of reasons why you might be interested in seeing
calculations of this form. Here's why i'm interested. For the past
several centuries there has been no competitor to the ``Newtonian''
account of dynamics. As a result the predominant share of accounts of
dynamical systems and situations have had to be formulated in terms of
the Newtonian machinery. i view this as an intellectually dangerous
position to occupy. Everything, despite it's intrinsic shape, turns
into a nail to be hit with this hammer. Recently, however, the theory
of computation has matured to the point where we have candidates for
theories of dynamics that offer very different perspective on
reasoning about dynamical systems and situations. Testing these
candidates against very successful accounts of dynamical situations,
like quantum mechanics, is going to give us some sense of how mature
they are and some measure of the quality of these accounts of
dynamics.

\subsection{Summary of contributions and outline of paper}

So, we're going to develop an interpretation of the operations of
quantum mechanics normally interpreted by Hilbert spaces and
operators. We're going to do this over a theory of computation. Note
that this is very different than the usual quantum computation program
which develops notions of computation over quantum mechanics. Rather,
we are developing a story that aligns with Wheeler's slogan: It from
Bit. To do this we will first provide an account of the theory of
computation at play here. Then we will dive into a calculation-driven
interpretation of the operations of quantum mechanics.

The reason we take this approach is that -- until very recently --
there hasn't been an axiomatic account of quantum mechanics. As a
result there has been no sharp delineation of the mathematical theory
supporting interpretation of the physical theory and the physical
theory, itself. So, ambient features of the maths are free to be
exploited (or supressed) without a real accounting of their physical
relevance. There is no sharp statement ``here's the physical theory''
qua \emph{theory} and ``here's the mathematical interpretation''
enabling a judgment of how faithful the interpretation is -- apart
from experimental observation. When there is an axiomatic account we
can judge how well a given mathematical formalism supports an
interpretation of the axioms, independent of
experimentation. Likewise, we can judge how well we have captured our
physical evidence and experience with our axiomatics, independent of
any specific mathematical implementation, with accidental detail that
may or may not have physical significance. 

In lieu of a fully fleshed out and vetted axiomatic account of quantum
mechanics, interpreting the operational notions in service of modeling
physical systems will have to suffice. In other words, we are not in
the business of providing a model of Hilbert spaces and operators. We
are in the business of providing a model of quantum mechanics because
we are motivated by testing our notions of dynamics against physical
theory; and, the predictive calculations of the physical theory must
serve as the best formulation -- shy of a fully fleshed out axiomatic
account -- of the physical theory itself (as they have for scientific
theories since time immemorial). Put another way, despite a
whole-hearted commitment to an It-from-Bit ontology, we are firmly
aligned with the shut-up-and-calculate camp as the best way to obtain
results either from the physical perspective or as a quality assurance
measure of our fledgling theory of dynamics.

In detail, we present a reflective process calculus. Then we develop
intuitive correspondences between the notions available in this
calculus and the usual physical notions supporting quantum mechanical
calculations. Thus, 

\begin{table}[htp]
  \center{
    \fbox{
      \begin{tabular}{c|c}
        quantum mechanics & process calculus \\
        \hline
        scalar & name \\
        state vector & process \\
        dual & contextual duals \\
        matrix & formal sums of process-context-dual pairs \\
        orthogonality & process annihilation \\
        inner product & execution-formula + quoting
      \end{tabular}
    }
  }
  \caption{QM - process calculi correspondences}
\end{table}

Then we tighten up these intuitions to operational definitions. We
employ the Dirac notation as the best proxy we can find for an
abstract syntax of the quantum mechanical notions. The definitions we
develop put us in contact with equational constraints coming from the
theory that we demonstrate the definitions and calculations satisfy.

This puts us in a position to shut up and calculate for the
Stern-Gerlach experimental set up, showing how these predictive
calculations become calculations on processes in our theory of a
reflective process calculus.

Penultimately, we demonstrate that the notion of metric coming from
the inner product coincides with the notion of metric available from
the theory of bisimulation. This demonstration gives us the right to
think of space as arising from behavior. Finally, we consider where we
might go from the new vantage point we have obtained.

% section introduction (end) 
 
% section introduction (end)

% \documentclass[12pt]{llncs}
%\documentclass{jktr}

\usepackage[pdftex]{hyperref}                   
\usepackage {listings}
\usepackage {mathpartir}
\usepackage{bcprules}
%\usepackage{listings}
                       
\usepackage{graphicx} 
%\usepackage[margins=2.5cm,nohead,nofoot]{geometry}
%\usepackage{geometry}
\usepackage{amsfonts}
\usepackage{amstext}
\usepackage{latexsym}
\usepackage{amssymb}
\usepackage{color}


%\include{myPreamble}
\include{qm2pi.local} 

%\ifpdf
%\usepackage[pdftex]{graphicx}
%\else
%\usepackage{graphicx}
%\fi

 % \ifpdf
%  \usepackage{pdfsync}
%  \if


%\title{Brief Article}
%\author{David F. Snyder}
%\author{L.G. Meredith}

%\address{Dept. of Math., Texas State University--San Marcos, San Marcos, TX 78666}
       
\pagestyle{empty}


\begin{document}

\lstset{language=[Objective]Caml,frame=shadowbox}

\input{qm2pi.front}

% section front matter (end)

\input{qm2pi.intro} 
 
% section introduction (end)

% \input{qm2pi.knotations} 

% section notation (end)

\input{qm2pi.process.calculi} 

% section concurrent_process_calculi_and_spatial_logics_ (end)
    
%\input{qm2pi.knots2pi} 

%\input{qm2pi.trefoil} 

%\input{qm2pi.mainthm} 

% subsection basic_interpretation (end)

%\input{qm2pi.rho.presentation} 
\subsection{The syntax and semantics of the notation system}\label{sub:the_syntax_and_semantics_of_the_notation_system} % (fold)

We now summarize a technical presentation of the calculus that
embodies our theory of dynamics. The typical presentation of such a
calculus follows the style of giving generators and relations on
them. The grammar, below, describing term constructors, freely
generates the set of processes, $\Proc$. This set is then quotiented
by a relation known as structural congruence and it is over this set
that the notion of dynamics is expressed. This presentation is
essentially that of \cite{MeredithR05} with the addition of
polyadicity and summation. For readability we have relegated some of
the technical subtleties to an appendix.

\subsubsection{Process grammar}\label{subsub:process_grammar}

\begin{mathpar}
  \inferrule* [lab=synchronization] {} {{M} \bc \pzero \;|\; x?F \;|\; x!C }
  \and
  \inferrule* [lab=abstraction] {} {{F} \bc (x)P}
  \and
  \inferrule* [lab=concretion] {} {{C} \bc \langle Q \rangle}
  \and
  \inferrule* [lab=process] {} {{P,Q} \bc M \;| \;P|Q \;|\; @{x}}
  \and
  \inferrule* [lab=name] {} {{x} \bc \quotep{P}}
\end{mathpar} 

Note that $\vec{x}$ (resp. $\vec{P}$) denotes a vector of names
(resp. processes) of length $|\vec{x}|$ (resp. $|\vec{P}|$). We adopt
the following useful abbreviations.

\begin{mathpar}
   x?(\vec{y}).P := x.(\vec{y})P \and  x\clift{\vec{P}} := x.\clift{\vec{P}}
   \and x!(y) := \lift{x}{\dropn{y}}
   \and \Pi_{i=0}^{n-1}P_i := P_0 | \ldots | P_{n-1}
\end{mathpar}

\subsubsection{Structural congruence}

\paragraph{Free and bound names and alpha-equivalence.} At the
core of structural equivalence is alpha-equivalence which identifies
process that are the same up to a change of variable. Formally, we
recognize the distinction between free and bound names. The free names
of a process, $\freenames{P}$, may be calculated recursively as
follows:

\begin{mathpar}
\freenames{\pzero} := \emptyset
  \and \\
  \freenames{x?(y).P} := \{ x \} \cup (\freenames{P} \setminus \{ y \})
  \and 
  \freenames{x!\langle P \rangle} := \{ x \} \cup \{ P \} 
  \and \\
  \freenames{P|Q} := \freenames{P} \cup \freenames{Q}
  \and \\
  \freenames{@{x}} := \{ x \}
\end{mathpar}

$\pi$
$\quotep{\pi}$

$\freenames{-} : \pi \to \mathcal{P}(\quotep{\pi})$

\begin{eqnarray*}
  \freenames{\pzero} & := & \emptyset \\
  \freenames{x?(y).P} & := & \{ x \} \cup (\freenames{P} \setminus \{ y \}) \\
  \freenames{x!\langle P \rangle} & := & \{ x \} \cup \{ P \} \\
  \freenames{P|Q} & := & \freenames{P} \cup \freenames{Q} \\
  \freenames{\dropn{x}} & := & \{ x \}
\end{eqnarray*}

The bound names of a process, $\boundnames{P}$, are those names occurring in $P$
that are not free. For example, in $x?(y).0$, the name $x$ is free, while $y$ is bound.

\begin{mathpar}
  \inferrule* [lab=monoidal-laws] {} { P|Q \equiv Q|P \and P|0 \equiv P \and P|(Q|R) \equiv (P|Q)|R }
\end{mathpar}

\begin{mathpar}
  \inferrule* [lab=alpha-equivalence] {} { (x)P \equiv (y)P\{y/x\} \and y \not\in \freenames{P} }
\end{mathpar}

\begin{definition}
Then two processes, $P,Q$, are alpha-equivalent if $P = Q\{\vec{y}/\vec{x}\}$ for
some $\vec{x} \in \boundnames{Q},\vec{y} \in \boundnames{P}$, where $Q\{\vec{y}/\vec{x}\}$
denotes the capture-avoiding substitution of $\vec{y}$ for $\vec{x}$ in $Q$.
\end{definition}

\begin{definition}
  The {\em structural congruence} \cite{SangiorgiWalker} , $\equiv$,
  between processes is the least congruence containing
  alpha-equivalence, satisfying the abelian monoid laws
  (associativity, commutativity and $\pzero$ as identity) for parallel
  composition $|$ and for summation $+$.
\end{definition}

\subsection{Name equivalence}

We take name equivalence, written $\nameeq$, to be the smallest
equivalence relation generated by the following rules.

\begin{mathpar}
\inferrule*[lab=Quote-drop]
{ }
{ \quotep{@{x}} \nameeq x }

\inferrule*[lab=Struct-equiv]
{ P \scong Q }
{ \quotep{P} \nameeq \quotep{Q} }
\end{mathpar}

The astute reader will have noticed that the mutual recursion of names
and processes imposes a mutual recursion on alpha-equivalence and
structural equivalence via name-equivalence. Fortunately, all of this
works out pleasantly and we may calculate in the natural way, free of
concern. The reader interested in the details is referred to the
appendix \ref{appendix:rho_details}.

\subsection{Substitution}

We use $\Proc$ for the set of processes, $\QProc$ for the set of
names, and $\id{\{}\vec{y} / \vec{x} \id{\}}$ to denote partial maps,
$s : \QProc \rightarrow \QProc$. A map, $s$ lifts, uniquely, to a map
on process terms, $\widehat{s} : \Proc \rightarrow \Proc$ by the
following equations.

\begin{mathpar}
  (0) \psubstp{Q}{P} := 0 \\
  (R \juxtap S) \psubstp{Q}{P}
  :=    
  (R)\psubstp{Q}{P} \juxtap (S) \psubstp{Q}{P} \\
  (x?(y).R) \psubstp{Q}{P}    
  :=    
  (x)\substp{Q}{P} (z)\concat( (R \psubstn{z}{y}) \psubstp{Q}{P} ) \\
  (\lift{x}{R}) \psubstp{Q}{P}  
  :=
  \lift{(x)\substp{Q}{P}}{ R \psubstp{Q}{P} } \\
%   (\dropn{x})  \psubstp{Q}{P}       
%   := 
%   \left\{ 
%     \begin{array}{ccc} 
%       \dropn{\quotep{Q}} & & x \nameeq \quotep{P} \\
%       \dropn{x} & & otherwise \\
%     \end{array}
%   \right. 
  (\dropn{x})  \psubstp{Q}{P}       
  := 
  \left\{ 
    \begin{array}{ccc} 
      Q & & x \nameeq \quotep{P} \\
      \dropn{x} & & otherwise \\
    \end{array}
  \right.
\end{mathpar}
 

where

\begin{eqnarray}
  (x)\id{\{} \lpquote Q \rpquote / \lpquote P \rpquote \id{\}}            = 
  \left\{ 
    \begin{array}{ccc}
      \lpquote Q \rpquote & & x \nameeq \lpquote P \rpquote \\
      x & & otherwise \\
    \end{array}
  \right. \nonumber
\end{eqnarray}

and $z$ is chosen distinct from $\quotep{P}$, $\quotep{Q}$, the free
names in $Q$, and all the names in $R$. Our $\alpha$-equivalence will
be built in the standard way from this substitution.

\begin{remark}\label{rem:no_self_referential_names}
  One consequence of these definitions is that $\forall P. \quotep{P}
  \not\in \freenames{P}$.
\end{remark}

\subsection{ Dynamic quote: an example }

Anticipating something of what's to come, consider applying the
substitution, $\widehat{\id{\{}u / z \id{\}}}$, to the following pair
of processes, $\lift{w}{y!(z)}$ and $w[ \lpquote y!(z) \rpquote ]$.

\begin{eqnarray}
	\lift{w}{y!(z)}\widehat{\id{\{}u / z \id{\}}}
		& = &
		\lift{w}{y!(u)} \nonumber\\
	w[ \lpquote y!(z) \rpquote ] \widehat{ \id{\{}u / z \id{\}} }
		& = &
		w[ \lpquote y!(z) \rpquote ] \nonumber
\end{eqnarray}

Because the body of the process between quotes is impervious to
substitution, we get radically different answers. In fact, by
examining the first process in an input context,
e.g. $x?(z).\lift{w}{y!(z)}$, we see that the process under the lift
operator may be shaped by prefixed inputs binding a name inside it. In
this sense, the lift operator will be seen as a way to dynamically
construct processes before reifying them as names.

Finally equipped with these standard features we can present the
dynamics of the calculus.

\subsubsection{Operational semantics} 

Finally, we introduce the computational dynamics. What marks these
algebras as distinct from other more traditionally studied algebraic
structures, e.g. vector spaces or polynomial rings, is the manner in
which dynamics is captured. In traditional structures, dynamics is typically
expressed through morphisms between such structures, as in linear maps
between vector spaces or morphisms between rings. In algebras
associated with the semantics of computation, the dynamics is
expressed as part of the algebraic structure itself, through a
reduction reduction relation typically denoted by $\red$. Below, we
give a recursive presentation of this relation for the calculus used
in the encoding.

$\red \subseteq \pi \times \pi$
$\red : \pi \to \mathcal{P}(\pi)$

\begin{mathpar}
  \inferrule* [lab=Comm] { \textsf{match}( x_{src}, x_{trgt} ) } { x_{trgt}?(y)P \; | \; x_{src}!\langle {Q} \rangle \red P\{\quotep{Q}/y}\} }
  \and \\
  \inferrule* [lab=Par] {{P} \red {P}'} {{{P} | {Q}} \red {{P}' | {Q}}}
  \and
  \inferrule* [lab=Equiv]{{{P} \scong {P}'} \andalso {{P}' \red {Q}'} \andalso {{Q}' \scong {Q}}}{{P} \red {Q}}
\end{mathpar}

\begin{eqnarray*}
  match_{\equiv} (\quotep{P},\quotep{Q}) & := & P \equiv Q \\
  match_{\dagger}(\quotep{P},\quotep{Q}) & := & \forall R. P|Q \red^{*} R => R \red^{*} 0 \\
  match_{K}(\quotep{P},\quotep{Q}) & := & K \mbox{ for some context } K
\end{eqnarray*}

$u?(x)P | u!\langle Q \rangle \red P\{\quotep{Q}/x\}$

%We write $\wred$ for $\red^*$, and $P\red$ if $\exists Q $ such that $ P \red Q$.
We write $P\red$ if $\exists Q $ such that $ P \red Q$ and $P\not\red$, otherwise.

\section{Replication}

As mentioned before, it is known that replication (and hence
recursion) can be implemented in a higher-order process algebra
\cite{SangiorgiWalker}. As our first example of calculation with the
machinery thus far presented we give the construction explicitly in
the {\rhoc}.

\begin{eqnarray}
	D_{x} & := & \prefix{x}{y}{(\binpar{\outputp{x}{y}}{@{y}})} \nonumber\\
	\bangp_{x}{P} & := & \binpar{{x}!\langle{\binpar{D_{x}}{P}}\rangle}{D_{x}} \nonumber
\end{eqnarray}

\begin{eqnarray}
	\bangp_{x}{P} & & \nonumber\\
	=
	& {x}!\langle{(\prefix{x}{y}{(\outputp{x}{y} | @{y})) | P}}\rangle 
	      | \prefix{x}{y}{(\outputp{x}{y} | @{y})} & \nonumber\\
	\red
	& (\outputp{x}{y} | @{y})\substn{\quotep{(\prefix{x}{y}{(@{y} | \outputp{x}{y})) | P}}}{y} & \nonumber\\
	=
	& \outputp{x}{\quotep{(\prefix{x}{y}{(\outputp{x}{y} | @{y})) | P}}}
	  | {(\prefix{x}{y}{(\outputp{x}{y} | @{y})) | P}} & \nonumber\\
	\red
	& \ldots & \nonumber\\
	\red^*
	& P | P | \ldots & \nonumber
\end{eqnarray}

Of course, this encoding, as an implementation, runs away, unfolding
$\bangp{P}$ eagerly. A lazier and more implementable replication
operator, restricted to input-guarded processes, may be obtained as follows.

\begin{eqnarray}
\bangp{\prefix{u}{v}{P}} 
	:= 
	\binpar{\lift{x}{\prefix{u}{v}{(\binpar{D(x)}{P})}}}{D(x)} \nonumber
\end{eqnarray}

\begin{remark}
  Note that the lazier definition still does not deal with summation
  or mixed summation (i.e. sums over input and output). The reader is
  invited to construct definitions of replication that deal with these
  features. 

  Further, the definitions are parameterized in a name, $x$. Can you,
  gentle reader, make a definition that eliminates this parameter and
  guarantees no accidental interaction between the replication
  machinery and the process being replicated -- i.e. no accidental
  sharing of names used by the process to get its work done and the
  name(s) used by the replication to effect copying. This latter
  revision of the definition of replication is crucial to obtaining
  the expected identity $!!P \sim !P$.
\end{remark}

\begin{remark}\label{rem:paradoxical_combinator}
  The reader familiar with the lambda calculus will have noticed the
  similarity between $D$ and the paradoxical combinator.

  [Ed. note: the existence of this seems to suggest we have to be more
  restrictive on the set of processes and names we admit if we are to
  support no-cloning.]
\end{remark}

\subsubsection{Bisimulation}

The computational dynamics gives rise to another kind of equivalence,
the equivalence of computational behavior. As previously mentioned
this is typically captured \emph{via} some form of bisimulation.

% The notion we use in this paper is weak barbed bisimulation
% \cite{milner91polyadicpi}.

The notion we use in this paper is derived from weak barbed
bisimulation \cite{milner91polyadicpi}. 

\begin{definition}
An \emph{observation relation}, $\downarrow_{\mathcal N}$, over a set
of names, $\mathcal N$, is the smallest relation satisfying the rules
below.

\infrule[Out-barb]{y \in {\mathcal N}, \; x \nameeq y}
		  {\outputp{x}{v} \downarrow_{\mathcal N} x}
\infrule[Par-barb]{\mbox{$P\downarrow_{\mathcal N} x$ or $Q\downarrow_{\mathcal N} x$}}
		  {\binpar{P}{Q} \downarrow_{\mathcal N} x}

We write $P \Downarrow_{\mathcal N} x$ if there is $Q$ such that 
$P \wred Q$ and $Q \downarrow_{\mathcal N} x$.
\end{definition}

\begin{definition}
%\label{def.bbisim}
An  ${\mathcal N}$-\emph{barbed bisimulation} over a set of names, ${\mathcal N}$, is a symmetric binary relation 
${\mathcal S}_{\mathcal N}$ between agents such that $P\rel{S}_{\mathcal N}Q$ implies:
\begin{enumerate}
\item If $P \red P'$ then $Q \wred Q'$ and $P'\rel{S}_{\mathcal N} Q'$.
\item If $P\downarrow_{\mathcal N} x$, then $Q\Downarrow_{\mathcal N} x$.
\end{enumerate}
$P$ is ${\mathcal N}$-barbed bisimilar to $Q$, written
$P \wbbisim_{\mathcal N} Q$, if $P \rel{S}_{\mathcal N} Q$ for some ${\mathcal N}$-barbed bisimulation ${\mathcal S}_{\mathcal N}$.
\end{definition}

$\mathcal{R} \subseteq \pi \times \pi$

$P \mathcal{R} Q => \forall P'. P \red P' \Rightarrow \exists Q'. Q \red Q', P' \mathcal{R} Q'$

$P \vdash x \Rightarrow Q \vdash x$

\begin{mathpar}
  \inferrule*[lab=Out-barb]{x \nameeq y}{{y}!\langle{Q}\rangle \vdash x}
  \and
  \inferrule*[lab=Par-barb]{\mbox{$P\vdash x$ or $Q\vdash x$}}{\binpar{P}{Q} \vdash x}
\end{mathpar}

\subsubsection{Contexts}

One of the principle advantages of computational calculi like the
$\pi$-calculus is a well-defined notion of context,
contextual-equivalence and a correlation between
contextual-equivalence and notions of bisimulation. The notion of
context allows the decomposition of a process into (sub-)process and
its syntactic environment, its context. Thus, a context may be
thought of as a process with a ``hole'' (written $\Box$) in it. The
application of a context $M$ to a process $P$, written $M[P]$, is
tantamount to filling the hole in $M$ with $P$. In this paper we do
not need the full weight of this theory, but do make use of the notion
of context in the proof the main theorem. 

\begin{mathpar}
  \inferrule* [lab=summation] {} {{M_{M},M_{N}} \bc \Box \;|\; x.M_{A} \;|\; M_{M}+M_{N}}
  \and
  \inferrule* [lab=agent] {} {{M_{A}} \bc (\vec{x})M_{P} \;| \; \clift{P_0,\ldots,M_{P},\ldots,P_N}}
  \and \\
  \inferrule* [lab=process] {} {{M_{P}} \bc M_{N} \;| \;P|M_{P} }
\end{mathpar} 

\begin{mathpar}
  \inferrule* [lab=sychronization] {} {M_{N} \bc \Box \;|\; x?M_{F} \;|\; x!M_{C}}
  \and
  \inferrule* [lab=abstraction] {} {{M_{F}} \bc (x)M_{P} }
  \and
  \inferrule* [lab=concretion] {} {{M_{C}} \bc \langle M_{P} \rangle }
  \and \\
  \inferrule* [lab=process] {} {{M_{P}} \bc M_{N} \;| \;P|M_{P} }
\end{mathpar}

\begin{definition}[contextual application] Given a context $M$, and
  process $P$, we define the \emph{contextual application}, $M[P] :=
  M\{P/\Box\}$. That is, the contextual application of M to P is the
  substitution of $P$ for $\Box$ in $M$.
\end{definition}

$\meaningof{-} : L \to \mathcal{P}(\pi)$

\begin{mathpar}
  \inferrule* [lab=collection] {} {\meaningof{true} = \pi, \and \meaningof{~E} = \pi \setminus \meaningof{E}, \and \meaningof{E_{1} \& E_{2}} = \meaningof{E_{1}} \cap \meaningof{E_{2}}}
\end{mathpar}

\begin{mathpar}
  \inferrule* [lab=structure] {} {\meaningof{0} = \{ P \in \pi | P \equiv 0 \}, \and \\ \meaningof{E_1 | E_2} = \{ P \in \pi | P \equiv P_{1} | P_{2}, P_{1} \in \meaningof{E_{1}}, P_{2} \in \meaningof{E_2}\} }
\end{mathpar}

\begin{mathpar}
 \inferrule* [lab=behavior] {} {\meaningof{\langle a?b \rangle E} = \{ P \in \pi | P \equiv Q | u?(y)P', \\ \and \\\\ \and \\ \;\;\; u \in \meaningof{a}, \forall z.P'\{z/y\} \in \meaningof{E\{z/b\}}\}, \and \\ \meaningof{a!E} = \{ P \in \pi | P \equiv Q | x!\langle P' \rangle, x \in \meaningof{a} P' \in \meaningof{E}\} }
\end{mathpar}

\begin{mathpar}
 \inferrule* [lab=nominal] {} {\meaningof{\quotep{E}} = \{ \quotep{P} \in \quotep{\pi} | P \in \meaningof{E} \}, \and \meaningof{\quotep{P}} = \{ \quotep{Q} \in \quotep{\pi} | P \equiv Q \} \and \\ \meaningof{@\quotep{E}} = \{ P \in \pi | P \equiv @x, x \in \meaningof{E} \}}
\end{mathpar}

\begin{eqnarray*}
  \\
  \meaningof{-} : TS \to ST
\end{eqnarray*}

\begin{eqnarray*}
  \\
  L : TS \to ST
\end{eqnarray*}

\begin{eqnarray*}
  \\
  P \models E \iff P \in \meaningof{E}
\end{eqnarray*}

\begin{eqnarray*}
  P \approx_{L} Q \iff \forall E \in L. P \models E \iff Q \models E
\end{eqnarray*}

\begin{eqnarray*}
  P \approx_{K} Q
\end{eqnarray*}

\begin{eqnarray*}
  P \approx Q
\end{eqnarray*}

$\approx_{K} = \approx = \approx_{L}$

\subsubsection{Contextual duality}

Note that contexts extend the quotation operation to a family of
operations from processes to names. Given a context, $M$, we can
define a \emph{nominal context}, $\quotep{M}$ by $\quotep{M}[P] :=
\quotep{M[P]}$. To foreshadow what is to come we observe that these
operations enjoy a duality with processes very much like the duality
between vectors and maps from vectors to scalars.

Further, because the calculus is essentially higher-order, we have a
correspondence between contexts and processes. More specifically,
given a name $x$ and a context $M$ we can construct $M^{*}_{x}$ such
that 

\begin{mathpar}
  M^{*}_{x} | \lift{x}{P} \red M[P]
\end{mathpar}

namely,

\begin{mathpar}
  M^{*}_{x} := x?(u).M[\dropn{u}]
\end{mathpar}

The dependence of $M^{*}_{x}$ on a name makes it an abstraction, 

\begin{mathpar}
  M^{*} := (x)x?(u).M[\dropn{u}]
\end{mathpar}

\subsection{Additional notation}

It will sometimes be convenient to denote the process a name
quotes. We already have the notation $x = \quotep{P}$, but it will be
convenient to introduce an alternate notation, $\procn{x}$, when we
want to emphasize the connection to the use of the name. Note that, by
virtue of name equivalence, $\quotep{\procn{x}} \nameeq x$; so, the
notation is consistent with previous definitions.

Further, because names have structure it is possible to effect
substitutions on the basis of that structure. This means we need to
upgrade our notation for substitutions, which we accomplish by
adapting comprehension notation. Thus,

\begin{mathpar}
  P\{ y / x : x \in S \}
\end{mathpar}

is interpreted to mean the process derived from P by replacing (in a
capture-avoiding manner) each occurrence of $x$ in $S$ by $y$. For example,

\begin{mathpar}
  P\{ \quotep{\procn{x}|\procn{x}} / x : x \in \freenames{P} \}
\end{mathpar}

will replace each (occurrence) of a free name $x$ in $P$ by
$\quotep{\procn{x}|\procn{x}}$.

Also, we will avail ourselves of the notation $x^{L}$ and $x^{R}$ to
denote injections of a name into disjoint copies of the name
space. There are numerous ways to accomplish this. One example can be
found in \cite{MeredithR05}. This notation overloads to vectors of
names: $\vec{x}^{\pi} := (x_{i}^{\pi} \; : \; 0 \leq i < |\vec{x}| )$ where $\pi \in \{L,R\}$.

We also use $P^{\Box} := P|\Box$.

In \cite{MeredithR05} an interpretation of the new operator is
given. It turns out that there are several possible interpretations
all enjoying the requisite algebraic properties of the operator (see
\cite{milner91polyadicpi}). We will therefore make liberal use of
$(\nu\; \vec{x})P$.

% subsection the_syntax_and_semantics_of_the_notation_system (end)   

\input{qm2pi.qmops} 

\input{qm2pi.sterngerlach} 

\input{qm2pi.metric} 

% section concurrent_process_calculi (end)

%\input{qm2pi.proofsketch}

% section proof sketch (end)

%\input{qm2pi.slviaknots} 

% section spatial logic via knots (end)

\input{qm2pi.conclusion}

% section conclusion (end)

%\input{qm2pi.dtcodes} 

% section wiring algorithm (end)

\input{qm2pi.ack} 

% section acknowledgments (end)

\newpage


\bibliographystyle{plain}   
\bibliography{../../biblios/main.bib}

\input{qm2pi.rhodetails}

\end{document}

 

% section notation (end)

\input{qm2pi.process.calculi} 

% section concurrent_process_calculi_and_spatial_logics_ (end)
    
%\documentclass[12pt]{llncs}
%\documentclass{jktr}

\usepackage[pdftex]{hyperref}                   
\usepackage {listings}
\usepackage {mathpartir}
\usepackage{bcprules}
%\usepackage{listings}
                       
\usepackage{graphicx} 
%\usepackage[margins=2.5cm,nohead,nofoot]{geometry}
%\usepackage{geometry}
\usepackage{amsfonts}
\usepackage{amstext}
\usepackage{latexsym}
\usepackage{amssymb}
\usepackage{color}


%\include{myPreamble}
\include{qm2pi.local} 

%\ifpdf
%\usepackage[pdftex]{graphicx}
%\else
%\usepackage{graphicx}
%\fi

 % \ifpdf
%  \usepackage{pdfsync}
%  \if


%\title{Brief Article}
%\author{David F. Snyder}
%\author{L.G. Meredith}

%\address{Dept. of Math., Texas State University--San Marcos, San Marcos, TX 78666}
       
\pagestyle{empty}


\begin{document}

\lstset{language=[Objective]Caml,frame=shadowbox}

\input{qm2pi.front}

% section front matter (end)

\input{qm2pi.intro} 
 
% section introduction (end)

% \input{qm2pi.knotations} 

% section notation (end)

\input{qm2pi.process.calculi} 

% section concurrent_process_calculi_and_spatial_logics_ (end)
    
%\input{qm2pi.knots2pi} 

%\input{qm2pi.trefoil} 

%\input{qm2pi.mainthm} 

% subsection basic_interpretation (end)

%\input{qm2pi.rho.presentation} 
\subsection{The syntax and semantics of the notation system}\label{sub:the_syntax_and_semantics_of_the_notation_system} % (fold)

We now summarize a technical presentation of the calculus that
embodies our theory of dynamics. The typical presentation of such a
calculus follows the style of giving generators and relations on
them. The grammar, below, describing term constructors, freely
generates the set of processes, $\Proc$. This set is then quotiented
by a relation known as structural congruence and it is over this set
that the notion of dynamics is expressed. This presentation is
essentially that of \cite{MeredithR05} with the addition of
polyadicity and summation. For readability we have relegated some of
the technical subtleties to an appendix.

\subsubsection{Process grammar}\label{subsub:process_grammar}

\begin{mathpar}
  \inferrule* [lab=synchronization] {} {{M} \bc \pzero \;|\; x?F \;|\; x!C }
  \and
  \inferrule* [lab=abstraction] {} {{F} \bc (x)P}
  \and
  \inferrule* [lab=concretion] {} {{C} \bc \langle Q \rangle}
  \and
  \inferrule* [lab=process] {} {{P,Q} \bc M \;| \;P|Q \;|\; @{x}}
  \and
  \inferrule* [lab=name] {} {{x} \bc \quotep{P}}
\end{mathpar} 

Note that $\vec{x}$ (resp. $\vec{P}$) denotes a vector of names
(resp. processes) of length $|\vec{x}|$ (resp. $|\vec{P}|$). We adopt
the following useful abbreviations.

\begin{mathpar}
   x?(\vec{y}).P := x.(\vec{y})P \and  x\clift{\vec{P}} := x.\clift{\vec{P}}
   \and x!(y) := \lift{x}{\dropn{y}}
   \and \Pi_{i=0}^{n-1}P_i := P_0 | \ldots | P_{n-1}
\end{mathpar}

\subsubsection{Structural congruence}

\paragraph{Free and bound names and alpha-equivalence.} At the
core of structural equivalence is alpha-equivalence which identifies
process that are the same up to a change of variable. Formally, we
recognize the distinction between free and bound names. The free names
of a process, $\freenames{P}$, may be calculated recursively as
follows:

\begin{mathpar}
\freenames{\pzero} := \emptyset
  \and \\
  \freenames{x?(y).P} := \{ x \} \cup (\freenames{P} \setminus \{ y \})
  \and 
  \freenames{x!\langle P \rangle} := \{ x \} \cup \{ P \} 
  \and \\
  \freenames{P|Q} := \freenames{P} \cup \freenames{Q}
  \and \\
  \freenames{@{x}} := \{ x \}
\end{mathpar}

$\pi$
$\quotep{\pi}$

$\freenames{-} : \pi \to \mathcal{P}(\quotep{\pi})$

\begin{eqnarray*}
  \freenames{\pzero} & := & \emptyset \\
  \freenames{x?(y).P} & := & \{ x \} \cup (\freenames{P} \setminus \{ y \}) \\
  \freenames{x!\langle P \rangle} & := & \{ x \} \cup \{ P \} \\
  \freenames{P|Q} & := & \freenames{P} \cup \freenames{Q} \\
  \freenames{\dropn{x}} & := & \{ x \}
\end{eqnarray*}

The bound names of a process, $\boundnames{P}$, are those names occurring in $P$
that are not free. For example, in $x?(y).0$, the name $x$ is free, while $y$ is bound.

\begin{mathpar}
  \inferrule* [lab=monoidal-laws] {} { P|Q \equiv Q|P \and P|0 \equiv P \and P|(Q|R) \equiv (P|Q)|R }
\end{mathpar}

\begin{mathpar}
  \inferrule* [lab=alpha-equivalence] {} { (x)P \equiv (y)P\{y/x\} \and y \not\in \freenames{P} }
\end{mathpar}

\begin{definition}
Then two processes, $P,Q$, are alpha-equivalent if $P = Q\{\vec{y}/\vec{x}\}$ for
some $\vec{x} \in \boundnames{Q},\vec{y} \in \boundnames{P}$, where $Q\{\vec{y}/\vec{x}\}$
denotes the capture-avoiding substitution of $\vec{y}$ for $\vec{x}$ in $Q$.
\end{definition}

\begin{definition}
  The {\em structural congruence} \cite{SangiorgiWalker} , $\equiv$,
  between processes is the least congruence containing
  alpha-equivalence, satisfying the abelian monoid laws
  (associativity, commutativity and $\pzero$ as identity) for parallel
  composition $|$ and for summation $+$.
\end{definition}

\subsection{Name equivalence}

We take name equivalence, written $\nameeq$, to be the smallest
equivalence relation generated by the following rules.

\begin{mathpar}
\inferrule*[lab=Quote-drop]
{ }
{ \quotep{@{x}} \nameeq x }

\inferrule*[lab=Struct-equiv]
{ P \scong Q }
{ \quotep{P} \nameeq \quotep{Q} }
\end{mathpar}

The astute reader will have noticed that the mutual recursion of names
and processes imposes a mutual recursion on alpha-equivalence and
structural equivalence via name-equivalence. Fortunately, all of this
works out pleasantly and we may calculate in the natural way, free of
concern. The reader interested in the details is referred to the
appendix \ref{appendix:rho_details}.

\subsection{Substitution}

We use $\Proc$ for the set of processes, $\QProc$ for the set of
names, and $\id{\{}\vec{y} / \vec{x} \id{\}}$ to denote partial maps,
$s : \QProc \rightarrow \QProc$. A map, $s$ lifts, uniquely, to a map
on process terms, $\widehat{s} : \Proc \rightarrow \Proc$ by the
following equations.

\begin{mathpar}
  (0) \psubstp{Q}{P} := 0 \\
  (R \juxtap S) \psubstp{Q}{P}
  :=    
  (R)\psubstp{Q}{P} \juxtap (S) \psubstp{Q}{P} \\
  (x?(y).R) \psubstp{Q}{P}    
  :=    
  (x)\substp{Q}{P} (z)\concat( (R \psubstn{z}{y}) \psubstp{Q}{P} ) \\
  (\lift{x}{R}) \psubstp{Q}{P}  
  :=
  \lift{(x)\substp{Q}{P}}{ R \psubstp{Q}{P} } \\
%   (\dropn{x})  \psubstp{Q}{P}       
%   := 
%   \left\{ 
%     \begin{array}{ccc} 
%       \dropn{\quotep{Q}} & & x \nameeq \quotep{P} \\
%       \dropn{x} & & otherwise \\
%     \end{array}
%   \right. 
  (\dropn{x})  \psubstp{Q}{P}       
  := 
  \left\{ 
    \begin{array}{ccc} 
      Q & & x \nameeq \quotep{P} \\
      \dropn{x} & & otherwise \\
    \end{array}
  \right.
\end{mathpar}
 

where

\begin{eqnarray}
  (x)\id{\{} \lpquote Q \rpquote / \lpquote P \rpquote \id{\}}            = 
  \left\{ 
    \begin{array}{ccc}
      \lpquote Q \rpquote & & x \nameeq \lpquote P \rpquote \\
      x & & otherwise \\
    \end{array}
  \right. \nonumber
\end{eqnarray}

and $z$ is chosen distinct from $\quotep{P}$, $\quotep{Q}$, the free
names in $Q$, and all the names in $R$. Our $\alpha$-equivalence will
be built in the standard way from this substitution.

\begin{remark}\label{rem:no_self_referential_names}
  One consequence of these definitions is that $\forall P. \quotep{P}
  \not\in \freenames{P}$.
\end{remark}

\subsection{ Dynamic quote: an example }

Anticipating something of what's to come, consider applying the
substitution, $\widehat{\id{\{}u / z \id{\}}}$, to the following pair
of processes, $\lift{w}{y!(z)}$ and $w[ \lpquote y!(z) \rpquote ]$.

\begin{eqnarray}
	\lift{w}{y!(z)}\widehat{\id{\{}u / z \id{\}}}
		& = &
		\lift{w}{y!(u)} \nonumber\\
	w[ \lpquote y!(z) \rpquote ] \widehat{ \id{\{}u / z \id{\}} }
		& = &
		w[ \lpquote y!(z) \rpquote ] \nonumber
\end{eqnarray}

Because the body of the process between quotes is impervious to
substitution, we get radically different answers. In fact, by
examining the first process in an input context,
e.g. $x?(z).\lift{w}{y!(z)}$, we see that the process under the lift
operator may be shaped by prefixed inputs binding a name inside it. In
this sense, the lift operator will be seen as a way to dynamically
construct processes before reifying them as names.

Finally equipped with these standard features we can present the
dynamics of the calculus.

\subsubsection{Operational semantics} 

Finally, we introduce the computational dynamics. What marks these
algebras as distinct from other more traditionally studied algebraic
structures, e.g. vector spaces or polynomial rings, is the manner in
which dynamics is captured. In traditional structures, dynamics is typically
expressed through morphisms between such structures, as in linear maps
between vector spaces or morphisms between rings. In algebras
associated with the semantics of computation, the dynamics is
expressed as part of the algebraic structure itself, through a
reduction reduction relation typically denoted by $\red$. Below, we
give a recursive presentation of this relation for the calculus used
in the encoding.

$\red \subseteq \pi \times \pi$
$\red : \pi \to \mathcal{P}(\pi)$

\begin{mathpar}
  \inferrule* [lab=Comm] { \textsf{match}( x_{src}, x_{trgt} ) } { x_{trgt}?(y)P \; | \; x_{src}!\langle {Q} \rangle \red P\{\quotep{Q}/y}\} }
  \and \\
  \inferrule* [lab=Par] {{P} \red {P}'} {{{P} | {Q}} \red {{P}' | {Q}}}
  \and
  \inferrule* [lab=Equiv]{{{P} \scong {P}'} \andalso {{P}' \red {Q}'} \andalso {{Q}' \scong {Q}}}{{P} \red {Q}}
\end{mathpar}

\begin{eqnarray*}
  match_{\equiv} (\quotep{P},\quotep{Q}) & := & P \equiv Q \\
  match_{\dagger}(\quotep{P},\quotep{Q}) & := & \forall R. P|Q \red^{*} R => R \red^{*} 0 \\
  match_{K}(\quotep{P},\quotep{Q}) & := & K \mbox{ for some context } K
\end{eqnarray*}

$u?(x)P | u!\langle Q \rangle \red P\{\quotep{Q}/x\}$

%We write $\wred$ for $\red^*$, and $P\red$ if $\exists Q $ such that $ P \red Q$.
We write $P\red$ if $\exists Q $ such that $ P \red Q$ and $P\not\red$, otherwise.

\section{Replication}

As mentioned before, it is known that replication (and hence
recursion) can be implemented in a higher-order process algebra
\cite{SangiorgiWalker}. As our first example of calculation with the
machinery thus far presented we give the construction explicitly in
the {\rhoc}.

\begin{eqnarray}
	D_{x} & := & \prefix{x}{y}{(\binpar{\outputp{x}{y}}{@{y}})} \nonumber\\
	\bangp_{x}{P} & := & \binpar{{x}!\langle{\binpar{D_{x}}{P}}\rangle}{D_{x}} \nonumber
\end{eqnarray}

\begin{eqnarray}
	\bangp_{x}{P} & & \nonumber\\
	=
	& {x}!\langle{(\prefix{x}{y}{(\outputp{x}{y} | @{y})) | P}}\rangle 
	      | \prefix{x}{y}{(\outputp{x}{y} | @{y})} & \nonumber\\
	\red
	& (\outputp{x}{y} | @{y})\substn{\quotep{(\prefix{x}{y}{(@{y} | \outputp{x}{y})) | P}}}{y} & \nonumber\\
	=
	& \outputp{x}{\quotep{(\prefix{x}{y}{(\outputp{x}{y} | @{y})) | P}}}
	  | {(\prefix{x}{y}{(\outputp{x}{y} | @{y})) | P}} & \nonumber\\
	\red
	& \ldots & \nonumber\\
	\red^*
	& P | P | \ldots & \nonumber
\end{eqnarray}

Of course, this encoding, as an implementation, runs away, unfolding
$\bangp{P}$ eagerly. A lazier and more implementable replication
operator, restricted to input-guarded processes, may be obtained as follows.

\begin{eqnarray}
\bangp{\prefix{u}{v}{P}} 
	:= 
	\binpar{\lift{x}{\prefix{u}{v}{(\binpar{D(x)}{P})}}}{D(x)} \nonumber
\end{eqnarray}

\begin{remark}
  Note that the lazier definition still does not deal with summation
  or mixed summation (i.e. sums over input and output). The reader is
  invited to construct definitions of replication that deal with these
  features. 

  Further, the definitions are parameterized in a name, $x$. Can you,
  gentle reader, make a definition that eliminates this parameter and
  guarantees no accidental interaction between the replication
  machinery and the process being replicated -- i.e. no accidental
  sharing of names used by the process to get its work done and the
  name(s) used by the replication to effect copying. This latter
  revision of the definition of replication is crucial to obtaining
  the expected identity $!!P \sim !P$.
\end{remark}

\begin{remark}\label{rem:paradoxical_combinator}
  The reader familiar with the lambda calculus will have noticed the
  similarity between $D$ and the paradoxical combinator.

  [Ed. note: the existence of this seems to suggest we have to be more
  restrictive on the set of processes and names we admit if we are to
  support no-cloning.]
\end{remark}

\subsubsection{Bisimulation}

The computational dynamics gives rise to another kind of equivalence,
the equivalence of computational behavior. As previously mentioned
this is typically captured \emph{via} some form of bisimulation.

% The notion we use in this paper is weak barbed bisimulation
% \cite{milner91polyadicpi}.

The notion we use in this paper is derived from weak barbed
bisimulation \cite{milner91polyadicpi}. 

\begin{definition}
An \emph{observation relation}, $\downarrow_{\mathcal N}$, over a set
of names, $\mathcal N$, is the smallest relation satisfying the rules
below.

\infrule[Out-barb]{y \in {\mathcal N}, \; x \nameeq y}
		  {\outputp{x}{v} \downarrow_{\mathcal N} x}
\infrule[Par-barb]{\mbox{$P\downarrow_{\mathcal N} x$ or $Q\downarrow_{\mathcal N} x$}}
		  {\binpar{P}{Q} \downarrow_{\mathcal N} x}

We write $P \Downarrow_{\mathcal N} x$ if there is $Q$ such that 
$P \wred Q$ and $Q \downarrow_{\mathcal N} x$.
\end{definition}

\begin{definition}
%\label{def.bbisim}
An  ${\mathcal N}$-\emph{barbed bisimulation} over a set of names, ${\mathcal N}$, is a symmetric binary relation 
${\mathcal S}_{\mathcal N}$ between agents such that $P\rel{S}_{\mathcal N}Q$ implies:
\begin{enumerate}
\item If $P \red P'$ then $Q \wred Q'$ and $P'\rel{S}_{\mathcal N} Q'$.
\item If $P\downarrow_{\mathcal N} x$, then $Q\Downarrow_{\mathcal N} x$.
\end{enumerate}
$P$ is ${\mathcal N}$-barbed bisimilar to $Q$, written
$P \wbbisim_{\mathcal N} Q$, if $P \rel{S}_{\mathcal N} Q$ for some ${\mathcal N}$-barbed bisimulation ${\mathcal S}_{\mathcal N}$.
\end{definition}

$\mathcal{R} \subseteq \pi \times \pi$

$P \mathcal{R} Q => \forall P'. P \red P' \Rightarrow \exists Q'. Q \red Q', P' \mathcal{R} Q'$

$P \vdash x \Rightarrow Q \vdash x$

\begin{mathpar}
  \inferrule*[lab=Out-barb]{x \nameeq y}{{y}!\langle{Q}\rangle \vdash x}
  \and
  \inferrule*[lab=Par-barb]{\mbox{$P\vdash x$ or $Q\vdash x$}}{\binpar{P}{Q} \vdash x}
\end{mathpar}

\subsubsection{Contexts}

One of the principle advantages of computational calculi like the
$\pi$-calculus is a well-defined notion of context,
contextual-equivalence and a correlation between
contextual-equivalence and notions of bisimulation. The notion of
context allows the decomposition of a process into (sub-)process and
its syntactic environment, its context. Thus, a context may be
thought of as a process with a ``hole'' (written $\Box$) in it. The
application of a context $M$ to a process $P$, written $M[P]$, is
tantamount to filling the hole in $M$ with $P$. In this paper we do
not need the full weight of this theory, but do make use of the notion
of context in the proof the main theorem. 

\begin{mathpar}
  \inferrule* [lab=summation] {} {{M_{M},M_{N}} \bc \Box \;|\; x.M_{A} \;|\; M_{M}+M_{N}}
  \and
  \inferrule* [lab=agent] {} {{M_{A}} \bc (\vec{x})M_{P} \;| \; \clift{P_0,\ldots,M_{P},\ldots,P_N}}
  \and \\
  \inferrule* [lab=process] {} {{M_{P}} \bc M_{N} \;| \;P|M_{P} }
\end{mathpar} 

\begin{mathpar}
  \inferrule* [lab=sychronization] {} {M_{N} \bc \Box \;|\; x?M_{F} \;|\; x!M_{C}}
  \and
  \inferrule* [lab=abstraction] {} {{M_{F}} \bc (x)M_{P} }
  \and
  \inferrule* [lab=concretion] {} {{M_{C}} \bc \langle M_{P} \rangle }
  \and \\
  \inferrule* [lab=process] {} {{M_{P}} \bc M_{N} \;| \;P|M_{P} }
\end{mathpar}

\begin{definition}[contextual application] Given a context $M$, and
  process $P$, we define the \emph{contextual application}, $M[P] :=
  M\{P/\Box\}$. That is, the contextual application of M to P is the
  substitution of $P$ for $\Box$ in $M$.
\end{definition}

$\meaningof{-} : L \to \mathcal{P}(\pi)$

\begin{mathpar}
  \inferrule* [lab=collection] {} {\meaningof{true} = \pi, \and \meaningof{~E} = \pi \setminus \meaningof{E}, \and \meaningof{E_{1} \& E_{2}} = \meaningof{E_{1}} \cap \meaningof{E_{2}}}
\end{mathpar}

\begin{mathpar}
  \inferrule* [lab=structure] {} {\meaningof{0} = \{ P \in \pi | P \equiv 0 \}, \and \\ \meaningof{E_1 | E_2} = \{ P \in \pi | P \equiv P_{1} | P_{2}, P_{1} \in \meaningof{E_{1}}, P_{2} \in \meaningof{E_2}\} }
\end{mathpar}

\begin{mathpar}
 \inferrule* [lab=behavior] {} {\meaningof{\langle a?b \rangle E} = \{ P \in \pi | P \equiv Q | u?(y)P', \\ \and \\\\ \and \\ \;\;\; u \in \meaningof{a}, \forall z.P'\{z/y\} \in \meaningof{E\{z/b\}}\}, \and \\ \meaningof{a!E} = \{ P \in \pi | P \equiv Q | x!\langle P' \rangle, x \in \meaningof{a} P' \in \meaningof{E}\} }
\end{mathpar}

\begin{mathpar}
 \inferrule* [lab=nominal] {} {\meaningof{\quotep{E}} = \{ \quotep{P} \in \quotep{\pi} | P \in \meaningof{E} \}, \and \meaningof{\quotep{P}} = \{ \quotep{Q} \in \quotep{\pi} | P \equiv Q \} \and \\ \meaningof{@\quotep{E}} = \{ P \in \pi | P \equiv @x, x \in \meaningof{E} \}}
\end{mathpar}

\begin{eqnarray*}
  \\
  \meaningof{-} : TS \to ST
\end{eqnarray*}

\begin{eqnarray*}
  \\
  L : TS \to ST
\end{eqnarray*}

\begin{eqnarray*}
  \\
  P \models E \iff P \in \meaningof{E}
\end{eqnarray*}

\begin{eqnarray*}
  P \approx_{L} Q \iff \forall E \in L. P \models E \iff Q \models E
\end{eqnarray*}

\begin{eqnarray*}
  P \approx_{K} Q
\end{eqnarray*}

\begin{eqnarray*}
  P \approx Q
\end{eqnarray*}

$\approx_{K} = \approx = \approx_{L}$

\subsubsection{Contextual duality}

Note that contexts extend the quotation operation to a family of
operations from processes to names. Given a context, $M$, we can
define a \emph{nominal context}, $\quotep{M}$ by $\quotep{M}[P] :=
\quotep{M[P]}$. To foreshadow what is to come we observe that these
operations enjoy a duality with processes very much like the duality
between vectors and maps from vectors to scalars.

Further, because the calculus is essentially higher-order, we have a
correspondence between contexts and processes. More specifically,
given a name $x$ and a context $M$ we can construct $M^{*}_{x}$ such
that 

\begin{mathpar}
  M^{*}_{x} | \lift{x}{P} \red M[P]
\end{mathpar}

namely,

\begin{mathpar}
  M^{*}_{x} := x?(u).M[\dropn{u}]
\end{mathpar}

The dependence of $M^{*}_{x}$ on a name makes it an abstraction, 

\begin{mathpar}
  M^{*} := (x)x?(u).M[\dropn{u}]
\end{mathpar}

\subsection{Additional notation}

It will sometimes be convenient to denote the process a name
quotes. We already have the notation $x = \quotep{P}$, but it will be
convenient to introduce an alternate notation, $\procn{x}$, when we
want to emphasize the connection to the use of the name. Note that, by
virtue of name equivalence, $\quotep{\procn{x}} \nameeq x$; so, the
notation is consistent with previous definitions.

Further, because names have structure it is possible to effect
substitutions on the basis of that structure. This means we need to
upgrade our notation for substitutions, which we accomplish by
adapting comprehension notation. Thus,

\begin{mathpar}
  P\{ y / x : x \in S \}
\end{mathpar}

is interpreted to mean the process derived from P by replacing (in a
capture-avoiding manner) each occurrence of $x$ in $S$ by $y$. For example,

\begin{mathpar}
  P\{ \quotep{\procn{x}|\procn{x}} / x : x \in \freenames{P} \}
\end{mathpar}

will replace each (occurrence) of a free name $x$ in $P$ by
$\quotep{\procn{x}|\procn{x}}$.

Also, we will avail ourselves of the notation $x^{L}$ and $x^{R}$ to
denote injections of a name into disjoint copies of the name
space. There are numerous ways to accomplish this. One example can be
found in \cite{MeredithR05}. This notation overloads to vectors of
names: $\vec{x}^{\pi} := (x_{i}^{\pi} \; : \; 0 \leq i < |\vec{x}| )$ where $\pi \in \{L,R\}$.

We also use $P^{\Box} := P|\Box$.

In \cite{MeredithR05} an interpretation of the new operator is
given. It turns out that there are several possible interpretations
all enjoying the requisite algebraic properties of the operator (see
\cite{milner91polyadicpi}). We will therefore make liberal use of
$(\nu\; \vec{x})P$.

% subsection the_syntax_and_semantics_of_the_notation_system (end)   

\input{qm2pi.qmops} 

\input{qm2pi.sterngerlach} 

\input{qm2pi.metric} 

% section concurrent_process_calculi (end)

%\input{qm2pi.proofsketch}

% section proof sketch (end)

%\input{qm2pi.slviaknots} 

% section spatial logic via knots (end)

\input{qm2pi.conclusion}

% section conclusion (end)

%\input{qm2pi.dtcodes} 

% section wiring algorithm (end)

\input{qm2pi.ack} 

% section acknowledgments (end)

\newpage


\bibliographystyle{plain}   
\bibliography{../../biblios/main.bib}

\input{qm2pi.rhodetails}

\end{document}

 

%\documentclass[12pt]{llncs}
%\documentclass{jktr}

\usepackage[pdftex]{hyperref}                   
\usepackage {listings}
\usepackage {mathpartir}
\usepackage{bcprules}
%\usepackage{listings}
                       
\usepackage{graphicx} 
%\usepackage[margins=2.5cm,nohead,nofoot]{geometry}
%\usepackage{geometry}
\usepackage{amsfonts}
\usepackage{amstext}
\usepackage{latexsym}
\usepackage{amssymb}
\usepackage{color}


%\include{myPreamble}
\include{qm2pi.local} 

%\ifpdf
%\usepackage[pdftex]{graphicx}
%\else
%\usepackage{graphicx}
%\fi

 % \ifpdf
%  \usepackage{pdfsync}
%  \if


%\title{Brief Article}
%\author{David F. Snyder}
%\author{L.G. Meredith}

%\address{Dept. of Math., Texas State University--San Marcos, San Marcos, TX 78666}
       
\pagestyle{empty}


\begin{document}

\lstset{language=[Objective]Caml,frame=shadowbox}

\input{qm2pi.front}

% section front matter (end)

\input{qm2pi.intro} 
 
% section introduction (end)

% \input{qm2pi.knotations} 

% section notation (end)

\input{qm2pi.process.calculi} 

% section concurrent_process_calculi_and_spatial_logics_ (end)
    
%\input{qm2pi.knots2pi} 

%\input{qm2pi.trefoil} 

%\input{qm2pi.mainthm} 

% subsection basic_interpretation (end)

%\input{qm2pi.rho.presentation} 
\subsection{The syntax and semantics of the notation system}\label{sub:the_syntax_and_semantics_of_the_notation_system} % (fold)

We now summarize a technical presentation of the calculus that
embodies our theory of dynamics. The typical presentation of such a
calculus follows the style of giving generators and relations on
them. The grammar, below, describing term constructors, freely
generates the set of processes, $\Proc$. This set is then quotiented
by a relation known as structural congruence and it is over this set
that the notion of dynamics is expressed. This presentation is
essentially that of \cite{MeredithR05} with the addition of
polyadicity and summation. For readability we have relegated some of
the technical subtleties to an appendix.

\subsubsection{Process grammar}\label{subsub:process_grammar}

\begin{mathpar}
  \inferrule* [lab=synchronization] {} {{M} \bc \pzero \;|\; x?F \;|\; x!C }
  \and
  \inferrule* [lab=abstraction] {} {{F} \bc (x)P}
  \and
  \inferrule* [lab=concretion] {} {{C} \bc \langle Q \rangle}
  \and
  \inferrule* [lab=process] {} {{P,Q} \bc M \;| \;P|Q \;|\; @{x}}
  \and
  \inferrule* [lab=name] {} {{x} \bc \quotep{P}}
\end{mathpar} 

Note that $\vec{x}$ (resp. $\vec{P}$) denotes a vector of names
(resp. processes) of length $|\vec{x}|$ (resp. $|\vec{P}|$). We adopt
the following useful abbreviations.

\begin{mathpar}
   x?(\vec{y}).P := x.(\vec{y})P \and  x\clift{\vec{P}} := x.\clift{\vec{P}}
   \and x!(y) := \lift{x}{\dropn{y}}
   \and \Pi_{i=0}^{n-1}P_i := P_0 | \ldots | P_{n-1}
\end{mathpar}

\subsubsection{Structural congruence}

\paragraph{Free and bound names and alpha-equivalence.} At the
core of structural equivalence is alpha-equivalence which identifies
process that are the same up to a change of variable. Formally, we
recognize the distinction between free and bound names. The free names
of a process, $\freenames{P}$, may be calculated recursively as
follows:

\begin{mathpar}
\freenames{\pzero} := \emptyset
  \and \\
  \freenames{x?(y).P} := \{ x \} \cup (\freenames{P} \setminus \{ y \})
  \and 
  \freenames{x!\langle P \rangle} := \{ x \} \cup \{ P \} 
  \and \\
  \freenames{P|Q} := \freenames{P} \cup \freenames{Q}
  \and \\
  \freenames{@{x}} := \{ x \}
\end{mathpar}

$\pi$
$\quotep{\pi}$

$\freenames{-} : \pi \to \mathcal{P}(\quotep{\pi})$

\begin{eqnarray*}
  \freenames{\pzero} & := & \emptyset \\
  \freenames{x?(y).P} & := & \{ x \} \cup (\freenames{P} \setminus \{ y \}) \\
  \freenames{x!\langle P \rangle} & := & \{ x \} \cup \{ P \} \\
  \freenames{P|Q} & := & \freenames{P} \cup \freenames{Q} \\
  \freenames{\dropn{x}} & := & \{ x \}
\end{eqnarray*}

The bound names of a process, $\boundnames{P}$, are those names occurring in $P$
that are not free. For example, in $x?(y).0$, the name $x$ is free, while $y$ is bound.

\begin{mathpar}
  \inferrule* [lab=monoidal-laws] {} { P|Q \equiv Q|P \and P|0 \equiv P \and P|(Q|R) \equiv (P|Q)|R }
\end{mathpar}

\begin{mathpar}
  \inferrule* [lab=alpha-equivalence] {} { (x)P \equiv (y)P\{y/x\} \and y \not\in \freenames{P} }
\end{mathpar}

\begin{definition}
Then two processes, $P,Q$, are alpha-equivalent if $P = Q\{\vec{y}/\vec{x}\}$ for
some $\vec{x} \in \boundnames{Q},\vec{y} \in \boundnames{P}$, where $Q\{\vec{y}/\vec{x}\}$
denotes the capture-avoiding substitution of $\vec{y}$ for $\vec{x}$ in $Q$.
\end{definition}

\begin{definition}
  The {\em structural congruence} \cite{SangiorgiWalker} , $\equiv$,
  between processes is the least congruence containing
  alpha-equivalence, satisfying the abelian monoid laws
  (associativity, commutativity and $\pzero$ as identity) for parallel
  composition $|$ and for summation $+$.
\end{definition}

\subsection{Name equivalence}

We take name equivalence, written $\nameeq$, to be the smallest
equivalence relation generated by the following rules.

\begin{mathpar}
\inferrule*[lab=Quote-drop]
{ }
{ \quotep{@{x}} \nameeq x }

\inferrule*[lab=Struct-equiv]
{ P \scong Q }
{ \quotep{P} \nameeq \quotep{Q} }
\end{mathpar}

The astute reader will have noticed that the mutual recursion of names
and processes imposes a mutual recursion on alpha-equivalence and
structural equivalence via name-equivalence. Fortunately, all of this
works out pleasantly and we may calculate in the natural way, free of
concern. The reader interested in the details is referred to the
appendix \ref{appendix:rho_details}.

\subsection{Substitution}

We use $\Proc$ for the set of processes, $\QProc$ for the set of
names, and $\id{\{}\vec{y} / \vec{x} \id{\}}$ to denote partial maps,
$s : \QProc \rightarrow \QProc$. A map, $s$ lifts, uniquely, to a map
on process terms, $\widehat{s} : \Proc \rightarrow \Proc$ by the
following equations.

\begin{mathpar}
  (0) \psubstp{Q}{P} := 0 \\
  (R \juxtap S) \psubstp{Q}{P}
  :=    
  (R)\psubstp{Q}{P} \juxtap (S) \psubstp{Q}{P} \\
  (x?(y).R) \psubstp{Q}{P}    
  :=    
  (x)\substp{Q}{P} (z)\concat( (R \psubstn{z}{y}) \psubstp{Q}{P} ) \\
  (\lift{x}{R}) \psubstp{Q}{P}  
  :=
  \lift{(x)\substp{Q}{P}}{ R \psubstp{Q}{P} } \\
%   (\dropn{x})  \psubstp{Q}{P}       
%   := 
%   \left\{ 
%     \begin{array}{ccc} 
%       \dropn{\quotep{Q}} & & x \nameeq \quotep{P} \\
%       \dropn{x} & & otherwise \\
%     \end{array}
%   \right. 
  (\dropn{x})  \psubstp{Q}{P}       
  := 
  \left\{ 
    \begin{array}{ccc} 
      Q & & x \nameeq \quotep{P} \\
      \dropn{x} & & otherwise \\
    \end{array}
  \right.
\end{mathpar}
 

where

\begin{eqnarray}
  (x)\id{\{} \lpquote Q \rpquote / \lpquote P \rpquote \id{\}}            = 
  \left\{ 
    \begin{array}{ccc}
      \lpquote Q \rpquote & & x \nameeq \lpquote P \rpquote \\
      x & & otherwise \\
    \end{array}
  \right. \nonumber
\end{eqnarray}

and $z$ is chosen distinct from $\quotep{P}$, $\quotep{Q}$, the free
names in $Q$, and all the names in $R$. Our $\alpha$-equivalence will
be built in the standard way from this substitution.

\begin{remark}\label{rem:no_self_referential_names}
  One consequence of these definitions is that $\forall P. \quotep{P}
  \not\in \freenames{P}$.
\end{remark}

\subsection{ Dynamic quote: an example }

Anticipating something of what's to come, consider applying the
substitution, $\widehat{\id{\{}u / z \id{\}}}$, to the following pair
of processes, $\lift{w}{y!(z)}$ and $w[ \lpquote y!(z) \rpquote ]$.

\begin{eqnarray}
	\lift{w}{y!(z)}\widehat{\id{\{}u / z \id{\}}}
		& = &
		\lift{w}{y!(u)} \nonumber\\
	w[ \lpquote y!(z) \rpquote ] \widehat{ \id{\{}u / z \id{\}} }
		& = &
		w[ \lpquote y!(z) \rpquote ] \nonumber
\end{eqnarray}

Because the body of the process between quotes is impervious to
substitution, we get radically different answers. In fact, by
examining the first process in an input context,
e.g. $x?(z).\lift{w}{y!(z)}$, we see that the process under the lift
operator may be shaped by prefixed inputs binding a name inside it. In
this sense, the lift operator will be seen as a way to dynamically
construct processes before reifying them as names.

Finally equipped with these standard features we can present the
dynamics of the calculus.

\subsubsection{Operational semantics} 

Finally, we introduce the computational dynamics. What marks these
algebras as distinct from other more traditionally studied algebraic
structures, e.g. vector spaces or polynomial rings, is the manner in
which dynamics is captured. In traditional structures, dynamics is typically
expressed through morphisms between such structures, as in linear maps
between vector spaces or morphisms between rings. In algebras
associated with the semantics of computation, the dynamics is
expressed as part of the algebraic structure itself, through a
reduction reduction relation typically denoted by $\red$. Below, we
give a recursive presentation of this relation for the calculus used
in the encoding.

$\red \subseteq \pi \times \pi$
$\red : \pi \to \mathcal{P}(\pi)$

\begin{mathpar}
  \inferrule* [lab=Comm] { \textsf{match}( x_{src}, x_{trgt} ) } { x_{trgt}?(y)P \; | \; x_{src}!\langle {Q} \rangle \red P\{\quotep{Q}/y}\} }
  \and \\
  \inferrule* [lab=Par] {{P} \red {P}'} {{{P} | {Q}} \red {{P}' | {Q}}}
  \and
  \inferrule* [lab=Equiv]{{{P} \scong {P}'} \andalso {{P}' \red {Q}'} \andalso {{Q}' \scong {Q}}}{{P} \red {Q}}
\end{mathpar}

\begin{eqnarray*}
  match_{\equiv} (\quotep{P},\quotep{Q}) & := & P \equiv Q \\
  match_{\dagger}(\quotep{P},\quotep{Q}) & := & \forall R. P|Q \red^{*} R => R \red^{*} 0 \\
  match_{K}(\quotep{P},\quotep{Q}) & := & K \mbox{ for some context } K
\end{eqnarray*}

$u?(x)P | u!\langle Q \rangle \red P\{\quotep{Q}/x\}$

%We write $\wred$ for $\red^*$, and $P\red$ if $\exists Q $ such that $ P \red Q$.
We write $P\red$ if $\exists Q $ such that $ P \red Q$ and $P\not\red$, otherwise.

\section{Replication}

As mentioned before, it is known that replication (and hence
recursion) can be implemented in a higher-order process algebra
\cite{SangiorgiWalker}. As our first example of calculation with the
machinery thus far presented we give the construction explicitly in
the {\rhoc}.

\begin{eqnarray}
	D_{x} & := & \prefix{x}{y}{(\binpar{\outputp{x}{y}}{@{y}})} \nonumber\\
	\bangp_{x}{P} & := & \binpar{{x}!\langle{\binpar{D_{x}}{P}}\rangle}{D_{x}} \nonumber
\end{eqnarray}

\begin{eqnarray}
	\bangp_{x}{P} & & \nonumber\\
	=
	& {x}!\langle{(\prefix{x}{y}{(\outputp{x}{y} | @{y})) | P}}\rangle 
	      | \prefix{x}{y}{(\outputp{x}{y} | @{y})} & \nonumber\\
	\red
	& (\outputp{x}{y} | @{y})\substn{\quotep{(\prefix{x}{y}{(@{y} | \outputp{x}{y})) | P}}}{y} & \nonumber\\
	=
	& \outputp{x}{\quotep{(\prefix{x}{y}{(\outputp{x}{y} | @{y})) | P}}}
	  | {(\prefix{x}{y}{(\outputp{x}{y} | @{y})) | P}} & \nonumber\\
	\red
	& \ldots & \nonumber\\
	\red^*
	& P | P | \ldots & \nonumber
\end{eqnarray}

Of course, this encoding, as an implementation, runs away, unfolding
$\bangp{P}$ eagerly. A lazier and more implementable replication
operator, restricted to input-guarded processes, may be obtained as follows.

\begin{eqnarray}
\bangp{\prefix{u}{v}{P}} 
	:= 
	\binpar{\lift{x}{\prefix{u}{v}{(\binpar{D(x)}{P})}}}{D(x)} \nonumber
\end{eqnarray}

\begin{remark}
  Note that the lazier definition still does not deal with summation
  or mixed summation (i.e. sums over input and output). The reader is
  invited to construct definitions of replication that deal with these
  features. 

  Further, the definitions are parameterized in a name, $x$. Can you,
  gentle reader, make a definition that eliminates this parameter and
  guarantees no accidental interaction between the replication
  machinery and the process being replicated -- i.e. no accidental
  sharing of names used by the process to get its work done and the
  name(s) used by the replication to effect copying. This latter
  revision of the definition of replication is crucial to obtaining
  the expected identity $!!P \sim !P$.
\end{remark}

\begin{remark}\label{rem:paradoxical_combinator}
  The reader familiar with the lambda calculus will have noticed the
  similarity between $D$ and the paradoxical combinator.

  [Ed. note: the existence of this seems to suggest we have to be more
  restrictive on the set of processes and names we admit if we are to
  support no-cloning.]
\end{remark}

\subsubsection{Bisimulation}

The computational dynamics gives rise to another kind of equivalence,
the equivalence of computational behavior. As previously mentioned
this is typically captured \emph{via} some form of bisimulation.

% The notion we use in this paper is weak barbed bisimulation
% \cite{milner91polyadicpi}.

The notion we use in this paper is derived from weak barbed
bisimulation \cite{milner91polyadicpi}. 

\begin{definition}
An \emph{observation relation}, $\downarrow_{\mathcal N}$, over a set
of names, $\mathcal N$, is the smallest relation satisfying the rules
below.

\infrule[Out-barb]{y \in {\mathcal N}, \; x \nameeq y}
		  {\outputp{x}{v} \downarrow_{\mathcal N} x}
\infrule[Par-barb]{\mbox{$P\downarrow_{\mathcal N} x$ or $Q\downarrow_{\mathcal N} x$}}
		  {\binpar{P}{Q} \downarrow_{\mathcal N} x}

We write $P \Downarrow_{\mathcal N} x$ if there is $Q$ such that 
$P \wred Q$ and $Q \downarrow_{\mathcal N} x$.
\end{definition}

\begin{definition}
%\label{def.bbisim}
An  ${\mathcal N}$-\emph{barbed bisimulation} over a set of names, ${\mathcal N}$, is a symmetric binary relation 
${\mathcal S}_{\mathcal N}$ between agents such that $P\rel{S}_{\mathcal N}Q$ implies:
\begin{enumerate}
\item If $P \red P'$ then $Q \wred Q'$ and $P'\rel{S}_{\mathcal N} Q'$.
\item If $P\downarrow_{\mathcal N} x$, then $Q\Downarrow_{\mathcal N} x$.
\end{enumerate}
$P$ is ${\mathcal N}$-barbed bisimilar to $Q$, written
$P \wbbisim_{\mathcal N} Q$, if $P \rel{S}_{\mathcal N} Q$ for some ${\mathcal N}$-barbed bisimulation ${\mathcal S}_{\mathcal N}$.
\end{definition}

$\mathcal{R} \subseteq \pi \times \pi$

$P \mathcal{R} Q => \forall P'. P \red P' \Rightarrow \exists Q'. Q \red Q', P' \mathcal{R} Q'$

$P \vdash x \Rightarrow Q \vdash x$

\begin{mathpar}
  \inferrule*[lab=Out-barb]{x \nameeq y}{{y}!\langle{Q}\rangle \vdash x}
  \and
  \inferrule*[lab=Par-barb]{\mbox{$P\vdash x$ or $Q\vdash x$}}{\binpar{P}{Q} \vdash x}
\end{mathpar}

\subsubsection{Contexts}

One of the principle advantages of computational calculi like the
$\pi$-calculus is a well-defined notion of context,
contextual-equivalence and a correlation between
contextual-equivalence and notions of bisimulation. The notion of
context allows the decomposition of a process into (sub-)process and
its syntactic environment, its context. Thus, a context may be
thought of as a process with a ``hole'' (written $\Box$) in it. The
application of a context $M$ to a process $P$, written $M[P]$, is
tantamount to filling the hole in $M$ with $P$. In this paper we do
not need the full weight of this theory, but do make use of the notion
of context in the proof the main theorem. 

\begin{mathpar}
  \inferrule* [lab=summation] {} {{M_{M},M_{N}} \bc \Box \;|\; x.M_{A} \;|\; M_{M}+M_{N}}
  \and
  \inferrule* [lab=agent] {} {{M_{A}} \bc (\vec{x})M_{P} \;| \; \clift{P_0,\ldots,M_{P},\ldots,P_N}}
  \and \\
  \inferrule* [lab=process] {} {{M_{P}} \bc M_{N} \;| \;P|M_{P} }
\end{mathpar} 

\begin{mathpar}
  \inferrule* [lab=sychronization] {} {M_{N} \bc \Box \;|\; x?M_{F} \;|\; x!M_{C}}
  \and
  \inferrule* [lab=abstraction] {} {{M_{F}} \bc (x)M_{P} }
  \and
  \inferrule* [lab=concretion] {} {{M_{C}} \bc \langle M_{P} \rangle }
  \and \\
  \inferrule* [lab=process] {} {{M_{P}} \bc M_{N} \;| \;P|M_{P} }
\end{mathpar}

\begin{definition}[contextual application] Given a context $M$, and
  process $P$, we define the \emph{contextual application}, $M[P] :=
  M\{P/\Box\}$. That is, the contextual application of M to P is the
  substitution of $P$ for $\Box$ in $M$.
\end{definition}

$\meaningof{-} : L \to \mathcal{P}(\pi)$

\begin{mathpar}
  \inferrule* [lab=collection] {} {\meaningof{true} = \pi, \and \meaningof{~E} = \pi \setminus \meaningof{E}, \and \meaningof{E_{1} \& E_{2}} = \meaningof{E_{1}} \cap \meaningof{E_{2}}}
\end{mathpar}

\begin{mathpar}
  \inferrule* [lab=structure] {} {\meaningof{0} = \{ P \in \pi | P \equiv 0 \}, \and \\ \meaningof{E_1 | E_2} = \{ P \in \pi | P \equiv P_{1} | P_{2}, P_{1} \in \meaningof{E_{1}}, P_{2} \in \meaningof{E_2}\} }
\end{mathpar}

\begin{mathpar}
 \inferrule* [lab=behavior] {} {\meaningof{\langle a?b \rangle E} = \{ P \in \pi | P \equiv Q | u?(y)P', \\ \and \\\\ \and \\ \;\;\; u \in \meaningof{a}, \forall z.P'\{z/y\} \in \meaningof{E\{z/b\}}\}, \and \\ \meaningof{a!E} = \{ P \in \pi | P \equiv Q | x!\langle P' \rangle, x \in \meaningof{a} P' \in \meaningof{E}\} }
\end{mathpar}

\begin{mathpar}
 \inferrule* [lab=nominal] {} {\meaningof{\quotep{E}} = \{ \quotep{P} \in \quotep{\pi} | P \in \meaningof{E} \}, \and \meaningof{\quotep{P}} = \{ \quotep{Q} \in \quotep{\pi} | P \equiv Q \} \and \\ \meaningof{@\quotep{E}} = \{ P \in \pi | P \equiv @x, x \in \meaningof{E} \}}
\end{mathpar}

\begin{eqnarray*}
  \\
  \meaningof{-} : TS \to ST
\end{eqnarray*}

\begin{eqnarray*}
  \\
  L : TS \to ST
\end{eqnarray*}

\begin{eqnarray*}
  \\
  P \models E \iff P \in \meaningof{E}
\end{eqnarray*}

\begin{eqnarray*}
  P \approx_{L} Q \iff \forall E \in L. P \models E \iff Q \models E
\end{eqnarray*}

\begin{eqnarray*}
  P \approx_{K} Q
\end{eqnarray*}

\begin{eqnarray*}
  P \approx Q
\end{eqnarray*}

$\approx_{K} = \approx = \approx_{L}$

\subsubsection{Contextual duality}

Note that contexts extend the quotation operation to a family of
operations from processes to names. Given a context, $M$, we can
define a \emph{nominal context}, $\quotep{M}$ by $\quotep{M}[P] :=
\quotep{M[P]}$. To foreshadow what is to come we observe that these
operations enjoy a duality with processes very much like the duality
between vectors and maps from vectors to scalars.

Further, because the calculus is essentially higher-order, we have a
correspondence between contexts and processes. More specifically,
given a name $x$ and a context $M$ we can construct $M^{*}_{x}$ such
that 

\begin{mathpar}
  M^{*}_{x} | \lift{x}{P} \red M[P]
\end{mathpar}

namely,

\begin{mathpar}
  M^{*}_{x} := x?(u).M[\dropn{u}]
\end{mathpar}

The dependence of $M^{*}_{x}$ on a name makes it an abstraction, 

\begin{mathpar}
  M^{*} := (x)x?(u).M[\dropn{u}]
\end{mathpar}

\subsection{Additional notation}

It will sometimes be convenient to denote the process a name
quotes. We already have the notation $x = \quotep{P}$, but it will be
convenient to introduce an alternate notation, $\procn{x}$, when we
want to emphasize the connection to the use of the name. Note that, by
virtue of name equivalence, $\quotep{\procn{x}} \nameeq x$; so, the
notation is consistent with previous definitions.

Further, because names have structure it is possible to effect
substitutions on the basis of that structure. This means we need to
upgrade our notation for substitutions, which we accomplish by
adapting comprehension notation. Thus,

\begin{mathpar}
  P\{ y / x : x \in S \}
\end{mathpar}

is interpreted to mean the process derived from P by replacing (in a
capture-avoiding manner) each occurrence of $x$ in $S$ by $y$. For example,

\begin{mathpar}
  P\{ \quotep{\procn{x}|\procn{x}} / x : x \in \freenames{P} \}
\end{mathpar}

will replace each (occurrence) of a free name $x$ in $P$ by
$\quotep{\procn{x}|\procn{x}}$.

Also, we will avail ourselves of the notation $x^{L}$ and $x^{R}$ to
denote injections of a name into disjoint copies of the name
space. There are numerous ways to accomplish this. One example can be
found in \cite{MeredithR05}. This notation overloads to vectors of
names: $\vec{x}^{\pi} := (x_{i}^{\pi} \; : \; 0 \leq i < |\vec{x}| )$ where $\pi \in \{L,R\}$.

We also use $P^{\Box} := P|\Box$.

In \cite{MeredithR05} an interpretation of the new operator is
given. It turns out that there are several possible interpretations
all enjoying the requisite algebraic properties of the operator (see
\cite{milner91polyadicpi}). We will therefore make liberal use of
$(\nu\; \vec{x})P$.

% subsection the_syntax_and_semantics_of_the_notation_system (end)   

\input{qm2pi.qmops} 

\input{qm2pi.sterngerlach} 

\input{qm2pi.metric} 

% section concurrent_process_calculi (end)

%\input{qm2pi.proofsketch}

% section proof sketch (end)

%\input{qm2pi.slviaknots} 

% section spatial logic via knots (end)

\input{qm2pi.conclusion}

% section conclusion (end)

%\input{qm2pi.dtcodes} 

% section wiring algorithm (end)

\input{qm2pi.ack} 

% section acknowledgments (end)

\newpage


\bibliographystyle{plain}   
\bibliography{../../biblios/main.bib}

\input{qm2pi.rhodetails}

\end{document}

 

%\documentclass[12pt]{llncs}
%\documentclass{jktr}

\usepackage[pdftex]{hyperref}                   
\usepackage {listings}
\usepackage {mathpartir}
\usepackage{bcprules}
%\usepackage{listings}
                       
\usepackage{graphicx} 
%\usepackage[margins=2.5cm,nohead,nofoot]{geometry}
%\usepackage{geometry}
\usepackage{amsfonts}
\usepackage{amstext}
\usepackage{latexsym}
\usepackage{amssymb}
\usepackage{color}


%\include{myPreamble}
\include{qm2pi.local} 

%\ifpdf
%\usepackage[pdftex]{graphicx}
%\else
%\usepackage{graphicx}
%\fi

 % \ifpdf
%  \usepackage{pdfsync}
%  \if


%\title{Brief Article}
%\author{David F. Snyder}
%\author{L.G. Meredith}

%\address{Dept. of Math., Texas State University--San Marcos, San Marcos, TX 78666}
       
\pagestyle{empty}


\begin{document}

\lstset{language=[Objective]Caml,frame=shadowbox}

\input{qm2pi.front}

% section front matter (end)

\input{qm2pi.intro} 
 
% section introduction (end)

% \input{qm2pi.knotations} 

% section notation (end)

\input{qm2pi.process.calculi} 

% section concurrent_process_calculi_and_spatial_logics_ (end)
    
%\input{qm2pi.knots2pi} 

%\input{qm2pi.trefoil} 

%\input{qm2pi.mainthm} 

% subsection basic_interpretation (end)

%\input{qm2pi.rho.presentation} 
\subsection{The syntax and semantics of the notation system}\label{sub:the_syntax_and_semantics_of_the_notation_system} % (fold)

We now summarize a technical presentation of the calculus that
embodies our theory of dynamics. The typical presentation of such a
calculus follows the style of giving generators and relations on
them. The grammar, below, describing term constructors, freely
generates the set of processes, $\Proc$. This set is then quotiented
by a relation known as structural congruence and it is over this set
that the notion of dynamics is expressed. This presentation is
essentially that of \cite{MeredithR05} with the addition of
polyadicity and summation. For readability we have relegated some of
the technical subtleties to an appendix.

\subsubsection{Process grammar}\label{subsub:process_grammar}

\begin{mathpar}
  \inferrule* [lab=synchronization] {} {{M} \bc \pzero \;|\; x?F \;|\; x!C }
  \and
  \inferrule* [lab=abstraction] {} {{F} \bc (x)P}
  \and
  \inferrule* [lab=concretion] {} {{C} \bc \langle Q \rangle}
  \and
  \inferrule* [lab=process] {} {{P,Q} \bc M \;| \;P|Q \;|\; @{x}}
  \and
  \inferrule* [lab=name] {} {{x} \bc \quotep{P}}
\end{mathpar} 

Note that $\vec{x}$ (resp. $\vec{P}$) denotes a vector of names
(resp. processes) of length $|\vec{x}|$ (resp. $|\vec{P}|$). We adopt
the following useful abbreviations.

\begin{mathpar}
   x?(\vec{y}).P := x.(\vec{y})P \and  x\clift{\vec{P}} := x.\clift{\vec{P}}
   \and x!(y) := \lift{x}{\dropn{y}}
   \and \Pi_{i=0}^{n-1}P_i := P_0 | \ldots | P_{n-1}
\end{mathpar}

\subsubsection{Structural congruence}

\paragraph{Free and bound names and alpha-equivalence.} At the
core of structural equivalence is alpha-equivalence which identifies
process that are the same up to a change of variable. Formally, we
recognize the distinction between free and bound names. The free names
of a process, $\freenames{P}$, may be calculated recursively as
follows:

\begin{mathpar}
\freenames{\pzero} := \emptyset
  \and \\
  \freenames{x?(y).P} := \{ x \} \cup (\freenames{P} \setminus \{ y \})
  \and 
  \freenames{x!\langle P \rangle} := \{ x \} \cup \{ P \} 
  \and \\
  \freenames{P|Q} := \freenames{P} \cup \freenames{Q}
  \and \\
  \freenames{@{x}} := \{ x \}
\end{mathpar}

$\pi$
$\quotep{\pi}$

$\freenames{-} : \pi \to \mathcal{P}(\quotep{\pi})$

\begin{eqnarray*}
  \freenames{\pzero} & := & \emptyset \\
  \freenames{x?(y).P} & := & \{ x \} \cup (\freenames{P} \setminus \{ y \}) \\
  \freenames{x!\langle P \rangle} & := & \{ x \} \cup \{ P \} \\
  \freenames{P|Q} & := & \freenames{P} \cup \freenames{Q} \\
  \freenames{\dropn{x}} & := & \{ x \}
\end{eqnarray*}

The bound names of a process, $\boundnames{P}$, are those names occurring in $P$
that are not free. For example, in $x?(y).0$, the name $x$ is free, while $y$ is bound.

\begin{mathpar}
  \inferrule* [lab=monoidal-laws] {} { P|Q \equiv Q|P \and P|0 \equiv P \and P|(Q|R) \equiv (P|Q)|R }
\end{mathpar}

\begin{mathpar}
  \inferrule* [lab=alpha-equivalence] {} { (x)P \equiv (y)P\{y/x\} \and y \not\in \freenames{P} }
\end{mathpar}

\begin{definition}
Then two processes, $P,Q$, are alpha-equivalent if $P = Q\{\vec{y}/\vec{x}\}$ for
some $\vec{x} \in \boundnames{Q},\vec{y} \in \boundnames{P}$, where $Q\{\vec{y}/\vec{x}\}$
denotes the capture-avoiding substitution of $\vec{y}$ for $\vec{x}$ in $Q$.
\end{definition}

\begin{definition}
  The {\em structural congruence} \cite{SangiorgiWalker} , $\equiv$,
  between processes is the least congruence containing
  alpha-equivalence, satisfying the abelian monoid laws
  (associativity, commutativity and $\pzero$ as identity) for parallel
  composition $|$ and for summation $+$.
\end{definition}

\subsection{Name equivalence}

We take name equivalence, written $\nameeq$, to be the smallest
equivalence relation generated by the following rules.

\begin{mathpar}
\inferrule*[lab=Quote-drop]
{ }
{ \quotep{@{x}} \nameeq x }

\inferrule*[lab=Struct-equiv]
{ P \scong Q }
{ \quotep{P} \nameeq \quotep{Q} }
\end{mathpar}

The astute reader will have noticed that the mutual recursion of names
and processes imposes a mutual recursion on alpha-equivalence and
structural equivalence via name-equivalence. Fortunately, all of this
works out pleasantly and we may calculate in the natural way, free of
concern. The reader interested in the details is referred to the
appendix \ref{appendix:rho_details}.

\subsection{Substitution}

We use $\Proc$ for the set of processes, $\QProc$ for the set of
names, and $\id{\{}\vec{y} / \vec{x} \id{\}}$ to denote partial maps,
$s : \QProc \rightarrow \QProc$. A map, $s$ lifts, uniquely, to a map
on process terms, $\widehat{s} : \Proc \rightarrow \Proc$ by the
following equations.

\begin{mathpar}
  (0) \psubstp{Q}{P} := 0 \\
  (R \juxtap S) \psubstp{Q}{P}
  :=    
  (R)\psubstp{Q}{P} \juxtap (S) \psubstp{Q}{P} \\
  (x?(y).R) \psubstp{Q}{P}    
  :=    
  (x)\substp{Q}{P} (z)\concat( (R \psubstn{z}{y}) \psubstp{Q}{P} ) \\
  (\lift{x}{R}) \psubstp{Q}{P}  
  :=
  \lift{(x)\substp{Q}{P}}{ R \psubstp{Q}{P} } \\
%   (\dropn{x})  \psubstp{Q}{P}       
%   := 
%   \left\{ 
%     \begin{array}{ccc} 
%       \dropn{\quotep{Q}} & & x \nameeq \quotep{P} \\
%       \dropn{x} & & otherwise \\
%     \end{array}
%   \right. 
  (\dropn{x})  \psubstp{Q}{P}       
  := 
  \left\{ 
    \begin{array}{ccc} 
      Q & & x \nameeq \quotep{P} \\
      \dropn{x} & & otherwise \\
    \end{array}
  \right.
\end{mathpar}
 

where

\begin{eqnarray}
  (x)\id{\{} \lpquote Q \rpquote / \lpquote P \rpquote \id{\}}            = 
  \left\{ 
    \begin{array}{ccc}
      \lpquote Q \rpquote & & x \nameeq \lpquote P \rpquote \\
      x & & otherwise \\
    \end{array}
  \right. \nonumber
\end{eqnarray}

and $z$ is chosen distinct from $\quotep{P}$, $\quotep{Q}$, the free
names in $Q$, and all the names in $R$. Our $\alpha$-equivalence will
be built in the standard way from this substitution.

\begin{remark}\label{rem:no_self_referential_names}
  One consequence of these definitions is that $\forall P. \quotep{P}
  \not\in \freenames{P}$.
\end{remark}

\subsection{ Dynamic quote: an example }

Anticipating something of what's to come, consider applying the
substitution, $\widehat{\id{\{}u / z \id{\}}}$, to the following pair
of processes, $\lift{w}{y!(z)}$ and $w[ \lpquote y!(z) \rpquote ]$.

\begin{eqnarray}
	\lift{w}{y!(z)}\widehat{\id{\{}u / z \id{\}}}
		& = &
		\lift{w}{y!(u)} \nonumber\\
	w[ \lpquote y!(z) \rpquote ] \widehat{ \id{\{}u / z \id{\}} }
		& = &
		w[ \lpquote y!(z) \rpquote ] \nonumber
\end{eqnarray}

Because the body of the process between quotes is impervious to
substitution, we get radically different answers. In fact, by
examining the first process in an input context,
e.g. $x?(z).\lift{w}{y!(z)}$, we see that the process under the lift
operator may be shaped by prefixed inputs binding a name inside it. In
this sense, the lift operator will be seen as a way to dynamically
construct processes before reifying them as names.

Finally equipped with these standard features we can present the
dynamics of the calculus.

\subsubsection{Operational semantics} 

Finally, we introduce the computational dynamics. What marks these
algebras as distinct from other more traditionally studied algebraic
structures, e.g. vector spaces or polynomial rings, is the manner in
which dynamics is captured. In traditional structures, dynamics is typically
expressed through morphisms between such structures, as in linear maps
between vector spaces or morphisms between rings. In algebras
associated with the semantics of computation, the dynamics is
expressed as part of the algebraic structure itself, through a
reduction reduction relation typically denoted by $\red$. Below, we
give a recursive presentation of this relation for the calculus used
in the encoding.

$\red \subseteq \pi \times \pi$
$\red : \pi \to \mathcal{P}(\pi)$

\begin{mathpar}
  \inferrule* [lab=Comm] { \textsf{match}( x_{src}, x_{trgt} ) } { x_{trgt}?(y)P \; | \; x_{src}!\langle {Q} \rangle \red P\{\quotep{Q}/y}\} }
  \and \\
  \inferrule* [lab=Par] {{P} \red {P}'} {{{P} | {Q}} \red {{P}' | {Q}}}
  \and
  \inferrule* [lab=Equiv]{{{P} \scong {P}'} \andalso {{P}' \red {Q}'} \andalso {{Q}' \scong {Q}}}{{P} \red {Q}}
\end{mathpar}

\begin{eqnarray*}
  match_{\equiv} (\quotep{P},\quotep{Q}) & := & P \equiv Q \\
  match_{\dagger}(\quotep{P},\quotep{Q}) & := & \forall R. P|Q \red^{*} R => R \red^{*} 0 \\
  match_{K}(\quotep{P},\quotep{Q}) & := & K \mbox{ for some context } K
\end{eqnarray*}

$u?(x)P | u!\langle Q \rangle \red P\{\quotep{Q}/x\}$

%We write $\wred$ for $\red^*$, and $P\red$ if $\exists Q $ such that $ P \red Q$.
We write $P\red$ if $\exists Q $ such that $ P \red Q$ and $P\not\red$, otherwise.

\section{Replication}

As mentioned before, it is known that replication (and hence
recursion) can be implemented in a higher-order process algebra
\cite{SangiorgiWalker}. As our first example of calculation with the
machinery thus far presented we give the construction explicitly in
the {\rhoc}.

\begin{eqnarray}
	D_{x} & := & \prefix{x}{y}{(\binpar{\outputp{x}{y}}{@{y}})} \nonumber\\
	\bangp_{x}{P} & := & \binpar{{x}!\langle{\binpar{D_{x}}{P}}\rangle}{D_{x}} \nonumber
\end{eqnarray}

\begin{eqnarray}
	\bangp_{x}{P} & & \nonumber\\
	=
	& {x}!\langle{(\prefix{x}{y}{(\outputp{x}{y} | @{y})) | P}}\rangle 
	      | \prefix{x}{y}{(\outputp{x}{y} | @{y})} & \nonumber\\
	\red
	& (\outputp{x}{y} | @{y})\substn{\quotep{(\prefix{x}{y}{(@{y} | \outputp{x}{y})) | P}}}{y} & \nonumber\\
	=
	& \outputp{x}{\quotep{(\prefix{x}{y}{(\outputp{x}{y} | @{y})) | P}}}
	  | {(\prefix{x}{y}{(\outputp{x}{y} | @{y})) | P}} & \nonumber\\
	\red
	& \ldots & \nonumber\\
	\red^*
	& P | P | \ldots & \nonumber
\end{eqnarray}

Of course, this encoding, as an implementation, runs away, unfolding
$\bangp{P}$ eagerly. A lazier and more implementable replication
operator, restricted to input-guarded processes, may be obtained as follows.

\begin{eqnarray}
\bangp{\prefix{u}{v}{P}} 
	:= 
	\binpar{\lift{x}{\prefix{u}{v}{(\binpar{D(x)}{P})}}}{D(x)} \nonumber
\end{eqnarray}

\begin{remark}
  Note that the lazier definition still does not deal with summation
  or mixed summation (i.e. sums over input and output). The reader is
  invited to construct definitions of replication that deal with these
  features. 

  Further, the definitions are parameterized in a name, $x$. Can you,
  gentle reader, make a definition that eliminates this parameter and
  guarantees no accidental interaction between the replication
  machinery and the process being replicated -- i.e. no accidental
  sharing of names used by the process to get its work done and the
  name(s) used by the replication to effect copying. This latter
  revision of the definition of replication is crucial to obtaining
  the expected identity $!!P \sim !P$.
\end{remark}

\begin{remark}\label{rem:paradoxical_combinator}
  The reader familiar with the lambda calculus will have noticed the
  similarity between $D$ and the paradoxical combinator.

  [Ed. note: the existence of this seems to suggest we have to be more
  restrictive on the set of processes and names we admit if we are to
  support no-cloning.]
\end{remark}

\subsubsection{Bisimulation}

The computational dynamics gives rise to another kind of equivalence,
the equivalence of computational behavior. As previously mentioned
this is typically captured \emph{via} some form of bisimulation.

% The notion we use in this paper is weak barbed bisimulation
% \cite{milner91polyadicpi}.

The notion we use in this paper is derived from weak barbed
bisimulation \cite{milner91polyadicpi}. 

\begin{definition}
An \emph{observation relation}, $\downarrow_{\mathcal N}$, over a set
of names, $\mathcal N$, is the smallest relation satisfying the rules
below.

\infrule[Out-barb]{y \in {\mathcal N}, \; x \nameeq y}
		  {\outputp{x}{v} \downarrow_{\mathcal N} x}
\infrule[Par-barb]{\mbox{$P\downarrow_{\mathcal N} x$ or $Q\downarrow_{\mathcal N} x$}}
		  {\binpar{P}{Q} \downarrow_{\mathcal N} x}

We write $P \Downarrow_{\mathcal N} x$ if there is $Q$ such that 
$P \wred Q$ and $Q \downarrow_{\mathcal N} x$.
\end{definition}

\begin{definition}
%\label{def.bbisim}
An  ${\mathcal N}$-\emph{barbed bisimulation} over a set of names, ${\mathcal N}$, is a symmetric binary relation 
${\mathcal S}_{\mathcal N}$ between agents such that $P\rel{S}_{\mathcal N}Q$ implies:
\begin{enumerate}
\item If $P \red P'$ then $Q \wred Q'$ and $P'\rel{S}_{\mathcal N} Q'$.
\item If $P\downarrow_{\mathcal N} x$, then $Q\Downarrow_{\mathcal N} x$.
\end{enumerate}
$P$ is ${\mathcal N}$-barbed bisimilar to $Q$, written
$P \wbbisim_{\mathcal N} Q$, if $P \rel{S}_{\mathcal N} Q$ for some ${\mathcal N}$-barbed bisimulation ${\mathcal S}_{\mathcal N}$.
\end{definition}

$\mathcal{R} \subseteq \pi \times \pi$

$P \mathcal{R} Q => \forall P'. P \red P' \Rightarrow \exists Q'. Q \red Q', P' \mathcal{R} Q'$

$P \vdash x \Rightarrow Q \vdash x$

\begin{mathpar}
  \inferrule*[lab=Out-barb]{x \nameeq y}{{y}!\langle{Q}\rangle \vdash x}
  \and
  \inferrule*[lab=Par-barb]{\mbox{$P\vdash x$ or $Q\vdash x$}}{\binpar{P}{Q} \vdash x}
\end{mathpar}

\subsubsection{Contexts}

One of the principle advantages of computational calculi like the
$\pi$-calculus is a well-defined notion of context,
contextual-equivalence and a correlation between
contextual-equivalence and notions of bisimulation. The notion of
context allows the decomposition of a process into (sub-)process and
its syntactic environment, its context. Thus, a context may be
thought of as a process with a ``hole'' (written $\Box$) in it. The
application of a context $M$ to a process $P$, written $M[P]$, is
tantamount to filling the hole in $M$ with $P$. In this paper we do
not need the full weight of this theory, but do make use of the notion
of context in the proof the main theorem. 

\begin{mathpar}
  \inferrule* [lab=summation] {} {{M_{M},M_{N}} \bc \Box \;|\; x.M_{A} \;|\; M_{M}+M_{N}}
  \and
  \inferrule* [lab=agent] {} {{M_{A}} \bc (\vec{x})M_{P} \;| \; \clift{P_0,\ldots,M_{P},\ldots,P_N}}
  \and \\
  \inferrule* [lab=process] {} {{M_{P}} \bc M_{N} \;| \;P|M_{P} }
\end{mathpar} 

\begin{mathpar}
  \inferrule* [lab=sychronization] {} {M_{N} \bc \Box \;|\; x?M_{F} \;|\; x!M_{C}}
  \and
  \inferrule* [lab=abstraction] {} {{M_{F}} \bc (x)M_{P} }
  \and
  \inferrule* [lab=concretion] {} {{M_{C}} \bc \langle M_{P} \rangle }
  \and \\
  \inferrule* [lab=process] {} {{M_{P}} \bc M_{N} \;| \;P|M_{P} }
\end{mathpar}

\begin{definition}[contextual application] Given a context $M$, and
  process $P$, we define the \emph{contextual application}, $M[P] :=
  M\{P/\Box\}$. That is, the contextual application of M to P is the
  substitution of $P$ for $\Box$ in $M$.
\end{definition}

$\meaningof{-} : L \to \mathcal{P}(\pi)$

\begin{mathpar}
  \inferrule* [lab=collection] {} {\meaningof{true} = \pi, \and \meaningof{~E} = \pi \setminus \meaningof{E}, \and \meaningof{E_{1} \& E_{2}} = \meaningof{E_{1}} \cap \meaningof{E_{2}}}
\end{mathpar}

\begin{mathpar}
  \inferrule* [lab=structure] {} {\meaningof{0} = \{ P \in \pi | P \equiv 0 \}, \and \\ \meaningof{E_1 | E_2} = \{ P \in \pi | P \equiv P_{1} | P_{2}, P_{1} \in \meaningof{E_{1}}, P_{2} \in \meaningof{E_2}\} }
\end{mathpar}

\begin{mathpar}
 \inferrule* [lab=behavior] {} {\meaningof{\langle a?b \rangle E} = \{ P \in \pi | P \equiv Q | u?(y)P', \\ \and \\\\ \and \\ \;\;\; u \in \meaningof{a}, \forall z.P'\{z/y\} \in \meaningof{E\{z/b\}}\}, \and \\ \meaningof{a!E} = \{ P \in \pi | P \equiv Q | x!\langle P' \rangle, x \in \meaningof{a} P' \in \meaningof{E}\} }
\end{mathpar}

\begin{mathpar}
 \inferrule* [lab=nominal] {} {\meaningof{\quotep{E}} = \{ \quotep{P} \in \quotep{\pi} | P \in \meaningof{E} \}, \and \meaningof{\quotep{P}} = \{ \quotep{Q} \in \quotep{\pi} | P \equiv Q \} \and \\ \meaningof{@\quotep{E}} = \{ P \in \pi | P \equiv @x, x \in \meaningof{E} \}}
\end{mathpar}

\begin{eqnarray*}
  \\
  \meaningof{-} : TS \to ST
\end{eqnarray*}

\begin{eqnarray*}
  \\
  L : TS \to ST
\end{eqnarray*}

\begin{eqnarray*}
  \\
  P \models E \iff P \in \meaningof{E}
\end{eqnarray*}

\begin{eqnarray*}
  P \approx_{L} Q \iff \forall E \in L. P \models E \iff Q \models E
\end{eqnarray*}

\begin{eqnarray*}
  P \approx_{K} Q
\end{eqnarray*}

\begin{eqnarray*}
  P \approx Q
\end{eqnarray*}

$\approx_{K} = \approx = \approx_{L}$

\subsubsection{Contextual duality}

Note that contexts extend the quotation operation to a family of
operations from processes to names. Given a context, $M$, we can
define a \emph{nominal context}, $\quotep{M}$ by $\quotep{M}[P] :=
\quotep{M[P]}$. To foreshadow what is to come we observe that these
operations enjoy a duality with processes very much like the duality
between vectors and maps from vectors to scalars.

Further, because the calculus is essentially higher-order, we have a
correspondence between contexts and processes. More specifically,
given a name $x$ and a context $M$ we can construct $M^{*}_{x}$ such
that 

\begin{mathpar}
  M^{*}_{x} | \lift{x}{P} \red M[P]
\end{mathpar}

namely,

\begin{mathpar}
  M^{*}_{x} := x?(u).M[\dropn{u}]
\end{mathpar}

The dependence of $M^{*}_{x}$ on a name makes it an abstraction, 

\begin{mathpar}
  M^{*} := (x)x?(u).M[\dropn{u}]
\end{mathpar}

\subsection{Additional notation}

It will sometimes be convenient to denote the process a name
quotes. We already have the notation $x = \quotep{P}$, but it will be
convenient to introduce an alternate notation, $\procn{x}$, when we
want to emphasize the connection to the use of the name. Note that, by
virtue of name equivalence, $\quotep{\procn{x}} \nameeq x$; so, the
notation is consistent with previous definitions.

Further, because names have structure it is possible to effect
substitutions on the basis of that structure. This means we need to
upgrade our notation for substitutions, which we accomplish by
adapting comprehension notation. Thus,

\begin{mathpar}
  P\{ y / x : x \in S \}
\end{mathpar}

is interpreted to mean the process derived from P by replacing (in a
capture-avoiding manner) each occurrence of $x$ in $S$ by $y$. For example,

\begin{mathpar}
  P\{ \quotep{\procn{x}|\procn{x}} / x : x \in \freenames{P} \}
\end{mathpar}

will replace each (occurrence) of a free name $x$ in $P$ by
$\quotep{\procn{x}|\procn{x}}$.

Also, we will avail ourselves of the notation $x^{L}$ and $x^{R}$ to
denote injections of a name into disjoint copies of the name
space. There are numerous ways to accomplish this. One example can be
found in \cite{MeredithR05}. This notation overloads to vectors of
names: $\vec{x}^{\pi} := (x_{i}^{\pi} \; : \; 0 \leq i < |\vec{x}| )$ where $\pi \in \{L,R\}$.

We also use $P^{\Box} := P|\Box$.

In \cite{MeredithR05} an interpretation of the new operator is
given. It turns out that there are several possible interpretations
all enjoying the requisite algebraic properties of the operator (see
\cite{milner91polyadicpi}). We will therefore make liberal use of
$(\nu\; \vec{x})P$.

% subsection the_syntax_and_semantics_of_the_notation_system (end)   

\input{qm2pi.qmops} 

\input{qm2pi.sterngerlach} 

\input{qm2pi.metric} 

% section concurrent_process_calculi (end)

%\input{qm2pi.proofsketch}

% section proof sketch (end)

%\input{qm2pi.slviaknots} 

% section spatial logic via knots (end)

\input{qm2pi.conclusion}

% section conclusion (end)

%\input{qm2pi.dtcodes} 

% section wiring algorithm (end)

\input{qm2pi.ack} 

% section acknowledgments (end)

\newpage


\bibliographystyle{plain}   
\bibliography{../../biblios/main.bib}

\input{qm2pi.rhodetails}

\end{document}

 

% subsection basic_interpretation (end)

%\input{qm2pi.rho.presentation} 
\subsection{The syntax and semantics of the notation system}\label{sub:the_syntax_and_semantics_of_the_notation_system} % (fold)

We now summarize a technical presentation of the calculus that
embodies our theory of dynamics. The typical presentation of such a
calculus follows the style of giving generators and relations on
them. The grammar, below, describing term constructors, freely
generates the set of processes, $\Proc$. This set is then quotiented
by a relation known as structural congruence and it is over this set
that the notion of dynamics is expressed. This presentation is
essentially that of \cite{MeredithR05} with the addition of
polyadicity and summation. For readability we have relegated some of
the technical subtleties to an appendix.

\subsubsection{Process grammar}\label{subsub:process_grammar}

\begin{mathpar}
  \inferrule* [lab=synchronization] {} {{M} \bc \pzero \;|\; x?F \;|\; x!C }
  \and
  \inferrule* [lab=abstraction] {} {{F} \bc (x)P}
  \and
  \inferrule* [lab=concretion] {} {{C} \bc \langle Q \rangle}
  \and
  \inferrule* [lab=process] {} {{P,Q} \bc M \;| \;P|Q \;|\; @{x}}
  \and
  \inferrule* [lab=name] {} {{x} \bc \quotep{P}}
\end{mathpar} 

Note that $\vec{x}$ (resp. $\vec{P}$) denotes a vector of names
(resp. processes) of length $|\vec{x}|$ (resp. $|\vec{P}|$). We adopt
the following useful abbreviations.

\begin{mathpar}
   x?(\vec{y}).P := x.(\vec{y})P \and  x\clift{\vec{P}} := x.\clift{\vec{P}}
   \and x!(y) := \lift{x}{\dropn{y}}
   \and \Pi_{i=0}^{n-1}P_i := P_0 | \ldots | P_{n-1}
\end{mathpar}

\subsubsection{Structural congruence}

\paragraph{Free and bound names and alpha-equivalence.} At the
core of structural equivalence is alpha-equivalence which identifies
process that are the same up to a change of variable. Formally, we
recognize the distinction between free and bound names. The free names
of a process, $\freenames{P}$, may be calculated recursively as
follows:

\begin{mathpar}
\freenames{\pzero} := \emptyset
  \and \\
  \freenames{x?(y).P} := \{ x \} \cup (\freenames{P} \setminus \{ y \})
  \and 
  \freenames{x!\langle P \rangle} := \{ x \} \cup \{ P \} 
  \and \\
  \freenames{P|Q} := \freenames{P} \cup \freenames{Q}
  \and \\
  \freenames{@{x}} := \{ x \}
\end{mathpar}

$\pi$
$\quotep{\pi}$

$\freenames{-} : \pi \to \mathcal{P}(\quotep{\pi})$

\begin{eqnarray*}
  \freenames{\pzero} & := & \emptyset \\
  \freenames{x?(y).P} & := & \{ x \} \cup (\freenames{P} \setminus \{ y \}) \\
  \freenames{x!\langle P \rangle} & := & \{ x \} \cup \{ P \} \\
  \freenames{P|Q} & := & \freenames{P} \cup \freenames{Q} \\
  \freenames{\dropn{x}} & := & \{ x \}
\end{eqnarray*}

The bound names of a process, $\boundnames{P}$, are those names occurring in $P$
that are not free. For example, in $x?(y).0$, the name $x$ is free, while $y$ is bound.

\begin{mathpar}
  \inferrule* [lab=monoidal-laws] {} { P|Q \equiv Q|P \and P|0 \equiv P \and P|(Q|R) \equiv (P|Q)|R }
\end{mathpar}

\begin{mathpar}
  \inferrule* [lab=alpha-equivalence] {} { (x)P \equiv (y)P\{y/x\} \and y \not\in \freenames{P} }
\end{mathpar}

\begin{definition}
Then two processes, $P,Q$, are alpha-equivalent if $P = Q\{\vec{y}/\vec{x}\}$ for
some $\vec{x} \in \boundnames{Q},\vec{y} \in \boundnames{P}$, where $Q\{\vec{y}/\vec{x}\}$
denotes the capture-avoiding substitution of $\vec{y}$ for $\vec{x}$ in $Q$.
\end{definition}

\begin{definition}
  The {\em structural congruence} \cite{SangiorgiWalker} , $\equiv$,
  between processes is the least congruence containing
  alpha-equivalence, satisfying the abelian monoid laws
  (associativity, commutativity and $\pzero$ as identity) for parallel
  composition $|$ and for summation $+$.
\end{definition}

\subsection{Name equivalence}

We take name equivalence, written $\nameeq$, to be the smallest
equivalence relation generated by the following rules.

\begin{mathpar}
\inferrule*[lab=Quote-drop]
{ }
{ \quotep{@{x}} \nameeq x }

\inferrule*[lab=Struct-equiv]
{ P \scong Q }
{ \quotep{P} \nameeq \quotep{Q} }
\end{mathpar}

The astute reader will have noticed that the mutual recursion of names
and processes imposes a mutual recursion on alpha-equivalence and
structural equivalence via name-equivalence. Fortunately, all of this
works out pleasantly and we may calculate in the natural way, free of
concern. The reader interested in the details is referred to the
appendix \ref{appendix:rho_details}.

\subsection{Substitution}

We use $\Proc$ for the set of processes, $\QProc$ for the set of
names, and $\id{\{}\vec{y} / \vec{x} \id{\}}$ to denote partial maps,
$s : \QProc \rightarrow \QProc$. A map, $s$ lifts, uniquely, to a map
on process terms, $\widehat{s} : \Proc \rightarrow \Proc$ by the
following equations.

\begin{mathpar}
  (0) \psubstp{Q}{P} := 0 \\
  (R \juxtap S) \psubstp{Q}{P}
  :=    
  (R)\psubstp{Q}{P} \juxtap (S) \psubstp{Q}{P} \\
  (x?(y).R) \psubstp{Q}{P}    
  :=    
  (x)\substp{Q}{P} (z)\concat( (R \psubstn{z}{y}) \psubstp{Q}{P} ) \\
  (\lift{x}{R}) \psubstp{Q}{P}  
  :=
  \lift{(x)\substp{Q}{P}}{ R \psubstp{Q}{P} } \\
%   (\dropn{x})  \psubstp{Q}{P}       
%   := 
%   \left\{ 
%     \begin{array}{ccc} 
%       \dropn{\quotep{Q}} & & x \nameeq \quotep{P} \\
%       \dropn{x} & & otherwise \\
%     \end{array}
%   \right. 
  (\dropn{x})  \psubstp{Q}{P}       
  := 
  \left\{ 
    \begin{array}{ccc} 
      Q & & x \nameeq \quotep{P} \\
      \dropn{x} & & otherwise \\
    \end{array}
  \right.
\end{mathpar}
 

where

\begin{eqnarray}
  (x)\id{\{} \lpquote Q \rpquote / \lpquote P \rpquote \id{\}}            = 
  \left\{ 
    \begin{array}{ccc}
      \lpquote Q \rpquote & & x \nameeq \lpquote P \rpquote \\
      x & & otherwise \\
    \end{array}
  \right. \nonumber
\end{eqnarray}

and $z$ is chosen distinct from $\quotep{P}$, $\quotep{Q}$, the free
names in $Q$, and all the names in $R$. Our $\alpha$-equivalence will
be built in the standard way from this substitution.

\begin{remark}\label{rem:no_self_referential_names}
  One consequence of these definitions is that $\forall P. \quotep{P}
  \not\in \freenames{P}$.
\end{remark}

\subsection{ Dynamic quote: an example }

Anticipating something of what's to come, consider applying the
substitution, $\widehat{\id{\{}u / z \id{\}}}$, to the following pair
of processes, $\lift{w}{y!(z)}$ and $w[ \lpquote y!(z) \rpquote ]$.

\begin{eqnarray}
	\lift{w}{y!(z)}\widehat{\id{\{}u / z \id{\}}}
		& = &
		\lift{w}{y!(u)} \nonumber\\
	w[ \lpquote y!(z) \rpquote ] \widehat{ \id{\{}u / z \id{\}} }
		& = &
		w[ \lpquote y!(z) \rpquote ] \nonumber
\end{eqnarray}

Because the body of the process between quotes is impervious to
substitution, we get radically different answers. In fact, by
examining the first process in an input context,
e.g. $x?(z).\lift{w}{y!(z)}$, we see that the process under the lift
operator may be shaped by prefixed inputs binding a name inside it. In
this sense, the lift operator will be seen as a way to dynamically
construct processes before reifying them as names.

Finally equipped with these standard features we can present the
dynamics of the calculus.

\subsubsection{Operational semantics} 

Finally, we introduce the computational dynamics. What marks these
algebras as distinct from other more traditionally studied algebraic
structures, e.g. vector spaces or polynomial rings, is the manner in
which dynamics is captured. In traditional structures, dynamics is typically
expressed through morphisms between such structures, as in linear maps
between vector spaces or morphisms between rings. In algebras
associated with the semantics of computation, the dynamics is
expressed as part of the algebraic structure itself, through a
reduction reduction relation typically denoted by $\red$. Below, we
give a recursive presentation of this relation for the calculus used
in the encoding.

$\red \subseteq \pi \times \pi$
$\red : \pi \to \mathcal{P}(\pi)$

\begin{mathpar}
  \inferrule* [lab=Comm] { \textsf{match}( x_{src}, x_{trgt} ) } { x_{trgt}?(y)P \; | \; x_{src}!\langle {Q} \rangle \red P\{\quotep{Q}/y}\} }
  \and \\
  \inferrule* [lab=Par] {{P} \red {P}'} {{{P} | {Q}} \red {{P}' | {Q}}}
  \and
  \inferrule* [lab=Equiv]{{{P} \scong {P}'} \andalso {{P}' \red {Q}'} \andalso {{Q}' \scong {Q}}}{{P} \red {Q}}
\end{mathpar}

\begin{eqnarray*}
  match_{\equiv} (\quotep{P},\quotep{Q}) & := & P \equiv Q \\
  match_{\dagger}(\quotep{P},\quotep{Q}) & := & \forall R. P|Q \red^{*} R => R \red^{*} 0 \\
  match_{K}(\quotep{P},\quotep{Q}) & := & K \mbox{ for some context } K
\end{eqnarray*}

$u?(x)P | u!\langle Q \rangle \red P\{\quotep{Q}/x\}$

%We write $\wred$ for $\red^*$, and $P\red$ if $\exists Q $ such that $ P \red Q$.
We write $P\red$ if $\exists Q $ such that $ P \red Q$ and $P\not\red$, otherwise.

\section{Replication}

As mentioned before, it is known that replication (and hence
recursion) can be implemented in a higher-order process algebra
\cite{SangiorgiWalker}. As our first example of calculation with the
machinery thus far presented we give the construction explicitly in
the {\rhoc}.

\begin{eqnarray}
	D_{x} & := & \prefix{x}{y}{(\binpar{\outputp{x}{y}}{@{y}})} \nonumber\\
	\bangp_{x}{P} & := & \binpar{{x}!\langle{\binpar{D_{x}}{P}}\rangle}{D_{x}} \nonumber
\end{eqnarray}

\begin{eqnarray}
	\bangp_{x}{P} & & \nonumber\\
	=
	& {x}!\langle{(\prefix{x}{y}{(\outputp{x}{y} | @{y})) | P}}\rangle 
	      | \prefix{x}{y}{(\outputp{x}{y} | @{y})} & \nonumber\\
	\red
	& (\outputp{x}{y} | @{y})\substn{\quotep{(\prefix{x}{y}{(@{y} | \outputp{x}{y})) | P}}}{y} & \nonumber\\
	=
	& \outputp{x}{\quotep{(\prefix{x}{y}{(\outputp{x}{y} | @{y})) | P}}}
	  | {(\prefix{x}{y}{(\outputp{x}{y} | @{y})) | P}} & \nonumber\\
	\red
	& \ldots & \nonumber\\
	\red^*
	& P | P | \ldots & \nonumber
\end{eqnarray}

Of course, this encoding, as an implementation, runs away, unfolding
$\bangp{P}$ eagerly. A lazier and more implementable replication
operator, restricted to input-guarded processes, may be obtained as follows.

\begin{eqnarray}
\bangp{\prefix{u}{v}{P}} 
	:= 
	\binpar{\lift{x}{\prefix{u}{v}{(\binpar{D(x)}{P})}}}{D(x)} \nonumber
\end{eqnarray}

\begin{remark}
  Note that the lazier definition still does not deal with summation
  or mixed summation (i.e. sums over input and output). The reader is
  invited to construct definitions of replication that deal with these
  features. 

  Further, the definitions are parameterized in a name, $x$. Can you,
  gentle reader, make a definition that eliminates this parameter and
  guarantees no accidental interaction between the replication
  machinery and the process being replicated -- i.e. no accidental
  sharing of names used by the process to get its work done and the
  name(s) used by the replication to effect copying. This latter
  revision of the definition of replication is crucial to obtaining
  the expected identity $!!P \sim !P$.
\end{remark}

\begin{remark}\label{rem:paradoxical_combinator}
  The reader familiar with the lambda calculus will have noticed the
  similarity between $D$ and the paradoxical combinator.

  [Ed. note: the existence of this seems to suggest we have to be more
  restrictive on the set of processes and names we admit if we are to
  support no-cloning.]
\end{remark}

\subsubsection{Bisimulation}

The computational dynamics gives rise to another kind of equivalence,
the equivalence of computational behavior. As previously mentioned
this is typically captured \emph{via} some form of bisimulation.

% The notion we use in this paper is weak barbed bisimulation
% \cite{milner91polyadicpi}.

The notion we use in this paper is derived from weak barbed
bisimulation \cite{milner91polyadicpi}. 

\begin{definition}
An \emph{observation relation}, $\downarrow_{\mathcal N}$, over a set
of names, $\mathcal N$, is the smallest relation satisfying the rules
below.

\infrule[Out-barb]{y \in {\mathcal N}, \; x \nameeq y}
		  {\outputp{x}{v} \downarrow_{\mathcal N} x}
\infrule[Par-barb]{\mbox{$P\downarrow_{\mathcal N} x$ or $Q\downarrow_{\mathcal N} x$}}
		  {\binpar{P}{Q} \downarrow_{\mathcal N} x}

We write $P \Downarrow_{\mathcal N} x$ if there is $Q$ such that 
$P \wred Q$ and $Q \downarrow_{\mathcal N} x$.
\end{definition}

\begin{definition}
%\label{def.bbisim}
An  ${\mathcal N}$-\emph{barbed bisimulation} over a set of names, ${\mathcal N}$, is a symmetric binary relation 
${\mathcal S}_{\mathcal N}$ between agents such that $P\rel{S}_{\mathcal N}Q$ implies:
\begin{enumerate}
\item If $P \red P'$ then $Q \wred Q'$ and $P'\rel{S}_{\mathcal N} Q'$.
\item If $P\downarrow_{\mathcal N} x$, then $Q\Downarrow_{\mathcal N} x$.
\end{enumerate}
$P$ is ${\mathcal N}$-barbed bisimilar to $Q$, written
$P \wbbisim_{\mathcal N} Q$, if $P \rel{S}_{\mathcal N} Q$ for some ${\mathcal N}$-barbed bisimulation ${\mathcal S}_{\mathcal N}$.
\end{definition}

$\mathcal{R} \subseteq \pi \times \pi$

$P \mathcal{R} Q => \forall P'. P \red P' \Rightarrow \exists Q'. Q \red Q', P' \mathcal{R} Q'$

$P \vdash x \Rightarrow Q \vdash x$

\begin{mathpar}
  \inferrule*[lab=Out-barb]{x \nameeq y}{{y}!\langle{Q}\rangle \vdash x}
  \and
  \inferrule*[lab=Par-barb]{\mbox{$P\vdash x$ or $Q\vdash x$}}{\binpar{P}{Q} \vdash x}
\end{mathpar}

\subsubsection{Contexts}

One of the principle advantages of computational calculi like the
$\pi$-calculus is a well-defined notion of context,
contextual-equivalence and a correlation between
contextual-equivalence and notions of bisimulation. The notion of
context allows the decomposition of a process into (sub-)process and
its syntactic environment, its context. Thus, a context may be
thought of as a process with a ``hole'' (written $\Box$) in it. The
application of a context $M$ to a process $P$, written $M[P]$, is
tantamount to filling the hole in $M$ with $P$. In this paper we do
not need the full weight of this theory, but do make use of the notion
of context in the proof the main theorem. 

\begin{mathpar}
  \inferrule* [lab=summation] {} {{M_{M},M_{N}} \bc \Box \;|\; x.M_{A} \;|\; M_{M}+M_{N}}
  \and
  \inferrule* [lab=agent] {} {{M_{A}} \bc (\vec{x})M_{P} \;| \; \clift{P_0,\ldots,M_{P},\ldots,P_N}}
  \and \\
  \inferrule* [lab=process] {} {{M_{P}} \bc M_{N} \;| \;P|M_{P} }
\end{mathpar} 

\begin{mathpar}
  \inferrule* [lab=sychronization] {} {M_{N} \bc \Box \;|\; x?M_{F} \;|\; x!M_{C}}
  \and
  \inferrule* [lab=abstraction] {} {{M_{F}} \bc (x)M_{P} }
  \and
  \inferrule* [lab=concretion] {} {{M_{C}} \bc \langle M_{P} \rangle }
  \and \\
  \inferrule* [lab=process] {} {{M_{P}} \bc M_{N} \;| \;P|M_{P} }
\end{mathpar}

\begin{definition}[contextual application] Given a context $M$, and
  process $P$, we define the \emph{contextual application}, $M[P] :=
  M\{P/\Box\}$. That is, the contextual application of M to P is the
  substitution of $P$ for $\Box$ in $M$.
\end{definition}

$\meaningof{-} : L \to \mathcal{P}(\pi)$

\begin{mathpar}
  \inferrule* [lab=collection] {} {\meaningof{true} = \pi, \and \meaningof{~E} = \pi \setminus \meaningof{E}, \and \meaningof{E_{1} \& E_{2}} = \meaningof{E_{1}} \cap \meaningof{E_{2}}}
\end{mathpar}

\begin{mathpar}
  \inferrule* [lab=structure] {} {\meaningof{0} = \{ P \in \pi | P \equiv 0 \}, \and \\ \meaningof{E_1 | E_2} = \{ P \in \pi | P \equiv P_{1} | P_{2}, P_{1} \in \meaningof{E_{1}}, P_{2} \in \meaningof{E_2}\} }
\end{mathpar}

\begin{mathpar}
 \inferrule* [lab=behavior] {} {\meaningof{\langle a?b \rangle E} = \{ P \in \pi | P \equiv Q | u?(y)P', \\ \and \\\\ \and \\ \;\;\; u \in \meaningof{a}, \forall z.P'\{z/y\} \in \meaningof{E\{z/b\}}\}, \and \\ \meaningof{a!E} = \{ P \in \pi | P \equiv Q | x!\langle P' \rangle, x \in \meaningof{a} P' \in \meaningof{E}\} }
\end{mathpar}

\begin{mathpar}
 \inferrule* [lab=nominal] {} {\meaningof{\quotep{E}} = \{ \quotep{P} \in \quotep{\pi} | P \in \meaningof{E} \}, \and \meaningof{\quotep{P}} = \{ \quotep{Q} \in \quotep{\pi} | P \equiv Q \} \and \\ \meaningof{@\quotep{E}} = \{ P \in \pi | P \equiv @x, x \in \meaningof{E} \}}
\end{mathpar}

\begin{eqnarray*}
  \\
  \meaningof{-} : TS \to ST
\end{eqnarray*}

\begin{eqnarray*}
  \\
  L : TS \to ST
\end{eqnarray*}

\begin{eqnarray*}
  \\
  P \models E \iff P \in \meaningof{E}
\end{eqnarray*}

\begin{eqnarray*}
  P \approx_{L} Q \iff \forall E \in L. P \models E \iff Q \models E
\end{eqnarray*}

\begin{eqnarray*}
  P \approx_{K} Q
\end{eqnarray*}

\begin{eqnarray*}
  P \approx Q
\end{eqnarray*}

$\approx_{K} = \approx = \approx_{L}$

\subsubsection{Contextual duality}

Note that contexts extend the quotation operation to a family of
operations from processes to names. Given a context, $M$, we can
define a \emph{nominal context}, $\quotep{M}$ by $\quotep{M}[P] :=
\quotep{M[P]}$. To foreshadow what is to come we observe that these
operations enjoy a duality with processes very much like the duality
between vectors and maps from vectors to scalars.

Further, because the calculus is essentially higher-order, we have a
correspondence between contexts and processes. More specifically,
given a name $x$ and a context $M$ we can construct $M^{*}_{x}$ such
that 

\begin{mathpar}
  M^{*}_{x} | \lift{x}{P} \red M[P]
\end{mathpar}

namely,

\begin{mathpar}
  M^{*}_{x} := x?(u).M[\dropn{u}]
\end{mathpar}

The dependence of $M^{*}_{x}$ on a name makes it an abstraction, 

\begin{mathpar}
  M^{*} := (x)x?(u).M[\dropn{u}]
\end{mathpar}

\subsection{Additional notation}

It will sometimes be convenient to denote the process a name
quotes. We already have the notation $x = \quotep{P}$, but it will be
convenient to introduce an alternate notation, $\procn{x}$, when we
want to emphasize the connection to the use of the name. Note that, by
virtue of name equivalence, $\quotep{\procn{x}} \nameeq x$; so, the
notation is consistent with previous definitions.

Further, because names have structure it is possible to effect
substitutions on the basis of that structure. This means we need to
upgrade our notation for substitutions, which we accomplish by
adapting comprehension notation. Thus,

\begin{mathpar}
  P\{ y / x : x \in S \}
\end{mathpar}

is interpreted to mean the process derived from P by replacing (in a
capture-avoiding manner) each occurrence of $x$ in $S$ by $y$. For example,

\begin{mathpar}
  P\{ \quotep{\procn{x}|\procn{x}} / x : x \in \freenames{P} \}
\end{mathpar}

will replace each (occurrence) of a free name $x$ in $P$ by
$\quotep{\procn{x}|\procn{x}}$.

Also, we will avail ourselves of the notation $x^{L}$ and $x^{R}$ to
denote injections of a name into disjoint copies of the name
space. There are numerous ways to accomplish this. One example can be
found in \cite{MeredithR05}. This notation overloads to vectors of
names: $\vec{x}^{\pi} := (x_{i}^{\pi} \; : \; 0 \leq i < |\vec{x}| )$ where $\pi \in \{L,R\}$.

We also use $P^{\Box} := P|\Box$.

In \cite{MeredithR05} an interpretation of the new operator is
given. It turns out that there are several possible interpretations
all enjoying the requisite algebraic properties of the operator (see
\cite{milner91polyadicpi}). We will therefore make liberal use of
$(\nu\; \vec{x})P$.

% subsection the_syntax_and_semantics_of_the_notation_system (end)   

\section{Interpretation of QM}
\subsection{Supporting definitions}
\subsubsection{Multiplication}
\begin{mathpar}
  \quotep{Q} \cdot \quotep{R} := \quotep{Q|R}
  \and \\
  \quotep{Q} \cdot P := P\{ \quotep{Q|R} / \quotep{R} : \quotep{R} \in \freenames{P} \}
\end{mathpar}

\paragraph{Discussion}
The first line needs little explanation. The second line says that
each free name of the process is replaced with the multiplication of
that name by the scalar. Multiplication of a scalar (name) by a state
(process) results in a process all the names of which have been `moved
over' by parallel composition with the process the scalar
quotes. There is a subtlety that the bound names have to be
manipulated so that multiplied names aren't accidentally
captured. There are many ways to achieve this.

\begin{remark}\label{rem:multiplication_identities}
  The reader is invited to verify that for all $x,y,z \in \QProc$ and $P \in \Proc$
  \begin{mathpar}
    x \cdot \quotep{0} \equiv x 
    \and
    x \cdot y \equiv y \cdot x
    \and
    x \cdot (y \cdot z) \equiv (x \cdot y) \cdot z
    \and \\
    \quotep{0} \cdot P \equiv P
    \and \\
    x \cdot (y \cdot P) \equiv (x \cdot y) \cdot P
    \and \\
    x \cdot (P|Q) \equiv (x \cdot P) | (x \cdot Q)
    \and \\    
  \end{mathpar}
\end{remark}

\subsubsection{Tensor product}

We define a tensor product on processes by structural induction.

\paragraph{Tensor of sums} First note that all summations, including
$\pzero$ and sequence, can be written $\Sigma_{i} x_{i}.A_{i} +
\Sigma_{j} x_{j}.C_{j}$, where we have grouped input-guarded processes
together and output-guarded processes together.

Thus, we can define the tensor product of two summations, $N_{1}\otimes N_{2}$, where

\begin{mathpar}
  N_{1} := \Sigma_{i} x_{i}.A_{i} + \Sigma_{j} x_{j}.C_{j}
  \and
  N_{2} := \Sigma_{i'} y_{i'}.B_{i'} + \Sigma_{j'} y_{j'}.D_{j'} 
\end{mathpar}

as follows.

\begin{mathpar}
  \Sigma_{i} x_{i}.A_{i} + \Sigma_{j} x_{j}.C_{j} \otimes \Sigma_{i'}
  y_{i'}.B_{i'} + \Sigma_{j'} y_{j'}.D_{j'} 
  \and \\
  := \; \Sigma_{i} \Sigma_{i'} \quotep{\stackrel{\vee}{x_{i}}| \stackrel{\vee}{y_{i'}}}.(A_{i}\otimes B_{i'}) \; | \; \Sigma_{i'} \Sigma_{i} \quotep{\stackrel{\vee}{y_{i'}}|\stackrel{\vee}{x_{i}}}.(B_{i'}\otimes A_{i})
  \and
  \;\; | \;\; \Sigma_{j} \Sigma_{j'} \quotep{\stackrel{\vee}{x_{j}}|\stackrel{\vee}{y_{j'}}}.(A_{j}\otimes B_{j'}) \; | \; \Sigma_{j'} \Sigma_{j} \quotep{\stackrel{\vee}{y_{j'}}|\stackrel{\vee}{x_{j}}}.(B_{j'}\otimes A_{j})
\end{mathpar}

\begin{remark}
  Do we need to $x^{L}$ and $y^{R}$ for this construction as well?
\end{remark}

\paragraph{Tensor of parallel compositions} Next, we distribute tensor
over par.

\begin{mathpar}
  P_{1}|P_{2} \otimes Q_{1}|Q_{2} := (P_{1} \otimes Q_{1}) | (P_{1}
  \otimes Q_{2}) | (P_{2} \otimes Q_{1}) | (P_{2} \otimes Q_{2})
\end{mathpar}

\paragraph{Tensor with dropped names} We treat tensor of a
process with a dropped name as parallel composition.

\begin{mathpar}
  P \otimes \dropn{x} := P | \dropn{x}
\end{mathpar}

\paragraph{Tensor of agents}

Finally, we need to define tensor on agents. Note that the definition
of tensor on normal products only tensors inputs with inputs and
outputs with outputs. Thus, we only have to define the operation on
``homogeneous'' pairings.

\begin{mathpar}
  (\vec{x})P \otimes (\vec{y})Q
  \and \\
  := (x_{0}^{L}|y_{0}^{R},\ldots,x_{0}^{L}|y_{n}^{R},\ldots,x_{m}^{L}|y_{0}^{R},\ldots,x_{m}^{L}|y_{n}^R)(P\{ \vec{x}^{L}/\vec{x}\} \otimes Q \{ \vec{y}^{R}/\vec{y}\})
  \and \\
  \clift{\vec{P}} \otimes \clift{\vec{Q}}
  \and \\
  := \clift{P_{0}\otimes Q_{0},\ldots,P_{0}\otimes Q_{n},\ldots,P_{m}\otimes Q_{0},\ldots,P_{m}\otimes Q_{n}}
\end{mathpar}

\begin{remark}
  Observe that arities of tensored abstractions matches arities of
  tensored concretions if the original arities matched. Note also that
  the length of the arities corresponds to the increase in dimension
  we see in ordinary vector space tensor product.
\end{remark}

\begin{remark}
  Operationally, this definition distributes the tensor down to
  components ``linked'' by summation. Tensor over summation is
  intriguing in that it mixes names. Moreover, as a consequence of the
  way it mixes names we have the identities for all $x \in \QProc$ and
  $P,Q \in \Proc$

  \begin{mathpar}
    (x \cdot P) \otimes Q \equiv x \cdot (P \otimes Q) \equiv P \otimes (x \cdot Q)
    \and
    P \otimes \pzero \equiv P
  \end{mathpar}

  that the reader is invited to verify.
\end{remark}

\subsubsection{Annihilation}
\begin{mathpar}
  P^{\perp} := \{ Q | \forall R. P|Q \red^{*} R \Rightarrow R \red^{*} \pzero \}
  \and \\
  P^{\underline{\perp}} := \Sigma_{Q \in P^{\perp}} \quotep{Q}?(y).(\dropn{y}|Q) | \Sigma_{Q \in P^{\perp}} \quotep{Q}\clift{\Box}
\end{mathpar}

\paragraph{Discussion} The reader will note that $P^{\perp}$ is a
\emph{set} of processes, while $P^{\underline{\perp}}$ is a
\emph{context}. We call the set $P^{\perp}$ the \emph{annihilators} of
$P$. The parallel composition of a process in the annihilators of $P$
with $P$ will result in a process, the state space of which has all
paths eventually leading to $\pzero$. Execution may endure loops; but
under reasonable conditions of fairness (naturally guaranteed under
most notions of bisimulation) such a composite process cannot get
stuck in such a loop and will, eventually pop out and terminate.

The context $P^{\underline{\perp}}$ is ready and willing to ``take the
$P$ out of'' the process to which it is applied. It will effectively
transmit the code of the process to which it is applied to one of the
annihilators and run the process against it.

\subsubsection{Evaluation}
We fix $M$ a domain of fully abstract interpretation with an equality
coincident with bisimulation. We take $\meaningof{\cdot} : \Proc \to
M$ to be the map interpreting processes and $\nmeaningof{\cdot} : \M
\to Proc$ to be the map running the other way. Then we define

\begin{mathpar}
  \int P := \nmeaningof{\meaningof{P}}
\end{mathpar}

\paragraph{Discussion}
There are many fully abstract interpretations of Milner's
$\pi$-calculus. Any of them can be used as a basis for interpreting
the reflective calculus here. Equipped with such a domain it is
largely a matter of grinding through to check that the Yoneda
construction for the normalization-by-evaluation program can be
extended to this setting.

\begin{remark}
  The reader is invited to verify that $\int (P^{\underline{\perp}}[P]) = 0$.
\end{remark}

\subsection{Quantum mechanics}

Table \ref{tbl:core_qm_op_defns} gives the core operational definitions

\begin{table}[htp]\label{tbl:core_qm_op_defns}
  \center{
    \fbox{
      \begin{tabular}{c|c}
        quantum mechanics & process calculus \\
        \hline
        scalar & $x := \quotep{P}$ \\
        state vector & $\state{P} := P$ \\
        dual & $\state{P}^{*} := \event{P^{\underline{\perp}}} := \quotep{P^{\underline{\perp}}}[-]$ \\
        matrix & $ \Sigma_{\alpha} \state{P_{\alpha}}x_{\alpha}\event{Q_{\alpha}}$ \\
        vector addition & $\state{P} + \state{Q} := \state{P | Q}$ \\
        tensor product & $\state{P} \otimes \state{Q} := \state{P \otimes Q}$ \\
        inner product & $\innerprod{P}{Q} := \quotep{\int P^{\underline{\perp}}[Q]}$ \\
      \end{tabular}
    }
  }
  \caption{QM - operational definitions}
\end{table}

where

\begin{mathpar}
  \prmatrix{P}{Q} := \fprmatrix{P}{\quotep{\pzero}}{Q}
  \and
  \fprmatrix{P}{x}{Q} := (\state{P},x,\event{Q})
  \and
  (\fprmatrix{P}{x}{Q})(\state{R}) := x \cdot \innerprod{Q}{R} \cdot \state{P}
  \and
  (\fprmatrix{P}{x}{Q})(\event{R}) := x \cdot \innerprod{R}{P} \cdot \event{Q}
\end{mathpar}

\paragraph{Discussion}
As promised: vectors (aka states) are represented as processes; duals
as contextual duals; inner product definition should be compared with
standard inner product definition for ....

\begin{remark}
  Assuming $\int (P^{\underline{\perp}}[P]) = 0$, the reader is
  invited to verify that $(\fprmatrix{P}{x}{P})(\state{P}) = x \cdot \state{P}$.
\end{remark}

\begin{remark}
  The reader is invited to verify that $\innerprod{P}{Q}$ could
  equally well have been written $\quotep{\int \stackrel{\vee}{x}}$
  where $x = \event{P^{\underline{\perp}}}(Q)$.

  One of the motivations for this remark is that there is another way
  to factor these operations. We could package up evaluation in the dual:

  \begin{mathpar}
    \state{P}^{*} := \event{\int P^{\underline{\perp}}} := \quotep{\int P^{\underline{\perp}}}[-]
  \end{mathpar}

  and then have inner product defined by
  
  \begin{mathpar}
    \innerprod{P}{Q} := \event{P}(Q)
  \end{mathpar}

  Hopefully, experience with the calculations will provide guidance on
  the best factoring.
\end{remark}

\begin{remark}
  Assuming $\int (P^{\underline{\perp}}[P]) = 0$, the reader is
  invited to verify that $\forall P,Q. (\prmatrix{0}{Q})(\state{0}) =
  \state{0}$ and dually $(\prmatrix{P}{0})(\event{0}) = \event{0}$.
\end{remark}

\begin{remark}
  i'm a little worried that i don't (yet) have proper support for
  complex conjugacy. But, the observation above may give us a
  clue. According to Abramsky, it must be the case that the scalars
  are iso to the homset of the identity for the tensor -- which the
  observation above characterizes. 

  For now, we will simply bookmark the notion with $\overline{x}$.
\end{remark}

\subsubsection{Adjointness}

We need to give a definition of $(\cdot)^{\dagger}$ for matrices. The
obvious candidate definition is
\begin{mathpar}
(\Sigma_{\alpha}\fprmatrix{P_{\alpha}}{x_{\alpha}}{Q_{\alpha}})^{\dagger}
= \Sigma_{\alpha}\fprmatrix{(Q_{\alpha}^{\underline{\perp}})^{*}}{\overline{x}_{\alpha}}{P_{\alpha}^{\underline{\perp}}} 
\end{mathpar}

But, $(Q_{\alpha}^{\underline{\perp}})^{*}$ requires a name along
which to communicate the process to achieve the context application.

\subsubsection{Basis for a basis}
If processes label states and ``addition'' of states (a.k.a. vector
addition) is interpreted as parallel composition, what corresponds to
notions of linear independence and basis? Here, we recall that Yoshida
has developed a set of \emph{combinators} for an asynchronous verison
of Milner's $\pi$-calculus. These are a finite set of processes such
any process can be expressed as parallel composition of these
combinators together with liberal uses of the new operator and
replication. We can simply give a translation of these into the
present calculus and have reasonable expectation that the property
carries over. That is, that the resultant set allows to express all
processes via parallel composition. Note, however, that there is no
new operator or replication in this calculus. As a result, we expect
that the corresponding set is actually infinite. That is, we expect
that the space is actually infinite dimensional.

\begin{remark}
  The attentive reader may be a bit concerned. Certainly, the
  collection $S$, $K$ and $I$ is a finite set of
  combinators. Shouldn't we expect to see a finite set of combinators
  for an effectively equivalent system? i am very sympathetic to this
  critique and feel it warrants full attention. On the other hand, i
  also have in mind the following analogy. The natural numbers, as a
  monoid under addition, has exactly $1$ generator, while the natural
  numbers, as a monoid under multiplication, has countably many
  generators (the primes). We observe that the application of the
  lambda calculus is much less resource sensitive than the parallel
  composition of the $\pi$-calculus. Could it be the case that we have
  an analogy of the form
  
  \begin{mathpar}
    m + n : MN :: m*n : M|N
  \end{mathpar}

  giving a similar blow up in the set of ``primes''?  This is such a
  wonderful thought that, even if it's not true, i think it's worth
  writing down.
\end{remark}
 

\documentclass[12pt]{llncs}
%\documentclass{jktr}

\usepackage[pdftex]{hyperref}                   
\usepackage {listings}
\usepackage {mathpartir}
\usepackage{bcprules}
%\usepackage{listings}
                       
\usepackage{graphicx} 
%\usepackage[margins=2.5cm,nohead,nofoot]{geometry}
%\usepackage{geometry}
\usepackage{amsfonts}
\usepackage{amstext}
\usepackage{latexsym}
\usepackage{amssymb}
\usepackage{color}


%\include{myPreamble}
\include{qm2pi.local} 

%\ifpdf
%\usepackage[pdftex]{graphicx}
%\else
%\usepackage{graphicx}
%\fi

 % \ifpdf
%  \usepackage{pdfsync}
%  \if


%\title{Brief Article}
%\author{David F. Snyder}
%\author{L.G. Meredith}

%\address{Dept. of Math., Texas State University--San Marcos, San Marcos, TX 78666}
       
\pagestyle{empty}


\begin{document}

\lstset{language=[Objective]Caml,frame=shadowbox}

\input{qm2pi.front}

% section front matter (end)

\input{qm2pi.intro} 
 
% section introduction (end)

% \input{qm2pi.knotations} 

% section notation (end)

\input{qm2pi.process.calculi} 

% section concurrent_process_calculi_and_spatial_logics_ (end)
    
%\input{qm2pi.knots2pi} 

%\input{qm2pi.trefoil} 

%\input{qm2pi.mainthm} 

% subsection basic_interpretation (end)

%\input{qm2pi.rho.presentation} 
\subsection{The syntax and semantics of the notation system}\label{sub:the_syntax_and_semantics_of_the_notation_system} % (fold)

We now summarize a technical presentation of the calculus that
embodies our theory of dynamics. The typical presentation of such a
calculus follows the style of giving generators and relations on
them. The grammar, below, describing term constructors, freely
generates the set of processes, $\Proc$. This set is then quotiented
by a relation known as structural congruence and it is over this set
that the notion of dynamics is expressed. This presentation is
essentially that of \cite{MeredithR05} with the addition of
polyadicity and summation. For readability we have relegated some of
the technical subtleties to an appendix.

\subsubsection{Process grammar}\label{subsub:process_grammar}

\begin{mathpar}
  \inferrule* [lab=synchronization] {} {{M} \bc \pzero \;|\; x?F \;|\; x!C }
  \and
  \inferrule* [lab=abstraction] {} {{F} \bc (x)P}
  \and
  \inferrule* [lab=concretion] {} {{C} \bc \langle Q \rangle}
  \and
  \inferrule* [lab=process] {} {{P,Q} \bc M \;| \;P|Q \;|\; @{x}}
  \and
  \inferrule* [lab=name] {} {{x} \bc \quotep{P}}
\end{mathpar} 

Note that $\vec{x}$ (resp. $\vec{P}$) denotes a vector of names
(resp. processes) of length $|\vec{x}|$ (resp. $|\vec{P}|$). We adopt
the following useful abbreviations.

\begin{mathpar}
   x?(\vec{y}).P := x.(\vec{y})P \and  x\clift{\vec{P}} := x.\clift{\vec{P}}
   \and x!(y) := \lift{x}{\dropn{y}}
   \and \Pi_{i=0}^{n-1}P_i := P_0 | \ldots | P_{n-1}
\end{mathpar}

\subsubsection{Structural congruence}

\paragraph{Free and bound names and alpha-equivalence.} At the
core of structural equivalence is alpha-equivalence which identifies
process that are the same up to a change of variable. Formally, we
recognize the distinction between free and bound names. The free names
of a process, $\freenames{P}$, may be calculated recursively as
follows:

\begin{mathpar}
\freenames{\pzero} := \emptyset
  \and \\
  \freenames{x?(y).P} := \{ x \} \cup (\freenames{P} \setminus \{ y \})
  \and 
  \freenames{x!\langle P \rangle} := \{ x \} \cup \{ P \} 
  \and \\
  \freenames{P|Q} := \freenames{P} \cup \freenames{Q}
  \and \\
  \freenames{@{x}} := \{ x \}
\end{mathpar}

$\pi$
$\quotep{\pi}$

$\freenames{-} : \pi \to \mathcal{P}(\quotep{\pi})$

\begin{eqnarray*}
  \freenames{\pzero} & := & \emptyset \\
  \freenames{x?(y).P} & := & \{ x \} \cup (\freenames{P} \setminus \{ y \}) \\
  \freenames{x!\langle P \rangle} & := & \{ x \} \cup \{ P \} \\
  \freenames{P|Q} & := & \freenames{P} \cup \freenames{Q} \\
  \freenames{\dropn{x}} & := & \{ x \}
\end{eqnarray*}

The bound names of a process, $\boundnames{P}$, are those names occurring in $P$
that are not free. For example, in $x?(y).0$, the name $x$ is free, while $y$ is bound.

\begin{mathpar}
  \inferrule* [lab=monoidal-laws] {} { P|Q \equiv Q|P \and P|0 \equiv P \and P|(Q|R) \equiv (P|Q)|R }
\end{mathpar}

\begin{mathpar}
  \inferrule* [lab=alpha-equivalence] {} { (x)P \equiv (y)P\{y/x\} \and y \not\in \freenames{P} }
\end{mathpar}

\begin{definition}
Then two processes, $P,Q$, are alpha-equivalent if $P = Q\{\vec{y}/\vec{x}\}$ for
some $\vec{x} \in \boundnames{Q},\vec{y} \in \boundnames{P}$, where $Q\{\vec{y}/\vec{x}\}$
denotes the capture-avoiding substitution of $\vec{y}$ for $\vec{x}$ in $Q$.
\end{definition}

\begin{definition}
  The {\em structural congruence} \cite{SangiorgiWalker} , $\equiv$,
  between processes is the least congruence containing
  alpha-equivalence, satisfying the abelian monoid laws
  (associativity, commutativity and $\pzero$ as identity) for parallel
  composition $|$ and for summation $+$.
\end{definition}

\subsection{Name equivalence}

We take name equivalence, written $\nameeq$, to be the smallest
equivalence relation generated by the following rules.

\begin{mathpar}
\inferrule*[lab=Quote-drop]
{ }
{ \quotep{@{x}} \nameeq x }

\inferrule*[lab=Struct-equiv]
{ P \scong Q }
{ \quotep{P} \nameeq \quotep{Q} }
\end{mathpar}

The astute reader will have noticed that the mutual recursion of names
and processes imposes a mutual recursion on alpha-equivalence and
structural equivalence via name-equivalence. Fortunately, all of this
works out pleasantly and we may calculate in the natural way, free of
concern. The reader interested in the details is referred to the
appendix \ref{appendix:rho_details}.

\subsection{Substitution}

We use $\Proc$ for the set of processes, $\QProc$ for the set of
names, and $\id{\{}\vec{y} / \vec{x} \id{\}}$ to denote partial maps,
$s : \QProc \rightarrow \QProc$. A map, $s$ lifts, uniquely, to a map
on process terms, $\widehat{s} : \Proc \rightarrow \Proc$ by the
following equations.

\begin{mathpar}
  (0) \psubstp{Q}{P} := 0 \\
  (R \juxtap S) \psubstp{Q}{P}
  :=    
  (R)\psubstp{Q}{P} \juxtap (S) \psubstp{Q}{P} \\
  (x?(y).R) \psubstp{Q}{P}    
  :=    
  (x)\substp{Q}{P} (z)\concat( (R \psubstn{z}{y}) \psubstp{Q}{P} ) \\
  (\lift{x}{R}) \psubstp{Q}{P}  
  :=
  \lift{(x)\substp{Q}{P}}{ R \psubstp{Q}{P} } \\
%   (\dropn{x})  \psubstp{Q}{P}       
%   := 
%   \left\{ 
%     \begin{array}{ccc} 
%       \dropn{\quotep{Q}} & & x \nameeq \quotep{P} \\
%       \dropn{x} & & otherwise \\
%     \end{array}
%   \right. 
  (\dropn{x})  \psubstp{Q}{P}       
  := 
  \left\{ 
    \begin{array}{ccc} 
      Q & & x \nameeq \quotep{P} \\
      \dropn{x} & & otherwise \\
    \end{array}
  \right.
\end{mathpar}
 

where

\begin{eqnarray}
  (x)\id{\{} \lpquote Q \rpquote / \lpquote P \rpquote \id{\}}            = 
  \left\{ 
    \begin{array}{ccc}
      \lpquote Q \rpquote & & x \nameeq \lpquote P \rpquote \\
      x & & otherwise \\
    \end{array}
  \right. \nonumber
\end{eqnarray}

and $z$ is chosen distinct from $\quotep{P}$, $\quotep{Q}$, the free
names in $Q$, and all the names in $R$. Our $\alpha$-equivalence will
be built in the standard way from this substitution.

\begin{remark}\label{rem:no_self_referential_names}
  One consequence of these definitions is that $\forall P. \quotep{P}
  \not\in \freenames{P}$.
\end{remark}

\subsection{ Dynamic quote: an example }

Anticipating something of what's to come, consider applying the
substitution, $\widehat{\id{\{}u / z \id{\}}}$, to the following pair
of processes, $\lift{w}{y!(z)}$ and $w[ \lpquote y!(z) \rpquote ]$.

\begin{eqnarray}
	\lift{w}{y!(z)}\widehat{\id{\{}u / z \id{\}}}
		& = &
		\lift{w}{y!(u)} \nonumber\\
	w[ \lpquote y!(z) \rpquote ] \widehat{ \id{\{}u / z \id{\}} }
		& = &
		w[ \lpquote y!(z) \rpquote ] \nonumber
\end{eqnarray}

Because the body of the process between quotes is impervious to
substitution, we get radically different answers. In fact, by
examining the first process in an input context,
e.g. $x?(z).\lift{w}{y!(z)}$, we see that the process under the lift
operator may be shaped by prefixed inputs binding a name inside it. In
this sense, the lift operator will be seen as a way to dynamically
construct processes before reifying them as names.

Finally equipped with these standard features we can present the
dynamics of the calculus.

\subsubsection{Operational semantics} 

Finally, we introduce the computational dynamics. What marks these
algebras as distinct from other more traditionally studied algebraic
structures, e.g. vector spaces or polynomial rings, is the manner in
which dynamics is captured. In traditional structures, dynamics is typically
expressed through morphisms between such structures, as in linear maps
between vector spaces or morphisms between rings. In algebras
associated with the semantics of computation, the dynamics is
expressed as part of the algebraic structure itself, through a
reduction reduction relation typically denoted by $\red$. Below, we
give a recursive presentation of this relation for the calculus used
in the encoding.

$\red \subseteq \pi \times \pi$
$\red : \pi \to \mathcal{P}(\pi)$

\begin{mathpar}
  \inferrule* [lab=Comm] { \textsf{match}( x_{src}, x_{trgt} ) } { x_{trgt}?(y)P \; | \; x_{src}!\langle {Q} \rangle \red P\{\quotep{Q}/y}\} }
  \and \\
  \inferrule* [lab=Par] {{P} \red {P}'} {{{P} | {Q}} \red {{P}' | {Q}}}
  \and
  \inferrule* [lab=Equiv]{{{P} \scong {P}'} \andalso {{P}' \red {Q}'} \andalso {{Q}' \scong {Q}}}{{P} \red {Q}}
\end{mathpar}

\begin{eqnarray*}
  match_{\equiv} (\quotep{P},\quotep{Q}) & := & P \equiv Q \\
  match_{\dagger}(\quotep{P},\quotep{Q}) & := & \forall R. P|Q \red^{*} R => R \red^{*} 0 \\
  match_{K}(\quotep{P},\quotep{Q}) & := & K \mbox{ for some context } K
\end{eqnarray*}

$u?(x)P | u!\langle Q \rangle \red P\{\quotep{Q}/x\}$

%We write $\wred$ for $\red^*$, and $P\red$ if $\exists Q $ such that $ P \red Q$.
We write $P\red$ if $\exists Q $ such that $ P \red Q$ and $P\not\red$, otherwise.

\section{Replication}

As mentioned before, it is known that replication (and hence
recursion) can be implemented in a higher-order process algebra
\cite{SangiorgiWalker}. As our first example of calculation with the
machinery thus far presented we give the construction explicitly in
the {\rhoc}.

\begin{eqnarray}
	D_{x} & := & \prefix{x}{y}{(\binpar{\outputp{x}{y}}{@{y}})} \nonumber\\
	\bangp_{x}{P} & := & \binpar{{x}!\langle{\binpar{D_{x}}{P}}\rangle}{D_{x}} \nonumber
\end{eqnarray}

\begin{eqnarray}
	\bangp_{x}{P} & & \nonumber\\
	=
	& {x}!\langle{(\prefix{x}{y}{(\outputp{x}{y} | @{y})) | P}}\rangle 
	      | \prefix{x}{y}{(\outputp{x}{y} | @{y})} & \nonumber\\
	\red
	& (\outputp{x}{y} | @{y})\substn{\quotep{(\prefix{x}{y}{(@{y} | \outputp{x}{y})) | P}}}{y} & \nonumber\\
	=
	& \outputp{x}{\quotep{(\prefix{x}{y}{(\outputp{x}{y} | @{y})) | P}}}
	  | {(\prefix{x}{y}{(\outputp{x}{y} | @{y})) | P}} & \nonumber\\
	\red
	& \ldots & \nonumber\\
	\red^*
	& P | P | \ldots & \nonumber
\end{eqnarray}

Of course, this encoding, as an implementation, runs away, unfolding
$\bangp{P}$ eagerly. A lazier and more implementable replication
operator, restricted to input-guarded processes, may be obtained as follows.

\begin{eqnarray}
\bangp{\prefix{u}{v}{P}} 
	:= 
	\binpar{\lift{x}{\prefix{u}{v}{(\binpar{D(x)}{P})}}}{D(x)} \nonumber
\end{eqnarray}

\begin{remark}
  Note that the lazier definition still does not deal with summation
  or mixed summation (i.e. sums over input and output). The reader is
  invited to construct definitions of replication that deal with these
  features. 

  Further, the definitions are parameterized in a name, $x$. Can you,
  gentle reader, make a definition that eliminates this parameter and
  guarantees no accidental interaction between the replication
  machinery and the process being replicated -- i.e. no accidental
  sharing of names used by the process to get its work done and the
  name(s) used by the replication to effect copying. This latter
  revision of the definition of replication is crucial to obtaining
  the expected identity $!!P \sim !P$.
\end{remark}

\begin{remark}\label{rem:paradoxical_combinator}
  The reader familiar with the lambda calculus will have noticed the
  similarity between $D$ and the paradoxical combinator.

  [Ed. note: the existence of this seems to suggest we have to be more
  restrictive on the set of processes and names we admit if we are to
  support no-cloning.]
\end{remark}

\subsubsection{Bisimulation}

The computational dynamics gives rise to another kind of equivalence,
the equivalence of computational behavior. As previously mentioned
this is typically captured \emph{via} some form of bisimulation.

% The notion we use in this paper is weak barbed bisimulation
% \cite{milner91polyadicpi}.

The notion we use in this paper is derived from weak barbed
bisimulation \cite{milner91polyadicpi}. 

\begin{definition}
An \emph{observation relation}, $\downarrow_{\mathcal N}$, over a set
of names, $\mathcal N$, is the smallest relation satisfying the rules
below.

\infrule[Out-barb]{y \in {\mathcal N}, \; x \nameeq y}
		  {\outputp{x}{v} \downarrow_{\mathcal N} x}
\infrule[Par-barb]{\mbox{$P\downarrow_{\mathcal N} x$ or $Q\downarrow_{\mathcal N} x$}}
		  {\binpar{P}{Q} \downarrow_{\mathcal N} x}

We write $P \Downarrow_{\mathcal N} x$ if there is $Q$ such that 
$P \wred Q$ and $Q \downarrow_{\mathcal N} x$.
\end{definition}

\begin{definition}
%\label{def.bbisim}
An  ${\mathcal N}$-\emph{barbed bisimulation} over a set of names, ${\mathcal N}$, is a symmetric binary relation 
${\mathcal S}_{\mathcal N}$ between agents such that $P\rel{S}_{\mathcal N}Q$ implies:
\begin{enumerate}
\item If $P \red P'$ then $Q \wred Q'$ and $P'\rel{S}_{\mathcal N} Q'$.
\item If $P\downarrow_{\mathcal N} x$, then $Q\Downarrow_{\mathcal N} x$.
\end{enumerate}
$P$ is ${\mathcal N}$-barbed bisimilar to $Q$, written
$P \wbbisim_{\mathcal N} Q$, if $P \rel{S}_{\mathcal N} Q$ for some ${\mathcal N}$-barbed bisimulation ${\mathcal S}_{\mathcal N}$.
\end{definition}

$\mathcal{R} \subseteq \pi \times \pi$

$P \mathcal{R} Q => \forall P'. P \red P' \Rightarrow \exists Q'. Q \red Q', P' \mathcal{R} Q'$

$P \vdash x \Rightarrow Q \vdash x$

\begin{mathpar}
  \inferrule*[lab=Out-barb]{x \nameeq y}{{y}!\langle{Q}\rangle \vdash x}
  \and
  \inferrule*[lab=Par-barb]{\mbox{$P\vdash x$ or $Q\vdash x$}}{\binpar{P}{Q} \vdash x}
\end{mathpar}

\subsubsection{Contexts}

One of the principle advantages of computational calculi like the
$\pi$-calculus is a well-defined notion of context,
contextual-equivalence and a correlation between
contextual-equivalence and notions of bisimulation. The notion of
context allows the decomposition of a process into (sub-)process and
its syntactic environment, its context. Thus, a context may be
thought of as a process with a ``hole'' (written $\Box$) in it. The
application of a context $M$ to a process $P$, written $M[P]$, is
tantamount to filling the hole in $M$ with $P$. In this paper we do
not need the full weight of this theory, but do make use of the notion
of context in the proof the main theorem. 

\begin{mathpar}
  \inferrule* [lab=summation] {} {{M_{M},M_{N}} \bc \Box \;|\; x.M_{A} \;|\; M_{M}+M_{N}}
  \and
  \inferrule* [lab=agent] {} {{M_{A}} \bc (\vec{x})M_{P} \;| \; \clift{P_0,\ldots,M_{P},\ldots,P_N}}
  \and \\
  \inferrule* [lab=process] {} {{M_{P}} \bc M_{N} \;| \;P|M_{P} }
\end{mathpar} 

\begin{mathpar}
  \inferrule* [lab=sychronization] {} {M_{N} \bc \Box \;|\; x?M_{F} \;|\; x!M_{C}}
  \and
  \inferrule* [lab=abstraction] {} {{M_{F}} \bc (x)M_{P} }
  \and
  \inferrule* [lab=concretion] {} {{M_{C}} \bc \langle M_{P} \rangle }
  \and \\
  \inferrule* [lab=process] {} {{M_{P}} \bc M_{N} \;| \;P|M_{P} }
\end{mathpar}

\begin{definition}[contextual application] Given a context $M$, and
  process $P$, we define the \emph{contextual application}, $M[P] :=
  M\{P/\Box\}$. That is, the contextual application of M to P is the
  substitution of $P$ for $\Box$ in $M$.
\end{definition}

$\meaningof{-} : L \to \mathcal{P}(\pi)$

\begin{mathpar}
  \inferrule* [lab=collection] {} {\meaningof{true} = \pi, \and \meaningof{~E} = \pi \setminus \meaningof{E}, \and \meaningof{E_{1} \& E_{2}} = \meaningof{E_{1}} \cap \meaningof{E_{2}}}
\end{mathpar}

\begin{mathpar}
  \inferrule* [lab=structure] {} {\meaningof{0} = \{ P \in \pi | P \equiv 0 \}, \and \\ \meaningof{E_1 | E_2} = \{ P \in \pi | P \equiv P_{1} | P_{2}, P_{1} \in \meaningof{E_{1}}, P_{2} \in \meaningof{E_2}\} }
\end{mathpar}

\begin{mathpar}
 \inferrule* [lab=behavior] {} {\meaningof{\langle a?b \rangle E} = \{ P \in \pi | P \equiv Q | u?(y)P', \\ \and \\\\ \and \\ \;\;\; u \in \meaningof{a}, \forall z.P'\{z/y\} \in \meaningof{E\{z/b\}}\}, \and \\ \meaningof{a!E} = \{ P \in \pi | P \equiv Q | x!\langle P' \rangle, x \in \meaningof{a} P' \in \meaningof{E}\} }
\end{mathpar}

\begin{mathpar}
 \inferrule* [lab=nominal] {} {\meaningof{\quotep{E}} = \{ \quotep{P} \in \quotep{\pi} | P \in \meaningof{E} \}, \and \meaningof{\quotep{P}} = \{ \quotep{Q} \in \quotep{\pi} | P \equiv Q \} \and \\ \meaningof{@\quotep{E}} = \{ P \in \pi | P \equiv @x, x \in \meaningof{E} \}}
\end{mathpar}

\begin{eqnarray*}
  \\
  \meaningof{-} : TS \to ST
\end{eqnarray*}

\begin{eqnarray*}
  \\
  L : TS \to ST
\end{eqnarray*}

\begin{eqnarray*}
  \\
  P \models E \iff P \in \meaningof{E}
\end{eqnarray*}

\begin{eqnarray*}
  P \approx_{L} Q \iff \forall E \in L. P \models E \iff Q \models E
\end{eqnarray*}

\begin{eqnarray*}
  P \approx_{K} Q
\end{eqnarray*}

\begin{eqnarray*}
  P \approx Q
\end{eqnarray*}

$\approx_{K} = \approx = \approx_{L}$

\subsubsection{Contextual duality}

Note that contexts extend the quotation operation to a family of
operations from processes to names. Given a context, $M$, we can
define a \emph{nominal context}, $\quotep{M}$ by $\quotep{M}[P] :=
\quotep{M[P]}$. To foreshadow what is to come we observe that these
operations enjoy a duality with processes very much like the duality
between vectors and maps from vectors to scalars.

Further, because the calculus is essentially higher-order, we have a
correspondence between contexts and processes. More specifically,
given a name $x$ and a context $M$ we can construct $M^{*}_{x}$ such
that 

\begin{mathpar}
  M^{*}_{x} | \lift{x}{P} \red M[P]
\end{mathpar}

namely,

\begin{mathpar}
  M^{*}_{x} := x?(u).M[\dropn{u}]
\end{mathpar}

The dependence of $M^{*}_{x}$ on a name makes it an abstraction, 

\begin{mathpar}
  M^{*} := (x)x?(u).M[\dropn{u}]
\end{mathpar}

\subsection{Additional notation}

It will sometimes be convenient to denote the process a name
quotes. We already have the notation $x = \quotep{P}$, but it will be
convenient to introduce an alternate notation, $\procn{x}$, when we
want to emphasize the connection to the use of the name. Note that, by
virtue of name equivalence, $\quotep{\procn{x}} \nameeq x$; so, the
notation is consistent with previous definitions.

Further, because names have structure it is possible to effect
substitutions on the basis of that structure. This means we need to
upgrade our notation for substitutions, which we accomplish by
adapting comprehension notation. Thus,

\begin{mathpar}
  P\{ y / x : x \in S \}
\end{mathpar}

is interpreted to mean the process derived from P by replacing (in a
capture-avoiding manner) each occurrence of $x$ in $S$ by $y$. For example,

\begin{mathpar}
  P\{ \quotep{\procn{x}|\procn{x}} / x : x \in \freenames{P} \}
\end{mathpar}

will replace each (occurrence) of a free name $x$ in $P$ by
$\quotep{\procn{x}|\procn{x}}$.

Also, we will avail ourselves of the notation $x^{L}$ and $x^{R}$ to
denote injections of a name into disjoint copies of the name
space. There are numerous ways to accomplish this. One example can be
found in \cite{MeredithR05}. This notation overloads to vectors of
names: $\vec{x}^{\pi} := (x_{i}^{\pi} \; : \; 0 \leq i < |\vec{x}| )$ where $\pi \in \{L,R\}$.

We also use $P^{\Box} := P|\Box$.

In \cite{MeredithR05} an interpretation of the new operator is
given. It turns out that there are several possible interpretations
all enjoying the requisite algebraic properties of the operator (see
\cite{milner91polyadicpi}). We will therefore make liberal use of
$(\nu\; \vec{x})P$.

% subsection the_syntax_and_semantics_of_the_notation_system (end)   

\input{qm2pi.qmops} 

\input{qm2pi.sterngerlach} 

\input{qm2pi.metric} 

% section concurrent_process_calculi (end)

%\input{qm2pi.proofsketch}

% section proof sketch (end)

%\input{qm2pi.slviaknots} 

% section spatial logic via knots (end)

\input{qm2pi.conclusion}

% section conclusion (end)

%\input{qm2pi.dtcodes} 

% section wiring algorithm (end)

\input{qm2pi.ack} 

% section acknowledgments (end)

\newpage


\bibliographystyle{plain}   
\bibliography{../../biblios/main.bib}

\input{qm2pi.rhodetails}

\end{document}

 

\documentclass[12pt]{llncs}
%\documentclass{jktr}

\usepackage[pdftex]{hyperref}                   
\usepackage {listings}
\usepackage {mathpartir}
\usepackage{bcprules}
%\usepackage{listings}
                       
\usepackage{graphicx} 
%\usepackage[margins=2.5cm,nohead,nofoot]{geometry}
%\usepackage{geometry}
\usepackage{amsfonts}
\usepackage{amstext}
\usepackage{latexsym}
\usepackage{amssymb}
\usepackage{color}


%\include{myPreamble}
\include{qm2pi.local} 

%\ifpdf
%\usepackage[pdftex]{graphicx}
%\else
%\usepackage{graphicx}
%\fi

 % \ifpdf
%  \usepackage{pdfsync}
%  \if


%\title{Brief Article}
%\author{David F. Snyder}
%\author{L.G. Meredith}

%\address{Dept. of Math., Texas State University--San Marcos, San Marcos, TX 78666}
       
\pagestyle{empty}


\begin{document}

\lstset{language=[Objective]Caml,frame=shadowbox}

\input{qm2pi.front}

% section front matter (end)

\input{qm2pi.intro} 
 
% section introduction (end)

% \input{qm2pi.knotations} 

% section notation (end)

\input{qm2pi.process.calculi} 

% section concurrent_process_calculi_and_spatial_logics_ (end)
    
%\input{qm2pi.knots2pi} 

%\input{qm2pi.trefoil} 

%\input{qm2pi.mainthm} 

% subsection basic_interpretation (end)

%\input{qm2pi.rho.presentation} 
\subsection{The syntax and semantics of the notation system}\label{sub:the_syntax_and_semantics_of_the_notation_system} % (fold)

We now summarize a technical presentation of the calculus that
embodies our theory of dynamics. The typical presentation of such a
calculus follows the style of giving generators and relations on
them. The grammar, below, describing term constructors, freely
generates the set of processes, $\Proc$. This set is then quotiented
by a relation known as structural congruence and it is over this set
that the notion of dynamics is expressed. This presentation is
essentially that of \cite{MeredithR05} with the addition of
polyadicity and summation. For readability we have relegated some of
the technical subtleties to an appendix.

\subsubsection{Process grammar}\label{subsub:process_grammar}

\begin{mathpar}
  \inferrule* [lab=synchronization] {} {{M} \bc \pzero \;|\; x?F \;|\; x!C }
  \and
  \inferrule* [lab=abstraction] {} {{F} \bc (x)P}
  \and
  \inferrule* [lab=concretion] {} {{C} \bc \langle Q \rangle}
  \and
  \inferrule* [lab=process] {} {{P,Q} \bc M \;| \;P|Q \;|\; @{x}}
  \and
  \inferrule* [lab=name] {} {{x} \bc \quotep{P}}
\end{mathpar} 

Note that $\vec{x}$ (resp. $\vec{P}$) denotes a vector of names
(resp. processes) of length $|\vec{x}|$ (resp. $|\vec{P}|$). We adopt
the following useful abbreviations.

\begin{mathpar}
   x?(\vec{y}).P := x.(\vec{y})P \and  x\clift{\vec{P}} := x.\clift{\vec{P}}
   \and x!(y) := \lift{x}{\dropn{y}}
   \and \Pi_{i=0}^{n-1}P_i := P_0 | \ldots | P_{n-1}
\end{mathpar}

\subsubsection{Structural congruence}

\paragraph{Free and bound names and alpha-equivalence.} At the
core of structural equivalence is alpha-equivalence which identifies
process that are the same up to a change of variable. Formally, we
recognize the distinction between free and bound names. The free names
of a process, $\freenames{P}$, may be calculated recursively as
follows:

\begin{mathpar}
\freenames{\pzero} := \emptyset
  \and \\
  \freenames{x?(y).P} := \{ x \} \cup (\freenames{P} \setminus \{ y \})
  \and 
  \freenames{x!\langle P \rangle} := \{ x \} \cup \{ P \} 
  \and \\
  \freenames{P|Q} := \freenames{P} \cup \freenames{Q}
  \and \\
  \freenames{@{x}} := \{ x \}
\end{mathpar}

$\pi$
$\quotep{\pi}$

$\freenames{-} : \pi \to \mathcal{P}(\quotep{\pi})$

\begin{eqnarray*}
  \freenames{\pzero} & := & \emptyset \\
  \freenames{x?(y).P} & := & \{ x \} \cup (\freenames{P} \setminus \{ y \}) \\
  \freenames{x!\langle P \rangle} & := & \{ x \} \cup \{ P \} \\
  \freenames{P|Q} & := & \freenames{P} \cup \freenames{Q} \\
  \freenames{\dropn{x}} & := & \{ x \}
\end{eqnarray*}

The bound names of a process, $\boundnames{P}$, are those names occurring in $P$
that are not free. For example, in $x?(y).0$, the name $x$ is free, while $y$ is bound.

\begin{mathpar}
  \inferrule* [lab=monoidal-laws] {} { P|Q \equiv Q|P \and P|0 \equiv P \and P|(Q|R) \equiv (P|Q)|R }
\end{mathpar}

\begin{mathpar}
  \inferrule* [lab=alpha-equivalence] {} { (x)P \equiv (y)P\{y/x\} \and y \not\in \freenames{P} }
\end{mathpar}

\begin{definition}
Then two processes, $P,Q$, are alpha-equivalent if $P = Q\{\vec{y}/\vec{x}\}$ for
some $\vec{x} \in \boundnames{Q},\vec{y} \in \boundnames{P}$, where $Q\{\vec{y}/\vec{x}\}$
denotes the capture-avoiding substitution of $\vec{y}$ for $\vec{x}$ in $Q$.
\end{definition}

\begin{definition}
  The {\em structural congruence} \cite{SangiorgiWalker} , $\equiv$,
  between processes is the least congruence containing
  alpha-equivalence, satisfying the abelian monoid laws
  (associativity, commutativity and $\pzero$ as identity) for parallel
  composition $|$ and for summation $+$.
\end{definition}

\subsection{Name equivalence}

We take name equivalence, written $\nameeq$, to be the smallest
equivalence relation generated by the following rules.

\begin{mathpar}
\inferrule*[lab=Quote-drop]
{ }
{ \quotep{@{x}} \nameeq x }

\inferrule*[lab=Struct-equiv]
{ P \scong Q }
{ \quotep{P} \nameeq \quotep{Q} }
\end{mathpar}

The astute reader will have noticed that the mutual recursion of names
and processes imposes a mutual recursion on alpha-equivalence and
structural equivalence via name-equivalence. Fortunately, all of this
works out pleasantly and we may calculate in the natural way, free of
concern. The reader interested in the details is referred to the
appendix \ref{appendix:rho_details}.

\subsection{Substitution}

We use $\Proc$ for the set of processes, $\QProc$ for the set of
names, and $\id{\{}\vec{y} / \vec{x} \id{\}}$ to denote partial maps,
$s : \QProc \rightarrow \QProc$. A map, $s$ lifts, uniquely, to a map
on process terms, $\widehat{s} : \Proc \rightarrow \Proc$ by the
following equations.

\begin{mathpar}
  (0) \psubstp{Q}{P} := 0 \\
  (R \juxtap S) \psubstp{Q}{P}
  :=    
  (R)\psubstp{Q}{P} \juxtap (S) \psubstp{Q}{P} \\
  (x?(y).R) \psubstp{Q}{P}    
  :=    
  (x)\substp{Q}{P} (z)\concat( (R \psubstn{z}{y}) \psubstp{Q}{P} ) \\
  (\lift{x}{R}) \psubstp{Q}{P}  
  :=
  \lift{(x)\substp{Q}{P}}{ R \psubstp{Q}{P} } \\
%   (\dropn{x})  \psubstp{Q}{P}       
%   := 
%   \left\{ 
%     \begin{array}{ccc} 
%       \dropn{\quotep{Q}} & & x \nameeq \quotep{P} \\
%       \dropn{x} & & otherwise \\
%     \end{array}
%   \right. 
  (\dropn{x})  \psubstp{Q}{P}       
  := 
  \left\{ 
    \begin{array}{ccc} 
      Q & & x \nameeq \quotep{P} \\
      \dropn{x} & & otherwise \\
    \end{array}
  \right.
\end{mathpar}
 

where

\begin{eqnarray}
  (x)\id{\{} \lpquote Q \rpquote / \lpquote P \rpquote \id{\}}            = 
  \left\{ 
    \begin{array}{ccc}
      \lpquote Q \rpquote & & x \nameeq \lpquote P \rpquote \\
      x & & otherwise \\
    \end{array}
  \right. \nonumber
\end{eqnarray}

and $z$ is chosen distinct from $\quotep{P}$, $\quotep{Q}$, the free
names in $Q$, and all the names in $R$. Our $\alpha$-equivalence will
be built in the standard way from this substitution.

\begin{remark}\label{rem:no_self_referential_names}
  One consequence of these definitions is that $\forall P. \quotep{P}
  \not\in \freenames{P}$.
\end{remark}

\subsection{ Dynamic quote: an example }

Anticipating something of what's to come, consider applying the
substitution, $\widehat{\id{\{}u / z \id{\}}}$, to the following pair
of processes, $\lift{w}{y!(z)}$ and $w[ \lpquote y!(z) \rpquote ]$.

\begin{eqnarray}
	\lift{w}{y!(z)}\widehat{\id{\{}u / z \id{\}}}
		& = &
		\lift{w}{y!(u)} \nonumber\\
	w[ \lpquote y!(z) \rpquote ] \widehat{ \id{\{}u / z \id{\}} }
		& = &
		w[ \lpquote y!(z) \rpquote ] \nonumber
\end{eqnarray}

Because the body of the process between quotes is impervious to
substitution, we get radically different answers. In fact, by
examining the first process in an input context,
e.g. $x?(z).\lift{w}{y!(z)}$, we see that the process under the lift
operator may be shaped by prefixed inputs binding a name inside it. In
this sense, the lift operator will be seen as a way to dynamically
construct processes before reifying them as names.

Finally equipped with these standard features we can present the
dynamics of the calculus.

\subsubsection{Operational semantics} 

Finally, we introduce the computational dynamics. What marks these
algebras as distinct from other more traditionally studied algebraic
structures, e.g. vector spaces or polynomial rings, is the manner in
which dynamics is captured. In traditional structures, dynamics is typically
expressed through morphisms between such structures, as in linear maps
between vector spaces or morphisms between rings. In algebras
associated with the semantics of computation, the dynamics is
expressed as part of the algebraic structure itself, through a
reduction reduction relation typically denoted by $\red$. Below, we
give a recursive presentation of this relation for the calculus used
in the encoding.

$\red \subseteq \pi \times \pi$
$\red : \pi \to \mathcal{P}(\pi)$

\begin{mathpar}
  \inferrule* [lab=Comm] { \textsf{match}( x_{src}, x_{trgt} ) } { x_{trgt}?(y)P \; | \; x_{src}!\langle {Q} \rangle \red P\{\quotep{Q}/y}\} }
  \and \\
  \inferrule* [lab=Par] {{P} \red {P}'} {{{P} | {Q}} \red {{P}' | {Q}}}
  \and
  \inferrule* [lab=Equiv]{{{P} \scong {P}'} \andalso {{P}' \red {Q}'} \andalso {{Q}' \scong {Q}}}{{P} \red {Q}}
\end{mathpar}

\begin{eqnarray*}
  match_{\equiv} (\quotep{P},\quotep{Q}) & := & P \equiv Q \\
  match_{\dagger}(\quotep{P},\quotep{Q}) & := & \forall R. P|Q \red^{*} R => R \red^{*} 0 \\
  match_{K}(\quotep{P},\quotep{Q}) & := & K \mbox{ for some context } K
\end{eqnarray*}

$u?(x)P | u!\langle Q \rangle \red P\{\quotep{Q}/x\}$

%We write $\wred$ for $\red^*$, and $P\red$ if $\exists Q $ such that $ P \red Q$.
We write $P\red$ if $\exists Q $ such that $ P \red Q$ and $P\not\red$, otherwise.

\section{Replication}

As mentioned before, it is known that replication (and hence
recursion) can be implemented in a higher-order process algebra
\cite{SangiorgiWalker}. As our first example of calculation with the
machinery thus far presented we give the construction explicitly in
the {\rhoc}.

\begin{eqnarray}
	D_{x} & := & \prefix{x}{y}{(\binpar{\outputp{x}{y}}{@{y}})} \nonumber\\
	\bangp_{x}{P} & := & \binpar{{x}!\langle{\binpar{D_{x}}{P}}\rangle}{D_{x}} \nonumber
\end{eqnarray}

\begin{eqnarray}
	\bangp_{x}{P} & & \nonumber\\
	=
	& {x}!\langle{(\prefix{x}{y}{(\outputp{x}{y} | @{y})) | P}}\rangle 
	      | \prefix{x}{y}{(\outputp{x}{y} | @{y})} & \nonumber\\
	\red
	& (\outputp{x}{y} | @{y})\substn{\quotep{(\prefix{x}{y}{(@{y} | \outputp{x}{y})) | P}}}{y} & \nonumber\\
	=
	& \outputp{x}{\quotep{(\prefix{x}{y}{(\outputp{x}{y} | @{y})) | P}}}
	  | {(\prefix{x}{y}{(\outputp{x}{y} | @{y})) | P}} & \nonumber\\
	\red
	& \ldots & \nonumber\\
	\red^*
	& P | P | \ldots & \nonumber
\end{eqnarray}

Of course, this encoding, as an implementation, runs away, unfolding
$\bangp{P}$ eagerly. A lazier and more implementable replication
operator, restricted to input-guarded processes, may be obtained as follows.

\begin{eqnarray}
\bangp{\prefix{u}{v}{P}} 
	:= 
	\binpar{\lift{x}{\prefix{u}{v}{(\binpar{D(x)}{P})}}}{D(x)} \nonumber
\end{eqnarray}

\begin{remark}
  Note that the lazier definition still does not deal with summation
  or mixed summation (i.e. sums over input and output). The reader is
  invited to construct definitions of replication that deal with these
  features. 

  Further, the definitions are parameterized in a name, $x$. Can you,
  gentle reader, make a definition that eliminates this parameter and
  guarantees no accidental interaction between the replication
  machinery and the process being replicated -- i.e. no accidental
  sharing of names used by the process to get its work done and the
  name(s) used by the replication to effect copying. This latter
  revision of the definition of replication is crucial to obtaining
  the expected identity $!!P \sim !P$.
\end{remark}

\begin{remark}\label{rem:paradoxical_combinator}
  The reader familiar with the lambda calculus will have noticed the
  similarity between $D$ and the paradoxical combinator.

  [Ed. note: the existence of this seems to suggest we have to be more
  restrictive on the set of processes and names we admit if we are to
  support no-cloning.]
\end{remark}

\subsubsection{Bisimulation}

The computational dynamics gives rise to another kind of equivalence,
the equivalence of computational behavior. As previously mentioned
this is typically captured \emph{via} some form of bisimulation.

% The notion we use in this paper is weak barbed bisimulation
% \cite{milner91polyadicpi}.

The notion we use in this paper is derived from weak barbed
bisimulation \cite{milner91polyadicpi}. 

\begin{definition}
An \emph{observation relation}, $\downarrow_{\mathcal N}$, over a set
of names, $\mathcal N$, is the smallest relation satisfying the rules
below.

\infrule[Out-barb]{y \in {\mathcal N}, \; x \nameeq y}
		  {\outputp{x}{v} \downarrow_{\mathcal N} x}
\infrule[Par-barb]{\mbox{$P\downarrow_{\mathcal N} x$ or $Q\downarrow_{\mathcal N} x$}}
		  {\binpar{P}{Q} \downarrow_{\mathcal N} x}

We write $P \Downarrow_{\mathcal N} x$ if there is $Q$ such that 
$P \wred Q$ and $Q \downarrow_{\mathcal N} x$.
\end{definition}

\begin{definition}
%\label{def.bbisim}
An  ${\mathcal N}$-\emph{barbed bisimulation} over a set of names, ${\mathcal N}$, is a symmetric binary relation 
${\mathcal S}_{\mathcal N}$ between agents such that $P\rel{S}_{\mathcal N}Q$ implies:
\begin{enumerate}
\item If $P \red P'$ then $Q \wred Q'$ and $P'\rel{S}_{\mathcal N} Q'$.
\item If $P\downarrow_{\mathcal N} x$, then $Q\Downarrow_{\mathcal N} x$.
\end{enumerate}
$P$ is ${\mathcal N}$-barbed bisimilar to $Q$, written
$P \wbbisim_{\mathcal N} Q$, if $P \rel{S}_{\mathcal N} Q$ for some ${\mathcal N}$-barbed bisimulation ${\mathcal S}_{\mathcal N}$.
\end{definition}

$\mathcal{R} \subseteq \pi \times \pi$

$P \mathcal{R} Q => \forall P'. P \red P' \Rightarrow \exists Q'. Q \red Q', P' \mathcal{R} Q'$

$P \vdash x \Rightarrow Q \vdash x$

\begin{mathpar}
  \inferrule*[lab=Out-barb]{x \nameeq y}{{y}!\langle{Q}\rangle \vdash x}
  \and
  \inferrule*[lab=Par-barb]{\mbox{$P\vdash x$ or $Q\vdash x$}}{\binpar{P}{Q} \vdash x}
\end{mathpar}

\subsubsection{Contexts}

One of the principle advantages of computational calculi like the
$\pi$-calculus is a well-defined notion of context,
contextual-equivalence and a correlation between
contextual-equivalence and notions of bisimulation. The notion of
context allows the decomposition of a process into (sub-)process and
its syntactic environment, its context. Thus, a context may be
thought of as a process with a ``hole'' (written $\Box$) in it. The
application of a context $M$ to a process $P$, written $M[P]$, is
tantamount to filling the hole in $M$ with $P$. In this paper we do
not need the full weight of this theory, but do make use of the notion
of context in the proof the main theorem. 

\begin{mathpar}
  \inferrule* [lab=summation] {} {{M_{M},M_{N}} \bc \Box \;|\; x.M_{A} \;|\; M_{M}+M_{N}}
  \and
  \inferrule* [lab=agent] {} {{M_{A}} \bc (\vec{x})M_{P} \;| \; \clift{P_0,\ldots,M_{P},\ldots,P_N}}
  \and \\
  \inferrule* [lab=process] {} {{M_{P}} \bc M_{N} \;| \;P|M_{P} }
\end{mathpar} 

\begin{mathpar}
  \inferrule* [lab=sychronization] {} {M_{N} \bc \Box \;|\; x?M_{F} \;|\; x!M_{C}}
  \and
  \inferrule* [lab=abstraction] {} {{M_{F}} \bc (x)M_{P} }
  \and
  \inferrule* [lab=concretion] {} {{M_{C}} \bc \langle M_{P} \rangle }
  \and \\
  \inferrule* [lab=process] {} {{M_{P}} \bc M_{N} \;| \;P|M_{P} }
\end{mathpar}

\begin{definition}[contextual application] Given a context $M$, and
  process $P$, we define the \emph{contextual application}, $M[P] :=
  M\{P/\Box\}$. That is, the contextual application of M to P is the
  substitution of $P$ for $\Box$ in $M$.
\end{definition}

$\meaningof{-} : L \to \mathcal{P}(\pi)$

\begin{mathpar}
  \inferrule* [lab=collection] {} {\meaningof{true} = \pi, \and \meaningof{~E} = \pi \setminus \meaningof{E}, \and \meaningof{E_{1} \& E_{2}} = \meaningof{E_{1}} \cap \meaningof{E_{2}}}
\end{mathpar}

\begin{mathpar}
  \inferrule* [lab=structure] {} {\meaningof{0} = \{ P \in \pi | P \equiv 0 \}, \and \\ \meaningof{E_1 | E_2} = \{ P \in \pi | P \equiv P_{1} | P_{2}, P_{1} \in \meaningof{E_{1}}, P_{2} \in \meaningof{E_2}\} }
\end{mathpar}

\begin{mathpar}
 \inferrule* [lab=behavior] {} {\meaningof{\langle a?b \rangle E} = \{ P \in \pi | P \equiv Q | u?(y)P', \\ \and \\\\ \and \\ \;\;\; u \in \meaningof{a}, \forall z.P'\{z/y\} \in \meaningof{E\{z/b\}}\}, \and \\ \meaningof{a!E} = \{ P \in \pi | P \equiv Q | x!\langle P' \rangle, x \in \meaningof{a} P' \in \meaningof{E}\} }
\end{mathpar}

\begin{mathpar}
 \inferrule* [lab=nominal] {} {\meaningof{\quotep{E}} = \{ \quotep{P} \in \quotep{\pi} | P \in \meaningof{E} \}, \and \meaningof{\quotep{P}} = \{ \quotep{Q} \in \quotep{\pi} | P \equiv Q \} \and \\ \meaningof{@\quotep{E}} = \{ P \in \pi | P \equiv @x, x \in \meaningof{E} \}}
\end{mathpar}

\begin{eqnarray*}
  \\
  \meaningof{-} : TS \to ST
\end{eqnarray*}

\begin{eqnarray*}
  \\
  L : TS \to ST
\end{eqnarray*}

\begin{eqnarray*}
  \\
  P \models E \iff P \in \meaningof{E}
\end{eqnarray*}

\begin{eqnarray*}
  P \approx_{L} Q \iff \forall E \in L. P \models E \iff Q \models E
\end{eqnarray*}

\begin{eqnarray*}
  P \approx_{K} Q
\end{eqnarray*}

\begin{eqnarray*}
  P \approx Q
\end{eqnarray*}

$\approx_{K} = \approx = \approx_{L}$

\subsubsection{Contextual duality}

Note that contexts extend the quotation operation to a family of
operations from processes to names. Given a context, $M$, we can
define a \emph{nominal context}, $\quotep{M}$ by $\quotep{M}[P] :=
\quotep{M[P]}$. To foreshadow what is to come we observe that these
operations enjoy a duality with processes very much like the duality
between vectors and maps from vectors to scalars.

Further, because the calculus is essentially higher-order, we have a
correspondence between contexts and processes. More specifically,
given a name $x$ and a context $M$ we can construct $M^{*}_{x}$ such
that 

\begin{mathpar}
  M^{*}_{x} | \lift{x}{P} \red M[P]
\end{mathpar}

namely,

\begin{mathpar}
  M^{*}_{x} := x?(u).M[\dropn{u}]
\end{mathpar}

The dependence of $M^{*}_{x}$ on a name makes it an abstraction, 

\begin{mathpar}
  M^{*} := (x)x?(u).M[\dropn{u}]
\end{mathpar}

\subsection{Additional notation}

It will sometimes be convenient to denote the process a name
quotes. We already have the notation $x = \quotep{P}$, but it will be
convenient to introduce an alternate notation, $\procn{x}$, when we
want to emphasize the connection to the use of the name. Note that, by
virtue of name equivalence, $\quotep{\procn{x}} \nameeq x$; so, the
notation is consistent with previous definitions.

Further, because names have structure it is possible to effect
substitutions on the basis of that structure. This means we need to
upgrade our notation for substitutions, which we accomplish by
adapting comprehension notation. Thus,

\begin{mathpar}
  P\{ y / x : x \in S \}
\end{mathpar}

is interpreted to mean the process derived from P by replacing (in a
capture-avoiding manner) each occurrence of $x$ in $S$ by $y$. For example,

\begin{mathpar}
  P\{ \quotep{\procn{x}|\procn{x}} / x : x \in \freenames{P} \}
\end{mathpar}

will replace each (occurrence) of a free name $x$ in $P$ by
$\quotep{\procn{x}|\procn{x}}$.

Also, we will avail ourselves of the notation $x^{L}$ and $x^{R}$ to
denote injections of a name into disjoint copies of the name
space. There are numerous ways to accomplish this. One example can be
found in \cite{MeredithR05}. This notation overloads to vectors of
names: $\vec{x}^{\pi} := (x_{i}^{\pi} \; : \; 0 \leq i < |\vec{x}| )$ where $\pi \in \{L,R\}$.

We also use $P^{\Box} := P|\Box$.

In \cite{MeredithR05} an interpretation of the new operator is
given. It turns out that there are several possible interpretations
all enjoying the requisite algebraic properties of the operator (see
\cite{milner91polyadicpi}). We will therefore make liberal use of
$(\nu\; \vec{x})P$.

% subsection the_syntax_and_semantics_of_the_notation_system (end)   

\input{qm2pi.qmops} 

\input{qm2pi.sterngerlach} 

\input{qm2pi.metric} 

% section concurrent_process_calculi (end)

%\input{qm2pi.proofsketch}

% section proof sketch (end)

%\input{qm2pi.slviaknots} 

% section spatial logic via knots (end)

\input{qm2pi.conclusion}

% section conclusion (end)

%\input{qm2pi.dtcodes} 

% section wiring algorithm (end)

\input{qm2pi.ack} 

% section acknowledgments (end)

\newpage


\bibliographystyle{plain}   
\bibliography{../../biblios/main.bib}

\input{qm2pi.rhodetails}

\end{document}

 

% section concurrent_process_calculi (end)

%\documentclass[12pt]{llncs}
%\documentclass{jktr}

\usepackage[pdftex]{hyperref}                   
\usepackage {listings}
\usepackage {mathpartir}
\usepackage{bcprules}
%\usepackage{listings}
                       
\usepackage{graphicx} 
%\usepackage[margins=2.5cm,nohead,nofoot]{geometry}
%\usepackage{geometry}
\usepackage{amsfonts}
\usepackage{amstext}
\usepackage{latexsym}
\usepackage{amssymb}
\usepackage{color}


%\include{myPreamble}
\include{qm2pi.local} 

%\ifpdf
%\usepackage[pdftex]{graphicx}
%\else
%\usepackage{graphicx}
%\fi

 % \ifpdf
%  \usepackage{pdfsync}
%  \if


%\title{Brief Article}
%\author{David F. Snyder}
%\author{L.G. Meredith}

%\address{Dept. of Math., Texas State University--San Marcos, San Marcos, TX 78666}
       
\pagestyle{empty}


\begin{document}

\lstset{language=[Objective]Caml,frame=shadowbox}

\input{qm2pi.front}

% section front matter (end)

\input{qm2pi.intro} 
 
% section introduction (end)

% \input{qm2pi.knotations} 

% section notation (end)

\input{qm2pi.process.calculi} 

% section concurrent_process_calculi_and_spatial_logics_ (end)
    
%\input{qm2pi.knots2pi} 

%\input{qm2pi.trefoil} 

%\input{qm2pi.mainthm} 

% subsection basic_interpretation (end)

%\input{qm2pi.rho.presentation} 
\subsection{The syntax and semantics of the notation system}\label{sub:the_syntax_and_semantics_of_the_notation_system} % (fold)

We now summarize a technical presentation of the calculus that
embodies our theory of dynamics. The typical presentation of such a
calculus follows the style of giving generators and relations on
them. The grammar, below, describing term constructors, freely
generates the set of processes, $\Proc$. This set is then quotiented
by a relation known as structural congruence and it is over this set
that the notion of dynamics is expressed. This presentation is
essentially that of \cite{MeredithR05} with the addition of
polyadicity and summation. For readability we have relegated some of
the technical subtleties to an appendix.

\subsubsection{Process grammar}\label{subsub:process_grammar}

\begin{mathpar}
  \inferrule* [lab=synchronization] {} {{M} \bc \pzero \;|\; x?F \;|\; x!C }
  \and
  \inferrule* [lab=abstraction] {} {{F} \bc (x)P}
  \and
  \inferrule* [lab=concretion] {} {{C} \bc \langle Q \rangle}
  \and
  \inferrule* [lab=process] {} {{P,Q} \bc M \;| \;P|Q \;|\; @{x}}
  \and
  \inferrule* [lab=name] {} {{x} \bc \quotep{P}}
\end{mathpar} 

Note that $\vec{x}$ (resp. $\vec{P}$) denotes a vector of names
(resp. processes) of length $|\vec{x}|$ (resp. $|\vec{P}|$). We adopt
the following useful abbreviations.

\begin{mathpar}
   x?(\vec{y}).P := x.(\vec{y})P \and  x\clift{\vec{P}} := x.\clift{\vec{P}}
   \and x!(y) := \lift{x}{\dropn{y}}
   \and \Pi_{i=0}^{n-1}P_i := P_0 | \ldots | P_{n-1}
\end{mathpar}

\subsubsection{Structural congruence}

\paragraph{Free and bound names and alpha-equivalence.} At the
core of structural equivalence is alpha-equivalence which identifies
process that are the same up to a change of variable. Formally, we
recognize the distinction between free and bound names. The free names
of a process, $\freenames{P}$, may be calculated recursively as
follows:

\begin{mathpar}
\freenames{\pzero} := \emptyset
  \and \\
  \freenames{x?(y).P} := \{ x \} \cup (\freenames{P} \setminus \{ y \})
  \and 
  \freenames{x!\langle P \rangle} := \{ x \} \cup \{ P \} 
  \and \\
  \freenames{P|Q} := \freenames{P} \cup \freenames{Q}
  \and \\
  \freenames{@{x}} := \{ x \}
\end{mathpar}

$\pi$
$\quotep{\pi}$

$\freenames{-} : \pi \to \mathcal{P}(\quotep{\pi})$

\begin{eqnarray*}
  \freenames{\pzero} & := & \emptyset \\
  \freenames{x?(y).P} & := & \{ x \} \cup (\freenames{P} \setminus \{ y \}) \\
  \freenames{x!\langle P \rangle} & := & \{ x \} \cup \{ P \} \\
  \freenames{P|Q} & := & \freenames{P} \cup \freenames{Q} \\
  \freenames{\dropn{x}} & := & \{ x \}
\end{eqnarray*}

The bound names of a process, $\boundnames{P}$, are those names occurring in $P$
that are not free. For example, in $x?(y).0$, the name $x$ is free, while $y$ is bound.

\begin{mathpar}
  \inferrule* [lab=monoidal-laws] {} { P|Q \equiv Q|P \and P|0 \equiv P \and P|(Q|R) \equiv (P|Q)|R }
\end{mathpar}

\begin{mathpar}
  \inferrule* [lab=alpha-equivalence] {} { (x)P \equiv (y)P\{y/x\} \and y \not\in \freenames{P} }
\end{mathpar}

\begin{definition}
Then two processes, $P,Q$, are alpha-equivalent if $P = Q\{\vec{y}/\vec{x}\}$ for
some $\vec{x} \in \boundnames{Q},\vec{y} \in \boundnames{P}$, where $Q\{\vec{y}/\vec{x}\}$
denotes the capture-avoiding substitution of $\vec{y}$ for $\vec{x}$ in $Q$.
\end{definition}

\begin{definition}
  The {\em structural congruence} \cite{SangiorgiWalker} , $\equiv$,
  between processes is the least congruence containing
  alpha-equivalence, satisfying the abelian monoid laws
  (associativity, commutativity and $\pzero$ as identity) for parallel
  composition $|$ and for summation $+$.
\end{definition}

\subsection{Name equivalence}

We take name equivalence, written $\nameeq$, to be the smallest
equivalence relation generated by the following rules.

\begin{mathpar}
\inferrule*[lab=Quote-drop]
{ }
{ \quotep{@{x}} \nameeq x }

\inferrule*[lab=Struct-equiv]
{ P \scong Q }
{ \quotep{P} \nameeq \quotep{Q} }
\end{mathpar}

The astute reader will have noticed that the mutual recursion of names
and processes imposes a mutual recursion on alpha-equivalence and
structural equivalence via name-equivalence. Fortunately, all of this
works out pleasantly and we may calculate in the natural way, free of
concern. The reader interested in the details is referred to the
appendix \ref{appendix:rho_details}.

\subsection{Substitution}

We use $\Proc$ for the set of processes, $\QProc$ for the set of
names, and $\id{\{}\vec{y} / \vec{x} \id{\}}$ to denote partial maps,
$s : \QProc \rightarrow \QProc$. A map, $s$ lifts, uniquely, to a map
on process terms, $\widehat{s} : \Proc \rightarrow \Proc$ by the
following equations.

\begin{mathpar}
  (0) \psubstp{Q}{P} := 0 \\
  (R \juxtap S) \psubstp{Q}{P}
  :=    
  (R)\psubstp{Q}{P} \juxtap (S) \psubstp{Q}{P} \\
  (x?(y).R) \psubstp{Q}{P}    
  :=    
  (x)\substp{Q}{P} (z)\concat( (R \psubstn{z}{y}) \psubstp{Q}{P} ) \\
  (\lift{x}{R}) \psubstp{Q}{P}  
  :=
  \lift{(x)\substp{Q}{P}}{ R \psubstp{Q}{P} } \\
%   (\dropn{x})  \psubstp{Q}{P}       
%   := 
%   \left\{ 
%     \begin{array}{ccc} 
%       \dropn{\quotep{Q}} & & x \nameeq \quotep{P} \\
%       \dropn{x} & & otherwise \\
%     \end{array}
%   \right. 
  (\dropn{x})  \psubstp{Q}{P}       
  := 
  \left\{ 
    \begin{array}{ccc} 
      Q & & x \nameeq \quotep{P} \\
      \dropn{x} & & otherwise \\
    \end{array}
  \right.
\end{mathpar}
 

where

\begin{eqnarray}
  (x)\id{\{} \lpquote Q \rpquote / \lpquote P \rpquote \id{\}}            = 
  \left\{ 
    \begin{array}{ccc}
      \lpquote Q \rpquote & & x \nameeq \lpquote P \rpquote \\
      x & & otherwise \\
    \end{array}
  \right. \nonumber
\end{eqnarray}

and $z$ is chosen distinct from $\quotep{P}$, $\quotep{Q}$, the free
names in $Q$, and all the names in $R$. Our $\alpha$-equivalence will
be built in the standard way from this substitution.

\begin{remark}\label{rem:no_self_referential_names}
  One consequence of these definitions is that $\forall P. \quotep{P}
  \not\in \freenames{P}$.
\end{remark}

\subsection{ Dynamic quote: an example }

Anticipating something of what's to come, consider applying the
substitution, $\widehat{\id{\{}u / z \id{\}}}$, to the following pair
of processes, $\lift{w}{y!(z)}$ and $w[ \lpquote y!(z) \rpquote ]$.

\begin{eqnarray}
	\lift{w}{y!(z)}\widehat{\id{\{}u / z \id{\}}}
		& = &
		\lift{w}{y!(u)} \nonumber\\
	w[ \lpquote y!(z) \rpquote ] \widehat{ \id{\{}u / z \id{\}} }
		& = &
		w[ \lpquote y!(z) \rpquote ] \nonumber
\end{eqnarray}

Because the body of the process between quotes is impervious to
substitution, we get radically different answers. In fact, by
examining the first process in an input context,
e.g. $x?(z).\lift{w}{y!(z)}$, we see that the process under the lift
operator may be shaped by prefixed inputs binding a name inside it. In
this sense, the lift operator will be seen as a way to dynamically
construct processes before reifying them as names.

Finally equipped with these standard features we can present the
dynamics of the calculus.

\subsubsection{Operational semantics} 

Finally, we introduce the computational dynamics. What marks these
algebras as distinct from other more traditionally studied algebraic
structures, e.g. vector spaces or polynomial rings, is the manner in
which dynamics is captured. In traditional structures, dynamics is typically
expressed through morphisms between such structures, as in linear maps
between vector spaces or morphisms between rings. In algebras
associated with the semantics of computation, the dynamics is
expressed as part of the algebraic structure itself, through a
reduction reduction relation typically denoted by $\red$. Below, we
give a recursive presentation of this relation for the calculus used
in the encoding.

$\red \subseteq \pi \times \pi$
$\red : \pi \to \mathcal{P}(\pi)$

\begin{mathpar}
  \inferrule* [lab=Comm] { \textsf{match}( x_{src}, x_{trgt} ) } { x_{trgt}?(y)P \; | \; x_{src}!\langle {Q} \rangle \red P\{\quotep{Q}/y}\} }
  \and \\
  \inferrule* [lab=Par] {{P} \red {P}'} {{{P} | {Q}} \red {{P}' | {Q}}}
  \and
  \inferrule* [lab=Equiv]{{{P} \scong {P}'} \andalso {{P}' \red {Q}'} \andalso {{Q}' \scong {Q}}}{{P} \red {Q}}
\end{mathpar}

\begin{eqnarray*}
  match_{\equiv} (\quotep{P},\quotep{Q}) & := & P \equiv Q \\
  match_{\dagger}(\quotep{P},\quotep{Q}) & := & \forall R. P|Q \red^{*} R => R \red^{*} 0 \\
  match_{K}(\quotep{P},\quotep{Q}) & := & K \mbox{ for some context } K
\end{eqnarray*}

$u?(x)P | u!\langle Q \rangle \red P\{\quotep{Q}/x\}$

%We write $\wred$ for $\red^*$, and $P\red$ if $\exists Q $ such that $ P \red Q$.
We write $P\red$ if $\exists Q $ such that $ P \red Q$ and $P\not\red$, otherwise.

\section{Replication}

As mentioned before, it is known that replication (and hence
recursion) can be implemented in a higher-order process algebra
\cite{SangiorgiWalker}. As our first example of calculation with the
machinery thus far presented we give the construction explicitly in
the {\rhoc}.

\begin{eqnarray}
	D_{x} & := & \prefix{x}{y}{(\binpar{\outputp{x}{y}}{@{y}})} \nonumber\\
	\bangp_{x}{P} & := & \binpar{{x}!\langle{\binpar{D_{x}}{P}}\rangle}{D_{x}} \nonumber
\end{eqnarray}

\begin{eqnarray}
	\bangp_{x}{P} & & \nonumber\\
	=
	& {x}!\langle{(\prefix{x}{y}{(\outputp{x}{y} | @{y})) | P}}\rangle 
	      | \prefix{x}{y}{(\outputp{x}{y} | @{y})} & \nonumber\\
	\red
	& (\outputp{x}{y} | @{y})\substn{\quotep{(\prefix{x}{y}{(@{y} | \outputp{x}{y})) | P}}}{y} & \nonumber\\
	=
	& \outputp{x}{\quotep{(\prefix{x}{y}{(\outputp{x}{y} | @{y})) | P}}}
	  | {(\prefix{x}{y}{(\outputp{x}{y} | @{y})) | P}} & \nonumber\\
	\red
	& \ldots & \nonumber\\
	\red^*
	& P | P | \ldots & \nonumber
\end{eqnarray}

Of course, this encoding, as an implementation, runs away, unfolding
$\bangp{P}$ eagerly. A lazier and more implementable replication
operator, restricted to input-guarded processes, may be obtained as follows.

\begin{eqnarray}
\bangp{\prefix{u}{v}{P}} 
	:= 
	\binpar{\lift{x}{\prefix{u}{v}{(\binpar{D(x)}{P})}}}{D(x)} \nonumber
\end{eqnarray}

\begin{remark}
  Note that the lazier definition still does not deal with summation
  or mixed summation (i.e. sums over input and output). The reader is
  invited to construct definitions of replication that deal with these
  features. 

  Further, the definitions are parameterized in a name, $x$. Can you,
  gentle reader, make a definition that eliminates this parameter and
  guarantees no accidental interaction between the replication
  machinery and the process being replicated -- i.e. no accidental
  sharing of names used by the process to get its work done and the
  name(s) used by the replication to effect copying. This latter
  revision of the definition of replication is crucial to obtaining
  the expected identity $!!P \sim !P$.
\end{remark}

\begin{remark}\label{rem:paradoxical_combinator}
  The reader familiar with the lambda calculus will have noticed the
  similarity between $D$ and the paradoxical combinator.

  [Ed. note: the existence of this seems to suggest we have to be more
  restrictive on the set of processes and names we admit if we are to
  support no-cloning.]
\end{remark}

\subsubsection{Bisimulation}

The computational dynamics gives rise to another kind of equivalence,
the equivalence of computational behavior. As previously mentioned
this is typically captured \emph{via} some form of bisimulation.

% The notion we use in this paper is weak barbed bisimulation
% \cite{milner91polyadicpi}.

The notion we use in this paper is derived from weak barbed
bisimulation \cite{milner91polyadicpi}. 

\begin{definition}
An \emph{observation relation}, $\downarrow_{\mathcal N}$, over a set
of names, $\mathcal N$, is the smallest relation satisfying the rules
below.

\infrule[Out-barb]{y \in {\mathcal N}, \; x \nameeq y}
		  {\outputp{x}{v} \downarrow_{\mathcal N} x}
\infrule[Par-barb]{\mbox{$P\downarrow_{\mathcal N} x$ or $Q\downarrow_{\mathcal N} x$}}
		  {\binpar{P}{Q} \downarrow_{\mathcal N} x}

We write $P \Downarrow_{\mathcal N} x$ if there is $Q$ such that 
$P \wred Q$ and $Q \downarrow_{\mathcal N} x$.
\end{definition}

\begin{definition}
%\label{def.bbisim}
An  ${\mathcal N}$-\emph{barbed bisimulation} over a set of names, ${\mathcal N}$, is a symmetric binary relation 
${\mathcal S}_{\mathcal N}$ between agents such that $P\rel{S}_{\mathcal N}Q$ implies:
\begin{enumerate}
\item If $P \red P'$ then $Q \wred Q'$ and $P'\rel{S}_{\mathcal N} Q'$.
\item If $P\downarrow_{\mathcal N} x$, then $Q\Downarrow_{\mathcal N} x$.
\end{enumerate}
$P$ is ${\mathcal N}$-barbed bisimilar to $Q$, written
$P \wbbisim_{\mathcal N} Q$, if $P \rel{S}_{\mathcal N} Q$ for some ${\mathcal N}$-barbed bisimulation ${\mathcal S}_{\mathcal N}$.
\end{definition}

$\mathcal{R} \subseteq \pi \times \pi$

$P \mathcal{R} Q => \forall P'. P \red P' \Rightarrow \exists Q'. Q \red Q', P' \mathcal{R} Q'$

$P \vdash x \Rightarrow Q \vdash x$

\begin{mathpar}
  \inferrule*[lab=Out-barb]{x \nameeq y}{{y}!\langle{Q}\rangle \vdash x}
  \and
  \inferrule*[lab=Par-barb]{\mbox{$P\vdash x$ or $Q\vdash x$}}{\binpar{P}{Q} \vdash x}
\end{mathpar}

\subsubsection{Contexts}

One of the principle advantages of computational calculi like the
$\pi$-calculus is a well-defined notion of context,
contextual-equivalence and a correlation between
contextual-equivalence and notions of bisimulation. The notion of
context allows the decomposition of a process into (sub-)process and
its syntactic environment, its context. Thus, a context may be
thought of as a process with a ``hole'' (written $\Box$) in it. The
application of a context $M$ to a process $P$, written $M[P]$, is
tantamount to filling the hole in $M$ with $P$. In this paper we do
not need the full weight of this theory, but do make use of the notion
of context in the proof the main theorem. 

\begin{mathpar}
  \inferrule* [lab=summation] {} {{M_{M},M_{N}} \bc \Box \;|\; x.M_{A} \;|\; M_{M}+M_{N}}
  \and
  \inferrule* [lab=agent] {} {{M_{A}} \bc (\vec{x})M_{P} \;| \; \clift{P_0,\ldots,M_{P},\ldots,P_N}}
  \and \\
  \inferrule* [lab=process] {} {{M_{P}} \bc M_{N} \;| \;P|M_{P} }
\end{mathpar} 

\begin{mathpar}
  \inferrule* [lab=sychronization] {} {M_{N} \bc \Box \;|\; x?M_{F} \;|\; x!M_{C}}
  \and
  \inferrule* [lab=abstraction] {} {{M_{F}} \bc (x)M_{P} }
  \and
  \inferrule* [lab=concretion] {} {{M_{C}} \bc \langle M_{P} \rangle }
  \and \\
  \inferrule* [lab=process] {} {{M_{P}} \bc M_{N} \;| \;P|M_{P} }
\end{mathpar}

\begin{definition}[contextual application] Given a context $M$, and
  process $P$, we define the \emph{contextual application}, $M[P] :=
  M\{P/\Box\}$. That is, the contextual application of M to P is the
  substitution of $P$ for $\Box$ in $M$.
\end{definition}

$\meaningof{-} : L \to \mathcal{P}(\pi)$

\begin{mathpar}
  \inferrule* [lab=collection] {} {\meaningof{true} = \pi, \and \meaningof{~E} = \pi \setminus \meaningof{E}, \and \meaningof{E_{1} \& E_{2}} = \meaningof{E_{1}} \cap \meaningof{E_{2}}}
\end{mathpar}

\begin{mathpar}
  \inferrule* [lab=structure] {} {\meaningof{0} = \{ P \in \pi | P \equiv 0 \}, \and \\ \meaningof{E_1 | E_2} = \{ P \in \pi | P \equiv P_{1} | P_{2}, P_{1} \in \meaningof{E_{1}}, P_{2} \in \meaningof{E_2}\} }
\end{mathpar}

\begin{mathpar}
 \inferrule* [lab=behavior] {} {\meaningof{\langle a?b \rangle E} = \{ P \in \pi | P \equiv Q | u?(y)P', \\ \and \\\\ \and \\ \;\;\; u \in \meaningof{a}, \forall z.P'\{z/y\} \in \meaningof{E\{z/b\}}\}, \and \\ \meaningof{a!E} = \{ P \in \pi | P \equiv Q | x!\langle P' \rangle, x \in \meaningof{a} P' \in \meaningof{E}\} }
\end{mathpar}

\begin{mathpar}
 \inferrule* [lab=nominal] {} {\meaningof{\quotep{E}} = \{ \quotep{P} \in \quotep{\pi} | P \in \meaningof{E} \}, \and \meaningof{\quotep{P}} = \{ \quotep{Q} \in \quotep{\pi} | P \equiv Q \} \and \\ \meaningof{@\quotep{E}} = \{ P \in \pi | P \equiv @x, x \in \meaningof{E} \}}
\end{mathpar}

\begin{eqnarray*}
  \\
  \meaningof{-} : TS \to ST
\end{eqnarray*}

\begin{eqnarray*}
  \\
  L : TS \to ST
\end{eqnarray*}

\begin{eqnarray*}
  \\
  P \models E \iff P \in \meaningof{E}
\end{eqnarray*}

\begin{eqnarray*}
  P \approx_{L} Q \iff \forall E \in L. P \models E \iff Q \models E
\end{eqnarray*}

\begin{eqnarray*}
  P \approx_{K} Q
\end{eqnarray*}

\begin{eqnarray*}
  P \approx Q
\end{eqnarray*}

$\approx_{K} = \approx = \approx_{L}$

\subsubsection{Contextual duality}

Note that contexts extend the quotation operation to a family of
operations from processes to names. Given a context, $M$, we can
define a \emph{nominal context}, $\quotep{M}$ by $\quotep{M}[P] :=
\quotep{M[P]}$. To foreshadow what is to come we observe that these
operations enjoy a duality with processes very much like the duality
between vectors and maps from vectors to scalars.

Further, because the calculus is essentially higher-order, we have a
correspondence between contexts and processes. More specifically,
given a name $x$ and a context $M$ we can construct $M^{*}_{x}$ such
that 

\begin{mathpar}
  M^{*}_{x} | \lift{x}{P} \red M[P]
\end{mathpar}

namely,

\begin{mathpar}
  M^{*}_{x} := x?(u).M[\dropn{u}]
\end{mathpar}

The dependence of $M^{*}_{x}$ on a name makes it an abstraction, 

\begin{mathpar}
  M^{*} := (x)x?(u).M[\dropn{u}]
\end{mathpar}

\subsection{Additional notation}

It will sometimes be convenient to denote the process a name
quotes. We already have the notation $x = \quotep{P}$, but it will be
convenient to introduce an alternate notation, $\procn{x}$, when we
want to emphasize the connection to the use of the name. Note that, by
virtue of name equivalence, $\quotep{\procn{x}} \nameeq x$; so, the
notation is consistent with previous definitions.

Further, because names have structure it is possible to effect
substitutions on the basis of that structure. This means we need to
upgrade our notation for substitutions, which we accomplish by
adapting comprehension notation. Thus,

\begin{mathpar}
  P\{ y / x : x \in S \}
\end{mathpar}

is interpreted to mean the process derived from P by replacing (in a
capture-avoiding manner) each occurrence of $x$ in $S$ by $y$. For example,

\begin{mathpar}
  P\{ \quotep{\procn{x}|\procn{x}} / x : x \in \freenames{P} \}
\end{mathpar}

will replace each (occurrence) of a free name $x$ in $P$ by
$\quotep{\procn{x}|\procn{x}}$.

Also, we will avail ourselves of the notation $x^{L}$ and $x^{R}$ to
denote injections of a name into disjoint copies of the name
space. There are numerous ways to accomplish this. One example can be
found in \cite{MeredithR05}. This notation overloads to vectors of
names: $\vec{x}^{\pi} := (x_{i}^{\pi} \; : \; 0 \leq i < |\vec{x}| )$ where $\pi \in \{L,R\}$.

We also use $P^{\Box} := P|\Box$.

In \cite{MeredithR05} an interpretation of the new operator is
given. It turns out that there are several possible interpretations
all enjoying the requisite algebraic properties of the operator (see
\cite{milner91polyadicpi}). We will therefore make liberal use of
$(\nu\; \vec{x})P$.

% subsection the_syntax_and_semantics_of_the_notation_system (end)   

\input{qm2pi.qmops} 

\input{qm2pi.sterngerlach} 

\input{qm2pi.metric} 

% section concurrent_process_calculi (end)

%\input{qm2pi.proofsketch}

% section proof sketch (end)

%\input{qm2pi.slviaknots} 

% section spatial logic via knots (end)

\input{qm2pi.conclusion}

% section conclusion (end)

%\input{qm2pi.dtcodes} 

% section wiring algorithm (end)

\input{qm2pi.ack} 

% section acknowledgments (end)

\newpage


\bibliographystyle{plain}   
\bibliography{../../biblios/main.bib}

\input{qm2pi.rhodetails}

\end{document}



% section proof sketch (end)

%\section{Unlikely characters: spatial logic for
  knots}\label{sub:characteristic_formulae} % (fold)

Associated to the mobile process calculi are a family of logics known
as the Hennessy-Milner logics. These logics typically enjoy a
semantics interpreting formulae as sets of processes that when
factored through the encoding outlined above allows an identification
of classes of knots with logical formulae. In the context of this
encoding the sub-family known as the spatial logics \cite{CairesC03}
\cite{CairesC04} \cite{Caires04} are of particular interest providing
several important features for expressing and reasoning about
properties (i.e. classes) of knots. We hint here at how this may be done.

%\begin{description}
%\item [structural connectives] 
\subsubsection{Structural connectives} The spatial logics enjoy
structural connectives corresponding, at the logical level, to the
parallel composition ($P | Q$) and new name ($(\nu \; x)P$)
connectives for processes. As illustrated in the examples below, these
connectives are extremely expressive given the shape of our encoding.
%\item [decideable satisfaction]

\subsubsection{Decideable satisfaction}
In \cite{Caires04} the satisfaction relation is shown to be decideable
for a rich class of processes. It further turns out that the image of
the our encoding is a proper subset of that class. This result
provides the basis for an algorithm by which to search for knots
enjoying a given property.
%\item [characteristic formulae]

\subsubsection{Characteristic formulae}
In the same paper \cite{Caires04} , Caires presents a means of calculating
characteristic formulae, selecting equivalence classes of processes
up to a pre--specified depth limit on the support set of names. Composed with our
encoding, this characteristic formula can be used to select
characteristic formulae for knots.
%\end{description}

\subsubsection{Spatial logic formulae}

The grammar below (segmented for comprehension) summarizes the syntax
of spatial logic formulae. We employ illustrative examples in the
sequel to provide an intuitive understanding of their meaning
referring the reader to \cite{Caires04} for a more detailed explication
of the semantics.

\begin{mathpar}
  \inferrule* [lab=boolean] {} {{A,B} \bc T \;|\; \neg A \;|\; A \wedge B \;|\; \eta = \eta'}
  \and
  \inferrule* [lab=spatial] {} {|\; \pzero \;|\; A | B \;|\; x \text{\textregistered} A \;|\; \forall x . A \;|\;  H x . A}
  \and
  \inferrule* [lab=behavioral] {} {|\; \alpha . A}
  \and 
  \inferrule* [lab=recursion] {} {|\; X(\vec{u}) \;|\; \mu X(\vec{u}) . A}
  \and
  \inferrule* [lab=action] {} {\alpha \bc \langle x?(\vec{y}) \rangle \;|\; \langle x!(\vec{y}) \rangle \;|\; \langle \tau \rangle}
  \and 
  \inferrule* [lab=name] {} {\eta \bc x \;|\; \tau}
\end{mathpar} 

% subsection characteristic_formulae (end)   	 

\subsection{Example formulae}\label{sub:example_formulae_} % (fold)

\subsubsection{Crossing as formula.}
% 
% \begin{align*}
%   \frac{d}{dx} \sin x &= \cos x 
%   & \frac{d}{dx} e^x &= e^x \\
%   \frac{d}{dx} \cos x &= - \sin x 
%   & \frac{d}{dx} \log x &= \frac{1}{x} \\
% \end{align*} 

\begin{align*}
 \mu C(x_{0},x_{1},y_{0},y_{1},u).&(\langle x_{0}?(z) \rangle(\langle u! \rangle\langle y_{1}!z \rangle C(x_{0},x_{1},y_{0},y_{1},u)) & \\
  & \wedge \langle y_{1}?(z) \rangle (\langle u! \rangle \langle x_{0}!z \rangle C(x_{0},x_{1},y_{0},y_{1},u)) & \\
  & \wedge \langle x_{1}?(z) \rangle (\langle u? \rangle \langle y_{0}!z \rangle C(x_{0},x_{1},y_{0},y_{1},u)) & \\
  & \wedge \langle y_{0}?(z) \rangle (\langle u? \rangle \langle x_{1}!z \rangle C(x_{0},x_{1},y_{0},y_{1},u))) &
\end{align*}

The lexicographical similarity between the shape of this formulae and
the shape of definition of the process representing a crossing reveals
the intuitive meaning of this formulae. It describes the capabilities
of a process that has the right to represent a crossing. For example
it picks out processes that may perform an input on the port $x_0$ in
its initial menu of capabilities. What differentiates the formula
from the process, however, is that the crossing process is the
smallest candidate to satisfy the formula. Infinitely many other
processes -- with internal behavior hidden behind this interface, so
to speak -- also satisfy this formula. Even this simple formula,
then, can be seen to open a new view onto knots, providing a
computational interpretation of \emph{virtual} knots.

Note that this formula is derived by hand. A similar formula can be
derived by employing Caires' calculation of characteristic formula
\cite{Caires04} to the process representing a crossing. In light of
this discussion, we let
$\meaningof{C}_{\phi}(x0,x1,y0,y1,u)$ denote a formula specifying the
dynamics we wish to capture of a crossing. To guarantee we preserve
the shape of the interface and minimal semantics we demand that
$\meaningof{C}_{\phi}(x0,x1,y0,y1,u) \Rightarrow
\textbf{C}(x0,x1,y0,y1,u)$ where $\textbf{C}(x0,x1,y0,y1,u)$ denotes
the formula above.
                            
\subsubsection{Crossing number constraints.}
The moral content of the context lemma (Lemma \ref{context}) is that the notion of
``locality'' in the Reidemeister moves is effectively captured by the
parallel composition operator of the process calculus. This intuition
extends through the logic. Given a formula,
$\meaningof{C}_{\phi}(x0,x1,y0,y1,u)$, we can use the structural
connectives to specify constraints on crossing numbers, such as at
least $n$ crossings, or exactly $n$ crossings.
\begin{mathpar}
  \inferrule* [lab=at-least-n] {} { K^{\geq n}_{\phi}(\vec{xs},\vec{ys}) := \Pi_{i=0}^{n-1} Hu . \meaningof{C}_{\phi}(xs_i,ys_i,u) | T }
  \and 
  \inferrule* [lab=exactly-n] {} { K^{= n}_{\phi}(\vec{xs},\vec{ys}) := \Pi_{i=0}^{n-1} Hu . \meaningof{C}_{\phi}(xs_i,ys_i,u) | \neg (\forall x_0,y_0,x_1,y_1,u . \meaningof{C}_{\phi}(x_0,y_0,x_1,y_1,u) | T) }
\end{mathpar}

To round out this section, recall that the encoding of an $n$-crossing
knot decomposes into a parallel composition of $n$ \emph{copies} of a
crossing process together with a wiring harness. To specify different
knot classes with the same crossing number amounts to specifying
logical constraints on the wiring harness. In the interest of space,
we defer examples to a forthcoming paper. Suffice it to say that both
the conditions ``alternating knot'' and ``contains the tangle
corresponding to 5/3'' are expressible. For example, it is possible to
calculate the characteristic formula of a process corresponding to the
tangle 5/3 and conjoin it into the classifying formula via the
composition connective of the logic.

Finally, we wish to observe that it is entirely within reason to
contemplate a more domain-specific version of spatial logic tailored
to the shape of processes in the image of the encoding. Such a
domain-specific logic would have a better claim to the title formal
language of knot properties.

% subsection example_formulae_ (end)

% section knots_as_processes (end) 

% section spatial logic via knots (end)

\section{Conclusions and future work}

\paragraph{Testing physical space}
You, gentle reader, may wonder why of all the theorems to be proved
given this set up we pick the one above. In some sense it's hardly
central to quantum mechanics. We see it as central in the sense that
it firmly establishes a notion of physical space arising from a notion
of the equivalence of behavior. Relating bisimulation to a metric is a
big step forward, but one is faced with interpreting the relationship
of that metric space to something more physical. Quantum mechanical
notions of ``physical'' space are still far from intuitive, but by
relating this idea of distance as testing to calculations that predict
physical circumstances we are making a not insignificant step forward
toward an understanding of the physical space we inhabit as
essentially dynamic.

\paragraph{Effectivity and simulation}
One of the observations we have yet to make is that the entire program
spelled out here is effective. We have built various interpreters for
the reflective calculus at work in this interpretation. In principle,
then, we can simulate quantum mechanics on a computer. The place where
the simulation may lose fidelity is the infinitely branching summation
for the annihilator.

In this connection i also want to point out that the evaluation style
calculation of the inner product puts the non-determinism of the
summation right at the heart of measurement. This suggests that
Milner's original reduction-based formulation of the dynamics of his
calculi in terms of sums was not just notationally suggestive of a
notion of measure-and-continue but captured some significant part of
the physics.

\paragraph{Quantum continuations}
In light of this last observation i want to point out that the
predominant account of quantum mechanics is missing a key aspect of a
truly compositional story of the physical situation. In a real lab,
when a measurement is made the observation can be made to feed into
another device that then makes another measurement conditioned on the
results of the first. This means that after the superposition was
collapsed the entire experimental set up remained in
superposition. While QM offers a means of writing this down it doesn't
quite line up well with the well-trodden formulation of computation
and continuation that we see so succinctly expressed in Milner's
calculi. This suggests that there might be advantages to this account
of dynamics waiting to be explored.

\paragraph{Quantum logic}
In this connection, we also note that by virtue of having the
Hennessy-Milner construction, we can pull the construction through the
interpretation of QM. This gives us a natural candidate for a quantum
logic that enjoys an extremely tight connection with it's domain of
interpretation, making the construction much less ad hoc (rather it is
the image of functor!).

\paragraph{Quantum probabiity}
i have questions about the basis of the interpretation of inner
product as probability amplitude. In particular, using which
axiomatization of probability theory does the notion of probability
amplitude earn the right to be so dubbed? In other words, where is the
proof that the operation for calculating a probability amplitude (and
then squaring) satisfies the axioms of what it means to calculate a
probability? Even if such a proof exists (i have yet to find it in the
literature), i wonder if it might not be possible to turn things on
their heads. Can we view the calculation of the probability amplitude
as an axiomatization of probability? If so, then the definition we
give for calculating probability amplitude may provide the basis for
an \emph{effective} theory of probability.

\paragraph{Quantum vs ``biological'' information}
Finally, i want to conclude with a more philosophical observation. At
a recent workshop in which QM was a predominant topic i noticed
something about quantum information. The speaker was giving a riveting
discussion of axiomatic QM and showing how properties of ``no
cloning'' and ``no deleting'' emerged as consequences of the
axiomatization. Theorems of this form are necessary to give us a sense
of confidence that our axioms characterize the physical theory. What
struck me, though, was that if quantum information is neither erasable
nor replicable it is markedly different from \emph{life}. Two of the
things we know about life is that

\begin{itemize}
  \item it ends;
  \item to gain some measure of persistence, to transcend it's
    finitude it is imminently copyable.
\end{itemize}

Both of these qualities are summarized succinctly in the aphorism: all
flesh is grass. For me these two kinds of ``information'' -- call them
quantum and biological -- are end points on a spectrum of strategies
for persistence. At one end, we have those curious entities that enjoy
uniqueness and permanence; at the other, we have those who in the face
of a certain end and an uncertain present make a go of passing
something on. To me one of the more remarkable aspects of the latter
strategy is that in the presence of noise (and certain features of
copying) we get a kind of dynamism, a chance for improvement against a
given persistent condition.

% subsection other_calculi_other_bisimulations_and_geometry_as_behavior (end)




% section conclusion (end)

%\documentclass[12pt]{llncs}
%\documentclass{jktr}

\usepackage[pdftex]{hyperref}                   
\usepackage {listings}
\usepackage {mathpartir}
\usepackage{bcprules}
%\usepackage{listings}
                       
\usepackage{graphicx} 
%\usepackage[margins=2.5cm,nohead,nofoot]{geometry}
%\usepackage{geometry}
\usepackage{amsfonts}
\usepackage{amstext}
\usepackage{latexsym}
\usepackage{amssymb}
\usepackage{color}


%\include{myPreamble}
\include{qm2pi.local} 

%\ifpdf
%\usepackage[pdftex]{graphicx}
%\else
%\usepackage{graphicx}
%\fi

 % \ifpdf
%  \usepackage{pdfsync}
%  \if


%\title{Brief Article}
%\author{David F. Snyder}
%\author{L.G. Meredith}

%\address{Dept. of Math., Texas State University--San Marcos, San Marcos, TX 78666}
       
\pagestyle{empty}


\begin{document}

\lstset{language=[Objective]Caml,frame=shadowbox}

\input{qm2pi.front}

% section front matter (end)

\input{qm2pi.intro} 
 
% section introduction (end)

% \input{qm2pi.knotations} 

% section notation (end)

\input{qm2pi.process.calculi} 

% section concurrent_process_calculi_and_spatial_logics_ (end)
    
%\input{qm2pi.knots2pi} 

%\input{qm2pi.trefoil} 

%\input{qm2pi.mainthm} 

% subsection basic_interpretation (end)

%\input{qm2pi.rho.presentation} 
\subsection{The syntax and semantics of the notation system}\label{sub:the_syntax_and_semantics_of_the_notation_system} % (fold)

We now summarize a technical presentation of the calculus that
embodies our theory of dynamics. The typical presentation of such a
calculus follows the style of giving generators and relations on
them. The grammar, below, describing term constructors, freely
generates the set of processes, $\Proc$. This set is then quotiented
by a relation known as structural congruence and it is over this set
that the notion of dynamics is expressed. This presentation is
essentially that of \cite{MeredithR05} with the addition of
polyadicity and summation. For readability we have relegated some of
the technical subtleties to an appendix.

\subsubsection{Process grammar}\label{subsub:process_grammar}

\begin{mathpar}
  \inferrule* [lab=synchronization] {} {{M} \bc \pzero \;|\; x?F \;|\; x!C }
  \and
  \inferrule* [lab=abstraction] {} {{F} \bc (x)P}
  \and
  \inferrule* [lab=concretion] {} {{C} \bc \langle Q \rangle}
  \and
  \inferrule* [lab=process] {} {{P,Q} \bc M \;| \;P|Q \;|\; @{x}}
  \and
  \inferrule* [lab=name] {} {{x} \bc \quotep{P}}
\end{mathpar} 

Note that $\vec{x}$ (resp. $\vec{P}$) denotes a vector of names
(resp. processes) of length $|\vec{x}|$ (resp. $|\vec{P}|$). We adopt
the following useful abbreviations.

\begin{mathpar}
   x?(\vec{y}).P := x.(\vec{y})P \and  x\clift{\vec{P}} := x.\clift{\vec{P}}
   \and x!(y) := \lift{x}{\dropn{y}}
   \and \Pi_{i=0}^{n-1}P_i := P_0 | \ldots | P_{n-1}
\end{mathpar}

\subsubsection{Structural congruence}

\paragraph{Free and bound names and alpha-equivalence.} At the
core of structural equivalence is alpha-equivalence which identifies
process that are the same up to a change of variable. Formally, we
recognize the distinction between free and bound names. The free names
of a process, $\freenames{P}$, may be calculated recursively as
follows:

\begin{mathpar}
\freenames{\pzero} := \emptyset
  \and \\
  \freenames{x?(y).P} := \{ x \} \cup (\freenames{P} \setminus \{ y \})
  \and 
  \freenames{x!\langle P \rangle} := \{ x \} \cup \{ P \} 
  \and \\
  \freenames{P|Q} := \freenames{P} \cup \freenames{Q}
  \and \\
  \freenames{@{x}} := \{ x \}
\end{mathpar}

$\pi$
$\quotep{\pi}$

$\freenames{-} : \pi \to \mathcal{P}(\quotep{\pi})$

\begin{eqnarray*}
  \freenames{\pzero} & := & \emptyset \\
  \freenames{x?(y).P} & := & \{ x \} \cup (\freenames{P} \setminus \{ y \}) \\
  \freenames{x!\langle P \rangle} & := & \{ x \} \cup \{ P \} \\
  \freenames{P|Q} & := & \freenames{P} \cup \freenames{Q} \\
  \freenames{\dropn{x}} & := & \{ x \}
\end{eqnarray*}

The bound names of a process, $\boundnames{P}$, are those names occurring in $P$
that are not free. For example, in $x?(y).0$, the name $x$ is free, while $y$ is bound.

\begin{mathpar}
  \inferrule* [lab=monoidal-laws] {} { P|Q \equiv Q|P \and P|0 \equiv P \and P|(Q|R) \equiv (P|Q)|R }
\end{mathpar}

\begin{mathpar}
  \inferrule* [lab=alpha-equivalence] {} { (x)P \equiv (y)P\{y/x\} \and y \not\in \freenames{P} }
\end{mathpar}

\begin{definition}
Then two processes, $P,Q$, are alpha-equivalent if $P = Q\{\vec{y}/\vec{x}\}$ for
some $\vec{x} \in \boundnames{Q},\vec{y} \in \boundnames{P}$, where $Q\{\vec{y}/\vec{x}\}$
denotes the capture-avoiding substitution of $\vec{y}$ for $\vec{x}$ in $Q$.
\end{definition}

\begin{definition}
  The {\em structural congruence} \cite{SangiorgiWalker} , $\equiv$,
  between processes is the least congruence containing
  alpha-equivalence, satisfying the abelian monoid laws
  (associativity, commutativity and $\pzero$ as identity) for parallel
  composition $|$ and for summation $+$.
\end{definition}

\subsection{Name equivalence}

We take name equivalence, written $\nameeq$, to be the smallest
equivalence relation generated by the following rules.

\begin{mathpar}
\inferrule*[lab=Quote-drop]
{ }
{ \quotep{@{x}} \nameeq x }

\inferrule*[lab=Struct-equiv]
{ P \scong Q }
{ \quotep{P} \nameeq \quotep{Q} }
\end{mathpar}

The astute reader will have noticed that the mutual recursion of names
and processes imposes a mutual recursion on alpha-equivalence and
structural equivalence via name-equivalence. Fortunately, all of this
works out pleasantly and we may calculate in the natural way, free of
concern. The reader interested in the details is referred to the
appendix \ref{appendix:rho_details}.

\subsection{Substitution}

We use $\Proc$ for the set of processes, $\QProc$ for the set of
names, and $\id{\{}\vec{y} / \vec{x} \id{\}}$ to denote partial maps,
$s : \QProc \rightarrow \QProc$. A map, $s$ lifts, uniquely, to a map
on process terms, $\widehat{s} : \Proc \rightarrow \Proc$ by the
following equations.

\begin{mathpar}
  (0) \psubstp{Q}{P} := 0 \\
  (R \juxtap S) \psubstp{Q}{P}
  :=    
  (R)\psubstp{Q}{P} \juxtap (S) \psubstp{Q}{P} \\
  (x?(y).R) \psubstp{Q}{P}    
  :=    
  (x)\substp{Q}{P} (z)\concat( (R \psubstn{z}{y}) \psubstp{Q}{P} ) \\
  (\lift{x}{R}) \psubstp{Q}{P}  
  :=
  \lift{(x)\substp{Q}{P}}{ R \psubstp{Q}{P} } \\
%   (\dropn{x})  \psubstp{Q}{P}       
%   := 
%   \left\{ 
%     \begin{array}{ccc} 
%       \dropn{\quotep{Q}} & & x \nameeq \quotep{P} \\
%       \dropn{x} & & otherwise \\
%     \end{array}
%   \right. 
  (\dropn{x})  \psubstp{Q}{P}       
  := 
  \left\{ 
    \begin{array}{ccc} 
      Q & & x \nameeq \quotep{P} \\
      \dropn{x} & & otherwise \\
    \end{array}
  \right.
\end{mathpar}
 

where

\begin{eqnarray}
  (x)\id{\{} \lpquote Q \rpquote / \lpquote P \rpquote \id{\}}            = 
  \left\{ 
    \begin{array}{ccc}
      \lpquote Q \rpquote & & x \nameeq \lpquote P \rpquote \\
      x & & otherwise \\
    \end{array}
  \right. \nonumber
\end{eqnarray}

and $z$ is chosen distinct from $\quotep{P}$, $\quotep{Q}$, the free
names in $Q$, and all the names in $R$. Our $\alpha$-equivalence will
be built in the standard way from this substitution.

\begin{remark}\label{rem:no_self_referential_names}
  One consequence of these definitions is that $\forall P. \quotep{P}
  \not\in \freenames{P}$.
\end{remark}

\subsection{ Dynamic quote: an example }

Anticipating something of what's to come, consider applying the
substitution, $\widehat{\id{\{}u / z \id{\}}}$, to the following pair
of processes, $\lift{w}{y!(z)}$ and $w[ \lpquote y!(z) \rpquote ]$.

\begin{eqnarray}
	\lift{w}{y!(z)}\widehat{\id{\{}u / z \id{\}}}
		& = &
		\lift{w}{y!(u)} \nonumber\\
	w[ \lpquote y!(z) \rpquote ] \widehat{ \id{\{}u / z \id{\}} }
		& = &
		w[ \lpquote y!(z) \rpquote ] \nonumber
\end{eqnarray}

Because the body of the process between quotes is impervious to
substitution, we get radically different answers. In fact, by
examining the first process in an input context,
e.g. $x?(z).\lift{w}{y!(z)}$, we see that the process under the lift
operator may be shaped by prefixed inputs binding a name inside it. In
this sense, the lift operator will be seen as a way to dynamically
construct processes before reifying them as names.

Finally equipped with these standard features we can present the
dynamics of the calculus.

\subsubsection{Operational semantics} 

Finally, we introduce the computational dynamics. What marks these
algebras as distinct from other more traditionally studied algebraic
structures, e.g. vector spaces or polynomial rings, is the manner in
which dynamics is captured. In traditional structures, dynamics is typically
expressed through morphisms between such structures, as in linear maps
between vector spaces or morphisms between rings. In algebras
associated with the semantics of computation, the dynamics is
expressed as part of the algebraic structure itself, through a
reduction reduction relation typically denoted by $\red$. Below, we
give a recursive presentation of this relation for the calculus used
in the encoding.

$\red \subseteq \pi \times \pi$
$\red : \pi \to \mathcal{P}(\pi)$

\begin{mathpar}
  \inferrule* [lab=Comm] { \textsf{match}( x_{src}, x_{trgt} ) } { x_{trgt}?(y)P \; | \; x_{src}!\langle {Q} \rangle \red P\{\quotep{Q}/y}\} }
  \and \\
  \inferrule* [lab=Par] {{P} \red {P}'} {{{P} | {Q}} \red {{P}' | {Q}}}
  \and
  \inferrule* [lab=Equiv]{{{P} \scong {P}'} \andalso {{P}' \red {Q}'} \andalso {{Q}' \scong {Q}}}{{P} \red {Q}}
\end{mathpar}

\begin{eqnarray*}
  match_{\equiv} (\quotep{P},\quotep{Q}) & := & P \equiv Q \\
  match_{\dagger}(\quotep{P},\quotep{Q}) & := & \forall R. P|Q \red^{*} R => R \red^{*} 0 \\
  match_{K}(\quotep{P},\quotep{Q}) & := & K \mbox{ for some context } K
\end{eqnarray*}

$u?(x)P | u!\langle Q \rangle \red P\{\quotep{Q}/x\}$

%We write $\wred$ for $\red^*$, and $P\red$ if $\exists Q $ such that $ P \red Q$.
We write $P\red$ if $\exists Q $ such that $ P \red Q$ and $P\not\red$, otherwise.

\section{Replication}

As mentioned before, it is known that replication (and hence
recursion) can be implemented in a higher-order process algebra
\cite{SangiorgiWalker}. As our first example of calculation with the
machinery thus far presented we give the construction explicitly in
the {\rhoc}.

\begin{eqnarray}
	D_{x} & := & \prefix{x}{y}{(\binpar{\outputp{x}{y}}{@{y}})} \nonumber\\
	\bangp_{x}{P} & := & \binpar{{x}!\langle{\binpar{D_{x}}{P}}\rangle}{D_{x}} \nonumber
\end{eqnarray}

\begin{eqnarray}
	\bangp_{x}{P} & & \nonumber\\
	=
	& {x}!\langle{(\prefix{x}{y}{(\outputp{x}{y} | @{y})) | P}}\rangle 
	      | \prefix{x}{y}{(\outputp{x}{y} | @{y})} & \nonumber\\
	\red
	& (\outputp{x}{y} | @{y})\substn{\quotep{(\prefix{x}{y}{(@{y} | \outputp{x}{y})) | P}}}{y} & \nonumber\\
	=
	& \outputp{x}{\quotep{(\prefix{x}{y}{(\outputp{x}{y} | @{y})) | P}}}
	  | {(\prefix{x}{y}{(\outputp{x}{y} | @{y})) | P}} & \nonumber\\
	\red
	& \ldots & \nonumber\\
	\red^*
	& P | P | \ldots & \nonumber
\end{eqnarray}

Of course, this encoding, as an implementation, runs away, unfolding
$\bangp{P}$ eagerly. A lazier and more implementable replication
operator, restricted to input-guarded processes, may be obtained as follows.

\begin{eqnarray}
\bangp{\prefix{u}{v}{P}} 
	:= 
	\binpar{\lift{x}{\prefix{u}{v}{(\binpar{D(x)}{P})}}}{D(x)} \nonumber
\end{eqnarray}

\begin{remark}
  Note that the lazier definition still does not deal with summation
  or mixed summation (i.e. sums over input and output). The reader is
  invited to construct definitions of replication that deal with these
  features. 

  Further, the definitions are parameterized in a name, $x$. Can you,
  gentle reader, make a definition that eliminates this parameter and
  guarantees no accidental interaction between the replication
  machinery and the process being replicated -- i.e. no accidental
  sharing of names used by the process to get its work done and the
  name(s) used by the replication to effect copying. This latter
  revision of the definition of replication is crucial to obtaining
  the expected identity $!!P \sim !P$.
\end{remark}

\begin{remark}\label{rem:paradoxical_combinator}
  The reader familiar with the lambda calculus will have noticed the
  similarity between $D$ and the paradoxical combinator.

  [Ed. note: the existence of this seems to suggest we have to be more
  restrictive on the set of processes and names we admit if we are to
  support no-cloning.]
\end{remark}

\subsubsection{Bisimulation}

The computational dynamics gives rise to another kind of equivalence,
the equivalence of computational behavior. As previously mentioned
this is typically captured \emph{via} some form of bisimulation.

% The notion we use in this paper is weak barbed bisimulation
% \cite{milner91polyadicpi}.

The notion we use in this paper is derived from weak barbed
bisimulation \cite{milner91polyadicpi}. 

\begin{definition}
An \emph{observation relation}, $\downarrow_{\mathcal N}$, over a set
of names, $\mathcal N$, is the smallest relation satisfying the rules
below.

\infrule[Out-barb]{y \in {\mathcal N}, \; x \nameeq y}
		  {\outputp{x}{v} \downarrow_{\mathcal N} x}
\infrule[Par-barb]{\mbox{$P\downarrow_{\mathcal N} x$ or $Q\downarrow_{\mathcal N} x$}}
		  {\binpar{P}{Q} \downarrow_{\mathcal N} x}

We write $P \Downarrow_{\mathcal N} x$ if there is $Q$ such that 
$P \wred Q$ and $Q \downarrow_{\mathcal N} x$.
\end{definition}

\begin{definition}
%\label{def.bbisim}
An  ${\mathcal N}$-\emph{barbed bisimulation} over a set of names, ${\mathcal N}$, is a symmetric binary relation 
${\mathcal S}_{\mathcal N}$ between agents such that $P\rel{S}_{\mathcal N}Q$ implies:
\begin{enumerate}
\item If $P \red P'$ then $Q \wred Q'$ and $P'\rel{S}_{\mathcal N} Q'$.
\item If $P\downarrow_{\mathcal N} x$, then $Q\Downarrow_{\mathcal N} x$.
\end{enumerate}
$P$ is ${\mathcal N}$-barbed bisimilar to $Q$, written
$P \wbbisim_{\mathcal N} Q$, if $P \rel{S}_{\mathcal N} Q$ for some ${\mathcal N}$-barbed bisimulation ${\mathcal S}_{\mathcal N}$.
\end{definition}

$\mathcal{R} \subseteq \pi \times \pi$

$P \mathcal{R} Q => \forall P'. P \red P' \Rightarrow \exists Q'. Q \red Q', P' \mathcal{R} Q'$

$P \vdash x \Rightarrow Q \vdash x$

\begin{mathpar}
  \inferrule*[lab=Out-barb]{x \nameeq y}{{y}!\langle{Q}\rangle \vdash x}
  \and
  \inferrule*[lab=Par-barb]{\mbox{$P\vdash x$ or $Q\vdash x$}}{\binpar{P}{Q} \vdash x}
\end{mathpar}

\subsubsection{Contexts}

One of the principle advantages of computational calculi like the
$\pi$-calculus is a well-defined notion of context,
contextual-equivalence and a correlation between
contextual-equivalence and notions of bisimulation. The notion of
context allows the decomposition of a process into (sub-)process and
its syntactic environment, its context. Thus, a context may be
thought of as a process with a ``hole'' (written $\Box$) in it. The
application of a context $M$ to a process $P$, written $M[P]$, is
tantamount to filling the hole in $M$ with $P$. In this paper we do
not need the full weight of this theory, but do make use of the notion
of context in the proof the main theorem. 

\begin{mathpar}
  \inferrule* [lab=summation] {} {{M_{M},M_{N}} \bc \Box \;|\; x.M_{A} \;|\; M_{M}+M_{N}}
  \and
  \inferrule* [lab=agent] {} {{M_{A}} \bc (\vec{x})M_{P} \;| \; \clift{P_0,\ldots,M_{P},\ldots,P_N}}
  \and \\
  \inferrule* [lab=process] {} {{M_{P}} \bc M_{N} \;| \;P|M_{P} }
\end{mathpar} 

\begin{mathpar}
  \inferrule* [lab=sychronization] {} {M_{N} \bc \Box \;|\; x?M_{F} \;|\; x!M_{C}}
  \and
  \inferrule* [lab=abstraction] {} {{M_{F}} \bc (x)M_{P} }
  \and
  \inferrule* [lab=concretion] {} {{M_{C}} \bc \langle M_{P} \rangle }
  \and \\
  \inferrule* [lab=process] {} {{M_{P}} \bc M_{N} \;| \;P|M_{P} }
\end{mathpar}

\begin{definition}[contextual application] Given a context $M$, and
  process $P$, we define the \emph{contextual application}, $M[P] :=
  M\{P/\Box\}$. That is, the contextual application of M to P is the
  substitution of $P$ for $\Box$ in $M$.
\end{definition}

$\meaningof{-} : L \to \mathcal{P}(\pi)$

\begin{mathpar}
  \inferrule* [lab=collection] {} {\meaningof{true} = \pi, \and \meaningof{~E} = \pi \setminus \meaningof{E}, \and \meaningof{E_{1} \& E_{2}} = \meaningof{E_{1}} \cap \meaningof{E_{2}}}
\end{mathpar}

\begin{mathpar}
  \inferrule* [lab=structure] {} {\meaningof{0} = \{ P \in \pi | P \equiv 0 \}, \and \\ \meaningof{E_1 | E_2} = \{ P \in \pi | P \equiv P_{1} | P_{2}, P_{1} \in \meaningof{E_{1}}, P_{2} \in \meaningof{E_2}\} }
\end{mathpar}

\begin{mathpar}
 \inferrule* [lab=behavior] {} {\meaningof{\langle a?b \rangle E} = \{ P \in \pi | P \equiv Q | u?(y)P', \\ \and \\\\ \and \\ \;\;\; u \in \meaningof{a}, \forall z.P'\{z/y\} \in \meaningof{E\{z/b\}}\}, \and \\ \meaningof{a!E} = \{ P \in \pi | P \equiv Q | x!\langle P' \rangle, x \in \meaningof{a} P' \in \meaningof{E}\} }
\end{mathpar}

\begin{mathpar}
 \inferrule* [lab=nominal] {} {\meaningof{\quotep{E}} = \{ \quotep{P} \in \quotep{\pi} | P \in \meaningof{E} \}, \and \meaningof{\quotep{P}} = \{ \quotep{Q} \in \quotep{\pi} | P \equiv Q \} \and \\ \meaningof{@\quotep{E}} = \{ P \in \pi | P \equiv @x, x \in \meaningof{E} \}}
\end{mathpar}

\begin{eqnarray*}
  \\
  \meaningof{-} : TS \to ST
\end{eqnarray*}

\begin{eqnarray*}
  \\
  L : TS \to ST
\end{eqnarray*}

\begin{eqnarray*}
  \\
  P \models E \iff P \in \meaningof{E}
\end{eqnarray*}

\begin{eqnarray*}
  P \approx_{L} Q \iff \forall E \in L. P \models E \iff Q \models E
\end{eqnarray*}

\begin{eqnarray*}
  P \approx_{K} Q
\end{eqnarray*}

\begin{eqnarray*}
  P \approx Q
\end{eqnarray*}

$\approx_{K} = \approx = \approx_{L}$

\subsubsection{Contextual duality}

Note that contexts extend the quotation operation to a family of
operations from processes to names. Given a context, $M$, we can
define a \emph{nominal context}, $\quotep{M}$ by $\quotep{M}[P] :=
\quotep{M[P]}$. To foreshadow what is to come we observe that these
operations enjoy a duality with processes very much like the duality
between vectors and maps from vectors to scalars.

Further, because the calculus is essentially higher-order, we have a
correspondence between contexts and processes. More specifically,
given a name $x$ and a context $M$ we can construct $M^{*}_{x}$ such
that 

\begin{mathpar}
  M^{*}_{x} | \lift{x}{P} \red M[P]
\end{mathpar}

namely,

\begin{mathpar}
  M^{*}_{x} := x?(u).M[\dropn{u}]
\end{mathpar}

The dependence of $M^{*}_{x}$ on a name makes it an abstraction, 

\begin{mathpar}
  M^{*} := (x)x?(u).M[\dropn{u}]
\end{mathpar}

\subsection{Additional notation}

It will sometimes be convenient to denote the process a name
quotes. We already have the notation $x = \quotep{P}$, but it will be
convenient to introduce an alternate notation, $\procn{x}$, when we
want to emphasize the connection to the use of the name. Note that, by
virtue of name equivalence, $\quotep{\procn{x}} \nameeq x$; so, the
notation is consistent with previous definitions.

Further, because names have structure it is possible to effect
substitutions on the basis of that structure. This means we need to
upgrade our notation for substitutions, which we accomplish by
adapting comprehension notation. Thus,

\begin{mathpar}
  P\{ y / x : x \in S \}
\end{mathpar}

is interpreted to mean the process derived from P by replacing (in a
capture-avoiding manner) each occurrence of $x$ in $S$ by $y$. For example,

\begin{mathpar}
  P\{ \quotep{\procn{x}|\procn{x}} / x : x \in \freenames{P} \}
\end{mathpar}

will replace each (occurrence) of a free name $x$ in $P$ by
$\quotep{\procn{x}|\procn{x}}$.

Also, we will avail ourselves of the notation $x^{L}$ and $x^{R}$ to
denote injections of a name into disjoint copies of the name
space. There are numerous ways to accomplish this. One example can be
found in \cite{MeredithR05}. This notation overloads to vectors of
names: $\vec{x}^{\pi} := (x_{i}^{\pi} \; : \; 0 \leq i < |\vec{x}| )$ where $\pi \in \{L,R\}$.

We also use $P^{\Box} := P|\Box$.

In \cite{MeredithR05} an interpretation of the new operator is
given. It turns out that there are several possible interpretations
all enjoying the requisite algebraic properties of the operator (see
\cite{milner91polyadicpi}). We will therefore make liberal use of
$(\nu\; \vec{x})P$.

% subsection the_syntax_and_semantics_of_the_notation_system (end)   

\input{qm2pi.qmops} 

\input{qm2pi.sterngerlach} 

\input{qm2pi.metric} 

% section concurrent_process_calculi (end)

%\input{qm2pi.proofsketch}

% section proof sketch (end)

%\input{qm2pi.slviaknots} 

% section spatial logic via knots (end)

\input{qm2pi.conclusion}

% section conclusion (end)

%\input{qm2pi.dtcodes} 

% section wiring algorithm (end)

\input{qm2pi.ack} 

% section acknowledgments (end)

\newpage


\bibliographystyle{plain}   
\bibliography{../../biblios/main.bib}

\input{qm2pi.rhodetails}

\end{document}

 

% section wiring algorithm (end)

\documentclass[12pt]{llncs}
%\documentclass{jktr}

\usepackage[pdftex]{hyperref}                   
\usepackage {listings}
\usepackage {mathpartir}
\usepackage{bcprules}
%\usepackage{listings}
                       
\usepackage{graphicx} 
%\usepackage[margins=2.5cm,nohead,nofoot]{geometry}
%\usepackage{geometry}
\usepackage{amsfonts}
\usepackage{amstext}
\usepackage{latexsym}
\usepackage{amssymb}
\usepackage{color}


%\include{myPreamble}
\include{qm2pi.local} 

%\ifpdf
%\usepackage[pdftex]{graphicx}
%\else
%\usepackage{graphicx}
%\fi

 % \ifpdf
%  \usepackage{pdfsync}
%  \if


%\title{Brief Article}
%\author{David F. Snyder}
%\author{L.G. Meredith}

%\address{Dept. of Math., Texas State University--San Marcos, San Marcos, TX 78666}
       
\pagestyle{empty}


\begin{document}

\lstset{language=[Objective]Caml,frame=shadowbox}

\input{qm2pi.front}

% section front matter (end)

\input{qm2pi.intro} 
 
% section introduction (end)

% \input{qm2pi.knotations} 

% section notation (end)

\input{qm2pi.process.calculi} 

% section concurrent_process_calculi_and_spatial_logics_ (end)
    
%\input{qm2pi.knots2pi} 

%\input{qm2pi.trefoil} 

%\input{qm2pi.mainthm} 

% subsection basic_interpretation (end)

%\input{qm2pi.rho.presentation} 
\subsection{The syntax and semantics of the notation system}\label{sub:the_syntax_and_semantics_of_the_notation_system} % (fold)

We now summarize a technical presentation of the calculus that
embodies our theory of dynamics. The typical presentation of such a
calculus follows the style of giving generators and relations on
them. The grammar, below, describing term constructors, freely
generates the set of processes, $\Proc$. This set is then quotiented
by a relation known as structural congruence and it is over this set
that the notion of dynamics is expressed. This presentation is
essentially that of \cite{MeredithR05} with the addition of
polyadicity and summation. For readability we have relegated some of
the technical subtleties to an appendix.

\subsubsection{Process grammar}\label{subsub:process_grammar}

\begin{mathpar}
  \inferrule* [lab=synchronization] {} {{M} \bc \pzero \;|\; x?F \;|\; x!C }
  \and
  \inferrule* [lab=abstraction] {} {{F} \bc (x)P}
  \and
  \inferrule* [lab=concretion] {} {{C} \bc \langle Q \rangle}
  \and
  \inferrule* [lab=process] {} {{P,Q} \bc M \;| \;P|Q \;|\; @{x}}
  \and
  \inferrule* [lab=name] {} {{x} \bc \quotep{P}}
\end{mathpar} 

Note that $\vec{x}$ (resp. $\vec{P}$) denotes a vector of names
(resp. processes) of length $|\vec{x}|$ (resp. $|\vec{P}|$). We adopt
the following useful abbreviations.

\begin{mathpar}
   x?(\vec{y}).P := x.(\vec{y})P \and  x\clift{\vec{P}} := x.\clift{\vec{P}}
   \and x!(y) := \lift{x}{\dropn{y}}
   \and \Pi_{i=0}^{n-1}P_i := P_0 | \ldots | P_{n-1}
\end{mathpar}

\subsubsection{Structural congruence}

\paragraph{Free and bound names and alpha-equivalence.} At the
core of structural equivalence is alpha-equivalence which identifies
process that are the same up to a change of variable. Formally, we
recognize the distinction between free and bound names. The free names
of a process, $\freenames{P}$, may be calculated recursively as
follows:

\begin{mathpar}
\freenames{\pzero} := \emptyset
  \and \\
  \freenames{x?(y).P} := \{ x \} \cup (\freenames{P} \setminus \{ y \})
  \and 
  \freenames{x!\langle P \rangle} := \{ x \} \cup \{ P \} 
  \and \\
  \freenames{P|Q} := \freenames{P} \cup \freenames{Q}
  \and \\
  \freenames{@{x}} := \{ x \}
\end{mathpar}

$\pi$
$\quotep{\pi}$

$\freenames{-} : \pi \to \mathcal{P}(\quotep{\pi})$

\begin{eqnarray*}
  \freenames{\pzero} & := & \emptyset \\
  \freenames{x?(y).P} & := & \{ x \} \cup (\freenames{P} \setminus \{ y \}) \\
  \freenames{x!\langle P \rangle} & := & \{ x \} \cup \{ P \} \\
  \freenames{P|Q} & := & \freenames{P} \cup \freenames{Q} \\
  \freenames{\dropn{x}} & := & \{ x \}
\end{eqnarray*}

The bound names of a process, $\boundnames{P}$, are those names occurring in $P$
that are not free. For example, in $x?(y).0$, the name $x$ is free, while $y$ is bound.

\begin{mathpar}
  \inferrule* [lab=monoidal-laws] {} { P|Q \equiv Q|P \and P|0 \equiv P \and P|(Q|R) \equiv (P|Q)|R }
\end{mathpar}

\begin{mathpar}
  \inferrule* [lab=alpha-equivalence] {} { (x)P \equiv (y)P\{y/x\} \and y \not\in \freenames{P} }
\end{mathpar}

\begin{definition}
Then two processes, $P,Q$, are alpha-equivalent if $P = Q\{\vec{y}/\vec{x}\}$ for
some $\vec{x} \in \boundnames{Q},\vec{y} \in \boundnames{P}$, where $Q\{\vec{y}/\vec{x}\}$
denotes the capture-avoiding substitution of $\vec{y}$ for $\vec{x}$ in $Q$.
\end{definition}

\begin{definition}
  The {\em structural congruence} \cite{SangiorgiWalker} , $\equiv$,
  between processes is the least congruence containing
  alpha-equivalence, satisfying the abelian monoid laws
  (associativity, commutativity and $\pzero$ as identity) for parallel
  composition $|$ and for summation $+$.
\end{definition}

\subsection{Name equivalence}

We take name equivalence, written $\nameeq$, to be the smallest
equivalence relation generated by the following rules.

\begin{mathpar}
\inferrule*[lab=Quote-drop]
{ }
{ \quotep{@{x}} \nameeq x }

\inferrule*[lab=Struct-equiv]
{ P \scong Q }
{ \quotep{P} \nameeq \quotep{Q} }
\end{mathpar}

The astute reader will have noticed that the mutual recursion of names
and processes imposes a mutual recursion on alpha-equivalence and
structural equivalence via name-equivalence. Fortunately, all of this
works out pleasantly and we may calculate in the natural way, free of
concern. The reader interested in the details is referred to the
appendix \ref{appendix:rho_details}.

\subsection{Substitution}

We use $\Proc$ for the set of processes, $\QProc$ for the set of
names, and $\id{\{}\vec{y} / \vec{x} \id{\}}$ to denote partial maps,
$s : \QProc \rightarrow \QProc$. A map, $s$ lifts, uniquely, to a map
on process terms, $\widehat{s} : \Proc \rightarrow \Proc$ by the
following equations.

\begin{mathpar}
  (0) \psubstp{Q}{P} := 0 \\
  (R \juxtap S) \psubstp{Q}{P}
  :=    
  (R)\psubstp{Q}{P} \juxtap (S) \psubstp{Q}{P} \\
  (x?(y).R) \psubstp{Q}{P}    
  :=    
  (x)\substp{Q}{P} (z)\concat( (R \psubstn{z}{y}) \psubstp{Q}{P} ) \\
  (\lift{x}{R}) \psubstp{Q}{P}  
  :=
  \lift{(x)\substp{Q}{P}}{ R \psubstp{Q}{P} } \\
%   (\dropn{x})  \psubstp{Q}{P}       
%   := 
%   \left\{ 
%     \begin{array}{ccc} 
%       \dropn{\quotep{Q}} & & x \nameeq \quotep{P} \\
%       \dropn{x} & & otherwise \\
%     \end{array}
%   \right. 
  (\dropn{x})  \psubstp{Q}{P}       
  := 
  \left\{ 
    \begin{array}{ccc} 
      Q & & x \nameeq \quotep{P} \\
      \dropn{x} & & otherwise \\
    \end{array}
  \right.
\end{mathpar}
 

where

\begin{eqnarray}
  (x)\id{\{} \lpquote Q \rpquote / \lpquote P \rpquote \id{\}}            = 
  \left\{ 
    \begin{array}{ccc}
      \lpquote Q \rpquote & & x \nameeq \lpquote P \rpquote \\
      x & & otherwise \\
    \end{array}
  \right. \nonumber
\end{eqnarray}

and $z$ is chosen distinct from $\quotep{P}$, $\quotep{Q}$, the free
names in $Q$, and all the names in $R$. Our $\alpha$-equivalence will
be built in the standard way from this substitution.

\begin{remark}\label{rem:no_self_referential_names}
  One consequence of these definitions is that $\forall P. \quotep{P}
  \not\in \freenames{P}$.
\end{remark}

\subsection{ Dynamic quote: an example }

Anticipating something of what's to come, consider applying the
substitution, $\widehat{\id{\{}u / z \id{\}}}$, to the following pair
of processes, $\lift{w}{y!(z)}$ and $w[ \lpquote y!(z) \rpquote ]$.

\begin{eqnarray}
	\lift{w}{y!(z)}\widehat{\id{\{}u / z \id{\}}}
		& = &
		\lift{w}{y!(u)} \nonumber\\
	w[ \lpquote y!(z) \rpquote ] \widehat{ \id{\{}u / z \id{\}} }
		& = &
		w[ \lpquote y!(z) \rpquote ] \nonumber
\end{eqnarray}

Because the body of the process between quotes is impervious to
substitution, we get radically different answers. In fact, by
examining the first process in an input context,
e.g. $x?(z).\lift{w}{y!(z)}$, we see that the process under the lift
operator may be shaped by prefixed inputs binding a name inside it. In
this sense, the lift operator will be seen as a way to dynamically
construct processes before reifying them as names.

Finally equipped with these standard features we can present the
dynamics of the calculus.

\subsubsection{Operational semantics} 

Finally, we introduce the computational dynamics. What marks these
algebras as distinct from other more traditionally studied algebraic
structures, e.g. vector spaces or polynomial rings, is the manner in
which dynamics is captured. In traditional structures, dynamics is typically
expressed through morphisms between such structures, as in linear maps
between vector spaces or morphisms between rings. In algebras
associated with the semantics of computation, the dynamics is
expressed as part of the algebraic structure itself, through a
reduction reduction relation typically denoted by $\red$. Below, we
give a recursive presentation of this relation for the calculus used
in the encoding.

$\red \subseteq \pi \times \pi$
$\red : \pi \to \mathcal{P}(\pi)$

\begin{mathpar}
  \inferrule* [lab=Comm] { \textsf{match}( x_{src}, x_{trgt} ) } { x_{trgt}?(y)P \; | \; x_{src}!\langle {Q} \rangle \red P\{\quotep{Q}/y}\} }
  \and \\
  \inferrule* [lab=Par] {{P} \red {P}'} {{{P} | {Q}} \red {{P}' | {Q}}}
  \and
  \inferrule* [lab=Equiv]{{{P} \scong {P}'} \andalso {{P}' \red {Q}'} \andalso {{Q}' \scong {Q}}}{{P} \red {Q}}
\end{mathpar}

\begin{eqnarray*}
  match_{\equiv} (\quotep{P},\quotep{Q}) & := & P \equiv Q \\
  match_{\dagger}(\quotep{P},\quotep{Q}) & := & \forall R. P|Q \red^{*} R => R \red^{*} 0 \\
  match_{K}(\quotep{P},\quotep{Q}) & := & K \mbox{ for some context } K
\end{eqnarray*}

$u?(x)P | u!\langle Q \rangle \red P\{\quotep{Q}/x\}$

%We write $\wred$ for $\red^*$, and $P\red$ if $\exists Q $ such that $ P \red Q$.
We write $P\red$ if $\exists Q $ such that $ P \red Q$ and $P\not\red$, otherwise.

\section{Replication}

As mentioned before, it is known that replication (and hence
recursion) can be implemented in a higher-order process algebra
\cite{SangiorgiWalker}. As our first example of calculation with the
machinery thus far presented we give the construction explicitly in
the {\rhoc}.

\begin{eqnarray}
	D_{x} & := & \prefix{x}{y}{(\binpar{\outputp{x}{y}}{@{y}})} \nonumber\\
	\bangp_{x}{P} & := & \binpar{{x}!\langle{\binpar{D_{x}}{P}}\rangle}{D_{x}} \nonumber
\end{eqnarray}

\begin{eqnarray}
	\bangp_{x}{P} & & \nonumber\\
	=
	& {x}!\langle{(\prefix{x}{y}{(\outputp{x}{y} | @{y})) | P}}\rangle 
	      | \prefix{x}{y}{(\outputp{x}{y} | @{y})} & \nonumber\\
	\red
	& (\outputp{x}{y} | @{y})\substn{\quotep{(\prefix{x}{y}{(@{y} | \outputp{x}{y})) | P}}}{y} & \nonumber\\
	=
	& \outputp{x}{\quotep{(\prefix{x}{y}{(\outputp{x}{y} | @{y})) | P}}}
	  | {(\prefix{x}{y}{(\outputp{x}{y} | @{y})) | P}} & \nonumber\\
	\red
	& \ldots & \nonumber\\
	\red^*
	& P | P | \ldots & \nonumber
\end{eqnarray}

Of course, this encoding, as an implementation, runs away, unfolding
$\bangp{P}$ eagerly. A lazier and more implementable replication
operator, restricted to input-guarded processes, may be obtained as follows.

\begin{eqnarray}
\bangp{\prefix{u}{v}{P}} 
	:= 
	\binpar{\lift{x}{\prefix{u}{v}{(\binpar{D(x)}{P})}}}{D(x)} \nonumber
\end{eqnarray}

\begin{remark}
  Note that the lazier definition still does not deal with summation
  or mixed summation (i.e. sums over input and output). The reader is
  invited to construct definitions of replication that deal with these
  features. 

  Further, the definitions are parameterized in a name, $x$. Can you,
  gentle reader, make a definition that eliminates this parameter and
  guarantees no accidental interaction between the replication
  machinery and the process being replicated -- i.e. no accidental
  sharing of names used by the process to get its work done and the
  name(s) used by the replication to effect copying. This latter
  revision of the definition of replication is crucial to obtaining
  the expected identity $!!P \sim !P$.
\end{remark}

\begin{remark}\label{rem:paradoxical_combinator}
  The reader familiar with the lambda calculus will have noticed the
  similarity between $D$ and the paradoxical combinator.

  [Ed. note: the existence of this seems to suggest we have to be more
  restrictive on the set of processes and names we admit if we are to
  support no-cloning.]
\end{remark}

\subsubsection{Bisimulation}

The computational dynamics gives rise to another kind of equivalence,
the equivalence of computational behavior. As previously mentioned
this is typically captured \emph{via} some form of bisimulation.

% The notion we use in this paper is weak barbed bisimulation
% \cite{milner91polyadicpi}.

The notion we use in this paper is derived from weak barbed
bisimulation \cite{milner91polyadicpi}. 

\begin{definition}
An \emph{observation relation}, $\downarrow_{\mathcal N}$, over a set
of names, $\mathcal N$, is the smallest relation satisfying the rules
below.

\infrule[Out-barb]{y \in {\mathcal N}, \; x \nameeq y}
		  {\outputp{x}{v} \downarrow_{\mathcal N} x}
\infrule[Par-barb]{\mbox{$P\downarrow_{\mathcal N} x$ or $Q\downarrow_{\mathcal N} x$}}
		  {\binpar{P}{Q} \downarrow_{\mathcal N} x}

We write $P \Downarrow_{\mathcal N} x$ if there is $Q$ such that 
$P \wred Q$ and $Q \downarrow_{\mathcal N} x$.
\end{definition}

\begin{definition}
%\label{def.bbisim}
An  ${\mathcal N}$-\emph{barbed bisimulation} over a set of names, ${\mathcal N}$, is a symmetric binary relation 
${\mathcal S}_{\mathcal N}$ between agents such that $P\rel{S}_{\mathcal N}Q$ implies:
\begin{enumerate}
\item If $P \red P'$ then $Q \wred Q'$ and $P'\rel{S}_{\mathcal N} Q'$.
\item If $P\downarrow_{\mathcal N} x$, then $Q\Downarrow_{\mathcal N} x$.
\end{enumerate}
$P$ is ${\mathcal N}$-barbed bisimilar to $Q$, written
$P \wbbisim_{\mathcal N} Q$, if $P \rel{S}_{\mathcal N} Q$ for some ${\mathcal N}$-barbed bisimulation ${\mathcal S}_{\mathcal N}$.
\end{definition}

$\mathcal{R} \subseteq \pi \times \pi$

$P \mathcal{R} Q => \forall P'. P \red P' \Rightarrow \exists Q'. Q \red Q', P' \mathcal{R} Q'$

$P \vdash x \Rightarrow Q \vdash x$

\begin{mathpar}
  \inferrule*[lab=Out-barb]{x \nameeq y}{{y}!\langle{Q}\rangle \vdash x}
  \and
  \inferrule*[lab=Par-barb]{\mbox{$P\vdash x$ or $Q\vdash x$}}{\binpar{P}{Q} \vdash x}
\end{mathpar}

\subsubsection{Contexts}

One of the principle advantages of computational calculi like the
$\pi$-calculus is a well-defined notion of context,
contextual-equivalence and a correlation between
contextual-equivalence and notions of bisimulation. The notion of
context allows the decomposition of a process into (sub-)process and
its syntactic environment, its context. Thus, a context may be
thought of as a process with a ``hole'' (written $\Box$) in it. The
application of a context $M$ to a process $P$, written $M[P]$, is
tantamount to filling the hole in $M$ with $P$. In this paper we do
not need the full weight of this theory, but do make use of the notion
of context in the proof the main theorem. 

\begin{mathpar}
  \inferrule* [lab=summation] {} {{M_{M},M_{N}} \bc \Box \;|\; x.M_{A} \;|\; M_{M}+M_{N}}
  \and
  \inferrule* [lab=agent] {} {{M_{A}} \bc (\vec{x})M_{P} \;| \; \clift{P_0,\ldots,M_{P},\ldots,P_N}}
  \and \\
  \inferrule* [lab=process] {} {{M_{P}} \bc M_{N} \;| \;P|M_{P} }
\end{mathpar} 

\begin{mathpar}
  \inferrule* [lab=sychronization] {} {M_{N} \bc \Box \;|\; x?M_{F} \;|\; x!M_{C}}
  \and
  \inferrule* [lab=abstraction] {} {{M_{F}} \bc (x)M_{P} }
  \and
  \inferrule* [lab=concretion] {} {{M_{C}} \bc \langle M_{P} \rangle }
  \and \\
  \inferrule* [lab=process] {} {{M_{P}} \bc M_{N} \;| \;P|M_{P} }
\end{mathpar}

\begin{definition}[contextual application] Given a context $M$, and
  process $P$, we define the \emph{contextual application}, $M[P] :=
  M\{P/\Box\}$. That is, the contextual application of M to P is the
  substitution of $P$ for $\Box$ in $M$.
\end{definition}

$\meaningof{-} : L \to \mathcal{P}(\pi)$

\begin{mathpar}
  \inferrule* [lab=collection] {} {\meaningof{true} = \pi, \and \meaningof{~E} = \pi \setminus \meaningof{E}, \and \meaningof{E_{1} \& E_{2}} = \meaningof{E_{1}} \cap \meaningof{E_{2}}}
\end{mathpar}

\begin{mathpar}
  \inferrule* [lab=structure] {} {\meaningof{0} = \{ P \in \pi | P \equiv 0 \}, \and \\ \meaningof{E_1 | E_2} = \{ P \in \pi | P \equiv P_{1} | P_{2}, P_{1} \in \meaningof{E_{1}}, P_{2} \in \meaningof{E_2}\} }
\end{mathpar}

\begin{mathpar}
 \inferrule* [lab=behavior] {} {\meaningof{\langle a?b \rangle E} = \{ P \in \pi | P \equiv Q | u?(y)P', \\ \and \\\\ \and \\ \;\;\; u \in \meaningof{a}, \forall z.P'\{z/y\} \in \meaningof{E\{z/b\}}\}, \and \\ \meaningof{a!E} = \{ P \in \pi | P \equiv Q | x!\langle P' \rangle, x \in \meaningof{a} P' \in \meaningof{E}\} }
\end{mathpar}

\begin{mathpar}
 \inferrule* [lab=nominal] {} {\meaningof{\quotep{E}} = \{ \quotep{P} \in \quotep{\pi} | P \in \meaningof{E} \}, \and \meaningof{\quotep{P}} = \{ \quotep{Q} \in \quotep{\pi} | P \equiv Q \} \and \\ \meaningof{@\quotep{E}} = \{ P \in \pi | P \equiv @x, x \in \meaningof{E} \}}
\end{mathpar}

\begin{eqnarray*}
  \\
  \meaningof{-} : TS \to ST
\end{eqnarray*}

\begin{eqnarray*}
  \\
  L : TS \to ST
\end{eqnarray*}

\begin{eqnarray*}
  \\
  P \models E \iff P \in \meaningof{E}
\end{eqnarray*}

\begin{eqnarray*}
  P \approx_{L} Q \iff \forall E \in L. P \models E \iff Q \models E
\end{eqnarray*}

\begin{eqnarray*}
  P \approx_{K} Q
\end{eqnarray*}

\begin{eqnarray*}
  P \approx Q
\end{eqnarray*}

$\approx_{K} = \approx = \approx_{L}$

\subsubsection{Contextual duality}

Note that contexts extend the quotation operation to a family of
operations from processes to names. Given a context, $M$, we can
define a \emph{nominal context}, $\quotep{M}$ by $\quotep{M}[P] :=
\quotep{M[P]}$. To foreshadow what is to come we observe that these
operations enjoy a duality with processes very much like the duality
between vectors and maps from vectors to scalars.

Further, because the calculus is essentially higher-order, we have a
correspondence between contexts and processes. More specifically,
given a name $x$ and a context $M$ we can construct $M^{*}_{x}$ such
that 

\begin{mathpar}
  M^{*}_{x} | \lift{x}{P} \red M[P]
\end{mathpar}

namely,

\begin{mathpar}
  M^{*}_{x} := x?(u).M[\dropn{u}]
\end{mathpar}

The dependence of $M^{*}_{x}$ on a name makes it an abstraction, 

\begin{mathpar}
  M^{*} := (x)x?(u).M[\dropn{u}]
\end{mathpar}

\subsection{Additional notation}

It will sometimes be convenient to denote the process a name
quotes. We already have the notation $x = \quotep{P}$, but it will be
convenient to introduce an alternate notation, $\procn{x}$, when we
want to emphasize the connection to the use of the name. Note that, by
virtue of name equivalence, $\quotep{\procn{x}} \nameeq x$; so, the
notation is consistent with previous definitions.

Further, because names have structure it is possible to effect
substitutions on the basis of that structure. This means we need to
upgrade our notation for substitutions, which we accomplish by
adapting comprehension notation. Thus,

\begin{mathpar}
  P\{ y / x : x \in S \}
\end{mathpar}

is interpreted to mean the process derived from P by replacing (in a
capture-avoiding manner) each occurrence of $x$ in $S$ by $y$. For example,

\begin{mathpar}
  P\{ \quotep{\procn{x}|\procn{x}} / x : x \in \freenames{P} \}
\end{mathpar}

will replace each (occurrence) of a free name $x$ in $P$ by
$\quotep{\procn{x}|\procn{x}}$.

Also, we will avail ourselves of the notation $x^{L}$ and $x^{R}$ to
denote injections of a name into disjoint copies of the name
space. There are numerous ways to accomplish this. One example can be
found in \cite{MeredithR05}. This notation overloads to vectors of
names: $\vec{x}^{\pi} := (x_{i}^{\pi} \; : \; 0 \leq i < |\vec{x}| )$ where $\pi \in \{L,R\}$.

We also use $P^{\Box} := P|\Box$.

In \cite{MeredithR05} an interpretation of the new operator is
given. It turns out that there are several possible interpretations
all enjoying the requisite algebraic properties of the operator (see
\cite{milner91polyadicpi}). We will therefore make liberal use of
$(\nu\; \vec{x})P$.

% subsection the_syntax_and_semantics_of_the_notation_system (end)   

\input{qm2pi.qmops} 

\input{qm2pi.sterngerlach} 

\input{qm2pi.metric} 

% section concurrent_process_calculi (end)

%\input{qm2pi.proofsketch}

% section proof sketch (end)

%\input{qm2pi.slviaknots} 

% section spatial logic via knots (end)

\input{qm2pi.conclusion}

% section conclusion (end)

%\input{qm2pi.dtcodes} 

% section wiring algorithm (end)

\input{qm2pi.ack} 

% section acknowledgments (end)

\newpage


\bibliographystyle{plain}   
\bibliography{../../biblios/main.bib}

\input{qm2pi.rhodetails}

\end{document}

 

% section acknowledgments (end)

\newpage


\bibliographystyle{plain}   
\bibliography{../../biblios/main.bib}

\documentclass[12pt]{llncs}
%\documentclass{jktr}

\usepackage[pdftex]{hyperref}                   
\usepackage {listings}
\usepackage {mathpartir}
\usepackage{bcprules}
%\usepackage{listings}
                       
\usepackage{graphicx} 
%\usepackage[margins=2.5cm,nohead,nofoot]{geometry}
%\usepackage{geometry}
\usepackage{amsfonts}
\usepackage{amstext}
\usepackage{latexsym}
\usepackage{amssymb}
\usepackage{color}


%\include{myPreamble}
\include{qm2pi.local} 

%\ifpdf
%\usepackage[pdftex]{graphicx}
%\else
%\usepackage{graphicx}
%\fi

 % \ifpdf
%  \usepackage{pdfsync}
%  \if


%\title{Brief Article}
%\author{David F. Snyder}
%\author{L.G. Meredith}

%\address{Dept. of Math., Texas State University--San Marcos, San Marcos, TX 78666}
       
\pagestyle{empty}


\begin{document}

\lstset{language=[Objective]Caml,frame=shadowbox}

\input{qm2pi.front}

% section front matter (end)

\input{qm2pi.intro} 
 
% section introduction (end)

% \input{qm2pi.knotations} 

% section notation (end)

\input{qm2pi.process.calculi} 

% section concurrent_process_calculi_and_spatial_logics_ (end)
    
%\input{qm2pi.knots2pi} 

%\input{qm2pi.trefoil} 

%\input{qm2pi.mainthm} 

% subsection basic_interpretation (end)

%\input{qm2pi.rho.presentation} 
\subsection{The syntax and semantics of the notation system}\label{sub:the_syntax_and_semantics_of_the_notation_system} % (fold)

We now summarize a technical presentation of the calculus that
embodies our theory of dynamics. The typical presentation of such a
calculus follows the style of giving generators and relations on
them. The grammar, below, describing term constructors, freely
generates the set of processes, $\Proc$. This set is then quotiented
by a relation known as structural congruence and it is over this set
that the notion of dynamics is expressed. This presentation is
essentially that of \cite{MeredithR05} with the addition of
polyadicity and summation. For readability we have relegated some of
the technical subtleties to an appendix.

\subsubsection{Process grammar}\label{subsub:process_grammar}

\begin{mathpar}
  \inferrule* [lab=synchronization] {} {{M} \bc \pzero \;|\; x?F \;|\; x!C }
  \and
  \inferrule* [lab=abstraction] {} {{F} \bc (x)P}
  \and
  \inferrule* [lab=concretion] {} {{C} \bc \langle Q \rangle}
  \and
  \inferrule* [lab=process] {} {{P,Q} \bc M \;| \;P|Q \;|\; @{x}}
  \and
  \inferrule* [lab=name] {} {{x} \bc \quotep{P}}
\end{mathpar} 

Note that $\vec{x}$ (resp. $\vec{P}$) denotes a vector of names
(resp. processes) of length $|\vec{x}|$ (resp. $|\vec{P}|$). We adopt
the following useful abbreviations.

\begin{mathpar}
   x?(\vec{y}).P := x.(\vec{y})P \and  x\clift{\vec{P}} := x.\clift{\vec{P}}
   \and x!(y) := \lift{x}{\dropn{y}}
   \and \Pi_{i=0}^{n-1}P_i := P_0 | \ldots | P_{n-1}
\end{mathpar}

\subsubsection{Structural congruence}

\paragraph{Free and bound names and alpha-equivalence.} At the
core of structural equivalence is alpha-equivalence which identifies
process that are the same up to a change of variable. Formally, we
recognize the distinction between free and bound names. The free names
of a process, $\freenames{P}$, may be calculated recursively as
follows:

\begin{mathpar}
\freenames{\pzero} := \emptyset
  \and \\
  \freenames{x?(y).P} := \{ x \} \cup (\freenames{P} \setminus \{ y \})
  \and 
  \freenames{x!\langle P \rangle} := \{ x \} \cup \{ P \} 
  \and \\
  \freenames{P|Q} := \freenames{P} \cup \freenames{Q}
  \and \\
  \freenames{@{x}} := \{ x \}
\end{mathpar}

$\pi$
$\quotep{\pi}$

$\freenames{-} : \pi \to \mathcal{P}(\quotep{\pi})$

\begin{eqnarray*}
  \freenames{\pzero} & := & \emptyset \\
  \freenames{x?(y).P} & := & \{ x \} \cup (\freenames{P} \setminus \{ y \}) \\
  \freenames{x!\langle P \rangle} & := & \{ x \} \cup \{ P \} \\
  \freenames{P|Q} & := & \freenames{P} \cup \freenames{Q} \\
  \freenames{\dropn{x}} & := & \{ x \}
\end{eqnarray*}

The bound names of a process, $\boundnames{P}$, are those names occurring in $P$
that are not free. For example, in $x?(y).0$, the name $x$ is free, while $y$ is bound.

\begin{mathpar}
  \inferrule* [lab=monoidal-laws] {} { P|Q \equiv Q|P \and P|0 \equiv P \and P|(Q|R) \equiv (P|Q)|R }
\end{mathpar}

\begin{mathpar}
  \inferrule* [lab=alpha-equivalence] {} { (x)P \equiv (y)P\{y/x\} \and y \not\in \freenames{P} }
\end{mathpar}

\begin{definition}
Then two processes, $P,Q$, are alpha-equivalent if $P = Q\{\vec{y}/\vec{x}\}$ for
some $\vec{x} \in \boundnames{Q},\vec{y} \in \boundnames{P}$, where $Q\{\vec{y}/\vec{x}\}$
denotes the capture-avoiding substitution of $\vec{y}$ for $\vec{x}$ in $Q$.
\end{definition}

\begin{definition}
  The {\em structural congruence} \cite{SangiorgiWalker} , $\equiv$,
  between processes is the least congruence containing
  alpha-equivalence, satisfying the abelian monoid laws
  (associativity, commutativity and $\pzero$ as identity) for parallel
  composition $|$ and for summation $+$.
\end{definition}

\subsection{Name equivalence}

We take name equivalence, written $\nameeq$, to be the smallest
equivalence relation generated by the following rules.

\begin{mathpar}
\inferrule*[lab=Quote-drop]
{ }
{ \quotep{@{x}} \nameeq x }

\inferrule*[lab=Struct-equiv]
{ P \scong Q }
{ \quotep{P} \nameeq \quotep{Q} }
\end{mathpar}

The astute reader will have noticed that the mutual recursion of names
and processes imposes a mutual recursion on alpha-equivalence and
structural equivalence via name-equivalence. Fortunately, all of this
works out pleasantly and we may calculate in the natural way, free of
concern. The reader interested in the details is referred to the
appendix \ref{appendix:rho_details}.

\subsection{Substitution}

We use $\Proc$ for the set of processes, $\QProc$ for the set of
names, and $\id{\{}\vec{y} / \vec{x} \id{\}}$ to denote partial maps,
$s : \QProc \rightarrow \QProc$. A map, $s$ lifts, uniquely, to a map
on process terms, $\widehat{s} : \Proc \rightarrow \Proc$ by the
following equations.

\begin{mathpar}
  (0) \psubstp{Q}{P} := 0 \\
  (R \juxtap S) \psubstp{Q}{P}
  :=    
  (R)\psubstp{Q}{P} \juxtap (S) \psubstp{Q}{P} \\
  (x?(y).R) \psubstp{Q}{P}    
  :=    
  (x)\substp{Q}{P} (z)\concat( (R \psubstn{z}{y}) \psubstp{Q}{P} ) \\
  (\lift{x}{R}) \psubstp{Q}{P}  
  :=
  \lift{(x)\substp{Q}{P}}{ R \psubstp{Q}{P} } \\
%   (\dropn{x})  \psubstp{Q}{P}       
%   := 
%   \left\{ 
%     \begin{array}{ccc} 
%       \dropn{\quotep{Q}} & & x \nameeq \quotep{P} \\
%       \dropn{x} & & otherwise \\
%     \end{array}
%   \right. 
  (\dropn{x})  \psubstp{Q}{P}       
  := 
  \left\{ 
    \begin{array}{ccc} 
      Q & & x \nameeq \quotep{P} \\
      \dropn{x} & & otherwise \\
    \end{array}
  \right.
\end{mathpar}
 

where

\begin{eqnarray}
  (x)\id{\{} \lpquote Q \rpquote / \lpquote P \rpquote \id{\}}            = 
  \left\{ 
    \begin{array}{ccc}
      \lpquote Q \rpquote & & x \nameeq \lpquote P \rpquote \\
      x & & otherwise \\
    \end{array}
  \right. \nonumber
\end{eqnarray}

and $z$ is chosen distinct from $\quotep{P}$, $\quotep{Q}$, the free
names in $Q$, and all the names in $R$. Our $\alpha$-equivalence will
be built in the standard way from this substitution.

\begin{remark}\label{rem:no_self_referential_names}
  One consequence of these definitions is that $\forall P. \quotep{P}
  \not\in \freenames{P}$.
\end{remark}

\subsection{ Dynamic quote: an example }

Anticipating something of what's to come, consider applying the
substitution, $\widehat{\id{\{}u / z \id{\}}}$, to the following pair
of processes, $\lift{w}{y!(z)}$ and $w[ \lpquote y!(z) \rpquote ]$.

\begin{eqnarray}
	\lift{w}{y!(z)}\widehat{\id{\{}u / z \id{\}}}
		& = &
		\lift{w}{y!(u)} \nonumber\\
	w[ \lpquote y!(z) \rpquote ] \widehat{ \id{\{}u / z \id{\}} }
		& = &
		w[ \lpquote y!(z) \rpquote ] \nonumber
\end{eqnarray}

Because the body of the process between quotes is impervious to
substitution, we get radically different answers. In fact, by
examining the first process in an input context,
e.g. $x?(z).\lift{w}{y!(z)}$, we see that the process under the lift
operator may be shaped by prefixed inputs binding a name inside it. In
this sense, the lift operator will be seen as a way to dynamically
construct processes before reifying them as names.

Finally equipped with these standard features we can present the
dynamics of the calculus.

\subsubsection{Operational semantics} 

Finally, we introduce the computational dynamics. What marks these
algebras as distinct from other more traditionally studied algebraic
structures, e.g. vector spaces or polynomial rings, is the manner in
which dynamics is captured. In traditional structures, dynamics is typically
expressed through morphisms between such structures, as in linear maps
between vector spaces or morphisms between rings. In algebras
associated with the semantics of computation, the dynamics is
expressed as part of the algebraic structure itself, through a
reduction reduction relation typically denoted by $\red$. Below, we
give a recursive presentation of this relation for the calculus used
in the encoding.

$\red \subseteq \pi \times \pi$
$\red : \pi \to \mathcal{P}(\pi)$

\begin{mathpar}
  \inferrule* [lab=Comm] { \textsf{match}( x_{src}, x_{trgt} ) } { x_{trgt}?(y)P \; | \; x_{src}!\langle {Q} \rangle \red P\{\quotep{Q}/y}\} }
  \and \\
  \inferrule* [lab=Par] {{P} \red {P}'} {{{P} | {Q}} \red {{P}' | {Q}}}
  \and
  \inferrule* [lab=Equiv]{{{P} \scong {P}'} \andalso {{P}' \red {Q}'} \andalso {{Q}' \scong {Q}}}{{P} \red {Q}}
\end{mathpar}

\begin{eqnarray*}
  match_{\equiv} (\quotep{P},\quotep{Q}) & := & P \equiv Q \\
  match_{\dagger}(\quotep{P},\quotep{Q}) & := & \forall R. P|Q \red^{*} R => R \red^{*} 0 \\
  match_{K}(\quotep{P},\quotep{Q}) & := & K \mbox{ for some context } K
\end{eqnarray*}

$u?(x)P | u!\langle Q \rangle \red P\{\quotep{Q}/x\}$

%We write $\wred$ for $\red^*$, and $P\red$ if $\exists Q $ such that $ P \red Q$.
We write $P\red$ if $\exists Q $ such that $ P \red Q$ and $P\not\red$, otherwise.

\section{Replication}

As mentioned before, it is known that replication (and hence
recursion) can be implemented in a higher-order process algebra
\cite{SangiorgiWalker}. As our first example of calculation with the
machinery thus far presented we give the construction explicitly in
the {\rhoc}.

\begin{eqnarray}
	D_{x} & := & \prefix{x}{y}{(\binpar{\outputp{x}{y}}{@{y}})} \nonumber\\
	\bangp_{x}{P} & := & \binpar{{x}!\langle{\binpar{D_{x}}{P}}\rangle}{D_{x}} \nonumber
\end{eqnarray}

\begin{eqnarray}
	\bangp_{x}{P} & & \nonumber\\
	=
	& {x}!\langle{(\prefix{x}{y}{(\outputp{x}{y} | @{y})) | P}}\rangle 
	      | \prefix{x}{y}{(\outputp{x}{y} | @{y})} & \nonumber\\
	\red
	& (\outputp{x}{y} | @{y})\substn{\quotep{(\prefix{x}{y}{(@{y} | \outputp{x}{y})) | P}}}{y} & \nonumber\\
	=
	& \outputp{x}{\quotep{(\prefix{x}{y}{(\outputp{x}{y} | @{y})) | P}}}
	  | {(\prefix{x}{y}{(\outputp{x}{y} | @{y})) | P}} & \nonumber\\
	\red
	& \ldots & \nonumber\\
	\red^*
	& P | P | \ldots & \nonumber
\end{eqnarray}

Of course, this encoding, as an implementation, runs away, unfolding
$\bangp{P}$ eagerly. A lazier and more implementable replication
operator, restricted to input-guarded processes, may be obtained as follows.

\begin{eqnarray}
\bangp{\prefix{u}{v}{P}} 
	:= 
	\binpar{\lift{x}{\prefix{u}{v}{(\binpar{D(x)}{P})}}}{D(x)} \nonumber
\end{eqnarray}

\begin{remark}
  Note that the lazier definition still does not deal with summation
  or mixed summation (i.e. sums over input and output). The reader is
  invited to construct definitions of replication that deal with these
  features. 

  Further, the definitions are parameterized in a name, $x$. Can you,
  gentle reader, make a definition that eliminates this parameter and
  guarantees no accidental interaction between the replication
  machinery and the process being replicated -- i.e. no accidental
  sharing of names used by the process to get its work done and the
  name(s) used by the replication to effect copying. This latter
  revision of the definition of replication is crucial to obtaining
  the expected identity $!!P \sim !P$.
\end{remark}

\begin{remark}\label{rem:paradoxical_combinator}
  The reader familiar with the lambda calculus will have noticed the
  similarity between $D$ and the paradoxical combinator.

  [Ed. note: the existence of this seems to suggest we have to be more
  restrictive on the set of processes and names we admit if we are to
  support no-cloning.]
\end{remark}

\subsubsection{Bisimulation}

The computational dynamics gives rise to another kind of equivalence,
the equivalence of computational behavior. As previously mentioned
this is typically captured \emph{via} some form of bisimulation.

% The notion we use in this paper is weak barbed bisimulation
% \cite{milner91polyadicpi}.

The notion we use in this paper is derived from weak barbed
bisimulation \cite{milner91polyadicpi}. 

\begin{definition}
An \emph{observation relation}, $\downarrow_{\mathcal N}$, over a set
of names, $\mathcal N$, is the smallest relation satisfying the rules
below.

\infrule[Out-barb]{y \in {\mathcal N}, \; x \nameeq y}
		  {\outputp{x}{v} \downarrow_{\mathcal N} x}
\infrule[Par-barb]{\mbox{$P\downarrow_{\mathcal N} x$ or $Q\downarrow_{\mathcal N} x$}}
		  {\binpar{P}{Q} \downarrow_{\mathcal N} x}

We write $P \Downarrow_{\mathcal N} x$ if there is $Q$ such that 
$P \wred Q$ and $Q \downarrow_{\mathcal N} x$.
\end{definition}

\begin{definition}
%\label{def.bbisim}
An  ${\mathcal N}$-\emph{barbed bisimulation} over a set of names, ${\mathcal N}$, is a symmetric binary relation 
${\mathcal S}_{\mathcal N}$ between agents such that $P\rel{S}_{\mathcal N}Q$ implies:
\begin{enumerate}
\item If $P \red P'$ then $Q \wred Q'$ and $P'\rel{S}_{\mathcal N} Q'$.
\item If $P\downarrow_{\mathcal N} x$, then $Q\Downarrow_{\mathcal N} x$.
\end{enumerate}
$P$ is ${\mathcal N}$-barbed bisimilar to $Q$, written
$P \wbbisim_{\mathcal N} Q$, if $P \rel{S}_{\mathcal N} Q$ for some ${\mathcal N}$-barbed bisimulation ${\mathcal S}_{\mathcal N}$.
\end{definition}

$\mathcal{R} \subseteq \pi \times \pi$

$P \mathcal{R} Q => \forall P'. P \red P' \Rightarrow \exists Q'. Q \red Q', P' \mathcal{R} Q'$

$P \vdash x \Rightarrow Q \vdash x$

\begin{mathpar}
  \inferrule*[lab=Out-barb]{x \nameeq y}{{y}!\langle{Q}\rangle \vdash x}
  \and
  \inferrule*[lab=Par-barb]{\mbox{$P\vdash x$ or $Q\vdash x$}}{\binpar{P}{Q} \vdash x}
\end{mathpar}

\subsubsection{Contexts}

One of the principle advantages of computational calculi like the
$\pi$-calculus is a well-defined notion of context,
contextual-equivalence and a correlation between
contextual-equivalence and notions of bisimulation. The notion of
context allows the decomposition of a process into (sub-)process and
its syntactic environment, its context. Thus, a context may be
thought of as a process with a ``hole'' (written $\Box$) in it. The
application of a context $M$ to a process $P$, written $M[P]$, is
tantamount to filling the hole in $M$ with $P$. In this paper we do
not need the full weight of this theory, but do make use of the notion
of context in the proof the main theorem. 

\begin{mathpar}
  \inferrule* [lab=summation] {} {{M_{M},M_{N}} \bc \Box \;|\; x.M_{A} \;|\; M_{M}+M_{N}}
  \and
  \inferrule* [lab=agent] {} {{M_{A}} \bc (\vec{x})M_{P} \;| \; \clift{P_0,\ldots,M_{P},\ldots,P_N}}
  \and \\
  \inferrule* [lab=process] {} {{M_{P}} \bc M_{N} \;| \;P|M_{P} }
\end{mathpar} 

\begin{mathpar}
  \inferrule* [lab=sychronization] {} {M_{N} \bc \Box \;|\; x?M_{F} \;|\; x!M_{C}}
  \and
  \inferrule* [lab=abstraction] {} {{M_{F}} \bc (x)M_{P} }
  \and
  \inferrule* [lab=concretion] {} {{M_{C}} \bc \langle M_{P} \rangle }
  \and \\
  \inferrule* [lab=process] {} {{M_{P}} \bc M_{N} \;| \;P|M_{P} }
\end{mathpar}

\begin{definition}[contextual application] Given a context $M$, and
  process $P$, we define the \emph{contextual application}, $M[P] :=
  M\{P/\Box\}$. That is, the contextual application of M to P is the
  substitution of $P$ for $\Box$ in $M$.
\end{definition}

$\meaningof{-} : L \to \mathcal{P}(\pi)$

\begin{mathpar}
  \inferrule* [lab=collection] {} {\meaningof{true} = \pi, \and \meaningof{~E} = \pi \setminus \meaningof{E}, \and \meaningof{E_{1} \& E_{2}} = \meaningof{E_{1}} \cap \meaningof{E_{2}}}
\end{mathpar}

\begin{mathpar}
  \inferrule* [lab=structure] {} {\meaningof{0} = \{ P \in \pi | P \equiv 0 \}, \and \\ \meaningof{E_1 | E_2} = \{ P \in \pi | P \equiv P_{1} | P_{2}, P_{1} \in \meaningof{E_{1}}, P_{2} \in \meaningof{E_2}\} }
\end{mathpar}

\begin{mathpar}
 \inferrule* [lab=behavior] {} {\meaningof{\langle a?b \rangle E} = \{ P \in \pi | P \equiv Q | u?(y)P', \\ \and \\\\ \and \\ \;\;\; u \in \meaningof{a}, \forall z.P'\{z/y\} \in \meaningof{E\{z/b\}}\}, \and \\ \meaningof{a!E} = \{ P \in \pi | P \equiv Q | x!\langle P' \rangle, x \in \meaningof{a} P' \in \meaningof{E}\} }
\end{mathpar}

\begin{mathpar}
 \inferrule* [lab=nominal] {} {\meaningof{\quotep{E}} = \{ \quotep{P} \in \quotep{\pi} | P \in \meaningof{E} \}, \and \meaningof{\quotep{P}} = \{ \quotep{Q} \in \quotep{\pi} | P \equiv Q \} \and \\ \meaningof{@\quotep{E}} = \{ P \in \pi | P \equiv @x, x \in \meaningof{E} \}}
\end{mathpar}

\begin{eqnarray*}
  \\
  \meaningof{-} : TS \to ST
\end{eqnarray*}

\begin{eqnarray*}
  \\
  L : TS \to ST
\end{eqnarray*}

\begin{eqnarray*}
  \\
  P \models E \iff P \in \meaningof{E}
\end{eqnarray*}

\begin{eqnarray*}
  P \approx_{L} Q \iff \forall E \in L. P \models E \iff Q \models E
\end{eqnarray*}

\begin{eqnarray*}
  P \approx_{K} Q
\end{eqnarray*}

\begin{eqnarray*}
  P \approx Q
\end{eqnarray*}

$\approx_{K} = \approx = \approx_{L}$

\subsubsection{Contextual duality}

Note that contexts extend the quotation operation to a family of
operations from processes to names. Given a context, $M$, we can
define a \emph{nominal context}, $\quotep{M}$ by $\quotep{M}[P] :=
\quotep{M[P]}$. To foreshadow what is to come we observe that these
operations enjoy a duality with processes very much like the duality
between vectors and maps from vectors to scalars.

Further, because the calculus is essentially higher-order, we have a
correspondence between contexts and processes. More specifically,
given a name $x$ and a context $M$ we can construct $M^{*}_{x}$ such
that 

\begin{mathpar}
  M^{*}_{x} | \lift{x}{P} \red M[P]
\end{mathpar}

namely,

\begin{mathpar}
  M^{*}_{x} := x?(u).M[\dropn{u}]
\end{mathpar}

The dependence of $M^{*}_{x}$ on a name makes it an abstraction, 

\begin{mathpar}
  M^{*} := (x)x?(u).M[\dropn{u}]
\end{mathpar}

\subsection{Additional notation}

It will sometimes be convenient to denote the process a name
quotes. We already have the notation $x = \quotep{P}$, but it will be
convenient to introduce an alternate notation, $\procn{x}$, when we
want to emphasize the connection to the use of the name. Note that, by
virtue of name equivalence, $\quotep{\procn{x}} \nameeq x$; so, the
notation is consistent with previous definitions.

Further, because names have structure it is possible to effect
substitutions on the basis of that structure. This means we need to
upgrade our notation for substitutions, which we accomplish by
adapting comprehension notation. Thus,

\begin{mathpar}
  P\{ y / x : x \in S \}
\end{mathpar}

is interpreted to mean the process derived from P by replacing (in a
capture-avoiding manner) each occurrence of $x$ in $S$ by $y$. For example,

\begin{mathpar}
  P\{ \quotep{\procn{x}|\procn{x}} / x : x \in \freenames{P} \}
\end{mathpar}

will replace each (occurrence) of a free name $x$ in $P$ by
$\quotep{\procn{x}|\procn{x}}$.

Also, we will avail ourselves of the notation $x^{L}$ and $x^{R}$ to
denote injections of a name into disjoint copies of the name
space. There are numerous ways to accomplish this. One example can be
found in \cite{MeredithR05}. This notation overloads to vectors of
names: $\vec{x}^{\pi} := (x_{i}^{\pi} \; : \; 0 \leq i < |\vec{x}| )$ where $\pi \in \{L,R\}$.

We also use $P^{\Box} := P|\Box$.

In \cite{MeredithR05} an interpretation of the new operator is
given. It turns out that there are several possible interpretations
all enjoying the requisite algebraic properties of the operator (see
\cite{milner91polyadicpi}). We will therefore make liberal use of
$(\nu\; \vec{x})P$.

% subsection the_syntax_and_semantics_of_the_notation_system (end)   

\input{qm2pi.qmops} 

\input{qm2pi.sterngerlach} 

\input{qm2pi.metric} 

% section concurrent_process_calculi (end)

%\input{qm2pi.proofsketch}

% section proof sketch (end)

%\input{qm2pi.slviaknots} 

% section spatial logic via knots (end)

\input{qm2pi.conclusion}

% section conclusion (end)

%\input{qm2pi.dtcodes} 

% section wiring algorithm (end)

\input{qm2pi.ack} 

% section acknowledgments (end)

\newpage


\bibliographystyle{plain}   
\bibliography{../../biblios/main.bib}

\input{qm2pi.rhodetails}

\end{document}



\end{document}

 

%\ifpdf
%\usepackage[pdftex]{graphicx}
%\else
%\usepackage{graphicx}
%\fi

 % \ifpdf
%  \usepackage{pdfsync}
%  \if


%\title{Brief Article}
%\author{David F. Snyder}
%\author{L.G. Meredith}

%\address{Dept. of Math., Texas State University--San Marcos, San Marcos, TX 78666}
       
\pagestyle{empty}


\begin{document}

\lstset{language=[Objective]Caml,frame=shadowbox}

\documentclass[12pt]{llncs}
%\documentclass{jktr}

\usepackage[pdftex]{hyperref}                   
\usepackage {listings}
\usepackage {mathpartir}
\usepackage{bcprules}
%\usepackage{listings}
                       
\usepackage{graphicx} 
%\usepackage[margins=2.5cm,nohead,nofoot]{geometry}
%\usepackage{geometry}
\usepackage{amsfonts}
\usepackage{amstext}
\usepackage{latexsym}
\usepackage{amssymb}
\usepackage{color}


%\include{myPreamble}
\documentclass[12pt]{llncs}
%\documentclass{jktr}

\usepackage[pdftex]{hyperref}                   
\usepackage {listings}
\usepackage {mathpartir}
\usepackage{bcprules}
%\usepackage{listings}
                       
\usepackage{graphicx} 
%\usepackage[margins=2.5cm,nohead,nofoot]{geometry}
%\usepackage{geometry}
\usepackage{amsfonts}
\usepackage{amstext}
\usepackage{latexsym}
\usepackage{amssymb}
\usepackage{color}


%\include{myPreamble}
\include{qm2pi.local} 

%\ifpdf
%\usepackage[pdftex]{graphicx}
%\else
%\usepackage{graphicx}
%\fi

 % \ifpdf
%  \usepackage{pdfsync}
%  \if


%\title{Brief Article}
%\author{David F. Snyder}
%\author{L.G. Meredith}

%\address{Dept. of Math., Texas State University--San Marcos, San Marcos, TX 78666}
       
\pagestyle{empty}


\begin{document}

\lstset{language=[Objective]Caml,frame=shadowbox}

\input{qm2pi.front}

% section front matter (end)

\input{qm2pi.intro} 
 
% section introduction (end)

% \input{qm2pi.knotations} 

% section notation (end)

\input{qm2pi.process.calculi} 

% section concurrent_process_calculi_and_spatial_logics_ (end)
    
%\input{qm2pi.knots2pi} 

%\input{qm2pi.trefoil} 

%\input{qm2pi.mainthm} 

% subsection basic_interpretation (end)

%\input{qm2pi.rho.presentation} 
\subsection{The syntax and semantics of the notation system}\label{sub:the_syntax_and_semantics_of_the_notation_system} % (fold)

We now summarize a technical presentation of the calculus that
embodies our theory of dynamics. The typical presentation of such a
calculus follows the style of giving generators and relations on
them. The grammar, below, describing term constructors, freely
generates the set of processes, $\Proc$. This set is then quotiented
by a relation known as structural congruence and it is over this set
that the notion of dynamics is expressed. This presentation is
essentially that of \cite{MeredithR05} with the addition of
polyadicity and summation. For readability we have relegated some of
the technical subtleties to an appendix.

\subsubsection{Process grammar}\label{subsub:process_grammar}

\begin{mathpar}
  \inferrule* [lab=synchronization] {} {{M} \bc \pzero \;|\; x?F \;|\; x!C }
  \and
  \inferrule* [lab=abstraction] {} {{F} \bc (x)P}
  \and
  \inferrule* [lab=concretion] {} {{C} \bc \langle Q \rangle}
  \and
  \inferrule* [lab=process] {} {{P,Q} \bc M \;| \;P|Q \;|\; @{x}}
  \and
  \inferrule* [lab=name] {} {{x} \bc \quotep{P}}
\end{mathpar} 

Note that $\vec{x}$ (resp. $\vec{P}$) denotes a vector of names
(resp. processes) of length $|\vec{x}|$ (resp. $|\vec{P}|$). We adopt
the following useful abbreviations.

\begin{mathpar}
   x?(\vec{y}).P := x.(\vec{y})P \and  x\clift{\vec{P}} := x.\clift{\vec{P}}
   \and x!(y) := \lift{x}{\dropn{y}}
   \and \Pi_{i=0}^{n-1}P_i := P_0 | \ldots | P_{n-1}
\end{mathpar}

\subsubsection{Structural congruence}

\paragraph{Free and bound names and alpha-equivalence.} At the
core of structural equivalence is alpha-equivalence which identifies
process that are the same up to a change of variable. Formally, we
recognize the distinction between free and bound names. The free names
of a process, $\freenames{P}$, may be calculated recursively as
follows:

\begin{mathpar}
\freenames{\pzero} := \emptyset
  \and \\
  \freenames{x?(y).P} := \{ x \} \cup (\freenames{P} \setminus \{ y \})
  \and 
  \freenames{x!\langle P \rangle} := \{ x \} \cup \{ P \} 
  \and \\
  \freenames{P|Q} := \freenames{P} \cup \freenames{Q}
  \and \\
  \freenames{@{x}} := \{ x \}
\end{mathpar}

$\pi$
$\quotep{\pi}$

$\freenames{-} : \pi \to \mathcal{P}(\quotep{\pi})$

\begin{eqnarray*}
  \freenames{\pzero} & := & \emptyset \\
  \freenames{x?(y).P} & := & \{ x \} \cup (\freenames{P} \setminus \{ y \}) \\
  \freenames{x!\langle P \rangle} & := & \{ x \} \cup \{ P \} \\
  \freenames{P|Q} & := & \freenames{P} \cup \freenames{Q} \\
  \freenames{\dropn{x}} & := & \{ x \}
\end{eqnarray*}

The bound names of a process, $\boundnames{P}$, are those names occurring in $P$
that are not free. For example, in $x?(y).0$, the name $x$ is free, while $y$ is bound.

\begin{mathpar}
  \inferrule* [lab=monoidal-laws] {} { P|Q \equiv Q|P \and P|0 \equiv P \and P|(Q|R) \equiv (P|Q)|R }
\end{mathpar}

\begin{mathpar}
  \inferrule* [lab=alpha-equivalence] {} { (x)P \equiv (y)P\{y/x\} \and y \not\in \freenames{P} }
\end{mathpar}

\begin{definition}
Then two processes, $P,Q$, are alpha-equivalent if $P = Q\{\vec{y}/\vec{x}\}$ for
some $\vec{x} \in \boundnames{Q},\vec{y} \in \boundnames{P}$, where $Q\{\vec{y}/\vec{x}\}$
denotes the capture-avoiding substitution of $\vec{y}$ for $\vec{x}$ in $Q$.
\end{definition}

\begin{definition}
  The {\em structural congruence} \cite{SangiorgiWalker} , $\equiv$,
  between processes is the least congruence containing
  alpha-equivalence, satisfying the abelian monoid laws
  (associativity, commutativity and $\pzero$ as identity) for parallel
  composition $|$ and for summation $+$.
\end{definition}

\subsection{Name equivalence}

We take name equivalence, written $\nameeq$, to be the smallest
equivalence relation generated by the following rules.

\begin{mathpar}
\inferrule*[lab=Quote-drop]
{ }
{ \quotep{@{x}} \nameeq x }

\inferrule*[lab=Struct-equiv]
{ P \scong Q }
{ \quotep{P} \nameeq \quotep{Q} }
\end{mathpar}

The astute reader will have noticed that the mutual recursion of names
and processes imposes a mutual recursion on alpha-equivalence and
structural equivalence via name-equivalence. Fortunately, all of this
works out pleasantly and we may calculate in the natural way, free of
concern. The reader interested in the details is referred to the
appendix \ref{appendix:rho_details}.

\subsection{Substitution}

We use $\Proc$ for the set of processes, $\QProc$ for the set of
names, and $\id{\{}\vec{y} / \vec{x} \id{\}}$ to denote partial maps,
$s : \QProc \rightarrow \QProc$. A map, $s$ lifts, uniquely, to a map
on process terms, $\widehat{s} : \Proc \rightarrow \Proc$ by the
following equations.

\begin{mathpar}
  (0) \psubstp{Q}{P} := 0 \\
  (R \juxtap S) \psubstp{Q}{P}
  :=    
  (R)\psubstp{Q}{P} \juxtap (S) \psubstp{Q}{P} \\
  (x?(y).R) \psubstp{Q}{P}    
  :=    
  (x)\substp{Q}{P} (z)\concat( (R \psubstn{z}{y}) \psubstp{Q}{P} ) \\
  (\lift{x}{R}) \psubstp{Q}{P}  
  :=
  \lift{(x)\substp{Q}{P}}{ R \psubstp{Q}{P} } \\
%   (\dropn{x})  \psubstp{Q}{P}       
%   := 
%   \left\{ 
%     \begin{array}{ccc} 
%       \dropn{\quotep{Q}} & & x \nameeq \quotep{P} \\
%       \dropn{x} & & otherwise \\
%     \end{array}
%   \right. 
  (\dropn{x})  \psubstp{Q}{P}       
  := 
  \left\{ 
    \begin{array}{ccc} 
      Q & & x \nameeq \quotep{P} \\
      \dropn{x} & & otherwise \\
    \end{array}
  \right.
\end{mathpar}
 

where

\begin{eqnarray}
  (x)\id{\{} \lpquote Q \rpquote / \lpquote P \rpquote \id{\}}            = 
  \left\{ 
    \begin{array}{ccc}
      \lpquote Q \rpquote & & x \nameeq \lpquote P \rpquote \\
      x & & otherwise \\
    \end{array}
  \right. \nonumber
\end{eqnarray}

and $z$ is chosen distinct from $\quotep{P}$, $\quotep{Q}$, the free
names in $Q$, and all the names in $R$. Our $\alpha$-equivalence will
be built in the standard way from this substitution.

\begin{remark}\label{rem:no_self_referential_names}
  One consequence of these definitions is that $\forall P. \quotep{P}
  \not\in \freenames{P}$.
\end{remark}

\subsection{ Dynamic quote: an example }

Anticipating something of what's to come, consider applying the
substitution, $\widehat{\id{\{}u / z \id{\}}}$, to the following pair
of processes, $\lift{w}{y!(z)}$ and $w[ \lpquote y!(z) \rpquote ]$.

\begin{eqnarray}
	\lift{w}{y!(z)}\widehat{\id{\{}u / z \id{\}}}
		& = &
		\lift{w}{y!(u)} \nonumber\\
	w[ \lpquote y!(z) \rpquote ] \widehat{ \id{\{}u / z \id{\}} }
		& = &
		w[ \lpquote y!(z) \rpquote ] \nonumber
\end{eqnarray}

Because the body of the process between quotes is impervious to
substitution, we get radically different answers. In fact, by
examining the first process in an input context,
e.g. $x?(z).\lift{w}{y!(z)}$, we see that the process under the lift
operator may be shaped by prefixed inputs binding a name inside it. In
this sense, the lift operator will be seen as a way to dynamically
construct processes before reifying them as names.

Finally equipped with these standard features we can present the
dynamics of the calculus.

\subsubsection{Operational semantics} 

Finally, we introduce the computational dynamics. What marks these
algebras as distinct from other more traditionally studied algebraic
structures, e.g. vector spaces or polynomial rings, is the manner in
which dynamics is captured. In traditional structures, dynamics is typically
expressed through morphisms between such structures, as in linear maps
between vector spaces or morphisms between rings. In algebras
associated with the semantics of computation, the dynamics is
expressed as part of the algebraic structure itself, through a
reduction reduction relation typically denoted by $\red$. Below, we
give a recursive presentation of this relation for the calculus used
in the encoding.

$\red \subseteq \pi \times \pi$
$\red : \pi \to \mathcal{P}(\pi)$

\begin{mathpar}
  \inferrule* [lab=Comm] { \textsf{match}( x_{src}, x_{trgt} ) } { x_{trgt}?(y)P \; | \; x_{src}!\langle {Q} \rangle \red P\{\quotep{Q}/y}\} }
  \and \\
  \inferrule* [lab=Par] {{P} \red {P}'} {{{P} | {Q}} \red {{P}' | {Q}}}
  \and
  \inferrule* [lab=Equiv]{{{P} \scong {P}'} \andalso {{P}' \red {Q}'} \andalso {{Q}' \scong {Q}}}{{P} \red {Q}}
\end{mathpar}

\begin{eqnarray*}
  match_{\equiv} (\quotep{P},\quotep{Q}) & := & P \equiv Q \\
  match_{\dagger}(\quotep{P},\quotep{Q}) & := & \forall R. P|Q \red^{*} R => R \red^{*} 0 \\
  match_{K}(\quotep{P},\quotep{Q}) & := & K \mbox{ for some context } K
\end{eqnarray*}

$u?(x)P | u!\langle Q \rangle \red P\{\quotep{Q}/x\}$

%We write $\wred$ for $\red^*$, and $P\red$ if $\exists Q $ such that $ P \red Q$.
We write $P\red$ if $\exists Q $ such that $ P \red Q$ and $P\not\red$, otherwise.

\section{Replication}

As mentioned before, it is known that replication (and hence
recursion) can be implemented in a higher-order process algebra
\cite{SangiorgiWalker}. As our first example of calculation with the
machinery thus far presented we give the construction explicitly in
the {\rhoc}.

\begin{eqnarray}
	D_{x} & := & \prefix{x}{y}{(\binpar{\outputp{x}{y}}{@{y}})} \nonumber\\
	\bangp_{x}{P} & := & \binpar{{x}!\langle{\binpar{D_{x}}{P}}\rangle}{D_{x}} \nonumber
\end{eqnarray}

\begin{eqnarray}
	\bangp_{x}{P} & & \nonumber\\
	=
	& {x}!\langle{(\prefix{x}{y}{(\outputp{x}{y} | @{y})) | P}}\rangle 
	      | \prefix{x}{y}{(\outputp{x}{y} | @{y})} & \nonumber\\
	\red
	& (\outputp{x}{y} | @{y})\substn{\quotep{(\prefix{x}{y}{(@{y} | \outputp{x}{y})) | P}}}{y} & \nonumber\\
	=
	& \outputp{x}{\quotep{(\prefix{x}{y}{(\outputp{x}{y} | @{y})) | P}}}
	  | {(\prefix{x}{y}{(\outputp{x}{y} | @{y})) | P}} & \nonumber\\
	\red
	& \ldots & \nonumber\\
	\red^*
	& P | P | \ldots & \nonumber
\end{eqnarray}

Of course, this encoding, as an implementation, runs away, unfolding
$\bangp{P}$ eagerly. A lazier and more implementable replication
operator, restricted to input-guarded processes, may be obtained as follows.

\begin{eqnarray}
\bangp{\prefix{u}{v}{P}} 
	:= 
	\binpar{\lift{x}{\prefix{u}{v}{(\binpar{D(x)}{P})}}}{D(x)} \nonumber
\end{eqnarray}

\begin{remark}
  Note that the lazier definition still does not deal with summation
  or mixed summation (i.e. sums over input and output). The reader is
  invited to construct definitions of replication that deal with these
  features. 

  Further, the definitions are parameterized in a name, $x$. Can you,
  gentle reader, make a definition that eliminates this parameter and
  guarantees no accidental interaction between the replication
  machinery and the process being replicated -- i.e. no accidental
  sharing of names used by the process to get its work done and the
  name(s) used by the replication to effect copying. This latter
  revision of the definition of replication is crucial to obtaining
  the expected identity $!!P \sim !P$.
\end{remark}

\begin{remark}\label{rem:paradoxical_combinator}
  The reader familiar with the lambda calculus will have noticed the
  similarity between $D$ and the paradoxical combinator.

  [Ed. note: the existence of this seems to suggest we have to be more
  restrictive on the set of processes and names we admit if we are to
  support no-cloning.]
\end{remark}

\subsubsection{Bisimulation}

The computational dynamics gives rise to another kind of equivalence,
the equivalence of computational behavior. As previously mentioned
this is typically captured \emph{via} some form of bisimulation.

% The notion we use in this paper is weak barbed bisimulation
% \cite{milner91polyadicpi}.

The notion we use in this paper is derived from weak barbed
bisimulation \cite{milner91polyadicpi}. 

\begin{definition}
An \emph{observation relation}, $\downarrow_{\mathcal N}$, over a set
of names, $\mathcal N$, is the smallest relation satisfying the rules
below.

\infrule[Out-barb]{y \in {\mathcal N}, \; x \nameeq y}
		  {\outputp{x}{v} \downarrow_{\mathcal N} x}
\infrule[Par-barb]{\mbox{$P\downarrow_{\mathcal N} x$ or $Q\downarrow_{\mathcal N} x$}}
		  {\binpar{P}{Q} \downarrow_{\mathcal N} x}

We write $P \Downarrow_{\mathcal N} x$ if there is $Q$ such that 
$P \wred Q$ and $Q \downarrow_{\mathcal N} x$.
\end{definition}

\begin{definition}
%\label{def.bbisim}
An  ${\mathcal N}$-\emph{barbed bisimulation} over a set of names, ${\mathcal N}$, is a symmetric binary relation 
${\mathcal S}_{\mathcal N}$ between agents such that $P\rel{S}_{\mathcal N}Q$ implies:
\begin{enumerate}
\item If $P \red P'$ then $Q \wred Q'$ and $P'\rel{S}_{\mathcal N} Q'$.
\item If $P\downarrow_{\mathcal N} x$, then $Q\Downarrow_{\mathcal N} x$.
\end{enumerate}
$P$ is ${\mathcal N}$-barbed bisimilar to $Q$, written
$P \wbbisim_{\mathcal N} Q$, if $P \rel{S}_{\mathcal N} Q$ for some ${\mathcal N}$-barbed bisimulation ${\mathcal S}_{\mathcal N}$.
\end{definition}

$\mathcal{R} \subseteq \pi \times \pi$

$P \mathcal{R} Q => \forall P'. P \red P' \Rightarrow \exists Q'. Q \red Q', P' \mathcal{R} Q'$

$P \vdash x \Rightarrow Q \vdash x$

\begin{mathpar}
  \inferrule*[lab=Out-barb]{x \nameeq y}{{y}!\langle{Q}\rangle \vdash x}
  \and
  \inferrule*[lab=Par-barb]{\mbox{$P\vdash x$ or $Q\vdash x$}}{\binpar{P}{Q} \vdash x}
\end{mathpar}

\subsubsection{Contexts}

One of the principle advantages of computational calculi like the
$\pi$-calculus is a well-defined notion of context,
contextual-equivalence and a correlation between
contextual-equivalence and notions of bisimulation. The notion of
context allows the decomposition of a process into (sub-)process and
its syntactic environment, its context. Thus, a context may be
thought of as a process with a ``hole'' (written $\Box$) in it. The
application of a context $M$ to a process $P$, written $M[P]$, is
tantamount to filling the hole in $M$ with $P$. In this paper we do
not need the full weight of this theory, but do make use of the notion
of context in the proof the main theorem. 

\begin{mathpar}
  \inferrule* [lab=summation] {} {{M_{M},M_{N}} \bc \Box \;|\; x.M_{A} \;|\; M_{M}+M_{N}}
  \and
  \inferrule* [lab=agent] {} {{M_{A}} \bc (\vec{x})M_{P} \;| \; \clift{P_0,\ldots,M_{P},\ldots,P_N}}
  \and \\
  \inferrule* [lab=process] {} {{M_{P}} \bc M_{N} \;| \;P|M_{P} }
\end{mathpar} 

\begin{mathpar}
  \inferrule* [lab=sychronization] {} {M_{N} \bc \Box \;|\; x?M_{F} \;|\; x!M_{C}}
  \and
  \inferrule* [lab=abstraction] {} {{M_{F}} \bc (x)M_{P} }
  \and
  \inferrule* [lab=concretion] {} {{M_{C}} \bc \langle M_{P} \rangle }
  \and \\
  \inferrule* [lab=process] {} {{M_{P}} \bc M_{N} \;| \;P|M_{P} }
\end{mathpar}

\begin{definition}[contextual application] Given a context $M$, and
  process $P$, we define the \emph{contextual application}, $M[P] :=
  M\{P/\Box\}$. That is, the contextual application of M to P is the
  substitution of $P$ for $\Box$ in $M$.
\end{definition}

$\meaningof{-} : L \to \mathcal{P}(\pi)$

\begin{mathpar}
  \inferrule* [lab=collection] {} {\meaningof{true} = \pi, \and \meaningof{~E} = \pi \setminus \meaningof{E}, \and \meaningof{E_{1} \& E_{2}} = \meaningof{E_{1}} \cap \meaningof{E_{2}}}
\end{mathpar}

\begin{mathpar}
  \inferrule* [lab=structure] {} {\meaningof{0} = \{ P \in \pi | P \equiv 0 \}, \and \\ \meaningof{E_1 | E_2} = \{ P \in \pi | P \equiv P_{1} | P_{2}, P_{1} \in \meaningof{E_{1}}, P_{2} \in \meaningof{E_2}\} }
\end{mathpar}

\begin{mathpar}
 \inferrule* [lab=behavior] {} {\meaningof{\langle a?b \rangle E} = \{ P \in \pi | P \equiv Q | u?(y)P', \\ \and \\\\ \and \\ \;\;\; u \in \meaningof{a}, \forall z.P'\{z/y\} \in \meaningof{E\{z/b\}}\}, \and \\ \meaningof{a!E} = \{ P \in \pi | P \equiv Q | x!\langle P' \rangle, x \in \meaningof{a} P' \in \meaningof{E}\} }
\end{mathpar}

\begin{mathpar}
 \inferrule* [lab=nominal] {} {\meaningof{\quotep{E}} = \{ \quotep{P} \in \quotep{\pi} | P \in \meaningof{E} \}, \and \meaningof{\quotep{P}} = \{ \quotep{Q} \in \quotep{\pi} | P \equiv Q \} \and \\ \meaningof{@\quotep{E}} = \{ P \in \pi | P \equiv @x, x \in \meaningof{E} \}}
\end{mathpar}

\begin{eqnarray*}
  \\
  \meaningof{-} : TS \to ST
\end{eqnarray*}

\begin{eqnarray*}
  \\
  L : TS \to ST
\end{eqnarray*}

\begin{eqnarray*}
  \\
  P \models E \iff P \in \meaningof{E}
\end{eqnarray*}

\begin{eqnarray*}
  P \approx_{L} Q \iff \forall E \in L. P \models E \iff Q \models E
\end{eqnarray*}

\begin{eqnarray*}
  P \approx_{K} Q
\end{eqnarray*}

\begin{eqnarray*}
  P \approx Q
\end{eqnarray*}

$\approx_{K} = \approx = \approx_{L}$

\subsubsection{Contextual duality}

Note that contexts extend the quotation operation to a family of
operations from processes to names. Given a context, $M$, we can
define a \emph{nominal context}, $\quotep{M}$ by $\quotep{M}[P] :=
\quotep{M[P]}$. To foreshadow what is to come we observe that these
operations enjoy a duality with processes very much like the duality
between vectors and maps from vectors to scalars.

Further, because the calculus is essentially higher-order, we have a
correspondence between contexts and processes. More specifically,
given a name $x$ and a context $M$ we can construct $M^{*}_{x}$ such
that 

\begin{mathpar}
  M^{*}_{x} | \lift{x}{P} \red M[P]
\end{mathpar}

namely,

\begin{mathpar}
  M^{*}_{x} := x?(u).M[\dropn{u}]
\end{mathpar}

The dependence of $M^{*}_{x}$ on a name makes it an abstraction, 

\begin{mathpar}
  M^{*} := (x)x?(u).M[\dropn{u}]
\end{mathpar}

\subsection{Additional notation}

It will sometimes be convenient to denote the process a name
quotes. We already have the notation $x = \quotep{P}$, but it will be
convenient to introduce an alternate notation, $\procn{x}$, when we
want to emphasize the connection to the use of the name. Note that, by
virtue of name equivalence, $\quotep{\procn{x}} \nameeq x$; so, the
notation is consistent with previous definitions.

Further, because names have structure it is possible to effect
substitutions on the basis of that structure. This means we need to
upgrade our notation for substitutions, which we accomplish by
adapting comprehension notation. Thus,

\begin{mathpar}
  P\{ y / x : x \in S \}
\end{mathpar}

is interpreted to mean the process derived from P by replacing (in a
capture-avoiding manner) each occurrence of $x$ in $S$ by $y$. For example,

\begin{mathpar}
  P\{ \quotep{\procn{x}|\procn{x}} / x : x \in \freenames{P} \}
\end{mathpar}

will replace each (occurrence) of a free name $x$ in $P$ by
$\quotep{\procn{x}|\procn{x}}$.

Also, we will avail ourselves of the notation $x^{L}$ and $x^{R}$ to
denote injections of a name into disjoint copies of the name
space. There are numerous ways to accomplish this. One example can be
found in \cite{MeredithR05}. This notation overloads to vectors of
names: $\vec{x}^{\pi} := (x_{i}^{\pi} \; : \; 0 \leq i < |\vec{x}| )$ where $\pi \in \{L,R\}$.

We also use $P^{\Box} := P|\Box$.

In \cite{MeredithR05} an interpretation of the new operator is
given. It turns out that there are several possible interpretations
all enjoying the requisite algebraic properties of the operator (see
\cite{milner91polyadicpi}). We will therefore make liberal use of
$(\nu\; \vec{x})P$.

% subsection the_syntax_and_semantics_of_the_notation_system (end)   

\input{qm2pi.qmops} 

\input{qm2pi.sterngerlach} 

\input{qm2pi.metric} 

% section concurrent_process_calculi (end)

%\input{qm2pi.proofsketch}

% section proof sketch (end)

%\input{qm2pi.slviaknots} 

% section spatial logic via knots (end)

\input{qm2pi.conclusion}

% section conclusion (end)

%\input{qm2pi.dtcodes} 

% section wiring algorithm (end)

\input{qm2pi.ack} 

% section acknowledgments (end)

\newpage


\bibliographystyle{plain}   
\bibliography{../../biblios/main.bib}

\input{qm2pi.rhodetails}

\end{document}

 

%\ifpdf
%\usepackage[pdftex]{graphicx}
%\else
%\usepackage{graphicx}
%\fi

 % \ifpdf
%  \usepackage{pdfsync}
%  \if


%\title{Brief Article}
%\author{David F. Snyder}
%\author{L.G. Meredith}

%\address{Dept. of Math., Texas State University--San Marcos, San Marcos, TX 78666}
       
\pagestyle{empty}


\begin{document}

\lstset{language=[Objective]Caml,frame=shadowbox}

\documentclass[12pt]{llncs}
%\documentclass{jktr}

\usepackage[pdftex]{hyperref}                   
\usepackage {listings}
\usepackage {mathpartir}
\usepackage{bcprules}
%\usepackage{listings}
                       
\usepackage{graphicx} 
%\usepackage[margins=2.5cm,nohead,nofoot]{geometry}
%\usepackage{geometry}
\usepackage{amsfonts}
\usepackage{amstext}
\usepackage{latexsym}
\usepackage{amssymb}
\usepackage{color}


%\include{myPreamble}
\include{qm2pi.local} 

%\ifpdf
%\usepackage[pdftex]{graphicx}
%\else
%\usepackage{graphicx}
%\fi

 % \ifpdf
%  \usepackage{pdfsync}
%  \if


%\title{Brief Article}
%\author{David F. Snyder}
%\author{L.G. Meredith}

%\address{Dept. of Math., Texas State University--San Marcos, San Marcos, TX 78666}
       
\pagestyle{empty}


\begin{document}

\lstset{language=[Objective]Caml,frame=shadowbox}

\input{qm2pi.front}

% section front matter (end)

\input{qm2pi.intro} 
 
% section introduction (end)

% \input{qm2pi.knotations} 

% section notation (end)

\input{qm2pi.process.calculi} 

% section concurrent_process_calculi_and_spatial_logics_ (end)
    
%\input{qm2pi.knots2pi} 

%\input{qm2pi.trefoil} 

%\input{qm2pi.mainthm} 

% subsection basic_interpretation (end)

%\input{qm2pi.rho.presentation} 
\subsection{The syntax and semantics of the notation system}\label{sub:the_syntax_and_semantics_of_the_notation_system} % (fold)

We now summarize a technical presentation of the calculus that
embodies our theory of dynamics. The typical presentation of such a
calculus follows the style of giving generators and relations on
them. The grammar, below, describing term constructors, freely
generates the set of processes, $\Proc$. This set is then quotiented
by a relation known as structural congruence and it is over this set
that the notion of dynamics is expressed. This presentation is
essentially that of \cite{MeredithR05} with the addition of
polyadicity and summation. For readability we have relegated some of
the technical subtleties to an appendix.

\subsubsection{Process grammar}\label{subsub:process_grammar}

\begin{mathpar}
  \inferrule* [lab=synchronization] {} {{M} \bc \pzero \;|\; x?F \;|\; x!C }
  \and
  \inferrule* [lab=abstraction] {} {{F} \bc (x)P}
  \and
  \inferrule* [lab=concretion] {} {{C} \bc \langle Q \rangle}
  \and
  \inferrule* [lab=process] {} {{P,Q} \bc M \;| \;P|Q \;|\; @{x}}
  \and
  \inferrule* [lab=name] {} {{x} \bc \quotep{P}}
\end{mathpar} 

Note that $\vec{x}$ (resp. $\vec{P}$) denotes a vector of names
(resp. processes) of length $|\vec{x}|$ (resp. $|\vec{P}|$). We adopt
the following useful abbreviations.

\begin{mathpar}
   x?(\vec{y}).P := x.(\vec{y})P \and  x\clift{\vec{P}} := x.\clift{\vec{P}}
   \and x!(y) := \lift{x}{\dropn{y}}
   \and \Pi_{i=0}^{n-1}P_i := P_0 | \ldots | P_{n-1}
\end{mathpar}

\subsubsection{Structural congruence}

\paragraph{Free and bound names and alpha-equivalence.} At the
core of structural equivalence is alpha-equivalence which identifies
process that are the same up to a change of variable. Formally, we
recognize the distinction between free and bound names. The free names
of a process, $\freenames{P}$, may be calculated recursively as
follows:

\begin{mathpar}
\freenames{\pzero} := \emptyset
  \and \\
  \freenames{x?(y).P} := \{ x \} \cup (\freenames{P} \setminus \{ y \})
  \and 
  \freenames{x!\langle P \rangle} := \{ x \} \cup \{ P \} 
  \and \\
  \freenames{P|Q} := \freenames{P} \cup \freenames{Q}
  \and \\
  \freenames{@{x}} := \{ x \}
\end{mathpar}

$\pi$
$\quotep{\pi}$

$\freenames{-} : \pi \to \mathcal{P}(\quotep{\pi})$

\begin{eqnarray*}
  \freenames{\pzero} & := & \emptyset \\
  \freenames{x?(y).P} & := & \{ x \} \cup (\freenames{P} \setminus \{ y \}) \\
  \freenames{x!\langle P \rangle} & := & \{ x \} \cup \{ P \} \\
  \freenames{P|Q} & := & \freenames{P} \cup \freenames{Q} \\
  \freenames{\dropn{x}} & := & \{ x \}
\end{eqnarray*}

The bound names of a process, $\boundnames{P}$, are those names occurring in $P$
that are not free. For example, in $x?(y).0$, the name $x$ is free, while $y$ is bound.

\begin{mathpar}
  \inferrule* [lab=monoidal-laws] {} { P|Q \equiv Q|P \and P|0 \equiv P \and P|(Q|R) \equiv (P|Q)|R }
\end{mathpar}

\begin{mathpar}
  \inferrule* [lab=alpha-equivalence] {} { (x)P \equiv (y)P\{y/x\} \and y \not\in \freenames{P} }
\end{mathpar}

\begin{definition}
Then two processes, $P,Q$, are alpha-equivalent if $P = Q\{\vec{y}/\vec{x}\}$ for
some $\vec{x} \in \boundnames{Q},\vec{y} \in \boundnames{P}$, where $Q\{\vec{y}/\vec{x}\}$
denotes the capture-avoiding substitution of $\vec{y}$ for $\vec{x}$ in $Q$.
\end{definition}

\begin{definition}
  The {\em structural congruence} \cite{SangiorgiWalker} , $\equiv$,
  between processes is the least congruence containing
  alpha-equivalence, satisfying the abelian monoid laws
  (associativity, commutativity and $\pzero$ as identity) for parallel
  composition $|$ and for summation $+$.
\end{definition}

\subsection{Name equivalence}

We take name equivalence, written $\nameeq$, to be the smallest
equivalence relation generated by the following rules.

\begin{mathpar}
\inferrule*[lab=Quote-drop]
{ }
{ \quotep{@{x}} \nameeq x }

\inferrule*[lab=Struct-equiv]
{ P \scong Q }
{ \quotep{P} \nameeq \quotep{Q} }
\end{mathpar}

The astute reader will have noticed that the mutual recursion of names
and processes imposes a mutual recursion on alpha-equivalence and
structural equivalence via name-equivalence. Fortunately, all of this
works out pleasantly and we may calculate in the natural way, free of
concern. The reader interested in the details is referred to the
appendix \ref{appendix:rho_details}.

\subsection{Substitution}

We use $\Proc$ for the set of processes, $\QProc$ for the set of
names, and $\id{\{}\vec{y} / \vec{x} \id{\}}$ to denote partial maps,
$s : \QProc \rightarrow \QProc$. A map, $s$ lifts, uniquely, to a map
on process terms, $\widehat{s} : \Proc \rightarrow \Proc$ by the
following equations.

\begin{mathpar}
  (0) \psubstp{Q}{P} := 0 \\
  (R \juxtap S) \psubstp{Q}{P}
  :=    
  (R)\psubstp{Q}{P} \juxtap (S) \psubstp{Q}{P} \\
  (x?(y).R) \psubstp{Q}{P}    
  :=    
  (x)\substp{Q}{P} (z)\concat( (R \psubstn{z}{y}) \psubstp{Q}{P} ) \\
  (\lift{x}{R}) \psubstp{Q}{P}  
  :=
  \lift{(x)\substp{Q}{P}}{ R \psubstp{Q}{P} } \\
%   (\dropn{x})  \psubstp{Q}{P}       
%   := 
%   \left\{ 
%     \begin{array}{ccc} 
%       \dropn{\quotep{Q}} & & x \nameeq \quotep{P} \\
%       \dropn{x} & & otherwise \\
%     \end{array}
%   \right. 
  (\dropn{x})  \psubstp{Q}{P}       
  := 
  \left\{ 
    \begin{array}{ccc} 
      Q & & x \nameeq \quotep{P} \\
      \dropn{x} & & otherwise \\
    \end{array}
  \right.
\end{mathpar}
 

where

\begin{eqnarray}
  (x)\id{\{} \lpquote Q \rpquote / \lpquote P \rpquote \id{\}}            = 
  \left\{ 
    \begin{array}{ccc}
      \lpquote Q \rpquote & & x \nameeq \lpquote P \rpquote \\
      x & & otherwise \\
    \end{array}
  \right. \nonumber
\end{eqnarray}

and $z$ is chosen distinct from $\quotep{P}$, $\quotep{Q}$, the free
names in $Q$, and all the names in $R$. Our $\alpha$-equivalence will
be built in the standard way from this substitution.

\begin{remark}\label{rem:no_self_referential_names}
  One consequence of these definitions is that $\forall P. \quotep{P}
  \not\in \freenames{P}$.
\end{remark}

\subsection{ Dynamic quote: an example }

Anticipating something of what's to come, consider applying the
substitution, $\widehat{\id{\{}u / z \id{\}}}$, to the following pair
of processes, $\lift{w}{y!(z)}$ and $w[ \lpquote y!(z) \rpquote ]$.

\begin{eqnarray}
	\lift{w}{y!(z)}\widehat{\id{\{}u / z \id{\}}}
		& = &
		\lift{w}{y!(u)} \nonumber\\
	w[ \lpquote y!(z) \rpquote ] \widehat{ \id{\{}u / z \id{\}} }
		& = &
		w[ \lpquote y!(z) \rpquote ] \nonumber
\end{eqnarray}

Because the body of the process between quotes is impervious to
substitution, we get radically different answers. In fact, by
examining the first process in an input context,
e.g. $x?(z).\lift{w}{y!(z)}$, we see that the process under the lift
operator may be shaped by prefixed inputs binding a name inside it. In
this sense, the lift operator will be seen as a way to dynamically
construct processes before reifying them as names.

Finally equipped with these standard features we can present the
dynamics of the calculus.

\subsubsection{Operational semantics} 

Finally, we introduce the computational dynamics. What marks these
algebras as distinct from other more traditionally studied algebraic
structures, e.g. vector spaces or polynomial rings, is the manner in
which dynamics is captured. In traditional structures, dynamics is typically
expressed through morphisms between such structures, as in linear maps
between vector spaces or morphisms between rings. In algebras
associated with the semantics of computation, the dynamics is
expressed as part of the algebraic structure itself, through a
reduction reduction relation typically denoted by $\red$. Below, we
give a recursive presentation of this relation for the calculus used
in the encoding.

$\red \subseteq \pi \times \pi$
$\red : \pi \to \mathcal{P}(\pi)$

\begin{mathpar}
  \inferrule* [lab=Comm] { \textsf{match}( x_{src}, x_{trgt} ) } { x_{trgt}?(y)P \; | \; x_{src}!\langle {Q} \rangle \red P\{\quotep{Q}/y}\} }
  \and \\
  \inferrule* [lab=Par] {{P} \red {P}'} {{{P} | {Q}} \red {{P}' | {Q}}}
  \and
  \inferrule* [lab=Equiv]{{{P} \scong {P}'} \andalso {{P}' \red {Q}'} \andalso {{Q}' \scong {Q}}}{{P} \red {Q}}
\end{mathpar}

\begin{eqnarray*}
  match_{\equiv} (\quotep{P},\quotep{Q}) & := & P \equiv Q \\
  match_{\dagger}(\quotep{P},\quotep{Q}) & := & \forall R. P|Q \red^{*} R => R \red^{*} 0 \\
  match_{K}(\quotep{P},\quotep{Q}) & := & K \mbox{ for some context } K
\end{eqnarray*}

$u?(x)P | u!\langle Q \rangle \red P\{\quotep{Q}/x\}$

%We write $\wred$ for $\red^*$, and $P\red$ if $\exists Q $ such that $ P \red Q$.
We write $P\red$ if $\exists Q $ such that $ P \red Q$ and $P\not\red$, otherwise.

\section{Replication}

As mentioned before, it is known that replication (and hence
recursion) can be implemented in a higher-order process algebra
\cite{SangiorgiWalker}. As our first example of calculation with the
machinery thus far presented we give the construction explicitly in
the {\rhoc}.

\begin{eqnarray}
	D_{x} & := & \prefix{x}{y}{(\binpar{\outputp{x}{y}}{@{y}})} \nonumber\\
	\bangp_{x}{P} & := & \binpar{{x}!\langle{\binpar{D_{x}}{P}}\rangle}{D_{x}} \nonumber
\end{eqnarray}

\begin{eqnarray}
	\bangp_{x}{P} & & \nonumber\\
	=
	& {x}!\langle{(\prefix{x}{y}{(\outputp{x}{y} | @{y})) | P}}\rangle 
	      | \prefix{x}{y}{(\outputp{x}{y} | @{y})} & \nonumber\\
	\red
	& (\outputp{x}{y} | @{y})\substn{\quotep{(\prefix{x}{y}{(@{y} | \outputp{x}{y})) | P}}}{y} & \nonumber\\
	=
	& \outputp{x}{\quotep{(\prefix{x}{y}{(\outputp{x}{y} | @{y})) | P}}}
	  | {(\prefix{x}{y}{(\outputp{x}{y} | @{y})) | P}} & \nonumber\\
	\red
	& \ldots & \nonumber\\
	\red^*
	& P | P | \ldots & \nonumber
\end{eqnarray}

Of course, this encoding, as an implementation, runs away, unfolding
$\bangp{P}$ eagerly. A lazier and more implementable replication
operator, restricted to input-guarded processes, may be obtained as follows.

\begin{eqnarray}
\bangp{\prefix{u}{v}{P}} 
	:= 
	\binpar{\lift{x}{\prefix{u}{v}{(\binpar{D(x)}{P})}}}{D(x)} \nonumber
\end{eqnarray}

\begin{remark}
  Note that the lazier definition still does not deal with summation
  or mixed summation (i.e. sums over input and output). The reader is
  invited to construct definitions of replication that deal with these
  features. 

  Further, the definitions are parameterized in a name, $x$. Can you,
  gentle reader, make a definition that eliminates this parameter and
  guarantees no accidental interaction between the replication
  machinery and the process being replicated -- i.e. no accidental
  sharing of names used by the process to get its work done and the
  name(s) used by the replication to effect copying. This latter
  revision of the definition of replication is crucial to obtaining
  the expected identity $!!P \sim !P$.
\end{remark}

\begin{remark}\label{rem:paradoxical_combinator}
  The reader familiar with the lambda calculus will have noticed the
  similarity between $D$ and the paradoxical combinator.

  [Ed. note: the existence of this seems to suggest we have to be more
  restrictive on the set of processes and names we admit if we are to
  support no-cloning.]
\end{remark}

\subsubsection{Bisimulation}

The computational dynamics gives rise to another kind of equivalence,
the equivalence of computational behavior. As previously mentioned
this is typically captured \emph{via} some form of bisimulation.

% The notion we use in this paper is weak barbed bisimulation
% \cite{milner91polyadicpi}.

The notion we use in this paper is derived from weak barbed
bisimulation \cite{milner91polyadicpi}. 

\begin{definition}
An \emph{observation relation}, $\downarrow_{\mathcal N}$, over a set
of names, $\mathcal N$, is the smallest relation satisfying the rules
below.

\infrule[Out-barb]{y \in {\mathcal N}, \; x \nameeq y}
		  {\outputp{x}{v} \downarrow_{\mathcal N} x}
\infrule[Par-barb]{\mbox{$P\downarrow_{\mathcal N} x$ or $Q\downarrow_{\mathcal N} x$}}
		  {\binpar{P}{Q} \downarrow_{\mathcal N} x}

We write $P \Downarrow_{\mathcal N} x$ if there is $Q$ such that 
$P \wred Q$ and $Q \downarrow_{\mathcal N} x$.
\end{definition}

\begin{definition}
%\label{def.bbisim}
An  ${\mathcal N}$-\emph{barbed bisimulation} over a set of names, ${\mathcal N}$, is a symmetric binary relation 
${\mathcal S}_{\mathcal N}$ between agents such that $P\rel{S}_{\mathcal N}Q$ implies:
\begin{enumerate}
\item If $P \red P'$ then $Q \wred Q'$ and $P'\rel{S}_{\mathcal N} Q'$.
\item If $P\downarrow_{\mathcal N} x$, then $Q\Downarrow_{\mathcal N} x$.
\end{enumerate}
$P$ is ${\mathcal N}$-barbed bisimilar to $Q$, written
$P \wbbisim_{\mathcal N} Q$, if $P \rel{S}_{\mathcal N} Q$ for some ${\mathcal N}$-barbed bisimulation ${\mathcal S}_{\mathcal N}$.
\end{definition}

$\mathcal{R} \subseteq \pi \times \pi$

$P \mathcal{R} Q => \forall P'. P \red P' \Rightarrow \exists Q'. Q \red Q', P' \mathcal{R} Q'$

$P \vdash x \Rightarrow Q \vdash x$

\begin{mathpar}
  \inferrule*[lab=Out-barb]{x \nameeq y}{{y}!\langle{Q}\rangle \vdash x}
  \and
  \inferrule*[lab=Par-barb]{\mbox{$P\vdash x$ or $Q\vdash x$}}{\binpar{P}{Q} \vdash x}
\end{mathpar}

\subsubsection{Contexts}

One of the principle advantages of computational calculi like the
$\pi$-calculus is a well-defined notion of context,
contextual-equivalence and a correlation between
contextual-equivalence and notions of bisimulation. The notion of
context allows the decomposition of a process into (sub-)process and
its syntactic environment, its context. Thus, a context may be
thought of as a process with a ``hole'' (written $\Box$) in it. The
application of a context $M$ to a process $P$, written $M[P]$, is
tantamount to filling the hole in $M$ with $P$. In this paper we do
not need the full weight of this theory, but do make use of the notion
of context in the proof the main theorem. 

\begin{mathpar}
  \inferrule* [lab=summation] {} {{M_{M},M_{N}} \bc \Box \;|\; x.M_{A} \;|\; M_{M}+M_{N}}
  \and
  \inferrule* [lab=agent] {} {{M_{A}} \bc (\vec{x})M_{P} \;| \; \clift{P_0,\ldots,M_{P},\ldots,P_N}}
  \and \\
  \inferrule* [lab=process] {} {{M_{P}} \bc M_{N} \;| \;P|M_{P} }
\end{mathpar} 

\begin{mathpar}
  \inferrule* [lab=sychronization] {} {M_{N} \bc \Box \;|\; x?M_{F} \;|\; x!M_{C}}
  \and
  \inferrule* [lab=abstraction] {} {{M_{F}} \bc (x)M_{P} }
  \and
  \inferrule* [lab=concretion] {} {{M_{C}} \bc \langle M_{P} \rangle }
  \and \\
  \inferrule* [lab=process] {} {{M_{P}} \bc M_{N} \;| \;P|M_{P} }
\end{mathpar}

\begin{definition}[contextual application] Given a context $M$, and
  process $P$, we define the \emph{contextual application}, $M[P] :=
  M\{P/\Box\}$. That is, the contextual application of M to P is the
  substitution of $P$ for $\Box$ in $M$.
\end{definition}

$\meaningof{-} : L \to \mathcal{P}(\pi)$

\begin{mathpar}
  \inferrule* [lab=collection] {} {\meaningof{true} = \pi, \and \meaningof{~E} = \pi \setminus \meaningof{E}, \and \meaningof{E_{1} \& E_{2}} = \meaningof{E_{1}} \cap \meaningof{E_{2}}}
\end{mathpar}

\begin{mathpar}
  \inferrule* [lab=structure] {} {\meaningof{0} = \{ P \in \pi | P \equiv 0 \}, \and \\ \meaningof{E_1 | E_2} = \{ P \in \pi | P \equiv P_{1} | P_{2}, P_{1} \in \meaningof{E_{1}}, P_{2} \in \meaningof{E_2}\} }
\end{mathpar}

\begin{mathpar}
 \inferrule* [lab=behavior] {} {\meaningof{\langle a?b \rangle E} = \{ P \in \pi | P \equiv Q | u?(y)P', \\ \and \\\\ \and \\ \;\;\; u \in \meaningof{a}, \forall z.P'\{z/y\} \in \meaningof{E\{z/b\}}\}, \and \\ \meaningof{a!E} = \{ P \in \pi | P \equiv Q | x!\langle P' \rangle, x \in \meaningof{a} P' \in \meaningof{E}\} }
\end{mathpar}

\begin{mathpar}
 \inferrule* [lab=nominal] {} {\meaningof{\quotep{E}} = \{ \quotep{P} \in \quotep{\pi} | P \in \meaningof{E} \}, \and \meaningof{\quotep{P}} = \{ \quotep{Q} \in \quotep{\pi} | P \equiv Q \} \and \\ \meaningof{@\quotep{E}} = \{ P \in \pi | P \equiv @x, x \in \meaningof{E} \}}
\end{mathpar}

\begin{eqnarray*}
  \\
  \meaningof{-} : TS \to ST
\end{eqnarray*}

\begin{eqnarray*}
  \\
  L : TS \to ST
\end{eqnarray*}

\begin{eqnarray*}
  \\
  P \models E \iff P \in \meaningof{E}
\end{eqnarray*}

\begin{eqnarray*}
  P \approx_{L} Q \iff \forall E \in L. P \models E \iff Q \models E
\end{eqnarray*}

\begin{eqnarray*}
  P \approx_{K} Q
\end{eqnarray*}

\begin{eqnarray*}
  P \approx Q
\end{eqnarray*}

$\approx_{K} = \approx = \approx_{L}$

\subsubsection{Contextual duality}

Note that contexts extend the quotation operation to a family of
operations from processes to names. Given a context, $M$, we can
define a \emph{nominal context}, $\quotep{M}$ by $\quotep{M}[P] :=
\quotep{M[P]}$. To foreshadow what is to come we observe that these
operations enjoy a duality with processes very much like the duality
between vectors and maps from vectors to scalars.

Further, because the calculus is essentially higher-order, we have a
correspondence between contexts and processes. More specifically,
given a name $x$ and a context $M$ we can construct $M^{*}_{x}$ such
that 

\begin{mathpar}
  M^{*}_{x} | \lift{x}{P} \red M[P]
\end{mathpar}

namely,

\begin{mathpar}
  M^{*}_{x} := x?(u).M[\dropn{u}]
\end{mathpar}

The dependence of $M^{*}_{x}$ on a name makes it an abstraction, 

\begin{mathpar}
  M^{*} := (x)x?(u).M[\dropn{u}]
\end{mathpar}

\subsection{Additional notation}

It will sometimes be convenient to denote the process a name
quotes. We already have the notation $x = \quotep{P}$, but it will be
convenient to introduce an alternate notation, $\procn{x}$, when we
want to emphasize the connection to the use of the name. Note that, by
virtue of name equivalence, $\quotep{\procn{x}} \nameeq x$; so, the
notation is consistent with previous definitions.

Further, because names have structure it is possible to effect
substitutions on the basis of that structure. This means we need to
upgrade our notation for substitutions, which we accomplish by
adapting comprehension notation. Thus,

\begin{mathpar}
  P\{ y / x : x \in S \}
\end{mathpar}

is interpreted to mean the process derived from P by replacing (in a
capture-avoiding manner) each occurrence of $x$ in $S$ by $y$. For example,

\begin{mathpar}
  P\{ \quotep{\procn{x}|\procn{x}} / x : x \in \freenames{P} \}
\end{mathpar}

will replace each (occurrence) of a free name $x$ in $P$ by
$\quotep{\procn{x}|\procn{x}}$.

Also, we will avail ourselves of the notation $x^{L}$ and $x^{R}$ to
denote injections of a name into disjoint copies of the name
space. There are numerous ways to accomplish this. One example can be
found in \cite{MeredithR05}. This notation overloads to vectors of
names: $\vec{x}^{\pi} := (x_{i}^{\pi} \; : \; 0 \leq i < |\vec{x}| )$ where $\pi \in \{L,R\}$.

We also use $P^{\Box} := P|\Box$.

In \cite{MeredithR05} an interpretation of the new operator is
given. It turns out that there are several possible interpretations
all enjoying the requisite algebraic properties of the operator (see
\cite{milner91polyadicpi}). We will therefore make liberal use of
$(\nu\; \vec{x})P$.

% subsection the_syntax_and_semantics_of_the_notation_system (end)   

\input{qm2pi.qmops} 

\input{qm2pi.sterngerlach} 

\input{qm2pi.metric} 

% section concurrent_process_calculi (end)

%\input{qm2pi.proofsketch}

% section proof sketch (end)

%\input{qm2pi.slviaknots} 

% section spatial logic via knots (end)

\input{qm2pi.conclusion}

% section conclusion (end)

%\input{qm2pi.dtcodes} 

% section wiring algorithm (end)

\input{qm2pi.ack} 

% section acknowledgments (end)

\newpage


\bibliographystyle{plain}   
\bibliography{../../biblios/main.bib}

\input{qm2pi.rhodetails}

\end{document}



% section front matter (end)

\section{Introduction}\label{sec:introduction} % (fold)
In this draft of the material i am going to have to dispense with the
usual writing conventions adopted in papers on these topics. i'm going
to have adopt whatever tone i need at the time i'm writing up the
calculations. Sometimes this may be very conversational; others it may
be the barest mathematical grunts; others still it may be that i have
lifted text from one of my other papers because the exposition of some
point was better said there. i hope that my readers are not unduly put
out by this decision. i'm not doing this to flout convention or be
rebellious. i find these calculations very technically challenging. To
keep everything going technically, something has to give; i have to
let go of some cognitive burden. So, the academic writing style --
with all of its trade-offs in terms of facilitating technical
communication -- is what i'm letting go of. Perhaps subsequent drafts
can be tightened and polished, but for now, i'm going to speak as if
we were sitting together in a coffee shop with a laptop, wifi and a
pad of paper and a pencil.

So, here's what i have to say. We -- you and i, comfortably ensconced
in our coffee shop and well-equipped with our tools -- can realize and
carry out the calculations of quantum mechanics over a very different
formal theory of dynamics, a formal theory of dynamics that
corresponds to a theory of concurrent computation with
\emph{reflection}. It has the advantage that the underlying theory is
already `quantized', but supports analogues all of the continuuous
operations. Strikingly, this underlying theory has recently been
connected with a notion of metric that we can show, by calculating
together, coincides with the metric induced by the inner product.

There are a lot of reasons why you might be interested in seeing
calculations of this form. Here's why i'm interested. For the past
several centuries there has been no competitor to the ``Newtonian''
account of dynamics. As a result the predominant share of accounts of
dynamical systems and situations have had to be formulated in terms of
the Newtonian machinery. i view this as an intellectually dangerous
position to occupy. Everything, despite it's intrinsic shape, turns
into a nail to be hit with this hammer. Recently, however, the theory
of computation has matured to the point where we have candidates for
theories of dynamics that offer very different perspective on
reasoning about dynamical systems and situations. Testing these
candidates against very successful accounts of dynamical situations,
like quantum mechanics, is going to give us some sense of how mature
they are and some measure of the quality of these accounts of
dynamics.

\subsection{Summary of contributions and outline of paper}

So, we're going to develop an interpretation of the operations of
quantum mechanics normally interpreted by Hilbert spaces and
operators. We're going to do this over a theory of computation. Note
that this is very different than the usual quantum computation program
which develops notions of computation over quantum mechanics. Rather,
we are developing a story that aligns with Wheeler's slogan: It from
Bit. To do this we will first provide an account of the theory of
computation at play here. Then we will dive into a calculation-driven
interpretation of the operations of quantum mechanics.

The reason we take this approach is that -- until very recently --
there hasn't been an axiomatic account of quantum mechanics. As a
result there has been no sharp delineation of the mathematical theory
supporting interpretation of the physical theory and the physical
theory, itself. So, ambient features of the maths are free to be
exploited (or supressed) without a real accounting of their physical
relevance. There is no sharp statement ``here's the physical theory''
qua \emph{theory} and ``here's the mathematical interpretation''
enabling a judgment of how faithful the interpretation is -- apart
from experimental observation. When there is an axiomatic account we
can judge how well a given mathematical formalism supports an
interpretation of the axioms, independent of
experimentation. Likewise, we can judge how well we have captured our
physical evidence and experience with our axiomatics, independent of
any specific mathematical implementation, with accidental detail that
may or may not have physical significance. 

In lieu of a fully fleshed out and vetted axiomatic account of quantum
mechanics, interpreting the operational notions in service of modeling
physical systems will have to suffice. In other words, we are not in
the business of providing a model of Hilbert spaces and operators. We
are in the business of providing a model of quantum mechanics because
we are motivated by testing our notions of dynamics against physical
theory; and, the predictive calculations of the physical theory must
serve as the best formulation -- shy of a fully fleshed out axiomatic
account -- of the physical theory itself (as they have for scientific
theories since time immemorial). Put another way, despite a
whole-hearted commitment to an It-from-Bit ontology, we are firmly
aligned with the shut-up-and-calculate camp as the best way to obtain
results either from the physical perspective or as a quality assurance
measure of our fledgling theory of dynamics.

In detail, we present a reflective process calculus. Then we develop
intuitive correspondences between the notions available in this
calculus and the usual physical notions supporting quantum mechanical
calculations. Thus, 

\begin{table}[htp]
  \center{
    \fbox{
      \begin{tabular}{c|c}
        quantum mechanics & process calculus \\
        \hline
        scalar & name \\
        state vector & process \\
        dual & contextual duals \\
        matrix & formal sums of process-context-dual pairs \\
        orthogonality & process annihilation \\
        inner product & execution-formula + quoting
      \end{tabular}
    }
  }
  \caption{QM - process calculi correspondences}
\end{table}

Then we tighten up these intuitions to operational definitions. We
employ the Dirac notation as the best proxy we can find for an
abstract syntax of the quantum mechanical notions. The definitions we
develop put us in contact with equational constraints coming from the
theory that we demonstrate the definitions and calculations satisfy.

This puts us in a position to shut up and calculate for the
Stern-Gerlach experimental set up, showing how these predictive
calculations become calculations on processes in our theory of a
reflective process calculus.

Penultimately, we demonstrate that the notion of metric coming from
the inner product coincides with the notion of metric available from
the theory of bisimulation. This demonstration gives us the right to
think of space as arising from behavior. Finally, we consider where we
might go from the new vantage point we have obtained.

% section introduction (end) 
 
% section introduction (end)

% \documentclass[12pt]{llncs}
%\documentclass{jktr}

\usepackage[pdftex]{hyperref}                   
\usepackage {listings}
\usepackage {mathpartir}
\usepackage{bcprules}
%\usepackage{listings}
                       
\usepackage{graphicx} 
%\usepackage[margins=2.5cm,nohead,nofoot]{geometry}
%\usepackage{geometry}
\usepackage{amsfonts}
\usepackage{amstext}
\usepackage{latexsym}
\usepackage{amssymb}
\usepackage{color}


%\include{myPreamble}
\include{qm2pi.local} 

%\ifpdf
%\usepackage[pdftex]{graphicx}
%\else
%\usepackage{graphicx}
%\fi

 % \ifpdf
%  \usepackage{pdfsync}
%  \if


%\title{Brief Article}
%\author{David F. Snyder}
%\author{L.G. Meredith}

%\address{Dept. of Math., Texas State University--San Marcos, San Marcos, TX 78666}
       
\pagestyle{empty}


\begin{document}

\lstset{language=[Objective]Caml,frame=shadowbox}

\input{qm2pi.front}

% section front matter (end)

\input{qm2pi.intro} 
 
% section introduction (end)

% \input{qm2pi.knotations} 

% section notation (end)

\input{qm2pi.process.calculi} 

% section concurrent_process_calculi_and_spatial_logics_ (end)
    
%\input{qm2pi.knots2pi} 

%\input{qm2pi.trefoil} 

%\input{qm2pi.mainthm} 

% subsection basic_interpretation (end)

%\input{qm2pi.rho.presentation} 
\subsection{The syntax and semantics of the notation system}\label{sub:the_syntax_and_semantics_of_the_notation_system} % (fold)

We now summarize a technical presentation of the calculus that
embodies our theory of dynamics. The typical presentation of such a
calculus follows the style of giving generators and relations on
them. The grammar, below, describing term constructors, freely
generates the set of processes, $\Proc$. This set is then quotiented
by a relation known as structural congruence and it is over this set
that the notion of dynamics is expressed. This presentation is
essentially that of \cite{MeredithR05} with the addition of
polyadicity and summation. For readability we have relegated some of
the technical subtleties to an appendix.

\subsubsection{Process grammar}\label{subsub:process_grammar}

\begin{mathpar}
  \inferrule* [lab=synchronization] {} {{M} \bc \pzero \;|\; x?F \;|\; x!C }
  \and
  \inferrule* [lab=abstraction] {} {{F} \bc (x)P}
  \and
  \inferrule* [lab=concretion] {} {{C} \bc \langle Q \rangle}
  \and
  \inferrule* [lab=process] {} {{P,Q} \bc M \;| \;P|Q \;|\; @{x}}
  \and
  \inferrule* [lab=name] {} {{x} \bc \quotep{P}}
\end{mathpar} 

Note that $\vec{x}$ (resp. $\vec{P}$) denotes a vector of names
(resp. processes) of length $|\vec{x}|$ (resp. $|\vec{P}|$). We adopt
the following useful abbreviations.

\begin{mathpar}
   x?(\vec{y}).P := x.(\vec{y})P \and  x\clift{\vec{P}} := x.\clift{\vec{P}}
   \and x!(y) := \lift{x}{\dropn{y}}
   \and \Pi_{i=0}^{n-1}P_i := P_0 | \ldots | P_{n-1}
\end{mathpar}

\subsubsection{Structural congruence}

\paragraph{Free and bound names and alpha-equivalence.} At the
core of structural equivalence is alpha-equivalence which identifies
process that are the same up to a change of variable. Formally, we
recognize the distinction between free and bound names. The free names
of a process, $\freenames{P}$, may be calculated recursively as
follows:

\begin{mathpar}
\freenames{\pzero} := \emptyset
  \and \\
  \freenames{x?(y).P} := \{ x \} \cup (\freenames{P} \setminus \{ y \})
  \and 
  \freenames{x!\langle P \rangle} := \{ x \} \cup \{ P \} 
  \and \\
  \freenames{P|Q} := \freenames{P} \cup \freenames{Q}
  \and \\
  \freenames{@{x}} := \{ x \}
\end{mathpar}

$\pi$
$\quotep{\pi}$

$\freenames{-} : \pi \to \mathcal{P}(\quotep{\pi})$

\begin{eqnarray*}
  \freenames{\pzero} & := & \emptyset \\
  \freenames{x?(y).P} & := & \{ x \} \cup (\freenames{P} \setminus \{ y \}) \\
  \freenames{x!\langle P \rangle} & := & \{ x \} \cup \{ P \} \\
  \freenames{P|Q} & := & \freenames{P} \cup \freenames{Q} \\
  \freenames{\dropn{x}} & := & \{ x \}
\end{eqnarray*}

The bound names of a process, $\boundnames{P}$, are those names occurring in $P$
that are not free. For example, in $x?(y).0$, the name $x$ is free, while $y$ is bound.

\begin{mathpar}
  \inferrule* [lab=monoidal-laws] {} { P|Q \equiv Q|P \and P|0 \equiv P \and P|(Q|R) \equiv (P|Q)|R }
\end{mathpar}

\begin{mathpar}
  \inferrule* [lab=alpha-equivalence] {} { (x)P \equiv (y)P\{y/x\} \and y \not\in \freenames{P} }
\end{mathpar}

\begin{definition}
Then two processes, $P,Q$, are alpha-equivalent if $P = Q\{\vec{y}/\vec{x}\}$ for
some $\vec{x} \in \boundnames{Q},\vec{y} \in \boundnames{P}$, where $Q\{\vec{y}/\vec{x}\}$
denotes the capture-avoiding substitution of $\vec{y}$ for $\vec{x}$ in $Q$.
\end{definition}

\begin{definition}
  The {\em structural congruence} \cite{SangiorgiWalker} , $\equiv$,
  between processes is the least congruence containing
  alpha-equivalence, satisfying the abelian monoid laws
  (associativity, commutativity and $\pzero$ as identity) for parallel
  composition $|$ and for summation $+$.
\end{definition}

\subsection{Name equivalence}

We take name equivalence, written $\nameeq$, to be the smallest
equivalence relation generated by the following rules.

\begin{mathpar}
\inferrule*[lab=Quote-drop]
{ }
{ \quotep{@{x}} \nameeq x }

\inferrule*[lab=Struct-equiv]
{ P \scong Q }
{ \quotep{P} \nameeq \quotep{Q} }
\end{mathpar}

The astute reader will have noticed that the mutual recursion of names
and processes imposes a mutual recursion on alpha-equivalence and
structural equivalence via name-equivalence. Fortunately, all of this
works out pleasantly and we may calculate in the natural way, free of
concern. The reader interested in the details is referred to the
appendix \ref{appendix:rho_details}.

\subsection{Substitution}

We use $\Proc$ for the set of processes, $\QProc$ for the set of
names, and $\id{\{}\vec{y} / \vec{x} \id{\}}$ to denote partial maps,
$s : \QProc \rightarrow \QProc$. A map, $s$ lifts, uniquely, to a map
on process terms, $\widehat{s} : \Proc \rightarrow \Proc$ by the
following equations.

\begin{mathpar}
  (0) \psubstp{Q}{P} := 0 \\
  (R \juxtap S) \psubstp{Q}{P}
  :=    
  (R)\psubstp{Q}{P} \juxtap (S) \psubstp{Q}{P} \\
  (x?(y).R) \psubstp{Q}{P}    
  :=    
  (x)\substp{Q}{P} (z)\concat( (R \psubstn{z}{y}) \psubstp{Q}{P} ) \\
  (\lift{x}{R}) \psubstp{Q}{P}  
  :=
  \lift{(x)\substp{Q}{P}}{ R \psubstp{Q}{P} } \\
%   (\dropn{x})  \psubstp{Q}{P}       
%   := 
%   \left\{ 
%     \begin{array}{ccc} 
%       \dropn{\quotep{Q}} & & x \nameeq \quotep{P} \\
%       \dropn{x} & & otherwise \\
%     \end{array}
%   \right. 
  (\dropn{x})  \psubstp{Q}{P}       
  := 
  \left\{ 
    \begin{array}{ccc} 
      Q & & x \nameeq \quotep{P} \\
      \dropn{x} & & otherwise \\
    \end{array}
  \right.
\end{mathpar}
 

where

\begin{eqnarray}
  (x)\id{\{} \lpquote Q \rpquote / \lpquote P \rpquote \id{\}}            = 
  \left\{ 
    \begin{array}{ccc}
      \lpquote Q \rpquote & & x \nameeq \lpquote P \rpquote \\
      x & & otherwise \\
    \end{array}
  \right. \nonumber
\end{eqnarray}

and $z$ is chosen distinct from $\quotep{P}$, $\quotep{Q}$, the free
names in $Q$, and all the names in $R$. Our $\alpha$-equivalence will
be built in the standard way from this substitution.

\begin{remark}\label{rem:no_self_referential_names}
  One consequence of these definitions is that $\forall P. \quotep{P}
  \not\in \freenames{P}$.
\end{remark}

\subsection{ Dynamic quote: an example }

Anticipating something of what's to come, consider applying the
substitution, $\widehat{\id{\{}u / z \id{\}}}$, to the following pair
of processes, $\lift{w}{y!(z)}$ and $w[ \lpquote y!(z) \rpquote ]$.

\begin{eqnarray}
	\lift{w}{y!(z)}\widehat{\id{\{}u / z \id{\}}}
		& = &
		\lift{w}{y!(u)} \nonumber\\
	w[ \lpquote y!(z) \rpquote ] \widehat{ \id{\{}u / z \id{\}} }
		& = &
		w[ \lpquote y!(z) \rpquote ] \nonumber
\end{eqnarray}

Because the body of the process between quotes is impervious to
substitution, we get radically different answers. In fact, by
examining the first process in an input context,
e.g. $x?(z).\lift{w}{y!(z)}$, we see that the process under the lift
operator may be shaped by prefixed inputs binding a name inside it. In
this sense, the lift operator will be seen as a way to dynamically
construct processes before reifying them as names.

Finally equipped with these standard features we can present the
dynamics of the calculus.

\subsubsection{Operational semantics} 

Finally, we introduce the computational dynamics. What marks these
algebras as distinct from other more traditionally studied algebraic
structures, e.g. vector spaces or polynomial rings, is the manner in
which dynamics is captured. In traditional structures, dynamics is typically
expressed through morphisms between such structures, as in linear maps
between vector spaces or morphisms between rings. In algebras
associated with the semantics of computation, the dynamics is
expressed as part of the algebraic structure itself, through a
reduction reduction relation typically denoted by $\red$. Below, we
give a recursive presentation of this relation for the calculus used
in the encoding.

$\red \subseteq \pi \times \pi$
$\red : \pi \to \mathcal{P}(\pi)$

\begin{mathpar}
  \inferrule* [lab=Comm] { \textsf{match}( x_{src}, x_{trgt} ) } { x_{trgt}?(y)P \; | \; x_{src}!\langle {Q} \rangle \red P\{\quotep{Q}/y}\} }
  \and \\
  \inferrule* [lab=Par] {{P} \red {P}'} {{{P} | {Q}} \red {{P}' | {Q}}}
  \and
  \inferrule* [lab=Equiv]{{{P} \scong {P}'} \andalso {{P}' \red {Q}'} \andalso {{Q}' \scong {Q}}}{{P} \red {Q}}
\end{mathpar}

\begin{eqnarray*}
  match_{\equiv} (\quotep{P},\quotep{Q}) & := & P \equiv Q \\
  match_{\dagger}(\quotep{P},\quotep{Q}) & := & \forall R. P|Q \red^{*} R => R \red^{*} 0 \\
  match_{K}(\quotep{P},\quotep{Q}) & := & K \mbox{ for some context } K
\end{eqnarray*}

$u?(x)P | u!\langle Q \rangle \red P\{\quotep{Q}/x\}$

%We write $\wred$ for $\red^*$, and $P\red$ if $\exists Q $ such that $ P \red Q$.
We write $P\red$ if $\exists Q $ such that $ P \red Q$ and $P\not\red$, otherwise.

\section{Replication}

As mentioned before, it is known that replication (and hence
recursion) can be implemented in a higher-order process algebra
\cite{SangiorgiWalker}. As our first example of calculation with the
machinery thus far presented we give the construction explicitly in
the {\rhoc}.

\begin{eqnarray}
	D_{x} & := & \prefix{x}{y}{(\binpar{\outputp{x}{y}}{@{y}})} \nonumber\\
	\bangp_{x}{P} & := & \binpar{{x}!\langle{\binpar{D_{x}}{P}}\rangle}{D_{x}} \nonumber
\end{eqnarray}

\begin{eqnarray}
	\bangp_{x}{P} & & \nonumber\\
	=
	& {x}!\langle{(\prefix{x}{y}{(\outputp{x}{y} | @{y})) | P}}\rangle 
	      | \prefix{x}{y}{(\outputp{x}{y} | @{y})} & \nonumber\\
	\red
	& (\outputp{x}{y} | @{y})\substn{\quotep{(\prefix{x}{y}{(@{y} | \outputp{x}{y})) | P}}}{y} & \nonumber\\
	=
	& \outputp{x}{\quotep{(\prefix{x}{y}{(\outputp{x}{y} | @{y})) | P}}}
	  | {(\prefix{x}{y}{(\outputp{x}{y} | @{y})) | P}} & \nonumber\\
	\red
	& \ldots & \nonumber\\
	\red^*
	& P | P | \ldots & \nonumber
\end{eqnarray}

Of course, this encoding, as an implementation, runs away, unfolding
$\bangp{P}$ eagerly. A lazier and more implementable replication
operator, restricted to input-guarded processes, may be obtained as follows.

\begin{eqnarray}
\bangp{\prefix{u}{v}{P}} 
	:= 
	\binpar{\lift{x}{\prefix{u}{v}{(\binpar{D(x)}{P})}}}{D(x)} \nonumber
\end{eqnarray}

\begin{remark}
  Note that the lazier definition still does not deal with summation
  or mixed summation (i.e. sums over input and output). The reader is
  invited to construct definitions of replication that deal with these
  features. 

  Further, the definitions are parameterized in a name, $x$. Can you,
  gentle reader, make a definition that eliminates this parameter and
  guarantees no accidental interaction between the replication
  machinery and the process being replicated -- i.e. no accidental
  sharing of names used by the process to get its work done and the
  name(s) used by the replication to effect copying. This latter
  revision of the definition of replication is crucial to obtaining
  the expected identity $!!P \sim !P$.
\end{remark}

\begin{remark}\label{rem:paradoxical_combinator}
  The reader familiar with the lambda calculus will have noticed the
  similarity between $D$ and the paradoxical combinator.

  [Ed. note: the existence of this seems to suggest we have to be more
  restrictive on the set of processes and names we admit if we are to
  support no-cloning.]
\end{remark}

\subsubsection{Bisimulation}

The computational dynamics gives rise to another kind of equivalence,
the equivalence of computational behavior. As previously mentioned
this is typically captured \emph{via} some form of bisimulation.

% The notion we use in this paper is weak barbed bisimulation
% \cite{milner91polyadicpi}.

The notion we use in this paper is derived from weak barbed
bisimulation \cite{milner91polyadicpi}. 

\begin{definition}
An \emph{observation relation}, $\downarrow_{\mathcal N}$, over a set
of names, $\mathcal N$, is the smallest relation satisfying the rules
below.

\infrule[Out-barb]{y \in {\mathcal N}, \; x \nameeq y}
		  {\outputp{x}{v} \downarrow_{\mathcal N} x}
\infrule[Par-barb]{\mbox{$P\downarrow_{\mathcal N} x$ or $Q\downarrow_{\mathcal N} x$}}
		  {\binpar{P}{Q} \downarrow_{\mathcal N} x}

We write $P \Downarrow_{\mathcal N} x$ if there is $Q$ such that 
$P \wred Q$ and $Q \downarrow_{\mathcal N} x$.
\end{definition}

\begin{definition}
%\label{def.bbisim}
An  ${\mathcal N}$-\emph{barbed bisimulation} over a set of names, ${\mathcal N}$, is a symmetric binary relation 
${\mathcal S}_{\mathcal N}$ between agents such that $P\rel{S}_{\mathcal N}Q$ implies:
\begin{enumerate}
\item If $P \red P'$ then $Q \wred Q'$ and $P'\rel{S}_{\mathcal N} Q'$.
\item If $P\downarrow_{\mathcal N} x$, then $Q\Downarrow_{\mathcal N} x$.
\end{enumerate}
$P$ is ${\mathcal N}$-barbed bisimilar to $Q$, written
$P \wbbisim_{\mathcal N} Q$, if $P \rel{S}_{\mathcal N} Q$ for some ${\mathcal N}$-barbed bisimulation ${\mathcal S}_{\mathcal N}$.
\end{definition}

$\mathcal{R} \subseteq \pi \times \pi$

$P \mathcal{R} Q => \forall P'. P \red P' \Rightarrow \exists Q'. Q \red Q', P' \mathcal{R} Q'$

$P \vdash x \Rightarrow Q \vdash x$

\begin{mathpar}
  \inferrule*[lab=Out-barb]{x \nameeq y}{{y}!\langle{Q}\rangle \vdash x}
  \and
  \inferrule*[lab=Par-barb]{\mbox{$P\vdash x$ or $Q\vdash x$}}{\binpar{P}{Q} \vdash x}
\end{mathpar}

\subsubsection{Contexts}

One of the principle advantages of computational calculi like the
$\pi$-calculus is a well-defined notion of context,
contextual-equivalence and a correlation between
contextual-equivalence and notions of bisimulation. The notion of
context allows the decomposition of a process into (sub-)process and
its syntactic environment, its context. Thus, a context may be
thought of as a process with a ``hole'' (written $\Box$) in it. The
application of a context $M$ to a process $P$, written $M[P]$, is
tantamount to filling the hole in $M$ with $P$. In this paper we do
not need the full weight of this theory, but do make use of the notion
of context in the proof the main theorem. 

\begin{mathpar}
  \inferrule* [lab=summation] {} {{M_{M},M_{N}} \bc \Box \;|\; x.M_{A} \;|\; M_{M}+M_{N}}
  \and
  \inferrule* [lab=agent] {} {{M_{A}} \bc (\vec{x})M_{P} \;| \; \clift{P_0,\ldots,M_{P},\ldots,P_N}}
  \and \\
  \inferrule* [lab=process] {} {{M_{P}} \bc M_{N} \;| \;P|M_{P} }
\end{mathpar} 

\begin{mathpar}
  \inferrule* [lab=sychronization] {} {M_{N} \bc \Box \;|\; x?M_{F} \;|\; x!M_{C}}
  \and
  \inferrule* [lab=abstraction] {} {{M_{F}} \bc (x)M_{P} }
  \and
  \inferrule* [lab=concretion] {} {{M_{C}} \bc \langle M_{P} \rangle }
  \and \\
  \inferrule* [lab=process] {} {{M_{P}} \bc M_{N} \;| \;P|M_{P} }
\end{mathpar}

\begin{definition}[contextual application] Given a context $M$, and
  process $P$, we define the \emph{contextual application}, $M[P] :=
  M\{P/\Box\}$. That is, the contextual application of M to P is the
  substitution of $P$ for $\Box$ in $M$.
\end{definition}

$\meaningof{-} : L \to \mathcal{P}(\pi)$

\begin{mathpar}
  \inferrule* [lab=collection] {} {\meaningof{true} = \pi, \and \meaningof{~E} = \pi \setminus \meaningof{E}, \and \meaningof{E_{1} \& E_{2}} = \meaningof{E_{1}} \cap \meaningof{E_{2}}}
\end{mathpar}

\begin{mathpar}
  \inferrule* [lab=structure] {} {\meaningof{0} = \{ P \in \pi | P \equiv 0 \}, \and \\ \meaningof{E_1 | E_2} = \{ P \in \pi | P \equiv P_{1} | P_{2}, P_{1} \in \meaningof{E_{1}}, P_{2} \in \meaningof{E_2}\} }
\end{mathpar}

\begin{mathpar}
 \inferrule* [lab=behavior] {} {\meaningof{\langle a?b \rangle E} = \{ P \in \pi | P \equiv Q | u?(y)P', \\ \and \\\\ \and \\ \;\;\; u \in \meaningof{a}, \forall z.P'\{z/y\} \in \meaningof{E\{z/b\}}\}, \and \\ \meaningof{a!E} = \{ P \in \pi | P \equiv Q | x!\langle P' \rangle, x \in \meaningof{a} P' \in \meaningof{E}\} }
\end{mathpar}

\begin{mathpar}
 \inferrule* [lab=nominal] {} {\meaningof{\quotep{E}} = \{ \quotep{P} \in \quotep{\pi} | P \in \meaningof{E} \}, \and \meaningof{\quotep{P}} = \{ \quotep{Q} \in \quotep{\pi} | P \equiv Q \} \and \\ \meaningof{@\quotep{E}} = \{ P \in \pi | P \equiv @x, x \in \meaningof{E} \}}
\end{mathpar}

\begin{eqnarray*}
  \\
  \meaningof{-} : TS \to ST
\end{eqnarray*}

\begin{eqnarray*}
  \\
  L : TS \to ST
\end{eqnarray*}

\begin{eqnarray*}
  \\
  P \models E \iff P \in \meaningof{E}
\end{eqnarray*}

\begin{eqnarray*}
  P \approx_{L} Q \iff \forall E \in L. P \models E \iff Q \models E
\end{eqnarray*}

\begin{eqnarray*}
  P \approx_{K} Q
\end{eqnarray*}

\begin{eqnarray*}
  P \approx Q
\end{eqnarray*}

$\approx_{K} = \approx = \approx_{L}$

\subsubsection{Contextual duality}

Note that contexts extend the quotation operation to a family of
operations from processes to names. Given a context, $M$, we can
define a \emph{nominal context}, $\quotep{M}$ by $\quotep{M}[P] :=
\quotep{M[P]}$. To foreshadow what is to come we observe that these
operations enjoy a duality with processes very much like the duality
between vectors and maps from vectors to scalars.

Further, because the calculus is essentially higher-order, we have a
correspondence between contexts and processes. More specifically,
given a name $x$ and a context $M$ we can construct $M^{*}_{x}$ such
that 

\begin{mathpar}
  M^{*}_{x} | \lift{x}{P} \red M[P]
\end{mathpar}

namely,

\begin{mathpar}
  M^{*}_{x} := x?(u).M[\dropn{u}]
\end{mathpar}

The dependence of $M^{*}_{x}$ on a name makes it an abstraction, 

\begin{mathpar}
  M^{*} := (x)x?(u).M[\dropn{u}]
\end{mathpar}

\subsection{Additional notation}

It will sometimes be convenient to denote the process a name
quotes. We already have the notation $x = \quotep{P}$, but it will be
convenient to introduce an alternate notation, $\procn{x}$, when we
want to emphasize the connection to the use of the name. Note that, by
virtue of name equivalence, $\quotep{\procn{x}} \nameeq x$; so, the
notation is consistent with previous definitions.

Further, because names have structure it is possible to effect
substitutions on the basis of that structure. This means we need to
upgrade our notation for substitutions, which we accomplish by
adapting comprehension notation. Thus,

\begin{mathpar}
  P\{ y / x : x \in S \}
\end{mathpar}

is interpreted to mean the process derived from P by replacing (in a
capture-avoiding manner) each occurrence of $x$ in $S$ by $y$. For example,

\begin{mathpar}
  P\{ \quotep{\procn{x}|\procn{x}} / x : x \in \freenames{P} \}
\end{mathpar}

will replace each (occurrence) of a free name $x$ in $P$ by
$\quotep{\procn{x}|\procn{x}}$.

Also, we will avail ourselves of the notation $x^{L}$ and $x^{R}$ to
denote injections of a name into disjoint copies of the name
space. There are numerous ways to accomplish this. One example can be
found in \cite{MeredithR05}. This notation overloads to vectors of
names: $\vec{x}^{\pi} := (x_{i}^{\pi} \; : \; 0 \leq i < |\vec{x}| )$ where $\pi \in \{L,R\}$.

We also use $P^{\Box} := P|\Box$.

In \cite{MeredithR05} an interpretation of the new operator is
given. It turns out that there are several possible interpretations
all enjoying the requisite algebraic properties of the operator (see
\cite{milner91polyadicpi}). We will therefore make liberal use of
$(\nu\; \vec{x})P$.

% subsection the_syntax_and_semantics_of_the_notation_system (end)   

\input{qm2pi.qmops} 

\input{qm2pi.sterngerlach} 

\input{qm2pi.metric} 

% section concurrent_process_calculi (end)

%\input{qm2pi.proofsketch}

% section proof sketch (end)

%\input{qm2pi.slviaknots} 

% section spatial logic via knots (end)

\input{qm2pi.conclusion}

% section conclusion (end)

%\input{qm2pi.dtcodes} 

% section wiring algorithm (end)

\input{qm2pi.ack} 

% section acknowledgments (end)

\newpage


\bibliographystyle{plain}   
\bibliography{../../biblios/main.bib}

\input{qm2pi.rhodetails}

\end{document}

 

% section notation (end)

\input{qm2pi.process.calculi} 

% section concurrent_process_calculi_and_spatial_logics_ (end)
    
%\documentclass[12pt]{llncs}
%\documentclass{jktr}

\usepackage[pdftex]{hyperref}                   
\usepackage {listings}
\usepackage {mathpartir}
\usepackage{bcprules}
%\usepackage{listings}
                       
\usepackage{graphicx} 
%\usepackage[margins=2.5cm,nohead,nofoot]{geometry}
%\usepackage{geometry}
\usepackage{amsfonts}
\usepackage{amstext}
\usepackage{latexsym}
\usepackage{amssymb}
\usepackage{color}


%\include{myPreamble}
\include{qm2pi.local} 

%\ifpdf
%\usepackage[pdftex]{graphicx}
%\else
%\usepackage{graphicx}
%\fi

 % \ifpdf
%  \usepackage{pdfsync}
%  \if


%\title{Brief Article}
%\author{David F. Snyder}
%\author{L.G. Meredith}

%\address{Dept. of Math., Texas State University--San Marcos, San Marcos, TX 78666}
       
\pagestyle{empty}


\begin{document}

\lstset{language=[Objective]Caml,frame=shadowbox}

\input{qm2pi.front}

% section front matter (end)

\input{qm2pi.intro} 
 
% section introduction (end)

% \input{qm2pi.knotations} 

% section notation (end)

\input{qm2pi.process.calculi} 

% section concurrent_process_calculi_and_spatial_logics_ (end)
    
%\input{qm2pi.knots2pi} 

%\input{qm2pi.trefoil} 

%\input{qm2pi.mainthm} 

% subsection basic_interpretation (end)

%\input{qm2pi.rho.presentation} 
\subsection{The syntax and semantics of the notation system}\label{sub:the_syntax_and_semantics_of_the_notation_system} % (fold)

We now summarize a technical presentation of the calculus that
embodies our theory of dynamics. The typical presentation of such a
calculus follows the style of giving generators and relations on
them. The grammar, below, describing term constructors, freely
generates the set of processes, $\Proc$. This set is then quotiented
by a relation known as structural congruence and it is over this set
that the notion of dynamics is expressed. This presentation is
essentially that of \cite{MeredithR05} with the addition of
polyadicity and summation. For readability we have relegated some of
the technical subtleties to an appendix.

\subsubsection{Process grammar}\label{subsub:process_grammar}

\begin{mathpar}
  \inferrule* [lab=synchronization] {} {{M} \bc \pzero \;|\; x?F \;|\; x!C }
  \and
  \inferrule* [lab=abstraction] {} {{F} \bc (x)P}
  \and
  \inferrule* [lab=concretion] {} {{C} \bc \langle Q \rangle}
  \and
  \inferrule* [lab=process] {} {{P,Q} \bc M \;| \;P|Q \;|\; @{x}}
  \and
  \inferrule* [lab=name] {} {{x} \bc \quotep{P}}
\end{mathpar} 

Note that $\vec{x}$ (resp. $\vec{P}$) denotes a vector of names
(resp. processes) of length $|\vec{x}|$ (resp. $|\vec{P}|$). We adopt
the following useful abbreviations.

\begin{mathpar}
   x?(\vec{y}).P := x.(\vec{y})P \and  x\clift{\vec{P}} := x.\clift{\vec{P}}
   \and x!(y) := \lift{x}{\dropn{y}}
   \and \Pi_{i=0}^{n-1}P_i := P_0 | \ldots | P_{n-1}
\end{mathpar}

\subsubsection{Structural congruence}

\paragraph{Free and bound names and alpha-equivalence.} At the
core of structural equivalence is alpha-equivalence which identifies
process that are the same up to a change of variable. Formally, we
recognize the distinction between free and bound names. The free names
of a process, $\freenames{P}$, may be calculated recursively as
follows:

\begin{mathpar}
\freenames{\pzero} := \emptyset
  \and \\
  \freenames{x?(y).P} := \{ x \} \cup (\freenames{P} \setminus \{ y \})
  \and 
  \freenames{x!\langle P \rangle} := \{ x \} \cup \{ P \} 
  \and \\
  \freenames{P|Q} := \freenames{P} \cup \freenames{Q}
  \and \\
  \freenames{@{x}} := \{ x \}
\end{mathpar}

$\pi$
$\quotep{\pi}$

$\freenames{-} : \pi \to \mathcal{P}(\quotep{\pi})$

\begin{eqnarray*}
  \freenames{\pzero} & := & \emptyset \\
  \freenames{x?(y).P} & := & \{ x \} \cup (\freenames{P} \setminus \{ y \}) \\
  \freenames{x!\langle P \rangle} & := & \{ x \} \cup \{ P \} \\
  \freenames{P|Q} & := & \freenames{P} \cup \freenames{Q} \\
  \freenames{\dropn{x}} & := & \{ x \}
\end{eqnarray*}

The bound names of a process, $\boundnames{P}$, are those names occurring in $P$
that are not free. For example, in $x?(y).0$, the name $x$ is free, while $y$ is bound.

\begin{mathpar}
  \inferrule* [lab=monoidal-laws] {} { P|Q \equiv Q|P \and P|0 \equiv P \and P|(Q|R) \equiv (P|Q)|R }
\end{mathpar}

\begin{mathpar}
  \inferrule* [lab=alpha-equivalence] {} { (x)P \equiv (y)P\{y/x\} \and y \not\in \freenames{P} }
\end{mathpar}

\begin{definition}
Then two processes, $P,Q$, are alpha-equivalent if $P = Q\{\vec{y}/\vec{x}\}$ for
some $\vec{x} \in \boundnames{Q},\vec{y} \in \boundnames{P}$, where $Q\{\vec{y}/\vec{x}\}$
denotes the capture-avoiding substitution of $\vec{y}$ for $\vec{x}$ in $Q$.
\end{definition}

\begin{definition}
  The {\em structural congruence} \cite{SangiorgiWalker} , $\equiv$,
  between processes is the least congruence containing
  alpha-equivalence, satisfying the abelian monoid laws
  (associativity, commutativity and $\pzero$ as identity) for parallel
  composition $|$ and for summation $+$.
\end{definition}

\subsection{Name equivalence}

We take name equivalence, written $\nameeq$, to be the smallest
equivalence relation generated by the following rules.

\begin{mathpar}
\inferrule*[lab=Quote-drop]
{ }
{ \quotep{@{x}} \nameeq x }

\inferrule*[lab=Struct-equiv]
{ P \scong Q }
{ \quotep{P} \nameeq \quotep{Q} }
\end{mathpar}

The astute reader will have noticed that the mutual recursion of names
and processes imposes a mutual recursion on alpha-equivalence and
structural equivalence via name-equivalence. Fortunately, all of this
works out pleasantly and we may calculate in the natural way, free of
concern. The reader interested in the details is referred to the
appendix \ref{appendix:rho_details}.

\subsection{Substitution}

We use $\Proc$ for the set of processes, $\QProc$ for the set of
names, and $\id{\{}\vec{y} / \vec{x} \id{\}}$ to denote partial maps,
$s : \QProc \rightarrow \QProc$. A map, $s$ lifts, uniquely, to a map
on process terms, $\widehat{s} : \Proc \rightarrow \Proc$ by the
following equations.

\begin{mathpar}
  (0) \psubstp{Q}{P} := 0 \\
  (R \juxtap S) \psubstp{Q}{P}
  :=    
  (R)\psubstp{Q}{P} \juxtap (S) \psubstp{Q}{P} \\
  (x?(y).R) \psubstp{Q}{P}    
  :=    
  (x)\substp{Q}{P} (z)\concat( (R \psubstn{z}{y}) \psubstp{Q}{P} ) \\
  (\lift{x}{R}) \psubstp{Q}{P}  
  :=
  \lift{(x)\substp{Q}{P}}{ R \psubstp{Q}{P} } \\
%   (\dropn{x})  \psubstp{Q}{P}       
%   := 
%   \left\{ 
%     \begin{array}{ccc} 
%       \dropn{\quotep{Q}} & & x \nameeq \quotep{P} \\
%       \dropn{x} & & otherwise \\
%     \end{array}
%   \right. 
  (\dropn{x})  \psubstp{Q}{P}       
  := 
  \left\{ 
    \begin{array}{ccc} 
      Q & & x \nameeq \quotep{P} \\
      \dropn{x} & & otherwise \\
    \end{array}
  \right.
\end{mathpar}
 

where

\begin{eqnarray}
  (x)\id{\{} \lpquote Q \rpquote / \lpquote P \rpquote \id{\}}            = 
  \left\{ 
    \begin{array}{ccc}
      \lpquote Q \rpquote & & x \nameeq \lpquote P \rpquote \\
      x & & otherwise \\
    \end{array}
  \right. \nonumber
\end{eqnarray}

and $z$ is chosen distinct from $\quotep{P}$, $\quotep{Q}$, the free
names in $Q$, and all the names in $R$. Our $\alpha$-equivalence will
be built in the standard way from this substitution.

\begin{remark}\label{rem:no_self_referential_names}
  One consequence of these definitions is that $\forall P. \quotep{P}
  \not\in \freenames{P}$.
\end{remark}

\subsection{ Dynamic quote: an example }

Anticipating something of what's to come, consider applying the
substitution, $\widehat{\id{\{}u / z \id{\}}}$, to the following pair
of processes, $\lift{w}{y!(z)}$ and $w[ \lpquote y!(z) \rpquote ]$.

\begin{eqnarray}
	\lift{w}{y!(z)}\widehat{\id{\{}u / z \id{\}}}
		& = &
		\lift{w}{y!(u)} \nonumber\\
	w[ \lpquote y!(z) \rpquote ] \widehat{ \id{\{}u / z \id{\}} }
		& = &
		w[ \lpquote y!(z) \rpquote ] \nonumber
\end{eqnarray}

Because the body of the process between quotes is impervious to
substitution, we get radically different answers. In fact, by
examining the first process in an input context,
e.g. $x?(z).\lift{w}{y!(z)}$, we see that the process under the lift
operator may be shaped by prefixed inputs binding a name inside it. In
this sense, the lift operator will be seen as a way to dynamically
construct processes before reifying them as names.

Finally equipped with these standard features we can present the
dynamics of the calculus.

\subsubsection{Operational semantics} 

Finally, we introduce the computational dynamics. What marks these
algebras as distinct from other more traditionally studied algebraic
structures, e.g. vector spaces or polynomial rings, is the manner in
which dynamics is captured. In traditional structures, dynamics is typically
expressed through morphisms between such structures, as in linear maps
between vector spaces or morphisms between rings. In algebras
associated with the semantics of computation, the dynamics is
expressed as part of the algebraic structure itself, through a
reduction reduction relation typically denoted by $\red$. Below, we
give a recursive presentation of this relation for the calculus used
in the encoding.

$\red \subseteq \pi \times \pi$
$\red : \pi \to \mathcal{P}(\pi)$

\begin{mathpar}
  \inferrule* [lab=Comm] { \textsf{match}( x_{src}, x_{trgt} ) } { x_{trgt}?(y)P \; | \; x_{src}!\langle {Q} \rangle \red P\{\quotep{Q}/y}\} }
  \and \\
  \inferrule* [lab=Par] {{P} \red {P}'} {{{P} | {Q}} \red {{P}' | {Q}}}
  \and
  \inferrule* [lab=Equiv]{{{P} \scong {P}'} \andalso {{P}' \red {Q}'} \andalso {{Q}' \scong {Q}}}{{P} \red {Q}}
\end{mathpar}

\begin{eqnarray*}
  match_{\equiv} (\quotep{P},\quotep{Q}) & := & P \equiv Q \\
  match_{\dagger}(\quotep{P},\quotep{Q}) & := & \forall R. P|Q \red^{*} R => R \red^{*} 0 \\
  match_{K}(\quotep{P},\quotep{Q}) & := & K \mbox{ for some context } K
\end{eqnarray*}

$u?(x)P | u!\langle Q \rangle \red P\{\quotep{Q}/x\}$

%We write $\wred$ for $\red^*$, and $P\red$ if $\exists Q $ such that $ P \red Q$.
We write $P\red$ if $\exists Q $ such that $ P \red Q$ and $P\not\red$, otherwise.

\section{Replication}

As mentioned before, it is known that replication (and hence
recursion) can be implemented in a higher-order process algebra
\cite{SangiorgiWalker}. As our first example of calculation with the
machinery thus far presented we give the construction explicitly in
the {\rhoc}.

\begin{eqnarray}
	D_{x} & := & \prefix{x}{y}{(\binpar{\outputp{x}{y}}{@{y}})} \nonumber\\
	\bangp_{x}{P} & := & \binpar{{x}!\langle{\binpar{D_{x}}{P}}\rangle}{D_{x}} \nonumber
\end{eqnarray}

\begin{eqnarray}
	\bangp_{x}{P} & & \nonumber\\
	=
	& {x}!\langle{(\prefix{x}{y}{(\outputp{x}{y} | @{y})) | P}}\rangle 
	      | \prefix{x}{y}{(\outputp{x}{y} | @{y})} & \nonumber\\
	\red
	& (\outputp{x}{y} | @{y})\substn{\quotep{(\prefix{x}{y}{(@{y} | \outputp{x}{y})) | P}}}{y} & \nonumber\\
	=
	& \outputp{x}{\quotep{(\prefix{x}{y}{(\outputp{x}{y} | @{y})) | P}}}
	  | {(\prefix{x}{y}{(\outputp{x}{y} | @{y})) | P}} & \nonumber\\
	\red
	& \ldots & \nonumber\\
	\red^*
	& P | P | \ldots & \nonumber
\end{eqnarray}

Of course, this encoding, as an implementation, runs away, unfolding
$\bangp{P}$ eagerly. A lazier and more implementable replication
operator, restricted to input-guarded processes, may be obtained as follows.

\begin{eqnarray}
\bangp{\prefix{u}{v}{P}} 
	:= 
	\binpar{\lift{x}{\prefix{u}{v}{(\binpar{D(x)}{P})}}}{D(x)} \nonumber
\end{eqnarray}

\begin{remark}
  Note that the lazier definition still does not deal with summation
  or mixed summation (i.e. sums over input and output). The reader is
  invited to construct definitions of replication that deal with these
  features. 

  Further, the definitions are parameterized in a name, $x$. Can you,
  gentle reader, make a definition that eliminates this parameter and
  guarantees no accidental interaction between the replication
  machinery and the process being replicated -- i.e. no accidental
  sharing of names used by the process to get its work done and the
  name(s) used by the replication to effect copying. This latter
  revision of the definition of replication is crucial to obtaining
  the expected identity $!!P \sim !P$.
\end{remark}

\begin{remark}\label{rem:paradoxical_combinator}
  The reader familiar with the lambda calculus will have noticed the
  similarity between $D$ and the paradoxical combinator.

  [Ed. note: the existence of this seems to suggest we have to be more
  restrictive on the set of processes and names we admit if we are to
  support no-cloning.]
\end{remark}

\subsubsection{Bisimulation}

The computational dynamics gives rise to another kind of equivalence,
the equivalence of computational behavior. As previously mentioned
this is typically captured \emph{via} some form of bisimulation.

% The notion we use in this paper is weak barbed bisimulation
% \cite{milner91polyadicpi}.

The notion we use in this paper is derived from weak barbed
bisimulation \cite{milner91polyadicpi}. 

\begin{definition}
An \emph{observation relation}, $\downarrow_{\mathcal N}$, over a set
of names, $\mathcal N$, is the smallest relation satisfying the rules
below.

\infrule[Out-barb]{y \in {\mathcal N}, \; x \nameeq y}
		  {\outputp{x}{v} \downarrow_{\mathcal N} x}
\infrule[Par-barb]{\mbox{$P\downarrow_{\mathcal N} x$ or $Q\downarrow_{\mathcal N} x$}}
		  {\binpar{P}{Q} \downarrow_{\mathcal N} x}

We write $P \Downarrow_{\mathcal N} x$ if there is $Q$ such that 
$P \wred Q$ and $Q \downarrow_{\mathcal N} x$.
\end{definition}

\begin{definition}
%\label{def.bbisim}
An  ${\mathcal N}$-\emph{barbed bisimulation} over a set of names, ${\mathcal N}$, is a symmetric binary relation 
${\mathcal S}_{\mathcal N}$ between agents such that $P\rel{S}_{\mathcal N}Q$ implies:
\begin{enumerate}
\item If $P \red P'$ then $Q \wred Q'$ and $P'\rel{S}_{\mathcal N} Q'$.
\item If $P\downarrow_{\mathcal N} x$, then $Q\Downarrow_{\mathcal N} x$.
\end{enumerate}
$P$ is ${\mathcal N}$-barbed bisimilar to $Q$, written
$P \wbbisim_{\mathcal N} Q$, if $P \rel{S}_{\mathcal N} Q$ for some ${\mathcal N}$-barbed bisimulation ${\mathcal S}_{\mathcal N}$.
\end{definition}

$\mathcal{R} \subseteq \pi \times \pi$

$P \mathcal{R} Q => \forall P'. P \red P' \Rightarrow \exists Q'. Q \red Q', P' \mathcal{R} Q'$

$P \vdash x \Rightarrow Q \vdash x$

\begin{mathpar}
  \inferrule*[lab=Out-barb]{x \nameeq y}{{y}!\langle{Q}\rangle \vdash x}
  \and
  \inferrule*[lab=Par-barb]{\mbox{$P\vdash x$ or $Q\vdash x$}}{\binpar{P}{Q} \vdash x}
\end{mathpar}

\subsubsection{Contexts}

One of the principle advantages of computational calculi like the
$\pi$-calculus is a well-defined notion of context,
contextual-equivalence and a correlation between
contextual-equivalence and notions of bisimulation. The notion of
context allows the decomposition of a process into (sub-)process and
its syntactic environment, its context. Thus, a context may be
thought of as a process with a ``hole'' (written $\Box$) in it. The
application of a context $M$ to a process $P$, written $M[P]$, is
tantamount to filling the hole in $M$ with $P$. In this paper we do
not need the full weight of this theory, but do make use of the notion
of context in the proof the main theorem. 

\begin{mathpar}
  \inferrule* [lab=summation] {} {{M_{M},M_{N}} \bc \Box \;|\; x.M_{A} \;|\; M_{M}+M_{N}}
  \and
  \inferrule* [lab=agent] {} {{M_{A}} \bc (\vec{x})M_{P} \;| \; \clift{P_0,\ldots,M_{P},\ldots,P_N}}
  \and \\
  \inferrule* [lab=process] {} {{M_{P}} \bc M_{N} \;| \;P|M_{P} }
\end{mathpar} 

\begin{mathpar}
  \inferrule* [lab=sychronization] {} {M_{N} \bc \Box \;|\; x?M_{F} \;|\; x!M_{C}}
  \and
  \inferrule* [lab=abstraction] {} {{M_{F}} \bc (x)M_{P} }
  \and
  \inferrule* [lab=concretion] {} {{M_{C}} \bc \langle M_{P} \rangle }
  \and \\
  \inferrule* [lab=process] {} {{M_{P}} \bc M_{N} \;| \;P|M_{P} }
\end{mathpar}

\begin{definition}[contextual application] Given a context $M$, and
  process $P$, we define the \emph{contextual application}, $M[P] :=
  M\{P/\Box\}$. That is, the contextual application of M to P is the
  substitution of $P$ for $\Box$ in $M$.
\end{definition}

$\meaningof{-} : L \to \mathcal{P}(\pi)$

\begin{mathpar}
  \inferrule* [lab=collection] {} {\meaningof{true} = \pi, \and \meaningof{~E} = \pi \setminus \meaningof{E}, \and \meaningof{E_{1} \& E_{2}} = \meaningof{E_{1}} \cap \meaningof{E_{2}}}
\end{mathpar}

\begin{mathpar}
  \inferrule* [lab=structure] {} {\meaningof{0} = \{ P \in \pi | P \equiv 0 \}, \and \\ \meaningof{E_1 | E_2} = \{ P \in \pi | P \equiv P_{1} | P_{2}, P_{1} \in \meaningof{E_{1}}, P_{2} \in \meaningof{E_2}\} }
\end{mathpar}

\begin{mathpar}
 \inferrule* [lab=behavior] {} {\meaningof{\langle a?b \rangle E} = \{ P \in \pi | P \equiv Q | u?(y)P', \\ \and \\\\ \and \\ \;\;\; u \in \meaningof{a}, \forall z.P'\{z/y\} \in \meaningof{E\{z/b\}}\}, \and \\ \meaningof{a!E} = \{ P \in \pi | P \equiv Q | x!\langle P' \rangle, x \in \meaningof{a} P' \in \meaningof{E}\} }
\end{mathpar}

\begin{mathpar}
 \inferrule* [lab=nominal] {} {\meaningof{\quotep{E}} = \{ \quotep{P} \in \quotep{\pi} | P \in \meaningof{E} \}, \and \meaningof{\quotep{P}} = \{ \quotep{Q} \in \quotep{\pi} | P \equiv Q \} \and \\ \meaningof{@\quotep{E}} = \{ P \in \pi | P \equiv @x, x \in \meaningof{E} \}}
\end{mathpar}

\begin{eqnarray*}
  \\
  \meaningof{-} : TS \to ST
\end{eqnarray*}

\begin{eqnarray*}
  \\
  L : TS \to ST
\end{eqnarray*}

\begin{eqnarray*}
  \\
  P \models E \iff P \in \meaningof{E}
\end{eqnarray*}

\begin{eqnarray*}
  P \approx_{L} Q \iff \forall E \in L. P \models E \iff Q \models E
\end{eqnarray*}

\begin{eqnarray*}
  P \approx_{K} Q
\end{eqnarray*}

\begin{eqnarray*}
  P \approx Q
\end{eqnarray*}

$\approx_{K} = \approx = \approx_{L}$

\subsubsection{Contextual duality}

Note that contexts extend the quotation operation to a family of
operations from processes to names. Given a context, $M$, we can
define a \emph{nominal context}, $\quotep{M}$ by $\quotep{M}[P] :=
\quotep{M[P]}$. To foreshadow what is to come we observe that these
operations enjoy a duality with processes very much like the duality
between vectors and maps from vectors to scalars.

Further, because the calculus is essentially higher-order, we have a
correspondence between contexts and processes. More specifically,
given a name $x$ and a context $M$ we can construct $M^{*}_{x}$ such
that 

\begin{mathpar}
  M^{*}_{x} | \lift{x}{P} \red M[P]
\end{mathpar}

namely,

\begin{mathpar}
  M^{*}_{x} := x?(u).M[\dropn{u}]
\end{mathpar}

The dependence of $M^{*}_{x}$ on a name makes it an abstraction, 

\begin{mathpar}
  M^{*} := (x)x?(u).M[\dropn{u}]
\end{mathpar}

\subsection{Additional notation}

It will sometimes be convenient to denote the process a name
quotes. We already have the notation $x = \quotep{P}$, but it will be
convenient to introduce an alternate notation, $\procn{x}$, when we
want to emphasize the connection to the use of the name. Note that, by
virtue of name equivalence, $\quotep{\procn{x}} \nameeq x$; so, the
notation is consistent with previous definitions.

Further, because names have structure it is possible to effect
substitutions on the basis of that structure. This means we need to
upgrade our notation for substitutions, which we accomplish by
adapting comprehension notation. Thus,

\begin{mathpar}
  P\{ y / x : x \in S \}
\end{mathpar}

is interpreted to mean the process derived from P by replacing (in a
capture-avoiding manner) each occurrence of $x$ in $S$ by $y$. For example,

\begin{mathpar}
  P\{ \quotep{\procn{x}|\procn{x}} / x : x \in \freenames{P} \}
\end{mathpar}

will replace each (occurrence) of a free name $x$ in $P$ by
$\quotep{\procn{x}|\procn{x}}$.

Also, we will avail ourselves of the notation $x^{L}$ and $x^{R}$ to
denote injections of a name into disjoint copies of the name
space. There are numerous ways to accomplish this. One example can be
found in \cite{MeredithR05}. This notation overloads to vectors of
names: $\vec{x}^{\pi} := (x_{i}^{\pi} \; : \; 0 \leq i < |\vec{x}| )$ where $\pi \in \{L,R\}$.

We also use $P^{\Box} := P|\Box$.

In \cite{MeredithR05} an interpretation of the new operator is
given. It turns out that there are several possible interpretations
all enjoying the requisite algebraic properties of the operator (see
\cite{milner91polyadicpi}). We will therefore make liberal use of
$(\nu\; \vec{x})P$.

% subsection the_syntax_and_semantics_of_the_notation_system (end)   

\input{qm2pi.qmops} 

\input{qm2pi.sterngerlach} 

\input{qm2pi.metric} 

% section concurrent_process_calculi (end)

%\input{qm2pi.proofsketch}

% section proof sketch (end)

%\input{qm2pi.slviaknots} 

% section spatial logic via knots (end)

\input{qm2pi.conclusion}

% section conclusion (end)

%\input{qm2pi.dtcodes} 

% section wiring algorithm (end)

\input{qm2pi.ack} 

% section acknowledgments (end)

\newpage


\bibliographystyle{plain}   
\bibliography{../../biblios/main.bib}

\input{qm2pi.rhodetails}

\end{document}

 

%\documentclass[12pt]{llncs}
%\documentclass{jktr}

\usepackage[pdftex]{hyperref}                   
\usepackage {listings}
\usepackage {mathpartir}
\usepackage{bcprules}
%\usepackage{listings}
                       
\usepackage{graphicx} 
%\usepackage[margins=2.5cm,nohead,nofoot]{geometry}
%\usepackage{geometry}
\usepackage{amsfonts}
\usepackage{amstext}
\usepackage{latexsym}
\usepackage{amssymb}
\usepackage{color}


%\include{myPreamble}
\include{qm2pi.local} 

%\ifpdf
%\usepackage[pdftex]{graphicx}
%\else
%\usepackage{graphicx}
%\fi

 % \ifpdf
%  \usepackage{pdfsync}
%  \if


%\title{Brief Article}
%\author{David F. Snyder}
%\author{L.G. Meredith}

%\address{Dept. of Math., Texas State University--San Marcos, San Marcos, TX 78666}
       
\pagestyle{empty}


\begin{document}

\lstset{language=[Objective]Caml,frame=shadowbox}

\input{qm2pi.front}

% section front matter (end)

\input{qm2pi.intro} 
 
% section introduction (end)

% \input{qm2pi.knotations} 

% section notation (end)

\input{qm2pi.process.calculi} 

% section concurrent_process_calculi_and_spatial_logics_ (end)
    
%\input{qm2pi.knots2pi} 

%\input{qm2pi.trefoil} 

%\input{qm2pi.mainthm} 

% subsection basic_interpretation (end)

%\input{qm2pi.rho.presentation} 
\subsection{The syntax and semantics of the notation system}\label{sub:the_syntax_and_semantics_of_the_notation_system} % (fold)

We now summarize a technical presentation of the calculus that
embodies our theory of dynamics. The typical presentation of such a
calculus follows the style of giving generators and relations on
them. The grammar, below, describing term constructors, freely
generates the set of processes, $\Proc$. This set is then quotiented
by a relation known as structural congruence and it is over this set
that the notion of dynamics is expressed. This presentation is
essentially that of \cite{MeredithR05} with the addition of
polyadicity and summation. For readability we have relegated some of
the technical subtleties to an appendix.

\subsubsection{Process grammar}\label{subsub:process_grammar}

\begin{mathpar}
  \inferrule* [lab=synchronization] {} {{M} \bc \pzero \;|\; x?F \;|\; x!C }
  \and
  \inferrule* [lab=abstraction] {} {{F} \bc (x)P}
  \and
  \inferrule* [lab=concretion] {} {{C} \bc \langle Q \rangle}
  \and
  \inferrule* [lab=process] {} {{P,Q} \bc M \;| \;P|Q \;|\; @{x}}
  \and
  \inferrule* [lab=name] {} {{x} \bc \quotep{P}}
\end{mathpar} 

Note that $\vec{x}$ (resp. $\vec{P}$) denotes a vector of names
(resp. processes) of length $|\vec{x}|$ (resp. $|\vec{P}|$). We adopt
the following useful abbreviations.

\begin{mathpar}
   x?(\vec{y}).P := x.(\vec{y})P \and  x\clift{\vec{P}} := x.\clift{\vec{P}}
   \and x!(y) := \lift{x}{\dropn{y}}
   \and \Pi_{i=0}^{n-1}P_i := P_0 | \ldots | P_{n-1}
\end{mathpar}

\subsubsection{Structural congruence}

\paragraph{Free and bound names and alpha-equivalence.} At the
core of structural equivalence is alpha-equivalence which identifies
process that are the same up to a change of variable. Formally, we
recognize the distinction between free and bound names. The free names
of a process, $\freenames{P}$, may be calculated recursively as
follows:

\begin{mathpar}
\freenames{\pzero} := \emptyset
  \and \\
  \freenames{x?(y).P} := \{ x \} \cup (\freenames{P} \setminus \{ y \})
  \and 
  \freenames{x!\langle P \rangle} := \{ x \} \cup \{ P \} 
  \and \\
  \freenames{P|Q} := \freenames{P} \cup \freenames{Q}
  \and \\
  \freenames{@{x}} := \{ x \}
\end{mathpar}

$\pi$
$\quotep{\pi}$

$\freenames{-} : \pi \to \mathcal{P}(\quotep{\pi})$

\begin{eqnarray*}
  \freenames{\pzero} & := & \emptyset \\
  \freenames{x?(y).P} & := & \{ x \} \cup (\freenames{P} \setminus \{ y \}) \\
  \freenames{x!\langle P \rangle} & := & \{ x \} \cup \{ P \} \\
  \freenames{P|Q} & := & \freenames{P} \cup \freenames{Q} \\
  \freenames{\dropn{x}} & := & \{ x \}
\end{eqnarray*}

The bound names of a process, $\boundnames{P}$, are those names occurring in $P$
that are not free. For example, in $x?(y).0$, the name $x$ is free, while $y$ is bound.

\begin{mathpar}
  \inferrule* [lab=monoidal-laws] {} { P|Q \equiv Q|P \and P|0 \equiv P \and P|(Q|R) \equiv (P|Q)|R }
\end{mathpar}

\begin{mathpar}
  \inferrule* [lab=alpha-equivalence] {} { (x)P \equiv (y)P\{y/x\} \and y \not\in \freenames{P} }
\end{mathpar}

\begin{definition}
Then two processes, $P,Q$, are alpha-equivalent if $P = Q\{\vec{y}/\vec{x}\}$ for
some $\vec{x} \in \boundnames{Q},\vec{y} \in \boundnames{P}$, where $Q\{\vec{y}/\vec{x}\}$
denotes the capture-avoiding substitution of $\vec{y}$ for $\vec{x}$ in $Q$.
\end{definition}

\begin{definition}
  The {\em structural congruence} \cite{SangiorgiWalker} , $\equiv$,
  between processes is the least congruence containing
  alpha-equivalence, satisfying the abelian monoid laws
  (associativity, commutativity and $\pzero$ as identity) for parallel
  composition $|$ and for summation $+$.
\end{definition}

\subsection{Name equivalence}

We take name equivalence, written $\nameeq$, to be the smallest
equivalence relation generated by the following rules.

\begin{mathpar}
\inferrule*[lab=Quote-drop]
{ }
{ \quotep{@{x}} \nameeq x }

\inferrule*[lab=Struct-equiv]
{ P \scong Q }
{ \quotep{P} \nameeq \quotep{Q} }
\end{mathpar}

The astute reader will have noticed that the mutual recursion of names
and processes imposes a mutual recursion on alpha-equivalence and
structural equivalence via name-equivalence. Fortunately, all of this
works out pleasantly and we may calculate in the natural way, free of
concern. The reader interested in the details is referred to the
appendix \ref{appendix:rho_details}.

\subsection{Substitution}

We use $\Proc$ for the set of processes, $\QProc$ for the set of
names, and $\id{\{}\vec{y} / \vec{x} \id{\}}$ to denote partial maps,
$s : \QProc \rightarrow \QProc$. A map, $s$ lifts, uniquely, to a map
on process terms, $\widehat{s} : \Proc \rightarrow \Proc$ by the
following equations.

\begin{mathpar}
  (0) \psubstp{Q}{P} := 0 \\
  (R \juxtap S) \psubstp{Q}{P}
  :=    
  (R)\psubstp{Q}{P} \juxtap (S) \psubstp{Q}{P} \\
  (x?(y).R) \psubstp{Q}{P}    
  :=    
  (x)\substp{Q}{P} (z)\concat( (R \psubstn{z}{y}) \psubstp{Q}{P} ) \\
  (\lift{x}{R}) \psubstp{Q}{P}  
  :=
  \lift{(x)\substp{Q}{P}}{ R \psubstp{Q}{P} } \\
%   (\dropn{x})  \psubstp{Q}{P}       
%   := 
%   \left\{ 
%     \begin{array}{ccc} 
%       \dropn{\quotep{Q}} & & x \nameeq \quotep{P} \\
%       \dropn{x} & & otherwise \\
%     \end{array}
%   \right. 
  (\dropn{x})  \psubstp{Q}{P}       
  := 
  \left\{ 
    \begin{array}{ccc} 
      Q & & x \nameeq \quotep{P} \\
      \dropn{x} & & otherwise \\
    \end{array}
  \right.
\end{mathpar}
 

where

\begin{eqnarray}
  (x)\id{\{} \lpquote Q \rpquote / \lpquote P \rpquote \id{\}}            = 
  \left\{ 
    \begin{array}{ccc}
      \lpquote Q \rpquote & & x \nameeq \lpquote P \rpquote \\
      x & & otherwise \\
    \end{array}
  \right. \nonumber
\end{eqnarray}

and $z$ is chosen distinct from $\quotep{P}$, $\quotep{Q}$, the free
names in $Q$, and all the names in $R$. Our $\alpha$-equivalence will
be built in the standard way from this substitution.

\begin{remark}\label{rem:no_self_referential_names}
  One consequence of these definitions is that $\forall P. \quotep{P}
  \not\in \freenames{P}$.
\end{remark}

\subsection{ Dynamic quote: an example }

Anticipating something of what's to come, consider applying the
substitution, $\widehat{\id{\{}u / z \id{\}}}$, to the following pair
of processes, $\lift{w}{y!(z)}$ and $w[ \lpquote y!(z) \rpquote ]$.

\begin{eqnarray}
	\lift{w}{y!(z)}\widehat{\id{\{}u / z \id{\}}}
		& = &
		\lift{w}{y!(u)} \nonumber\\
	w[ \lpquote y!(z) \rpquote ] \widehat{ \id{\{}u / z \id{\}} }
		& = &
		w[ \lpquote y!(z) \rpquote ] \nonumber
\end{eqnarray}

Because the body of the process between quotes is impervious to
substitution, we get radically different answers. In fact, by
examining the first process in an input context,
e.g. $x?(z).\lift{w}{y!(z)}$, we see that the process under the lift
operator may be shaped by prefixed inputs binding a name inside it. In
this sense, the lift operator will be seen as a way to dynamically
construct processes before reifying them as names.

Finally equipped with these standard features we can present the
dynamics of the calculus.

\subsubsection{Operational semantics} 

Finally, we introduce the computational dynamics. What marks these
algebras as distinct from other more traditionally studied algebraic
structures, e.g. vector spaces or polynomial rings, is the manner in
which dynamics is captured. In traditional structures, dynamics is typically
expressed through morphisms between such structures, as in linear maps
between vector spaces or morphisms between rings. In algebras
associated with the semantics of computation, the dynamics is
expressed as part of the algebraic structure itself, through a
reduction reduction relation typically denoted by $\red$. Below, we
give a recursive presentation of this relation for the calculus used
in the encoding.

$\red \subseteq \pi \times \pi$
$\red : \pi \to \mathcal{P}(\pi)$

\begin{mathpar}
  \inferrule* [lab=Comm] { \textsf{match}( x_{src}, x_{trgt} ) } { x_{trgt}?(y)P \; | \; x_{src}!\langle {Q} \rangle \red P\{\quotep{Q}/y}\} }
  \and \\
  \inferrule* [lab=Par] {{P} \red {P}'} {{{P} | {Q}} \red {{P}' | {Q}}}
  \and
  \inferrule* [lab=Equiv]{{{P} \scong {P}'} \andalso {{P}' \red {Q}'} \andalso {{Q}' \scong {Q}}}{{P} \red {Q}}
\end{mathpar}

\begin{eqnarray*}
  match_{\equiv} (\quotep{P},\quotep{Q}) & := & P \equiv Q \\
  match_{\dagger}(\quotep{P},\quotep{Q}) & := & \forall R. P|Q \red^{*} R => R \red^{*} 0 \\
  match_{K}(\quotep{P},\quotep{Q}) & := & K \mbox{ for some context } K
\end{eqnarray*}

$u?(x)P | u!\langle Q \rangle \red P\{\quotep{Q}/x\}$

%We write $\wred$ for $\red^*$, and $P\red$ if $\exists Q $ such that $ P \red Q$.
We write $P\red$ if $\exists Q $ such that $ P \red Q$ and $P\not\red$, otherwise.

\section{Replication}

As mentioned before, it is known that replication (and hence
recursion) can be implemented in a higher-order process algebra
\cite{SangiorgiWalker}. As our first example of calculation with the
machinery thus far presented we give the construction explicitly in
the {\rhoc}.

\begin{eqnarray}
	D_{x} & := & \prefix{x}{y}{(\binpar{\outputp{x}{y}}{@{y}})} \nonumber\\
	\bangp_{x}{P} & := & \binpar{{x}!\langle{\binpar{D_{x}}{P}}\rangle}{D_{x}} \nonumber
\end{eqnarray}

\begin{eqnarray}
	\bangp_{x}{P} & & \nonumber\\
	=
	& {x}!\langle{(\prefix{x}{y}{(\outputp{x}{y} | @{y})) | P}}\rangle 
	      | \prefix{x}{y}{(\outputp{x}{y} | @{y})} & \nonumber\\
	\red
	& (\outputp{x}{y} | @{y})\substn{\quotep{(\prefix{x}{y}{(@{y} | \outputp{x}{y})) | P}}}{y} & \nonumber\\
	=
	& \outputp{x}{\quotep{(\prefix{x}{y}{(\outputp{x}{y} | @{y})) | P}}}
	  | {(\prefix{x}{y}{(\outputp{x}{y} | @{y})) | P}} & \nonumber\\
	\red
	& \ldots & \nonumber\\
	\red^*
	& P | P | \ldots & \nonumber
\end{eqnarray}

Of course, this encoding, as an implementation, runs away, unfolding
$\bangp{P}$ eagerly. A lazier and more implementable replication
operator, restricted to input-guarded processes, may be obtained as follows.

\begin{eqnarray}
\bangp{\prefix{u}{v}{P}} 
	:= 
	\binpar{\lift{x}{\prefix{u}{v}{(\binpar{D(x)}{P})}}}{D(x)} \nonumber
\end{eqnarray}

\begin{remark}
  Note that the lazier definition still does not deal with summation
  or mixed summation (i.e. sums over input and output). The reader is
  invited to construct definitions of replication that deal with these
  features. 

  Further, the definitions are parameterized in a name, $x$. Can you,
  gentle reader, make a definition that eliminates this parameter and
  guarantees no accidental interaction between the replication
  machinery and the process being replicated -- i.e. no accidental
  sharing of names used by the process to get its work done and the
  name(s) used by the replication to effect copying. This latter
  revision of the definition of replication is crucial to obtaining
  the expected identity $!!P \sim !P$.
\end{remark}

\begin{remark}\label{rem:paradoxical_combinator}
  The reader familiar with the lambda calculus will have noticed the
  similarity between $D$ and the paradoxical combinator.

  [Ed. note: the existence of this seems to suggest we have to be more
  restrictive on the set of processes and names we admit if we are to
  support no-cloning.]
\end{remark}

\subsubsection{Bisimulation}

The computational dynamics gives rise to another kind of equivalence,
the equivalence of computational behavior. As previously mentioned
this is typically captured \emph{via} some form of bisimulation.

% The notion we use in this paper is weak barbed bisimulation
% \cite{milner91polyadicpi}.

The notion we use in this paper is derived from weak barbed
bisimulation \cite{milner91polyadicpi}. 

\begin{definition}
An \emph{observation relation}, $\downarrow_{\mathcal N}$, over a set
of names, $\mathcal N$, is the smallest relation satisfying the rules
below.

\infrule[Out-barb]{y \in {\mathcal N}, \; x \nameeq y}
		  {\outputp{x}{v} \downarrow_{\mathcal N} x}
\infrule[Par-barb]{\mbox{$P\downarrow_{\mathcal N} x$ or $Q\downarrow_{\mathcal N} x$}}
		  {\binpar{P}{Q} \downarrow_{\mathcal N} x}

We write $P \Downarrow_{\mathcal N} x$ if there is $Q$ such that 
$P \wred Q$ and $Q \downarrow_{\mathcal N} x$.
\end{definition}

\begin{definition}
%\label{def.bbisim}
An  ${\mathcal N}$-\emph{barbed bisimulation} over a set of names, ${\mathcal N}$, is a symmetric binary relation 
${\mathcal S}_{\mathcal N}$ between agents such that $P\rel{S}_{\mathcal N}Q$ implies:
\begin{enumerate}
\item If $P \red P'$ then $Q \wred Q'$ and $P'\rel{S}_{\mathcal N} Q'$.
\item If $P\downarrow_{\mathcal N} x$, then $Q\Downarrow_{\mathcal N} x$.
\end{enumerate}
$P$ is ${\mathcal N}$-barbed bisimilar to $Q$, written
$P \wbbisim_{\mathcal N} Q$, if $P \rel{S}_{\mathcal N} Q$ for some ${\mathcal N}$-barbed bisimulation ${\mathcal S}_{\mathcal N}$.
\end{definition}

$\mathcal{R} \subseteq \pi \times \pi$

$P \mathcal{R} Q => \forall P'. P \red P' \Rightarrow \exists Q'. Q \red Q', P' \mathcal{R} Q'$

$P \vdash x \Rightarrow Q \vdash x$

\begin{mathpar}
  \inferrule*[lab=Out-barb]{x \nameeq y}{{y}!\langle{Q}\rangle \vdash x}
  \and
  \inferrule*[lab=Par-barb]{\mbox{$P\vdash x$ or $Q\vdash x$}}{\binpar{P}{Q} \vdash x}
\end{mathpar}

\subsubsection{Contexts}

One of the principle advantages of computational calculi like the
$\pi$-calculus is a well-defined notion of context,
contextual-equivalence and a correlation between
contextual-equivalence and notions of bisimulation. The notion of
context allows the decomposition of a process into (sub-)process and
its syntactic environment, its context. Thus, a context may be
thought of as a process with a ``hole'' (written $\Box$) in it. The
application of a context $M$ to a process $P$, written $M[P]$, is
tantamount to filling the hole in $M$ with $P$. In this paper we do
not need the full weight of this theory, but do make use of the notion
of context in the proof the main theorem. 

\begin{mathpar}
  \inferrule* [lab=summation] {} {{M_{M},M_{N}} \bc \Box \;|\; x.M_{A} \;|\; M_{M}+M_{N}}
  \and
  \inferrule* [lab=agent] {} {{M_{A}} \bc (\vec{x})M_{P} \;| \; \clift{P_0,\ldots,M_{P},\ldots,P_N}}
  \and \\
  \inferrule* [lab=process] {} {{M_{P}} \bc M_{N} \;| \;P|M_{P} }
\end{mathpar} 

\begin{mathpar}
  \inferrule* [lab=sychronization] {} {M_{N} \bc \Box \;|\; x?M_{F} \;|\; x!M_{C}}
  \and
  \inferrule* [lab=abstraction] {} {{M_{F}} \bc (x)M_{P} }
  \and
  \inferrule* [lab=concretion] {} {{M_{C}} \bc \langle M_{P} \rangle }
  \and \\
  \inferrule* [lab=process] {} {{M_{P}} \bc M_{N} \;| \;P|M_{P} }
\end{mathpar}

\begin{definition}[contextual application] Given a context $M$, and
  process $P$, we define the \emph{contextual application}, $M[P] :=
  M\{P/\Box\}$. That is, the contextual application of M to P is the
  substitution of $P$ for $\Box$ in $M$.
\end{definition}

$\meaningof{-} : L \to \mathcal{P}(\pi)$

\begin{mathpar}
  \inferrule* [lab=collection] {} {\meaningof{true} = \pi, \and \meaningof{~E} = \pi \setminus \meaningof{E}, \and \meaningof{E_{1} \& E_{2}} = \meaningof{E_{1}} \cap \meaningof{E_{2}}}
\end{mathpar}

\begin{mathpar}
  \inferrule* [lab=structure] {} {\meaningof{0} = \{ P \in \pi | P \equiv 0 \}, \and \\ \meaningof{E_1 | E_2} = \{ P \in \pi | P \equiv P_{1} | P_{2}, P_{1} \in \meaningof{E_{1}}, P_{2} \in \meaningof{E_2}\} }
\end{mathpar}

\begin{mathpar}
 \inferrule* [lab=behavior] {} {\meaningof{\langle a?b \rangle E} = \{ P \in \pi | P \equiv Q | u?(y)P', \\ \and \\\\ \and \\ \;\;\; u \in \meaningof{a}, \forall z.P'\{z/y\} \in \meaningof{E\{z/b\}}\}, \and \\ \meaningof{a!E} = \{ P \in \pi | P \equiv Q | x!\langle P' \rangle, x \in \meaningof{a} P' \in \meaningof{E}\} }
\end{mathpar}

\begin{mathpar}
 \inferrule* [lab=nominal] {} {\meaningof{\quotep{E}} = \{ \quotep{P} \in \quotep{\pi} | P \in \meaningof{E} \}, \and \meaningof{\quotep{P}} = \{ \quotep{Q} \in \quotep{\pi} | P \equiv Q \} \and \\ \meaningof{@\quotep{E}} = \{ P \in \pi | P \equiv @x, x \in \meaningof{E} \}}
\end{mathpar}

\begin{eqnarray*}
  \\
  \meaningof{-} : TS \to ST
\end{eqnarray*}

\begin{eqnarray*}
  \\
  L : TS \to ST
\end{eqnarray*}

\begin{eqnarray*}
  \\
  P \models E \iff P \in \meaningof{E}
\end{eqnarray*}

\begin{eqnarray*}
  P \approx_{L} Q \iff \forall E \in L. P \models E \iff Q \models E
\end{eqnarray*}

\begin{eqnarray*}
  P \approx_{K} Q
\end{eqnarray*}

\begin{eqnarray*}
  P \approx Q
\end{eqnarray*}

$\approx_{K} = \approx = \approx_{L}$

\subsubsection{Contextual duality}

Note that contexts extend the quotation operation to a family of
operations from processes to names. Given a context, $M$, we can
define a \emph{nominal context}, $\quotep{M}$ by $\quotep{M}[P] :=
\quotep{M[P]}$. To foreshadow what is to come we observe that these
operations enjoy a duality with processes very much like the duality
between vectors and maps from vectors to scalars.

Further, because the calculus is essentially higher-order, we have a
correspondence between contexts and processes. More specifically,
given a name $x$ and a context $M$ we can construct $M^{*}_{x}$ such
that 

\begin{mathpar}
  M^{*}_{x} | \lift{x}{P} \red M[P]
\end{mathpar}

namely,

\begin{mathpar}
  M^{*}_{x} := x?(u).M[\dropn{u}]
\end{mathpar}

The dependence of $M^{*}_{x}$ on a name makes it an abstraction, 

\begin{mathpar}
  M^{*} := (x)x?(u).M[\dropn{u}]
\end{mathpar}

\subsection{Additional notation}

It will sometimes be convenient to denote the process a name
quotes. We already have the notation $x = \quotep{P}$, but it will be
convenient to introduce an alternate notation, $\procn{x}$, when we
want to emphasize the connection to the use of the name. Note that, by
virtue of name equivalence, $\quotep{\procn{x}} \nameeq x$; so, the
notation is consistent with previous definitions.

Further, because names have structure it is possible to effect
substitutions on the basis of that structure. This means we need to
upgrade our notation for substitutions, which we accomplish by
adapting comprehension notation. Thus,

\begin{mathpar}
  P\{ y / x : x \in S \}
\end{mathpar}

is interpreted to mean the process derived from P by replacing (in a
capture-avoiding manner) each occurrence of $x$ in $S$ by $y$. For example,

\begin{mathpar}
  P\{ \quotep{\procn{x}|\procn{x}} / x : x \in \freenames{P} \}
\end{mathpar}

will replace each (occurrence) of a free name $x$ in $P$ by
$\quotep{\procn{x}|\procn{x}}$.

Also, we will avail ourselves of the notation $x^{L}$ and $x^{R}$ to
denote injections of a name into disjoint copies of the name
space. There are numerous ways to accomplish this. One example can be
found in \cite{MeredithR05}. This notation overloads to vectors of
names: $\vec{x}^{\pi} := (x_{i}^{\pi} \; : \; 0 \leq i < |\vec{x}| )$ where $\pi \in \{L,R\}$.

We also use $P^{\Box} := P|\Box$.

In \cite{MeredithR05} an interpretation of the new operator is
given. It turns out that there are several possible interpretations
all enjoying the requisite algebraic properties of the operator (see
\cite{milner91polyadicpi}). We will therefore make liberal use of
$(\nu\; \vec{x})P$.

% subsection the_syntax_and_semantics_of_the_notation_system (end)   

\input{qm2pi.qmops} 

\input{qm2pi.sterngerlach} 

\input{qm2pi.metric} 

% section concurrent_process_calculi (end)

%\input{qm2pi.proofsketch}

% section proof sketch (end)

%\input{qm2pi.slviaknots} 

% section spatial logic via knots (end)

\input{qm2pi.conclusion}

% section conclusion (end)

%\input{qm2pi.dtcodes} 

% section wiring algorithm (end)

\input{qm2pi.ack} 

% section acknowledgments (end)

\newpage


\bibliographystyle{plain}   
\bibliography{../../biblios/main.bib}

\input{qm2pi.rhodetails}

\end{document}

 

%\documentclass[12pt]{llncs}
%\documentclass{jktr}

\usepackage[pdftex]{hyperref}                   
\usepackage {listings}
\usepackage {mathpartir}
\usepackage{bcprules}
%\usepackage{listings}
                       
\usepackage{graphicx} 
%\usepackage[margins=2.5cm,nohead,nofoot]{geometry}
%\usepackage{geometry}
\usepackage{amsfonts}
\usepackage{amstext}
\usepackage{latexsym}
\usepackage{amssymb}
\usepackage{color}


%\include{myPreamble}
\include{qm2pi.local} 

%\ifpdf
%\usepackage[pdftex]{graphicx}
%\else
%\usepackage{graphicx}
%\fi

 % \ifpdf
%  \usepackage{pdfsync}
%  \if


%\title{Brief Article}
%\author{David F. Snyder}
%\author{L.G. Meredith}

%\address{Dept. of Math., Texas State University--San Marcos, San Marcos, TX 78666}
       
\pagestyle{empty}


\begin{document}

\lstset{language=[Objective]Caml,frame=shadowbox}

\input{qm2pi.front}

% section front matter (end)

\input{qm2pi.intro} 
 
% section introduction (end)

% \input{qm2pi.knotations} 

% section notation (end)

\input{qm2pi.process.calculi} 

% section concurrent_process_calculi_and_spatial_logics_ (end)
    
%\input{qm2pi.knots2pi} 

%\input{qm2pi.trefoil} 

%\input{qm2pi.mainthm} 

% subsection basic_interpretation (end)

%\input{qm2pi.rho.presentation} 
\subsection{The syntax and semantics of the notation system}\label{sub:the_syntax_and_semantics_of_the_notation_system} % (fold)

We now summarize a technical presentation of the calculus that
embodies our theory of dynamics. The typical presentation of such a
calculus follows the style of giving generators and relations on
them. The grammar, below, describing term constructors, freely
generates the set of processes, $\Proc$. This set is then quotiented
by a relation known as structural congruence and it is over this set
that the notion of dynamics is expressed. This presentation is
essentially that of \cite{MeredithR05} with the addition of
polyadicity and summation. For readability we have relegated some of
the technical subtleties to an appendix.

\subsubsection{Process grammar}\label{subsub:process_grammar}

\begin{mathpar}
  \inferrule* [lab=synchronization] {} {{M} \bc \pzero \;|\; x?F \;|\; x!C }
  \and
  \inferrule* [lab=abstraction] {} {{F} \bc (x)P}
  \and
  \inferrule* [lab=concretion] {} {{C} \bc \langle Q \rangle}
  \and
  \inferrule* [lab=process] {} {{P,Q} \bc M \;| \;P|Q \;|\; @{x}}
  \and
  \inferrule* [lab=name] {} {{x} \bc \quotep{P}}
\end{mathpar} 

Note that $\vec{x}$ (resp. $\vec{P}$) denotes a vector of names
(resp. processes) of length $|\vec{x}|$ (resp. $|\vec{P}|$). We adopt
the following useful abbreviations.

\begin{mathpar}
   x?(\vec{y}).P := x.(\vec{y})P \and  x\clift{\vec{P}} := x.\clift{\vec{P}}
   \and x!(y) := \lift{x}{\dropn{y}}
   \and \Pi_{i=0}^{n-1}P_i := P_0 | \ldots | P_{n-1}
\end{mathpar}

\subsubsection{Structural congruence}

\paragraph{Free and bound names and alpha-equivalence.} At the
core of structural equivalence is alpha-equivalence which identifies
process that are the same up to a change of variable. Formally, we
recognize the distinction between free and bound names. The free names
of a process, $\freenames{P}$, may be calculated recursively as
follows:

\begin{mathpar}
\freenames{\pzero} := \emptyset
  \and \\
  \freenames{x?(y).P} := \{ x \} \cup (\freenames{P} \setminus \{ y \})
  \and 
  \freenames{x!\langle P \rangle} := \{ x \} \cup \{ P \} 
  \and \\
  \freenames{P|Q} := \freenames{P} \cup \freenames{Q}
  \and \\
  \freenames{@{x}} := \{ x \}
\end{mathpar}

$\pi$
$\quotep{\pi}$

$\freenames{-} : \pi \to \mathcal{P}(\quotep{\pi})$

\begin{eqnarray*}
  \freenames{\pzero} & := & \emptyset \\
  \freenames{x?(y).P} & := & \{ x \} \cup (\freenames{P} \setminus \{ y \}) \\
  \freenames{x!\langle P \rangle} & := & \{ x \} \cup \{ P \} \\
  \freenames{P|Q} & := & \freenames{P} \cup \freenames{Q} \\
  \freenames{\dropn{x}} & := & \{ x \}
\end{eqnarray*}

The bound names of a process, $\boundnames{P}$, are those names occurring in $P$
that are not free. For example, in $x?(y).0$, the name $x$ is free, while $y$ is bound.

\begin{mathpar}
  \inferrule* [lab=monoidal-laws] {} { P|Q \equiv Q|P \and P|0 \equiv P \and P|(Q|R) \equiv (P|Q)|R }
\end{mathpar}

\begin{mathpar}
  \inferrule* [lab=alpha-equivalence] {} { (x)P \equiv (y)P\{y/x\} \and y \not\in \freenames{P} }
\end{mathpar}

\begin{definition}
Then two processes, $P,Q$, are alpha-equivalent if $P = Q\{\vec{y}/\vec{x}\}$ for
some $\vec{x} \in \boundnames{Q},\vec{y} \in \boundnames{P}$, where $Q\{\vec{y}/\vec{x}\}$
denotes the capture-avoiding substitution of $\vec{y}$ for $\vec{x}$ in $Q$.
\end{definition}

\begin{definition}
  The {\em structural congruence} \cite{SangiorgiWalker} , $\equiv$,
  between processes is the least congruence containing
  alpha-equivalence, satisfying the abelian monoid laws
  (associativity, commutativity and $\pzero$ as identity) for parallel
  composition $|$ and for summation $+$.
\end{definition}

\subsection{Name equivalence}

We take name equivalence, written $\nameeq$, to be the smallest
equivalence relation generated by the following rules.

\begin{mathpar}
\inferrule*[lab=Quote-drop]
{ }
{ \quotep{@{x}} \nameeq x }

\inferrule*[lab=Struct-equiv]
{ P \scong Q }
{ \quotep{P} \nameeq \quotep{Q} }
\end{mathpar}

The astute reader will have noticed that the mutual recursion of names
and processes imposes a mutual recursion on alpha-equivalence and
structural equivalence via name-equivalence. Fortunately, all of this
works out pleasantly and we may calculate in the natural way, free of
concern. The reader interested in the details is referred to the
appendix \ref{appendix:rho_details}.

\subsection{Substitution}

We use $\Proc$ for the set of processes, $\QProc$ for the set of
names, and $\id{\{}\vec{y} / \vec{x} \id{\}}$ to denote partial maps,
$s : \QProc \rightarrow \QProc$. A map, $s$ lifts, uniquely, to a map
on process terms, $\widehat{s} : \Proc \rightarrow \Proc$ by the
following equations.

\begin{mathpar}
  (0) \psubstp{Q}{P} := 0 \\
  (R \juxtap S) \psubstp{Q}{P}
  :=    
  (R)\psubstp{Q}{P} \juxtap (S) \psubstp{Q}{P} \\
  (x?(y).R) \psubstp{Q}{P}    
  :=    
  (x)\substp{Q}{P} (z)\concat( (R \psubstn{z}{y}) \psubstp{Q}{P} ) \\
  (\lift{x}{R}) \psubstp{Q}{P}  
  :=
  \lift{(x)\substp{Q}{P}}{ R \psubstp{Q}{P} } \\
%   (\dropn{x})  \psubstp{Q}{P}       
%   := 
%   \left\{ 
%     \begin{array}{ccc} 
%       \dropn{\quotep{Q}} & & x \nameeq \quotep{P} \\
%       \dropn{x} & & otherwise \\
%     \end{array}
%   \right. 
  (\dropn{x})  \psubstp{Q}{P}       
  := 
  \left\{ 
    \begin{array}{ccc} 
      Q & & x \nameeq \quotep{P} \\
      \dropn{x} & & otherwise \\
    \end{array}
  \right.
\end{mathpar}
 

where

\begin{eqnarray}
  (x)\id{\{} \lpquote Q \rpquote / \lpquote P \rpquote \id{\}}            = 
  \left\{ 
    \begin{array}{ccc}
      \lpquote Q \rpquote & & x \nameeq \lpquote P \rpquote \\
      x & & otherwise \\
    \end{array}
  \right. \nonumber
\end{eqnarray}

and $z$ is chosen distinct from $\quotep{P}$, $\quotep{Q}$, the free
names in $Q$, and all the names in $R$. Our $\alpha$-equivalence will
be built in the standard way from this substitution.

\begin{remark}\label{rem:no_self_referential_names}
  One consequence of these definitions is that $\forall P. \quotep{P}
  \not\in \freenames{P}$.
\end{remark}

\subsection{ Dynamic quote: an example }

Anticipating something of what's to come, consider applying the
substitution, $\widehat{\id{\{}u / z \id{\}}}$, to the following pair
of processes, $\lift{w}{y!(z)}$ and $w[ \lpquote y!(z) \rpquote ]$.

\begin{eqnarray}
	\lift{w}{y!(z)}\widehat{\id{\{}u / z \id{\}}}
		& = &
		\lift{w}{y!(u)} \nonumber\\
	w[ \lpquote y!(z) \rpquote ] \widehat{ \id{\{}u / z \id{\}} }
		& = &
		w[ \lpquote y!(z) \rpquote ] \nonumber
\end{eqnarray}

Because the body of the process between quotes is impervious to
substitution, we get radically different answers. In fact, by
examining the first process in an input context,
e.g. $x?(z).\lift{w}{y!(z)}$, we see that the process under the lift
operator may be shaped by prefixed inputs binding a name inside it. In
this sense, the lift operator will be seen as a way to dynamically
construct processes before reifying them as names.

Finally equipped with these standard features we can present the
dynamics of the calculus.

\subsubsection{Operational semantics} 

Finally, we introduce the computational dynamics. What marks these
algebras as distinct from other more traditionally studied algebraic
structures, e.g. vector spaces or polynomial rings, is the manner in
which dynamics is captured. In traditional structures, dynamics is typically
expressed through morphisms between such structures, as in linear maps
between vector spaces or morphisms between rings. In algebras
associated with the semantics of computation, the dynamics is
expressed as part of the algebraic structure itself, through a
reduction reduction relation typically denoted by $\red$. Below, we
give a recursive presentation of this relation for the calculus used
in the encoding.

$\red \subseteq \pi \times \pi$
$\red : \pi \to \mathcal{P}(\pi)$

\begin{mathpar}
  \inferrule* [lab=Comm] { \textsf{match}( x_{src}, x_{trgt} ) } { x_{trgt}?(y)P \; | \; x_{src}!\langle {Q} \rangle \red P\{\quotep{Q}/y}\} }
  \and \\
  \inferrule* [lab=Par] {{P} \red {P}'} {{{P} | {Q}} \red {{P}' | {Q}}}
  \and
  \inferrule* [lab=Equiv]{{{P} \scong {P}'} \andalso {{P}' \red {Q}'} \andalso {{Q}' \scong {Q}}}{{P} \red {Q}}
\end{mathpar}

\begin{eqnarray*}
  match_{\equiv} (\quotep{P},\quotep{Q}) & := & P \equiv Q \\
  match_{\dagger}(\quotep{P},\quotep{Q}) & := & \forall R. P|Q \red^{*} R => R \red^{*} 0 \\
  match_{K}(\quotep{P},\quotep{Q}) & := & K \mbox{ for some context } K
\end{eqnarray*}

$u?(x)P | u!\langle Q \rangle \red P\{\quotep{Q}/x\}$

%We write $\wred$ for $\red^*$, and $P\red$ if $\exists Q $ such that $ P \red Q$.
We write $P\red$ if $\exists Q $ such that $ P \red Q$ and $P\not\red$, otherwise.

\section{Replication}

As mentioned before, it is known that replication (and hence
recursion) can be implemented in a higher-order process algebra
\cite{SangiorgiWalker}. As our first example of calculation with the
machinery thus far presented we give the construction explicitly in
the {\rhoc}.

\begin{eqnarray}
	D_{x} & := & \prefix{x}{y}{(\binpar{\outputp{x}{y}}{@{y}})} \nonumber\\
	\bangp_{x}{P} & := & \binpar{{x}!\langle{\binpar{D_{x}}{P}}\rangle}{D_{x}} \nonumber
\end{eqnarray}

\begin{eqnarray}
	\bangp_{x}{P} & & \nonumber\\
	=
	& {x}!\langle{(\prefix{x}{y}{(\outputp{x}{y} | @{y})) | P}}\rangle 
	      | \prefix{x}{y}{(\outputp{x}{y} | @{y})} & \nonumber\\
	\red
	& (\outputp{x}{y} | @{y})\substn{\quotep{(\prefix{x}{y}{(@{y} | \outputp{x}{y})) | P}}}{y} & \nonumber\\
	=
	& \outputp{x}{\quotep{(\prefix{x}{y}{(\outputp{x}{y} | @{y})) | P}}}
	  | {(\prefix{x}{y}{(\outputp{x}{y} | @{y})) | P}} & \nonumber\\
	\red
	& \ldots & \nonumber\\
	\red^*
	& P | P | \ldots & \nonumber
\end{eqnarray}

Of course, this encoding, as an implementation, runs away, unfolding
$\bangp{P}$ eagerly. A lazier and more implementable replication
operator, restricted to input-guarded processes, may be obtained as follows.

\begin{eqnarray}
\bangp{\prefix{u}{v}{P}} 
	:= 
	\binpar{\lift{x}{\prefix{u}{v}{(\binpar{D(x)}{P})}}}{D(x)} \nonumber
\end{eqnarray}

\begin{remark}
  Note that the lazier definition still does not deal with summation
  or mixed summation (i.e. sums over input and output). The reader is
  invited to construct definitions of replication that deal with these
  features. 

  Further, the definitions are parameterized in a name, $x$. Can you,
  gentle reader, make a definition that eliminates this parameter and
  guarantees no accidental interaction between the replication
  machinery and the process being replicated -- i.e. no accidental
  sharing of names used by the process to get its work done and the
  name(s) used by the replication to effect copying. This latter
  revision of the definition of replication is crucial to obtaining
  the expected identity $!!P \sim !P$.
\end{remark}

\begin{remark}\label{rem:paradoxical_combinator}
  The reader familiar with the lambda calculus will have noticed the
  similarity between $D$ and the paradoxical combinator.

  [Ed. note: the existence of this seems to suggest we have to be more
  restrictive on the set of processes and names we admit if we are to
  support no-cloning.]
\end{remark}

\subsubsection{Bisimulation}

The computational dynamics gives rise to another kind of equivalence,
the equivalence of computational behavior. As previously mentioned
this is typically captured \emph{via} some form of bisimulation.

% The notion we use in this paper is weak barbed bisimulation
% \cite{milner91polyadicpi}.

The notion we use in this paper is derived from weak barbed
bisimulation \cite{milner91polyadicpi}. 

\begin{definition}
An \emph{observation relation}, $\downarrow_{\mathcal N}$, over a set
of names, $\mathcal N$, is the smallest relation satisfying the rules
below.

\infrule[Out-barb]{y \in {\mathcal N}, \; x \nameeq y}
		  {\outputp{x}{v} \downarrow_{\mathcal N} x}
\infrule[Par-barb]{\mbox{$P\downarrow_{\mathcal N} x$ or $Q\downarrow_{\mathcal N} x$}}
		  {\binpar{P}{Q} \downarrow_{\mathcal N} x}

We write $P \Downarrow_{\mathcal N} x$ if there is $Q$ such that 
$P \wred Q$ and $Q \downarrow_{\mathcal N} x$.
\end{definition}

\begin{definition}
%\label{def.bbisim}
An  ${\mathcal N}$-\emph{barbed bisimulation} over a set of names, ${\mathcal N}$, is a symmetric binary relation 
${\mathcal S}_{\mathcal N}$ between agents such that $P\rel{S}_{\mathcal N}Q$ implies:
\begin{enumerate}
\item If $P \red P'$ then $Q \wred Q'$ and $P'\rel{S}_{\mathcal N} Q'$.
\item If $P\downarrow_{\mathcal N} x$, then $Q\Downarrow_{\mathcal N} x$.
\end{enumerate}
$P$ is ${\mathcal N}$-barbed bisimilar to $Q$, written
$P \wbbisim_{\mathcal N} Q$, if $P \rel{S}_{\mathcal N} Q$ for some ${\mathcal N}$-barbed bisimulation ${\mathcal S}_{\mathcal N}$.
\end{definition}

$\mathcal{R} \subseteq \pi \times \pi$

$P \mathcal{R} Q => \forall P'. P \red P' \Rightarrow \exists Q'. Q \red Q', P' \mathcal{R} Q'$

$P \vdash x \Rightarrow Q \vdash x$

\begin{mathpar}
  \inferrule*[lab=Out-barb]{x \nameeq y}{{y}!\langle{Q}\rangle \vdash x}
  \and
  \inferrule*[lab=Par-barb]{\mbox{$P\vdash x$ or $Q\vdash x$}}{\binpar{P}{Q} \vdash x}
\end{mathpar}

\subsubsection{Contexts}

One of the principle advantages of computational calculi like the
$\pi$-calculus is a well-defined notion of context,
contextual-equivalence and a correlation between
contextual-equivalence and notions of bisimulation. The notion of
context allows the decomposition of a process into (sub-)process and
its syntactic environment, its context. Thus, a context may be
thought of as a process with a ``hole'' (written $\Box$) in it. The
application of a context $M$ to a process $P$, written $M[P]$, is
tantamount to filling the hole in $M$ with $P$. In this paper we do
not need the full weight of this theory, but do make use of the notion
of context in the proof the main theorem. 

\begin{mathpar}
  \inferrule* [lab=summation] {} {{M_{M},M_{N}} \bc \Box \;|\; x.M_{A} \;|\; M_{M}+M_{N}}
  \and
  \inferrule* [lab=agent] {} {{M_{A}} \bc (\vec{x})M_{P} \;| \; \clift{P_0,\ldots,M_{P},\ldots,P_N}}
  \and \\
  \inferrule* [lab=process] {} {{M_{P}} \bc M_{N} \;| \;P|M_{P} }
\end{mathpar} 

\begin{mathpar}
  \inferrule* [lab=sychronization] {} {M_{N} \bc \Box \;|\; x?M_{F} \;|\; x!M_{C}}
  \and
  \inferrule* [lab=abstraction] {} {{M_{F}} \bc (x)M_{P} }
  \and
  \inferrule* [lab=concretion] {} {{M_{C}} \bc \langle M_{P} \rangle }
  \and \\
  \inferrule* [lab=process] {} {{M_{P}} \bc M_{N} \;| \;P|M_{P} }
\end{mathpar}

\begin{definition}[contextual application] Given a context $M$, and
  process $P$, we define the \emph{contextual application}, $M[P] :=
  M\{P/\Box\}$. That is, the contextual application of M to P is the
  substitution of $P$ for $\Box$ in $M$.
\end{definition}

$\meaningof{-} : L \to \mathcal{P}(\pi)$

\begin{mathpar}
  \inferrule* [lab=collection] {} {\meaningof{true} = \pi, \and \meaningof{~E} = \pi \setminus \meaningof{E}, \and \meaningof{E_{1} \& E_{2}} = \meaningof{E_{1}} \cap \meaningof{E_{2}}}
\end{mathpar}

\begin{mathpar}
  \inferrule* [lab=structure] {} {\meaningof{0} = \{ P \in \pi | P \equiv 0 \}, \and \\ \meaningof{E_1 | E_2} = \{ P \in \pi | P \equiv P_{1} | P_{2}, P_{1} \in \meaningof{E_{1}}, P_{2} \in \meaningof{E_2}\} }
\end{mathpar}

\begin{mathpar}
 \inferrule* [lab=behavior] {} {\meaningof{\langle a?b \rangle E} = \{ P \in \pi | P \equiv Q | u?(y)P', \\ \and \\\\ \and \\ \;\;\; u \in \meaningof{a}, \forall z.P'\{z/y\} \in \meaningof{E\{z/b\}}\}, \and \\ \meaningof{a!E} = \{ P \in \pi | P \equiv Q | x!\langle P' \rangle, x \in \meaningof{a} P' \in \meaningof{E}\} }
\end{mathpar}

\begin{mathpar}
 \inferrule* [lab=nominal] {} {\meaningof{\quotep{E}} = \{ \quotep{P} \in \quotep{\pi} | P \in \meaningof{E} \}, \and \meaningof{\quotep{P}} = \{ \quotep{Q} \in \quotep{\pi} | P \equiv Q \} \and \\ \meaningof{@\quotep{E}} = \{ P \in \pi | P \equiv @x, x \in \meaningof{E} \}}
\end{mathpar}

\begin{eqnarray*}
  \\
  \meaningof{-} : TS \to ST
\end{eqnarray*}

\begin{eqnarray*}
  \\
  L : TS \to ST
\end{eqnarray*}

\begin{eqnarray*}
  \\
  P \models E \iff P \in \meaningof{E}
\end{eqnarray*}

\begin{eqnarray*}
  P \approx_{L} Q \iff \forall E \in L. P \models E \iff Q \models E
\end{eqnarray*}

\begin{eqnarray*}
  P \approx_{K} Q
\end{eqnarray*}

\begin{eqnarray*}
  P \approx Q
\end{eqnarray*}

$\approx_{K} = \approx = \approx_{L}$

\subsubsection{Contextual duality}

Note that contexts extend the quotation operation to a family of
operations from processes to names. Given a context, $M$, we can
define a \emph{nominal context}, $\quotep{M}$ by $\quotep{M}[P] :=
\quotep{M[P]}$. To foreshadow what is to come we observe that these
operations enjoy a duality with processes very much like the duality
between vectors and maps from vectors to scalars.

Further, because the calculus is essentially higher-order, we have a
correspondence between contexts and processes. More specifically,
given a name $x$ and a context $M$ we can construct $M^{*}_{x}$ such
that 

\begin{mathpar}
  M^{*}_{x} | \lift{x}{P} \red M[P]
\end{mathpar}

namely,

\begin{mathpar}
  M^{*}_{x} := x?(u).M[\dropn{u}]
\end{mathpar}

The dependence of $M^{*}_{x}$ on a name makes it an abstraction, 

\begin{mathpar}
  M^{*} := (x)x?(u).M[\dropn{u}]
\end{mathpar}

\subsection{Additional notation}

It will sometimes be convenient to denote the process a name
quotes. We already have the notation $x = \quotep{P}$, but it will be
convenient to introduce an alternate notation, $\procn{x}$, when we
want to emphasize the connection to the use of the name. Note that, by
virtue of name equivalence, $\quotep{\procn{x}} \nameeq x$; so, the
notation is consistent with previous definitions.

Further, because names have structure it is possible to effect
substitutions on the basis of that structure. This means we need to
upgrade our notation for substitutions, which we accomplish by
adapting comprehension notation. Thus,

\begin{mathpar}
  P\{ y / x : x \in S \}
\end{mathpar}

is interpreted to mean the process derived from P by replacing (in a
capture-avoiding manner) each occurrence of $x$ in $S$ by $y$. For example,

\begin{mathpar}
  P\{ \quotep{\procn{x}|\procn{x}} / x : x \in \freenames{P} \}
\end{mathpar}

will replace each (occurrence) of a free name $x$ in $P$ by
$\quotep{\procn{x}|\procn{x}}$.

Also, we will avail ourselves of the notation $x^{L}$ and $x^{R}$ to
denote injections of a name into disjoint copies of the name
space. There are numerous ways to accomplish this. One example can be
found in \cite{MeredithR05}. This notation overloads to vectors of
names: $\vec{x}^{\pi} := (x_{i}^{\pi} \; : \; 0 \leq i < |\vec{x}| )$ where $\pi \in \{L,R\}$.

We also use $P^{\Box} := P|\Box$.

In \cite{MeredithR05} an interpretation of the new operator is
given. It turns out that there are several possible interpretations
all enjoying the requisite algebraic properties of the operator (see
\cite{milner91polyadicpi}). We will therefore make liberal use of
$(\nu\; \vec{x})P$.

% subsection the_syntax_and_semantics_of_the_notation_system (end)   

\input{qm2pi.qmops} 

\input{qm2pi.sterngerlach} 

\input{qm2pi.metric} 

% section concurrent_process_calculi (end)

%\input{qm2pi.proofsketch}

% section proof sketch (end)

%\input{qm2pi.slviaknots} 

% section spatial logic via knots (end)

\input{qm2pi.conclusion}

% section conclusion (end)

%\input{qm2pi.dtcodes} 

% section wiring algorithm (end)

\input{qm2pi.ack} 

% section acknowledgments (end)

\newpage


\bibliographystyle{plain}   
\bibliography{../../biblios/main.bib}

\input{qm2pi.rhodetails}

\end{document}

 

% subsection basic_interpretation (end)

%\input{qm2pi.rho.presentation} 
\subsection{The syntax and semantics of the notation system}\label{sub:the_syntax_and_semantics_of_the_notation_system} % (fold)

We now summarize a technical presentation of the calculus that
embodies our theory of dynamics. The typical presentation of such a
calculus follows the style of giving generators and relations on
them. The grammar, below, describing term constructors, freely
generates the set of processes, $\Proc$. This set is then quotiented
by a relation known as structural congruence and it is over this set
that the notion of dynamics is expressed. This presentation is
essentially that of \cite{MeredithR05} with the addition of
polyadicity and summation. For readability we have relegated some of
the technical subtleties to an appendix.

\subsubsection{Process grammar}\label{subsub:process_grammar}

\begin{mathpar}
  \inferrule* [lab=synchronization] {} {{M} \bc \pzero \;|\; x?F \;|\; x!C }
  \and
  \inferrule* [lab=abstraction] {} {{F} \bc (x)P}
  \and
  \inferrule* [lab=concretion] {} {{C} \bc \langle Q \rangle}
  \and
  \inferrule* [lab=process] {} {{P,Q} \bc M \;| \;P|Q \;|\; @{x}}
  \and
  \inferrule* [lab=name] {} {{x} \bc \quotep{P}}
\end{mathpar} 

Note that $\vec{x}$ (resp. $\vec{P}$) denotes a vector of names
(resp. processes) of length $|\vec{x}|$ (resp. $|\vec{P}|$). We adopt
the following useful abbreviations.

\begin{mathpar}
   x?(\vec{y}).P := x.(\vec{y})P \and  x\clift{\vec{P}} := x.\clift{\vec{P}}
   \and x!(y) := \lift{x}{\dropn{y}}
   \and \Pi_{i=0}^{n-1}P_i := P_0 | \ldots | P_{n-1}
\end{mathpar}

\subsubsection{Structural congruence}

\paragraph{Free and bound names and alpha-equivalence.} At the
core of structural equivalence is alpha-equivalence which identifies
process that are the same up to a change of variable. Formally, we
recognize the distinction between free and bound names. The free names
of a process, $\freenames{P}$, may be calculated recursively as
follows:

\begin{mathpar}
\freenames{\pzero} := \emptyset
  \and \\
  \freenames{x?(y).P} := \{ x \} \cup (\freenames{P} \setminus \{ y \})
  \and 
  \freenames{x!\langle P \rangle} := \{ x \} \cup \{ P \} 
  \and \\
  \freenames{P|Q} := \freenames{P} \cup \freenames{Q}
  \and \\
  \freenames{@{x}} := \{ x \}
\end{mathpar}

$\pi$
$\quotep{\pi}$

$\freenames{-} : \pi \to \mathcal{P}(\quotep{\pi})$

\begin{eqnarray*}
  \freenames{\pzero} & := & \emptyset \\
  \freenames{x?(y).P} & := & \{ x \} \cup (\freenames{P} \setminus \{ y \}) \\
  \freenames{x!\langle P \rangle} & := & \{ x \} \cup \{ P \} \\
  \freenames{P|Q} & := & \freenames{P} \cup \freenames{Q} \\
  \freenames{\dropn{x}} & := & \{ x \}
\end{eqnarray*}

The bound names of a process, $\boundnames{P}$, are those names occurring in $P$
that are not free. For example, in $x?(y).0$, the name $x$ is free, while $y$ is bound.

\begin{mathpar}
  \inferrule* [lab=monoidal-laws] {} { P|Q \equiv Q|P \and P|0 \equiv P \and P|(Q|R) \equiv (P|Q)|R }
\end{mathpar}

\begin{mathpar}
  \inferrule* [lab=alpha-equivalence] {} { (x)P \equiv (y)P\{y/x\} \and y \not\in \freenames{P} }
\end{mathpar}

\begin{definition}
Then two processes, $P,Q$, are alpha-equivalent if $P = Q\{\vec{y}/\vec{x}\}$ for
some $\vec{x} \in \boundnames{Q},\vec{y} \in \boundnames{P}$, where $Q\{\vec{y}/\vec{x}\}$
denotes the capture-avoiding substitution of $\vec{y}$ for $\vec{x}$ in $Q$.
\end{definition}

\begin{definition}
  The {\em structural congruence} \cite{SangiorgiWalker} , $\equiv$,
  between processes is the least congruence containing
  alpha-equivalence, satisfying the abelian monoid laws
  (associativity, commutativity and $\pzero$ as identity) for parallel
  composition $|$ and for summation $+$.
\end{definition}

\subsection{Name equivalence}

We take name equivalence, written $\nameeq$, to be the smallest
equivalence relation generated by the following rules.

\begin{mathpar}
\inferrule*[lab=Quote-drop]
{ }
{ \quotep{@{x}} \nameeq x }

\inferrule*[lab=Struct-equiv]
{ P \scong Q }
{ \quotep{P} \nameeq \quotep{Q} }
\end{mathpar}

The astute reader will have noticed that the mutual recursion of names
and processes imposes a mutual recursion on alpha-equivalence and
structural equivalence via name-equivalence. Fortunately, all of this
works out pleasantly and we may calculate in the natural way, free of
concern. The reader interested in the details is referred to the
appendix \ref{appendix:rho_details}.

\subsection{Substitution}

We use $\Proc$ for the set of processes, $\QProc$ for the set of
names, and $\id{\{}\vec{y} / \vec{x} \id{\}}$ to denote partial maps,
$s : \QProc \rightarrow \QProc$. A map, $s$ lifts, uniquely, to a map
on process terms, $\widehat{s} : \Proc \rightarrow \Proc$ by the
following equations.

\begin{mathpar}
  (0) \psubstp{Q}{P} := 0 \\
  (R \juxtap S) \psubstp{Q}{P}
  :=    
  (R)\psubstp{Q}{P} \juxtap (S) \psubstp{Q}{P} \\
  (x?(y).R) \psubstp{Q}{P}    
  :=    
  (x)\substp{Q}{P} (z)\concat( (R \psubstn{z}{y}) \psubstp{Q}{P} ) \\
  (\lift{x}{R}) \psubstp{Q}{P}  
  :=
  \lift{(x)\substp{Q}{P}}{ R \psubstp{Q}{P} } \\
%   (\dropn{x})  \psubstp{Q}{P}       
%   := 
%   \left\{ 
%     \begin{array}{ccc} 
%       \dropn{\quotep{Q}} & & x \nameeq \quotep{P} \\
%       \dropn{x} & & otherwise \\
%     \end{array}
%   \right. 
  (\dropn{x})  \psubstp{Q}{P}       
  := 
  \left\{ 
    \begin{array}{ccc} 
      Q & & x \nameeq \quotep{P} \\
      \dropn{x} & & otherwise \\
    \end{array}
  \right.
\end{mathpar}
 

where

\begin{eqnarray}
  (x)\id{\{} \lpquote Q \rpquote / \lpquote P \rpquote \id{\}}            = 
  \left\{ 
    \begin{array}{ccc}
      \lpquote Q \rpquote & & x \nameeq \lpquote P \rpquote \\
      x & & otherwise \\
    \end{array}
  \right. \nonumber
\end{eqnarray}

and $z$ is chosen distinct from $\quotep{P}$, $\quotep{Q}$, the free
names in $Q$, and all the names in $R$. Our $\alpha$-equivalence will
be built in the standard way from this substitution.

\begin{remark}\label{rem:no_self_referential_names}
  One consequence of these definitions is that $\forall P. \quotep{P}
  \not\in \freenames{P}$.
\end{remark}

\subsection{ Dynamic quote: an example }

Anticipating something of what's to come, consider applying the
substitution, $\widehat{\id{\{}u / z \id{\}}}$, to the following pair
of processes, $\lift{w}{y!(z)}$ and $w[ \lpquote y!(z) \rpquote ]$.

\begin{eqnarray}
	\lift{w}{y!(z)}\widehat{\id{\{}u / z \id{\}}}
		& = &
		\lift{w}{y!(u)} \nonumber\\
	w[ \lpquote y!(z) \rpquote ] \widehat{ \id{\{}u / z \id{\}} }
		& = &
		w[ \lpquote y!(z) \rpquote ] \nonumber
\end{eqnarray}

Because the body of the process between quotes is impervious to
substitution, we get radically different answers. In fact, by
examining the first process in an input context,
e.g. $x?(z).\lift{w}{y!(z)}$, we see that the process under the lift
operator may be shaped by prefixed inputs binding a name inside it. In
this sense, the lift operator will be seen as a way to dynamically
construct processes before reifying them as names.

Finally equipped with these standard features we can present the
dynamics of the calculus.

\subsubsection{Operational semantics} 

Finally, we introduce the computational dynamics. What marks these
algebras as distinct from other more traditionally studied algebraic
structures, e.g. vector spaces or polynomial rings, is the manner in
which dynamics is captured. In traditional structures, dynamics is typically
expressed through morphisms between such structures, as in linear maps
between vector spaces or morphisms between rings. In algebras
associated with the semantics of computation, the dynamics is
expressed as part of the algebraic structure itself, through a
reduction reduction relation typically denoted by $\red$. Below, we
give a recursive presentation of this relation for the calculus used
in the encoding.

$\red \subseteq \pi \times \pi$
$\red : \pi \to \mathcal{P}(\pi)$

\begin{mathpar}
  \inferrule* [lab=Comm] { \textsf{match}( x_{src}, x_{trgt} ) } { x_{trgt}?(y)P \; | \; x_{src}!\langle {Q} \rangle \red P\{\quotep{Q}/y}\} }
  \and \\
  \inferrule* [lab=Par] {{P} \red {P}'} {{{P} | {Q}} \red {{P}' | {Q}}}
  \and
  \inferrule* [lab=Equiv]{{{P} \scong {P}'} \andalso {{P}' \red {Q}'} \andalso {{Q}' \scong {Q}}}{{P} \red {Q}}
\end{mathpar}

\begin{eqnarray*}
  match_{\equiv} (\quotep{P},\quotep{Q}) & := & P \equiv Q \\
  match_{\dagger}(\quotep{P},\quotep{Q}) & := & \forall R. P|Q \red^{*} R => R \red^{*} 0 \\
  match_{K}(\quotep{P},\quotep{Q}) & := & K \mbox{ for some context } K
\end{eqnarray*}

$u?(x)P | u!\langle Q \rangle \red P\{\quotep{Q}/x\}$

%We write $\wred$ for $\red^*$, and $P\red$ if $\exists Q $ such that $ P \red Q$.
We write $P\red$ if $\exists Q $ such that $ P \red Q$ and $P\not\red$, otherwise.

\section{Replication}

As mentioned before, it is known that replication (and hence
recursion) can be implemented in a higher-order process algebra
\cite{SangiorgiWalker}. As our first example of calculation with the
machinery thus far presented we give the construction explicitly in
the {\rhoc}.

\begin{eqnarray}
	D_{x} & := & \prefix{x}{y}{(\binpar{\outputp{x}{y}}{@{y}})} \nonumber\\
	\bangp_{x}{P} & := & \binpar{{x}!\langle{\binpar{D_{x}}{P}}\rangle}{D_{x}} \nonumber
\end{eqnarray}

\begin{eqnarray}
	\bangp_{x}{P} & & \nonumber\\
	=
	& {x}!\langle{(\prefix{x}{y}{(\outputp{x}{y} | @{y})) | P}}\rangle 
	      | \prefix{x}{y}{(\outputp{x}{y} | @{y})} & \nonumber\\
	\red
	& (\outputp{x}{y} | @{y})\substn{\quotep{(\prefix{x}{y}{(@{y} | \outputp{x}{y})) | P}}}{y} & \nonumber\\
	=
	& \outputp{x}{\quotep{(\prefix{x}{y}{(\outputp{x}{y} | @{y})) | P}}}
	  | {(\prefix{x}{y}{(\outputp{x}{y} | @{y})) | P}} & \nonumber\\
	\red
	& \ldots & \nonumber\\
	\red^*
	& P | P | \ldots & \nonumber
\end{eqnarray}

Of course, this encoding, as an implementation, runs away, unfolding
$\bangp{P}$ eagerly. A lazier and more implementable replication
operator, restricted to input-guarded processes, may be obtained as follows.

\begin{eqnarray}
\bangp{\prefix{u}{v}{P}} 
	:= 
	\binpar{\lift{x}{\prefix{u}{v}{(\binpar{D(x)}{P})}}}{D(x)} \nonumber
\end{eqnarray}

\begin{remark}
  Note that the lazier definition still does not deal with summation
  or mixed summation (i.e. sums over input and output). The reader is
  invited to construct definitions of replication that deal with these
  features. 

  Further, the definitions are parameterized in a name, $x$. Can you,
  gentle reader, make a definition that eliminates this parameter and
  guarantees no accidental interaction between the replication
  machinery and the process being replicated -- i.e. no accidental
  sharing of names used by the process to get its work done and the
  name(s) used by the replication to effect copying. This latter
  revision of the definition of replication is crucial to obtaining
  the expected identity $!!P \sim !P$.
\end{remark}

\begin{remark}\label{rem:paradoxical_combinator}
  The reader familiar with the lambda calculus will have noticed the
  similarity between $D$ and the paradoxical combinator.

  [Ed. note: the existence of this seems to suggest we have to be more
  restrictive on the set of processes and names we admit if we are to
  support no-cloning.]
\end{remark}

\subsubsection{Bisimulation}

The computational dynamics gives rise to another kind of equivalence,
the equivalence of computational behavior. As previously mentioned
this is typically captured \emph{via} some form of bisimulation.

% The notion we use in this paper is weak barbed bisimulation
% \cite{milner91polyadicpi}.

The notion we use in this paper is derived from weak barbed
bisimulation \cite{milner91polyadicpi}. 

\begin{definition}
An \emph{observation relation}, $\downarrow_{\mathcal N}$, over a set
of names, $\mathcal N$, is the smallest relation satisfying the rules
below.

\infrule[Out-barb]{y \in {\mathcal N}, \; x \nameeq y}
		  {\outputp{x}{v} \downarrow_{\mathcal N} x}
\infrule[Par-barb]{\mbox{$P\downarrow_{\mathcal N} x$ or $Q\downarrow_{\mathcal N} x$}}
		  {\binpar{P}{Q} \downarrow_{\mathcal N} x}

We write $P \Downarrow_{\mathcal N} x$ if there is $Q$ such that 
$P \wred Q$ and $Q \downarrow_{\mathcal N} x$.
\end{definition}

\begin{definition}
%\label{def.bbisim}
An  ${\mathcal N}$-\emph{barbed bisimulation} over a set of names, ${\mathcal N}$, is a symmetric binary relation 
${\mathcal S}_{\mathcal N}$ between agents such that $P\rel{S}_{\mathcal N}Q$ implies:
\begin{enumerate}
\item If $P \red P'$ then $Q \wred Q'$ and $P'\rel{S}_{\mathcal N} Q'$.
\item If $P\downarrow_{\mathcal N} x$, then $Q\Downarrow_{\mathcal N} x$.
\end{enumerate}
$P$ is ${\mathcal N}$-barbed bisimilar to $Q$, written
$P \wbbisim_{\mathcal N} Q$, if $P \rel{S}_{\mathcal N} Q$ for some ${\mathcal N}$-barbed bisimulation ${\mathcal S}_{\mathcal N}$.
\end{definition}

$\mathcal{R} \subseteq \pi \times \pi$

$P \mathcal{R} Q => \forall P'. P \red P' \Rightarrow \exists Q'. Q \red Q', P' \mathcal{R} Q'$

$P \vdash x \Rightarrow Q \vdash x$

\begin{mathpar}
  \inferrule*[lab=Out-barb]{x \nameeq y}{{y}!\langle{Q}\rangle \vdash x}
  \and
  \inferrule*[lab=Par-barb]{\mbox{$P\vdash x$ or $Q\vdash x$}}{\binpar{P}{Q} \vdash x}
\end{mathpar}

\subsubsection{Contexts}

One of the principle advantages of computational calculi like the
$\pi$-calculus is a well-defined notion of context,
contextual-equivalence and a correlation between
contextual-equivalence and notions of bisimulation. The notion of
context allows the decomposition of a process into (sub-)process and
its syntactic environment, its context. Thus, a context may be
thought of as a process with a ``hole'' (written $\Box$) in it. The
application of a context $M$ to a process $P$, written $M[P]$, is
tantamount to filling the hole in $M$ with $P$. In this paper we do
not need the full weight of this theory, but do make use of the notion
of context in the proof the main theorem. 

\begin{mathpar}
  \inferrule* [lab=summation] {} {{M_{M},M_{N}} \bc \Box \;|\; x.M_{A} \;|\; M_{M}+M_{N}}
  \and
  \inferrule* [lab=agent] {} {{M_{A}} \bc (\vec{x})M_{P} \;| \; \clift{P_0,\ldots,M_{P},\ldots,P_N}}
  \and \\
  \inferrule* [lab=process] {} {{M_{P}} \bc M_{N} \;| \;P|M_{P} }
\end{mathpar} 

\begin{mathpar}
  \inferrule* [lab=sychronization] {} {M_{N} \bc \Box \;|\; x?M_{F} \;|\; x!M_{C}}
  \and
  \inferrule* [lab=abstraction] {} {{M_{F}} \bc (x)M_{P} }
  \and
  \inferrule* [lab=concretion] {} {{M_{C}} \bc \langle M_{P} \rangle }
  \and \\
  \inferrule* [lab=process] {} {{M_{P}} \bc M_{N} \;| \;P|M_{P} }
\end{mathpar}

\begin{definition}[contextual application] Given a context $M$, and
  process $P$, we define the \emph{contextual application}, $M[P] :=
  M\{P/\Box\}$. That is, the contextual application of M to P is the
  substitution of $P$ for $\Box$ in $M$.
\end{definition}

$\meaningof{-} : L \to \mathcal{P}(\pi)$

\begin{mathpar}
  \inferrule* [lab=collection] {} {\meaningof{true} = \pi, \and \meaningof{~E} = \pi \setminus \meaningof{E}, \and \meaningof{E_{1} \& E_{2}} = \meaningof{E_{1}} \cap \meaningof{E_{2}}}
\end{mathpar}

\begin{mathpar}
  \inferrule* [lab=structure] {} {\meaningof{0} = \{ P \in \pi | P \equiv 0 \}, \and \\ \meaningof{E_1 | E_2} = \{ P \in \pi | P \equiv P_{1} | P_{2}, P_{1} \in \meaningof{E_{1}}, P_{2} \in \meaningof{E_2}\} }
\end{mathpar}

\begin{mathpar}
 \inferrule* [lab=behavior] {} {\meaningof{\langle a?b \rangle E} = \{ P \in \pi | P \equiv Q | u?(y)P', \\ \and \\\\ \and \\ \;\;\; u \in \meaningof{a}, \forall z.P'\{z/y\} \in \meaningof{E\{z/b\}}\}, \and \\ \meaningof{a!E} = \{ P \in \pi | P \equiv Q | x!\langle P' \rangle, x \in \meaningof{a} P' \in \meaningof{E}\} }
\end{mathpar}

\begin{mathpar}
 \inferrule* [lab=nominal] {} {\meaningof{\quotep{E}} = \{ \quotep{P} \in \quotep{\pi} | P \in \meaningof{E} \}, \and \meaningof{\quotep{P}} = \{ \quotep{Q} \in \quotep{\pi} | P \equiv Q \} \and \\ \meaningof{@\quotep{E}} = \{ P \in \pi | P \equiv @x, x \in \meaningof{E} \}}
\end{mathpar}

\begin{eqnarray*}
  \\
  \meaningof{-} : TS \to ST
\end{eqnarray*}

\begin{eqnarray*}
  \\
  L : TS \to ST
\end{eqnarray*}

\begin{eqnarray*}
  \\
  P \models E \iff P \in \meaningof{E}
\end{eqnarray*}

\begin{eqnarray*}
  P \approx_{L} Q \iff \forall E \in L. P \models E \iff Q \models E
\end{eqnarray*}

\begin{eqnarray*}
  P \approx_{K} Q
\end{eqnarray*}

\begin{eqnarray*}
  P \approx Q
\end{eqnarray*}

$\approx_{K} = \approx = \approx_{L}$

\subsubsection{Contextual duality}

Note that contexts extend the quotation operation to a family of
operations from processes to names. Given a context, $M$, we can
define a \emph{nominal context}, $\quotep{M}$ by $\quotep{M}[P] :=
\quotep{M[P]}$. To foreshadow what is to come we observe that these
operations enjoy a duality with processes very much like the duality
between vectors and maps from vectors to scalars.

Further, because the calculus is essentially higher-order, we have a
correspondence between contexts and processes. More specifically,
given a name $x$ and a context $M$ we can construct $M^{*}_{x}$ such
that 

\begin{mathpar}
  M^{*}_{x} | \lift{x}{P} \red M[P]
\end{mathpar}

namely,

\begin{mathpar}
  M^{*}_{x} := x?(u).M[\dropn{u}]
\end{mathpar}

The dependence of $M^{*}_{x}$ on a name makes it an abstraction, 

\begin{mathpar}
  M^{*} := (x)x?(u).M[\dropn{u}]
\end{mathpar}

\subsection{Additional notation}

It will sometimes be convenient to denote the process a name
quotes. We already have the notation $x = \quotep{P}$, but it will be
convenient to introduce an alternate notation, $\procn{x}$, when we
want to emphasize the connection to the use of the name. Note that, by
virtue of name equivalence, $\quotep{\procn{x}} \nameeq x$; so, the
notation is consistent with previous definitions.

Further, because names have structure it is possible to effect
substitutions on the basis of that structure. This means we need to
upgrade our notation for substitutions, which we accomplish by
adapting comprehension notation. Thus,

\begin{mathpar}
  P\{ y / x : x \in S \}
\end{mathpar}

is interpreted to mean the process derived from P by replacing (in a
capture-avoiding manner) each occurrence of $x$ in $S$ by $y$. For example,

\begin{mathpar}
  P\{ \quotep{\procn{x}|\procn{x}} / x : x \in \freenames{P} \}
\end{mathpar}

will replace each (occurrence) of a free name $x$ in $P$ by
$\quotep{\procn{x}|\procn{x}}$.

Also, we will avail ourselves of the notation $x^{L}$ and $x^{R}$ to
denote injections of a name into disjoint copies of the name
space. There are numerous ways to accomplish this. One example can be
found in \cite{MeredithR05}. This notation overloads to vectors of
names: $\vec{x}^{\pi} := (x_{i}^{\pi} \; : \; 0 \leq i < |\vec{x}| )$ where $\pi \in \{L,R\}$.

We also use $P^{\Box} := P|\Box$.

In \cite{MeredithR05} an interpretation of the new operator is
given. It turns out that there are several possible interpretations
all enjoying the requisite algebraic properties of the operator (see
\cite{milner91polyadicpi}). We will therefore make liberal use of
$(\nu\; \vec{x})P$.

% subsection the_syntax_and_semantics_of_the_notation_system (end)   

\section{Interpretation of QM}
\subsection{Supporting definitions}
\subsubsection{Multiplication}
\begin{mathpar}
  \quotep{Q} \cdot \quotep{R} := \quotep{Q|R}
  \and \\
  \quotep{Q} \cdot P := P\{ \quotep{Q|R} / \quotep{R} : \quotep{R} \in \freenames{P} \}
\end{mathpar}

\paragraph{Discussion}
The first line needs little explanation. The second line says that
each free name of the process is replaced with the multiplication of
that name by the scalar. Multiplication of a scalar (name) by a state
(process) results in a process all the names of which have been `moved
over' by parallel composition with the process the scalar
quotes. There is a subtlety that the bound names have to be
manipulated so that multiplied names aren't accidentally
captured. There are many ways to achieve this.

\begin{remark}\label{rem:multiplication_identities}
  The reader is invited to verify that for all $x,y,z \in \QProc$ and $P \in \Proc$
  \begin{mathpar}
    x \cdot \quotep{0} \equiv x 
    \and
    x \cdot y \equiv y \cdot x
    \and
    x \cdot (y \cdot z) \equiv (x \cdot y) \cdot z
    \and \\
    \quotep{0} \cdot P \equiv P
    \and \\
    x \cdot (y \cdot P) \equiv (x \cdot y) \cdot P
    \and \\
    x \cdot (P|Q) \equiv (x \cdot P) | (x \cdot Q)
    \and \\    
  \end{mathpar}
\end{remark}

\subsubsection{Tensor product}

We define a tensor product on processes by structural induction.

\paragraph{Tensor of sums} First note that all summations, including
$\pzero$ and sequence, can be written $\Sigma_{i} x_{i}.A_{i} +
\Sigma_{j} x_{j}.C_{j}$, where we have grouped input-guarded processes
together and output-guarded processes together.

Thus, we can define the tensor product of two summations, $N_{1}\otimes N_{2}$, where

\begin{mathpar}
  N_{1} := \Sigma_{i} x_{i}.A_{i} + \Sigma_{j} x_{j}.C_{j}
  \and
  N_{2} := \Sigma_{i'} y_{i'}.B_{i'} + \Sigma_{j'} y_{j'}.D_{j'} 
\end{mathpar}

as follows.

\begin{mathpar}
  \Sigma_{i} x_{i}.A_{i} + \Sigma_{j} x_{j}.C_{j} \otimes \Sigma_{i'}
  y_{i'}.B_{i'} + \Sigma_{j'} y_{j'}.D_{j'} 
  \and \\
  := \; \Sigma_{i} \Sigma_{i'} \quotep{\stackrel{\vee}{x_{i}}| \stackrel{\vee}{y_{i'}}}.(A_{i}\otimes B_{i'}) \; | \; \Sigma_{i'} \Sigma_{i} \quotep{\stackrel{\vee}{y_{i'}}|\stackrel{\vee}{x_{i}}}.(B_{i'}\otimes A_{i})
  \and
  \;\; | \;\; \Sigma_{j} \Sigma_{j'} \quotep{\stackrel{\vee}{x_{j}}|\stackrel{\vee}{y_{j'}}}.(A_{j}\otimes B_{j'}) \; | \; \Sigma_{j'} \Sigma_{j} \quotep{\stackrel{\vee}{y_{j'}}|\stackrel{\vee}{x_{j}}}.(B_{j'}\otimes A_{j})
\end{mathpar}

\begin{remark}
  Do we need to $x^{L}$ and $y^{R}$ for this construction as well?
\end{remark}

\paragraph{Tensor of parallel compositions} Next, we distribute tensor
over par.

\begin{mathpar}
  P_{1}|P_{2} \otimes Q_{1}|Q_{2} := (P_{1} \otimes Q_{1}) | (P_{1}
  \otimes Q_{2}) | (P_{2} \otimes Q_{1}) | (P_{2} \otimes Q_{2})
\end{mathpar}

\paragraph{Tensor with dropped names} We treat tensor of a
process with a dropped name as parallel composition.

\begin{mathpar}
  P \otimes \dropn{x} := P | \dropn{x}
\end{mathpar}

\paragraph{Tensor of agents}

Finally, we need to define tensor on agents. Note that the definition
of tensor on normal products only tensors inputs with inputs and
outputs with outputs. Thus, we only have to define the operation on
``homogeneous'' pairings.

\begin{mathpar}
  (\vec{x})P \otimes (\vec{y})Q
  \and \\
  := (x_{0}^{L}|y_{0}^{R},\ldots,x_{0}^{L}|y_{n}^{R},\ldots,x_{m}^{L}|y_{0}^{R},\ldots,x_{m}^{L}|y_{n}^R)(P\{ \vec{x}^{L}/\vec{x}\} \otimes Q \{ \vec{y}^{R}/\vec{y}\})
  \and \\
  \clift{\vec{P}} \otimes \clift{\vec{Q}}
  \and \\
  := \clift{P_{0}\otimes Q_{0},\ldots,P_{0}\otimes Q_{n},\ldots,P_{m}\otimes Q_{0},\ldots,P_{m}\otimes Q_{n}}
\end{mathpar}

\begin{remark}
  Observe that arities of tensored abstractions matches arities of
  tensored concretions if the original arities matched. Note also that
  the length of the arities corresponds to the increase in dimension
  we see in ordinary vector space tensor product.
\end{remark}

\begin{remark}
  Operationally, this definition distributes the tensor down to
  components ``linked'' by summation. Tensor over summation is
  intriguing in that it mixes names. Moreover, as a consequence of the
  way it mixes names we have the identities for all $x \in \QProc$ and
  $P,Q \in \Proc$

  \begin{mathpar}
    (x \cdot P) \otimes Q \equiv x \cdot (P \otimes Q) \equiv P \otimes (x \cdot Q)
    \and
    P \otimes \pzero \equiv P
  \end{mathpar}

  that the reader is invited to verify.
\end{remark}

\subsubsection{Annihilation}
\begin{mathpar}
  P^{\perp} := \{ Q | \forall R. P|Q \red^{*} R \Rightarrow R \red^{*} \pzero \}
  \and \\
  P^{\underline{\perp}} := \Sigma_{Q \in P^{\perp}} \quotep{Q}?(y).(\dropn{y}|Q) | \Sigma_{Q \in P^{\perp}} \quotep{Q}\clift{\Box}
\end{mathpar}

\paragraph{Discussion} The reader will note that $P^{\perp}$ is a
\emph{set} of processes, while $P^{\underline{\perp}}$ is a
\emph{context}. We call the set $P^{\perp}$ the \emph{annihilators} of
$P$. The parallel composition of a process in the annihilators of $P$
with $P$ will result in a process, the state space of which has all
paths eventually leading to $\pzero$. Execution may endure loops; but
under reasonable conditions of fairness (naturally guaranteed under
most notions of bisimulation) such a composite process cannot get
stuck in such a loop and will, eventually pop out and terminate.

The context $P^{\underline{\perp}}$ is ready and willing to ``take the
$P$ out of'' the process to which it is applied. It will effectively
transmit the code of the process to which it is applied to one of the
annihilators and run the process against it.

\subsubsection{Evaluation}
We fix $M$ a domain of fully abstract interpretation with an equality
coincident with bisimulation. We take $\meaningof{\cdot} : \Proc \to
M$ to be the map interpreting processes and $\nmeaningof{\cdot} : \M
\to Proc$ to be the map running the other way. Then we define

\begin{mathpar}
  \int P := \nmeaningof{\meaningof{P}}
\end{mathpar}

\paragraph{Discussion}
There are many fully abstract interpretations of Milner's
$\pi$-calculus. Any of them can be used as a basis for interpreting
the reflective calculus here. Equipped with such a domain it is
largely a matter of grinding through to check that the Yoneda
construction for the normalization-by-evaluation program can be
extended to this setting.

\begin{remark}
  The reader is invited to verify that $\int (P^{\underline{\perp}}[P]) = 0$.
\end{remark}

\subsection{Quantum mechanics}

Table \ref{tbl:core_qm_op_defns} gives the core operational definitions

\begin{table}[htp]\label{tbl:core_qm_op_defns}
  \center{
    \fbox{
      \begin{tabular}{c|c}
        quantum mechanics & process calculus \\
        \hline
        scalar & $x := \quotep{P}$ \\
        state vector & $\state{P} := P$ \\
        dual & $\state{P}^{*} := \event{P^{\underline{\perp}}} := \quotep{P^{\underline{\perp}}}[-]$ \\
        matrix & $ \Sigma_{\alpha} \state{P_{\alpha}}x_{\alpha}\event{Q_{\alpha}}$ \\
        vector addition & $\state{P} + \state{Q} := \state{P | Q}$ \\
        tensor product & $\state{P} \otimes \state{Q} := \state{P \otimes Q}$ \\
        inner product & $\innerprod{P}{Q} := \quotep{\int P^{\underline{\perp}}[Q]}$ \\
      \end{tabular}
    }
  }
  \caption{QM - operational definitions}
\end{table}

where

\begin{mathpar}
  \prmatrix{P}{Q} := \fprmatrix{P}{\quotep{\pzero}}{Q}
  \and
  \fprmatrix{P}{x}{Q} := (\state{P},x,\event{Q})
  \and
  (\fprmatrix{P}{x}{Q})(\state{R}) := x \cdot \innerprod{Q}{R} \cdot \state{P}
  \and
  (\fprmatrix{P}{x}{Q})(\event{R}) := x \cdot \innerprod{R}{P} \cdot \event{Q}
\end{mathpar}

\paragraph{Discussion}
As promised: vectors (aka states) are represented as processes; duals
as contextual duals; inner product definition should be compared with
standard inner product definition for ....

\begin{remark}
  Assuming $\int (P^{\underline{\perp}}[P]) = 0$, the reader is
  invited to verify that $(\fprmatrix{P}{x}{P})(\state{P}) = x \cdot \state{P}$.
\end{remark}

\begin{remark}
  The reader is invited to verify that $\innerprod{P}{Q}$ could
  equally well have been written $\quotep{\int \stackrel{\vee}{x}}$
  where $x = \event{P^{\underline{\perp}}}(Q)$.

  One of the motivations for this remark is that there is another way
  to factor these operations. We could package up evaluation in the dual:

  \begin{mathpar}
    \state{P}^{*} := \event{\int P^{\underline{\perp}}} := \quotep{\int P^{\underline{\perp}}}[-]
  \end{mathpar}

  and then have inner product defined by
  
  \begin{mathpar}
    \innerprod{P}{Q} := \event{P}(Q)
  \end{mathpar}

  Hopefully, experience with the calculations will provide guidance on
  the best factoring.
\end{remark}

\begin{remark}
  Assuming $\int (P^{\underline{\perp}}[P]) = 0$, the reader is
  invited to verify that $\forall P,Q. (\prmatrix{0}{Q})(\state{0}) =
  \state{0}$ and dually $(\prmatrix{P}{0})(\event{0}) = \event{0}$.
\end{remark}

\begin{remark}
  i'm a little worried that i don't (yet) have proper support for
  complex conjugacy. But, the observation above may give us a
  clue. According to Abramsky, it must be the case that the scalars
  are iso to the homset of the identity for the tensor -- which the
  observation above characterizes. 

  For now, we will simply bookmark the notion with $\overline{x}$.
\end{remark}

\subsubsection{Adjointness}

We need to give a definition of $(\cdot)^{\dagger}$ for matrices. The
obvious candidate definition is
\begin{mathpar}
(\Sigma_{\alpha}\fprmatrix{P_{\alpha}}{x_{\alpha}}{Q_{\alpha}})^{\dagger}
= \Sigma_{\alpha}\fprmatrix{(Q_{\alpha}^{\underline{\perp}})^{*}}{\overline{x}_{\alpha}}{P_{\alpha}^{\underline{\perp}}} 
\end{mathpar}

But, $(Q_{\alpha}^{\underline{\perp}})^{*}$ requires a name along
which to communicate the process to achieve the context application.

\subsubsection{Basis for a basis}
If processes label states and ``addition'' of states (a.k.a. vector
addition) is interpreted as parallel composition, what corresponds to
notions of linear independence and basis? Here, we recall that Yoshida
has developed a set of \emph{combinators} for an asynchronous verison
of Milner's $\pi$-calculus. These are a finite set of processes such
any process can be expressed as parallel composition of these
combinators together with liberal uses of the new operator and
replication. We can simply give a translation of these into the
present calculus and have reasonable expectation that the property
carries over. That is, that the resultant set allows to express all
processes via parallel composition. Note, however, that there is no
new operator or replication in this calculus. As a result, we expect
that the corresponding set is actually infinite. That is, we expect
that the space is actually infinite dimensional.

\begin{remark}
  The attentive reader may be a bit concerned. Certainly, the
  collection $S$, $K$ and $I$ is a finite set of
  combinators. Shouldn't we expect to see a finite set of combinators
  for an effectively equivalent system? i am very sympathetic to this
  critique and feel it warrants full attention. On the other hand, i
  also have in mind the following analogy. The natural numbers, as a
  monoid under addition, has exactly $1$ generator, while the natural
  numbers, as a monoid under multiplication, has countably many
  generators (the primes). We observe that the application of the
  lambda calculus is much less resource sensitive than the parallel
  composition of the $\pi$-calculus. Could it be the case that we have
  an analogy of the form
  
  \begin{mathpar}
    m + n : MN :: m*n : M|N
  \end{mathpar}

  giving a similar blow up in the set of ``primes''?  This is such a
  wonderful thought that, even if it's not true, i think it's worth
  writing down.
\end{remark}
 

\documentclass[12pt]{llncs}
%\documentclass{jktr}

\usepackage[pdftex]{hyperref}                   
\usepackage {listings}
\usepackage {mathpartir}
\usepackage{bcprules}
%\usepackage{listings}
                       
\usepackage{graphicx} 
%\usepackage[margins=2.5cm,nohead,nofoot]{geometry}
%\usepackage{geometry}
\usepackage{amsfonts}
\usepackage{amstext}
\usepackage{latexsym}
\usepackage{amssymb}
\usepackage{color}


%\include{myPreamble}
\include{qm2pi.local} 

%\ifpdf
%\usepackage[pdftex]{graphicx}
%\else
%\usepackage{graphicx}
%\fi

 % \ifpdf
%  \usepackage{pdfsync}
%  \if


%\title{Brief Article}
%\author{David F. Snyder}
%\author{L.G. Meredith}

%\address{Dept. of Math., Texas State University--San Marcos, San Marcos, TX 78666}
       
\pagestyle{empty}


\begin{document}

\lstset{language=[Objective]Caml,frame=shadowbox}

\input{qm2pi.front}

% section front matter (end)

\input{qm2pi.intro} 
 
% section introduction (end)

% \input{qm2pi.knotations} 

% section notation (end)

\input{qm2pi.process.calculi} 

% section concurrent_process_calculi_and_spatial_logics_ (end)
    
%\input{qm2pi.knots2pi} 

%\input{qm2pi.trefoil} 

%\input{qm2pi.mainthm} 

% subsection basic_interpretation (end)

%\input{qm2pi.rho.presentation} 
\subsection{The syntax and semantics of the notation system}\label{sub:the_syntax_and_semantics_of_the_notation_system} % (fold)

We now summarize a technical presentation of the calculus that
embodies our theory of dynamics. The typical presentation of such a
calculus follows the style of giving generators and relations on
them. The grammar, below, describing term constructors, freely
generates the set of processes, $\Proc$. This set is then quotiented
by a relation known as structural congruence and it is over this set
that the notion of dynamics is expressed. This presentation is
essentially that of \cite{MeredithR05} with the addition of
polyadicity and summation. For readability we have relegated some of
the technical subtleties to an appendix.

\subsubsection{Process grammar}\label{subsub:process_grammar}

\begin{mathpar}
  \inferrule* [lab=synchronization] {} {{M} \bc \pzero \;|\; x?F \;|\; x!C }
  \and
  \inferrule* [lab=abstraction] {} {{F} \bc (x)P}
  \and
  \inferrule* [lab=concretion] {} {{C} \bc \langle Q \rangle}
  \and
  \inferrule* [lab=process] {} {{P,Q} \bc M \;| \;P|Q \;|\; @{x}}
  \and
  \inferrule* [lab=name] {} {{x} \bc \quotep{P}}
\end{mathpar} 

Note that $\vec{x}$ (resp. $\vec{P}$) denotes a vector of names
(resp. processes) of length $|\vec{x}|$ (resp. $|\vec{P}|$). We adopt
the following useful abbreviations.

\begin{mathpar}
   x?(\vec{y}).P := x.(\vec{y})P \and  x\clift{\vec{P}} := x.\clift{\vec{P}}
   \and x!(y) := \lift{x}{\dropn{y}}
   \and \Pi_{i=0}^{n-1}P_i := P_0 | \ldots | P_{n-1}
\end{mathpar}

\subsubsection{Structural congruence}

\paragraph{Free and bound names and alpha-equivalence.} At the
core of structural equivalence is alpha-equivalence which identifies
process that are the same up to a change of variable. Formally, we
recognize the distinction between free and bound names. The free names
of a process, $\freenames{P}$, may be calculated recursively as
follows:

\begin{mathpar}
\freenames{\pzero} := \emptyset
  \and \\
  \freenames{x?(y).P} := \{ x \} \cup (\freenames{P} \setminus \{ y \})
  \and 
  \freenames{x!\langle P \rangle} := \{ x \} \cup \{ P \} 
  \and \\
  \freenames{P|Q} := \freenames{P} \cup \freenames{Q}
  \and \\
  \freenames{@{x}} := \{ x \}
\end{mathpar}

$\pi$
$\quotep{\pi}$

$\freenames{-} : \pi \to \mathcal{P}(\quotep{\pi})$

\begin{eqnarray*}
  \freenames{\pzero} & := & \emptyset \\
  \freenames{x?(y).P} & := & \{ x \} \cup (\freenames{P} \setminus \{ y \}) \\
  \freenames{x!\langle P \rangle} & := & \{ x \} \cup \{ P \} \\
  \freenames{P|Q} & := & \freenames{P} \cup \freenames{Q} \\
  \freenames{\dropn{x}} & := & \{ x \}
\end{eqnarray*}

The bound names of a process, $\boundnames{P}$, are those names occurring in $P$
that are not free. For example, in $x?(y).0$, the name $x$ is free, while $y$ is bound.

\begin{mathpar}
  \inferrule* [lab=monoidal-laws] {} { P|Q \equiv Q|P \and P|0 \equiv P \and P|(Q|R) \equiv (P|Q)|R }
\end{mathpar}

\begin{mathpar}
  \inferrule* [lab=alpha-equivalence] {} { (x)P \equiv (y)P\{y/x\} \and y \not\in \freenames{P} }
\end{mathpar}

\begin{definition}
Then two processes, $P,Q$, are alpha-equivalent if $P = Q\{\vec{y}/\vec{x}\}$ for
some $\vec{x} \in \boundnames{Q},\vec{y} \in \boundnames{P}$, where $Q\{\vec{y}/\vec{x}\}$
denotes the capture-avoiding substitution of $\vec{y}$ for $\vec{x}$ in $Q$.
\end{definition}

\begin{definition}
  The {\em structural congruence} \cite{SangiorgiWalker} , $\equiv$,
  between processes is the least congruence containing
  alpha-equivalence, satisfying the abelian monoid laws
  (associativity, commutativity and $\pzero$ as identity) for parallel
  composition $|$ and for summation $+$.
\end{definition}

\subsection{Name equivalence}

We take name equivalence, written $\nameeq$, to be the smallest
equivalence relation generated by the following rules.

\begin{mathpar}
\inferrule*[lab=Quote-drop]
{ }
{ \quotep{@{x}} \nameeq x }

\inferrule*[lab=Struct-equiv]
{ P \scong Q }
{ \quotep{P} \nameeq \quotep{Q} }
\end{mathpar}

The astute reader will have noticed that the mutual recursion of names
and processes imposes a mutual recursion on alpha-equivalence and
structural equivalence via name-equivalence. Fortunately, all of this
works out pleasantly and we may calculate in the natural way, free of
concern. The reader interested in the details is referred to the
appendix \ref{appendix:rho_details}.

\subsection{Substitution}

We use $\Proc$ for the set of processes, $\QProc$ for the set of
names, and $\id{\{}\vec{y} / \vec{x} \id{\}}$ to denote partial maps,
$s : \QProc \rightarrow \QProc$. A map, $s$ lifts, uniquely, to a map
on process terms, $\widehat{s} : \Proc \rightarrow \Proc$ by the
following equations.

\begin{mathpar}
  (0) \psubstp{Q}{P} := 0 \\
  (R \juxtap S) \psubstp{Q}{P}
  :=    
  (R)\psubstp{Q}{P} \juxtap (S) \psubstp{Q}{P} \\
  (x?(y).R) \psubstp{Q}{P}    
  :=    
  (x)\substp{Q}{P} (z)\concat( (R \psubstn{z}{y}) \psubstp{Q}{P} ) \\
  (\lift{x}{R}) \psubstp{Q}{P}  
  :=
  \lift{(x)\substp{Q}{P}}{ R \psubstp{Q}{P} } \\
%   (\dropn{x})  \psubstp{Q}{P}       
%   := 
%   \left\{ 
%     \begin{array}{ccc} 
%       \dropn{\quotep{Q}} & & x \nameeq \quotep{P} \\
%       \dropn{x} & & otherwise \\
%     \end{array}
%   \right. 
  (\dropn{x})  \psubstp{Q}{P}       
  := 
  \left\{ 
    \begin{array}{ccc} 
      Q & & x \nameeq \quotep{P} \\
      \dropn{x} & & otherwise \\
    \end{array}
  \right.
\end{mathpar}
 

where

\begin{eqnarray}
  (x)\id{\{} \lpquote Q \rpquote / \lpquote P \rpquote \id{\}}            = 
  \left\{ 
    \begin{array}{ccc}
      \lpquote Q \rpquote & & x \nameeq \lpquote P \rpquote \\
      x & & otherwise \\
    \end{array}
  \right. \nonumber
\end{eqnarray}

and $z$ is chosen distinct from $\quotep{P}$, $\quotep{Q}$, the free
names in $Q$, and all the names in $R$. Our $\alpha$-equivalence will
be built in the standard way from this substitution.

\begin{remark}\label{rem:no_self_referential_names}
  One consequence of these definitions is that $\forall P. \quotep{P}
  \not\in \freenames{P}$.
\end{remark}

\subsection{ Dynamic quote: an example }

Anticipating something of what's to come, consider applying the
substitution, $\widehat{\id{\{}u / z \id{\}}}$, to the following pair
of processes, $\lift{w}{y!(z)}$ and $w[ \lpquote y!(z) \rpquote ]$.

\begin{eqnarray}
	\lift{w}{y!(z)}\widehat{\id{\{}u / z \id{\}}}
		& = &
		\lift{w}{y!(u)} \nonumber\\
	w[ \lpquote y!(z) \rpquote ] \widehat{ \id{\{}u / z \id{\}} }
		& = &
		w[ \lpquote y!(z) \rpquote ] \nonumber
\end{eqnarray}

Because the body of the process between quotes is impervious to
substitution, we get radically different answers. In fact, by
examining the first process in an input context,
e.g. $x?(z).\lift{w}{y!(z)}$, we see that the process under the lift
operator may be shaped by prefixed inputs binding a name inside it. In
this sense, the lift operator will be seen as a way to dynamically
construct processes before reifying them as names.

Finally equipped with these standard features we can present the
dynamics of the calculus.

\subsubsection{Operational semantics} 

Finally, we introduce the computational dynamics. What marks these
algebras as distinct from other more traditionally studied algebraic
structures, e.g. vector spaces or polynomial rings, is the manner in
which dynamics is captured. In traditional structures, dynamics is typically
expressed through morphisms between such structures, as in linear maps
between vector spaces or morphisms between rings. In algebras
associated with the semantics of computation, the dynamics is
expressed as part of the algebraic structure itself, through a
reduction reduction relation typically denoted by $\red$. Below, we
give a recursive presentation of this relation for the calculus used
in the encoding.

$\red \subseteq \pi \times \pi$
$\red : \pi \to \mathcal{P}(\pi)$

\begin{mathpar}
  \inferrule* [lab=Comm] { \textsf{match}( x_{src}, x_{trgt} ) } { x_{trgt}?(y)P \; | \; x_{src}!\langle {Q} \rangle \red P\{\quotep{Q}/y}\} }
  \and \\
  \inferrule* [lab=Par] {{P} \red {P}'} {{{P} | {Q}} \red {{P}' | {Q}}}
  \and
  \inferrule* [lab=Equiv]{{{P} \scong {P}'} \andalso {{P}' \red {Q}'} \andalso {{Q}' \scong {Q}}}{{P} \red {Q}}
\end{mathpar}

\begin{eqnarray*}
  match_{\equiv} (\quotep{P},\quotep{Q}) & := & P \equiv Q \\
  match_{\dagger}(\quotep{P},\quotep{Q}) & := & \forall R. P|Q \red^{*} R => R \red^{*} 0 \\
  match_{K}(\quotep{P},\quotep{Q}) & := & K \mbox{ for some context } K
\end{eqnarray*}

$u?(x)P | u!\langle Q \rangle \red P\{\quotep{Q}/x\}$

%We write $\wred$ for $\red^*$, and $P\red$ if $\exists Q $ such that $ P \red Q$.
We write $P\red$ if $\exists Q $ such that $ P \red Q$ and $P\not\red$, otherwise.

\section{Replication}

As mentioned before, it is known that replication (and hence
recursion) can be implemented in a higher-order process algebra
\cite{SangiorgiWalker}. As our first example of calculation with the
machinery thus far presented we give the construction explicitly in
the {\rhoc}.

\begin{eqnarray}
	D_{x} & := & \prefix{x}{y}{(\binpar{\outputp{x}{y}}{@{y}})} \nonumber\\
	\bangp_{x}{P} & := & \binpar{{x}!\langle{\binpar{D_{x}}{P}}\rangle}{D_{x}} \nonumber
\end{eqnarray}

\begin{eqnarray}
	\bangp_{x}{P} & & \nonumber\\
	=
	& {x}!\langle{(\prefix{x}{y}{(\outputp{x}{y} | @{y})) | P}}\rangle 
	      | \prefix{x}{y}{(\outputp{x}{y} | @{y})} & \nonumber\\
	\red
	& (\outputp{x}{y} | @{y})\substn{\quotep{(\prefix{x}{y}{(@{y} | \outputp{x}{y})) | P}}}{y} & \nonumber\\
	=
	& \outputp{x}{\quotep{(\prefix{x}{y}{(\outputp{x}{y} | @{y})) | P}}}
	  | {(\prefix{x}{y}{(\outputp{x}{y} | @{y})) | P}} & \nonumber\\
	\red
	& \ldots & \nonumber\\
	\red^*
	& P | P | \ldots & \nonumber
\end{eqnarray}

Of course, this encoding, as an implementation, runs away, unfolding
$\bangp{P}$ eagerly. A lazier and more implementable replication
operator, restricted to input-guarded processes, may be obtained as follows.

\begin{eqnarray}
\bangp{\prefix{u}{v}{P}} 
	:= 
	\binpar{\lift{x}{\prefix{u}{v}{(\binpar{D(x)}{P})}}}{D(x)} \nonumber
\end{eqnarray}

\begin{remark}
  Note that the lazier definition still does not deal with summation
  or mixed summation (i.e. sums over input and output). The reader is
  invited to construct definitions of replication that deal with these
  features. 

  Further, the definitions are parameterized in a name, $x$. Can you,
  gentle reader, make a definition that eliminates this parameter and
  guarantees no accidental interaction between the replication
  machinery and the process being replicated -- i.e. no accidental
  sharing of names used by the process to get its work done and the
  name(s) used by the replication to effect copying. This latter
  revision of the definition of replication is crucial to obtaining
  the expected identity $!!P \sim !P$.
\end{remark}

\begin{remark}\label{rem:paradoxical_combinator}
  The reader familiar with the lambda calculus will have noticed the
  similarity between $D$ and the paradoxical combinator.

  [Ed. note: the existence of this seems to suggest we have to be more
  restrictive on the set of processes and names we admit if we are to
  support no-cloning.]
\end{remark}

\subsubsection{Bisimulation}

The computational dynamics gives rise to another kind of equivalence,
the equivalence of computational behavior. As previously mentioned
this is typically captured \emph{via} some form of bisimulation.

% The notion we use in this paper is weak barbed bisimulation
% \cite{milner91polyadicpi}.

The notion we use in this paper is derived from weak barbed
bisimulation \cite{milner91polyadicpi}. 

\begin{definition}
An \emph{observation relation}, $\downarrow_{\mathcal N}$, over a set
of names, $\mathcal N$, is the smallest relation satisfying the rules
below.

\infrule[Out-barb]{y \in {\mathcal N}, \; x \nameeq y}
		  {\outputp{x}{v} \downarrow_{\mathcal N} x}
\infrule[Par-barb]{\mbox{$P\downarrow_{\mathcal N} x$ or $Q\downarrow_{\mathcal N} x$}}
		  {\binpar{P}{Q} \downarrow_{\mathcal N} x}

We write $P \Downarrow_{\mathcal N} x$ if there is $Q$ such that 
$P \wred Q$ and $Q \downarrow_{\mathcal N} x$.
\end{definition}

\begin{definition}
%\label{def.bbisim}
An  ${\mathcal N}$-\emph{barbed bisimulation} over a set of names, ${\mathcal N}$, is a symmetric binary relation 
${\mathcal S}_{\mathcal N}$ between agents such that $P\rel{S}_{\mathcal N}Q$ implies:
\begin{enumerate}
\item If $P \red P'$ then $Q \wred Q'$ and $P'\rel{S}_{\mathcal N} Q'$.
\item If $P\downarrow_{\mathcal N} x$, then $Q\Downarrow_{\mathcal N} x$.
\end{enumerate}
$P$ is ${\mathcal N}$-barbed bisimilar to $Q$, written
$P \wbbisim_{\mathcal N} Q$, if $P \rel{S}_{\mathcal N} Q$ for some ${\mathcal N}$-barbed bisimulation ${\mathcal S}_{\mathcal N}$.
\end{definition}

$\mathcal{R} \subseteq \pi \times \pi$

$P \mathcal{R} Q => \forall P'. P \red P' \Rightarrow \exists Q'. Q \red Q', P' \mathcal{R} Q'$

$P \vdash x \Rightarrow Q \vdash x$

\begin{mathpar}
  \inferrule*[lab=Out-barb]{x \nameeq y}{{y}!\langle{Q}\rangle \vdash x}
  \and
  \inferrule*[lab=Par-barb]{\mbox{$P\vdash x$ or $Q\vdash x$}}{\binpar{P}{Q} \vdash x}
\end{mathpar}

\subsubsection{Contexts}

One of the principle advantages of computational calculi like the
$\pi$-calculus is a well-defined notion of context,
contextual-equivalence and a correlation between
contextual-equivalence and notions of bisimulation. The notion of
context allows the decomposition of a process into (sub-)process and
its syntactic environment, its context. Thus, a context may be
thought of as a process with a ``hole'' (written $\Box$) in it. The
application of a context $M$ to a process $P$, written $M[P]$, is
tantamount to filling the hole in $M$ with $P$. In this paper we do
not need the full weight of this theory, but do make use of the notion
of context in the proof the main theorem. 

\begin{mathpar}
  \inferrule* [lab=summation] {} {{M_{M},M_{N}} \bc \Box \;|\; x.M_{A} \;|\; M_{M}+M_{N}}
  \and
  \inferrule* [lab=agent] {} {{M_{A}} \bc (\vec{x})M_{P} \;| \; \clift{P_0,\ldots,M_{P},\ldots,P_N}}
  \and \\
  \inferrule* [lab=process] {} {{M_{P}} \bc M_{N} \;| \;P|M_{P} }
\end{mathpar} 

\begin{mathpar}
  \inferrule* [lab=sychronization] {} {M_{N} \bc \Box \;|\; x?M_{F} \;|\; x!M_{C}}
  \and
  \inferrule* [lab=abstraction] {} {{M_{F}} \bc (x)M_{P} }
  \and
  \inferrule* [lab=concretion] {} {{M_{C}} \bc \langle M_{P} \rangle }
  \and \\
  \inferrule* [lab=process] {} {{M_{P}} \bc M_{N} \;| \;P|M_{P} }
\end{mathpar}

\begin{definition}[contextual application] Given a context $M$, and
  process $P$, we define the \emph{contextual application}, $M[P] :=
  M\{P/\Box\}$. That is, the contextual application of M to P is the
  substitution of $P$ for $\Box$ in $M$.
\end{definition}

$\meaningof{-} : L \to \mathcal{P}(\pi)$

\begin{mathpar}
  \inferrule* [lab=collection] {} {\meaningof{true} = \pi, \and \meaningof{~E} = \pi \setminus \meaningof{E}, \and \meaningof{E_{1} \& E_{2}} = \meaningof{E_{1}} \cap \meaningof{E_{2}}}
\end{mathpar}

\begin{mathpar}
  \inferrule* [lab=structure] {} {\meaningof{0} = \{ P \in \pi | P \equiv 0 \}, \and \\ \meaningof{E_1 | E_2} = \{ P \in \pi | P \equiv P_{1} | P_{2}, P_{1} \in \meaningof{E_{1}}, P_{2} \in \meaningof{E_2}\} }
\end{mathpar}

\begin{mathpar}
 \inferrule* [lab=behavior] {} {\meaningof{\langle a?b \rangle E} = \{ P \in \pi | P \equiv Q | u?(y)P', \\ \and \\\\ \and \\ \;\;\; u \in \meaningof{a}, \forall z.P'\{z/y\} \in \meaningof{E\{z/b\}}\}, \and \\ \meaningof{a!E} = \{ P \in \pi | P \equiv Q | x!\langle P' \rangle, x \in \meaningof{a} P' \in \meaningof{E}\} }
\end{mathpar}

\begin{mathpar}
 \inferrule* [lab=nominal] {} {\meaningof{\quotep{E}} = \{ \quotep{P} \in \quotep{\pi} | P \in \meaningof{E} \}, \and \meaningof{\quotep{P}} = \{ \quotep{Q} \in \quotep{\pi} | P \equiv Q \} \and \\ \meaningof{@\quotep{E}} = \{ P \in \pi | P \equiv @x, x \in \meaningof{E} \}}
\end{mathpar}

\begin{eqnarray*}
  \\
  \meaningof{-} : TS \to ST
\end{eqnarray*}

\begin{eqnarray*}
  \\
  L : TS \to ST
\end{eqnarray*}

\begin{eqnarray*}
  \\
  P \models E \iff P \in \meaningof{E}
\end{eqnarray*}

\begin{eqnarray*}
  P \approx_{L} Q \iff \forall E \in L. P \models E \iff Q \models E
\end{eqnarray*}

\begin{eqnarray*}
  P \approx_{K} Q
\end{eqnarray*}

\begin{eqnarray*}
  P \approx Q
\end{eqnarray*}

$\approx_{K} = \approx = \approx_{L}$

\subsubsection{Contextual duality}

Note that contexts extend the quotation operation to a family of
operations from processes to names. Given a context, $M$, we can
define a \emph{nominal context}, $\quotep{M}$ by $\quotep{M}[P] :=
\quotep{M[P]}$. To foreshadow what is to come we observe that these
operations enjoy a duality with processes very much like the duality
between vectors and maps from vectors to scalars.

Further, because the calculus is essentially higher-order, we have a
correspondence between contexts and processes. More specifically,
given a name $x$ and a context $M$ we can construct $M^{*}_{x}$ such
that 

\begin{mathpar}
  M^{*}_{x} | \lift{x}{P} \red M[P]
\end{mathpar}

namely,

\begin{mathpar}
  M^{*}_{x} := x?(u).M[\dropn{u}]
\end{mathpar}

The dependence of $M^{*}_{x}$ on a name makes it an abstraction, 

\begin{mathpar}
  M^{*} := (x)x?(u).M[\dropn{u}]
\end{mathpar}

\subsection{Additional notation}

It will sometimes be convenient to denote the process a name
quotes. We already have the notation $x = \quotep{P}$, but it will be
convenient to introduce an alternate notation, $\procn{x}$, when we
want to emphasize the connection to the use of the name. Note that, by
virtue of name equivalence, $\quotep{\procn{x}} \nameeq x$; so, the
notation is consistent with previous definitions.

Further, because names have structure it is possible to effect
substitutions on the basis of that structure. This means we need to
upgrade our notation for substitutions, which we accomplish by
adapting comprehension notation. Thus,

\begin{mathpar}
  P\{ y / x : x \in S \}
\end{mathpar}

is interpreted to mean the process derived from P by replacing (in a
capture-avoiding manner) each occurrence of $x$ in $S$ by $y$. For example,

\begin{mathpar}
  P\{ \quotep{\procn{x}|\procn{x}} / x : x \in \freenames{P} \}
\end{mathpar}

will replace each (occurrence) of a free name $x$ in $P$ by
$\quotep{\procn{x}|\procn{x}}$.

Also, we will avail ourselves of the notation $x^{L}$ and $x^{R}$ to
denote injections of a name into disjoint copies of the name
space. There are numerous ways to accomplish this. One example can be
found in \cite{MeredithR05}. This notation overloads to vectors of
names: $\vec{x}^{\pi} := (x_{i}^{\pi} \; : \; 0 \leq i < |\vec{x}| )$ where $\pi \in \{L,R\}$.

We also use $P^{\Box} := P|\Box$.

In \cite{MeredithR05} an interpretation of the new operator is
given. It turns out that there are several possible interpretations
all enjoying the requisite algebraic properties of the operator (see
\cite{milner91polyadicpi}). We will therefore make liberal use of
$(\nu\; \vec{x})P$.

% subsection the_syntax_and_semantics_of_the_notation_system (end)   

\input{qm2pi.qmops} 

\input{qm2pi.sterngerlach} 

\input{qm2pi.metric} 

% section concurrent_process_calculi (end)

%\input{qm2pi.proofsketch}

% section proof sketch (end)

%\input{qm2pi.slviaknots} 

% section spatial logic via knots (end)

\input{qm2pi.conclusion}

% section conclusion (end)

%\input{qm2pi.dtcodes} 

% section wiring algorithm (end)

\input{qm2pi.ack} 

% section acknowledgments (end)

\newpage


\bibliographystyle{plain}   
\bibliography{../../biblios/main.bib}

\input{qm2pi.rhodetails}

\end{document}

 

\documentclass[12pt]{llncs}
%\documentclass{jktr}

\usepackage[pdftex]{hyperref}                   
\usepackage {listings}
\usepackage {mathpartir}
\usepackage{bcprules}
%\usepackage{listings}
                       
\usepackage{graphicx} 
%\usepackage[margins=2.5cm,nohead,nofoot]{geometry}
%\usepackage{geometry}
\usepackage{amsfonts}
\usepackage{amstext}
\usepackage{latexsym}
\usepackage{amssymb}
\usepackage{color}


%\include{myPreamble}
\include{qm2pi.local} 

%\ifpdf
%\usepackage[pdftex]{graphicx}
%\else
%\usepackage{graphicx}
%\fi

 % \ifpdf
%  \usepackage{pdfsync}
%  \if


%\title{Brief Article}
%\author{David F. Snyder}
%\author{L.G. Meredith}

%\address{Dept. of Math., Texas State University--San Marcos, San Marcos, TX 78666}
       
\pagestyle{empty}


\begin{document}

\lstset{language=[Objective]Caml,frame=shadowbox}

\input{qm2pi.front}

% section front matter (end)

\input{qm2pi.intro} 
 
% section introduction (end)

% \input{qm2pi.knotations} 

% section notation (end)

\input{qm2pi.process.calculi} 

% section concurrent_process_calculi_and_spatial_logics_ (end)
    
%\input{qm2pi.knots2pi} 

%\input{qm2pi.trefoil} 

%\input{qm2pi.mainthm} 

% subsection basic_interpretation (end)

%\input{qm2pi.rho.presentation} 
\subsection{The syntax and semantics of the notation system}\label{sub:the_syntax_and_semantics_of_the_notation_system} % (fold)

We now summarize a technical presentation of the calculus that
embodies our theory of dynamics. The typical presentation of such a
calculus follows the style of giving generators and relations on
them. The grammar, below, describing term constructors, freely
generates the set of processes, $\Proc$. This set is then quotiented
by a relation known as structural congruence and it is over this set
that the notion of dynamics is expressed. This presentation is
essentially that of \cite{MeredithR05} with the addition of
polyadicity and summation. For readability we have relegated some of
the technical subtleties to an appendix.

\subsubsection{Process grammar}\label{subsub:process_grammar}

\begin{mathpar}
  \inferrule* [lab=synchronization] {} {{M} \bc \pzero \;|\; x?F \;|\; x!C }
  \and
  \inferrule* [lab=abstraction] {} {{F} \bc (x)P}
  \and
  \inferrule* [lab=concretion] {} {{C} \bc \langle Q \rangle}
  \and
  \inferrule* [lab=process] {} {{P,Q} \bc M \;| \;P|Q \;|\; @{x}}
  \and
  \inferrule* [lab=name] {} {{x} \bc \quotep{P}}
\end{mathpar} 

Note that $\vec{x}$ (resp. $\vec{P}$) denotes a vector of names
(resp. processes) of length $|\vec{x}|$ (resp. $|\vec{P}|$). We adopt
the following useful abbreviations.

\begin{mathpar}
   x?(\vec{y}).P := x.(\vec{y})P \and  x\clift{\vec{P}} := x.\clift{\vec{P}}
   \and x!(y) := \lift{x}{\dropn{y}}
   \and \Pi_{i=0}^{n-1}P_i := P_0 | \ldots | P_{n-1}
\end{mathpar}

\subsubsection{Structural congruence}

\paragraph{Free and bound names and alpha-equivalence.} At the
core of structural equivalence is alpha-equivalence which identifies
process that are the same up to a change of variable. Formally, we
recognize the distinction between free and bound names. The free names
of a process, $\freenames{P}$, may be calculated recursively as
follows:

\begin{mathpar}
\freenames{\pzero} := \emptyset
  \and \\
  \freenames{x?(y).P} := \{ x \} \cup (\freenames{P} \setminus \{ y \})
  \and 
  \freenames{x!\langle P \rangle} := \{ x \} \cup \{ P \} 
  \and \\
  \freenames{P|Q} := \freenames{P} \cup \freenames{Q}
  \and \\
  \freenames{@{x}} := \{ x \}
\end{mathpar}

$\pi$
$\quotep{\pi}$

$\freenames{-} : \pi \to \mathcal{P}(\quotep{\pi})$

\begin{eqnarray*}
  \freenames{\pzero} & := & \emptyset \\
  \freenames{x?(y).P} & := & \{ x \} \cup (\freenames{P} \setminus \{ y \}) \\
  \freenames{x!\langle P \rangle} & := & \{ x \} \cup \{ P \} \\
  \freenames{P|Q} & := & \freenames{P} \cup \freenames{Q} \\
  \freenames{\dropn{x}} & := & \{ x \}
\end{eqnarray*}

The bound names of a process, $\boundnames{P}$, are those names occurring in $P$
that are not free. For example, in $x?(y).0$, the name $x$ is free, while $y$ is bound.

\begin{mathpar}
  \inferrule* [lab=monoidal-laws] {} { P|Q \equiv Q|P \and P|0 \equiv P \and P|(Q|R) \equiv (P|Q)|R }
\end{mathpar}

\begin{mathpar}
  \inferrule* [lab=alpha-equivalence] {} { (x)P \equiv (y)P\{y/x\} \and y \not\in \freenames{P} }
\end{mathpar}

\begin{definition}
Then two processes, $P,Q$, are alpha-equivalent if $P = Q\{\vec{y}/\vec{x}\}$ for
some $\vec{x} \in \boundnames{Q},\vec{y} \in \boundnames{P}$, where $Q\{\vec{y}/\vec{x}\}$
denotes the capture-avoiding substitution of $\vec{y}$ for $\vec{x}$ in $Q$.
\end{definition}

\begin{definition}
  The {\em structural congruence} \cite{SangiorgiWalker} , $\equiv$,
  between processes is the least congruence containing
  alpha-equivalence, satisfying the abelian monoid laws
  (associativity, commutativity and $\pzero$ as identity) for parallel
  composition $|$ and for summation $+$.
\end{definition}

\subsection{Name equivalence}

We take name equivalence, written $\nameeq$, to be the smallest
equivalence relation generated by the following rules.

\begin{mathpar}
\inferrule*[lab=Quote-drop]
{ }
{ \quotep{@{x}} \nameeq x }

\inferrule*[lab=Struct-equiv]
{ P \scong Q }
{ \quotep{P} \nameeq \quotep{Q} }
\end{mathpar}

The astute reader will have noticed that the mutual recursion of names
and processes imposes a mutual recursion on alpha-equivalence and
structural equivalence via name-equivalence. Fortunately, all of this
works out pleasantly and we may calculate in the natural way, free of
concern. The reader interested in the details is referred to the
appendix \ref{appendix:rho_details}.

\subsection{Substitution}

We use $\Proc$ for the set of processes, $\QProc$ for the set of
names, and $\id{\{}\vec{y} / \vec{x} \id{\}}$ to denote partial maps,
$s : \QProc \rightarrow \QProc$. A map, $s$ lifts, uniquely, to a map
on process terms, $\widehat{s} : \Proc \rightarrow \Proc$ by the
following equations.

\begin{mathpar}
  (0) \psubstp{Q}{P} := 0 \\
  (R \juxtap S) \psubstp{Q}{P}
  :=    
  (R)\psubstp{Q}{P} \juxtap (S) \psubstp{Q}{P} \\
  (x?(y).R) \psubstp{Q}{P}    
  :=    
  (x)\substp{Q}{P} (z)\concat( (R \psubstn{z}{y}) \psubstp{Q}{P} ) \\
  (\lift{x}{R}) \psubstp{Q}{P}  
  :=
  \lift{(x)\substp{Q}{P}}{ R \psubstp{Q}{P} } \\
%   (\dropn{x})  \psubstp{Q}{P}       
%   := 
%   \left\{ 
%     \begin{array}{ccc} 
%       \dropn{\quotep{Q}} & & x \nameeq \quotep{P} \\
%       \dropn{x} & & otherwise \\
%     \end{array}
%   \right. 
  (\dropn{x})  \psubstp{Q}{P}       
  := 
  \left\{ 
    \begin{array}{ccc} 
      Q & & x \nameeq \quotep{P} \\
      \dropn{x} & & otherwise \\
    \end{array}
  \right.
\end{mathpar}
 

where

\begin{eqnarray}
  (x)\id{\{} \lpquote Q \rpquote / \lpquote P \rpquote \id{\}}            = 
  \left\{ 
    \begin{array}{ccc}
      \lpquote Q \rpquote & & x \nameeq \lpquote P \rpquote \\
      x & & otherwise \\
    \end{array}
  \right. \nonumber
\end{eqnarray}

and $z$ is chosen distinct from $\quotep{P}$, $\quotep{Q}$, the free
names in $Q$, and all the names in $R$. Our $\alpha$-equivalence will
be built in the standard way from this substitution.

\begin{remark}\label{rem:no_self_referential_names}
  One consequence of these definitions is that $\forall P. \quotep{P}
  \not\in \freenames{P}$.
\end{remark}

\subsection{ Dynamic quote: an example }

Anticipating something of what's to come, consider applying the
substitution, $\widehat{\id{\{}u / z \id{\}}}$, to the following pair
of processes, $\lift{w}{y!(z)}$ and $w[ \lpquote y!(z) \rpquote ]$.

\begin{eqnarray}
	\lift{w}{y!(z)}\widehat{\id{\{}u / z \id{\}}}
		& = &
		\lift{w}{y!(u)} \nonumber\\
	w[ \lpquote y!(z) \rpquote ] \widehat{ \id{\{}u / z \id{\}} }
		& = &
		w[ \lpquote y!(z) \rpquote ] \nonumber
\end{eqnarray}

Because the body of the process between quotes is impervious to
substitution, we get radically different answers. In fact, by
examining the first process in an input context,
e.g. $x?(z).\lift{w}{y!(z)}$, we see that the process under the lift
operator may be shaped by prefixed inputs binding a name inside it. In
this sense, the lift operator will be seen as a way to dynamically
construct processes before reifying them as names.

Finally equipped with these standard features we can present the
dynamics of the calculus.

\subsubsection{Operational semantics} 

Finally, we introduce the computational dynamics. What marks these
algebras as distinct from other more traditionally studied algebraic
structures, e.g. vector spaces or polynomial rings, is the manner in
which dynamics is captured. In traditional structures, dynamics is typically
expressed through morphisms between such structures, as in linear maps
between vector spaces or morphisms between rings. In algebras
associated with the semantics of computation, the dynamics is
expressed as part of the algebraic structure itself, through a
reduction reduction relation typically denoted by $\red$. Below, we
give a recursive presentation of this relation for the calculus used
in the encoding.

$\red \subseteq \pi \times \pi$
$\red : \pi \to \mathcal{P}(\pi)$

\begin{mathpar}
  \inferrule* [lab=Comm] { \textsf{match}( x_{src}, x_{trgt} ) } { x_{trgt}?(y)P \; | \; x_{src}!\langle {Q} \rangle \red P\{\quotep{Q}/y}\} }
  \and \\
  \inferrule* [lab=Par] {{P} \red {P}'} {{{P} | {Q}} \red {{P}' | {Q}}}
  \and
  \inferrule* [lab=Equiv]{{{P} \scong {P}'} \andalso {{P}' \red {Q}'} \andalso {{Q}' \scong {Q}}}{{P} \red {Q}}
\end{mathpar}

\begin{eqnarray*}
  match_{\equiv} (\quotep{P},\quotep{Q}) & := & P \equiv Q \\
  match_{\dagger}(\quotep{P},\quotep{Q}) & := & \forall R. P|Q \red^{*} R => R \red^{*} 0 \\
  match_{K}(\quotep{P},\quotep{Q}) & := & K \mbox{ for some context } K
\end{eqnarray*}

$u?(x)P | u!\langle Q \rangle \red P\{\quotep{Q}/x\}$

%We write $\wred$ for $\red^*$, and $P\red$ if $\exists Q $ such that $ P \red Q$.
We write $P\red$ if $\exists Q $ such that $ P \red Q$ and $P\not\red$, otherwise.

\section{Replication}

As mentioned before, it is known that replication (and hence
recursion) can be implemented in a higher-order process algebra
\cite{SangiorgiWalker}. As our first example of calculation with the
machinery thus far presented we give the construction explicitly in
the {\rhoc}.

\begin{eqnarray}
	D_{x} & := & \prefix{x}{y}{(\binpar{\outputp{x}{y}}{@{y}})} \nonumber\\
	\bangp_{x}{P} & := & \binpar{{x}!\langle{\binpar{D_{x}}{P}}\rangle}{D_{x}} \nonumber
\end{eqnarray}

\begin{eqnarray}
	\bangp_{x}{P} & & \nonumber\\
	=
	& {x}!\langle{(\prefix{x}{y}{(\outputp{x}{y} | @{y})) | P}}\rangle 
	      | \prefix{x}{y}{(\outputp{x}{y} | @{y})} & \nonumber\\
	\red
	& (\outputp{x}{y} | @{y})\substn{\quotep{(\prefix{x}{y}{(@{y} | \outputp{x}{y})) | P}}}{y} & \nonumber\\
	=
	& \outputp{x}{\quotep{(\prefix{x}{y}{(\outputp{x}{y} | @{y})) | P}}}
	  | {(\prefix{x}{y}{(\outputp{x}{y} | @{y})) | P}} & \nonumber\\
	\red
	& \ldots & \nonumber\\
	\red^*
	& P | P | \ldots & \nonumber
\end{eqnarray}

Of course, this encoding, as an implementation, runs away, unfolding
$\bangp{P}$ eagerly. A lazier and more implementable replication
operator, restricted to input-guarded processes, may be obtained as follows.

\begin{eqnarray}
\bangp{\prefix{u}{v}{P}} 
	:= 
	\binpar{\lift{x}{\prefix{u}{v}{(\binpar{D(x)}{P})}}}{D(x)} \nonumber
\end{eqnarray}

\begin{remark}
  Note that the lazier definition still does not deal with summation
  or mixed summation (i.e. sums over input and output). The reader is
  invited to construct definitions of replication that deal with these
  features. 

  Further, the definitions are parameterized in a name, $x$. Can you,
  gentle reader, make a definition that eliminates this parameter and
  guarantees no accidental interaction between the replication
  machinery and the process being replicated -- i.e. no accidental
  sharing of names used by the process to get its work done and the
  name(s) used by the replication to effect copying. This latter
  revision of the definition of replication is crucial to obtaining
  the expected identity $!!P \sim !P$.
\end{remark}

\begin{remark}\label{rem:paradoxical_combinator}
  The reader familiar with the lambda calculus will have noticed the
  similarity between $D$ and the paradoxical combinator.

  [Ed. note: the existence of this seems to suggest we have to be more
  restrictive on the set of processes and names we admit if we are to
  support no-cloning.]
\end{remark}

\subsubsection{Bisimulation}

The computational dynamics gives rise to another kind of equivalence,
the equivalence of computational behavior. As previously mentioned
this is typically captured \emph{via} some form of bisimulation.

% The notion we use in this paper is weak barbed bisimulation
% \cite{milner91polyadicpi}.

The notion we use in this paper is derived from weak barbed
bisimulation \cite{milner91polyadicpi}. 

\begin{definition}
An \emph{observation relation}, $\downarrow_{\mathcal N}$, over a set
of names, $\mathcal N$, is the smallest relation satisfying the rules
below.

\infrule[Out-barb]{y \in {\mathcal N}, \; x \nameeq y}
		  {\outputp{x}{v} \downarrow_{\mathcal N} x}
\infrule[Par-barb]{\mbox{$P\downarrow_{\mathcal N} x$ or $Q\downarrow_{\mathcal N} x$}}
		  {\binpar{P}{Q} \downarrow_{\mathcal N} x}

We write $P \Downarrow_{\mathcal N} x$ if there is $Q$ such that 
$P \wred Q$ and $Q \downarrow_{\mathcal N} x$.
\end{definition}

\begin{definition}
%\label{def.bbisim}
An  ${\mathcal N}$-\emph{barbed bisimulation} over a set of names, ${\mathcal N}$, is a symmetric binary relation 
${\mathcal S}_{\mathcal N}$ between agents such that $P\rel{S}_{\mathcal N}Q$ implies:
\begin{enumerate}
\item If $P \red P'$ then $Q \wred Q'$ and $P'\rel{S}_{\mathcal N} Q'$.
\item If $P\downarrow_{\mathcal N} x$, then $Q\Downarrow_{\mathcal N} x$.
\end{enumerate}
$P$ is ${\mathcal N}$-barbed bisimilar to $Q$, written
$P \wbbisim_{\mathcal N} Q$, if $P \rel{S}_{\mathcal N} Q$ for some ${\mathcal N}$-barbed bisimulation ${\mathcal S}_{\mathcal N}$.
\end{definition}

$\mathcal{R} \subseteq \pi \times \pi$

$P \mathcal{R} Q => \forall P'. P \red P' \Rightarrow \exists Q'. Q \red Q', P' \mathcal{R} Q'$

$P \vdash x \Rightarrow Q \vdash x$

\begin{mathpar}
  \inferrule*[lab=Out-barb]{x \nameeq y}{{y}!\langle{Q}\rangle \vdash x}
  \and
  \inferrule*[lab=Par-barb]{\mbox{$P\vdash x$ or $Q\vdash x$}}{\binpar{P}{Q} \vdash x}
\end{mathpar}

\subsubsection{Contexts}

One of the principle advantages of computational calculi like the
$\pi$-calculus is a well-defined notion of context,
contextual-equivalence and a correlation between
contextual-equivalence and notions of bisimulation. The notion of
context allows the decomposition of a process into (sub-)process and
its syntactic environment, its context. Thus, a context may be
thought of as a process with a ``hole'' (written $\Box$) in it. The
application of a context $M$ to a process $P$, written $M[P]$, is
tantamount to filling the hole in $M$ with $P$. In this paper we do
not need the full weight of this theory, but do make use of the notion
of context in the proof the main theorem. 

\begin{mathpar}
  \inferrule* [lab=summation] {} {{M_{M},M_{N}} \bc \Box \;|\; x.M_{A} \;|\; M_{M}+M_{N}}
  \and
  \inferrule* [lab=agent] {} {{M_{A}} \bc (\vec{x})M_{P} \;| \; \clift{P_0,\ldots,M_{P},\ldots,P_N}}
  \and \\
  \inferrule* [lab=process] {} {{M_{P}} \bc M_{N} \;| \;P|M_{P} }
\end{mathpar} 

\begin{mathpar}
  \inferrule* [lab=sychronization] {} {M_{N} \bc \Box \;|\; x?M_{F} \;|\; x!M_{C}}
  \and
  \inferrule* [lab=abstraction] {} {{M_{F}} \bc (x)M_{P} }
  \and
  \inferrule* [lab=concretion] {} {{M_{C}} \bc \langle M_{P} \rangle }
  \and \\
  \inferrule* [lab=process] {} {{M_{P}} \bc M_{N} \;| \;P|M_{P} }
\end{mathpar}

\begin{definition}[contextual application] Given a context $M$, and
  process $P$, we define the \emph{contextual application}, $M[P] :=
  M\{P/\Box\}$. That is, the contextual application of M to P is the
  substitution of $P$ for $\Box$ in $M$.
\end{definition}

$\meaningof{-} : L \to \mathcal{P}(\pi)$

\begin{mathpar}
  \inferrule* [lab=collection] {} {\meaningof{true} = \pi, \and \meaningof{~E} = \pi \setminus \meaningof{E}, \and \meaningof{E_{1} \& E_{2}} = \meaningof{E_{1}} \cap \meaningof{E_{2}}}
\end{mathpar}

\begin{mathpar}
  \inferrule* [lab=structure] {} {\meaningof{0} = \{ P \in \pi | P \equiv 0 \}, \and \\ \meaningof{E_1 | E_2} = \{ P \in \pi | P \equiv P_{1} | P_{2}, P_{1} \in \meaningof{E_{1}}, P_{2} \in \meaningof{E_2}\} }
\end{mathpar}

\begin{mathpar}
 \inferrule* [lab=behavior] {} {\meaningof{\langle a?b \rangle E} = \{ P \in \pi | P \equiv Q | u?(y)P', \\ \and \\\\ \and \\ \;\;\; u \in \meaningof{a}, \forall z.P'\{z/y\} \in \meaningof{E\{z/b\}}\}, \and \\ \meaningof{a!E} = \{ P \in \pi | P \equiv Q | x!\langle P' \rangle, x \in \meaningof{a} P' \in \meaningof{E}\} }
\end{mathpar}

\begin{mathpar}
 \inferrule* [lab=nominal] {} {\meaningof{\quotep{E}} = \{ \quotep{P} \in \quotep{\pi} | P \in \meaningof{E} \}, \and \meaningof{\quotep{P}} = \{ \quotep{Q} \in \quotep{\pi} | P \equiv Q \} \and \\ \meaningof{@\quotep{E}} = \{ P \in \pi | P \equiv @x, x \in \meaningof{E} \}}
\end{mathpar}

\begin{eqnarray*}
  \\
  \meaningof{-} : TS \to ST
\end{eqnarray*}

\begin{eqnarray*}
  \\
  L : TS \to ST
\end{eqnarray*}

\begin{eqnarray*}
  \\
  P \models E \iff P \in \meaningof{E}
\end{eqnarray*}

\begin{eqnarray*}
  P \approx_{L} Q \iff \forall E \in L. P \models E \iff Q \models E
\end{eqnarray*}

\begin{eqnarray*}
  P \approx_{K} Q
\end{eqnarray*}

\begin{eqnarray*}
  P \approx Q
\end{eqnarray*}

$\approx_{K} = \approx = \approx_{L}$

\subsubsection{Contextual duality}

Note that contexts extend the quotation operation to a family of
operations from processes to names. Given a context, $M$, we can
define a \emph{nominal context}, $\quotep{M}$ by $\quotep{M}[P] :=
\quotep{M[P]}$. To foreshadow what is to come we observe that these
operations enjoy a duality with processes very much like the duality
between vectors and maps from vectors to scalars.

Further, because the calculus is essentially higher-order, we have a
correspondence between contexts and processes. More specifically,
given a name $x$ and a context $M$ we can construct $M^{*}_{x}$ such
that 

\begin{mathpar}
  M^{*}_{x} | \lift{x}{P} \red M[P]
\end{mathpar}

namely,

\begin{mathpar}
  M^{*}_{x} := x?(u).M[\dropn{u}]
\end{mathpar}

The dependence of $M^{*}_{x}$ on a name makes it an abstraction, 

\begin{mathpar}
  M^{*} := (x)x?(u).M[\dropn{u}]
\end{mathpar}

\subsection{Additional notation}

It will sometimes be convenient to denote the process a name
quotes. We already have the notation $x = \quotep{P}$, but it will be
convenient to introduce an alternate notation, $\procn{x}$, when we
want to emphasize the connection to the use of the name. Note that, by
virtue of name equivalence, $\quotep{\procn{x}} \nameeq x$; so, the
notation is consistent with previous definitions.

Further, because names have structure it is possible to effect
substitutions on the basis of that structure. This means we need to
upgrade our notation for substitutions, which we accomplish by
adapting comprehension notation. Thus,

\begin{mathpar}
  P\{ y / x : x \in S \}
\end{mathpar}

is interpreted to mean the process derived from P by replacing (in a
capture-avoiding manner) each occurrence of $x$ in $S$ by $y$. For example,

\begin{mathpar}
  P\{ \quotep{\procn{x}|\procn{x}} / x : x \in \freenames{P} \}
\end{mathpar}

will replace each (occurrence) of a free name $x$ in $P$ by
$\quotep{\procn{x}|\procn{x}}$.

Also, we will avail ourselves of the notation $x^{L}$ and $x^{R}$ to
denote injections of a name into disjoint copies of the name
space. There are numerous ways to accomplish this. One example can be
found in \cite{MeredithR05}. This notation overloads to vectors of
names: $\vec{x}^{\pi} := (x_{i}^{\pi} \; : \; 0 \leq i < |\vec{x}| )$ where $\pi \in \{L,R\}$.

We also use $P^{\Box} := P|\Box$.

In \cite{MeredithR05} an interpretation of the new operator is
given. It turns out that there are several possible interpretations
all enjoying the requisite algebraic properties of the operator (see
\cite{milner91polyadicpi}). We will therefore make liberal use of
$(\nu\; \vec{x})P$.

% subsection the_syntax_and_semantics_of_the_notation_system (end)   

\input{qm2pi.qmops} 

\input{qm2pi.sterngerlach} 

\input{qm2pi.metric} 

% section concurrent_process_calculi (end)

%\input{qm2pi.proofsketch}

% section proof sketch (end)

%\input{qm2pi.slviaknots} 

% section spatial logic via knots (end)

\input{qm2pi.conclusion}

% section conclusion (end)

%\input{qm2pi.dtcodes} 

% section wiring algorithm (end)

\input{qm2pi.ack} 

% section acknowledgments (end)

\newpage


\bibliographystyle{plain}   
\bibliography{../../biblios/main.bib}

\input{qm2pi.rhodetails}

\end{document}

 

% section concurrent_process_calculi (end)

%\documentclass[12pt]{llncs}
%\documentclass{jktr}

\usepackage[pdftex]{hyperref}                   
\usepackage {listings}
\usepackage {mathpartir}
\usepackage{bcprules}
%\usepackage{listings}
                       
\usepackage{graphicx} 
%\usepackage[margins=2.5cm,nohead,nofoot]{geometry}
%\usepackage{geometry}
\usepackage{amsfonts}
\usepackage{amstext}
\usepackage{latexsym}
\usepackage{amssymb}
\usepackage{color}


%\include{myPreamble}
\include{qm2pi.local} 

%\ifpdf
%\usepackage[pdftex]{graphicx}
%\else
%\usepackage{graphicx}
%\fi

 % \ifpdf
%  \usepackage{pdfsync}
%  \if


%\title{Brief Article}
%\author{David F. Snyder}
%\author{L.G. Meredith}

%\address{Dept. of Math., Texas State University--San Marcos, San Marcos, TX 78666}
       
\pagestyle{empty}


\begin{document}

\lstset{language=[Objective]Caml,frame=shadowbox}

\input{qm2pi.front}

% section front matter (end)

\input{qm2pi.intro} 
 
% section introduction (end)

% \input{qm2pi.knotations} 

% section notation (end)

\input{qm2pi.process.calculi} 

% section concurrent_process_calculi_and_spatial_logics_ (end)
    
%\input{qm2pi.knots2pi} 

%\input{qm2pi.trefoil} 

%\input{qm2pi.mainthm} 

% subsection basic_interpretation (end)

%\input{qm2pi.rho.presentation} 
\subsection{The syntax and semantics of the notation system}\label{sub:the_syntax_and_semantics_of_the_notation_system} % (fold)

We now summarize a technical presentation of the calculus that
embodies our theory of dynamics. The typical presentation of such a
calculus follows the style of giving generators and relations on
them. The grammar, below, describing term constructors, freely
generates the set of processes, $\Proc$. This set is then quotiented
by a relation known as structural congruence and it is over this set
that the notion of dynamics is expressed. This presentation is
essentially that of \cite{MeredithR05} with the addition of
polyadicity and summation. For readability we have relegated some of
the technical subtleties to an appendix.

\subsubsection{Process grammar}\label{subsub:process_grammar}

\begin{mathpar}
  \inferrule* [lab=synchronization] {} {{M} \bc \pzero \;|\; x?F \;|\; x!C }
  \and
  \inferrule* [lab=abstraction] {} {{F} \bc (x)P}
  \and
  \inferrule* [lab=concretion] {} {{C} \bc \langle Q \rangle}
  \and
  \inferrule* [lab=process] {} {{P,Q} \bc M \;| \;P|Q \;|\; @{x}}
  \and
  \inferrule* [lab=name] {} {{x} \bc \quotep{P}}
\end{mathpar} 

Note that $\vec{x}$ (resp. $\vec{P}$) denotes a vector of names
(resp. processes) of length $|\vec{x}|$ (resp. $|\vec{P}|$). We adopt
the following useful abbreviations.

\begin{mathpar}
   x?(\vec{y}).P := x.(\vec{y})P \and  x\clift{\vec{P}} := x.\clift{\vec{P}}
   \and x!(y) := \lift{x}{\dropn{y}}
   \and \Pi_{i=0}^{n-1}P_i := P_0 | \ldots | P_{n-1}
\end{mathpar}

\subsubsection{Structural congruence}

\paragraph{Free and bound names and alpha-equivalence.} At the
core of structural equivalence is alpha-equivalence which identifies
process that are the same up to a change of variable. Formally, we
recognize the distinction between free and bound names. The free names
of a process, $\freenames{P}$, may be calculated recursively as
follows:

\begin{mathpar}
\freenames{\pzero} := \emptyset
  \and \\
  \freenames{x?(y).P} := \{ x \} \cup (\freenames{P} \setminus \{ y \})
  \and 
  \freenames{x!\langle P \rangle} := \{ x \} \cup \{ P \} 
  \and \\
  \freenames{P|Q} := \freenames{P} \cup \freenames{Q}
  \and \\
  \freenames{@{x}} := \{ x \}
\end{mathpar}

$\pi$
$\quotep{\pi}$

$\freenames{-} : \pi \to \mathcal{P}(\quotep{\pi})$

\begin{eqnarray*}
  \freenames{\pzero} & := & \emptyset \\
  \freenames{x?(y).P} & := & \{ x \} \cup (\freenames{P} \setminus \{ y \}) \\
  \freenames{x!\langle P \rangle} & := & \{ x \} \cup \{ P \} \\
  \freenames{P|Q} & := & \freenames{P} \cup \freenames{Q} \\
  \freenames{\dropn{x}} & := & \{ x \}
\end{eqnarray*}

The bound names of a process, $\boundnames{P}$, are those names occurring in $P$
that are not free. For example, in $x?(y).0$, the name $x$ is free, while $y$ is bound.

\begin{mathpar}
  \inferrule* [lab=monoidal-laws] {} { P|Q \equiv Q|P \and P|0 \equiv P \and P|(Q|R) \equiv (P|Q)|R }
\end{mathpar}

\begin{mathpar}
  \inferrule* [lab=alpha-equivalence] {} { (x)P \equiv (y)P\{y/x\} \and y \not\in \freenames{P} }
\end{mathpar}

\begin{definition}
Then two processes, $P,Q$, are alpha-equivalent if $P = Q\{\vec{y}/\vec{x}\}$ for
some $\vec{x} \in \boundnames{Q},\vec{y} \in \boundnames{P}$, where $Q\{\vec{y}/\vec{x}\}$
denotes the capture-avoiding substitution of $\vec{y}$ for $\vec{x}$ in $Q$.
\end{definition}

\begin{definition}
  The {\em structural congruence} \cite{SangiorgiWalker} , $\equiv$,
  between processes is the least congruence containing
  alpha-equivalence, satisfying the abelian monoid laws
  (associativity, commutativity and $\pzero$ as identity) for parallel
  composition $|$ and for summation $+$.
\end{definition}

\subsection{Name equivalence}

We take name equivalence, written $\nameeq$, to be the smallest
equivalence relation generated by the following rules.

\begin{mathpar}
\inferrule*[lab=Quote-drop]
{ }
{ \quotep{@{x}} \nameeq x }

\inferrule*[lab=Struct-equiv]
{ P \scong Q }
{ \quotep{P} \nameeq \quotep{Q} }
\end{mathpar}

The astute reader will have noticed that the mutual recursion of names
and processes imposes a mutual recursion on alpha-equivalence and
structural equivalence via name-equivalence. Fortunately, all of this
works out pleasantly and we may calculate in the natural way, free of
concern. The reader interested in the details is referred to the
appendix \ref{appendix:rho_details}.

\subsection{Substitution}

We use $\Proc$ for the set of processes, $\QProc$ for the set of
names, and $\id{\{}\vec{y} / \vec{x} \id{\}}$ to denote partial maps,
$s : \QProc \rightarrow \QProc$. A map, $s$ lifts, uniquely, to a map
on process terms, $\widehat{s} : \Proc \rightarrow \Proc$ by the
following equations.

\begin{mathpar}
  (0) \psubstp{Q}{P} := 0 \\
  (R \juxtap S) \psubstp{Q}{P}
  :=    
  (R)\psubstp{Q}{P} \juxtap (S) \psubstp{Q}{P} \\
  (x?(y).R) \psubstp{Q}{P}    
  :=    
  (x)\substp{Q}{P} (z)\concat( (R \psubstn{z}{y}) \psubstp{Q}{P} ) \\
  (\lift{x}{R}) \psubstp{Q}{P}  
  :=
  \lift{(x)\substp{Q}{P}}{ R \psubstp{Q}{P} } \\
%   (\dropn{x})  \psubstp{Q}{P}       
%   := 
%   \left\{ 
%     \begin{array}{ccc} 
%       \dropn{\quotep{Q}} & & x \nameeq \quotep{P} \\
%       \dropn{x} & & otherwise \\
%     \end{array}
%   \right. 
  (\dropn{x})  \psubstp{Q}{P}       
  := 
  \left\{ 
    \begin{array}{ccc} 
      Q & & x \nameeq \quotep{P} \\
      \dropn{x} & & otherwise \\
    \end{array}
  \right.
\end{mathpar}
 

where

\begin{eqnarray}
  (x)\id{\{} \lpquote Q \rpquote / \lpquote P \rpquote \id{\}}            = 
  \left\{ 
    \begin{array}{ccc}
      \lpquote Q \rpquote & & x \nameeq \lpquote P \rpquote \\
      x & & otherwise \\
    \end{array}
  \right. \nonumber
\end{eqnarray}

and $z$ is chosen distinct from $\quotep{P}$, $\quotep{Q}$, the free
names in $Q$, and all the names in $R$. Our $\alpha$-equivalence will
be built in the standard way from this substitution.

\begin{remark}\label{rem:no_self_referential_names}
  One consequence of these definitions is that $\forall P. \quotep{P}
  \not\in \freenames{P}$.
\end{remark}

\subsection{ Dynamic quote: an example }

Anticipating something of what's to come, consider applying the
substitution, $\widehat{\id{\{}u / z \id{\}}}$, to the following pair
of processes, $\lift{w}{y!(z)}$ and $w[ \lpquote y!(z) \rpquote ]$.

\begin{eqnarray}
	\lift{w}{y!(z)}\widehat{\id{\{}u / z \id{\}}}
		& = &
		\lift{w}{y!(u)} \nonumber\\
	w[ \lpquote y!(z) \rpquote ] \widehat{ \id{\{}u / z \id{\}} }
		& = &
		w[ \lpquote y!(z) \rpquote ] \nonumber
\end{eqnarray}

Because the body of the process between quotes is impervious to
substitution, we get radically different answers. In fact, by
examining the first process in an input context,
e.g. $x?(z).\lift{w}{y!(z)}$, we see that the process under the lift
operator may be shaped by prefixed inputs binding a name inside it. In
this sense, the lift operator will be seen as a way to dynamically
construct processes before reifying them as names.

Finally equipped with these standard features we can present the
dynamics of the calculus.

\subsubsection{Operational semantics} 

Finally, we introduce the computational dynamics. What marks these
algebras as distinct from other more traditionally studied algebraic
structures, e.g. vector spaces or polynomial rings, is the manner in
which dynamics is captured. In traditional structures, dynamics is typically
expressed through morphisms between such structures, as in linear maps
between vector spaces or morphisms between rings. In algebras
associated with the semantics of computation, the dynamics is
expressed as part of the algebraic structure itself, through a
reduction reduction relation typically denoted by $\red$. Below, we
give a recursive presentation of this relation for the calculus used
in the encoding.

$\red \subseteq \pi \times \pi$
$\red : \pi \to \mathcal{P}(\pi)$

\begin{mathpar}
  \inferrule* [lab=Comm] { \textsf{match}( x_{src}, x_{trgt} ) } { x_{trgt}?(y)P \; | \; x_{src}!\langle {Q} \rangle \red P\{\quotep{Q}/y}\} }
  \and \\
  \inferrule* [lab=Par] {{P} \red {P}'} {{{P} | {Q}} \red {{P}' | {Q}}}
  \and
  \inferrule* [lab=Equiv]{{{P} \scong {P}'} \andalso {{P}' \red {Q}'} \andalso {{Q}' \scong {Q}}}{{P} \red {Q}}
\end{mathpar}

\begin{eqnarray*}
  match_{\equiv} (\quotep{P},\quotep{Q}) & := & P \equiv Q \\
  match_{\dagger}(\quotep{P},\quotep{Q}) & := & \forall R. P|Q \red^{*} R => R \red^{*} 0 \\
  match_{K}(\quotep{P},\quotep{Q}) & := & K \mbox{ for some context } K
\end{eqnarray*}

$u?(x)P | u!\langle Q \rangle \red P\{\quotep{Q}/x\}$

%We write $\wred$ for $\red^*$, and $P\red$ if $\exists Q $ such that $ P \red Q$.
We write $P\red$ if $\exists Q $ such that $ P \red Q$ and $P\not\red$, otherwise.

\section{Replication}

As mentioned before, it is known that replication (and hence
recursion) can be implemented in a higher-order process algebra
\cite{SangiorgiWalker}. As our first example of calculation with the
machinery thus far presented we give the construction explicitly in
the {\rhoc}.

\begin{eqnarray}
	D_{x} & := & \prefix{x}{y}{(\binpar{\outputp{x}{y}}{@{y}})} \nonumber\\
	\bangp_{x}{P} & := & \binpar{{x}!\langle{\binpar{D_{x}}{P}}\rangle}{D_{x}} \nonumber
\end{eqnarray}

\begin{eqnarray}
	\bangp_{x}{P} & & \nonumber\\
	=
	& {x}!\langle{(\prefix{x}{y}{(\outputp{x}{y} | @{y})) | P}}\rangle 
	      | \prefix{x}{y}{(\outputp{x}{y} | @{y})} & \nonumber\\
	\red
	& (\outputp{x}{y} | @{y})\substn{\quotep{(\prefix{x}{y}{(@{y} | \outputp{x}{y})) | P}}}{y} & \nonumber\\
	=
	& \outputp{x}{\quotep{(\prefix{x}{y}{(\outputp{x}{y} | @{y})) | P}}}
	  | {(\prefix{x}{y}{(\outputp{x}{y} | @{y})) | P}} & \nonumber\\
	\red
	& \ldots & \nonumber\\
	\red^*
	& P | P | \ldots & \nonumber
\end{eqnarray}

Of course, this encoding, as an implementation, runs away, unfolding
$\bangp{P}$ eagerly. A lazier and more implementable replication
operator, restricted to input-guarded processes, may be obtained as follows.

\begin{eqnarray}
\bangp{\prefix{u}{v}{P}} 
	:= 
	\binpar{\lift{x}{\prefix{u}{v}{(\binpar{D(x)}{P})}}}{D(x)} \nonumber
\end{eqnarray}

\begin{remark}
  Note that the lazier definition still does not deal with summation
  or mixed summation (i.e. sums over input and output). The reader is
  invited to construct definitions of replication that deal with these
  features. 

  Further, the definitions are parameterized in a name, $x$. Can you,
  gentle reader, make a definition that eliminates this parameter and
  guarantees no accidental interaction between the replication
  machinery and the process being replicated -- i.e. no accidental
  sharing of names used by the process to get its work done and the
  name(s) used by the replication to effect copying. This latter
  revision of the definition of replication is crucial to obtaining
  the expected identity $!!P \sim !P$.
\end{remark}

\begin{remark}\label{rem:paradoxical_combinator}
  The reader familiar with the lambda calculus will have noticed the
  similarity between $D$ and the paradoxical combinator.

  [Ed. note: the existence of this seems to suggest we have to be more
  restrictive on the set of processes and names we admit if we are to
  support no-cloning.]
\end{remark}

\subsubsection{Bisimulation}

The computational dynamics gives rise to another kind of equivalence,
the equivalence of computational behavior. As previously mentioned
this is typically captured \emph{via} some form of bisimulation.

% The notion we use in this paper is weak barbed bisimulation
% \cite{milner91polyadicpi}.

The notion we use in this paper is derived from weak barbed
bisimulation \cite{milner91polyadicpi}. 

\begin{definition}
An \emph{observation relation}, $\downarrow_{\mathcal N}$, over a set
of names, $\mathcal N$, is the smallest relation satisfying the rules
below.

\infrule[Out-barb]{y \in {\mathcal N}, \; x \nameeq y}
		  {\outputp{x}{v} \downarrow_{\mathcal N} x}
\infrule[Par-barb]{\mbox{$P\downarrow_{\mathcal N} x$ or $Q\downarrow_{\mathcal N} x$}}
		  {\binpar{P}{Q} \downarrow_{\mathcal N} x}

We write $P \Downarrow_{\mathcal N} x$ if there is $Q$ such that 
$P \wred Q$ and $Q \downarrow_{\mathcal N} x$.
\end{definition}

\begin{definition}
%\label{def.bbisim}
An  ${\mathcal N}$-\emph{barbed bisimulation} over a set of names, ${\mathcal N}$, is a symmetric binary relation 
${\mathcal S}_{\mathcal N}$ between agents such that $P\rel{S}_{\mathcal N}Q$ implies:
\begin{enumerate}
\item If $P \red P'$ then $Q \wred Q'$ and $P'\rel{S}_{\mathcal N} Q'$.
\item If $P\downarrow_{\mathcal N} x$, then $Q\Downarrow_{\mathcal N} x$.
\end{enumerate}
$P$ is ${\mathcal N}$-barbed bisimilar to $Q$, written
$P \wbbisim_{\mathcal N} Q$, if $P \rel{S}_{\mathcal N} Q$ for some ${\mathcal N}$-barbed bisimulation ${\mathcal S}_{\mathcal N}$.
\end{definition}

$\mathcal{R} \subseteq \pi \times \pi$

$P \mathcal{R} Q => \forall P'. P \red P' \Rightarrow \exists Q'. Q \red Q', P' \mathcal{R} Q'$

$P \vdash x \Rightarrow Q \vdash x$

\begin{mathpar}
  \inferrule*[lab=Out-barb]{x \nameeq y}{{y}!\langle{Q}\rangle \vdash x}
  \and
  \inferrule*[lab=Par-barb]{\mbox{$P\vdash x$ or $Q\vdash x$}}{\binpar{P}{Q} \vdash x}
\end{mathpar}

\subsubsection{Contexts}

One of the principle advantages of computational calculi like the
$\pi$-calculus is a well-defined notion of context,
contextual-equivalence and a correlation between
contextual-equivalence and notions of bisimulation. The notion of
context allows the decomposition of a process into (sub-)process and
its syntactic environment, its context. Thus, a context may be
thought of as a process with a ``hole'' (written $\Box$) in it. The
application of a context $M$ to a process $P$, written $M[P]$, is
tantamount to filling the hole in $M$ with $P$. In this paper we do
not need the full weight of this theory, but do make use of the notion
of context in the proof the main theorem. 

\begin{mathpar}
  \inferrule* [lab=summation] {} {{M_{M},M_{N}} \bc \Box \;|\; x.M_{A} \;|\; M_{M}+M_{N}}
  \and
  \inferrule* [lab=agent] {} {{M_{A}} \bc (\vec{x})M_{P} \;| \; \clift{P_0,\ldots,M_{P},\ldots,P_N}}
  \and \\
  \inferrule* [lab=process] {} {{M_{P}} \bc M_{N} \;| \;P|M_{P} }
\end{mathpar} 

\begin{mathpar}
  \inferrule* [lab=sychronization] {} {M_{N} \bc \Box \;|\; x?M_{F} \;|\; x!M_{C}}
  \and
  \inferrule* [lab=abstraction] {} {{M_{F}} \bc (x)M_{P} }
  \and
  \inferrule* [lab=concretion] {} {{M_{C}} \bc \langle M_{P} \rangle }
  \and \\
  \inferrule* [lab=process] {} {{M_{P}} \bc M_{N} \;| \;P|M_{P} }
\end{mathpar}

\begin{definition}[contextual application] Given a context $M$, and
  process $P$, we define the \emph{contextual application}, $M[P] :=
  M\{P/\Box\}$. That is, the contextual application of M to P is the
  substitution of $P$ for $\Box$ in $M$.
\end{definition}

$\meaningof{-} : L \to \mathcal{P}(\pi)$

\begin{mathpar}
  \inferrule* [lab=collection] {} {\meaningof{true} = \pi, \and \meaningof{~E} = \pi \setminus \meaningof{E}, \and \meaningof{E_{1} \& E_{2}} = \meaningof{E_{1}} \cap \meaningof{E_{2}}}
\end{mathpar}

\begin{mathpar}
  \inferrule* [lab=structure] {} {\meaningof{0} = \{ P \in \pi | P \equiv 0 \}, \and \\ \meaningof{E_1 | E_2} = \{ P \in \pi | P \equiv P_{1} | P_{2}, P_{1} \in \meaningof{E_{1}}, P_{2} \in \meaningof{E_2}\} }
\end{mathpar}

\begin{mathpar}
 \inferrule* [lab=behavior] {} {\meaningof{\langle a?b \rangle E} = \{ P \in \pi | P \equiv Q | u?(y)P', \\ \and \\\\ \and \\ \;\;\; u \in \meaningof{a}, \forall z.P'\{z/y\} \in \meaningof{E\{z/b\}}\}, \and \\ \meaningof{a!E} = \{ P \in \pi | P \equiv Q | x!\langle P' \rangle, x \in \meaningof{a} P' \in \meaningof{E}\} }
\end{mathpar}

\begin{mathpar}
 \inferrule* [lab=nominal] {} {\meaningof{\quotep{E}} = \{ \quotep{P} \in \quotep{\pi} | P \in \meaningof{E} \}, \and \meaningof{\quotep{P}} = \{ \quotep{Q} \in \quotep{\pi} | P \equiv Q \} \and \\ \meaningof{@\quotep{E}} = \{ P \in \pi | P \equiv @x, x \in \meaningof{E} \}}
\end{mathpar}

\begin{eqnarray*}
  \\
  \meaningof{-} : TS \to ST
\end{eqnarray*}

\begin{eqnarray*}
  \\
  L : TS \to ST
\end{eqnarray*}

\begin{eqnarray*}
  \\
  P \models E \iff P \in \meaningof{E}
\end{eqnarray*}

\begin{eqnarray*}
  P \approx_{L} Q \iff \forall E \in L. P \models E \iff Q \models E
\end{eqnarray*}

\begin{eqnarray*}
  P \approx_{K} Q
\end{eqnarray*}

\begin{eqnarray*}
  P \approx Q
\end{eqnarray*}

$\approx_{K} = \approx = \approx_{L}$

\subsubsection{Contextual duality}

Note that contexts extend the quotation operation to a family of
operations from processes to names. Given a context, $M$, we can
define a \emph{nominal context}, $\quotep{M}$ by $\quotep{M}[P] :=
\quotep{M[P]}$. To foreshadow what is to come we observe that these
operations enjoy a duality with processes very much like the duality
between vectors and maps from vectors to scalars.

Further, because the calculus is essentially higher-order, we have a
correspondence between contexts and processes. More specifically,
given a name $x$ and a context $M$ we can construct $M^{*}_{x}$ such
that 

\begin{mathpar}
  M^{*}_{x} | \lift{x}{P} \red M[P]
\end{mathpar}

namely,

\begin{mathpar}
  M^{*}_{x} := x?(u).M[\dropn{u}]
\end{mathpar}

The dependence of $M^{*}_{x}$ on a name makes it an abstraction, 

\begin{mathpar}
  M^{*} := (x)x?(u).M[\dropn{u}]
\end{mathpar}

\subsection{Additional notation}

It will sometimes be convenient to denote the process a name
quotes. We already have the notation $x = \quotep{P}$, but it will be
convenient to introduce an alternate notation, $\procn{x}$, when we
want to emphasize the connection to the use of the name. Note that, by
virtue of name equivalence, $\quotep{\procn{x}} \nameeq x$; so, the
notation is consistent with previous definitions.

Further, because names have structure it is possible to effect
substitutions on the basis of that structure. This means we need to
upgrade our notation for substitutions, which we accomplish by
adapting comprehension notation. Thus,

\begin{mathpar}
  P\{ y / x : x \in S \}
\end{mathpar}

is interpreted to mean the process derived from P by replacing (in a
capture-avoiding manner) each occurrence of $x$ in $S$ by $y$. For example,

\begin{mathpar}
  P\{ \quotep{\procn{x}|\procn{x}} / x : x \in \freenames{P} \}
\end{mathpar}

will replace each (occurrence) of a free name $x$ in $P$ by
$\quotep{\procn{x}|\procn{x}}$.

Also, we will avail ourselves of the notation $x^{L}$ and $x^{R}$ to
denote injections of a name into disjoint copies of the name
space. There are numerous ways to accomplish this. One example can be
found in \cite{MeredithR05}. This notation overloads to vectors of
names: $\vec{x}^{\pi} := (x_{i}^{\pi} \; : \; 0 \leq i < |\vec{x}| )$ where $\pi \in \{L,R\}$.

We also use $P^{\Box} := P|\Box$.

In \cite{MeredithR05} an interpretation of the new operator is
given. It turns out that there are several possible interpretations
all enjoying the requisite algebraic properties of the operator (see
\cite{milner91polyadicpi}). We will therefore make liberal use of
$(\nu\; \vec{x})P$.

% subsection the_syntax_and_semantics_of_the_notation_system (end)   

\input{qm2pi.qmops} 

\input{qm2pi.sterngerlach} 

\input{qm2pi.metric} 

% section concurrent_process_calculi (end)

%\input{qm2pi.proofsketch}

% section proof sketch (end)

%\input{qm2pi.slviaknots} 

% section spatial logic via knots (end)

\input{qm2pi.conclusion}

% section conclusion (end)

%\input{qm2pi.dtcodes} 

% section wiring algorithm (end)

\input{qm2pi.ack} 

% section acknowledgments (end)

\newpage


\bibliographystyle{plain}   
\bibliography{../../biblios/main.bib}

\input{qm2pi.rhodetails}

\end{document}



% section proof sketch (end)

%\section{Unlikely characters: spatial logic for
  knots}\label{sub:characteristic_formulae} % (fold)

Associated to the mobile process calculi are a family of logics known
as the Hennessy-Milner logics. These logics typically enjoy a
semantics interpreting formulae as sets of processes that when
factored through the encoding outlined above allows an identification
of classes of knots with logical formulae. In the context of this
encoding the sub-family known as the spatial logics \cite{CairesC03}
\cite{CairesC04} \cite{Caires04} are of particular interest providing
several important features for expressing and reasoning about
properties (i.e. classes) of knots. We hint here at how this may be done.

%\begin{description}
%\item [structural connectives] 
\subsubsection{Structural connectives} The spatial logics enjoy
structural connectives corresponding, at the logical level, to the
parallel composition ($P | Q$) and new name ($(\nu \; x)P$)
connectives for processes. As illustrated in the examples below, these
connectives are extremely expressive given the shape of our encoding.
%\item [decideable satisfaction]

\subsubsection{Decideable satisfaction}
In \cite{Caires04} the satisfaction relation is shown to be decideable
for a rich class of processes. It further turns out that the image of
the our encoding is a proper subset of that class. This result
provides the basis for an algorithm by which to search for knots
enjoying a given property.
%\item [characteristic formulae]

\subsubsection{Characteristic formulae}
In the same paper \cite{Caires04} , Caires presents a means of calculating
characteristic formulae, selecting equivalence classes of processes
up to a pre--specified depth limit on the support set of names. Composed with our
encoding, this characteristic formula can be used to select
characteristic formulae for knots.
%\end{description}

\subsubsection{Spatial logic formulae}

The grammar below (segmented for comprehension) summarizes the syntax
of spatial logic formulae. We employ illustrative examples in the
sequel to provide an intuitive understanding of their meaning
referring the reader to \cite{Caires04} for a more detailed explication
of the semantics.

\begin{mathpar}
  \inferrule* [lab=boolean] {} {{A,B} \bc T \;|\; \neg A \;|\; A \wedge B \;|\; \eta = \eta'}
  \and
  \inferrule* [lab=spatial] {} {|\; \pzero \;|\; A | B \;|\; x \text{\textregistered} A \;|\; \forall x . A \;|\;  H x . A}
  \and
  \inferrule* [lab=behavioral] {} {|\; \alpha . A}
  \and 
  \inferrule* [lab=recursion] {} {|\; X(\vec{u}) \;|\; \mu X(\vec{u}) . A}
  \and
  \inferrule* [lab=action] {} {\alpha \bc \langle x?(\vec{y}) \rangle \;|\; \langle x!(\vec{y}) \rangle \;|\; \langle \tau \rangle}
  \and 
  \inferrule* [lab=name] {} {\eta \bc x \;|\; \tau}
\end{mathpar} 

% subsection characteristic_formulae (end)   	 

\subsection{Example formulae}\label{sub:example_formulae_} % (fold)

\subsubsection{Crossing as formula.}
% 
% \begin{align*}
%   \frac{d}{dx} \sin x &= \cos x 
%   & \frac{d}{dx} e^x &= e^x \\
%   \frac{d}{dx} \cos x &= - \sin x 
%   & \frac{d}{dx} \log x &= \frac{1}{x} \\
% \end{align*} 

\begin{align*}
 \mu C(x_{0},x_{1},y_{0},y_{1},u).&(\langle x_{0}?(z) \rangle(\langle u! \rangle\langle y_{1}!z \rangle C(x_{0},x_{1},y_{0},y_{1},u)) & \\
  & \wedge \langle y_{1}?(z) \rangle (\langle u! \rangle \langle x_{0}!z \rangle C(x_{0},x_{1},y_{0},y_{1},u)) & \\
  & \wedge \langle x_{1}?(z) \rangle (\langle u? \rangle \langle y_{0}!z \rangle C(x_{0},x_{1},y_{0},y_{1},u)) & \\
  & \wedge \langle y_{0}?(z) \rangle (\langle u? \rangle \langle x_{1}!z \rangle C(x_{0},x_{1},y_{0},y_{1},u))) &
\end{align*}

The lexicographical similarity between the shape of this formulae and
the shape of definition of the process representing a crossing reveals
the intuitive meaning of this formulae. It describes the capabilities
of a process that has the right to represent a crossing. For example
it picks out processes that may perform an input on the port $x_0$ in
its initial menu of capabilities. What differentiates the formula
from the process, however, is that the crossing process is the
smallest candidate to satisfy the formula. Infinitely many other
processes -- with internal behavior hidden behind this interface, so
to speak -- also satisfy this formula. Even this simple formula,
then, can be seen to open a new view onto knots, providing a
computational interpretation of \emph{virtual} knots.

Note that this formula is derived by hand. A similar formula can be
derived by employing Caires' calculation of characteristic formula
\cite{Caires04} to the process representing a crossing. In light of
this discussion, we let
$\meaningof{C}_{\phi}(x0,x1,y0,y1,u)$ denote a formula specifying the
dynamics we wish to capture of a crossing. To guarantee we preserve
the shape of the interface and minimal semantics we demand that
$\meaningof{C}_{\phi}(x0,x1,y0,y1,u) \Rightarrow
\textbf{C}(x0,x1,y0,y1,u)$ where $\textbf{C}(x0,x1,y0,y1,u)$ denotes
the formula above.
                            
\subsubsection{Crossing number constraints.}
The moral content of the context lemma (Lemma \ref{context}) is that the notion of
``locality'' in the Reidemeister moves is effectively captured by the
parallel composition operator of the process calculus. This intuition
extends through the logic. Given a formula,
$\meaningof{C}_{\phi}(x0,x1,y0,y1,u)$, we can use the structural
connectives to specify constraints on crossing numbers, such as at
least $n$ crossings, or exactly $n$ crossings.
\begin{mathpar}
  \inferrule* [lab=at-least-n] {} { K^{\geq n}_{\phi}(\vec{xs},\vec{ys}) := \Pi_{i=0}^{n-1} Hu . \meaningof{C}_{\phi}(xs_i,ys_i,u) | T }
  \and 
  \inferrule* [lab=exactly-n] {} { K^{= n}_{\phi}(\vec{xs},\vec{ys}) := \Pi_{i=0}^{n-1} Hu . \meaningof{C}_{\phi}(xs_i,ys_i,u) | \neg (\forall x_0,y_0,x_1,y_1,u . \meaningof{C}_{\phi}(x_0,y_0,x_1,y_1,u) | T) }
\end{mathpar}

To round out this section, recall that the encoding of an $n$-crossing
knot decomposes into a parallel composition of $n$ \emph{copies} of a
crossing process together with a wiring harness. To specify different
knot classes with the same crossing number amounts to specifying
logical constraints on the wiring harness. In the interest of space,
we defer examples to a forthcoming paper. Suffice it to say that both
the conditions ``alternating knot'' and ``contains the tangle
corresponding to 5/3'' are expressible. For example, it is possible to
calculate the characteristic formula of a process corresponding to the
tangle 5/3 and conjoin it into the classifying formula via the
composition connective of the logic.

Finally, we wish to observe that it is entirely within reason to
contemplate a more domain-specific version of spatial logic tailored
to the shape of processes in the image of the encoding. Such a
domain-specific logic would have a better claim to the title formal
language of knot properties.

% subsection example_formulae_ (end)

% section knots_as_processes (end) 

% section spatial logic via knots (end)

\section{Conclusions and future work}

\paragraph{Testing physical space}
You, gentle reader, may wonder why of all the theorems to be proved
given this set up we pick the one above. In some sense it's hardly
central to quantum mechanics. We see it as central in the sense that
it firmly establishes a notion of physical space arising from a notion
of the equivalence of behavior. Relating bisimulation to a metric is a
big step forward, but one is faced with interpreting the relationship
of that metric space to something more physical. Quantum mechanical
notions of ``physical'' space are still far from intuitive, but by
relating this idea of distance as testing to calculations that predict
physical circumstances we are making a not insignificant step forward
toward an understanding of the physical space we inhabit as
essentially dynamic.

\paragraph{Effectivity and simulation}
One of the observations we have yet to make is that the entire program
spelled out here is effective. We have built various interpreters for
the reflective calculus at work in this interpretation. In principle,
then, we can simulate quantum mechanics on a computer. The place where
the simulation may lose fidelity is the infinitely branching summation
for the annihilator.

In this connection i also want to point out that the evaluation style
calculation of the inner product puts the non-determinism of the
summation right at the heart of measurement. This suggests that
Milner's original reduction-based formulation of the dynamics of his
calculi in terms of sums was not just notationally suggestive of a
notion of measure-and-continue but captured some significant part of
the physics.

\paragraph{Quantum continuations}
In light of this last observation i want to point out that the
predominant account of quantum mechanics is missing a key aspect of a
truly compositional story of the physical situation. In a real lab,
when a measurement is made the observation can be made to feed into
another device that then makes another measurement conditioned on the
results of the first. This means that after the superposition was
collapsed the entire experimental set up remained in
superposition. While QM offers a means of writing this down it doesn't
quite line up well with the well-trodden formulation of computation
and continuation that we see so succinctly expressed in Milner's
calculi. This suggests that there might be advantages to this account
of dynamics waiting to be explored.

\paragraph{Quantum logic}
In this connection, we also note that by virtue of having the
Hennessy-Milner construction, we can pull the construction through the
interpretation of QM. This gives us a natural candidate for a quantum
logic that enjoys an extremely tight connection with it's domain of
interpretation, making the construction much less ad hoc (rather it is
the image of functor!).

\paragraph{Quantum probabiity}
i have questions about the basis of the interpretation of inner
product as probability amplitude. In particular, using which
axiomatization of probability theory does the notion of probability
amplitude earn the right to be so dubbed? In other words, where is the
proof that the operation for calculating a probability amplitude (and
then squaring) satisfies the axioms of what it means to calculate a
probability? Even if such a proof exists (i have yet to find it in the
literature), i wonder if it might not be possible to turn things on
their heads. Can we view the calculation of the probability amplitude
as an axiomatization of probability? If so, then the definition we
give for calculating probability amplitude may provide the basis for
an \emph{effective} theory of probability.

\paragraph{Quantum vs ``biological'' information}
Finally, i want to conclude with a more philosophical observation. At
a recent workshop in which QM was a predominant topic i noticed
something about quantum information. The speaker was giving a riveting
discussion of axiomatic QM and showing how properties of ``no
cloning'' and ``no deleting'' emerged as consequences of the
axiomatization. Theorems of this form are necessary to give us a sense
of confidence that our axioms characterize the physical theory. What
struck me, though, was that if quantum information is neither erasable
nor replicable it is markedly different from \emph{life}. Two of the
things we know about life is that

\begin{itemize}
  \item it ends;
  \item to gain some measure of persistence, to transcend it's
    finitude it is imminently copyable.
\end{itemize}

Both of these qualities are summarized succinctly in the aphorism: all
flesh is grass. For me these two kinds of ``information'' -- call them
quantum and biological -- are end points on a spectrum of strategies
for persistence. At one end, we have those curious entities that enjoy
uniqueness and permanence; at the other, we have those who in the face
of a certain end and an uncertain present make a go of passing
something on. To me one of the more remarkable aspects of the latter
strategy is that in the presence of noise (and certain features of
copying) we get a kind of dynamism, a chance for improvement against a
given persistent condition.

% subsection other_calculi_other_bisimulations_and_geometry_as_behavior (end)




% section conclusion (end)

%\documentclass[12pt]{llncs}
%\documentclass{jktr}

\usepackage[pdftex]{hyperref}                   
\usepackage {listings}
\usepackage {mathpartir}
\usepackage{bcprules}
%\usepackage{listings}
                       
\usepackage{graphicx} 
%\usepackage[margins=2.5cm,nohead,nofoot]{geometry}
%\usepackage{geometry}
\usepackage{amsfonts}
\usepackage{amstext}
\usepackage{latexsym}
\usepackage{amssymb}
\usepackage{color}


%\include{myPreamble}
\include{qm2pi.local} 

%\ifpdf
%\usepackage[pdftex]{graphicx}
%\else
%\usepackage{graphicx}
%\fi

 % \ifpdf
%  \usepackage{pdfsync}
%  \if


%\title{Brief Article}
%\author{David F. Snyder}
%\author{L.G. Meredith}

%\address{Dept. of Math., Texas State University--San Marcos, San Marcos, TX 78666}
       
\pagestyle{empty}


\begin{document}

\lstset{language=[Objective]Caml,frame=shadowbox}

\input{qm2pi.front}

% section front matter (end)

\input{qm2pi.intro} 
 
% section introduction (end)

% \input{qm2pi.knotations} 

% section notation (end)

\input{qm2pi.process.calculi} 

% section concurrent_process_calculi_and_spatial_logics_ (end)
    
%\input{qm2pi.knots2pi} 

%\input{qm2pi.trefoil} 

%\input{qm2pi.mainthm} 

% subsection basic_interpretation (end)

%\input{qm2pi.rho.presentation} 
\subsection{The syntax and semantics of the notation system}\label{sub:the_syntax_and_semantics_of_the_notation_system} % (fold)

We now summarize a technical presentation of the calculus that
embodies our theory of dynamics. The typical presentation of such a
calculus follows the style of giving generators and relations on
them. The grammar, below, describing term constructors, freely
generates the set of processes, $\Proc$. This set is then quotiented
by a relation known as structural congruence and it is over this set
that the notion of dynamics is expressed. This presentation is
essentially that of \cite{MeredithR05} with the addition of
polyadicity and summation. For readability we have relegated some of
the technical subtleties to an appendix.

\subsubsection{Process grammar}\label{subsub:process_grammar}

\begin{mathpar}
  \inferrule* [lab=synchronization] {} {{M} \bc \pzero \;|\; x?F \;|\; x!C }
  \and
  \inferrule* [lab=abstraction] {} {{F} \bc (x)P}
  \and
  \inferrule* [lab=concretion] {} {{C} \bc \langle Q \rangle}
  \and
  \inferrule* [lab=process] {} {{P,Q} \bc M \;| \;P|Q \;|\; @{x}}
  \and
  \inferrule* [lab=name] {} {{x} \bc \quotep{P}}
\end{mathpar} 

Note that $\vec{x}$ (resp. $\vec{P}$) denotes a vector of names
(resp. processes) of length $|\vec{x}|$ (resp. $|\vec{P}|$). We adopt
the following useful abbreviations.

\begin{mathpar}
   x?(\vec{y}).P := x.(\vec{y})P \and  x\clift{\vec{P}} := x.\clift{\vec{P}}
   \and x!(y) := \lift{x}{\dropn{y}}
   \and \Pi_{i=0}^{n-1}P_i := P_0 | \ldots | P_{n-1}
\end{mathpar}

\subsubsection{Structural congruence}

\paragraph{Free and bound names and alpha-equivalence.} At the
core of structural equivalence is alpha-equivalence which identifies
process that are the same up to a change of variable. Formally, we
recognize the distinction between free and bound names. The free names
of a process, $\freenames{P}$, may be calculated recursively as
follows:

\begin{mathpar}
\freenames{\pzero} := \emptyset
  \and \\
  \freenames{x?(y).P} := \{ x \} \cup (\freenames{P} \setminus \{ y \})
  \and 
  \freenames{x!\langle P \rangle} := \{ x \} \cup \{ P \} 
  \and \\
  \freenames{P|Q} := \freenames{P} \cup \freenames{Q}
  \and \\
  \freenames{@{x}} := \{ x \}
\end{mathpar}

$\pi$
$\quotep{\pi}$

$\freenames{-} : \pi \to \mathcal{P}(\quotep{\pi})$

\begin{eqnarray*}
  \freenames{\pzero} & := & \emptyset \\
  \freenames{x?(y).P} & := & \{ x \} \cup (\freenames{P} \setminus \{ y \}) \\
  \freenames{x!\langle P \rangle} & := & \{ x \} \cup \{ P \} \\
  \freenames{P|Q} & := & \freenames{P} \cup \freenames{Q} \\
  \freenames{\dropn{x}} & := & \{ x \}
\end{eqnarray*}

The bound names of a process, $\boundnames{P}$, are those names occurring in $P$
that are not free. For example, in $x?(y).0$, the name $x$ is free, while $y$ is bound.

\begin{mathpar}
  \inferrule* [lab=monoidal-laws] {} { P|Q \equiv Q|P \and P|0 \equiv P \and P|(Q|R) \equiv (P|Q)|R }
\end{mathpar}

\begin{mathpar}
  \inferrule* [lab=alpha-equivalence] {} { (x)P \equiv (y)P\{y/x\} \and y \not\in \freenames{P} }
\end{mathpar}

\begin{definition}
Then two processes, $P,Q$, are alpha-equivalent if $P = Q\{\vec{y}/\vec{x}\}$ for
some $\vec{x} \in \boundnames{Q},\vec{y} \in \boundnames{P}$, where $Q\{\vec{y}/\vec{x}\}$
denotes the capture-avoiding substitution of $\vec{y}$ for $\vec{x}$ in $Q$.
\end{definition}

\begin{definition}
  The {\em structural congruence} \cite{SangiorgiWalker} , $\equiv$,
  between processes is the least congruence containing
  alpha-equivalence, satisfying the abelian monoid laws
  (associativity, commutativity and $\pzero$ as identity) for parallel
  composition $|$ and for summation $+$.
\end{definition}

\subsection{Name equivalence}

We take name equivalence, written $\nameeq$, to be the smallest
equivalence relation generated by the following rules.

\begin{mathpar}
\inferrule*[lab=Quote-drop]
{ }
{ \quotep{@{x}} \nameeq x }

\inferrule*[lab=Struct-equiv]
{ P \scong Q }
{ \quotep{P} \nameeq \quotep{Q} }
\end{mathpar}

The astute reader will have noticed that the mutual recursion of names
and processes imposes a mutual recursion on alpha-equivalence and
structural equivalence via name-equivalence. Fortunately, all of this
works out pleasantly and we may calculate in the natural way, free of
concern. The reader interested in the details is referred to the
appendix \ref{appendix:rho_details}.

\subsection{Substitution}

We use $\Proc$ for the set of processes, $\QProc$ for the set of
names, and $\id{\{}\vec{y} / \vec{x} \id{\}}$ to denote partial maps,
$s : \QProc \rightarrow \QProc$. A map, $s$ lifts, uniquely, to a map
on process terms, $\widehat{s} : \Proc \rightarrow \Proc$ by the
following equations.

\begin{mathpar}
  (0) \psubstp{Q}{P} := 0 \\
  (R \juxtap S) \psubstp{Q}{P}
  :=    
  (R)\psubstp{Q}{P} \juxtap (S) \psubstp{Q}{P} \\
  (x?(y).R) \psubstp{Q}{P}    
  :=    
  (x)\substp{Q}{P} (z)\concat( (R \psubstn{z}{y}) \psubstp{Q}{P} ) \\
  (\lift{x}{R}) \psubstp{Q}{P}  
  :=
  \lift{(x)\substp{Q}{P}}{ R \psubstp{Q}{P} } \\
%   (\dropn{x})  \psubstp{Q}{P}       
%   := 
%   \left\{ 
%     \begin{array}{ccc} 
%       \dropn{\quotep{Q}} & & x \nameeq \quotep{P} \\
%       \dropn{x} & & otherwise \\
%     \end{array}
%   \right. 
  (\dropn{x})  \psubstp{Q}{P}       
  := 
  \left\{ 
    \begin{array}{ccc} 
      Q & & x \nameeq \quotep{P} \\
      \dropn{x} & & otherwise \\
    \end{array}
  \right.
\end{mathpar}
 

where

\begin{eqnarray}
  (x)\id{\{} \lpquote Q \rpquote / \lpquote P \rpquote \id{\}}            = 
  \left\{ 
    \begin{array}{ccc}
      \lpquote Q \rpquote & & x \nameeq \lpquote P \rpquote \\
      x & & otherwise \\
    \end{array}
  \right. \nonumber
\end{eqnarray}

and $z$ is chosen distinct from $\quotep{P}$, $\quotep{Q}$, the free
names in $Q$, and all the names in $R$. Our $\alpha$-equivalence will
be built in the standard way from this substitution.

\begin{remark}\label{rem:no_self_referential_names}
  One consequence of these definitions is that $\forall P. \quotep{P}
  \not\in \freenames{P}$.
\end{remark}

\subsection{ Dynamic quote: an example }

Anticipating something of what's to come, consider applying the
substitution, $\widehat{\id{\{}u / z \id{\}}}$, to the following pair
of processes, $\lift{w}{y!(z)}$ and $w[ \lpquote y!(z) \rpquote ]$.

\begin{eqnarray}
	\lift{w}{y!(z)}\widehat{\id{\{}u / z \id{\}}}
		& = &
		\lift{w}{y!(u)} \nonumber\\
	w[ \lpquote y!(z) \rpquote ] \widehat{ \id{\{}u / z \id{\}} }
		& = &
		w[ \lpquote y!(z) \rpquote ] \nonumber
\end{eqnarray}

Because the body of the process between quotes is impervious to
substitution, we get radically different answers. In fact, by
examining the first process in an input context,
e.g. $x?(z).\lift{w}{y!(z)}$, we see that the process under the lift
operator may be shaped by prefixed inputs binding a name inside it. In
this sense, the lift operator will be seen as a way to dynamically
construct processes before reifying them as names.

Finally equipped with these standard features we can present the
dynamics of the calculus.

\subsubsection{Operational semantics} 

Finally, we introduce the computational dynamics. What marks these
algebras as distinct from other more traditionally studied algebraic
structures, e.g. vector spaces or polynomial rings, is the manner in
which dynamics is captured. In traditional structures, dynamics is typically
expressed through morphisms between such structures, as in linear maps
between vector spaces or morphisms between rings. In algebras
associated with the semantics of computation, the dynamics is
expressed as part of the algebraic structure itself, through a
reduction reduction relation typically denoted by $\red$. Below, we
give a recursive presentation of this relation for the calculus used
in the encoding.

$\red \subseteq \pi \times \pi$
$\red : \pi \to \mathcal{P}(\pi)$

\begin{mathpar}
  \inferrule* [lab=Comm] { \textsf{match}( x_{src}, x_{trgt} ) } { x_{trgt}?(y)P \; | \; x_{src}!\langle {Q} \rangle \red P\{\quotep{Q}/y}\} }
  \and \\
  \inferrule* [lab=Par] {{P} \red {P}'} {{{P} | {Q}} \red {{P}' | {Q}}}
  \and
  \inferrule* [lab=Equiv]{{{P} \scong {P}'} \andalso {{P}' \red {Q}'} \andalso {{Q}' \scong {Q}}}{{P} \red {Q}}
\end{mathpar}

\begin{eqnarray*}
  match_{\equiv} (\quotep{P},\quotep{Q}) & := & P \equiv Q \\
  match_{\dagger}(\quotep{P},\quotep{Q}) & := & \forall R. P|Q \red^{*} R => R \red^{*} 0 \\
  match_{K}(\quotep{P},\quotep{Q}) & := & K \mbox{ for some context } K
\end{eqnarray*}

$u?(x)P | u!\langle Q \rangle \red P\{\quotep{Q}/x\}$

%We write $\wred$ for $\red^*$, and $P\red$ if $\exists Q $ such that $ P \red Q$.
We write $P\red$ if $\exists Q $ such that $ P \red Q$ and $P\not\red$, otherwise.

\section{Replication}

As mentioned before, it is known that replication (and hence
recursion) can be implemented in a higher-order process algebra
\cite{SangiorgiWalker}. As our first example of calculation with the
machinery thus far presented we give the construction explicitly in
the {\rhoc}.

\begin{eqnarray}
	D_{x} & := & \prefix{x}{y}{(\binpar{\outputp{x}{y}}{@{y}})} \nonumber\\
	\bangp_{x}{P} & := & \binpar{{x}!\langle{\binpar{D_{x}}{P}}\rangle}{D_{x}} \nonumber
\end{eqnarray}

\begin{eqnarray}
	\bangp_{x}{P} & & \nonumber\\
	=
	& {x}!\langle{(\prefix{x}{y}{(\outputp{x}{y} | @{y})) | P}}\rangle 
	      | \prefix{x}{y}{(\outputp{x}{y} | @{y})} & \nonumber\\
	\red
	& (\outputp{x}{y} | @{y})\substn{\quotep{(\prefix{x}{y}{(@{y} | \outputp{x}{y})) | P}}}{y} & \nonumber\\
	=
	& \outputp{x}{\quotep{(\prefix{x}{y}{(\outputp{x}{y} | @{y})) | P}}}
	  | {(\prefix{x}{y}{(\outputp{x}{y} | @{y})) | P}} & \nonumber\\
	\red
	& \ldots & \nonumber\\
	\red^*
	& P | P | \ldots & \nonumber
\end{eqnarray}

Of course, this encoding, as an implementation, runs away, unfolding
$\bangp{P}$ eagerly. A lazier and more implementable replication
operator, restricted to input-guarded processes, may be obtained as follows.

\begin{eqnarray}
\bangp{\prefix{u}{v}{P}} 
	:= 
	\binpar{\lift{x}{\prefix{u}{v}{(\binpar{D(x)}{P})}}}{D(x)} \nonumber
\end{eqnarray}

\begin{remark}
  Note that the lazier definition still does not deal with summation
  or mixed summation (i.e. sums over input and output). The reader is
  invited to construct definitions of replication that deal with these
  features. 

  Further, the definitions are parameterized in a name, $x$. Can you,
  gentle reader, make a definition that eliminates this parameter and
  guarantees no accidental interaction between the replication
  machinery and the process being replicated -- i.e. no accidental
  sharing of names used by the process to get its work done and the
  name(s) used by the replication to effect copying. This latter
  revision of the definition of replication is crucial to obtaining
  the expected identity $!!P \sim !P$.
\end{remark}

\begin{remark}\label{rem:paradoxical_combinator}
  The reader familiar with the lambda calculus will have noticed the
  similarity between $D$ and the paradoxical combinator.

  [Ed. note: the existence of this seems to suggest we have to be more
  restrictive on the set of processes and names we admit if we are to
  support no-cloning.]
\end{remark}

\subsubsection{Bisimulation}

The computational dynamics gives rise to another kind of equivalence,
the equivalence of computational behavior. As previously mentioned
this is typically captured \emph{via} some form of bisimulation.

% The notion we use in this paper is weak barbed bisimulation
% \cite{milner91polyadicpi}.

The notion we use in this paper is derived from weak barbed
bisimulation \cite{milner91polyadicpi}. 

\begin{definition}
An \emph{observation relation}, $\downarrow_{\mathcal N}$, over a set
of names, $\mathcal N$, is the smallest relation satisfying the rules
below.

\infrule[Out-barb]{y \in {\mathcal N}, \; x \nameeq y}
		  {\outputp{x}{v} \downarrow_{\mathcal N} x}
\infrule[Par-barb]{\mbox{$P\downarrow_{\mathcal N} x$ or $Q\downarrow_{\mathcal N} x$}}
		  {\binpar{P}{Q} \downarrow_{\mathcal N} x}

We write $P \Downarrow_{\mathcal N} x$ if there is $Q$ such that 
$P \wred Q$ and $Q \downarrow_{\mathcal N} x$.
\end{definition}

\begin{definition}
%\label{def.bbisim}
An  ${\mathcal N}$-\emph{barbed bisimulation} over a set of names, ${\mathcal N}$, is a symmetric binary relation 
${\mathcal S}_{\mathcal N}$ between agents such that $P\rel{S}_{\mathcal N}Q$ implies:
\begin{enumerate}
\item If $P \red P'$ then $Q \wred Q'$ and $P'\rel{S}_{\mathcal N} Q'$.
\item If $P\downarrow_{\mathcal N} x$, then $Q\Downarrow_{\mathcal N} x$.
\end{enumerate}
$P$ is ${\mathcal N}$-barbed bisimilar to $Q$, written
$P \wbbisim_{\mathcal N} Q$, if $P \rel{S}_{\mathcal N} Q$ for some ${\mathcal N}$-barbed bisimulation ${\mathcal S}_{\mathcal N}$.
\end{definition}

$\mathcal{R} \subseteq \pi \times \pi$

$P \mathcal{R} Q => \forall P'. P \red P' \Rightarrow \exists Q'. Q \red Q', P' \mathcal{R} Q'$

$P \vdash x \Rightarrow Q \vdash x$

\begin{mathpar}
  \inferrule*[lab=Out-barb]{x \nameeq y}{{y}!\langle{Q}\rangle \vdash x}
  \and
  \inferrule*[lab=Par-barb]{\mbox{$P\vdash x$ or $Q\vdash x$}}{\binpar{P}{Q} \vdash x}
\end{mathpar}

\subsubsection{Contexts}

One of the principle advantages of computational calculi like the
$\pi$-calculus is a well-defined notion of context,
contextual-equivalence and a correlation between
contextual-equivalence and notions of bisimulation. The notion of
context allows the decomposition of a process into (sub-)process and
its syntactic environment, its context. Thus, a context may be
thought of as a process with a ``hole'' (written $\Box$) in it. The
application of a context $M$ to a process $P$, written $M[P]$, is
tantamount to filling the hole in $M$ with $P$. In this paper we do
not need the full weight of this theory, but do make use of the notion
of context in the proof the main theorem. 

\begin{mathpar}
  \inferrule* [lab=summation] {} {{M_{M},M_{N}} \bc \Box \;|\; x.M_{A} \;|\; M_{M}+M_{N}}
  \and
  \inferrule* [lab=agent] {} {{M_{A}} \bc (\vec{x})M_{P} \;| \; \clift{P_0,\ldots,M_{P},\ldots,P_N}}
  \and \\
  \inferrule* [lab=process] {} {{M_{P}} \bc M_{N} \;| \;P|M_{P} }
\end{mathpar} 

\begin{mathpar}
  \inferrule* [lab=sychronization] {} {M_{N} \bc \Box \;|\; x?M_{F} \;|\; x!M_{C}}
  \and
  \inferrule* [lab=abstraction] {} {{M_{F}} \bc (x)M_{P} }
  \and
  \inferrule* [lab=concretion] {} {{M_{C}} \bc \langle M_{P} \rangle }
  \and \\
  \inferrule* [lab=process] {} {{M_{P}} \bc M_{N} \;| \;P|M_{P} }
\end{mathpar}

\begin{definition}[contextual application] Given a context $M$, and
  process $P$, we define the \emph{contextual application}, $M[P] :=
  M\{P/\Box\}$. That is, the contextual application of M to P is the
  substitution of $P$ for $\Box$ in $M$.
\end{definition}

$\meaningof{-} : L \to \mathcal{P}(\pi)$

\begin{mathpar}
  \inferrule* [lab=collection] {} {\meaningof{true} = \pi, \and \meaningof{~E} = \pi \setminus \meaningof{E}, \and \meaningof{E_{1} \& E_{2}} = \meaningof{E_{1}} \cap \meaningof{E_{2}}}
\end{mathpar}

\begin{mathpar}
  \inferrule* [lab=structure] {} {\meaningof{0} = \{ P \in \pi | P \equiv 0 \}, \and \\ \meaningof{E_1 | E_2} = \{ P \in \pi | P \equiv P_{1} | P_{2}, P_{1} \in \meaningof{E_{1}}, P_{2} \in \meaningof{E_2}\} }
\end{mathpar}

\begin{mathpar}
 \inferrule* [lab=behavior] {} {\meaningof{\langle a?b \rangle E} = \{ P \in \pi | P \equiv Q | u?(y)P', \\ \and \\\\ \and \\ \;\;\; u \in \meaningof{a}, \forall z.P'\{z/y\} \in \meaningof{E\{z/b\}}\}, \and \\ \meaningof{a!E} = \{ P \in \pi | P \equiv Q | x!\langle P' \rangle, x \in \meaningof{a} P' \in \meaningof{E}\} }
\end{mathpar}

\begin{mathpar}
 \inferrule* [lab=nominal] {} {\meaningof{\quotep{E}} = \{ \quotep{P} \in \quotep{\pi} | P \in \meaningof{E} \}, \and \meaningof{\quotep{P}} = \{ \quotep{Q} \in \quotep{\pi} | P \equiv Q \} \and \\ \meaningof{@\quotep{E}} = \{ P \in \pi | P \equiv @x, x \in \meaningof{E} \}}
\end{mathpar}

\begin{eqnarray*}
  \\
  \meaningof{-} : TS \to ST
\end{eqnarray*}

\begin{eqnarray*}
  \\
  L : TS \to ST
\end{eqnarray*}

\begin{eqnarray*}
  \\
  P \models E \iff P \in \meaningof{E}
\end{eqnarray*}

\begin{eqnarray*}
  P \approx_{L} Q \iff \forall E \in L. P \models E \iff Q \models E
\end{eqnarray*}

\begin{eqnarray*}
  P \approx_{K} Q
\end{eqnarray*}

\begin{eqnarray*}
  P \approx Q
\end{eqnarray*}

$\approx_{K} = \approx = \approx_{L}$

\subsubsection{Contextual duality}

Note that contexts extend the quotation operation to a family of
operations from processes to names. Given a context, $M$, we can
define a \emph{nominal context}, $\quotep{M}$ by $\quotep{M}[P] :=
\quotep{M[P]}$. To foreshadow what is to come we observe that these
operations enjoy a duality with processes very much like the duality
between vectors and maps from vectors to scalars.

Further, because the calculus is essentially higher-order, we have a
correspondence between contexts and processes. More specifically,
given a name $x$ and a context $M$ we can construct $M^{*}_{x}$ such
that 

\begin{mathpar}
  M^{*}_{x} | \lift{x}{P} \red M[P]
\end{mathpar}

namely,

\begin{mathpar}
  M^{*}_{x} := x?(u).M[\dropn{u}]
\end{mathpar}

The dependence of $M^{*}_{x}$ on a name makes it an abstraction, 

\begin{mathpar}
  M^{*} := (x)x?(u).M[\dropn{u}]
\end{mathpar}

\subsection{Additional notation}

It will sometimes be convenient to denote the process a name
quotes. We already have the notation $x = \quotep{P}$, but it will be
convenient to introduce an alternate notation, $\procn{x}$, when we
want to emphasize the connection to the use of the name. Note that, by
virtue of name equivalence, $\quotep{\procn{x}} \nameeq x$; so, the
notation is consistent with previous definitions.

Further, because names have structure it is possible to effect
substitutions on the basis of that structure. This means we need to
upgrade our notation for substitutions, which we accomplish by
adapting comprehension notation. Thus,

\begin{mathpar}
  P\{ y / x : x \in S \}
\end{mathpar}

is interpreted to mean the process derived from P by replacing (in a
capture-avoiding manner) each occurrence of $x$ in $S$ by $y$. For example,

\begin{mathpar}
  P\{ \quotep{\procn{x}|\procn{x}} / x : x \in \freenames{P} \}
\end{mathpar}

will replace each (occurrence) of a free name $x$ in $P$ by
$\quotep{\procn{x}|\procn{x}}$.

Also, we will avail ourselves of the notation $x^{L}$ and $x^{R}$ to
denote injections of a name into disjoint copies of the name
space. There are numerous ways to accomplish this. One example can be
found in \cite{MeredithR05}. This notation overloads to vectors of
names: $\vec{x}^{\pi} := (x_{i}^{\pi} \; : \; 0 \leq i < |\vec{x}| )$ where $\pi \in \{L,R\}$.

We also use $P^{\Box} := P|\Box$.

In \cite{MeredithR05} an interpretation of the new operator is
given. It turns out that there are several possible interpretations
all enjoying the requisite algebraic properties of the operator (see
\cite{milner91polyadicpi}). We will therefore make liberal use of
$(\nu\; \vec{x})P$.

% subsection the_syntax_and_semantics_of_the_notation_system (end)   

\input{qm2pi.qmops} 

\input{qm2pi.sterngerlach} 

\input{qm2pi.metric} 

% section concurrent_process_calculi (end)

%\input{qm2pi.proofsketch}

% section proof sketch (end)

%\input{qm2pi.slviaknots} 

% section spatial logic via knots (end)

\input{qm2pi.conclusion}

% section conclusion (end)

%\input{qm2pi.dtcodes} 

% section wiring algorithm (end)

\input{qm2pi.ack} 

% section acknowledgments (end)

\newpage


\bibliographystyle{plain}   
\bibliography{../../biblios/main.bib}

\input{qm2pi.rhodetails}

\end{document}

 

% section wiring algorithm (end)

\documentclass[12pt]{llncs}
%\documentclass{jktr}

\usepackage[pdftex]{hyperref}                   
\usepackage {listings}
\usepackage {mathpartir}
\usepackage{bcprules}
%\usepackage{listings}
                       
\usepackage{graphicx} 
%\usepackage[margins=2.5cm,nohead,nofoot]{geometry}
%\usepackage{geometry}
\usepackage{amsfonts}
\usepackage{amstext}
\usepackage{latexsym}
\usepackage{amssymb}
\usepackage{color}


%\include{myPreamble}
\include{qm2pi.local} 

%\ifpdf
%\usepackage[pdftex]{graphicx}
%\else
%\usepackage{graphicx}
%\fi

 % \ifpdf
%  \usepackage{pdfsync}
%  \if


%\title{Brief Article}
%\author{David F. Snyder}
%\author{L.G. Meredith}

%\address{Dept. of Math., Texas State University--San Marcos, San Marcos, TX 78666}
       
\pagestyle{empty}


\begin{document}

\lstset{language=[Objective]Caml,frame=shadowbox}

\input{qm2pi.front}

% section front matter (end)

\input{qm2pi.intro} 
 
% section introduction (end)

% \input{qm2pi.knotations} 

% section notation (end)

\input{qm2pi.process.calculi} 

% section concurrent_process_calculi_and_spatial_logics_ (end)
    
%\input{qm2pi.knots2pi} 

%\input{qm2pi.trefoil} 

%\input{qm2pi.mainthm} 

% subsection basic_interpretation (end)

%\input{qm2pi.rho.presentation} 
\subsection{The syntax and semantics of the notation system}\label{sub:the_syntax_and_semantics_of_the_notation_system} % (fold)

We now summarize a technical presentation of the calculus that
embodies our theory of dynamics. The typical presentation of such a
calculus follows the style of giving generators and relations on
them. The grammar, below, describing term constructors, freely
generates the set of processes, $\Proc$. This set is then quotiented
by a relation known as structural congruence and it is over this set
that the notion of dynamics is expressed. This presentation is
essentially that of \cite{MeredithR05} with the addition of
polyadicity and summation. For readability we have relegated some of
the technical subtleties to an appendix.

\subsubsection{Process grammar}\label{subsub:process_grammar}

\begin{mathpar}
  \inferrule* [lab=synchronization] {} {{M} \bc \pzero \;|\; x?F \;|\; x!C }
  \and
  \inferrule* [lab=abstraction] {} {{F} \bc (x)P}
  \and
  \inferrule* [lab=concretion] {} {{C} \bc \langle Q \rangle}
  \and
  \inferrule* [lab=process] {} {{P,Q} \bc M \;| \;P|Q \;|\; @{x}}
  \and
  \inferrule* [lab=name] {} {{x} \bc \quotep{P}}
\end{mathpar} 

Note that $\vec{x}$ (resp. $\vec{P}$) denotes a vector of names
(resp. processes) of length $|\vec{x}|$ (resp. $|\vec{P}|$). We adopt
the following useful abbreviations.

\begin{mathpar}
   x?(\vec{y}).P := x.(\vec{y})P \and  x\clift{\vec{P}} := x.\clift{\vec{P}}
   \and x!(y) := \lift{x}{\dropn{y}}
   \and \Pi_{i=0}^{n-1}P_i := P_0 | \ldots | P_{n-1}
\end{mathpar}

\subsubsection{Structural congruence}

\paragraph{Free and bound names and alpha-equivalence.} At the
core of structural equivalence is alpha-equivalence which identifies
process that are the same up to a change of variable. Formally, we
recognize the distinction between free and bound names. The free names
of a process, $\freenames{P}$, may be calculated recursively as
follows:

\begin{mathpar}
\freenames{\pzero} := \emptyset
  \and \\
  \freenames{x?(y).P} := \{ x \} \cup (\freenames{P} \setminus \{ y \})
  \and 
  \freenames{x!\langle P \rangle} := \{ x \} \cup \{ P \} 
  \and \\
  \freenames{P|Q} := \freenames{P} \cup \freenames{Q}
  \and \\
  \freenames{@{x}} := \{ x \}
\end{mathpar}

$\pi$
$\quotep{\pi}$

$\freenames{-} : \pi \to \mathcal{P}(\quotep{\pi})$

\begin{eqnarray*}
  \freenames{\pzero} & := & \emptyset \\
  \freenames{x?(y).P} & := & \{ x \} \cup (\freenames{P} \setminus \{ y \}) \\
  \freenames{x!\langle P \rangle} & := & \{ x \} \cup \{ P \} \\
  \freenames{P|Q} & := & \freenames{P} \cup \freenames{Q} \\
  \freenames{\dropn{x}} & := & \{ x \}
\end{eqnarray*}

The bound names of a process, $\boundnames{P}$, are those names occurring in $P$
that are not free. For example, in $x?(y).0$, the name $x$ is free, while $y$ is bound.

\begin{mathpar}
  \inferrule* [lab=monoidal-laws] {} { P|Q \equiv Q|P \and P|0 \equiv P \and P|(Q|R) \equiv (P|Q)|R }
\end{mathpar}

\begin{mathpar}
  \inferrule* [lab=alpha-equivalence] {} { (x)P \equiv (y)P\{y/x\} \and y \not\in \freenames{P} }
\end{mathpar}

\begin{definition}
Then two processes, $P,Q$, are alpha-equivalent if $P = Q\{\vec{y}/\vec{x}\}$ for
some $\vec{x} \in \boundnames{Q},\vec{y} \in \boundnames{P}$, where $Q\{\vec{y}/\vec{x}\}$
denotes the capture-avoiding substitution of $\vec{y}$ for $\vec{x}$ in $Q$.
\end{definition}

\begin{definition}
  The {\em structural congruence} \cite{SangiorgiWalker} , $\equiv$,
  between processes is the least congruence containing
  alpha-equivalence, satisfying the abelian monoid laws
  (associativity, commutativity and $\pzero$ as identity) for parallel
  composition $|$ and for summation $+$.
\end{definition}

\subsection{Name equivalence}

We take name equivalence, written $\nameeq$, to be the smallest
equivalence relation generated by the following rules.

\begin{mathpar}
\inferrule*[lab=Quote-drop]
{ }
{ \quotep{@{x}} \nameeq x }

\inferrule*[lab=Struct-equiv]
{ P \scong Q }
{ \quotep{P} \nameeq \quotep{Q} }
\end{mathpar}

The astute reader will have noticed that the mutual recursion of names
and processes imposes a mutual recursion on alpha-equivalence and
structural equivalence via name-equivalence. Fortunately, all of this
works out pleasantly and we may calculate in the natural way, free of
concern. The reader interested in the details is referred to the
appendix \ref{appendix:rho_details}.

\subsection{Substitution}

We use $\Proc$ for the set of processes, $\QProc$ for the set of
names, and $\id{\{}\vec{y} / \vec{x} \id{\}}$ to denote partial maps,
$s : \QProc \rightarrow \QProc$. A map, $s$ lifts, uniquely, to a map
on process terms, $\widehat{s} : \Proc \rightarrow \Proc$ by the
following equations.

\begin{mathpar}
  (0) \psubstp{Q}{P} := 0 \\
  (R \juxtap S) \psubstp{Q}{P}
  :=    
  (R)\psubstp{Q}{P} \juxtap (S) \psubstp{Q}{P} \\
  (x?(y).R) \psubstp{Q}{P}    
  :=    
  (x)\substp{Q}{P} (z)\concat( (R \psubstn{z}{y}) \psubstp{Q}{P} ) \\
  (\lift{x}{R}) \psubstp{Q}{P}  
  :=
  \lift{(x)\substp{Q}{P}}{ R \psubstp{Q}{P} } \\
%   (\dropn{x})  \psubstp{Q}{P}       
%   := 
%   \left\{ 
%     \begin{array}{ccc} 
%       \dropn{\quotep{Q}} & & x \nameeq \quotep{P} \\
%       \dropn{x} & & otherwise \\
%     \end{array}
%   \right. 
  (\dropn{x})  \psubstp{Q}{P}       
  := 
  \left\{ 
    \begin{array}{ccc} 
      Q & & x \nameeq \quotep{P} \\
      \dropn{x} & & otherwise \\
    \end{array}
  \right.
\end{mathpar}
 

where

\begin{eqnarray}
  (x)\id{\{} \lpquote Q \rpquote / \lpquote P \rpquote \id{\}}            = 
  \left\{ 
    \begin{array}{ccc}
      \lpquote Q \rpquote & & x \nameeq \lpquote P \rpquote \\
      x & & otherwise \\
    \end{array}
  \right. \nonumber
\end{eqnarray}

and $z$ is chosen distinct from $\quotep{P}$, $\quotep{Q}$, the free
names in $Q$, and all the names in $R$. Our $\alpha$-equivalence will
be built in the standard way from this substitution.

\begin{remark}\label{rem:no_self_referential_names}
  One consequence of these definitions is that $\forall P. \quotep{P}
  \not\in \freenames{P}$.
\end{remark}

\subsection{ Dynamic quote: an example }

Anticipating something of what's to come, consider applying the
substitution, $\widehat{\id{\{}u / z \id{\}}}$, to the following pair
of processes, $\lift{w}{y!(z)}$ and $w[ \lpquote y!(z) \rpquote ]$.

\begin{eqnarray}
	\lift{w}{y!(z)}\widehat{\id{\{}u / z \id{\}}}
		& = &
		\lift{w}{y!(u)} \nonumber\\
	w[ \lpquote y!(z) \rpquote ] \widehat{ \id{\{}u / z \id{\}} }
		& = &
		w[ \lpquote y!(z) \rpquote ] \nonumber
\end{eqnarray}

Because the body of the process between quotes is impervious to
substitution, we get radically different answers. In fact, by
examining the first process in an input context,
e.g. $x?(z).\lift{w}{y!(z)}$, we see that the process under the lift
operator may be shaped by prefixed inputs binding a name inside it. In
this sense, the lift operator will be seen as a way to dynamically
construct processes before reifying them as names.

Finally equipped with these standard features we can present the
dynamics of the calculus.

\subsubsection{Operational semantics} 

Finally, we introduce the computational dynamics. What marks these
algebras as distinct from other more traditionally studied algebraic
structures, e.g. vector spaces or polynomial rings, is the manner in
which dynamics is captured. In traditional structures, dynamics is typically
expressed through morphisms between such structures, as in linear maps
between vector spaces or morphisms between rings. In algebras
associated with the semantics of computation, the dynamics is
expressed as part of the algebraic structure itself, through a
reduction reduction relation typically denoted by $\red$. Below, we
give a recursive presentation of this relation for the calculus used
in the encoding.

$\red \subseteq \pi \times \pi$
$\red : \pi \to \mathcal{P}(\pi)$

\begin{mathpar}
  \inferrule* [lab=Comm] { \textsf{match}( x_{src}, x_{trgt} ) } { x_{trgt}?(y)P \; | \; x_{src}!\langle {Q} \rangle \red P\{\quotep{Q}/y}\} }
  \and \\
  \inferrule* [lab=Par] {{P} \red {P}'} {{{P} | {Q}} \red {{P}' | {Q}}}
  \and
  \inferrule* [lab=Equiv]{{{P} \scong {P}'} \andalso {{P}' \red {Q}'} \andalso {{Q}' \scong {Q}}}{{P} \red {Q}}
\end{mathpar}

\begin{eqnarray*}
  match_{\equiv} (\quotep{P},\quotep{Q}) & := & P \equiv Q \\
  match_{\dagger}(\quotep{P},\quotep{Q}) & := & \forall R. P|Q \red^{*} R => R \red^{*} 0 \\
  match_{K}(\quotep{P},\quotep{Q}) & := & K \mbox{ for some context } K
\end{eqnarray*}

$u?(x)P | u!\langle Q \rangle \red P\{\quotep{Q}/x\}$

%We write $\wred$ for $\red^*$, and $P\red$ if $\exists Q $ such that $ P \red Q$.
We write $P\red$ if $\exists Q $ such that $ P \red Q$ and $P\not\red$, otherwise.

\section{Replication}

As mentioned before, it is known that replication (and hence
recursion) can be implemented in a higher-order process algebra
\cite{SangiorgiWalker}. As our first example of calculation with the
machinery thus far presented we give the construction explicitly in
the {\rhoc}.

\begin{eqnarray}
	D_{x} & := & \prefix{x}{y}{(\binpar{\outputp{x}{y}}{@{y}})} \nonumber\\
	\bangp_{x}{P} & := & \binpar{{x}!\langle{\binpar{D_{x}}{P}}\rangle}{D_{x}} \nonumber
\end{eqnarray}

\begin{eqnarray}
	\bangp_{x}{P} & & \nonumber\\
	=
	& {x}!\langle{(\prefix{x}{y}{(\outputp{x}{y} | @{y})) | P}}\rangle 
	      | \prefix{x}{y}{(\outputp{x}{y} | @{y})} & \nonumber\\
	\red
	& (\outputp{x}{y} | @{y})\substn{\quotep{(\prefix{x}{y}{(@{y} | \outputp{x}{y})) | P}}}{y} & \nonumber\\
	=
	& \outputp{x}{\quotep{(\prefix{x}{y}{(\outputp{x}{y} | @{y})) | P}}}
	  | {(\prefix{x}{y}{(\outputp{x}{y} | @{y})) | P}} & \nonumber\\
	\red
	& \ldots & \nonumber\\
	\red^*
	& P | P | \ldots & \nonumber
\end{eqnarray}

Of course, this encoding, as an implementation, runs away, unfolding
$\bangp{P}$ eagerly. A lazier and more implementable replication
operator, restricted to input-guarded processes, may be obtained as follows.

\begin{eqnarray}
\bangp{\prefix{u}{v}{P}} 
	:= 
	\binpar{\lift{x}{\prefix{u}{v}{(\binpar{D(x)}{P})}}}{D(x)} \nonumber
\end{eqnarray}

\begin{remark}
  Note that the lazier definition still does not deal with summation
  or mixed summation (i.e. sums over input and output). The reader is
  invited to construct definitions of replication that deal with these
  features. 

  Further, the definitions are parameterized in a name, $x$. Can you,
  gentle reader, make a definition that eliminates this parameter and
  guarantees no accidental interaction between the replication
  machinery and the process being replicated -- i.e. no accidental
  sharing of names used by the process to get its work done and the
  name(s) used by the replication to effect copying. This latter
  revision of the definition of replication is crucial to obtaining
  the expected identity $!!P \sim !P$.
\end{remark}

\begin{remark}\label{rem:paradoxical_combinator}
  The reader familiar with the lambda calculus will have noticed the
  similarity between $D$ and the paradoxical combinator.

  [Ed. note: the existence of this seems to suggest we have to be more
  restrictive on the set of processes and names we admit if we are to
  support no-cloning.]
\end{remark}

\subsubsection{Bisimulation}

The computational dynamics gives rise to another kind of equivalence,
the equivalence of computational behavior. As previously mentioned
this is typically captured \emph{via} some form of bisimulation.

% The notion we use in this paper is weak barbed bisimulation
% \cite{milner91polyadicpi}.

The notion we use in this paper is derived from weak barbed
bisimulation \cite{milner91polyadicpi}. 

\begin{definition}
An \emph{observation relation}, $\downarrow_{\mathcal N}$, over a set
of names, $\mathcal N$, is the smallest relation satisfying the rules
below.

\infrule[Out-barb]{y \in {\mathcal N}, \; x \nameeq y}
		  {\outputp{x}{v} \downarrow_{\mathcal N} x}
\infrule[Par-barb]{\mbox{$P\downarrow_{\mathcal N} x$ or $Q\downarrow_{\mathcal N} x$}}
		  {\binpar{P}{Q} \downarrow_{\mathcal N} x}

We write $P \Downarrow_{\mathcal N} x$ if there is $Q$ such that 
$P \wred Q$ and $Q \downarrow_{\mathcal N} x$.
\end{definition}

\begin{definition}
%\label{def.bbisim}
An  ${\mathcal N}$-\emph{barbed bisimulation} over a set of names, ${\mathcal N}$, is a symmetric binary relation 
${\mathcal S}_{\mathcal N}$ between agents such that $P\rel{S}_{\mathcal N}Q$ implies:
\begin{enumerate}
\item If $P \red P'$ then $Q \wred Q'$ and $P'\rel{S}_{\mathcal N} Q'$.
\item If $P\downarrow_{\mathcal N} x$, then $Q\Downarrow_{\mathcal N} x$.
\end{enumerate}
$P$ is ${\mathcal N}$-barbed bisimilar to $Q$, written
$P \wbbisim_{\mathcal N} Q$, if $P \rel{S}_{\mathcal N} Q$ for some ${\mathcal N}$-barbed bisimulation ${\mathcal S}_{\mathcal N}$.
\end{definition}

$\mathcal{R} \subseteq \pi \times \pi$

$P \mathcal{R} Q => \forall P'. P \red P' \Rightarrow \exists Q'. Q \red Q', P' \mathcal{R} Q'$

$P \vdash x \Rightarrow Q \vdash x$

\begin{mathpar}
  \inferrule*[lab=Out-barb]{x \nameeq y}{{y}!\langle{Q}\rangle \vdash x}
  \and
  \inferrule*[lab=Par-barb]{\mbox{$P\vdash x$ or $Q\vdash x$}}{\binpar{P}{Q} \vdash x}
\end{mathpar}

\subsubsection{Contexts}

One of the principle advantages of computational calculi like the
$\pi$-calculus is a well-defined notion of context,
contextual-equivalence and a correlation between
contextual-equivalence and notions of bisimulation. The notion of
context allows the decomposition of a process into (sub-)process and
its syntactic environment, its context. Thus, a context may be
thought of as a process with a ``hole'' (written $\Box$) in it. The
application of a context $M$ to a process $P$, written $M[P]$, is
tantamount to filling the hole in $M$ with $P$. In this paper we do
not need the full weight of this theory, but do make use of the notion
of context in the proof the main theorem. 

\begin{mathpar}
  \inferrule* [lab=summation] {} {{M_{M},M_{N}} \bc \Box \;|\; x.M_{A} \;|\; M_{M}+M_{N}}
  \and
  \inferrule* [lab=agent] {} {{M_{A}} \bc (\vec{x})M_{P} \;| \; \clift{P_0,\ldots,M_{P},\ldots,P_N}}
  \and \\
  \inferrule* [lab=process] {} {{M_{P}} \bc M_{N} \;| \;P|M_{P} }
\end{mathpar} 

\begin{mathpar}
  \inferrule* [lab=sychronization] {} {M_{N} \bc \Box \;|\; x?M_{F} \;|\; x!M_{C}}
  \and
  \inferrule* [lab=abstraction] {} {{M_{F}} \bc (x)M_{P} }
  \and
  \inferrule* [lab=concretion] {} {{M_{C}} \bc \langle M_{P} \rangle }
  \and \\
  \inferrule* [lab=process] {} {{M_{P}} \bc M_{N} \;| \;P|M_{P} }
\end{mathpar}

\begin{definition}[contextual application] Given a context $M$, and
  process $P$, we define the \emph{contextual application}, $M[P] :=
  M\{P/\Box\}$. That is, the contextual application of M to P is the
  substitution of $P$ for $\Box$ in $M$.
\end{definition}

$\meaningof{-} : L \to \mathcal{P}(\pi)$

\begin{mathpar}
  \inferrule* [lab=collection] {} {\meaningof{true} = \pi, \and \meaningof{~E} = \pi \setminus \meaningof{E}, \and \meaningof{E_{1} \& E_{2}} = \meaningof{E_{1}} \cap \meaningof{E_{2}}}
\end{mathpar}

\begin{mathpar}
  \inferrule* [lab=structure] {} {\meaningof{0} = \{ P \in \pi | P \equiv 0 \}, \and \\ \meaningof{E_1 | E_2} = \{ P \in \pi | P \equiv P_{1} | P_{2}, P_{1} \in \meaningof{E_{1}}, P_{2} \in \meaningof{E_2}\} }
\end{mathpar}

\begin{mathpar}
 \inferrule* [lab=behavior] {} {\meaningof{\langle a?b \rangle E} = \{ P \in \pi | P \equiv Q | u?(y)P', \\ \and \\\\ \and \\ \;\;\; u \in \meaningof{a}, \forall z.P'\{z/y\} \in \meaningof{E\{z/b\}}\}, \and \\ \meaningof{a!E} = \{ P \in \pi | P \equiv Q | x!\langle P' \rangle, x \in \meaningof{a} P' \in \meaningof{E}\} }
\end{mathpar}

\begin{mathpar}
 \inferrule* [lab=nominal] {} {\meaningof{\quotep{E}} = \{ \quotep{P} \in \quotep{\pi} | P \in \meaningof{E} \}, \and \meaningof{\quotep{P}} = \{ \quotep{Q} \in \quotep{\pi} | P \equiv Q \} \and \\ \meaningof{@\quotep{E}} = \{ P \in \pi | P \equiv @x, x \in \meaningof{E} \}}
\end{mathpar}

\begin{eqnarray*}
  \\
  \meaningof{-} : TS \to ST
\end{eqnarray*}

\begin{eqnarray*}
  \\
  L : TS \to ST
\end{eqnarray*}

\begin{eqnarray*}
  \\
  P \models E \iff P \in \meaningof{E}
\end{eqnarray*}

\begin{eqnarray*}
  P \approx_{L} Q \iff \forall E \in L. P \models E \iff Q \models E
\end{eqnarray*}

\begin{eqnarray*}
  P \approx_{K} Q
\end{eqnarray*}

\begin{eqnarray*}
  P \approx Q
\end{eqnarray*}

$\approx_{K} = \approx = \approx_{L}$

\subsubsection{Contextual duality}

Note that contexts extend the quotation operation to a family of
operations from processes to names. Given a context, $M$, we can
define a \emph{nominal context}, $\quotep{M}$ by $\quotep{M}[P] :=
\quotep{M[P]}$. To foreshadow what is to come we observe that these
operations enjoy a duality with processes very much like the duality
between vectors and maps from vectors to scalars.

Further, because the calculus is essentially higher-order, we have a
correspondence between contexts and processes. More specifically,
given a name $x$ and a context $M$ we can construct $M^{*}_{x}$ such
that 

\begin{mathpar}
  M^{*}_{x} | \lift{x}{P} \red M[P]
\end{mathpar}

namely,

\begin{mathpar}
  M^{*}_{x} := x?(u).M[\dropn{u}]
\end{mathpar}

The dependence of $M^{*}_{x}$ on a name makes it an abstraction, 

\begin{mathpar}
  M^{*} := (x)x?(u).M[\dropn{u}]
\end{mathpar}

\subsection{Additional notation}

It will sometimes be convenient to denote the process a name
quotes. We already have the notation $x = \quotep{P}$, but it will be
convenient to introduce an alternate notation, $\procn{x}$, when we
want to emphasize the connection to the use of the name. Note that, by
virtue of name equivalence, $\quotep{\procn{x}} \nameeq x$; so, the
notation is consistent with previous definitions.

Further, because names have structure it is possible to effect
substitutions on the basis of that structure. This means we need to
upgrade our notation for substitutions, which we accomplish by
adapting comprehension notation. Thus,

\begin{mathpar}
  P\{ y / x : x \in S \}
\end{mathpar}

is interpreted to mean the process derived from P by replacing (in a
capture-avoiding manner) each occurrence of $x$ in $S$ by $y$. For example,

\begin{mathpar}
  P\{ \quotep{\procn{x}|\procn{x}} / x : x \in \freenames{P} \}
\end{mathpar}

will replace each (occurrence) of a free name $x$ in $P$ by
$\quotep{\procn{x}|\procn{x}}$.

Also, we will avail ourselves of the notation $x^{L}$ and $x^{R}$ to
denote injections of a name into disjoint copies of the name
space. There are numerous ways to accomplish this. One example can be
found in \cite{MeredithR05}. This notation overloads to vectors of
names: $\vec{x}^{\pi} := (x_{i}^{\pi} \; : \; 0 \leq i < |\vec{x}| )$ where $\pi \in \{L,R\}$.

We also use $P^{\Box} := P|\Box$.

In \cite{MeredithR05} an interpretation of the new operator is
given. It turns out that there are several possible interpretations
all enjoying the requisite algebraic properties of the operator (see
\cite{milner91polyadicpi}). We will therefore make liberal use of
$(\nu\; \vec{x})P$.

% subsection the_syntax_and_semantics_of_the_notation_system (end)   

\input{qm2pi.qmops} 

\input{qm2pi.sterngerlach} 

\input{qm2pi.metric} 

% section concurrent_process_calculi (end)

%\input{qm2pi.proofsketch}

% section proof sketch (end)

%\input{qm2pi.slviaknots} 

% section spatial logic via knots (end)

\input{qm2pi.conclusion}

% section conclusion (end)

%\input{qm2pi.dtcodes} 

% section wiring algorithm (end)

\input{qm2pi.ack} 

% section acknowledgments (end)

\newpage


\bibliographystyle{plain}   
\bibliography{../../biblios/main.bib}

\input{qm2pi.rhodetails}

\end{document}

 

% section acknowledgments (end)

\newpage


\bibliographystyle{plain}   
\bibliography{../../biblios/main.bib}

\documentclass[12pt]{llncs}
%\documentclass{jktr}

\usepackage[pdftex]{hyperref}                   
\usepackage {listings}
\usepackage {mathpartir}
\usepackage{bcprules}
%\usepackage{listings}
                       
\usepackage{graphicx} 
%\usepackage[margins=2.5cm,nohead,nofoot]{geometry}
%\usepackage{geometry}
\usepackage{amsfonts}
\usepackage{amstext}
\usepackage{latexsym}
\usepackage{amssymb}
\usepackage{color}


%\include{myPreamble}
\include{qm2pi.local} 

%\ifpdf
%\usepackage[pdftex]{graphicx}
%\else
%\usepackage{graphicx}
%\fi

 % \ifpdf
%  \usepackage{pdfsync}
%  \if


%\title{Brief Article}
%\author{David F. Snyder}
%\author{L.G. Meredith}

%\address{Dept. of Math., Texas State University--San Marcos, San Marcos, TX 78666}
       
\pagestyle{empty}


\begin{document}

\lstset{language=[Objective]Caml,frame=shadowbox}

\input{qm2pi.front}

% section front matter (end)

\input{qm2pi.intro} 
 
% section introduction (end)

% \input{qm2pi.knotations} 

% section notation (end)

\input{qm2pi.process.calculi} 

% section concurrent_process_calculi_and_spatial_logics_ (end)
    
%\input{qm2pi.knots2pi} 

%\input{qm2pi.trefoil} 

%\input{qm2pi.mainthm} 

% subsection basic_interpretation (end)

%\input{qm2pi.rho.presentation} 
\subsection{The syntax and semantics of the notation system}\label{sub:the_syntax_and_semantics_of_the_notation_system} % (fold)

We now summarize a technical presentation of the calculus that
embodies our theory of dynamics. The typical presentation of such a
calculus follows the style of giving generators and relations on
them. The grammar, below, describing term constructors, freely
generates the set of processes, $\Proc$. This set is then quotiented
by a relation known as structural congruence and it is over this set
that the notion of dynamics is expressed. This presentation is
essentially that of \cite{MeredithR05} with the addition of
polyadicity and summation. For readability we have relegated some of
the technical subtleties to an appendix.

\subsubsection{Process grammar}\label{subsub:process_grammar}

\begin{mathpar}
  \inferrule* [lab=synchronization] {} {{M} \bc \pzero \;|\; x?F \;|\; x!C }
  \and
  \inferrule* [lab=abstraction] {} {{F} \bc (x)P}
  \and
  \inferrule* [lab=concretion] {} {{C} \bc \langle Q \rangle}
  \and
  \inferrule* [lab=process] {} {{P,Q} \bc M \;| \;P|Q \;|\; @{x}}
  \and
  \inferrule* [lab=name] {} {{x} \bc \quotep{P}}
\end{mathpar} 

Note that $\vec{x}$ (resp. $\vec{P}$) denotes a vector of names
(resp. processes) of length $|\vec{x}|$ (resp. $|\vec{P}|$). We adopt
the following useful abbreviations.

\begin{mathpar}
   x?(\vec{y}).P := x.(\vec{y})P \and  x\clift{\vec{P}} := x.\clift{\vec{P}}
   \and x!(y) := \lift{x}{\dropn{y}}
   \and \Pi_{i=0}^{n-1}P_i := P_0 | \ldots | P_{n-1}
\end{mathpar}

\subsubsection{Structural congruence}

\paragraph{Free and bound names and alpha-equivalence.} At the
core of structural equivalence is alpha-equivalence which identifies
process that are the same up to a change of variable. Formally, we
recognize the distinction between free and bound names. The free names
of a process, $\freenames{P}$, may be calculated recursively as
follows:

\begin{mathpar}
\freenames{\pzero} := \emptyset
  \and \\
  \freenames{x?(y).P} := \{ x \} \cup (\freenames{P} \setminus \{ y \})
  \and 
  \freenames{x!\langle P \rangle} := \{ x \} \cup \{ P \} 
  \and \\
  \freenames{P|Q} := \freenames{P} \cup \freenames{Q}
  \and \\
  \freenames{@{x}} := \{ x \}
\end{mathpar}

$\pi$
$\quotep{\pi}$

$\freenames{-} : \pi \to \mathcal{P}(\quotep{\pi})$

\begin{eqnarray*}
  \freenames{\pzero} & := & \emptyset \\
  \freenames{x?(y).P} & := & \{ x \} \cup (\freenames{P} \setminus \{ y \}) \\
  \freenames{x!\langle P \rangle} & := & \{ x \} \cup \{ P \} \\
  \freenames{P|Q} & := & \freenames{P} \cup \freenames{Q} \\
  \freenames{\dropn{x}} & := & \{ x \}
\end{eqnarray*}

The bound names of a process, $\boundnames{P}$, are those names occurring in $P$
that are not free. For example, in $x?(y).0$, the name $x$ is free, while $y$ is bound.

\begin{mathpar}
  \inferrule* [lab=monoidal-laws] {} { P|Q \equiv Q|P \and P|0 \equiv P \and P|(Q|R) \equiv (P|Q)|R }
\end{mathpar}

\begin{mathpar}
  \inferrule* [lab=alpha-equivalence] {} { (x)P \equiv (y)P\{y/x\} \and y \not\in \freenames{P} }
\end{mathpar}

\begin{definition}
Then two processes, $P,Q$, are alpha-equivalent if $P = Q\{\vec{y}/\vec{x}\}$ for
some $\vec{x} \in \boundnames{Q},\vec{y} \in \boundnames{P}$, where $Q\{\vec{y}/\vec{x}\}$
denotes the capture-avoiding substitution of $\vec{y}$ for $\vec{x}$ in $Q$.
\end{definition}

\begin{definition}
  The {\em structural congruence} \cite{SangiorgiWalker} , $\equiv$,
  between processes is the least congruence containing
  alpha-equivalence, satisfying the abelian monoid laws
  (associativity, commutativity and $\pzero$ as identity) for parallel
  composition $|$ and for summation $+$.
\end{definition}

\subsection{Name equivalence}

We take name equivalence, written $\nameeq$, to be the smallest
equivalence relation generated by the following rules.

\begin{mathpar}
\inferrule*[lab=Quote-drop]
{ }
{ \quotep{@{x}} \nameeq x }

\inferrule*[lab=Struct-equiv]
{ P \scong Q }
{ \quotep{P} \nameeq \quotep{Q} }
\end{mathpar}

The astute reader will have noticed that the mutual recursion of names
and processes imposes a mutual recursion on alpha-equivalence and
structural equivalence via name-equivalence. Fortunately, all of this
works out pleasantly and we may calculate in the natural way, free of
concern. The reader interested in the details is referred to the
appendix \ref{appendix:rho_details}.

\subsection{Substitution}

We use $\Proc$ for the set of processes, $\QProc$ for the set of
names, and $\id{\{}\vec{y} / \vec{x} \id{\}}$ to denote partial maps,
$s : \QProc \rightarrow \QProc$. A map, $s$ lifts, uniquely, to a map
on process terms, $\widehat{s} : \Proc \rightarrow \Proc$ by the
following equations.

\begin{mathpar}
  (0) \psubstp{Q}{P} := 0 \\
  (R \juxtap S) \psubstp{Q}{P}
  :=    
  (R)\psubstp{Q}{P} \juxtap (S) \psubstp{Q}{P} \\
  (x?(y).R) \psubstp{Q}{P}    
  :=    
  (x)\substp{Q}{P} (z)\concat( (R \psubstn{z}{y}) \psubstp{Q}{P} ) \\
  (\lift{x}{R}) \psubstp{Q}{P}  
  :=
  \lift{(x)\substp{Q}{P}}{ R \psubstp{Q}{P} } \\
%   (\dropn{x})  \psubstp{Q}{P}       
%   := 
%   \left\{ 
%     \begin{array}{ccc} 
%       \dropn{\quotep{Q}} & & x \nameeq \quotep{P} \\
%       \dropn{x} & & otherwise \\
%     \end{array}
%   \right. 
  (\dropn{x})  \psubstp{Q}{P}       
  := 
  \left\{ 
    \begin{array}{ccc} 
      Q & & x \nameeq \quotep{P} \\
      \dropn{x} & & otherwise \\
    \end{array}
  \right.
\end{mathpar}
 

where

\begin{eqnarray}
  (x)\id{\{} \lpquote Q \rpquote / \lpquote P \rpquote \id{\}}            = 
  \left\{ 
    \begin{array}{ccc}
      \lpquote Q \rpquote & & x \nameeq \lpquote P \rpquote \\
      x & & otherwise \\
    \end{array}
  \right. \nonumber
\end{eqnarray}

and $z$ is chosen distinct from $\quotep{P}$, $\quotep{Q}$, the free
names in $Q$, and all the names in $R$. Our $\alpha$-equivalence will
be built in the standard way from this substitution.

\begin{remark}\label{rem:no_self_referential_names}
  One consequence of these definitions is that $\forall P. \quotep{P}
  \not\in \freenames{P}$.
\end{remark}

\subsection{ Dynamic quote: an example }

Anticipating something of what's to come, consider applying the
substitution, $\widehat{\id{\{}u / z \id{\}}}$, to the following pair
of processes, $\lift{w}{y!(z)}$ and $w[ \lpquote y!(z) \rpquote ]$.

\begin{eqnarray}
	\lift{w}{y!(z)}\widehat{\id{\{}u / z \id{\}}}
		& = &
		\lift{w}{y!(u)} \nonumber\\
	w[ \lpquote y!(z) \rpquote ] \widehat{ \id{\{}u / z \id{\}} }
		& = &
		w[ \lpquote y!(z) \rpquote ] \nonumber
\end{eqnarray}

Because the body of the process between quotes is impervious to
substitution, we get radically different answers. In fact, by
examining the first process in an input context,
e.g. $x?(z).\lift{w}{y!(z)}$, we see that the process under the lift
operator may be shaped by prefixed inputs binding a name inside it. In
this sense, the lift operator will be seen as a way to dynamically
construct processes before reifying them as names.

Finally equipped with these standard features we can present the
dynamics of the calculus.

\subsubsection{Operational semantics} 

Finally, we introduce the computational dynamics. What marks these
algebras as distinct from other more traditionally studied algebraic
structures, e.g. vector spaces or polynomial rings, is the manner in
which dynamics is captured. In traditional structures, dynamics is typically
expressed through morphisms between such structures, as in linear maps
between vector spaces or morphisms between rings. In algebras
associated with the semantics of computation, the dynamics is
expressed as part of the algebraic structure itself, through a
reduction reduction relation typically denoted by $\red$. Below, we
give a recursive presentation of this relation for the calculus used
in the encoding.

$\red \subseteq \pi \times \pi$
$\red : \pi \to \mathcal{P}(\pi)$

\begin{mathpar}
  \inferrule* [lab=Comm] { \textsf{match}( x_{src}, x_{trgt} ) } { x_{trgt}?(y)P \; | \; x_{src}!\langle {Q} \rangle \red P\{\quotep{Q}/y}\} }
  \and \\
  \inferrule* [lab=Par] {{P} \red {P}'} {{{P} | {Q}} \red {{P}' | {Q}}}
  \and
  \inferrule* [lab=Equiv]{{{P} \scong {P}'} \andalso {{P}' \red {Q}'} \andalso {{Q}' \scong {Q}}}{{P} \red {Q}}
\end{mathpar}

\begin{eqnarray*}
  match_{\equiv} (\quotep{P},\quotep{Q}) & := & P \equiv Q \\
  match_{\dagger}(\quotep{P},\quotep{Q}) & := & \forall R. P|Q \red^{*} R => R \red^{*} 0 \\
  match_{K}(\quotep{P},\quotep{Q}) & := & K \mbox{ for some context } K
\end{eqnarray*}

$u?(x)P | u!\langle Q \rangle \red P\{\quotep{Q}/x\}$

%We write $\wred$ for $\red^*$, and $P\red$ if $\exists Q $ such that $ P \red Q$.
We write $P\red$ if $\exists Q $ such that $ P \red Q$ and $P\not\red$, otherwise.

\section{Replication}

As mentioned before, it is known that replication (and hence
recursion) can be implemented in a higher-order process algebra
\cite{SangiorgiWalker}. As our first example of calculation with the
machinery thus far presented we give the construction explicitly in
the {\rhoc}.

\begin{eqnarray}
	D_{x} & := & \prefix{x}{y}{(\binpar{\outputp{x}{y}}{@{y}})} \nonumber\\
	\bangp_{x}{P} & := & \binpar{{x}!\langle{\binpar{D_{x}}{P}}\rangle}{D_{x}} \nonumber
\end{eqnarray}

\begin{eqnarray}
	\bangp_{x}{P} & & \nonumber\\
	=
	& {x}!\langle{(\prefix{x}{y}{(\outputp{x}{y} | @{y})) | P}}\rangle 
	      | \prefix{x}{y}{(\outputp{x}{y} | @{y})} & \nonumber\\
	\red
	& (\outputp{x}{y} | @{y})\substn{\quotep{(\prefix{x}{y}{(@{y} | \outputp{x}{y})) | P}}}{y} & \nonumber\\
	=
	& \outputp{x}{\quotep{(\prefix{x}{y}{(\outputp{x}{y} | @{y})) | P}}}
	  | {(\prefix{x}{y}{(\outputp{x}{y} | @{y})) | P}} & \nonumber\\
	\red
	& \ldots & \nonumber\\
	\red^*
	& P | P | \ldots & \nonumber
\end{eqnarray}

Of course, this encoding, as an implementation, runs away, unfolding
$\bangp{P}$ eagerly. A lazier and more implementable replication
operator, restricted to input-guarded processes, may be obtained as follows.

\begin{eqnarray}
\bangp{\prefix{u}{v}{P}} 
	:= 
	\binpar{\lift{x}{\prefix{u}{v}{(\binpar{D(x)}{P})}}}{D(x)} \nonumber
\end{eqnarray}

\begin{remark}
  Note that the lazier definition still does not deal with summation
  or mixed summation (i.e. sums over input and output). The reader is
  invited to construct definitions of replication that deal with these
  features. 

  Further, the definitions are parameterized in a name, $x$. Can you,
  gentle reader, make a definition that eliminates this parameter and
  guarantees no accidental interaction between the replication
  machinery and the process being replicated -- i.e. no accidental
  sharing of names used by the process to get its work done and the
  name(s) used by the replication to effect copying. This latter
  revision of the definition of replication is crucial to obtaining
  the expected identity $!!P \sim !P$.
\end{remark}

\begin{remark}\label{rem:paradoxical_combinator}
  The reader familiar with the lambda calculus will have noticed the
  similarity between $D$ and the paradoxical combinator.

  [Ed. note: the existence of this seems to suggest we have to be more
  restrictive on the set of processes and names we admit if we are to
  support no-cloning.]
\end{remark}

\subsubsection{Bisimulation}

The computational dynamics gives rise to another kind of equivalence,
the equivalence of computational behavior. As previously mentioned
this is typically captured \emph{via} some form of bisimulation.

% The notion we use in this paper is weak barbed bisimulation
% \cite{milner91polyadicpi}.

The notion we use in this paper is derived from weak barbed
bisimulation \cite{milner91polyadicpi}. 

\begin{definition}
An \emph{observation relation}, $\downarrow_{\mathcal N}$, over a set
of names, $\mathcal N$, is the smallest relation satisfying the rules
below.

\infrule[Out-barb]{y \in {\mathcal N}, \; x \nameeq y}
		  {\outputp{x}{v} \downarrow_{\mathcal N} x}
\infrule[Par-barb]{\mbox{$P\downarrow_{\mathcal N} x$ or $Q\downarrow_{\mathcal N} x$}}
		  {\binpar{P}{Q} \downarrow_{\mathcal N} x}

We write $P \Downarrow_{\mathcal N} x$ if there is $Q$ such that 
$P \wred Q$ and $Q \downarrow_{\mathcal N} x$.
\end{definition}

\begin{definition}
%\label{def.bbisim}
An  ${\mathcal N}$-\emph{barbed bisimulation} over a set of names, ${\mathcal N}$, is a symmetric binary relation 
${\mathcal S}_{\mathcal N}$ between agents such that $P\rel{S}_{\mathcal N}Q$ implies:
\begin{enumerate}
\item If $P \red P'$ then $Q \wred Q'$ and $P'\rel{S}_{\mathcal N} Q'$.
\item If $P\downarrow_{\mathcal N} x$, then $Q\Downarrow_{\mathcal N} x$.
\end{enumerate}
$P$ is ${\mathcal N}$-barbed bisimilar to $Q$, written
$P \wbbisim_{\mathcal N} Q$, if $P \rel{S}_{\mathcal N} Q$ for some ${\mathcal N}$-barbed bisimulation ${\mathcal S}_{\mathcal N}$.
\end{definition}

$\mathcal{R} \subseteq \pi \times \pi$

$P \mathcal{R} Q => \forall P'. P \red P' \Rightarrow \exists Q'. Q \red Q', P' \mathcal{R} Q'$

$P \vdash x \Rightarrow Q \vdash x$

\begin{mathpar}
  \inferrule*[lab=Out-barb]{x \nameeq y}{{y}!\langle{Q}\rangle \vdash x}
  \and
  \inferrule*[lab=Par-barb]{\mbox{$P\vdash x$ or $Q\vdash x$}}{\binpar{P}{Q} \vdash x}
\end{mathpar}

\subsubsection{Contexts}

One of the principle advantages of computational calculi like the
$\pi$-calculus is a well-defined notion of context,
contextual-equivalence and a correlation between
contextual-equivalence and notions of bisimulation. The notion of
context allows the decomposition of a process into (sub-)process and
its syntactic environment, its context. Thus, a context may be
thought of as a process with a ``hole'' (written $\Box$) in it. The
application of a context $M$ to a process $P$, written $M[P]$, is
tantamount to filling the hole in $M$ with $P$. In this paper we do
not need the full weight of this theory, but do make use of the notion
of context in the proof the main theorem. 

\begin{mathpar}
  \inferrule* [lab=summation] {} {{M_{M},M_{N}} \bc \Box \;|\; x.M_{A} \;|\; M_{M}+M_{N}}
  \and
  \inferrule* [lab=agent] {} {{M_{A}} \bc (\vec{x})M_{P} \;| \; \clift{P_0,\ldots,M_{P},\ldots,P_N}}
  \and \\
  \inferrule* [lab=process] {} {{M_{P}} \bc M_{N} \;| \;P|M_{P} }
\end{mathpar} 

\begin{mathpar}
  \inferrule* [lab=sychronization] {} {M_{N} \bc \Box \;|\; x?M_{F} \;|\; x!M_{C}}
  \and
  \inferrule* [lab=abstraction] {} {{M_{F}} \bc (x)M_{P} }
  \and
  \inferrule* [lab=concretion] {} {{M_{C}} \bc \langle M_{P} \rangle }
  \and \\
  \inferrule* [lab=process] {} {{M_{P}} \bc M_{N} \;| \;P|M_{P} }
\end{mathpar}

\begin{definition}[contextual application] Given a context $M$, and
  process $P$, we define the \emph{contextual application}, $M[P] :=
  M\{P/\Box\}$. That is, the contextual application of M to P is the
  substitution of $P$ for $\Box$ in $M$.
\end{definition}

$\meaningof{-} : L \to \mathcal{P}(\pi)$

\begin{mathpar}
  \inferrule* [lab=collection] {} {\meaningof{true} = \pi, \and \meaningof{~E} = \pi \setminus \meaningof{E}, \and \meaningof{E_{1} \& E_{2}} = \meaningof{E_{1}} \cap \meaningof{E_{2}}}
\end{mathpar}

\begin{mathpar}
  \inferrule* [lab=structure] {} {\meaningof{0} = \{ P \in \pi | P \equiv 0 \}, \and \\ \meaningof{E_1 | E_2} = \{ P \in \pi | P \equiv P_{1} | P_{2}, P_{1} \in \meaningof{E_{1}}, P_{2} \in \meaningof{E_2}\} }
\end{mathpar}

\begin{mathpar}
 \inferrule* [lab=behavior] {} {\meaningof{\langle a?b \rangle E} = \{ P \in \pi | P \equiv Q | u?(y)P', \\ \and \\\\ \and \\ \;\;\; u \in \meaningof{a}, \forall z.P'\{z/y\} \in \meaningof{E\{z/b\}}\}, \and \\ \meaningof{a!E} = \{ P \in \pi | P \equiv Q | x!\langle P' \rangle, x \in \meaningof{a} P' \in \meaningof{E}\} }
\end{mathpar}

\begin{mathpar}
 \inferrule* [lab=nominal] {} {\meaningof{\quotep{E}} = \{ \quotep{P} \in \quotep{\pi} | P \in \meaningof{E} \}, \and \meaningof{\quotep{P}} = \{ \quotep{Q} \in \quotep{\pi} | P \equiv Q \} \and \\ \meaningof{@\quotep{E}} = \{ P \in \pi | P \equiv @x, x \in \meaningof{E} \}}
\end{mathpar}

\begin{eqnarray*}
  \\
  \meaningof{-} : TS \to ST
\end{eqnarray*}

\begin{eqnarray*}
  \\
  L : TS \to ST
\end{eqnarray*}

\begin{eqnarray*}
  \\
  P \models E \iff P \in \meaningof{E}
\end{eqnarray*}

\begin{eqnarray*}
  P \approx_{L} Q \iff \forall E \in L. P \models E \iff Q \models E
\end{eqnarray*}

\begin{eqnarray*}
  P \approx_{K} Q
\end{eqnarray*}

\begin{eqnarray*}
  P \approx Q
\end{eqnarray*}

$\approx_{K} = \approx = \approx_{L}$

\subsubsection{Contextual duality}

Note that contexts extend the quotation operation to a family of
operations from processes to names. Given a context, $M$, we can
define a \emph{nominal context}, $\quotep{M}$ by $\quotep{M}[P] :=
\quotep{M[P]}$. To foreshadow what is to come we observe that these
operations enjoy a duality with processes very much like the duality
between vectors and maps from vectors to scalars.

Further, because the calculus is essentially higher-order, we have a
correspondence between contexts and processes. More specifically,
given a name $x$ and a context $M$ we can construct $M^{*}_{x}$ such
that 

\begin{mathpar}
  M^{*}_{x} | \lift{x}{P} \red M[P]
\end{mathpar}

namely,

\begin{mathpar}
  M^{*}_{x} := x?(u).M[\dropn{u}]
\end{mathpar}

The dependence of $M^{*}_{x}$ on a name makes it an abstraction, 

\begin{mathpar}
  M^{*} := (x)x?(u).M[\dropn{u}]
\end{mathpar}

\subsection{Additional notation}

It will sometimes be convenient to denote the process a name
quotes. We already have the notation $x = \quotep{P}$, but it will be
convenient to introduce an alternate notation, $\procn{x}$, when we
want to emphasize the connection to the use of the name. Note that, by
virtue of name equivalence, $\quotep{\procn{x}} \nameeq x$; so, the
notation is consistent with previous definitions.

Further, because names have structure it is possible to effect
substitutions on the basis of that structure. This means we need to
upgrade our notation for substitutions, which we accomplish by
adapting comprehension notation. Thus,

\begin{mathpar}
  P\{ y / x : x \in S \}
\end{mathpar}

is interpreted to mean the process derived from P by replacing (in a
capture-avoiding manner) each occurrence of $x$ in $S$ by $y$. For example,

\begin{mathpar}
  P\{ \quotep{\procn{x}|\procn{x}} / x : x \in \freenames{P} \}
\end{mathpar}

will replace each (occurrence) of a free name $x$ in $P$ by
$\quotep{\procn{x}|\procn{x}}$.

Also, we will avail ourselves of the notation $x^{L}$ and $x^{R}$ to
denote injections of a name into disjoint copies of the name
space. There are numerous ways to accomplish this. One example can be
found in \cite{MeredithR05}. This notation overloads to vectors of
names: $\vec{x}^{\pi} := (x_{i}^{\pi} \; : \; 0 \leq i < |\vec{x}| )$ where $\pi \in \{L,R\}$.

We also use $P^{\Box} := P|\Box$.

In \cite{MeredithR05} an interpretation of the new operator is
given. It turns out that there are several possible interpretations
all enjoying the requisite algebraic properties of the operator (see
\cite{milner91polyadicpi}). We will therefore make liberal use of
$(\nu\; \vec{x})P$.

% subsection the_syntax_and_semantics_of_the_notation_system (end)   

\input{qm2pi.qmops} 

\input{qm2pi.sterngerlach} 

\input{qm2pi.metric} 

% section concurrent_process_calculi (end)

%\input{qm2pi.proofsketch}

% section proof sketch (end)

%\input{qm2pi.slviaknots} 

% section spatial logic via knots (end)

\input{qm2pi.conclusion}

% section conclusion (end)

%\input{qm2pi.dtcodes} 

% section wiring algorithm (end)

\input{qm2pi.ack} 

% section acknowledgments (end)

\newpage


\bibliographystyle{plain}   
\bibliography{../../biblios/main.bib}

\input{qm2pi.rhodetails}

\end{document}



\end{document}



% section front matter (end)

\section{Introduction}\label{sec:introduction} % (fold)
In this draft of the material i am going to have to dispense with the
usual writing conventions adopted in papers on these topics. i'm going
to have adopt whatever tone i need at the time i'm writing up the
calculations. Sometimes this may be very conversational; others it may
be the barest mathematical grunts; others still it may be that i have
lifted text from one of my other papers because the exposition of some
point was better said there. i hope that my readers are not unduly put
out by this decision. i'm not doing this to flout convention or be
rebellious. i find these calculations very technically challenging. To
keep everything going technically, something has to give; i have to
let go of some cognitive burden. So, the academic writing style --
with all of its trade-offs in terms of facilitating technical
communication -- is what i'm letting go of. Perhaps subsequent drafts
can be tightened and polished, but for now, i'm going to speak as if
we were sitting together in a coffee shop with a laptop, wifi and a
pad of paper and a pencil.

So, here's what i have to say. We -- you and i, comfortably ensconced
in our coffee shop and well-equipped with our tools -- can realize and
carry out the calculations of quantum mechanics over a very different
formal theory of dynamics, a formal theory of dynamics that
corresponds to a theory of concurrent computation with
\emph{reflection}. It has the advantage that the underlying theory is
already `quantized', but supports analogues all of the continuuous
operations. Strikingly, this underlying theory has recently been
connected with a notion of metric that we can show, by calculating
together, coincides with the metric induced by the inner product.

There are a lot of reasons why you might be interested in seeing
calculations of this form. Here's why i'm interested. For the past
several centuries there has been no competitor to the ``Newtonian''
account of dynamics. As a result the predominant share of accounts of
dynamical systems and situations have had to be formulated in terms of
the Newtonian machinery. i view this as an intellectually dangerous
position to occupy. Everything, despite it's intrinsic shape, turns
into a nail to be hit with this hammer. Recently, however, the theory
of computation has matured to the point where we have candidates for
theories of dynamics that offer very different perspective on
reasoning about dynamical systems and situations. Testing these
candidates against very successful accounts of dynamical situations,
like quantum mechanics, is going to give us some sense of how mature
they are and some measure of the quality of these accounts of
dynamics.

\subsection{Summary of contributions and outline of paper}

So, we're going to develop an interpretation of the operations of
quantum mechanics normally interpreted by Hilbert spaces and
operators. We're going to do this over a theory of computation. Note
that this is very different than the usual quantum computation program
which develops notions of computation over quantum mechanics. Rather,
we are developing a story that aligns with Wheeler's slogan: It from
Bit. To do this we will first provide an account of the theory of
computation at play here. Then we will dive into a calculation-driven
interpretation of the operations of quantum mechanics.

The reason we take this approach is that -- until very recently --
there hasn't been an axiomatic account of quantum mechanics. As a
result there has been no sharp delineation of the mathematical theory
supporting interpretation of the physical theory and the physical
theory, itself. So, ambient features of the maths are free to be
exploited (or supressed) without a real accounting of their physical
relevance. There is no sharp statement ``here's the physical theory''
qua \emph{theory} and ``here's the mathematical interpretation''
enabling a judgment of how faithful the interpretation is -- apart
from experimental observation. When there is an axiomatic account we
can judge how well a given mathematical formalism supports an
interpretation of the axioms, independent of
experimentation. Likewise, we can judge how well we have captured our
physical evidence and experience with our axiomatics, independent of
any specific mathematical implementation, with accidental detail that
may or may not have physical significance. 

In lieu of a fully fleshed out and vetted axiomatic account of quantum
mechanics, interpreting the operational notions in service of modeling
physical systems will have to suffice. In other words, we are not in
the business of providing a model of Hilbert spaces and operators. We
are in the business of providing a model of quantum mechanics because
we are motivated by testing our notions of dynamics against physical
theory; and, the predictive calculations of the physical theory must
serve as the best formulation -- shy of a fully fleshed out axiomatic
account -- of the physical theory itself (as they have for scientific
theories since time immemorial). Put another way, despite a
whole-hearted commitment to an It-from-Bit ontology, we are firmly
aligned with the shut-up-and-calculate camp as the best way to obtain
results either from the physical perspective or as a quality assurance
measure of our fledgling theory of dynamics.

In detail, we present a reflective process calculus. Then we develop
intuitive correspondences between the notions available in this
calculus and the usual physical notions supporting quantum mechanical
calculations. Thus, 

\begin{table}[htp]
  \center{
    \fbox{
      \begin{tabular}{c|c}
        quantum mechanics & process calculus \\
        \hline
        scalar & name \\
        state vector & process \\
        dual & contextual duals \\
        matrix & formal sums of process-context-dual pairs \\
        orthogonality & process annihilation \\
        inner product & execution-formula + quoting
      \end{tabular}
    }
  }
  \caption{QM - process calculi correspondences}
\end{table}

Then we tighten up these intuitions to operational definitions. We
employ the Dirac notation as the best proxy we can find for an
abstract syntax of the quantum mechanical notions. The definitions we
develop put us in contact with equational constraints coming from the
theory that we demonstrate the definitions and calculations satisfy.

This puts us in a position to shut up and calculate for the
Stern-Gerlach experimental set up, showing how these predictive
calculations become calculations on processes in our theory of a
reflective process calculus.

Penultimately, we demonstrate that the notion of metric coming from
the inner product coincides with the notion of metric available from
the theory of bisimulation. This demonstration gives us the right to
think of space as arising from behavior. Finally, we consider where we
might go from the new vantage point we have obtained.

% section introduction (end) 
 
% section introduction (end)

% \documentclass[12pt]{llncs}
%\documentclass{jktr}

\usepackage[pdftex]{hyperref}                   
\usepackage {listings}
\usepackage {mathpartir}
\usepackage{bcprules}
%\usepackage{listings}
                       
\usepackage{graphicx} 
%\usepackage[margins=2.5cm,nohead,nofoot]{geometry}
%\usepackage{geometry}
\usepackage{amsfonts}
\usepackage{amstext}
\usepackage{latexsym}
\usepackage{amssymb}
\usepackage{color}


%\include{myPreamble}
\documentclass[12pt]{llncs}
%\documentclass{jktr}

\usepackage[pdftex]{hyperref}                   
\usepackage {listings}
\usepackage {mathpartir}
\usepackage{bcprules}
%\usepackage{listings}
                       
\usepackage{graphicx} 
%\usepackage[margins=2.5cm,nohead,nofoot]{geometry}
%\usepackage{geometry}
\usepackage{amsfonts}
\usepackage{amstext}
\usepackage{latexsym}
\usepackage{amssymb}
\usepackage{color}


%\include{myPreamble}
\include{qm2pi.local} 

%\ifpdf
%\usepackage[pdftex]{graphicx}
%\else
%\usepackage{graphicx}
%\fi

 % \ifpdf
%  \usepackage{pdfsync}
%  \if


%\title{Brief Article}
%\author{David F. Snyder}
%\author{L.G. Meredith}

%\address{Dept. of Math., Texas State University--San Marcos, San Marcos, TX 78666}
       
\pagestyle{empty}


\begin{document}

\lstset{language=[Objective]Caml,frame=shadowbox}

\input{qm2pi.front}

% section front matter (end)

\input{qm2pi.intro} 
 
% section introduction (end)

% \input{qm2pi.knotations} 

% section notation (end)

\input{qm2pi.process.calculi} 

% section concurrent_process_calculi_and_spatial_logics_ (end)
    
%\input{qm2pi.knots2pi} 

%\input{qm2pi.trefoil} 

%\input{qm2pi.mainthm} 

% subsection basic_interpretation (end)

%\input{qm2pi.rho.presentation} 
\subsection{The syntax and semantics of the notation system}\label{sub:the_syntax_and_semantics_of_the_notation_system} % (fold)

We now summarize a technical presentation of the calculus that
embodies our theory of dynamics. The typical presentation of such a
calculus follows the style of giving generators and relations on
them. The grammar, below, describing term constructors, freely
generates the set of processes, $\Proc$. This set is then quotiented
by a relation known as structural congruence and it is over this set
that the notion of dynamics is expressed. This presentation is
essentially that of \cite{MeredithR05} with the addition of
polyadicity and summation. For readability we have relegated some of
the technical subtleties to an appendix.

\subsubsection{Process grammar}\label{subsub:process_grammar}

\begin{mathpar}
  \inferrule* [lab=synchronization] {} {{M} \bc \pzero \;|\; x?F \;|\; x!C }
  \and
  \inferrule* [lab=abstraction] {} {{F} \bc (x)P}
  \and
  \inferrule* [lab=concretion] {} {{C} \bc \langle Q \rangle}
  \and
  \inferrule* [lab=process] {} {{P,Q} \bc M \;| \;P|Q \;|\; @{x}}
  \and
  \inferrule* [lab=name] {} {{x} \bc \quotep{P}}
\end{mathpar} 

Note that $\vec{x}$ (resp. $\vec{P}$) denotes a vector of names
(resp. processes) of length $|\vec{x}|$ (resp. $|\vec{P}|$). We adopt
the following useful abbreviations.

\begin{mathpar}
   x?(\vec{y}).P := x.(\vec{y})P \and  x\clift{\vec{P}} := x.\clift{\vec{P}}
   \and x!(y) := \lift{x}{\dropn{y}}
   \and \Pi_{i=0}^{n-1}P_i := P_0 | \ldots | P_{n-1}
\end{mathpar}

\subsubsection{Structural congruence}

\paragraph{Free and bound names and alpha-equivalence.} At the
core of structural equivalence is alpha-equivalence which identifies
process that are the same up to a change of variable. Formally, we
recognize the distinction between free and bound names. The free names
of a process, $\freenames{P}$, may be calculated recursively as
follows:

\begin{mathpar}
\freenames{\pzero} := \emptyset
  \and \\
  \freenames{x?(y).P} := \{ x \} \cup (\freenames{P} \setminus \{ y \})
  \and 
  \freenames{x!\langle P \rangle} := \{ x \} \cup \{ P \} 
  \and \\
  \freenames{P|Q} := \freenames{P} \cup \freenames{Q}
  \and \\
  \freenames{@{x}} := \{ x \}
\end{mathpar}

$\pi$
$\quotep{\pi}$

$\freenames{-} : \pi \to \mathcal{P}(\quotep{\pi})$

\begin{eqnarray*}
  \freenames{\pzero} & := & \emptyset \\
  \freenames{x?(y).P} & := & \{ x \} \cup (\freenames{P} \setminus \{ y \}) \\
  \freenames{x!\langle P \rangle} & := & \{ x \} \cup \{ P \} \\
  \freenames{P|Q} & := & \freenames{P} \cup \freenames{Q} \\
  \freenames{\dropn{x}} & := & \{ x \}
\end{eqnarray*}

The bound names of a process, $\boundnames{P}$, are those names occurring in $P$
that are not free. For example, in $x?(y).0$, the name $x$ is free, while $y$ is bound.

\begin{mathpar}
  \inferrule* [lab=monoidal-laws] {} { P|Q \equiv Q|P \and P|0 \equiv P \and P|(Q|R) \equiv (P|Q)|R }
\end{mathpar}

\begin{mathpar}
  \inferrule* [lab=alpha-equivalence] {} { (x)P \equiv (y)P\{y/x\} \and y \not\in \freenames{P} }
\end{mathpar}

\begin{definition}
Then two processes, $P,Q$, are alpha-equivalent if $P = Q\{\vec{y}/\vec{x}\}$ for
some $\vec{x} \in \boundnames{Q},\vec{y} \in \boundnames{P}$, where $Q\{\vec{y}/\vec{x}\}$
denotes the capture-avoiding substitution of $\vec{y}$ for $\vec{x}$ in $Q$.
\end{definition}

\begin{definition}
  The {\em structural congruence} \cite{SangiorgiWalker} , $\equiv$,
  between processes is the least congruence containing
  alpha-equivalence, satisfying the abelian monoid laws
  (associativity, commutativity and $\pzero$ as identity) for parallel
  composition $|$ and for summation $+$.
\end{definition}

\subsection{Name equivalence}

We take name equivalence, written $\nameeq$, to be the smallest
equivalence relation generated by the following rules.

\begin{mathpar}
\inferrule*[lab=Quote-drop]
{ }
{ \quotep{@{x}} \nameeq x }

\inferrule*[lab=Struct-equiv]
{ P \scong Q }
{ \quotep{P} \nameeq \quotep{Q} }
\end{mathpar}

The astute reader will have noticed that the mutual recursion of names
and processes imposes a mutual recursion on alpha-equivalence and
structural equivalence via name-equivalence. Fortunately, all of this
works out pleasantly and we may calculate in the natural way, free of
concern. The reader interested in the details is referred to the
appendix \ref{appendix:rho_details}.

\subsection{Substitution}

We use $\Proc$ for the set of processes, $\QProc$ for the set of
names, and $\id{\{}\vec{y} / \vec{x} \id{\}}$ to denote partial maps,
$s : \QProc \rightarrow \QProc$. A map, $s$ lifts, uniquely, to a map
on process terms, $\widehat{s} : \Proc \rightarrow \Proc$ by the
following equations.

\begin{mathpar}
  (0) \psubstp{Q}{P} := 0 \\
  (R \juxtap S) \psubstp{Q}{P}
  :=    
  (R)\psubstp{Q}{P} \juxtap (S) \psubstp{Q}{P} \\
  (x?(y).R) \psubstp{Q}{P}    
  :=    
  (x)\substp{Q}{P} (z)\concat( (R \psubstn{z}{y}) \psubstp{Q}{P} ) \\
  (\lift{x}{R}) \psubstp{Q}{P}  
  :=
  \lift{(x)\substp{Q}{P}}{ R \psubstp{Q}{P} } \\
%   (\dropn{x})  \psubstp{Q}{P}       
%   := 
%   \left\{ 
%     \begin{array}{ccc} 
%       \dropn{\quotep{Q}} & & x \nameeq \quotep{P} \\
%       \dropn{x} & & otherwise \\
%     \end{array}
%   \right. 
  (\dropn{x})  \psubstp{Q}{P}       
  := 
  \left\{ 
    \begin{array}{ccc} 
      Q & & x \nameeq \quotep{P} \\
      \dropn{x} & & otherwise \\
    \end{array}
  \right.
\end{mathpar}
 

where

\begin{eqnarray}
  (x)\id{\{} \lpquote Q \rpquote / \lpquote P \rpquote \id{\}}            = 
  \left\{ 
    \begin{array}{ccc}
      \lpquote Q \rpquote & & x \nameeq \lpquote P \rpquote \\
      x & & otherwise \\
    \end{array}
  \right. \nonumber
\end{eqnarray}

and $z$ is chosen distinct from $\quotep{P}$, $\quotep{Q}$, the free
names in $Q$, and all the names in $R$. Our $\alpha$-equivalence will
be built in the standard way from this substitution.

\begin{remark}\label{rem:no_self_referential_names}
  One consequence of these definitions is that $\forall P. \quotep{P}
  \not\in \freenames{P}$.
\end{remark}

\subsection{ Dynamic quote: an example }

Anticipating something of what's to come, consider applying the
substitution, $\widehat{\id{\{}u / z \id{\}}}$, to the following pair
of processes, $\lift{w}{y!(z)}$ and $w[ \lpquote y!(z) \rpquote ]$.

\begin{eqnarray}
	\lift{w}{y!(z)}\widehat{\id{\{}u / z \id{\}}}
		& = &
		\lift{w}{y!(u)} \nonumber\\
	w[ \lpquote y!(z) \rpquote ] \widehat{ \id{\{}u / z \id{\}} }
		& = &
		w[ \lpquote y!(z) \rpquote ] \nonumber
\end{eqnarray}

Because the body of the process between quotes is impervious to
substitution, we get radically different answers. In fact, by
examining the first process in an input context,
e.g. $x?(z).\lift{w}{y!(z)}$, we see that the process under the lift
operator may be shaped by prefixed inputs binding a name inside it. In
this sense, the lift operator will be seen as a way to dynamically
construct processes before reifying them as names.

Finally equipped with these standard features we can present the
dynamics of the calculus.

\subsubsection{Operational semantics} 

Finally, we introduce the computational dynamics. What marks these
algebras as distinct from other more traditionally studied algebraic
structures, e.g. vector spaces or polynomial rings, is the manner in
which dynamics is captured. In traditional structures, dynamics is typically
expressed through morphisms between such structures, as in linear maps
between vector spaces or morphisms between rings. In algebras
associated with the semantics of computation, the dynamics is
expressed as part of the algebraic structure itself, through a
reduction reduction relation typically denoted by $\red$. Below, we
give a recursive presentation of this relation for the calculus used
in the encoding.

$\red \subseteq \pi \times \pi$
$\red : \pi \to \mathcal{P}(\pi)$

\begin{mathpar}
  \inferrule* [lab=Comm] { \textsf{match}( x_{src}, x_{trgt} ) } { x_{trgt}?(y)P \; | \; x_{src}!\langle {Q} \rangle \red P\{\quotep{Q}/y}\} }
  \and \\
  \inferrule* [lab=Par] {{P} \red {P}'} {{{P} | {Q}} \red {{P}' | {Q}}}
  \and
  \inferrule* [lab=Equiv]{{{P} \scong {P}'} \andalso {{P}' \red {Q}'} \andalso {{Q}' \scong {Q}}}{{P} \red {Q}}
\end{mathpar}

\begin{eqnarray*}
  match_{\equiv} (\quotep{P},\quotep{Q}) & := & P \equiv Q \\
  match_{\dagger}(\quotep{P},\quotep{Q}) & := & \forall R. P|Q \red^{*} R => R \red^{*} 0 \\
  match_{K}(\quotep{P},\quotep{Q}) & := & K \mbox{ for some context } K
\end{eqnarray*}

$u?(x)P | u!\langle Q \rangle \red P\{\quotep{Q}/x\}$

%We write $\wred$ for $\red^*$, and $P\red$ if $\exists Q $ such that $ P \red Q$.
We write $P\red$ if $\exists Q $ such that $ P \red Q$ and $P\not\red$, otherwise.

\section{Replication}

As mentioned before, it is known that replication (and hence
recursion) can be implemented in a higher-order process algebra
\cite{SangiorgiWalker}. As our first example of calculation with the
machinery thus far presented we give the construction explicitly in
the {\rhoc}.

\begin{eqnarray}
	D_{x} & := & \prefix{x}{y}{(\binpar{\outputp{x}{y}}{@{y}})} \nonumber\\
	\bangp_{x}{P} & := & \binpar{{x}!\langle{\binpar{D_{x}}{P}}\rangle}{D_{x}} \nonumber
\end{eqnarray}

\begin{eqnarray}
	\bangp_{x}{P} & & \nonumber\\
	=
	& {x}!\langle{(\prefix{x}{y}{(\outputp{x}{y} | @{y})) | P}}\rangle 
	      | \prefix{x}{y}{(\outputp{x}{y} | @{y})} & \nonumber\\
	\red
	& (\outputp{x}{y} | @{y})\substn{\quotep{(\prefix{x}{y}{(@{y} | \outputp{x}{y})) | P}}}{y} & \nonumber\\
	=
	& \outputp{x}{\quotep{(\prefix{x}{y}{(\outputp{x}{y} | @{y})) | P}}}
	  | {(\prefix{x}{y}{(\outputp{x}{y} | @{y})) | P}} & \nonumber\\
	\red
	& \ldots & \nonumber\\
	\red^*
	& P | P | \ldots & \nonumber
\end{eqnarray}

Of course, this encoding, as an implementation, runs away, unfolding
$\bangp{P}$ eagerly. A lazier and more implementable replication
operator, restricted to input-guarded processes, may be obtained as follows.

\begin{eqnarray}
\bangp{\prefix{u}{v}{P}} 
	:= 
	\binpar{\lift{x}{\prefix{u}{v}{(\binpar{D(x)}{P})}}}{D(x)} \nonumber
\end{eqnarray}

\begin{remark}
  Note that the lazier definition still does not deal with summation
  or mixed summation (i.e. sums over input and output). The reader is
  invited to construct definitions of replication that deal with these
  features. 

  Further, the definitions are parameterized in a name, $x$. Can you,
  gentle reader, make a definition that eliminates this parameter and
  guarantees no accidental interaction between the replication
  machinery and the process being replicated -- i.e. no accidental
  sharing of names used by the process to get its work done and the
  name(s) used by the replication to effect copying. This latter
  revision of the definition of replication is crucial to obtaining
  the expected identity $!!P \sim !P$.
\end{remark}

\begin{remark}\label{rem:paradoxical_combinator}
  The reader familiar with the lambda calculus will have noticed the
  similarity between $D$ and the paradoxical combinator.

  [Ed. note: the existence of this seems to suggest we have to be more
  restrictive on the set of processes and names we admit if we are to
  support no-cloning.]
\end{remark}

\subsubsection{Bisimulation}

The computational dynamics gives rise to another kind of equivalence,
the equivalence of computational behavior. As previously mentioned
this is typically captured \emph{via} some form of bisimulation.

% The notion we use in this paper is weak barbed bisimulation
% \cite{milner91polyadicpi}.

The notion we use in this paper is derived from weak barbed
bisimulation \cite{milner91polyadicpi}. 

\begin{definition}
An \emph{observation relation}, $\downarrow_{\mathcal N}$, over a set
of names, $\mathcal N$, is the smallest relation satisfying the rules
below.

\infrule[Out-barb]{y \in {\mathcal N}, \; x \nameeq y}
		  {\outputp{x}{v} \downarrow_{\mathcal N} x}
\infrule[Par-barb]{\mbox{$P\downarrow_{\mathcal N} x$ or $Q\downarrow_{\mathcal N} x$}}
		  {\binpar{P}{Q} \downarrow_{\mathcal N} x}

We write $P \Downarrow_{\mathcal N} x$ if there is $Q$ such that 
$P \wred Q$ and $Q \downarrow_{\mathcal N} x$.
\end{definition}

\begin{definition}
%\label{def.bbisim}
An  ${\mathcal N}$-\emph{barbed bisimulation} over a set of names, ${\mathcal N}$, is a symmetric binary relation 
${\mathcal S}_{\mathcal N}$ between agents such that $P\rel{S}_{\mathcal N}Q$ implies:
\begin{enumerate}
\item If $P \red P'$ then $Q \wred Q'$ and $P'\rel{S}_{\mathcal N} Q'$.
\item If $P\downarrow_{\mathcal N} x$, then $Q\Downarrow_{\mathcal N} x$.
\end{enumerate}
$P$ is ${\mathcal N}$-barbed bisimilar to $Q$, written
$P \wbbisim_{\mathcal N} Q$, if $P \rel{S}_{\mathcal N} Q$ for some ${\mathcal N}$-barbed bisimulation ${\mathcal S}_{\mathcal N}$.
\end{definition}

$\mathcal{R} \subseteq \pi \times \pi$

$P \mathcal{R} Q => \forall P'. P \red P' \Rightarrow \exists Q'. Q \red Q', P' \mathcal{R} Q'$

$P \vdash x \Rightarrow Q \vdash x$

\begin{mathpar}
  \inferrule*[lab=Out-barb]{x \nameeq y}{{y}!\langle{Q}\rangle \vdash x}
  \and
  \inferrule*[lab=Par-barb]{\mbox{$P\vdash x$ or $Q\vdash x$}}{\binpar{P}{Q} \vdash x}
\end{mathpar}

\subsubsection{Contexts}

One of the principle advantages of computational calculi like the
$\pi$-calculus is a well-defined notion of context,
contextual-equivalence and a correlation between
contextual-equivalence and notions of bisimulation. The notion of
context allows the decomposition of a process into (sub-)process and
its syntactic environment, its context. Thus, a context may be
thought of as a process with a ``hole'' (written $\Box$) in it. The
application of a context $M$ to a process $P$, written $M[P]$, is
tantamount to filling the hole in $M$ with $P$. In this paper we do
not need the full weight of this theory, but do make use of the notion
of context in the proof the main theorem. 

\begin{mathpar}
  \inferrule* [lab=summation] {} {{M_{M},M_{N}} \bc \Box \;|\; x.M_{A} \;|\; M_{M}+M_{N}}
  \and
  \inferrule* [lab=agent] {} {{M_{A}} \bc (\vec{x})M_{P} \;| \; \clift{P_0,\ldots,M_{P},\ldots,P_N}}
  \and \\
  \inferrule* [lab=process] {} {{M_{P}} \bc M_{N} \;| \;P|M_{P} }
\end{mathpar} 

\begin{mathpar}
  \inferrule* [lab=sychronization] {} {M_{N} \bc \Box \;|\; x?M_{F} \;|\; x!M_{C}}
  \and
  \inferrule* [lab=abstraction] {} {{M_{F}} \bc (x)M_{P} }
  \and
  \inferrule* [lab=concretion] {} {{M_{C}} \bc \langle M_{P} \rangle }
  \and \\
  \inferrule* [lab=process] {} {{M_{P}} \bc M_{N} \;| \;P|M_{P} }
\end{mathpar}

\begin{definition}[contextual application] Given a context $M$, and
  process $P$, we define the \emph{contextual application}, $M[P] :=
  M\{P/\Box\}$. That is, the contextual application of M to P is the
  substitution of $P$ for $\Box$ in $M$.
\end{definition}

$\meaningof{-} : L \to \mathcal{P}(\pi)$

\begin{mathpar}
  \inferrule* [lab=collection] {} {\meaningof{true} = \pi, \and \meaningof{~E} = \pi \setminus \meaningof{E}, \and \meaningof{E_{1} \& E_{2}} = \meaningof{E_{1}} \cap \meaningof{E_{2}}}
\end{mathpar}

\begin{mathpar}
  \inferrule* [lab=structure] {} {\meaningof{0} = \{ P \in \pi | P \equiv 0 \}, \and \\ \meaningof{E_1 | E_2} = \{ P \in \pi | P \equiv P_{1} | P_{2}, P_{1} \in \meaningof{E_{1}}, P_{2} \in \meaningof{E_2}\} }
\end{mathpar}

\begin{mathpar}
 \inferrule* [lab=behavior] {} {\meaningof{\langle a?b \rangle E} = \{ P \in \pi | P \equiv Q | u?(y)P', \\ \and \\\\ \and \\ \;\;\; u \in \meaningof{a}, \forall z.P'\{z/y\} \in \meaningof{E\{z/b\}}\}, \and \\ \meaningof{a!E} = \{ P \in \pi | P \equiv Q | x!\langle P' \rangle, x \in \meaningof{a} P' \in \meaningof{E}\} }
\end{mathpar}

\begin{mathpar}
 \inferrule* [lab=nominal] {} {\meaningof{\quotep{E}} = \{ \quotep{P} \in \quotep{\pi} | P \in \meaningof{E} \}, \and \meaningof{\quotep{P}} = \{ \quotep{Q} \in \quotep{\pi} | P \equiv Q \} \and \\ \meaningof{@\quotep{E}} = \{ P \in \pi | P \equiv @x, x \in \meaningof{E} \}}
\end{mathpar}

\begin{eqnarray*}
  \\
  \meaningof{-} : TS \to ST
\end{eqnarray*}

\begin{eqnarray*}
  \\
  L : TS \to ST
\end{eqnarray*}

\begin{eqnarray*}
  \\
  P \models E \iff P \in \meaningof{E}
\end{eqnarray*}

\begin{eqnarray*}
  P \approx_{L} Q \iff \forall E \in L. P \models E \iff Q \models E
\end{eqnarray*}

\begin{eqnarray*}
  P \approx_{K} Q
\end{eqnarray*}

\begin{eqnarray*}
  P \approx Q
\end{eqnarray*}

$\approx_{K} = \approx = \approx_{L}$

\subsubsection{Contextual duality}

Note that contexts extend the quotation operation to a family of
operations from processes to names. Given a context, $M$, we can
define a \emph{nominal context}, $\quotep{M}$ by $\quotep{M}[P] :=
\quotep{M[P]}$. To foreshadow what is to come we observe that these
operations enjoy a duality with processes very much like the duality
between vectors and maps from vectors to scalars.

Further, because the calculus is essentially higher-order, we have a
correspondence between contexts and processes. More specifically,
given a name $x$ and a context $M$ we can construct $M^{*}_{x}$ such
that 

\begin{mathpar}
  M^{*}_{x} | \lift{x}{P} \red M[P]
\end{mathpar}

namely,

\begin{mathpar}
  M^{*}_{x} := x?(u).M[\dropn{u}]
\end{mathpar}

The dependence of $M^{*}_{x}$ on a name makes it an abstraction, 

\begin{mathpar}
  M^{*} := (x)x?(u).M[\dropn{u}]
\end{mathpar}

\subsection{Additional notation}

It will sometimes be convenient to denote the process a name
quotes. We already have the notation $x = \quotep{P}$, but it will be
convenient to introduce an alternate notation, $\procn{x}$, when we
want to emphasize the connection to the use of the name. Note that, by
virtue of name equivalence, $\quotep{\procn{x}} \nameeq x$; so, the
notation is consistent with previous definitions.

Further, because names have structure it is possible to effect
substitutions on the basis of that structure. This means we need to
upgrade our notation for substitutions, which we accomplish by
adapting comprehension notation. Thus,

\begin{mathpar}
  P\{ y / x : x \in S \}
\end{mathpar}

is interpreted to mean the process derived from P by replacing (in a
capture-avoiding manner) each occurrence of $x$ in $S$ by $y$. For example,

\begin{mathpar}
  P\{ \quotep{\procn{x}|\procn{x}} / x : x \in \freenames{P} \}
\end{mathpar}

will replace each (occurrence) of a free name $x$ in $P$ by
$\quotep{\procn{x}|\procn{x}}$.

Also, we will avail ourselves of the notation $x^{L}$ and $x^{R}$ to
denote injections of a name into disjoint copies of the name
space. There are numerous ways to accomplish this. One example can be
found in \cite{MeredithR05}. This notation overloads to vectors of
names: $\vec{x}^{\pi} := (x_{i}^{\pi} \; : \; 0 \leq i < |\vec{x}| )$ where $\pi \in \{L,R\}$.

We also use $P^{\Box} := P|\Box$.

In \cite{MeredithR05} an interpretation of the new operator is
given. It turns out that there are several possible interpretations
all enjoying the requisite algebraic properties of the operator (see
\cite{milner91polyadicpi}). We will therefore make liberal use of
$(\nu\; \vec{x})P$.

% subsection the_syntax_and_semantics_of_the_notation_system (end)   

\input{qm2pi.qmops} 

\input{qm2pi.sterngerlach} 

\input{qm2pi.metric} 

% section concurrent_process_calculi (end)

%\input{qm2pi.proofsketch}

% section proof sketch (end)

%\input{qm2pi.slviaknots} 

% section spatial logic via knots (end)

\input{qm2pi.conclusion}

% section conclusion (end)

%\input{qm2pi.dtcodes} 

% section wiring algorithm (end)

\input{qm2pi.ack} 

% section acknowledgments (end)

\newpage


\bibliographystyle{plain}   
\bibliography{../../biblios/main.bib}

\input{qm2pi.rhodetails}

\end{document}

 

%\ifpdf
%\usepackage[pdftex]{graphicx}
%\else
%\usepackage{graphicx}
%\fi

 % \ifpdf
%  \usepackage{pdfsync}
%  \if


%\title{Brief Article}
%\author{David F. Snyder}
%\author{L.G. Meredith}

%\address{Dept. of Math., Texas State University--San Marcos, San Marcos, TX 78666}
       
\pagestyle{empty}


\begin{document}

\lstset{language=[Objective]Caml,frame=shadowbox}

\documentclass[12pt]{llncs}
%\documentclass{jktr}

\usepackage[pdftex]{hyperref}                   
\usepackage {listings}
\usepackage {mathpartir}
\usepackage{bcprules}
%\usepackage{listings}
                       
\usepackage{graphicx} 
%\usepackage[margins=2.5cm,nohead,nofoot]{geometry}
%\usepackage{geometry}
\usepackage{amsfonts}
\usepackage{amstext}
\usepackage{latexsym}
\usepackage{amssymb}
\usepackage{color}


%\include{myPreamble}
\include{qm2pi.local} 

%\ifpdf
%\usepackage[pdftex]{graphicx}
%\else
%\usepackage{graphicx}
%\fi

 % \ifpdf
%  \usepackage{pdfsync}
%  \if


%\title{Brief Article}
%\author{David F. Snyder}
%\author{L.G. Meredith}

%\address{Dept. of Math., Texas State University--San Marcos, San Marcos, TX 78666}
       
\pagestyle{empty}


\begin{document}

\lstset{language=[Objective]Caml,frame=shadowbox}

\input{qm2pi.front}

% section front matter (end)

\input{qm2pi.intro} 
 
% section introduction (end)

% \input{qm2pi.knotations} 

% section notation (end)

\input{qm2pi.process.calculi} 

% section concurrent_process_calculi_and_spatial_logics_ (end)
    
%\input{qm2pi.knots2pi} 

%\input{qm2pi.trefoil} 

%\input{qm2pi.mainthm} 

% subsection basic_interpretation (end)

%\input{qm2pi.rho.presentation} 
\subsection{The syntax and semantics of the notation system}\label{sub:the_syntax_and_semantics_of_the_notation_system} % (fold)

We now summarize a technical presentation of the calculus that
embodies our theory of dynamics. The typical presentation of such a
calculus follows the style of giving generators and relations on
them. The grammar, below, describing term constructors, freely
generates the set of processes, $\Proc$. This set is then quotiented
by a relation known as structural congruence and it is over this set
that the notion of dynamics is expressed. This presentation is
essentially that of \cite{MeredithR05} with the addition of
polyadicity and summation. For readability we have relegated some of
the technical subtleties to an appendix.

\subsubsection{Process grammar}\label{subsub:process_grammar}

\begin{mathpar}
  \inferrule* [lab=synchronization] {} {{M} \bc \pzero \;|\; x?F \;|\; x!C }
  \and
  \inferrule* [lab=abstraction] {} {{F} \bc (x)P}
  \and
  \inferrule* [lab=concretion] {} {{C} \bc \langle Q \rangle}
  \and
  \inferrule* [lab=process] {} {{P,Q} \bc M \;| \;P|Q \;|\; @{x}}
  \and
  \inferrule* [lab=name] {} {{x} \bc \quotep{P}}
\end{mathpar} 

Note that $\vec{x}$ (resp. $\vec{P}$) denotes a vector of names
(resp. processes) of length $|\vec{x}|$ (resp. $|\vec{P}|$). We adopt
the following useful abbreviations.

\begin{mathpar}
   x?(\vec{y}).P := x.(\vec{y})P \and  x\clift{\vec{P}} := x.\clift{\vec{P}}
   \and x!(y) := \lift{x}{\dropn{y}}
   \and \Pi_{i=0}^{n-1}P_i := P_0 | \ldots | P_{n-1}
\end{mathpar}

\subsubsection{Structural congruence}

\paragraph{Free and bound names and alpha-equivalence.} At the
core of structural equivalence is alpha-equivalence which identifies
process that are the same up to a change of variable. Formally, we
recognize the distinction between free and bound names. The free names
of a process, $\freenames{P}$, may be calculated recursively as
follows:

\begin{mathpar}
\freenames{\pzero} := \emptyset
  \and \\
  \freenames{x?(y).P} := \{ x \} \cup (\freenames{P} \setminus \{ y \})
  \and 
  \freenames{x!\langle P \rangle} := \{ x \} \cup \{ P \} 
  \and \\
  \freenames{P|Q} := \freenames{P} \cup \freenames{Q}
  \and \\
  \freenames{@{x}} := \{ x \}
\end{mathpar}

$\pi$
$\quotep{\pi}$

$\freenames{-} : \pi \to \mathcal{P}(\quotep{\pi})$

\begin{eqnarray*}
  \freenames{\pzero} & := & \emptyset \\
  \freenames{x?(y).P} & := & \{ x \} \cup (\freenames{P} \setminus \{ y \}) \\
  \freenames{x!\langle P \rangle} & := & \{ x \} \cup \{ P \} \\
  \freenames{P|Q} & := & \freenames{P} \cup \freenames{Q} \\
  \freenames{\dropn{x}} & := & \{ x \}
\end{eqnarray*}

The bound names of a process, $\boundnames{P}$, are those names occurring in $P$
that are not free. For example, in $x?(y).0$, the name $x$ is free, while $y$ is bound.

\begin{mathpar}
  \inferrule* [lab=monoidal-laws] {} { P|Q \equiv Q|P \and P|0 \equiv P \and P|(Q|R) \equiv (P|Q)|R }
\end{mathpar}

\begin{mathpar}
  \inferrule* [lab=alpha-equivalence] {} { (x)P \equiv (y)P\{y/x\} \and y \not\in \freenames{P} }
\end{mathpar}

\begin{definition}
Then two processes, $P,Q$, are alpha-equivalent if $P = Q\{\vec{y}/\vec{x}\}$ for
some $\vec{x} \in \boundnames{Q},\vec{y} \in \boundnames{P}$, where $Q\{\vec{y}/\vec{x}\}$
denotes the capture-avoiding substitution of $\vec{y}$ for $\vec{x}$ in $Q$.
\end{definition}

\begin{definition}
  The {\em structural congruence} \cite{SangiorgiWalker} , $\equiv$,
  between processes is the least congruence containing
  alpha-equivalence, satisfying the abelian monoid laws
  (associativity, commutativity and $\pzero$ as identity) for parallel
  composition $|$ and for summation $+$.
\end{definition}

\subsection{Name equivalence}

We take name equivalence, written $\nameeq$, to be the smallest
equivalence relation generated by the following rules.

\begin{mathpar}
\inferrule*[lab=Quote-drop]
{ }
{ \quotep{@{x}} \nameeq x }

\inferrule*[lab=Struct-equiv]
{ P \scong Q }
{ \quotep{P} \nameeq \quotep{Q} }
\end{mathpar}

The astute reader will have noticed that the mutual recursion of names
and processes imposes a mutual recursion on alpha-equivalence and
structural equivalence via name-equivalence. Fortunately, all of this
works out pleasantly and we may calculate in the natural way, free of
concern. The reader interested in the details is referred to the
appendix \ref{appendix:rho_details}.

\subsection{Substitution}

We use $\Proc$ for the set of processes, $\QProc$ for the set of
names, and $\id{\{}\vec{y} / \vec{x} \id{\}}$ to denote partial maps,
$s : \QProc \rightarrow \QProc$. A map, $s$ lifts, uniquely, to a map
on process terms, $\widehat{s} : \Proc \rightarrow \Proc$ by the
following equations.

\begin{mathpar}
  (0) \psubstp{Q}{P} := 0 \\
  (R \juxtap S) \psubstp{Q}{P}
  :=    
  (R)\psubstp{Q}{P} \juxtap (S) \psubstp{Q}{P} \\
  (x?(y).R) \psubstp{Q}{P}    
  :=    
  (x)\substp{Q}{P} (z)\concat( (R \psubstn{z}{y}) \psubstp{Q}{P} ) \\
  (\lift{x}{R}) \psubstp{Q}{P}  
  :=
  \lift{(x)\substp{Q}{P}}{ R \psubstp{Q}{P} } \\
%   (\dropn{x})  \psubstp{Q}{P}       
%   := 
%   \left\{ 
%     \begin{array}{ccc} 
%       \dropn{\quotep{Q}} & & x \nameeq \quotep{P} \\
%       \dropn{x} & & otherwise \\
%     \end{array}
%   \right. 
  (\dropn{x})  \psubstp{Q}{P}       
  := 
  \left\{ 
    \begin{array}{ccc} 
      Q & & x \nameeq \quotep{P} \\
      \dropn{x} & & otherwise \\
    \end{array}
  \right.
\end{mathpar}
 

where

\begin{eqnarray}
  (x)\id{\{} \lpquote Q \rpquote / \lpquote P \rpquote \id{\}}            = 
  \left\{ 
    \begin{array}{ccc}
      \lpquote Q \rpquote & & x \nameeq \lpquote P \rpquote \\
      x & & otherwise \\
    \end{array}
  \right. \nonumber
\end{eqnarray}

and $z$ is chosen distinct from $\quotep{P}$, $\quotep{Q}$, the free
names in $Q$, and all the names in $R$. Our $\alpha$-equivalence will
be built in the standard way from this substitution.

\begin{remark}\label{rem:no_self_referential_names}
  One consequence of these definitions is that $\forall P. \quotep{P}
  \not\in \freenames{P}$.
\end{remark}

\subsection{ Dynamic quote: an example }

Anticipating something of what's to come, consider applying the
substitution, $\widehat{\id{\{}u / z \id{\}}}$, to the following pair
of processes, $\lift{w}{y!(z)}$ and $w[ \lpquote y!(z) \rpquote ]$.

\begin{eqnarray}
	\lift{w}{y!(z)}\widehat{\id{\{}u / z \id{\}}}
		& = &
		\lift{w}{y!(u)} \nonumber\\
	w[ \lpquote y!(z) \rpquote ] \widehat{ \id{\{}u / z \id{\}} }
		& = &
		w[ \lpquote y!(z) \rpquote ] \nonumber
\end{eqnarray}

Because the body of the process between quotes is impervious to
substitution, we get radically different answers. In fact, by
examining the first process in an input context,
e.g. $x?(z).\lift{w}{y!(z)}$, we see that the process under the lift
operator may be shaped by prefixed inputs binding a name inside it. In
this sense, the lift operator will be seen as a way to dynamically
construct processes before reifying them as names.

Finally equipped with these standard features we can present the
dynamics of the calculus.

\subsubsection{Operational semantics} 

Finally, we introduce the computational dynamics. What marks these
algebras as distinct from other more traditionally studied algebraic
structures, e.g. vector spaces or polynomial rings, is the manner in
which dynamics is captured. In traditional structures, dynamics is typically
expressed through morphisms between such structures, as in linear maps
between vector spaces or morphisms between rings. In algebras
associated with the semantics of computation, the dynamics is
expressed as part of the algebraic structure itself, through a
reduction reduction relation typically denoted by $\red$. Below, we
give a recursive presentation of this relation for the calculus used
in the encoding.

$\red \subseteq \pi \times \pi$
$\red : \pi \to \mathcal{P}(\pi)$

\begin{mathpar}
  \inferrule* [lab=Comm] { \textsf{match}( x_{src}, x_{trgt} ) } { x_{trgt}?(y)P \; | \; x_{src}!\langle {Q} \rangle \red P\{\quotep{Q}/y}\} }
  \and \\
  \inferrule* [lab=Par] {{P} \red {P}'} {{{P} | {Q}} \red {{P}' | {Q}}}
  \and
  \inferrule* [lab=Equiv]{{{P} \scong {P}'} \andalso {{P}' \red {Q}'} \andalso {{Q}' \scong {Q}}}{{P} \red {Q}}
\end{mathpar}

\begin{eqnarray*}
  match_{\equiv} (\quotep{P},\quotep{Q}) & := & P \equiv Q \\
  match_{\dagger}(\quotep{P},\quotep{Q}) & := & \forall R. P|Q \red^{*} R => R \red^{*} 0 \\
  match_{K}(\quotep{P},\quotep{Q}) & := & K \mbox{ for some context } K
\end{eqnarray*}

$u?(x)P | u!\langle Q \rangle \red P\{\quotep{Q}/x\}$

%We write $\wred$ for $\red^*$, and $P\red$ if $\exists Q $ such that $ P \red Q$.
We write $P\red$ if $\exists Q $ such that $ P \red Q$ and $P\not\red$, otherwise.

\section{Replication}

As mentioned before, it is known that replication (and hence
recursion) can be implemented in a higher-order process algebra
\cite{SangiorgiWalker}. As our first example of calculation with the
machinery thus far presented we give the construction explicitly in
the {\rhoc}.

\begin{eqnarray}
	D_{x} & := & \prefix{x}{y}{(\binpar{\outputp{x}{y}}{@{y}})} \nonumber\\
	\bangp_{x}{P} & := & \binpar{{x}!\langle{\binpar{D_{x}}{P}}\rangle}{D_{x}} \nonumber
\end{eqnarray}

\begin{eqnarray}
	\bangp_{x}{P} & & \nonumber\\
	=
	& {x}!\langle{(\prefix{x}{y}{(\outputp{x}{y} | @{y})) | P}}\rangle 
	      | \prefix{x}{y}{(\outputp{x}{y} | @{y})} & \nonumber\\
	\red
	& (\outputp{x}{y} | @{y})\substn{\quotep{(\prefix{x}{y}{(@{y} | \outputp{x}{y})) | P}}}{y} & \nonumber\\
	=
	& \outputp{x}{\quotep{(\prefix{x}{y}{(\outputp{x}{y} | @{y})) | P}}}
	  | {(\prefix{x}{y}{(\outputp{x}{y} | @{y})) | P}} & \nonumber\\
	\red
	& \ldots & \nonumber\\
	\red^*
	& P | P | \ldots & \nonumber
\end{eqnarray}

Of course, this encoding, as an implementation, runs away, unfolding
$\bangp{P}$ eagerly. A lazier and more implementable replication
operator, restricted to input-guarded processes, may be obtained as follows.

\begin{eqnarray}
\bangp{\prefix{u}{v}{P}} 
	:= 
	\binpar{\lift{x}{\prefix{u}{v}{(\binpar{D(x)}{P})}}}{D(x)} \nonumber
\end{eqnarray}

\begin{remark}
  Note that the lazier definition still does not deal with summation
  or mixed summation (i.e. sums over input and output). The reader is
  invited to construct definitions of replication that deal with these
  features. 

  Further, the definitions are parameterized in a name, $x$. Can you,
  gentle reader, make a definition that eliminates this parameter and
  guarantees no accidental interaction between the replication
  machinery and the process being replicated -- i.e. no accidental
  sharing of names used by the process to get its work done and the
  name(s) used by the replication to effect copying. This latter
  revision of the definition of replication is crucial to obtaining
  the expected identity $!!P \sim !P$.
\end{remark}

\begin{remark}\label{rem:paradoxical_combinator}
  The reader familiar with the lambda calculus will have noticed the
  similarity between $D$ and the paradoxical combinator.

  [Ed. note: the existence of this seems to suggest we have to be more
  restrictive on the set of processes and names we admit if we are to
  support no-cloning.]
\end{remark}

\subsubsection{Bisimulation}

The computational dynamics gives rise to another kind of equivalence,
the equivalence of computational behavior. As previously mentioned
this is typically captured \emph{via} some form of bisimulation.

% The notion we use in this paper is weak barbed bisimulation
% \cite{milner91polyadicpi}.

The notion we use in this paper is derived from weak barbed
bisimulation \cite{milner91polyadicpi}. 

\begin{definition}
An \emph{observation relation}, $\downarrow_{\mathcal N}$, over a set
of names, $\mathcal N$, is the smallest relation satisfying the rules
below.

\infrule[Out-barb]{y \in {\mathcal N}, \; x \nameeq y}
		  {\outputp{x}{v} \downarrow_{\mathcal N} x}
\infrule[Par-barb]{\mbox{$P\downarrow_{\mathcal N} x$ or $Q\downarrow_{\mathcal N} x$}}
		  {\binpar{P}{Q} \downarrow_{\mathcal N} x}

We write $P \Downarrow_{\mathcal N} x$ if there is $Q$ such that 
$P \wred Q$ and $Q \downarrow_{\mathcal N} x$.
\end{definition}

\begin{definition}
%\label{def.bbisim}
An  ${\mathcal N}$-\emph{barbed bisimulation} over a set of names, ${\mathcal N}$, is a symmetric binary relation 
${\mathcal S}_{\mathcal N}$ between agents such that $P\rel{S}_{\mathcal N}Q$ implies:
\begin{enumerate}
\item If $P \red P'$ then $Q \wred Q'$ and $P'\rel{S}_{\mathcal N} Q'$.
\item If $P\downarrow_{\mathcal N} x$, then $Q\Downarrow_{\mathcal N} x$.
\end{enumerate}
$P$ is ${\mathcal N}$-barbed bisimilar to $Q$, written
$P \wbbisim_{\mathcal N} Q$, if $P \rel{S}_{\mathcal N} Q$ for some ${\mathcal N}$-barbed bisimulation ${\mathcal S}_{\mathcal N}$.
\end{definition}

$\mathcal{R} \subseteq \pi \times \pi$

$P \mathcal{R} Q => \forall P'. P \red P' \Rightarrow \exists Q'. Q \red Q', P' \mathcal{R} Q'$

$P \vdash x \Rightarrow Q \vdash x$

\begin{mathpar}
  \inferrule*[lab=Out-barb]{x \nameeq y}{{y}!\langle{Q}\rangle \vdash x}
  \and
  \inferrule*[lab=Par-barb]{\mbox{$P\vdash x$ or $Q\vdash x$}}{\binpar{P}{Q} \vdash x}
\end{mathpar}

\subsubsection{Contexts}

One of the principle advantages of computational calculi like the
$\pi$-calculus is a well-defined notion of context,
contextual-equivalence and a correlation between
contextual-equivalence and notions of bisimulation. The notion of
context allows the decomposition of a process into (sub-)process and
its syntactic environment, its context. Thus, a context may be
thought of as a process with a ``hole'' (written $\Box$) in it. The
application of a context $M$ to a process $P$, written $M[P]$, is
tantamount to filling the hole in $M$ with $P$. In this paper we do
not need the full weight of this theory, but do make use of the notion
of context in the proof the main theorem. 

\begin{mathpar}
  \inferrule* [lab=summation] {} {{M_{M},M_{N}} \bc \Box \;|\; x.M_{A} \;|\; M_{M}+M_{N}}
  \and
  \inferrule* [lab=agent] {} {{M_{A}} \bc (\vec{x})M_{P} \;| \; \clift{P_0,\ldots,M_{P},\ldots,P_N}}
  \and \\
  \inferrule* [lab=process] {} {{M_{P}} \bc M_{N} \;| \;P|M_{P} }
\end{mathpar} 

\begin{mathpar}
  \inferrule* [lab=sychronization] {} {M_{N} \bc \Box \;|\; x?M_{F} \;|\; x!M_{C}}
  \and
  \inferrule* [lab=abstraction] {} {{M_{F}} \bc (x)M_{P} }
  \and
  \inferrule* [lab=concretion] {} {{M_{C}} \bc \langle M_{P} \rangle }
  \and \\
  \inferrule* [lab=process] {} {{M_{P}} \bc M_{N} \;| \;P|M_{P} }
\end{mathpar}

\begin{definition}[contextual application] Given a context $M$, and
  process $P$, we define the \emph{contextual application}, $M[P] :=
  M\{P/\Box\}$. That is, the contextual application of M to P is the
  substitution of $P$ for $\Box$ in $M$.
\end{definition}

$\meaningof{-} : L \to \mathcal{P}(\pi)$

\begin{mathpar}
  \inferrule* [lab=collection] {} {\meaningof{true} = \pi, \and \meaningof{~E} = \pi \setminus \meaningof{E}, \and \meaningof{E_{1} \& E_{2}} = \meaningof{E_{1}} \cap \meaningof{E_{2}}}
\end{mathpar}

\begin{mathpar}
  \inferrule* [lab=structure] {} {\meaningof{0} = \{ P \in \pi | P \equiv 0 \}, \and \\ \meaningof{E_1 | E_2} = \{ P \in \pi | P \equiv P_{1} | P_{2}, P_{1} \in \meaningof{E_{1}}, P_{2} \in \meaningof{E_2}\} }
\end{mathpar}

\begin{mathpar}
 \inferrule* [lab=behavior] {} {\meaningof{\langle a?b \rangle E} = \{ P \in \pi | P \equiv Q | u?(y)P', \\ \and \\\\ \and \\ \;\;\; u \in \meaningof{a}, \forall z.P'\{z/y\} \in \meaningof{E\{z/b\}}\}, \and \\ \meaningof{a!E} = \{ P \in \pi | P \equiv Q | x!\langle P' \rangle, x \in \meaningof{a} P' \in \meaningof{E}\} }
\end{mathpar}

\begin{mathpar}
 \inferrule* [lab=nominal] {} {\meaningof{\quotep{E}} = \{ \quotep{P} \in \quotep{\pi} | P \in \meaningof{E} \}, \and \meaningof{\quotep{P}} = \{ \quotep{Q} \in \quotep{\pi} | P \equiv Q \} \and \\ \meaningof{@\quotep{E}} = \{ P \in \pi | P \equiv @x, x \in \meaningof{E} \}}
\end{mathpar}

\begin{eqnarray*}
  \\
  \meaningof{-} : TS \to ST
\end{eqnarray*}

\begin{eqnarray*}
  \\
  L : TS \to ST
\end{eqnarray*}

\begin{eqnarray*}
  \\
  P \models E \iff P \in \meaningof{E}
\end{eqnarray*}

\begin{eqnarray*}
  P \approx_{L} Q \iff \forall E \in L. P \models E \iff Q \models E
\end{eqnarray*}

\begin{eqnarray*}
  P \approx_{K} Q
\end{eqnarray*}

\begin{eqnarray*}
  P \approx Q
\end{eqnarray*}

$\approx_{K} = \approx = \approx_{L}$

\subsubsection{Contextual duality}

Note that contexts extend the quotation operation to a family of
operations from processes to names. Given a context, $M$, we can
define a \emph{nominal context}, $\quotep{M}$ by $\quotep{M}[P] :=
\quotep{M[P]}$. To foreshadow what is to come we observe that these
operations enjoy a duality with processes very much like the duality
between vectors and maps from vectors to scalars.

Further, because the calculus is essentially higher-order, we have a
correspondence between contexts and processes. More specifically,
given a name $x$ and a context $M$ we can construct $M^{*}_{x}$ such
that 

\begin{mathpar}
  M^{*}_{x} | \lift{x}{P} \red M[P]
\end{mathpar}

namely,

\begin{mathpar}
  M^{*}_{x} := x?(u).M[\dropn{u}]
\end{mathpar}

The dependence of $M^{*}_{x}$ on a name makes it an abstraction, 

\begin{mathpar}
  M^{*} := (x)x?(u).M[\dropn{u}]
\end{mathpar}

\subsection{Additional notation}

It will sometimes be convenient to denote the process a name
quotes. We already have the notation $x = \quotep{P}$, but it will be
convenient to introduce an alternate notation, $\procn{x}$, when we
want to emphasize the connection to the use of the name. Note that, by
virtue of name equivalence, $\quotep{\procn{x}} \nameeq x$; so, the
notation is consistent with previous definitions.

Further, because names have structure it is possible to effect
substitutions on the basis of that structure. This means we need to
upgrade our notation for substitutions, which we accomplish by
adapting comprehension notation. Thus,

\begin{mathpar}
  P\{ y / x : x \in S \}
\end{mathpar}

is interpreted to mean the process derived from P by replacing (in a
capture-avoiding manner) each occurrence of $x$ in $S$ by $y$. For example,

\begin{mathpar}
  P\{ \quotep{\procn{x}|\procn{x}} / x : x \in \freenames{P} \}
\end{mathpar}

will replace each (occurrence) of a free name $x$ in $P$ by
$\quotep{\procn{x}|\procn{x}}$.

Also, we will avail ourselves of the notation $x^{L}$ and $x^{R}$ to
denote injections of a name into disjoint copies of the name
space. There are numerous ways to accomplish this. One example can be
found in \cite{MeredithR05}. This notation overloads to vectors of
names: $\vec{x}^{\pi} := (x_{i}^{\pi} \; : \; 0 \leq i < |\vec{x}| )$ where $\pi \in \{L,R\}$.

We also use $P^{\Box} := P|\Box$.

In \cite{MeredithR05} an interpretation of the new operator is
given. It turns out that there are several possible interpretations
all enjoying the requisite algebraic properties of the operator (see
\cite{milner91polyadicpi}). We will therefore make liberal use of
$(\nu\; \vec{x})P$.

% subsection the_syntax_and_semantics_of_the_notation_system (end)   

\input{qm2pi.qmops} 

\input{qm2pi.sterngerlach} 

\input{qm2pi.metric} 

% section concurrent_process_calculi (end)

%\input{qm2pi.proofsketch}

% section proof sketch (end)

%\input{qm2pi.slviaknots} 

% section spatial logic via knots (end)

\input{qm2pi.conclusion}

% section conclusion (end)

%\input{qm2pi.dtcodes} 

% section wiring algorithm (end)

\input{qm2pi.ack} 

% section acknowledgments (end)

\newpage


\bibliographystyle{plain}   
\bibliography{../../biblios/main.bib}

\input{qm2pi.rhodetails}

\end{document}



% section front matter (end)

\section{Introduction}\label{sec:introduction} % (fold)
In this draft of the material i am going to have to dispense with the
usual writing conventions adopted in papers on these topics. i'm going
to have adopt whatever tone i need at the time i'm writing up the
calculations. Sometimes this may be very conversational; others it may
be the barest mathematical grunts; others still it may be that i have
lifted text from one of my other papers because the exposition of some
point was better said there. i hope that my readers are not unduly put
out by this decision. i'm not doing this to flout convention or be
rebellious. i find these calculations very technically challenging. To
keep everything going technically, something has to give; i have to
let go of some cognitive burden. So, the academic writing style --
with all of its trade-offs in terms of facilitating technical
communication -- is what i'm letting go of. Perhaps subsequent drafts
can be tightened and polished, but for now, i'm going to speak as if
we were sitting together in a coffee shop with a laptop, wifi and a
pad of paper and a pencil.

So, here's what i have to say. We -- you and i, comfortably ensconced
in our coffee shop and well-equipped with our tools -- can realize and
carry out the calculations of quantum mechanics over a very different
formal theory of dynamics, a formal theory of dynamics that
corresponds to a theory of concurrent computation with
\emph{reflection}. It has the advantage that the underlying theory is
already `quantized', but supports analogues all of the continuuous
operations. Strikingly, this underlying theory has recently been
connected with a notion of metric that we can show, by calculating
together, coincides with the metric induced by the inner product.

There are a lot of reasons why you might be interested in seeing
calculations of this form. Here's why i'm interested. For the past
several centuries there has been no competitor to the ``Newtonian''
account of dynamics. As a result the predominant share of accounts of
dynamical systems and situations have had to be formulated in terms of
the Newtonian machinery. i view this as an intellectually dangerous
position to occupy. Everything, despite it's intrinsic shape, turns
into a nail to be hit with this hammer. Recently, however, the theory
of computation has matured to the point where we have candidates for
theories of dynamics that offer very different perspective on
reasoning about dynamical systems and situations. Testing these
candidates against very successful accounts of dynamical situations,
like quantum mechanics, is going to give us some sense of how mature
they are and some measure of the quality of these accounts of
dynamics.

\subsection{Summary of contributions and outline of paper}

So, we're going to develop an interpretation of the operations of
quantum mechanics normally interpreted by Hilbert spaces and
operators. We're going to do this over a theory of computation. Note
that this is very different than the usual quantum computation program
which develops notions of computation over quantum mechanics. Rather,
we are developing a story that aligns with Wheeler's slogan: It from
Bit. To do this we will first provide an account of the theory of
computation at play here. Then we will dive into a calculation-driven
interpretation of the operations of quantum mechanics.

The reason we take this approach is that -- until very recently --
there hasn't been an axiomatic account of quantum mechanics. As a
result there has been no sharp delineation of the mathematical theory
supporting interpretation of the physical theory and the physical
theory, itself. So, ambient features of the maths are free to be
exploited (or supressed) without a real accounting of their physical
relevance. There is no sharp statement ``here's the physical theory''
qua \emph{theory} and ``here's the mathematical interpretation''
enabling a judgment of how faithful the interpretation is -- apart
from experimental observation. When there is an axiomatic account we
can judge how well a given mathematical formalism supports an
interpretation of the axioms, independent of
experimentation. Likewise, we can judge how well we have captured our
physical evidence and experience with our axiomatics, independent of
any specific mathematical implementation, with accidental detail that
may or may not have physical significance. 

In lieu of a fully fleshed out and vetted axiomatic account of quantum
mechanics, interpreting the operational notions in service of modeling
physical systems will have to suffice. In other words, we are not in
the business of providing a model of Hilbert spaces and operators. We
are in the business of providing a model of quantum mechanics because
we are motivated by testing our notions of dynamics against physical
theory; and, the predictive calculations of the physical theory must
serve as the best formulation -- shy of a fully fleshed out axiomatic
account -- of the physical theory itself (as they have for scientific
theories since time immemorial). Put another way, despite a
whole-hearted commitment to an It-from-Bit ontology, we are firmly
aligned with the shut-up-and-calculate camp as the best way to obtain
results either from the physical perspective or as a quality assurance
measure of our fledgling theory of dynamics.

In detail, we present a reflective process calculus. Then we develop
intuitive correspondences between the notions available in this
calculus and the usual physical notions supporting quantum mechanical
calculations. Thus, 

\begin{table}[htp]
  \center{
    \fbox{
      \begin{tabular}{c|c}
        quantum mechanics & process calculus \\
        \hline
        scalar & name \\
        state vector & process \\
        dual & contextual duals \\
        matrix & formal sums of process-context-dual pairs \\
        orthogonality & process annihilation \\
        inner product & execution-formula + quoting
      \end{tabular}
    }
  }
  \caption{QM - process calculi correspondences}
\end{table}

Then we tighten up these intuitions to operational definitions. We
employ the Dirac notation as the best proxy we can find for an
abstract syntax of the quantum mechanical notions. The definitions we
develop put us in contact with equational constraints coming from the
theory that we demonstrate the definitions and calculations satisfy.

This puts us in a position to shut up and calculate for the
Stern-Gerlach experimental set up, showing how these predictive
calculations become calculations on processes in our theory of a
reflective process calculus.

Penultimately, we demonstrate that the notion of metric coming from
the inner product coincides with the notion of metric available from
the theory of bisimulation. This demonstration gives us the right to
think of space as arising from behavior. Finally, we consider where we
might go from the new vantage point we have obtained.

% section introduction (end) 
 
% section introduction (end)

% \documentclass[12pt]{llncs}
%\documentclass{jktr}

\usepackage[pdftex]{hyperref}                   
\usepackage {listings}
\usepackage {mathpartir}
\usepackage{bcprules}
%\usepackage{listings}
                       
\usepackage{graphicx} 
%\usepackage[margins=2.5cm,nohead,nofoot]{geometry}
%\usepackage{geometry}
\usepackage{amsfonts}
\usepackage{amstext}
\usepackage{latexsym}
\usepackage{amssymb}
\usepackage{color}


%\include{myPreamble}
\include{qm2pi.local} 

%\ifpdf
%\usepackage[pdftex]{graphicx}
%\else
%\usepackage{graphicx}
%\fi

 % \ifpdf
%  \usepackage{pdfsync}
%  \if


%\title{Brief Article}
%\author{David F. Snyder}
%\author{L.G. Meredith}

%\address{Dept. of Math., Texas State University--San Marcos, San Marcos, TX 78666}
       
\pagestyle{empty}


\begin{document}

\lstset{language=[Objective]Caml,frame=shadowbox}

\input{qm2pi.front}

% section front matter (end)

\input{qm2pi.intro} 
 
% section introduction (end)

% \input{qm2pi.knotations} 

% section notation (end)

\input{qm2pi.process.calculi} 

% section concurrent_process_calculi_and_spatial_logics_ (end)
    
%\input{qm2pi.knots2pi} 

%\input{qm2pi.trefoil} 

%\input{qm2pi.mainthm} 

% subsection basic_interpretation (end)

%\input{qm2pi.rho.presentation} 
\subsection{The syntax and semantics of the notation system}\label{sub:the_syntax_and_semantics_of_the_notation_system} % (fold)

We now summarize a technical presentation of the calculus that
embodies our theory of dynamics. The typical presentation of such a
calculus follows the style of giving generators and relations on
them. The grammar, below, describing term constructors, freely
generates the set of processes, $\Proc$. This set is then quotiented
by a relation known as structural congruence and it is over this set
that the notion of dynamics is expressed. This presentation is
essentially that of \cite{MeredithR05} with the addition of
polyadicity and summation. For readability we have relegated some of
the technical subtleties to an appendix.

\subsubsection{Process grammar}\label{subsub:process_grammar}

\begin{mathpar}
  \inferrule* [lab=synchronization] {} {{M} \bc \pzero \;|\; x?F \;|\; x!C }
  \and
  \inferrule* [lab=abstraction] {} {{F} \bc (x)P}
  \and
  \inferrule* [lab=concretion] {} {{C} \bc \langle Q \rangle}
  \and
  \inferrule* [lab=process] {} {{P,Q} \bc M \;| \;P|Q \;|\; @{x}}
  \and
  \inferrule* [lab=name] {} {{x} \bc \quotep{P}}
\end{mathpar} 

Note that $\vec{x}$ (resp. $\vec{P}$) denotes a vector of names
(resp. processes) of length $|\vec{x}|$ (resp. $|\vec{P}|$). We adopt
the following useful abbreviations.

\begin{mathpar}
   x?(\vec{y}).P := x.(\vec{y})P \and  x\clift{\vec{P}} := x.\clift{\vec{P}}
   \and x!(y) := \lift{x}{\dropn{y}}
   \and \Pi_{i=0}^{n-1}P_i := P_0 | \ldots | P_{n-1}
\end{mathpar}

\subsubsection{Structural congruence}

\paragraph{Free and bound names and alpha-equivalence.} At the
core of structural equivalence is alpha-equivalence which identifies
process that are the same up to a change of variable. Formally, we
recognize the distinction between free and bound names. The free names
of a process, $\freenames{P}$, may be calculated recursively as
follows:

\begin{mathpar}
\freenames{\pzero} := \emptyset
  \and \\
  \freenames{x?(y).P} := \{ x \} \cup (\freenames{P} \setminus \{ y \})
  \and 
  \freenames{x!\langle P \rangle} := \{ x \} \cup \{ P \} 
  \and \\
  \freenames{P|Q} := \freenames{P} \cup \freenames{Q}
  \and \\
  \freenames{@{x}} := \{ x \}
\end{mathpar}

$\pi$
$\quotep{\pi}$

$\freenames{-} : \pi \to \mathcal{P}(\quotep{\pi})$

\begin{eqnarray*}
  \freenames{\pzero} & := & \emptyset \\
  \freenames{x?(y).P} & := & \{ x \} \cup (\freenames{P} \setminus \{ y \}) \\
  \freenames{x!\langle P \rangle} & := & \{ x \} \cup \{ P \} \\
  \freenames{P|Q} & := & \freenames{P} \cup \freenames{Q} \\
  \freenames{\dropn{x}} & := & \{ x \}
\end{eqnarray*}

The bound names of a process, $\boundnames{P}$, are those names occurring in $P$
that are not free. For example, in $x?(y).0$, the name $x$ is free, while $y$ is bound.

\begin{mathpar}
  \inferrule* [lab=monoidal-laws] {} { P|Q \equiv Q|P \and P|0 \equiv P \and P|(Q|R) \equiv (P|Q)|R }
\end{mathpar}

\begin{mathpar}
  \inferrule* [lab=alpha-equivalence] {} { (x)P \equiv (y)P\{y/x\} \and y \not\in \freenames{P} }
\end{mathpar}

\begin{definition}
Then two processes, $P,Q$, are alpha-equivalent if $P = Q\{\vec{y}/\vec{x}\}$ for
some $\vec{x} \in \boundnames{Q},\vec{y} \in \boundnames{P}$, where $Q\{\vec{y}/\vec{x}\}$
denotes the capture-avoiding substitution of $\vec{y}$ for $\vec{x}$ in $Q$.
\end{definition}

\begin{definition}
  The {\em structural congruence} \cite{SangiorgiWalker} , $\equiv$,
  between processes is the least congruence containing
  alpha-equivalence, satisfying the abelian monoid laws
  (associativity, commutativity and $\pzero$ as identity) for parallel
  composition $|$ and for summation $+$.
\end{definition}

\subsection{Name equivalence}

We take name equivalence, written $\nameeq$, to be the smallest
equivalence relation generated by the following rules.

\begin{mathpar}
\inferrule*[lab=Quote-drop]
{ }
{ \quotep{@{x}} \nameeq x }

\inferrule*[lab=Struct-equiv]
{ P \scong Q }
{ \quotep{P} \nameeq \quotep{Q} }
\end{mathpar}

The astute reader will have noticed that the mutual recursion of names
and processes imposes a mutual recursion on alpha-equivalence and
structural equivalence via name-equivalence. Fortunately, all of this
works out pleasantly and we may calculate in the natural way, free of
concern. The reader interested in the details is referred to the
appendix \ref{appendix:rho_details}.

\subsection{Substitution}

We use $\Proc$ for the set of processes, $\QProc$ for the set of
names, and $\id{\{}\vec{y} / \vec{x} \id{\}}$ to denote partial maps,
$s : \QProc \rightarrow \QProc$. A map, $s$ lifts, uniquely, to a map
on process terms, $\widehat{s} : \Proc \rightarrow \Proc$ by the
following equations.

\begin{mathpar}
  (0) \psubstp{Q}{P} := 0 \\
  (R \juxtap S) \psubstp{Q}{P}
  :=    
  (R)\psubstp{Q}{P} \juxtap (S) \psubstp{Q}{P} \\
  (x?(y).R) \psubstp{Q}{P}    
  :=    
  (x)\substp{Q}{P} (z)\concat( (R \psubstn{z}{y}) \psubstp{Q}{P} ) \\
  (\lift{x}{R}) \psubstp{Q}{P}  
  :=
  \lift{(x)\substp{Q}{P}}{ R \psubstp{Q}{P} } \\
%   (\dropn{x})  \psubstp{Q}{P}       
%   := 
%   \left\{ 
%     \begin{array}{ccc} 
%       \dropn{\quotep{Q}} & & x \nameeq \quotep{P} \\
%       \dropn{x} & & otherwise \\
%     \end{array}
%   \right. 
  (\dropn{x})  \psubstp{Q}{P}       
  := 
  \left\{ 
    \begin{array}{ccc} 
      Q & & x \nameeq \quotep{P} \\
      \dropn{x} & & otherwise \\
    \end{array}
  \right.
\end{mathpar}
 

where

\begin{eqnarray}
  (x)\id{\{} \lpquote Q \rpquote / \lpquote P \rpquote \id{\}}            = 
  \left\{ 
    \begin{array}{ccc}
      \lpquote Q \rpquote & & x \nameeq \lpquote P \rpquote \\
      x & & otherwise \\
    \end{array}
  \right. \nonumber
\end{eqnarray}

and $z$ is chosen distinct from $\quotep{P}$, $\quotep{Q}$, the free
names in $Q$, and all the names in $R$. Our $\alpha$-equivalence will
be built in the standard way from this substitution.

\begin{remark}\label{rem:no_self_referential_names}
  One consequence of these definitions is that $\forall P. \quotep{P}
  \not\in \freenames{P}$.
\end{remark}

\subsection{ Dynamic quote: an example }

Anticipating something of what's to come, consider applying the
substitution, $\widehat{\id{\{}u / z \id{\}}}$, to the following pair
of processes, $\lift{w}{y!(z)}$ and $w[ \lpquote y!(z) \rpquote ]$.

\begin{eqnarray}
	\lift{w}{y!(z)}\widehat{\id{\{}u / z \id{\}}}
		& = &
		\lift{w}{y!(u)} \nonumber\\
	w[ \lpquote y!(z) \rpquote ] \widehat{ \id{\{}u / z \id{\}} }
		& = &
		w[ \lpquote y!(z) \rpquote ] \nonumber
\end{eqnarray}

Because the body of the process between quotes is impervious to
substitution, we get radically different answers. In fact, by
examining the first process in an input context,
e.g. $x?(z).\lift{w}{y!(z)}$, we see that the process under the lift
operator may be shaped by prefixed inputs binding a name inside it. In
this sense, the lift operator will be seen as a way to dynamically
construct processes before reifying them as names.

Finally equipped with these standard features we can present the
dynamics of the calculus.

\subsubsection{Operational semantics} 

Finally, we introduce the computational dynamics. What marks these
algebras as distinct from other more traditionally studied algebraic
structures, e.g. vector spaces or polynomial rings, is the manner in
which dynamics is captured. In traditional structures, dynamics is typically
expressed through morphisms between such structures, as in linear maps
between vector spaces or morphisms between rings. In algebras
associated with the semantics of computation, the dynamics is
expressed as part of the algebraic structure itself, through a
reduction reduction relation typically denoted by $\red$. Below, we
give a recursive presentation of this relation for the calculus used
in the encoding.

$\red \subseteq \pi \times \pi$
$\red : \pi \to \mathcal{P}(\pi)$

\begin{mathpar}
  \inferrule* [lab=Comm] { \textsf{match}( x_{src}, x_{trgt} ) } { x_{trgt}?(y)P \; | \; x_{src}!\langle {Q} \rangle \red P\{\quotep{Q}/y}\} }
  \and \\
  \inferrule* [lab=Par] {{P} \red {P}'} {{{P} | {Q}} \red {{P}' | {Q}}}
  \and
  \inferrule* [lab=Equiv]{{{P} \scong {P}'} \andalso {{P}' \red {Q}'} \andalso {{Q}' \scong {Q}}}{{P} \red {Q}}
\end{mathpar}

\begin{eqnarray*}
  match_{\equiv} (\quotep{P},\quotep{Q}) & := & P \equiv Q \\
  match_{\dagger}(\quotep{P},\quotep{Q}) & := & \forall R. P|Q \red^{*} R => R \red^{*} 0 \\
  match_{K}(\quotep{P},\quotep{Q}) & := & K \mbox{ for some context } K
\end{eqnarray*}

$u?(x)P | u!\langle Q \rangle \red P\{\quotep{Q}/x\}$

%We write $\wred$ for $\red^*$, and $P\red$ if $\exists Q $ such that $ P \red Q$.
We write $P\red$ if $\exists Q $ such that $ P \red Q$ and $P\not\red$, otherwise.

\section{Replication}

As mentioned before, it is known that replication (and hence
recursion) can be implemented in a higher-order process algebra
\cite{SangiorgiWalker}. As our first example of calculation with the
machinery thus far presented we give the construction explicitly in
the {\rhoc}.

\begin{eqnarray}
	D_{x} & := & \prefix{x}{y}{(\binpar{\outputp{x}{y}}{@{y}})} \nonumber\\
	\bangp_{x}{P} & := & \binpar{{x}!\langle{\binpar{D_{x}}{P}}\rangle}{D_{x}} \nonumber
\end{eqnarray}

\begin{eqnarray}
	\bangp_{x}{P} & & \nonumber\\
	=
	& {x}!\langle{(\prefix{x}{y}{(\outputp{x}{y} | @{y})) | P}}\rangle 
	      | \prefix{x}{y}{(\outputp{x}{y} | @{y})} & \nonumber\\
	\red
	& (\outputp{x}{y} | @{y})\substn{\quotep{(\prefix{x}{y}{(@{y} | \outputp{x}{y})) | P}}}{y} & \nonumber\\
	=
	& \outputp{x}{\quotep{(\prefix{x}{y}{(\outputp{x}{y} | @{y})) | P}}}
	  | {(\prefix{x}{y}{(\outputp{x}{y} | @{y})) | P}} & \nonumber\\
	\red
	& \ldots & \nonumber\\
	\red^*
	& P | P | \ldots & \nonumber
\end{eqnarray}

Of course, this encoding, as an implementation, runs away, unfolding
$\bangp{P}$ eagerly. A lazier and more implementable replication
operator, restricted to input-guarded processes, may be obtained as follows.

\begin{eqnarray}
\bangp{\prefix{u}{v}{P}} 
	:= 
	\binpar{\lift{x}{\prefix{u}{v}{(\binpar{D(x)}{P})}}}{D(x)} \nonumber
\end{eqnarray}

\begin{remark}
  Note that the lazier definition still does not deal with summation
  or mixed summation (i.e. sums over input and output). The reader is
  invited to construct definitions of replication that deal with these
  features. 

  Further, the definitions are parameterized in a name, $x$. Can you,
  gentle reader, make a definition that eliminates this parameter and
  guarantees no accidental interaction between the replication
  machinery and the process being replicated -- i.e. no accidental
  sharing of names used by the process to get its work done and the
  name(s) used by the replication to effect copying. This latter
  revision of the definition of replication is crucial to obtaining
  the expected identity $!!P \sim !P$.
\end{remark}

\begin{remark}\label{rem:paradoxical_combinator}
  The reader familiar with the lambda calculus will have noticed the
  similarity between $D$ and the paradoxical combinator.

  [Ed. note: the existence of this seems to suggest we have to be more
  restrictive on the set of processes and names we admit if we are to
  support no-cloning.]
\end{remark}

\subsubsection{Bisimulation}

The computational dynamics gives rise to another kind of equivalence,
the equivalence of computational behavior. As previously mentioned
this is typically captured \emph{via} some form of bisimulation.

% The notion we use in this paper is weak barbed bisimulation
% \cite{milner91polyadicpi}.

The notion we use in this paper is derived from weak barbed
bisimulation \cite{milner91polyadicpi}. 

\begin{definition}
An \emph{observation relation}, $\downarrow_{\mathcal N}$, over a set
of names, $\mathcal N$, is the smallest relation satisfying the rules
below.

\infrule[Out-barb]{y \in {\mathcal N}, \; x \nameeq y}
		  {\outputp{x}{v} \downarrow_{\mathcal N} x}
\infrule[Par-barb]{\mbox{$P\downarrow_{\mathcal N} x$ or $Q\downarrow_{\mathcal N} x$}}
		  {\binpar{P}{Q} \downarrow_{\mathcal N} x}

We write $P \Downarrow_{\mathcal N} x$ if there is $Q$ such that 
$P \wred Q$ and $Q \downarrow_{\mathcal N} x$.
\end{definition}

\begin{definition}
%\label{def.bbisim}
An  ${\mathcal N}$-\emph{barbed bisimulation} over a set of names, ${\mathcal N}$, is a symmetric binary relation 
${\mathcal S}_{\mathcal N}$ between agents such that $P\rel{S}_{\mathcal N}Q$ implies:
\begin{enumerate}
\item If $P \red P'$ then $Q \wred Q'$ and $P'\rel{S}_{\mathcal N} Q'$.
\item If $P\downarrow_{\mathcal N} x$, then $Q\Downarrow_{\mathcal N} x$.
\end{enumerate}
$P$ is ${\mathcal N}$-barbed bisimilar to $Q$, written
$P \wbbisim_{\mathcal N} Q$, if $P \rel{S}_{\mathcal N} Q$ for some ${\mathcal N}$-barbed bisimulation ${\mathcal S}_{\mathcal N}$.
\end{definition}

$\mathcal{R} \subseteq \pi \times \pi$

$P \mathcal{R} Q => \forall P'. P \red P' \Rightarrow \exists Q'. Q \red Q', P' \mathcal{R} Q'$

$P \vdash x \Rightarrow Q \vdash x$

\begin{mathpar}
  \inferrule*[lab=Out-barb]{x \nameeq y}{{y}!\langle{Q}\rangle \vdash x}
  \and
  \inferrule*[lab=Par-barb]{\mbox{$P\vdash x$ or $Q\vdash x$}}{\binpar{P}{Q} \vdash x}
\end{mathpar}

\subsubsection{Contexts}

One of the principle advantages of computational calculi like the
$\pi$-calculus is a well-defined notion of context,
contextual-equivalence and a correlation between
contextual-equivalence and notions of bisimulation. The notion of
context allows the decomposition of a process into (sub-)process and
its syntactic environment, its context. Thus, a context may be
thought of as a process with a ``hole'' (written $\Box$) in it. The
application of a context $M$ to a process $P$, written $M[P]$, is
tantamount to filling the hole in $M$ with $P$. In this paper we do
not need the full weight of this theory, but do make use of the notion
of context in the proof the main theorem. 

\begin{mathpar}
  \inferrule* [lab=summation] {} {{M_{M},M_{N}} \bc \Box \;|\; x.M_{A} \;|\; M_{M}+M_{N}}
  \and
  \inferrule* [lab=agent] {} {{M_{A}} \bc (\vec{x})M_{P} \;| \; \clift{P_0,\ldots,M_{P},\ldots,P_N}}
  \and \\
  \inferrule* [lab=process] {} {{M_{P}} \bc M_{N} \;| \;P|M_{P} }
\end{mathpar} 

\begin{mathpar}
  \inferrule* [lab=sychronization] {} {M_{N} \bc \Box \;|\; x?M_{F} \;|\; x!M_{C}}
  \and
  \inferrule* [lab=abstraction] {} {{M_{F}} \bc (x)M_{P} }
  \and
  \inferrule* [lab=concretion] {} {{M_{C}} \bc \langle M_{P} \rangle }
  \and \\
  \inferrule* [lab=process] {} {{M_{P}} \bc M_{N} \;| \;P|M_{P} }
\end{mathpar}

\begin{definition}[contextual application] Given a context $M$, and
  process $P$, we define the \emph{contextual application}, $M[P] :=
  M\{P/\Box\}$. That is, the contextual application of M to P is the
  substitution of $P$ for $\Box$ in $M$.
\end{definition}

$\meaningof{-} : L \to \mathcal{P}(\pi)$

\begin{mathpar}
  \inferrule* [lab=collection] {} {\meaningof{true} = \pi, \and \meaningof{~E} = \pi \setminus \meaningof{E}, \and \meaningof{E_{1} \& E_{2}} = \meaningof{E_{1}} \cap \meaningof{E_{2}}}
\end{mathpar}

\begin{mathpar}
  \inferrule* [lab=structure] {} {\meaningof{0} = \{ P \in \pi | P \equiv 0 \}, \and \\ \meaningof{E_1 | E_2} = \{ P \in \pi | P \equiv P_{1} | P_{2}, P_{1} \in \meaningof{E_{1}}, P_{2} \in \meaningof{E_2}\} }
\end{mathpar}

\begin{mathpar}
 \inferrule* [lab=behavior] {} {\meaningof{\langle a?b \rangle E} = \{ P \in \pi | P \equiv Q | u?(y)P', \\ \and \\\\ \and \\ \;\;\; u \in \meaningof{a}, \forall z.P'\{z/y\} \in \meaningof{E\{z/b\}}\}, \and \\ \meaningof{a!E} = \{ P \in \pi | P \equiv Q | x!\langle P' \rangle, x \in \meaningof{a} P' \in \meaningof{E}\} }
\end{mathpar}

\begin{mathpar}
 \inferrule* [lab=nominal] {} {\meaningof{\quotep{E}} = \{ \quotep{P} \in \quotep{\pi} | P \in \meaningof{E} \}, \and \meaningof{\quotep{P}} = \{ \quotep{Q} \in \quotep{\pi} | P \equiv Q \} \and \\ \meaningof{@\quotep{E}} = \{ P \in \pi | P \equiv @x, x \in \meaningof{E} \}}
\end{mathpar}

\begin{eqnarray*}
  \\
  \meaningof{-} : TS \to ST
\end{eqnarray*}

\begin{eqnarray*}
  \\
  L : TS \to ST
\end{eqnarray*}

\begin{eqnarray*}
  \\
  P \models E \iff P \in \meaningof{E}
\end{eqnarray*}

\begin{eqnarray*}
  P \approx_{L} Q \iff \forall E \in L. P \models E \iff Q \models E
\end{eqnarray*}

\begin{eqnarray*}
  P \approx_{K} Q
\end{eqnarray*}

\begin{eqnarray*}
  P \approx Q
\end{eqnarray*}

$\approx_{K} = \approx = \approx_{L}$

\subsubsection{Contextual duality}

Note that contexts extend the quotation operation to a family of
operations from processes to names. Given a context, $M$, we can
define a \emph{nominal context}, $\quotep{M}$ by $\quotep{M}[P] :=
\quotep{M[P]}$. To foreshadow what is to come we observe that these
operations enjoy a duality with processes very much like the duality
between vectors and maps from vectors to scalars.

Further, because the calculus is essentially higher-order, we have a
correspondence between contexts and processes. More specifically,
given a name $x$ and a context $M$ we can construct $M^{*}_{x}$ such
that 

\begin{mathpar}
  M^{*}_{x} | \lift{x}{P} \red M[P]
\end{mathpar}

namely,

\begin{mathpar}
  M^{*}_{x} := x?(u).M[\dropn{u}]
\end{mathpar}

The dependence of $M^{*}_{x}$ on a name makes it an abstraction, 

\begin{mathpar}
  M^{*} := (x)x?(u).M[\dropn{u}]
\end{mathpar}

\subsection{Additional notation}

It will sometimes be convenient to denote the process a name
quotes. We already have the notation $x = \quotep{P}$, but it will be
convenient to introduce an alternate notation, $\procn{x}$, when we
want to emphasize the connection to the use of the name. Note that, by
virtue of name equivalence, $\quotep{\procn{x}} \nameeq x$; so, the
notation is consistent with previous definitions.

Further, because names have structure it is possible to effect
substitutions on the basis of that structure. This means we need to
upgrade our notation for substitutions, which we accomplish by
adapting comprehension notation. Thus,

\begin{mathpar}
  P\{ y / x : x \in S \}
\end{mathpar}

is interpreted to mean the process derived from P by replacing (in a
capture-avoiding manner) each occurrence of $x$ in $S$ by $y$. For example,

\begin{mathpar}
  P\{ \quotep{\procn{x}|\procn{x}} / x : x \in \freenames{P} \}
\end{mathpar}

will replace each (occurrence) of a free name $x$ in $P$ by
$\quotep{\procn{x}|\procn{x}}$.

Also, we will avail ourselves of the notation $x^{L}$ and $x^{R}$ to
denote injections of a name into disjoint copies of the name
space. There are numerous ways to accomplish this. One example can be
found in \cite{MeredithR05}. This notation overloads to vectors of
names: $\vec{x}^{\pi} := (x_{i}^{\pi} \; : \; 0 \leq i < |\vec{x}| )$ where $\pi \in \{L,R\}$.

We also use $P^{\Box} := P|\Box$.

In \cite{MeredithR05} an interpretation of the new operator is
given. It turns out that there are several possible interpretations
all enjoying the requisite algebraic properties of the operator (see
\cite{milner91polyadicpi}). We will therefore make liberal use of
$(\nu\; \vec{x})P$.

% subsection the_syntax_and_semantics_of_the_notation_system (end)   

\input{qm2pi.qmops} 

\input{qm2pi.sterngerlach} 

\input{qm2pi.metric} 

% section concurrent_process_calculi (end)

%\input{qm2pi.proofsketch}

% section proof sketch (end)

%\input{qm2pi.slviaknots} 

% section spatial logic via knots (end)

\input{qm2pi.conclusion}

% section conclusion (end)

%\input{qm2pi.dtcodes} 

% section wiring algorithm (end)

\input{qm2pi.ack} 

% section acknowledgments (end)

\newpage


\bibliographystyle{plain}   
\bibliography{../../biblios/main.bib}

\input{qm2pi.rhodetails}

\end{document}

 

% section notation (end)

\input{qm2pi.process.calculi} 

% section concurrent_process_calculi_and_spatial_logics_ (end)
    
%\documentclass[12pt]{llncs}
%\documentclass{jktr}

\usepackage[pdftex]{hyperref}                   
\usepackage {listings}
\usepackage {mathpartir}
\usepackage{bcprules}
%\usepackage{listings}
                       
\usepackage{graphicx} 
%\usepackage[margins=2.5cm,nohead,nofoot]{geometry}
%\usepackage{geometry}
\usepackage{amsfonts}
\usepackage{amstext}
\usepackage{latexsym}
\usepackage{amssymb}
\usepackage{color}


%\include{myPreamble}
\include{qm2pi.local} 

%\ifpdf
%\usepackage[pdftex]{graphicx}
%\else
%\usepackage{graphicx}
%\fi

 % \ifpdf
%  \usepackage{pdfsync}
%  \if


%\title{Brief Article}
%\author{David F. Snyder}
%\author{L.G. Meredith}

%\address{Dept. of Math., Texas State University--San Marcos, San Marcos, TX 78666}
       
\pagestyle{empty}


\begin{document}

\lstset{language=[Objective]Caml,frame=shadowbox}

\input{qm2pi.front}

% section front matter (end)

\input{qm2pi.intro} 
 
% section introduction (end)

% \input{qm2pi.knotations} 

% section notation (end)

\input{qm2pi.process.calculi} 

% section concurrent_process_calculi_and_spatial_logics_ (end)
    
%\input{qm2pi.knots2pi} 

%\input{qm2pi.trefoil} 

%\input{qm2pi.mainthm} 

% subsection basic_interpretation (end)

%\input{qm2pi.rho.presentation} 
\subsection{The syntax and semantics of the notation system}\label{sub:the_syntax_and_semantics_of_the_notation_system} % (fold)

We now summarize a technical presentation of the calculus that
embodies our theory of dynamics. The typical presentation of such a
calculus follows the style of giving generators and relations on
them. The grammar, below, describing term constructors, freely
generates the set of processes, $\Proc$. This set is then quotiented
by a relation known as structural congruence and it is over this set
that the notion of dynamics is expressed. This presentation is
essentially that of \cite{MeredithR05} with the addition of
polyadicity and summation. For readability we have relegated some of
the technical subtleties to an appendix.

\subsubsection{Process grammar}\label{subsub:process_grammar}

\begin{mathpar}
  \inferrule* [lab=synchronization] {} {{M} \bc \pzero \;|\; x?F \;|\; x!C }
  \and
  \inferrule* [lab=abstraction] {} {{F} \bc (x)P}
  \and
  \inferrule* [lab=concretion] {} {{C} \bc \langle Q \rangle}
  \and
  \inferrule* [lab=process] {} {{P,Q} \bc M \;| \;P|Q \;|\; @{x}}
  \and
  \inferrule* [lab=name] {} {{x} \bc \quotep{P}}
\end{mathpar} 

Note that $\vec{x}$ (resp. $\vec{P}$) denotes a vector of names
(resp. processes) of length $|\vec{x}|$ (resp. $|\vec{P}|$). We adopt
the following useful abbreviations.

\begin{mathpar}
   x?(\vec{y}).P := x.(\vec{y})P \and  x\clift{\vec{P}} := x.\clift{\vec{P}}
   \and x!(y) := \lift{x}{\dropn{y}}
   \and \Pi_{i=0}^{n-1}P_i := P_0 | \ldots | P_{n-1}
\end{mathpar}

\subsubsection{Structural congruence}

\paragraph{Free and bound names and alpha-equivalence.} At the
core of structural equivalence is alpha-equivalence which identifies
process that are the same up to a change of variable. Formally, we
recognize the distinction between free and bound names. The free names
of a process, $\freenames{P}$, may be calculated recursively as
follows:

\begin{mathpar}
\freenames{\pzero} := \emptyset
  \and \\
  \freenames{x?(y).P} := \{ x \} \cup (\freenames{P} \setminus \{ y \})
  \and 
  \freenames{x!\langle P \rangle} := \{ x \} \cup \{ P \} 
  \and \\
  \freenames{P|Q} := \freenames{P} \cup \freenames{Q}
  \and \\
  \freenames{@{x}} := \{ x \}
\end{mathpar}

$\pi$
$\quotep{\pi}$

$\freenames{-} : \pi \to \mathcal{P}(\quotep{\pi})$

\begin{eqnarray*}
  \freenames{\pzero} & := & \emptyset \\
  \freenames{x?(y).P} & := & \{ x \} \cup (\freenames{P} \setminus \{ y \}) \\
  \freenames{x!\langle P \rangle} & := & \{ x \} \cup \{ P \} \\
  \freenames{P|Q} & := & \freenames{P} \cup \freenames{Q} \\
  \freenames{\dropn{x}} & := & \{ x \}
\end{eqnarray*}

The bound names of a process, $\boundnames{P}$, are those names occurring in $P$
that are not free. For example, in $x?(y).0$, the name $x$ is free, while $y$ is bound.

\begin{mathpar}
  \inferrule* [lab=monoidal-laws] {} { P|Q \equiv Q|P \and P|0 \equiv P \and P|(Q|R) \equiv (P|Q)|R }
\end{mathpar}

\begin{mathpar}
  \inferrule* [lab=alpha-equivalence] {} { (x)P \equiv (y)P\{y/x\} \and y \not\in \freenames{P} }
\end{mathpar}

\begin{definition}
Then two processes, $P,Q$, are alpha-equivalent if $P = Q\{\vec{y}/\vec{x}\}$ for
some $\vec{x} \in \boundnames{Q},\vec{y} \in \boundnames{P}$, where $Q\{\vec{y}/\vec{x}\}$
denotes the capture-avoiding substitution of $\vec{y}$ for $\vec{x}$ in $Q$.
\end{definition}

\begin{definition}
  The {\em structural congruence} \cite{SangiorgiWalker} , $\equiv$,
  between processes is the least congruence containing
  alpha-equivalence, satisfying the abelian monoid laws
  (associativity, commutativity and $\pzero$ as identity) for parallel
  composition $|$ and for summation $+$.
\end{definition}

\subsection{Name equivalence}

We take name equivalence, written $\nameeq$, to be the smallest
equivalence relation generated by the following rules.

\begin{mathpar}
\inferrule*[lab=Quote-drop]
{ }
{ \quotep{@{x}} \nameeq x }

\inferrule*[lab=Struct-equiv]
{ P \scong Q }
{ \quotep{P} \nameeq \quotep{Q} }
\end{mathpar}

The astute reader will have noticed that the mutual recursion of names
and processes imposes a mutual recursion on alpha-equivalence and
structural equivalence via name-equivalence. Fortunately, all of this
works out pleasantly and we may calculate in the natural way, free of
concern. The reader interested in the details is referred to the
appendix \ref{appendix:rho_details}.

\subsection{Substitution}

We use $\Proc$ for the set of processes, $\QProc$ for the set of
names, and $\id{\{}\vec{y} / \vec{x} \id{\}}$ to denote partial maps,
$s : \QProc \rightarrow \QProc$. A map, $s$ lifts, uniquely, to a map
on process terms, $\widehat{s} : \Proc \rightarrow \Proc$ by the
following equations.

\begin{mathpar}
  (0) \psubstp{Q}{P} := 0 \\
  (R \juxtap S) \psubstp{Q}{P}
  :=    
  (R)\psubstp{Q}{P} \juxtap (S) \psubstp{Q}{P} \\
  (x?(y).R) \psubstp{Q}{P}    
  :=    
  (x)\substp{Q}{P} (z)\concat( (R \psubstn{z}{y}) \psubstp{Q}{P} ) \\
  (\lift{x}{R}) \psubstp{Q}{P}  
  :=
  \lift{(x)\substp{Q}{P}}{ R \psubstp{Q}{P} } \\
%   (\dropn{x})  \psubstp{Q}{P}       
%   := 
%   \left\{ 
%     \begin{array}{ccc} 
%       \dropn{\quotep{Q}} & & x \nameeq \quotep{P} \\
%       \dropn{x} & & otherwise \\
%     \end{array}
%   \right. 
  (\dropn{x})  \psubstp{Q}{P}       
  := 
  \left\{ 
    \begin{array}{ccc} 
      Q & & x \nameeq \quotep{P} \\
      \dropn{x} & & otherwise \\
    \end{array}
  \right.
\end{mathpar}
 

where

\begin{eqnarray}
  (x)\id{\{} \lpquote Q \rpquote / \lpquote P \rpquote \id{\}}            = 
  \left\{ 
    \begin{array}{ccc}
      \lpquote Q \rpquote & & x \nameeq \lpquote P \rpquote \\
      x & & otherwise \\
    \end{array}
  \right. \nonumber
\end{eqnarray}

and $z$ is chosen distinct from $\quotep{P}$, $\quotep{Q}$, the free
names in $Q$, and all the names in $R$. Our $\alpha$-equivalence will
be built in the standard way from this substitution.

\begin{remark}\label{rem:no_self_referential_names}
  One consequence of these definitions is that $\forall P. \quotep{P}
  \not\in \freenames{P}$.
\end{remark}

\subsection{ Dynamic quote: an example }

Anticipating something of what's to come, consider applying the
substitution, $\widehat{\id{\{}u / z \id{\}}}$, to the following pair
of processes, $\lift{w}{y!(z)}$ and $w[ \lpquote y!(z) \rpquote ]$.

\begin{eqnarray}
	\lift{w}{y!(z)}\widehat{\id{\{}u / z \id{\}}}
		& = &
		\lift{w}{y!(u)} \nonumber\\
	w[ \lpquote y!(z) \rpquote ] \widehat{ \id{\{}u / z \id{\}} }
		& = &
		w[ \lpquote y!(z) \rpquote ] \nonumber
\end{eqnarray}

Because the body of the process between quotes is impervious to
substitution, we get radically different answers. In fact, by
examining the first process in an input context,
e.g. $x?(z).\lift{w}{y!(z)}$, we see that the process under the lift
operator may be shaped by prefixed inputs binding a name inside it. In
this sense, the lift operator will be seen as a way to dynamically
construct processes before reifying them as names.

Finally equipped with these standard features we can present the
dynamics of the calculus.

\subsubsection{Operational semantics} 

Finally, we introduce the computational dynamics. What marks these
algebras as distinct from other more traditionally studied algebraic
structures, e.g. vector spaces or polynomial rings, is the manner in
which dynamics is captured. In traditional structures, dynamics is typically
expressed through morphisms between such structures, as in linear maps
between vector spaces or morphisms between rings. In algebras
associated with the semantics of computation, the dynamics is
expressed as part of the algebraic structure itself, through a
reduction reduction relation typically denoted by $\red$. Below, we
give a recursive presentation of this relation for the calculus used
in the encoding.

$\red \subseteq \pi \times \pi$
$\red : \pi \to \mathcal{P}(\pi)$

\begin{mathpar}
  \inferrule* [lab=Comm] { \textsf{match}( x_{src}, x_{trgt} ) } { x_{trgt}?(y)P \; | \; x_{src}!\langle {Q} \rangle \red P\{\quotep{Q}/y}\} }
  \and \\
  \inferrule* [lab=Par] {{P} \red {P}'} {{{P} | {Q}} \red {{P}' | {Q}}}
  \and
  \inferrule* [lab=Equiv]{{{P} \scong {P}'} \andalso {{P}' \red {Q}'} \andalso {{Q}' \scong {Q}}}{{P} \red {Q}}
\end{mathpar}

\begin{eqnarray*}
  match_{\equiv} (\quotep{P},\quotep{Q}) & := & P \equiv Q \\
  match_{\dagger}(\quotep{P},\quotep{Q}) & := & \forall R. P|Q \red^{*} R => R \red^{*} 0 \\
  match_{K}(\quotep{P},\quotep{Q}) & := & K \mbox{ for some context } K
\end{eqnarray*}

$u?(x)P | u!\langle Q \rangle \red P\{\quotep{Q}/x\}$

%We write $\wred$ for $\red^*$, and $P\red$ if $\exists Q $ such that $ P \red Q$.
We write $P\red$ if $\exists Q $ such that $ P \red Q$ and $P\not\red$, otherwise.

\section{Replication}

As mentioned before, it is known that replication (and hence
recursion) can be implemented in a higher-order process algebra
\cite{SangiorgiWalker}. As our first example of calculation with the
machinery thus far presented we give the construction explicitly in
the {\rhoc}.

\begin{eqnarray}
	D_{x} & := & \prefix{x}{y}{(\binpar{\outputp{x}{y}}{@{y}})} \nonumber\\
	\bangp_{x}{P} & := & \binpar{{x}!\langle{\binpar{D_{x}}{P}}\rangle}{D_{x}} \nonumber
\end{eqnarray}

\begin{eqnarray}
	\bangp_{x}{P} & & \nonumber\\
	=
	& {x}!\langle{(\prefix{x}{y}{(\outputp{x}{y} | @{y})) | P}}\rangle 
	      | \prefix{x}{y}{(\outputp{x}{y} | @{y})} & \nonumber\\
	\red
	& (\outputp{x}{y} | @{y})\substn{\quotep{(\prefix{x}{y}{(@{y} | \outputp{x}{y})) | P}}}{y} & \nonumber\\
	=
	& \outputp{x}{\quotep{(\prefix{x}{y}{(\outputp{x}{y} | @{y})) | P}}}
	  | {(\prefix{x}{y}{(\outputp{x}{y} | @{y})) | P}} & \nonumber\\
	\red
	& \ldots & \nonumber\\
	\red^*
	& P | P | \ldots & \nonumber
\end{eqnarray}

Of course, this encoding, as an implementation, runs away, unfolding
$\bangp{P}$ eagerly. A lazier and more implementable replication
operator, restricted to input-guarded processes, may be obtained as follows.

\begin{eqnarray}
\bangp{\prefix{u}{v}{P}} 
	:= 
	\binpar{\lift{x}{\prefix{u}{v}{(\binpar{D(x)}{P})}}}{D(x)} \nonumber
\end{eqnarray}

\begin{remark}
  Note that the lazier definition still does not deal with summation
  or mixed summation (i.e. sums over input and output). The reader is
  invited to construct definitions of replication that deal with these
  features. 

  Further, the definitions are parameterized in a name, $x$. Can you,
  gentle reader, make a definition that eliminates this parameter and
  guarantees no accidental interaction between the replication
  machinery and the process being replicated -- i.e. no accidental
  sharing of names used by the process to get its work done and the
  name(s) used by the replication to effect copying. This latter
  revision of the definition of replication is crucial to obtaining
  the expected identity $!!P \sim !P$.
\end{remark}

\begin{remark}\label{rem:paradoxical_combinator}
  The reader familiar with the lambda calculus will have noticed the
  similarity between $D$ and the paradoxical combinator.

  [Ed. note: the existence of this seems to suggest we have to be more
  restrictive on the set of processes and names we admit if we are to
  support no-cloning.]
\end{remark}

\subsubsection{Bisimulation}

The computational dynamics gives rise to another kind of equivalence,
the equivalence of computational behavior. As previously mentioned
this is typically captured \emph{via} some form of bisimulation.

% The notion we use in this paper is weak barbed bisimulation
% \cite{milner91polyadicpi}.

The notion we use in this paper is derived from weak barbed
bisimulation \cite{milner91polyadicpi}. 

\begin{definition}
An \emph{observation relation}, $\downarrow_{\mathcal N}$, over a set
of names, $\mathcal N$, is the smallest relation satisfying the rules
below.

\infrule[Out-barb]{y \in {\mathcal N}, \; x \nameeq y}
		  {\outputp{x}{v} \downarrow_{\mathcal N} x}
\infrule[Par-barb]{\mbox{$P\downarrow_{\mathcal N} x$ or $Q\downarrow_{\mathcal N} x$}}
		  {\binpar{P}{Q} \downarrow_{\mathcal N} x}

We write $P \Downarrow_{\mathcal N} x$ if there is $Q$ such that 
$P \wred Q$ and $Q \downarrow_{\mathcal N} x$.
\end{definition}

\begin{definition}
%\label{def.bbisim}
An  ${\mathcal N}$-\emph{barbed bisimulation} over a set of names, ${\mathcal N}$, is a symmetric binary relation 
${\mathcal S}_{\mathcal N}$ between agents such that $P\rel{S}_{\mathcal N}Q$ implies:
\begin{enumerate}
\item If $P \red P'$ then $Q \wred Q'$ and $P'\rel{S}_{\mathcal N} Q'$.
\item If $P\downarrow_{\mathcal N} x$, then $Q\Downarrow_{\mathcal N} x$.
\end{enumerate}
$P$ is ${\mathcal N}$-barbed bisimilar to $Q$, written
$P \wbbisim_{\mathcal N} Q$, if $P \rel{S}_{\mathcal N} Q$ for some ${\mathcal N}$-barbed bisimulation ${\mathcal S}_{\mathcal N}$.
\end{definition}

$\mathcal{R} \subseteq \pi \times \pi$

$P \mathcal{R} Q => \forall P'. P \red P' \Rightarrow \exists Q'. Q \red Q', P' \mathcal{R} Q'$

$P \vdash x \Rightarrow Q \vdash x$

\begin{mathpar}
  \inferrule*[lab=Out-barb]{x \nameeq y}{{y}!\langle{Q}\rangle \vdash x}
  \and
  \inferrule*[lab=Par-barb]{\mbox{$P\vdash x$ or $Q\vdash x$}}{\binpar{P}{Q} \vdash x}
\end{mathpar}

\subsubsection{Contexts}

One of the principle advantages of computational calculi like the
$\pi$-calculus is a well-defined notion of context,
contextual-equivalence and a correlation between
contextual-equivalence and notions of bisimulation. The notion of
context allows the decomposition of a process into (sub-)process and
its syntactic environment, its context. Thus, a context may be
thought of as a process with a ``hole'' (written $\Box$) in it. The
application of a context $M$ to a process $P$, written $M[P]$, is
tantamount to filling the hole in $M$ with $P$. In this paper we do
not need the full weight of this theory, but do make use of the notion
of context in the proof the main theorem. 

\begin{mathpar}
  \inferrule* [lab=summation] {} {{M_{M},M_{N}} \bc \Box \;|\; x.M_{A} \;|\; M_{M}+M_{N}}
  \and
  \inferrule* [lab=agent] {} {{M_{A}} \bc (\vec{x})M_{P} \;| \; \clift{P_0,\ldots,M_{P},\ldots,P_N}}
  \and \\
  \inferrule* [lab=process] {} {{M_{P}} \bc M_{N} \;| \;P|M_{P} }
\end{mathpar} 

\begin{mathpar}
  \inferrule* [lab=sychronization] {} {M_{N} \bc \Box \;|\; x?M_{F} \;|\; x!M_{C}}
  \and
  \inferrule* [lab=abstraction] {} {{M_{F}} \bc (x)M_{P} }
  \and
  \inferrule* [lab=concretion] {} {{M_{C}} \bc \langle M_{P} \rangle }
  \and \\
  \inferrule* [lab=process] {} {{M_{P}} \bc M_{N} \;| \;P|M_{P} }
\end{mathpar}

\begin{definition}[contextual application] Given a context $M$, and
  process $P$, we define the \emph{contextual application}, $M[P] :=
  M\{P/\Box\}$. That is, the contextual application of M to P is the
  substitution of $P$ for $\Box$ in $M$.
\end{definition}

$\meaningof{-} : L \to \mathcal{P}(\pi)$

\begin{mathpar}
  \inferrule* [lab=collection] {} {\meaningof{true} = \pi, \and \meaningof{~E} = \pi \setminus \meaningof{E}, \and \meaningof{E_{1} \& E_{2}} = \meaningof{E_{1}} \cap \meaningof{E_{2}}}
\end{mathpar}

\begin{mathpar}
  \inferrule* [lab=structure] {} {\meaningof{0} = \{ P \in \pi | P \equiv 0 \}, \and \\ \meaningof{E_1 | E_2} = \{ P \in \pi | P \equiv P_{1} | P_{2}, P_{1} \in \meaningof{E_{1}}, P_{2} \in \meaningof{E_2}\} }
\end{mathpar}

\begin{mathpar}
 \inferrule* [lab=behavior] {} {\meaningof{\langle a?b \rangle E} = \{ P \in \pi | P \equiv Q | u?(y)P', \\ \and \\\\ \and \\ \;\;\; u \in \meaningof{a}, \forall z.P'\{z/y\} \in \meaningof{E\{z/b\}}\}, \and \\ \meaningof{a!E} = \{ P \in \pi | P \equiv Q | x!\langle P' \rangle, x \in \meaningof{a} P' \in \meaningof{E}\} }
\end{mathpar}

\begin{mathpar}
 \inferrule* [lab=nominal] {} {\meaningof{\quotep{E}} = \{ \quotep{P} \in \quotep{\pi} | P \in \meaningof{E} \}, \and \meaningof{\quotep{P}} = \{ \quotep{Q} \in \quotep{\pi} | P \equiv Q \} \and \\ \meaningof{@\quotep{E}} = \{ P \in \pi | P \equiv @x, x \in \meaningof{E} \}}
\end{mathpar}

\begin{eqnarray*}
  \\
  \meaningof{-} : TS \to ST
\end{eqnarray*}

\begin{eqnarray*}
  \\
  L : TS \to ST
\end{eqnarray*}

\begin{eqnarray*}
  \\
  P \models E \iff P \in \meaningof{E}
\end{eqnarray*}

\begin{eqnarray*}
  P \approx_{L} Q \iff \forall E \in L. P \models E \iff Q \models E
\end{eqnarray*}

\begin{eqnarray*}
  P \approx_{K} Q
\end{eqnarray*}

\begin{eqnarray*}
  P \approx Q
\end{eqnarray*}

$\approx_{K} = \approx = \approx_{L}$

\subsubsection{Contextual duality}

Note that contexts extend the quotation operation to a family of
operations from processes to names. Given a context, $M$, we can
define a \emph{nominal context}, $\quotep{M}$ by $\quotep{M}[P] :=
\quotep{M[P]}$. To foreshadow what is to come we observe that these
operations enjoy a duality with processes very much like the duality
between vectors and maps from vectors to scalars.

Further, because the calculus is essentially higher-order, we have a
correspondence between contexts and processes. More specifically,
given a name $x$ and a context $M$ we can construct $M^{*}_{x}$ such
that 

\begin{mathpar}
  M^{*}_{x} | \lift{x}{P} \red M[P]
\end{mathpar}

namely,

\begin{mathpar}
  M^{*}_{x} := x?(u).M[\dropn{u}]
\end{mathpar}

The dependence of $M^{*}_{x}$ on a name makes it an abstraction, 

\begin{mathpar}
  M^{*} := (x)x?(u).M[\dropn{u}]
\end{mathpar}

\subsection{Additional notation}

It will sometimes be convenient to denote the process a name
quotes. We already have the notation $x = \quotep{P}$, but it will be
convenient to introduce an alternate notation, $\procn{x}$, when we
want to emphasize the connection to the use of the name. Note that, by
virtue of name equivalence, $\quotep{\procn{x}} \nameeq x$; so, the
notation is consistent with previous definitions.

Further, because names have structure it is possible to effect
substitutions on the basis of that structure. This means we need to
upgrade our notation for substitutions, which we accomplish by
adapting comprehension notation. Thus,

\begin{mathpar}
  P\{ y / x : x \in S \}
\end{mathpar}

is interpreted to mean the process derived from P by replacing (in a
capture-avoiding manner) each occurrence of $x$ in $S$ by $y$. For example,

\begin{mathpar}
  P\{ \quotep{\procn{x}|\procn{x}} / x : x \in \freenames{P} \}
\end{mathpar}

will replace each (occurrence) of a free name $x$ in $P$ by
$\quotep{\procn{x}|\procn{x}}$.

Also, we will avail ourselves of the notation $x^{L}$ and $x^{R}$ to
denote injections of a name into disjoint copies of the name
space. There are numerous ways to accomplish this. One example can be
found in \cite{MeredithR05}. This notation overloads to vectors of
names: $\vec{x}^{\pi} := (x_{i}^{\pi} \; : \; 0 \leq i < |\vec{x}| )$ where $\pi \in \{L,R\}$.

We also use $P^{\Box} := P|\Box$.

In \cite{MeredithR05} an interpretation of the new operator is
given. It turns out that there are several possible interpretations
all enjoying the requisite algebraic properties of the operator (see
\cite{milner91polyadicpi}). We will therefore make liberal use of
$(\nu\; \vec{x})P$.

% subsection the_syntax_and_semantics_of_the_notation_system (end)   

\input{qm2pi.qmops} 

\input{qm2pi.sterngerlach} 

\input{qm2pi.metric} 

% section concurrent_process_calculi (end)

%\input{qm2pi.proofsketch}

% section proof sketch (end)

%\input{qm2pi.slviaknots} 

% section spatial logic via knots (end)

\input{qm2pi.conclusion}

% section conclusion (end)

%\input{qm2pi.dtcodes} 

% section wiring algorithm (end)

\input{qm2pi.ack} 

% section acknowledgments (end)

\newpage


\bibliographystyle{plain}   
\bibliography{../../biblios/main.bib}

\input{qm2pi.rhodetails}

\end{document}

 

%\documentclass[12pt]{llncs}
%\documentclass{jktr}

\usepackage[pdftex]{hyperref}                   
\usepackage {listings}
\usepackage {mathpartir}
\usepackage{bcprules}
%\usepackage{listings}
                       
\usepackage{graphicx} 
%\usepackage[margins=2.5cm,nohead,nofoot]{geometry}
%\usepackage{geometry}
\usepackage{amsfonts}
\usepackage{amstext}
\usepackage{latexsym}
\usepackage{amssymb}
\usepackage{color}


%\include{myPreamble}
\include{qm2pi.local} 

%\ifpdf
%\usepackage[pdftex]{graphicx}
%\else
%\usepackage{graphicx}
%\fi

 % \ifpdf
%  \usepackage{pdfsync}
%  \if


%\title{Brief Article}
%\author{David F. Snyder}
%\author{L.G. Meredith}

%\address{Dept. of Math., Texas State University--San Marcos, San Marcos, TX 78666}
       
\pagestyle{empty}


\begin{document}

\lstset{language=[Objective]Caml,frame=shadowbox}

\input{qm2pi.front}

% section front matter (end)

\input{qm2pi.intro} 
 
% section introduction (end)

% \input{qm2pi.knotations} 

% section notation (end)

\input{qm2pi.process.calculi} 

% section concurrent_process_calculi_and_spatial_logics_ (end)
    
%\input{qm2pi.knots2pi} 

%\input{qm2pi.trefoil} 

%\input{qm2pi.mainthm} 

% subsection basic_interpretation (end)

%\input{qm2pi.rho.presentation} 
\subsection{The syntax and semantics of the notation system}\label{sub:the_syntax_and_semantics_of_the_notation_system} % (fold)

We now summarize a technical presentation of the calculus that
embodies our theory of dynamics. The typical presentation of such a
calculus follows the style of giving generators and relations on
them. The grammar, below, describing term constructors, freely
generates the set of processes, $\Proc$. This set is then quotiented
by a relation known as structural congruence and it is over this set
that the notion of dynamics is expressed. This presentation is
essentially that of \cite{MeredithR05} with the addition of
polyadicity and summation. For readability we have relegated some of
the technical subtleties to an appendix.

\subsubsection{Process grammar}\label{subsub:process_grammar}

\begin{mathpar}
  \inferrule* [lab=synchronization] {} {{M} \bc \pzero \;|\; x?F \;|\; x!C }
  \and
  \inferrule* [lab=abstraction] {} {{F} \bc (x)P}
  \and
  \inferrule* [lab=concretion] {} {{C} \bc \langle Q \rangle}
  \and
  \inferrule* [lab=process] {} {{P,Q} \bc M \;| \;P|Q \;|\; @{x}}
  \and
  \inferrule* [lab=name] {} {{x} \bc \quotep{P}}
\end{mathpar} 

Note that $\vec{x}$ (resp. $\vec{P}$) denotes a vector of names
(resp. processes) of length $|\vec{x}|$ (resp. $|\vec{P}|$). We adopt
the following useful abbreviations.

\begin{mathpar}
   x?(\vec{y}).P := x.(\vec{y})P \and  x\clift{\vec{P}} := x.\clift{\vec{P}}
   \and x!(y) := \lift{x}{\dropn{y}}
   \and \Pi_{i=0}^{n-1}P_i := P_0 | \ldots | P_{n-1}
\end{mathpar}

\subsubsection{Structural congruence}

\paragraph{Free and bound names and alpha-equivalence.} At the
core of structural equivalence is alpha-equivalence which identifies
process that are the same up to a change of variable. Formally, we
recognize the distinction between free and bound names. The free names
of a process, $\freenames{P}$, may be calculated recursively as
follows:

\begin{mathpar}
\freenames{\pzero} := \emptyset
  \and \\
  \freenames{x?(y).P} := \{ x \} \cup (\freenames{P} \setminus \{ y \})
  \and 
  \freenames{x!\langle P \rangle} := \{ x \} \cup \{ P \} 
  \and \\
  \freenames{P|Q} := \freenames{P} \cup \freenames{Q}
  \and \\
  \freenames{@{x}} := \{ x \}
\end{mathpar}

$\pi$
$\quotep{\pi}$

$\freenames{-} : \pi \to \mathcal{P}(\quotep{\pi})$

\begin{eqnarray*}
  \freenames{\pzero} & := & \emptyset \\
  \freenames{x?(y).P} & := & \{ x \} \cup (\freenames{P} \setminus \{ y \}) \\
  \freenames{x!\langle P \rangle} & := & \{ x \} \cup \{ P \} \\
  \freenames{P|Q} & := & \freenames{P} \cup \freenames{Q} \\
  \freenames{\dropn{x}} & := & \{ x \}
\end{eqnarray*}

The bound names of a process, $\boundnames{P}$, are those names occurring in $P$
that are not free. For example, in $x?(y).0$, the name $x$ is free, while $y$ is bound.

\begin{mathpar}
  \inferrule* [lab=monoidal-laws] {} { P|Q \equiv Q|P \and P|0 \equiv P \and P|(Q|R) \equiv (P|Q)|R }
\end{mathpar}

\begin{mathpar}
  \inferrule* [lab=alpha-equivalence] {} { (x)P \equiv (y)P\{y/x\} \and y \not\in \freenames{P} }
\end{mathpar}

\begin{definition}
Then two processes, $P,Q$, are alpha-equivalent if $P = Q\{\vec{y}/\vec{x}\}$ for
some $\vec{x} \in \boundnames{Q},\vec{y} \in \boundnames{P}$, where $Q\{\vec{y}/\vec{x}\}$
denotes the capture-avoiding substitution of $\vec{y}$ for $\vec{x}$ in $Q$.
\end{definition}

\begin{definition}
  The {\em structural congruence} \cite{SangiorgiWalker} , $\equiv$,
  between processes is the least congruence containing
  alpha-equivalence, satisfying the abelian monoid laws
  (associativity, commutativity and $\pzero$ as identity) for parallel
  composition $|$ and for summation $+$.
\end{definition}

\subsection{Name equivalence}

We take name equivalence, written $\nameeq$, to be the smallest
equivalence relation generated by the following rules.

\begin{mathpar}
\inferrule*[lab=Quote-drop]
{ }
{ \quotep{@{x}} \nameeq x }

\inferrule*[lab=Struct-equiv]
{ P \scong Q }
{ \quotep{P} \nameeq \quotep{Q} }
\end{mathpar}

The astute reader will have noticed that the mutual recursion of names
and processes imposes a mutual recursion on alpha-equivalence and
structural equivalence via name-equivalence. Fortunately, all of this
works out pleasantly and we may calculate in the natural way, free of
concern. The reader interested in the details is referred to the
appendix \ref{appendix:rho_details}.

\subsection{Substitution}

We use $\Proc$ for the set of processes, $\QProc$ for the set of
names, and $\id{\{}\vec{y} / \vec{x} \id{\}}$ to denote partial maps,
$s : \QProc \rightarrow \QProc$. A map, $s$ lifts, uniquely, to a map
on process terms, $\widehat{s} : \Proc \rightarrow \Proc$ by the
following equations.

\begin{mathpar}
  (0) \psubstp{Q}{P} := 0 \\
  (R \juxtap S) \psubstp{Q}{P}
  :=    
  (R)\psubstp{Q}{P} \juxtap (S) \psubstp{Q}{P} \\
  (x?(y).R) \psubstp{Q}{P}    
  :=    
  (x)\substp{Q}{P} (z)\concat( (R \psubstn{z}{y}) \psubstp{Q}{P} ) \\
  (\lift{x}{R}) \psubstp{Q}{P}  
  :=
  \lift{(x)\substp{Q}{P}}{ R \psubstp{Q}{P} } \\
%   (\dropn{x})  \psubstp{Q}{P}       
%   := 
%   \left\{ 
%     \begin{array}{ccc} 
%       \dropn{\quotep{Q}} & & x \nameeq \quotep{P} \\
%       \dropn{x} & & otherwise \\
%     \end{array}
%   \right. 
  (\dropn{x})  \psubstp{Q}{P}       
  := 
  \left\{ 
    \begin{array}{ccc} 
      Q & & x \nameeq \quotep{P} \\
      \dropn{x} & & otherwise \\
    \end{array}
  \right.
\end{mathpar}
 

where

\begin{eqnarray}
  (x)\id{\{} \lpquote Q \rpquote / \lpquote P \rpquote \id{\}}            = 
  \left\{ 
    \begin{array}{ccc}
      \lpquote Q \rpquote & & x \nameeq \lpquote P \rpquote \\
      x & & otherwise \\
    \end{array}
  \right. \nonumber
\end{eqnarray}

and $z$ is chosen distinct from $\quotep{P}$, $\quotep{Q}$, the free
names in $Q$, and all the names in $R$. Our $\alpha$-equivalence will
be built in the standard way from this substitution.

\begin{remark}\label{rem:no_self_referential_names}
  One consequence of these definitions is that $\forall P. \quotep{P}
  \not\in \freenames{P}$.
\end{remark}

\subsection{ Dynamic quote: an example }

Anticipating something of what's to come, consider applying the
substitution, $\widehat{\id{\{}u / z \id{\}}}$, to the following pair
of processes, $\lift{w}{y!(z)}$ and $w[ \lpquote y!(z) \rpquote ]$.

\begin{eqnarray}
	\lift{w}{y!(z)}\widehat{\id{\{}u / z \id{\}}}
		& = &
		\lift{w}{y!(u)} \nonumber\\
	w[ \lpquote y!(z) \rpquote ] \widehat{ \id{\{}u / z \id{\}} }
		& = &
		w[ \lpquote y!(z) \rpquote ] \nonumber
\end{eqnarray}

Because the body of the process between quotes is impervious to
substitution, we get radically different answers. In fact, by
examining the first process in an input context,
e.g. $x?(z).\lift{w}{y!(z)}$, we see that the process under the lift
operator may be shaped by prefixed inputs binding a name inside it. In
this sense, the lift operator will be seen as a way to dynamically
construct processes before reifying them as names.

Finally equipped with these standard features we can present the
dynamics of the calculus.

\subsubsection{Operational semantics} 

Finally, we introduce the computational dynamics. What marks these
algebras as distinct from other more traditionally studied algebraic
structures, e.g. vector spaces or polynomial rings, is the manner in
which dynamics is captured. In traditional structures, dynamics is typically
expressed through morphisms between such structures, as in linear maps
between vector spaces or morphisms between rings. In algebras
associated with the semantics of computation, the dynamics is
expressed as part of the algebraic structure itself, through a
reduction reduction relation typically denoted by $\red$. Below, we
give a recursive presentation of this relation for the calculus used
in the encoding.

$\red \subseteq \pi \times \pi$
$\red : \pi \to \mathcal{P}(\pi)$

\begin{mathpar}
  \inferrule* [lab=Comm] { \textsf{match}( x_{src}, x_{trgt} ) } { x_{trgt}?(y)P \; | \; x_{src}!\langle {Q} \rangle \red P\{\quotep{Q}/y}\} }
  \and \\
  \inferrule* [lab=Par] {{P} \red {P}'} {{{P} | {Q}} \red {{P}' | {Q}}}
  \and
  \inferrule* [lab=Equiv]{{{P} \scong {P}'} \andalso {{P}' \red {Q}'} \andalso {{Q}' \scong {Q}}}{{P} \red {Q}}
\end{mathpar}

\begin{eqnarray*}
  match_{\equiv} (\quotep{P},\quotep{Q}) & := & P \equiv Q \\
  match_{\dagger}(\quotep{P},\quotep{Q}) & := & \forall R. P|Q \red^{*} R => R \red^{*} 0 \\
  match_{K}(\quotep{P},\quotep{Q}) & := & K \mbox{ for some context } K
\end{eqnarray*}

$u?(x)P | u!\langle Q \rangle \red P\{\quotep{Q}/x\}$

%We write $\wred$ for $\red^*$, and $P\red$ if $\exists Q $ such that $ P \red Q$.
We write $P\red$ if $\exists Q $ such that $ P \red Q$ and $P\not\red$, otherwise.

\section{Replication}

As mentioned before, it is known that replication (and hence
recursion) can be implemented in a higher-order process algebra
\cite{SangiorgiWalker}. As our first example of calculation with the
machinery thus far presented we give the construction explicitly in
the {\rhoc}.

\begin{eqnarray}
	D_{x} & := & \prefix{x}{y}{(\binpar{\outputp{x}{y}}{@{y}})} \nonumber\\
	\bangp_{x}{P} & := & \binpar{{x}!\langle{\binpar{D_{x}}{P}}\rangle}{D_{x}} \nonumber
\end{eqnarray}

\begin{eqnarray}
	\bangp_{x}{P} & & \nonumber\\
	=
	& {x}!\langle{(\prefix{x}{y}{(\outputp{x}{y} | @{y})) | P}}\rangle 
	      | \prefix{x}{y}{(\outputp{x}{y} | @{y})} & \nonumber\\
	\red
	& (\outputp{x}{y} | @{y})\substn{\quotep{(\prefix{x}{y}{(@{y} | \outputp{x}{y})) | P}}}{y} & \nonumber\\
	=
	& \outputp{x}{\quotep{(\prefix{x}{y}{(\outputp{x}{y} | @{y})) | P}}}
	  | {(\prefix{x}{y}{(\outputp{x}{y} | @{y})) | P}} & \nonumber\\
	\red
	& \ldots & \nonumber\\
	\red^*
	& P | P | \ldots & \nonumber
\end{eqnarray}

Of course, this encoding, as an implementation, runs away, unfolding
$\bangp{P}$ eagerly. A lazier and more implementable replication
operator, restricted to input-guarded processes, may be obtained as follows.

\begin{eqnarray}
\bangp{\prefix{u}{v}{P}} 
	:= 
	\binpar{\lift{x}{\prefix{u}{v}{(\binpar{D(x)}{P})}}}{D(x)} \nonumber
\end{eqnarray}

\begin{remark}
  Note that the lazier definition still does not deal with summation
  or mixed summation (i.e. sums over input and output). The reader is
  invited to construct definitions of replication that deal with these
  features. 

  Further, the definitions are parameterized in a name, $x$. Can you,
  gentle reader, make a definition that eliminates this parameter and
  guarantees no accidental interaction between the replication
  machinery and the process being replicated -- i.e. no accidental
  sharing of names used by the process to get its work done and the
  name(s) used by the replication to effect copying. This latter
  revision of the definition of replication is crucial to obtaining
  the expected identity $!!P \sim !P$.
\end{remark}

\begin{remark}\label{rem:paradoxical_combinator}
  The reader familiar with the lambda calculus will have noticed the
  similarity between $D$ and the paradoxical combinator.

  [Ed. note: the existence of this seems to suggest we have to be more
  restrictive on the set of processes and names we admit if we are to
  support no-cloning.]
\end{remark}

\subsubsection{Bisimulation}

The computational dynamics gives rise to another kind of equivalence,
the equivalence of computational behavior. As previously mentioned
this is typically captured \emph{via} some form of bisimulation.

% The notion we use in this paper is weak barbed bisimulation
% \cite{milner91polyadicpi}.

The notion we use in this paper is derived from weak barbed
bisimulation \cite{milner91polyadicpi}. 

\begin{definition}
An \emph{observation relation}, $\downarrow_{\mathcal N}$, over a set
of names, $\mathcal N$, is the smallest relation satisfying the rules
below.

\infrule[Out-barb]{y \in {\mathcal N}, \; x \nameeq y}
		  {\outputp{x}{v} \downarrow_{\mathcal N} x}
\infrule[Par-barb]{\mbox{$P\downarrow_{\mathcal N} x$ or $Q\downarrow_{\mathcal N} x$}}
		  {\binpar{P}{Q} \downarrow_{\mathcal N} x}

We write $P \Downarrow_{\mathcal N} x$ if there is $Q$ such that 
$P \wred Q$ and $Q \downarrow_{\mathcal N} x$.
\end{definition}

\begin{definition}
%\label{def.bbisim}
An  ${\mathcal N}$-\emph{barbed bisimulation} over a set of names, ${\mathcal N}$, is a symmetric binary relation 
${\mathcal S}_{\mathcal N}$ between agents such that $P\rel{S}_{\mathcal N}Q$ implies:
\begin{enumerate}
\item If $P \red P'$ then $Q \wred Q'$ and $P'\rel{S}_{\mathcal N} Q'$.
\item If $P\downarrow_{\mathcal N} x$, then $Q\Downarrow_{\mathcal N} x$.
\end{enumerate}
$P$ is ${\mathcal N}$-barbed bisimilar to $Q$, written
$P \wbbisim_{\mathcal N} Q$, if $P \rel{S}_{\mathcal N} Q$ for some ${\mathcal N}$-barbed bisimulation ${\mathcal S}_{\mathcal N}$.
\end{definition}

$\mathcal{R} \subseteq \pi \times \pi$

$P \mathcal{R} Q => \forall P'. P \red P' \Rightarrow \exists Q'. Q \red Q', P' \mathcal{R} Q'$

$P \vdash x \Rightarrow Q \vdash x$

\begin{mathpar}
  \inferrule*[lab=Out-barb]{x \nameeq y}{{y}!\langle{Q}\rangle \vdash x}
  \and
  \inferrule*[lab=Par-barb]{\mbox{$P\vdash x$ or $Q\vdash x$}}{\binpar{P}{Q} \vdash x}
\end{mathpar}

\subsubsection{Contexts}

One of the principle advantages of computational calculi like the
$\pi$-calculus is a well-defined notion of context,
contextual-equivalence and a correlation between
contextual-equivalence and notions of bisimulation. The notion of
context allows the decomposition of a process into (sub-)process and
its syntactic environment, its context. Thus, a context may be
thought of as a process with a ``hole'' (written $\Box$) in it. The
application of a context $M$ to a process $P$, written $M[P]$, is
tantamount to filling the hole in $M$ with $P$. In this paper we do
not need the full weight of this theory, but do make use of the notion
of context in the proof the main theorem. 

\begin{mathpar}
  \inferrule* [lab=summation] {} {{M_{M},M_{N}} \bc \Box \;|\; x.M_{A} \;|\; M_{M}+M_{N}}
  \and
  \inferrule* [lab=agent] {} {{M_{A}} \bc (\vec{x})M_{P} \;| \; \clift{P_0,\ldots,M_{P},\ldots,P_N}}
  \and \\
  \inferrule* [lab=process] {} {{M_{P}} \bc M_{N} \;| \;P|M_{P} }
\end{mathpar} 

\begin{mathpar}
  \inferrule* [lab=sychronization] {} {M_{N} \bc \Box \;|\; x?M_{F} \;|\; x!M_{C}}
  \and
  \inferrule* [lab=abstraction] {} {{M_{F}} \bc (x)M_{P} }
  \and
  \inferrule* [lab=concretion] {} {{M_{C}} \bc \langle M_{P} \rangle }
  \and \\
  \inferrule* [lab=process] {} {{M_{P}} \bc M_{N} \;| \;P|M_{P} }
\end{mathpar}

\begin{definition}[contextual application] Given a context $M$, and
  process $P$, we define the \emph{contextual application}, $M[P] :=
  M\{P/\Box\}$. That is, the contextual application of M to P is the
  substitution of $P$ for $\Box$ in $M$.
\end{definition}

$\meaningof{-} : L \to \mathcal{P}(\pi)$

\begin{mathpar}
  \inferrule* [lab=collection] {} {\meaningof{true} = \pi, \and \meaningof{~E} = \pi \setminus \meaningof{E}, \and \meaningof{E_{1} \& E_{2}} = \meaningof{E_{1}} \cap \meaningof{E_{2}}}
\end{mathpar}

\begin{mathpar}
  \inferrule* [lab=structure] {} {\meaningof{0} = \{ P \in \pi | P \equiv 0 \}, \and \\ \meaningof{E_1 | E_2} = \{ P \in \pi | P \equiv P_{1} | P_{2}, P_{1} \in \meaningof{E_{1}}, P_{2} \in \meaningof{E_2}\} }
\end{mathpar}

\begin{mathpar}
 \inferrule* [lab=behavior] {} {\meaningof{\langle a?b \rangle E} = \{ P \in \pi | P \equiv Q | u?(y)P', \\ \and \\\\ \and \\ \;\;\; u \in \meaningof{a}, \forall z.P'\{z/y\} \in \meaningof{E\{z/b\}}\}, \and \\ \meaningof{a!E} = \{ P \in \pi | P \equiv Q | x!\langle P' \rangle, x \in \meaningof{a} P' \in \meaningof{E}\} }
\end{mathpar}

\begin{mathpar}
 \inferrule* [lab=nominal] {} {\meaningof{\quotep{E}} = \{ \quotep{P} \in \quotep{\pi} | P \in \meaningof{E} \}, \and \meaningof{\quotep{P}} = \{ \quotep{Q} \in \quotep{\pi} | P \equiv Q \} \and \\ \meaningof{@\quotep{E}} = \{ P \in \pi | P \equiv @x, x \in \meaningof{E} \}}
\end{mathpar}

\begin{eqnarray*}
  \\
  \meaningof{-} : TS \to ST
\end{eqnarray*}

\begin{eqnarray*}
  \\
  L : TS \to ST
\end{eqnarray*}

\begin{eqnarray*}
  \\
  P \models E \iff P \in \meaningof{E}
\end{eqnarray*}

\begin{eqnarray*}
  P \approx_{L} Q \iff \forall E \in L. P \models E \iff Q \models E
\end{eqnarray*}

\begin{eqnarray*}
  P \approx_{K} Q
\end{eqnarray*}

\begin{eqnarray*}
  P \approx Q
\end{eqnarray*}

$\approx_{K} = \approx = \approx_{L}$

\subsubsection{Contextual duality}

Note that contexts extend the quotation operation to a family of
operations from processes to names. Given a context, $M$, we can
define a \emph{nominal context}, $\quotep{M}$ by $\quotep{M}[P] :=
\quotep{M[P]}$. To foreshadow what is to come we observe that these
operations enjoy a duality with processes very much like the duality
between vectors and maps from vectors to scalars.

Further, because the calculus is essentially higher-order, we have a
correspondence between contexts and processes. More specifically,
given a name $x$ and a context $M$ we can construct $M^{*}_{x}$ such
that 

\begin{mathpar}
  M^{*}_{x} | \lift{x}{P} \red M[P]
\end{mathpar}

namely,

\begin{mathpar}
  M^{*}_{x} := x?(u).M[\dropn{u}]
\end{mathpar}

The dependence of $M^{*}_{x}$ on a name makes it an abstraction, 

\begin{mathpar}
  M^{*} := (x)x?(u).M[\dropn{u}]
\end{mathpar}

\subsection{Additional notation}

It will sometimes be convenient to denote the process a name
quotes. We already have the notation $x = \quotep{P}$, but it will be
convenient to introduce an alternate notation, $\procn{x}$, when we
want to emphasize the connection to the use of the name. Note that, by
virtue of name equivalence, $\quotep{\procn{x}} \nameeq x$; so, the
notation is consistent with previous definitions.

Further, because names have structure it is possible to effect
substitutions on the basis of that structure. This means we need to
upgrade our notation for substitutions, which we accomplish by
adapting comprehension notation. Thus,

\begin{mathpar}
  P\{ y / x : x \in S \}
\end{mathpar}

is interpreted to mean the process derived from P by replacing (in a
capture-avoiding manner) each occurrence of $x$ in $S$ by $y$. For example,

\begin{mathpar}
  P\{ \quotep{\procn{x}|\procn{x}} / x : x \in \freenames{P} \}
\end{mathpar}

will replace each (occurrence) of a free name $x$ in $P$ by
$\quotep{\procn{x}|\procn{x}}$.

Also, we will avail ourselves of the notation $x^{L}$ and $x^{R}$ to
denote injections of a name into disjoint copies of the name
space. There are numerous ways to accomplish this. One example can be
found in \cite{MeredithR05}. This notation overloads to vectors of
names: $\vec{x}^{\pi} := (x_{i}^{\pi} \; : \; 0 \leq i < |\vec{x}| )$ where $\pi \in \{L,R\}$.

We also use $P^{\Box} := P|\Box$.

In \cite{MeredithR05} an interpretation of the new operator is
given. It turns out that there are several possible interpretations
all enjoying the requisite algebraic properties of the operator (see
\cite{milner91polyadicpi}). We will therefore make liberal use of
$(\nu\; \vec{x})P$.

% subsection the_syntax_and_semantics_of_the_notation_system (end)   

\input{qm2pi.qmops} 

\input{qm2pi.sterngerlach} 

\input{qm2pi.metric} 

% section concurrent_process_calculi (end)

%\input{qm2pi.proofsketch}

% section proof sketch (end)

%\input{qm2pi.slviaknots} 

% section spatial logic via knots (end)

\input{qm2pi.conclusion}

% section conclusion (end)

%\input{qm2pi.dtcodes} 

% section wiring algorithm (end)

\input{qm2pi.ack} 

% section acknowledgments (end)

\newpage


\bibliographystyle{plain}   
\bibliography{../../biblios/main.bib}

\input{qm2pi.rhodetails}

\end{document}

 

%\documentclass[12pt]{llncs}
%\documentclass{jktr}

\usepackage[pdftex]{hyperref}                   
\usepackage {listings}
\usepackage {mathpartir}
\usepackage{bcprules}
%\usepackage{listings}
                       
\usepackage{graphicx} 
%\usepackage[margins=2.5cm,nohead,nofoot]{geometry}
%\usepackage{geometry}
\usepackage{amsfonts}
\usepackage{amstext}
\usepackage{latexsym}
\usepackage{amssymb}
\usepackage{color}


%\include{myPreamble}
\include{qm2pi.local} 

%\ifpdf
%\usepackage[pdftex]{graphicx}
%\else
%\usepackage{graphicx}
%\fi

 % \ifpdf
%  \usepackage{pdfsync}
%  \if


%\title{Brief Article}
%\author{David F. Snyder}
%\author{L.G. Meredith}

%\address{Dept. of Math., Texas State University--San Marcos, San Marcos, TX 78666}
       
\pagestyle{empty}


\begin{document}

\lstset{language=[Objective]Caml,frame=shadowbox}

\input{qm2pi.front}

% section front matter (end)

\input{qm2pi.intro} 
 
% section introduction (end)

% \input{qm2pi.knotations} 

% section notation (end)

\input{qm2pi.process.calculi} 

% section concurrent_process_calculi_and_spatial_logics_ (end)
    
%\input{qm2pi.knots2pi} 

%\input{qm2pi.trefoil} 

%\input{qm2pi.mainthm} 

% subsection basic_interpretation (end)

%\input{qm2pi.rho.presentation} 
\subsection{The syntax and semantics of the notation system}\label{sub:the_syntax_and_semantics_of_the_notation_system} % (fold)

We now summarize a technical presentation of the calculus that
embodies our theory of dynamics. The typical presentation of such a
calculus follows the style of giving generators and relations on
them. The grammar, below, describing term constructors, freely
generates the set of processes, $\Proc$. This set is then quotiented
by a relation known as structural congruence and it is over this set
that the notion of dynamics is expressed. This presentation is
essentially that of \cite{MeredithR05} with the addition of
polyadicity and summation. For readability we have relegated some of
the technical subtleties to an appendix.

\subsubsection{Process grammar}\label{subsub:process_grammar}

\begin{mathpar}
  \inferrule* [lab=synchronization] {} {{M} \bc \pzero \;|\; x?F \;|\; x!C }
  \and
  \inferrule* [lab=abstraction] {} {{F} \bc (x)P}
  \and
  \inferrule* [lab=concretion] {} {{C} \bc \langle Q \rangle}
  \and
  \inferrule* [lab=process] {} {{P,Q} \bc M \;| \;P|Q \;|\; @{x}}
  \and
  \inferrule* [lab=name] {} {{x} \bc \quotep{P}}
\end{mathpar} 

Note that $\vec{x}$ (resp. $\vec{P}$) denotes a vector of names
(resp. processes) of length $|\vec{x}|$ (resp. $|\vec{P}|$). We adopt
the following useful abbreviations.

\begin{mathpar}
   x?(\vec{y}).P := x.(\vec{y})P \and  x\clift{\vec{P}} := x.\clift{\vec{P}}
   \and x!(y) := \lift{x}{\dropn{y}}
   \and \Pi_{i=0}^{n-1}P_i := P_0 | \ldots | P_{n-1}
\end{mathpar}

\subsubsection{Structural congruence}

\paragraph{Free and bound names and alpha-equivalence.} At the
core of structural equivalence is alpha-equivalence which identifies
process that are the same up to a change of variable. Formally, we
recognize the distinction between free and bound names. The free names
of a process, $\freenames{P}$, may be calculated recursively as
follows:

\begin{mathpar}
\freenames{\pzero} := \emptyset
  \and \\
  \freenames{x?(y).P} := \{ x \} \cup (\freenames{P} \setminus \{ y \})
  \and 
  \freenames{x!\langle P \rangle} := \{ x \} \cup \{ P \} 
  \and \\
  \freenames{P|Q} := \freenames{P} \cup \freenames{Q}
  \and \\
  \freenames{@{x}} := \{ x \}
\end{mathpar}

$\pi$
$\quotep{\pi}$

$\freenames{-} : \pi \to \mathcal{P}(\quotep{\pi})$

\begin{eqnarray*}
  \freenames{\pzero} & := & \emptyset \\
  \freenames{x?(y).P} & := & \{ x \} \cup (\freenames{P} \setminus \{ y \}) \\
  \freenames{x!\langle P \rangle} & := & \{ x \} \cup \{ P \} \\
  \freenames{P|Q} & := & \freenames{P} \cup \freenames{Q} \\
  \freenames{\dropn{x}} & := & \{ x \}
\end{eqnarray*}

The bound names of a process, $\boundnames{P}$, are those names occurring in $P$
that are not free. For example, in $x?(y).0$, the name $x$ is free, while $y$ is bound.

\begin{mathpar}
  \inferrule* [lab=monoidal-laws] {} { P|Q \equiv Q|P \and P|0 \equiv P \and P|(Q|R) \equiv (P|Q)|R }
\end{mathpar}

\begin{mathpar}
  \inferrule* [lab=alpha-equivalence] {} { (x)P \equiv (y)P\{y/x\} \and y \not\in \freenames{P} }
\end{mathpar}

\begin{definition}
Then two processes, $P,Q$, are alpha-equivalent if $P = Q\{\vec{y}/\vec{x}\}$ for
some $\vec{x} \in \boundnames{Q},\vec{y} \in \boundnames{P}$, where $Q\{\vec{y}/\vec{x}\}$
denotes the capture-avoiding substitution of $\vec{y}$ for $\vec{x}$ in $Q$.
\end{definition}

\begin{definition}
  The {\em structural congruence} \cite{SangiorgiWalker} , $\equiv$,
  between processes is the least congruence containing
  alpha-equivalence, satisfying the abelian monoid laws
  (associativity, commutativity and $\pzero$ as identity) for parallel
  composition $|$ and for summation $+$.
\end{definition}

\subsection{Name equivalence}

We take name equivalence, written $\nameeq$, to be the smallest
equivalence relation generated by the following rules.

\begin{mathpar}
\inferrule*[lab=Quote-drop]
{ }
{ \quotep{@{x}} \nameeq x }

\inferrule*[lab=Struct-equiv]
{ P \scong Q }
{ \quotep{P} \nameeq \quotep{Q} }
\end{mathpar}

The astute reader will have noticed that the mutual recursion of names
and processes imposes a mutual recursion on alpha-equivalence and
structural equivalence via name-equivalence. Fortunately, all of this
works out pleasantly and we may calculate in the natural way, free of
concern. The reader interested in the details is referred to the
appendix \ref{appendix:rho_details}.

\subsection{Substitution}

We use $\Proc$ for the set of processes, $\QProc$ for the set of
names, and $\id{\{}\vec{y} / \vec{x} \id{\}}$ to denote partial maps,
$s : \QProc \rightarrow \QProc$. A map, $s$ lifts, uniquely, to a map
on process terms, $\widehat{s} : \Proc \rightarrow \Proc$ by the
following equations.

\begin{mathpar}
  (0) \psubstp{Q}{P} := 0 \\
  (R \juxtap S) \psubstp{Q}{P}
  :=    
  (R)\psubstp{Q}{P} \juxtap (S) \psubstp{Q}{P} \\
  (x?(y).R) \psubstp{Q}{P}    
  :=    
  (x)\substp{Q}{P} (z)\concat( (R \psubstn{z}{y}) \psubstp{Q}{P} ) \\
  (\lift{x}{R}) \psubstp{Q}{P}  
  :=
  \lift{(x)\substp{Q}{P}}{ R \psubstp{Q}{P} } \\
%   (\dropn{x})  \psubstp{Q}{P}       
%   := 
%   \left\{ 
%     \begin{array}{ccc} 
%       \dropn{\quotep{Q}} & & x \nameeq \quotep{P} \\
%       \dropn{x} & & otherwise \\
%     \end{array}
%   \right. 
  (\dropn{x})  \psubstp{Q}{P}       
  := 
  \left\{ 
    \begin{array}{ccc} 
      Q & & x \nameeq \quotep{P} \\
      \dropn{x} & & otherwise \\
    \end{array}
  \right.
\end{mathpar}
 

where

\begin{eqnarray}
  (x)\id{\{} \lpquote Q \rpquote / \lpquote P \rpquote \id{\}}            = 
  \left\{ 
    \begin{array}{ccc}
      \lpquote Q \rpquote & & x \nameeq \lpquote P \rpquote \\
      x & & otherwise \\
    \end{array}
  \right. \nonumber
\end{eqnarray}

and $z$ is chosen distinct from $\quotep{P}$, $\quotep{Q}$, the free
names in $Q$, and all the names in $R$. Our $\alpha$-equivalence will
be built in the standard way from this substitution.

\begin{remark}\label{rem:no_self_referential_names}
  One consequence of these definitions is that $\forall P. \quotep{P}
  \not\in \freenames{P}$.
\end{remark}

\subsection{ Dynamic quote: an example }

Anticipating something of what's to come, consider applying the
substitution, $\widehat{\id{\{}u / z \id{\}}}$, to the following pair
of processes, $\lift{w}{y!(z)}$ and $w[ \lpquote y!(z) \rpquote ]$.

\begin{eqnarray}
	\lift{w}{y!(z)}\widehat{\id{\{}u / z \id{\}}}
		& = &
		\lift{w}{y!(u)} \nonumber\\
	w[ \lpquote y!(z) \rpquote ] \widehat{ \id{\{}u / z \id{\}} }
		& = &
		w[ \lpquote y!(z) \rpquote ] \nonumber
\end{eqnarray}

Because the body of the process between quotes is impervious to
substitution, we get radically different answers. In fact, by
examining the first process in an input context,
e.g. $x?(z).\lift{w}{y!(z)}$, we see that the process under the lift
operator may be shaped by prefixed inputs binding a name inside it. In
this sense, the lift operator will be seen as a way to dynamically
construct processes before reifying them as names.

Finally equipped with these standard features we can present the
dynamics of the calculus.

\subsubsection{Operational semantics} 

Finally, we introduce the computational dynamics. What marks these
algebras as distinct from other more traditionally studied algebraic
structures, e.g. vector spaces or polynomial rings, is the manner in
which dynamics is captured. In traditional structures, dynamics is typically
expressed through morphisms between such structures, as in linear maps
between vector spaces or morphisms between rings. In algebras
associated with the semantics of computation, the dynamics is
expressed as part of the algebraic structure itself, through a
reduction reduction relation typically denoted by $\red$. Below, we
give a recursive presentation of this relation for the calculus used
in the encoding.

$\red \subseteq \pi \times \pi$
$\red : \pi \to \mathcal{P}(\pi)$

\begin{mathpar}
  \inferrule* [lab=Comm] { \textsf{match}( x_{src}, x_{trgt} ) } { x_{trgt}?(y)P \; | \; x_{src}!\langle {Q} \rangle \red P\{\quotep{Q}/y}\} }
  \and \\
  \inferrule* [lab=Par] {{P} \red {P}'} {{{P} | {Q}} \red {{P}' | {Q}}}
  \and
  \inferrule* [lab=Equiv]{{{P} \scong {P}'} \andalso {{P}' \red {Q}'} \andalso {{Q}' \scong {Q}}}{{P} \red {Q}}
\end{mathpar}

\begin{eqnarray*}
  match_{\equiv} (\quotep{P},\quotep{Q}) & := & P \equiv Q \\
  match_{\dagger}(\quotep{P},\quotep{Q}) & := & \forall R. P|Q \red^{*} R => R \red^{*} 0 \\
  match_{K}(\quotep{P},\quotep{Q}) & := & K \mbox{ for some context } K
\end{eqnarray*}

$u?(x)P | u!\langle Q \rangle \red P\{\quotep{Q}/x\}$

%We write $\wred$ for $\red^*$, and $P\red$ if $\exists Q $ such that $ P \red Q$.
We write $P\red$ if $\exists Q $ such that $ P \red Q$ and $P\not\red$, otherwise.

\section{Replication}

As mentioned before, it is known that replication (and hence
recursion) can be implemented in a higher-order process algebra
\cite{SangiorgiWalker}. As our first example of calculation with the
machinery thus far presented we give the construction explicitly in
the {\rhoc}.

\begin{eqnarray}
	D_{x} & := & \prefix{x}{y}{(\binpar{\outputp{x}{y}}{@{y}})} \nonumber\\
	\bangp_{x}{P} & := & \binpar{{x}!\langle{\binpar{D_{x}}{P}}\rangle}{D_{x}} \nonumber
\end{eqnarray}

\begin{eqnarray}
	\bangp_{x}{P} & & \nonumber\\
	=
	& {x}!\langle{(\prefix{x}{y}{(\outputp{x}{y} | @{y})) | P}}\rangle 
	      | \prefix{x}{y}{(\outputp{x}{y} | @{y})} & \nonumber\\
	\red
	& (\outputp{x}{y} | @{y})\substn{\quotep{(\prefix{x}{y}{(@{y} | \outputp{x}{y})) | P}}}{y} & \nonumber\\
	=
	& \outputp{x}{\quotep{(\prefix{x}{y}{(\outputp{x}{y} | @{y})) | P}}}
	  | {(\prefix{x}{y}{(\outputp{x}{y} | @{y})) | P}} & \nonumber\\
	\red
	& \ldots & \nonumber\\
	\red^*
	& P | P | \ldots & \nonumber
\end{eqnarray}

Of course, this encoding, as an implementation, runs away, unfolding
$\bangp{P}$ eagerly. A lazier and more implementable replication
operator, restricted to input-guarded processes, may be obtained as follows.

\begin{eqnarray}
\bangp{\prefix{u}{v}{P}} 
	:= 
	\binpar{\lift{x}{\prefix{u}{v}{(\binpar{D(x)}{P})}}}{D(x)} \nonumber
\end{eqnarray}

\begin{remark}
  Note that the lazier definition still does not deal with summation
  or mixed summation (i.e. sums over input and output). The reader is
  invited to construct definitions of replication that deal with these
  features. 

  Further, the definitions are parameterized in a name, $x$. Can you,
  gentle reader, make a definition that eliminates this parameter and
  guarantees no accidental interaction between the replication
  machinery and the process being replicated -- i.e. no accidental
  sharing of names used by the process to get its work done and the
  name(s) used by the replication to effect copying. This latter
  revision of the definition of replication is crucial to obtaining
  the expected identity $!!P \sim !P$.
\end{remark}

\begin{remark}\label{rem:paradoxical_combinator}
  The reader familiar with the lambda calculus will have noticed the
  similarity between $D$ and the paradoxical combinator.

  [Ed. note: the existence of this seems to suggest we have to be more
  restrictive on the set of processes and names we admit if we are to
  support no-cloning.]
\end{remark}

\subsubsection{Bisimulation}

The computational dynamics gives rise to another kind of equivalence,
the equivalence of computational behavior. As previously mentioned
this is typically captured \emph{via} some form of bisimulation.

% The notion we use in this paper is weak barbed bisimulation
% \cite{milner91polyadicpi}.

The notion we use in this paper is derived from weak barbed
bisimulation \cite{milner91polyadicpi}. 

\begin{definition}
An \emph{observation relation}, $\downarrow_{\mathcal N}$, over a set
of names, $\mathcal N$, is the smallest relation satisfying the rules
below.

\infrule[Out-barb]{y \in {\mathcal N}, \; x \nameeq y}
		  {\outputp{x}{v} \downarrow_{\mathcal N} x}
\infrule[Par-barb]{\mbox{$P\downarrow_{\mathcal N} x$ or $Q\downarrow_{\mathcal N} x$}}
		  {\binpar{P}{Q} \downarrow_{\mathcal N} x}

We write $P \Downarrow_{\mathcal N} x$ if there is $Q$ such that 
$P \wred Q$ and $Q \downarrow_{\mathcal N} x$.
\end{definition}

\begin{definition}
%\label{def.bbisim}
An  ${\mathcal N}$-\emph{barbed bisimulation} over a set of names, ${\mathcal N}$, is a symmetric binary relation 
${\mathcal S}_{\mathcal N}$ between agents such that $P\rel{S}_{\mathcal N}Q$ implies:
\begin{enumerate}
\item If $P \red P'$ then $Q \wred Q'$ and $P'\rel{S}_{\mathcal N} Q'$.
\item If $P\downarrow_{\mathcal N} x$, then $Q\Downarrow_{\mathcal N} x$.
\end{enumerate}
$P$ is ${\mathcal N}$-barbed bisimilar to $Q$, written
$P \wbbisim_{\mathcal N} Q$, if $P \rel{S}_{\mathcal N} Q$ for some ${\mathcal N}$-barbed bisimulation ${\mathcal S}_{\mathcal N}$.
\end{definition}

$\mathcal{R} \subseteq \pi \times \pi$

$P \mathcal{R} Q => \forall P'. P \red P' \Rightarrow \exists Q'. Q \red Q', P' \mathcal{R} Q'$

$P \vdash x \Rightarrow Q \vdash x$

\begin{mathpar}
  \inferrule*[lab=Out-barb]{x \nameeq y}{{y}!\langle{Q}\rangle \vdash x}
  \and
  \inferrule*[lab=Par-barb]{\mbox{$P\vdash x$ or $Q\vdash x$}}{\binpar{P}{Q} \vdash x}
\end{mathpar}

\subsubsection{Contexts}

One of the principle advantages of computational calculi like the
$\pi$-calculus is a well-defined notion of context,
contextual-equivalence and a correlation between
contextual-equivalence and notions of bisimulation. The notion of
context allows the decomposition of a process into (sub-)process and
its syntactic environment, its context. Thus, a context may be
thought of as a process with a ``hole'' (written $\Box$) in it. The
application of a context $M$ to a process $P$, written $M[P]$, is
tantamount to filling the hole in $M$ with $P$. In this paper we do
not need the full weight of this theory, but do make use of the notion
of context in the proof the main theorem. 

\begin{mathpar}
  \inferrule* [lab=summation] {} {{M_{M},M_{N}} \bc \Box \;|\; x.M_{A} \;|\; M_{M}+M_{N}}
  \and
  \inferrule* [lab=agent] {} {{M_{A}} \bc (\vec{x})M_{P} \;| \; \clift{P_0,\ldots,M_{P},\ldots,P_N}}
  \and \\
  \inferrule* [lab=process] {} {{M_{P}} \bc M_{N} \;| \;P|M_{P} }
\end{mathpar} 

\begin{mathpar}
  \inferrule* [lab=sychronization] {} {M_{N} \bc \Box \;|\; x?M_{F} \;|\; x!M_{C}}
  \and
  \inferrule* [lab=abstraction] {} {{M_{F}} \bc (x)M_{P} }
  \and
  \inferrule* [lab=concretion] {} {{M_{C}} \bc \langle M_{P} \rangle }
  \and \\
  \inferrule* [lab=process] {} {{M_{P}} \bc M_{N} \;| \;P|M_{P} }
\end{mathpar}

\begin{definition}[contextual application] Given a context $M$, and
  process $P$, we define the \emph{contextual application}, $M[P] :=
  M\{P/\Box\}$. That is, the contextual application of M to P is the
  substitution of $P$ for $\Box$ in $M$.
\end{definition}

$\meaningof{-} : L \to \mathcal{P}(\pi)$

\begin{mathpar}
  \inferrule* [lab=collection] {} {\meaningof{true} = \pi, \and \meaningof{~E} = \pi \setminus \meaningof{E}, \and \meaningof{E_{1} \& E_{2}} = \meaningof{E_{1}} \cap \meaningof{E_{2}}}
\end{mathpar}

\begin{mathpar}
  \inferrule* [lab=structure] {} {\meaningof{0} = \{ P \in \pi | P \equiv 0 \}, \and \\ \meaningof{E_1 | E_2} = \{ P \in \pi | P \equiv P_{1} | P_{2}, P_{1} \in \meaningof{E_{1}}, P_{2} \in \meaningof{E_2}\} }
\end{mathpar}

\begin{mathpar}
 \inferrule* [lab=behavior] {} {\meaningof{\langle a?b \rangle E} = \{ P \in \pi | P \equiv Q | u?(y)P', \\ \and \\\\ \and \\ \;\;\; u \in \meaningof{a}, \forall z.P'\{z/y\} \in \meaningof{E\{z/b\}}\}, \and \\ \meaningof{a!E} = \{ P \in \pi | P \equiv Q | x!\langle P' \rangle, x \in \meaningof{a} P' \in \meaningof{E}\} }
\end{mathpar}

\begin{mathpar}
 \inferrule* [lab=nominal] {} {\meaningof{\quotep{E}} = \{ \quotep{P} \in \quotep{\pi} | P \in \meaningof{E} \}, \and \meaningof{\quotep{P}} = \{ \quotep{Q} \in \quotep{\pi} | P \equiv Q \} \and \\ \meaningof{@\quotep{E}} = \{ P \in \pi | P \equiv @x, x \in \meaningof{E} \}}
\end{mathpar}

\begin{eqnarray*}
  \\
  \meaningof{-} : TS \to ST
\end{eqnarray*}

\begin{eqnarray*}
  \\
  L : TS \to ST
\end{eqnarray*}

\begin{eqnarray*}
  \\
  P \models E \iff P \in \meaningof{E}
\end{eqnarray*}

\begin{eqnarray*}
  P \approx_{L} Q \iff \forall E \in L. P \models E \iff Q \models E
\end{eqnarray*}

\begin{eqnarray*}
  P \approx_{K} Q
\end{eqnarray*}

\begin{eqnarray*}
  P \approx Q
\end{eqnarray*}

$\approx_{K} = \approx = \approx_{L}$

\subsubsection{Contextual duality}

Note that contexts extend the quotation operation to a family of
operations from processes to names. Given a context, $M$, we can
define a \emph{nominal context}, $\quotep{M}$ by $\quotep{M}[P] :=
\quotep{M[P]}$. To foreshadow what is to come we observe that these
operations enjoy a duality with processes very much like the duality
between vectors and maps from vectors to scalars.

Further, because the calculus is essentially higher-order, we have a
correspondence between contexts and processes. More specifically,
given a name $x$ and a context $M$ we can construct $M^{*}_{x}$ such
that 

\begin{mathpar}
  M^{*}_{x} | \lift{x}{P} \red M[P]
\end{mathpar}

namely,

\begin{mathpar}
  M^{*}_{x} := x?(u).M[\dropn{u}]
\end{mathpar}

The dependence of $M^{*}_{x}$ on a name makes it an abstraction, 

\begin{mathpar}
  M^{*} := (x)x?(u).M[\dropn{u}]
\end{mathpar}

\subsection{Additional notation}

It will sometimes be convenient to denote the process a name
quotes. We already have the notation $x = \quotep{P}$, but it will be
convenient to introduce an alternate notation, $\procn{x}$, when we
want to emphasize the connection to the use of the name. Note that, by
virtue of name equivalence, $\quotep{\procn{x}} \nameeq x$; so, the
notation is consistent with previous definitions.

Further, because names have structure it is possible to effect
substitutions on the basis of that structure. This means we need to
upgrade our notation for substitutions, which we accomplish by
adapting comprehension notation. Thus,

\begin{mathpar}
  P\{ y / x : x \in S \}
\end{mathpar}

is interpreted to mean the process derived from P by replacing (in a
capture-avoiding manner) each occurrence of $x$ in $S$ by $y$. For example,

\begin{mathpar}
  P\{ \quotep{\procn{x}|\procn{x}} / x : x \in \freenames{P} \}
\end{mathpar}

will replace each (occurrence) of a free name $x$ in $P$ by
$\quotep{\procn{x}|\procn{x}}$.

Also, we will avail ourselves of the notation $x^{L}$ and $x^{R}$ to
denote injections of a name into disjoint copies of the name
space. There are numerous ways to accomplish this. One example can be
found in \cite{MeredithR05}. This notation overloads to vectors of
names: $\vec{x}^{\pi} := (x_{i}^{\pi} \; : \; 0 \leq i < |\vec{x}| )$ where $\pi \in \{L,R\}$.

We also use $P^{\Box} := P|\Box$.

In \cite{MeredithR05} an interpretation of the new operator is
given. It turns out that there are several possible interpretations
all enjoying the requisite algebraic properties of the operator (see
\cite{milner91polyadicpi}). We will therefore make liberal use of
$(\nu\; \vec{x})P$.

% subsection the_syntax_and_semantics_of_the_notation_system (end)   

\input{qm2pi.qmops} 

\input{qm2pi.sterngerlach} 

\input{qm2pi.metric} 

% section concurrent_process_calculi (end)

%\input{qm2pi.proofsketch}

% section proof sketch (end)

%\input{qm2pi.slviaknots} 

% section spatial logic via knots (end)

\input{qm2pi.conclusion}

% section conclusion (end)

%\input{qm2pi.dtcodes} 

% section wiring algorithm (end)

\input{qm2pi.ack} 

% section acknowledgments (end)

\newpage


\bibliographystyle{plain}   
\bibliography{../../biblios/main.bib}

\input{qm2pi.rhodetails}

\end{document}

 

% subsection basic_interpretation (end)

%\input{qm2pi.rho.presentation} 
\subsection{The syntax and semantics of the notation system}\label{sub:the_syntax_and_semantics_of_the_notation_system} % (fold)

We now summarize a technical presentation of the calculus that
embodies our theory of dynamics. The typical presentation of such a
calculus follows the style of giving generators and relations on
them. The grammar, below, describing term constructors, freely
generates the set of processes, $\Proc$. This set is then quotiented
by a relation known as structural congruence and it is over this set
that the notion of dynamics is expressed. This presentation is
essentially that of \cite{MeredithR05} with the addition of
polyadicity and summation. For readability we have relegated some of
the technical subtleties to an appendix.

\subsubsection{Process grammar}\label{subsub:process_grammar}

\begin{mathpar}
  \inferrule* [lab=synchronization] {} {{M} \bc \pzero \;|\; x?F \;|\; x!C }
  \and
  \inferrule* [lab=abstraction] {} {{F} \bc (x)P}
  \and
  \inferrule* [lab=concretion] {} {{C} \bc \langle Q \rangle}
  \and
  \inferrule* [lab=process] {} {{P,Q} \bc M \;| \;P|Q \;|\; @{x}}
  \and
  \inferrule* [lab=name] {} {{x} \bc \quotep{P}}
\end{mathpar} 

Note that $\vec{x}$ (resp. $\vec{P}$) denotes a vector of names
(resp. processes) of length $|\vec{x}|$ (resp. $|\vec{P}|$). We adopt
the following useful abbreviations.

\begin{mathpar}
   x?(\vec{y}).P := x.(\vec{y})P \and  x\clift{\vec{P}} := x.\clift{\vec{P}}
   \and x!(y) := \lift{x}{\dropn{y}}
   \and \Pi_{i=0}^{n-1}P_i := P_0 | \ldots | P_{n-1}
\end{mathpar}

\subsubsection{Structural congruence}

\paragraph{Free and bound names and alpha-equivalence.} At the
core of structural equivalence is alpha-equivalence which identifies
process that are the same up to a change of variable. Formally, we
recognize the distinction between free and bound names. The free names
of a process, $\freenames{P}$, may be calculated recursively as
follows:

\begin{mathpar}
\freenames{\pzero} := \emptyset
  \and \\
  \freenames{x?(y).P} := \{ x \} \cup (\freenames{P} \setminus \{ y \})
  \and 
  \freenames{x!\langle P \rangle} := \{ x \} \cup \{ P \} 
  \and \\
  \freenames{P|Q} := \freenames{P} \cup \freenames{Q}
  \and \\
  \freenames{@{x}} := \{ x \}
\end{mathpar}

$\pi$
$\quotep{\pi}$

$\freenames{-} : \pi \to \mathcal{P}(\quotep{\pi})$

\begin{eqnarray*}
  \freenames{\pzero} & := & \emptyset \\
  \freenames{x?(y).P} & := & \{ x \} \cup (\freenames{P} \setminus \{ y \}) \\
  \freenames{x!\langle P \rangle} & := & \{ x \} \cup \{ P \} \\
  \freenames{P|Q} & := & \freenames{P} \cup \freenames{Q} \\
  \freenames{\dropn{x}} & := & \{ x \}
\end{eqnarray*}

The bound names of a process, $\boundnames{P}$, are those names occurring in $P$
that are not free. For example, in $x?(y).0$, the name $x$ is free, while $y$ is bound.

\begin{mathpar}
  \inferrule* [lab=monoidal-laws] {} { P|Q \equiv Q|P \and P|0 \equiv P \and P|(Q|R) \equiv (P|Q)|R }
\end{mathpar}

\begin{mathpar}
  \inferrule* [lab=alpha-equivalence] {} { (x)P \equiv (y)P\{y/x\} \and y \not\in \freenames{P} }
\end{mathpar}

\begin{definition}
Then two processes, $P,Q$, are alpha-equivalent if $P = Q\{\vec{y}/\vec{x}\}$ for
some $\vec{x} \in \boundnames{Q},\vec{y} \in \boundnames{P}$, where $Q\{\vec{y}/\vec{x}\}$
denotes the capture-avoiding substitution of $\vec{y}$ for $\vec{x}$ in $Q$.
\end{definition}

\begin{definition}
  The {\em structural congruence} \cite{SangiorgiWalker} , $\equiv$,
  between processes is the least congruence containing
  alpha-equivalence, satisfying the abelian monoid laws
  (associativity, commutativity and $\pzero$ as identity) for parallel
  composition $|$ and for summation $+$.
\end{definition}

\subsection{Name equivalence}

We take name equivalence, written $\nameeq$, to be the smallest
equivalence relation generated by the following rules.

\begin{mathpar}
\inferrule*[lab=Quote-drop]
{ }
{ \quotep{@{x}} \nameeq x }

\inferrule*[lab=Struct-equiv]
{ P \scong Q }
{ \quotep{P} \nameeq \quotep{Q} }
\end{mathpar}

The astute reader will have noticed that the mutual recursion of names
and processes imposes a mutual recursion on alpha-equivalence and
structural equivalence via name-equivalence. Fortunately, all of this
works out pleasantly and we may calculate in the natural way, free of
concern. The reader interested in the details is referred to the
appendix \ref{appendix:rho_details}.

\subsection{Substitution}

We use $\Proc$ for the set of processes, $\QProc$ for the set of
names, and $\id{\{}\vec{y} / \vec{x} \id{\}}$ to denote partial maps,
$s : \QProc \rightarrow \QProc$. A map, $s$ lifts, uniquely, to a map
on process terms, $\widehat{s} : \Proc \rightarrow \Proc$ by the
following equations.

\begin{mathpar}
  (0) \psubstp{Q}{P} := 0 \\
  (R \juxtap S) \psubstp{Q}{P}
  :=    
  (R)\psubstp{Q}{P} \juxtap (S) \psubstp{Q}{P} \\
  (x?(y).R) \psubstp{Q}{P}    
  :=    
  (x)\substp{Q}{P} (z)\concat( (R \psubstn{z}{y}) \psubstp{Q}{P} ) \\
  (\lift{x}{R}) \psubstp{Q}{P}  
  :=
  \lift{(x)\substp{Q}{P}}{ R \psubstp{Q}{P} } \\
%   (\dropn{x})  \psubstp{Q}{P}       
%   := 
%   \left\{ 
%     \begin{array}{ccc} 
%       \dropn{\quotep{Q}} & & x \nameeq \quotep{P} \\
%       \dropn{x} & & otherwise \\
%     \end{array}
%   \right. 
  (\dropn{x})  \psubstp{Q}{P}       
  := 
  \left\{ 
    \begin{array}{ccc} 
      Q & & x \nameeq \quotep{P} \\
      \dropn{x} & & otherwise \\
    \end{array}
  \right.
\end{mathpar}
 

where

\begin{eqnarray}
  (x)\id{\{} \lpquote Q \rpquote / \lpquote P \rpquote \id{\}}            = 
  \left\{ 
    \begin{array}{ccc}
      \lpquote Q \rpquote & & x \nameeq \lpquote P \rpquote \\
      x & & otherwise \\
    \end{array}
  \right. \nonumber
\end{eqnarray}

and $z$ is chosen distinct from $\quotep{P}$, $\quotep{Q}$, the free
names in $Q$, and all the names in $R$. Our $\alpha$-equivalence will
be built in the standard way from this substitution.

\begin{remark}\label{rem:no_self_referential_names}
  One consequence of these definitions is that $\forall P. \quotep{P}
  \not\in \freenames{P}$.
\end{remark}

\subsection{ Dynamic quote: an example }

Anticipating something of what's to come, consider applying the
substitution, $\widehat{\id{\{}u / z \id{\}}}$, to the following pair
of processes, $\lift{w}{y!(z)}$ and $w[ \lpquote y!(z) \rpquote ]$.

\begin{eqnarray}
	\lift{w}{y!(z)}\widehat{\id{\{}u / z \id{\}}}
		& = &
		\lift{w}{y!(u)} \nonumber\\
	w[ \lpquote y!(z) \rpquote ] \widehat{ \id{\{}u / z \id{\}} }
		& = &
		w[ \lpquote y!(z) \rpquote ] \nonumber
\end{eqnarray}

Because the body of the process between quotes is impervious to
substitution, we get radically different answers. In fact, by
examining the first process in an input context,
e.g. $x?(z).\lift{w}{y!(z)}$, we see that the process under the lift
operator may be shaped by prefixed inputs binding a name inside it. In
this sense, the lift operator will be seen as a way to dynamically
construct processes before reifying them as names.

Finally equipped with these standard features we can present the
dynamics of the calculus.

\subsubsection{Operational semantics} 

Finally, we introduce the computational dynamics. What marks these
algebras as distinct from other more traditionally studied algebraic
structures, e.g. vector spaces or polynomial rings, is the manner in
which dynamics is captured. In traditional structures, dynamics is typically
expressed through morphisms between such structures, as in linear maps
between vector spaces or morphisms between rings. In algebras
associated with the semantics of computation, the dynamics is
expressed as part of the algebraic structure itself, through a
reduction reduction relation typically denoted by $\red$. Below, we
give a recursive presentation of this relation for the calculus used
in the encoding.

$\red \subseteq \pi \times \pi$
$\red : \pi \to \mathcal{P}(\pi)$

\begin{mathpar}
  \inferrule* [lab=Comm] { \textsf{match}( x_{src}, x_{trgt} ) } { x_{trgt}?(y)P \; | \; x_{src}!\langle {Q} \rangle \red P\{\quotep{Q}/y}\} }
  \and \\
  \inferrule* [lab=Par] {{P} \red {P}'} {{{P} | {Q}} \red {{P}' | {Q}}}
  \and
  \inferrule* [lab=Equiv]{{{P} \scong {P}'} \andalso {{P}' \red {Q}'} \andalso {{Q}' \scong {Q}}}{{P} \red {Q}}
\end{mathpar}

\begin{eqnarray*}
  match_{\equiv} (\quotep{P},\quotep{Q}) & := & P \equiv Q \\
  match_{\dagger}(\quotep{P},\quotep{Q}) & := & \forall R. P|Q \red^{*} R => R \red^{*} 0 \\
  match_{K}(\quotep{P},\quotep{Q}) & := & K \mbox{ for some context } K
\end{eqnarray*}

$u?(x)P | u!\langle Q \rangle \red P\{\quotep{Q}/x\}$

%We write $\wred$ for $\red^*$, and $P\red$ if $\exists Q $ such that $ P \red Q$.
We write $P\red$ if $\exists Q $ such that $ P \red Q$ and $P\not\red$, otherwise.

\section{Replication}

As mentioned before, it is known that replication (and hence
recursion) can be implemented in a higher-order process algebra
\cite{SangiorgiWalker}. As our first example of calculation with the
machinery thus far presented we give the construction explicitly in
the {\rhoc}.

\begin{eqnarray}
	D_{x} & := & \prefix{x}{y}{(\binpar{\outputp{x}{y}}{@{y}})} \nonumber\\
	\bangp_{x}{P} & := & \binpar{{x}!\langle{\binpar{D_{x}}{P}}\rangle}{D_{x}} \nonumber
\end{eqnarray}

\begin{eqnarray}
	\bangp_{x}{P} & & \nonumber\\
	=
	& {x}!\langle{(\prefix{x}{y}{(\outputp{x}{y} | @{y})) | P}}\rangle 
	      | \prefix{x}{y}{(\outputp{x}{y} | @{y})} & \nonumber\\
	\red
	& (\outputp{x}{y} | @{y})\substn{\quotep{(\prefix{x}{y}{(@{y} | \outputp{x}{y})) | P}}}{y} & \nonumber\\
	=
	& \outputp{x}{\quotep{(\prefix{x}{y}{(\outputp{x}{y} | @{y})) | P}}}
	  | {(\prefix{x}{y}{(\outputp{x}{y} | @{y})) | P}} & \nonumber\\
	\red
	& \ldots & \nonumber\\
	\red^*
	& P | P | \ldots & \nonumber
\end{eqnarray}

Of course, this encoding, as an implementation, runs away, unfolding
$\bangp{P}$ eagerly. A lazier and more implementable replication
operator, restricted to input-guarded processes, may be obtained as follows.

\begin{eqnarray}
\bangp{\prefix{u}{v}{P}} 
	:= 
	\binpar{\lift{x}{\prefix{u}{v}{(\binpar{D(x)}{P})}}}{D(x)} \nonumber
\end{eqnarray}

\begin{remark}
  Note that the lazier definition still does not deal with summation
  or mixed summation (i.e. sums over input and output). The reader is
  invited to construct definitions of replication that deal with these
  features. 

  Further, the definitions are parameterized in a name, $x$. Can you,
  gentle reader, make a definition that eliminates this parameter and
  guarantees no accidental interaction between the replication
  machinery and the process being replicated -- i.e. no accidental
  sharing of names used by the process to get its work done and the
  name(s) used by the replication to effect copying. This latter
  revision of the definition of replication is crucial to obtaining
  the expected identity $!!P \sim !P$.
\end{remark}

\begin{remark}\label{rem:paradoxical_combinator}
  The reader familiar with the lambda calculus will have noticed the
  similarity between $D$ and the paradoxical combinator.

  [Ed. note: the existence of this seems to suggest we have to be more
  restrictive on the set of processes and names we admit if we are to
  support no-cloning.]
\end{remark}

\subsubsection{Bisimulation}

The computational dynamics gives rise to another kind of equivalence,
the equivalence of computational behavior. As previously mentioned
this is typically captured \emph{via} some form of bisimulation.

% The notion we use in this paper is weak barbed bisimulation
% \cite{milner91polyadicpi}.

The notion we use in this paper is derived from weak barbed
bisimulation \cite{milner91polyadicpi}. 

\begin{definition}
An \emph{observation relation}, $\downarrow_{\mathcal N}$, over a set
of names, $\mathcal N$, is the smallest relation satisfying the rules
below.

\infrule[Out-barb]{y \in {\mathcal N}, \; x \nameeq y}
		  {\outputp{x}{v} \downarrow_{\mathcal N} x}
\infrule[Par-barb]{\mbox{$P\downarrow_{\mathcal N} x$ or $Q\downarrow_{\mathcal N} x$}}
		  {\binpar{P}{Q} \downarrow_{\mathcal N} x}

We write $P \Downarrow_{\mathcal N} x$ if there is $Q$ such that 
$P \wred Q$ and $Q \downarrow_{\mathcal N} x$.
\end{definition}

\begin{definition}
%\label{def.bbisim}
An  ${\mathcal N}$-\emph{barbed bisimulation} over a set of names, ${\mathcal N}$, is a symmetric binary relation 
${\mathcal S}_{\mathcal N}$ between agents such that $P\rel{S}_{\mathcal N}Q$ implies:
\begin{enumerate}
\item If $P \red P'$ then $Q \wred Q'$ and $P'\rel{S}_{\mathcal N} Q'$.
\item If $P\downarrow_{\mathcal N} x$, then $Q\Downarrow_{\mathcal N} x$.
\end{enumerate}
$P$ is ${\mathcal N}$-barbed bisimilar to $Q$, written
$P \wbbisim_{\mathcal N} Q$, if $P \rel{S}_{\mathcal N} Q$ for some ${\mathcal N}$-barbed bisimulation ${\mathcal S}_{\mathcal N}$.
\end{definition}

$\mathcal{R} \subseteq \pi \times \pi$

$P \mathcal{R} Q => \forall P'. P \red P' \Rightarrow \exists Q'. Q \red Q', P' \mathcal{R} Q'$

$P \vdash x \Rightarrow Q \vdash x$

\begin{mathpar}
  \inferrule*[lab=Out-barb]{x \nameeq y}{{y}!\langle{Q}\rangle \vdash x}
  \and
  \inferrule*[lab=Par-barb]{\mbox{$P\vdash x$ or $Q\vdash x$}}{\binpar{P}{Q} \vdash x}
\end{mathpar}

\subsubsection{Contexts}

One of the principle advantages of computational calculi like the
$\pi$-calculus is a well-defined notion of context,
contextual-equivalence and a correlation between
contextual-equivalence and notions of bisimulation. The notion of
context allows the decomposition of a process into (sub-)process and
its syntactic environment, its context. Thus, a context may be
thought of as a process with a ``hole'' (written $\Box$) in it. The
application of a context $M$ to a process $P$, written $M[P]$, is
tantamount to filling the hole in $M$ with $P$. In this paper we do
not need the full weight of this theory, but do make use of the notion
of context in the proof the main theorem. 

\begin{mathpar}
  \inferrule* [lab=summation] {} {{M_{M},M_{N}} \bc \Box \;|\; x.M_{A} \;|\; M_{M}+M_{N}}
  \and
  \inferrule* [lab=agent] {} {{M_{A}} \bc (\vec{x})M_{P} \;| \; \clift{P_0,\ldots,M_{P},\ldots,P_N}}
  \and \\
  \inferrule* [lab=process] {} {{M_{P}} \bc M_{N} \;| \;P|M_{P} }
\end{mathpar} 

\begin{mathpar}
  \inferrule* [lab=sychronization] {} {M_{N} \bc \Box \;|\; x?M_{F} \;|\; x!M_{C}}
  \and
  \inferrule* [lab=abstraction] {} {{M_{F}} \bc (x)M_{P} }
  \and
  \inferrule* [lab=concretion] {} {{M_{C}} \bc \langle M_{P} \rangle }
  \and \\
  \inferrule* [lab=process] {} {{M_{P}} \bc M_{N} \;| \;P|M_{P} }
\end{mathpar}

\begin{definition}[contextual application] Given a context $M$, and
  process $P$, we define the \emph{contextual application}, $M[P] :=
  M\{P/\Box\}$. That is, the contextual application of M to P is the
  substitution of $P$ for $\Box$ in $M$.
\end{definition}

$\meaningof{-} : L \to \mathcal{P}(\pi)$

\begin{mathpar}
  \inferrule* [lab=collection] {} {\meaningof{true} = \pi, \and \meaningof{~E} = \pi \setminus \meaningof{E}, \and \meaningof{E_{1} \& E_{2}} = \meaningof{E_{1}} \cap \meaningof{E_{2}}}
\end{mathpar}

\begin{mathpar}
  \inferrule* [lab=structure] {} {\meaningof{0} = \{ P \in \pi | P \equiv 0 \}, \and \\ \meaningof{E_1 | E_2} = \{ P \in \pi | P \equiv P_{1} | P_{2}, P_{1} \in \meaningof{E_{1}}, P_{2} \in \meaningof{E_2}\} }
\end{mathpar}

\begin{mathpar}
 \inferrule* [lab=behavior] {} {\meaningof{\langle a?b \rangle E} = \{ P \in \pi | P \equiv Q | u?(y)P', \\ \and \\\\ \and \\ \;\;\; u \in \meaningof{a}, \forall z.P'\{z/y\} \in \meaningof{E\{z/b\}}\}, \and \\ \meaningof{a!E} = \{ P \in \pi | P \equiv Q | x!\langle P' \rangle, x \in \meaningof{a} P' \in \meaningof{E}\} }
\end{mathpar}

\begin{mathpar}
 \inferrule* [lab=nominal] {} {\meaningof{\quotep{E}} = \{ \quotep{P} \in \quotep{\pi} | P \in \meaningof{E} \}, \and \meaningof{\quotep{P}} = \{ \quotep{Q} \in \quotep{\pi} | P \equiv Q \} \and \\ \meaningof{@\quotep{E}} = \{ P \in \pi | P \equiv @x, x \in \meaningof{E} \}}
\end{mathpar}

\begin{eqnarray*}
  \\
  \meaningof{-} : TS \to ST
\end{eqnarray*}

\begin{eqnarray*}
  \\
  L : TS \to ST
\end{eqnarray*}

\begin{eqnarray*}
  \\
  P \models E \iff P \in \meaningof{E}
\end{eqnarray*}

\begin{eqnarray*}
  P \approx_{L} Q \iff \forall E \in L. P \models E \iff Q \models E
\end{eqnarray*}

\begin{eqnarray*}
  P \approx_{K} Q
\end{eqnarray*}

\begin{eqnarray*}
  P \approx Q
\end{eqnarray*}

$\approx_{K} = \approx = \approx_{L}$

\subsubsection{Contextual duality}

Note that contexts extend the quotation operation to a family of
operations from processes to names. Given a context, $M$, we can
define a \emph{nominal context}, $\quotep{M}$ by $\quotep{M}[P] :=
\quotep{M[P]}$. To foreshadow what is to come we observe that these
operations enjoy a duality with processes very much like the duality
between vectors and maps from vectors to scalars.

Further, because the calculus is essentially higher-order, we have a
correspondence between contexts and processes. More specifically,
given a name $x$ and a context $M$ we can construct $M^{*}_{x}$ such
that 

\begin{mathpar}
  M^{*}_{x} | \lift{x}{P} \red M[P]
\end{mathpar}

namely,

\begin{mathpar}
  M^{*}_{x} := x?(u).M[\dropn{u}]
\end{mathpar}

The dependence of $M^{*}_{x}$ on a name makes it an abstraction, 

\begin{mathpar}
  M^{*} := (x)x?(u).M[\dropn{u}]
\end{mathpar}

\subsection{Additional notation}

It will sometimes be convenient to denote the process a name
quotes. We already have the notation $x = \quotep{P}$, but it will be
convenient to introduce an alternate notation, $\procn{x}$, when we
want to emphasize the connection to the use of the name. Note that, by
virtue of name equivalence, $\quotep{\procn{x}} \nameeq x$; so, the
notation is consistent with previous definitions.

Further, because names have structure it is possible to effect
substitutions on the basis of that structure. This means we need to
upgrade our notation for substitutions, which we accomplish by
adapting comprehension notation. Thus,

\begin{mathpar}
  P\{ y / x : x \in S \}
\end{mathpar}

is interpreted to mean the process derived from P by replacing (in a
capture-avoiding manner) each occurrence of $x$ in $S$ by $y$. For example,

\begin{mathpar}
  P\{ \quotep{\procn{x}|\procn{x}} / x : x \in \freenames{P} \}
\end{mathpar}

will replace each (occurrence) of a free name $x$ in $P$ by
$\quotep{\procn{x}|\procn{x}}$.

Also, we will avail ourselves of the notation $x^{L}$ and $x^{R}$ to
denote injections of a name into disjoint copies of the name
space. There are numerous ways to accomplish this. One example can be
found in \cite{MeredithR05}. This notation overloads to vectors of
names: $\vec{x}^{\pi} := (x_{i}^{\pi} \; : \; 0 \leq i < |\vec{x}| )$ where $\pi \in \{L,R\}$.

We also use $P^{\Box} := P|\Box$.

In \cite{MeredithR05} an interpretation of the new operator is
given. It turns out that there are several possible interpretations
all enjoying the requisite algebraic properties of the operator (see
\cite{milner91polyadicpi}). We will therefore make liberal use of
$(\nu\; \vec{x})P$.

% subsection the_syntax_and_semantics_of_the_notation_system (end)   

\section{Interpretation of QM}
\subsection{Supporting definitions}
\subsubsection{Multiplication}
\begin{mathpar}
  \quotep{Q} \cdot \quotep{R} := \quotep{Q|R}
  \and \\
  \quotep{Q} \cdot P := P\{ \quotep{Q|R} / \quotep{R} : \quotep{R} \in \freenames{P} \}
\end{mathpar}

\paragraph{Discussion}
The first line needs little explanation. The second line says that
each free name of the process is replaced with the multiplication of
that name by the scalar. Multiplication of a scalar (name) by a state
(process) results in a process all the names of which have been `moved
over' by parallel composition with the process the scalar
quotes. There is a subtlety that the bound names have to be
manipulated so that multiplied names aren't accidentally
captured. There are many ways to achieve this.

\begin{remark}\label{rem:multiplication_identities}
  The reader is invited to verify that for all $x,y,z \in \QProc$ and $P \in \Proc$
  \begin{mathpar}
    x \cdot \quotep{0} \equiv x 
    \and
    x \cdot y \equiv y \cdot x
    \and
    x \cdot (y \cdot z) \equiv (x \cdot y) \cdot z
    \and \\
    \quotep{0} \cdot P \equiv P
    \and \\
    x \cdot (y \cdot P) \equiv (x \cdot y) \cdot P
    \and \\
    x \cdot (P|Q) \equiv (x \cdot P) | (x \cdot Q)
    \and \\    
  \end{mathpar}
\end{remark}

\subsubsection{Tensor product}

We define a tensor product on processes by structural induction.

\paragraph{Tensor of sums} First note that all summations, including
$\pzero$ and sequence, can be written $\Sigma_{i} x_{i}.A_{i} +
\Sigma_{j} x_{j}.C_{j}$, where we have grouped input-guarded processes
together and output-guarded processes together.

Thus, we can define the tensor product of two summations, $N_{1}\otimes N_{2}$, where

\begin{mathpar}
  N_{1} := \Sigma_{i} x_{i}.A_{i} + \Sigma_{j} x_{j}.C_{j}
  \and
  N_{2} := \Sigma_{i'} y_{i'}.B_{i'} + \Sigma_{j'} y_{j'}.D_{j'} 
\end{mathpar}

as follows.

\begin{mathpar}
  \Sigma_{i} x_{i}.A_{i} + \Sigma_{j} x_{j}.C_{j} \otimes \Sigma_{i'}
  y_{i'}.B_{i'} + \Sigma_{j'} y_{j'}.D_{j'} 
  \and \\
  := \; \Sigma_{i} \Sigma_{i'} \quotep{\stackrel{\vee}{x_{i}}| \stackrel{\vee}{y_{i'}}}.(A_{i}\otimes B_{i'}) \; | \; \Sigma_{i'} \Sigma_{i} \quotep{\stackrel{\vee}{y_{i'}}|\stackrel{\vee}{x_{i}}}.(B_{i'}\otimes A_{i})
  \and
  \;\; | \;\; \Sigma_{j} \Sigma_{j'} \quotep{\stackrel{\vee}{x_{j}}|\stackrel{\vee}{y_{j'}}}.(A_{j}\otimes B_{j'}) \; | \; \Sigma_{j'} \Sigma_{j} \quotep{\stackrel{\vee}{y_{j'}}|\stackrel{\vee}{x_{j}}}.(B_{j'}\otimes A_{j})
\end{mathpar}

\begin{remark}
  Do we need to $x^{L}$ and $y^{R}$ for this construction as well?
\end{remark}

\paragraph{Tensor of parallel compositions} Next, we distribute tensor
over par.

\begin{mathpar}
  P_{1}|P_{2} \otimes Q_{1}|Q_{2} := (P_{1} \otimes Q_{1}) | (P_{1}
  \otimes Q_{2}) | (P_{2} \otimes Q_{1}) | (P_{2} \otimes Q_{2})
\end{mathpar}

\paragraph{Tensor with dropped names} We treat tensor of a
process with a dropped name as parallel composition.

\begin{mathpar}
  P \otimes \dropn{x} := P | \dropn{x}
\end{mathpar}

\paragraph{Tensor of agents}

Finally, we need to define tensor on agents. Note that the definition
of tensor on normal products only tensors inputs with inputs and
outputs with outputs. Thus, we only have to define the operation on
``homogeneous'' pairings.

\begin{mathpar}
  (\vec{x})P \otimes (\vec{y})Q
  \and \\
  := (x_{0}^{L}|y_{0}^{R},\ldots,x_{0}^{L}|y_{n}^{R},\ldots,x_{m}^{L}|y_{0}^{R},\ldots,x_{m}^{L}|y_{n}^R)(P\{ \vec{x}^{L}/\vec{x}\} \otimes Q \{ \vec{y}^{R}/\vec{y}\})
  \and \\
  \clift{\vec{P}} \otimes \clift{\vec{Q}}
  \and \\
  := \clift{P_{0}\otimes Q_{0},\ldots,P_{0}\otimes Q_{n},\ldots,P_{m}\otimes Q_{0},\ldots,P_{m}\otimes Q_{n}}
\end{mathpar}

\begin{remark}
  Observe that arities of tensored abstractions matches arities of
  tensored concretions if the original arities matched. Note also that
  the length of the arities corresponds to the increase in dimension
  we see in ordinary vector space tensor product.
\end{remark}

\begin{remark}
  Operationally, this definition distributes the tensor down to
  components ``linked'' by summation. Tensor over summation is
  intriguing in that it mixes names. Moreover, as a consequence of the
  way it mixes names we have the identities for all $x \in \QProc$ and
  $P,Q \in \Proc$

  \begin{mathpar}
    (x \cdot P) \otimes Q \equiv x \cdot (P \otimes Q) \equiv P \otimes (x \cdot Q)
    \and
    P \otimes \pzero \equiv P
  \end{mathpar}

  that the reader is invited to verify.
\end{remark}

\subsubsection{Annihilation}
\begin{mathpar}
  P^{\perp} := \{ Q | \forall R. P|Q \red^{*} R \Rightarrow R \red^{*} \pzero \}
  \and \\
  P^{\underline{\perp}} := \Sigma_{Q \in P^{\perp}} \quotep{Q}?(y).(\dropn{y}|Q) | \Sigma_{Q \in P^{\perp}} \quotep{Q}\clift{\Box}
\end{mathpar}

\paragraph{Discussion} The reader will note that $P^{\perp}$ is a
\emph{set} of processes, while $P^{\underline{\perp}}$ is a
\emph{context}. We call the set $P^{\perp}$ the \emph{annihilators} of
$P$. The parallel composition of a process in the annihilators of $P$
with $P$ will result in a process, the state space of which has all
paths eventually leading to $\pzero$. Execution may endure loops; but
under reasonable conditions of fairness (naturally guaranteed under
most notions of bisimulation) such a composite process cannot get
stuck in such a loop and will, eventually pop out and terminate.

The context $P^{\underline{\perp}}$ is ready and willing to ``take the
$P$ out of'' the process to which it is applied. It will effectively
transmit the code of the process to which it is applied to one of the
annihilators and run the process against it.

\subsubsection{Evaluation}
We fix $M$ a domain of fully abstract interpretation with an equality
coincident with bisimulation. We take $\meaningof{\cdot} : \Proc \to
M$ to be the map interpreting processes and $\nmeaningof{\cdot} : \M
\to Proc$ to be the map running the other way. Then we define

\begin{mathpar}
  \int P := \nmeaningof{\meaningof{P}}
\end{mathpar}

\paragraph{Discussion}
There are many fully abstract interpretations of Milner's
$\pi$-calculus. Any of them can be used as a basis for interpreting
the reflective calculus here. Equipped with such a domain it is
largely a matter of grinding through to check that the Yoneda
construction for the normalization-by-evaluation program can be
extended to this setting.

\begin{remark}
  The reader is invited to verify that $\int (P^{\underline{\perp}}[P]) = 0$.
\end{remark}

\subsection{Quantum mechanics}

Table \ref{tbl:core_qm_op_defns} gives the core operational definitions

\begin{table}[htp]\label{tbl:core_qm_op_defns}
  \center{
    \fbox{
      \begin{tabular}{c|c}
        quantum mechanics & process calculus \\
        \hline
        scalar & $x := \quotep{P}$ \\
        state vector & $\state{P} := P$ \\
        dual & $\state{P}^{*} := \event{P^{\underline{\perp}}} := \quotep{P^{\underline{\perp}}}[-]$ \\
        matrix & $ \Sigma_{\alpha} \state{P_{\alpha}}x_{\alpha}\event{Q_{\alpha}}$ \\
        vector addition & $\state{P} + \state{Q} := \state{P | Q}$ \\
        tensor product & $\state{P} \otimes \state{Q} := \state{P \otimes Q}$ \\
        inner product & $\innerprod{P}{Q} := \quotep{\int P^{\underline{\perp}}[Q]}$ \\
      \end{tabular}
    }
  }
  \caption{QM - operational definitions}
\end{table}

where

\begin{mathpar}
  \prmatrix{P}{Q} := \fprmatrix{P}{\quotep{\pzero}}{Q}
  \and
  \fprmatrix{P}{x}{Q} := (\state{P},x,\event{Q})
  \and
  (\fprmatrix{P}{x}{Q})(\state{R}) := x \cdot \innerprod{Q}{R} \cdot \state{P}
  \and
  (\fprmatrix{P}{x}{Q})(\event{R}) := x \cdot \innerprod{R}{P} \cdot \event{Q}
\end{mathpar}

\paragraph{Discussion}
As promised: vectors (aka states) are represented as processes; duals
as contextual duals; inner product definition should be compared with
standard inner product definition for ....

\begin{remark}
  Assuming $\int (P^{\underline{\perp}}[P]) = 0$, the reader is
  invited to verify that $(\fprmatrix{P}{x}{P})(\state{P}) = x \cdot \state{P}$.
\end{remark}

\begin{remark}
  The reader is invited to verify that $\innerprod{P}{Q}$ could
  equally well have been written $\quotep{\int \stackrel{\vee}{x}}$
  where $x = \event{P^{\underline{\perp}}}(Q)$.

  One of the motivations for this remark is that there is another way
  to factor these operations. We could package up evaluation in the dual:

  \begin{mathpar}
    \state{P}^{*} := \event{\int P^{\underline{\perp}}} := \quotep{\int P^{\underline{\perp}}}[-]
  \end{mathpar}

  and then have inner product defined by
  
  \begin{mathpar}
    \innerprod{P}{Q} := \event{P}(Q)
  \end{mathpar}

  Hopefully, experience with the calculations will provide guidance on
  the best factoring.
\end{remark}

\begin{remark}
  Assuming $\int (P^{\underline{\perp}}[P]) = 0$, the reader is
  invited to verify that $\forall P,Q. (\prmatrix{0}{Q})(\state{0}) =
  \state{0}$ and dually $(\prmatrix{P}{0})(\event{0}) = \event{0}$.
\end{remark}

\begin{remark}
  i'm a little worried that i don't (yet) have proper support for
  complex conjugacy. But, the observation above may give us a
  clue. According to Abramsky, it must be the case that the scalars
  are iso to the homset of the identity for the tensor -- which the
  observation above characterizes. 

  For now, we will simply bookmark the notion with $\overline{x}$.
\end{remark}

\subsubsection{Adjointness}

We need to give a definition of $(\cdot)^{\dagger}$ for matrices. The
obvious candidate definition is
\begin{mathpar}
(\Sigma_{\alpha}\fprmatrix{P_{\alpha}}{x_{\alpha}}{Q_{\alpha}})^{\dagger}
= \Sigma_{\alpha}\fprmatrix{(Q_{\alpha}^{\underline{\perp}})^{*}}{\overline{x}_{\alpha}}{P_{\alpha}^{\underline{\perp}}} 
\end{mathpar}

But, $(Q_{\alpha}^{\underline{\perp}})^{*}$ requires a name along
which to communicate the process to achieve the context application.

\subsubsection{Basis for a basis}
If processes label states and ``addition'' of states (a.k.a. vector
addition) is interpreted as parallel composition, what corresponds to
notions of linear independence and basis? Here, we recall that Yoshida
has developed a set of \emph{combinators} for an asynchronous verison
of Milner's $\pi$-calculus. These are a finite set of processes such
any process can be expressed as parallel composition of these
combinators together with liberal uses of the new operator and
replication. We can simply give a translation of these into the
present calculus and have reasonable expectation that the property
carries over. That is, that the resultant set allows to express all
processes via parallel composition. Note, however, that there is no
new operator or replication in this calculus. As a result, we expect
that the corresponding set is actually infinite. That is, we expect
that the space is actually infinite dimensional.

\begin{remark}
  The attentive reader may be a bit concerned. Certainly, the
  collection $S$, $K$ and $I$ is a finite set of
  combinators. Shouldn't we expect to see a finite set of combinators
  for an effectively equivalent system? i am very sympathetic to this
  critique and feel it warrants full attention. On the other hand, i
  also have in mind the following analogy. The natural numbers, as a
  monoid under addition, has exactly $1$ generator, while the natural
  numbers, as a monoid under multiplication, has countably many
  generators (the primes). We observe that the application of the
  lambda calculus is much less resource sensitive than the parallel
  composition of the $\pi$-calculus. Could it be the case that we have
  an analogy of the form
  
  \begin{mathpar}
    m + n : MN :: m*n : M|N
  \end{mathpar}

  giving a similar blow up in the set of ``primes''?  This is such a
  wonderful thought that, even if it's not true, i think it's worth
  writing down.
\end{remark}
 

\documentclass[12pt]{llncs}
%\documentclass{jktr}

\usepackage[pdftex]{hyperref}                   
\usepackage {listings}
\usepackage {mathpartir}
\usepackage{bcprules}
%\usepackage{listings}
                       
\usepackage{graphicx} 
%\usepackage[margins=2.5cm,nohead,nofoot]{geometry}
%\usepackage{geometry}
\usepackage{amsfonts}
\usepackage{amstext}
\usepackage{latexsym}
\usepackage{amssymb}
\usepackage{color}


%\include{myPreamble}
\include{qm2pi.local} 

%\ifpdf
%\usepackage[pdftex]{graphicx}
%\else
%\usepackage{graphicx}
%\fi

 % \ifpdf
%  \usepackage{pdfsync}
%  \if


%\title{Brief Article}
%\author{David F. Snyder}
%\author{L.G. Meredith}

%\address{Dept. of Math., Texas State University--San Marcos, San Marcos, TX 78666}
       
\pagestyle{empty}


\begin{document}

\lstset{language=[Objective]Caml,frame=shadowbox}

\input{qm2pi.front}

% section front matter (end)

\input{qm2pi.intro} 
 
% section introduction (end)

% \input{qm2pi.knotations} 

% section notation (end)

\input{qm2pi.process.calculi} 

% section concurrent_process_calculi_and_spatial_logics_ (end)
    
%\input{qm2pi.knots2pi} 

%\input{qm2pi.trefoil} 

%\input{qm2pi.mainthm} 

% subsection basic_interpretation (end)

%\input{qm2pi.rho.presentation} 
\subsection{The syntax and semantics of the notation system}\label{sub:the_syntax_and_semantics_of_the_notation_system} % (fold)

We now summarize a technical presentation of the calculus that
embodies our theory of dynamics. The typical presentation of such a
calculus follows the style of giving generators and relations on
them. The grammar, below, describing term constructors, freely
generates the set of processes, $\Proc$. This set is then quotiented
by a relation known as structural congruence and it is over this set
that the notion of dynamics is expressed. This presentation is
essentially that of \cite{MeredithR05} with the addition of
polyadicity and summation. For readability we have relegated some of
the technical subtleties to an appendix.

\subsubsection{Process grammar}\label{subsub:process_grammar}

\begin{mathpar}
  \inferrule* [lab=synchronization] {} {{M} \bc \pzero \;|\; x?F \;|\; x!C }
  \and
  \inferrule* [lab=abstraction] {} {{F} \bc (x)P}
  \and
  \inferrule* [lab=concretion] {} {{C} \bc \langle Q \rangle}
  \and
  \inferrule* [lab=process] {} {{P,Q} \bc M \;| \;P|Q \;|\; @{x}}
  \and
  \inferrule* [lab=name] {} {{x} \bc \quotep{P}}
\end{mathpar} 

Note that $\vec{x}$ (resp. $\vec{P}$) denotes a vector of names
(resp. processes) of length $|\vec{x}|$ (resp. $|\vec{P}|$). We adopt
the following useful abbreviations.

\begin{mathpar}
   x?(\vec{y}).P := x.(\vec{y})P \and  x\clift{\vec{P}} := x.\clift{\vec{P}}
   \and x!(y) := \lift{x}{\dropn{y}}
   \and \Pi_{i=0}^{n-1}P_i := P_0 | \ldots | P_{n-1}
\end{mathpar}

\subsubsection{Structural congruence}

\paragraph{Free and bound names and alpha-equivalence.} At the
core of structural equivalence is alpha-equivalence which identifies
process that are the same up to a change of variable. Formally, we
recognize the distinction between free and bound names. The free names
of a process, $\freenames{P}$, may be calculated recursively as
follows:

\begin{mathpar}
\freenames{\pzero} := \emptyset
  \and \\
  \freenames{x?(y).P} := \{ x \} \cup (\freenames{P} \setminus \{ y \})
  \and 
  \freenames{x!\langle P \rangle} := \{ x \} \cup \{ P \} 
  \and \\
  \freenames{P|Q} := \freenames{P} \cup \freenames{Q}
  \and \\
  \freenames{@{x}} := \{ x \}
\end{mathpar}

$\pi$
$\quotep{\pi}$

$\freenames{-} : \pi \to \mathcal{P}(\quotep{\pi})$

\begin{eqnarray*}
  \freenames{\pzero} & := & \emptyset \\
  \freenames{x?(y).P} & := & \{ x \} \cup (\freenames{P} \setminus \{ y \}) \\
  \freenames{x!\langle P \rangle} & := & \{ x \} \cup \{ P \} \\
  \freenames{P|Q} & := & \freenames{P} \cup \freenames{Q} \\
  \freenames{\dropn{x}} & := & \{ x \}
\end{eqnarray*}

The bound names of a process, $\boundnames{P}$, are those names occurring in $P$
that are not free. For example, in $x?(y).0$, the name $x$ is free, while $y$ is bound.

\begin{mathpar}
  \inferrule* [lab=monoidal-laws] {} { P|Q \equiv Q|P \and P|0 \equiv P \and P|(Q|R) \equiv (P|Q)|R }
\end{mathpar}

\begin{mathpar}
  \inferrule* [lab=alpha-equivalence] {} { (x)P \equiv (y)P\{y/x\} \and y \not\in \freenames{P} }
\end{mathpar}

\begin{definition}
Then two processes, $P,Q$, are alpha-equivalent if $P = Q\{\vec{y}/\vec{x}\}$ for
some $\vec{x} \in \boundnames{Q},\vec{y} \in \boundnames{P}$, where $Q\{\vec{y}/\vec{x}\}$
denotes the capture-avoiding substitution of $\vec{y}$ for $\vec{x}$ in $Q$.
\end{definition}

\begin{definition}
  The {\em structural congruence} \cite{SangiorgiWalker} , $\equiv$,
  between processes is the least congruence containing
  alpha-equivalence, satisfying the abelian monoid laws
  (associativity, commutativity and $\pzero$ as identity) for parallel
  composition $|$ and for summation $+$.
\end{definition}

\subsection{Name equivalence}

We take name equivalence, written $\nameeq$, to be the smallest
equivalence relation generated by the following rules.

\begin{mathpar}
\inferrule*[lab=Quote-drop]
{ }
{ \quotep{@{x}} \nameeq x }

\inferrule*[lab=Struct-equiv]
{ P \scong Q }
{ \quotep{P} \nameeq \quotep{Q} }
\end{mathpar}

The astute reader will have noticed that the mutual recursion of names
and processes imposes a mutual recursion on alpha-equivalence and
structural equivalence via name-equivalence. Fortunately, all of this
works out pleasantly and we may calculate in the natural way, free of
concern. The reader interested in the details is referred to the
appendix \ref{appendix:rho_details}.

\subsection{Substitution}

We use $\Proc$ for the set of processes, $\QProc$ for the set of
names, and $\id{\{}\vec{y} / \vec{x} \id{\}}$ to denote partial maps,
$s : \QProc \rightarrow \QProc$. A map, $s$ lifts, uniquely, to a map
on process terms, $\widehat{s} : \Proc \rightarrow \Proc$ by the
following equations.

\begin{mathpar}
  (0) \psubstp{Q}{P} := 0 \\
  (R \juxtap S) \psubstp{Q}{P}
  :=    
  (R)\psubstp{Q}{P} \juxtap (S) \psubstp{Q}{P} \\
  (x?(y).R) \psubstp{Q}{P}    
  :=    
  (x)\substp{Q}{P} (z)\concat( (R \psubstn{z}{y}) \psubstp{Q}{P} ) \\
  (\lift{x}{R}) \psubstp{Q}{P}  
  :=
  \lift{(x)\substp{Q}{P}}{ R \psubstp{Q}{P} } \\
%   (\dropn{x})  \psubstp{Q}{P}       
%   := 
%   \left\{ 
%     \begin{array}{ccc} 
%       \dropn{\quotep{Q}} & & x \nameeq \quotep{P} \\
%       \dropn{x} & & otherwise \\
%     \end{array}
%   \right. 
  (\dropn{x})  \psubstp{Q}{P}       
  := 
  \left\{ 
    \begin{array}{ccc} 
      Q & & x \nameeq \quotep{P} \\
      \dropn{x} & & otherwise \\
    \end{array}
  \right.
\end{mathpar}
 

where

\begin{eqnarray}
  (x)\id{\{} \lpquote Q \rpquote / \lpquote P \rpquote \id{\}}            = 
  \left\{ 
    \begin{array}{ccc}
      \lpquote Q \rpquote & & x \nameeq \lpquote P \rpquote \\
      x & & otherwise \\
    \end{array}
  \right. \nonumber
\end{eqnarray}

and $z$ is chosen distinct from $\quotep{P}$, $\quotep{Q}$, the free
names in $Q$, and all the names in $R$. Our $\alpha$-equivalence will
be built in the standard way from this substitution.

\begin{remark}\label{rem:no_self_referential_names}
  One consequence of these definitions is that $\forall P. \quotep{P}
  \not\in \freenames{P}$.
\end{remark}

\subsection{ Dynamic quote: an example }

Anticipating something of what's to come, consider applying the
substitution, $\widehat{\id{\{}u / z \id{\}}}$, to the following pair
of processes, $\lift{w}{y!(z)}$ and $w[ \lpquote y!(z) \rpquote ]$.

\begin{eqnarray}
	\lift{w}{y!(z)}\widehat{\id{\{}u / z \id{\}}}
		& = &
		\lift{w}{y!(u)} \nonumber\\
	w[ \lpquote y!(z) \rpquote ] \widehat{ \id{\{}u / z \id{\}} }
		& = &
		w[ \lpquote y!(z) \rpquote ] \nonumber
\end{eqnarray}

Because the body of the process between quotes is impervious to
substitution, we get radically different answers. In fact, by
examining the first process in an input context,
e.g. $x?(z).\lift{w}{y!(z)}$, we see that the process under the lift
operator may be shaped by prefixed inputs binding a name inside it. In
this sense, the lift operator will be seen as a way to dynamically
construct processes before reifying them as names.

Finally equipped with these standard features we can present the
dynamics of the calculus.

\subsubsection{Operational semantics} 

Finally, we introduce the computational dynamics. What marks these
algebras as distinct from other more traditionally studied algebraic
structures, e.g. vector spaces or polynomial rings, is the manner in
which dynamics is captured. In traditional structures, dynamics is typically
expressed through morphisms between such structures, as in linear maps
between vector spaces or morphisms between rings. In algebras
associated with the semantics of computation, the dynamics is
expressed as part of the algebraic structure itself, through a
reduction reduction relation typically denoted by $\red$. Below, we
give a recursive presentation of this relation for the calculus used
in the encoding.

$\red \subseteq \pi \times \pi$
$\red : \pi \to \mathcal{P}(\pi)$

\begin{mathpar}
  \inferrule* [lab=Comm] { \textsf{match}( x_{src}, x_{trgt} ) } { x_{trgt}?(y)P \; | \; x_{src}!\langle {Q} \rangle \red P\{\quotep{Q}/y}\} }
  \and \\
  \inferrule* [lab=Par] {{P} \red {P}'} {{{P} | {Q}} \red {{P}' | {Q}}}
  \and
  \inferrule* [lab=Equiv]{{{P} \scong {P}'} \andalso {{P}' \red {Q}'} \andalso {{Q}' \scong {Q}}}{{P} \red {Q}}
\end{mathpar}

\begin{eqnarray*}
  match_{\equiv} (\quotep{P},\quotep{Q}) & := & P \equiv Q \\
  match_{\dagger}(\quotep{P},\quotep{Q}) & := & \forall R. P|Q \red^{*} R => R \red^{*} 0 \\
  match_{K}(\quotep{P},\quotep{Q}) & := & K \mbox{ for some context } K
\end{eqnarray*}

$u?(x)P | u!\langle Q \rangle \red P\{\quotep{Q}/x\}$

%We write $\wred$ for $\red^*$, and $P\red$ if $\exists Q $ such that $ P \red Q$.
We write $P\red$ if $\exists Q $ such that $ P \red Q$ and $P\not\red$, otherwise.

\section{Replication}

As mentioned before, it is known that replication (and hence
recursion) can be implemented in a higher-order process algebra
\cite{SangiorgiWalker}. As our first example of calculation with the
machinery thus far presented we give the construction explicitly in
the {\rhoc}.

\begin{eqnarray}
	D_{x} & := & \prefix{x}{y}{(\binpar{\outputp{x}{y}}{@{y}})} \nonumber\\
	\bangp_{x}{P} & := & \binpar{{x}!\langle{\binpar{D_{x}}{P}}\rangle}{D_{x}} \nonumber
\end{eqnarray}

\begin{eqnarray}
	\bangp_{x}{P} & & \nonumber\\
	=
	& {x}!\langle{(\prefix{x}{y}{(\outputp{x}{y} | @{y})) | P}}\rangle 
	      | \prefix{x}{y}{(\outputp{x}{y} | @{y})} & \nonumber\\
	\red
	& (\outputp{x}{y} | @{y})\substn{\quotep{(\prefix{x}{y}{(@{y} | \outputp{x}{y})) | P}}}{y} & \nonumber\\
	=
	& \outputp{x}{\quotep{(\prefix{x}{y}{(\outputp{x}{y} | @{y})) | P}}}
	  | {(\prefix{x}{y}{(\outputp{x}{y} | @{y})) | P}} & \nonumber\\
	\red
	& \ldots & \nonumber\\
	\red^*
	& P | P | \ldots & \nonumber
\end{eqnarray}

Of course, this encoding, as an implementation, runs away, unfolding
$\bangp{P}$ eagerly. A lazier and more implementable replication
operator, restricted to input-guarded processes, may be obtained as follows.

\begin{eqnarray}
\bangp{\prefix{u}{v}{P}} 
	:= 
	\binpar{\lift{x}{\prefix{u}{v}{(\binpar{D(x)}{P})}}}{D(x)} \nonumber
\end{eqnarray}

\begin{remark}
  Note that the lazier definition still does not deal with summation
  or mixed summation (i.e. sums over input and output). The reader is
  invited to construct definitions of replication that deal with these
  features. 

  Further, the definitions are parameterized in a name, $x$. Can you,
  gentle reader, make a definition that eliminates this parameter and
  guarantees no accidental interaction between the replication
  machinery and the process being replicated -- i.e. no accidental
  sharing of names used by the process to get its work done and the
  name(s) used by the replication to effect copying. This latter
  revision of the definition of replication is crucial to obtaining
  the expected identity $!!P \sim !P$.
\end{remark}

\begin{remark}\label{rem:paradoxical_combinator}
  The reader familiar with the lambda calculus will have noticed the
  similarity between $D$ and the paradoxical combinator.

  [Ed. note: the existence of this seems to suggest we have to be more
  restrictive on the set of processes and names we admit if we are to
  support no-cloning.]
\end{remark}

\subsubsection{Bisimulation}

The computational dynamics gives rise to another kind of equivalence,
the equivalence of computational behavior. As previously mentioned
this is typically captured \emph{via} some form of bisimulation.

% The notion we use in this paper is weak barbed bisimulation
% \cite{milner91polyadicpi}.

The notion we use in this paper is derived from weak barbed
bisimulation \cite{milner91polyadicpi}. 

\begin{definition}
An \emph{observation relation}, $\downarrow_{\mathcal N}$, over a set
of names, $\mathcal N$, is the smallest relation satisfying the rules
below.

\infrule[Out-barb]{y \in {\mathcal N}, \; x \nameeq y}
		  {\outputp{x}{v} \downarrow_{\mathcal N} x}
\infrule[Par-barb]{\mbox{$P\downarrow_{\mathcal N} x$ or $Q\downarrow_{\mathcal N} x$}}
		  {\binpar{P}{Q} \downarrow_{\mathcal N} x}

We write $P \Downarrow_{\mathcal N} x$ if there is $Q$ such that 
$P \wred Q$ and $Q \downarrow_{\mathcal N} x$.
\end{definition}

\begin{definition}
%\label{def.bbisim}
An  ${\mathcal N}$-\emph{barbed bisimulation} over a set of names, ${\mathcal N}$, is a symmetric binary relation 
${\mathcal S}_{\mathcal N}$ between agents such that $P\rel{S}_{\mathcal N}Q$ implies:
\begin{enumerate}
\item If $P \red P'$ then $Q \wred Q'$ and $P'\rel{S}_{\mathcal N} Q'$.
\item If $P\downarrow_{\mathcal N} x$, then $Q\Downarrow_{\mathcal N} x$.
\end{enumerate}
$P$ is ${\mathcal N}$-barbed bisimilar to $Q$, written
$P \wbbisim_{\mathcal N} Q$, if $P \rel{S}_{\mathcal N} Q$ for some ${\mathcal N}$-barbed bisimulation ${\mathcal S}_{\mathcal N}$.
\end{definition}

$\mathcal{R} \subseteq \pi \times \pi$

$P \mathcal{R} Q => \forall P'. P \red P' \Rightarrow \exists Q'. Q \red Q', P' \mathcal{R} Q'$

$P \vdash x \Rightarrow Q \vdash x$

\begin{mathpar}
  \inferrule*[lab=Out-barb]{x \nameeq y}{{y}!\langle{Q}\rangle \vdash x}
  \and
  \inferrule*[lab=Par-barb]{\mbox{$P\vdash x$ or $Q\vdash x$}}{\binpar{P}{Q} \vdash x}
\end{mathpar}

\subsubsection{Contexts}

One of the principle advantages of computational calculi like the
$\pi$-calculus is a well-defined notion of context,
contextual-equivalence and a correlation between
contextual-equivalence and notions of bisimulation. The notion of
context allows the decomposition of a process into (sub-)process and
its syntactic environment, its context. Thus, a context may be
thought of as a process with a ``hole'' (written $\Box$) in it. The
application of a context $M$ to a process $P$, written $M[P]$, is
tantamount to filling the hole in $M$ with $P$. In this paper we do
not need the full weight of this theory, but do make use of the notion
of context in the proof the main theorem. 

\begin{mathpar}
  \inferrule* [lab=summation] {} {{M_{M},M_{N}} \bc \Box \;|\; x.M_{A} \;|\; M_{M}+M_{N}}
  \and
  \inferrule* [lab=agent] {} {{M_{A}} \bc (\vec{x})M_{P} \;| \; \clift{P_0,\ldots,M_{P},\ldots,P_N}}
  \and \\
  \inferrule* [lab=process] {} {{M_{P}} \bc M_{N} \;| \;P|M_{P} }
\end{mathpar} 

\begin{mathpar}
  \inferrule* [lab=sychronization] {} {M_{N} \bc \Box \;|\; x?M_{F} \;|\; x!M_{C}}
  \and
  \inferrule* [lab=abstraction] {} {{M_{F}} \bc (x)M_{P} }
  \and
  \inferrule* [lab=concretion] {} {{M_{C}} \bc \langle M_{P} \rangle }
  \and \\
  \inferrule* [lab=process] {} {{M_{P}} \bc M_{N} \;| \;P|M_{P} }
\end{mathpar}

\begin{definition}[contextual application] Given a context $M$, and
  process $P$, we define the \emph{contextual application}, $M[P] :=
  M\{P/\Box\}$. That is, the contextual application of M to P is the
  substitution of $P$ for $\Box$ in $M$.
\end{definition}

$\meaningof{-} : L \to \mathcal{P}(\pi)$

\begin{mathpar}
  \inferrule* [lab=collection] {} {\meaningof{true} = \pi, \and \meaningof{~E} = \pi \setminus \meaningof{E}, \and \meaningof{E_{1} \& E_{2}} = \meaningof{E_{1}} \cap \meaningof{E_{2}}}
\end{mathpar}

\begin{mathpar}
  \inferrule* [lab=structure] {} {\meaningof{0} = \{ P \in \pi | P \equiv 0 \}, \and \\ \meaningof{E_1 | E_2} = \{ P \in \pi | P \equiv P_{1} | P_{2}, P_{1} \in \meaningof{E_{1}}, P_{2} \in \meaningof{E_2}\} }
\end{mathpar}

\begin{mathpar}
 \inferrule* [lab=behavior] {} {\meaningof{\langle a?b \rangle E} = \{ P \in \pi | P \equiv Q | u?(y)P', \\ \and \\\\ \and \\ \;\;\; u \in \meaningof{a}, \forall z.P'\{z/y\} \in \meaningof{E\{z/b\}}\}, \and \\ \meaningof{a!E} = \{ P \in \pi | P \equiv Q | x!\langle P' \rangle, x \in \meaningof{a} P' \in \meaningof{E}\} }
\end{mathpar}

\begin{mathpar}
 \inferrule* [lab=nominal] {} {\meaningof{\quotep{E}} = \{ \quotep{P} \in \quotep{\pi} | P \in \meaningof{E} \}, \and \meaningof{\quotep{P}} = \{ \quotep{Q} \in \quotep{\pi} | P \equiv Q \} \and \\ \meaningof{@\quotep{E}} = \{ P \in \pi | P \equiv @x, x \in \meaningof{E} \}}
\end{mathpar}

\begin{eqnarray*}
  \\
  \meaningof{-} : TS \to ST
\end{eqnarray*}

\begin{eqnarray*}
  \\
  L : TS \to ST
\end{eqnarray*}

\begin{eqnarray*}
  \\
  P \models E \iff P \in \meaningof{E}
\end{eqnarray*}

\begin{eqnarray*}
  P \approx_{L} Q \iff \forall E \in L. P \models E \iff Q \models E
\end{eqnarray*}

\begin{eqnarray*}
  P \approx_{K} Q
\end{eqnarray*}

\begin{eqnarray*}
  P \approx Q
\end{eqnarray*}

$\approx_{K} = \approx = \approx_{L}$

\subsubsection{Contextual duality}

Note that contexts extend the quotation operation to a family of
operations from processes to names. Given a context, $M$, we can
define a \emph{nominal context}, $\quotep{M}$ by $\quotep{M}[P] :=
\quotep{M[P]}$. To foreshadow what is to come we observe that these
operations enjoy a duality with processes very much like the duality
between vectors and maps from vectors to scalars.

Further, because the calculus is essentially higher-order, we have a
correspondence between contexts and processes. More specifically,
given a name $x$ and a context $M$ we can construct $M^{*}_{x}$ such
that 

\begin{mathpar}
  M^{*}_{x} | \lift{x}{P} \red M[P]
\end{mathpar}

namely,

\begin{mathpar}
  M^{*}_{x} := x?(u).M[\dropn{u}]
\end{mathpar}

The dependence of $M^{*}_{x}$ on a name makes it an abstraction, 

\begin{mathpar}
  M^{*} := (x)x?(u).M[\dropn{u}]
\end{mathpar}

\subsection{Additional notation}

It will sometimes be convenient to denote the process a name
quotes. We already have the notation $x = \quotep{P}$, but it will be
convenient to introduce an alternate notation, $\procn{x}$, when we
want to emphasize the connection to the use of the name. Note that, by
virtue of name equivalence, $\quotep{\procn{x}} \nameeq x$; so, the
notation is consistent with previous definitions.

Further, because names have structure it is possible to effect
substitutions on the basis of that structure. This means we need to
upgrade our notation for substitutions, which we accomplish by
adapting comprehension notation. Thus,

\begin{mathpar}
  P\{ y / x : x \in S \}
\end{mathpar}

is interpreted to mean the process derived from P by replacing (in a
capture-avoiding manner) each occurrence of $x$ in $S$ by $y$. For example,

\begin{mathpar}
  P\{ \quotep{\procn{x}|\procn{x}} / x : x \in \freenames{P} \}
\end{mathpar}

will replace each (occurrence) of a free name $x$ in $P$ by
$\quotep{\procn{x}|\procn{x}}$.

Also, we will avail ourselves of the notation $x^{L}$ and $x^{R}$ to
denote injections of a name into disjoint copies of the name
space. There are numerous ways to accomplish this. One example can be
found in \cite{MeredithR05}. This notation overloads to vectors of
names: $\vec{x}^{\pi} := (x_{i}^{\pi} \; : \; 0 \leq i < |\vec{x}| )$ where $\pi \in \{L,R\}$.

We also use $P^{\Box} := P|\Box$.

In \cite{MeredithR05} an interpretation of the new operator is
given. It turns out that there are several possible interpretations
all enjoying the requisite algebraic properties of the operator (see
\cite{milner91polyadicpi}). We will therefore make liberal use of
$(\nu\; \vec{x})P$.

% subsection the_syntax_and_semantics_of_the_notation_system (end)   

\input{qm2pi.qmops} 

\input{qm2pi.sterngerlach} 

\input{qm2pi.metric} 

% section concurrent_process_calculi (end)

%\input{qm2pi.proofsketch}

% section proof sketch (end)

%\input{qm2pi.slviaknots} 

% section spatial logic via knots (end)

\input{qm2pi.conclusion}

% section conclusion (end)

%\input{qm2pi.dtcodes} 

% section wiring algorithm (end)

\input{qm2pi.ack} 

% section acknowledgments (end)

\newpage


\bibliographystyle{plain}   
\bibliography{../../biblios/main.bib}

\input{qm2pi.rhodetails}

\end{document}

 

\documentclass[12pt]{llncs}
%\documentclass{jktr}

\usepackage[pdftex]{hyperref}                   
\usepackage {listings}
\usepackage {mathpartir}
\usepackage{bcprules}
%\usepackage{listings}
                       
\usepackage{graphicx} 
%\usepackage[margins=2.5cm,nohead,nofoot]{geometry}
%\usepackage{geometry}
\usepackage{amsfonts}
\usepackage{amstext}
\usepackage{latexsym}
\usepackage{amssymb}
\usepackage{color}


%\include{myPreamble}
\include{qm2pi.local} 

%\ifpdf
%\usepackage[pdftex]{graphicx}
%\else
%\usepackage{graphicx}
%\fi

 % \ifpdf
%  \usepackage{pdfsync}
%  \if


%\title{Brief Article}
%\author{David F. Snyder}
%\author{L.G. Meredith}

%\address{Dept. of Math., Texas State University--San Marcos, San Marcos, TX 78666}
       
\pagestyle{empty}


\begin{document}

\lstset{language=[Objective]Caml,frame=shadowbox}

\input{qm2pi.front}

% section front matter (end)

\input{qm2pi.intro} 
 
% section introduction (end)

% \input{qm2pi.knotations} 

% section notation (end)

\input{qm2pi.process.calculi} 

% section concurrent_process_calculi_and_spatial_logics_ (end)
    
%\input{qm2pi.knots2pi} 

%\input{qm2pi.trefoil} 

%\input{qm2pi.mainthm} 

% subsection basic_interpretation (end)

%\input{qm2pi.rho.presentation} 
\subsection{The syntax and semantics of the notation system}\label{sub:the_syntax_and_semantics_of_the_notation_system} % (fold)

We now summarize a technical presentation of the calculus that
embodies our theory of dynamics. The typical presentation of such a
calculus follows the style of giving generators and relations on
them. The grammar, below, describing term constructors, freely
generates the set of processes, $\Proc$. This set is then quotiented
by a relation known as structural congruence and it is over this set
that the notion of dynamics is expressed. This presentation is
essentially that of \cite{MeredithR05} with the addition of
polyadicity and summation. For readability we have relegated some of
the technical subtleties to an appendix.

\subsubsection{Process grammar}\label{subsub:process_grammar}

\begin{mathpar}
  \inferrule* [lab=synchronization] {} {{M} \bc \pzero \;|\; x?F \;|\; x!C }
  \and
  \inferrule* [lab=abstraction] {} {{F} \bc (x)P}
  \and
  \inferrule* [lab=concretion] {} {{C} \bc \langle Q \rangle}
  \and
  \inferrule* [lab=process] {} {{P,Q} \bc M \;| \;P|Q \;|\; @{x}}
  \and
  \inferrule* [lab=name] {} {{x} \bc \quotep{P}}
\end{mathpar} 

Note that $\vec{x}$ (resp. $\vec{P}$) denotes a vector of names
(resp. processes) of length $|\vec{x}|$ (resp. $|\vec{P}|$). We adopt
the following useful abbreviations.

\begin{mathpar}
   x?(\vec{y}).P := x.(\vec{y})P \and  x\clift{\vec{P}} := x.\clift{\vec{P}}
   \and x!(y) := \lift{x}{\dropn{y}}
   \and \Pi_{i=0}^{n-1}P_i := P_0 | \ldots | P_{n-1}
\end{mathpar}

\subsubsection{Structural congruence}

\paragraph{Free and bound names and alpha-equivalence.} At the
core of structural equivalence is alpha-equivalence which identifies
process that are the same up to a change of variable. Formally, we
recognize the distinction between free and bound names. The free names
of a process, $\freenames{P}$, may be calculated recursively as
follows:

\begin{mathpar}
\freenames{\pzero} := \emptyset
  \and \\
  \freenames{x?(y).P} := \{ x \} \cup (\freenames{P} \setminus \{ y \})
  \and 
  \freenames{x!\langle P \rangle} := \{ x \} \cup \{ P \} 
  \and \\
  \freenames{P|Q} := \freenames{P} \cup \freenames{Q}
  \and \\
  \freenames{@{x}} := \{ x \}
\end{mathpar}

$\pi$
$\quotep{\pi}$

$\freenames{-} : \pi \to \mathcal{P}(\quotep{\pi})$

\begin{eqnarray*}
  \freenames{\pzero} & := & \emptyset \\
  \freenames{x?(y).P} & := & \{ x \} \cup (\freenames{P} \setminus \{ y \}) \\
  \freenames{x!\langle P \rangle} & := & \{ x \} \cup \{ P \} \\
  \freenames{P|Q} & := & \freenames{P} \cup \freenames{Q} \\
  \freenames{\dropn{x}} & := & \{ x \}
\end{eqnarray*}

The bound names of a process, $\boundnames{P}$, are those names occurring in $P$
that are not free. For example, in $x?(y).0$, the name $x$ is free, while $y$ is bound.

\begin{mathpar}
  \inferrule* [lab=monoidal-laws] {} { P|Q \equiv Q|P \and P|0 \equiv P \and P|(Q|R) \equiv (P|Q)|R }
\end{mathpar}

\begin{mathpar}
  \inferrule* [lab=alpha-equivalence] {} { (x)P \equiv (y)P\{y/x\} \and y \not\in \freenames{P} }
\end{mathpar}

\begin{definition}
Then two processes, $P,Q$, are alpha-equivalent if $P = Q\{\vec{y}/\vec{x}\}$ for
some $\vec{x} \in \boundnames{Q},\vec{y} \in \boundnames{P}$, where $Q\{\vec{y}/\vec{x}\}$
denotes the capture-avoiding substitution of $\vec{y}$ for $\vec{x}$ in $Q$.
\end{definition}

\begin{definition}
  The {\em structural congruence} \cite{SangiorgiWalker} , $\equiv$,
  between processes is the least congruence containing
  alpha-equivalence, satisfying the abelian monoid laws
  (associativity, commutativity and $\pzero$ as identity) for parallel
  composition $|$ and for summation $+$.
\end{definition}

\subsection{Name equivalence}

We take name equivalence, written $\nameeq$, to be the smallest
equivalence relation generated by the following rules.

\begin{mathpar}
\inferrule*[lab=Quote-drop]
{ }
{ \quotep{@{x}} \nameeq x }

\inferrule*[lab=Struct-equiv]
{ P \scong Q }
{ \quotep{P} \nameeq \quotep{Q} }
\end{mathpar}

The astute reader will have noticed that the mutual recursion of names
and processes imposes a mutual recursion on alpha-equivalence and
structural equivalence via name-equivalence. Fortunately, all of this
works out pleasantly and we may calculate in the natural way, free of
concern. The reader interested in the details is referred to the
appendix \ref{appendix:rho_details}.

\subsection{Substitution}

We use $\Proc$ for the set of processes, $\QProc$ for the set of
names, and $\id{\{}\vec{y} / \vec{x} \id{\}}$ to denote partial maps,
$s : \QProc \rightarrow \QProc$. A map, $s$ lifts, uniquely, to a map
on process terms, $\widehat{s} : \Proc \rightarrow \Proc$ by the
following equations.

\begin{mathpar}
  (0) \psubstp{Q}{P} := 0 \\
  (R \juxtap S) \psubstp{Q}{P}
  :=    
  (R)\psubstp{Q}{P} \juxtap (S) \psubstp{Q}{P} \\
  (x?(y).R) \psubstp{Q}{P}    
  :=    
  (x)\substp{Q}{P} (z)\concat( (R \psubstn{z}{y}) \psubstp{Q}{P} ) \\
  (\lift{x}{R}) \psubstp{Q}{P}  
  :=
  \lift{(x)\substp{Q}{P}}{ R \psubstp{Q}{P} } \\
%   (\dropn{x})  \psubstp{Q}{P}       
%   := 
%   \left\{ 
%     \begin{array}{ccc} 
%       \dropn{\quotep{Q}} & & x \nameeq \quotep{P} \\
%       \dropn{x} & & otherwise \\
%     \end{array}
%   \right. 
  (\dropn{x})  \psubstp{Q}{P}       
  := 
  \left\{ 
    \begin{array}{ccc} 
      Q & & x \nameeq \quotep{P} \\
      \dropn{x} & & otherwise \\
    \end{array}
  \right.
\end{mathpar}
 

where

\begin{eqnarray}
  (x)\id{\{} \lpquote Q \rpquote / \lpquote P \rpquote \id{\}}            = 
  \left\{ 
    \begin{array}{ccc}
      \lpquote Q \rpquote & & x \nameeq \lpquote P \rpquote \\
      x & & otherwise \\
    \end{array}
  \right. \nonumber
\end{eqnarray}

and $z$ is chosen distinct from $\quotep{P}$, $\quotep{Q}$, the free
names in $Q$, and all the names in $R$. Our $\alpha$-equivalence will
be built in the standard way from this substitution.

\begin{remark}\label{rem:no_self_referential_names}
  One consequence of these definitions is that $\forall P. \quotep{P}
  \not\in \freenames{P}$.
\end{remark}

\subsection{ Dynamic quote: an example }

Anticipating something of what's to come, consider applying the
substitution, $\widehat{\id{\{}u / z \id{\}}}$, to the following pair
of processes, $\lift{w}{y!(z)}$ and $w[ \lpquote y!(z) \rpquote ]$.

\begin{eqnarray}
	\lift{w}{y!(z)}\widehat{\id{\{}u / z \id{\}}}
		& = &
		\lift{w}{y!(u)} \nonumber\\
	w[ \lpquote y!(z) \rpquote ] \widehat{ \id{\{}u / z \id{\}} }
		& = &
		w[ \lpquote y!(z) \rpquote ] \nonumber
\end{eqnarray}

Because the body of the process between quotes is impervious to
substitution, we get radically different answers. In fact, by
examining the first process in an input context,
e.g. $x?(z).\lift{w}{y!(z)}$, we see that the process under the lift
operator may be shaped by prefixed inputs binding a name inside it. In
this sense, the lift operator will be seen as a way to dynamically
construct processes before reifying them as names.

Finally equipped with these standard features we can present the
dynamics of the calculus.

\subsubsection{Operational semantics} 

Finally, we introduce the computational dynamics. What marks these
algebras as distinct from other more traditionally studied algebraic
structures, e.g. vector spaces or polynomial rings, is the manner in
which dynamics is captured. In traditional structures, dynamics is typically
expressed through morphisms between such structures, as in linear maps
between vector spaces or morphisms between rings. In algebras
associated with the semantics of computation, the dynamics is
expressed as part of the algebraic structure itself, through a
reduction reduction relation typically denoted by $\red$. Below, we
give a recursive presentation of this relation for the calculus used
in the encoding.

$\red \subseteq \pi \times \pi$
$\red : \pi \to \mathcal{P}(\pi)$

\begin{mathpar}
  \inferrule* [lab=Comm] { \textsf{match}( x_{src}, x_{trgt} ) } { x_{trgt}?(y)P \; | \; x_{src}!\langle {Q} \rangle \red P\{\quotep{Q}/y}\} }
  \and \\
  \inferrule* [lab=Par] {{P} \red {P}'} {{{P} | {Q}} \red {{P}' | {Q}}}
  \and
  \inferrule* [lab=Equiv]{{{P} \scong {P}'} \andalso {{P}' \red {Q}'} \andalso {{Q}' \scong {Q}}}{{P} \red {Q}}
\end{mathpar}

\begin{eqnarray*}
  match_{\equiv} (\quotep{P},\quotep{Q}) & := & P \equiv Q \\
  match_{\dagger}(\quotep{P},\quotep{Q}) & := & \forall R. P|Q \red^{*} R => R \red^{*} 0 \\
  match_{K}(\quotep{P},\quotep{Q}) & := & K \mbox{ for some context } K
\end{eqnarray*}

$u?(x)P | u!\langle Q \rangle \red P\{\quotep{Q}/x\}$

%We write $\wred$ for $\red^*$, and $P\red$ if $\exists Q $ such that $ P \red Q$.
We write $P\red$ if $\exists Q $ such that $ P \red Q$ and $P\not\red$, otherwise.

\section{Replication}

As mentioned before, it is known that replication (and hence
recursion) can be implemented in a higher-order process algebra
\cite{SangiorgiWalker}. As our first example of calculation with the
machinery thus far presented we give the construction explicitly in
the {\rhoc}.

\begin{eqnarray}
	D_{x} & := & \prefix{x}{y}{(\binpar{\outputp{x}{y}}{@{y}})} \nonumber\\
	\bangp_{x}{P} & := & \binpar{{x}!\langle{\binpar{D_{x}}{P}}\rangle}{D_{x}} \nonumber
\end{eqnarray}

\begin{eqnarray}
	\bangp_{x}{P} & & \nonumber\\
	=
	& {x}!\langle{(\prefix{x}{y}{(\outputp{x}{y} | @{y})) | P}}\rangle 
	      | \prefix{x}{y}{(\outputp{x}{y} | @{y})} & \nonumber\\
	\red
	& (\outputp{x}{y} | @{y})\substn{\quotep{(\prefix{x}{y}{(@{y} | \outputp{x}{y})) | P}}}{y} & \nonumber\\
	=
	& \outputp{x}{\quotep{(\prefix{x}{y}{(\outputp{x}{y} | @{y})) | P}}}
	  | {(\prefix{x}{y}{(\outputp{x}{y} | @{y})) | P}} & \nonumber\\
	\red
	& \ldots & \nonumber\\
	\red^*
	& P | P | \ldots & \nonumber
\end{eqnarray}

Of course, this encoding, as an implementation, runs away, unfolding
$\bangp{P}$ eagerly. A lazier and more implementable replication
operator, restricted to input-guarded processes, may be obtained as follows.

\begin{eqnarray}
\bangp{\prefix{u}{v}{P}} 
	:= 
	\binpar{\lift{x}{\prefix{u}{v}{(\binpar{D(x)}{P})}}}{D(x)} \nonumber
\end{eqnarray}

\begin{remark}
  Note that the lazier definition still does not deal with summation
  or mixed summation (i.e. sums over input and output). The reader is
  invited to construct definitions of replication that deal with these
  features. 

  Further, the definitions are parameterized in a name, $x$. Can you,
  gentle reader, make a definition that eliminates this parameter and
  guarantees no accidental interaction between the replication
  machinery and the process being replicated -- i.e. no accidental
  sharing of names used by the process to get its work done and the
  name(s) used by the replication to effect copying. This latter
  revision of the definition of replication is crucial to obtaining
  the expected identity $!!P \sim !P$.
\end{remark}

\begin{remark}\label{rem:paradoxical_combinator}
  The reader familiar with the lambda calculus will have noticed the
  similarity between $D$ and the paradoxical combinator.

  [Ed. note: the existence of this seems to suggest we have to be more
  restrictive on the set of processes and names we admit if we are to
  support no-cloning.]
\end{remark}

\subsubsection{Bisimulation}

The computational dynamics gives rise to another kind of equivalence,
the equivalence of computational behavior. As previously mentioned
this is typically captured \emph{via} some form of bisimulation.

% The notion we use in this paper is weak barbed bisimulation
% \cite{milner91polyadicpi}.

The notion we use in this paper is derived from weak barbed
bisimulation \cite{milner91polyadicpi}. 

\begin{definition}
An \emph{observation relation}, $\downarrow_{\mathcal N}$, over a set
of names, $\mathcal N$, is the smallest relation satisfying the rules
below.

\infrule[Out-barb]{y \in {\mathcal N}, \; x \nameeq y}
		  {\outputp{x}{v} \downarrow_{\mathcal N} x}
\infrule[Par-barb]{\mbox{$P\downarrow_{\mathcal N} x$ or $Q\downarrow_{\mathcal N} x$}}
		  {\binpar{P}{Q} \downarrow_{\mathcal N} x}

We write $P \Downarrow_{\mathcal N} x$ if there is $Q$ such that 
$P \wred Q$ and $Q \downarrow_{\mathcal N} x$.
\end{definition}

\begin{definition}
%\label{def.bbisim}
An  ${\mathcal N}$-\emph{barbed bisimulation} over a set of names, ${\mathcal N}$, is a symmetric binary relation 
${\mathcal S}_{\mathcal N}$ between agents such that $P\rel{S}_{\mathcal N}Q$ implies:
\begin{enumerate}
\item If $P \red P'$ then $Q \wred Q'$ and $P'\rel{S}_{\mathcal N} Q'$.
\item If $P\downarrow_{\mathcal N} x$, then $Q\Downarrow_{\mathcal N} x$.
\end{enumerate}
$P$ is ${\mathcal N}$-barbed bisimilar to $Q$, written
$P \wbbisim_{\mathcal N} Q$, if $P \rel{S}_{\mathcal N} Q$ for some ${\mathcal N}$-barbed bisimulation ${\mathcal S}_{\mathcal N}$.
\end{definition}

$\mathcal{R} \subseteq \pi \times \pi$

$P \mathcal{R} Q => \forall P'. P \red P' \Rightarrow \exists Q'. Q \red Q', P' \mathcal{R} Q'$

$P \vdash x \Rightarrow Q \vdash x$

\begin{mathpar}
  \inferrule*[lab=Out-barb]{x \nameeq y}{{y}!\langle{Q}\rangle \vdash x}
  \and
  \inferrule*[lab=Par-barb]{\mbox{$P\vdash x$ or $Q\vdash x$}}{\binpar{P}{Q} \vdash x}
\end{mathpar}

\subsubsection{Contexts}

One of the principle advantages of computational calculi like the
$\pi$-calculus is a well-defined notion of context,
contextual-equivalence and a correlation between
contextual-equivalence and notions of bisimulation. The notion of
context allows the decomposition of a process into (sub-)process and
its syntactic environment, its context. Thus, a context may be
thought of as a process with a ``hole'' (written $\Box$) in it. The
application of a context $M$ to a process $P$, written $M[P]$, is
tantamount to filling the hole in $M$ with $P$. In this paper we do
not need the full weight of this theory, but do make use of the notion
of context in the proof the main theorem. 

\begin{mathpar}
  \inferrule* [lab=summation] {} {{M_{M},M_{N}} \bc \Box \;|\; x.M_{A} \;|\; M_{M}+M_{N}}
  \and
  \inferrule* [lab=agent] {} {{M_{A}} \bc (\vec{x})M_{P} \;| \; \clift{P_0,\ldots,M_{P},\ldots,P_N}}
  \and \\
  \inferrule* [lab=process] {} {{M_{P}} \bc M_{N} \;| \;P|M_{P} }
\end{mathpar} 

\begin{mathpar}
  \inferrule* [lab=sychronization] {} {M_{N} \bc \Box \;|\; x?M_{F} \;|\; x!M_{C}}
  \and
  \inferrule* [lab=abstraction] {} {{M_{F}} \bc (x)M_{P} }
  \and
  \inferrule* [lab=concretion] {} {{M_{C}} \bc \langle M_{P} \rangle }
  \and \\
  \inferrule* [lab=process] {} {{M_{P}} \bc M_{N} \;| \;P|M_{P} }
\end{mathpar}

\begin{definition}[contextual application] Given a context $M$, and
  process $P$, we define the \emph{contextual application}, $M[P] :=
  M\{P/\Box\}$. That is, the contextual application of M to P is the
  substitution of $P$ for $\Box$ in $M$.
\end{definition}

$\meaningof{-} : L \to \mathcal{P}(\pi)$

\begin{mathpar}
  \inferrule* [lab=collection] {} {\meaningof{true} = \pi, \and \meaningof{~E} = \pi \setminus \meaningof{E}, \and \meaningof{E_{1} \& E_{2}} = \meaningof{E_{1}} \cap \meaningof{E_{2}}}
\end{mathpar}

\begin{mathpar}
  \inferrule* [lab=structure] {} {\meaningof{0} = \{ P \in \pi | P \equiv 0 \}, \and \\ \meaningof{E_1 | E_2} = \{ P \in \pi | P \equiv P_{1} | P_{2}, P_{1} \in \meaningof{E_{1}}, P_{2} \in \meaningof{E_2}\} }
\end{mathpar}

\begin{mathpar}
 \inferrule* [lab=behavior] {} {\meaningof{\langle a?b \rangle E} = \{ P \in \pi | P \equiv Q | u?(y)P', \\ \and \\\\ \and \\ \;\;\; u \in \meaningof{a}, \forall z.P'\{z/y\} \in \meaningof{E\{z/b\}}\}, \and \\ \meaningof{a!E} = \{ P \in \pi | P \equiv Q | x!\langle P' \rangle, x \in \meaningof{a} P' \in \meaningof{E}\} }
\end{mathpar}

\begin{mathpar}
 \inferrule* [lab=nominal] {} {\meaningof{\quotep{E}} = \{ \quotep{P} \in \quotep{\pi} | P \in \meaningof{E} \}, \and \meaningof{\quotep{P}} = \{ \quotep{Q} \in \quotep{\pi} | P \equiv Q \} \and \\ \meaningof{@\quotep{E}} = \{ P \in \pi | P \equiv @x, x \in \meaningof{E} \}}
\end{mathpar}

\begin{eqnarray*}
  \\
  \meaningof{-} : TS \to ST
\end{eqnarray*}

\begin{eqnarray*}
  \\
  L : TS \to ST
\end{eqnarray*}

\begin{eqnarray*}
  \\
  P \models E \iff P \in \meaningof{E}
\end{eqnarray*}

\begin{eqnarray*}
  P \approx_{L} Q \iff \forall E \in L. P \models E \iff Q \models E
\end{eqnarray*}

\begin{eqnarray*}
  P \approx_{K} Q
\end{eqnarray*}

\begin{eqnarray*}
  P \approx Q
\end{eqnarray*}

$\approx_{K} = \approx = \approx_{L}$

\subsubsection{Contextual duality}

Note that contexts extend the quotation operation to a family of
operations from processes to names. Given a context, $M$, we can
define a \emph{nominal context}, $\quotep{M}$ by $\quotep{M}[P] :=
\quotep{M[P]}$. To foreshadow what is to come we observe that these
operations enjoy a duality with processes very much like the duality
between vectors and maps from vectors to scalars.

Further, because the calculus is essentially higher-order, we have a
correspondence between contexts and processes. More specifically,
given a name $x$ and a context $M$ we can construct $M^{*}_{x}$ such
that 

\begin{mathpar}
  M^{*}_{x} | \lift{x}{P} \red M[P]
\end{mathpar}

namely,

\begin{mathpar}
  M^{*}_{x} := x?(u).M[\dropn{u}]
\end{mathpar}

The dependence of $M^{*}_{x}$ on a name makes it an abstraction, 

\begin{mathpar}
  M^{*} := (x)x?(u).M[\dropn{u}]
\end{mathpar}

\subsection{Additional notation}

It will sometimes be convenient to denote the process a name
quotes. We already have the notation $x = \quotep{P}$, but it will be
convenient to introduce an alternate notation, $\procn{x}$, when we
want to emphasize the connection to the use of the name. Note that, by
virtue of name equivalence, $\quotep{\procn{x}} \nameeq x$; so, the
notation is consistent with previous definitions.

Further, because names have structure it is possible to effect
substitutions on the basis of that structure. This means we need to
upgrade our notation for substitutions, which we accomplish by
adapting comprehension notation. Thus,

\begin{mathpar}
  P\{ y / x : x \in S \}
\end{mathpar}

is interpreted to mean the process derived from P by replacing (in a
capture-avoiding manner) each occurrence of $x$ in $S$ by $y$. For example,

\begin{mathpar}
  P\{ \quotep{\procn{x}|\procn{x}} / x : x \in \freenames{P} \}
\end{mathpar}

will replace each (occurrence) of a free name $x$ in $P$ by
$\quotep{\procn{x}|\procn{x}}$.

Also, we will avail ourselves of the notation $x^{L}$ and $x^{R}$ to
denote injections of a name into disjoint copies of the name
space. There are numerous ways to accomplish this. One example can be
found in \cite{MeredithR05}. This notation overloads to vectors of
names: $\vec{x}^{\pi} := (x_{i}^{\pi} \; : \; 0 \leq i < |\vec{x}| )$ where $\pi \in \{L,R\}$.

We also use $P^{\Box} := P|\Box$.

In \cite{MeredithR05} an interpretation of the new operator is
given. It turns out that there are several possible interpretations
all enjoying the requisite algebraic properties of the operator (see
\cite{milner91polyadicpi}). We will therefore make liberal use of
$(\nu\; \vec{x})P$.

% subsection the_syntax_and_semantics_of_the_notation_system (end)   

\input{qm2pi.qmops} 

\input{qm2pi.sterngerlach} 

\input{qm2pi.metric} 

% section concurrent_process_calculi (end)

%\input{qm2pi.proofsketch}

% section proof sketch (end)

%\input{qm2pi.slviaknots} 

% section spatial logic via knots (end)

\input{qm2pi.conclusion}

% section conclusion (end)

%\input{qm2pi.dtcodes} 

% section wiring algorithm (end)

\input{qm2pi.ack} 

% section acknowledgments (end)

\newpage


\bibliographystyle{plain}   
\bibliography{../../biblios/main.bib}

\input{qm2pi.rhodetails}

\end{document}

 

% section concurrent_process_calculi (end)

%\documentclass[12pt]{llncs}
%\documentclass{jktr}

\usepackage[pdftex]{hyperref}                   
\usepackage {listings}
\usepackage {mathpartir}
\usepackage{bcprules}
%\usepackage{listings}
                       
\usepackage{graphicx} 
%\usepackage[margins=2.5cm,nohead,nofoot]{geometry}
%\usepackage{geometry}
\usepackage{amsfonts}
\usepackage{amstext}
\usepackage{latexsym}
\usepackage{amssymb}
\usepackage{color}


%\include{myPreamble}
\include{qm2pi.local} 

%\ifpdf
%\usepackage[pdftex]{graphicx}
%\else
%\usepackage{graphicx}
%\fi

 % \ifpdf
%  \usepackage{pdfsync}
%  \if


%\title{Brief Article}
%\author{David F. Snyder}
%\author{L.G. Meredith}

%\address{Dept. of Math., Texas State University--San Marcos, San Marcos, TX 78666}
       
\pagestyle{empty}


\begin{document}

\lstset{language=[Objective]Caml,frame=shadowbox}

\input{qm2pi.front}

% section front matter (end)

\input{qm2pi.intro} 
 
% section introduction (end)

% \input{qm2pi.knotations} 

% section notation (end)

\input{qm2pi.process.calculi} 

% section concurrent_process_calculi_and_spatial_logics_ (end)
    
%\input{qm2pi.knots2pi} 

%\input{qm2pi.trefoil} 

%\input{qm2pi.mainthm} 

% subsection basic_interpretation (end)

%\input{qm2pi.rho.presentation} 
\subsection{The syntax and semantics of the notation system}\label{sub:the_syntax_and_semantics_of_the_notation_system} % (fold)

We now summarize a technical presentation of the calculus that
embodies our theory of dynamics. The typical presentation of such a
calculus follows the style of giving generators and relations on
them. The grammar, below, describing term constructors, freely
generates the set of processes, $\Proc$. This set is then quotiented
by a relation known as structural congruence and it is over this set
that the notion of dynamics is expressed. This presentation is
essentially that of \cite{MeredithR05} with the addition of
polyadicity and summation. For readability we have relegated some of
the technical subtleties to an appendix.

\subsubsection{Process grammar}\label{subsub:process_grammar}

\begin{mathpar}
  \inferrule* [lab=synchronization] {} {{M} \bc \pzero \;|\; x?F \;|\; x!C }
  \and
  \inferrule* [lab=abstraction] {} {{F} \bc (x)P}
  \and
  \inferrule* [lab=concretion] {} {{C} \bc \langle Q \rangle}
  \and
  \inferrule* [lab=process] {} {{P,Q} \bc M \;| \;P|Q \;|\; @{x}}
  \and
  \inferrule* [lab=name] {} {{x} \bc \quotep{P}}
\end{mathpar} 

Note that $\vec{x}$ (resp. $\vec{P}$) denotes a vector of names
(resp. processes) of length $|\vec{x}|$ (resp. $|\vec{P}|$). We adopt
the following useful abbreviations.

\begin{mathpar}
   x?(\vec{y}).P := x.(\vec{y})P \and  x\clift{\vec{P}} := x.\clift{\vec{P}}
   \and x!(y) := \lift{x}{\dropn{y}}
   \and \Pi_{i=0}^{n-1}P_i := P_0 | \ldots | P_{n-1}
\end{mathpar}

\subsubsection{Structural congruence}

\paragraph{Free and bound names and alpha-equivalence.} At the
core of structural equivalence is alpha-equivalence which identifies
process that are the same up to a change of variable. Formally, we
recognize the distinction between free and bound names. The free names
of a process, $\freenames{P}$, may be calculated recursively as
follows:

\begin{mathpar}
\freenames{\pzero} := \emptyset
  \and \\
  \freenames{x?(y).P} := \{ x \} \cup (\freenames{P} \setminus \{ y \})
  \and 
  \freenames{x!\langle P \rangle} := \{ x \} \cup \{ P \} 
  \and \\
  \freenames{P|Q} := \freenames{P} \cup \freenames{Q}
  \and \\
  \freenames{@{x}} := \{ x \}
\end{mathpar}

$\pi$
$\quotep{\pi}$

$\freenames{-} : \pi \to \mathcal{P}(\quotep{\pi})$

\begin{eqnarray*}
  \freenames{\pzero} & := & \emptyset \\
  \freenames{x?(y).P} & := & \{ x \} \cup (\freenames{P} \setminus \{ y \}) \\
  \freenames{x!\langle P \rangle} & := & \{ x \} \cup \{ P \} \\
  \freenames{P|Q} & := & \freenames{P} \cup \freenames{Q} \\
  \freenames{\dropn{x}} & := & \{ x \}
\end{eqnarray*}

The bound names of a process, $\boundnames{P}$, are those names occurring in $P$
that are not free. For example, in $x?(y).0$, the name $x$ is free, while $y$ is bound.

\begin{mathpar}
  \inferrule* [lab=monoidal-laws] {} { P|Q \equiv Q|P \and P|0 \equiv P \and P|(Q|R) \equiv (P|Q)|R }
\end{mathpar}

\begin{mathpar}
  \inferrule* [lab=alpha-equivalence] {} { (x)P \equiv (y)P\{y/x\} \and y \not\in \freenames{P} }
\end{mathpar}

\begin{definition}
Then two processes, $P,Q$, are alpha-equivalent if $P = Q\{\vec{y}/\vec{x}\}$ for
some $\vec{x} \in \boundnames{Q},\vec{y} \in \boundnames{P}$, where $Q\{\vec{y}/\vec{x}\}$
denotes the capture-avoiding substitution of $\vec{y}$ for $\vec{x}$ in $Q$.
\end{definition}

\begin{definition}
  The {\em structural congruence} \cite{SangiorgiWalker} , $\equiv$,
  between processes is the least congruence containing
  alpha-equivalence, satisfying the abelian monoid laws
  (associativity, commutativity and $\pzero$ as identity) for parallel
  composition $|$ and for summation $+$.
\end{definition}

\subsection{Name equivalence}

We take name equivalence, written $\nameeq$, to be the smallest
equivalence relation generated by the following rules.

\begin{mathpar}
\inferrule*[lab=Quote-drop]
{ }
{ \quotep{@{x}} \nameeq x }

\inferrule*[lab=Struct-equiv]
{ P \scong Q }
{ \quotep{P} \nameeq \quotep{Q} }
\end{mathpar}

The astute reader will have noticed that the mutual recursion of names
and processes imposes a mutual recursion on alpha-equivalence and
structural equivalence via name-equivalence. Fortunately, all of this
works out pleasantly and we may calculate in the natural way, free of
concern. The reader interested in the details is referred to the
appendix \ref{appendix:rho_details}.

\subsection{Substitution}

We use $\Proc$ for the set of processes, $\QProc$ for the set of
names, and $\id{\{}\vec{y} / \vec{x} \id{\}}$ to denote partial maps,
$s : \QProc \rightarrow \QProc$. A map, $s$ lifts, uniquely, to a map
on process terms, $\widehat{s} : \Proc \rightarrow \Proc$ by the
following equations.

\begin{mathpar}
  (0) \psubstp{Q}{P} := 0 \\
  (R \juxtap S) \psubstp{Q}{P}
  :=    
  (R)\psubstp{Q}{P} \juxtap (S) \psubstp{Q}{P} \\
  (x?(y).R) \psubstp{Q}{P}    
  :=    
  (x)\substp{Q}{P} (z)\concat( (R \psubstn{z}{y}) \psubstp{Q}{P} ) \\
  (\lift{x}{R}) \psubstp{Q}{P}  
  :=
  \lift{(x)\substp{Q}{P}}{ R \psubstp{Q}{P} } \\
%   (\dropn{x})  \psubstp{Q}{P}       
%   := 
%   \left\{ 
%     \begin{array}{ccc} 
%       \dropn{\quotep{Q}} & & x \nameeq \quotep{P} \\
%       \dropn{x} & & otherwise \\
%     \end{array}
%   \right. 
  (\dropn{x})  \psubstp{Q}{P}       
  := 
  \left\{ 
    \begin{array}{ccc} 
      Q & & x \nameeq \quotep{P} \\
      \dropn{x} & & otherwise \\
    \end{array}
  \right.
\end{mathpar}
 

where

\begin{eqnarray}
  (x)\id{\{} \lpquote Q \rpquote / \lpquote P \rpquote \id{\}}            = 
  \left\{ 
    \begin{array}{ccc}
      \lpquote Q \rpquote & & x \nameeq \lpquote P \rpquote \\
      x & & otherwise \\
    \end{array}
  \right. \nonumber
\end{eqnarray}

and $z$ is chosen distinct from $\quotep{P}$, $\quotep{Q}$, the free
names in $Q$, and all the names in $R$. Our $\alpha$-equivalence will
be built in the standard way from this substitution.

\begin{remark}\label{rem:no_self_referential_names}
  One consequence of these definitions is that $\forall P. \quotep{P}
  \not\in \freenames{P}$.
\end{remark}

\subsection{ Dynamic quote: an example }

Anticipating something of what's to come, consider applying the
substitution, $\widehat{\id{\{}u / z \id{\}}}$, to the following pair
of processes, $\lift{w}{y!(z)}$ and $w[ \lpquote y!(z) \rpquote ]$.

\begin{eqnarray}
	\lift{w}{y!(z)}\widehat{\id{\{}u / z \id{\}}}
		& = &
		\lift{w}{y!(u)} \nonumber\\
	w[ \lpquote y!(z) \rpquote ] \widehat{ \id{\{}u / z \id{\}} }
		& = &
		w[ \lpquote y!(z) \rpquote ] \nonumber
\end{eqnarray}

Because the body of the process between quotes is impervious to
substitution, we get radically different answers. In fact, by
examining the first process in an input context,
e.g. $x?(z).\lift{w}{y!(z)}$, we see that the process under the lift
operator may be shaped by prefixed inputs binding a name inside it. In
this sense, the lift operator will be seen as a way to dynamically
construct processes before reifying them as names.

Finally equipped with these standard features we can present the
dynamics of the calculus.

\subsubsection{Operational semantics} 

Finally, we introduce the computational dynamics. What marks these
algebras as distinct from other more traditionally studied algebraic
structures, e.g. vector spaces or polynomial rings, is the manner in
which dynamics is captured. In traditional structures, dynamics is typically
expressed through morphisms between such structures, as in linear maps
between vector spaces or morphisms between rings. In algebras
associated with the semantics of computation, the dynamics is
expressed as part of the algebraic structure itself, through a
reduction reduction relation typically denoted by $\red$. Below, we
give a recursive presentation of this relation for the calculus used
in the encoding.

$\red \subseteq \pi \times \pi$
$\red : \pi \to \mathcal{P}(\pi)$

\begin{mathpar}
  \inferrule* [lab=Comm] { \textsf{match}( x_{src}, x_{trgt} ) } { x_{trgt}?(y)P \; | \; x_{src}!\langle {Q} \rangle \red P\{\quotep{Q}/y}\} }
  \and \\
  \inferrule* [lab=Par] {{P} \red {P}'} {{{P} | {Q}} \red {{P}' | {Q}}}
  \and
  \inferrule* [lab=Equiv]{{{P} \scong {P}'} \andalso {{P}' \red {Q}'} \andalso {{Q}' \scong {Q}}}{{P} \red {Q}}
\end{mathpar}

\begin{eqnarray*}
  match_{\equiv} (\quotep{P},\quotep{Q}) & := & P \equiv Q \\
  match_{\dagger}(\quotep{P},\quotep{Q}) & := & \forall R. P|Q \red^{*} R => R \red^{*} 0 \\
  match_{K}(\quotep{P},\quotep{Q}) & := & K \mbox{ for some context } K
\end{eqnarray*}

$u?(x)P | u!\langle Q \rangle \red P\{\quotep{Q}/x\}$

%We write $\wred$ for $\red^*$, and $P\red$ if $\exists Q $ such that $ P \red Q$.
We write $P\red$ if $\exists Q $ such that $ P \red Q$ and $P\not\red$, otherwise.

\section{Replication}

As mentioned before, it is known that replication (and hence
recursion) can be implemented in a higher-order process algebra
\cite{SangiorgiWalker}. As our first example of calculation with the
machinery thus far presented we give the construction explicitly in
the {\rhoc}.

\begin{eqnarray}
	D_{x} & := & \prefix{x}{y}{(\binpar{\outputp{x}{y}}{@{y}})} \nonumber\\
	\bangp_{x}{P} & := & \binpar{{x}!\langle{\binpar{D_{x}}{P}}\rangle}{D_{x}} \nonumber
\end{eqnarray}

\begin{eqnarray}
	\bangp_{x}{P} & & \nonumber\\
	=
	& {x}!\langle{(\prefix{x}{y}{(\outputp{x}{y} | @{y})) | P}}\rangle 
	      | \prefix{x}{y}{(\outputp{x}{y} | @{y})} & \nonumber\\
	\red
	& (\outputp{x}{y} | @{y})\substn{\quotep{(\prefix{x}{y}{(@{y} | \outputp{x}{y})) | P}}}{y} & \nonumber\\
	=
	& \outputp{x}{\quotep{(\prefix{x}{y}{(\outputp{x}{y} | @{y})) | P}}}
	  | {(\prefix{x}{y}{(\outputp{x}{y} | @{y})) | P}} & \nonumber\\
	\red
	& \ldots & \nonumber\\
	\red^*
	& P | P | \ldots & \nonumber
\end{eqnarray}

Of course, this encoding, as an implementation, runs away, unfolding
$\bangp{P}$ eagerly. A lazier and more implementable replication
operator, restricted to input-guarded processes, may be obtained as follows.

\begin{eqnarray}
\bangp{\prefix{u}{v}{P}} 
	:= 
	\binpar{\lift{x}{\prefix{u}{v}{(\binpar{D(x)}{P})}}}{D(x)} \nonumber
\end{eqnarray}

\begin{remark}
  Note that the lazier definition still does not deal with summation
  or mixed summation (i.e. sums over input and output). The reader is
  invited to construct definitions of replication that deal with these
  features. 

  Further, the definitions are parameterized in a name, $x$. Can you,
  gentle reader, make a definition that eliminates this parameter and
  guarantees no accidental interaction between the replication
  machinery and the process being replicated -- i.e. no accidental
  sharing of names used by the process to get its work done and the
  name(s) used by the replication to effect copying. This latter
  revision of the definition of replication is crucial to obtaining
  the expected identity $!!P \sim !P$.
\end{remark}

\begin{remark}\label{rem:paradoxical_combinator}
  The reader familiar with the lambda calculus will have noticed the
  similarity between $D$ and the paradoxical combinator.

  [Ed. note: the existence of this seems to suggest we have to be more
  restrictive on the set of processes and names we admit if we are to
  support no-cloning.]
\end{remark}

\subsubsection{Bisimulation}

The computational dynamics gives rise to another kind of equivalence,
the equivalence of computational behavior. As previously mentioned
this is typically captured \emph{via} some form of bisimulation.

% The notion we use in this paper is weak barbed bisimulation
% \cite{milner91polyadicpi}.

The notion we use in this paper is derived from weak barbed
bisimulation \cite{milner91polyadicpi}. 

\begin{definition}
An \emph{observation relation}, $\downarrow_{\mathcal N}$, over a set
of names, $\mathcal N$, is the smallest relation satisfying the rules
below.

\infrule[Out-barb]{y \in {\mathcal N}, \; x \nameeq y}
		  {\outputp{x}{v} \downarrow_{\mathcal N} x}
\infrule[Par-barb]{\mbox{$P\downarrow_{\mathcal N} x$ or $Q\downarrow_{\mathcal N} x$}}
		  {\binpar{P}{Q} \downarrow_{\mathcal N} x}

We write $P \Downarrow_{\mathcal N} x$ if there is $Q$ such that 
$P \wred Q$ and $Q \downarrow_{\mathcal N} x$.
\end{definition}

\begin{definition}
%\label{def.bbisim}
An  ${\mathcal N}$-\emph{barbed bisimulation} over a set of names, ${\mathcal N}$, is a symmetric binary relation 
${\mathcal S}_{\mathcal N}$ between agents such that $P\rel{S}_{\mathcal N}Q$ implies:
\begin{enumerate}
\item If $P \red P'$ then $Q \wred Q'$ and $P'\rel{S}_{\mathcal N} Q'$.
\item If $P\downarrow_{\mathcal N} x$, then $Q\Downarrow_{\mathcal N} x$.
\end{enumerate}
$P$ is ${\mathcal N}$-barbed bisimilar to $Q$, written
$P \wbbisim_{\mathcal N} Q$, if $P \rel{S}_{\mathcal N} Q$ for some ${\mathcal N}$-barbed bisimulation ${\mathcal S}_{\mathcal N}$.
\end{definition}

$\mathcal{R} \subseteq \pi \times \pi$

$P \mathcal{R} Q => \forall P'. P \red P' \Rightarrow \exists Q'. Q \red Q', P' \mathcal{R} Q'$

$P \vdash x \Rightarrow Q \vdash x$

\begin{mathpar}
  \inferrule*[lab=Out-barb]{x \nameeq y}{{y}!\langle{Q}\rangle \vdash x}
  \and
  \inferrule*[lab=Par-barb]{\mbox{$P\vdash x$ or $Q\vdash x$}}{\binpar{P}{Q} \vdash x}
\end{mathpar}

\subsubsection{Contexts}

One of the principle advantages of computational calculi like the
$\pi$-calculus is a well-defined notion of context,
contextual-equivalence and a correlation between
contextual-equivalence and notions of bisimulation. The notion of
context allows the decomposition of a process into (sub-)process and
its syntactic environment, its context. Thus, a context may be
thought of as a process with a ``hole'' (written $\Box$) in it. The
application of a context $M$ to a process $P$, written $M[P]$, is
tantamount to filling the hole in $M$ with $P$. In this paper we do
not need the full weight of this theory, but do make use of the notion
of context in the proof the main theorem. 

\begin{mathpar}
  \inferrule* [lab=summation] {} {{M_{M},M_{N}} \bc \Box \;|\; x.M_{A} \;|\; M_{M}+M_{N}}
  \and
  \inferrule* [lab=agent] {} {{M_{A}} \bc (\vec{x})M_{P} \;| \; \clift{P_0,\ldots,M_{P},\ldots,P_N}}
  \and \\
  \inferrule* [lab=process] {} {{M_{P}} \bc M_{N} \;| \;P|M_{P} }
\end{mathpar} 

\begin{mathpar}
  \inferrule* [lab=sychronization] {} {M_{N} \bc \Box \;|\; x?M_{F} \;|\; x!M_{C}}
  \and
  \inferrule* [lab=abstraction] {} {{M_{F}} \bc (x)M_{P} }
  \and
  \inferrule* [lab=concretion] {} {{M_{C}} \bc \langle M_{P} \rangle }
  \and \\
  \inferrule* [lab=process] {} {{M_{P}} \bc M_{N} \;| \;P|M_{P} }
\end{mathpar}

\begin{definition}[contextual application] Given a context $M$, and
  process $P$, we define the \emph{contextual application}, $M[P] :=
  M\{P/\Box\}$. That is, the contextual application of M to P is the
  substitution of $P$ for $\Box$ in $M$.
\end{definition}

$\meaningof{-} : L \to \mathcal{P}(\pi)$

\begin{mathpar}
  \inferrule* [lab=collection] {} {\meaningof{true} = \pi, \and \meaningof{~E} = \pi \setminus \meaningof{E}, \and \meaningof{E_{1} \& E_{2}} = \meaningof{E_{1}} \cap \meaningof{E_{2}}}
\end{mathpar}

\begin{mathpar}
  \inferrule* [lab=structure] {} {\meaningof{0} = \{ P \in \pi | P \equiv 0 \}, \and \\ \meaningof{E_1 | E_2} = \{ P \in \pi | P \equiv P_{1} | P_{2}, P_{1} \in \meaningof{E_{1}}, P_{2} \in \meaningof{E_2}\} }
\end{mathpar}

\begin{mathpar}
 \inferrule* [lab=behavior] {} {\meaningof{\langle a?b \rangle E} = \{ P \in \pi | P \equiv Q | u?(y)P', \\ \and \\\\ \and \\ \;\;\; u \in \meaningof{a}, \forall z.P'\{z/y\} \in \meaningof{E\{z/b\}}\}, \and \\ \meaningof{a!E} = \{ P \in \pi | P \equiv Q | x!\langle P' \rangle, x \in \meaningof{a} P' \in \meaningof{E}\} }
\end{mathpar}

\begin{mathpar}
 \inferrule* [lab=nominal] {} {\meaningof{\quotep{E}} = \{ \quotep{P} \in \quotep{\pi} | P \in \meaningof{E} \}, \and \meaningof{\quotep{P}} = \{ \quotep{Q} \in \quotep{\pi} | P \equiv Q \} \and \\ \meaningof{@\quotep{E}} = \{ P \in \pi | P \equiv @x, x \in \meaningof{E} \}}
\end{mathpar}

\begin{eqnarray*}
  \\
  \meaningof{-} : TS \to ST
\end{eqnarray*}

\begin{eqnarray*}
  \\
  L : TS \to ST
\end{eqnarray*}

\begin{eqnarray*}
  \\
  P \models E \iff P \in \meaningof{E}
\end{eqnarray*}

\begin{eqnarray*}
  P \approx_{L} Q \iff \forall E \in L. P \models E \iff Q \models E
\end{eqnarray*}

\begin{eqnarray*}
  P \approx_{K} Q
\end{eqnarray*}

\begin{eqnarray*}
  P \approx Q
\end{eqnarray*}

$\approx_{K} = \approx = \approx_{L}$

\subsubsection{Contextual duality}

Note that contexts extend the quotation operation to a family of
operations from processes to names. Given a context, $M$, we can
define a \emph{nominal context}, $\quotep{M}$ by $\quotep{M}[P] :=
\quotep{M[P]}$. To foreshadow what is to come we observe that these
operations enjoy a duality with processes very much like the duality
between vectors and maps from vectors to scalars.

Further, because the calculus is essentially higher-order, we have a
correspondence between contexts and processes. More specifically,
given a name $x$ and a context $M$ we can construct $M^{*}_{x}$ such
that 

\begin{mathpar}
  M^{*}_{x} | \lift{x}{P} \red M[P]
\end{mathpar}

namely,

\begin{mathpar}
  M^{*}_{x} := x?(u).M[\dropn{u}]
\end{mathpar}

The dependence of $M^{*}_{x}$ on a name makes it an abstraction, 

\begin{mathpar}
  M^{*} := (x)x?(u).M[\dropn{u}]
\end{mathpar}

\subsection{Additional notation}

It will sometimes be convenient to denote the process a name
quotes. We already have the notation $x = \quotep{P}$, but it will be
convenient to introduce an alternate notation, $\procn{x}$, when we
want to emphasize the connection to the use of the name. Note that, by
virtue of name equivalence, $\quotep{\procn{x}} \nameeq x$; so, the
notation is consistent with previous definitions.

Further, because names have structure it is possible to effect
substitutions on the basis of that structure. This means we need to
upgrade our notation for substitutions, which we accomplish by
adapting comprehension notation. Thus,

\begin{mathpar}
  P\{ y / x : x \in S \}
\end{mathpar}

is interpreted to mean the process derived from P by replacing (in a
capture-avoiding manner) each occurrence of $x$ in $S$ by $y$. For example,

\begin{mathpar}
  P\{ \quotep{\procn{x}|\procn{x}} / x : x \in \freenames{P} \}
\end{mathpar}

will replace each (occurrence) of a free name $x$ in $P$ by
$\quotep{\procn{x}|\procn{x}}$.

Also, we will avail ourselves of the notation $x^{L}$ and $x^{R}$ to
denote injections of a name into disjoint copies of the name
space. There are numerous ways to accomplish this. One example can be
found in \cite{MeredithR05}. This notation overloads to vectors of
names: $\vec{x}^{\pi} := (x_{i}^{\pi} \; : \; 0 \leq i < |\vec{x}| )$ where $\pi \in \{L,R\}$.

We also use $P^{\Box} := P|\Box$.

In \cite{MeredithR05} an interpretation of the new operator is
given. It turns out that there are several possible interpretations
all enjoying the requisite algebraic properties of the operator (see
\cite{milner91polyadicpi}). We will therefore make liberal use of
$(\nu\; \vec{x})P$.

% subsection the_syntax_and_semantics_of_the_notation_system (end)   

\input{qm2pi.qmops} 

\input{qm2pi.sterngerlach} 

\input{qm2pi.metric} 

% section concurrent_process_calculi (end)

%\input{qm2pi.proofsketch}

% section proof sketch (end)

%\input{qm2pi.slviaknots} 

% section spatial logic via knots (end)

\input{qm2pi.conclusion}

% section conclusion (end)

%\input{qm2pi.dtcodes} 

% section wiring algorithm (end)

\input{qm2pi.ack} 

% section acknowledgments (end)

\newpage


\bibliographystyle{plain}   
\bibliography{../../biblios/main.bib}

\input{qm2pi.rhodetails}

\end{document}



% section proof sketch (end)

%\section{Unlikely characters: spatial logic for
  knots}\label{sub:characteristic_formulae} % (fold)

Associated to the mobile process calculi are a family of logics known
as the Hennessy-Milner logics. These logics typically enjoy a
semantics interpreting formulae as sets of processes that when
factored through the encoding outlined above allows an identification
of classes of knots with logical formulae. In the context of this
encoding the sub-family known as the spatial logics \cite{CairesC03}
\cite{CairesC04} \cite{Caires04} are of particular interest providing
several important features for expressing and reasoning about
properties (i.e. classes) of knots. We hint here at how this may be done.

%\begin{description}
%\item [structural connectives] 
\subsubsection{Structural connectives} The spatial logics enjoy
structural connectives corresponding, at the logical level, to the
parallel composition ($P | Q$) and new name ($(\nu \; x)P$)
connectives for processes. As illustrated in the examples below, these
connectives are extremely expressive given the shape of our encoding.
%\item [decideable satisfaction]

\subsubsection{Decideable satisfaction}
In \cite{Caires04} the satisfaction relation is shown to be decideable
for a rich class of processes. It further turns out that the image of
the our encoding is a proper subset of that class. This result
provides the basis for an algorithm by which to search for knots
enjoying a given property.
%\item [characteristic formulae]

\subsubsection{Characteristic formulae}
In the same paper \cite{Caires04} , Caires presents a means of calculating
characteristic formulae, selecting equivalence classes of processes
up to a pre--specified depth limit on the support set of names. Composed with our
encoding, this characteristic formula can be used to select
characteristic formulae for knots.
%\end{description}

\subsubsection{Spatial logic formulae}

The grammar below (segmented for comprehension) summarizes the syntax
of spatial logic formulae. We employ illustrative examples in the
sequel to provide an intuitive understanding of their meaning
referring the reader to \cite{Caires04} for a more detailed explication
of the semantics.

\begin{mathpar}
  \inferrule* [lab=boolean] {} {{A,B} \bc T \;|\; \neg A \;|\; A \wedge B \;|\; \eta = \eta'}
  \and
  \inferrule* [lab=spatial] {} {|\; \pzero \;|\; A | B \;|\; x \text{\textregistered} A \;|\; \forall x . A \;|\;  H x . A}
  \and
  \inferrule* [lab=behavioral] {} {|\; \alpha . A}
  \and 
  \inferrule* [lab=recursion] {} {|\; X(\vec{u}) \;|\; \mu X(\vec{u}) . A}
  \and
  \inferrule* [lab=action] {} {\alpha \bc \langle x?(\vec{y}) \rangle \;|\; \langle x!(\vec{y}) \rangle \;|\; \langle \tau \rangle}
  \and 
  \inferrule* [lab=name] {} {\eta \bc x \;|\; \tau}
\end{mathpar} 

% subsection characteristic_formulae (end)   	 

\subsection{Example formulae}\label{sub:example_formulae_} % (fold)

\subsubsection{Crossing as formula.}
% 
% \begin{align*}
%   \frac{d}{dx} \sin x &= \cos x 
%   & \frac{d}{dx} e^x &= e^x \\
%   \frac{d}{dx} \cos x &= - \sin x 
%   & \frac{d}{dx} \log x &= \frac{1}{x} \\
% \end{align*} 

\begin{align*}
 \mu C(x_{0},x_{1},y_{0},y_{1},u).&(\langle x_{0}?(z) \rangle(\langle u! \rangle\langle y_{1}!z \rangle C(x_{0},x_{1},y_{0},y_{1},u)) & \\
  & \wedge \langle y_{1}?(z) \rangle (\langle u! \rangle \langle x_{0}!z \rangle C(x_{0},x_{1},y_{0},y_{1},u)) & \\
  & \wedge \langle x_{1}?(z) \rangle (\langle u? \rangle \langle y_{0}!z \rangle C(x_{0},x_{1},y_{0},y_{1},u)) & \\
  & \wedge \langle y_{0}?(z) \rangle (\langle u? \rangle \langle x_{1}!z \rangle C(x_{0},x_{1},y_{0},y_{1},u))) &
\end{align*}

The lexicographical similarity between the shape of this formulae and
the shape of definition of the process representing a crossing reveals
the intuitive meaning of this formulae. It describes the capabilities
of a process that has the right to represent a crossing. For example
it picks out processes that may perform an input on the port $x_0$ in
its initial menu of capabilities. What differentiates the formula
from the process, however, is that the crossing process is the
smallest candidate to satisfy the formula. Infinitely many other
processes -- with internal behavior hidden behind this interface, so
to speak -- also satisfy this formula. Even this simple formula,
then, can be seen to open a new view onto knots, providing a
computational interpretation of \emph{virtual} knots.

Note that this formula is derived by hand. A similar formula can be
derived by employing Caires' calculation of characteristic formula
\cite{Caires04} to the process representing a crossing. In light of
this discussion, we let
$\meaningof{C}_{\phi}(x0,x1,y0,y1,u)$ denote a formula specifying the
dynamics we wish to capture of a crossing. To guarantee we preserve
the shape of the interface and minimal semantics we demand that
$\meaningof{C}_{\phi}(x0,x1,y0,y1,u) \Rightarrow
\textbf{C}(x0,x1,y0,y1,u)$ where $\textbf{C}(x0,x1,y0,y1,u)$ denotes
the formula above.
                            
\subsubsection{Crossing number constraints.}
The moral content of the context lemma (Lemma \ref{context}) is that the notion of
``locality'' in the Reidemeister moves is effectively captured by the
parallel composition operator of the process calculus. This intuition
extends through the logic. Given a formula,
$\meaningof{C}_{\phi}(x0,x1,y0,y1,u)$, we can use the structural
connectives to specify constraints on crossing numbers, such as at
least $n$ crossings, or exactly $n$ crossings.
\begin{mathpar}
  \inferrule* [lab=at-least-n] {} { K^{\geq n}_{\phi}(\vec{xs},\vec{ys}) := \Pi_{i=0}^{n-1} Hu . \meaningof{C}_{\phi}(xs_i,ys_i,u) | T }
  \and 
  \inferrule* [lab=exactly-n] {} { K^{= n}_{\phi}(\vec{xs},\vec{ys}) := \Pi_{i=0}^{n-1} Hu . \meaningof{C}_{\phi}(xs_i,ys_i,u) | \neg (\forall x_0,y_0,x_1,y_1,u . \meaningof{C}_{\phi}(x_0,y_0,x_1,y_1,u) | T) }
\end{mathpar}

To round out this section, recall that the encoding of an $n$-crossing
knot decomposes into a parallel composition of $n$ \emph{copies} of a
crossing process together with a wiring harness. To specify different
knot classes with the same crossing number amounts to specifying
logical constraints on the wiring harness. In the interest of space,
we defer examples to a forthcoming paper. Suffice it to say that both
the conditions ``alternating knot'' and ``contains the tangle
corresponding to 5/3'' are expressible. For example, it is possible to
calculate the characteristic formula of a process corresponding to the
tangle 5/3 and conjoin it into the classifying formula via the
composition connective of the logic.

Finally, we wish to observe that it is entirely within reason to
contemplate a more domain-specific version of spatial logic tailored
to the shape of processes in the image of the encoding. Such a
domain-specific logic would have a better claim to the title formal
language of knot properties.

% subsection example_formulae_ (end)

% section knots_as_processes (end) 

% section spatial logic via knots (end)

\section{Conclusions and future work}

\paragraph{Testing physical space}
You, gentle reader, may wonder why of all the theorems to be proved
given this set up we pick the one above. In some sense it's hardly
central to quantum mechanics. We see it as central in the sense that
it firmly establishes a notion of physical space arising from a notion
of the equivalence of behavior. Relating bisimulation to a metric is a
big step forward, but one is faced with interpreting the relationship
of that metric space to something more physical. Quantum mechanical
notions of ``physical'' space are still far from intuitive, but by
relating this idea of distance as testing to calculations that predict
physical circumstances we are making a not insignificant step forward
toward an understanding of the physical space we inhabit as
essentially dynamic.

\paragraph{Effectivity and simulation}
One of the observations we have yet to make is that the entire program
spelled out here is effective. We have built various interpreters for
the reflective calculus at work in this interpretation. In principle,
then, we can simulate quantum mechanics on a computer. The place where
the simulation may lose fidelity is the infinitely branching summation
for the annihilator.

In this connection i also want to point out that the evaluation style
calculation of the inner product puts the non-determinism of the
summation right at the heart of measurement. This suggests that
Milner's original reduction-based formulation of the dynamics of his
calculi in terms of sums was not just notationally suggestive of a
notion of measure-and-continue but captured some significant part of
the physics.

\paragraph{Quantum continuations}
In light of this last observation i want to point out that the
predominant account of quantum mechanics is missing a key aspect of a
truly compositional story of the physical situation. In a real lab,
when a measurement is made the observation can be made to feed into
another device that then makes another measurement conditioned on the
results of the first. This means that after the superposition was
collapsed the entire experimental set up remained in
superposition. While QM offers a means of writing this down it doesn't
quite line up well with the well-trodden formulation of computation
and continuation that we see so succinctly expressed in Milner's
calculi. This suggests that there might be advantages to this account
of dynamics waiting to be explored.

\paragraph{Quantum logic}
In this connection, we also note that by virtue of having the
Hennessy-Milner construction, we can pull the construction through the
interpretation of QM. This gives us a natural candidate for a quantum
logic that enjoys an extremely tight connection with it's domain of
interpretation, making the construction much less ad hoc (rather it is
the image of functor!).

\paragraph{Quantum probabiity}
i have questions about the basis of the interpretation of inner
product as probability amplitude. In particular, using which
axiomatization of probability theory does the notion of probability
amplitude earn the right to be so dubbed? In other words, where is the
proof that the operation for calculating a probability amplitude (and
then squaring) satisfies the axioms of what it means to calculate a
probability? Even if such a proof exists (i have yet to find it in the
literature), i wonder if it might not be possible to turn things on
their heads. Can we view the calculation of the probability amplitude
as an axiomatization of probability? If so, then the definition we
give for calculating probability amplitude may provide the basis for
an \emph{effective} theory of probability.

\paragraph{Quantum vs ``biological'' information}
Finally, i want to conclude with a more philosophical observation. At
a recent workshop in which QM was a predominant topic i noticed
something about quantum information. The speaker was giving a riveting
discussion of axiomatic QM and showing how properties of ``no
cloning'' and ``no deleting'' emerged as consequences of the
axiomatization. Theorems of this form are necessary to give us a sense
of confidence that our axioms characterize the physical theory. What
struck me, though, was that if quantum information is neither erasable
nor replicable it is markedly different from \emph{life}. Two of the
things we know about life is that

\begin{itemize}
  \item it ends;
  \item to gain some measure of persistence, to transcend it's
    finitude it is imminently copyable.
\end{itemize}

Both of these qualities are summarized succinctly in the aphorism: all
flesh is grass. For me these two kinds of ``information'' -- call them
quantum and biological -- are end points on a spectrum of strategies
for persistence. At one end, we have those curious entities that enjoy
uniqueness and permanence; at the other, we have those who in the face
of a certain end and an uncertain present make a go of passing
something on. To me one of the more remarkable aspects of the latter
strategy is that in the presence of noise (and certain features of
copying) we get a kind of dynamism, a chance for improvement against a
given persistent condition.

% subsection other_calculi_other_bisimulations_and_geometry_as_behavior (end)




% section conclusion (end)

%\documentclass[12pt]{llncs}
%\documentclass{jktr}

\usepackage[pdftex]{hyperref}                   
\usepackage {listings}
\usepackage {mathpartir}
\usepackage{bcprules}
%\usepackage{listings}
                       
\usepackage{graphicx} 
%\usepackage[margins=2.5cm,nohead,nofoot]{geometry}
%\usepackage{geometry}
\usepackage{amsfonts}
\usepackage{amstext}
\usepackage{latexsym}
\usepackage{amssymb}
\usepackage{color}


%\include{myPreamble}
\include{qm2pi.local} 

%\ifpdf
%\usepackage[pdftex]{graphicx}
%\else
%\usepackage{graphicx}
%\fi

 % \ifpdf
%  \usepackage{pdfsync}
%  \if


%\title{Brief Article}
%\author{David F. Snyder}
%\author{L.G. Meredith}

%\address{Dept. of Math., Texas State University--San Marcos, San Marcos, TX 78666}
       
\pagestyle{empty}


\begin{document}

\lstset{language=[Objective]Caml,frame=shadowbox}

\input{qm2pi.front}

% section front matter (end)

\input{qm2pi.intro} 
 
% section introduction (end)

% \input{qm2pi.knotations} 

% section notation (end)

\input{qm2pi.process.calculi} 

% section concurrent_process_calculi_and_spatial_logics_ (end)
    
%\input{qm2pi.knots2pi} 

%\input{qm2pi.trefoil} 

%\input{qm2pi.mainthm} 

% subsection basic_interpretation (end)

%\input{qm2pi.rho.presentation} 
\subsection{The syntax and semantics of the notation system}\label{sub:the_syntax_and_semantics_of_the_notation_system} % (fold)

We now summarize a technical presentation of the calculus that
embodies our theory of dynamics. The typical presentation of such a
calculus follows the style of giving generators and relations on
them. The grammar, below, describing term constructors, freely
generates the set of processes, $\Proc$. This set is then quotiented
by a relation known as structural congruence and it is over this set
that the notion of dynamics is expressed. This presentation is
essentially that of \cite{MeredithR05} with the addition of
polyadicity and summation. For readability we have relegated some of
the technical subtleties to an appendix.

\subsubsection{Process grammar}\label{subsub:process_grammar}

\begin{mathpar}
  \inferrule* [lab=synchronization] {} {{M} \bc \pzero \;|\; x?F \;|\; x!C }
  \and
  \inferrule* [lab=abstraction] {} {{F} \bc (x)P}
  \and
  \inferrule* [lab=concretion] {} {{C} \bc \langle Q \rangle}
  \and
  \inferrule* [lab=process] {} {{P,Q} \bc M \;| \;P|Q \;|\; @{x}}
  \and
  \inferrule* [lab=name] {} {{x} \bc \quotep{P}}
\end{mathpar} 

Note that $\vec{x}$ (resp. $\vec{P}$) denotes a vector of names
(resp. processes) of length $|\vec{x}|$ (resp. $|\vec{P}|$). We adopt
the following useful abbreviations.

\begin{mathpar}
   x?(\vec{y}).P := x.(\vec{y})P \and  x\clift{\vec{P}} := x.\clift{\vec{P}}
   \and x!(y) := \lift{x}{\dropn{y}}
   \and \Pi_{i=0}^{n-1}P_i := P_0 | \ldots | P_{n-1}
\end{mathpar}

\subsubsection{Structural congruence}

\paragraph{Free and bound names and alpha-equivalence.} At the
core of structural equivalence is alpha-equivalence which identifies
process that are the same up to a change of variable. Formally, we
recognize the distinction between free and bound names. The free names
of a process, $\freenames{P}$, may be calculated recursively as
follows:

\begin{mathpar}
\freenames{\pzero} := \emptyset
  \and \\
  \freenames{x?(y).P} := \{ x \} \cup (\freenames{P} \setminus \{ y \})
  \and 
  \freenames{x!\langle P \rangle} := \{ x \} \cup \{ P \} 
  \and \\
  \freenames{P|Q} := \freenames{P} \cup \freenames{Q}
  \and \\
  \freenames{@{x}} := \{ x \}
\end{mathpar}

$\pi$
$\quotep{\pi}$

$\freenames{-} : \pi \to \mathcal{P}(\quotep{\pi})$

\begin{eqnarray*}
  \freenames{\pzero} & := & \emptyset \\
  \freenames{x?(y).P} & := & \{ x \} \cup (\freenames{P} \setminus \{ y \}) \\
  \freenames{x!\langle P \rangle} & := & \{ x \} \cup \{ P \} \\
  \freenames{P|Q} & := & \freenames{P} \cup \freenames{Q} \\
  \freenames{\dropn{x}} & := & \{ x \}
\end{eqnarray*}

The bound names of a process, $\boundnames{P}$, are those names occurring in $P$
that are not free. For example, in $x?(y).0$, the name $x$ is free, while $y$ is bound.

\begin{mathpar}
  \inferrule* [lab=monoidal-laws] {} { P|Q \equiv Q|P \and P|0 \equiv P \and P|(Q|R) \equiv (P|Q)|R }
\end{mathpar}

\begin{mathpar}
  \inferrule* [lab=alpha-equivalence] {} { (x)P \equiv (y)P\{y/x\} \and y \not\in \freenames{P} }
\end{mathpar}

\begin{definition}
Then two processes, $P,Q$, are alpha-equivalent if $P = Q\{\vec{y}/\vec{x}\}$ for
some $\vec{x} \in \boundnames{Q},\vec{y} \in \boundnames{P}$, where $Q\{\vec{y}/\vec{x}\}$
denotes the capture-avoiding substitution of $\vec{y}$ for $\vec{x}$ in $Q$.
\end{definition}

\begin{definition}
  The {\em structural congruence} \cite{SangiorgiWalker} , $\equiv$,
  between processes is the least congruence containing
  alpha-equivalence, satisfying the abelian monoid laws
  (associativity, commutativity and $\pzero$ as identity) for parallel
  composition $|$ and for summation $+$.
\end{definition}

\subsection{Name equivalence}

We take name equivalence, written $\nameeq$, to be the smallest
equivalence relation generated by the following rules.

\begin{mathpar}
\inferrule*[lab=Quote-drop]
{ }
{ \quotep{@{x}} \nameeq x }

\inferrule*[lab=Struct-equiv]
{ P \scong Q }
{ \quotep{P} \nameeq \quotep{Q} }
\end{mathpar}

The astute reader will have noticed that the mutual recursion of names
and processes imposes a mutual recursion on alpha-equivalence and
structural equivalence via name-equivalence. Fortunately, all of this
works out pleasantly and we may calculate in the natural way, free of
concern. The reader interested in the details is referred to the
appendix \ref{appendix:rho_details}.

\subsection{Substitution}

We use $\Proc$ for the set of processes, $\QProc$ for the set of
names, and $\id{\{}\vec{y} / \vec{x} \id{\}}$ to denote partial maps,
$s : \QProc \rightarrow \QProc$. A map, $s$ lifts, uniquely, to a map
on process terms, $\widehat{s} : \Proc \rightarrow \Proc$ by the
following equations.

\begin{mathpar}
  (0) \psubstp{Q}{P} := 0 \\
  (R \juxtap S) \psubstp{Q}{P}
  :=    
  (R)\psubstp{Q}{P} \juxtap (S) \psubstp{Q}{P} \\
  (x?(y).R) \psubstp{Q}{P}    
  :=    
  (x)\substp{Q}{P} (z)\concat( (R \psubstn{z}{y}) \psubstp{Q}{P} ) \\
  (\lift{x}{R}) \psubstp{Q}{P}  
  :=
  \lift{(x)\substp{Q}{P}}{ R \psubstp{Q}{P} } \\
%   (\dropn{x})  \psubstp{Q}{P}       
%   := 
%   \left\{ 
%     \begin{array}{ccc} 
%       \dropn{\quotep{Q}} & & x \nameeq \quotep{P} \\
%       \dropn{x} & & otherwise \\
%     \end{array}
%   \right. 
  (\dropn{x})  \psubstp{Q}{P}       
  := 
  \left\{ 
    \begin{array}{ccc} 
      Q & & x \nameeq \quotep{P} \\
      \dropn{x} & & otherwise \\
    \end{array}
  \right.
\end{mathpar}
 

where

\begin{eqnarray}
  (x)\id{\{} \lpquote Q \rpquote / \lpquote P \rpquote \id{\}}            = 
  \left\{ 
    \begin{array}{ccc}
      \lpquote Q \rpquote & & x \nameeq \lpquote P \rpquote \\
      x & & otherwise \\
    \end{array}
  \right. \nonumber
\end{eqnarray}

and $z$ is chosen distinct from $\quotep{P}$, $\quotep{Q}$, the free
names in $Q$, and all the names in $R$. Our $\alpha$-equivalence will
be built in the standard way from this substitution.

\begin{remark}\label{rem:no_self_referential_names}
  One consequence of these definitions is that $\forall P. \quotep{P}
  \not\in \freenames{P}$.
\end{remark}

\subsection{ Dynamic quote: an example }

Anticipating something of what's to come, consider applying the
substitution, $\widehat{\id{\{}u / z \id{\}}}$, to the following pair
of processes, $\lift{w}{y!(z)}$ and $w[ \lpquote y!(z) \rpquote ]$.

\begin{eqnarray}
	\lift{w}{y!(z)}\widehat{\id{\{}u / z \id{\}}}
		& = &
		\lift{w}{y!(u)} \nonumber\\
	w[ \lpquote y!(z) \rpquote ] \widehat{ \id{\{}u / z \id{\}} }
		& = &
		w[ \lpquote y!(z) \rpquote ] \nonumber
\end{eqnarray}

Because the body of the process between quotes is impervious to
substitution, we get radically different answers. In fact, by
examining the first process in an input context,
e.g. $x?(z).\lift{w}{y!(z)}$, we see that the process under the lift
operator may be shaped by prefixed inputs binding a name inside it. In
this sense, the lift operator will be seen as a way to dynamically
construct processes before reifying them as names.

Finally equipped with these standard features we can present the
dynamics of the calculus.

\subsubsection{Operational semantics} 

Finally, we introduce the computational dynamics. What marks these
algebras as distinct from other more traditionally studied algebraic
structures, e.g. vector spaces or polynomial rings, is the manner in
which dynamics is captured. In traditional structures, dynamics is typically
expressed through morphisms between such structures, as in linear maps
between vector spaces or morphisms between rings. In algebras
associated with the semantics of computation, the dynamics is
expressed as part of the algebraic structure itself, through a
reduction reduction relation typically denoted by $\red$. Below, we
give a recursive presentation of this relation for the calculus used
in the encoding.

$\red \subseteq \pi \times \pi$
$\red : \pi \to \mathcal{P}(\pi)$

\begin{mathpar}
  \inferrule* [lab=Comm] { \textsf{match}( x_{src}, x_{trgt} ) } { x_{trgt}?(y)P \; | \; x_{src}!\langle {Q} \rangle \red P\{\quotep{Q}/y}\} }
  \and \\
  \inferrule* [lab=Par] {{P} \red {P}'} {{{P} | {Q}} \red {{P}' | {Q}}}
  \and
  \inferrule* [lab=Equiv]{{{P} \scong {P}'} \andalso {{P}' \red {Q}'} \andalso {{Q}' \scong {Q}}}{{P} \red {Q}}
\end{mathpar}

\begin{eqnarray*}
  match_{\equiv} (\quotep{P},\quotep{Q}) & := & P \equiv Q \\
  match_{\dagger}(\quotep{P},\quotep{Q}) & := & \forall R. P|Q \red^{*} R => R \red^{*} 0 \\
  match_{K}(\quotep{P},\quotep{Q}) & := & K \mbox{ for some context } K
\end{eqnarray*}

$u?(x)P | u!\langle Q \rangle \red P\{\quotep{Q}/x\}$

%We write $\wred$ for $\red^*$, and $P\red$ if $\exists Q $ such that $ P \red Q$.
We write $P\red$ if $\exists Q $ such that $ P \red Q$ and $P\not\red$, otherwise.

\section{Replication}

As mentioned before, it is known that replication (and hence
recursion) can be implemented in a higher-order process algebra
\cite{SangiorgiWalker}. As our first example of calculation with the
machinery thus far presented we give the construction explicitly in
the {\rhoc}.

\begin{eqnarray}
	D_{x} & := & \prefix{x}{y}{(\binpar{\outputp{x}{y}}{@{y}})} \nonumber\\
	\bangp_{x}{P} & := & \binpar{{x}!\langle{\binpar{D_{x}}{P}}\rangle}{D_{x}} \nonumber
\end{eqnarray}

\begin{eqnarray}
	\bangp_{x}{P} & & \nonumber\\
	=
	& {x}!\langle{(\prefix{x}{y}{(\outputp{x}{y} | @{y})) | P}}\rangle 
	      | \prefix{x}{y}{(\outputp{x}{y} | @{y})} & \nonumber\\
	\red
	& (\outputp{x}{y} | @{y})\substn{\quotep{(\prefix{x}{y}{(@{y} | \outputp{x}{y})) | P}}}{y} & \nonumber\\
	=
	& \outputp{x}{\quotep{(\prefix{x}{y}{(\outputp{x}{y} | @{y})) | P}}}
	  | {(\prefix{x}{y}{(\outputp{x}{y} | @{y})) | P}} & \nonumber\\
	\red
	& \ldots & \nonumber\\
	\red^*
	& P | P | \ldots & \nonumber
\end{eqnarray}

Of course, this encoding, as an implementation, runs away, unfolding
$\bangp{P}$ eagerly. A lazier and more implementable replication
operator, restricted to input-guarded processes, may be obtained as follows.

\begin{eqnarray}
\bangp{\prefix{u}{v}{P}} 
	:= 
	\binpar{\lift{x}{\prefix{u}{v}{(\binpar{D(x)}{P})}}}{D(x)} \nonumber
\end{eqnarray}

\begin{remark}
  Note that the lazier definition still does not deal with summation
  or mixed summation (i.e. sums over input and output). The reader is
  invited to construct definitions of replication that deal with these
  features. 

  Further, the definitions are parameterized in a name, $x$. Can you,
  gentle reader, make a definition that eliminates this parameter and
  guarantees no accidental interaction between the replication
  machinery and the process being replicated -- i.e. no accidental
  sharing of names used by the process to get its work done and the
  name(s) used by the replication to effect copying. This latter
  revision of the definition of replication is crucial to obtaining
  the expected identity $!!P \sim !P$.
\end{remark}

\begin{remark}\label{rem:paradoxical_combinator}
  The reader familiar with the lambda calculus will have noticed the
  similarity between $D$ and the paradoxical combinator.

  [Ed. note: the existence of this seems to suggest we have to be more
  restrictive on the set of processes and names we admit if we are to
  support no-cloning.]
\end{remark}

\subsubsection{Bisimulation}

The computational dynamics gives rise to another kind of equivalence,
the equivalence of computational behavior. As previously mentioned
this is typically captured \emph{via} some form of bisimulation.

% The notion we use in this paper is weak barbed bisimulation
% \cite{milner91polyadicpi}.

The notion we use in this paper is derived from weak barbed
bisimulation \cite{milner91polyadicpi}. 

\begin{definition}
An \emph{observation relation}, $\downarrow_{\mathcal N}$, over a set
of names, $\mathcal N$, is the smallest relation satisfying the rules
below.

\infrule[Out-barb]{y \in {\mathcal N}, \; x \nameeq y}
		  {\outputp{x}{v} \downarrow_{\mathcal N} x}
\infrule[Par-barb]{\mbox{$P\downarrow_{\mathcal N} x$ or $Q\downarrow_{\mathcal N} x$}}
		  {\binpar{P}{Q} \downarrow_{\mathcal N} x}

We write $P \Downarrow_{\mathcal N} x$ if there is $Q$ such that 
$P \wred Q$ and $Q \downarrow_{\mathcal N} x$.
\end{definition}

\begin{definition}
%\label{def.bbisim}
An  ${\mathcal N}$-\emph{barbed bisimulation} over a set of names, ${\mathcal N}$, is a symmetric binary relation 
${\mathcal S}_{\mathcal N}$ between agents such that $P\rel{S}_{\mathcal N}Q$ implies:
\begin{enumerate}
\item If $P \red P'$ then $Q \wred Q'$ and $P'\rel{S}_{\mathcal N} Q'$.
\item If $P\downarrow_{\mathcal N} x$, then $Q\Downarrow_{\mathcal N} x$.
\end{enumerate}
$P$ is ${\mathcal N}$-barbed bisimilar to $Q$, written
$P \wbbisim_{\mathcal N} Q$, if $P \rel{S}_{\mathcal N} Q$ for some ${\mathcal N}$-barbed bisimulation ${\mathcal S}_{\mathcal N}$.
\end{definition}

$\mathcal{R} \subseteq \pi \times \pi$

$P \mathcal{R} Q => \forall P'. P \red P' \Rightarrow \exists Q'. Q \red Q', P' \mathcal{R} Q'$

$P \vdash x \Rightarrow Q \vdash x$

\begin{mathpar}
  \inferrule*[lab=Out-barb]{x \nameeq y}{{y}!\langle{Q}\rangle \vdash x}
  \and
  \inferrule*[lab=Par-barb]{\mbox{$P\vdash x$ or $Q\vdash x$}}{\binpar{P}{Q} \vdash x}
\end{mathpar}

\subsubsection{Contexts}

One of the principle advantages of computational calculi like the
$\pi$-calculus is a well-defined notion of context,
contextual-equivalence and a correlation between
contextual-equivalence and notions of bisimulation. The notion of
context allows the decomposition of a process into (sub-)process and
its syntactic environment, its context. Thus, a context may be
thought of as a process with a ``hole'' (written $\Box$) in it. The
application of a context $M$ to a process $P$, written $M[P]$, is
tantamount to filling the hole in $M$ with $P$. In this paper we do
not need the full weight of this theory, but do make use of the notion
of context in the proof the main theorem. 

\begin{mathpar}
  \inferrule* [lab=summation] {} {{M_{M},M_{N}} \bc \Box \;|\; x.M_{A} \;|\; M_{M}+M_{N}}
  \and
  \inferrule* [lab=agent] {} {{M_{A}} \bc (\vec{x})M_{P} \;| \; \clift{P_0,\ldots,M_{P},\ldots,P_N}}
  \and \\
  \inferrule* [lab=process] {} {{M_{P}} \bc M_{N} \;| \;P|M_{P} }
\end{mathpar} 

\begin{mathpar}
  \inferrule* [lab=sychronization] {} {M_{N} \bc \Box \;|\; x?M_{F} \;|\; x!M_{C}}
  \and
  \inferrule* [lab=abstraction] {} {{M_{F}} \bc (x)M_{P} }
  \and
  \inferrule* [lab=concretion] {} {{M_{C}} \bc \langle M_{P} \rangle }
  \and \\
  \inferrule* [lab=process] {} {{M_{P}} \bc M_{N} \;| \;P|M_{P} }
\end{mathpar}

\begin{definition}[contextual application] Given a context $M$, and
  process $P$, we define the \emph{contextual application}, $M[P] :=
  M\{P/\Box\}$. That is, the contextual application of M to P is the
  substitution of $P$ for $\Box$ in $M$.
\end{definition}

$\meaningof{-} : L \to \mathcal{P}(\pi)$

\begin{mathpar}
  \inferrule* [lab=collection] {} {\meaningof{true} = \pi, \and \meaningof{~E} = \pi \setminus \meaningof{E}, \and \meaningof{E_{1} \& E_{2}} = \meaningof{E_{1}} \cap \meaningof{E_{2}}}
\end{mathpar}

\begin{mathpar}
  \inferrule* [lab=structure] {} {\meaningof{0} = \{ P \in \pi | P \equiv 0 \}, \and \\ \meaningof{E_1 | E_2} = \{ P \in \pi | P \equiv P_{1} | P_{2}, P_{1} \in \meaningof{E_{1}}, P_{2} \in \meaningof{E_2}\} }
\end{mathpar}

\begin{mathpar}
 \inferrule* [lab=behavior] {} {\meaningof{\langle a?b \rangle E} = \{ P \in \pi | P \equiv Q | u?(y)P', \\ \and \\\\ \and \\ \;\;\; u \in \meaningof{a}, \forall z.P'\{z/y\} \in \meaningof{E\{z/b\}}\}, \and \\ \meaningof{a!E} = \{ P \in \pi | P \equiv Q | x!\langle P' \rangle, x \in \meaningof{a} P' \in \meaningof{E}\} }
\end{mathpar}

\begin{mathpar}
 \inferrule* [lab=nominal] {} {\meaningof{\quotep{E}} = \{ \quotep{P} \in \quotep{\pi} | P \in \meaningof{E} \}, \and \meaningof{\quotep{P}} = \{ \quotep{Q} \in \quotep{\pi} | P \equiv Q \} \and \\ \meaningof{@\quotep{E}} = \{ P \in \pi | P \equiv @x, x \in \meaningof{E} \}}
\end{mathpar}

\begin{eqnarray*}
  \\
  \meaningof{-} : TS \to ST
\end{eqnarray*}

\begin{eqnarray*}
  \\
  L : TS \to ST
\end{eqnarray*}

\begin{eqnarray*}
  \\
  P \models E \iff P \in \meaningof{E}
\end{eqnarray*}

\begin{eqnarray*}
  P \approx_{L} Q \iff \forall E \in L. P \models E \iff Q \models E
\end{eqnarray*}

\begin{eqnarray*}
  P \approx_{K} Q
\end{eqnarray*}

\begin{eqnarray*}
  P \approx Q
\end{eqnarray*}

$\approx_{K} = \approx = \approx_{L}$

\subsubsection{Contextual duality}

Note that contexts extend the quotation operation to a family of
operations from processes to names. Given a context, $M$, we can
define a \emph{nominal context}, $\quotep{M}$ by $\quotep{M}[P] :=
\quotep{M[P]}$. To foreshadow what is to come we observe that these
operations enjoy a duality with processes very much like the duality
between vectors and maps from vectors to scalars.

Further, because the calculus is essentially higher-order, we have a
correspondence between contexts and processes. More specifically,
given a name $x$ and a context $M$ we can construct $M^{*}_{x}$ such
that 

\begin{mathpar}
  M^{*}_{x} | \lift{x}{P} \red M[P]
\end{mathpar}

namely,

\begin{mathpar}
  M^{*}_{x} := x?(u).M[\dropn{u}]
\end{mathpar}

The dependence of $M^{*}_{x}$ on a name makes it an abstraction, 

\begin{mathpar}
  M^{*} := (x)x?(u).M[\dropn{u}]
\end{mathpar}

\subsection{Additional notation}

It will sometimes be convenient to denote the process a name
quotes. We already have the notation $x = \quotep{P}$, but it will be
convenient to introduce an alternate notation, $\procn{x}$, when we
want to emphasize the connection to the use of the name. Note that, by
virtue of name equivalence, $\quotep{\procn{x}} \nameeq x$; so, the
notation is consistent with previous definitions.

Further, because names have structure it is possible to effect
substitutions on the basis of that structure. This means we need to
upgrade our notation for substitutions, which we accomplish by
adapting comprehension notation. Thus,

\begin{mathpar}
  P\{ y / x : x \in S \}
\end{mathpar}

is interpreted to mean the process derived from P by replacing (in a
capture-avoiding manner) each occurrence of $x$ in $S$ by $y$. For example,

\begin{mathpar}
  P\{ \quotep{\procn{x}|\procn{x}} / x : x \in \freenames{P} \}
\end{mathpar}

will replace each (occurrence) of a free name $x$ in $P$ by
$\quotep{\procn{x}|\procn{x}}$.

Also, we will avail ourselves of the notation $x^{L}$ and $x^{R}$ to
denote injections of a name into disjoint copies of the name
space. There are numerous ways to accomplish this. One example can be
found in \cite{MeredithR05}. This notation overloads to vectors of
names: $\vec{x}^{\pi} := (x_{i}^{\pi} \; : \; 0 \leq i < |\vec{x}| )$ where $\pi \in \{L,R\}$.

We also use $P^{\Box} := P|\Box$.

In \cite{MeredithR05} an interpretation of the new operator is
given. It turns out that there are several possible interpretations
all enjoying the requisite algebraic properties of the operator (see
\cite{milner91polyadicpi}). We will therefore make liberal use of
$(\nu\; \vec{x})P$.

% subsection the_syntax_and_semantics_of_the_notation_system (end)   

\input{qm2pi.qmops} 

\input{qm2pi.sterngerlach} 

\input{qm2pi.metric} 

% section concurrent_process_calculi (end)

%\input{qm2pi.proofsketch}

% section proof sketch (end)

%\input{qm2pi.slviaknots} 

% section spatial logic via knots (end)

\input{qm2pi.conclusion}

% section conclusion (end)

%\input{qm2pi.dtcodes} 

% section wiring algorithm (end)

\input{qm2pi.ack} 

% section acknowledgments (end)

\newpage


\bibliographystyle{plain}   
\bibliography{../../biblios/main.bib}

\input{qm2pi.rhodetails}

\end{document}

 

% section wiring algorithm (end)

\documentclass[12pt]{llncs}
%\documentclass{jktr}

\usepackage[pdftex]{hyperref}                   
\usepackage {listings}
\usepackage {mathpartir}
\usepackage{bcprules}
%\usepackage{listings}
                       
\usepackage{graphicx} 
%\usepackage[margins=2.5cm,nohead,nofoot]{geometry}
%\usepackage{geometry}
\usepackage{amsfonts}
\usepackage{amstext}
\usepackage{latexsym}
\usepackage{amssymb}
\usepackage{color}


%\include{myPreamble}
\include{qm2pi.local} 

%\ifpdf
%\usepackage[pdftex]{graphicx}
%\else
%\usepackage{graphicx}
%\fi

 % \ifpdf
%  \usepackage{pdfsync}
%  \if


%\title{Brief Article}
%\author{David F. Snyder}
%\author{L.G. Meredith}

%\address{Dept. of Math., Texas State University--San Marcos, San Marcos, TX 78666}
       
\pagestyle{empty}


\begin{document}

\lstset{language=[Objective]Caml,frame=shadowbox}

\input{qm2pi.front}

% section front matter (end)

\input{qm2pi.intro} 
 
% section introduction (end)

% \input{qm2pi.knotations} 

% section notation (end)

\input{qm2pi.process.calculi} 

% section concurrent_process_calculi_and_spatial_logics_ (end)
    
%\input{qm2pi.knots2pi} 

%\input{qm2pi.trefoil} 

%\input{qm2pi.mainthm} 

% subsection basic_interpretation (end)

%\input{qm2pi.rho.presentation} 
\subsection{The syntax and semantics of the notation system}\label{sub:the_syntax_and_semantics_of_the_notation_system} % (fold)

We now summarize a technical presentation of the calculus that
embodies our theory of dynamics. The typical presentation of such a
calculus follows the style of giving generators and relations on
them. The grammar, below, describing term constructors, freely
generates the set of processes, $\Proc$. This set is then quotiented
by a relation known as structural congruence and it is over this set
that the notion of dynamics is expressed. This presentation is
essentially that of \cite{MeredithR05} with the addition of
polyadicity and summation. For readability we have relegated some of
the technical subtleties to an appendix.

\subsubsection{Process grammar}\label{subsub:process_grammar}

\begin{mathpar}
  \inferrule* [lab=synchronization] {} {{M} \bc \pzero \;|\; x?F \;|\; x!C }
  \and
  \inferrule* [lab=abstraction] {} {{F} \bc (x)P}
  \and
  \inferrule* [lab=concretion] {} {{C} \bc \langle Q \rangle}
  \and
  \inferrule* [lab=process] {} {{P,Q} \bc M \;| \;P|Q \;|\; @{x}}
  \and
  \inferrule* [lab=name] {} {{x} \bc \quotep{P}}
\end{mathpar} 

Note that $\vec{x}$ (resp. $\vec{P}$) denotes a vector of names
(resp. processes) of length $|\vec{x}|$ (resp. $|\vec{P}|$). We adopt
the following useful abbreviations.

\begin{mathpar}
   x?(\vec{y}).P := x.(\vec{y})P \and  x\clift{\vec{P}} := x.\clift{\vec{P}}
   \and x!(y) := \lift{x}{\dropn{y}}
   \and \Pi_{i=0}^{n-1}P_i := P_0 | \ldots | P_{n-1}
\end{mathpar}

\subsubsection{Structural congruence}

\paragraph{Free and bound names and alpha-equivalence.} At the
core of structural equivalence is alpha-equivalence which identifies
process that are the same up to a change of variable. Formally, we
recognize the distinction between free and bound names. The free names
of a process, $\freenames{P}$, may be calculated recursively as
follows:

\begin{mathpar}
\freenames{\pzero} := \emptyset
  \and \\
  \freenames{x?(y).P} := \{ x \} \cup (\freenames{P} \setminus \{ y \})
  \and 
  \freenames{x!\langle P \rangle} := \{ x \} \cup \{ P \} 
  \and \\
  \freenames{P|Q} := \freenames{P} \cup \freenames{Q}
  \and \\
  \freenames{@{x}} := \{ x \}
\end{mathpar}

$\pi$
$\quotep{\pi}$

$\freenames{-} : \pi \to \mathcal{P}(\quotep{\pi})$

\begin{eqnarray*}
  \freenames{\pzero} & := & \emptyset \\
  \freenames{x?(y).P} & := & \{ x \} \cup (\freenames{P} \setminus \{ y \}) \\
  \freenames{x!\langle P \rangle} & := & \{ x \} \cup \{ P \} \\
  \freenames{P|Q} & := & \freenames{P} \cup \freenames{Q} \\
  \freenames{\dropn{x}} & := & \{ x \}
\end{eqnarray*}

The bound names of a process, $\boundnames{P}$, are those names occurring in $P$
that are not free. For example, in $x?(y).0$, the name $x$ is free, while $y$ is bound.

\begin{mathpar}
  \inferrule* [lab=monoidal-laws] {} { P|Q \equiv Q|P \and P|0 \equiv P \and P|(Q|R) \equiv (P|Q)|R }
\end{mathpar}

\begin{mathpar}
  \inferrule* [lab=alpha-equivalence] {} { (x)P \equiv (y)P\{y/x\} \and y \not\in \freenames{P} }
\end{mathpar}

\begin{definition}
Then two processes, $P,Q$, are alpha-equivalent if $P = Q\{\vec{y}/\vec{x}\}$ for
some $\vec{x} \in \boundnames{Q},\vec{y} \in \boundnames{P}$, where $Q\{\vec{y}/\vec{x}\}$
denotes the capture-avoiding substitution of $\vec{y}$ for $\vec{x}$ in $Q$.
\end{definition}

\begin{definition}
  The {\em structural congruence} \cite{SangiorgiWalker} , $\equiv$,
  between processes is the least congruence containing
  alpha-equivalence, satisfying the abelian monoid laws
  (associativity, commutativity and $\pzero$ as identity) for parallel
  composition $|$ and for summation $+$.
\end{definition}

\subsection{Name equivalence}

We take name equivalence, written $\nameeq$, to be the smallest
equivalence relation generated by the following rules.

\begin{mathpar}
\inferrule*[lab=Quote-drop]
{ }
{ \quotep{@{x}} \nameeq x }

\inferrule*[lab=Struct-equiv]
{ P \scong Q }
{ \quotep{P} \nameeq \quotep{Q} }
\end{mathpar}

The astute reader will have noticed that the mutual recursion of names
and processes imposes a mutual recursion on alpha-equivalence and
structural equivalence via name-equivalence. Fortunately, all of this
works out pleasantly and we may calculate in the natural way, free of
concern. The reader interested in the details is referred to the
appendix \ref{appendix:rho_details}.

\subsection{Substitution}

We use $\Proc$ for the set of processes, $\QProc$ for the set of
names, and $\id{\{}\vec{y} / \vec{x} \id{\}}$ to denote partial maps,
$s : \QProc \rightarrow \QProc$. A map, $s$ lifts, uniquely, to a map
on process terms, $\widehat{s} : \Proc \rightarrow \Proc$ by the
following equations.

\begin{mathpar}
  (0) \psubstp{Q}{P} := 0 \\
  (R \juxtap S) \psubstp{Q}{P}
  :=    
  (R)\psubstp{Q}{P} \juxtap (S) \psubstp{Q}{P} \\
  (x?(y).R) \psubstp{Q}{P}    
  :=    
  (x)\substp{Q}{P} (z)\concat( (R \psubstn{z}{y}) \psubstp{Q}{P} ) \\
  (\lift{x}{R}) \psubstp{Q}{P}  
  :=
  \lift{(x)\substp{Q}{P}}{ R \psubstp{Q}{P} } \\
%   (\dropn{x})  \psubstp{Q}{P}       
%   := 
%   \left\{ 
%     \begin{array}{ccc} 
%       \dropn{\quotep{Q}} & & x \nameeq \quotep{P} \\
%       \dropn{x} & & otherwise \\
%     \end{array}
%   \right. 
  (\dropn{x})  \psubstp{Q}{P}       
  := 
  \left\{ 
    \begin{array}{ccc} 
      Q & & x \nameeq \quotep{P} \\
      \dropn{x} & & otherwise \\
    \end{array}
  \right.
\end{mathpar}
 

where

\begin{eqnarray}
  (x)\id{\{} \lpquote Q \rpquote / \lpquote P \rpquote \id{\}}            = 
  \left\{ 
    \begin{array}{ccc}
      \lpquote Q \rpquote & & x \nameeq \lpquote P \rpquote \\
      x & & otherwise \\
    \end{array}
  \right. \nonumber
\end{eqnarray}

and $z$ is chosen distinct from $\quotep{P}$, $\quotep{Q}$, the free
names in $Q$, and all the names in $R$. Our $\alpha$-equivalence will
be built in the standard way from this substitution.

\begin{remark}\label{rem:no_self_referential_names}
  One consequence of these definitions is that $\forall P. \quotep{P}
  \not\in \freenames{P}$.
\end{remark}

\subsection{ Dynamic quote: an example }

Anticipating something of what's to come, consider applying the
substitution, $\widehat{\id{\{}u / z \id{\}}}$, to the following pair
of processes, $\lift{w}{y!(z)}$ and $w[ \lpquote y!(z) \rpquote ]$.

\begin{eqnarray}
	\lift{w}{y!(z)}\widehat{\id{\{}u / z \id{\}}}
		& = &
		\lift{w}{y!(u)} \nonumber\\
	w[ \lpquote y!(z) \rpquote ] \widehat{ \id{\{}u / z \id{\}} }
		& = &
		w[ \lpquote y!(z) \rpquote ] \nonumber
\end{eqnarray}

Because the body of the process between quotes is impervious to
substitution, we get radically different answers. In fact, by
examining the first process in an input context,
e.g. $x?(z).\lift{w}{y!(z)}$, we see that the process under the lift
operator may be shaped by prefixed inputs binding a name inside it. In
this sense, the lift operator will be seen as a way to dynamically
construct processes before reifying them as names.

Finally equipped with these standard features we can present the
dynamics of the calculus.

\subsubsection{Operational semantics} 

Finally, we introduce the computational dynamics. What marks these
algebras as distinct from other more traditionally studied algebraic
structures, e.g. vector spaces or polynomial rings, is the manner in
which dynamics is captured. In traditional structures, dynamics is typically
expressed through morphisms between such structures, as in linear maps
between vector spaces or morphisms between rings. In algebras
associated with the semantics of computation, the dynamics is
expressed as part of the algebraic structure itself, through a
reduction reduction relation typically denoted by $\red$. Below, we
give a recursive presentation of this relation for the calculus used
in the encoding.

$\red \subseteq \pi \times \pi$
$\red : \pi \to \mathcal{P}(\pi)$

\begin{mathpar}
  \inferrule* [lab=Comm] { \textsf{match}( x_{src}, x_{trgt} ) } { x_{trgt}?(y)P \; | \; x_{src}!\langle {Q} \rangle \red P\{\quotep{Q}/y}\} }
  \and \\
  \inferrule* [lab=Par] {{P} \red {P}'} {{{P} | {Q}} \red {{P}' | {Q}}}
  \and
  \inferrule* [lab=Equiv]{{{P} \scong {P}'} \andalso {{P}' \red {Q}'} \andalso {{Q}' \scong {Q}}}{{P} \red {Q}}
\end{mathpar}

\begin{eqnarray*}
  match_{\equiv} (\quotep{P},\quotep{Q}) & := & P \equiv Q \\
  match_{\dagger}(\quotep{P},\quotep{Q}) & := & \forall R. P|Q \red^{*} R => R \red^{*} 0 \\
  match_{K}(\quotep{P},\quotep{Q}) & := & K \mbox{ for some context } K
\end{eqnarray*}

$u?(x)P | u!\langle Q \rangle \red P\{\quotep{Q}/x\}$

%We write $\wred$ for $\red^*$, and $P\red$ if $\exists Q $ such that $ P \red Q$.
We write $P\red$ if $\exists Q $ such that $ P \red Q$ and $P\not\red$, otherwise.

\section{Replication}

As mentioned before, it is known that replication (and hence
recursion) can be implemented in a higher-order process algebra
\cite{SangiorgiWalker}. As our first example of calculation with the
machinery thus far presented we give the construction explicitly in
the {\rhoc}.

\begin{eqnarray}
	D_{x} & := & \prefix{x}{y}{(\binpar{\outputp{x}{y}}{@{y}})} \nonumber\\
	\bangp_{x}{P} & := & \binpar{{x}!\langle{\binpar{D_{x}}{P}}\rangle}{D_{x}} \nonumber
\end{eqnarray}

\begin{eqnarray}
	\bangp_{x}{P} & & \nonumber\\
	=
	& {x}!\langle{(\prefix{x}{y}{(\outputp{x}{y} | @{y})) | P}}\rangle 
	      | \prefix{x}{y}{(\outputp{x}{y} | @{y})} & \nonumber\\
	\red
	& (\outputp{x}{y} | @{y})\substn{\quotep{(\prefix{x}{y}{(@{y} | \outputp{x}{y})) | P}}}{y} & \nonumber\\
	=
	& \outputp{x}{\quotep{(\prefix{x}{y}{(\outputp{x}{y} | @{y})) | P}}}
	  | {(\prefix{x}{y}{(\outputp{x}{y} | @{y})) | P}} & \nonumber\\
	\red
	& \ldots & \nonumber\\
	\red^*
	& P | P | \ldots & \nonumber
\end{eqnarray}

Of course, this encoding, as an implementation, runs away, unfolding
$\bangp{P}$ eagerly. A lazier and more implementable replication
operator, restricted to input-guarded processes, may be obtained as follows.

\begin{eqnarray}
\bangp{\prefix{u}{v}{P}} 
	:= 
	\binpar{\lift{x}{\prefix{u}{v}{(\binpar{D(x)}{P})}}}{D(x)} \nonumber
\end{eqnarray}

\begin{remark}
  Note that the lazier definition still does not deal with summation
  or mixed summation (i.e. sums over input and output). The reader is
  invited to construct definitions of replication that deal with these
  features. 

  Further, the definitions are parameterized in a name, $x$. Can you,
  gentle reader, make a definition that eliminates this parameter and
  guarantees no accidental interaction between the replication
  machinery and the process being replicated -- i.e. no accidental
  sharing of names used by the process to get its work done and the
  name(s) used by the replication to effect copying. This latter
  revision of the definition of replication is crucial to obtaining
  the expected identity $!!P \sim !P$.
\end{remark}

\begin{remark}\label{rem:paradoxical_combinator}
  The reader familiar with the lambda calculus will have noticed the
  similarity between $D$ and the paradoxical combinator.

  [Ed. note: the existence of this seems to suggest we have to be more
  restrictive on the set of processes and names we admit if we are to
  support no-cloning.]
\end{remark}

\subsubsection{Bisimulation}

The computational dynamics gives rise to another kind of equivalence,
the equivalence of computational behavior. As previously mentioned
this is typically captured \emph{via} some form of bisimulation.

% The notion we use in this paper is weak barbed bisimulation
% \cite{milner91polyadicpi}.

The notion we use in this paper is derived from weak barbed
bisimulation \cite{milner91polyadicpi}. 

\begin{definition}
An \emph{observation relation}, $\downarrow_{\mathcal N}$, over a set
of names, $\mathcal N$, is the smallest relation satisfying the rules
below.

\infrule[Out-barb]{y \in {\mathcal N}, \; x \nameeq y}
		  {\outputp{x}{v} \downarrow_{\mathcal N} x}
\infrule[Par-barb]{\mbox{$P\downarrow_{\mathcal N} x$ or $Q\downarrow_{\mathcal N} x$}}
		  {\binpar{P}{Q} \downarrow_{\mathcal N} x}

We write $P \Downarrow_{\mathcal N} x$ if there is $Q$ such that 
$P \wred Q$ and $Q \downarrow_{\mathcal N} x$.
\end{definition}

\begin{definition}
%\label{def.bbisim}
An  ${\mathcal N}$-\emph{barbed bisimulation} over a set of names, ${\mathcal N}$, is a symmetric binary relation 
${\mathcal S}_{\mathcal N}$ between agents such that $P\rel{S}_{\mathcal N}Q$ implies:
\begin{enumerate}
\item If $P \red P'$ then $Q \wred Q'$ and $P'\rel{S}_{\mathcal N} Q'$.
\item If $P\downarrow_{\mathcal N} x$, then $Q\Downarrow_{\mathcal N} x$.
\end{enumerate}
$P$ is ${\mathcal N}$-barbed bisimilar to $Q$, written
$P \wbbisim_{\mathcal N} Q$, if $P \rel{S}_{\mathcal N} Q$ for some ${\mathcal N}$-barbed bisimulation ${\mathcal S}_{\mathcal N}$.
\end{definition}

$\mathcal{R} \subseteq \pi \times \pi$

$P \mathcal{R} Q => \forall P'. P \red P' \Rightarrow \exists Q'. Q \red Q', P' \mathcal{R} Q'$

$P \vdash x \Rightarrow Q \vdash x$

\begin{mathpar}
  \inferrule*[lab=Out-barb]{x \nameeq y}{{y}!\langle{Q}\rangle \vdash x}
  \and
  \inferrule*[lab=Par-barb]{\mbox{$P\vdash x$ or $Q\vdash x$}}{\binpar{P}{Q} \vdash x}
\end{mathpar}

\subsubsection{Contexts}

One of the principle advantages of computational calculi like the
$\pi$-calculus is a well-defined notion of context,
contextual-equivalence and a correlation between
contextual-equivalence and notions of bisimulation. The notion of
context allows the decomposition of a process into (sub-)process and
its syntactic environment, its context. Thus, a context may be
thought of as a process with a ``hole'' (written $\Box$) in it. The
application of a context $M$ to a process $P$, written $M[P]$, is
tantamount to filling the hole in $M$ with $P$. In this paper we do
not need the full weight of this theory, but do make use of the notion
of context in the proof the main theorem. 

\begin{mathpar}
  \inferrule* [lab=summation] {} {{M_{M},M_{N}} \bc \Box \;|\; x.M_{A} \;|\; M_{M}+M_{N}}
  \and
  \inferrule* [lab=agent] {} {{M_{A}} \bc (\vec{x})M_{P} \;| \; \clift{P_0,\ldots,M_{P},\ldots,P_N}}
  \and \\
  \inferrule* [lab=process] {} {{M_{P}} \bc M_{N} \;| \;P|M_{P} }
\end{mathpar} 

\begin{mathpar}
  \inferrule* [lab=sychronization] {} {M_{N} \bc \Box \;|\; x?M_{F} \;|\; x!M_{C}}
  \and
  \inferrule* [lab=abstraction] {} {{M_{F}} \bc (x)M_{P} }
  \and
  \inferrule* [lab=concretion] {} {{M_{C}} \bc \langle M_{P} \rangle }
  \and \\
  \inferrule* [lab=process] {} {{M_{P}} \bc M_{N} \;| \;P|M_{P} }
\end{mathpar}

\begin{definition}[contextual application] Given a context $M$, and
  process $P$, we define the \emph{contextual application}, $M[P] :=
  M\{P/\Box\}$. That is, the contextual application of M to P is the
  substitution of $P$ for $\Box$ in $M$.
\end{definition}

$\meaningof{-} : L \to \mathcal{P}(\pi)$

\begin{mathpar}
  \inferrule* [lab=collection] {} {\meaningof{true} = \pi, \and \meaningof{~E} = \pi \setminus \meaningof{E}, \and \meaningof{E_{1} \& E_{2}} = \meaningof{E_{1}} \cap \meaningof{E_{2}}}
\end{mathpar}

\begin{mathpar}
  \inferrule* [lab=structure] {} {\meaningof{0} = \{ P \in \pi | P \equiv 0 \}, \and \\ \meaningof{E_1 | E_2} = \{ P \in \pi | P \equiv P_{1} | P_{2}, P_{1} \in \meaningof{E_{1}}, P_{2} \in \meaningof{E_2}\} }
\end{mathpar}

\begin{mathpar}
 \inferrule* [lab=behavior] {} {\meaningof{\langle a?b \rangle E} = \{ P \in \pi | P \equiv Q | u?(y)P', \\ \and \\\\ \and \\ \;\;\; u \in \meaningof{a}, \forall z.P'\{z/y\} \in \meaningof{E\{z/b\}}\}, \and \\ \meaningof{a!E} = \{ P \in \pi | P \equiv Q | x!\langle P' \rangle, x \in \meaningof{a} P' \in \meaningof{E}\} }
\end{mathpar}

\begin{mathpar}
 \inferrule* [lab=nominal] {} {\meaningof{\quotep{E}} = \{ \quotep{P} \in \quotep{\pi} | P \in \meaningof{E} \}, \and \meaningof{\quotep{P}} = \{ \quotep{Q} \in \quotep{\pi} | P \equiv Q \} \and \\ \meaningof{@\quotep{E}} = \{ P \in \pi | P \equiv @x, x \in \meaningof{E} \}}
\end{mathpar}

\begin{eqnarray*}
  \\
  \meaningof{-} : TS \to ST
\end{eqnarray*}

\begin{eqnarray*}
  \\
  L : TS \to ST
\end{eqnarray*}

\begin{eqnarray*}
  \\
  P \models E \iff P \in \meaningof{E}
\end{eqnarray*}

\begin{eqnarray*}
  P \approx_{L} Q \iff \forall E \in L. P \models E \iff Q \models E
\end{eqnarray*}

\begin{eqnarray*}
  P \approx_{K} Q
\end{eqnarray*}

\begin{eqnarray*}
  P \approx Q
\end{eqnarray*}

$\approx_{K} = \approx = \approx_{L}$

\subsubsection{Contextual duality}

Note that contexts extend the quotation operation to a family of
operations from processes to names. Given a context, $M$, we can
define a \emph{nominal context}, $\quotep{M}$ by $\quotep{M}[P] :=
\quotep{M[P]}$. To foreshadow what is to come we observe that these
operations enjoy a duality with processes very much like the duality
between vectors and maps from vectors to scalars.

Further, because the calculus is essentially higher-order, we have a
correspondence between contexts and processes. More specifically,
given a name $x$ and a context $M$ we can construct $M^{*}_{x}$ such
that 

\begin{mathpar}
  M^{*}_{x} | \lift{x}{P} \red M[P]
\end{mathpar}

namely,

\begin{mathpar}
  M^{*}_{x} := x?(u).M[\dropn{u}]
\end{mathpar}

The dependence of $M^{*}_{x}$ on a name makes it an abstraction, 

\begin{mathpar}
  M^{*} := (x)x?(u).M[\dropn{u}]
\end{mathpar}

\subsection{Additional notation}

It will sometimes be convenient to denote the process a name
quotes. We already have the notation $x = \quotep{P}$, but it will be
convenient to introduce an alternate notation, $\procn{x}$, when we
want to emphasize the connection to the use of the name. Note that, by
virtue of name equivalence, $\quotep{\procn{x}} \nameeq x$; so, the
notation is consistent with previous definitions.

Further, because names have structure it is possible to effect
substitutions on the basis of that structure. This means we need to
upgrade our notation for substitutions, which we accomplish by
adapting comprehension notation. Thus,

\begin{mathpar}
  P\{ y / x : x \in S \}
\end{mathpar}

is interpreted to mean the process derived from P by replacing (in a
capture-avoiding manner) each occurrence of $x$ in $S$ by $y$. For example,

\begin{mathpar}
  P\{ \quotep{\procn{x}|\procn{x}} / x : x \in \freenames{P} \}
\end{mathpar}

will replace each (occurrence) of a free name $x$ in $P$ by
$\quotep{\procn{x}|\procn{x}}$.

Also, we will avail ourselves of the notation $x^{L}$ and $x^{R}$ to
denote injections of a name into disjoint copies of the name
space. There are numerous ways to accomplish this. One example can be
found in \cite{MeredithR05}. This notation overloads to vectors of
names: $\vec{x}^{\pi} := (x_{i}^{\pi} \; : \; 0 \leq i < |\vec{x}| )$ where $\pi \in \{L,R\}$.

We also use $P^{\Box} := P|\Box$.

In \cite{MeredithR05} an interpretation of the new operator is
given. It turns out that there are several possible interpretations
all enjoying the requisite algebraic properties of the operator (see
\cite{milner91polyadicpi}). We will therefore make liberal use of
$(\nu\; \vec{x})P$.

% subsection the_syntax_and_semantics_of_the_notation_system (end)   

\input{qm2pi.qmops} 

\input{qm2pi.sterngerlach} 

\input{qm2pi.metric} 

% section concurrent_process_calculi (end)

%\input{qm2pi.proofsketch}

% section proof sketch (end)

%\input{qm2pi.slviaknots} 

% section spatial logic via knots (end)

\input{qm2pi.conclusion}

% section conclusion (end)

%\input{qm2pi.dtcodes} 

% section wiring algorithm (end)

\input{qm2pi.ack} 

% section acknowledgments (end)

\newpage


\bibliographystyle{plain}   
\bibliography{../../biblios/main.bib}

\input{qm2pi.rhodetails}

\end{document}

 

% section acknowledgments (end)

\newpage


\bibliographystyle{plain}   
\bibliography{../../biblios/main.bib}

\documentclass[12pt]{llncs}
%\documentclass{jktr}

\usepackage[pdftex]{hyperref}                   
\usepackage {listings}
\usepackage {mathpartir}
\usepackage{bcprules}
%\usepackage{listings}
                       
\usepackage{graphicx} 
%\usepackage[margins=2.5cm,nohead,nofoot]{geometry}
%\usepackage{geometry}
\usepackage{amsfonts}
\usepackage{amstext}
\usepackage{latexsym}
\usepackage{amssymb}
\usepackage{color}


%\include{myPreamble}
\include{qm2pi.local} 

%\ifpdf
%\usepackage[pdftex]{graphicx}
%\else
%\usepackage{graphicx}
%\fi

 % \ifpdf
%  \usepackage{pdfsync}
%  \if


%\title{Brief Article}
%\author{David F. Snyder}
%\author{L.G. Meredith}

%\address{Dept. of Math., Texas State University--San Marcos, San Marcos, TX 78666}
       
\pagestyle{empty}


\begin{document}

\lstset{language=[Objective]Caml,frame=shadowbox}

\input{qm2pi.front}

% section front matter (end)

\input{qm2pi.intro} 
 
% section introduction (end)

% \input{qm2pi.knotations} 

% section notation (end)

\input{qm2pi.process.calculi} 

% section concurrent_process_calculi_and_spatial_logics_ (end)
    
%\input{qm2pi.knots2pi} 

%\input{qm2pi.trefoil} 

%\input{qm2pi.mainthm} 

% subsection basic_interpretation (end)

%\input{qm2pi.rho.presentation} 
\subsection{The syntax and semantics of the notation system}\label{sub:the_syntax_and_semantics_of_the_notation_system} % (fold)

We now summarize a technical presentation of the calculus that
embodies our theory of dynamics. The typical presentation of such a
calculus follows the style of giving generators and relations on
them. The grammar, below, describing term constructors, freely
generates the set of processes, $\Proc$. This set is then quotiented
by a relation known as structural congruence and it is over this set
that the notion of dynamics is expressed. This presentation is
essentially that of \cite{MeredithR05} with the addition of
polyadicity and summation. For readability we have relegated some of
the technical subtleties to an appendix.

\subsubsection{Process grammar}\label{subsub:process_grammar}

\begin{mathpar}
  \inferrule* [lab=synchronization] {} {{M} \bc \pzero \;|\; x?F \;|\; x!C }
  \and
  \inferrule* [lab=abstraction] {} {{F} \bc (x)P}
  \and
  \inferrule* [lab=concretion] {} {{C} \bc \langle Q \rangle}
  \and
  \inferrule* [lab=process] {} {{P,Q} \bc M \;| \;P|Q \;|\; @{x}}
  \and
  \inferrule* [lab=name] {} {{x} \bc \quotep{P}}
\end{mathpar} 

Note that $\vec{x}$ (resp. $\vec{P}$) denotes a vector of names
(resp. processes) of length $|\vec{x}|$ (resp. $|\vec{P}|$). We adopt
the following useful abbreviations.

\begin{mathpar}
   x?(\vec{y}).P := x.(\vec{y})P \and  x\clift{\vec{P}} := x.\clift{\vec{P}}
   \and x!(y) := \lift{x}{\dropn{y}}
   \and \Pi_{i=0}^{n-1}P_i := P_0 | \ldots | P_{n-1}
\end{mathpar}

\subsubsection{Structural congruence}

\paragraph{Free and bound names and alpha-equivalence.} At the
core of structural equivalence is alpha-equivalence which identifies
process that are the same up to a change of variable. Formally, we
recognize the distinction between free and bound names. The free names
of a process, $\freenames{P}$, may be calculated recursively as
follows:

\begin{mathpar}
\freenames{\pzero} := \emptyset
  \and \\
  \freenames{x?(y).P} := \{ x \} \cup (\freenames{P} \setminus \{ y \})
  \and 
  \freenames{x!\langle P \rangle} := \{ x \} \cup \{ P \} 
  \and \\
  \freenames{P|Q} := \freenames{P} \cup \freenames{Q}
  \and \\
  \freenames{@{x}} := \{ x \}
\end{mathpar}

$\pi$
$\quotep{\pi}$

$\freenames{-} : \pi \to \mathcal{P}(\quotep{\pi})$

\begin{eqnarray*}
  \freenames{\pzero} & := & \emptyset \\
  \freenames{x?(y).P} & := & \{ x \} \cup (\freenames{P} \setminus \{ y \}) \\
  \freenames{x!\langle P \rangle} & := & \{ x \} \cup \{ P \} \\
  \freenames{P|Q} & := & \freenames{P} \cup \freenames{Q} \\
  \freenames{\dropn{x}} & := & \{ x \}
\end{eqnarray*}

The bound names of a process, $\boundnames{P}$, are those names occurring in $P$
that are not free. For example, in $x?(y).0$, the name $x$ is free, while $y$ is bound.

\begin{mathpar}
  \inferrule* [lab=monoidal-laws] {} { P|Q \equiv Q|P \and P|0 \equiv P \and P|(Q|R) \equiv (P|Q)|R }
\end{mathpar}

\begin{mathpar}
  \inferrule* [lab=alpha-equivalence] {} { (x)P \equiv (y)P\{y/x\} \and y \not\in \freenames{P} }
\end{mathpar}

\begin{definition}
Then two processes, $P,Q$, are alpha-equivalent if $P = Q\{\vec{y}/\vec{x}\}$ for
some $\vec{x} \in \boundnames{Q},\vec{y} \in \boundnames{P}$, where $Q\{\vec{y}/\vec{x}\}$
denotes the capture-avoiding substitution of $\vec{y}$ for $\vec{x}$ in $Q$.
\end{definition}

\begin{definition}
  The {\em structural congruence} \cite{SangiorgiWalker} , $\equiv$,
  between processes is the least congruence containing
  alpha-equivalence, satisfying the abelian monoid laws
  (associativity, commutativity and $\pzero$ as identity) for parallel
  composition $|$ and for summation $+$.
\end{definition}

\subsection{Name equivalence}

We take name equivalence, written $\nameeq$, to be the smallest
equivalence relation generated by the following rules.

\begin{mathpar}
\inferrule*[lab=Quote-drop]
{ }
{ \quotep{@{x}} \nameeq x }

\inferrule*[lab=Struct-equiv]
{ P \scong Q }
{ \quotep{P} \nameeq \quotep{Q} }
\end{mathpar}

The astute reader will have noticed that the mutual recursion of names
and processes imposes a mutual recursion on alpha-equivalence and
structural equivalence via name-equivalence. Fortunately, all of this
works out pleasantly and we may calculate in the natural way, free of
concern. The reader interested in the details is referred to the
appendix \ref{appendix:rho_details}.

\subsection{Substitution}

We use $\Proc$ for the set of processes, $\QProc$ for the set of
names, and $\id{\{}\vec{y} / \vec{x} \id{\}}$ to denote partial maps,
$s : \QProc \rightarrow \QProc$. A map, $s$ lifts, uniquely, to a map
on process terms, $\widehat{s} : \Proc \rightarrow \Proc$ by the
following equations.

\begin{mathpar}
  (0) \psubstp{Q}{P} := 0 \\
  (R \juxtap S) \psubstp{Q}{P}
  :=    
  (R)\psubstp{Q}{P} \juxtap (S) \psubstp{Q}{P} \\
  (x?(y).R) \psubstp{Q}{P}    
  :=    
  (x)\substp{Q}{P} (z)\concat( (R \psubstn{z}{y}) \psubstp{Q}{P} ) \\
  (\lift{x}{R}) \psubstp{Q}{P}  
  :=
  \lift{(x)\substp{Q}{P}}{ R \psubstp{Q}{P} } \\
%   (\dropn{x})  \psubstp{Q}{P}       
%   := 
%   \left\{ 
%     \begin{array}{ccc} 
%       \dropn{\quotep{Q}} & & x \nameeq \quotep{P} \\
%       \dropn{x} & & otherwise \\
%     \end{array}
%   \right. 
  (\dropn{x})  \psubstp{Q}{P}       
  := 
  \left\{ 
    \begin{array}{ccc} 
      Q & & x \nameeq \quotep{P} \\
      \dropn{x} & & otherwise \\
    \end{array}
  \right.
\end{mathpar}
 

where

\begin{eqnarray}
  (x)\id{\{} \lpquote Q \rpquote / \lpquote P \rpquote \id{\}}            = 
  \left\{ 
    \begin{array}{ccc}
      \lpquote Q \rpquote & & x \nameeq \lpquote P \rpquote \\
      x & & otherwise \\
    \end{array}
  \right. \nonumber
\end{eqnarray}

and $z$ is chosen distinct from $\quotep{P}$, $\quotep{Q}$, the free
names in $Q$, and all the names in $R$. Our $\alpha$-equivalence will
be built in the standard way from this substitution.

\begin{remark}\label{rem:no_self_referential_names}
  One consequence of these definitions is that $\forall P. \quotep{P}
  \not\in \freenames{P}$.
\end{remark}

\subsection{ Dynamic quote: an example }

Anticipating something of what's to come, consider applying the
substitution, $\widehat{\id{\{}u / z \id{\}}}$, to the following pair
of processes, $\lift{w}{y!(z)}$ and $w[ \lpquote y!(z) \rpquote ]$.

\begin{eqnarray}
	\lift{w}{y!(z)}\widehat{\id{\{}u / z \id{\}}}
		& = &
		\lift{w}{y!(u)} \nonumber\\
	w[ \lpquote y!(z) \rpquote ] \widehat{ \id{\{}u / z \id{\}} }
		& = &
		w[ \lpquote y!(z) \rpquote ] \nonumber
\end{eqnarray}

Because the body of the process between quotes is impervious to
substitution, we get radically different answers. In fact, by
examining the first process in an input context,
e.g. $x?(z).\lift{w}{y!(z)}$, we see that the process under the lift
operator may be shaped by prefixed inputs binding a name inside it. In
this sense, the lift operator will be seen as a way to dynamically
construct processes before reifying them as names.

Finally equipped with these standard features we can present the
dynamics of the calculus.

\subsubsection{Operational semantics} 

Finally, we introduce the computational dynamics. What marks these
algebras as distinct from other more traditionally studied algebraic
structures, e.g. vector spaces or polynomial rings, is the manner in
which dynamics is captured. In traditional structures, dynamics is typically
expressed through morphisms between such structures, as in linear maps
between vector spaces or morphisms between rings. In algebras
associated with the semantics of computation, the dynamics is
expressed as part of the algebraic structure itself, through a
reduction reduction relation typically denoted by $\red$. Below, we
give a recursive presentation of this relation for the calculus used
in the encoding.

$\red \subseteq \pi \times \pi$
$\red : \pi \to \mathcal{P}(\pi)$

\begin{mathpar}
  \inferrule* [lab=Comm] { \textsf{match}( x_{src}, x_{trgt} ) } { x_{trgt}?(y)P \; | \; x_{src}!\langle {Q} \rangle \red P\{\quotep{Q}/y}\} }
  \and \\
  \inferrule* [lab=Par] {{P} \red {P}'} {{{P} | {Q}} \red {{P}' | {Q}}}
  \and
  \inferrule* [lab=Equiv]{{{P} \scong {P}'} \andalso {{P}' \red {Q}'} \andalso {{Q}' \scong {Q}}}{{P} \red {Q}}
\end{mathpar}

\begin{eqnarray*}
  match_{\equiv} (\quotep{P},\quotep{Q}) & := & P \equiv Q \\
  match_{\dagger}(\quotep{P},\quotep{Q}) & := & \forall R. P|Q \red^{*} R => R \red^{*} 0 \\
  match_{K}(\quotep{P},\quotep{Q}) & := & K \mbox{ for some context } K
\end{eqnarray*}

$u?(x)P | u!\langle Q \rangle \red P\{\quotep{Q}/x\}$

%We write $\wred$ for $\red^*$, and $P\red$ if $\exists Q $ such that $ P \red Q$.
We write $P\red$ if $\exists Q $ such that $ P \red Q$ and $P\not\red$, otherwise.

\section{Replication}

As mentioned before, it is known that replication (and hence
recursion) can be implemented in a higher-order process algebra
\cite{SangiorgiWalker}. As our first example of calculation with the
machinery thus far presented we give the construction explicitly in
the {\rhoc}.

\begin{eqnarray}
	D_{x} & := & \prefix{x}{y}{(\binpar{\outputp{x}{y}}{@{y}})} \nonumber\\
	\bangp_{x}{P} & := & \binpar{{x}!\langle{\binpar{D_{x}}{P}}\rangle}{D_{x}} \nonumber
\end{eqnarray}

\begin{eqnarray}
	\bangp_{x}{P} & & \nonumber\\
	=
	& {x}!\langle{(\prefix{x}{y}{(\outputp{x}{y} | @{y})) | P}}\rangle 
	      | \prefix{x}{y}{(\outputp{x}{y} | @{y})} & \nonumber\\
	\red
	& (\outputp{x}{y} | @{y})\substn{\quotep{(\prefix{x}{y}{(@{y} | \outputp{x}{y})) | P}}}{y} & \nonumber\\
	=
	& \outputp{x}{\quotep{(\prefix{x}{y}{(\outputp{x}{y} | @{y})) | P}}}
	  | {(\prefix{x}{y}{(\outputp{x}{y} | @{y})) | P}} & \nonumber\\
	\red
	& \ldots & \nonumber\\
	\red^*
	& P | P | \ldots & \nonumber
\end{eqnarray}

Of course, this encoding, as an implementation, runs away, unfolding
$\bangp{P}$ eagerly. A lazier and more implementable replication
operator, restricted to input-guarded processes, may be obtained as follows.

\begin{eqnarray}
\bangp{\prefix{u}{v}{P}} 
	:= 
	\binpar{\lift{x}{\prefix{u}{v}{(\binpar{D(x)}{P})}}}{D(x)} \nonumber
\end{eqnarray}

\begin{remark}
  Note that the lazier definition still does not deal with summation
  or mixed summation (i.e. sums over input and output). The reader is
  invited to construct definitions of replication that deal with these
  features. 

  Further, the definitions are parameterized in a name, $x$. Can you,
  gentle reader, make a definition that eliminates this parameter and
  guarantees no accidental interaction between the replication
  machinery and the process being replicated -- i.e. no accidental
  sharing of names used by the process to get its work done and the
  name(s) used by the replication to effect copying. This latter
  revision of the definition of replication is crucial to obtaining
  the expected identity $!!P \sim !P$.
\end{remark}

\begin{remark}\label{rem:paradoxical_combinator}
  The reader familiar with the lambda calculus will have noticed the
  similarity between $D$ and the paradoxical combinator.

  [Ed. note: the existence of this seems to suggest we have to be more
  restrictive on the set of processes and names we admit if we are to
  support no-cloning.]
\end{remark}

\subsubsection{Bisimulation}

The computational dynamics gives rise to another kind of equivalence,
the equivalence of computational behavior. As previously mentioned
this is typically captured \emph{via} some form of bisimulation.

% The notion we use in this paper is weak barbed bisimulation
% \cite{milner91polyadicpi}.

The notion we use in this paper is derived from weak barbed
bisimulation \cite{milner91polyadicpi}. 

\begin{definition}
An \emph{observation relation}, $\downarrow_{\mathcal N}$, over a set
of names, $\mathcal N$, is the smallest relation satisfying the rules
below.

\infrule[Out-barb]{y \in {\mathcal N}, \; x \nameeq y}
		  {\outputp{x}{v} \downarrow_{\mathcal N} x}
\infrule[Par-barb]{\mbox{$P\downarrow_{\mathcal N} x$ or $Q\downarrow_{\mathcal N} x$}}
		  {\binpar{P}{Q} \downarrow_{\mathcal N} x}

We write $P \Downarrow_{\mathcal N} x$ if there is $Q$ such that 
$P \wred Q$ and $Q \downarrow_{\mathcal N} x$.
\end{definition}

\begin{definition}
%\label{def.bbisim}
An  ${\mathcal N}$-\emph{barbed bisimulation} over a set of names, ${\mathcal N}$, is a symmetric binary relation 
${\mathcal S}_{\mathcal N}$ between agents such that $P\rel{S}_{\mathcal N}Q$ implies:
\begin{enumerate}
\item If $P \red P'$ then $Q \wred Q'$ and $P'\rel{S}_{\mathcal N} Q'$.
\item If $P\downarrow_{\mathcal N} x$, then $Q\Downarrow_{\mathcal N} x$.
\end{enumerate}
$P$ is ${\mathcal N}$-barbed bisimilar to $Q$, written
$P \wbbisim_{\mathcal N} Q$, if $P \rel{S}_{\mathcal N} Q$ for some ${\mathcal N}$-barbed bisimulation ${\mathcal S}_{\mathcal N}$.
\end{definition}

$\mathcal{R} \subseteq \pi \times \pi$

$P \mathcal{R} Q => \forall P'. P \red P' \Rightarrow \exists Q'. Q \red Q', P' \mathcal{R} Q'$

$P \vdash x \Rightarrow Q \vdash x$

\begin{mathpar}
  \inferrule*[lab=Out-barb]{x \nameeq y}{{y}!\langle{Q}\rangle \vdash x}
  \and
  \inferrule*[lab=Par-barb]{\mbox{$P\vdash x$ or $Q\vdash x$}}{\binpar{P}{Q} \vdash x}
\end{mathpar}

\subsubsection{Contexts}

One of the principle advantages of computational calculi like the
$\pi$-calculus is a well-defined notion of context,
contextual-equivalence and a correlation between
contextual-equivalence and notions of bisimulation. The notion of
context allows the decomposition of a process into (sub-)process and
its syntactic environment, its context. Thus, a context may be
thought of as a process with a ``hole'' (written $\Box$) in it. The
application of a context $M$ to a process $P$, written $M[P]$, is
tantamount to filling the hole in $M$ with $P$. In this paper we do
not need the full weight of this theory, but do make use of the notion
of context in the proof the main theorem. 

\begin{mathpar}
  \inferrule* [lab=summation] {} {{M_{M},M_{N}} \bc \Box \;|\; x.M_{A} \;|\; M_{M}+M_{N}}
  \and
  \inferrule* [lab=agent] {} {{M_{A}} \bc (\vec{x})M_{P} \;| \; \clift{P_0,\ldots,M_{P},\ldots,P_N}}
  \and \\
  \inferrule* [lab=process] {} {{M_{P}} \bc M_{N} \;| \;P|M_{P} }
\end{mathpar} 

\begin{mathpar}
  \inferrule* [lab=sychronization] {} {M_{N} \bc \Box \;|\; x?M_{F} \;|\; x!M_{C}}
  \and
  \inferrule* [lab=abstraction] {} {{M_{F}} \bc (x)M_{P} }
  \and
  \inferrule* [lab=concretion] {} {{M_{C}} \bc \langle M_{P} \rangle }
  \and \\
  \inferrule* [lab=process] {} {{M_{P}} \bc M_{N} \;| \;P|M_{P} }
\end{mathpar}

\begin{definition}[contextual application] Given a context $M$, and
  process $P$, we define the \emph{contextual application}, $M[P] :=
  M\{P/\Box\}$. That is, the contextual application of M to P is the
  substitution of $P$ for $\Box$ in $M$.
\end{definition}

$\meaningof{-} : L \to \mathcal{P}(\pi)$

\begin{mathpar}
  \inferrule* [lab=collection] {} {\meaningof{true} = \pi, \and \meaningof{~E} = \pi \setminus \meaningof{E}, \and \meaningof{E_{1} \& E_{2}} = \meaningof{E_{1}} \cap \meaningof{E_{2}}}
\end{mathpar}

\begin{mathpar}
  \inferrule* [lab=structure] {} {\meaningof{0} = \{ P \in \pi | P \equiv 0 \}, \and \\ \meaningof{E_1 | E_2} = \{ P \in \pi | P \equiv P_{1} | P_{2}, P_{1} \in \meaningof{E_{1}}, P_{2} \in \meaningof{E_2}\} }
\end{mathpar}

\begin{mathpar}
 \inferrule* [lab=behavior] {} {\meaningof{\langle a?b \rangle E} = \{ P \in \pi | P \equiv Q | u?(y)P', \\ \and \\\\ \and \\ \;\;\; u \in \meaningof{a}, \forall z.P'\{z/y\} \in \meaningof{E\{z/b\}}\}, \and \\ \meaningof{a!E} = \{ P \in \pi | P \equiv Q | x!\langle P' \rangle, x \in \meaningof{a} P' \in \meaningof{E}\} }
\end{mathpar}

\begin{mathpar}
 \inferrule* [lab=nominal] {} {\meaningof{\quotep{E}} = \{ \quotep{P} \in \quotep{\pi} | P \in \meaningof{E} \}, \and \meaningof{\quotep{P}} = \{ \quotep{Q} \in \quotep{\pi} | P \equiv Q \} \and \\ \meaningof{@\quotep{E}} = \{ P \in \pi | P \equiv @x, x \in \meaningof{E} \}}
\end{mathpar}

\begin{eqnarray*}
  \\
  \meaningof{-} : TS \to ST
\end{eqnarray*}

\begin{eqnarray*}
  \\
  L : TS \to ST
\end{eqnarray*}

\begin{eqnarray*}
  \\
  P \models E \iff P \in \meaningof{E}
\end{eqnarray*}

\begin{eqnarray*}
  P \approx_{L} Q \iff \forall E \in L. P \models E \iff Q \models E
\end{eqnarray*}

\begin{eqnarray*}
  P \approx_{K} Q
\end{eqnarray*}

\begin{eqnarray*}
  P \approx Q
\end{eqnarray*}

$\approx_{K} = \approx = \approx_{L}$

\subsubsection{Contextual duality}

Note that contexts extend the quotation operation to a family of
operations from processes to names. Given a context, $M$, we can
define a \emph{nominal context}, $\quotep{M}$ by $\quotep{M}[P] :=
\quotep{M[P]}$. To foreshadow what is to come we observe that these
operations enjoy a duality with processes very much like the duality
between vectors and maps from vectors to scalars.

Further, because the calculus is essentially higher-order, we have a
correspondence between contexts and processes. More specifically,
given a name $x$ and a context $M$ we can construct $M^{*}_{x}$ such
that 

\begin{mathpar}
  M^{*}_{x} | \lift{x}{P} \red M[P]
\end{mathpar}

namely,

\begin{mathpar}
  M^{*}_{x} := x?(u).M[\dropn{u}]
\end{mathpar}

The dependence of $M^{*}_{x}$ on a name makes it an abstraction, 

\begin{mathpar}
  M^{*} := (x)x?(u).M[\dropn{u}]
\end{mathpar}

\subsection{Additional notation}

It will sometimes be convenient to denote the process a name
quotes. We already have the notation $x = \quotep{P}$, but it will be
convenient to introduce an alternate notation, $\procn{x}$, when we
want to emphasize the connection to the use of the name. Note that, by
virtue of name equivalence, $\quotep{\procn{x}} \nameeq x$; so, the
notation is consistent with previous definitions.

Further, because names have structure it is possible to effect
substitutions on the basis of that structure. This means we need to
upgrade our notation for substitutions, which we accomplish by
adapting comprehension notation. Thus,

\begin{mathpar}
  P\{ y / x : x \in S \}
\end{mathpar}

is interpreted to mean the process derived from P by replacing (in a
capture-avoiding manner) each occurrence of $x$ in $S$ by $y$. For example,

\begin{mathpar}
  P\{ \quotep{\procn{x}|\procn{x}} / x : x \in \freenames{P} \}
\end{mathpar}

will replace each (occurrence) of a free name $x$ in $P$ by
$\quotep{\procn{x}|\procn{x}}$.

Also, we will avail ourselves of the notation $x^{L}$ and $x^{R}$ to
denote injections of a name into disjoint copies of the name
space. There are numerous ways to accomplish this. One example can be
found in \cite{MeredithR05}. This notation overloads to vectors of
names: $\vec{x}^{\pi} := (x_{i}^{\pi} \; : \; 0 \leq i < |\vec{x}| )$ where $\pi \in \{L,R\}$.

We also use $P^{\Box} := P|\Box$.

In \cite{MeredithR05} an interpretation of the new operator is
given. It turns out that there are several possible interpretations
all enjoying the requisite algebraic properties of the operator (see
\cite{milner91polyadicpi}). We will therefore make liberal use of
$(\nu\; \vec{x})P$.

% subsection the_syntax_and_semantics_of_the_notation_system (end)   

\input{qm2pi.qmops} 

\input{qm2pi.sterngerlach} 

\input{qm2pi.metric} 

% section concurrent_process_calculi (end)

%\input{qm2pi.proofsketch}

% section proof sketch (end)

%\input{qm2pi.slviaknots} 

% section spatial logic via knots (end)

\input{qm2pi.conclusion}

% section conclusion (end)

%\input{qm2pi.dtcodes} 

% section wiring algorithm (end)

\input{qm2pi.ack} 

% section acknowledgments (end)

\newpage


\bibliographystyle{plain}   
\bibliography{../../biblios/main.bib}

\input{qm2pi.rhodetails}

\end{document}



\end{document}

 

% section notation (end)

\input{qm2pi.process.calculi} 

% section concurrent_process_calculi_and_spatial_logics_ (end)
    
%\documentclass[12pt]{llncs}
%\documentclass{jktr}

\usepackage[pdftex]{hyperref}                   
\usepackage {listings}
\usepackage {mathpartir}
\usepackage{bcprules}
%\usepackage{listings}
                       
\usepackage{graphicx} 
%\usepackage[margins=2.5cm,nohead,nofoot]{geometry}
%\usepackage{geometry}
\usepackage{amsfonts}
\usepackage{amstext}
\usepackage{latexsym}
\usepackage{amssymb}
\usepackage{color}


%\include{myPreamble}
\documentclass[12pt]{llncs}
%\documentclass{jktr}

\usepackage[pdftex]{hyperref}                   
\usepackage {listings}
\usepackage {mathpartir}
\usepackage{bcprules}
%\usepackage{listings}
                       
\usepackage{graphicx} 
%\usepackage[margins=2.5cm,nohead,nofoot]{geometry}
%\usepackage{geometry}
\usepackage{amsfonts}
\usepackage{amstext}
\usepackage{latexsym}
\usepackage{amssymb}
\usepackage{color}


%\include{myPreamble}
\include{qm2pi.local} 

%\ifpdf
%\usepackage[pdftex]{graphicx}
%\else
%\usepackage{graphicx}
%\fi

 % \ifpdf
%  \usepackage{pdfsync}
%  \if


%\title{Brief Article}
%\author{David F. Snyder}
%\author{L.G. Meredith}

%\address{Dept. of Math., Texas State University--San Marcos, San Marcos, TX 78666}
       
\pagestyle{empty}


\begin{document}

\lstset{language=[Objective]Caml,frame=shadowbox}

\input{qm2pi.front}

% section front matter (end)

\input{qm2pi.intro} 
 
% section introduction (end)

% \input{qm2pi.knotations} 

% section notation (end)

\input{qm2pi.process.calculi} 

% section concurrent_process_calculi_and_spatial_logics_ (end)
    
%\input{qm2pi.knots2pi} 

%\input{qm2pi.trefoil} 

%\input{qm2pi.mainthm} 

% subsection basic_interpretation (end)

%\input{qm2pi.rho.presentation} 
\subsection{The syntax and semantics of the notation system}\label{sub:the_syntax_and_semantics_of_the_notation_system} % (fold)

We now summarize a technical presentation of the calculus that
embodies our theory of dynamics. The typical presentation of such a
calculus follows the style of giving generators and relations on
them. The grammar, below, describing term constructors, freely
generates the set of processes, $\Proc$. This set is then quotiented
by a relation known as structural congruence and it is over this set
that the notion of dynamics is expressed. This presentation is
essentially that of \cite{MeredithR05} with the addition of
polyadicity and summation. For readability we have relegated some of
the technical subtleties to an appendix.

\subsubsection{Process grammar}\label{subsub:process_grammar}

\begin{mathpar}
  \inferrule* [lab=synchronization] {} {{M} \bc \pzero \;|\; x?F \;|\; x!C }
  \and
  \inferrule* [lab=abstraction] {} {{F} \bc (x)P}
  \and
  \inferrule* [lab=concretion] {} {{C} \bc \langle Q \rangle}
  \and
  \inferrule* [lab=process] {} {{P,Q} \bc M \;| \;P|Q \;|\; @{x}}
  \and
  \inferrule* [lab=name] {} {{x} \bc \quotep{P}}
\end{mathpar} 

Note that $\vec{x}$ (resp. $\vec{P}$) denotes a vector of names
(resp. processes) of length $|\vec{x}|$ (resp. $|\vec{P}|$). We adopt
the following useful abbreviations.

\begin{mathpar}
   x?(\vec{y}).P := x.(\vec{y})P \and  x\clift{\vec{P}} := x.\clift{\vec{P}}
   \and x!(y) := \lift{x}{\dropn{y}}
   \and \Pi_{i=0}^{n-1}P_i := P_0 | \ldots | P_{n-1}
\end{mathpar}

\subsubsection{Structural congruence}

\paragraph{Free and bound names and alpha-equivalence.} At the
core of structural equivalence is alpha-equivalence which identifies
process that are the same up to a change of variable. Formally, we
recognize the distinction between free and bound names. The free names
of a process, $\freenames{P}$, may be calculated recursively as
follows:

\begin{mathpar}
\freenames{\pzero} := \emptyset
  \and \\
  \freenames{x?(y).P} := \{ x \} \cup (\freenames{P} \setminus \{ y \})
  \and 
  \freenames{x!\langle P \rangle} := \{ x \} \cup \{ P \} 
  \and \\
  \freenames{P|Q} := \freenames{P} \cup \freenames{Q}
  \and \\
  \freenames{@{x}} := \{ x \}
\end{mathpar}

$\pi$
$\quotep{\pi}$

$\freenames{-} : \pi \to \mathcal{P}(\quotep{\pi})$

\begin{eqnarray*}
  \freenames{\pzero} & := & \emptyset \\
  \freenames{x?(y).P} & := & \{ x \} \cup (\freenames{P} \setminus \{ y \}) \\
  \freenames{x!\langle P \rangle} & := & \{ x \} \cup \{ P \} \\
  \freenames{P|Q} & := & \freenames{P} \cup \freenames{Q} \\
  \freenames{\dropn{x}} & := & \{ x \}
\end{eqnarray*}

The bound names of a process, $\boundnames{P}$, are those names occurring in $P$
that are not free. For example, in $x?(y).0$, the name $x$ is free, while $y$ is bound.

\begin{mathpar}
  \inferrule* [lab=monoidal-laws] {} { P|Q \equiv Q|P \and P|0 \equiv P \and P|(Q|R) \equiv (P|Q)|R }
\end{mathpar}

\begin{mathpar}
  \inferrule* [lab=alpha-equivalence] {} { (x)P \equiv (y)P\{y/x\} \and y \not\in \freenames{P} }
\end{mathpar}

\begin{definition}
Then two processes, $P,Q$, are alpha-equivalent if $P = Q\{\vec{y}/\vec{x}\}$ for
some $\vec{x} \in \boundnames{Q},\vec{y} \in \boundnames{P}$, where $Q\{\vec{y}/\vec{x}\}$
denotes the capture-avoiding substitution of $\vec{y}$ for $\vec{x}$ in $Q$.
\end{definition}

\begin{definition}
  The {\em structural congruence} \cite{SangiorgiWalker} , $\equiv$,
  between processes is the least congruence containing
  alpha-equivalence, satisfying the abelian monoid laws
  (associativity, commutativity and $\pzero$ as identity) for parallel
  composition $|$ and for summation $+$.
\end{definition}

\subsection{Name equivalence}

We take name equivalence, written $\nameeq$, to be the smallest
equivalence relation generated by the following rules.

\begin{mathpar}
\inferrule*[lab=Quote-drop]
{ }
{ \quotep{@{x}} \nameeq x }

\inferrule*[lab=Struct-equiv]
{ P \scong Q }
{ \quotep{P} \nameeq \quotep{Q} }
\end{mathpar}

The astute reader will have noticed that the mutual recursion of names
and processes imposes a mutual recursion on alpha-equivalence and
structural equivalence via name-equivalence. Fortunately, all of this
works out pleasantly and we may calculate in the natural way, free of
concern. The reader interested in the details is referred to the
appendix \ref{appendix:rho_details}.

\subsection{Substitution}

We use $\Proc$ for the set of processes, $\QProc$ for the set of
names, and $\id{\{}\vec{y} / \vec{x} \id{\}}$ to denote partial maps,
$s : \QProc \rightarrow \QProc$. A map, $s$ lifts, uniquely, to a map
on process terms, $\widehat{s} : \Proc \rightarrow \Proc$ by the
following equations.

\begin{mathpar}
  (0) \psubstp{Q}{P} := 0 \\
  (R \juxtap S) \psubstp{Q}{P}
  :=    
  (R)\psubstp{Q}{P} \juxtap (S) \psubstp{Q}{P} \\
  (x?(y).R) \psubstp{Q}{P}    
  :=    
  (x)\substp{Q}{P} (z)\concat( (R \psubstn{z}{y}) \psubstp{Q}{P} ) \\
  (\lift{x}{R}) \psubstp{Q}{P}  
  :=
  \lift{(x)\substp{Q}{P}}{ R \psubstp{Q}{P} } \\
%   (\dropn{x})  \psubstp{Q}{P}       
%   := 
%   \left\{ 
%     \begin{array}{ccc} 
%       \dropn{\quotep{Q}} & & x \nameeq \quotep{P} \\
%       \dropn{x} & & otherwise \\
%     \end{array}
%   \right. 
  (\dropn{x})  \psubstp{Q}{P}       
  := 
  \left\{ 
    \begin{array}{ccc} 
      Q & & x \nameeq \quotep{P} \\
      \dropn{x} & & otherwise \\
    \end{array}
  \right.
\end{mathpar}
 

where

\begin{eqnarray}
  (x)\id{\{} \lpquote Q \rpquote / \lpquote P \rpquote \id{\}}            = 
  \left\{ 
    \begin{array}{ccc}
      \lpquote Q \rpquote & & x \nameeq \lpquote P \rpquote \\
      x & & otherwise \\
    \end{array}
  \right. \nonumber
\end{eqnarray}

and $z$ is chosen distinct from $\quotep{P}$, $\quotep{Q}$, the free
names in $Q$, and all the names in $R$. Our $\alpha$-equivalence will
be built in the standard way from this substitution.

\begin{remark}\label{rem:no_self_referential_names}
  One consequence of these definitions is that $\forall P. \quotep{P}
  \not\in \freenames{P}$.
\end{remark}

\subsection{ Dynamic quote: an example }

Anticipating something of what's to come, consider applying the
substitution, $\widehat{\id{\{}u / z \id{\}}}$, to the following pair
of processes, $\lift{w}{y!(z)}$ and $w[ \lpquote y!(z) \rpquote ]$.

\begin{eqnarray}
	\lift{w}{y!(z)}\widehat{\id{\{}u / z \id{\}}}
		& = &
		\lift{w}{y!(u)} \nonumber\\
	w[ \lpquote y!(z) \rpquote ] \widehat{ \id{\{}u / z \id{\}} }
		& = &
		w[ \lpquote y!(z) \rpquote ] \nonumber
\end{eqnarray}

Because the body of the process between quotes is impervious to
substitution, we get radically different answers. In fact, by
examining the first process in an input context,
e.g. $x?(z).\lift{w}{y!(z)}$, we see that the process under the lift
operator may be shaped by prefixed inputs binding a name inside it. In
this sense, the lift operator will be seen as a way to dynamically
construct processes before reifying them as names.

Finally equipped with these standard features we can present the
dynamics of the calculus.

\subsubsection{Operational semantics} 

Finally, we introduce the computational dynamics. What marks these
algebras as distinct from other more traditionally studied algebraic
structures, e.g. vector spaces or polynomial rings, is the manner in
which dynamics is captured. In traditional structures, dynamics is typically
expressed through morphisms between such structures, as in linear maps
between vector spaces or morphisms between rings. In algebras
associated with the semantics of computation, the dynamics is
expressed as part of the algebraic structure itself, through a
reduction reduction relation typically denoted by $\red$. Below, we
give a recursive presentation of this relation for the calculus used
in the encoding.

$\red \subseteq \pi \times \pi$
$\red : \pi \to \mathcal{P}(\pi)$

\begin{mathpar}
  \inferrule* [lab=Comm] { \textsf{match}( x_{src}, x_{trgt} ) } { x_{trgt}?(y)P \; | \; x_{src}!\langle {Q} \rangle \red P\{\quotep{Q}/y}\} }
  \and \\
  \inferrule* [lab=Par] {{P} \red {P}'} {{{P} | {Q}} \red {{P}' | {Q}}}
  \and
  \inferrule* [lab=Equiv]{{{P} \scong {P}'} \andalso {{P}' \red {Q}'} \andalso {{Q}' \scong {Q}}}{{P} \red {Q}}
\end{mathpar}

\begin{eqnarray*}
  match_{\equiv} (\quotep{P},\quotep{Q}) & := & P \equiv Q \\
  match_{\dagger}(\quotep{P},\quotep{Q}) & := & \forall R. P|Q \red^{*} R => R \red^{*} 0 \\
  match_{K}(\quotep{P},\quotep{Q}) & := & K \mbox{ for some context } K
\end{eqnarray*}

$u?(x)P | u!\langle Q \rangle \red P\{\quotep{Q}/x\}$

%We write $\wred$ for $\red^*$, and $P\red$ if $\exists Q $ such that $ P \red Q$.
We write $P\red$ if $\exists Q $ such that $ P \red Q$ and $P\not\red$, otherwise.

\section{Replication}

As mentioned before, it is known that replication (and hence
recursion) can be implemented in a higher-order process algebra
\cite{SangiorgiWalker}. As our first example of calculation with the
machinery thus far presented we give the construction explicitly in
the {\rhoc}.

\begin{eqnarray}
	D_{x} & := & \prefix{x}{y}{(\binpar{\outputp{x}{y}}{@{y}})} \nonumber\\
	\bangp_{x}{P} & := & \binpar{{x}!\langle{\binpar{D_{x}}{P}}\rangle}{D_{x}} \nonumber
\end{eqnarray}

\begin{eqnarray}
	\bangp_{x}{P} & & \nonumber\\
	=
	& {x}!\langle{(\prefix{x}{y}{(\outputp{x}{y} | @{y})) | P}}\rangle 
	      | \prefix{x}{y}{(\outputp{x}{y} | @{y})} & \nonumber\\
	\red
	& (\outputp{x}{y} | @{y})\substn{\quotep{(\prefix{x}{y}{(@{y} | \outputp{x}{y})) | P}}}{y} & \nonumber\\
	=
	& \outputp{x}{\quotep{(\prefix{x}{y}{(\outputp{x}{y} | @{y})) | P}}}
	  | {(\prefix{x}{y}{(\outputp{x}{y} | @{y})) | P}} & \nonumber\\
	\red
	& \ldots & \nonumber\\
	\red^*
	& P | P | \ldots & \nonumber
\end{eqnarray}

Of course, this encoding, as an implementation, runs away, unfolding
$\bangp{P}$ eagerly. A lazier and more implementable replication
operator, restricted to input-guarded processes, may be obtained as follows.

\begin{eqnarray}
\bangp{\prefix{u}{v}{P}} 
	:= 
	\binpar{\lift{x}{\prefix{u}{v}{(\binpar{D(x)}{P})}}}{D(x)} \nonumber
\end{eqnarray}

\begin{remark}
  Note that the lazier definition still does not deal with summation
  or mixed summation (i.e. sums over input and output). The reader is
  invited to construct definitions of replication that deal with these
  features. 

  Further, the definitions are parameterized in a name, $x$. Can you,
  gentle reader, make a definition that eliminates this parameter and
  guarantees no accidental interaction between the replication
  machinery and the process being replicated -- i.e. no accidental
  sharing of names used by the process to get its work done and the
  name(s) used by the replication to effect copying. This latter
  revision of the definition of replication is crucial to obtaining
  the expected identity $!!P \sim !P$.
\end{remark}

\begin{remark}\label{rem:paradoxical_combinator}
  The reader familiar with the lambda calculus will have noticed the
  similarity between $D$ and the paradoxical combinator.

  [Ed. note: the existence of this seems to suggest we have to be more
  restrictive on the set of processes and names we admit if we are to
  support no-cloning.]
\end{remark}

\subsubsection{Bisimulation}

The computational dynamics gives rise to another kind of equivalence,
the equivalence of computational behavior. As previously mentioned
this is typically captured \emph{via} some form of bisimulation.

% The notion we use in this paper is weak barbed bisimulation
% \cite{milner91polyadicpi}.

The notion we use in this paper is derived from weak barbed
bisimulation \cite{milner91polyadicpi}. 

\begin{definition}
An \emph{observation relation}, $\downarrow_{\mathcal N}$, over a set
of names, $\mathcal N$, is the smallest relation satisfying the rules
below.

\infrule[Out-barb]{y \in {\mathcal N}, \; x \nameeq y}
		  {\outputp{x}{v} \downarrow_{\mathcal N} x}
\infrule[Par-barb]{\mbox{$P\downarrow_{\mathcal N} x$ or $Q\downarrow_{\mathcal N} x$}}
		  {\binpar{P}{Q} \downarrow_{\mathcal N} x}

We write $P \Downarrow_{\mathcal N} x$ if there is $Q$ such that 
$P \wred Q$ and $Q \downarrow_{\mathcal N} x$.
\end{definition}

\begin{definition}
%\label{def.bbisim}
An  ${\mathcal N}$-\emph{barbed bisimulation} over a set of names, ${\mathcal N}$, is a symmetric binary relation 
${\mathcal S}_{\mathcal N}$ between agents such that $P\rel{S}_{\mathcal N}Q$ implies:
\begin{enumerate}
\item If $P \red P'$ then $Q \wred Q'$ and $P'\rel{S}_{\mathcal N} Q'$.
\item If $P\downarrow_{\mathcal N} x$, then $Q\Downarrow_{\mathcal N} x$.
\end{enumerate}
$P$ is ${\mathcal N}$-barbed bisimilar to $Q$, written
$P \wbbisim_{\mathcal N} Q$, if $P \rel{S}_{\mathcal N} Q$ for some ${\mathcal N}$-barbed bisimulation ${\mathcal S}_{\mathcal N}$.
\end{definition}

$\mathcal{R} \subseteq \pi \times \pi$

$P \mathcal{R} Q => \forall P'. P \red P' \Rightarrow \exists Q'. Q \red Q', P' \mathcal{R} Q'$

$P \vdash x \Rightarrow Q \vdash x$

\begin{mathpar}
  \inferrule*[lab=Out-barb]{x \nameeq y}{{y}!\langle{Q}\rangle \vdash x}
  \and
  \inferrule*[lab=Par-barb]{\mbox{$P\vdash x$ or $Q\vdash x$}}{\binpar{P}{Q} \vdash x}
\end{mathpar}

\subsubsection{Contexts}

One of the principle advantages of computational calculi like the
$\pi$-calculus is a well-defined notion of context,
contextual-equivalence and a correlation between
contextual-equivalence and notions of bisimulation. The notion of
context allows the decomposition of a process into (sub-)process and
its syntactic environment, its context. Thus, a context may be
thought of as a process with a ``hole'' (written $\Box$) in it. The
application of a context $M$ to a process $P$, written $M[P]$, is
tantamount to filling the hole in $M$ with $P$. In this paper we do
not need the full weight of this theory, but do make use of the notion
of context in the proof the main theorem. 

\begin{mathpar}
  \inferrule* [lab=summation] {} {{M_{M},M_{N}} \bc \Box \;|\; x.M_{A} \;|\; M_{M}+M_{N}}
  \and
  \inferrule* [lab=agent] {} {{M_{A}} \bc (\vec{x})M_{P} \;| \; \clift{P_0,\ldots,M_{P},\ldots,P_N}}
  \and \\
  \inferrule* [lab=process] {} {{M_{P}} \bc M_{N} \;| \;P|M_{P} }
\end{mathpar} 

\begin{mathpar}
  \inferrule* [lab=sychronization] {} {M_{N} \bc \Box \;|\; x?M_{F} \;|\; x!M_{C}}
  \and
  \inferrule* [lab=abstraction] {} {{M_{F}} \bc (x)M_{P} }
  \and
  \inferrule* [lab=concretion] {} {{M_{C}} \bc \langle M_{P} \rangle }
  \and \\
  \inferrule* [lab=process] {} {{M_{P}} \bc M_{N} \;| \;P|M_{P} }
\end{mathpar}

\begin{definition}[contextual application] Given a context $M$, and
  process $P$, we define the \emph{contextual application}, $M[P] :=
  M\{P/\Box\}$. That is, the contextual application of M to P is the
  substitution of $P$ for $\Box$ in $M$.
\end{definition}

$\meaningof{-} : L \to \mathcal{P}(\pi)$

\begin{mathpar}
  \inferrule* [lab=collection] {} {\meaningof{true} = \pi, \and \meaningof{~E} = \pi \setminus \meaningof{E}, \and \meaningof{E_{1} \& E_{2}} = \meaningof{E_{1}} \cap \meaningof{E_{2}}}
\end{mathpar}

\begin{mathpar}
  \inferrule* [lab=structure] {} {\meaningof{0} = \{ P \in \pi | P \equiv 0 \}, \and \\ \meaningof{E_1 | E_2} = \{ P \in \pi | P \equiv P_{1} | P_{2}, P_{1} \in \meaningof{E_{1}}, P_{2} \in \meaningof{E_2}\} }
\end{mathpar}

\begin{mathpar}
 \inferrule* [lab=behavior] {} {\meaningof{\langle a?b \rangle E} = \{ P \in \pi | P \equiv Q | u?(y)P', \\ \and \\\\ \and \\ \;\;\; u \in \meaningof{a}, \forall z.P'\{z/y\} \in \meaningof{E\{z/b\}}\}, \and \\ \meaningof{a!E} = \{ P \in \pi | P \equiv Q | x!\langle P' \rangle, x \in \meaningof{a} P' \in \meaningof{E}\} }
\end{mathpar}

\begin{mathpar}
 \inferrule* [lab=nominal] {} {\meaningof{\quotep{E}} = \{ \quotep{P} \in \quotep{\pi} | P \in \meaningof{E} \}, \and \meaningof{\quotep{P}} = \{ \quotep{Q} \in \quotep{\pi} | P \equiv Q \} \and \\ \meaningof{@\quotep{E}} = \{ P \in \pi | P \equiv @x, x \in \meaningof{E} \}}
\end{mathpar}

\begin{eqnarray*}
  \\
  \meaningof{-} : TS \to ST
\end{eqnarray*}

\begin{eqnarray*}
  \\
  L : TS \to ST
\end{eqnarray*}

\begin{eqnarray*}
  \\
  P \models E \iff P \in \meaningof{E}
\end{eqnarray*}

\begin{eqnarray*}
  P \approx_{L} Q \iff \forall E \in L. P \models E \iff Q \models E
\end{eqnarray*}

\begin{eqnarray*}
  P \approx_{K} Q
\end{eqnarray*}

\begin{eqnarray*}
  P \approx Q
\end{eqnarray*}

$\approx_{K} = \approx = \approx_{L}$

\subsubsection{Contextual duality}

Note that contexts extend the quotation operation to a family of
operations from processes to names. Given a context, $M$, we can
define a \emph{nominal context}, $\quotep{M}$ by $\quotep{M}[P] :=
\quotep{M[P]}$. To foreshadow what is to come we observe that these
operations enjoy a duality with processes very much like the duality
between vectors and maps from vectors to scalars.

Further, because the calculus is essentially higher-order, we have a
correspondence between contexts and processes. More specifically,
given a name $x$ and a context $M$ we can construct $M^{*}_{x}$ such
that 

\begin{mathpar}
  M^{*}_{x} | \lift{x}{P} \red M[P]
\end{mathpar}

namely,

\begin{mathpar}
  M^{*}_{x} := x?(u).M[\dropn{u}]
\end{mathpar}

The dependence of $M^{*}_{x}$ on a name makes it an abstraction, 

\begin{mathpar}
  M^{*} := (x)x?(u).M[\dropn{u}]
\end{mathpar}

\subsection{Additional notation}

It will sometimes be convenient to denote the process a name
quotes. We already have the notation $x = \quotep{P}$, but it will be
convenient to introduce an alternate notation, $\procn{x}$, when we
want to emphasize the connection to the use of the name. Note that, by
virtue of name equivalence, $\quotep{\procn{x}} \nameeq x$; so, the
notation is consistent with previous definitions.

Further, because names have structure it is possible to effect
substitutions on the basis of that structure. This means we need to
upgrade our notation for substitutions, which we accomplish by
adapting comprehension notation. Thus,

\begin{mathpar}
  P\{ y / x : x \in S \}
\end{mathpar}

is interpreted to mean the process derived from P by replacing (in a
capture-avoiding manner) each occurrence of $x$ in $S$ by $y$. For example,

\begin{mathpar}
  P\{ \quotep{\procn{x}|\procn{x}} / x : x \in \freenames{P} \}
\end{mathpar}

will replace each (occurrence) of a free name $x$ in $P$ by
$\quotep{\procn{x}|\procn{x}}$.

Also, we will avail ourselves of the notation $x^{L}$ and $x^{R}$ to
denote injections of a name into disjoint copies of the name
space. There are numerous ways to accomplish this. One example can be
found in \cite{MeredithR05}. This notation overloads to vectors of
names: $\vec{x}^{\pi} := (x_{i}^{\pi} \; : \; 0 \leq i < |\vec{x}| )$ where $\pi \in \{L,R\}$.

We also use $P^{\Box} := P|\Box$.

In \cite{MeredithR05} an interpretation of the new operator is
given. It turns out that there are several possible interpretations
all enjoying the requisite algebraic properties of the operator (see
\cite{milner91polyadicpi}). We will therefore make liberal use of
$(\nu\; \vec{x})P$.

% subsection the_syntax_and_semantics_of_the_notation_system (end)   

\input{qm2pi.qmops} 

\input{qm2pi.sterngerlach} 

\input{qm2pi.metric} 

% section concurrent_process_calculi (end)

%\input{qm2pi.proofsketch}

% section proof sketch (end)

%\input{qm2pi.slviaknots} 

% section spatial logic via knots (end)

\input{qm2pi.conclusion}

% section conclusion (end)

%\input{qm2pi.dtcodes} 

% section wiring algorithm (end)

\input{qm2pi.ack} 

% section acknowledgments (end)

\newpage


\bibliographystyle{plain}   
\bibliography{../../biblios/main.bib}

\input{qm2pi.rhodetails}

\end{document}

 

%\ifpdf
%\usepackage[pdftex]{graphicx}
%\else
%\usepackage{graphicx}
%\fi

 % \ifpdf
%  \usepackage{pdfsync}
%  \if


%\title{Brief Article}
%\author{David F. Snyder}
%\author{L.G. Meredith}

%\address{Dept. of Math., Texas State University--San Marcos, San Marcos, TX 78666}
       
\pagestyle{empty}


\begin{document}

\lstset{language=[Objective]Caml,frame=shadowbox}

\documentclass[12pt]{llncs}
%\documentclass{jktr}

\usepackage[pdftex]{hyperref}                   
\usepackage {listings}
\usepackage {mathpartir}
\usepackage{bcprules}
%\usepackage{listings}
                       
\usepackage{graphicx} 
%\usepackage[margins=2.5cm,nohead,nofoot]{geometry}
%\usepackage{geometry}
\usepackage{amsfonts}
\usepackage{amstext}
\usepackage{latexsym}
\usepackage{amssymb}
\usepackage{color}


%\include{myPreamble}
\include{qm2pi.local} 

%\ifpdf
%\usepackage[pdftex]{graphicx}
%\else
%\usepackage{graphicx}
%\fi

 % \ifpdf
%  \usepackage{pdfsync}
%  \if


%\title{Brief Article}
%\author{David F. Snyder}
%\author{L.G. Meredith}

%\address{Dept. of Math., Texas State University--San Marcos, San Marcos, TX 78666}
       
\pagestyle{empty}


\begin{document}

\lstset{language=[Objective]Caml,frame=shadowbox}

\input{qm2pi.front}

% section front matter (end)

\input{qm2pi.intro} 
 
% section introduction (end)

% \input{qm2pi.knotations} 

% section notation (end)

\input{qm2pi.process.calculi} 

% section concurrent_process_calculi_and_spatial_logics_ (end)
    
%\input{qm2pi.knots2pi} 

%\input{qm2pi.trefoil} 

%\input{qm2pi.mainthm} 

% subsection basic_interpretation (end)

%\input{qm2pi.rho.presentation} 
\subsection{The syntax and semantics of the notation system}\label{sub:the_syntax_and_semantics_of_the_notation_system} % (fold)

We now summarize a technical presentation of the calculus that
embodies our theory of dynamics. The typical presentation of such a
calculus follows the style of giving generators and relations on
them. The grammar, below, describing term constructors, freely
generates the set of processes, $\Proc$. This set is then quotiented
by a relation known as structural congruence and it is over this set
that the notion of dynamics is expressed. This presentation is
essentially that of \cite{MeredithR05} with the addition of
polyadicity and summation. For readability we have relegated some of
the technical subtleties to an appendix.

\subsubsection{Process grammar}\label{subsub:process_grammar}

\begin{mathpar}
  \inferrule* [lab=synchronization] {} {{M} \bc \pzero \;|\; x?F \;|\; x!C }
  \and
  \inferrule* [lab=abstraction] {} {{F} \bc (x)P}
  \and
  \inferrule* [lab=concretion] {} {{C} \bc \langle Q \rangle}
  \and
  \inferrule* [lab=process] {} {{P,Q} \bc M \;| \;P|Q \;|\; @{x}}
  \and
  \inferrule* [lab=name] {} {{x} \bc \quotep{P}}
\end{mathpar} 

Note that $\vec{x}$ (resp. $\vec{P}$) denotes a vector of names
(resp. processes) of length $|\vec{x}|$ (resp. $|\vec{P}|$). We adopt
the following useful abbreviations.

\begin{mathpar}
   x?(\vec{y}).P := x.(\vec{y})P \and  x\clift{\vec{P}} := x.\clift{\vec{P}}
   \and x!(y) := \lift{x}{\dropn{y}}
   \and \Pi_{i=0}^{n-1}P_i := P_0 | \ldots | P_{n-1}
\end{mathpar}

\subsubsection{Structural congruence}

\paragraph{Free and bound names and alpha-equivalence.} At the
core of structural equivalence is alpha-equivalence which identifies
process that are the same up to a change of variable. Formally, we
recognize the distinction between free and bound names. The free names
of a process, $\freenames{P}$, may be calculated recursively as
follows:

\begin{mathpar}
\freenames{\pzero} := \emptyset
  \and \\
  \freenames{x?(y).P} := \{ x \} \cup (\freenames{P} \setminus \{ y \})
  \and 
  \freenames{x!\langle P \rangle} := \{ x \} \cup \{ P \} 
  \and \\
  \freenames{P|Q} := \freenames{P} \cup \freenames{Q}
  \and \\
  \freenames{@{x}} := \{ x \}
\end{mathpar}

$\pi$
$\quotep{\pi}$

$\freenames{-} : \pi \to \mathcal{P}(\quotep{\pi})$

\begin{eqnarray*}
  \freenames{\pzero} & := & \emptyset \\
  \freenames{x?(y).P} & := & \{ x \} \cup (\freenames{P} \setminus \{ y \}) \\
  \freenames{x!\langle P \rangle} & := & \{ x \} \cup \{ P \} \\
  \freenames{P|Q} & := & \freenames{P} \cup \freenames{Q} \\
  \freenames{\dropn{x}} & := & \{ x \}
\end{eqnarray*}

The bound names of a process, $\boundnames{P}$, are those names occurring in $P$
that are not free. For example, in $x?(y).0$, the name $x$ is free, while $y$ is bound.

\begin{mathpar}
  \inferrule* [lab=monoidal-laws] {} { P|Q \equiv Q|P \and P|0 \equiv P \and P|(Q|R) \equiv (P|Q)|R }
\end{mathpar}

\begin{mathpar}
  \inferrule* [lab=alpha-equivalence] {} { (x)P \equiv (y)P\{y/x\} \and y \not\in \freenames{P} }
\end{mathpar}

\begin{definition}
Then two processes, $P,Q$, are alpha-equivalent if $P = Q\{\vec{y}/\vec{x}\}$ for
some $\vec{x} \in \boundnames{Q},\vec{y} \in \boundnames{P}$, where $Q\{\vec{y}/\vec{x}\}$
denotes the capture-avoiding substitution of $\vec{y}$ for $\vec{x}$ in $Q$.
\end{definition}

\begin{definition}
  The {\em structural congruence} \cite{SangiorgiWalker} , $\equiv$,
  between processes is the least congruence containing
  alpha-equivalence, satisfying the abelian monoid laws
  (associativity, commutativity and $\pzero$ as identity) for parallel
  composition $|$ and for summation $+$.
\end{definition}

\subsection{Name equivalence}

We take name equivalence, written $\nameeq$, to be the smallest
equivalence relation generated by the following rules.

\begin{mathpar}
\inferrule*[lab=Quote-drop]
{ }
{ \quotep{@{x}} \nameeq x }

\inferrule*[lab=Struct-equiv]
{ P \scong Q }
{ \quotep{P} \nameeq \quotep{Q} }
\end{mathpar}

The astute reader will have noticed that the mutual recursion of names
and processes imposes a mutual recursion on alpha-equivalence and
structural equivalence via name-equivalence. Fortunately, all of this
works out pleasantly and we may calculate in the natural way, free of
concern. The reader interested in the details is referred to the
appendix \ref{appendix:rho_details}.

\subsection{Substitution}

We use $\Proc$ for the set of processes, $\QProc$ for the set of
names, and $\id{\{}\vec{y} / \vec{x} \id{\}}$ to denote partial maps,
$s : \QProc \rightarrow \QProc$. A map, $s$ lifts, uniquely, to a map
on process terms, $\widehat{s} : \Proc \rightarrow \Proc$ by the
following equations.

\begin{mathpar}
  (0) \psubstp{Q}{P} := 0 \\
  (R \juxtap S) \psubstp{Q}{P}
  :=    
  (R)\psubstp{Q}{P} \juxtap (S) \psubstp{Q}{P} \\
  (x?(y).R) \psubstp{Q}{P}    
  :=    
  (x)\substp{Q}{P} (z)\concat( (R \psubstn{z}{y}) \psubstp{Q}{P} ) \\
  (\lift{x}{R}) \psubstp{Q}{P}  
  :=
  \lift{(x)\substp{Q}{P}}{ R \psubstp{Q}{P} } \\
%   (\dropn{x})  \psubstp{Q}{P}       
%   := 
%   \left\{ 
%     \begin{array}{ccc} 
%       \dropn{\quotep{Q}} & & x \nameeq \quotep{P} \\
%       \dropn{x} & & otherwise \\
%     \end{array}
%   \right. 
  (\dropn{x})  \psubstp{Q}{P}       
  := 
  \left\{ 
    \begin{array}{ccc} 
      Q & & x \nameeq \quotep{P} \\
      \dropn{x} & & otherwise \\
    \end{array}
  \right.
\end{mathpar}
 

where

\begin{eqnarray}
  (x)\id{\{} \lpquote Q \rpquote / \lpquote P \rpquote \id{\}}            = 
  \left\{ 
    \begin{array}{ccc}
      \lpquote Q \rpquote & & x \nameeq \lpquote P \rpquote \\
      x & & otherwise \\
    \end{array}
  \right. \nonumber
\end{eqnarray}

and $z$ is chosen distinct from $\quotep{P}$, $\quotep{Q}$, the free
names in $Q$, and all the names in $R$. Our $\alpha$-equivalence will
be built in the standard way from this substitution.

\begin{remark}\label{rem:no_self_referential_names}
  One consequence of these definitions is that $\forall P. \quotep{P}
  \not\in \freenames{P}$.
\end{remark}

\subsection{ Dynamic quote: an example }

Anticipating something of what's to come, consider applying the
substitution, $\widehat{\id{\{}u / z \id{\}}}$, to the following pair
of processes, $\lift{w}{y!(z)}$ and $w[ \lpquote y!(z) \rpquote ]$.

\begin{eqnarray}
	\lift{w}{y!(z)}\widehat{\id{\{}u / z \id{\}}}
		& = &
		\lift{w}{y!(u)} \nonumber\\
	w[ \lpquote y!(z) \rpquote ] \widehat{ \id{\{}u / z \id{\}} }
		& = &
		w[ \lpquote y!(z) \rpquote ] \nonumber
\end{eqnarray}

Because the body of the process between quotes is impervious to
substitution, we get radically different answers. In fact, by
examining the first process in an input context,
e.g. $x?(z).\lift{w}{y!(z)}$, we see that the process under the lift
operator may be shaped by prefixed inputs binding a name inside it. In
this sense, the lift operator will be seen as a way to dynamically
construct processes before reifying them as names.

Finally equipped with these standard features we can present the
dynamics of the calculus.

\subsubsection{Operational semantics} 

Finally, we introduce the computational dynamics. What marks these
algebras as distinct from other more traditionally studied algebraic
structures, e.g. vector spaces or polynomial rings, is the manner in
which dynamics is captured. In traditional structures, dynamics is typically
expressed through morphisms between such structures, as in linear maps
between vector spaces or morphisms between rings. In algebras
associated with the semantics of computation, the dynamics is
expressed as part of the algebraic structure itself, through a
reduction reduction relation typically denoted by $\red$. Below, we
give a recursive presentation of this relation for the calculus used
in the encoding.

$\red \subseteq \pi \times \pi$
$\red : \pi \to \mathcal{P}(\pi)$

\begin{mathpar}
  \inferrule* [lab=Comm] { \textsf{match}( x_{src}, x_{trgt} ) } { x_{trgt}?(y)P \; | \; x_{src}!\langle {Q} \rangle \red P\{\quotep{Q}/y}\} }
  \and \\
  \inferrule* [lab=Par] {{P} \red {P}'} {{{P} | {Q}} \red {{P}' | {Q}}}
  \and
  \inferrule* [lab=Equiv]{{{P} \scong {P}'} \andalso {{P}' \red {Q}'} \andalso {{Q}' \scong {Q}}}{{P} \red {Q}}
\end{mathpar}

\begin{eqnarray*}
  match_{\equiv} (\quotep{P},\quotep{Q}) & := & P \equiv Q \\
  match_{\dagger}(\quotep{P},\quotep{Q}) & := & \forall R. P|Q \red^{*} R => R \red^{*} 0 \\
  match_{K}(\quotep{P},\quotep{Q}) & := & K \mbox{ for some context } K
\end{eqnarray*}

$u?(x)P | u!\langle Q \rangle \red P\{\quotep{Q}/x\}$

%We write $\wred$ for $\red^*$, and $P\red$ if $\exists Q $ such that $ P \red Q$.
We write $P\red$ if $\exists Q $ such that $ P \red Q$ and $P\not\red$, otherwise.

\section{Replication}

As mentioned before, it is known that replication (and hence
recursion) can be implemented in a higher-order process algebra
\cite{SangiorgiWalker}. As our first example of calculation with the
machinery thus far presented we give the construction explicitly in
the {\rhoc}.

\begin{eqnarray}
	D_{x} & := & \prefix{x}{y}{(\binpar{\outputp{x}{y}}{@{y}})} \nonumber\\
	\bangp_{x}{P} & := & \binpar{{x}!\langle{\binpar{D_{x}}{P}}\rangle}{D_{x}} \nonumber
\end{eqnarray}

\begin{eqnarray}
	\bangp_{x}{P} & & \nonumber\\
	=
	& {x}!\langle{(\prefix{x}{y}{(\outputp{x}{y} | @{y})) | P}}\rangle 
	      | \prefix{x}{y}{(\outputp{x}{y} | @{y})} & \nonumber\\
	\red
	& (\outputp{x}{y} | @{y})\substn{\quotep{(\prefix{x}{y}{(@{y} | \outputp{x}{y})) | P}}}{y} & \nonumber\\
	=
	& \outputp{x}{\quotep{(\prefix{x}{y}{(\outputp{x}{y} | @{y})) | P}}}
	  | {(\prefix{x}{y}{(\outputp{x}{y} | @{y})) | P}} & \nonumber\\
	\red
	& \ldots & \nonumber\\
	\red^*
	& P | P | \ldots & \nonumber
\end{eqnarray}

Of course, this encoding, as an implementation, runs away, unfolding
$\bangp{P}$ eagerly. A lazier and more implementable replication
operator, restricted to input-guarded processes, may be obtained as follows.

\begin{eqnarray}
\bangp{\prefix{u}{v}{P}} 
	:= 
	\binpar{\lift{x}{\prefix{u}{v}{(\binpar{D(x)}{P})}}}{D(x)} \nonumber
\end{eqnarray}

\begin{remark}
  Note that the lazier definition still does not deal with summation
  or mixed summation (i.e. sums over input and output). The reader is
  invited to construct definitions of replication that deal with these
  features. 

  Further, the definitions are parameterized in a name, $x$. Can you,
  gentle reader, make a definition that eliminates this parameter and
  guarantees no accidental interaction between the replication
  machinery and the process being replicated -- i.e. no accidental
  sharing of names used by the process to get its work done and the
  name(s) used by the replication to effect copying. This latter
  revision of the definition of replication is crucial to obtaining
  the expected identity $!!P \sim !P$.
\end{remark}

\begin{remark}\label{rem:paradoxical_combinator}
  The reader familiar with the lambda calculus will have noticed the
  similarity between $D$ and the paradoxical combinator.

  [Ed. note: the existence of this seems to suggest we have to be more
  restrictive on the set of processes and names we admit if we are to
  support no-cloning.]
\end{remark}

\subsubsection{Bisimulation}

The computational dynamics gives rise to another kind of equivalence,
the equivalence of computational behavior. As previously mentioned
this is typically captured \emph{via} some form of bisimulation.

% The notion we use in this paper is weak barbed bisimulation
% \cite{milner91polyadicpi}.

The notion we use in this paper is derived from weak barbed
bisimulation \cite{milner91polyadicpi}. 

\begin{definition}
An \emph{observation relation}, $\downarrow_{\mathcal N}$, over a set
of names, $\mathcal N$, is the smallest relation satisfying the rules
below.

\infrule[Out-barb]{y \in {\mathcal N}, \; x \nameeq y}
		  {\outputp{x}{v} \downarrow_{\mathcal N} x}
\infrule[Par-barb]{\mbox{$P\downarrow_{\mathcal N} x$ or $Q\downarrow_{\mathcal N} x$}}
		  {\binpar{P}{Q} \downarrow_{\mathcal N} x}

We write $P \Downarrow_{\mathcal N} x$ if there is $Q$ such that 
$P \wred Q$ and $Q \downarrow_{\mathcal N} x$.
\end{definition}

\begin{definition}
%\label{def.bbisim}
An  ${\mathcal N}$-\emph{barbed bisimulation} over a set of names, ${\mathcal N}$, is a symmetric binary relation 
${\mathcal S}_{\mathcal N}$ between agents such that $P\rel{S}_{\mathcal N}Q$ implies:
\begin{enumerate}
\item If $P \red P'$ then $Q \wred Q'$ and $P'\rel{S}_{\mathcal N} Q'$.
\item If $P\downarrow_{\mathcal N} x$, then $Q\Downarrow_{\mathcal N} x$.
\end{enumerate}
$P$ is ${\mathcal N}$-barbed bisimilar to $Q$, written
$P \wbbisim_{\mathcal N} Q$, if $P \rel{S}_{\mathcal N} Q$ for some ${\mathcal N}$-barbed bisimulation ${\mathcal S}_{\mathcal N}$.
\end{definition}

$\mathcal{R} \subseteq \pi \times \pi$

$P \mathcal{R} Q => \forall P'. P \red P' \Rightarrow \exists Q'. Q \red Q', P' \mathcal{R} Q'$

$P \vdash x \Rightarrow Q \vdash x$

\begin{mathpar}
  \inferrule*[lab=Out-barb]{x \nameeq y}{{y}!\langle{Q}\rangle \vdash x}
  \and
  \inferrule*[lab=Par-barb]{\mbox{$P\vdash x$ or $Q\vdash x$}}{\binpar{P}{Q} \vdash x}
\end{mathpar}

\subsubsection{Contexts}

One of the principle advantages of computational calculi like the
$\pi$-calculus is a well-defined notion of context,
contextual-equivalence and a correlation between
contextual-equivalence and notions of bisimulation. The notion of
context allows the decomposition of a process into (sub-)process and
its syntactic environment, its context. Thus, a context may be
thought of as a process with a ``hole'' (written $\Box$) in it. The
application of a context $M$ to a process $P$, written $M[P]$, is
tantamount to filling the hole in $M$ with $P$. In this paper we do
not need the full weight of this theory, but do make use of the notion
of context in the proof the main theorem. 

\begin{mathpar}
  \inferrule* [lab=summation] {} {{M_{M},M_{N}} \bc \Box \;|\; x.M_{A} \;|\; M_{M}+M_{N}}
  \and
  \inferrule* [lab=agent] {} {{M_{A}} \bc (\vec{x})M_{P} \;| \; \clift{P_0,\ldots,M_{P},\ldots,P_N}}
  \and \\
  \inferrule* [lab=process] {} {{M_{P}} \bc M_{N} \;| \;P|M_{P} }
\end{mathpar} 

\begin{mathpar}
  \inferrule* [lab=sychronization] {} {M_{N} \bc \Box \;|\; x?M_{F} \;|\; x!M_{C}}
  \and
  \inferrule* [lab=abstraction] {} {{M_{F}} \bc (x)M_{P} }
  \and
  \inferrule* [lab=concretion] {} {{M_{C}} \bc \langle M_{P} \rangle }
  \and \\
  \inferrule* [lab=process] {} {{M_{P}} \bc M_{N} \;| \;P|M_{P} }
\end{mathpar}

\begin{definition}[contextual application] Given a context $M$, and
  process $P$, we define the \emph{contextual application}, $M[P] :=
  M\{P/\Box\}$. That is, the contextual application of M to P is the
  substitution of $P$ for $\Box$ in $M$.
\end{definition}

$\meaningof{-} : L \to \mathcal{P}(\pi)$

\begin{mathpar}
  \inferrule* [lab=collection] {} {\meaningof{true} = \pi, \and \meaningof{~E} = \pi \setminus \meaningof{E}, \and \meaningof{E_{1} \& E_{2}} = \meaningof{E_{1}} \cap \meaningof{E_{2}}}
\end{mathpar}

\begin{mathpar}
  \inferrule* [lab=structure] {} {\meaningof{0} = \{ P \in \pi | P \equiv 0 \}, \and \\ \meaningof{E_1 | E_2} = \{ P \in \pi | P \equiv P_{1} | P_{2}, P_{1} \in \meaningof{E_{1}}, P_{2} \in \meaningof{E_2}\} }
\end{mathpar}

\begin{mathpar}
 \inferrule* [lab=behavior] {} {\meaningof{\langle a?b \rangle E} = \{ P \in \pi | P \equiv Q | u?(y)P', \\ \and \\\\ \and \\ \;\;\; u \in \meaningof{a}, \forall z.P'\{z/y\} \in \meaningof{E\{z/b\}}\}, \and \\ \meaningof{a!E} = \{ P \in \pi | P \equiv Q | x!\langle P' \rangle, x \in \meaningof{a} P' \in \meaningof{E}\} }
\end{mathpar}

\begin{mathpar}
 \inferrule* [lab=nominal] {} {\meaningof{\quotep{E}} = \{ \quotep{P} \in \quotep{\pi} | P \in \meaningof{E} \}, \and \meaningof{\quotep{P}} = \{ \quotep{Q} \in \quotep{\pi} | P \equiv Q \} \and \\ \meaningof{@\quotep{E}} = \{ P \in \pi | P \equiv @x, x \in \meaningof{E} \}}
\end{mathpar}

\begin{eqnarray*}
  \\
  \meaningof{-} : TS \to ST
\end{eqnarray*}

\begin{eqnarray*}
  \\
  L : TS \to ST
\end{eqnarray*}

\begin{eqnarray*}
  \\
  P \models E \iff P \in \meaningof{E}
\end{eqnarray*}

\begin{eqnarray*}
  P \approx_{L} Q \iff \forall E \in L. P \models E \iff Q \models E
\end{eqnarray*}

\begin{eqnarray*}
  P \approx_{K} Q
\end{eqnarray*}

\begin{eqnarray*}
  P \approx Q
\end{eqnarray*}

$\approx_{K} = \approx = \approx_{L}$

\subsubsection{Contextual duality}

Note that contexts extend the quotation operation to a family of
operations from processes to names. Given a context, $M$, we can
define a \emph{nominal context}, $\quotep{M}$ by $\quotep{M}[P] :=
\quotep{M[P]}$. To foreshadow what is to come we observe that these
operations enjoy a duality with processes very much like the duality
between vectors and maps from vectors to scalars.

Further, because the calculus is essentially higher-order, we have a
correspondence between contexts and processes. More specifically,
given a name $x$ and a context $M$ we can construct $M^{*}_{x}$ such
that 

\begin{mathpar}
  M^{*}_{x} | \lift{x}{P} \red M[P]
\end{mathpar}

namely,

\begin{mathpar}
  M^{*}_{x} := x?(u).M[\dropn{u}]
\end{mathpar}

The dependence of $M^{*}_{x}$ on a name makes it an abstraction, 

\begin{mathpar}
  M^{*} := (x)x?(u).M[\dropn{u}]
\end{mathpar}

\subsection{Additional notation}

It will sometimes be convenient to denote the process a name
quotes. We already have the notation $x = \quotep{P}$, but it will be
convenient to introduce an alternate notation, $\procn{x}$, when we
want to emphasize the connection to the use of the name. Note that, by
virtue of name equivalence, $\quotep{\procn{x}} \nameeq x$; so, the
notation is consistent with previous definitions.

Further, because names have structure it is possible to effect
substitutions on the basis of that structure. This means we need to
upgrade our notation for substitutions, which we accomplish by
adapting comprehension notation. Thus,

\begin{mathpar}
  P\{ y / x : x \in S \}
\end{mathpar}

is interpreted to mean the process derived from P by replacing (in a
capture-avoiding manner) each occurrence of $x$ in $S$ by $y$. For example,

\begin{mathpar}
  P\{ \quotep{\procn{x}|\procn{x}} / x : x \in \freenames{P} \}
\end{mathpar}

will replace each (occurrence) of a free name $x$ in $P$ by
$\quotep{\procn{x}|\procn{x}}$.

Also, we will avail ourselves of the notation $x^{L}$ and $x^{R}$ to
denote injections of a name into disjoint copies of the name
space. There are numerous ways to accomplish this. One example can be
found in \cite{MeredithR05}. This notation overloads to vectors of
names: $\vec{x}^{\pi} := (x_{i}^{\pi} \; : \; 0 \leq i < |\vec{x}| )$ where $\pi \in \{L,R\}$.

We also use $P^{\Box} := P|\Box$.

In \cite{MeredithR05} an interpretation of the new operator is
given. It turns out that there are several possible interpretations
all enjoying the requisite algebraic properties of the operator (see
\cite{milner91polyadicpi}). We will therefore make liberal use of
$(\nu\; \vec{x})P$.

% subsection the_syntax_and_semantics_of_the_notation_system (end)   

\input{qm2pi.qmops} 

\input{qm2pi.sterngerlach} 

\input{qm2pi.metric} 

% section concurrent_process_calculi (end)

%\input{qm2pi.proofsketch}

% section proof sketch (end)

%\input{qm2pi.slviaknots} 

% section spatial logic via knots (end)

\input{qm2pi.conclusion}

% section conclusion (end)

%\input{qm2pi.dtcodes} 

% section wiring algorithm (end)

\input{qm2pi.ack} 

% section acknowledgments (end)

\newpage


\bibliographystyle{plain}   
\bibliography{../../biblios/main.bib}

\input{qm2pi.rhodetails}

\end{document}



% section front matter (end)

\section{Introduction}\label{sec:introduction} % (fold)
In this draft of the material i am going to have to dispense with the
usual writing conventions adopted in papers on these topics. i'm going
to have adopt whatever tone i need at the time i'm writing up the
calculations. Sometimes this may be very conversational; others it may
be the barest mathematical grunts; others still it may be that i have
lifted text from one of my other papers because the exposition of some
point was better said there. i hope that my readers are not unduly put
out by this decision. i'm not doing this to flout convention or be
rebellious. i find these calculations very technically challenging. To
keep everything going technically, something has to give; i have to
let go of some cognitive burden. So, the academic writing style --
with all of its trade-offs in terms of facilitating technical
communication -- is what i'm letting go of. Perhaps subsequent drafts
can be tightened and polished, but for now, i'm going to speak as if
we were sitting together in a coffee shop with a laptop, wifi and a
pad of paper and a pencil.

So, here's what i have to say. We -- you and i, comfortably ensconced
in our coffee shop and well-equipped with our tools -- can realize and
carry out the calculations of quantum mechanics over a very different
formal theory of dynamics, a formal theory of dynamics that
corresponds to a theory of concurrent computation with
\emph{reflection}. It has the advantage that the underlying theory is
already `quantized', but supports analogues all of the continuuous
operations. Strikingly, this underlying theory has recently been
connected with a notion of metric that we can show, by calculating
together, coincides with the metric induced by the inner product.

There are a lot of reasons why you might be interested in seeing
calculations of this form. Here's why i'm interested. For the past
several centuries there has been no competitor to the ``Newtonian''
account of dynamics. As a result the predominant share of accounts of
dynamical systems and situations have had to be formulated in terms of
the Newtonian machinery. i view this as an intellectually dangerous
position to occupy. Everything, despite it's intrinsic shape, turns
into a nail to be hit with this hammer. Recently, however, the theory
of computation has matured to the point where we have candidates for
theories of dynamics that offer very different perspective on
reasoning about dynamical systems and situations. Testing these
candidates against very successful accounts of dynamical situations,
like quantum mechanics, is going to give us some sense of how mature
they are and some measure of the quality of these accounts of
dynamics.

\subsection{Summary of contributions and outline of paper}

So, we're going to develop an interpretation of the operations of
quantum mechanics normally interpreted by Hilbert spaces and
operators. We're going to do this over a theory of computation. Note
that this is very different than the usual quantum computation program
which develops notions of computation over quantum mechanics. Rather,
we are developing a story that aligns with Wheeler's slogan: It from
Bit. To do this we will first provide an account of the theory of
computation at play here. Then we will dive into a calculation-driven
interpretation of the operations of quantum mechanics.

The reason we take this approach is that -- until very recently --
there hasn't been an axiomatic account of quantum mechanics. As a
result there has been no sharp delineation of the mathematical theory
supporting interpretation of the physical theory and the physical
theory, itself. So, ambient features of the maths are free to be
exploited (or supressed) without a real accounting of their physical
relevance. There is no sharp statement ``here's the physical theory''
qua \emph{theory} and ``here's the mathematical interpretation''
enabling a judgment of how faithful the interpretation is -- apart
from experimental observation. When there is an axiomatic account we
can judge how well a given mathematical formalism supports an
interpretation of the axioms, independent of
experimentation. Likewise, we can judge how well we have captured our
physical evidence and experience with our axiomatics, independent of
any specific mathematical implementation, with accidental detail that
may or may not have physical significance. 

In lieu of a fully fleshed out and vetted axiomatic account of quantum
mechanics, interpreting the operational notions in service of modeling
physical systems will have to suffice. In other words, we are not in
the business of providing a model of Hilbert spaces and operators. We
are in the business of providing a model of quantum mechanics because
we are motivated by testing our notions of dynamics against physical
theory; and, the predictive calculations of the physical theory must
serve as the best formulation -- shy of a fully fleshed out axiomatic
account -- of the physical theory itself (as they have for scientific
theories since time immemorial). Put another way, despite a
whole-hearted commitment to an It-from-Bit ontology, we are firmly
aligned with the shut-up-and-calculate camp as the best way to obtain
results either from the physical perspective or as a quality assurance
measure of our fledgling theory of dynamics.

In detail, we present a reflective process calculus. Then we develop
intuitive correspondences between the notions available in this
calculus and the usual physical notions supporting quantum mechanical
calculations. Thus, 

\begin{table}[htp]
  \center{
    \fbox{
      \begin{tabular}{c|c}
        quantum mechanics & process calculus \\
        \hline
        scalar & name \\
        state vector & process \\
        dual & contextual duals \\
        matrix & formal sums of process-context-dual pairs \\
        orthogonality & process annihilation \\
        inner product & execution-formula + quoting
      \end{tabular}
    }
  }
  \caption{QM - process calculi correspondences}
\end{table}

Then we tighten up these intuitions to operational definitions. We
employ the Dirac notation as the best proxy we can find for an
abstract syntax of the quantum mechanical notions. The definitions we
develop put us in contact with equational constraints coming from the
theory that we demonstrate the definitions and calculations satisfy.

This puts us in a position to shut up and calculate for the
Stern-Gerlach experimental set up, showing how these predictive
calculations become calculations on processes in our theory of a
reflective process calculus.

Penultimately, we demonstrate that the notion of metric coming from
the inner product coincides with the notion of metric available from
the theory of bisimulation. This demonstration gives us the right to
think of space as arising from behavior. Finally, we consider where we
might go from the new vantage point we have obtained.

% section introduction (end) 
 
% section introduction (end)

% \documentclass[12pt]{llncs}
%\documentclass{jktr}

\usepackage[pdftex]{hyperref}                   
\usepackage {listings}
\usepackage {mathpartir}
\usepackage{bcprules}
%\usepackage{listings}
                       
\usepackage{graphicx} 
%\usepackage[margins=2.5cm,nohead,nofoot]{geometry}
%\usepackage{geometry}
\usepackage{amsfonts}
\usepackage{amstext}
\usepackage{latexsym}
\usepackage{amssymb}
\usepackage{color}


%\include{myPreamble}
\include{qm2pi.local} 

%\ifpdf
%\usepackage[pdftex]{graphicx}
%\else
%\usepackage{graphicx}
%\fi

 % \ifpdf
%  \usepackage{pdfsync}
%  \if


%\title{Brief Article}
%\author{David F. Snyder}
%\author{L.G. Meredith}

%\address{Dept. of Math., Texas State University--San Marcos, San Marcos, TX 78666}
       
\pagestyle{empty}


\begin{document}

\lstset{language=[Objective]Caml,frame=shadowbox}

\input{qm2pi.front}

% section front matter (end)

\input{qm2pi.intro} 
 
% section introduction (end)

% \input{qm2pi.knotations} 

% section notation (end)

\input{qm2pi.process.calculi} 

% section concurrent_process_calculi_and_spatial_logics_ (end)
    
%\input{qm2pi.knots2pi} 

%\input{qm2pi.trefoil} 

%\input{qm2pi.mainthm} 

% subsection basic_interpretation (end)

%\input{qm2pi.rho.presentation} 
\subsection{The syntax and semantics of the notation system}\label{sub:the_syntax_and_semantics_of_the_notation_system} % (fold)

We now summarize a technical presentation of the calculus that
embodies our theory of dynamics. The typical presentation of such a
calculus follows the style of giving generators and relations on
them. The grammar, below, describing term constructors, freely
generates the set of processes, $\Proc$. This set is then quotiented
by a relation known as structural congruence and it is over this set
that the notion of dynamics is expressed. This presentation is
essentially that of \cite{MeredithR05} with the addition of
polyadicity and summation. For readability we have relegated some of
the technical subtleties to an appendix.

\subsubsection{Process grammar}\label{subsub:process_grammar}

\begin{mathpar}
  \inferrule* [lab=synchronization] {} {{M} \bc \pzero \;|\; x?F \;|\; x!C }
  \and
  \inferrule* [lab=abstraction] {} {{F} \bc (x)P}
  \and
  \inferrule* [lab=concretion] {} {{C} \bc \langle Q \rangle}
  \and
  \inferrule* [lab=process] {} {{P,Q} \bc M \;| \;P|Q \;|\; @{x}}
  \and
  \inferrule* [lab=name] {} {{x} \bc \quotep{P}}
\end{mathpar} 

Note that $\vec{x}$ (resp. $\vec{P}$) denotes a vector of names
(resp. processes) of length $|\vec{x}|$ (resp. $|\vec{P}|$). We adopt
the following useful abbreviations.

\begin{mathpar}
   x?(\vec{y}).P := x.(\vec{y})P \and  x\clift{\vec{P}} := x.\clift{\vec{P}}
   \and x!(y) := \lift{x}{\dropn{y}}
   \and \Pi_{i=0}^{n-1}P_i := P_0 | \ldots | P_{n-1}
\end{mathpar}

\subsubsection{Structural congruence}

\paragraph{Free and bound names and alpha-equivalence.} At the
core of structural equivalence is alpha-equivalence which identifies
process that are the same up to a change of variable. Formally, we
recognize the distinction between free and bound names. The free names
of a process, $\freenames{P}$, may be calculated recursively as
follows:

\begin{mathpar}
\freenames{\pzero} := \emptyset
  \and \\
  \freenames{x?(y).P} := \{ x \} \cup (\freenames{P} \setminus \{ y \})
  \and 
  \freenames{x!\langle P \rangle} := \{ x \} \cup \{ P \} 
  \and \\
  \freenames{P|Q} := \freenames{P} \cup \freenames{Q}
  \and \\
  \freenames{@{x}} := \{ x \}
\end{mathpar}

$\pi$
$\quotep{\pi}$

$\freenames{-} : \pi \to \mathcal{P}(\quotep{\pi})$

\begin{eqnarray*}
  \freenames{\pzero} & := & \emptyset \\
  \freenames{x?(y).P} & := & \{ x \} \cup (\freenames{P} \setminus \{ y \}) \\
  \freenames{x!\langle P \rangle} & := & \{ x \} \cup \{ P \} \\
  \freenames{P|Q} & := & \freenames{P} \cup \freenames{Q} \\
  \freenames{\dropn{x}} & := & \{ x \}
\end{eqnarray*}

The bound names of a process, $\boundnames{P}$, are those names occurring in $P$
that are not free. For example, in $x?(y).0$, the name $x$ is free, while $y$ is bound.

\begin{mathpar}
  \inferrule* [lab=monoidal-laws] {} { P|Q \equiv Q|P \and P|0 \equiv P \and P|(Q|R) \equiv (P|Q)|R }
\end{mathpar}

\begin{mathpar}
  \inferrule* [lab=alpha-equivalence] {} { (x)P \equiv (y)P\{y/x\} \and y \not\in \freenames{P} }
\end{mathpar}

\begin{definition}
Then two processes, $P,Q$, are alpha-equivalent if $P = Q\{\vec{y}/\vec{x}\}$ for
some $\vec{x} \in \boundnames{Q},\vec{y} \in \boundnames{P}$, where $Q\{\vec{y}/\vec{x}\}$
denotes the capture-avoiding substitution of $\vec{y}$ for $\vec{x}$ in $Q$.
\end{definition}

\begin{definition}
  The {\em structural congruence} \cite{SangiorgiWalker} , $\equiv$,
  between processes is the least congruence containing
  alpha-equivalence, satisfying the abelian monoid laws
  (associativity, commutativity and $\pzero$ as identity) for parallel
  composition $|$ and for summation $+$.
\end{definition}

\subsection{Name equivalence}

We take name equivalence, written $\nameeq$, to be the smallest
equivalence relation generated by the following rules.

\begin{mathpar}
\inferrule*[lab=Quote-drop]
{ }
{ \quotep{@{x}} \nameeq x }

\inferrule*[lab=Struct-equiv]
{ P \scong Q }
{ \quotep{P} \nameeq \quotep{Q} }
\end{mathpar}

The astute reader will have noticed that the mutual recursion of names
and processes imposes a mutual recursion on alpha-equivalence and
structural equivalence via name-equivalence. Fortunately, all of this
works out pleasantly and we may calculate in the natural way, free of
concern. The reader interested in the details is referred to the
appendix \ref{appendix:rho_details}.

\subsection{Substitution}

We use $\Proc$ for the set of processes, $\QProc$ for the set of
names, and $\id{\{}\vec{y} / \vec{x} \id{\}}$ to denote partial maps,
$s : \QProc \rightarrow \QProc$. A map, $s$ lifts, uniquely, to a map
on process terms, $\widehat{s} : \Proc \rightarrow \Proc$ by the
following equations.

\begin{mathpar}
  (0) \psubstp{Q}{P} := 0 \\
  (R \juxtap S) \psubstp{Q}{P}
  :=    
  (R)\psubstp{Q}{P} \juxtap (S) \psubstp{Q}{P} \\
  (x?(y).R) \psubstp{Q}{P}    
  :=    
  (x)\substp{Q}{P} (z)\concat( (R \psubstn{z}{y}) \psubstp{Q}{P} ) \\
  (\lift{x}{R}) \psubstp{Q}{P}  
  :=
  \lift{(x)\substp{Q}{P}}{ R \psubstp{Q}{P} } \\
%   (\dropn{x})  \psubstp{Q}{P}       
%   := 
%   \left\{ 
%     \begin{array}{ccc} 
%       \dropn{\quotep{Q}} & & x \nameeq \quotep{P} \\
%       \dropn{x} & & otherwise \\
%     \end{array}
%   \right. 
  (\dropn{x})  \psubstp{Q}{P}       
  := 
  \left\{ 
    \begin{array}{ccc} 
      Q & & x \nameeq \quotep{P} \\
      \dropn{x} & & otherwise \\
    \end{array}
  \right.
\end{mathpar}
 

where

\begin{eqnarray}
  (x)\id{\{} \lpquote Q \rpquote / \lpquote P \rpquote \id{\}}            = 
  \left\{ 
    \begin{array}{ccc}
      \lpquote Q \rpquote & & x \nameeq \lpquote P \rpquote \\
      x & & otherwise \\
    \end{array}
  \right. \nonumber
\end{eqnarray}

and $z$ is chosen distinct from $\quotep{P}$, $\quotep{Q}$, the free
names in $Q$, and all the names in $R$. Our $\alpha$-equivalence will
be built in the standard way from this substitution.

\begin{remark}\label{rem:no_self_referential_names}
  One consequence of these definitions is that $\forall P. \quotep{P}
  \not\in \freenames{P}$.
\end{remark}

\subsection{ Dynamic quote: an example }

Anticipating something of what's to come, consider applying the
substitution, $\widehat{\id{\{}u / z \id{\}}}$, to the following pair
of processes, $\lift{w}{y!(z)}$ and $w[ \lpquote y!(z) \rpquote ]$.

\begin{eqnarray}
	\lift{w}{y!(z)}\widehat{\id{\{}u / z \id{\}}}
		& = &
		\lift{w}{y!(u)} \nonumber\\
	w[ \lpquote y!(z) \rpquote ] \widehat{ \id{\{}u / z \id{\}} }
		& = &
		w[ \lpquote y!(z) \rpquote ] \nonumber
\end{eqnarray}

Because the body of the process between quotes is impervious to
substitution, we get radically different answers. In fact, by
examining the first process in an input context,
e.g. $x?(z).\lift{w}{y!(z)}$, we see that the process under the lift
operator may be shaped by prefixed inputs binding a name inside it. In
this sense, the lift operator will be seen as a way to dynamically
construct processes before reifying them as names.

Finally equipped with these standard features we can present the
dynamics of the calculus.

\subsubsection{Operational semantics} 

Finally, we introduce the computational dynamics. What marks these
algebras as distinct from other more traditionally studied algebraic
structures, e.g. vector spaces or polynomial rings, is the manner in
which dynamics is captured. In traditional structures, dynamics is typically
expressed through morphisms between such structures, as in linear maps
between vector spaces or morphisms between rings. In algebras
associated with the semantics of computation, the dynamics is
expressed as part of the algebraic structure itself, through a
reduction reduction relation typically denoted by $\red$. Below, we
give a recursive presentation of this relation for the calculus used
in the encoding.

$\red \subseteq \pi \times \pi$
$\red : \pi \to \mathcal{P}(\pi)$

\begin{mathpar}
  \inferrule* [lab=Comm] { \textsf{match}( x_{src}, x_{trgt} ) } { x_{trgt}?(y)P \; | \; x_{src}!\langle {Q} \rangle \red P\{\quotep{Q}/y}\} }
  \and \\
  \inferrule* [lab=Par] {{P} \red {P}'} {{{P} | {Q}} \red {{P}' | {Q}}}
  \and
  \inferrule* [lab=Equiv]{{{P} \scong {P}'} \andalso {{P}' \red {Q}'} \andalso {{Q}' \scong {Q}}}{{P} \red {Q}}
\end{mathpar}

\begin{eqnarray*}
  match_{\equiv} (\quotep{P},\quotep{Q}) & := & P \equiv Q \\
  match_{\dagger}(\quotep{P},\quotep{Q}) & := & \forall R. P|Q \red^{*} R => R \red^{*} 0 \\
  match_{K}(\quotep{P},\quotep{Q}) & := & K \mbox{ for some context } K
\end{eqnarray*}

$u?(x)P | u!\langle Q \rangle \red P\{\quotep{Q}/x\}$

%We write $\wred$ for $\red^*$, and $P\red$ if $\exists Q $ such that $ P \red Q$.
We write $P\red$ if $\exists Q $ such that $ P \red Q$ and $P\not\red$, otherwise.

\section{Replication}

As mentioned before, it is known that replication (and hence
recursion) can be implemented in a higher-order process algebra
\cite{SangiorgiWalker}. As our first example of calculation with the
machinery thus far presented we give the construction explicitly in
the {\rhoc}.

\begin{eqnarray}
	D_{x} & := & \prefix{x}{y}{(\binpar{\outputp{x}{y}}{@{y}})} \nonumber\\
	\bangp_{x}{P} & := & \binpar{{x}!\langle{\binpar{D_{x}}{P}}\rangle}{D_{x}} \nonumber
\end{eqnarray}

\begin{eqnarray}
	\bangp_{x}{P} & & \nonumber\\
	=
	& {x}!\langle{(\prefix{x}{y}{(\outputp{x}{y} | @{y})) | P}}\rangle 
	      | \prefix{x}{y}{(\outputp{x}{y} | @{y})} & \nonumber\\
	\red
	& (\outputp{x}{y} | @{y})\substn{\quotep{(\prefix{x}{y}{(@{y} | \outputp{x}{y})) | P}}}{y} & \nonumber\\
	=
	& \outputp{x}{\quotep{(\prefix{x}{y}{(\outputp{x}{y} | @{y})) | P}}}
	  | {(\prefix{x}{y}{(\outputp{x}{y} | @{y})) | P}} & \nonumber\\
	\red
	& \ldots & \nonumber\\
	\red^*
	& P | P | \ldots & \nonumber
\end{eqnarray}

Of course, this encoding, as an implementation, runs away, unfolding
$\bangp{P}$ eagerly. A lazier and more implementable replication
operator, restricted to input-guarded processes, may be obtained as follows.

\begin{eqnarray}
\bangp{\prefix{u}{v}{P}} 
	:= 
	\binpar{\lift{x}{\prefix{u}{v}{(\binpar{D(x)}{P})}}}{D(x)} \nonumber
\end{eqnarray}

\begin{remark}
  Note that the lazier definition still does not deal with summation
  or mixed summation (i.e. sums over input and output). The reader is
  invited to construct definitions of replication that deal with these
  features. 

  Further, the definitions are parameterized in a name, $x$. Can you,
  gentle reader, make a definition that eliminates this parameter and
  guarantees no accidental interaction between the replication
  machinery and the process being replicated -- i.e. no accidental
  sharing of names used by the process to get its work done and the
  name(s) used by the replication to effect copying. This latter
  revision of the definition of replication is crucial to obtaining
  the expected identity $!!P \sim !P$.
\end{remark}

\begin{remark}\label{rem:paradoxical_combinator}
  The reader familiar with the lambda calculus will have noticed the
  similarity between $D$ and the paradoxical combinator.

  [Ed. note: the existence of this seems to suggest we have to be more
  restrictive on the set of processes and names we admit if we are to
  support no-cloning.]
\end{remark}

\subsubsection{Bisimulation}

The computational dynamics gives rise to another kind of equivalence,
the equivalence of computational behavior. As previously mentioned
this is typically captured \emph{via} some form of bisimulation.

% The notion we use in this paper is weak barbed bisimulation
% \cite{milner91polyadicpi}.

The notion we use in this paper is derived from weak barbed
bisimulation \cite{milner91polyadicpi}. 

\begin{definition}
An \emph{observation relation}, $\downarrow_{\mathcal N}$, over a set
of names, $\mathcal N$, is the smallest relation satisfying the rules
below.

\infrule[Out-barb]{y \in {\mathcal N}, \; x \nameeq y}
		  {\outputp{x}{v} \downarrow_{\mathcal N} x}
\infrule[Par-barb]{\mbox{$P\downarrow_{\mathcal N} x$ or $Q\downarrow_{\mathcal N} x$}}
		  {\binpar{P}{Q} \downarrow_{\mathcal N} x}

We write $P \Downarrow_{\mathcal N} x$ if there is $Q$ such that 
$P \wred Q$ and $Q \downarrow_{\mathcal N} x$.
\end{definition}

\begin{definition}
%\label{def.bbisim}
An  ${\mathcal N}$-\emph{barbed bisimulation} over a set of names, ${\mathcal N}$, is a symmetric binary relation 
${\mathcal S}_{\mathcal N}$ between agents such that $P\rel{S}_{\mathcal N}Q$ implies:
\begin{enumerate}
\item If $P \red P'$ then $Q \wred Q'$ and $P'\rel{S}_{\mathcal N} Q'$.
\item If $P\downarrow_{\mathcal N} x$, then $Q\Downarrow_{\mathcal N} x$.
\end{enumerate}
$P$ is ${\mathcal N}$-barbed bisimilar to $Q$, written
$P \wbbisim_{\mathcal N} Q$, if $P \rel{S}_{\mathcal N} Q$ for some ${\mathcal N}$-barbed bisimulation ${\mathcal S}_{\mathcal N}$.
\end{definition}

$\mathcal{R} \subseteq \pi \times \pi$

$P \mathcal{R} Q => \forall P'. P \red P' \Rightarrow \exists Q'. Q \red Q', P' \mathcal{R} Q'$

$P \vdash x \Rightarrow Q \vdash x$

\begin{mathpar}
  \inferrule*[lab=Out-barb]{x \nameeq y}{{y}!\langle{Q}\rangle \vdash x}
  \and
  \inferrule*[lab=Par-barb]{\mbox{$P\vdash x$ or $Q\vdash x$}}{\binpar{P}{Q} \vdash x}
\end{mathpar}

\subsubsection{Contexts}

One of the principle advantages of computational calculi like the
$\pi$-calculus is a well-defined notion of context,
contextual-equivalence and a correlation between
contextual-equivalence and notions of bisimulation. The notion of
context allows the decomposition of a process into (sub-)process and
its syntactic environment, its context. Thus, a context may be
thought of as a process with a ``hole'' (written $\Box$) in it. The
application of a context $M$ to a process $P$, written $M[P]$, is
tantamount to filling the hole in $M$ with $P$. In this paper we do
not need the full weight of this theory, but do make use of the notion
of context in the proof the main theorem. 

\begin{mathpar}
  \inferrule* [lab=summation] {} {{M_{M},M_{N}} \bc \Box \;|\; x.M_{A} \;|\; M_{M}+M_{N}}
  \and
  \inferrule* [lab=agent] {} {{M_{A}} \bc (\vec{x})M_{P} \;| \; \clift{P_0,\ldots,M_{P},\ldots,P_N}}
  \and \\
  \inferrule* [lab=process] {} {{M_{P}} \bc M_{N} \;| \;P|M_{P} }
\end{mathpar} 

\begin{mathpar}
  \inferrule* [lab=sychronization] {} {M_{N} \bc \Box \;|\; x?M_{F} \;|\; x!M_{C}}
  \and
  \inferrule* [lab=abstraction] {} {{M_{F}} \bc (x)M_{P} }
  \and
  \inferrule* [lab=concretion] {} {{M_{C}} \bc \langle M_{P} \rangle }
  \and \\
  \inferrule* [lab=process] {} {{M_{P}} \bc M_{N} \;| \;P|M_{P} }
\end{mathpar}

\begin{definition}[contextual application] Given a context $M$, and
  process $P$, we define the \emph{contextual application}, $M[P] :=
  M\{P/\Box\}$. That is, the contextual application of M to P is the
  substitution of $P$ for $\Box$ in $M$.
\end{definition}

$\meaningof{-} : L \to \mathcal{P}(\pi)$

\begin{mathpar}
  \inferrule* [lab=collection] {} {\meaningof{true} = \pi, \and \meaningof{~E} = \pi \setminus \meaningof{E}, \and \meaningof{E_{1} \& E_{2}} = \meaningof{E_{1}} \cap \meaningof{E_{2}}}
\end{mathpar}

\begin{mathpar}
  \inferrule* [lab=structure] {} {\meaningof{0} = \{ P \in \pi | P \equiv 0 \}, \and \\ \meaningof{E_1 | E_2} = \{ P \in \pi | P \equiv P_{1} | P_{2}, P_{1} \in \meaningof{E_{1}}, P_{2} \in \meaningof{E_2}\} }
\end{mathpar}

\begin{mathpar}
 \inferrule* [lab=behavior] {} {\meaningof{\langle a?b \rangle E} = \{ P \in \pi | P \equiv Q | u?(y)P', \\ \and \\\\ \and \\ \;\;\; u \in \meaningof{a}, \forall z.P'\{z/y\} \in \meaningof{E\{z/b\}}\}, \and \\ \meaningof{a!E} = \{ P \in \pi | P \equiv Q | x!\langle P' \rangle, x \in \meaningof{a} P' \in \meaningof{E}\} }
\end{mathpar}

\begin{mathpar}
 \inferrule* [lab=nominal] {} {\meaningof{\quotep{E}} = \{ \quotep{P} \in \quotep{\pi} | P \in \meaningof{E} \}, \and \meaningof{\quotep{P}} = \{ \quotep{Q} \in \quotep{\pi} | P \equiv Q \} \and \\ \meaningof{@\quotep{E}} = \{ P \in \pi | P \equiv @x, x \in \meaningof{E} \}}
\end{mathpar}

\begin{eqnarray*}
  \\
  \meaningof{-} : TS \to ST
\end{eqnarray*}

\begin{eqnarray*}
  \\
  L : TS \to ST
\end{eqnarray*}

\begin{eqnarray*}
  \\
  P \models E \iff P \in \meaningof{E}
\end{eqnarray*}

\begin{eqnarray*}
  P \approx_{L} Q \iff \forall E \in L. P \models E \iff Q \models E
\end{eqnarray*}

\begin{eqnarray*}
  P \approx_{K} Q
\end{eqnarray*}

\begin{eqnarray*}
  P \approx Q
\end{eqnarray*}

$\approx_{K} = \approx = \approx_{L}$

\subsubsection{Contextual duality}

Note that contexts extend the quotation operation to a family of
operations from processes to names. Given a context, $M$, we can
define a \emph{nominal context}, $\quotep{M}$ by $\quotep{M}[P] :=
\quotep{M[P]}$. To foreshadow what is to come we observe that these
operations enjoy a duality with processes very much like the duality
between vectors and maps from vectors to scalars.

Further, because the calculus is essentially higher-order, we have a
correspondence between contexts and processes. More specifically,
given a name $x$ and a context $M$ we can construct $M^{*}_{x}$ such
that 

\begin{mathpar}
  M^{*}_{x} | \lift{x}{P} \red M[P]
\end{mathpar}

namely,

\begin{mathpar}
  M^{*}_{x} := x?(u).M[\dropn{u}]
\end{mathpar}

The dependence of $M^{*}_{x}$ on a name makes it an abstraction, 

\begin{mathpar}
  M^{*} := (x)x?(u).M[\dropn{u}]
\end{mathpar}

\subsection{Additional notation}

It will sometimes be convenient to denote the process a name
quotes. We already have the notation $x = \quotep{P}$, but it will be
convenient to introduce an alternate notation, $\procn{x}$, when we
want to emphasize the connection to the use of the name. Note that, by
virtue of name equivalence, $\quotep{\procn{x}} \nameeq x$; so, the
notation is consistent with previous definitions.

Further, because names have structure it is possible to effect
substitutions on the basis of that structure. This means we need to
upgrade our notation for substitutions, which we accomplish by
adapting comprehension notation. Thus,

\begin{mathpar}
  P\{ y / x : x \in S \}
\end{mathpar}

is interpreted to mean the process derived from P by replacing (in a
capture-avoiding manner) each occurrence of $x$ in $S$ by $y$. For example,

\begin{mathpar}
  P\{ \quotep{\procn{x}|\procn{x}} / x : x \in \freenames{P} \}
\end{mathpar}

will replace each (occurrence) of a free name $x$ in $P$ by
$\quotep{\procn{x}|\procn{x}}$.

Also, we will avail ourselves of the notation $x^{L}$ and $x^{R}$ to
denote injections of a name into disjoint copies of the name
space. There are numerous ways to accomplish this. One example can be
found in \cite{MeredithR05}. This notation overloads to vectors of
names: $\vec{x}^{\pi} := (x_{i}^{\pi} \; : \; 0 \leq i < |\vec{x}| )$ where $\pi \in \{L,R\}$.

We also use $P^{\Box} := P|\Box$.

In \cite{MeredithR05} an interpretation of the new operator is
given. It turns out that there are several possible interpretations
all enjoying the requisite algebraic properties of the operator (see
\cite{milner91polyadicpi}). We will therefore make liberal use of
$(\nu\; \vec{x})P$.

% subsection the_syntax_and_semantics_of_the_notation_system (end)   

\input{qm2pi.qmops} 

\input{qm2pi.sterngerlach} 

\input{qm2pi.metric} 

% section concurrent_process_calculi (end)

%\input{qm2pi.proofsketch}

% section proof sketch (end)

%\input{qm2pi.slviaknots} 

% section spatial logic via knots (end)

\input{qm2pi.conclusion}

% section conclusion (end)

%\input{qm2pi.dtcodes} 

% section wiring algorithm (end)

\input{qm2pi.ack} 

% section acknowledgments (end)

\newpage


\bibliographystyle{plain}   
\bibliography{../../biblios/main.bib}

\input{qm2pi.rhodetails}

\end{document}

 

% section notation (end)

\input{qm2pi.process.calculi} 

% section concurrent_process_calculi_and_spatial_logics_ (end)
    
%\documentclass[12pt]{llncs}
%\documentclass{jktr}

\usepackage[pdftex]{hyperref}                   
\usepackage {listings}
\usepackage {mathpartir}
\usepackage{bcprules}
%\usepackage{listings}
                       
\usepackage{graphicx} 
%\usepackage[margins=2.5cm,nohead,nofoot]{geometry}
%\usepackage{geometry}
\usepackage{amsfonts}
\usepackage{amstext}
\usepackage{latexsym}
\usepackage{amssymb}
\usepackage{color}


%\include{myPreamble}
\include{qm2pi.local} 

%\ifpdf
%\usepackage[pdftex]{graphicx}
%\else
%\usepackage{graphicx}
%\fi

 % \ifpdf
%  \usepackage{pdfsync}
%  \if


%\title{Brief Article}
%\author{David F. Snyder}
%\author{L.G. Meredith}

%\address{Dept. of Math., Texas State University--San Marcos, San Marcos, TX 78666}
       
\pagestyle{empty}


\begin{document}

\lstset{language=[Objective]Caml,frame=shadowbox}

\input{qm2pi.front}

% section front matter (end)

\input{qm2pi.intro} 
 
% section introduction (end)

% \input{qm2pi.knotations} 

% section notation (end)

\input{qm2pi.process.calculi} 

% section concurrent_process_calculi_and_spatial_logics_ (end)
    
%\input{qm2pi.knots2pi} 

%\input{qm2pi.trefoil} 

%\input{qm2pi.mainthm} 

% subsection basic_interpretation (end)

%\input{qm2pi.rho.presentation} 
\subsection{The syntax and semantics of the notation system}\label{sub:the_syntax_and_semantics_of_the_notation_system} % (fold)

We now summarize a technical presentation of the calculus that
embodies our theory of dynamics. The typical presentation of such a
calculus follows the style of giving generators and relations on
them. The grammar, below, describing term constructors, freely
generates the set of processes, $\Proc$. This set is then quotiented
by a relation known as structural congruence and it is over this set
that the notion of dynamics is expressed. This presentation is
essentially that of \cite{MeredithR05} with the addition of
polyadicity and summation. For readability we have relegated some of
the technical subtleties to an appendix.

\subsubsection{Process grammar}\label{subsub:process_grammar}

\begin{mathpar}
  \inferrule* [lab=synchronization] {} {{M} \bc \pzero \;|\; x?F \;|\; x!C }
  \and
  \inferrule* [lab=abstraction] {} {{F} \bc (x)P}
  \and
  \inferrule* [lab=concretion] {} {{C} \bc \langle Q \rangle}
  \and
  \inferrule* [lab=process] {} {{P,Q} \bc M \;| \;P|Q \;|\; @{x}}
  \and
  \inferrule* [lab=name] {} {{x} \bc \quotep{P}}
\end{mathpar} 

Note that $\vec{x}$ (resp. $\vec{P}$) denotes a vector of names
(resp. processes) of length $|\vec{x}|$ (resp. $|\vec{P}|$). We adopt
the following useful abbreviations.

\begin{mathpar}
   x?(\vec{y}).P := x.(\vec{y})P \and  x\clift{\vec{P}} := x.\clift{\vec{P}}
   \and x!(y) := \lift{x}{\dropn{y}}
   \and \Pi_{i=0}^{n-1}P_i := P_0 | \ldots | P_{n-1}
\end{mathpar}

\subsubsection{Structural congruence}

\paragraph{Free and bound names and alpha-equivalence.} At the
core of structural equivalence is alpha-equivalence which identifies
process that are the same up to a change of variable. Formally, we
recognize the distinction between free and bound names. The free names
of a process, $\freenames{P}$, may be calculated recursively as
follows:

\begin{mathpar}
\freenames{\pzero} := \emptyset
  \and \\
  \freenames{x?(y).P} := \{ x \} \cup (\freenames{P} \setminus \{ y \})
  \and 
  \freenames{x!\langle P \rangle} := \{ x \} \cup \{ P \} 
  \and \\
  \freenames{P|Q} := \freenames{P} \cup \freenames{Q}
  \and \\
  \freenames{@{x}} := \{ x \}
\end{mathpar}

$\pi$
$\quotep{\pi}$

$\freenames{-} : \pi \to \mathcal{P}(\quotep{\pi})$

\begin{eqnarray*}
  \freenames{\pzero} & := & \emptyset \\
  \freenames{x?(y).P} & := & \{ x \} \cup (\freenames{P} \setminus \{ y \}) \\
  \freenames{x!\langle P \rangle} & := & \{ x \} \cup \{ P \} \\
  \freenames{P|Q} & := & \freenames{P} \cup \freenames{Q} \\
  \freenames{\dropn{x}} & := & \{ x \}
\end{eqnarray*}

The bound names of a process, $\boundnames{P}$, are those names occurring in $P$
that are not free. For example, in $x?(y).0$, the name $x$ is free, while $y$ is bound.

\begin{mathpar}
  \inferrule* [lab=monoidal-laws] {} { P|Q \equiv Q|P \and P|0 \equiv P \and P|(Q|R) \equiv (P|Q)|R }
\end{mathpar}

\begin{mathpar}
  \inferrule* [lab=alpha-equivalence] {} { (x)P \equiv (y)P\{y/x\} \and y \not\in \freenames{P} }
\end{mathpar}

\begin{definition}
Then two processes, $P,Q$, are alpha-equivalent if $P = Q\{\vec{y}/\vec{x}\}$ for
some $\vec{x} \in \boundnames{Q},\vec{y} \in \boundnames{P}$, where $Q\{\vec{y}/\vec{x}\}$
denotes the capture-avoiding substitution of $\vec{y}$ for $\vec{x}$ in $Q$.
\end{definition}

\begin{definition}
  The {\em structural congruence} \cite{SangiorgiWalker} , $\equiv$,
  between processes is the least congruence containing
  alpha-equivalence, satisfying the abelian monoid laws
  (associativity, commutativity and $\pzero$ as identity) for parallel
  composition $|$ and for summation $+$.
\end{definition}

\subsection{Name equivalence}

We take name equivalence, written $\nameeq$, to be the smallest
equivalence relation generated by the following rules.

\begin{mathpar}
\inferrule*[lab=Quote-drop]
{ }
{ \quotep{@{x}} \nameeq x }

\inferrule*[lab=Struct-equiv]
{ P \scong Q }
{ \quotep{P} \nameeq \quotep{Q} }
\end{mathpar}

The astute reader will have noticed that the mutual recursion of names
and processes imposes a mutual recursion on alpha-equivalence and
structural equivalence via name-equivalence. Fortunately, all of this
works out pleasantly and we may calculate in the natural way, free of
concern. The reader interested in the details is referred to the
appendix \ref{appendix:rho_details}.

\subsection{Substitution}

We use $\Proc$ for the set of processes, $\QProc$ for the set of
names, and $\id{\{}\vec{y} / \vec{x} \id{\}}$ to denote partial maps,
$s : \QProc \rightarrow \QProc$. A map, $s$ lifts, uniquely, to a map
on process terms, $\widehat{s} : \Proc \rightarrow \Proc$ by the
following equations.

\begin{mathpar}
  (0) \psubstp{Q}{P} := 0 \\
  (R \juxtap S) \psubstp{Q}{P}
  :=    
  (R)\psubstp{Q}{P} \juxtap (S) \psubstp{Q}{P} \\
  (x?(y).R) \psubstp{Q}{P}    
  :=    
  (x)\substp{Q}{P} (z)\concat( (R \psubstn{z}{y}) \psubstp{Q}{P} ) \\
  (\lift{x}{R}) \psubstp{Q}{P}  
  :=
  \lift{(x)\substp{Q}{P}}{ R \psubstp{Q}{P} } \\
%   (\dropn{x})  \psubstp{Q}{P}       
%   := 
%   \left\{ 
%     \begin{array}{ccc} 
%       \dropn{\quotep{Q}} & & x \nameeq \quotep{P} \\
%       \dropn{x} & & otherwise \\
%     \end{array}
%   \right. 
  (\dropn{x})  \psubstp{Q}{P}       
  := 
  \left\{ 
    \begin{array}{ccc} 
      Q & & x \nameeq \quotep{P} \\
      \dropn{x} & & otherwise \\
    \end{array}
  \right.
\end{mathpar}
 

where

\begin{eqnarray}
  (x)\id{\{} \lpquote Q \rpquote / \lpquote P \rpquote \id{\}}            = 
  \left\{ 
    \begin{array}{ccc}
      \lpquote Q \rpquote & & x \nameeq \lpquote P \rpquote \\
      x & & otherwise \\
    \end{array}
  \right. \nonumber
\end{eqnarray}

and $z$ is chosen distinct from $\quotep{P}$, $\quotep{Q}$, the free
names in $Q$, and all the names in $R$. Our $\alpha$-equivalence will
be built in the standard way from this substitution.

\begin{remark}\label{rem:no_self_referential_names}
  One consequence of these definitions is that $\forall P. \quotep{P}
  \not\in \freenames{P}$.
\end{remark}

\subsection{ Dynamic quote: an example }

Anticipating something of what's to come, consider applying the
substitution, $\widehat{\id{\{}u / z \id{\}}}$, to the following pair
of processes, $\lift{w}{y!(z)}$ and $w[ \lpquote y!(z) \rpquote ]$.

\begin{eqnarray}
	\lift{w}{y!(z)}\widehat{\id{\{}u / z \id{\}}}
		& = &
		\lift{w}{y!(u)} \nonumber\\
	w[ \lpquote y!(z) \rpquote ] \widehat{ \id{\{}u / z \id{\}} }
		& = &
		w[ \lpquote y!(z) \rpquote ] \nonumber
\end{eqnarray}

Because the body of the process between quotes is impervious to
substitution, we get radically different answers. In fact, by
examining the first process in an input context,
e.g. $x?(z).\lift{w}{y!(z)}$, we see that the process under the lift
operator may be shaped by prefixed inputs binding a name inside it. In
this sense, the lift operator will be seen as a way to dynamically
construct processes before reifying them as names.

Finally equipped with these standard features we can present the
dynamics of the calculus.

\subsubsection{Operational semantics} 

Finally, we introduce the computational dynamics. What marks these
algebras as distinct from other more traditionally studied algebraic
structures, e.g. vector spaces or polynomial rings, is the manner in
which dynamics is captured. In traditional structures, dynamics is typically
expressed through morphisms between such structures, as in linear maps
between vector spaces or morphisms between rings. In algebras
associated with the semantics of computation, the dynamics is
expressed as part of the algebraic structure itself, through a
reduction reduction relation typically denoted by $\red$. Below, we
give a recursive presentation of this relation for the calculus used
in the encoding.

$\red \subseteq \pi \times \pi$
$\red : \pi \to \mathcal{P}(\pi)$

\begin{mathpar}
  \inferrule* [lab=Comm] { \textsf{match}( x_{src}, x_{trgt} ) } { x_{trgt}?(y)P \; | \; x_{src}!\langle {Q} \rangle \red P\{\quotep{Q}/y}\} }
  \and \\
  \inferrule* [lab=Par] {{P} \red {P}'} {{{P} | {Q}} \red {{P}' | {Q}}}
  \and
  \inferrule* [lab=Equiv]{{{P} \scong {P}'} \andalso {{P}' \red {Q}'} \andalso {{Q}' \scong {Q}}}{{P} \red {Q}}
\end{mathpar}

\begin{eqnarray*}
  match_{\equiv} (\quotep{P},\quotep{Q}) & := & P \equiv Q \\
  match_{\dagger}(\quotep{P},\quotep{Q}) & := & \forall R. P|Q \red^{*} R => R \red^{*} 0 \\
  match_{K}(\quotep{P},\quotep{Q}) & := & K \mbox{ for some context } K
\end{eqnarray*}

$u?(x)P | u!\langle Q \rangle \red P\{\quotep{Q}/x\}$

%We write $\wred$ for $\red^*$, and $P\red$ if $\exists Q $ such that $ P \red Q$.
We write $P\red$ if $\exists Q $ such that $ P \red Q$ and $P\not\red$, otherwise.

\section{Replication}

As mentioned before, it is known that replication (and hence
recursion) can be implemented in a higher-order process algebra
\cite{SangiorgiWalker}. As our first example of calculation with the
machinery thus far presented we give the construction explicitly in
the {\rhoc}.

\begin{eqnarray}
	D_{x} & := & \prefix{x}{y}{(\binpar{\outputp{x}{y}}{@{y}})} \nonumber\\
	\bangp_{x}{P} & := & \binpar{{x}!\langle{\binpar{D_{x}}{P}}\rangle}{D_{x}} \nonumber
\end{eqnarray}

\begin{eqnarray}
	\bangp_{x}{P} & & \nonumber\\
	=
	& {x}!\langle{(\prefix{x}{y}{(\outputp{x}{y} | @{y})) | P}}\rangle 
	      | \prefix{x}{y}{(\outputp{x}{y} | @{y})} & \nonumber\\
	\red
	& (\outputp{x}{y} | @{y})\substn{\quotep{(\prefix{x}{y}{(@{y} | \outputp{x}{y})) | P}}}{y} & \nonumber\\
	=
	& \outputp{x}{\quotep{(\prefix{x}{y}{(\outputp{x}{y} | @{y})) | P}}}
	  | {(\prefix{x}{y}{(\outputp{x}{y} | @{y})) | P}} & \nonumber\\
	\red
	& \ldots & \nonumber\\
	\red^*
	& P | P | \ldots & \nonumber
\end{eqnarray}

Of course, this encoding, as an implementation, runs away, unfolding
$\bangp{P}$ eagerly. A lazier and more implementable replication
operator, restricted to input-guarded processes, may be obtained as follows.

\begin{eqnarray}
\bangp{\prefix{u}{v}{P}} 
	:= 
	\binpar{\lift{x}{\prefix{u}{v}{(\binpar{D(x)}{P})}}}{D(x)} \nonumber
\end{eqnarray}

\begin{remark}
  Note that the lazier definition still does not deal with summation
  or mixed summation (i.e. sums over input and output). The reader is
  invited to construct definitions of replication that deal with these
  features. 

  Further, the definitions are parameterized in a name, $x$. Can you,
  gentle reader, make a definition that eliminates this parameter and
  guarantees no accidental interaction between the replication
  machinery and the process being replicated -- i.e. no accidental
  sharing of names used by the process to get its work done and the
  name(s) used by the replication to effect copying. This latter
  revision of the definition of replication is crucial to obtaining
  the expected identity $!!P \sim !P$.
\end{remark}

\begin{remark}\label{rem:paradoxical_combinator}
  The reader familiar with the lambda calculus will have noticed the
  similarity between $D$ and the paradoxical combinator.

  [Ed. note: the existence of this seems to suggest we have to be more
  restrictive on the set of processes and names we admit if we are to
  support no-cloning.]
\end{remark}

\subsubsection{Bisimulation}

The computational dynamics gives rise to another kind of equivalence,
the equivalence of computational behavior. As previously mentioned
this is typically captured \emph{via} some form of bisimulation.

% The notion we use in this paper is weak barbed bisimulation
% \cite{milner91polyadicpi}.

The notion we use in this paper is derived from weak barbed
bisimulation \cite{milner91polyadicpi}. 

\begin{definition}
An \emph{observation relation}, $\downarrow_{\mathcal N}$, over a set
of names, $\mathcal N$, is the smallest relation satisfying the rules
below.

\infrule[Out-barb]{y \in {\mathcal N}, \; x \nameeq y}
		  {\outputp{x}{v} \downarrow_{\mathcal N} x}
\infrule[Par-barb]{\mbox{$P\downarrow_{\mathcal N} x$ or $Q\downarrow_{\mathcal N} x$}}
		  {\binpar{P}{Q} \downarrow_{\mathcal N} x}

We write $P \Downarrow_{\mathcal N} x$ if there is $Q$ such that 
$P \wred Q$ and $Q \downarrow_{\mathcal N} x$.
\end{definition}

\begin{definition}
%\label{def.bbisim}
An  ${\mathcal N}$-\emph{barbed bisimulation} over a set of names, ${\mathcal N}$, is a symmetric binary relation 
${\mathcal S}_{\mathcal N}$ between agents such that $P\rel{S}_{\mathcal N}Q$ implies:
\begin{enumerate}
\item If $P \red P'$ then $Q \wred Q'$ and $P'\rel{S}_{\mathcal N} Q'$.
\item If $P\downarrow_{\mathcal N} x$, then $Q\Downarrow_{\mathcal N} x$.
\end{enumerate}
$P$ is ${\mathcal N}$-barbed bisimilar to $Q$, written
$P \wbbisim_{\mathcal N} Q$, if $P \rel{S}_{\mathcal N} Q$ for some ${\mathcal N}$-barbed bisimulation ${\mathcal S}_{\mathcal N}$.
\end{definition}

$\mathcal{R} \subseteq \pi \times \pi$

$P \mathcal{R} Q => \forall P'. P \red P' \Rightarrow \exists Q'. Q \red Q', P' \mathcal{R} Q'$

$P \vdash x \Rightarrow Q \vdash x$

\begin{mathpar}
  \inferrule*[lab=Out-barb]{x \nameeq y}{{y}!\langle{Q}\rangle \vdash x}
  \and
  \inferrule*[lab=Par-barb]{\mbox{$P\vdash x$ or $Q\vdash x$}}{\binpar{P}{Q} \vdash x}
\end{mathpar}

\subsubsection{Contexts}

One of the principle advantages of computational calculi like the
$\pi$-calculus is a well-defined notion of context,
contextual-equivalence and a correlation between
contextual-equivalence and notions of bisimulation. The notion of
context allows the decomposition of a process into (sub-)process and
its syntactic environment, its context. Thus, a context may be
thought of as a process with a ``hole'' (written $\Box$) in it. The
application of a context $M$ to a process $P$, written $M[P]$, is
tantamount to filling the hole in $M$ with $P$. In this paper we do
not need the full weight of this theory, but do make use of the notion
of context in the proof the main theorem. 

\begin{mathpar}
  \inferrule* [lab=summation] {} {{M_{M},M_{N}} \bc \Box \;|\; x.M_{A} \;|\; M_{M}+M_{N}}
  \and
  \inferrule* [lab=agent] {} {{M_{A}} \bc (\vec{x})M_{P} \;| \; \clift{P_0,\ldots,M_{P},\ldots,P_N}}
  \and \\
  \inferrule* [lab=process] {} {{M_{P}} \bc M_{N} \;| \;P|M_{P} }
\end{mathpar} 

\begin{mathpar}
  \inferrule* [lab=sychronization] {} {M_{N} \bc \Box \;|\; x?M_{F} \;|\; x!M_{C}}
  \and
  \inferrule* [lab=abstraction] {} {{M_{F}} \bc (x)M_{P} }
  \and
  \inferrule* [lab=concretion] {} {{M_{C}} \bc \langle M_{P} \rangle }
  \and \\
  \inferrule* [lab=process] {} {{M_{P}} \bc M_{N} \;| \;P|M_{P} }
\end{mathpar}

\begin{definition}[contextual application] Given a context $M$, and
  process $P$, we define the \emph{contextual application}, $M[P] :=
  M\{P/\Box\}$. That is, the contextual application of M to P is the
  substitution of $P$ for $\Box$ in $M$.
\end{definition}

$\meaningof{-} : L \to \mathcal{P}(\pi)$

\begin{mathpar}
  \inferrule* [lab=collection] {} {\meaningof{true} = \pi, \and \meaningof{~E} = \pi \setminus \meaningof{E}, \and \meaningof{E_{1} \& E_{2}} = \meaningof{E_{1}} \cap \meaningof{E_{2}}}
\end{mathpar}

\begin{mathpar}
  \inferrule* [lab=structure] {} {\meaningof{0} = \{ P \in \pi | P \equiv 0 \}, \and \\ \meaningof{E_1 | E_2} = \{ P \in \pi | P \equiv P_{1} | P_{2}, P_{1} \in \meaningof{E_{1}}, P_{2} \in \meaningof{E_2}\} }
\end{mathpar}

\begin{mathpar}
 \inferrule* [lab=behavior] {} {\meaningof{\langle a?b \rangle E} = \{ P \in \pi | P \equiv Q | u?(y)P', \\ \and \\\\ \and \\ \;\;\; u \in \meaningof{a}, \forall z.P'\{z/y\} \in \meaningof{E\{z/b\}}\}, \and \\ \meaningof{a!E} = \{ P \in \pi | P \equiv Q | x!\langle P' \rangle, x \in \meaningof{a} P' \in \meaningof{E}\} }
\end{mathpar}

\begin{mathpar}
 \inferrule* [lab=nominal] {} {\meaningof{\quotep{E}} = \{ \quotep{P} \in \quotep{\pi} | P \in \meaningof{E} \}, \and \meaningof{\quotep{P}} = \{ \quotep{Q} \in \quotep{\pi} | P \equiv Q \} \and \\ \meaningof{@\quotep{E}} = \{ P \in \pi | P \equiv @x, x \in \meaningof{E} \}}
\end{mathpar}

\begin{eqnarray*}
  \\
  \meaningof{-} : TS \to ST
\end{eqnarray*}

\begin{eqnarray*}
  \\
  L : TS \to ST
\end{eqnarray*}

\begin{eqnarray*}
  \\
  P \models E \iff P \in \meaningof{E}
\end{eqnarray*}

\begin{eqnarray*}
  P \approx_{L} Q \iff \forall E \in L. P \models E \iff Q \models E
\end{eqnarray*}

\begin{eqnarray*}
  P \approx_{K} Q
\end{eqnarray*}

\begin{eqnarray*}
  P \approx Q
\end{eqnarray*}

$\approx_{K} = \approx = \approx_{L}$

\subsubsection{Contextual duality}

Note that contexts extend the quotation operation to a family of
operations from processes to names. Given a context, $M$, we can
define a \emph{nominal context}, $\quotep{M}$ by $\quotep{M}[P] :=
\quotep{M[P]}$. To foreshadow what is to come we observe that these
operations enjoy a duality with processes very much like the duality
between vectors and maps from vectors to scalars.

Further, because the calculus is essentially higher-order, we have a
correspondence between contexts and processes. More specifically,
given a name $x$ and a context $M$ we can construct $M^{*}_{x}$ such
that 

\begin{mathpar}
  M^{*}_{x} | \lift{x}{P} \red M[P]
\end{mathpar}

namely,

\begin{mathpar}
  M^{*}_{x} := x?(u).M[\dropn{u}]
\end{mathpar}

The dependence of $M^{*}_{x}$ on a name makes it an abstraction, 

\begin{mathpar}
  M^{*} := (x)x?(u).M[\dropn{u}]
\end{mathpar}

\subsection{Additional notation}

It will sometimes be convenient to denote the process a name
quotes. We already have the notation $x = \quotep{P}$, but it will be
convenient to introduce an alternate notation, $\procn{x}$, when we
want to emphasize the connection to the use of the name. Note that, by
virtue of name equivalence, $\quotep{\procn{x}} \nameeq x$; so, the
notation is consistent with previous definitions.

Further, because names have structure it is possible to effect
substitutions on the basis of that structure. This means we need to
upgrade our notation for substitutions, which we accomplish by
adapting comprehension notation. Thus,

\begin{mathpar}
  P\{ y / x : x \in S \}
\end{mathpar}

is interpreted to mean the process derived from P by replacing (in a
capture-avoiding manner) each occurrence of $x$ in $S$ by $y$. For example,

\begin{mathpar}
  P\{ \quotep{\procn{x}|\procn{x}} / x : x \in \freenames{P} \}
\end{mathpar}

will replace each (occurrence) of a free name $x$ in $P$ by
$\quotep{\procn{x}|\procn{x}}$.

Also, we will avail ourselves of the notation $x^{L}$ and $x^{R}$ to
denote injections of a name into disjoint copies of the name
space. There are numerous ways to accomplish this. One example can be
found in \cite{MeredithR05}. This notation overloads to vectors of
names: $\vec{x}^{\pi} := (x_{i}^{\pi} \; : \; 0 \leq i < |\vec{x}| )$ where $\pi \in \{L,R\}$.

We also use $P^{\Box} := P|\Box$.

In \cite{MeredithR05} an interpretation of the new operator is
given. It turns out that there are several possible interpretations
all enjoying the requisite algebraic properties of the operator (see
\cite{milner91polyadicpi}). We will therefore make liberal use of
$(\nu\; \vec{x})P$.

% subsection the_syntax_and_semantics_of_the_notation_system (end)   

\input{qm2pi.qmops} 

\input{qm2pi.sterngerlach} 

\input{qm2pi.metric} 

% section concurrent_process_calculi (end)

%\input{qm2pi.proofsketch}

% section proof sketch (end)

%\input{qm2pi.slviaknots} 

% section spatial logic via knots (end)

\input{qm2pi.conclusion}

% section conclusion (end)

%\input{qm2pi.dtcodes} 

% section wiring algorithm (end)

\input{qm2pi.ack} 

% section acknowledgments (end)

\newpage


\bibliographystyle{plain}   
\bibliography{../../biblios/main.bib}

\input{qm2pi.rhodetails}

\end{document}

 

%\documentclass[12pt]{llncs}
%\documentclass{jktr}

\usepackage[pdftex]{hyperref}                   
\usepackage {listings}
\usepackage {mathpartir}
\usepackage{bcprules}
%\usepackage{listings}
                       
\usepackage{graphicx} 
%\usepackage[margins=2.5cm,nohead,nofoot]{geometry}
%\usepackage{geometry}
\usepackage{amsfonts}
\usepackage{amstext}
\usepackage{latexsym}
\usepackage{amssymb}
\usepackage{color}


%\include{myPreamble}
\include{qm2pi.local} 

%\ifpdf
%\usepackage[pdftex]{graphicx}
%\else
%\usepackage{graphicx}
%\fi

 % \ifpdf
%  \usepackage{pdfsync}
%  \if


%\title{Brief Article}
%\author{David F. Snyder}
%\author{L.G. Meredith}

%\address{Dept. of Math., Texas State University--San Marcos, San Marcos, TX 78666}
       
\pagestyle{empty}


\begin{document}

\lstset{language=[Objective]Caml,frame=shadowbox}

\input{qm2pi.front}

% section front matter (end)

\input{qm2pi.intro} 
 
% section introduction (end)

% \input{qm2pi.knotations} 

% section notation (end)

\input{qm2pi.process.calculi} 

% section concurrent_process_calculi_and_spatial_logics_ (end)
    
%\input{qm2pi.knots2pi} 

%\input{qm2pi.trefoil} 

%\input{qm2pi.mainthm} 

% subsection basic_interpretation (end)

%\input{qm2pi.rho.presentation} 
\subsection{The syntax and semantics of the notation system}\label{sub:the_syntax_and_semantics_of_the_notation_system} % (fold)

We now summarize a technical presentation of the calculus that
embodies our theory of dynamics. The typical presentation of such a
calculus follows the style of giving generators and relations on
them. The grammar, below, describing term constructors, freely
generates the set of processes, $\Proc$. This set is then quotiented
by a relation known as structural congruence and it is over this set
that the notion of dynamics is expressed. This presentation is
essentially that of \cite{MeredithR05} with the addition of
polyadicity and summation. For readability we have relegated some of
the technical subtleties to an appendix.

\subsubsection{Process grammar}\label{subsub:process_grammar}

\begin{mathpar}
  \inferrule* [lab=synchronization] {} {{M} \bc \pzero \;|\; x?F \;|\; x!C }
  \and
  \inferrule* [lab=abstraction] {} {{F} \bc (x)P}
  \and
  \inferrule* [lab=concretion] {} {{C} \bc \langle Q \rangle}
  \and
  \inferrule* [lab=process] {} {{P,Q} \bc M \;| \;P|Q \;|\; @{x}}
  \and
  \inferrule* [lab=name] {} {{x} \bc \quotep{P}}
\end{mathpar} 

Note that $\vec{x}$ (resp. $\vec{P}$) denotes a vector of names
(resp. processes) of length $|\vec{x}|$ (resp. $|\vec{P}|$). We adopt
the following useful abbreviations.

\begin{mathpar}
   x?(\vec{y}).P := x.(\vec{y})P \and  x\clift{\vec{P}} := x.\clift{\vec{P}}
   \and x!(y) := \lift{x}{\dropn{y}}
   \and \Pi_{i=0}^{n-1}P_i := P_0 | \ldots | P_{n-1}
\end{mathpar}

\subsubsection{Structural congruence}

\paragraph{Free and bound names and alpha-equivalence.} At the
core of structural equivalence is alpha-equivalence which identifies
process that are the same up to a change of variable. Formally, we
recognize the distinction between free and bound names. The free names
of a process, $\freenames{P}$, may be calculated recursively as
follows:

\begin{mathpar}
\freenames{\pzero} := \emptyset
  \and \\
  \freenames{x?(y).P} := \{ x \} \cup (\freenames{P} \setminus \{ y \})
  \and 
  \freenames{x!\langle P \rangle} := \{ x \} \cup \{ P \} 
  \and \\
  \freenames{P|Q} := \freenames{P} \cup \freenames{Q}
  \and \\
  \freenames{@{x}} := \{ x \}
\end{mathpar}

$\pi$
$\quotep{\pi}$

$\freenames{-} : \pi \to \mathcal{P}(\quotep{\pi})$

\begin{eqnarray*}
  \freenames{\pzero} & := & \emptyset \\
  \freenames{x?(y).P} & := & \{ x \} \cup (\freenames{P} \setminus \{ y \}) \\
  \freenames{x!\langle P \rangle} & := & \{ x \} \cup \{ P \} \\
  \freenames{P|Q} & := & \freenames{P} \cup \freenames{Q} \\
  \freenames{\dropn{x}} & := & \{ x \}
\end{eqnarray*}

The bound names of a process, $\boundnames{P}$, are those names occurring in $P$
that are not free. For example, in $x?(y).0$, the name $x$ is free, while $y$ is bound.

\begin{mathpar}
  \inferrule* [lab=monoidal-laws] {} { P|Q \equiv Q|P \and P|0 \equiv P \and P|(Q|R) \equiv (P|Q)|R }
\end{mathpar}

\begin{mathpar}
  \inferrule* [lab=alpha-equivalence] {} { (x)P \equiv (y)P\{y/x\} \and y \not\in \freenames{P} }
\end{mathpar}

\begin{definition}
Then two processes, $P,Q$, are alpha-equivalent if $P = Q\{\vec{y}/\vec{x}\}$ for
some $\vec{x} \in \boundnames{Q},\vec{y} \in \boundnames{P}$, where $Q\{\vec{y}/\vec{x}\}$
denotes the capture-avoiding substitution of $\vec{y}$ for $\vec{x}$ in $Q$.
\end{definition}

\begin{definition}
  The {\em structural congruence} \cite{SangiorgiWalker} , $\equiv$,
  between processes is the least congruence containing
  alpha-equivalence, satisfying the abelian monoid laws
  (associativity, commutativity and $\pzero$ as identity) for parallel
  composition $|$ and for summation $+$.
\end{definition}

\subsection{Name equivalence}

We take name equivalence, written $\nameeq$, to be the smallest
equivalence relation generated by the following rules.

\begin{mathpar}
\inferrule*[lab=Quote-drop]
{ }
{ \quotep{@{x}} \nameeq x }

\inferrule*[lab=Struct-equiv]
{ P \scong Q }
{ \quotep{P} \nameeq \quotep{Q} }
\end{mathpar}

The astute reader will have noticed that the mutual recursion of names
and processes imposes a mutual recursion on alpha-equivalence and
structural equivalence via name-equivalence. Fortunately, all of this
works out pleasantly and we may calculate in the natural way, free of
concern. The reader interested in the details is referred to the
appendix \ref{appendix:rho_details}.

\subsection{Substitution}

We use $\Proc$ for the set of processes, $\QProc$ for the set of
names, and $\id{\{}\vec{y} / \vec{x} \id{\}}$ to denote partial maps,
$s : \QProc \rightarrow \QProc$. A map, $s$ lifts, uniquely, to a map
on process terms, $\widehat{s} : \Proc \rightarrow \Proc$ by the
following equations.

\begin{mathpar}
  (0) \psubstp{Q}{P} := 0 \\
  (R \juxtap S) \psubstp{Q}{P}
  :=    
  (R)\psubstp{Q}{P} \juxtap (S) \psubstp{Q}{P} \\
  (x?(y).R) \psubstp{Q}{P}    
  :=    
  (x)\substp{Q}{P} (z)\concat( (R \psubstn{z}{y}) \psubstp{Q}{P} ) \\
  (\lift{x}{R}) \psubstp{Q}{P}  
  :=
  \lift{(x)\substp{Q}{P}}{ R \psubstp{Q}{P} } \\
%   (\dropn{x})  \psubstp{Q}{P}       
%   := 
%   \left\{ 
%     \begin{array}{ccc} 
%       \dropn{\quotep{Q}} & & x \nameeq \quotep{P} \\
%       \dropn{x} & & otherwise \\
%     \end{array}
%   \right. 
  (\dropn{x})  \psubstp{Q}{P}       
  := 
  \left\{ 
    \begin{array}{ccc} 
      Q & & x \nameeq \quotep{P} \\
      \dropn{x} & & otherwise \\
    \end{array}
  \right.
\end{mathpar}
 

where

\begin{eqnarray}
  (x)\id{\{} \lpquote Q \rpquote / \lpquote P \rpquote \id{\}}            = 
  \left\{ 
    \begin{array}{ccc}
      \lpquote Q \rpquote & & x \nameeq \lpquote P \rpquote \\
      x & & otherwise \\
    \end{array}
  \right. \nonumber
\end{eqnarray}

and $z$ is chosen distinct from $\quotep{P}$, $\quotep{Q}$, the free
names in $Q$, and all the names in $R$. Our $\alpha$-equivalence will
be built in the standard way from this substitution.

\begin{remark}\label{rem:no_self_referential_names}
  One consequence of these definitions is that $\forall P. \quotep{P}
  \not\in \freenames{P}$.
\end{remark}

\subsection{ Dynamic quote: an example }

Anticipating something of what's to come, consider applying the
substitution, $\widehat{\id{\{}u / z \id{\}}}$, to the following pair
of processes, $\lift{w}{y!(z)}$ and $w[ \lpquote y!(z) \rpquote ]$.

\begin{eqnarray}
	\lift{w}{y!(z)}\widehat{\id{\{}u / z \id{\}}}
		& = &
		\lift{w}{y!(u)} \nonumber\\
	w[ \lpquote y!(z) \rpquote ] \widehat{ \id{\{}u / z \id{\}} }
		& = &
		w[ \lpquote y!(z) \rpquote ] \nonumber
\end{eqnarray}

Because the body of the process between quotes is impervious to
substitution, we get radically different answers. In fact, by
examining the first process in an input context,
e.g. $x?(z).\lift{w}{y!(z)}$, we see that the process under the lift
operator may be shaped by prefixed inputs binding a name inside it. In
this sense, the lift operator will be seen as a way to dynamically
construct processes before reifying them as names.

Finally equipped with these standard features we can present the
dynamics of the calculus.

\subsubsection{Operational semantics} 

Finally, we introduce the computational dynamics. What marks these
algebras as distinct from other more traditionally studied algebraic
structures, e.g. vector spaces or polynomial rings, is the manner in
which dynamics is captured. In traditional structures, dynamics is typically
expressed through morphisms between such structures, as in linear maps
between vector spaces or morphisms between rings. In algebras
associated with the semantics of computation, the dynamics is
expressed as part of the algebraic structure itself, through a
reduction reduction relation typically denoted by $\red$. Below, we
give a recursive presentation of this relation for the calculus used
in the encoding.

$\red \subseteq \pi \times \pi$
$\red : \pi \to \mathcal{P}(\pi)$

\begin{mathpar}
  \inferrule* [lab=Comm] { \textsf{match}( x_{src}, x_{trgt} ) } { x_{trgt}?(y)P \; | \; x_{src}!\langle {Q} \rangle \red P\{\quotep{Q}/y}\} }
  \and \\
  \inferrule* [lab=Par] {{P} \red {P}'} {{{P} | {Q}} \red {{P}' | {Q}}}
  \and
  \inferrule* [lab=Equiv]{{{P} \scong {P}'} \andalso {{P}' \red {Q}'} \andalso {{Q}' \scong {Q}}}{{P} \red {Q}}
\end{mathpar}

\begin{eqnarray*}
  match_{\equiv} (\quotep{P},\quotep{Q}) & := & P \equiv Q \\
  match_{\dagger}(\quotep{P},\quotep{Q}) & := & \forall R. P|Q \red^{*} R => R \red^{*} 0 \\
  match_{K}(\quotep{P},\quotep{Q}) & := & K \mbox{ for some context } K
\end{eqnarray*}

$u?(x)P | u!\langle Q \rangle \red P\{\quotep{Q}/x\}$

%We write $\wred$ for $\red^*$, and $P\red$ if $\exists Q $ such that $ P \red Q$.
We write $P\red$ if $\exists Q $ such that $ P \red Q$ and $P\not\red$, otherwise.

\section{Replication}

As mentioned before, it is known that replication (and hence
recursion) can be implemented in a higher-order process algebra
\cite{SangiorgiWalker}. As our first example of calculation with the
machinery thus far presented we give the construction explicitly in
the {\rhoc}.

\begin{eqnarray}
	D_{x} & := & \prefix{x}{y}{(\binpar{\outputp{x}{y}}{@{y}})} \nonumber\\
	\bangp_{x}{P} & := & \binpar{{x}!\langle{\binpar{D_{x}}{P}}\rangle}{D_{x}} \nonumber
\end{eqnarray}

\begin{eqnarray}
	\bangp_{x}{P} & & \nonumber\\
	=
	& {x}!\langle{(\prefix{x}{y}{(\outputp{x}{y} | @{y})) | P}}\rangle 
	      | \prefix{x}{y}{(\outputp{x}{y} | @{y})} & \nonumber\\
	\red
	& (\outputp{x}{y} | @{y})\substn{\quotep{(\prefix{x}{y}{(@{y} | \outputp{x}{y})) | P}}}{y} & \nonumber\\
	=
	& \outputp{x}{\quotep{(\prefix{x}{y}{(\outputp{x}{y} | @{y})) | P}}}
	  | {(\prefix{x}{y}{(\outputp{x}{y} | @{y})) | P}} & \nonumber\\
	\red
	& \ldots & \nonumber\\
	\red^*
	& P | P | \ldots & \nonumber
\end{eqnarray}

Of course, this encoding, as an implementation, runs away, unfolding
$\bangp{P}$ eagerly. A lazier and more implementable replication
operator, restricted to input-guarded processes, may be obtained as follows.

\begin{eqnarray}
\bangp{\prefix{u}{v}{P}} 
	:= 
	\binpar{\lift{x}{\prefix{u}{v}{(\binpar{D(x)}{P})}}}{D(x)} \nonumber
\end{eqnarray}

\begin{remark}
  Note that the lazier definition still does not deal with summation
  or mixed summation (i.e. sums over input and output). The reader is
  invited to construct definitions of replication that deal with these
  features. 

  Further, the definitions are parameterized in a name, $x$. Can you,
  gentle reader, make a definition that eliminates this parameter and
  guarantees no accidental interaction between the replication
  machinery and the process being replicated -- i.e. no accidental
  sharing of names used by the process to get its work done and the
  name(s) used by the replication to effect copying. This latter
  revision of the definition of replication is crucial to obtaining
  the expected identity $!!P \sim !P$.
\end{remark}

\begin{remark}\label{rem:paradoxical_combinator}
  The reader familiar with the lambda calculus will have noticed the
  similarity between $D$ and the paradoxical combinator.

  [Ed. note: the existence of this seems to suggest we have to be more
  restrictive on the set of processes and names we admit if we are to
  support no-cloning.]
\end{remark}

\subsubsection{Bisimulation}

The computational dynamics gives rise to another kind of equivalence,
the equivalence of computational behavior. As previously mentioned
this is typically captured \emph{via} some form of bisimulation.

% The notion we use in this paper is weak barbed bisimulation
% \cite{milner91polyadicpi}.

The notion we use in this paper is derived from weak barbed
bisimulation \cite{milner91polyadicpi}. 

\begin{definition}
An \emph{observation relation}, $\downarrow_{\mathcal N}$, over a set
of names, $\mathcal N$, is the smallest relation satisfying the rules
below.

\infrule[Out-barb]{y \in {\mathcal N}, \; x \nameeq y}
		  {\outputp{x}{v} \downarrow_{\mathcal N} x}
\infrule[Par-barb]{\mbox{$P\downarrow_{\mathcal N} x$ or $Q\downarrow_{\mathcal N} x$}}
		  {\binpar{P}{Q} \downarrow_{\mathcal N} x}

We write $P \Downarrow_{\mathcal N} x$ if there is $Q$ such that 
$P \wred Q$ and $Q \downarrow_{\mathcal N} x$.
\end{definition}

\begin{definition}
%\label{def.bbisim}
An  ${\mathcal N}$-\emph{barbed bisimulation} over a set of names, ${\mathcal N}$, is a symmetric binary relation 
${\mathcal S}_{\mathcal N}$ between agents such that $P\rel{S}_{\mathcal N}Q$ implies:
\begin{enumerate}
\item If $P \red P'$ then $Q \wred Q'$ and $P'\rel{S}_{\mathcal N} Q'$.
\item If $P\downarrow_{\mathcal N} x$, then $Q\Downarrow_{\mathcal N} x$.
\end{enumerate}
$P$ is ${\mathcal N}$-barbed bisimilar to $Q$, written
$P \wbbisim_{\mathcal N} Q$, if $P \rel{S}_{\mathcal N} Q$ for some ${\mathcal N}$-barbed bisimulation ${\mathcal S}_{\mathcal N}$.
\end{definition}

$\mathcal{R} \subseteq \pi \times \pi$

$P \mathcal{R} Q => \forall P'. P \red P' \Rightarrow \exists Q'. Q \red Q', P' \mathcal{R} Q'$

$P \vdash x \Rightarrow Q \vdash x$

\begin{mathpar}
  \inferrule*[lab=Out-barb]{x \nameeq y}{{y}!\langle{Q}\rangle \vdash x}
  \and
  \inferrule*[lab=Par-barb]{\mbox{$P\vdash x$ or $Q\vdash x$}}{\binpar{P}{Q} \vdash x}
\end{mathpar}

\subsubsection{Contexts}

One of the principle advantages of computational calculi like the
$\pi$-calculus is a well-defined notion of context,
contextual-equivalence and a correlation between
contextual-equivalence and notions of bisimulation. The notion of
context allows the decomposition of a process into (sub-)process and
its syntactic environment, its context. Thus, a context may be
thought of as a process with a ``hole'' (written $\Box$) in it. The
application of a context $M$ to a process $P$, written $M[P]$, is
tantamount to filling the hole in $M$ with $P$. In this paper we do
not need the full weight of this theory, but do make use of the notion
of context in the proof the main theorem. 

\begin{mathpar}
  \inferrule* [lab=summation] {} {{M_{M},M_{N}} \bc \Box \;|\; x.M_{A} \;|\; M_{M}+M_{N}}
  \and
  \inferrule* [lab=agent] {} {{M_{A}} \bc (\vec{x})M_{P} \;| \; \clift{P_0,\ldots,M_{P},\ldots,P_N}}
  \and \\
  \inferrule* [lab=process] {} {{M_{P}} \bc M_{N} \;| \;P|M_{P} }
\end{mathpar} 

\begin{mathpar}
  \inferrule* [lab=sychronization] {} {M_{N} \bc \Box \;|\; x?M_{F} \;|\; x!M_{C}}
  \and
  \inferrule* [lab=abstraction] {} {{M_{F}} \bc (x)M_{P} }
  \and
  \inferrule* [lab=concretion] {} {{M_{C}} \bc \langle M_{P} \rangle }
  \and \\
  \inferrule* [lab=process] {} {{M_{P}} \bc M_{N} \;| \;P|M_{P} }
\end{mathpar}

\begin{definition}[contextual application] Given a context $M$, and
  process $P$, we define the \emph{contextual application}, $M[P] :=
  M\{P/\Box\}$. That is, the contextual application of M to P is the
  substitution of $P$ for $\Box$ in $M$.
\end{definition}

$\meaningof{-} : L \to \mathcal{P}(\pi)$

\begin{mathpar}
  \inferrule* [lab=collection] {} {\meaningof{true} = \pi, \and \meaningof{~E} = \pi \setminus \meaningof{E}, \and \meaningof{E_{1} \& E_{2}} = \meaningof{E_{1}} \cap \meaningof{E_{2}}}
\end{mathpar}

\begin{mathpar}
  \inferrule* [lab=structure] {} {\meaningof{0} = \{ P \in \pi | P \equiv 0 \}, \and \\ \meaningof{E_1 | E_2} = \{ P \in \pi | P \equiv P_{1} | P_{2}, P_{1} \in \meaningof{E_{1}}, P_{2} \in \meaningof{E_2}\} }
\end{mathpar}

\begin{mathpar}
 \inferrule* [lab=behavior] {} {\meaningof{\langle a?b \rangle E} = \{ P \in \pi | P \equiv Q | u?(y)P', \\ \and \\\\ \and \\ \;\;\; u \in \meaningof{a}, \forall z.P'\{z/y\} \in \meaningof{E\{z/b\}}\}, \and \\ \meaningof{a!E} = \{ P \in \pi | P \equiv Q | x!\langle P' \rangle, x \in \meaningof{a} P' \in \meaningof{E}\} }
\end{mathpar}

\begin{mathpar}
 \inferrule* [lab=nominal] {} {\meaningof{\quotep{E}} = \{ \quotep{P} \in \quotep{\pi} | P \in \meaningof{E} \}, \and \meaningof{\quotep{P}} = \{ \quotep{Q} \in \quotep{\pi} | P \equiv Q \} \and \\ \meaningof{@\quotep{E}} = \{ P \in \pi | P \equiv @x, x \in \meaningof{E} \}}
\end{mathpar}

\begin{eqnarray*}
  \\
  \meaningof{-} : TS \to ST
\end{eqnarray*}

\begin{eqnarray*}
  \\
  L : TS \to ST
\end{eqnarray*}

\begin{eqnarray*}
  \\
  P \models E \iff P \in \meaningof{E}
\end{eqnarray*}

\begin{eqnarray*}
  P \approx_{L} Q \iff \forall E \in L. P \models E \iff Q \models E
\end{eqnarray*}

\begin{eqnarray*}
  P \approx_{K} Q
\end{eqnarray*}

\begin{eqnarray*}
  P \approx Q
\end{eqnarray*}

$\approx_{K} = \approx = \approx_{L}$

\subsubsection{Contextual duality}

Note that contexts extend the quotation operation to a family of
operations from processes to names. Given a context, $M$, we can
define a \emph{nominal context}, $\quotep{M}$ by $\quotep{M}[P] :=
\quotep{M[P]}$. To foreshadow what is to come we observe that these
operations enjoy a duality with processes very much like the duality
between vectors and maps from vectors to scalars.

Further, because the calculus is essentially higher-order, we have a
correspondence between contexts and processes. More specifically,
given a name $x$ and a context $M$ we can construct $M^{*}_{x}$ such
that 

\begin{mathpar}
  M^{*}_{x} | \lift{x}{P} \red M[P]
\end{mathpar}

namely,

\begin{mathpar}
  M^{*}_{x} := x?(u).M[\dropn{u}]
\end{mathpar}

The dependence of $M^{*}_{x}$ on a name makes it an abstraction, 

\begin{mathpar}
  M^{*} := (x)x?(u).M[\dropn{u}]
\end{mathpar}

\subsection{Additional notation}

It will sometimes be convenient to denote the process a name
quotes. We already have the notation $x = \quotep{P}$, but it will be
convenient to introduce an alternate notation, $\procn{x}$, when we
want to emphasize the connection to the use of the name. Note that, by
virtue of name equivalence, $\quotep{\procn{x}} \nameeq x$; so, the
notation is consistent with previous definitions.

Further, because names have structure it is possible to effect
substitutions on the basis of that structure. This means we need to
upgrade our notation for substitutions, which we accomplish by
adapting comprehension notation. Thus,

\begin{mathpar}
  P\{ y / x : x \in S \}
\end{mathpar}

is interpreted to mean the process derived from P by replacing (in a
capture-avoiding manner) each occurrence of $x$ in $S$ by $y$. For example,

\begin{mathpar}
  P\{ \quotep{\procn{x}|\procn{x}} / x : x \in \freenames{P} \}
\end{mathpar}

will replace each (occurrence) of a free name $x$ in $P$ by
$\quotep{\procn{x}|\procn{x}}$.

Also, we will avail ourselves of the notation $x^{L}$ and $x^{R}$ to
denote injections of a name into disjoint copies of the name
space. There are numerous ways to accomplish this. One example can be
found in \cite{MeredithR05}. This notation overloads to vectors of
names: $\vec{x}^{\pi} := (x_{i}^{\pi} \; : \; 0 \leq i < |\vec{x}| )$ where $\pi \in \{L,R\}$.

We also use $P^{\Box} := P|\Box$.

In \cite{MeredithR05} an interpretation of the new operator is
given. It turns out that there are several possible interpretations
all enjoying the requisite algebraic properties of the operator (see
\cite{milner91polyadicpi}). We will therefore make liberal use of
$(\nu\; \vec{x})P$.

% subsection the_syntax_and_semantics_of_the_notation_system (end)   

\input{qm2pi.qmops} 

\input{qm2pi.sterngerlach} 

\input{qm2pi.metric} 

% section concurrent_process_calculi (end)

%\input{qm2pi.proofsketch}

% section proof sketch (end)

%\input{qm2pi.slviaknots} 

% section spatial logic via knots (end)

\input{qm2pi.conclusion}

% section conclusion (end)

%\input{qm2pi.dtcodes} 

% section wiring algorithm (end)

\input{qm2pi.ack} 

% section acknowledgments (end)

\newpage


\bibliographystyle{plain}   
\bibliography{../../biblios/main.bib}

\input{qm2pi.rhodetails}

\end{document}

 

%\documentclass[12pt]{llncs}
%\documentclass{jktr}

\usepackage[pdftex]{hyperref}                   
\usepackage {listings}
\usepackage {mathpartir}
\usepackage{bcprules}
%\usepackage{listings}
                       
\usepackage{graphicx} 
%\usepackage[margins=2.5cm,nohead,nofoot]{geometry}
%\usepackage{geometry}
\usepackage{amsfonts}
\usepackage{amstext}
\usepackage{latexsym}
\usepackage{amssymb}
\usepackage{color}


%\include{myPreamble}
\include{qm2pi.local} 

%\ifpdf
%\usepackage[pdftex]{graphicx}
%\else
%\usepackage{graphicx}
%\fi

 % \ifpdf
%  \usepackage{pdfsync}
%  \if


%\title{Brief Article}
%\author{David F. Snyder}
%\author{L.G. Meredith}

%\address{Dept. of Math., Texas State University--San Marcos, San Marcos, TX 78666}
       
\pagestyle{empty}


\begin{document}

\lstset{language=[Objective]Caml,frame=shadowbox}

\input{qm2pi.front}

% section front matter (end)

\input{qm2pi.intro} 
 
% section introduction (end)

% \input{qm2pi.knotations} 

% section notation (end)

\input{qm2pi.process.calculi} 

% section concurrent_process_calculi_and_spatial_logics_ (end)
    
%\input{qm2pi.knots2pi} 

%\input{qm2pi.trefoil} 

%\input{qm2pi.mainthm} 

% subsection basic_interpretation (end)

%\input{qm2pi.rho.presentation} 
\subsection{The syntax and semantics of the notation system}\label{sub:the_syntax_and_semantics_of_the_notation_system} % (fold)

We now summarize a technical presentation of the calculus that
embodies our theory of dynamics. The typical presentation of such a
calculus follows the style of giving generators and relations on
them. The grammar, below, describing term constructors, freely
generates the set of processes, $\Proc$. This set is then quotiented
by a relation known as structural congruence and it is over this set
that the notion of dynamics is expressed. This presentation is
essentially that of \cite{MeredithR05} with the addition of
polyadicity and summation. For readability we have relegated some of
the technical subtleties to an appendix.

\subsubsection{Process grammar}\label{subsub:process_grammar}

\begin{mathpar}
  \inferrule* [lab=synchronization] {} {{M} \bc \pzero \;|\; x?F \;|\; x!C }
  \and
  \inferrule* [lab=abstraction] {} {{F} \bc (x)P}
  \and
  \inferrule* [lab=concretion] {} {{C} \bc \langle Q \rangle}
  \and
  \inferrule* [lab=process] {} {{P,Q} \bc M \;| \;P|Q \;|\; @{x}}
  \and
  \inferrule* [lab=name] {} {{x} \bc \quotep{P}}
\end{mathpar} 

Note that $\vec{x}$ (resp. $\vec{P}$) denotes a vector of names
(resp. processes) of length $|\vec{x}|$ (resp. $|\vec{P}|$). We adopt
the following useful abbreviations.

\begin{mathpar}
   x?(\vec{y}).P := x.(\vec{y})P \and  x\clift{\vec{P}} := x.\clift{\vec{P}}
   \and x!(y) := \lift{x}{\dropn{y}}
   \and \Pi_{i=0}^{n-1}P_i := P_0 | \ldots | P_{n-1}
\end{mathpar}

\subsubsection{Structural congruence}

\paragraph{Free and bound names and alpha-equivalence.} At the
core of structural equivalence is alpha-equivalence which identifies
process that are the same up to a change of variable. Formally, we
recognize the distinction between free and bound names. The free names
of a process, $\freenames{P}$, may be calculated recursively as
follows:

\begin{mathpar}
\freenames{\pzero} := \emptyset
  \and \\
  \freenames{x?(y).P} := \{ x \} \cup (\freenames{P} \setminus \{ y \})
  \and 
  \freenames{x!\langle P \rangle} := \{ x \} \cup \{ P \} 
  \and \\
  \freenames{P|Q} := \freenames{P} \cup \freenames{Q}
  \and \\
  \freenames{@{x}} := \{ x \}
\end{mathpar}

$\pi$
$\quotep{\pi}$

$\freenames{-} : \pi \to \mathcal{P}(\quotep{\pi})$

\begin{eqnarray*}
  \freenames{\pzero} & := & \emptyset \\
  \freenames{x?(y).P} & := & \{ x \} \cup (\freenames{P} \setminus \{ y \}) \\
  \freenames{x!\langle P \rangle} & := & \{ x \} \cup \{ P \} \\
  \freenames{P|Q} & := & \freenames{P} \cup \freenames{Q} \\
  \freenames{\dropn{x}} & := & \{ x \}
\end{eqnarray*}

The bound names of a process, $\boundnames{P}$, are those names occurring in $P$
that are not free. For example, in $x?(y).0$, the name $x$ is free, while $y$ is bound.

\begin{mathpar}
  \inferrule* [lab=monoidal-laws] {} { P|Q \equiv Q|P \and P|0 \equiv P \and P|(Q|R) \equiv (P|Q)|R }
\end{mathpar}

\begin{mathpar}
  \inferrule* [lab=alpha-equivalence] {} { (x)P \equiv (y)P\{y/x\} \and y \not\in \freenames{P} }
\end{mathpar}

\begin{definition}
Then two processes, $P,Q$, are alpha-equivalent if $P = Q\{\vec{y}/\vec{x}\}$ for
some $\vec{x} \in \boundnames{Q},\vec{y} \in \boundnames{P}$, where $Q\{\vec{y}/\vec{x}\}$
denotes the capture-avoiding substitution of $\vec{y}$ for $\vec{x}$ in $Q$.
\end{definition}

\begin{definition}
  The {\em structural congruence} \cite{SangiorgiWalker} , $\equiv$,
  between processes is the least congruence containing
  alpha-equivalence, satisfying the abelian monoid laws
  (associativity, commutativity and $\pzero$ as identity) for parallel
  composition $|$ and for summation $+$.
\end{definition}

\subsection{Name equivalence}

We take name equivalence, written $\nameeq$, to be the smallest
equivalence relation generated by the following rules.

\begin{mathpar}
\inferrule*[lab=Quote-drop]
{ }
{ \quotep{@{x}} \nameeq x }

\inferrule*[lab=Struct-equiv]
{ P \scong Q }
{ \quotep{P} \nameeq \quotep{Q} }
\end{mathpar}

The astute reader will have noticed that the mutual recursion of names
and processes imposes a mutual recursion on alpha-equivalence and
structural equivalence via name-equivalence. Fortunately, all of this
works out pleasantly and we may calculate in the natural way, free of
concern. The reader interested in the details is referred to the
appendix \ref{appendix:rho_details}.

\subsection{Substitution}

We use $\Proc$ for the set of processes, $\QProc$ for the set of
names, and $\id{\{}\vec{y} / \vec{x} \id{\}}$ to denote partial maps,
$s : \QProc \rightarrow \QProc$. A map, $s$ lifts, uniquely, to a map
on process terms, $\widehat{s} : \Proc \rightarrow \Proc$ by the
following equations.

\begin{mathpar}
  (0) \psubstp{Q}{P} := 0 \\
  (R \juxtap S) \psubstp{Q}{P}
  :=    
  (R)\psubstp{Q}{P} \juxtap (S) \psubstp{Q}{P} \\
  (x?(y).R) \psubstp{Q}{P}    
  :=    
  (x)\substp{Q}{P} (z)\concat( (R \psubstn{z}{y}) \psubstp{Q}{P} ) \\
  (\lift{x}{R}) \psubstp{Q}{P}  
  :=
  \lift{(x)\substp{Q}{P}}{ R \psubstp{Q}{P} } \\
%   (\dropn{x})  \psubstp{Q}{P}       
%   := 
%   \left\{ 
%     \begin{array}{ccc} 
%       \dropn{\quotep{Q}} & & x \nameeq \quotep{P} \\
%       \dropn{x} & & otherwise \\
%     \end{array}
%   \right. 
  (\dropn{x})  \psubstp{Q}{P}       
  := 
  \left\{ 
    \begin{array}{ccc} 
      Q & & x \nameeq \quotep{P} \\
      \dropn{x} & & otherwise \\
    \end{array}
  \right.
\end{mathpar}
 

where

\begin{eqnarray}
  (x)\id{\{} \lpquote Q \rpquote / \lpquote P \rpquote \id{\}}            = 
  \left\{ 
    \begin{array}{ccc}
      \lpquote Q \rpquote & & x \nameeq \lpquote P \rpquote \\
      x & & otherwise \\
    \end{array}
  \right. \nonumber
\end{eqnarray}

and $z$ is chosen distinct from $\quotep{P}$, $\quotep{Q}$, the free
names in $Q$, and all the names in $R$. Our $\alpha$-equivalence will
be built in the standard way from this substitution.

\begin{remark}\label{rem:no_self_referential_names}
  One consequence of these definitions is that $\forall P. \quotep{P}
  \not\in \freenames{P}$.
\end{remark}

\subsection{ Dynamic quote: an example }

Anticipating something of what's to come, consider applying the
substitution, $\widehat{\id{\{}u / z \id{\}}}$, to the following pair
of processes, $\lift{w}{y!(z)}$ and $w[ \lpquote y!(z) \rpquote ]$.

\begin{eqnarray}
	\lift{w}{y!(z)}\widehat{\id{\{}u / z \id{\}}}
		& = &
		\lift{w}{y!(u)} \nonumber\\
	w[ \lpquote y!(z) \rpquote ] \widehat{ \id{\{}u / z \id{\}} }
		& = &
		w[ \lpquote y!(z) \rpquote ] \nonumber
\end{eqnarray}

Because the body of the process between quotes is impervious to
substitution, we get radically different answers. In fact, by
examining the first process in an input context,
e.g. $x?(z).\lift{w}{y!(z)}$, we see that the process under the lift
operator may be shaped by prefixed inputs binding a name inside it. In
this sense, the lift operator will be seen as a way to dynamically
construct processes before reifying them as names.

Finally equipped with these standard features we can present the
dynamics of the calculus.

\subsubsection{Operational semantics} 

Finally, we introduce the computational dynamics. What marks these
algebras as distinct from other more traditionally studied algebraic
structures, e.g. vector spaces or polynomial rings, is the manner in
which dynamics is captured. In traditional structures, dynamics is typically
expressed through morphisms between such structures, as in linear maps
between vector spaces or morphisms between rings. In algebras
associated with the semantics of computation, the dynamics is
expressed as part of the algebraic structure itself, through a
reduction reduction relation typically denoted by $\red$. Below, we
give a recursive presentation of this relation for the calculus used
in the encoding.

$\red \subseteq \pi \times \pi$
$\red : \pi \to \mathcal{P}(\pi)$

\begin{mathpar}
  \inferrule* [lab=Comm] { \textsf{match}( x_{src}, x_{trgt} ) } { x_{trgt}?(y)P \; | \; x_{src}!\langle {Q} \rangle \red P\{\quotep{Q}/y}\} }
  \and \\
  \inferrule* [lab=Par] {{P} \red {P}'} {{{P} | {Q}} \red {{P}' | {Q}}}
  \and
  \inferrule* [lab=Equiv]{{{P} \scong {P}'} \andalso {{P}' \red {Q}'} \andalso {{Q}' \scong {Q}}}{{P} \red {Q}}
\end{mathpar}

\begin{eqnarray*}
  match_{\equiv} (\quotep{P},\quotep{Q}) & := & P \equiv Q \\
  match_{\dagger}(\quotep{P},\quotep{Q}) & := & \forall R. P|Q \red^{*} R => R \red^{*} 0 \\
  match_{K}(\quotep{P},\quotep{Q}) & := & K \mbox{ for some context } K
\end{eqnarray*}

$u?(x)P | u!\langle Q \rangle \red P\{\quotep{Q}/x\}$

%We write $\wred$ for $\red^*$, and $P\red$ if $\exists Q $ such that $ P \red Q$.
We write $P\red$ if $\exists Q $ such that $ P \red Q$ and $P\not\red$, otherwise.

\section{Replication}

As mentioned before, it is known that replication (and hence
recursion) can be implemented in a higher-order process algebra
\cite{SangiorgiWalker}. As our first example of calculation with the
machinery thus far presented we give the construction explicitly in
the {\rhoc}.

\begin{eqnarray}
	D_{x} & := & \prefix{x}{y}{(\binpar{\outputp{x}{y}}{@{y}})} \nonumber\\
	\bangp_{x}{P} & := & \binpar{{x}!\langle{\binpar{D_{x}}{P}}\rangle}{D_{x}} \nonumber
\end{eqnarray}

\begin{eqnarray}
	\bangp_{x}{P} & & \nonumber\\
	=
	& {x}!\langle{(\prefix{x}{y}{(\outputp{x}{y} | @{y})) | P}}\rangle 
	      | \prefix{x}{y}{(\outputp{x}{y} | @{y})} & \nonumber\\
	\red
	& (\outputp{x}{y} | @{y})\substn{\quotep{(\prefix{x}{y}{(@{y} | \outputp{x}{y})) | P}}}{y} & \nonumber\\
	=
	& \outputp{x}{\quotep{(\prefix{x}{y}{(\outputp{x}{y} | @{y})) | P}}}
	  | {(\prefix{x}{y}{(\outputp{x}{y} | @{y})) | P}} & \nonumber\\
	\red
	& \ldots & \nonumber\\
	\red^*
	& P | P | \ldots & \nonumber
\end{eqnarray}

Of course, this encoding, as an implementation, runs away, unfolding
$\bangp{P}$ eagerly. A lazier and more implementable replication
operator, restricted to input-guarded processes, may be obtained as follows.

\begin{eqnarray}
\bangp{\prefix{u}{v}{P}} 
	:= 
	\binpar{\lift{x}{\prefix{u}{v}{(\binpar{D(x)}{P})}}}{D(x)} \nonumber
\end{eqnarray}

\begin{remark}
  Note that the lazier definition still does not deal with summation
  or mixed summation (i.e. sums over input and output). The reader is
  invited to construct definitions of replication that deal with these
  features. 

  Further, the definitions are parameterized in a name, $x$. Can you,
  gentle reader, make a definition that eliminates this parameter and
  guarantees no accidental interaction between the replication
  machinery and the process being replicated -- i.e. no accidental
  sharing of names used by the process to get its work done and the
  name(s) used by the replication to effect copying. This latter
  revision of the definition of replication is crucial to obtaining
  the expected identity $!!P \sim !P$.
\end{remark}

\begin{remark}\label{rem:paradoxical_combinator}
  The reader familiar with the lambda calculus will have noticed the
  similarity between $D$ and the paradoxical combinator.

  [Ed. note: the existence of this seems to suggest we have to be more
  restrictive on the set of processes and names we admit if we are to
  support no-cloning.]
\end{remark}

\subsubsection{Bisimulation}

The computational dynamics gives rise to another kind of equivalence,
the equivalence of computational behavior. As previously mentioned
this is typically captured \emph{via} some form of bisimulation.

% The notion we use in this paper is weak barbed bisimulation
% \cite{milner91polyadicpi}.

The notion we use in this paper is derived from weak barbed
bisimulation \cite{milner91polyadicpi}. 

\begin{definition}
An \emph{observation relation}, $\downarrow_{\mathcal N}$, over a set
of names, $\mathcal N$, is the smallest relation satisfying the rules
below.

\infrule[Out-barb]{y \in {\mathcal N}, \; x \nameeq y}
		  {\outputp{x}{v} \downarrow_{\mathcal N} x}
\infrule[Par-barb]{\mbox{$P\downarrow_{\mathcal N} x$ or $Q\downarrow_{\mathcal N} x$}}
		  {\binpar{P}{Q} \downarrow_{\mathcal N} x}

We write $P \Downarrow_{\mathcal N} x$ if there is $Q$ such that 
$P \wred Q$ and $Q \downarrow_{\mathcal N} x$.
\end{definition}

\begin{definition}
%\label{def.bbisim}
An  ${\mathcal N}$-\emph{barbed bisimulation} over a set of names, ${\mathcal N}$, is a symmetric binary relation 
${\mathcal S}_{\mathcal N}$ between agents such that $P\rel{S}_{\mathcal N}Q$ implies:
\begin{enumerate}
\item If $P \red P'$ then $Q \wred Q'$ and $P'\rel{S}_{\mathcal N} Q'$.
\item If $P\downarrow_{\mathcal N} x$, then $Q\Downarrow_{\mathcal N} x$.
\end{enumerate}
$P$ is ${\mathcal N}$-barbed bisimilar to $Q$, written
$P \wbbisim_{\mathcal N} Q$, if $P \rel{S}_{\mathcal N} Q$ for some ${\mathcal N}$-barbed bisimulation ${\mathcal S}_{\mathcal N}$.
\end{definition}

$\mathcal{R} \subseteq \pi \times \pi$

$P \mathcal{R} Q => \forall P'. P \red P' \Rightarrow \exists Q'. Q \red Q', P' \mathcal{R} Q'$

$P \vdash x \Rightarrow Q \vdash x$

\begin{mathpar}
  \inferrule*[lab=Out-barb]{x \nameeq y}{{y}!\langle{Q}\rangle \vdash x}
  \and
  \inferrule*[lab=Par-barb]{\mbox{$P\vdash x$ or $Q\vdash x$}}{\binpar{P}{Q} \vdash x}
\end{mathpar}

\subsubsection{Contexts}

One of the principle advantages of computational calculi like the
$\pi$-calculus is a well-defined notion of context,
contextual-equivalence and a correlation between
contextual-equivalence and notions of bisimulation. The notion of
context allows the decomposition of a process into (sub-)process and
its syntactic environment, its context. Thus, a context may be
thought of as a process with a ``hole'' (written $\Box$) in it. The
application of a context $M$ to a process $P$, written $M[P]$, is
tantamount to filling the hole in $M$ with $P$. In this paper we do
not need the full weight of this theory, but do make use of the notion
of context in the proof the main theorem. 

\begin{mathpar}
  \inferrule* [lab=summation] {} {{M_{M},M_{N}} \bc \Box \;|\; x.M_{A} \;|\; M_{M}+M_{N}}
  \and
  \inferrule* [lab=agent] {} {{M_{A}} \bc (\vec{x})M_{P} \;| \; \clift{P_0,\ldots,M_{P},\ldots,P_N}}
  \and \\
  \inferrule* [lab=process] {} {{M_{P}} \bc M_{N} \;| \;P|M_{P} }
\end{mathpar} 

\begin{mathpar}
  \inferrule* [lab=sychronization] {} {M_{N} \bc \Box \;|\; x?M_{F} \;|\; x!M_{C}}
  \and
  \inferrule* [lab=abstraction] {} {{M_{F}} \bc (x)M_{P} }
  \and
  \inferrule* [lab=concretion] {} {{M_{C}} \bc \langle M_{P} \rangle }
  \and \\
  \inferrule* [lab=process] {} {{M_{P}} \bc M_{N} \;| \;P|M_{P} }
\end{mathpar}

\begin{definition}[contextual application] Given a context $M$, and
  process $P$, we define the \emph{contextual application}, $M[P] :=
  M\{P/\Box\}$. That is, the contextual application of M to P is the
  substitution of $P$ for $\Box$ in $M$.
\end{definition}

$\meaningof{-} : L \to \mathcal{P}(\pi)$

\begin{mathpar}
  \inferrule* [lab=collection] {} {\meaningof{true} = \pi, \and \meaningof{~E} = \pi \setminus \meaningof{E}, \and \meaningof{E_{1} \& E_{2}} = \meaningof{E_{1}} \cap \meaningof{E_{2}}}
\end{mathpar}

\begin{mathpar}
  \inferrule* [lab=structure] {} {\meaningof{0} = \{ P \in \pi | P \equiv 0 \}, \and \\ \meaningof{E_1 | E_2} = \{ P \in \pi | P \equiv P_{1} | P_{2}, P_{1} \in \meaningof{E_{1}}, P_{2} \in \meaningof{E_2}\} }
\end{mathpar}

\begin{mathpar}
 \inferrule* [lab=behavior] {} {\meaningof{\langle a?b \rangle E} = \{ P \in \pi | P \equiv Q | u?(y)P', \\ \and \\\\ \and \\ \;\;\; u \in \meaningof{a}, \forall z.P'\{z/y\} \in \meaningof{E\{z/b\}}\}, \and \\ \meaningof{a!E} = \{ P \in \pi | P \equiv Q | x!\langle P' \rangle, x \in \meaningof{a} P' \in \meaningof{E}\} }
\end{mathpar}

\begin{mathpar}
 \inferrule* [lab=nominal] {} {\meaningof{\quotep{E}} = \{ \quotep{P} \in \quotep{\pi} | P \in \meaningof{E} \}, \and \meaningof{\quotep{P}} = \{ \quotep{Q} \in \quotep{\pi} | P \equiv Q \} \and \\ \meaningof{@\quotep{E}} = \{ P \in \pi | P \equiv @x, x \in \meaningof{E} \}}
\end{mathpar}

\begin{eqnarray*}
  \\
  \meaningof{-} : TS \to ST
\end{eqnarray*}

\begin{eqnarray*}
  \\
  L : TS \to ST
\end{eqnarray*}

\begin{eqnarray*}
  \\
  P \models E \iff P \in \meaningof{E}
\end{eqnarray*}

\begin{eqnarray*}
  P \approx_{L} Q \iff \forall E \in L. P \models E \iff Q \models E
\end{eqnarray*}

\begin{eqnarray*}
  P \approx_{K} Q
\end{eqnarray*}

\begin{eqnarray*}
  P \approx Q
\end{eqnarray*}

$\approx_{K} = \approx = \approx_{L}$

\subsubsection{Contextual duality}

Note that contexts extend the quotation operation to a family of
operations from processes to names. Given a context, $M$, we can
define a \emph{nominal context}, $\quotep{M}$ by $\quotep{M}[P] :=
\quotep{M[P]}$. To foreshadow what is to come we observe that these
operations enjoy a duality with processes very much like the duality
between vectors and maps from vectors to scalars.

Further, because the calculus is essentially higher-order, we have a
correspondence between contexts and processes. More specifically,
given a name $x$ and a context $M$ we can construct $M^{*}_{x}$ such
that 

\begin{mathpar}
  M^{*}_{x} | \lift{x}{P} \red M[P]
\end{mathpar}

namely,

\begin{mathpar}
  M^{*}_{x} := x?(u).M[\dropn{u}]
\end{mathpar}

The dependence of $M^{*}_{x}$ on a name makes it an abstraction, 

\begin{mathpar}
  M^{*} := (x)x?(u).M[\dropn{u}]
\end{mathpar}

\subsection{Additional notation}

It will sometimes be convenient to denote the process a name
quotes. We already have the notation $x = \quotep{P}$, but it will be
convenient to introduce an alternate notation, $\procn{x}$, when we
want to emphasize the connection to the use of the name. Note that, by
virtue of name equivalence, $\quotep{\procn{x}} \nameeq x$; so, the
notation is consistent with previous definitions.

Further, because names have structure it is possible to effect
substitutions on the basis of that structure. This means we need to
upgrade our notation for substitutions, which we accomplish by
adapting comprehension notation. Thus,

\begin{mathpar}
  P\{ y / x : x \in S \}
\end{mathpar}

is interpreted to mean the process derived from P by replacing (in a
capture-avoiding manner) each occurrence of $x$ in $S$ by $y$. For example,

\begin{mathpar}
  P\{ \quotep{\procn{x}|\procn{x}} / x : x \in \freenames{P} \}
\end{mathpar}

will replace each (occurrence) of a free name $x$ in $P$ by
$\quotep{\procn{x}|\procn{x}}$.

Also, we will avail ourselves of the notation $x^{L}$ and $x^{R}$ to
denote injections of a name into disjoint copies of the name
space. There are numerous ways to accomplish this. One example can be
found in \cite{MeredithR05}. This notation overloads to vectors of
names: $\vec{x}^{\pi} := (x_{i}^{\pi} \; : \; 0 \leq i < |\vec{x}| )$ where $\pi \in \{L,R\}$.

We also use $P^{\Box} := P|\Box$.

In \cite{MeredithR05} an interpretation of the new operator is
given. It turns out that there are several possible interpretations
all enjoying the requisite algebraic properties of the operator (see
\cite{milner91polyadicpi}). We will therefore make liberal use of
$(\nu\; \vec{x})P$.

% subsection the_syntax_and_semantics_of_the_notation_system (end)   

\input{qm2pi.qmops} 

\input{qm2pi.sterngerlach} 

\input{qm2pi.metric} 

% section concurrent_process_calculi (end)

%\input{qm2pi.proofsketch}

% section proof sketch (end)

%\input{qm2pi.slviaknots} 

% section spatial logic via knots (end)

\input{qm2pi.conclusion}

% section conclusion (end)

%\input{qm2pi.dtcodes} 

% section wiring algorithm (end)

\input{qm2pi.ack} 

% section acknowledgments (end)

\newpage


\bibliographystyle{plain}   
\bibliography{../../biblios/main.bib}

\input{qm2pi.rhodetails}

\end{document}

 

% subsection basic_interpretation (end)

%\input{qm2pi.rho.presentation} 
\subsection{The syntax and semantics of the notation system}\label{sub:the_syntax_and_semantics_of_the_notation_system} % (fold)

We now summarize a technical presentation of the calculus that
embodies our theory of dynamics. The typical presentation of such a
calculus follows the style of giving generators and relations on
them. The grammar, below, describing term constructors, freely
generates the set of processes, $\Proc$. This set is then quotiented
by a relation known as structural congruence and it is over this set
that the notion of dynamics is expressed. This presentation is
essentially that of \cite{MeredithR05} with the addition of
polyadicity and summation. For readability we have relegated some of
the technical subtleties to an appendix.

\subsubsection{Process grammar}\label{subsub:process_grammar}

\begin{mathpar}
  \inferrule* [lab=synchronization] {} {{M} \bc \pzero \;|\; x?F \;|\; x!C }
  \and
  \inferrule* [lab=abstraction] {} {{F} \bc (x)P}
  \and
  \inferrule* [lab=concretion] {} {{C} \bc \langle Q \rangle}
  \and
  \inferrule* [lab=process] {} {{P,Q} \bc M \;| \;P|Q \;|\; @{x}}
  \and
  \inferrule* [lab=name] {} {{x} \bc \quotep{P}}
\end{mathpar} 

Note that $\vec{x}$ (resp. $\vec{P}$) denotes a vector of names
(resp. processes) of length $|\vec{x}|$ (resp. $|\vec{P}|$). We adopt
the following useful abbreviations.

\begin{mathpar}
   x?(\vec{y}).P := x.(\vec{y})P \and  x\clift{\vec{P}} := x.\clift{\vec{P}}
   \and x!(y) := \lift{x}{\dropn{y}}
   \and \Pi_{i=0}^{n-1}P_i := P_0 | \ldots | P_{n-1}
\end{mathpar}

\subsubsection{Structural congruence}

\paragraph{Free and bound names and alpha-equivalence.} At the
core of structural equivalence is alpha-equivalence which identifies
process that are the same up to a change of variable. Formally, we
recognize the distinction between free and bound names. The free names
of a process, $\freenames{P}$, may be calculated recursively as
follows:

\begin{mathpar}
\freenames{\pzero} := \emptyset
  \and \\
  \freenames{x?(y).P} := \{ x \} \cup (\freenames{P} \setminus \{ y \})
  \and 
  \freenames{x!\langle P \rangle} := \{ x \} \cup \{ P \} 
  \and \\
  \freenames{P|Q} := \freenames{P} \cup \freenames{Q}
  \and \\
  \freenames{@{x}} := \{ x \}
\end{mathpar}

$\pi$
$\quotep{\pi}$

$\freenames{-} : \pi \to \mathcal{P}(\quotep{\pi})$

\begin{eqnarray*}
  \freenames{\pzero} & := & \emptyset \\
  \freenames{x?(y).P} & := & \{ x \} \cup (\freenames{P} \setminus \{ y \}) \\
  \freenames{x!\langle P \rangle} & := & \{ x \} \cup \{ P \} \\
  \freenames{P|Q} & := & \freenames{P} \cup \freenames{Q} \\
  \freenames{\dropn{x}} & := & \{ x \}
\end{eqnarray*}

The bound names of a process, $\boundnames{P}$, are those names occurring in $P$
that are not free. For example, in $x?(y).0$, the name $x$ is free, while $y$ is bound.

\begin{mathpar}
  \inferrule* [lab=monoidal-laws] {} { P|Q \equiv Q|P \and P|0 \equiv P \and P|(Q|R) \equiv (P|Q)|R }
\end{mathpar}

\begin{mathpar}
  \inferrule* [lab=alpha-equivalence] {} { (x)P \equiv (y)P\{y/x\} \and y \not\in \freenames{P} }
\end{mathpar}

\begin{definition}
Then two processes, $P,Q$, are alpha-equivalent if $P = Q\{\vec{y}/\vec{x}\}$ for
some $\vec{x} \in \boundnames{Q},\vec{y} \in \boundnames{P}$, where $Q\{\vec{y}/\vec{x}\}$
denotes the capture-avoiding substitution of $\vec{y}$ for $\vec{x}$ in $Q$.
\end{definition}

\begin{definition}
  The {\em structural congruence} \cite{SangiorgiWalker} , $\equiv$,
  between processes is the least congruence containing
  alpha-equivalence, satisfying the abelian monoid laws
  (associativity, commutativity and $\pzero$ as identity) for parallel
  composition $|$ and for summation $+$.
\end{definition}

\subsection{Name equivalence}

We take name equivalence, written $\nameeq$, to be the smallest
equivalence relation generated by the following rules.

\begin{mathpar}
\inferrule*[lab=Quote-drop]
{ }
{ \quotep{@{x}} \nameeq x }

\inferrule*[lab=Struct-equiv]
{ P \scong Q }
{ \quotep{P} \nameeq \quotep{Q} }
\end{mathpar}

The astute reader will have noticed that the mutual recursion of names
and processes imposes a mutual recursion on alpha-equivalence and
structural equivalence via name-equivalence. Fortunately, all of this
works out pleasantly and we may calculate in the natural way, free of
concern. The reader interested in the details is referred to the
appendix \ref{appendix:rho_details}.

\subsection{Substitution}

We use $\Proc$ for the set of processes, $\QProc$ for the set of
names, and $\id{\{}\vec{y} / \vec{x} \id{\}}$ to denote partial maps,
$s : \QProc \rightarrow \QProc$. A map, $s$ lifts, uniquely, to a map
on process terms, $\widehat{s} : \Proc \rightarrow \Proc$ by the
following equations.

\begin{mathpar}
  (0) \psubstp{Q}{P} := 0 \\
  (R \juxtap S) \psubstp{Q}{P}
  :=    
  (R)\psubstp{Q}{P} \juxtap (S) \psubstp{Q}{P} \\
  (x?(y).R) \psubstp{Q}{P}    
  :=    
  (x)\substp{Q}{P} (z)\concat( (R \psubstn{z}{y}) \psubstp{Q}{P} ) \\
  (\lift{x}{R}) \psubstp{Q}{P}  
  :=
  \lift{(x)\substp{Q}{P}}{ R \psubstp{Q}{P} } \\
%   (\dropn{x})  \psubstp{Q}{P}       
%   := 
%   \left\{ 
%     \begin{array}{ccc} 
%       \dropn{\quotep{Q}} & & x \nameeq \quotep{P} \\
%       \dropn{x} & & otherwise \\
%     \end{array}
%   \right. 
  (\dropn{x})  \psubstp{Q}{P}       
  := 
  \left\{ 
    \begin{array}{ccc} 
      Q & & x \nameeq \quotep{P} \\
      \dropn{x} & & otherwise \\
    \end{array}
  \right.
\end{mathpar}
 

where

\begin{eqnarray}
  (x)\id{\{} \lpquote Q \rpquote / \lpquote P \rpquote \id{\}}            = 
  \left\{ 
    \begin{array}{ccc}
      \lpquote Q \rpquote & & x \nameeq \lpquote P \rpquote \\
      x & & otherwise \\
    \end{array}
  \right. \nonumber
\end{eqnarray}

and $z$ is chosen distinct from $\quotep{P}$, $\quotep{Q}$, the free
names in $Q$, and all the names in $R$. Our $\alpha$-equivalence will
be built in the standard way from this substitution.

\begin{remark}\label{rem:no_self_referential_names}
  One consequence of these definitions is that $\forall P. \quotep{P}
  \not\in \freenames{P}$.
\end{remark}

\subsection{ Dynamic quote: an example }

Anticipating something of what's to come, consider applying the
substitution, $\widehat{\id{\{}u / z \id{\}}}$, to the following pair
of processes, $\lift{w}{y!(z)}$ and $w[ \lpquote y!(z) \rpquote ]$.

\begin{eqnarray}
	\lift{w}{y!(z)}\widehat{\id{\{}u / z \id{\}}}
		& = &
		\lift{w}{y!(u)} \nonumber\\
	w[ \lpquote y!(z) \rpquote ] \widehat{ \id{\{}u / z \id{\}} }
		& = &
		w[ \lpquote y!(z) \rpquote ] \nonumber
\end{eqnarray}

Because the body of the process between quotes is impervious to
substitution, we get radically different answers. In fact, by
examining the first process in an input context,
e.g. $x?(z).\lift{w}{y!(z)}$, we see that the process under the lift
operator may be shaped by prefixed inputs binding a name inside it. In
this sense, the lift operator will be seen as a way to dynamically
construct processes before reifying them as names.

Finally equipped with these standard features we can present the
dynamics of the calculus.

\subsubsection{Operational semantics} 

Finally, we introduce the computational dynamics. What marks these
algebras as distinct from other more traditionally studied algebraic
structures, e.g. vector spaces or polynomial rings, is the manner in
which dynamics is captured. In traditional structures, dynamics is typically
expressed through morphisms between such structures, as in linear maps
between vector spaces or morphisms between rings. In algebras
associated with the semantics of computation, the dynamics is
expressed as part of the algebraic structure itself, through a
reduction reduction relation typically denoted by $\red$. Below, we
give a recursive presentation of this relation for the calculus used
in the encoding.

$\red \subseteq \pi \times \pi$
$\red : \pi \to \mathcal{P}(\pi)$

\begin{mathpar}
  \inferrule* [lab=Comm] { \textsf{match}( x_{src}, x_{trgt} ) } { x_{trgt}?(y)P \; | \; x_{src}!\langle {Q} \rangle \red P\{\quotep{Q}/y}\} }
  \and \\
  \inferrule* [lab=Par] {{P} \red {P}'} {{{P} | {Q}} \red {{P}' | {Q}}}
  \and
  \inferrule* [lab=Equiv]{{{P} \scong {P}'} \andalso {{P}' \red {Q}'} \andalso {{Q}' \scong {Q}}}{{P} \red {Q}}
\end{mathpar}

\begin{eqnarray*}
  match_{\equiv} (\quotep{P},\quotep{Q}) & := & P \equiv Q \\
  match_{\dagger}(\quotep{P},\quotep{Q}) & := & \forall R. P|Q \red^{*} R => R \red^{*} 0 \\
  match_{K}(\quotep{P},\quotep{Q}) & := & K \mbox{ for some context } K
\end{eqnarray*}

$u?(x)P | u!\langle Q \rangle \red P\{\quotep{Q}/x\}$

%We write $\wred$ for $\red^*$, and $P\red$ if $\exists Q $ such that $ P \red Q$.
We write $P\red$ if $\exists Q $ such that $ P \red Q$ and $P\not\red$, otherwise.

\section{Replication}

As mentioned before, it is known that replication (and hence
recursion) can be implemented in a higher-order process algebra
\cite{SangiorgiWalker}. As our first example of calculation with the
machinery thus far presented we give the construction explicitly in
the {\rhoc}.

\begin{eqnarray}
	D_{x} & := & \prefix{x}{y}{(\binpar{\outputp{x}{y}}{@{y}})} \nonumber\\
	\bangp_{x}{P} & := & \binpar{{x}!\langle{\binpar{D_{x}}{P}}\rangle}{D_{x}} \nonumber
\end{eqnarray}

\begin{eqnarray}
	\bangp_{x}{P} & & \nonumber\\
	=
	& {x}!\langle{(\prefix{x}{y}{(\outputp{x}{y} | @{y})) | P}}\rangle 
	      | \prefix{x}{y}{(\outputp{x}{y} | @{y})} & \nonumber\\
	\red
	& (\outputp{x}{y} | @{y})\substn{\quotep{(\prefix{x}{y}{(@{y} | \outputp{x}{y})) | P}}}{y} & \nonumber\\
	=
	& \outputp{x}{\quotep{(\prefix{x}{y}{(\outputp{x}{y} | @{y})) | P}}}
	  | {(\prefix{x}{y}{(\outputp{x}{y} | @{y})) | P}} & \nonumber\\
	\red
	& \ldots & \nonumber\\
	\red^*
	& P | P | \ldots & \nonumber
\end{eqnarray}

Of course, this encoding, as an implementation, runs away, unfolding
$\bangp{P}$ eagerly. A lazier and more implementable replication
operator, restricted to input-guarded processes, may be obtained as follows.

\begin{eqnarray}
\bangp{\prefix{u}{v}{P}} 
	:= 
	\binpar{\lift{x}{\prefix{u}{v}{(\binpar{D(x)}{P})}}}{D(x)} \nonumber
\end{eqnarray}

\begin{remark}
  Note that the lazier definition still does not deal with summation
  or mixed summation (i.e. sums over input and output). The reader is
  invited to construct definitions of replication that deal with these
  features. 

  Further, the definitions are parameterized in a name, $x$. Can you,
  gentle reader, make a definition that eliminates this parameter and
  guarantees no accidental interaction between the replication
  machinery and the process being replicated -- i.e. no accidental
  sharing of names used by the process to get its work done and the
  name(s) used by the replication to effect copying. This latter
  revision of the definition of replication is crucial to obtaining
  the expected identity $!!P \sim !P$.
\end{remark}

\begin{remark}\label{rem:paradoxical_combinator}
  The reader familiar with the lambda calculus will have noticed the
  similarity between $D$ and the paradoxical combinator.

  [Ed. note: the existence of this seems to suggest we have to be more
  restrictive on the set of processes and names we admit if we are to
  support no-cloning.]
\end{remark}

\subsubsection{Bisimulation}

The computational dynamics gives rise to another kind of equivalence,
the equivalence of computational behavior. As previously mentioned
this is typically captured \emph{via} some form of bisimulation.

% The notion we use in this paper is weak barbed bisimulation
% \cite{milner91polyadicpi}.

The notion we use in this paper is derived from weak barbed
bisimulation \cite{milner91polyadicpi}. 

\begin{definition}
An \emph{observation relation}, $\downarrow_{\mathcal N}$, over a set
of names, $\mathcal N$, is the smallest relation satisfying the rules
below.

\infrule[Out-barb]{y \in {\mathcal N}, \; x \nameeq y}
		  {\outputp{x}{v} \downarrow_{\mathcal N} x}
\infrule[Par-barb]{\mbox{$P\downarrow_{\mathcal N} x$ or $Q\downarrow_{\mathcal N} x$}}
		  {\binpar{P}{Q} \downarrow_{\mathcal N} x}

We write $P \Downarrow_{\mathcal N} x$ if there is $Q$ such that 
$P \wred Q$ and $Q \downarrow_{\mathcal N} x$.
\end{definition}

\begin{definition}
%\label{def.bbisim}
An  ${\mathcal N}$-\emph{barbed bisimulation} over a set of names, ${\mathcal N}$, is a symmetric binary relation 
${\mathcal S}_{\mathcal N}$ between agents such that $P\rel{S}_{\mathcal N}Q$ implies:
\begin{enumerate}
\item If $P \red P'$ then $Q \wred Q'$ and $P'\rel{S}_{\mathcal N} Q'$.
\item If $P\downarrow_{\mathcal N} x$, then $Q\Downarrow_{\mathcal N} x$.
\end{enumerate}
$P$ is ${\mathcal N}$-barbed bisimilar to $Q$, written
$P \wbbisim_{\mathcal N} Q$, if $P \rel{S}_{\mathcal N} Q$ for some ${\mathcal N}$-barbed bisimulation ${\mathcal S}_{\mathcal N}$.
\end{definition}

$\mathcal{R} \subseteq \pi \times \pi$

$P \mathcal{R} Q => \forall P'. P \red P' \Rightarrow \exists Q'. Q \red Q', P' \mathcal{R} Q'$

$P \vdash x \Rightarrow Q \vdash x$

\begin{mathpar}
  \inferrule*[lab=Out-barb]{x \nameeq y}{{y}!\langle{Q}\rangle \vdash x}
  \and
  \inferrule*[lab=Par-barb]{\mbox{$P\vdash x$ or $Q\vdash x$}}{\binpar{P}{Q} \vdash x}
\end{mathpar}

\subsubsection{Contexts}

One of the principle advantages of computational calculi like the
$\pi$-calculus is a well-defined notion of context,
contextual-equivalence and a correlation between
contextual-equivalence and notions of bisimulation. The notion of
context allows the decomposition of a process into (sub-)process and
its syntactic environment, its context. Thus, a context may be
thought of as a process with a ``hole'' (written $\Box$) in it. The
application of a context $M$ to a process $P$, written $M[P]$, is
tantamount to filling the hole in $M$ with $P$. In this paper we do
not need the full weight of this theory, but do make use of the notion
of context in the proof the main theorem. 

\begin{mathpar}
  \inferrule* [lab=summation] {} {{M_{M},M_{N}} \bc \Box \;|\; x.M_{A} \;|\; M_{M}+M_{N}}
  \and
  \inferrule* [lab=agent] {} {{M_{A}} \bc (\vec{x})M_{P} \;| \; \clift{P_0,\ldots,M_{P},\ldots,P_N}}
  \and \\
  \inferrule* [lab=process] {} {{M_{P}} \bc M_{N} \;| \;P|M_{P} }
\end{mathpar} 

\begin{mathpar}
  \inferrule* [lab=sychronization] {} {M_{N} \bc \Box \;|\; x?M_{F} \;|\; x!M_{C}}
  \and
  \inferrule* [lab=abstraction] {} {{M_{F}} \bc (x)M_{P} }
  \and
  \inferrule* [lab=concretion] {} {{M_{C}} \bc \langle M_{P} \rangle }
  \and \\
  \inferrule* [lab=process] {} {{M_{P}} \bc M_{N} \;| \;P|M_{P} }
\end{mathpar}

\begin{definition}[contextual application] Given a context $M$, and
  process $P$, we define the \emph{contextual application}, $M[P] :=
  M\{P/\Box\}$. That is, the contextual application of M to P is the
  substitution of $P$ for $\Box$ in $M$.
\end{definition}

$\meaningof{-} : L \to \mathcal{P}(\pi)$

\begin{mathpar}
  \inferrule* [lab=collection] {} {\meaningof{true} = \pi, \and \meaningof{~E} = \pi \setminus \meaningof{E}, \and \meaningof{E_{1} \& E_{2}} = \meaningof{E_{1}} \cap \meaningof{E_{2}}}
\end{mathpar}

\begin{mathpar}
  \inferrule* [lab=structure] {} {\meaningof{0} = \{ P \in \pi | P \equiv 0 \}, \and \\ \meaningof{E_1 | E_2} = \{ P \in \pi | P \equiv P_{1} | P_{2}, P_{1} \in \meaningof{E_{1}}, P_{2} \in \meaningof{E_2}\} }
\end{mathpar}

\begin{mathpar}
 \inferrule* [lab=behavior] {} {\meaningof{\langle a?b \rangle E} = \{ P \in \pi | P \equiv Q | u?(y)P', \\ \and \\\\ \and \\ \;\;\; u \in \meaningof{a}, \forall z.P'\{z/y\} \in \meaningof{E\{z/b\}}\}, \and \\ \meaningof{a!E} = \{ P \in \pi | P \equiv Q | x!\langle P' \rangle, x \in \meaningof{a} P' \in \meaningof{E}\} }
\end{mathpar}

\begin{mathpar}
 \inferrule* [lab=nominal] {} {\meaningof{\quotep{E}} = \{ \quotep{P} \in \quotep{\pi} | P \in \meaningof{E} \}, \and \meaningof{\quotep{P}} = \{ \quotep{Q} \in \quotep{\pi} | P \equiv Q \} \and \\ \meaningof{@\quotep{E}} = \{ P \in \pi | P \equiv @x, x \in \meaningof{E} \}}
\end{mathpar}

\begin{eqnarray*}
  \\
  \meaningof{-} : TS \to ST
\end{eqnarray*}

\begin{eqnarray*}
  \\
  L : TS \to ST
\end{eqnarray*}

\begin{eqnarray*}
  \\
  P \models E \iff P \in \meaningof{E}
\end{eqnarray*}

\begin{eqnarray*}
  P \approx_{L} Q \iff \forall E \in L. P \models E \iff Q \models E
\end{eqnarray*}

\begin{eqnarray*}
  P \approx_{K} Q
\end{eqnarray*}

\begin{eqnarray*}
  P \approx Q
\end{eqnarray*}

$\approx_{K} = \approx = \approx_{L}$

\subsubsection{Contextual duality}

Note that contexts extend the quotation operation to a family of
operations from processes to names. Given a context, $M$, we can
define a \emph{nominal context}, $\quotep{M}$ by $\quotep{M}[P] :=
\quotep{M[P]}$. To foreshadow what is to come we observe that these
operations enjoy a duality with processes very much like the duality
between vectors and maps from vectors to scalars.

Further, because the calculus is essentially higher-order, we have a
correspondence between contexts and processes. More specifically,
given a name $x$ and a context $M$ we can construct $M^{*}_{x}$ such
that 

\begin{mathpar}
  M^{*}_{x} | \lift{x}{P} \red M[P]
\end{mathpar}

namely,

\begin{mathpar}
  M^{*}_{x} := x?(u).M[\dropn{u}]
\end{mathpar}

The dependence of $M^{*}_{x}$ on a name makes it an abstraction, 

\begin{mathpar}
  M^{*} := (x)x?(u).M[\dropn{u}]
\end{mathpar}

\subsection{Additional notation}

It will sometimes be convenient to denote the process a name
quotes. We already have the notation $x = \quotep{P}$, but it will be
convenient to introduce an alternate notation, $\procn{x}$, when we
want to emphasize the connection to the use of the name. Note that, by
virtue of name equivalence, $\quotep{\procn{x}} \nameeq x$; so, the
notation is consistent with previous definitions.

Further, because names have structure it is possible to effect
substitutions on the basis of that structure. This means we need to
upgrade our notation for substitutions, which we accomplish by
adapting comprehension notation. Thus,

\begin{mathpar}
  P\{ y / x : x \in S \}
\end{mathpar}

is interpreted to mean the process derived from P by replacing (in a
capture-avoiding manner) each occurrence of $x$ in $S$ by $y$. For example,

\begin{mathpar}
  P\{ \quotep{\procn{x}|\procn{x}} / x : x \in \freenames{P} \}
\end{mathpar}

will replace each (occurrence) of a free name $x$ in $P$ by
$\quotep{\procn{x}|\procn{x}}$.

Also, we will avail ourselves of the notation $x^{L}$ and $x^{R}$ to
denote injections of a name into disjoint copies of the name
space. There are numerous ways to accomplish this. One example can be
found in \cite{MeredithR05}. This notation overloads to vectors of
names: $\vec{x}^{\pi} := (x_{i}^{\pi} \; : \; 0 \leq i < |\vec{x}| )$ where $\pi \in \{L,R\}$.

We also use $P^{\Box} := P|\Box$.

In \cite{MeredithR05} an interpretation of the new operator is
given. It turns out that there are several possible interpretations
all enjoying the requisite algebraic properties of the operator (see
\cite{milner91polyadicpi}). We will therefore make liberal use of
$(\nu\; \vec{x})P$.

% subsection the_syntax_and_semantics_of_the_notation_system (end)   

\section{Interpretation of QM}
\subsection{Supporting definitions}
\subsubsection{Multiplication}
\begin{mathpar}
  \quotep{Q} \cdot \quotep{R} := \quotep{Q|R}
  \and \\
  \quotep{Q} \cdot P := P\{ \quotep{Q|R} / \quotep{R} : \quotep{R} \in \freenames{P} \}
\end{mathpar}

\paragraph{Discussion}
The first line needs little explanation. The second line says that
each free name of the process is replaced with the multiplication of
that name by the scalar. Multiplication of a scalar (name) by a state
(process) results in a process all the names of which have been `moved
over' by parallel composition with the process the scalar
quotes. There is a subtlety that the bound names have to be
manipulated so that multiplied names aren't accidentally
captured. There are many ways to achieve this.

\begin{remark}\label{rem:multiplication_identities}
  The reader is invited to verify that for all $x,y,z \in \QProc$ and $P \in \Proc$
  \begin{mathpar}
    x \cdot \quotep{0} \equiv x 
    \and
    x \cdot y \equiv y \cdot x
    \and
    x \cdot (y \cdot z) \equiv (x \cdot y) \cdot z
    \and \\
    \quotep{0} \cdot P \equiv P
    \and \\
    x \cdot (y \cdot P) \equiv (x \cdot y) \cdot P
    \and \\
    x \cdot (P|Q) \equiv (x \cdot P) | (x \cdot Q)
    \and \\    
  \end{mathpar}
\end{remark}

\subsubsection{Tensor product}

We define a tensor product on processes by structural induction.

\paragraph{Tensor of sums} First note that all summations, including
$\pzero$ and sequence, can be written $\Sigma_{i} x_{i}.A_{i} +
\Sigma_{j} x_{j}.C_{j}$, where we have grouped input-guarded processes
together and output-guarded processes together.

Thus, we can define the tensor product of two summations, $N_{1}\otimes N_{2}$, where

\begin{mathpar}
  N_{1} := \Sigma_{i} x_{i}.A_{i} + \Sigma_{j} x_{j}.C_{j}
  \and
  N_{2} := \Sigma_{i'} y_{i'}.B_{i'} + \Sigma_{j'} y_{j'}.D_{j'} 
\end{mathpar}

as follows.

\begin{mathpar}
  \Sigma_{i} x_{i}.A_{i} + \Sigma_{j} x_{j}.C_{j} \otimes \Sigma_{i'}
  y_{i'}.B_{i'} + \Sigma_{j'} y_{j'}.D_{j'} 
  \and \\
  := \; \Sigma_{i} \Sigma_{i'} \quotep{\stackrel{\vee}{x_{i}}| \stackrel{\vee}{y_{i'}}}.(A_{i}\otimes B_{i'}) \; | \; \Sigma_{i'} \Sigma_{i} \quotep{\stackrel{\vee}{y_{i'}}|\stackrel{\vee}{x_{i}}}.(B_{i'}\otimes A_{i})
  \and
  \;\; | \;\; \Sigma_{j} \Sigma_{j'} \quotep{\stackrel{\vee}{x_{j}}|\stackrel{\vee}{y_{j'}}}.(A_{j}\otimes B_{j'}) \; | \; \Sigma_{j'} \Sigma_{j} \quotep{\stackrel{\vee}{y_{j'}}|\stackrel{\vee}{x_{j}}}.(B_{j'}\otimes A_{j})
\end{mathpar}

\begin{remark}
  Do we need to $x^{L}$ and $y^{R}$ for this construction as well?
\end{remark}

\paragraph{Tensor of parallel compositions} Next, we distribute tensor
over par.

\begin{mathpar}
  P_{1}|P_{2} \otimes Q_{1}|Q_{2} := (P_{1} \otimes Q_{1}) | (P_{1}
  \otimes Q_{2}) | (P_{2} \otimes Q_{1}) | (P_{2} \otimes Q_{2})
\end{mathpar}

\paragraph{Tensor with dropped names} We treat tensor of a
process with a dropped name as parallel composition.

\begin{mathpar}
  P \otimes \dropn{x} := P | \dropn{x}
\end{mathpar}

\paragraph{Tensor of agents}

Finally, we need to define tensor on agents. Note that the definition
of tensor on normal products only tensors inputs with inputs and
outputs with outputs. Thus, we only have to define the operation on
``homogeneous'' pairings.

\begin{mathpar}
  (\vec{x})P \otimes (\vec{y})Q
  \and \\
  := (x_{0}^{L}|y_{0}^{R},\ldots,x_{0}^{L}|y_{n}^{R},\ldots,x_{m}^{L}|y_{0}^{R},\ldots,x_{m}^{L}|y_{n}^R)(P\{ \vec{x}^{L}/\vec{x}\} \otimes Q \{ \vec{y}^{R}/\vec{y}\})
  \and \\
  \clift{\vec{P}} \otimes \clift{\vec{Q}}
  \and \\
  := \clift{P_{0}\otimes Q_{0},\ldots,P_{0}\otimes Q_{n},\ldots,P_{m}\otimes Q_{0},\ldots,P_{m}\otimes Q_{n}}
\end{mathpar}

\begin{remark}
  Observe that arities of tensored abstractions matches arities of
  tensored concretions if the original arities matched. Note also that
  the length of the arities corresponds to the increase in dimension
  we see in ordinary vector space tensor product.
\end{remark}

\begin{remark}
  Operationally, this definition distributes the tensor down to
  components ``linked'' by summation. Tensor over summation is
  intriguing in that it mixes names. Moreover, as a consequence of the
  way it mixes names we have the identities for all $x \in \QProc$ and
  $P,Q \in \Proc$

  \begin{mathpar}
    (x \cdot P) \otimes Q \equiv x \cdot (P \otimes Q) \equiv P \otimes (x \cdot Q)
    \and
    P \otimes \pzero \equiv P
  \end{mathpar}

  that the reader is invited to verify.
\end{remark}

\subsubsection{Annihilation}
\begin{mathpar}
  P^{\perp} := \{ Q | \forall R. P|Q \red^{*} R \Rightarrow R \red^{*} \pzero \}
  \and \\
  P^{\underline{\perp}} := \Sigma_{Q \in P^{\perp}} \quotep{Q}?(y).(\dropn{y}|Q) | \Sigma_{Q \in P^{\perp}} \quotep{Q}\clift{\Box}
\end{mathpar}

\paragraph{Discussion} The reader will note that $P^{\perp}$ is a
\emph{set} of processes, while $P^{\underline{\perp}}$ is a
\emph{context}. We call the set $P^{\perp}$ the \emph{annihilators} of
$P$. The parallel composition of a process in the annihilators of $P$
with $P$ will result in a process, the state space of which has all
paths eventually leading to $\pzero$. Execution may endure loops; but
under reasonable conditions of fairness (naturally guaranteed under
most notions of bisimulation) such a composite process cannot get
stuck in such a loop and will, eventually pop out and terminate.

The context $P^{\underline{\perp}}$ is ready and willing to ``take the
$P$ out of'' the process to which it is applied. It will effectively
transmit the code of the process to which it is applied to one of the
annihilators and run the process against it.

\subsubsection{Evaluation}
We fix $M$ a domain of fully abstract interpretation with an equality
coincident with bisimulation. We take $\meaningof{\cdot} : \Proc \to
M$ to be the map interpreting processes and $\nmeaningof{\cdot} : \M
\to Proc$ to be the map running the other way. Then we define

\begin{mathpar}
  \int P := \nmeaningof{\meaningof{P}}
\end{mathpar}

\paragraph{Discussion}
There are many fully abstract interpretations of Milner's
$\pi$-calculus. Any of them can be used as a basis for interpreting
the reflective calculus here. Equipped with such a domain it is
largely a matter of grinding through to check that the Yoneda
construction for the normalization-by-evaluation program can be
extended to this setting.

\begin{remark}
  The reader is invited to verify that $\int (P^{\underline{\perp}}[P]) = 0$.
\end{remark}

\subsection{Quantum mechanics}

Table \ref{tbl:core_qm_op_defns} gives the core operational definitions

\begin{table}[htp]\label{tbl:core_qm_op_defns}
  \center{
    \fbox{
      \begin{tabular}{c|c}
        quantum mechanics & process calculus \\
        \hline
        scalar & $x := \quotep{P}$ \\
        state vector & $\state{P} := P$ \\
        dual & $\state{P}^{*} := \event{P^{\underline{\perp}}} := \quotep{P^{\underline{\perp}}}[-]$ \\
        matrix & $ \Sigma_{\alpha} \state{P_{\alpha}}x_{\alpha}\event{Q_{\alpha}}$ \\
        vector addition & $\state{P} + \state{Q} := \state{P | Q}$ \\
        tensor product & $\state{P} \otimes \state{Q} := \state{P \otimes Q}$ \\
        inner product & $\innerprod{P}{Q} := \quotep{\int P^{\underline{\perp}}[Q]}$ \\
      \end{tabular}
    }
  }
  \caption{QM - operational definitions}
\end{table}

where

\begin{mathpar}
  \prmatrix{P}{Q} := \fprmatrix{P}{\quotep{\pzero}}{Q}
  \and
  \fprmatrix{P}{x}{Q} := (\state{P},x,\event{Q})
  \and
  (\fprmatrix{P}{x}{Q})(\state{R}) := x \cdot \innerprod{Q}{R} \cdot \state{P}
  \and
  (\fprmatrix{P}{x}{Q})(\event{R}) := x \cdot \innerprod{R}{P} \cdot \event{Q}
\end{mathpar}

\paragraph{Discussion}
As promised: vectors (aka states) are represented as processes; duals
as contextual duals; inner product definition should be compared with
standard inner product definition for ....

\begin{remark}
  Assuming $\int (P^{\underline{\perp}}[P]) = 0$, the reader is
  invited to verify that $(\fprmatrix{P}{x}{P})(\state{P}) = x \cdot \state{P}$.
\end{remark}

\begin{remark}
  The reader is invited to verify that $\innerprod{P}{Q}$ could
  equally well have been written $\quotep{\int \stackrel{\vee}{x}}$
  where $x = \event{P^{\underline{\perp}}}(Q)$.

  One of the motivations for this remark is that there is another way
  to factor these operations. We could package up evaluation in the dual:

  \begin{mathpar}
    \state{P}^{*} := \event{\int P^{\underline{\perp}}} := \quotep{\int P^{\underline{\perp}}}[-]
  \end{mathpar}

  and then have inner product defined by
  
  \begin{mathpar}
    \innerprod{P}{Q} := \event{P}(Q)
  \end{mathpar}

  Hopefully, experience with the calculations will provide guidance on
  the best factoring.
\end{remark}

\begin{remark}
  Assuming $\int (P^{\underline{\perp}}[P]) = 0$, the reader is
  invited to verify that $\forall P,Q. (\prmatrix{0}{Q})(\state{0}) =
  \state{0}$ and dually $(\prmatrix{P}{0})(\event{0}) = \event{0}$.
\end{remark}

\begin{remark}
  i'm a little worried that i don't (yet) have proper support for
  complex conjugacy. But, the observation above may give us a
  clue. According to Abramsky, it must be the case that the scalars
  are iso to the homset of the identity for the tensor -- which the
  observation above characterizes. 

  For now, we will simply bookmark the notion with $\overline{x}$.
\end{remark}

\subsubsection{Adjointness}

We need to give a definition of $(\cdot)^{\dagger}$ for matrices. The
obvious candidate definition is
\begin{mathpar}
(\Sigma_{\alpha}\fprmatrix{P_{\alpha}}{x_{\alpha}}{Q_{\alpha}})^{\dagger}
= \Sigma_{\alpha}\fprmatrix{(Q_{\alpha}^{\underline{\perp}})^{*}}{\overline{x}_{\alpha}}{P_{\alpha}^{\underline{\perp}}} 
\end{mathpar}

But, $(Q_{\alpha}^{\underline{\perp}})^{*}$ requires a name along
which to communicate the process to achieve the context application.

\subsubsection{Basis for a basis}
If processes label states and ``addition'' of states (a.k.a. vector
addition) is interpreted as parallel composition, what corresponds to
notions of linear independence and basis? Here, we recall that Yoshida
has developed a set of \emph{combinators} for an asynchronous verison
of Milner's $\pi$-calculus. These are a finite set of processes such
any process can be expressed as parallel composition of these
combinators together with liberal uses of the new operator and
replication. We can simply give a translation of these into the
present calculus and have reasonable expectation that the property
carries over. That is, that the resultant set allows to express all
processes via parallel composition. Note, however, that there is no
new operator or replication in this calculus. As a result, we expect
that the corresponding set is actually infinite. That is, we expect
that the space is actually infinite dimensional.

\begin{remark}
  The attentive reader may be a bit concerned. Certainly, the
  collection $S$, $K$ and $I$ is a finite set of
  combinators. Shouldn't we expect to see a finite set of combinators
  for an effectively equivalent system? i am very sympathetic to this
  critique and feel it warrants full attention. On the other hand, i
  also have in mind the following analogy. The natural numbers, as a
  monoid under addition, has exactly $1$ generator, while the natural
  numbers, as a monoid under multiplication, has countably many
  generators (the primes). We observe that the application of the
  lambda calculus is much less resource sensitive than the parallel
  composition of the $\pi$-calculus. Could it be the case that we have
  an analogy of the form
  
  \begin{mathpar}
    m + n : MN :: m*n : M|N
  \end{mathpar}

  giving a similar blow up in the set of ``primes''?  This is such a
  wonderful thought that, even if it's not true, i think it's worth
  writing down.
\end{remark}
 

\documentclass[12pt]{llncs}
%\documentclass{jktr}

\usepackage[pdftex]{hyperref}                   
\usepackage {listings}
\usepackage {mathpartir}
\usepackage{bcprules}
%\usepackage{listings}
                       
\usepackage{graphicx} 
%\usepackage[margins=2.5cm,nohead,nofoot]{geometry}
%\usepackage{geometry}
\usepackage{amsfonts}
\usepackage{amstext}
\usepackage{latexsym}
\usepackage{amssymb}
\usepackage{color}


%\include{myPreamble}
\include{qm2pi.local} 

%\ifpdf
%\usepackage[pdftex]{graphicx}
%\else
%\usepackage{graphicx}
%\fi

 % \ifpdf
%  \usepackage{pdfsync}
%  \if


%\title{Brief Article}
%\author{David F. Snyder}
%\author{L.G. Meredith}

%\address{Dept. of Math., Texas State University--San Marcos, San Marcos, TX 78666}
       
\pagestyle{empty}


\begin{document}

\lstset{language=[Objective]Caml,frame=shadowbox}

\input{qm2pi.front}

% section front matter (end)

\input{qm2pi.intro} 
 
% section introduction (end)

% \input{qm2pi.knotations} 

% section notation (end)

\input{qm2pi.process.calculi} 

% section concurrent_process_calculi_and_spatial_logics_ (end)
    
%\input{qm2pi.knots2pi} 

%\input{qm2pi.trefoil} 

%\input{qm2pi.mainthm} 

% subsection basic_interpretation (end)

%\input{qm2pi.rho.presentation} 
\subsection{The syntax and semantics of the notation system}\label{sub:the_syntax_and_semantics_of_the_notation_system} % (fold)

We now summarize a technical presentation of the calculus that
embodies our theory of dynamics. The typical presentation of such a
calculus follows the style of giving generators and relations on
them. The grammar, below, describing term constructors, freely
generates the set of processes, $\Proc$. This set is then quotiented
by a relation known as structural congruence and it is over this set
that the notion of dynamics is expressed. This presentation is
essentially that of \cite{MeredithR05} with the addition of
polyadicity and summation. For readability we have relegated some of
the technical subtleties to an appendix.

\subsubsection{Process grammar}\label{subsub:process_grammar}

\begin{mathpar}
  \inferrule* [lab=synchronization] {} {{M} \bc \pzero \;|\; x?F \;|\; x!C }
  \and
  \inferrule* [lab=abstraction] {} {{F} \bc (x)P}
  \and
  \inferrule* [lab=concretion] {} {{C} \bc \langle Q \rangle}
  \and
  \inferrule* [lab=process] {} {{P,Q} \bc M \;| \;P|Q \;|\; @{x}}
  \and
  \inferrule* [lab=name] {} {{x} \bc \quotep{P}}
\end{mathpar} 

Note that $\vec{x}$ (resp. $\vec{P}$) denotes a vector of names
(resp. processes) of length $|\vec{x}|$ (resp. $|\vec{P}|$). We adopt
the following useful abbreviations.

\begin{mathpar}
   x?(\vec{y}).P := x.(\vec{y})P \and  x\clift{\vec{P}} := x.\clift{\vec{P}}
   \and x!(y) := \lift{x}{\dropn{y}}
   \and \Pi_{i=0}^{n-1}P_i := P_0 | \ldots | P_{n-1}
\end{mathpar}

\subsubsection{Structural congruence}

\paragraph{Free and bound names and alpha-equivalence.} At the
core of structural equivalence is alpha-equivalence which identifies
process that are the same up to a change of variable. Formally, we
recognize the distinction between free and bound names. The free names
of a process, $\freenames{P}$, may be calculated recursively as
follows:

\begin{mathpar}
\freenames{\pzero} := \emptyset
  \and \\
  \freenames{x?(y).P} := \{ x \} \cup (\freenames{P} \setminus \{ y \})
  \and 
  \freenames{x!\langle P \rangle} := \{ x \} \cup \{ P \} 
  \and \\
  \freenames{P|Q} := \freenames{P} \cup \freenames{Q}
  \and \\
  \freenames{@{x}} := \{ x \}
\end{mathpar}

$\pi$
$\quotep{\pi}$

$\freenames{-} : \pi \to \mathcal{P}(\quotep{\pi})$

\begin{eqnarray*}
  \freenames{\pzero} & := & \emptyset \\
  \freenames{x?(y).P} & := & \{ x \} \cup (\freenames{P} \setminus \{ y \}) \\
  \freenames{x!\langle P \rangle} & := & \{ x \} \cup \{ P \} \\
  \freenames{P|Q} & := & \freenames{P} \cup \freenames{Q} \\
  \freenames{\dropn{x}} & := & \{ x \}
\end{eqnarray*}

The bound names of a process, $\boundnames{P}$, are those names occurring in $P$
that are not free. For example, in $x?(y).0$, the name $x$ is free, while $y$ is bound.

\begin{mathpar}
  \inferrule* [lab=monoidal-laws] {} { P|Q \equiv Q|P \and P|0 \equiv P \and P|(Q|R) \equiv (P|Q)|R }
\end{mathpar}

\begin{mathpar}
  \inferrule* [lab=alpha-equivalence] {} { (x)P \equiv (y)P\{y/x\} \and y \not\in \freenames{P} }
\end{mathpar}

\begin{definition}
Then two processes, $P,Q$, are alpha-equivalent if $P = Q\{\vec{y}/\vec{x}\}$ for
some $\vec{x} \in \boundnames{Q},\vec{y} \in \boundnames{P}$, where $Q\{\vec{y}/\vec{x}\}$
denotes the capture-avoiding substitution of $\vec{y}$ for $\vec{x}$ in $Q$.
\end{definition}

\begin{definition}
  The {\em structural congruence} \cite{SangiorgiWalker} , $\equiv$,
  between processes is the least congruence containing
  alpha-equivalence, satisfying the abelian monoid laws
  (associativity, commutativity and $\pzero$ as identity) for parallel
  composition $|$ and for summation $+$.
\end{definition}

\subsection{Name equivalence}

We take name equivalence, written $\nameeq$, to be the smallest
equivalence relation generated by the following rules.

\begin{mathpar}
\inferrule*[lab=Quote-drop]
{ }
{ \quotep{@{x}} \nameeq x }

\inferrule*[lab=Struct-equiv]
{ P \scong Q }
{ \quotep{P} \nameeq \quotep{Q} }
\end{mathpar}

The astute reader will have noticed that the mutual recursion of names
and processes imposes a mutual recursion on alpha-equivalence and
structural equivalence via name-equivalence. Fortunately, all of this
works out pleasantly and we may calculate in the natural way, free of
concern. The reader interested in the details is referred to the
appendix \ref{appendix:rho_details}.

\subsection{Substitution}

We use $\Proc$ for the set of processes, $\QProc$ for the set of
names, and $\id{\{}\vec{y} / \vec{x} \id{\}}$ to denote partial maps,
$s : \QProc \rightarrow \QProc$. A map, $s$ lifts, uniquely, to a map
on process terms, $\widehat{s} : \Proc \rightarrow \Proc$ by the
following equations.

\begin{mathpar}
  (0) \psubstp{Q}{P} := 0 \\
  (R \juxtap S) \psubstp{Q}{P}
  :=    
  (R)\psubstp{Q}{P} \juxtap (S) \psubstp{Q}{P} \\
  (x?(y).R) \psubstp{Q}{P}    
  :=    
  (x)\substp{Q}{P} (z)\concat( (R \psubstn{z}{y}) \psubstp{Q}{P} ) \\
  (\lift{x}{R}) \psubstp{Q}{P}  
  :=
  \lift{(x)\substp{Q}{P}}{ R \psubstp{Q}{P} } \\
%   (\dropn{x})  \psubstp{Q}{P}       
%   := 
%   \left\{ 
%     \begin{array}{ccc} 
%       \dropn{\quotep{Q}} & & x \nameeq \quotep{P} \\
%       \dropn{x} & & otherwise \\
%     \end{array}
%   \right. 
  (\dropn{x})  \psubstp{Q}{P}       
  := 
  \left\{ 
    \begin{array}{ccc} 
      Q & & x \nameeq \quotep{P} \\
      \dropn{x} & & otherwise \\
    \end{array}
  \right.
\end{mathpar}
 

where

\begin{eqnarray}
  (x)\id{\{} \lpquote Q \rpquote / \lpquote P \rpquote \id{\}}            = 
  \left\{ 
    \begin{array}{ccc}
      \lpquote Q \rpquote & & x \nameeq \lpquote P \rpquote \\
      x & & otherwise \\
    \end{array}
  \right. \nonumber
\end{eqnarray}

and $z$ is chosen distinct from $\quotep{P}$, $\quotep{Q}$, the free
names in $Q$, and all the names in $R$. Our $\alpha$-equivalence will
be built in the standard way from this substitution.

\begin{remark}\label{rem:no_self_referential_names}
  One consequence of these definitions is that $\forall P. \quotep{P}
  \not\in \freenames{P}$.
\end{remark}

\subsection{ Dynamic quote: an example }

Anticipating something of what's to come, consider applying the
substitution, $\widehat{\id{\{}u / z \id{\}}}$, to the following pair
of processes, $\lift{w}{y!(z)}$ and $w[ \lpquote y!(z) \rpquote ]$.

\begin{eqnarray}
	\lift{w}{y!(z)}\widehat{\id{\{}u / z \id{\}}}
		& = &
		\lift{w}{y!(u)} \nonumber\\
	w[ \lpquote y!(z) \rpquote ] \widehat{ \id{\{}u / z \id{\}} }
		& = &
		w[ \lpquote y!(z) \rpquote ] \nonumber
\end{eqnarray}

Because the body of the process between quotes is impervious to
substitution, we get radically different answers. In fact, by
examining the first process in an input context,
e.g. $x?(z).\lift{w}{y!(z)}$, we see that the process under the lift
operator may be shaped by prefixed inputs binding a name inside it. In
this sense, the lift operator will be seen as a way to dynamically
construct processes before reifying them as names.

Finally equipped with these standard features we can present the
dynamics of the calculus.

\subsubsection{Operational semantics} 

Finally, we introduce the computational dynamics. What marks these
algebras as distinct from other more traditionally studied algebraic
structures, e.g. vector spaces or polynomial rings, is the manner in
which dynamics is captured. In traditional structures, dynamics is typically
expressed through morphisms between such structures, as in linear maps
between vector spaces or morphisms between rings. In algebras
associated with the semantics of computation, the dynamics is
expressed as part of the algebraic structure itself, through a
reduction reduction relation typically denoted by $\red$. Below, we
give a recursive presentation of this relation for the calculus used
in the encoding.

$\red \subseteq \pi \times \pi$
$\red : \pi \to \mathcal{P}(\pi)$

\begin{mathpar}
  \inferrule* [lab=Comm] { \textsf{match}( x_{src}, x_{trgt} ) } { x_{trgt}?(y)P \; | \; x_{src}!\langle {Q} \rangle \red P\{\quotep{Q}/y}\} }
  \and \\
  \inferrule* [lab=Par] {{P} \red {P}'} {{{P} | {Q}} \red {{P}' | {Q}}}
  \and
  \inferrule* [lab=Equiv]{{{P} \scong {P}'} \andalso {{P}' \red {Q}'} \andalso {{Q}' \scong {Q}}}{{P} \red {Q}}
\end{mathpar}

\begin{eqnarray*}
  match_{\equiv} (\quotep{P},\quotep{Q}) & := & P \equiv Q \\
  match_{\dagger}(\quotep{P},\quotep{Q}) & := & \forall R. P|Q \red^{*} R => R \red^{*} 0 \\
  match_{K}(\quotep{P},\quotep{Q}) & := & K \mbox{ for some context } K
\end{eqnarray*}

$u?(x)P | u!\langle Q \rangle \red P\{\quotep{Q}/x\}$

%We write $\wred$ for $\red^*$, and $P\red$ if $\exists Q $ such that $ P \red Q$.
We write $P\red$ if $\exists Q $ such that $ P \red Q$ and $P\not\red$, otherwise.

\section{Replication}

As mentioned before, it is known that replication (and hence
recursion) can be implemented in a higher-order process algebra
\cite{SangiorgiWalker}. As our first example of calculation with the
machinery thus far presented we give the construction explicitly in
the {\rhoc}.

\begin{eqnarray}
	D_{x} & := & \prefix{x}{y}{(\binpar{\outputp{x}{y}}{@{y}})} \nonumber\\
	\bangp_{x}{P} & := & \binpar{{x}!\langle{\binpar{D_{x}}{P}}\rangle}{D_{x}} \nonumber
\end{eqnarray}

\begin{eqnarray}
	\bangp_{x}{P} & & \nonumber\\
	=
	& {x}!\langle{(\prefix{x}{y}{(\outputp{x}{y} | @{y})) | P}}\rangle 
	      | \prefix{x}{y}{(\outputp{x}{y} | @{y})} & \nonumber\\
	\red
	& (\outputp{x}{y} | @{y})\substn{\quotep{(\prefix{x}{y}{(@{y} | \outputp{x}{y})) | P}}}{y} & \nonumber\\
	=
	& \outputp{x}{\quotep{(\prefix{x}{y}{(\outputp{x}{y} | @{y})) | P}}}
	  | {(\prefix{x}{y}{(\outputp{x}{y} | @{y})) | P}} & \nonumber\\
	\red
	& \ldots & \nonumber\\
	\red^*
	& P | P | \ldots & \nonumber
\end{eqnarray}

Of course, this encoding, as an implementation, runs away, unfolding
$\bangp{P}$ eagerly. A lazier and more implementable replication
operator, restricted to input-guarded processes, may be obtained as follows.

\begin{eqnarray}
\bangp{\prefix{u}{v}{P}} 
	:= 
	\binpar{\lift{x}{\prefix{u}{v}{(\binpar{D(x)}{P})}}}{D(x)} \nonumber
\end{eqnarray}

\begin{remark}
  Note that the lazier definition still does not deal with summation
  or mixed summation (i.e. sums over input and output). The reader is
  invited to construct definitions of replication that deal with these
  features. 

  Further, the definitions are parameterized in a name, $x$. Can you,
  gentle reader, make a definition that eliminates this parameter and
  guarantees no accidental interaction between the replication
  machinery and the process being replicated -- i.e. no accidental
  sharing of names used by the process to get its work done and the
  name(s) used by the replication to effect copying. This latter
  revision of the definition of replication is crucial to obtaining
  the expected identity $!!P \sim !P$.
\end{remark}

\begin{remark}\label{rem:paradoxical_combinator}
  The reader familiar with the lambda calculus will have noticed the
  similarity between $D$ and the paradoxical combinator.

  [Ed. note: the existence of this seems to suggest we have to be more
  restrictive on the set of processes and names we admit if we are to
  support no-cloning.]
\end{remark}

\subsubsection{Bisimulation}

The computational dynamics gives rise to another kind of equivalence,
the equivalence of computational behavior. As previously mentioned
this is typically captured \emph{via} some form of bisimulation.

% The notion we use in this paper is weak barbed bisimulation
% \cite{milner91polyadicpi}.

The notion we use in this paper is derived from weak barbed
bisimulation \cite{milner91polyadicpi}. 

\begin{definition}
An \emph{observation relation}, $\downarrow_{\mathcal N}$, over a set
of names, $\mathcal N$, is the smallest relation satisfying the rules
below.

\infrule[Out-barb]{y \in {\mathcal N}, \; x \nameeq y}
		  {\outputp{x}{v} \downarrow_{\mathcal N} x}
\infrule[Par-barb]{\mbox{$P\downarrow_{\mathcal N} x$ or $Q\downarrow_{\mathcal N} x$}}
		  {\binpar{P}{Q} \downarrow_{\mathcal N} x}

We write $P \Downarrow_{\mathcal N} x$ if there is $Q$ such that 
$P \wred Q$ and $Q \downarrow_{\mathcal N} x$.
\end{definition}

\begin{definition}
%\label{def.bbisim}
An  ${\mathcal N}$-\emph{barbed bisimulation} over a set of names, ${\mathcal N}$, is a symmetric binary relation 
${\mathcal S}_{\mathcal N}$ between agents such that $P\rel{S}_{\mathcal N}Q$ implies:
\begin{enumerate}
\item If $P \red P'$ then $Q \wred Q'$ and $P'\rel{S}_{\mathcal N} Q'$.
\item If $P\downarrow_{\mathcal N} x$, then $Q\Downarrow_{\mathcal N} x$.
\end{enumerate}
$P$ is ${\mathcal N}$-barbed bisimilar to $Q$, written
$P \wbbisim_{\mathcal N} Q$, if $P \rel{S}_{\mathcal N} Q$ for some ${\mathcal N}$-barbed bisimulation ${\mathcal S}_{\mathcal N}$.
\end{definition}

$\mathcal{R} \subseteq \pi \times \pi$

$P \mathcal{R} Q => \forall P'. P \red P' \Rightarrow \exists Q'. Q \red Q', P' \mathcal{R} Q'$

$P \vdash x \Rightarrow Q \vdash x$

\begin{mathpar}
  \inferrule*[lab=Out-barb]{x \nameeq y}{{y}!\langle{Q}\rangle \vdash x}
  \and
  \inferrule*[lab=Par-barb]{\mbox{$P\vdash x$ or $Q\vdash x$}}{\binpar{P}{Q} \vdash x}
\end{mathpar}

\subsubsection{Contexts}

One of the principle advantages of computational calculi like the
$\pi$-calculus is a well-defined notion of context,
contextual-equivalence and a correlation between
contextual-equivalence and notions of bisimulation. The notion of
context allows the decomposition of a process into (sub-)process and
its syntactic environment, its context. Thus, a context may be
thought of as a process with a ``hole'' (written $\Box$) in it. The
application of a context $M$ to a process $P$, written $M[P]$, is
tantamount to filling the hole in $M$ with $P$. In this paper we do
not need the full weight of this theory, but do make use of the notion
of context in the proof the main theorem. 

\begin{mathpar}
  \inferrule* [lab=summation] {} {{M_{M},M_{N}} \bc \Box \;|\; x.M_{A} \;|\; M_{M}+M_{N}}
  \and
  \inferrule* [lab=agent] {} {{M_{A}} \bc (\vec{x})M_{P} \;| \; \clift{P_0,\ldots,M_{P},\ldots,P_N}}
  \and \\
  \inferrule* [lab=process] {} {{M_{P}} \bc M_{N} \;| \;P|M_{P} }
\end{mathpar} 

\begin{mathpar}
  \inferrule* [lab=sychronization] {} {M_{N} \bc \Box \;|\; x?M_{F} \;|\; x!M_{C}}
  \and
  \inferrule* [lab=abstraction] {} {{M_{F}} \bc (x)M_{P} }
  \and
  \inferrule* [lab=concretion] {} {{M_{C}} \bc \langle M_{P} \rangle }
  \and \\
  \inferrule* [lab=process] {} {{M_{P}} \bc M_{N} \;| \;P|M_{P} }
\end{mathpar}

\begin{definition}[contextual application] Given a context $M$, and
  process $P$, we define the \emph{contextual application}, $M[P] :=
  M\{P/\Box\}$. That is, the contextual application of M to P is the
  substitution of $P$ for $\Box$ in $M$.
\end{definition}

$\meaningof{-} : L \to \mathcal{P}(\pi)$

\begin{mathpar}
  \inferrule* [lab=collection] {} {\meaningof{true} = \pi, \and \meaningof{~E} = \pi \setminus \meaningof{E}, \and \meaningof{E_{1} \& E_{2}} = \meaningof{E_{1}} \cap \meaningof{E_{2}}}
\end{mathpar}

\begin{mathpar}
  \inferrule* [lab=structure] {} {\meaningof{0} = \{ P \in \pi | P \equiv 0 \}, \and \\ \meaningof{E_1 | E_2} = \{ P \in \pi | P \equiv P_{1} | P_{2}, P_{1} \in \meaningof{E_{1}}, P_{2} \in \meaningof{E_2}\} }
\end{mathpar}

\begin{mathpar}
 \inferrule* [lab=behavior] {} {\meaningof{\langle a?b \rangle E} = \{ P \in \pi | P \equiv Q | u?(y)P', \\ \and \\\\ \and \\ \;\;\; u \in \meaningof{a}, \forall z.P'\{z/y\} \in \meaningof{E\{z/b\}}\}, \and \\ \meaningof{a!E} = \{ P \in \pi | P \equiv Q | x!\langle P' \rangle, x \in \meaningof{a} P' \in \meaningof{E}\} }
\end{mathpar}

\begin{mathpar}
 \inferrule* [lab=nominal] {} {\meaningof{\quotep{E}} = \{ \quotep{P} \in \quotep{\pi} | P \in \meaningof{E} \}, \and \meaningof{\quotep{P}} = \{ \quotep{Q} \in \quotep{\pi} | P \equiv Q \} \and \\ \meaningof{@\quotep{E}} = \{ P \in \pi | P \equiv @x, x \in \meaningof{E} \}}
\end{mathpar}

\begin{eqnarray*}
  \\
  \meaningof{-} : TS \to ST
\end{eqnarray*}

\begin{eqnarray*}
  \\
  L : TS \to ST
\end{eqnarray*}

\begin{eqnarray*}
  \\
  P \models E \iff P \in \meaningof{E}
\end{eqnarray*}

\begin{eqnarray*}
  P \approx_{L} Q \iff \forall E \in L. P \models E \iff Q \models E
\end{eqnarray*}

\begin{eqnarray*}
  P \approx_{K} Q
\end{eqnarray*}

\begin{eqnarray*}
  P \approx Q
\end{eqnarray*}

$\approx_{K} = \approx = \approx_{L}$

\subsubsection{Contextual duality}

Note that contexts extend the quotation operation to a family of
operations from processes to names. Given a context, $M$, we can
define a \emph{nominal context}, $\quotep{M}$ by $\quotep{M}[P] :=
\quotep{M[P]}$. To foreshadow what is to come we observe that these
operations enjoy a duality with processes very much like the duality
between vectors and maps from vectors to scalars.

Further, because the calculus is essentially higher-order, we have a
correspondence between contexts and processes. More specifically,
given a name $x$ and a context $M$ we can construct $M^{*}_{x}$ such
that 

\begin{mathpar}
  M^{*}_{x} | \lift{x}{P} \red M[P]
\end{mathpar}

namely,

\begin{mathpar}
  M^{*}_{x} := x?(u).M[\dropn{u}]
\end{mathpar}

The dependence of $M^{*}_{x}$ on a name makes it an abstraction, 

\begin{mathpar}
  M^{*} := (x)x?(u).M[\dropn{u}]
\end{mathpar}

\subsection{Additional notation}

It will sometimes be convenient to denote the process a name
quotes. We already have the notation $x = \quotep{P}$, but it will be
convenient to introduce an alternate notation, $\procn{x}$, when we
want to emphasize the connection to the use of the name. Note that, by
virtue of name equivalence, $\quotep{\procn{x}} \nameeq x$; so, the
notation is consistent with previous definitions.

Further, because names have structure it is possible to effect
substitutions on the basis of that structure. This means we need to
upgrade our notation for substitutions, which we accomplish by
adapting comprehension notation. Thus,

\begin{mathpar}
  P\{ y / x : x \in S \}
\end{mathpar}

is interpreted to mean the process derived from P by replacing (in a
capture-avoiding manner) each occurrence of $x$ in $S$ by $y$. For example,

\begin{mathpar}
  P\{ \quotep{\procn{x}|\procn{x}} / x : x \in \freenames{P} \}
\end{mathpar}

will replace each (occurrence) of a free name $x$ in $P$ by
$\quotep{\procn{x}|\procn{x}}$.

Also, we will avail ourselves of the notation $x^{L}$ and $x^{R}$ to
denote injections of a name into disjoint copies of the name
space. There are numerous ways to accomplish this. One example can be
found in \cite{MeredithR05}. This notation overloads to vectors of
names: $\vec{x}^{\pi} := (x_{i}^{\pi} \; : \; 0 \leq i < |\vec{x}| )$ where $\pi \in \{L,R\}$.

We also use $P^{\Box} := P|\Box$.

In \cite{MeredithR05} an interpretation of the new operator is
given. It turns out that there are several possible interpretations
all enjoying the requisite algebraic properties of the operator (see
\cite{milner91polyadicpi}). We will therefore make liberal use of
$(\nu\; \vec{x})P$.

% subsection the_syntax_and_semantics_of_the_notation_system (end)   

\input{qm2pi.qmops} 

\input{qm2pi.sterngerlach} 

\input{qm2pi.metric} 

% section concurrent_process_calculi (end)

%\input{qm2pi.proofsketch}

% section proof sketch (end)

%\input{qm2pi.slviaknots} 

% section spatial logic via knots (end)

\input{qm2pi.conclusion}

% section conclusion (end)

%\input{qm2pi.dtcodes} 

% section wiring algorithm (end)

\input{qm2pi.ack} 

% section acknowledgments (end)

\newpage


\bibliographystyle{plain}   
\bibliography{../../biblios/main.bib}

\input{qm2pi.rhodetails}

\end{document}

 

\documentclass[12pt]{llncs}
%\documentclass{jktr}

\usepackage[pdftex]{hyperref}                   
\usepackage {listings}
\usepackage {mathpartir}
\usepackage{bcprules}
%\usepackage{listings}
                       
\usepackage{graphicx} 
%\usepackage[margins=2.5cm,nohead,nofoot]{geometry}
%\usepackage{geometry}
\usepackage{amsfonts}
\usepackage{amstext}
\usepackage{latexsym}
\usepackage{amssymb}
\usepackage{color}


%\include{myPreamble}
\include{qm2pi.local} 

%\ifpdf
%\usepackage[pdftex]{graphicx}
%\else
%\usepackage{graphicx}
%\fi

 % \ifpdf
%  \usepackage{pdfsync}
%  \if


%\title{Brief Article}
%\author{David F. Snyder}
%\author{L.G. Meredith}

%\address{Dept. of Math., Texas State University--San Marcos, San Marcos, TX 78666}
       
\pagestyle{empty}


\begin{document}

\lstset{language=[Objective]Caml,frame=shadowbox}

\input{qm2pi.front}

% section front matter (end)

\input{qm2pi.intro} 
 
% section introduction (end)

% \input{qm2pi.knotations} 

% section notation (end)

\input{qm2pi.process.calculi} 

% section concurrent_process_calculi_and_spatial_logics_ (end)
    
%\input{qm2pi.knots2pi} 

%\input{qm2pi.trefoil} 

%\input{qm2pi.mainthm} 

% subsection basic_interpretation (end)

%\input{qm2pi.rho.presentation} 
\subsection{The syntax and semantics of the notation system}\label{sub:the_syntax_and_semantics_of_the_notation_system} % (fold)

We now summarize a technical presentation of the calculus that
embodies our theory of dynamics. The typical presentation of such a
calculus follows the style of giving generators and relations on
them. The grammar, below, describing term constructors, freely
generates the set of processes, $\Proc$. This set is then quotiented
by a relation known as structural congruence and it is over this set
that the notion of dynamics is expressed. This presentation is
essentially that of \cite{MeredithR05} with the addition of
polyadicity and summation. For readability we have relegated some of
the technical subtleties to an appendix.

\subsubsection{Process grammar}\label{subsub:process_grammar}

\begin{mathpar}
  \inferrule* [lab=synchronization] {} {{M} \bc \pzero \;|\; x?F \;|\; x!C }
  \and
  \inferrule* [lab=abstraction] {} {{F} \bc (x)P}
  \and
  \inferrule* [lab=concretion] {} {{C} \bc \langle Q \rangle}
  \and
  \inferrule* [lab=process] {} {{P,Q} \bc M \;| \;P|Q \;|\; @{x}}
  \and
  \inferrule* [lab=name] {} {{x} \bc \quotep{P}}
\end{mathpar} 

Note that $\vec{x}$ (resp. $\vec{P}$) denotes a vector of names
(resp. processes) of length $|\vec{x}|$ (resp. $|\vec{P}|$). We adopt
the following useful abbreviations.

\begin{mathpar}
   x?(\vec{y}).P := x.(\vec{y})P \and  x\clift{\vec{P}} := x.\clift{\vec{P}}
   \and x!(y) := \lift{x}{\dropn{y}}
   \and \Pi_{i=0}^{n-1}P_i := P_0 | \ldots | P_{n-1}
\end{mathpar}

\subsubsection{Structural congruence}

\paragraph{Free and bound names and alpha-equivalence.} At the
core of structural equivalence is alpha-equivalence which identifies
process that are the same up to a change of variable. Formally, we
recognize the distinction between free and bound names. The free names
of a process, $\freenames{P}$, may be calculated recursively as
follows:

\begin{mathpar}
\freenames{\pzero} := \emptyset
  \and \\
  \freenames{x?(y).P} := \{ x \} \cup (\freenames{P} \setminus \{ y \})
  \and 
  \freenames{x!\langle P \rangle} := \{ x \} \cup \{ P \} 
  \and \\
  \freenames{P|Q} := \freenames{P} \cup \freenames{Q}
  \and \\
  \freenames{@{x}} := \{ x \}
\end{mathpar}

$\pi$
$\quotep{\pi}$

$\freenames{-} : \pi \to \mathcal{P}(\quotep{\pi})$

\begin{eqnarray*}
  \freenames{\pzero} & := & \emptyset \\
  \freenames{x?(y).P} & := & \{ x \} \cup (\freenames{P} \setminus \{ y \}) \\
  \freenames{x!\langle P \rangle} & := & \{ x \} \cup \{ P \} \\
  \freenames{P|Q} & := & \freenames{P} \cup \freenames{Q} \\
  \freenames{\dropn{x}} & := & \{ x \}
\end{eqnarray*}

The bound names of a process, $\boundnames{P}$, are those names occurring in $P$
that are not free. For example, in $x?(y).0$, the name $x$ is free, while $y$ is bound.

\begin{mathpar}
  \inferrule* [lab=monoidal-laws] {} { P|Q \equiv Q|P \and P|0 \equiv P \and P|(Q|R) \equiv (P|Q)|R }
\end{mathpar}

\begin{mathpar}
  \inferrule* [lab=alpha-equivalence] {} { (x)P \equiv (y)P\{y/x\} \and y \not\in \freenames{P} }
\end{mathpar}

\begin{definition}
Then two processes, $P,Q$, are alpha-equivalent if $P = Q\{\vec{y}/\vec{x}\}$ for
some $\vec{x} \in \boundnames{Q},\vec{y} \in \boundnames{P}$, where $Q\{\vec{y}/\vec{x}\}$
denotes the capture-avoiding substitution of $\vec{y}$ for $\vec{x}$ in $Q$.
\end{definition}

\begin{definition}
  The {\em structural congruence} \cite{SangiorgiWalker} , $\equiv$,
  between processes is the least congruence containing
  alpha-equivalence, satisfying the abelian monoid laws
  (associativity, commutativity and $\pzero$ as identity) for parallel
  composition $|$ and for summation $+$.
\end{definition}

\subsection{Name equivalence}

We take name equivalence, written $\nameeq$, to be the smallest
equivalence relation generated by the following rules.

\begin{mathpar}
\inferrule*[lab=Quote-drop]
{ }
{ \quotep{@{x}} \nameeq x }

\inferrule*[lab=Struct-equiv]
{ P \scong Q }
{ \quotep{P} \nameeq \quotep{Q} }
\end{mathpar}

The astute reader will have noticed that the mutual recursion of names
and processes imposes a mutual recursion on alpha-equivalence and
structural equivalence via name-equivalence. Fortunately, all of this
works out pleasantly and we may calculate in the natural way, free of
concern. The reader interested in the details is referred to the
appendix \ref{appendix:rho_details}.

\subsection{Substitution}

We use $\Proc$ for the set of processes, $\QProc$ for the set of
names, and $\id{\{}\vec{y} / \vec{x} \id{\}}$ to denote partial maps,
$s : \QProc \rightarrow \QProc$. A map, $s$ lifts, uniquely, to a map
on process terms, $\widehat{s} : \Proc \rightarrow \Proc$ by the
following equations.

\begin{mathpar}
  (0) \psubstp{Q}{P} := 0 \\
  (R \juxtap S) \psubstp{Q}{P}
  :=    
  (R)\psubstp{Q}{P} \juxtap (S) \psubstp{Q}{P} \\
  (x?(y).R) \psubstp{Q}{P}    
  :=    
  (x)\substp{Q}{P} (z)\concat( (R \psubstn{z}{y}) \psubstp{Q}{P} ) \\
  (\lift{x}{R}) \psubstp{Q}{P}  
  :=
  \lift{(x)\substp{Q}{P}}{ R \psubstp{Q}{P} } \\
%   (\dropn{x})  \psubstp{Q}{P}       
%   := 
%   \left\{ 
%     \begin{array}{ccc} 
%       \dropn{\quotep{Q}} & & x \nameeq \quotep{P} \\
%       \dropn{x} & & otherwise \\
%     \end{array}
%   \right. 
  (\dropn{x})  \psubstp{Q}{P}       
  := 
  \left\{ 
    \begin{array}{ccc} 
      Q & & x \nameeq \quotep{P} \\
      \dropn{x} & & otherwise \\
    \end{array}
  \right.
\end{mathpar}
 

where

\begin{eqnarray}
  (x)\id{\{} \lpquote Q \rpquote / \lpquote P \rpquote \id{\}}            = 
  \left\{ 
    \begin{array}{ccc}
      \lpquote Q \rpquote & & x \nameeq \lpquote P \rpquote \\
      x & & otherwise \\
    \end{array}
  \right. \nonumber
\end{eqnarray}

and $z$ is chosen distinct from $\quotep{P}$, $\quotep{Q}$, the free
names in $Q$, and all the names in $R$. Our $\alpha$-equivalence will
be built in the standard way from this substitution.

\begin{remark}\label{rem:no_self_referential_names}
  One consequence of these definitions is that $\forall P. \quotep{P}
  \not\in \freenames{P}$.
\end{remark}

\subsection{ Dynamic quote: an example }

Anticipating something of what's to come, consider applying the
substitution, $\widehat{\id{\{}u / z \id{\}}}$, to the following pair
of processes, $\lift{w}{y!(z)}$ and $w[ \lpquote y!(z) \rpquote ]$.

\begin{eqnarray}
	\lift{w}{y!(z)}\widehat{\id{\{}u / z \id{\}}}
		& = &
		\lift{w}{y!(u)} \nonumber\\
	w[ \lpquote y!(z) \rpquote ] \widehat{ \id{\{}u / z \id{\}} }
		& = &
		w[ \lpquote y!(z) \rpquote ] \nonumber
\end{eqnarray}

Because the body of the process between quotes is impervious to
substitution, we get radically different answers. In fact, by
examining the first process in an input context,
e.g. $x?(z).\lift{w}{y!(z)}$, we see that the process under the lift
operator may be shaped by prefixed inputs binding a name inside it. In
this sense, the lift operator will be seen as a way to dynamically
construct processes before reifying them as names.

Finally equipped with these standard features we can present the
dynamics of the calculus.

\subsubsection{Operational semantics} 

Finally, we introduce the computational dynamics. What marks these
algebras as distinct from other more traditionally studied algebraic
structures, e.g. vector spaces or polynomial rings, is the manner in
which dynamics is captured. In traditional structures, dynamics is typically
expressed through morphisms between such structures, as in linear maps
between vector spaces or morphisms between rings. In algebras
associated with the semantics of computation, the dynamics is
expressed as part of the algebraic structure itself, through a
reduction reduction relation typically denoted by $\red$. Below, we
give a recursive presentation of this relation for the calculus used
in the encoding.

$\red \subseteq \pi \times \pi$
$\red : \pi \to \mathcal{P}(\pi)$

\begin{mathpar}
  \inferrule* [lab=Comm] { \textsf{match}( x_{src}, x_{trgt} ) } { x_{trgt}?(y)P \; | \; x_{src}!\langle {Q} \rangle \red P\{\quotep{Q}/y}\} }
  \and \\
  \inferrule* [lab=Par] {{P} \red {P}'} {{{P} | {Q}} \red {{P}' | {Q}}}
  \and
  \inferrule* [lab=Equiv]{{{P} \scong {P}'} \andalso {{P}' \red {Q}'} \andalso {{Q}' \scong {Q}}}{{P} \red {Q}}
\end{mathpar}

\begin{eqnarray*}
  match_{\equiv} (\quotep{P},\quotep{Q}) & := & P \equiv Q \\
  match_{\dagger}(\quotep{P},\quotep{Q}) & := & \forall R. P|Q \red^{*} R => R \red^{*} 0 \\
  match_{K}(\quotep{P},\quotep{Q}) & := & K \mbox{ for some context } K
\end{eqnarray*}

$u?(x)P | u!\langle Q \rangle \red P\{\quotep{Q}/x\}$

%We write $\wred$ for $\red^*$, and $P\red$ if $\exists Q $ such that $ P \red Q$.
We write $P\red$ if $\exists Q $ such that $ P \red Q$ and $P\not\red$, otherwise.

\section{Replication}

As mentioned before, it is known that replication (and hence
recursion) can be implemented in a higher-order process algebra
\cite{SangiorgiWalker}. As our first example of calculation with the
machinery thus far presented we give the construction explicitly in
the {\rhoc}.

\begin{eqnarray}
	D_{x} & := & \prefix{x}{y}{(\binpar{\outputp{x}{y}}{@{y}})} \nonumber\\
	\bangp_{x}{P} & := & \binpar{{x}!\langle{\binpar{D_{x}}{P}}\rangle}{D_{x}} \nonumber
\end{eqnarray}

\begin{eqnarray}
	\bangp_{x}{P} & & \nonumber\\
	=
	& {x}!\langle{(\prefix{x}{y}{(\outputp{x}{y} | @{y})) | P}}\rangle 
	      | \prefix{x}{y}{(\outputp{x}{y} | @{y})} & \nonumber\\
	\red
	& (\outputp{x}{y} | @{y})\substn{\quotep{(\prefix{x}{y}{(@{y} | \outputp{x}{y})) | P}}}{y} & \nonumber\\
	=
	& \outputp{x}{\quotep{(\prefix{x}{y}{(\outputp{x}{y} | @{y})) | P}}}
	  | {(\prefix{x}{y}{(\outputp{x}{y} | @{y})) | P}} & \nonumber\\
	\red
	& \ldots & \nonumber\\
	\red^*
	& P | P | \ldots & \nonumber
\end{eqnarray}

Of course, this encoding, as an implementation, runs away, unfolding
$\bangp{P}$ eagerly. A lazier and more implementable replication
operator, restricted to input-guarded processes, may be obtained as follows.

\begin{eqnarray}
\bangp{\prefix{u}{v}{P}} 
	:= 
	\binpar{\lift{x}{\prefix{u}{v}{(\binpar{D(x)}{P})}}}{D(x)} \nonumber
\end{eqnarray}

\begin{remark}
  Note that the lazier definition still does not deal with summation
  or mixed summation (i.e. sums over input and output). The reader is
  invited to construct definitions of replication that deal with these
  features. 

  Further, the definitions are parameterized in a name, $x$. Can you,
  gentle reader, make a definition that eliminates this parameter and
  guarantees no accidental interaction between the replication
  machinery and the process being replicated -- i.e. no accidental
  sharing of names used by the process to get its work done and the
  name(s) used by the replication to effect copying. This latter
  revision of the definition of replication is crucial to obtaining
  the expected identity $!!P \sim !P$.
\end{remark}

\begin{remark}\label{rem:paradoxical_combinator}
  The reader familiar with the lambda calculus will have noticed the
  similarity between $D$ and the paradoxical combinator.

  [Ed. note: the existence of this seems to suggest we have to be more
  restrictive on the set of processes and names we admit if we are to
  support no-cloning.]
\end{remark}

\subsubsection{Bisimulation}

The computational dynamics gives rise to another kind of equivalence,
the equivalence of computational behavior. As previously mentioned
this is typically captured \emph{via} some form of bisimulation.

% The notion we use in this paper is weak barbed bisimulation
% \cite{milner91polyadicpi}.

The notion we use in this paper is derived from weak barbed
bisimulation \cite{milner91polyadicpi}. 

\begin{definition}
An \emph{observation relation}, $\downarrow_{\mathcal N}$, over a set
of names, $\mathcal N$, is the smallest relation satisfying the rules
below.

\infrule[Out-barb]{y \in {\mathcal N}, \; x \nameeq y}
		  {\outputp{x}{v} \downarrow_{\mathcal N} x}
\infrule[Par-barb]{\mbox{$P\downarrow_{\mathcal N} x$ or $Q\downarrow_{\mathcal N} x$}}
		  {\binpar{P}{Q} \downarrow_{\mathcal N} x}

We write $P \Downarrow_{\mathcal N} x$ if there is $Q$ such that 
$P \wred Q$ and $Q \downarrow_{\mathcal N} x$.
\end{definition}

\begin{definition}
%\label{def.bbisim}
An  ${\mathcal N}$-\emph{barbed bisimulation} over a set of names, ${\mathcal N}$, is a symmetric binary relation 
${\mathcal S}_{\mathcal N}$ between agents such that $P\rel{S}_{\mathcal N}Q$ implies:
\begin{enumerate}
\item If $P \red P'$ then $Q \wred Q'$ and $P'\rel{S}_{\mathcal N} Q'$.
\item If $P\downarrow_{\mathcal N} x$, then $Q\Downarrow_{\mathcal N} x$.
\end{enumerate}
$P$ is ${\mathcal N}$-barbed bisimilar to $Q$, written
$P \wbbisim_{\mathcal N} Q$, if $P \rel{S}_{\mathcal N} Q$ for some ${\mathcal N}$-barbed bisimulation ${\mathcal S}_{\mathcal N}$.
\end{definition}

$\mathcal{R} \subseteq \pi \times \pi$

$P \mathcal{R} Q => \forall P'. P \red P' \Rightarrow \exists Q'. Q \red Q', P' \mathcal{R} Q'$

$P \vdash x \Rightarrow Q \vdash x$

\begin{mathpar}
  \inferrule*[lab=Out-barb]{x \nameeq y}{{y}!\langle{Q}\rangle \vdash x}
  \and
  \inferrule*[lab=Par-barb]{\mbox{$P\vdash x$ or $Q\vdash x$}}{\binpar{P}{Q} \vdash x}
\end{mathpar}

\subsubsection{Contexts}

One of the principle advantages of computational calculi like the
$\pi$-calculus is a well-defined notion of context,
contextual-equivalence and a correlation between
contextual-equivalence and notions of bisimulation. The notion of
context allows the decomposition of a process into (sub-)process and
its syntactic environment, its context. Thus, a context may be
thought of as a process with a ``hole'' (written $\Box$) in it. The
application of a context $M$ to a process $P$, written $M[P]$, is
tantamount to filling the hole in $M$ with $P$. In this paper we do
not need the full weight of this theory, but do make use of the notion
of context in the proof the main theorem. 

\begin{mathpar}
  \inferrule* [lab=summation] {} {{M_{M},M_{N}} \bc \Box \;|\; x.M_{A} \;|\; M_{M}+M_{N}}
  \and
  \inferrule* [lab=agent] {} {{M_{A}} \bc (\vec{x})M_{P} \;| \; \clift{P_0,\ldots,M_{P},\ldots,P_N}}
  \and \\
  \inferrule* [lab=process] {} {{M_{P}} \bc M_{N} \;| \;P|M_{P} }
\end{mathpar} 

\begin{mathpar}
  \inferrule* [lab=sychronization] {} {M_{N} \bc \Box \;|\; x?M_{F} \;|\; x!M_{C}}
  \and
  \inferrule* [lab=abstraction] {} {{M_{F}} \bc (x)M_{P} }
  \and
  \inferrule* [lab=concretion] {} {{M_{C}} \bc \langle M_{P} \rangle }
  \and \\
  \inferrule* [lab=process] {} {{M_{P}} \bc M_{N} \;| \;P|M_{P} }
\end{mathpar}

\begin{definition}[contextual application] Given a context $M$, and
  process $P$, we define the \emph{contextual application}, $M[P] :=
  M\{P/\Box\}$. That is, the contextual application of M to P is the
  substitution of $P$ for $\Box$ in $M$.
\end{definition}

$\meaningof{-} : L \to \mathcal{P}(\pi)$

\begin{mathpar}
  \inferrule* [lab=collection] {} {\meaningof{true} = \pi, \and \meaningof{~E} = \pi \setminus \meaningof{E}, \and \meaningof{E_{1} \& E_{2}} = \meaningof{E_{1}} \cap \meaningof{E_{2}}}
\end{mathpar}

\begin{mathpar}
  \inferrule* [lab=structure] {} {\meaningof{0} = \{ P \in \pi | P \equiv 0 \}, \and \\ \meaningof{E_1 | E_2} = \{ P \in \pi | P \equiv P_{1} | P_{2}, P_{1} \in \meaningof{E_{1}}, P_{2} \in \meaningof{E_2}\} }
\end{mathpar}

\begin{mathpar}
 \inferrule* [lab=behavior] {} {\meaningof{\langle a?b \rangle E} = \{ P \in \pi | P \equiv Q | u?(y)P', \\ \and \\\\ \and \\ \;\;\; u \in \meaningof{a}, \forall z.P'\{z/y\} \in \meaningof{E\{z/b\}}\}, \and \\ \meaningof{a!E} = \{ P \in \pi | P \equiv Q | x!\langle P' \rangle, x \in \meaningof{a} P' \in \meaningof{E}\} }
\end{mathpar}

\begin{mathpar}
 \inferrule* [lab=nominal] {} {\meaningof{\quotep{E}} = \{ \quotep{P} \in \quotep{\pi} | P \in \meaningof{E} \}, \and \meaningof{\quotep{P}} = \{ \quotep{Q} \in \quotep{\pi} | P \equiv Q \} \and \\ \meaningof{@\quotep{E}} = \{ P \in \pi | P \equiv @x, x \in \meaningof{E} \}}
\end{mathpar}

\begin{eqnarray*}
  \\
  \meaningof{-} : TS \to ST
\end{eqnarray*}

\begin{eqnarray*}
  \\
  L : TS \to ST
\end{eqnarray*}

\begin{eqnarray*}
  \\
  P \models E \iff P \in \meaningof{E}
\end{eqnarray*}

\begin{eqnarray*}
  P \approx_{L} Q \iff \forall E \in L. P \models E \iff Q \models E
\end{eqnarray*}

\begin{eqnarray*}
  P \approx_{K} Q
\end{eqnarray*}

\begin{eqnarray*}
  P \approx Q
\end{eqnarray*}

$\approx_{K} = \approx = \approx_{L}$

\subsubsection{Contextual duality}

Note that contexts extend the quotation operation to a family of
operations from processes to names. Given a context, $M$, we can
define a \emph{nominal context}, $\quotep{M}$ by $\quotep{M}[P] :=
\quotep{M[P]}$. To foreshadow what is to come we observe that these
operations enjoy a duality with processes very much like the duality
between vectors and maps from vectors to scalars.

Further, because the calculus is essentially higher-order, we have a
correspondence between contexts and processes. More specifically,
given a name $x$ and a context $M$ we can construct $M^{*}_{x}$ such
that 

\begin{mathpar}
  M^{*}_{x} | \lift{x}{P} \red M[P]
\end{mathpar}

namely,

\begin{mathpar}
  M^{*}_{x} := x?(u).M[\dropn{u}]
\end{mathpar}

The dependence of $M^{*}_{x}$ on a name makes it an abstraction, 

\begin{mathpar}
  M^{*} := (x)x?(u).M[\dropn{u}]
\end{mathpar}

\subsection{Additional notation}

It will sometimes be convenient to denote the process a name
quotes. We already have the notation $x = \quotep{P}$, but it will be
convenient to introduce an alternate notation, $\procn{x}$, when we
want to emphasize the connection to the use of the name. Note that, by
virtue of name equivalence, $\quotep{\procn{x}} \nameeq x$; so, the
notation is consistent with previous definitions.

Further, because names have structure it is possible to effect
substitutions on the basis of that structure. This means we need to
upgrade our notation for substitutions, which we accomplish by
adapting comprehension notation. Thus,

\begin{mathpar}
  P\{ y / x : x \in S \}
\end{mathpar}

is interpreted to mean the process derived from P by replacing (in a
capture-avoiding manner) each occurrence of $x$ in $S$ by $y$. For example,

\begin{mathpar}
  P\{ \quotep{\procn{x}|\procn{x}} / x : x \in \freenames{P} \}
\end{mathpar}

will replace each (occurrence) of a free name $x$ in $P$ by
$\quotep{\procn{x}|\procn{x}}$.

Also, we will avail ourselves of the notation $x^{L}$ and $x^{R}$ to
denote injections of a name into disjoint copies of the name
space. There are numerous ways to accomplish this. One example can be
found in \cite{MeredithR05}. This notation overloads to vectors of
names: $\vec{x}^{\pi} := (x_{i}^{\pi} \; : \; 0 \leq i < |\vec{x}| )$ where $\pi \in \{L,R\}$.

We also use $P^{\Box} := P|\Box$.

In \cite{MeredithR05} an interpretation of the new operator is
given. It turns out that there are several possible interpretations
all enjoying the requisite algebraic properties of the operator (see
\cite{milner91polyadicpi}). We will therefore make liberal use of
$(\nu\; \vec{x})P$.

% subsection the_syntax_and_semantics_of_the_notation_system (end)   

\input{qm2pi.qmops} 

\input{qm2pi.sterngerlach} 

\input{qm2pi.metric} 

% section concurrent_process_calculi (end)

%\input{qm2pi.proofsketch}

% section proof sketch (end)

%\input{qm2pi.slviaknots} 

% section spatial logic via knots (end)

\input{qm2pi.conclusion}

% section conclusion (end)

%\input{qm2pi.dtcodes} 

% section wiring algorithm (end)

\input{qm2pi.ack} 

% section acknowledgments (end)

\newpage


\bibliographystyle{plain}   
\bibliography{../../biblios/main.bib}

\input{qm2pi.rhodetails}

\end{document}

 

% section concurrent_process_calculi (end)

%\documentclass[12pt]{llncs}
%\documentclass{jktr}

\usepackage[pdftex]{hyperref}                   
\usepackage {listings}
\usepackage {mathpartir}
\usepackage{bcprules}
%\usepackage{listings}
                       
\usepackage{graphicx} 
%\usepackage[margins=2.5cm,nohead,nofoot]{geometry}
%\usepackage{geometry}
\usepackage{amsfonts}
\usepackage{amstext}
\usepackage{latexsym}
\usepackage{amssymb}
\usepackage{color}


%\include{myPreamble}
\include{qm2pi.local} 

%\ifpdf
%\usepackage[pdftex]{graphicx}
%\else
%\usepackage{graphicx}
%\fi

 % \ifpdf
%  \usepackage{pdfsync}
%  \if


%\title{Brief Article}
%\author{David F. Snyder}
%\author{L.G. Meredith}

%\address{Dept. of Math., Texas State University--San Marcos, San Marcos, TX 78666}
       
\pagestyle{empty}


\begin{document}

\lstset{language=[Objective]Caml,frame=shadowbox}

\input{qm2pi.front}

% section front matter (end)

\input{qm2pi.intro} 
 
% section introduction (end)

% \input{qm2pi.knotations} 

% section notation (end)

\input{qm2pi.process.calculi} 

% section concurrent_process_calculi_and_spatial_logics_ (end)
    
%\input{qm2pi.knots2pi} 

%\input{qm2pi.trefoil} 

%\input{qm2pi.mainthm} 

% subsection basic_interpretation (end)

%\input{qm2pi.rho.presentation} 
\subsection{The syntax and semantics of the notation system}\label{sub:the_syntax_and_semantics_of_the_notation_system} % (fold)

We now summarize a technical presentation of the calculus that
embodies our theory of dynamics. The typical presentation of such a
calculus follows the style of giving generators and relations on
them. The grammar, below, describing term constructors, freely
generates the set of processes, $\Proc$. This set is then quotiented
by a relation known as structural congruence and it is over this set
that the notion of dynamics is expressed. This presentation is
essentially that of \cite{MeredithR05} with the addition of
polyadicity and summation. For readability we have relegated some of
the technical subtleties to an appendix.

\subsubsection{Process grammar}\label{subsub:process_grammar}

\begin{mathpar}
  \inferrule* [lab=synchronization] {} {{M} \bc \pzero \;|\; x?F \;|\; x!C }
  \and
  \inferrule* [lab=abstraction] {} {{F} \bc (x)P}
  \and
  \inferrule* [lab=concretion] {} {{C} \bc \langle Q \rangle}
  \and
  \inferrule* [lab=process] {} {{P,Q} \bc M \;| \;P|Q \;|\; @{x}}
  \and
  \inferrule* [lab=name] {} {{x} \bc \quotep{P}}
\end{mathpar} 

Note that $\vec{x}$ (resp. $\vec{P}$) denotes a vector of names
(resp. processes) of length $|\vec{x}|$ (resp. $|\vec{P}|$). We adopt
the following useful abbreviations.

\begin{mathpar}
   x?(\vec{y}).P := x.(\vec{y})P \and  x\clift{\vec{P}} := x.\clift{\vec{P}}
   \and x!(y) := \lift{x}{\dropn{y}}
   \and \Pi_{i=0}^{n-1}P_i := P_0 | \ldots | P_{n-1}
\end{mathpar}

\subsubsection{Structural congruence}

\paragraph{Free and bound names and alpha-equivalence.} At the
core of structural equivalence is alpha-equivalence which identifies
process that are the same up to a change of variable. Formally, we
recognize the distinction between free and bound names. The free names
of a process, $\freenames{P}$, may be calculated recursively as
follows:

\begin{mathpar}
\freenames{\pzero} := \emptyset
  \and \\
  \freenames{x?(y).P} := \{ x \} \cup (\freenames{P} \setminus \{ y \})
  \and 
  \freenames{x!\langle P \rangle} := \{ x \} \cup \{ P \} 
  \and \\
  \freenames{P|Q} := \freenames{P} \cup \freenames{Q}
  \and \\
  \freenames{@{x}} := \{ x \}
\end{mathpar}

$\pi$
$\quotep{\pi}$

$\freenames{-} : \pi \to \mathcal{P}(\quotep{\pi})$

\begin{eqnarray*}
  \freenames{\pzero} & := & \emptyset \\
  \freenames{x?(y).P} & := & \{ x \} \cup (\freenames{P} \setminus \{ y \}) \\
  \freenames{x!\langle P \rangle} & := & \{ x \} \cup \{ P \} \\
  \freenames{P|Q} & := & \freenames{P} \cup \freenames{Q} \\
  \freenames{\dropn{x}} & := & \{ x \}
\end{eqnarray*}

The bound names of a process, $\boundnames{P}$, are those names occurring in $P$
that are not free. For example, in $x?(y).0$, the name $x$ is free, while $y$ is bound.

\begin{mathpar}
  \inferrule* [lab=monoidal-laws] {} { P|Q \equiv Q|P \and P|0 \equiv P \and P|(Q|R) \equiv (P|Q)|R }
\end{mathpar}

\begin{mathpar}
  \inferrule* [lab=alpha-equivalence] {} { (x)P \equiv (y)P\{y/x\} \and y \not\in \freenames{P} }
\end{mathpar}

\begin{definition}
Then two processes, $P,Q$, are alpha-equivalent if $P = Q\{\vec{y}/\vec{x}\}$ for
some $\vec{x} \in \boundnames{Q},\vec{y} \in \boundnames{P}$, where $Q\{\vec{y}/\vec{x}\}$
denotes the capture-avoiding substitution of $\vec{y}$ for $\vec{x}$ in $Q$.
\end{definition}

\begin{definition}
  The {\em structural congruence} \cite{SangiorgiWalker} , $\equiv$,
  between processes is the least congruence containing
  alpha-equivalence, satisfying the abelian monoid laws
  (associativity, commutativity and $\pzero$ as identity) for parallel
  composition $|$ and for summation $+$.
\end{definition}

\subsection{Name equivalence}

We take name equivalence, written $\nameeq$, to be the smallest
equivalence relation generated by the following rules.

\begin{mathpar}
\inferrule*[lab=Quote-drop]
{ }
{ \quotep{@{x}} \nameeq x }

\inferrule*[lab=Struct-equiv]
{ P \scong Q }
{ \quotep{P} \nameeq \quotep{Q} }
\end{mathpar}

The astute reader will have noticed that the mutual recursion of names
and processes imposes a mutual recursion on alpha-equivalence and
structural equivalence via name-equivalence. Fortunately, all of this
works out pleasantly and we may calculate in the natural way, free of
concern. The reader interested in the details is referred to the
appendix \ref{appendix:rho_details}.

\subsection{Substitution}

We use $\Proc$ for the set of processes, $\QProc$ for the set of
names, and $\id{\{}\vec{y} / \vec{x} \id{\}}$ to denote partial maps,
$s : \QProc \rightarrow \QProc$. A map, $s$ lifts, uniquely, to a map
on process terms, $\widehat{s} : \Proc \rightarrow \Proc$ by the
following equations.

\begin{mathpar}
  (0) \psubstp{Q}{P} := 0 \\
  (R \juxtap S) \psubstp{Q}{P}
  :=    
  (R)\psubstp{Q}{P} \juxtap (S) \psubstp{Q}{P} \\
  (x?(y).R) \psubstp{Q}{P}    
  :=    
  (x)\substp{Q}{P} (z)\concat( (R \psubstn{z}{y}) \psubstp{Q}{P} ) \\
  (\lift{x}{R}) \psubstp{Q}{P}  
  :=
  \lift{(x)\substp{Q}{P}}{ R \psubstp{Q}{P} } \\
%   (\dropn{x})  \psubstp{Q}{P}       
%   := 
%   \left\{ 
%     \begin{array}{ccc} 
%       \dropn{\quotep{Q}} & & x \nameeq \quotep{P} \\
%       \dropn{x} & & otherwise \\
%     \end{array}
%   \right. 
  (\dropn{x})  \psubstp{Q}{P}       
  := 
  \left\{ 
    \begin{array}{ccc} 
      Q & & x \nameeq \quotep{P} \\
      \dropn{x} & & otherwise \\
    \end{array}
  \right.
\end{mathpar}
 

where

\begin{eqnarray}
  (x)\id{\{} \lpquote Q \rpquote / \lpquote P \rpquote \id{\}}            = 
  \left\{ 
    \begin{array}{ccc}
      \lpquote Q \rpquote & & x \nameeq \lpquote P \rpquote \\
      x & & otherwise \\
    \end{array}
  \right. \nonumber
\end{eqnarray}

and $z$ is chosen distinct from $\quotep{P}$, $\quotep{Q}$, the free
names in $Q$, and all the names in $R$. Our $\alpha$-equivalence will
be built in the standard way from this substitution.

\begin{remark}\label{rem:no_self_referential_names}
  One consequence of these definitions is that $\forall P. \quotep{P}
  \not\in \freenames{P}$.
\end{remark}

\subsection{ Dynamic quote: an example }

Anticipating something of what's to come, consider applying the
substitution, $\widehat{\id{\{}u / z \id{\}}}$, to the following pair
of processes, $\lift{w}{y!(z)}$ and $w[ \lpquote y!(z) \rpquote ]$.

\begin{eqnarray}
	\lift{w}{y!(z)}\widehat{\id{\{}u / z \id{\}}}
		& = &
		\lift{w}{y!(u)} \nonumber\\
	w[ \lpquote y!(z) \rpquote ] \widehat{ \id{\{}u / z \id{\}} }
		& = &
		w[ \lpquote y!(z) \rpquote ] \nonumber
\end{eqnarray}

Because the body of the process between quotes is impervious to
substitution, we get radically different answers. In fact, by
examining the first process in an input context,
e.g. $x?(z).\lift{w}{y!(z)}$, we see that the process under the lift
operator may be shaped by prefixed inputs binding a name inside it. In
this sense, the lift operator will be seen as a way to dynamically
construct processes before reifying them as names.

Finally equipped with these standard features we can present the
dynamics of the calculus.

\subsubsection{Operational semantics} 

Finally, we introduce the computational dynamics. What marks these
algebras as distinct from other more traditionally studied algebraic
structures, e.g. vector spaces or polynomial rings, is the manner in
which dynamics is captured. In traditional structures, dynamics is typically
expressed through morphisms between such structures, as in linear maps
between vector spaces or morphisms between rings. In algebras
associated with the semantics of computation, the dynamics is
expressed as part of the algebraic structure itself, through a
reduction reduction relation typically denoted by $\red$. Below, we
give a recursive presentation of this relation for the calculus used
in the encoding.

$\red \subseteq \pi \times \pi$
$\red : \pi \to \mathcal{P}(\pi)$

\begin{mathpar}
  \inferrule* [lab=Comm] { \textsf{match}( x_{src}, x_{trgt} ) } { x_{trgt}?(y)P \; | \; x_{src}!\langle {Q} \rangle \red P\{\quotep{Q}/y}\} }
  \and \\
  \inferrule* [lab=Par] {{P} \red {P}'} {{{P} | {Q}} \red {{P}' | {Q}}}
  \and
  \inferrule* [lab=Equiv]{{{P} \scong {P}'} \andalso {{P}' \red {Q}'} \andalso {{Q}' \scong {Q}}}{{P} \red {Q}}
\end{mathpar}

\begin{eqnarray*}
  match_{\equiv} (\quotep{P},\quotep{Q}) & := & P \equiv Q \\
  match_{\dagger}(\quotep{P},\quotep{Q}) & := & \forall R. P|Q \red^{*} R => R \red^{*} 0 \\
  match_{K}(\quotep{P},\quotep{Q}) & := & K \mbox{ for some context } K
\end{eqnarray*}

$u?(x)P | u!\langle Q \rangle \red P\{\quotep{Q}/x\}$

%We write $\wred$ for $\red^*$, and $P\red$ if $\exists Q $ such that $ P \red Q$.
We write $P\red$ if $\exists Q $ such that $ P \red Q$ and $P\not\red$, otherwise.

\section{Replication}

As mentioned before, it is known that replication (and hence
recursion) can be implemented in a higher-order process algebra
\cite{SangiorgiWalker}. As our first example of calculation with the
machinery thus far presented we give the construction explicitly in
the {\rhoc}.

\begin{eqnarray}
	D_{x} & := & \prefix{x}{y}{(\binpar{\outputp{x}{y}}{@{y}})} \nonumber\\
	\bangp_{x}{P} & := & \binpar{{x}!\langle{\binpar{D_{x}}{P}}\rangle}{D_{x}} \nonumber
\end{eqnarray}

\begin{eqnarray}
	\bangp_{x}{P} & & \nonumber\\
	=
	& {x}!\langle{(\prefix{x}{y}{(\outputp{x}{y} | @{y})) | P}}\rangle 
	      | \prefix{x}{y}{(\outputp{x}{y} | @{y})} & \nonumber\\
	\red
	& (\outputp{x}{y} | @{y})\substn{\quotep{(\prefix{x}{y}{(@{y} | \outputp{x}{y})) | P}}}{y} & \nonumber\\
	=
	& \outputp{x}{\quotep{(\prefix{x}{y}{(\outputp{x}{y} | @{y})) | P}}}
	  | {(\prefix{x}{y}{(\outputp{x}{y} | @{y})) | P}} & \nonumber\\
	\red
	& \ldots & \nonumber\\
	\red^*
	& P | P | \ldots & \nonumber
\end{eqnarray}

Of course, this encoding, as an implementation, runs away, unfolding
$\bangp{P}$ eagerly. A lazier and more implementable replication
operator, restricted to input-guarded processes, may be obtained as follows.

\begin{eqnarray}
\bangp{\prefix{u}{v}{P}} 
	:= 
	\binpar{\lift{x}{\prefix{u}{v}{(\binpar{D(x)}{P})}}}{D(x)} \nonumber
\end{eqnarray}

\begin{remark}
  Note that the lazier definition still does not deal with summation
  or mixed summation (i.e. sums over input and output). The reader is
  invited to construct definitions of replication that deal with these
  features. 

  Further, the definitions are parameterized in a name, $x$. Can you,
  gentle reader, make a definition that eliminates this parameter and
  guarantees no accidental interaction between the replication
  machinery and the process being replicated -- i.e. no accidental
  sharing of names used by the process to get its work done and the
  name(s) used by the replication to effect copying. This latter
  revision of the definition of replication is crucial to obtaining
  the expected identity $!!P \sim !P$.
\end{remark}

\begin{remark}\label{rem:paradoxical_combinator}
  The reader familiar with the lambda calculus will have noticed the
  similarity between $D$ and the paradoxical combinator.

  [Ed. note: the existence of this seems to suggest we have to be more
  restrictive on the set of processes and names we admit if we are to
  support no-cloning.]
\end{remark}

\subsubsection{Bisimulation}

The computational dynamics gives rise to another kind of equivalence,
the equivalence of computational behavior. As previously mentioned
this is typically captured \emph{via} some form of bisimulation.

% The notion we use in this paper is weak barbed bisimulation
% \cite{milner91polyadicpi}.

The notion we use in this paper is derived from weak barbed
bisimulation \cite{milner91polyadicpi}. 

\begin{definition}
An \emph{observation relation}, $\downarrow_{\mathcal N}$, over a set
of names, $\mathcal N$, is the smallest relation satisfying the rules
below.

\infrule[Out-barb]{y \in {\mathcal N}, \; x \nameeq y}
		  {\outputp{x}{v} \downarrow_{\mathcal N} x}
\infrule[Par-barb]{\mbox{$P\downarrow_{\mathcal N} x$ or $Q\downarrow_{\mathcal N} x$}}
		  {\binpar{P}{Q} \downarrow_{\mathcal N} x}

We write $P \Downarrow_{\mathcal N} x$ if there is $Q$ such that 
$P \wred Q$ and $Q \downarrow_{\mathcal N} x$.
\end{definition}

\begin{definition}
%\label{def.bbisim}
An  ${\mathcal N}$-\emph{barbed bisimulation} over a set of names, ${\mathcal N}$, is a symmetric binary relation 
${\mathcal S}_{\mathcal N}$ between agents such that $P\rel{S}_{\mathcal N}Q$ implies:
\begin{enumerate}
\item If $P \red P'$ then $Q \wred Q'$ and $P'\rel{S}_{\mathcal N} Q'$.
\item If $P\downarrow_{\mathcal N} x$, then $Q\Downarrow_{\mathcal N} x$.
\end{enumerate}
$P$ is ${\mathcal N}$-barbed bisimilar to $Q$, written
$P \wbbisim_{\mathcal N} Q$, if $P \rel{S}_{\mathcal N} Q$ for some ${\mathcal N}$-barbed bisimulation ${\mathcal S}_{\mathcal N}$.
\end{definition}

$\mathcal{R} \subseteq \pi \times \pi$

$P \mathcal{R} Q => \forall P'. P \red P' \Rightarrow \exists Q'. Q \red Q', P' \mathcal{R} Q'$

$P \vdash x \Rightarrow Q \vdash x$

\begin{mathpar}
  \inferrule*[lab=Out-barb]{x \nameeq y}{{y}!\langle{Q}\rangle \vdash x}
  \and
  \inferrule*[lab=Par-barb]{\mbox{$P\vdash x$ or $Q\vdash x$}}{\binpar{P}{Q} \vdash x}
\end{mathpar}

\subsubsection{Contexts}

One of the principle advantages of computational calculi like the
$\pi$-calculus is a well-defined notion of context,
contextual-equivalence and a correlation between
contextual-equivalence and notions of bisimulation. The notion of
context allows the decomposition of a process into (sub-)process and
its syntactic environment, its context. Thus, a context may be
thought of as a process with a ``hole'' (written $\Box$) in it. The
application of a context $M$ to a process $P$, written $M[P]$, is
tantamount to filling the hole in $M$ with $P$. In this paper we do
not need the full weight of this theory, but do make use of the notion
of context in the proof the main theorem. 

\begin{mathpar}
  \inferrule* [lab=summation] {} {{M_{M},M_{N}} \bc \Box \;|\; x.M_{A} \;|\; M_{M}+M_{N}}
  \and
  \inferrule* [lab=agent] {} {{M_{A}} \bc (\vec{x})M_{P} \;| \; \clift{P_0,\ldots,M_{P},\ldots,P_N}}
  \and \\
  \inferrule* [lab=process] {} {{M_{P}} \bc M_{N} \;| \;P|M_{P} }
\end{mathpar} 

\begin{mathpar}
  \inferrule* [lab=sychronization] {} {M_{N} \bc \Box \;|\; x?M_{F} \;|\; x!M_{C}}
  \and
  \inferrule* [lab=abstraction] {} {{M_{F}} \bc (x)M_{P} }
  \and
  \inferrule* [lab=concretion] {} {{M_{C}} \bc \langle M_{P} \rangle }
  \and \\
  \inferrule* [lab=process] {} {{M_{P}} \bc M_{N} \;| \;P|M_{P} }
\end{mathpar}

\begin{definition}[contextual application] Given a context $M$, and
  process $P$, we define the \emph{contextual application}, $M[P] :=
  M\{P/\Box\}$. That is, the contextual application of M to P is the
  substitution of $P$ for $\Box$ in $M$.
\end{definition}

$\meaningof{-} : L \to \mathcal{P}(\pi)$

\begin{mathpar}
  \inferrule* [lab=collection] {} {\meaningof{true} = \pi, \and \meaningof{~E} = \pi \setminus \meaningof{E}, \and \meaningof{E_{1} \& E_{2}} = \meaningof{E_{1}} \cap \meaningof{E_{2}}}
\end{mathpar}

\begin{mathpar}
  \inferrule* [lab=structure] {} {\meaningof{0} = \{ P \in \pi | P \equiv 0 \}, \and \\ \meaningof{E_1 | E_2} = \{ P \in \pi | P \equiv P_{1} | P_{2}, P_{1} \in \meaningof{E_{1}}, P_{2} \in \meaningof{E_2}\} }
\end{mathpar}

\begin{mathpar}
 \inferrule* [lab=behavior] {} {\meaningof{\langle a?b \rangle E} = \{ P \in \pi | P \equiv Q | u?(y)P', \\ \and \\\\ \and \\ \;\;\; u \in \meaningof{a}, \forall z.P'\{z/y\} \in \meaningof{E\{z/b\}}\}, \and \\ \meaningof{a!E} = \{ P \in \pi | P \equiv Q | x!\langle P' \rangle, x \in \meaningof{a} P' \in \meaningof{E}\} }
\end{mathpar}

\begin{mathpar}
 \inferrule* [lab=nominal] {} {\meaningof{\quotep{E}} = \{ \quotep{P} \in \quotep{\pi} | P \in \meaningof{E} \}, \and \meaningof{\quotep{P}} = \{ \quotep{Q} \in \quotep{\pi} | P \equiv Q \} \and \\ \meaningof{@\quotep{E}} = \{ P \in \pi | P \equiv @x, x \in \meaningof{E} \}}
\end{mathpar}

\begin{eqnarray*}
  \\
  \meaningof{-} : TS \to ST
\end{eqnarray*}

\begin{eqnarray*}
  \\
  L : TS \to ST
\end{eqnarray*}

\begin{eqnarray*}
  \\
  P \models E \iff P \in \meaningof{E}
\end{eqnarray*}

\begin{eqnarray*}
  P \approx_{L} Q \iff \forall E \in L. P \models E \iff Q \models E
\end{eqnarray*}

\begin{eqnarray*}
  P \approx_{K} Q
\end{eqnarray*}

\begin{eqnarray*}
  P \approx Q
\end{eqnarray*}

$\approx_{K} = \approx = \approx_{L}$

\subsubsection{Contextual duality}

Note that contexts extend the quotation operation to a family of
operations from processes to names. Given a context, $M$, we can
define a \emph{nominal context}, $\quotep{M}$ by $\quotep{M}[P] :=
\quotep{M[P]}$. To foreshadow what is to come we observe that these
operations enjoy a duality with processes very much like the duality
between vectors and maps from vectors to scalars.

Further, because the calculus is essentially higher-order, we have a
correspondence between contexts and processes. More specifically,
given a name $x$ and a context $M$ we can construct $M^{*}_{x}$ such
that 

\begin{mathpar}
  M^{*}_{x} | \lift{x}{P} \red M[P]
\end{mathpar}

namely,

\begin{mathpar}
  M^{*}_{x} := x?(u).M[\dropn{u}]
\end{mathpar}

The dependence of $M^{*}_{x}$ on a name makes it an abstraction, 

\begin{mathpar}
  M^{*} := (x)x?(u).M[\dropn{u}]
\end{mathpar}

\subsection{Additional notation}

It will sometimes be convenient to denote the process a name
quotes. We already have the notation $x = \quotep{P}$, but it will be
convenient to introduce an alternate notation, $\procn{x}$, when we
want to emphasize the connection to the use of the name. Note that, by
virtue of name equivalence, $\quotep{\procn{x}} \nameeq x$; so, the
notation is consistent with previous definitions.

Further, because names have structure it is possible to effect
substitutions on the basis of that structure. This means we need to
upgrade our notation for substitutions, which we accomplish by
adapting comprehension notation. Thus,

\begin{mathpar}
  P\{ y / x : x \in S \}
\end{mathpar}

is interpreted to mean the process derived from P by replacing (in a
capture-avoiding manner) each occurrence of $x$ in $S$ by $y$. For example,

\begin{mathpar}
  P\{ \quotep{\procn{x}|\procn{x}} / x : x \in \freenames{P} \}
\end{mathpar}

will replace each (occurrence) of a free name $x$ in $P$ by
$\quotep{\procn{x}|\procn{x}}$.

Also, we will avail ourselves of the notation $x^{L}$ and $x^{R}$ to
denote injections of a name into disjoint copies of the name
space. There are numerous ways to accomplish this. One example can be
found in \cite{MeredithR05}. This notation overloads to vectors of
names: $\vec{x}^{\pi} := (x_{i}^{\pi} \; : \; 0 \leq i < |\vec{x}| )$ where $\pi \in \{L,R\}$.

We also use $P^{\Box} := P|\Box$.

In \cite{MeredithR05} an interpretation of the new operator is
given. It turns out that there are several possible interpretations
all enjoying the requisite algebraic properties of the operator (see
\cite{milner91polyadicpi}). We will therefore make liberal use of
$(\nu\; \vec{x})P$.

% subsection the_syntax_and_semantics_of_the_notation_system (end)   

\input{qm2pi.qmops} 

\input{qm2pi.sterngerlach} 

\input{qm2pi.metric} 

% section concurrent_process_calculi (end)

%\input{qm2pi.proofsketch}

% section proof sketch (end)

%\input{qm2pi.slviaknots} 

% section spatial logic via knots (end)

\input{qm2pi.conclusion}

% section conclusion (end)

%\input{qm2pi.dtcodes} 

% section wiring algorithm (end)

\input{qm2pi.ack} 

% section acknowledgments (end)

\newpage


\bibliographystyle{plain}   
\bibliography{../../biblios/main.bib}

\input{qm2pi.rhodetails}

\end{document}



% section proof sketch (end)

%\section{Unlikely characters: spatial logic for
  knots}\label{sub:characteristic_formulae} % (fold)

Associated to the mobile process calculi are a family of logics known
as the Hennessy-Milner logics. These logics typically enjoy a
semantics interpreting formulae as sets of processes that when
factored through the encoding outlined above allows an identification
of classes of knots with logical formulae. In the context of this
encoding the sub-family known as the spatial logics \cite{CairesC03}
\cite{CairesC04} \cite{Caires04} are of particular interest providing
several important features for expressing and reasoning about
properties (i.e. classes) of knots. We hint here at how this may be done.

%\begin{description}
%\item [structural connectives] 
\subsubsection{Structural connectives} The spatial logics enjoy
structural connectives corresponding, at the logical level, to the
parallel composition ($P | Q$) and new name ($(\nu \; x)P$)
connectives for processes. As illustrated in the examples below, these
connectives are extremely expressive given the shape of our encoding.
%\item [decideable satisfaction]

\subsubsection{Decideable satisfaction}
In \cite{Caires04} the satisfaction relation is shown to be decideable
for a rich class of processes. It further turns out that the image of
the our encoding is a proper subset of that class. This result
provides the basis for an algorithm by which to search for knots
enjoying a given property.
%\item [characteristic formulae]

\subsubsection{Characteristic formulae}
In the same paper \cite{Caires04} , Caires presents a means of calculating
characteristic formulae, selecting equivalence classes of processes
up to a pre--specified depth limit on the support set of names. Composed with our
encoding, this characteristic formula can be used to select
characteristic formulae for knots.
%\end{description}

\subsubsection{Spatial logic formulae}

The grammar below (segmented for comprehension) summarizes the syntax
of spatial logic formulae. We employ illustrative examples in the
sequel to provide an intuitive understanding of their meaning
referring the reader to \cite{Caires04} for a more detailed explication
of the semantics.

\begin{mathpar}
  \inferrule* [lab=boolean] {} {{A,B} \bc T \;|\; \neg A \;|\; A \wedge B \;|\; \eta = \eta'}
  \and
  \inferrule* [lab=spatial] {} {|\; \pzero \;|\; A | B \;|\; x \text{\textregistered} A \;|\; \forall x . A \;|\;  H x . A}
  \and
  \inferrule* [lab=behavioral] {} {|\; \alpha . A}
  \and 
  \inferrule* [lab=recursion] {} {|\; X(\vec{u}) \;|\; \mu X(\vec{u}) . A}
  \and
  \inferrule* [lab=action] {} {\alpha \bc \langle x?(\vec{y}) \rangle \;|\; \langle x!(\vec{y}) \rangle \;|\; \langle \tau \rangle}
  \and 
  \inferrule* [lab=name] {} {\eta \bc x \;|\; \tau}
\end{mathpar} 

% subsection characteristic_formulae (end)   	 

\subsection{Example formulae}\label{sub:example_formulae_} % (fold)

\subsubsection{Crossing as formula.}
% 
% \begin{align*}
%   \frac{d}{dx} \sin x &= \cos x 
%   & \frac{d}{dx} e^x &= e^x \\
%   \frac{d}{dx} \cos x &= - \sin x 
%   & \frac{d}{dx} \log x &= \frac{1}{x} \\
% \end{align*} 

\begin{align*}
 \mu C(x_{0},x_{1},y_{0},y_{1},u).&(\langle x_{0}?(z) \rangle(\langle u! \rangle\langle y_{1}!z \rangle C(x_{0},x_{1},y_{0},y_{1},u)) & \\
  & \wedge \langle y_{1}?(z) \rangle (\langle u! \rangle \langle x_{0}!z \rangle C(x_{0},x_{1},y_{0},y_{1},u)) & \\
  & \wedge \langle x_{1}?(z) \rangle (\langle u? \rangle \langle y_{0}!z \rangle C(x_{0},x_{1},y_{0},y_{1},u)) & \\
  & \wedge \langle y_{0}?(z) \rangle (\langle u? \rangle \langle x_{1}!z \rangle C(x_{0},x_{1},y_{0},y_{1},u))) &
\end{align*}

The lexicographical similarity between the shape of this formulae and
the shape of definition of the process representing a crossing reveals
the intuitive meaning of this formulae. It describes the capabilities
of a process that has the right to represent a crossing. For example
it picks out processes that may perform an input on the port $x_0$ in
its initial menu of capabilities. What differentiates the formula
from the process, however, is that the crossing process is the
smallest candidate to satisfy the formula. Infinitely many other
processes -- with internal behavior hidden behind this interface, so
to speak -- also satisfy this formula. Even this simple formula,
then, can be seen to open a new view onto knots, providing a
computational interpretation of \emph{virtual} knots.

Note that this formula is derived by hand. A similar formula can be
derived by employing Caires' calculation of characteristic formula
\cite{Caires04} to the process representing a crossing. In light of
this discussion, we let
$\meaningof{C}_{\phi}(x0,x1,y0,y1,u)$ denote a formula specifying the
dynamics we wish to capture of a crossing. To guarantee we preserve
the shape of the interface and minimal semantics we demand that
$\meaningof{C}_{\phi}(x0,x1,y0,y1,u) \Rightarrow
\textbf{C}(x0,x1,y0,y1,u)$ where $\textbf{C}(x0,x1,y0,y1,u)$ denotes
the formula above.
                            
\subsubsection{Crossing number constraints.}
The moral content of the context lemma (Lemma \ref{context}) is that the notion of
``locality'' in the Reidemeister moves is effectively captured by the
parallel composition operator of the process calculus. This intuition
extends through the logic. Given a formula,
$\meaningof{C}_{\phi}(x0,x1,y0,y1,u)$, we can use the structural
connectives to specify constraints on crossing numbers, such as at
least $n$ crossings, or exactly $n$ crossings.
\begin{mathpar}
  \inferrule* [lab=at-least-n] {} { K^{\geq n}_{\phi}(\vec{xs},\vec{ys}) := \Pi_{i=0}^{n-1} Hu . \meaningof{C}_{\phi}(xs_i,ys_i,u) | T }
  \and 
  \inferrule* [lab=exactly-n] {} { K^{= n}_{\phi}(\vec{xs},\vec{ys}) := \Pi_{i=0}^{n-1} Hu . \meaningof{C}_{\phi}(xs_i,ys_i,u) | \neg (\forall x_0,y_0,x_1,y_1,u . \meaningof{C}_{\phi}(x_0,y_0,x_1,y_1,u) | T) }
\end{mathpar}

To round out this section, recall that the encoding of an $n$-crossing
knot decomposes into a parallel composition of $n$ \emph{copies} of a
crossing process together with a wiring harness. To specify different
knot classes with the same crossing number amounts to specifying
logical constraints on the wiring harness. In the interest of space,
we defer examples to a forthcoming paper. Suffice it to say that both
the conditions ``alternating knot'' and ``contains the tangle
corresponding to 5/3'' are expressible. For example, it is possible to
calculate the characteristic formula of a process corresponding to the
tangle 5/3 and conjoin it into the classifying formula via the
composition connective of the logic.

Finally, we wish to observe that it is entirely within reason to
contemplate a more domain-specific version of spatial logic tailored
to the shape of processes in the image of the encoding. Such a
domain-specific logic would have a better claim to the title formal
language of knot properties.

% subsection example_formulae_ (end)

% section knots_as_processes (end) 

% section spatial logic via knots (end)

\section{Conclusions and future work}

\paragraph{Testing physical space}
You, gentle reader, may wonder why of all the theorems to be proved
given this set up we pick the one above. In some sense it's hardly
central to quantum mechanics. We see it as central in the sense that
it firmly establishes a notion of physical space arising from a notion
of the equivalence of behavior. Relating bisimulation to a metric is a
big step forward, but one is faced with interpreting the relationship
of that metric space to something more physical. Quantum mechanical
notions of ``physical'' space are still far from intuitive, but by
relating this idea of distance as testing to calculations that predict
physical circumstances we are making a not insignificant step forward
toward an understanding of the physical space we inhabit as
essentially dynamic.

\paragraph{Effectivity and simulation}
One of the observations we have yet to make is that the entire program
spelled out here is effective. We have built various interpreters for
the reflective calculus at work in this interpretation. In principle,
then, we can simulate quantum mechanics on a computer. The place where
the simulation may lose fidelity is the infinitely branching summation
for the annihilator.

In this connection i also want to point out that the evaluation style
calculation of the inner product puts the non-determinism of the
summation right at the heart of measurement. This suggests that
Milner's original reduction-based formulation of the dynamics of his
calculi in terms of sums was not just notationally suggestive of a
notion of measure-and-continue but captured some significant part of
the physics.

\paragraph{Quantum continuations}
In light of this last observation i want to point out that the
predominant account of quantum mechanics is missing a key aspect of a
truly compositional story of the physical situation. In a real lab,
when a measurement is made the observation can be made to feed into
another device that then makes another measurement conditioned on the
results of the first. This means that after the superposition was
collapsed the entire experimental set up remained in
superposition. While QM offers a means of writing this down it doesn't
quite line up well with the well-trodden formulation of computation
and continuation that we see so succinctly expressed in Milner's
calculi. This suggests that there might be advantages to this account
of dynamics waiting to be explored.

\paragraph{Quantum logic}
In this connection, we also note that by virtue of having the
Hennessy-Milner construction, we can pull the construction through the
interpretation of QM. This gives us a natural candidate for a quantum
logic that enjoys an extremely tight connection with it's domain of
interpretation, making the construction much less ad hoc (rather it is
the image of functor!).

\paragraph{Quantum probabiity}
i have questions about the basis of the interpretation of inner
product as probability amplitude. In particular, using which
axiomatization of probability theory does the notion of probability
amplitude earn the right to be so dubbed? In other words, where is the
proof that the operation for calculating a probability amplitude (and
then squaring) satisfies the axioms of what it means to calculate a
probability? Even if such a proof exists (i have yet to find it in the
literature), i wonder if it might not be possible to turn things on
their heads. Can we view the calculation of the probability amplitude
as an axiomatization of probability? If so, then the definition we
give for calculating probability amplitude may provide the basis for
an \emph{effective} theory of probability.

\paragraph{Quantum vs ``biological'' information}
Finally, i want to conclude with a more philosophical observation. At
a recent workshop in which QM was a predominant topic i noticed
something about quantum information. The speaker was giving a riveting
discussion of axiomatic QM and showing how properties of ``no
cloning'' and ``no deleting'' emerged as consequences of the
axiomatization. Theorems of this form are necessary to give us a sense
of confidence that our axioms characterize the physical theory. What
struck me, though, was that if quantum information is neither erasable
nor replicable it is markedly different from \emph{life}. Two of the
things we know about life is that

\begin{itemize}
  \item it ends;
  \item to gain some measure of persistence, to transcend it's
    finitude it is imminently copyable.
\end{itemize}

Both of these qualities are summarized succinctly in the aphorism: all
flesh is grass. For me these two kinds of ``information'' -- call them
quantum and biological -- are end points on a spectrum of strategies
for persistence. At one end, we have those curious entities that enjoy
uniqueness and permanence; at the other, we have those who in the face
of a certain end and an uncertain present make a go of passing
something on. To me one of the more remarkable aspects of the latter
strategy is that in the presence of noise (and certain features of
copying) we get a kind of dynamism, a chance for improvement against a
given persistent condition.

% subsection other_calculi_other_bisimulations_and_geometry_as_behavior (end)




% section conclusion (end)

%\documentclass[12pt]{llncs}
%\documentclass{jktr}

\usepackage[pdftex]{hyperref}                   
\usepackage {listings}
\usepackage {mathpartir}
\usepackage{bcprules}
%\usepackage{listings}
                       
\usepackage{graphicx} 
%\usepackage[margins=2.5cm,nohead,nofoot]{geometry}
%\usepackage{geometry}
\usepackage{amsfonts}
\usepackage{amstext}
\usepackage{latexsym}
\usepackage{amssymb}
\usepackage{color}


%\include{myPreamble}
\include{qm2pi.local} 

%\ifpdf
%\usepackage[pdftex]{graphicx}
%\else
%\usepackage{graphicx}
%\fi

 % \ifpdf
%  \usepackage{pdfsync}
%  \if


%\title{Brief Article}
%\author{David F. Snyder}
%\author{L.G. Meredith}

%\address{Dept. of Math., Texas State University--San Marcos, San Marcos, TX 78666}
       
\pagestyle{empty}


\begin{document}

\lstset{language=[Objective]Caml,frame=shadowbox}

\input{qm2pi.front}

% section front matter (end)

\input{qm2pi.intro} 
 
% section introduction (end)

% \input{qm2pi.knotations} 

% section notation (end)

\input{qm2pi.process.calculi} 

% section concurrent_process_calculi_and_spatial_logics_ (end)
    
%\input{qm2pi.knots2pi} 

%\input{qm2pi.trefoil} 

%\input{qm2pi.mainthm} 

% subsection basic_interpretation (end)

%\input{qm2pi.rho.presentation} 
\subsection{The syntax and semantics of the notation system}\label{sub:the_syntax_and_semantics_of_the_notation_system} % (fold)

We now summarize a technical presentation of the calculus that
embodies our theory of dynamics. The typical presentation of such a
calculus follows the style of giving generators and relations on
them. The grammar, below, describing term constructors, freely
generates the set of processes, $\Proc$. This set is then quotiented
by a relation known as structural congruence and it is over this set
that the notion of dynamics is expressed. This presentation is
essentially that of \cite{MeredithR05} with the addition of
polyadicity and summation. For readability we have relegated some of
the technical subtleties to an appendix.

\subsubsection{Process grammar}\label{subsub:process_grammar}

\begin{mathpar}
  \inferrule* [lab=synchronization] {} {{M} \bc \pzero \;|\; x?F \;|\; x!C }
  \and
  \inferrule* [lab=abstraction] {} {{F} \bc (x)P}
  \and
  \inferrule* [lab=concretion] {} {{C} \bc \langle Q \rangle}
  \and
  \inferrule* [lab=process] {} {{P,Q} \bc M \;| \;P|Q \;|\; @{x}}
  \and
  \inferrule* [lab=name] {} {{x} \bc \quotep{P}}
\end{mathpar} 

Note that $\vec{x}$ (resp. $\vec{P}$) denotes a vector of names
(resp. processes) of length $|\vec{x}|$ (resp. $|\vec{P}|$). We adopt
the following useful abbreviations.

\begin{mathpar}
   x?(\vec{y}).P := x.(\vec{y})P \and  x\clift{\vec{P}} := x.\clift{\vec{P}}
   \and x!(y) := \lift{x}{\dropn{y}}
   \and \Pi_{i=0}^{n-1}P_i := P_0 | \ldots | P_{n-1}
\end{mathpar}

\subsubsection{Structural congruence}

\paragraph{Free and bound names and alpha-equivalence.} At the
core of structural equivalence is alpha-equivalence which identifies
process that are the same up to a change of variable. Formally, we
recognize the distinction between free and bound names. The free names
of a process, $\freenames{P}$, may be calculated recursively as
follows:

\begin{mathpar}
\freenames{\pzero} := \emptyset
  \and \\
  \freenames{x?(y).P} := \{ x \} \cup (\freenames{P} \setminus \{ y \})
  \and 
  \freenames{x!\langle P \rangle} := \{ x \} \cup \{ P \} 
  \and \\
  \freenames{P|Q} := \freenames{P} \cup \freenames{Q}
  \and \\
  \freenames{@{x}} := \{ x \}
\end{mathpar}

$\pi$
$\quotep{\pi}$

$\freenames{-} : \pi \to \mathcal{P}(\quotep{\pi})$

\begin{eqnarray*}
  \freenames{\pzero} & := & \emptyset \\
  \freenames{x?(y).P} & := & \{ x \} \cup (\freenames{P} \setminus \{ y \}) \\
  \freenames{x!\langle P \rangle} & := & \{ x \} \cup \{ P \} \\
  \freenames{P|Q} & := & \freenames{P} \cup \freenames{Q} \\
  \freenames{\dropn{x}} & := & \{ x \}
\end{eqnarray*}

The bound names of a process, $\boundnames{P}$, are those names occurring in $P$
that are not free. For example, in $x?(y).0$, the name $x$ is free, while $y$ is bound.

\begin{mathpar}
  \inferrule* [lab=monoidal-laws] {} { P|Q \equiv Q|P \and P|0 \equiv P \and P|(Q|R) \equiv (P|Q)|R }
\end{mathpar}

\begin{mathpar}
  \inferrule* [lab=alpha-equivalence] {} { (x)P \equiv (y)P\{y/x\} \and y \not\in \freenames{P} }
\end{mathpar}

\begin{definition}
Then two processes, $P,Q$, are alpha-equivalent if $P = Q\{\vec{y}/\vec{x}\}$ for
some $\vec{x} \in \boundnames{Q},\vec{y} \in \boundnames{P}$, where $Q\{\vec{y}/\vec{x}\}$
denotes the capture-avoiding substitution of $\vec{y}$ for $\vec{x}$ in $Q$.
\end{definition}

\begin{definition}
  The {\em structural congruence} \cite{SangiorgiWalker} , $\equiv$,
  between processes is the least congruence containing
  alpha-equivalence, satisfying the abelian monoid laws
  (associativity, commutativity and $\pzero$ as identity) for parallel
  composition $|$ and for summation $+$.
\end{definition}

\subsection{Name equivalence}

We take name equivalence, written $\nameeq$, to be the smallest
equivalence relation generated by the following rules.

\begin{mathpar}
\inferrule*[lab=Quote-drop]
{ }
{ \quotep{@{x}} \nameeq x }

\inferrule*[lab=Struct-equiv]
{ P \scong Q }
{ \quotep{P} \nameeq \quotep{Q} }
\end{mathpar}

The astute reader will have noticed that the mutual recursion of names
and processes imposes a mutual recursion on alpha-equivalence and
structural equivalence via name-equivalence. Fortunately, all of this
works out pleasantly and we may calculate in the natural way, free of
concern. The reader interested in the details is referred to the
appendix \ref{appendix:rho_details}.

\subsection{Substitution}

We use $\Proc$ for the set of processes, $\QProc$ for the set of
names, and $\id{\{}\vec{y} / \vec{x} \id{\}}$ to denote partial maps,
$s : \QProc \rightarrow \QProc$. A map, $s$ lifts, uniquely, to a map
on process terms, $\widehat{s} : \Proc \rightarrow \Proc$ by the
following equations.

\begin{mathpar}
  (0) \psubstp{Q}{P} := 0 \\
  (R \juxtap S) \psubstp{Q}{P}
  :=    
  (R)\psubstp{Q}{P} \juxtap (S) \psubstp{Q}{P} \\
  (x?(y).R) \psubstp{Q}{P}    
  :=    
  (x)\substp{Q}{P} (z)\concat( (R \psubstn{z}{y}) \psubstp{Q}{P} ) \\
  (\lift{x}{R}) \psubstp{Q}{P}  
  :=
  \lift{(x)\substp{Q}{P}}{ R \psubstp{Q}{P} } \\
%   (\dropn{x})  \psubstp{Q}{P}       
%   := 
%   \left\{ 
%     \begin{array}{ccc} 
%       \dropn{\quotep{Q}} & & x \nameeq \quotep{P} \\
%       \dropn{x} & & otherwise \\
%     \end{array}
%   \right. 
  (\dropn{x})  \psubstp{Q}{P}       
  := 
  \left\{ 
    \begin{array}{ccc} 
      Q & & x \nameeq \quotep{P} \\
      \dropn{x} & & otherwise \\
    \end{array}
  \right.
\end{mathpar}
 

where

\begin{eqnarray}
  (x)\id{\{} \lpquote Q \rpquote / \lpquote P \rpquote \id{\}}            = 
  \left\{ 
    \begin{array}{ccc}
      \lpquote Q \rpquote & & x \nameeq \lpquote P \rpquote \\
      x & & otherwise \\
    \end{array}
  \right. \nonumber
\end{eqnarray}

and $z$ is chosen distinct from $\quotep{P}$, $\quotep{Q}$, the free
names in $Q$, and all the names in $R$. Our $\alpha$-equivalence will
be built in the standard way from this substitution.

\begin{remark}\label{rem:no_self_referential_names}
  One consequence of these definitions is that $\forall P. \quotep{P}
  \not\in \freenames{P}$.
\end{remark}

\subsection{ Dynamic quote: an example }

Anticipating something of what's to come, consider applying the
substitution, $\widehat{\id{\{}u / z \id{\}}}$, to the following pair
of processes, $\lift{w}{y!(z)}$ and $w[ \lpquote y!(z) \rpquote ]$.

\begin{eqnarray}
	\lift{w}{y!(z)}\widehat{\id{\{}u / z \id{\}}}
		& = &
		\lift{w}{y!(u)} \nonumber\\
	w[ \lpquote y!(z) \rpquote ] \widehat{ \id{\{}u / z \id{\}} }
		& = &
		w[ \lpquote y!(z) \rpquote ] \nonumber
\end{eqnarray}

Because the body of the process between quotes is impervious to
substitution, we get radically different answers. In fact, by
examining the first process in an input context,
e.g. $x?(z).\lift{w}{y!(z)}$, we see that the process under the lift
operator may be shaped by prefixed inputs binding a name inside it. In
this sense, the lift operator will be seen as a way to dynamically
construct processes before reifying them as names.

Finally equipped with these standard features we can present the
dynamics of the calculus.

\subsubsection{Operational semantics} 

Finally, we introduce the computational dynamics. What marks these
algebras as distinct from other more traditionally studied algebraic
structures, e.g. vector spaces or polynomial rings, is the manner in
which dynamics is captured. In traditional structures, dynamics is typically
expressed through morphisms between such structures, as in linear maps
between vector spaces or morphisms between rings. In algebras
associated with the semantics of computation, the dynamics is
expressed as part of the algebraic structure itself, through a
reduction reduction relation typically denoted by $\red$. Below, we
give a recursive presentation of this relation for the calculus used
in the encoding.

$\red \subseteq \pi \times \pi$
$\red : \pi \to \mathcal{P}(\pi)$

\begin{mathpar}
  \inferrule* [lab=Comm] { \textsf{match}( x_{src}, x_{trgt} ) } { x_{trgt}?(y)P \; | \; x_{src}!\langle {Q} \rangle \red P\{\quotep{Q}/y}\} }
  \and \\
  \inferrule* [lab=Par] {{P} \red {P}'} {{{P} | {Q}} \red {{P}' | {Q}}}
  \and
  \inferrule* [lab=Equiv]{{{P} \scong {P}'} \andalso {{P}' \red {Q}'} \andalso {{Q}' \scong {Q}}}{{P} \red {Q}}
\end{mathpar}

\begin{eqnarray*}
  match_{\equiv} (\quotep{P},\quotep{Q}) & := & P \equiv Q \\
  match_{\dagger}(\quotep{P},\quotep{Q}) & := & \forall R. P|Q \red^{*} R => R \red^{*} 0 \\
  match_{K}(\quotep{P},\quotep{Q}) & := & K \mbox{ for some context } K
\end{eqnarray*}

$u?(x)P | u!\langle Q \rangle \red P\{\quotep{Q}/x\}$

%We write $\wred$ for $\red^*$, and $P\red$ if $\exists Q $ such that $ P \red Q$.
We write $P\red$ if $\exists Q $ such that $ P \red Q$ and $P\not\red$, otherwise.

\section{Replication}

As mentioned before, it is known that replication (and hence
recursion) can be implemented in a higher-order process algebra
\cite{SangiorgiWalker}. As our first example of calculation with the
machinery thus far presented we give the construction explicitly in
the {\rhoc}.

\begin{eqnarray}
	D_{x} & := & \prefix{x}{y}{(\binpar{\outputp{x}{y}}{@{y}})} \nonumber\\
	\bangp_{x}{P} & := & \binpar{{x}!\langle{\binpar{D_{x}}{P}}\rangle}{D_{x}} \nonumber
\end{eqnarray}

\begin{eqnarray}
	\bangp_{x}{P} & & \nonumber\\
	=
	& {x}!\langle{(\prefix{x}{y}{(\outputp{x}{y} | @{y})) | P}}\rangle 
	      | \prefix{x}{y}{(\outputp{x}{y} | @{y})} & \nonumber\\
	\red
	& (\outputp{x}{y} | @{y})\substn{\quotep{(\prefix{x}{y}{(@{y} | \outputp{x}{y})) | P}}}{y} & \nonumber\\
	=
	& \outputp{x}{\quotep{(\prefix{x}{y}{(\outputp{x}{y} | @{y})) | P}}}
	  | {(\prefix{x}{y}{(\outputp{x}{y} | @{y})) | P}} & \nonumber\\
	\red
	& \ldots & \nonumber\\
	\red^*
	& P | P | \ldots & \nonumber
\end{eqnarray}

Of course, this encoding, as an implementation, runs away, unfolding
$\bangp{P}$ eagerly. A lazier and more implementable replication
operator, restricted to input-guarded processes, may be obtained as follows.

\begin{eqnarray}
\bangp{\prefix{u}{v}{P}} 
	:= 
	\binpar{\lift{x}{\prefix{u}{v}{(\binpar{D(x)}{P})}}}{D(x)} \nonumber
\end{eqnarray}

\begin{remark}
  Note that the lazier definition still does not deal with summation
  or mixed summation (i.e. sums over input and output). The reader is
  invited to construct definitions of replication that deal with these
  features. 

  Further, the definitions are parameterized in a name, $x$. Can you,
  gentle reader, make a definition that eliminates this parameter and
  guarantees no accidental interaction between the replication
  machinery and the process being replicated -- i.e. no accidental
  sharing of names used by the process to get its work done and the
  name(s) used by the replication to effect copying. This latter
  revision of the definition of replication is crucial to obtaining
  the expected identity $!!P \sim !P$.
\end{remark}

\begin{remark}\label{rem:paradoxical_combinator}
  The reader familiar with the lambda calculus will have noticed the
  similarity between $D$ and the paradoxical combinator.

  [Ed. note: the existence of this seems to suggest we have to be more
  restrictive on the set of processes and names we admit if we are to
  support no-cloning.]
\end{remark}

\subsubsection{Bisimulation}

The computational dynamics gives rise to another kind of equivalence,
the equivalence of computational behavior. As previously mentioned
this is typically captured \emph{via} some form of bisimulation.

% The notion we use in this paper is weak barbed bisimulation
% \cite{milner91polyadicpi}.

The notion we use in this paper is derived from weak barbed
bisimulation \cite{milner91polyadicpi}. 

\begin{definition}
An \emph{observation relation}, $\downarrow_{\mathcal N}$, over a set
of names, $\mathcal N$, is the smallest relation satisfying the rules
below.

\infrule[Out-barb]{y \in {\mathcal N}, \; x \nameeq y}
		  {\outputp{x}{v} \downarrow_{\mathcal N} x}
\infrule[Par-barb]{\mbox{$P\downarrow_{\mathcal N} x$ or $Q\downarrow_{\mathcal N} x$}}
		  {\binpar{P}{Q} \downarrow_{\mathcal N} x}

We write $P \Downarrow_{\mathcal N} x$ if there is $Q$ such that 
$P \wred Q$ and $Q \downarrow_{\mathcal N} x$.
\end{definition}

\begin{definition}
%\label{def.bbisim}
An  ${\mathcal N}$-\emph{barbed bisimulation} over a set of names, ${\mathcal N}$, is a symmetric binary relation 
${\mathcal S}_{\mathcal N}$ between agents such that $P\rel{S}_{\mathcal N}Q$ implies:
\begin{enumerate}
\item If $P \red P'$ then $Q \wred Q'$ and $P'\rel{S}_{\mathcal N} Q'$.
\item If $P\downarrow_{\mathcal N} x$, then $Q\Downarrow_{\mathcal N} x$.
\end{enumerate}
$P$ is ${\mathcal N}$-barbed bisimilar to $Q$, written
$P \wbbisim_{\mathcal N} Q$, if $P \rel{S}_{\mathcal N} Q$ for some ${\mathcal N}$-barbed bisimulation ${\mathcal S}_{\mathcal N}$.
\end{definition}

$\mathcal{R} \subseteq \pi \times \pi$

$P \mathcal{R} Q => \forall P'. P \red P' \Rightarrow \exists Q'. Q \red Q', P' \mathcal{R} Q'$

$P \vdash x \Rightarrow Q \vdash x$

\begin{mathpar}
  \inferrule*[lab=Out-barb]{x \nameeq y}{{y}!\langle{Q}\rangle \vdash x}
  \and
  \inferrule*[lab=Par-barb]{\mbox{$P\vdash x$ or $Q\vdash x$}}{\binpar{P}{Q} \vdash x}
\end{mathpar}

\subsubsection{Contexts}

One of the principle advantages of computational calculi like the
$\pi$-calculus is a well-defined notion of context,
contextual-equivalence and a correlation between
contextual-equivalence and notions of bisimulation. The notion of
context allows the decomposition of a process into (sub-)process and
its syntactic environment, its context. Thus, a context may be
thought of as a process with a ``hole'' (written $\Box$) in it. The
application of a context $M$ to a process $P$, written $M[P]$, is
tantamount to filling the hole in $M$ with $P$. In this paper we do
not need the full weight of this theory, but do make use of the notion
of context in the proof the main theorem. 

\begin{mathpar}
  \inferrule* [lab=summation] {} {{M_{M},M_{N}} \bc \Box \;|\; x.M_{A} \;|\; M_{M}+M_{N}}
  \and
  \inferrule* [lab=agent] {} {{M_{A}} \bc (\vec{x})M_{P} \;| \; \clift{P_0,\ldots,M_{P},\ldots,P_N}}
  \and \\
  \inferrule* [lab=process] {} {{M_{P}} \bc M_{N} \;| \;P|M_{P} }
\end{mathpar} 

\begin{mathpar}
  \inferrule* [lab=sychronization] {} {M_{N} \bc \Box \;|\; x?M_{F} \;|\; x!M_{C}}
  \and
  \inferrule* [lab=abstraction] {} {{M_{F}} \bc (x)M_{P} }
  \and
  \inferrule* [lab=concretion] {} {{M_{C}} \bc \langle M_{P} \rangle }
  \and \\
  \inferrule* [lab=process] {} {{M_{P}} \bc M_{N} \;| \;P|M_{P} }
\end{mathpar}

\begin{definition}[contextual application] Given a context $M$, and
  process $P$, we define the \emph{contextual application}, $M[P] :=
  M\{P/\Box\}$. That is, the contextual application of M to P is the
  substitution of $P$ for $\Box$ in $M$.
\end{definition}

$\meaningof{-} : L \to \mathcal{P}(\pi)$

\begin{mathpar}
  \inferrule* [lab=collection] {} {\meaningof{true} = \pi, \and \meaningof{~E} = \pi \setminus \meaningof{E}, \and \meaningof{E_{1} \& E_{2}} = \meaningof{E_{1}} \cap \meaningof{E_{2}}}
\end{mathpar}

\begin{mathpar}
  \inferrule* [lab=structure] {} {\meaningof{0} = \{ P \in \pi | P \equiv 0 \}, \and \\ \meaningof{E_1 | E_2} = \{ P \in \pi | P \equiv P_{1} | P_{2}, P_{1} \in \meaningof{E_{1}}, P_{2} \in \meaningof{E_2}\} }
\end{mathpar}

\begin{mathpar}
 \inferrule* [lab=behavior] {} {\meaningof{\langle a?b \rangle E} = \{ P \in \pi | P \equiv Q | u?(y)P', \\ \and \\\\ \and \\ \;\;\; u \in \meaningof{a}, \forall z.P'\{z/y\} \in \meaningof{E\{z/b\}}\}, \and \\ \meaningof{a!E} = \{ P \in \pi | P \equiv Q | x!\langle P' \rangle, x \in \meaningof{a} P' \in \meaningof{E}\} }
\end{mathpar}

\begin{mathpar}
 \inferrule* [lab=nominal] {} {\meaningof{\quotep{E}} = \{ \quotep{P} \in \quotep{\pi} | P \in \meaningof{E} \}, \and \meaningof{\quotep{P}} = \{ \quotep{Q} \in \quotep{\pi} | P \equiv Q \} \and \\ \meaningof{@\quotep{E}} = \{ P \in \pi | P \equiv @x, x \in \meaningof{E} \}}
\end{mathpar}

\begin{eqnarray*}
  \\
  \meaningof{-} : TS \to ST
\end{eqnarray*}

\begin{eqnarray*}
  \\
  L : TS \to ST
\end{eqnarray*}

\begin{eqnarray*}
  \\
  P \models E \iff P \in \meaningof{E}
\end{eqnarray*}

\begin{eqnarray*}
  P \approx_{L} Q \iff \forall E \in L. P \models E \iff Q \models E
\end{eqnarray*}

\begin{eqnarray*}
  P \approx_{K} Q
\end{eqnarray*}

\begin{eqnarray*}
  P \approx Q
\end{eqnarray*}

$\approx_{K} = \approx = \approx_{L}$

\subsubsection{Contextual duality}

Note that contexts extend the quotation operation to a family of
operations from processes to names. Given a context, $M$, we can
define a \emph{nominal context}, $\quotep{M}$ by $\quotep{M}[P] :=
\quotep{M[P]}$. To foreshadow what is to come we observe that these
operations enjoy a duality with processes very much like the duality
between vectors and maps from vectors to scalars.

Further, because the calculus is essentially higher-order, we have a
correspondence between contexts and processes. More specifically,
given a name $x$ and a context $M$ we can construct $M^{*}_{x}$ such
that 

\begin{mathpar}
  M^{*}_{x} | \lift{x}{P} \red M[P]
\end{mathpar}

namely,

\begin{mathpar}
  M^{*}_{x} := x?(u).M[\dropn{u}]
\end{mathpar}

The dependence of $M^{*}_{x}$ on a name makes it an abstraction, 

\begin{mathpar}
  M^{*} := (x)x?(u).M[\dropn{u}]
\end{mathpar}

\subsection{Additional notation}

It will sometimes be convenient to denote the process a name
quotes. We already have the notation $x = \quotep{P}$, but it will be
convenient to introduce an alternate notation, $\procn{x}$, when we
want to emphasize the connection to the use of the name. Note that, by
virtue of name equivalence, $\quotep{\procn{x}} \nameeq x$; so, the
notation is consistent with previous definitions.

Further, because names have structure it is possible to effect
substitutions on the basis of that structure. This means we need to
upgrade our notation for substitutions, which we accomplish by
adapting comprehension notation. Thus,

\begin{mathpar}
  P\{ y / x : x \in S \}
\end{mathpar}

is interpreted to mean the process derived from P by replacing (in a
capture-avoiding manner) each occurrence of $x$ in $S$ by $y$. For example,

\begin{mathpar}
  P\{ \quotep{\procn{x}|\procn{x}} / x : x \in \freenames{P} \}
\end{mathpar}

will replace each (occurrence) of a free name $x$ in $P$ by
$\quotep{\procn{x}|\procn{x}}$.

Also, we will avail ourselves of the notation $x^{L}$ and $x^{R}$ to
denote injections of a name into disjoint copies of the name
space. There are numerous ways to accomplish this. One example can be
found in \cite{MeredithR05}. This notation overloads to vectors of
names: $\vec{x}^{\pi} := (x_{i}^{\pi} \; : \; 0 \leq i < |\vec{x}| )$ where $\pi \in \{L,R\}$.

We also use $P^{\Box} := P|\Box$.

In \cite{MeredithR05} an interpretation of the new operator is
given. It turns out that there are several possible interpretations
all enjoying the requisite algebraic properties of the operator (see
\cite{milner91polyadicpi}). We will therefore make liberal use of
$(\nu\; \vec{x})P$.

% subsection the_syntax_and_semantics_of_the_notation_system (end)   

\input{qm2pi.qmops} 

\input{qm2pi.sterngerlach} 

\input{qm2pi.metric} 

% section concurrent_process_calculi (end)

%\input{qm2pi.proofsketch}

% section proof sketch (end)

%\input{qm2pi.slviaknots} 

% section spatial logic via knots (end)

\input{qm2pi.conclusion}

% section conclusion (end)

%\input{qm2pi.dtcodes} 

% section wiring algorithm (end)

\input{qm2pi.ack} 

% section acknowledgments (end)

\newpage


\bibliographystyle{plain}   
\bibliography{../../biblios/main.bib}

\input{qm2pi.rhodetails}

\end{document}

 

% section wiring algorithm (end)

\documentclass[12pt]{llncs}
%\documentclass{jktr}

\usepackage[pdftex]{hyperref}                   
\usepackage {listings}
\usepackage {mathpartir}
\usepackage{bcprules}
%\usepackage{listings}
                       
\usepackage{graphicx} 
%\usepackage[margins=2.5cm,nohead,nofoot]{geometry}
%\usepackage{geometry}
\usepackage{amsfonts}
\usepackage{amstext}
\usepackage{latexsym}
\usepackage{amssymb}
\usepackage{color}


%\include{myPreamble}
\include{qm2pi.local} 

%\ifpdf
%\usepackage[pdftex]{graphicx}
%\else
%\usepackage{graphicx}
%\fi

 % \ifpdf
%  \usepackage{pdfsync}
%  \if


%\title{Brief Article}
%\author{David F. Snyder}
%\author{L.G. Meredith}

%\address{Dept. of Math., Texas State University--San Marcos, San Marcos, TX 78666}
       
\pagestyle{empty}


\begin{document}

\lstset{language=[Objective]Caml,frame=shadowbox}

\input{qm2pi.front}

% section front matter (end)

\input{qm2pi.intro} 
 
% section introduction (end)

% \input{qm2pi.knotations} 

% section notation (end)

\input{qm2pi.process.calculi} 

% section concurrent_process_calculi_and_spatial_logics_ (end)
    
%\input{qm2pi.knots2pi} 

%\input{qm2pi.trefoil} 

%\input{qm2pi.mainthm} 

% subsection basic_interpretation (end)

%\input{qm2pi.rho.presentation} 
\subsection{The syntax and semantics of the notation system}\label{sub:the_syntax_and_semantics_of_the_notation_system} % (fold)

We now summarize a technical presentation of the calculus that
embodies our theory of dynamics. The typical presentation of such a
calculus follows the style of giving generators and relations on
them. The grammar, below, describing term constructors, freely
generates the set of processes, $\Proc$. This set is then quotiented
by a relation known as structural congruence and it is over this set
that the notion of dynamics is expressed. This presentation is
essentially that of \cite{MeredithR05} with the addition of
polyadicity and summation. For readability we have relegated some of
the technical subtleties to an appendix.

\subsubsection{Process grammar}\label{subsub:process_grammar}

\begin{mathpar}
  \inferrule* [lab=synchronization] {} {{M} \bc \pzero \;|\; x?F \;|\; x!C }
  \and
  \inferrule* [lab=abstraction] {} {{F} \bc (x)P}
  \and
  \inferrule* [lab=concretion] {} {{C} \bc \langle Q \rangle}
  \and
  \inferrule* [lab=process] {} {{P,Q} \bc M \;| \;P|Q \;|\; @{x}}
  \and
  \inferrule* [lab=name] {} {{x} \bc \quotep{P}}
\end{mathpar} 

Note that $\vec{x}$ (resp. $\vec{P}$) denotes a vector of names
(resp. processes) of length $|\vec{x}|$ (resp. $|\vec{P}|$). We adopt
the following useful abbreviations.

\begin{mathpar}
   x?(\vec{y}).P := x.(\vec{y})P \and  x\clift{\vec{P}} := x.\clift{\vec{P}}
   \and x!(y) := \lift{x}{\dropn{y}}
   \and \Pi_{i=0}^{n-1}P_i := P_0 | \ldots | P_{n-1}
\end{mathpar}

\subsubsection{Structural congruence}

\paragraph{Free and bound names and alpha-equivalence.} At the
core of structural equivalence is alpha-equivalence which identifies
process that are the same up to a change of variable. Formally, we
recognize the distinction between free and bound names. The free names
of a process, $\freenames{P}$, may be calculated recursively as
follows:

\begin{mathpar}
\freenames{\pzero} := \emptyset
  \and \\
  \freenames{x?(y).P} := \{ x \} \cup (\freenames{P} \setminus \{ y \})
  \and 
  \freenames{x!\langle P \rangle} := \{ x \} \cup \{ P \} 
  \and \\
  \freenames{P|Q} := \freenames{P} \cup \freenames{Q}
  \and \\
  \freenames{@{x}} := \{ x \}
\end{mathpar}

$\pi$
$\quotep{\pi}$

$\freenames{-} : \pi \to \mathcal{P}(\quotep{\pi})$

\begin{eqnarray*}
  \freenames{\pzero} & := & \emptyset \\
  \freenames{x?(y).P} & := & \{ x \} \cup (\freenames{P} \setminus \{ y \}) \\
  \freenames{x!\langle P \rangle} & := & \{ x \} \cup \{ P \} \\
  \freenames{P|Q} & := & \freenames{P} \cup \freenames{Q} \\
  \freenames{\dropn{x}} & := & \{ x \}
\end{eqnarray*}

The bound names of a process, $\boundnames{P}$, are those names occurring in $P$
that are not free. For example, in $x?(y).0$, the name $x$ is free, while $y$ is bound.

\begin{mathpar}
  \inferrule* [lab=monoidal-laws] {} { P|Q \equiv Q|P \and P|0 \equiv P \and P|(Q|R) \equiv (P|Q)|R }
\end{mathpar}

\begin{mathpar}
  \inferrule* [lab=alpha-equivalence] {} { (x)P \equiv (y)P\{y/x\} \and y \not\in \freenames{P} }
\end{mathpar}

\begin{definition}
Then two processes, $P,Q$, are alpha-equivalent if $P = Q\{\vec{y}/\vec{x}\}$ for
some $\vec{x} \in \boundnames{Q},\vec{y} \in \boundnames{P}$, where $Q\{\vec{y}/\vec{x}\}$
denotes the capture-avoiding substitution of $\vec{y}$ for $\vec{x}$ in $Q$.
\end{definition}

\begin{definition}
  The {\em structural congruence} \cite{SangiorgiWalker} , $\equiv$,
  between processes is the least congruence containing
  alpha-equivalence, satisfying the abelian monoid laws
  (associativity, commutativity and $\pzero$ as identity) for parallel
  composition $|$ and for summation $+$.
\end{definition}

\subsection{Name equivalence}

We take name equivalence, written $\nameeq$, to be the smallest
equivalence relation generated by the following rules.

\begin{mathpar}
\inferrule*[lab=Quote-drop]
{ }
{ \quotep{@{x}} \nameeq x }

\inferrule*[lab=Struct-equiv]
{ P \scong Q }
{ \quotep{P} \nameeq \quotep{Q} }
\end{mathpar}

The astute reader will have noticed that the mutual recursion of names
and processes imposes a mutual recursion on alpha-equivalence and
structural equivalence via name-equivalence. Fortunately, all of this
works out pleasantly and we may calculate in the natural way, free of
concern. The reader interested in the details is referred to the
appendix \ref{appendix:rho_details}.

\subsection{Substitution}

We use $\Proc$ for the set of processes, $\QProc$ for the set of
names, and $\id{\{}\vec{y} / \vec{x} \id{\}}$ to denote partial maps,
$s : \QProc \rightarrow \QProc$. A map, $s$ lifts, uniquely, to a map
on process terms, $\widehat{s} : \Proc \rightarrow \Proc$ by the
following equations.

\begin{mathpar}
  (0) \psubstp{Q}{P} := 0 \\
  (R \juxtap S) \psubstp{Q}{P}
  :=    
  (R)\psubstp{Q}{P} \juxtap (S) \psubstp{Q}{P} \\
  (x?(y).R) \psubstp{Q}{P}    
  :=    
  (x)\substp{Q}{P} (z)\concat( (R \psubstn{z}{y}) \psubstp{Q}{P} ) \\
  (\lift{x}{R}) \psubstp{Q}{P}  
  :=
  \lift{(x)\substp{Q}{P}}{ R \psubstp{Q}{P} } \\
%   (\dropn{x})  \psubstp{Q}{P}       
%   := 
%   \left\{ 
%     \begin{array}{ccc} 
%       \dropn{\quotep{Q}} & & x \nameeq \quotep{P} \\
%       \dropn{x} & & otherwise \\
%     \end{array}
%   \right. 
  (\dropn{x})  \psubstp{Q}{P}       
  := 
  \left\{ 
    \begin{array}{ccc} 
      Q & & x \nameeq \quotep{P} \\
      \dropn{x} & & otherwise \\
    \end{array}
  \right.
\end{mathpar}
 

where

\begin{eqnarray}
  (x)\id{\{} \lpquote Q \rpquote / \lpquote P \rpquote \id{\}}            = 
  \left\{ 
    \begin{array}{ccc}
      \lpquote Q \rpquote & & x \nameeq \lpquote P \rpquote \\
      x & & otherwise \\
    \end{array}
  \right. \nonumber
\end{eqnarray}

and $z$ is chosen distinct from $\quotep{P}$, $\quotep{Q}$, the free
names in $Q$, and all the names in $R$. Our $\alpha$-equivalence will
be built in the standard way from this substitution.

\begin{remark}\label{rem:no_self_referential_names}
  One consequence of these definitions is that $\forall P. \quotep{P}
  \not\in \freenames{P}$.
\end{remark}

\subsection{ Dynamic quote: an example }

Anticipating something of what's to come, consider applying the
substitution, $\widehat{\id{\{}u / z \id{\}}}$, to the following pair
of processes, $\lift{w}{y!(z)}$ and $w[ \lpquote y!(z) \rpquote ]$.

\begin{eqnarray}
	\lift{w}{y!(z)}\widehat{\id{\{}u / z \id{\}}}
		& = &
		\lift{w}{y!(u)} \nonumber\\
	w[ \lpquote y!(z) \rpquote ] \widehat{ \id{\{}u / z \id{\}} }
		& = &
		w[ \lpquote y!(z) \rpquote ] \nonumber
\end{eqnarray}

Because the body of the process between quotes is impervious to
substitution, we get radically different answers. In fact, by
examining the first process in an input context,
e.g. $x?(z).\lift{w}{y!(z)}$, we see that the process under the lift
operator may be shaped by prefixed inputs binding a name inside it. In
this sense, the lift operator will be seen as a way to dynamically
construct processes before reifying them as names.

Finally equipped with these standard features we can present the
dynamics of the calculus.

\subsubsection{Operational semantics} 

Finally, we introduce the computational dynamics. What marks these
algebras as distinct from other more traditionally studied algebraic
structures, e.g. vector spaces or polynomial rings, is the manner in
which dynamics is captured. In traditional structures, dynamics is typically
expressed through morphisms between such structures, as in linear maps
between vector spaces or morphisms between rings. In algebras
associated with the semantics of computation, the dynamics is
expressed as part of the algebraic structure itself, through a
reduction reduction relation typically denoted by $\red$. Below, we
give a recursive presentation of this relation for the calculus used
in the encoding.

$\red \subseteq \pi \times \pi$
$\red : \pi \to \mathcal{P}(\pi)$

\begin{mathpar}
  \inferrule* [lab=Comm] { \textsf{match}( x_{src}, x_{trgt} ) } { x_{trgt}?(y)P \; | \; x_{src}!\langle {Q} \rangle \red P\{\quotep{Q}/y}\} }
  \and \\
  \inferrule* [lab=Par] {{P} \red {P}'} {{{P} | {Q}} \red {{P}' | {Q}}}
  \and
  \inferrule* [lab=Equiv]{{{P} \scong {P}'} \andalso {{P}' \red {Q}'} \andalso {{Q}' \scong {Q}}}{{P} \red {Q}}
\end{mathpar}

\begin{eqnarray*}
  match_{\equiv} (\quotep{P},\quotep{Q}) & := & P \equiv Q \\
  match_{\dagger}(\quotep{P},\quotep{Q}) & := & \forall R. P|Q \red^{*} R => R \red^{*} 0 \\
  match_{K}(\quotep{P},\quotep{Q}) & := & K \mbox{ for some context } K
\end{eqnarray*}

$u?(x)P | u!\langle Q \rangle \red P\{\quotep{Q}/x\}$

%We write $\wred$ for $\red^*$, and $P\red$ if $\exists Q $ such that $ P \red Q$.
We write $P\red$ if $\exists Q $ such that $ P \red Q$ and $P\not\red$, otherwise.

\section{Replication}

As mentioned before, it is known that replication (and hence
recursion) can be implemented in a higher-order process algebra
\cite{SangiorgiWalker}. As our first example of calculation with the
machinery thus far presented we give the construction explicitly in
the {\rhoc}.

\begin{eqnarray}
	D_{x} & := & \prefix{x}{y}{(\binpar{\outputp{x}{y}}{@{y}})} \nonumber\\
	\bangp_{x}{P} & := & \binpar{{x}!\langle{\binpar{D_{x}}{P}}\rangle}{D_{x}} \nonumber
\end{eqnarray}

\begin{eqnarray}
	\bangp_{x}{P} & & \nonumber\\
	=
	& {x}!\langle{(\prefix{x}{y}{(\outputp{x}{y} | @{y})) | P}}\rangle 
	      | \prefix{x}{y}{(\outputp{x}{y} | @{y})} & \nonumber\\
	\red
	& (\outputp{x}{y} | @{y})\substn{\quotep{(\prefix{x}{y}{(@{y} | \outputp{x}{y})) | P}}}{y} & \nonumber\\
	=
	& \outputp{x}{\quotep{(\prefix{x}{y}{(\outputp{x}{y} | @{y})) | P}}}
	  | {(\prefix{x}{y}{(\outputp{x}{y} | @{y})) | P}} & \nonumber\\
	\red
	& \ldots & \nonumber\\
	\red^*
	& P | P | \ldots & \nonumber
\end{eqnarray}

Of course, this encoding, as an implementation, runs away, unfolding
$\bangp{P}$ eagerly. A lazier and more implementable replication
operator, restricted to input-guarded processes, may be obtained as follows.

\begin{eqnarray}
\bangp{\prefix{u}{v}{P}} 
	:= 
	\binpar{\lift{x}{\prefix{u}{v}{(\binpar{D(x)}{P})}}}{D(x)} \nonumber
\end{eqnarray}

\begin{remark}
  Note that the lazier definition still does not deal with summation
  or mixed summation (i.e. sums over input and output). The reader is
  invited to construct definitions of replication that deal with these
  features. 

  Further, the definitions are parameterized in a name, $x$. Can you,
  gentle reader, make a definition that eliminates this parameter and
  guarantees no accidental interaction between the replication
  machinery and the process being replicated -- i.e. no accidental
  sharing of names used by the process to get its work done and the
  name(s) used by the replication to effect copying. This latter
  revision of the definition of replication is crucial to obtaining
  the expected identity $!!P \sim !P$.
\end{remark}

\begin{remark}\label{rem:paradoxical_combinator}
  The reader familiar with the lambda calculus will have noticed the
  similarity between $D$ and the paradoxical combinator.

  [Ed. note: the existence of this seems to suggest we have to be more
  restrictive on the set of processes and names we admit if we are to
  support no-cloning.]
\end{remark}

\subsubsection{Bisimulation}

The computational dynamics gives rise to another kind of equivalence,
the equivalence of computational behavior. As previously mentioned
this is typically captured \emph{via} some form of bisimulation.

% The notion we use in this paper is weak barbed bisimulation
% \cite{milner91polyadicpi}.

The notion we use in this paper is derived from weak barbed
bisimulation \cite{milner91polyadicpi}. 

\begin{definition}
An \emph{observation relation}, $\downarrow_{\mathcal N}$, over a set
of names, $\mathcal N$, is the smallest relation satisfying the rules
below.

\infrule[Out-barb]{y \in {\mathcal N}, \; x \nameeq y}
		  {\outputp{x}{v} \downarrow_{\mathcal N} x}
\infrule[Par-barb]{\mbox{$P\downarrow_{\mathcal N} x$ or $Q\downarrow_{\mathcal N} x$}}
		  {\binpar{P}{Q} \downarrow_{\mathcal N} x}

We write $P \Downarrow_{\mathcal N} x$ if there is $Q$ such that 
$P \wred Q$ and $Q \downarrow_{\mathcal N} x$.
\end{definition}

\begin{definition}
%\label{def.bbisim}
An  ${\mathcal N}$-\emph{barbed bisimulation} over a set of names, ${\mathcal N}$, is a symmetric binary relation 
${\mathcal S}_{\mathcal N}$ between agents such that $P\rel{S}_{\mathcal N}Q$ implies:
\begin{enumerate}
\item If $P \red P'$ then $Q \wred Q'$ and $P'\rel{S}_{\mathcal N} Q'$.
\item If $P\downarrow_{\mathcal N} x$, then $Q\Downarrow_{\mathcal N} x$.
\end{enumerate}
$P$ is ${\mathcal N}$-barbed bisimilar to $Q$, written
$P \wbbisim_{\mathcal N} Q$, if $P \rel{S}_{\mathcal N} Q$ for some ${\mathcal N}$-barbed bisimulation ${\mathcal S}_{\mathcal N}$.
\end{definition}

$\mathcal{R} \subseteq \pi \times \pi$

$P \mathcal{R} Q => \forall P'. P \red P' \Rightarrow \exists Q'. Q \red Q', P' \mathcal{R} Q'$

$P \vdash x \Rightarrow Q \vdash x$

\begin{mathpar}
  \inferrule*[lab=Out-barb]{x \nameeq y}{{y}!\langle{Q}\rangle \vdash x}
  \and
  \inferrule*[lab=Par-barb]{\mbox{$P\vdash x$ or $Q\vdash x$}}{\binpar{P}{Q} \vdash x}
\end{mathpar}

\subsubsection{Contexts}

One of the principle advantages of computational calculi like the
$\pi$-calculus is a well-defined notion of context,
contextual-equivalence and a correlation between
contextual-equivalence and notions of bisimulation. The notion of
context allows the decomposition of a process into (sub-)process and
its syntactic environment, its context. Thus, a context may be
thought of as a process with a ``hole'' (written $\Box$) in it. The
application of a context $M$ to a process $P$, written $M[P]$, is
tantamount to filling the hole in $M$ with $P$. In this paper we do
not need the full weight of this theory, but do make use of the notion
of context in the proof the main theorem. 

\begin{mathpar}
  \inferrule* [lab=summation] {} {{M_{M},M_{N}} \bc \Box \;|\; x.M_{A} \;|\; M_{M}+M_{N}}
  \and
  \inferrule* [lab=agent] {} {{M_{A}} \bc (\vec{x})M_{P} \;| \; \clift{P_0,\ldots,M_{P},\ldots,P_N}}
  \and \\
  \inferrule* [lab=process] {} {{M_{P}} \bc M_{N} \;| \;P|M_{P} }
\end{mathpar} 

\begin{mathpar}
  \inferrule* [lab=sychronization] {} {M_{N} \bc \Box \;|\; x?M_{F} \;|\; x!M_{C}}
  \and
  \inferrule* [lab=abstraction] {} {{M_{F}} \bc (x)M_{P} }
  \and
  \inferrule* [lab=concretion] {} {{M_{C}} \bc \langle M_{P} \rangle }
  \and \\
  \inferrule* [lab=process] {} {{M_{P}} \bc M_{N} \;| \;P|M_{P} }
\end{mathpar}

\begin{definition}[contextual application] Given a context $M$, and
  process $P$, we define the \emph{contextual application}, $M[P] :=
  M\{P/\Box\}$. That is, the contextual application of M to P is the
  substitution of $P$ for $\Box$ in $M$.
\end{definition}

$\meaningof{-} : L \to \mathcal{P}(\pi)$

\begin{mathpar}
  \inferrule* [lab=collection] {} {\meaningof{true} = \pi, \and \meaningof{~E} = \pi \setminus \meaningof{E}, \and \meaningof{E_{1} \& E_{2}} = \meaningof{E_{1}} \cap \meaningof{E_{2}}}
\end{mathpar}

\begin{mathpar}
  \inferrule* [lab=structure] {} {\meaningof{0} = \{ P \in \pi | P \equiv 0 \}, \and \\ \meaningof{E_1 | E_2} = \{ P \in \pi | P \equiv P_{1} | P_{2}, P_{1} \in \meaningof{E_{1}}, P_{2} \in \meaningof{E_2}\} }
\end{mathpar}

\begin{mathpar}
 \inferrule* [lab=behavior] {} {\meaningof{\langle a?b \rangle E} = \{ P \in \pi | P \equiv Q | u?(y)P', \\ \and \\\\ \and \\ \;\;\; u \in \meaningof{a}, \forall z.P'\{z/y\} \in \meaningof{E\{z/b\}}\}, \and \\ \meaningof{a!E} = \{ P \in \pi | P \equiv Q | x!\langle P' \rangle, x \in \meaningof{a} P' \in \meaningof{E}\} }
\end{mathpar}

\begin{mathpar}
 \inferrule* [lab=nominal] {} {\meaningof{\quotep{E}} = \{ \quotep{P} \in \quotep{\pi} | P \in \meaningof{E} \}, \and \meaningof{\quotep{P}} = \{ \quotep{Q} \in \quotep{\pi} | P \equiv Q \} \and \\ \meaningof{@\quotep{E}} = \{ P \in \pi | P \equiv @x, x \in \meaningof{E} \}}
\end{mathpar}

\begin{eqnarray*}
  \\
  \meaningof{-} : TS \to ST
\end{eqnarray*}

\begin{eqnarray*}
  \\
  L : TS \to ST
\end{eqnarray*}

\begin{eqnarray*}
  \\
  P \models E \iff P \in \meaningof{E}
\end{eqnarray*}

\begin{eqnarray*}
  P \approx_{L} Q \iff \forall E \in L. P \models E \iff Q \models E
\end{eqnarray*}

\begin{eqnarray*}
  P \approx_{K} Q
\end{eqnarray*}

\begin{eqnarray*}
  P \approx Q
\end{eqnarray*}

$\approx_{K} = \approx = \approx_{L}$

\subsubsection{Contextual duality}

Note that contexts extend the quotation operation to a family of
operations from processes to names. Given a context, $M$, we can
define a \emph{nominal context}, $\quotep{M}$ by $\quotep{M}[P] :=
\quotep{M[P]}$. To foreshadow what is to come we observe that these
operations enjoy a duality with processes very much like the duality
between vectors and maps from vectors to scalars.

Further, because the calculus is essentially higher-order, we have a
correspondence between contexts and processes. More specifically,
given a name $x$ and a context $M$ we can construct $M^{*}_{x}$ such
that 

\begin{mathpar}
  M^{*}_{x} | \lift{x}{P} \red M[P]
\end{mathpar}

namely,

\begin{mathpar}
  M^{*}_{x} := x?(u).M[\dropn{u}]
\end{mathpar}

The dependence of $M^{*}_{x}$ on a name makes it an abstraction, 

\begin{mathpar}
  M^{*} := (x)x?(u).M[\dropn{u}]
\end{mathpar}

\subsection{Additional notation}

It will sometimes be convenient to denote the process a name
quotes. We already have the notation $x = \quotep{P}$, but it will be
convenient to introduce an alternate notation, $\procn{x}$, when we
want to emphasize the connection to the use of the name. Note that, by
virtue of name equivalence, $\quotep{\procn{x}} \nameeq x$; so, the
notation is consistent with previous definitions.

Further, because names have structure it is possible to effect
substitutions on the basis of that structure. This means we need to
upgrade our notation for substitutions, which we accomplish by
adapting comprehension notation. Thus,

\begin{mathpar}
  P\{ y / x : x \in S \}
\end{mathpar}

is interpreted to mean the process derived from P by replacing (in a
capture-avoiding manner) each occurrence of $x$ in $S$ by $y$. For example,

\begin{mathpar}
  P\{ \quotep{\procn{x}|\procn{x}} / x : x \in \freenames{P} \}
\end{mathpar}

will replace each (occurrence) of a free name $x$ in $P$ by
$\quotep{\procn{x}|\procn{x}}$.

Also, we will avail ourselves of the notation $x^{L}$ and $x^{R}$ to
denote injections of a name into disjoint copies of the name
space. There are numerous ways to accomplish this. One example can be
found in \cite{MeredithR05}. This notation overloads to vectors of
names: $\vec{x}^{\pi} := (x_{i}^{\pi} \; : \; 0 \leq i < |\vec{x}| )$ where $\pi \in \{L,R\}$.

We also use $P^{\Box} := P|\Box$.

In \cite{MeredithR05} an interpretation of the new operator is
given. It turns out that there are several possible interpretations
all enjoying the requisite algebraic properties of the operator (see
\cite{milner91polyadicpi}). We will therefore make liberal use of
$(\nu\; \vec{x})P$.

% subsection the_syntax_and_semantics_of_the_notation_system (end)   

\input{qm2pi.qmops} 

\input{qm2pi.sterngerlach} 

\input{qm2pi.metric} 

% section concurrent_process_calculi (end)

%\input{qm2pi.proofsketch}

% section proof sketch (end)

%\input{qm2pi.slviaknots} 

% section spatial logic via knots (end)

\input{qm2pi.conclusion}

% section conclusion (end)

%\input{qm2pi.dtcodes} 

% section wiring algorithm (end)

\input{qm2pi.ack} 

% section acknowledgments (end)

\newpage


\bibliographystyle{plain}   
\bibliography{../../biblios/main.bib}

\input{qm2pi.rhodetails}

\end{document}

 

% section acknowledgments (end)

\newpage


\bibliographystyle{plain}   
\bibliography{../../biblios/main.bib}

\documentclass[12pt]{llncs}
%\documentclass{jktr}

\usepackage[pdftex]{hyperref}                   
\usepackage {listings}
\usepackage {mathpartir}
\usepackage{bcprules}
%\usepackage{listings}
                       
\usepackage{graphicx} 
%\usepackage[margins=2.5cm,nohead,nofoot]{geometry}
%\usepackage{geometry}
\usepackage{amsfonts}
\usepackage{amstext}
\usepackage{latexsym}
\usepackage{amssymb}
\usepackage{color}


%\include{myPreamble}
\include{qm2pi.local} 

%\ifpdf
%\usepackage[pdftex]{graphicx}
%\else
%\usepackage{graphicx}
%\fi

 % \ifpdf
%  \usepackage{pdfsync}
%  \if


%\title{Brief Article}
%\author{David F. Snyder}
%\author{L.G. Meredith}

%\address{Dept. of Math., Texas State University--San Marcos, San Marcos, TX 78666}
       
\pagestyle{empty}


\begin{document}

\lstset{language=[Objective]Caml,frame=shadowbox}

\input{qm2pi.front}

% section front matter (end)

\input{qm2pi.intro} 
 
% section introduction (end)

% \input{qm2pi.knotations} 

% section notation (end)

\input{qm2pi.process.calculi} 

% section concurrent_process_calculi_and_spatial_logics_ (end)
    
%\input{qm2pi.knots2pi} 

%\input{qm2pi.trefoil} 

%\input{qm2pi.mainthm} 

% subsection basic_interpretation (end)

%\input{qm2pi.rho.presentation} 
\subsection{The syntax and semantics of the notation system}\label{sub:the_syntax_and_semantics_of_the_notation_system} % (fold)

We now summarize a technical presentation of the calculus that
embodies our theory of dynamics. The typical presentation of such a
calculus follows the style of giving generators and relations on
them. The grammar, below, describing term constructors, freely
generates the set of processes, $\Proc$. This set is then quotiented
by a relation known as structural congruence and it is over this set
that the notion of dynamics is expressed. This presentation is
essentially that of \cite{MeredithR05} with the addition of
polyadicity and summation. For readability we have relegated some of
the technical subtleties to an appendix.

\subsubsection{Process grammar}\label{subsub:process_grammar}

\begin{mathpar}
  \inferrule* [lab=synchronization] {} {{M} \bc \pzero \;|\; x?F \;|\; x!C }
  \and
  \inferrule* [lab=abstraction] {} {{F} \bc (x)P}
  \and
  \inferrule* [lab=concretion] {} {{C} \bc \langle Q \rangle}
  \and
  \inferrule* [lab=process] {} {{P,Q} \bc M \;| \;P|Q \;|\; @{x}}
  \and
  \inferrule* [lab=name] {} {{x} \bc \quotep{P}}
\end{mathpar} 

Note that $\vec{x}$ (resp. $\vec{P}$) denotes a vector of names
(resp. processes) of length $|\vec{x}|$ (resp. $|\vec{P}|$). We adopt
the following useful abbreviations.

\begin{mathpar}
   x?(\vec{y}).P := x.(\vec{y})P \and  x\clift{\vec{P}} := x.\clift{\vec{P}}
   \and x!(y) := \lift{x}{\dropn{y}}
   \and \Pi_{i=0}^{n-1}P_i := P_0 | \ldots | P_{n-1}
\end{mathpar}

\subsubsection{Structural congruence}

\paragraph{Free and bound names and alpha-equivalence.} At the
core of structural equivalence is alpha-equivalence which identifies
process that are the same up to a change of variable. Formally, we
recognize the distinction between free and bound names. The free names
of a process, $\freenames{P}$, may be calculated recursively as
follows:

\begin{mathpar}
\freenames{\pzero} := \emptyset
  \and \\
  \freenames{x?(y).P} := \{ x \} \cup (\freenames{P} \setminus \{ y \})
  \and 
  \freenames{x!\langle P \rangle} := \{ x \} \cup \{ P \} 
  \and \\
  \freenames{P|Q} := \freenames{P} \cup \freenames{Q}
  \and \\
  \freenames{@{x}} := \{ x \}
\end{mathpar}

$\pi$
$\quotep{\pi}$

$\freenames{-} : \pi \to \mathcal{P}(\quotep{\pi})$

\begin{eqnarray*}
  \freenames{\pzero} & := & \emptyset \\
  \freenames{x?(y).P} & := & \{ x \} \cup (\freenames{P} \setminus \{ y \}) \\
  \freenames{x!\langle P \rangle} & := & \{ x \} \cup \{ P \} \\
  \freenames{P|Q} & := & \freenames{P} \cup \freenames{Q} \\
  \freenames{\dropn{x}} & := & \{ x \}
\end{eqnarray*}

The bound names of a process, $\boundnames{P}$, are those names occurring in $P$
that are not free. For example, in $x?(y).0$, the name $x$ is free, while $y$ is bound.

\begin{mathpar}
  \inferrule* [lab=monoidal-laws] {} { P|Q \equiv Q|P \and P|0 \equiv P \and P|(Q|R) \equiv (P|Q)|R }
\end{mathpar}

\begin{mathpar}
  \inferrule* [lab=alpha-equivalence] {} { (x)P \equiv (y)P\{y/x\} \and y \not\in \freenames{P} }
\end{mathpar}

\begin{definition}
Then two processes, $P,Q$, are alpha-equivalent if $P = Q\{\vec{y}/\vec{x}\}$ for
some $\vec{x} \in \boundnames{Q},\vec{y} \in \boundnames{P}$, where $Q\{\vec{y}/\vec{x}\}$
denotes the capture-avoiding substitution of $\vec{y}$ for $\vec{x}$ in $Q$.
\end{definition}

\begin{definition}
  The {\em structural congruence} \cite{SangiorgiWalker} , $\equiv$,
  between processes is the least congruence containing
  alpha-equivalence, satisfying the abelian monoid laws
  (associativity, commutativity and $\pzero$ as identity) for parallel
  composition $|$ and for summation $+$.
\end{definition}

\subsection{Name equivalence}

We take name equivalence, written $\nameeq$, to be the smallest
equivalence relation generated by the following rules.

\begin{mathpar}
\inferrule*[lab=Quote-drop]
{ }
{ \quotep{@{x}} \nameeq x }

\inferrule*[lab=Struct-equiv]
{ P \scong Q }
{ \quotep{P} \nameeq \quotep{Q} }
\end{mathpar}

The astute reader will have noticed that the mutual recursion of names
and processes imposes a mutual recursion on alpha-equivalence and
structural equivalence via name-equivalence. Fortunately, all of this
works out pleasantly and we may calculate in the natural way, free of
concern. The reader interested in the details is referred to the
appendix \ref{appendix:rho_details}.

\subsection{Substitution}

We use $\Proc$ for the set of processes, $\QProc$ for the set of
names, and $\id{\{}\vec{y} / \vec{x} \id{\}}$ to denote partial maps,
$s : \QProc \rightarrow \QProc$. A map, $s$ lifts, uniquely, to a map
on process terms, $\widehat{s} : \Proc \rightarrow \Proc$ by the
following equations.

\begin{mathpar}
  (0) \psubstp{Q}{P} := 0 \\
  (R \juxtap S) \psubstp{Q}{P}
  :=    
  (R)\psubstp{Q}{P} \juxtap (S) \psubstp{Q}{P} \\
  (x?(y).R) \psubstp{Q}{P}    
  :=    
  (x)\substp{Q}{P} (z)\concat( (R \psubstn{z}{y}) \psubstp{Q}{P} ) \\
  (\lift{x}{R}) \psubstp{Q}{P}  
  :=
  \lift{(x)\substp{Q}{P}}{ R \psubstp{Q}{P} } \\
%   (\dropn{x})  \psubstp{Q}{P}       
%   := 
%   \left\{ 
%     \begin{array}{ccc} 
%       \dropn{\quotep{Q}} & & x \nameeq \quotep{P} \\
%       \dropn{x} & & otherwise \\
%     \end{array}
%   \right. 
  (\dropn{x})  \psubstp{Q}{P}       
  := 
  \left\{ 
    \begin{array}{ccc} 
      Q & & x \nameeq \quotep{P} \\
      \dropn{x} & & otherwise \\
    \end{array}
  \right.
\end{mathpar}
 

where

\begin{eqnarray}
  (x)\id{\{} \lpquote Q \rpquote / \lpquote P \rpquote \id{\}}            = 
  \left\{ 
    \begin{array}{ccc}
      \lpquote Q \rpquote & & x \nameeq \lpquote P \rpquote \\
      x & & otherwise \\
    \end{array}
  \right. \nonumber
\end{eqnarray}

and $z$ is chosen distinct from $\quotep{P}$, $\quotep{Q}$, the free
names in $Q$, and all the names in $R$. Our $\alpha$-equivalence will
be built in the standard way from this substitution.

\begin{remark}\label{rem:no_self_referential_names}
  One consequence of these definitions is that $\forall P. \quotep{P}
  \not\in \freenames{P}$.
\end{remark}

\subsection{ Dynamic quote: an example }

Anticipating something of what's to come, consider applying the
substitution, $\widehat{\id{\{}u / z \id{\}}}$, to the following pair
of processes, $\lift{w}{y!(z)}$ and $w[ \lpquote y!(z) \rpquote ]$.

\begin{eqnarray}
	\lift{w}{y!(z)}\widehat{\id{\{}u / z \id{\}}}
		& = &
		\lift{w}{y!(u)} \nonumber\\
	w[ \lpquote y!(z) \rpquote ] \widehat{ \id{\{}u / z \id{\}} }
		& = &
		w[ \lpquote y!(z) \rpquote ] \nonumber
\end{eqnarray}

Because the body of the process between quotes is impervious to
substitution, we get radically different answers. In fact, by
examining the first process in an input context,
e.g. $x?(z).\lift{w}{y!(z)}$, we see that the process under the lift
operator may be shaped by prefixed inputs binding a name inside it. In
this sense, the lift operator will be seen as a way to dynamically
construct processes before reifying them as names.

Finally equipped with these standard features we can present the
dynamics of the calculus.

\subsubsection{Operational semantics} 

Finally, we introduce the computational dynamics. What marks these
algebras as distinct from other more traditionally studied algebraic
structures, e.g. vector spaces or polynomial rings, is the manner in
which dynamics is captured. In traditional structures, dynamics is typically
expressed through morphisms between such structures, as in linear maps
between vector spaces or morphisms between rings. In algebras
associated with the semantics of computation, the dynamics is
expressed as part of the algebraic structure itself, through a
reduction reduction relation typically denoted by $\red$. Below, we
give a recursive presentation of this relation for the calculus used
in the encoding.

$\red \subseteq \pi \times \pi$
$\red : \pi \to \mathcal{P}(\pi)$

\begin{mathpar}
  \inferrule* [lab=Comm] { \textsf{match}( x_{src}, x_{trgt} ) } { x_{trgt}?(y)P \; | \; x_{src}!\langle {Q} \rangle \red P\{\quotep{Q}/y}\} }
  \and \\
  \inferrule* [lab=Par] {{P} \red {P}'} {{{P} | {Q}} \red {{P}' | {Q}}}
  \and
  \inferrule* [lab=Equiv]{{{P} \scong {P}'} \andalso {{P}' \red {Q}'} \andalso {{Q}' \scong {Q}}}{{P} \red {Q}}
\end{mathpar}

\begin{eqnarray*}
  match_{\equiv} (\quotep{P},\quotep{Q}) & := & P \equiv Q \\
  match_{\dagger}(\quotep{P},\quotep{Q}) & := & \forall R. P|Q \red^{*} R => R \red^{*} 0 \\
  match_{K}(\quotep{P},\quotep{Q}) & := & K \mbox{ for some context } K
\end{eqnarray*}

$u?(x)P | u!\langle Q \rangle \red P\{\quotep{Q}/x\}$

%We write $\wred$ for $\red^*$, and $P\red$ if $\exists Q $ such that $ P \red Q$.
We write $P\red$ if $\exists Q $ such that $ P \red Q$ and $P\not\red$, otherwise.

\section{Replication}

As mentioned before, it is known that replication (and hence
recursion) can be implemented in a higher-order process algebra
\cite{SangiorgiWalker}. As our first example of calculation with the
machinery thus far presented we give the construction explicitly in
the {\rhoc}.

\begin{eqnarray}
	D_{x} & := & \prefix{x}{y}{(\binpar{\outputp{x}{y}}{@{y}})} \nonumber\\
	\bangp_{x}{P} & := & \binpar{{x}!\langle{\binpar{D_{x}}{P}}\rangle}{D_{x}} \nonumber
\end{eqnarray}

\begin{eqnarray}
	\bangp_{x}{P} & & \nonumber\\
	=
	& {x}!\langle{(\prefix{x}{y}{(\outputp{x}{y} | @{y})) | P}}\rangle 
	      | \prefix{x}{y}{(\outputp{x}{y} | @{y})} & \nonumber\\
	\red
	& (\outputp{x}{y} | @{y})\substn{\quotep{(\prefix{x}{y}{(@{y} | \outputp{x}{y})) | P}}}{y} & \nonumber\\
	=
	& \outputp{x}{\quotep{(\prefix{x}{y}{(\outputp{x}{y} | @{y})) | P}}}
	  | {(\prefix{x}{y}{(\outputp{x}{y} | @{y})) | P}} & \nonumber\\
	\red
	& \ldots & \nonumber\\
	\red^*
	& P | P | \ldots & \nonumber
\end{eqnarray}

Of course, this encoding, as an implementation, runs away, unfolding
$\bangp{P}$ eagerly. A lazier and more implementable replication
operator, restricted to input-guarded processes, may be obtained as follows.

\begin{eqnarray}
\bangp{\prefix{u}{v}{P}} 
	:= 
	\binpar{\lift{x}{\prefix{u}{v}{(\binpar{D(x)}{P})}}}{D(x)} \nonumber
\end{eqnarray}

\begin{remark}
  Note that the lazier definition still does not deal with summation
  or mixed summation (i.e. sums over input and output). The reader is
  invited to construct definitions of replication that deal with these
  features. 

  Further, the definitions are parameterized in a name, $x$. Can you,
  gentle reader, make a definition that eliminates this parameter and
  guarantees no accidental interaction between the replication
  machinery and the process being replicated -- i.e. no accidental
  sharing of names used by the process to get its work done and the
  name(s) used by the replication to effect copying. This latter
  revision of the definition of replication is crucial to obtaining
  the expected identity $!!P \sim !P$.
\end{remark}

\begin{remark}\label{rem:paradoxical_combinator}
  The reader familiar with the lambda calculus will have noticed the
  similarity between $D$ and the paradoxical combinator.

  [Ed. note: the existence of this seems to suggest we have to be more
  restrictive on the set of processes and names we admit if we are to
  support no-cloning.]
\end{remark}

\subsubsection{Bisimulation}

The computational dynamics gives rise to another kind of equivalence,
the equivalence of computational behavior. As previously mentioned
this is typically captured \emph{via} some form of bisimulation.

% The notion we use in this paper is weak barbed bisimulation
% \cite{milner91polyadicpi}.

The notion we use in this paper is derived from weak barbed
bisimulation \cite{milner91polyadicpi}. 

\begin{definition}
An \emph{observation relation}, $\downarrow_{\mathcal N}$, over a set
of names, $\mathcal N$, is the smallest relation satisfying the rules
below.

\infrule[Out-barb]{y \in {\mathcal N}, \; x \nameeq y}
		  {\outputp{x}{v} \downarrow_{\mathcal N} x}
\infrule[Par-barb]{\mbox{$P\downarrow_{\mathcal N} x$ or $Q\downarrow_{\mathcal N} x$}}
		  {\binpar{P}{Q} \downarrow_{\mathcal N} x}

We write $P \Downarrow_{\mathcal N} x$ if there is $Q$ such that 
$P \wred Q$ and $Q \downarrow_{\mathcal N} x$.
\end{definition}

\begin{definition}
%\label{def.bbisim}
An  ${\mathcal N}$-\emph{barbed bisimulation} over a set of names, ${\mathcal N}$, is a symmetric binary relation 
${\mathcal S}_{\mathcal N}$ between agents such that $P\rel{S}_{\mathcal N}Q$ implies:
\begin{enumerate}
\item If $P \red P'$ then $Q \wred Q'$ and $P'\rel{S}_{\mathcal N} Q'$.
\item If $P\downarrow_{\mathcal N} x$, then $Q\Downarrow_{\mathcal N} x$.
\end{enumerate}
$P$ is ${\mathcal N}$-barbed bisimilar to $Q$, written
$P \wbbisim_{\mathcal N} Q$, if $P \rel{S}_{\mathcal N} Q$ for some ${\mathcal N}$-barbed bisimulation ${\mathcal S}_{\mathcal N}$.
\end{definition}

$\mathcal{R} \subseteq \pi \times \pi$

$P \mathcal{R} Q => \forall P'. P \red P' \Rightarrow \exists Q'. Q \red Q', P' \mathcal{R} Q'$

$P \vdash x \Rightarrow Q \vdash x$

\begin{mathpar}
  \inferrule*[lab=Out-barb]{x \nameeq y}{{y}!\langle{Q}\rangle \vdash x}
  \and
  \inferrule*[lab=Par-barb]{\mbox{$P\vdash x$ or $Q\vdash x$}}{\binpar{P}{Q} \vdash x}
\end{mathpar}

\subsubsection{Contexts}

One of the principle advantages of computational calculi like the
$\pi$-calculus is a well-defined notion of context,
contextual-equivalence and a correlation between
contextual-equivalence and notions of bisimulation. The notion of
context allows the decomposition of a process into (sub-)process and
its syntactic environment, its context. Thus, a context may be
thought of as a process with a ``hole'' (written $\Box$) in it. The
application of a context $M$ to a process $P$, written $M[P]$, is
tantamount to filling the hole in $M$ with $P$. In this paper we do
not need the full weight of this theory, but do make use of the notion
of context in the proof the main theorem. 

\begin{mathpar}
  \inferrule* [lab=summation] {} {{M_{M},M_{N}} \bc \Box \;|\; x.M_{A} \;|\; M_{M}+M_{N}}
  \and
  \inferrule* [lab=agent] {} {{M_{A}} \bc (\vec{x})M_{P} \;| \; \clift{P_0,\ldots,M_{P},\ldots,P_N}}
  \and \\
  \inferrule* [lab=process] {} {{M_{P}} \bc M_{N} \;| \;P|M_{P} }
\end{mathpar} 

\begin{mathpar}
  \inferrule* [lab=sychronization] {} {M_{N} \bc \Box \;|\; x?M_{F} \;|\; x!M_{C}}
  \and
  \inferrule* [lab=abstraction] {} {{M_{F}} \bc (x)M_{P} }
  \and
  \inferrule* [lab=concretion] {} {{M_{C}} \bc \langle M_{P} \rangle }
  \and \\
  \inferrule* [lab=process] {} {{M_{P}} \bc M_{N} \;| \;P|M_{P} }
\end{mathpar}

\begin{definition}[contextual application] Given a context $M$, and
  process $P$, we define the \emph{contextual application}, $M[P] :=
  M\{P/\Box\}$. That is, the contextual application of M to P is the
  substitution of $P$ for $\Box$ in $M$.
\end{definition}

$\meaningof{-} : L \to \mathcal{P}(\pi)$

\begin{mathpar}
  \inferrule* [lab=collection] {} {\meaningof{true} = \pi, \and \meaningof{~E} = \pi \setminus \meaningof{E}, \and \meaningof{E_{1} \& E_{2}} = \meaningof{E_{1}} \cap \meaningof{E_{2}}}
\end{mathpar}

\begin{mathpar}
  \inferrule* [lab=structure] {} {\meaningof{0} = \{ P \in \pi | P \equiv 0 \}, \and \\ \meaningof{E_1 | E_2} = \{ P \in \pi | P \equiv P_{1} | P_{2}, P_{1} \in \meaningof{E_{1}}, P_{2} \in \meaningof{E_2}\} }
\end{mathpar}

\begin{mathpar}
 \inferrule* [lab=behavior] {} {\meaningof{\langle a?b \rangle E} = \{ P \in \pi | P \equiv Q | u?(y)P', \\ \and \\\\ \and \\ \;\;\; u \in \meaningof{a}, \forall z.P'\{z/y\} \in \meaningof{E\{z/b\}}\}, \and \\ \meaningof{a!E} = \{ P \in \pi | P \equiv Q | x!\langle P' \rangle, x \in \meaningof{a} P' \in \meaningof{E}\} }
\end{mathpar}

\begin{mathpar}
 \inferrule* [lab=nominal] {} {\meaningof{\quotep{E}} = \{ \quotep{P} \in \quotep{\pi} | P \in \meaningof{E} \}, \and \meaningof{\quotep{P}} = \{ \quotep{Q} \in \quotep{\pi} | P \equiv Q \} \and \\ \meaningof{@\quotep{E}} = \{ P \in \pi | P \equiv @x, x \in \meaningof{E} \}}
\end{mathpar}

\begin{eqnarray*}
  \\
  \meaningof{-} : TS \to ST
\end{eqnarray*}

\begin{eqnarray*}
  \\
  L : TS \to ST
\end{eqnarray*}

\begin{eqnarray*}
  \\
  P \models E \iff P \in \meaningof{E}
\end{eqnarray*}

\begin{eqnarray*}
  P \approx_{L} Q \iff \forall E \in L. P \models E \iff Q \models E
\end{eqnarray*}

\begin{eqnarray*}
  P \approx_{K} Q
\end{eqnarray*}

\begin{eqnarray*}
  P \approx Q
\end{eqnarray*}

$\approx_{K} = \approx = \approx_{L}$

\subsubsection{Contextual duality}

Note that contexts extend the quotation operation to a family of
operations from processes to names. Given a context, $M$, we can
define a \emph{nominal context}, $\quotep{M}$ by $\quotep{M}[P] :=
\quotep{M[P]}$. To foreshadow what is to come we observe that these
operations enjoy a duality with processes very much like the duality
between vectors and maps from vectors to scalars.

Further, because the calculus is essentially higher-order, we have a
correspondence between contexts and processes. More specifically,
given a name $x$ and a context $M$ we can construct $M^{*}_{x}$ such
that 

\begin{mathpar}
  M^{*}_{x} | \lift{x}{P} \red M[P]
\end{mathpar}

namely,

\begin{mathpar}
  M^{*}_{x} := x?(u).M[\dropn{u}]
\end{mathpar}

The dependence of $M^{*}_{x}$ on a name makes it an abstraction, 

\begin{mathpar}
  M^{*} := (x)x?(u).M[\dropn{u}]
\end{mathpar}

\subsection{Additional notation}

It will sometimes be convenient to denote the process a name
quotes. We already have the notation $x = \quotep{P}$, but it will be
convenient to introduce an alternate notation, $\procn{x}$, when we
want to emphasize the connection to the use of the name. Note that, by
virtue of name equivalence, $\quotep{\procn{x}} \nameeq x$; so, the
notation is consistent with previous definitions.

Further, because names have structure it is possible to effect
substitutions on the basis of that structure. This means we need to
upgrade our notation for substitutions, which we accomplish by
adapting comprehension notation. Thus,

\begin{mathpar}
  P\{ y / x : x \in S \}
\end{mathpar}

is interpreted to mean the process derived from P by replacing (in a
capture-avoiding manner) each occurrence of $x$ in $S$ by $y$. For example,

\begin{mathpar}
  P\{ \quotep{\procn{x}|\procn{x}} / x : x \in \freenames{P} \}
\end{mathpar}

will replace each (occurrence) of a free name $x$ in $P$ by
$\quotep{\procn{x}|\procn{x}}$.

Also, we will avail ourselves of the notation $x^{L}$ and $x^{R}$ to
denote injections of a name into disjoint copies of the name
space. There are numerous ways to accomplish this. One example can be
found in \cite{MeredithR05}. This notation overloads to vectors of
names: $\vec{x}^{\pi} := (x_{i}^{\pi} \; : \; 0 \leq i < |\vec{x}| )$ where $\pi \in \{L,R\}$.

We also use $P^{\Box} := P|\Box$.

In \cite{MeredithR05} an interpretation of the new operator is
given. It turns out that there are several possible interpretations
all enjoying the requisite algebraic properties of the operator (see
\cite{milner91polyadicpi}). We will therefore make liberal use of
$(\nu\; \vec{x})P$.

% subsection the_syntax_and_semantics_of_the_notation_system (end)   

\input{qm2pi.qmops} 

\input{qm2pi.sterngerlach} 

\input{qm2pi.metric} 

% section concurrent_process_calculi (end)

%\input{qm2pi.proofsketch}

% section proof sketch (end)

%\input{qm2pi.slviaknots} 

% section spatial logic via knots (end)

\input{qm2pi.conclusion}

% section conclusion (end)

%\input{qm2pi.dtcodes} 

% section wiring algorithm (end)

\input{qm2pi.ack} 

% section acknowledgments (end)

\newpage


\bibliographystyle{plain}   
\bibliography{../../biblios/main.bib}

\input{qm2pi.rhodetails}

\end{document}



\end{document}

 

%\documentclass[12pt]{llncs}
%\documentclass{jktr}

\usepackage[pdftex]{hyperref}                   
\usepackage {listings}
\usepackage {mathpartir}
\usepackage{bcprules}
%\usepackage{listings}
                       
\usepackage{graphicx} 
%\usepackage[margins=2.5cm,nohead,nofoot]{geometry}
%\usepackage{geometry}
\usepackage{amsfonts}
\usepackage{amstext}
\usepackage{latexsym}
\usepackage{amssymb}
\usepackage{color}


%\include{myPreamble}
\documentclass[12pt]{llncs}
%\documentclass{jktr}

\usepackage[pdftex]{hyperref}                   
\usepackage {listings}
\usepackage {mathpartir}
\usepackage{bcprules}
%\usepackage{listings}
                       
\usepackage{graphicx} 
%\usepackage[margins=2.5cm,nohead,nofoot]{geometry}
%\usepackage{geometry}
\usepackage{amsfonts}
\usepackage{amstext}
\usepackage{latexsym}
\usepackage{amssymb}
\usepackage{color}


%\include{myPreamble}
\include{qm2pi.local} 

%\ifpdf
%\usepackage[pdftex]{graphicx}
%\else
%\usepackage{graphicx}
%\fi

 % \ifpdf
%  \usepackage{pdfsync}
%  \if


%\title{Brief Article}
%\author{David F. Snyder}
%\author{L.G. Meredith}

%\address{Dept. of Math., Texas State University--San Marcos, San Marcos, TX 78666}
       
\pagestyle{empty}


\begin{document}

\lstset{language=[Objective]Caml,frame=shadowbox}

\input{qm2pi.front}

% section front matter (end)

\input{qm2pi.intro} 
 
% section introduction (end)

% \input{qm2pi.knotations} 

% section notation (end)

\input{qm2pi.process.calculi} 

% section concurrent_process_calculi_and_spatial_logics_ (end)
    
%\input{qm2pi.knots2pi} 

%\input{qm2pi.trefoil} 

%\input{qm2pi.mainthm} 

% subsection basic_interpretation (end)

%\input{qm2pi.rho.presentation} 
\subsection{The syntax and semantics of the notation system}\label{sub:the_syntax_and_semantics_of_the_notation_system} % (fold)

We now summarize a technical presentation of the calculus that
embodies our theory of dynamics. The typical presentation of such a
calculus follows the style of giving generators and relations on
them. The grammar, below, describing term constructors, freely
generates the set of processes, $\Proc$. This set is then quotiented
by a relation known as structural congruence and it is over this set
that the notion of dynamics is expressed. This presentation is
essentially that of \cite{MeredithR05} with the addition of
polyadicity and summation. For readability we have relegated some of
the technical subtleties to an appendix.

\subsubsection{Process grammar}\label{subsub:process_grammar}

\begin{mathpar}
  \inferrule* [lab=synchronization] {} {{M} \bc \pzero \;|\; x?F \;|\; x!C }
  \and
  \inferrule* [lab=abstraction] {} {{F} \bc (x)P}
  \and
  \inferrule* [lab=concretion] {} {{C} \bc \langle Q \rangle}
  \and
  \inferrule* [lab=process] {} {{P,Q} \bc M \;| \;P|Q \;|\; @{x}}
  \and
  \inferrule* [lab=name] {} {{x} \bc \quotep{P}}
\end{mathpar} 

Note that $\vec{x}$ (resp. $\vec{P}$) denotes a vector of names
(resp. processes) of length $|\vec{x}|$ (resp. $|\vec{P}|$). We adopt
the following useful abbreviations.

\begin{mathpar}
   x?(\vec{y}).P := x.(\vec{y})P \and  x\clift{\vec{P}} := x.\clift{\vec{P}}
   \and x!(y) := \lift{x}{\dropn{y}}
   \and \Pi_{i=0}^{n-1}P_i := P_0 | \ldots | P_{n-1}
\end{mathpar}

\subsubsection{Structural congruence}

\paragraph{Free and bound names and alpha-equivalence.} At the
core of structural equivalence is alpha-equivalence which identifies
process that are the same up to a change of variable. Formally, we
recognize the distinction between free and bound names. The free names
of a process, $\freenames{P}$, may be calculated recursively as
follows:

\begin{mathpar}
\freenames{\pzero} := \emptyset
  \and \\
  \freenames{x?(y).P} := \{ x \} \cup (\freenames{P} \setminus \{ y \})
  \and 
  \freenames{x!\langle P \rangle} := \{ x \} \cup \{ P \} 
  \and \\
  \freenames{P|Q} := \freenames{P} \cup \freenames{Q}
  \and \\
  \freenames{@{x}} := \{ x \}
\end{mathpar}

$\pi$
$\quotep{\pi}$

$\freenames{-} : \pi \to \mathcal{P}(\quotep{\pi})$

\begin{eqnarray*}
  \freenames{\pzero} & := & \emptyset \\
  \freenames{x?(y).P} & := & \{ x \} \cup (\freenames{P} \setminus \{ y \}) \\
  \freenames{x!\langle P \rangle} & := & \{ x \} \cup \{ P \} \\
  \freenames{P|Q} & := & \freenames{P} \cup \freenames{Q} \\
  \freenames{\dropn{x}} & := & \{ x \}
\end{eqnarray*}

The bound names of a process, $\boundnames{P}$, are those names occurring in $P$
that are not free. For example, in $x?(y).0$, the name $x$ is free, while $y$ is bound.

\begin{mathpar}
  \inferrule* [lab=monoidal-laws] {} { P|Q \equiv Q|P \and P|0 \equiv P \and P|(Q|R) \equiv (P|Q)|R }
\end{mathpar}

\begin{mathpar}
  \inferrule* [lab=alpha-equivalence] {} { (x)P \equiv (y)P\{y/x\} \and y \not\in \freenames{P} }
\end{mathpar}

\begin{definition}
Then two processes, $P,Q$, are alpha-equivalent if $P = Q\{\vec{y}/\vec{x}\}$ for
some $\vec{x} \in \boundnames{Q},\vec{y} \in \boundnames{P}$, where $Q\{\vec{y}/\vec{x}\}$
denotes the capture-avoiding substitution of $\vec{y}$ for $\vec{x}$ in $Q$.
\end{definition}

\begin{definition}
  The {\em structural congruence} \cite{SangiorgiWalker} , $\equiv$,
  between processes is the least congruence containing
  alpha-equivalence, satisfying the abelian monoid laws
  (associativity, commutativity and $\pzero$ as identity) for parallel
  composition $|$ and for summation $+$.
\end{definition}

\subsection{Name equivalence}

We take name equivalence, written $\nameeq$, to be the smallest
equivalence relation generated by the following rules.

\begin{mathpar}
\inferrule*[lab=Quote-drop]
{ }
{ \quotep{@{x}} \nameeq x }

\inferrule*[lab=Struct-equiv]
{ P \scong Q }
{ \quotep{P} \nameeq \quotep{Q} }
\end{mathpar}

The astute reader will have noticed that the mutual recursion of names
and processes imposes a mutual recursion on alpha-equivalence and
structural equivalence via name-equivalence. Fortunately, all of this
works out pleasantly and we may calculate in the natural way, free of
concern. The reader interested in the details is referred to the
appendix \ref{appendix:rho_details}.

\subsection{Substitution}

We use $\Proc$ for the set of processes, $\QProc$ for the set of
names, and $\id{\{}\vec{y} / \vec{x} \id{\}}$ to denote partial maps,
$s : \QProc \rightarrow \QProc$. A map, $s$ lifts, uniquely, to a map
on process terms, $\widehat{s} : \Proc \rightarrow \Proc$ by the
following equations.

\begin{mathpar}
  (0) \psubstp{Q}{P} := 0 \\
  (R \juxtap S) \psubstp{Q}{P}
  :=    
  (R)\psubstp{Q}{P} \juxtap (S) \psubstp{Q}{P} \\
  (x?(y).R) \psubstp{Q}{P}    
  :=    
  (x)\substp{Q}{P} (z)\concat( (R \psubstn{z}{y}) \psubstp{Q}{P} ) \\
  (\lift{x}{R}) \psubstp{Q}{P}  
  :=
  \lift{(x)\substp{Q}{P}}{ R \psubstp{Q}{P} } \\
%   (\dropn{x})  \psubstp{Q}{P}       
%   := 
%   \left\{ 
%     \begin{array}{ccc} 
%       \dropn{\quotep{Q}} & & x \nameeq \quotep{P} \\
%       \dropn{x} & & otherwise \\
%     \end{array}
%   \right. 
  (\dropn{x})  \psubstp{Q}{P}       
  := 
  \left\{ 
    \begin{array}{ccc} 
      Q & & x \nameeq \quotep{P} \\
      \dropn{x} & & otherwise \\
    \end{array}
  \right.
\end{mathpar}
 

where

\begin{eqnarray}
  (x)\id{\{} \lpquote Q \rpquote / \lpquote P \rpquote \id{\}}            = 
  \left\{ 
    \begin{array}{ccc}
      \lpquote Q \rpquote & & x \nameeq \lpquote P \rpquote \\
      x & & otherwise \\
    \end{array}
  \right. \nonumber
\end{eqnarray}

and $z$ is chosen distinct from $\quotep{P}$, $\quotep{Q}$, the free
names in $Q$, and all the names in $R$. Our $\alpha$-equivalence will
be built in the standard way from this substitution.

\begin{remark}\label{rem:no_self_referential_names}
  One consequence of these definitions is that $\forall P. \quotep{P}
  \not\in \freenames{P}$.
\end{remark}

\subsection{ Dynamic quote: an example }

Anticipating something of what's to come, consider applying the
substitution, $\widehat{\id{\{}u / z \id{\}}}$, to the following pair
of processes, $\lift{w}{y!(z)}$ and $w[ \lpquote y!(z) \rpquote ]$.

\begin{eqnarray}
	\lift{w}{y!(z)}\widehat{\id{\{}u / z \id{\}}}
		& = &
		\lift{w}{y!(u)} \nonumber\\
	w[ \lpquote y!(z) \rpquote ] \widehat{ \id{\{}u / z \id{\}} }
		& = &
		w[ \lpquote y!(z) \rpquote ] \nonumber
\end{eqnarray}

Because the body of the process between quotes is impervious to
substitution, we get radically different answers. In fact, by
examining the first process in an input context,
e.g. $x?(z).\lift{w}{y!(z)}$, we see that the process under the lift
operator may be shaped by prefixed inputs binding a name inside it. In
this sense, the lift operator will be seen as a way to dynamically
construct processes before reifying them as names.

Finally equipped with these standard features we can present the
dynamics of the calculus.

\subsubsection{Operational semantics} 

Finally, we introduce the computational dynamics. What marks these
algebras as distinct from other more traditionally studied algebraic
structures, e.g. vector spaces or polynomial rings, is the manner in
which dynamics is captured. In traditional structures, dynamics is typically
expressed through morphisms between such structures, as in linear maps
between vector spaces or morphisms between rings. In algebras
associated with the semantics of computation, the dynamics is
expressed as part of the algebraic structure itself, through a
reduction reduction relation typically denoted by $\red$. Below, we
give a recursive presentation of this relation for the calculus used
in the encoding.

$\red \subseteq \pi \times \pi$
$\red : \pi \to \mathcal{P}(\pi)$

\begin{mathpar}
  \inferrule* [lab=Comm] { \textsf{match}( x_{src}, x_{trgt} ) } { x_{trgt}?(y)P \; | \; x_{src}!\langle {Q} \rangle \red P\{\quotep{Q}/y}\} }
  \and \\
  \inferrule* [lab=Par] {{P} \red {P}'} {{{P} | {Q}} \red {{P}' | {Q}}}
  \and
  \inferrule* [lab=Equiv]{{{P} \scong {P}'} \andalso {{P}' \red {Q}'} \andalso {{Q}' \scong {Q}}}{{P} \red {Q}}
\end{mathpar}

\begin{eqnarray*}
  match_{\equiv} (\quotep{P},\quotep{Q}) & := & P \equiv Q \\
  match_{\dagger}(\quotep{P},\quotep{Q}) & := & \forall R. P|Q \red^{*} R => R \red^{*} 0 \\
  match_{K}(\quotep{P},\quotep{Q}) & := & K \mbox{ for some context } K
\end{eqnarray*}

$u?(x)P | u!\langle Q \rangle \red P\{\quotep{Q}/x\}$

%We write $\wred$ for $\red^*$, and $P\red$ if $\exists Q $ such that $ P \red Q$.
We write $P\red$ if $\exists Q $ such that $ P \red Q$ and $P\not\red$, otherwise.

\section{Replication}

As mentioned before, it is known that replication (and hence
recursion) can be implemented in a higher-order process algebra
\cite{SangiorgiWalker}. As our first example of calculation with the
machinery thus far presented we give the construction explicitly in
the {\rhoc}.

\begin{eqnarray}
	D_{x} & := & \prefix{x}{y}{(\binpar{\outputp{x}{y}}{@{y}})} \nonumber\\
	\bangp_{x}{P} & := & \binpar{{x}!\langle{\binpar{D_{x}}{P}}\rangle}{D_{x}} \nonumber
\end{eqnarray}

\begin{eqnarray}
	\bangp_{x}{P} & & \nonumber\\
	=
	& {x}!\langle{(\prefix{x}{y}{(\outputp{x}{y} | @{y})) | P}}\rangle 
	      | \prefix{x}{y}{(\outputp{x}{y} | @{y})} & \nonumber\\
	\red
	& (\outputp{x}{y} | @{y})\substn{\quotep{(\prefix{x}{y}{(@{y} | \outputp{x}{y})) | P}}}{y} & \nonumber\\
	=
	& \outputp{x}{\quotep{(\prefix{x}{y}{(\outputp{x}{y} | @{y})) | P}}}
	  | {(\prefix{x}{y}{(\outputp{x}{y} | @{y})) | P}} & \nonumber\\
	\red
	& \ldots & \nonumber\\
	\red^*
	& P | P | \ldots & \nonumber
\end{eqnarray}

Of course, this encoding, as an implementation, runs away, unfolding
$\bangp{P}$ eagerly. A lazier and more implementable replication
operator, restricted to input-guarded processes, may be obtained as follows.

\begin{eqnarray}
\bangp{\prefix{u}{v}{P}} 
	:= 
	\binpar{\lift{x}{\prefix{u}{v}{(\binpar{D(x)}{P})}}}{D(x)} \nonumber
\end{eqnarray}

\begin{remark}
  Note that the lazier definition still does not deal with summation
  or mixed summation (i.e. sums over input and output). The reader is
  invited to construct definitions of replication that deal with these
  features. 

  Further, the definitions are parameterized in a name, $x$. Can you,
  gentle reader, make a definition that eliminates this parameter and
  guarantees no accidental interaction between the replication
  machinery and the process being replicated -- i.e. no accidental
  sharing of names used by the process to get its work done and the
  name(s) used by the replication to effect copying. This latter
  revision of the definition of replication is crucial to obtaining
  the expected identity $!!P \sim !P$.
\end{remark}

\begin{remark}\label{rem:paradoxical_combinator}
  The reader familiar with the lambda calculus will have noticed the
  similarity between $D$ and the paradoxical combinator.

  [Ed. note: the existence of this seems to suggest we have to be more
  restrictive on the set of processes and names we admit if we are to
  support no-cloning.]
\end{remark}

\subsubsection{Bisimulation}

The computational dynamics gives rise to another kind of equivalence,
the equivalence of computational behavior. As previously mentioned
this is typically captured \emph{via} some form of bisimulation.

% The notion we use in this paper is weak barbed bisimulation
% \cite{milner91polyadicpi}.

The notion we use in this paper is derived from weak barbed
bisimulation \cite{milner91polyadicpi}. 

\begin{definition}
An \emph{observation relation}, $\downarrow_{\mathcal N}$, over a set
of names, $\mathcal N$, is the smallest relation satisfying the rules
below.

\infrule[Out-barb]{y \in {\mathcal N}, \; x \nameeq y}
		  {\outputp{x}{v} \downarrow_{\mathcal N} x}
\infrule[Par-barb]{\mbox{$P\downarrow_{\mathcal N} x$ or $Q\downarrow_{\mathcal N} x$}}
		  {\binpar{P}{Q} \downarrow_{\mathcal N} x}

We write $P \Downarrow_{\mathcal N} x$ if there is $Q$ such that 
$P \wred Q$ and $Q \downarrow_{\mathcal N} x$.
\end{definition}

\begin{definition}
%\label{def.bbisim}
An  ${\mathcal N}$-\emph{barbed bisimulation} over a set of names, ${\mathcal N}$, is a symmetric binary relation 
${\mathcal S}_{\mathcal N}$ between agents such that $P\rel{S}_{\mathcal N}Q$ implies:
\begin{enumerate}
\item If $P \red P'$ then $Q \wred Q'$ and $P'\rel{S}_{\mathcal N} Q'$.
\item If $P\downarrow_{\mathcal N} x$, then $Q\Downarrow_{\mathcal N} x$.
\end{enumerate}
$P$ is ${\mathcal N}$-barbed bisimilar to $Q$, written
$P \wbbisim_{\mathcal N} Q$, if $P \rel{S}_{\mathcal N} Q$ for some ${\mathcal N}$-barbed bisimulation ${\mathcal S}_{\mathcal N}$.
\end{definition}

$\mathcal{R} \subseteq \pi \times \pi$

$P \mathcal{R} Q => \forall P'. P \red P' \Rightarrow \exists Q'. Q \red Q', P' \mathcal{R} Q'$

$P \vdash x \Rightarrow Q \vdash x$

\begin{mathpar}
  \inferrule*[lab=Out-barb]{x \nameeq y}{{y}!\langle{Q}\rangle \vdash x}
  \and
  \inferrule*[lab=Par-barb]{\mbox{$P\vdash x$ or $Q\vdash x$}}{\binpar{P}{Q} \vdash x}
\end{mathpar}

\subsubsection{Contexts}

One of the principle advantages of computational calculi like the
$\pi$-calculus is a well-defined notion of context,
contextual-equivalence and a correlation between
contextual-equivalence and notions of bisimulation. The notion of
context allows the decomposition of a process into (sub-)process and
its syntactic environment, its context. Thus, a context may be
thought of as a process with a ``hole'' (written $\Box$) in it. The
application of a context $M$ to a process $P$, written $M[P]$, is
tantamount to filling the hole in $M$ with $P$. In this paper we do
not need the full weight of this theory, but do make use of the notion
of context in the proof the main theorem. 

\begin{mathpar}
  \inferrule* [lab=summation] {} {{M_{M},M_{N}} \bc \Box \;|\; x.M_{A} \;|\; M_{M}+M_{N}}
  \and
  \inferrule* [lab=agent] {} {{M_{A}} \bc (\vec{x})M_{P} \;| \; \clift{P_0,\ldots,M_{P},\ldots,P_N}}
  \and \\
  \inferrule* [lab=process] {} {{M_{P}} \bc M_{N} \;| \;P|M_{P} }
\end{mathpar} 

\begin{mathpar}
  \inferrule* [lab=sychronization] {} {M_{N} \bc \Box \;|\; x?M_{F} \;|\; x!M_{C}}
  \and
  \inferrule* [lab=abstraction] {} {{M_{F}} \bc (x)M_{P} }
  \and
  \inferrule* [lab=concretion] {} {{M_{C}} \bc \langle M_{P} \rangle }
  \and \\
  \inferrule* [lab=process] {} {{M_{P}} \bc M_{N} \;| \;P|M_{P} }
\end{mathpar}

\begin{definition}[contextual application] Given a context $M$, and
  process $P$, we define the \emph{contextual application}, $M[P] :=
  M\{P/\Box\}$. That is, the contextual application of M to P is the
  substitution of $P$ for $\Box$ in $M$.
\end{definition}

$\meaningof{-} : L \to \mathcal{P}(\pi)$

\begin{mathpar}
  \inferrule* [lab=collection] {} {\meaningof{true} = \pi, \and \meaningof{~E} = \pi \setminus \meaningof{E}, \and \meaningof{E_{1} \& E_{2}} = \meaningof{E_{1}} \cap \meaningof{E_{2}}}
\end{mathpar}

\begin{mathpar}
  \inferrule* [lab=structure] {} {\meaningof{0} = \{ P \in \pi | P \equiv 0 \}, \and \\ \meaningof{E_1 | E_2} = \{ P \in \pi | P \equiv P_{1} | P_{2}, P_{1} \in \meaningof{E_{1}}, P_{2} \in \meaningof{E_2}\} }
\end{mathpar}

\begin{mathpar}
 \inferrule* [lab=behavior] {} {\meaningof{\langle a?b \rangle E} = \{ P \in \pi | P \equiv Q | u?(y)P', \\ \and \\\\ \and \\ \;\;\; u \in \meaningof{a}, \forall z.P'\{z/y\} \in \meaningof{E\{z/b\}}\}, \and \\ \meaningof{a!E} = \{ P \in \pi | P \equiv Q | x!\langle P' \rangle, x \in \meaningof{a} P' \in \meaningof{E}\} }
\end{mathpar}

\begin{mathpar}
 \inferrule* [lab=nominal] {} {\meaningof{\quotep{E}} = \{ \quotep{P} \in \quotep{\pi} | P \in \meaningof{E} \}, \and \meaningof{\quotep{P}} = \{ \quotep{Q} \in \quotep{\pi} | P \equiv Q \} \and \\ \meaningof{@\quotep{E}} = \{ P \in \pi | P \equiv @x, x \in \meaningof{E} \}}
\end{mathpar}

\begin{eqnarray*}
  \\
  \meaningof{-} : TS \to ST
\end{eqnarray*}

\begin{eqnarray*}
  \\
  L : TS \to ST
\end{eqnarray*}

\begin{eqnarray*}
  \\
  P \models E \iff P \in \meaningof{E}
\end{eqnarray*}

\begin{eqnarray*}
  P \approx_{L} Q \iff \forall E \in L. P \models E \iff Q \models E
\end{eqnarray*}

\begin{eqnarray*}
  P \approx_{K} Q
\end{eqnarray*}

\begin{eqnarray*}
  P \approx Q
\end{eqnarray*}

$\approx_{K} = \approx = \approx_{L}$

\subsubsection{Contextual duality}

Note that contexts extend the quotation operation to a family of
operations from processes to names. Given a context, $M$, we can
define a \emph{nominal context}, $\quotep{M}$ by $\quotep{M}[P] :=
\quotep{M[P]}$. To foreshadow what is to come we observe that these
operations enjoy a duality with processes very much like the duality
between vectors and maps from vectors to scalars.

Further, because the calculus is essentially higher-order, we have a
correspondence between contexts and processes. More specifically,
given a name $x$ and a context $M$ we can construct $M^{*}_{x}$ such
that 

\begin{mathpar}
  M^{*}_{x} | \lift{x}{P} \red M[P]
\end{mathpar}

namely,

\begin{mathpar}
  M^{*}_{x} := x?(u).M[\dropn{u}]
\end{mathpar}

The dependence of $M^{*}_{x}$ on a name makes it an abstraction, 

\begin{mathpar}
  M^{*} := (x)x?(u).M[\dropn{u}]
\end{mathpar}

\subsection{Additional notation}

It will sometimes be convenient to denote the process a name
quotes. We already have the notation $x = \quotep{P}$, but it will be
convenient to introduce an alternate notation, $\procn{x}$, when we
want to emphasize the connection to the use of the name. Note that, by
virtue of name equivalence, $\quotep{\procn{x}} \nameeq x$; so, the
notation is consistent with previous definitions.

Further, because names have structure it is possible to effect
substitutions on the basis of that structure. This means we need to
upgrade our notation for substitutions, which we accomplish by
adapting comprehension notation. Thus,

\begin{mathpar}
  P\{ y / x : x \in S \}
\end{mathpar}

is interpreted to mean the process derived from P by replacing (in a
capture-avoiding manner) each occurrence of $x$ in $S$ by $y$. For example,

\begin{mathpar}
  P\{ \quotep{\procn{x}|\procn{x}} / x : x \in \freenames{P} \}
\end{mathpar}

will replace each (occurrence) of a free name $x$ in $P$ by
$\quotep{\procn{x}|\procn{x}}$.

Also, we will avail ourselves of the notation $x^{L}$ and $x^{R}$ to
denote injections of a name into disjoint copies of the name
space. There are numerous ways to accomplish this. One example can be
found in \cite{MeredithR05}. This notation overloads to vectors of
names: $\vec{x}^{\pi} := (x_{i}^{\pi} \; : \; 0 \leq i < |\vec{x}| )$ where $\pi \in \{L,R\}$.

We also use $P^{\Box} := P|\Box$.

In \cite{MeredithR05} an interpretation of the new operator is
given. It turns out that there are several possible interpretations
all enjoying the requisite algebraic properties of the operator (see
\cite{milner91polyadicpi}). We will therefore make liberal use of
$(\nu\; \vec{x})P$.

% subsection the_syntax_and_semantics_of_the_notation_system (end)   

\input{qm2pi.qmops} 

\input{qm2pi.sterngerlach} 

\input{qm2pi.metric} 

% section concurrent_process_calculi (end)

%\input{qm2pi.proofsketch}

% section proof sketch (end)

%\input{qm2pi.slviaknots} 

% section spatial logic via knots (end)

\input{qm2pi.conclusion}

% section conclusion (end)

%\input{qm2pi.dtcodes} 

% section wiring algorithm (end)

\input{qm2pi.ack} 

% section acknowledgments (end)

\newpage


\bibliographystyle{plain}   
\bibliography{../../biblios/main.bib}

\input{qm2pi.rhodetails}

\end{document}

 

%\ifpdf
%\usepackage[pdftex]{graphicx}
%\else
%\usepackage{graphicx}
%\fi

 % \ifpdf
%  \usepackage{pdfsync}
%  \if


%\title{Brief Article}
%\author{David F. Snyder}
%\author{L.G. Meredith}

%\address{Dept. of Math., Texas State University--San Marcos, San Marcos, TX 78666}
       
\pagestyle{empty}


\begin{document}

\lstset{language=[Objective]Caml,frame=shadowbox}

\documentclass[12pt]{llncs}
%\documentclass{jktr}

\usepackage[pdftex]{hyperref}                   
\usepackage {listings}
\usepackage {mathpartir}
\usepackage{bcprules}
%\usepackage{listings}
                       
\usepackage{graphicx} 
%\usepackage[margins=2.5cm,nohead,nofoot]{geometry}
%\usepackage{geometry}
\usepackage{amsfonts}
\usepackage{amstext}
\usepackage{latexsym}
\usepackage{amssymb}
\usepackage{color}


%\include{myPreamble}
\include{qm2pi.local} 

%\ifpdf
%\usepackage[pdftex]{graphicx}
%\else
%\usepackage{graphicx}
%\fi

 % \ifpdf
%  \usepackage{pdfsync}
%  \if


%\title{Brief Article}
%\author{David F. Snyder}
%\author{L.G. Meredith}

%\address{Dept. of Math., Texas State University--San Marcos, San Marcos, TX 78666}
       
\pagestyle{empty}


\begin{document}

\lstset{language=[Objective]Caml,frame=shadowbox}

\input{qm2pi.front}

% section front matter (end)

\input{qm2pi.intro} 
 
% section introduction (end)

% \input{qm2pi.knotations} 

% section notation (end)

\input{qm2pi.process.calculi} 

% section concurrent_process_calculi_and_spatial_logics_ (end)
    
%\input{qm2pi.knots2pi} 

%\input{qm2pi.trefoil} 

%\input{qm2pi.mainthm} 

% subsection basic_interpretation (end)

%\input{qm2pi.rho.presentation} 
\subsection{The syntax and semantics of the notation system}\label{sub:the_syntax_and_semantics_of_the_notation_system} % (fold)

We now summarize a technical presentation of the calculus that
embodies our theory of dynamics. The typical presentation of such a
calculus follows the style of giving generators and relations on
them. The grammar, below, describing term constructors, freely
generates the set of processes, $\Proc$. This set is then quotiented
by a relation known as structural congruence and it is over this set
that the notion of dynamics is expressed. This presentation is
essentially that of \cite{MeredithR05} with the addition of
polyadicity and summation. For readability we have relegated some of
the technical subtleties to an appendix.

\subsubsection{Process grammar}\label{subsub:process_grammar}

\begin{mathpar}
  \inferrule* [lab=synchronization] {} {{M} \bc \pzero \;|\; x?F \;|\; x!C }
  \and
  \inferrule* [lab=abstraction] {} {{F} \bc (x)P}
  \and
  \inferrule* [lab=concretion] {} {{C} \bc \langle Q \rangle}
  \and
  \inferrule* [lab=process] {} {{P,Q} \bc M \;| \;P|Q \;|\; @{x}}
  \and
  \inferrule* [lab=name] {} {{x} \bc \quotep{P}}
\end{mathpar} 

Note that $\vec{x}$ (resp. $\vec{P}$) denotes a vector of names
(resp. processes) of length $|\vec{x}|$ (resp. $|\vec{P}|$). We adopt
the following useful abbreviations.

\begin{mathpar}
   x?(\vec{y}).P := x.(\vec{y})P \and  x\clift{\vec{P}} := x.\clift{\vec{P}}
   \and x!(y) := \lift{x}{\dropn{y}}
   \and \Pi_{i=0}^{n-1}P_i := P_0 | \ldots | P_{n-1}
\end{mathpar}

\subsubsection{Structural congruence}

\paragraph{Free and bound names and alpha-equivalence.} At the
core of structural equivalence is alpha-equivalence which identifies
process that are the same up to a change of variable. Formally, we
recognize the distinction between free and bound names. The free names
of a process, $\freenames{P}$, may be calculated recursively as
follows:

\begin{mathpar}
\freenames{\pzero} := \emptyset
  \and \\
  \freenames{x?(y).P} := \{ x \} \cup (\freenames{P} \setminus \{ y \})
  \and 
  \freenames{x!\langle P \rangle} := \{ x \} \cup \{ P \} 
  \and \\
  \freenames{P|Q} := \freenames{P} \cup \freenames{Q}
  \and \\
  \freenames{@{x}} := \{ x \}
\end{mathpar}

$\pi$
$\quotep{\pi}$

$\freenames{-} : \pi \to \mathcal{P}(\quotep{\pi})$

\begin{eqnarray*}
  \freenames{\pzero} & := & \emptyset \\
  \freenames{x?(y).P} & := & \{ x \} \cup (\freenames{P} \setminus \{ y \}) \\
  \freenames{x!\langle P \rangle} & := & \{ x \} \cup \{ P \} \\
  \freenames{P|Q} & := & \freenames{P} \cup \freenames{Q} \\
  \freenames{\dropn{x}} & := & \{ x \}
\end{eqnarray*}

The bound names of a process, $\boundnames{P}$, are those names occurring in $P$
that are not free. For example, in $x?(y).0$, the name $x$ is free, while $y$ is bound.

\begin{mathpar}
  \inferrule* [lab=monoidal-laws] {} { P|Q \equiv Q|P \and P|0 \equiv P \and P|(Q|R) \equiv (P|Q)|R }
\end{mathpar}

\begin{mathpar}
  \inferrule* [lab=alpha-equivalence] {} { (x)P \equiv (y)P\{y/x\} \and y \not\in \freenames{P} }
\end{mathpar}

\begin{definition}
Then two processes, $P,Q$, are alpha-equivalent if $P = Q\{\vec{y}/\vec{x}\}$ for
some $\vec{x} \in \boundnames{Q},\vec{y} \in \boundnames{P}$, where $Q\{\vec{y}/\vec{x}\}$
denotes the capture-avoiding substitution of $\vec{y}$ for $\vec{x}$ in $Q$.
\end{definition}

\begin{definition}
  The {\em structural congruence} \cite{SangiorgiWalker} , $\equiv$,
  between processes is the least congruence containing
  alpha-equivalence, satisfying the abelian monoid laws
  (associativity, commutativity and $\pzero$ as identity) for parallel
  composition $|$ and for summation $+$.
\end{definition}

\subsection{Name equivalence}

We take name equivalence, written $\nameeq$, to be the smallest
equivalence relation generated by the following rules.

\begin{mathpar}
\inferrule*[lab=Quote-drop]
{ }
{ \quotep{@{x}} \nameeq x }

\inferrule*[lab=Struct-equiv]
{ P \scong Q }
{ \quotep{P} \nameeq \quotep{Q} }
\end{mathpar}

The astute reader will have noticed that the mutual recursion of names
and processes imposes a mutual recursion on alpha-equivalence and
structural equivalence via name-equivalence. Fortunately, all of this
works out pleasantly and we may calculate in the natural way, free of
concern. The reader interested in the details is referred to the
appendix \ref{appendix:rho_details}.

\subsection{Substitution}

We use $\Proc$ for the set of processes, $\QProc$ for the set of
names, and $\id{\{}\vec{y} / \vec{x} \id{\}}$ to denote partial maps,
$s : \QProc \rightarrow \QProc$. A map, $s$ lifts, uniquely, to a map
on process terms, $\widehat{s} : \Proc \rightarrow \Proc$ by the
following equations.

\begin{mathpar}
  (0) \psubstp{Q}{P} := 0 \\
  (R \juxtap S) \psubstp{Q}{P}
  :=    
  (R)\psubstp{Q}{P} \juxtap (S) \psubstp{Q}{P} \\
  (x?(y).R) \psubstp{Q}{P}    
  :=    
  (x)\substp{Q}{P} (z)\concat( (R \psubstn{z}{y}) \psubstp{Q}{P} ) \\
  (\lift{x}{R}) \psubstp{Q}{P}  
  :=
  \lift{(x)\substp{Q}{P}}{ R \psubstp{Q}{P} } \\
%   (\dropn{x})  \psubstp{Q}{P}       
%   := 
%   \left\{ 
%     \begin{array}{ccc} 
%       \dropn{\quotep{Q}} & & x \nameeq \quotep{P} \\
%       \dropn{x} & & otherwise \\
%     \end{array}
%   \right. 
  (\dropn{x})  \psubstp{Q}{P}       
  := 
  \left\{ 
    \begin{array}{ccc} 
      Q & & x \nameeq \quotep{P} \\
      \dropn{x} & & otherwise \\
    \end{array}
  \right.
\end{mathpar}
 

where

\begin{eqnarray}
  (x)\id{\{} \lpquote Q \rpquote / \lpquote P \rpquote \id{\}}            = 
  \left\{ 
    \begin{array}{ccc}
      \lpquote Q \rpquote & & x \nameeq \lpquote P \rpquote \\
      x & & otherwise \\
    \end{array}
  \right. \nonumber
\end{eqnarray}

and $z$ is chosen distinct from $\quotep{P}$, $\quotep{Q}$, the free
names in $Q$, and all the names in $R$. Our $\alpha$-equivalence will
be built in the standard way from this substitution.

\begin{remark}\label{rem:no_self_referential_names}
  One consequence of these definitions is that $\forall P. \quotep{P}
  \not\in \freenames{P}$.
\end{remark}

\subsection{ Dynamic quote: an example }

Anticipating something of what's to come, consider applying the
substitution, $\widehat{\id{\{}u / z \id{\}}}$, to the following pair
of processes, $\lift{w}{y!(z)}$ and $w[ \lpquote y!(z) \rpquote ]$.

\begin{eqnarray}
	\lift{w}{y!(z)}\widehat{\id{\{}u / z \id{\}}}
		& = &
		\lift{w}{y!(u)} \nonumber\\
	w[ \lpquote y!(z) \rpquote ] \widehat{ \id{\{}u / z \id{\}} }
		& = &
		w[ \lpquote y!(z) \rpquote ] \nonumber
\end{eqnarray}

Because the body of the process between quotes is impervious to
substitution, we get radically different answers. In fact, by
examining the first process in an input context,
e.g. $x?(z).\lift{w}{y!(z)}$, we see that the process under the lift
operator may be shaped by prefixed inputs binding a name inside it. In
this sense, the lift operator will be seen as a way to dynamically
construct processes before reifying them as names.

Finally equipped with these standard features we can present the
dynamics of the calculus.

\subsubsection{Operational semantics} 

Finally, we introduce the computational dynamics. What marks these
algebras as distinct from other more traditionally studied algebraic
structures, e.g. vector spaces or polynomial rings, is the manner in
which dynamics is captured. In traditional structures, dynamics is typically
expressed through morphisms between such structures, as in linear maps
between vector spaces or morphisms between rings. In algebras
associated with the semantics of computation, the dynamics is
expressed as part of the algebraic structure itself, through a
reduction reduction relation typically denoted by $\red$. Below, we
give a recursive presentation of this relation for the calculus used
in the encoding.

$\red \subseteq \pi \times \pi$
$\red : \pi \to \mathcal{P}(\pi)$

\begin{mathpar}
  \inferrule* [lab=Comm] { \textsf{match}( x_{src}, x_{trgt} ) } { x_{trgt}?(y)P \; | \; x_{src}!\langle {Q} \rangle \red P\{\quotep{Q}/y}\} }
  \and \\
  \inferrule* [lab=Par] {{P} \red {P}'} {{{P} | {Q}} \red {{P}' | {Q}}}
  \and
  \inferrule* [lab=Equiv]{{{P} \scong {P}'} \andalso {{P}' \red {Q}'} \andalso {{Q}' \scong {Q}}}{{P} \red {Q}}
\end{mathpar}

\begin{eqnarray*}
  match_{\equiv} (\quotep{P},\quotep{Q}) & := & P \equiv Q \\
  match_{\dagger}(\quotep{P},\quotep{Q}) & := & \forall R. P|Q \red^{*} R => R \red^{*} 0 \\
  match_{K}(\quotep{P},\quotep{Q}) & := & K \mbox{ for some context } K
\end{eqnarray*}

$u?(x)P | u!\langle Q \rangle \red P\{\quotep{Q}/x\}$

%We write $\wred$ for $\red^*$, and $P\red$ if $\exists Q $ such that $ P \red Q$.
We write $P\red$ if $\exists Q $ such that $ P \red Q$ and $P\not\red$, otherwise.

\section{Replication}

As mentioned before, it is known that replication (and hence
recursion) can be implemented in a higher-order process algebra
\cite{SangiorgiWalker}. As our first example of calculation with the
machinery thus far presented we give the construction explicitly in
the {\rhoc}.

\begin{eqnarray}
	D_{x} & := & \prefix{x}{y}{(\binpar{\outputp{x}{y}}{@{y}})} \nonumber\\
	\bangp_{x}{P} & := & \binpar{{x}!\langle{\binpar{D_{x}}{P}}\rangle}{D_{x}} \nonumber
\end{eqnarray}

\begin{eqnarray}
	\bangp_{x}{P} & & \nonumber\\
	=
	& {x}!\langle{(\prefix{x}{y}{(\outputp{x}{y} | @{y})) | P}}\rangle 
	      | \prefix{x}{y}{(\outputp{x}{y} | @{y})} & \nonumber\\
	\red
	& (\outputp{x}{y} | @{y})\substn{\quotep{(\prefix{x}{y}{(@{y} | \outputp{x}{y})) | P}}}{y} & \nonumber\\
	=
	& \outputp{x}{\quotep{(\prefix{x}{y}{(\outputp{x}{y} | @{y})) | P}}}
	  | {(\prefix{x}{y}{(\outputp{x}{y} | @{y})) | P}} & \nonumber\\
	\red
	& \ldots & \nonumber\\
	\red^*
	& P | P | \ldots & \nonumber
\end{eqnarray}

Of course, this encoding, as an implementation, runs away, unfolding
$\bangp{P}$ eagerly. A lazier and more implementable replication
operator, restricted to input-guarded processes, may be obtained as follows.

\begin{eqnarray}
\bangp{\prefix{u}{v}{P}} 
	:= 
	\binpar{\lift{x}{\prefix{u}{v}{(\binpar{D(x)}{P})}}}{D(x)} \nonumber
\end{eqnarray}

\begin{remark}
  Note that the lazier definition still does not deal with summation
  or mixed summation (i.e. sums over input and output). The reader is
  invited to construct definitions of replication that deal with these
  features. 

  Further, the definitions are parameterized in a name, $x$. Can you,
  gentle reader, make a definition that eliminates this parameter and
  guarantees no accidental interaction between the replication
  machinery and the process being replicated -- i.e. no accidental
  sharing of names used by the process to get its work done and the
  name(s) used by the replication to effect copying. This latter
  revision of the definition of replication is crucial to obtaining
  the expected identity $!!P \sim !P$.
\end{remark}

\begin{remark}\label{rem:paradoxical_combinator}
  The reader familiar with the lambda calculus will have noticed the
  similarity between $D$ and the paradoxical combinator.

  [Ed. note: the existence of this seems to suggest we have to be more
  restrictive on the set of processes and names we admit if we are to
  support no-cloning.]
\end{remark}

\subsubsection{Bisimulation}

The computational dynamics gives rise to another kind of equivalence,
the equivalence of computational behavior. As previously mentioned
this is typically captured \emph{via} some form of bisimulation.

% The notion we use in this paper is weak barbed bisimulation
% \cite{milner91polyadicpi}.

The notion we use in this paper is derived from weak barbed
bisimulation \cite{milner91polyadicpi}. 

\begin{definition}
An \emph{observation relation}, $\downarrow_{\mathcal N}$, over a set
of names, $\mathcal N$, is the smallest relation satisfying the rules
below.

\infrule[Out-barb]{y \in {\mathcal N}, \; x \nameeq y}
		  {\outputp{x}{v} \downarrow_{\mathcal N} x}
\infrule[Par-barb]{\mbox{$P\downarrow_{\mathcal N} x$ or $Q\downarrow_{\mathcal N} x$}}
		  {\binpar{P}{Q} \downarrow_{\mathcal N} x}

We write $P \Downarrow_{\mathcal N} x$ if there is $Q$ such that 
$P \wred Q$ and $Q \downarrow_{\mathcal N} x$.
\end{definition}

\begin{definition}
%\label{def.bbisim}
An  ${\mathcal N}$-\emph{barbed bisimulation} over a set of names, ${\mathcal N}$, is a symmetric binary relation 
${\mathcal S}_{\mathcal N}$ between agents such that $P\rel{S}_{\mathcal N}Q$ implies:
\begin{enumerate}
\item If $P \red P'$ then $Q \wred Q'$ and $P'\rel{S}_{\mathcal N} Q'$.
\item If $P\downarrow_{\mathcal N} x$, then $Q\Downarrow_{\mathcal N} x$.
\end{enumerate}
$P$ is ${\mathcal N}$-barbed bisimilar to $Q$, written
$P \wbbisim_{\mathcal N} Q$, if $P \rel{S}_{\mathcal N} Q$ for some ${\mathcal N}$-barbed bisimulation ${\mathcal S}_{\mathcal N}$.
\end{definition}

$\mathcal{R} \subseteq \pi \times \pi$

$P \mathcal{R} Q => \forall P'. P \red P' \Rightarrow \exists Q'. Q \red Q', P' \mathcal{R} Q'$

$P \vdash x \Rightarrow Q \vdash x$

\begin{mathpar}
  \inferrule*[lab=Out-barb]{x \nameeq y}{{y}!\langle{Q}\rangle \vdash x}
  \and
  \inferrule*[lab=Par-barb]{\mbox{$P\vdash x$ or $Q\vdash x$}}{\binpar{P}{Q} \vdash x}
\end{mathpar}

\subsubsection{Contexts}

One of the principle advantages of computational calculi like the
$\pi$-calculus is a well-defined notion of context,
contextual-equivalence and a correlation between
contextual-equivalence and notions of bisimulation. The notion of
context allows the decomposition of a process into (sub-)process and
its syntactic environment, its context. Thus, a context may be
thought of as a process with a ``hole'' (written $\Box$) in it. The
application of a context $M$ to a process $P$, written $M[P]$, is
tantamount to filling the hole in $M$ with $P$. In this paper we do
not need the full weight of this theory, but do make use of the notion
of context in the proof the main theorem. 

\begin{mathpar}
  \inferrule* [lab=summation] {} {{M_{M},M_{N}} \bc \Box \;|\; x.M_{A} \;|\; M_{M}+M_{N}}
  \and
  \inferrule* [lab=agent] {} {{M_{A}} \bc (\vec{x})M_{P} \;| \; \clift{P_0,\ldots,M_{P},\ldots,P_N}}
  \and \\
  \inferrule* [lab=process] {} {{M_{P}} \bc M_{N} \;| \;P|M_{P} }
\end{mathpar} 

\begin{mathpar}
  \inferrule* [lab=sychronization] {} {M_{N} \bc \Box \;|\; x?M_{F} \;|\; x!M_{C}}
  \and
  \inferrule* [lab=abstraction] {} {{M_{F}} \bc (x)M_{P} }
  \and
  \inferrule* [lab=concretion] {} {{M_{C}} \bc \langle M_{P} \rangle }
  \and \\
  \inferrule* [lab=process] {} {{M_{P}} \bc M_{N} \;| \;P|M_{P} }
\end{mathpar}

\begin{definition}[contextual application] Given a context $M$, and
  process $P$, we define the \emph{contextual application}, $M[P] :=
  M\{P/\Box\}$. That is, the contextual application of M to P is the
  substitution of $P$ for $\Box$ in $M$.
\end{definition}

$\meaningof{-} : L \to \mathcal{P}(\pi)$

\begin{mathpar}
  \inferrule* [lab=collection] {} {\meaningof{true} = \pi, \and \meaningof{~E} = \pi \setminus \meaningof{E}, \and \meaningof{E_{1} \& E_{2}} = \meaningof{E_{1}} \cap \meaningof{E_{2}}}
\end{mathpar}

\begin{mathpar}
  \inferrule* [lab=structure] {} {\meaningof{0} = \{ P \in \pi | P \equiv 0 \}, \and \\ \meaningof{E_1 | E_2} = \{ P \in \pi | P \equiv P_{1} | P_{2}, P_{1} \in \meaningof{E_{1}}, P_{2} \in \meaningof{E_2}\} }
\end{mathpar}

\begin{mathpar}
 \inferrule* [lab=behavior] {} {\meaningof{\langle a?b \rangle E} = \{ P \in \pi | P \equiv Q | u?(y)P', \\ \and \\\\ \and \\ \;\;\; u \in \meaningof{a}, \forall z.P'\{z/y\} \in \meaningof{E\{z/b\}}\}, \and \\ \meaningof{a!E} = \{ P \in \pi | P \equiv Q | x!\langle P' \rangle, x \in \meaningof{a} P' \in \meaningof{E}\} }
\end{mathpar}

\begin{mathpar}
 \inferrule* [lab=nominal] {} {\meaningof{\quotep{E}} = \{ \quotep{P} \in \quotep{\pi} | P \in \meaningof{E} \}, \and \meaningof{\quotep{P}} = \{ \quotep{Q} \in \quotep{\pi} | P \equiv Q \} \and \\ \meaningof{@\quotep{E}} = \{ P \in \pi | P \equiv @x, x \in \meaningof{E} \}}
\end{mathpar}

\begin{eqnarray*}
  \\
  \meaningof{-} : TS \to ST
\end{eqnarray*}

\begin{eqnarray*}
  \\
  L : TS \to ST
\end{eqnarray*}

\begin{eqnarray*}
  \\
  P \models E \iff P \in \meaningof{E}
\end{eqnarray*}

\begin{eqnarray*}
  P \approx_{L} Q \iff \forall E \in L. P \models E \iff Q \models E
\end{eqnarray*}

\begin{eqnarray*}
  P \approx_{K} Q
\end{eqnarray*}

\begin{eqnarray*}
  P \approx Q
\end{eqnarray*}

$\approx_{K} = \approx = \approx_{L}$

\subsubsection{Contextual duality}

Note that contexts extend the quotation operation to a family of
operations from processes to names. Given a context, $M$, we can
define a \emph{nominal context}, $\quotep{M}$ by $\quotep{M}[P] :=
\quotep{M[P]}$. To foreshadow what is to come we observe that these
operations enjoy a duality with processes very much like the duality
between vectors and maps from vectors to scalars.

Further, because the calculus is essentially higher-order, we have a
correspondence between contexts and processes. More specifically,
given a name $x$ and a context $M$ we can construct $M^{*}_{x}$ such
that 

\begin{mathpar}
  M^{*}_{x} | \lift{x}{P} \red M[P]
\end{mathpar}

namely,

\begin{mathpar}
  M^{*}_{x} := x?(u).M[\dropn{u}]
\end{mathpar}

The dependence of $M^{*}_{x}$ on a name makes it an abstraction, 

\begin{mathpar}
  M^{*} := (x)x?(u).M[\dropn{u}]
\end{mathpar}

\subsection{Additional notation}

It will sometimes be convenient to denote the process a name
quotes. We already have the notation $x = \quotep{P}$, but it will be
convenient to introduce an alternate notation, $\procn{x}$, when we
want to emphasize the connection to the use of the name. Note that, by
virtue of name equivalence, $\quotep{\procn{x}} \nameeq x$; so, the
notation is consistent with previous definitions.

Further, because names have structure it is possible to effect
substitutions on the basis of that structure. This means we need to
upgrade our notation for substitutions, which we accomplish by
adapting comprehension notation. Thus,

\begin{mathpar}
  P\{ y / x : x \in S \}
\end{mathpar}

is interpreted to mean the process derived from P by replacing (in a
capture-avoiding manner) each occurrence of $x$ in $S$ by $y$. For example,

\begin{mathpar}
  P\{ \quotep{\procn{x}|\procn{x}} / x : x \in \freenames{P} \}
\end{mathpar}

will replace each (occurrence) of a free name $x$ in $P$ by
$\quotep{\procn{x}|\procn{x}}$.

Also, we will avail ourselves of the notation $x^{L}$ and $x^{R}$ to
denote injections of a name into disjoint copies of the name
space. There are numerous ways to accomplish this. One example can be
found in \cite{MeredithR05}. This notation overloads to vectors of
names: $\vec{x}^{\pi} := (x_{i}^{\pi} \; : \; 0 \leq i < |\vec{x}| )$ where $\pi \in \{L,R\}$.

We also use $P^{\Box} := P|\Box$.

In \cite{MeredithR05} an interpretation of the new operator is
given. It turns out that there are several possible interpretations
all enjoying the requisite algebraic properties of the operator (see
\cite{milner91polyadicpi}). We will therefore make liberal use of
$(\nu\; \vec{x})P$.

% subsection the_syntax_and_semantics_of_the_notation_system (end)   

\input{qm2pi.qmops} 

\input{qm2pi.sterngerlach} 

\input{qm2pi.metric} 

% section concurrent_process_calculi (end)

%\input{qm2pi.proofsketch}

% section proof sketch (end)

%\input{qm2pi.slviaknots} 

% section spatial logic via knots (end)

\input{qm2pi.conclusion}

% section conclusion (end)

%\input{qm2pi.dtcodes} 

% section wiring algorithm (end)

\input{qm2pi.ack} 

% section acknowledgments (end)

\newpage


\bibliographystyle{plain}   
\bibliography{../../biblios/main.bib}

\input{qm2pi.rhodetails}

\end{document}



% section front matter (end)

\section{Introduction}\label{sec:introduction} % (fold)
In this draft of the material i am going to have to dispense with the
usual writing conventions adopted in papers on these topics. i'm going
to have adopt whatever tone i need at the time i'm writing up the
calculations. Sometimes this may be very conversational; others it may
be the barest mathematical grunts; others still it may be that i have
lifted text from one of my other papers because the exposition of some
point was better said there. i hope that my readers are not unduly put
out by this decision. i'm not doing this to flout convention or be
rebellious. i find these calculations very technically challenging. To
keep everything going technically, something has to give; i have to
let go of some cognitive burden. So, the academic writing style --
with all of its trade-offs in terms of facilitating technical
communication -- is what i'm letting go of. Perhaps subsequent drafts
can be tightened and polished, but for now, i'm going to speak as if
we were sitting together in a coffee shop with a laptop, wifi and a
pad of paper and a pencil.

So, here's what i have to say. We -- you and i, comfortably ensconced
in our coffee shop and well-equipped with our tools -- can realize and
carry out the calculations of quantum mechanics over a very different
formal theory of dynamics, a formal theory of dynamics that
corresponds to a theory of concurrent computation with
\emph{reflection}. It has the advantage that the underlying theory is
already `quantized', but supports analogues all of the continuuous
operations. Strikingly, this underlying theory has recently been
connected with a notion of metric that we can show, by calculating
together, coincides with the metric induced by the inner product.

There are a lot of reasons why you might be interested in seeing
calculations of this form. Here's why i'm interested. For the past
several centuries there has been no competitor to the ``Newtonian''
account of dynamics. As a result the predominant share of accounts of
dynamical systems and situations have had to be formulated in terms of
the Newtonian machinery. i view this as an intellectually dangerous
position to occupy. Everything, despite it's intrinsic shape, turns
into a nail to be hit with this hammer. Recently, however, the theory
of computation has matured to the point where we have candidates for
theories of dynamics that offer very different perspective on
reasoning about dynamical systems and situations. Testing these
candidates against very successful accounts of dynamical situations,
like quantum mechanics, is going to give us some sense of how mature
they are and some measure of the quality of these accounts of
dynamics.

\subsection{Summary of contributions and outline of paper}

So, we're going to develop an interpretation of the operations of
quantum mechanics normally interpreted by Hilbert spaces and
operators. We're going to do this over a theory of computation. Note
that this is very different than the usual quantum computation program
which develops notions of computation over quantum mechanics. Rather,
we are developing a story that aligns with Wheeler's slogan: It from
Bit. To do this we will first provide an account of the theory of
computation at play here. Then we will dive into a calculation-driven
interpretation of the operations of quantum mechanics.

The reason we take this approach is that -- until very recently --
there hasn't been an axiomatic account of quantum mechanics. As a
result there has been no sharp delineation of the mathematical theory
supporting interpretation of the physical theory and the physical
theory, itself. So, ambient features of the maths are free to be
exploited (or supressed) without a real accounting of their physical
relevance. There is no sharp statement ``here's the physical theory''
qua \emph{theory} and ``here's the mathematical interpretation''
enabling a judgment of how faithful the interpretation is -- apart
from experimental observation. When there is an axiomatic account we
can judge how well a given mathematical formalism supports an
interpretation of the axioms, independent of
experimentation. Likewise, we can judge how well we have captured our
physical evidence and experience with our axiomatics, independent of
any specific mathematical implementation, with accidental detail that
may or may not have physical significance. 

In lieu of a fully fleshed out and vetted axiomatic account of quantum
mechanics, interpreting the operational notions in service of modeling
physical systems will have to suffice. In other words, we are not in
the business of providing a model of Hilbert spaces and operators. We
are in the business of providing a model of quantum mechanics because
we are motivated by testing our notions of dynamics against physical
theory; and, the predictive calculations of the physical theory must
serve as the best formulation -- shy of a fully fleshed out axiomatic
account -- of the physical theory itself (as they have for scientific
theories since time immemorial). Put another way, despite a
whole-hearted commitment to an It-from-Bit ontology, we are firmly
aligned with the shut-up-and-calculate camp as the best way to obtain
results either from the physical perspective or as a quality assurance
measure of our fledgling theory of dynamics.

In detail, we present a reflective process calculus. Then we develop
intuitive correspondences between the notions available in this
calculus and the usual physical notions supporting quantum mechanical
calculations. Thus, 

\begin{table}[htp]
  \center{
    \fbox{
      \begin{tabular}{c|c}
        quantum mechanics & process calculus \\
        \hline
        scalar & name \\
        state vector & process \\
        dual & contextual duals \\
        matrix & formal sums of process-context-dual pairs \\
        orthogonality & process annihilation \\
        inner product & execution-formula + quoting
      \end{tabular}
    }
  }
  \caption{QM - process calculi correspondences}
\end{table}

Then we tighten up these intuitions to operational definitions. We
employ the Dirac notation as the best proxy we can find for an
abstract syntax of the quantum mechanical notions. The definitions we
develop put us in contact with equational constraints coming from the
theory that we demonstrate the definitions and calculations satisfy.

This puts us in a position to shut up and calculate for the
Stern-Gerlach experimental set up, showing how these predictive
calculations become calculations on processes in our theory of a
reflective process calculus.

Penultimately, we demonstrate that the notion of metric coming from
the inner product coincides with the notion of metric available from
the theory of bisimulation. This demonstration gives us the right to
think of space as arising from behavior. Finally, we consider where we
might go from the new vantage point we have obtained.

% section introduction (end) 
 
% section introduction (end)

% \documentclass[12pt]{llncs}
%\documentclass{jktr}

\usepackage[pdftex]{hyperref}                   
\usepackage {listings}
\usepackage {mathpartir}
\usepackage{bcprules}
%\usepackage{listings}
                       
\usepackage{graphicx} 
%\usepackage[margins=2.5cm,nohead,nofoot]{geometry}
%\usepackage{geometry}
\usepackage{amsfonts}
\usepackage{amstext}
\usepackage{latexsym}
\usepackage{amssymb}
\usepackage{color}


%\include{myPreamble}
\include{qm2pi.local} 

%\ifpdf
%\usepackage[pdftex]{graphicx}
%\else
%\usepackage{graphicx}
%\fi

 % \ifpdf
%  \usepackage{pdfsync}
%  \if


%\title{Brief Article}
%\author{David F. Snyder}
%\author{L.G. Meredith}

%\address{Dept. of Math., Texas State University--San Marcos, San Marcos, TX 78666}
       
\pagestyle{empty}


\begin{document}

\lstset{language=[Objective]Caml,frame=shadowbox}

\input{qm2pi.front}

% section front matter (end)

\input{qm2pi.intro} 
 
% section introduction (end)

% \input{qm2pi.knotations} 

% section notation (end)

\input{qm2pi.process.calculi} 

% section concurrent_process_calculi_and_spatial_logics_ (end)
    
%\input{qm2pi.knots2pi} 

%\input{qm2pi.trefoil} 

%\input{qm2pi.mainthm} 

% subsection basic_interpretation (end)

%\input{qm2pi.rho.presentation} 
\subsection{The syntax and semantics of the notation system}\label{sub:the_syntax_and_semantics_of_the_notation_system} % (fold)

We now summarize a technical presentation of the calculus that
embodies our theory of dynamics. The typical presentation of such a
calculus follows the style of giving generators and relations on
them. The grammar, below, describing term constructors, freely
generates the set of processes, $\Proc$. This set is then quotiented
by a relation known as structural congruence and it is over this set
that the notion of dynamics is expressed. This presentation is
essentially that of \cite{MeredithR05} with the addition of
polyadicity and summation. For readability we have relegated some of
the technical subtleties to an appendix.

\subsubsection{Process grammar}\label{subsub:process_grammar}

\begin{mathpar}
  \inferrule* [lab=synchronization] {} {{M} \bc \pzero \;|\; x?F \;|\; x!C }
  \and
  \inferrule* [lab=abstraction] {} {{F} \bc (x)P}
  \and
  \inferrule* [lab=concretion] {} {{C} \bc \langle Q \rangle}
  \and
  \inferrule* [lab=process] {} {{P,Q} \bc M \;| \;P|Q \;|\; @{x}}
  \and
  \inferrule* [lab=name] {} {{x} \bc \quotep{P}}
\end{mathpar} 

Note that $\vec{x}$ (resp. $\vec{P}$) denotes a vector of names
(resp. processes) of length $|\vec{x}|$ (resp. $|\vec{P}|$). We adopt
the following useful abbreviations.

\begin{mathpar}
   x?(\vec{y}).P := x.(\vec{y})P \and  x\clift{\vec{P}} := x.\clift{\vec{P}}
   \and x!(y) := \lift{x}{\dropn{y}}
   \and \Pi_{i=0}^{n-1}P_i := P_0 | \ldots | P_{n-1}
\end{mathpar}

\subsubsection{Structural congruence}

\paragraph{Free and bound names and alpha-equivalence.} At the
core of structural equivalence is alpha-equivalence which identifies
process that are the same up to a change of variable. Formally, we
recognize the distinction between free and bound names. The free names
of a process, $\freenames{P}$, may be calculated recursively as
follows:

\begin{mathpar}
\freenames{\pzero} := \emptyset
  \and \\
  \freenames{x?(y).P} := \{ x \} \cup (\freenames{P} \setminus \{ y \})
  \and 
  \freenames{x!\langle P \rangle} := \{ x \} \cup \{ P \} 
  \and \\
  \freenames{P|Q} := \freenames{P} \cup \freenames{Q}
  \and \\
  \freenames{@{x}} := \{ x \}
\end{mathpar}

$\pi$
$\quotep{\pi}$

$\freenames{-} : \pi \to \mathcal{P}(\quotep{\pi})$

\begin{eqnarray*}
  \freenames{\pzero} & := & \emptyset \\
  \freenames{x?(y).P} & := & \{ x \} \cup (\freenames{P} \setminus \{ y \}) \\
  \freenames{x!\langle P \rangle} & := & \{ x \} \cup \{ P \} \\
  \freenames{P|Q} & := & \freenames{P} \cup \freenames{Q} \\
  \freenames{\dropn{x}} & := & \{ x \}
\end{eqnarray*}

The bound names of a process, $\boundnames{P}$, are those names occurring in $P$
that are not free. For example, in $x?(y).0$, the name $x$ is free, while $y$ is bound.

\begin{mathpar}
  \inferrule* [lab=monoidal-laws] {} { P|Q \equiv Q|P \and P|0 \equiv P \and P|(Q|R) \equiv (P|Q)|R }
\end{mathpar}

\begin{mathpar}
  \inferrule* [lab=alpha-equivalence] {} { (x)P \equiv (y)P\{y/x\} \and y \not\in \freenames{P} }
\end{mathpar}

\begin{definition}
Then two processes, $P,Q$, are alpha-equivalent if $P = Q\{\vec{y}/\vec{x}\}$ for
some $\vec{x} \in \boundnames{Q},\vec{y} \in \boundnames{P}$, where $Q\{\vec{y}/\vec{x}\}$
denotes the capture-avoiding substitution of $\vec{y}$ for $\vec{x}$ in $Q$.
\end{definition}

\begin{definition}
  The {\em structural congruence} \cite{SangiorgiWalker} , $\equiv$,
  between processes is the least congruence containing
  alpha-equivalence, satisfying the abelian monoid laws
  (associativity, commutativity and $\pzero$ as identity) for parallel
  composition $|$ and for summation $+$.
\end{definition}

\subsection{Name equivalence}

We take name equivalence, written $\nameeq$, to be the smallest
equivalence relation generated by the following rules.

\begin{mathpar}
\inferrule*[lab=Quote-drop]
{ }
{ \quotep{@{x}} \nameeq x }

\inferrule*[lab=Struct-equiv]
{ P \scong Q }
{ \quotep{P} \nameeq \quotep{Q} }
\end{mathpar}

The astute reader will have noticed that the mutual recursion of names
and processes imposes a mutual recursion on alpha-equivalence and
structural equivalence via name-equivalence. Fortunately, all of this
works out pleasantly and we may calculate in the natural way, free of
concern. The reader interested in the details is referred to the
appendix \ref{appendix:rho_details}.

\subsection{Substitution}

We use $\Proc$ for the set of processes, $\QProc$ for the set of
names, and $\id{\{}\vec{y} / \vec{x} \id{\}}$ to denote partial maps,
$s : \QProc \rightarrow \QProc$. A map, $s$ lifts, uniquely, to a map
on process terms, $\widehat{s} : \Proc \rightarrow \Proc$ by the
following equations.

\begin{mathpar}
  (0) \psubstp{Q}{P} := 0 \\
  (R \juxtap S) \psubstp{Q}{P}
  :=    
  (R)\psubstp{Q}{P} \juxtap (S) \psubstp{Q}{P} \\
  (x?(y).R) \psubstp{Q}{P}    
  :=    
  (x)\substp{Q}{P} (z)\concat( (R \psubstn{z}{y}) \psubstp{Q}{P} ) \\
  (\lift{x}{R}) \psubstp{Q}{P}  
  :=
  \lift{(x)\substp{Q}{P}}{ R \psubstp{Q}{P} } \\
%   (\dropn{x})  \psubstp{Q}{P}       
%   := 
%   \left\{ 
%     \begin{array}{ccc} 
%       \dropn{\quotep{Q}} & & x \nameeq \quotep{P} \\
%       \dropn{x} & & otherwise \\
%     \end{array}
%   \right. 
  (\dropn{x})  \psubstp{Q}{P}       
  := 
  \left\{ 
    \begin{array}{ccc} 
      Q & & x \nameeq \quotep{P} \\
      \dropn{x} & & otherwise \\
    \end{array}
  \right.
\end{mathpar}
 

where

\begin{eqnarray}
  (x)\id{\{} \lpquote Q \rpquote / \lpquote P \rpquote \id{\}}            = 
  \left\{ 
    \begin{array}{ccc}
      \lpquote Q \rpquote & & x \nameeq \lpquote P \rpquote \\
      x & & otherwise \\
    \end{array}
  \right. \nonumber
\end{eqnarray}

and $z$ is chosen distinct from $\quotep{P}$, $\quotep{Q}$, the free
names in $Q$, and all the names in $R$. Our $\alpha$-equivalence will
be built in the standard way from this substitution.

\begin{remark}\label{rem:no_self_referential_names}
  One consequence of these definitions is that $\forall P. \quotep{P}
  \not\in \freenames{P}$.
\end{remark}

\subsection{ Dynamic quote: an example }

Anticipating something of what's to come, consider applying the
substitution, $\widehat{\id{\{}u / z \id{\}}}$, to the following pair
of processes, $\lift{w}{y!(z)}$ and $w[ \lpquote y!(z) \rpquote ]$.

\begin{eqnarray}
	\lift{w}{y!(z)}\widehat{\id{\{}u / z \id{\}}}
		& = &
		\lift{w}{y!(u)} \nonumber\\
	w[ \lpquote y!(z) \rpquote ] \widehat{ \id{\{}u / z \id{\}} }
		& = &
		w[ \lpquote y!(z) \rpquote ] \nonumber
\end{eqnarray}

Because the body of the process between quotes is impervious to
substitution, we get radically different answers. In fact, by
examining the first process in an input context,
e.g. $x?(z).\lift{w}{y!(z)}$, we see that the process under the lift
operator may be shaped by prefixed inputs binding a name inside it. In
this sense, the lift operator will be seen as a way to dynamically
construct processes before reifying them as names.

Finally equipped with these standard features we can present the
dynamics of the calculus.

\subsubsection{Operational semantics} 

Finally, we introduce the computational dynamics. What marks these
algebras as distinct from other more traditionally studied algebraic
structures, e.g. vector spaces or polynomial rings, is the manner in
which dynamics is captured. In traditional structures, dynamics is typically
expressed through morphisms between such structures, as in linear maps
between vector spaces or morphisms between rings. In algebras
associated with the semantics of computation, the dynamics is
expressed as part of the algebraic structure itself, through a
reduction reduction relation typically denoted by $\red$. Below, we
give a recursive presentation of this relation for the calculus used
in the encoding.

$\red \subseteq \pi \times \pi$
$\red : \pi \to \mathcal{P}(\pi)$

\begin{mathpar}
  \inferrule* [lab=Comm] { \textsf{match}( x_{src}, x_{trgt} ) } { x_{trgt}?(y)P \; | \; x_{src}!\langle {Q} \rangle \red P\{\quotep{Q}/y}\} }
  \and \\
  \inferrule* [lab=Par] {{P} \red {P}'} {{{P} | {Q}} \red {{P}' | {Q}}}
  \and
  \inferrule* [lab=Equiv]{{{P} \scong {P}'} \andalso {{P}' \red {Q}'} \andalso {{Q}' \scong {Q}}}{{P} \red {Q}}
\end{mathpar}

\begin{eqnarray*}
  match_{\equiv} (\quotep{P},\quotep{Q}) & := & P \equiv Q \\
  match_{\dagger}(\quotep{P},\quotep{Q}) & := & \forall R. P|Q \red^{*} R => R \red^{*} 0 \\
  match_{K}(\quotep{P},\quotep{Q}) & := & K \mbox{ for some context } K
\end{eqnarray*}

$u?(x)P | u!\langle Q \rangle \red P\{\quotep{Q}/x\}$

%We write $\wred$ for $\red^*$, and $P\red$ if $\exists Q $ such that $ P \red Q$.
We write $P\red$ if $\exists Q $ such that $ P \red Q$ and $P\not\red$, otherwise.

\section{Replication}

As mentioned before, it is known that replication (and hence
recursion) can be implemented in a higher-order process algebra
\cite{SangiorgiWalker}. As our first example of calculation with the
machinery thus far presented we give the construction explicitly in
the {\rhoc}.

\begin{eqnarray}
	D_{x} & := & \prefix{x}{y}{(\binpar{\outputp{x}{y}}{@{y}})} \nonumber\\
	\bangp_{x}{P} & := & \binpar{{x}!\langle{\binpar{D_{x}}{P}}\rangle}{D_{x}} \nonumber
\end{eqnarray}

\begin{eqnarray}
	\bangp_{x}{P} & & \nonumber\\
	=
	& {x}!\langle{(\prefix{x}{y}{(\outputp{x}{y} | @{y})) | P}}\rangle 
	      | \prefix{x}{y}{(\outputp{x}{y} | @{y})} & \nonumber\\
	\red
	& (\outputp{x}{y} | @{y})\substn{\quotep{(\prefix{x}{y}{(@{y} | \outputp{x}{y})) | P}}}{y} & \nonumber\\
	=
	& \outputp{x}{\quotep{(\prefix{x}{y}{(\outputp{x}{y} | @{y})) | P}}}
	  | {(\prefix{x}{y}{(\outputp{x}{y} | @{y})) | P}} & \nonumber\\
	\red
	& \ldots & \nonumber\\
	\red^*
	& P | P | \ldots & \nonumber
\end{eqnarray}

Of course, this encoding, as an implementation, runs away, unfolding
$\bangp{P}$ eagerly. A lazier and more implementable replication
operator, restricted to input-guarded processes, may be obtained as follows.

\begin{eqnarray}
\bangp{\prefix{u}{v}{P}} 
	:= 
	\binpar{\lift{x}{\prefix{u}{v}{(\binpar{D(x)}{P})}}}{D(x)} \nonumber
\end{eqnarray}

\begin{remark}
  Note that the lazier definition still does not deal with summation
  or mixed summation (i.e. sums over input and output). The reader is
  invited to construct definitions of replication that deal with these
  features. 

  Further, the definitions are parameterized in a name, $x$. Can you,
  gentle reader, make a definition that eliminates this parameter and
  guarantees no accidental interaction between the replication
  machinery and the process being replicated -- i.e. no accidental
  sharing of names used by the process to get its work done and the
  name(s) used by the replication to effect copying. This latter
  revision of the definition of replication is crucial to obtaining
  the expected identity $!!P \sim !P$.
\end{remark}

\begin{remark}\label{rem:paradoxical_combinator}
  The reader familiar with the lambda calculus will have noticed the
  similarity between $D$ and the paradoxical combinator.

  [Ed. note: the existence of this seems to suggest we have to be more
  restrictive on the set of processes and names we admit if we are to
  support no-cloning.]
\end{remark}

\subsubsection{Bisimulation}

The computational dynamics gives rise to another kind of equivalence,
the equivalence of computational behavior. As previously mentioned
this is typically captured \emph{via} some form of bisimulation.

% The notion we use in this paper is weak barbed bisimulation
% \cite{milner91polyadicpi}.

The notion we use in this paper is derived from weak barbed
bisimulation \cite{milner91polyadicpi}. 

\begin{definition}
An \emph{observation relation}, $\downarrow_{\mathcal N}$, over a set
of names, $\mathcal N$, is the smallest relation satisfying the rules
below.

\infrule[Out-barb]{y \in {\mathcal N}, \; x \nameeq y}
		  {\outputp{x}{v} \downarrow_{\mathcal N} x}
\infrule[Par-barb]{\mbox{$P\downarrow_{\mathcal N} x$ or $Q\downarrow_{\mathcal N} x$}}
		  {\binpar{P}{Q} \downarrow_{\mathcal N} x}

We write $P \Downarrow_{\mathcal N} x$ if there is $Q$ such that 
$P \wred Q$ and $Q \downarrow_{\mathcal N} x$.
\end{definition}

\begin{definition}
%\label{def.bbisim}
An  ${\mathcal N}$-\emph{barbed bisimulation} over a set of names, ${\mathcal N}$, is a symmetric binary relation 
${\mathcal S}_{\mathcal N}$ between agents such that $P\rel{S}_{\mathcal N}Q$ implies:
\begin{enumerate}
\item If $P \red P'$ then $Q \wred Q'$ and $P'\rel{S}_{\mathcal N} Q'$.
\item If $P\downarrow_{\mathcal N} x$, then $Q\Downarrow_{\mathcal N} x$.
\end{enumerate}
$P$ is ${\mathcal N}$-barbed bisimilar to $Q$, written
$P \wbbisim_{\mathcal N} Q$, if $P \rel{S}_{\mathcal N} Q$ for some ${\mathcal N}$-barbed bisimulation ${\mathcal S}_{\mathcal N}$.
\end{definition}

$\mathcal{R} \subseteq \pi \times \pi$

$P \mathcal{R} Q => \forall P'. P \red P' \Rightarrow \exists Q'. Q \red Q', P' \mathcal{R} Q'$

$P \vdash x \Rightarrow Q \vdash x$

\begin{mathpar}
  \inferrule*[lab=Out-barb]{x \nameeq y}{{y}!\langle{Q}\rangle \vdash x}
  \and
  \inferrule*[lab=Par-barb]{\mbox{$P\vdash x$ or $Q\vdash x$}}{\binpar{P}{Q} \vdash x}
\end{mathpar}

\subsubsection{Contexts}

One of the principle advantages of computational calculi like the
$\pi$-calculus is a well-defined notion of context,
contextual-equivalence and a correlation between
contextual-equivalence and notions of bisimulation. The notion of
context allows the decomposition of a process into (sub-)process and
its syntactic environment, its context. Thus, a context may be
thought of as a process with a ``hole'' (written $\Box$) in it. The
application of a context $M$ to a process $P$, written $M[P]$, is
tantamount to filling the hole in $M$ with $P$. In this paper we do
not need the full weight of this theory, but do make use of the notion
of context in the proof the main theorem. 

\begin{mathpar}
  \inferrule* [lab=summation] {} {{M_{M},M_{N}} \bc \Box \;|\; x.M_{A} \;|\; M_{M}+M_{N}}
  \and
  \inferrule* [lab=agent] {} {{M_{A}} \bc (\vec{x})M_{P} \;| \; \clift{P_0,\ldots,M_{P},\ldots,P_N}}
  \and \\
  \inferrule* [lab=process] {} {{M_{P}} \bc M_{N} \;| \;P|M_{P} }
\end{mathpar} 

\begin{mathpar}
  \inferrule* [lab=sychronization] {} {M_{N} \bc \Box \;|\; x?M_{F} \;|\; x!M_{C}}
  \and
  \inferrule* [lab=abstraction] {} {{M_{F}} \bc (x)M_{P} }
  \and
  \inferrule* [lab=concretion] {} {{M_{C}} \bc \langle M_{P} \rangle }
  \and \\
  \inferrule* [lab=process] {} {{M_{P}} \bc M_{N} \;| \;P|M_{P} }
\end{mathpar}

\begin{definition}[contextual application] Given a context $M$, and
  process $P$, we define the \emph{contextual application}, $M[P] :=
  M\{P/\Box\}$. That is, the contextual application of M to P is the
  substitution of $P$ for $\Box$ in $M$.
\end{definition}

$\meaningof{-} : L \to \mathcal{P}(\pi)$

\begin{mathpar}
  \inferrule* [lab=collection] {} {\meaningof{true} = \pi, \and \meaningof{~E} = \pi \setminus \meaningof{E}, \and \meaningof{E_{1} \& E_{2}} = \meaningof{E_{1}} \cap \meaningof{E_{2}}}
\end{mathpar}

\begin{mathpar}
  \inferrule* [lab=structure] {} {\meaningof{0} = \{ P \in \pi | P \equiv 0 \}, \and \\ \meaningof{E_1 | E_2} = \{ P \in \pi | P \equiv P_{1} | P_{2}, P_{1} \in \meaningof{E_{1}}, P_{2} \in \meaningof{E_2}\} }
\end{mathpar}

\begin{mathpar}
 \inferrule* [lab=behavior] {} {\meaningof{\langle a?b \rangle E} = \{ P \in \pi | P \equiv Q | u?(y)P', \\ \and \\\\ \and \\ \;\;\; u \in \meaningof{a}, \forall z.P'\{z/y\} \in \meaningof{E\{z/b\}}\}, \and \\ \meaningof{a!E} = \{ P \in \pi | P \equiv Q | x!\langle P' \rangle, x \in \meaningof{a} P' \in \meaningof{E}\} }
\end{mathpar}

\begin{mathpar}
 \inferrule* [lab=nominal] {} {\meaningof{\quotep{E}} = \{ \quotep{P} \in \quotep{\pi} | P \in \meaningof{E} \}, \and \meaningof{\quotep{P}} = \{ \quotep{Q} \in \quotep{\pi} | P \equiv Q \} \and \\ \meaningof{@\quotep{E}} = \{ P \in \pi | P \equiv @x, x \in \meaningof{E} \}}
\end{mathpar}

\begin{eqnarray*}
  \\
  \meaningof{-} : TS \to ST
\end{eqnarray*}

\begin{eqnarray*}
  \\
  L : TS \to ST
\end{eqnarray*}

\begin{eqnarray*}
  \\
  P \models E \iff P \in \meaningof{E}
\end{eqnarray*}

\begin{eqnarray*}
  P \approx_{L} Q \iff \forall E \in L. P \models E \iff Q \models E
\end{eqnarray*}

\begin{eqnarray*}
  P \approx_{K} Q
\end{eqnarray*}

\begin{eqnarray*}
  P \approx Q
\end{eqnarray*}

$\approx_{K} = \approx = \approx_{L}$

\subsubsection{Contextual duality}

Note that contexts extend the quotation operation to a family of
operations from processes to names. Given a context, $M$, we can
define a \emph{nominal context}, $\quotep{M}$ by $\quotep{M}[P] :=
\quotep{M[P]}$. To foreshadow what is to come we observe that these
operations enjoy a duality with processes very much like the duality
between vectors and maps from vectors to scalars.

Further, because the calculus is essentially higher-order, we have a
correspondence between contexts and processes. More specifically,
given a name $x$ and a context $M$ we can construct $M^{*}_{x}$ such
that 

\begin{mathpar}
  M^{*}_{x} | \lift{x}{P} \red M[P]
\end{mathpar}

namely,

\begin{mathpar}
  M^{*}_{x} := x?(u).M[\dropn{u}]
\end{mathpar}

The dependence of $M^{*}_{x}$ on a name makes it an abstraction, 

\begin{mathpar}
  M^{*} := (x)x?(u).M[\dropn{u}]
\end{mathpar}

\subsection{Additional notation}

It will sometimes be convenient to denote the process a name
quotes. We already have the notation $x = \quotep{P}$, but it will be
convenient to introduce an alternate notation, $\procn{x}$, when we
want to emphasize the connection to the use of the name. Note that, by
virtue of name equivalence, $\quotep{\procn{x}} \nameeq x$; so, the
notation is consistent with previous definitions.

Further, because names have structure it is possible to effect
substitutions on the basis of that structure. This means we need to
upgrade our notation for substitutions, which we accomplish by
adapting comprehension notation. Thus,

\begin{mathpar}
  P\{ y / x : x \in S \}
\end{mathpar}

is interpreted to mean the process derived from P by replacing (in a
capture-avoiding manner) each occurrence of $x$ in $S$ by $y$. For example,

\begin{mathpar}
  P\{ \quotep{\procn{x}|\procn{x}} / x : x \in \freenames{P} \}
\end{mathpar}

will replace each (occurrence) of a free name $x$ in $P$ by
$\quotep{\procn{x}|\procn{x}}$.

Also, we will avail ourselves of the notation $x^{L}$ and $x^{R}$ to
denote injections of a name into disjoint copies of the name
space. There are numerous ways to accomplish this. One example can be
found in \cite{MeredithR05}. This notation overloads to vectors of
names: $\vec{x}^{\pi} := (x_{i}^{\pi} \; : \; 0 \leq i < |\vec{x}| )$ where $\pi \in \{L,R\}$.

We also use $P^{\Box} := P|\Box$.

In \cite{MeredithR05} an interpretation of the new operator is
given. It turns out that there are several possible interpretations
all enjoying the requisite algebraic properties of the operator (see
\cite{milner91polyadicpi}). We will therefore make liberal use of
$(\nu\; \vec{x})P$.

% subsection the_syntax_and_semantics_of_the_notation_system (end)   

\input{qm2pi.qmops} 

\input{qm2pi.sterngerlach} 

\input{qm2pi.metric} 

% section concurrent_process_calculi (end)

%\input{qm2pi.proofsketch}

% section proof sketch (end)

%\input{qm2pi.slviaknots} 

% section spatial logic via knots (end)

\input{qm2pi.conclusion}

% section conclusion (end)

%\input{qm2pi.dtcodes} 

% section wiring algorithm (end)

\input{qm2pi.ack} 

% section acknowledgments (end)

\newpage


\bibliographystyle{plain}   
\bibliography{../../biblios/main.bib}

\input{qm2pi.rhodetails}

\end{document}

 

% section notation (end)

\input{qm2pi.process.calculi} 

% section concurrent_process_calculi_and_spatial_logics_ (end)
    
%\documentclass[12pt]{llncs}
%\documentclass{jktr}

\usepackage[pdftex]{hyperref}                   
\usepackage {listings}
\usepackage {mathpartir}
\usepackage{bcprules}
%\usepackage{listings}
                       
\usepackage{graphicx} 
%\usepackage[margins=2.5cm,nohead,nofoot]{geometry}
%\usepackage{geometry}
\usepackage{amsfonts}
\usepackage{amstext}
\usepackage{latexsym}
\usepackage{amssymb}
\usepackage{color}


%\include{myPreamble}
\include{qm2pi.local} 

%\ifpdf
%\usepackage[pdftex]{graphicx}
%\else
%\usepackage{graphicx}
%\fi

 % \ifpdf
%  \usepackage{pdfsync}
%  \if


%\title{Brief Article}
%\author{David F. Snyder}
%\author{L.G. Meredith}

%\address{Dept. of Math., Texas State University--San Marcos, San Marcos, TX 78666}
       
\pagestyle{empty}


\begin{document}

\lstset{language=[Objective]Caml,frame=shadowbox}

\input{qm2pi.front}

% section front matter (end)

\input{qm2pi.intro} 
 
% section introduction (end)

% \input{qm2pi.knotations} 

% section notation (end)

\input{qm2pi.process.calculi} 

% section concurrent_process_calculi_and_spatial_logics_ (end)
    
%\input{qm2pi.knots2pi} 

%\input{qm2pi.trefoil} 

%\input{qm2pi.mainthm} 

% subsection basic_interpretation (end)

%\input{qm2pi.rho.presentation} 
\subsection{The syntax and semantics of the notation system}\label{sub:the_syntax_and_semantics_of_the_notation_system} % (fold)

We now summarize a technical presentation of the calculus that
embodies our theory of dynamics. The typical presentation of such a
calculus follows the style of giving generators and relations on
them. The grammar, below, describing term constructors, freely
generates the set of processes, $\Proc$. This set is then quotiented
by a relation known as structural congruence and it is over this set
that the notion of dynamics is expressed. This presentation is
essentially that of \cite{MeredithR05} with the addition of
polyadicity and summation. For readability we have relegated some of
the technical subtleties to an appendix.

\subsubsection{Process grammar}\label{subsub:process_grammar}

\begin{mathpar}
  \inferrule* [lab=synchronization] {} {{M} \bc \pzero \;|\; x?F \;|\; x!C }
  \and
  \inferrule* [lab=abstraction] {} {{F} \bc (x)P}
  \and
  \inferrule* [lab=concretion] {} {{C} \bc \langle Q \rangle}
  \and
  \inferrule* [lab=process] {} {{P,Q} \bc M \;| \;P|Q \;|\; @{x}}
  \and
  \inferrule* [lab=name] {} {{x} \bc \quotep{P}}
\end{mathpar} 

Note that $\vec{x}$ (resp. $\vec{P}$) denotes a vector of names
(resp. processes) of length $|\vec{x}|$ (resp. $|\vec{P}|$). We adopt
the following useful abbreviations.

\begin{mathpar}
   x?(\vec{y}).P := x.(\vec{y})P \and  x\clift{\vec{P}} := x.\clift{\vec{P}}
   \and x!(y) := \lift{x}{\dropn{y}}
   \and \Pi_{i=0}^{n-1}P_i := P_0 | \ldots | P_{n-1}
\end{mathpar}

\subsubsection{Structural congruence}

\paragraph{Free and bound names and alpha-equivalence.} At the
core of structural equivalence is alpha-equivalence which identifies
process that are the same up to a change of variable. Formally, we
recognize the distinction between free and bound names. The free names
of a process, $\freenames{P}$, may be calculated recursively as
follows:

\begin{mathpar}
\freenames{\pzero} := \emptyset
  \and \\
  \freenames{x?(y).P} := \{ x \} \cup (\freenames{P} \setminus \{ y \})
  \and 
  \freenames{x!\langle P \rangle} := \{ x \} \cup \{ P \} 
  \and \\
  \freenames{P|Q} := \freenames{P} \cup \freenames{Q}
  \and \\
  \freenames{@{x}} := \{ x \}
\end{mathpar}

$\pi$
$\quotep{\pi}$

$\freenames{-} : \pi \to \mathcal{P}(\quotep{\pi})$

\begin{eqnarray*}
  \freenames{\pzero} & := & \emptyset \\
  \freenames{x?(y).P} & := & \{ x \} \cup (\freenames{P} \setminus \{ y \}) \\
  \freenames{x!\langle P \rangle} & := & \{ x \} \cup \{ P \} \\
  \freenames{P|Q} & := & \freenames{P} \cup \freenames{Q} \\
  \freenames{\dropn{x}} & := & \{ x \}
\end{eqnarray*}

The bound names of a process, $\boundnames{P}$, are those names occurring in $P$
that are not free. For example, in $x?(y).0$, the name $x$ is free, while $y$ is bound.

\begin{mathpar}
  \inferrule* [lab=monoidal-laws] {} { P|Q \equiv Q|P \and P|0 \equiv P \and P|(Q|R) \equiv (P|Q)|R }
\end{mathpar}

\begin{mathpar}
  \inferrule* [lab=alpha-equivalence] {} { (x)P \equiv (y)P\{y/x\} \and y \not\in \freenames{P} }
\end{mathpar}

\begin{definition}
Then two processes, $P,Q$, are alpha-equivalent if $P = Q\{\vec{y}/\vec{x}\}$ for
some $\vec{x} \in \boundnames{Q},\vec{y} \in \boundnames{P}$, where $Q\{\vec{y}/\vec{x}\}$
denotes the capture-avoiding substitution of $\vec{y}$ for $\vec{x}$ in $Q$.
\end{definition}

\begin{definition}
  The {\em structural congruence} \cite{SangiorgiWalker} , $\equiv$,
  between processes is the least congruence containing
  alpha-equivalence, satisfying the abelian monoid laws
  (associativity, commutativity and $\pzero$ as identity) for parallel
  composition $|$ and for summation $+$.
\end{definition}

\subsection{Name equivalence}

We take name equivalence, written $\nameeq$, to be the smallest
equivalence relation generated by the following rules.

\begin{mathpar}
\inferrule*[lab=Quote-drop]
{ }
{ \quotep{@{x}} \nameeq x }

\inferrule*[lab=Struct-equiv]
{ P \scong Q }
{ \quotep{P} \nameeq \quotep{Q} }
\end{mathpar}

The astute reader will have noticed that the mutual recursion of names
and processes imposes a mutual recursion on alpha-equivalence and
structural equivalence via name-equivalence. Fortunately, all of this
works out pleasantly and we may calculate in the natural way, free of
concern. The reader interested in the details is referred to the
appendix \ref{appendix:rho_details}.

\subsection{Substitution}

We use $\Proc$ for the set of processes, $\QProc$ for the set of
names, and $\id{\{}\vec{y} / \vec{x} \id{\}}$ to denote partial maps,
$s : \QProc \rightarrow \QProc$. A map, $s$ lifts, uniquely, to a map
on process terms, $\widehat{s} : \Proc \rightarrow \Proc$ by the
following equations.

\begin{mathpar}
  (0) \psubstp{Q}{P} := 0 \\
  (R \juxtap S) \psubstp{Q}{P}
  :=    
  (R)\psubstp{Q}{P} \juxtap (S) \psubstp{Q}{P} \\
  (x?(y).R) \psubstp{Q}{P}    
  :=    
  (x)\substp{Q}{P} (z)\concat( (R \psubstn{z}{y}) \psubstp{Q}{P} ) \\
  (\lift{x}{R}) \psubstp{Q}{P}  
  :=
  \lift{(x)\substp{Q}{P}}{ R \psubstp{Q}{P} } \\
%   (\dropn{x})  \psubstp{Q}{P}       
%   := 
%   \left\{ 
%     \begin{array}{ccc} 
%       \dropn{\quotep{Q}} & & x \nameeq \quotep{P} \\
%       \dropn{x} & & otherwise \\
%     \end{array}
%   \right. 
  (\dropn{x})  \psubstp{Q}{P}       
  := 
  \left\{ 
    \begin{array}{ccc} 
      Q & & x \nameeq \quotep{P} \\
      \dropn{x} & & otherwise \\
    \end{array}
  \right.
\end{mathpar}
 

where

\begin{eqnarray}
  (x)\id{\{} \lpquote Q \rpquote / \lpquote P \rpquote \id{\}}            = 
  \left\{ 
    \begin{array}{ccc}
      \lpquote Q \rpquote & & x \nameeq \lpquote P \rpquote \\
      x & & otherwise \\
    \end{array}
  \right. \nonumber
\end{eqnarray}

and $z$ is chosen distinct from $\quotep{P}$, $\quotep{Q}$, the free
names in $Q$, and all the names in $R$. Our $\alpha$-equivalence will
be built in the standard way from this substitution.

\begin{remark}\label{rem:no_self_referential_names}
  One consequence of these definitions is that $\forall P. \quotep{P}
  \not\in \freenames{P}$.
\end{remark}

\subsection{ Dynamic quote: an example }

Anticipating something of what's to come, consider applying the
substitution, $\widehat{\id{\{}u / z \id{\}}}$, to the following pair
of processes, $\lift{w}{y!(z)}$ and $w[ \lpquote y!(z) \rpquote ]$.

\begin{eqnarray}
	\lift{w}{y!(z)}\widehat{\id{\{}u / z \id{\}}}
		& = &
		\lift{w}{y!(u)} \nonumber\\
	w[ \lpquote y!(z) \rpquote ] \widehat{ \id{\{}u / z \id{\}} }
		& = &
		w[ \lpquote y!(z) \rpquote ] \nonumber
\end{eqnarray}

Because the body of the process between quotes is impervious to
substitution, we get radically different answers. In fact, by
examining the first process in an input context,
e.g. $x?(z).\lift{w}{y!(z)}$, we see that the process under the lift
operator may be shaped by prefixed inputs binding a name inside it. In
this sense, the lift operator will be seen as a way to dynamically
construct processes before reifying them as names.

Finally equipped with these standard features we can present the
dynamics of the calculus.

\subsubsection{Operational semantics} 

Finally, we introduce the computational dynamics. What marks these
algebras as distinct from other more traditionally studied algebraic
structures, e.g. vector spaces or polynomial rings, is the manner in
which dynamics is captured. In traditional structures, dynamics is typically
expressed through morphisms between such structures, as in linear maps
between vector spaces or morphisms between rings. In algebras
associated with the semantics of computation, the dynamics is
expressed as part of the algebraic structure itself, through a
reduction reduction relation typically denoted by $\red$. Below, we
give a recursive presentation of this relation for the calculus used
in the encoding.

$\red \subseteq \pi \times \pi$
$\red : \pi \to \mathcal{P}(\pi)$

\begin{mathpar}
  \inferrule* [lab=Comm] { \textsf{match}( x_{src}, x_{trgt} ) } { x_{trgt}?(y)P \; | \; x_{src}!\langle {Q} \rangle \red P\{\quotep{Q}/y}\} }
  \and \\
  \inferrule* [lab=Par] {{P} \red {P}'} {{{P} | {Q}} \red {{P}' | {Q}}}
  \and
  \inferrule* [lab=Equiv]{{{P} \scong {P}'} \andalso {{P}' \red {Q}'} \andalso {{Q}' \scong {Q}}}{{P} \red {Q}}
\end{mathpar}

\begin{eqnarray*}
  match_{\equiv} (\quotep{P},\quotep{Q}) & := & P \equiv Q \\
  match_{\dagger}(\quotep{P},\quotep{Q}) & := & \forall R. P|Q \red^{*} R => R \red^{*} 0 \\
  match_{K}(\quotep{P},\quotep{Q}) & := & K \mbox{ for some context } K
\end{eqnarray*}

$u?(x)P | u!\langle Q \rangle \red P\{\quotep{Q}/x\}$

%We write $\wred$ for $\red^*$, and $P\red$ if $\exists Q $ such that $ P \red Q$.
We write $P\red$ if $\exists Q $ such that $ P \red Q$ and $P\not\red$, otherwise.

\section{Replication}

As mentioned before, it is known that replication (and hence
recursion) can be implemented in a higher-order process algebra
\cite{SangiorgiWalker}. As our first example of calculation with the
machinery thus far presented we give the construction explicitly in
the {\rhoc}.

\begin{eqnarray}
	D_{x} & := & \prefix{x}{y}{(\binpar{\outputp{x}{y}}{@{y}})} \nonumber\\
	\bangp_{x}{P} & := & \binpar{{x}!\langle{\binpar{D_{x}}{P}}\rangle}{D_{x}} \nonumber
\end{eqnarray}

\begin{eqnarray}
	\bangp_{x}{P} & & \nonumber\\
	=
	& {x}!\langle{(\prefix{x}{y}{(\outputp{x}{y} | @{y})) | P}}\rangle 
	      | \prefix{x}{y}{(\outputp{x}{y} | @{y})} & \nonumber\\
	\red
	& (\outputp{x}{y} | @{y})\substn{\quotep{(\prefix{x}{y}{(@{y} | \outputp{x}{y})) | P}}}{y} & \nonumber\\
	=
	& \outputp{x}{\quotep{(\prefix{x}{y}{(\outputp{x}{y} | @{y})) | P}}}
	  | {(\prefix{x}{y}{(\outputp{x}{y} | @{y})) | P}} & \nonumber\\
	\red
	& \ldots & \nonumber\\
	\red^*
	& P | P | \ldots & \nonumber
\end{eqnarray}

Of course, this encoding, as an implementation, runs away, unfolding
$\bangp{P}$ eagerly. A lazier and more implementable replication
operator, restricted to input-guarded processes, may be obtained as follows.

\begin{eqnarray}
\bangp{\prefix{u}{v}{P}} 
	:= 
	\binpar{\lift{x}{\prefix{u}{v}{(\binpar{D(x)}{P})}}}{D(x)} \nonumber
\end{eqnarray}

\begin{remark}
  Note that the lazier definition still does not deal with summation
  or mixed summation (i.e. sums over input and output). The reader is
  invited to construct definitions of replication that deal with these
  features. 

  Further, the definitions are parameterized in a name, $x$. Can you,
  gentle reader, make a definition that eliminates this parameter and
  guarantees no accidental interaction between the replication
  machinery and the process being replicated -- i.e. no accidental
  sharing of names used by the process to get its work done and the
  name(s) used by the replication to effect copying. This latter
  revision of the definition of replication is crucial to obtaining
  the expected identity $!!P \sim !P$.
\end{remark}

\begin{remark}\label{rem:paradoxical_combinator}
  The reader familiar with the lambda calculus will have noticed the
  similarity between $D$ and the paradoxical combinator.

  [Ed. note: the existence of this seems to suggest we have to be more
  restrictive on the set of processes and names we admit if we are to
  support no-cloning.]
\end{remark}

\subsubsection{Bisimulation}

The computational dynamics gives rise to another kind of equivalence,
the equivalence of computational behavior. As previously mentioned
this is typically captured \emph{via} some form of bisimulation.

% The notion we use in this paper is weak barbed bisimulation
% \cite{milner91polyadicpi}.

The notion we use in this paper is derived from weak barbed
bisimulation \cite{milner91polyadicpi}. 

\begin{definition}
An \emph{observation relation}, $\downarrow_{\mathcal N}$, over a set
of names, $\mathcal N$, is the smallest relation satisfying the rules
below.

\infrule[Out-barb]{y \in {\mathcal N}, \; x \nameeq y}
		  {\outputp{x}{v} \downarrow_{\mathcal N} x}
\infrule[Par-barb]{\mbox{$P\downarrow_{\mathcal N} x$ or $Q\downarrow_{\mathcal N} x$}}
		  {\binpar{P}{Q} \downarrow_{\mathcal N} x}

We write $P \Downarrow_{\mathcal N} x$ if there is $Q$ such that 
$P \wred Q$ and $Q \downarrow_{\mathcal N} x$.
\end{definition}

\begin{definition}
%\label{def.bbisim}
An  ${\mathcal N}$-\emph{barbed bisimulation} over a set of names, ${\mathcal N}$, is a symmetric binary relation 
${\mathcal S}_{\mathcal N}$ between agents such that $P\rel{S}_{\mathcal N}Q$ implies:
\begin{enumerate}
\item If $P \red P'$ then $Q \wred Q'$ and $P'\rel{S}_{\mathcal N} Q'$.
\item If $P\downarrow_{\mathcal N} x$, then $Q\Downarrow_{\mathcal N} x$.
\end{enumerate}
$P$ is ${\mathcal N}$-barbed bisimilar to $Q$, written
$P \wbbisim_{\mathcal N} Q$, if $P \rel{S}_{\mathcal N} Q$ for some ${\mathcal N}$-barbed bisimulation ${\mathcal S}_{\mathcal N}$.
\end{definition}

$\mathcal{R} \subseteq \pi \times \pi$

$P \mathcal{R} Q => \forall P'. P \red P' \Rightarrow \exists Q'. Q \red Q', P' \mathcal{R} Q'$

$P \vdash x \Rightarrow Q \vdash x$

\begin{mathpar}
  \inferrule*[lab=Out-barb]{x \nameeq y}{{y}!\langle{Q}\rangle \vdash x}
  \and
  \inferrule*[lab=Par-barb]{\mbox{$P\vdash x$ or $Q\vdash x$}}{\binpar{P}{Q} \vdash x}
\end{mathpar}

\subsubsection{Contexts}

One of the principle advantages of computational calculi like the
$\pi$-calculus is a well-defined notion of context,
contextual-equivalence and a correlation between
contextual-equivalence and notions of bisimulation. The notion of
context allows the decomposition of a process into (sub-)process and
its syntactic environment, its context. Thus, a context may be
thought of as a process with a ``hole'' (written $\Box$) in it. The
application of a context $M$ to a process $P$, written $M[P]$, is
tantamount to filling the hole in $M$ with $P$. In this paper we do
not need the full weight of this theory, but do make use of the notion
of context in the proof the main theorem. 

\begin{mathpar}
  \inferrule* [lab=summation] {} {{M_{M},M_{N}} \bc \Box \;|\; x.M_{A} \;|\; M_{M}+M_{N}}
  \and
  \inferrule* [lab=agent] {} {{M_{A}} \bc (\vec{x})M_{P} \;| \; \clift{P_0,\ldots,M_{P},\ldots,P_N}}
  \and \\
  \inferrule* [lab=process] {} {{M_{P}} \bc M_{N} \;| \;P|M_{P} }
\end{mathpar} 

\begin{mathpar}
  \inferrule* [lab=sychronization] {} {M_{N} \bc \Box \;|\; x?M_{F} \;|\; x!M_{C}}
  \and
  \inferrule* [lab=abstraction] {} {{M_{F}} \bc (x)M_{P} }
  \and
  \inferrule* [lab=concretion] {} {{M_{C}} \bc \langle M_{P} \rangle }
  \and \\
  \inferrule* [lab=process] {} {{M_{P}} \bc M_{N} \;| \;P|M_{P} }
\end{mathpar}

\begin{definition}[contextual application] Given a context $M$, and
  process $P$, we define the \emph{contextual application}, $M[P] :=
  M\{P/\Box\}$. That is, the contextual application of M to P is the
  substitution of $P$ for $\Box$ in $M$.
\end{definition}

$\meaningof{-} : L \to \mathcal{P}(\pi)$

\begin{mathpar}
  \inferrule* [lab=collection] {} {\meaningof{true} = \pi, \and \meaningof{~E} = \pi \setminus \meaningof{E}, \and \meaningof{E_{1} \& E_{2}} = \meaningof{E_{1}} \cap \meaningof{E_{2}}}
\end{mathpar}

\begin{mathpar}
  \inferrule* [lab=structure] {} {\meaningof{0} = \{ P \in \pi | P \equiv 0 \}, \and \\ \meaningof{E_1 | E_2} = \{ P \in \pi | P \equiv P_{1} | P_{2}, P_{1} \in \meaningof{E_{1}}, P_{2} \in \meaningof{E_2}\} }
\end{mathpar}

\begin{mathpar}
 \inferrule* [lab=behavior] {} {\meaningof{\langle a?b \rangle E} = \{ P \in \pi | P \equiv Q | u?(y)P', \\ \and \\\\ \and \\ \;\;\; u \in \meaningof{a}, \forall z.P'\{z/y\} \in \meaningof{E\{z/b\}}\}, \and \\ \meaningof{a!E} = \{ P \in \pi | P \equiv Q | x!\langle P' \rangle, x \in \meaningof{a} P' \in \meaningof{E}\} }
\end{mathpar}

\begin{mathpar}
 \inferrule* [lab=nominal] {} {\meaningof{\quotep{E}} = \{ \quotep{P} \in \quotep{\pi} | P \in \meaningof{E} \}, \and \meaningof{\quotep{P}} = \{ \quotep{Q} \in \quotep{\pi} | P \equiv Q \} \and \\ \meaningof{@\quotep{E}} = \{ P \in \pi | P \equiv @x, x \in \meaningof{E} \}}
\end{mathpar}

\begin{eqnarray*}
  \\
  \meaningof{-} : TS \to ST
\end{eqnarray*}

\begin{eqnarray*}
  \\
  L : TS \to ST
\end{eqnarray*}

\begin{eqnarray*}
  \\
  P \models E \iff P \in \meaningof{E}
\end{eqnarray*}

\begin{eqnarray*}
  P \approx_{L} Q \iff \forall E \in L. P \models E \iff Q \models E
\end{eqnarray*}

\begin{eqnarray*}
  P \approx_{K} Q
\end{eqnarray*}

\begin{eqnarray*}
  P \approx Q
\end{eqnarray*}

$\approx_{K} = \approx = \approx_{L}$

\subsubsection{Contextual duality}

Note that contexts extend the quotation operation to a family of
operations from processes to names. Given a context, $M$, we can
define a \emph{nominal context}, $\quotep{M}$ by $\quotep{M}[P] :=
\quotep{M[P]}$. To foreshadow what is to come we observe that these
operations enjoy a duality with processes very much like the duality
between vectors and maps from vectors to scalars.

Further, because the calculus is essentially higher-order, we have a
correspondence between contexts and processes. More specifically,
given a name $x$ and a context $M$ we can construct $M^{*}_{x}$ such
that 

\begin{mathpar}
  M^{*}_{x} | \lift{x}{P} \red M[P]
\end{mathpar}

namely,

\begin{mathpar}
  M^{*}_{x} := x?(u).M[\dropn{u}]
\end{mathpar}

The dependence of $M^{*}_{x}$ on a name makes it an abstraction, 

\begin{mathpar}
  M^{*} := (x)x?(u).M[\dropn{u}]
\end{mathpar}

\subsection{Additional notation}

It will sometimes be convenient to denote the process a name
quotes. We already have the notation $x = \quotep{P}$, but it will be
convenient to introduce an alternate notation, $\procn{x}$, when we
want to emphasize the connection to the use of the name. Note that, by
virtue of name equivalence, $\quotep{\procn{x}} \nameeq x$; so, the
notation is consistent with previous definitions.

Further, because names have structure it is possible to effect
substitutions on the basis of that structure. This means we need to
upgrade our notation for substitutions, which we accomplish by
adapting comprehension notation. Thus,

\begin{mathpar}
  P\{ y / x : x \in S \}
\end{mathpar}

is interpreted to mean the process derived from P by replacing (in a
capture-avoiding manner) each occurrence of $x$ in $S$ by $y$. For example,

\begin{mathpar}
  P\{ \quotep{\procn{x}|\procn{x}} / x : x \in \freenames{P} \}
\end{mathpar}

will replace each (occurrence) of a free name $x$ in $P$ by
$\quotep{\procn{x}|\procn{x}}$.

Also, we will avail ourselves of the notation $x^{L}$ and $x^{R}$ to
denote injections of a name into disjoint copies of the name
space. There are numerous ways to accomplish this. One example can be
found in \cite{MeredithR05}. This notation overloads to vectors of
names: $\vec{x}^{\pi} := (x_{i}^{\pi} \; : \; 0 \leq i < |\vec{x}| )$ where $\pi \in \{L,R\}$.

We also use $P^{\Box} := P|\Box$.

In \cite{MeredithR05} an interpretation of the new operator is
given. It turns out that there are several possible interpretations
all enjoying the requisite algebraic properties of the operator (see
\cite{milner91polyadicpi}). We will therefore make liberal use of
$(\nu\; \vec{x})P$.

% subsection the_syntax_and_semantics_of_the_notation_system (end)   

\input{qm2pi.qmops} 

\input{qm2pi.sterngerlach} 

\input{qm2pi.metric} 

% section concurrent_process_calculi (end)

%\input{qm2pi.proofsketch}

% section proof sketch (end)

%\input{qm2pi.slviaknots} 

% section spatial logic via knots (end)

\input{qm2pi.conclusion}

% section conclusion (end)

%\input{qm2pi.dtcodes} 

% section wiring algorithm (end)

\input{qm2pi.ack} 

% section acknowledgments (end)

\newpage


\bibliographystyle{plain}   
\bibliography{../../biblios/main.bib}

\input{qm2pi.rhodetails}

\end{document}

 

%\documentclass[12pt]{llncs}
%\documentclass{jktr}

\usepackage[pdftex]{hyperref}                   
\usepackage {listings}
\usepackage {mathpartir}
\usepackage{bcprules}
%\usepackage{listings}
                       
\usepackage{graphicx} 
%\usepackage[margins=2.5cm,nohead,nofoot]{geometry}
%\usepackage{geometry}
\usepackage{amsfonts}
\usepackage{amstext}
\usepackage{latexsym}
\usepackage{amssymb}
\usepackage{color}


%\include{myPreamble}
\include{qm2pi.local} 

%\ifpdf
%\usepackage[pdftex]{graphicx}
%\else
%\usepackage{graphicx}
%\fi

 % \ifpdf
%  \usepackage{pdfsync}
%  \if


%\title{Brief Article}
%\author{David F. Snyder}
%\author{L.G. Meredith}

%\address{Dept. of Math., Texas State University--San Marcos, San Marcos, TX 78666}
       
\pagestyle{empty}


\begin{document}

\lstset{language=[Objective]Caml,frame=shadowbox}

\input{qm2pi.front}

% section front matter (end)

\input{qm2pi.intro} 
 
% section introduction (end)

% \input{qm2pi.knotations} 

% section notation (end)

\input{qm2pi.process.calculi} 

% section concurrent_process_calculi_and_spatial_logics_ (end)
    
%\input{qm2pi.knots2pi} 

%\input{qm2pi.trefoil} 

%\input{qm2pi.mainthm} 

% subsection basic_interpretation (end)

%\input{qm2pi.rho.presentation} 
\subsection{The syntax and semantics of the notation system}\label{sub:the_syntax_and_semantics_of_the_notation_system} % (fold)

We now summarize a technical presentation of the calculus that
embodies our theory of dynamics. The typical presentation of such a
calculus follows the style of giving generators and relations on
them. The grammar, below, describing term constructors, freely
generates the set of processes, $\Proc$. This set is then quotiented
by a relation known as structural congruence and it is over this set
that the notion of dynamics is expressed. This presentation is
essentially that of \cite{MeredithR05} with the addition of
polyadicity and summation. For readability we have relegated some of
the technical subtleties to an appendix.

\subsubsection{Process grammar}\label{subsub:process_grammar}

\begin{mathpar}
  \inferrule* [lab=synchronization] {} {{M} \bc \pzero \;|\; x?F \;|\; x!C }
  \and
  \inferrule* [lab=abstraction] {} {{F} \bc (x)P}
  \and
  \inferrule* [lab=concretion] {} {{C} \bc \langle Q \rangle}
  \and
  \inferrule* [lab=process] {} {{P,Q} \bc M \;| \;P|Q \;|\; @{x}}
  \and
  \inferrule* [lab=name] {} {{x} \bc \quotep{P}}
\end{mathpar} 

Note that $\vec{x}$ (resp. $\vec{P}$) denotes a vector of names
(resp. processes) of length $|\vec{x}|$ (resp. $|\vec{P}|$). We adopt
the following useful abbreviations.

\begin{mathpar}
   x?(\vec{y}).P := x.(\vec{y})P \and  x\clift{\vec{P}} := x.\clift{\vec{P}}
   \and x!(y) := \lift{x}{\dropn{y}}
   \and \Pi_{i=0}^{n-1}P_i := P_0 | \ldots | P_{n-1}
\end{mathpar}

\subsubsection{Structural congruence}

\paragraph{Free and bound names and alpha-equivalence.} At the
core of structural equivalence is alpha-equivalence which identifies
process that are the same up to a change of variable. Formally, we
recognize the distinction between free and bound names. The free names
of a process, $\freenames{P}$, may be calculated recursively as
follows:

\begin{mathpar}
\freenames{\pzero} := \emptyset
  \and \\
  \freenames{x?(y).P} := \{ x \} \cup (\freenames{P} \setminus \{ y \})
  \and 
  \freenames{x!\langle P \rangle} := \{ x \} \cup \{ P \} 
  \and \\
  \freenames{P|Q} := \freenames{P} \cup \freenames{Q}
  \and \\
  \freenames{@{x}} := \{ x \}
\end{mathpar}

$\pi$
$\quotep{\pi}$

$\freenames{-} : \pi \to \mathcal{P}(\quotep{\pi})$

\begin{eqnarray*}
  \freenames{\pzero} & := & \emptyset \\
  \freenames{x?(y).P} & := & \{ x \} \cup (\freenames{P} \setminus \{ y \}) \\
  \freenames{x!\langle P \rangle} & := & \{ x \} \cup \{ P \} \\
  \freenames{P|Q} & := & \freenames{P} \cup \freenames{Q} \\
  \freenames{\dropn{x}} & := & \{ x \}
\end{eqnarray*}

The bound names of a process, $\boundnames{P}$, are those names occurring in $P$
that are not free. For example, in $x?(y).0$, the name $x$ is free, while $y$ is bound.

\begin{mathpar}
  \inferrule* [lab=monoidal-laws] {} { P|Q \equiv Q|P \and P|0 \equiv P \and P|(Q|R) \equiv (P|Q)|R }
\end{mathpar}

\begin{mathpar}
  \inferrule* [lab=alpha-equivalence] {} { (x)P \equiv (y)P\{y/x\} \and y \not\in \freenames{P} }
\end{mathpar}

\begin{definition}
Then two processes, $P,Q$, are alpha-equivalent if $P = Q\{\vec{y}/\vec{x}\}$ for
some $\vec{x} \in \boundnames{Q},\vec{y} \in \boundnames{P}$, where $Q\{\vec{y}/\vec{x}\}$
denotes the capture-avoiding substitution of $\vec{y}$ for $\vec{x}$ in $Q$.
\end{definition}

\begin{definition}
  The {\em structural congruence} \cite{SangiorgiWalker} , $\equiv$,
  between processes is the least congruence containing
  alpha-equivalence, satisfying the abelian monoid laws
  (associativity, commutativity and $\pzero$ as identity) for parallel
  composition $|$ and for summation $+$.
\end{definition}

\subsection{Name equivalence}

We take name equivalence, written $\nameeq$, to be the smallest
equivalence relation generated by the following rules.

\begin{mathpar}
\inferrule*[lab=Quote-drop]
{ }
{ \quotep{@{x}} \nameeq x }

\inferrule*[lab=Struct-equiv]
{ P \scong Q }
{ \quotep{P} \nameeq \quotep{Q} }
\end{mathpar}

The astute reader will have noticed that the mutual recursion of names
and processes imposes a mutual recursion on alpha-equivalence and
structural equivalence via name-equivalence. Fortunately, all of this
works out pleasantly and we may calculate in the natural way, free of
concern. The reader interested in the details is referred to the
appendix \ref{appendix:rho_details}.

\subsection{Substitution}

We use $\Proc$ for the set of processes, $\QProc$ for the set of
names, and $\id{\{}\vec{y} / \vec{x} \id{\}}$ to denote partial maps,
$s : \QProc \rightarrow \QProc$. A map, $s$ lifts, uniquely, to a map
on process terms, $\widehat{s} : \Proc \rightarrow \Proc$ by the
following equations.

\begin{mathpar}
  (0) \psubstp{Q}{P} := 0 \\
  (R \juxtap S) \psubstp{Q}{P}
  :=    
  (R)\psubstp{Q}{P} \juxtap (S) \psubstp{Q}{P} \\
  (x?(y).R) \psubstp{Q}{P}    
  :=    
  (x)\substp{Q}{P} (z)\concat( (R \psubstn{z}{y}) \psubstp{Q}{P} ) \\
  (\lift{x}{R}) \psubstp{Q}{P}  
  :=
  \lift{(x)\substp{Q}{P}}{ R \psubstp{Q}{P} } \\
%   (\dropn{x})  \psubstp{Q}{P}       
%   := 
%   \left\{ 
%     \begin{array}{ccc} 
%       \dropn{\quotep{Q}} & & x \nameeq \quotep{P} \\
%       \dropn{x} & & otherwise \\
%     \end{array}
%   \right. 
  (\dropn{x})  \psubstp{Q}{P}       
  := 
  \left\{ 
    \begin{array}{ccc} 
      Q & & x \nameeq \quotep{P} \\
      \dropn{x} & & otherwise \\
    \end{array}
  \right.
\end{mathpar}
 

where

\begin{eqnarray}
  (x)\id{\{} \lpquote Q \rpquote / \lpquote P \rpquote \id{\}}            = 
  \left\{ 
    \begin{array}{ccc}
      \lpquote Q \rpquote & & x \nameeq \lpquote P \rpquote \\
      x & & otherwise \\
    \end{array}
  \right. \nonumber
\end{eqnarray}

and $z$ is chosen distinct from $\quotep{P}$, $\quotep{Q}$, the free
names in $Q$, and all the names in $R$. Our $\alpha$-equivalence will
be built in the standard way from this substitution.

\begin{remark}\label{rem:no_self_referential_names}
  One consequence of these definitions is that $\forall P. \quotep{P}
  \not\in \freenames{P}$.
\end{remark}

\subsection{ Dynamic quote: an example }

Anticipating something of what's to come, consider applying the
substitution, $\widehat{\id{\{}u / z \id{\}}}$, to the following pair
of processes, $\lift{w}{y!(z)}$ and $w[ \lpquote y!(z) \rpquote ]$.

\begin{eqnarray}
	\lift{w}{y!(z)}\widehat{\id{\{}u / z \id{\}}}
		& = &
		\lift{w}{y!(u)} \nonumber\\
	w[ \lpquote y!(z) \rpquote ] \widehat{ \id{\{}u / z \id{\}} }
		& = &
		w[ \lpquote y!(z) \rpquote ] \nonumber
\end{eqnarray}

Because the body of the process between quotes is impervious to
substitution, we get radically different answers. In fact, by
examining the first process in an input context,
e.g. $x?(z).\lift{w}{y!(z)}$, we see that the process under the lift
operator may be shaped by prefixed inputs binding a name inside it. In
this sense, the lift operator will be seen as a way to dynamically
construct processes before reifying them as names.

Finally equipped with these standard features we can present the
dynamics of the calculus.

\subsubsection{Operational semantics} 

Finally, we introduce the computational dynamics. What marks these
algebras as distinct from other more traditionally studied algebraic
structures, e.g. vector spaces or polynomial rings, is the manner in
which dynamics is captured. In traditional structures, dynamics is typically
expressed through morphisms between such structures, as in linear maps
between vector spaces or morphisms between rings. In algebras
associated with the semantics of computation, the dynamics is
expressed as part of the algebraic structure itself, through a
reduction reduction relation typically denoted by $\red$. Below, we
give a recursive presentation of this relation for the calculus used
in the encoding.

$\red \subseteq \pi \times \pi$
$\red : \pi \to \mathcal{P}(\pi)$

\begin{mathpar}
  \inferrule* [lab=Comm] { \textsf{match}( x_{src}, x_{trgt} ) } { x_{trgt}?(y)P \; | \; x_{src}!\langle {Q} \rangle \red P\{\quotep{Q}/y}\} }
  \and \\
  \inferrule* [lab=Par] {{P} \red {P}'} {{{P} | {Q}} \red {{P}' | {Q}}}
  \and
  \inferrule* [lab=Equiv]{{{P} \scong {P}'} \andalso {{P}' \red {Q}'} \andalso {{Q}' \scong {Q}}}{{P} \red {Q}}
\end{mathpar}

\begin{eqnarray*}
  match_{\equiv} (\quotep{P},\quotep{Q}) & := & P \equiv Q \\
  match_{\dagger}(\quotep{P},\quotep{Q}) & := & \forall R. P|Q \red^{*} R => R \red^{*} 0 \\
  match_{K}(\quotep{P},\quotep{Q}) & := & K \mbox{ for some context } K
\end{eqnarray*}

$u?(x)P | u!\langle Q \rangle \red P\{\quotep{Q}/x\}$

%We write $\wred$ for $\red^*$, and $P\red$ if $\exists Q $ such that $ P \red Q$.
We write $P\red$ if $\exists Q $ such that $ P \red Q$ and $P\not\red$, otherwise.

\section{Replication}

As mentioned before, it is known that replication (and hence
recursion) can be implemented in a higher-order process algebra
\cite{SangiorgiWalker}. As our first example of calculation with the
machinery thus far presented we give the construction explicitly in
the {\rhoc}.

\begin{eqnarray}
	D_{x} & := & \prefix{x}{y}{(\binpar{\outputp{x}{y}}{@{y}})} \nonumber\\
	\bangp_{x}{P} & := & \binpar{{x}!\langle{\binpar{D_{x}}{P}}\rangle}{D_{x}} \nonumber
\end{eqnarray}

\begin{eqnarray}
	\bangp_{x}{P} & & \nonumber\\
	=
	& {x}!\langle{(\prefix{x}{y}{(\outputp{x}{y} | @{y})) | P}}\rangle 
	      | \prefix{x}{y}{(\outputp{x}{y} | @{y})} & \nonumber\\
	\red
	& (\outputp{x}{y} | @{y})\substn{\quotep{(\prefix{x}{y}{(@{y} | \outputp{x}{y})) | P}}}{y} & \nonumber\\
	=
	& \outputp{x}{\quotep{(\prefix{x}{y}{(\outputp{x}{y} | @{y})) | P}}}
	  | {(\prefix{x}{y}{(\outputp{x}{y} | @{y})) | P}} & \nonumber\\
	\red
	& \ldots & \nonumber\\
	\red^*
	& P | P | \ldots & \nonumber
\end{eqnarray}

Of course, this encoding, as an implementation, runs away, unfolding
$\bangp{P}$ eagerly. A lazier and more implementable replication
operator, restricted to input-guarded processes, may be obtained as follows.

\begin{eqnarray}
\bangp{\prefix{u}{v}{P}} 
	:= 
	\binpar{\lift{x}{\prefix{u}{v}{(\binpar{D(x)}{P})}}}{D(x)} \nonumber
\end{eqnarray}

\begin{remark}
  Note that the lazier definition still does not deal with summation
  or mixed summation (i.e. sums over input and output). The reader is
  invited to construct definitions of replication that deal with these
  features. 

  Further, the definitions are parameterized in a name, $x$. Can you,
  gentle reader, make a definition that eliminates this parameter and
  guarantees no accidental interaction between the replication
  machinery and the process being replicated -- i.e. no accidental
  sharing of names used by the process to get its work done and the
  name(s) used by the replication to effect copying. This latter
  revision of the definition of replication is crucial to obtaining
  the expected identity $!!P \sim !P$.
\end{remark}

\begin{remark}\label{rem:paradoxical_combinator}
  The reader familiar with the lambda calculus will have noticed the
  similarity between $D$ and the paradoxical combinator.

  [Ed. note: the existence of this seems to suggest we have to be more
  restrictive on the set of processes and names we admit if we are to
  support no-cloning.]
\end{remark}

\subsubsection{Bisimulation}

The computational dynamics gives rise to another kind of equivalence,
the equivalence of computational behavior. As previously mentioned
this is typically captured \emph{via} some form of bisimulation.

% The notion we use in this paper is weak barbed bisimulation
% \cite{milner91polyadicpi}.

The notion we use in this paper is derived from weak barbed
bisimulation \cite{milner91polyadicpi}. 

\begin{definition}
An \emph{observation relation}, $\downarrow_{\mathcal N}$, over a set
of names, $\mathcal N$, is the smallest relation satisfying the rules
below.

\infrule[Out-barb]{y \in {\mathcal N}, \; x \nameeq y}
		  {\outputp{x}{v} \downarrow_{\mathcal N} x}
\infrule[Par-barb]{\mbox{$P\downarrow_{\mathcal N} x$ or $Q\downarrow_{\mathcal N} x$}}
		  {\binpar{P}{Q} \downarrow_{\mathcal N} x}

We write $P \Downarrow_{\mathcal N} x$ if there is $Q$ such that 
$P \wred Q$ and $Q \downarrow_{\mathcal N} x$.
\end{definition}

\begin{definition}
%\label{def.bbisim}
An  ${\mathcal N}$-\emph{barbed bisimulation} over a set of names, ${\mathcal N}$, is a symmetric binary relation 
${\mathcal S}_{\mathcal N}$ between agents such that $P\rel{S}_{\mathcal N}Q$ implies:
\begin{enumerate}
\item If $P \red P'$ then $Q \wred Q'$ and $P'\rel{S}_{\mathcal N} Q'$.
\item If $P\downarrow_{\mathcal N} x$, then $Q\Downarrow_{\mathcal N} x$.
\end{enumerate}
$P$ is ${\mathcal N}$-barbed bisimilar to $Q$, written
$P \wbbisim_{\mathcal N} Q$, if $P \rel{S}_{\mathcal N} Q$ for some ${\mathcal N}$-barbed bisimulation ${\mathcal S}_{\mathcal N}$.
\end{definition}

$\mathcal{R} \subseteq \pi \times \pi$

$P \mathcal{R} Q => \forall P'. P \red P' \Rightarrow \exists Q'. Q \red Q', P' \mathcal{R} Q'$

$P \vdash x \Rightarrow Q \vdash x$

\begin{mathpar}
  \inferrule*[lab=Out-barb]{x \nameeq y}{{y}!\langle{Q}\rangle \vdash x}
  \and
  \inferrule*[lab=Par-barb]{\mbox{$P\vdash x$ or $Q\vdash x$}}{\binpar{P}{Q} \vdash x}
\end{mathpar}

\subsubsection{Contexts}

One of the principle advantages of computational calculi like the
$\pi$-calculus is a well-defined notion of context,
contextual-equivalence and a correlation between
contextual-equivalence and notions of bisimulation. The notion of
context allows the decomposition of a process into (sub-)process and
its syntactic environment, its context. Thus, a context may be
thought of as a process with a ``hole'' (written $\Box$) in it. The
application of a context $M$ to a process $P$, written $M[P]$, is
tantamount to filling the hole in $M$ with $P$. In this paper we do
not need the full weight of this theory, but do make use of the notion
of context in the proof the main theorem. 

\begin{mathpar}
  \inferrule* [lab=summation] {} {{M_{M},M_{N}} \bc \Box \;|\; x.M_{A} \;|\; M_{M}+M_{N}}
  \and
  \inferrule* [lab=agent] {} {{M_{A}} \bc (\vec{x})M_{P} \;| \; \clift{P_0,\ldots,M_{P},\ldots,P_N}}
  \and \\
  \inferrule* [lab=process] {} {{M_{P}} \bc M_{N} \;| \;P|M_{P} }
\end{mathpar} 

\begin{mathpar}
  \inferrule* [lab=sychronization] {} {M_{N} \bc \Box \;|\; x?M_{F} \;|\; x!M_{C}}
  \and
  \inferrule* [lab=abstraction] {} {{M_{F}} \bc (x)M_{P} }
  \and
  \inferrule* [lab=concretion] {} {{M_{C}} \bc \langle M_{P} \rangle }
  \and \\
  \inferrule* [lab=process] {} {{M_{P}} \bc M_{N} \;| \;P|M_{P} }
\end{mathpar}

\begin{definition}[contextual application] Given a context $M$, and
  process $P$, we define the \emph{contextual application}, $M[P] :=
  M\{P/\Box\}$. That is, the contextual application of M to P is the
  substitution of $P$ for $\Box$ in $M$.
\end{definition}

$\meaningof{-} : L \to \mathcal{P}(\pi)$

\begin{mathpar}
  \inferrule* [lab=collection] {} {\meaningof{true} = \pi, \and \meaningof{~E} = \pi \setminus \meaningof{E}, \and \meaningof{E_{1} \& E_{2}} = \meaningof{E_{1}} \cap \meaningof{E_{2}}}
\end{mathpar}

\begin{mathpar}
  \inferrule* [lab=structure] {} {\meaningof{0} = \{ P \in \pi | P \equiv 0 \}, \and \\ \meaningof{E_1 | E_2} = \{ P \in \pi | P \equiv P_{1} | P_{2}, P_{1} \in \meaningof{E_{1}}, P_{2} \in \meaningof{E_2}\} }
\end{mathpar}

\begin{mathpar}
 \inferrule* [lab=behavior] {} {\meaningof{\langle a?b \rangle E} = \{ P \in \pi | P \equiv Q | u?(y)P', \\ \and \\\\ \and \\ \;\;\; u \in \meaningof{a}, \forall z.P'\{z/y\} \in \meaningof{E\{z/b\}}\}, \and \\ \meaningof{a!E} = \{ P \in \pi | P \equiv Q | x!\langle P' \rangle, x \in \meaningof{a} P' \in \meaningof{E}\} }
\end{mathpar}

\begin{mathpar}
 \inferrule* [lab=nominal] {} {\meaningof{\quotep{E}} = \{ \quotep{P} \in \quotep{\pi} | P \in \meaningof{E} \}, \and \meaningof{\quotep{P}} = \{ \quotep{Q} \in \quotep{\pi} | P \equiv Q \} \and \\ \meaningof{@\quotep{E}} = \{ P \in \pi | P \equiv @x, x \in \meaningof{E} \}}
\end{mathpar}

\begin{eqnarray*}
  \\
  \meaningof{-} : TS \to ST
\end{eqnarray*}

\begin{eqnarray*}
  \\
  L : TS \to ST
\end{eqnarray*}

\begin{eqnarray*}
  \\
  P \models E \iff P \in \meaningof{E}
\end{eqnarray*}

\begin{eqnarray*}
  P \approx_{L} Q \iff \forall E \in L. P \models E \iff Q \models E
\end{eqnarray*}

\begin{eqnarray*}
  P \approx_{K} Q
\end{eqnarray*}

\begin{eqnarray*}
  P \approx Q
\end{eqnarray*}

$\approx_{K} = \approx = \approx_{L}$

\subsubsection{Contextual duality}

Note that contexts extend the quotation operation to a family of
operations from processes to names. Given a context, $M$, we can
define a \emph{nominal context}, $\quotep{M}$ by $\quotep{M}[P] :=
\quotep{M[P]}$. To foreshadow what is to come we observe that these
operations enjoy a duality with processes very much like the duality
between vectors and maps from vectors to scalars.

Further, because the calculus is essentially higher-order, we have a
correspondence between contexts and processes. More specifically,
given a name $x$ and a context $M$ we can construct $M^{*}_{x}$ such
that 

\begin{mathpar}
  M^{*}_{x} | \lift{x}{P} \red M[P]
\end{mathpar}

namely,

\begin{mathpar}
  M^{*}_{x} := x?(u).M[\dropn{u}]
\end{mathpar}

The dependence of $M^{*}_{x}$ on a name makes it an abstraction, 

\begin{mathpar}
  M^{*} := (x)x?(u).M[\dropn{u}]
\end{mathpar}

\subsection{Additional notation}

It will sometimes be convenient to denote the process a name
quotes. We already have the notation $x = \quotep{P}$, but it will be
convenient to introduce an alternate notation, $\procn{x}$, when we
want to emphasize the connection to the use of the name. Note that, by
virtue of name equivalence, $\quotep{\procn{x}} \nameeq x$; so, the
notation is consistent with previous definitions.

Further, because names have structure it is possible to effect
substitutions on the basis of that structure. This means we need to
upgrade our notation for substitutions, which we accomplish by
adapting comprehension notation. Thus,

\begin{mathpar}
  P\{ y / x : x \in S \}
\end{mathpar}

is interpreted to mean the process derived from P by replacing (in a
capture-avoiding manner) each occurrence of $x$ in $S$ by $y$. For example,

\begin{mathpar}
  P\{ \quotep{\procn{x}|\procn{x}} / x : x \in \freenames{P} \}
\end{mathpar}

will replace each (occurrence) of a free name $x$ in $P$ by
$\quotep{\procn{x}|\procn{x}}$.

Also, we will avail ourselves of the notation $x^{L}$ and $x^{R}$ to
denote injections of a name into disjoint copies of the name
space. There are numerous ways to accomplish this. One example can be
found in \cite{MeredithR05}. This notation overloads to vectors of
names: $\vec{x}^{\pi} := (x_{i}^{\pi} \; : \; 0 \leq i < |\vec{x}| )$ where $\pi \in \{L,R\}$.

We also use $P^{\Box} := P|\Box$.

In \cite{MeredithR05} an interpretation of the new operator is
given. It turns out that there are several possible interpretations
all enjoying the requisite algebraic properties of the operator (see
\cite{milner91polyadicpi}). We will therefore make liberal use of
$(\nu\; \vec{x})P$.

% subsection the_syntax_and_semantics_of_the_notation_system (end)   

\input{qm2pi.qmops} 

\input{qm2pi.sterngerlach} 

\input{qm2pi.metric} 

% section concurrent_process_calculi (end)

%\input{qm2pi.proofsketch}

% section proof sketch (end)

%\input{qm2pi.slviaknots} 

% section spatial logic via knots (end)

\input{qm2pi.conclusion}

% section conclusion (end)

%\input{qm2pi.dtcodes} 

% section wiring algorithm (end)

\input{qm2pi.ack} 

% section acknowledgments (end)

\newpage


\bibliographystyle{plain}   
\bibliography{../../biblios/main.bib}

\input{qm2pi.rhodetails}

\end{document}

 

%\documentclass[12pt]{llncs}
%\documentclass{jktr}

\usepackage[pdftex]{hyperref}                   
\usepackage {listings}
\usepackage {mathpartir}
\usepackage{bcprules}
%\usepackage{listings}
                       
\usepackage{graphicx} 
%\usepackage[margins=2.5cm,nohead,nofoot]{geometry}
%\usepackage{geometry}
\usepackage{amsfonts}
\usepackage{amstext}
\usepackage{latexsym}
\usepackage{amssymb}
\usepackage{color}


%\include{myPreamble}
\include{qm2pi.local} 

%\ifpdf
%\usepackage[pdftex]{graphicx}
%\else
%\usepackage{graphicx}
%\fi

 % \ifpdf
%  \usepackage{pdfsync}
%  \if


%\title{Brief Article}
%\author{David F. Snyder}
%\author{L.G. Meredith}

%\address{Dept. of Math., Texas State University--San Marcos, San Marcos, TX 78666}
       
\pagestyle{empty}


\begin{document}

\lstset{language=[Objective]Caml,frame=shadowbox}

\input{qm2pi.front}

% section front matter (end)

\input{qm2pi.intro} 
 
% section introduction (end)

% \input{qm2pi.knotations} 

% section notation (end)

\input{qm2pi.process.calculi} 

% section concurrent_process_calculi_and_spatial_logics_ (end)
    
%\input{qm2pi.knots2pi} 

%\input{qm2pi.trefoil} 

%\input{qm2pi.mainthm} 

% subsection basic_interpretation (end)

%\input{qm2pi.rho.presentation} 
\subsection{The syntax and semantics of the notation system}\label{sub:the_syntax_and_semantics_of_the_notation_system} % (fold)

We now summarize a technical presentation of the calculus that
embodies our theory of dynamics. The typical presentation of such a
calculus follows the style of giving generators and relations on
them. The grammar, below, describing term constructors, freely
generates the set of processes, $\Proc$. This set is then quotiented
by a relation known as structural congruence and it is over this set
that the notion of dynamics is expressed. This presentation is
essentially that of \cite{MeredithR05} with the addition of
polyadicity and summation. For readability we have relegated some of
the technical subtleties to an appendix.

\subsubsection{Process grammar}\label{subsub:process_grammar}

\begin{mathpar}
  \inferrule* [lab=synchronization] {} {{M} \bc \pzero \;|\; x?F \;|\; x!C }
  \and
  \inferrule* [lab=abstraction] {} {{F} \bc (x)P}
  \and
  \inferrule* [lab=concretion] {} {{C} \bc \langle Q \rangle}
  \and
  \inferrule* [lab=process] {} {{P,Q} \bc M \;| \;P|Q \;|\; @{x}}
  \and
  \inferrule* [lab=name] {} {{x} \bc \quotep{P}}
\end{mathpar} 

Note that $\vec{x}$ (resp. $\vec{P}$) denotes a vector of names
(resp. processes) of length $|\vec{x}|$ (resp. $|\vec{P}|$). We adopt
the following useful abbreviations.

\begin{mathpar}
   x?(\vec{y}).P := x.(\vec{y})P \and  x\clift{\vec{P}} := x.\clift{\vec{P}}
   \and x!(y) := \lift{x}{\dropn{y}}
   \and \Pi_{i=0}^{n-1}P_i := P_0 | \ldots | P_{n-1}
\end{mathpar}

\subsubsection{Structural congruence}

\paragraph{Free and bound names and alpha-equivalence.} At the
core of structural equivalence is alpha-equivalence which identifies
process that are the same up to a change of variable. Formally, we
recognize the distinction between free and bound names. The free names
of a process, $\freenames{P}$, may be calculated recursively as
follows:

\begin{mathpar}
\freenames{\pzero} := \emptyset
  \and \\
  \freenames{x?(y).P} := \{ x \} \cup (\freenames{P} \setminus \{ y \})
  \and 
  \freenames{x!\langle P \rangle} := \{ x \} \cup \{ P \} 
  \and \\
  \freenames{P|Q} := \freenames{P} \cup \freenames{Q}
  \and \\
  \freenames{@{x}} := \{ x \}
\end{mathpar}

$\pi$
$\quotep{\pi}$

$\freenames{-} : \pi \to \mathcal{P}(\quotep{\pi})$

\begin{eqnarray*}
  \freenames{\pzero} & := & \emptyset \\
  \freenames{x?(y).P} & := & \{ x \} \cup (\freenames{P} \setminus \{ y \}) \\
  \freenames{x!\langle P \rangle} & := & \{ x \} \cup \{ P \} \\
  \freenames{P|Q} & := & \freenames{P} \cup \freenames{Q} \\
  \freenames{\dropn{x}} & := & \{ x \}
\end{eqnarray*}

The bound names of a process, $\boundnames{P}$, are those names occurring in $P$
that are not free. For example, in $x?(y).0$, the name $x$ is free, while $y$ is bound.

\begin{mathpar}
  \inferrule* [lab=monoidal-laws] {} { P|Q \equiv Q|P \and P|0 \equiv P \and P|(Q|R) \equiv (P|Q)|R }
\end{mathpar}

\begin{mathpar}
  \inferrule* [lab=alpha-equivalence] {} { (x)P \equiv (y)P\{y/x\} \and y \not\in \freenames{P} }
\end{mathpar}

\begin{definition}
Then two processes, $P,Q$, are alpha-equivalent if $P = Q\{\vec{y}/\vec{x}\}$ for
some $\vec{x} \in \boundnames{Q},\vec{y} \in \boundnames{P}$, where $Q\{\vec{y}/\vec{x}\}$
denotes the capture-avoiding substitution of $\vec{y}$ for $\vec{x}$ in $Q$.
\end{definition}

\begin{definition}
  The {\em structural congruence} \cite{SangiorgiWalker} , $\equiv$,
  between processes is the least congruence containing
  alpha-equivalence, satisfying the abelian monoid laws
  (associativity, commutativity and $\pzero$ as identity) for parallel
  composition $|$ and for summation $+$.
\end{definition}

\subsection{Name equivalence}

We take name equivalence, written $\nameeq$, to be the smallest
equivalence relation generated by the following rules.

\begin{mathpar}
\inferrule*[lab=Quote-drop]
{ }
{ \quotep{@{x}} \nameeq x }

\inferrule*[lab=Struct-equiv]
{ P \scong Q }
{ \quotep{P} \nameeq \quotep{Q} }
\end{mathpar}

The astute reader will have noticed that the mutual recursion of names
and processes imposes a mutual recursion on alpha-equivalence and
structural equivalence via name-equivalence. Fortunately, all of this
works out pleasantly and we may calculate in the natural way, free of
concern. The reader interested in the details is referred to the
appendix \ref{appendix:rho_details}.

\subsection{Substitution}

We use $\Proc$ for the set of processes, $\QProc$ for the set of
names, and $\id{\{}\vec{y} / \vec{x} \id{\}}$ to denote partial maps,
$s : \QProc \rightarrow \QProc$. A map, $s$ lifts, uniquely, to a map
on process terms, $\widehat{s} : \Proc \rightarrow \Proc$ by the
following equations.

\begin{mathpar}
  (0) \psubstp{Q}{P} := 0 \\
  (R \juxtap S) \psubstp{Q}{P}
  :=    
  (R)\psubstp{Q}{P} \juxtap (S) \psubstp{Q}{P} \\
  (x?(y).R) \psubstp{Q}{P}    
  :=    
  (x)\substp{Q}{P} (z)\concat( (R \psubstn{z}{y}) \psubstp{Q}{P} ) \\
  (\lift{x}{R}) \psubstp{Q}{P}  
  :=
  \lift{(x)\substp{Q}{P}}{ R \psubstp{Q}{P} } \\
%   (\dropn{x})  \psubstp{Q}{P}       
%   := 
%   \left\{ 
%     \begin{array}{ccc} 
%       \dropn{\quotep{Q}} & & x \nameeq \quotep{P} \\
%       \dropn{x} & & otherwise \\
%     \end{array}
%   \right. 
  (\dropn{x})  \psubstp{Q}{P}       
  := 
  \left\{ 
    \begin{array}{ccc} 
      Q & & x \nameeq \quotep{P} \\
      \dropn{x} & & otherwise \\
    \end{array}
  \right.
\end{mathpar}
 

where

\begin{eqnarray}
  (x)\id{\{} \lpquote Q \rpquote / \lpquote P \rpquote \id{\}}            = 
  \left\{ 
    \begin{array}{ccc}
      \lpquote Q \rpquote & & x \nameeq \lpquote P \rpquote \\
      x & & otherwise \\
    \end{array}
  \right. \nonumber
\end{eqnarray}

and $z$ is chosen distinct from $\quotep{P}$, $\quotep{Q}$, the free
names in $Q$, and all the names in $R$. Our $\alpha$-equivalence will
be built in the standard way from this substitution.

\begin{remark}\label{rem:no_self_referential_names}
  One consequence of these definitions is that $\forall P. \quotep{P}
  \not\in \freenames{P}$.
\end{remark}

\subsection{ Dynamic quote: an example }

Anticipating something of what's to come, consider applying the
substitution, $\widehat{\id{\{}u / z \id{\}}}$, to the following pair
of processes, $\lift{w}{y!(z)}$ and $w[ \lpquote y!(z) \rpquote ]$.

\begin{eqnarray}
	\lift{w}{y!(z)}\widehat{\id{\{}u / z \id{\}}}
		& = &
		\lift{w}{y!(u)} \nonumber\\
	w[ \lpquote y!(z) \rpquote ] \widehat{ \id{\{}u / z \id{\}} }
		& = &
		w[ \lpquote y!(z) \rpquote ] \nonumber
\end{eqnarray}

Because the body of the process between quotes is impervious to
substitution, we get radically different answers. In fact, by
examining the first process in an input context,
e.g. $x?(z).\lift{w}{y!(z)}$, we see that the process under the lift
operator may be shaped by prefixed inputs binding a name inside it. In
this sense, the lift operator will be seen as a way to dynamically
construct processes before reifying them as names.

Finally equipped with these standard features we can present the
dynamics of the calculus.

\subsubsection{Operational semantics} 

Finally, we introduce the computational dynamics. What marks these
algebras as distinct from other more traditionally studied algebraic
structures, e.g. vector spaces or polynomial rings, is the manner in
which dynamics is captured. In traditional structures, dynamics is typically
expressed through morphisms between such structures, as in linear maps
between vector spaces or morphisms between rings. In algebras
associated with the semantics of computation, the dynamics is
expressed as part of the algebraic structure itself, through a
reduction reduction relation typically denoted by $\red$. Below, we
give a recursive presentation of this relation for the calculus used
in the encoding.

$\red \subseteq \pi \times \pi$
$\red : \pi \to \mathcal{P}(\pi)$

\begin{mathpar}
  \inferrule* [lab=Comm] { \textsf{match}( x_{src}, x_{trgt} ) } { x_{trgt}?(y)P \; | \; x_{src}!\langle {Q} \rangle \red P\{\quotep{Q}/y}\} }
  \and \\
  \inferrule* [lab=Par] {{P} \red {P}'} {{{P} | {Q}} \red {{P}' | {Q}}}
  \and
  \inferrule* [lab=Equiv]{{{P} \scong {P}'} \andalso {{P}' \red {Q}'} \andalso {{Q}' \scong {Q}}}{{P} \red {Q}}
\end{mathpar}

\begin{eqnarray*}
  match_{\equiv} (\quotep{P},\quotep{Q}) & := & P \equiv Q \\
  match_{\dagger}(\quotep{P},\quotep{Q}) & := & \forall R. P|Q \red^{*} R => R \red^{*} 0 \\
  match_{K}(\quotep{P},\quotep{Q}) & := & K \mbox{ for some context } K
\end{eqnarray*}

$u?(x)P | u!\langle Q \rangle \red P\{\quotep{Q}/x\}$

%We write $\wred$ for $\red^*$, and $P\red$ if $\exists Q $ such that $ P \red Q$.
We write $P\red$ if $\exists Q $ such that $ P \red Q$ and $P\not\red$, otherwise.

\section{Replication}

As mentioned before, it is known that replication (and hence
recursion) can be implemented in a higher-order process algebra
\cite{SangiorgiWalker}. As our first example of calculation with the
machinery thus far presented we give the construction explicitly in
the {\rhoc}.

\begin{eqnarray}
	D_{x} & := & \prefix{x}{y}{(\binpar{\outputp{x}{y}}{@{y}})} \nonumber\\
	\bangp_{x}{P} & := & \binpar{{x}!\langle{\binpar{D_{x}}{P}}\rangle}{D_{x}} \nonumber
\end{eqnarray}

\begin{eqnarray}
	\bangp_{x}{P} & & \nonumber\\
	=
	& {x}!\langle{(\prefix{x}{y}{(\outputp{x}{y} | @{y})) | P}}\rangle 
	      | \prefix{x}{y}{(\outputp{x}{y} | @{y})} & \nonumber\\
	\red
	& (\outputp{x}{y} | @{y})\substn{\quotep{(\prefix{x}{y}{(@{y} | \outputp{x}{y})) | P}}}{y} & \nonumber\\
	=
	& \outputp{x}{\quotep{(\prefix{x}{y}{(\outputp{x}{y} | @{y})) | P}}}
	  | {(\prefix{x}{y}{(\outputp{x}{y} | @{y})) | P}} & \nonumber\\
	\red
	& \ldots & \nonumber\\
	\red^*
	& P | P | \ldots & \nonumber
\end{eqnarray}

Of course, this encoding, as an implementation, runs away, unfolding
$\bangp{P}$ eagerly. A lazier and more implementable replication
operator, restricted to input-guarded processes, may be obtained as follows.

\begin{eqnarray}
\bangp{\prefix{u}{v}{P}} 
	:= 
	\binpar{\lift{x}{\prefix{u}{v}{(\binpar{D(x)}{P})}}}{D(x)} \nonumber
\end{eqnarray}

\begin{remark}
  Note that the lazier definition still does not deal with summation
  or mixed summation (i.e. sums over input and output). The reader is
  invited to construct definitions of replication that deal with these
  features. 

  Further, the definitions are parameterized in a name, $x$. Can you,
  gentle reader, make a definition that eliminates this parameter and
  guarantees no accidental interaction between the replication
  machinery and the process being replicated -- i.e. no accidental
  sharing of names used by the process to get its work done and the
  name(s) used by the replication to effect copying. This latter
  revision of the definition of replication is crucial to obtaining
  the expected identity $!!P \sim !P$.
\end{remark}

\begin{remark}\label{rem:paradoxical_combinator}
  The reader familiar with the lambda calculus will have noticed the
  similarity between $D$ and the paradoxical combinator.

  [Ed. note: the existence of this seems to suggest we have to be more
  restrictive on the set of processes and names we admit if we are to
  support no-cloning.]
\end{remark}

\subsubsection{Bisimulation}

The computational dynamics gives rise to another kind of equivalence,
the equivalence of computational behavior. As previously mentioned
this is typically captured \emph{via} some form of bisimulation.

% The notion we use in this paper is weak barbed bisimulation
% \cite{milner91polyadicpi}.

The notion we use in this paper is derived from weak barbed
bisimulation \cite{milner91polyadicpi}. 

\begin{definition}
An \emph{observation relation}, $\downarrow_{\mathcal N}$, over a set
of names, $\mathcal N$, is the smallest relation satisfying the rules
below.

\infrule[Out-barb]{y \in {\mathcal N}, \; x \nameeq y}
		  {\outputp{x}{v} \downarrow_{\mathcal N} x}
\infrule[Par-barb]{\mbox{$P\downarrow_{\mathcal N} x$ or $Q\downarrow_{\mathcal N} x$}}
		  {\binpar{P}{Q} \downarrow_{\mathcal N} x}

We write $P \Downarrow_{\mathcal N} x$ if there is $Q$ such that 
$P \wred Q$ and $Q \downarrow_{\mathcal N} x$.
\end{definition}

\begin{definition}
%\label{def.bbisim}
An  ${\mathcal N}$-\emph{barbed bisimulation} over a set of names, ${\mathcal N}$, is a symmetric binary relation 
${\mathcal S}_{\mathcal N}$ between agents such that $P\rel{S}_{\mathcal N}Q$ implies:
\begin{enumerate}
\item If $P \red P'$ then $Q \wred Q'$ and $P'\rel{S}_{\mathcal N} Q'$.
\item If $P\downarrow_{\mathcal N} x$, then $Q\Downarrow_{\mathcal N} x$.
\end{enumerate}
$P$ is ${\mathcal N}$-barbed bisimilar to $Q$, written
$P \wbbisim_{\mathcal N} Q$, if $P \rel{S}_{\mathcal N} Q$ for some ${\mathcal N}$-barbed bisimulation ${\mathcal S}_{\mathcal N}$.
\end{definition}

$\mathcal{R} \subseteq \pi \times \pi$

$P \mathcal{R} Q => \forall P'. P \red P' \Rightarrow \exists Q'. Q \red Q', P' \mathcal{R} Q'$

$P \vdash x \Rightarrow Q \vdash x$

\begin{mathpar}
  \inferrule*[lab=Out-barb]{x \nameeq y}{{y}!\langle{Q}\rangle \vdash x}
  \and
  \inferrule*[lab=Par-barb]{\mbox{$P\vdash x$ or $Q\vdash x$}}{\binpar{P}{Q} \vdash x}
\end{mathpar}

\subsubsection{Contexts}

One of the principle advantages of computational calculi like the
$\pi$-calculus is a well-defined notion of context,
contextual-equivalence and a correlation between
contextual-equivalence and notions of bisimulation. The notion of
context allows the decomposition of a process into (sub-)process and
its syntactic environment, its context. Thus, a context may be
thought of as a process with a ``hole'' (written $\Box$) in it. The
application of a context $M$ to a process $P$, written $M[P]$, is
tantamount to filling the hole in $M$ with $P$. In this paper we do
not need the full weight of this theory, but do make use of the notion
of context in the proof the main theorem. 

\begin{mathpar}
  \inferrule* [lab=summation] {} {{M_{M},M_{N}} \bc \Box \;|\; x.M_{A} \;|\; M_{M}+M_{N}}
  \and
  \inferrule* [lab=agent] {} {{M_{A}} \bc (\vec{x})M_{P} \;| \; \clift{P_0,\ldots,M_{P},\ldots,P_N}}
  \and \\
  \inferrule* [lab=process] {} {{M_{P}} \bc M_{N} \;| \;P|M_{P} }
\end{mathpar} 

\begin{mathpar}
  \inferrule* [lab=sychronization] {} {M_{N} \bc \Box \;|\; x?M_{F} \;|\; x!M_{C}}
  \and
  \inferrule* [lab=abstraction] {} {{M_{F}} \bc (x)M_{P} }
  \and
  \inferrule* [lab=concretion] {} {{M_{C}} \bc \langle M_{P} \rangle }
  \and \\
  \inferrule* [lab=process] {} {{M_{P}} \bc M_{N} \;| \;P|M_{P} }
\end{mathpar}

\begin{definition}[contextual application] Given a context $M$, and
  process $P$, we define the \emph{contextual application}, $M[P] :=
  M\{P/\Box\}$. That is, the contextual application of M to P is the
  substitution of $P$ for $\Box$ in $M$.
\end{definition}

$\meaningof{-} : L \to \mathcal{P}(\pi)$

\begin{mathpar}
  \inferrule* [lab=collection] {} {\meaningof{true} = \pi, \and \meaningof{~E} = \pi \setminus \meaningof{E}, \and \meaningof{E_{1} \& E_{2}} = \meaningof{E_{1}} \cap \meaningof{E_{2}}}
\end{mathpar}

\begin{mathpar}
  \inferrule* [lab=structure] {} {\meaningof{0} = \{ P \in \pi | P \equiv 0 \}, \and \\ \meaningof{E_1 | E_2} = \{ P \in \pi | P \equiv P_{1} | P_{2}, P_{1} \in \meaningof{E_{1}}, P_{2} \in \meaningof{E_2}\} }
\end{mathpar}

\begin{mathpar}
 \inferrule* [lab=behavior] {} {\meaningof{\langle a?b \rangle E} = \{ P \in \pi | P \equiv Q | u?(y)P', \\ \and \\\\ \and \\ \;\;\; u \in \meaningof{a}, \forall z.P'\{z/y\} \in \meaningof{E\{z/b\}}\}, \and \\ \meaningof{a!E} = \{ P \in \pi | P \equiv Q | x!\langle P' \rangle, x \in \meaningof{a} P' \in \meaningof{E}\} }
\end{mathpar}

\begin{mathpar}
 \inferrule* [lab=nominal] {} {\meaningof{\quotep{E}} = \{ \quotep{P} \in \quotep{\pi} | P \in \meaningof{E} \}, \and \meaningof{\quotep{P}} = \{ \quotep{Q} \in \quotep{\pi} | P \equiv Q \} \and \\ \meaningof{@\quotep{E}} = \{ P \in \pi | P \equiv @x, x \in \meaningof{E} \}}
\end{mathpar}

\begin{eqnarray*}
  \\
  \meaningof{-} : TS \to ST
\end{eqnarray*}

\begin{eqnarray*}
  \\
  L : TS \to ST
\end{eqnarray*}

\begin{eqnarray*}
  \\
  P \models E \iff P \in \meaningof{E}
\end{eqnarray*}

\begin{eqnarray*}
  P \approx_{L} Q \iff \forall E \in L. P \models E \iff Q \models E
\end{eqnarray*}

\begin{eqnarray*}
  P \approx_{K} Q
\end{eqnarray*}

\begin{eqnarray*}
  P \approx Q
\end{eqnarray*}

$\approx_{K} = \approx = \approx_{L}$

\subsubsection{Contextual duality}

Note that contexts extend the quotation operation to a family of
operations from processes to names. Given a context, $M$, we can
define a \emph{nominal context}, $\quotep{M}$ by $\quotep{M}[P] :=
\quotep{M[P]}$. To foreshadow what is to come we observe that these
operations enjoy a duality with processes very much like the duality
between vectors and maps from vectors to scalars.

Further, because the calculus is essentially higher-order, we have a
correspondence between contexts and processes. More specifically,
given a name $x$ and a context $M$ we can construct $M^{*}_{x}$ such
that 

\begin{mathpar}
  M^{*}_{x} | \lift{x}{P} \red M[P]
\end{mathpar}

namely,

\begin{mathpar}
  M^{*}_{x} := x?(u).M[\dropn{u}]
\end{mathpar}

The dependence of $M^{*}_{x}$ on a name makes it an abstraction, 

\begin{mathpar}
  M^{*} := (x)x?(u).M[\dropn{u}]
\end{mathpar}

\subsection{Additional notation}

It will sometimes be convenient to denote the process a name
quotes. We already have the notation $x = \quotep{P}$, but it will be
convenient to introduce an alternate notation, $\procn{x}$, when we
want to emphasize the connection to the use of the name. Note that, by
virtue of name equivalence, $\quotep{\procn{x}} \nameeq x$; so, the
notation is consistent with previous definitions.

Further, because names have structure it is possible to effect
substitutions on the basis of that structure. This means we need to
upgrade our notation for substitutions, which we accomplish by
adapting comprehension notation. Thus,

\begin{mathpar}
  P\{ y / x : x \in S \}
\end{mathpar}

is interpreted to mean the process derived from P by replacing (in a
capture-avoiding manner) each occurrence of $x$ in $S$ by $y$. For example,

\begin{mathpar}
  P\{ \quotep{\procn{x}|\procn{x}} / x : x \in \freenames{P} \}
\end{mathpar}

will replace each (occurrence) of a free name $x$ in $P$ by
$\quotep{\procn{x}|\procn{x}}$.

Also, we will avail ourselves of the notation $x^{L}$ and $x^{R}$ to
denote injections of a name into disjoint copies of the name
space. There are numerous ways to accomplish this. One example can be
found in \cite{MeredithR05}. This notation overloads to vectors of
names: $\vec{x}^{\pi} := (x_{i}^{\pi} \; : \; 0 \leq i < |\vec{x}| )$ where $\pi \in \{L,R\}$.

We also use $P^{\Box} := P|\Box$.

In \cite{MeredithR05} an interpretation of the new operator is
given. It turns out that there are several possible interpretations
all enjoying the requisite algebraic properties of the operator (see
\cite{milner91polyadicpi}). We will therefore make liberal use of
$(\nu\; \vec{x})P$.

% subsection the_syntax_and_semantics_of_the_notation_system (end)   

\input{qm2pi.qmops} 

\input{qm2pi.sterngerlach} 

\input{qm2pi.metric} 

% section concurrent_process_calculi (end)

%\input{qm2pi.proofsketch}

% section proof sketch (end)

%\input{qm2pi.slviaknots} 

% section spatial logic via knots (end)

\input{qm2pi.conclusion}

% section conclusion (end)

%\input{qm2pi.dtcodes} 

% section wiring algorithm (end)

\input{qm2pi.ack} 

% section acknowledgments (end)

\newpage


\bibliographystyle{plain}   
\bibliography{../../biblios/main.bib}

\input{qm2pi.rhodetails}

\end{document}

 

% subsection basic_interpretation (end)

%\input{qm2pi.rho.presentation} 
\subsection{The syntax and semantics of the notation system}\label{sub:the_syntax_and_semantics_of_the_notation_system} % (fold)

We now summarize a technical presentation of the calculus that
embodies our theory of dynamics. The typical presentation of such a
calculus follows the style of giving generators and relations on
them. The grammar, below, describing term constructors, freely
generates the set of processes, $\Proc$. This set is then quotiented
by a relation known as structural congruence and it is over this set
that the notion of dynamics is expressed. This presentation is
essentially that of \cite{MeredithR05} with the addition of
polyadicity and summation. For readability we have relegated some of
the technical subtleties to an appendix.

\subsubsection{Process grammar}\label{subsub:process_grammar}

\begin{mathpar}
  \inferrule* [lab=synchronization] {} {{M} \bc \pzero \;|\; x?F \;|\; x!C }
  \and
  \inferrule* [lab=abstraction] {} {{F} \bc (x)P}
  \and
  \inferrule* [lab=concretion] {} {{C} \bc \langle Q \rangle}
  \and
  \inferrule* [lab=process] {} {{P,Q} \bc M \;| \;P|Q \;|\; @{x}}
  \and
  \inferrule* [lab=name] {} {{x} \bc \quotep{P}}
\end{mathpar} 

Note that $\vec{x}$ (resp. $\vec{P}$) denotes a vector of names
(resp. processes) of length $|\vec{x}|$ (resp. $|\vec{P}|$). We adopt
the following useful abbreviations.

\begin{mathpar}
   x?(\vec{y}).P := x.(\vec{y})P \and  x\clift{\vec{P}} := x.\clift{\vec{P}}
   \and x!(y) := \lift{x}{\dropn{y}}
   \and \Pi_{i=0}^{n-1}P_i := P_0 | \ldots | P_{n-1}
\end{mathpar}

\subsubsection{Structural congruence}

\paragraph{Free and bound names and alpha-equivalence.} At the
core of structural equivalence is alpha-equivalence which identifies
process that are the same up to a change of variable. Formally, we
recognize the distinction between free and bound names. The free names
of a process, $\freenames{P}$, may be calculated recursively as
follows:

\begin{mathpar}
\freenames{\pzero} := \emptyset
  \and \\
  \freenames{x?(y).P} := \{ x \} \cup (\freenames{P} \setminus \{ y \})
  \and 
  \freenames{x!\langle P \rangle} := \{ x \} \cup \{ P \} 
  \and \\
  \freenames{P|Q} := \freenames{P} \cup \freenames{Q}
  \and \\
  \freenames{@{x}} := \{ x \}
\end{mathpar}

$\pi$
$\quotep{\pi}$

$\freenames{-} : \pi \to \mathcal{P}(\quotep{\pi})$

\begin{eqnarray*}
  \freenames{\pzero} & := & \emptyset \\
  \freenames{x?(y).P} & := & \{ x \} \cup (\freenames{P} \setminus \{ y \}) \\
  \freenames{x!\langle P \rangle} & := & \{ x \} \cup \{ P \} \\
  \freenames{P|Q} & := & \freenames{P} \cup \freenames{Q} \\
  \freenames{\dropn{x}} & := & \{ x \}
\end{eqnarray*}

The bound names of a process, $\boundnames{P}$, are those names occurring in $P$
that are not free. For example, in $x?(y).0$, the name $x$ is free, while $y$ is bound.

\begin{mathpar}
  \inferrule* [lab=monoidal-laws] {} { P|Q \equiv Q|P \and P|0 \equiv P \and P|(Q|R) \equiv (P|Q)|R }
\end{mathpar}

\begin{mathpar}
  \inferrule* [lab=alpha-equivalence] {} { (x)P \equiv (y)P\{y/x\} \and y \not\in \freenames{P} }
\end{mathpar}

\begin{definition}
Then two processes, $P,Q$, are alpha-equivalent if $P = Q\{\vec{y}/\vec{x}\}$ for
some $\vec{x} \in \boundnames{Q},\vec{y} \in \boundnames{P}$, where $Q\{\vec{y}/\vec{x}\}$
denotes the capture-avoiding substitution of $\vec{y}$ for $\vec{x}$ in $Q$.
\end{definition}

\begin{definition}
  The {\em structural congruence} \cite{SangiorgiWalker} , $\equiv$,
  between processes is the least congruence containing
  alpha-equivalence, satisfying the abelian monoid laws
  (associativity, commutativity and $\pzero$ as identity) for parallel
  composition $|$ and for summation $+$.
\end{definition}

\subsection{Name equivalence}

We take name equivalence, written $\nameeq$, to be the smallest
equivalence relation generated by the following rules.

\begin{mathpar}
\inferrule*[lab=Quote-drop]
{ }
{ \quotep{@{x}} \nameeq x }

\inferrule*[lab=Struct-equiv]
{ P \scong Q }
{ \quotep{P} \nameeq \quotep{Q} }
\end{mathpar}

The astute reader will have noticed that the mutual recursion of names
and processes imposes a mutual recursion on alpha-equivalence and
structural equivalence via name-equivalence. Fortunately, all of this
works out pleasantly and we may calculate in the natural way, free of
concern. The reader interested in the details is referred to the
appendix \ref{appendix:rho_details}.

\subsection{Substitution}

We use $\Proc$ for the set of processes, $\QProc$ for the set of
names, and $\id{\{}\vec{y} / \vec{x} \id{\}}$ to denote partial maps,
$s : \QProc \rightarrow \QProc$. A map, $s$ lifts, uniquely, to a map
on process terms, $\widehat{s} : \Proc \rightarrow \Proc$ by the
following equations.

\begin{mathpar}
  (0) \psubstp{Q}{P} := 0 \\
  (R \juxtap S) \psubstp{Q}{P}
  :=    
  (R)\psubstp{Q}{P} \juxtap (S) \psubstp{Q}{P} \\
  (x?(y).R) \psubstp{Q}{P}    
  :=    
  (x)\substp{Q}{P} (z)\concat( (R \psubstn{z}{y}) \psubstp{Q}{P} ) \\
  (\lift{x}{R}) \psubstp{Q}{P}  
  :=
  \lift{(x)\substp{Q}{P}}{ R \psubstp{Q}{P} } \\
%   (\dropn{x})  \psubstp{Q}{P}       
%   := 
%   \left\{ 
%     \begin{array}{ccc} 
%       \dropn{\quotep{Q}} & & x \nameeq \quotep{P} \\
%       \dropn{x} & & otherwise \\
%     \end{array}
%   \right. 
  (\dropn{x})  \psubstp{Q}{P}       
  := 
  \left\{ 
    \begin{array}{ccc} 
      Q & & x \nameeq \quotep{P} \\
      \dropn{x} & & otherwise \\
    \end{array}
  \right.
\end{mathpar}
 

where

\begin{eqnarray}
  (x)\id{\{} \lpquote Q \rpquote / \lpquote P \rpquote \id{\}}            = 
  \left\{ 
    \begin{array}{ccc}
      \lpquote Q \rpquote & & x \nameeq \lpquote P \rpquote \\
      x & & otherwise \\
    \end{array}
  \right. \nonumber
\end{eqnarray}

and $z$ is chosen distinct from $\quotep{P}$, $\quotep{Q}$, the free
names in $Q$, and all the names in $R$. Our $\alpha$-equivalence will
be built in the standard way from this substitution.

\begin{remark}\label{rem:no_self_referential_names}
  One consequence of these definitions is that $\forall P. \quotep{P}
  \not\in \freenames{P}$.
\end{remark}

\subsection{ Dynamic quote: an example }

Anticipating something of what's to come, consider applying the
substitution, $\widehat{\id{\{}u / z \id{\}}}$, to the following pair
of processes, $\lift{w}{y!(z)}$ and $w[ \lpquote y!(z) \rpquote ]$.

\begin{eqnarray}
	\lift{w}{y!(z)}\widehat{\id{\{}u / z \id{\}}}
		& = &
		\lift{w}{y!(u)} \nonumber\\
	w[ \lpquote y!(z) \rpquote ] \widehat{ \id{\{}u / z \id{\}} }
		& = &
		w[ \lpquote y!(z) \rpquote ] \nonumber
\end{eqnarray}

Because the body of the process between quotes is impervious to
substitution, we get radically different answers. In fact, by
examining the first process in an input context,
e.g. $x?(z).\lift{w}{y!(z)}$, we see that the process under the lift
operator may be shaped by prefixed inputs binding a name inside it. In
this sense, the lift operator will be seen as a way to dynamically
construct processes before reifying them as names.

Finally equipped with these standard features we can present the
dynamics of the calculus.

\subsubsection{Operational semantics} 

Finally, we introduce the computational dynamics. What marks these
algebras as distinct from other more traditionally studied algebraic
structures, e.g. vector spaces or polynomial rings, is the manner in
which dynamics is captured. In traditional structures, dynamics is typically
expressed through morphisms between such structures, as in linear maps
between vector spaces or morphisms between rings. In algebras
associated with the semantics of computation, the dynamics is
expressed as part of the algebraic structure itself, through a
reduction reduction relation typically denoted by $\red$. Below, we
give a recursive presentation of this relation for the calculus used
in the encoding.

$\red \subseteq \pi \times \pi$
$\red : \pi \to \mathcal{P}(\pi)$

\begin{mathpar}
  \inferrule* [lab=Comm] { \textsf{match}( x_{src}, x_{trgt} ) } { x_{trgt}?(y)P \; | \; x_{src}!\langle {Q} \rangle \red P\{\quotep{Q}/y}\} }
  \and \\
  \inferrule* [lab=Par] {{P} \red {P}'} {{{P} | {Q}} \red {{P}' | {Q}}}
  \and
  \inferrule* [lab=Equiv]{{{P} \scong {P}'} \andalso {{P}' \red {Q}'} \andalso {{Q}' \scong {Q}}}{{P} \red {Q}}
\end{mathpar}

\begin{eqnarray*}
  match_{\equiv} (\quotep{P},\quotep{Q}) & := & P \equiv Q \\
  match_{\dagger}(\quotep{P},\quotep{Q}) & := & \forall R. P|Q \red^{*} R => R \red^{*} 0 \\
  match_{K}(\quotep{P},\quotep{Q}) & := & K \mbox{ for some context } K
\end{eqnarray*}

$u?(x)P | u!\langle Q \rangle \red P\{\quotep{Q}/x\}$

%We write $\wred$ for $\red^*$, and $P\red$ if $\exists Q $ such that $ P \red Q$.
We write $P\red$ if $\exists Q $ such that $ P \red Q$ and $P\not\red$, otherwise.

\section{Replication}

As mentioned before, it is known that replication (and hence
recursion) can be implemented in a higher-order process algebra
\cite{SangiorgiWalker}. As our first example of calculation with the
machinery thus far presented we give the construction explicitly in
the {\rhoc}.

\begin{eqnarray}
	D_{x} & := & \prefix{x}{y}{(\binpar{\outputp{x}{y}}{@{y}})} \nonumber\\
	\bangp_{x}{P} & := & \binpar{{x}!\langle{\binpar{D_{x}}{P}}\rangle}{D_{x}} \nonumber
\end{eqnarray}

\begin{eqnarray}
	\bangp_{x}{P} & & \nonumber\\
	=
	& {x}!\langle{(\prefix{x}{y}{(\outputp{x}{y} | @{y})) | P}}\rangle 
	      | \prefix{x}{y}{(\outputp{x}{y} | @{y})} & \nonumber\\
	\red
	& (\outputp{x}{y} | @{y})\substn{\quotep{(\prefix{x}{y}{(@{y} | \outputp{x}{y})) | P}}}{y} & \nonumber\\
	=
	& \outputp{x}{\quotep{(\prefix{x}{y}{(\outputp{x}{y} | @{y})) | P}}}
	  | {(\prefix{x}{y}{(\outputp{x}{y} | @{y})) | P}} & \nonumber\\
	\red
	& \ldots & \nonumber\\
	\red^*
	& P | P | \ldots & \nonumber
\end{eqnarray}

Of course, this encoding, as an implementation, runs away, unfolding
$\bangp{P}$ eagerly. A lazier and more implementable replication
operator, restricted to input-guarded processes, may be obtained as follows.

\begin{eqnarray}
\bangp{\prefix{u}{v}{P}} 
	:= 
	\binpar{\lift{x}{\prefix{u}{v}{(\binpar{D(x)}{P})}}}{D(x)} \nonumber
\end{eqnarray}

\begin{remark}
  Note that the lazier definition still does not deal with summation
  or mixed summation (i.e. sums over input and output). The reader is
  invited to construct definitions of replication that deal with these
  features. 

  Further, the definitions are parameterized in a name, $x$. Can you,
  gentle reader, make a definition that eliminates this parameter and
  guarantees no accidental interaction between the replication
  machinery and the process being replicated -- i.e. no accidental
  sharing of names used by the process to get its work done and the
  name(s) used by the replication to effect copying. This latter
  revision of the definition of replication is crucial to obtaining
  the expected identity $!!P \sim !P$.
\end{remark}

\begin{remark}\label{rem:paradoxical_combinator}
  The reader familiar with the lambda calculus will have noticed the
  similarity between $D$ and the paradoxical combinator.

  [Ed. note: the existence of this seems to suggest we have to be more
  restrictive on the set of processes and names we admit if we are to
  support no-cloning.]
\end{remark}

\subsubsection{Bisimulation}

The computational dynamics gives rise to another kind of equivalence,
the equivalence of computational behavior. As previously mentioned
this is typically captured \emph{via} some form of bisimulation.

% The notion we use in this paper is weak barbed bisimulation
% \cite{milner91polyadicpi}.

The notion we use in this paper is derived from weak barbed
bisimulation \cite{milner91polyadicpi}. 

\begin{definition}
An \emph{observation relation}, $\downarrow_{\mathcal N}$, over a set
of names, $\mathcal N$, is the smallest relation satisfying the rules
below.

\infrule[Out-barb]{y \in {\mathcal N}, \; x \nameeq y}
		  {\outputp{x}{v} \downarrow_{\mathcal N} x}
\infrule[Par-barb]{\mbox{$P\downarrow_{\mathcal N} x$ or $Q\downarrow_{\mathcal N} x$}}
		  {\binpar{P}{Q} \downarrow_{\mathcal N} x}

We write $P \Downarrow_{\mathcal N} x$ if there is $Q$ such that 
$P \wred Q$ and $Q \downarrow_{\mathcal N} x$.
\end{definition}

\begin{definition}
%\label{def.bbisim}
An  ${\mathcal N}$-\emph{barbed bisimulation} over a set of names, ${\mathcal N}$, is a symmetric binary relation 
${\mathcal S}_{\mathcal N}$ between agents such that $P\rel{S}_{\mathcal N}Q$ implies:
\begin{enumerate}
\item If $P \red P'$ then $Q \wred Q'$ and $P'\rel{S}_{\mathcal N} Q'$.
\item If $P\downarrow_{\mathcal N} x$, then $Q\Downarrow_{\mathcal N} x$.
\end{enumerate}
$P$ is ${\mathcal N}$-barbed bisimilar to $Q$, written
$P \wbbisim_{\mathcal N} Q$, if $P \rel{S}_{\mathcal N} Q$ for some ${\mathcal N}$-barbed bisimulation ${\mathcal S}_{\mathcal N}$.
\end{definition}

$\mathcal{R} \subseteq \pi \times \pi$

$P \mathcal{R} Q => \forall P'. P \red P' \Rightarrow \exists Q'. Q \red Q', P' \mathcal{R} Q'$

$P \vdash x \Rightarrow Q \vdash x$

\begin{mathpar}
  \inferrule*[lab=Out-barb]{x \nameeq y}{{y}!\langle{Q}\rangle \vdash x}
  \and
  \inferrule*[lab=Par-barb]{\mbox{$P\vdash x$ or $Q\vdash x$}}{\binpar{P}{Q} \vdash x}
\end{mathpar}

\subsubsection{Contexts}

One of the principle advantages of computational calculi like the
$\pi$-calculus is a well-defined notion of context,
contextual-equivalence and a correlation between
contextual-equivalence and notions of bisimulation. The notion of
context allows the decomposition of a process into (sub-)process and
its syntactic environment, its context. Thus, a context may be
thought of as a process with a ``hole'' (written $\Box$) in it. The
application of a context $M$ to a process $P$, written $M[P]$, is
tantamount to filling the hole in $M$ with $P$. In this paper we do
not need the full weight of this theory, but do make use of the notion
of context in the proof the main theorem. 

\begin{mathpar}
  \inferrule* [lab=summation] {} {{M_{M},M_{N}} \bc \Box \;|\; x.M_{A} \;|\; M_{M}+M_{N}}
  \and
  \inferrule* [lab=agent] {} {{M_{A}} \bc (\vec{x})M_{P} \;| \; \clift{P_0,\ldots,M_{P},\ldots,P_N}}
  \and \\
  \inferrule* [lab=process] {} {{M_{P}} \bc M_{N} \;| \;P|M_{P} }
\end{mathpar} 

\begin{mathpar}
  \inferrule* [lab=sychronization] {} {M_{N} \bc \Box \;|\; x?M_{F} \;|\; x!M_{C}}
  \and
  \inferrule* [lab=abstraction] {} {{M_{F}} \bc (x)M_{P} }
  \and
  \inferrule* [lab=concretion] {} {{M_{C}} \bc \langle M_{P} \rangle }
  \and \\
  \inferrule* [lab=process] {} {{M_{P}} \bc M_{N} \;| \;P|M_{P} }
\end{mathpar}

\begin{definition}[contextual application] Given a context $M$, and
  process $P$, we define the \emph{contextual application}, $M[P] :=
  M\{P/\Box\}$. That is, the contextual application of M to P is the
  substitution of $P$ for $\Box$ in $M$.
\end{definition}

$\meaningof{-} : L \to \mathcal{P}(\pi)$

\begin{mathpar}
  \inferrule* [lab=collection] {} {\meaningof{true} = \pi, \and \meaningof{~E} = \pi \setminus \meaningof{E}, \and \meaningof{E_{1} \& E_{2}} = \meaningof{E_{1}} \cap \meaningof{E_{2}}}
\end{mathpar}

\begin{mathpar}
  \inferrule* [lab=structure] {} {\meaningof{0} = \{ P \in \pi | P \equiv 0 \}, \and \\ \meaningof{E_1 | E_2} = \{ P \in \pi | P \equiv P_{1} | P_{2}, P_{1} \in \meaningof{E_{1}}, P_{2} \in \meaningof{E_2}\} }
\end{mathpar}

\begin{mathpar}
 \inferrule* [lab=behavior] {} {\meaningof{\langle a?b \rangle E} = \{ P \in \pi | P \equiv Q | u?(y)P', \\ \and \\\\ \and \\ \;\;\; u \in \meaningof{a}, \forall z.P'\{z/y\} \in \meaningof{E\{z/b\}}\}, \and \\ \meaningof{a!E} = \{ P \in \pi | P \equiv Q | x!\langle P' \rangle, x \in \meaningof{a} P' \in \meaningof{E}\} }
\end{mathpar}

\begin{mathpar}
 \inferrule* [lab=nominal] {} {\meaningof{\quotep{E}} = \{ \quotep{P} \in \quotep{\pi} | P \in \meaningof{E} \}, \and \meaningof{\quotep{P}} = \{ \quotep{Q} \in \quotep{\pi} | P \equiv Q \} \and \\ \meaningof{@\quotep{E}} = \{ P \in \pi | P \equiv @x, x \in \meaningof{E} \}}
\end{mathpar}

\begin{eqnarray*}
  \\
  \meaningof{-} : TS \to ST
\end{eqnarray*}

\begin{eqnarray*}
  \\
  L : TS \to ST
\end{eqnarray*}

\begin{eqnarray*}
  \\
  P \models E \iff P \in \meaningof{E}
\end{eqnarray*}

\begin{eqnarray*}
  P \approx_{L} Q \iff \forall E \in L. P \models E \iff Q \models E
\end{eqnarray*}

\begin{eqnarray*}
  P \approx_{K} Q
\end{eqnarray*}

\begin{eqnarray*}
  P \approx Q
\end{eqnarray*}

$\approx_{K} = \approx = \approx_{L}$

\subsubsection{Contextual duality}

Note that contexts extend the quotation operation to a family of
operations from processes to names. Given a context, $M$, we can
define a \emph{nominal context}, $\quotep{M}$ by $\quotep{M}[P] :=
\quotep{M[P]}$. To foreshadow what is to come we observe that these
operations enjoy a duality with processes very much like the duality
between vectors and maps from vectors to scalars.

Further, because the calculus is essentially higher-order, we have a
correspondence between contexts and processes. More specifically,
given a name $x$ and a context $M$ we can construct $M^{*}_{x}$ such
that 

\begin{mathpar}
  M^{*}_{x} | \lift{x}{P} \red M[P]
\end{mathpar}

namely,

\begin{mathpar}
  M^{*}_{x} := x?(u).M[\dropn{u}]
\end{mathpar}

The dependence of $M^{*}_{x}$ on a name makes it an abstraction, 

\begin{mathpar}
  M^{*} := (x)x?(u).M[\dropn{u}]
\end{mathpar}

\subsection{Additional notation}

It will sometimes be convenient to denote the process a name
quotes. We already have the notation $x = \quotep{P}$, but it will be
convenient to introduce an alternate notation, $\procn{x}$, when we
want to emphasize the connection to the use of the name. Note that, by
virtue of name equivalence, $\quotep{\procn{x}} \nameeq x$; so, the
notation is consistent with previous definitions.

Further, because names have structure it is possible to effect
substitutions on the basis of that structure. This means we need to
upgrade our notation for substitutions, which we accomplish by
adapting comprehension notation. Thus,

\begin{mathpar}
  P\{ y / x : x \in S \}
\end{mathpar}

is interpreted to mean the process derived from P by replacing (in a
capture-avoiding manner) each occurrence of $x$ in $S$ by $y$. For example,

\begin{mathpar}
  P\{ \quotep{\procn{x}|\procn{x}} / x : x \in \freenames{P} \}
\end{mathpar}

will replace each (occurrence) of a free name $x$ in $P$ by
$\quotep{\procn{x}|\procn{x}}$.

Also, we will avail ourselves of the notation $x^{L}$ and $x^{R}$ to
denote injections of a name into disjoint copies of the name
space. There are numerous ways to accomplish this. One example can be
found in \cite{MeredithR05}. This notation overloads to vectors of
names: $\vec{x}^{\pi} := (x_{i}^{\pi} \; : \; 0 \leq i < |\vec{x}| )$ where $\pi \in \{L,R\}$.

We also use $P^{\Box} := P|\Box$.

In \cite{MeredithR05} an interpretation of the new operator is
given. It turns out that there are several possible interpretations
all enjoying the requisite algebraic properties of the operator (see
\cite{milner91polyadicpi}). We will therefore make liberal use of
$(\nu\; \vec{x})P$.

% subsection the_syntax_and_semantics_of_the_notation_system (end)   

\section{Interpretation of QM}
\subsection{Supporting definitions}
\subsubsection{Multiplication}
\begin{mathpar}
  \quotep{Q} \cdot \quotep{R} := \quotep{Q|R}
  \and \\
  \quotep{Q} \cdot P := P\{ \quotep{Q|R} / \quotep{R} : \quotep{R} \in \freenames{P} \}
\end{mathpar}

\paragraph{Discussion}
The first line needs little explanation. The second line says that
each free name of the process is replaced with the multiplication of
that name by the scalar. Multiplication of a scalar (name) by a state
(process) results in a process all the names of which have been `moved
over' by parallel composition with the process the scalar
quotes. There is a subtlety that the bound names have to be
manipulated so that multiplied names aren't accidentally
captured. There are many ways to achieve this.

\begin{remark}\label{rem:multiplication_identities}
  The reader is invited to verify that for all $x,y,z \in \QProc$ and $P \in \Proc$
  \begin{mathpar}
    x \cdot \quotep{0} \equiv x 
    \and
    x \cdot y \equiv y \cdot x
    \and
    x \cdot (y \cdot z) \equiv (x \cdot y) \cdot z
    \and \\
    \quotep{0} \cdot P \equiv P
    \and \\
    x \cdot (y \cdot P) \equiv (x \cdot y) \cdot P
    \and \\
    x \cdot (P|Q) \equiv (x \cdot P) | (x \cdot Q)
    \and \\    
  \end{mathpar}
\end{remark}

\subsubsection{Tensor product}

We define a tensor product on processes by structural induction.

\paragraph{Tensor of sums} First note that all summations, including
$\pzero$ and sequence, can be written $\Sigma_{i} x_{i}.A_{i} +
\Sigma_{j} x_{j}.C_{j}$, where we have grouped input-guarded processes
together and output-guarded processes together.

Thus, we can define the tensor product of two summations, $N_{1}\otimes N_{2}$, where

\begin{mathpar}
  N_{1} := \Sigma_{i} x_{i}.A_{i} + \Sigma_{j} x_{j}.C_{j}
  \and
  N_{2} := \Sigma_{i'} y_{i'}.B_{i'} + \Sigma_{j'} y_{j'}.D_{j'} 
\end{mathpar}

as follows.

\begin{mathpar}
  \Sigma_{i} x_{i}.A_{i} + \Sigma_{j} x_{j}.C_{j} \otimes \Sigma_{i'}
  y_{i'}.B_{i'} + \Sigma_{j'} y_{j'}.D_{j'} 
  \and \\
  := \; \Sigma_{i} \Sigma_{i'} \quotep{\stackrel{\vee}{x_{i}}| \stackrel{\vee}{y_{i'}}}.(A_{i}\otimes B_{i'}) \; | \; \Sigma_{i'} \Sigma_{i} \quotep{\stackrel{\vee}{y_{i'}}|\stackrel{\vee}{x_{i}}}.(B_{i'}\otimes A_{i})
  \and
  \;\; | \;\; \Sigma_{j} \Sigma_{j'} \quotep{\stackrel{\vee}{x_{j}}|\stackrel{\vee}{y_{j'}}}.(A_{j}\otimes B_{j'}) \; | \; \Sigma_{j'} \Sigma_{j} \quotep{\stackrel{\vee}{y_{j'}}|\stackrel{\vee}{x_{j}}}.(B_{j'}\otimes A_{j})
\end{mathpar}

\begin{remark}
  Do we need to $x^{L}$ and $y^{R}$ for this construction as well?
\end{remark}

\paragraph{Tensor of parallel compositions} Next, we distribute tensor
over par.

\begin{mathpar}
  P_{1}|P_{2} \otimes Q_{1}|Q_{2} := (P_{1} \otimes Q_{1}) | (P_{1}
  \otimes Q_{2}) | (P_{2} \otimes Q_{1}) | (P_{2} \otimes Q_{2})
\end{mathpar}

\paragraph{Tensor with dropped names} We treat tensor of a
process with a dropped name as parallel composition.

\begin{mathpar}
  P \otimes \dropn{x} := P | \dropn{x}
\end{mathpar}

\paragraph{Tensor of agents}

Finally, we need to define tensor on agents. Note that the definition
of tensor on normal products only tensors inputs with inputs and
outputs with outputs. Thus, we only have to define the operation on
``homogeneous'' pairings.

\begin{mathpar}
  (\vec{x})P \otimes (\vec{y})Q
  \and \\
  := (x_{0}^{L}|y_{0}^{R},\ldots,x_{0}^{L}|y_{n}^{R},\ldots,x_{m}^{L}|y_{0}^{R},\ldots,x_{m}^{L}|y_{n}^R)(P\{ \vec{x}^{L}/\vec{x}\} \otimes Q \{ \vec{y}^{R}/\vec{y}\})
  \and \\
  \clift{\vec{P}} \otimes \clift{\vec{Q}}
  \and \\
  := \clift{P_{0}\otimes Q_{0},\ldots,P_{0}\otimes Q_{n},\ldots,P_{m}\otimes Q_{0},\ldots,P_{m}\otimes Q_{n}}
\end{mathpar}

\begin{remark}
  Observe that arities of tensored abstractions matches arities of
  tensored concretions if the original arities matched. Note also that
  the length of the arities corresponds to the increase in dimension
  we see in ordinary vector space tensor product.
\end{remark}

\begin{remark}
  Operationally, this definition distributes the tensor down to
  components ``linked'' by summation. Tensor over summation is
  intriguing in that it mixes names. Moreover, as a consequence of the
  way it mixes names we have the identities for all $x \in \QProc$ and
  $P,Q \in \Proc$

  \begin{mathpar}
    (x \cdot P) \otimes Q \equiv x \cdot (P \otimes Q) \equiv P \otimes (x \cdot Q)
    \and
    P \otimes \pzero \equiv P
  \end{mathpar}

  that the reader is invited to verify.
\end{remark}

\subsubsection{Annihilation}
\begin{mathpar}
  P^{\perp} := \{ Q | \forall R. P|Q \red^{*} R \Rightarrow R \red^{*} \pzero \}
  \and \\
  P^{\underline{\perp}} := \Sigma_{Q \in P^{\perp}} \quotep{Q}?(y).(\dropn{y}|Q) | \Sigma_{Q \in P^{\perp}} \quotep{Q}\clift{\Box}
\end{mathpar}

\paragraph{Discussion} The reader will note that $P^{\perp}$ is a
\emph{set} of processes, while $P^{\underline{\perp}}$ is a
\emph{context}. We call the set $P^{\perp}$ the \emph{annihilators} of
$P$. The parallel composition of a process in the annihilators of $P$
with $P$ will result in a process, the state space of which has all
paths eventually leading to $\pzero$. Execution may endure loops; but
under reasonable conditions of fairness (naturally guaranteed under
most notions of bisimulation) such a composite process cannot get
stuck in such a loop and will, eventually pop out and terminate.

The context $P^{\underline{\perp}}$ is ready and willing to ``take the
$P$ out of'' the process to which it is applied. It will effectively
transmit the code of the process to which it is applied to one of the
annihilators and run the process against it.

\subsubsection{Evaluation}
We fix $M$ a domain of fully abstract interpretation with an equality
coincident with bisimulation. We take $\meaningof{\cdot} : \Proc \to
M$ to be the map interpreting processes and $\nmeaningof{\cdot} : \M
\to Proc$ to be the map running the other way. Then we define

\begin{mathpar}
  \int P := \nmeaningof{\meaningof{P}}
\end{mathpar}

\paragraph{Discussion}
There are many fully abstract interpretations of Milner's
$\pi$-calculus. Any of them can be used as a basis for interpreting
the reflective calculus here. Equipped with such a domain it is
largely a matter of grinding through to check that the Yoneda
construction for the normalization-by-evaluation program can be
extended to this setting.

\begin{remark}
  The reader is invited to verify that $\int (P^{\underline{\perp}}[P]) = 0$.
\end{remark}

\subsection{Quantum mechanics}

Table \ref{tbl:core_qm_op_defns} gives the core operational definitions

\begin{table}[htp]\label{tbl:core_qm_op_defns}
  \center{
    \fbox{
      \begin{tabular}{c|c}
        quantum mechanics & process calculus \\
        \hline
        scalar & $x := \quotep{P}$ \\
        state vector & $\state{P} := P$ \\
        dual & $\state{P}^{*} := \event{P^{\underline{\perp}}} := \quotep{P^{\underline{\perp}}}[-]$ \\
        matrix & $ \Sigma_{\alpha} \state{P_{\alpha}}x_{\alpha}\event{Q_{\alpha}}$ \\
        vector addition & $\state{P} + \state{Q} := \state{P | Q}$ \\
        tensor product & $\state{P} \otimes \state{Q} := \state{P \otimes Q}$ \\
        inner product & $\innerprod{P}{Q} := \quotep{\int P^{\underline{\perp}}[Q]}$ \\
      \end{tabular}
    }
  }
  \caption{QM - operational definitions}
\end{table}

where

\begin{mathpar}
  \prmatrix{P}{Q} := \fprmatrix{P}{\quotep{\pzero}}{Q}
  \and
  \fprmatrix{P}{x}{Q} := (\state{P},x,\event{Q})
  \and
  (\fprmatrix{P}{x}{Q})(\state{R}) := x \cdot \innerprod{Q}{R} \cdot \state{P}
  \and
  (\fprmatrix{P}{x}{Q})(\event{R}) := x \cdot \innerprod{R}{P} \cdot \event{Q}
\end{mathpar}

\paragraph{Discussion}
As promised: vectors (aka states) are represented as processes; duals
as contextual duals; inner product definition should be compared with
standard inner product definition for ....

\begin{remark}
  Assuming $\int (P^{\underline{\perp}}[P]) = 0$, the reader is
  invited to verify that $(\fprmatrix{P}{x}{P})(\state{P}) = x \cdot \state{P}$.
\end{remark}

\begin{remark}
  The reader is invited to verify that $\innerprod{P}{Q}$ could
  equally well have been written $\quotep{\int \stackrel{\vee}{x}}$
  where $x = \event{P^{\underline{\perp}}}(Q)$.

  One of the motivations for this remark is that there is another way
  to factor these operations. We could package up evaluation in the dual:

  \begin{mathpar}
    \state{P}^{*} := \event{\int P^{\underline{\perp}}} := \quotep{\int P^{\underline{\perp}}}[-]
  \end{mathpar}

  and then have inner product defined by
  
  \begin{mathpar}
    \innerprod{P}{Q} := \event{P}(Q)
  \end{mathpar}

  Hopefully, experience with the calculations will provide guidance on
  the best factoring.
\end{remark}

\begin{remark}
  Assuming $\int (P^{\underline{\perp}}[P]) = 0$, the reader is
  invited to verify that $\forall P,Q. (\prmatrix{0}{Q})(\state{0}) =
  \state{0}$ and dually $(\prmatrix{P}{0})(\event{0}) = \event{0}$.
\end{remark}

\begin{remark}
  i'm a little worried that i don't (yet) have proper support for
  complex conjugacy. But, the observation above may give us a
  clue. According to Abramsky, it must be the case that the scalars
  are iso to the homset of the identity for the tensor -- which the
  observation above characterizes. 

  For now, we will simply bookmark the notion with $\overline{x}$.
\end{remark}

\subsubsection{Adjointness}

We need to give a definition of $(\cdot)^{\dagger}$ for matrices. The
obvious candidate definition is
\begin{mathpar}
(\Sigma_{\alpha}\fprmatrix{P_{\alpha}}{x_{\alpha}}{Q_{\alpha}})^{\dagger}
= \Sigma_{\alpha}\fprmatrix{(Q_{\alpha}^{\underline{\perp}})^{*}}{\overline{x}_{\alpha}}{P_{\alpha}^{\underline{\perp}}} 
\end{mathpar}

But, $(Q_{\alpha}^{\underline{\perp}})^{*}$ requires a name along
which to communicate the process to achieve the context application.

\subsubsection{Basis for a basis}
If processes label states and ``addition'' of states (a.k.a. vector
addition) is interpreted as parallel composition, what corresponds to
notions of linear independence and basis? Here, we recall that Yoshida
has developed a set of \emph{combinators} for an asynchronous verison
of Milner's $\pi$-calculus. These are a finite set of processes such
any process can be expressed as parallel composition of these
combinators together with liberal uses of the new operator and
replication. We can simply give a translation of these into the
present calculus and have reasonable expectation that the property
carries over. That is, that the resultant set allows to express all
processes via parallel composition. Note, however, that there is no
new operator or replication in this calculus. As a result, we expect
that the corresponding set is actually infinite. That is, we expect
that the space is actually infinite dimensional.

\begin{remark}
  The attentive reader may be a bit concerned. Certainly, the
  collection $S$, $K$ and $I$ is a finite set of
  combinators. Shouldn't we expect to see a finite set of combinators
  for an effectively equivalent system? i am very sympathetic to this
  critique and feel it warrants full attention. On the other hand, i
  also have in mind the following analogy. The natural numbers, as a
  monoid under addition, has exactly $1$ generator, while the natural
  numbers, as a monoid under multiplication, has countably many
  generators (the primes). We observe that the application of the
  lambda calculus is much less resource sensitive than the parallel
  composition of the $\pi$-calculus. Could it be the case that we have
  an analogy of the form
  
  \begin{mathpar}
    m + n : MN :: m*n : M|N
  \end{mathpar}

  giving a similar blow up in the set of ``primes''?  This is such a
  wonderful thought that, even if it's not true, i think it's worth
  writing down.
\end{remark}
 

\documentclass[12pt]{llncs}
%\documentclass{jktr}

\usepackage[pdftex]{hyperref}                   
\usepackage {listings}
\usepackage {mathpartir}
\usepackage{bcprules}
%\usepackage{listings}
                       
\usepackage{graphicx} 
%\usepackage[margins=2.5cm,nohead,nofoot]{geometry}
%\usepackage{geometry}
\usepackage{amsfonts}
\usepackage{amstext}
\usepackage{latexsym}
\usepackage{amssymb}
\usepackage{color}


%\include{myPreamble}
\include{qm2pi.local} 

%\ifpdf
%\usepackage[pdftex]{graphicx}
%\else
%\usepackage{graphicx}
%\fi

 % \ifpdf
%  \usepackage{pdfsync}
%  \if


%\title{Brief Article}
%\author{David F. Snyder}
%\author{L.G. Meredith}

%\address{Dept. of Math., Texas State University--San Marcos, San Marcos, TX 78666}
       
\pagestyle{empty}


\begin{document}

\lstset{language=[Objective]Caml,frame=shadowbox}

\input{qm2pi.front}

% section front matter (end)

\input{qm2pi.intro} 
 
% section introduction (end)

% \input{qm2pi.knotations} 

% section notation (end)

\input{qm2pi.process.calculi} 

% section concurrent_process_calculi_and_spatial_logics_ (end)
    
%\input{qm2pi.knots2pi} 

%\input{qm2pi.trefoil} 

%\input{qm2pi.mainthm} 

% subsection basic_interpretation (end)

%\input{qm2pi.rho.presentation} 
\subsection{The syntax and semantics of the notation system}\label{sub:the_syntax_and_semantics_of_the_notation_system} % (fold)

We now summarize a technical presentation of the calculus that
embodies our theory of dynamics. The typical presentation of such a
calculus follows the style of giving generators and relations on
them. The grammar, below, describing term constructors, freely
generates the set of processes, $\Proc$. This set is then quotiented
by a relation known as structural congruence and it is over this set
that the notion of dynamics is expressed. This presentation is
essentially that of \cite{MeredithR05} with the addition of
polyadicity and summation. For readability we have relegated some of
the technical subtleties to an appendix.

\subsubsection{Process grammar}\label{subsub:process_grammar}

\begin{mathpar}
  \inferrule* [lab=synchronization] {} {{M} \bc \pzero \;|\; x?F \;|\; x!C }
  \and
  \inferrule* [lab=abstraction] {} {{F} \bc (x)P}
  \and
  \inferrule* [lab=concretion] {} {{C} \bc \langle Q \rangle}
  \and
  \inferrule* [lab=process] {} {{P,Q} \bc M \;| \;P|Q \;|\; @{x}}
  \and
  \inferrule* [lab=name] {} {{x} \bc \quotep{P}}
\end{mathpar} 

Note that $\vec{x}$ (resp. $\vec{P}$) denotes a vector of names
(resp. processes) of length $|\vec{x}|$ (resp. $|\vec{P}|$). We adopt
the following useful abbreviations.

\begin{mathpar}
   x?(\vec{y}).P := x.(\vec{y})P \and  x\clift{\vec{P}} := x.\clift{\vec{P}}
   \and x!(y) := \lift{x}{\dropn{y}}
   \and \Pi_{i=0}^{n-1}P_i := P_0 | \ldots | P_{n-1}
\end{mathpar}

\subsubsection{Structural congruence}

\paragraph{Free and bound names and alpha-equivalence.} At the
core of structural equivalence is alpha-equivalence which identifies
process that are the same up to a change of variable. Formally, we
recognize the distinction between free and bound names. The free names
of a process, $\freenames{P}$, may be calculated recursively as
follows:

\begin{mathpar}
\freenames{\pzero} := \emptyset
  \and \\
  \freenames{x?(y).P} := \{ x \} \cup (\freenames{P} \setminus \{ y \})
  \and 
  \freenames{x!\langle P \rangle} := \{ x \} \cup \{ P \} 
  \and \\
  \freenames{P|Q} := \freenames{P} \cup \freenames{Q}
  \and \\
  \freenames{@{x}} := \{ x \}
\end{mathpar}

$\pi$
$\quotep{\pi}$

$\freenames{-} : \pi \to \mathcal{P}(\quotep{\pi})$

\begin{eqnarray*}
  \freenames{\pzero} & := & \emptyset \\
  \freenames{x?(y).P} & := & \{ x \} \cup (\freenames{P} \setminus \{ y \}) \\
  \freenames{x!\langle P \rangle} & := & \{ x \} \cup \{ P \} \\
  \freenames{P|Q} & := & \freenames{P} \cup \freenames{Q} \\
  \freenames{\dropn{x}} & := & \{ x \}
\end{eqnarray*}

The bound names of a process, $\boundnames{P}$, are those names occurring in $P$
that are not free. For example, in $x?(y).0$, the name $x$ is free, while $y$ is bound.

\begin{mathpar}
  \inferrule* [lab=monoidal-laws] {} { P|Q \equiv Q|P \and P|0 \equiv P \and P|(Q|R) \equiv (P|Q)|R }
\end{mathpar}

\begin{mathpar}
  \inferrule* [lab=alpha-equivalence] {} { (x)P \equiv (y)P\{y/x\} \and y \not\in \freenames{P} }
\end{mathpar}

\begin{definition}
Then two processes, $P,Q$, are alpha-equivalent if $P = Q\{\vec{y}/\vec{x}\}$ for
some $\vec{x} \in \boundnames{Q},\vec{y} \in \boundnames{P}$, where $Q\{\vec{y}/\vec{x}\}$
denotes the capture-avoiding substitution of $\vec{y}$ for $\vec{x}$ in $Q$.
\end{definition}

\begin{definition}
  The {\em structural congruence} \cite{SangiorgiWalker} , $\equiv$,
  between processes is the least congruence containing
  alpha-equivalence, satisfying the abelian monoid laws
  (associativity, commutativity and $\pzero$ as identity) for parallel
  composition $|$ and for summation $+$.
\end{definition}

\subsection{Name equivalence}

We take name equivalence, written $\nameeq$, to be the smallest
equivalence relation generated by the following rules.

\begin{mathpar}
\inferrule*[lab=Quote-drop]
{ }
{ \quotep{@{x}} \nameeq x }

\inferrule*[lab=Struct-equiv]
{ P \scong Q }
{ \quotep{P} \nameeq \quotep{Q} }
\end{mathpar}

The astute reader will have noticed that the mutual recursion of names
and processes imposes a mutual recursion on alpha-equivalence and
structural equivalence via name-equivalence. Fortunately, all of this
works out pleasantly and we may calculate in the natural way, free of
concern. The reader interested in the details is referred to the
appendix \ref{appendix:rho_details}.

\subsection{Substitution}

We use $\Proc$ for the set of processes, $\QProc$ for the set of
names, and $\id{\{}\vec{y} / \vec{x} \id{\}}$ to denote partial maps,
$s : \QProc \rightarrow \QProc$. A map, $s$ lifts, uniquely, to a map
on process terms, $\widehat{s} : \Proc \rightarrow \Proc$ by the
following equations.

\begin{mathpar}
  (0) \psubstp{Q}{P} := 0 \\
  (R \juxtap S) \psubstp{Q}{P}
  :=    
  (R)\psubstp{Q}{P} \juxtap (S) \psubstp{Q}{P} \\
  (x?(y).R) \psubstp{Q}{P}    
  :=    
  (x)\substp{Q}{P} (z)\concat( (R \psubstn{z}{y}) \psubstp{Q}{P} ) \\
  (\lift{x}{R}) \psubstp{Q}{P}  
  :=
  \lift{(x)\substp{Q}{P}}{ R \psubstp{Q}{P} } \\
%   (\dropn{x})  \psubstp{Q}{P}       
%   := 
%   \left\{ 
%     \begin{array}{ccc} 
%       \dropn{\quotep{Q}} & & x \nameeq \quotep{P} \\
%       \dropn{x} & & otherwise \\
%     \end{array}
%   \right. 
  (\dropn{x})  \psubstp{Q}{P}       
  := 
  \left\{ 
    \begin{array}{ccc} 
      Q & & x \nameeq \quotep{P} \\
      \dropn{x} & & otherwise \\
    \end{array}
  \right.
\end{mathpar}
 

where

\begin{eqnarray}
  (x)\id{\{} \lpquote Q \rpquote / \lpquote P \rpquote \id{\}}            = 
  \left\{ 
    \begin{array}{ccc}
      \lpquote Q \rpquote & & x \nameeq \lpquote P \rpquote \\
      x & & otherwise \\
    \end{array}
  \right. \nonumber
\end{eqnarray}

and $z$ is chosen distinct from $\quotep{P}$, $\quotep{Q}$, the free
names in $Q$, and all the names in $R$. Our $\alpha$-equivalence will
be built in the standard way from this substitution.

\begin{remark}\label{rem:no_self_referential_names}
  One consequence of these definitions is that $\forall P. \quotep{P}
  \not\in \freenames{P}$.
\end{remark}

\subsection{ Dynamic quote: an example }

Anticipating something of what's to come, consider applying the
substitution, $\widehat{\id{\{}u / z \id{\}}}$, to the following pair
of processes, $\lift{w}{y!(z)}$ and $w[ \lpquote y!(z) \rpquote ]$.

\begin{eqnarray}
	\lift{w}{y!(z)}\widehat{\id{\{}u / z \id{\}}}
		& = &
		\lift{w}{y!(u)} \nonumber\\
	w[ \lpquote y!(z) \rpquote ] \widehat{ \id{\{}u / z \id{\}} }
		& = &
		w[ \lpquote y!(z) \rpquote ] \nonumber
\end{eqnarray}

Because the body of the process between quotes is impervious to
substitution, we get radically different answers. In fact, by
examining the first process in an input context,
e.g. $x?(z).\lift{w}{y!(z)}$, we see that the process under the lift
operator may be shaped by prefixed inputs binding a name inside it. In
this sense, the lift operator will be seen as a way to dynamically
construct processes before reifying them as names.

Finally equipped with these standard features we can present the
dynamics of the calculus.

\subsubsection{Operational semantics} 

Finally, we introduce the computational dynamics. What marks these
algebras as distinct from other more traditionally studied algebraic
structures, e.g. vector spaces or polynomial rings, is the manner in
which dynamics is captured. In traditional structures, dynamics is typically
expressed through morphisms between such structures, as in linear maps
between vector spaces or morphisms between rings. In algebras
associated with the semantics of computation, the dynamics is
expressed as part of the algebraic structure itself, through a
reduction reduction relation typically denoted by $\red$. Below, we
give a recursive presentation of this relation for the calculus used
in the encoding.

$\red \subseteq \pi \times \pi$
$\red : \pi \to \mathcal{P}(\pi)$

\begin{mathpar}
  \inferrule* [lab=Comm] { \textsf{match}( x_{src}, x_{trgt} ) } { x_{trgt}?(y)P \; | \; x_{src}!\langle {Q} \rangle \red P\{\quotep{Q}/y}\} }
  \and \\
  \inferrule* [lab=Par] {{P} \red {P}'} {{{P} | {Q}} \red {{P}' | {Q}}}
  \and
  \inferrule* [lab=Equiv]{{{P} \scong {P}'} \andalso {{P}' \red {Q}'} \andalso {{Q}' \scong {Q}}}{{P} \red {Q}}
\end{mathpar}

\begin{eqnarray*}
  match_{\equiv} (\quotep{P},\quotep{Q}) & := & P \equiv Q \\
  match_{\dagger}(\quotep{P},\quotep{Q}) & := & \forall R. P|Q \red^{*} R => R \red^{*} 0 \\
  match_{K}(\quotep{P},\quotep{Q}) & := & K \mbox{ for some context } K
\end{eqnarray*}

$u?(x)P | u!\langle Q \rangle \red P\{\quotep{Q}/x\}$

%We write $\wred$ for $\red^*$, and $P\red$ if $\exists Q $ such that $ P \red Q$.
We write $P\red$ if $\exists Q $ such that $ P \red Q$ and $P\not\red$, otherwise.

\section{Replication}

As mentioned before, it is known that replication (and hence
recursion) can be implemented in a higher-order process algebra
\cite{SangiorgiWalker}. As our first example of calculation with the
machinery thus far presented we give the construction explicitly in
the {\rhoc}.

\begin{eqnarray}
	D_{x} & := & \prefix{x}{y}{(\binpar{\outputp{x}{y}}{@{y}})} \nonumber\\
	\bangp_{x}{P} & := & \binpar{{x}!\langle{\binpar{D_{x}}{P}}\rangle}{D_{x}} \nonumber
\end{eqnarray}

\begin{eqnarray}
	\bangp_{x}{P} & & \nonumber\\
	=
	& {x}!\langle{(\prefix{x}{y}{(\outputp{x}{y} | @{y})) | P}}\rangle 
	      | \prefix{x}{y}{(\outputp{x}{y} | @{y})} & \nonumber\\
	\red
	& (\outputp{x}{y} | @{y})\substn{\quotep{(\prefix{x}{y}{(@{y} | \outputp{x}{y})) | P}}}{y} & \nonumber\\
	=
	& \outputp{x}{\quotep{(\prefix{x}{y}{(\outputp{x}{y} | @{y})) | P}}}
	  | {(\prefix{x}{y}{(\outputp{x}{y} | @{y})) | P}} & \nonumber\\
	\red
	& \ldots & \nonumber\\
	\red^*
	& P | P | \ldots & \nonumber
\end{eqnarray}

Of course, this encoding, as an implementation, runs away, unfolding
$\bangp{P}$ eagerly. A lazier and more implementable replication
operator, restricted to input-guarded processes, may be obtained as follows.

\begin{eqnarray}
\bangp{\prefix{u}{v}{P}} 
	:= 
	\binpar{\lift{x}{\prefix{u}{v}{(\binpar{D(x)}{P})}}}{D(x)} \nonumber
\end{eqnarray}

\begin{remark}
  Note that the lazier definition still does not deal with summation
  or mixed summation (i.e. sums over input and output). The reader is
  invited to construct definitions of replication that deal with these
  features. 

  Further, the definitions are parameterized in a name, $x$. Can you,
  gentle reader, make a definition that eliminates this parameter and
  guarantees no accidental interaction between the replication
  machinery and the process being replicated -- i.e. no accidental
  sharing of names used by the process to get its work done and the
  name(s) used by the replication to effect copying. This latter
  revision of the definition of replication is crucial to obtaining
  the expected identity $!!P \sim !P$.
\end{remark}

\begin{remark}\label{rem:paradoxical_combinator}
  The reader familiar with the lambda calculus will have noticed the
  similarity between $D$ and the paradoxical combinator.

  [Ed. note: the existence of this seems to suggest we have to be more
  restrictive on the set of processes and names we admit if we are to
  support no-cloning.]
\end{remark}

\subsubsection{Bisimulation}

The computational dynamics gives rise to another kind of equivalence,
the equivalence of computational behavior. As previously mentioned
this is typically captured \emph{via} some form of bisimulation.

% The notion we use in this paper is weak barbed bisimulation
% \cite{milner91polyadicpi}.

The notion we use in this paper is derived from weak barbed
bisimulation \cite{milner91polyadicpi}. 

\begin{definition}
An \emph{observation relation}, $\downarrow_{\mathcal N}$, over a set
of names, $\mathcal N$, is the smallest relation satisfying the rules
below.

\infrule[Out-barb]{y \in {\mathcal N}, \; x \nameeq y}
		  {\outputp{x}{v} \downarrow_{\mathcal N} x}
\infrule[Par-barb]{\mbox{$P\downarrow_{\mathcal N} x$ or $Q\downarrow_{\mathcal N} x$}}
		  {\binpar{P}{Q} \downarrow_{\mathcal N} x}

We write $P \Downarrow_{\mathcal N} x$ if there is $Q$ such that 
$P \wred Q$ and $Q \downarrow_{\mathcal N} x$.
\end{definition}

\begin{definition}
%\label{def.bbisim}
An  ${\mathcal N}$-\emph{barbed bisimulation} over a set of names, ${\mathcal N}$, is a symmetric binary relation 
${\mathcal S}_{\mathcal N}$ between agents such that $P\rel{S}_{\mathcal N}Q$ implies:
\begin{enumerate}
\item If $P \red P'$ then $Q \wred Q'$ and $P'\rel{S}_{\mathcal N} Q'$.
\item If $P\downarrow_{\mathcal N} x$, then $Q\Downarrow_{\mathcal N} x$.
\end{enumerate}
$P$ is ${\mathcal N}$-barbed bisimilar to $Q$, written
$P \wbbisim_{\mathcal N} Q$, if $P \rel{S}_{\mathcal N} Q$ for some ${\mathcal N}$-barbed bisimulation ${\mathcal S}_{\mathcal N}$.
\end{definition}

$\mathcal{R} \subseteq \pi \times \pi$

$P \mathcal{R} Q => \forall P'. P \red P' \Rightarrow \exists Q'. Q \red Q', P' \mathcal{R} Q'$

$P \vdash x \Rightarrow Q \vdash x$

\begin{mathpar}
  \inferrule*[lab=Out-barb]{x \nameeq y}{{y}!\langle{Q}\rangle \vdash x}
  \and
  \inferrule*[lab=Par-barb]{\mbox{$P\vdash x$ or $Q\vdash x$}}{\binpar{P}{Q} \vdash x}
\end{mathpar}

\subsubsection{Contexts}

One of the principle advantages of computational calculi like the
$\pi$-calculus is a well-defined notion of context,
contextual-equivalence and a correlation between
contextual-equivalence and notions of bisimulation. The notion of
context allows the decomposition of a process into (sub-)process and
its syntactic environment, its context. Thus, a context may be
thought of as a process with a ``hole'' (written $\Box$) in it. The
application of a context $M$ to a process $P$, written $M[P]$, is
tantamount to filling the hole in $M$ with $P$. In this paper we do
not need the full weight of this theory, but do make use of the notion
of context in the proof the main theorem. 

\begin{mathpar}
  \inferrule* [lab=summation] {} {{M_{M},M_{N}} \bc \Box \;|\; x.M_{A} \;|\; M_{M}+M_{N}}
  \and
  \inferrule* [lab=agent] {} {{M_{A}} \bc (\vec{x})M_{P} \;| \; \clift{P_0,\ldots,M_{P},\ldots,P_N}}
  \and \\
  \inferrule* [lab=process] {} {{M_{P}} \bc M_{N} \;| \;P|M_{P} }
\end{mathpar} 

\begin{mathpar}
  \inferrule* [lab=sychronization] {} {M_{N} \bc \Box \;|\; x?M_{F} \;|\; x!M_{C}}
  \and
  \inferrule* [lab=abstraction] {} {{M_{F}} \bc (x)M_{P} }
  \and
  \inferrule* [lab=concretion] {} {{M_{C}} \bc \langle M_{P} \rangle }
  \and \\
  \inferrule* [lab=process] {} {{M_{P}} \bc M_{N} \;| \;P|M_{P} }
\end{mathpar}

\begin{definition}[contextual application] Given a context $M$, and
  process $P$, we define the \emph{contextual application}, $M[P] :=
  M\{P/\Box\}$. That is, the contextual application of M to P is the
  substitution of $P$ for $\Box$ in $M$.
\end{definition}

$\meaningof{-} : L \to \mathcal{P}(\pi)$

\begin{mathpar}
  \inferrule* [lab=collection] {} {\meaningof{true} = \pi, \and \meaningof{~E} = \pi \setminus \meaningof{E}, \and \meaningof{E_{1} \& E_{2}} = \meaningof{E_{1}} \cap \meaningof{E_{2}}}
\end{mathpar}

\begin{mathpar}
  \inferrule* [lab=structure] {} {\meaningof{0} = \{ P \in \pi | P \equiv 0 \}, \and \\ \meaningof{E_1 | E_2} = \{ P \in \pi | P \equiv P_{1} | P_{2}, P_{1} \in \meaningof{E_{1}}, P_{2} \in \meaningof{E_2}\} }
\end{mathpar}

\begin{mathpar}
 \inferrule* [lab=behavior] {} {\meaningof{\langle a?b \rangle E} = \{ P \in \pi | P \equiv Q | u?(y)P', \\ \and \\\\ \and \\ \;\;\; u \in \meaningof{a}, \forall z.P'\{z/y\} \in \meaningof{E\{z/b\}}\}, \and \\ \meaningof{a!E} = \{ P \in \pi | P \equiv Q | x!\langle P' \rangle, x \in \meaningof{a} P' \in \meaningof{E}\} }
\end{mathpar}

\begin{mathpar}
 \inferrule* [lab=nominal] {} {\meaningof{\quotep{E}} = \{ \quotep{P} \in \quotep{\pi} | P \in \meaningof{E} \}, \and \meaningof{\quotep{P}} = \{ \quotep{Q} \in \quotep{\pi} | P \equiv Q \} \and \\ \meaningof{@\quotep{E}} = \{ P \in \pi | P \equiv @x, x \in \meaningof{E} \}}
\end{mathpar}

\begin{eqnarray*}
  \\
  \meaningof{-} : TS \to ST
\end{eqnarray*}

\begin{eqnarray*}
  \\
  L : TS \to ST
\end{eqnarray*}

\begin{eqnarray*}
  \\
  P \models E \iff P \in \meaningof{E}
\end{eqnarray*}

\begin{eqnarray*}
  P \approx_{L} Q \iff \forall E \in L. P \models E \iff Q \models E
\end{eqnarray*}

\begin{eqnarray*}
  P \approx_{K} Q
\end{eqnarray*}

\begin{eqnarray*}
  P \approx Q
\end{eqnarray*}

$\approx_{K} = \approx = \approx_{L}$

\subsubsection{Contextual duality}

Note that contexts extend the quotation operation to a family of
operations from processes to names. Given a context, $M$, we can
define a \emph{nominal context}, $\quotep{M}$ by $\quotep{M}[P] :=
\quotep{M[P]}$. To foreshadow what is to come we observe that these
operations enjoy a duality with processes very much like the duality
between vectors and maps from vectors to scalars.

Further, because the calculus is essentially higher-order, we have a
correspondence between contexts and processes. More specifically,
given a name $x$ and a context $M$ we can construct $M^{*}_{x}$ such
that 

\begin{mathpar}
  M^{*}_{x} | \lift{x}{P} \red M[P]
\end{mathpar}

namely,

\begin{mathpar}
  M^{*}_{x} := x?(u).M[\dropn{u}]
\end{mathpar}

The dependence of $M^{*}_{x}$ on a name makes it an abstraction, 

\begin{mathpar}
  M^{*} := (x)x?(u).M[\dropn{u}]
\end{mathpar}

\subsection{Additional notation}

It will sometimes be convenient to denote the process a name
quotes. We already have the notation $x = \quotep{P}$, but it will be
convenient to introduce an alternate notation, $\procn{x}$, when we
want to emphasize the connection to the use of the name. Note that, by
virtue of name equivalence, $\quotep{\procn{x}} \nameeq x$; so, the
notation is consistent with previous definitions.

Further, because names have structure it is possible to effect
substitutions on the basis of that structure. This means we need to
upgrade our notation for substitutions, which we accomplish by
adapting comprehension notation. Thus,

\begin{mathpar}
  P\{ y / x : x \in S \}
\end{mathpar}

is interpreted to mean the process derived from P by replacing (in a
capture-avoiding manner) each occurrence of $x$ in $S$ by $y$. For example,

\begin{mathpar}
  P\{ \quotep{\procn{x}|\procn{x}} / x : x \in \freenames{P} \}
\end{mathpar}

will replace each (occurrence) of a free name $x$ in $P$ by
$\quotep{\procn{x}|\procn{x}}$.

Also, we will avail ourselves of the notation $x^{L}$ and $x^{R}$ to
denote injections of a name into disjoint copies of the name
space. There are numerous ways to accomplish this. One example can be
found in \cite{MeredithR05}. This notation overloads to vectors of
names: $\vec{x}^{\pi} := (x_{i}^{\pi} \; : \; 0 \leq i < |\vec{x}| )$ where $\pi \in \{L,R\}$.

We also use $P^{\Box} := P|\Box$.

In \cite{MeredithR05} an interpretation of the new operator is
given. It turns out that there are several possible interpretations
all enjoying the requisite algebraic properties of the operator (see
\cite{milner91polyadicpi}). We will therefore make liberal use of
$(\nu\; \vec{x})P$.

% subsection the_syntax_and_semantics_of_the_notation_system (end)   

\input{qm2pi.qmops} 

\input{qm2pi.sterngerlach} 

\input{qm2pi.metric} 

% section concurrent_process_calculi (end)

%\input{qm2pi.proofsketch}

% section proof sketch (end)

%\input{qm2pi.slviaknots} 

% section spatial logic via knots (end)

\input{qm2pi.conclusion}

% section conclusion (end)

%\input{qm2pi.dtcodes} 

% section wiring algorithm (end)

\input{qm2pi.ack} 

% section acknowledgments (end)

\newpage


\bibliographystyle{plain}   
\bibliography{../../biblios/main.bib}

\input{qm2pi.rhodetails}

\end{document}

 

\documentclass[12pt]{llncs}
%\documentclass{jktr}

\usepackage[pdftex]{hyperref}                   
\usepackage {listings}
\usepackage {mathpartir}
\usepackage{bcprules}
%\usepackage{listings}
                       
\usepackage{graphicx} 
%\usepackage[margins=2.5cm,nohead,nofoot]{geometry}
%\usepackage{geometry}
\usepackage{amsfonts}
\usepackage{amstext}
\usepackage{latexsym}
\usepackage{amssymb}
\usepackage{color}


%\include{myPreamble}
\include{qm2pi.local} 

%\ifpdf
%\usepackage[pdftex]{graphicx}
%\else
%\usepackage{graphicx}
%\fi

 % \ifpdf
%  \usepackage{pdfsync}
%  \if


%\title{Brief Article}
%\author{David F. Snyder}
%\author{L.G. Meredith}

%\address{Dept. of Math., Texas State University--San Marcos, San Marcos, TX 78666}
       
\pagestyle{empty}


\begin{document}

\lstset{language=[Objective]Caml,frame=shadowbox}

\input{qm2pi.front}

% section front matter (end)

\input{qm2pi.intro} 
 
% section introduction (end)

% \input{qm2pi.knotations} 

% section notation (end)

\input{qm2pi.process.calculi} 

% section concurrent_process_calculi_and_spatial_logics_ (end)
    
%\input{qm2pi.knots2pi} 

%\input{qm2pi.trefoil} 

%\input{qm2pi.mainthm} 

% subsection basic_interpretation (end)

%\input{qm2pi.rho.presentation} 
\subsection{The syntax and semantics of the notation system}\label{sub:the_syntax_and_semantics_of_the_notation_system} % (fold)

We now summarize a technical presentation of the calculus that
embodies our theory of dynamics. The typical presentation of such a
calculus follows the style of giving generators and relations on
them. The grammar, below, describing term constructors, freely
generates the set of processes, $\Proc$. This set is then quotiented
by a relation known as structural congruence and it is over this set
that the notion of dynamics is expressed. This presentation is
essentially that of \cite{MeredithR05} with the addition of
polyadicity and summation. For readability we have relegated some of
the technical subtleties to an appendix.

\subsubsection{Process grammar}\label{subsub:process_grammar}

\begin{mathpar}
  \inferrule* [lab=synchronization] {} {{M} \bc \pzero \;|\; x?F \;|\; x!C }
  \and
  \inferrule* [lab=abstraction] {} {{F} \bc (x)P}
  \and
  \inferrule* [lab=concretion] {} {{C} \bc \langle Q \rangle}
  \and
  \inferrule* [lab=process] {} {{P,Q} \bc M \;| \;P|Q \;|\; @{x}}
  \and
  \inferrule* [lab=name] {} {{x} \bc \quotep{P}}
\end{mathpar} 

Note that $\vec{x}$ (resp. $\vec{P}$) denotes a vector of names
(resp. processes) of length $|\vec{x}|$ (resp. $|\vec{P}|$). We adopt
the following useful abbreviations.

\begin{mathpar}
   x?(\vec{y}).P := x.(\vec{y})P \and  x\clift{\vec{P}} := x.\clift{\vec{P}}
   \and x!(y) := \lift{x}{\dropn{y}}
   \and \Pi_{i=0}^{n-1}P_i := P_0 | \ldots | P_{n-1}
\end{mathpar}

\subsubsection{Structural congruence}

\paragraph{Free and bound names and alpha-equivalence.} At the
core of structural equivalence is alpha-equivalence which identifies
process that are the same up to a change of variable. Formally, we
recognize the distinction between free and bound names. The free names
of a process, $\freenames{P}$, may be calculated recursively as
follows:

\begin{mathpar}
\freenames{\pzero} := \emptyset
  \and \\
  \freenames{x?(y).P} := \{ x \} \cup (\freenames{P} \setminus \{ y \})
  \and 
  \freenames{x!\langle P \rangle} := \{ x \} \cup \{ P \} 
  \and \\
  \freenames{P|Q} := \freenames{P} \cup \freenames{Q}
  \and \\
  \freenames{@{x}} := \{ x \}
\end{mathpar}

$\pi$
$\quotep{\pi}$

$\freenames{-} : \pi \to \mathcal{P}(\quotep{\pi})$

\begin{eqnarray*}
  \freenames{\pzero} & := & \emptyset \\
  \freenames{x?(y).P} & := & \{ x \} \cup (\freenames{P} \setminus \{ y \}) \\
  \freenames{x!\langle P \rangle} & := & \{ x \} \cup \{ P \} \\
  \freenames{P|Q} & := & \freenames{P} \cup \freenames{Q} \\
  \freenames{\dropn{x}} & := & \{ x \}
\end{eqnarray*}

The bound names of a process, $\boundnames{P}$, are those names occurring in $P$
that are not free. For example, in $x?(y).0$, the name $x$ is free, while $y$ is bound.

\begin{mathpar}
  \inferrule* [lab=monoidal-laws] {} { P|Q \equiv Q|P \and P|0 \equiv P \and P|(Q|R) \equiv (P|Q)|R }
\end{mathpar}

\begin{mathpar}
  \inferrule* [lab=alpha-equivalence] {} { (x)P \equiv (y)P\{y/x\} \and y \not\in \freenames{P} }
\end{mathpar}

\begin{definition}
Then two processes, $P,Q$, are alpha-equivalent if $P = Q\{\vec{y}/\vec{x}\}$ for
some $\vec{x} \in \boundnames{Q},\vec{y} \in \boundnames{P}$, where $Q\{\vec{y}/\vec{x}\}$
denotes the capture-avoiding substitution of $\vec{y}$ for $\vec{x}$ in $Q$.
\end{definition}

\begin{definition}
  The {\em structural congruence} \cite{SangiorgiWalker} , $\equiv$,
  between processes is the least congruence containing
  alpha-equivalence, satisfying the abelian monoid laws
  (associativity, commutativity and $\pzero$ as identity) for parallel
  composition $|$ and for summation $+$.
\end{definition}

\subsection{Name equivalence}

We take name equivalence, written $\nameeq$, to be the smallest
equivalence relation generated by the following rules.

\begin{mathpar}
\inferrule*[lab=Quote-drop]
{ }
{ \quotep{@{x}} \nameeq x }

\inferrule*[lab=Struct-equiv]
{ P \scong Q }
{ \quotep{P} \nameeq \quotep{Q} }
\end{mathpar}

The astute reader will have noticed that the mutual recursion of names
and processes imposes a mutual recursion on alpha-equivalence and
structural equivalence via name-equivalence. Fortunately, all of this
works out pleasantly and we may calculate in the natural way, free of
concern. The reader interested in the details is referred to the
appendix \ref{appendix:rho_details}.

\subsection{Substitution}

We use $\Proc$ for the set of processes, $\QProc$ for the set of
names, and $\id{\{}\vec{y} / \vec{x} \id{\}}$ to denote partial maps,
$s : \QProc \rightarrow \QProc$. A map, $s$ lifts, uniquely, to a map
on process terms, $\widehat{s} : \Proc \rightarrow \Proc$ by the
following equations.

\begin{mathpar}
  (0) \psubstp{Q}{P} := 0 \\
  (R \juxtap S) \psubstp{Q}{P}
  :=    
  (R)\psubstp{Q}{P} \juxtap (S) \psubstp{Q}{P} \\
  (x?(y).R) \psubstp{Q}{P}    
  :=    
  (x)\substp{Q}{P} (z)\concat( (R \psubstn{z}{y}) \psubstp{Q}{P} ) \\
  (\lift{x}{R}) \psubstp{Q}{P}  
  :=
  \lift{(x)\substp{Q}{P}}{ R \psubstp{Q}{P} } \\
%   (\dropn{x})  \psubstp{Q}{P}       
%   := 
%   \left\{ 
%     \begin{array}{ccc} 
%       \dropn{\quotep{Q}} & & x \nameeq \quotep{P} \\
%       \dropn{x} & & otherwise \\
%     \end{array}
%   \right. 
  (\dropn{x})  \psubstp{Q}{P}       
  := 
  \left\{ 
    \begin{array}{ccc} 
      Q & & x \nameeq \quotep{P} \\
      \dropn{x} & & otherwise \\
    \end{array}
  \right.
\end{mathpar}
 

where

\begin{eqnarray}
  (x)\id{\{} \lpquote Q \rpquote / \lpquote P \rpquote \id{\}}            = 
  \left\{ 
    \begin{array}{ccc}
      \lpquote Q \rpquote & & x \nameeq \lpquote P \rpquote \\
      x & & otherwise \\
    \end{array}
  \right. \nonumber
\end{eqnarray}

and $z$ is chosen distinct from $\quotep{P}$, $\quotep{Q}$, the free
names in $Q$, and all the names in $R$. Our $\alpha$-equivalence will
be built in the standard way from this substitution.

\begin{remark}\label{rem:no_self_referential_names}
  One consequence of these definitions is that $\forall P. \quotep{P}
  \not\in \freenames{P}$.
\end{remark}

\subsection{ Dynamic quote: an example }

Anticipating something of what's to come, consider applying the
substitution, $\widehat{\id{\{}u / z \id{\}}}$, to the following pair
of processes, $\lift{w}{y!(z)}$ and $w[ \lpquote y!(z) \rpquote ]$.

\begin{eqnarray}
	\lift{w}{y!(z)}\widehat{\id{\{}u / z \id{\}}}
		& = &
		\lift{w}{y!(u)} \nonumber\\
	w[ \lpquote y!(z) \rpquote ] \widehat{ \id{\{}u / z \id{\}} }
		& = &
		w[ \lpquote y!(z) \rpquote ] \nonumber
\end{eqnarray}

Because the body of the process between quotes is impervious to
substitution, we get radically different answers. In fact, by
examining the first process in an input context,
e.g. $x?(z).\lift{w}{y!(z)}$, we see that the process under the lift
operator may be shaped by prefixed inputs binding a name inside it. In
this sense, the lift operator will be seen as a way to dynamically
construct processes before reifying them as names.

Finally equipped with these standard features we can present the
dynamics of the calculus.

\subsubsection{Operational semantics} 

Finally, we introduce the computational dynamics. What marks these
algebras as distinct from other more traditionally studied algebraic
structures, e.g. vector spaces or polynomial rings, is the manner in
which dynamics is captured. In traditional structures, dynamics is typically
expressed through morphisms between such structures, as in linear maps
between vector spaces or morphisms between rings. In algebras
associated with the semantics of computation, the dynamics is
expressed as part of the algebraic structure itself, through a
reduction reduction relation typically denoted by $\red$. Below, we
give a recursive presentation of this relation for the calculus used
in the encoding.

$\red \subseteq \pi \times \pi$
$\red : \pi \to \mathcal{P}(\pi)$

\begin{mathpar}
  \inferrule* [lab=Comm] { \textsf{match}( x_{src}, x_{trgt} ) } { x_{trgt}?(y)P \; | \; x_{src}!\langle {Q} \rangle \red P\{\quotep{Q}/y}\} }
  \and \\
  \inferrule* [lab=Par] {{P} \red {P}'} {{{P} | {Q}} \red {{P}' | {Q}}}
  \and
  \inferrule* [lab=Equiv]{{{P} \scong {P}'} \andalso {{P}' \red {Q}'} \andalso {{Q}' \scong {Q}}}{{P} \red {Q}}
\end{mathpar}

\begin{eqnarray*}
  match_{\equiv} (\quotep{P},\quotep{Q}) & := & P \equiv Q \\
  match_{\dagger}(\quotep{P},\quotep{Q}) & := & \forall R. P|Q \red^{*} R => R \red^{*} 0 \\
  match_{K}(\quotep{P},\quotep{Q}) & := & K \mbox{ for some context } K
\end{eqnarray*}

$u?(x)P | u!\langle Q \rangle \red P\{\quotep{Q}/x\}$

%We write $\wred$ for $\red^*$, and $P\red$ if $\exists Q $ such that $ P \red Q$.
We write $P\red$ if $\exists Q $ such that $ P \red Q$ and $P\not\red$, otherwise.

\section{Replication}

As mentioned before, it is known that replication (and hence
recursion) can be implemented in a higher-order process algebra
\cite{SangiorgiWalker}. As our first example of calculation with the
machinery thus far presented we give the construction explicitly in
the {\rhoc}.

\begin{eqnarray}
	D_{x} & := & \prefix{x}{y}{(\binpar{\outputp{x}{y}}{@{y}})} \nonumber\\
	\bangp_{x}{P} & := & \binpar{{x}!\langle{\binpar{D_{x}}{P}}\rangle}{D_{x}} \nonumber
\end{eqnarray}

\begin{eqnarray}
	\bangp_{x}{P} & & \nonumber\\
	=
	& {x}!\langle{(\prefix{x}{y}{(\outputp{x}{y} | @{y})) | P}}\rangle 
	      | \prefix{x}{y}{(\outputp{x}{y} | @{y})} & \nonumber\\
	\red
	& (\outputp{x}{y} | @{y})\substn{\quotep{(\prefix{x}{y}{(@{y} | \outputp{x}{y})) | P}}}{y} & \nonumber\\
	=
	& \outputp{x}{\quotep{(\prefix{x}{y}{(\outputp{x}{y} | @{y})) | P}}}
	  | {(\prefix{x}{y}{(\outputp{x}{y} | @{y})) | P}} & \nonumber\\
	\red
	& \ldots & \nonumber\\
	\red^*
	& P | P | \ldots & \nonumber
\end{eqnarray}

Of course, this encoding, as an implementation, runs away, unfolding
$\bangp{P}$ eagerly. A lazier and more implementable replication
operator, restricted to input-guarded processes, may be obtained as follows.

\begin{eqnarray}
\bangp{\prefix{u}{v}{P}} 
	:= 
	\binpar{\lift{x}{\prefix{u}{v}{(\binpar{D(x)}{P})}}}{D(x)} \nonumber
\end{eqnarray}

\begin{remark}
  Note that the lazier definition still does not deal with summation
  or mixed summation (i.e. sums over input and output). The reader is
  invited to construct definitions of replication that deal with these
  features. 

  Further, the definitions are parameterized in a name, $x$. Can you,
  gentle reader, make a definition that eliminates this parameter and
  guarantees no accidental interaction between the replication
  machinery and the process being replicated -- i.e. no accidental
  sharing of names used by the process to get its work done and the
  name(s) used by the replication to effect copying. This latter
  revision of the definition of replication is crucial to obtaining
  the expected identity $!!P \sim !P$.
\end{remark}

\begin{remark}\label{rem:paradoxical_combinator}
  The reader familiar with the lambda calculus will have noticed the
  similarity between $D$ and the paradoxical combinator.

  [Ed. note: the existence of this seems to suggest we have to be more
  restrictive on the set of processes and names we admit if we are to
  support no-cloning.]
\end{remark}

\subsubsection{Bisimulation}

The computational dynamics gives rise to another kind of equivalence,
the equivalence of computational behavior. As previously mentioned
this is typically captured \emph{via} some form of bisimulation.

% The notion we use in this paper is weak barbed bisimulation
% \cite{milner91polyadicpi}.

The notion we use in this paper is derived from weak barbed
bisimulation \cite{milner91polyadicpi}. 

\begin{definition}
An \emph{observation relation}, $\downarrow_{\mathcal N}$, over a set
of names, $\mathcal N$, is the smallest relation satisfying the rules
below.

\infrule[Out-barb]{y \in {\mathcal N}, \; x \nameeq y}
		  {\outputp{x}{v} \downarrow_{\mathcal N} x}
\infrule[Par-barb]{\mbox{$P\downarrow_{\mathcal N} x$ or $Q\downarrow_{\mathcal N} x$}}
		  {\binpar{P}{Q} \downarrow_{\mathcal N} x}

We write $P \Downarrow_{\mathcal N} x$ if there is $Q$ such that 
$P \wred Q$ and $Q \downarrow_{\mathcal N} x$.
\end{definition}

\begin{definition}
%\label{def.bbisim}
An  ${\mathcal N}$-\emph{barbed bisimulation} over a set of names, ${\mathcal N}$, is a symmetric binary relation 
${\mathcal S}_{\mathcal N}$ between agents such that $P\rel{S}_{\mathcal N}Q$ implies:
\begin{enumerate}
\item If $P \red P'$ then $Q \wred Q'$ and $P'\rel{S}_{\mathcal N} Q'$.
\item If $P\downarrow_{\mathcal N} x$, then $Q\Downarrow_{\mathcal N} x$.
\end{enumerate}
$P$ is ${\mathcal N}$-barbed bisimilar to $Q$, written
$P \wbbisim_{\mathcal N} Q$, if $P \rel{S}_{\mathcal N} Q$ for some ${\mathcal N}$-barbed bisimulation ${\mathcal S}_{\mathcal N}$.
\end{definition}

$\mathcal{R} \subseteq \pi \times \pi$

$P \mathcal{R} Q => \forall P'. P \red P' \Rightarrow \exists Q'. Q \red Q', P' \mathcal{R} Q'$

$P \vdash x \Rightarrow Q \vdash x$

\begin{mathpar}
  \inferrule*[lab=Out-barb]{x \nameeq y}{{y}!\langle{Q}\rangle \vdash x}
  \and
  \inferrule*[lab=Par-barb]{\mbox{$P\vdash x$ or $Q\vdash x$}}{\binpar{P}{Q} \vdash x}
\end{mathpar}

\subsubsection{Contexts}

One of the principle advantages of computational calculi like the
$\pi$-calculus is a well-defined notion of context,
contextual-equivalence and a correlation between
contextual-equivalence and notions of bisimulation. The notion of
context allows the decomposition of a process into (sub-)process and
its syntactic environment, its context. Thus, a context may be
thought of as a process with a ``hole'' (written $\Box$) in it. The
application of a context $M$ to a process $P$, written $M[P]$, is
tantamount to filling the hole in $M$ with $P$. In this paper we do
not need the full weight of this theory, but do make use of the notion
of context in the proof the main theorem. 

\begin{mathpar}
  \inferrule* [lab=summation] {} {{M_{M},M_{N}} \bc \Box \;|\; x.M_{A} \;|\; M_{M}+M_{N}}
  \and
  \inferrule* [lab=agent] {} {{M_{A}} \bc (\vec{x})M_{P} \;| \; \clift{P_0,\ldots,M_{P},\ldots,P_N}}
  \and \\
  \inferrule* [lab=process] {} {{M_{P}} \bc M_{N} \;| \;P|M_{P} }
\end{mathpar} 

\begin{mathpar}
  \inferrule* [lab=sychronization] {} {M_{N} \bc \Box \;|\; x?M_{F} \;|\; x!M_{C}}
  \and
  \inferrule* [lab=abstraction] {} {{M_{F}} \bc (x)M_{P} }
  \and
  \inferrule* [lab=concretion] {} {{M_{C}} \bc \langle M_{P} \rangle }
  \and \\
  \inferrule* [lab=process] {} {{M_{P}} \bc M_{N} \;| \;P|M_{P} }
\end{mathpar}

\begin{definition}[contextual application] Given a context $M$, and
  process $P$, we define the \emph{contextual application}, $M[P] :=
  M\{P/\Box\}$. That is, the contextual application of M to P is the
  substitution of $P$ for $\Box$ in $M$.
\end{definition}

$\meaningof{-} : L \to \mathcal{P}(\pi)$

\begin{mathpar}
  \inferrule* [lab=collection] {} {\meaningof{true} = \pi, \and \meaningof{~E} = \pi \setminus \meaningof{E}, \and \meaningof{E_{1} \& E_{2}} = \meaningof{E_{1}} \cap \meaningof{E_{2}}}
\end{mathpar}

\begin{mathpar}
  \inferrule* [lab=structure] {} {\meaningof{0} = \{ P \in \pi | P \equiv 0 \}, \and \\ \meaningof{E_1 | E_2} = \{ P \in \pi | P \equiv P_{1} | P_{2}, P_{1} \in \meaningof{E_{1}}, P_{2} \in \meaningof{E_2}\} }
\end{mathpar}

\begin{mathpar}
 \inferrule* [lab=behavior] {} {\meaningof{\langle a?b \rangle E} = \{ P \in \pi | P \equiv Q | u?(y)P', \\ \and \\\\ \and \\ \;\;\; u \in \meaningof{a}, \forall z.P'\{z/y\} \in \meaningof{E\{z/b\}}\}, \and \\ \meaningof{a!E} = \{ P \in \pi | P \equiv Q | x!\langle P' \rangle, x \in \meaningof{a} P' \in \meaningof{E}\} }
\end{mathpar}

\begin{mathpar}
 \inferrule* [lab=nominal] {} {\meaningof{\quotep{E}} = \{ \quotep{P} \in \quotep{\pi} | P \in \meaningof{E} \}, \and \meaningof{\quotep{P}} = \{ \quotep{Q} \in \quotep{\pi} | P \equiv Q \} \and \\ \meaningof{@\quotep{E}} = \{ P \in \pi | P \equiv @x, x \in \meaningof{E} \}}
\end{mathpar}

\begin{eqnarray*}
  \\
  \meaningof{-} : TS \to ST
\end{eqnarray*}

\begin{eqnarray*}
  \\
  L : TS \to ST
\end{eqnarray*}

\begin{eqnarray*}
  \\
  P \models E \iff P \in \meaningof{E}
\end{eqnarray*}

\begin{eqnarray*}
  P \approx_{L} Q \iff \forall E \in L. P \models E \iff Q \models E
\end{eqnarray*}

\begin{eqnarray*}
  P \approx_{K} Q
\end{eqnarray*}

\begin{eqnarray*}
  P \approx Q
\end{eqnarray*}

$\approx_{K} = \approx = \approx_{L}$

\subsubsection{Contextual duality}

Note that contexts extend the quotation operation to a family of
operations from processes to names. Given a context, $M$, we can
define a \emph{nominal context}, $\quotep{M}$ by $\quotep{M}[P] :=
\quotep{M[P]}$. To foreshadow what is to come we observe that these
operations enjoy a duality with processes very much like the duality
between vectors and maps from vectors to scalars.

Further, because the calculus is essentially higher-order, we have a
correspondence between contexts and processes. More specifically,
given a name $x$ and a context $M$ we can construct $M^{*}_{x}$ such
that 

\begin{mathpar}
  M^{*}_{x} | \lift{x}{P} \red M[P]
\end{mathpar}

namely,

\begin{mathpar}
  M^{*}_{x} := x?(u).M[\dropn{u}]
\end{mathpar}

The dependence of $M^{*}_{x}$ on a name makes it an abstraction, 

\begin{mathpar}
  M^{*} := (x)x?(u).M[\dropn{u}]
\end{mathpar}

\subsection{Additional notation}

It will sometimes be convenient to denote the process a name
quotes. We already have the notation $x = \quotep{P}$, but it will be
convenient to introduce an alternate notation, $\procn{x}$, when we
want to emphasize the connection to the use of the name. Note that, by
virtue of name equivalence, $\quotep{\procn{x}} \nameeq x$; so, the
notation is consistent with previous definitions.

Further, because names have structure it is possible to effect
substitutions on the basis of that structure. This means we need to
upgrade our notation for substitutions, which we accomplish by
adapting comprehension notation. Thus,

\begin{mathpar}
  P\{ y / x : x \in S \}
\end{mathpar}

is interpreted to mean the process derived from P by replacing (in a
capture-avoiding manner) each occurrence of $x$ in $S$ by $y$. For example,

\begin{mathpar}
  P\{ \quotep{\procn{x}|\procn{x}} / x : x \in \freenames{P} \}
\end{mathpar}

will replace each (occurrence) of a free name $x$ in $P$ by
$\quotep{\procn{x}|\procn{x}}$.

Also, we will avail ourselves of the notation $x^{L}$ and $x^{R}$ to
denote injections of a name into disjoint copies of the name
space. There are numerous ways to accomplish this. One example can be
found in \cite{MeredithR05}. This notation overloads to vectors of
names: $\vec{x}^{\pi} := (x_{i}^{\pi} \; : \; 0 \leq i < |\vec{x}| )$ where $\pi \in \{L,R\}$.

We also use $P^{\Box} := P|\Box$.

In \cite{MeredithR05} an interpretation of the new operator is
given. It turns out that there are several possible interpretations
all enjoying the requisite algebraic properties of the operator (see
\cite{milner91polyadicpi}). We will therefore make liberal use of
$(\nu\; \vec{x})P$.

% subsection the_syntax_and_semantics_of_the_notation_system (end)   

\input{qm2pi.qmops} 

\input{qm2pi.sterngerlach} 

\input{qm2pi.metric} 

% section concurrent_process_calculi (end)

%\input{qm2pi.proofsketch}

% section proof sketch (end)

%\input{qm2pi.slviaknots} 

% section spatial logic via knots (end)

\input{qm2pi.conclusion}

% section conclusion (end)

%\input{qm2pi.dtcodes} 

% section wiring algorithm (end)

\input{qm2pi.ack} 

% section acknowledgments (end)

\newpage


\bibliographystyle{plain}   
\bibliography{../../biblios/main.bib}

\input{qm2pi.rhodetails}

\end{document}

 

% section concurrent_process_calculi (end)

%\documentclass[12pt]{llncs}
%\documentclass{jktr}

\usepackage[pdftex]{hyperref}                   
\usepackage {listings}
\usepackage {mathpartir}
\usepackage{bcprules}
%\usepackage{listings}
                       
\usepackage{graphicx} 
%\usepackage[margins=2.5cm,nohead,nofoot]{geometry}
%\usepackage{geometry}
\usepackage{amsfonts}
\usepackage{amstext}
\usepackage{latexsym}
\usepackage{amssymb}
\usepackage{color}


%\include{myPreamble}
\include{qm2pi.local} 

%\ifpdf
%\usepackage[pdftex]{graphicx}
%\else
%\usepackage{graphicx}
%\fi

 % \ifpdf
%  \usepackage{pdfsync}
%  \if


%\title{Brief Article}
%\author{David F. Snyder}
%\author{L.G. Meredith}

%\address{Dept. of Math., Texas State University--San Marcos, San Marcos, TX 78666}
       
\pagestyle{empty}


\begin{document}

\lstset{language=[Objective]Caml,frame=shadowbox}

\input{qm2pi.front}

% section front matter (end)

\input{qm2pi.intro} 
 
% section introduction (end)

% \input{qm2pi.knotations} 

% section notation (end)

\input{qm2pi.process.calculi} 

% section concurrent_process_calculi_and_spatial_logics_ (end)
    
%\input{qm2pi.knots2pi} 

%\input{qm2pi.trefoil} 

%\input{qm2pi.mainthm} 

% subsection basic_interpretation (end)

%\input{qm2pi.rho.presentation} 
\subsection{The syntax and semantics of the notation system}\label{sub:the_syntax_and_semantics_of_the_notation_system} % (fold)

We now summarize a technical presentation of the calculus that
embodies our theory of dynamics. The typical presentation of such a
calculus follows the style of giving generators and relations on
them. The grammar, below, describing term constructors, freely
generates the set of processes, $\Proc$. This set is then quotiented
by a relation known as structural congruence and it is over this set
that the notion of dynamics is expressed. This presentation is
essentially that of \cite{MeredithR05} with the addition of
polyadicity and summation. For readability we have relegated some of
the technical subtleties to an appendix.

\subsubsection{Process grammar}\label{subsub:process_grammar}

\begin{mathpar}
  \inferrule* [lab=synchronization] {} {{M} \bc \pzero \;|\; x?F \;|\; x!C }
  \and
  \inferrule* [lab=abstraction] {} {{F} \bc (x)P}
  \and
  \inferrule* [lab=concretion] {} {{C} \bc \langle Q \rangle}
  \and
  \inferrule* [lab=process] {} {{P,Q} \bc M \;| \;P|Q \;|\; @{x}}
  \and
  \inferrule* [lab=name] {} {{x} \bc \quotep{P}}
\end{mathpar} 

Note that $\vec{x}$ (resp. $\vec{P}$) denotes a vector of names
(resp. processes) of length $|\vec{x}|$ (resp. $|\vec{P}|$). We adopt
the following useful abbreviations.

\begin{mathpar}
   x?(\vec{y}).P := x.(\vec{y})P \and  x\clift{\vec{P}} := x.\clift{\vec{P}}
   \and x!(y) := \lift{x}{\dropn{y}}
   \and \Pi_{i=0}^{n-1}P_i := P_0 | \ldots | P_{n-1}
\end{mathpar}

\subsubsection{Structural congruence}

\paragraph{Free and bound names and alpha-equivalence.} At the
core of structural equivalence is alpha-equivalence which identifies
process that are the same up to a change of variable. Formally, we
recognize the distinction between free and bound names. The free names
of a process, $\freenames{P}$, may be calculated recursively as
follows:

\begin{mathpar}
\freenames{\pzero} := \emptyset
  \and \\
  \freenames{x?(y).P} := \{ x \} \cup (\freenames{P} \setminus \{ y \})
  \and 
  \freenames{x!\langle P \rangle} := \{ x \} \cup \{ P \} 
  \and \\
  \freenames{P|Q} := \freenames{P} \cup \freenames{Q}
  \and \\
  \freenames{@{x}} := \{ x \}
\end{mathpar}

$\pi$
$\quotep{\pi}$

$\freenames{-} : \pi \to \mathcal{P}(\quotep{\pi})$

\begin{eqnarray*}
  \freenames{\pzero} & := & \emptyset \\
  \freenames{x?(y).P} & := & \{ x \} \cup (\freenames{P} \setminus \{ y \}) \\
  \freenames{x!\langle P \rangle} & := & \{ x \} \cup \{ P \} \\
  \freenames{P|Q} & := & \freenames{P} \cup \freenames{Q} \\
  \freenames{\dropn{x}} & := & \{ x \}
\end{eqnarray*}

The bound names of a process, $\boundnames{P}$, are those names occurring in $P$
that are not free. For example, in $x?(y).0$, the name $x$ is free, while $y$ is bound.

\begin{mathpar}
  \inferrule* [lab=monoidal-laws] {} { P|Q \equiv Q|P \and P|0 \equiv P \and P|(Q|R) \equiv (P|Q)|R }
\end{mathpar}

\begin{mathpar}
  \inferrule* [lab=alpha-equivalence] {} { (x)P \equiv (y)P\{y/x\} \and y \not\in \freenames{P} }
\end{mathpar}

\begin{definition}
Then two processes, $P,Q$, are alpha-equivalent if $P = Q\{\vec{y}/\vec{x}\}$ for
some $\vec{x} \in \boundnames{Q},\vec{y} \in \boundnames{P}$, where $Q\{\vec{y}/\vec{x}\}$
denotes the capture-avoiding substitution of $\vec{y}$ for $\vec{x}$ in $Q$.
\end{definition}

\begin{definition}
  The {\em structural congruence} \cite{SangiorgiWalker} , $\equiv$,
  between processes is the least congruence containing
  alpha-equivalence, satisfying the abelian monoid laws
  (associativity, commutativity and $\pzero$ as identity) for parallel
  composition $|$ and for summation $+$.
\end{definition}

\subsection{Name equivalence}

We take name equivalence, written $\nameeq$, to be the smallest
equivalence relation generated by the following rules.

\begin{mathpar}
\inferrule*[lab=Quote-drop]
{ }
{ \quotep{@{x}} \nameeq x }

\inferrule*[lab=Struct-equiv]
{ P \scong Q }
{ \quotep{P} \nameeq \quotep{Q} }
\end{mathpar}

The astute reader will have noticed that the mutual recursion of names
and processes imposes a mutual recursion on alpha-equivalence and
structural equivalence via name-equivalence. Fortunately, all of this
works out pleasantly and we may calculate in the natural way, free of
concern. The reader interested in the details is referred to the
appendix \ref{appendix:rho_details}.

\subsection{Substitution}

We use $\Proc$ for the set of processes, $\QProc$ for the set of
names, and $\id{\{}\vec{y} / \vec{x} \id{\}}$ to denote partial maps,
$s : \QProc \rightarrow \QProc$. A map, $s$ lifts, uniquely, to a map
on process terms, $\widehat{s} : \Proc \rightarrow \Proc$ by the
following equations.

\begin{mathpar}
  (0) \psubstp{Q}{P} := 0 \\
  (R \juxtap S) \psubstp{Q}{P}
  :=    
  (R)\psubstp{Q}{P} \juxtap (S) \psubstp{Q}{P} \\
  (x?(y).R) \psubstp{Q}{P}    
  :=    
  (x)\substp{Q}{P} (z)\concat( (R \psubstn{z}{y}) \psubstp{Q}{P} ) \\
  (\lift{x}{R}) \psubstp{Q}{P}  
  :=
  \lift{(x)\substp{Q}{P}}{ R \psubstp{Q}{P} } \\
%   (\dropn{x})  \psubstp{Q}{P}       
%   := 
%   \left\{ 
%     \begin{array}{ccc} 
%       \dropn{\quotep{Q}} & & x \nameeq \quotep{P} \\
%       \dropn{x} & & otherwise \\
%     \end{array}
%   \right. 
  (\dropn{x})  \psubstp{Q}{P}       
  := 
  \left\{ 
    \begin{array}{ccc} 
      Q & & x \nameeq \quotep{P} \\
      \dropn{x} & & otherwise \\
    \end{array}
  \right.
\end{mathpar}
 

where

\begin{eqnarray}
  (x)\id{\{} \lpquote Q \rpquote / \lpquote P \rpquote \id{\}}            = 
  \left\{ 
    \begin{array}{ccc}
      \lpquote Q \rpquote & & x \nameeq \lpquote P \rpquote \\
      x & & otherwise \\
    \end{array}
  \right. \nonumber
\end{eqnarray}

and $z$ is chosen distinct from $\quotep{P}$, $\quotep{Q}$, the free
names in $Q$, and all the names in $R$. Our $\alpha$-equivalence will
be built in the standard way from this substitution.

\begin{remark}\label{rem:no_self_referential_names}
  One consequence of these definitions is that $\forall P. \quotep{P}
  \not\in \freenames{P}$.
\end{remark}

\subsection{ Dynamic quote: an example }

Anticipating something of what's to come, consider applying the
substitution, $\widehat{\id{\{}u / z \id{\}}}$, to the following pair
of processes, $\lift{w}{y!(z)}$ and $w[ \lpquote y!(z) \rpquote ]$.

\begin{eqnarray}
	\lift{w}{y!(z)}\widehat{\id{\{}u / z \id{\}}}
		& = &
		\lift{w}{y!(u)} \nonumber\\
	w[ \lpquote y!(z) \rpquote ] \widehat{ \id{\{}u / z \id{\}} }
		& = &
		w[ \lpquote y!(z) \rpquote ] \nonumber
\end{eqnarray}

Because the body of the process between quotes is impervious to
substitution, we get radically different answers. In fact, by
examining the first process in an input context,
e.g. $x?(z).\lift{w}{y!(z)}$, we see that the process under the lift
operator may be shaped by prefixed inputs binding a name inside it. In
this sense, the lift operator will be seen as a way to dynamically
construct processes before reifying them as names.

Finally equipped with these standard features we can present the
dynamics of the calculus.

\subsubsection{Operational semantics} 

Finally, we introduce the computational dynamics. What marks these
algebras as distinct from other more traditionally studied algebraic
structures, e.g. vector spaces or polynomial rings, is the manner in
which dynamics is captured. In traditional structures, dynamics is typically
expressed through morphisms between such structures, as in linear maps
between vector spaces or morphisms between rings. In algebras
associated with the semantics of computation, the dynamics is
expressed as part of the algebraic structure itself, through a
reduction reduction relation typically denoted by $\red$. Below, we
give a recursive presentation of this relation for the calculus used
in the encoding.

$\red \subseteq \pi \times \pi$
$\red : \pi \to \mathcal{P}(\pi)$

\begin{mathpar}
  \inferrule* [lab=Comm] { \textsf{match}( x_{src}, x_{trgt} ) } { x_{trgt}?(y)P \; | \; x_{src}!\langle {Q} \rangle \red P\{\quotep{Q}/y}\} }
  \and \\
  \inferrule* [lab=Par] {{P} \red {P}'} {{{P} | {Q}} \red {{P}' | {Q}}}
  \and
  \inferrule* [lab=Equiv]{{{P} \scong {P}'} \andalso {{P}' \red {Q}'} \andalso {{Q}' \scong {Q}}}{{P} \red {Q}}
\end{mathpar}

\begin{eqnarray*}
  match_{\equiv} (\quotep{P},\quotep{Q}) & := & P \equiv Q \\
  match_{\dagger}(\quotep{P},\quotep{Q}) & := & \forall R. P|Q \red^{*} R => R \red^{*} 0 \\
  match_{K}(\quotep{P},\quotep{Q}) & := & K \mbox{ for some context } K
\end{eqnarray*}

$u?(x)P | u!\langle Q \rangle \red P\{\quotep{Q}/x\}$

%We write $\wred$ for $\red^*$, and $P\red$ if $\exists Q $ such that $ P \red Q$.
We write $P\red$ if $\exists Q $ such that $ P \red Q$ and $P\not\red$, otherwise.

\section{Replication}

As mentioned before, it is known that replication (and hence
recursion) can be implemented in a higher-order process algebra
\cite{SangiorgiWalker}. As our first example of calculation with the
machinery thus far presented we give the construction explicitly in
the {\rhoc}.

\begin{eqnarray}
	D_{x} & := & \prefix{x}{y}{(\binpar{\outputp{x}{y}}{@{y}})} \nonumber\\
	\bangp_{x}{P} & := & \binpar{{x}!\langle{\binpar{D_{x}}{P}}\rangle}{D_{x}} \nonumber
\end{eqnarray}

\begin{eqnarray}
	\bangp_{x}{P} & & \nonumber\\
	=
	& {x}!\langle{(\prefix{x}{y}{(\outputp{x}{y} | @{y})) | P}}\rangle 
	      | \prefix{x}{y}{(\outputp{x}{y} | @{y})} & \nonumber\\
	\red
	& (\outputp{x}{y} | @{y})\substn{\quotep{(\prefix{x}{y}{(@{y} | \outputp{x}{y})) | P}}}{y} & \nonumber\\
	=
	& \outputp{x}{\quotep{(\prefix{x}{y}{(\outputp{x}{y} | @{y})) | P}}}
	  | {(\prefix{x}{y}{(\outputp{x}{y} | @{y})) | P}} & \nonumber\\
	\red
	& \ldots & \nonumber\\
	\red^*
	& P | P | \ldots & \nonumber
\end{eqnarray}

Of course, this encoding, as an implementation, runs away, unfolding
$\bangp{P}$ eagerly. A lazier and more implementable replication
operator, restricted to input-guarded processes, may be obtained as follows.

\begin{eqnarray}
\bangp{\prefix{u}{v}{P}} 
	:= 
	\binpar{\lift{x}{\prefix{u}{v}{(\binpar{D(x)}{P})}}}{D(x)} \nonumber
\end{eqnarray}

\begin{remark}
  Note that the lazier definition still does not deal with summation
  or mixed summation (i.e. sums over input and output). The reader is
  invited to construct definitions of replication that deal with these
  features. 

  Further, the definitions are parameterized in a name, $x$. Can you,
  gentle reader, make a definition that eliminates this parameter and
  guarantees no accidental interaction between the replication
  machinery and the process being replicated -- i.e. no accidental
  sharing of names used by the process to get its work done and the
  name(s) used by the replication to effect copying. This latter
  revision of the definition of replication is crucial to obtaining
  the expected identity $!!P \sim !P$.
\end{remark}

\begin{remark}\label{rem:paradoxical_combinator}
  The reader familiar with the lambda calculus will have noticed the
  similarity between $D$ and the paradoxical combinator.

  [Ed. note: the existence of this seems to suggest we have to be more
  restrictive on the set of processes and names we admit if we are to
  support no-cloning.]
\end{remark}

\subsubsection{Bisimulation}

The computational dynamics gives rise to another kind of equivalence,
the equivalence of computational behavior. As previously mentioned
this is typically captured \emph{via} some form of bisimulation.

% The notion we use in this paper is weak barbed bisimulation
% \cite{milner91polyadicpi}.

The notion we use in this paper is derived from weak barbed
bisimulation \cite{milner91polyadicpi}. 

\begin{definition}
An \emph{observation relation}, $\downarrow_{\mathcal N}$, over a set
of names, $\mathcal N$, is the smallest relation satisfying the rules
below.

\infrule[Out-barb]{y \in {\mathcal N}, \; x \nameeq y}
		  {\outputp{x}{v} \downarrow_{\mathcal N} x}
\infrule[Par-barb]{\mbox{$P\downarrow_{\mathcal N} x$ or $Q\downarrow_{\mathcal N} x$}}
		  {\binpar{P}{Q} \downarrow_{\mathcal N} x}

We write $P \Downarrow_{\mathcal N} x$ if there is $Q$ such that 
$P \wred Q$ and $Q \downarrow_{\mathcal N} x$.
\end{definition}

\begin{definition}
%\label{def.bbisim}
An  ${\mathcal N}$-\emph{barbed bisimulation} over a set of names, ${\mathcal N}$, is a symmetric binary relation 
${\mathcal S}_{\mathcal N}$ between agents such that $P\rel{S}_{\mathcal N}Q$ implies:
\begin{enumerate}
\item If $P \red P'$ then $Q \wred Q'$ and $P'\rel{S}_{\mathcal N} Q'$.
\item If $P\downarrow_{\mathcal N} x$, then $Q\Downarrow_{\mathcal N} x$.
\end{enumerate}
$P$ is ${\mathcal N}$-barbed bisimilar to $Q$, written
$P \wbbisim_{\mathcal N} Q$, if $P \rel{S}_{\mathcal N} Q$ for some ${\mathcal N}$-barbed bisimulation ${\mathcal S}_{\mathcal N}$.
\end{definition}

$\mathcal{R} \subseteq \pi \times \pi$

$P \mathcal{R} Q => \forall P'. P \red P' \Rightarrow \exists Q'. Q \red Q', P' \mathcal{R} Q'$

$P \vdash x \Rightarrow Q \vdash x$

\begin{mathpar}
  \inferrule*[lab=Out-barb]{x \nameeq y}{{y}!\langle{Q}\rangle \vdash x}
  \and
  \inferrule*[lab=Par-barb]{\mbox{$P\vdash x$ or $Q\vdash x$}}{\binpar{P}{Q} \vdash x}
\end{mathpar}

\subsubsection{Contexts}

One of the principle advantages of computational calculi like the
$\pi$-calculus is a well-defined notion of context,
contextual-equivalence and a correlation between
contextual-equivalence and notions of bisimulation. The notion of
context allows the decomposition of a process into (sub-)process and
its syntactic environment, its context. Thus, a context may be
thought of as a process with a ``hole'' (written $\Box$) in it. The
application of a context $M$ to a process $P$, written $M[P]$, is
tantamount to filling the hole in $M$ with $P$. In this paper we do
not need the full weight of this theory, but do make use of the notion
of context in the proof the main theorem. 

\begin{mathpar}
  \inferrule* [lab=summation] {} {{M_{M},M_{N}} \bc \Box \;|\; x.M_{A} \;|\; M_{M}+M_{N}}
  \and
  \inferrule* [lab=agent] {} {{M_{A}} \bc (\vec{x})M_{P} \;| \; \clift{P_0,\ldots,M_{P},\ldots,P_N}}
  \and \\
  \inferrule* [lab=process] {} {{M_{P}} \bc M_{N} \;| \;P|M_{P} }
\end{mathpar} 

\begin{mathpar}
  \inferrule* [lab=sychronization] {} {M_{N} \bc \Box \;|\; x?M_{F} \;|\; x!M_{C}}
  \and
  \inferrule* [lab=abstraction] {} {{M_{F}} \bc (x)M_{P} }
  \and
  \inferrule* [lab=concretion] {} {{M_{C}} \bc \langle M_{P} \rangle }
  \and \\
  \inferrule* [lab=process] {} {{M_{P}} \bc M_{N} \;| \;P|M_{P} }
\end{mathpar}

\begin{definition}[contextual application] Given a context $M$, and
  process $P$, we define the \emph{contextual application}, $M[P] :=
  M\{P/\Box\}$. That is, the contextual application of M to P is the
  substitution of $P$ for $\Box$ in $M$.
\end{definition}

$\meaningof{-} : L \to \mathcal{P}(\pi)$

\begin{mathpar}
  \inferrule* [lab=collection] {} {\meaningof{true} = \pi, \and \meaningof{~E} = \pi \setminus \meaningof{E}, \and \meaningof{E_{1} \& E_{2}} = \meaningof{E_{1}} \cap \meaningof{E_{2}}}
\end{mathpar}

\begin{mathpar}
  \inferrule* [lab=structure] {} {\meaningof{0} = \{ P \in \pi | P \equiv 0 \}, \and \\ \meaningof{E_1 | E_2} = \{ P \in \pi | P \equiv P_{1} | P_{2}, P_{1} \in \meaningof{E_{1}}, P_{2} \in \meaningof{E_2}\} }
\end{mathpar}

\begin{mathpar}
 \inferrule* [lab=behavior] {} {\meaningof{\langle a?b \rangle E} = \{ P \in \pi | P \equiv Q | u?(y)P', \\ \and \\\\ \and \\ \;\;\; u \in \meaningof{a}, \forall z.P'\{z/y\} \in \meaningof{E\{z/b\}}\}, \and \\ \meaningof{a!E} = \{ P \in \pi | P \equiv Q | x!\langle P' \rangle, x \in \meaningof{a} P' \in \meaningof{E}\} }
\end{mathpar}

\begin{mathpar}
 \inferrule* [lab=nominal] {} {\meaningof{\quotep{E}} = \{ \quotep{P} \in \quotep{\pi} | P \in \meaningof{E} \}, \and \meaningof{\quotep{P}} = \{ \quotep{Q} \in \quotep{\pi} | P \equiv Q \} \and \\ \meaningof{@\quotep{E}} = \{ P \in \pi | P \equiv @x, x \in \meaningof{E} \}}
\end{mathpar}

\begin{eqnarray*}
  \\
  \meaningof{-} : TS \to ST
\end{eqnarray*}

\begin{eqnarray*}
  \\
  L : TS \to ST
\end{eqnarray*}

\begin{eqnarray*}
  \\
  P \models E \iff P \in \meaningof{E}
\end{eqnarray*}

\begin{eqnarray*}
  P \approx_{L} Q \iff \forall E \in L. P \models E \iff Q \models E
\end{eqnarray*}

\begin{eqnarray*}
  P \approx_{K} Q
\end{eqnarray*}

\begin{eqnarray*}
  P \approx Q
\end{eqnarray*}

$\approx_{K} = \approx = \approx_{L}$

\subsubsection{Contextual duality}

Note that contexts extend the quotation operation to a family of
operations from processes to names. Given a context, $M$, we can
define a \emph{nominal context}, $\quotep{M}$ by $\quotep{M}[P] :=
\quotep{M[P]}$. To foreshadow what is to come we observe that these
operations enjoy a duality with processes very much like the duality
between vectors and maps from vectors to scalars.

Further, because the calculus is essentially higher-order, we have a
correspondence between contexts and processes. More specifically,
given a name $x$ and a context $M$ we can construct $M^{*}_{x}$ such
that 

\begin{mathpar}
  M^{*}_{x} | \lift{x}{P} \red M[P]
\end{mathpar}

namely,

\begin{mathpar}
  M^{*}_{x} := x?(u).M[\dropn{u}]
\end{mathpar}

The dependence of $M^{*}_{x}$ on a name makes it an abstraction, 

\begin{mathpar}
  M^{*} := (x)x?(u).M[\dropn{u}]
\end{mathpar}

\subsection{Additional notation}

It will sometimes be convenient to denote the process a name
quotes. We already have the notation $x = \quotep{P}$, but it will be
convenient to introduce an alternate notation, $\procn{x}$, when we
want to emphasize the connection to the use of the name. Note that, by
virtue of name equivalence, $\quotep{\procn{x}} \nameeq x$; so, the
notation is consistent with previous definitions.

Further, because names have structure it is possible to effect
substitutions on the basis of that structure. This means we need to
upgrade our notation for substitutions, which we accomplish by
adapting comprehension notation. Thus,

\begin{mathpar}
  P\{ y / x : x \in S \}
\end{mathpar}

is interpreted to mean the process derived from P by replacing (in a
capture-avoiding manner) each occurrence of $x$ in $S$ by $y$. For example,

\begin{mathpar}
  P\{ \quotep{\procn{x}|\procn{x}} / x : x \in \freenames{P} \}
\end{mathpar}

will replace each (occurrence) of a free name $x$ in $P$ by
$\quotep{\procn{x}|\procn{x}}$.

Also, we will avail ourselves of the notation $x^{L}$ and $x^{R}$ to
denote injections of a name into disjoint copies of the name
space. There are numerous ways to accomplish this. One example can be
found in \cite{MeredithR05}. This notation overloads to vectors of
names: $\vec{x}^{\pi} := (x_{i}^{\pi} \; : \; 0 \leq i < |\vec{x}| )$ where $\pi \in \{L,R\}$.

We also use $P^{\Box} := P|\Box$.

In \cite{MeredithR05} an interpretation of the new operator is
given. It turns out that there are several possible interpretations
all enjoying the requisite algebraic properties of the operator (see
\cite{milner91polyadicpi}). We will therefore make liberal use of
$(\nu\; \vec{x})P$.

% subsection the_syntax_and_semantics_of_the_notation_system (end)   

\input{qm2pi.qmops} 

\input{qm2pi.sterngerlach} 

\input{qm2pi.metric} 

% section concurrent_process_calculi (end)

%\input{qm2pi.proofsketch}

% section proof sketch (end)

%\input{qm2pi.slviaknots} 

% section spatial logic via knots (end)

\input{qm2pi.conclusion}

% section conclusion (end)

%\input{qm2pi.dtcodes} 

% section wiring algorithm (end)

\input{qm2pi.ack} 

% section acknowledgments (end)

\newpage


\bibliographystyle{plain}   
\bibliography{../../biblios/main.bib}

\input{qm2pi.rhodetails}

\end{document}



% section proof sketch (end)

%\section{Unlikely characters: spatial logic for
  knots}\label{sub:characteristic_formulae} % (fold)

Associated to the mobile process calculi are a family of logics known
as the Hennessy-Milner logics. These logics typically enjoy a
semantics interpreting formulae as sets of processes that when
factored through the encoding outlined above allows an identification
of classes of knots with logical formulae. In the context of this
encoding the sub-family known as the spatial logics \cite{CairesC03}
\cite{CairesC04} \cite{Caires04} are of particular interest providing
several important features for expressing and reasoning about
properties (i.e. classes) of knots. We hint here at how this may be done.

%\begin{description}
%\item [structural connectives] 
\subsubsection{Structural connectives} The spatial logics enjoy
structural connectives corresponding, at the logical level, to the
parallel composition ($P | Q$) and new name ($(\nu \; x)P$)
connectives for processes. As illustrated in the examples below, these
connectives are extremely expressive given the shape of our encoding.
%\item [decideable satisfaction]

\subsubsection{Decideable satisfaction}
In \cite{Caires04} the satisfaction relation is shown to be decideable
for a rich class of processes. It further turns out that the image of
the our encoding is a proper subset of that class. This result
provides the basis for an algorithm by which to search for knots
enjoying a given property.
%\item [characteristic formulae]

\subsubsection{Characteristic formulae}
In the same paper \cite{Caires04} , Caires presents a means of calculating
characteristic formulae, selecting equivalence classes of processes
up to a pre--specified depth limit on the support set of names. Composed with our
encoding, this characteristic formula can be used to select
characteristic formulae for knots.
%\end{description}

\subsubsection{Spatial logic formulae}

The grammar below (segmented for comprehension) summarizes the syntax
of spatial logic formulae. We employ illustrative examples in the
sequel to provide an intuitive understanding of their meaning
referring the reader to \cite{Caires04} for a more detailed explication
of the semantics.

\begin{mathpar}
  \inferrule* [lab=boolean] {} {{A,B} \bc T \;|\; \neg A \;|\; A \wedge B \;|\; \eta = \eta'}
  \and
  \inferrule* [lab=spatial] {} {|\; \pzero \;|\; A | B \;|\; x \text{\textregistered} A \;|\; \forall x . A \;|\;  H x . A}
  \and
  \inferrule* [lab=behavioral] {} {|\; \alpha . A}
  \and 
  \inferrule* [lab=recursion] {} {|\; X(\vec{u}) \;|\; \mu X(\vec{u}) . A}
  \and
  \inferrule* [lab=action] {} {\alpha \bc \langle x?(\vec{y}) \rangle \;|\; \langle x!(\vec{y}) \rangle \;|\; \langle \tau \rangle}
  \and 
  \inferrule* [lab=name] {} {\eta \bc x \;|\; \tau}
\end{mathpar} 

% subsection characteristic_formulae (end)   	 

\subsection{Example formulae}\label{sub:example_formulae_} % (fold)

\subsubsection{Crossing as formula.}
% 
% \begin{align*}
%   \frac{d}{dx} \sin x &= \cos x 
%   & \frac{d}{dx} e^x &= e^x \\
%   \frac{d}{dx} \cos x &= - \sin x 
%   & \frac{d}{dx} \log x &= \frac{1}{x} \\
% \end{align*} 

\begin{align*}
 \mu C(x_{0},x_{1},y_{0},y_{1},u).&(\langle x_{0}?(z) \rangle(\langle u! \rangle\langle y_{1}!z \rangle C(x_{0},x_{1},y_{0},y_{1},u)) & \\
  & \wedge \langle y_{1}?(z) \rangle (\langle u! \rangle \langle x_{0}!z \rangle C(x_{0},x_{1},y_{0},y_{1},u)) & \\
  & \wedge \langle x_{1}?(z) \rangle (\langle u? \rangle \langle y_{0}!z \rangle C(x_{0},x_{1},y_{0},y_{1},u)) & \\
  & \wedge \langle y_{0}?(z) \rangle (\langle u? \rangle \langle x_{1}!z \rangle C(x_{0},x_{1},y_{0},y_{1},u))) &
\end{align*}

The lexicographical similarity between the shape of this formulae and
the shape of definition of the process representing a crossing reveals
the intuitive meaning of this formulae. It describes the capabilities
of a process that has the right to represent a crossing. For example
it picks out processes that may perform an input on the port $x_0$ in
its initial menu of capabilities. What differentiates the formula
from the process, however, is that the crossing process is the
smallest candidate to satisfy the formula. Infinitely many other
processes -- with internal behavior hidden behind this interface, so
to speak -- also satisfy this formula. Even this simple formula,
then, can be seen to open a new view onto knots, providing a
computational interpretation of \emph{virtual} knots.

Note that this formula is derived by hand. A similar formula can be
derived by employing Caires' calculation of characteristic formula
\cite{Caires04} to the process representing a crossing. In light of
this discussion, we let
$\meaningof{C}_{\phi}(x0,x1,y0,y1,u)$ denote a formula specifying the
dynamics we wish to capture of a crossing. To guarantee we preserve
the shape of the interface and minimal semantics we demand that
$\meaningof{C}_{\phi}(x0,x1,y0,y1,u) \Rightarrow
\textbf{C}(x0,x1,y0,y1,u)$ where $\textbf{C}(x0,x1,y0,y1,u)$ denotes
the formula above.
                            
\subsubsection{Crossing number constraints.}
The moral content of the context lemma (Lemma \ref{context}) is that the notion of
``locality'' in the Reidemeister moves is effectively captured by the
parallel composition operator of the process calculus. This intuition
extends through the logic. Given a formula,
$\meaningof{C}_{\phi}(x0,x1,y0,y1,u)$, we can use the structural
connectives to specify constraints on crossing numbers, such as at
least $n$ crossings, or exactly $n$ crossings.
\begin{mathpar}
  \inferrule* [lab=at-least-n] {} { K^{\geq n}_{\phi}(\vec{xs},\vec{ys}) := \Pi_{i=0}^{n-1} Hu . \meaningof{C}_{\phi}(xs_i,ys_i,u) | T }
  \and 
  \inferrule* [lab=exactly-n] {} { K^{= n}_{\phi}(\vec{xs},\vec{ys}) := \Pi_{i=0}^{n-1} Hu . \meaningof{C}_{\phi}(xs_i,ys_i,u) | \neg (\forall x_0,y_0,x_1,y_1,u . \meaningof{C}_{\phi}(x_0,y_0,x_1,y_1,u) | T) }
\end{mathpar}

To round out this section, recall that the encoding of an $n$-crossing
knot decomposes into a parallel composition of $n$ \emph{copies} of a
crossing process together with a wiring harness. To specify different
knot classes with the same crossing number amounts to specifying
logical constraints on the wiring harness. In the interest of space,
we defer examples to a forthcoming paper. Suffice it to say that both
the conditions ``alternating knot'' and ``contains the tangle
corresponding to 5/3'' are expressible. For example, it is possible to
calculate the characteristic formula of a process corresponding to the
tangle 5/3 and conjoin it into the classifying formula via the
composition connective of the logic.

Finally, we wish to observe that it is entirely within reason to
contemplate a more domain-specific version of spatial logic tailored
to the shape of processes in the image of the encoding. Such a
domain-specific logic would have a better claim to the title formal
language of knot properties.

% subsection example_formulae_ (end)

% section knots_as_processes (end) 

% section spatial logic via knots (end)

\section{Conclusions and future work}

\paragraph{Testing physical space}
You, gentle reader, may wonder why of all the theorems to be proved
given this set up we pick the one above. In some sense it's hardly
central to quantum mechanics. We see it as central in the sense that
it firmly establishes a notion of physical space arising from a notion
of the equivalence of behavior. Relating bisimulation to a metric is a
big step forward, but one is faced with interpreting the relationship
of that metric space to something more physical. Quantum mechanical
notions of ``physical'' space are still far from intuitive, but by
relating this idea of distance as testing to calculations that predict
physical circumstances we are making a not insignificant step forward
toward an understanding of the physical space we inhabit as
essentially dynamic.

\paragraph{Effectivity and simulation}
One of the observations we have yet to make is that the entire program
spelled out here is effective. We have built various interpreters for
the reflective calculus at work in this interpretation. In principle,
then, we can simulate quantum mechanics on a computer. The place where
the simulation may lose fidelity is the infinitely branching summation
for the annihilator.

In this connection i also want to point out that the evaluation style
calculation of the inner product puts the non-determinism of the
summation right at the heart of measurement. This suggests that
Milner's original reduction-based formulation of the dynamics of his
calculi in terms of sums was not just notationally suggestive of a
notion of measure-and-continue but captured some significant part of
the physics.

\paragraph{Quantum continuations}
In light of this last observation i want to point out that the
predominant account of quantum mechanics is missing a key aspect of a
truly compositional story of the physical situation. In a real lab,
when a measurement is made the observation can be made to feed into
another device that then makes another measurement conditioned on the
results of the first. This means that after the superposition was
collapsed the entire experimental set up remained in
superposition. While QM offers a means of writing this down it doesn't
quite line up well with the well-trodden formulation of computation
and continuation that we see so succinctly expressed in Milner's
calculi. This suggests that there might be advantages to this account
of dynamics waiting to be explored.

\paragraph{Quantum logic}
In this connection, we also note that by virtue of having the
Hennessy-Milner construction, we can pull the construction through the
interpretation of QM. This gives us a natural candidate for a quantum
logic that enjoys an extremely tight connection with it's domain of
interpretation, making the construction much less ad hoc (rather it is
the image of functor!).

\paragraph{Quantum probabiity}
i have questions about the basis of the interpretation of inner
product as probability amplitude. In particular, using which
axiomatization of probability theory does the notion of probability
amplitude earn the right to be so dubbed? In other words, where is the
proof that the operation for calculating a probability amplitude (and
then squaring) satisfies the axioms of what it means to calculate a
probability? Even if such a proof exists (i have yet to find it in the
literature), i wonder if it might not be possible to turn things on
their heads. Can we view the calculation of the probability amplitude
as an axiomatization of probability? If so, then the definition we
give for calculating probability amplitude may provide the basis for
an \emph{effective} theory of probability.

\paragraph{Quantum vs ``biological'' information}
Finally, i want to conclude with a more philosophical observation. At
a recent workshop in which QM was a predominant topic i noticed
something about quantum information. The speaker was giving a riveting
discussion of axiomatic QM and showing how properties of ``no
cloning'' and ``no deleting'' emerged as consequences of the
axiomatization. Theorems of this form are necessary to give us a sense
of confidence that our axioms characterize the physical theory. What
struck me, though, was that if quantum information is neither erasable
nor replicable it is markedly different from \emph{life}. Two of the
things we know about life is that

\begin{itemize}
  \item it ends;
  \item to gain some measure of persistence, to transcend it's
    finitude it is imminently copyable.
\end{itemize}

Both of these qualities are summarized succinctly in the aphorism: all
flesh is grass. For me these two kinds of ``information'' -- call them
quantum and biological -- are end points on a spectrum of strategies
for persistence. At one end, we have those curious entities that enjoy
uniqueness and permanence; at the other, we have those who in the face
of a certain end and an uncertain present make a go of passing
something on. To me one of the more remarkable aspects of the latter
strategy is that in the presence of noise (and certain features of
copying) we get a kind of dynamism, a chance for improvement against a
given persistent condition.

% subsection other_calculi_other_bisimulations_and_geometry_as_behavior (end)




% section conclusion (end)

%\documentclass[12pt]{llncs}
%\documentclass{jktr}

\usepackage[pdftex]{hyperref}                   
\usepackage {listings}
\usepackage {mathpartir}
\usepackage{bcprules}
%\usepackage{listings}
                       
\usepackage{graphicx} 
%\usepackage[margins=2.5cm,nohead,nofoot]{geometry}
%\usepackage{geometry}
\usepackage{amsfonts}
\usepackage{amstext}
\usepackage{latexsym}
\usepackage{amssymb}
\usepackage{color}


%\include{myPreamble}
\include{qm2pi.local} 

%\ifpdf
%\usepackage[pdftex]{graphicx}
%\else
%\usepackage{graphicx}
%\fi

 % \ifpdf
%  \usepackage{pdfsync}
%  \if


%\title{Brief Article}
%\author{David F. Snyder}
%\author{L.G. Meredith}

%\address{Dept. of Math., Texas State University--San Marcos, San Marcos, TX 78666}
       
\pagestyle{empty}


\begin{document}

\lstset{language=[Objective]Caml,frame=shadowbox}

\input{qm2pi.front}

% section front matter (end)

\input{qm2pi.intro} 
 
% section introduction (end)

% \input{qm2pi.knotations} 

% section notation (end)

\input{qm2pi.process.calculi} 

% section concurrent_process_calculi_and_spatial_logics_ (end)
    
%\input{qm2pi.knots2pi} 

%\input{qm2pi.trefoil} 

%\input{qm2pi.mainthm} 

% subsection basic_interpretation (end)

%\input{qm2pi.rho.presentation} 
\subsection{The syntax and semantics of the notation system}\label{sub:the_syntax_and_semantics_of_the_notation_system} % (fold)

We now summarize a technical presentation of the calculus that
embodies our theory of dynamics. The typical presentation of such a
calculus follows the style of giving generators and relations on
them. The grammar, below, describing term constructors, freely
generates the set of processes, $\Proc$. This set is then quotiented
by a relation known as structural congruence and it is over this set
that the notion of dynamics is expressed. This presentation is
essentially that of \cite{MeredithR05} with the addition of
polyadicity and summation. For readability we have relegated some of
the technical subtleties to an appendix.

\subsubsection{Process grammar}\label{subsub:process_grammar}

\begin{mathpar}
  \inferrule* [lab=synchronization] {} {{M} \bc \pzero \;|\; x?F \;|\; x!C }
  \and
  \inferrule* [lab=abstraction] {} {{F} \bc (x)P}
  \and
  \inferrule* [lab=concretion] {} {{C} \bc \langle Q \rangle}
  \and
  \inferrule* [lab=process] {} {{P,Q} \bc M \;| \;P|Q \;|\; @{x}}
  \and
  \inferrule* [lab=name] {} {{x} \bc \quotep{P}}
\end{mathpar} 

Note that $\vec{x}$ (resp. $\vec{P}$) denotes a vector of names
(resp. processes) of length $|\vec{x}|$ (resp. $|\vec{P}|$). We adopt
the following useful abbreviations.

\begin{mathpar}
   x?(\vec{y}).P := x.(\vec{y})P \and  x\clift{\vec{P}} := x.\clift{\vec{P}}
   \and x!(y) := \lift{x}{\dropn{y}}
   \and \Pi_{i=0}^{n-1}P_i := P_0 | \ldots | P_{n-1}
\end{mathpar}

\subsubsection{Structural congruence}

\paragraph{Free and bound names and alpha-equivalence.} At the
core of structural equivalence is alpha-equivalence which identifies
process that are the same up to a change of variable. Formally, we
recognize the distinction between free and bound names. The free names
of a process, $\freenames{P}$, may be calculated recursively as
follows:

\begin{mathpar}
\freenames{\pzero} := \emptyset
  \and \\
  \freenames{x?(y).P} := \{ x \} \cup (\freenames{P} \setminus \{ y \})
  \and 
  \freenames{x!\langle P \rangle} := \{ x \} \cup \{ P \} 
  \and \\
  \freenames{P|Q} := \freenames{P} \cup \freenames{Q}
  \and \\
  \freenames{@{x}} := \{ x \}
\end{mathpar}

$\pi$
$\quotep{\pi}$

$\freenames{-} : \pi \to \mathcal{P}(\quotep{\pi})$

\begin{eqnarray*}
  \freenames{\pzero} & := & \emptyset \\
  \freenames{x?(y).P} & := & \{ x \} \cup (\freenames{P} \setminus \{ y \}) \\
  \freenames{x!\langle P \rangle} & := & \{ x \} \cup \{ P \} \\
  \freenames{P|Q} & := & \freenames{P} \cup \freenames{Q} \\
  \freenames{\dropn{x}} & := & \{ x \}
\end{eqnarray*}

The bound names of a process, $\boundnames{P}$, are those names occurring in $P$
that are not free. For example, in $x?(y).0$, the name $x$ is free, while $y$ is bound.

\begin{mathpar}
  \inferrule* [lab=monoidal-laws] {} { P|Q \equiv Q|P \and P|0 \equiv P \and P|(Q|R) \equiv (P|Q)|R }
\end{mathpar}

\begin{mathpar}
  \inferrule* [lab=alpha-equivalence] {} { (x)P \equiv (y)P\{y/x\} \and y \not\in \freenames{P} }
\end{mathpar}

\begin{definition}
Then two processes, $P,Q$, are alpha-equivalent if $P = Q\{\vec{y}/\vec{x}\}$ for
some $\vec{x} \in \boundnames{Q},\vec{y} \in \boundnames{P}$, where $Q\{\vec{y}/\vec{x}\}$
denotes the capture-avoiding substitution of $\vec{y}$ for $\vec{x}$ in $Q$.
\end{definition}

\begin{definition}
  The {\em structural congruence} \cite{SangiorgiWalker} , $\equiv$,
  between processes is the least congruence containing
  alpha-equivalence, satisfying the abelian monoid laws
  (associativity, commutativity and $\pzero$ as identity) for parallel
  composition $|$ and for summation $+$.
\end{definition}

\subsection{Name equivalence}

We take name equivalence, written $\nameeq$, to be the smallest
equivalence relation generated by the following rules.

\begin{mathpar}
\inferrule*[lab=Quote-drop]
{ }
{ \quotep{@{x}} \nameeq x }

\inferrule*[lab=Struct-equiv]
{ P \scong Q }
{ \quotep{P} \nameeq \quotep{Q} }
\end{mathpar}

The astute reader will have noticed that the mutual recursion of names
and processes imposes a mutual recursion on alpha-equivalence and
structural equivalence via name-equivalence. Fortunately, all of this
works out pleasantly and we may calculate in the natural way, free of
concern. The reader interested in the details is referred to the
appendix \ref{appendix:rho_details}.

\subsection{Substitution}

We use $\Proc$ for the set of processes, $\QProc$ for the set of
names, and $\id{\{}\vec{y} / \vec{x} \id{\}}$ to denote partial maps,
$s : \QProc \rightarrow \QProc$. A map, $s$ lifts, uniquely, to a map
on process terms, $\widehat{s} : \Proc \rightarrow \Proc$ by the
following equations.

\begin{mathpar}
  (0) \psubstp{Q}{P} := 0 \\
  (R \juxtap S) \psubstp{Q}{P}
  :=    
  (R)\psubstp{Q}{P} \juxtap (S) \psubstp{Q}{P} \\
  (x?(y).R) \psubstp{Q}{P}    
  :=    
  (x)\substp{Q}{P} (z)\concat( (R \psubstn{z}{y}) \psubstp{Q}{P} ) \\
  (\lift{x}{R}) \psubstp{Q}{P}  
  :=
  \lift{(x)\substp{Q}{P}}{ R \psubstp{Q}{P} } \\
%   (\dropn{x})  \psubstp{Q}{P}       
%   := 
%   \left\{ 
%     \begin{array}{ccc} 
%       \dropn{\quotep{Q}} & & x \nameeq \quotep{P} \\
%       \dropn{x} & & otherwise \\
%     \end{array}
%   \right. 
  (\dropn{x})  \psubstp{Q}{P}       
  := 
  \left\{ 
    \begin{array}{ccc} 
      Q & & x \nameeq \quotep{P} \\
      \dropn{x} & & otherwise \\
    \end{array}
  \right.
\end{mathpar}
 

where

\begin{eqnarray}
  (x)\id{\{} \lpquote Q \rpquote / \lpquote P \rpquote \id{\}}            = 
  \left\{ 
    \begin{array}{ccc}
      \lpquote Q \rpquote & & x \nameeq \lpquote P \rpquote \\
      x & & otherwise \\
    \end{array}
  \right. \nonumber
\end{eqnarray}

and $z$ is chosen distinct from $\quotep{P}$, $\quotep{Q}$, the free
names in $Q$, and all the names in $R$. Our $\alpha$-equivalence will
be built in the standard way from this substitution.

\begin{remark}\label{rem:no_self_referential_names}
  One consequence of these definitions is that $\forall P. \quotep{P}
  \not\in \freenames{P}$.
\end{remark}

\subsection{ Dynamic quote: an example }

Anticipating something of what's to come, consider applying the
substitution, $\widehat{\id{\{}u / z \id{\}}}$, to the following pair
of processes, $\lift{w}{y!(z)}$ and $w[ \lpquote y!(z) \rpquote ]$.

\begin{eqnarray}
	\lift{w}{y!(z)}\widehat{\id{\{}u / z \id{\}}}
		& = &
		\lift{w}{y!(u)} \nonumber\\
	w[ \lpquote y!(z) \rpquote ] \widehat{ \id{\{}u / z \id{\}} }
		& = &
		w[ \lpquote y!(z) \rpquote ] \nonumber
\end{eqnarray}

Because the body of the process between quotes is impervious to
substitution, we get radically different answers. In fact, by
examining the first process in an input context,
e.g. $x?(z).\lift{w}{y!(z)}$, we see that the process under the lift
operator may be shaped by prefixed inputs binding a name inside it. In
this sense, the lift operator will be seen as a way to dynamically
construct processes before reifying them as names.

Finally equipped with these standard features we can present the
dynamics of the calculus.

\subsubsection{Operational semantics} 

Finally, we introduce the computational dynamics. What marks these
algebras as distinct from other more traditionally studied algebraic
structures, e.g. vector spaces or polynomial rings, is the manner in
which dynamics is captured. In traditional structures, dynamics is typically
expressed through morphisms between such structures, as in linear maps
between vector spaces or morphisms between rings. In algebras
associated with the semantics of computation, the dynamics is
expressed as part of the algebraic structure itself, through a
reduction reduction relation typically denoted by $\red$. Below, we
give a recursive presentation of this relation for the calculus used
in the encoding.

$\red \subseteq \pi \times \pi$
$\red : \pi \to \mathcal{P}(\pi)$

\begin{mathpar}
  \inferrule* [lab=Comm] { \textsf{match}( x_{src}, x_{trgt} ) } { x_{trgt}?(y)P \; | \; x_{src}!\langle {Q} \rangle \red P\{\quotep{Q}/y}\} }
  \and \\
  \inferrule* [lab=Par] {{P} \red {P}'} {{{P} | {Q}} \red {{P}' | {Q}}}
  \and
  \inferrule* [lab=Equiv]{{{P} \scong {P}'} \andalso {{P}' \red {Q}'} \andalso {{Q}' \scong {Q}}}{{P} \red {Q}}
\end{mathpar}

\begin{eqnarray*}
  match_{\equiv} (\quotep{P},\quotep{Q}) & := & P \equiv Q \\
  match_{\dagger}(\quotep{P},\quotep{Q}) & := & \forall R. P|Q \red^{*} R => R \red^{*} 0 \\
  match_{K}(\quotep{P},\quotep{Q}) & := & K \mbox{ for some context } K
\end{eqnarray*}

$u?(x)P | u!\langle Q \rangle \red P\{\quotep{Q}/x\}$

%We write $\wred$ for $\red^*$, and $P\red$ if $\exists Q $ such that $ P \red Q$.
We write $P\red$ if $\exists Q $ such that $ P \red Q$ and $P\not\red$, otherwise.

\section{Replication}

As mentioned before, it is known that replication (and hence
recursion) can be implemented in a higher-order process algebra
\cite{SangiorgiWalker}. As our first example of calculation with the
machinery thus far presented we give the construction explicitly in
the {\rhoc}.

\begin{eqnarray}
	D_{x} & := & \prefix{x}{y}{(\binpar{\outputp{x}{y}}{@{y}})} \nonumber\\
	\bangp_{x}{P} & := & \binpar{{x}!\langle{\binpar{D_{x}}{P}}\rangle}{D_{x}} \nonumber
\end{eqnarray}

\begin{eqnarray}
	\bangp_{x}{P} & & \nonumber\\
	=
	& {x}!\langle{(\prefix{x}{y}{(\outputp{x}{y} | @{y})) | P}}\rangle 
	      | \prefix{x}{y}{(\outputp{x}{y} | @{y})} & \nonumber\\
	\red
	& (\outputp{x}{y} | @{y})\substn{\quotep{(\prefix{x}{y}{(@{y} | \outputp{x}{y})) | P}}}{y} & \nonumber\\
	=
	& \outputp{x}{\quotep{(\prefix{x}{y}{(\outputp{x}{y} | @{y})) | P}}}
	  | {(\prefix{x}{y}{(\outputp{x}{y} | @{y})) | P}} & \nonumber\\
	\red
	& \ldots & \nonumber\\
	\red^*
	& P | P | \ldots & \nonumber
\end{eqnarray}

Of course, this encoding, as an implementation, runs away, unfolding
$\bangp{P}$ eagerly. A lazier and more implementable replication
operator, restricted to input-guarded processes, may be obtained as follows.

\begin{eqnarray}
\bangp{\prefix{u}{v}{P}} 
	:= 
	\binpar{\lift{x}{\prefix{u}{v}{(\binpar{D(x)}{P})}}}{D(x)} \nonumber
\end{eqnarray}

\begin{remark}
  Note that the lazier definition still does not deal with summation
  or mixed summation (i.e. sums over input and output). The reader is
  invited to construct definitions of replication that deal with these
  features. 

  Further, the definitions are parameterized in a name, $x$. Can you,
  gentle reader, make a definition that eliminates this parameter and
  guarantees no accidental interaction between the replication
  machinery and the process being replicated -- i.e. no accidental
  sharing of names used by the process to get its work done and the
  name(s) used by the replication to effect copying. This latter
  revision of the definition of replication is crucial to obtaining
  the expected identity $!!P \sim !P$.
\end{remark}

\begin{remark}\label{rem:paradoxical_combinator}
  The reader familiar with the lambda calculus will have noticed the
  similarity between $D$ and the paradoxical combinator.

  [Ed. note: the existence of this seems to suggest we have to be more
  restrictive on the set of processes and names we admit if we are to
  support no-cloning.]
\end{remark}

\subsubsection{Bisimulation}

The computational dynamics gives rise to another kind of equivalence,
the equivalence of computational behavior. As previously mentioned
this is typically captured \emph{via} some form of bisimulation.

% The notion we use in this paper is weak barbed bisimulation
% \cite{milner91polyadicpi}.

The notion we use in this paper is derived from weak barbed
bisimulation \cite{milner91polyadicpi}. 

\begin{definition}
An \emph{observation relation}, $\downarrow_{\mathcal N}$, over a set
of names, $\mathcal N$, is the smallest relation satisfying the rules
below.

\infrule[Out-barb]{y \in {\mathcal N}, \; x \nameeq y}
		  {\outputp{x}{v} \downarrow_{\mathcal N} x}
\infrule[Par-barb]{\mbox{$P\downarrow_{\mathcal N} x$ or $Q\downarrow_{\mathcal N} x$}}
		  {\binpar{P}{Q} \downarrow_{\mathcal N} x}

We write $P \Downarrow_{\mathcal N} x$ if there is $Q$ such that 
$P \wred Q$ and $Q \downarrow_{\mathcal N} x$.
\end{definition}

\begin{definition}
%\label{def.bbisim}
An  ${\mathcal N}$-\emph{barbed bisimulation} over a set of names, ${\mathcal N}$, is a symmetric binary relation 
${\mathcal S}_{\mathcal N}$ between agents such that $P\rel{S}_{\mathcal N}Q$ implies:
\begin{enumerate}
\item If $P \red P'$ then $Q \wred Q'$ and $P'\rel{S}_{\mathcal N} Q'$.
\item If $P\downarrow_{\mathcal N} x$, then $Q\Downarrow_{\mathcal N} x$.
\end{enumerate}
$P$ is ${\mathcal N}$-barbed bisimilar to $Q$, written
$P \wbbisim_{\mathcal N} Q$, if $P \rel{S}_{\mathcal N} Q$ for some ${\mathcal N}$-barbed bisimulation ${\mathcal S}_{\mathcal N}$.
\end{definition}

$\mathcal{R} \subseteq \pi \times \pi$

$P \mathcal{R} Q => \forall P'. P \red P' \Rightarrow \exists Q'. Q \red Q', P' \mathcal{R} Q'$

$P \vdash x \Rightarrow Q \vdash x$

\begin{mathpar}
  \inferrule*[lab=Out-barb]{x \nameeq y}{{y}!\langle{Q}\rangle \vdash x}
  \and
  \inferrule*[lab=Par-barb]{\mbox{$P\vdash x$ or $Q\vdash x$}}{\binpar{P}{Q} \vdash x}
\end{mathpar}

\subsubsection{Contexts}

One of the principle advantages of computational calculi like the
$\pi$-calculus is a well-defined notion of context,
contextual-equivalence and a correlation between
contextual-equivalence and notions of bisimulation. The notion of
context allows the decomposition of a process into (sub-)process and
its syntactic environment, its context. Thus, a context may be
thought of as a process with a ``hole'' (written $\Box$) in it. The
application of a context $M$ to a process $P$, written $M[P]$, is
tantamount to filling the hole in $M$ with $P$. In this paper we do
not need the full weight of this theory, but do make use of the notion
of context in the proof the main theorem. 

\begin{mathpar}
  \inferrule* [lab=summation] {} {{M_{M},M_{N}} \bc \Box \;|\; x.M_{A} \;|\; M_{M}+M_{N}}
  \and
  \inferrule* [lab=agent] {} {{M_{A}} \bc (\vec{x})M_{P} \;| \; \clift{P_0,\ldots,M_{P},\ldots,P_N}}
  \and \\
  \inferrule* [lab=process] {} {{M_{P}} \bc M_{N} \;| \;P|M_{P} }
\end{mathpar} 

\begin{mathpar}
  \inferrule* [lab=sychronization] {} {M_{N} \bc \Box \;|\; x?M_{F} \;|\; x!M_{C}}
  \and
  \inferrule* [lab=abstraction] {} {{M_{F}} \bc (x)M_{P} }
  \and
  \inferrule* [lab=concretion] {} {{M_{C}} \bc \langle M_{P} \rangle }
  \and \\
  \inferrule* [lab=process] {} {{M_{P}} \bc M_{N} \;| \;P|M_{P} }
\end{mathpar}

\begin{definition}[contextual application] Given a context $M$, and
  process $P$, we define the \emph{contextual application}, $M[P] :=
  M\{P/\Box\}$. That is, the contextual application of M to P is the
  substitution of $P$ for $\Box$ in $M$.
\end{definition}

$\meaningof{-} : L \to \mathcal{P}(\pi)$

\begin{mathpar}
  \inferrule* [lab=collection] {} {\meaningof{true} = \pi, \and \meaningof{~E} = \pi \setminus \meaningof{E}, \and \meaningof{E_{1} \& E_{2}} = \meaningof{E_{1}} \cap \meaningof{E_{2}}}
\end{mathpar}

\begin{mathpar}
  \inferrule* [lab=structure] {} {\meaningof{0} = \{ P \in \pi | P \equiv 0 \}, \and \\ \meaningof{E_1 | E_2} = \{ P \in \pi | P \equiv P_{1} | P_{2}, P_{1} \in \meaningof{E_{1}}, P_{2} \in \meaningof{E_2}\} }
\end{mathpar}

\begin{mathpar}
 \inferrule* [lab=behavior] {} {\meaningof{\langle a?b \rangle E} = \{ P \in \pi | P \equiv Q | u?(y)P', \\ \and \\\\ \and \\ \;\;\; u \in \meaningof{a}, \forall z.P'\{z/y\} \in \meaningof{E\{z/b\}}\}, \and \\ \meaningof{a!E} = \{ P \in \pi | P \equiv Q | x!\langle P' \rangle, x \in \meaningof{a} P' \in \meaningof{E}\} }
\end{mathpar}

\begin{mathpar}
 \inferrule* [lab=nominal] {} {\meaningof{\quotep{E}} = \{ \quotep{P} \in \quotep{\pi} | P \in \meaningof{E} \}, \and \meaningof{\quotep{P}} = \{ \quotep{Q} \in \quotep{\pi} | P \equiv Q \} \and \\ \meaningof{@\quotep{E}} = \{ P \in \pi | P \equiv @x, x \in \meaningof{E} \}}
\end{mathpar}

\begin{eqnarray*}
  \\
  \meaningof{-} : TS \to ST
\end{eqnarray*}

\begin{eqnarray*}
  \\
  L : TS \to ST
\end{eqnarray*}

\begin{eqnarray*}
  \\
  P \models E \iff P \in \meaningof{E}
\end{eqnarray*}

\begin{eqnarray*}
  P \approx_{L} Q \iff \forall E \in L. P \models E \iff Q \models E
\end{eqnarray*}

\begin{eqnarray*}
  P \approx_{K} Q
\end{eqnarray*}

\begin{eqnarray*}
  P \approx Q
\end{eqnarray*}

$\approx_{K} = \approx = \approx_{L}$

\subsubsection{Contextual duality}

Note that contexts extend the quotation operation to a family of
operations from processes to names. Given a context, $M$, we can
define a \emph{nominal context}, $\quotep{M}$ by $\quotep{M}[P] :=
\quotep{M[P]}$. To foreshadow what is to come we observe that these
operations enjoy a duality with processes very much like the duality
between vectors and maps from vectors to scalars.

Further, because the calculus is essentially higher-order, we have a
correspondence between contexts and processes. More specifically,
given a name $x$ and a context $M$ we can construct $M^{*}_{x}$ such
that 

\begin{mathpar}
  M^{*}_{x} | \lift{x}{P} \red M[P]
\end{mathpar}

namely,

\begin{mathpar}
  M^{*}_{x} := x?(u).M[\dropn{u}]
\end{mathpar}

The dependence of $M^{*}_{x}$ on a name makes it an abstraction, 

\begin{mathpar}
  M^{*} := (x)x?(u).M[\dropn{u}]
\end{mathpar}

\subsection{Additional notation}

It will sometimes be convenient to denote the process a name
quotes. We already have the notation $x = \quotep{P}$, but it will be
convenient to introduce an alternate notation, $\procn{x}$, when we
want to emphasize the connection to the use of the name. Note that, by
virtue of name equivalence, $\quotep{\procn{x}} \nameeq x$; so, the
notation is consistent with previous definitions.

Further, because names have structure it is possible to effect
substitutions on the basis of that structure. This means we need to
upgrade our notation for substitutions, which we accomplish by
adapting comprehension notation. Thus,

\begin{mathpar}
  P\{ y / x : x \in S \}
\end{mathpar}

is interpreted to mean the process derived from P by replacing (in a
capture-avoiding manner) each occurrence of $x$ in $S$ by $y$. For example,

\begin{mathpar}
  P\{ \quotep{\procn{x}|\procn{x}} / x : x \in \freenames{P} \}
\end{mathpar}

will replace each (occurrence) of a free name $x$ in $P$ by
$\quotep{\procn{x}|\procn{x}}$.

Also, we will avail ourselves of the notation $x^{L}$ and $x^{R}$ to
denote injections of a name into disjoint copies of the name
space. There are numerous ways to accomplish this. One example can be
found in \cite{MeredithR05}. This notation overloads to vectors of
names: $\vec{x}^{\pi} := (x_{i}^{\pi} \; : \; 0 \leq i < |\vec{x}| )$ where $\pi \in \{L,R\}$.

We also use $P^{\Box} := P|\Box$.

In \cite{MeredithR05} an interpretation of the new operator is
given. It turns out that there are several possible interpretations
all enjoying the requisite algebraic properties of the operator (see
\cite{milner91polyadicpi}). We will therefore make liberal use of
$(\nu\; \vec{x})P$.

% subsection the_syntax_and_semantics_of_the_notation_system (end)   

\input{qm2pi.qmops} 

\input{qm2pi.sterngerlach} 

\input{qm2pi.metric} 

% section concurrent_process_calculi (end)

%\input{qm2pi.proofsketch}

% section proof sketch (end)

%\input{qm2pi.slviaknots} 

% section spatial logic via knots (end)

\input{qm2pi.conclusion}

% section conclusion (end)

%\input{qm2pi.dtcodes} 

% section wiring algorithm (end)

\input{qm2pi.ack} 

% section acknowledgments (end)

\newpage


\bibliographystyle{plain}   
\bibliography{../../biblios/main.bib}

\input{qm2pi.rhodetails}

\end{document}

 

% section wiring algorithm (end)

\documentclass[12pt]{llncs}
%\documentclass{jktr}

\usepackage[pdftex]{hyperref}                   
\usepackage {listings}
\usepackage {mathpartir}
\usepackage{bcprules}
%\usepackage{listings}
                       
\usepackage{graphicx} 
%\usepackage[margins=2.5cm,nohead,nofoot]{geometry}
%\usepackage{geometry}
\usepackage{amsfonts}
\usepackage{amstext}
\usepackage{latexsym}
\usepackage{amssymb}
\usepackage{color}


%\include{myPreamble}
\include{qm2pi.local} 

%\ifpdf
%\usepackage[pdftex]{graphicx}
%\else
%\usepackage{graphicx}
%\fi

 % \ifpdf
%  \usepackage{pdfsync}
%  \if


%\title{Brief Article}
%\author{David F. Snyder}
%\author{L.G. Meredith}

%\address{Dept. of Math., Texas State University--San Marcos, San Marcos, TX 78666}
       
\pagestyle{empty}


\begin{document}

\lstset{language=[Objective]Caml,frame=shadowbox}

\input{qm2pi.front}

% section front matter (end)

\input{qm2pi.intro} 
 
% section introduction (end)

% \input{qm2pi.knotations} 

% section notation (end)

\input{qm2pi.process.calculi} 

% section concurrent_process_calculi_and_spatial_logics_ (end)
    
%\input{qm2pi.knots2pi} 

%\input{qm2pi.trefoil} 

%\input{qm2pi.mainthm} 

% subsection basic_interpretation (end)

%\input{qm2pi.rho.presentation} 
\subsection{The syntax and semantics of the notation system}\label{sub:the_syntax_and_semantics_of_the_notation_system} % (fold)

We now summarize a technical presentation of the calculus that
embodies our theory of dynamics. The typical presentation of such a
calculus follows the style of giving generators and relations on
them. The grammar, below, describing term constructors, freely
generates the set of processes, $\Proc$. This set is then quotiented
by a relation known as structural congruence and it is over this set
that the notion of dynamics is expressed. This presentation is
essentially that of \cite{MeredithR05} with the addition of
polyadicity and summation. For readability we have relegated some of
the technical subtleties to an appendix.

\subsubsection{Process grammar}\label{subsub:process_grammar}

\begin{mathpar}
  \inferrule* [lab=synchronization] {} {{M} \bc \pzero \;|\; x?F \;|\; x!C }
  \and
  \inferrule* [lab=abstraction] {} {{F} \bc (x)P}
  \and
  \inferrule* [lab=concretion] {} {{C} \bc \langle Q \rangle}
  \and
  \inferrule* [lab=process] {} {{P,Q} \bc M \;| \;P|Q \;|\; @{x}}
  \and
  \inferrule* [lab=name] {} {{x} \bc \quotep{P}}
\end{mathpar} 

Note that $\vec{x}$ (resp. $\vec{P}$) denotes a vector of names
(resp. processes) of length $|\vec{x}|$ (resp. $|\vec{P}|$). We adopt
the following useful abbreviations.

\begin{mathpar}
   x?(\vec{y}).P := x.(\vec{y})P \and  x\clift{\vec{P}} := x.\clift{\vec{P}}
   \and x!(y) := \lift{x}{\dropn{y}}
   \and \Pi_{i=0}^{n-1}P_i := P_0 | \ldots | P_{n-1}
\end{mathpar}

\subsubsection{Structural congruence}

\paragraph{Free and bound names and alpha-equivalence.} At the
core of structural equivalence is alpha-equivalence which identifies
process that are the same up to a change of variable. Formally, we
recognize the distinction between free and bound names. The free names
of a process, $\freenames{P}$, may be calculated recursively as
follows:

\begin{mathpar}
\freenames{\pzero} := \emptyset
  \and \\
  \freenames{x?(y).P} := \{ x \} \cup (\freenames{P} \setminus \{ y \})
  \and 
  \freenames{x!\langle P \rangle} := \{ x \} \cup \{ P \} 
  \and \\
  \freenames{P|Q} := \freenames{P} \cup \freenames{Q}
  \and \\
  \freenames{@{x}} := \{ x \}
\end{mathpar}

$\pi$
$\quotep{\pi}$

$\freenames{-} : \pi \to \mathcal{P}(\quotep{\pi})$

\begin{eqnarray*}
  \freenames{\pzero} & := & \emptyset \\
  \freenames{x?(y).P} & := & \{ x \} \cup (\freenames{P} \setminus \{ y \}) \\
  \freenames{x!\langle P \rangle} & := & \{ x \} \cup \{ P \} \\
  \freenames{P|Q} & := & \freenames{P} \cup \freenames{Q} \\
  \freenames{\dropn{x}} & := & \{ x \}
\end{eqnarray*}

The bound names of a process, $\boundnames{P}$, are those names occurring in $P$
that are not free. For example, in $x?(y).0$, the name $x$ is free, while $y$ is bound.

\begin{mathpar}
  \inferrule* [lab=monoidal-laws] {} { P|Q \equiv Q|P \and P|0 \equiv P \and P|(Q|R) \equiv (P|Q)|R }
\end{mathpar}

\begin{mathpar}
  \inferrule* [lab=alpha-equivalence] {} { (x)P \equiv (y)P\{y/x\} \and y \not\in \freenames{P} }
\end{mathpar}

\begin{definition}
Then two processes, $P,Q$, are alpha-equivalent if $P = Q\{\vec{y}/\vec{x}\}$ for
some $\vec{x} \in \boundnames{Q},\vec{y} \in \boundnames{P}$, where $Q\{\vec{y}/\vec{x}\}$
denotes the capture-avoiding substitution of $\vec{y}$ for $\vec{x}$ in $Q$.
\end{definition}

\begin{definition}
  The {\em structural congruence} \cite{SangiorgiWalker} , $\equiv$,
  between processes is the least congruence containing
  alpha-equivalence, satisfying the abelian monoid laws
  (associativity, commutativity and $\pzero$ as identity) for parallel
  composition $|$ and for summation $+$.
\end{definition}

\subsection{Name equivalence}

We take name equivalence, written $\nameeq$, to be the smallest
equivalence relation generated by the following rules.

\begin{mathpar}
\inferrule*[lab=Quote-drop]
{ }
{ \quotep{@{x}} \nameeq x }

\inferrule*[lab=Struct-equiv]
{ P \scong Q }
{ \quotep{P} \nameeq \quotep{Q} }
\end{mathpar}

The astute reader will have noticed that the mutual recursion of names
and processes imposes a mutual recursion on alpha-equivalence and
structural equivalence via name-equivalence. Fortunately, all of this
works out pleasantly and we may calculate in the natural way, free of
concern. The reader interested in the details is referred to the
appendix \ref{appendix:rho_details}.

\subsection{Substitution}

We use $\Proc$ for the set of processes, $\QProc$ for the set of
names, and $\id{\{}\vec{y} / \vec{x} \id{\}}$ to denote partial maps,
$s : \QProc \rightarrow \QProc$. A map, $s$ lifts, uniquely, to a map
on process terms, $\widehat{s} : \Proc \rightarrow \Proc$ by the
following equations.

\begin{mathpar}
  (0) \psubstp{Q}{P} := 0 \\
  (R \juxtap S) \psubstp{Q}{P}
  :=    
  (R)\psubstp{Q}{P} \juxtap (S) \psubstp{Q}{P} \\
  (x?(y).R) \psubstp{Q}{P}    
  :=    
  (x)\substp{Q}{P} (z)\concat( (R \psubstn{z}{y}) \psubstp{Q}{P} ) \\
  (\lift{x}{R}) \psubstp{Q}{P}  
  :=
  \lift{(x)\substp{Q}{P}}{ R \psubstp{Q}{P} } \\
%   (\dropn{x})  \psubstp{Q}{P}       
%   := 
%   \left\{ 
%     \begin{array}{ccc} 
%       \dropn{\quotep{Q}} & & x \nameeq \quotep{P} \\
%       \dropn{x} & & otherwise \\
%     \end{array}
%   \right. 
  (\dropn{x})  \psubstp{Q}{P}       
  := 
  \left\{ 
    \begin{array}{ccc} 
      Q & & x \nameeq \quotep{P} \\
      \dropn{x} & & otherwise \\
    \end{array}
  \right.
\end{mathpar}
 

where

\begin{eqnarray}
  (x)\id{\{} \lpquote Q \rpquote / \lpquote P \rpquote \id{\}}            = 
  \left\{ 
    \begin{array}{ccc}
      \lpquote Q \rpquote & & x \nameeq \lpquote P \rpquote \\
      x & & otherwise \\
    \end{array}
  \right. \nonumber
\end{eqnarray}

and $z$ is chosen distinct from $\quotep{P}$, $\quotep{Q}$, the free
names in $Q$, and all the names in $R$. Our $\alpha$-equivalence will
be built in the standard way from this substitution.

\begin{remark}\label{rem:no_self_referential_names}
  One consequence of these definitions is that $\forall P. \quotep{P}
  \not\in \freenames{P}$.
\end{remark}

\subsection{ Dynamic quote: an example }

Anticipating something of what's to come, consider applying the
substitution, $\widehat{\id{\{}u / z \id{\}}}$, to the following pair
of processes, $\lift{w}{y!(z)}$ and $w[ \lpquote y!(z) \rpquote ]$.

\begin{eqnarray}
	\lift{w}{y!(z)}\widehat{\id{\{}u / z \id{\}}}
		& = &
		\lift{w}{y!(u)} \nonumber\\
	w[ \lpquote y!(z) \rpquote ] \widehat{ \id{\{}u / z \id{\}} }
		& = &
		w[ \lpquote y!(z) \rpquote ] \nonumber
\end{eqnarray}

Because the body of the process between quotes is impervious to
substitution, we get radically different answers. In fact, by
examining the first process in an input context,
e.g. $x?(z).\lift{w}{y!(z)}$, we see that the process under the lift
operator may be shaped by prefixed inputs binding a name inside it. In
this sense, the lift operator will be seen as a way to dynamically
construct processes before reifying them as names.

Finally equipped with these standard features we can present the
dynamics of the calculus.

\subsubsection{Operational semantics} 

Finally, we introduce the computational dynamics. What marks these
algebras as distinct from other more traditionally studied algebraic
structures, e.g. vector spaces or polynomial rings, is the manner in
which dynamics is captured. In traditional structures, dynamics is typically
expressed through morphisms between such structures, as in linear maps
between vector spaces or morphisms between rings. In algebras
associated with the semantics of computation, the dynamics is
expressed as part of the algebraic structure itself, through a
reduction reduction relation typically denoted by $\red$. Below, we
give a recursive presentation of this relation for the calculus used
in the encoding.

$\red \subseteq \pi \times \pi$
$\red : \pi \to \mathcal{P}(\pi)$

\begin{mathpar}
  \inferrule* [lab=Comm] { \textsf{match}( x_{src}, x_{trgt} ) } { x_{trgt}?(y)P \; | \; x_{src}!\langle {Q} \rangle \red P\{\quotep{Q}/y}\} }
  \and \\
  \inferrule* [lab=Par] {{P} \red {P}'} {{{P} | {Q}} \red {{P}' | {Q}}}
  \and
  \inferrule* [lab=Equiv]{{{P} \scong {P}'} \andalso {{P}' \red {Q}'} \andalso {{Q}' \scong {Q}}}{{P} \red {Q}}
\end{mathpar}

\begin{eqnarray*}
  match_{\equiv} (\quotep{P},\quotep{Q}) & := & P \equiv Q \\
  match_{\dagger}(\quotep{P},\quotep{Q}) & := & \forall R. P|Q \red^{*} R => R \red^{*} 0 \\
  match_{K}(\quotep{P},\quotep{Q}) & := & K \mbox{ for some context } K
\end{eqnarray*}

$u?(x)P | u!\langle Q \rangle \red P\{\quotep{Q}/x\}$

%We write $\wred$ for $\red^*$, and $P\red$ if $\exists Q $ such that $ P \red Q$.
We write $P\red$ if $\exists Q $ such that $ P \red Q$ and $P\not\red$, otherwise.

\section{Replication}

As mentioned before, it is known that replication (and hence
recursion) can be implemented in a higher-order process algebra
\cite{SangiorgiWalker}. As our first example of calculation with the
machinery thus far presented we give the construction explicitly in
the {\rhoc}.

\begin{eqnarray}
	D_{x} & := & \prefix{x}{y}{(\binpar{\outputp{x}{y}}{@{y}})} \nonumber\\
	\bangp_{x}{P} & := & \binpar{{x}!\langle{\binpar{D_{x}}{P}}\rangle}{D_{x}} \nonumber
\end{eqnarray}

\begin{eqnarray}
	\bangp_{x}{P} & & \nonumber\\
	=
	& {x}!\langle{(\prefix{x}{y}{(\outputp{x}{y} | @{y})) | P}}\rangle 
	      | \prefix{x}{y}{(\outputp{x}{y} | @{y})} & \nonumber\\
	\red
	& (\outputp{x}{y} | @{y})\substn{\quotep{(\prefix{x}{y}{(@{y} | \outputp{x}{y})) | P}}}{y} & \nonumber\\
	=
	& \outputp{x}{\quotep{(\prefix{x}{y}{(\outputp{x}{y} | @{y})) | P}}}
	  | {(\prefix{x}{y}{(\outputp{x}{y} | @{y})) | P}} & \nonumber\\
	\red
	& \ldots & \nonumber\\
	\red^*
	& P | P | \ldots & \nonumber
\end{eqnarray}

Of course, this encoding, as an implementation, runs away, unfolding
$\bangp{P}$ eagerly. A lazier and more implementable replication
operator, restricted to input-guarded processes, may be obtained as follows.

\begin{eqnarray}
\bangp{\prefix{u}{v}{P}} 
	:= 
	\binpar{\lift{x}{\prefix{u}{v}{(\binpar{D(x)}{P})}}}{D(x)} \nonumber
\end{eqnarray}

\begin{remark}
  Note that the lazier definition still does not deal with summation
  or mixed summation (i.e. sums over input and output). The reader is
  invited to construct definitions of replication that deal with these
  features. 

  Further, the definitions are parameterized in a name, $x$. Can you,
  gentle reader, make a definition that eliminates this parameter and
  guarantees no accidental interaction between the replication
  machinery and the process being replicated -- i.e. no accidental
  sharing of names used by the process to get its work done and the
  name(s) used by the replication to effect copying. This latter
  revision of the definition of replication is crucial to obtaining
  the expected identity $!!P \sim !P$.
\end{remark}

\begin{remark}\label{rem:paradoxical_combinator}
  The reader familiar with the lambda calculus will have noticed the
  similarity between $D$ and the paradoxical combinator.

  [Ed. note: the existence of this seems to suggest we have to be more
  restrictive on the set of processes and names we admit if we are to
  support no-cloning.]
\end{remark}

\subsubsection{Bisimulation}

The computational dynamics gives rise to another kind of equivalence,
the equivalence of computational behavior. As previously mentioned
this is typically captured \emph{via} some form of bisimulation.

% The notion we use in this paper is weak barbed bisimulation
% \cite{milner91polyadicpi}.

The notion we use in this paper is derived from weak barbed
bisimulation \cite{milner91polyadicpi}. 

\begin{definition}
An \emph{observation relation}, $\downarrow_{\mathcal N}$, over a set
of names, $\mathcal N$, is the smallest relation satisfying the rules
below.

\infrule[Out-barb]{y \in {\mathcal N}, \; x \nameeq y}
		  {\outputp{x}{v} \downarrow_{\mathcal N} x}
\infrule[Par-barb]{\mbox{$P\downarrow_{\mathcal N} x$ or $Q\downarrow_{\mathcal N} x$}}
		  {\binpar{P}{Q} \downarrow_{\mathcal N} x}

We write $P \Downarrow_{\mathcal N} x$ if there is $Q$ such that 
$P \wred Q$ and $Q \downarrow_{\mathcal N} x$.
\end{definition}

\begin{definition}
%\label{def.bbisim}
An  ${\mathcal N}$-\emph{barbed bisimulation} over a set of names, ${\mathcal N}$, is a symmetric binary relation 
${\mathcal S}_{\mathcal N}$ between agents such that $P\rel{S}_{\mathcal N}Q$ implies:
\begin{enumerate}
\item If $P \red P'$ then $Q \wred Q'$ and $P'\rel{S}_{\mathcal N} Q'$.
\item If $P\downarrow_{\mathcal N} x$, then $Q\Downarrow_{\mathcal N} x$.
\end{enumerate}
$P$ is ${\mathcal N}$-barbed bisimilar to $Q$, written
$P \wbbisim_{\mathcal N} Q$, if $P \rel{S}_{\mathcal N} Q$ for some ${\mathcal N}$-barbed bisimulation ${\mathcal S}_{\mathcal N}$.
\end{definition}

$\mathcal{R} \subseteq \pi \times \pi$

$P \mathcal{R} Q => \forall P'. P \red P' \Rightarrow \exists Q'. Q \red Q', P' \mathcal{R} Q'$

$P \vdash x \Rightarrow Q \vdash x$

\begin{mathpar}
  \inferrule*[lab=Out-barb]{x \nameeq y}{{y}!\langle{Q}\rangle \vdash x}
  \and
  \inferrule*[lab=Par-barb]{\mbox{$P\vdash x$ or $Q\vdash x$}}{\binpar{P}{Q} \vdash x}
\end{mathpar}

\subsubsection{Contexts}

One of the principle advantages of computational calculi like the
$\pi$-calculus is a well-defined notion of context,
contextual-equivalence and a correlation between
contextual-equivalence and notions of bisimulation. The notion of
context allows the decomposition of a process into (sub-)process and
its syntactic environment, its context. Thus, a context may be
thought of as a process with a ``hole'' (written $\Box$) in it. The
application of a context $M$ to a process $P$, written $M[P]$, is
tantamount to filling the hole in $M$ with $P$. In this paper we do
not need the full weight of this theory, but do make use of the notion
of context in the proof the main theorem. 

\begin{mathpar}
  \inferrule* [lab=summation] {} {{M_{M},M_{N}} \bc \Box \;|\; x.M_{A} \;|\; M_{M}+M_{N}}
  \and
  \inferrule* [lab=agent] {} {{M_{A}} \bc (\vec{x})M_{P} \;| \; \clift{P_0,\ldots,M_{P},\ldots,P_N}}
  \and \\
  \inferrule* [lab=process] {} {{M_{P}} \bc M_{N} \;| \;P|M_{P} }
\end{mathpar} 

\begin{mathpar}
  \inferrule* [lab=sychronization] {} {M_{N} \bc \Box \;|\; x?M_{F} \;|\; x!M_{C}}
  \and
  \inferrule* [lab=abstraction] {} {{M_{F}} \bc (x)M_{P} }
  \and
  \inferrule* [lab=concretion] {} {{M_{C}} \bc \langle M_{P} \rangle }
  \and \\
  \inferrule* [lab=process] {} {{M_{P}} \bc M_{N} \;| \;P|M_{P} }
\end{mathpar}

\begin{definition}[contextual application] Given a context $M$, and
  process $P$, we define the \emph{contextual application}, $M[P] :=
  M\{P/\Box\}$. That is, the contextual application of M to P is the
  substitution of $P$ for $\Box$ in $M$.
\end{definition}

$\meaningof{-} : L \to \mathcal{P}(\pi)$

\begin{mathpar}
  \inferrule* [lab=collection] {} {\meaningof{true} = \pi, \and \meaningof{~E} = \pi \setminus \meaningof{E}, \and \meaningof{E_{1} \& E_{2}} = \meaningof{E_{1}} \cap \meaningof{E_{2}}}
\end{mathpar}

\begin{mathpar}
  \inferrule* [lab=structure] {} {\meaningof{0} = \{ P \in \pi | P \equiv 0 \}, \and \\ \meaningof{E_1 | E_2} = \{ P \in \pi | P \equiv P_{1} | P_{2}, P_{1} \in \meaningof{E_{1}}, P_{2} \in \meaningof{E_2}\} }
\end{mathpar}

\begin{mathpar}
 \inferrule* [lab=behavior] {} {\meaningof{\langle a?b \rangle E} = \{ P \in \pi | P \equiv Q | u?(y)P', \\ \and \\\\ \and \\ \;\;\; u \in \meaningof{a}, \forall z.P'\{z/y\} \in \meaningof{E\{z/b\}}\}, \and \\ \meaningof{a!E} = \{ P \in \pi | P \equiv Q | x!\langle P' \rangle, x \in \meaningof{a} P' \in \meaningof{E}\} }
\end{mathpar}

\begin{mathpar}
 \inferrule* [lab=nominal] {} {\meaningof{\quotep{E}} = \{ \quotep{P} \in \quotep{\pi} | P \in \meaningof{E} \}, \and \meaningof{\quotep{P}} = \{ \quotep{Q} \in \quotep{\pi} | P \equiv Q \} \and \\ \meaningof{@\quotep{E}} = \{ P \in \pi | P \equiv @x, x \in \meaningof{E} \}}
\end{mathpar}

\begin{eqnarray*}
  \\
  \meaningof{-} : TS \to ST
\end{eqnarray*}

\begin{eqnarray*}
  \\
  L : TS \to ST
\end{eqnarray*}

\begin{eqnarray*}
  \\
  P \models E \iff P \in \meaningof{E}
\end{eqnarray*}

\begin{eqnarray*}
  P \approx_{L} Q \iff \forall E \in L. P \models E \iff Q \models E
\end{eqnarray*}

\begin{eqnarray*}
  P \approx_{K} Q
\end{eqnarray*}

\begin{eqnarray*}
  P \approx Q
\end{eqnarray*}

$\approx_{K} = \approx = \approx_{L}$

\subsubsection{Contextual duality}

Note that contexts extend the quotation operation to a family of
operations from processes to names. Given a context, $M$, we can
define a \emph{nominal context}, $\quotep{M}$ by $\quotep{M}[P] :=
\quotep{M[P]}$. To foreshadow what is to come we observe that these
operations enjoy a duality with processes very much like the duality
between vectors and maps from vectors to scalars.

Further, because the calculus is essentially higher-order, we have a
correspondence between contexts and processes. More specifically,
given a name $x$ and a context $M$ we can construct $M^{*}_{x}$ such
that 

\begin{mathpar}
  M^{*}_{x} | \lift{x}{P} \red M[P]
\end{mathpar}

namely,

\begin{mathpar}
  M^{*}_{x} := x?(u).M[\dropn{u}]
\end{mathpar}

The dependence of $M^{*}_{x}$ on a name makes it an abstraction, 

\begin{mathpar}
  M^{*} := (x)x?(u).M[\dropn{u}]
\end{mathpar}

\subsection{Additional notation}

It will sometimes be convenient to denote the process a name
quotes. We already have the notation $x = \quotep{P}$, but it will be
convenient to introduce an alternate notation, $\procn{x}$, when we
want to emphasize the connection to the use of the name. Note that, by
virtue of name equivalence, $\quotep{\procn{x}} \nameeq x$; so, the
notation is consistent with previous definitions.

Further, because names have structure it is possible to effect
substitutions on the basis of that structure. This means we need to
upgrade our notation for substitutions, which we accomplish by
adapting comprehension notation. Thus,

\begin{mathpar}
  P\{ y / x : x \in S \}
\end{mathpar}

is interpreted to mean the process derived from P by replacing (in a
capture-avoiding manner) each occurrence of $x$ in $S$ by $y$. For example,

\begin{mathpar}
  P\{ \quotep{\procn{x}|\procn{x}} / x : x \in \freenames{P} \}
\end{mathpar}

will replace each (occurrence) of a free name $x$ in $P$ by
$\quotep{\procn{x}|\procn{x}}$.

Also, we will avail ourselves of the notation $x^{L}$ and $x^{R}$ to
denote injections of a name into disjoint copies of the name
space. There are numerous ways to accomplish this. One example can be
found in \cite{MeredithR05}. This notation overloads to vectors of
names: $\vec{x}^{\pi} := (x_{i}^{\pi} \; : \; 0 \leq i < |\vec{x}| )$ where $\pi \in \{L,R\}$.

We also use $P^{\Box} := P|\Box$.

In \cite{MeredithR05} an interpretation of the new operator is
given. It turns out that there are several possible interpretations
all enjoying the requisite algebraic properties of the operator (see
\cite{milner91polyadicpi}). We will therefore make liberal use of
$(\nu\; \vec{x})P$.

% subsection the_syntax_and_semantics_of_the_notation_system (end)   

\input{qm2pi.qmops} 

\input{qm2pi.sterngerlach} 

\input{qm2pi.metric} 

% section concurrent_process_calculi (end)

%\input{qm2pi.proofsketch}

% section proof sketch (end)

%\input{qm2pi.slviaknots} 

% section spatial logic via knots (end)

\input{qm2pi.conclusion}

% section conclusion (end)

%\input{qm2pi.dtcodes} 

% section wiring algorithm (end)

\input{qm2pi.ack} 

% section acknowledgments (end)

\newpage


\bibliographystyle{plain}   
\bibliography{../../biblios/main.bib}

\input{qm2pi.rhodetails}

\end{document}

 

% section acknowledgments (end)

\newpage


\bibliographystyle{plain}   
\bibliography{../../biblios/main.bib}

\documentclass[12pt]{llncs}
%\documentclass{jktr}

\usepackage[pdftex]{hyperref}                   
\usepackage {listings}
\usepackage {mathpartir}
\usepackage{bcprules}
%\usepackage{listings}
                       
\usepackage{graphicx} 
%\usepackage[margins=2.5cm,nohead,nofoot]{geometry}
%\usepackage{geometry}
\usepackage{amsfonts}
\usepackage{amstext}
\usepackage{latexsym}
\usepackage{amssymb}
\usepackage{color}


%\include{myPreamble}
\include{qm2pi.local} 

%\ifpdf
%\usepackage[pdftex]{graphicx}
%\else
%\usepackage{graphicx}
%\fi

 % \ifpdf
%  \usepackage{pdfsync}
%  \if


%\title{Brief Article}
%\author{David F. Snyder}
%\author{L.G. Meredith}

%\address{Dept. of Math., Texas State University--San Marcos, San Marcos, TX 78666}
       
\pagestyle{empty}


\begin{document}

\lstset{language=[Objective]Caml,frame=shadowbox}

\input{qm2pi.front}

% section front matter (end)

\input{qm2pi.intro} 
 
% section introduction (end)

% \input{qm2pi.knotations} 

% section notation (end)

\input{qm2pi.process.calculi} 

% section concurrent_process_calculi_and_spatial_logics_ (end)
    
%\input{qm2pi.knots2pi} 

%\input{qm2pi.trefoil} 

%\input{qm2pi.mainthm} 

% subsection basic_interpretation (end)

%\input{qm2pi.rho.presentation} 
\subsection{The syntax and semantics of the notation system}\label{sub:the_syntax_and_semantics_of_the_notation_system} % (fold)

We now summarize a technical presentation of the calculus that
embodies our theory of dynamics. The typical presentation of such a
calculus follows the style of giving generators and relations on
them. The grammar, below, describing term constructors, freely
generates the set of processes, $\Proc$. This set is then quotiented
by a relation known as structural congruence and it is over this set
that the notion of dynamics is expressed. This presentation is
essentially that of \cite{MeredithR05} with the addition of
polyadicity and summation. For readability we have relegated some of
the technical subtleties to an appendix.

\subsubsection{Process grammar}\label{subsub:process_grammar}

\begin{mathpar}
  \inferrule* [lab=synchronization] {} {{M} \bc \pzero \;|\; x?F \;|\; x!C }
  \and
  \inferrule* [lab=abstraction] {} {{F} \bc (x)P}
  \and
  \inferrule* [lab=concretion] {} {{C} \bc \langle Q \rangle}
  \and
  \inferrule* [lab=process] {} {{P,Q} \bc M \;| \;P|Q \;|\; @{x}}
  \and
  \inferrule* [lab=name] {} {{x} \bc \quotep{P}}
\end{mathpar} 

Note that $\vec{x}$ (resp. $\vec{P}$) denotes a vector of names
(resp. processes) of length $|\vec{x}|$ (resp. $|\vec{P}|$). We adopt
the following useful abbreviations.

\begin{mathpar}
   x?(\vec{y}).P := x.(\vec{y})P \and  x\clift{\vec{P}} := x.\clift{\vec{P}}
   \and x!(y) := \lift{x}{\dropn{y}}
   \and \Pi_{i=0}^{n-1}P_i := P_0 | \ldots | P_{n-1}
\end{mathpar}

\subsubsection{Structural congruence}

\paragraph{Free and bound names and alpha-equivalence.} At the
core of structural equivalence is alpha-equivalence which identifies
process that are the same up to a change of variable. Formally, we
recognize the distinction between free and bound names. The free names
of a process, $\freenames{P}$, may be calculated recursively as
follows:

\begin{mathpar}
\freenames{\pzero} := \emptyset
  \and \\
  \freenames{x?(y).P} := \{ x \} \cup (\freenames{P} \setminus \{ y \})
  \and 
  \freenames{x!\langle P \rangle} := \{ x \} \cup \{ P \} 
  \and \\
  \freenames{P|Q} := \freenames{P} \cup \freenames{Q}
  \and \\
  \freenames{@{x}} := \{ x \}
\end{mathpar}

$\pi$
$\quotep{\pi}$

$\freenames{-} : \pi \to \mathcal{P}(\quotep{\pi})$

\begin{eqnarray*}
  \freenames{\pzero} & := & \emptyset \\
  \freenames{x?(y).P} & := & \{ x \} \cup (\freenames{P} \setminus \{ y \}) \\
  \freenames{x!\langle P \rangle} & := & \{ x \} \cup \{ P \} \\
  \freenames{P|Q} & := & \freenames{P} \cup \freenames{Q} \\
  \freenames{\dropn{x}} & := & \{ x \}
\end{eqnarray*}

The bound names of a process, $\boundnames{P}$, are those names occurring in $P$
that are not free. For example, in $x?(y).0$, the name $x$ is free, while $y$ is bound.

\begin{mathpar}
  \inferrule* [lab=monoidal-laws] {} { P|Q \equiv Q|P \and P|0 \equiv P \and P|(Q|R) \equiv (P|Q)|R }
\end{mathpar}

\begin{mathpar}
  \inferrule* [lab=alpha-equivalence] {} { (x)P \equiv (y)P\{y/x\} \and y \not\in \freenames{P} }
\end{mathpar}

\begin{definition}
Then two processes, $P,Q$, are alpha-equivalent if $P = Q\{\vec{y}/\vec{x}\}$ for
some $\vec{x} \in \boundnames{Q},\vec{y} \in \boundnames{P}$, where $Q\{\vec{y}/\vec{x}\}$
denotes the capture-avoiding substitution of $\vec{y}$ for $\vec{x}$ in $Q$.
\end{definition}

\begin{definition}
  The {\em structural congruence} \cite{SangiorgiWalker} , $\equiv$,
  between processes is the least congruence containing
  alpha-equivalence, satisfying the abelian monoid laws
  (associativity, commutativity and $\pzero$ as identity) for parallel
  composition $|$ and for summation $+$.
\end{definition}

\subsection{Name equivalence}

We take name equivalence, written $\nameeq$, to be the smallest
equivalence relation generated by the following rules.

\begin{mathpar}
\inferrule*[lab=Quote-drop]
{ }
{ \quotep{@{x}} \nameeq x }

\inferrule*[lab=Struct-equiv]
{ P \scong Q }
{ \quotep{P} \nameeq \quotep{Q} }
\end{mathpar}

The astute reader will have noticed that the mutual recursion of names
and processes imposes a mutual recursion on alpha-equivalence and
structural equivalence via name-equivalence. Fortunately, all of this
works out pleasantly and we may calculate in the natural way, free of
concern. The reader interested in the details is referred to the
appendix \ref{appendix:rho_details}.

\subsection{Substitution}

We use $\Proc$ for the set of processes, $\QProc$ for the set of
names, and $\id{\{}\vec{y} / \vec{x} \id{\}}$ to denote partial maps,
$s : \QProc \rightarrow \QProc$. A map, $s$ lifts, uniquely, to a map
on process terms, $\widehat{s} : \Proc \rightarrow \Proc$ by the
following equations.

\begin{mathpar}
  (0) \psubstp{Q}{P} := 0 \\
  (R \juxtap S) \psubstp{Q}{P}
  :=    
  (R)\psubstp{Q}{P} \juxtap (S) \psubstp{Q}{P} \\
  (x?(y).R) \psubstp{Q}{P}    
  :=    
  (x)\substp{Q}{P} (z)\concat( (R \psubstn{z}{y}) \psubstp{Q}{P} ) \\
  (\lift{x}{R}) \psubstp{Q}{P}  
  :=
  \lift{(x)\substp{Q}{P}}{ R \psubstp{Q}{P} } \\
%   (\dropn{x})  \psubstp{Q}{P}       
%   := 
%   \left\{ 
%     \begin{array}{ccc} 
%       \dropn{\quotep{Q}} & & x \nameeq \quotep{P} \\
%       \dropn{x} & & otherwise \\
%     \end{array}
%   \right. 
  (\dropn{x})  \psubstp{Q}{P}       
  := 
  \left\{ 
    \begin{array}{ccc} 
      Q & & x \nameeq \quotep{P} \\
      \dropn{x} & & otherwise \\
    \end{array}
  \right.
\end{mathpar}
 

where

\begin{eqnarray}
  (x)\id{\{} \lpquote Q \rpquote / \lpquote P \rpquote \id{\}}            = 
  \left\{ 
    \begin{array}{ccc}
      \lpquote Q \rpquote & & x \nameeq \lpquote P \rpquote \\
      x & & otherwise \\
    \end{array}
  \right. \nonumber
\end{eqnarray}

and $z$ is chosen distinct from $\quotep{P}$, $\quotep{Q}$, the free
names in $Q$, and all the names in $R$. Our $\alpha$-equivalence will
be built in the standard way from this substitution.

\begin{remark}\label{rem:no_self_referential_names}
  One consequence of these definitions is that $\forall P. \quotep{P}
  \not\in \freenames{P}$.
\end{remark}

\subsection{ Dynamic quote: an example }

Anticipating something of what's to come, consider applying the
substitution, $\widehat{\id{\{}u / z \id{\}}}$, to the following pair
of processes, $\lift{w}{y!(z)}$ and $w[ \lpquote y!(z) \rpquote ]$.

\begin{eqnarray}
	\lift{w}{y!(z)}\widehat{\id{\{}u / z \id{\}}}
		& = &
		\lift{w}{y!(u)} \nonumber\\
	w[ \lpquote y!(z) \rpquote ] \widehat{ \id{\{}u / z \id{\}} }
		& = &
		w[ \lpquote y!(z) \rpquote ] \nonumber
\end{eqnarray}

Because the body of the process between quotes is impervious to
substitution, we get radically different answers. In fact, by
examining the first process in an input context,
e.g. $x?(z).\lift{w}{y!(z)}$, we see that the process under the lift
operator may be shaped by prefixed inputs binding a name inside it. In
this sense, the lift operator will be seen as a way to dynamically
construct processes before reifying them as names.

Finally equipped with these standard features we can present the
dynamics of the calculus.

\subsubsection{Operational semantics} 

Finally, we introduce the computational dynamics. What marks these
algebras as distinct from other more traditionally studied algebraic
structures, e.g. vector spaces or polynomial rings, is the manner in
which dynamics is captured. In traditional structures, dynamics is typically
expressed through morphisms between such structures, as in linear maps
between vector spaces or morphisms between rings. In algebras
associated with the semantics of computation, the dynamics is
expressed as part of the algebraic structure itself, through a
reduction reduction relation typically denoted by $\red$. Below, we
give a recursive presentation of this relation for the calculus used
in the encoding.

$\red \subseteq \pi \times \pi$
$\red : \pi \to \mathcal{P}(\pi)$

\begin{mathpar}
  \inferrule* [lab=Comm] { \textsf{match}( x_{src}, x_{trgt} ) } { x_{trgt}?(y)P \; | \; x_{src}!\langle {Q} \rangle \red P\{\quotep{Q}/y}\} }
  \and \\
  \inferrule* [lab=Par] {{P} \red {P}'} {{{P} | {Q}} \red {{P}' | {Q}}}
  \and
  \inferrule* [lab=Equiv]{{{P} \scong {P}'} \andalso {{P}' \red {Q}'} \andalso {{Q}' \scong {Q}}}{{P} \red {Q}}
\end{mathpar}

\begin{eqnarray*}
  match_{\equiv} (\quotep{P},\quotep{Q}) & := & P \equiv Q \\
  match_{\dagger}(\quotep{P},\quotep{Q}) & := & \forall R. P|Q \red^{*} R => R \red^{*} 0 \\
  match_{K}(\quotep{P},\quotep{Q}) & := & K \mbox{ for some context } K
\end{eqnarray*}

$u?(x)P | u!\langle Q \rangle \red P\{\quotep{Q}/x\}$

%We write $\wred$ for $\red^*$, and $P\red$ if $\exists Q $ such that $ P \red Q$.
We write $P\red$ if $\exists Q $ such that $ P \red Q$ and $P\not\red$, otherwise.

\section{Replication}

As mentioned before, it is known that replication (and hence
recursion) can be implemented in a higher-order process algebra
\cite{SangiorgiWalker}. As our first example of calculation with the
machinery thus far presented we give the construction explicitly in
the {\rhoc}.

\begin{eqnarray}
	D_{x} & := & \prefix{x}{y}{(\binpar{\outputp{x}{y}}{@{y}})} \nonumber\\
	\bangp_{x}{P} & := & \binpar{{x}!\langle{\binpar{D_{x}}{P}}\rangle}{D_{x}} \nonumber
\end{eqnarray}

\begin{eqnarray}
	\bangp_{x}{P} & & \nonumber\\
	=
	& {x}!\langle{(\prefix{x}{y}{(\outputp{x}{y} | @{y})) | P}}\rangle 
	      | \prefix{x}{y}{(\outputp{x}{y} | @{y})} & \nonumber\\
	\red
	& (\outputp{x}{y} | @{y})\substn{\quotep{(\prefix{x}{y}{(@{y} | \outputp{x}{y})) | P}}}{y} & \nonumber\\
	=
	& \outputp{x}{\quotep{(\prefix{x}{y}{(\outputp{x}{y} | @{y})) | P}}}
	  | {(\prefix{x}{y}{(\outputp{x}{y} | @{y})) | P}} & \nonumber\\
	\red
	& \ldots & \nonumber\\
	\red^*
	& P | P | \ldots & \nonumber
\end{eqnarray}

Of course, this encoding, as an implementation, runs away, unfolding
$\bangp{P}$ eagerly. A lazier and more implementable replication
operator, restricted to input-guarded processes, may be obtained as follows.

\begin{eqnarray}
\bangp{\prefix{u}{v}{P}} 
	:= 
	\binpar{\lift{x}{\prefix{u}{v}{(\binpar{D(x)}{P})}}}{D(x)} \nonumber
\end{eqnarray}

\begin{remark}
  Note that the lazier definition still does not deal with summation
  or mixed summation (i.e. sums over input and output). The reader is
  invited to construct definitions of replication that deal with these
  features. 

  Further, the definitions are parameterized in a name, $x$. Can you,
  gentle reader, make a definition that eliminates this parameter and
  guarantees no accidental interaction between the replication
  machinery and the process being replicated -- i.e. no accidental
  sharing of names used by the process to get its work done and the
  name(s) used by the replication to effect copying. This latter
  revision of the definition of replication is crucial to obtaining
  the expected identity $!!P \sim !P$.
\end{remark}

\begin{remark}\label{rem:paradoxical_combinator}
  The reader familiar with the lambda calculus will have noticed the
  similarity between $D$ and the paradoxical combinator.

  [Ed. note: the existence of this seems to suggest we have to be more
  restrictive on the set of processes and names we admit if we are to
  support no-cloning.]
\end{remark}

\subsubsection{Bisimulation}

The computational dynamics gives rise to another kind of equivalence,
the equivalence of computational behavior. As previously mentioned
this is typically captured \emph{via} some form of bisimulation.

% The notion we use in this paper is weak barbed bisimulation
% \cite{milner91polyadicpi}.

The notion we use in this paper is derived from weak barbed
bisimulation \cite{milner91polyadicpi}. 

\begin{definition}
An \emph{observation relation}, $\downarrow_{\mathcal N}$, over a set
of names, $\mathcal N$, is the smallest relation satisfying the rules
below.

\infrule[Out-barb]{y \in {\mathcal N}, \; x \nameeq y}
		  {\outputp{x}{v} \downarrow_{\mathcal N} x}
\infrule[Par-barb]{\mbox{$P\downarrow_{\mathcal N} x$ or $Q\downarrow_{\mathcal N} x$}}
		  {\binpar{P}{Q} \downarrow_{\mathcal N} x}

We write $P \Downarrow_{\mathcal N} x$ if there is $Q$ such that 
$P \wred Q$ and $Q \downarrow_{\mathcal N} x$.
\end{definition}

\begin{definition}
%\label{def.bbisim}
An  ${\mathcal N}$-\emph{barbed bisimulation} over a set of names, ${\mathcal N}$, is a symmetric binary relation 
${\mathcal S}_{\mathcal N}$ between agents such that $P\rel{S}_{\mathcal N}Q$ implies:
\begin{enumerate}
\item If $P \red P'$ then $Q \wred Q'$ and $P'\rel{S}_{\mathcal N} Q'$.
\item If $P\downarrow_{\mathcal N} x$, then $Q\Downarrow_{\mathcal N} x$.
\end{enumerate}
$P$ is ${\mathcal N}$-barbed bisimilar to $Q$, written
$P \wbbisim_{\mathcal N} Q$, if $P \rel{S}_{\mathcal N} Q$ for some ${\mathcal N}$-barbed bisimulation ${\mathcal S}_{\mathcal N}$.
\end{definition}

$\mathcal{R} \subseteq \pi \times \pi$

$P \mathcal{R} Q => \forall P'. P \red P' \Rightarrow \exists Q'. Q \red Q', P' \mathcal{R} Q'$

$P \vdash x \Rightarrow Q \vdash x$

\begin{mathpar}
  \inferrule*[lab=Out-barb]{x \nameeq y}{{y}!\langle{Q}\rangle \vdash x}
  \and
  \inferrule*[lab=Par-barb]{\mbox{$P\vdash x$ or $Q\vdash x$}}{\binpar{P}{Q} \vdash x}
\end{mathpar}

\subsubsection{Contexts}

One of the principle advantages of computational calculi like the
$\pi$-calculus is a well-defined notion of context,
contextual-equivalence and a correlation between
contextual-equivalence and notions of bisimulation. The notion of
context allows the decomposition of a process into (sub-)process and
its syntactic environment, its context. Thus, a context may be
thought of as a process with a ``hole'' (written $\Box$) in it. The
application of a context $M$ to a process $P$, written $M[P]$, is
tantamount to filling the hole in $M$ with $P$. In this paper we do
not need the full weight of this theory, but do make use of the notion
of context in the proof the main theorem. 

\begin{mathpar}
  \inferrule* [lab=summation] {} {{M_{M},M_{N}} \bc \Box \;|\; x.M_{A} \;|\; M_{M}+M_{N}}
  \and
  \inferrule* [lab=agent] {} {{M_{A}} \bc (\vec{x})M_{P} \;| \; \clift{P_0,\ldots,M_{P},\ldots,P_N}}
  \and \\
  \inferrule* [lab=process] {} {{M_{P}} \bc M_{N} \;| \;P|M_{P} }
\end{mathpar} 

\begin{mathpar}
  \inferrule* [lab=sychronization] {} {M_{N} \bc \Box \;|\; x?M_{F} \;|\; x!M_{C}}
  \and
  \inferrule* [lab=abstraction] {} {{M_{F}} \bc (x)M_{P} }
  \and
  \inferrule* [lab=concretion] {} {{M_{C}} \bc \langle M_{P} \rangle }
  \and \\
  \inferrule* [lab=process] {} {{M_{P}} \bc M_{N} \;| \;P|M_{P} }
\end{mathpar}

\begin{definition}[contextual application] Given a context $M$, and
  process $P$, we define the \emph{contextual application}, $M[P] :=
  M\{P/\Box\}$. That is, the contextual application of M to P is the
  substitution of $P$ for $\Box$ in $M$.
\end{definition}

$\meaningof{-} : L \to \mathcal{P}(\pi)$

\begin{mathpar}
  \inferrule* [lab=collection] {} {\meaningof{true} = \pi, \and \meaningof{~E} = \pi \setminus \meaningof{E}, \and \meaningof{E_{1} \& E_{2}} = \meaningof{E_{1}} \cap \meaningof{E_{2}}}
\end{mathpar}

\begin{mathpar}
  \inferrule* [lab=structure] {} {\meaningof{0} = \{ P \in \pi | P \equiv 0 \}, \and \\ \meaningof{E_1 | E_2} = \{ P \in \pi | P \equiv P_{1} | P_{2}, P_{1} \in \meaningof{E_{1}}, P_{2} \in \meaningof{E_2}\} }
\end{mathpar}

\begin{mathpar}
 \inferrule* [lab=behavior] {} {\meaningof{\langle a?b \rangle E} = \{ P \in \pi | P \equiv Q | u?(y)P', \\ \and \\\\ \and \\ \;\;\; u \in \meaningof{a}, \forall z.P'\{z/y\} \in \meaningof{E\{z/b\}}\}, \and \\ \meaningof{a!E} = \{ P \in \pi | P \equiv Q | x!\langle P' \rangle, x \in \meaningof{a} P' \in \meaningof{E}\} }
\end{mathpar}

\begin{mathpar}
 \inferrule* [lab=nominal] {} {\meaningof{\quotep{E}} = \{ \quotep{P} \in \quotep{\pi} | P \in \meaningof{E} \}, \and \meaningof{\quotep{P}} = \{ \quotep{Q} \in \quotep{\pi} | P \equiv Q \} \and \\ \meaningof{@\quotep{E}} = \{ P \in \pi | P \equiv @x, x \in \meaningof{E} \}}
\end{mathpar}

\begin{eqnarray*}
  \\
  \meaningof{-} : TS \to ST
\end{eqnarray*}

\begin{eqnarray*}
  \\
  L : TS \to ST
\end{eqnarray*}

\begin{eqnarray*}
  \\
  P \models E \iff P \in \meaningof{E}
\end{eqnarray*}

\begin{eqnarray*}
  P \approx_{L} Q \iff \forall E \in L. P \models E \iff Q \models E
\end{eqnarray*}

\begin{eqnarray*}
  P \approx_{K} Q
\end{eqnarray*}

\begin{eqnarray*}
  P \approx Q
\end{eqnarray*}

$\approx_{K} = \approx = \approx_{L}$

\subsubsection{Contextual duality}

Note that contexts extend the quotation operation to a family of
operations from processes to names. Given a context, $M$, we can
define a \emph{nominal context}, $\quotep{M}$ by $\quotep{M}[P] :=
\quotep{M[P]}$. To foreshadow what is to come we observe that these
operations enjoy a duality with processes very much like the duality
between vectors and maps from vectors to scalars.

Further, because the calculus is essentially higher-order, we have a
correspondence between contexts and processes. More specifically,
given a name $x$ and a context $M$ we can construct $M^{*}_{x}$ such
that 

\begin{mathpar}
  M^{*}_{x} | \lift{x}{P} \red M[P]
\end{mathpar}

namely,

\begin{mathpar}
  M^{*}_{x} := x?(u).M[\dropn{u}]
\end{mathpar}

The dependence of $M^{*}_{x}$ on a name makes it an abstraction, 

\begin{mathpar}
  M^{*} := (x)x?(u).M[\dropn{u}]
\end{mathpar}

\subsection{Additional notation}

It will sometimes be convenient to denote the process a name
quotes. We already have the notation $x = \quotep{P}$, but it will be
convenient to introduce an alternate notation, $\procn{x}$, when we
want to emphasize the connection to the use of the name. Note that, by
virtue of name equivalence, $\quotep{\procn{x}} \nameeq x$; so, the
notation is consistent with previous definitions.

Further, because names have structure it is possible to effect
substitutions on the basis of that structure. This means we need to
upgrade our notation for substitutions, which we accomplish by
adapting comprehension notation. Thus,

\begin{mathpar}
  P\{ y / x : x \in S \}
\end{mathpar}

is interpreted to mean the process derived from P by replacing (in a
capture-avoiding manner) each occurrence of $x$ in $S$ by $y$. For example,

\begin{mathpar}
  P\{ \quotep{\procn{x}|\procn{x}} / x : x \in \freenames{P} \}
\end{mathpar}

will replace each (occurrence) of a free name $x$ in $P$ by
$\quotep{\procn{x}|\procn{x}}$.

Also, we will avail ourselves of the notation $x^{L}$ and $x^{R}$ to
denote injections of a name into disjoint copies of the name
space. There are numerous ways to accomplish this. One example can be
found in \cite{MeredithR05}. This notation overloads to vectors of
names: $\vec{x}^{\pi} := (x_{i}^{\pi} \; : \; 0 \leq i < |\vec{x}| )$ where $\pi \in \{L,R\}$.

We also use $P^{\Box} := P|\Box$.

In \cite{MeredithR05} an interpretation of the new operator is
given. It turns out that there are several possible interpretations
all enjoying the requisite algebraic properties of the operator (see
\cite{milner91polyadicpi}). We will therefore make liberal use of
$(\nu\; \vec{x})P$.

% subsection the_syntax_and_semantics_of_the_notation_system (end)   

\input{qm2pi.qmops} 

\input{qm2pi.sterngerlach} 

\input{qm2pi.metric} 

% section concurrent_process_calculi (end)

%\input{qm2pi.proofsketch}

% section proof sketch (end)

%\input{qm2pi.slviaknots} 

% section spatial logic via knots (end)

\input{qm2pi.conclusion}

% section conclusion (end)

%\input{qm2pi.dtcodes} 

% section wiring algorithm (end)

\input{qm2pi.ack} 

% section acknowledgments (end)

\newpage


\bibliographystyle{plain}   
\bibliography{../../biblios/main.bib}

\input{qm2pi.rhodetails}

\end{document}



\end{document}

 

%\documentclass[12pt]{llncs}
%\documentclass{jktr}

\usepackage[pdftex]{hyperref}                   
\usepackage {listings}
\usepackage {mathpartir}
\usepackage{bcprules}
%\usepackage{listings}
                       
\usepackage{graphicx} 
%\usepackage[margins=2.5cm,nohead,nofoot]{geometry}
%\usepackage{geometry}
\usepackage{amsfonts}
\usepackage{amstext}
\usepackage{latexsym}
\usepackage{amssymb}
\usepackage{color}


%\include{myPreamble}
\documentclass[12pt]{llncs}
%\documentclass{jktr}

\usepackage[pdftex]{hyperref}                   
\usepackage {listings}
\usepackage {mathpartir}
\usepackage{bcprules}
%\usepackage{listings}
                       
\usepackage{graphicx} 
%\usepackage[margins=2.5cm,nohead,nofoot]{geometry}
%\usepackage{geometry}
\usepackage{amsfonts}
\usepackage{amstext}
\usepackage{latexsym}
\usepackage{amssymb}
\usepackage{color}


%\include{myPreamble}
\include{qm2pi.local} 

%\ifpdf
%\usepackage[pdftex]{graphicx}
%\else
%\usepackage{graphicx}
%\fi

 % \ifpdf
%  \usepackage{pdfsync}
%  \if


%\title{Brief Article}
%\author{David F. Snyder}
%\author{L.G. Meredith}

%\address{Dept. of Math., Texas State University--San Marcos, San Marcos, TX 78666}
       
\pagestyle{empty}


\begin{document}

\lstset{language=[Objective]Caml,frame=shadowbox}

\input{qm2pi.front}

% section front matter (end)

\input{qm2pi.intro} 
 
% section introduction (end)

% \input{qm2pi.knotations} 

% section notation (end)

\input{qm2pi.process.calculi} 

% section concurrent_process_calculi_and_spatial_logics_ (end)
    
%\input{qm2pi.knots2pi} 

%\input{qm2pi.trefoil} 

%\input{qm2pi.mainthm} 

% subsection basic_interpretation (end)

%\input{qm2pi.rho.presentation} 
\subsection{The syntax and semantics of the notation system}\label{sub:the_syntax_and_semantics_of_the_notation_system} % (fold)

We now summarize a technical presentation of the calculus that
embodies our theory of dynamics. The typical presentation of such a
calculus follows the style of giving generators and relations on
them. The grammar, below, describing term constructors, freely
generates the set of processes, $\Proc$. This set is then quotiented
by a relation known as structural congruence and it is over this set
that the notion of dynamics is expressed. This presentation is
essentially that of \cite{MeredithR05} with the addition of
polyadicity and summation. For readability we have relegated some of
the technical subtleties to an appendix.

\subsubsection{Process grammar}\label{subsub:process_grammar}

\begin{mathpar}
  \inferrule* [lab=synchronization] {} {{M} \bc \pzero \;|\; x?F \;|\; x!C }
  \and
  \inferrule* [lab=abstraction] {} {{F} \bc (x)P}
  \and
  \inferrule* [lab=concretion] {} {{C} \bc \langle Q \rangle}
  \and
  \inferrule* [lab=process] {} {{P,Q} \bc M \;| \;P|Q \;|\; @{x}}
  \and
  \inferrule* [lab=name] {} {{x} \bc \quotep{P}}
\end{mathpar} 

Note that $\vec{x}$ (resp. $\vec{P}$) denotes a vector of names
(resp. processes) of length $|\vec{x}|$ (resp. $|\vec{P}|$). We adopt
the following useful abbreviations.

\begin{mathpar}
   x?(\vec{y}).P := x.(\vec{y})P \and  x\clift{\vec{P}} := x.\clift{\vec{P}}
   \and x!(y) := \lift{x}{\dropn{y}}
   \and \Pi_{i=0}^{n-1}P_i := P_0 | \ldots | P_{n-1}
\end{mathpar}

\subsubsection{Structural congruence}

\paragraph{Free and bound names and alpha-equivalence.} At the
core of structural equivalence is alpha-equivalence which identifies
process that are the same up to a change of variable. Formally, we
recognize the distinction between free and bound names. The free names
of a process, $\freenames{P}$, may be calculated recursively as
follows:

\begin{mathpar}
\freenames{\pzero} := \emptyset
  \and \\
  \freenames{x?(y).P} := \{ x \} \cup (\freenames{P} \setminus \{ y \})
  \and 
  \freenames{x!\langle P \rangle} := \{ x \} \cup \{ P \} 
  \and \\
  \freenames{P|Q} := \freenames{P} \cup \freenames{Q}
  \and \\
  \freenames{@{x}} := \{ x \}
\end{mathpar}

$\pi$
$\quotep{\pi}$

$\freenames{-} : \pi \to \mathcal{P}(\quotep{\pi})$

\begin{eqnarray*}
  \freenames{\pzero} & := & \emptyset \\
  \freenames{x?(y).P} & := & \{ x \} \cup (\freenames{P} \setminus \{ y \}) \\
  \freenames{x!\langle P \rangle} & := & \{ x \} \cup \{ P \} \\
  \freenames{P|Q} & := & \freenames{P} \cup \freenames{Q} \\
  \freenames{\dropn{x}} & := & \{ x \}
\end{eqnarray*}

The bound names of a process, $\boundnames{P}$, are those names occurring in $P$
that are not free. For example, in $x?(y).0$, the name $x$ is free, while $y$ is bound.

\begin{mathpar}
  \inferrule* [lab=monoidal-laws] {} { P|Q \equiv Q|P \and P|0 \equiv P \and P|(Q|R) \equiv (P|Q)|R }
\end{mathpar}

\begin{mathpar}
  \inferrule* [lab=alpha-equivalence] {} { (x)P \equiv (y)P\{y/x\} \and y \not\in \freenames{P} }
\end{mathpar}

\begin{definition}
Then two processes, $P,Q$, are alpha-equivalent if $P = Q\{\vec{y}/\vec{x}\}$ for
some $\vec{x} \in \boundnames{Q},\vec{y} \in \boundnames{P}$, where $Q\{\vec{y}/\vec{x}\}$
denotes the capture-avoiding substitution of $\vec{y}$ for $\vec{x}$ in $Q$.
\end{definition}

\begin{definition}
  The {\em structural congruence} \cite{SangiorgiWalker} , $\equiv$,
  between processes is the least congruence containing
  alpha-equivalence, satisfying the abelian monoid laws
  (associativity, commutativity and $\pzero$ as identity) for parallel
  composition $|$ and for summation $+$.
\end{definition}

\subsection{Name equivalence}

We take name equivalence, written $\nameeq$, to be the smallest
equivalence relation generated by the following rules.

\begin{mathpar}
\inferrule*[lab=Quote-drop]
{ }
{ \quotep{@{x}} \nameeq x }

\inferrule*[lab=Struct-equiv]
{ P \scong Q }
{ \quotep{P} \nameeq \quotep{Q} }
\end{mathpar}

The astute reader will have noticed that the mutual recursion of names
and processes imposes a mutual recursion on alpha-equivalence and
structural equivalence via name-equivalence. Fortunately, all of this
works out pleasantly and we may calculate in the natural way, free of
concern. The reader interested in the details is referred to the
appendix \ref{appendix:rho_details}.

\subsection{Substitution}

We use $\Proc$ for the set of processes, $\QProc$ for the set of
names, and $\id{\{}\vec{y} / \vec{x} \id{\}}$ to denote partial maps,
$s : \QProc \rightarrow \QProc$. A map, $s$ lifts, uniquely, to a map
on process terms, $\widehat{s} : \Proc \rightarrow \Proc$ by the
following equations.

\begin{mathpar}
  (0) \psubstp{Q}{P} := 0 \\
  (R \juxtap S) \psubstp{Q}{P}
  :=    
  (R)\psubstp{Q}{P} \juxtap (S) \psubstp{Q}{P} \\
  (x?(y).R) \psubstp{Q}{P}    
  :=    
  (x)\substp{Q}{P} (z)\concat( (R \psubstn{z}{y}) \psubstp{Q}{P} ) \\
  (\lift{x}{R}) \psubstp{Q}{P}  
  :=
  \lift{(x)\substp{Q}{P}}{ R \psubstp{Q}{P} } \\
%   (\dropn{x})  \psubstp{Q}{P}       
%   := 
%   \left\{ 
%     \begin{array}{ccc} 
%       \dropn{\quotep{Q}} & & x \nameeq \quotep{P} \\
%       \dropn{x} & & otherwise \\
%     \end{array}
%   \right. 
  (\dropn{x})  \psubstp{Q}{P}       
  := 
  \left\{ 
    \begin{array}{ccc} 
      Q & & x \nameeq \quotep{P} \\
      \dropn{x} & & otherwise \\
    \end{array}
  \right.
\end{mathpar}
 

where

\begin{eqnarray}
  (x)\id{\{} \lpquote Q \rpquote / \lpquote P \rpquote \id{\}}            = 
  \left\{ 
    \begin{array}{ccc}
      \lpquote Q \rpquote & & x \nameeq \lpquote P \rpquote \\
      x & & otherwise \\
    \end{array}
  \right. \nonumber
\end{eqnarray}

and $z$ is chosen distinct from $\quotep{P}$, $\quotep{Q}$, the free
names in $Q$, and all the names in $R$. Our $\alpha$-equivalence will
be built in the standard way from this substitution.

\begin{remark}\label{rem:no_self_referential_names}
  One consequence of these definitions is that $\forall P. \quotep{P}
  \not\in \freenames{P}$.
\end{remark}

\subsection{ Dynamic quote: an example }

Anticipating something of what's to come, consider applying the
substitution, $\widehat{\id{\{}u / z \id{\}}}$, to the following pair
of processes, $\lift{w}{y!(z)}$ and $w[ \lpquote y!(z) \rpquote ]$.

\begin{eqnarray}
	\lift{w}{y!(z)}\widehat{\id{\{}u / z \id{\}}}
		& = &
		\lift{w}{y!(u)} \nonumber\\
	w[ \lpquote y!(z) \rpquote ] \widehat{ \id{\{}u / z \id{\}} }
		& = &
		w[ \lpquote y!(z) \rpquote ] \nonumber
\end{eqnarray}

Because the body of the process between quotes is impervious to
substitution, we get radically different answers. In fact, by
examining the first process in an input context,
e.g. $x?(z).\lift{w}{y!(z)}$, we see that the process under the lift
operator may be shaped by prefixed inputs binding a name inside it. In
this sense, the lift operator will be seen as a way to dynamically
construct processes before reifying them as names.

Finally equipped with these standard features we can present the
dynamics of the calculus.

\subsubsection{Operational semantics} 

Finally, we introduce the computational dynamics. What marks these
algebras as distinct from other more traditionally studied algebraic
structures, e.g. vector spaces or polynomial rings, is the manner in
which dynamics is captured. In traditional structures, dynamics is typically
expressed through morphisms between such structures, as in linear maps
between vector spaces or morphisms between rings. In algebras
associated with the semantics of computation, the dynamics is
expressed as part of the algebraic structure itself, through a
reduction reduction relation typically denoted by $\red$. Below, we
give a recursive presentation of this relation for the calculus used
in the encoding.

$\red \subseteq \pi \times \pi$
$\red : \pi \to \mathcal{P}(\pi)$

\begin{mathpar}
  \inferrule* [lab=Comm] { \textsf{match}( x_{src}, x_{trgt} ) } { x_{trgt}?(y)P \; | \; x_{src}!\langle {Q} \rangle \red P\{\quotep{Q}/y}\} }
  \and \\
  \inferrule* [lab=Par] {{P} \red {P}'} {{{P} | {Q}} \red {{P}' | {Q}}}
  \and
  \inferrule* [lab=Equiv]{{{P} \scong {P}'} \andalso {{P}' \red {Q}'} \andalso {{Q}' \scong {Q}}}{{P} \red {Q}}
\end{mathpar}

\begin{eqnarray*}
  match_{\equiv} (\quotep{P},\quotep{Q}) & := & P \equiv Q \\
  match_{\dagger}(\quotep{P},\quotep{Q}) & := & \forall R. P|Q \red^{*} R => R \red^{*} 0 \\
  match_{K}(\quotep{P},\quotep{Q}) & := & K \mbox{ for some context } K
\end{eqnarray*}

$u?(x)P | u!\langle Q \rangle \red P\{\quotep{Q}/x\}$

%We write $\wred$ for $\red^*$, and $P\red$ if $\exists Q $ such that $ P \red Q$.
We write $P\red$ if $\exists Q $ such that $ P \red Q$ and $P\not\red$, otherwise.

\section{Replication}

As mentioned before, it is known that replication (and hence
recursion) can be implemented in a higher-order process algebra
\cite{SangiorgiWalker}. As our first example of calculation with the
machinery thus far presented we give the construction explicitly in
the {\rhoc}.

\begin{eqnarray}
	D_{x} & := & \prefix{x}{y}{(\binpar{\outputp{x}{y}}{@{y}})} \nonumber\\
	\bangp_{x}{P} & := & \binpar{{x}!\langle{\binpar{D_{x}}{P}}\rangle}{D_{x}} \nonumber
\end{eqnarray}

\begin{eqnarray}
	\bangp_{x}{P} & & \nonumber\\
	=
	& {x}!\langle{(\prefix{x}{y}{(\outputp{x}{y} | @{y})) | P}}\rangle 
	      | \prefix{x}{y}{(\outputp{x}{y} | @{y})} & \nonumber\\
	\red
	& (\outputp{x}{y} | @{y})\substn{\quotep{(\prefix{x}{y}{(@{y} | \outputp{x}{y})) | P}}}{y} & \nonumber\\
	=
	& \outputp{x}{\quotep{(\prefix{x}{y}{(\outputp{x}{y} | @{y})) | P}}}
	  | {(\prefix{x}{y}{(\outputp{x}{y} | @{y})) | P}} & \nonumber\\
	\red
	& \ldots & \nonumber\\
	\red^*
	& P | P | \ldots & \nonumber
\end{eqnarray}

Of course, this encoding, as an implementation, runs away, unfolding
$\bangp{P}$ eagerly. A lazier and more implementable replication
operator, restricted to input-guarded processes, may be obtained as follows.

\begin{eqnarray}
\bangp{\prefix{u}{v}{P}} 
	:= 
	\binpar{\lift{x}{\prefix{u}{v}{(\binpar{D(x)}{P})}}}{D(x)} \nonumber
\end{eqnarray}

\begin{remark}
  Note that the lazier definition still does not deal with summation
  or mixed summation (i.e. sums over input and output). The reader is
  invited to construct definitions of replication that deal with these
  features. 

  Further, the definitions are parameterized in a name, $x$. Can you,
  gentle reader, make a definition that eliminates this parameter and
  guarantees no accidental interaction between the replication
  machinery and the process being replicated -- i.e. no accidental
  sharing of names used by the process to get its work done and the
  name(s) used by the replication to effect copying. This latter
  revision of the definition of replication is crucial to obtaining
  the expected identity $!!P \sim !P$.
\end{remark}

\begin{remark}\label{rem:paradoxical_combinator}
  The reader familiar with the lambda calculus will have noticed the
  similarity between $D$ and the paradoxical combinator.

  [Ed. note: the existence of this seems to suggest we have to be more
  restrictive on the set of processes and names we admit if we are to
  support no-cloning.]
\end{remark}

\subsubsection{Bisimulation}

The computational dynamics gives rise to another kind of equivalence,
the equivalence of computational behavior. As previously mentioned
this is typically captured \emph{via} some form of bisimulation.

% The notion we use in this paper is weak barbed bisimulation
% \cite{milner91polyadicpi}.

The notion we use in this paper is derived from weak barbed
bisimulation \cite{milner91polyadicpi}. 

\begin{definition}
An \emph{observation relation}, $\downarrow_{\mathcal N}$, over a set
of names, $\mathcal N$, is the smallest relation satisfying the rules
below.

\infrule[Out-barb]{y \in {\mathcal N}, \; x \nameeq y}
		  {\outputp{x}{v} \downarrow_{\mathcal N} x}
\infrule[Par-barb]{\mbox{$P\downarrow_{\mathcal N} x$ or $Q\downarrow_{\mathcal N} x$}}
		  {\binpar{P}{Q} \downarrow_{\mathcal N} x}

We write $P \Downarrow_{\mathcal N} x$ if there is $Q$ such that 
$P \wred Q$ and $Q \downarrow_{\mathcal N} x$.
\end{definition}

\begin{definition}
%\label{def.bbisim}
An  ${\mathcal N}$-\emph{barbed bisimulation} over a set of names, ${\mathcal N}$, is a symmetric binary relation 
${\mathcal S}_{\mathcal N}$ between agents such that $P\rel{S}_{\mathcal N}Q$ implies:
\begin{enumerate}
\item If $P \red P'$ then $Q \wred Q'$ and $P'\rel{S}_{\mathcal N} Q'$.
\item If $P\downarrow_{\mathcal N} x$, then $Q\Downarrow_{\mathcal N} x$.
\end{enumerate}
$P$ is ${\mathcal N}$-barbed bisimilar to $Q$, written
$P \wbbisim_{\mathcal N} Q$, if $P \rel{S}_{\mathcal N} Q$ for some ${\mathcal N}$-barbed bisimulation ${\mathcal S}_{\mathcal N}$.
\end{definition}

$\mathcal{R} \subseteq \pi \times \pi$

$P \mathcal{R} Q => \forall P'. P \red P' \Rightarrow \exists Q'. Q \red Q', P' \mathcal{R} Q'$

$P \vdash x \Rightarrow Q \vdash x$

\begin{mathpar}
  \inferrule*[lab=Out-barb]{x \nameeq y}{{y}!\langle{Q}\rangle \vdash x}
  \and
  \inferrule*[lab=Par-barb]{\mbox{$P\vdash x$ or $Q\vdash x$}}{\binpar{P}{Q} \vdash x}
\end{mathpar}

\subsubsection{Contexts}

One of the principle advantages of computational calculi like the
$\pi$-calculus is a well-defined notion of context,
contextual-equivalence and a correlation between
contextual-equivalence and notions of bisimulation. The notion of
context allows the decomposition of a process into (sub-)process and
its syntactic environment, its context. Thus, a context may be
thought of as a process with a ``hole'' (written $\Box$) in it. The
application of a context $M$ to a process $P$, written $M[P]$, is
tantamount to filling the hole in $M$ with $P$. In this paper we do
not need the full weight of this theory, but do make use of the notion
of context in the proof the main theorem. 

\begin{mathpar}
  \inferrule* [lab=summation] {} {{M_{M},M_{N}} \bc \Box \;|\; x.M_{A} \;|\; M_{M}+M_{N}}
  \and
  \inferrule* [lab=agent] {} {{M_{A}} \bc (\vec{x})M_{P} \;| \; \clift{P_0,\ldots,M_{P},\ldots,P_N}}
  \and \\
  \inferrule* [lab=process] {} {{M_{P}} \bc M_{N} \;| \;P|M_{P} }
\end{mathpar} 

\begin{mathpar}
  \inferrule* [lab=sychronization] {} {M_{N} \bc \Box \;|\; x?M_{F} \;|\; x!M_{C}}
  \and
  \inferrule* [lab=abstraction] {} {{M_{F}} \bc (x)M_{P} }
  \and
  \inferrule* [lab=concretion] {} {{M_{C}} \bc \langle M_{P} \rangle }
  \and \\
  \inferrule* [lab=process] {} {{M_{P}} \bc M_{N} \;| \;P|M_{P} }
\end{mathpar}

\begin{definition}[contextual application] Given a context $M$, and
  process $P$, we define the \emph{contextual application}, $M[P] :=
  M\{P/\Box\}$. That is, the contextual application of M to P is the
  substitution of $P$ for $\Box$ in $M$.
\end{definition}

$\meaningof{-} : L \to \mathcal{P}(\pi)$

\begin{mathpar}
  \inferrule* [lab=collection] {} {\meaningof{true} = \pi, \and \meaningof{~E} = \pi \setminus \meaningof{E}, \and \meaningof{E_{1} \& E_{2}} = \meaningof{E_{1}} \cap \meaningof{E_{2}}}
\end{mathpar}

\begin{mathpar}
  \inferrule* [lab=structure] {} {\meaningof{0} = \{ P \in \pi | P \equiv 0 \}, \and \\ \meaningof{E_1 | E_2} = \{ P \in \pi | P \equiv P_{1} | P_{2}, P_{1} \in \meaningof{E_{1}}, P_{2} \in \meaningof{E_2}\} }
\end{mathpar}

\begin{mathpar}
 \inferrule* [lab=behavior] {} {\meaningof{\langle a?b \rangle E} = \{ P \in \pi | P \equiv Q | u?(y)P', \\ \and \\\\ \and \\ \;\;\; u \in \meaningof{a}, \forall z.P'\{z/y\} \in \meaningof{E\{z/b\}}\}, \and \\ \meaningof{a!E} = \{ P \in \pi | P \equiv Q | x!\langle P' \rangle, x \in \meaningof{a} P' \in \meaningof{E}\} }
\end{mathpar}

\begin{mathpar}
 \inferrule* [lab=nominal] {} {\meaningof{\quotep{E}} = \{ \quotep{P} \in \quotep{\pi} | P \in \meaningof{E} \}, \and \meaningof{\quotep{P}} = \{ \quotep{Q} \in \quotep{\pi} | P \equiv Q \} \and \\ \meaningof{@\quotep{E}} = \{ P \in \pi | P \equiv @x, x \in \meaningof{E} \}}
\end{mathpar}

\begin{eqnarray*}
  \\
  \meaningof{-} : TS \to ST
\end{eqnarray*}

\begin{eqnarray*}
  \\
  L : TS \to ST
\end{eqnarray*}

\begin{eqnarray*}
  \\
  P \models E \iff P \in \meaningof{E}
\end{eqnarray*}

\begin{eqnarray*}
  P \approx_{L} Q \iff \forall E \in L. P \models E \iff Q \models E
\end{eqnarray*}

\begin{eqnarray*}
  P \approx_{K} Q
\end{eqnarray*}

\begin{eqnarray*}
  P \approx Q
\end{eqnarray*}

$\approx_{K} = \approx = \approx_{L}$

\subsubsection{Contextual duality}

Note that contexts extend the quotation operation to a family of
operations from processes to names. Given a context, $M$, we can
define a \emph{nominal context}, $\quotep{M}$ by $\quotep{M}[P] :=
\quotep{M[P]}$. To foreshadow what is to come we observe that these
operations enjoy a duality with processes very much like the duality
between vectors and maps from vectors to scalars.

Further, because the calculus is essentially higher-order, we have a
correspondence between contexts and processes. More specifically,
given a name $x$ and a context $M$ we can construct $M^{*}_{x}$ such
that 

\begin{mathpar}
  M^{*}_{x} | \lift{x}{P} \red M[P]
\end{mathpar}

namely,

\begin{mathpar}
  M^{*}_{x} := x?(u).M[\dropn{u}]
\end{mathpar}

The dependence of $M^{*}_{x}$ on a name makes it an abstraction, 

\begin{mathpar}
  M^{*} := (x)x?(u).M[\dropn{u}]
\end{mathpar}

\subsection{Additional notation}

It will sometimes be convenient to denote the process a name
quotes. We already have the notation $x = \quotep{P}$, but it will be
convenient to introduce an alternate notation, $\procn{x}$, when we
want to emphasize the connection to the use of the name. Note that, by
virtue of name equivalence, $\quotep{\procn{x}} \nameeq x$; so, the
notation is consistent with previous definitions.

Further, because names have structure it is possible to effect
substitutions on the basis of that structure. This means we need to
upgrade our notation for substitutions, which we accomplish by
adapting comprehension notation. Thus,

\begin{mathpar}
  P\{ y / x : x \in S \}
\end{mathpar}

is interpreted to mean the process derived from P by replacing (in a
capture-avoiding manner) each occurrence of $x$ in $S$ by $y$. For example,

\begin{mathpar}
  P\{ \quotep{\procn{x}|\procn{x}} / x : x \in \freenames{P} \}
\end{mathpar}

will replace each (occurrence) of a free name $x$ in $P$ by
$\quotep{\procn{x}|\procn{x}}$.

Also, we will avail ourselves of the notation $x^{L}$ and $x^{R}$ to
denote injections of a name into disjoint copies of the name
space. There are numerous ways to accomplish this. One example can be
found in \cite{MeredithR05}. This notation overloads to vectors of
names: $\vec{x}^{\pi} := (x_{i}^{\pi} \; : \; 0 \leq i < |\vec{x}| )$ where $\pi \in \{L,R\}$.

We also use $P^{\Box} := P|\Box$.

In \cite{MeredithR05} an interpretation of the new operator is
given. It turns out that there are several possible interpretations
all enjoying the requisite algebraic properties of the operator (see
\cite{milner91polyadicpi}). We will therefore make liberal use of
$(\nu\; \vec{x})P$.

% subsection the_syntax_and_semantics_of_the_notation_system (end)   

\input{qm2pi.qmops} 

\input{qm2pi.sterngerlach} 

\input{qm2pi.metric} 

% section concurrent_process_calculi (end)

%\input{qm2pi.proofsketch}

% section proof sketch (end)

%\input{qm2pi.slviaknots} 

% section spatial logic via knots (end)

\input{qm2pi.conclusion}

% section conclusion (end)

%\input{qm2pi.dtcodes} 

% section wiring algorithm (end)

\input{qm2pi.ack} 

% section acknowledgments (end)

\newpage


\bibliographystyle{plain}   
\bibliography{../../biblios/main.bib}

\input{qm2pi.rhodetails}

\end{document}

 

%\ifpdf
%\usepackage[pdftex]{graphicx}
%\else
%\usepackage{graphicx}
%\fi

 % \ifpdf
%  \usepackage{pdfsync}
%  \if


%\title{Brief Article}
%\author{David F. Snyder}
%\author{L.G. Meredith}

%\address{Dept. of Math., Texas State University--San Marcos, San Marcos, TX 78666}
       
\pagestyle{empty}


\begin{document}

\lstset{language=[Objective]Caml,frame=shadowbox}

\documentclass[12pt]{llncs}
%\documentclass{jktr}

\usepackage[pdftex]{hyperref}                   
\usepackage {listings}
\usepackage {mathpartir}
\usepackage{bcprules}
%\usepackage{listings}
                       
\usepackage{graphicx} 
%\usepackage[margins=2.5cm,nohead,nofoot]{geometry}
%\usepackage{geometry}
\usepackage{amsfonts}
\usepackage{amstext}
\usepackage{latexsym}
\usepackage{amssymb}
\usepackage{color}


%\include{myPreamble}
\include{qm2pi.local} 

%\ifpdf
%\usepackage[pdftex]{graphicx}
%\else
%\usepackage{graphicx}
%\fi

 % \ifpdf
%  \usepackage{pdfsync}
%  \if


%\title{Brief Article}
%\author{David F. Snyder}
%\author{L.G. Meredith}

%\address{Dept. of Math., Texas State University--San Marcos, San Marcos, TX 78666}
       
\pagestyle{empty}


\begin{document}

\lstset{language=[Objective]Caml,frame=shadowbox}

\input{qm2pi.front}

% section front matter (end)

\input{qm2pi.intro} 
 
% section introduction (end)

% \input{qm2pi.knotations} 

% section notation (end)

\input{qm2pi.process.calculi} 

% section concurrent_process_calculi_and_spatial_logics_ (end)
    
%\input{qm2pi.knots2pi} 

%\input{qm2pi.trefoil} 

%\input{qm2pi.mainthm} 

% subsection basic_interpretation (end)

%\input{qm2pi.rho.presentation} 
\subsection{The syntax and semantics of the notation system}\label{sub:the_syntax_and_semantics_of_the_notation_system} % (fold)

We now summarize a technical presentation of the calculus that
embodies our theory of dynamics. The typical presentation of such a
calculus follows the style of giving generators and relations on
them. The grammar, below, describing term constructors, freely
generates the set of processes, $\Proc$. This set is then quotiented
by a relation known as structural congruence and it is over this set
that the notion of dynamics is expressed. This presentation is
essentially that of \cite{MeredithR05} with the addition of
polyadicity and summation. For readability we have relegated some of
the technical subtleties to an appendix.

\subsubsection{Process grammar}\label{subsub:process_grammar}

\begin{mathpar}
  \inferrule* [lab=synchronization] {} {{M} \bc \pzero \;|\; x?F \;|\; x!C }
  \and
  \inferrule* [lab=abstraction] {} {{F} \bc (x)P}
  \and
  \inferrule* [lab=concretion] {} {{C} \bc \langle Q \rangle}
  \and
  \inferrule* [lab=process] {} {{P,Q} \bc M \;| \;P|Q \;|\; @{x}}
  \and
  \inferrule* [lab=name] {} {{x} \bc \quotep{P}}
\end{mathpar} 

Note that $\vec{x}$ (resp. $\vec{P}$) denotes a vector of names
(resp. processes) of length $|\vec{x}|$ (resp. $|\vec{P}|$). We adopt
the following useful abbreviations.

\begin{mathpar}
   x?(\vec{y}).P := x.(\vec{y})P \and  x\clift{\vec{P}} := x.\clift{\vec{P}}
   \and x!(y) := \lift{x}{\dropn{y}}
   \and \Pi_{i=0}^{n-1}P_i := P_0 | \ldots | P_{n-1}
\end{mathpar}

\subsubsection{Structural congruence}

\paragraph{Free and bound names and alpha-equivalence.} At the
core of structural equivalence is alpha-equivalence which identifies
process that are the same up to a change of variable. Formally, we
recognize the distinction between free and bound names. The free names
of a process, $\freenames{P}$, may be calculated recursively as
follows:

\begin{mathpar}
\freenames{\pzero} := \emptyset
  \and \\
  \freenames{x?(y).P} := \{ x \} \cup (\freenames{P} \setminus \{ y \})
  \and 
  \freenames{x!\langle P \rangle} := \{ x \} \cup \{ P \} 
  \and \\
  \freenames{P|Q} := \freenames{P} \cup \freenames{Q}
  \and \\
  \freenames{@{x}} := \{ x \}
\end{mathpar}

$\pi$
$\quotep{\pi}$

$\freenames{-} : \pi \to \mathcal{P}(\quotep{\pi})$

\begin{eqnarray*}
  \freenames{\pzero} & := & \emptyset \\
  \freenames{x?(y).P} & := & \{ x \} \cup (\freenames{P} \setminus \{ y \}) \\
  \freenames{x!\langle P \rangle} & := & \{ x \} \cup \{ P \} \\
  \freenames{P|Q} & := & \freenames{P} \cup \freenames{Q} \\
  \freenames{\dropn{x}} & := & \{ x \}
\end{eqnarray*}

The bound names of a process, $\boundnames{P}$, are those names occurring in $P$
that are not free. For example, in $x?(y).0$, the name $x$ is free, while $y$ is bound.

\begin{mathpar}
  \inferrule* [lab=monoidal-laws] {} { P|Q \equiv Q|P \and P|0 \equiv P \and P|(Q|R) \equiv (P|Q)|R }
\end{mathpar}

\begin{mathpar}
  \inferrule* [lab=alpha-equivalence] {} { (x)P \equiv (y)P\{y/x\} \and y \not\in \freenames{P} }
\end{mathpar}

\begin{definition}
Then two processes, $P,Q$, are alpha-equivalent if $P = Q\{\vec{y}/\vec{x}\}$ for
some $\vec{x} \in \boundnames{Q},\vec{y} \in \boundnames{P}$, where $Q\{\vec{y}/\vec{x}\}$
denotes the capture-avoiding substitution of $\vec{y}$ for $\vec{x}$ in $Q$.
\end{definition}

\begin{definition}
  The {\em structural congruence} \cite{SangiorgiWalker} , $\equiv$,
  between processes is the least congruence containing
  alpha-equivalence, satisfying the abelian monoid laws
  (associativity, commutativity and $\pzero$ as identity) for parallel
  composition $|$ and for summation $+$.
\end{definition}

\subsection{Name equivalence}

We take name equivalence, written $\nameeq$, to be the smallest
equivalence relation generated by the following rules.

\begin{mathpar}
\inferrule*[lab=Quote-drop]
{ }
{ \quotep{@{x}} \nameeq x }

\inferrule*[lab=Struct-equiv]
{ P \scong Q }
{ \quotep{P} \nameeq \quotep{Q} }
\end{mathpar}

The astute reader will have noticed that the mutual recursion of names
and processes imposes a mutual recursion on alpha-equivalence and
structural equivalence via name-equivalence. Fortunately, all of this
works out pleasantly and we may calculate in the natural way, free of
concern. The reader interested in the details is referred to the
appendix \ref{appendix:rho_details}.

\subsection{Substitution}

We use $\Proc$ for the set of processes, $\QProc$ for the set of
names, and $\id{\{}\vec{y} / \vec{x} \id{\}}$ to denote partial maps,
$s : \QProc \rightarrow \QProc$. A map, $s$ lifts, uniquely, to a map
on process terms, $\widehat{s} : \Proc \rightarrow \Proc$ by the
following equations.

\begin{mathpar}
  (0) \psubstp{Q}{P} := 0 \\
  (R \juxtap S) \psubstp{Q}{P}
  :=    
  (R)\psubstp{Q}{P} \juxtap (S) \psubstp{Q}{P} \\
  (x?(y).R) \psubstp{Q}{P}    
  :=    
  (x)\substp{Q}{P} (z)\concat( (R \psubstn{z}{y}) \psubstp{Q}{P} ) \\
  (\lift{x}{R}) \psubstp{Q}{P}  
  :=
  \lift{(x)\substp{Q}{P}}{ R \psubstp{Q}{P} } \\
%   (\dropn{x})  \psubstp{Q}{P}       
%   := 
%   \left\{ 
%     \begin{array}{ccc} 
%       \dropn{\quotep{Q}} & & x \nameeq \quotep{P} \\
%       \dropn{x} & & otherwise \\
%     \end{array}
%   \right. 
  (\dropn{x})  \psubstp{Q}{P}       
  := 
  \left\{ 
    \begin{array}{ccc} 
      Q & & x \nameeq \quotep{P} \\
      \dropn{x} & & otherwise \\
    \end{array}
  \right.
\end{mathpar}
 

where

\begin{eqnarray}
  (x)\id{\{} \lpquote Q \rpquote / \lpquote P \rpquote \id{\}}            = 
  \left\{ 
    \begin{array}{ccc}
      \lpquote Q \rpquote & & x \nameeq \lpquote P \rpquote \\
      x & & otherwise \\
    \end{array}
  \right. \nonumber
\end{eqnarray}

and $z$ is chosen distinct from $\quotep{P}$, $\quotep{Q}$, the free
names in $Q$, and all the names in $R$. Our $\alpha$-equivalence will
be built in the standard way from this substitution.

\begin{remark}\label{rem:no_self_referential_names}
  One consequence of these definitions is that $\forall P. \quotep{P}
  \not\in \freenames{P}$.
\end{remark}

\subsection{ Dynamic quote: an example }

Anticipating something of what's to come, consider applying the
substitution, $\widehat{\id{\{}u / z \id{\}}}$, to the following pair
of processes, $\lift{w}{y!(z)}$ and $w[ \lpquote y!(z) \rpquote ]$.

\begin{eqnarray}
	\lift{w}{y!(z)}\widehat{\id{\{}u / z \id{\}}}
		& = &
		\lift{w}{y!(u)} \nonumber\\
	w[ \lpquote y!(z) \rpquote ] \widehat{ \id{\{}u / z \id{\}} }
		& = &
		w[ \lpquote y!(z) \rpquote ] \nonumber
\end{eqnarray}

Because the body of the process between quotes is impervious to
substitution, we get radically different answers. In fact, by
examining the first process in an input context,
e.g. $x?(z).\lift{w}{y!(z)}$, we see that the process under the lift
operator may be shaped by prefixed inputs binding a name inside it. In
this sense, the lift operator will be seen as a way to dynamically
construct processes before reifying them as names.

Finally equipped with these standard features we can present the
dynamics of the calculus.

\subsubsection{Operational semantics} 

Finally, we introduce the computational dynamics. What marks these
algebras as distinct from other more traditionally studied algebraic
structures, e.g. vector spaces or polynomial rings, is the manner in
which dynamics is captured. In traditional structures, dynamics is typically
expressed through morphisms between such structures, as in linear maps
between vector spaces or morphisms between rings. In algebras
associated with the semantics of computation, the dynamics is
expressed as part of the algebraic structure itself, through a
reduction reduction relation typically denoted by $\red$. Below, we
give a recursive presentation of this relation for the calculus used
in the encoding.

$\red \subseteq \pi \times \pi$
$\red : \pi \to \mathcal{P}(\pi)$

\begin{mathpar}
  \inferrule* [lab=Comm] { \textsf{match}( x_{src}, x_{trgt} ) } { x_{trgt}?(y)P \; | \; x_{src}!\langle {Q} \rangle \red P\{\quotep{Q}/y}\} }
  \and \\
  \inferrule* [lab=Par] {{P} \red {P}'} {{{P} | {Q}} \red {{P}' | {Q}}}
  \and
  \inferrule* [lab=Equiv]{{{P} \scong {P}'} \andalso {{P}' \red {Q}'} \andalso {{Q}' \scong {Q}}}{{P} \red {Q}}
\end{mathpar}

\begin{eqnarray*}
  match_{\equiv} (\quotep{P},\quotep{Q}) & := & P \equiv Q \\
  match_{\dagger}(\quotep{P},\quotep{Q}) & := & \forall R. P|Q \red^{*} R => R \red^{*} 0 \\
  match_{K}(\quotep{P},\quotep{Q}) & := & K \mbox{ for some context } K
\end{eqnarray*}

$u?(x)P | u!\langle Q \rangle \red P\{\quotep{Q}/x\}$

%We write $\wred$ for $\red^*$, and $P\red$ if $\exists Q $ such that $ P \red Q$.
We write $P\red$ if $\exists Q $ such that $ P \red Q$ and $P\not\red$, otherwise.

\section{Replication}

As mentioned before, it is known that replication (and hence
recursion) can be implemented in a higher-order process algebra
\cite{SangiorgiWalker}. As our first example of calculation with the
machinery thus far presented we give the construction explicitly in
the {\rhoc}.

\begin{eqnarray}
	D_{x} & := & \prefix{x}{y}{(\binpar{\outputp{x}{y}}{@{y}})} \nonumber\\
	\bangp_{x}{P} & := & \binpar{{x}!\langle{\binpar{D_{x}}{P}}\rangle}{D_{x}} \nonumber
\end{eqnarray}

\begin{eqnarray}
	\bangp_{x}{P} & & \nonumber\\
	=
	& {x}!\langle{(\prefix{x}{y}{(\outputp{x}{y} | @{y})) | P}}\rangle 
	      | \prefix{x}{y}{(\outputp{x}{y} | @{y})} & \nonumber\\
	\red
	& (\outputp{x}{y} | @{y})\substn{\quotep{(\prefix{x}{y}{(@{y} | \outputp{x}{y})) | P}}}{y} & \nonumber\\
	=
	& \outputp{x}{\quotep{(\prefix{x}{y}{(\outputp{x}{y} | @{y})) | P}}}
	  | {(\prefix{x}{y}{(\outputp{x}{y} | @{y})) | P}} & \nonumber\\
	\red
	& \ldots & \nonumber\\
	\red^*
	& P | P | \ldots & \nonumber
\end{eqnarray}

Of course, this encoding, as an implementation, runs away, unfolding
$\bangp{P}$ eagerly. A lazier and more implementable replication
operator, restricted to input-guarded processes, may be obtained as follows.

\begin{eqnarray}
\bangp{\prefix{u}{v}{P}} 
	:= 
	\binpar{\lift{x}{\prefix{u}{v}{(\binpar{D(x)}{P})}}}{D(x)} \nonumber
\end{eqnarray}

\begin{remark}
  Note that the lazier definition still does not deal with summation
  or mixed summation (i.e. sums over input and output). The reader is
  invited to construct definitions of replication that deal with these
  features. 

  Further, the definitions are parameterized in a name, $x$. Can you,
  gentle reader, make a definition that eliminates this parameter and
  guarantees no accidental interaction between the replication
  machinery and the process being replicated -- i.e. no accidental
  sharing of names used by the process to get its work done and the
  name(s) used by the replication to effect copying. This latter
  revision of the definition of replication is crucial to obtaining
  the expected identity $!!P \sim !P$.
\end{remark}

\begin{remark}\label{rem:paradoxical_combinator}
  The reader familiar with the lambda calculus will have noticed the
  similarity between $D$ and the paradoxical combinator.

  [Ed. note: the existence of this seems to suggest we have to be more
  restrictive on the set of processes and names we admit if we are to
  support no-cloning.]
\end{remark}

\subsubsection{Bisimulation}

The computational dynamics gives rise to another kind of equivalence,
the equivalence of computational behavior. As previously mentioned
this is typically captured \emph{via} some form of bisimulation.

% The notion we use in this paper is weak barbed bisimulation
% \cite{milner91polyadicpi}.

The notion we use in this paper is derived from weak barbed
bisimulation \cite{milner91polyadicpi}. 

\begin{definition}
An \emph{observation relation}, $\downarrow_{\mathcal N}$, over a set
of names, $\mathcal N$, is the smallest relation satisfying the rules
below.

\infrule[Out-barb]{y \in {\mathcal N}, \; x \nameeq y}
		  {\outputp{x}{v} \downarrow_{\mathcal N} x}
\infrule[Par-barb]{\mbox{$P\downarrow_{\mathcal N} x$ or $Q\downarrow_{\mathcal N} x$}}
		  {\binpar{P}{Q} \downarrow_{\mathcal N} x}

We write $P \Downarrow_{\mathcal N} x$ if there is $Q$ such that 
$P \wred Q$ and $Q \downarrow_{\mathcal N} x$.
\end{definition}

\begin{definition}
%\label{def.bbisim}
An  ${\mathcal N}$-\emph{barbed bisimulation} over a set of names, ${\mathcal N}$, is a symmetric binary relation 
${\mathcal S}_{\mathcal N}$ between agents such that $P\rel{S}_{\mathcal N}Q$ implies:
\begin{enumerate}
\item If $P \red P'$ then $Q \wred Q'$ and $P'\rel{S}_{\mathcal N} Q'$.
\item If $P\downarrow_{\mathcal N} x$, then $Q\Downarrow_{\mathcal N} x$.
\end{enumerate}
$P$ is ${\mathcal N}$-barbed bisimilar to $Q$, written
$P \wbbisim_{\mathcal N} Q$, if $P \rel{S}_{\mathcal N} Q$ for some ${\mathcal N}$-barbed bisimulation ${\mathcal S}_{\mathcal N}$.
\end{definition}

$\mathcal{R} \subseteq \pi \times \pi$

$P \mathcal{R} Q => \forall P'. P \red P' \Rightarrow \exists Q'. Q \red Q', P' \mathcal{R} Q'$

$P \vdash x \Rightarrow Q \vdash x$

\begin{mathpar}
  \inferrule*[lab=Out-barb]{x \nameeq y}{{y}!\langle{Q}\rangle \vdash x}
  \and
  \inferrule*[lab=Par-barb]{\mbox{$P\vdash x$ or $Q\vdash x$}}{\binpar{P}{Q} \vdash x}
\end{mathpar}

\subsubsection{Contexts}

One of the principle advantages of computational calculi like the
$\pi$-calculus is a well-defined notion of context,
contextual-equivalence and a correlation between
contextual-equivalence and notions of bisimulation. The notion of
context allows the decomposition of a process into (sub-)process and
its syntactic environment, its context. Thus, a context may be
thought of as a process with a ``hole'' (written $\Box$) in it. The
application of a context $M$ to a process $P$, written $M[P]$, is
tantamount to filling the hole in $M$ with $P$. In this paper we do
not need the full weight of this theory, but do make use of the notion
of context in the proof the main theorem. 

\begin{mathpar}
  \inferrule* [lab=summation] {} {{M_{M},M_{N}} \bc \Box \;|\; x.M_{A} \;|\; M_{M}+M_{N}}
  \and
  \inferrule* [lab=agent] {} {{M_{A}} \bc (\vec{x})M_{P} \;| \; \clift{P_0,\ldots,M_{P},\ldots,P_N}}
  \and \\
  \inferrule* [lab=process] {} {{M_{P}} \bc M_{N} \;| \;P|M_{P} }
\end{mathpar} 

\begin{mathpar}
  \inferrule* [lab=sychronization] {} {M_{N} \bc \Box \;|\; x?M_{F} \;|\; x!M_{C}}
  \and
  \inferrule* [lab=abstraction] {} {{M_{F}} \bc (x)M_{P} }
  \and
  \inferrule* [lab=concretion] {} {{M_{C}} \bc \langle M_{P} \rangle }
  \and \\
  \inferrule* [lab=process] {} {{M_{P}} \bc M_{N} \;| \;P|M_{P} }
\end{mathpar}

\begin{definition}[contextual application] Given a context $M$, and
  process $P$, we define the \emph{contextual application}, $M[P] :=
  M\{P/\Box\}$. That is, the contextual application of M to P is the
  substitution of $P$ for $\Box$ in $M$.
\end{definition}

$\meaningof{-} : L \to \mathcal{P}(\pi)$

\begin{mathpar}
  \inferrule* [lab=collection] {} {\meaningof{true} = \pi, \and \meaningof{~E} = \pi \setminus \meaningof{E}, \and \meaningof{E_{1} \& E_{2}} = \meaningof{E_{1}} \cap \meaningof{E_{2}}}
\end{mathpar}

\begin{mathpar}
  \inferrule* [lab=structure] {} {\meaningof{0} = \{ P \in \pi | P \equiv 0 \}, \and \\ \meaningof{E_1 | E_2} = \{ P \in \pi | P \equiv P_{1} | P_{2}, P_{1} \in \meaningof{E_{1}}, P_{2} \in \meaningof{E_2}\} }
\end{mathpar}

\begin{mathpar}
 \inferrule* [lab=behavior] {} {\meaningof{\langle a?b \rangle E} = \{ P \in \pi | P \equiv Q | u?(y)P', \\ \and \\\\ \and \\ \;\;\; u \in \meaningof{a}, \forall z.P'\{z/y\} \in \meaningof{E\{z/b\}}\}, \and \\ \meaningof{a!E} = \{ P \in \pi | P \equiv Q | x!\langle P' \rangle, x \in \meaningof{a} P' \in \meaningof{E}\} }
\end{mathpar}

\begin{mathpar}
 \inferrule* [lab=nominal] {} {\meaningof{\quotep{E}} = \{ \quotep{P} \in \quotep{\pi} | P \in \meaningof{E} \}, \and \meaningof{\quotep{P}} = \{ \quotep{Q} \in \quotep{\pi} | P \equiv Q \} \and \\ \meaningof{@\quotep{E}} = \{ P \in \pi | P \equiv @x, x \in \meaningof{E} \}}
\end{mathpar}

\begin{eqnarray*}
  \\
  \meaningof{-} : TS \to ST
\end{eqnarray*}

\begin{eqnarray*}
  \\
  L : TS \to ST
\end{eqnarray*}

\begin{eqnarray*}
  \\
  P \models E \iff P \in \meaningof{E}
\end{eqnarray*}

\begin{eqnarray*}
  P \approx_{L} Q \iff \forall E \in L. P \models E \iff Q \models E
\end{eqnarray*}

\begin{eqnarray*}
  P \approx_{K} Q
\end{eqnarray*}

\begin{eqnarray*}
  P \approx Q
\end{eqnarray*}

$\approx_{K} = \approx = \approx_{L}$

\subsubsection{Contextual duality}

Note that contexts extend the quotation operation to a family of
operations from processes to names. Given a context, $M$, we can
define a \emph{nominal context}, $\quotep{M}$ by $\quotep{M}[P] :=
\quotep{M[P]}$. To foreshadow what is to come we observe that these
operations enjoy a duality with processes very much like the duality
between vectors and maps from vectors to scalars.

Further, because the calculus is essentially higher-order, we have a
correspondence between contexts and processes. More specifically,
given a name $x$ and a context $M$ we can construct $M^{*}_{x}$ such
that 

\begin{mathpar}
  M^{*}_{x} | \lift{x}{P} \red M[P]
\end{mathpar}

namely,

\begin{mathpar}
  M^{*}_{x} := x?(u).M[\dropn{u}]
\end{mathpar}

The dependence of $M^{*}_{x}$ on a name makes it an abstraction, 

\begin{mathpar}
  M^{*} := (x)x?(u).M[\dropn{u}]
\end{mathpar}

\subsection{Additional notation}

It will sometimes be convenient to denote the process a name
quotes. We already have the notation $x = \quotep{P}$, but it will be
convenient to introduce an alternate notation, $\procn{x}$, when we
want to emphasize the connection to the use of the name. Note that, by
virtue of name equivalence, $\quotep{\procn{x}} \nameeq x$; so, the
notation is consistent with previous definitions.

Further, because names have structure it is possible to effect
substitutions on the basis of that structure. This means we need to
upgrade our notation for substitutions, which we accomplish by
adapting comprehension notation. Thus,

\begin{mathpar}
  P\{ y / x : x \in S \}
\end{mathpar}

is interpreted to mean the process derived from P by replacing (in a
capture-avoiding manner) each occurrence of $x$ in $S$ by $y$. For example,

\begin{mathpar}
  P\{ \quotep{\procn{x}|\procn{x}} / x : x \in \freenames{P} \}
\end{mathpar}

will replace each (occurrence) of a free name $x$ in $P$ by
$\quotep{\procn{x}|\procn{x}}$.

Also, we will avail ourselves of the notation $x^{L}$ and $x^{R}$ to
denote injections of a name into disjoint copies of the name
space. There are numerous ways to accomplish this. One example can be
found in \cite{MeredithR05}. This notation overloads to vectors of
names: $\vec{x}^{\pi} := (x_{i}^{\pi} \; : \; 0 \leq i < |\vec{x}| )$ where $\pi \in \{L,R\}$.

We also use $P^{\Box} := P|\Box$.

In \cite{MeredithR05} an interpretation of the new operator is
given. It turns out that there are several possible interpretations
all enjoying the requisite algebraic properties of the operator (see
\cite{milner91polyadicpi}). We will therefore make liberal use of
$(\nu\; \vec{x})P$.

% subsection the_syntax_and_semantics_of_the_notation_system (end)   

\input{qm2pi.qmops} 

\input{qm2pi.sterngerlach} 

\input{qm2pi.metric} 

% section concurrent_process_calculi (end)

%\input{qm2pi.proofsketch}

% section proof sketch (end)

%\input{qm2pi.slviaknots} 

% section spatial logic via knots (end)

\input{qm2pi.conclusion}

% section conclusion (end)

%\input{qm2pi.dtcodes} 

% section wiring algorithm (end)

\input{qm2pi.ack} 

% section acknowledgments (end)

\newpage


\bibliographystyle{plain}   
\bibliography{../../biblios/main.bib}

\input{qm2pi.rhodetails}

\end{document}



% section front matter (end)

\section{Introduction}\label{sec:introduction} % (fold)
In this draft of the material i am going to have to dispense with the
usual writing conventions adopted in papers on these topics. i'm going
to have adopt whatever tone i need at the time i'm writing up the
calculations. Sometimes this may be very conversational; others it may
be the barest mathematical grunts; others still it may be that i have
lifted text from one of my other papers because the exposition of some
point was better said there. i hope that my readers are not unduly put
out by this decision. i'm not doing this to flout convention or be
rebellious. i find these calculations very technically challenging. To
keep everything going technically, something has to give; i have to
let go of some cognitive burden. So, the academic writing style --
with all of its trade-offs in terms of facilitating technical
communication -- is what i'm letting go of. Perhaps subsequent drafts
can be tightened and polished, but for now, i'm going to speak as if
we were sitting together in a coffee shop with a laptop, wifi and a
pad of paper and a pencil.

So, here's what i have to say. We -- you and i, comfortably ensconced
in our coffee shop and well-equipped with our tools -- can realize and
carry out the calculations of quantum mechanics over a very different
formal theory of dynamics, a formal theory of dynamics that
corresponds to a theory of concurrent computation with
\emph{reflection}. It has the advantage that the underlying theory is
already `quantized', but supports analogues all of the continuuous
operations. Strikingly, this underlying theory has recently been
connected with a notion of metric that we can show, by calculating
together, coincides with the metric induced by the inner product.

There are a lot of reasons why you might be interested in seeing
calculations of this form. Here's why i'm interested. For the past
several centuries there has been no competitor to the ``Newtonian''
account of dynamics. As a result the predominant share of accounts of
dynamical systems and situations have had to be formulated in terms of
the Newtonian machinery. i view this as an intellectually dangerous
position to occupy. Everything, despite it's intrinsic shape, turns
into a nail to be hit with this hammer. Recently, however, the theory
of computation has matured to the point where we have candidates for
theories of dynamics that offer very different perspective on
reasoning about dynamical systems and situations. Testing these
candidates against very successful accounts of dynamical situations,
like quantum mechanics, is going to give us some sense of how mature
they are and some measure of the quality of these accounts of
dynamics.

\subsection{Summary of contributions and outline of paper}

So, we're going to develop an interpretation of the operations of
quantum mechanics normally interpreted by Hilbert spaces and
operators. We're going to do this over a theory of computation. Note
that this is very different than the usual quantum computation program
which develops notions of computation over quantum mechanics. Rather,
we are developing a story that aligns with Wheeler's slogan: It from
Bit. To do this we will first provide an account of the theory of
computation at play here. Then we will dive into a calculation-driven
interpretation of the operations of quantum mechanics.

The reason we take this approach is that -- until very recently --
there hasn't been an axiomatic account of quantum mechanics. As a
result there has been no sharp delineation of the mathematical theory
supporting interpretation of the physical theory and the physical
theory, itself. So, ambient features of the maths are free to be
exploited (or supressed) without a real accounting of their physical
relevance. There is no sharp statement ``here's the physical theory''
qua \emph{theory} and ``here's the mathematical interpretation''
enabling a judgment of how faithful the interpretation is -- apart
from experimental observation. When there is an axiomatic account we
can judge how well a given mathematical formalism supports an
interpretation of the axioms, independent of
experimentation. Likewise, we can judge how well we have captured our
physical evidence and experience with our axiomatics, independent of
any specific mathematical implementation, with accidental detail that
may or may not have physical significance. 

In lieu of a fully fleshed out and vetted axiomatic account of quantum
mechanics, interpreting the operational notions in service of modeling
physical systems will have to suffice. In other words, we are not in
the business of providing a model of Hilbert spaces and operators. We
are in the business of providing a model of quantum mechanics because
we are motivated by testing our notions of dynamics against physical
theory; and, the predictive calculations of the physical theory must
serve as the best formulation -- shy of a fully fleshed out axiomatic
account -- of the physical theory itself (as they have for scientific
theories since time immemorial). Put another way, despite a
whole-hearted commitment to an It-from-Bit ontology, we are firmly
aligned with the shut-up-and-calculate camp as the best way to obtain
results either from the physical perspective or as a quality assurance
measure of our fledgling theory of dynamics.

In detail, we present a reflective process calculus. Then we develop
intuitive correspondences between the notions available in this
calculus and the usual physical notions supporting quantum mechanical
calculations. Thus, 

\begin{table}[htp]
  \center{
    \fbox{
      \begin{tabular}{c|c}
        quantum mechanics & process calculus \\
        \hline
        scalar & name \\
        state vector & process \\
        dual & contextual duals \\
        matrix & formal sums of process-context-dual pairs \\
        orthogonality & process annihilation \\
        inner product & execution-formula + quoting
      \end{tabular}
    }
  }
  \caption{QM - process calculi correspondences}
\end{table}

Then we tighten up these intuitions to operational definitions. We
employ the Dirac notation as the best proxy we can find for an
abstract syntax of the quantum mechanical notions. The definitions we
develop put us in contact with equational constraints coming from the
theory that we demonstrate the definitions and calculations satisfy.

This puts us in a position to shut up and calculate for the
Stern-Gerlach experimental set up, showing how these predictive
calculations become calculations on processes in our theory of a
reflective process calculus.

Penultimately, we demonstrate that the notion of metric coming from
the inner product coincides with the notion of metric available from
the theory of bisimulation. This demonstration gives us the right to
think of space as arising from behavior. Finally, we consider where we
might go from the new vantage point we have obtained.

% section introduction (end) 
 
% section introduction (end)

% \documentclass[12pt]{llncs}
%\documentclass{jktr}

\usepackage[pdftex]{hyperref}                   
\usepackage {listings}
\usepackage {mathpartir}
\usepackage{bcprules}
%\usepackage{listings}
                       
\usepackage{graphicx} 
%\usepackage[margins=2.5cm,nohead,nofoot]{geometry}
%\usepackage{geometry}
\usepackage{amsfonts}
\usepackage{amstext}
\usepackage{latexsym}
\usepackage{amssymb}
\usepackage{color}


%\include{myPreamble}
\include{qm2pi.local} 

%\ifpdf
%\usepackage[pdftex]{graphicx}
%\else
%\usepackage{graphicx}
%\fi

 % \ifpdf
%  \usepackage{pdfsync}
%  \if


%\title{Brief Article}
%\author{David F. Snyder}
%\author{L.G. Meredith}

%\address{Dept. of Math., Texas State University--San Marcos, San Marcos, TX 78666}
       
\pagestyle{empty}


\begin{document}

\lstset{language=[Objective]Caml,frame=shadowbox}

\input{qm2pi.front}

% section front matter (end)

\input{qm2pi.intro} 
 
% section introduction (end)

% \input{qm2pi.knotations} 

% section notation (end)

\input{qm2pi.process.calculi} 

% section concurrent_process_calculi_and_spatial_logics_ (end)
    
%\input{qm2pi.knots2pi} 

%\input{qm2pi.trefoil} 

%\input{qm2pi.mainthm} 

% subsection basic_interpretation (end)

%\input{qm2pi.rho.presentation} 
\subsection{The syntax and semantics of the notation system}\label{sub:the_syntax_and_semantics_of_the_notation_system} % (fold)

We now summarize a technical presentation of the calculus that
embodies our theory of dynamics. The typical presentation of such a
calculus follows the style of giving generators and relations on
them. The grammar, below, describing term constructors, freely
generates the set of processes, $\Proc$. This set is then quotiented
by a relation known as structural congruence and it is over this set
that the notion of dynamics is expressed. This presentation is
essentially that of \cite{MeredithR05} with the addition of
polyadicity and summation. For readability we have relegated some of
the technical subtleties to an appendix.

\subsubsection{Process grammar}\label{subsub:process_grammar}

\begin{mathpar}
  \inferrule* [lab=synchronization] {} {{M} \bc \pzero \;|\; x?F \;|\; x!C }
  \and
  \inferrule* [lab=abstraction] {} {{F} \bc (x)P}
  \and
  \inferrule* [lab=concretion] {} {{C} \bc \langle Q \rangle}
  \and
  \inferrule* [lab=process] {} {{P,Q} \bc M \;| \;P|Q \;|\; @{x}}
  \and
  \inferrule* [lab=name] {} {{x} \bc \quotep{P}}
\end{mathpar} 

Note that $\vec{x}$ (resp. $\vec{P}$) denotes a vector of names
(resp. processes) of length $|\vec{x}|$ (resp. $|\vec{P}|$). We adopt
the following useful abbreviations.

\begin{mathpar}
   x?(\vec{y}).P := x.(\vec{y})P \and  x\clift{\vec{P}} := x.\clift{\vec{P}}
   \and x!(y) := \lift{x}{\dropn{y}}
   \and \Pi_{i=0}^{n-1}P_i := P_0 | \ldots | P_{n-1}
\end{mathpar}

\subsubsection{Structural congruence}

\paragraph{Free and bound names and alpha-equivalence.} At the
core of structural equivalence is alpha-equivalence which identifies
process that are the same up to a change of variable. Formally, we
recognize the distinction between free and bound names. The free names
of a process, $\freenames{P}$, may be calculated recursively as
follows:

\begin{mathpar}
\freenames{\pzero} := \emptyset
  \and \\
  \freenames{x?(y).P} := \{ x \} \cup (\freenames{P} \setminus \{ y \})
  \and 
  \freenames{x!\langle P \rangle} := \{ x \} \cup \{ P \} 
  \and \\
  \freenames{P|Q} := \freenames{P} \cup \freenames{Q}
  \and \\
  \freenames{@{x}} := \{ x \}
\end{mathpar}

$\pi$
$\quotep{\pi}$

$\freenames{-} : \pi \to \mathcal{P}(\quotep{\pi})$

\begin{eqnarray*}
  \freenames{\pzero} & := & \emptyset \\
  \freenames{x?(y).P} & := & \{ x \} \cup (\freenames{P} \setminus \{ y \}) \\
  \freenames{x!\langle P \rangle} & := & \{ x \} \cup \{ P \} \\
  \freenames{P|Q} & := & \freenames{P} \cup \freenames{Q} \\
  \freenames{\dropn{x}} & := & \{ x \}
\end{eqnarray*}

The bound names of a process, $\boundnames{P}$, are those names occurring in $P$
that are not free. For example, in $x?(y).0$, the name $x$ is free, while $y$ is bound.

\begin{mathpar}
  \inferrule* [lab=monoidal-laws] {} { P|Q \equiv Q|P \and P|0 \equiv P \and P|(Q|R) \equiv (P|Q)|R }
\end{mathpar}

\begin{mathpar}
  \inferrule* [lab=alpha-equivalence] {} { (x)P \equiv (y)P\{y/x\} \and y \not\in \freenames{P} }
\end{mathpar}

\begin{definition}
Then two processes, $P,Q$, are alpha-equivalent if $P = Q\{\vec{y}/\vec{x}\}$ for
some $\vec{x} \in \boundnames{Q},\vec{y} \in \boundnames{P}$, where $Q\{\vec{y}/\vec{x}\}$
denotes the capture-avoiding substitution of $\vec{y}$ for $\vec{x}$ in $Q$.
\end{definition}

\begin{definition}
  The {\em structural congruence} \cite{SangiorgiWalker} , $\equiv$,
  between processes is the least congruence containing
  alpha-equivalence, satisfying the abelian monoid laws
  (associativity, commutativity and $\pzero$ as identity) for parallel
  composition $|$ and for summation $+$.
\end{definition}

\subsection{Name equivalence}

We take name equivalence, written $\nameeq$, to be the smallest
equivalence relation generated by the following rules.

\begin{mathpar}
\inferrule*[lab=Quote-drop]
{ }
{ \quotep{@{x}} \nameeq x }

\inferrule*[lab=Struct-equiv]
{ P \scong Q }
{ \quotep{P} \nameeq \quotep{Q} }
\end{mathpar}

The astute reader will have noticed that the mutual recursion of names
and processes imposes a mutual recursion on alpha-equivalence and
structural equivalence via name-equivalence. Fortunately, all of this
works out pleasantly and we may calculate in the natural way, free of
concern. The reader interested in the details is referred to the
appendix \ref{appendix:rho_details}.

\subsection{Substitution}

We use $\Proc$ for the set of processes, $\QProc$ for the set of
names, and $\id{\{}\vec{y} / \vec{x} \id{\}}$ to denote partial maps,
$s : \QProc \rightarrow \QProc$. A map, $s$ lifts, uniquely, to a map
on process terms, $\widehat{s} : \Proc \rightarrow \Proc$ by the
following equations.

\begin{mathpar}
  (0) \psubstp{Q}{P} := 0 \\
  (R \juxtap S) \psubstp{Q}{P}
  :=    
  (R)\psubstp{Q}{P} \juxtap (S) \psubstp{Q}{P} \\
  (x?(y).R) \psubstp{Q}{P}    
  :=    
  (x)\substp{Q}{P} (z)\concat( (R \psubstn{z}{y}) \psubstp{Q}{P} ) \\
  (\lift{x}{R}) \psubstp{Q}{P}  
  :=
  \lift{(x)\substp{Q}{P}}{ R \psubstp{Q}{P} } \\
%   (\dropn{x})  \psubstp{Q}{P}       
%   := 
%   \left\{ 
%     \begin{array}{ccc} 
%       \dropn{\quotep{Q}} & & x \nameeq \quotep{P} \\
%       \dropn{x} & & otherwise \\
%     \end{array}
%   \right. 
  (\dropn{x})  \psubstp{Q}{P}       
  := 
  \left\{ 
    \begin{array}{ccc} 
      Q & & x \nameeq \quotep{P} \\
      \dropn{x} & & otherwise \\
    \end{array}
  \right.
\end{mathpar}
 

where

\begin{eqnarray}
  (x)\id{\{} \lpquote Q \rpquote / \lpquote P \rpquote \id{\}}            = 
  \left\{ 
    \begin{array}{ccc}
      \lpquote Q \rpquote & & x \nameeq \lpquote P \rpquote \\
      x & & otherwise \\
    \end{array}
  \right. \nonumber
\end{eqnarray}

and $z$ is chosen distinct from $\quotep{P}$, $\quotep{Q}$, the free
names in $Q$, and all the names in $R$. Our $\alpha$-equivalence will
be built in the standard way from this substitution.

\begin{remark}\label{rem:no_self_referential_names}
  One consequence of these definitions is that $\forall P. \quotep{P}
  \not\in \freenames{P}$.
\end{remark}

\subsection{ Dynamic quote: an example }

Anticipating something of what's to come, consider applying the
substitution, $\widehat{\id{\{}u / z \id{\}}}$, to the following pair
of processes, $\lift{w}{y!(z)}$ and $w[ \lpquote y!(z) \rpquote ]$.

\begin{eqnarray}
	\lift{w}{y!(z)}\widehat{\id{\{}u / z \id{\}}}
		& = &
		\lift{w}{y!(u)} \nonumber\\
	w[ \lpquote y!(z) \rpquote ] \widehat{ \id{\{}u / z \id{\}} }
		& = &
		w[ \lpquote y!(z) \rpquote ] \nonumber
\end{eqnarray}

Because the body of the process between quotes is impervious to
substitution, we get radically different answers. In fact, by
examining the first process in an input context,
e.g. $x?(z).\lift{w}{y!(z)}$, we see that the process under the lift
operator may be shaped by prefixed inputs binding a name inside it. In
this sense, the lift operator will be seen as a way to dynamically
construct processes before reifying them as names.

Finally equipped with these standard features we can present the
dynamics of the calculus.

\subsubsection{Operational semantics} 

Finally, we introduce the computational dynamics. What marks these
algebras as distinct from other more traditionally studied algebraic
structures, e.g. vector spaces or polynomial rings, is the manner in
which dynamics is captured. In traditional structures, dynamics is typically
expressed through morphisms between such structures, as in linear maps
between vector spaces or morphisms between rings. In algebras
associated with the semantics of computation, the dynamics is
expressed as part of the algebraic structure itself, through a
reduction reduction relation typically denoted by $\red$. Below, we
give a recursive presentation of this relation for the calculus used
in the encoding.

$\red \subseteq \pi \times \pi$
$\red : \pi \to \mathcal{P}(\pi)$

\begin{mathpar}
  \inferrule* [lab=Comm] { \textsf{match}( x_{src}, x_{trgt} ) } { x_{trgt}?(y)P \; | \; x_{src}!\langle {Q} \rangle \red P\{\quotep{Q}/y}\} }
  \and \\
  \inferrule* [lab=Par] {{P} \red {P}'} {{{P} | {Q}} \red {{P}' | {Q}}}
  \and
  \inferrule* [lab=Equiv]{{{P} \scong {P}'} \andalso {{P}' \red {Q}'} \andalso {{Q}' \scong {Q}}}{{P} \red {Q}}
\end{mathpar}

\begin{eqnarray*}
  match_{\equiv} (\quotep{P},\quotep{Q}) & := & P \equiv Q \\
  match_{\dagger}(\quotep{P},\quotep{Q}) & := & \forall R. P|Q \red^{*} R => R \red^{*} 0 \\
  match_{K}(\quotep{P},\quotep{Q}) & := & K \mbox{ for some context } K
\end{eqnarray*}

$u?(x)P | u!\langle Q \rangle \red P\{\quotep{Q}/x\}$

%We write $\wred$ for $\red^*$, and $P\red$ if $\exists Q $ such that $ P \red Q$.
We write $P\red$ if $\exists Q $ such that $ P \red Q$ and $P\not\red$, otherwise.

\section{Replication}

As mentioned before, it is known that replication (and hence
recursion) can be implemented in a higher-order process algebra
\cite{SangiorgiWalker}. As our first example of calculation with the
machinery thus far presented we give the construction explicitly in
the {\rhoc}.

\begin{eqnarray}
	D_{x} & := & \prefix{x}{y}{(\binpar{\outputp{x}{y}}{@{y}})} \nonumber\\
	\bangp_{x}{P} & := & \binpar{{x}!\langle{\binpar{D_{x}}{P}}\rangle}{D_{x}} \nonumber
\end{eqnarray}

\begin{eqnarray}
	\bangp_{x}{P} & & \nonumber\\
	=
	& {x}!\langle{(\prefix{x}{y}{(\outputp{x}{y} | @{y})) | P}}\rangle 
	      | \prefix{x}{y}{(\outputp{x}{y} | @{y})} & \nonumber\\
	\red
	& (\outputp{x}{y} | @{y})\substn{\quotep{(\prefix{x}{y}{(@{y} | \outputp{x}{y})) | P}}}{y} & \nonumber\\
	=
	& \outputp{x}{\quotep{(\prefix{x}{y}{(\outputp{x}{y} | @{y})) | P}}}
	  | {(\prefix{x}{y}{(\outputp{x}{y} | @{y})) | P}} & \nonumber\\
	\red
	& \ldots & \nonumber\\
	\red^*
	& P | P | \ldots & \nonumber
\end{eqnarray}

Of course, this encoding, as an implementation, runs away, unfolding
$\bangp{P}$ eagerly. A lazier and more implementable replication
operator, restricted to input-guarded processes, may be obtained as follows.

\begin{eqnarray}
\bangp{\prefix{u}{v}{P}} 
	:= 
	\binpar{\lift{x}{\prefix{u}{v}{(\binpar{D(x)}{P})}}}{D(x)} \nonumber
\end{eqnarray}

\begin{remark}
  Note that the lazier definition still does not deal with summation
  or mixed summation (i.e. sums over input and output). The reader is
  invited to construct definitions of replication that deal with these
  features. 

  Further, the definitions are parameterized in a name, $x$. Can you,
  gentle reader, make a definition that eliminates this parameter and
  guarantees no accidental interaction between the replication
  machinery and the process being replicated -- i.e. no accidental
  sharing of names used by the process to get its work done and the
  name(s) used by the replication to effect copying. This latter
  revision of the definition of replication is crucial to obtaining
  the expected identity $!!P \sim !P$.
\end{remark}

\begin{remark}\label{rem:paradoxical_combinator}
  The reader familiar with the lambda calculus will have noticed the
  similarity between $D$ and the paradoxical combinator.

  [Ed. note: the existence of this seems to suggest we have to be more
  restrictive on the set of processes and names we admit if we are to
  support no-cloning.]
\end{remark}

\subsubsection{Bisimulation}

The computational dynamics gives rise to another kind of equivalence,
the equivalence of computational behavior. As previously mentioned
this is typically captured \emph{via} some form of bisimulation.

% The notion we use in this paper is weak barbed bisimulation
% \cite{milner91polyadicpi}.

The notion we use in this paper is derived from weak barbed
bisimulation \cite{milner91polyadicpi}. 

\begin{definition}
An \emph{observation relation}, $\downarrow_{\mathcal N}$, over a set
of names, $\mathcal N$, is the smallest relation satisfying the rules
below.

\infrule[Out-barb]{y \in {\mathcal N}, \; x \nameeq y}
		  {\outputp{x}{v} \downarrow_{\mathcal N} x}
\infrule[Par-barb]{\mbox{$P\downarrow_{\mathcal N} x$ or $Q\downarrow_{\mathcal N} x$}}
		  {\binpar{P}{Q} \downarrow_{\mathcal N} x}

We write $P \Downarrow_{\mathcal N} x$ if there is $Q$ such that 
$P \wred Q$ and $Q \downarrow_{\mathcal N} x$.
\end{definition}

\begin{definition}
%\label{def.bbisim}
An  ${\mathcal N}$-\emph{barbed bisimulation} over a set of names, ${\mathcal N}$, is a symmetric binary relation 
${\mathcal S}_{\mathcal N}$ between agents such that $P\rel{S}_{\mathcal N}Q$ implies:
\begin{enumerate}
\item If $P \red P'$ then $Q \wred Q'$ and $P'\rel{S}_{\mathcal N} Q'$.
\item If $P\downarrow_{\mathcal N} x$, then $Q\Downarrow_{\mathcal N} x$.
\end{enumerate}
$P$ is ${\mathcal N}$-barbed bisimilar to $Q$, written
$P \wbbisim_{\mathcal N} Q$, if $P \rel{S}_{\mathcal N} Q$ for some ${\mathcal N}$-barbed bisimulation ${\mathcal S}_{\mathcal N}$.
\end{definition}

$\mathcal{R} \subseteq \pi \times \pi$

$P \mathcal{R} Q => \forall P'. P \red P' \Rightarrow \exists Q'. Q \red Q', P' \mathcal{R} Q'$

$P \vdash x \Rightarrow Q \vdash x$

\begin{mathpar}
  \inferrule*[lab=Out-barb]{x \nameeq y}{{y}!\langle{Q}\rangle \vdash x}
  \and
  \inferrule*[lab=Par-barb]{\mbox{$P\vdash x$ or $Q\vdash x$}}{\binpar{P}{Q} \vdash x}
\end{mathpar}

\subsubsection{Contexts}

One of the principle advantages of computational calculi like the
$\pi$-calculus is a well-defined notion of context,
contextual-equivalence and a correlation between
contextual-equivalence and notions of bisimulation. The notion of
context allows the decomposition of a process into (sub-)process and
its syntactic environment, its context. Thus, a context may be
thought of as a process with a ``hole'' (written $\Box$) in it. The
application of a context $M$ to a process $P$, written $M[P]$, is
tantamount to filling the hole in $M$ with $P$. In this paper we do
not need the full weight of this theory, but do make use of the notion
of context in the proof the main theorem. 

\begin{mathpar}
  \inferrule* [lab=summation] {} {{M_{M},M_{N}} \bc \Box \;|\; x.M_{A} \;|\; M_{M}+M_{N}}
  \and
  \inferrule* [lab=agent] {} {{M_{A}} \bc (\vec{x})M_{P} \;| \; \clift{P_0,\ldots,M_{P},\ldots,P_N}}
  \and \\
  \inferrule* [lab=process] {} {{M_{P}} \bc M_{N} \;| \;P|M_{P} }
\end{mathpar} 

\begin{mathpar}
  \inferrule* [lab=sychronization] {} {M_{N} \bc \Box \;|\; x?M_{F} \;|\; x!M_{C}}
  \and
  \inferrule* [lab=abstraction] {} {{M_{F}} \bc (x)M_{P} }
  \and
  \inferrule* [lab=concretion] {} {{M_{C}} \bc \langle M_{P} \rangle }
  \and \\
  \inferrule* [lab=process] {} {{M_{P}} \bc M_{N} \;| \;P|M_{P} }
\end{mathpar}

\begin{definition}[contextual application] Given a context $M$, and
  process $P$, we define the \emph{contextual application}, $M[P] :=
  M\{P/\Box\}$. That is, the contextual application of M to P is the
  substitution of $P$ for $\Box$ in $M$.
\end{definition}

$\meaningof{-} : L \to \mathcal{P}(\pi)$

\begin{mathpar}
  \inferrule* [lab=collection] {} {\meaningof{true} = \pi, \and \meaningof{~E} = \pi \setminus \meaningof{E}, \and \meaningof{E_{1} \& E_{2}} = \meaningof{E_{1}} \cap \meaningof{E_{2}}}
\end{mathpar}

\begin{mathpar}
  \inferrule* [lab=structure] {} {\meaningof{0} = \{ P \in \pi | P \equiv 0 \}, \and \\ \meaningof{E_1 | E_2} = \{ P \in \pi | P \equiv P_{1} | P_{2}, P_{1} \in \meaningof{E_{1}}, P_{2} \in \meaningof{E_2}\} }
\end{mathpar}

\begin{mathpar}
 \inferrule* [lab=behavior] {} {\meaningof{\langle a?b \rangle E} = \{ P \in \pi | P \equiv Q | u?(y)P', \\ \and \\\\ \and \\ \;\;\; u \in \meaningof{a}, \forall z.P'\{z/y\} \in \meaningof{E\{z/b\}}\}, \and \\ \meaningof{a!E} = \{ P \in \pi | P \equiv Q | x!\langle P' \rangle, x \in \meaningof{a} P' \in \meaningof{E}\} }
\end{mathpar}

\begin{mathpar}
 \inferrule* [lab=nominal] {} {\meaningof{\quotep{E}} = \{ \quotep{P} \in \quotep{\pi} | P \in \meaningof{E} \}, \and \meaningof{\quotep{P}} = \{ \quotep{Q} \in \quotep{\pi} | P \equiv Q \} \and \\ \meaningof{@\quotep{E}} = \{ P \in \pi | P \equiv @x, x \in \meaningof{E} \}}
\end{mathpar}

\begin{eqnarray*}
  \\
  \meaningof{-} : TS \to ST
\end{eqnarray*}

\begin{eqnarray*}
  \\
  L : TS \to ST
\end{eqnarray*}

\begin{eqnarray*}
  \\
  P \models E \iff P \in \meaningof{E}
\end{eqnarray*}

\begin{eqnarray*}
  P \approx_{L} Q \iff \forall E \in L. P \models E \iff Q \models E
\end{eqnarray*}

\begin{eqnarray*}
  P \approx_{K} Q
\end{eqnarray*}

\begin{eqnarray*}
  P \approx Q
\end{eqnarray*}

$\approx_{K} = \approx = \approx_{L}$

\subsubsection{Contextual duality}

Note that contexts extend the quotation operation to a family of
operations from processes to names. Given a context, $M$, we can
define a \emph{nominal context}, $\quotep{M}$ by $\quotep{M}[P] :=
\quotep{M[P]}$. To foreshadow what is to come we observe that these
operations enjoy a duality with processes very much like the duality
between vectors and maps from vectors to scalars.

Further, because the calculus is essentially higher-order, we have a
correspondence between contexts and processes. More specifically,
given a name $x$ and a context $M$ we can construct $M^{*}_{x}$ such
that 

\begin{mathpar}
  M^{*}_{x} | \lift{x}{P} \red M[P]
\end{mathpar}

namely,

\begin{mathpar}
  M^{*}_{x} := x?(u).M[\dropn{u}]
\end{mathpar}

The dependence of $M^{*}_{x}$ on a name makes it an abstraction, 

\begin{mathpar}
  M^{*} := (x)x?(u).M[\dropn{u}]
\end{mathpar}

\subsection{Additional notation}

It will sometimes be convenient to denote the process a name
quotes. We already have the notation $x = \quotep{P}$, but it will be
convenient to introduce an alternate notation, $\procn{x}$, when we
want to emphasize the connection to the use of the name. Note that, by
virtue of name equivalence, $\quotep{\procn{x}} \nameeq x$; so, the
notation is consistent with previous definitions.

Further, because names have structure it is possible to effect
substitutions on the basis of that structure. This means we need to
upgrade our notation for substitutions, which we accomplish by
adapting comprehension notation. Thus,

\begin{mathpar}
  P\{ y / x : x \in S \}
\end{mathpar}

is interpreted to mean the process derived from P by replacing (in a
capture-avoiding manner) each occurrence of $x$ in $S$ by $y$. For example,

\begin{mathpar}
  P\{ \quotep{\procn{x}|\procn{x}} / x : x \in \freenames{P} \}
\end{mathpar}

will replace each (occurrence) of a free name $x$ in $P$ by
$\quotep{\procn{x}|\procn{x}}$.

Also, we will avail ourselves of the notation $x^{L}$ and $x^{R}$ to
denote injections of a name into disjoint copies of the name
space. There are numerous ways to accomplish this. One example can be
found in \cite{MeredithR05}. This notation overloads to vectors of
names: $\vec{x}^{\pi} := (x_{i}^{\pi} \; : \; 0 \leq i < |\vec{x}| )$ where $\pi \in \{L,R\}$.

We also use $P^{\Box} := P|\Box$.

In \cite{MeredithR05} an interpretation of the new operator is
given. It turns out that there are several possible interpretations
all enjoying the requisite algebraic properties of the operator (see
\cite{milner91polyadicpi}). We will therefore make liberal use of
$(\nu\; \vec{x})P$.

% subsection the_syntax_and_semantics_of_the_notation_system (end)   

\input{qm2pi.qmops} 

\input{qm2pi.sterngerlach} 

\input{qm2pi.metric} 

% section concurrent_process_calculi (end)

%\input{qm2pi.proofsketch}

% section proof sketch (end)

%\input{qm2pi.slviaknots} 

% section spatial logic via knots (end)

\input{qm2pi.conclusion}

% section conclusion (end)

%\input{qm2pi.dtcodes} 

% section wiring algorithm (end)

\input{qm2pi.ack} 

% section acknowledgments (end)

\newpage


\bibliographystyle{plain}   
\bibliography{../../biblios/main.bib}

\input{qm2pi.rhodetails}

\end{document}

 

% section notation (end)

\input{qm2pi.process.calculi} 

% section concurrent_process_calculi_and_spatial_logics_ (end)
    
%\documentclass[12pt]{llncs}
%\documentclass{jktr}

\usepackage[pdftex]{hyperref}                   
\usepackage {listings}
\usepackage {mathpartir}
\usepackage{bcprules}
%\usepackage{listings}
                       
\usepackage{graphicx} 
%\usepackage[margins=2.5cm,nohead,nofoot]{geometry}
%\usepackage{geometry}
\usepackage{amsfonts}
\usepackage{amstext}
\usepackage{latexsym}
\usepackage{amssymb}
\usepackage{color}


%\include{myPreamble}
\include{qm2pi.local} 

%\ifpdf
%\usepackage[pdftex]{graphicx}
%\else
%\usepackage{graphicx}
%\fi

 % \ifpdf
%  \usepackage{pdfsync}
%  \if


%\title{Brief Article}
%\author{David F. Snyder}
%\author{L.G. Meredith}

%\address{Dept. of Math., Texas State University--San Marcos, San Marcos, TX 78666}
       
\pagestyle{empty}


\begin{document}

\lstset{language=[Objective]Caml,frame=shadowbox}

\input{qm2pi.front}

% section front matter (end)

\input{qm2pi.intro} 
 
% section introduction (end)

% \input{qm2pi.knotations} 

% section notation (end)

\input{qm2pi.process.calculi} 

% section concurrent_process_calculi_and_spatial_logics_ (end)
    
%\input{qm2pi.knots2pi} 

%\input{qm2pi.trefoil} 

%\input{qm2pi.mainthm} 

% subsection basic_interpretation (end)

%\input{qm2pi.rho.presentation} 
\subsection{The syntax and semantics of the notation system}\label{sub:the_syntax_and_semantics_of_the_notation_system} % (fold)

We now summarize a technical presentation of the calculus that
embodies our theory of dynamics. The typical presentation of such a
calculus follows the style of giving generators and relations on
them. The grammar, below, describing term constructors, freely
generates the set of processes, $\Proc$. This set is then quotiented
by a relation known as structural congruence and it is over this set
that the notion of dynamics is expressed. This presentation is
essentially that of \cite{MeredithR05} with the addition of
polyadicity and summation. For readability we have relegated some of
the technical subtleties to an appendix.

\subsubsection{Process grammar}\label{subsub:process_grammar}

\begin{mathpar}
  \inferrule* [lab=synchronization] {} {{M} \bc \pzero \;|\; x?F \;|\; x!C }
  \and
  \inferrule* [lab=abstraction] {} {{F} \bc (x)P}
  \and
  \inferrule* [lab=concretion] {} {{C} \bc \langle Q \rangle}
  \and
  \inferrule* [lab=process] {} {{P,Q} \bc M \;| \;P|Q \;|\; @{x}}
  \and
  \inferrule* [lab=name] {} {{x} \bc \quotep{P}}
\end{mathpar} 

Note that $\vec{x}$ (resp. $\vec{P}$) denotes a vector of names
(resp. processes) of length $|\vec{x}|$ (resp. $|\vec{P}|$). We adopt
the following useful abbreviations.

\begin{mathpar}
   x?(\vec{y}).P := x.(\vec{y})P \and  x\clift{\vec{P}} := x.\clift{\vec{P}}
   \and x!(y) := \lift{x}{\dropn{y}}
   \and \Pi_{i=0}^{n-1}P_i := P_0 | \ldots | P_{n-1}
\end{mathpar}

\subsubsection{Structural congruence}

\paragraph{Free and bound names and alpha-equivalence.} At the
core of structural equivalence is alpha-equivalence which identifies
process that are the same up to a change of variable. Formally, we
recognize the distinction between free and bound names. The free names
of a process, $\freenames{P}$, may be calculated recursively as
follows:

\begin{mathpar}
\freenames{\pzero} := \emptyset
  \and \\
  \freenames{x?(y).P} := \{ x \} \cup (\freenames{P} \setminus \{ y \})
  \and 
  \freenames{x!\langle P \rangle} := \{ x \} \cup \{ P \} 
  \and \\
  \freenames{P|Q} := \freenames{P} \cup \freenames{Q}
  \and \\
  \freenames{@{x}} := \{ x \}
\end{mathpar}

$\pi$
$\quotep{\pi}$

$\freenames{-} : \pi \to \mathcal{P}(\quotep{\pi})$

\begin{eqnarray*}
  \freenames{\pzero} & := & \emptyset \\
  \freenames{x?(y).P} & := & \{ x \} \cup (\freenames{P} \setminus \{ y \}) \\
  \freenames{x!\langle P \rangle} & := & \{ x \} \cup \{ P \} \\
  \freenames{P|Q} & := & \freenames{P} \cup \freenames{Q} \\
  \freenames{\dropn{x}} & := & \{ x \}
\end{eqnarray*}

The bound names of a process, $\boundnames{P}$, are those names occurring in $P$
that are not free. For example, in $x?(y).0$, the name $x$ is free, while $y$ is bound.

\begin{mathpar}
  \inferrule* [lab=monoidal-laws] {} { P|Q \equiv Q|P \and P|0 \equiv P \and P|(Q|R) \equiv (P|Q)|R }
\end{mathpar}

\begin{mathpar}
  \inferrule* [lab=alpha-equivalence] {} { (x)P \equiv (y)P\{y/x\} \and y \not\in \freenames{P} }
\end{mathpar}

\begin{definition}
Then two processes, $P,Q$, are alpha-equivalent if $P = Q\{\vec{y}/\vec{x}\}$ for
some $\vec{x} \in \boundnames{Q},\vec{y} \in \boundnames{P}$, where $Q\{\vec{y}/\vec{x}\}$
denotes the capture-avoiding substitution of $\vec{y}$ for $\vec{x}$ in $Q$.
\end{definition}

\begin{definition}
  The {\em structural congruence} \cite{SangiorgiWalker} , $\equiv$,
  between processes is the least congruence containing
  alpha-equivalence, satisfying the abelian monoid laws
  (associativity, commutativity and $\pzero$ as identity) for parallel
  composition $|$ and for summation $+$.
\end{definition}

\subsection{Name equivalence}

We take name equivalence, written $\nameeq$, to be the smallest
equivalence relation generated by the following rules.

\begin{mathpar}
\inferrule*[lab=Quote-drop]
{ }
{ \quotep{@{x}} \nameeq x }

\inferrule*[lab=Struct-equiv]
{ P \scong Q }
{ \quotep{P} \nameeq \quotep{Q} }
\end{mathpar}

The astute reader will have noticed that the mutual recursion of names
and processes imposes a mutual recursion on alpha-equivalence and
structural equivalence via name-equivalence. Fortunately, all of this
works out pleasantly and we may calculate in the natural way, free of
concern. The reader interested in the details is referred to the
appendix \ref{appendix:rho_details}.

\subsection{Substitution}

We use $\Proc$ for the set of processes, $\QProc$ for the set of
names, and $\id{\{}\vec{y} / \vec{x} \id{\}}$ to denote partial maps,
$s : \QProc \rightarrow \QProc$. A map, $s$ lifts, uniquely, to a map
on process terms, $\widehat{s} : \Proc \rightarrow \Proc$ by the
following equations.

\begin{mathpar}
  (0) \psubstp{Q}{P} := 0 \\
  (R \juxtap S) \psubstp{Q}{P}
  :=    
  (R)\psubstp{Q}{P} \juxtap (S) \psubstp{Q}{P} \\
  (x?(y).R) \psubstp{Q}{P}    
  :=    
  (x)\substp{Q}{P} (z)\concat( (R \psubstn{z}{y}) \psubstp{Q}{P} ) \\
  (\lift{x}{R}) \psubstp{Q}{P}  
  :=
  \lift{(x)\substp{Q}{P}}{ R \psubstp{Q}{P} } \\
%   (\dropn{x})  \psubstp{Q}{P}       
%   := 
%   \left\{ 
%     \begin{array}{ccc} 
%       \dropn{\quotep{Q}} & & x \nameeq \quotep{P} \\
%       \dropn{x} & & otherwise \\
%     \end{array}
%   \right. 
  (\dropn{x})  \psubstp{Q}{P}       
  := 
  \left\{ 
    \begin{array}{ccc} 
      Q & & x \nameeq \quotep{P} \\
      \dropn{x} & & otherwise \\
    \end{array}
  \right.
\end{mathpar}
 

where

\begin{eqnarray}
  (x)\id{\{} \lpquote Q \rpquote / \lpquote P \rpquote \id{\}}            = 
  \left\{ 
    \begin{array}{ccc}
      \lpquote Q \rpquote & & x \nameeq \lpquote P \rpquote \\
      x & & otherwise \\
    \end{array}
  \right. \nonumber
\end{eqnarray}

and $z$ is chosen distinct from $\quotep{P}$, $\quotep{Q}$, the free
names in $Q$, and all the names in $R$. Our $\alpha$-equivalence will
be built in the standard way from this substitution.

\begin{remark}\label{rem:no_self_referential_names}
  One consequence of these definitions is that $\forall P. \quotep{P}
  \not\in \freenames{P}$.
\end{remark}

\subsection{ Dynamic quote: an example }

Anticipating something of what's to come, consider applying the
substitution, $\widehat{\id{\{}u / z \id{\}}}$, to the following pair
of processes, $\lift{w}{y!(z)}$ and $w[ \lpquote y!(z) \rpquote ]$.

\begin{eqnarray}
	\lift{w}{y!(z)}\widehat{\id{\{}u / z \id{\}}}
		& = &
		\lift{w}{y!(u)} \nonumber\\
	w[ \lpquote y!(z) \rpquote ] \widehat{ \id{\{}u / z \id{\}} }
		& = &
		w[ \lpquote y!(z) \rpquote ] \nonumber
\end{eqnarray}

Because the body of the process between quotes is impervious to
substitution, we get radically different answers. In fact, by
examining the first process in an input context,
e.g. $x?(z).\lift{w}{y!(z)}$, we see that the process under the lift
operator may be shaped by prefixed inputs binding a name inside it. In
this sense, the lift operator will be seen as a way to dynamically
construct processes before reifying them as names.

Finally equipped with these standard features we can present the
dynamics of the calculus.

\subsubsection{Operational semantics} 

Finally, we introduce the computational dynamics. What marks these
algebras as distinct from other more traditionally studied algebraic
structures, e.g. vector spaces or polynomial rings, is the manner in
which dynamics is captured. In traditional structures, dynamics is typically
expressed through morphisms between such structures, as in linear maps
between vector spaces or morphisms between rings. In algebras
associated with the semantics of computation, the dynamics is
expressed as part of the algebraic structure itself, through a
reduction reduction relation typically denoted by $\red$. Below, we
give a recursive presentation of this relation for the calculus used
in the encoding.

$\red \subseteq \pi \times \pi$
$\red : \pi \to \mathcal{P}(\pi)$

\begin{mathpar}
  \inferrule* [lab=Comm] { \textsf{match}( x_{src}, x_{trgt} ) } { x_{trgt}?(y)P \; | \; x_{src}!\langle {Q} \rangle \red P\{\quotep{Q}/y}\} }
  \and \\
  \inferrule* [lab=Par] {{P} \red {P}'} {{{P} | {Q}} \red {{P}' | {Q}}}
  \and
  \inferrule* [lab=Equiv]{{{P} \scong {P}'} \andalso {{P}' \red {Q}'} \andalso {{Q}' \scong {Q}}}{{P} \red {Q}}
\end{mathpar}

\begin{eqnarray*}
  match_{\equiv} (\quotep{P},\quotep{Q}) & := & P \equiv Q \\
  match_{\dagger}(\quotep{P},\quotep{Q}) & := & \forall R. P|Q \red^{*} R => R \red^{*} 0 \\
  match_{K}(\quotep{P},\quotep{Q}) & := & K \mbox{ for some context } K
\end{eqnarray*}

$u?(x)P | u!\langle Q \rangle \red P\{\quotep{Q}/x\}$

%We write $\wred$ for $\red^*$, and $P\red$ if $\exists Q $ such that $ P \red Q$.
We write $P\red$ if $\exists Q $ such that $ P \red Q$ and $P\not\red$, otherwise.

\section{Replication}

As mentioned before, it is known that replication (and hence
recursion) can be implemented in a higher-order process algebra
\cite{SangiorgiWalker}. As our first example of calculation with the
machinery thus far presented we give the construction explicitly in
the {\rhoc}.

\begin{eqnarray}
	D_{x} & := & \prefix{x}{y}{(\binpar{\outputp{x}{y}}{@{y}})} \nonumber\\
	\bangp_{x}{P} & := & \binpar{{x}!\langle{\binpar{D_{x}}{P}}\rangle}{D_{x}} \nonumber
\end{eqnarray}

\begin{eqnarray}
	\bangp_{x}{P} & & \nonumber\\
	=
	& {x}!\langle{(\prefix{x}{y}{(\outputp{x}{y} | @{y})) | P}}\rangle 
	      | \prefix{x}{y}{(\outputp{x}{y} | @{y})} & \nonumber\\
	\red
	& (\outputp{x}{y} | @{y})\substn{\quotep{(\prefix{x}{y}{(@{y} | \outputp{x}{y})) | P}}}{y} & \nonumber\\
	=
	& \outputp{x}{\quotep{(\prefix{x}{y}{(\outputp{x}{y} | @{y})) | P}}}
	  | {(\prefix{x}{y}{(\outputp{x}{y} | @{y})) | P}} & \nonumber\\
	\red
	& \ldots & \nonumber\\
	\red^*
	& P | P | \ldots & \nonumber
\end{eqnarray}

Of course, this encoding, as an implementation, runs away, unfolding
$\bangp{P}$ eagerly. A lazier and more implementable replication
operator, restricted to input-guarded processes, may be obtained as follows.

\begin{eqnarray}
\bangp{\prefix{u}{v}{P}} 
	:= 
	\binpar{\lift{x}{\prefix{u}{v}{(\binpar{D(x)}{P})}}}{D(x)} \nonumber
\end{eqnarray}

\begin{remark}
  Note that the lazier definition still does not deal with summation
  or mixed summation (i.e. sums over input and output). The reader is
  invited to construct definitions of replication that deal with these
  features. 

  Further, the definitions are parameterized in a name, $x$. Can you,
  gentle reader, make a definition that eliminates this parameter and
  guarantees no accidental interaction between the replication
  machinery and the process being replicated -- i.e. no accidental
  sharing of names used by the process to get its work done and the
  name(s) used by the replication to effect copying. This latter
  revision of the definition of replication is crucial to obtaining
  the expected identity $!!P \sim !P$.
\end{remark}

\begin{remark}\label{rem:paradoxical_combinator}
  The reader familiar with the lambda calculus will have noticed the
  similarity between $D$ and the paradoxical combinator.

  [Ed. note: the existence of this seems to suggest we have to be more
  restrictive on the set of processes and names we admit if we are to
  support no-cloning.]
\end{remark}

\subsubsection{Bisimulation}

The computational dynamics gives rise to another kind of equivalence,
the equivalence of computational behavior. As previously mentioned
this is typically captured \emph{via} some form of bisimulation.

% The notion we use in this paper is weak barbed bisimulation
% \cite{milner91polyadicpi}.

The notion we use in this paper is derived from weak barbed
bisimulation \cite{milner91polyadicpi}. 

\begin{definition}
An \emph{observation relation}, $\downarrow_{\mathcal N}$, over a set
of names, $\mathcal N$, is the smallest relation satisfying the rules
below.

\infrule[Out-barb]{y \in {\mathcal N}, \; x \nameeq y}
		  {\outputp{x}{v} \downarrow_{\mathcal N} x}
\infrule[Par-barb]{\mbox{$P\downarrow_{\mathcal N} x$ or $Q\downarrow_{\mathcal N} x$}}
		  {\binpar{P}{Q} \downarrow_{\mathcal N} x}

We write $P \Downarrow_{\mathcal N} x$ if there is $Q$ such that 
$P \wred Q$ and $Q \downarrow_{\mathcal N} x$.
\end{definition}

\begin{definition}
%\label{def.bbisim}
An  ${\mathcal N}$-\emph{barbed bisimulation} over a set of names, ${\mathcal N}$, is a symmetric binary relation 
${\mathcal S}_{\mathcal N}$ between agents such that $P\rel{S}_{\mathcal N}Q$ implies:
\begin{enumerate}
\item If $P \red P'$ then $Q \wred Q'$ and $P'\rel{S}_{\mathcal N} Q'$.
\item If $P\downarrow_{\mathcal N} x$, then $Q\Downarrow_{\mathcal N} x$.
\end{enumerate}
$P$ is ${\mathcal N}$-barbed bisimilar to $Q$, written
$P \wbbisim_{\mathcal N} Q$, if $P \rel{S}_{\mathcal N} Q$ for some ${\mathcal N}$-barbed bisimulation ${\mathcal S}_{\mathcal N}$.
\end{definition}

$\mathcal{R} \subseteq \pi \times \pi$

$P \mathcal{R} Q => \forall P'. P \red P' \Rightarrow \exists Q'. Q \red Q', P' \mathcal{R} Q'$

$P \vdash x \Rightarrow Q \vdash x$

\begin{mathpar}
  \inferrule*[lab=Out-barb]{x \nameeq y}{{y}!\langle{Q}\rangle \vdash x}
  \and
  \inferrule*[lab=Par-barb]{\mbox{$P\vdash x$ or $Q\vdash x$}}{\binpar{P}{Q} \vdash x}
\end{mathpar}

\subsubsection{Contexts}

One of the principle advantages of computational calculi like the
$\pi$-calculus is a well-defined notion of context,
contextual-equivalence and a correlation between
contextual-equivalence and notions of bisimulation. The notion of
context allows the decomposition of a process into (sub-)process and
its syntactic environment, its context. Thus, a context may be
thought of as a process with a ``hole'' (written $\Box$) in it. The
application of a context $M$ to a process $P$, written $M[P]$, is
tantamount to filling the hole in $M$ with $P$. In this paper we do
not need the full weight of this theory, but do make use of the notion
of context in the proof the main theorem. 

\begin{mathpar}
  \inferrule* [lab=summation] {} {{M_{M},M_{N}} \bc \Box \;|\; x.M_{A} \;|\; M_{M}+M_{N}}
  \and
  \inferrule* [lab=agent] {} {{M_{A}} \bc (\vec{x})M_{P} \;| \; \clift{P_0,\ldots,M_{P},\ldots,P_N}}
  \and \\
  \inferrule* [lab=process] {} {{M_{P}} \bc M_{N} \;| \;P|M_{P} }
\end{mathpar} 

\begin{mathpar}
  \inferrule* [lab=sychronization] {} {M_{N} \bc \Box \;|\; x?M_{F} \;|\; x!M_{C}}
  \and
  \inferrule* [lab=abstraction] {} {{M_{F}} \bc (x)M_{P} }
  \and
  \inferrule* [lab=concretion] {} {{M_{C}} \bc \langle M_{P} \rangle }
  \and \\
  \inferrule* [lab=process] {} {{M_{P}} \bc M_{N} \;| \;P|M_{P} }
\end{mathpar}

\begin{definition}[contextual application] Given a context $M$, and
  process $P$, we define the \emph{contextual application}, $M[P] :=
  M\{P/\Box\}$. That is, the contextual application of M to P is the
  substitution of $P$ for $\Box$ in $M$.
\end{definition}

$\meaningof{-} : L \to \mathcal{P}(\pi)$

\begin{mathpar}
  \inferrule* [lab=collection] {} {\meaningof{true} = \pi, \and \meaningof{~E} = \pi \setminus \meaningof{E}, \and \meaningof{E_{1} \& E_{2}} = \meaningof{E_{1}} \cap \meaningof{E_{2}}}
\end{mathpar}

\begin{mathpar}
  \inferrule* [lab=structure] {} {\meaningof{0} = \{ P \in \pi | P \equiv 0 \}, \and \\ \meaningof{E_1 | E_2} = \{ P \in \pi | P \equiv P_{1} | P_{2}, P_{1} \in \meaningof{E_{1}}, P_{2} \in \meaningof{E_2}\} }
\end{mathpar}

\begin{mathpar}
 \inferrule* [lab=behavior] {} {\meaningof{\langle a?b \rangle E} = \{ P \in \pi | P \equiv Q | u?(y)P', \\ \and \\\\ \and \\ \;\;\; u \in \meaningof{a}, \forall z.P'\{z/y\} \in \meaningof{E\{z/b\}}\}, \and \\ \meaningof{a!E} = \{ P \in \pi | P \equiv Q | x!\langle P' \rangle, x \in \meaningof{a} P' \in \meaningof{E}\} }
\end{mathpar}

\begin{mathpar}
 \inferrule* [lab=nominal] {} {\meaningof{\quotep{E}} = \{ \quotep{P} \in \quotep{\pi} | P \in \meaningof{E} \}, \and \meaningof{\quotep{P}} = \{ \quotep{Q} \in \quotep{\pi} | P \equiv Q \} \and \\ \meaningof{@\quotep{E}} = \{ P \in \pi | P \equiv @x, x \in \meaningof{E} \}}
\end{mathpar}

\begin{eqnarray*}
  \\
  \meaningof{-} : TS \to ST
\end{eqnarray*}

\begin{eqnarray*}
  \\
  L : TS \to ST
\end{eqnarray*}

\begin{eqnarray*}
  \\
  P \models E \iff P \in \meaningof{E}
\end{eqnarray*}

\begin{eqnarray*}
  P \approx_{L} Q \iff \forall E \in L. P \models E \iff Q \models E
\end{eqnarray*}

\begin{eqnarray*}
  P \approx_{K} Q
\end{eqnarray*}

\begin{eqnarray*}
  P \approx Q
\end{eqnarray*}

$\approx_{K} = \approx = \approx_{L}$

\subsubsection{Contextual duality}

Note that contexts extend the quotation operation to a family of
operations from processes to names. Given a context, $M$, we can
define a \emph{nominal context}, $\quotep{M}$ by $\quotep{M}[P] :=
\quotep{M[P]}$. To foreshadow what is to come we observe that these
operations enjoy a duality with processes very much like the duality
between vectors and maps from vectors to scalars.

Further, because the calculus is essentially higher-order, we have a
correspondence between contexts and processes. More specifically,
given a name $x$ and a context $M$ we can construct $M^{*}_{x}$ such
that 

\begin{mathpar}
  M^{*}_{x} | \lift{x}{P} \red M[P]
\end{mathpar}

namely,

\begin{mathpar}
  M^{*}_{x} := x?(u).M[\dropn{u}]
\end{mathpar}

The dependence of $M^{*}_{x}$ on a name makes it an abstraction, 

\begin{mathpar}
  M^{*} := (x)x?(u).M[\dropn{u}]
\end{mathpar}

\subsection{Additional notation}

It will sometimes be convenient to denote the process a name
quotes. We already have the notation $x = \quotep{P}$, but it will be
convenient to introduce an alternate notation, $\procn{x}$, when we
want to emphasize the connection to the use of the name. Note that, by
virtue of name equivalence, $\quotep{\procn{x}} \nameeq x$; so, the
notation is consistent with previous definitions.

Further, because names have structure it is possible to effect
substitutions on the basis of that structure. This means we need to
upgrade our notation for substitutions, which we accomplish by
adapting comprehension notation. Thus,

\begin{mathpar}
  P\{ y / x : x \in S \}
\end{mathpar}

is interpreted to mean the process derived from P by replacing (in a
capture-avoiding manner) each occurrence of $x$ in $S$ by $y$. For example,

\begin{mathpar}
  P\{ \quotep{\procn{x}|\procn{x}} / x : x \in \freenames{P} \}
\end{mathpar}

will replace each (occurrence) of a free name $x$ in $P$ by
$\quotep{\procn{x}|\procn{x}}$.

Also, we will avail ourselves of the notation $x^{L}$ and $x^{R}$ to
denote injections of a name into disjoint copies of the name
space. There are numerous ways to accomplish this. One example can be
found in \cite{MeredithR05}. This notation overloads to vectors of
names: $\vec{x}^{\pi} := (x_{i}^{\pi} \; : \; 0 \leq i < |\vec{x}| )$ where $\pi \in \{L,R\}$.

We also use $P^{\Box} := P|\Box$.

In \cite{MeredithR05} an interpretation of the new operator is
given. It turns out that there are several possible interpretations
all enjoying the requisite algebraic properties of the operator (see
\cite{milner91polyadicpi}). We will therefore make liberal use of
$(\nu\; \vec{x})P$.

% subsection the_syntax_and_semantics_of_the_notation_system (end)   

\input{qm2pi.qmops} 

\input{qm2pi.sterngerlach} 

\input{qm2pi.metric} 

% section concurrent_process_calculi (end)

%\input{qm2pi.proofsketch}

% section proof sketch (end)

%\input{qm2pi.slviaknots} 

% section spatial logic via knots (end)

\input{qm2pi.conclusion}

% section conclusion (end)

%\input{qm2pi.dtcodes} 

% section wiring algorithm (end)

\input{qm2pi.ack} 

% section acknowledgments (end)

\newpage


\bibliographystyle{plain}   
\bibliography{../../biblios/main.bib}

\input{qm2pi.rhodetails}

\end{document}

 

%\documentclass[12pt]{llncs}
%\documentclass{jktr}

\usepackage[pdftex]{hyperref}                   
\usepackage {listings}
\usepackage {mathpartir}
\usepackage{bcprules}
%\usepackage{listings}
                       
\usepackage{graphicx} 
%\usepackage[margins=2.5cm,nohead,nofoot]{geometry}
%\usepackage{geometry}
\usepackage{amsfonts}
\usepackage{amstext}
\usepackage{latexsym}
\usepackage{amssymb}
\usepackage{color}


%\include{myPreamble}
\include{qm2pi.local} 

%\ifpdf
%\usepackage[pdftex]{graphicx}
%\else
%\usepackage{graphicx}
%\fi

 % \ifpdf
%  \usepackage{pdfsync}
%  \if


%\title{Brief Article}
%\author{David F. Snyder}
%\author{L.G. Meredith}

%\address{Dept. of Math., Texas State University--San Marcos, San Marcos, TX 78666}
       
\pagestyle{empty}


\begin{document}

\lstset{language=[Objective]Caml,frame=shadowbox}

\input{qm2pi.front}

% section front matter (end)

\input{qm2pi.intro} 
 
% section introduction (end)

% \input{qm2pi.knotations} 

% section notation (end)

\input{qm2pi.process.calculi} 

% section concurrent_process_calculi_and_spatial_logics_ (end)
    
%\input{qm2pi.knots2pi} 

%\input{qm2pi.trefoil} 

%\input{qm2pi.mainthm} 

% subsection basic_interpretation (end)

%\input{qm2pi.rho.presentation} 
\subsection{The syntax and semantics of the notation system}\label{sub:the_syntax_and_semantics_of_the_notation_system} % (fold)

We now summarize a technical presentation of the calculus that
embodies our theory of dynamics. The typical presentation of such a
calculus follows the style of giving generators and relations on
them. The grammar, below, describing term constructors, freely
generates the set of processes, $\Proc$. This set is then quotiented
by a relation known as structural congruence and it is over this set
that the notion of dynamics is expressed. This presentation is
essentially that of \cite{MeredithR05} with the addition of
polyadicity and summation. For readability we have relegated some of
the technical subtleties to an appendix.

\subsubsection{Process grammar}\label{subsub:process_grammar}

\begin{mathpar}
  \inferrule* [lab=synchronization] {} {{M} \bc \pzero \;|\; x?F \;|\; x!C }
  \and
  \inferrule* [lab=abstraction] {} {{F} \bc (x)P}
  \and
  \inferrule* [lab=concretion] {} {{C} \bc \langle Q \rangle}
  \and
  \inferrule* [lab=process] {} {{P,Q} \bc M \;| \;P|Q \;|\; @{x}}
  \and
  \inferrule* [lab=name] {} {{x} \bc \quotep{P}}
\end{mathpar} 

Note that $\vec{x}$ (resp. $\vec{P}$) denotes a vector of names
(resp. processes) of length $|\vec{x}|$ (resp. $|\vec{P}|$). We adopt
the following useful abbreviations.

\begin{mathpar}
   x?(\vec{y}).P := x.(\vec{y})P \and  x\clift{\vec{P}} := x.\clift{\vec{P}}
   \and x!(y) := \lift{x}{\dropn{y}}
   \and \Pi_{i=0}^{n-1}P_i := P_0 | \ldots | P_{n-1}
\end{mathpar}

\subsubsection{Structural congruence}

\paragraph{Free and bound names and alpha-equivalence.} At the
core of structural equivalence is alpha-equivalence which identifies
process that are the same up to a change of variable. Formally, we
recognize the distinction between free and bound names. The free names
of a process, $\freenames{P}$, may be calculated recursively as
follows:

\begin{mathpar}
\freenames{\pzero} := \emptyset
  \and \\
  \freenames{x?(y).P} := \{ x \} \cup (\freenames{P} \setminus \{ y \})
  \and 
  \freenames{x!\langle P \rangle} := \{ x \} \cup \{ P \} 
  \and \\
  \freenames{P|Q} := \freenames{P} \cup \freenames{Q}
  \and \\
  \freenames{@{x}} := \{ x \}
\end{mathpar}

$\pi$
$\quotep{\pi}$

$\freenames{-} : \pi \to \mathcal{P}(\quotep{\pi})$

\begin{eqnarray*}
  \freenames{\pzero} & := & \emptyset \\
  \freenames{x?(y).P} & := & \{ x \} \cup (\freenames{P} \setminus \{ y \}) \\
  \freenames{x!\langle P \rangle} & := & \{ x \} \cup \{ P \} \\
  \freenames{P|Q} & := & \freenames{P} \cup \freenames{Q} \\
  \freenames{\dropn{x}} & := & \{ x \}
\end{eqnarray*}

The bound names of a process, $\boundnames{P}$, are those names occurring in $P$
that are not free. For example, in $x?(y).0$, the name $x$ is free, while $y$ is bound.

\begin{mathpar}
  \inferrule* [lab=monoidal-laws] {} { P|Q \equiv Q|P \and P|0 \equiv P \and P|(Q|R) \equiv (P|Q)|R }
\end{mathpar}

\begin{mathpar}
  \inferrule* [lab=alpha-equivalence] {} { (x)P \equiv (y)P\{y/x\} \and y \not\in \freenames{P} }
\end{mathpar}

\begin{definition}
Then two processes, $P,Q$, are alpha-equivalent if $P = Q\{\vec{y}/\vec{x}\}$ for
some $\vec{x} \in \boundnames{Q},\vec{y} \in \boundnames{P}$, where $Q\{\vec{y}/\vec{x}\}$
denotes the capture-avoiding substitution of $\vec{y}$ for $\vec{x}$ in $Q$.
\end{definition}

\begin{definition}
  The {\em structural congruence} \cite{SangiorgiWalker} , $\equiv$,
  between processes is the least congruence containing
  alpha-equivalence, satisfying the abelian monoid laws
  (associativity, commutativity and $\pzero$ as identity) for parallel
  composition $|$ and for summation $+$.
\end{definition}

\subsection{Name equivalence}

We take name equivalence, written $\nameeq$, to be the smallest
equivalence relation generated by the following rules.

\begin{mathpar}
\inferrule*[lab=Quote-drop]
{ }
{ \quotep{@{x}} \nameeq x }

\inferrule*[lab=Struct-equiv]
{ P \scong Q }
{ \quotep{P} \nameeq \quotep{Q} }
\end{mathpar}

The astute reader will have noticed that the mutual recursion of names
and processes imposes a mutual recursion on alpha-equivalence and
structural equivalence via name-equivalence. Fortunately, all of this
works out pleasantly and we may calculate in the natural way, free of
concern. The reader interested in the details is referred to the
appendix \ref{appendix:rho_details}.

\subsection{Substitution}

We use $\Proc$ for the set of processes, $\QProc$ for the set of
names, and $\id{\{}\vec{y} / \vec{x} \id{\}}$ to denote partial maps,
$s : \QProc \rightarrow \QProc$. A map, $s$ lifts, uniquely, to a map
on process terms, $\widehat{s} : \Proc \rightarrow \Proc$ by the
following equations.

\begin{mathpar}
  (0) \psubstp{Q}{P} := 0 \\
  (R \juxtap S) \psubstp{Q}{P}
  :=    
  (R)\psubstp{Q}{P} \juxtap (S) \psubstp{Q}{P} \\
  (x?(y).R) \psubstp{Q}{P}    
  :=    
  (x)\substp{Q}{P} (z)\concat( (R \psubstn{z}{y}) \psubstp{Q}{P} ) \\
  (\lift{x}{R}) \psubstp{Q}{P}  
  :=
  \lift{(x)\substp{Q}{P}}{ R \psubstp{Q}{P} } \\
%   (\dropn{x})  \psubstp{Q}{P}       
%   := 
%   \left\{ 
%     \begin{array}{ccc} 
%       \dropn{\quotep{Q}} & & x \nameeq \quotep{P} \\
%       \dropn{x} & & otherwise \\
%     \end{array}
%   \right. 
  (\dropn{x})  \psubstp{Q}{P}       
  := 
  \left\{ 
    \begin{array}{ccc} 
      Q & & x \nameeq \quotep{P} \\
      \dropn{x} & & otherwise \\
    \end{array}
  \right.
\end{mathpar}
 

where

\begin{eqnarray}
  (x)\id{\{} \lpquote Q \rpquote / \lpquote P \rpquote \id{\}}            = 
  \left\{ 
    \begin{array}{ccc}
      \lpquote Q \rpquote & & x \nameeq \lpquote P \rpquote \\
      x & & otherwise \\
    \end{array}
  \right. \nonumber
\end{eqnarray}

and $z$ is chosen distinct from $\quotep{P}$, $\quotep{Q}$, the free
names in $Q$, and all the names in $R$. Our $\alpha$-equivalence will
be built in the standard way from this substitution.

\begin{remark}\label{rem:no_self_referential_names}
  One consequence of these definitions is that $\forall P. \quotep{P}
  \not\in \freenames{P}$.
\end{remark}

\subsection{ Dynamic quote: an example }

Anticipating something of what's to come, consider applying the
substitution, $\widehat{\id{\{}u / z \id{\}}}$, to the following pair
of processes, $\lift{w}{y!(z)}$ and $w[ \lpquote y!(z) \rpquote ]$.

\begin{eqnarray}
	\lift{w}{y!(z)}\widehat{\id{\{}u / z \id{\}}}
		& = &
		\lift{w}{y!(u)} \nonumber\\
	w[ \lpquote y!(z) \rpquote ] \widehat{ \id{\{}u / z \id{\}} }
		& = &
		w[ \lpquote y!(z) \rpquote ] \nonumber
\end{eqnarray}

Because the body of the process between quotes is impervious to
substitution, we get radically different answers. In fact, by
examining the first process in an input context,
e.g. $x?(z).\lift{w}{y!(z)}$, we see that the process under the lift
operator may be shaped by prefixed inputs binding a name inside it. In
this sense, the lift operator will be seen as a way to dynamically
construct processes before reifying them as names.

Finally equipped with these standard features we can present the
dynamics of the calculus.

\subsubsection{Operational semantics} 

Finally, we introduce the computational dynamics. What marks these
algebras as distinct from other more traditionally studied algebraic
structures, e.g. vector spaces or polynomial rings, is the manner in
which dynamics is captured. In traditional structures, dynamics is typically
expressed through morphisms between such structures, as in linear maps
between vector spaces or morphisms between rings. In algebras
associated with the semantics of computation, the dynamics is
expressed as part of the algebraic structure itself, through a
reduction reduction relation typically denoted by $\red$. Below, we
give a recursive presentation of this relation for the calculus used
in the encoding.

$\red \subseteq \pi \times \pi$
$\red : \pi \to \mathcal{P}(\pi)$

\begin{mathpar}
  \inferrule* [lab=Comm] { \textsf{match}( x_{src}, x_{trgt} ) } { x_{trgt}?(y)P \; | \; x_{src}!\langle {Q} \rangle \red P\{\quotep{Q}/y}\} }
  \and \\
  \inferrule* [lab=Par] {{P} \red {P}'} {{{P} | {Q}} \red {{P}' | {Q}}}
  \and
  \inferrule* [lab=Equiv]{{{P} \scong {P}'} \andalso {{P}' \red {Q}'} \andalso {{Q}' \scong {Q}}}{{P} \red {Q}}
\end{mathpar}

\begin{eqnarray*}
  match_{\equiv} (\quotep{P},\quotep{Q}) & := & P \equiv Q \\
  match_{\dagger}(\quotep{P},\quotep{Q}) & := & \forall R. P|Q \red^{*} R => R \red^{*} 0 \\
  match_{K}(\quotep{P},\quotep{Q}) & := & K \mbox{ for some context } K
\end{eqnarray*}

$u?(x)P | u!\langle Q \rangle \red P\{\quotep{Q}/x\}$

%We write $\wred$ for $\red^*$, and $P\red$ if $\exists Q $ such that $ P \red Q$.
We write $P\red$ if $\exists Q $ such that $ P \red Q$ and $P\not\red$, otherwise.

\section{Replication}

As mentioned before, it is known that replication (and hence
recursion) can be implemented in a higher-order process algebra
\cite{SangiorgiWalker}. As our first example of calculation with the
machinery thus far presented we give the construction explicitly in
the {\rhoc}.

\begin{eqnarray}
	D_{x} & := & \prefix{x}{y}{(\binpar{\outputp{x}{y}}{@{y}})} \nonumber\\
	\bangp_{x}{P} & := & \binpar{{x}!\langle{\binpar{D_{x}}{P}}\rangle}{D_{x}} \nonumber
\end{eqnarray}

\begin{eqnarray}
	\bangp_{x}{P} & & \nonumber\\
	=
	& {x}!\langle{(\prefix{x}{y}{(\outputp{x}{y} | @{y})) | P}}\rangle 
	      | \prefix{x}{y}{(\outputp{x}{y} | @{y})} & \nonumber\\
	\red
	& (\outputp{x}{y} | @{y})\substn{\quotep{(\prefix{x}{y}{(@{y} | \outputp{x}{y})) | P}}}{y} & \nonumber\\
	=
	& \outputp{x}{\quotep{(\prefix{x}{y}{(\outputp{x}{y} | @{y})) | P}}}
	  | {(\prefix{x}{y}{(\outputp{x}{y} | @{y})) | P}} & \nonumber\\
	\red
	& \ldots & \nonumber\\
	\red^*
	& P | P | \ldots & \nonumber
\end{eqnarray}

Of course, this encoding, as an implementation, runs away, unfolding
$\bangp{P}$ eagerly. A lazier and more implementable replication
operator, restricted to input-guarded processes, may be obtained as follows.

\begin{eqnarray}
\bangp{\prefix{u}{v}{P}} 
	:= 
	\binpar{\lift{x}{\prefix{u}{v}{(\binpar{D(x)}{P})}}}{D(x)} \nonumber
\end{eqnarray}

\begin{remark}
  Note that the lazier definition still does not deal with summation
  or mixed summation (i.e. sums over input and output). The reader is
  invited to construct definitions of replication that deal with these
  features. 

  Further, the definitions are parameterized in a name, $x$. Can you,
  gentle reader, make a definition that eliminates this parameter and
  guarantees no accidental interaction between the replication
  machinery and the process being replicated -- i.e. no accidental
  sharing of names used by the process to get its work done and the
  name(s) used by the replication to effect copying. This latter
  revision of the definition of replication is crucial to obtaining
  the expected identity $!!P \sim !P$.
\end{remark}

\begin{remark}\label{rem:paradoxical_combinator}
  The reader familiar with the lambda calculus will have noticed the
  similarity between $D$ and the paradoxical combinator.

  [Ed. note: the existence of this seems to suggest we have to be more
  restrictive on the set of processes and names we admit if we are to
  support no-cloning.]
\end{remark}

\subsubsection{Bisimulation}

The computational dynamics gives rise to another kind of equivalence,
the equivalence of computational behavior. As previously mentioned
this is typically captured \emph{via} some form of bisimulation.

% The notion we use in this paper is weak barbed bisimulation
% \cite{milner91polyadicpi}.

The notion we use in this paper is derived from weak barbed
bisimulation \cite{milner91polyadicpi}. 

\begin{definition}
An \emph{observation relation}, $\downarrow_{\mathcal N}$, over a set
of names, $\mathcal N$, is the smallest relation satisfying the rules
below.

\infrule[Out-barb]{y \in {\mathcal N}, \; x \nameeq y}
		  {\outputp{x}{v} \downarrow_{\mathcal N} x}
\infrule[Par-barb]{\mbox{$P\downarrow_{\mathcal N} x$ or $Q\downarrow_{\mathcal N} x$}}
		  {\binpar{P}{Q} \downarrow_{\mathcal N} x}

We write $P \Downarrow_{\mathcal N} x$ if there is $Q$ such that 
$P \wred Q$ and $Q \downarrow_{\mathcal N} x$.
\end{definition}

\begin{definition}
%\label{def.bbisim}
An  ${\mathcal N}$-\emph{barbed bisimulation} over a set of names, ${\mathcal N}$, is a symmetric binary relation 
${\mathcal S}_{\mathcal N}$ between agents such that $P\rel{S}_{\mathcal N}Q$ implies:
\begin{enumerate}
\item If $P \red P'$ then $Q \wred Q'$ and $P'\rel{S}_{\mathcal N} Q'$.
\item If $P\downarrow_{\mathcal N} x$, then $Q\Downarrow_{\mathcal N} x$.
\end{enumerate}
$P$ is ${\mathcal N}$-barbed bisimilar to $Q$, written
$P \wbbisim_{\mathcal N} Q$, if $P \rel{S}_{\mathcal N} Q$ for some ${\mathcal N}$-barbed bisimulation ${\mathcal S}_{\mathcal N}$.
\end{definition}

$\mathcal{R} \subseteq \pi \times \pi$

$P \mathcal{R} Q => \forall P'. P \red P' \Rightarrow \exists Q'. Q \red Q', P' \mathcal{R} Q'$

$P \vdash x \Rightarrow Q \vdash x$

\begin{mathpar}
  \inferrule*[lab=Out-barb]{x \nameeq y}{{y}!\langle{Q}\rangle \vdash x}
  \and
  \inferrule*[lab=Par-barb]{\mbox{$P\vdash x$ or $Q\vdash x$}}{\binpar{P}{Q} \vdash x}
\end{mathpar}

\subsubsection{Contexts}

One of the principle advantages of computational calculi like the
$\pi$-calculus is a well-defined notion of context,
contextual-equivalence and a correlation between
contextual-equivalence and notions of bisimulation. The notion of
context allows the decomposition of a process into (sub-)process and
its syntactic environment, its context. Thus, a context may be
thought of as a process with a ``hole'' (written $\Box$) in it. The
application of a context $M$ to a process $P$, written $M[P]$, is
tantamount to filling the hole in $M$ with $P$. In this paper we do
not need the full weight of this theory, but do make use of the notion
of context in the proof the main theorem. 

\begin{mathpar}
  \inferrule* [lab=summation] {} {{M_{M},M_{N}} \bc \Box \;|\; x.M_{A} \;|\; M_{M}+M_{N}}
  \and
  \inferrule* [lab=agent] {} {{M_{A}} \bc (\vec{x})M_{P} \;| \; \clift{P_0,\ldots,M_{P},\ldots,P_N}}
  \and \\
  \inferrule* [lab=process] {} {{M_{P}} \bc M_{N} \;| \;P|M_{P} }
\end{mathpar} 

\begin{mathpar}
  \inferrule* [lab=sychronization] {} {M_{N} \bc \Box \;|\; x?M_{F} \;|\; x!M_{C}}
  \and
  \inferrule* [lab=abstraction] {} {{M_{F}} \bc (x)M_{P} }
  \and
  \inferrule* [lab=concretion] {} {{M_{C}} \bc \langle M_{P} \rangle }
  \and \\
  \inferrule* [lab=process] {} {{M_{P}} \bc M_{N} \;| \;P|M_{P} }
\end{mathpar}

\begin{definition}[contextual application] Given a context $M$, and
  process $P$, we define the \emph{contextual application}, $M[P] :=
  M\{P/\Box\}$. That is, the contextual application of M to P is the
  substitution of $P$ for $\Box$ in $M$.
\end{definition}

$\meaningof{-} : L \to \mathcal{P}(\pi)$

\begin{mathpar}
  \inferrule* [lab=collection] {} {\meaningof{true} = \pi, \and \meaningof{~E} = \pi \setminus \meaningof{E}, \and \meaningof{E_{1} \& E_{2}} = \meaningof{E_{1}} \cap \meaningof{E_{2}}}
\end{mathpar}

\begin{mathpar}
  \inferrule* [lab=structure] {} {\meaningof{0} = \{ P \in \pi | P \equiv 0 \}, \and \\ \meaningof{E_1 | E_2} = \{ P \in \pi | P \equiv P_{1} | P_{2}, P_{1} \in \meaningof{E_{1}}, P_{2} \in \meaningof{E_2}\} }
\end{mathpar}

\begin{mathpar}
 \inferrule* [lab=behavior] {} {\meaningof{\langle a?b \rangle E} = \{ P \in \pi | P \equiv Q | u?(y)P', \\ \and \\\\ \and \\ \;\;\; u \in \meaningof{a}, \forall z.P'\{z/y\} \in \meaningof{E\{z/b\}}\}, \and \\ \meaningof{a!E} = \{ P \in \pi | P \equiv Q | x!\langle P' \rangle, x \in \meaningof{a} P' \in \meaningof{E}\} }
\end{mathpar}

\begin{mathpar}
 \inferrule* [lab=nominal] {} {\meaningof{\quotep{E}} = \{ \quotep{P} \in \quotep{\pi} | P \in \meaningof{E} \}, \and \meaningof{\quotep{P}} = \{ \quotep{Q} \in \quotep{\pi} | P \equiv Q \} \and \\ \meaningof{@\quotep{E}} = \{ P \in \pi | P \equiv @x, x \in \meaningof{E} \}}
\end{mathpar}

\begin{eqnarray*}
  \\
  \meaningof{-} : TS \to ST
\end{eqnarray*}

\begin{eqnarray*}
  \\
  L : TS \to ST
\end{eqnarray*}

\begin{eqnarray*}
  \\
  P \models E \iff P \in \meaningof{E}
\end{eqnarray*}

\begin{eqnarray*}
  P \approx_{L} Q \iff \forall E \in L. P \models E \iff Q \models E
\end{eqnarray*}

\begin{eqnarray*}
  P \approx_{K} Q
\end{eqnarray*}

\begin{eqnarray*}
  P \approx Q
\end{eqnarray*}

$\approx_{K} = \approx = \approx_{L}$

\subsubsection{Contextual duality}

Note that contexts extend the quotation operation to a family of
operations from processes to names. Given a context, $M$, we can
define a \emph{nominal context}, $\quotep{M}$ by $\quotep{M}[P] :=
\quotep{M[P]}$. To foreshadow what is to come we observe that these
operations enjoy a duality with processes very much like the duality
between vectors and maps from vectors to scalars.

Further, because the calculus is essentially higher-order, we have a
correspondence between contexts and processes. More specifically,
given a name $x$ and a context $M$ we can construct $M^{*}_{x}$ such
that 

\begin{mathpar}
  M^{*}_{x} | \lift{x}{P} \red M[P]
\end{mathpar}

namely,

\begin{mathpar}
  M^{*}_{x} := x?(u).M[\dropn{u}]
\end{mathpar}

The dependence of $M^{*}_{x}$ on a name makes it an abstraction, 

\begin{mathpar}
  M^{*} := (x)x?(u).M[\dropn{u}]
\end{mathpar}

\subsection{Additional notation}

It will sometimes be convenient to denote the process a name
quotes. We already have the notation $x = \quotep{P}$, but it will be
convenient to introduce an alternate notation, $\procn{x}$, when we
want to emphasize the connection to the use of the name. Note that, by
virtue of name equivalence, $\quotep{\procn{x}} \nameeq x$; so, the
notation is consistent with previous definitions.

Further, because names have structure it is possible to effect
substitutions on the basis of that structure. This means we need to
upgrade our notation for substitutions, which we accomplish by
adapting comprehension notation. Thus,

\begin{mathpar}
  P\{ y / x : x \in S \}
\end{mathpar}

is interpreted to mean the process derived from P by replacing (in a
capture-avoiding manner) each occurrence of $x$ in $S$ by $y$. For example,

\begin{mathpar}
  P\{ \quotep{\procn{x}|\procn{x}} / x : x \in \freenames{P} \}
\end{mathpar}

will replace each (occurrence) of a free name $x$ in $P$ by
$\quotep{\procn{x}|\procn{x}}$.

Also, we will avail ourselves of the notation $x^{L}$ and $x^{R}$ to
denote injections of a name into disjoint copies of the name
space. There are numerous ways to accomplish this. One example can be
found in \cite{MeredithR05}. This notation overloads to vectors of
names: $\vec{x}^{\pi} := (x_{i}^{\pi} \; : \; 0 \leq i < |\vec{x}| )$ where $\pi \in \{L,R\}$.

We also use $P^{\Box} := P|\Box$.

In \cite{MeredithR05} an interpretation of the new operator is
given. It turns out that there are several possible interpretations
all enjoying the requisite algebraic properties of the operator (see
\cite{milner91polyadicpi}). We will therefore make liberal use of
$(\nu\; \vec{x})P$.

% subsection the_syntax_and_semantics_of_the_notation_system (end)   

\input{qm2pi.qmops} 

\input{qm2pi.sterngerlach} 

\input{qm2pi.metric} 

% section concurrent_process_calculi (end)

%\input{qm2pi.proofsketch}

% section proof sketch (end)

%\input{qm2pi.slviaknots} 

% section spatial logic via knots (end)

\input{qm2pi.conclusion}

% section conclusion (end)

%\input{qm2pi.dtcodes} 

% section wiring algorithm (end)

\input{qm2pi.ack} 

% section acknowledgments (end)

\newpage


\bibliographystyle{plain}   
\bibliography{../../biblios/main.bib}

\input{qm2pi.rhodetails}

\end{document}

 

%\documentclass[12pt]{llncs}
%\documentclass{jktr}

\usepackage[pdftex]{hyperref}                   
\usepackage {listings}
\usepackage {mathpartir}
\usepackage{bcprules}
%\usepackage{listings}
                       
\usepackage{graphicx} 
%\usepackage[margins=2.5cm,nohead,nofoot]{geometry}
%\usepackage{geometry}
\usepackage{amsfonts}
\usepackage{amstext}
\usepackage{latexsym}
\usepackage{amssymb}
\usepackage{color}


%\include{myPreamble}
\include{qm2pi.local} 

%\ifpdf
%\usepackage[pdftex]{graphicx}
%\else
%\usepackage{graphicx}
%\fi

 % \ifpdf
%  \usepackage{pdfsync}
%  \if


%\title{Brief Article}
%\author{David F. Snyder}
%\author{L.G. Meredith}

%\address{Dept. of Math., Texas State University--San Marcos, San Marcos, TX 78666}
       
\pagestyle{empty}


\begin{document}

\lstset{language=[Objective]Caml,frame=shadowbox}

\input{qm2pi.front}

% section front matter (end)

\input{qm2pi.intro} 
 
% section introduction (end)

% \input{qm2pi.knotations} 

% section notation (end)

\input{qm2pi.process.calculi} 

% section concurrent_process_calculi_and_spatial_logics_ (end)
    
%\input{qm2pi.knots2pi} 

%\input{qm2pi.trefoil} 

%\input{qm2pi.mainthm} 

% subsection basic_interpretation (end)

%\input{qm2pi.rho.presentation} 
\subsection{The syntax and semantics of the notation system}\label{sub:the_syntax_and_semantics_of_the_notation_system} % (fold)

We now summarize a technical presentation of the calculus that
embodies our theory of dynamics. The typical presentation of such a
calculus follows the style of giving generators and relations on
them. The grammar, below, describing term constructors, freely
generates the set of processes, $\Proc$. This set is then quotiented
by a relation known as structural congruence and it is over this set
that the notion of dynamics is expressed. This presentation is
essentially that of \cite{MeredithR05} with the addition of
polyadicity and summation. For readability we have relegated some of
the technical subtleties to an appendix.

\subsubsection{Process grammar}\label{subsub:process_grammar}

\begin{mathpar}
  \inferrule* [lab=synchronization] {} {{M} \bc \pzero \;|\; x?F \;|\; x!C }
  \and
  \inferrule* [lab=abstraction] {} {{F} \bc (x)P}
  \and
  \inferrule* [lab=concretion] {} {{C} \bc \langle Q \rangle}
  \and
  \inferrule* [lab=process] {} {{P,Q} \bc M \;| \;P|Q \;|\; @{x}}
  \and
  \inferrule* [lab=name] {} {{x} \bc \quotep{P}}
\end{mathpar} 

Note that $\vec{x}$ (resp. $\vec{P}$) denotes a vector of names
(resp. processes) of length $|\vec{x}|$ (resp. $|\vec{P}|$). We adopt
the following useful abbreviations.

\begin{mathpar}
   x?(\vec{y}).P := x.(\vec{y})P \and  x\clift{\vec{P}} := x.\clift{\vec{P}}
   \and x!(y) := \lift{x}{\dropn{y}}
   \and \Pi_{i=0}^{n-1}P_i := P_0 | \ldots | P_{n-1}
\end{mathpar}

\subsubsection{Structural congruence}

\paragraph{Free and bound names and alpha-equivalence.} At the
core of structural equivalence is alpha-equivalence which identifies
process that are the same up to a change of variable. Formally, we
recognize the distinction between free and bound names. The free names
of a process, $\freenames{P}$, may be calculated recursively as
follows:

\begin{mathpar}
\freenames{\pzero} := \emptyset
  \and \\
  \freenames{x?(y).P} := \{ x \} \cup (\freenames{P} \setminus \{ y \})
  \and 
  \freenames{x!\langle P \rangle} := \{ x \} \cup \{ P \} 
  \and \\
  \freenames{P|Q} := \freenames{P} \cup \freenames{Q}
  \and \\
  \freenames{@{x}} := \{ x \}
\end{mathpar}

$\pi$
$\quotep{\pi}$

$\freenames{-} : \pi \to \mathcal{P}(\quotep{\pi})$

\begin{eqnarray*}
  \freenames{\pzero} & := & \emptyset \\
  \freenames{x?(y).P} & := & \{ x \} \cup (\freenames{P} \setminus \{ y \}) \\
  \freenames{x!\langle P \rangle} & := & \{ x \} \cup \{ P \} \\
  \freenames{P|Q} & := & \freenames{P} \cup \freenames{Q} \\
  \freenames{\dropn{x}} & := & \{ x \}
\end{eqnarray*}

The bound names of a process, $\boundnames{P}$, are those names occurring in $P$
that are not free. For example, in $x?(y).0$, the name $x$ is free, while $y$ is bound.

\begin{mathpar}
  \inferrule* [lab=monoidal-laws] {} { P|Q \equiv Q|P \and P|0 \equiv P \and P|(Q|R) \equiv (P|Q)|R }
\end{mathpar}

\begin{mathpar}
  \inferrule* [lab=alpha-equivalence] {} { (x)P \equiv (y)P\{y/x\} \and y \not\in \freenames{P} }
\end{mathpar}

\begin{definition}
Then two processes, $P,Q$, are alpha-equivalent if $P = Q\{\vec{y}/\vec{x}\}$ for
some $\vec{x} \in \boundnames{Q},\vec{y} \in \boundnames{P}$, where $Q\{\vec{y}/\vec{x}\}$
denotes the capture-avoiding substitution of $\vec{y}$ for $\vec{x}$ in $Q$.
\end{definition}

\begin{definition}
  The {\em structural congruence} \cite{SangiorgiWalker} , $\equiv$,
  between processes is the least congruence containing
  alpha-equivalence, satisfying the abelian monoid laws
  (associativity, commutativity and $\pzero$ as identity) for parallel
  composition $|$ and for summation $+$.
\end{definition}

\subsection{Name equivalence}

We take name equivalence, written $\nameeq$, to be the smallest
equivalence relation generated by the following rules.

\begin{mathpar}
\inferrule*[lab=Quote-drop]
{ }
{ \quotep{@{x}} \nameeq x }

\inferrule*[lab=Struct-equiv]
{ P \scong Q }
{ \quotep{P} \nameeq \quotep{Q} }
\end{mathpar}

The astute reader will have noticed that the mutual recursion of names
and processes imposes a mutual recursion on alpha-equivalence and
structural equivalence via name-equivalence. Fortunately, all of this
works out pleasantly and we may calculate in the natural way, free of
concern. The reader interested in the details is referred to the
appendix \ref{appendix:rho_details}.

\subsection{Substitution}

We use $\Proc$ for the set of processes, $\QProc$ for the set of
names, and $\id{\{}\vec{y} / \vec{x} \id{\}}$ to denote partial maps,
$s : \QProc \rightarrow \QProc$. A map, $s$ lifts, uniquely, to a map
on process terms, $\widehat{s} : \Proc \rightarrow \Proc$ by the
following equations.

\begin{mathpar}
  (0) \psubstp{Q}{P} := 0 \\
  (R \juxtap S) \psubstp{Q}{P}
  :=    
  (R)\psubstp{Q}{P} \juxtap (S) \psubstp{Q}{P} \\
  (x?(y).R) \psubstp{Q}{P}    
  :=    
  (x)\substp{Q}{P} (z)\concat( (R \psubstn{z}{y}) \psubstp{Q}{P} ) \\
  (\lift{x}{R}) \psubstp{Q}{P}  
  :=
  \lift{(x)\substp{Q}{P}}{ R \psubstp{Q}{P} } \\
%   (\dropn{x})  \psubstp{Q}{P}       
%   := 
%   \left\{ 
%     \begin{array}{ccc} 
%       \dropn{\quotep{Q}} & & x \nameeq \quotep{P} \\
%       \dropn{x} & & otherwise \\
%     \end{array}
%   \right. 
  (\dropn{x})  \psubstp{Q}{P}       
  := 
  \left\{ 
    \begin{array}{ccc} 
      Q & & x \nameeq \quotep{P} \\
      \dropn{x} & & otherwise \\
    \end{array}
  \right.
\end{mathpar}
 

where

\begin{eqnarray}
  (x)\id{\{} \lpquote Q \rpquote / \lpquote P \rpquote \id{\}}            = 
  \left\{ 
    \begin{array}{ccc}
      \lpquote Q \rpquote & & x \nameeq \lpquote P \rpquote \\
      x & & otherwise \\
    \end{array}
  \right. \nonumber
\end{eqnarray}

and $z$ is chosen distinct from $\quotep{P}$, $\quotep{Q}$, the free
names in $Q$, and all the names in $R$. Our $\alpha$-equivalence will
be built in the standard way from this substitution.

\begin{remark}\label{rem:no_self_referential_names}
  One consequence of these definitions is that $\forall P. \quotep{P}
  \not\in \freenames{P}$.
\end{remark}

\subsection{ Dynamic quote: an example }

Anticipating something of what's to come, consider applying the
substitution, $\widehat{\id{\{}u / z \id{\}}}$, to the following pair
of processes, $\lift{w}{y!(z)}$ and $w[ \lpquote y!(z) \rpquote ]$.

\begin{eqnarray}
	\lift{w}{y!(z)}\widehat{\id{\{}u / z \id{\}}}
		& = &
		\lift{w}{y!(u)} \nonumber\\
	w[ \lpquote y!(z) \rpquote ] \widehat{ \id{\{}u / z \id{\}} }
		& = &
		w[ \lpquote y!(z) \rpquote ] \nonumber
\end{eqnarray}

Because the body of the process between quotes is impervious to
substitution, we get radically different answers. In fact, by
examining the first process in an input context,
e.g. $x?(z).\lift{w}{y!(z)}$, we see that the process under the lift
operator may be shaped by prefixed inputs binding a name inside it. In
this sense, the lift operator will be seen as a way to dynamically
construct processes before reifying them as names.

Finally equipped with these standard features we can present the
dynamics of the calculus.

\subsubsection{Operational semantics} 

Finally, we introduce the computational dynamics. What marks these
algebras as distinct from other more traditionally studied algebraic
structures, e.g. vector spaces or polynomial rings, is the manner in
which dynamics is captured. In traditional structures, dynamics is typically
expressed through morphisms between such structures, as in linear maps
between vector spaces or morphisms between rings. In algebras
associated with the semantics of computation, the dynamics is
expressed as part of the algebraic structure itself, through a
reduction reduction relation typically denoted by $\red$. Below, we
give a recursive presentation of this relation for the calculus used
in the encoding.

$\red \subseteq \pi \times \pi$
$\red : \pi \to \mathcal{P}(\pi)$

\begin{mathpar}
  \inferrule* [lab=Comm] { \textsf{match}( x_{src}, x_{trgt} ) } { x_{trgt}?(y)P \; | \; x_{src}!\langle {Q} \rangle \red P\{\quotep{Q}/y}\} }
  \and \\
  \inferrule* [lab=Par] {{P} \red {P}'} {{{P} | {Q}} \red {{P}' | {Q}}}
  \and
  \inferrule* [lab=Equiv]{{{P} \scong {P}'} \andalso {{P}' \red {Q}'} \andalso {{Q}' \scong {Q}}}{{P} \red {Q}}
\end{mathpar}

\begin{eqnarray*}
  match_{\equiv} (\quotep{P},\quotep{Q}) & := & P \equiv Q \\
  match_{\dagger}(\quotep{P},\quotep{Q}) & := & \forall R. P|Q \red^{*} R => R \red^{*} 0 \\
  match_{K}(\quotep{P},\quotep{Q}) & := & K \mbox{ for some context } K
\end{eqnarray*}

$u?(x)P | u!\langle Q \rangle \red P\{\quotep{Q}/x\}$

%We write $\wred$ for $\red^*$, and $P\red$ if $\exists Q $ such that $ P \red Q$.
We write $P\red$ if $\exists Q $ such that $ P \red Q$ and $P\not\red$, otherwise.

\section{Replication}

As mentioned before, it is known that replication (and hence
recursion) can be implemented in a higher-order process algebra
\cite{SangiorgiWalker}. As our first example of calculation with the
machinery thus far presented we give the construction explicitly in
the {\rhoc}.

\begin{eqnarray}
	D_{x} & := & \prefix{x}{y}{(\binpar{\outputp{x}{y}}{@{y}})} \nonumber\\
	\bangp_{x}{P} & := & \binpar{{x}!\langle{\binpar{D_{x}}{P}}\rangle}{D_{x}} \nonumber
\end{eqnarray}

\begin{eqnarray}
	\bangp_{x}{P} & & \nonumber\\
	=
	& {x}!\langle{(\prefix{x}{y}{(\outputp{x}{y} | @{y})) | P}}\rangle 
	      | \prefix{x}{y}{(\outputp{x}{y} | @{y})} & \nonumber\\
	\red
	& (\outputp{x}{y} | @{y})\substn{\quotep{(\prefix{x}{y}{(@{y} | \outputp{x}{y})) | P}}}{y} & \nonumber\\
	=
	& \outputp{x}{\quotep{(\prefix{x}{y}{(\outputp{x}{y} | @{y})) | P}}}
	  | {(\prefix{x}{y}{(\outputp{x}{y} | @{y})) | P}} & \nonumber\\
	\red
	& \ldots & \nonumber\\
	\red^*
	& P | P | \ldots & \nonumber
\end{eqnarray}

Of course, this encoding, as an implementation, runs away, unfolding
$\bangp{P}$ eagerly. A lazier and more implementable replication
operator, restricted to input-guarded processes, may be obtained as follows.

\begin{eqnarray}
\bangp{\prefix{u}{v}{P}} 
	:= 
	\binpar{\lift{x}{\prefix{u}{v}{(\binpar{D(x)}{P})}}}{D(x)} \nonumber
\end{eqnarray}

\begin{remark}
  Note that the lazier definition still does not deal with summation
  or mixed summation (i.e. sums over input and output). The reader is
  invited to construct definitions of replication that deal with these
  features. 

  Further, the definitions are parameterized in a name, $x$. Can you,
  gentle reader, make a definition that eliminates this parameter and
  guarantees no accidental interaction between the replication
  machinery and the process being replicated -- i.e. no accidental
  sharing of names used by the process to get its work done and the
  name(s) used by the replication to effect copying. This latter
  revision of the definition of replication is crucial to obtaining
  the expected identity $!!P \sim !P$.
\end{remark}

\begin{remark}\label{rem:paradoxical_combinator}
  The reader familiar with the lambda calculus will have noticed the
  similarity between $D$ and the paradoxical combinator.

  [Ed. note: the existence of this seems to suggest we have to be more
  restrictive on the set of processes and names we admit if we are to
  support no-cloning.]
\end{remark}

\subsubsection{Bisimulation}

The computational dynamics gives rise to another kind of equivalence,
the equivalence of computational behavior. As previously mentioned
this is typically captured \emph{via} some form of bisimulation.

% The notion we use in this paper is weak barbed bisimulation
% \cite{milner91polyadicpi}.

The notion we use in this paper is derived from weak barbed
bisimulation \cite{milner91polyadicpi}. 

\begin{definition}
An \emph{observation relation}, $\downarrow_{\mathcal N}$, over a set
of names, $\mathcal N$, is the smallest relation satisfying the rules
below.

\infrule[Out-barb]{y \in {\mathcal N}, \; x \nameeq y}
		  {\outputp{x}{v} \downarrow_{\mathcal N} x}
\infrule[Par-barb]{\mbox{$P\downarrow_{\mathcal N} x$ or $Q\downarrow_{\mathcal N} x$}}
		  {\binpar{P}{Q} \downarrow_{\mathcal N} x}

We write $P \Downarrow_{\mathcal N} x$ if there is $Q$ such that 
$P \wred Q$ and $Q \downarrow_{\mathcal N} x$.
\end{definition}

\begin{definition}
%\label{def.bbisim}
An  ${\mathcal N}$-\emph{barbed bisimulation} over a set of names, ${\mathcal N}$, is a symmetric binary relation 
${\mathcal S}_{\mathcal N}$ between agents such that $P\rel{S}_{\mathcal N}Q$ implies:
\begin{enumerate}
\item If $P \red P'$ then $Q \wred Q'$ and $P'\rel{S}_{\mathcal N} Q'$.
\item If $P\downarrow_{\mathcal N} x$, then $Q\Downarrow_{\mathcal N} x$.
\end{enumerate}
$P$ is ${\mathcal N}$-barbed bisimilar to $Q$, written
$P \wbbisim_{\mathcal N} Q$, if $P \rel{S}_{\mathcal N} Q$ for some ${\mathcal N}$-barbed bisimulation ${\mathcal S}_{\mathcal N}$.
\end{definition}

$\mathcal{R} \subseteq \pi \times \pi$

$P \mathcal{R} Q => \forall P'. P \red P' \Rightarrow \exists Q'. Q \red Q', P' \mathcal{R} Q'$

$P \vdash x \Rightarrow Q \vdash x$

\begin{mathpar}
  \inferrule*[lab=Out-barb]{x \nameeq y}{{y}!\langle{Q}\rangle \vdash x}
  \and
  \inferrule*[lab=Par-barb]{\mbox{$P\vdash x$ or $Q\vdash x$}}{\binpar{P}{Q} \vdash x}
\end{mathpar}

\subsubsection{Contexts}

One of the principle advantages of computational calculi like the
$\pi$-calculus is a well-defined notion of context,
contextual-equivalence and a correlation between
contextual-equivalence and notions of bisimulation. The notion of
context allows the decomposition of a process into (sub-)process and
its syntactic environment, its context. Thus, a context may be
thought of as a process with a ``hole'' (written $\Box$) in it. The
application of a context $M$ to a process $P$, written $M[P]$, is
tantamount to filling the hole in $M$ with $P$. In this paper we do
not need the full weight of this theory, but do make use of the notion
of context in the proof the main theorem. 

\begin{mathpar}
  \inferrule* [lab=summation] {} {{M_{M},M_{N}} \bc \Box \;|\; x.M_{A} \;|\; M_{M}+M_{N}}
  \and
  \inferrule* [lab=agent] {} {{M_{A}} \bc (\vec{x})M_{P} \;| \; \clift{P_0,\ldots,M_{P},\ldots,P_N}}
  \and \\
  \inferrule* [lab=process] {} {{M_{P}} \bc M_{N} \;| \;P|M_{P} }
\end{mathpar} 

\begin{mathpar}
  \inferrule* [lab=sychronization] {} {M_{N} \bc \Box \;|\; x?M_{F} \;|\; x!M_{C}}
  \and
  \inferrule* [lab=abstraction] {} {{M_{F}} \bc (x)M_{P} }
  \and
  \inferrule* [lab=concretion] {} {{M_{C}} \bc \langle M_{P} \rangle }
  \and \\
  \inferrule* [lab=process] {} {{M_{P}} \bc M_{N} \;| \;P|M_{P} }
\end{mathpar}

\begin{definition}[contextual application] Given a context $M$, and
  process $P$, we define the \emph{contextual application}, $M[P] :=
  M\{P/\Box\}$. That is, the contextual application of M to P is the
  substitution of $P$ for $\Box$ in $M$.
\end{definition}

$\meaningof{-} : L \to \mathcal{P}(\pi)$

\begin{mathpar}
  \inferrule* [lab=collection] {} {\meaningof{true} = \pi, \and \meaningof{~E} = \pi \setminus \meaningof{E}, \and \meaningof{E_{1} \& E_{2}} = \meaningof{E_{1}} \cap \meaningof{E_{2}}}
\end{mathpar}

\begin{mathpar}
  \inferrule* [lab=structure] {} {\meaningof{0} = \{ P \in \pi | P \equiv 0 \}, \and \\ \meaningof{E_1 | E_2} = \{ P \in \pi | P \equiv P_{1} | P_{2}, P_{1} \in \meaningof{E_{1}}, P_{2} \in \meaningof{E_2}\} }
\end{mathpar}

\begin{mathpar}
 \inferrule* [lab=behavior] {} {\meaningof{\langle a?b \rangle E} = \{ P \in \pi | P \equiv Q | u?(y)P', \\ \and \\\\ \and \\ \;\;\; u \in \meaningof{a}, \forall z.P'\{z/y\} \in \meaningof{E\{z/b\}}\}, \and \\ \meaningof{a!E} = \{ P \in \pi | P \equiv Q | x!\langle P' \rangle, x \in \meaningof{a} P' \in \meaningof{E}\} }
\end{mathpar}

\begin{mathpar}
 \inferrule* [lab=nominal] {} {\meaningof{\quotep{E}} = \{ \quotep{P} \in \quotep{\pi} | P \in \meaningof{E} \}, \and \meaningof{\quotep{P}} = \{ \quotep{Q} \in \quotep{\pi} | P \equiv Q \} \and \\ \meaningof{@\quotep{E}} = \{ P \in \pi | P \equiv @x, x \in \meaningof{E} \}}
\end{mathpar}

\begin{eqnarray*}
  \\
  \meaningof{-} : TS \to ST
\end{eqnarray*}

\begin{eqnarray*}
  \\
  L : TS \to ST
\end{eqnarray*}

\begin{eqnarray*}
  \\
  P \models E \iff P \in \meaningof{E}
\end{eqnarray*}

\begin{eqnarray*}
  P \approx_{L} Q \iff \forall E \in L. P \models E \iff Q \models E
\end{eqnarray*}

\begin{eqnarray*}
  P \approx_{K} Q
\end{eqnarray*}

\begin{eqnarray*}
  P \approx Q
\end{eqnarray*}

$\approx_{K} = \approx = \approx_{L}$

\subsubsection{Contextual duality}

Note that contexts extend the quotation operation to a family of
operations from processes to names. Given a context, $M$, we can
define a \emph{nominal context}, $\quotep{M}$ by $\quotep{M}[P] :=
\quotep{M[P]}$. To foreshadow what is to come we observe that these
operations enjoy a duality with processes very much like the duality
between vectors and maps from vectors to scalars.

Further, because the calculus is essentially higher-order, we have a
correspondence between contexts and processes. More specifically,
given a name $x$ and a context $M$ we can construct $M^{*}_{x}$ such
that 

\begin{mathpar}
  M^{*}_{x} | \lift{x}{P} \red M[P]
\end{mathpar}

namely,

\begin{mathpar}
  M^{*}_{x} := x?(u).M[\dropn{u}]
\end{mathpar}

The dependence of $M^{*}_{x}$ on a name makes it an abstraction, 

\begin{mathpar}
  M^{*} := (x)x?(u).M[\dropn{u}]
\end{mathpar}

\subsection{Additional notation}

It will sometimes be convenient to denote the process a name
quotes. We already have the notation $x = \quotep{P}$, but it will be
convenient to introduce an alternate notation, $\procn{x}$, when we
want to emphasize the connection to the use of the name. Note that, by
virtue of name equivalence, $\quotep{\procn{x}} \nameeq x$; so, the
notation is consistent with previous definitions.

Further, because names have structure it is possible to effect
substitutions on the basis of that structure. This means we need to
upgrade our notation for substitutions, which we accomplish by
adapting comprehension notation. Thus,

\begin{mathpar}
  P\{ y / x : x \in S \}
\end{mathpar}

is interpreted to mean the process derived from P by replacing (in a
capture-avoiding manner) each occurrence of $x$ in $S$ by $y$. For example,

\begin{mathpar}
  P\{ \quotep{\procn{x}|\procn{x}} / x : x \in \freenames{P} \}
\end{mathpar}

will replace each (occurrence) of a free name $x$ in $P$ by
$\quotep{\procn{x}|\procn{x}}$.

Also, we will avail ourselves of the notation $x^{L}$ and $x^{R}$ to
denote injections of a name into disjoint copies of the name
space. There are numerous ways to accomplish this. One example can be
found in \cite{MeredithR05}. This notation overloads to vectors of
names: $\vec{x}^{\pi} := (x_{i}^{\pi} \; : \; 0 \leq i < |\vec{x}| )$ where $\pi \in \{L,R\}$.

We also use $P^{\Box} := P|\Box$.

In \cite{MeredithR05} an interpretation of the new operator is
given. It turns out that there are several possible interpretations
all enjoying the requisite algebraic properties of the operator (see
\cite{milner91polyadicpi}). We will therefore make liberal use of
$(\nu\; \vec{x})P$.

% subsection the_syntax_and_semantics_of_the_notation_system (end)   

\input{qm2pi.qmops} 

\input{qm2pi.sterngerlach} 

\input{qm2pi.metric} 

% section concurrent_process_calculi (end)

%\input{qm2pi.proofsketch}

% section proof sketch (end)

%\input{qm2pi.slviaknots} 

% section spatial logic via knots (end)

\input{qm2pi.conclusion}

% section conclusion (end)

%\input{qm2pi.dtcodes} 

% section wiring algorithm (end)

\input{qm2pi.ack} 

% section acknowledgments (end)

\newpage


\bibliographystyle{plain}   
\bibliography{../../biblios/main.bib}

\input{qm2pi.rhodetails}

\end{document}

 

% subsection basic_interpretation (end)

%\input{qm2pi.rho.presentation} 
\subsection{The syntax and semantics of the notation system}\label{sub:the_syntax_and_semantics_of_the_notation_system} % (fold)

We now summarize a technical presentation of the calculus that
embodies our theory of dynamics. The typical presentation of such a
calculus follows the style of giving generators and relations on
them. The grammar, below, describing term constructors, freely
generates the set of processes, $\Proc$. This set is then quotiented
by a relation known as structural congruence and it is over this set
that the notion of dynamics is expressed. This presentation is
essentially that of \cite{MeredithR05} with the addition of
polyadicity and summation. For readability we have relegated some of
the technical subtleties to an appendix.

\subsubsection{Process grammar}\label{subsub:process_grammar}

\begin{mathpar}
  \inferrule* [lab=synchronization] {} {{M} \bc \pzero \;|\; x?F \;|\; x!C }
  \and
  \inferrule* [lab=abstraction] {} {{F} \bc (x)P}
  \and
  \inferrule* [lab=concretion] {} {{C} \bc \langle Q \rangle}
  \and
  \inferrule* [lab=process] {} {{P,Q} \bc M \;| \;P|Q \;|\; @{x}}
  \and
  \inferrule* [lab=name] {} {{x} \bc \quotep{P}}
\end{mathpar} 

Note that $\vec{x}$ (resp. $\vec{P}$) denotes a vector of names
(resp. processes) of length $|\vec{x}|$ (resp. $|\vec{P}|$). We adopt
the following useful abbreviations.

\begin{mathpar}
   x?(\vec{y}).P := x.(\vec{y})P \and  x\clift{\vec{P}} := x.\clift{\vec{P}}
   \and x!(y) := \lift{x}{\dropn{y}}
   \and \Pi_{i=0}^{n-1}P_i := P_0 | \ldots | P_{n-1}
\end{mathpar}

\subsubsection{Structural congruence}

\paragraph{Free and bound names and alpha-equivalence.} At the
core of structural equivalence is alpha-equivalence which identifies
process that are the same up to a change of variable. Formally, we
recognize the distinction between free and bound names. The free names
of a process, $\freenames{P}$, may be calculated recursively as
follows:

\begin{mathpar}
\freenames{\pzero} := \emptyset
  \and \\
  \freenames{x?(y).P} := \{ x \} \cup (\freenames{P} \setminus \{ y \})
  \and 
  \freenames{x!\langle P \rangle} := \{ x \} \cup \{ P \} 
  \and \\
  \freenames{P|Q} := \freenames{P} \cup \freenames{Q}
  \and \\
  \freenames{@{x}} := \{ x \}
\end{mathpar}

$\pi$
$\quotep{\pi}$

$\freenames{-} : \pi \to \mathcal{P}(\quotep{\pi})$

\begin{eqnarray*}
  \freenames{\pzero} & := & \emptyset \\
  \freenames{x?(y).P} & := & \{ x \} \cup (\freenames{P} \setminus \{ y \}) \\
  \freenames{x!\langle P \rangle} & := & \{ x \} \cup \{ P \} \\
  \freenames{P|Q} & := & \freenames{P} \cup \freenames{Q} \\
  \freenames{\dropn{x}} & := & \{ x \}
\end{eqnarray*}

The bound names of a process, $\boundnames{P}$, are those names occurring in $P$
that are not free. For example, in $x?(y).0$, the name $x$ is free, while $y$ is bound.

\begin{mathpar}
  \inferrule* [lab=monoidal-laws] {} { P|Q \equiv Q|P \and P|0 \equiv P \and P|(Q|R) \equiv (P|Q)|R }
\end{mathpar}

\begin{mathpar}
  \inferrule* [lab=alpha-equivalence] {} { (x)P \equiv (y)P\{y/x\} \and y \not\in \freenames{P} }
\end{mathpar}

\begin{definition}
Then two processes, $P,Q$, are alpha-equivalent if $P = Q\{\vec{y}/\vec{x}\}$ for
some $\vec{x} \in \boundnames{Q},\vec{y} \in \boundnames{P}$, where $Q\{\vec{y}/\vec{x}\}$
denotes the capture-avoiding substitution of $\vec{y}$ for $\vec{x}$ in $Q$.
\end{definition}

\begin{definition}
  The {\em structural congruence} \cite{SangiorgiWalker} , $\equiv$,
  between processes is the least congruence containing
  alpha-equivalence, satisfying the abelian monoid laws
  (associativity, commutativity and $\pzero$ as identity) for parallel
  composition $|$ and for summation $+$.
\end{definition}

\subsection{Name equivalence}

We take name equivalence, written $\nameeq$, to be the smallest
equivalence relation generated by the following rules.

\begin{mathpar}
\inferrule*[lab=Quote-drop]
{ }
{ \quotep{@{x}} \nameeq x }

\inferrule*[lab=Struct-equiv]
{ P \scong Q }
{ \quotep{P} \nameeq \quotep{Q} }
\end{mathpar}

The astute reader will have noticed that the mutual recursion of names
and processes imposes a mutual recursion on alpha-equivalence and
structural equivalence via name-equivalence. Fortunately, all of this
works out pleasantly and we may calculate in the natural way, free of
concern. The reader interested in the details is referred to the
appendix \ref{appendix:rho_details}.

\subsection{Substitution}

We use $\Proc$ for the set of processes, $\QProc$ for the set of
names, and $\id{\{}\vec{y} / \vec{x} \id{\}}$ to denote partial maps,
$s : \QProc \rightarrow \QProc$. A map, $s$ lifts, uniquely, to a map
on process terms, $\widehat{s} : \Proc \rightarrow \Proc$ by the
following equations.

\begin{mathpar}
  (0) \psubstp{Q}{P} := 0 \\
  (R \juxtap S) \psubstp{Q}{P}
  :=    
  (R)\psubstp{Q}{P} \juxtap (S) \psubstp{Q}{P} \\
  (x?(y).R) \psubstp{Q}{P}    
  :=    
  (x)\substp{Q}{P} (z)\concat( (R \psubstn{z}{y}) \psubstp{Q}{P} ) \\
  (\lift{x}{R}) \psubstp{Q}{P}  
  :=
  \lift{(x)\substp{Q}{P}}{ R \psubstp{Q}{P} } \\
%   (\dropn{x})  \psubstp{Q}{P}       
%   := 
%   \left\{ 
%     \begin{array}{ccc} 
%       \dropn{\quotep{Q}} & & x \nameeq \quotep{P} \\
%       \dropn{x} & & otherwise \\
%     \end{array}
%   \right. 
  (\dropn{x})  \psubstp{Q}{P}       
  := 
  \left\{ 
    \begin{array}{ccc} 
      Q & & x \nameeq \quotep{P} \\
      \dropn{x} & & otherwise \\
    \end{array}
  \right.
\end{mathpar}
 

where

\begin{eqnarray}
  (x)\id{\{} \lpquote Q \rpquote / \lpquote P \rpquote \id{\}}            = 
  \left\{ 
    \begin{array}{ccc}
      \lpquote Q \rpquote & & x \nameeq \lpquote P \rpquote \\
      x & & otherwise \\
    \end{array}
  \right. \nonumber
\end{eqnarray}

and $z$ is chosen distinct from $\quotep{P}$, $\quotep{Q}$, the free
names in $Q$, and all the names in $R$. Our $\alpha$-equivalence will
be built in the standard way from this substitution.

\begin{remark}\label{rem:no_self_referential_names}
  One consequence of these definitions is that $\forall P. \quotep{P}
  \not\in \freenames{P}$.
\end{remark}

\subsection{ Dynamic quote: an example }

Anticipating something of what's to come, consider applying the
substitution, $\widehat{\id{\{}u / z \id{\}}}$, to the following pair
of processes, $\lift{w}{y!(z)}$ and $w[ \lpquote y!(z) \rpquote ]$.

\begin{eqnarray}
	\lift{w}{y!(z)}\widehat{\id{\{}u / z \id{\}}}
		& = &
		\lift{w}{y!(u)} \nonumber\\
	w[ \lpquote y!(z) \rpquote ] \widehat{ \id{\{}u / z \id{\}} }
		& = &
		w[ \lpquote y!(z) \rpquote ] \nonumber
\end{eqnarray}

Because the body of the process between quotes is impervious to
substitution, we get radically different answers. In fact, by
examining the first process in an input context,
e.g. $x?(z).\lift{w}{y!(z)}$, we see that the process under the lift
operator may be shaped by prefixed inputs binding a name inside it. In
this sense, the lift operator will be seen as a way to dynamically
construct processes before reifying them as names.

Finally equipped with these standard features we can present the
dynamics of the calculus.

\subsubsection{Operational semantics} 

Finally, we introduce the computational dynamics. What marks these
algebras as distinct from other more traditionally studied algebraic
structures, e.g. vector spaces or polynomial rings, is the manner in
which dynamics is captured. In traditional structures, dynamics is typically
expressed through morphisms between such structures, as in linear maps
between vector spaces or morphisms between rings. In algebras
associated with the semantics of computation, the dynamics is
expressed as part of the algebraic structure itself, through a
reduction reduction relation typically denoted by $\red$. Below, we
give a recursive presentation of this relation for the calculus used
in the encoding.

$\red \subseteq \pi \times \pi$
$\red : \pi \to \mathcal{P}(\pi)$

\begin{mathpar}
  \inferrule* [lab=Comm] { \textsf{match}( x_{src}, x_{trgt} ) } { x_{trgt}?(y)P \; | \; x_{src}!\langle {Q} \rangle \red P\{\quotep{Q}/y}\} }
  \and \\
  \inferrule* [lab=Par] {{P} \red {P}'} {{{P} | {Q}} \red {{P}' | {Q}}}
  \and
  \inferrule* [lab=Equiv]{{{P} \scong {P}'} \andalso {{P}' \red {Q}'} \andalso {{Q}' \scong {Q}}}{{P} \red {Q}}
\end{mathpar}

\begin{eqnarray*}
  match_{\equiv} (\quotep{P},\quotep{Q}) & := & P \equiv Q \\
  match_{\dagger}(\quotep{P},\quotep{Q}) & := & \forall R. P|Q \red^{*} R => R \red^{*} 0 \\
  match_{K}(\quotep{P},\quotep{Q}) & := & K \mbox{ for some context } K
\end{eqnarray*}

$u?(x)P | u!\langle Q \rangle \red P\{\quotep{Q}/x\}$

%We write $\wred$ for $\red^*$, and $P\red$ if $\exists Q $ such that $ P \red Q$.
We write $P\red$ if $\exists Q $ such that $ P \red Q$ and $P\not\red$, otherwise.

\section{Replication}

As mentioned before, it is known that replication (and hence
recursion) can be implemented in a higher-order process algebra
\cite{SangiorgiWalker}. As our first example of calculation with the
machinery thus far presented we give the construction explicitly in
the {\rhoc}.

\begin{eqnarray}
	D_{x} & := & \prefix{x}{y}{(\binpar{\outputp{x}{y}}{@{y}})} \nonumber\\
	\bangp_{x}{P} & := & \binpar{{x}!\langle{\binpar{D_{x}}{P}}\rangle}{D_{x}} \nonumber
\end{eqnarray}

\begin{eqnarray}
	\bangp_{x}{P} & & \nonumber\\
	=
	& {x}!\langle{(\prefix{x}{y}{(\outputp{x}{y} | @{y})) | P}}\rangle 
	      | \prefix{x}{y}{(\outputp{x}{y} | @{y})} & \nonumber\\
	\red
	& (\outputp{x}{y} | @{y})\substn{\quotep{(\prefix{x}{y}{(@{y} | \outputp{x}{y})) | P}}}{y} & \nonumber\\
	=
	& \outputp{x}{\quotep{(\prefix{x}{y}{(\outputp{x}{y} | @{y})) | P}}}
	  | {(\prefix{x}{y}{(\outputp{x}{y} | @{y})) | P}} & \nonumber\\
	\red
	& \ldots & \nonumber\\
	\red^*
	& P | P | \ldots & \nonumber
\end{eqnarray}

Of course, this encoding, as an implementation, runs away, unfolding
$\bangp{P}$ eagerly. A lazier and more implementable replication
operator, restricted to input-guarded processes, may be obtained as follows.

\begin{eqnarray}
\bangp{\prefix{u}{v}{P}} 
	:= 
	\binpar{\lift{x}{\prefix{u}{v}{(\binpar{D(x)}{P})}}}{D(x)} \nonumber
\end{eqnarray}

\begin{remark}
  Note that the lazier definition still does not deal with summation
  or mixed summation (i.e. sums over input and output). The reader is
  invited to construct definitions of replication that deal with these
  features. 

  Further, the definitions are parameterized in a name, $x$. Can you,
  gentle reader, make a definition that eliminates this parameter and
  guarantees no accidental interaction between the replication
  machinery and the process being replicated -- i.e. no accidental
  sharing of names used by the process to get its work done and the
  name(s) used by the replication to effect copying. This latter
  revision of the definition of replication is crucial to obtaining
  the expected identity $!!P \sim !P$.
\end{remark}

\begin{remark}\label{rem:paradoxical_combinator}
  The reader familiar with the lambda calculus will have noticed the
  similarity between $D$ and the paradoxical combinator.

  [Ed. note: the existence of this seems to suggest we have to be more
  restrictive on the set of processes and names we admit if we are to
  support no-cloning.]
\end{remark}

\subsubsection{Bisimulation}

The computational dynamics gives rise to another kind of equivalence,
the equivalence of computational behavior. As previously mentioned
this is typically captured \emph{via} some form of bisimulation.

% The notion we use in this paper is weak barbed bisimulation
% \cite{milner91polyadicpi}.

The notion we use in this paper is derived from weak barbed
bisimulation \cite{milner91polyadicpi}. 

\begin{definition}
An \emph{observation relation}, $\downarrow_{\mathcal N}$, over a set
of names, $\mathcal N$, is the smallest relation satisfying the rules
below.

\infrule[Out-barb]{y \in {\mathcal N}, \; x \nameeq y}
		  {\outputp{x}{v} \downarrow_{\mathcal N} x}
\infrule[Par-barb]{\mbox{$P\downarrow_{\mathcal N} x$ or $Q\downarrow_{\mathcal N} x$}}
		  {\binpar{P}{Q} \downarrow_{\mathcal N} x}

We write $P \Downarrow_{\mathcal N} x$ if there is $Q$ such that 
$P \wred Q$ and $Q \downarrow_{\mathcal N} x$.
\end{definition}

\begin{definition}
%\label{def.bbisim}
An  ${\mathcal N}$-\emph{barbed bisimulation} over a set of names, ${\mathcal N}$, is a symmetric binary relation 
${\mathcal S}_{\mathcal N}$ between agents such that $P\rel{S}_{\mathcal N}Q$ implies:
\begin{enumerate}
\item If $P \red P'$ then $Q \wred Q'$ and $P'\rel{S}_{\mathcal N} Q'$.
\item If $P\downarrow_{\mathcal N} x$, then $Q\Downarrow_{\mathcal N} x$.
\end{enumerate}
$P$ is ${\mathcal N}$-barbed bisimilar to $Q$, written
$P \wbbisim_{\mathcal N} Q$, if $P \rel{S}_{\mathcal N} Q$ for some ${\mathcal N}$-barbed bisimulation ${\mathcal S}_{\mathcal N}$.
\end{definition}

$\mathcal{R} \subseteq \pi \times \pi$

$P \mathcal{R} Q => \forall P'. P \red P' \Rightarrow \exists Q'. Q \red Q', P' \mathcal{R} Q'$

$P \vdash x \Rightarrow Q \vdash x$

\begin{mathpar}
  \inferrule*[lab=Out-barb]{x \nameeq y}{{y}!\langle{Q}\rangle \vdash x}
  \and
  \inferrule*[lab=Par-barb]{\mbox{$P\vdash x$ or $Q\vdash x$}}{\binpar{P}{Q} \vdash x}
\end{mathpar}

\subsubsection{Contexts}

One of the principle advantages of computational calculi like the
$\pi$-calculus is a well-defined notion of context,
contextual-equivalence and a correlation between
contextual-equivalence and notions of bisimulation. The notion of
context allows the decomposition of a process into (sub-)process and
its syntactic environment, its context. Thus, a context may be
thought of as a process with a ``hole'' (written $\Box$) in it. The
application of a context $M$ to a process $P$, written $M[P]$, is
tantamount to filling the hole in $M$ with $P$. In this paper we do
not need the full weight of this theory, but do make use of the notion
of context in the proof the main theorem. 

\begin{mathpar}
  \inferrule* [lab=summation] {} {{M_{M},M_{N}} \bc \Box \;|\; x.M_{A} \;|\; M_{M}+M_{N}}
  \and
  \inferrule* [lab=agent] {} {{M_{A}} \bc (\vec{x})M_{P} \;| \; \clift{P_0,\ldots,M_{P},\ldots,P_N}}
  \and \\
  \inferrule* [lab=process] {} {{M_{P}} \bc M_{N} \;| \;P|M_{P} }
\end{mathpar} 

\begin{mathpar}
  \inferrule* [lab=sychronization] {} {M_{N} \bc \Box \;|\; x?M_{F} \;|\; x!M_{C}}
  \and
  \inferrule* [lab=abstraction] {} {{M_{F}} \bc (x)M_{P} }
  \and
  \inferrule* [lab=concretion] {} {{M_{C}} \bc \langle M_{P} \rangle }
  \and \\
  \inferrule* [lab=process] {} {{M_{P}} \bc M_{N} \;| \;P|M_{P} }
\end{mathpar}

\begin{definition}[contextual application] Given a context $M$, and
  process $P$, we define the \emph{contextual application}, $M[P] :=
  M\{P/\Box\}$. That is, the contextual application of M to P is the
  substitution of $P$ for $\Box$ in $M$.
\end{definition}

$\meaningof{-} : L \to \mathcal{P}(\pi)$

\begin{mathpar}
  \inferrule* [lab=collection] {} {\meaningof{true} = \pi, \and \meaningof{~E} = \pi \setminus \meaningof{E}, \and \meaningof{E_{1} \& E_{2}} = \meaningof{E_{1}} \cap \meaningof{E_{2}}}
\end{mathpar}

\begin{mathpar}
  \inferrule* [lab=structure] {} {\meaningof{0} = \{ P \in \pi | P \equiv 0 \}, \and \\ \meaningof{E_1 | E_2} = \{ P \in \pi | P \equiv P_{1} | P_{2}, P_{1} \in \meaningof{E_{1}}, P_{2} \in \meaningof{E_2}\} }
\end{mathpar}

\begin{mathpar}
 \inferrule* [lab=behavior] {} {\meaningof{\langle a?b \rangle E} = \{ P \in \pi | P \equiv Q | u?(y)P', \\ \and \\\\ \and \\ \;\;\; u \in \meaningof{a}, \forall z.P'\{z/y\} \in \meaningof{E\{z/b\}}\}, \and \\ \meaningof{a!E} = \{ P \in \pi | P \equiv Q | x!\langle P' \rangle, x \in \meaningof{a} P' \in \meaningof{E}\} }
\end{mathpar}

\begin{mathpar}
 \inferrule* [lab=nominal] {} {\meaningof{\quotep{E}} = \{ \quotep{P} \in \quotep{\pi} | P \in \meaningof{E} \}, \and \meaningof{\quotep{P}} = \{ \quotep{Q} \in \quotep{\pi} | P \equiv Q \} \and \\ \meaningof{@\quotep{E}} = \{ P \in \pi | P \equiv @x, x \in \meaningof{E} \}}
\end{mathpar}

\begin{eqnarray*}
  \\
  \meaningof{-} : TS \to ST
\end{eqnarray*}

\begin{eqnarray*}
  \\
  L : TS \to ST
\end{eqnarray*}

\begin{eqnarray*}
  \\
  P \models E \iff P \in \meaningof{E}
\end{eqnarray*}

\begin{eqnarray*}
  P \approx_{L} Q \iff \forall E \in L. P \models E \iff Q \models E
\end{eqnarray*}

\begin{eqnarray*}
  P \approx_{K} Q
\end{eqnarray*}

\begin{eqnarray*}
  P \approx Q
\end{eqnarray*}

$\approx_{K} = \approx = \approx_{L}$

\subsubsection{Contextual duality}

Note that contexts extend the quotation operation to a family of
operations from processes to names. Given a context, $M$, we can
define a \emph{nominal context}, $\quotep{M}$ by $\quotep{M}[P] :=
\quotep{M[P]}$. To foreshadow what is to come we observe that these
operations enjoy a duality with processes very much like the duality
between vectors and maps from vectors to scalars.

Further, because the calculus is essentially higher-order, we have a
correspondence between contexts and processes. More specifically,
given a name $x$ and a context $M$ we can construct $M^{*}_{x}$ such
that 

\begin{mathpar}
  M^{*}_{x} | \lift{x}{P} \red M[P]
\end{mathpar}

namely,

\begin{mathpar}
  M^{*}_{x} := x?(u).M[\dropn{u}]
\end{mathpar}

The dependence of $M^{*}_{x}$ on a name makes it an abstraction, 

\begin{mathpar}
  M^{*} := (x)x?(u).M[\dropn{u}]
\end{mathpar}

\subsection{Additional notation}

It will sometimes be convenient to denote the process a name
quotes. We already have the notation $x = \quotep{P}$, but it will be
convenient to introduce an alternate notation, $\procn{x}$, when we
want to emphasize the connection to the use of the name. Note that, by
virtue of name equivalence, $\quotep{\procn{x}} \nameeq x$; so, the
notation is consistent with previous definitions.

Further, because names have structure it is possible to effect
substitutions on the basis of that structure. This means we need to
upgrade our notation for substitutions, which we accomplish by
adapting comprehension notation. Thus,

\begin{mathpar}
  P\{ y / x : x \in S \}
\end{mathpar}

is interpreted to mean the process derived from P by replacing (in a
capture-avoiding manner) each occurrence of $x$ in $S$ by $y$. For example,

\begin{mathpar}
  P\{ \quotep{\procn{x}|\procn{x}} / x : x \in \freenames{P} \}
\end{mathpar}

will replace each (occurrence) of a free name $x$ in $P$ by
$\quotep{\procn{x}|\procn{x}}$.

Also, we will avail ourselves of the notation $x^{L}$ and $x^{R}$ to
denote injections of a name into disjoint copies of the name
space. There are numerous ways to accomplish this. One example can be
found in \cite{MeredithR05}. This notation overloads to vectors of
names: $\vec{x}^{\pi} := (x_{i}^{\pi} \; : \; 0 \leq i < |\vec{x}| )$ where $\pi \in \{L,R\}$.

We also use $P^{\Box} := P|\Box$.

In \cite{MeredithR05} an interpretation of the new operator is
given. It turns out that there are several possible interpretations
all enjoying the requisite algebraic properties of the operator (see
\cite{milner91polyadicpi}). We will therefore make liberal use of
$(\nu\; \vec{x})P$.

% subsection the_syntax_and_semantics_of_the_notation_system (end)   

\section{Interpretation of QM}
\subsection{Supporting definitions}
\subsubsection{Multiplication}
\begin{mathpar}
  \quotep{Q} \cdot \quotep{R} := \quotep{Q|R}
  \and \\
  \quotep{Q} \cdot P := P\{ \quotep{Q|R} / \quotep{R} : \quotep{R} \in \freenames{P} \}
\end{mathpar}

\paragraph{Discussion}
The first line needs little explanation. The second line says that
each free name of the process is replaced with the multiplication of
that name by the scalar. Multiplication of a scalar (name) by a state
(process) results in a process all the names of which have been `moved
over' by parallel composition with the process the scalar
quotes. There is a subtlety that the bound names have to be
manipulated so that multiplied names aren't accidentally
captured. There are many ways to achieve this.

\begin{remark}\label{rem:multiplication_identities}
  The reader is invited to verify that for all $x,y,z \in \QProc$ and $P \in \Proc$
  \begin{mathpar}
    x \cdot \quotep{0} \equiv x 
    \and
    x \cdot y \equiv y \cdot x
    \and
    x \cdot (y \cdot z) \equiv (x \cdot y) \cdot z
    \and \\
    \quotep{0} \cdot P \equiv P
    \and \\
    x \cdot (y \cdot P) \equiv (x \cdot y) \cdot P
    \and \\
    x \cdot (P|Q) \equiv (x \cdot P) | (x \cdot Q)
    \and \\    
  \end{mathpar}
\end{remark}

\subsubsection{Tensor product}

We define a tensor product on processes by structural induction.

\paragraph{Tensor of sums} First note that all summations, including
$\pzero$ and sequence, can be written $\Sigma_{i} x_{i}.A_{i} +
\Sigma_{j} x_{j}.C_{j}$, where we have grouped input-guarded processes
together and output-guarded processes together.

Thus, we can define the tensor product of two summations, $N_{1}\otimes N_{2}$, where

\begin{mathpar}
  N_{1} := \Sigma_{i} x_{i}.A_{i} + \Sigma_{j} x_{j}.C_{j}
  \and
  N_{2} := \Sigma_{i'} y_{i'}.B_{i'} + \Sigma_{j'} y_{j'}.D_{j'} 
\end{mathpar}

as follows.

\begin{mathpar}
  \Sigma_{i} x_{i}.A_{i} + \Sigma_{j} x_{j}.C_{j} \otimes \Sigma_{i'}
  y_{i'}.B_{i'} + \Sigma_{j'} y_{j'}.D_{j'} 
  \and \\
  := \; \Sigma_{i} \Sigma_{i'} \quotep{\stackrel{\vee}{x_{i}}| \stackrel{\vee}{y_{i'}}}.(A_{i}\otimes B_{i'}) \; | \; \Sigma_{i'} \Sigma_{i} \quotep{\stackrel{\vee}{y_{i'}}|\stackrel{\vee}{x_{i}}}.(B_{i'}\otimes A_{i})
  \and
  \;\; | \;\; \Sigma_{j} \Sigma_{j'} \quotep{\stackrel{\vee}{x_{j}}|\stackrel{\vee}{y_{j'}}}.(A_{j}\otimes B_{j'}) \; | \; \Sigma_{j'} \Sigma_{j} \quotep{\stackrel{\vee}{y_{j'}}|\stackrel{\vee}{x_{j}}}.(B_{j'}\otimes A_{j})
\end{mathpar}

\begin{remark}
  Do we need to $x^{L}$ and $y^{R}$ for this construction as well?
\end{remark}

\paragraph{Tensor of parallel compositions} Next, we distribute tensor
over par.

\begin{mathpar}
  P_{1}|P_{2} \otimes Q_{1}|Q_{2} := (P_{1} \otimes Q_{1}) | (P_{1}
  \otimes Q_{2}) | (P_{2} \otimes Q_{1}) | (P_{2} \otimes Q_{2})
\end{mathpar}

\paragraph{Tensor with dropped names} We treat tensor of a
process with a dropped name as parallel composition.

\begin{mathpar}
  P \otimes \dropn{x} := P | \dropn{x}
\end{mathpar}

\paragraph{Tensor of agents}

Finally, we need to define tensor on agents. Note that the definition
of tensor on normal products only tensors inputs with inputs and
outputs with outputs. Thus, we only have to define the operation on
``homogeneous'' pairings.

\begin{mathpar}
  (\vec{x})P \otimes (\vec{y})Q
  \and \\
  := (x_{0}^{L}|y_{0}^{R},\ldots,x_{0}^{L}|y_{n}^{R},\ldots,x_{m}^{L}|y_{0}^{R},\ldots,x_{m}^{L}|y_{n}^R)(P\{ \vec{x}^{L}/\vec{x}\} \otimes Q \{ \vec{y}^{R}/\vec{y}\})
  \and \\
  \clift{\vec{P}} \otimes \clift{\vec{Q}}
  \and \\
  := \clift{P_{0}\otimes Q_{0},\ldots,P_{0}\otimes Q_{n},\ldots,P_{m}\otimes Q_{0},\ldots,P_{m}\otimes Q_{n}}
\end{mathpar}

\begin{remark}
  Observe that arities of tensored abstractions matches arities of
  tensored concretions if the original arities matched. Note also that
  the length of the arities corresponds to the increase in dimension
  we see in ordinary vector space tensor product.
\end{remark}

\begin{remark}
  Operationally, this definition distributes the tensor down to
  components ``linked'' by summation. Tensor over summation is
  intriguing in that it mixes names. Moreover, as a consequence of the
  way it mixes names we have the identities for all $x \in \QProc$ and
  $P,Q \in \Proc$

  \begin{mathpar}
    (x \cdot P) \otimes Q \equiv x \cdot (P \otimes Q) \equiv P \otimes (x \cdot Q)
    \and
    P \otimes \pzero \equiv P
  \end{mathpar}

  that the reader is invited to verify.
\end{remark}

\subsubsection{Annihilation}
\begin{mathpar}
  P^{\perp} := \{ Q | \forall R. P|Q \red^{*} R \Rightarrow R \red^{*} \pzero \}
  \and \\
  P^{\underline{\perp}} := \Sigma_{Q \in P^{\perp}} \quotep{Q}?(y).(\dropn{y}|Q) | \Sigma_{Q \in P^{\perp}} \quotep{Q}\clift{\Box}
\end{mathpar}

\paragraph{Discussion} The reader will note that $P^{\perp}$ is a
\emph{set} of processes, while $P^{\underline{\perp}}$ is a
\emph{context}. We call the set $P^{\perp}$ the \emph{annihilators} of
$P$. The parallel composition of a process in the annihilators of $P$
with $P$ will result in a process, the state space of which has all
paths eventually leading to $\pzero$. Execution may endure loops; but
under reasonable conditions of fairness (naturally guaranteed under
most notions of bisimulation) such a composite process cannot get
stuck in such a loop and will, eventually pop out and terminate.

The context $P^{\underline{\perp}}$ is ready and willing to ``take the
$P$ out of'' the process to which it is applied. It will effectively
transmit the code of the process to which it is applied to one of the
annihilators and run the process against it.

\subsubsection{Evaluation}
We fix $M$ a domain of fully abstract interpretation with an equality
coincident with bisimulation. We take $\meaningof{\cdot} : \Proc \to
M$ to be the map interpreting processes and $\nmeaningof{\cdot} : \M
\to Proc$ to be the map running the other way. Then we define

\begin{mathpar}
  \int P := \nmeaningof{\meaningof{P}}
\end{mathpar}

\paragraph{Discussion}
There are many fully abstract interpretations of Milner's
$\pi$-calculus. Any of them can be used as a basis for interpreting
the reflective calculus here. Equipped with such a domain it is
largely a matter of grinding through to check that the Yoneda
construction for the normalization-by-evaluation program can be
extended to this setting.

\begin{remark}
  The reader is invited to verify that $\int (P^{\underline{\perp}}[P]) = 0$.
\end{remark}

\subsection{Quantum mechanics}

Table \ref{tbl:core_qm_op_defns} gives the core operational definitions

\begin{table}[htp]\label{tbl:core_qm_op_defns}
  \center{
    \fbox{
      \begin{tabular}{c|c}
        quantum mechanics & process calculus \\
        \hline
        scalar & $x := \quotep{P}$ \\
        state vector & $\state{P} := P$ \\
        dual & $\state{P}^{*} := \event{P^{\underline{\perp}}} := \quotep{P^{\underline{\perp}}}[-]$ \\
        matrix & $ \Sigma_{\alpha} \state{P_{\alpha}}x_{\alpha}\event{Q_{\alpha}}$ \\
        vector addition & $\state{P} + \state{Q} := \state{P | Q}$ \\
        tensor product & $\state{P} \otimes \state{Q} := \state{P \otimes Q}$ \\
        inner product & $\innerprod{P}{Q} := \quotep{\int P^{\underline{\perp}}[Q]}$ \\
      \end{tabular}
    }
  }
  \caption{QM - operational definitions}
\end{table}

where

\begin{mathpar}
  \prmatrix{P}{Q} := \fprmatrix{P}{\quotep{\pzero}}{Q}
  \and
  \fprmatrix{P}{x}{Q} := (\state{P},x,\event{Q})
  \and
  (\fprmatrix{P}{x}{Q})(\state{R}) := x \cdot \innerprod{Q}{R} \cdot \state{P}
  \and
  (\fprmatrix{P}{x}{Q})(\event{R}) := x \cdot \innerprod{R}{P} \cdot \event{Q}
\end{mathpar}

\paragraph{Discussion}
As promised: vectors (aka states) are represented as processes; duals
as contextual duals; inner product definition should be compared with
standard inner product definition for ....

\begin{remark}
  Assuming $\int (P^{\underline{\perp}}[P]) = 0$, the reader is
  invited to verify that $(\fprmatrix{P}{x}{P})(\state{P}) = x \cdot \state{P}$.
\end{remark}

\begin{remark}
  The reader is invited to verify that $\innerprod{P}{Q}$ could
  equally well have been written $\quotep{\int \stackrel{\vee}{x}}$
  where $x = \event{P^{\underline{\perp}}}(Q)$.

  One of the motivations for this remark is that there is another way
  to factor these operations. We could package up evaluation in the dual:

  \begin{mathpar}
    \state{P}^{*} := \event{\int P^{\underline{\perp}}} := \quotep{\int P^{\underline{\perp}}}[-]
  \end{mathpar}

  and then have inner product defined by
  
  \begin{mathpar}
    \innerprod{P}{Q} := \event{P}(Q)
  \end{mathpar}

  Hopefully, experience with the calculations will provide guidance on
  the best factoring.
\end{remark}

\begin{remark}
  Assuming $\int (P^{\underline{\perp}}[P]) = 0$, the reader is
  invited to verify that $\forall P,Q. (\prmatrix{0}{Q})(\state{0}) =
  \state{0}$ and dually $(\prmatrix{P}{0})(\event{0}) = \event{0}$.
\end{remark}

\begin{remark}
  i'm a little worried that i don't (yet) have proper support for
  complex conjugacy. But, the observation above may give us a
  clue. According to Abramsky, it must be the case that the scalars
  are iso to the homset of the identity for the tensor -- which the
  observation above characterizes. 

  For now, we will simply bookmark the notion with $\overline{x}$.
\end{remark}

\subsubsection{Adjointness}

We need to give a definition of $(\cdot)^{\dagger}$ for matrices. The
obvious candidate definition is
\begin{mathpar}
(\Sigma_{\alpha}\fprmatrix{P_{\alpha}}{x_{\alpha}}{Q_{\alpha}})^{\dagger}
= \Sigma_{\alpha}\fprmatrix{(Q_{\alpha}^{\underline{\perp}})^{*}}{\overline{x}_{\alpha}}{P_{\alpha}^{\underline{\perp}}} 
\end{mathpar}

But, $(Q_{\alpha}^{\underline{\perp}})^{*}$ requires a name along
which to communicate the process to achieve the context application.

\subsubsection{Basis for a basis}
If processes label states and ``addition'' of states (a.k.a. vector
addition) is interpreted as parallel composition, what corresponds to
notions of linear independence and basis? Here, we recall that Yoshida
has developed a set of \emph{combinators} for an asynchronous verison
of Milner's $\pi$-calculus. These are a finite set of processes such
any process can be expressed as parallel composition of these
combinators together with liberal uses of the new operator and
replication. We can simply give a translation of these into the
present calculus and have reasonable expectation that the property
carries over. That is, that the resultant set allows to express all
processes via parallel composition. Note, however, that there is no
new operator or replication in this calculus. As a result, we expect
that the corresponding set is actually infinite. That is, we expect
that the space is actually infinite dimensional.

\begin{remark}
  The attentive reader may be a bit concerned. Certainly, the
  collection $S$, $K$ and $I$ is a finite set of
  combinators. Shouldn't we expect to see a finite set of combinators
  for an effectively equivalent system? i am very sympathetic to this
  critique and feel it warrants full attention. On the other hand, i
  also have in mind the following analogy. The natural numbers, as a
  monoid under addition, has exactly $1$ generator, while the natural
  numbers, as a monoid under multiplication, has countably many
  generators (the primes). We observe that the application of the
  lambda calculus is much less resource sensitive than the parallel
  composition of the $\pi$-calculus. Could it be the case that we have
  an analogy of the form
  
  \begin{mathpar}
    m + n : MN :: m*n : M|N
  \end{mathpar}

  giving a similar blow up in the set of ``primes''?  This is such a
  wonderful thought that, even if it's not true, i think it's worth
  writing down.
\end{remark}
 

\documentclass[12pt]{llncs}
%\documentclass{jktr}

\usepackage[pdftex]{hyperref}                   
\usepackage {listings}
\usepackage {mathpartir}
\usepackage{bcprules}
%\usepackage{listings}
                       
\usepackage{graphicx} 
%\usepackage[margins=2.5cm,nohead,nofoot]{geometry}
%\usepackage{geometry}
\usepackage{amsfonts}
\usepackage{amstext}
\usepackage{latexsym}
\usepackage{amssymb}
\usepackage{color}


%\include{myPreamble}
\include{qm2pi.local} 

%\ifpdf
%\usepackage[pdftex]{graphicx}
%\else
%\usepackage{graphicx}
%\fi

 % \ifpdf
%  \usepackage{pdfsync}
%  \if


%\title{Brief Article}
%\author{David F. Snyder}
%\author{L.G. Meredith}

%\address{Dept. of Math., Texas State University--San Marcos, San Marcos, TX 78666}
       
\pagestyle{empty}


\begin{document}

\lstset{language=[Objective]Caml,frame=shadowbox}

\input{qm2pi.front}

% section front matter (end)

\input{qm2pi.intro} 
 
% section introduction (end)

% \input{qm2pi.knotations} 

% section notation (end)

\input{qm2pi.process.calculi} 

% section concurrent_process_calculi_and_spatial_logics_ (end)
    
%\input{qm2pi.knots2pi} 

%\input{qm2pi.trefoil} 

%\input{qm2pi.mainthm} 

% subsection basic_interpretation (end)

%\input{qm2pi.rho.presentation} 
\subsection{The syntax and semantics of the notation system}\label{sub:the_syntax_and_semantics_of_the_notation_system} % (fold)

We now summarize a technical presentation of the calculus that
embodies our theory of dynamics. The typical presentation of such a
calculus follows the style of giving generators and relations on
them. The grammar, below, describing term constructors, freely
generates the set of processes, $\Proc$. This set is then quotiented
by a relation known as structural congruence and it is over this set
that the notion of dynamics is expressed. This presentation is
essentially that of \cite{MeredithR05} with the addition of
polyadicity and summation. For readability we have relegated some of
the technical subtleties to an appendix.

\subsubsection{Process grammar}\label{subsub:process_grammar}

\begin{mathpar}
  \inferrule* [lab=synchronization] {} {{M} \bc \pzero \;|\; x?F \;|\; x!C }
  \and
  \inferrule* [lab=abstraction] {} {{F} \bc (x)P}
  \and
  \inferrule* [lab=concretion] {} {{C} \bc \langle Q \rangle}
  \and
  \inferrule* [lab=process] {} {{P,Q} \bc M \;| \;P|Q \;|\; @{x}}
  \and
  \inferrule* [lab=name] {} {{x} \bc \quotep{P}}
\end{mathpar} 

Note that $\vec{x}$ (resp. $\vec{P}$) denotes a vector of names
(resp. processes) of length $|\vec{x}|$ (resp. $|\vec{P}|$). We adopt
the following useful abbreviations.

\begin{mathpar}
   x?(\vec{y}).P := x.(\vec{y})P \and  x\clift{\vec{P}} := x.\clift{\vec{P}}
   \and x!(y) := \lift{x}{\dropn{y}}
   \and \Pi_{i=0}^{n-1}P_i := P_0 | \ldots | P_{n-1}
\end{mathpar}

\subsubsection{Structural congruence}

\paragraph{Free and bound names and alpha-equivalence.} At the
core of structural equivalence is alpha-equivalence which identifies
process that are the same up to a change of variable. Formally, we
recognize the distinction between free and bound names. The free names
of a process, $\freenames{P}$, may be calculated recursively as
follows:

\begin{mathpar}
\freenames{\pzero} := \emptyset
  \and \\
  \freenames{x?(y).P} := \{ x \} \cup (\freenames{P} \setminus \{ y \})
  \and 
  \freenames{x!\langle P \rangle} := \{ x \} \cup \{ P \} 
  \and \\
  \freenames{P|Q} := \freenames{P} \cup \freenames{Q}
  \and \\
  \freenames{@{x}} := \{ x \}
\end{mathpar}

$\pi$
$\quotep{\pi}$

$\freenames{-} : \pi \to \mathcal{P}(\quotep{\pi})$

\begin{eqnarray*}
  \freenames{\pzero} & := & \emptyset \\
  \freenames{x?(y).P} & := & \{ x \} \cup (\freenames{P} \setminus \{ y \}) \\
  \freenames{x!\langle P \rangle} & := & \{ x \} \cup \{ P \} \\
  \freenames{P|Q} & := & \freenames{P} \cup \freenames{Q} \\
  \freenames{\dropn{x}} & := & \{ x \}
\end{eqnarray*}

The bound names of a process, $\boundnames{P}$, are those names occurring in $P$
that are not free. For example, in $x?(y).0$, the name $x$ is free, while $y$ is bound.

\begin{mathpar}
  \inferrule* [lab=monoidal-laws] {} { P|Q \equiv Q|P \and P|0 \equiv P \and P|(Q|R) \equiv (P|Q)|R }
\end{mathpar}

\begin{mathpar}
  \inferrule* [lab=alpha-equivalence] {} { (x)P \equiv (y)P\{y/x\} \and y \not\in \freenames{P} }
\end{mathpar}

\begin{definition}
Then two processes, $P,Q$, are alpha-equivalent if $P = Q\{\vec{y}/\vec{x}\}$ for
some $\vec{x} \in \boundnames{Q},\vec{y} \in \boundnames{P}$, where $Q\{\vec{y}/\vec{x}\}$
denotes the capture-avoiding substitution of $\vec{y}$ for $\vec{x}$ in $Q$.
\end{definition}

\begin{definition}
  The {\em structural congruence} \cite{SangiorgiWalker} , $\equiv$,
  between processes is the least congruence containing
  alpha-equivalence, satisfying the abelian monoid laws
  (associativity, commutativity and $\pzero$ as identity) for parallel
  composition $|$ and for summation $+$.
\end{definition}

\subsection{Name equivalence}

We take name equivalence, written $\nameeq$, to be the smallest
equivalence relation generated by the following rules.

\begin{mathpar}
\inferrule*[lab=Quote-drop]
{ }
{ \quotep{@{x}} \nameeq x }

\inferrule*[lab=Struct-equiv]
{ P \scong Q }
{ \quotep{P} \nameeq \quotep{Q} }
\end{mathpar}

The astute reader will have noticed that the mutual recursion of names
and processes imposes a mutual recursion on alpha-equivalence and
structural equivalence via name-equivalence. Fortunately, all of this
works out pleasantly and we may calculate in the natural way, free of
concern. The reader interested in the details is referred to the
appendix \ref{appendix:rho_details}.

\subsection{Substitution}

We use $\Proc$ for the set of processes, $\QProc$ for the set of
names, and $\id{\{}\vec{y} / \vec{x} \id{\}}$ to denote partial maps,
$s : \QProc \rightarrow \QProc$. A map, $s$ lifts, uniquely, to a map
on process terms, $\widehat{s} : \Proc \rightarrow \Proc$ by the
following equations.

\begin{mathpar}
  (0) \psubstp{Q}{P} := 0 \\
  (R \juxtap S) \psubstp{Q}{P}
  :=    
  (R)\psubstp{Q}{P} \juxtap (S) \psubstp{Q}{P} \\
  (x?(y).R) \psubstp{Q}{P}    
  :=    
  (x)\substp{Q}{P} (z)\concat( (R \psubstn{z}{y}) \psubstp{Q}{P} ) \\
  (\lift{x}{R}) \psubstp{Q}{P}  
  :=
  \lift{(x)\substp{Q}{P}}{ R \psubstp{Q}{P} } \\
%   (\dropn{x})  \psubstp{Q}{P}       
%   := 
%   \left\{ 
%     \begin{array}{ccc} 
%       \dropn{\quotep{Q}} & & x \nameeq \quotep{P} \\
%       \dropn{x} & & otherwise \\
%     \end{array}
%   \right. 
  (\dropn{x})  \psubstp{Q}{P}       
  := 
  \left\{ 
    \begin{array}{ccc} 
      Q & & x \nameeq \quotep{P} \\
      \dropn{x} & & otherwise \\
    \end{array}
  \right.
\end{mathpar}
 

where

\begin{eqnarray}
  (x)\id{\{} \lpquote Q \rpquote / \lpquote P \rpquote \id{\}}            = 
  \left\{ 
    \begin{array}{ccc}
      \lpquote Q \rpquote & & x \nameeq \lpquote P \rpquote \\
      x & & otherwise \\
    \end{array}
  \right. \nonumber
\end{eqnarray}

and $z$ is chosen distinct from $\quotep{P}$, $\quotep{Q}$, the free
names in $Q$, and all the names in $R$. Our $\alpha$-equivalence will
be built in the standard way from this substitution.

\begin{remark}\label{rem:no_self_referential_names}
  One consequence of these definitions is that $\forall P. \quotep{P}
  \not\in \freenames{P}$.
\end{remark}

\subsection{ Dynamic quote: an example }

Anticipating something of what's to come, consider applying the
substitution, $\widehat{\id{\{}u / z \id{\}}}$, to the following pair
of processes, $\lift{w}{y!(z)}$ and $w[ \lpquote y!(z) \rpquote ]$.

\begin{eqnarray}
	\lift{w}{y!(z)}\widehat{\id{\{}u / z \id{\}}}
		& = &
		\lift{w}{y!(u)} \nonumber\\
	w[ \lpquote y!(z) \rpquote ] \widehat{ \id{\{}u / z \id{\}} }
		& = &
		w[ \lpquote y!(z) \rpquote ] \nonumber
\end{eqnarray}

Because the body of the process between quotes is impervious to
substitution, we get radically different answers. In fact, by
examining the first process in an input context,
e.g. $x?(z).\lift{w}{y!(z)}$, we see that the process under the lift
operator may be shaped by prefixed inputs binding a name inside it. In
this sense, the lift operator will be seen as a way to dynamically
construct processes before reifying them as names.

Finally equipped with these standard features we can present the
dynamics of the calculus.

\subsubsection{Operational semantics} 

Finally, we introduce the computational dynamics. What marks these
algebras as distinct from other more traditionally studied algebraic
structures, e.g. vector spaces or polynomial rings, is the manner in
which dynamics is captured. In traditional structures, dynamics is typically
expressed through morphisms between such structures, as in linear maps
between vector spaces or morphisms between rings. In algebras
associated with the semantics of computation, the dynamics is
expressed as part of the algebraic structure itself, through a
reduction reduction relation typically denoted by $\red$. Below, we
give a recursive presentation of this relation for the calculus used
in the encoding.

$\red \subseteq \pi \times \pi$
$\red : \pi \to \mathcal{P}(\pi)$

\begin{mathpar}
  \inferrule* [lab=Comm] { \textsf{match}( x_{src}, x_{trgt} ) } { x_{trgt}?(y)P \; | \; x_{src}!\langle {Q} \rangle \red P\{\quotep{Q}/y}\} }
  \and \\
  \inferrule* [lab=Par] {{P} \red {P}'} {{{P} | {Q}} \red {{P}' | {Q}}}
  \and
  \inferrule* [lab=Equiv]{{{P} \scong {P}'} \andalso {{P}' \red {Q}'} \andalso {{Q}' \scong {Q}}}{{P} \red {Q}}
\end{mathpar}

\begin{eqnarray*}
  match_{\equiv} (\quotep{P},\quotep{Q}) & := & P \equiv Q \\
  match_{\dagger}(\quotep{P},\quotep{Q}) & := & \forall R. P|Q \red^{*} R => R \red^{*} 0 \\
  match_{K}(\quotep{P},\quotep{Q}) & := & K \mbox{ for some context } K
\end{eqnarray*}

$u?(x)P | u!\langle Q \rangle \red P\{\quotep{Q}/x\}$

%We write $\wred$ for $\red^*$, and $P\red$ if $\exists Q $ such that $ P \red Q$.
We write $P\red$ if $\exists Q $ such that $ P \red Q$ and $P\not\red$, otherwise.

\section{Replication}

As mentioned before, it is known that replication (and hence
recursion) can be implemented in a higher-order process algebra
\cite{SangiorgiWalker}. As our first example of calculation with the
machinery thus far presented we give the construction explicitly in
the {\rhoc}.

\begin{eqnarray}
	D_{x} & := & \prefix{x}{y}{(\binpar{\outputp{x}{y}}{@{y}})} \nonumber\\
	\bangp_{x}{P} & := & \binpar{{x}!\langle{\binpar{D_{x}}{P}}\rangle}{D_{x}} \nonumber
\end{eqnarray}

\begin{eqnarray}
	\bangp_{x}{P} & & \nonumber\\
	=
	& {x}!\langle{(\prefix{x}{y}{(\outputp{x}{y} | @{y})) | P}}\rangle 
	      | \prefix{x}{y}{(\outputp{x}{y} | @{y})} & \nonumber\\
	\red
	& (\outputp{x}{y} | @{y})\substn{\quotep{(\prefix{x}{y}{(@{y} | \outputp{x}{y})) | P}}}{y} & \nonumber\\
	=
	& \outputp{x}{\quotep{(\prefix{x}{y}{(\outputp{x}{y} | @{y})) | P}}}
	  | {(\prefix{x}{y}{(\outputp{x}{y} | @{y})) | P}} & \nonumber\\
	\red
	& \ldots & \nonumber\\
	\red^*
	& P | P | \ldots & \nonumber
\end{eqnarray}

Of course, this encoding, as an implementation, runs away, unfolding
$\bangp{P}$ eagerly. A lazier and more implementable replication
operator, restricted to input-guarded processes, may be obtained as follows.

\begin{eqnarray}
\bangp{\prefix{u}{v}{P}} 
	:= 
	\binpar{\lift{x}{\prefix{u}{v}{(\binpar{D(x)}{P})}}}{D(x)} \nonumber
\end{eqnarray}

\begin{remark}
  Note that the lazier definition still does not deal with summation
  or mixed summation (i.e. sums over input and output). The reader is
  invited to construct definitions of replication that deal with these
  features. 

  Further, the definitions are parameterized in a name, $x$. Can you,
  gentle reader, make a definition that eliminates this parameter and
  guarantees no accidental interaction between the replication
  machinery and the process being replicated -- i.e. no accidental
  sharing of names used by the process to get its work done and the
  name(s) used by the replication to effect copying. This latter
  revision of the definition of replication is crucial to obtaining
  the expected identity $!!P \sim !P$.
\end{remark}

\begin{remark}\label{rem:paradoxical_combinator}
  The reader familiar with the lambda calculus will have noticed the
  similarity between $D$ and the paradoxical combinator.

  [Ed. note: the existence of this seems to suggest we have to be more
  restrictive on the set of processes and names we admit if we are to
  support no-cloning.]
\end{remark}

\subsubsection{Bisimulation}

The computational dynamics gives rise to another kind of equivalence,
the equivalence of computational behavior. As previously mentioned
this is typically captured \emph{via} some form of bisimulation.

% The notion we use in this paper is weak barbed bisimulation
% \cite{milner91polyadicpi}.

The notion we use in this paper is derived from weak barbed
bisimulation \cite{milner91polyadicpi}. 

\begin{definition}
An \emph{observation relation}, $\downarrow_{\mathcal N}$, over a set
of names, $\mathcal N$, is the smallest relation satisfying the rules
below.

\infrule[Out-barb]{y \in {\mathcal N}, \; x \nameeq y}
		  {\outputp{x}{v} \downarrow_{\mathcal N} x}
\infrule[Par-barb]{\mbox{$P\downarrow_{\mathcal N} x$ or $Q\downarrow_{\mathcal N} x$}}
		  {\binpar{P}{Q} \downarrow_{\mathcal N} x}

We write $P \Downarrow_{\mathcal N} x$ if there is $Q$ such that 
$P \wred Q$ and $Q \downarrow_{\mathcal N} x$.
\end{definition}

\begin{definition}
%\label{def.bbisim}
An  ${\mathcal N}$-\emph{barbed bisimulation} over a set of names, ${\mathcal N}$, is a symmetric binary relation 
${\mathcal S}_{\mathcal N}$ between agents such that $P\rel{S}_{\mathcal N}Q$ implies:
\begin{enumerate}
\item If $P \red P'$ then $Q \wred Q'$ and $P'\rel{S}_{\mathcal N} Q'$.
\item If $P\downarrow_{\mathcal N} x$, then $Q\Downarrow_{\mathcal N} x$.
\end{enumerate}
$P$ is ${\mathcal N}$-barbed bisimilar to $Q$, written
$P \wbbisim_{\mathcal N} Q$, if $P \rel{S}_{\mathcal N} Q$ for some ${\mathcal N}$-barbed bisimulation ${\mathcal S}_{\mathcal N}$.
\end{definition}

$\mathcal{R} \subseteq \pi \times \pi$

$P \mathcal{R} Q => \forall P'. P \red P' \Rightarrow \exists Q'. Q \red Q', P' \mathcal{R} Q'$

$P \vdash x \Rightarrow Q \vdash x$

\begin{mathpar}
  \inferrule*[lab=Out-barb]{x \nameeq y}{{y}!\langle{Q}\rangle \vdash x}
  \and
  \inferrule*[lab=Par-barb]{\mbox{$P\vdash x$ or $Q\vdash x$}}{\binpar{P}{Q} \vdash x}
\end{mathpar}

\subsubsection{Contexts}

One of the principle advantages of computational calculi like the
$\pi$-calculus is a well-defined notion of context,
contextual-equivalence and a correlation between
contextual-equivalence and notions of bisimulation. The notion of
context allows the decomposition of a process into (sub-)process and
its syntactic environment, its context. Thus, a context may be
thought of as a process with a ``hole'' (written $\Box$) in it. The
application of a context $M$ to a process $P$, written $M[P]$, is
tantamount to filling the hole in $M$ with $P$. In this paper we do
not need the full weight of this theory, but do make use of the notion
of context in the proof the main theorem. 

\begin{mathpar}
  \inferrule* [lab=summation] {} {{M_{M},M_{N}} \bc \Box \;|\; x.M_{A} \;|\; M_{M}+M_{N}}
  \and
  \inferrule* [lab=agent] {} {{M_{A}} \bc (\vec{x})M_{P} \;| \; \clift{P_0,\ldots,M_{P},\ldots,P_N}}
  \and \\
  \inferrule* [lab=process] {} {{M_{P}} \bc M_{N} \;| \;P|M_{P} }
\end{mathpar} 

\begin{mathpar}
  \inferrule* [lab=sychronization] {} {M_{N} \bc \Box \;|\; x?M_{F} \;|\; x!M_{C}}
  \and
  \inferrule* [lab=abstraction] {} {{M_{F}} \bc (x)M_{P} }
  \and
  \inferrule* [lab=concretion] {} {{M_{C}} \bc \langle M_{P} \rangle }
  \and \\
  \inferrule* [lab=process] {} {{M_{P}} \bc M_{N} \;| \;P|M_{P} }
\end{mathpar}

\begin{definition}[contextual application] Given a context $M$, and
  process $P$, we define the \emph{contextual application}, $M[P] :=
  M\{P/\Box\}$. That is, the contextual application of M to P is the
  substitution of $P$ for $\Box$ in $M$.
\end{definition}

$\meaningof{-} : L \to \mathcal{P}(\pi)$

\begin{mathpar}
  \inferrule* [lab=collection] {} {\meaningof{true} = \pi, \and \meaningof{~E} = \pi \setminus \meaningof{E}, \and \meaningof{E_{1} \& E_{2}} = \meaningof{E_{1}} \cap \meaningof{E_{2}}}
\end{mathpar}

\begin{mathpar}
  \inferrule* [lab=structure] {} {\meaningof{0} = \{ P \in \pi | P \equiv 0 \}, \and \\ \meaningof{E_1 | E_2} = \{ P \in \pi | P \equiv P_{1} | P_{2}, P_{1} \in \meaningof{E_{1}}, P_{2} \in \meaningof{E_2}\} }
\end{mathpar}

\begin{mathpar}
 \inferrule* [lab=behavior] {} {\meaningof{\langle a?b \rangle E} = \{ P \in \pi | P \equiv Q | u?(y)P', \\ \and \\\\ \and \\ \;\;\; u \in \meaningof{a}, \forall z.P'\{z/y\} \in \meaningof{E\{z/b\}}\}, \and \\ \meaningof{a!E} = \{ P \in \pi | P \equiv Q | x!\langle P' \rangle, x \in \meaningof{a} P' \in \meaningof{E}\} }
\end{mathpar}

\begin{mathpar}
 \inferrule* [lab=nominal] {} {\meaningof{\quotep{E}} = \{ \quotep{P} \in \quotep{\pi} | P \in \meaningof{E} \}, \and \meaningof{\quotep{P}} = \{ \quotep{Q} \in \quotep{\pi} | P \equiv Q \} \and \\ \meaningof{@\quotep{E}} = \{ P \in \pi | P \equiv @x, x \in \meaningof{E} \}}
\end{mathpar}

\begin{eqnarray*}
  \\
  \meaningof{-} : TS \to ST
\end{eqnarray*}

\begin{eqnarray*}
  \\
  L : TS \to ST
\end{eqnarray*}

\begin{eqnarray*}
  \\
  P \models E \iff P \in \meaningof{E}
\end{eqnarray*}

\begin{eqnarray*}
  P \approx_{L} Q \iff \forall E \in L. P \models E \iff Q \models E
\end{eqnarray*}

\begin{eqnarray*}
  P \approx_{K} Q
\end{eqnarray*}

\begin{eqnarray*}
  P \approx Q
\end{eqnarray*}

$\approx_{K} = \approx = \approx_{L}$

\subsubsection{Contextual duality}

Note that contexts extend the quotation operation to a family of
operations from processes to names. Given a context, $M$, we can
define a \emph{nominal context}, $\quotep{M}$ by $\quotep{M}[P] :=
\quotep{M[P]}$. To foreshadow what is to come we observe that these
operations enjoy a duality with processes very much like the duality
between vectors and maps from vectors to scalars.

Further, because the calculus is essentially higher-order, we have a
correspondence between contexts and processes. More specifically,
given a name $x$ and a context $M$ we can construct $M^{*}_{x}$ such
that 

\begin{mathpar}
  M^{*}_{x} | \lift{x}{P} \red M[P]
\end{mathpar}

namely,

\begin{mathpar}
  M^{*}_{x} := x?(u).M[\dropn{u}]
\end{mathpar}

The dependence of $M^{*}_{x}$ on a name makes it an abstraction, 

\begin{mathpar}
  M^{*} := (x)x?(u).M[\dropn{u}]
\end{mathpar}

\subsection{Additional notation}

It will sometimes be convenient to denote the process a name
quotes. We already have the notation $x = \quotep{P}$, but it will be
convenient to introduce an alternate notation, $\procn{x}$, when we
want to emphasize the connection to the use of the name. Note that, by
virtue of name equivalence, $\quotep{\procn{x}} \nameeq x$; so, the
notation is consistent with previous definitions.

Further, because names have structure it is possible to effect
substitutions on the basis of that structure. This means we need to
upgrade our notation for substitutions, which we accomplish by
adapting comprehension notation. Thus,

\begin{mathpar}
  P\{ y / x : x \in S \}
\end{mathpar}

is interpreted to mean the process derived from P by replacing (in a
capture-avoiding manner) each occurrence of $x$ in $S$ by $y$. For example,

\begin{mathpar}
  P\{ \quotep{\procn{x}|\procn{x}} / x : x \in \freenames{P} \}
\end{mathpar}

will replace each (occurrence) of a free name $x$ in $P$ by
$\quotep{\procn{x}|\procn{x}}$.

Also, we will avail ourselves of the notation $x^{L}$ and $x^{R}$ to
denote injections of a name into disjoint copies of the name
space. There are numerous ways to accomplish this. One example can be
found in \cite{MeredithR05}. This notation overloads to vectors of
names: $\vec{x}^{\pi} := (x_{i}^{\pi} \; : \; 0 \leq i < |\vec{x}| )$ where $\pi \in \{L,R\}$.

We also use $P^{\Box} := P|\Box$.

In \cite{MeredithR05} an interpretation of the new operator is
given. It turns out that there are several possible interpretations
all enjoying the requisite algebraic properties of the operator (see
\cite{milner91polyadicpi}). We will therefore make liberal use of
$(\nu\; \vec{x})P$.

% subsection the_syntax_and_semantics_of_the_notation_system (end)   

\input{qm2pi.qmops} 

\input{qm2pi.sterngerlach} 

\input{qm2pi.metric} 

% section concurrent_process_calculi (end)

%\input{qm2pi.proofsketch}

% section proof sketch (end)

%\input{qm2pi.slviaknots} 

% section spatial logic via knots (end)

\input{qm2pi.conclusion}

% section conclusion (end)

%\input{qm2pi.dtcodes} 

% section wiring algorithm (end)

\input{qm2pi.ack} 

% section acknowledgments (end)

\newpage


\bibliographystyle{plain}   
\bibliography{../../biblios/main.bib}

\input{qm2pi.rhodetails}

\end{document}

 

\documentclass[12pt]{llncs}
%\documentclass{jktr}

\usepackage[pdftex]{hyperref}                   
\usepackage {listings}
\usepackage {mathpartir}
\usepackage{bcprules}
%\usepackage{listings}
                       
\usepackage{graphicx} 
%\usepackage[margins=2.5cm,nohead,nofoot]{geometry}
%\usepackage{geometry}
\usepackage{amsfonts}
\usepackage{amstext}
\usepackage{latexsym}
\usepackage{amssymb}
\usepackage{color}


%\include{myPreamble}
\include{qm2pi.local} 

%\ifpdf
%\usepackage[pdftex]{graphicx}
%\else
%\usepackage{graphicx}
%\fi

 % \ifpdf
%  \usepackage{pdfsync}
%  \if


%\title{Brief Article}
%\author{David F. Snyder}
%\author{L.G. Meredith}

%\address{Dept. of Math., Texas State University--San Marcos, San Marcos, TX 78666}
       
\pagestyle{empty}


\begin{document}

\lstset{language=[Objective]Caml,frame=shadowbox}

\input{qm2pi.front}

% section front matter (end)

\input{qm2pi.intro} 
 
% section introduction (end)

% \input{qm2pi.knotations} 

% section notation (end)

\input{qm2pi.process.calculi} 

% section concurrent_process_calculi_and_spatial_logics_ (end)
    
%\input{qm2pi.knots2pi} 

%\input{qm2pi.trefoil} 

%\input{qm2pi.mainthm} 

% subsection basic_interpretation (end)

%\input{qm2pi.rho.presentation} 
\subsection{The syntax and semantics of the notation system}\label{sub:the_syntax_and_semantics_of_the_notation_system} % (fold)

We now summarize a technical presentation of the calculus that
embodies our theory of dynamics. The typical presentation of such a
calculus follows the style of giving generators and relations on
them. The grammar, below, describing term constructors, freely
generates the set of processes, $\Proc$. This set is then quotiented
by a relation known as structural congruence and it is over this set
that the notion of dynamics is expressed. This presentation is
essentially that of \cite{MeredithR05} with the addition of
polyadicity and summation. For readability we have relegated some of
the technical subtleties to an appendix.

\subsubsection{Process grammar}\label{subsub:process_grammar}

\begin{mathpar}
  \inferrule* [lab=synchronization] {} {{M} \bc \pzero \;|\; x?F \;|\; x!C }
  \and
  \inferrule* [lab=abstraction] {} {{F} \bc (x)P}
  \and
  \inferrule* [lab=concretion] {} {{C} \bc \langle Q \rangle}
  \and
  \inferrule* [lab=process] {} {{P,Q} \bc M \;| \;P|Q \;|\; @{x}}
  \and
  \inferrule* [lab=name] {} {{x} \bc \quotep{P}}
\end{mathpar} 

Note that $\vec{x}$ (resp. $\vec{P}$) denotes a vector of names
(resp. processes) of length $|\vec{x}|$ (resp. $|\vec{P}|$). We adopt
the following useful abbreviations.

\begin{mathpar}
   x?(\vec{y}).P := x.(\vec{y})P \and  x\clift{\vec{P}} := x.\clift{\vec{P}}
   \and x!(y) := \lift{x}{\dropn{y}}
   \and \Pi_{i=0}^{n-1}P_i := P_0 | \ldots | P_{n-1}
\end{mathpar}

\subsubsection{Structural congruence}

\paragraph{Free and bound names and alpha-equivalence.} At the
core of structural equivalence is alpha-equivalence which identifies
process that are the same up to a change of variable. Formally, we
recognize the distinction between free and bound names. The free names
of a process, $\freenames{P}$, may be calculated recursively as
follows:

\begin{mathpar}
\freenames{\pzero} := \emptyset
  \and \\
  \freenames{x?(y).P} := \{ x \} \cup (\freenames{P} \setminus \{ y \})
  \and 
  \freenames{x!\langle P \rangle} := \{ x \} \cup \{ P \} 
  \and \\
  \freenames{P|Q} := \freenames{P} \cup \freenames{Q}
  \and \\
  \freenames{@{x}} := \{ x \}
\end{mathpar}

$\pi$
$\quotep{\pi}$

$\freenames{-} : \pi \to \mathcal{P}(\quotep{\pi})$

\begin{eqnarray*}
  \freenames{\pzero} & := & \emptyset \\
  \freenames{x?(y).P} & := & \{ x \} \cup (\freenames{P} \setminus \{ y \}) \\
  \freenames{x!\langle P \rangle} & := & \{ x \} \cup \{ P \} \\
  \freenames{P|Q} & := & \freenames{P} \cup \freenames{Q} \\
  \freenames{\dropn{x}} & := & \{ x \}
\end{eqnarray*}

The bound names of a process, $\boundnames{P}$, are those names occurring in $P$
that are not free. For example, in $x?(y).0$, the name $x$ is free, while $y$ is bound.

\begin{mathpar}
  \inferrule* [lab=monoidal-laws] {} { P|Q \equiv Q|P \and P|0 \equiv P \and P|(Q|R) \equiv (P|Q)|R }
\end{mathpar}

\begin{mathpar}
  \inferrule* [lab=alpha-equivalence] {} { (x)P \equiv (y)P\{y/x\} \and y \not\in \freenames{P} }
\end{mathpar}

\begin{definition}
Then two processes, $P,Q$, are alpha-equivalent if $P = Q\{\vec{y}/\vec{x}\}$ for
some $\vec{x} \in \boundnames{Q},\vec{y} \in \boundnames{P}$, where $Q\{\vec{y}/\vec{x}\}$
denotes the capture-avoiding substitution of $\vec{y}$ for $\vec{x}$ in $Q$.
\end{definition}

\begin{definition}
  The {\em structural congruence} \cite{SangiorgiWalker} , $\equiv$,
  between processes is the least congruence containing
  alpha-equivalence, satisfying the abelian monoid laws
  (associativity, commutativity and $\pzero$ as identity) for parallel
  composition $|$ and for summation $+$.
\end{definition}

\subsection{Name equivalence}

We take name equivalence, written $\nameeq$, to be the smallest
equivalence relation generated by the following rules.

\begin{mathpar}
\inferrule*[lab=Quote-drop]
{ }
{ \quotep{@{x}} \nameeq x }

\inferrule*[lab=Struct-equiv]
{ P \scong Q }
{ \quotep{P} \nameeq \quotep{Q} }
\end{mathpar}

The astute reader will have noticed that the mutual recursion of names
and processes imposes a mutual recursion on alpha-equivalence and
structural equivalence via name-equivalence. Fortunately, all of this
works out pleasantly and we may calculate in the natural way, free of
concern. The reader interested in the details is referred to the
appendix \ref{appendix:rho_details}.

\subsection{Substitution}

We use $\Proc$ for the set of processes, $\QProc$ for the set of
names, and $\id{\{}\vec{y} / \vec{x} \id{\}}$ to denote partial maps,
$s : \QProc \rightarrow \QProc$. A map, $s$ lifts, uniquely, to a map
on process terms, $\widehat{s} : \Proc \rightarrow \Proc$ by the
following equations.

\begin{mathpar}
  (0) \psubstp{Q}{P} := 0 \\
  (R \juxtap S) \psubstp{Q}{P}
  :=    
  (R)\psubstp{Q}{P} \juxtap (S) \psubstp{Q}{P} \\
  (x?(y).R) \psubstp{Q}{P}    
  :=    
  (x)\substp{Q}{P} (z)\concat( (R \psubstn{z}{y}) \psubstp{Q}{P} ) \\
  (\lift{x}{R}) \psubstp{Q}{P}  
  :=
  \lift{(x)\substp{Q}{P}}{ R \psubstp{Q}{P} } \\
%   (\dropn{x})  \psubstp{Q}{P}       
%   := 
%   \left\{ 
%     \begin{array}{ccc} 
%       \dropn{\quotep{Q}} & & x \nameeq \quotep{P} \\
%       \dropn{x} & & otherwise \\
%     \end{array}
%   \right. 
  (\dropn{x})  \psubstp{Q}{P}       
  := 
  \left\{ 
    \begin{array}{ccc} 
      Q & & x \nameeq \quotep{P} \\
      \dropn{x} & & otherwise \\
    \end{array}
  \right.
\end{mathpar}
 

where

\begin{eqnarray}
  (x)\id{\{} \lpquote Q \rpquote / \lpquote P \rpquote \id{\}}            = 
  \left\{ 
    \begin{array}{ccc}
      \lpquote Q \rpquote & & x \nameeq \lpquote P \rpquote \\
      x & & otherwise \\
    \end{array}
  \right. \nonumber
\end{eqnarray}

and $z$ is chosen distinct from $\quotep{P}$, $\quotep{Q}$, the free
names in $Q$, and all the names in $R$. Our $\alpha$-equivalence will
be built in the standard way from this substitution.

\begin{remark}\label{rem:no_self_referential_names}
  One consequence of these definitions is that $\forall P. \quotep{P}
  \not\in \freenames{P}$.
\end{remark}

\subsection{ Dynamic quote: an example }

Anticipating something of what's to come, consider applying the
substitution, $\widehat{\id{\{}u / z \id{\}}}$, to the following pair
of processes, $\lift{w}{y!(z)}$ and $w[ \lpquote y!(z) \rpquote ]$.

\begin{eqnarray}
	\lift{w}{y!(z)}\widehat{\id{\{}u / z \id{\}}}
		& = &
		\lift{w}{y!(u)} \nonumber\\
	w[ \lpquote y!(z) \rpquote ] \widehat{ \id{\{}u / z \id{\}} }
		& = &
		w[ \lpquote y!(z) \rpquote ] \nonumber
\end{eqnarray}

Because the body of the process between quotes is impervious to
substitution, we get radically different answers. In fact, by
examining the first process in an input context,
e.g. $x?(z).\lift{w}{y!(z)}$, we see that the process under the lift
operator may be shaped by prefixed inputs binding a name inside it. In
this sense, the lift operator will be seen as a way to dynamically
construct processes before reifying them as names.

Finally equipped with these standard features we can present the
dynamics of the calculus.

\subsubsection{Operational semantics} 

Finally, we introduce the computational dynamics. What marks these
algebras as distinct from other more traditionally studied algebraic
structures, e.g. vector spaces or polynomial rings, is the manner in
which dynamics is captured. In traditional structures, dynamics is typically
expressed through morphisms between such structures, as in linear maps
between vector spaces or morphisms between rings. In algebras
associated with the semantics of computation, the dynamics is
expressed as part of the algebraic structure itself, through a
reduction reduction relation typically denoted by $\red$. Below, we
give a recursive presentation of this relation for the calculus used
in the encoding.

$\red \subseteq \pi \times \pi$
$\red : \pi \to \mathcal{P}(\pi)$

\begin{mathpar}
  \inferrule* [lab=Comm] { \textsf{match}( x_{src}, x_{trgt} ) } { x_{trgt}?(y)P \; | \; x_{src}!\langle {Q} \rangle \red P\{\quotep{Q}/y}\} }
  \and \\
  \inferrule* [lab=Par] {{P} \red {P}'} {{{P} | {Q}} \red {{P}' | {Q}}}
  \and
  \inferrule* [lab=Equiv]{{{P} \scong {P}'} \andalso {{P}' \red {Q}'} \andalso {{Q}' \scong {Q}}}{{P} \red {Q}}
\end{mathpar}

\begin{eqnarray*}
  match_{\equiv} (\quotep{P},\quotep{Q}) & := & P \equiv Q \\
  match_{\dagger}(\quotep{P},\quotep{Q}) & := & \forall R. P|Q \red^{*} R => R \red^{*} 0 \\
  match_{K}(\quotep{P},\quotep{Q}) & := & K \mbox{ for some context } K
\end{eqnarray*}

$u?(x)P | u!\langle Q \rangle \red P\{\quotep{Q}/x\}$

%We write $\wred$ for $\red^*$, and $P\red$ if $\exists Q $ such that $ P \red Q$.
We write $P\red$ if $\exists Q $ such that $ P \red Q$ and $P\not\red$, otherwise.

\section{Replication}

As mentioned before, it is known that replication (and hence
recursion) can be implemented in a higher-order process algebra
\cite{SangiorgiWalker}. As our first example of calculation with the
machinery thus far presented we give the construction explicitly in
the {\rhoc}.

\begin{eqnarray}
	D_{x} & := & \prefix{x}{y}{(\binpar{\outputp{x}{y}}{@{y}})} \nonumber\\
	\bangp_{x}{P} & := & \binpar{{x}!\langle{\binpar{D_{x}}{P}}\rangle}{D_{x}} \nonumber
\end{eqnarray}

\begin{eqnarray}
	\bangp_{x}{P} & & \nonumber\\
	=
	& {x}!\langle{(\prefix{x}{y}{(\outputp{x}{y} | @{y})) | P}}\rangle 
	      | \prefix{x}{y}{(\outputp{x}{y} | @{y})} & \nonumber\\
	\red
	& (\outputp{x}{y} | @{y})\substn{\quotep{(\prefix{x}{y}{(@{y} | \outputp{x}{y})) | P}}}{y} & \nonumber\\
	=
	& \outputp{x}{\quotep{(\prefix{x}{y}{(\outputp{x}{y} | @{y})) | P}}}
	  | {(\prefix{x}{y}{(\outputp{x}{y} | @{y})) | P}} & \nonumber\\
	\red
	& \ldots & \nonumber\\
	\red^*
	& P | P | \ldots & \nonumber
\end{eqnarray}

Of course, this encoding, as an implementation, runs away, unfolding
$\bangp{P}$ eagerly. A lazier and more implementable replication
operator, restricted to input-guarded processes, may be obtained as follows.

\begin{eqnarray}
\bangp{\prefix{u}{v}{P}} 
	:= 
	\binpar{\lift{x}{\prefix{u}{v}{(\binpar{D(x)}{P})}}}{D(x)} \nonumber
\end{eqnarray}

\begin{remark}
  Note that the lazier definition still does not deal with summation
  or mixed summation (i.e. sums over input and output). The reader is
  invited to construct definitions of replication that deal with these
  features. 

  Further, the definitions are parameterized in a name, $x$. Can you,
  gentle reader, make a definition that eliminates this parameter and
  guarantees no accidental interaction between the replication
  machinery and the process being replicated -- i.e. no accidental
  sharing of names used by the process to get its work done and the
  name(s) used by the replication to effect copying. This latter
  revision of the definition of replication is crucial to obtaining
  the expected identity $!!P \sim !P$.
\end{remark}

\begin{remark}\label{rem:paradoxical_combinator}
  The reader familiar with the lambda calculus will have noticed the
  similarity between $D$ and the paradoxical combinator.

  [Ed. note: the existence of this seems to suggest we have to be more
  restrictive on the set of processes and names we admit if we are to
  support no-cloning.]
\end{remark}

\subsubsection{Bisimulation}

The computational dynamics gives rise to another kind of equivalence,
the equivalence of computational behavior. As previously mentioned
this is typically captured \emph{via} some form of bisimulation.

% The notion we use in this paper is weak barbed bisimulation
% \cite{milner91polyadicpi}.

The notion we use in this paper is derived from weak barbed
bisimulation \cite{milner91polyadicpi}. 

\begin{definition}
An \emph{observation relation}, $\downarrow_{\mathcal N}$, over a set
of names, $\mathcal N$, is the smallest relation satisfying the rules
below.

\infrule[Out-barb]{y \in {\mathcal N}, \; x \nameeq y}
		  {\outputp{x}{v} \downarrow_{\mathcal N} x}
\infrule[Par-barb]{\mbox{$P\downarrow_{\mathcal N} x$ or $Q\downarrow_{\mathcal N} x$}}
		  {\binpar{P}{Q} \downarrow_{\mathcal N} x}

We write $P \Downarrow_{\mathcal N} x$ if there is $Q$ such that 
$P \wred Q$ and $Q \downarrow_{\mathcal N} x$.
\end{definition}

\begin{definition}
%\label{def.bbisim}
An  ${\mathcal N}$-\emph{barbed bisimulation} over a set of names, ${\mathcal N}$, is a symmetric binary relation 
${\mathcal S}_{\mathcal N}$ between agents such that $P\rel{S}_{\mathcal N}Q$ implies:
\begin{enumerate}
\item If $P \red P'$ then $Q \wred Q'$ and $P'\rel{S}_{\mathcal N} Q'$.
\item If $P\downarrow_{\mathcal N} x$, then $Q\Downarrow_{\mathcal N} x$.
\end{enumerate}
$P$ is ${\mathcal N}$-barbed bisimilar to $Q$, written
$P \wbbisim_{\mathcal N} Q$, if $P \rel{S}_{\mathcal N} Q$ for some ${\mathcal N}$-barbed bisimulation ${\mathcal S}_{\mathcal N}$.
\end{definition}

$\mathcal{R} \subseteq \pi \times \pi$

$P \mathcal{R} Q => \forall P'. P \red P' \Rightarrow \exists Q'. Q \red Q', P' \mathcal{R} Q'$

$P \vdash x \Rightarrow Q \vdash x$

\begin{mathpar}
  \inferrule*[lab=Out-barb]{x \nameeq y}{{y}!\langle{Q}\rangle \vdash x}
  \and
  \inferrule*[lab=Par-barb]{\mbox{$P\vdash x$ or $Q\vdash x$}}{\binpar{P}{Q} \vdash x}
\end{mathpar}

\subsubsection{Contexts}

One of the principle advantages of computational calculi like the
$\pi$-calculus is a well-defined notion of context,
contextual-equivalence and a correlation between
contextual-equivalence and notions of bisimulation. The notion of
context allows the decomposition of a process into (sub-)process and
its syntactic environment, its context. Thus, a context may be
thought of as a process with a ``hole'' (written $\Box$) in it. The
application of a context $M$ to a process $P$, written $M[P]$, is
tantamount to filling the hole in $M$ with $P$. In this paper we do
not need the full weight of this theory, but do make use of the notion
of context in the proof the main theorem. 

\begin{mathpar}
  \inferrule* [lab=summation] {} {{M_{M},M_{N}} \bc \Box \;|\; x.M_{A} \;|\; M_{M}+M_{N}}
  \and
  \inferrule* [lab=agent] {} {{M_{A}} \bc (\vec{x})M_{P} \;| \; \clift{P_0,\ldots,M_{P},\ldots,P_N}}
  \and \\
  \inferrule* [lab=process] {} {{M_{P}} \bc M_{N} \;| \;P|M_{P} }
\end{mathpar} 

\begin{mathpar}
  \inferrule* [lab=sychronization] {} {M_{N} \bc \Box \;|\; x?M_{F} \;|\; x!M_{C}}
  \and
  \inferrule* [lab=abstraction] {} {{M_{F}} \bc (x)M_{P} }
  \and
  \inferrule* [lab=concretion] {} {{M_{C}} \bc \langle M_{P} \rangle }
  \and \\
  \inferrule* [lab=process] {} {{M_{P}} \bc M_{N} \;| \;P|M_{P} }
\end{mathpar}

\begin{definition}[contextual application] Given a context $M$, and
  process $P$, we define the \emph{contextual application}, $M[P] :=
  M\{P/\Box\}$. That is, the contextual application of M to P is the
  substitution of $P$ for $\Box$ in $M$.
\end{definition}

$\meaningof{-} : L \to \mathcal{P}(\pi)$

\begin{mathpar}
  \inferrule* [lab=collection] {} {\meaningof{true} = \pi, \and \meaningof{~E} = \pi \setminus \meaningof{E}, \and \meaningof{E_{1} \& E_{2}} = \meaningof{E_{1}} \cap \meaningof{E_{2}}}
\end{mathpar}

\begin{mathpar}
  \inferrule* [lab=structure] {} {\meaningof{0} = \{ P \in \pi | P \equiv 0 \}, \and \\ \meaningof{E_1 | E_2} = \{ P \in \pi | P \equiv P_{1} | P_{2}, P_{1} \in \meaningof{E_{1}}, P_{2} \in \meaningof{E_2}\} }
\end{mathpar}

\begin{mathpar}
 \inferrule* [lab=behavior] {} {\meaningof{\langle a?b \rangle E} = \{ P \in \pi | P \equiv Q | u?(y)P', \\ \and \\\\ \and \\ \;\;\; u \in \meaningof{a}, \forall z.P'\{z/y\} \in \meaningof{E\{z/b\}}\}, \and \\ \meaningof{a!E} = \{ P \in \pi | P \equiv Q | x!\langle P' \rangle, x \in \meaningof{a} P' \in \meaningof{E}\} }
\end{mathpar}

\begin{mathpar}
 \inferrule* [lab=nominal] {} {\meaningof{\quotep{E}} = \{ \quotep{P} \in \quotep{\pi} | P \in \meaningof{E} \}, \and \meaningof{\quotep{P}} = \{ \quotep{Q} \in \quotep{\pi} | P \equiv Q \} \and \\ \meaningof{@\quotep{E}} = \{ P \in \pi | P \equiv @x, x \in \meaningof{E} \}}
\end{mathpar}

\begin{eqnarray*}
  \\
  \meaningof{-} : TS \to ST
\end{eqnarray*}

\begin{eqnarray*}
  \\
  L : TS \to ST
\end{eqnarray*}

\begin{eqnarray*}
  \\
  P \models E \iff P \in \meaningof{E}
\end{eqnarray*}

\begin{eqnarray*}
  P \approx_{L} Q \iff \forall E \in L. P \models E \iff Q \models E
\end{eqnarray*}

\begin{eqnarray*}
  P \approx_{K} Q
\end{eqnarray*}

\begin{eqnarray*}
  P \approx Q
\end{eqnarray*}

$\approx_{K} = \approx = \approx_{L}$

\subsubsection{Contextual duality}

Note that contexts extend the quotation operation to a family of
operations from processes to names. Given a context, $M$, we can
define a \emph{nominal context}, $\quotep{M}$ by $\quotep{M}[P] :=
\quotep{M[P]}$. To foreshadow what is to come we observe that these
operations enjoy a duality with processes very much like the duality
between vectors and maps from vectors to scalars.

Further, because the calculus is essentially higher-order, we have a
correspondence between contexts and processes. More specifically,
given a name $x$ and a context $M$ we can construct $M^{*}_{x}$ such
that 

\begin{mathpar}
  M^{*}_{x} | \lift{x}{P} \red M[P]
\end{mathpar}

namely,

\begin{mathpar}
  M^{*}_{x} := x?(u).M[\dropn{u}]
\end{mathpar}

The dependence of $M^{*}_{x}$ on a name makes it an abstraction, 

\begin{mathpar}
  M^{*} := (x)x?(u).M[\dropn{u}]
\end{mathpar}

\subsection{Additional notation}

It will sometimes be convenient to denote the process a name
quotes. We already have the notation $x = \quotep{P}$, but it will be
convenient to introduce an alternate notation, $\procn{x}$, when we
want to emphasize the connection to the use of the name. Note that, by
virtue of name equivalence, $\quotep{\procn{x}} \nameeq x$; so, the
notation is consistent with previous definitions.

Further, because names have structure it is possible to effect
substitutions on the basis of that structure. This means we need to
upgrade our notation for substitutions, which we accomplish by
adapting comprehension notation. Thus,

\begin{mathpar}
  P\{ y / x : x \in S \}
\end{mathpar}

is interpreted to mean the process derived from P by replacing (in a
capture-avoiding manner) each occurrence of $x$ in $S$ by $y$. For example,

\begin{mathpar}
  P\{ \quotep{\procn{x}|\procn{x}} / x : x \in \freenames{P} \}
\end{mathpar}

will replace each (occurrence) of a free name $x$ in $P$ by
$\quotep{\procn{x}|\procn{x}}$.

Also, we will avail ourselves of the notation $x^{L}$ and $x^{R}$ to
denote injections of a name into disjoint copies of the name
space. There are numerous ways to accomplish this. One example can be
found in \cite{MeredithR05}. This notation overloads to vectors of
names: $\vec{x}^{\pi} := (x_{i}^{\pi} \; : \; 0 \leq i < |\vec{x}| )$ where $\pi \in \{L,R\}$.

We also use $P^{\Box} := P|\Box$.

In \cite{MeredithR05} an interpretation of the new operator is
given. It turns out that there are several possible interpretations
all enjoying the requisite algebraic properties of the operator (see
\cite{milner91polyadicpi}). We will therefore make liberal use of
$(\nu\; \vec{x})P$.

% subsection the_syntax_and_semantics_of_the_notation_system (end)   

\input{qm2pi.qmops} 

\input{qm2pi.sterngerlach} 

\input{qm2pi.metric} 

% section concurrent_process_calculi (end)

%\input{qm2pi.proofsketch}

% section proof sketch (end)

%\input{qm2pi.slviaknots} 

% section spatial logic via knots (end)

\input{qm2pi.conclusion}

% section conclusion (end)

%\input{qm2pi.dtcodes} 

% section wiring algorithm (end)

\input{qm2pi.ack} 

% section acknowledgments (end)

\newpage


\bibliographystyle{plain}   
\bibliography{../../biblios/main.bib}

\input{qm2pi.rhodetails}

\end{document}

 

% section concurrent_process_calculi (end)

%\documentclass[12pt]{llncs}
%\documentclass{jktr}

\usepackage[pdftex]{hyperref}                   
\usepackage {listings}
\usepackage {mathpartir}
\usepackage{bcprules}
%\usepackage{listings}
                       
\usepackage{graphicx} 
%\usepackage[margins=2.5cm,nohead,nofoot]{geometry}
%\usepackage{geometry}
\usepackage{amsfonts}
\usepackage{amstext}
\usepackage{latexsym}
\usepackage{amssymb}
\usepackage{color}


%\include{myPreamble}
\include{qm2pi.local} 

%\ifpdf
%\usepackage[pdftex]{graphicx}
%\else
%\usepackage{graphicx}
%\fi

 % \ifpdf
%  \usepackage{pdfsync}
%  \if


%\title{Brief Article}
%\author{David F. Snyder}
%\author{L.G. Meredith}

%\address{Dept. of Math., Texas State University--San Marcos, San Marcos, TX 78666}
       
\pagestyle{empty}


\begin{document}

\lstset{language=[Objective]Caml,frame=shadowbox}

\input{qm2pi.front}

% section front matter (end)

\input{qm2pi.intro} 
 
% section introduction (end)

% \input{qm2pi.knotations} 

% section notation (end)

\input{qm2pi.process.calculi} 

% section concurrent_process_calculi_and_spatial_logics_ (end)
    
%\input{qm2pi.knots2pi} 

%\input{qm2pi.trefoil} 

%\input{qm2pi.mainthm} 

% subsection basic_interpretation (end)

%\input{qm2pi.rho.presentation} 
\subsection{The syntax and semantics of the notation system}\label{sub:the_syntax_and_semantics_of_the_notation_system} % (fold)

We now summarize a technical presentation of the calculus that
embodies our theory of dynamics. The typical presentation of such a
calculus follows the style of giving generators and relations on
them. The grammar, below, describing term constructors, freely
generates the set of processes, $\Proc$. This set is then quotiented
by a relation known as structural congruence and it is over this set
that the notion of dynamics is expressed. This presentation is
essentially that of \cite{MeredithR05} with the addition of
polyadicity and summation. For readability we have relegated some of
the technical subtleties to an appendix.

\subsubsection{Process grammar}\label{subsub:process_grammar}

\begin{mathpar}
  \inferrule* [lab=synchronization] {} {{M} \bc \pzero \;|\; x?F \;|\; x!C }
  \and
  \inferrule* [lab=abstraction] {} {{F} \bc (x)P}
  \and
  \inferrule* [lab=concretion] {} {{C} \bc \langle Q \rangle}
  \and
  \inferrule* [lab=process] {} {{P,Q} \bc M \;| \;P|Q \;|\; @{x}}
  \and
  \inferrule* [lab=name] {} {{x} \bc \quotep{P}}
\end{mathpar} 

Note that $\vec{x}$ (resp. $\vec{P}$) denotes a vector of names
(resp. processes) of length $|\vec{x}|$ (resp. $|\vec{P}|$). We adopt
the following useful abbreviations.

\begin{mathpar}
   x?(\vec{y}).P := x.(\vec{y})P \and  x\clift{\vec{P}} := x.\clift{\vec{P}}
   \and x!(y) := \lift{x}{\dropn{y}}
   \and \Pi_{i=0}^{n-1}P_i := P_0 | \ldots | P_{n-1}
\end{mathpar}

\subsubsection{Structural congruence}

\paragraph{Free and bound names and alpha-equivalence.} At the
core of structural equivalence is alpha-equivalence which identifies
process that are the same up to a change of variable. Formally, we
recognize the distinction between free and bound names. The free names
of a process, $\freenames{P}$, may be calculated recursively as
follows:

\begin{mathpar}
\freenames{\pzero} := \emptyset
  \and \\
  \freenames{x?(y).P} := \{ x \} \cup (\freenames{P} \setminus \{ y \})
  \and 
  \freenames{x!\langle P \rangle} := \{ x \} \cup \{ P \} 
  \and \\
  \freenames{P|Q} := \freenames{P} \cup \freenames{Q}
  \and \\
  \freenames{@{x}} := \{ x \}
\end{mathpar}

$\pi$
$\quotep{\pi}$

$\freenames{-} : \pi \to \mathcal{P}(\quotep{\pi})$

\begin{eqnarray*}
  \freenames{\pzero} & := & \emptyset \\
  \freenames{x?(y).P} & := & \{ x \} \cup (\freenames{P} \setminus \{ y \}) \\
  \freenames{x!\langle P \rangle} & := & \{ x \} \cup \{ P \} \\
  \freenames{P|Q} & := & \freenames{P} \cup \freenames{Q} \\
  \freenames{\dropn{x}} & := & \{ x \}
\end{eqnarray*}

The bound names of a process, $\boundnames{P}$, are those names occurring in $P$
that are not free. For example, in $x?(y).0$, the name $x$ is free, while $y$ is bound.

\begin{mathpar}
  \inferrule* [lab=monoidal-laws] {} { P|Q \equiv Q|P \and P|0 \equiv P \and P|(Q|R) \equiv (P|Q)|R }
\end{mathpar}

\begin{mathpar}
  \inferrule* [lab=alpha-equivalence] {} { (x)P \equiv (y)P\{y/x\} \and y \not\in \freenames{P} }
\end{mathpar}

\begin{definition}
Then two processes, $P,Q$, are alpha-equivalent if $P = Q\{\vec{y}/\vec{x}\}$ for
some $\vec{x} \in \boundnames{Q},\vec{y} \in \boundnames{P}$, where $Q\{\vec{y}/\vec{x}\}$
denotes the capture-avoiding substitution of $\vec{y}$ for $\vec{x}$ in $Q$.
\end{definition}

\begin{definition}
  The {\em structural congruence} \cite{SangiorgiWalker} , $\equiv$,
  between processes is the least congruence containing
  alpha-equivalence, satisfying the abelian monoid laws
  (associativity, commutativity and $\pzero$ as identity) for parallel
  composition $|$ and for summation $+$.
\end{definition}

\subsection{Name equivalence}

We take name equivalence, written $\nameeq$, to be the smallest
equivalence relation generated by the following rules.

\begin{mathpar}
\inferrule*[lab=Quote-drop]
{ }
{ \quotep{@{x}} \nameeq x }

\inferrule*[lab=Struct-equiv]
{ P \scong Q }
{ \quotep{P} \nameeq \quotep{Q} }
\end{mathpar}

The astute reader will have noticed that the mutual recursion of names
and processes imposes a mutual recursion on alpha-equivalence and
structural equivalence via name-equivalence. Fortunately, all of this
works out pleasantly and we may calculate in the natural way, free of
concern. The reader interested in the details is referred to the
appendix \ref{appendix:rho_details}.

\subsection{Substitution}

We use $\Proc$ for the set of processes, $\QProc$ for the set of
names, and $\id{\{}\vec{y} / \vec{x} \id{\}}$ to denote partial maps,
$s : \QProc \rightarrow \QProc$. A map, $s$ lifts, uniquely, to a map
on process terms, $\widehat{s} : \Proc \rightarrow \Proc$ by the
following equations.

\begin{mathpar}
  (0) \psubstp{Q}{P} := 0 \\
  (R \juxtap S) \psubstp{Q}{P}
  :=    
  (R)\psubstp{Q}{P} \juxtap (S) \psubstp{Q}{P} \\
  (x?(y).R) \psubstp{Q}{P}    
  :=    
  (x)\substp{Q}{P} (z)\concat( (R \psubstn{z}{y}) \psubstp{Q}{P} ) \\
  (\lift{x}{R}) \psubstp{Q}{P}  
  :=
  \lift{(x)\substp{Q}{P}}{ R \psubstp{Q}{P} } \\
%   (\dropn{x})  \psubstp{Q}{P}       
%   := 
%   \left\{ 
%     \begin{array}{ccc} 
%       \dropn{\quotep{Q}} & & x \nameeq \quotep{P} \\
%       \dropn{x} & & otherwise \\
%     \end{array}
%   \right. 
  (\dropn{x})  \psubstp{Q}{P}       
  := 
  \left\{ 
    \begin{array}{ccc} 
      Q & & x \nameeq \quotep{P} \\
      \dropn{x} & & otherwise \\
    \end{array}
  \right.
\end{mathpar}
 

where

\begin{eqnarray}
  (x)\id{\{} \lpquote Q \rpquote / \lpquote P \rpquote \id{\}}            = 
  \left\{ 
    \begin{array}{ccc}
      \lpquote Q \rpquote & & x \nameeq \lpquote P \rpquote \\
      x & & otherwise \\
    \end{array}
  \right. \nonumber
\end{eqnarray}

and $z$ is chosen distinct from $\quotep{P}$, $\quotep{Q}$, the free
names in $Q$, and all the names in $R$. Our $\alpha$-equivalence will
be built in the standard way from this substitution.

\begin{remark}\label{rem:no_self_referential_names}
  One consequence of these definitions is that $\forall P. \quotep{P}
  \not\in \freenames{P}$.
\end{remark}

\subsection{ Dynamic quote: an example }

Anticipating something of what's to come, consider applying the
substitution, $\widehat{\id{\{}u / z \id{\}}}$, to the following pair
of processes, $\lift{w}{y!(z)}$ and $w[ \lpquote y!(z) \rpquote ]$.

\begin{eqnarray}
	\lift{w}{y!(z)}\widehat{\id{\{}u / z \id{\}}}
		& = &
		\lift{w}{y!(u)} \nonumber\\
	w[ \lpquote y!(z) \rpquote ] \widehat{ \id{\{}u / z \id{\}} }
		& = &
		w[ \lpquote y!(z) \rpquote ] \nonumber
\end{eqnarray}

Because the body of the process between quotes is impervious to
substitution, we get radically different answers. In fact, by
examining the first process in an input context,
e.g. $x?(z).\lift{w}{y!(z)}$, we see that the process under the lift
operator may be shaped by prefixed inputs binding a name inside it. In
this sense, the lift operator will be seen as a way to dynamically
construct processes before reifying them as names.

Finally equipped with these standard features we can present the
dynamics of the calculus.

\subsubsection{Operational semantics} 

Finally, we introduce the computational dynamics. What marks these
algebras as distinct from other more traditionally studied algebraic
structures, e.g. vector spaces or polynomial rings, is the manner in
which dynamics is captured. In traditional structures, dynamics is typically
expressed through morphisms between such structures, as in linear maps
between vector spaces or morphisms between rings. In algebras
associated with the semantics of computation, the dynamics is
expressed as part of the algebraic structure itself, through a
reduction reduction relation typically denoted by $\red$. Below, we
give a recursive presentation of this relation for the calculus used
in the encoding.

$\red \subseteq \pi \times \pi$
$\red : \pi \to \mathcal{P}(\pi)$

\begin{mathpar}
  \inferrule* [lab=Comm] { \textsf{match}( x_{src}, x_{trgt} ) } { x_{trgt}?(y)P \; | \; x_{src}!\langle {Q} \rangle \red P\{\quotep{Q}/y}\} }
  \and \\
  \inferrule* [lab=Par] {{P} \red {P}'} {{{P} | {Q}} \red {{P}' | {Q}}}
  \and
  \inferrule* [lab=Equiv]{{{P} \scong {P}'} \andalso {{P}' \red {Q}'} \andalso {{Q}' \scong {Q}}}{{P} \red {Q}}
\end{mathpar}

\begin{eqnarray*}
  match_{\equiv} (\quotep{P},\quotep{Q}) & := & P \equiv Q \\
  match_{\dagger}(\quotep{P},\quotep{Q}) & := & \forall R. P|Q \red^{*} R => R \red^{*} 0 \\
  match_{K}(\quotep{P},\quotep{Q}) & := & K \mbox{ for some context } K
\end{eqnarray*}

$u?(x)P | u!\langle Q \rangle \red P\{\quotep{Q}/x\}$

%We write $\wred$ for $\red^*$, and $P\red$ if $\exists Q $ such that $ P \red Q$.
We write $P\red$ if $\exists Q $ such that $ P \red Q$ and $P\not\red$, otherwise.

\section{Replication}

As mentioned before, it is known that replication (and hence
recursion) can be implemented in a higher-order process algebra
\cite{SangiorgiWalker}. As our first example of calculation with the
machinery thus far presented we give the construction explicitly in
the {\rhoc}.

\begin{eqnarray}
	D_{x} & := & \prefix{x}{y}{(\binpar{\outputp{x}{y}}{@{y}})} \nonumber\\
	\bangp_{x}{P} & := & \binpar{{x}!\langle{\binpar{D_{x}}{P}}\rangle}{D_{x}} \nonumber
\end{eqnarray}

\begin{eqnarray}
	\bangp_{x}{P} & & \nonumber\\
	=
	& {x}!\langle{(\prefix{x}{y}{(\outputp{x}{y} | @{y})) | P}}\rangle 
	      | \prefix{x}{y}{(\outputp{x}{y} | @{y})} & \nonumber\\
	\red
	& (\outputp{x}{y} | @{y})\substn{\quotep{(\prefix{x}{y}{(@{y} | \outputp{x}{y})) | P}}}{y} & \nonumber\\
	=
	& \outputp{x}{\quotep{(\prefix{x}{y}{(\outputp{x}{y} | @{y})) | P}}}
	  | {(\prefix{x}{y}{(\outputp{x}{y} | @{y})) | P}} & \nonumber\\
	\red
	& \ldots & \nonumber\\
	\red^*
	& P | P | \ldots & \nonumber
\end{eqnarray}

Of course, this encoding, as an implementation, runs away, unfolding
$\bangp{P}$ eagerly. A lazier and more implementable replication
operator, restricted to input-guarded processes, may be obtained as follows.

\begin{eqnarray}
\bangp{\prefix{u}{v}{P}} 
	:= 
	\binpar{\lift{x}{\prefix{u}{v}{(\binpar{D(x)}{P})}}}{D(x)} \nonumber
\end{eqnarray}

\begin{remark}
  Note that the lazier definition still does not deal with summation
  or mixed summation (i.e. sums over input and output). The reader is
  invited to construct definitions of replication that deal with these
  features. 

  Further, the definitions are parameterized in a name, $x$. Can you,
  gentle reader, make a definition that eliminates this parameter and
  guarantees no accidental interaction between the replication
  machinery and the process being replicated -- i.e. no accidental
  sharing of names used by the process to get its work done and the
  name(s) used by the replication to effect copying. This latter
  revision of the definition of replication is crucial to obtaining
  the expected identity $!!P \sim !P$.
\end{remark}

\begin{remark}\label{rem:paradoxical_combinator}
  The reader familiar with the lambda calculus will have noticed the
  similarity between $D$ and the paradoxical combinator.

  [Ed. note: the existence of this seems to suggest we have to be more
  restrictive on the set of processes and names we admit if we are to
  support no-cloning.]
\end{remark}

\subsubsection{Bisimulation}

The computational dynamics gives rise to another kind of equivalence,
the equivalence of computational behavior. As previously mentioned
this is typically captured \emph{via} some form of bisimulation.

% The notion we use in this paper is weak barbed bisimulation
% \cite{milner91polyadicpi}.

The notion we use in this paper is derived from weak barbed
bisimulation \cite{milner91polyadicpi}. 

\begin{definition}
An \emph{observation relation}, $\downarrow_{\mathcal N}$, over a set
of names, $\mathcal N$, is the smallest relation satisfying the rules
below.

\infrule[Out-barb]{y \in {\mathcal N}, \; x \nameeq y}
		  {\outputp{x}{v} \downarrow_{\mathcal N} x}
\infrule[Par-barb]{\mbox{$P\downarrow_{\mathcal N} x$ or $Q\downarrow_{\mathcal N} x$}}
		  {\binpar{P}{Q} \downarrow_{\mathcal N} x}

We write $P \Downarrow_{\mathcal N} x$ if there is $Q$ such that 
$P \wred Q$ and $Q \downarrow_{\mathcal N} x$.
\end{definition}

\begin{definition}
%\label{def.bbisim}
An  ${\mathcal N}$-\emph{barbed bisimulation} over a set of names, ${\mathcal N}$, is a symmetric binary relation 
${\mathcal S}_{\mathcal N}$ between agents such that $P\rel{S}_{\mathcal N}Q$ implies:
\begin{enumerate}
\item If $P \red P'$ then $Q \wred Q'$ and $P'\rel{S}_{\mathcal N} Q'$.
\item If $P\downarrow_{\mathcal N} x$, then $Q\Downarrow_{\mathcal N} x$.
\end{enumerate}
$P$ is ${\mathcal N}$-barbed bisimilar to $Q$, written
$P \wbbisim_{\mathcal N} Q$, if $P \rel{S}_{\mathcal N} Q$ for some ${\mathcal N}$-barbed bisimulation ${\mathcal S}_{\mathcal N}$.
\end{definition}

$\mathcal{R} \subseteq \pi \times \pi$

$P \mathcal{R} Q => \forall P'. P \red P' \Rightarrow \exists Q'. Q \red Q', P' \mathcal{R} Q'$

$P \vdash x \Rightarrow Q \vdash x$

\begin{mathpar}
  \inferrule*[lab=Out-barb]{x \nameeq y}{{y}!\langle{Q}\rangle \vdash x}
  \and
  \inferrule*[lab=Par-barb]{\mbox{$P\vdash x$ or $Q\vdash x$}}{\binpar{P}{Q} \vdash x}
\end{mathpar}

\subsubsection{Contexts}

One of the principle advantages of computational calculi like the
$\pi$-calculus is a well-defined notion of context,
contextual-equivalence and a correlation between
contextual-equivalence and notions of bisimulation. The notion of
context allows the decomposition of a process into (sub-)process and
its syntactic environment, its context. Thus, a context may be
thought of as a process with a ``hole'' (written $\Box$) in it. The
application of a context $M$ to a process $P$, written $M[P]$, is
tantamount to filling the hole in $M$ with $P$. In this paper we do
not need the full weight of this theory, but do make use of the notion
of context in the proof the main theorem. 

\begin{mathpar}
  \inferrule* [lab=summation] {} {{M_{M},M_{N}} \bc \Box \;|\; x.M_{A} \;|\; M_{M}+M_{N}}
  \and
  \inferrule* [lab=agent] {} {{M_{A}} \bc (\vec{x})M_{P} \;| \; \clift{P_0,\ldots,M_{P},\ldots,P_N}}
  \and \\
  \inferrule* [lab=process] {} {{M_{P}} \bc M_{N} \;| \;P|M_{P} }
\end{mathpar} 

\begin{mathpar}
  \inferrule* [lab=sychronization] {} {M_{N} \bc \Box \;|\; x?M_{F} \;|\; x!M_{C}}
  \and
  \inferrule* [lab=abstraction] {} {{M_{F}} \bc (x)M_{P} }
  \and
  \inferrule* [lab=concretion] {} {{M_{C}} \bc \langle M_{P} \rangle }
  \and \\
  \inferrule* [lab=process] {} {{M_{P}} \bc M_{N} \;| \;P|M_{P} }
\end{mathpar}

\begin{definition}[contextual application] Given a context $M$, and
  process $P$, we define the \emph{contextual application}, $M[P] :=
  M\{P/\Box\}$. That is, the contextual application of M to P is the
  substitution of $P$ for $\Box$ in $M$.
\end{definition}

$\meaningof{-} : L \to \mathcal{P}(\pi)$

\begin{mathpar}
  \inferrule* [lab=collection] {} {\meaningof{true} = \pi, \and \meaningof{~E} = \pi \setminus \meaningof{E}, \and \meaningof{E_{1} \& E_{2}} = \meaningof{E_{1}} \cap \meaningof{E_{2}}}
\end{mathpar}

\begin{mathpar}
  \inferrule* [lab=structure] {} {\meaningof{0} = \{ P \in \pi | P \equiv 0 \}, \and \\ \meaningof{E_1 | E_2} = \{ P \in \pi | P \equiv P_{1} | P_{2}, P_{1} \in \meaningof{E_{1}}, P_{2} \in \meaningof{E_2}\} }
\end{mathpar}

\begin{mathpar}
 \inferrule* [lab=behavior] {} {\meaningof{\langle a?b \rangle E} = \{ P \in \pi | P \equiv Q | u?(y)P', \\ \and \\\\ \and \\ \;\;\; u \in \meaningof{a}, \forall z.P'\{z/y\} \in \meaningof{E\{z/b\}}\}, \and \\ \meaningof{a!E} = \{ P \in \pi | P \equiv Q | x!\langle P' \rangle, x \in \meaningof{a} P' \in \meaningof{E}\} }
\end{mathpar}

\begin{mathpar}
 \inferrule* [lab=nominal] {} {\meaningof{\quotep{E}} = \{ \quotep{P} \in \quotep{\pi} | P \in \meaningof{E} \}, \and \meaningof{\quotep{P}} = \{ \quotep{Q} \in \quotep{\pi} | P \equiv Q \} \and \\ \meaningof{@\quotep{E}} = \{ P \in \pi | P \equiv @x, x \in \meaningof{E} \}}
\end{mathpar}

\begin{eqnarray*}
  \\
  \meaningof{-} : TS \to ST
\end{eqnarray*}

\begin{eqnarray*}
  \\
  L : TS \to ST
\end{eqnarray*}

\begin{eqnarray*}
  \\
  P \models E \iff P \in \meaningof{E}
\end{eqnarray*}

\begin{eqnarray*}
  P \approx_{L} Q \iff \forall E \in L. P \models E \iff Q \models E
\end{eqnarray*}

\begin{eqnarray*}
  P \approx_{K} Q
\end{eqnarray*}

\begin{eqnarray*}
  P \approx Q
\end{eqnarray*}

$\approx_{K} = \approx = \approx_{L}$

\subsubsection{Contextual duality}

Note that contexts extend the quotation operation to a family of
operations from processes to names. Given a context, $M$, we can
define a \emph{nominal context}, $\quotep{M}$ by $\quotep{M}[P] :=
\quotep{M[P]}$. To foreshadow what is to come we observe that these
operations enjoy a duality with processes very much like the duality
between vectors and maps from vectors to scalars.

Further, because the calculus is essentially higher-order, we have a
correspondence between contexts and processes. More specifically,
given a name $x$ and a context $M$ we can construct $M^{*}_{x}$ such
that 

\begin{mathpar}
  M^{*}_{x} | \lift{x}{P} \red M[P]
\end{mathpar}

namely,

\begin{mathpar}
  M^{*}_{x} := x?(u).M[\dropn{u}]
\end{mathpar}

The dependence of $M^{*}_{x}$ on a name makes it an abstraction, 

\begin{mathpar}
  M^{*} := (x)x?(u).M[\dropn{u}]
\end{mathpar}

\subsection{Additional notation}

It will sometimes be convenient to denote the process a name
quotes. We already have the notation $x = \quotep{P}$, but it will be
convenient to introduce an alternate notation, $\procn{x}$, when we
want to emphasize the connection to the use of the name. Note that, by
virtue of name equivalence, $\quotep{\procn{x}} \nameeq x$; so, the
notation is consistent with previous definitions.

Further, because names have structure it is possible to effect
substitutions on the basis of that structure. This means we need to
upgrade our notation for substitutions, which we accomplish by
adapting comprehension notation. Thus,

\begin{mathpar}
  P\{ y / x : x \in S \}
\end{mathpar}

is interpreted to mean the process derived from P by replacing (in a
capture-avoiding manner) each occurrence of $x$ in $S$ by $y$. For example,

\begin{mathpar}
  P\{ \quotep{\procn{x}|\procn{x}} / x : x \in \freenames{P} \}
\end{mathpar}

will replace each (occurrence) of a free name $x$ in $P$ by
$\quotep{\procn{x}|\procn{x}}$.

Also, we will avail ourselves of the notation $x^{L}$ and $x^{R}$ to
denote injections of a name into disjoint copies of the name
space. There are numerous ways to accomplish this. One example can be
found in \cite{MeredithR05}. This notation overloads to vectors of
names: $\vec{x}^{\pi} := (x_{i}^{\pi} \; : \; 0 \leq i < |\vec{x}| )$ where $\pi \in \{L,R\}$.

We also use $P^{\Box} := P|\Box$.

In \cite{MeredithR05} an interpretation of the new operator is
given. It turns out that there are several possible interpretations
all enjoying the requisite algebraic properties of the operator (see
\cite{milner91polyadicpi}). We will therefore make liberal use of
$(\nu\; \vec{x})P$.

% subsection the_syntax_and_semantics_of_the_notation_system (end)   

\input{qm2pi.qmops} 

\input{qm2pi.sterngerlach} 

\input{qm2pi.metric} 

% section concurrent_process_calculi (end)

%\input{qm2pi.proofsketch}

% section proof sketch (end)

%\input{qm2pi.slviaknots} 

% section spatial logic via knots (end)

\input{qm2pi.conclusion}

% section conclusion (end)

%\input{qm2pi.dtcodes} 

% section wiring algorithm (end)

\input{qm2pi.ack} 

% section acknowledgments (end)

\newpage


\bibliographystyle{plain}   
\bibliography{../../biblios/main.bib}

\input{qm2pi.rhodetails}

\end{document}



% section proof sketch (end)

%\section{Unlikely characters: spatial logic for
  knots}\label{sub:characteristic_formulae} % (fold)

Associated to the mobile process calculi are a family of logics known
as the Hennessy-Milner logics. These logics typically enjoy a
semantics interpreting formulae as sets of processes that when
factored through the encoding outlined above allows an identification
of classes of knots with logical formulae. In the context of this
encoding the sub-family known as the spatial logics \cite{CairesC03}
\cite{CairesC04} \cite{Caires04} are of particular interest providing
several important features for expressing and reasoning about
properties (i.e. classes) of knots. We hint here at how this may be done.

%\begin{description}
%\item [structural connectives] 
\subsubsection{Structural connectives} The spatial logics enjoy
structural connectives corresponding, at the logical level, to the
parallel composition ($P | Q$) and new name ($(\nu \; x)P$)
connectives for processes. As illustrated in the examples below, these
connectives are extremely expressive given the shape of our encoding.
%\item [decideable satisfaction]

\subsubsection{Decideable satisfaction}
In \cite{Caires04} the satisfaction relation is shown to be decideable
for a rich class of processes. It further turns out that the image of
the our encoding is a proper subset of that class. This result
provides the basis for an algorithm by which to search for knots
enjoying a given property.
%\item [characteristic formulae]

\subsubsection{Characteristic formulae}
In the same paper \cite{Caires04} , Caires presents a means of calculating
characteristic formulae, selecting equivalence classes of processes
up to a pre--specified depth limit on the support set of names. Composed with our
encoding, this characteristic formula can be used to select
characteristic formulae for knots.
%\end{description}

\subsubsection{Spatial logic formulae}

The grammar below (segmented for comprehension) summarizes the syntax
of spatial logic formulae. We employ illustrative examples in the
sequel to provide an intuitive understanding of their meaning
referring the reader to \cite{Caires04} for a more detailed explication
of the semantics.

\begin{mathpar}
  \inferrule* [lab=boolean] {} {{A,B} \bc T \;|\; \neg A \;|\; A \wedge B \;|\; \eta = \eta'}
  \and
  \inferrule* [lab=spatial] {} {|\; \pzero \;|\; A | B \;|\; x \text{\textregistered} A \;|\; \forall x . A \;|\;  H x . A}
  \and
  \inferrule* [lab=behavioral] {} {|\; \alpha . A}
  \and 
  \inferrule* [lab=recursion] {} {|\; X(\vec{u}) \;|\; \mu X(\vec{u}) . A}
  \and
  \inferrule* [lab=action] {} {\alpha \bc \langle x?(\vec{y}) \rangle \;|\; \langle x!(\vec{y}) \rangle \;|\; \langle \tau \rangle}
  \and 
  \inferrule* [lab=name] {} {\eta \bc x \;|\; \tau}
\end{mathpar} 

% subsection characteristic_formulae (end)   	 

\subsection{Example formulae}\label{sub:example_formulae_} % (fold)

\subsubsection{Crossing as formula.}
% 
% \begin{align*}
%   \frac{d}{dx} \sin x &= \cos x 
%   & \frac{d}{dx} e^x &= e^x \\
%   \frac{d}{dx} \cos x &= - \sin x 
%   & \frac{d}{dx} \log x &= \frac{1}{x} \\
% \end{align*} 

\begin{align*}
 \mu C(x_{0},x_{1},y_{0},y_{1},u).&(\langle x_{0}?(z) \rangle(\langle u! \rangle\langle y_{1}!z \rangle C(x_{0},x_{1},y_{0},y_{1},u)) & \\
  & \wedge \langle y_{1}?(z) \rangle (\langle u! \rangle \langle x_{0}!z \rangle C(x_{0},x_{1},y_{0},y_{1},u)) & \\
  & \wedge \langle x_{1}?(z) \rangle (\langle u? \rangle \langle y_{0}!z \rangle C(x_{0},x_{1},y_{0},y_{1},u)) & \\
  & \wedge \langle y_{0}?(z) \rangle (\langle u? \rangle \langle x_{1}!z \rangle C(x_{0},x_{1},y_{0},y_{1},u))) &
\end{align*}

The lexicographical similarity between the shape of this formulae and
the shape of definition of the process representing a crossing reveals
the intuitive meaning of this formulae. It describes the capabilities
of a process that has the right to represent a crossing. For example
it picks out processes that may perform an input on the port $x_0$ in
its initial menu of capabilities. What differentiates the formula
from the process, however, is that the crossing process is the
smallest candidate to satisfy the formula. Infinitely many other
processes -- with internal behavior hidden behind this interface, so
to speak -- also satisfy this formula. Even this simple formula,
then, can be seen to open a new view onto knots, providing a
computational interpretation of \emph{virtual} knots.

Note that this formula is derived by hand. A similar formula can be
derived by employing Caires' calculation of characteristic formula
\cite{Caires04} to the process representing a crossing. In light of
this discussion, we let
$\meaningof{C}_{\phi}(x0,x1,y0,y1,u)$ denote a formula specifying the
dynamics we wish to capture of a crossing. To guarantee we preserve
the shape of the interface and minimal semantics we demand that
$\meaningof{C}_{\phi}(x0,x1,y0,y1,u) \Rightarrow
\textbf{C}(x0,x1,y0,y1,u)$ where $\textbf{C}(x0,x1,y0,y1,u)$ denotes
the formula above.
                            
\subsubsection{Crossing number constraints.}
The moral content of the context lemma (Lemma \ref{context}) is that the notion of
``locality'' in the Reidemeister moves is effectively captured by the
parallel composition operator of the process calculus. This intuition
extends through the logic. Given a formula,
$\meaningof{C}_{\phi}(x0,x1,y0,y1,u)$, we can use the structural
connectives to specify constraints on crossing numbers, such as at
least $n$ crossings, or exactly $n$ crossings.
\begin{mathpar}
  \inferrule* [lab=at-least-n] {} { K^{\geq n}_{\phi}(\vec{xs},\vec{ys}) := \Pi_{i=0}^{n-1} Hu . \meaningof{C}_{\phi}(xs_i,ys_i,u) | T }
  \and 
  \inferrule* [lab=exactly-n] {} { K^{= n}_{\phi}(\vec{xs},\vec{ys}) := \Pi_{i=0}^{n-1} Hu . \meaningof{C}_{\phi}(xs_i,ys_i,u) | \neg (\forall x_0,y_0,x_1,y_1,u . \meaningof{C}_{\phi}(x_0,y_0,x_1,y_1,u) | T) }
\end{mathpar}

To round out this section, recall that the encoding of an $n$-crossing
knot decomposes into a parallel composition of $n$ \emph{copies} of a
crossing process together with a wiring harness. To specify different
knot classes with the same crossing number amounts to specifying
logical constraints on the wiring harness. In the interest of space,
we defer examples to a forthcoming paper. Suffice it to say that both
the conditions ``alternating knot'' and ``contains the tangle
corresponding to 5/3'' are expressible. For example, it is possible to
calculate the characteristic formula of a process corresponding to the
tangle 5/3 and conjoin it into the classifying formula via the
composition connective of the logic.

Finally, we wish to observe that it is entirely within reason to
contemplate a more domain-specific version of spatial logic tailored
to the shape of processes in the image of the encoding. Such a
domain-specific logic would have a better claim to the title formal
language of knot properties.

% subsection example_formulae_ (end)

% section knots_as_processes (end) 

% section spatial logic via knots (end)

\section{Conclusions and future work}

\paragraph{Testing physical space}
You, gentle reader, may wonder why of all the theorems to be proved
given this set up we pick the one above. In some sense it's hardly
central to quantum mechanics. We see it as central in the sense that
it firmly establishes a notion of physical space arising from a notion
of the equivalence of behavior. Relating bisimulation to a metric is a
big step forward, but one is faced with interpreting the relationship
of that metric space to something more physical. Quantum mechanical
notions of ``physical'' space are still far from intuitive, but by
relating this idea of distance as testing to calculations that predict
physical circumstances we are making a not insignificant step forward
toward an understanding of the physical space we inhabit as
essentially dynamic.

\paragraph{Effectivity and simulation}
One of the observations we have yet to make is that the entire program
spelled out here is effective. We have built various interpreters for
the reflective calculus at work in this interpretation. In principle,
then, we can simulate quantum mechanics on a computer. The place where
the simulation may lose fidelity is the infinitely branching summation
for the annihilator.

In this connection i also want to point out that the evaluation style
calculation of the inner product puts the non-determinism of the
summation right at the heart of measurement. This suggests that
Milner's original reduction-based formulation of the dynamics of his
calculi in terms of sums was not just notationally suggestive of a
notion of measure-and-continue but captured some significant part of
the physics.

\paragraph{Quantum continuations}
In light of this last observation i want to point out that the
predominant account of quantum mechanics is missing a key aspect of a
truly compositional story of the physical situation. In a real lab,
when a measurement is made the observation can be made to feed into
another device that then makes another measurement conditioned on the
results of the first. This means that after the superposition was
collapsed the entire experimental set up remained in
superposition. While QM offers a means of writing this down it doesn't
quite line up well with the well-trodden formulation of computation
and continuation that we see so succinctly expressed in Milner's
calculi. This suggests that there might be advantages to this account
of dynamics waiting to be explored.

\paragraph{Quantum logic}
In this connection, we also note that by virtue of having the
Hennessy-Milner construction, we can pull the construction through the
interpretation of QM. This gives us a natural candidate for a quantum
logic that enjoys an extremely tight connection with it's domain of
interpretation, making the construction much less ad hoc (rather it is
the image of functor!).

\paragraph{Quantum probabiity}
i have questions about the basis of the interpretation of inner
product as probability amplitude. In particular, using which
axiomatization of probability theory does the notion of probability
amplitude earn the right to be so dubbed? In other words, where is the
proof that the operation for calculating a probability amplitude (and
then squaring) satisfies the axioms of what it means to calculate a
probability? Even if such a proof exists (i have yet to find it in the
literature), i wonder if it might not be possible to turn things on
their heads. Can we view the calculation of the probability amplitude
as an axiomatization of probability? If so, then the definition we
give for calculating probability amplitude may provide the basis for
an \emph{effective} theory of probability.

\paragraph{Quantum vs ``biological'' information}
Finally, i want to conclude with a more philosophical observation. At
a recent workshop in which QM was a predominant topic i noticed
something about quantum information. The speaker was giving a riveting
discussion of axiomatic QM and showing how properties of ``no
cloning'' and ``no deleting'' emerged as consequences of the
axiomatization. Theorems of this form are necessary to give us a sense
of confidence that our axioms characterize the physical theory. What
struck me, though, was that if quantum information is neither erasable
nor replicable it is markedly different from \emph{life}. Two of the
things we know about life is that

\begin{itemize}
  \item it ends;
  \item to gain some measure of persistence, to transcend it's
    finitude it is imminently copyable.
\end{itemize}

Both of these qualities are summarized succinctly in the aphorism: all
flesh is grass. For me these two kinds of ``information'' -- call them
quantum and biological -- are end points on a spectrum of strategies
for persistence. At one end, we have those curious entities that enjoy
uniqueness and permanence; at the other, we have those who in the face
of a certain end and an uncertain present make a go of passing
something on. To me one of the more remarkable aspects of the latter
strategy is that in the presence of noise (and certain features of
copying) we get a kind of dynamism, a chance for improvement against a
given persistent condition.

% subsection other_calculi_other_bisimulations_and_geometry_as_behavior (end)




% section conclusion (end)

%\documentclass[12pt]{llncs}
%\documentclass{jktr}

\usepackage[pdftex]{hyperref}                   
\usepackage {listings}
\usepackage {mathpartir}
\usepackage{bcprules}
%\usepackage{listings}
                       
\usepackage{graphicx} 
%\usepackage[margins=2.5cm,nohead,nofoot]{geometry}
%\usepackage{geometry}
\usepackage{amsfonts}
\usepackage{amstext}
\usepackage{latexsym}
\usepackage{amssymb}
\usepackage{color}


%\include{myPreamble}
\include{qm2pi.local} 

%\ifpdf
%\usepackage[pdftex]{graphicx}
%\else
%\usepackage{graphicx}
%\fi

 % \ifpdf
%  \usepackage{pdfsync}
%  \if


%\title{Brief Article}
%\author{David F. Snyder}
%\author{L.G. Meredith}

%\address{Dept. of Math., Texas State University--San Marcos, San Marcos, TX 78666}
       
\pagestyle{empty}


\begin{document}

\lstset{language=[Objective]Caml,frame=shadowbox}

\input{qm2pi.front}

% section front matter (end)

\input{qm2pi.intro} 
 
% section introduction (end)

% \input{qm2pi.knotations} 

% section notation (end)

\input{qm2pi.process.calculi} 

% section concurrent_process_calculi_and_spatial_logics_ (end)
    
%\input{qm2pi.knots2pi} 

%\input{qm2pi.trefoil} 

%\input{qm2pi.mainthm} 

% subsection basic_interpretation (end)

%\input{qm2pi.rho.presentation} 
\subsection{The syntax and semantics of the notation system}\label{sub:the_syntax_and_semantics_of_the_notation_system} % (fold)

We now summarize a technical presentation of the calculus that
embodies our theory of dynamics. The typical presentation of such a
calculus follows the style of giving generators and relations on
them. The grammar, below, describing term constructors, freely
generates the set of processes, $\Proc$. This set is then quotiented
by a relation known as structural congruence and it is over this set
that the notion of dynamics is expressed. This presentation is
essentially that of \cite{MeredithR05} with the addition of
polyadicity and summation. For readability we have relegated some of
the technical subtleties to an appendix.

\subsubsection{Process grammar}\label{subsub:process_grammar}

\begin{mathpar}
  \inferrule* [lab=synchronization] {} {{M} \bc \pzero \;|\; x?F \;|\; x!C }
  \and
  \inferrule* [lab=abstraction] {} {{F} \bc (x)P}
  \and
  \inferrule* [lab=concretion] {} {{C} \bc \langle Q \rangle}
  \and
  \inferrule* [lab=process] {} {{P,Q} \bc M \;| \;P|Q \;|\; @{x}}
  \and
  \inferrule* [lab=name] {} {{x} \bc \quotep{P}}
\end{mathpar} 

Note that $\vec{x}$ (resp. $\vec{P}$) denotes a vector of names
(resp. processes) of length $|\vec{x}|$ (resp. $|\vec{P}|$). We adopt
the following useful abbreviations.

\begin{mathpar}
   x?(\vec{y}).P := x.(\vec{y})P \and  x\clift{\vec{P}} := x.\clift{\vec{P}}
   \and x!(y) := \lift{x}{\dropn{y}}
   \and \Pi_{i=0}^{n-1}P_i := P_0 | \ldots | P_{n-1}
\end{mathpar}

\subsubsection{Structural congruence}

\paragraph{Free and bound names and alpha-equivalence.} At the
core of structural equivalence is alpha-equivalence which identifies
process that are the same up to a change of variable. Formally, we
recognize the distinction between free and bound names. The free names
of a process, $\freenames{P}$, may be calculated recursively as
follows:

\begin{mathpar}
\freenames{\pzero} := \emptyset
  \and \\
  \freenames{x?(y).P} := \{ x \} \cup (\freenames{P} \setminus \{ y \})
  \and 
  \freenames{x!\langle P \rangle} := \{ x \} \cup \{ P \} 
  \and \\
  \freenames{P|Q} := \freenames{P} \cup \freenames{Q}
  \and \\
  \freenames{@{x}} := \{ x \}
\end{mathpar}

$\pi$
$\quotep{\pi}$

$\freenames{-} : \pi \to \mathcal{P}(\quotep{\pi})$

\begin{eqnarray*}
  \freenames{\pzero} & := & \emptyset \\
  \freenames{x?(y).P} & := & \{ x \} \cup (\freenames{P} \setminus \{ y \}) \\
  \freenames{x!\langle P \rangle} & := & \{ x \} \cup \{ P \} \\
  \freenames{P|Q} & := & \freenames{P} \cup \freenames{Q} \\
  \freenames{\dropn{x}} & := & \{ x \}
\end{eqnarray*}

The bound names of a process, $\boundnames{P}$, are those names occurring in $P$
that are not free. For example, in $x?(y).0$, the name $x$ is free, while $y$ is bound.

\begin{mathpar}
  \inferrule* [lab=monoidal-laws] {} { P|Q \equiv Q|P \and P|0 \equiv P \and P|(Q|R) \equiv (P|Q)|R }
\end{mathpar}

\begin{mathpar}
  \inferrule* [lab=alpha-equivalence] {} { (x)P \equiv (y)P\{y/x\} \and y \not\in \freenames{P} }
\end{mathpar}

\begin{definition}
Then two processes, $P,Q$, are alpha-equivalent if $P = Q\{\vec{y}/\vec{x}\}$ for
some $\vec{x} \in \boundnames{Q},\vec{y} \in \boundnames{P}$, where $Q\{\vec{y}/\vec{x}\}$
denotes the capture-avoiding substitution of $\vec{y}$ for $\vec{x}$ in $Q$.
\end{definition}

\begin{definition}
  The {\em structural congruence} \cite{SangiorgiWalker} , $\equiv$,
  between processes is the least congruence containing
  alpha-equivalence, satisfying the abelian monoid laws
  (associativity, commutativity and $\pzero$ as identity) for parallel
  composition $|$ and for summation $+$.
\end{definition}

\subsection{Name equivalence}

We take name equivalence, written $\nameeq$, to be the smallest
equivalence relation generated by the following rules.

\begin{mathpar}
\inferrule*[lab=Quote-drop]
{ }
{ \quotep{@{x}} \nameeq x }

\inferrule*[lab=Struct-equiv]
{ P \scong Q }
{ \quotep{P} \nameeq \quotep{Q} }
\end{mathpar}

The astute reader will have noticed that the mutual recursion of names
and processes imposes a mutual recursion on alpha-equivalence and
structural equivalence via name-equivalence. Fortunately, all of this
works out pleasantly and we may calculate in the natural way, free of
concern. The reader interested in the details is referred to the
appendix \ref{appendix:rho_details}.

\subsection{Substitution}

We use $\Proc$ for the set of processes, $\QProc$ for the set of
names, and $\id{\{}\vec{y} / \vec{x} \id{\}}$ to denote partial maps,
$s : \QProc \rightarrow \QProc$. A map, $s$ lifts, uniquely, to a map
on process terms, $\widehat{s} : \Proc \rightarrow \Proc$ by the
following equations.

\begin{mathpar}
  (0) \psubstp{Q}{P} := 0 \\
  (R \juxtap S) \psubstp{Q}{P}
  :=    
  (R)\psubstp{Q}{P} \juxtap (S) \psubstp{Q}{P} \\
  (x?(y).R) \psubstp{Q}{P}    
  :=    
  (x)\substp{Q}{P} (z)\concat( (R \psubstn{z}{y}) \psubstp{Q}{P} ) \\
  (\lift{x}{R}) \psubstp{Q}{P}  
  :=
  \lift{(x)\substp{Q}{P}}{ R \psubstp{Q}{P} } \\
%   (\dropn{x})  \psubstp{Q}{P}       
%   := 
%   \left\{ 
%     \begin{array}{ccc} 
%       \dropn{\quotep{Q}} & & x \nameeq \quotep{P} \\
%       \dropn{x} & & otherwise \\
%     \end{array}
%   \right. 
  (\dropn{x})  \psubstp{Q}{P}       
  := 
  \left\{ 
    \begin{array}{ccc} 
      Q & & x \nameeq \quotep{P} \\
      \dropn{x} & & otherwise \\
    \end{array}
  \right.
\end{mathpar}
 

where

\begin{eqnarray}
  (x)\id{\{} \lpquote Q \rpquote / \lpquote P \rpquote \id{\}}            = 
  \left\{ 
    \begin{array}{ccc}
      \lpquote Q \rpquote & & x \nameeq \lpquote P \rpquote \\
      x & & otherwise \\
    \end{array}
  \right. \nonumber
\end{eqnarray}

and $z$ is chosen distinct from $\quotep{P}$, $\quotep{Q}$, the free
names in $Q$, and all the names in $R$. Our $\alpha$-equivalence will
be built in the standard way from this substitution.

\begin{remark}\label{rem:no_self_referential_names}
  One consequence of these definitions is that $\forall P. \quotep{P}
  \not\in \freenames{P}$.
\end{remark}

\subsection{ Dynamic quote: an example }

Anticipating something of what's to come, consider applying the
substitution, $\widehat{\id{\{}u / z \id{\}}}$, to the following pair
of processes, $\lift{w}{y!(z)}$ and $w[ \lpquote y!(z) \rpquote ]$.

\begin{eqnarray}
	\lift{w}{y!(z)}\widehat{\id{\{}u / z \id{\}}}
		& = &
		\lift{w}{y!(u)} \nonumber\\
	w[ \lpquote y!(z) \rpquote ] \widehat{ \id{\{}u / z \id{\}} }
		& = &
		w[ \lpquote y!(z) \rpquote ] \nonumber
\end{eqnarray}

Because the body of the process between quotes is impervious to
substitution, we get radically different answers. In fact, by
examining the first process in an input context,
e.g. $x?(z).\lift{w}{y!(z)}$, we see that the process under the lift
operator may be shaped by prefixed inputs binding a name inside it. In
this sense, the lift operator will be seen as a way to dynamically
construct processes before reifying them as names.

Finally equipped with these standard features we can present the
dynamics of the calculus.

\subsubsection{Operational semantics} 

Finally, we introduce the computational dynamics. What marks these
algebras as distinct from other more traditionally studied algebraic
structures, e.g. vector spaces or polynomial rings, is the manner in
which dynamics is captured. In traditional structures, dynamics is typically
expressed through morphisms between such structures, as in linear maps
between vector spaces or morphisms between rings. In algebras
associated with the semantics of computation, the dynamics is
expressed as part of the algebraic structure itself, through a
reduction reduction relation typically denoted by $\red$. Below, we
give a recursive presentation of this relation for the calculus used
in the encoding.

$\red \subseteq \pi \times \pi$
$\red : \pi \to \mathcal{P}(\pi)$

\begin{mathpar}
  \inferrule* [lab=Comm] { \textsf{match}( x_{src}, x_{trgt} ) } { x_{trgt}?(y)P \; | \; x_{src}!\langle {Q} \rangle \red P\{\quotep{Q}/y}\} }
  \and \\
  \inferrule* [lab=Par] {{P} \red {P}'} {{{P} | {Q}} \red {{P}' | {Q}}}
  \and
  \inferrule* [lab=Equiv]{{{P} \scong {P}'} \andalso {{P}' \red {Q}'} \andalso {{Q}' \scong {Q}}}{{P} \red {Q}}
\end{mathpar}

\begin{eqnarray*}
  match_{\equiv} (\quotep{P},\quotep{Q}) & := & P \equiv Q \\
  match_{\dagger}(\quotep{P},\quotep{Q}) & := & \forall R. P|Q \red^{*} R => R \red^{*} 0 \\
  match_{K}(\quotep{P},\quotep{Q}) & := & K \mbox{ for some context } K
\end{eqnarray*}

$u?(x)P | u!\langle Q \rangle \red P\{\quotep{Q}/x\}$

%We write $\wred$ for $\red^*$, and $P\red$ if $\exists Q $ such that $ P \red Q$.
We write $P\red$ if $\exists Q $ such that $ P \red Q$ and $P\not\red$, otherwise.

\section{Replication}

As mentioned before, it is known that replication (and hence
recursion) can be implemented in a higher-order process algebra
\cite{SangiorgiWalker}. As our first example of calculation with the
machinery thus far presented we give the construction explicitly in
the {\rhoc}.

\begin{eqnarray}
	D_{x} & := & \prefix{x}{y}{(\binpar{\outputp{x}{y}}{@{y}})} \nonumber\\
	\bangp_{x}{P} & := & \binpar{{x}!\langle{\binpar{D_{x}}{P}}\rangle}{D_{x}} \nonumber
\end{eqnarray}

\begin{eqnarray}
	\bangp_{x}{P} & & \nonumber\\
	=
	& {x}!\langle{(\prefix{x}{y}{(\outputp{x}{y} | @{y})) | P}}\rangle 
	      | \prefix{x}{y}{(\outputp{x}{y} | @{y})} & \nonumber\\
	\red
	& (\outputp{x}{y} | @{y})\substn{\quotep{(\prefix{x}{y}{(@{y} | \outputp{x}{y})) | P}}}{y} & \nonumber\\
	=
	& \outputp{x}{\quotep{(\prefix{x}{y}{(\outputp{x}{y} | @{y})) | P}}}
	  | {(\prefix{x}{y}{(\outputp{x}{y} | @{y})) | P}} & \nonumber\\
	\red
	& \ldots & \nonumber\\
	\red^*
	& P | P | \ldots & \nonumber
\end{eqnarray}

Of course, this encoding, as an implementation, runs away, unfolding
$\bangp{P}$ eagerly. A lazier and more implementable replication
operator, restricted to input-guarded processes, may be obtained as follows.

\begin{eqnarray}
\bangp{\prefix{u}{v}{P}} 
	:= 
	\binpar{\lift{x}{\prefix{u}{v}{(\binpar{D(x)}{P})}}}{D(x)} \nonumber
\end{eqnarray}

\begin{remark}
  Note that the lazier definition still does not deal with summation
  or mixed summation (i.e. sums over input and output). The reader is
  invited to construct definitions of replication that deal with these
  features. 

  Further, the definitions are parameterized in a name, $x$. Can you,
  gentle reader, make a definition that eliminates this parameter and
  guarantees no accidental interaction between the replication
  machinery and the process being replicated -- i.e. no accidental
  sharing of names used by the process to get its work done and the
  name(s) used by the replication to effect copying. This latter
  revision of the definition of replication is crucial to obtaining
  the expected identity $!!P \sim !P$.
\end{remark}

\begin{remark}\label{rem:paradoxical_combinator}
  The reader familiar with the lambda calculus will have noticed the
  similarity between $D$ and the paradoxical combinator.

  [Ed. note: the existence of this seems to suggest we have to be more
  restrictive on the set of processes and names we admit if we are to
  support no-cloning.]
\end{remark}

\subsubsection{Bisimulation}

The computational dynamics gives rise to another kind of equivalence,
the equivalence of computational behavior. As previously mentioned
this is typically captured \emph{via} some form of bisimulation.

% The notion we use in this paper is weak barbed bisimulation
% \cite{milner91polyadicpi}.

The notion we use in this paper is derived from weak barbed
bisimulation \cite{milner91polyadicpi}. 

\begin{definition}
An \emph{observation relation}, $\downarrow_{\mathcal N}$, over a set
of names, $\mathcal N$, is the smallest relation satisfying the rules
below.

\infrule[Out-barb]{y \in {\mathcal N}, \; x \nameeq y}
		  {\outputp{x}{v} \downarrow_{\mathcal N} x}
\infrule[Par-barb]{\mbox{$P\downarrow_{\mathcal N} x$ or $Q\downarrow_{\mathcal N} x$}}
		  {\binpar{P}{Q} \downarrow_{\mathcal N} x}

We write $P \Downarrow_{\mathcal N} x$ if there is $Q$ such that 
$P \wred Q$ and $Q \downarrow_{\mathcal N} x$.
\end{definition}

\begin{definition}
%\label{def.bbisim}
An  ${\mathcal N}$-\emph{barbed bisimulation} over a set of names, ${\mathcal N}$, is a symmetric binary relation 
${\mathcal S}_{\mathcal N}$ between agents such that $P\rel{S}_{\mathcal N}Q$ implies:
\begin{enumerate}
\item If $P \red P'$ then $Q \wred Q'$ and $P'\rel{S}_{\mathcal N} Q'$.
\item If $P\downarrow_{\mathcal N} x$, then $Q\Downarrow_{\mathcal N} x$.
\end{enumerate}
$P$ is ${\mathcal N}$-barbed bisimilar to $Q$, written
$P \wbbisim_{\mathcal N} Q$, if $P \rel{S}_{\mathcal N} Q$ for some ${\mathcal N}$-barbed bisimulation ${\mathcal S}_{\mathcal N}$.
\end{definition}

$\mathcal{R} \subseteq \pi \times \pi$

$P \mathcal{R} Q => \forall P'. P \red P' \Rightarrow \exists Q'. Q \red Q', P' \mathcal{R} Q'$

$P \vdash x \Rightarrow Q \vdash x$

\begin{mathpar}
  \inferrule*[lab=Out-barb]{x \nameeq y}{{y}!\langle{Q}\rangle \vdash x}
  \and
  \inferrule*[lab=Par-barb]{\mbox{$P\vdash x$ or $Q\vdash x$}}{\binpar{P}{Q} \vdash x}
\end{mathpar}

\subsubsection{Contexts}

One of the principle advantages of computational calculi like the
$\pi$-calculus is a well-defined notion of context,
contextual-equivalence and a correlation between
contextual-equivalence and notions of bisimulation. The notion of
context allows the decomposition of a process into (sub-)process and
its syntactic environment, its context. Thus, a context may be
thought of as a process with a ``hole'' (written $\Box$) in it. The
application of a context $M$ to a process $P$, written $M[P]$, is
tantamount to filling the hole in $M$ with $P$. In this paper we do
not need the full weight of this theory, but do make use of the notion
of context in the proof the main theorem. 

\begin{mathpar}
  \inferrule* [lab=summation] {} {{M_{M},M_{N}} \bc \Box \;|\; x.M_{A} \;|\; M_{M}+M_{N}}
  \and
  \inferrule* [lab=agent] {} {{M_{A}} \bc (\vec{x})M_{P} \;| \; \clift{P_0,\ldots,M_{P},\ldots,P_N}}
  \and \\
  \inferrule* [lab=process] {} {{M_{P}} \bc M_{N} \;| \;P|M_{P} }
\end{mathpar} 

\begin{mathpar}
  \inferrule* [lab=sychronization] {} {M_{N} \bc \Box \;|\; x?M_{F} \;|\; x!M_{C}}
  \and
  \inferrule* [lab=abstraction] {} {{M_{F}} \bc (x)M_{P} }
  \and
  \inferrule* [lab=concretion] {} {{M_{C}} \bc \langle M_{P} \rangle }
  \and \\
  \inferrule* [lab=process] {} {{M_{P}} \bc M_{N} \;| \;P|M_{P} }
\end{mathpar}

\begin{definition}[contextual application] Given a context $M$, and
  process $P$, we define the \emph{contextual application}, $M[P] :=
  M\{P/\Box\}$. That is, the contextual application of M to P is the
  substitution of $P$ for $\Box$ in $M$.
\end{definition}

$\meaningof{-} : L \to \mathcal{P}(\pi)$

\begin{mathpar}
  \inferrule* [lab=collection] {} {\meaningof{true} = \pi, \and \meaningof{~E} = \pi \setminus \meaningof{E}, \and \meaningof{E_{1} \& E_{2}} = \meaningof{E_{1}} \cap \meaningof{E_{2}}}
\end{mathpar}

\begin{mathpar}
  \inferrule* [lab=structure] {} {\meaningof{0} = \{ P \in \pi | P \equiv 0 \}, \and \\ \meaningof{E_1 | E_2} = \{ P \in \pi | P \equiv P_{1} | P_{2}, P_{1} \in \meaningof{E_{1}}, P_{2} \in \meaningof{E_2}\} }
\end{mathpar}

\begin{mathpar}
 \inferrule* [lab=behavior] {} {\meaningof{\langle a?b \rangle E} = \{ P \in \pi | P \equiv Q | u?(y)P', \\ \and \\\\ \and \\ \;\;\; u \in \meaningof{a}, \forall z.P'\{z/y\} \in \meaningof{E\{z/b\}}\}, \and \\ \meaningof{a!E} = \{ P \in \pi | P \equiv Q | x!\langle P' \rangle, x \in \meaningof{a} P' \in \meaningof{E}\} }
\end{mathpar}

\begin{mathpar}
 \inferrule* [lab=nominal] {} {\meaningof{\quotep{E}} = \{ \quotep{P} \in \quotep{\pi} | P \in \meaningof{E} \}, \and \meaningof{\quotep{P}} = \{ \quotep{Q} \in \quotep{\pi} | P \equiv Q \} \and \\ \meaningof{@\quotep{E}} = \{ P \in \pi | P \equiv @x, x \in \meaningof{E} \}}
\end{mathpar}

\begin{eqnarray*}
  \\
  \meaningof{-} : TS \to ST
\end{eqnarray*}

\begin{eqnarray*}
  \\
  L : TS \to ST
\end{eqnarray*}

\begin{eqnarray*}
  \\
  P \models E \iff P \in \meaningof{E}
\end{eqnarray*}

\begin{eqnarray*}
  P \approx_{L} Q \iff \forall E \in L. P \models E \iff Q \models E
\end{eqnarray*}

\begin{eqnarray*}
  P \approx_{K} Q
\end{eqnarray*}

\begin{eqnarray*}
  P \approx Q
\end{eqnarray*}

$\approx_{K} = \approx = \approx_{L}$

\subsubsection{Contextual duality}

Note that contexts extend the quotation operation to a family of
operations from processes to names. Given a context, $M$, we can
define a \emph{nominal context}, $\quotep{M}$ by $\quotep{M}[P] :=
\quotep{M[P]}$. To foreshadow what is to come we observe that these
operations enjoy a duality with processes very much like the duality
between vectors and maps from vectors to scalars.

Further, because the calculus is essentially higher-order, we have a
correspondence between contexts and processes. More specifically,
given a name $x$ and a context $M$ we can construct $M^{*}_{x}$ such
that 

\begin{mathpar}
  M^{*}_{x} | \lift{x}{P} \red M[P]
\end{mathpar}

namely,

\begin{mathpar}
  M^{*}_{x} := x?(u).M[\dropn{u}]
\end{mathpar}

The dependence of $M^{*}_{x}$ on a name makes it an abstraction, 

\begin{mathpar}
  M^{*} := (x)x?(u).M[\dropn{u}]
\end{mathpar}

\subsection{Additional notation}

It will sometimes be convenient to denote the process a name
quotes. We already have the notation $x = \quotep{P}$, but it will be
convenient to introduce an alternate notation, $\procn{x}$, when we
want to emphasize the connection to the use of the name. Note that, by
virtue of name equivalence, $\quotep{\procn{x}} \nameeq x$; so, the
notation is consistent with previous definitions.

Further, because names have structure it is possible to effect
substitutions on the basis of that structure. This means we need to
upgrade our notation for substitutions, which we accomplish by
adapting comprehension notation. Thus,

\begin{mathpar}
  P\{ y / x : x \in S \}
\end{mathpar}

is interpreted to mean the process derived from P by replacing (in a
capture-avoiding manner) each occurrence of $x$ in $S$ by $y$. For example,

\begin{mathpar}
  P\{ \quotep{\procn{x}|\procn{x}} / x : x \in \freenames{P} \}
\end{mathpar}

will replace each (occurrence) of a free name $x$ in $P$ by
$\quotep{\procn{x}|\procn{x}}$.

Also, we will avail ourselves of the notation $x^{L}$ and $x^{R}$ to
denote injections of a name into disjoint copies of the name
space. There are numerous ways to accomplish this. One example can be
found in \cite{MeredithR05}. This notation overloads to vectors of
names: $\vec{x}^{\pi} := (x_{i}^{\pi} \; : \; 0 \leq i < |\vec{x}| )$ where $\pi \in \{L,R\}$.

We also use $P^{\Box} := P|\Box$.

In \cite{MeredithR05} an interpretation of the new operator is
given. It turns out that there are several possible interpretations
all enjoying the requisite algebraic properties of the operator (see
\cite{milner91polyadicpi}). We will therefore make liberal use of
$(\nu\; \vec{x})P$.

% subsection the_syntax_and_semantics_of_the_notation_system (end)   

\input{qm2pi.qmops} 

\input{qm2pi.sterngerlach} 

\input{qm2pi.metric} 

% section concurrent_process_calculi (end)

%\input{qm2pi.proofsketch}

% section proof sketch (end)

%\input{qm2pi.slviaknots} 

% section spatial logic via knots (end)

\input{qm2pi.conclusion}

% section conclusion (end)

%\input{qm2pi.dtcodes} 

% section wiring algorithm (end)

\input{qm2pi.ack} 

% section acknowledgments (end)

\newpage


\bibliographystyle{plain}   
\bibliography{../../biblios/main.bib}

\input{qm2pi.rhodetails}

\end{document}

 

% section wiring algorithm (end)

\documentclass[12pt]{llncs}
%\documentclass{jktr}

\usepackage[pdftex]{hyperref}                   
\usepackage {listings}
\usepackage {mathpartir}
\usepackage{bcprules}
%\usepackage{listings}
                       
\usepackage{graphicx} 
%\usepackage[margins=2.5cm,nohead,nofoot]{geometry}
%\usepackage{geometry}
\usepackage{amsfonts}
\usepackage{amstext}
\usepackage{latexsym}
\usepackage{amssymb}
\usepackage{color}


%\include{myPreamble}
\include{qm2pi.local} 

%\ifpdf
%\usepackage[pdftex]{graphicx}
%\else
%\usepackage{graphicx}
%\fi

 % \ifpdf
%  \usepackage{pdfsync}
%  \if


%\title{Brief Article}
%\author{David F. Snyder}
%\author{L.G. Meredith}

%\address{Dept. of Math., Texas State University--San Marcos, San Marcos, TX 78666}
       
\pagestyle{empty}


\begin{document}

\lstset{language=[Objective]Caml,frame=shadowbox}

\input{qm2pi.front}

% section front matter (end)

\input{qm2pi.intro} 
 
% section introduction (end)

% \input{qm2pi.knotations} 

% section notation (end)

\input{qm2pi.process.calculi} 

% section concurrent_process_calculi_and_spatial_logics_ (end)
    
%\input{qm2pi.knots2pi} 

%\input{qm2pi.trefoil} 

%\input{qm2pi.mainthm} 

% subsection basic_interpretation (end)

%\input{qm2pi.rho.presentation} 
\subsection{The syntax and semantics of the notation system}\label{sub:the_syntax_and_semantics_of_the_notation_system} % (fold)

We now summarize a technical presentation of the calculus that
embodies our theory of dynamics. The typical presentation of such a
calculus follows the style of giving generators and relations on
them. The grammar, below, describing term constructors, freely
generates the set of processes, $\Proc$. This set is then quotiented
by a relation known as structural congruence and it is over this set
that the notion of dynamics is expressed. This presentation is
essentially that of \cite{MeredithR05} with the addition of
polyadicity and summation. For readability we have relegated some of
the technical subtleties to an appendix.

\subsubsection{Process grammar}\label{subsub:process_grammar}

\begin{mathpar}
  \inferrule* [lab=synchronization] {} {{M} \bc \pzero \;|\; x?F \;|\; x!C }
  \and
  \inferrule* [lab=abstraction] {} {{F} \bc (x)P}
  \and
  \inferrule* [lab=concretion] {} {{C} \bc \langle Q \rangle}
  \and
  \inferrule* [lab=process] {} {{P,Q} \bc M \;| \;P|Q \;|\; @{x}}
  \and
  \inferrule* [lab=name] {} {{x} \bc \quotep{P}}
\end{mathpar} 

Note that $\vec{x}$ (resp. $\vec{P}$) denotes a vector of names
(resp. processes) of length $|\vec{x}|$ (resp. $|\vec{P}|$). We adopt
the following useful abbreviations.

\begin{mathpar}
   x?(\vec{y}).P := x.(\vec{y})P \and  x\clift{\vec{P}} := x.\clift{\vec{P}}
   \and x!(y) := \lift{x}{\dropn{y}}
   \and \Pi_{i=0}^{n-1}P_i := P_0 | \ldots | P_{n-1}
\end{mathpar}

\subsubsection{Structural congruence}

\paragraph{Free and bound names and alpha-equivalence.} At the
core of structural equivalence is alpha-equivalence which identifies
process that are the same up to a change of variable. Formally, we
recognize the distinction between free and bound names. The free names
of a process, $\freenames{P}$, may be calculated recursively as
follows:

\begin{mathpar}
\freenames{\pzero} := \emptyset
  \and \\
  \freenames{x?(y).P} := \{ x \} \cup (\freenames{P} \setminus \{ y \})
  \and 
  \freenames{x!\langle P \rangle} := \{ x \} \cup \{ P \} 
  \and \\
  \freenames{P|Q} := \freenames{P} \cup \freenames{Q}
  \and \\
  \freenames{@{x}} := \{ x \}
\end{mathpar}

$\pi$
$\quotep{\pi}$

$\freenames{-} : \pi \to \mathcal{P}(\quotep{\pi})$

\begin{eqnarray*}
  \freenames{\pzero} & := & \emptyset \\
  \freenames{x?(y).P} & := & \{ x \} \cup (\freenames{P} \setminus \{ y \}) \\
  \freenames{x!\langle P \rangle} & := & \{ x \} \cup \{ P \} \\
  \freenames{P|Q} & := & \freenames{P} \cup \freenames{Q} \\
  \freenames{\dropn{x}} & := & \{ x \}
\end{eqnarray*}

The bound names of a process, $\boundnames{P}$, are those names occurring in $P$
that are not free. For example, in $x?(y).0$, the name $x$ is free, while $y$ is bound.

\begin{mathpar}
  \inferrule* [lab=monoidal-laws] {} { P|Q \equiv Q|P \and P|0 \equiv P \and P|(Q|R) \equiv (P|Q)|R }
\end{mathpar}

\begin{mathpar}
  \inferrule* [lab=alpha-equivalence] {} { (x)P \equiv (y)P\{y/x\} \and y \not\in \freenames{P} }
\end{mathpar}

\begin{definition}
Then two processes, $P,Q$, are alpha-equivalent if $P = Q\{\vec{y}/\vec{x}\}$ for
some $\vec{x} \in \boundnames{Q},\vec{y} \in \boundnames{P}$, where $Q\{\vec{y}/\vec{x}\}$
denotes the capture-avoiding substitution of $\vec{y}$ for $\vec{x}$ in $Q$.
\end{definition}

\begin{definition}
  The {\em structural congruence} \cite{SangiorgiWalker} , $\equiv$,
  between processes is the least congruence containing
  alpha-equivalence, satisfying the abelian monoid laws
  (associativity, commutativity and $\pzero$ as identity) for parallel
  composition $|$ and for summation $+$.
\end{definition}

\subsection{Name equivalence}

We take name equivalence, written $\nameeq$, to be the smallest
equivalence relation generated by the following rules.

\begin{mathpar}
\inferrule*[lab=Quote-drop]
{ }
{ \quotep{@{x}} \nameeq x }

\inferrule*[lab=Struct-equiv]
{ P \scong Q }
{ \quotep{P} \nameeq \quotep{Q} }
\end{mathpar}

The astute reader will have noticed that the mutual recursion of names
and processes imposes a mutual recursion on alpha-equivalence and
structural equivalence via name-equivalence. Fortunately, all of this
works out pleasantly and we may calculate in the natural way, free of
concern. The reader interested in the details is referred to the
appendix \ref{appendix:rho_details}.

\subsection{Substitution}

We use $\Proc$ for the set of processes, $\QProc$ for the set of
names, and $\id{\{}\vec{y} / \vec{x} \id{\}}$ to denote partial maps,
$s : \QProc \rightarrow \QProc$. A map, $s$ lifts, uniquely, to a map
on process terms, $\widehat{s} : \Proc \rightarrow \Proc$ by the
following equations.

\begin{mathpar}
  (0) \psubstp{Q}{P} := 0 \\
  (R \juxtap S) \psubstp{Q}{P}
  :=    
  (R)\psubstp{Q}{P} \juxtap (S) \psubstp{Q}{P} \\
  (x?(y).R) \psubstp{Q}{P}    
  :=    
  (x)\substp{Q}{P} (z)\concat( (R \psubstn{z}{y}) \psubstp{Q}{P} ) \\
  (\lift{x}{R}) \psubstp{Q}{P}  
  :=
  \lift{(x)\substp{Q}{P}}{ R \psubstp{Q}{P} } \\
%   (\dropn{x})  \psubstp{Q}{P}       
%   := 
%   \left\{ 
%     \begin{array}{ccc} 
%       \dropn{\quotep{Q}} & & x \nameeq \quotep{P} \\
%       \dropn{x} & & otherwise \\
%     \end{array}
%   \right. 
  (\dropn{x})  \psubstp{Q}{P}       
  := 
  \left\{ 
    \begin{array}{ccc} 
      Q & & x \nameeq \quotep{P} \\
      \dropn{x} & & otherwise \\
    \end{array}
  \right.
\end{mathpar}
 

where

\begin{eqnarray}
  (x)\id{\{} \lpquote Q \rpquote / \lpquote P \rpquote \id{\}}            = 
  \left\{ 
    \begin{array}{ccc}
      \lpquote Q \rpquote & & x \nameeq \lpquote P \rpquote \\
      x & & otherwise \\
    \end{array}
  \right. \nonumber
\end{eqnarray}

and $z$ is chosen distinct from $\quotep{P}$, $\quotep{Q}$, the free
names in $Q$, and all the names in $R$. Our $\alpha$-equivalence will
be built in the standard way from this substitution.

\begin{remark}\label{rem:no_self_referential_names}
  One consequence of these definitions is that $\forall P. \quotep{P}
  \not\in \freenames{P}$.
\end{remark}

\subsection{ Dynamic quote: an example }

Anticipating something of what's to come, consider applying the
substitution, $\widehat{\id{\{}u / z \id{\}}}$, to the following pair
of processes, $\lift{w}{y!(z)}$ and $w[ \lpquote y!(z) \rpquote ]$.

\begin{eqnarray}
	\lift{w}{y!(z)}\widehat{\id{\{}u / z \id{\}}}
		& = &
		\lift{w}{y!(u)} \nonumber\\
	w[ \lpquote y!(z) \rpquote ] \widehat{ \id{\{}u / z \id{\}} }
		& = &
		w[ \lpquote y!(z) \rpquote ] \nonumber
\end{eqnarray}

Because the body of the process between quotes is impervious to
substitution, we get radically different answers. In fact, by
examining the first process in an input context,
e.g. $x?(z).\lift{w}{y!(z)}$, we see that the process under the lift
operator may be shaped by prefixed inputs binding a name inside it. In
this sense, the lift operator will be seen as a way to dynamically
construct processes before reifying them as names.

Finally equipped with these standard features we can present the
dynamics of the calculus.

\subsubsection{Operational semantics} 

Finally, we introduce the computational dynamics. What marks these
algebras as distinct from other more traditionally studied algebraic
structures, e.g. vector spaces or polynomial rings, is the manner in
which dynamics is captured. In traditional structures, dynamics is typically
expressed through morphisms between such structures, as in linear maps
between vector spaces or morphisms between rings. In algebras
associated with the semantics of computation, the dynamics is
expressed as part of the algebraic structure itself, through a
reduction reduction relation typically denoted by $\red$. Below, we
give a recursive presentation of this relation for the calculus used
in the encoding.

$\red \subseteq \pi \times \pi$
$\red : \pi \to \mathcal{P}(\pi)$

\begin{mathpar}
  \inferrule* [lab=Comm] { \textsf{match}( x_{src}, x_{trgt} ) } { x_{trgt}?(y)P \; | \; x_{src}!\langle {Q} \rangle \red P\{\quotep{Q}/y}\} }
  \and \\
  \inferrule* [lab=Par] {{P} \red {P}'} {{{P} | {Q}} \red {{P}' | {Q}}}
  \and
  \inferrule* [lab=Equiv]{{{P} \scong {P}'} \andalso {{P}' \red {Q}'} \andalso {{Q}' \scong {Q}}}{{P} \red {Q}}
\end{mathpar}

\begin{eqnarray*}
  match_{\equiv} (\quotep{P},\quotep{Q}) & := & P \equiv Q \\
  match_{\dagger}(\quotep{P},\quotep{Q}) & := & \forall R. P|Q \red^{*} R => R \red^{*} 0 \\
  match_{K}(\quotep{P},\quotep{Q}) & := & K \mbox{ for some context } K
\end{eqnarray*}

$u?(x)P | u!\langle Q \rangle \red P\{\quotep{Q}/x\}$

%We write $\wred$ for $\red^*$, and $P\red$ if $\exists Q $ such that $ P \red Q$.
We write $P\red$ if $\exists Q $ such that $ P \red Q$ and $P\not\red$, otherwise.

\section{Replication}

As mentioned before, it is known that replication (and hence
recursion) can be implemented in a higher-order process algebra
\cite{SangiorgiWalker}. As our first example of calculation with the
machinery thus far presented we give the construction explicitly in
the {\rhoc}.

\begin{eqnarray}
	D_{x} & := & \prefix{x}{y}{(\binpar{\outputp{x}{y}}{@{y}})} \nonumber\\
	\bangp_{x}{P} & := & \binpar{{x}!\langle{\binpar{D_{x}}{P}}\rangle}{D_{x}} \nonumber
\end{eqnarray}

\begin{eqnarray}
	\bangp_{x}{P} & & \nonumber\\
	=
	& {x}!\langle{(\prefix{x}{y}{(\outputp{x}{y} | @{y})) | P}}\rangle 
	      | \prefix{x}{y}{(\outputp{x}{y} | @{y})} & \nonumber\\
	\red
	& (\outputp{x}{y} | @{y})\substn{\quotep{(\prefix{x}{y}{(@{y} | \outputp{x}{y})) | P}}}{y} & \nonumber\\
	=
	& \outputp{x}{\quotep{(\prefix{x}{y}{(\outputp{x}{y} | @{y})) | P}}}
	  | {(\prefix{x}{y}{(\outputp{x}{y} | @{y})) | P}} & \nonumber\\
	\red
	& \ldots & \nonumber\\
	\red^*
	& P | P | \ldots & \nonumber
\end{eqnarray}

Of course, this encoding, as an implementation, runs away, unfolding
$\bangp{P}$ eagerly. A lazier and more implementable replication
operator, restricted to input-guarded processes, may be obtained as follows.

\begin{eqnarray}
\bangp{\prefix{u}{v}{P}} 
	:= 
	\binpar{\lift{x}{\prefix{u}{v}{(\binpar{D(x)}{P})}}}{D(x)} \nonumber
\end{eqnarray}

\begin{remark}
  Note that the lazier definition still does not deal with summation
  or mixed summation (i.e. sums over input and output). The reader is
  invited to construct definitions of replication that deal with these
  features. 

  Further, the definitions are parameterized in a name, $x$. Can you,
  gentle reader, make a definition that eliminates this parameter and
  guarantees no accidental interaction between the replication
  machinery and the process being replicated -- i.e. no accidental
  sharing of names used by the process to get its work done and the
  name(s) used by the replication to effect copying. This latter
  revision of the definition of replication is crucial to obtaining
  the expected identity $!!P \sim !P$.
\end{remark}

\begin{remark}\label{rem:paradoxical_combinator}
  The reader familiar with the lambda calculus will have noticed the
  similarity between $D$ and the paradoxical combinator.

  [Ed. note: the existence of this seems to suggest we have to be more
  restrictive on the set of processes and names we admit if we are to
  support no-cloning.]
\end{remark}

\subsubsection{Bisimulation}

The computational dynamics gives rise to another kind of equivalence,
the equivalence of computational behavior. As previously mentioned
this is typically captured \emph{via} some form of bisimulation.

% The notion we use in this paper is weak barbed bisimulation
% \cite{milner91polyadicpi}.

The notion we use in this paper is derived from weak barbed
bisimulation \cite{milner91polyadicpi}. 

\begin{definition}
An \emph{observation relation}, $\downarrow_{\mathcal N}$, over a set
of names, $\mathcal N$, is the smallest relation satisfying the rules
below.

\infrule[Out-barb]{y \in {\mathcal N}, \; x \nameeq y}
		  {\outputp{x}{v} \downarrow_{\mathcal N} x}
\infrule[Par-barb]{\mbox{$P\downarrow_{\mathcal N} x$ or $Q\downarrow_{\mathcal N} x$}}
		  {\binpar{P}{Q} \downarrow_{\mathcal N} x}

We write $P \Downarrow_{\mathcal N} x$ if there is $Q$ such that 
$P \wred Q$ and $Q \downarrow_{\mathcal N} x$.
\end{definition}

\begin{definition}
%\label{def.bbisim}
An  ${\mathcal N}$-\emph{barbed bisimulation} over a set of names, ${\mathcal N}$, is a symmetric binary relation 
${\mathcal S}_{\mathcal N}$ between agents such that $P\rel{S}_{\mathcal N}Q$ implies:
\begin{enumerate}
\item If $P \red P'$ then $Q \wred Q'$ and $P'\rel{S}_{\mathcal N} Q'$.
\item If $P\downarrow_{\mathcal N} x$, then $Q\Downarrow_{\mathcal N} x$.
\end{enumerate}
$P$ is ${\mathcal N}$-barbed bisimilar to $Q$, written
$P \wbbisim_{\mathcal N} Q$, if $P \rel{S}_{\mathcal N} Q$ for some ${\mathcal N}$-barbed bisimulation ${\mathcal S}_{\mathcal N}$.
\end{definition}

$\mathcal{R} \subseteq \pi \times \pi$

$P \mathcal{R} Q => \forall P'. P \red P' \Rightarrow \exists Q'. Q \red Q', P' \mathcal{R} Q'$

$P \vdash x \Rightarrow Q \vdash x$

\begin{mathpar}
  \inferrule*[lab=Out-barb]{x \nameeq y}{{y}!\langle{Q}\rangle \vdash x}
  \and
  \inferrule*[lab=Par-barb]{\mbox{$P\vdash x$ or $Q\vdash x$}}{\binpar{P}{Q} \vdash x}
\end{mathpar}

\subsubsection{Contexts}

One of the principle advantages of computational calculi like the
$\pi$-calculus is a well-defined notion of context,
contextual-equivalence and a correlation between
contextual-equivalence and notions of bisimulation. The notion of
context allows the decomposition of a process into (sub-)process and
its syntactic environment, its context. Thus, a context may be
thought of as a process with a ``hole'' (written $\Box$) in it. The
application of a context $M$ to a process $P$, written $M[P]$, is
tantamount to filling the hole in $M$ with $P$. In this paper we do
not need the full weight of this theory, but do make use of the notion
of context in the proof the main theorem. 

\begin{mathpar}
  \inferrule* [lab=summation] {} {{M_{M},M_{N}} \bc \Box \;|\; x.M_{A} \;|\; M_{M}+M_{N}}
  \and
  \inferrule* [lab=agent] {} {{M_{A}} \bc (\vec{x})M_{P} \;| \; \clift{P_0,\ldots,M_{P},\ldots,P_N}}
  \and \\
  \inferrule* [lab=process] {} {{M_{P}} \bc M_{N} \;| \;P|M_{P} }
\end{mathpar} 

\begin{mathpar}
  \inferrule* [lab=sychronization] {} {M_{N} \bc \Box \;|\; x?M_{F} \;|\; x!M_{C}}
  \and
  \inferrule* [lab=abstraction] {} {{M_{F}} \bc (x)M_{P} }
  \and
  \inferrule* [lab=concretion] {} {{M_{C}} \bc \langle M_{P} \rangle }
  \and \\
  \inferrule* [lab=process] {} {{M_{P}} \bc M_{N} \;| \;P|M_{P} }
\end{mathpar}

\begin{definition}[contextual application] Given a context $M$, and
  process $P$, we define the \emph{contextual application}, $M[P] :=
  M\{P/\Box\}$. That is, the contextual application of M to P is the
  substitution of $P$ for $\Box$ in $M$.
\end{definition}

$\meaningof{-} : L \to \mathcal{P}(\pi)$

\begin{mathpar}
  \inferrule* [lab=collection] {} {\meaningof{true} = \pi, \and \meaningof{~E} = \pi \setminus \meaningof{E}, \and \meaningof{E_{1} \& E_{2}} = \meaningof{E_{1}} \cap \meaningof{E_{2}}}
\end{mathpar}

\begin{mathpar}
  \inferrule* [lab=structure] {} {\meaningof{0} = \{ P \in \pi | P \equiv 0 \}, \and \\ \meaningof{E_1 | E_2} = \{ P \in \pi | P \equiv P_{1} | P_{2}, P_{1} \in \meaningof{E_{1}}, P_{2} \in \meaningof{E_2}\} }
\end{mathpar}

\begin{mathpar}
 \inferrule* [lab=behavior] {} {\meaningof{\langle a?b \rangle E} = \{ P \in \pi | P \equiv Q | u?(y)P', \\ \and \\\\ \and \\ \;\;\; u \in \meaningof{a}, \forall z.P'\{z/y\} \in \meaningof{E\{z/b\}}\}, \and \\ \meaningof{a!E} = \{ P \in \pi | P \equiv Q | x!\langle P' \rangle, x \in \meaningof{a} P' \in \meaningof{E}\} }
\end{mathpar}

\begin{mathpar}
 \inferrule* [lab=nominal] {} {\meaningof{\quotep{E}} = \{ \quotep{P} \in \quotep{\pi} | P \in \meaningof{E} \}, \and \meaningof{\quotep{P}} = \{ \quotep{Q} \in \quotep{\pi} | P \equiv Q \} \and \\ \meaningof{@\quotep{E}} = \{ P \in \pi | P \equiv @x, x \in \meaningof{E} \}}
\end{mathpar}

\begin{eqnarray*}
  \\
  \meaningof{-} : TS \to ST
\end{eqnarray*}

\begin{eqnarray*}
  \\
  L : TS \to ST
\end{eqnarray*}

\begin{eqnarray*}
  \\
  P \models E \iff P \in \meaningof{E}
\end{eqnarray*}

\begin{eqnarray*}
  P \approx_{L} Q \iff \forall E \in L. P \models E \iff Q \models E
\end{eqnarray*}

\begin{eqnarray*}
  P \approx_{K} Q
\end{eqnarray*}

\begin{eqnarray*}
  P \approx Q
\end{eqnarray*}

$\approx_{K} = \approx = \approx_{L}$

\subsubsection{Contextual duality}

Note that contexts extend the quotation operation to a family of
operations from processes to names. Given a context, $M$, we can
define a \emph{nominal context}, $\quotep{M}$ by $\quotep{M}[P] :=
\quotep{M[P]}$. To foreshadow what is to come we observe that these
operations enjoy a duality with processes very much like the duality
between vectors and maps from vectors to scalars.

Further, because the calculus is essentially higher-order, we have a
correspondence between contexts and processes. More specifically,
given a name $x$ and a context $M$ we can construct $M^{*}_{x}$ such
that 

\begin{mathpar}
  M^{*}_{x} | \lift{x}{P} \red M[P]
\end{mathpar}

namely,

\begin{mathpar}
  M^{*}_{x} := x?(u).M[\dropn{u}]
\end{mathpar}

The dependence of $M^{*}_{x}$ on a name makes it an abstraction, 

\begin{mathpar}
  M^{*} := (x)x?(u).M[\dropn{u}]
\end{mathpar}

\subsection{Additional notation}

It will sometimes be convenient to denote the process a name
quotes. We already have the notation $x = \quotep{P}$, but it will be
convenient to introduce an alternate notation, $\procn{x}$, when we
want to emphasize the connection to the use of the name. Note that, by
virtue of name equivalence, $\quotep{\procn{x}} \nameeq x$; so, the
notation is consistent with previous definitions.

Further, because names have structure it is possible to effect
substitutions on the basis of that structure. This means we need to
upgrade our notation for substitutions, which we accomplish by
adapting comprehension notation. Thus,

\begin{mathpar}
  P\{ y / x : x \in S \}
\end{mathpar}

is interpreted to mean the process derived from P by replacing (in a
capture-avoiding manner) each occurrence of $x$ in $S$ by $y$. For example,

\begin{mathpar}
  P\{ \quotep{\procn{x}|\procn{x}} / x : x \in \freenames{P} \}
\end{mathpar}

will replace each (occurrence) of a free name $x$ in $P$ by
$\quotep{\procn{x}|\procn{x}}$.

Also, we will avail ourselves of the notation $x^{L}$ and $x^{R}$ to
denote injections of a name into disjoint copies of the name
space. There are numerous ways to accomplish this. One example can be
found in \cite{MeredithR05}. This notation overloads to vectors of
names: $\vec{x}^{\pi} := (x_{i}^{\pi} \; : \; 0 \leq i < |\vec{x}| )$ where $\pi \in \{L,R\}$.

We also use $P^{\Box} := P|\Box$.

In \cite{MeredithR05} an interpretation of the new operator is
given. It turns out that there are several possible interpretations
all enjoying the requisite algebraic properties of the operator (see
\cite{milner91polyadicpi}). We will therefore make liberal use of
$(\nu\; \vec{x})P$.

% subsection the_syntax_and_semantics_of_the_notation_system (end)   

\input{qm2pi.qmops} 

\input{qm2pi.sterngerlach} 

\input{qm2pi.metric} 

% section concurrent_process_calculi (end)

%\input{qm2pi.proofsketch}

% section proof sketch (end)

%\input{qm2pi.slviaknots} 

% section spatial logic via knots (end)

\input{qm2pi.conclusion}

% section conclusion (end)

%\input{qm2pi.dtcodes} 

% section wiring algorithm (end)

\input{qm2pi.ack} 

% section acknowledgments (end)

\newpage


\bibliographystyle{plain}   
\bibliography{../../biblios/main.bib}

\input{qm2pi.rhodetails}

\end{document}

 

% section acknowledgments (end)

\newpage


\bibliographystyle{plain}   
\bibliography{../../biblios/main.bib}

\documentclass[12pt]{llncs}
%\documentclass{jktr}

\usepackage[pdftex]{hyperref}                   
\usepackage {listings}
\usepackage {mathpartir}
\usepackage{bcprules}
%\usepackage{listings}
                       
\usepackage{graphicx} 
%\usepackage[margins=2.5cm,nohead,nofoot]{geometry}
%\usepackage{geometry}
\usepackage{amsfonts}
\usepackage{amstext}
\usepackage{latexsym}
\usepackage{amssymb}
\usepackage{color}


%\include{myPreamble}
\include{qm2pi.local} 

%\ifpdf
%\usepackage[pdftex]{graphicx}
%\else
%\usepackage{graphicx}
%\fi

 % \ifpdf
%  \usepackage{pdfsync}
%  \if


%\title{Brief Article}
%\author{David F. Snyder}
%\author{L.G. Meredith}

%\address{Dept. of Math., Texas State University--San Marcos, San Marcos, TX 78666}
       
\pagestyle{empty}


\begin{document}

\lstset{language=[Objective]Caml,frame=shadowbox}

\input{qm2pi.front}

% section front matter (end)

\input{qm2pi.intro} 
 
% section introduction (end)

% \input{qm2pi.knotations} 

% section notation (end)

\input{qm2pi.process.calculi} 

% section concurrent_process_calculi_and_spatial_logics_ (end)
    
%\input{qm2pi.knots2pi} 

%\input{qm2pi.trefoil} 

%\input{qm2pi.mainthm} 

% subsection basic_interpretation (end)

%\input{qm2pi.rho.presentation} 
\subsection{The syntax and semantics of the notation system}\label{sub:the_syntax_and_semantics_of_the_notation_system} % (fold)

We now summarize a technical presentation of the calculus that
embodies our theory of dynamics. The typical presentation of such a
calculus follows the style of giving generators and relations on
them. The grammar, below, describing term constructors, freely
generates the set of processes, $\Proc$. This set is then quotiented
by a relation known as structural congruence and it is over this set
that the notion of dynamics is expressed. This presentation is
essentially that of \cite{MeredithR05} with the addition of
polyadicity and summation. For readability we have relegated some of
the technical subtleties to an appendix.

\subsubsection{Process grammar}\label{subsub:process_grammar}

\begin{mathpar}
  \inferrule* [lab=synchronization] {} {{M} \bc \pzero \;|\; x?F \;|\; x!C }
  \and
  \inferrule* [lab=abstraction] {} {{F} \bc (x)P}
  \and
  \inferrule* [lab=concretion] {} {{C} \bc \langle Q \rangle}
  \and
  \inferrule* [lab=process] {} {{P,Q} \bc M \;| \;P|Q \;|\; @{x}}
  \and
  \inferrule* [lab=name] {} {{x} \bc \quotep{P}}
\end{mathpar} 

Note that $\vec{x}$ (resp. $\vec{P}$) denotes a vector of names
(resp. processes) of length $|\vec{x}|$ (resp. $|\vec{P}|$). We adopt
the following useful abbreviations.

\begin{mathpar}
   x?(\vec{y}).P := x.(\vec{y})P \and  x\clift{\vec{P}} := x.\clift{\vec{P}}
   \and x!(y) := \lift{x}{\dropn{y}}
   \and \Pi_{i=0}^{n-1}P_i := P_0 | \ldots | P_{n-1}
\end{mathpar}

\subsubsection{Structural congruence}

\paragraph{Free and bound names and alpha-equivalence.} At the
core of structural equivalence is alpha-equivalence which identifies
process that are the same up to a change of variable. Formally, we
recognize the distinction between free and bound names. The free names
of a process, $\freenames{P}$, may be calculated recursively as
follows:

\begin{mathpar}
\freenames{\pzero} := \emptyset
  \and \\
  \freenames{x?(y).P} := \{ x \} \cup (\freenames{P} \setminus \{ y \})
  \and 
  \freenames{x!\langle P \rangle} := \{ x \} \cup \{ P \} 
  \and \\
  \freenames{P|Q} := \freenames{P} \cup \freenames{Q}
  \and \\
  \freenames{@{x}} := \{ x \}
\end{mathpar}

$\pi$
$\quotep{\pi}$

$\freenames{-} : \pi \to \mathcal{P}(\quotep{\pi})$

\begin{eqnarray*}
  \freenames{\pzero} & := & \emptyset \\
  \freenames{x?(y).P} & := & \{ x \} \cup (\freenames{P} \setminus \{ y \}) \\
  \freenames{x!\langle P \rangle} & := & \{ x \} \cup \{ P \} \\
  \freenames{P|Q} & := & \freenames{P} \cup \freenames{Q} \\
  \freenames{\dropn{x}} & := & \{ x \}
\end{eqnarray*}

The bound names of a process, $\boundnames{P}$, are those names occurring in $P$
that are not free. For example, in $x?(y).0$, the name $x$ is free, while $y$ is bound.

\begin{mathpar}
  \inferrule* [lab=monoidal-laws] {} { P|Q \equiv Q|P \and P|0 \equiv P \and P|(Q|R) \equiv (P|Q)|R }
\end{mathpar}

\begin{mathpar}
  \inferrule* [lab=alpha-equivalence] {} { (x)P \equiv (y)P\{y/x\} \and y \not\in \freenames{P} }
\end{mathpar}

\begin{definition}
Then two processes, $P,Q$, are alpha-equivalent if $P = Q\{\vec{y}/\vec{x}\}$ for
some $\vec{x} \in \boundnames{Q},\vec{y} \in \boundnames{P}$, where $Q\{\vec{y}/\vec{x}\}$
denotes the capture-avoiding substitution of $\vec{y}$ for $\vec{x}$ in $Q$.
\end{definition}

\begin{definition}
  The {\em structural congruence} \cite{SangiorgiWalker} , $\equiv$,
  between processes is the least congruence containing
  alpha-equivalence, satisfying the abelian monoid laws
  (associativity, commutativity and $\pzero$ as identity) for parallel
  composition $|$ and for summation $+$.
\end{definition}

\subsection{Name equivalence}

We take name equivalence, written $\nameeq$, to be the smallest
equivalence relation generated by the following rules.

\begin{mathpar}
\inferrule*[lab=Quote-drop]
{ }
{ \quotep{@{x}} \nameeq x }

\inferrule*[lab=Struct-equiv]
{ P \scong Q }
{ \quotep{P} \nameeq \quotep{Q} }
\end{mathpar}

The astute reader will have noticed that the mutual recursion of names
and processes imposes a mutual recursion on alpha-equivalence and
structural equivalence via name-equivalence. Fortunately, all of this
works out pleasantly and we may calculate in the natural way, free of
concern. The reader interested in the details is referred to the
appendix \ref{appendix:rho_details}.

\subsection{Substitution}

We use $\Proc$ for the set of processes, $\QProc$ for the set of
names, and $\id{\{}\vec{y} / \vec{x} \id{\}}$ to denote partial maps,
$s : \QProc \rightarrow \QProc$. A map, $s$ lifts, uniquely, to a map
on process terms, $\widehat{s} : \Proc \rightarrow \Proc$ by the
following equations.

\begin{mathpar}
  (0) \psubstp{Q}{P} := 0 \\
  (R \juxtap S) \psubstp{Q}{P}
  :=    
  (R)\psubstp{Q}{P} \juxtap (S) \psubstp{Q}{P} \\
  (x?(y).R) \psubstp{Q}{P}    
  :=    
  (x)\substp{Q}{P} (z)\concat( (R \psubstn{z}{y}) \psubstp{Q}{P} ) \\
  (\lift{x}{R}) \psubstp{Q}{P}  
  :=
  \lift{(x)\substp{Q}{P}}{ R \psubstp{Q}{P} } \\
%   (\dropn{x})  \psubstp{Q}{P}       
%   := 
%   \left\{ 
%     \begin{array}{ccc} 
%       \dropn{\quotep{Q}} & & x \nameeq \quotep{P} \\
%       \dropn{x} & & otherwise \\
%     \end{array}
%   \right. 
  (\dropn{x})  \psubstp{Q}{P}       
  := 
  \left\{ 
    \begin{array}{ccc} 
      Q & & x \nameeq \quotep{P} \\
      \dropn{x} & & otherwise \\
    \end{array}
  \right.
\end{mathpar}
 

where

\begin{eqnarray}
  (x)\id{\{} \lpquote Q \rpquote / \lpquote P \rpquote \id{\}}            = 
  \left\{ 
    \begin{array}{ccc}
      \lpquote Q \rpquote & & x \nameeq \lpquote P \rpquote \\
      x & & otherwise \\
    \end{array}
  \right. \nonumber
\end{eqnarray}

and $z$ is chosen distinct from $\quotep{P}$, $\quotep{Q}$, the free
names in $Q$, and all the names in $R$. Our $\alpha$-equivalence will
be built in the standard way from this substitution.

\begin{remark}\label{rem:no_self_referential_names}
  One consequence of these definitions is that $\forall P. \quotep{P}
  \not\in \freenames{P}$.
\end{remark}

\subsection{ Dynamic quote: an example }

Anticipating something of what's to come, consider applying the
substitution, $\widehat{\id{\{}u / z \id{\}}}$, to the following pair
of processes, $\lift{w}{y!(z)}$ and $w[ \lpquote y!(z) \rpquote ]$.

\begin{eqnarray}
	\lift{w}{y!(z)}\widehat{\id{\{}u / z \id{\}}}
		& = &
		\lift{w}{y!(u)} \nonumber\\
	w[ \lpquote y!(z) \rpquote ] \widehat{ \id{\{}u / z \id{\}} }
		& = &
		w[ \lpquote y!(z) \rpquote ] \nonumber
\end{eqnarray}

Because the body of the process between quotes is impervious to
substitution, we get radically different answers. In fact, by
examining the first process in an input context,
e.g. $x?(z).\lift{w}{y!(z)}$, we see that the process under the lift
operator may be shaped by prefixed inputs binding a name inside it. In
this sense, the lift operator will be seen as a way to dynamically
construct processes before reifying them as names.

Finally equipped with these standard features we can present the
dynamics of the calculus.

\subsubsection{Operational semantics} 

Finally, we introduce the computational dynamics. What marks these
algebras as distinct from other more traditionally studied algebraic
structures, e.g. vector spaces or polynomial rings, is the manner in
which dynamics is captured. In traditional structures, dynamics is typically
expressed through morphisms between such structures, as in linear maps
between vector spaces or morphisms between rings. In algebras
associated with the semantics of computation, the dynamics is
expressed as part of the algebraic structure itself, through a
reduction reduction relation typically denoted by $\red$. Below, we
give a recursive presentation of this relation for the calculus used
in the encoding.

$\red \subseteq \pi \times \pi$
$\red : \pi \to \mathcal{P}(\pi)$

\begin{mathpar}
  \inferrule* [lab=Comm] { \textsf{match}( x_{src}, x_{trgt} ) } { x_{trgt}?(y)P \; | \; x_{src}!\langle {Q} \rangle \red P\{\quotep{Q}/y}\} }
  \and \\
  \inferrule* [lab=Par] {{P} \red {P}'} {{{P} | {Q}} \red {{P}' | {Q}}}
  \and
  \inferrule* [lab=Equiv]{{{P} \scong {P}'} \andalso {{P}' \red {Q}'} \andalso {{Q}' \scong {Q}}}{{P} \red {Q}}
\end{mathpar}

\begin{eqnarray*}
  match_{\equiv} (\quotep{P},\quotep{Q}) & := & P \equiv Q \\
  match_{\dagger}(\quotep{P},\quotep{Q}) & := & \forall R. P|Q \red^{*} R => R \red^{*} 0 \\
  match_{K}(\quotep{P},\quotep{Q}) & := & K \mbox{ for some context } K
\end{eqnarray*}

$u?(x)P | u!\langle Q \rangle \red P\{\quotep{Q}/x\}$

%We write $\wred$ for $\red^*$, and $P\red$ if $\exists Q $ such that $ P \red Q$.
We write $P\red$ if $\exists Q $ such that $ P \red Q$ and $P\not\red$, otherwise.

\section{Replication}

As mentioned before, it is known that replication (and hence
recursion) can be implemented in a higher-order process algebra
\cite{SangiorgiWalker}. As our first example of calculation with the
machinery thus far presented we give the construction explicitly in
the {\rhoc}.

\begin{eqnarray}
	D_{x} & := & \prefix{x}{y}{(\binpar{\outputp{x}{y}}{@{y}})} \nonumber\\
	\bangp_{x}{P} & := & \binpar{{x}!\langle{\binpar{D_{x}}{P}}\rangle}{D_{x}} \nonumber
\end{eqnarray}

\begin{eqnarray}
	\bangp_{x}{P} & & \nonumber\\
	=
	& {x}!\langle{(\prefix{x}{y}{(\outputp{x}{y} | @{y})) | P}}\rangle 
	      | \prefix{x}{y}{(\outputp{x}{y} | @{y})} & \nonumber\\
	\red
	& (\outputp{x}{y} | @{y})\substn{\quotep{(\prefix{x}{y}{(@{y} | \outputp{x}{y})) | P}}}{y} & \nonumber\\
	=
	& \outputp{x}{\quotep{(\prefix{x}{y}{(\outputp{x}{y} | @{y})) | P}}}
	  | {(\prefix{x}{y}{(\outputp{x}{y} | @{y})) | P}} & \nonumber\\
	\red
	& \ldots & \nonumber\\
	\red^*
	& P | P | \ldots & \nonumber
\end{eqnarray}

Of course, this encoding, as an implementation, runs away, unfolding
$\bangp{P}$ eagerly. A lazier and more implementable replication
operator, restricted to input-guarded processes, may be obtained as follows.

\begin{eqnarray}
\bangp{\prefix{u}{v}{P}} 
	:= 
	\binpar{\lift{x}{\prefix{u}{v}{(\binpar{D(x)}{P})}}}{D(x)} \nonumber
\end{eqnarray}

\begin{remark}
  Note that the lazier definition still does not deal with summation
  or mixed summation (i.e. sums over input and output). The reader is
  invited to construct definitions of replication that deal with these
  features. 

  Further, the definitions are parameterized in a name, $x$. Can you,
  gentle reader, make a definition that eliminates this parameter and
  guarantees no accidental interaction between the replication
  machinery and the process being replicated -- i.e. no accidental
  sharing of names used by the process to get its work done and the
  name(s) used by the replication to effect copying. This latter
  revision of the definition of replication is crucial to obtaining
  the expected identity $!!P \sim !P$.
\end{remark}

\begin{remark}\label{rem:paradoxical_combinator}
  The reader familiar with the lambda calculus will have noticed the
  similarity between $D$ and the paradoxical combinator.

  [Ed. note: the existence of this seems to suggest we have to be more
  restrictive on the set of processes and names we admit if we are to
  support no-cloning.]
\end{remark}

\subsubsection{Bisimulation}

The computational dynamics gives rise to another kind of equivalence,
the equivalence of computational behavior. As previously mentioned
this is typically captured \emph{via} some form of bisimulation.

% The notion we use in this paper is weak barbed bisimulation
% \cite{milner91polyadicpi}.

The notion we use in this paper is derived from weak barbed
bisimulation \cite{milner91polyadicpi}. 

\begin{definition}
An \emph{observation relation}, $\downarrow_{\mathcal N}$, over a set
of names, $\mathcal N$, is the smallest relation satisfying the rules
below.

\infrule[Out-barb]{y \in {\mathcal N}, \; x \nameeq y}
		  {\outputp{x}{v} \downarrow_{\mathcal N} x}
\infrule[Par-barb]{\mbox{$P\downarrow_{\mathcal N} x$ or $Q\downarrow_{\mathcal N} x$}}
		  {\binpar{P}{Q} \downarrow_{\mathcal N} x}

We write $P \Downarrow_{\mathcal N} x$ if there is $Q$ such that 
$P \wred Q$ and $Q \downarrow_{\mathcal N} x$.
\end{definition}

\begin{definition}
%\label{def.bbisim}
An  ${\mathcal N}$-\emph{barbed bisimulation} over a set of names, ${\mathcal N}$, is a symmetric binary relation 
${\mathcal S}_{\mathcal N}$ between agents such that $P\rel{S}_{\mathcal N}Q$ implies:
\begin{enumerate}
\item If $P \red P'$ then $Q \wred Q'$ and $P'\rel{S}_{\mathcal N} Q'$.
\item If $P\downarrow_{\mathcal N} x$, then $Q\Downarrow_{\mathcal N} x$.
\end{enumerate}
$P$ is ${\mathcal N}$-barbed bisimilar to $Q$, written
$P \wbbisim_{\mathcal N} Q$, if $P \rel{S}_{\mathcal N} Q$ for some ${\mathcal N}$-barbed bisimulation ${\mathcal S}_{\mathcal N}$.
\end{definition}

$\mathcal{R} \subseteq \pi \times \pi$

$P \mathcal{R} Q => \forall P'. P \red P' \Rightarrow \exists Q'. Q \red Q', P' \mathcal{R} Q'$

$P \vdash x \Rightarrow Q \vdash x$

\begin{mathpar}
  \inferrule*[lab=Out-barb]{x \nameeq y}{{y}!\langle{Q}\rangle \vdash x}
  \and
  \inferrule*[lab=Par-barb]{\mbox{$P\vdash x$ or $Q\vdash x$}}{\binpar{P}{Q} \vdash x}
\end{mathpar}

\subsubsection{Contexts}

One of the principle advantages of computational calculi like the
$\pi$-calculus is a well-defined notion of context,
contextual-equivalence and a correlation between
contextual-equivalence and notions of bisimulation. The notion of
context allows the decomposition of a process into (sub-)process and
its syntactic environment, its context. Thus, a context may be
thought of as a process with a ``hole'' (written $\Box$) in it. The
application of a context $M$ to a process $P$, written $M[P]$, is
tantamount to filling the hole in $M$ with $P$. In this paper we do
not need the full weight of this theory, but do make use of the notion
of context in the proof the main theorem. 

\begin{mathpar}
  \inferrule* [lab=summation] {} {{M_{M},M_{N}} \bc \Box \;|\; x.M_{A} \;|\; M_{M}+M_{N}}
  \and
  \inferrule* [lab=agent] {} {{M_{A}} \bc (\vec{x})M_{P} \;| \; \clift{P_0,\ldots,M_{P},\ldots,P_N}}
  \and \\
  \inferrule* [lab=process] {} {{M_{P}} \bc M_{N} \;| \;P|M_{P} }
\end{mathpar} 

\begin{mathpar}
  \inferrule* [lab=sychronization] {} {M_{N} \bc \Box \;|\; x?M_{F} \;|\; x!M_{C}}
  \and
  \inferrule* [lab=abstraction] {} {{M_{F}} \bc (x)M_{P} }
  \and
  \inferrule* [lab=concretion] {} {{M_{C}} \bc \langle M_{P} \rangle }
  \and \\
  \inferrule* [lab=process] {} {{M_{P}} \bc M_{N} \;| \;P|M_{P} }
\end{mathpar}

\begin{definition}[contextual application] Given a context $M$, and
  process $P$, we define the \emph{contextual application}, $M[P] :=
  M\{P/\Box\}$. That is, the contextual application of M to P is the
  substitution of $P$ for $\Box$ in $M$.
\end{definition}

$\meaningof{-} : L \to \mathcal{P}(\pi)$

\begin{mathpar}
  \inferrule* [lab=collection] {} {\meaningof{true} = \pi, \and \meaningof{~E} = \pi \setminus \meaningof{E}, \and \meaningof{E_{1} \& E_{2}} = \meaningof{E_{1}} \cap \meaningof{E_{2}}}
\end{mathpar}

\begin{mathpar}
  \inferrule* [lab=structure] {} {\meaningof{0} = \{ P \in \pi | P \equiv 0 \}, \and \\ \meaningof{E_1 | E_2} = \{ P \in \pi | P \equiv P_{1} | P_{2}, P_{1} \in \meaningof{E_{1}}, P_{2} \in \meaningof{E_2}\} }
\end{mathpar}

\begin{mathpar}
 \inferrule* [lab=behavior] {} {\meaningof{\langle a?b \rangle E} = \{ P \in \pi | P \equiv Q | u?(y)P', \\ \and \\\\ \and \\ \;\;\; u \in \meaningof{a}, \forall z.P'\{z/y\} \in \meaningof{E\{z/b\}}\}, \and \\ \meaningof{a!E} = \{ P \in \pi | P \equiv Q | x!\langle P' \rangle, x \in \meaningof{a} P' \in \meaningof{E}\} }
\end{mathpar}

\begin{mathpar}
 \inferrule* [lab=nominal] {} {\meaningof{\quotep{E}} = \{ \quotep{P} \in \quotep{\pi} | P \in \meaningof{E} \}, \and \meaningof{\quotep{P}} = \{ \quotep{Q} \in \quotep{\pi} | P \equiv Q \} \and \\ \meaningof{@\quotep{E}} = \{ P \in \pi | P \equiv @x, x \in \meaningof{E} \}}
\end{mathpar}

\begin{eqnarray*}
  \\
  \meaningof{-} : TS \to ST
\end{eqnarray*}

\begin{eqnarray*}
  \\
  L : TS \to ST
\end{eqnarray*}

\begin{eqnarray*}
  \\
  P \models E \iff P \in \meaningof{E}
\end{eqnarray*}

\begin{eqnarray*}
  P \approx_{L} Q \iff \forall E \in L. P \models E \iff Q \models E
\end{eqnarray*}

\begin{eqnarray*}
  P \approx_{K} Q
\end{eqnarray*}

\begin{eqnarray*}
  P \approx Q
\end{eqnarray*}

$\approx_{K} = \approx = \approx_{L}$

\subsubsection{Contextual duality}

Note that contexts extend the quotation operation to a family of
operations from processes to names. Given a context, $M$, we can
define a \emph{nominal context}, $\quotep{M}$ by $\quotep{M}[P] :=
\quotep{M[P]}$. To foreshadow what is to come we observe that these
operations enjoy a duality with processes very much like the duality
between vectors and maps from vectors to scalars.

Further, because the calculus is essentially higher-order, we have a
correspondence between contexts and processes. More specifically,
given a name $x$ and a context $M$ we can construct $M^{*}_{x}$ such
that 

\begin{mathpar}
  M^{*}_{x} | \lift{x}{P} \red M[P]
\end{mathpar}

namely,

\begin{mathpar}
  M^{*}_{x} := x?(u).M[\dropn{u}]
\end{mathpar}

The dependence of $M^{*}_{x}$ on a name makes it an abstraction, 

\begin{mathpar}
  M^{*} := (x)x?(u).M[\dropn{u}]
\end{mathpar}

\subsection{Additional notation}

It will sometimes be convenient to denote the process a name
quotes. We already have the notation $x = \quotep{P}$, but it will be
convenient to introduce an alternate notation, $\procn{x}$, when we
want to emphasize the connection to the use of the name. Note that, by
virtue of name equivalence, $\quotep{\procn{x}} \nameeq x$; so, the
notation is consistent with previous definitions.

Further, because names have structure it is possible to effect
substitutions on the basis of that structure. This means we need to
upgrade our notation for substitutions, which we accomplish by
adapting comprehension notation. Thus,

\begin{mathpar}
  P\{ y / x : x \in S \}
\end{mathpar}

is interpreted to mean the process derived from P by replacing (in a
capture-avoiding manner) each occurrence of $x$ in $S$ by $y$. For example,

\begin{mathpar}
  P\{ \quotep{\procn{x}|\procn{x}} / x : x \in \freenames{P} \}
\end{mathpar}

will replace each (occurrence) of a free name $x$ in $P$ by
$\quotep{\procn{x}|\procn{x}}$.

Also, we will avail ourselves of the notation $x^{L}$ and $x^{R}$ to
denote injections of a name into disjoint copies of the name
space. There are numerous ways to accomplish this. One example can be
found in \cite{MeredithR05}. This notation overloads to vectors of
names: $\vec{x}^{\pi} := (x_{i}^{\pi} \; : \; 0 \leq i < |\vec{x}| )$ where $\pi \in \{L,R\}$.

We also use $P^{\Box} := P|\Box$.

In \cite{MeredithR05} an interpretation of the new operator is
given. It turns out that there are several possible interpretations
all enjoying the requisite algebraic properties of the operator (see
\cite{milner91polyadicpi}). We will therefore make liberal use of
$(\nu\; \vec{x})P$.

% subsection the_syntax_and_semantics_of_the_notation_system (end)   

\input{qm2pi.qmops} 

\input{qm2pi.sterngerlach} 

\input{qm2pi.metric} 

% section concurrent_process_calculi (end)

%\input{qm2pi.proofsketch}

% section proof sketch (end)

%\input{qm2pi.slviaknots} 

% section spatial logic via knots (end)

\input{qm2pi.conclusion}

% section conclusion (end)

%\input{qm2pi.dtcodes} 

% section wiring algorithm (end)

\input{qm2pi.ack} 

% section acknowledgments (end)

\newpage


\bibliographystyle{plain}   
\bibliography{../../biblios/main.bib}

\input{qm2pi.rhodetails}

\end{document}



\end{document}

 

% subsection basic_interpretation (end)

%\input{qm2pi.rho.presentation} 
\subsection{The syntax and semantics of the notation system}\label{sub:the_syntax_and_semantics_of_the_notation_system} % (fold)

We now summarize a technical presentation of the calculus that
embodies our theory of dynamics. The typical presentation of such a
calculus follows the style of giving generators and relations on
them. The grammar, below, describing term constructors, freely
generates the set of processes, $\Proc$. This set is then quotiented
by a relation known as structural congruence and it is over this set
that the notion of dynamics is expressed. This presentation is
essentially that of \cite{MeredithR05} with the addition of
polyadicity and summation. For readability we have relegated some of
the technical subtleties to an appendix.

\subsubsection{Process grammar}\label{subsub:process_grammar}

\begin{mathpar}
  \inferrule* [lab=synchronization] {} {{M} \bc \pzero \;|\; x?F \;|\; x!C }
  \and
  \inferrule* [lab=abstraction] {} {{F} \bc (x)P}
  \and
  \inferrule* [lab=concretion] {} {{C} \bc \langle Q \rangle}
  \and
  \inferrule* [lab=process] {} {{P,Q} \bc M \;| \;P|Q \;|\; @{x}}
  \and
  \inferrule* [lab=name] {} {{x} \bc \quotep{P}}
\end{mathpar} 

Note that $\vec{x}$ (resp. $\vec{P}$) denotes a vector of names
(resp. processes) of length $|\vec{x}|$ (resp. $|\vec{P}|$). We adopt
the following useful abbreviations.

\begin{mathpar}
   x?(\vec{y}).P := x.(\vec{y})P \and  x\clift{\vec{P}} := x.\clift{\vec{P}}
   \and x!(y) := \lift{x}{\dropn{y}}
   \and \Pi_{i=0}^{n-1}P_i := P_0 | \ldots | P_{n-1}
\end{mathpar}

\subsubsection{Structural congruence}

\paragraph{Free and bound names and alpha-equivalence.} At the
core of structural equivalence is alpha-equivalence which identifies
process that are the same up to a change of variable. Formally, we
recognize the distinction between free and bound names. The free names
of a process, $\freenames{P}$, may be calculated recursively as
follows:

\begin{mathpar}
\freenames{\pzero} := \emptyset
  \and \\
  \freenames{x?(y).P} := \{ x \} \cup (\freenames{P} \setminus \{ y \})
  \and 
  \freenames{x!\langle P \rangle} := \{ x \} \cup \{ P \} 
  \and \\
  \freenames{P|Q} := \freenames{P} \cup \freenames{Q}
  \and \\
  \freenames{@{x}} := \{ x \}
\end{mathpar}

$\pi$
$\quotep{\pi}$

$\freenames{-} : \pi \to \mathcal{P}(\quotep{\pi})$

\begin{eqnarray*}
  \freenames{\pzero} & := & \emptyset \\
  \freenames{x?(y).P} & := & \{ x \} \cup (\freenames{P} \setminus \{ y \}) \\
  \freenames{x!\langle P \rangle} & := & \{ x \} \cup \{ P \} \\
  \freenames{P|Q} & := & \freenames{P} \cup \freenames{Q} \\
  \freenames{\dropn{x}} & := & \{ x \}
\end{eqnarray*}

The bound names of a process, $\boundnames{P}$, are those names occurring in $P$
that are not free. For example, in $x?(y).0$, the name $x$ is free, while $y$ is bound.

\begin{mathpar}
  \inferrule* [lab=monoidal-laws] {} { P|Q \equiv Q|P \and P|0 \equiv P \and P|(Q|R) \equiv (P|Q)|R }
\end{mathpar}

\begin{mathpar}
  \inferrule* [lab=alpha-equivalence] {} { (x)P \equiv (y)P\{y/x\} \and y \not\in \freenames{P} }
\end{mathpar}

\begin{definition}
Then two processes, $P,Q$, are alpha-equivalent if $P = Q\{\vec{y}/\vec{x}\}$ for
some $\vec{x} \in \boundnames{Q},\vec{y} \in \boundnames{P}$, where $Q\{\vec{y}/\vec{x}\}$
denotes the capture-avoiding substitution of $\vec{y}$ for $\vec{x}$ in $Q$.
\end{definition}

\begin{definition}
  The {\em structural congruence} \cite{SangiorgiWalker} , $\equiv$,
  between processes is the least congruence containing
  alpha-equivalence, satisfying the abelian monoid laws
  (associativity, commutativity and $\pzero$ as identity) for parallel
  composition $|$ and for summation $+$.
\end{definition}

\subsection{Name equivalence}

We take name equivalence, written $\nameeq$, to be the smallest
equivalence relation generated by the following rules.

\begin{mathpar}
\inferrule*[lab=Quote-drop]
{ }
{ \quotep{@{x}} \nameeq x }

\inferrule*[lab=Struct-equiv]
{ P \scong Q }
{ \quotep{P} \nameeq \quotep{Q} }
\end{mathpar}

The astute reader will have noticed that the mutual recursion of names
and processes imposes a mutual recursion on alpha-equivalence and
structural equivalence via name-equivalence. Fortunately, all of this
works out pleasantly and we may calculate in the natural way, free of
concern. The reader interested in the details is referred to the
appendix \ref{appendix:rho_details}.

\subsection{Substitution}

We use $\Proc$ for the set of processes, $\QProc$ for the set of
names, and $\id{\{}\vec{y} / \vec{x} \id{\}}$ to denote partial maps,
$s : \QProc \rightarrow \QProc$. A map, $s$ lifts, uniquely, to a map
on process terms, $\widehat{s} : \Proc \rightarrow \Proc$ by the
following equations.

\begin{mathpar}
  (0) \psubstp{Q}{P} := 0 \\
  (R \juxtap S) \psubstp{Q}{P}
  :=    
  (R)\psubstp{Q}{P} \juxtap (S) \psubstp{Q}{P} \\
  (x?(y).R) \psubstp{Q}{P}    
  :=    
  (x)\substp{Q}{P} (z)\concat( (R \psubstn{z}{y}) \psubstp{Q}{P} ) \\
  (\lift{x}{R}) \psubstp{Q}{P}  
  :=
  \lift{(x)\substp{Q}{P}}{ R \psubstp{Q}{P} } \\
%   (\dropn{x})  \psubstp{Q}{P}       
%   := 
%   \left\{ 
%     \begin{array}{ccc} 
%       \dropn{\quotep{Q}} & & x \nameeq \quotep{P} \\
%       \dropn{x} & & otherwise \\
%     \end{array}
%   \right. 
  (\dropn{x})  \psubstp{Q}{P}       
  := 
  \left\{ 
    \begin{array}{ccc} 
      Q & & x \nameeq \quotep{P} \\
      \dropn{x} & & otherwise \\
    \end{array}
  \right.
\end{mathpar}
 

where

\begin{eqnarray}
  (x)\id{\{} \lpquote Q \rpquote / \lpquote P \rpquote \id{\}}            = 
  \left\{ 
    \begin{array}{ccc}
      \lpquote Q \rpquote & & x \nameeq \lpquote P \rpquote \\
      x & & otherwise \\
    \end{array}
  \right. \nonumber
\end{eqnarray}

and $z$ is chosen distinct from $\quotep{P}$, $\quotep{Q}$, the free
names in $Q$, and all the names in $R$. Our $\alpha$-equivalence will
be built in the standard way from this substitution.

\begin{remark}\label{rem:no_self_referential_names}
  One consequence of these definitions is that $\forall P. \quotep{P}
  \not\in \freenames{P}$.
\end{remark}

\subsection{ Dynamic quote: an example }

Anticipating something of what's to come, consider applying the
substitution, $\widehat{\id{\{}u / z \id{\}}}$, to the following pair
of processes, $\lift{w}{y!(z)}$ and $w[ \lpquote y!(z) \rpquote ]$.

\begin{eqnarray}
	\lift{w}{y!(z)}\widehat{\id{\{}u / z \id{\}}}
		& = &
		\lift{w}{y!(u)} \nonumber\\
	w[ \lpquote y!(z) \rpquote ] \widehat{ \id{\{}u / z \id{\}} }
		& = &
		w[ \lpquote y!(z) \rpquote ] \nonumber
\end{eqnarray}

Because the body of the process between quotes is impervious to
substitution, we get radically different answers. In fact, by
examining the first process in an input context,
e.g. $x?(z).\lift{w}{y!(z)}$, we see that the process under the lift
operator may be shaped by prefixed inputs binding a name inside it. In
this sense, the lift operator will be seen as a way to dynamically
construct processes before reifying them as names.

Finally equipped with these standard features we can present the
dynamics of the calculus.

\subsubsection{Operational semantics} 

Finally, we introduce the computational dynamics. What marks these
algebras as distinct from other more traditionally studied algebraic
structures, e.g. vector spaces or polynomial rings, is the manner in
which dynamics is captured. In traditional structures, dynamics is typically
expressed through morphisms between such structures, as in linear maps
between vector spaces or morphisms between rings. In algebras
associated with the semantics of computation, the dynamics is
expressed as part of the algebraic structure itself, through a
reduction reduction relation typically denoted by $\red$. Below, we
give a recursive presentation of this relation for the calculus used
in the encoding.

$\red \subseteq \pi \times \pi$
$\red : \pi \to \mathcal{P}(\pi)$

\begin{mathpar}
  \inferrule* [lab=Comm] { \textsf{match}( x_{src}, x_{trgt} ) } { x_{trgt}?(y)P \; | \; x_{src}!\langle {Q} \rangle \red P\{\quotep{Q}/y}\} }
  \and \\
  \inferrule* [lab=Par] {{P} \red {P}'} {{{P} | {Q}} \red {{P}' | {Q}}}
  \and
  \inferrule* [lab=Equiv]{{{P} \scong {P}'} \andalso {{P}' \red {Q}'} \andalso {{Q}' \scong {Q}}}{{P} \red {Q}}
\end{mathpar}

\begin{eqnarray*}
  match_{\equiv} (\quotep{P},\quotep{Q}) & := & P \equiv Q \\
  match_{\dagger}(\quotep{P},\quotep{Q}) & := & \forall R. P|Q \red^{*} R => R \red^{*} 0 \\
  match_{K}(\quotep{P},\quotep{Q}) & := & K \mbox{ for some context } K
\end{eqnarray*}

$u?(x)P | u!\langle Q \rangle \red P\{\quotep{Q}/x\}$

%We write $\wred$ for $\red^*$, and $P\red$ if $\exists Q $ such that $ P \red Q$.
We write $P\red$ if $\exists Q $ such that $ P \red Q$ and $P\not\red$, otherwise.

\section{Replication}

As mentioned before, it is known that replication (and hence
recursion) can be implemented in a higher-order process algebra
\cite{SangiorgiWalker}. As our first example of calculation with the
machinery thus far presented we give the construction explicitly in
the {\rhoc}.

\begin{eqnarray}
	D_{x} & := & \prefix{x}{y}{(\binpar{\outputp{x}{y}}{@{y}})} \nonumber\\
	\bangp_{x}{P} & := & \binpar{{x}!\langle{\binpar{D_{x}}{P}}\rangle}{D_{x}} \nonumber
\end{eqnarray}

\begin{eqnarray}
	\bangp_{x}{P} & & \nonumber\\
	=
	& {x}!\langle{(\prefix{x}{y}{(\outputp{x}{y} | @{y})) | P}}\rangle 
	      | \prefix{x}{y}{(\outputp{x}{y} | @{y})} & \nonumber\\
	\red
	& (\outputp{x}{y} | @{y})\substn{\quotep{(\prefix{x}{y}{(@{y} | \outputp{x}{y})) | P}}}{y} & \nonumber\\
	=
	& \outputp{x}{\quotep{(\prefix{x}{y}{(\outputp{x}{y} | @{y})) | P}}}
	  | {(\prefix{x}{y}{(\outputp{x}{y} | @{y})) | P}} & \nonumber\\
	\red
	& \ldots & \nonumber\\
	\red^*
	& P | P | \ldots & \nonumber
\end{eqnarray}

Of course, this encoding, as an implementation, runs away, unfolding
$\bangp{P}$ eagerly. A lazier and more implementable replication
operator, restricted to input-guarded processes, may be obtained as follows.

\begin{eqnarray}
\bangp{\prefix{u}{v}{P}} 
	:= 
	\binpar{\lift{x}{\prefix{u}{v}{(\binpar{D(x)}{P})}}}{D(x)} \nonumber
\end{eqnarray}

\begin{remark}
  Note that the lazier definition still does not deal with summation
  or mixed summation (i.e. sums over input and output). The reader is
  invited to construct definitions of replication that deal with these
  features. 

  Further, the definitions are parameterized in a name, $x$. Can you,
  gentle reader, make a definition that eliminates this parameter and
  guarantees no accidental interaction between the replication
  machinery and the process being replicated -- i.e. no accidental
  sharing of names used by the process to get its work done and the
  name(s) used by the replication to effect copying. This latter
  revision of the definition of replication is crucial to obtaining
  the expected identity $!!P \sim !P$.
\end{remark}

\begin{remark}\label{rem:paradoxical_combinator}
  The reader familiar with the lambda calculus will have noticed the
  similarity between $D$ and the paradoxical combinator.

  [Ed. note: the existence of this seems to suggest we have to be more
  restrictive on the set of processes and names we admit if we are to
  support no-cloning.]
\end{remark}

\subsubsection{Bisimulation}

The computational dynamics gives rise to another kind of equivalence,
the equivalence of computational behavior. As previously mentioned
this is typically captured \emph{via} some form of bisimulation.

% The notion we use in this paper is weak barbed bisimulation
% \cite{milner91polyadicpi}.

The notion we use in this paper is derived from weak barbed
bisimulation \cite{milner91polyadicpi}. 

\begin{definition}
An \emph{observation relation}, $\downarrow_{\mathcal N}$, over a set
of names, $\mathcal N$, is the smallest relation satisfying the rules
below.

\infrule[Out-barb]{y \in {\mathcal N}, \; x \nameeq y}
		  {\outputp{x}{v} \downarrow_{\mathcal N} x}
\infrule[Par-barb]{\mbox{$P\downarrow_{\mathcal N} x$ or $Q\downarrow_{\mathcal N} x$}}
		  {\binpar{P}{Q} \downarrow_{\mathcal N} x}

We write $P \Downarrow_{\mathcal N} x$ if there is $Q$ such that 
$P \wred Q$ and $Q \downarrow_{\mathcal N} x$.
\end{definition}

\begin{definition}
%\label{def.bbisim}
An  ${\mathcal N}$-\emph{barbed bisimulation} over a set of names, ${\mathcal N}$, is a symmetric binary relation 
${\mathcal S}_{\mathcal N}$ between agents such that $P\rel{S}_{\mathcal N}Q$ implies:
\begin{enumerate}
\item If $P \red P'$ then $Q \wred Q'$ and $P'\rel{S}_{\mathcal N} Q'$.
\item If $P\downarrow_{\mathcal N} x$, then $Q\Downarrow_{\mathcal N} x$.
\end{enumerate}
$P$ is ${\mathcal N}$-barbed bisimilar to $Q$, written
$P \wbbisim_{\mathcal N} Q$, if $P \rel{S}_{\mathcal N} Q$ for some ${\mathcal N}$-barbed bisimulation ${\mathcal S}_{\mathcal N}$.
\end{definition}

$\mathcal{R} \subseteq \pi \times \pi$

$P \mathcal{R} Q => \forall P'. P \red P' \Rightarrow \exists Q'. Q \red Q', P' \mathcal{R} Q'$

$P \vdash x \Rightarrow Q \vdash x$

\begin{mathpar}
  \inferrule*[lab=Out-barb]{x \nameeq y}{{y}!\langle{Q}\rangle \vdash x}
  \and
  \inferrule*[lab=Par-barb]{\mbox{$P\vdash x$ or $Q\vdash x$}}{\binpar{P}{Q} \vdash x}
\end{mathpar}

\subsubsection{Contexts}

One of the principle advantages of computational calculi like the
$\pi$-calculus is a well-defined notion of context,
contextual-equivalence and a correlation between
contextual-equivalence and notions of bisimulation. The notion of
context allows the decomposition of a process into (sub-)process and
its syntactic environment, its context. Thus, a context may be
thought of as a process with a ``hole'' (written $\Box$) in it. The
application of a context $M$ to a process $P$, written $M[P]$, is
tantamount to filling the hole in $M$ with $P$. In this paper we do
not need the full weight of this theory, but do make use of the notion
of context in the proof the main theorem. 

\begin{mathpar}
  \inferrule* [lab=summation] {} {{M_{M},M_{N}} \bc \Box \;|\; x.M_{A} \;|\; M_{M}+M_{N}}
  \and
  \inferrule* [lab=agent] {} {{M_{A}} \bc (\vec{x})M_{P} \;| \; \clift{P_0,\ldots,M_{P},\ldots,P_N}}
  \and \\
  \inferrule* [lab=process] {} {{M_{P}} \bc M_{N} \;| \;P|M_{P} }
\end{mathpar} 

\begin{mathpar}
  \inferrule* [lab=sychronization] {} {M_{N} \bc \Box \;|\; x?M_{F} \;|\; x!M_{C}}
  \and
  \inferrule* [lab=abstraction] {} {{M_{F}} \bc (x)M_{P} }
  \and
  \inferrule* [lab=concretion] {} {{M_{C}} \bc \langle M_{P} \rangle }
  \and \\
  \inferrule* [lab=process] {} {{M_{P}} \bc M_{N} \;| \;P|M_{P} }
\end{mathpar}

\begin{definition}[contextual application] Given a context $M$, and
  process $P$, we define the \emph{contextual application}, $M[P] :=
  M\{P/\Box\}$. That is, the contextual application of M to P is the
  substitution of $P$ for $\Box$ in $M$.
\end{definition}

$\meaningof{-} : L \to \mathcal{P}(\pi)$

\begin{mathpar}
  \inferrule* [lab=collection] {} {\meaningof{true} = \pi, \and \meaningof{~E} = \pi \setminus \meaningof{E}, \and \meaningof{E_{1} \& E_{2}} = \meaningof{E_{1}} \cap \meaningof{E_{2}}}
\end{mathpar}

\begin{mathpar}
  \inferrule* [lab=structure] {} {\meaningof{0} = \{ P \in \pi | P \equiv 0 \}, \and \\ \meaningof{E_1 | E_2} = \{ P \in \pi | P \equiv P_{1} | P_{2}, P_{1} \in \meaningof{E_{1}}, P_{2} \in \meaningof{E_2}\} }
\end{mathpar}

\begin{mathpar}
 \inferrule* [lab=behavior] {} {\meaningof{\langle a?b \rangle E} = \{ P \in \pi | P \equiv Q | u?(y)P', \\ \and \\\\ \and \\ \;\;\; u \in \meaningof{a}, \forall z.P'\{z/y\} \in \meaningof{E\{z/b\}}\}, \and \\ \meaningof{a!E} = \{ P \in \pi | P \equiv Q | x!\langle P' \rangle, x \in \meaningof{a} P' \in \meaningof{E}\} }
\end{mathpar}

\begin{mathpar}
 \inferrule* [lab=nominal] {} {\meaningof{\quotep{E}} = \{ \quotep{P} \in \quotep{\pi} | P \in \meaningof{E} \}, \and \meaningof{\quotep{P}} = \{ \quotep{Q} \in \quotep{\pi} | P \equiv Q \} \and \\ \meaningof{@\quotep{E}} = \{ P \in \pi | P \equiv @x, x \in \meaningof{E} \}}
\end{mathpar}

\begin{eqnarray*}
  \\
  \meaningof{-} : TS \to ST
\end{eqnarray*}

\begin{eqnarray*}
  \\
  L : TS \to ST
\end{eqnarray*}

\begin{eqnarray*}
  \\
  P \models E \iff P \in \meaningof{E}
\end{eqnarray*}

\begin{eqnarray*}
  P \approx_{L} Q \iff \forall E \in L. P \models E \iff Q \models E
\end{eqnarray*}

\begin{eqnarray*}
  P \approx_{K} Q
\end{eqnarray*}

\begin{eqnarray*}
  P \approx Q
\end{eqnarray*}

$\approx_{K} = \approx = \approx_{L}$

\subsubsection{Contextual duality}

Note that contexts extend the quotation operation to a family of
operations from processes to names. Given a context, $M$, we can
define a \emph{nominal context}, $\quotep{M}$ by $\quotep{M}[P] :=
\quotep{M[P]}$. To foreshadow what is to come we observe that these
operations enjoy a duality with processes very much like the duality
between vectors and maps from vectors to scalars.

Further, because the calculus is essentially higher-order, we have a
correspondence between contexts and processes. More specifically,
given a name $x$ and a context $M$ we can construct $M^{*}_{x}$ such
that 

\begin{mathpar}
  M^{*}_{x} | \lift{x}{P} \red M[P]
\end{mathpar}

namely,

\begin{mathpar}
  M^{*}_{x} := x?(u).M[\dropn{u}]
\end{mathpar}

The dependence of $M^{*}_{x}$ on a name makes it an abstraction, 

\begin{mathpar}
  M^{*} := (x)x?(u).M[\dropn{u}]
\end{mathpar}

\subsection{Additional notation}

It will sometimes be convenient to denote the process a name
quotes. We already have the notation $x = \quotep{P}$, but it will be
convenient to introduce an alternate notation, $\procn{x}$, when we
want to emphasize the connection to the use of the name. Note that, by
virtue of name equivalence, $\quotep{\procn{x}} \nameeq x$; so, the
notation is consistent with previous definitions.

Further, because names have structure it is possible to effect
substitutions on the basis of that structure. This means we need to
upgrade our notation for substitutions, which we accomplish by
adapting comprehension notation. Thus,

\begin{mathpar}
  P\{ y / x : x \in S \}
\end{mathpar}

is interpreted to mean the process derived from P by replacing (in a
capture-avoiding manner) each occurrence of $x$ in $S$ by $y$. For example,

\begin{mathpar}
  P\{ \quotep{\procn{x}|\procn{x}} / x : x \in \freenames{P} \}
\end{mathpar}

will replace each (occurrence) of a free name $x$ in $P$ by
$\quotep{\procn{x}|\procn{x}}$.

Also, we will avail ourselves of the notation $x^{L}$ and $x^{R}$ to
denote injections of a name into disjoint copies of the name
space. There are numerous ways to accomplish this. One example can be
found in \cite{MeredithR05}. This notation overloads to vectors of
names: $\vec{x}^{\pi} := (x_{i}^{\pi} \; : \; 0 \leq i < |\vec{x}| )$ where $\pi \in \{L,R\}$.

We also use $P^{\Box} := P|\Box$.

In \cite{MeredithR05} an interpretation of the new operator is
given. It turns out that there are several possible interpretations
all enjoying the requisite algebraic properties of the operator (see
\cite{milner91polyadicpi}). We will therefore make liberal use of
$(\nu\; \vec{x})P$.

% subsection the_syntax_and_semantics_of_the_notation_system (end)   

\section{Interpretation of QM}
\subsection{Supporting definitions}
\subsubsection{Multiplication}
\begin{mathpar}
  \quotep{Q} \cdot \quotep{R} := \quotep{Q|R}
  \and \\
  \quotep{Q} \cdot P := P\{ \quotep{Q|R} / \quotep{R} : \quotep{R} \in \freenames{P} \}
\end{mathpar}

\paragraph{Discussion}
The first line needs little explanation. The second line says that
each free name of the process is replaced with the multiplication of
that name by the scalar. Multiplication of a scalar (name) by a state
(process) results in a process all the names of which have been `moved
over' by parallel composition with the process the scalar
quotes. There is a subtlety that the bound names have to be
manipulated so that multiplied names aren't accidentally
captured. There are many ways to achieve this.

\begin{remark}\label{rem:multiplication_identities}
  The reader is invited to verify that for all $x,y,z \in \QProc$ and $P \in \Proc$
  \begin{mathpar}
    x \cdot \quotep{0} \equiv x 
    \and
    x \cdot y \equiv y \cdot x
    \and
    x \cdot (y \cdot z) \equiv (x \cdot y) \cdot z
    \and \\
    \quotep{0} \cdot P \equiv P
    \and \\
    x \cdot (y \cdot P) \equiv (x \cdot y) \cdot P
    \and \\
    x \cdot (P|Q) \equiv (x \cdot P) | (x \cdot Q)
    \and \\    
  \end{mathpar}
\end{remark}

\subsubsection{Tensor product}

We define a tensor product on processes by structural induction.

\paragraph{Tensor of sums} First note that all summations, including
$\pzero$ and sequence, can be written $\Sigma_{i} x_{i}.A_{i} +
\Sigma_{j} x_{j}.C_{j}$, where we have grouped input-guarded processes
together and output-guarded processes together.

Thus, we can define the tensor product of two summations, $N_{1}\otimes N_{2}$, where

\begin{mathpar}
  N_{1} := \Sigma_{i} x_{i}.A_{i} + \Sigma_{j} x_{j}.C_{j}
  \and
  N_{2} := \Sigma_{i'} y_{i'}.B_{i'} + \Sigma_{j'} y_{j'}.D_{j'} 
\end{mathpar}

as follows.

\begin{mathpar}
  \Sigma_{i} x_{i}.A_{i} + \Sigma_{j} x_{j}.C_{j} \otimes \Sigma_{i'}
  y_{i'}.B_{i'} + \Sigma_{j'} y_{j'}.D_{j'} 
  \and \\
  := \; \Sigma_{i} \Sigma_{i'} \quotep{\stackrel{\vee}{x_{i}}| \stackrel{\vee}{y_{i'}}}.(A_{i}\otimes B_{i'}) \; | \; \Sigma_{i'} \Sigma_{i} \quotep{\stackrel{\vee}{y_{i'}}|\stackrel{\vee}{x_{i}}}.(B_{i'}\otimes A_{i})
  \and
  \;\; | \;\; \Sigma_{j} \Sigma_{j'} \quotep{\stackrel{\vee}{x_{j}}|\stackrel{\vee}{y_{j'}}}.(A_{j}\otimes B_{j'}) \; | \; \Sigma_{j'} \Sigma_{j} \quotep{\stackrel{\vee}{y_{j'}}|\stackrel{\vee}{x_{j}}}.(B_{j'}\otimes A_{j})
\end{mathpar}

\begin{remark}
  Do we need to $x^{L}$ and $y^{R}$ for this construction as well?
\end{remark}

\paragraph{Tensor of parallel compositions} Next, we distribute tensor
over par.

\begin{mathpar}
  P_{1}|P_{2} \otimes Q_{1}|Q_{2} := (P_{1} \otimes Q_{1}) | (P_{1}
  \otimes Q_{2}) | (P_{2} \otimes Q_{1}) | (P_{2} \otimes Q_{2})
\end{mathpar}

\paragraph{Tensor with dropped names} We treat tensor of a
process with a dropped name as parallel composition.

\begin{mathpar}
  P \otimes \dropn{x} := P | \dropn{x}
\end{mathpar}

\paragraph{Tensor of agents}

Finally, we need to define tensor on agents. Note that the definition
of tensor on normal products only tensors inputs with inputs and
outputs with outputs. Thus, we only have to define the operation on
``homogeneous'' pairings.

\begin{mathpar}
  (\vec{x})P \otimes (\vec{y})Q
  \and \\
  := (x_{0}^{L}|y_{0}^{R},\ldots,x_{0}^{L}|y_{n}^{R},\ldots,x_{m}^{L}|y_{0}^{R},\ldots,x_{m}^{L}|y_{n}^R)(P\{ \vec{x}^{L}/\vec{x}\} \otimes Q \{ \vec{y}^{R}/\vec{y}\})
  \and \\
  \clift{\vec{P}} \otimes \clift{\vec{Q}}
  \and \\
  := \clift{P_{0}\otimes Q_{0},\ldots,P_{0}\otimes Q_{n},\ldots,P_{m}\otimes Q_{0},\ldots,P_{m}\otimes Q_{n}}
\end{mathpar}

\begin{remark}
  Observe that arities of tensored abstractions matches arities of
  tensored concretions if the original arities matched. Note also that
  the length of the arities corresponds to the increase in dimension
  we see in ordinary vector space tensor product.
\end{remark}

\begin{remark}
  Operationally, this definition distributes the tensor down to
  components ``linked'' by summation. Tensor over summation is
  intriguing in that it mixes names. Moreover, as a consequence of the
  way it mixes names we have the identities for all $x \in \QProc$ and
  $P,Q \in \Proc$

  \begin{mathpar}
    (x \cdot P) \otimes Q \equiv x \cdot (P \otimes Q) \equiv P \otimes (x \cdot Q)
    \and
    P \otimes \pzero \equiv P
  \end{mathpar}

  that the reader is invited to verify.
\end{remark}

\subsubsection{Annihilation}
\begin{mathpar}
  P^{\perp} := \{ Q | \forall R. P|Q \red^{*} R \Rightarrow R \red^{*} \pzero \}
  \and \\
  P^{\underline{\perp}} := \Sigma_{Q \in P^{\perp}} \quotep{Q}?(y).(\dropn{y}|Q) | \Sigma_{Q \in P^{\perp}} \quotep{Q}\clift{\Box}
\end{mathpar}

\paragraph{Discussion} The reader will note that $P^{\perp}$ is a
\emph{set} of processes, while $P^{\underline{\perp}}$ is a
\emph{context}. We call the set $P^{\perp}$ the \emph{annihilators} of
$P$. The parallel composition of a process in the annihilators of $P$
with $P$ will result in a process, the state space of which has all
paths eventually leading to $\pzero$. Execution may endure loops; but
under reasonable conditions of fairness (naturally guaranteed under
most notions of bisimulation) such a composite process cannot get
stuck in such a loop and will, eventually pop out and terminate.

The context $P^{\underline{\perp}}$ is ready and willing to ``take the
$P$ out of'' the process to which it is applied. It will effectively
transmit the code of the process to which it is applied to one of the
annihilators and run the process against it.

\subsubsection{Evaluation}
We fix $M$ a domain of fully abstract interpretation with an equality
coincident with bisimulation. We take $\meaningof{\cdot} : \Proc \to
M$ to be the map interpreting processes and $\nmeaningof{\cdot} : \M
\to Proc$ to be the map running the other way. Then we define

\begin{mathpar}
  \int P := \nmeaningof{\meaningof{P}}
\end{mathpar}

\paragraph{Discussion}
There are many fully abstract interpretations of Milner's
$\pi$-calculus. Any of them can be used as a basis for interpreting
the reflective calculus here. Equipped with such a domain it is
largely a matter of grinding through to check that the Yoneda
construction for the normalization-by-evaluation program can be
extended to this setting.

\begin{remark}
  The reader is invited to verify that $\int (P^{\underline{\perp}}[P]) = 0$.
\end{remark}

\subsection{Quantum mechanics}

Table \ref{tbl:core_qm_op_defns} gives the core operational definitions

\begin{table}[htp]\label{tbl:core_qm_op_defns}
  \center{
    \fbox{
      \begin{tabular}{c|c}
        quantum mechanics & process calculus \\
        \hline
        scalar & $x := \quotep{P}$ \\
        state vector & $\state{P} := P$ \\
        dual & $\state{P}^{*} := \event{P^{\underline{\perp}}} := \quotep{P^{\underline{\perp}}}[-]$ \\
        matrix & $ \Sigma_{\alpha} \state{P_{\alpha}}x_{\alpha}\event{Q_{\alpha}}$ \\
        vector addition & $\state{P} + \state{Q} := \state{P | Q}$ \\
        tensor product & $\state{P} \otimes \state{Q} := \state{P \otimes Q}$ \\
        inner product & $\innerprod{P}{Q} := \quotep{\int P^{\underline{\perp}}[Q]}$ \\
      \end{tabular}
    }
  }
  \caption{QM - operational definitions}
\end{table}

where

\begin{mathpar}
  \prmatrix{P}{Q} := \fprmatrix{P}{\quotep{\pzero}}{Q}
  \and
  \fprmatrix{P}{x}{Q} := (\state{P},x,\event{Q})
  \and
  (\fprmatrix{P}{x}{Q})(\state{R}) := x \cdot \innerprod{Q}{R} \cdot \state{P}
  \and
  (\fprmatrix{P}{x}{Q})(\event{R}) := x \cdot \innerprod{R}{P} \cdot \event{Q}
\end{mathpar}

\paragraph{Discussion}
As promised: vectors (aka states) are represented as processes; duals
as contextual duals; inner product definition should be compared with
standard inner product definition for ....

\begin{remark}
  Assuming $\int (P^{\underline{\perp}}[P]) = 0$, the reader is
  invited to verify that $(\fprmatrix{P}{x}{P})(\state{P}) = x \cdot \state{P}$.
\end{remark}

\begin{remark}
  The reader is invited to verify that $\innerprod{P}{Q}$ could
  equally well have been written $\quotep{\int \stackrel{\vee}{x}}$
  where $x = \event{P^{\underline{\perp}}}(Q)$.

  One of the motivations for this remark is that there is another way
  to factor these operations. We could package up evaluation in the dual:

  \begin{mathpar}
    \state{P}^{*} := \event{\int P^{\underline{\perp}}} := \quotep{\int P^{\underline{\perp}}}[-]
  \end{mathpar}

  and then have inner product defined by
  
  \begin{mathpar}
    \innerprod{P}{Q} := \event{P}(Q)
  \end{mathpar}

  Hopefully, experience with the calculations will provide guidance on
  the best factoring.
\end{remark}

\begin{remark}
  Assuming $\int (P^{\underline{\perp}}[P]) = 0$, the reader is
  invited to verify that $\forall P,Q. (\prmatrix{0}{Q})(\state{0}) =
  \state{0}$ and dually $(\prmatrix{P}{0})(\event{0}) = \event{0}$.
\end{remark}

\begin{remark}
  i'm a little worried that i don't (yet) have proper support for
  complex conjugacy. But, the observation above may give us a
  clue. According to Abramsky, it must be the case that the scalars
  are iso to the homset of the identity for the tensor -- which the
  observation above characterizes. 

  For now, we will simply bookmark the notion with $\overline{x}$.
\end{remark}

\subsubsection{Adjointness}

We need to give a definition of $(\cdot)^{\dagger}$ for matrices. The
obvious candidate definition is
\begin{mathpar}
(\Sigma_{\alpha}\fprmatrix{P_{\alpha}}{x_{\alpha}}{Q_{\alpha}})^{\dagger}
= \Sigma_{\alpha}\fprmatrix{(Q_{\alpha}^{\underline{\perp}})^{*}}{\overline{x}_{\alpha}}{P_{\alpha}^{\underline{\perp}}} 
\end{mathpar}

But, $(Q_{\alpha}^{\underline{\perp}})^{*}$ requires a name along
which to communicate the process to achieve the context application.

\subsubsection{Basis for a basis}
If processes label states and ``addition'' of states (a.k.a. vector
addition) is interpreted as parallel composition, what corresponds to
notions of linear independence and basis? Here, we recall that Yoshida
has developed a set of \emph{combinators} for an asynchronous verison
of Milner's $\pi$-calculus. These are a finite set of processes such
any process can be expressed as parallel composition of these
combinators together with liberal uses of the new operator and
replication. We can simply give a translation of these into the
present calculus and have reasonable expectation that the property
carries over. That is, that the resultant set allows to express all
processes via parallel composition. Note, however, that there is no
new operator or replication in this calculus. As a result, we expect
that the corresponding set is actually infinite. That is, we expect
that the space is actually infinite dimensional.

\begin{remark}
  The attentive reader may be a bit concerned. Certainly, the
  collection $S$, $K$ and $I$ is a finite set of
  combinators. Shouldn't we expect to see a finite set of combinators
  for an effectively equivalent system? i am very sympathetic to this
  critique and feel it warrants full attention. On the other hand, i
  also have in mind the following analogy. The natural numbers, as a
  monoid under addition, has exactly $1$ generator, while the natural
  numbers, as a monoid under multiplication, has countably many
  generators (the primes). We observe that the application of the
  lambda calculus is much less resource sensitive than the parallel
  composition of the $\pi$-calculus. Could it be the case that we have
  an analogy of the form
  
  \begin{mathpar}
    m + n : MN :: m*n : M|N
  \end{mathpar}

  giving a similar blow up in the set of ``primes''?  This is such a
  wonderful thought that, even if it's not true, i think it's worth
  writing down.
\end{remark}
 

\documentclass[12pt]{llncs}
%\documentclass{jktr}

\usepackage[pdftex]{hyperref}                   
\usepackage {listings}
\usepackage {mathpartir}
\usepackage{bcprules}
%\usepackage{listings}
                       
\usepackage{graphicx} 
%\usepackage[margins=2.5cm,nohead,nofoot]{geometry}
%\usepackage{geometry}
\usepackage{amsfonts}
\usepackage{amstext}
\usepackage{latexsym}
\usepackage{amssymb}
\usepackage{color}


%\include{myPreamble}
\documentclass[12pt]{llncs}
%\documentclass{jktr}

\usepackage[pdftex]{hyperref}                   
\usepackage {listings}
\usepackage {mathpartir}
\usepackage{bcprules}
%\usepackage{listings}
                       
\usepackage{graphicx} 
%\usepackage[margins=2.5cm,nohead,nofoot]{geometry}
%\usepackage{geometry}
\usepackage{amsfonts}
\usepackage{amstext}
\usepackage{latexsym}
\usepackage{amssymb}
\usepackage{color}


%\include{myPreamble}
\include{qm2pi.local} 

%\ifpdf
%\usepackage[pdftex]{graphicx}
%\else
%\usepackage{graphicx}
%\fi

 % \ifpdf
%  \usepackage{pdfsync}
%  \if


%\title{Brief Article}
%\author{David F. Snyder}
%\author{L.G. Meredith}

%\address{Dept. of Math., Texas State University--San Marcos, San Marcos, TX 78666}
       
\pagestyle{empty}


\begin{document}

\lstset{language=[Objective]Caml,frame=shadowbox}

\input{qm2pi.front}

% section front matter (end)

\input{qm2pi.intro} 
 
% section introduction (end)

% \input{qm2pi.knotations} 

% section notation (end)

\input{qm2pi.process.calculi} 

% section concurrent_process_calculi_and_spatial_logics_ (end)
    
%\input{qm2pi.knots2pi} 

%\input{qm2pi.trefoil} 

%\input{qm2pi.mainthm} 

% subsection basic_interpretation (end)

%\input{qm2pi.rho.presentation} 
\subsection{The syntax and semantics of the notation system}\label{sub:the_syntax_and_semantics_of_the_notation_system} % (fold)

We now summarize a technical presentation of the calculus that
embodies our theory of dynamics. The typical presentation of such a
calculus follows the style of giving generators and relations on
them. The grammar, below, describing term constructors, freely
generates the set of processes, $\Proc$. This set is then quotiented
by a relation known as structural congruence and it is over this set
that the notion of dynamics is expressed. This presentation is
essentially that of \cite{MeredithR05} with the addition of
polyadicity and summation. For readability we have relegated some of
the technical subtleties to an appendix.

\subsubsection{Process grammar}\label{subsub:process_grammar}

\begin{mathpar}
  \inferrule* [lab=synchronization] {} {{M} \bc \pzero \;|\; x?F \;|\; x!C }
  \and
  \inferrule* [lab=abstraction] {} {{F} \bc (x)P}
  \and
  \inferrule* [lab=concretion] {} {{C} \bc \langle Q \rangle}
  \and
  \inferrule* [lab=process] {} {{P,Q} \bc M \;| \;P|Q \;|\; @{x}}
  \and
  \inferrule* [lab=name] {} {{x} \bc \quotep{P}}
\end{mathpar} 

Note that $\vec{x}$ (resp. $\vec{P}$) denotes a vector of names
(resp. processes) of length $|\vec{x}|$ (resp. $|\vec{P}|$). We adopt
the following useful abbreviations.

\begin{mathpar}
   x?(\vec{y}).P := x.(\vec{y})P \and  x\clift{\vec{P}} := x.\clift{\vec{P}}
   \and x!(y) := \lift{x}{\dropn{y}}
   \and \Pi_{i=0}^{n-1}P_i := P_0 | \ldots | P_{n-1}
\end{mathpar}

\subsubsection{Structural congruence}

\paragraph{Free and bound names and alpha-equivalence.} At the
core of structural equivalence is alpha-equivalence which identifies
process that are the same up to a change of variable. Formally, we
recognize the distinction between free and bound names. The free names
of a process, $\freenames{P}$, may be calculated recursively as
follows:

\begin{mathpar}
\freenames{\pzero} := \emptyset
  \and \\
  \freenames{x?(y).P} := \{ x \} \cup (\freenames{P} \setminus \{ y \})
  \and 
  \freenames{x!\langle P \rangle} := \{ x \} \cup \{ P \} 
  \and \\
  \freenames{P|Q} := \freenames{P} \cup \freenames{Q}
  \and \\
  \freenames{@{x}} := \{ x \}
\end{mathpar}

$\pi$
$\quotep{\pi}$

$\freenames{-} : \pi \to \mathcal{P}(\quotep{\pi})$

\begin{eqnarray*}
  \freenames{\pzero} & := & \emptyset \\
  \freenames{x?(y).P} & := & \{ x \} \cup (\freenames{P} \setminus \{ y \}) \\
  \freenames{x!\langle P \rangle} & := & \{ x \} \cup \{ P \} \\
  \freenames{P|Q} & := & \freenames{P} \cup \freenames{Q} \\
  \freenames{\dropn{x}} & := & \{ x \}
\end{eqnarray*}

The bound names of a process, $\boundnames{P}$, are those names occurring in $P$
that are not free. For example, in $x?(y).0$, the name $x$ is free, while $y$ is bound.

\begin{mathpar}
  \inferrule* [lab=monoidal-laws] {} { P|Q \equiv Q|P \and P|0 \equiv P \and P|(Q|R) \equiv (P|Q)|R }
\end{mathpar}

\begin{mathpar}
  \inferrule* [lab=alpha-equivalence] {} { (x)P \equiv (y)P\{y/x\} \and y \not\in \freenames{P} }
\end{mathpar}

\begin{definition}
Then two processes, $P,Q$, are alpha-equivalent if $P = Q\{\vec{y}/\vec{x}\}$ for
some $\vec{x} \in \boundnames{Q},\vec{y} \in \boundnames{P}$, where $Q\{\vec{y}/\vec{x}\}$
denotes the capture-avoiding substitution of $\vec{y}$ for $\vec{x}$ in $Q$.
\end{definition}

\begin{definition}
  The {\em structural congruence} \cite{SangiorgiWalker} , $\equiv$,
  between processes is the least congruence containing
  alpha-equivalence, satisfying the abelian monoid laws
  (associativity, commutativity and $\pzero$ as identity) for parallel
  composition $|$ and for summation $+$.
\end{definition}

\subsection{Name equivalence}

We take name equivalence, written $\nameeq$, to be the smallest
equivalence relation generated by the following rules.

\begin{mathpar}
\inferrule*[lab=Quote-drop]
{ }
{ \quotep{@{x}} \nameeq x }

\inferrule*[lab=Struct-equiv]
{ P \scong Q }
{ \quotep{P} \nameeq \quotep{Q} }
\end{mathpar}

The astute reader will have noticed that the mutual recursion of names
and processes imposes a mutual recursion on alpha-equivalence and
structural equivalence via name-equivalence. Fortunately, all of this
works out pleasantly and we may calculate in the natural way, free of
concern. The reader interested in the details is referred to the
appendix \ref{appendix:rho_details}.

\subsection{Substitution}

We use $\Proc$ for the set of processes, $\QProc$ for the set of
names, and $\id{\{}\vec{y} / \vec{x} \id{\}}$ to denote partial maps,
$s : \QProc \rightarrow \QProc$. A map, $s$ lifts, uniquely, to a map
on process terms, $\widehat{s} : \Proc \rightarrow \Proc$ by the
following equations.

\begin{mathpar}
  (0) \psubstp{Q}{P} := 0 \\
  (R \juxtap S) \psubstp{Q}{P}
  :=    
  (R)\psubstp{Q}{P} \juxtap (S) \psubstp{Q}{P} \\
  (x?(y).R) \psubstp{Q}{P}    
  :=    
  (x)\substp{Q}{P} (z)\concat( (R \psubstn{z}{y}) \psubstp{Q}{P} ) \\
  (\lift{x}{R}) \psubstp{Q}{P}  
  :=
  \lift{(x)\substp{Q}{P}}{ R \psubstp{Q}{P} } \\
%   (\dropn{x})  \psubstp{Q}{P}       
%   := 
%   \left\{ 
%     \begin{array}{ccc} 
%       \dropn{\quotep{Q}} & & x \nameeq \quotep{P} \\
%       \dropn{x} & & otherwise \\
%     \end{array}
%   \right. 
  (\dropn{x})  \psubstp{Q}{P}       
  := 
  \left\{ 
    \begin{array}{ccc} 
      Q & & x \nameeq \quotep{P} \\
      \dropn{x} & & otherwise \\
    \end{array}
  \right.
\end{mathpar}
 

where

\begin{eqnarray}
  (x)\id{\{} \lpquote Q \rpquote / \lpquote P \rpquote \id{\}}            = 
  \left\{ 
    \begin{array}{ccc}
      \lpquote Q \rpquote & & x \nameeq \lpquote P \rpquote \\
      x & & otherwise \\
    \end{array}
  \right. \nonumber
\end{eqnarray}

and $z$ is chosen distinct from $\quotep{P}$, $\quotep{Q}$, the free
names in $Q$, and all the names in $R$. Our $\alpha$-equivalence will
be built in the standard way from this substitution.

\begin{remark}\label{rem:no_self_referential_names}
  One consequence of these definitions is that $\forall P. \quotep{P}
  \not\in \freenames{P}$.
\end{remark}

\subsection{ Dynamic quote: an example }

Anticipating something of what's to come, consider applying the
substitution, $\widehat{\id{\{}u / z \id{\}}}$, to the following pair
of processes, $\lift{w}{y!(z)}$ and $w[ \lpquote y!(z) \rpquote ]$.

\begin{eqnarray}
	\lift{w}{y!(z)}\widehat{\id{\{}u / z \id{\}}}
		& = &
		\lift{w}{y!(u)} \nonumber\\
	w[ \lpquote y!(z) \rpquote ] \widehat{ \id{\{}u / z \id{\}} }
		& = &
		w[ \lpquote y!(z) \rpquote ] \nonumber
\end{eqnarray}

Because the body of the process between quotes is impervious to
substitution, we get radically different answers. In fact, by
examining the first process in an input context,
e.g. $x?(z).\lift{w}{y!(z)}$, we see that the process under the lift
operator may be shaped by prefixed inputs binding a name inside it. In
this sense, the lift operator will be seen as a way to dynamically
construct processes before reifying them as names.

Finally equipped with these standard features we can present the
dynamics of the calculus.

\subsubsection{Operational semantics} 

Finally, we introduce the computational dynamics. What marks these
algebras as distinct from other more traditionally studied algebraic
structures, e.g. vector spaces or polynomial rings, is the manner in
which dynamics is captured. In traditional structures, dynamics is typically
expressed through morphisms between such structures, as in linear maps
between vector spaces or morphisms between rings. In algebras
associated with the semantics of computation, the dynamics is
expressed as part of the algebraic structure itself, through a
reduction reduction relation typically denoted by $\red$. Below, we
give a recursive presentation of this relation for the calculus used
in the encoding.

$\red \subseteq \pi \times \pi$
$\red : \pi \to \mathcal{P}(\pi)$

\begin{mathpar}
  \inferrule* [lab=Comm] { \textsf{match}( x_{src}, x_{trgt} ) } { x_{trgt}?(y)P \; | \; x_{src}!\langle {Q} \rangle \red P\{\quotep{Q}/y}\} }
  \and \\
  \inferrule* [lab=Par] {{P} \red {P}'} {{{P} | {Q}} \red {{P}' | {Q}}}
  \and
  \inferrule* [lab=Equiv]{{{P} \scong {P}'} \andalso {{P}' \red {Q}'} \andalso {{Q}' \scong {Q}}}{{P} \red {Q}}
\end{mathpar}

\begin{eqnarray*}
  match_{\equiv} (\quotep{P},\quotep{Q}) & := & P \equiv Q \\
  match_{\dagger}(\quotep{P},\quotep{Q}) & := & \forall R. P|Q \red^{*} R => R \red^{*} 0 \\
  match_{K}(\quotep{P},\quotep{Q}) & := & K \mbox{ for some context } K
\end{eqnarray*}

$u?(x)P | u!\langle Q \rangle \red P\{\quotep{Q}/x\}$

%We write $\wred$ for $\red^*$, and $P\red$ if $\exists Q $ such that $ P \red Q$.
We write $P\red$ if $\exists Q $ such that $ P \red Q$ and $P\not\red$, otherwise.

\section{Replication}

As mentioned before, it is known that replication (and hence
recursion) can be implemented in a higher-order process algebra
\cite{SangiorgiWalker}. As our first example of calculation with the
machinery thus far presented we give the construction explicitly in
the {\rhoc}.

\begin{eqnarray}
	D_{x} & := & \prefix{x}{y}{(\binpar{\outputp{x}{y}}{@{y}})} \nonumber\\
	\bangp_{x}{P} & := & \binpar{{x}!\langle{\binpar{D_{x}}{P}}\rangle}{D_{x}} \nonumber
\end{eqnarray}

\begin{eqnarray}
	\bangp_{x}{P} & & \nonumber\\
	=
	& {x}!\langle{(\prefix{x}{y}{(\outputp{x}{y} | @{y})) | P}}\rangle 
	      | \prefix{x}{y}{(\outputp{x}{y} | @{y})} & \nonumber\\
	\red
	& (\outputp{x}{y} | @{y})\substn{\quotep{(\prefix{x}{y}{(@{y} | \outputp{x}{y})) | P}}}{y} & \nonumber\\
	=
	& \outputp{x}{\quotep{(\prefix{x}{y}{(\outputp{x}{y} | @{y})) | P}}}
	  | {(\prefix{x}{y}{(\outputp{x}{y} | @{y})) | P}} & \nonumber\\
	\red
	& \ldots & \nonumber\\
	\red^*
	& P | P | \ldots & \nonumber
\end{eqnarray}

Of course, this encoding, as an implementation, runs away, unfolding
$\bangp{P}$ eagerly. A lazier and more implementable replication
operator, restricted to input-guarded processes, may be obtained as follows.

\begin{eqnarray}
\bangp{\prefix{u}{v}{P}} 
	:= 
	\binpar{\lift{x}{\prefix{u}{v}{(\binpar{D(x)}{P})}}}{D(x)} \nonumber
\end{eqnarray}

\begin{remark}
  Note that the lazier definition still does not deal with summation
  or mixed summation (i.e. sums over input and output). The reader is
  invited to construct definitions of replication that deal with these
  features. 

  Further, the definitions are parameterized in a name, $x$. Can you,
  gentle reader, make a definition that eliminates this parameter and
  guarantees no accidental interaction between the replication
  machinery and the process being replicated -- i.e. no accidental
  sharing of names used by the process to get its work done and the
  name(s) used by the replication to effect copying. This latter
  revision of the definition of replication is crucial to obtaining
  the expected identity $!!P \sim !P$.
\end{remark}

\begin{remark}\label{rem:paradoxical_combinator}
  The reader familiar with the lambda calculus will have noticed the
  similarity between $D$ and the paradoxical combinator.

  [Ed. note: the existence of this seems to suggest we have to be more
  restrictive on the set of processes and names we admit if we are to
  support no-cloning.]
\end{remark}

\subsubsection{Bisimulation}

The computational dynamics gives rise to another kind of equivalence,
the equivalence of computational behavior. As previously mentioned
this is typically captured \emph{via} some form of bisimulation.

% The notion we use in this paper is weak barbed bisimulation
% \cite{milner91polyadicpi}.

The notion we use in this paper is derived from weak barbed
bisimulation \cite{milner91polyadicpi}. 

\begin{definition}
An \emph{observation relation}, $\downarrow_{\mathcal N}$, over a set
of names, $\mathcal N$, is the smallest relation satisfying the rules
below.

\infrule[Out-barb]{y \in {\mathcal N}, \; x \nameeq y}
		  {\outputp{x}{v} \downarrow_{\mathcal N} x}
\infrule[Par-barb]{\mbox{$P\downarrow_{\mathcal N} x$ or $Q\downarrow_{\mathcal N} x$}}
		  {\binpar{P}{Q} \downarrow_{\mathcal N} x}

We write $P \Downarrow_{\mathcal N} x$ if there is $Q$ such that 
$P \wred Q$ and $Q \downarrow_{\mathcal N} x$.
\end{definition}

\begin{definition}
%\label{def.bbisim}
An  ${\mathcal N}$-\emph{barbed bisimulation} over a set of names, ${\mathcal N}$, is a symmetric binary relation 
${\mathcal S}_{\mathcal N}$ between agents such that $P\rel{S}_{\mathcal N}Q$ implies:
\begin{enumerate}
\item If $P \red P'$ then $Q \wred Q'$ and $P'\rel{S}_{\mathcal N} Q'$.
\item If $P\downarrow_{\mathcal N} x$, then $Q\Downarrow_{\mathcal N} x$.
\end{enumerate}
$P$ is ${\mathcal N}$-barbed bisimilar to $Q$, written
$P \wbbisim_{\mathcal N} Q$, if $P \rel{S}_{\mathcal N} Q$ for some ${\mathcal N}$-barbed bisimulation ${\mathcal S}_{\mathcal N}$.
\end{definition}

$\mathcal{R} \subseteq \pi \times \pi$

$P \mathcal{R} Q => \forall P'. P \red P' \Rightarrow \exists Q'. Q \red Q', P' \mathcal{R} Q'$

$P \vdash x \Rightarrow Q \vdash x$

\begin{mathpar}
  \inferrule*[lab=Out-barb]{x \nameeq y}{{y}!\langle{Q}\rangle \vdash x}
  \and
  \inferrule*[lab=Par-barb]{\mbox{$P\vdash x$ or $Q\vdash x$}}{\binpar{P}{Q} \vdash x}
\end{mathpar}

\subsubsection{Contexts}

One of the principle advantages of computational calculi like the
$\pi$-calculus is a well-defined notion of context,
contextual-equivalence and a correlation between
contextual-equivalence and notions of bisimulation. The notion of
context allows the decomposition of a process into (sub-)process and
its syntactic environment, its context. Thus, a context may be
thought of as a process with a ``hole'' (written $\Box$) in it. The
application of a context $M$ to a process $P$, written $M[P]$, is
tantamount to filling the hole in $M$ with $P$. In this paper we do
not need the full weight of this theory, but do make use of the notion
of context in the proof the main theorem. 

\begin{mathpar}
  \inferrule* [lab=summation] {} {{M_{M},M_{N}} \bc \Box \;|\; x.M_{A} \;|\; M_{M}+M_{N}}
  \and
  \inferrule* [lab=agent] {} {{M_{A}} \bc (\vec{x})M_{P} \;| \; \clift{P_0,\ldots,M_{P},\ldots,P_N}}
  \and \\
  \inferrule* [lab=process] {} {{M_{P}} \bc M_{N} \;| \;P|M_{P} }
\end{mathpar} 

\begin{mathpar}
  \inferrule* [lab=sychronization] {} {M_{N} \bc \Box \;|\; x?M_{F} \;|\; x!M_{C}}
  \and
  \inferrule* [lab=abstraction] {} {{M_{F}} \bc (x)M_{P} }
  \and
  \inferrule* [lab=concretion] {} {{M_{C}} \bc \langle M_{P} \rangle }
  \and \\
  \inferrule* [lab=process] {} {{M_{P}} \bc M_{N} \;| \;P|M_{P} }
\end{mathpar}

\begin{definition}[contextual application] Given a context $M$, and
  process $P$, we define the \emph{contextual application}, $M[P] :=
  M\{P/\Box\}$. That is, the contextual application of M to P is the
  substitution of $P$ for $\Box$ in $M$.
\end{definition}

$\meaningof{-} : L \to \mathcal{P}(\pi)$

\begin{mathpar}
  \inferrule* [lab=collection] {} {\meaningof{true} = \pi, \and \meaningof{~E} = \pi \setminus \meaningof{E}, \and \meaningof{E_{1} \& E_{2}} = \meaningof{E_{1}} \cap \meaningof{E_{2}}}
\end{mathpar}

\begin{mathpar}
  \inferrule* [lab=structure] {} {\meaningof{0} = \{ P \in \pi | P \equiv 0 \}, \and \\ \meaningof{E_1 | E_2} = \{ P \in \pi | P \equiv P_{1} | P_{2}, P_{1} \in \meaningof{E_{1}}, P_{2} \in \meaningof{E_2}\} }
\end{mathpar}

\begin{mathpar}
 \inferrule* [lab=behavior] {} {\meaningof{\langle a?b \rangle E} = \{ P \in \pi | P \equiv Q | u?(y)P', \\ \and \\\\ \and \\ \;\;\; u \in \meaningof{a}, \forall z.P'\{z/y\} \in \meaningof{E\{z/b\}}\}, \and \\ \meaningof{a!E} = \{ P \in \pi | P \equiv Q | x!\langle P' \rangle, x \in \meaningof{a} P' \in \meaningof{E}\} }
\end{mathpar}

\begin{mathpar}
 \inferrule* [lab=nominal] {} {\meaningof{\quotep{E}} = \{ \quotep{P} \in \quotep{\pi} | P \in \meaningof{E} \}, \and \meaningof{\quotep{P}} = \{ \quotep{Q} \in \quotep{\pi} | P \equiv Q \} \and \\ \meaningof{@\quotep{E}} = \{ P \in \pi | P \equiv @x, x \in \meaningof{E} \}}
\end{mathpar}

\begin{eqnarray*}
  \\
  \meaningof{-} : TS \to ST
\end{eqnarray*}

\begin{eqnarray*}
  \\
  L : TS \to ST
\end{eqnarray*}

\begin{eqnarray*}
  \\
  P \models E \iff P \in \meaningof{E}
\end{eqnarray*}

\begin{eqnarray*}
  P \approx_{L} Q \iff \forall E \in L. P \models E \iff Q \models E
\end{eqnarray*}

\begin{eqnarray*}
  P \approx_{K} Q
\end{eqnarray*}

\begin{eqnarray*}
  P \approx Q
\end{eqnarray*}

$\approx_{K} = \approx = \approx_{L}$

\subsubsection{Contextual duality}

Note that contexts extend the quotation operation to a family of
operations from processes to names. Given a context, $M$, we can
define a \emph{nominal context}, $\quotep{M}$ by $\quotep{M}[P] :=
\quotep{M[P]}$. To foreshadow what is to come we observe that these
operations enjoy a duality with processes very much like the duality
between vectors and maps from vectors to scalars.

Further, because the calculus is essentially higher-order, we have a
correspondence between contexts and processes. More specifically,
given a name $x$ and a context $M$ we can construct $M^{*}_{x}$ such
that 

\begin{mathpar}
  M^{*}_{x} | \lift{x}{P} \red M[P]
\end{mathpar}

namely,

\begin{mathpar}
  M^{*}_{x} := x?(u).M[\dropn{u}]
\end{mathpar}

The dependence of $M^{*}_{x}$ on a name makes it an abstraction, 

\begin{mathpar}
  M^{*} := (x)x?(u).M[\dropn{u}]
\end{mathpar}

\subsection{Additional notation}

It will sometimes be convenient to denote the process a name
quotes. We already have the notation $x = \quotep{P}$, but it will be
convenient to introduce an alternate notation, $\procn{x}$, when we
want to emphasize the connection to the use of the name. Note that, by
virtue of name equivalence, $\quotep{\procn{x}} \nameeq x$; so, the
notation is consistent with previous definitions.

Further, because names have structure it is possible to effect
substitutions on the basis of that structure. This means we need to
upgrade our notation for substitutions, which we accomplish by
adapting comprehension notation. Thus,

\begin{mathpar}
  P\{ y / x : x \in S \}
\end{mathpar}

is interpreted to mean the process derived from P by replacing (in a
capture-avoiding manner) each occurrence of $x$ in $S$ by $y$. For example,

\begin{mathpar}
  P\{ \quotep{\procn{x}|\procn{x}} / x : x \in \freenames{P} \}
\end{mathpar}

will replace each (occurrence) of a free name $x$ in $P$ by
$\quotep{\procn{x}|\procn{x}}$.

Also, we will avail ourselves of the notation $x^{L}$ and $x^{R}$ to
denote injections of a name into disjoint copies of the name
space. There are numerous ways to accomplish this. One example can be
found in \cite{MeredithR05}. This notation overloads to vectors of
names: $\vec{x}^{\pi} := (x_{i}^{\pi} \; : \; 0 \leq i < |\vec{x}| )$ where $\pi \in \{L,R\}$.

We also use $P^{\Box} := P|\Box$.

In \cite{MeredithR05} an interpretation of the new operator is
given. It turns out that there are several possible interpretations
all enjoying the requisite algebraic properties of the operator (see
\cite{milner91polyadicpi}). We will therefore make liberal use of
$(\nu\; \vec{x})P$.

% subsection the_syntax_and_semantics_of_the_notation_system (end)   

\input{qm2pi.qmops} 

\input{qm2pi.sterngerlach} 

\input{qm2pi.metric} 

% section concurrent_process_calculi (end)

%\input{qm2pi.proofsketch}

% section proof sketch (end)

%\input{qm2pi.slviaknots} 

% section spatial logic via knots (end)

\input{qm2pi.conclusion}

% section conclusion (end)

%\input{qm2pi.dtcodes} 

% section wiring algorithm (end)

\input{qm2pi.ack} 

% section acknowledgments (end)

\newpage


\bibliographystyle{plain}   
\bibliography{../../biblios/main.bib}

\input{qm2pi.rhodetails}

\end{document}

 

%\ifpdf
%\usepackage[pdftex]{graphicx}
%\else
%\usepackage{graphicx}
%\fi

 % \ifpdf
%  \usepackage{pdfsync}
%  \if


%\title{Brief Article}
%\author{David F. Snyder}
%\author{L.G. Meredith}

%\address{Dept. of Math., Texas State University--San Marcos, San Marcos, TX 78666}
       
\pagestyle{empty}


\begin{document}

\lstset{language=[Objective]Caml,frame=shadowbox}

\documentclass[12pt]{llncs}
%\documentclass{jktr}

\usepackage[pdftex]{hyperref}                   
\usepackage {listings}
\usepackage {mathpartir}
\usepackage{bcprules}
%\usepackage{listings}
                       
\usepackage{graphicx} 
%\usepackage[margins=2.5cm,nohead,nofoot]{geometry}
%\usepackage{geometry}
\usepackage{amsfonts}
\usepackage{amstext}
\usepackage{latexsym}
\usepackage{amssymb}
\usepackage{color}


%\include{myPreamble}
\include{qm2pi.local} 

%\ifpdf
%\usepackage[pdftex]{graphicx}
%\else
%\usepackage{graphicx}
%\fi

 % \ifpdf
%  \usepackage{pdfsync}
%  \if


%\title{Brief Article}
%\author{David F. Snyder}
%\author{L.G. Meredith}

%\address{Dept. of Math., Texas State University--San Marcos, San Marcos, TX 78666}
       
\pagestyle{empty}


\begin{document}

\lstset{language=[Objective]Caml,frame=shadowbox}

\input{qm2pi.front}

% section front matter (end)

\input{qm2pi.intro} 
 
% section introduction (end)

% \input{qm2pi.knotations} 

% section notation (end)

\input{qm2pi.process.calculi} 

% section concurrent_process_calculi_and_spatial_logics_ (end)
    
%\input{qm2pi.knots2pi} 

%\input{qm2pi.trefoil} 

%\input{qm2pi.mainthm} 

% subsection basic_interpretation (end)

%\input{qm2pi.rho.presentation} 
\subsection{The syntax and semantics of the notation system}\label{sub:the_syntax_and_semantics_of_the_notation_system} % (fold)

We now summarize a technical presentation of the calculus that
embodies our theory of dynamics. The typical presentation of such a
calculus follows the style of giving generators and relations on
them. The grammar, below, describing term constructors, freely
generates the set of processes, $\Proc$. This set is then quotiented
by a relation known as structural congruence and it is over this set
that the notion of dynamics is expressed. This presentation is
essentially that of \cite{MeredithR05} with the addition of
polyadicity and summation. For readability we have relegated some of
the technical subtleties to an appendix.

\subsubsection{Process grammar}\label{subsub:process_grammar}

\begin{mathpar}
  \inferrule* [lab=synchronization] {} {{M} \bc \pzero \;|\; x?F \;|\; x!C }
  \and
  \inferrule* [lab=abstraction] {} {{F} \bc (x)P}
  \and
  \inferrule* [lab=concretion] {} {{C} \bc \langle Q \rangle}
  \and
  \inferrule* [lab=process] {} {{P,Q} \bc M \;| \;P|Q \;|\; @{x}}
  \and
  \inferrule* [lab=name] {} {{x} \bc \quotep{P}}
\end{mathpar} 

Note that $\vec{x}$ (resp. $\vec{P}$) denotes a vector of names
(resp. processes) of length $|\vec{x}|$ (resp. $|\vec{P}|$). We adopt
the following useful abbreviations.

\begin{mathpar}
   x?(\vec{y}).P := x.(\vec{y})P \and  x\clift{\vec{P}} := x.\clift{\vec{P}}
   \and x!(y) := \lift{x}{\dropn{y}}
   \and \Pi_{i=0}^{n-1}P_i := P_0 | \ldots | P_{n-1}
\end{mathpar}

\subsubsection{Structural congruence}

\paragraph{Free and bound names and alpha-equivalence.} At the
core of structural equivalence is alpha-equivalence which identifies
process that are the same up to a change of variable. Formally, we
recognize the distinction between free and bound names. The free names
of a process, $\freenames{P}$, may be calculated recursively as
follows:

\begin{mathpar}
\freenames{\pzero} := \emptyset
  \and \\
  \freenames{x?(y).P} := \{ x \} \cup (\freenames{P} \setminus \{ y \})
  \and 
  \freenames{x!\langle P \rangle} := \{ x \} \cup \{ P \} 
  \and \\
  \freenames{P|Q} := \freenames{P} \cup \freenames{Q}
  \and \\
  \freenames{@{x}} := \{ x \}
\end{mathpar}

$\pi$
$\quotep{\pi}$

$\freenames{-} : \pi \to \mathcal{P}(\quotep{\pi})$

\begin{eqnarray*}
  \freenames{\pzero} & := & \emptyset \\
  \freenames{x?(y).P} & := & \{ x \} \cup (\freenames{P} \setminus \{ y \}) \\
  \freenames{x!\langle P \rangle} & := & \{ x \} \cup \{ P \} \\
  \freenames{P|Q} & := & \freenames{P} \cup \freenames{Q} \\
  \freenames{\dropn{x}} & := & \{ x \}
\end{eqnarray*}

The bound names of a process, $\boundnames{P}$, are those names occurring in $P$
that are not free. For example, in $x?(y).0$, the name $x$ is free, while $y$ is bound.

\begin{mathpar}
  \inferrule* [lab=monoidal-laws] {} { P|Q \equiv Q|P \and P|0 \equiv P \and P|(Q|R) \equiv (P|Q)|R }
\end{mathpar}

\begin{mathpar}
  \inferrule* [lab=alpha-equivalence] {} { (x)P \equiv (y)P\{y/x\} \and y \not\in \freenames{P} }
\end{mathpar}

\begin{definition}
Then two processes, $P,Q$, are alpha-equivalent if $P = Q\{\vec{y}/\vec{x}\}$ for
some $\vec{x} \in \boundnames{Q},\vec{y} \in \boundnames{P}$, where $Q\{\vec{y}/\vec{x}\}$
denotes the capture-avoiding substitution of $\vec{y}$ for $\vec{x}$ in $Q$.
\end{definition}

\begin{definition}
  The {\em structural congruence} \cite{SangiorgiWalker} , $\equiv$,
  between processes is the least congruence containing
  alpha-equivalence, satisfying the abelian monoid laws
  (associativity, commutativity and $\pzero$ as identity) for parallel
  composition $|$ and for summation $+$.
\end{definition}

\subsection{Name equivalence}

We take name equivalence, written $\nameeq$, to be the smallest
equivalence relation generated by the following rules.

\begin{mathpar}
\inferrule*[lab=Quote-drop]
{ }
{ \quotep{@{x}} \nameeq x }

\inferrule*[lab=Struct-equiv]
{ P \scong Q }
{ \quotep{P} \nameeq \quotep{Q} }
\end{mathpar}

The astute reader will have noticed that the mutual recursion of names
and processes imposes a mutual recursion on alpha-equivalence and
structural equivalence via name-equivalence. Fortunately, all of this
works out pleasantly and we may calculate in the natural way, free of
concern. The reader interested in the details is referred to the
appendix \ref{appendix:rho_details}.

\subsection{Substitution}

We use $\Proc$ for the set of processes, $\QProc$ for the set of
names, and $\id{\{}\vec{y} / \vec{x} \id{\}}$ to denote partial maps,
$s : \QProc \rightarrow \QProc$. A map, $s$ lifts, uniquely, to a map
on process terms, $\widehat{s} : \Proc \rightarrow \Proc$ by the
following equations.

\begin{mathpar}
  (0) \psubstp{Q}{P} := 0 \\
  (R \juxtap S) \psubstp{Q}{P}
  :=    
  (R)\psubstp{Q}{P} \juxtap (S) \psubstp{Q}{P} \\
  (x?(y).R) \psubstp{Q}{P}    
  :=    
  (x)\substp{Q}{P} (z)\concat( (R \psubstn{z}{y}) \psubstp{Q}{P} ) \\
  (\lift{x}{R}) \psubstp{Q}{P}  
  :=
  \lift{(x)\substp{Q}{P}}{ R \psubstp{Q}{P} } \\
%   (\dropn{x})  \psubstp{Q}{P}       
%   := 
%   \left\{ 
%     \begin{array}{ccc} 
%       \dropn{\quotep{Q}} & & x \nameeq \quotep{P} \\
%       \dropn{x} & & otherwise \\
%     \end{array}
%   \right. 
  (\dropn{x})  \psubstp{Q}{P}       
  := 
  \left\{ 
    \begin{array}{ccc} 
      Q & & x \nameeq \quotep{P} \\
      \dropn{x} & & otherwise \\
    \end{array}
  \right.
\end{mathpar}
 

where

\begin{eqnarray}
  (x)\id{\{} \lpquote Q \rpquote / \lpquote P \rpquote \id{\}}            = 
  \left\{ 
    \begin{array}{ccc}
      \lpquote Q \rpquote & & x \nameeq \lpquote P \rpquote \\
      x & & otherwise \\
    \end{array}
  \right. \nonumber
\end{eqnarray}

and $z$ is chosen distinct from $\quotep{P}$, $\quotep{Q}$, the free
names in $Q$, and all the names in $R$. Our $\alpha$-equivalence will
be built in the standard way from this substitution.

\begin{remark}\label{rem:no_self_referential_names}
  One consequence of these definitions is that $\forall P. \quotep{P}
  \not\in \freenames{P}$.
\end{remark}

\subsection{ Dynamic quote: an example }

Anticipating something of what's to come, consider applying the
substitution, $\widehat{\id{\{}u / z \id{\}}}$, to the following pair
of processes, $\lift{w}{y!(z)}$ and $w[ \lpquote y!(z) \rpquote ]$.

\begin{eqnarray}
	\lift{w}{y!(z)}\widehat{\id{\{}u / z \id{\}}}
		& = &
		\lift{w}{y!(u)} \nonumber\\
	w[ \lpquote y!(z) \rpquote ] \widehat{ \id{\{}u / z \id{\}} }
		& = &
		w[ \lpquote y!(z) \rpquote ] \nonumber
\end{eqnarray}

Because the body of the process between quotes is impervious to
substitution, we get radically different answers. In fact, by
examining the first process in an input context,
e.g. $x?(z).\lift{w}{y!(z)}$, we see that the process under the lift
operator may be shaped by prefixed inputs binding a name inside it. In
this sense, the lift operator will be seen as a way to dynamically
construct processes before reifying them as names.

Finally equipped with these standard features we can present the
dynamics of the calculus.

\subsubsection{Operational semantics} 

Finally, we introduce the computational dynamics. What marks these
algebras as distinct from other more traditionally studied algebraic
structures, e.g. vector spaces or polynomial rings, is the manner in
which dynamics is captured. In traditional structures, dynamics is typically
expressed through morphisms between such structures, as in linear maps
between vector spaces or morphisms between rings. In algebras
associated with the semantics of computation, the dynamics is
expressed as part of the algebraic structure itself, through a
reduction reduction relation typically denoted by $\red$. Below, we
give a recursive presentation of this relation for the calculus used
in the encoding.

$\red \subseteq \pi \times \pi$
$\red : \pi \to \mathcal{P}(\pi)$

\begin{mathpar}
  \inferrule* [lab=Comm] { \textsf{match}( x_{src}, x_{trgt} ) } { x_{trgt}?(y)P \; | \; x_{src}!\langle {Q} \rangle \red P\{\quotep{Q}/y}\} }
  \and \\
  \inferrule* [lab=Par] {{P} \red {P}'} {{{P} | {Q}} \red {{P}' | {Q}}}
  \and
  \inferrule* [lab=Equiv]{{{P} \scong {P}'} \andalso {{P}' \red {Q}'} \andalso {{Q}' \scong {Q}}}{{P} \red {Q}}
\end{mathpar}

\begin{eqnarray*}
  match_{\equiv} (\quotep{P},\quotep{Q}) & := & P \equiv Q \\
  match_{\dagger}(\quotep{P},\quotep{Q}) & := & \forall R. P|Q \red^{*} R => R \red^{*} 0 \\
  match_{K}(\quotep{P},\quotep{Q}) & := & K \mbox{ for some context } K
\end{eqnarray*}

$u?(x)P | u!\langle Q \rangle \red P\{\quotep{Q}/x\}$

%We write $\wred$ for $\red^*$, and $P\red$ if $\exists Q $ such that $ P \red Q$.
We write $P\red$ if $\exists Q $ such that $ P \red Q$ and $P\not\red$, otherwise.

\section{Replication}

As mentioned before, it is known that replication (and hence
recursion) can be implemented in a higher-order process algebra
\cite{SangiorgiWalker}. As our first example of calculation with the
machinery thus far presented we give the construction explicitly in
the {\rhoc}.

\begin{eqnarray}
	D_{x} & := & \prefix{x}{y}{(\binpar{\outputp{x}{y}}{@{y}})} \nonumber\\
	\bangp_{x}{P} & := & \binpar{{x}!\langle{\binpar{D_{x}}{P}}\rangle}{D_{x}} \nonumber
\end{eqnarray}

\begin{eqnarray}
	\bangp_{x}{P} & & \nonumber\\
	=
	& {x}!\langle{(\prefix{x}{y}{(\outputp{x}{y} | @{y})) | P}}\rangle 
	      | \prefix{x}{y}{(\outputp{x}{y} | @{y})} & \nonumber\\
	\red
	& (\outputp{x}{y} | @{y})\substn{\quotep{(\prefix{x}{y}{(@{y} | \outputp{x}{y})) | P}}}{y} & \nonumber\\
	=
	& \outputp{x}{\quotep{(\prefix{x}{y}{(\outputp{x}{y} | @{y})) | P}}}
	  | {(\prefix{x}{y}{(\outputp{x}{y} | @{y})) | P}} & \nonumber\\
	\red
	& \ldots & \nonumber\\
	\red^*
	& P | P | \ldots & \nonumber
\end{eqnarray}

Of course, this encoding, as an implementation, runs away, unfolding
$\bangp{P}$ eagerly. A lazier and more implementable replication
operator, restricted to input-guarded processes, may be obtained as follows.

\begin{eqnarray}
\bangp{\prefix{u}{v}{P}} 
	:= 
	\binpar{\lift{x}{\prefix{u}{v}{(\binpar{D(x)}{P})}}}{D(x)} \nonumber
\end{eqnarray}

\begin{remark}
  Note that the lazier definition still does not deal with summation
  or mixed summation (i.e. sums over input and output). The reader is
  invited to construct definitions of replication that deal with these
  features. 

  Further, the definitions are parameterized in a name, $x$. Can you,
  gentle reader, make a definition that eliminates this parameter and
  guarantees no accidental interaction between the replication
  machinery and the process being replicated -- i.e. no accidental
  sharing of names used by the process to get its work done and the
  name(s) used by the replication to effect copying. This latter
  revision of the definition of replication is crucial to obtaining
  the expected identity $!!P \sim !P$.
\end{remark}

\begin{remark}\label{rem:paradoxical_combinator}
  The reader familiar with the lambda calculus will have noticed the
  similarity between $D$ and the paradoxical combinator.

  [Ed. note: the existence of this seems to suggest we have to be more
  restrictive on the set of processes and names we admit if we are to
  support no-cloning.]
\end{remark}

\subsubsection{Bisimulation}

The computational dynamics gives rise to another kind of equivalence,
the equivalence of computational behavior. As previously mentioned
this is typically captured \emph{via} some form of bisimulation.

% The notion we use in this paper is weak barbed bisimulation
% \cite{milner91polyadicpi}.

The notion we use in this paper is derived from weak barbed
bisimulation \cite{milner91polyadicpi}. 

\begin{definition}
An \emph{observation relation}, $\downarrow_{\mathcal N}$, over a set
of names, $\mathcal N$, is the smallest relation satisfying the rules
below.

\infrule[Out-barb]{y \in {\mathcal N}, \; x \nameeq y}
		  {\outputp{x}{v} \downarrow_{\mathcal N} x}
\infrule[Par-barb]{\mbox{$P\downarrow_{\mathcal N} x$ or $Q\downarrow_{\mathcal N} x$}}
		  {\binpar{P}{Q} \downarrow_{\mathcal N} x}

We write $P \Downarrow_{\mathcal N} x$ if there is $Q$ such that 
$P \wred Q$ and $Q \downarrow_{\mathcal N} x$.
\end{definition}

\begin{definition}
%\label{def.bbisim}
An  ${\mathcal N}$-\emph{barbed bisimulation} over a set of names, ${\mathcal N}$, is a symmetric binary relation 
${\mathcal S}_{\mathcal N}$ between agents such that $P\rel{S}_{\mathcal N}Q$ implies:
\begin{enumerate}
\item If $P \red P'$ then $Q \wred Q'$ and $P'\rel{S}_{\mathcal N} Q'$.
\item If $P\downarrow_{\mathcal N} x$, then $Q\Downarrow_{\mathcal N} x$.
\end{enumerate}
$P$ is ${\mathcal N}$-barbed bisimilar to $Q$, written
$P \wbbisim_{\mathcal N} Q$, if $P \rel{S}_{\mathcal N} Q$ for some ${\mathcal N}$-barbed bisimulation ${\mathcal S}_{\mathcal N}$.
\end{definition}

$\mathcal{R} \subseteq \pi \times \pi$

$P \mathcal{R} Q => \forall P'. P \red P' \Rightarrow \exists Q'. Q \red Q', P' \mathcal{R} Q'$

$P \vdash x \Rightarrow Q \vdash x$

\begin{mathpar}
  \inferrule*[lab=Out-barb]{x \nameeq y}{{y}!\langle{Q}\rangle \vdash x}
  \and
  \inferrule*[lab=Par-barb]{\mbox{$P\vdash x$ or $Q\vdash x$}}{\binpar{P}{Q} \vdash x}
\end{mathpar}

\subsubsection{Contexts}

One of the principle advantages of computational calculi like the
$\pi$-calculus is a well-defined notion of context,
contextual-equivalence and a correlation between
contextual-equivalence and notions of bisimulation. The notion of
context allows the decomposition of a process into (sub-)process and
its syntactic environment, its context. Thus, a context may be
thought of as a process with a ``hole'' (written $\Box$) in it. The
application of a context $M$ to a process $P$, written $M[P]$, is
tantamount to filling the hole in $M$ with $P$. In this paper we do
not need the full weight of this theory, but do make use of the notion
of context in the proof the main theorem. 

\begin{mathpar}
  \inferrule* [lab=summation] {} {{M_{M},M_{N}} \bc \Box \;|\; x.M_{A} \;|\; M_{M}+M_{N}}
  \and
  \inferrule* [lab=agent] {} {{M_{A}} \bc (\vec{x})M_{P} \;| \; \clift{P_0,\ldots,M_{P},\ldots,P_N}}
  \and \\
  \inferrule* [lab=process] {} {{M_{P}} \bc M_{N} \;| \;P|M_{P} }
\end{mathpar} 

\begin{mathpar}
  \inferrule* [lab=sychronization] {} {M_{N} \bc \Box \;|\; x?M_{F} \;|\; x!M_{C}}
  \and
  \inferrule* [lab=abstraction] {} {{M_{F}} \bc (x)M_{P} }
  \and
  \inferrule* [lab=concretion] {} {{M_{C}} \bc \langle M_{P} \rangle }
  \and \\
  \inferrule* [lab=process] {} {{M_{P}} \bc M_{N} \;| \;P|M_{P} }
\end{mathpar}

\begin{definition}[contextual application] Given a context $M$, and
  process $P$, we define the \emph{contextual application}, $M[P] :=
  M\{P/\Box\}$. That is, the contextual application of M to P is the
  substitution of $P$ for $\Box$ in $M$.
\end{definition}

$\meaningof{-} : L \to \mathcal{P}(\pi)$

\begin{mathpar}
  \inferrule* [lab=collection] {} {\meaningof{true} = \pi, \and \meaningof{~E} = \pi \setminus \meaningof{E}, \and \meaningof{E_{1} \& E_{2}} = \meaningof{E_{1}} \cap \meaningof{E_{2}}}
\end{mathpar}

\begin{mathpar}
  \inferrule* [lab=structure] {} {\meaningof{0} = \{ P \in \pi | P \equiv 0 \}, \and \\ \meaningof{E_1 | E_2} = \{ P \in \pi | P \equiv P_{1} | P_{2}, P_{1} \in \meaningof{E_{1}}, P_{2} \in \meaningof{E_2}\} }
\end{mathpar}

\begin{mathpar}
 \inferrule* [lab=behavior] {} {\meaningof{\langle a?b \rangle E} = \{ P \in \pi | P \equiv Q | u?(y)P', \\ \and \\\\ \and \\ \;\;\; u \in \meaningof{a}, \forall z.P'\{z/y\} \in \meaningof{E\{z/b\}}\}, \and \\ \meaningof{a!E} = \{ P \in \pi | P \equiv Q | x!\langle P' \rangle, x \in \meaningof{a} P' \in \meaningof{E}\} }
\end{mathpar}

\begin{mathpar}
 \inferrule* [lab=nominal] {} {\meaningof{\quotep{E}} = \{ \quotep{P} \in \quotep{\pi} | P \in \meaningof{E} \}, \and \meaningof{\quotep{P}} = \{ \quotep{Q} \in \quotep{\pi} | P \equiv Q \} \and \\ \meaningof{@\quotep{E}} = \{ P \in \pi | P \equiv @x, x \in \meaningof{E} \}}
\end{mathpar}

\begin{eqnarray*}
  \\
  \meaningof{-} : TS \to ST
\end{eqnarray*}

\begin{eqnarray*}
  \\
  L : TS \to ST
\end{eqnarray*}

\begin{eqnarray*}
  \\
  P \models E \iff P \in \meaningof{E}
\end{eqnarray*}

\begin{eqnarray*}
  P \approx_{L} Q \iff \forall E \in L. P \models E \iff Q \models E
\end{eqnarray*}

\begin{eqnarray*}
  P \approx_{K} Q
\end{eqnarray*}

\begin{eqnarray*}
  P \approx Q
\end{eqnarray*}

$\approx_{K} = \approx = \approx_{L}$

\subsubsection{Contextual duality}

Note that contexts extend the quotation operation to a family of
operations from processes to names. Given a context, $M$, we can
define a \emph{nominal context}, $\quotep{M}$ by $\quotep{M}[P] :=
\quotep{M[P]}$. To foreshadow what is to come we observe that these
operations enjoy a duality with processes very much like the duality
between vectors and maps from vectors to scalars.

Further, because the calculus is essentially higher-order, we have a
correspondence between contexts and processes. More specifically,
given a name $x$ and a context $M$ we can construct $M^{*}_{x}$ such
that 

\begin{mathpar}
  M^{*}_{x} | \lift{x}{P} \red M[P]
\end{mathpar}

namely,

\begin{mathpar}
  M^{*}_{x} := x?(u).M[\dropn{u}]
\end{mathpar}

The dependence of $M^{*}_{x}$ on a name makes it an abstraction, 

\begin{mathpar}
  M^{*} := (x)x?(u).M[\dropn{u}]
\end{mathpar}

\subsection{Additional notation}

It will sometimes be convenient to denote the process a name
quotes. We already have the notation $x = \quotep{P}$, but it will be
convenient to introduce an alternate notation, $\procn{x}$, when we
want to emphasize the connection to the use of the name. Note that, by
virtue of name equivalence, $\quotep{\procn{x}} \nameeq x$; so, the
notation is consistent with previous definitions.

Further, because names have structure it is possible to effect
substitutions on the basis of that structure. This means we need to
upgrade our notation for substitutions, which we accomplish by
adapting comprehension notation. Thus,

\begin{mathpar}
  P\{ y / x : x \in S \}
\end{mathpar}

is interpreted to mean the process derived from P by replacing (in a
capture-avoiding manner) each occurrence of $x$ in $S$ by $y$. For example,

\begin{mathpar}
  P\{ \quotep{\procn{x}|\procn{x}} / x : x \in \freenames{P} \}
\end{mathpar}

will replace each (occurrence) of a free name $x$ in $P$ by
$\quotep{\procn{x}|\procn{x}}$.

Also, we will avail ourselves of the notation $x^{L}$ and $x^{R}$ to
denote injections of a name into disjoint copies of the name
space. There are numerous ways to accomplish this. One example can be
found in \cite{MeredithR05}. This notation overloads to vectors of
names: $\vec{x}^{\pi} := (x_{i}^{\pi} \; : \; 0 \leq i < |\vec{x}| )$ where $\pi \in \{L,R\}$.

We also use $P^{\Box} := P|\Box$.

In \cite{MeredithR05} an interpretation of the new operator is
given. It turns out that there are several possible interpretations
all enjoying the requisite algebraic properties of the operator (see
\cite{milner91polyadicpi}). We will therefore make liberal use of
$(\nu\; \vec{x})P$.

% subsection the_syntax_and_semantics_of_the_notation_system (end)   

\input{qm2pi.qmops} 

\input{qm2pi.sterngerlach} 

\input{qm2pi.metric} 

% section concurrent_process_calculi (end)

%\input{qm2pi.proofsketch}

% section proof sketch (end)

%\input{qm2pi.slviaknots} 

% section spatial logic via knots (end)

\input{qm2pi.conclusion}

% section conclusion (end)

%\input{qm2pi.dtcodes} 

% section wiring algorithm (end)

\input{qm2pi.ack} 

% section acknowledgments (end)

\newpage


\bibliographystyle{plain}   
\bibliography{../../biblios/main.bib}

\input{qm2pi.rhodetails}

\end{document}



% section front matter (end)

\section{Introduction}\label{sec:introduction} % (fold)
In this draft of the material i am going to have to dispense with the
usual writing conventions adopted in papers on these topics. i'm going
to have adopt whatever tone i need at the time i'm writing up the
calculations. Sometimes this may be very conversational; others it may
be the barest mathematical grunts; others still it may be that i have
lifted text from one of my other papers because the exposition of some
point was better said there. i hope that my readers are not unduly put
out by this decision. i'm not doing this to flout convention or be
rebellious. i find these calculations very technically challenging. To
keep everything going technically, something has to give; i have to
let go of some cognitive burden. So, the academic writing style --
with all of its trade-offs in terms of facilitating technical
communication -- is what i'm letting go of. Perhaps subsequent drafts
can be tightened and polished, but for now, i'm going to speak as if
we were sitting together in a coffee shop with a laptop, wifi and a
pad of paper and a pencil.

So, here's what i have to say. We -- you and i, comfortably ensconced
in our coffee shop and well-equipped with our tools -- can realize and
carry out the calculations of quantum mechanics over a very different
formal theory of dynamics, a formal theory of dynamics that
corresponds to a theory of concurrent computation with
\emph{reflection}. It has the advantage that the underlying theory is
already `quantized', but supports analogues all of the continuuous
operations. Strikingly, this underlying theory has recently been
connected with a notion of metric that we can show, by calculating
together, coincides with the metric induced by the inner product.

There are a lot of reasons why you might be interested in seeing
calculations of this form. Here's why i'm interested. For the past
several centuries there has been no competitor to the ``Newtonian''
account of dynamics. As a result the predominant share of accounts of
dynamical systems and situations have had to be formulated in terms of
the Newtonian machinery. i view this as an intellectually dangerous
position to occupy. Everything, despite it's intrinsic shape, turns
into a nail to be hit with this hammer. Recently, however, the theory
of computation has matured to the point where we have candidates for
theories of dynamics that offer very different perspective on
reasoning about dynamical systems and situations. Testing these
candidates against very successful accounts of dynamical situations,
like quantum mechanics, is going to give us some sense of how mature
they are and some measure of the quality of these accounts of
dynamics.

\subsection{Summary of contributions and outline of paper}

So, we're going to develop an interpretation of the operations of
quantum mechanics normally interpreted by Hilbert spaces and
operators. We're going to do this over a theory of computation. Note
that this is very different than the usual quantum computation program
which develops notions of computation over quantum mechanics. Rather,
we are developing a story that aligns with Wheeler's slogan: It from
Bit. To do this we will first provide an account of the theory of
computation at play here. Then we will dive into a calculation-driven
interpretation of the operations of quantum mechanics.

The reason we take this approach is that -- until very recently --
there hasn't been an axiomatic account of quantum mechanics. As a
result there has been no sharp delineation of the mathematical theory
supporting interpretation of the physical theory and the physical
theory, itself. So, ambient features of the maths are free to be
exploited (or supressed) without a real accounting of their physical
relevance. There is no sharp statement ``here's the physical theory''
qua \emph{theory} and ``here's the mathematical interpretation''
enabling a judgment of how faithful the interpretation is -- apart
from experimental observation. When there is an axiomatic account we
can judge how well a given mathematical formalism supports an
interpretation of the axioms, independent of
experimentation. Likewise, we can judge how well we have captured our
physical evidence and experience with our axiomatics, independent of
any specific mathematical implementation, with accidental detail that
may or may not have physical significance. 

In lieu of a fully fleshed out and vetted axiomatic account of quantum
mechanics, interpreting the operational notions in service of modeling
physical systems will have to suffice. In other words, we are not in
the business of providing a model of Hilbert spaces and operators. We
are in the business of providing a model of quantum mechanics because
we are motivated by testing our notions of dynamics against physical
theory; and, the predictive calculations of the physical theory must
serve as the best formulation -- shy of a fully fleshed out axiomatic
account -- of the physical theory itself (as they have for scientific
theories since time immemorial). Put another way, despite a
whole-hearted commitment to an It-from-Bit ontology, we are firmly
aligned with the shut-up-and-calculate camp as the best way to obtain
results either from the physical perspective or as a quality assurance
measure of our fledgling theory of dynamics.

In detail, we present a reflective process calculus. Then we develop
intuitive correspondences between the notions available in this
calculus and the usual physical notions supporting quantum mechanical
calculations. Thus, 

\begin{table}[htp]
  \center{
    \fbox{
      \begin{tabular}{c|c}
        quantum mechanics & process calculus \\
        \hline
        scalar & name \\
        state vector & process \\
        dual & contextual duals \\
        matrix & formal sums of process-context-dual pairs \\
        orthogonality & process annihilation \\
        inner product & execution-formula + quoting
      \end{tabular}
    }
  }
  \caption{QM - process calculi correspondences}
\end{table}

Then we tighten up these intuitions to operational definitions. We
employ the Dirac notation as the best proxy we can find for an
abstract syntax of the quantum mechanical notions. The definitions we
develop put us in contact with equational constraints coming from the
theory that we demonstrate the definitions and calculations satisfy.

This puts us in a position to shut up and calculate for the
Stern-Gerlach experimental set up, showing how these predictive
calculations become calculations on processes in our theory of a
reflective process calculus.

Penultimately, we demonstrate that the notion of metric coming from
the inner product coincides with the notion of metric available from
the theory of bisimulation. This demonstration gives us the right to
think of space as arising from behavior. Finally, we consider where we
might go from the new vantage point we have obtained.

% section introduction (end) 
 
% section introduction (end)

% \documentclass[12pt]{llncs}
%\documentclass{jktr}

\usepackage[pdftex]{hyperref}                   
\usepackage {listings}
\usepackage {mathpartir}
\usepackage{bcprules}
%\usepackage{listings}
                       
\usepackage{graphicx} 
%\usepackage[margins=2.5cm,nohead,nofoot]{geometry}
%\usepackage{geometry}
\usepackage{amsfonts}
\usepackage{amstext}
\usepackage{latexsym}
\usepackage{amssymb}
\usepackage{color}


%\include{myPreamble}
\include{qm2pi.local} 

%\ifpdf
%\usepackage[pdftex]{graphicx}
%\else
%\usepackage{graphicx}
%\fi

 % \ifpdf
%  \usepackage{pdfsync}
%  \if


%\title{Brief Article}
%\author{David F. Snyder}
%\author{L.G. Meredith}

%\address{Dept. of Math., Texas State University--San Marcos, San Marcos, TX 78666}
       
\pagestyle{empty}


\begin{document}

\lstset{language=[Objective]Caml,frame=shadowbox}

\input{qm2pi.front}

% section front matter (end)

\input{qm2pi.intro} 
 
% section introduction (end)

% \input{qm2pi.knotations} 

% section notation (end)

\input{qm2pi.process.calculi} 

% section concurrent_process_calculi_and_spatial_logics_ (end)
    
%\input{qm2pi.knots2pi} 

%\input{qm2pi.trefoil} 

%\input{qm2pi.mainthm} 

% subsection basic_interpretation (end)

%\input{qm2pi.rho.presentation} 
\subsection{The syntax and semantics of the notation system}\label{sub:the_syntax_and_semantics_of_the_notation_system} % (fold)

We now summarize a technical presentation of the calculus that
embodies our theory of dynamics. The typical presentation of such a
calculus follows the style of giving generators and relations on
them. The grammar, below, describing term constructors, freely
generates the set of processes, $\Proc$. This set is then quotiented
by a relation known as structural congruence and it is over this set
that the notion of dynamics is expressed. This presentation is
essentially that of \cite{MeredithR05} with the addition of
polyadicity and summation. For readability we have relegated some of
the technical subtleties to an appendix.

\subsubsection{Process grammar}\label{subsub:process_grammar}

\begin{mathpar}
  \inferrule* [lab=synchronization] {} {{M} \bc \pzero \;|\; x?F \;|\; x!C }
  \and
  \inferrule* [lab=abstraction] {} {{F} \bc (x)P}
  \and
  \inferrule* [lab=concretion] {} {{C} \bc \langle Q \rangle}
  \and
  \inferrule* [lab=process] {} {{P,Q} \bc M \;| \;P|Q \;|\; @{x}}
  \and
  \inferrule* [lab=name] {} {{x} \bc \quotep{P}}
\end{mathpar} 

Note that $\vec{x}$ (resp. $\vec{P}$) denotes a vector of names
(resp. processes) of length $|\vec{x}|$ (resp. $|\vec{P}|$). We adopt
the following useful abbreviations.

\begin{mathpar}
   x?(\vec{y}).P := x.(\vec{y})P \and  x\clift{\vec{P}} := x.\clift{\vec{P}}
   \and x!(y) := \lift{x}{\dropn{y}}
   \and \Pi_{i=0}^{n-1}P_i := P_0 | \ldots | P_{n-1}
\end{mathpar}

\subsubsection{Structural congruence}

\paragraph{Free and bound names and alpha-equivalence.} At the
core of structural equivalence is alpha-equivalence which identifies
process that are the same up to a change of variable. Formally, we
recognize the distinction between free and bound names. The free names
of a process, $\freenames{P}$, may be calculated recursively as
follows:

\begin{mathpar}
\freenames{\pzero} := \emptyset
  \and \\
  \freenames{x?(y).P} := \{ x \} \cup (\freenames{P} \setminus \{ y \})
  \and 
  \freenames{x!\langle P \rangle} := \{ x \} \cup \{ P \} 
  \and \\
  \freenames{P|Q} := \freenames{P} \cup \freenames{Q}
  \and \\
  \freenames{@{x}} := \{ x \}
\end{mathpar}

$\pi$
$\quotep{\pi}$

$\freenames{-} : \pi \to \mathcal{P}(\quotep{\pi})$

\begin{eqnarray*}
  \freenames{\pzero} & := & \emptyset \\
  \freenames{x?(y).P} & := & \{ x \} \cup (\freenames{P} \setminus \{ y \}) \\
  \freenames{x!\langle P \rangle} & := & \{ x \} \cup \{ P \} \\
  \freenames{P|Q} & := & \freenames{P} \cup \freenames{Q} \\
  \freenames{\dropn{x}} & := & \{ x \}
\end{eqnarray*}

The bound names of a process, $\boundnames{P}$, are those names occurring in $P$
that are not free. For example, in $x?(y).0$, the name $x$ is free, while $y$ is bound.

\begin{mathpar}
  \inferrule* [lab=monoidal-laws] {} { P|Q \equiv Q|P \and P|0 \equiv P \and P|(Q|R) \equiv (P|Q)|R }
\end{mathpar}

\begin{mathpar}
  \inferrule* [lab=alpha-equivalence] {} { (x)P \equiv (y)P\{y/x\} \and y \not\in \freenames{P} }
\end{mathpar}

\begin{definition}
Then two processes, $P,Q$, are alpha-equivalent if $P = Q\{\vec{y}/\vec{x}\}$ for
some $\vec{x} \in \boundnames{Q},\vec{y} \in \boundnames{P}$, where $Q\{\vec{y}/\vec{x}\}$
denotes the capture-avoiding substitution of $\vec{y}$ for $\vec{x}$ in $Q$.
\end{definition}

\begin{definition}
  The {\em structural congruence} \cite{SangiorgiWalker} , $\equiv$,
  between processes is the least congruence containing
  alpha-equivalence, satisfying the abelian monoid laws
  (associativity, commutativity and $\pzero$ as identity) for parallel
  composition $|$ and for summation $+$.
\end{definition}

\subsection{Name equivalence}

We take name equivalence, written $\nameeq$, to be the smallest
equivalence relation generated by the following rules.

\begin{mathpar}
\inferrule*[lab=Quote-drop]
{ }
{ \quotep{@{x}} \nameeq x }

\inferrule*[lab=Struct-equiv]
{ P \scong Q }
{ \quotep{P} \nameeq \quotep{Q} }
\end{mathpar}

The astute reader will have noticed that the mutual recursion of names
and processes imposes a mutual recursion on alpha-equivalence and
structural equivalence via name-equivalence. Fortunately, all of this
works out pleasantly and we may calculate in the natural way, free of
concern. The reader interested in the details is referred to the
appendix \ref{appendix:rho_details}.

\subsection{Substitution}

We use $\Proc$ for the set of processes, $\QProc$ for the set of
names, and $\id{\{}\vec{y} / \vec{x} \id{\}}$ to denote partial maps,
$s : \QProc \rightarrow \QProc$. A map, $s$ lifts, uniquely, to a map
on process terms, $\widehat{s} : \Proc \rightarrow \Proc$ by the
following equations.

\begin{mathpar}
  (0) \psubstp{Q}{P} := 0 \\
  (R \juxtap S) \psubstp{Q}{P}
  :=    
  (R)\psubstp{Q}{P} \juxtap (S) \psubstp{Q}{P} \\
  (x?(y).R) \psubstp{Q}{P}    
  :=    
  (x)\substp{Q}{P} (z)\concat( (R \psubstn{z}{y}) \psubstp{Q}{P} ) \\
  (\lift{x}{R}) \psubstp{Q}{P}  
  :=
  \lift{(x)\substp{Q}{P}}{ R \psubstp{Q}{P} } \\
%   (\dropn{x})  \psubstp{Q}{P}       
%   := 
%   \left\{ 
%     \begin{array}{ccc} 
%       \dropn{\quotep{Q}} & & x \nameeq \quotep{P} \\
%       \dropn{x} & & otherwise \\
%     \end{array}
%   \right. 
  (\dropn{x})  \psubstp{Q}{P}       
  := 
  \left\{ 
    \begin{array}{ccc} 
      Q & & x \nameeq \quotep{P} \\
      \dropn{x} & & otherwise \\
    \end{array}
  \right.
\end{mathpar}
 

where

\begin{eqnarray}
  (x)\id{\{} \lpquote Q \rpquote / \lpquote P \rpquote \id{\}}            = 
  \left\{ 
    \begin{array}{ccc}
      \lpquote Q \rpquote & & x \nameeq \lpquote P \rpquote \\
      x & & otherwise \\
    \end{array}
  \right. \nonumber
\end{eqnarray}

and $z$ is chosen distinct from $\quotep{P}$, $\quotep{Q}$, the free
names in $Q$, and all the names in $R$. Our $\alpha$-equivalence will
be built in the standard way from this substitution.

\begin{remark}\label{rem:no_self_referential_names}
  One consequence of these definitions is that $\forall P. \quotep{P}
  \not\in \freenames{P}$.
\end{remark}

\subsection{ Dynamic quote: an example }

Anticipating something of what's to come, consider applying the
substitution, $\widehat{\id{\{}u / z \id{\}}}$, to the following pair
of processes, $\lift{w}{y!(z)}$ and $w[ \lpquote y!(z) \rpquote ]$.

\begin{eqnarray}
	\lift{w}{y!(z)}\widehat{\id{\{}u / z \id{\}}}
		& = &
		\lift{w}{y!(u)} \nonumber\\
	w[ \lpquote y!(z) \rpquote ] \widehat{ \id{\{}u / z \id{\}} }
		& = &
		w[ \lpquote y!(z) \rpquote ] \nonumber
\end{eqnarray}

Because the body of the process between quotes is impervious to
substitution, we get radically different answers. In fact, by
examining the first process in an input context,
e.g. $x?(z).\lift{w}{y!(z)}$, we see that the process under the lift
operator may be shaped by prefixed inputs binding a name inside it. In
this sense, the lift operator will be seen as a way to dynamically
construct processes before reifying them as names.

Finally equipped with these standard features we can present the
dynamics of the calculus.

\subsubsection{Operational semantics} 

Finally, we introduce the computational dynamics. What marks these
algebras as distinct from other more traditionally studied algebraic
structures, e.g. vector spaces or polynomial rings, is the manner in
which dynamics is captured. In traditional structures, dynamics is typically
expressed through morphisms between such structures, as in linear maps
between vector spaces or morphisms between rings. In algebras
associated with the semantics of computation, the dynamics is
expressed as part of the algebraic structure itself, through a
reduction reduction relation typically denoted by $\red$. Below, we
give a recursive presentation of this relation for the calculus used
in the encoding.

$\red \subseteq \pi \times \pi$
$\red : \pi \to \mathcal{P}(\pi)$

\begin{mathpar}
  \inferrule* [lab=Comm] { \textsf{match}( x_{src}, x_{trgt} ) } { x_{trgt}?(y)P \; | \; x_{src}!\langle {Q} \rangle \red P\{\quotep{Q}/y}\} }
  \and \\
  \inferrule* [lab=Par] {{P} \red {P}'} {{{P} | {Q}} \red {{P}' | {Q}}}
  \and
  \inferrule* [lab=Equiv]{{{P} \scong {P}'} \andalso {{P}' \red {Q}'} \andalso {{Q}' \scong {Q}}}{{P} \red {Q}}
\end{mathpar}

\begin{eqnarray*}
  match_{\equiv} (\quotep{P},\quotep{Q}) & := & P \equiv Q \\
  match_{\dagger}(\quotep{P},\quotep{Q}) & := & \forall R. P|Q \red^{*} R => R \red^{*} 0 \\
  match_{K}(\quotep{P},\quotep{Q}) & := & K \mbox{ for some context } K
\end{eqnarray*}

$u?(x)P | u!\langle Q \rangle \red P\{\quotep{Q}/x\}$

%We write $\wred$ for $\red^*$, and $P\red$ if $\exists Q $ such that $ P \red Q$.
We write $P\red$ if $\exists Q $ such that $ P \red Q$ and $P\not\red$, otherwise.

\section{Replication}

As mentioned before, it is known that replication (and hence
recursion) can be implemented in a higher-order process algebra
\cite{SangiorgiWalker}. As our first example of calculation with the
machinery thus far presented we give the construction explicitly in
the {\rhoc}.

\begin{eqnarray}
	D_{x} & := & \prefix{x}{y}{(\binpar{\outputp{x}{y}}{@{y}})} \nonumber\\
	\bangp_{x}{P} & := & \binpar{{x}!\langle{\binpar{D_{x}}{P}}\rangle}{D_{x}} \nonumber
\end{eqnarray}

\begin{eqnarray}
	\bangp_{x}{P} & & \nonumber\\
	=
	& {x}!\langle{(\prefix{x}{y}{(\outputp{x}{y} | @{y})) | P}}\rangle 
	      | \prefix{x}{y}{(\outputp{x}{y} | @{y})} & \nonumber\\
	\red
	& (\outputp{x}{y} | @{y})\substn{\quotep{(\prefix{x}{y}{(@{y} | \outputp{x}{y})) | P}}}{y} & \nonumber\\
	=
	& \outputp{x}{\quotep{(\prefix{x}{y}{(\outputp{x}{y} | @{y})) | P}}}
	  | {(\prefix{x}{y}{(\outputp{x}{y} | @{y})) | P}} & \nonumber\\
	\red
	& \ldots & \nonumber\\
	\red^*
	& P | P | \ldots & \nonumber
\end{eqnarray}

Of course, this encoding, as an implementation, runs away, unfolding
$\bangp{P}$ eagerly. A lazier and more implementable replication
operator, restricted to input-guarded processes, may be obtained as follows.

\begin{eqnarray}
\bangp{\prefix{u}{v}{P}} 
	:= 
	\binpar{\lift{x}{\prefix{u}{v}{(\binpar{D(x)}{P})}}}{D(x)} \nonumber
\end{eqnarray}

\begin{remark}
  Note that the lazier definition still does not deal with summation
  or mixed summation (i.e. sums over input and output). The reader is
  invited to construct definitions of replication that deal with these
  features. 

  Further, the definitions are parameterized in a name, $x$. Can you,
  gentle reader, make a definition that eliminates this parameter and
  guarantees no accidental interaction between the replication
  machinery and the process being replicated -- i.e. no accidental
  sharing of names used by the process to get its work done and the
  name(s) used by the replication to effect copying. This latter
  revision of the definition of replication is crucial to obtaining
  the expected identity $!!P \sim !P$.
\end{remark}

\begin{remark}\label{rem:paradoxical_combinator}
  The reader familiar with the lambda calculus will have noticed the
  similarity between $D$ and the paradoxical combinator.

  [Ed. note: the existence of this seems to suggest we have to be more
  restrictive on the set of processes and names we admit if we are to
  support no-cloning.]
\end{remark}

\subsubsection{Bisimulation}

The computational dynamics gives rise to another kind of equivalence,
the equivalence of computational behavior. As previously mentioned
this is typically captured \emph{via} some form of bisimulation.

% The notion we use in this paper is weak barbed bisimulation
% \cite{milner91polyadicpi}.

The notion we use in this paper is derived from weak barbed
bisimulation \cite{milner91polyadicpi}. 

\begin{definition}
An \emph{observation relation}, $\downarrow_{\mathcal N}$, over a set
of names, $\mathcal N$, is the smallest relation satisfying the rules
below.

\infrule[Out-barb]{y \in {\mathcal N}, \; x \nameeq y}
		  {\outputp{x}{v} \downarrow_{\mathcal N} x}
\infrule[Par-barb]{\mbox{$P\downarrow_{\mathcal N} x$ or $Q\downarrow_{\mathcal N} x$}}
		  {\binpar{P}{Q} \downarrow_{\mathcal N} x}

We write $P \Downarrow_{\mathcal N} x$ if there is $Q$ such that 
$P \wred Q$ and $Q \downarrow_{\mathcal N} x$.
\end{definition}

\begin{definition}
%\label{def.bbisim}
An  ${\mathcal N}$-\emph{barbed bisimulation} over a set of names, ${\mathcal N}$, is a symmetric binary relation 
${\mathcal S}_{\mathcal N}$ between agents such that $P\rel{S}_{\mathcal N}Q$ implies:
\begin{enumerate}
\item If $P \red P'$ then $Q \wred Q'$ and $P'\rel{S}_{\mathcal N} Q'$.
\item If $P\downarrow_{\mathcal N} x$, then $Q\Downarrow_{\mathcal N} x$.
\end{enumerate}
$P$ is ${\mathcal N}$-barbed bisimilar to $Q$, written
$P \wbbisim_{\mathcal N} Q$, if $P \rel{S}_{\mathcal N} Q$ for some ${\mathcal N}$-barbed bisimulation ${\mathcal S}_{\mathcal N}$.
\end{definition}

$\mathcal{R} \subseteq \pi \times \pi$

$P \mathcal{R} Q => \forall P'. P \red P' \Rightarrow \exists Q'. Q \red Q', P' \mathcal{R} Q'$

$P \vdash x \Rightarrow Q \vdash x$

\begin{mathpar}
  \inferrule*[lab=Out-barb]{x \nameeq y}{{y}!\langle{Q}\rangle \vdash x}
  \and
  \inferrule*[lab=Par-barb]{\mbox{$P\vdash x$ or $Q\vdash x$}}{\binpar{P}{Q} \vdash x}
\end{mathpar}

\subsubsection{Contexts}

One of the principle advantages of computational calculi like the
$\pi$-calculus is a well-defined notion of context,
contextual-equivalence and a correlation between
contextual-equivalence and notions of bisimulation. The notion of
context allows the decomposition of a process into (sub-)process and
its syntactic environment, its context. Thus, a context may be
thought of as a process with a ``hole'' (written $\Box$) in it. The
application of a context $M$ to a process $P$, written $M[P]$, is
tantamount to filling the hole in $M$ with $P$. In this paper we do
not need the full weight of this theory, but do make use of the notion
of context in the proof the main theorem. 

\begin{mathpar}
  \inferrule* [lab=summation] {} {{M_{M},M_{N}} \bc \Box \;|\; x.M_{A} \;|\; M_{M}+M_{N}}
  \and
  \inferrule* [lab=agent] {} {{M_{A}} \bc (\vec{x})M_{P} \;| \; \clift{P_0,\ldots,M_{P},\ldots,P_N}}
  \and \\
  \inferrule* [lab=process] {} {{M_{P}} \bc M_{N} \;| \;P|M_{P} }
\end{mathpar} 

\begin{mathpar}
  \inferrule* [lab=sychronization] {} {M_{N} \bc \Box \;|\; x?M_{F} \;|\; x!M_{C}}
  \and
  \inferrule* [lab=abstraction] {} {{M_{F}} \bc (x)M_{P} }
  \and
  \inferrule* [lab=concretion] {} {{M_{C}} \bc \langle M_{P} \rangle }
  \and \\
  \inferrule* [lab=process] {} {{M_{P}} \bc M_{N} \;| \;P|M_{P} }
\end{mathpar}

\begin{definition}[contextual application] Given a context $M$, and
  process $P$, we define the \emph{contextual application}, $M[P] :=
  M\{P/\Box\}$. That is, the contextual application of M to P is the
  substitution of $P$ for $\Box$ in $M$.
\end{definition}

$\meaningof{-} : L \to \mathcal{P}(\pi)$

\begin{mathpar}
  \inferrule* [lab=collection] {} {\meaningof{true} = \pi, \and \meaningof{~E} = \pi \setminus \meaningof{E}, \and \meaningof{E_{1} \& E_{2}} = \meaningof{E_{1}} \cap \meaningof{E_{2}}}
\end{mathpar}

\begin{mathpar}
  \inferrule* [lab=structure] {} {\meaningof{0} = \{ P \in \pi | P \equiv 0 \}, \and \\ \meaningof{E_1 | E_2} = \{ P \in \pi | P \equiv P_{1} | P_{2}, P_{1} \in \meaningof{E_{1}}, P_{2} \in \meaningof{E_2}\} }
\end{mathpar}

\begin{mathpar}
 \inferrule* [lab=behavior] {} {\meaningof{\langle a?b \rangle E} = \{ P \in \pi | P \equiv Q | u?(y)P', \\ \and \\\\ \and \\ \;\;\; u \in \meaningof{a}, \forall z.P'\{z/y\} \in \meaningof{E\{z/b\}}\}, \and \\ \meaningof{a!E} = \{ P \in \pi | P \equiv Q | x!\langle P' \rangle, x \in \meaningof{a} P' \in \meaningof{E}\} }
\end{mathpar}

\begin{mathpar}
 \inferrule* [lab=nominal] {} {\meaningof{\quotep{E}} = \{ \quotep{P} \in \quotep{\pi} | P \in \meaningof{E} \}, \and \meaningof{\quotep{P}} = \{ \quotep{Q} \in \quotep{\pi} | P \equiv Q \} \and \\ \meaningof{@\quotep{E}} = \{ P \in \pi | P \equiv @x, x \in \meaningof{E} \}}
\end{mathpar}

\begin{eqnarray*}
  \\
  \meaningof{-} : TS \to ST
\end{eqnarray*}

\begin{eqnarray*}
  \\
  L : TS \to ST
\end{eqnarray*}

\begin{eqnarray*}
  \\
  P \models E \iff P \in \meaningof{E}
\end{eqnarray*}

\begin{eqnarray*}
  P \approx_{L} Q \iff \forall E \in L. P \models E \iff Q \models E
\end{eqnarray*}

\begin{eqnarray*}
  P \approx_{K} Q
\end{eqnarray*}

\begin{eqnarray*}
  P \approx Q
\end{eqnarray*}

$\approx_{K} = \approx = \approx_{L}$

\subsubsection{Contextual duality}

Note that contexts extend the quotation operation to a family of
operations from processes to names. Given a context, $M$, we can
define a \emph{nominal context}, $\quotep{M}$ by $\quotep{M}[P] :=
\quotep{M[P]}$. To foreshadow what is to come we observe that these
operations enjoy a duality with processes very much like the duality
between vectors and maps from vectors to scalars.

Further, because the calculus is essentially higher-order, we have a
correspondence between contexts and processes. More specifically,
given a name $x$ and a context $M$ we can construct $M^{*}_{x}$ such
that 

\begin{mathpar}
  M^{*}_{x} | \lift{x}{P} \red M[P]
\end{mathpar}

namely,

\begin{mathpar}
  M^{*}_{x} := x?(u).M[\dropn{u}]
\end{mathpar}

The dependence of $M^{*}_{x}$ on a name makes it an abstraction, 

\begin{mathpar}
  M^{*} := (x)x?(u).M[\dropn{u}]
\end{mathpar}

\subsection{Additional notation}

It will sometimes be convenient to denote the process a name
quotes. We already have the notation $x = \quotep{P}$, but it will be
convenient to introduce an alternate notation, $\procn{x}$, when we
want to emphasize the connection to the use of the name. Note that, by
virtue of name equivalence, $\quotep{\procn{x}} \nameeq x$; so, the
notation is consistent with previous definitions.

Further, because names have structure it is possible to effect
substitutions on the basis of that structure. This means we need to
upgrade our notation for substitutions, which we accomplish by
adapting comprehension notation. Thus,

\begin{mathpar}
  P\{ y / x : x \in S \}
\end{mathpar}

is interpreted to mean the process derived from P by replacing (in a
capture-avoiding manner) each occurrence of $x$ in $S$ by $y$. For example,

\begin{mathpar}
  P\{ \quotep{\procn{x}|\procn{x}} / x : x \in \freenames{P} \}
\end{mathpar}

will replace each (occurrence) of a free name $x$ in $P$ by
$\quotep{\procn{x}|\procn{x}}$.

Also, we will avail ourselves of the notation $x^{L}$ and $x^{R}$ to
denote injections of a name into disjoint copies of the name
space. There are numerous ways to accomplish this. One example can be
found in \cite{MeredithR05}. This notation overloads to vectors of
names: $\vec{x}^{\pi} := (x_{i}^{\pi} \; : \; 0 \leq i < |\vec{x}| )$ where $\pi \in \{L,R\}$.

We also use $P^{\Box} := P|\Box$.

In \cite{MeredithR05} an interpretation of the new operator is
given. It turns out that there are several possible interpretations
all enjoying the requisite algebraic properties of the operator (see
\cite{milner91polyadicpi}). We will therefore make liberal use of
$(\nu\; \vec{x})P$.

% subsection the_syntax_and_semantics_of_the_notation_system (end)   

\input{qm2pi.qmops} 

\input{qm2pi.sterngerlach} 

\input{qm2pi.metric} 

% section concurrent_process_calculi (end)

%\input{qm2pi.proofsketch}

% section proof sketch (end)

%\input{qm2pi.slviaknots} 

% section spatial logic via knots (end)

\input{qm2pi.conclusion}

% section conclusion (end)

%\input{qm2pi.dtcodes} 

% section wiring algorithm (end)

\input{qm2pi.ack} 

% section acknowledgments (end)

\newpage


\bibliographystyle{plain}   
\bibliography{../../biblios/main.bib}

\input{qm2pi.rhodetails}

\end{document}

 

% section notation (end)

\input{qm2pi.process.calculi} 

% section concurrent_process_calculi_and_spatial_logics_ (end)
    
%\documentclass[12pt]{llncs}
%\documentclass{jktr}

\usepackage[pdftex]{hyperref}                   
\usepackage {listings}
\usepackage {mathpartir}
\usepackage{bcprules}
%\usepackage{listings}
                       
\usepackage{graphicx} 
%\usepackage[margins=2.5cm,nohead,nofoot]{geometry}
%\usepackage{geometry}
\usepackage{amsfonts}
\usepackage{amstext}
\usepackage{latexsym}
\usepackage{amssymb}
\usepackage{color}


%\include{myPreamble}
\include{qm2pi.local} 

%\ifpdf
%\usepackage[pdftex]{graphicx}
%\else
%\usepackage{graphicx}
%\fi

 % \ifpdf
%  \usepackage{pdfsync}
%  \if


%\title{Brief Article}
%\author{David F. Snyder}
%\author{L.G. Meredith}

%\address{Dept. of Math., Texas State University--San Marcos, San Marcos, TX 78666}
       
\pagestyle{empty}


\begin{document}

\lstset{language=[Objective]Caml,frame=shadowbox}

\input{qm2pi.front}

% section front matter (end)

\input{qm2pi.intro} 
 
% section introduction (end)

% \input{qm2pi.knotations} 

% section notation (end)

\input{qm2pi.process.calculi} 

% section concurrent_process_calculi_and_spatial_logics_ (end)
    
%\input{qm2pi.knots2pi} 

%\input{qm2pi.trefoil} 

%\input{qm2pi.mainthm} 

% subsection basic_interpretation (end)

%\input{qm2pi.rho.presentation} 
\subsection{The syntax and semantics of the notation system}\label{sub:the_syntax_and_semantics_of_the_notation_system} % (fold)

We now summarize a technical presentation of the calculus that
embodies our theory of dynamics. The typical presentation of such a
calculus follows the style of giving generators and relations on
them. The grammar, below, describing term constructors, freely
generates the set of processes, $\Proc$. This set is then quotiented
by a relation known as structural congruence and it is over this set
that the notion of dynamics is expressed. This presentation is
essentially that of \cite{MeredithR05} with the addition of
polyadicity and summation. For readability we have relegated some of
the technical subtleties to an appendix.

\subsubsection{Process grammar}\label{subsub:process_grammar}

\begin{mathpar}
  \inferrule* [lab=synchronization] {} {{M} \bc \pzero \;|\; x?F \;|\; x!C }
  \and
  \inferrule* [lab=abstraction] {} {{F} \bc (x)P}
  \and
  \inferrule* [lab=concretion] {} {{C} \bc \langle Q \rangle}
  \and
  \inferrule* [lab=process] {} {{P,Q} \bc M \;| \;P|Q \;|\; @{x}}
  \and
  \inferrule* [lab=name] {} {{x} \bc \quotep{P}}
\end{mathpar} 

Note that $\vec{x}$ (resp. $\vec{P}$) denotes a vector of names
(resp. processes) of length $|\vec{x}|$ (resp. $|\vec{P}|$). We adopt
the following useful abbreviations.

\begin{mathpar}
   x?(\vec{y}).P := x.(\vec{y})P \and  x\clift{\vec{P}} := x.\clift{\vec{P}}
   \and x!(y) := \lift{x}{\dropn{y}}
   \and \Pi_{i=0}^{n-1}P_i := P_0 | \ldots | P_{n-1}
\end{mathpar}

\subsubsection{Structural congruence}

\paragraph{Free and bound names and alpha-equivalence.} At the
core of structural equivalence is alpha-equivalence which identifies
process that are the same up to a change of variable. Formally, we
recognize the distinction between free and bound names. The free names
of a process, $\freenames{P}$, may be calculated recursively as
follows:

\begin{mathpar}
\freenames{\pzero} := \emptyset
  \and \\
  \freenames{x?(y).P} := \{ x \} \cup (\freenames{P} \setminus \{ y \})
  \and 
  \freenames{x!\langle P \rangle} := \{ x \} \cup \{ P \} 
  \and \\
  \freenames{P|Q} := \freenames{P} \cup \freenames{Q}
  \and \\
  \freenames{@{x}} := \{ x \}
\end{mathpar}

$\pi$
$\quotep{\pi}$

$\freenames{-} : \pi \to \mathcal{P}(\quotep{\pi})$

\begin{eqnarray*}
  \freenames{\pzero} & := & \emptyset \\
  \freenames{x?(y).P} & := & \{ x \} \cup (\freenames{P} \setminus \{ y \}) \\
  \freenames{x!\langle P \rangle} & := & \{ x \} \cup \{ P \} \\
  \freenames{P|Q} & := & \freenames{P} \cup \freenames{Q} \\
  \freenames{\dropn{x}} & := & \{ x \}
\end{eqnarray*}

The bound names of a process, $\boundnames{P}$, are those names occurring in $P$
that are not free. For example, in $x?(y).0$, the name $x$ is free, while $y$ is bound.

\begin{mathpar}
  \inferrule* [lab=monoidal-laws] {} { P|Q \equiv Q|P \and P|0 \equiv P \and P|(Q|R) \equiv (P|Q)|R }
\end{mathpar}

\begin{mathpar}
  \inferrule* [lab=alpha-equivalence] {} { (x)P \equiv (y)P\{y/x\} \and y \not\in \freenames{P} }
\end{mathpar}

\begin{definition}
Then two processes, $P,Q$, are alpha-equivalent if $P = Q\{\vec{y}/\vec{x}\}$ for
some $\vec{x} \in \boundnames{Q},\vec{y} \in \boundnames{P}$, where $Q\{\vec{y}/\vec{x}\}$
denotes the capture-avoiding substitution of $\vec{y}$ for $\vec{x}$ in $Q$.
\end{definition}

\begin{definition}
  The {\em structural congruence} \cite{SangiorgiWalker} , $\equiv$,
  between processes is the least congruence containing
  alpha-equivalence, satisfying the abelian monoid laws
  (associativity, commutativity and $\pzero$ as identity) for parallel
  composition $|$ and for summation $+$.
\end{definition}

\subsection{Name equivalence}

We take name equivalence, written $\nameeq$, to be the smallest
equivalence relation generated by the following rules.

\begin{mathpar}
\inferrule*[lab=Quote-drop]
{ }
{ \quotep{@{x}} \nameeq x }

\inferrule*[lab=Struct-equiv]
{ P \scong Q }
{ \quotep{P} \nameeq \quotep{Q} }
\end{mathpar}

The astute reader will have noticed that the mutual recursion of names
and processes imposes a mutual recursion on alpha-equivalence and
structural equivalence via name-equivalence. Fortunately, all of this
works out pleasantly and we may calculate in the natural way, free of
concern. The reader interested in the details is referred to the
appendix \ref{appendix:rho_details}.

\subsection{Substitution}

We use $\Proc$ for the set of processes, $\QProc$ for the set of
names, and $\id{\{}\vec{y} / \vec{x} \id{\}}$ to denote partial maps,
$s : \QProc \rightarrow \QProc$. A map, $s$ lifts, uniquely, to a map
on process terms, $\widehat{s} : \Proc \rightarrow \Proc$ by the
following equations.

\begin{mathpar}
  (0) \psubstp{Q}{P} := 0 \\
  (R \juxtap S) \psubstp{Q}{P}
  :=    
  (R)\psubstp{Q}{P} \juxtap (S) \psubstp{Q}{P} \\
  (x?(y).R) \psubstp{Q}{P}    
  :=    
  (x)\substp{Q}{P} (z)\concat( (R \psubstn{z}{y}) \psubstp{Q}{P} ) \\
  (\lift{x}{R}) \psubstp{Q}{P}  
  :=
  \lift{(x)\substp{Q}{P}}{ R \psubstp{Q}{P} } \\
%   (\dropn{x})  \psubstp{Q}{P}       
%   := 
%   \left\{ 
%     \begin{array}{ccc} 
%       \dropn{\quotep{Q}} & & x \nameeq \quotep{P} \\
%       \dropn{x} & & otherwise \\
%     \end{array}
%   \right. 
  (\dropn{x})  \psubstp{Q}{P}       
  := 
  \left\{ 
    \begin{array}{ccc} 
      Q & & x \nameeq \quotep{P} \\
      \dropn{x} & & otherwise \\
    \end{array}
  \right.
\end{mathpar}
 

where

\begin{eqnarray}
  (x)\id{\{} \lpquote Q \rpquote / \lpquote P \rpquote \id{\}}            = 
  \left\{ 
    \begin{array}{ccc}
      \lpquote Q \rpquote & & x \nameeq \lpquote P \rpquote \\
      x & & otherwise \\
    \end{array}
  \right. \nonumber
\end{eqnarray}

and $z$ is chosen distinct from $\quotep{P}$, $\quotep{Q}$, the free
names in $Q$, and all the names in $R$. Our $\alpha$-equivalence will
be built in the standard way from this substitution.

\begin{remark}\label{rem:no_self_referential_names}
  One consequence of these definitions is that $\forall P. \quotep{P}
  \not\in \freenames{P}$.
\end{remark}

\subsection{ Dynamic quote: an example }

Anticipating something of what's to come, consider applying the
substitution, $\widehat{\id{\{}u / z \id{\}}}$, to the following pair
of processes, $\lift{w}{y!(z)}$ and $w[ \lpquote y!(z) \rpquote ]$.

\begin{eqnarray}
	\lift{w}{y!(z)}\widehat{\id{\{}u / z \id{\}}}
		& = &
		\lift{w}{y!(u)} \nonumber\\
	w[ \lpquote y!(z) \rpquote ] \widehat{ \id{\{}u / z \id{\}} }
		& = &
		w[ \lpquote y!(z) \rpquote ] \nonumber
\end{eqnarray}

Because the body of the process between quotes is impervious to
substitution, we get radically different answers. In fact, by
examining the first process in an input context,
e.g. $x?(z).\lift{w}{y!(z)}$, we see that the process under the lift
operator may be shaped by prefixed inputs binding a name inside it. In
this sense, the lift operator will be seen as a way to dynamically
construct processes before reifying them as names.

Finally equipped with these standard features we can present the
dynamics of the calculus.

\subsubsection{Operational semantics} 

Finally, we introduce the computational dynamics. What marks these
algebras as distinct from other more traditionally studied algebraic
structures, e.g. vector spaces or polynomial rings, is the manner in
which dynamics is captured. In traditional structures, dynamics is typically
expressed through morphisms between such structures, as in linear maps
between vector spaces or morphisms between rings. In algebras
associated with the semantics of computation, the dynamics is
expressed as part of the algebraic structure itself, through a
reduction reduction relation typically denoted by $\red$. Below, we
give a recursive presentation of this relation for the calculus used
in the encoding.

$\red \subseteq \pi \times \pi$
$\red : \pi \to \mathcal{P}(\pi)$

\begin{mathpar}
  \inferrule* [lab=Comm] { \textsf{match}( x_{src}, x_{trgt} ) } { x_{trgt}?(y)P \; | \; x_{src}!\langle {Q} \rangle \red P\{\quotep{Q}/y}\} }
  \and \\
  \inferrule* [lab=Par] {{P} \red {P}'} {{{P} | {Q}} \red {{P}' | {Q}}}
  \and
  \inferrule* [lab=Equiv]{{{P} \scong {P}'} \andalso {{P}' \red {Q}'} \andalso {{Q}' \scong {Q}}}{{P} \red {Q}}
\end{mathpar}

\begin{eqnarray*}
  match_{\equiv} (\quotep{P},\quotep{Q}) & := & P \equiv Q \\
  match_{\dagger}(\quotep{P},\quotep{Q}) & := & \forall R. P|Q \red^{*} R => R \red^{*} 0 \\
  match_{K}(\quotep{P},\quotep{Q}) & := & K \mbox{ for some context } K
\end{eqnarray*}

$u?(x)P | u!\langle Q \rangle \red P\{\quotep{Q}/x\}$

%We write $\wred$ for $\red^*$, and $P\red$ if $\exists Q $ such that $ P \red Q$.
We write $P\red$ if $\exists Q $ such that $ P \red Q$ and $P\not\red$, otherwise.

\section{Replication}

As mentioned before, it is known that replication (and hence
recursion) can be implemented in a higher-order process algebra
\cite{SangiorgiWalker}. As our first example of calculation with the
machinery thus far presented we give the construction explicitly in
the {\rhoc}.

\begin{eqnarray}
	D_{x} & := & \prefix{x}{y}{(\binpar{\outputp{x}{y}}{@{y}})} \nonumber\\
	\bangp_{x}{P} & := & \binpar{{x}!\langle{\binpar{D_{x}}{P}}\rangle}{D_{x}} \nonumber
\end{eqnarray}

\begin{eqnarray}
	\bangp_{x}{P} & & \nonumber\\
	=
	& {x}!\langle{(\prefix{x}{y}{(\outputp{x}{y} | @{y})) | P}}\rangle 
	      | \prefix{x}{y}{(\outputp{x}{y} | @{y})} & \nonumber\\
	\red
	& (\outputp{x}{y} | @{y})\substn{\quotep{(\prefix{x}{y}{(@{y} | \outputp{x}{y})) | P}}}{y} & \nonumber\\
	=
	& \outputp{x}{\quotep{(\prefix{x}{y}{(\outputp{x}{y} | @{y})) | P}}}
	  | {(\prefix{x}{y}{(\outputp{x}{y} | @{y})) | P}} & \nonumber\\
	\red
	& \ldots & \nonumber\\
	\red^*
	& P | P | \ldots & \nonumber
\end{eqnarray}

Of course, this encoding, as an implementation, runs away, unfolding
$\bangp{P}$ eagerly. A lazier and more implementable replication
operator, restricted to input-guarded processes, may be obtained as follows.

\begin{eqnarray}
\bangp{\prefix{u}{v}{P}} 
	:= 
	\binpar{\lift{x}{\prefix{u}{v}{(\binpar{D(x)}{P})}}}{D(x)} \nonumber
\end{eqnarray}

\begin{remark}
  Note that the lazier definition still does not deal with summation
  or mixed summation (i.e. sums over input and output). The reader is
  invited to construct definitions of replication that deal with these
  features. 

  Further, the definitions are parameterized in a name, $x$. Can you,
  gentle reader, make a definition that eliminates this parameter and
  guarantees no accidental interaction between the replication
  machinery and the process being replicated -- i.e. no accidental
  sharing of names used by the process to get its work done and the
  name(s) used by the replication to effect copying. This latter
  revision of the definition of replication is crucial to obtaining
  the expected identity $!!P \sim !P$.
\end{remark}

\begin{remark}\label{rem:paradoxical_combinator}
  The reader familiar with the lambda calculus will have noticed the
  similarity between $D$ and the paradoxical combinator.

  [Ed. note: the existence of this seems to suggest we have to be more
  restrictive on the set of processes and names we admit if we are to
  support no-cloning.]
\end{remark}

\subsubsection{Bisimulation}

The computational dynamics gives rise to another kind of equivalence,
the equivalence of computational behavior. As previously mentioned
this is typically captured \emph{via} some form of bisimulation.

% The notion we use in this paper is weak barbed bisimulation
% \cite{milner91polyadicpi}.

The notion we use in this paper is derived from weak barbed
bisimulation \cite{milner91polyadicpi}. 

\begin{definition}
An \emph{observation relation}, $\downarrow_{\mathcal N}$, over a set
of names, $\mathcal N$, is the smallest relation satisfying the rules
below.

\infrule[Out-barb]{y \in {\mathcal N}, \; x \nameeq y}
		  {\outputp{x}{v} \downarrow_{\mathcal N} x}
\infrule[Par-barb]{\mbox{$P\downarrow_{\mathcal N} x$ or $Q\downarrow_{\mathcal N} x$}}
		  {\binpar{P}{Q} \downarrow_{\mathcal N} x}

We write $P \Downarrow_{\mathcal N} x$ if there is $Q$ such that 
$P \wred Q$ and $Q \downarrow_{\mathcal N} x$.
\end{definition}

\begin{definition}
%\label{def.bbisim}
An  ${\mathcal N}$-\emph{barbed bisimulation} over a set of names, ${\mathcal N}$, is a symmetric binary relation 
${\mathcal S}_{\mathcal N}$ between agents such that $P\rel{S}_{\mathcal N}Q$ implies:
\begin{enumerate}
\item If $P \red P'$ then $Q \wred Q'$ and $P'\rel{S}_{\mathcal N} Q'$.
\item If $P\downarrow_{\mathcal N} x$, then $Q\Downarrow_{\mathcal N} x$.
\end{enumerate}
$P$ is ${\mathcal N}$-barbed bisimilar to $Q$, written
$P \wbbisim_{\mathcal N} Q$, if $P \rel{S}_{\mathcal N} Q$ for some ${\mathcal N}$-barbed bisimulation ${\mathcal S}_{\mathcal N}$.
\end{definition}

$\mathcal{R} \subseteq \pi \times \pi$

$P \mathcal{R} Q => \forall P'. P \red P' \Rightarrow \exists Q'. Q \red Q', P' \mathcal{R} Q'$

$P \vdash x \Rightarrow Q \vdash x$

\begin{mathpar}
  \inferrule*[lab=Out-barb]{x \nameeq y}{{y}!\langle{Q}\rangle \vdash x}
  \and
  \inferrule*[lab=Par-barb]{\mbox{$P\vdash x$ or $Q\vdash x$}}{\binpar{P}{Q} \vdash x}
\end{mathpar}

\subsubsection{Contexts}

One of the principle advantages of computational calculi like the
$\pi$-calculus is a well-defined notion of context,
contextual-equivalence and a correlation between
contextual-equivalence and notions of bisimulation. The notion of
context allows the decomposition of a process into (sub-)process and
its syntactic environment, its context. Thus, a context may be
thought of as a process with a ``hole'' (written $\Box$) in it. The
application of a context $M$ to a process $P$, written $M[P]$, is
tantamount to filling the hole in $M$ with $P$. In this paper we do
not need the full weight of this theory, but do make use of the notion
of context in the proof the main theorem. 

\begin{mathpar}
  \inferrule* [lab=summation] {} {{M_{M},M_{N}} \bc \Box \;|\; x.M_{A} \;|\; M_{M}+M_{N}}
  \and
  \inferrule* [lab=agent] {} {{M_{A}} \bc (\vec{x})M_{P} \;| \; \clift{P_0,\ldots,M_{P},\ldots,P_N}}
  \and \\
  \inferrule* [lab=process] {} {{M_{P}} \bc M_{N} \;| \;P|M_{P} }
\end{mathpar} 

\begin{mathpar}
  \inferrule* [lab=sychronization] {} {M_{N} \bc \Box \;|\; x?M_{F} \;|\; x!M_{C}}
  \and
  \inferrule* [lab=abstraction] {} {{M_{F}} \bc (x)M_{P} }
  \and
  \inferrule* [lab=concretion] {} {{M_{C}} \bc \langle M_{P} \rangle }
  \and \\
  \inferrule* [lab=process] {} {{M_{P}} \bc M_{N} \;| \;P|M_{P} }
\end{mathpar}

\begin{definition}[contextual application] Given a context $M$, and
  process $P$, we define the \emph{contextual application}, $M[P] :=
  M\{P/\Box\}$. That is, the contextual application of M to P is the
  substitution of $P$ for $\Box$ in $M$.
\end{definition}

$\meaningof{-} : L \to \mathcal{P}(\pi)$

\begin{mathpar}
  \inferrule* [lab=collection] {} {\meaningof{true} = \pi, \and \meaningof{~E} = \pi \setminus \meaningof{E}, \and \meaningof{E_{1} \& E_{2}} = \meaningof{E_{1}} \cap \meaningof{E_{2}}}
\end{mathpar}

\begin{mathpar}
  \inferrule* [lab=structure] {} {\meaningof{0} = \{ P \in \pi | P \equiv 0 \}, \and \\ \meaningof{E_1 | E_2} = \{ P \in \pi | P \equiv P_{1} | P_{2}, P_{1} \in \meaningof{E_{1}}, P_{2} \in \meaningof{E_2}\} }
\end{mathpar}

\begin{mathpar}
 \inferrule* [lab=behavior] {} {\meaningof{\langle a?b \rangle E} = \{ P \in \pi | P \equiv Q | u?(y)P', \\ \and \\\\ \and \\ \;\;\; u \in \meaningof{a}, \forall z.P'\{z/y\} \in \meaningof{E\{z/b\}}\}, \and \\ \meaningof{a!E} = \{ P \in \pi | P \equiv Q | x!\langle P' \rangle, x \in \meaningof{a} P' \in \meaningof{E}\} }
\end{mathpar}

\begin{mathpar}
 \inferrule* [lab=nominal] {} {\meaningof{\quotep{E}} = \{ \quotep{P} \in \quotep{\pi} | P \in \meaningof{E} \}, \and \meaningof{\quotep{P}} = \{ \quotep{Q} \in \quotep{\pi} | P \equiv Q \} \and \\ \meaningof{@\quotep{E}} = \{ P \in \pi | P \equiv @x, x \in \meaningof{E} \}}
\end{mathpar}

\begin{eqnarray*}
  \\
  \meaningof{-} : TS \to ST
\end{eqnarray*}

\begin{eqnarray*}
  \\
  L : TS \to ST
\end{eqnarray*}

\begin{eqnarray*}
  \\
  P \models E \iff P \in \meaningof{E}
\end{eqnarray*}

\begin{eqnarray*}
  P \approx_{L} Q \iff \forall E \in L. P \models E \iff Q \models E
\end{eqnarray*}

\begin{eqnarray*}
  P \approx_{K} Q
\end{eqnarray*}

\begin{eqnarray*}
  P \approx Q
\end{eqnarray*}

$\approx_{K} = \approx = \approx_{L}$

\subsubsection{Contextual duality}

Note that contexts extend the quotation operation to a family of
operations from processes to names. Given a context, $M$, we can
define a \emph{nominal context}, $\quotep{M}$ by $\quotep{M}[P] :=
\quotep{M[P]}$. To foreshadow what is to come we observe that these
operations enjoy a duality with processes very much like the duality
between vectors and maps from vectors to scalars.

Further, because the calculus is essentially higher-order, we have a
correspondence between contexts and processes. More specifically,
given a name $x$ and a context $M$ we can construct $M^{*}_{x}$ such
that 

\begin{mathpar}
  M^{*}_{x} | \lift{x}{P} \red M[P]
\end{mathpar}

namely,

\begin{mathpar}
  M^{*}_{x} := x?(u).M[\dropn{u}]
\end{mathpar}

The dependence of $M^{*}_{x}$ on a name makes it an abstraction, 

\begin{mathpar}
  M^{*} := (x)x?(u).M[\dropn{u}]
\end{mathpar}

\subsection{Additional notation}

It will sometimes be convenient to denote the process a name
quotes. We already have the notation $x = \quotep{P}$, but it will be
convenient to introduce an alternate notation, $\procn{x}$, when we
want to emphasize the connection to the use of the name. Note that, by
virtue of name equivalence, $\quotep{\procn{x}} \nameeq x$; so, the
notation is consistent with previous definitions.

Further, because names have structure it is possible to effect
substitutions on the basis of that structure. This means we need to
upgrade our notation for substitutions, which we accomplish by
adapting comprehension notation. Thus,

\begin{mathpar}
  P\{ y / x : x \in S \}
\end{mathpar}

is interpreted to mean the process derived from P by replacing (in a
capture-avoiding manner) each occurrence of $x$ in $S$ by $y$. For example,

\begin{mathpar}
  P\{ \quotep{\procn{x}|\procn{x}} / x : x \in \freenames{P} \}
\end{mathpar}

will replace each (occurrence) of a free name $x$ in $P$ by
$\quotep{\procn{x}|\procn{x}}$.

Also, we will avail ourselves of the notation $x^{L}$ and $x^{R}$ to
denote injections of a name into disjoint copies of the name
space. There are numerous ways to accomplish this. One example can be
found in \cite{MeredithR05}. This notation overloads to vectors of
names: $\vec{x}^{\pi} := (x_{i}^{\pi} \; : \; 0 \leq i < |\vec{x}| )$ where $\pi \in \{L,R\}$.

We also use $P^{\Box} := P|\Box$.

In \cite{MeredithR05} an interpretation of the new operator is
given. It turns out that there are several possible interpretations
all enjoying the requisite algebraic properties of the operator (see
\cite{milner91polyadicpi}). We will therefore make liberal use of
$(\nu\; \vec{x})P$.

% subsection the_syntax_and_semantics_of_the_notation_system (end)   

\input{qm2pi.qmops} 

\input{qm2pi.sterngerlach} 

\input{qm2pi.metric} 

% section concurrent_process_calculi (end)

%\input{qm2pi.proofsketch}

% section proof sketch (end)

%\input{qm2pi.slviaknots} 

% section spatial logic via knots (end)

\input{qm2pi.conclusion}

% section conclusion (end)

%\input{qm2pi.dtcodes} 

% section wiring algorithm (end)

\input{qm2pi.ack} 

% section acknowledgments (end)

\newpage


\bibliographystyle{plain}   
\bibliography{../../biblios/main.bib}

\input{qm2pi.rhodetails}

\end{document}

 

%\documentclass[12pt]{llncs}
%\documentclass{jktr}

\usepackage[pdftex]{hyperref}                   
\usepackage {listings}
\usepackage {mathpartir}
\usepackage{bcprules}
%\usepackage{listings}
                       
\usepackage{graphicx} 
%\usepackage[margins=2.5cm,nohead,nofoot]{geometry}
%\usepackage{geometry}
\usepackage{amsfonts}
\usepackage{amstext}
\usepackage{latexsym}
\usepackage{amssymb}
\usepackage{color}


%\include{myPreamble}
\include{qm2pi.local} 

%\ifpdf
%\usepackage[pdftex]{graphicx}
%\else
%\usepackage{graphicx}
%\fi

 % \ifpdf
%  \usepackage{pdfsync}
%  \if


%\title{Brief Article}
%\author{David F. Snyder}
%\author{L.G. Meredith}

%\address{Dept. of Math., Texas State University--San Marcos, San Marcos, TX 78666}
       
\pagestyle{empty}


\begin{document}

\lstset{language=[Objective]Caml,frame=shadowbox}

\input{qm2pi.front}

% section front matter (end)

\input{qm2pi.intro} 
 
% section introduction (end)

% \input{qm2pi.knotations} 

% section notation (end)

\input{qm2pi.process.calculi} 

% section concurrent_process_calculi_and_spatial_logics_ (end)
    
%\input{qm2pi.knots2pi} 

%\input{qm2pi.trefoil} 

%\input{qm2pi.mainthm} 

% subsection basic_interpretation (end)

%\input{qm2pi.rho.presentation} 
\subsection{The syntax and semantics of the notation system}\label{sub:the_syntax_and_semantics_of_the_notation_system} % (fold)

We now summarize a technical presentation of the calculus that
embodies our theory of dynamics. The typical presentation of such a
calculus follows the style of giving generators and relations on
them. The grammar, below, describing term constructors, freely
generates the set of processes, $\Proc$. This set is then quotiented
by a relation known as structural congruence and it is over this set
that the notion of dynamics is expressed. This presentation is
essentially that of \cite{MeredithR05} with the addition of
polyadicity and summation. For readability we have relegated some of
the technical subtleties to an appendix.

\subsubsection{Process grammar}\label{subsub:process_grammar}

\begin{mathpar}
  \inferrule* [lab=synchronization] {} {{M} \bc \pzero \;|\; x?F \;|\; x!C }
  \and
  \inferrule* [lab=abstraction] {} {{F} \bc (x)P}
  \and
  \inferrule* [lab=concretion] {} {{C} \bc \langle Q \rangle}
  \and
  \inferrule* [lab=process] {} {{P,Q} \bc M \;| \;P|Q \;|\; @{x}}
  \and
  \inferrule* [lab=name] {} {{x} \bc \quotep{P}}
\end{mathpar} 

Note that $\vec{x}$ (resp. $\vec{P}$) denotes a vector of names
(resp. processes) of length $|\vec{x}|$ (resp. $|\vec{P}|$). We adopt
the following useful abbreviations.

\begin{mathpar}
   x?(\vec{y}).P := x.(\vec{y})P \and  x\clift{\vec{P}} := x.\clift{\vec{P}}
   \and x!(y) := \lift{x}{\dropn{y}}
   \and \Pi_{i=0}^{n-1}P_i := P_0 | \ldots | P_{n-1}
\end{mathpar}

\subsubsection{Structural congruence}

\paragraph{Free and bound names and alpha-equivalence.} At the
core of structural equivalence is alpha-equivalence which identifies
process that are the same up to a change of variable. Formally, we
recognize the distinction between free and bound names. The free names
of a process, $\freenames{P}$, may be calculated recursively as
follows:

\begin{mathpar}
\freenames{\pzero} := \emptyset
  \and \\
  \freenames{x?(y).P} := \{ x \} \cup (\freenames{P} \setminus \{ y \})
  \and 
  \freenames{x!\langle P \rangle} := \{ x \} \cup \{ P \} 
  \and \\
  \freenames{P|Q} := \freenames{P} \cup \freenames{Q}
  \and \\
  \freenames{@{x}} := \{ x \}
\end{mathpar}

$\pi$
$\quotep{\pi}$

$\freenames{-} : \pi \to \mathcal{P}(\quotep{\pi})$

\begin{eqnarray*}
  \freenames{\pzero} & := & \emptyset \\
  \freenames{x?(y).P} & := & \{ x \} \cup (\freenames{P} \setminus \{ y \}) \\
  \freenames{x!\langle P \rangle} & := & \{ x \} \cup \{ P \} \\
  \freenames{P|Q} & := & \freenames{P} \cup \freenames{Q} \\
  \freenames{\dropn{x}} & := & \{ x \}
\end{eqnarray*}

The bound names of a process, $\boundnames{P}$, are those names occurring in $P$
that are not free. For example, in $x?(y).0$, the name $x$ is free, while $y$ is bound.

\begin{mathpar}
  \inferrule* [lab=monoidal-laws] {} { P|Q \equiv Q|P \and P|0 \equiv P \and P|(Q|R) \equiv (P|Q)|R }
\end{mathpar}

\begin{mathpar}
  \inferrule* [lab=alpha-equivalence] {} { (x)P \equiv (y)P\{y/x\} \and y \not\in \freenames{P} }
\end{mathpar}

\begin{definition}
Then two processes, $P,Q$, are alpha-equivalent if $P = Q\{\vec{y}/\vec{x}\}$ for
some $\vec{x} \in \boundnames{Q},\vec{y} \in \boundnames{P}$, where $Q\{\vec{y}/\vec{x}\}$
denotes the capture-avoiding substitution of $\vec{y}$ for $\vec{x}$ in $Q$.
\end{definition}

\begin{definition}
  The {\em structural congruence} \cite{SangiorgiWalker} , $\equiv$,
  between processes is the least congruence containing
  alpha-equivalence, satisfying the abelian monoid laws
  (associativity, commutativity and $\pzero$ as identity) for parallel
  composition $|$ and for summation $+$.
\end{definition}

\subsection{Name equivalence}

We take name equivalence, written $\nameeq$, to be the smallest
equivalence relation generated by the following rules.

\begin{mathpar}
\inferrule*[lab=Quote-drop]
{ }
{ \quotep{@{x}} \nameeq x }

\inferrule*[lab=Struct-equiv]
{ P \scong Q }
{ \quotep{P} \nameeq \quotep{Q} }
\end{mathpar}

The astute reader will have noticed that the mutual recursion of names
and processes imposes a mutual recursion on alpha-equivalence and
structural equivalence via name-equivalence. Fortunately, all of this
works out pleasantly and we may calculate in the natural way, free of
concern. The reader interested in the details is referred to the
appendix \ref{appendix:rho_details}.

\subsection{Substitution}

We use $\Proc$ for the set of processes, $\QProc$ for the set of
names, and $\id{\{}\vec{y} / \vec{x} \id{\}}$ to denote partial maps,
$s : \QProc \rightarrow \QProc$. A map, $s$ lifts, uniquely, to a map
on process terms, $\widehat{s} : \Proc \rightarrow \Proc$ by the
following equations.

\begin{mathpar}
  (0) \psubstp{Q}{P} := 0 \\
  (R \juxtap S) \psubstp{Q}{P}
  :=    
  (R)\psubstp{Q}{P} \juxtap (S) \psubstp{Q}{P} \\
  (x?(y).R) \psubstp{Q}{P}    
  :=    
  (x)\substp{Q}{P} (z)\concat( (R \psubstn{z}{y}) \psubstp{Q}{P} ) \\
  (\lift{x}{R}) \psubstp{Q}{P}  
  :=
  \lift{(x)\substp{Q}{P}}{ R \psubstp{Q}{P} } \\
%   (\dropn{x})  \psubstp{Q}{P}       
%   := 
%   \left\{ 
%     \begin{array}{ccc} 
%       \dropn{\quotep{Q}} & & x \nameeq \quotep{P} \\
%       \dropn{x} & & otherwise \\
%     \end{array}
%   \right. 
  (\dropn{x})  \psubstp{Q}{P}       
  := 
  \left\{ 
    \begin{array}{ccc} 
      Q & & x \nameeq \quotep{P} \\
      \dropn{x} & & otherwise \\
    \end{array}
  \right.
\end{mathpar}
 

where

\begin{eqnarray}
  (x)\id{\{} \lpquote Q \rpquote / \lpquote P \rpquote \id{\}}            = 
  \left\{ 
    \begin{array}{ccc}
      \lpquote Q \rpquote & & x \nameeq \lpquote P \rpquote \\
      x & & otherwise \\
    \end{array}
  \right. \nonumber
\end{eqnarray}

and $z$ is chosen distinct from $\quotep{P}$, $\quotep{Q}$, the free
names in $Q$, and all the names in $R$. Our $\alpha$-equivalence will
be built in the standard way from this substitution.

\begin{remark}\label{rem:no_self_referential_names}
  One consequence of these definitions is that $\forall P. \quotep{P}
  \not\in \freenames{P}$.
\end{remark}

\subsection{ Dynamic quote: an example }

Anticipating something of what's to come, consider applying the
substitution, $\widehat{\id{\{}u / z \id{\}}}$, to the following pair
of processes, $\lift{w}{y!(z)}$ and $w[ \lpquote y!(z) \rpquote ]$.

\begin{eqnarray}
	\lift{w}{y!(z)}\widehat{\id{\{}u / z \id{\}}}
		& = &
		\lift{w}{y!(u)} \nonumber\\
	w[ \lpquote y!(z) \rpquote ] \widehat{ \id{\{}u / z \id{\}} }
		& = &
		w[ \lpquote y!(z) \rpquote ] \nonumber
\end{eqnarray}

Because the body of the process between quotes is impervious to
substitution, we get radically different answers. In fact, by
examining the first process in an input context,
e.g. $x?(z).\lift{w}{y!(z)}$, we see that the process under the lift
operator may be shaped by prefixed inputs binding a name inside it. In
this sense, the lift operator will be seen as a way to dynamically
construct processes before reifying them as names.

Finally equipped with these standard features we can present the
dynamics of the calculus.

\subsubsection{Operational semantics} 

Finally, we introduce the computational dynamics. What marks these
algebras as distinct from other more traditionally studied algebraic
structures, e.g. vector spaces or polynomial rings, is the manner in
which dynamics is captured. In traditional structures, dynamics is typically
expressed through morphisms between such structures, as in linear maps
between vector spaces or morphisms between rings. In algebras
associated with the semantics of computation, the dynamics is
expressed as part of the algebraic structure itself, through a
reduction reduction relation typically denoted by $\red$. Below, we
give a recursive presentation of this relation for the calculus used
in the encoding.

$\red \subseteq \pi \times \pi$
$\red : \pi \to \mathcal{P}(\pi)$

\begin{mathpar}
  \inferrule* [lab=Comm] { \textsf{match}( x_{src}, x_{trgt} ) } { x_{trgt}?(y)P \; | \; x_{src}!\langle {Q} \rangle \red P\{\quotep{Q}/y}\} }
  \and \\
  \inferrule* [lab=Par] {{P} \red {P}'} {{{P} | {Q}} \red {{P}' | {Q}}}
  \and
  \inferrule* [lab=Equiv]{{{P} \scong {P}'} \andalso {{P}' \red {Q}'} \andalso {{Q}' \scong {Q}}}{{P} \red {Q}}
\end{mathpar}

\begin{eqnarray*}
  match_{\equiv} (\quotep{P},\quotep{Q}) & := & P \equiv Q \\
  match_{\dagger}(\quotep{P},\quotep{Q}) & := & \forall R. P|Q \red^{*} R => R \red^{*} 0 \\
  match_{K}(\quotep{P},\quotep{Q}) & := & K \mbox{ for some context } K
\end{eqnarray*}

$u?(x)P | u!\langle Q \rangle \red P\{\quotep{Q}/x\}$

%We write $\wred$ for $\red^*$, and $P\red$ if $\exists Q $ such that $ P \red Q$.
We write $P\red$ if $\exists Q $ such that $ P \red Q$ and $P\not\red$, otherwise.

\section{Replication}

As mentioned before, it is known that replication (and hence
recursion) can be implemented in a higher-order process algebra
\cite{SangiorgiWalker}. As our first example of calculation with the
machinery thus far presented we give the construction explicitly in
the {\rhoc}.

\begin{eqnarray}
	D_{x} & := & \prefix{x}{y}{(\binpar{\outputp{x}{y}}{@{y}})} \nonumber\\
	\bangp_{x}{P} & := & \binpar{{x}!\langle{\binpar{D_{x}}{P}}\rangle}{D_{x}} \nonumber
\end{eqnarray}

\begin{eqnarray}
	\bangp_{x}{P} & & \nonumber\\
	=
	& {x}!\langle{(\prefix{x}{y}{(\outputp{x}{y} | @{y})) | P}}\rangle 
	      | \prefix{x}{y}{(\outputp{x}{y} | @{y})} & \nonumber\\
	\red
	& (\outputp{x}{y} | @{y})\substn{\quotep{(\prefix{x}{y}{(@{y} | \outputp{x}{y})) | P}}}{y} & \nonumber\\
	=
	& \outputp{x}{\quotep{(\prefix{x}{y}{(\outputp{x}{y} | @{y})) | P}}}
	  | {(\prefix{x}{y}{(\outputp{x}{y} | @{y})) | P}} & \nonumber\\
	\red
	& \ldots & \nonumber\\
	\red^*
	& P | P | \ldots & \nonumber
\end{eqnarray}

Of course, this encoding, as an implementation, runs away, unfolding
$\bangp{P}$ eagerly. A lazier and more implementable replication
operator, restricted to input-guarded processes, may be obtained as follows.

\begin{eqnarray}
\bangp{\prefix{u}{v}{P}} 
	:= 
	\binpar{\lift{x}{\prefix{u}{v}{(\binpar{D(x)}{P})}}}{D(x)} \nonumber
\end{eqnarray}

\begin{remark}
  Note that the lazier definition still does not deal with summation
  or mixed summation (i.e. sums over input and output). The reader is
  invited to construct definitions of replication that deal with these
  features. 

  Further, the definitions are parameterized in a name, $x$. Can you,
  gentle reader, make a definition that eliminates this parameter and
  guarantees no accidental interaction between the replication
  machinery and the process being replicated -- i.e. no accidental
  sharing of names used by the process to get its work done and the
  name(s) used by the replication to effect copying. This latter
  revision of the definition of replication is crucial to obtaining
  the expected identity $!!P \sim !P$.
\end{remark}

\begin{remark}\label{rem:paradoxical_combinator}
  The reader familiar with the lambda calculus will have noticed the
  similarity between $D$ and the paradoxical combinator.

  [Ed. note: the existence of this seems to suggest we have to be more
  restrictive on the set of processes and names we admit if we are to
  support no-cloning.]
\end{remark}

\subsubsection{Bisimulation}

The computational dynamics gives rise to another kind of equivalence,
the equivalence of computational behavior. As previously mentioned
this is typically captured \emph{via} some form of bisimulation.

% The notion we use in this paper is weak barbed bisimulation
% \cite{milner91polyadicpi}.

The notion we use in this paper is derived from weak barbed
bisimulation \cite{milner91polyadicpi}. 

\begin{definition}
An \emph{observation relation}, $\downarrow_{\mathcal N}$, over a set
of names, $\mathcal N$, is the smallest relation satisfying the rules
below.

\infrule[Out-barb]{y \in {\mathcal N}, \; x \nameeq y}
		  {\outputp{x}{v} \downarrow_{\mathcal N} x}
\infrule[Par-barb]{\mbox{$P\downarrow_{\mathcal N} x$ or $Q\downarrow_{\mathcal N} x$}}
		  {\binpar{P}{Q} \downarrow_{\mathcal N} x}

We write $P \Downarrow_{\mathcal N} x$ if there is $Q$ such that 
$P \wred Q$ and $Q \downarrow_{\mathcal N} x$.
\end{definition}

\begin{definition}
%\label{def.bbisim}
An  ${\mathcal N}$-\emph{barbed bisimulation} over a set of names, ${\mathcal N}$, is a symmetric binary relation 
${\mathcal S}_{\mathcal N}$ between agents such that $P\rel{S}_{\mathcal N}Q$ implies:
\begin{enumerate}
\item If $P \red P'$ then $Q \wred Q'$ and $P'\rel{S}_{\mathcal N} Q'$.
\item If $P\downarrow_{\mathcal N} x$, then $Q\Downarrow_{\mathcal N} x$.
\end{enumerate}
$P$ is ${\mathcal N}$-barbed bisimilar to $Q$, written
$P \wbbisim_{\mathcal N} Q$, if $P \rel{S}_{\mathcal N} Q$ for some ${\mathcal N}$-barbed bisimulation ${\mathcal S}_{\mathcal N}$.
\end{definition}

$\mathcal{R} \subseteq \pi \times \pi$

$P \mathcal{R} Q => \forall P'. P \red P' \Rightarrow \exists Q'. Q \red Q', P' \mathcal{R} Q'$

$P \vdash x \Rightarrow Q \vdash x$

\begin{mathpar}
  \inferrule*[lab=Out-barb]{x \nameeq y}{{y}!\langle{Q}\rangle \vdash x}
  \and
  \inferrule*[lab=Par-barb]{\mbox{$P\vdash x$ or $Q\vdash x$}}{\binpar{P}{Q} \vdash x}
\end{mathpar}

\subsubsection{Contexts}

One of the principle advantages of computational calculi like the
$\pi$-calculus is a well-defined notion of context,
contextual-equivalence and a correlation between
contextual-equivalence and notions of bisimulation. The notion of
context allows the decomposition of a process into (sub-)process and
its syntactic environment, its context. Thus, a context may be
thought of as a process with a ``hole'' (written $\Box$) in it. The
application of a context $M$ to a process $P$, written $M[P]$, is
tantamount to filling the hole in $M$ with $P$. In this paper we do
not need the full weight of this theory, but do make use of the notion
of context in the proof the main theorem. 

\begin{mathpar}
  \inferrule* [lab=summation] {} {{M_{M},M_{N}} \bc \Box \;|\; x.M_{A} \;|\; M_{M}+M_{N}}
  \and
  \inferrule* [lab=agent] {} {{M_{A}} \bc (\vec{x})M_{P} \;| \; \clift{P_0,\ldots,M_{P},\ldots,P_N}}
  \and \\
  \inferrule* [lab=process] {} {{M_{P}} \bc M_{N} \;| \;P|M_{P} }
\end{mathpar} 

\begin{mathpar}
  \inferrule* [lab=sychronization] {} {M_{N} \bc \Box \;|\; x?M_{F} \;|\; x!M_{C}}
  \and
  \inferrule* [lab=abstraction] {} {{M_{F}} \bc (x)M_{P} }
  \and
  \inferrule* [lab=concretion] {} {{M_{C}} \bc \langle M_{P} \rangle }
  \and \\
  \inferrule* [lab=process] {} {{M_{P}} \bc M_{N} \;| \;P|M_{P} }
\end{mathpar}

\begin{definition}[contextual application] Given a context $M$, and
  process $P$, we define the \emph{contextual application}, $M[P] :=
  M\{P/\Box\}$. That is, the contextual application of M to P is the
  substitution of $P$ for $\Box$ in $M$.
\end{definition}

$\meaningof{-} : L \to \mathcal{P}(\pi)$

\begin{mathpar}
  \inferrule* [lab=collection] {} {\meaningof{true} = \pi, \and \meaningof{~E} = \pi \setminus \meaningof{E}, \and \meaningof{E_{1} \& E_{2}} = \meaningof{E_{1}} \cap \meaningof{E_{2}}}
\end{mathpar}

\begin{mathpar}
  \inferrule* [lab=structure] {} {\meaningof{0} = \{ P \in \pi | P \equiv 0 \}, \and \\ \meaningof{E_1 | E_2} = \{ P \in \pi | P \equiv P_{1} | P_{2}, P_{1} \in \meaningof{E_{1}}, P_{2} \in \meaningof{E_2}\} }
\end{mathpar}

\begin{mathpar}
 \inferrule* [lab=behavior] {} {\meaningof{\langle a?b \rangle E} = \{ P \in \pi | P \equiv Q | u?(y)P', \\ \and \\\\ \and \\ \;\;\; u \in \meaningof{a}, \forall z.P'\{z/y\} \in \meaningof{E\{z/b\}}\}, \and \\ \meaningof{a!E} = \{ P \in \pi | P \equiv Q | x!\langle P' \rangle, x \in \meaningof{a} P' \in \meaningof{E}\} }
\end{mathpar}

\begin{mathpar}
 \inferrule* [lab=nominal] {} {\meaningof{\quotep{E}} = \{ \quotep{P} \in \quotep{\pi} | P \in \meaningof{E} \}, \and \meaningof{\quotep{P}} = \{ \quotep{Q} \in \quotep{\pi} | P \equiv Q \} \and \\ \meaningof{@\quotep{E}} = \{ P \in \pi | P \equiv @x, x \in \meaningof{E} \}}
\end{mathpar}

\begin{eqnarray*}
  \\
  \meaningof{-} : TS \to ST
\end{eqnarray*}

\begin{eqnarray*}
  \\
  L : TS \to ST
\end{eqnarray*}

\begin{eqnarray*}
  \\
  P \models E \iff P \in \meaningof{E}
\end{eqnarray*}

\begin{eqnarray*}
  P \approx_{L} Q \iff \forall E \in L. P \models E \iff Q \models E
\end{eqnarray*}

\begin{eqnarray*}
  P \approx_{K} Q
\end{eqnarray*}

\begin{eqnarray*}
  P \approx Q
\end{eqnarray*}

$\approx_{K} = \approx = \approx_{L}$

\subsubsection{Contextual duality}

Note that contexts extend the quotation operation to a family of
operations from processes to names. Given a context, $M$, we can
define a \emph{nominal context}, $\quotep{M}$ by $\quotep{M}[P] :=
\quotep{M[P]}$. To foreshadow what is to come we observe that these
operations enjoy a duality with processes very much like the duality
between vectors and maps from vectors to scalars.

Further, because the calculus is essentially higher-order, we have a
correspondence between contexts and processes. More specifically,
given a name $x$ and a context $M$ we can construct $M^{*}_{x}$ such
that 

\begin{mathpar}
  M^{*}_{x} | \lift{x}{P} \red M[P]
\end{mathpar}

namely,

\begin{mathpar}
  M^{*}_{x} := x?(u).M[\dropn{u}]
\end{mathpar}

The dependence of $M^{*}_{x}$ on a name makes it an abstraction, 

\begin{mathpar}
  M^{*} := (x)x?(u).M[\dropn{u}]
\end{mathpar}

\subsection{Additional notation}

It will sometimes be convenient to denote the process a name
quotes. We already have the notation $x = \quotep{P}$, but it will be
convenient to introduce an alternate notation, $\procn{x}$, when we
want to emphasize the connection to the use of the name. Note that, by
virtue of name equivalence, $\quotep{\procn{x}} \nameeq x$; so, the
notation is consistent with previous definitions.

Further, because names have structure it is possible to effect
substitutions on the basis of that structure. This means we need to
upgrade our notation for substitutions, which we accomplish by
adapting comprehension notation. Thus,

\begin{mathpar}
  P\{ y / x : x \in S \}
\end{mathpar}

is interpreted to mean the process derived from P by replacing (in a
capture-avoiding manner) each occurrence of $x$ in $S$ by $y$. For example,

\begin{mathpar}
  P\{ \quotep{\procn{x}|\procn{x}} / x : x \in \freenames{P} \}
\end{mathpar}

will replace each (occurrence) of a free name $x$ in $P$ by
$\quotep{\procn{x}|\procn{x}}$.

Also, we will avail ourselves of the notation $x^{L}$ and $x^{R}$ to
denote injections of a name into disjoint copies of the name
space. There are numerous ways to accomplish this. One example can be
found in \cite{MeredithR05}. This notation overloads to vectors of
names: $\vec{x}^{\pi} := (x_{i}^{\pi} \; : \; 0 \leq i < |\vec{x}| )$ where $\pi \in \{L,R\}$.

We also use $P^{\Box} := P|\Box$.

In \cite{MeredithR05} an interpretation of the new operator is
given. It turns out that there are several possible interpretations
all enjoying the requisite algebraic properties of the operator (see
\cite{milner91polyadicpi}). We will therefore make liberal use of
$(\nu\; \vec{x})P$.

% subsection the_syntax_and_semantics_of_the_notation_system (end)   

\input{qm2pi.qmops} 

\input{qm2pi.sterngerlach} 

\input{qm2pi.metric} 

% section concurrent_process_calculi (end)

%\input{qm2pi.proofsketch}

% section proof sketch (end)

%\input{qm2pi.slviaknots} 

% section spatial logic via knots (end)

\input{qm2pi.conclusion}

% section conclusion (end)

%\input{qm2pi.dtcodes} 

% section wiring algorithm (end)

\input{qm2pi.ack} 

% section acknowledgments (end)

\newpage


\bibliographystyle{plain}   
\bibliography{../../biblios/main.bib}

\input{qm2pi.rhodetails}

\end{document}

 

%\documentclass[12pt]{llncs}
%\documentclass{jktr}

\usepackage[pdftex]{hyperref}                   
\usepackage {listings}
\usepackage {mathpartir}
\usepackage{bcprules}
%\usepackage{listings}
                       
\usepackage{graphicx} 
%\usepackage[margins=2.5cm,nohead,nofoot]{geometry}
%\usepackage{geometry}
\usepackage{amsfonts}
\usepackage{amstext}
\usepackage{latexsym}
\usepackage{amssymb}
\usepackage{color}


%\include{myPreamble}
\include{qm2pi.local} 

%\ifpdf
%\usepackage[pdftex]{graphicx}
%\else
%\usepackage{graphicx}
%\fi

 % \ifpdf
%  \usepackage{pdfsync}
%  \if


%\title{Brief Article}
%\author{David F. Snyder}
%\author{L.G. Meredith}

%\address{Dept. of Math., Texas State University--San Marcos, San Marcos, TX 78666}
       
\pagestyle{empty}


\begin{document}

\lstset{language=[Objective]Caml,frame=shadowbox}

\input{qm2pi.front}

% section front matter (end)

\input{qm2pi.intro} 
 
% section introduction (end)

% \input{qm2pi.knotations} 

% section notation (end)

\input{qm2pi.process.calculi} 

% section concurrent_process_calculi_and_spatial_logics_ (end)
    
%\input{qm2pi.knots2pi} 

%\input{qm2pi.trefoil} 

%\input{qm2pi.mainthm} 

% subsection basic_interpretation (end)

%\input{qm2pi.rho.presentation} 
\subsection{The syntax and semantics of the notation system}\label{sub:the_syntax_and_semantics_of_the_notation_system} % (fold)

We now summarize a technical presentation of the calculus that
embodies our theory of dynamics. The typical presentation of such a
calculus follows the style of giving generators and relations on
them. The grammar, below, describing term constructors, freely
generates the set of processes, $\Proc$. This set is then quotiented
by a relation known as structural congruence and it is over this set
that the notion of dynamics is expressed. This presentation is
essentially that of \cite{MeredithR05} with the addition of
polyadicity and summation. For readability we have relegated some of
the technical subtleties to an appendix.

\subsubsection{Process grammar}\label{subsub:process_grammar}

\begin{mathpar}
  \inferrule* [lab=synchronization] {} {{M} \bc \pzero \;|\; x?F \;|\; x!C }
  \and
  \inferrule* [lab=abstraction] {} {{F} \bc (x)P}
  \and
  \inferrule* [lab=concretion] {} {{C} \bc \langle Q \rangle}
  \and
  \inferrule* [lab=process] {} {{P,Q} \bc M \;| \;P|Q \;|\; @{x}}
  \and
  \inferrule* [lab=name] {} {{x} \bc \quotep{P}}
\end{mathpar} 

Note that $\vec{x}$ (resp. $\vec{P}$) denotes a vector of names
(resp. processes) of length $|\vec{x}|$ (resp. $|\vec{P}|$). We adopt
the following useful abbreviations.

\begin{mathpar}
   x?(\vec{y}).P := x.(\vec{y})P \and  x\clift{\vec{P}} := x.\clift{\vec{P}}
   \and x!(y) := \lift{x}{\dropn{y}}
   \and \Pi_{i=0}^{n-1}P_i := P_0 | \ldots | P_{n-1}
\end{mathpar}

\subsubsection{Structural congruence}

\paragraph{Free and bound names and alpha-equivalence.} At the
core of structural equivalence is alpha-equivalence which identifies
process that are the same up to a change of variable. Formally, we
recognize the distinction between free and bound names. The free names
of a process, $\freenames{P}$, may be calculated recursively as
follows:

\begin{mathpar}
\freenames{\pzero} := \emptyset
  \and \\
  \freenames{x?(y).P} := \{ x \} \cup (\freenames{P} \setminus \{ y \})
  \and 
  \freenames{x!\langle P \rangle} := \{ x \} \cup \{ P \} 
  \and \\
  \freenames{P|Q} := \freenames{P} \cup \freenames{Q}
  \and \\
  \freenames{@{x}} := \{ x \}
\end{mathpar}

$\pi$
$\quotep{\pi}$

$\freenames{-} : \pi \to \mathcal{P}(\quotep{\pi})$

\begin{eqnarray*}
  \freenames{\pzero} & := & \emptyset \\
  \freenames{x?(y).P} & := & \{ x \} \cup (\freenames{P} \setminus \{ y \}) \\
  \freenames{x!\langle P \rangle} & := & \{ x \} \cup \{ P \} \\
  \freenames{P|Q} & := & \freenames{P} \cup \freenames{Q} \\
  \freenames{\dropn{x}} & := & \{ x \}
\end{eqnarray*}

The bound names of a process, $\boundnames{P}$, are those names occurring in $P$
that are not free. For example, in $x?(y).0$, the name $x$ is free, while $y$ is bound.

\begin{mathpar}
  \inferrule* [lab=monoidal-laws] {} { P|Q \equiv Q|P \and P|0 \equiv P \and P|(Q|R) \equiv (P|Q)|R }
\end{mathpar}

\begin{mathpar}
  \inferrule* [lab=alpha-equivalence] {} { (x)P \equiv (y)P\{y/x\} \and y \not\in \freenames{P} }
\end{mathpar}

\begin{definition}
Then two processes, $P,Q$, are alpha-equivalent if $P = Q\{\vec{y}/\vec{x}\}$ for
some $\vec{x} \in \boundnames{Q},\vec{y} \in \boundnames{P}$, where $Q\{\vec{y}/\vec{x}\}$
denotes the capture-avoiding substitution of $\vec{y}$ for $\vec{x}$ in $Q$.
\end{definition}

\begin{definition}
  The {\em structural congruence} \cite{SangiorgiWalker} , $\equiv$,
  between processes is the least congruence containing
  alpha-equivalence, satisfying the abelian monoid laws
  (associativity, commutativity and $\pzero$ as identity) for parallel
  composition $|$ and for summation $+$.
\end{definition}

\subsection{Name equivalence}

We take name equivalence, written $\nameeq$, to be the smallest
equivalence relation generated by the following rules.

\begin{mathpar}
\inferrule*[lab=Quote-drop]
{ }
{ \quotep{@{x}} \nameeq x }

\inferrule*[lab=Struct-equiv]
{ P \scong Q }
{ \quotep{P} \nameeq \quotep{Q} }
\end{mathpar}

The astute reader will have noticed that the mutual recursion of names
and processes imposes a mutual recursion on alpha-equivalence and
structural equivalence via name-equivalence. Fortunately, all of this
works out pleasantly and we may calculate in the natural way, free of
concern. The reader interested in the details is referred to the
appendix \ref{appendix:rho_details}.

\subsection{Substitution}

We use $\Proc$ for the set of processes, $\QProc$ for the set of
names, and $\id{\{}\vec{y} / \vec{x} \id{\}}$ to denote partial maps,
$s : \QProc \rightarrow \QProc$. A map, $s$ lifts, uniquely, to a map
on process terms, $\widehat{s} : \Proc \rightarrow \Proc$ by the
following equations.

\begin{mathpar}
  (0) \psubstp{Q}{P} := 0 \\
  (R \juxtap S) \psubstp{Q}{P}
  :=    
  (R)\psubstp{Q}{P} \juxtap (S) \psubstp{Q}{P} \\
  (x?(y).R) \psubstp{Q}{P}    
  :=    
  (x)\substp{Q}{P} (z)\concat( (R \psubstn{z}{y}) \psubstp{Q}{P} ) \\
  (\lift{x}{R}) \psubstp{Q}{P}  
  :=
  \lift{(x)\substp{Q}{P}}{ R \psubstp{Q}{P} } \\
%   (\dropn{x})  \psubstp{Q}{P}       
%   := 
%   \left\{ 
%     \begin{array}{ccc} 
%       \dropn{\quotep{Q}} & & x \nameeq \quotep{P} \\
%       \dropn{x} & & otherwise \\
%     \end{array}
%   \right. 
  (\dropn{x})  \psubstp{Q}{P}       
  := 
  \left\{ 
    \begin{array}{ccc} 
      Q & & x \nameeq \quotep{P} \\
      \dropn{x} & & otherwise \\
    \end{array}
  \right.
\end{mathpar}
 

where

\begin{eqnarray}
  (x)\id{\{} \lpquote Q \rpquote / \lpquote P \rpquote \id{\}}            = 
  \left\{ 
    \begin{array}{ccc}
      \lpquote Q \rpquote & & x \nameeq \lpquote P \rpquote \\
      x & & otherwise \\
    \end{array}
  \right. \nonumber
\end{eqnarray}

and $z$ is chosen distinct from $\quotep{P}$, $\quotep{Q}$, the free
names in $Q$, and all the names in $R$. Our $\alpha$-equivalence will
be built in the standard way from this substitution.

\begin{remark}\label{rem:no_self_referential_names}
  One consequence of these definitions is that $\forall P. \quotep{P}
  \not\in \freenames{P}$.
\end{remark}

\subsection{ Dynamic quote: an example }

Anticipating something of what's to come, consider applying the
substitution, $\widehat{\id{\{}u / z \id{\}}}$, to the following pair
of processes, $\lift{w}{y!(z)}$ and $w[ \lpquote y!(z) \rpquote ]$.

\begin{eqnarray}
	\lift{w}{y!(z)}\widehat{\id{\{}u / z \id{\}}}
		& = &
		\lift{w}{y!(u)} \nonumber\\
	w[ \lpquote y!(z) \rpquote ] \widehat{ \id{\{}u / z \id{\}} }
		& = &
		w[ \lpquote y!(z) \rpquote ] \nonumber
\end{eqnarray}

Because the body of the process between quotes is impervious to
substitution, we get radically different answers. In fact, by
examining the first process in an input context,
e.g. $x?(z).\lift{w}{y!(z)}$, we see that the process under the lift
operator may be shaped by prefixed inputs binding a name inside it. In
this sense, the lift operator will be seen as a way to dynamically
construct processes before reifying them as names.

Finally equipped with these standard features we can present the
dynamics of the calculus.

\subsubsection{Operational semantics} 

Finally, we introduce the computational dynamics. What marks these
algebras as distinct from other more traditionally studied algebraic
structures, e.g. vector spaces or polynomial rings, is the manner in
which dynamics is captured. In traditional structures, dynamics is typically
expressed through morphisms between such structures, as in linear maps
between vector spaces or morphisms between rings. In algebras
associated with the semantics of computation, the dynamics is
expressed as part of the algebraic structure itself, through a
reduction reduction relation typically denoted by $\red$. Below, we
give a recursive presentation of this relation for the calculus used
in the encoding.

$\red \subseteq \pi \times \pi$
$\red : \pi \to \mathcal{P}(\pi)$

\begin{mathpar}
  \inferrule* [lab=Comm] { \textsf{match}( x_{src}, x_{trgt} ) } { x_{trgt}?(y)P \; | \; x_{src}!\langle {Q} \rangle \red P\{\quotep{Q}/y}\} }
  \and \\
  \inferrule* [lab=Par] {{P} \red {P}'} {{{P} | {Q}} \red {{P}' | {Q}}}
  \and
  \inferrule* [lab=Equiv]{{{P} \scong {P}'} \andalso {{P}' \red {Q}'} \andalso {{Q}' \scong {Q}}}{{P} \red {Q}}
\end{mathpar}

\begin{eqnarray*}
  match_{\equiv} (\quotep{P},\quotep{Q}) & := & P \equiv Q \\
  match_{\dagger}(\quotep{P},\quotep{Q}) & := & \forall R. P|Q \red^{*} R => R \red^{*} 0 \\
  match_{K}(\quotep{P},\quotep{Q}) & := & K \mbox{ for some context } K
\end{eqnarray*}

$u?(x)P | u!\langle Q \rangle \red P\{\quotep{Q}/x\}$

%We write $\wred$ for $\red^*$, and $P\red$ if $\exists Q $ such that $ P \red Q$.
We write $P\red$ if $\exists Q $ such that $ P \red Q$ and $P\not\red$, otherwise.

\section{Replication}

As mentioned before, it is known that replication (and hence
recursion) can be implemented in a higher-order process algebra
\cite{SangiorgiWalker}. As our first example of calculation with the
machinery thus far presented we give the construction explicitly in
the {\rhoc}.

\begin{eqnarray}
	D_{x} & := & \prefix{x}{y}{(\binpar{\outputp{x}{y}}{@{y}})} \nonumber\\
	\bangp_{x}{P} & := & \binpar{{x}!\langle{\binpar{D_{x}}{P}}\rangle}{D_{x}} \nonumber
\end{eqnarray}

\begin{eqnarray}
	\bangp_{x}{P} & & \nonumber\\
	=
	& {x}!\langle{(\prefix{x}{y}{(\outputp{x}{y} | @{y})) | P}}\rangle 
	      | \prefix{x}{y}{(\outputp{x}{y} | @{y})} & \nonumber\\
	\red
	& (\outputp{x}{y} | @{y})\substn{\quotep{(\prefix{x}{y}{(@{y} | \outputp{x}{y})) | P}}}{y} & \nonumber\\
	=
	& \outputp{x}{\quotep{(\prefix{x}{y}{(\outputp{x}{y} | @{y})) | P}}}
	  | {(\prefix{x}{y}{(\outputp{x}{y} | @{y})) | P}} & \nonumber\\
	\red
	& \ldots & \nonumber\\
	\red^*
	& P | P | \ldots & \nonumber
\end{eqnarray}

Of course, this encoding, as an implementation, runs away, unfolding
$\bangp{P}$ eagerly. A lazier and more implementable replication
operator, restricted to input-guarded processes, may be obtained as follows.

\begin{eqnarray}
\bangp{\prefix{u}{v}{P}} 
	:= 
	\binpar{\lift{x}{\prefix{u}{v}{(\binpar{D(x)}{P})}}}{D(x)} \nonumber
\end{eqnarray}

\begin{remark}
  Note that the lazier definition still does not deal with summation
  or mixed summation (i.e. sums over input and output). The reader is
  invited to construct definitions of replication that deal with these
  features. 

  Further, the definitions are parameterized in a name, $x$. Can you,
  gentle reader, make a definition that eliminates this parameter and
  guarantees no accidental interaction between the replication
  machinery and the process being replicated -- i.e. no accidental
  sharing of names used by the process to get its work done and the
  name(s) used by the replication to effect copying. This latter
  revision of the definition of replication is crucial to obtaining
  the expected identity $!!P \sim !P$.
\end{remark}

\begin{remark}\label{rem:paradoxical_combinator}
  The reader familiar with the lambda calculus will have noticed the
  similarity between $D$ and the paradoxical combinator.

  [Ed. note: the existence of this seems to suggest we have to be more
  restrictive on the set of processes and names we admit if we are to
  support no-cloning.]
\end{remark}

\subsubsection{Bisimulation}

The computational dynamics gives rise to another kind of equivalence,
the equivalence of computational behavior. As previously mentioned
this is typically captured \emph{via} some form of bisimulation.

% The notion we use in this paper is weak barbed bisimulation
% \cite{milner91polyadicpi}.

The notion we use in this paper is derived from weak barbed
bisimulation \cite{milner91polyadicpi}. 

\begin{definition}
An \emph{observation relation}, $\downarrow_{\mathcal N}$, over a set
of names, $\mathcal N$, is the smallest relation satisfying the rules
below.

\infrule[Out-barb]{y \in {\mathcal N}, \; x \nameeq y}
		  {\outputp{x}{v} \downarrow_{\mathcal N} x}
\infrule[Par-barb]{\mbox{$P\downarrow_{\mathcal N} x$ or $Q\downarrow_{\mathcal N} x$}}
		  {\binpar{P}{Q} \downarrow_{\mathcal N} x}

We write $P \Downarrow_{\mathcal N} x$ if there is $Q$ such that 
$P \wred Q$ and $Q \downarrow_{\mathcal N} x$.
\end{definition}

\begin{definition}
%\label{def.bbisim}
An  ${\mathcal N}$-\emph{barbed bisimulation} over a set of names, ${\mathcal N}$, is a symmetric binary relation 
${\mathcal S}_{\mathcal N}$ between agents such that $P\rel{S}_{\mathcal N}Q$ implies:
\begin{enumerate}
\item If $P \red P'$ then $Q \wred Q'$ and $P'\rel{S}_{\mathcal N} Q'$.
\item If $P\downarrow_{\mathcal N} x$, then $Q\Downarrow_{\mathcal N} x$.
\end{enumerate}
$P$ is ${\mathcal N}$-barbed bisimilar to $Q$, written
$P \wbbisim_{\mathcal N} Q$, if $P \rel{S}_{\mathcal N} Q$ for some ${\mathcal N}$-barbed bisimulation ${\mathcal S}_{\mathcal N}$.
\end{definition}

$\mathcal{R} \subseteq \pi \times \pi$

$P \mathcal{R} Q => \forall P'. P \red P' \Rightarrow \exists Q'. Q \red Q', P' \mathcal{R} Q'$

$P \vdash x \Rightarrow Q \vdash x$

\begin{mathpar}
  \inferrule*[lab=Out-barb]{x \nameeq y}{{y}!\langle{Q}\rangle \vdash x}
  \and
  \inferrule*[lab=Par-barb]{\mbox{$P\vdash x$ or $Q\vdash x$}}{\binpar{P}{Q} \vdash x}
\end{mathpar}

\subsubsection{Contexts}

One of the principle advantages of computational calculi like the
$\pi$-calculus is a well-defined notion of context,
contextual-equivalence and a correlation between
contextual-equivalence and notions of bisimulation. The notion of
context allows the decomposition of a process into (sub-)process and
its syntactic environment, its context. Thus, a context may be
thought of as a process with a ``hole'' (written $\Box$) in it. The
application of a context $M$ to a process $P$, written $M[P]$, is
tantamount to filling the hole in $M$ with $P$. In this paper we do
not need the full weight of this theory, but do make use of the notion
of context in the proof the main theorem. 

\begin{mathpar}
  \inferrule* [lab=summation] {} {{M_{M},M_{N}} \bc \Box \;|\; x.M_{A} \;|\; M_{M}+M_{N}}
  \and
  \inferrule* [lab=agent] {} {{M_{A}} \bc (\vec{x})M_{P} \;| \; \clift{P_0,\ldots,M_{P},\ldots,P_N}}
  \and \\
  \inferrule* [lab=process] {} {{M_{P}} \bc M_{N} \;| \;P|M_{P} }
\end{mathpar} 

\begin{mathpar}
  \inferrule* [lab=sychronization] {} {M_{N} \bc \Box \;|\; x?M_{F} \;|\; x!M_{C}}
  \and
  \inferrule* [lab=abstraction] {} {{M_{F}} \bc (x)M_{P} }
  \and
  \inferrule* [lab=concretion] {} {{M_{C}} \bc \langle M_{P} \rangle }
  \and \\
  \inferrule* [lab=process] {} {{M_{P}} \bc M_{N} \;| \;P|M_{P} }
\end{mathpar}

\begin{definition}[contextual application] Given a context $M$, and
  process $P$, we define the \emph{contextual application}, $M[P] :=
  M\{P/\Box\}$. That is, the contextual application of M to P is the
  substitution of $P$ for $\Box$ in $M$.
\end{definition}

$\meaningof{-} : L \to \mathcal{P}(\pi)$

\begin{mathpar}
  \inferrule* [lab=collection] {} {\meaningof{true} = \pi, \and \meaningof{~E} = \pi \setminus \meaningof{E}, \and \meaningof{E_{1} \& E_{2}} = \meaningof{E_{1}} \cap \meaningof{E_{2}}}
\end{mathpar}

\begin{mathpar}
  \inferrule* [lab=structure] {} {\meaningof{0} = \{ P \in \pi | P \equiv 0 \}, \and \\ \meaningof{E_1 | E_2} = \{ P \in \pi | P \equiv P_{1} | P_{2}, P_{1} \in \meaningof{E_{1}}, P_{2} \in \meaningof{E_2}\} }
\end{mathpar}

\begin{mathpar}
 \inferrule* [lab=behavior] {} {\meaningof{\langle a?b \rangle E} = \{ P \in \pi | P \equiv Q | u?(y)P', \\ \and \\\\ \and \\ \;\;\; u \in \meaningof{a}, \forall z.P'\{z/y\} \in \meaningof{E\{z/b\}}\}, \and \\ \meaningof{a!E} = \{ P \in \pi | P \equiv Q | x!\langle P' \rangle, x \in \meaningof{a} P' \in \meaningof{E}\} }
\end{mathpar}

\begin{mathpar}
 \inferrule* [lab=nominal] {} {\meaningof{\quotep{E}} = \{ \quotep{P} \in \quotep{\pi} | P \in \meaningof{E} \}, \and \meaningof{\quotep{P}} = \{ \quotep{Q} \in \quotep{\pi} | P \equiv Q \} \and \\ \meaningof{@\quotep{E}} = \{ P \in \pi | P \equiv @x, x \in \meaningof{E} \}}
\end{mathpar}

\begin{eqnarray*}
  \\
  \meaningof{-} : TS \to ST
\end{eqnarray*}

\begin{eqnarray*}
  \\
  L : TS \to ST
\end{eqnarray*}

\begin{eqnarray*}
  \\
  P \models E \iff P \in \meaningof{E}
\end{eqnarray*}

\begin{eqnarray*}
  P \approx_{L} Q \iff \forall E \in L. P \models E \iff Q \models E
\end{eqnarray*}

\begin{eqnarray*}
  P \approx_{K} Q
\end{eqnarray*}

\begin{eqnarray*}
  P \approx Q
\end{eqnarray*}

$\approx_{K} = \approx = \approx_{L}$

\subsubsection{Contextual duality}

Note that contexts extend the quotation operation to a family of
operations from processes to names. Given a context, $M$, we can
define a \emph{nominal context}, $\quotep{M}$ by $\quotep{M}[P] :=
\quotep{M[P]}$. To foreshadow what is to come we observe that these
operations enjoy a duality with processes very much like the duality
between vectors and maps from vectors to scalars.

Further, because the calculus is essentially higher-order, we have a
correspondence between contexts and processes. More specifically,
given a name $x$ and a context $M$ we can construct $M^{*}_{x}$ such
that 

\begin{mathpar}
  M^{*}_{x} | \lift{x}{P} \red M[P]
\end{mathpar}

namely,

\begin{mathpar}
  M^{*}_{x} := x?(u).M[\dropn{u}]
\end{mathpar}

The dependence of $M^{*}_{x}$ on a name makes it an abstraction, 

\begin{mathpar}
  M^{*} := (x)x?(u).M[\dropn{u}]
\end{mathpar}

\subsection{Additional notation}

It will sometimes be convenient to denote the process a name
quotes. We already have the notation $x = \quotep{P}$, but it will be
convenient to introduce an alternate notation, $\procn{x}$, when we
want to emphasize the connection to the use of the name. Note that, by
virtue of name equivalence, $\quotep{\procn{x}} \nameeq x$; so, the
notation is consistent with previous definitions.

Further, because names have structure it is possible to effect
substitutions on the basis of that structure. This means we need to
upgrade our notation for substitutions, which we accomplish by
adapting comprehension notation. Thus,

\begin{mathpar}
  P\{ y / x : x \in S \}
\end{mathpar}

is interpreted to mean the process derived from P by replacing (in a
capture-avoiding manner) each occurrence of $x$ in $S$ by $y$. For example,

\begin{mathpar}
  P\{ \quotep{\procn{x}|\procn{x}} / x : x \in \freenames{P} \}
\end{mathpar}

will replace each (occurrence) of a free name $x$ in $P$ by
$\quotep{\procn{x}|\procn{x}}$.

Also, we will avail ourselves of the notation $x^{L}$ and $x^{R}$ to
denote injections of a name into disjoint copies of the name
space. There are numerous ways to accomplish this. One example can be
found in \cite{MeredithR05}. This notation overloads to vectors of
names: $\vec{x}^{\pi} := (x_{i}^{\pi} \; : \; 0 \leq i < |\vec{x}| )$ where $\pi \in \{L,R\}$.

We also use $P^{\Box} := P|\Box$.

In \cite{MeredithR05} an interpretation of the new operator is
given. It turns out that there are several possible interpretations
all enjoying the requisite algebraic properties of the operator (see
\cite{milner91polyadicpi}). We will therefore make liberal use of
$(\nu\; \vec{x})P$.

% subsection the_syntax_and_semantics_of_the_notation_system (end)   

\input{qm2pi.qmops} 

\input{qm2pi.sterngerlach} 

\input{qm2pi.metric} 

% section concurrent_process_calculi (end)

%\input{qm2pi.proofsketch}

% section proof sketch (end)

%\input{qm2pi.slviaknots} 

% section spatial logic via knots (end)

\input{qm2pi.conclusion}

% section conclusion (end)

%\input{qm2pi.dtcodes} 

% section wiring algorithm (end)

\input{qm2pi.ack} 

% section acknowledgments (end)

\newpage


\bibliographystyle{plain}   
\bibliography{../../biblios/main.bib}

\input{qm2pi.rhodetails}

\end{document}

 

% subsection basic_interpretation (end)

%\input{qm2pi.rho.presentation} 
\subsection{The syntax and semantics of the notation system}\label{sub:the_syntax_and_semantics_of_the_notation_system} % (fold)

We now summarize a technical presentation of the calculus that
embodies our theory of dynamics. The typical presentation of such a
calculus follows the style of giving generators and relations on
them. The grammar, below, describing term constructors, freely
generates the set of processes, $\Proc$. This set is then quotiented
by a relation known as structural congruence and it is over this set
that the notion of dynamics is expressed. This presentation is
essentially that of \cite{MeredithR05} with the addition of
polyadicity and summation. For readability we have relegated some of
the technical subtleties to an appendix.

\subsubsection{Process grammar}\label{subsub:process_grammar}

\begin{mathpar}
  \inferrule* [lab=synchronization] {} {{M} \bc \pzero \;|\; x?F \;|\; x!C }
  \and
  \inferrule* [lab=abstraction] {} {{F} \bc (x)P}
  \and
  \inferrule* [lab=concretion] {} {{C} \bc \langle Q \rangle}
  \and
  \inferrule* [lab=process] {} {{P,Q} \bc M \;| \;P|Q \;|\; @{x}}
  \and
  \inferrule* [lab=name] {} {{x} \bc \quotep{P}}
\end{mathpar} 

Note that $\vec{x}$ (resp. $\vec{P}$) denotes a vector of names
(resp. processes) of length $|\vec{x}|$ (resp. $|\vec{P}|$). We adopt
the following useful abbreviations.

\begin{mathpar}
   x?(\vec{y}).P := x.(\vec{y})P \and  x\clift{\vec{P}} := x.\clift{\vec{P}}
   \and x!(y) := \lift{x}{\dropn{y}}
   \and \Pi_{i=0}^{n-1}P_i := P_0 | \ldots | P_{n-1}
\end{mathpar}

\subsubsection{Structural congruence}

\paragraph{Free and bound names and alpha-equivalence.} At the
core of structural equivalence is alpha-equivalence which identifies
process that are the same up to a change of variable. Formally, we
recognize the distinction between free and bound names. The free names
of a process, $\freenames{P}$, may be calculated recursively as
follows:

\begin{mathpar}
\freenames{\pzero} := \emptyset
  \and \\
  \freenames{x?(y).P} := \{ x \} \cup (\freenames{P} \setminus \{ y \})
  \and 
  \freenames{x!\langle P \rangle} := \{ x \} \cup \{ P \} 
  \and \\
  \freenames{P|Q} := \freenames{P} \cup \freenames{Q}
  \and \\
  \freenames{@{x}} := \{ x \}
\end{mathpar}

$\pi$
$\quotep{\pi}$

$\freenames{-} : \pi \to \mathcal{P}(\quotep{\pi})$

\begin{eqnarray*}
  \freenames{\pzero} & := & \emptyset \\
  \freenames{x?(y).P} & := & \{ x \} \cup (\freenames{P} \setminus \{ y \}) \\
  \freenames{x!\langle P \rangle} & := & \{ x \} \cup \{ P \} \\
  \freenames{P|Q} & := & \freenames{P} \cup \freenames{Q} \\
  \freenames{\dropn{x}} & := & \{ x \}
\end{eqnarray*}

The bound names of a process, $\boundnames{P}$, are those names occurring in $P$
that are not free. For example, in $x?(y).0$, the name $x$ is free, while $y$ is bound.

\begin{mathpar}
  \inferrule* [lab=monoidal-laws] {} { P|Q \equiv Q|P \and P|0 \equiv P \and P|(Q|R) \equiv (P|Q)|R }
\end{mathpar}

\begin{mathpar}
  \inferrule* [lab=alpha-equivalence] {} { (x)P \equiv (y)P\{y/x\} \and y \not\in \freenames{P} }
\end{mathpar}

\begin{definition}
Then two processes, $P,Q$, are alpha-equivalent if $P = Q\{\vec{y}/\vec{x}\}$ for
some $\vec{x} \in \boundnames{Q},\vec{y} \in \boundnames{P}$, where $Q\{\vec{y}/\vec{x}\}$
denotes the capture-avoiding substitution of $\vec{y}$ for $\vec{x}$ in $Q$.
\end{definition}

\begin{definition}
  The {\em structural congruence} \cite{SangiorgiWalker} , $\equiv$,
  between processes is the least congruence containing
  alpha-equivalence, satisfying the abelian monoid laws
  (associativity, commutativity and $\pzero$ as identity) for parallel
  composition $|$ and for summation $+$.
\end{definition}

\subsection{Name equivalence}

We take name equivalence, written $\nameeq$, to be the smallest
equivalence relation generated by the following rules.

\begin{mathpar}
\inferrule*[lab=Quote-drop]
{ }
{ \quotep{@{x}} \nameeq x }

\inferrule*[lab=Struct-equiv]
{ P \scong Q }
{ \quotep{P} \nameeq \quotep{Q} }
\end{mathpar}

The astute reader will have noticed that the mutual recursion of names
and processes imposes a mutual recursion on alpha-equivalence and
structural equivalence via name-equivalence. Fortunately, all of this
works out pleasantly and we may calculate in the natural way, free of
concern. The reader interested in the details is referred to the
appendix \ref{appendix:rho_details}.

\subsection{Substitution}

We use $\Proc$ for the set of processes, $\QProc$ for the set of
names, and $\id{\{}\vec{y} / \vec{x} \id{\}}$ to denote partial maps,
$s : \QProc \rightarrow \QProc$. A map, $s$ lifts, uniquely, to a map
on process terms, $\widehat{s} : \Proc \rightarrow \Proc$ by the
following equations.

\begin{mathpar}
  (0) \psubstp{Q}{P} := 0 \\
  (R \juxtap S) \psubstp{Q}{P}
  :=    
  (R)\psubstp{Q}{P} \juxtap (S) \psubstp{Q}{P} \\
  (x?(y).R) \psubstp{Q}{P}    
  :=    
  (x)\substp{Q}{P} (z)\concat( (R \psubstn{z}{y}) \psubstp{Q}{P} ) \\
  (\lift{x}{R}) \psubstp{Q}{P}  
  :=
  \lift{(x)\substp{Q}{P}}{ R \psubstp{Q}{P} } \\
%   (\dropn{x})  \psubstp{Q}{P}       
%   := 
%   \left\{ 
%     \begin{array}{ccc} 
%       \dropn{\quotep{Q}} & & x \nameeq \quotep{P} \\
%       \dropn{x} & & otherwise \\
%     \end{array}
%   \right. 
  (\dropn{x})  \psubstp{Q}{P}       
  := 
  \left\{ 
    \begin{array}{ccc} 
      Q & & x \nameeq \quotep{P} \\
      \dropn{x} & & otherwise \\
    \end{array}
  \right.
\end{mathpar}
 

where

\begin{eqnarray}
  (x)\id{\{} \lpquote Q \rpquote / \lpquote P \rpquote \id{\}}            = 
  \left\{ 
    \begin{array}{ccc}
      \lpquote Q \rpquote & & x \nameeq \lpquote P \rpquote \\
      x & & otherwise \\
    \end{array}
  \right. \nonumber
\end{eqnarray}

and $z$ is chosen distinct from $\quotep{P}$, $\quotep{Q}$, the free
names in $Q$, and all the names in $R$. Our $\alpha$-equivalence will
be built in the standard way from this substitution.

\begin{remark}\label{rem:no_self_referential_names}
  One consequence of these definitions is that $\forall P. \quotep{P}
  \not\in \freenames{P}$.
\end{remark}

\subsection{ Dynamic quote: an example }

Anticipating something of what's to come, consider applying the
substitution, $\widehat{\id{\{}u / z \id{\}}}$, to the following pair
of processes, $\lift{w}{y!(z)}$ and $w[ \lpquote y!(z) \rpquote ]$.

\begin{eqnarray}
	\lift{w}{y!(z)}\widehat{\id{\{}u / z \id{\}}}
		& = &
		\lift{w}{y!(u)} \nonumber\\
	w[ \lpquote y!(z) \rpquote ] \widehat{ \id{\{}u / z \id{\}} }
		& = &
		w[ \lpquote y!(z) \rpquote ] \nonumber
\end{eqnarray}

Because the body of the process between quotes is impervious to
substitution, we get radically different answers. In fact, by
examining the first process in an input context,
e.g. $x?(z).\lift{w}{y!(z)}$, we see that the process under the lift
operator may be shaped by prefixed inputs binding a name inside it. In
this sense, the lift operator will be seen as a way to dynamically
construct processes before reifying them as names.

Finally equipped with these standard features we can present the
dynamics of the calculus.

\subsubsection{Operational semantics} 

Finally, we introduce the computational dynamics. What marks these
algebras as distinct from other more traditionally studied algebraic
structures, e.g. vector spaces or polynomial rings, is the manner in
which dynamics is captured. In traditional structures, dynamics is typically
expressed through morphisms between such structures, as in linear maps
between vector spaces or morphisms between rings. In algebras
associated with the semantics of computation, the dynamics is
expressed as part of the algebraic structure itself, through a
reduction reduction relation typically denoted by $\red$. Below, we
give a recursive presentation of this relation for the calculus used
in the encoding.

$\red \subseteq \pi \times \pi$
$\red : \pi \to \mathcal{P}(\pi)$

\begin{mathpar}
  \inferrule* [lab=Comm] { \textsf{match}( x_{src}, x_{trgt} ) } { x_{trgt}?(y)P \; | \; x_{src}!\langle {Q} \rangle \red P\{\quotep{Q}/y}\} }
  \and \\
  \inferrule* [lab=Par] {{P} \red {P}'} {{{P} | {Q}} \red {{P}' | {Q}}}
  \and
  \inferrule* [lab=Equiv]{{{P} \scong {P}'} \andalso {{P}' \red {Q}'} \andalso {{Q}' \scong {Q}}}{{P} \red {Q}}
\end{mathpar}

\begin{eqnarray*}
  match_{\equiv} (\quotep{P},\quotep{Q}) & := & P \equiv Q \\
  match_{\dagger}(\quotep{P},\quotep{Q}) & := & \forall R. P|Q \red^{*} R => R \red^{*} 0 \\
  match_{K}(\quotep{P},\quotep{Q}) & := & K \mbox{ for some context } K
\end{eqnarray*}

$u?(x)P | u!\langle Q \rangle \red P\{\quotep{Q}/x\}$

%We write $\wred$ for $\red^*$, and $P\red$ if $\exists Q $ such that $ P \red Q$.
We write $P\red$ if $\exists Q $ such that $ P \red Q$ and $P\not\red$, otherwise.

\section{Replication}

As mentioned before, it is known that replication (and hence
recursion) can be implemented in a higher-order process algebra
\cite{SangiorgiWalker}. As our first example of calculation with the
machinery thus far presented we give the construction explicitly in
the {\rhoc}.

\begin{eqnarray}
	D_{x} & := & \prefix{x}{y}{(\binpar{\outputp{x}{y}}{@{y}})} \nonumber\\
	\bangp_{x}{P} & := & \binpar{{x}!\langle{\binpar{D_{x}}{P}}\rangle}{D_{x}} \nonumber
\end{eqnarray}

\begin{eqnarray}
	\bangp_{x}{P} & & \nonumber\\
	=
	& {x}!\langle{(\prefix{x}{y}{(\outputp{x}{y} | @{y})) | P}}\rangle 
	      | \prefix{x}{y}{(\outputp{x}{y} | @{y})} & \nonumber\\
	\red
	& (\outputp{x}{y} | @{y})\substn{\quotep{(\prefix{x}{y}{(@{y} | \outputp{x}{y})) | P}}}{y} & \nonumber\\
	=
	& \outputp{x}{\quotep{(\prefix{x}{y}{(\outputp{x}{y} | @{y})) | P}}}
	  | {(\prefix{x}{y}{(\outputp{x}{y} | @{y})) | P}} & \nonumber\\
	\red
	& \ldots & \nonumber\\
	\red^*
	& P | P | \ldots & \nonumber
\end{eqnarray}

Of course, this encoding, as an implementation, runs away, unfolding
$\bangp{P}$ eagerly. A lazier and more implementable replication
operator, restricted to input-guarded processes, may be obtained as follows.

\begin{eqnarray}
\bangp{\prefix{u}{v}{P}} 
	:= 
	\binpar{\lift{x}{\prefix{u}{v}{(\binpar{D(x)}{P})}}}{D(x)} \nonumber
\end{eqnarray}

\begin{remark}
  Note that the lazier definition still does not deal with summation
  or mixed summation (i.e. sums over input and output). The reader is
  invited to construct definitions of replication that deal with these
  features. 

  Further, the definitions are parameterized in a name, $x$. Can you,
  gentle reader, make a definition that eliminates this parameter and
  guarantees no accidental interaction between the replication
  machinery and the process being replicated -- i.e. no accidental
  sharing of names used by the process to get its work done and the
  name(s) used by the replication to effect copying. This latter
  revision of the definition of replication is crucial to obtaining
  the expected identity $!!P \sim !P$.
\end{remark}

\begin{remark}\label{rem:paradoxical_combinator}
  The reader familiar with the lambda calculus will have noticed the
  similarity between $D$ and the paradoxical combinator.

  [Ed. note: the existence of this seems to suggest we have to be more
  restrictive on the set of processes and names we admit if we are to
  support no-cloning.]
\end{remark}

\subsubsection{Bisimulation}

The computational dynamics gives rise to another kind of equivalence,
the equivalence of computational behavior. As previously mentioned
this is typically captured \emph{via} some form of bisimulation.

% The notion we use in this paper is weak barbed bisimulation
% \cite{milner91polyadicpi}.

The notion we use in this paper is derived from weak barbed
bisimulation \cite{milner91polyadicpi}. 

\begin{definition}
An \emph{observation relation}, $\downarrow_{\mathcal N}$, over a set
of names, $\mathcal N$, is the smallest relation satisfying the rules
below.

\infrule[Out-barb]{y \in {\mathcal N}, \; x \nameeq y}
		  {\outputp{x}{v} \downarrow_{\mathcal N} x}
\infrule[Par-barb]{\mbox{$P\downarrow_{\mathcal N} x$ or $Q\downarrow_{\mathcal N} x$}}
		  {\binpar{P}{Q} \downarrow_{\mathcal N} x}

We write $P \Downarrow_{\mathcal N} x$ if there is $Q$ such that 
$P \wred Q$ and $Q \downarrow_{\mathcal N} x$.
\end{definition}

\begin{definition}
%\label{def.bbisim}
An  ${\mathcal N}$-\emph{barbed bisimulation} over a set of names, ${\mathcal N}$, is a symmetric binary relation 
${\mathcal S}_{\mathcal N}$ between agents such that $P\rel{S}_{\mathcal N}Q$ implies:
\begin{enumerate}
\item If $P \red P'$ then $Q \wred Q'$ and $P'\rel{S}_{\mathcal N} Q'$.
\item If $P\downarrow_{\mathcal N} x$, then $Q\Downarrow_{\mathcal N} x$.
\end{enumerate}
$P$ is ${\mathcal N}$-barbed bisimilar to $Q$, written
$P \wbbisim_{\mathcal N} Q$, if $P \rel{S}_{\mathcal N} Q$ for some ${\mathcal N}$-barbed bisimulation ${\mathcal S}_{\mathcal N}$.
\end{definition}

$\mathcal{R} \subseteq \pi \times \pi$

$P \mathcal{R} Q => \forall P'. P \red P' \Rightarrow \exists Q'. Q \red Q', P' \mathcal{R} Q'$

$P \vdash x \Rightarrow Q \vdash x$

\begin{mathpar}
  \inferrule*[lab=Out-barb]{x \nameeq y}{{y}!\langle{Q}\rangle \vdash x}
  \and
  \inferrule*[lab=Par-barb]{\mbox{$P\vdash x$ or $Q\vdash x$}}{\binpar{P}{Q} \vdash x}
\end{mathpar}

\subsubsection{Contexts}

One of the principle advantages of computational calculi like the
$\pi$-calculus is a well-defined notion of context,
contextual-equivalence and a correlation between
contextual-equivalence and notions of bisimulation. The notion of
context allows the decomposition of a process into (sub-)process and
its syntactic environment, its context. Thus, a context may be
thought of as a process with a ``hole'' (written $\Box$) in it. The
application of a context $M$ to a process $P$, written $M[P]$, is
tantamount to filling the hole in $M$ with $P$. In this paper we do
not need the full weight of this theory, but do make use of the notion
of context in the proof the main theorem. 

\begin{mathpar}
  \inferrule* [lab=summation] {} {{M_{M},M_{N}} \bc \Box \;|\; x.M_{A} \;|\; M_{M}+M_{N}}
  \and
  \inferrule* [lab=agent] {} {{M_{A}} \bc (\vec{x})M_{P} \;| \; \clift{P_0,\ldots,M_{P},\ldots,P_N}}
  \and \\
  \inferrule* [lab=process] {} {{M_{P}} \bc M_{N} \;| \;P|M_{P} }
\end{mathpar} 

\begin{mathpar}
  \inferrule* [lab=sychronization] {} {M_{N} \bc \Box \;|\; x?M_{F} \;|\; x!M_{C}}
  \and
  \inferrule* [lab=abstraction] {} {{M_{F}} \bc (x)M_{P} }
  \and
  \inferrule* [lab=concretion] {} {{M_{C}} \bc \langle M_{P} \rangle }
  \and \\
  \inferrule* [lab=process] {} {{M_{P}} \bc M_{N} \;| \;P|M_{P} }
\end{mathpar}

\begin{definition}[contextual application] Given a context $M$, and
  process $P$, we define the \emph{contextual application}, $M[P] :=
  M\{P/\Box\}$. That is, the contextual application of M to P is the
  substitution of $P$ for $\Box$ in $M$.
\end{definition}

$\meaningof{-} : L \to \mathcal{P}(\pi)$

\begin{mathpar}
  \inferrule* [lab=collection] {} {\meaningof{true} = \pi, \and \meaningof{~E} = \pi \setminus \meaningof{E}, \and \meaningof{E_{1} \& E_{2}} = \meaningof{E_{1}} \cap \meaningof{E_{2}}}
\end{mathpar}

\begin{mathpar}
  \inferrule* [lab=structure] {} {\meaningof{0} = \{ P \in \pi | P \equiv 0 \}, \and \\ \meaningof{E_1 | E_2} = \{ P \in \pi | P \equiv P_{1} | P_{2}, P_{1} \in \meaningof{E_{1}}, P_{2} \in \meaningof{E_2}\} }
\end{mathpar}

\begin{mathpar}
 \inferrule* [lab=behavior] {} {\meaningof{\langle a?b \rangle E} = \{ P \in \pi | P \equiv Q | u?(y)P', \\ \and \\\\ \and \\ \;\;\; u \in \meaningof{a}, \forall z.P'\{z/y\} \in \meaningof{E\{z/b\}}\}, \and \\ \meaningof{a!E} = \{ P \in \pi | P \equiv Q | x!\langle P' \rangle, x \in \meaningof{a} P' \in \meaningof{E}\} }
\end{mathpar}

\begin{mathpar}
 \inferrule* [lab=nominal] {} {\meaningof{\quotep{E}} = \{ \quotep{P} \in \quotep{\pi} | P \in \meaningof{E} \}, \and \meaningof{\quotep{P}} = \{ \quotep{Q} \in \quotep{\pi} | P \equiv Q \} \and \\ \meaningof{@\quotep{E}} = \{ P \in \pi | P \equiv @x, x \in \meaningof{E} \}}
\end{mathpar}

\begin{eqnarray*}
  \\
  \meaningof{-} : TS \to ST
\end{eqnarray*}

\begin{eqnarray*}
  \\
  L : TS \to ST
\end{eqnarray*}

\begin{eqnarray*}
  \\
  P \models E \iff P \in \meaningof{E}
\end{eqnarray*}

\begin{eqnarray*}
  P \approx_{L} Q \iff \forall E \in L. P \models E \iff Q \models E
\end{eqnarray*}

\begin{eqnarray*}
  P \approx_{K} Q
\end{eqnarray*}

\begin{eqnarray*}
  P \approx Q
\end{eqnarray*}

$\approx_{K} = \approx = \approx_{L}$

\subsubsection{Contextual duality}

Note that contexts extend the quotation operation to a family of
operations from processes to names. Given a context, $M$, we can
define a \emph{nominal context}, $\quotep{M}$ by $\quotep{M}[P] :=
\quotep{M[P]}$. To foreshadow what is to come we observe that these
operations enjoy a duality with processes very much like the duality
between vectors and maps from vectors to scalars.

Further, because the calculus is essentially higher-order, we have a
correspondence between contexts and processes. More specifically,
given a name $x$ and a context $M$ we can construct $M^{*}_{x}$ such
that 

\begin{mathpar}
  M^{*}_{x} | \lift{x}{P} \red M[P]
\end{mathpar}

namely,

\begin{mathpar}
  M^{*}_{x} := x?(u).M[\dropn{u}]
\end{mathpar}

The dependence of $M^{*}_{x}$ on a name makes it an abstraction, 

\begin{mathpar}
  M^{*} := (x)x?(u).M[\dropn{u}]
\end{mathpar}

\subsection{Additional notation}

It will sometimes be convenient to denote the process a name
quotes. We already have the notation $x = \quotep{P}$, but it will be
convenient to introduce an alternate notation, $\procn{x}$, when we
want to emphasize the connection to the use of the name. Note that, by
virtue of name equivalence, $\quotep{\procn{x}} \nameeq x$; so, the
notation is consistent with previous definitions.

Further, because names have structure it is possible to effect
substitutions on the basis of that structure. This means we need to
upgrade our notation for substitutions, which we accomplish by
adapting comprehension notation. Thus,

\begin{mathpar}
  P\{ y / x : x \in S \}
\end{mathpar}

is interpreted to mean the process derived from P by replacing (in a
capture-avoiding manner) each occurrence of $x$ in $S$ by $y$. For example,

\begin{mathpar}
  P\{ \quotep{\procn{x}|\procn{x}} / x : x \in \freenames{P} \}
\end{mathpar}

will replace each (occurrence) of a free name $x$ in $P$ by
$\quotep{\procn{x}|\procn{x}}$.

Also, we will avail ourselves of the notation $x^{L}$ and $x^{R}$ to
denote injections of a name into disjoint copies of the name
space. There are numerous ways to accomplish this. One example can be
found in \cite{MeredithR05}. This notation overloads to vectors of
names: $\vec{x}^{\pi} := (x_{i}^{\pi} \; : \; 0 \leq i < |\vec{x}| )$ where $\pi \in \{L,R\}$.

We also use $P^{\Box} := P|\Box$.

In \cite{MeredithR05} an interpretation of the new operator is
given. It turns out that there are several possible interpretations
all enjoying the requisite algebraic properties of the operator (see
\cite{milner91polyadicpi}). We will therefore make liberal use of
$(\nu\; \vec{x})P$.

% subsection the_syntax_and_semantics_of_the_notation_system (end)   

\section{Interpretation of QM}
\subsection{Supporting definitions}
\subsubsection{Multiplication}
\begin{mathpar}
  \quotep{Q} \cdot \quotep{R} := \quotep{Q|R}
  \and \\
  \quotep{Q} \cdot P := P\{ \quotep{Q|R} / \quotep{R} : \quotep{R} \in \freenames{P} \}
\end{mathpar}

\paragraph{Discussion}
The first line needs little explanation. The second line says that
each free name of the process is replaced with the multiplication of
that name by the scalar. Multiplication of a scalar (name) by a state
(process) results in a process all the names of which have been `moved
over' by parallel composition with the process the scalar
quotes. There is a subtlety that the bound names have to be
manipulated so that multiplied names aren't accidentally
captured. There are many ways to achieve this.

\begin{remark}\label{rem:multiplication_identities}
  The reader is invited to verify that for all $x,y,z \in \QProc$ and $P \in \Proc$
  \begin{mathpar}
    x \cdot \quotep{0} \equiv x 
    \and
    x \cdot y \equiv y \cdot x
    \and
    x \cdot (y \cdot z) \equiv (x \cdot y) \cdot z
    \and \\
    \quotep{0} \cdot P \equiv P
    \and \\
    x \cdot (y \cdot P) \equiv (x \cdot y) \cdot P
    \and \\
    x \cdot (P|Q) \equiv (x \cdot P) | (x \cdot Q)
    \and \\    
  \end{mathpar}
\end{remark}

\subsubsection{Tensor product}

We define a tensor product on processes by structural induction.

\paragraph{Tensor of sums} First note that all summations, including
$\pzero$ and sequence, can be written $\Sigma_{i} x_{i}.A_{i} +
\Sigma_{j} x_{j}.C_{j}$, where we have grouped input-guarded processes
together and output-guarded processes together.

Thus, we can define the tensor product of two summations, $N_{1}\otimes N_{2}$, where

\begin{mathpar}
  N_{1} := \Sigma_{i} x_{i}.A_{i} + \Sigma_{j} x_{j}.C_{j}
  \and
  N_{2} := \Sigma_{i'} y_{i'}.B_{i'} + \Sigma_{j'} y_{j'}.D_{j'} 
\end{mathpar}

as follows.

\begin{mathpar}
  \Sigma_{i} x_{i}.A_{i} + \Sigma_{j} x_{j}.C_{j} \otimes \Sigma_{i'}
  y_{i'}.B_{i'} + \Sigma_{j'} y_{j'}.D_{j'} 
  \and \\
  := \; \Sigma_{i} \Sigma_{i'} \quotep{\stackrel{\vee}{x_{i}}| \stackrel{\vee}{y_{i'}}}.(A_{i}\otimes B_{i'}) \; | \; \Sigma_{i'} \Sigma_{i} \quotep{\stackrel{\vee}{y_{i'}}|\stackrel{\vee}{x_{i}}}.(B_{i'}\otimes A_{i})
  \and
  \;\; | \;\; \Sigma_{j} \Sigma_{j'} \quotep{\stackrel{\vee}{x_{j}}|\stackrel{\vee}{y_{j'}}}.(A_{j}\otimes B_{j'}) \; | \; \Sigma_{j'} \Sigma_{j} \quotep{\stackrel{\vee}{y_{j'}}|\stackrel{\vee}{x_{j}}}.(B_{j'}\otimes A_{j})
\end{mathpar}

\begin{remark}
  Do we need to $x^{L}$ and $y^{R}$ for this construction as well?
\end{remark}

\paragraph{Tensor of parallel compositions} Next, we distribute tensor
over par.

\begin{mathpar}
  P_{1}|P_{2} \otimes Q_{1}|Q_{2} := (P_{1} \otimes Q_{1}) | (P_{1}
  \otimes Q_{2}) | (P_{2} \otimes Q_{1}) | (P_{2} \otimes Q_{2})
\end{mathpar}

\paragraph{Tensor with dropped names} We treat tensor of a
process with a dropped name as parallel composition.

\begin{mathpar}
  P \otimes \dropn{x} := P | \dropn{x}
\end{mathpar}

\paragraph{Tensor of agents}

Finally, we need to define tensor on agents. Note that the definition
of tensor on normal products only tensors inputs with inputs and
outputs with outputs. Thus, we only have to define the operation on
``homogeneous'' pairings.

\begin{mathpar}
  (\vec{x})P \otimes (\vec{y})Q
  \and \\
  := (x_{0}^{L}|y_{0}^{R},\ldots,x_{0}^{L}|y_{n}^{R},\ldots,x_{m}^{L}|y_{0}^{R},\ldots,x_{m}^{L}|y_{n}^R)(P\{ \vec{x}^{L}/\vec{x}\} \otimes Q \{ \vec{y}^{R}/\vec{y}\})
  \and \\
  \clift{\vec{P}} \otimes \clift{\vec{Q}}
  \and \\
  := \clift{P_{0}\otimes Q_{0},\ldots,P_{0}\otimes Q_{n},\ldots,P_{m}\otimes Q_{0},\ldots,P_{m}\otimes Q_{n}}
\end{mathpar}

\begin{remark}
  Observe that arities of tensored abstractions matches arities of
  tensored concretions if the original arities matched. Note also that
  the length of the arities corresponds to the increase in dimension
  we see in ordinary vector space tensor product.
\end{remark}

\begin{remark}
  Operationally, this definition distributes the tensor down to
  components ``linked'' by summation. Tensor over summation is
  intriguing in that it mixes names. Moreover, as a consequence of the
  way it mixes names we have the identities for all $x \in \QProc$ and
  $P,Q \in \Proc$

  \begin{mathpar}
    (x \cdot P) \otimes Q \equiv x \cdot (P \otimes Q) \equiv P \otimes (x \cdot Q)
    \and
    P \otimes \pzero \equiv P
  \end{mathpar}

  that the reader is invited to verify.
\end{remark}

\subsubsection{Annihilation}
\begin{mathpar}
  P^{\perp} := \{ Q | \forall R. P|Q \red^{*} R \Rightarrow R \red^{*} \pzero \}
  \and \\
  P^{\underline{\perp}} := \Sigma_{Q \in P^{\perp}} \quotep{Q}?(y).(\dropn{y}|Q) | \Sigma_{Q \in P^{\perp}} \quotep{Q}\clift{\Box}
\end{mathpar}

\paragraph{Discussion} The reader will note that $P^{\perp}$ is a
\emph{set} of processes, while $P^{\underline{\perp}}$ is a
\emph{context}. We call the set $P^{\perp}$ the \emph{annihilators} of
$P$. The parallel composition of a process in the annihilators of $P$
with $P$ will result in a process, the state space of which has all
paths eventually leading to $\pzero$. Execution may endure loops; but
under reasonable conditions of fairness (naturally guaranteed under
most notions of bisimulation) such a composite process cannot get
stuck in such a loop and will, eventually pop out and terminate.

The context $P^{\underline{\perp}}$ is ready and willing to ``take the
$P$ out of'' the process to which it is applied. It will effectively
transmit the code of the process to which it is applied to one of the
annihilators and run the process against it.

\subsubsection{Evaluation}
We fix $M$ a domain of fully abstract interpretation with an equality
coincident with bisimulation. We take $\meaningof{\cdot} : \Proc \to
M$ to be the map interpreting processes and $\nmeaningof{\cdot} : \M
\to Proc$ to be the map running the other way. Then we define

\begin{mathpar}
  \int P := \nmeaningof{\meaningof{P}}
\end{mathpar}

\paragraph{Discussion}
There are many fully abstract interpretations of Milner's
$\pi$-calculus. Any of them can be used as a basis for interpreting
the reflective calculus here. Equipped with such a domain it is
largely a matter of grinding through to check that the Yoneda
construction for the normalization-by-evaluation program can be
extended to this setting.

\begin{remark}
  The reader is invited to verify that $\int (P^{\underline{\perp}}[P]) = 0$.
\end{remark}

\subsection{Quantum mechanics}

Table \ref{tbl:core_qm_op_defns} gives the core operational definitions

\begin{table}[htp]\label{tbl:core_qm_op_defns}
  \center{
    \fbox{
      \begin{tabular}{c|c}
        quantum mechanics & process calculus \\
        \hline
        scalar & $x := \quotep{P}$ \\
        state vector & $\state{P} := P$ \\
        dual & $\state{P}^{*} := \event{P^{\underline{\perp}}} := \quotep{P^{\underline{\perp}}}[-]$ \\
        matrix & $ \Sigma_{\alpha} \state{P_{\alpha}}x_{\alpha}\event{Q_{\alpha}}$ \\
        vector addition & $\state{P} + \state{Q} := \state{P | Q}$ \\
        tensor product & $\state{P} \otimes \state{Q} := \state{P \otimes Q}$ \\
        inner product & $\innerprod{P}{Q} := \quotep{\int P^{\underline{\perp}}[Q]}$ \\
      \end{tabular}
    }
  }
  \caption{QM - operational definitions}
\end{table}

where

\begin{mathpar}
  \prmatrix{P}{Q} := \fprmatrix{P}{\quotep{\pzero}}{Q}
  \and
  \fprmatrix{P}{x}{Q} := (\state{P},x,\event{Q})
  \and
  (\fprmatrix{P}{x}{Q})(\state{R}) := x \cdot \innerprod{Q}{R} \cdot \state{P}
  \and
  (\fprmatrix{P}{x}{Q})(\event{R}) := x \cdot \innerprod{R}{P} \cdot \event{Q}
\end{mathpar}

\paragraph{Discussion}
As promised: vectors (aka states) are represented as processes; duals
as contextual duals; inner product definition should be compared with
standard inner product definition for ....

\begin{remark}
  Assuming $\int (P^{\underline{\perp}}[P]) = 0$, the reader is
  invited to verify that $(\fprmatrix{P}{x}{P})(\state{P}) = x \cdot \state{P}$.
\end{remark}

\begin{remark}
  The reader is invited to verify that $\innerprod{P}{Q}$ could
  equally well have been written $\quotep{\int \stackrel{\vee}{x}}$
  where $x = \event{P^{\underline{\perp}}}(Q)$.

  One of the motivations for this remark is that there is another way
  to factor these operations. We could package up evaluation in the dual:

  \begin{mathpar}
    \state{P}^{*} := \event{\int P^{\underline{\perp}}} := \quotep{\int P^{\underline{\perp}}}[-]
  \end{mathpar}

  and then have inner product defined by
  
  \begin{mathpar}
    \innerprod{P}{Q} := \event{P}(Q)
  \end{mathpar}

  Hopefully, experience with the calculations will provide guidance on
  the best factoring.
\end{remark}

\begin{remark}
  Assuming $\int (P^{\underline{\perp}}[P]) = 0$, the reader is
  invited to verify that $\forall P,Q. (\prmatrix{0}{Q})(\state{0}) =
  \state{0}$ and dually $(\prmatrix{P}{0})(\event{0}) = \event{0}$.
\end{remark}

\begin{remark}
  i'm a little worried that i don't (yet) have proper support for
  complex conjugacy. But, the observation above may give us a
  clue. According to Abramsky, it must be the case that the scalars
  are iso to the homset of the identity for the tensor -- which the
  observation above characterizes. 

  For now, we will simply bookmark the notion with $\overline{x}$.
\end{remark}

\subsubsection{Adjointness}

We need to give a definition of $(\cdot)^{\dagger}$ for matrices. The
obvious candidate definition is
\begin{mathpar}
(\Sigma_{\alpha}\fprmatrix{P_{\alpha}}{x_{\alpha}}{Q_{\alpha}})^{\dagger}
= \Sigma_{\alpha}\fprmatrix{(Q_{\alpha}^{\underline{\perp}})^{*}}{\overline{x}_{\alpha}}{P_{\alpha}^{\underline{\perp}}} 
\end{mathpar}

But, $(Q_{\alpha}^{\underline{\perp}})^{*}$ requires a name along
which to communicate the process to achieve the context application.

\subsubsection{Basis for a basis}
If processes label states and ``addition'' of states (a.k.a. vector
addition) is interpreted as parallel composition, what corresponds to
notions of linear independence and basis? Here, we recall that Yoshida
has developed a set of \emph{combinators} for an asynchronous verison
of Milner's $\pi$-calculus. These are a finite set of processes such
any process can be expressed as parallel composition of these
combinators together with liberal uses of the new operator and
replication. We can simply give a translation of these into the
present calculus and have reasonable expectation that the property
carries over. That is, that the resultant set allows to express all
processes via parallel composition. Note, however, that there is no
new operator or replication in this calculus. As a result, we expect
that the corresponding set is actually infinite. That is, we expect
that the space is actually infinite dimensional.

\begin{remark}
  The attentive reader may be a bit concerned. Certainly, the
  collection $S$, $K$ and $I$ is a finite set of
  combinators. Shouldn't we expect to see a finite set of combinators
  for an effectively equivalent system? i am very sympathetic to this
  critique and feel it warrants full attention. On the other hand, i
  also have in mind the following analogy. The natural numbers, as a
  monoid under addition, has exactly $1$ generator, while the natural
  numbers, as a monoid under multiplication, has countably many
  generators (the primes). We observe that the application of the
  lambda calculus is much less resource sensitive than the parallel
  composition of the $\pi$-calculus. Could it be the case that we have
  an analogy of the form
  
  \begin{mathpar}
    m + n : MN :: m*n : M|N
  \end{mathpar}

  giving a similar blow up in the set of ``primes''?  This is such a
  wonderful thought that, even if it's not true, i think it's worth
  writing down.
\end{remark}
 

\documentclass[12pt]{llncs}
%\documentclass{jktr}

\usepackage[pdftex]{hyperref}                   
\usepackage {listings}
\usepackage {mathpartir}
\usepackage{bcprules}
%\usepackage{listings}
                       
\usepackage{graphicx} 
%\usepackage[margins=2.5cm,nohead,nofoot]{geometry}
%\usepackage{geometry}
\usepackage{amsfonts}
\usepackage{amstext}
\usepackage{latexsym}
\usepackage{amssymb}
\usepackage{color}


%\include{myPreamble}
\include{qm2pi.local} 

%\ifpdf
%\usepackage[pdftex]{graphicx}
%\else
%\usepackage{graphicx}
%\fi

 % \ifpdf
%  \usepackage{pdfsync}
%  \if


%\title{Brief Article}
%\author{David F. Snyder}
%\author{L.G. Meredith}

%\address{Dept. of Math., Texas State University--San Marcos, San Marcos, TX 78666}
       
\pagestyle{empty}


\begin{document}

\lstset{language=[Objective]Caml,frame=shadowbox}

\input{qm2pi.front}

% section front matter (end)

\input{qm2pi.intro} 
 
% section introduction (end)

% \input{qm2pi.knotations} 

% section notation (end)

\input{qm2pi.process.calculi} 

% section concurrent_process_calculi_and_spatial_logics_ (end)
    
%\input{qm2pi.knots2pi} 

%\input{qm2pi.trefoil} 

%\input{qm2pi.mainthm} 

% subsection basic_interpretation (end)

%\input{qm2pi.rho.presentation} 
\subsection{The syntax and semantics of the notation system}\label{sub:the_syntax_and_semantics_of_the_notation_system} % (fold)

We now summarize a technical presentation of the calculus that
embodies our theory of dynamics. The typical presentation of such a
calculus follows the style of giving generators and relations on
them. The grammar, below, describing term constructors, freely
generates the set of processes, $\Proc$. This set is then quotiented
by a relation known as structural congruence and it is over this set
that the notion of dynamics is expressed. This presentation is
essentially that of \cite{MeredithR05} with the addition of
polyadicity and summation. For readability we have relegated some of
the technical subtleties to an appendix.

\subsubsection{Process grammar}\label{subsub:process_grammar}

\begin{mathpar}
  \inferrule* [lab=synchronization] {} {{M} \bc \pzero \;|\; x?F \;|\; x!C }
  \and
  \inferrule* [lab=abstraction] {} {{F} \bc (x)P}
  \and
  \inferrule* [lab=concretion] {} {{C} \bc \langle Q \rangle}
  \and
  \inferrule* [lab=process] {} {{P,Q} \bc M \;| \;P|Q \;|\; @{x}}
  \and
  \inferrule* [lab=name] {} {{x} \bc \quotep{P}}
\end{mathpar} 

Note that $\vec{x}$ (resp. $\vec{P}$) denotes a vector of names
(resp. processes) of length $|\vec{x}|$ (resp. $|\vec{P}|$). We adopt
the following useful abbreviations.

\begin{mathpar}
   x?(\vec{y}).P := x.(\vec{y})P \and  x\clift{\vec{P}} := x.\clift{\vec{P}}
   \and x!(y) := \lift{x}{\dropn{y}}
   \and \Pi_{i=0}^{n-1}P_i := P_0 | \ldots | P_{n-1}
\end{mathpar}

\subsubsection{Structural congruence}

\paragraph{Free and bound names and alpha-equivalence.} At the
core of structural equivalence is alpha-equivalence which identifies
process that are the same up to a change of variable. Formally, we
recognize the distinction between free and bound names. The free names
of a process, $\freenames{P}$, may be calculated recursively as
follows:

\begin{mathpar}
\freenames{\pzero} := \emptyset
  \and \\
  \freenames{x?(y).P} := \{ x \} \cup (\freenames{P} \setminus \{ y \})
  \and 
  \freenames{x!\langle P \rangle} := \{ x \} \cup \{ P \} 
  \and \\
  \freenames{P|Q} := \freenames{P} \cup \freenames{Q}
  \and \\
  \freenames{@{x}} := \{ x \}
\end{mathpar}

$\pi$
$\quotep{\pi}$

$\freenames{-} : \pi \to \mathcal{P}(\quotep{\pi})$

\begin{eqnarray*}
  \freenames{\pzero} & := & \emptyset \\
  \freenames{x?(y).P} & := & \{ x \} \cup (\freenames{P} \setminus \{ y \}) \\
  \freenames{x!\langle P \rangle} & := & \{ x \} \cup \{ P \} \\
  \freenames{P|Q} & := & \freenames{P} \cup \freenames{Q} \\
  \freenames{\dropn{x}} & := & \{ x \}
\end{eqnarray*}

The bound names of a process, $\boundnames{P}$, are those names occurring in $P$
that are not free. For example, in $x?(y).0$, the name $x$ is free, while $y$ is bound.

\begin{mathpar}
  \inferrule* [lab=monoidal-laws] {} { P|Q \equiv Q|P \and P|0 \equiv P \and P|(Q|R) \equiv (P|Q)|R }
\end{mathpar}

\begin{mathpar}
  \inferrule* [lab=alpha-equivalence] {} { (x)P \equiv (y)P\{y/x\} \and y \not\in \freenames{P} }
\end{mathpar}

\begin{definition}
Then two processes, $P,Q$, are alpha-equivalent if $P = Q\{\vec{y}/\vec{x}\}$ for
some $\vec{x} \in \boundnames{Q},\vec{y} \in \boundnames{P}$, where $Q\{\vec{y}/\vec{x}\}$
denotes the capture-avoiding substitution of $\vec{y}$ for $\vec{x}$ in $Q$.
\end{definition}

\begin{definition}
  The {\em structural congruence} \cite{SangiorgiWalker} , $\equiv$,
  between processes is the least congruence containing
  alpha-equivalence, satisfying the abelian monoid laws
  (associativity, commutativity and $\pzero$ as identity) for parallel
  composition $|$ and for summation $+$.
\end{definition}

\subsection{Name equivalence}

We take name equivalence, written $\nameeq$, to be the smallest
equivalence relation generated by the following rules.

\begin{mathpar}
\inferrule*[lab=Quote-drop]
{ }
{ \quotep{@{x}} \nameeq x }

\inferrule*[lab=Struct-equiv]
{ P \scong Q }
{ \quotep{P} \nameeq \quotep{Q} }
\end{mathpar}

The astute reader will have noticed that the mutual recursion of names
and processes imposes a mutual recursion on alpha-equivalence and
structural equivalence via name-equivalence. Fortunately, all of this
works out pleasantly and we may calculate in the natural way, free of
concern. The reader interested in the details is referred to the
appendix \ref{appendix:rho_details}.

\subsection{Substitution}

We use $\Proc$ for the set of processes, $\QProc$ for the set of
names, and $\id{\{}\vec{y} / \vec{x} \id{\}}$ to denote partial maps,
$s : \QProc \rightarrow \QProc$. A map, $s$ lifts, uniquely, to a map
on process terms, $\widehat{s} : \Proc \rightarrow \Proc$ by the
following equations.

\begin{mathpar}
  (0) \psubstp{Q}{P} := 0 \\
  (R \juxtap S) \psubstp{Q}{P}
  :=    
  (R)\psubstp{Q}{P} \juxtap (S) \psubstp{Q}{P} \\
  (x?(y).R) \psubstp{Q}{P}    
  :=    
  (x)\substp{Q}{P} (z)\concat( (R \psubstn{z}{y}) \psubstp{Q}{P} ) \\
  (\lift{x}{R}) \psubstp{Q}{P}  
  :=
  \lift{(x)\substp{Q}{P}}{ R \psubstp{Q}{P} } \\
%   (\dropn{x})  \psubstp{Q}{P}       
%   := 
%   \left\{ 
%     \begin{array}{ccc} 
%       \dropn{\quotep{Q}} & & x \nameeq \quotep{P} \\
%       \dropn{x} & & otherwise \\
%     \end{array}
%   \right. 
  (\dropn{x})  \psubstp{Q}{P}       
  := 
  \left\{ 
    \begin{array}{ccc} 
      Q & & x \nameeq \quotep{P} \\
      \dropn{x} & & otherwise \\
    \end{array}
  \right.
\end{mathpar}
 

where

\begin{eqnarray}
  (x)\id{\{} \lpquote Q \rpquote / \lpquote P \rpquote \id{\}}            = 
  \left\{ 
    \begin{array}{ccc}
      \lpquote Q \rpquote & & x \nameeq \lpquote P \rpquote \\
      x & & otherwise \\
    \end{array}
  \right. \nonumber
\end{eqnarray}

and $z$ is chosen distinct from $\quotep{P}$, $\quotep{Q}$, the free
names in $Q$, and all the names in $R$. Our $\alpha$-equivalence will
be built in the standard way from this substitution.

\begin{remark}\label{rem:no_self_referential_names}
  One consequence of these definitions is that $\forall P. \quotep{P}
  \not\in \freenames{P}$.
\end{remark}

\subsection{ Dynamic quote: an example }

Anticipating something of what's to come, consider applying the
substitution, $\widehat{\id{\{}u / z \id{\}}}$, to the following pair
of processes, $\lift{w}{y!(z)}$ and $w[ \lpquote y!(z) \rpquote ]$.

\begin{eqnarray}
	\lift{w}{y!(z)}\widehat{\id{\{}u / z \id{\}}}
		& = &
		\lift{w}{y!(u)} \nonumber\\
	w[ \lpquote y!(z) \rpquote ] \widehat{ \id{\{}u / z \id{\}} }
		& = &
		w[ \lpquote y!(z) \rpquote ] \nonumber
\end{eqnarray}

Because the body of the process between quotes is impervious to
substitution, we get radically different answers. In fact, by
examining the first process in an input context,
e.g. $x?(z).\lift{w}{y!(z)}$, we see that the process under the lift
operator may be shaped by prefixed inputs binding a name inside it. In
this sense, the lift operator will be seen as a way to dynamically
construct processes before reifying them as names.

Finally equipped with these standard features we can present the
dynamics of the calculus.

\subsubsection{Operational semantics} 

Finally, we introduce the computational dynamics. What marks these
algebras as distinct from other more traditionally studied algebraic
structures, e.g. vector spaces or polynomial rings, is the manner in
which dynamics is captured. In traditional structures, dynamics is typically
expressed through morphisms between such structures, as in linear maps
between vector spaces or morphisms between rings. In algebras
associated with the semantics of computation, the dynamics is
expressed as part of the algebraic structure itself, through a
reduction reduction relation typically denoted by $\red$. Below, we
give a recursive presentation of this relation for the calculus used
in the encoding.

$\red \subseteq \pi \times \pi$
$\red : \pi \to \mathcal{P}(\pi)$

\begin{mathpar}
  \inferrule* [lab=Comm] { \textsf{match}( x_{src}, x_{trgt} ) } { x_{trgt}?(y)P \; | \; x_{src}!\langle {Q} \rangle \red P\{\quotep{Q}/y}\} }
  \and \\
  \inferrule* [lab=Par] {{P} \red {P}'} {{{P} | {Q}} \red {{P}' | {Q}}}
  \and
  \inferrule* [lab=Equiv]{{{P} \scong {P}'} \andalso {{P}' \red {Q}'} \andalso {{Q}' \scong {Q}}}{{P} \red {Q}}
\end{mathpar}

\begin{eqnarray*}
  match_{\equiv} (\quotep{P},\quotep{Q}) & := & P \equiv Q \\
  match_{\dagger}(\quotep{P},\quotep{Q}) & := & \forall R. P|Q \red^{*} R => R \red^{*} 0 \\
  match_{K}(\quotep{P},\quotep{Q}) & := & K \mbox{ for some context } K
\end{eqnarray*}

$u?(x)P | u!\langle Q \rangle \red P\{\quotep{Q}/x\}$

%We write $\wred$ for $\red^*$, and $P\red$ if $\exists Q $ such that $ P \red Q$.
We write $P\red$ if $\exists Q $ such that $ P \red Q$ and $P\not\red$, otherwise.

\section{Replication}

As mentioned before, it is known that replication (and hence
recursion) can be implemented in a higher-order process algebra
\cite{SangiorgiWalker}. As our first example of calculation with the
machinery thus far presented we give the construction explicitly in
the {\rhoc}.

\begin{eqnarray}
	D_{x} & := & \prefix{x}{y}{(\binpar{\outputp{x}{y}}{@{y}})} \nonumber\\
	\bangp_{x}{P} & := & \binpar{{x}!\langle{\binpar{D_{x}}{P}}\rangle}{D_{x}} \nonumber
\end{eqnarray}

\begin{eqnarray}
	\bangp_{x}{P} & & \nonumber\\
	=
	& {x}!\langle{(\prefix{x}{y}{(\outputp{x}{y} | @{y})) | P}}\rangle 
	      | \prefix{x}{y}{(\outputp{x}{y} | @{y})} & \nonumber\\
	\red
	& (\outputp{x}{y} | @{y})\substn{\quotep{(\prefix{x}{y}{(@{y} | \outputp{x}{y})) | P}}}{y} & \nonumber\\
	=
	& \outputp{x}{\quotep{(\prefix{x}{y}{(\outputp{x}{y} | @{y})) | P}}}
	  | {(\prefix{x}{y}{(\outputp{x}{y} | @{y})) | P}} & \nonumber\\
	\red
	& \ldots & \nonumber\\
	\red^*
	& P | P | \ldots & \nonumber
\end{eqnarray}

Of course, this encoding, as an implementation, runs away, unfolding
$\bangp{P}$ eagerly. A lazier and more implementable replication
operator, restricted to input-guarded processes, may be obtained as follows.

\begin{eqnarray}
\bangp{\prefix{u}{v}{P}} 
	:= 
	\binpar{\lift{x}{\prefix{u}{v}{(\binpar{D(x)}{P})}}}{D(x)} \nonumber
\end{eqnarray}

\begin{remark}
  Note that the lazier definition still does not deal with summation
  or mixed summation (i.e. sums over input and output). The reader is
  invited to construct definitions of replication that deal with these
  features. 

  Further, the definitions are parameterized in a name, $x$. Can you,
  gentle reader, make a definition that eliminates this parameter and
  guarantees no accidental interaction between the replication
  machinery and the process being replicated -- i.e. no accidental
  sharing of names used by the process to get its work done and the
  name(s) used by the replication to effect copying. This latter
  revision of the definition of replication is crucial to obtaining
  the expected identity $!!P \sim !P$.
\end{remark}

\begin{remark}\label{rem:paradoxical_combinator}
  The reader familiar with the lambda calculus will have noticed the
  similarity between $D$ and the paradoxical combinator.

  [Ed. note: the existence of this seems to suggest we have to be more
  restrictive on the set of processes and names we admit if we are to
  support no-cloning.]
\end{remark}

\subsubsection{Bisimulation}

The computational dynamics gives rise to another kind of equivalence,
the equivalence of computational behavior. As previously mentioned
this is typically captured \emph{via} some form of bisimulation.

% The notion we use in this paper is weak barbed bisimulation
% \cite{milner91polyadicpi}.

The notion we use in this paper is derived from weak barbed
bisimulation \cite{milner91polyadicpi}. 

\begin{definition}
An \emph{observation relation}, $\downarrow_{\mathcal N}$, over a set
of names, $\mathcal N$, is the smallest relation satisfying the rules
below.

\infrule[Out-barb]{y \in {\mathcal N}, \; x \nameeq y}
		  {\outputp{x}{v} \downarrow_{\mathcal N} x}
\infrule[Par-barb]{\mbox{$P\downarrow_{\mathcal N} x$ or $Q\downarrow_{\mathcal N} x$}}
		  {\binpar{P}{Q} \downarrow_{\mathcal N} x}

We write $P \Downarrow_{\mathcal N} x$ if there is $Q$ such that 
$P \wred Q$ and $Q \downarrow_{\mathcal N} x$.
\end{definition}

\begin{definition}
%\label{def.bbisim}
An  ${\mathcal N}$-\emph{barbed bisimulation} over a set of names, ${\mathcal N}$, is a symmetric binary relation 
${\mathcal S}_{\mathcal N}$ between agents such that $P\rel{S}_{\mathcal N}Q$ implies:
\begin{enumerate}
\item If $P \red P'$ then $Q \wred Q'$ and $P'\rel{S}_{\mathcal N} Q'$.
\item If $P\downarrow_{\mathcal N} x$, then $Q\Downarrow_{\mathcal N} x$.
\end{enumerate}
$P$ is ${\mathcal N}$-barbed bisimilar to $Q$, written
$P \wbbisim_{\mathcal N} Q$, if $P \rel{S}_{\mathcal N} Q$ for some ${\mathcal N}$-barbed bisimulation ${\mathcal S}_{\mathcal N}$.
\end{definition}

$\mathcal{R} \subseteq \pi \times \pi$

$P \mathcal{R} Q => \forall P'. P \red P' \Rightarrow \exists Q'. Q \red Q', P' \mathcal{R} Q'$

$P \vdash x \Rightarrow Q \vdash x$

\begin{mathpar}
  \inferrule*[lab=Out-barb]{x \nameeq y}{{y}!\langle{Q}\rangle \vdash x}
  \and
  \inferrule*[lab=Par-barb]{\mbox{$P\vdash x$ or $Q\vdash x$}}{\binpar{P}{Q} \vdash x}
\end{mathpar}

\subsubsection{Contexts}

One of the principle advantages of computational calculi like the
$\pi$-calculus is a well-defined notion of context,
contextual-equivalence and a correlation between
contextual-equivalence and notions of bisimulation. The notion of
context allows the decomposition of a process into (sub-)process and
its syntactic environment, its context. Thus, a context may be
thought of as a process with a ``hole'' (written $\Box$) in it. The
application of a context $M$ to a process $P$, written $M[P]$, is
tantamount to filling the hole in $M$ with $P$. In this paper we do
not need the full weight of this theory, but do make use of the notion
of context in the proof the main theorem. 

\begin{mathpar}
  \inferrule* [lab=summation] {} {{M_{M},M_{N}} \bc \Box \;|\; x.M_{A} \;|\; M_{M}+M_{N}}
  \and
  \inferrule* [lab=agent] {} {{M_{A}} \bc (\vec{x})M_{P} \;| \; \clift{P_0,\ldots,M_{P},\ldots,P_N}}
  \and \\
  \inferrule* [lab=process] {} {{M_{P}} \bc M_{N} \;| \;P|M_{P} }
\end{mathpar} 

\begin{mathpar}
  \inferrule* [lab=sychronization] {} {M_{N} \bc \Box \;|\; x?M_{F} \;|\; x!M_{C}}
  \and
  \inferrule* [lab=abstraction] {} {{M_{F}} \bc (x)M_{P} }
  \and
  \inferrule* [lab=concretion] {} {{M_{C}} \bc \langle M_{P} \rangle }
  \and \\
  \inferrule* [lab=process] {} {{M_{P}} \bc M_{N} \;| \;P|M_{P} }
\end{mathpar}

\begin{definition}[contextual application] Given a context $M$, and
  process $P$, we define the \emph{contextual application}, $M[P] :=
  M\{P/\Box\}$. That is, the contextual application of M to P is the
  substitution of $P$ for $\Box$ in $M$.
\end{definition}

$\meaningof{-} : L \to \mathcal{P}(\pi)$

\begin{mathpar}
  \inferrule* [lab=collection] {} {\meaningof{true} = \pi, \and \meaningof{~E} = \pi \setminus \meaningof{E}, \and \meaningof{E_{1} \& E_{2}} = \meaningof{E_{1}} \cap \meaningof{E_{2}}}
\end{mathpar}

\begin{mathpar}
  \inferrule* [lab=structure] {} {\meaningof{0} = \{ P \in \pi | P \equiv 0 \}, \and \\ \meaningof{E_1 | E_2} = \{ P \in \pi | P \equiv P_{1} | P_{2}, P_{1} \in \meaningof{E_{1}}, P_{2} \in \meaningof{E_2}\} }
\end{mathpar}

\begin{mathpar}
 \inferrule* [lab=behavior] {} {\meaningof{\langle a?b \rangle E} = \{ P \in \pi | P \equiv Q | u?(y)P', \\ \and \\\\ \and \\ \;\;\; u \in \meaningof{a}, \forall z.P'\{z/y\} \in \meaningof{E\{z/b\}}\}, \and \\ \meaningof{a!E} = \{ P \in \pi | P \equiv Q | x!\langle P' \rangle, x \in \meaningof{a} P' \in \meaningof{E}\} }
\end{mathpar}

\begin{mathpar}
 \inferrule* [lab=nominal] {} {\meaningof{\quotep{E}} = \{ \quotep{P} \in \quotep{\pi} | P \in \meaningof{E} \}, \and \meaningof{\quotep{P}} = \{ \quotep{Q} \in \quotep{\pi} | P \equiv Q \} \and \\ \meaningof{@\quotep{E}} = \{ P \in \pi | P \equiv @x, x \in \meaningof{E} \}}
\end{mathpar}

\begin{eqnarray*}
  \\
  \meaningof{-} : TS \to ST
\end{eqnarray*}

\begin{eqnarray*}
  \\
  L : TS \to ST
\end{eqnarray*}

\begin{eqnarray*}
  \\
  P \models E \iff P \in \meaningof{E}
\end{eqnarray*}

\begin{eqnarray*}
  P \approx_{L} Q \iff \forall E \in L. P \models E \iff Q \models E
\end{eqnarray*}

\begin{eqnarray*}
  P \approx_{K} Q
\end{eqnarray*}

\begin{eqnarray*}
  P \approx Q
\end{eqnarray*}

$\approx_{K} = \approx = \approx_{L}$

\subsubsection{Contextual duality}

Note that contexts extend the quotation operation to a family of
operations from processes to names. Given a context, $M$, we can
define a \emph{nominal context}, $\quotep{M}$ by $\quotep{M}[P] :=
\quotep{M[P]}$. To foreshadow what is to come we observe that these
operations enjoy a duality with processes very much like the duality
between vectors and maps from vectors to scalars.

Further, because the calculus is essentially higher-order, we have a
correspondence between contexts and processes. More specifically,
given a name $x$ and a context $M$ we can construct $M^{*}_{x}$ such
that 

\begin{mathpar}
  M^{*}_{x} | \lift{x}{P} \red M[P]
\end{mathpar}

namely,

\begin{mathpar}
  M^{*}_{x} := x?(u).M[\dropn{u}]
\end{mathpar}

The dependence of $M^{*}_{x}$ on a name makes it an abstraction, 

\begin{mathpar}
  M^{*} := (x)x?(u).M[\dropn{u}]
\end{mathpar}

\subsection{Additional notation}

It will sometimes be convenient to denote the process a name
quotes. We already have the notation $x = \quotep{P}$, but it will be
convenient to introduce an alternate notation, $\procn{x}$, when we
want to emphasize the connection to the use of the name. Note that, by
virtue of name equivalence, $\quotep{\procn{x}} \nameeq x$; so, the
notation is consistent with previous definitions.

Further, because names have structure it is possible to effect
substitutions on the basis of that structure. This means we need to
upgrade our notation for substitutions, which we accomplish by
adapting comprehension notation. Thus,

\begin{mathpar}
  P\{ y / x : x \in S \}
\end{mathpar}

is interpreted to mean the process derived from P by replacing (in a
capture-avoiding manner) each occurrence of $x$ in $S$ by $y$. For example,

\begin{mathpar}
  P\{ \quotep{\procn{x}|\procn{x}} / x : x \in \freenames{P} \}
\end{mathpar}

will replace each (occurrence) of a free name $x$ in $P$ by
$\quotep{\procn{x}|\procn{x}}$.

Also, we will avail ourselves of the notation $x^{L}$ and $x^{R}$ to
denote injections of a name into disjoint copies of the name
space. There are numerous ways to accomplish this. One example can be
found in \cite{MeredithR05}. This notation overloads to vectors of
names: $\vec{x}^{\pi} := (x_{i}^{\pi} \; : \; 0 \leq i < |\vec{x}| )$ where $\pi \in \{L,R\}$.

We also use $P^{\Box} := P|\Box$.

In \cite{MeredithR05} an interpretation of the new operator is
given. It turns out that there are several possible interpretations
all enjoying the requisite algebraic properties of the operator (see
\cite{milner91polyadicpi}). We will therefore make liberal use of
$(\nu\; \vec{x})P$.

% subsection the_syntax_and_semantics_of_the_notation_system (end)   

\input{qm2pi.qmops} 

\input{qm2pi.sterngerlach} 

\input{qm2pi.metric} 

% section concurrent_process_calculi (end)

%\input{qm2pi.proofsketch}

% section proof sketch (end)

%\input{qm2pi.slviaknots} 

% section spatial logic via knots (end)

\input{qm2pi.conclusion}

% section conclusion (end)

%\input{qm2pi.dtcodes} 

% section wiring algorithm (end)

\input{qm2pi.ack} 

% section acknowledgments (end)

\newpage


\bibliographystyle{plain}   
\bibliography{../../biblios/main.bib}

\input{qm2pi.rhodetails}

\end{document}

 

\documentclass[12pt]{llncs}
%\documentclass{jktr}

\usepackage[pdftex]{hyperref}                   
\usepackage {listings}
\usepackage {mathpartir}
\usepackage{bcprules}
%\usepackage{listings}
                       
\usepackage{graphicx} 
%\usepackage[margins=2.5cm,nohead,nofoot]{geometry}
%\usepackage{geometry}
\usepackage{amsfonts}
\usepackage{amstext}
\usepackage{latexsym}
\usepackage{amssymb}
\usepackage{color}


%\include{myPreamble}
\include{qm2pi.local} 

%\ifpdf
%\usepackage[pdftex]{graphicx}
%\else
%\usepackage{graphicx}
%\fi

 % \ifpdf
%  \usepackage{pdfsync}
%  \if


%\title{Brief Article}
%\author{David F. Snyder}
%\author{L.G. Meredith}

%\address{Dept. of Math., Texas State University--San Marcos, San Marcos, TX 78666}
       
\pagestyle{empty}


\begin{document}

\lstset{language=[Objective]Caml,frame=shadowbox}

\input{qm2pi.front}

% section front matter (end)

\input{qm2pi.intro} 
 
% section introduction (end)

% \input{qm2pi.knotations} 

% section notation (end)

\input{qm2pi.process.calculi} 

% section concurrent_process_calculi_and_spatial_logics_ (end)
    
%\input{qm2pi.knots2pi} 

%\input{qm2pi.trefoil} 

%\input{qm2pi.mainthm} 

% subsection basic_interpretation (end)

%\input{qm2pi.rho.presentation} 
\subsection{The syntax and semantics of the notation system}\label{sub:the_syntax_and_semantics_of_the_notation_system} % (fold)

We now summarize a technical presentation of the calculus that
embodies our theory of dynamics. The typical presentation of such a
calculus follows the style of giving generators and relations on
them. The grammar, below, describing term constructors, freely
generates the set of processes, $\Proc$. This set is then quotiented
by a relation known as structural congruence and it is over this set
that the notion of dynamics is expressed. This presentation is
essentially that of \cite{MeredithR05} with the addition of
polyadicity and summation. For readability we have relegated some of
the technical subtleties to an appendix.

\subsubsection{Process grammar}\label{subsub:process_grammar}

\begin{mathpar}
  \inferrule* [lab=synchronization] {} {{M} \bc \pzero \;|\; x?F \;|\; x!C }
  \and
  \inferrule* [lab=abstraction] {} {{F} \bc (x)P}
  \and
  \inferrule* [lab=concretion] {} {{C} \bc \langle Q \rangle}
  \and
  \inferrule* [lab=process] {} {{P,Q} \bc M \;| \;P|Q \;|\; @{x}}
  \and
  \inferrule* [lab=name] {} {{x} \bc \quotep{P}}
\end{mathpar} 

Note that $\vec{x}$ (resp. $\vec{P}$) denotes a vector of names
(resp. processes) of length $|\vec{x}|$ (resp. $|\vec{P}|$). We adopt
the following useful abbreviations.

\begin{mathpar}
   x?(\vec{y}).P := x.(\vec{y})P \and  x\clift{\vec{P}} := x.\clift{\vec{P}}
   \and x!(y) := \lift{x}{\dropn{y}}
   \and \Pi_{i=0}^{n-1}P_i := P_0 | \ldots | P_{n-1}
\end{mathpar}

\subsubsection{Structural congruence}

\paragraph{Free and bound names and alpha-equivalence.} At the
core of structural equivalence is alpha-equivalence which identifies
process that are the same up to a change of variable. Formally, we
recognize the distinction between free and bound names. The free names
of a process, $\freenames{P}$, may be calculated recursively as
follows:

\begin{mathpar}
\freenames{\pzero} := \emptyset
  \and \\
  \freenames{x?(y).P} := \{ x \} \cup (\freenames{P} \setminus \{ y \})
  \and 
  \freenames{x!\langle P \rangle} := \{ x \} \cup \{ P \} 
  \and \\
  \freenames{P|Q} := \freenames{P} \cup \freenames{Q}
  \and \\
  \freenames{@{x}} := \{ x \}
\end{mathpar}

$\pi$
$\quotep{\pi}$

$\freenames{-} : \pi \to \mathcal{P}(\quotep{\pi})$

\begin{eqnarray*}
  \freenames{\pzero} & := & \emptyset \\
  \freenames{x?(y).P} & := & \{ x \} \cup (\freenames{P} \setminus \{ y \}) \\
  \freenames{x!\langle P \rangle} & := & \{ x \} \cup \{ P \} \\
  \freenames{P|Q} & := & \freenames{P} \cup \freenames{Q} \\
  \freenames{\dropn{x}} & := & \{ x \}
\end{eqnarray*}

The bound names of a process, $\boundnames{P}$, are those names occurring in $P$
that are not free. For example, in $x?(y).0$, the name $x$ is free, while $y$ is bound.

\begin{mathpar}
  \inferrule* [lab=monoidal-laws] {} { P|Q \equiv Q|P \and P|0 \equiv P \and P|(Q|R) \equiv (P|Q)|R }
\end{mathpar}

\begin{mathpar}
  \inferrule* [lab=alpha-equivalence] {} { (x)P \equiv (y)P\{y/x\} \and y \not\in \freenames{P} }
\end{mathpar}

\begin{definition}
Then two processes, $P,Q$, are alpha-equivalent if $P = Q\{\vec{y}/\vec{x}\}$ for
some $\vec{x} \in \boundnames{Q},\vec{y} \in \boundnames{P}$, where $Q\{\vec{y}/\vec{x}\}$
denotes the capture-avoiding substitution of $\vec{y}$ for $\vec{x}$ in $Q$.
\end{definition}

\begin{definition}
  The {\em structural congruence} \cite{SangiorgiWalker} , $\equiv$,
  between processes is the least congruence containing
  alpha-equivalence, satisfying the abelian monoid laws
  (associativity, commutativity and $\pzero$ as identity) for parallel
  composition $|$ and for summation $+$.
\end{definition}

\subsection{Name equivalence}

We take name equivalence, written $\nameeq$, to be the smallest
equivalence relation generated by the following rules.

\begin{mathpar}
\inferrule*[lab=Quote-drop]
{ }
{ \quotep{@{x}} \nameeq x }

\inferrule*[lab=Struct-equiv]
{ P \scong Q }
{ \quotep{P} \nameeq \quotep{Q} }
\end{mathpar}

The astute reader will have noticed that the mutual recursion of names
and processes imposes a mutual recursion on alpha-equivalence and
structural equivalence via name-equivalence. Fortunately, all of this
works out pleasantly and we may calculate in the natural way, free of
concern. The reader interested in the details is referred to the
appendix \ref{appendix:rho_details}.

\subsection{Substitution}

We use $\Proc$ for the set of processes, $\QProc$ for the set of
names, and $\id{\{}\vec{y} / \vec{x} \id{\}}$ to denote partial maps,
$s : \QProc \rightarrow \QProc$. A map, $s$ lifts, uniquely, to a map
on process terms, $\widehat{s} : \Proc \rightarrow \Proc$ by the
following equations.

\begin{mathpar}
  (0) \psubstp{Q}{P} := 0 \\
  (R \juxtap S) \psubstp{Q}{P}
  :=    
  (R)\psubstp{Q}{P} \juxtap (S) \psubstp{Q}{P} \\
  (x?(y).R) \psubstp{Q}{P}    
  :=    
  (x)\substp{Q}{P} (z)\concat( (R \psubstn{z}{y}) \psubstp{Q}{P} ) \\
  (\lift{x}{R}) \psubstp{Q}{P}  
  :=
  \lift{(x)\substp{Q}{P}}{ R \psubstp{Q}{P} } \\
%   (\dropn{x})  \psubstp{Q}{P}       
%   := 
%   \left\{ 
%     \begin{array}{ccc} 
%       \dropn{\quotep{Q}} & & x \nameeq \quotep{P} \\
%       \dropn{x} & & otherwise \\
%     \end{array}
%   \right. 
  (\dropn{x})  \psubstp{Q}{P}       
  := 
  \left\{ 
    \begin{array}{ccc} 
      Q & & x \nameeq \quotep{P} \\
      \dropn{x} & & otherwise \\
    \end{array}
  \right.
\end{mathpar}
 

where

\begin{eqnarray}
  (x)\id{\{} \lpquote Q \rpquote / \lpquote P \rpquote \id{\}}            = 
  \left\{ 
    \begin{array}{ccc}
      \lpquote Q \rpquote & & x \nameeq \lpquote P \rpquote \\
      x & & otherwise \\
    \end{array}
  \right. \nonumber
\end{eqnarray}

and $z$ is chosen distinct from $\quotep{P}$, $\quotep{Q}$, the free
names in $Q$, and all the names in $R$. Our $\alpha$-equivalence will
be built in the standard way from this substitution.

\begin{remark}\label{rem:no_self_referential_names}
  One consequence of these definitions is that $\forall P. \quotep{P}
  \not\in \freenames{P}$.
\end{remark}

\subsection{ Dynamic quote: an example }

Anticipating something of what's to come, consider applying the
substitution, $\widehat{\id{\{}u / z \id{\}}}$, to the following pair
of processes, $\lift{w}{y!(z)}$ and $w[ \lpquote y!(z) \rpquote ]$.

\begin{eqnarray}
	\lift{w}{y!(z)}\widehat{\id{\{}u / z \id{\}}}
		& = &
		\lift{w}{y!(u)} \nonumber\\
	w[ \lpquote y!(z) \rpquote ] \widehat{ \id{\{}u / z \id{\}} }
		& = &
		w[ \lpquote y!(z) \rpquote ] \nonumber
\end{eqnarray}

Because the body of the process between quotes is impervious to
substitution, we get radically different answers. In fact, by
examining the first process in an input context,
e.g. $x?(z).\lift{w}{y!(z)}$, we see that the process under the lift
operator may be shaped by prefixed inputs binding a name inside it. In
this sense, the lift operator will be seen as a way to dynamically
construct processes before reifying them as names.

Finally equipped with these standard features we can present the
dynamics of the calculus.

\subsubsection{Operational semantics} 

Finally, we introduce the computational dynamics. What marks these
algebras as distinct from other more traditionally studied algebraic
structures, e.g. vector spaces or polynomial rings, is the manner in
which dynamics is captured. In traditional structures, dynamics is typically
expressed through morphisms between such structures, as in linear maps
between vector spaces or morphisms between rings. In algebras
associated with the semantics of computation, the dynamics is
expressed as part of the algebraic structure itself, through a
reduction reduction relation typically denoted by $\red$. Below, we
give a recursive presentation of this relation for the calculus used
in the encoding.

$\red \subseteq \pi \times \pi$
$\red : \pi \to \mathcal{P}(\pi)$

\begin{mathpar}
  \inferrule* [lab=Comm] { \textsf{match}( x_{src}, x_{trgt} ) } { x_{trgt}?(y)P \; | \; x_{src}!\langle {Q} \rangle \red P\{\quotep{Q}/y}\} }
  \and \\
  \inferrule* [lab=Par] {{P} \red {P}'} {{{P} | {Q}} \red {{P}' | {Q}}}
  \and
  \inferrule* [lab=Equiv]{{{P} \scong {P}'} \andalso {{P}' \red {Q}'} \andalso {{Q}' \scong {Q}}}{{P} \red {Q}}
\end{mathpar}

\begin{eqnarray*}
  match_{\equiv} (\quotep{P},\quotep{Q}) & := & P \equiv Q \\
  match_{\dagger}(\quotep{P},\quotep{Q}) & := & \forall R. P|Q \red^{*} R => R \red^{*} 0 \\
  match_{K}(\quotep{P},\quotep{Q}) & := & K \mbox{ for some context } K
\end{eqnarray*}

$u?(x)P | u!\langle Q \rangle \red P\{\quotep{Q}/x\}$

%We write $\wred$ for $\red^*$, and $P\red$ if $\exists Q $ such that $ P \red Q$.
We write $P\red$ if $\exists Q $ such that $ P \red Q$ and $P\not\red$, otherwise.

\section{Replication}

As mentioned before, it is known that replication (and hence
recursion) can be implemented in a higher-order process algebra
\cite{SangiorgiWalker}. As our first example of calculation with the
machinery thus far presented we give the construction explicitly in
the {\rhoc}.

\begin{eqnarray}
	D_{x} & := & \prefix{x}{y}{(\binpar{\outputp{x}{y}}{@{y}})} \nonumber\\
	\bangp_{x}{P} & := & \binpar{{x}!\langle{\binpar{D_{x}}{P}}\rangle}{D_{x}} \nonumber
\end{eqnarray}

\begin{eqnarray}
	\bangp_{x}{P} & & \nonumber\\
	=
	& {x}!\langle{(\prefix{x}{y}{(\outputp{x}{y} | @{y})) | P}}\rangle 
	      | \prefix{x}{y}{(\outputp{x}{y} | @{y})} & \nonumber\\
	\red
	& (\outputp{x}{y} | @{y})\substn{\quotep{(\prefix{x}{y}{(@{y} | \outputp{x}{y})) | P}}}{y} & \nonumber\\
	=
	& \outputp{x}{\quotep{(\prefix{x}{y}{(\outputp{x}{y} | @{y})) | P}}}
	  | {(\prefix{x}{y}{(\outputp{x}{y} | @{y})) | P}} & \nonumber\\
	\red
	& \ldots & \nonumber\\
	\red^*
	& P | P | \ldots & \nonumber
\end{eqnarray}

Of course, this encoding, as an implementation, runs away, unfolding
$\bangp{P}$ eagerly. A lazier and more implementable replication
operator, restricted to input-guarded processes, may be obtained as follows.

\begin{eqnarray}
\bangp{\prefix{u}{v}{P}} 
	:= 
	\binpar{\lift{x}{\prefix{u}{v}{(\binpar{D(x)}{P})}}}{D(x)} \nonumber
\end{eqnarray}

\begin{remark}
  Note that the lazier definition still does not deal with summation
  or mixed summation (i.e. sums over input and output). The reader is
  invited to construct definitions of replication that deal with these
  features. 

  Further, the definitions are parameterized in a name, $x$. Can you,
  gentle reader, make a definition that eliminates this parameter and
  guarantees no accidental interaction between the replication
  machinery and the process being replicated -- i.e. no accidental
  sharing of names used by the process to get its work done and the
  name(s) used by the replication to effect copying. This latter
  revision of the definition of replication is crucial to obtaining
  the expected identity $!!P \sim !P$.
\end{remark}

\begin{remark}\label{rem:paradoxical_combinator}
  The reader familiar with the lambda calculus will have noticed the
  similarity between $D$ and the paradoxical combinator.

  [Ed. note: the existence of this seems to suggest we have to be more
  restrictive on the set of processes and names we admit if we are to
  support no-cloning.]
\end{remark}

\subsubsection{Bisimulation}

The computational dynamics gives rise to another kind of equivalence,
the equivalence of computational behavior. As previously mentioned
this is typically captured \emph{via} some form of bisimulation.

% The notion we use in this paper is weak barbed bisimulation
% \cite{milner91polyadicpi}.

The notion we use in this paper is derived from weak barbed
bisimulation \cite{milner91polyadicpi}. 

\begin{definition}
An \emph{observation relation}, $\downarrow_{\mathcal N}$, over a set
of names, $\mathcal N$, is the smallest relation satisfying the rules
below.

\infrule[Out-barb]{y \in {\mathcal N}, \; x \nameeq y}
		  {\outputp{x}{v} \downarrow_{\mathcal N} x}
\infrule[Par-barb]{\mbox{$P\downarrow_{\mathcal N} x$ or $Q\downarrow_{\mathcal N} x$}}
		  {\binpar{P}{Q} \downarrow_{\mathcal N} x}

We write $P \Downarrow_{\mathcal N} x$ if there is $Q$ such that 
$P \wred Q$ and $Q \downarrow_{\mathcal N} x$.
\end{definition}

\begin{definition}
%\label{def.bbisim}
An  ${\mathcal N}$-\emph{barbed bisimulation} over a set of names, ${\mathcal N}$, is a symmetric binary relation 
${\mathcal S}_{\mathcal N}$ between agents such that $P\rel{S}_{\mathcal N}Q$ implies:
\begin{enumerate}
\item If $P \red P'$ then $Q \wred Q'$ and $P'\rel{S}_{\mathcal N} Q'$.
\item If $P\downarrow_{\mathcal N} x$, then $Q\Downarrow_{\mathcal N} x$.
\end{enumerate}
$P$ is ${\mathcal N}$-barbed bisimilar to $Q$, written
$P \wbbisim_{\mathcal N} Q$, if $P \rel{S}_{\mathcal N} Q$ for some ${\mathcal N}$-barbed bisimulation ${\mathcal S}_{\mathcal N}$.
\end{definition}

$\mathcal{R} \subseteq \pi \times \pi$

$P \mathcal{R} Q => \forall P'. P \red P' \Rightarrow \exists Q'. Q \red Q', P' \mathcal{R} Q'$

$P \vdash x \Rightarrow Q \vdash x$

\begin{mathpar}
  \inferrule*[lab=Out-barb]{x \nameeq y}{{y}!\langle{Q}\rangle \vdash x}
  \and
  \inferrule*[lab=Par-barb]{\mbox{$P\vdash x$ or $Q\vdash x$}}{\binpar{P}{Q} \vdash x}
\end{mathpar}

\subsubsection{Contexts}

One of the principle advantages of computational calculi like the
$\pi$-calculus is a well-defined notion of context,
contextual-equivalence and a correlation between
contextual-equivalence and notions of bisimulation. The notion of
context allows the decomposition of a process into (sub-)process and
its syntactic environment, its context. Thus, a context may be
thought of as a process with a ``hole'' (written $\Box$) in it. The
application of a context $M$ to a process $P$, written $M[P]$, is
tantamount to filling the hole in $M$ with $P$. In this paper we do
not need the full weight of this theory, but do make use of the notion
of context in the proof the main theorem. 

\begin{mathpar}
  \inferrule* [lab=summation] {} {{M_{M},M_{N}} \bc \Box \;|\; x.M_{A} \;|\; M_{M}+M_{N}}
  \and
  \inferrule* [lab=agent] {} {{M_{A}} \bc (\vec{x})M_{P} \;| \; \clift{P_0,\ldots,M_{P},\ldots,P_N}}
  \and \\
  \inferrule* [lab=process] {} {{M_{P}} \bc M_{N} \;| \;P|M_{P} }
\end{mathpar} 

\begin{mathpar}
  \inferrule* [lab=sychronization] {} {M_{N} \bc \Box \;|\; x?M_{F} \;|\; x!M_{C}}
  \and
  \inferrule* [lab=abstraction] {} {{M_{F}} \bc (x)M_{P} }
  \and
  \inferrule* [lab=concretion] {} {{M_{C}} \bc \langle M_{P} \rangle }
  \and \\
  \inferrule* [lab=process] {} {{M_{P}} \bc M_{N} \;| \;P|M_{P} }
\end{mathpar}

\begin{definition}[contextual application] Given a context $M$, and
  process $P$, we define the \emph{contextual application}, $M[P] :=
  M\{P/\Box\}$. That is, the contextual application of M to P is the
  substitution of $P$ for $\Box$ in $M$.
\end{definition}

$\meaningof{-} : L \to \mathcal{P}(\pi)$

\begin{mathpar}
  \inferrule* [lab=collection] {} {\meaningof{true} = \pi, \and \meaningof{~E} = \pi \setminus \meaningof{E}, \and \meaningof{E_{1} \& E_{2}} = \meaningof{E_{1}} \cap \meaningof{E_{2}}}
\end{mathpar}

\begin{mathpar}
  \inferrule* [lab=structure] {} {\meaningof{0} = \{ P \in \pi | P \equiv 0 \}, \and \\ \meaningof{E_1 | E_2} = \{ P \in \pi | P \equiv P_{1} | P_{2}, P_{1} \in \meaningof{E_{1}}, P_{2} \in \meaningof{E_2}\} }
\end{mathpar}

\begin{mathpar}
 \inferrule* [lab=behavior] {} {\meaningof{\langle a?b \rangle E} = \{ P \in \pi | P \equiv Q | u?(y)P', \\ \and \\\\ \and \\ \;\;\; u \in \meaningof{a}, \forall z.P'\{z/y\} \in \meaningof{E\{z/b\}}\}, \and \\ \meaningof{a!E} = \{ P \in \pi | P \equiv Q | x!\langle P' \rangle, x \in \meaningof{a} P' \in \meaningof{E}\} }
\end{mathpar}

\begin{mathpar}
 \inferrule* [lab=nominal] {} {\meaningof{\quotep{E}} = \{ \quotep{P} \in \quotep{\pi} | P \in \meaningof{E} \}, \and \meaningof{\quotep{P}} = \{ \quotep{Q} \in \quotep{\pi} | P \equiv Q \} \and \\ \meaningof{@\quotep{E}} = \{ P \in \pi | P \equiv @x, x \in \meaningof{E} \}}
\end{mathpar}

\begin{eqnarray*}
  \\
  \meaningof{-} : TS \to ST
\end{eqnarray*}

\begin{eqnarray*}
  \\
  L : TS \to ST
\end{eqnarray*}

\begin{eqnarray*}
  \\
  P \models E \iff P \in \meaningof{E}
\end{eqnarray*}

\begin{eqnarray*}
  P \approx_{L} Q \iff \forall E \in L. P \models E \iff Q \models E
\end{eqnarray*}

\begin{eqnarray*}
  P \approx_{K} Q
\end{eqnarray*}

\begin{eqnarray*}
  P \approx Q
\end{eqnarray*}

$\approx_{K} = \approx = \approx_{L}$

\subsubsection{Contextual duality}

Note that contexts extend the quotation operation to a family of
operations from processes to names. Given a context, $M$, we can
define a \emph{nominal context}, $\quotep{M}$ by $\quotep{M}[P] :=
\quotep{M[P]}$. To foreshadow what is to come we observe that these
operations enjoy a duality with processes very much like the duality
between vectors and maps from vectors to scalars.

Further, because the calculus is essentially higher-order, we have a
correspondence between contexts and processes. More specifically,
given a name $x$ and a context $M$ we can construct $M^{*}_{x}$ such
that 

\begin{mathpar}
  M^{*}_{x} | \lift{x}{P} \red M[P]
\end{mathpar}

namely,

\begin{mathpar}
  M^{*}_{x} := x?(u).M[\dropn{u}]
\end{mathpar}

The dependence of $M^{*}_{x}$ on a name makes it an abstraction, 

\begin{mathpar}
  M^{*} := (x)x?(u).M[\dropn{u}]
\end{mathpar}

\subsection{Additional notation}

It will sometimes be convenient to denote the process a name
quotes. We already have the notation $x = \quotep{P}$, but it will be
convenient to introduce an alternate notation, $\procn{x}$, when we
want to emphasize the connection to the use of the name. Note that, by
virtue of name equivalence, $\quotep{\procn{x}} \nameeq x$; so, the
notation is consistent with previous definitions.

Further, because names have structure it is possible to effect
substitutions on the basis of that structure. This means we need to
upgrade our notation for substitutions, which we accomplish by
adapting comprehension notation. Thus,

\begin{mathpar}
  P\{ y / x : x \in S \}
\end{mathpar}

is interpreted to mean the process derived from P by replacing (in a
capture-avoiding manner) each occurrence of $x$ in $S$ by $y$. For example,

\begin{mathpar}
  P\{ \quotep{\procn{x}|\procn{x}} / x : x \in \freenames{P} \}
\end{mathpar}

will replace each (occurrence) of a free name $x$ in $P$ by
$\quotep{\procn{x}|\procn{x}}$.

Also, we will avail ourselves of the notation $x^{L}$ and $x^{R}$ to
denote injections of a name into disjoint copies of the name
space. There are numerous ways to accomplish this. One example can be
found in \cite{MeredithR05}. This notation overloads to vectors of
names: $\vec{x}^{\pi} := (x_{i}^{\pi} \; : \; 0 \leq i < |\vec{x}| )$ where $\pi \in \{L,R\}$.

We also use $P^{\Box} := P|\Box$.

In \cite{MeredithR05} an interpretation of the new operator is
given. It turns out that there are several possible interpretations
all enjoying the requisite algebraic properties of the operator (see
\cite{milner91polyadicpi}). We will therefore make liberal use of
$(\nu\; \vec{x})P$.

% subsection the_syntax_and_semantics_of_the_notation_system (end)   

\input{qm2pi.qmops} 

\input{qm2pi.sterngerlach} 

\input{qm2pi.metric} 

% section concurrent_process_calculi (end)

%\input{qm2pi.proofsketch}

% section proof sketch (end)

%\input{qm2pi.slviaknots} 

% section spatial logic via knots (end)

\input{qm2pi.conclusion}

% section conclusion (end)

%\input{qm2pi.dtcodes} 

% section wiring algorithm (end)

\input{qm2pi.ack} 

% section acknowledgments (end)

\newpage


\bibliographystyle{plain}   
\bibliography{../../biblios/main.bib}

\input{qm2pi.rhodetails}

\end{document}

 

% section concurrent_process_calculi (end)

%\documentclass[12pt]{llncs}
%\documentclass{jktr}

\usepackage[pdftex]{hyperref}                   
\usepackage {listings}
\usepackage {mathpartir}
\usepackage{bcprules}
%\usepackage{listings}
                       
\usepackage{graphicx} 
%\usepackage[margins=2.5cm,nohead,nofoot]{geometry}
%\usepackage{geometry}
\usepackage{amsfonts}
\usepackage{amstext}
\usepackage{latexsym}
\usepackage{amssymb}
\usepackage{color}


%\include{myPreamble}
\include{qm2pi.local} 

%\ifpdf
%\usepackage[pdftex]{graphicx}
%\else
%\usepackage{graphicx}
%\fi

 % \ifpdf
%  \usepackage{pdfsync}
%  \if


%\title{Brief Article}
%\author{David F. Snyder}
%\author{L.G. Meredith}

%\address{Dept. of Math., Texas State University--San Marcos, San Marcos, TX 78666}
       
\pagestyle{empty}


\begin{document}

\lstset{language=[Objective]Caml,frame=shadowbox}

\input{qm2pi.front}

% section front matter (end)

\input{qm2pi.intro} 
 
% section introduction (end)

% \input{qm2pi.knotations} 

% section notation (end)

\input{qm2pi.process.calculi} 

% section concurrent_process_calculi_and_spatial_logics_ (end)
    
%\input{qm2pi.knots2pi} 

%\input{qm2pi.trefoil} 

%\input{qm2pi.mainthm} 

% subsection basic_interpretation (end)

%\input{qm2pi.rho.presentation} 
\subsection{The syntax and semantics of the notation system}\label{sub:the_syntax_and_semantics_of_the_notation_system} % (fold)

We now summarize a technical presentation of the calculus that
embodies our theory of dynamics. The typical presentation of such a
calculus follows the style of giving generators and relations on
them. The grammar, below, describing term constructors, freely
generates the set of processes, $\Proc$. This set is then quotiented
by a relation known as structural congruence and it is over this set
that the notion of dynamics is expressed. This presentation is
essentially that of \cite{MeredithR05} with the addition of
polyadicity and summation. For readability we have relegated some of
the technical subtleties to an appendix.

\subsubsection{Process grammar}\label{subsub:process_grammar}

\begin{mathpar}
  \inferrule* [lab=synchronization] {} {{M} \bc \pzero \;|\; x?F \;|\; x!C }
  \and
  \inferrule* [lab=abstraction] {} {{F} \bc (x)P}
  \and
  \inferrule* [lab=concretion] {} {{C} \bc \langle Q \rangle}
  \and
  \inferrule* [lab=process] {} {{P,Q} \bc M \;| \;P|Q \;|\; @{x}}
  \and
  \inferrule* [lab=name] {} {{x} \bc \quotep{P}}
\end{mathpar} 

Note that $\vec{x}$ (resp. $\vec{P}$) denotes a vector of names
(resp. processes) of length $|\vec{x}|$ (resp. $|\vec{P}|$). We adopt
the following useful abbreviations.

\begin{mathpar}
   x?(\vec{y}).P := x.(\vec{y})P \and  x\clift{\vec{P}} := x.\clift{\vec{P}}
   \and x!(y) := \lift{x}{\dropn{y}}
   \and \Pi_{i=0}^{n-1}P_i := P_0 | \ldots | P_{n-1}
\end{mathpar}

\subsubsection{Structural congruence}

\paragraph{Free and bound names and alpha-equivalence.} At the
core of structural equivalence is alpha-equivalence which identifies
process that are the same up to a change of variable. Formally, we
recognize the distinction between free and bound names. The free names
of a process, $\freenames{P}$, may be calculated recursively as
follows:

\begin{mathpar}
\freenames{\pzero} := \emptyset
  \and \\
  \freenames{x?(y).P} := \{ x \} \cup (\freenames{P} \setminus \{ y \})
  \and 
  \freenames{x!\langle P \rangle} := \{ x \} \cup \{ P \} 
  \and \\
  \freenames{P|Q} := \freenames{P} \cup \freenames{Q}
  \and \\
  \freenames{@{x}} := \{ x \}
\end{mathpar}

$\pi$
$\quotep{\pi}$

$\freenames{-} : \pi \to \mathcal{P}(\quotep{\pi})$

\begin{eqnarray*}
  \freenames{\pzero} & := & \emptyset \\
  \freenames{x?(y).P} & := & \{ x \} \cup (\freenames{P} \setminus \{ y \}) \\
  \freenames{x!\langle P \rangle} & := & \{ x \} \cup \{ P \} \\
  \freenames{P|Q} & := & \freenames{P} \cup \freenames{Q} \\
  \freenames{\dropn{x}} & := & \{ x \}
\end{eqnarray*}

The bound names of a process, $\boundnames{P}$, are those names occurring in $P$
that are not free. For example, in $x?(y).0$, the name $x$ is free, while $y$ is bound.

\begin{mathpar}
  \inferrule* [lab=monoidal-laws] {} { P|Q \equiv Q|P \and P|0 \equiv P \and P|(Q|R) \equiv (P|Q)|R }
\end{mathpar}

\begin{mathpar}
  \inferrule* [lab=alpha-equivalence] {} { (x)P \equiv (y)P\{y/x\} \and y \not\in \freenames{P} }
\end{mathpar}

\begin{definition}
Then two processes, $P,Q$, are alpha-equivalent if $P = Q\{\vec{y}/\vec{x}\}$ for
some $\vec{x} \in \boundnames{Q},\vec{y} \in \boundnames{P}$, where $Q\{\vec{y}/\vec{x}\}$
denotes the capture-avoiding substitution of $\vec{y}$ for $\vec{x}$ in $Q$.
\end{definition}

\begin{definition}
  The {\em structural congruence} \cite{SangiorgiWalker} , $\equiv$,
  between processes is the least congruence containing
  alpha-equivalence, satisfying the abelian monoid laws
  (associativity, commutativity and $\pzero$ as identity) for parallel
  composition $|$ and for summation $+$.
\end{definition}

\subsection{Name equivalence}

We take name equivalence, written $\nameeq$, to be the smallest
equivalence relation generated by the following rules.

\begin{mathpar}
\inferrule*[lab=Quote-drop]
{ }
{ \quotep{@{x}} \nameeq x }

\inferrule*[lab=Struct-equiv]
{ P \scong Q }
{ \quotep{P} \nameeq \quotep{Q} }
\end{mathpar}

The astute reader will have noticed that the mutual recursion of names
and processes imposes a mutual recursion on alpha-equivalence and
structural equivalence via name-equivalence. Fortunately, all of this
works out pleasantly and we may calculate in the natural way, free of
concern. The reader interested in the details is referred to the
appendix \ref{appendix:rho_details}.

\subsection{Substitution}

We use $\Proc$ for the set of processes, $\QProc$ for the set of
names, and $\id{\{}\vec{y} / \vec{x} \id{\}}$ to denote partial maps,
$s : \QProc \rightarrow \QProc$. A map, $s$ lifts, uniquely, to a map
on process terms, $\widehat{s} : \Proc \rightarrow \Proc$ by the
following equations.

\begin{mathpar}
  (0) \psubstp{Q}{P} := 0 \\
  (R \juxtap S) \psubstp{Q}{P}
  :=    
  (R)\psubstp{Q}{P} \juxtap (S) \psubstp{Q}{P} \\
  (x?(y).R) \psubstp{Q}{P}    
  :=    
  (x)\substp{Q}{P} (z)\concat( (R \psubstn{z}{y}) \psubstp{Q}{P} ) \\
  (\lift{x}{R}) \psubstp{Q}{P}  
  :=
  \lift{(x)\substp{Q}{P}}{ R \psubstp{Q}{P} } \\
%   (\dropn{x})  \psubstp{Q}{P}       
%   := 
%   \left\{ 
%     \begin{array}{ccc} 
%       \dropn{\quotep{Q}} & & x \nameeq \quotep{P} \\
%       \dropn{x} & & otherwise \\
%     \end{array}
%   \right. 
  (\dropn{x})  \psubstp{Q}{P}       
  := 
  \left\{ 
    \begin{array}{ccc} 
      Q & & x \nameeq \quotep{P} \\
      \dropn{x} & & otherwise \\
    \end{array}
  \right.
\end{mathpar}
 

where

\begin{eqnarray}
  (x)\id{\{} \lpquote Q \rpquote / \lpquote P \rpquote \id{\}}            = 
  \left\{ 
    \begin{array}{ccc}
      \lpquote Q \rpquote & & x \nameeq \lpquote P \rpquote \\
      x & & otherwise \\
    \end{array}
  \right. \nonumber
\end{eqnarray}

and $z$ is chosen distinct from $\quotep{P}$, $\quotep{Q}$, the free
names in $Q$, and all the names in $R$. Our $\alpha$-equivalence will
be built in the standard way from this substitution.

\begin{remark}\label{rem:no_self_referential_names}
  One consequence of these definitions is that $\forall P. \quotep{P}
  \not\in \freenames{P}$.
\end{remark}

\subsection{ Dynamic quote: an example }

Anticipating something of what's to come, consider applying the
substitution, $\widehat{\id{\{}u / z \id{\}}}$, to the following pair
of processes, $\lift{w}{y!(z)}$ and $w[ \lpquote y!(z) \rpquote ]$.

\begin{eqnarray}
	\lift{w}{y!(z)}\widehat{\id{\{}u / z \id{\}}}
		& = &
		\lift{w}{y!(u)} \nonumber\\
	w[ \lpquote y!(z) \rpquote ] \widehat{ \id{\{}u / z \id{\}} }
		& = &
		w[ \lpquote y!(z) \rpquote ] \nonumber
\end{eqnarray}

Because the body of the process between quotes is impervious to
substitution, we get radically different answers. In fact, by
examining the first process in an input context,
e.g. $x?(z).\lift{w}{y!(z)}$, we see that the process under the lift
operator may be shaped by prefixed inputs binding a name inside it. In
this sense, the lift operator will be seen as a way to dynamically
construct processes before reifying them as names.

Finally equipped with these standard features we can present the
dynamics of the calculus.

\subsubsection{Operational semantics} 

Finally, we introduce the computational dynamics. What marks these
algebras as distinct from other more traditionally studied algebraic
structures, e.g. vector spaces or polynomial rings, is the manner in
which dynamics is captured. In traditional structures, dynamics is typically
expressed through morphisms between such structures, as in linear maps
between vector spaces or morphisms between rings. In algebras
associated with the semantics of computation, the dynamics is
expressed as part of the algebraic structure itself, through a
reduction reduction relation typically denoted by $\red$. Below, we
give a recursive presentation of this relation for the calculus used
in the encoding.

$\red \subseteq \pi \times \pi$
$\red : \pi \to \mathcal{P}(\pi)$

\begin{mathpar}
  \inferrule* [lab=Comm] { \textsf{match}( x_{src}, x_{trgt} ) } { x_{trgt}?(y)P \; | \; x_{src}!\langle {Q} \rangle \red P\{\quotep{Q}/y}\} }
  \and \\
  \inferrule* [lab=Par] {{P} \red {P}'} {{{P} | {Q}} \red {{P}' | {Q}}}
  \and
  \inferrule* [lab=Equiv]{{{P} \scong {P}'} \andalso {{P}' \red {Q}'} \andalso {{Q}' \scong {Q}}}{{P} \red {Q}}
\end{mathpar}

\begin{eqnarray*}
  match_{\equiv} (\quotep{P},\quotep{Q}) & := & P \equiv Q \\
  match_{\dagger}(\quotep{P},\quotep{Q}) & := & \forall R. P|Q \red^{*} R => R \red^{*} 0 \\
  match_{K}(\quotep{P},\quotep{Q}) & := & K \mbox{ for some context } K
\end{eqnarray*}

$u?(x)P | u!\langle Q \rangle \red P\{\quotep{Q}/x\}$

%We write $\wred$ for $\red^*$, and $P\red$ if $\exists Q $ such that $ P \red Q$.
We write $P\red$ if $\exists Q $ such that $ P \red Q$ and $P\not\red$, otherwise.

\section{Replication}

As mentioned before, it is known that replication (and hence
recursion) can be implemented in a higher-order process algebra
\cite{SangiorgiWalker}. As our first example of calculation with the
machinery thus far presented we give the construction explicitly in
the {\rhoc}.

\begin{eqnarray}
	D_{x} & := & \prefix{x}{y}{(\binpar{\outputp{x}{y}}{@{y}})} \nonumber\\
	\bangp_{x}{P} & := & \binpar{{x}!\langle{\binpar{D_{x}}{P}}\rangle}{D_{x}} \nonumber
\end{eqnarray}

\begin{eqnarray}
	\bangp_{x}{P} & & \nonumber\\
	=
	& {x}!\langle{(\prefix{x}{y}{(\outputp{x}{y} | @{y})) | P}}\rangle 
	      | \prefix{x}{y}{(\outputp{x}{y} | @{y})} & \nonumber\\
	\red
	& (\outputp{x}{y} | @{y})\substn{\quotep{(\prefix{x}{y}{(@{y} | \outputp{x}{y})) | P}}}{y} & \nonumber\\
	=
	& \outputp{x}{\quotep{(\prefix{x}{y}{(\outputp{x}{y} | @{y})) | P}}}
	  | {(\prefix{x}{y}{(\outputp{x}{y} | @{y})) | P}} & \nonumber\\
	\red
	& \ldots & \nonumber\\
	\red^*
	& P | P | \ldots & \nonumber
\end{eqnarray}

Of course, this encoding, as an implementation, runs away, unfolding
$\bangp{P}$ eagerly. A lazier and more implementable replication
operator, restricted to input-guarded processes, may be obtained as follows.

\begin{eqnarray}
\bangp{\prefix{u}{v}{P}} 
	:= 
	\binpar{\lift{x}{\prefix{u}{v}{(\binpar{D(x)}{P})}}}{D(x)} \nonumber
\end{eqnarray}

\begin{remark}
  Note that the lazier definition still does not deal with summation
  or mixed summation (i.e. sums over input and output). The reader is
  invited to construct definitions of replication that deal with these
  features. 

  Further, the definitions are parameterized in a name, $x$. Can you,
  gentle reader, make a definition that eliminates this parameter and
  guarantees no accidental interaction between the replication
  machinery and the process being replicated -- i.e. no accidental
  sharing of names used by the process to get its work done and the
  name(s) used by the replication to effect copying. This latter
  revision of the definition of replication is crucial to obtaining
  the expected identity $!!P \sim !P$.
\end{remark}

\begin{remark}\label{rem:paradoxical_combinator}
  The reader familiar with the lambda calculus will have noticed the
  similarity between $D$ and the paradoxical combinator.

  [Ed. note: the existence of this seems to suggest we have to be more
  restrictive on the set of processes and names we admit if we are to
  support no-cloning.]
\end{remark}

\subsubsection{Bisimulation}

The computational dynamics gives rise to another kind of equivalence,
the equivalence of computational behavior. As previously mentioned
this is typically captured \emph{via} some form of bisimulation.

% The notion we use in this paper is weak barbed bisimulation
% \cite{milner91polyadicpi}.

The notion we use in this paper is derived from weak barbed
bisimulation \cite{milner91polyadicpi}. 

\begin{definition}
An \emph{observation relation}, $\downarrow_{\mathcal N}$, over a set
of names, $\mathcal N$, is the smallest relation satisfying the rules
below.

\infrule[Out-barb]{y \in {\mathcal N}, \; x \nameeq y}
		  {\outputp{x}{v} \downarrow_{\mathcal N} x}
\infrule[Par-barb]{\mbox{$P\downarrow_{\mathcal N} x$ or $Q\downarrow_{\mathcal N} x$}}
		  {\binpar{P}{Q} \downarrow_{\mathcal N} x}

We write $P \Downarrow_{\mathcal N} x$ if there is $Q$ such that 
$P \wred Q$ and $Q \downarrow_{\mathcal N} x$.
\end{definition}

\begin{definition}
%\label{def.bbisim}
An  ${\mathcal N}$-\emph{barbed bisimulation} over a set of names, ${\mathcal N}$, is a symmetric binary relation 
${\mathcal S}_{\mathcal N}$ between agents such that $P\rel{S}_{\mathcal N}Q$ implies:
\begin{enumerate}
\item If $P \red P'$ then $Q \wred Q'$ and $P'\rel{S}_{\mathcal N} Q'$.
\item If $P\downarrow_{\mathcal N} x$, then $Q\Downarrow_{\mathcal N} x$.
\end{enumerate}
$P$ is ${\mathcal N}$-barbed bisimilar to $Q$, written
$P \wbbisim_{\mathcal N} Q$, if $P \rel{S}_{\mathcal N} Q$ for some ${\mathcal N}$-barbed bisimulation ${\mathcal S}_{\mathcal N}$.
\end{definition}

$\mathcal{R} \subseteq \pi \times \pi$

$P \mathcal{R} Q => \forall P'. P \red P' \Rightarrow \exists Q'. Q \red Q', P' \mathcal{R} Q'$

$P \vdash x \Rightarrow Q \vdash x$

\begin{mathpar}
  \inferrule*[lab=Out-barb]{x \nameeq y}{{y}!\langle{Q}\rangle \vdash x}
  \and
  \inferrule*[lab=Par-barb]{\mbox{$P\vdash x$ or $Q\vdash x$}}{\binpar{P}{Q} \vdash x}
\end{mathpar}

\subsubsection{Contexts}

One of the principle advantages of computational calculi like the
$\pi$-calculus is a well-defined notion of context,
contextual-equivalence and a correlation between
contextual-equivalence and notions of bisimulation. The notion of
context allows the decomposition of a process into (sub-)process and
its syntactic environment, its context. Thus, a context may be
thought of as a process with a ``hole'' (written $\Box$) in it. The
application of a context $M$ to a process $P$, written $M[P]$, is
tantamount to filling the hole in $M$ with $P$. In this paper we do
not need the full weight of this theory, but do make use of the notion
of context in the proof the main theorem. 

\begin{mathpar}
  \inferrule* [lab=summation] {} {{M_{M},M_{N}} \bc \Box \;|\; x.M_{A} \;|\; M_{M}+M_{N}}
  \and
  \inferrule* [lab=agent] {} {{M_{A}} \bc (\vec{x})M_{P} \;| \; \clift{P_0,\ldots,M_{P},\ldots,P_N}}
  \and \\
  \inferrule* [lab=process] {} {{M_{P}} \bc M_{N} \;| \;P|M_{P} }
\end{mathpar} 

\begin{mathpar}
  \inferrule* [lab=sychronization] {} {M_{N} \bc \Box \;|\; x?M_{F} \;|\; x!M_{C}}
  \and
  \inferrule* [lab=abstraction] {} {{M_{F}} \bc (x)M_{P} }
  \and
  \inferrule* [lab=concretion] {} {{M_{C}} \bc \langle M_{P} \rangle }
  \and \\
  \inferrule* [lab=process] {} {{M_{P}} \bc M_{N} \;| \;P|M_{P} }
\end{mathpar}

\begin{definition}[contextual application] Given a context $M$, and
  process $P$, we define the \emph{contextual application}, $M[P] :=
  M\{P/\Box\}$. That is, the contextual application of M to P is the
  substitution of $P$ for $\Box$ in $M$.
\end{definition}

$\meaningof{-} : L \to \mathcal{P}(\pi)$

\begin{mathpar}
  \inferrule* [lab=collection] {} {\meaningof{true} = \pi, \and \meaningof{~E} = \pi \setminus \meaningof{E}, \and \meaningof{E_{1} \& E_{2}} = \meaningof{E_{1}} \cap \meaningof{E_{2}}}
\end{mathpar}

\begin{mathpar}
  \inferrule* [lab=structure] {} {\meaningof{0} = \{ P \in \pi | P \equiv 0 \}, \and \\ \meaningof{E_1 | E_2} = \{ P \in \pi | P \equiv P_{1} | P_{2}, P_{1} \in \meaningof{E_{1}}, P_{2} \in \meaningof{E_2}\} }
\end{mathpar}

\begin{mathpar}
 \inferrule* [lab=behavior] {} {\meaningof{\langle a?b \rangle E} = \{ P \in \pi | P \equiv Q | u?(y)P', \\ \and \\\\ \and \\ \;\;\; u \in \meaningof{a}, \forall z.P'\{z/y\} \in \meaningof{E\{z/b\}}\}, \and \\ \meaningof{a!E} = \{ P \in \pi | P \equiv Q | x!\langle P' \rangle, x \in \meaningof{a} P' \in \meaningof{E}\} }
\end{mathpar}

\begin{mathpar}
 \inferrule* [lab=nominal] {} {\meaningof{\quotep{E}} = \{ \quotep{P} \in \quotep{\pi} | P \in \meaningof{E} \}, \and \meaningof{\quotep{P}} = \{ \quotep{Q} \in \quotep{\pi} | P \equiv Q \} \and \\ \meaningof{@\quotep{E}} = \{ P \in \pi | P \equiv @x, x \in \meaningof{E} \}}
\end{mathpar}

\begin{eqnarray*}
  \\
  \meaningof{-} : TS \to ST
\end{eqnarray*}

\begin{eqnarray*}
  \\
  L : TS \to ST
\end{eqnarray*}

\begin{eqnarray*}
  \\
  P \models E \iff P \in \meaningof{E}
\end{eqnarray*}

\begin{eqnarray*}
  P \approx_{L} Q \iff \forall E \in L. P \models E \iff Q \models E
\end{eqnarray*}

\begin{eqnarray*}
  P \approx_{K} Q
\end{eqnarray*}

\begin{eqnarray*}
  P \approx Q
\end{eqnarray*}

$\approx_{K} = \approx = \approx_{L}$

\subsubsection{Contextual duality}

Note that contexts extend the quotation operation to a family of
operations from processes to names. Given a context, $M$, we can
define a \emph{nominal context}, $\quotep{M}$ by $\quotep{M}[P] :=
\quotep{M[P]}$. To foreshadow what is to come we observe that these
operations enjoy a duality with processes very much like the duality
between vectors and maps from vectors to scalars.

Further, because the calculus is essentially higher-order, we have a
correspondence between contexts and processes. More specifically,
given a name $x$ and a context $M$ we can construct $M^{*}_{x}$ such
that 

\begin{mathpar}
  M^{*}_{x} | \lift{x}{P} \red M[P]
\end{mathpar}

namely,

\begin{mathpar}
  M^{*}_{x} := x?(u).M[\dropn{u}]
\end{mathpar}

The dependence of $M^{*}_{x}$ on a name makes it an abstraction, 

\begin{mathpar}
  M^{*} := (x)x?(u).M[\dropn{u}]
\end{mathpar}

\subsection{Additional notation}

It will sometimes be convenient to denote the process a name
quotes. We already have the notation $x = \quotep{P}$, but it will be
convenient to introduce an alternate notation, $\procn{x}$, when we
want to emphasize the connection to the use of the name. Note that, by
virtue of name equivalence, $\quotep{\procn{x}} \nameeq x$; so, the
notation is consistent with previous definitions.

Further, because names have structure it is possible to effect
substitutions on the basis of that structure. This means we need to
upgrade our notation for substitutions, which we accomplish by
adapting comprehension notation. Thus,

\begin{mathpar}
  P\{ y / x : x \in S \}
\end{mathpar}

is interpreted to mean the process derived from P by replacing (in a
capture-avoiding manner) each occurrence of $x$ in $S$ by $y$. For example,

\begin{mathpar}
  P\{ \quotep{\procn{x}|\procn{x}} / x : x \in \freenames{P} \}
\end{mathpar}

will replace each (occurrence) of a free name $x$ in $P$ by
$\quotep{\procn{x}|\procn{x}}$.

Also, we will avail ourselves of the notation $x^{L}$ and $x^{R}$ to
denote injections of a name into disjoint copies of the name
space. There are numerous ways to accomplish this. One example can be
found in \cite{MeredithR05}. This notation overloads to vectors of
names: $\vec{x}^{\pi} := (x_{i}^{\pi} \; : \; 0 \leq i < |\vec{x}| )$ where $\pi \in \{L,R\}$.

We also use $P^{\Box} := P|\Box$.

In \cite{MeredithR05} an interpretation of the new operator is
given. It turns out that there are several possible interpretations
all enjoying the requisite algebraic properties of the operator (see
\cite{milner91polyadicpi}). We will therefore make liberal use of
$(\nu\; \vec{x})P$.

% subsection the_syntax_and_semantics_of_the_notation_system (end)   

\input{qm2pi.qmops} 

\input{qm2pi.sterngerlach} 

\input{qm2pi.metric} 

% section concurrent_process_calculi (end)

%\input{qm2pi.proofsketch}

% section proof sketch (end)

%\input{qm2pi.slviaknots} 

% section spatial logic via knots (end)

\input{qm2pi.conclusion}

% section conclusion (end)

%\input{qm2pi.dtcodes} 

% section wiring algorithm (end)

\input{qm2pi.ack} 

% section acknowledgments (end)

\newpage


\bibliographystyle{plain}   
\bibliography{../../biblios/main.bib}

\input{qm2pi.rhodetails}

\end{document}



% section proof sketch (end)

%\section{Unlikely characters: spatial logic for
  knots}\label{sub:characteristic_formulae} % (fold)

Associated to the mobile process calculi are a family of logics known
as the Hennessy-Milner logics. These logics typically enjoy a
semantics interpreting formulae as sets of processes that when
factored through the encoding outlined above allows an identification
of classes of knots with logical formulae. In the context of this
encoding the sub-family known as the spatial logics \cite{CairesC03}
\cite{CairesC04} \cite{Caires04} are of particular interest providing
several important features for expressing and reasoning about
properties (i.e. classes) of knots. We hint here at how this may be done.

%\begin{description}
%\item [structural connectives] 
\subsubsection{Structural connectives} The spatial logics enjoy
structural connectives corresponding, at the logical level, to the
parallel composition ($P | Q$) and new name ($(\nu \; x)P$)
connectives for processes. As illustrated in the examples below, these
connectives are extremely expressive given the shape of our encoding.
%\item [decideable satisfaction]

\subsubsection{Decideable satisfaction}
In \cite{Caires04} the satisfaction relation is shown to be decideable
for a rich class of processes. It further turns out that the image of
the our encoding is a proper subset of that class. This result
provides the basis for an algorithm by which to search for knots
enjoying a given property.
%\item [characteristic formulae]

\subsubsection{Characteristic formulae}
In the same paper \cite{Caires04} , Caires presents a means of calculating
characteristic formulae, selecting equivalence classes of processes
up to a pre--specified depth limit on the support set of names. Composed with our
encoding, this characteristic formula can be used to select
characteristic formulae for knots.
%\end{description}

\subsubsection{Spatial logic formulae}

The grammar below (segmented for comprehension) summarizes the syntax
of spatial logic formulae. We employ illustrative examples in the
sequel to provide an intuitive understanding of their meaning
referring the reader to \cite{Caires04} for a more detailed explication
of the semantics.

\begin{mathpar}
  \inferrule* [lab=boolean] {} {{A,B} \bc T \;|\; \neg A \;|\; A \wedge B \;|\; \eta = \eta'}
  \and
  \inferrule* [lab=spatial] {} {|\; \pzero \;|\; A | B \;|\; x \text{\textregistered} A \;|\; \forall x . A \;|\;  H x . A}
  \and
  \inferrule* [lab=behavioral] {} {|\; \alpha . A}
  \and 
  \inferrule* [lab=recursion] {} {|\; X(\vec{u}) \;|\; \mu X(\vec{u}) . A}
  \and
  \inferrule* [lab=action] {} {\alpha \bc \langle x?(\vec{y}) \rangle \;|\; \langle x!(\vec{y}) \rangle \;|\; \langle \tau \rangle}
  \and 
  \inferrule* [lab=name] {} {\eta \bc x \;|\; \tau}
\end{mathpar} 

% subsection characteristic_formulae (end)   	 

\subsection{Example formulae}\label{sub:example_formulae_} % (fold)

\subsubsection{Crossing as formula.}
% 
% \begin{align*}
%   \frac{d}{dx} \sin x &= \cos x 
%   & \frac{d}{dx} e^x &= e^x \\
%   \frac{d}{dx} \cos x &= - \sin x 
%   & \frac{d}{dx} \log x &= \frac{1}{x} \\
% \end{align*} 

\begin{align*}
 \mu C(x_{0},x_{1},y_{0},y_{1},u).&(\langle x_{0}?(z) \rangle(\langle u! \rangle\langle y_{1}!z \rangle C(x_{0},x_{1},y_{0},y_{1},u)) & \\
  & \wedge \langle y_{1}?(z) \rangle (\langle u! \rangle \langle x_{0}!z \rangle C(x_{0},x_{1},y_{0},y_{1},u)) & \\
  & \wedge \langle x_{1}?(z) \rangle (\langle u? \rangle \langle y_{0}!z \rangle C(x_{0},x_{1},y_{0},y_{1},u)) & \\
  & \wedge \langle y_{0}?(z) \rangle (\langle u? \rangle \langle x_{1}!z \rangle C(x_{0},x_{1},y_{0},y_{1},u))) &
\end{align*}

The lexicographical similarity between the shape of this formulae and
the shape of definition of the process representing a crossing reveals
the intuitive meaning of this formulae. It describes the capabilities
of a process that has the right to represent a crossing. For example
it picks out processes that may perform an input on the port $x_0$ in
its initial menu of capabilities. What differentiates the formula
from the process, however, is that the crossing process is the
smallest candidate to satisfy the formula. Infinitely many other
processes -- with internal behavior hidden behind this interface, so
to speak -- also satisfy this formula. Even this simple formula,
then, can be seen to open a new view onto knots, providing a
computational interpretation of \emph{virtual} knots.

Note that this formula is derived by hand. A similar formula can be
derived by employing Caires' calculation of characteristic formula
\cite{Caires04} to the process representing a crossing. In light of
this discussion, we let
$\meaningof{C}_{\phi}(x0,x1,y0,y1,u)$ denote a formula specifying the
dynamics we wish to capture of a crossing. To guarantee we preserve
the shape of the interface and minimal semantics we demand that
$\meaningof{C}_{\phi}(x0,x1,y0,y1,u) \Rightarrow
\textbf{C}(x0,x1,y0,y1,u)$ where $\textbf{C}(x0,x1,y0,y1,u)$ denotes
the formula above.
                            
\subsubsection{Crossing number constraints.}
The moral content of the context lemma (Lemma \ref{context}) is that the notion of
``locality'' in the Reidemeister moves is effectively captured by the
parallel composition operator of the process calculus. This intuition
extends through the logic. Given a formula,
$\meaningof{C}_{\phi}(x0,x1,y0,y1,u)$, we can use the structural
connectives to specify constraints on crossing numbers, such as at
least $n$ crossings, or exactly $n$ crossings.
\begin{mathpar}
  \inferrule* [lab=at-least-n] {} { K^{\geq n}_{\phi}(\vec{xs},\vec{ys}) := \Pi_{i=0}^{n-1} Hu . \meaningof{C}_{\phi}(xs_i,ys_i,u) | T }
  \and 
  \inferrule* [lab=exactly-n] {} { K^{= n}_{\phi}(\vec{xs},\vec{ys}) := \Pi_{i=0}^{n-1} Hu . \meaningof{C}_{\phi}(xs_i,ys_i,u) | \neg (\forall x_0,y_0,x_1,y_1,u . \meaningof{C}_{\phi}(x_0,y_0,x_1,y_1,u) | T) }
\end{mathpar}

To round out this section, recall that the encoding of an $n$-crossing
knot decomposes into a parallel composition of $n$ \emph{copies} of a
crossing process together with a wiring harness. To specify different
knot classes with the same crossing number amounts to specifying
logical constraints on the wiring harness. In the interest of space,
we defer examples to a forthcoming paper. Suffice it to say that both
the conditions ``alternating knot'' and ``contains the tangle
corresponding to 5/3'' are expressible. For example, it is possible to
calculate the characteristic formula of a process corresponding to the
tangle 5/3 and conjoin it into the classifying formula via the
composition connective of the logic.

Finally, we wish to observe that it is entirely within reason to
contemplate a more domain-specific version of spatial logic tailored
to the shape of processes in the image of the encoding. Such a
domain-specific logic would have a better claim to the title formal
language of knot properties.

% subsection example_formulae_ (end)

% section knots_as_processes (end) 

% section spatial logic via knots (end)

\section{Conclusions and future work}

\paragraph{Testing physical space}
You, gentle reader, may wonder why of all the theorems to be proved
given this set up we pick the one above. In some sense it's hardly
central to quantum mechanics. We see it as central in the sense that
it firmly establishes a notion of physical space arising from a notion
of the equivalence of behavior. Relating bisimulation to a metric is a
big step forward, but one is faced with interpreting the relationship
of that metric space to something more physical. Quantum mechanical
notions of ``physical'' space are still far from intuitive, but by
relating this idea of distance as testing to calculations that predict
physical circumstances we are making a not insignificant step forward
toward an understanding of the physical space we inhabit as
essentially dynamic.

\paragraph{Effectivity and simulation}
One of the observations we have yet to make is that the entire program
spelled out here is effective. We have built various interpreters for
the reflective calculus at work in this interpretation. In principle,
then, we can simulate quantum mechanics on a computer. The place where
the simulation may lose fidelity is the infinitely branching summation
for the annihilator.

In this connection i also want to point out that the evaluation style
calculation of the inner product puts the non-determinism of the
summation right at the heart of measurement. This suggests that
Milner's original reduction-based formulation of the dynamics of his
calculi in terms of sums was not just notationally suggestive of a
notion of measure-and-continue but captured some significant part of
the physics.

\paragraph{Quantum continuations}
In light of this last observation i want to point out that the
predominant account of quantum mechanics is missing a key aspect of a
truly compositional story of the physical situation. In a real lab,
when a measurement is made the observation can be made to feed into
another device that then makes another measurement conditioned on the
results of the first. This means that after the superposition was
collapsed the entire experimental set up remained in
superposition. While QM offers a means of writing this down it doesn't
quite line up well with the well-trodden formulation of computation
and continuation that we see so succinctly expressed in Milner's
calculi. This suggests that there might be advantages to this account
of dynamics waiting to be explored.

\paragraph{Quantum logic}
In this connection, we also note that by virtue of having the
Hennessy-Milner construction, we can pull the construction through the
interpretation of QM. This gives us a natural candidate for a quantum
logic that enjoys an extremely tight connection with it's domain of
interpretation, making the construction much less ad hoc (rather it is
the image of functor!).

\paragraph{Quantum probabiity}
i have questions about the basis of the interpretation of inner
product as probability amplitude. In particular, using which
axiomatization of probability theory does the notion of probability
amplitude earn the right to be so dubbed? In other words, where is the
proof that the operation for calculating a probability amplitude (and
then squaring) satisfies the axioms of what it means to calculate a
probability? Even if such a proof exists (i have yet to find it in the
literature), i wonder if it might not be possible to turn things on
their heads. Can we view the calculation of the probability amplitude
as an axiomatization of probability? If so, then the definition we
give for calculating probability amplitude may provide the basis for
an \emph{effective} theory of probability.

\paragraph{Quantum vs ``biological'' information}
Finally, i want to conclude with a more philosophical observation. At
a recent workshop in which QM was a predominant topic i noticed
something about quantum information. The speaker was giving a riveting
discussion of axiomatic QM and showing how properties of ``no
cloning'' and ``no deleting'' emerged as consequences of the
axiomatization. Theorems of this form are necessary to give us a sense
of confidence that our axioms characterize the physical theory. What
struck me, though, was that if quantum information is neither erasable
nor replicable it is markedly different from \emph{life}. Two of the
things we know about life is that

\begin{itemize}
  \item it ends;
  \item to gain some measure of persistence, to transcend it's
    finitude it is imminently copyable.
\end{itemize}

Both of these qualities are summarized succinctly in the aphorism: all
flesh is grass. For me these two kinds of ``information'' -- call them
quantum and biological -- are end points on a spectrum of strategies
for persistence. At one end, we have those curious entities that enjoy
uniqueness and permanence; at the other, we have those who in the face
of a certain end and an uncertain present make a go of passing
something on. To me one of the more remarkable aspects of the latter
strategy is that in the presence of noise (and certain features of
copying) we get a kind of dynamism, a chance for improvement against a
given persistent condition.

% subsection other_calculi_other_bisimulations_and_geometry_as_behavior (end)




% section conclusion (end)

%\documentclass[12pt]{llncs}
%\documentclass{jktr}

\usepackage[pdftex]{hyperref}                   
\usepackage {listings}
\usepackage {mathpartir}
\usepackage{bcprules}
%\usepackage{listings}
                       
\usepackage{graphicx} 
%\usepackage[margins=2.5cm,nohead,nofoot]{geometry}
%\usepackage{geometry}
\usepackage{amsfonts}
\usepackage{amstext}
\usepackage{latexsym}
\usepackage{amssymb}
\usepackage{color}


%\include{myPreamble}
\include{qm2pi.local} 

%\ifpdf
%\usepackage[pdftex]{graphicx}
%\else
%\usepackage{graphicx}
%\fi

 % \ifpdf
%  \usepackage{pdfsync}
%  \if


%\title{Brief Article}
%\author{David F. Snyder}
%\author{L.G. Meredith}

%\address{Dept. of Math., Texas State University--San Marcos, San Marcos, TX 78666}
       
\pagestyle{empty}


\begin{document}

\lstset{language=[Objective]Caml,frame=shadowbox}

\input{qm2pi.front}

% section front matter (end)

\input{qm2pi.intro} 
 
% section introduction (end)

% \input{qm2pi.knotations} 

% section notation (end)

\input{qm2pi.process.calculi} 

% section concurrent_process_calculi_and_spatial_logics_ (end)
    
%\input{qm2pi.knots2pi} 

%\input{qm2pi.trefoil} 

%\input{qm2pi.mainthm} 

% subsection basic_interpretation (end)

%\input{qm2pi.rho.presentation} 
\subsection{The syntax and semantics of the notation system}\label{sub:the_syntax_and_semantics_of_the_notation_system} % (fold)

We now summarize a technical presentation of the calculus that
embodies our theory of dynamics. The typical presentation of such a
calculus follows the style of giving generators and relations on
them. The grammar, below, describing term constructors, freely
generates the set of processes, $\Proc$. This set is then quotiented
by a relation known as structural congruence and it is over this set
that the notion of dynamics is expressed. This presentation is
essentially that of \cite{MeredithR05} with the addition of
polyadicity and summation. For readability we have relegated some of
the technical subtleties to an appendix.

\subsubsection{Process grammar}\label{subsub:process_grammar}

\begin{mathpar}
  \inferrule* [lab=synchronization] {} {{M} \bc \pzero \;|\; x?F \;|\; x!C }
  \and
  \inferrule* [lab=abstraction] {} {{F} \bc (x)P}
  \and
  \inferrule* [lab=concretion] {} {{C} \bc \langle Q \rangle}
  \and
  \inferrule* [lab=process] {} {{P,Q} \bc M \;| \;P|Q \;|\; @{x}}
  \and
  \inferrule* [lab=name] {} {{x} \bc \quotep{P}}
\end{mathpar} 

Note that $\vec{x}$ (resp. $\vec{P}$) denotes a vector of names
(resp. processes) of length $|\vec{x}|$ (resp. $|\vec{P}|$). We adopt
the following useful abbreviations.

\begin{mathpar}
   x?(\vec{y}).P := x.(\vec{y})P \and  x\clift{\vec{P}} := x.\clift{\vec{P}}
   \and x!(y) := \lift{x}{\dropn{y}}
   \and \Pi_{i=0}^{n-1}P_i := P_0 | \ldots | P_{n-1}
\end{mathpar}

\subsubsection{Structural congruence}

\paragraph{Free and bound names and alpha-equivalence.} At the
core of structural equivalence is alpha-equivalence which identifies
process that are the same up to a change of variable. Formally, we
recognize the distinction between free and bound names. The free names
of a process, $\freenames{P}$, may be calculated recursively as
follows:

\begin{mathpar}
\freenames{\pzero} := \emptyset
  \and \\
  \freenames{x?(y).P} := \{ x \} \cup (\freenames{P} \setminus \{ y \})
  \and 
  \freenames{x!\langle P \rangle} := \{ x \} \cup \{ P \} 
  \and \\
  \freenames{P|Q} := \freenames{P} \cup \freenames{Q}
  \and \\
  \freenames{@{x}} := \{ x \}
\end{mathpar}

$\pi$
$\quotep{\pi}$

$\freenames{-} : \pi \to \mathcal{P}(\quotep{\pi})$

\begin{eqnarray*}
  \freenames{\pzero} & := & \emptyset \\
  \freenames{x?(y).P} & := & \{ x \} \cup (\freenames{P} \setminus \{ y \}) \\
  \freenames{x!\langle P \rangle} & := & \{ x \} \cup \{ P \} \\
  \freenames{P|Q} & := & \freenames{P} \cup \freenames{Q} \\
  \freenames{\dropn{x}} & := & \{ x \}
\end{eqnarray*}

The bound names of a process, $\boundnames{P}$, are those names occurring in $P$
that are not free. For example, in $x?(y).0$, the name $x$ is free, while $y$ is bound.

\begin{mathpar}
  \inferrule* [lab=monoidal-laws] {} { P|Q \equiv Q|P \and P|0 \equiv P \and P|(Q|R) \equiv (P|Q)|R }
\end{mathpar}

\begin{mathpar}
  \inferrule* [lab=alpha-equivalence] {} { (x)P \equiv (y)P\{y/x\} \and y \not\in \freenames{P} }
\end{mathpar}

\begin{definition}
Then two processes, $P,Q$, are alpha-equivalent if $P = Q\{\vec{y}/\vec{x}\}$ for
some $\vec{x} \in \boundnames{Q},\vec{y} \in \boundnames{P}$, where $Q\{\vec{y}/\vec{x}\}$
denotes the capture-avoiding substitution of $\vec{y}$ for $\vec{x}$ in $Q$.
\end{definition}

\begin{definition}
  The {\em structural congruence} \cite{SangiorgiWalker} , $\equiv$,
  between processes is the least congruence containing
  alpha-equivalence, satisfying the abelian monoid laws
  (associativity, commutativity and $\pzero$ as identity) for parallel
  composition $|$ and for summation $+$.
\end{definition}

\subsection{Name equivalence}

We take name equivalence, written $\nameeq$, to be the smallest
equivalence relation generated by the following rules.

\begin{mathpar}
\inferrule*[lab=Quote-drop]
{ }
{ \quotep{@{x}} \nameeq x }

\inferrule*[lab=Struct-equiv]
{ P \scong Q }
{ \quotep{P} \nameeq \quotep{Q} }
\end{mathpar}

The astute reader will have noticed that the mutual recursion of names
and processes imposes a mutual recursion on alpha-equivalence and
structural equivalence via name-equivalence. Fortunately, all of this
works out pleasantly and we may calculate in the natural way, free of
concern. The reader interested in the details is referred to the
appendix \ref{appendix:rho_details}.

\subsection{Substitution}

We use $\Proc$ for the set of processes, $\QProc$ for the set of
names, and $\id{\{}\vec{y} / \vec{x} \id{\}}$ to denote partial maps,
$s : \QProc \rightarrow \QProc$. A map, $s$ lifts, uniquely, to a map
on process terms, $\widehat{s} : \Proc \rightarrow \Proc$ by the
following equations.

\begin{mathpar}
  (0) \psubstp{Q}{P} := 0 \\
  (R \juxtap S) \psubstp{Q}{P}
  :=    
  (R)\psubstp{Q}{P} \juxtap (S) \psubstp{Q}{P} \\
  (x?(y).R) \psubstp{Q}{P}    
  :=    
  (x)\substp{Q}{P} (z)\concat( (R \psubstn{z}{y}) \psubstp{Q}{P} ) \\
  (\lift{x}{R}) \psubstp{Q}{P}  
  :=
  \lift{(x)\substp{Q}{P}}{ R \psubstp{Q}{P} } \\
%   (\dropn{x})  \psubstp{Q}{P}       
%   := 
%   \left\{ 
%     \begin{array}{ccc} 
%       \dropn{\quotep{Q}} & & x \nameeq \quotep{P} \\
%       \dropn{x} & & otherwise \\
%     \end{array}
%   \right. 
  (\dropn{x})  \psubstp{Q}{P}       
  := 
  \left\{ 
    \begin{array}{ccc} 
      Q & & x \nameeq \quotep{P} \\
      \dropn{x} & & otherwise \\
    \end{array}
  \right.
\end{mathpar}
 

where

\begin{eqnarray}
  (x)\id{\{} \lpquote Q \rpquote / \lpquote P \rpquote \id{\}}            = 
  \left\{ 
    \begin{array}{ccc}
      \lpquote Q \rpquote & & x \nameeq \lpquote P \rpquote \\
      x & & otherwise \\
    \end{array}
  \right. \nonumber
\end{eqnarray}

and $z$ is chosen distinct from $\quotep{P}$, $\quotep{Q}$, the free
names in $Q$, and all the names in $R$. Our $\alpha$-equivalence will
be built in the standard way from this substitution.

\begin{remark}\label{rem:no_self_referential_names}
  One consequence of these definitions is that $\forall P. \quotep{P}
  \not\in \freenames{P}$.
\end{remark}

\subsection{ Dynamic quote: an example }

Anticipating something of what's to come, consider applying the
substitution, $\widehat{\id{\{}u / z \id{\}}}$, to the following pair
of processes, $\lift{w}{y!(z)}$ and $w[ \lpquote y!(z) \rpquote ]$.

\begin{eqnarray}
	\lift{w}{y!(z)}\widehat{\id{\{}u / z \id{\}}}
		& = &
		\lift{w}{y!(u)} \nonumber\\
	w[ \lpquote y!(z) \rpquote ] \widehat{ \id{\{}u / z \id{\}} }
		& = &
		w[ \lpquote y!(z) \rpquote ] \nonumber
\end{eqnarray}

Because the body of the process between quotes is impervious to
substitution, we get radically different answers. In fact, by
examining the first process in an input context,
e.g. $x?(z).\lift{w}{y!(z)}$, we see that the process under the lift
operator may be shaped by prefixed inputs binding a name inside it. In
this sense, the lift operator will be seen as a way to dynamically
construct processes before reifying them as names.

Finally equipped with these standard features we can present the
dynamics of the calculus.

\subsubsection{Operational semantics} 

Finally, we introduce the computational dynamics. What marks these
algebras as distinct from other more traditionally studied algebraic
structures, e.g. vector spaces or polynomial rings, is the manner in
which dynamics is captured. In traditional structures, dynamics is typically
expressed through morphisms between such structures, as in linear maps
between vector spaces or morphisms between rings. In algebras
associated with the semantics of computation, the dynamics is
expressed as part of the algebraic structure itself, through a
reduction reduction relation typically denoted by $\red$. Below, we
give a recursive presentation of this relation for the calculus used
in the encoding.

$\red \subseteq \pi \times \pi$
$\red : \pi \to \mathcal{P}(\pi)$

\begin{mathpar}
  \inferrule* [lab=Comm] { \textsf{match}( x_{src}, x_{trgt} ) } { x_{trgt}?(y)P \; | \; x_{src}!\langle {Q} \rangle \red P\{\quotep{Q}/y}\} }
  \and \\
  \inferrule* [lab=Par] {{P} \red {P}'} {{{P} | {Q}} \red {{P}' | {Q}}}
  \and
  \inferrule* [lab=Equiv]{{{P} \scong {P}'} \andalso {{P}' \red {Q}'} \andalso {{Q}' \scong {Q}}}{{P} \red {Q}}
\end{mathpar}

\begin{eqnarray*}
  match_{\equiv} (\quotep{P},\quotep{Q}) & := & P \equiv Q \\
  match_{\dagger}(\quotep{P},\quotep{Q}) & := & \forall R. P|Q \red^{*} R => R \red^{*} 0 \\
  match_{K}(\quotep{P},\quotep{Q}) & := & K \mbox{ for some context } K
\end{eqnarray*}

$u?(x)P | u!\langle Q \rangle \red P\{\quotep{Q}/x\}$

%We write $\wred$ for $\red^*$, and $P\red$ if $\exists Q $ such that $ P \red Q$.
We write $P\red$ if $\exists Q $ such that $ P \red Q$ and $P\not\red$, otherwise.

\section{Replication}

As mentioned before, it is known that replication (and hence
recursion) can be implemented in a higher-order process algebra
\cite{SangiorgiWalker}. As our first example of calculation with the
machinery thus far presented we give the construction explicitly in
the {\rhoc}.

\begin{eqnarray}
	D_{x} & := & \prefix{x}{y}{(\binpar{\outputp{x}{y}}{@{y}})} \nonumber\\
	\bangp_{x}{P} & := & \binpar{{x}!\langle{\binpar{D_{x}}{P}}\rangle}{D_{x}} \nonumber
\end{eqnarray}

\begin{eqnarray}
	\bangp_{x}{P} & & \nonumber\\
	=
	& {x}!\langle{(\prefix{x}{y}{(\outputp{x}{y} | @{y})) | P}}\rangle 
	      | \prefix{x}{y}{(\outputp{x}{y} | @{y})} & \nonumber\\
	\red
	& (\outputp{x}{y} | @{y})\substn{\quotep{(\prefix{x}{y}{(@{y} | \outputp{x}{y})) | P}}}{y} & \nonumber\\
	=
	& \outputp{x}{\quotep{(\prefix{x}{y}{(\outputp{x}{y} | @{y})) | P}}}
	  | {(\prefix{x}{y}{(\outputp{x}{y} | @{y})) | P}} & \nonumber\\
	\red
	& \ldots & \nonumber\\
	\red^*
	& P | P | \ldots & \nonumber
\end{eqnarray}

Of course, this encoding, as an implementation, runs away, unfolding
$\bangp{P}$ eagerly. A lazier and more implementable replication
operator, restricted to input-guarded processes, may be obtained as follows.

\begin{eqnarray}
\bangp{\prefix{u}{v}{P}} 
	:= 
	\binpar{\lift{x}{\prefix{u}{v}{(\binpar{D(x)}{P})}}}{D(x)} \nonumber
\end{eqnarray}

\begin{remark}
  Note that the lazier definition still does not deal with summation
  or mixed summation (i.e. sums over input and output). The reader is
  invited to construct definitions of replication that deal with these
  features. 

  Further, the definitions are parameterized in a name, $x$. Can you,
  gentle reader, make a definition that eliminates this parameter and
  guarantees no accidental interaction between the replication
  machinery and the process being replicated -- i.e. no accidental
  sharing of names used by the process to get its work done and the
  name(s) used by the replication to effect copying. This latter
  revision of the definition of replication is crucial to obtaining
  the expected identity $!!P \sim !P$.
\end{remark}

\begin{remark}\label{rem:paradoxical_combinator}
  The reader familiar with the lambda calculus will have noticed the
  similarity between $D$ and the paradoxical combinator.

  [Ed. note: the existence of this seems to suggest we have to be more
  restrictive on the set of processes and names we admit if we are to
  support no-cloning.]
\end{remark}

\subsubsection{Bisimulation}

The computational dynamics gives rise to another kind of equivalence,
the equivalence of computational behavior. As previously mentioned
this is typically captured \emph{via} some form of bisimulation.

% The notion we use in this paper is weak barbed bisimulation
% \cite{milner91polyadicpi}.

The notion we use in this paper is derived from weak barbed
bisimulation \cite{milner91polyadicpi}. 

\begin{definition}
An \emph{observation relation}, $\downarrow_{\mathcal N}$, over a set
of names, $\mathcal N$, is the smallest relation satisfying the rules
below.

\infrule[Out-barb]{y \in {\mathcal N}, \; x \nameeq y}
		  {\outputp{x}{v} \downarrow_{\mathcal N} x}
\infrule[Par-barb]{\mbox{$P\downarrow_{\mathcal N} x$ or $Q\downarrow_{\mathcal N} x$}}
		  {\binpar{P}{Q} \downarrow_{\mathcal N} x}

We write $P \Downarrow_{\mathcal N} x$ if there is $Q$ such that 
$P \wred Q$ and $Q \downarrow_{\mathcal N} x$.
\end{definition}

\begin{definition}
%\label{def.bbisim}
An  ${\mathcal N}$-\emph{barbed bisimulation} over a set of names, ${\mathcal N}$, is a symmetric binary relation 
${\mathcal S}_{\mathcal N}$ between agents such that $P\rel{S}_{\mathcal N}Q$ implies:
\begin{enumerate}
\item If $P \red P'$ then $Q \wred Q'$ and $P'\rel{S}_{\mathcal N} Q'$.
\item If $P\downarrow_{\mathcal N} x$, then $Q\Downarrow_{\mathcal N} x$.
\end{enumerate}
$P$ is ${\mathcal N}$-barbed bisimilar to $Q$, written
$P \wbbisim_{\mathcal N} Q$, if $P \rel{S}_{\mathcal N} Q$ for some ${\mathcal N}$-barbed bisimulation ${\mathcal S}_{\mathcal N}$.
\end{definition}

$\mathcal{R} \subseteq \pi \times \pi$

$P \mathcal{R} Q => \forall P'. P \red P' \Rightarrow \exists Q'. Q \red Q', P' \mathcal{R} Q'$

$P \vdash x \Rightarrow Q \vdash x$

\begin{mathpar}
  \inferrule*[lab=Out-barb]{x \nameeq y}{{y}!\langle{Q}\rangle \vdash x}
  \and
  \inferrule*[lab=Par-barb]{\mbox{$P\vdash x$ or $Q\vdash x$}}{\binpar{P}{Q} \vdash x}
\end{mathpar}

\subsubsection{Contexts}

One of the principle advantages of computational calculi like the
$\pi$-calculus is a well-defined notion of context,
contextual-equivalence and a correlation between
contextual-equivalence and notions of bisimulation. The notion of
context allows the decomposition of a process into (sub-)process and
its syntactic environment, its context. Thus, a context may be
thought of as a process with a ``hole'' (written $\Box$) in it. The
application of a context $M$ to a process $P$, written $M[P]$, is
tantamount to filling the hole in $M$ with $P$. In this paper we do
not need the full weight of this theory, but do make use of the notion
of context in the proof the main theorem. 

\begin{mathpar}
  \inferrule* [lab=summation] {} {{M_{M},M_{N}} \bc \Box \;|\; x.M_{A} \;|\; M_{M}+M_{N}}
  \and
  \inferrule* [lab=agent] {} {{M_{A}} \bc (\vec{x})M_{P} \;| \; \clift{P_0,\ldots,M_{P},\ldots,P_N}}
  \and \\
  \inferrule* [lab=process] {} {{M_{P}} \bc M_{N} \;| \;P|M_{P} }
\end{mathpar} 

\begin{mathpar}
  \inferrule* [lab=sychronization] {} {M_{N} \bc \Box \;|\; x?M_{F} \;|\; x!M_{C}}
  \and
  \inferrule* [lab=abstraction] {} {{M_{F}} \bc (x)M_{P} }
  \and
  \inferrule* [lab=concretion] {} {{M_{C}} \bc \langle M_{P} \rangle }
  \and \\
  \inferrule* [lab=process] {} {{M_{P}} \bc M_{N} \;| \;P|M_{P} }
\end{mathpar}

\begin{definition}[contextual application] Given a context $M$, and
  process $P$, we define the \emph{contextual application}, $M[P] :=
  M\{P/\Box\}$. That is, the contextual application of M to P is the
  substitution of $P$ for $\Box$ in $M$.
\end{definition}

$\meaningof{-} : L \to \mathcal{P}(\pi)$

\begin{mathpar}
  \inferrule* [lab=collection] {} {\meaningof{true} = \pi, \and \meaningof{~E} = \pi \setminus \meaningof{E}, \and \meaningof{E_{1} \& E_{2}} = \meaningof{E_{1}} \cap \meaningof{E_{2}}}
\end{mathpar}

\begin{mathpar}
  \inferrule* [lab=structure] {} {\meaningof{0} = \{ P \in \pi | P \equiv 0 \}, \and \\ \meaningof{E_1 | E_2} = \{ P \in \pi | P \equiv P_{1} | P_{2}, P_{1} \in \meaningof{E_{1}}, P_{2} \in \meaningof{E_2}\} }
\end{mathpar}

\begin{mathpar}
 \inferrule* [lab=behavior] {} {\meaningof{\langle a?b \rangle E} = \{ P \in \pi | P \equiv Q | u?(y)P', \\ \and \\\\ \and \\ \;\;\; u \in \meaningof{a}, \forall z.P'\{z/y\} \in \meaningof{E\{z/b\}}\}, \and \\ \meaningof{a!E} = \{ P \in \pi | P \equiv Q | x!\langle P' \rangle, x \in \meaningof{a} P' \in \meaningof{E}\} }
\end{mathpar}

\begin{mathpar}
 \inferrule* [lab=nominal] {} {\meaningof{\quotep{E}} = \{ \quotep{P} \in \quotep{\pi} | P \in \meaningof{E} \}, \and \meaningof{\quotep{P}} = \{ \quotep{Q} \in \quotep{\pi} | P \equiv Q \} \and \\ \meaningof{@\quotep{E}} = \{ P \in \pi | P \equiv @x, x \in \meaningof{E} \}}
\end{mathpar}

\begin{eqnarray*}
  \\
  \meaningof{-} : TS \to ST
\end{eqnarray*}

\begin{eqnarray*}
  \\
  L : TS \to ST
\end{eqnarray*}

\begin{eqnarray*}
  \\
  P \models E \iff P \in \meaningof{E}
\end{eqnarray*}

\begin{eqnarray*}
  P \approx_{L} Q \iff \forall E \in L. P \models E \iff Q \models E
\end{eqnarray*}

\begin{eqnarray*}
  P \approx_{K} Q
\end{eqnarray*}

\begin{eqnarray*}
  P \approx Q
\end{eqnarray*}

$\approx_{K} = \approx = \approx_{L}$

\subsubsection{Contextual duality}

Note that contexts extend the quotation operation to a family of
operations from processes to names. Given a context, $M$, we can
define a \emph{nominal context}, $\quotep{M}$ by $\quotep{M}[P] :=
\quotep{M[P]}$. To foreshadow what is to come we observe that these
operations enjoy a duality with processes very much like the duality
between vectors and maps from vectors to scalars.

Further, because the calculus is essentially higher-order, we have a
correspondence between contexts and processes. More specifically,
given a name $x$ and a context $M$ we can construct $M^{*}_{x}$ such
that 

\begin{mathpar}
  M^{*}_{x} | \lift{x}{P} \red M[P]
\end{mathpar}

namely,

\begin{mathpar}
  M^{*}_{x} := x?(u).M[\dropn{u}]
\end{mathpar}

The dependence of $M^{*}_{x}$ on a name makes it an abstraction, 

\begin{mathpar}
  M^{*} := (x)x?(u).M[\dropn{u}]
\end{mathpar}

\subsection{Additional notation}

It will sometimes be convenient to denote the process a name
quotes. We already have the notation $x = \quotep{P}$, but it will be
convenient to introduce an alternate notation, $\procn{x}$, when we
want to emphasize the connection to the use of the name. Note that, by
virtue of name equivalence, $\quotep{\procn{x}} \nameeq x$; so, the
notation is consistent with previous definitions.

Further, because names have structure it is possible to effect
substitutions on the basis of that structure. This means we need to
upgrade our notation for substitutions, which we accomplish by
adapting comprehension notation. Thus,

\begin{mathpar}
  P\{ y / x : x \in S \}
\end{mathpar}

is interpreted to mean the process derived from P by replacing (in a
capture-avoiding manner) each occurrence of $x$ in $S$ by $y$. For example,

\begin{mathpar}
  P\{ \quotep{\procn{x}|\procn{x}} / x : x \in \freenames{P} \}
\end{mathpar}

will replace each (occurrence) of a free name $x$ in $P$ by
$\quotep{\procn{x}|\procn{x}}$.

Also, we will avail ourselves of the notation $x^{L}$ and $x^{R}$ to
denote injections of a name into disjoint copies of the name
space. There are numerous ways to accomplish this. One example can be
found in \cite{MeredithR05}. This notation overloads to vectors of
names: $\vec{x}^{\pi} := (x_{i}^{\pi} \; : \; 0 \leq i < |\vec{x}| )$ where $\pi \in \{L,R\}$.

We also use $P^{\Box} := P|\Box$.

In \cite{MeredithR05} an interpretation of the new operator is
given. It turns out that there are several possible interpretations
all enjoying the requisite algebraic properties of the operator (see
\cite{milner91polyadicpi}). We will therefore make liberal use of
$(\nu\; \vec{x})P$.

% subsection the_syntax_and_semantics_of_the_notation_system (end)   

\input{qm2pi.qmops} 

\input{qm2pi.sterngerlach} 

\input{qm2pi.metric} 

% section concurrent_process_calculi (end)

%\input{qm2pi.proofsketch}

% section proof sketch (end)

%\input{qm2pi.slviaknots} 

% section spatial logic via knots (end)

\input{qm2pi.conclusion}

% section conclusion (end)

%\input{qm2pi.dtcodes} 

% section wiring algorithm (end)

\input{qm2pi.ack} 

% section acknowledgments (end)

\newpage


\bibliographystyle{plain}   
\bibliography{../../biblios/main.bib}

\input{qm2pi.rhodetails}

\end{document}

 

% section wiring algorithm (end)

\documentclass[12pt]{llncs}
%\documentclass{jktr}

\usepackage[pdftex]{hyperref}                   
\usepackage {listings}
\usepackage {mathpartir}
\usepackage{bcprules}
%\usepackage{listings}
                       
\usepackage{graphicx} 
%\usepackage[margins=2.5cm,nohead,nofoot]{geometry}
%\usepackage{geometry}
\usepackage{amsfonts}
\usepackage{amstext}
\usepackage{latexsym}
\usepackage{amssymb}
\usepackage{color}


%\include{myPreamble}
\include{qm2pi.local} 

%\ifpdf
%\usepackage[pdftex]{graphicx}
%\else
%\usepackage{graphicx}
%\fi

 % \ifpdf
%  \usepackage{pdfsync}
%  \if


%\title{Brief Article}
%\author{David F. Snyder}
%\author{L.G. Meredith}

%\address{Dept. of Math., Texas State University--San Marcos, San Marcos, TX 78666}
       
\pagestyle{empty}


\begin{document}

\lstset{language=[Objective]Caml,frame=shadowbox}

\input{qm2pi.front}

% section front matter (end)

\input{qm2pi.intro} 
 
% section introduction (end)

% \input{qm2pi.knotations} 

% section notation (end)

\input{qm2pi.process.calculi} 

% section concurrent_process_calculi_and_spatial_logics_ (end)
    
%\input{qm2pi.knots2pi} 

%\input{qm2pi.trefoil} 

%\input{qm2pi.mainthm} 

% subsection basic_interpretation (end)

%\input{qm2pi.rho.presentation} 
\subsection{The syntax and semantics of the notation system}\label{sub:the_syntax_and_semantics_of_the_notation_system} % (fold)

We now summarize a technical presentation of the calculus that
embodies our theory of dynamics. The typical presentation of such a
calculus follows the style of giving generators and relations on
them. The grammar, below, describing term constructors, freely
generates the set of processes, $\Proc$. This set is then quotiented
by a relation known as structural congruence and it is over this set
that the notion of dynamics is expressed. This presentation is
essentially that of \cite{MeredithR05} with the addition of
polyadicity and summation. For readability we have relegated some of
the technical subtleties to an appendix.

\subsubsection{Process grammar}\label{subsub:process_grammar}

\begin{mathpar}
  \inferrule* [lab=synchronization] {} {{M} \bc \pzero \;|\; x?F \;|\; x!C }
  \and
  \inferrule* [lab=abstraction] {} {{F} \bc (x)P}
  \and
  \inferrule* [lab=concretion] {} {{C} \bc \langle Q \rangle}
  \and
  \inferrule* [lab=process] {} {{P,Q} \bc M \;| \;P|Q \;|\; @{x}}
  \and
  \inferrule* [lab=name] {} {{x} \bc \quotep{P}}
\end{mathpar} 

Note that $\vec{x}$ (resp. $\vec{P}$) denotes a vector of names
(resp. processes) of length $|\vec{x}|$ (resp. $|\vec{P}|$). We adopt
the following useful abbreviations.

\begin{mathpar}
   x?(\vec{y}).P := x.(\vec{y})P \and  x\clift{\vec{P}} := x.\clift{\vec{P}}
   \and x!(y) := \lift{x}{\dropn{y}}
   \and \Pi_{i=0}^{n-1}P_i := P_0 | \ldots | P_{n-1}
\end{mathpar}

\subsubsection{Structural congruence}

\paragraph{Free and bound names and alpha-equivalence.} At the
core of structural equivalence is alpha-equivalence which identifies
process that are the same up to a change of variable. Formally, we
recognize the distinction between free and bound names. The free names
of a process, $\freenames{P}$, may be calculated recursively as
follows:

\begin{mathpar}
\freenames{\pzero} := \emptyset
  \and \\
  \freenames{x?(y).P} := \{ x \} \cup (\freenames{P} \setminus \{ y \})
  \and 
  \freenames{x!\langle P \rangle} := \{ x \} \cup \{ P \} 
  \and \\
  \freenames{P|Q} := \freenames{P} \cup \freenames{Q}
  \and \\
  \freenames{@{x}} := \{ x \}
\end{mathpar}

$\pi$
$\quotep{\pi}$

$\freenames{-} : \pi \to \mathcal{P}(\quotep{\pi})$

\begin{eqnarray*}
  \freenames{\pzero} & := & \emptyset \\
  \freenames{x?(y).P} & := & \{ x \} \cup (\freenames{P} \setminus \{ y \}) \\
  \freenames{x!\langle P \rangle} & := & \{ x \} \cup \{ P \} \\
  \freenames{P|Q} & := & \freenames{P} \cup \freenames{Q} \\
  \freenames{\dropn{x}} & := & \{ x \}
\end{eqnarray*}

The bound names of a process, $\boundnames{P}$, are those names occurring in $P$
that are not free. For example, in $x?(y).0$, the name $x$ is free, while $y$ is bound.

\begin{mathpar}
  \inferrule* [lab=monoidal-laws] {} { P|Q \equiv Q|P \and P|0 \equiv P \and P|(Q|R) \equiv (P|Q)|R }
\end{mathpar}

\begin{mathpar}
  \inferrule* [lab=alpha-equivalence] {} { (x)P \equiv (y)P\{y/x\} \and y \not\in \freenames{P} }
\end{mathpar}

\begin{definition}
Then two processes, $P,Q$, are alpha-equivalent if $P = Q\{\vec{y}/\vec{x}\}$ for
some $\vec{x} \in \boundnames{Q},\vec{y} \in \boundnames{P}$, where $Q\{\vec{y}/\vec{x}\}$
denotes the capture-avoiding substitution of $\vec{y}$ for $\vec{x}$ in $Q$.
\end{definition}

\begin{definition}
  The {\em structural congruence} \cite{SangiorgiWalker} , $\equiv$,
  between processes is the least congruence containing
  alpha-equivalence, satisfying the abelian monoid laws
  (associativity, commutativity and $\pzero$ as identity) for parallel
  composition $|$ and for summation $+$.
\end{definition}

\subsection{Name equivalence}

We take name equivalence, written $\nameeq$, to be the smallest
equivalence relation generated by the following rules.

\begin{mathpar}
\inferrule*[lab=Quote-drop]
{ }
{ \quotep{@{x}} \nameeq x }

\inferrule*[lab=Struct-equiv]
{ P \scong Q }
{ \quotep{P} \nameeq \quotep{Q} }
\end{mathpar}

The astute reader will have noticed that the mutual recursion of names
and processes imposes a mutual recursion on alpha-equivalence and
structural equivalence via name-equivalence. Fortunately, all of this
works out pleasantly and we may calculate in the natural way, free of
concern. The reader interested in the details is referred to the
appendix \ref{appendix:rho_details}.

\subsection{Substitution}

We use $\Proc$ for the set of processes, $\QProc$ for the set of
names, and $\id{\{}\vec{y} / \vec{x} \id{\}}$ to denote partial maps,
$s : \QProc \rightarrow \QProc$. A map, $s$ lifts, uniquely, to a map
on process terms, $\widehat{s} : \Proc \rightarrow \Proc$ by the
following equations.

\begin{mathpar}
  (0) \psubstp{Q}{P} := 0 \\
  (R \juxtap S) \psubstp{Q}{P}
  :=    
  (R)\psubstp{Q}{P} \juxtap (S) \psubstp{Q}{P} \\
  (x?(y).R) \psubstp{Q}{P}    
  :=    
  (x)\substp{Q}{P} (z)\concat( (R \psubstn{z}{y}) \psubstp{Q}{P} ) \\
  (\lift{x}{R}) \psubstp{Q}{P}  
  :=
  \lift{(x)\substp{Q}{P}}{ R \psubstp{Q}{P} } \\
%   (\dropn{x})  \psubstp{Q}{P}       
%   := 
%   \left\{ 
%     \begin{array}{ccc} 
%       \dropn{\quotep{Q}} & & x \nameeq \quotep{P} \\
%       \dropn{x} & & otherwise \\
%     \end{array}
%   \right. 
  (\dropn{x})  \psubstp{Q}{P}       
  := 
  \left\{ 
    \begin{array}{ccc} 
      Q & & x \nameeq \quotep{P} \\
      \dropn{x} & & otherwise \\
    \end{array}
  \right.
\end{mathpar}
 

where

\begin{eqnarray}
  (x)\id{\{} \lpquote Q \rpquote / \lpquote P \rpquote \id{\}}            = 
  \left\{ 
    \begin{array}{ccc}
      \lpquote Q \rpquote & & x \nameeq \lpquote P \rpquote \\
      x & & otherwise \\
    \end{array}
  \right. \nonumber
\end{eqnarray}

and $z$ is chosen distinct from $\quotep{P}$, $\quotep{Q}$, the free
names in $Q$, and all the names in $R$. Our $\alpha$-equivalence will
be built in the standard way from this substitution.

\begin{remark}\label{rem:no_self_referential_names}
  One consequence of these definitions is that $\forall P. \quotep{P}
  \not\in \freenames{P}$.
\end{remark}

\subsection{ Dynamic quote: an example }

Anticipating something of what's to come, consider applying the
substitution, $\widehat{\id{\{}u / z \id{\}}}$, to the following pair
of processes, $\lift{w}{y!(z)}$ and $w[ \lpquote y!(z) \rpquote ]$.

\begin{eqnarray}
	\lift{w}{y!(z)}\widehat{\id{\{}u / z \id{\}}}
		& = &
		\lift{w}{y!(u)} \nonumber\\
	w[ \lpquote y!(z) \rpquote ] \widehat{ \id{\{}u / z \id{\}} }
		& = &
		w[ \lpquote y!(z) \rpquote ] \nonumber
\end{eqnarray}

Because the body of the process between quotes is impervious to
substitution, we get radically different answers. In fact, by
examining the first process in an input context,
e.g. $x?(z).\lift{w}{y!(z)}$, we see that the process under the lift
operator may be shaped by prefixed inputs binding a name inside it. In
this sense, the lift operator will be seen as a way to dynamically
construct processes before reifying them as names.

Finally equipped with these standard features we can present the
dynamics of the calculus.

\subsubsection{Operational semantics} 

Finally, we introduce the computational dynamics. What marks these
algebras as distinct from other more traditionally studied algebraic
structures, e.g. vector spaces or polynomial rings, is the manner in
which dynamics is captured. In traditional structures, dynamics is typically
expressed through morphisms between such structures, as in linear maps
between vector spaces or morphisms between rings. In algebras
associated with the semantics of computation, the dynamics is
expressed as part of the algebraic structure itself, through a
reduction reduction relation typically denoted by $\red$. Below, we
give a recursive presentation of this relation for the calculus used
in the encoding.

$\red \subseteq \pi \times \pi$
$\red : \pi \to \mathcal{P}(\pi)$

\begin{mathpar}
  \inferrule* [lab=Comm] { \textsf{match}( x_{src}, x_{trgt} ) } { x_{trgt}?(y)P \; | \; x_{src}!\langle {Q} \rangle \red P\{\quotep{Q}/y}\} }
  \and \\
  \inferrule* [lab=Par] {{P} \red {P}'} {{{P} | {Q}} \red {{P}' | {Q}}}
  \and
  \inferrule* [lab=Equiv]{{{P} \scong {P}'} \andalso {{P}' \red {Q}'} \andalso {{Q}' \scong {Q}}}{{P} \red {Q}}
\end{mathpar}

\begin{eqnarray*}
  match_{\equiv} (\quotep{P},\quotep{Q}) & := & P \equiv Q \\
  match_{\dagger}(\quotep{P},\quotep{Q}) & := & \forall R. P|Q \red^{*} R => R \red^{*} 0 \\
  match_{K}(\quotep{P},\quotep{Q}) & := & K \mbox{ for some context } K
\end{eqnarray*}

$u?(x)P | u!\langle Q \rangle \red P\{\quotep{Q}/x\}$

%We write $\wred$ for $\red^*$, and $P\red$ if $\exists Q $ such that $ P \red Q$.
We write $P\red$ if $\exists Q $ such that $ P \red Q$ and $P\not\red$, otherwise.

\section{Replication}

As mentioned before, it is known that replication (and hence
recursion) can be implemented in a higher-order process algebra
\cite{SangiorgiWalker}. As our first example of calculation with the
machinery thus far presented we give the construction explicitly in
the {\rhoc}.

\begin{eqnarray}
	D_{x} & := & \prefix{x}{y}{(\binpar{\outputp{x}{y}}{@{y}})} \nonumber\\
	\bangp_{x}{P} & := & \binpar{{x}!\langle{\binpar{D_{x}}{P}}\rangle}{D_{x}} \nonumber
\end{eqnarray}

\begin{eqnarray}
	\bangp_{x}{P} & & \nonumber\\
	=
	& {x}!\langle{(\prefix{x}{y}{(\outputp{x}{y} | @{y})) | P}}\rangle 
	      | \prefix{x}{y}{(\outputp{x}{y} | @{y})} & \nonumber\\
	\red
	& (\outputp{x}{y} | @{y})\substn{\quotep{(\prefix{x}{y}{(@{y} | \outputp{x}{y})) | P}}}{y} & \nonumber\\
	=
	& \outputp{x}{\quotep{(\prefix{x}{y}{(\outputp{x}{y} | @{y})) | P}}}
	  | {(\prefix{x}{y}{(\outputp{x}{y} | @{y})) | P}} & \nonumber\\
	\red
	& \ldots & \nonumber\\
	\red^*
	& P | P | \ldots & \nonumber
\end{eqnarray}

Of course, this encoding, as an implementation, runs away, unfolding
$\bangp{P}$ eagerly. A lazier and more implementable replication
operator, restricted to input-guarded processes, may be obtained as follows.

\begin{eqnarray}
\bangp{\prefix{u}{v}{P}} 
	:= 
	\binpar{\lift{x}{\prefix{u}{v}{(\binpar{D(x)}{P})}}}{D(x)} \nonumber
\end{eqnarray}

\begin{remark}
  Note that the lazier definition still does not deal with summation
  or mixed summation (i.e. sums over input and output). The reader is
  invited to construct definitions of replication that deal with these
  features. 

  Further, the definitions are parameterized in a name, $x$. Can you,
  gentle reader, make a definition that eliminates this parameter and
  guarantees no accidental interaction between the replication
  machinery and the process being replicated -- i.e. no accidental
  sharing of names used by the process to get its work done and the
  name(s) used by the replication to effect copying. This latter
  revision of the definition of replication is crucial to obtaining
  the expected identity $!!P \sim !P$.
\end{remark}

\begin{remark}\label{rem:paradoxical_combinator}
  The reader familiar with the lambda calculus will have noticed the
  similarity between $D$ and the paradoxical combinator.

  [Ed. note: the existence of this seems to suggest we have to be more
  restrictive on the set of processes and names we admit if we are to
  support no-cloning.]
\end{remark}

\subsubsection{Bisimulation}

The computational dynamics gives rise to another kind of equivalence,
the equivalence of computational behavior. As previously mentioned
this is typically captured \emph{via} some form of bisimulation.

% The notion we use in this paper is weak barbed bisimulation
% \cite{milner91polyadicpi}.

The notion we use in this paper is derived from weak barbed
bisimulation \cite{milner91polyadicpi}. 

\begin{definition}
An \emph{observation relation}, $\downarrow_{\mathcal N}$, over a set
of names, $\mathcal N$, is the smallest relation satisfying the rules
below.

\infrule[Out-barb]{y \in {\mathcal N}, \; x \nameeq y}
		  {\outputp{x}{v} \downarrow_{\mathcal N} x}
\infrule[Par-barb]{\mbox{$P\downarrow_{\mathcal N} x$ or $Q\downarrow_{\mathcal N} x$}}
		  {\binpar{P}{Q} \downarrow_{\mathcal N} x}

We write $P \Downarrow_{\mathcal N} x$ if there is $Q$ such that 
$P \wred Q$ and $Q \downarrow_{\mathcal N} x$.
\end{definition}

\begin{definition}
%\label{def.bbisim}
An  ${\mathcal N}$-\emph{barbed bisimulation} over a set of names, ${\mathcal N}$, is a symmetric binary relation 
${\mathcal S}_{\mathcal N}$ between agents such that $P\rel{S}_{\mathcal N}Q$ implies:
\begin{enumerate}
\item If $P \red P'$ then $Q \wred Q'$ and $P'\rel{S}_{\mathcal N} Q'$.
\item If $P\downarrow_{\mathcal N} x$, then $Q\Downarrow_{\mathcal N} x$.
\end{enumerate}
$P$ is ${\mathcal N}$-barbed bisimilar to $Q$, written
$P \wbbisim_{\mathcal N} Q$, if $P \rel{S}_{\mathcal N} Q$ for some ${\mathcal N}$-barbed bisimulation ${\mathcal S}_{\mathcal N}$.
\end{definition}

$\mathcal{R} \subseteq \pi \times \pi$

$P \mathcal{R} Q => \forall P'. P \red P' \Rightarrow \exists Q'. Q \red Q', P' \mathcal{R} Q'$

$P \vdash x \Rightarrow Q \vdash x$

\begin{mathpar}
  \inferrule*[lab=Out-barb]{x \nameeq y}{{y}!\langle{Q}\rangle \vdash x}
  \and
  \inferrule*[lab=Par-barb]{\mbox{$P\vdash x$ or $Q\vdash x$}}{\binpar{P}{Q} \vdash x}
\end{mathpar}

\subsubsection{Contexts}

One of the principle advantages of computational calculi like the
$\pi$-calculus is a well-defined notion of context,
contextual-equivalence and a correlation between
contextual-equivalence and notions of bisimulation. The notion of
context allows the decomposition of a process into (sub-)process and
its syntactic environment, its context. Thus, a context may be
thought of as a process with a ``hole'' (written $\Box$) in it. The
application of a context $M$ to a process $P$, written $M[P]$, is
tantamount to filling the hole in $M$ with $P$. In this paper we do
not need the full weight of this theory, but do make use of the notion
of context in the proof the main theorem. 

\begin{mathpar}
  \inferrule* [lab=summation] {} {{M_{M},M_{N}} \bc \Box \;|\; x.M_{A} \;|\; M_{M}+M_{N}}
  \and
  \inferrule* [lab=agent] {} {{M_{A}} \bc (\vec{x})M_{P} \;| \; \clift{P_0,\ldots,M_{P},\ldots,P_N}}
  \and \\
  \inferrule* [lab=process] {} {{M_{P}} \bc M_{N} \;| \;P|M_{P} }
\end{mathpar} 

\begin{mathpar}
  \inferrule* [lab=sychronization] {} {M_{N} \bc \Box \;|\; x?M_{F} \;|\; x!M_{C}}
  \and
  \inferrule* [lab=abstraction] {} {{M_{F}} \bc (x)M_{P} }
  \and
  \inferrule* [lab=concretion] {} {{M_{C}} \bc \langle M_{P} \rangle }
  \and \\
  \inferrule* [lab=process] {} {{M_{P}} \bc M_{N} \;| \;P|M_{P} }
\end{mathpar}

\begin{definition}[contextual application] Given a context $M$, and
  process $P$, we define the \emph{contextual application}, $M[P] :=
  M\{P/\Box\}$. That is, the contextual application of M to P is the
  substitution of $P$ for $\Box$ in $M$.
\end{definition}

$\meaningof{-} : L \to \mathcal{P}(\pi)$

\begin{mathpar}
  \inferrule* [lab=collection] {} {\meaningof{true} = \pi, \and \meaningof{~E} = \pi \setminus \meaningof{E}, \and \meaningof{E_{1} \& E_{2}} = \meaningof{E_{1}} \cap \meaningof{E_{2}}}
\end{mathpar}

\begin{mathpar}
  \inferrule* [lab=structure] {} {\meaningof{0} = \{ P \in \pi | P \equiv 0 \}, \and \\ \meaningof{E_1 | E_2} = \{ P \in \pi | P \equiv P_{1} | P_{2}, P_{1} \in \meaningof{E_{1}}, P_{2} \in \meaningof{E_2}\} }
\end{mathpar}

\begin{mathpar}
 \inferrule* [lab=behavior] {} {\meaningof{\langle a?b \rangle E} = \{ P \in \pi | P \equiv Q | u?(y)P', \\ \and \\\\ \and \\ \;\;\; u \in \meaningof{a}, \forall z.P'\{z/y\} \in \meaningof{E\{z/b\}}\}, \and \\ \meaningof{a!E} = \{ P \in \pi | P \equiv Q | x!\langle P' \rangle, x \in \meaningof{a} P' \in \meaningof{E}\} }
\end{mathpar}

\begin{mathpar}
 \inferrule* [lab=nominal] {} {\meaningof{\quotep{E}} = \{ \quotep{P} \in \quotep{\pi} | P \in \meaningof{E} \}, \and \meaningof{\quotep{P}} = \{ \quotep{Q} \in \quotep{\pi} | P \equiv Q \} \and \\ \meaningof{@\quotep{E}} = \{ P \in \pi | P \equiv @x, x \in \meaningof{E} \}}
\end{mathpar}

\begin{eqnarray*}
  \\
  \meaningof{-} : TS \to ST
\end{eqnarray*}

\begin{eqnarray*}
  \\
  L : TS \to ST
\end{eqnarray*}

\begin{eqnarray*}
  \\
  P \models E \iff P \in \meaningof{E}
\end{eqnarray*}

\begin{eqnarray*}
  P \approx_{L} Q \iff \forall E \in L. P \models E \iff Q \models E
\end{eqnarray*}

\begin{eqnarray*}
  P \approx_{K} Q
\end{eqnarray*}

\begin{eqnarray*}
  P \approx Q
\end{eqnarray*}

$\approx_{K} = \approx = \approx_{L}$

\subsubsection{Contextual duality}

Note that contexts extend the quotation operation to a family of
operations from processes to names. Given a context, $M$, we can
define a \emph{nominal context}, $\quotep{M}$ by $\quotep{M}[P] :=
\quotep{M[P]}$. To foreshadow what is to come we observe that these
operations enjoy a duality with processes very much like the duality
between vectors and maps from vectors to scalars.

Further, because the calculus is essentially higher-order, we have a
correspondence between contexts and processes. More specifically,
given a name $x$ and a context $M$ we can construct $M^{*}_{x}$ such
that 

\begin{mathpar}
  M^{*}_{x} | \lift{x}{P} \red M[P]
\end{mathpar}

namely,

\begin{mathpar}
  M^{*}_{x} := x?(u).M[\dropn{u}]
\end{mathpar}

The dependence of $M^{*}_{x}$ on a name makes it an abstraction, 

\begin{mathpar}
  M^{*} := (x)x?(u).M[\dropn{u}]
\end{mathpar}

\subsection{Additional notation}

It will sometimes be convenient to denote the process a name
quotes. We already have the notation $x = \quotep{P}$, but it will be
convenient to introduce an alternate notation, $\procn{x}$, when we
want to emphasize the connection to the use of the name. Note that, by
virtue of name equivalence, $\quotep{\procn{x}} \nameeq x$; so, the
notation is consistent with previous definitions.

Further, because names have structure it is possible to effect
substitutions on the basis of that structure. This means we need to
upgrade our notation for substitutions, which we accomplish by
adapting comprehension notation. Thus,

\begin{mathpar}
  P\{ y / x : x \in S \}
\end{mathpar}

is interpreted to mean the process derived from P by replacing (in a
capture-avoiding manner) each occurrence of $x$ in $S$ by $y$. For example,

\begin{mathpar}
  P\{ \quotep{\procn{x}|\procn{x}} / x : x \in \freenames{P} \}
\end{mathpar}

will replace each (occurrence) of a free name $x$ in $P$ by
$\quotep{\procn{x}|\procn{x}}$.

Also, we will avail ourselves of the notation $x^{L}$ and $x^{R}$ to
denote injections of a name into disjoint copies of the name
space. There are numerous ways to accomplish this. One example can be
found in \cite{MeredithR05}. This notation overloads to vectors of
names: $\vec{x}^{\pi} := (x_{i}^{\pi} \; : \; 0 \leq i < |\vec{x}| )$ where $\pi \in \{L,R\}$.

We also use $P^{\Box} := P|\Box$.

In \cite{MeredithR05} an interpretation of the new operator is
given. It turns out that there are several possible interpretations
all enjoying the requisite algebraic properties of the operator (see
\cite{milner91polyadicpi}). We will therefore make liberal use of
$(\nu\; \vec{x})P$.

% subsection the_syntax_and_semantics_of_the_notation_system (end)   

\input{qm2pi.qmops} 

\input{qm2pi.sterngerlach} 

\input{qm2pi.metric} 

% section concurrent_process_calculi (end)

%\input{qm2pi.proofsketch}

% section proof sketch (end)

%\input{qm2pi.slviaknots} 

% section spatial logic via knots (end)

\input{qm2pi.conclusion}

% section conclusion (end)

%\input{qm2pi.dtcodes} 

% section wiring algorithm (end)

\input{qm2pi.ack} 

% section acknowledgments (end)

\newpage


\bibliographystyle{plain}   
\bibliography{../../biblios/main.bib}

\input{qm2pi.rhodetails}

\end{document}

 

% section acknowledgments (end)

\newpage


\bibliographystyle{plain}   
\bibliography{../../biblios/main.bib}

\documentclass[12pt]{llncs}
%\documentclass{jktr}

\usepackage[pdftex]{hyperref}                   
\usepackage {listings}
\usepackage {mathpartir}
\usepackage{bcprules}
%\usepackage{listings}
                       
\usepackage{graphicx} 
%\usepackage[margins=2.5cm,nohead,nofoot]{geometry}
%\usepackage{geometry}
\usepackage{amsfonts}
\usepackage{amstext}
\usepackage{latexsym}
\usepackage{amssymb}
\usepackage{color}


%\include{myPreamble}
\include{qm2pi.local} 

%\ifpdf
%\usepackage[pdftex]{graphicx}
%\else
%\usepackage{graphicx}
%\fi

 % \ifpdf
%  \usepackage{pdfsync}
%  \if


%\title{Brief Article}
%\author{David F. Snyder}
%\author{L.G. Meredith}

%\address{Dept. of Math., Texas State University--San Marcos, San Marcos, TX 78666}
       
\pagestyle{empty}


\begin{document}

\lstset{language=[Objective]Caml,frame=shadowbox}

\input{qm2pi.front}

% section front matter (end)

\input{qm2pi.intro} 
 
% section introduction (end)

% \input{qm2pi.knotations} 

% section notation (end)

\input{qm2pi.process.calculi} 

% section concurrent_process_calculi_and_spatial_logics_ (end)
    
%\input{qm2pi.knots2pi} 

%\input{qm2pi.trefoil} 

%\input{qm2pi.mainthm} 

% subsection basic_interpretation (end)

%\input{qm2pi.rho.presentation} 
\subsection{The syntax and semantics of the notation system}\label{sub:the_syntax_and_semantics_of_the_notation_system} % (fold)

We now summarize a technical presentation of the calculus that
embodies our theory of dynamics. The typical presentation of such a
calculus follows the style of giving generators and relations on
them. The grammar, below, describing term constructors, freely
generates the set of processes, $\Proc$. This set is then quotiented
by a relation known as structural congruence and it is over this set
that the notion of dynamics is expressed. This presentation is
essentially that of \cite{MeredithR05} with the addition of
polyadicity and summation. For readability we have relegated some of
the technical subtleties to an appendix.

\subsubsection{Process grammar}\label{subsub:process_grammar}

\begin{mathpar}
  \inferrule* [lab=synchronization] {} {{M} \bc \pzero \;|\; x?F \;|\; x!C }
  \and
  \inferrule* [lab=abstraction] {} {{F} \bc (x)P}
  \and
  \inferrule* [lab=concretion] {} {{C} \bc \langle Q \rangle}
  \and
  \inferrule* [lab=process] {} {{P,Q} \bc M \;| \;P|Q \;|\; @{x}}
  \and
  \inferrule* [lab=name] {} {{x} \bc \quotep{P}}
\end{mathpar} 

Note that $\vec{x}$ (resp. $\vec{P}$) denotes a vector of names
(resp. processes) of length $|\vec{x}|$ (resp. $|\vec{P}|$). We adopt
the following useful abbreviations.

\begin{mathpar}
   x?(\vec{y}).P := x.(\vec{y})P \and  x\clift{\vec{P}} := x.\clift{\vec{P}}
   \and x!(y) := \lift{x}{\dropn{y}}
   \and \Pi_{i=0}^{n-1}P_i := P_0 | \ldots | P_{n-1}
\end{mathpar}

\subsubsection{Structural congruence}

\paragraph{Free and bound names and alpha-equivalence.} At the
core of structural equivalence is alpha-equivalence which identifies
process that are the same up to a change of variable. Formally, we
recognize the distinction between free and bound names. The free names
of a process, $\freenames{P}$, may be calculated recursively as
follows:

\begin{mathpar}
\freenames{\pzero} := \emptyset
  \and \\
  \freenames{x?(y).P} := \{ x \} \cup (\freenames{P} \setminus \{ y \})
  \and 
  \freenames{x!\langle P \rangle} := \{ x \} \cup \{ P \} 
  \and \\
  \freenames{P|Q} := \freenames{P} \cup \freenames{Q}
  \and \\
  \freenames{@{x}} := \{ x \}
\end{mathpar}

$\pi$
$\quotep{\pi}$

$\freenames{-} : \pi \to \mathcal{P}(\quotep{\pi})$

\begin{eqnarray*}
  \freenames{\pzero} & := & \emptyset \\
  \freenames{x?(y).P} & := & \{ x \} \cup (\freenames{P} \setminus \{ y \}) \\
  \freenames{x!\langle P \rangle} & := & \{ x \} \cup \{ P \} \\
  \freenames{P|Q} & := & \freenames{P} \cup \freenames{Q} \\
  \freenames{\dropn{x}} & := & \{ x \}
\end{eqnarray*}

The bound names of a process, $\boundnames{P}$, are those names occurring in $P$
that are not free. For example, in $x?(y).0$, the name $x$ is free, while $y$ is bound.

\begin{mathpar}
  \inferrule* [lab=monoidal-laws] {} { P|Q \equiv Q|P \and P|0 \equiv P \and P|(Q|R) \equiv (P|Q)|R }
\end{mathpar}

\begin{mathpar}
  \inferrule* [lab=alpha-equivalence] {} { (x)P \equiv (y)P\{y/x\} \and y \not\in \freenames{P} }
\end{mathpar}

\begin{definition}
Then two processes, $P,Q$, are alpha-equivalent if $P = Q\{\vec{y}/\vec{x}\}$ for
some $\vec{x} \in \boundnames{Q},\vec{y} \in \boundnames{P}$, where $Q\{\vec{y}/\vec{x}\}$
denotes the capture-avoiding substitution of $\vec{y}$ for $\vec{x}$ in $Q$.
\end{definition}

\begin{definition}
  The {\em structural congruence} \cite{SangiorgiWalker} , $\equiv$,
  between processes is the least congruence containing
  alpha-equivalence, satisfying the abelian monoid laws
  (associativity, commutativity and $\pzero$ as identity) for parallel
  composition $|$ and for summation $+$.
\end{definition}

\subsection{Name equivalence}

We take name equivalence, written $\nameeq$, to be the smallest
equivalence relation generated by the following rules.

\begin{mathpar}
\inferrule*[lab=Quote-drop]
{ }
{ \quotep{@{x}} \nameeq x }

\inferrule*[lab=Struct-equiv]
{ P \scong Q }
{ \quotep{P} \nameeq \quotep{Q} }
\end{mathpar}

The astute reader will have noticed that the mutual recursion of names
and processes imposes a mutual recursion on alpha-equivalence and
structural equivalence via name-equivalence. Fortunately, all of this
works out pleasantly and we may calculate in the natural way, free of
concern. The reader interested in the details is referred to the
appendix \ref{appendix:rho_details}.

\subsection{Substitution}

We use $\Proc$ for the set of processes, $\QProc$ for the set of
names, and $\id{\{}\vec{y} / \vec{x} \id{\}}$ to denote partial maps,
$s : \QProc \rightarrow \QProc$. A map, $s$ lifts, uniquely, to a map
on process terms, $\widehat{s} : \Proc \rightarrow \Proc$ by the
following equations.

\begin{mathpar}
  (0) \psubstp{Q}{P} := 0 \\
  (R \juxtap S) \psubstp{Q}{P}
  :=    
  (R)\psubstp{Q}{P} \juxtap (S) \psubstp{Q}{P} \\
  (x?(y).R) \psubstp{Q}{P}    
  :=    
  (x)\substp{Q}{P} (z)\concat( (R \psubstn{z}{y}) \psubstp{Q}{P} ) \\
  (\lift{x}{R}) \psubstp{Q}{P}  
  :=
  \lift{(x)\substp{Q}{P}}{ R \psubstp{Q}{P} } \\
%   (\dropn{x})  \psubstp{Q}{P}       
%   := 
%   \left\{ 
%     \begin{array}{ccc} 
%       \dropn{\quotep{Q}} & & x \nameeq \quotep{P} \\
%       \dropn{x} & & otherwise \\
%     \end{array}
%   \right. 
  (\dropn{x})  \psubstp{Q}{P}       
  := 
  \left\{ 
    \begin{array}{ccc} 
      Q & & x \nameeq \quotep{P} \\
      \dropn{x} & & otherwise \\
    \end{array}
  \right.
\end{mathpar}
 

where

\begin{eqnarray}
  (x)\id{\{} \lpquote Q \rpquote / \lpquote P \rpquote \id{\}}            = 
  \left\{ 
    \begin{array}{ccc}
      \lpquote Q \rpquote & & x \nameeq \lpquote P \rpquote \\
      x & & otherwise \\
    \end{array}
  \right. \nonumber
\end{eqnarray}

and $z$ is chosen distinct from $\quotep{P}$, $\quotep{Q}$, the free
names in $Q$, and all the names in $R$. Our $\alpha$-equivalence will
be built in the standard way from this substitution.

\begin{remark}\label{rem:no_self_referential_names}
  One consequence of these definitions is that $\forall P. \quotep{P}
  \not\in \freenames{P}$.
\end{remark}

\subsection{ Dynamic quote: an example }

Anticipating something of what's to come, consider applying the
substitution, $\widehat{\id{\{}u / z \id{\}}}$, to the following pair
of processes, $\lift{w}{y!(z)}$ and $w[ \lpquote y!(z) \rpquote ]$.

\begin{eqnarray}
	\lift{w}{y!(z)}\widehat{\id{\{}u / z \id{\}}}
		& = &
		\lift{w}{y!(u)} \nonumber\\
	w[ \lpquote y!(z) \rpquote ] \widehat{ \id{\{}u / z \id{\}} }
		& = &
		w[ \lpquote y!(z) \rpquote ] \nonumber
\end{eqnarray}

Because the body of the process between quotes is impervious to
substitution, we get radically different answers. In fact, by
examining the first process in an input context,
e.g. $x?(z).\lift{w}{y!(z)}$, we see that the process under the lift
operator may be shaped by prefixed inputs binding a name inside it. In
this sense, the lift operator will be seen as a way to dynamically
construct processes before reifying them as names.

Finally equipped with these standard features we can present the
dynamics of the calculus.

\subsubsection{Operational semantics} 

Finally, we introduce the computational dynamics. What marks these
algebras as distinct from other more traditionally studied algebraic
structures, e.g. vector spaces or polynomial rings, is the manner in
which dynamics is captured. In traditional structures, dynamics is typically
expressed through morphisms between such structures, as in linear maps
between vector spaces or morphisms between rings. In algebras
associated with the semantics of computation, the dynamics is
expressed as part of the algebraic structure itself, through a
reduction reduction relation typically denoted by $\red$. Below, we
give a recursive presentation of this relation for the calculus used
in the encoding.

$\red \subseteq \pi \times \pi$
$\red : \pi \to \mathcal{P}(\pi)$

\begin{mathpar}
  \inferrule* [lab=Comm] { \textsf{match}( x_{src}, x_{trgt} ) } { x_{trgt}?(y)P \; | \; x_{src}!\langle {Q} \rangle \red P\{\quotep{Q}/y}\} }
  \and \\
  \inferrule* [lab=Par] {{P} \red {P}'} {{{P} | {Q}} \red {{P}' | {Q}}}
  \and
  \inferrule* [lab=Equiv]{{{P} \scong {P}'} \andalso {{P}' \red {Q}'} \andalso {{Q}' \scong {Q}}}{{P} \red {Q}}
\end{mathpar}

\begin{eqnarray*}
  match_{\equiv} (\quotep{P},\quotep{Q}) & := & P \equiv Q \\
  match_{\dagger}(\quotep{P},\quotep{Q}) & := & \forall R. P|Q \red^{*} R => R \red^{*} 0 \\
  match_{K}(\quotep{P},\quotep{Q}) & := & K \mbox{ for some context } K
\end{eqnarray*}

$u?(x)P | u!\langle Q \rangle \red P\{\quotep{Q}/x\}$

%We write $\wred$ for $\red^*$, and $P\red$ if $\exists Q $ such that $ P \red Q$.
We write $P\red$ if $\exists Q $ such that $ P \red Q$ and $P\not\red$, otherwise.

\section{Replication}

As mentioned before, it is known that replication (and hence
recursion) can be implemented in a higher-order process algebra
\cite{SangiorgiWalker}. As our first example of calculation with the
machinery thus far presented we give the construction explicitly in
the {\rhoc}.

\begin{eqnarray}
	D_{x} & := & \prefix{x}{y}{(\binpar{\outputp{x}{y}}{@{y}})} \nonumber\\
	\bangp_{x}{P} & := & \binpar{{x}!\langle{\binpar{D_{x}}{P}}\rangle}{D_{x}} \nonumber
\end{eqnarray}

\begin{eqnarray}
	\bangp_{x}{P} & & \nonumber\\
	=
	& {x}!\langle{(\prefix{x}{y}{(\outputp{x}{y} | @{y})) | P}}\rangle 
	      | \prefix{x}{y}{(\outputp{x}{y} | @{y})} & \nonumber\\
	\red
	& (\outputp{x}{y} | @{y})\substn{\quotep{(\prefix{x}{y}{(@{y} | \outputp{x}{y})) | P}}}{y} & \nonumber\\
	=
	& \outputp{x}{\quotep{(\prefix{x}{y}{(\outputp{x}{y} | @{y})) | P}}}
	  | {(\prefix{x}{y}{(\outputp{x}{y} | @{y})) | P}} & \nonumber\\
	\red
	& \ldots & \nonumber\\
	\red^*
	& P | P | \ldots & \nonumber
\end{eqnarray}

Of course, this encoding, as an implementation, runs away, unfolding
$\bangp{P}$ eagerly. A lazier and more implementable replication
operator, restricted to input-guarded processes, may be obtained as follows.

\begin{eqnarray}
\bangp{\prefix{u}{v}{P}} 
	:= 
	\binpar{\lift{x}{\prefix{u}{v}{(\binpar{D(x)}{P})}}}{D(x)} \nonumber
\end{eqnarray}

\begin{remark}
  Note that the lazier definition still does not deal with summation
  or mixed summation (i.e. sums over input and output). The reader is
  invited to construct definitions of replication that deal with these
  features. 

  Further, the definitions are parameterized in a name, $x$. Can you,
  gentle reader, make a definition that eliminates this parameter and
  guarantees no accidental interaction between the replication
  machinery and the process being replicated -- i.e. no accidental
  sharing of names used by the process to get its work done and the
  name(s) used by the replication to effect copying. This latter
  revision of the definition of replication is crucial to obtaining
  the expected identity $!!P \sim !P$.
\end{remark}

\begin{remark}\label{rem:paradoxical_combinator}
  The reader familiar with the lambda calculus will have noticed the
  similarity between $D$ and the paradoxical combinator.

  [Ed. note: the existence of this seems to suggest we have to be more
  restrictive on the set of processes and names we admit if we are to
  support no-cloning.]
\end{remark}

\subsubsection{Bisimulation}

The computational dynamics gives rise to another kind of equivalence,
the equivalence of computational behavior. As previously mentioned
this is typically captured \emph{via} some form of bisimulation.

% The notion we use in this paper is weak barbed bisimulation
% \cite{milner91polyadicpi}.

The notion we use in this paper is derived from weak barbed
bisimulation \cite{milner91polyadicpi}. 

\begin{definition}
An \emph{observation relation}, $\downarrow_{\mathcal N}$, over a set
of names, $\mathcal N$, is the smallest relation satisfying the rules
below.

\infrule[Out-barb]{y \in {\mathcal N}, \; x \nameeq y}
		  {\outputp{x}{v} \downarrow_{\mathcal N} x}
\infrule[Par-barb]{\mbox{$P\downarrow_{\mathcal N} x$ or $Q\downarrow_{\mathcal N} x$}}
		  {\binpar{P}{Q} \downarrow_{\mathcal N} x}

We write $P \Downarrow_{\mathcal N} x$ if there is $Q$ such that 
$P \wred Q$ and $Q \downarrow_{\mathcal N} x$.
\end{definition}

\begin{definition}
%\label{def.bbisim}
An  ${\mathcal N}$-\emph{barbed bisimulation} over a set of names, ${\mathcal N}$, is a symmetric binary relation 
${\mathcal S}_{\mathcal N}$ between agents such that $P\rel{S}_{\mathcal N}Q$ implies:
\begin{enumerate}
\item If $P \red P'$ then $Q \wred Q'$ and $P'\rel{S}_{\mathcal N} Q'$.
\item If $P\downarrow_{\mathcal N} x$, then $Q\Downarrow_{\mathcal N} x$.
\end{enumerate}
$P$ is ${\mathcal N}$-barbed bisimilar to $Q$, written
$P \wbbisim_{\mathcal N} Q$, if $P \rel{S}_{\mathcal N} Q$ for some ${\mathcal N}$-barbed bisimulation ${\mathcal S}_{\mathcal N}$.
\end{definition}

$\mathcal{R} \subseteq \pi \times \pi$

$P \mathcal{R} Q => \forall P'. P \red P' \Rightarrow \exists Q'. Q \red Q', P' \mathcal{R} Q'$

$P \vdash x \Rightarrow Q \vdash x$

\begin{mathpar}
  \inferrule*[lab=Out-barb]{x \nameeq y}{{y}!\langle{Q}\rangle \vdash x}
  \and
  \inferrule*[lab=Par-barb]{\mbox{$P\vdash x$ or $Q\vdash x$}}{\binpar{P}{Q} \vdash x}
\end{mathpar}

\subsubsection{Contexts}

One of the principle advantages of computational calculi like the
$\pi$-calculus is a well-defined notion of context,
contextual-equivalence and a correlation between
contextual-equivalence and notions of bisimulation. The notion of
context allows the decomposition of a process into (sub-)process and
its syntactic environment, its context. Thus, a context may be
thought of as a process with a ``hole'' (written $\Box$) in it. The
application of a context $M$ to a process $P$, written $M[P]$, is
tantamount to filling the hole in $M$ with $P$. In this paper we do
not need the full weight of this theory, but do make use of the notion
of context in the proof the main theorem. 

\begin{mathpar}
  \inferrule* [lab=summation] {} {{M_{M},M_{N}} \bc \Box \;|\; x.M_{A} \;|\; M_{M}+M_{N}}
  \and
  \inferrule* [lab=agent] {} {{M_{A}} \bc (\vec{x})M_{P} \;| \; \clift{P_0,\ldots,M_{P},\ldots,P_N}}
  \and \\
  \inferrule* [lab=process] {} {{M_{P}} \bc M_{N} \;| \;P|M_{P} }
\end{mathpar} 

\begin{mathpar}
  \inferrule* [lab=sychronization] {} {M_{N} \bc \Box \;|\; x?M_{F} \;|\; x!M_{C}}
  \and
  \inferrule* [lab=abstraction] {} {{M_{F}} \bc (x)M_{P} }
  \and
  \inferrule* [lab=concretion] {} {{M_{C}} \bc \langle M_{P} \rangle }
  \and \\
  \inferrule* [lab=process] {} {{M_{P}} \bc M_{N} \;| \;P|M_{P} }
\end{mathpar}

\begin{definition}[contextual application] Given a context $M$, and
  process $P$, we define the \emph{contextual application}, $M[P] :=
  M\{P/\Box\}$. That is, the contextual application of M to P is the
  substitution of $P$ for $\Box$ in $M$.
\end{definition}

$\meaningof{-} : L \to \mathcal{P}(\pi)$

\begin{mathpar}
  \inferrule* [lab=collection] {} {\meaningof{true} = \pi, \and \meaningof{~E} = \pi \setminus \meaningof{E}, \and \meaningof{E_{1} \& E_{2}} = \meaningof{E_{1}} \cap \meaningof{E_{2}}}
\end{mathpar}

\begin{mathpar}
  \inferrule* [lab=structure] {} {\meaningof{0} = \{ P \in \pi | P \equiv 0 \}, \and \\ \meaningof{E_1 | E_2} = \{ P \in \pi | P \equiv P_{1} | P_{2}, P_{1} \in \meaningof{E_{1}}, P_{2} \in \meaningof{E_2}\} }
\end{mathpar}

\begin{mathpar}
 \inferrule* [lab=behavior] {} {\meaningof{\langle a?b \rangle E} = \{ P \in \pi | P \equiv Q | u?(y)P', \\ \and \\\\ \and \\ \;\;\; u \in \meaningof{a}, \forall z.P'\{z/y\} \in \meaningof{E\{z/b\}}\}, \and \\ \meaningof{a!E} = \{ P \in \pi | P \equiv Q | x!\langle P' \rangle, x \in \meaningof{a} P' \in \meaningof{E}\} }
\end{mathpar}

\begin{mathpar}
 \inferrule* [lab=nominal] {} {\meaningof{\quotep{E}} = \{ \quotep{P} \in \quotep{\pi} | P \in \meaningof{E} \}, \and \meaningof{\quotep{P}} = \{ \quotep{Q} \in \quotep{\pi} | P \equiv Q \} \and \\ \meaningof{@\quotep{E}} = \{ P \in \pi | P \equiv @x, x \in \meaningof{E} \}}
\end{mathpar}

\begin{eqnarray*}
  \\
  \meaningof{-} : TS \to ST
\end{eqnarray*}

\begin{eqnarray*}
  \\
  L : TS \to ST
\end{eqnarray*}

\begin{eqnarray*}
  \\
  P \models E \iff P \in \meaningof{E}
\end{eqnarray*}

\begin{eqnarray*}
  P \approx_{L} Q \iff \forall E \in L. P \models E \iff Q \models E
\end{eqnarray*}

\begin{eqnarray*}
  P \approx_{K} Q
\end{eqnarray*}

\begin{eqnarray*}
  P \approx Q
\end{eqnarray*}

$\approx_{K} = \approx = \approx_{L}$

\subsubsection{Contextual duality}

Note that contexts extend the quotation operation to a family of
operations from processes to names. Given a context, $M$, we can
define a \emph{nominal context}, $\quotep{M}$ by $\quotep{M}[P] :=
\quotep{M[P]}$. To foreshadow what is to come we observe that these
operations enjoy a duality with processes very much like the duality
between vectors and maps from vectors to scalars.

Further, because the calculus is essentially higher-order, we have a
correspondence between contexts and processes. More specifically,
given a name $x$ and a context $M$ we can construct $M^{*}_{x}$ such
that 

\begin{mathpar}
  M^{*}_{x} | \lift{x}{P} \red M[P]
\end{mathpar}

namely,

\begin{mathpar}
  M^{*}_{x} := x?(u).M[\dropn{u}]
\end{mathpar}

The dependence of $M^{*}_{x}$ on a name makes it an abstraction, 

\begin{mathpar}
  M^{*} := (x)x?(u).M[\dropn{u}]
\end{mathpar}

\subsection{Additional notation}

It will sometimes be convenient to denote the process a name
quotes. We already have the notation $x = \quotep{P}$, but it will be
convenient to introduce an alternate notation, $\procn{x}$, when we
want to emphasize the connection to the use of the name. Note that, by
virtue of name equivalence, $\quotep{\procn{x}} \nameeq x$; so, the
notation is consistent with previous definitions.

Further, because names have structure it is possible to effect
substitutions on the basis of that structure. This means we need to
upgrade our notation for substitutions, which we accomplish by
adapting comprehension notation. Thus,

\begin{mathpar}
  P\{ y / x : x \in S \}
\end{mathpar}

is interpreted to mean the process derived from P by replacing (in a
capture-avoiding manner) each occurrence of $x$ in $S$ by $y$. For example,

\begin{mathpar}
  P\{ \quotep{\procn{x}|\procn{x}} / x : x \in \freenames{P} \}
\end{mathpar}

will replace each (occurrence) of a free name $x$ in $P$ by
$\quotep{\procn{x}|\procn{x}}$.

Also, we will avail ourselves of the notation $x^{L}$ and $x^{R}$ to
denote injections of a name into disjoint copies of the name
space. There are numerous ways to accomplish this. One example can be
found in \cite{MeredithR05}. This notation overloads to vectors of
names: $\vec{x}^{\pi} := (x_{i}^{\pi} \; : \; 0 \leq i < |\vec{x}| )$ where $\pi \in \{L,R\}$.

We also use $P^{\Box} := P|\Box$.

In \cite{MeredithR05} an interpretation of the new operator is
given. It turns out that there are several possible interpretations
all enjoying the requisite algebraic properties of the operator (see
\cite{milner91polyadicpi}). We will therefore make liberal use of
$(\nu\; \vec{x})P$.

% subsection the_syntax_and_semantics_of_the_notation_system (end)   

\input{qm2pi.qmops} 

\input{qm2pi.sterngerlach} 

\input{qm2pi.metric} 

% section concurrent_process_calculi (end)

%\input{qm2pi.proofsketch}

% section proof sketch (end)

%\input{qm2pi.slviaknots} 

% section spatial logic via knots (end)

\input{qm2pi.conclusion}

% section conclusion (end)

%\input{qm2pi.dtcodes} 

% section wiring algorithm (end)

\input{qm2pi.ack} 

% section acknowledgments (end)

\newpage


\bibliographystyle{plain}   
\bibliography{../../biblios/main.bib}

\input{qm2pi.rhodetails}

\end{document}



\end{document}

 

\documentclass[12pt]{llncs}
%\documentclass{jktr}

\usepackage[pdftex]{hyperref}                   
\usepackage {listings}
\usepackage {mathpartir}
\usepackage{bcprules}
%\usepackage{listings}
                       
\usepackage{graphicx} 
%\usepackage[margins=2.5cm,nohead,nofoot]{geometry}
%\usepackage{geometry}
\usepackage{amsfonts}
\usepackage{amstext}
\usepackage{latexsym}
\usepackage{amssymb}
\usepackage{color}


%\include{myPreamble}
\documentclass[12pt]{llncs}
%\documentclass{jktr}

\usepackage[pdftex]{hyperref}                   
\usepackage {listings}
\usepackage {mathpartir}
\usepackage{bcprules}
%\usepackage{listings}
                       
\usepackage{graphicx} 
%\usepackage[margins=2.5cm,nohead,nofoot]{geometry}
%\usepackage{geometry}
\usepackage{amsfonts}
\usepackage{amstext}
\usepackage{latexsym}
\usepackage{amssymb}
\usepackage{color}


%\include{myPreamble}
\include{qm2pi.local} 

%\ifpdf
%\usepackage[pdftex]{graphicx}
%\else
%\usepackage{graphicx}
%\fi

 % \ifpdf
%  \usepackage{pdfsync}
%  \if


%\title{Brief Article}
%\author{David F. Snyder}
%\author{L.G. Meredith}

%\address{Dept. of Math., Texas State University--San Marcos, San Marcos, TX 78666}
       
\pagestyle{empty}


\begin{document}

\lstset{language=[Objective]Caml,frame=shadowbox}

\input{qm2pi.front}

% section front matter (end)

\input{qm2pi.intro} 
 
% section introduction (end)

% \input{qm2pi.knotations} 

% section notation (end)

\input{qm2pi.process.calculi} 

% section concurrent_process_calculi_and_spatial_logics_ (end)
    
%\input{qm2pi.knots2pi} 

%\input{qm2pi.trefoil} 

%\input{qm2pi.mainthm} 

% subsection basic_interpretation (end)

%\input{qm2pi.rho.presentation} 
\subsection{The syntax and semantics of the notation system}\label{sub:the_syntax_and_semantics_of_the_notation_system} % (fold)

We now summarize a technical presentation of the calculus that
embodies our theory of dynamics. The typical presentation of such a
calculus follows the style of giving generators and relations on
them. The grammar, below, describing term constructors, freely
generates the set of processes, $\Proc$. This set is then quotiented
by a relation known as structural congruence and it is over this set
that the notion of dynamics is expressed. This presentation is
essentially that of \cite{MeredithR05} with the addition of
polyadicity and summation. For readability we have relegated some of
the technical subtleties to an appendix.

\subsubsection{Process grammar}\label{subsub:process_grammar}

\begin{mathpar}
  \inferrule* [lab=synchronization] {} {{M} \bc \pzero \;|\; x?F \;|\; x!C }
  \and
  \inferrule* [lab=abstraction] {} {{F} \bc (x)P}
  \and
  \inferrule* [lab=concretion] {} {{C} \bc \langle Q \rangle}
  \and
  \inferrule* [lab=process] {} {{P,Q} \bc M \;| \;P|Q \;|\; @{x}}
  \and
  \inferrule* [lab=name] {} {{x} \bc \quotep{P}}
\end{mathpar} 

Note that $\vec{x}$ (resp. $\vec{P}$) denotes a vector of names
(resp. processes) of length $|\vec{x}|$ (resp. $|\vec{P}|$). We adopt
the following useful abbreviations.

\begin{mathpar}
   x?(\vec{y}).P := x.(\vec{y})P \and  x\clift{\vec{P}} := x.\clift{\vec{P}}
   \and x!(y) := \lift{x}{\dropn{y}}
   \and \Pi_{i=0}^{n-1}P_i := P_0 | \ldots | P_{n-1}
\end{mathpar}

\subsubsection{Structural congruence}

\paragraph{Free and bound names and alpha-equivalence.} At the
core of structural equivalence is alpha-equivalence which identifies
process that are the same up to a change of variable. Formally, we
recognize the distinction between free and bound names. The free names
of a process, $\freenames{P}$, may be calculated recursively as
follows:

\begin{mathpar}
\freenames{\pzero} := \emptyset
  \and \\
  \freenames{x?(y).P} := \{ x \} \cup (\freenames{P} \setminus \{ y \})
  \and 
  \freenames{x!\langle P \rangle} := \{ x \} \cup \{ P \} 
  \and \\
  \freenames{P|Q} := \freenames{P} \cup \freenames{Q}
  \and \\
  \freenames{@{x}} := \{ x \}
\end{mathpar}

$\pi$
$\quotep{\pi}$

$\freenames{-} : \pi \to \mathcal{P}(\quotep{\pi})$

\begin{eqnarray*}
  \freenames{\pzero} & := & \emptyset \\
  \freenames{x?(y).P} & := & \{ x \} \cup (\freenames{P} \setminus \{ y \}) \\
  \freenames{x!\langle P \rangle} & := & \{ x \} \cup \{ P \} \\
  \freenames{P|Q} & := & \freenames{P} \cup \freenames{Q} \\
  \freenames{\dropn{x}} & := & \{ x \}
\end{eqnarray*}

The bound names of a process, $\boundnames{P}$, are those names occurring in $P$
that are not free. For example, in $x?(y).0$, the name $x$ is free, while $y$ is bound.

\begin{mathpar}
  \inferrule* [lab=monoidal-laws] {} { P|Q \equiv Q|P \and P|0 \equiv P \and P|(Q|R) \equiv (P|Q)|R }
\end{mathpar}

\begin{mathpar}
  \inferrule* [lab=alpha-equivalence] {} { (x)P \equiv (y)P\{y/x\} \and y \not\in \freenames{P} }
\end{mathpar}

\begin{definition}
Then two processes, $P,Q$, are alpha-equivalent if $P = Q\{\vec{y}/\vec{x}\}$ for
some $\vec{x} \in \boundnames{Q},\vec{y} \in \boundnames{P}$, where $Q\{\vec{y}/\vec{x}\}$
denotes the capture-avoiding substitution of $\vec{y}$ for $\vec{x}$ in $Q$.
\end{definition}

\begin{definition}
  The {\em structural congruence} \cite{SangiorgiWalker} , $\equiv$,
  between processes is the least congruence containing
  alpha-equivalence, satisfying the abelian monoid laws
  (associativity, commutativity and $\pzero$ as identity) for parallel
  composition $|$ and for summation $+$.
\end{definition}

\subsection{Name equivalence}

We take name equivalence, written $\nameeq$, to be the smallest
equivalence relation generated by the following rules.

\begin{mathpar}
\inferrule*[lab=Quote-drop]
{ }
{ \quotep{@{x}} \nameeq x }

\inferrule*[lab=Struct-equiv]
{ P \scong Q }
{ \quotep{P} \nameeq \quotep{Q} }
\end{mathpar}

The astute reader will have noticed that the mutual recursion of names
and processes imposes a mutual recursion on alpha-equivalence and
structural equivalence via name-equivalence. Fortunately, all of this
works out pleasantly and we may calculate in the natural way, free of
concern. The reader interested in the details is referred to the
appendix \ref{appendix:rho_details}.

\subsection{Substitution}

We use $\Proc$ for the set of processes, $\QProc$ for the set of
names, and $\id{\{}\vec{y} / \vec{x} \id{\}}$ to denote partial maps,
$s : \QProc \rightarrow \QProc$. A map, $s$ lifts, uniquely, to a map
on process terms, $\widehat{s} : \Proc \rightarrow \Proc$ by the
following equations.

\begin{mathpar}
  (0) \psubstp{Q}{P} := 0 \\
  (R \juxtap S) \psubstp{Q}{P}
  :=    
  (R)\psubstp{Q}{P} \juxtap (S) \psubstp{Q}{P} \\
  (x?(y).R) \psubstp{Q}{P}    
  :=    
  (x)\substp{Q}{P} (z)\concat( (R \psubstn{z}{y}) \psubstp{Q}{P} ) \\
  (\lift{x}{R}) \psubstp{Q}{P}  
  :=
  \lift{(x)\substp{Q}{P}}{ R \psubstp{Q}{P} } \\
%   (\dropn{x})  \psubstp{Q}{P}       
%   := 
%   \left\{ 
%     \begin{array}{ccc} 
%       \dropn{\quotep{Q}} & & x \nameeq \quotep{P} \\
%       \dropn{x} & & otherwise \\
%     \end{array}
%   \right. 
  (\dropn{x})  \psubstp{Q}{P}       
  := 
  \left\{ 
    \begin{array}{ccc} 
      Q & & x \nameeq \quotep{P} \\
      \dropn{x} & & otherwise \\
    \end{array}
  \right.
\end{mathpar}
 

where

\begin{eqnarray}
  (x)\id{\{} \lpquote Q \rpquote / \lpquote P \rpquote \id{\}}            = 
  \left\{ 
    \begin{array}{ccc}
      \lpquote Q \rpquote & & x \nameeq \lpquote P \rpquote \\
      x & & otherwise \\
    \end{array}
  \right. \nonumber
\end{eqnarray}

and $z$ is chosen distinct from $\quotep{P}$, $\quotep{Q}$, the free
names in $Q$, and all the names in $R$. Our $\alpha$-equivalence will
be built in the standard way from this substitution.

\begin{remark}\label{rem:no_self_referential_names}
  One consequence of these definitions is that $\forall P. \quotep{P}
  \not\in \freenames{P}$.
\end{remark}

\subsection{ Dynamic quote: an example }

Anticipating something of what's to come, consider applying the
substitution, $\widehat{\id{\{}u / z \id{\}}}$, to the following pair
of processes, $\lift{w}{y!(z)}$ and $w[ \lpquote y!(z) \rpquote ]$.

\begin{eqnarray}
	\lift{w}{y!(z)}\widehat{\id{\{}u / z \id{\}}}
		& = &
		\lift{w}{y!(u)} \nonumber\\
	w[ \lpquote y!(z) \rpquote ] \widehat{ \id{\{}u / z \id{\}} }
		& = &
		w[ \lpquote y!(z) \rpquote ] \nonumber
\end{eqnarray}

Because the body of the process between quotes is impervious to
substitution, we get radically different answers. In fact, by
examining the first process in an input context,
e.g. $x?(z).\lift{w}{y!(z)}$, we see that the process under the lift
operator may be shaped by prefixed inputs binding a name inside it. In
this sense, the lift operator will be seen as a way to dynamically
construct processes before reifying them as names.

Finally equipped with these standard features we can present the
dynamics of the calculus.

\subsubsection{Operational semantics} 

Finally, we introduce the computational dynamics. What marks these
algebras as distinct from other more traditionally studied algebraic
structures, e.g. vector spaces or polynomial rings, is the manner in
which dynamics is captured. In traditional structures, dynamics is typically
expressed through morphisms between such structures, as in linear maps
between vector spaces or morphisms between rings. In algebras
associated with the semantics of computation, the dynamics is
expressed as part of the algebraic structure itself, through a
reduction reduction relation typically denoted by $\red$. Below, we
give a recursive presentation of this relation for the calculus used
in the encoding.

$\red \subseteq \pi \times \pi$
$\red : \pi \to \mathcal{P}(\pi)$

\begin{mathpar}
  \inferrule* [lab=Comm] { \textsf{match}( x_{src}, x_{trgt} ) } { x_{trgt}?(y)P \; | \; x_{src}!\langle {Q} \rangle \red P\{\quotep{Q}/y}\} }
  \and \\
  \inferrule* [lab=Par] {{P} \red {P}'} {{{P} | {Q}} \red {{P}' | {Q}}}
  \and
  \inferrule* [lab=Equiv]{{{P} \scong {P}'} \andalso {{P}' \red {Q}'} \andalso {{Q}' \scong {Q}}}{{P} \red {Q}}
\end{mathpar}

\begin{eqnarray*}
  match_{\equiv} (\quotep{P},\quotep{Q}) & := & P \equiv Q \\
  match_{\dagger}(\quotep{P},\quotep{Q}) & := & \forall R. P|Q \red^{*} R => R \red^{*} 0 \\
  match_{K}(\quotep{P},\quotep{Q}) & := & K \mbox{ for some context } K
\end{eqnarray*}

$u?(x)P | u!\langle Q \rangle \red P\{\quotep{Q}/x\}$

%We write $\wred$ for $\red^*$, and $P\red$ if $\exists Q $ such that $ P \red Q$.
We write $P\red$ if $\exists Q $ such that $ P \red Q$ and $P\not\red$, otherwise.

\section{Replication}

As mentioned before, it is known that replication (and hence
recursion) can be implemented in a higher-order process algebra
\cite{SangiorgiWalker}. As our first example of calculation with the
machinery thus far presented we give the construction explicitly in
the {\rhoc}.

\begin{eqnarray}
	D_{x} & := & \prefix{x}{y}{(\binpar{\outputp{x}{y}}{@{y}})} \nonumber\\
	\bangp_{x}{P} & := & \binpar{{x}!\langle{\binpar{D_{x}}{P}}\rangle}{D_{x}} \nonumber
\end{eqnarray}

\begin{eqnarray}
	\bangp_{x}{P} & & \nonumber\\
	=
	& {x}!\langle{(\prefix{x}{y}{(\outputp{x}{y} | @{y})) | P}}\rangle 
	      | \prefix{x}{y}{(\outputp{x}{y} | @{y})} & \nonumber\\
	\red
	& (\outputp{x}{y} | @{y})\substn{\quotep{(\prefix{x}{y}{(@{y} | \outputp{x}{y})) | P}}}{y} & \nonumber\\
	=
	& \outputp{x}{\quotep{(\prefix{x}{y}{(\outputp{x}{y} | @{y})) | P}}}
	  | {(\prefix{x}{y}{(\outputp{x}{y} | @{y})) | P}} & \nonumber\\
	\red
	& \ldots & \nonumber\\
	\red^*
	& P | P | \ldots & \nonumber
\end{eqnarray}

Of course, this encoding, as an implementation, runs away, unfolding
$\bangp{P}$ eagerly. A lazier and more implementable replication
operator, restricted to input-guarded processes, may be obtained as follows.

\begin{eqnarray}
\bangp{\prefix{u}{v}{P}} 
	:= 
	\binpar{\lift{x}{\prefix{u}{v}{(\binpar{D(x)}{P})}}}{D(x)} \nonumber
\end{eqnarray}

\begin{remark}
  Note that the lazier definition still does not deal with summation
  or mixed summation (i.e. sums over input and output). The reader is
  invited to construct definitions of replication that deal with these
  features. 

  Further, the definitions are parameterized in a name, $x$. Can you,
  gentle reader, make a definition that eliminates this parameter and
  guarantees no accidental interaction between the replication
  machinery and the process being replicated -- i.e. no accidental
  sharing of names used by the process to get its work done and the
  name(s) used by the replication to effect copying. This latter
  revision of the definition of replication is crucial to obtaining
  the expected identity $!!P \sim !P$.
\end{remark}

\begin{remark}\label{rem:paradoxical_combinator}
  The reader familiar with the lambda calculus will have noticed the
  similarity between $D$ and the paradoxical combinator.

  [Ed. note: the existence of this seems to suggest we have to be more
  restrictive on the set of processes and names we admit if we are to
  support no-cloning.]
\end{remark}

\subsubsection{Bisimulation}

The computational dynamics gives rise to another kind of equivalence,
the equivalence of computational behavior. As previously mentioned
this is typically captured \emph{via} some form of bisimulation.

% The notion we use in this paper is weak barbed bisimulation
% \cite{milner91polyadicpi}.

The notion we use in this paper is derived from weak barbed
bisimulation \cite{milner91polyadicpi}. 

\begin{definition}
An \emph{observation relation}, $\downarrow_{\mathcal N}$, over a set
of names, $\mathcal N$, is the smallest relation satisfying the rules
below.

\infrule[Out-barb]{y \in {\mathcal N}, \; x \nameeq y}
		  {\outputp{x}{v} \downarrow_{\mathcal N} x}
\infrule[Par-barb]{\mbox{$P\downarrow_{\mathcal N} x$ or $Q\downarrow_{\mathcal N} x$}}
		  {\binpar{P}{Q} \downarrow_{\mathcal N} x}

We write $P \Downarrow_{\mathcal N} x$ if there is $Q$ such that 
$P \wred Q$ and $Q \downarrow_{\mathcal N} x$.
\end{definition}

\begin{definition}
%\label{def.bbisim}
An  ${\mathcal N}$-\emph{barbed bisimulation} over a set of names, ${\mathcal N}$, is a symmetric binary relation 
${\mathcal S}_{\mathcal N}$ between agents such that $P\rel{S}_{\mathcal N}Q$ implies:
\begin{enumerate}
\item If $P \red P'$ then $Q \wred Q'$ and $P'\rel{S}_{\mathcal N} Q'$.
\item If $P\downarrow_{\mathcal N} x$, then $Q\Downarrow_{\mathcal N} x$.
\end{enumerate}
$P$ is ${\mathcal N}$-barbed bisimilar to $Q$, written
$P \wbbisim_{\mathcal N} Q$, if $P \rel{S}_{\mathcal N} Q$ for some ${\mathcal N}$-barbed bisimulation ${\mathcal S}_{\mathcal N}$.
\end{definition}

$\mathcal{R} \subseteq \pi \times \pi$

$P \mathcal{R} Q => \forall P'. P \red P' \Rightarrow \exists Q'. Q \red Q', P' \mathcal{R} Q'$

$P \vdash x \Rightarrow Q \vdash x$

\begin{mathpar}
  \inferrule*[lab=Out-barb]{x \nameeq y}{{y}!\langle{Q}\rangle \vdash x}
  \and
  \inferrule*[lab=Par-barb]{\mbox{$P\vdash x$ or $Q\vdash x$}}{\binpar{P}{Q} \vdash x}
\end{mathpar}

\subsubsection{Contexts}

One of the principle advantages of computational calculi like the
$\pi$-calculus is a well-defined notion of context,
contextual-equivalence and a correlation between
contextual-equivalence and notions of bisimulation. The notion of
context allows the decomposition of a process into (sub-)process and
its syntactic environment, its context. Thus, a context may be
thought of as a process with a ``hole'' (written $\Box$) in it. The
application of a context $M$ to a process $P$, written $M[P]$, is
tantamount to filling the hole in $M$ with $P$. In this paper we do
not need the full weight of this theory, but do make use of the notion
of context in the proof the main theorem. 

\begin{mathpar}
  \inferrule* [lab=summation] {} {{M_{M},M_{N}} \bc \Box \;|\; x.M_{A} \;|\; M_{M}+M_{N}}
  \and
  \inferrule* [lab=agent] {} {{M_{A}} \bc (\vec{x})M_{P} \;| \; \clift{P_0,\ldots,M_{P},\ldots,P_N}}
  \and \\
  \inferrule* [lab=process] {} {{M_{P}} \bc M_{N} \;| \;P|M_{P} }
\end{mathpar} 

\begin{mathpar}
  \inferrule* [lab=sychronization] {} {M_{N} \bc \Box \;|\; x?M_{F} \;|\; x!M_{C}}
  \and
  \inferrule* [lab=abstraction] {} {{M_{F}} \bc (x)M_{P} }
  \and
  \inferrule* [lab=concretion] {} {{M_{C}} \bc \langle M_{P} \rangle }
  \and \\
  \inferrule* [lab=process] {} {{M_{P}} \bc M_{N} \;| \;P|M_{P} }
\end{mathpar}

\begin{definition}[contextual application] Given a context $M$, and
  process $P$, we define the \emph{contextual application}, $M[P] :=
  M\{P/\Box\}$. That is, the contextual application of M to P is the
  substitution of $P$ for $\Box$ in $M$.
\end{definition}

$\meaningof{-} : L \to \mathcal{P}(\pi)$

\begin{mathpar}
  \inferrule* [lab=collection] {} {\meaningof{true} = \pi, \and \meaningof{~E} = \pi \setminus \meaningof{E}, \and \meaningof{E_{1} \& E_{2}} = \meaningof{E_{1}} \cap \meaningof{E_{2}}}
\end{mathpar}

\begin{mathpar}
  \inferrule* [lab=structure] {} {\meaningof{0} = \{ P \in \pi | P \equiv 0 \}, \and \\ \meaningof{E_1 | E_2} = \{ P \in \pi | P \equiv P_{1} | P_{2}, P_{1} \in \meaningof{E_{1}}, P_{2} \in \meaningof{E_2}\} }
\end{mathpar}

\begin{mathpar}
 \inferrule* [lab=behavior] {} {\meaningof{\langle a?b \rangle E} = \{ P \in \pi | P \equiv Q | u?(y)P', \\ \and \\\\ \and \\ \;\;\; u \in \meaningof{a}, \forall z.P'\{z/y\} \in \meaningof{E\{z/b\}}\}, \and \\ \meaningof{a!E} = \{ P \in \pi | P \equiv Q | x!\langle P' \rangle, x \in \meaningof{a} P' \in \meaningof{E}\} }
\end{mathpar}

\begin{mathpar}
 \inferrule* [lab=nominal] {} {\meaningof{\quotep{E}} = \{ \quotep{P} \in \quotep{\pi} | P \in \meaningof{E} \}, \and \meaningof{\quotep{P}} = \{ \quotep{Q} \in \quotep{\pi} | P \equiv Q \} \and \\ \meaningof{@\quotep{E}} = \{ P \in \pi | P \equiv @x, x \in \meaningof{E} \}}
\end{mathpar}

\begin{eqnarray*}
  \\
  \meaningof{-} : TS \to ST
\end{eqnarray*}

\begin{eqnarray*}
  \\
  L : TS \to ST
\end{eqnarray*}

\begin{eqnarray*}
  \\
  P \models E \iff P \in \meaningof{E}
\end{eqnarray*}

\begin{eqnarray*}
  P \approx_{L} Q \iff \forall E \in L. P \models E \iff Q \models E
\end{eqnarray*}

\begin{eqnarray*}
  P \approx_{K} Q
\end{eqnarray*}

\begin{eqnarray*}
  P \approx Q
\end{eqnarray*}

$\approx_{K} = \approx = \approx_{L}$

\subsubsection{Contextual duality}

Note that contexts extend the quotation operation to a family of
operations from processes to names. Given a context, $M$, we can
define a \emph{nominal context}, $\quotep{M}$ by $\quotep{M}[P] :=
\quotep{M[P]}$. To foreshadow what is to come we observe that these
operations enjoy a duality with processes very much like the duality
between vectors and maps from vectors to scalars.

Further, because the calculus is essentially higher-order, we have a
correspondence between contexts and processes. More specifically,
given a name $x$ and a context $M$ we can construct $M^{*}_{x}$ such
that 

\begin{mathpar}
  M^{*}_{x} | \lift{x}{P} \red M[P]
\end{mathpar}

namely,

\begin{mathpar}
  M^{*}_{x} := x?(u).M[\dropn{u}]
\end{mathpar}

The dependence of $M^{*}_{x}$ on a name makes it an abstraction, 

\begin{mathpar}
  M^{*} := (x)x?(u).M[\dropn{u}]
\end{mathpar}

\subsection{Additional notation}

It will sometimes be convenient to denote the process a name
quotes. We already have the notation $x = \quotep{P}$, but it will be
convenient to introduce an alternate notation, $\procn{x}$, when we
want to emphasize the connection to the use of the name. Note that, by
virtue of name equivalence, $\quotep{\procn{x}} \nameeq x$; so, the
notation is consistent with previous definitions.

Further, because names have structure it is possible to effect
substitutions on the basis of that structure. This means we need to
upgrade our notation for substitutions, which we accomplish by
adapting comprehension notation. Thus,

\begin{mathpar}
  P\{ y / x : x \in S \}
\end{mathpar}

is interpreted to mean the process derived from P by replacing (in a
capture-avoiding manner) each occurrence of $x$ in $S$ by $y$. For example,

\begin{mathpar}
  P\{ \quotep{\procn{x}|\procn{x}} / x : x \in \freenames{P} \}
\end{mathpar}

will replace each (occurrence) of a free name $x$ in $P$ by
$\quotep{\procn{x}|\procn{x}}$.

Also, we will avail ourselves of the notation $x^{L}$ and $x^{R}$ to
denote injections of a name into disjoint copies of the name
space. There are numerous ways to accomplish this. One example can be
found in \cite{MeredithR05}. This notation overloads to vectors of
names: $\vec{x}^{\pi} := (x_{i}^{\pi} \; : \; 0 \leq i < |\vec{x}| )$ where $\pi \in \{L,R\}$.

We also use $P^{\Box} := P|\Box$.

In \cite{MeredithR05} an interpretation of the new operator is
given. It turns out that there are several possible interpretations
all enjoying the requisite algebraic properties of the operator (see
\cite{milner91polyadicpi}). We will therefore make liberal use of
$(\nu\; \vec{x})P$.

% subsection the_syntax_and_semantics_of_the_notation_system (end)   

\input{qm2pi.qmops} 

\input{qm2pi.sterngerlach} 

\input{qm2pi.metric} 

% section concurrent_process_calculi (end)

%\input{qm2pi.proofsketch}

% section proof sketch (end)

%\input{qm2pi.slviaknots} 

% section spatial logic via knots (end)

\input{qm2pi.conclusion}

% section conclusion (end)

%\input{qm2pi.dtcodes} 

% section wiring algorithm (end)

\input{qm2pi.ack} 

% section acknowledgments (end)

\newpage


\bibliographystyle{plain}   
\bibliography{../../biblios/main.bib}

\input{qm2pi.rhodetails}

\end{document}

 

%\ifpdf
%\usepackage[pdftex]{graphicx}
%\else
%\usepackage{graphicx}
%\fi

 % \ifpdf
%  \usepackage{pdfsync}
%  \if


%\title{Brief Article}
%\author{David F. Snyder}
%\author{L.G. Meredith}

%\address{Dept. of Math., Texas State University--San Marcos, San Marcos, TX 78666}
       
\pagestyle{empty}


\begin{document}

\lstset{language=[Objective]Caml,frame=shadowbox}

\documentclass[12pt]{llncs}
%\documentclass{jktr}

\usepackage[pdftex]{hyperref}                   
\usepackage {listings}
\usepackage {mathpartir}
\usepackage{bcprules}
%\usepackage{listings}
                       
\usepackage{graphicx} 
%\usepackage[margins=2.5cm,nohead,nofoot]{geometry}
%\usepackage{geometry}
\usepackage{amsfonts}
\usepackage{amstext}
\usepackage{latexsym}
\usepackage{amssymb}
\usepackage{color}


%\include{myPreamble}
\include{qm2pi.local} 

%\ifpdf
%\usepackage[pdftex]{graphicx}
%\else
%\usepackage{graphicx}
%\fi

 % \ifpdf
%  \usepackage{pdfsync}
%  \if


%\title{Brief Article}
%\author{David F. Snyder}
%\author{L.G. Meredith}

%\address{Dept. of Math., Texas State University--San Marcos, San Marcos, TX 78666}
       
\pagestyle{empty}


\begin{document}

\lstset{language=[Objective]Caml,frame=shadowbox}

\input{qm2pi.front}

% section front matter (end)

\input{qm2pi.intro} 
 
% section introduction (end)

% \input{qm2pi.knotations} 

% section notation (end)

\input{qm2pi.process.calculi} 

% section concurrent_process_calculi_and_spatial_logics_ (end)
    
%\input{qm2pi.knots2pi} 

%\input{qm2pi.trefoil} 

%\input{qm2pi.mainthm} 

% subsection basic_interpretation (end)

%\input{qm2pi.rho.presentation} 
\subsection{The syntax and semantics of the notation system}\label{sub:the_syntax_and_semantics_of_the_notation_system} % (fold)

We now summarize a technical presentation of the calculus that
embodies our theory of dynamics. The typical presentation of such a
calculus follows the style of giving generators and relations on
them. The grammar, below, describing term constructors, freely
generates the set of processes, $\Proc$. This set is then quotiented
by a relation known as structural congruence and it is over this set
that the notion of dynamics is expressed. This presentation is
essentially that of \cite{MeredithR05} with the addition of
polyadicity and summation. For readability we have relegated some of
the technical subtleties to an appendix.

\subsubsection{Process grammar}\label{subsub:process_grammar}

\begin{mathpar}
  \inferrule* [lab=synchronization] {} {{M} \bc \pzero \;|\; x?F \;|\; x!C }
  \and
  \inferrule* [lab=abstraction] {} {{F} \bc (x)P}
  \and
  \inferrule* [lab=concretion] {} {{C} \bc \langle Q \rangle}
  \and
  \inferrule* [lab=process] {} {{P,Q} \bc M \;| \;P|Q \;|\; @{x}}
  \and
  \inferrule* [lab=name] {} {{x} \bc \quotep{P}}
\end{mathpar} 

Note that $\vec{x}$ (resp. $\vec{P}$) denotes a vector of names
(resp. processes) of length $|\vec{x}|$ (resp. $|\vec{P}|$). We adopt
the following useful abbreviations.

\begin{mathpar}
   x?(\vec{y}).P := x.(\vec{y})P \and  x\clift{\vec{P}} := x.\clift{\vec{P}}
   \and x!(y) := \lift{x}{\dropn{y}}
   \and \Pi_{i=0}^{n-1}P_i := P_0 | \ldots | P_{n-1}
\end{mathpar}

\subsubsection{Structural congruence}

\paragraph{Free and bound names and alpha-equivalence.} At the
core of structural equivalence is alpha-equivalence which identifies
process that are the same up to a change of variable. Formally, we
recognize the distinction between free and bound names. The free names
of a process, $\freenames{P}$, may be calculated recursively as
follows:

\begin{mathpar}
\freenames{\pzero} := \emptyset
  \and \\
  \freenames{x?(y).P} := \{ x \} \cup (\freenames{P} \setminus \{ y \})
  \and 
  \freenames{x!\langle P \rangle} := \{ x \} \cup \{ P \} 
  \and \\
  \freenames{P|Q} := \freenames{P} \cup \freenames{Q}
  \and \\
  \freenames{@{x}} := \{ x \}
\end{mathpar}

$\pi$
$\quotep{\pi}$

$\freenames{-} : \pi \to \mathcal{P}(\quotep{\pi})$

\begin{eqnarray*}
  \freenames{\pzero} & := & \emptyset \\
  \freenames{x?(y).P} & := & \{ x \} \cup (\freenames{P} \setminus \{ y \}) \\
  \freenames{x!\langle P \rangle} & := & \{ x \} \cup \{ P \} \\
  \freenames{P|Q} & := & \freenames{P} \cup \freenames{Q} \\
  \freenames{\dropn{x}} & := & \{ x \}
\end{eqnarray*}

The bound names of a process, $\boundnames{P}$, are those names occurring in $P$
that are not free. For example, in $x?(y).0$, the name $x$ is free, while $y$ is bound.

\begin{mathpar}
  \inferrule* [lab=monoidal-laws] {} { P|Q \equiv Q|P \and P|0 \equiv P \and P|(Q|R) \equiv (P|Q)|R }
\end{mathpar}

\begin{mathpar}
  \inferrule* [lab=alpha-equivalence] {} { (x)P \equiv (y)P\{y/x\} \and y \not\in \freenames{P} }
\end{mathpar}

\begin{definition}
Then two processes, $P,Q$, are alpha-equivalent if $P = Q\{\vec{y}/\vec{x}\}$ for
some $\vec{x} \in \boundnames{Q},\vec{y} \in \boundnames{P}$, where $Q\{\vec{y}/\vec{x}\}$
denotes the capture-avoiding substitution of $\vec{y}$ for $\vec{x}$ in $Q$.
\end{definition}

\begin{definition}
  The {\em structural congruence} \cite{SangiorgiWalker} , $\equiv$,
  between processes is the least congruence containing
  alpha-equivalence, satisfying the abelian monoid laws
  (associativity, commutativity and $\pzero$ as identity) for parallel
  composition $|$ and for summation $+$.
\end{definition}

\subsection{Name equivalence}

We take name equivalence, written $\nameeq$, to be the smallest
equivalence relation generated by the following rules.

\begin{mathpar}
\inferrule*[lab=Quote-drop]
{ }
{ \quotep{@{x}} \nameeq x }

\inferrule*[lab=Struct-equiv]
{ P \scong Q }
{ \quotep{P} \nameeq \quotep{Q} }
\end{mathpar}

The astute reader will have noticed that the mutual recursion of names
and processes imposes a mutual recursion on alpha-equivalence and
structural equivalence via name-equivalence. Fortunately, all of this
works out pleasantly and we may calculate in the natural way, free of
concern. The reader interested in the details is referred to the
appendix \ref{appendix:rho_details}.

\subsection{Substitution}

We use $\Proc$ for the set of processes, $\QProc$ for the set of
names, and $\id{\{}\vec{y} / \vec{x} \id{\}}$ to denote partial maps,
$s : \QProc \rightarrow \QProc$. A map, $s$ lifts, uniquely, to a map
on process terms, $\widehat{s} : \Proc \rightarrow \Proc$ by the
following equations.

\begin{mathpar}
  (0) \psubstp{Q}{P} := 0 \\
  (R \juxtap S) \psubstp{Q}{P}
  :=    
  (R)\psubstp{Q}{P} \juxtap (S) \psubstp{Q}{P} \\
  (x?(y).R) \psubstp{Q}{P}    
  :=    
  (x)\substp{Q}{P} (z)\concat( (R \psubstn{z}{y}) \psubstp{Q}{P} ) \\
  (\lift{x}{R}) \psubstp{Q}{P}  
  :=
  \lift{(x)\substp{Q}{P}}{ R \psubstp{Q}{P} } \\
%   (\dropn{x})  \psubstp{Q}{P}       
%   := 
%   \left\{ 
%     \begin{array}{ccc} 
%       \dropn{\quotep{Q}} & & x \nameeq \quotep{P} \\
%       \dropn{x} & & otherwise \\
%     \end{array}
%   \right. 
  (\dropn{x})  \psubstp{Q}{P}       
  := 
  \left\{ 
    \begin{array}{ccc} 
      Q & & x \nameeq \quotep{P} \\
      \dropn{x} & & otherwise \\
    \end{array}
  \right.
\end{mathpar}
 

where

\begin{eqnarray}
  (x)\id{\{} \lpquote Q \rpquote / \lpquote P \rpquote \id{\}}            = 
  \left\{ 
    \begin{array}{ccc}
      \lpquote Q \rpquote & & x \nameeq \lpquote P \rpquote \\
      x & & otherwise \\
    \end{array}
  \right. \nonumber
\end{eqnarray}

and $z$ is chosen distinct from $\quotep{P}$, $\quotep{Q}$, the free
names in $Q$, and all the names in $R$. Our $\alpha$-equivalence will
be built in the standard way from this substitution.

\begin{remark}\label{rem:no_self_referential_names}
  One consequence of these definitions is that $\forall P. \quotep{P}
  \not\in \freenames{P}$.
\end{remark}

\subsection{ Dynamic quote: an example }

Anticipating something of what's to come, consider applying the
substitution, $\widehat{\id{\{}u / z \id{\}}}$, to the following pair
of processes, $\lift{w}{y!(z)}$ and $w[ \lpquote y!(z) \rpquote ]$.

\begin{eqnarray}
	\lift{w}{y!(z)}\widehat{\id{\{}u / z \id{\}}}
		& = &
		\lift{w}{y!(u)} \nonumber\\
	w[ \lpquote y!(z) \rpquote ] \widehat{ \id{\{}u / z \id{\}} }
		& = &
		w[ \lpquote y!(z) \rpquote ] \nonumber
\end{eqnarray}

Because the body of the process between quotes is impervious to
substitution, we get radically different answers. In fact, by
examining the first process in an input context,
e.g. $x?(z).\lift{w}{y!(z)}$, we see that the process under the lift
operator may be shaped by prefixed inputs binding a name inside it. In
this sense, the lift operator will be seen as a way to dynamically
construct processes before reifying them as names.

Finally equipped with these standard features we can present the
dynamics of the calculus.

\subsubsection{Operational semantics} 

Finally, we introduce the computational dynamics. What marks these
algebras as distinct from other more traditionally studied algebraic
structures, e.g. vector spaces or polynomial rings, is the manner in
which dynamics is captured. In traditional structures, dynamics is typically
expressed through morphisms between such structures, as in linear maps
between vector spaces or morphisms between rings. In algebras
associated with the semantics of computation, the dynamics is
expressed as part of the algebraic structure itself, through a
reduction reduction relation typically denoted by $\red$. Below, we
give a recursive presentation of this relation for the calculus used
in the encoding.

$\red \subseteq \pi \times \pi$
$\red : \pi \to \mathcal{P}(\pi)$

\begin{mathpar}
  \inferrule* [lab=Comm] { \textsf{match}( x_{src}, x_{trgt} ) } { x_{trgt}?(y)P \; | \; x_{src}!\langle {Q} \rangle \red P\{\quotep{Q}/y}\} }
  \and \\
  \inferrule* [lab=Par] {{P} \red {P}'} {{{P} | {Q}} \red {{P}' | {Q}}}
  \and
  \inferrule* [lab=Equiv]{{{P} \scong {P}'} \andalso {{P}' \red {Q}'} \andalso {{Q}' \scong {Q}}}{{P} \red {Q}}
\end{mathpar}

\begin{eqnarray*}
  match_{\equiv} (\quotep{P},\quotep{Q}) & := & P \equiv Q \\
  match_{\dagger}(\quotep{P},\quotep{Q}) & := & \forall R. P|Q \red^{*} R => R \red^{*} 0 \\
  match_{K}(\quotep{P},\quotep{Q}) & := & K \mbox{ for some context } K
\end{eqnarray*}

$u?(x)P | u!\langle Q \rangle \red P\{\quotep{Q}/x\}$

%We write $\wred$ for $\red^*$, and $P\red$ if $\exists Q $ such that $ P \red Q$.
We write $P\red$ if $\exists Q $ such that $ P \red Q$ and $P\not\red$, otherwise.

\section{Replication}

As mentioned before, it is known that replication (and hence
recursion) can be implemented in a higher-order process algebra
\cite{SangiorgiWalker}. As our first example of calculation with the
machinery thus far presented we give the construction explicitly in
the {\rhoc}.

\begin{eqnarray}
	D_{x} & := & \prefix{x}{y}{(\binpar{\outputp{x}{y}}{@{y}})} \nonumber\\
	\bangp_{x}{P} & := & \binpar{{x}!\langle{\binpar{D_{x}}{P}}\rangle}{D_{x}} \nonumber
\end{eqnarray}

\begin{eqnarray}
	\bangp_{x}{P} & & \nonumber\\
	=
	& {x}!\langle{(\prefix{x}{y}{(\outputp{x}{y} | @{y})) | P}}\rangle 
	      | \prefix{x}{y}{(\outputp{x}{y} | @{y})} & \nonumber\\
	\red
	& (\outputp{x}{y} | @{y})\substn{\quotep{(\prefix{x}{y}{(@{y} | \outputp{x}{y})) | P}}}{y} & \nonumber\\
	=
	& \outputp{x}{\quotep{(\prefix{x}{y}{(\outputp{x}{y} | @{y})) | P}}}
	  | {(\prefix{x}{y}{(\outputp{x}{y} | @{y})) | P}} & \nonumber\\
	\red
	& \ldots & \nonumber\\
	\red^*
	& P | P | \ldots & \nonumber
\end{eqnarray}

Of course, this encoding, as an implementation, runs away, unfolding
$\bangp{P}$ eagerly. A lazier and more implementable replication
operator, restricted to input-guarded processes, may be obtained as follows.

\begin{eqnarray}
\bangp{\prefix{u}{v}{P}} 
	:= 
	\binpar{\lift{x}{\prefix{u}{v}{(\binpar{D(x)}{P})}}}{D(x)} \nonumber
\end{eqnarray}

\begin{remark}
  Note that the lazier definition still does not deal with summation
  or mixed summation (i.e. sums over input and output). The reader is
  invited to construct definitions of replication that deal with these
  features. 

  Further, the definitions are parameterized in a name, $x$. Can you,
  gentle reader, make a definition that eliminates this parameter and
  guarantees no accidental interaction between the replication
  machinery and the process being replicated -- i.e. no accidental
  sharing of names used by the process to get its work done and the
  name(s) used by the replication to effect copying. This latter
  revision of the definition of replication is crucial to obtaining
  the expected identity $!!P \sim !P$.
\end{remark}

\begin{remark}\label{rem:paradoxical_combinator}
  The reader familiar with the lambda calculus will have noticed the
  similarity between $D$ and the paradoxical combinator.

  [Ed. note: the existence of this seems to suggest we have to be more
  restrictive on the set of processes and names we admit if we are to
  support no-cloning.]
\end{remark}

\subsubsection{Bisimulation}

The computational dynamics gives rise to another kind of equivalence,
the equivalence of computational behavior. As previously mentioned
this is typically captured \emph{via} some form of bisimulation.

% The notion we use in this paper is weak barbed bisimulation
% \cite{milner91polyadicpi}.

The notion we use in this paper is derived from weak barbed
bisimulation \cite{milner91polyadicpi}. 

\begin{definition}
An \emph{observation relation}, $\downarrow_{\mathcal N}$, over a set
of names, $\mathcal N$, is the smallest relation satisfying the rules
below.

\infrule[Out-barb]{y \in {\mathcal N}, \; x \nameeq y}
		  {\outputp{x}{v} \downarrow_{\mathcal N} x}
\infrule[Par-barb]{\mbox{$P\downarrow_{\mathcal N} x$ or $Q\downarrow_{\mathcal N} x$}}
		  {\binpar{P}{Q} \downarrow_{\mathcal N} x}

We write $P \Downarrow_{\mathcal N} x$ if there is $Q$ such that 
$P \wred Q$ and $Q \downarrow_{\mathcal N} x$.
\end{definition}

\begin{definition}
%\label{def.bbisim}
An  ${\mathcal N}$-\emph{barbed bisimulation} over a set of names, ${\mathcal N}$, is a symmetric binary relation 
${\mathcal S}_{\mathcal N}$ between agents such that $P\rel{S}_{\mathcal N}Q$ implies:
\begin{enumerate}
\item If $P \red P'$ then $Q \wred Q'$ and $P'\rel{S}_{\mathcal N} Q'$.
\item If $P\downarrow_{\mathcal N} x$, then $Q\Downarrow_{\mathcal N} x$.
\end{enumerate}
$P$ is ${\mathcal N}$-barbed bisimilar to $Q$, written
$P \wbbisim_{\mathcal N} Q$, if $P \rel{S}_{\mathcal N} Q$ for some ${\mathcal N}$-barbed bisimulation ${\mathcal S}_{\mathcal N}$.
\end{definition}

$\mathcal{R} \subseteq \pi \times \pi$

$P \mathcal{R} Q => \forall P'. P \red P' \Rightarrow \exists Q'. Q \red Q', P' \mathcal{R} Q'$

$P \vdash x \Rightarrow Q \vdash x$

\begin{mathpar}
  \inferrule*[lab=Out-barb]{x \nameeq y}{{y}!\langle{Q}\rangle \vdash x}
  \and
  \inferrule*[lab=Par-barb]{\mbox{$P\vdash x$ or $Q\vdash x$}}{\binpar{P}{Q} \vdash x}
\end{mathpar}

\subsubsection{Contexts}

One of the principle advantages of computational calculi like the
$\pi$-calculus is a well-defined notion of context,
contextual-equivalence and a correlation between
contextual-equivalence and notions of bisimulation. The notion of
context allows the decomposition of a process into (sub-)process and
its syntactic environment, its context. Thus, a context may be
thought of as a process with a ``hole'' (written $\Box$) in it. The
application of a context $M$ to a process $P$, written $M[P]$, is
tantamount to filling the hole in $M$ with $P$. In this paper we do
not need the full weight of this theory, but do make use of the notion
of context in the proof the main theorem. 

\begin{mathpar}
  \inferrule* [lab=summation] {} {{M_{M},M_{N}} \bc \Box \;|\; x.M_{A} \;|\; M_{M}+M_{N}}
  \and
  \inferrule* [lab=agent] {} {{M_{A}} \bc (\vec{x})M_{P} \;| \; \clift{P_0,\ldots,M_{P},\ldots,P_N}}
  \and \\
  \inferrule* [lab=process] {} {{M_{P}} \bc M_{N} \;| \;P|M_{P} }
\end{mathpar} 

\begin{mathpar}
  \inferrule* [lab=sychronization] {} {M_{N} \bc \Box \;|\; x?M_{F} \;|\; x!M_{C}}
  \and
  \inferrule* [lab=abstraction] {} {{M_{F}} \bc (x)M_{P} }
  \and
  \inferrule* [lab=concretion] {} {{M_{C}} \bc \langle M_{P} \rangle }
  \and \\
  \inferrule* [lab=process] {} {{M_{P}} \bc M_{N} \;| \;P|M_{P} }
\end{mathpar}

\begin{definition}[contextual application] Given a context $M$, and
  process $P$, we define the \emph{contextual application}, $M[P] :=
  M\{P/\Box\}$. That is, the contextual application of M to P is the
  substitution of $P$ for $\Box$ in $M$.
\end{definition}

$\meaningof{-} : L \to \mathcal{P}(\pi)$

\begin{mathpar}
  \inferrule* [lab=collection] {} {\meaningof{true} = \pi, \and \meaningof{~E} = \pi \setminus \meaningof{E}, \and \meaningof{E_{1} \& E_{2}} = \meaningof{E_{1}} \cap \meaningof{E_{2}}}
\end{mathpar}

\begin{mathpar}
  \inferrule* [lab=structure] {} {\meaningof{0} = \{ P \in \pi | P \equiv 0 \}, \and \\ \meaningof{E_1 | E_2} = \{ P \in \pi | P \equiv P_{1} | P_{2}, P_{1} \in \meaningof{E_{1}}, P_{2} \in \meaningof{E_2}\} }
\end{mathpar}

\begin{mathpar}
 \inferrule* [lab=behavior] {} {\meaningof{\langle a?b \rangle E} = \{ P \in \pi | P \equiv Q | u?(y)P', \\ \and \\\\ \and \\ \;\;\; u \in \meaningof{a}, \forall z.P'\{z/y\} \in \meaningof{E\{z/b\}}\}, \and \\ \meaningof{a!E} = \{ P \in \pi | P \equiv Q | x!\langle P' \rangle, x \in \meaningof{a} P' \in \meaningof{E}\} }
\end{mathpar}

\begin{mathpar}
 \inferrule* [lab=nominal] {} {\meaningof{\quotep{E}} = \{ \quotep{P} \in \quotep{\pi} | P \in \meaningof{E} \}, \and \meaningof{\quotep{P}} = \{ \quotep{Q} \in \quotep{\pi} | P \equiv Q \} \and \\ \meaningof{@\quotep{E}} = \{ P \in \pi | P \equiv @x, x \in \meaningof{E} \}}
\end{mathpar}

\begin{eqnarray*}
  \\
  \meaningof{-} : TS \to ST
\end{eqnarray*}

\begin{eqnarray*}
  \\
  L : TS \to ST
\end{eqnarray*}

\begin{eqnarray*}
  \\
  P \models E \iff P \in \meaningof{E}
\end{eqnarray*}

\begin{eqnarray*}
  P \approx_{L} Q \iff \forall E \in L. P \models E \iff Q \models E
\end{eqnarray*}

\begin{eqnarray*}
  P \approx_{K} Q
\end{eqnarray*}

\begin{eqnarray*}
  P \approx Q
\end{eqnarray*}

$\approx_{K} = \approx = \approx_{L}$

\subsubsection{Contextual duality}

Note that contexts extend the quotation operation to a family of
operations from processes to names. Given a context, $M$, we can
define a \emph{nominal context}, $\quotep{M}$ by $\quotep{M}[P] :=
\quotep{M[P]}$. To foreshadow what is to come we observe that these
operations enjoy a duality with processes very much like the duality
between vectors and maps from vectors to scalars.

Further, because the calculus is essentially higher-order, we have a
correspondence between contexts and processes. More specifically,
given a name $x$ and a context $M$ we can construct $M^{*}_{x}$ such
that 

\begin{mathpar}
  M^{*}_{x} | \lift{x}{P} \red M[P]
\end{mathpar}

namely,

\begin{mathpar}
  M^{*}_{x} := x?(u).M[\dropn{u}]
\end{mathpar}

The dependence of $M^{*}_{x}$ on a name makes it an abstraction, 

\begin{mathpar}
  M^{*} := (x)x?(u).M[\dropn{u}]
\end{mathpar}

\subsection{Additional notation}

It will sometimes be convenient to denote the process a name
quotes. We already have the notation $x = \quotep{P}$, but it will be
convenient to introduce an alternate notation, $\procn{x}$, when we
want to emphasize the connection to the use of the name. Note that, by
virtue of name equivalence, $\quotep{\procn{x}} \nameeq x$; so, the
notation is consistent with previous definitions.

Further, because names have structure it is possible to effect
substitutions on the basis of that structure. This means we need to
upgrade our notation for substitutions, which we accomplish by
adapting comprehension notation. Thus,

\begin{mathpar}
  P\{ y / x : x \in S \}
\end{mathpar}

is interpreted to mean the process derived from P by replacing (in a
capture-avoiding manner) each occurrence of $x$ in $S$ by $y$. For example,

\begin{mathpar}
  P\{ \quotep{\procn{x}|\procn{x}} / x : x \in \freenames{P} \}
\end{mathpar}

will replace each (occurrence) of a free name $x$ in $P$ by
$\quotep{\procn{x}|\procn{x}}$.

Also, we will avail ourselves of the notation $x^{L}$ and $x^{R}$ to
denote injections of a name into disjoint copies of the name
space. There are numerous ways to accomplish this. One example can be
found in \cite{MeredithR05}. This notation overloads to vectors of
names: $\vec{x}^{\pi} := (x_{i}^{\pi} \; : \; 0 \leq i < |\vec{x}| )$ where $\pi \in \{L,R\}$.

We also use $P^{\Box} := P|\Box$.

In \cite{MeredithR05} an interpretation of the new operator is
given. It turns out that there are several possible interpretations
all enjoying the requisite algebraic properties of the operator (see
\cite{milner91polyadicpi}). We will therefore make liberal use of
$(\nu\; \vec{x})P$.

% subsection the_syntax_and_semantics_of_the_notation_system (end)   

\input{qm2pi.qmops} 

\input{qm2pi.sterngerlach} 

\input{qm2pi.metric} 

% section concurrent_process_calculi (end)

%\input{qm2pi.proofsketch}

% section proof sketch (end)

%\input{qm2pi.slviaknots} 

% section spatial logic via knots (end)

\input{qm2pi.conclusion}

% section conclusion (end)

%\input{qm2pi.dtcodes} 

% section wiring algorithm (end)

\input{qm2pi.ack} 

% section acknowledgments (end)

\newpage


\bibliographystyle{plain}   
\bibliography{../../biblios/main.bib}

\input{qm2pi.rhodetails}

\end{document}



% section front matter (end)

\section{Introduction}\label{sec:introduction} % (fold)
In this draft of the material i am going to have to dispense with the
usual writing conventions adopted in papers on these topics. i'm going
to have adopt whatever tone i need at the time i'm writing up the
calculations. Sometimes this may be very conversational; others it may
be the barest mathematical grunts; others still it may be that i have
lifted text from one of my other papers because the exposition of some
point was better said there. i hope that my readers are not unduly put
out by this decision. i'm not doing this to flout convention or be
rebellious. i find these calculations very technically challenging. To
keep everything going technically, something has to give; i have to
let go of some cognitive burden. So, the academic writing style --
with all of its trade-offs in terms of facilitating technical
communication -- is what i'm letting go of. Perhaps subsequent drafts
can be tightened and polished, but for now, i'm going to speak as if
we were sitting together in a coffee shop with a laptop, wifi and a
pad of paper and a pencil.

So, here's what i have to say. We -- you and i, comfortably ensconced
in our coffee shop and well-equipped with our tools -- can realize and
carry out the calculations of quantum mechanics over a very different
formal theory of dynamics, a formal theory of dynamics that
corresponds to a theory of concurrent computation with
\emph{reflection}. It has the advantage that the underlying theory is
already `quantized', but supports analogues all of the continuuous
operations. Strikingly, this underlying theory has recently been
connected with a notion of metric that we can show, by calculating
together, coincides with the metric induced by the inner product.

There are a lot of reasons why you might be interested in seeing
calculations of this form. Here's why i'm interested. For the past
several centuries there has been no competitor to the ``Newtonian''
account of dynamics. As a result the predominant share of accounts of
dynamical systems and situations have had to be formulated in terms of
the Newtonian machinery. i view this as an intellectually dangerous
position to occupy. Everything, despite it's intrinsic shape, turns
into a nail to be hit with this hammer. Recently, however, the theory
of computation has matured to the point where we have candidates for
theories of dynamics that offer very different perspective on
reasoning about dynamical systems and situations. Testing these
candidates against very successful accounts of dynamical situations,
like quantum mechanics, is going to give us some sense of how mature
they are and some measure of the quality of these accounts of
dynamics.

\subsection{Summary of contributions and outline of paper}

So, we're going to develop an interpretation of the operations of
quantum mechanics normally interpreted by Hilbert spaces and
operators. We're going to do this over a theory of computation. Note
that this is very different than the usual quantum computation program
which develops notions of computation over quantum mechanics. Rather,
we are developing a story that aligns with Wheeler's slogan: It from
Bit. To do this we will first provide an account of the theory of
computation at play here. Then we will dive into a calculation-driven
interpretation of the operations of quantum mechanics.

The reason we take this approach is that -- until very recently --
there hasn't been an axiomatic account of quantum mechanics. As a
result there has been no sharp delineation of the mathematical theory
supporting interpretation of the physical theory and the physical
theory, itself. So, ambient features of the maths are free to be
exploited (or supressed) without a real accounting of their physical
relevance. There is no sharp statement ``here's the physical theory''
qua \emph{theory} and ``here's the mathematical interpretation''
enabling a judgment of how faithful the interpretation is -- apart
from experimental observation. When there is an axiomatic account we
can judge how well a given mathematical formalism supports an
interpretation of the axioms, independent of
experimentation. Likewise, we can judge how well we have captured our
physical evidence and experience with our axiomatics, independent of
any specific mathematical implementation, with accidental detail that
may or may not have physical significance. 

In lieu of a fully fleshed out and vetted axiomatic account of quantum
mechanics, interpreting the operational notions in service of modeling
physical systems will have to suffice. In other words, we are not in
the business of providing a model of Hilbert spaces and operators. We
are in the business of providing a model of quantum mechanics because
we are motivated by testing our notions of dynamics against physical
theory; and, the predictive calculations of the physical theory must
serve as the best formulation -- shy of a fully fleshed out axiomatic
account -- of the physical theory itself (as they have for scientific
theories since time immemorial). Put another way, despite a
whole-hearted commitment to an It-from-Bit ontology, we are firmly
aligned with the shut-up-and-calculate camp as the best way to obtain
results either from the physical perspective or as a quality assurance
measure of our fledgling theory of dynamics.

In detail, we present a reflective process calculus. Then we develop
intuitive correspondences between the notions available in this
calculus and the usual physical notions supporting quantum mechanical
calculations. Thus, 

\begin{table}[htp]
  \center{
    \fbox{
      \begin{tabular}{c|c}
        quantum mechanics & process calculus \\
        \hline
        scalar & name \\
        state vector & process \\
        dual & contextual duals \\
        matrix & formal sums of process-context-dual pairs \\
        orthogonality & process annihilation \\
        inner product & execution-formula + quoting
      \end{tabular}
    }
  }
  \caption{QM - process calculi correspondences}
\end{table}

Then we tighten up these intuitions to operational definitions. We
employ the Dirac notation as the best proxy we can find for an
abstract syntax of the quantum mechanical notions. The definitions we
develop put us in contact with equational constraints coming from the
theory that we demonstrate the definitions and calculations satisfy.

This puts us in a position to shut up and calculate for the
Stern-Gerlach experimental set up, showing how these predictive
calculations become calculations on processes in our theory of a
reflective process calculus.

Penultimately, we demonstrate that the notion of metric coming from
the inner product coincides with the notion of metric available from
the theory of bisimulation. This demonstration gives us the right to
think of space as arising from behavior. Finally, we consider where we
might go from the new vantage point we have obtained.

% section introduction (end) 
 
% section introduction (end)

% \documentclass[12pt]{llncs}
%\documentclass{jktr}

\usepackage[pdftex]{hyperref}                   
\usepackage {listings}
\usepackage {mathpartir}
\usepackage{bcprules}
%\usepackage{listings}
                       
\usepackage{graphicx} 
%\usepackage[margins=2.5cm,nohead,nofoot]{geometry}
%\usepackage{geometry}
\usepackage{amsfonts}
\usepackage{amstext}
\usepackage{latexsym}
\usepackage{amssymb}
\usepackage{color}


%\include{myPreamble}
\include{qm2pi.local} 

%\ifpdf
%\usepackage[pdftex]{graphicx}
%\else
%\usepackage{graphicx}
%\fi

 % \ifpdf
%  \usepackage{pdfsync}
%  \if


%\title{Brief Article}
%\author{David F. Snyder}
%\author{L.G. Meredith}

%\address{Dept. of Math., Texas State University--San Marcos, San Marcos, TX 78666}
       
\pagestyle{empty}


\begin{document}

\lstset{language=[Objective]Caml,frame=shadowbox}

\input{qm2pi.front}

% section front matter (end)

\input{qm2pi.intro} 
 
% section introduction (end)

% \input{qm2pi.knotations} 

% section notation (end)

\input{qm2pi.process.calculi} 

% section concurrent_process_calculi_and_spatial_logics_ (end)
    
%\input{qm2pi.knots2pi} 

%\input{qm2pi.trefoil} 

%\input{qm2pi.mainthm} 

% subsection basic_interpretation (end)

%\input{qm2pi.rho.presentation} 
\subsection{The syntax and semantics of the notation system}\label{sub:the_syntax_and_semantics_of_the_notation_system} % (fold)

We now summarize a technical presentation of the calculus that
embodies our theory of dynamics. The typical presentation of such a
calculus follows the style of giving generators and relations on
them. The grammar, below, describing term constructors, freely
generates the set of processes, $\Proc$. This set is then quotiented
by a relation known as structural congruence and it is over this set
that the notion of dynamics is expressed. This presentation is
essentially that of \cite{MeredithR05} with the addition of
polyadicity and summation. For readability we have relegated some of
the technical subtleties to an appendix.

\subsubsection{Process grammar}\label{subsub:process_grammar}

\begin{mathpar}
  \inferrule* [lab=synchronization] {} {{M} \bc \pzero \;|\; x?F \;|\; x!C }
  \and
  \inferrule* [lab=abstraction] {} {{F} \bc (x)P}
  \and
  \inferrule* [lab=concretion] {} {{C} \bc \langle Q \rangle}
  \and
  \inferrule* [lab=process] {} {{P,Q} \bc M \;| \;P|Q \;|\; @{x}}
  \and
  \inferrule* [lab=name] {} {{x} \bc \quotep{P}}
\end{mathpar} 

Note that $\vec{x}$ (resp. $\vec{P}$) denotes a vector of names
(resp. processes) of length $|\vec{x}|$ (resp. $|\vec{P}|$). We adopt
the following useful abbreviations.

\begin{mathpar}
   x?(\vec{y}).P := x.(\vec{y})P \and  x\clift{\vec{P}} := x.\clift{\vec{P}}
   \and x!(y) := \lift{x}{\dropn{y}}
   \and \Pi_{i=0}^{n-1}P_i := P_0 | \ldots | P_{n-1}
\end{mathpar}

\subsubsection{Structural congruence}

\paragraph{Free and bound names and alpha-equivalence.} At the
core of structural equivalence is alpha-equivalence which identifies
process that are the same up to a change of variable. Formally, we
recognize the distinction between free and bound names. The free names
of a process, $\freenames{P}$, may be calculated recursively as
follows:

\begin{mathpar}
\freenames{\pzero} := \emptyset
  \and \\
  \freenames{x?(y).P} := \{ x \} \cup (\freenames{P} \setminus \{ y \})
  \and 
  \freenames{x!\langle P \rangle} := \{ x \} \cup \{ P \} 
  \and \\
  \freenames{P|Q} := \freenames{P} \cup \freenames{Q}
  \and \\
  \freenames{@{x}} := \{ x \}
\end{mathpar}

$\pi$
$\quotep{\pi}$

$\freenames{-} : \pi \to \mathcal{P}(\quotep{\pi})$

\begin{eqnarray*}
  \freenames{\pzero} & := & \emptyset \\
  \freenames{x?(y).P} & := & \{ x \} \cup (\freenames{P} \setminus \{ y \}) \\
  \freenames{x!\langle P \rangle} & := & \{ x \} \cup \{ P \} \\
  \freenames{P|Q} & := & \freenames{P} \cup \freenames{Q} \\
  \freenames{\dropn{x}} & := & \{ x \}
\end{eqnarray*}

The bound names of a process, $\boundnames{P}$, are those names occurring in $P$
that are not free. For example, in $x?(y).0$, the name $x$ is free, while $y$ is bound.

\begin{mathpar}
  \inferrule* [lab=monoidal-laws] {} { P|Q \equiv Q|P \and P|0 \equiv P \and P|(Q|R) \equiv (P|Q)|R }
\end{mathpar}

\begin{mathpar}
  \inferrule* [lab=alpha-equivalence] {} { (x)P \equiv (y)P\{y/x\} \and y \not\in \freenames{P} }
\end{mathpar}

\begin{definition}
Then two processes, $P,Q$, are alpha-equivalent if $P = Q\{\vec{y}/\vec{x}\}$ for
some $\vec{x} \in \boundnames{Q},\vec{y} \in \boundnames{P}$, where $Q\{\vec{y}/\vec{x}\}$
denotes the capture-avoiding substitution of $\vec{y}$ for $\vec{x}$ in $Q$.
\end{definition}

\begin{definition}
  The {\em structural congruence} \cite{SangiorgiWalker} , $\equiv$,
  between processes is the least congruence containing
  alpha-equivalence, satisfying the abelian monoid laws
  (associativity, commutativity and $\pzero$ as identity) for parallel
  composition $|$ and for summation $+$.
\end{definition}

\subsection{Name equivalence}

We take name equivalence, written $\nameeq$, to be the smallest
equivalence relation generated by the following rules.

\begin{mathpar}
\inferrule*[lab=Quote-drop]
{ }
{ \quotep{@{x}} \nameeq x }

\inferrule*[lab=Struct-equiv]
{ P \scong Q }
{ \quotep{P} \nameeq \quotep{Q} }
\end{mathpar}

The astute reader will have noticed that the mutual recursion of names
and processes imposes a mutual recursion on alpha-equivalence and
structural equivalence via name-equivalence. Fortunately, all of this
works out pleasantly and we may calculate in the natural way, free of
concern. The reader interested in the details is referred to the
appendix \ref{appendix:rho_details}.

\subsection{Substitution}

We use $\Proc$ for the set of processes, $\QProc$ for the set of
names, and $\id{\{}\vec{y} / \vec{x} \id{\}}$ to denote partial maps,
$s : \QProc \rightarrow \QProc$. A map, $s$ lifts, uniquely, to a map
on process terms, $\widehat{s} : \Proc \rightarrow \Proc$ by the
following equations.

\begin{mathpar}
  (0) \psubstp{Q}{P} := 0 \\
  (R \juxtap S) \psubstp{Q}{P}
  :=    
  (R)\psubstp{Q}{P} \juxtap (S) \psubstp{Q}{P} \\
  (x?(y).R) \psubstp{Q}{P}    
  :=    
  (x)\substp{Q}{P} (z)\concat( (R \psubstn{z}{y}) \psubstp{Q}{P} ) \\
  (\lift{x}{R}) \psubstp{Q}{P}  
  :=
  \lift{(x)\substp{Q}{P}}{ R \psubstp{Q}{P} } \\
%   (\dropn{x})  \psubstp{Q}{P}       
%   := 
%   \left\{ 
%     \begin{array}{ccc} 
%       \dropn{\quotep{Q}} & & x \nameeq \quotep{P} \\
%       \dropn{x} & & otherwise \\
%     \end{array}
%   \right. 
  (\dropn{x})  \psubstp{Q}{P}       
  := 
  \left\{ 
    \begin{array}{ccc} 
      Q & & x \nameeq \quotep{P} \\
      \dropn{x} & & otherwise \\
    \end{array}
  \right.
\end{mathpar}
 

where

\begin{eqnarray}
  (x)\id{\{} \lpquote Q \rpquote / \lpquote P \rpquote \id{\}}            = 
  \left\{ 
    \begin{array}{ccc}
      \lpquote Q \rpquote & & x \nameeq \lpquote P \rpquote \\
      x & & otherwise \\
    \end{array}
  \right. \nonumber
\end{eqnarray}

and $z$ is chosen distinct from $\quotep{P}$, $\quotep{Q}$, the free
names in $Q$, and all the names in $R$. Our $\alpha$-equivalence will
be built in the standard way from this substitution.

\begin{remark}\label{rem:no_self_referential_names}
  One consequence of these definitions is that $\forall P. \quotep{P}
  \not\in \freenames{P}$.
\end{remark}

\subsection{ Dynamic quote: an example }

Anticipating something of what's to come, consider applying the
substitution, $\widehat{\id{\{}u / z \id{\}}}$, to the following pair
of processes, $\lift{w}{y!(z)}$ and $w[ \lpquote y!(z) \rpquote ]$.

\begin{eqnarray}
	\lift{w}{y!(z)}\widehat{\id{\{}u / z \id{\}}}
		& = &
		\lift{w}{y!(u)} \nonumber\\
	w[ \lpquote y!(z) \rpquote ] \widehat{ \id{\{}u / z \id{\}} }
		& = &
		w[ \lpquote y!(z) \rpquote ] \nonumber
\end{eqnarray}

Because the body of the process between quotes is impervious to
substitution, we get radically different answers. In fact, by
examining the first process in an input context,
e.g. $x?(z).\lift{w}{y!(z)}$, we see that the process under the lift
operator may be shaped by prefixed inputs binding a name inside it. In
this sense, the lift operator will be seen as a way to dynamically
construct processes before reifying them as names.

Finally equipped with these standard features we can present the
dynamics of the calculus.

\subsubsection{Operational semantics} 

Finally, we introduce the computational dynamics. What marks these
algebras as distinct from other more traditionally studied algebraic
structures, e.g. vector spaces or polynomial rings, is the manner in
which dynamics is captured. In traditional structures, dynamics is typically
expressed through morphisms between such structures, as in linear maps
between vector spaces or morphisms between rings. In algebras
associated with the semantics of computation, the dynamics is
expressed as part of the algebraic structure itself, through a
reduction reduction relation typically denoted by $\red$. Below, we
give a recursive presentation of this relation for the calculus used
in the encoding.

$\red \subseteq \pi \times \pi$
$\red : \pi \to \mathcal{P}(\pi)$

\begin{mathpar}
  \inferrule* [lab=Comm] { \textsf{match}( x_{src}, x_{trgt} ) } { x_{trgt}?(y)P \; | \; x_{src}!\langle {Q} \rangle \red P\{\quotep{Q}/y}\} }
  \and \\
  \inferrule* [lab=Par] {{P} \red {P}'} {{{P} | {Q}} \red {{P}' | {Q}}}
  \and
  \inferrule* [lab=Equiv]{{{P} \scong {P}'} \andalso {{P}' \red {Q}'} \andalso {{Q}' \scong {Q}}}{{P} \red {Q}}
\end{mathpar}

\begin{eqnarray*}
  match_{\equiv} (\quotep{P},\quotep{Q}) & := & P \equiv Q \\
  match_{\dagger}(\quotep{P},\quotep{Q}) & := & \forall R. P|Q \red^{*} R => R \red^{*} 0 \\
  match_{K}(\quotep{P},\quotep{Q}) & := & K \mbox{ for some context } K
\end{eqnarray*}

$u?(x)P | u!\langle Q \rangle \red P\{\quotep{Q}/x\}$

%We write $\wred$ for $\red^*$, and $P\red$ if $\exists Q $ such that $ P \red Q$.
We write $P\red$ if $\exists Q $ such that $ P \red Q$ and $P\not\red$, otherwise.

\section{Replication}

As mentioned before, it is known that replication (and hence
recursion) can be implemented in a higher-order process algebra
\cite{SangiorgiWalker}. As our first example of calculation with the
machinery thus far presented we give the construction explicitly in
the {\rhoc}.

\begin{eqnarray}
	D_{x} & := & \prefix{x}{y}{(\binpar{\outputp{x}{y}}{@{y}})} \nonumber\\
	\bangp_{x}{P} & := & \binpar{{x}!\langle{\binpar{D_{x}}{P}}\rangle}{D_{x}} \nonumber
\end{eqnarray}

\begin{eqnarray}
	\bangp_{x}{P} & & \nonumber\\
	=
	& {x}!\langle{(\prefix{x}{y}{(\outputp{x}{y} | @{y})) | P}}\rangle 
	      | \prefix{x}{y}{(\outputp{x}{y} | @{y})} & \nonumber\\
	\red
	& (\outputp{x}{y} | @{y})\substn{\quotep{(\prefix{x}{y}{(@{y} | \outputp{x}{y})) | P}}}{y} & \nonumber\\
	=
	& \outputp{x}{\quotep{(\prefix{x}{y}{(\outputp{x}{y} | @{y})) | P}}}
	  | {(\prefix{x}{y}{(\outputp{x}{y} | @{y})) | P}} & \nonumber\\
	\red
	& \ldots & \nonumber\\
	\red^*
	& P | P | \ldots & \nonumber
\end{eqnarray}

Of course, this encoding, as an implementation, runs away, unfolding
$\bangp{P}$ eagerly. A lazier and more implementable replication
operator, restricted to input-guarded processes, may be obtained as follows.

\begin{eqnarray}
\bangp{\prefix{u}{v}{P}} 
	:= 
	\binpar{\lift{x}{\prefix{u}{v}{(\binpar{D(x)}{P})}}}{D(x)} \nonumber
\end{eqnarray}

\begin{remark}
  Note that the lazier definition still does not deal with summation
  or mixed summation (i.e. sums over input and output). The reader is
  invited to construct definitions of replication that deal with these
  features. 

  Further, the definitions are parameterized in a name, $x$. Can you,
  gentle reader, make a definition that eliminates this parameter and
  guarantees no accidental interaction between the replication
  machinery and the process being replicated -- i.e. no accidental
  sharing of names used by the process to get its work done and the
  name(s) used by the replication to effect copying. This latter
  revision of the definition of replication is crucial to obtaining
  the expected identity $!!P \sim !P$.
\end{remark}

\begin{remark}\label{rem:paradoxical_combinator}
  The reader familiar with the lambda calculus will have noticed the
  similarity between $D$ and the paradoxical combinator.

  [Ed. note: the existence of this seems to suggest we have to be more
  restrictive on the set of processes and names we admit if we are to
  support no-cloning.]
\end{remark}

\subsubsection{Bisimulation}

The computational dynamics gives rise to another kind of equivalence,
the equivalence of computational behavior. As previously mentioned
this is typically captured \emph{via} some form of bisimulation.

% The notion we use in this paper is weak barbed bisimulation
% \cite{milner91polyadicpi}.

The notion we use in this paper is derived from weak barbed
bisimulation \cite{milner91polyadicpi}. 

\begin{definition}
An \emph{observation relation}, $\downarrow_{\mathcal N}$, over a set
of names, $\mathcal N$, is the smallest relation satisfying the rules
below.

\infrule[Out-barb]{y \in {\mathcal N}, \; x \nameeq y}
		  {\outputp{x}{v} \downarrow_{\mathcal N} x}
\infrule[Par-barb]{\mbox{$P\downarrow_{\mathcal N} x$ or $Q\downarrow_{\mathcal N} x$}}
		  {\binpar{P}{Q} \downarrow_{\mathcal N} x}

We write $P \Downarrow_{\mathcal N} x$ if there is $Q$ such that 
$P \wred Q$ and $Q \downarrow_{\mathcal N} x$.
\end{definition}

\begin{definition}
%\label{def.bbisim}
An  ${\mathcal N}$-\emph{barbed bisimulation} over a set of names, ${\mathcal N}$, is a symmetric binary relation 
${\mathcal S}_{\mathcal N}$ between agents such that $P\rel{S}_{\mathcal N}Q$ implies:
\begin{enumerate}
\item If $P \red P'$ then $Q \wred Q'$ and $P'\rel{S}_{\mathcal N} Q'$.
\item If $P\downarrow_{\mathcal N} x$, then $Q\Downarrow_{\mathcal N} x$.
\end{enumerate}
$P$ is ${\mathcal N}$-barbed bisimilar to $Q$, written
$P \wbbisim_{\mathcal N} Q$, if $P \rel{S}_{\mathcal N} Q$ for some ${\mathcal N}$-barbed bisimulation ${\mathcal S}_{\mathcal N}$.
\end{definition}

$\mathcal{R} \subseteq \pi \times \pi$

$P \mathcal{R} Q => \forall P'. P \red P' \Rightarrow \exists Q'. Q \red Q', P' \mathcal{R} Q'$

$P \vdash x \Rightarrow Q \vdash x$

\begin{mathpar}
  \inferrule*[lab=Out-barb]{x \nameeq y}{{y}!\langle{Q}\rangle \vdash x}
  \and
  \inferrule*[lab=Par-barb]{\mbox{$P\vdash x$ or $Q\vdash x$}}{\binpar{P}{Q} \vdash x}
\end{mathpar}

\subsubsection{Contexts}

One of the principle advantages of computational calculi like the
$\pi$-calculus is a well-defined notion of context,
contextual-equivalence and a correlation between
contextual-equivalence and notions of bisimulation. The notion of
context allows the decomposition of a process into (sub-)process and
its syntactic environment, its context. Thus, a context may be
thought of as a process with a ``hole'' (written $\Box$) in it. The
application of a context $M$ to a process $P$, written $M[P]$, is
tantamount to filling the hole in $M$ with $P$. In this paper we do
not need the full weight of this theory, but do make use of the notion
of context in the proof the main theorem. 

\begin{mathpar}
  \inferrule* [lab=summation] {} {{M_{M},M_{N}} \bc \Box \;|\; x.M_{A} \;|\; M_{M}+M_{N}}
  \and
  \inferrule* [lab=agent] {} {{M_{A}} \bc (\vec{x})M_{P} \;| \; \clift{P_0,\ldots,M_{P},\ldots,P_N}}
  \and \\
  \inferrule* [lab=process] {} {{M_{P}} \bc M_{N} \;| \;P|M_{P} }
\end{mathpar} 

\begin{mathpar}
  \inferrule* [lab=sychronization] {} {M_{N} \bc \Box \;|\; x?M_{F} \;|\; x!M_{C}}
  \and
  \inferrule* [lab=abstraction] {} {{M_{F}} \bc (x)M_{P} }
  \and
  \inferrule* [lab=concretion] {} {{M_{C}} \bc \langle M_{P} \rangle }
  \and \\
  \inferrule* [lab=process] {} {{M_{P}} \bc M_{N} \;| \;P|M_{P} }
\end{mathpar}

\begin{definition}[contextual application] Given a context $M$, and
  process $P$, we define the \emph{contextual application}, $M[P] :=
  M\{P/\Box\}$. That is, the contextual application of M to P is the
  substitution of $P$ for $\Box$ in $M$.
\end{definition}

$\meaningof{-} : L \to \mathcal{P}(\pi)$

\begin{mathpar}
  \inferrule* [lab=collection] {} {\meaningof{true} = \pi, \and \meaningof{~E} = \pi \setminus \meaningof{E}, \and \meaningof{E_{1} \& E_{2}} = \meaningof{E_{1}} \cap \meaningof{E_{2}}}
\end{mathpar}

\begin{mathpar}
  \inferrule* [lab=structure] {} {\meaningof{0} = \{ P \in \pi | P \equiv 0 \}, \and \\ \meaningof{E_1 | E_2} = \{ P \in \pi | P \equiv P_{1} | P_{2}, P_{1} \in \meaningof{E_{1}}, P_{2} \in \meaningof{E_2}\} }
\end{mathpar}

\begin{mathpar}
 \inferrule* [lab=behavior] {} {\meaningof{\langle a?b \rangle E} = \{ P \in \pi | P \equiv Q | u?(y)P', \\ \and \\\\ \and \\ \;\;\; u \in \meaningof{a}, \forall z.P'\{z/y\} \in \meaningof{E\{z/b\}}\}, \and \\ \meaningof{a!E} = \{ P \in \pi | P \equiv Q | x!\langle P' \rangle, x \in \meaningof{a} P' \in \meaningof{E}\} }
\end{mathpar}

\begin{mathpar}
 \inferrule* [lab=nominal] {} {\meaningof{\quotep{E}} = \{ \quotep{P} \in \quotep{\pi} | P \in \meaningof{E} \}, \and \meaningof{\quotep{P}} = \{ \quotep{Q} \in \quotep{\pi} | P \equiv Q \} \and \\ \meaningof{@\quotep{E}} = \{ P \in \pi | P \equiv @x, x \in \meaningof{E} \}}
\end{mathpar}

\begin{eqnarray*}
  \\
  \meaningof{-} : TS \to ST
\end{eqnarray*}

\begin{eqnarray*}
  \\
  L : TS \to ST
\end{eqnarray*}

\begin{eqnarray*}
  \\
  P \models E \iff P \in \meaningof{E}
\end{eqnarray*}

\begin{eqnarray*}
  P \approx_{L} Q \iff \forall E \in L. P \models E \iff Q \models E
\end{eqnarray*}

\begin{eqnarray*}
  P \approx_{K} Q
\end{eqnarray*}

\begin{eqnarray*}
  P \approx Q
\end{eqnarray*}

$\approx_{K} = \approx = \approx_{L}$

\subsubsection{Contextual duality}

Note that contexts extend the quotation operation to a family of
operations from processes to names. Given a context, $M$, we can
define a \emph{nominal context}, $\quotep{M}$ by $\quotep{M}[P] :=
\quotep{M[P]}$. To foreshadow what is to come we observe that these
operations enjoy a duality with processes very much like the duality
between vectors and maps from vectors to scalars.

Further, because the calculus is essentially higher-order, we have a
correspondence between contexts and processes. More specifically,
given a name $x$ and a context $M$ we can construct $M^{*}_{x}$ such
that 

\begin{mathpar}
  M^{*}_{x} | \lift{x}{P} \red M[P]
\end{mathpar}

namely,

\begin{mathpar}
  M^{*}_{x} := x?(u).M[\dropn{u}]
\end{mathpar}

The dependence of $M^{*}_{x}$ on a name makes it an abstraction, 

\begin{mathpar}
  M^{*} := (x)x?(u).M[\dropn{u}]
\end{mathpar}

\subsection{Additional notation}

It will sometimes be convenient to denote the process a name
quotes. We already have the notation $x = \quotep{P}$, but it will be
convenient to introduce an alternate notation, $\procn{x}$, when we
want to emphasize the connection to the use of the name. Note that, by
virtue of name equivalence, $\quotep{\procn{x}} \nameeq x$; so, the
notation is consistent with previous definitions.

Further, because names have structure it is possible to effect
substitutions on the basis of that structure. This means we need to
upgrade our notation for substitutions, which we accomplish by
adapting comprehension notation. Thus,

\begin{mathpar}
  P\{ y / x : x \in S \}
\end{mathpar}

is interpreted to mean the process derived from P by replacing (in a
capture-avoiding manner) each occurrence of $x$ in $S$ by $y$. For example,

\begin{mathpar}
  P\{ \quotep{\procn{x}|\procn{x}} / x : x \in \freenames{P} \}
\end{mathpar}

will replace each (occurrence) of a free name $x$ in $P$ by
$\quotep{\procn{x}|\procn{x}}$.

Also, we will avail ourselves of the notation $x^{L}$ and $x^{R}$ to
denote injections of a name into disjoint copies of the name
space. There are numerous ways to accomplish this. One example can be
found in \cite{MeredithR05}. This notation overloads to vectors of
names: $\vec{x}^{\pi} := (x_{i}^{\pi} \; : \; 0 \leq i < |\vec{x}| )$ where $\pi \in \{L,R\}$.

We also use $P^{\Box} := P|\Box$.

In \cite{MeredithR05} an interpretation of the new operator is
given. It turns out that there are several possible interpretations
all enjoying the requisite algebraic properties of the operator (see
\cite{milner91polyadicpi}). We will therefore make liberal use of
$(\nu\; \vec{x})P$.

% subsection the_syntax_and_semantics_of_the_notation_system (end)   

\input{qm2pi.qmops} 

\input{qm2pi.sterngerlach} 

\input{qm2pi.metric} 

% section concurrent_process_calculi (end)

%\input{qm2pi.proofsketch}

% section proof sketch (end)

%\input{qm2pi.slviaknots} 

% section spatial logic via knots (end)

\input{qm2pi.conclusion}

% section conclusion (end)

%\input{qm2pi.dtcodes} 

% section wiring algorithm (end)

\input{qm2pi.ack} 

% section acknowledgments (end)

\newpage


\bibliographystyle{plain}   
\bibliography{../../biblios/main.bib}

\input{qm2pi.rhodetails}

\end{document}

 

% section notation (end)

\input{qm2pi.process.calculi} 

% section concurrent_process_calculi_and_spatial_logics_ (end)
    
%\documentclass[12pt]{llncs}
%\documentclass{jktr}

\usepackage[pdftex]{hyperref}                   
\usepackage {listings}
\usepackage {mathpartir}
\usepackage{bcprules}
%\usepackage{listings}
                       
\usepackage{graphicx} 
%\usepackage[margins=2.5cm,nohead,nofoot]{geometry}
%\usepackage{geometry}
\usepackage{amsfonts}
\usepackage{amstext}
\usepackage{latexsym}
\usepackage{amssymb}
\usepackage{color}


%\include{myPreamble}
\include{qm2pi.local} 

%\ifpdf
%\usepackage[pdftex]{graphicx}
%\else
%\usepackage{graphicx}
%\fi

 % \ifpdf
%  \usepackage{pdfsync}
%  \if


%\title{Brief Article}
%\author{David F. Snyder}
%\author{L.G. Meredith}

%\address{Dept. of Math., Texas State University--San Marcos, San Marcos, TX 78666}
       
\pagestyle{empty}


\begin{document}

\lstset{language=[Objective]Caml,frame=shadowbox}

\input{qm2pi.front}

% section front matter (end)

\input{qm2pi.intro} 
 
% section introduction (end)

% \input{qm2pi.knotations} 

% section notation (end)

\input{qm2pi.process.calculi} 

% section concurrent_process_calculi_and_spatial_logics_ (end)
    
%\input{qm2pi.knots2pi} 

%\input{qm2pi.trefoil} 

%\input{qm2pi.mainthm} 

% subsection basic_interpretation (end)

%\input{qm2pi.rho.presentation} 
\subsection{The syntax and semantics of the notation system}\label{sub:the_syntax_and_semantics_of_the_notation_system} % (fold)

We now summarize a technical presentation of the calculus that
embodies our theory of dynamics. The typical presentation of such a
calculus follows the style of giving generators and relations on
them. The grammar, below, describing term constructors, freely
generates the set of processes, $\Proc$. This set is then quotiented
by a relation known as structural congruence and it is over this set
that the notion of dynamics is expressed. This presentation is
essentially that of \cite{MeredithR05} with the addition of
polyadicity and summation. For readability we have relegated some of
the technical subtleties to an appendix.

\subsubsection{Process grammar}\label{subsub:process_grammar}

\begin{mathpar}
  \inferrule* [lab=synchronization] {} {{M} \bc \pzero \;|\; x?F \;|\; x!C }
  \and
  \inferrule* [lab=abstraction] {} {{F} \bc (x)P}
  \and
  \inferrule* [lab=concretion] {} {{C} \bc \langle Q \rangle}
  \and
  \inferrule* [lab=process] {} {{P,Q} \bc M \;| \;P|Q \;|\; @{x}}
  \and
  \inferrule* [lab=name] {} {{x} \bc \quotep{P}}
\end{mathpar} 

Note that $\vec{x}$ (resp. $\vec{P}$) denotes a vector of names
(resp. processes) of length $|\vec{x}|$ (resp. $|\vec{P}|$). We adopt
the following useful abbreviations.

\begin{mathpar}
   x?(\vec{y}).P := x.(\vec{y})P \and  x\clift{\vec{P}} := x.\clift{\vec{P}}
   \and x!(y) := \lift{x}{\dropn{y}}
   \and \Pi_{i=0}^{n-1}P_i := P_0 | \ldots | P_{n-1}
\end{mathpar}

\subsubsection{Structural congruence}

\paragraph{Free and bound names and alpha-equivalence.} At the
core of structural equivalence is alpha-equivalence which identifies
process that are the same up to a change of variable. Formally, we
recognize the distinction between free and bound names. The free names
of a process, $\freenames{P}$, may be calculated recursively as
follows:

\begin{mathpar}
\freenames{\pzero} := \emptyset
  \and \\
  \freenames{x?(y).P} := \{ x \} \cup (\freenames{P} \setminus \{ y \})
  \and 
  \freenames{x!\langle P \rangle} := \{ x \} \cup \{ P \} 
  \and \\
  \freenames{P|Q} := \freenames{P} \cup \freenames{Q}
  \and \\
  \freenames{@{x}} := \{ x \}
\end{mathpar}

$\pi$
$\quotep{\pi}$

$\freenames{-} : \pi \to \mathcal{P}(\quotep{\pi})$

\begin{eqnarray*}
  \freenames{\pzero} & := & \emptyset \\
  \freenames{x?(y).P} & := & \{ x \} \cup (\freenames{P} \setminus \{ y \}) \\
  \freenames{x!\langle P \rangle} & := & \{ x \} \cup \{ P \} \\
  \freenames{P|Q} & := & \freenames{P} \cup \freenames{Q} \\
  \freenames{\dropn{x}} & := & \{ x \}
\end{eqnarray*}

The bound names of a process, $\boundnames{P}$, are those names occurring in $P$
that are not free. For example, in $x?(y).0$, the name $x$ is free, while $y$ is bound.

\begin{mathpar}
  \inferrule* [lab=monoidal-laws] {} { P|Q \equiv Q|P \and P|0 \equiv P \and P|(Q|R) \equiv (P|Q)|R }
\end{mathpar}

\begin{mathpar}
  \inferrule* [lab=alpha-equivalence] {} { (x)P \equiv (y)P\{y/x\} \and y \not\in \freenames{P} }
\end{mathpar}

\begin{definition}
Then two processes, $P,Q$, are alpha-equivalent if $P = Q\{\vec{y}/\vec{x}\}$ for
some $\vec{x} \in \boundnames{Q},\vec{y} \in \boundnames{P}$, where $Q\{\vec{y}/\vec{x}\}$
denotes the capture-avoiding substitution of $\vec{y}$ for $\vec{x}$ in $Q$.
\end{definition}

\begin{definition}
  The {\em structural congruence} \cite{SangiorgiWalker} , $\equiv$,
  between processes is the least congruence containing
  alpha-equivalence, satisfying the abelian monoid laws
  (associativity, commutativity and $\pzero$ as identity) for parallel
  composition $|$ and for summation $+$.
\end{definition}

\subsection{Name equivalence}

We take name equivalence, written $\nameeq$, to be the smallest
equivalence relation generated by the following rules.

\begin{mathpar}
\inferrule*[lab=Quote-drop]
{ }
{ \quotep{@{x}} \nameeq x }

\inferrule*[lab=Struct-equiv]
{ P \scong Q }
{ \quotep{P} \nameeq \quotep{Q} }
\end{mathpar}

The astute reader will have noticed that the mutual recursion of names
and processes imposes a mutual recursion on alpha-equivalence and
structural equivalence via name-equivalence. Fortunately, all of this
works out pleasantly and we may calculate in the natural way, free of
concern. The reader interested in the details is referred to the
appendix \ref{appendix:rho_details}.

\subsection{Substitution}

We use $\Proc$ for the set of processes, $\QProc$ for the set of
names, and $\id{\{}\vec{y} / \vec{x} \id{\}}$ to denote partial maps,
$s : \QProc \rightarrow \QProc$. A map, $s$ lifts, uniquely, to a map
on process terms, $\widehat{s} : \Proc \rightarrow \Proc$ by the
following equations.

\begin{mathpar}
  (0) \psubstp{Q}{P} := 0 \\
  (R \juxtap S) \psubstp{Q}{P}
  :=    
  (R)\psubstp{Q}{P} \juxtap (S) \psubstp{Q}{P} \\
  (x?(y).R) \psubstp{Q}{P}    
  :=    
  (x)\substp{Q}{P} (z)\concat( (R \psubstn{z}{y}) \psubstp{Q}{P} ) \\
  (\lift{x}{R}) \psubstp{Q}{P}  
  :=
  \lift{(x)\substp{Q}{P}}{ R \psubstp{Q}{P} } \\
%   (\dropn{x})  \psubstp{Q}{P}       
%   := 
%   \left\{ 
%     \begin{array}{ccc} 
%       \dropn{\quotep{Q}} & & x \nameeq \quotep{P} \\
%       \dropn{x} & & otherwise \\
%     \end{array}
%   \right. 
  (\dropn{x})  \psubstp{Q}{P}       
  := 
  \left\{ 
    \begin{array}{ccc} 
      Q & & x \nameeq \quotep{P} \\
      \dropn{x} & & otherwise \\
    \end{array}
  \right.
\end{mathpar}
 

where

\begin{eqnarray}
  (x)\id{\{} \lpquote Q \rpquote / \lpquote P \rpquote \id{\}}            = 
  \left\{ 
    \begin{array}{ccc}
      \lpquote Q \rpquote & & x \nameeq \lpquote P \rpquote \\
      x & & otherwise \\
    \end{array}
  \right. \nonumber
\end{eqnarray}

and $z$ is chosen distinct from $\quotep{P}$, $\quotep{Q}$, the free
names in $Q$, and all the names in $R$. Our $\alpha$-equivalence will
be built in the standard way from this substitution.

\begin{remark}\label{rem:no_self_referential_names}
  One consequence of these definitions is that $\forall P. \quotep{P}
  \not\in \freenames{P}$.
\end{remark}

\subsection{ Dynamic quote: an example }

Anticipating something of what's to come, consider applying the
substitution, $\widehat{\id{\{}u / z \id{\}}}$, to the following pair
of processes, $\lift{w}{y!(z)}$ and $w[ \lpquote y!(z) \rpquote ]$.

\begin{eqnarray}
	\lift{w}{y!(z)}\widehat{\id{\{}u / z \id{\}}}
		& = &
		\lift{w}{y!(u)} \nonumber\\
	w[ \lpquote y!(z) \rpquote ] \widehat{ \id{\{}u / z \id{\}} }
		& = &
		w[ \lpquote y!(z) \rpquote ] \nonumber
\end{eqnarray}

Because the body of the process between quotes is impervious to
substitution, we get radically different answers. In fact, by
examining the first process in an input context,
e.g. $x?(z).\lift{w}{y!(z)}$, we see that the process under the lift
operator may be shaped by prefixed inputs binding a name inside it. In
this sense, the lift operator will be seen as a way to dynamically
construct processes before reifying them as names.

Finally equipped with these standard features we can present the
dynamics of the calculus.

\subsubsection{Operational semantics} 

Finally, we introduce the computational dynamics. What marks these
algebras as distinct from other more traditionally studied algebraic
structures, e.g. vector spaces or polynomial rings, is the manner in
which dynamics is captured. In traditional structures, dynamics is typically
expressed through morphisms between such structures, as in linear maps
between vector spaces or morphisms between rings. In algebras
associated with the semantics of computation, the dynamics is
expressed as part of the algebraic structure itself, through a
reduction reduction relation typically denoted by $\red$. Below, we
give a recursive presentation of this relation for the calculus used
in the encoding.

$\red \subseteq \pi \times \pi$
$\red : \pi \to \mathcal{P}(\pi)$

\begin{mathpar}
  \inferrule* [lab=Comm] { \textsf{match}( x_{src}, x_{trgt} ) } { x_{trgt}?(y)P \; | \; x_{src}!\langle {Q} \rangle \red P\{\quotep{Q}/y}\} }
  \and \\
  \inferrule* [lab=Par] {{P} \red {P}'} {{{P} | {Q}} \red {{P}' | {Q}}}
  \and
  \inferrule* [lab=Equiv]{{{P} \scong {P}'} \andalso {{P}' \red {Q}'} \andalso {{Q}' \scong {Q}}}{{P} \red {Q}}
\end{mathpar}

\begin{eqnarray*}
  match_{\equiv} (\quotep{P},\quotep{Q}) & := & P \equiv Q \\
  match_{\dagger}(\quotep{P},\quotep{Q}) & := & \forall R. P|Q \red^{*} R => R \red^{*} 0 \\
  match_{K}(\quotep{P},\quotep{Q}) & := & K \mbox{ for some context } K
\end{eqnarray*}

$u?(x)P | u!\langle Q \rangle \red P\{\quotep{Q}/x\}$

%We write $\wred$ for $\red^*$, and $P\red$ if $\exists Q $ such that $ P \red Q$.
We write $P\red$ if $\exists Q $ such that $ P \red Q$ and $P\not\red$, otherwise.

\section{Replication}

As mentioned before, it is known that replication (and hence
recursion) can be implemented in a higher-order process algebra
\cite{SangiorgiWalker}. As our first example of calculation with the
machinery thus far presented we give the construction explicitly in
the {\rhoc}.

\begin{eqnarray}
	D_{x} & := & \prefix{x}{y}{(\binpar{\outputp{x}{y}}{@{y}})} \nonumber\\
	\bangp_{x}{P} & := & \binpar{{x}!\langle{\binpar{D_{x}}{P}}\rangle}{D_{x}} \nonumber
\end{eqnarray}

\begin{eqnarray}
	\bangp_{x}{P} & & \nonumber\\
	=
	& {x}!\langle{(\prefix{x}{y}{(\outputp{x}{y} | @{y})) | P}}\rangle 
	      | \prefix{x}{y}{(\outputp{x}{y} | @{y})} & \nonumber\\
	\red
	& (\outputp{x}{y} | @{y})\substn{\quotep{(\prefix{x}{y}{(@{y} | \outputp{x}{y})) | P}}}{y} & \nonumber\\
	=
	& \outputp{x}{\quotep{(\prefix{x}{y}{(\outputp{x}{y} | @{y})) | P}}}
	  | {(\prefix{x}{y}{(\outputp{x}{y} | @{y})) | P}} & \nonumber\\
	\red
	& \ldots & \nonumber\\
	\red^*
	& P | P | \ldots & \nonumber
\end{eqnarray}

Of course, this encoding, as an implementation, runs away, unfolding
$\bangp{P}$ eagerly. A lazier and more implementable replication
operator, restricted to input-guarded processes, may be obtained as follows.

\begin{eqnarray}
\bangp{\prefix{u}{v}{P}} 
	:= 
	\binpar{\lift{x}{\prefix{u}{v}{(\binpar{D(x)}{P})}}}{D(x)} \nonumber
\end{eqnarray}

\begin{remark}
  Note that the lazier definition still does not deal with summation
  or mixed summation (i.e. sums over input and output). The reader is
  invited to construct definitions of replication that deal with these
  features. 

  Further, the definitions are parameterized in a name, $x$. Can you,
  gentle reader, make a definition that eliminates this parameter and
  guarantees no accidental interaction between the replication
  machinery and the process being replicated -- i.e. no accidental
  sharing of names used by the process to get its work done and the
  name(s) used by the replication to effect copying. This latter
  revision of the definition of replication is crucial to obtaining
  the expected identity $!!P \sim !P$.
\end{remark}

\begin{remark}\label{rem:paradoxical_combinator}
  The reader familiar with the lambda calculus will have noticed the
  similarity between $D$ and the paradoxical combinator.

  [Ed. note: the existence of this seems to suggest we have to be more
  restrictive on the set of processes and names we admit if we are to
  support no-cloning.]
\end{remark}

\subsubsection{Bisimulation}

The computational dynamics gives rise to another kind of equivalence,
the equivalence of computational behavior. As previously mentioned
this is typically captured \emph{via} some form of bisimulation.

% The notion we use in this paper is weak barbed bisimulation
% \cite{milner91polyadicpi}.

The notion we use in this paper is derived from weak barbed
bisimulation \cite{milner91polyadicpi}. 

\begin{definition}
An \emph{observation relation}, $\downarrow_{\mathcal N}$, over a set
of names, $\mathcal N$, is the smallest relation satisfying the rules
below.

\infrule[Out-barb]{y \in {\mathcal N}, \; x \nameeq y}
		  {\outputp{x}{v} \downarrow_{\mathcal N} x}
\infrule[Par-barb]{\mbox{$P\downarrow_{\mathcal N} x$ or $Q\downarrow_{\mathcal N} x$}}
		  {\binpar{P}{Q} \downarrow_{\mathcal N} x}

We write $P \Downarrow_{\mathcal N} x$ if there is $Q$ such that 
$P \wred Q$ and $Q \downarrow_{\mathcal N} x$.
\end{definition}

\begin{definition}
%\label{def.bbisim}
An  ${\mathcal N}$-\emph{barbed bisimulation} over a set of names, ${\mathcal N}$, is a symmetric binary relation 
${\mathcal S}_{\mathcal N}$ between agents such that $P\rel{S}_{\mathcal N}Q$ implies:
\begin{enumerate}
\item If $P \red P'$ then $Q \wred Q'$ and $P'\rel{S}_{\mathcal N} Q'$.
\item If $P\downarrow_{\mathcal N} x$, then $Q\Downarrow_{\mathcal N} x$.
\end{enumerate}
$P$ is ${\mathcal N}$-barbed bisimilar to $Q$, written
$P \wbbisim_{\mathcal N} Q$, if $P \rel{S}_{\mathcal N} Q$ for some ${\mathcal N}$-barbed bisimulation ${\mathcal S}_{\mathcal N}$.
\end{definition}

$\mathcal{R} \subseteq \pi \times \pi$

$P \mathcal{R} Q => \forall P'. P \red P' \Rightarrow \exists Q'. Q \red Q', P' \mathcal{R} Q'$

$P \vdash x \Rightarrow Q \vdash x$

\begin{mathpar}
  \inferrule*[lab=Out-barb]{x \nameeq y}{{y}!\langle{Q}\rangle \vdash x}
  \and
  \inferrule*[lab=Par-barb]{\mbox{$P\vdash x$ or $Q\vdash x$}}{\binpar{P}{Q} \vdash x}
\end{mathpar}

\subsubsection{Contexts}

One of the principle advantages of computational calculi like the
$\pi$-calculus is a well-defined notion of context,
contextual-equivalence and a correlation between
contextual-equivalence and notions of bisimulation. The notion of
context allows the decomposition of a process into (sub-)process and
its syntactic environment, its context. Thus, a context may be
thought of as a process with a ``hole'' (written $\Box$) in it. The
application of a context $M$ to a process $P$, written $M[P]$, is
tantamount to filling the hole in $M$ with $P$. In this paper we do
not need the full weight of this theory, but do make use of the notion
of context in the proof the main theorem. 

\begin{mathpar}
  \inferrule* [lab=summation] {} {{M_{M},M_{N}} \bc \Box \;|\; x.M_{A} \;|\; M_{M}+M_{N}}
  \and
  \inferrule* [lab=agent] {} {{M_{A}} \bc (\vec{x})M_{P} \;| \; \clift{P_0,\ldots,M_{P},\ldots,P_N}}
  \and \\
  \inferrule* [lab=process] {} {{M_{P}} \bc M_{N} \;| \;P|M_{P} }
\end{mathpar} 

\begin{mathpar}
  \inferrule* [lab=sychronization] {} {M_{N} \bc \Box \;|\; x?M_{F} \;|\; x!M_{C}}
  \and
  \inferrule* [lab=abstraction] {} {{M_{F}} \bc (x)M_{P} }
  \and
  \inferrule* [lab=concretion] {} {{M_{C}} \bc \langle M_{P} \rangle }
  \and \\
  \inferrule* [lab=process] {} {{M_{P}} \bc M_{N} \;| \;P|M_{P} }
\end{mathpar}

\begin{definition}[contextual application] Given a context $M$, and
  process $P$, we define the \emph{contextual application}, $M[P] :=
  M\{P/\Box\}$. That is, the contextual application of M to P is the
  substitution of $P$ for $\Box$ in $M$.
\end{definition}

$\meaningof{-} : L \to \mathcal{P}(\pi)$

\begin{mathpar}
  \inferrule* [lab=collection] {} {\meaningof{true} = \pi, \and \meaningof{~E} = \pi \setminus \meaningof{E}, \and \meaningof{E_{1} \& E_{2}} = \meaningof{E_{1}} \cap \meaningof{E_{2}}}
\end{mathpar}

\begin{mathpar}
  \inferrule* [lab=structure] {} {\meaningof{0} = \{ P \in \pi | P \equiv 0 \}, \and \\ \meaningof{E_1 | E_2} = \{ P \in \pi | P \equiv P_{1} | P_{2}, P_{1} \in \meaningof{E_{1}}, P_{2} \in \meaningof{E_2}\} }
\end{mathpar}

\begin{mathpar}
 \inferrule* [lab=behavior] {} {\meaningof{\langle a?b \rangle E} = \{ P \in \pi | P \equiv Q | u?(y)P', \\ \and \\\\ \and \\ \;\;\; u \in \meaningof{a}, \forall z.P'\{z/y\} \in \meaningof{E\{z/b\}}\}, \and \\ \meaningof{a!E} = \{ P \in \pi | P \equiv Q | x!\langle P' \rangle, x \in \meaningof{a} P' \in \meaningof{E}\} }
\end{mathpar}

\begin{mathpar}
 \inferrule* [lab=nominal] {} {\meaningof{\quotep{E}} = \{ \quotep{P} \in \quotep{\pi} | P \in \meaningof{E} \}, \and \meaningof{\quotep{P}} = \{ \quotep{Q} \in \quotep{\pi} | P \equiv Q \} \and \\ \meaningof{@\quotep{E}} = \{ P \in \pi | P \equiv @x, x \in \meaningof{E} \}}
\end{mathpar}

\begin{eqnarray*}
  \\
  \meaningof{-} : TS \to ST
\end{eqnarray*}

\begin{eqnarray*}
  \\
  L : TS \to ST
\end{eqnarray*}

\begin{eqnarray*}
  \\
  P \models E \iff P \in \meaningof{E}
\end{eqnarray*}

\begin{eqnarray*}
  P \approx_{L} Q \iff \forall E \in L. P \models E \iff Q \models E
\end{eqnarray*}

\begin{eqnarray*}
  P \approx_{K} Q
\end{eqnarray*}

\begin{eqnarray*}
  P \approx Q
\end{eqnarray*}

$\approx_{K} = \approx = \approx_{L}$

\subsubsection{Contextual duality}

Note that contexts extend the quotation operation to a family of
operations from processes to names. Given a context, $M$, we can
define a \emph{nominal context}, $\quotep{M}$ by $\quotep{M}[P] :=
\quotep{M[P]}$. To foreshadow what is to come we observe that these
operations enjoy a duality with processes very much like the duality
between vectors and maps from vectors to scalars.

Further, because the calculus is essentially higher-order, we have a
correspondence between contexts and processes. More specifically,
given a name $x$ and a context $M$ we can construct $M^{*}_{x}$ such
that 

\begin{mathpar}
  M^{*}_{x} | \lift{x}{P} \red M[P]
\end{mathpar}

namely,

\begin{mathpar}
  M^{*}_{x} := x?(u).M[\dropn{u}]
\end{mathpar}

The dependence of $M^{*}_{x}$ on a name makes it an abstraction, 

\begin{mathpar}
  M^{*} := (x)x?(u).M[\dropn{u}]
\end{mathpar}

\subsection{Additional notation}

It will sometimes be convenient to denote the process a name
quotes. We already have the notation $x = \quotep{P}$, but it will be
convenient to introduce an alternate notation, $\procn{x}$, when we
want to emphasize the connection to the use of the name. Note that, by
virtue of name equivalence, $\quotep{\procn{x}} \nameeq x$; so, the
notation is consistent with previous definitions.

Further, because names have structure it is possible to effect
substitutions on the basis of that structure. This means we need to
upgrade our notation for substitutions, which we accomplish by
adapting comprehension notation. Thus,

\begin{mathpar}
  P\{ y / x : x \in S \}
\end{mathpar}

is interpreted to mean the process derived from P by replacing (in a
capture-avoiding manner) each occurrence of $x$ in $S$ by $y$. For example,

\begin{mathpar}
  P\{ \quotep{\procn{x}|\procn{x}} / x : x \in \freenames{P} \}
\end{mathpar}

will replace each (occurrence) of a free name $x$ in $P$ by
$\quotep{\procn{x}|\procn{x}}$.

Also, we will avail ourselves of the notation $x^{L}$ and $x^{R}$ to
denote injections of a name into disjoint copies of the name
space. There are numerous ways to accomplish this. One example can be
found in \cite{MeredithR05}. This notation overloads to vectors of
names: $\vec{x}^{\pi} := (x_{i}^{\pi} \; : \; 0 \leq i < |\vec{x}| )$ where $\pi \in \{L,R\}$.

We also use $P^{\Box} := P|\Box$.

In \cite{MeredithR05} an interpretation of the new operator is
given. It turns out that there are several possible interpretations
all enjoying the requisite algebraic properties of the operator (see
\cite{milner91polyadicpi}). We will therefore make liberal use of
$(\nu\; \vec{x})P$.

% subsection the_syntax_and_semantics_of_the_notation_system (end)   

\input{qm2pi.qmops} 

\input{qm2pi.sterngerlach} 

\input{qm2pi.metric} 

% section concurrent_process_calculi (end)

%\input{qm2pi.proofsketch}

% section proof sketch (end)

%\input{qm2pi.slviaknots} 

% section spatial logic via knots (end)

\input{qm2pi.conclusion}

% section conclusion (end)

%\input{qm2pi.dtcodes} 

% section wiring algorithm (end)

\input{qm2pi.ack} 

% section acknowledgments (end)

\newpage


\bibliographystyle{plain}   
\bibliography{../../biblios/main.bib}

\input{qm2pi.rhodetails}

\end{document}

 

%\documentclass[12pt]{llncs}
%\documentclass{jktr}

\usepackage[pdftex]{hyperref}                   
\usepackage {listings}
\usepackage {mathpartir}
\usepackage{bcprules}
%\usepackage{listings}
                       
\usepackage{graphicx} 
%\usepackage[margins=2.5cm,nohead,nofoot]{geometry}
%\usepackage{geometry}
\usepackage{amsfonts}
\usepackage{amstext}
\usepackage{latexsym}
\usepackage{amssymb}
\usepackage{color}


%\include{myPreamble}
\include{qm2pi.local} 

%\ifpdf
%\usepackage[pdftex]{graphicx}
%\else
%\usepackage{graphicx}
%\fi

 % \ifpdf
%  \usepackage{pdfsync}
%  \if


%\title{Brief Article}
%\author{David F. Snyder}
%\author{L.G. Meredith}

%\address{Dept. of Math., Texas State University--San Marcos, San Marcos, TX 78666}
       
\pagestyle{empty}


\begin{document}

\lstset{language=[Objective]Caml,frame=shadowbox}

\input{qm2pi.front}

% section front matter (end)

\input{qm2pi.intro} 
 
% section introduction (end)

% \input{qm2pi.knotations} 

% section notation (end)

\input{qm2pi.process.calculi} 

% section concurrent_process_calculi_and_spatial_logics_ (end)
    
%\input{qm2pi.knots2pi} 

%\input{qm2pi.trefoil} 

%\input{qm2pi.mainthm} 

% subsection basic_interpretation (end)

%\input{qm2pi.rho.presentation} 
\subsection{The syntax and semantics of the notation system}\label{sub:the_syntax_and_semantics_of_the_notation_system} % (fold)

We now summarize a technical presentation of the calculus that
embodies our theory of dynamics. The typical presentation of such a
calculus follows the style of giving generators and relations on
them. The grammar, below, describing term constructors, freely
generates the set of processes, $\Proc$. This set is then quotiented
by a relation known as structural congruence and it is over this set
that the notion of dynamics is expressed. This presentation is
essentially that of \cite{MeredithR05} with the addition of
polyadicity and summation. For readability we have relegated some of
the technical subtleties to an appendix.

\subsubsection{Process grammar}\label{subsub:process_grammar}

\begin{mathpar}
  \inferrule* [lab=synchronization] {} {{M} \bc \pzero \;|\; x?F \;|\; x!C }
  \and
  \inferrule* [lab=abstraction] {} {{F} \bc (x)P}
  \and
  \inferrule* [lab=concretion] {} {{C} \bc \langle Q \rangle}
  \and
  \inferrule* [lab=process] {} {{P,Q} \bc M \;| \;P|Q \;|\; @{x}}
  \and
  \inferrule* [lab=name] {} {{x} \bc \quotep{P}}
\end{mathpar} 

Note that $\vec{x}$ (resp. $\vec{P}$) denotes a vector of names
(resp. processes) of length $|\vec{x}|$ (resp. $|\vec{P}|$). We adopt
the following useful abbreviations.

\begin{mathpar}
   x?(\vec{y}).P := x.(\vec{y})P \and  x\clift{\vec{P}} := x.\clift{\vec{P}}
   \and x!(y) := \lift{x}{\dropn{y}}
   \and \Pi_{i=0}^{n-1}P_i := P_0 | \ldots | P_{n-1}
\end{mathpar}

\subsubsection{Structural congruence}

\paragraph{Free and bound names and alpha-equivalence.} At the
core of structural equivalence is alpha-equivalence which identifies
process that are the same up to a change of variable. Formally, we
recognize the distinction between free and bound names. The free names
of a process, $\freenames{P}$, may be calculated recursively as
follows:

\begin{mathpar}
\freenames{\pzero} := \emptyset
  \and \\
  \freenames{x?(y).P} := \{ x \} \cup (\freenames{P} \setminus \{ y \})
  \and 
  \freenames{x!\langle P \rangle} := \{ x \} \cup \{ P \} 
  \and \\
  \freenames{P|Q} := \freenames{P} \cup \freenames{Q}
  \and \\
  \freenames{@{x}} := \{ x \}
\end{mathpar}

$\pi$
$\quotep{\pi}$

$\freenames{-} : \pi \to \mathcal{P}(\quotep{\pi})$

\begin{eqnarray*}
  \freenames{\pzero} & := & \emptyset \\
  \freenames{x?(y).P} & := & \{ x \} \cup (\freenames{P} \setminus \{ y \}) \\
  \freenames{x!\langle P \rangle} & := & \{ x \} \cup \{ P \} \\
  \freenames{P|Q} & := & \freenames{P} \cup \freenames{Q} \\
  \freenames{\dropn{x}} & := & \{ x \}
\end{eqnarray*}

The bound names of a process, $\boundnames{P}$, are those names occurring in $P$
that are not free. For example, in $x?(y).0$, the name $x$ is free, while $y$ is bound.

\begin{mathpar}
  \inferrule* [lab=monoidal-laws] {} { P|Q \equiv Q|P \and P|0 \equiv P \and P|(Q|R) \equiv (P|Q)|R }
\end{mathpar}

\begin{mathpar}
  \inferrule* [lab=alpha-equivalence] {} { (x)P \equiv (y)P\{y/x\} \and y \not\in \freenames{P} }
\end{mathpar}

\begin{definition}
Then two processes, $P,Q$, are alpha-equivalent if $P = Q\{\vec{y}/\vec{x}\}$ for
some $\vec{x} \in \boundnames{Q},\vec{y} \in \boundnames{P}$, where $Q\{\vec{y}/\vec{x}\}$
denotes the capture-avoiding substitution of $\vec{y}$ for $\vec{x}$ in $Q$.
\end{definition}

\begin{definition}
  The {\em structural congruence} \cite{SangiorgiWalker} , $\equiv$,
  between processes is the least congruence containing
  alpha-equivalence, satisfying the abelian monoid laws
  (associativity, commutativity and $\pzero$ as identity) for parallel
  composition $|$ and for summation $+$.
\end{definition}

\subsection{Name equivalence}

We take name equivalence, written $\nameeq$, to be the smallest
equivalence relation generated by the following rules.

\begin{mathpar}
\inferrule*[lab=Quote-drop]
{ }
{ \quotep{@{x}} \nameeq x }

\inferrule*[lab=Struct-equiv]
{ P \scong Q }
{ \quotep{P} \nameeq \quotep{Q} }
\end{mathpar}

The astute reader will have noticed that the mutual recursion of names
and processes imposes a mutual recursion on alpha-equivalence and
structural equivalence via name-equivalence. Fortunately, all of this
works out pleasantly and we may calculate in the natural way, free of
concern. The reader interested in the details is referred to the
appendix \ref{appendix:rho_details}.

\subsection{Substitution}

We use $\Proc$ for the set of processes, $\QProc$ for the set of
names, and $\id{\{}\vec{y} / \vec{x} \id{\}}$ to denote partial maps,
$s : \QProc \rightarrow \QProc$. A map, $s$ lifts, uniquely, to a map
on process terms, $\widehat{s} : \Proc \rightarrow \Proc$ by the
following equations.

\begin{mathpar}
  (0) \psubstp{Q}{P} := 0 \\
  (R \juxtap S) \psubstp{Q}{P}
  :=    
  (R)\psubstp{Q}{P} \juxtap (S) \psubstp{Q}{P} \\
  (x?(y).R) \psubstp{Q}{P}    
  :=    
  (x)\substp{Q}{P} (z)\concat( (R \psubstn{z}{y}) \psubstp{Q}{P} ) \\
  (\lift{x}{R}) \psubstp{Q}{P}  
  :=
  \lift{(x)\substp{Q}{P}}{ R \psubstp{Q}{P} } \\
%   (\dropn{x})  \psubstp{Q}{P}       
%   := 
%   \left\{ 
%     \begin{array}{ccc} 
%       \dropn{\quotep{Q}} & & x \nameeq \quotep{P} \\
%       \dropn{x} & & otherwise \\
%     \end{array}
%   \right. 
  (\dropn{x})  \psubstp{Q}{P}       
  := 
  \left\{ 
    \begin{array}{ccc} 
      Q & & x \nameeq \quotep{P} \\
      \dropn{x} & & otherwise \\
    \end{array}
  \right.
\end{mathpar}
 

where

\begin{eqnarray}
  (x)\id{\{} \lpquote Q \rpquote / \lpquote P \rpquote \id{\}}            = 
  \left\{ 
    \begin{array}{ccc}
      \lpquote Q \rpquote & & x \nameeq \lpquote P \rpquote \\
      x & & otherwise \\
    \end{array}
  \right. \nonumber
\end{eqnarray}

and $z$ is chosen distinct from $\quotep{P}$, $\quotep{Q}$, the free
names in $Q$, and all the names in $R$. Our $\alpha$-equivalence will
be built in the standard way from this substitution.

\begin{remark}\label{rem:no_self_referential_names}
  One consequence of these definitions is that $\forall P. \quotep{P}
  \not\in \freenames{P}$.
\end{remark}

\subsection{ Dynamic quote: an example }

Anticipating something of what's to come, consider applying the
substitution, $\widehat{\id{\{}u / z \id{\}}}$, to the following pair
of processes, $\lift{w}{y!(z)}$ and $w[ \lpquote y!(z) \rpquote ]$.

\begin{eqnarray}
	\lift{w}{y!(z)}\widehat{\id{\{}u / z \id{\}}}
		& = &
		\lift{w}{y!(u)} \nonumber\\
	w[ \lpquote y!(z) \rpquote ] \widehat{ \id{\{}u / z \id{\}} }
		& = &
		w[ \lpquote y!(z) \rpquote ] \nonumber
\end{eqnarray}

Because the body of the process between quotes is impervious to
substitution, we get radically different answers. In fact, by
examining the first process in an input context,
e.g. $x?(z).\lift{w}{y!(z)}$, we see that the process under the lift
operator may be shaped by prefixed inputs binding a name inside it. In
this sense, the lift operator will be seen as a way to dynamically
construct processes before reifying them as names.

Finally equipped with these standard features we can present the
dynamics of the calculus.

\subsubsection{Operational semantics} 

Finally, we introduce the computational dynamics. What marks these
algebras as distinct from other more traditionally studied algebraic
structures, e.g. vector spaces or polynomial rings, is the manner in
which dynamics is captured. In traditional structures, dynamics is typically
expressed through morphisms between such structures, as in linear maps
between vector spaces or morphisms between rings. In algebras
associated with the semantics of computation, the dynamics is
expressed as part of the algebraic structure itself, through a
reduction reduction relation typically denoted by $\red$. Below, we
give a recursive presentation of this relation for the calculus used
in the encoding.

$\red \subseteq \pi \times \pi$
$\red : \pi \to \mathcal{P}(\pi)$

\begin{mathpar}
  \inferrule* [lab=Comm] { \textsf{match}( x_{src}, x_{trgt} ) } { x_{trgt}?(y)P \; | \; x_{src}!\langle {Q} \rangle \red P\{\quotep{Q}/y}\} }
  \and \\
  \inferrule* [lab=Par] {{P} \red {P}'} {{{P} | {Q}} \red {{P}' | {Q}}}
  \and
  \inferrule* [lab=Equiv]{{{P} \scong {P}'} \andalso {{P}' \red {Q}'} \andalso {{Q}' \scong {Q}}}{{P} \red {Q}}
\end{mathpar}

\begin{eqnarray*}
  match_{\equiv} (\quotep{P},\quotep{Q}) & := & P \equiv Q \\
  match_{\dagger}(\quotep{P},\quotep{Q}) & := & \forall R. P|Q \red^{*} R => R \red^{*} 0 \\
  match_{K}(\quotep{P},\quotep{Q}) & := & K \mbox{ for some context } K
\end{eqnarray*}

$u?(x)P | u!\langle Q \rangle \red P\{\quotep{Q}/x\}$

%We write $\wred$ for $\red^*$, and $P\red$ if $\exists Q $ such that $ P \red Q$.
We write $P\red$ if $\exists Q $ such that $ P \red Q$ and $P\not\red$, otherwise.

\section{Replication}

As mentioned before, it is known that replication (and hence
recursion) can be implemented in a higher-order process algebra
\cite{SangiorgiWalker}. As our first example of calculation with the
machinery thus far presented we give the construction explicitly in
the {\rhoc}.

\begin{eqnarray}
	D_{x} & := & \prefix{x}{y}{(\binpar{\outputp{x}{y}}{@{y}})} \nonumber\\
	\bangp_{x}{P} & := & \binpar{{x}!\langle{\binpar{D_{x}}{P}}\rangle}{D_{x}} \nonumber
\end{eqnarray}

\begin{eqnarray}
	\bangp_{x}{P} & & \nonumber\\
	=
	& {x}!\langle{(\prefix{x}{y}{(\outputp{x}{y} | @{y})) | P}}\rangle 
	      | \prefix{x}{y}{(\outputp{x}{y} | @{y})} & \nonumber\\
	\red
	& (\outputp{x}{y} | @{y})\substn{\quotep{(\prefix{x}{y}{(@{y} | \outputp{x}{y})) | P}}}{y} & \nonumber\\
	=
	& \outputp{x}{\quotep{(\prefix{x}{y}{(\outputp{x}{y} | @{y})) | P}}}
	  | {(\prefix{x}{y}{(\outputp{x}{y} | @{y})) | P}} & \nonumber\\
	\red
	& \ldots & \nonumber\\
	\red^*
	& P | P | \ldots & \nonumber
\end{eqnarray}

Of course, this encoding, as an implementation, runs away, unfolding
$\bangp{P}$ eagerly. A lazier and more implementable replication
operator, restricted to input-guarded processes, may be obtained as follows.

\begin{eqnarray}
\bangp{\prefix{u}{v}{P}} 
	:= 
	\binpar{\lift{x}{\prefix{u}{v}{(\binpar{D(x)}{P})}}}{D(x)} \nonumber
\end{eqnarray}

\begin{remark}
  Note that the lazier definition still does not deal with summation
  or mixed summation (i.e. sums over input and output). The reader is
  invited to construct definitions of replication that deal with these
  features. 

  Further, the definitions are parameterized in a name, $x$. Can you,
  gentle reader, make a definition that eliminates this parameter and
  guarantees no accidental interaction between the replication
  machinery and the process being replicated -- i.e. no accidental
  sharing of names used by the process to get its work done and the
  name(s) used by the replication to effect copying. This latter
  revision of the definition of replication is crucial to obtaining
  the expected identity $!!P \sim !P$.
\end{remark}

\begin{remark}\label{rem:paradoxical_combinator}
  The reader familiar with the lambda calculus will have noticed the
  similarity between $D$ and the paradoxical combinator.

  [Ed. note: the existence of this seems to suggest we have to be more
  restrictive on the set of processes and names we admit if we are to
  support no-cloning.]
\end{remark}

\subsubsection{Bisimulation}

The computational dynamics gives rise to another kind of equivalence,
the equivalence of computational behavior. As previously mentioned
this is typically captured \emph{via} some form of bisimulation.

% The notion we use in this paper is weak barbed bisimulation
% \cite{milner91polyadicpi}.

The notion we use in this paper is derived from weak barbed
bisimulation \cite{milner91polyadicpi}. 

\begin{definition}
An \emph{observation relation}, $\downarrow_{\mathcal N}$, over a set
of names, $\mathcal N$, is the smallest relation satisfying the rules
below.

\infrule[Out-barb]{y \in {\mathcal N}, \; x \nameeq y}
		  {\outputp{x}{v} \downarrow_{\mathcal N} x}
\infrule[Par-barb]{\mbox{$P\downarrow_{\mathcal N} x$ or $Q\downarrow_{\mathcal N} x$}}
		  {\binpar{P}{Q} \downarrow_{\mathcal N} x}

We write $P \Downarrow_{\mathcal N} x$ if there is $Q$ such that 
$P \wred Q$ and $Q \downarrow_{\mathcal N} x$.
\end{definition}

\begin{definition}
%\label{def.bbisim}
An  ${\mathcal N}$-\emph{barbed bisimulation} over a set of names, ${\mathcal N}$, is a symmetric binary relation 
${\mathcal S}_{\mathcal N}$ between agents such that $P\rel{S}_{\mathcal N}Q$ implies:
\begin{enumerate}
\item If $P \red P'$ then $Q \wred Q'$ and $P'\rel{S}_{\mathcal N} Q'$.
\item If $P\downarrow_{\mathcal N} x$, then $Q\Downarrow_{\mathcal N} x$.
\end{enumerate}
$P$ is ${\mathcal N}$-barbed bisimilar to $Q$, written
$P \wbbisim_{\mathcal N} Q$, if $P \rel{S}_{\mathcal N} Q$ for some ${\mathcal N}$-barbed bisimulation ${\mathcal S}_{\mathcal N}$.
\end{definition}

$\mathcal{R} \subseteq \pi \times \pi$

$P \mathcal{R} Q => \forall P'. P \red P' \Rightarrow \exists Q'. Q \red Q', P' \mathcal{R} Q'$

$P \vdash x \Rightarrow Q \vdash x$

\begin{mathpar}
  \inferrule*[lab=Out-barb]{x \nameeq y}{{y}!\langle{Q}\rangle \vdash x}
  \and
  \inferrule*[lab=Par-barb]{\mbox{$P\vdash x$ or $Q\vdash x$}}{\binpar{P}{Q} \vdash x}
\end{mathpar}

\subsubsection{Contexts}

One of the principle advantages of computational calculi like the
$\pi$-calculus is a well-defined notion of context,
contextual-equivalence and a correlation between
contextual-equivalence and notions of bisimulation. The notion of
context allows the decomposition of a process into (sub-)process and
its syntactic environment, its context. Thus, a context may be
thought of as a process with a ``hole'' (written $\Box$) in it. The
application of a context $M$ to a process $P$, written $M[P]$, is
tantamount to filling the hole in $M$ with $P$. In this paper we do
not need the full weight of this theory, but do make use of the notion
of context in the proof the main theorem. 

\begin{mathpar}
  \inferrule* [lab=summation] {} {{M_{M},M_{N}} \bc \Box \;|\; x.M_{A} \;|\; M_{M}+M_{N}}
  \and
  \inferrule* [lab=agent] {} {{M_{A}} \bc (\vec{x})M_{P} \;| \; \clift{P_0,\ldots,M_{P},\ldots,P_N}}
  \and \\
  \inferrule* [lab=process] {} {{M_{P}} \bc M_{N} \;| \;P|M_{P} }
\end{mathpar} 

\begin{mathpar}
  \inferrule* [lab=sychronization] {} {M_{N} \bc \Box \;|\; x?M_{F} \;|\; x!M_{C}}
  \and
  \inferrule* [lab=abstraction] {} {{M_{F}} \bc (x)M_{P} }
  \and
  \inferrule* [lab=concretion] {} {{M_{C}} \bc \langle M_{P} \rangle }
  \and \\
  \inferrule* [lab=process] {} {{M_{P}} \bc M_{N} \;| \;P|M_{P} }
\end{mathpar}

\begin{definition}[contextual application] Given a context $M$, and
  process $P$, we define the \emph{contextual application}, $M[P] :=
  M\{P/\Box\}$. That is, the contextual application of M to P is the
  substitution of $P$ for $\Box$ in $M$.
\end{definition}

$\meaningof{-} : L \to \mathcal{P}(\pi)$

\begin{mathpar}
  \inferrule* [lab=collection] {} {\meaningof{true} = \pi, \and \meaningof{~E} = \pi \setminus \meaningof{E}, \and \meaningof{E_{1} \& E_{2}} = \meaningof{E_{1}} \cap \meaningof{E_{2}}}
\end{mathpar}

\begin{mathpar}
  \inferrule* [lab=structure] {} {\meaningof{0} = \{ P \in \pi | P \equiv 0 \}, \and \\ \meaningof{E_1 | E_2} = \{ P \in \pi | P \equiv P_{1} | P_{2}, P_{1} \in \meaningof{E_{1}}, P_{2} \in \meaningof{E_2}\} }
\end{mathpar}

\begin{mathpar}
 \inferrule* [lab=behavior] {} {\meaningof{\langle a?b \rangle E} = \{ P \in \pi | P \equiv Q | u?(y)P', \\ \and \\\\ \and \\ \;\;\; u \in \meaningof{a}, \forall z.P'\{z/y\} \in \meaningof{E\{z/b\}}\}, \and \\ \meaningof{a!E} = \{ P \in \pi | P \equiv Q | x!\langle P' \rangle, x \in \meaningof{a} P' \in \meaningof{E}\} }
\end{mathpar}

\begin{mathpar}
 \inferrule* [lab=nominal] {} {\meaningof{\quotep{E}} = \{ \quotep{P} \in \quotep{\pi} | P \in \meaningof{E} \}, \and \meaningof{\quotep{P}} = \{ \quotep{Q} \in \quotep{\pi} | P \equiv Q \} \and \\ \meaningof{@\quotep{E}} = \{ P \in \pi | P \equiv @x, x \in \meaningof{E} \}}
\end{mathpar}

\begin{eqnarray*}
  \\
  \meaningof{-} : TS \to ST
\end{eqnarray*}

\begin{eqnarray*}
  \\
  L : TS \to ST
\end{eqnarray*}

\begin{eqnarray*}
  \\
  P \models E \iff P \in \meaningof{E}
\end{eqnarray*}

\begin{eqnarray*}
  P \approx_{L} Q \iff \forall E \in L. P \models E \iff Q \models E
\end{eqnarray*}

\begin{eqnarray*}
  P \approx_{K} Q
\end{eqnarray*}

\begin{eqnarray*}
  P \approx Q
\end{eqnarray*}

$\approx_{K} = \approx = \approx_{L}$

\subsubsection{Contextual duality}

Note that contexts extend the quotation operation to a family of
operations from processes to names. Given a context, $M$, we can
define a \emph{nominal context}, $\quotep{M}$ by $\quotep{M}[P] :=
\quotep{M[P]}$. To foreshadow what is to come we observe that these
operations enjoy a duality with processes very much like the duality
between vectors and maps from vectors to scalars.

Further, because the calculus is essentially higher-order, we have a
correspondence between contexts and processes. More specifically,
given a name $x$ and a context $M$ we can construct $M^{*}_{x}$ such
that 

\begin{mathpar}
  M^{*}_{x} | \lift{x}{P} \red M[P]
\end{mathpar}

namely,

\begin{mathpar}
  M^{*}_{x} := x?(u).M[\dropn{u}]
\end{mathpar}

The dependence of $M^{*}_{x}$ on a name makes it an abstraction, 

\begin{mathpar}
  M^{*} := (x)x?(u).M[\dropn{u}]
\end{mathpar}

\subsection{Additional notation}

It will sometimes be convenient to denote the process a name
quotes. We already have the notation $x = \quotep{P}$, but it will be
convenient to introduce an alternate notation, $\procn{x}$, when we
want to emphasize the connection to the use of the name. Note that, by
virtue of name equivalence, $\quotep{\procn{x}} \nameeq x$; so, the
notation is consistent with previous definitions.

Further, because names have structure it is possible to effect
substitutions on the basis of that structure. This means we need to
upgrade our notation for substitutions, which we accomplish by
adapting comprehension notation. Thus,

\begin{mathpar}
  P\{ y / x : x \in S \}
\end{mathpar}

is interpreted to mean the process derived from P by replacing (in a
capture-avoiding manner) each occurrence of $x$ in $S$ by $y$. For example,

\begin{mathpar}
  P\{ \quotep{\procn{x}|\procn{x}} / x : x \in \freenames{P} \}
\end{mathpar}

will replace each (occurrence) of a free name $x$ in $P$ by
$\quotep{\procn{x}|\procn{x}}$.

Also, we will avail ourselves of the notation $x^{L}$ and $x^{R}$ to
denote injections of a name into disjoint copies of the name
space. There are numerous ways to accomplish this. One example can be
found in \cite{MeredithR05}. This notation overloads to vectors of
names: $\vec{x}^{\pi} := (x_{i}^{\pi} \; : \; 0 \leq i < |\vec{x}| )$ where $\pi \in \{L,R\}$.

We also use $P^{\Box} := P|\Box$.

In \cite{MeredithR05} an interpretation of the new operator is
given. It turns out that there are several possible interpretations
all enjoying the requisite algebraic properties of the operator (see
\cite{milner91polyadicpi}). We will therefore make liberal use of
$(\nu\; \vec{x})P$.

% subsection the_syntax_and_semantics_of_the_notation_system (end)   

\input{qm2pi.qmops} 

\input{qm2pi.sterngerlach} 

\input{qm2pi.metric} 

% section concurrent_process_calculi (end)

%\input{qm2pi.proofsketch}

% section proof sketch (end)

%\input{qm2pi.slviaknots} 

% section spatial logic via knots (end)

\input{qm2pi.conclusion}

% section conclusion (end)

%\input{qm2pi.dtcodes} 

% section wiring algorithm (end)

\input{qm2pi.ack} 

% section acknowledgments (end)

\newpage


\bibliographystyle{plain}   
\bibliography{../../biblios/main.bib}

\input{qm2pi.rhodetails}

\end{document}

 

%\documentclass[12pt]{llncs}
%\documentclass{jktr}

\usepackage[pdftex]{hyperref}                   
\usepackage {listings}
\usepackage {mathpartir}
\usepackage{bcprules}
%\usepackage{listings}
                       
\usepackage{graphicx} 
%\usepackage[margins=2.5cm,nohead,nofoot]{geometry}
%\usepackage{geometry}
\usepackage{amsfonts}
\usepackage{amstext}
\usepackage{latexsym}
\usepackage{amssymb}
\usepackage{color}


%\include{myPreamble}
\include{qm2pi.local} 

%\ifpdf
%\usepackage[pdftex]{graphicx}
%\else
%\usepackage{graphicx}
%\fi

 % \ifpdf
%  \usepackage{pdfsync}
%  \if


%\title{Brief Article}
%\author{David F. Snyder}
%\author{L.G. Meredith}

%\address{Dept. of Math., Texas State University--San Marcos, San Marcos, TX 78666}
       
\pagestyle{empty}


\begin{document}

\lstset{language=[Objective]Caml,frame=shadowbox}

\input{qm2pi.front}

% section front matter (end)

\input{qm2pi.intro} 
 
% section introduction (end)

% \input{qm2pi.knotations} 

% section notation (end)

\input{qm2pi.process.calculi} 

% section concurrent_process_calculi_and_spatial_logics_ (end)
    
%\input{qm2pi.knots2pi} 

%\input{qm2pi.trefoil} 

%\input{qm2pi.mainthm} 

% subsection basic_interpretation (end)

%\input{qm2pi.rho.presentation} 
\subsection{The syntax and semantics of the notation system}\label{sub:the_syntax_and_semantics_of_the_notation_system} % (fold)

We now summarize a technical presentation of the calculus that
embodies our theory of dynamics. The typical presentation of such a
calculus follows the style of giving generators and relations on
them. The grammar, below, describing term constructors, freely
generates the set of processes, $\Proc$. This set is then quotiented
by a relation known as structural congruence and it is over this set
that the notion of dynamics is expressed. This presentation is
essentially that of \cite{MeredithR05} with the addition of
polyadicity and summation. For readability we have relegated some of
the technical subtleties to an appendix.

\subsubsection{Process grammar}\label{subsub:process_grammar}

\begin{mathpar}
  \inferrule* [lab=synchronization] {} {{M} \bc \pzero \;|\; x?F \;|\; x!C }
  \and
  \inferrule* [lab=abstraction] {} {{F} \bc (x)P}
  \and
  \inferrule* [lab=concretion] {} {{C} \bc \langle Q \rangle}
  \and
  \inferrule* [lab=process] {} {{P,Q} \bc M \;| \;P|Q \;|\; @{x}}
  \and
  \inferrule* [lab=name] {} {{x} \bc \quotep{P}}
\end{mathpar} 

Note that $\vec{x}$ (resp. $\vec{P}$) denotes a vector of names
(resp. processes) of length $|\vec{x}|$ (resp. $|\vec{P}|$). We adopt
the following useful abbreviations.

\begin{mathpar}
   x?(\vec{y}).P := x.(\vec{y})P \and  x\clift{\vec{P}} := x.\clift{\vec{P}}
   \and x!(y) := \lift{x}{\dropn{y}}
   \and \Pi_{i=0}^{n-1}P_i := P_0 | \ldots | P_{n-1}
\end{mathpar}

\subsubsection{Structural congruence}

\paragraph{Free and bound names and alpha-equivalence.} At the
core of structural equivalence is alpha-equivalence which identifies
process that are the same up to a change of variable. Formally, we
recognize the distinction between free and bound names. The free names
of a process, $\freenames{P}$, may be calculated recursively as
follows:

\begin{mathpar}
\freenames{\pzero} := \emptyset
  \and \\
  \freenames{x?(y).P} := \{ x \} \cup (\freenames{P} \setminus \{ y \})
  \and 
  \freenames{x!\langle P \rangle} := \{ x \} \cup \{ P \} 
  \and \\
  \freenames{P|Q} := \freenames{P} \cup \freenames{Q}
  \and \\
  \freenames{@{x}} := \{ x \}
\end{mathpar}

$\pi$
$\quotep{\pi}$

$\freenames{-} : \pi \to \mathcal{P}(\quotep{\pi})$

\begin{eqnarray*}
  \freenames{\pzero} & := & \emptyset \\
  \freenames{x?(y).P} & := & \{ x \} \cup (\freenames{P} \setminus \{ y \}) \\
  \freenames{x!\langle P \rangle} & := & \{ x \} \cup \{ P \} \\
  \freenames{P|Q} & := & \freenames{P} \cup \freenames{Q} \\
  \freenames{\dropn{x}} & := & \{ x \}
\end{eqnarray*}

The bound names of a process, $\boundnames{P}$, are those names occurring in $P$
that are not free. For example, in $x?(y).0$, the name $x$ is free, while $y$ is bound.

\begin{mathpar}
  \inferrule* [lab=monoidal-laws] {} { P|Q \equiv Q|P \and P|0 \equiv P \and P|(Q|R) \equiv (P|Q)|R }
\end{mathpar}

\begin{mathpar}
  \inferrule* [lab=alpha-equivalence] {} { (x)P \equiv (y)P\{y/x\} \and y \not\in \freenames{P} }
\end{mathpar}

\begin{definition}
Then two processes, $P,Q$, are alpha-equivalent if $P = Q\{\vec{y}/\vec{x}\}$ for
some $\vec{x} \in \boundnames{Q},\vec{y} \in \boundnames{P}$, where $Q\{\vec{y}/\vec{x}\}$
denotes the capture-avoiding substitution of $\vec{y}$ for $\vec{x}$ in $Q$.
\end{definition}

\begin{definition}
  The {\em structural congruence} \cite{SangiorgiWalker} , $\equiv$,
  between processes is the least congruence containing
  alpha-equivalence, satisfying the abelian monoid laws
  (associativity, commutativity and $\pzero$ as identity) for parallel
  composition $|$ and for summation $+$.
\end{definition}

\subsection{Name equivalence}

We take name equivalence, written $\nameeq$, to be the smallest
equivalence relation generated by the following rules.

\begin{mathpar}
\inferrule*[lab=Quote-drop]
{ }
{ \quotep{@{x}} \nameeq x }

\inferrule*[lab=Struct-equiv]
{ P \scong Q }
{ \quotep{P} \nameeq \quotep{Q} }
\end{mathpar}

The astute reader will have noticed that the mutual recursion of names
and processes imposes a mutual recursion on alpha-equivalence and
structural equivalence via name-equivalence. Fortunately, all of this
works out pleasantly and we may calculate in the natural way, free of
concern. The reader interested in the details is referred to the
appendix \ref{appendix:rho_details}.

\subsection{Substitution}

We use $\Proc$ for the set of processes, $\QProc$ for the set of
names, and $\id{\{}\vec{y} / \vec{x} \id{\}}$ to denote partial maps,
$s : \QProc \rightarrow \QProc$. A map, $s$ lifts, uniquely, to a map
on process terms, $\widehat{s} : \Proc \rightarrow \Proc$ by the
following equations.

\begin{mathpar}
  (0) \psubstp{Q}{P} := 0 \\
  (R \juxtap S) \psubstp{Q}{P}
  :=    
  (R)\psubstp{Q}{P} \juxtap (S) \psubstp{Q}{P} \\
  (x?(y).R) \psubstp{Q}{P}    
  :=    
  (x)\substp{Q}{P} (z)\concat( (R \psubstn{z}{y}) \psubstp{Q}{P} ) \\
  (\lift{x}{R}) \psubstp{Q}{P}  
  :=
  \lift{(x)\substp{Q}{P}}{ R \psubstp{Q}{P} } \\
%   (\dropn{x})  \psubstp{Q}{P}       
%   := 
%   \left\{ 
%     \begin{array}{ccc} 
%       \dropn{\quotep{Q}} & & x \nameeq \quotep{P} \\
%       \dropn{x} & & otherwise \\
%     \end{array}
%   \right. 
  (\dropn{x})  \psubstp{Q}{P}       
  := 
  \left\{ 
    \begin{array}{ccc} 
      Q & & x \nameeq \quotep{P} \\
      \dropn{x} & & otherwise \\
    \end{array}
  \right.
\end{mathpar}
 

where

\begin{eqnarray}
  (x)\id{\{} \lpquote Q \rpquote / \lpquote P \rpquote \id{\}}            = 
  \left\{ 
    \begin{array}{ccc}
      \lpquote Q \rpquote & & x \nameeq \lpquote P \rpquote \\
      x & & otherwise \\
    \end{array}
  \right. \nonumber
\end{eqnarray}

and $z$ is chosen distinct from $\quotep{P}$, $\quotep{Q}$, the free
names in $Q$, and all the names in $R$. Our $\alpha$-equivalence will
be built in the standard way from this substitution.

\begin{remark}\label{rem:no_self_referential_names}
  One consequence of these definitions is that $\forall P. \quotep{P}
  \not\in \freenames{P}$.
\end{remark}

\subsection{ Dynamic quote: an example }

Anticipating something of what's to come, consider applying the
substitution, $\widehat{\id{\{}u / z \id{\}}}$, to the following pair
of processes, $\lift{w}{y!(z)}$ and $w[ \lpquote y!(z) \rpquote ]$.

\begin{eqnarray}
	\lift{w}{y!(z)}\widehat{\id{\{}u / z \id{\}}}
		& = &
		\lift{w}{y!(u)} \nonumber\\
	w[ \lpquote y!(z) \rpquote ] \widehat{ \id{\{}u / z \id{\}} }
		& = &
		w[ \lpquote y!(z) \rpquote ] \nonumber
\end{eqnarray}

Because the body of the process between quotes is impervious to
substitution, we get radically different answers. In fact, by
examining the first process in an input context,
e.g. $x?(z).\lift{w}{y!(z)}$, we see that the process under the lift
operator may be shaped by prefixed inputs binding a name inside it. In
this sense, the lift operator will be seen as a way to dynamically
construct processes before reifying them as names.

Finally equipped with these standard features we can present the
dynamics of the calculus.

\subsubsection{Operational semantics} 

Finally, we introduce the computational dynamics. What marks these
algebras as distinct from other more traditionally studied algebraic
structures, e.g. vector spaces or polynomial rings, is the manner in
which dynamics is captured. In traditional structures, dynamics is typically
expressed through morphisms between such structures, as in linear maps
between vector spaces or morphisms between rings. In algebras
associated with the semantics of computation, the dynamics is
expressed as part of the algebraic structure itself, through a
reduction reduction relation typically denoted by $\red$. Below, we
give a recursive presentation of this relation for the calculus used
in the encoding.

$\red \subseteq \pi \times \pi$
$\red : \pi \to \mathcal{P}(\pi)$

\begin{mathpar}
  \inferrule* [lab=Comm] { \textsf{match}( x_{src}, x_{trgt} ) } { x_{trgt}?(y)P \; | \; x_{src}!\langle {Q} \rangle \red P\{\quotep{Q}/y}\} }
  \and \\
  \inferrule* [lab=Par] {{P} \red {P}'} {{{P} | {Q}} \red {{P}' | {Q}}}
  \and
  \inferrule* [lab=Equiv]{{{P} \scong {P}'} \andalso {{P}' \red {Q}'} \andalso {{Q}' \scong {Q}}}{{P} \red {Q}}
\end{mathpar}

\begin{eqnarray*}
  match_{\equiv} (\quotep{P},\quotep{Q}) & := & P \equiv Q \\
  match_{\dagger}(\quotep{P},\quotep{Q}) & := & \forall R. P|Q \red^{*} R => R \red^{*} 0 \\
  match_{K}(\quotep{P},\quotep{Q}) & := & K \mbox{ for some context } K
\end{eqnarray*}

$u?(x)P | u!\langle Q \rangle \red P\{\quotep{Q}/x\}$

%We write $\wred$ for $\red^*$, and $P\red$ if $\exists Q $ such that $ P \red Q$.
We write $P\red$ if $\exists Q $ such that $ P \red Q$ and $P\not\red$, otherwise.

\section{Replication}

As mentioned before, it is known that replication (and hence
recursion) can be implemented in a higher-order process algebra
\cite{SangiorgiWalker}. As our first example of calculation with the
machinery thus far presented we give the construction explicitly in
the {\rhoc}.

\begin{eqnarray}
	D_{x} & := & \prefix{x}{y}{(\binpar{\outputp{x}{y}}{@{y}})} \nonumber\\
	\bangp_{x}{P} & := & \binpar{{x}!\langle{\binpar{D_{x}}{P}}\rangle}{D_{x}} \nonumber
\end{eqnarray}

\begin{eqnarray}
	\bangp_{x}{P} & & \nonumber\\
	=
	& {x}!\langle{(\prefix{x}{y}{(\outputp{x}{y} | @{y})) | P}}\rangle 
	      | \prefix{x}{y}{(\outputp{x}{y} | @{y})} & \nonumber\\
	\red
	& (\outputp{x}{y} | @{y})\substn{\quotep{(\prefix{x}{y}{(@{y} | \outputp{x}{y})) | P}}}{y} & \nonumber\\
	=
	& \outputp{x}{\quotep{(\prefix{x}{y}{(\outputp{x}{y} | @{y})) | P}}}
	  | {(\prefix{x}{y}{(\outputp{x}{y} | @{y})) | P}} & \nonumber\\
	\red
	& \ldots & \nonumber\\
	\red^*
	& P | P | \ldots & \nonumber
\end{eqnarray}

Of course, this encoding, as an implementation, runs away, unfolding
$\bangp{P}$ eagerly. A lazier and more implementable replication
operator, restricted to input-guarded processes, may be obtained as follows.

\begin{eqnarray}
\bangp{\prefix{u}{v}{P}} 
	:= 
	\binpar{\lift{x}{\prefix{u}{v}{(\binpar{D(x)}{P})}}}{D(x)} \nonumber
\end{eqnarray}

\begin{remark}
  Note that the lazier definition still does not deal with summation
  or mixed summation (i.e. sums over input and output). The reader is
  invited to construct definitions of replication that deal with these
  features. 

  Further, the definitions are parameterized in a name, $x$. Can you,
  gentle reader, make a definition that eliminates this parameter and
  guarantees no accidental interaction between the replication
  machinery and the process being replicated -- i.e. no accidental
  sharing of names used by the process to get its work done and the
  name(s) used by the replication to effect copying. This latter
  revision of the definition of replication is crucial to obtaining
  the expected identity $!!P \sim !P$.
\end{remark}

\begin{remark}\label{rem:paradoxical_combinator}
  The reader familiar with the lambda calculus will have noticed the
  similarity between $D$ and the paradoxical combinator.

  [Ed. note: the existence of this seems to suggest we have to be more
  restrictive on the set of processes and names we admit if we are to
  support no-cloning.]
\end{remark}

\subsubsection{Bisimulation}

The computational dynamics gives rise to another kind of equivalence,
the equivalence of computational behavior. As previously mentioned
this is typically captured \emph{via} some form of bisimulation.

% The notion we use in this paper is weak barbed bisimulation
% \cite{milner91polyadicpi}.

The notion we use in this paper is derived from weak barbed
bisimulation \cite{milner91polyadicpi}. 

\begin{definition}
An \emph{observation relation}, $\downarrow_{\mathcal N}$, over a set
of names, $\mathcal N$, is the smallest relation satisfying the rules
below.

\infrule[Out-barb]{y \in {\mathcal N}, \; x \nameeq y}
		  {\outputp{x}{v} \downarrow_{\mathcal N} x}
\infrule[Par-barb]{\mbox{$P\downarrow_{\mathcal N} x$ or $Q\downarrow_{\mathcal N} x$}}
		  {\binpar{P}{Q} \downarrow_{\mathcal N} x}

We write $P \Downarrow_{\mathcal N} x$ if there is $Q$ such that 
$P \wred Q$ and $Q \downarrow_{\mathcal N} x$.
\end{definition}

\begin{definition}
%\label{def.bbisim}
An  ${\mathcal N}$-\emph{barbed bisimulation} over a set of names, ${\mathcal N}$, is a symmetric binary relation 
${\mathcal S}_{\mathcal N}$ between agents such that $P\rel{S}_{\mathcal N}Q$ implies:
\begin{enumerate}
\item If $P \red P'$ then $Q \wred Q'$ and $P'\rel{S}_{\mathcal N} Q'$.
\item If $P\downarrow_{\mathcal N} x$, then $Q\Downarrow_{\mathcal N} x$.
\end{enumerate}
$P$ is ${\mathcal N}$-barbed bisimilar to $Q$, written
$P \wbbisim_{\mathcal N} Q$, if $P \rel{S}_{\mathcal N} Q$ for some ${\mathcal N}$-barbed bisimulation ${\mathcal S}_{\mathcal N}$.
\end{definition}

$\mathcal{R} \subseteq \pi \times \pi$

$P \mathcal{R} Q => \forall P'. P \red P' \Rightarrow \exists Q'. Q \red Q', P' \mathcal{R} Q'$

$P \vdash x \Rightarrow Q \vdash x$

\begin{mathpar}
  \inferrule*[lab=Out-barb]{x \nameeq y}{{y}!\langle{Q}\rangle \vdash x}
  \and
  \inferrule*[lab=Par-barb]{\mbox{$P\vdash x$ or $Q\vdash x$}}{\binpar{P}{Q} \vdash x}
\end{mathpar}

\subsubsection{Contexts}

One of the principle advantages of computational calculi like the
$\pi$-calculus is a well-defined notion of context,
contextual-equivalence and a correlation between
contextual-equivalence and notions of bisimulation. The notion of
context allows the decomposition of a process into (sub-)process and
its syntactic environment, its context. Thus, a context may be
thought of as a process with a ``hole'' (written $\Box$) in it. The
application of a context $M$ to a process $P$, written $M[P]$, is
tantamount to filling the hole in $M$ with $P$. In this paper we do
not need the full weight of this theory, but do make use of the notion
of context in the proof the main theorem. 

\begin{mathpar}
  \inferrule* [lab=summation] {} {{M_{M},M_{N}} \bc \Box \;|\; x.M_{A} \;|\; M_{M}+M_{N}}
  \and
  \inferrule* [lab=agent] {} {{M_{A}} \bc (\vec{x})M_{P} \;| \; \clift{P_0,\ldots,M_{P},\ldots,P_N}}
  \and \\
  \inferrule* [lab=process] {} {{M_{P}} \bc M_{N} \;| \;P|M_{P} }
\end{mathpar} 

\begin{mathpar}
  \inferrule* [lab=sychronization] {} {M_{N} \bc \Box \;|\; x?M_{F} \;|\; x!M_{C}}
  \and
  \inferrule* [lab=abstraction] {} {{M_{F}} \bc (x)M_{P} }
  \and
  \inferrule* [lab=concretion] {} {{M_{C}} \bc \langle M_{P} \rangle }
  \and \\
  \inferrule* [lab=process] {} {{M_{P}} \bc M_{N} \;| \;P|M_{P} }
\end{mathpar}

\begin{definition}[contextual application] Given a context $M$, and
  process $P$, we define the \emph{contextual application}, $M[P] :=
  M\{P/\Box\}$. That is, the contextual application of M to P is the
  substitution of $P$ for $\Box$ in $M$.
\end{definition}

$\meaningof{-} : L \to \mathcal{P}(\pi)$

\begin{mathpar}
  \inferrule* [lab=collection] {} {\meaningof{true} = \pi, \and \meaningof{~E} = \pi \setminus \meaningof{E}, \and \meaningof{E_{1} \& E_{2}} = \meaningof{E_{1}} \cap \meaningof{E_{2}}}
\end{mathpar}

\begin{mathpar}
  \inferrule* [lab=structure] {} {\meaningof{0} = \{ P \in \pi | P \equiv 0 \}, \and \\ \meaningof{E_1 | E_2} = \{ P \in \pi | P \equiv P_{1} | P_{2}, P_{1} \in \meaningof{E_{1}}, P_{2} \in \meaningof{E_2}\} }
\end{mathpar}

\begin{mathpar}
 \inferrule* [lab=behavior] {} {\meaningof{\langle a?b \rangle E} = \{ P \in \pi | P \equiv Q | u?(y)P', \\ \and \\\\ \and \\ \;\;\; u \in \meaningof{a}, \forall z.P'\{z/y\} \in \meaningof{E\{z/b\}}\}, \and \\ \meaningof{a!E} = \{ P \in \pi | P \equiv Q | x!\langle P' \rangle, x \in \meaningof{a} P' \in \meaningof{E}\} }
\end{mathpar}

\begin{mathpar}
 \inferrule* [lab=nominal] {} {\meaningof{\quotep{E}} = \{ \quotep{P} \in \quotep{\pi} | P \in \meaningof{E} \}, \and \meaningof{\quotep{P}} = \{ \quotep{Q} \in \quotep{\pi} | P \equiv Q \} \and \\ \meaningof{@\quotep{E}} = \{ P \in \pi | P \equiv @x, x \in \meaningof{E} \}}
\end{mathpar}

\begin{eqnarray*}
  \\
  \meaningof{-} : TS \to ST
\end{eqnarray*}

\begin{eqnarray*}
  \\
  L : TS \to ST
\end{eqnarray*}

\begin{eqnarray*}
  \\
  P \models E \iff P \in \meaningof{E}
\end{eqnarray*}

\begin{eqnarray*}
  P \approx_{L} Q \iff \forall E \in L. P \models E \iff Q \models E
\end{eqnarray*}

\begin{eqnarray*}
  P \approx_{K} Q
\end{eqnarray*}

\begin{eqnarray*}
  P \approx Q
\end{eqnarray*}

$\approx_{K} = \approx = \approx_{L}$

\subsubsection{Contextual duality}

Note that contexts extend the quotation operation to a family of
operations from processes to names. Given a context, $M$, we can
define a \emph{nominal context}, $\quotep{M}$ by $\quotep{M}[P] :=
\quotep{M[P]}$. To foreshadow what is to come we observe that these
operations enjoy a duality with processes very much like the duality
between vectors and maps from vectors to scalars.

Further, because the calculus is essentially higher-order, we have a
correspondence between contexts and processes. More specifically,
given a name $x$ and a context $M$ we can construct $M^{*}_{x}$ such
that 

\begin{mathpar}
  M^{*}_{x} | \lift{x}{P} \red M[P]
\end{mathpar}

namely,

\begin{mathpar}
  M^{*}_{x} := x?(u).M[\dropn{u}]
\end{mathpar}

The dependence of $M^{*}_{x}$ on a name makes it an abstraction, 

\begin{mathpar}
  M^{*} := (x)x?(u).M[\dropn{u}]
\end{mathpar}

\subsection{Additional notation}

It will sometimes be convenient to denote the process a name
quotes. We already have the notation $x = \quotep{P}$, but it will be
convenient to introduce an alternate notation, $\procn{x}$, when we
want to emphasize the connection to the use of the name. Note that, by
virtue of name equivalence, $\quotep{\procn{x}} \nameeq x$; so, the
notation is consistent with previous definitions.

Further, because names have structure it is possible to effect
substitutions on the basis of that structure. This means we need to
upgrade our notation for substitutions, which we accomplish by
adapting comprehension notation. Thus,

\begin{mathpar}
  P\{ y / x : x \in S \}
\end{mathpar}

is interpreted to mean the process derived from P by replacing (in a
capture-avoiding manner) each occurrence of $x$ in $S$ by $y$. For example,

\begin{mathpar}
  P\{ \quotep{\procn{x}|\procn{x}} / x : x \in \freenames{P} \}
\end{mathpar}

will replace each (occurrence) of a free name $x$ in $P$ by
$\quotep{\procn{x}|\procn{x}}$.

Also, we will avail ourselves of the notation $x^{L}$ and $x^{R}$ to
denote injections of a name into disjoint copies of the name
space. There are numerous ways to accomplish this. One example can be
found in \cite{MeredithR05}. This notation overloads to vectors of
names: $\vec{x}^{\pi} := (x_{i}^{\pi} \; : \; 0 \leq i < |\vec{x}| )$ where $\pi \in \{L,R\}$.

We also use $P^{\Box} := P|\Box$.

In \cite{MeredithR05} an interpretation of the new operator is
given. It turns out that there are several possible interpretations
all enjoying the requisite algebraic properties of the operator (see
\cite{milner91polyadicpi}). We will therefore make liberal use of
$(\nu\; \vec{x})P$.

% subsection the_syntax_and_semantics_of_the_notation_system (end)   

\input{qm2pi.qmops} 

\input{qm2pi.sterngerlach} 

\input{qm2pi.metric} 

% section concurrent_process_calculi (end)

%\input{qm2pi.proofsketch}

% section proof sketch (end)

%\input{qm2pi.slviaknots} 

% section spatial logic via knots (end)

\input{qm2pi.conclusion}

% section conclusion (end)

%\input{qm2pi.dtcodes} 

% section wiring algorithm (end)

\input{qm2pi.ack} 

% section acknowledgments (end)

\newpage


\bibliographystyle{plain}   
\bibliography{../../biblios/main.bib}

\input{qm2pi.rhodetails}

\end{document}

 

% subsection basic_interpretation (end)

%\input{qm2pi.rho.presentation} 
\subsection{The syntax and semantics of the notation system}\label{sub:the_syntax_and_semantics_of_the_notation_system} % (fold)

We now summarize a technical presentation of the calculus that
embodies our theory of dynamics. The typical presentation of such a
calculus follows the style of giving generators and relations on
them. The grammar, below, describing term constructors, freely
generates the set of processes, $\Proc$. This set is then quotiented
by a relation known as structural congruence and it is over this set
that the notion of dynamics is expressed. This presentation is
essentially that of \cite{MeredithR05} with the addition of
polyadicity and summation. For readability we have relegated some of
the technical subtleties to an appendix.

\subsubsection{Process grammar}\label{subsub:process_grammar}

\begin{mathpar}
  \inferrule* [lab=synchronization] {} {{M} \bc \pzero \;|\; x?F \;|\; x!C }
  \and
  \inferrule* [lab=abstraction] {} {{F} \bc (x)P}
  \and
  \inferrule* [lab=concretion] {} {{C} \bc \langle Q \rangle}
  \and
  \inferrule* [lab=process] {} {{P,Q} \bc M \;| \;P|Q \;|\; @{x}}
  \and
  \inferrule* [lab=name] {} {{x} \bc \quotep{P}}
\end{mathpar} 

Note that $\vec{x}$ (resp. $\vec{P}$) denotes a vector of names
(resp. processes) of length $|\vec{x}|$ (resp. $|\vec{P}|$). We adopt
the following useful abbreviations.

\begin{mathpar}
   x?(\vec{y}).P := x.(\vec{y})P \and  x\clift{\vec{P}} := x.\clift{\vec{P}}
   \and x!(y) := \lift{x}{\dropn{y}}
   \and \Pi_{i=0}^{n-1}P_i := P_0 | \ldots | P_{n-1}
\end{mathpar}

\subsubsection{Structural congruence}

\paragraph{Free and bound names and alpha-equivalence.} At the
core of structural equivalence is alpha-equivalence which identifies
process that are the same up to a change of variable. Formally, we
recognize the distinction between free and bound names. The free names
of a process, $\freenames{P}$, may be calculated recursively as
follows:

\begin{mathpar}
\freenames{\pzero} := \emptyset
  \and \\
  \freenames{x?(y).P} := \{ x \} \cup (\freenames{P} \setminus \{ y \})
  \and 
  \freenames{x!\langle P \rangle} := \{ x \} \cup \{ P \} 
  \and \\
  \freenames{P|Q} := \freenames{P} \cup \freenames{Q}
  \and \\
  \freenames{@{x}} := \{ x \}
\end{mathpar}

$\pi$
$\quotep{\pi}$

$\freenames{-} : \pi \to \mathcal{P}(\quotep{\pi})$

\begin{eqnarray*}
  \freenames{\pzero} & := & \emptyset \\
  \freenames{x?(y).P} & := & \{ x \} \cup (\freenames{P} \setminus \{ y \}) \\
  \freenames{x!\langle P \rangle} & := & \{ x \} \cup \{ P \} \\
  \freenames{P|Q} & := & \freenames{P} \cup \freenames{Q} \\
  \freenames{\dropn{x}} & := & \{ x \}
\end{eqnarray*}

The bound names of a process, $\boundnames{P}$, are those names occurring in $P$
that are not free. For example, in $x?(y).0$, the name $x$ is free, while $y$ is bound.

\begin{mathpar}
  \inferrule* [lab=monoidal-laws] {} { P|Q \equiv Q|P \and P|0 \equiv P \and P|(Q|R) \equiv (P|Q)|R }
\end{mathpar}

\begin{mathpar}
  \inferrule* [lab=alpha-equivalence] {} { (x)P \equiv (y)P\{y/x\} \and y \not\in \freenames{P} }
\end{mathpar}

\begin{definition}
Then two processes, $P,Q$, are alpha-equivalent if $P = Q\{\vec{y}/\vec{x}\}$ for
some $\vec{x} \in \boundnames{Q},\vec{y} \in \boundnames{P}$, where $Q\{\vec{y}/\vec{x}\}$
denotes the capture-avoiding substitution of $\vec{y}$ for $\vec{x}$ in $Q$.
\end{definition}

\begin{definition}
  The {\em structural congruence} \cite{SangiorgiWalker} , $\equiv$,
  between processes is the least congruence containing
  alpha-equivalence, satisfying the abelian monoid laws
  (associativity, commutativity and $\pzero$ as identity) for parallel
  composition $|$ and for summation $+$.
\end{definition}

\subsection{Name equivalence}

We take name equivalence, written $\nameeq$, to be the smallest
equivalence relation generated by the following rules.

\begin{mathpar}
\inferrule*[lab=Quote-drop]
{ }
{ \quotep{@{x}} \nameeq x }

\inferrule*[lab=Struct-equiv]
{ P \scong Q }
{ \quotep{P} \nameeq \quotep{Q} }
\end{mathpar}

The astute reader will have noticed that the mutual recursion of names
and processes imposes a mutual recursion on alpha-equivalence and
structural equivalence via name-equivalence. Fortunately, all of this
works out pleasantly and we may calculate in the natural way, free of
concern. The reader interested in the details is referred to the
appendix \ref{appendix:rho_details}.

\subsection{Substitution}

We use $\Proc$ for the set of processes, $\QProc$ for the set of
names, and $\id{\{}\vec{y} / \vec{x} \id{\}}$ to denote partial maps,
$s : \QProc \rightarrow \QProc$. A map, $s$ lifts, uniquely, to a map
on process terms, $\widehat{s} : \Proc \rightarrow \Proc$ by the
following equations.

\begin{mathpar}
  (0) \psubstp{Q}{P} := 0 \\
  (R \juxtap S) \psubstp{Q}{P}
  :=    
  (R)\psubstp{Q}{P} \juxtap (S) \psubstp{Q}{P} \\
  (x?(y).R) \psubstp{Q}{P}    
  :=    
  (x)\substp{Q}{P} (z)\concat( (R \psubstn{z}{y}) \psubstp{Q}{P} ) \\
  (\lift{x}{R}) \psubstp{Q}{P}  
  :=
  \lift{(x)\substp{Q}{P}}{ R \psubstp{Q}{P} } \\
%   (\dropn{x})  \psubstp{Q}{P}       
%   := 
%   \left\{ 
%     \begin{array}{ccc} 
%       \dropn{\quotep{Q}} & & x \nameeq \quotep{P} \\
%       \dropn{x} & & otherwise \\
%     \end{array}
%   \right. 
  (\dropn{x})  \psubstp{Q}{P}       
  := 
  \left\{ 
    \begin{array}{ccc} 
      Q & & x \nameeq \quotep{P} \\
      \dropn{x} & & otherwise \\
    \end{array}
  \right.
\end{mathpar}
 

where

\begin{eqnarray}
  (x)\id{\{} \lpquote Q \rpquote / \lpquote P \rpquote \id{\}}            = 
  \left\{ 
    \begin{array}{ccc}
      \lpquote Q \rpquote & & x \nameeq \lpquote P \rpquote \\
      x & & otherwise \\
    \end{array}
  \right. \nonumber
\end{eqnarray}

and $z$ is chosen distinct from $\quotep{P}$, $\quotep{Q}$, the free
names in $Q$, and all the names in $R$. Our $\alpha$-equivalence will
be built in the standard way from this substitution.

\begin{remark}\label{rem:no_self_referential_names}
  One consequence of these definitions is that $\forall P. \quotep{P}
  \not\in \freenames{P}$.
\end{remark}

\subsection{ Dynamic quote: an example }

Anticipating something of what's to come, consider applying the
substitution, $\widehat{\id{\{}u / z \id{\}}}$, to the following pair
of processes, $\lift{w}{y!(z)}$ and $w[ \lpquote y!(z) \rpquote ]$.

\begin{eqnarray}
	\lift{w}{y!(z)}\widehat{\id{\{}u / z \id{\}}}
		& = &
		\lift{w}{y!(u)} \nonumber\\
	w[ \lpquote y!(z) \rpquote ] \widehat{ \id{\{}u / z \id{\}} }
		& = &
		w[ \lpquote y!(z) \rpquote ] \nonumber
\end{eqnarray}

Because the body of the process between quotes is impervious to
substitution, we get radically different answers. In fact, by
examining the first process in an input context,
e.g. $x?(z).\lift{w}{y!(z)}$, we see that the process under the lift
operator may be shaped by prefixed inputs binding a name inside it. In
this sense, the lift operator will be seen as a way to dynamically
construct processes before reifying them as names.

Finally equipped with these standard features we can present the
dynamics of the calculus.

\subsubsection{Operational semantics} 

Finally, we introduce the computational dynamics. What marks these
algebras as distinct from other more traditionally studied algebraic
structures, e.g. vector spaces or polynomial rings, is the manner in
which dynamics is captured. In traditional structures, dynamics is typically
expressed through morphisms between such structures, as in linear maps
between vector spaces or morphisms between rings. In algebras
associated with the semantics of computation, the dynamics is
expressed as part of the algebraic structure itself, through a
reduction reduction relation typically denoted by $\red$. Below, we
give a recursive presentation of this relation for the calculus used
in the encoding.

$\red \subseteq \pi \times \pi$
$\red : \pi \to \mathcal{P}(\pi)$

\begin{mathpar}
  \inferrule* [lab=Comm] { \textsf{match}( x_{src}, x_{trgt} ) } { x_{trgt}?(y)P \; | \; x_{src}!\langle {Q} \rangle \red P\{\quotep{Q}/y}\} }
  \and \\
  \inferrule* [lab=Par] {{P} \red {P}'} {{{P} | {Q}} \red {{P}' | {Q}}}
  \and
  \inferrule* [lab=Equiv]{{{P} \scong {P}'} \andalso {{P}' \red {Q}'} \andalso {{Q}' \scong {Q}}}{{P} \red {Q}}
\end{mathpar}

\begin{eqnarray*}
  match_{\equiv} (\quotep{P},\quotep{Q}) & := & P \equiv Q \\
  match_{\dagger}(\quotep{P},\quotep{Q}) & := & \forall R. P|Q \red^{*} R => R \red^{*} 0 \\
  match_{K}(\quotep{P},\quotep{Q}) & := & K \mbox{ for some context } K
\end{eqnarray*}

$u?(x)P | u!\langle Q \rangle \red P\{\quotep{Q}/x\}$

%We write $\wred$ for $\red^*$, and $P\red$ if $\exists Q $ such that $ P \red Q$.
We write $P\red$ if $\exists Q $ such that $ P \red Q$ and $P\not\red$, otherwise.

\section{Replication}

As mentioned before, it is known that replication (and hence
recursion) can be implemented in a higher-order process algebra
\cite{SangiorgiWalker}. As our first example of calculation with the
machinery thus far presented we give the construction explicitly in
the {\rhoc}.

\begin{eqnarray}
	D_{x} & := & \prefix{x}{y}{(\binpar{\outputp{x}{y}}{@{y}})} \nonumber\\
	\bangp_{x}{P} & := & \binpar{{x}!\langle{\binpar{D_{x}}{P}}\rangle}{D_{x}} \nonumber
\end{eqnarray}

\begin{eqnarray}
	\bangp_{x}{P} & & \nonumber\\
	=
	& {x}!\langle{(\prefix{x}{y}{(\outputp{x}{y} | @{y})) | P}}\rangle 
	      | \prefix{x}{y}{(\outputp{x}{y} | @{y})} & \nonumber\\
	\red
	& (\outputp{x}{y} | @{y})\substn{\quotep{(\prefix{x}{y}{(@{y} | \outputp{x}{y})) | P}}}{y} & \nonumber\\
	=
	& \outputp{x}{\quotep{(\prefix{x}{y}{(\outputp{x}{y} | @{y})) | P}}}
	  | {(\prefix{x}{y}{(\outputp{x}{y} | @{y})) | P}} & \nonumber\\
	\red
	& \ldots & \nonumber\\
	\red^*
	& P | P | \ldots & \nonumber
\end{eqnarray}

Of course, this encoding, as an implementation, runs away, unfolding
$\bangp{P}$ eagerly. A lazier and more implementable replication
operator, restricted to input-guarded processes, may be obtained as follows.

\begin{eqnarray}
\bangp{\prefix{u}{v}{P}} 
	:= 
	\binpar{\lift{x}{\prefix{u}{v}{(\binpar{D(x)}{P})}}}{D(x)} \nonumber
\end{eqnarray}

\begin{remark}
  Note that the lazier definition still does not deal with summation
  or mixed summation (i.e. sums over input and output). The reader is
  invited to construct definitions of replication that deal with these
  features. 

  Further, the definitions are parameterized in a name, $x$. Can you,
  gentle reader, make a definition that eliminates this parameter and
  guarantees no accidental interaction between the replication
  machinery and the process being replicated -- i.e. no accidental
  sharing of names used by the process to get its work done and the
  name(s) used by the replication to effect copying. This latter
  revision of the definition of replication is crucial to obtaining
  the expected identity $!!P \sim !P$.
\end{remark}

\begin{remark}\label{rem:paradoxical_combinator}
  The reader familiar with the lambda calculus will have noticed the
  similarity between $D$ and the paradoxical combinator.

  [Ed. note: the existence of this seems to suggest we have to be more
  restrictive on the set of processes and names we admit if we are to
  support no-cloning.]
\end{remark}

\subsubsection{Bisimulation}

The computational dynamics gives rise to another kind of equivalence,
the equivalence of computational behavior. As previously mentioned
this is typically captured \emph{via} some form of bisimulation.

% The notion we use in this paper is weak barbed bisimulation
% \cite{milner91polyadicpi}.

The notion we use in this paper is derived from weak barbed
bisimulation \cite{milner91polyadicpi}. 

\begin{definition}
An \emph{observation relation}, $\downarrow_{\mathcal N}$, over a set
of names, $\mathcal N$, is the smallest relation satisfying the rules
below.

\infrule[Out-barb]{y \in {\mathcal N}, \; x \nameeq y}
		  {\outputp{x}{v} \downarrow_{\mathcal N} x}
\infrule[Par-barb]{\mbox{$P\downarrow_{\mathcal N} x$ or $Q\downarrow_{\mathcal N} x$}}
		  {\binpar{P}{Q} \downarrow_{\mathcal N} x}

We write $P \Downarrow_{\mathcal N} x$ if there is $Q$ such that 
$P \wred Q$ and $Q \downarrow_{\mathcal N} x$.
\end{definition}

\begin{definition}
%\label{def.bbisim}
An  ${\mathcal N}$-\emph{barbed bisimulation} over a set of names, ${\mathcal N}$, is a symmetric binary relation 
${\mathcal S}_{\mathcal N}$ between agents such that $P\rel{S}_{\mathcal N}Q$ implies:
\begin{enumerate}
\item If $P \red P'$ then $Q \wred Q'$ and $P'\rel{S}_{\mathcal N} Q'$.
\item If $P\downarrow_{\mathcal N} x$, then $Q\Downarrow_{\mathcal N} x$.
\end{enumerate}
$P$ is ${\mathcal N}$-barbed bisimilar to $Q$, written
$P \wbbisim_{\mathcal N} Q$, if $P \rel{S}_{\mathcal N} Q$ for some ${\mathcal N}$-barbed bisimulation ${\mathcal S}_{\mathcal N}$.
\end{definition}

$\mathcal{R} \subseteq \pi \times \pi$

$P \mathcal{R} Q => \forall P'. P \red P' \Rightarrow \exists Q'. Q \red Q', P' \mathcal{R} Q'$

$P \vdash x \Rightarrow Q \vdash x$

\begin{mathpar}
  \inferrule*[lab=Out-barb]{x \nameeq y}{{y}!\langle{Q}\rangle \vdash x}
  \and
  \inferrule*[lab=Par-barb]{\mbox{$P\vdash x$ or $Q\vdash x$}}{\binpar{P}{Q} \vdash x}
\end{mathpar}

\subsubsection{Contexts}

One of the principle advantages of computational calculi like the
$\pi$-calculus is a well-defined notion of context,
contextual-equivalence and a correlation between
contextual-equivalence and notions of bisimulation. The notion of
context allows the decomposition of a process into (sub-)process and
its syntactic environment, its context. Thus, a context may be
thought of as a process with a ``hole'' (written $\Box$) in it. The
application of a context $M$ to a process $P$, written $M[P]$, is
tantamount to filling the hole in $M$ with $P$. In this paper we do
not need the full weight of this theory, but do make use of the notion
of context in the proof the main theorem. 

\begin{mathpar}
  \inferrule* [lab=summation] {} {{M_{M},M_{N}} \bc \Box \;|\; x.M_{A} \;|\; M_{M}+M_{N}}
  \and
  \inferrule* [lab=agent] {} {{M_{A}} \bc (\vec{x})M_{P} \;| \; \clift{P_0,\ldots,M_{P},\ldots,P_N}}
  \and \\
  \inferrule* [lab=process] {} {{M_{P}} \bc M_{N} \;| \;P|M_{P} }
\end{mathpar} 

\begin{mathpar}
  \inferrule* [lab=sychronization] {} {M_{N} \bc \Box \;|\; x?M_{F} \;|\; x!M_{C}}
  \and
  \inferrule* [lab=abstraction] {} {{M_{F}} \bc (x)M_{P} }
  \and
  \inferrule* [lab=concretion] {} {{M_{C}} \bc \langle M_{P} \rangle }
  \and \\
  \inferrule* [lab=process] {} {{M_{P}} \bc M_{N} \;| \;P|M_{P} }
\end{mathpar}

\begin{definition}[contextual application] Given a context $M$, and
  process $P$, we define the \emph{contextual application}, $M[P] :=
  M\{P/\Box\}$. That is, the contextual application of M to P is the
  substitution of $P$ for $\Box$ in $M$.
\end{definition}

$\meaningof{-} : L \to \mathcal{P}(\pi)$

\begin{mathpar}
  \inferrule* [lab=collection] {} {\meaningof{true} = \pi, \and \meaningof{~E} = \pi \setminus \meaningof{E}, \and \meaningof{E_{1} \& E_{2}} = \meaningof{E_{1}} \cap \meaningof{E_{2}}}
\end{mathpar}

\begin{mathpar}
  \inferrule* [lab=structure] {} {\meaningof{0} = \{ P \in \pi | P \equiv 0 \}, \and \\ \meaningof{E_1 | E_2} = \{ P \in \pi | P \equiv P_{1} | P_{2}, P_{1} \in \meaningof{E_{1}}, P_{2} \in \meaningof{E_2}\} }
\end{mathpar}

\begin{mathpar}
 \inferrule* [lab=behavior] {} {\meaningof{\langle a?b \rangle E} = \{ P \in \pi | P \equiv Q | u?(y)P', \\ \and \\\\ \and \\ \;\;\; u \in \meaningof{a}, \forall z.P'\{z/y\} \in \meaningof{E\{z/b\}}\}, \and \\ \meaningof{a!E} = \{ P \in \pi | P \equiv Q | x!\langle P' \rangle, x \in \meaningof{a} P' \in \meaningof{E}\} }
\end{mathpar}

\begin{mathpar}
 \inferrule* [lab=nominal] {} {\meaningof{\quotep{E}} = \{ \quotep{P} \in \quotep{\pi} | P \in \meaningof{E} \}, \and \meaningof{\quotep{P}} = \{ \quotep{Q} \in \quotep{\pi} | P \equiv Q \} \and \\ \meaningof{@\quotep{E}} = \{ P \in \pi | P \equiv @x, x \in \meaningof{E} \}}
\end{mathpar}

\begin{eqnarray*}
  \\
  \meaningof{-} : TS \to ST
\end{eqnarray*}

\begin{eqnarray*}
  \\
  L : TS \to ST
\end{eqnarray*}

\begin{eqnarray*}
  \\
  P \models E \iff P \in \meaningof{E}
\end{eqnarray*}

\begin{eqnarray*}
  P \approx_{L} Q \iff \forall E \in L. P \models E \iff Q \models E
\end{eqnarray*}

\begin{eqnarray*}
  P \approx_{K} Q
\end{eqnarray*}

\begin{eqnarray*}
  P \approx Q
\end{eqnarray*}

$\approx_{K} = \approx = \approx_{L}$

\subsubsection{Contextual duality}

Note that contexts extend the quotation operation to a family of
operations from processes to names. Given a context, $M$, we can
define a \emph{nominal context}, $\quotep{M}$ by $\quotep{M}[P] :=
\quotep{M[P]}$. To foreshadow what is to come we observe that these
operations enjoy a duality with processes very much like the duality
between vectors and maps from vectors to scalars.

Further, because the calculus is essentially higher-order, we have a
correspondence between contexts and processes. More specifically,
given a name $x$ and a context $M$ we can construct $M^{*}_{x}$ such
that 

\begin{mathpar}
  M^{*}_{x} | \lift{x}{P} \red M[P]
\end{mathpar}

namely,

\begin{mathpar}
  M^{*}_{x} := x?(u).M[\dropn{u}]
\end{mathpar}

The dependence of $M^{*}_{x}$ on a name makes it an abstraction, 

\begin{mathpar}
  M^{*} := (x)x?(u).M[\dropn{u}]
\end{mathpar}

\subsection{Additional notation}

It will sometimes be convenient to denote the process a name
quotes. We already have the notation $x = \quotep{P}$, but it will be
convenient to introduce an alternate notation, $\procn{x}$, when we
want to emphasize the connection to the use of the name. Note that, by
virtue of name equivalence, $\quotep{\procn{x}} \nameeq x$; so, the
notation is consistent with previous definitions.

Further, because names have structure it is possible to effect
substitutions on the basis of that structure. This means we need to
upgrade our notation for substitutions, which we accomplish by
adapting comprehension notation. Thus,

\begin{mathpar}
  P\{ y / x : x \in S \}
\end{mathpar}

is interpreted to mean the process derived from P by replacing (in a
capture-avoiding manner) each occurrence of $x$ in $S$ by $y$. For example,

\begin{mathpar}
  P\{ \quotep{\procn{x}|\procn{x}} / x : x \in \freenames{P} \}
\end{mathpar}

will replace each (occurrence) of a free name $x$ in $P$ by
$\quotep{\procn{x}|\procn{x}}$.

Also, we will avail ourselves of the notation $x^{L}$ and $x^{R}$ to
denote injections of a name into disjoint copies of the name
space. There are numerous ways to accomplish this. One example can be
found in \cite{MeredithR05}. This notation overloads to vectors of
names: $\vec{x}^{\pi} := (x_{i}^{\pi} \; : \; 0 \leq i < |\vec{x}| )$ where $\pi \in \{L,R\}$.

We also use $P^{\Box} := P|\Box$.

In \cite{MeredithR05} an interpretation of the new operator is
given. It turns out that there are several possible interpretations
all enjoying the requisite algebraic properties of the operator (see
\cite{milner91polyadicpi}). We will therefore make liberal use of
$(\nu\; \vec{x})P$.

% subsection the_syntax_and_semantics_of_the_notation_system (end)   

\section{Interpretation of QM}
\subsection{Supporting definitions}
\subsubsection{Multiplication}
\begin{mathpar}
  \quotep{Q} \cdot \quotep{R} := \quotep{Q|R}
  \and \\
  \quotep{Q} \cdot P := P\{ \quotep{Q|R} / \quotep{R} : \quotep{R} \in \freenames{P} \}
\end{mathpar}

\paragraph{Discussion}
The first line needs little explanation. The second line says that
each free name of the process is replaced with the multiplication of
that name by the scalar. Multiplication of a scalar (name) by a state
(process) results in a process all the names of which have been `moved
over' by parallel composition with the process the scalar
quotes. There is a subtlety that the bound names have to be
manipulated so that multiplied names aren't accidentally
captured. There are many ways to achieve this.

\begin{remark}\label{rem:multiplication_identities}
  The reader is invited to verify that for all $x,y,z \in \QProc$ and $P \in \Proc$
  \begin{mathpar}
    x \cdot \quotep{0} \equiv x 
    \and
    x \cdot y \equiv y \cdot x
    \and
    x \cdot (y \cdot z) \equiv (x \cdot y) \cdot z
    \and \\
    \quotep{0} \cdot P \equiv P
    \and \\
    x \cdot (y \cdot P) \equiv (x \cdot y) \cdot P
    \and \\
    x \cdot (P|Q) \equiv (x \cdot P) | (x \cdot Q)
    \and \\    
  \end{mathpar}
\end{remark}

\subsubsection{Tensor product}

We define a tensor product on processes by structural induction.

\paragraph{Tensor of sums} First note that all summations, including
$\pzero$ and sequence, can be written $\Sigma_{i} x_{i}.A_{i} +
\Sigma_{j} x_{j}.C_{j}$, where we have grouped input-guarded processes
together and output-guarded processes together.

Thus, we can define the tensor product of two summations, $N_{1}\otimes N_{2}$, where

\begin{mathpar}
  N_{1} := \Sigma_{i} x_{i}.A_{i} + \Sigma_{j} x_{j}.C_{j}
  \and
  N_{2} := \Sigma_{i'} y_{i'}.B_{i'} + \Sigma_{j'} y_{j'}.D_{j'} 
\end{mathpar}

as follows.

\begin{mathpar}
  \Sigma_{i} x_{i}.A_{i} + \Sigma_{j} x_{j}.C_{j} \otimes \Sigma_{i'}
  y_{i'}.B_{i'} + \Sigma_{j'} y_{j'}.D_{j'} 
  \and \\
  := \; \Sigma_{i} \Sigma_{i'} \quotep{\stackrel{\vee}{x_{i}}| \stackrel{\vee}{y_{i'}}}.(A_{i}\otimes B_{i'}) \; | \; \Sigma_{i'} \Sigma_{i} \quotep{\stackrel{\vee}{y_{i'}}|\stackrel{\vee}{x_{i}}}.(B_{i'}\otimes A_{i})
  \and
  \;\; | \;\; \Sigma_{j} \Sigma_{j'} \quotep{\stackrel{\vee}{x_{j}}|\stackrel{\vee}{y_{j'}}}.(A_{j}\otimes B_{j'}) \; | \; \Sigma_{j'} \Sigma_{j} \quotep{\stackrel{\vee}{y_{j'}}|\stackrel{\vee}{x_{j}}}.(B_{j'}\otimes A_{j})
\end{mathpar}

\begin{remark}
  Do we need to $x^{L}$ and $y^{R}$ for this construction as well?
\end{remark}

\paragraph{Tensor of parallel compositions} Next, we distribute tensor
over par.

\begin{mathpar}
  P_{1}|P_{2} \otimes Q_{1}|Q_{2} := (P_{1} \otimes Q_{1}) | (P_{1}
  \otimes Q_{2}) | (P_{2} \otimes Q_{1}) | (P_{2} \otimes Q_{2})
\end{mathpar}

\paragraph{Tensor with dropped names} We treat tensor of a
process with a dropped name as parallel composition.

\begin{mathpar}
  P \otimes \dropn{x} := P | \dropn{x}
\end{mathpar}

\paragraph{Tensor of agents}

Finally, we need to define tensor on agents. Note that the definition
of tensor on normal products only tensors inputs with inputs and
outputs with outputs. Thus, we only have to define the operation on
``homogeneous'' pairings.

\begin{mathpar}
  (\vec{x})P \otimes (\vec{y})Q
  \and \\
  := (x_{0}^{L}|y_{0}^{R},\ldots,x_{0}^{L}|y_{n}^{R},\ldots,x_{m}^{L}|y_{0}^{R},\ldots,x_{m}^{L}|y_{n}^R)(P\{ \vec{x}^{L}/\vec{x}\} \otimes Q \{ \vec{y}^{R}/\vec{y}\})
  \and \\
  \clift{\vec{P}} \otimes \clift{\vec{Q}}
  \and \\
  := \clift{P_{0}\otimes Q_{0},\ldots,P_{0}\otimes Q_{n},\ldots,P_{m}\otimes Q_{0},\ldots,P_{m}\otimes Q_{n}}
\end{mathpar}

\begin{remark}
  Observe that arities of tensored abstractions matches arities of
  tensored concretions if the original arities matched. Note also that
  the length of the arities corresponds to the increase in dimension
  we see in ordinary vector space tensor product.
\end{remark}

\begin{remark}
  Operationally, this definition distributes the tensor down to
  components ``linked'' by summation. Tensor over summation is
  intriguing in that it mixes names. Moreover, as a consequence of the
  way it mixes names we have the identities for all $x \in \QProc$ and
  $P,Q \in \Proc$

  \begin{mathpar}
    (x \cdot P) \otimes Q \equiv x \cdot (P \otimes Q) \equiv P \otimes (x \cdot Q)
    \and
    P \otimes \pzero \equiv P
  \end{mathpar}

  that the reader is invited to verify.
\end{remark}

\subsubsection{Annihilation}
\begin{mathpar}
  P^{\perp} := \{ Q | \forall R. P|Q \red^{*} R \Rightarrow R \red^{*} \pzero \}
  \and \\
  P^{\underline{\perp}} := \Sigma_{Q \in P^{\perp}} \quotep{Q}?(y).(\dropn{y}|Q) | \Sigma_{Q \in P^{\perp}} \quotep{Q}\clift{\Box}
\end{mathpar}

\paragraph{Discussion} The reader will note that $P^{\perp}$ is a
\emph{set} of processes, while $P^{\underline{\perp}}$ is a
\emph{context}. We call the set $P^{\perp}$ the \emph{annihilators} of
$P$. The parallel composition of a process in the annihilators of $P$
with $P$ will result in a process, the state space of which has all
paths eventually leading to $\pzero$. Execution may endure loops; but
under reasonable conditions of fairness (naturally guaranteed under
most notions of bisimulation) such a composite process cannot get
stuck in such a loop and will, eventually pop out and terminate.

The context $P^{\underline{\perp}}$ is ready and willing to ``take the
$P$ out of'' the process to which it is applied. It will effectively
transmit the code of the process to which it is applied to one of the
annihilators and run the process against it.

\subsubsection{Evaluation}
We fix $M$ a domain of fully abstract interpretation with an equality
coincident with bisimulation. We take $\meaningof{\cdot} : \Proc \to
M$ to be the map interpreting processes and $\nmeaningof{\cdot} : \M
\to Proc$ to be the map running the other way. Then we define

\begin{mathpar}
  \int P := \nmeaningof{\meaningof{P}}
\end{mathpar}

\paragraph{Discussion}
There are many fully abstract interpretations of Milner's
$\pi$-calculus. Any of them can be used as a basis for interpreting
the reflective calculus here. Equipped with such a domain it is
largely a matter of grinding through to check that the Yoneda
construction for the normalization-by-evaluation program can be
extended to this setting.

\begin{remark}
  The reader is invited to verify that $\int (P^{\underline{\perp}}[P]) = 0$.
\end{remark}

\subsection{Quantum mechanics}

Table \ref{tbl:core_qm_op_defns} gives the core operational definitions

\begin{table}[htp]\label{tbl:core_qm_op_defns}
  \center{
    \fbox{
      \begin{tabular}{c|c}
        quantum mechanics & process calculus \\
        \hline
        scalar & $x := \quotep{P}$ \\
        state vector & $\state{P} := P$ \\
        dual & $\state{P}^{*} := \event{P^{\underline{\perp}}} := \quotep{P^{\underline{\perp}}}[-]$ \\
        matrix & $ \Sigma_{\alpha} \state{P_{\alpha}}x_{\alpha}\event{Q_{\alpha}}$ \\
        vector addition & $\state{P} + \state{Q} := \state{P | Q}$ \\
        tensor product & $\state{P} \otimes \state{Q} := \state{P \otimes Q}$ \\
        inner product & $\innerprod{P}{Q} := \quotep{\int P^{\underline{\perp}}[Q]}$ \\
      \end{tabular}
    }
  }
  \caption{QM - operational definitions}
\end{table}

where

\begin{mathpar}
  \prmatrix{P}{Q} := \fprmatrix{P}{\quotep{\pzero}}{Q}
  \and
  \fprmatrix{P}{x}{Q} := (\state{P},x,\event{Q})
  \and
  (\fprmatrix{P}{x}{Q})(\state{R}) := x \cdot \innerprod{Q}{R} \cdot \state{P}
  \and
  (\fprmatrix{P}{x}{Q})(\event{R}) := x \cdot \innerprod{R}{P} \cdot \event{Q}
\end{mathpar}

\paragraph{Discussion}
As promised: vectors (aka states) are represented as processes; duals
as contextual duals; inner product definition should be compared with
standard inner product definition for ....

\begin{remark}
  Assuming $\int (P^{\underline{\perp}}[P]) = 0$, the reader is
  invited to verify that $(\fprmatrix{P}{x}{P})(\state{P}) = x \cdot \state{P}$.
\end{remark}

\begin{remark}
  The reader is invited to verify that $\innerprod{P}{Q}$ could
  equally well have been written $\quotep{\int \stackrel{\vee}{x}}$
  where $x = \event{P^{\underline{\perp}}}(Q)$.

  One of the motivations for this remark is that there is another way
  to factor these operations. We could package up evaluation in the dual:

  \begin{mathpar}
    \state{P}^{*} := \event{\int P^{\underline{\perp}}} := \quotep{\int P^{\underline{\perp}}}[-]
  \end{mathpar}

  and then have inner product defined by
  
  \begin{mathpar}
    \innerprod{P}{Q} := \event{P}(Q)
  \end{mathpar}

  Hopefully, experience with the calculations will provide guidance on
  the best factoring.
\end{remark}

\begin{remark}
  Assuming $\int (P^{\underline{\perp}}[P]) = 0$, the reader is
  invited to verify that $\forall P,Q. (\prmatrix{0}{Q})(\state{0}) =
  \state{0}$ and dually $(\prmatrix{P}{0})(\event{0}) = \event{0}$.
\end{remark}

\begin{remark}
  i'm a little worried that i don't (yet) have proper support for
  complex conjugacy. But, the observation above may give us a
  clue. According to Abramsky, it must be the case that the scalars
  are iso to the homset of the identity for the tensor -- which the
  observation above characterizes. 

  For now, we will simply bookmark the notion with $\overline{x}$.
\end{remark}

\subsubsection{Adjointness}

We need to give a definition of $(\cdot)^{\dagger}$ for matrices. The
obvious candidate definition is
\begin{mathpar}
(\Sigma_{\alpha}\fprmatrix{P_{\alpha}}{x_{\alpha}}{Q_{\alpha}})^{\dagger}
= \Sigma_{\alpha}\fprmatrix{(Q_{\alpha}^{\underline{\perp}})^{*}}{\overline{x}_{\alpha}}{P_{\alpha}^{\underline{\perp}}} 
\end{mathpar}

But, $(Q_{\alpha}^{\underline{\perp}})^{*}$ requires a name along
which to communicate the process to achieve the context application.

\subsubsection{Basis for a basis}
If processes label states and ``addition'' of states (a.k.a. vector
addition) is interpreted as parallel composition, what corresponds to
notions of linear independence and basis? Here, we recall that Yoshida
has developed a set of \emph{combinators} for an asynchronous verison
of Milner's $\pi$-calculus. These are a finite set of processes such
any process can be expressed as parallel composition of these
combinators together with liberal uses of the new operator and
replication. We can simply give a translation of these into the
present calculus and have reasonable expectation that the property
carries over. That is, that the resultant set allows to express all
processes via parallel composition. Note, however, that there is no
new operator or replication in this calculus. As a result, we expect
that the corresponding set is actually infinite. That is, we expect
that the space is actually infinite dimensional.

\begin{remark}
  The attentive reader may be a bit concerned. Certainly, the
  collection $S$, $K$ and $I$ is a finite set of
  combinators. Shouldn't we expect to see a finite set of combinators
  for an effectively equivalent system? i am very sympathetic to this
  critique and feel it warrants full attention. On the other hand, i
  also have in mind the following analogy. The natural numbers, as a
  monoid under addition, has exactly $1$ generator, while the natural
  numbers, as a monoid under multiplication, has countably many
  generators (the primes). We observe that the application of the
  lambda calculus is much less resource sensitive than the parallel
  composition of the $\pi$-calculus. Could it be the case that we have
  an analogy of the form
  
  \begin{mathpar}
    m + n : MN :: m*n : M|N
  \end{mathpar}

  giving a similar blow up in the set of ``primes''?  This is such a
  wonderful thought that, even if it's not true, i think it's worth
  writing down.
\end{remark}
 

\documentclass[12pt]{llncs}
%\documentclass{jktr}

\usepackage[pdftex]{hyperref}                   
\usepackage {listings}
\usepackage {mathpartir}
\usepackage{bcprules}
%\usepackage{listings}
                       
\usepackage{graphicx} 
%\usepackage[margins=2.5cm,nohead,nofoot]{geometry}
%\usepackage{geometry}
\usepackage{amsfonts}
\usepackage{amstext}
\usepackage{latexsym}
\usepackage{amssymb}
\usepackage{color}


%\include{myPreamble}
\include{qm2pi.local} 

%\ifpdf
%\usepackage[pdftex]{graphicx}
%\else
%\usepackage{graphicx}
%\fi

 % \ifpdf
%  \usepackage{pdfsync}
%  \if


%\title{Brief Article}
%\author{David F. Snyder}
%\author{L.G. Meredith}

%\address{Dept. of Math., Texas State University--San Marcos, San Marcos, TX 78666}
       
\pagestyle{empty}


\begin{document}

\lstset{language=[Objective]Caml,frame=shadowbox}

\input{qm2pi.front}

% section front matter (end)

\input{qm2pi.intro} 
 
% section introduction (end)

% \input{qm2pi.knotations} 

% section notation (end)

\input{qm2pi.process.calculi} 

% section concurrent_process_calculi_and_spatial_logics_ (end)
    
%\input{qm2pi.knots2pi} 

%\input{qm2pi.trefoil} 

%\input{qm2pi.mainthm} 

% subsection basic_interpretation (end)

%\input{qm2pi.rho.presentation} 
\subsection{The syntax and semantics of the notation system}\label{sub:the_syntax_and_semantics_of_the_notation_system} % (fold)

We now summarize a technical presentation of the calculus that
embodies our theory of dynamics. The typical presentation of such a
calculus follows the style of giving generators and relations on
them. The grammar, below, describing term constructors, freely
generates the set of processes, $\Proc$. This set is then quotiented
by a relation known as structural congruence and it is over this set
that the notion of dynamics is expressed. This presentation is
essentially that of \cite{MeredithR05} with the addition of
polyadicity and summation. For readability we have relegated some of
the technical subtleties to an appendix.

\subsubsection{Process grammar}\label{subsub:process_grammar}

\begin{mathpar}
  \inferrule* [lab=synchronization] {} {{M} \bc \pzero \;|\; x?F \;|\; x!C }
  \and
  \inferrule* [lab=abstraction] {} {{F} \bc (x)P}
  \and
  \inferrule* [lab=concretion] {} {{C} \bc \langle Q \rangle}
  \and
  \inferrule* [lab=process] {} {{P,Q} \bc M \;| \;P|Q \;|\; @{x}}
  \and
  \inferrule* [lab=name] {} {{x} \bc \quotep{P}}
\end{mathpar} 

Note that $\vec{x}$ (resp. $\vec{P}$) denotes a vector of names
(resp. processes) of length $|\vec{x}|$ (resp. $|\vec{P}|$). We adopt
the following useful abbreviations.

\begin{mathpar}
   x?(\vec{y}).P := x.(\vec{y})P \and  x\clift{\vec{P}} := x.\clift{\vec{P}}
   \and x!(y) := \lift{x}{\dropn{y}}
   \and \Pi_{i=0}^{n-1}P_i := P_0 | \ldots | P_{n-1}
\end{mathpar}

\subsubsection{Structural congruence}

\paragraph{Free and bound names and alpha-equivalence.} At the
core of structural equivalence is alpha-equivalence which identifies
process that are the same up to a change of variable. Formally, we
recognize the distinction between free and bound names. The free names
of a process, $\freenames{P}$, may be calculated recursively as
follows:

\begin{mathpar}
\freenames{\pzero} := \emptyset
  \and \\
  \freenames{x?(y).P} := \{ x \} \cup (\freenames{P} \setminus \{ y \})
  \and 
  \freenames{x!\langle P \rangle} := \{ x \} \cup \{ P \} 
  \and \\
  \freenames{P|Q} := \freenames{P} \cup \freenames{Q}
  \and \\
  \freenames{@{x}} := \{ x \}
\end{mathpar}

$\pi$
$\quotep{\pi}$

$\freenames{-} : \pi \to \mathcal{P}(\quotep{\pi})$

\begin{eqnarray*}
  \freenames{\pzero} & := & \emptyset \\
  \freenames{x?(y).P} & := & \{ x \} \cup (\freenames{P} \setminus \{ y \}) \\
  \freenames{x!\langle P \rangle} & := & \{ x \} \cup \{ P \} \\
  \freenames{P|Q} & := & \freenames{P} \cup \freenames{Q} \\
  \freenames{\dropn{x}} & := & \{ x \}
\end{eqnarray*}

The bound names of a process, $\boundnames{P}$, are those names occurring in $P$
that are not free. For example, in $x?(y).0$, the name $x$ is free, while $y$ is bound.

\begin{mathpar}
  \inferrule* [lab=monoidal-laws] {} { P|Q \equiv Q|P \and P|0 \equiv P \and P|(Q|R) \equiv (P|Q)|R }
\end{mathpar}

\begin{mathpar}
  \inferrule* [lab=alpha-equivalence] {} { (x)P \equiv (y)P\{y/x\} \and y \not\in \freenames{P} }
\end{mathpar}

\begin{definition}
Then two processes, $P,Q$, are alpha-equivalent if $P = Q\{\vec{y}/\vec{x}\}$ for
some $\vec{x} \in \boundnames{Q},\vec{y} \in \boundnames{P}$, where $Q\{\vec{y}/\vec{x}\}$
denotes the capture-avoiding substitution of $\vec{y}$ for $\vec{x}$ in $Q$.
\end{definition}

\begin{definition}
  The {\em structural congruence} \cite{SangiorgiWalker} , $\equiv$,
  between processes is the least congruence containing
  alpha-equivalence, satisfying the abelian monoid laws
  (associativity, commutativity and $\pzero$ as identity) for parallel
  composition $|$ and for summation $+$.
\end{definition}

\subsection{Name equivalence}

We take name equivalence, written $\nameeq$, to be the smallest
equivalence relation generated by the following rules.

\begin{mathpar}
\inferrule*[lab=Quote-drop]
{ }
{ \quotep{@{x}} \nameeq x }

\inferrule*[lab=Struct-equiv]
{ P \scong Q }
{ \quotep{P} \nameeq \quotep{Q} }
\end{mathpar}

The astute reader will have noticed that the mutual recursion of names
and processes imposes a mutual recursion on alpha-equivalence and
structural equivalence via name-equivalence. Fortunately, all of this
works out pleasantly and we may calculate in the natural way, free of
concern. The reader interested in the details is referred to the
appendix \ref{appendix:rho_details}.

\subsection{Substitution}

We use $\Proc$ for the set of processes, $\QProc$ for the set of
names, and $\id{\{}\vec{y} / \vec{x} \id{\}}$ to denote partial maps,
$s : \QProc \rightarrow \QProc$. A map, $s$ lifts, uniquely, to a map
on process terms, $\widehat{s} : \Proc \rightarrow \Proc$ by the
following equations.

\begin{mathpar}
  (0) \psubstp{Q}{P} := 0 \\
  (R \juxtap S) \psubstp{Q}{P}
  :=    
  (R)\psubstp{Q}{P} \juxtap (S) \psubstp{Q}{P} \\
  (x?(y).R) \psubstp{Q}{P}    
  :=    
  (x)\substp{Q}{P} (z)\concat( (R \psubstn{z}{y}) \psubstp{Q}{P} ) \\
  (\lift{x}{R}) \psubstp{Q}{P}  
  :=
  \lift{(x)\substp{Q}{P}}{ R \psubstp{Q}{P} } \\
%   (\dropn{x})  \psubstp{Q}{P}       
%   := 
%   \left\{ 
%     \begin{array}{ccc} 
%       \dropn{\quotep{Q}} & & x \nameeq \quotep{P} \\
%       \dropn{x} & & otherwise \\
%     \end{array}
%   \right. 
  (\dropn{x})  \psubstp{Q}{P}       
  := 
  \left\{ 
    \begin{array}{ccc} 
      Q & & x \nameeq \quotep{P} \\
      \dropn{x} & & otherwise \\
    \end{array}
  \right.
\end{mathpar}
 

where

\begin{eqnarray}
  (x)\id{\{} \lpquote Q \rpquote / \lpquote P \rpquote \id{\}}            = 
  \left\{ 
    \begin{array}{ccc}
      \lpquote Q \rpquote & & x \nameeq \lpquote P \rpquote \\
      x & & otherwise \\
    \end{array}
  \right. \nonumber
\end{eqnarray}

and $z$ is chosen distinct from $\quotep{P}$, $\quotep{Q}$, the free
names in $Q$, and all the names in $R$. Our $\alpha$-equivalence will
be built in the standard way from this substitution.

\begin{remark}\label{rem:no_self_referential_names}
  One consequence of these definitions is that $\forall P. \quotep{P}
  \not\in \freenames{P}$.
\end{remark}

\subsection{ Dynamic quote: an example }

Anticipating something of what's to come, consider applying the
substitution, $\widehat{\id{\{}u / z \id{\}}}$, to the following pair
of processes, $\lift{w}{y!(z)}$ and $w[ \lpquote y!(z) \rpquote ]$.

\begin{eqnarray}
	\lift{w}{y!(z)}\widehat{\id{\{}u / z \id{\}}}
		& = &
		\lift{w}{y!(u)} \nonumber\\
	w[ \lpquote y!(z) \rpquote ] \widehat{ \id{\{}u / z \id{\}} }
		& = &
		w[ \lpquote y!(z) \rpquote ] \nonumber
\end{eqnarray}

Because the body of the process between quotes is impervious to
substitution, we get radically different answers. In fact, by
examining the first process in an input context,
e.g. $x?(z).\lift{w}{y!(z)}$, we see that the process under the lift
operator may be shaped by prefixed inputs binding a name inside it. In
this sense, the lift operator will be seen as a way to dynamically
construct processes before reifying them as names.

Finally equipped with these standard features we can present the
dynamics of the calculus.

\subsubsection{Operational semantics} 

Finally, we introduce the computational dynamics. What marks these
algebras as distinct from other more traditionally studied algebraic
structures, e.g. vector spaces or polynomial rings, is the manner in
which dynamics is captured. In traditional structures, dynamics is typically
expressed through morphisms between such structures, as in linear maps
between vector spaces or morphisms between rings. In algebras
associated with the semantics of computation, the dynamics is
expressed as part of the algebraic structure itself, through a
reduction reduction relation typically denoted by $\red$. Below, we
give a recursive presentation of this relation for the calculus used
in the encoding.

$\red \subseteq \pi \times \pi$
$\red : \pi \to \mathcal{P}(\pi)$

\begin{mathpar}
  \inferrule* [lab=Comm] { \textsf{match}( x_{src}, x_{trgt} ) } { x_{trgt}?(y)P \; | \; x_{src}!\langle {Q} \rangle \red P\{\quotep{Q}/y}\} }
  \and \\
  \inferrule* [lab=Par] {{P} \red {P}'} {{{P} | {Q}} \red {{P}' | {Q}}}
  \and
  \inferrule* [lab=Equiv]{{{P} \scong {P}'} \andalso {{P}' \red {Q}'} \andalso {{Q}' \scong {Q}}}{{P} \red {Q}}
\end{mathpar}

\begin{eqnarray*}
  match_{\equiv} (\quotep{P},\quotep{Q}) & := & P \equiv Q \\
  match_{\dagger}(\quotep{P},\quotep{Q}) & := & \forall R. P|Q \red^{*} R => R \red^{*} 0 \\
  match_{K}(\quotep{P},\quotep{Q}) & := & K \mbox{ for some context } K
\end{eqnarray*}

$u?(x)P | u!\langle Q \rangle \red P\{\quotep{Q}/x\}$

%We write $\wred$ for $\red^*$, and $P\red$ if $\exists Q $ such that $ P \red Q$.
We write $P\red$ if $\exists Q $ such that $ P \red Q$ and $P\not\red$, otherwise.

\section{Replication}

As mentioned before, it is known that replication (and hence
recursion) can be implemented in a higher-order process algebra
\cite{SangiorgiWalker}. As our first example of calculation with the
machinery thus far presented we give the construction explicitly in
the {\rhoc}.

\begin{eqnarray}
	D_{x} & := & \prefix{x}{y}{(\binpar{\outputp{x}{y}}{@{y}})} \nonumber\\
	\bangp_{x}{P} & := & \binpar{{x}!\langle{\binpar{D_{x}}{P}}\rangle}{D_{x}} \nonumber
\end{eqnarray}

\begin{eqnarray}
	\bangp_{x}{P} & & \nonumber\\
	=
	& {x}!\langle{(\prefix{x}{y}{(\outputp{x}{y} | @{y})) | P}}\rangle 
	      | \prefix{x}{y}{(\outputp{x}{y} | @{y})} & \nonumber\\
	\red
	& (\outputp{x}{y} | @{y})\substn{\quotep{(\prefix{x}{y}{(@{y} | \outputp{x}{y})) | P}}}{y} & \nonumber\\
	=
	& \outputp{x}{\quotep{(\prefix{x}{y}{(\outputp{x}{y} | @{y})) | P}}}
	  | {(\prefix{x}{y}{(\outputp{x}{y} | @{y})) | P}} & \nonumber\\
	\red
	& \ldots & \nonumber\\
	\red^*
	& P | P | \ldots & \nonumber
\end{eqnarray}

Of course, this encoding, as an implementation, runs away, unfolding
$\bangp{P}$ eagerly. A lazier and more implementable replication
operator, restricted to input-guarded processes, may be obtained as follows.

\begin{eqnarray}
\bangp{\prefix{u}{v}{P}} 
	:= 
	\binpar{\lift{x}{\prefix{u}{v}{(\binpar{D(x)}{P})}}}{D(x)} \nonumber
\end{eqnarray}

\begin{remark}
  Note that the lazier definition still does not deal with summation
  or mixed summation (i.e. sums over input and output). The reader is
  invited to construct definitions of replication that deal with these
  features. 

  Further, the definitions are parameterized in a name, $x$. Can you,
  gentle reader, make a definition that eliminates this parameter and
  guarantees no accidental interaction between the replication
  machinery and the process being replicated -- i.e. no accidental
  sharing of names used by the process to get its work done and the
  name(s) used by the replication to effect copying. This latter
  revision of the definition of replication is crucial to obtaining
  the expected identity $!!P \sim !P$.
\end{remark}

\begin{remark}\label{rem:paradoxical_combinator}
  The reader familiar with the lambda calculus will have noticed the
  similarity between $D$ and the paradoxical combinator.

  [Ed. note: the existence of this seems to suggest we have to be more
  restrictive on the set of processes and names we admit if we are to
  support no-cloning.]
\end{remark}

\subsubsection{Bisimulation}

The computational dynamics gives rise to another kind of equivalence,
the equivalence of computational behavior. As previously mentioned
this is typically captured \emph{via} some form of bisimulation.

% The notion we use in this paper is weak barbed bisimulation
% \cite{milner91polyadicpi}.

The notion we use in this paper is derived from weak barbed
bisimulation \cite{milner91polyadicpi}. 

\begin{definition}
An \emph{observation relation}, $\downarrow_{\mathcal N}$, over a set
of names, $\mathcal N$, is the smallest relation satisfying the rules
below.

\infrule[Out-barb]{y \in {\mathcal N}, \; x \nameeq y}
		  {\outputp{x}{v} \downarrow_{\mathcal N} x}
\infrule[Par-barb]{\mbox{$P\downarrow_{\mathcal N} x$ or $Q\downarrow_{\mathcal N} x$}}
		  {\binpar{P}{Q} \downarrow_{\mathcal N} x}

We write $P \Downarrow_{\mathcal N} x$ if there is $Q$ such that 
$P \wred Q$ and $Q \downarrow_{\mathcal N} x$.
\end{definition}

\begin{definition}
%\label{def.bbisim}
An  ${\mathcal N}$-\emph{barbed bisimulation} over a set of names, ${\mathcal N}$, is a symmetric binary relation 
${\mathcal S}_{\mathcal N}$ between agents such that $P\rel{S}_{\mathcal N}Q$ implies:
\begin{enumerate}
\item If $P \red P'$ then $Q \wred Q'$ and $P'\rel{S}_{\mathcal N} Q'$.
\item If $P\downarrow_{\mathcal N} x$, then $Q\Downarrow_{\mathcal N} x$.
\end{enumerate}
$P$ is ${\mathcal N}$-barbed bisimilar to $Q$, written
$P \wbbisim_{\mathcal N} Q$, if $P \rel{S}_{\mathcal N} Q$ for some ${\mathcal N}$-barbed bisimulation ${\mathcal S}_{\mathcal N}$.
\end{definition}

$\mathcal{R} \subseteq \pi \times \pi$

$P \mathcal{R} Q => \forall P'. P \red P' \Rightarrow \exists Q'. Q \red Q', P' \mathcal{R} Q'$

$P \vdash x \Rightarrow Q \vdash x$

\begin{mathpar}
  \inferrule*[lab=Out-barb]{x \nameeq y}{{y}!\langle{Q}\rangle \vdash x}
  \and
  \inferrule*[lab=Par-barb]{\mbox{$P\vdash x$ or $Q\vdash x$}}{\binpar{P}{Q} \vdash x}
\end{mathpar}

\subsubsection{Contexts}

One of the principle advantages of computational calculi like the
$\pi$-calculus is a well-defined notion of context,
contextual-equivalence and a correlation between
contextual-equivalence and notions of bisimulation. The notion of
context allows the decomposition of a process into (sub-)process and
its syntactic environment, its context. Thus, a context may be
thought of as a process with a ``hole'' (written $\Box$) in it. The
application of a context $M$ to a process $P$, written $M[P]$, is
tantamount to filling the hole in $M$ with $P$. In this paper we do
not need the full weight of this theory, but do make use of the notion
of context in the proof the main theorem. 

\begin{mathpar}
  \inferrule* [lab=summation] {} {{M_{M},M_{N}} \bc \Box \;|\; x.M_{A} \;|\; M_{M}+M_{N}}
  \and
  \inferrule* [lab=agent] {} {{M_{A}} \bc (\vec{x})M_{P} \;| \; \clift{P_0,\ldots,M_{P},\ldots,P_N}}
  \and \\
  \inferrule* [lab=process] {} {{M_{P}} \bc M_{N} \;| \;P|M_{P} }
\end{mathpar} 

\begin{mathpar}
  \inferrule* [lab=sychronization] {} {M_{N} \bc \Box \;|\; x?M_{F} \;|\; x!M_{C}}
  \and
  \inferrule* [lab=abstraction] {} {{M_{F}} \bc (x)M_{P} }
  \and
  \inferrule* [lab=concretion] {} {{M_{C}} \bc \langle M_{P} \rangle }
  \and \\
  \inferrule* [lab=process] {} {{M_{P}} \bc M_{N} \;| \;P|M_{P} }
\end{mathpar}

\begin{definition}[contextual application] Given a context $M$, and
  process $P$, we define the \emph{contextual application}, $M[P] :=
  M\{P/\Box\}$. That is, the contextual application of M to P is the
  substitution of $P$ for $\Box$ in $M$.
\end{definition}

$\meaningof{-} : L \to \mathcal{P}(\pi)$

\begin{mathpar}
  \inferrule* [lab=collection] {} {\meaningof{true} = \pi, \and \meaningof{~E} = \pi \setminus \meaningof{E}, \and \meaningof{E_{1} \& E_{2}} = \meaningof{E_{1}} \cap \meaningof{E_{2}}}
\end{mathpar}

\begin{mathpar}
  \inferrule* [lab=structure] {} {\meaningof{0} = \{ P \in \pi | P \equiv 0 \}, \and \\ \meaningof{E_1 | E_2} = \{ P \in \pi | P \equiv P_{1} | P_{2}, P_{1} \in \meaningof{E_{1}}, P_{2} \in \meaningof{E_2}\} }
\end{mathpar}

\begin{mathpar}
 \inferrule* [lab=behavior] {} {\meaningof{\langle a?b \rangle E} = \{ P \in \pi | P \equiv Q | u?(y)P', \\ \and \\\\ \and \\ \;\;\; u \in \meaningof{a}, \forall z.P'\{z/y\} \in \meaningof{E\{z/b\}}\}, \and \\ \meaningof{a!E} = \{ P \in \pi | P \equiv Q | x!\langle P' \rangle, x \in \meaningof{a} P' \in \meaningof{E}\} }
\end{mathpar}

\begin{mathpar}
 \inferrule* [lab=nominal] {} {\meaningof{\quotep{E}} = \{ \quotep{P} \in \quotep{\pi} | P \in \meaningof{E} \}, \and \meaningof{\quotep{P}} = \{ \quotep{Q} \in \quotep{\pi} | P \equiv Q \} \and \\ \meaningof{@\quotep{E}} = \{ P \in \pi | P \equiv @x, x \in \meaningof{E} \}}
\end{mathpar}

\begin{eqnarray*}
  \\
  \meaningof{-} : TS \to ST
\end{eqnarray*}

\begin{eqnarray*}
  \\
  L : TS \to ST
\end{eqnarray*}

\begin{eqnarray*}
  \\
  P \models E \iff P \in \meaningof{E}
\end{eqnarray*}

\begin{eqnarray*}
  P \approx_{L} Q \iff \forall E \in L. P \models E \iff Q \models E
\end{eqnarray*}

\begin{eqnarray*}
  P \approx_{K} Q
\end{eqnarray*}

\begin{eqnarray*}
  P \approx Q
\end{eqnarray*}

$\approx_{K} = \approx = \approx_{L}$

\subsubsection{Contextual duality}

Note that contexts extend the quotation operation to a family of
operations from processes to names. Given a context, $M$, we can
define a \emph{nominal context}, $\quotep{M}$ by $\quotep{M}[P] :=
\quotep{M[P]}$. To foreshadow what is to come we observe that these
operations enjoy a duality with processes very much like the duality
between vectors and maps from vectors to scalars.

Further, because the calculus is essentially higher-order, we have a
correspondence between contexts and processes. More specifically,
given a name $x$ and a context $M$ we can construct $M^{*}_{x}$ such
that 

\begin{mathpar}
  M^{*}_{x} | \lift{x}{P} \red M[P]
\end{mathpar}

namely,

\begin{mathpar}
  M^{*}_{x} := x?(u).M[\dropn{u}]
\end{mathpar}

The dependence of $M^{*}_{x}$ on a name makes it an abstraction, 

\begin{mathpar}
  M^{*} := (x)x?(u).M[\dropn{u}]
\end{mathpar}

\subsection{Additional notation}

It will sometimes be convenient to denote the process a name
quotes. We already have the notation $x = \quotep{P}$, but it will be
convenient to introduce an alternate notation, $\procn{x}$, when we
want to emphasize the connection to the use of the name. Note that, by
virtue of name equivalence, $\quotep{\procn{x}} \nameeq x$; so, the
notation is consistent with previous definitions.

Further, because names have structure it is possible to effect
substitutions on the basis of that structure. This means we need to
upgrade our notation for substitutions, which we accomplish by
adapting comprehension notation. Thus,

\begin{mathpar}
  P\{ y / x : x \in S \}
\end{mathpar}

is interpreted to mean the process derived from P by replacing (in a
capture-avoiding manner) each occurrence of $x$ in $S$ by $y$. For example,

\begin{mathpar}
  P\{ \quotep{\procn{x}|\procn{x}} / x : x \in \freenames{P} \}
\end{mathpar}

will replace each (occurrence) of a free name $x$ in $P$ by
$\quotep{\procn{x}|\procn{x}}$.

Also, we will avail ourselves of the notation $x^{L}$ and $x^{R}$ to
denote injections of a name into disjoint copies of the name
space. There are numerous ways to accomplish this. One example can be
found in \cite{MeredithR05}. This notation overloads to vectors of
names: $\vec{x}^{\pi} := (x_{i}^{\pi} \; : \; 0 \leq i < |\vec{x}| )$ where $\pi \in \{L,R\}$.

We also use $P^{\Box} := P|\Box$.

In \cite{MeredithR05} an interpretation of the new operator is
given. It turns out that there are several possible interpretations
all enjoying the requisite algebraic properties of the operator (see
\cite{milner91polyadicpi}). We will therefore make liberal use of
$(\nu\; \vec{x})P$.

% subsection the_syntax_and_semantics_of_the_notation_system (end)   

\input{qm2pi.qmops} 

\input{qm2pi.sterngerlach} 

\input{qm2pi.metric} 

% section concurrent_process_calculi (end)

%\input{qm2pi.proofsketch}

% section proof sketch (end)

%\input{qm2pi.slviaknots} 

% section spatial logic via knots (end)

\input{qm2pi.conclusion}

% section conclusion (end)

%\input{qm2pi.dtcodes} 

% section wiring algorithm (end)

\input{qm2pi.ack} 

% section acknowledgments (end)

\newpage


\bibliographystyle{plain}   
\bibliography{../../biblios/main.bib}

\input{qm2pi.rhodetails}

\end{document}

 

\documentclass[12pt]{llncs}
%\documentclass{jktr}

\usepackage[pdftex]{hyperref}                   
\usepackage {listings}
\usepackage {mathpartir}
\usepackage{bcprules}
%\usepackage{listings}
                       
\usepackage{graphicx} 
%\usepackage[margins=2.5cm,nohead,nofoot]{geometry}
%\usepackage{geometry}
\usepackage{amsfonts}
\usepackage{amstext}
\usepackage{latexsym}
\usepackage{amssymb}
\usepackage{color}


%\include{myPreamble}
\include{qm2pi.local} 

%\ifpdf
%\usepackage[pdftex]{graphicx}
%\else
%\usepackage{graphicx}
%\fi

 % \ifpdf
%  \usepackage{pdfsync}
%  \if


%\title{Brief Article}
%\author{David F. Snyder}
%\author{L.G. Meredith}

%\address{Dept. of Math., Texas State University--San Marcos, San Marcos, TX 78666}
       
\pagestyle{empty}


\begin{document}

\lstset{language=[Objective]Caml,frame=shadowbox}

\input{qm2pi.front}

% section front matter (end)

\input{qm2pi.intro} 
 
% section introduction (end)

% \input{qm2pi.knotations} 

% section notation (end)

\input{qm2pi.process.calculi} 

% section concurrent_process_calculi_and_spatial_logics_ (end)
    
%\input{qm2pi.knots2pi} 

%\input{qm2pi.trefoil} 

%\input{qm2pi.mainthm} 

% subsection basic_interpretation (end)

%\input{qm2pi.rho.presentation} 
\subsection{The syntax and semantics of the notation system}\label{sub:the_syntax_and_semantics_of_the_notation_system} % (fold)

We now summarize a technical presentation of the calculus that
embodies our theory of dynamics. The typical presentation of such a
calculus follows the style of giving generators and relations on
them. The grammar, below, describing term constructors, freely
generates the set of processes, $\Proc$. This set is then quotiented
by a relation known as structural congruence and it is over this set
that the notion of dynamics is expressed. This presentation is
essentially that of \cite{MeredithR05} with the addition of
polyadicity and summation. For readability we have relegated some of
the technical subtleties to an appendix.

\subsubsection{Process grammar}\label{subsub:process_grammar}

\begin{mathpar}
  \inferrule* [lab=synchronization] {} {{M} \bc \pzero \;|\; x?F \;|\; x!C }
  \and
  \inferrule* [lab=abstraction] {} {{F} \bc (x)P}
  \and
  \inferrule* [lab=concretion] {} {{C} \bc \langle Q \rangle}
  \and
  \inferrule* [lab=process] {} {{P,Q} \bc M \;| \;P|Q \;|\; @{x}}
  \and
  \inferrule* [lab=name] {} {{x} \bc \quotep{P}}
\end{mathpar} 

Note that $\vec{x}$ (resp. $\vec{P}$) denotes a vector of names
(resp. processes) of length $|\vec{x}|$ (resp. $|\vec{P}|$). We adopt
the following useful abbreviations.

\begin{mathpar}
   x?(\vec{y}).P := x.(\vec{y})P \and  x\clift{\vec{P}} := x.\clift{\vec{P}}
   \and x!(y) := \lift{x}{\dropn{y}}
   \and \Pi_{i=0}^{n-1}P_i := P_0 | \ldots | P_{n-1}
\end{mathpar}

\subsubsection{Structural congruence}

\paragraph{Free and bound names and alpha-equivalence.} At the
core of structural equivalence is alpha-equivalence which identifies
process that are the same up to a change of variable. Formally, we
recognize the distinction between free and bound names. The free names
of a process, $\freenames{P}$, may be calculated recursively as
follows:

\begin{mathpar}
\freenames{\pzero} := \emptyset
  \and \\
  \freenames{x?(y).P} := \{ x \} \cup (\freenames{P} \setminus \{ y \})
  \and 
  \freenames{x!\langle P \rangle} := \{ x \} \cup \{ P \} 
  \and \\
  \freenames{P|Q} := \freenames{P} \cup \freenames{Q}
  \and \\
  \freenames{@{x}} := \{ x \}
\end{mathpar}

$\pi$
$\quotep{\pi}$

$\freenames{-} : \pi \to \mathcal{P}(\quotep{\pi})$

\begin{eqnarray*}
  \freenames{\pzero} & := & \emptyset \\
  \freenames{x?(y).P} & := & \{ x \} \cup (\freenames{P} \setminus \{ y \}) \\
  \freenames{x!\langle P \rangle} & := & \{ x \} \cup \{ P \} \\
  \freenames{P|Q} & := & \freenames{P} \cup \freenames{Q} \\
  \freenames{\dropn{x}} & := & \{ x \}
\end{eqnarray*}

The bound names of a process, $\boundnames{P}$, are those names occurring in $P$
that are not free. For example, in $x?(y).0$, the name $x$ is free, while $y$ is bound.

\begin{mathpar}
  \inferrule* [lab=monoidal-laws] {} { P|Q \equiv Q|P \and P|0 \equiv P \and P|(Q|R) \equiv (P|Q)|R }
\end{mathpar}

\begin{mathpar}
  \inferrule* [lab=alpha-equivalence] {} { (x)P \equiv (y)P\{y/x\} \and y \not\in \freenames{P} }
\end{mathpar}

\begin{definition}
Then two processes, $P,Q$, are alpha-equivalent if $P = Q\{\vec{y}/\vec{x}\}$ for
some $\vec{x} \in \boundnames{Q},\vec{y} \in \boundnames{P}$, where $Q\{\vec{y}/\vec{x}\}$
denotes the capture-avoiding substitution of $\vec{y}$ for $\vec{x}$ in $Q$.
\end{definition}

\begin{definition}
  The {\em structural congruence} \cite{SangiorgiWalker} , $\equiv$,
  between processes is the least congruence containing
  alpha-equivalence, satisfying the abelian monoid laws
  (associativity, commutativity and $\pzero$ as identity) for parallel
  composition $|$ and for summation $+$.
\end{definition}

\subsection{Name equivalence}

We take name equivalence, written $\nameeq$, to be the smallest
equivalence relation generated by the following rules.

\begin{mathpar}
\inferrule*[lab=Quote-drop]
{ }
{ \quotep{@{x}} \nameeq x }

\inferrule*[lab=Struct-equiv]
{ P \scong Q }
{ \quotep{P} \nameeq \quotep{Q} }
\end{mathpar}

The astute reader will have noticed that the mutual recursion of names
and processes imposes a mutual recursion on alpha-equivalence and
structural equivalence via name-equivalence. Fortunately, all of this
works out pleasantly and we may calculate in the natural way, free of
concern. The reader interested in the details is referred to the
appendix \ref{appendix:rho_details}.

\subsection{Substitution}

We use $\Proc$ for the set of processes, $\QProc$ for the set of
names, and $\id{\{}\vec{y} / \vec{x} \id{\}}$ to denote partial maps,
$s : \QProc \rightarrow \QProc$. A map, $s$ lifts, uniquely, to a map
on process terms, $\widehat{s} : \Proc \rightarrow \Proc$ by the
following equations.

\begin{mathpar}
  (0) \psubstp{Q}{P} := 0 \\
  (R \juxtap S) \psubstp{Q}{P}
  :=    
  (R)\psubstp{Q}{P} \juxtap (S) \psubstp{Q}{P} \\
  (x?(y).R) \psubstp{Q}{P}    
  :=    
  (x)\substp{Q}{P} (z)\concat( (R \psubstn{z}{y}) \psubstp{Q}{P} ) \\
  (\lift{x}{R}) \psubstp{Q}{P}  
  :=
  \lift{(x)\substp{Q}{P}}{ R \psubstp{Q}{P} } \\
%   (\dropn{x})  \psubstp{Q}{P}       
%   := 
%   \left\{ 
%     \begin{array}{ccc} 
%       \dropn{\quotep{Q}} & & x \nameeq \quotep{P} \\
%       \dropn{x} & & otherwise \\
%     \end{array}
%   \right. 
  (\dropn{x})  \psubstp{Q}{P}       
  := 
  \left\{ 
    \begin{array}{ccc} 
      Q & & x \nameeq \quotep{P} \\
      \dropn{x} & & otherwise \\
    \end{array}
  \right.
\end{mathpar}
 

where

\begin{eqnarray}
  (x)\id{\{} \lpquote Q \rpquote / \lpquote P \rpquote \id{\}}            = 
  \left\{ 
    \begin{array}{ccc}
      \lpquote Q \rpquote & & x \nameeq \lpquote P \rpquote \\
      x & & otherwise \\
    \end{array}
  \right. \nonumber
\end{eqnarray}

and $z$ is chosen distinct from $\quotep{P}$, $\quotep{Q}$, the free
names in $Q$, and all the names in $R$. Our $\alpha$-equivalence will
be built in the standard way from this substitution.

\begin{remark}\label{rem:no_self_referential_names}
  One consequence of these definitions is that $\forall P. \quotep{P}
  \not\in \freenames{P}$.
\end{remark}

\subsection{ Dynamic quote: an example }

Anticipating something of what's to come, consider applying the
substitution, $\widehat{\id{\{}u / z \id{\}}}$, to the following pair
of processes, $\lift{w}{y!(z)}$ and $w[ \lpquote y!(z) \rpquote ]$.

\begin{eqnarray}
	\lift{w}{y!(z)}\widehat{\id{\{}u / z \id{\}}}
		& = &
		\lift{w}{y!(u)} \nonumber\\
	w[ \lpquote y!(z) \rpquote ] \widehat{ \id{\{}u / z \id{\}} }
		& = &
		w[ \lpquote y!(z) \rpquote ] \nonumber
\end{eqnarray}

Because the body of the process between quotes is impervious to
substitution, we get radically different answers. In fact, by
examining the first process in an input context,
e.g. $x?(z).\lift{w}{y!(z)}$, we see that the process under the lift
operator may be shaped by prefixed inputs binding a name inside it. In
this sense, the lift operator will be seen as a way to dynamically
construct processes before reifying them as names.

Finally equipped with these standard features we can present the
dynamics of the calculus.

\subsubsection{Operational semantics} 

Finally, we introduce the computational dynamics. What marks these
algebras as distinct from other more traditionally studied algebraic
structures, e.g. vector spaces or polynomial rings, is the manner in
which dynamics is captured. In traditional structures, dynamics is typically
expressed through morphisms between such structures, as in linear maps
between vector spaces or morphisms between rings. In algebras
associated with the semantics of computation, the dynamics is
expressed as part of the algebraic structure itself, through a
reduction reduction relation typically denoted by $\red$. Below, we
give a recursive presentation of this relation for the calculus used
in the encoding.

$\red \subseteq \pi \times \pi$
$\red : \pi \to \mathcal{P}(\pi)$

\begin{mathpar}
  \inferrule* [lab=Comm] { \textsf{match}( x_{src}, x_{trgt} ) } { x_{trgt}?(y)P \; | \; x_{src}!\langle {Q} \rangle \red P\{\quotep{Q}/y}\} }
  \and \\
  \inferrule* [lab=Par] {{P} \red {P}'} {{{P} | {Q}} \red {{P}' | {Q}}}
  \and
  \inferrule* [lab=Equiv]{{{P} \scong {P}'} \andalso {{P}' \red {Q}'} \andalso {{Q}' \scong {Q}}}{{P} \red {Q}}
\end{mathpar}

\begin{eqnarray*}
  match_{\equiv} (\quotep{P},\quotep{Q}) & := & P \equiv Q \\
  match_{\dagger}(\quotep{P},\quotep{Q}) & := & \forall R. P|Q \red^{*} R => R \red^{*} 0 \\
  match_{K}(\quotep{P},\quotep{Q}) & := & K \mbox{ for some context } K
\end{eqnarray*}

$u?(x)P | u!\langle Q \rangle \red P\{\quotep{Q}/x\}$

%We write $\wred$ for $\red^*$, and $P\red$ if $\exists Q $ such that $ P \red Q$.
We write $P\red$ if $\exists Q $ such that $ P \red Q$ and $P\not\red$, otherwise.

\section{Replication}

As mentioned before, it is known that replication (and hence
recursion) can be implemented in a higher-order process algebra
\cite{SangiorgiWalker}. As our first example of calculation with the
machinery thus far presented we give the construction explicitly in
the {\rhoc}.

\begin{eqnarray}
	D_{x} & := & \prefix{x}{y}{(\binpar{\outputp{x}{y}}{@{y}})} \nonumber\\
	\bangp_{x}{P} & := & \binpar{{x}!\langle{\binpar{D_{x}}{P}}\rangle}{D_{x}} \nonumber
\end{eqnarray}

\begin{eqnarray}
	\bangp_{x}{P} & & \nonumber\\
	=
	& {x}!\langle{(\prefix{x}{y}{(\outputp{x}{y} | @{y})) | P}}\rangle 
	      | \prefix{x}{y}{(\outputp{x}{y} | @{y})} & \nonumber\\
	\red
	& (\outputp{x}{y} | @{y})\substn{\quotep{(\prefix{x}{y}{(@{y} | \outputp{x}{y})) | P}}}{y} & \nonumber\\
	=
	& \outputp{x}{\quotep{(\prefix{x}{y}{(\outputp{x}{y} | @{y})) | P}}}
	  | {(\prefix{x}{y}{(\outputp{x}{y} | @{y})) | P}} & \nonumber\\
	\red
	& \ldots & \nonumber\\
	\red^*
	& P | P | \ldots & \nonumber
\end{eqnarray}

Of course, this encoding, as an implementation, runs away, unfolding
$\bangp{P}$ eagerly. A lazier and more implementable replication
operator, restricted to input-guarded processes, may be obtained as follows.

\begin{eqnarray}
\bangp{\prefix{u}{v}{P}} 
	:= 
	\binpar{\lift{x}{\prefix{u}{v}{(\binpar{D(x)}{P})}}}{D(x)} \nonumber
\end{eqnarray}

\begin{remark}
  Note that the lazier definition still does not deal with summation
  or mixed summation (i.e. sums over input and output). The reader is
  invited to construct definitions of replication that deal with these
  features. 

  Further, the definitions are parameterized in a name, $x$. Can you,
  gentle reader, make a definition that eliminates this parameter and
  guarantees no accidental interaction between the replication
  machinery and the process being replicated -- i.e. no accidental
  sharing of names used by the process to get its work done and the
  name(s) used by the replication to effect copying. This latter
  revision of the definition of replication is crucial to obtaining
  the expected identity $!!P \sim !P$.
\end{remark}

\begin{remark}\label{rem:paradoxical_combinator}
  The reader familiar with the lambda calculus will have noticed the
  similarity between $D$ and the paradoxical combinator.

  [Ed. note: the existence of this seems to suggest we have to be more
  restrictive on the set of processes and names we admit if we are to
  support no-cloning.]
\end{remark}

\subsubsection{Bisimulation}

The computational dynamics gives rise to another kind of equivalence,
the equivalence of computational behavior. As previously mentioned
this is typically captured \emph{via} some form of bisimulation.

% The notion we use in this paper is weak barbed bisimulation
% \cite{milner91polyadicpi}.

The notion we use in this paper is derived from weak barbed
bisimulation \cite{milner91polyadicpi}. 

\begin{definition}
An \emph{observation relation}, $\downarrow_{\mathcal N}$, over a set
of names, $\mathcal N$, is the smallest relation satisfying the rules
below.

\infrule[Out-barb]{y \in {\mathcal N}, \; x \nameeq y}
		  {\outputp{x}{v} \downarrow_{\mathcal N} x}
\infrule[Par-barb]{\mbox{$P\downarrow_{\mathcal N} x$ or $Q\downarrow_{\mathcal N} x$}}
		  {\binpar{P}{Q} \downarrow_{\mathcal N} x}

We write $P \Downarrow_{\mathcal N} x$ if there is $Q$ such that 
$P \wred Q$ and $Q \downarrow_{\mathcal N} x$.
\end{definition}

\begin{definition}
%\label{def.bbisim}
An  ${\mathcal N}$-\emph{barbed bisimulation} over a set of names, ${\mathcal N}$, is a symmetric binary relation 
${\mathcal S}_{\mathcal N}$ between agents such that $P\rel{S}_{\mathcal N}Q$ implies:
\begin{enumerate}
\item If $P \red P'$ then $Q \wred Q'$ and $P'\rel{S}_{\mathcal N} Q'$.
\item If $P\downarrow_{\mathcal N} x$, then $Q\Downarrow_{\mathcal N} x$.
\end{enumerate}
$P$ is ${\mathcal N}$-barbed bisimilar to $Q$, written
$P \wbbisim_{\mathcal N} Q$, if $P \rel{S}_{\mathcal N} Q$ for some ${\mathcal N}$-barbed bisimulation ${\mathcal S}_{\mathcal N}$.
\end{definition}

$\mathcal{R} \subseteq \pi \times \pi$

$P \mathcal{R} Q => \forall P'. P \red P' \Rightarrow \exists Q'. Q \red Q', P' \mathcal{R} Q'$

$P \vdash x \Rightarrow Q \vdash x$

\begin{mathpar}
  \inferrule*[lab=Out-barb]{x \nameeq y}{{y}!\langle{Q}\rangle \vdash x}
  \and
  \inferrule*[lab=Par-barb]{\mbox{$P\vdash x$ or $Q\vdash x$}}{\binpar{P}{Q} \vdash x}
\end{mathpar}

\subsubsection{Contexts}

One of the principle advantages of computational calculi like the
$\pi$-calculus is a well-defined notion of context,
contextual-equivalence and a correlation between
contextual-equivalence and notions of bisimulation. The notion of
context allows the decomposition of a process into (sub-)process and
its syntactic environment, its context. Thus, a context may be
thought of as a process with a ``hole'' (written $\Box$) in it. The
application of a context $M$ to a process $P$, written $M[P]$, is
tantamount to filling the hole in $M$ with $P$. In this paper we do
not need the full weight of this theory, but do make use of the notion
of context in the proof the main theorem. 

\begin{mathpar}
  \inferrule* [lab=summation] {} {{M_{M},M_{N}} \bc \Box \;|\; x.M_{A} \;|\; M_{M}+M_{N}}
  \and
  \inferrule* [lab=agent] {} {{M_{A}} \bc (\vec{x})M_{P} \;| \; \clift{P_0,\ldots,M_{P},\ldots,P_N}}
  \and \\
  \inferrule* [lab=process] {} {{M_{P}} \bc M_{N} \;| \;P|M_{P} }
\end{mathpar} 

\begin{mathpar}
  \inferrule* [lab=sychronization] {} {M_{N} \bc \Box \;|\; x?M_{F} \;|\; x!M_{C}}
  \and
  \inferrule* [lab=abstraction] {} {{M_{F}} \bc (x)M_{P} }
  \and
  \inferrule* [lab=concretion] {} {{M_{C}} \bc \langle M_{P} \rangle }
  \and \\
  \inferrule* [lab=process] {} {{M_{P}} \bc M_{N} \;| \;P|M_{P} }
\end{mathpar}

\begin{definition}[contextual application] Given a context $M$, and
  process $P$, we define the \emph{contextual application}, $M[P] :=
  M\{P/\Box\}$. That is, the contextual application of M to P is the
  substitution of $P$ for $\Box$ in $M$.
\end{definition}

$\meaningof{-} : L \to \mathcal{P}(\pi)$

\begin{mathpar}
  \inferrule* [lab=collection] {} {\meaningof{true} = \pi, \and \meaningof{~E} = \pi \setminus \meaningof{E}, \and \meaningof{E_{1} \& E_{2}} = \meaningof{E_{1}} \cap \meaningof{E_{2}}}
\end{mathpar}

\begin{mathpar}
  \inferrule* [lab=structure] {} {\meaningof{0} = \{ P \in \pi | P \equiv 0 \}, \and \\ \meaningof{E_1 | E_2} = \{ P \in \pi | P \equiv P_{1} | P_{2}, P_{1} \in \meaningof{E_{1}}, P_{2} \in \meaningof{E_2}\} }
\end{mathpar}

\begin{mathpar}
 \inferrule* [lab=behavior] {} {\meaningof{\langle a?b \rangle E} = \{ P \in \pi | P \equiv Q | u?(y)P', \\ \and \\\\ \and \\ \;\;\; u \in \meaningof{a}, \forall z.P'\{z/y\} \in \meaningof{E\{z/b\}}\}, \and \\ \meaningof{a!E} = \{ P \in \pi | P \equiv Q | x!\langle P' \rangle, x \in \meaningof{a} P' \in \meaningof{E}\} }
\end{mathpar}

\begin{mathpar}
 \inferrule* [lab=nominal] {} {\meaningof{\quotep{E}} = \{ \quotep{P} \in \quotep{\pi} | P \in \meaningof{E} \}, \and \meaningof{\quotep{P}} = \{ \quotep{Q} \in \quotep{\pi} | P \equiv Q \} \and \\ \meaningof{@\quotep{E}} = \{ P \in \pi | P \equiv @x, x \in \meaningof{E} \}}
\end{mathpar}

\begin{eqnarray*}
  \\
  \meaningof{-} : TS \to ST
\end{eqnarray*}

\begin{eqnarray*}
  \\
  L : TS \to ST
\end{eqnarray*}

\begin{eqnarray*}
  \\
  P \models E \iff P \in \meaningof{E}
\end{eqnarray*}

\begin{eqnarray*}
  P \approx_{L} Q \iff \forall E \in L. P \models E \iff Q \models E
\end{eqnarray*}

\begin{eqnarray*}
  P \approx_{K} Q
\end{eqnarray*}

\begin{eqnarray*}
  P \approx Q
\end{eqnarray*}

$\approx_{K} = \approx = \approx_{L}$

\subsubsection{Contextual duality}

Note that contexts extend the quotation operation to a family of
operations from processes to names. Given a context, $M$, we can
define a \emph{nominal context}, $\quotep{M}$ by $\quotep{M}[P] :=
\quotep{M[P]}$. To foreshadow what is to come we observe that these
operations enjoy a duality with processes very much like the duality
between vectors and maps from vectors to scalars.

Further, because the calculus is essentially higher-order, we have a
correspondence between contexts and processes. More specifically,
given a name $x$ and a context $M$ we can construct $M^{*}_{x}$ such
that 

\begin{mathpar}
  M^{*}_{x} | \lift{x}{P} \red M[P]
\end{mathpar}

namely,

\begin{mathpar}
  M^{*}_{x} := x?(u).M[\dropn{u}]
\end{mathpar}

The dependence of $M^{*}_{x}$ on a name makes it an abstraction, 

\begin{mathpar}
  M^{*} := (x)x?(u).M[\dropn{u}]
\end{mathpar}

\subsection{Additional notation}

It will sometimes be convenient to denote the process a name
quotes. We already have the notation $x = \quotep{P}$, but it will be
convenient to introduce an alternate notation, $\procn{x}$, when we
want to emphasize the connection to the use of the name. Note that, by
virtue of name equivalence, $\quotep{\procn{x}} \nameeq x$; so, the
notation is consistent with previous definitions.

Further, because names have structure it is possible to effect
substitutions on the basis of that structure. This means we need to
upgrade our notation for substitutions, which we accomplish by
adapting comprehension notation. Thus,

\begin{mathpar}
  P\{ y / x : x \in S \}
\end{mathpar}

is interpreted to mean the process derived from P by replacing (in a
capture-avoiding manner) each occurrence of $x$ in $S$ by $y$. For example,

\begin{mathpar}
  P\{ \quotep{\procn{x}|\procn{x}} / x : x \in \freenames{P} \}
\end{mathpar}

will replace each (occurrence) of a free name $x$ in $P$ by
$\quotep{\procn{x}|\procn{x}}$.

Also, we will avail ourselves of the notation $x^{L}$ and $x^{R}$ to
denote injections of a name into disjoint copies of the name
space. There are numerous ways to accomplish this. One example can be
found in \cite{MeredithR05}. This notation overloads to vectors of
names: $\vec{x}^{\pi} := (x_{i}^{\pi} \; : \; 0 \leq i < |\vec{x}| )$ where $\pi \in \{L,R\}$.

We also use $P^{\Box} := P|\Box$.

In \cite{MeredithR05} an interpretation of the new operator is
given. It turns out that there are several possible interpretations
all enjoying the requisite algebraic properties of the operator (see
\cite{milner91polyadicpi}). We will therefore make liberal use of
$(\nu\; \vec{x})P$.

% subsection the_syntax_and_semantics_of_the_notation_system (end)   

\input{qm2pi.qmops} 

\input{qm2pi.sterngerlach} 

\input{qm2pi.metric} 

% section concurrent_process_calculi (end)

%\input{qm2pi.proofsketch}

% section proof sketch (end)

%\input{qm2pi.slviaknots} 

% section spatial logic via knots (end)

\input{qm2pi.conclusion}

% section conclusion (end)

%\input{qm2pi.dtcodes} 

% section wiring algorithm (end)

\input{qm2pi.ack} 

% section acknowledgments (end)

\newpage


\bibliographystyle{plain}   
\bibliography{../../biblios/main.bib}

\input{qm2pi.rhodetails}

\end{document}

 

% section concurrent_process_calculi (end)

%\documentclass[12pt]{llncs}
%\documentclass{jktr}

\usepackage[pdftex]{hyperref}                   
\usepackage {listings}
\usepackage {mathpartir}
\usepackage{bcprules}
%\usepackage{listings}
                       
\usepackage{graphicx} 
%\usepackage[margins=2.5cm,nohead,nofoot]{geometry}
%\usepackage{geometry}
\usepackage{amsfonts}
\usepackage{amstext}
\usepackage{latexsym}
\usepackage{amssymb}
\usepackage{color}


%\include{myPreamble}
\include{qm2pi.local} 

%\ifpdf
%\usepackage[pdftex]{graphicx}
%\else
%\usepackage{graphicx}
%\fi

 % \ifpdf
%  \usepackage{pdfsync}
%  \if


%\title{Brief Article}
%\author{David F. Snyder}
%\author{L.G. Meredith}

%\address{Dept. of Math., Texas State University--San Marcos, San Marcos, TX 78666}
       
\pagestyle{empty}


\begin{document}

\lstset{language=[Objective]Caml,frame=shadowbox}

\input{qm2pi.front}

% section front matter (end)

\input{qm2pi.intro} 
 
% section introduction (end)

% \input{qm2pi.knotations} 

% section notation (end)

\input{qm2pi.process.calculi} 

% section concurrent_process_calculi_and_spatial_logics_ (end)
    
%\input{qm2pi.knots2pi} 

%\input{qm2pi.trefoil} 

%\input{qm2pi.mainthm} 

% subsection basic_interpretation (end)

%\input{qm2pi.rho.presentation} 
\subsection{The syntax and semantics of the notation system}\label{sub:the_syntax_and_semantics_of_the_notation_system} % (fold)

We now summarize a technical presentation of the calculus that
embodies our theory of dynamics. The typical presentation of such a
calculus follows the style of giving generators and relations on
them. The grammar, below, describing term constructors, freely
generates the set of processes, $\Proc$. This set is then quotiented
by a relation known as structural congruence and it is over this set
that the notion of dynamics is expressed. This presentation is
essentially that of \cite{MeredithR05} with the addition of
polyadicity and summation. For readability we have relegated some of
the technical subtleties to an appendix.

\subsubsection{Process grammar}\label{subsub:process_grammar}

\begin{mathpar}
  \inferrule* [lab=synchronization] {} {{M} \bc \pzero \;|\; x?F \;|\; x!C }
  \and
  \inferrule* [lab=abstraction] {} {{F} \bc (x)P}
  \and
  \inferrule* [lab=concretion] {} {{C} \bc \langle Q \rangle}
  \and
  \inferrule* [lab=process] {} {{P,Q} \bc M \;| \;P|Q \;|\; @{x}}
  \and
  \inferrule* [lab=name] {} {{x} \bc \quotep{P}}
\end{mathpar} 

Note that $\vec{x}$ (resp. $\vec{P}$) denotes a vector of names
(resp. processes) of length $|\vec{x}|$ (resp. $|\vec{P}|$). We adopt
the following useful abbreviations.

\begin{mathpar}
   x?(\vec{y}).P := x.(\vec{y})P \and  x\clift{\vec{P}} := x.\clift{\vec{P}}
   \and x!(y) := \lift{x}{\dropn{y}}
   \and \Pi_{i=0}^{n-1}P_i := P_0 | \ldots | P_{n-1}
\end{mathpar}

\subsubsection{Structural congruence}

\paragraph{Free and bound names and alpha-equivalence.} At the
core of structural equivalence is alpha-equivalence which identifies
process that are the same up to a change of variable. Formally, we
recognize the distinction between free and bound names. The free names
of a process, $\freenames{P}$, may be calculated recursively as
follows:

\begin{mathpar}
\freenames{\pzero} := \emptyset
  \and \\
  \freenames{x?(y).P} := \{ x \} \cup (\freenames{P} \setminus \{ y \})
  \and 
  \freenames{x!\langle P \rangle} := \{ x \} \cup \{ P \} 
  \and \\
  \freenames{P|Q} := \freenames{P} \cup \freenames{Q}
  \and \\
  \freenames{@{x}} := \{ x \}
\end{mathpar}

$\pi$
$\quotep{\pi}$

$\freenames{-} : \pi \to \mathcal{P}(\quotep{\pi})$

\begin{eqnarray*}
  \freenames{\pzero} & := & \emptyset \\
  \freenames{x?(y).P} & := & \{ x \} \cup (\freenames{P} \setminus \{ y \}) \\
  \freenames{x!\langle P \rangle} & := & \{ x \} \cup \{ P \} \\
  \freenames{P|Q} & := & \freenames{P} \cup \freenames{Q} \\
  \freenames{\dropn{x}} & := & \{ x \}
\end{eqnarray*}

The bound names of a process, $\boundnames{P}$, are those names occurring in $P$
that are not free. For example, in $x?(y).0$, the name $x$ is free, while $y$ is bound.

\begin{mathpar}
  \inferrule* [lab=monoidal-laws] {} { P|Q \equiv Q|P \and P|0 \equiv P \and P|(Q|R) \equiv (P|Q)|R }
\end{mathpar}

\begin{mathpar}
  \inferrule* [lab=alpha-equivalence] {} { (x)P \equiv (y)P\{y/x\} \and y \not\in \freenames{P} }
\end{mathpar}

\begin{definition}
Then two processes, $P,Q$, are alpha-equivalent if $P = Q\{\vec{y}/\vec{x}\}$ for
some $\vec{x} \in \boundnames{Q},\vec{y} \in \boundnames{P}$, where $Q\{\vec{y}/\vec{x}\}$
denotes the capture-avoiding substitution of $\vec{y}$ for $\vec{x}$ in $Q$.
\end{definition}

\begin{definition}
  The {\em structural congruence} \cite{SangiorgiWalker} , $\equiv$,
  between processes is the least congruence containing
  alpha-equivalence, satisfying the abelian monoid laws
  (associativity, commutativity and $\pzero$ as identity) for parallel
  composition $|$ and for summation $+$.
\end{definition}

\subsection{Name equivalence}

We take name equivalence, written $\nameeq$, to be the smallest
equivalence relation generated by the following rules.

\begin{mathpar}
\inferrule*[lab=Quote-drop]
{ }
{ \quotep{@{x}} \nameeq x }

\inferrule*[lab=Struct-equiv]
{ P \scong Q }
{ \quotep{P} \nameeq \quotep{Q} }
\end{mathpar}

The astute reader will have noticed that the mutual recursion of names
and processes imposes a mutual recursion on alpha-equivalence and
structural equivalence via name-equivalence. Fortunately, all of this
works out pleasantly and we may calculate in the natural way, free of
concern. The reader interested in the details is referred to the
appendix \ref{appendix:rho_details}.

\subsection{Substitution}

We use $\Proc$ for the set of processes, $\QProc$ for the set of
names, and $\id{\{}\vec{y} / \vec{x} \id{\}}$ to denote partial maps,
$s : \QProc \rightarrow \QProc$. A map, $s$ lifts, uniquely, to a map
on process terms, $\widehat{s} : \Proc \rightarrow \Proc$ by the
following equations.

\begin{mathpar}
  (0) \psubstp{Q}{P} := 0 \\
  (R \juxtap S) \psubstp{Q}{P}
  :=    
  (R)\psubstp{Q}{P} \juxtap (S) \psubstp{Q}{P} \\
  (x?(y).R) \psubstp{Q}{P}    
  :=    
  (x)\substp{Q}{P} (z)\concat( (R \psubstn{z}{y}) \psubstp{Q}{P} ) \\
  (\lift{x}{R}) \psubstp{Q}{P}  
  :=
  \lift{(x)\substp{Q}{P}}{ R \psubstp{Q}{P} } \\
%   (\dropn{x})  \psubstp{Q}{P}       
%   := 
%   \left\{ 
%     \begin{array}{ccc} 
%       \dropn{\quotep{Q}} & & x \nameeq \quotep{P} \\
%       \dropn{x} & & otherwise \\
%     \end{array}
%   \right. 
  (\dropn{x})  \psubstp{Q}{P}       
  := 
  \left\{ 
    \begin{array}{ccc} 
      Q & & x \nameeq \quotep{P} \\
      \dropn{x} & & otherwise \\
    \end{array}
  \right.
\end{mathpar}
 

where

\begin{eqnarray}
  (x)\id{\{} \lpquote Q \rpquote / \lpquote P \rpquote \id{\}}            = 
  \left\{ 
    \begin{array}{ccc}
      \lpquote Q \rpquote & & x \nameeq \lpquote P \rpquote \\
      x & & otherwise \\
    \end{array}
  \right. \nonumber
\end{eqnarray}

and $z$ is chosen distinct from $\quotep{P}$, $\quotep{Q}$, the free
names in $Q$, and all the names in $R$. Our $\alpha$-equivalence will
be built in the standard way from this substitution.

\begin{remark}\label{rem:no_self_referential_names}
  One consequence of these definitions is that $\forall P. \quotep{P}
  \not\in \freenames{P}$.
\end{remark}

\subsection{ Dynamic quote: an example }

Anticipating something of what's to come, consider applying the
substitution, $\widehat{\id{\{}u / z \id{\}}}$, to the following pair
of processes, $\lift{w}{y!(z)}$ and $w[ \lpquote y!(z) \rpquote ]$.

\begin{eqnarray}
	\lift{w}{y!(z)}\widehat{\id{\{}u / z \id{\}}}
		& = &
		\lift{w}{y!(u)} \nonumber\\
	w[ \lpquote y!(z) \rpquote ] \widehat{ \id{\{}u / z \id{\}} }
		& = &
		w[ \lpquote y!(z) \rpquote ] \nonumber
\end{eqnarray}

Because the body of the process between quotes is impervious to
substitution, we get radically different answers. In fact, by
examining the first process in an input context,
e.g. $x?(z).\lift{w}{y!(z)}$, we see that the process under the lift
operator may be shaped by prefixed inputs binding a name inside it. In
this sense, the lift operator will be seen as a way to dynamically
construct processes before reifying them as names.

Finally equipped with these standard features we can present the
dynamics of the calculus.

\subsubsection{Operational semantics} 

Finally, we introduce the computational dynamics. What marks these
algebras as distinct from other more traditionally studied algebraic
structures, e.g. vector spaces or polynomial rings, is the manner in
which dynamics is captured. In traditional structures, dynamics is typically
expressed through morphisms between such structures, as in linear maps
between vector spaces or morphisms between rings. In algebras
associated with the semantics of computation, the dynamics is
expressed as part of the algebraic structure itself, through a
reduction reduction relation typically denoted by $\red$. Below, we
give a recursive presentation of this relation for the calculus used
in the encoding.

$\red \subseteq \pi \times \pi$
$\red : \pi \to \mathcal{P}(\pi)$

\begin{mathpar}
  \inferrule* [lab=Comm] { \textsf{match}( x_{src}, x_{trgt} ) } { x_{trgt}?(y)P \; | \; x_{src}!\langle {Q} \rangle \red P\{\quotep{Q}/y}\} }
  \and \\
  \inferrule* [lab=Par] {{P} \red {P}'} {{{P} | {Q}} \red {{P}' | {Q}}}
  \and
  \inferrule* [lab=Equiv]{{{P} \scong {P}'} \andalso {{P}' \red {Q}'} \andalso {{Q}' \scong {Q}}}{{P} \red {Q}}
\end{mathpar}

\begin{eqnarray*}
  match_{\equiv} (\quotep{P},\quotep{Q}) & := & P \equiv Q \\
  match_{\dagger}(\quotep{P},\quotep{Q}) & := & \forall R. P|Q \red^{*} R => R \red^{*} 0 \\
  match_{K}(\quotep{P},\quotep{Q}) & := & K \mbox{ for some context } K
\end{eqnarray*}

$u?(x)P | u!\langle Q \rangle \red P\{\quotep{Q}/x\}$

%We write $\wred$ for $\red^*$, and $P\red$ if $\exists Q $ such that $ P \red Q$.
We write $P\red$ if $\exists Q $ such that $ P \red Q$ and $P\not\red$, otherwise.

\section{Replication}

As mentioned before, it is known that replication (and hence
recursion) can be implemented in a higher-order process algebra
\cite{SangiorgiWalker}. As our first example of calculation with the
machinery thus far presented we give the construction explicitly in
the {\rhoc}.

\begin{eqnarray}
	D_{x} & := & \prefix{x}{y}{(\binpar{\outputp{x}{y}}{@{y}})} \nonumber\\
	\bangp_{x}{P} & := & \binpar{{x}!\langle{\binpar{D_{x}}{P}}\rangle}{D_{x}} \nonumber
\end{eqnarray}

\begin{eqnarray}
	\bangp_{x}{P} & & \nonumber\\
	=
	& {x}!\langle{(\prefix{x}{y}{(\outputp{x}{y} | @{y})) | P}}\rangle 
	      | \prefix{x}{y}{(\outputp{x}{y} | @{y})} & \nonumber\\
	\red
	& (\outputp{x}{y} | @{y})\substn{\quotep{(\prefix{x}{y}{(@{y} | \outputp{x}{y})) | P}}}{y} & \nonumber\\
	=
	& \outputp{x}{\quotep{(\prefix{x}{y}{(\outputp{x}{y} | @{y})) | P}}}
	  | {(\prefix{x}{y}{(\outputp{x}{y} | @{y})) | P}} & \nonumber\\
	\red
	& \ldots & \nonumber\\
	\red^*
	& P | P | \ldots & \nonumber
\end{eqnarray}

Of course, this encoding, as an implementation, runs away, unfolding
$\bangp{P}$ eagerly. A lazier and more implementable replication
operator, restricted to input-guarded processes, may be obtained as follows.

\begin{eqnarray}
\bangp{\prefix{u}{v}{P}} 
	:= 
	\binpar{\lift{x}{\prefix{u}{v}{(\binpar{D(x)}{P})}}}{D(x)} \nonumber
\end{eqnarray}

\begin{remark}
  Note that the lazier definition still does not deal with summation
  or mixed summation (i.e. sums over input and output). The reader is
  invited to construct definitions of replication that deal with these
  features. 

  Further, the definitions are parameterized in a name, $x$. Can you,
  gentle reader, make a definition that eliminates this parameter and
  guarantees no accidental interaction between the replication
  machinery and the process being replicated -- i.e. no accidental
  sharing of names used by the process to get its work done and the
  name(s) used by the replication to effect copying. This latter
  revision of the definition of replication is crucial to obtaining
  the expected identity $!!P \sim !P$.
\end{remark}

\begin{remark}\label{rem:paradoxical_combinator}
  The reader familiar with the lambda calculus will have noticed the
  similarity between $D$ and the paradoxical combinator.

  [Ed. note: the existence of this seems to suggest we have to be more
  restrictive on the set of processes and names we admit if we are to
  support no-cloning.]
\end{remark}

\subsubsection{Bisimulation}

The computational dynamics gives rise to another kind of equivalence,
the equivalence of computational behavior. As previously mentioned
this is typically captured \emph{via} some form of bisimulation.

% The notion we use in this paper is weak barbed bisimulation
% \cite{milner91polyadicpi}.

The notion we use in this paper is derived from weak barbed
bisimulation \cite{milner91polyadicpi}. 

\begin{definition}
An \emph{observation relation}, $\downarrow_{\mathcal N}$, over a set
of names, $\mathcal N$, is the smallest relation satisfying the rules
below.

\infrule[Out-barb]{y \in {\mathcal N}, \; x \nameeq y}
		  {\outputp{x}{v} \downarrow_{\mathcal N} x}
\infrule[Par-barb]{\mbox{$P\downarrow_{\mathcal N} x$ or $Q\downarrow_{\mathcal N} x$}}
		  {\binpar{P}{Q} \downarrow_{\mathcal N} x}

We write $P \Downarrow_{\mathcal N} x$ if there is $Q$ such that 
$P \wred Q$ and $Q \downarrow_{\mathcal N} x$.
\end{definition}

\begin{definition}
%\label{def.bbisim}
An  ${\mathcal N}$-\emph{barbed bisimulation} over a set of names, ${\mathcal N}$, is a symmetric binary relation 
${\mathcal S}_{\mathcal N}$ between agents such that $P\rel{S}_{\mathcal N}Q$ implies:
\begin{enumerate}
\item If $P \red P'$ then $Q \wred Q'$ and $P'\rel{S}_{\mathcal N} Q'$.
\item If $P\downarrow_{\mathcal N} x$, then $Q\Downarrow_{\mathcal N} x$.
\end{enumerate}
$P$ is ${\mathcal N}$-barbed bisimilar to $Q$, written
$P \wbbisim_{\mathcal N} Q$, if $P \rel{S}_{\mathcal N} Q$ for some ${\mathcal N}$-barbed bisimulation ${\mathcal S}_{\mathcal N}$.
\end{definition}

$\mathcal{R} \subseteq \pi \times \pi$

$P \mathcal{R} Q => \forall P'. P \red P' \Rightarrow \exists Q'. Q \red Q', P' \mathcal{R} Q'$

$P \vdash x \Rightarrow Q \vdash x$

\begin{mathpar}
  \inferrule*[lab=Out-barb]{x \nameeq y}{{y}!\langle{Q}\rangle \vdash x}
  \and
  \inferrule*[lab=Par-barb]{\mbox{$P\vdash x$ or $Q\vdash x$}}{\binpar{P}{Q} \vdash x}
\end{mathpar}

\subsubsection{Contexts}

One of the principle advantages of computational calculi like the
$\pi$-calculus is a well-defined notion of context,
contextual-equivalence and a correlation between
contextual-equivalence and notions of bisimulation. The notion of
context allows the decomposition of a process into (sub-)process and
its syntactic environment, its context. Thus, a context may be
thought of as a process with a ``hole'' (written $\Box$) in it. The
application of a context $M$ to a process $P$, written $M[P]$, is
tantamount to filling the hole in $M$ with $P$. In this paper we do
not need the full weight of this theory, but do make use of the notion
of context in the proof the main theorem. 

\begin{mathpar}
  \inferrule* [lab=summation] {} {{M_{M},M_{N}} \bc \Box \;|\; x.M_{A} \;|\; M_{M}+M_{N}}
  \and
  \inferrule* [lab=agent] {} {{M_{A}} \bc (\vec{x})M_{P} \;| \; \clift{P_0,\ldots,M_{P},\ldots,P_N}}
  \and \\
  \inferrule* [lab=process] {} {{M_{P}} \bc M_{N} \;| \;P|M_{P} }
\end{mathpar} 

\begin{mathpar}
  \inferrule* [lab=sychronization] {} {M_{N} \bc \Box \;|\; x?M_{F} \;|\; x!M_{C}}
  \and
  \inferrule* [lab=abstraction] {} {{M_{F}} \bc (x)M_{P} }
  \and
  \inferrule* [lab=concretion] {} {{M_{C}} \bc \langle M_{P} \rangle }
  \and \\
  \inferrule* [lab=process] {} {{M_{P}} \bc M_{N} \;| \;P|M_{P} }
\end{mathpar}

\begin{definition}[contextual application] Given a context $M$, and
  process $P$, we define the \emph{contextual application}, $M[P] :=
  M\{P/\Box\}$. That is, the contextual application of M to P is the
  substitution of $P$ for $\Box$ in $M$.
\end{definition}

$\meaningof{-} : L \to \mathcal{P}(\pi)$

\begin{mathpar}
  \inferrule* [lab=collection] {} {\meaningof{true} = \pi, \and \meaningof{~E} = \pi \setminus \meaningof{E}, \and \meaningof{E_{1} \& E_{2}} = \meaningof{E_{1}} \cap \meaningof{E_{2}}}
\end{mathpar}

\begin{mathpar}
  \inferrule* [lab=structure] {} {\meaningof{0} = \{ P \in \pi | P \equiv 0 \}, \and \\ \meaningof{E_1 | E_2} = \{ P \in \pi | P \equiv P_{1} | P_{2}, P_{1} \in \meaningof{E_{1}}, P_{2} \in \meaningof{E_2}\} }
\end{mathpar}

\begin{mathpar}
 \inferrule* [lab=behavior] {} {\meaningof{\langle a?b \rangle E} = \{ P \in \pi | P \equiv Q | u?(y)P', \\ \and \\\\ \and \\ \;\;\; u \in \meaningof{a}, \forall z.P'\{z/y\} \in \meaningof{E\{z/b\}}\}, \and \\ \meaningof{a!E} = \{ P \in \pi | P \equiv Q | x!\langle P' \rangle, x \in \meaningof{a} P' \in \meaningof{E}\} }
\end{mathpar}

\begin{mathpar}
 \inferrule* [lab=nominal] {} {\meaningof{\quotep{E}} = \{ \quotep{P} \in \quotep{\pi} | P \in \meaningof{E} \}, \and \meaningof{\quotep{P}} = \{ \quotep{Q} \in \quotep{\pi} | P \equiv Q \} \and \\ \meaningof{@\quotep{E}} = \{ P \in \pi | P \equiv @x, x \in \meaningof{E} \}}
\end{mathpar}

\begin{eqnarray*}
  \\
  \meaningof{-} : TS \to ST
\end{eqnarray*}

\begin{eqnarray*}
  \\
  L : TS \to ST
\end{eqnarray*}

\begin{eqnarray*}
  \\
  P \models E \iff P \in \meaningof{E}
\end{eqnarray*}

\begin{eqnarray*}
  P \approx_{L} Q \iff \forall E \in L. P \models E \iff Q \models E
\end{eqnarray*}

\begin{eqnarray*}
  P \approx_{K} Q
\end{eqnarray*}

\begin{eqnarray*}
  P \approx Q
\end{eqnarray*}

$\approx_{K} = \approx = \approx_{L}$

\subsubsection{Contextual duality}

Note that contexts extend the quotation operation to a family of
operations from processes to names. Given a context, $M$, we can
define a \emph{nominal context}, $\quotep{M}$ by $\quotep{M}[P] :=
\quotep{M[P]}$. To foreshadow what is to come we observe that these
operations enjoy a duality with processes very much like the duality
between vectors and maps from vectors to scalars.

Further, because the calculus is essentially higher-order, we have a
correspondence between contexts and processes. More specifically,
given a name $x$ and a context $M$ we can construct $M^{*}_{x}$ such
that 

\begin{mathpar}
  M^{*}_{x} | \lift{x}{P} \red M[P]
\end{mathpar}

namely,

\begin{mathpar}
  M^{*}_{x} := x?(u).M[\dropn{u}]
\end{mathpar}

The dependence of $M^{*}_{x}$ on a name makes it an abstraction, 

\begin{mathpar}
  M^{*} := (x)x?(u).M[\dropn{u}]
\end{mathpar}

\subsection{Additional notation}

It will sometimes be convenient to denote the process a name
quotes. We already have the notation $x = \quotep{P}$, but it will be
convenient to introduce an alternate notation, $\procn{x}$, when we
want to emphasize the connection to the use of the name. Note that, by
virtue of name equivalence, $\quotep{\procn{x}} \nameeq x$; so, the
notation is consistent with previous definitions.

Further, because names have structure it is possible to effect
substitutions on the basis of that structure. This means we need to
upgrade our notation for substitutions, which we accomplish by
adapting comprehension notation. Thus,

\begin{mathpar}
  P\{ y / x : x \in S \}
\end{mathpar}

is interpreted to mean the process derived from P by replacing (in a
capture-avoiding manner) each occurrence of $x$ in $S$ by $y$. For example,

\begin{mathpar}
  P\{ \quotep{\procn{x}|\procn{x}} / x : x \in \freenames{P} \}
\end{mathpar}

will replace each (occurrence) of a free name $x$ in $P$ by
$\quotep{\procn{x}|\procn{x}}$.

Also, we will avail ourselves of the notation $x^{L}$ and $x^{R}$ to
denote injections of a name into disjoint copies of the name
space. There are numerous ways to accomplish this. One example can be
found in \cite{MeredithR05}. This notation overloads to vectors of
names: $\vec{x}^{\pi} := (x_{i}^{\pi} \; : \; 0 \leq i < |\vec{x}| )$ where $\pi \in \{L,R\}$.

We also use $P^{\Box} := P|\Box$.

In \cite{MeredithR05} an interpretation of the new operator is
given. It turns out that there are several possible interpretations
all enjoying the requisite algebraic properties of the operator (see
\cite{milner91polyadicpi}). We will therefore make liberal use of
$(\nu\; \vec{x})P$.

% subsection the_syntax_and_semantics_of_the_notation_system (end)   

\input{qm2pi.qmops} 

\input{qm2pi.sterngerlach} 

\input{qm2pi.metric} 

% section concurrent_process_calculi (end)

%\input{qm2pi.proofsketch}

% section proof sketch (end)

%\input{qm2pi.slviaknots} 

% section spatial logic via knots (end)

\input{qm2pi.conclusion}

% section conclusion (end)

%\input{qm2pi.dtcodes} 

% section wiring algorithm (end)

\input{qm2pi.ack} 

% section acknowledgments (end)

\newpage


\bibliographystyle{plain}   
\bibliography{../../biblios/main.bib}

\input{qm2pi.rhodetails}

\end{document}



% section proof sketch (end)

%\section{Unlikely characters: spatial logic for
  knots}\label{sub:characteristic_formulae} % (fold)

Associated to the mobile process calculi are a family of logics known
as the Hennessy-Milner logics. These logics typically enjoy a
semantics interpreting formulae as sets of processes that when
factored through the encoding outlined above allows an identification
of classes of knots with logical formulae. In the context of this
encoding the sub-family known as the spatial logics \cite{CairesC03}
\cite{CairesC04} \cite{Caires04} are of particular interest providing
several important features for expressing and reasoning about
properties (i.e. classes) of knots. We hint here at how this may be done.

%\begin{description}
%\item [structural connectives] 
\subsubsection{Structural connectives} The spatial logics enjoy
structural connectives corresponding, at the logical level, to the
parallel composition ($P | Q$) and new name ($(\nu \; x)P$)
connectives for processes. As illustrated in the examples below, these
connectives are extremely expressive given the shape of our encoding.
%\item [decideable satisfaction]

\subsubsection{Decideable satisfaction}
In \cite{Caires04} the satisfaction relation is shown to be decideable
for a rich class of processes. It further turns out that the image of
the our encoding is a proper subset of that class. This result
provides the basis for an algorithm by which to search for knots
enjoying a given property.
%\item [characteristic formulae]

\subsubsection{Characteristic formulae}
In the same paper \cite{Caires04} , Caires presents a means of calculating
characteristic formulae, selecting equivalence classes of processes
up to a pre--specified depth limit on the support set of names. Composed with our
encoding, this characteristic formula can be used to select
characteristic formulae for knots.
%\end{description}

\subsubsection{Spatial logic formulae}

The grammar below (segmented for comprehension) summarizes the syntax
of spatial logic formulae. We employ illustrative examples in the
sequel to provide an intuitive understanding of their meaning
referring the reader to \cite{Caires04} for a more detailed explication
of the semantics.

\begin{mathpar}
  \inferrule* [lab=boolean] {} {{A,B} \bc T \;|\; \neg A \;|\; A \wedge B \;|\; \eta = \eta'}
  \and
  \inferrule* [lab=spatial] {} {|\; \pzero \;|\; A | B \;|\; x \text{\textregistered} A \;|\; \forall x . A \;|\;  H x . A}
  \and
  \inferrule* [lab=behavioral] {} {|\; \alpha . A}
  \and 
  \inferrule* [lab=recursion] {} {|\; X(\vec{u}) \;|\; \mu X(\vec{u}) . A}
  \and
  \inferrule* [lab=action] {} {\alpha \bc \langle x?(\vec{y}) \rangle \;|\; \langle x!(\vec{y}) \rangle \;|\; \langle \tau \rangle}
  \and 
  \inferrule* [lab=name] {} {\eta \bc x \;|\; \tau}
\end{mathpar} 

% subsection characteristic_formulae (end)   	 

\subsection{Example formulae}\label{sub:example_formulae_} % (fold)

\subsubsection{Crossing as formula.}
% 
% \begin{align*}
%   \frac{d}{dx} \sin x &= \cos x 
%   & \frac{d}{dx} e^x &= e^x \\
%   \frac{d}{dx} \cos x &= - \sin x 
%   & \frac{d}{dx} \log x &= \frac{1}{x} \\
% \end{align*} 

\begin{align*}
 \mu C(x_{0},x_{1},y_{0},y_{1},u).&(\langle x_{0}?(z) \rangle(\langle u! \rangle\langle y_{1}!z \rangle C(x_{0},x_{1},y_{0},y_{1},u)) & \\
  & \wedge \langle y_{1}?(z) \rangle (\langle u! \rangle \langle x_{0}!z \rangle C(x_{0},x_{1},y_{0},y_{1},u)) & \\
  & \wedge \langle x_{1}?(z) \rangle (\langle u? \rangle \langle y_{0}!z \rangle C(x_{0},x_{1},y_{0},y_{1},u)) & \\
  & \wedge \langle y_{0}?(z) \rangle (\langle u? \rangle \langle x_{1}!z \rangle C(x_{0},x_{1},y_{0},y_{1},u))) &
\end{align*}

The lexicographical similarity between the shape of this formulae and
the shape of definition of the process representing a crossing reveals
the intuitive meaning of this formulae. It describes the capabilities
of a process that has the right to represent a crossing. For example
it picks out processes that may perform an input on the port $x_0$ in
its initial menu of capabilities. What differentiates the formula
from the process, however, is that the crossing process is the
smallest candidate to satisfy the formula. Infinitely many other
processes -- with internal behavior hidden behind this interface, so
to speak -- also satisfy this formula. Even this simple formula,
then, can be seen to open a new view onto knots, providing a
computational interpretation of \emph{virtual} knots.

Note that this formula is derived by hand. A similar formula can be
derived by employing Caires' calculation of characteristic formula
\cite{Caires04} to the process representing a crossing. In light of
this discussion, we let
$\meaningof{C}_{\phi}(x0,x1,y0,y1,u)$ denote a formula specifying the
dynamics we wish to capture of a crossing. To guarantee we preserve
the shape of the interface and minimal semantics we demand that
$\meaningof{C}_{\phi}(x0,x1,y0,y1,u) \Rightarrow
\textbf{C}(x0,x1,y0,y1,u)$ where $\textbf{C}(x0,x1,y0,y1,u)$ denotes
the formula above.
                            
\subsubsection{Crossing number constraints.}
The moral content of the context lemma (Lemma \ref{context}) is that the notion of
``locality'' in the Reidemeister moves is effectively captured by the
parallel composition operator of the process calculus. This intuition
extends through the logic. Given a formula,
$\meaningof{C}_{\phi}(x0,x1,y0,y1,u)$, we can use the structural
connectives to specify constraints on crossing numbers, such as at
least $n$ crossings, or exactly $n$ crossings.
\begin{mathpar}
  \inferrule* [lab=at-least-n] {} { K^{\geq n}_{\phi}(\vec{xs},\vec{ys}) := \Pi_{i=0}^{n-1} Hu . \meaningof{C}_{\phi}(xs_i,ys_i,u) | T }
  \and 
  \inferrule* [lab=exactly-n] {} { K^{= n}_{\phi}(\vec{xs},\vec{ys}) := \Pi_{i=0}^{n-1} Hu . \meaningof{C}_{\phi}(xs_i,ys_i,u) | \neg (\forall x_0,y_0,x_1,y_1,u . \meaningof{C}_{\phi}(x_0,y_0,x_1,y_1,u) | T) }
\end{mathpar}

To round out this section, recall that the encoding of an $n$-crossing
knot decomposes into a parallel composition of $n$ \emph{copies} of a
crossing process together with a wiring harness. To specify different
knot classes with the same crossing number amounts to specifying
logical constraints on the wiring harness. In the interest of space,
we defer examples to a forthcoming paper. Suffice it to say that both
the conditions ``alternating knot'' and ``contains the tangle
corresponding to 5/3'' are expressible. For example, it is possible to
calculate the characteristic formula of a process corresponding to the
tangle 5/3 and conjoin it into the classifying formula via the
composition connective of the logic.

Finally, we wish to observe that it is entirely within reason to
contemplate a more domain-specific version of spatial logic tailored
to the shape of processes in the image of the encoding. Such a
domain-specific logic would have a better claim to the title formal
language of knot properties.

% subsection example_formulae_ (end)

% section knots_as_processes (end) 

% section spatial logic via knots (end)

\section{Conclusions and future work}

\paragraph{Testing physical space}
You, gentle reader, may wonder why of all the theorems to be proved
given this set up we pick the one above. In some sense it's hardly
central to quantum mechanics. We see it as central in the sense that
it firmly establishes a notion of physical space arising from a notion
of the equivalence of behavior. Relating bisimulation to a metric is a
big step forward, but one is faced with interpreting the relationship
of that metric space to something more physical. Quantum mechanical
notions of ``physical'' space are still far from intuitive, but by
relating this idea of distance as testing to calculations that predict
physical circumstances we are making a not insignificant step forward
toward an understanding of the physical space we inhabit as
essentially dynamic.

\paragraph{Effectivity and simulation}
One of the observations we have yet to make is that the entire program
spelled out here is effective. We have built various interpreters for
the reflective calculus at work in this interpretation. In principle,
then, we can simulate quantum mechanics on a computer. The place where
the simulation may lose fidelity is the infinitely branching summation
for the annihilator.

In this connection i also want to point out that the evaluation style
calculation of the inner product puts the non-determinism of the
summation right at the heart of measurement. This suggests that
Milner's original reduction-based formulation of the dynamics of his
calculi in terms of sums was not just notationally suggestive of a
notion of measure-and-continue but captured some significant part of
the physics.

\paragraph{Quantum continuations}
In light of this last observation i want to point out that the
predominant account of quantum mechanics is missing a key aspect of a
truly compositional story of the physical situation. In a real lab,
when a measurement is made the observation can be made to feed into
another device that then makes another measurement conditioned on the
results of the first. This means that after the superposition was
collapsed the entire experimental set up remained in
superposition. While QM offers a means of writing this down it doesn't
quite line up well with the well-trodden formulation of computation
and continuation that we see so succinctly expressed in Milner's
calculi. This suggests that there might be advantages to this account
of dynamics waiting to be explored.

\paragraph{Quantum logic}
In this connection, we also note that by virtue of having the
Hennessy-Milner construction, we can pull the construction through the
interpretation of QM. This gives us a natural candidate for a quantum
logic that enjoys an extremely tight connection with it's domain of
interpretation, making the construction much less ad hoc (rather it is
the image of functor!).

\paragraph{Quantum probabiity}
i have questions about the basis of the interpretation of inner
product as probability amplitude. In particular, using which
axiomatization of probability theory does the notion of probability
amplitude earn the right to be so dubbed? In other words, where is the
proof that the operation for calculating a probability amplitude (and
then squaring) satisfies the axioms of what it means to calculate a
probability? Even if such a proof exists (i have yet to find it in the
literature), i wonder if it might not be possible to turn things on
their heads. Can we view the calculation of the probability amplitude
as an axiomatization of probability? If so, then the definition we
give for calculating probability amplitude may provide the basis for
an \emph{effective} theory of probability.

\paragraph{Quantum vs ``biological'' information}
Finally, i want to conclude with a more philosophical observation. At
a recent workshop in which QM was a predominant topic i noticed
something about quantum information. The speaker was giving a riveting
discussion of axiomatic QM and showing how properties of ``no
cloning'' and ``no deleting'' emerged as consequences of the
axiomatization. Theorems of this form are necessary to give us a sense
of confidence that our axioms characterize the physical theory. What
struck me, though, was that if quantum information is neither erasable
nor replicable it is markedly different from \emph{life}. Two of the
things we know about life is that

\begin{itemize}
  \item it ends;
  \item to gain some measure of persistence, to transcend it's
    finitude it is imminently copyable.
\end{itemize}

Both of these qualities are summarized succinctly in the aphorism: all
flesh is grass. For me these two kinds of ``information'' -- call them
quantum and biological -- are end points on a spectrum of strategies
for persistence. At one end, we have those curious entities that enjoy
uniqueness and permanence; at the other, we have those who in the face
of a certain end and an uncertain present make a go of passing
something on. To me one of the more remarkable aspects of the latter
strategy is that in the presence of noise (and certain features of
copying) we get a kind of dynamism, a chance for improvement against a
given persistent condition.

% subsection other_calculi_other_bisimulations_and_geometry_as_behavior (end)




% section conclusion (end)

%\documentclass[12pt]{llncs}
%\documentclass{jktr}

\usepackage[pdftex]{hyperref}                   
\usepackage {listings}
\usepackage {mathpartir}
\usepackage{bcprules}
%\usepackage{listings}
                       
\usepackage{graphicx} 
%\usepackage[margins=2.5cm,nohead,nofoot]{geometry}
%\usepackage{geometry}
\usepackage{amsfonts}
\usepackage{amstext}
\usepackage{latexsym}
\usepackage{amssymb}
\usepackage{color}


%\include{myPreamble}
\include{qm2pi.local} 

%\ifpdf
%\usepackage[pdftex]{graphicx}
%\else
%\usepackage{graphicx}
%\fi

 % \ifpdf
%  \usepackage{pdfsync}
%  \if


%\title{Brief Article}
%\author{David F. Snyder}
%\author{L.G. Meredith}

%\address{Dept. of Math., Texas State University--San Marcos, San Marcos, TX 78666}
       
\pagestyle{empty}


\begin{document}

\lstset{language=[Objective]Caml,frame=shadowbox}

\input{qm2pi.front}

% section front matter (end)

\input{qm2pi.intro} 
 
% section introduction (end)

% \input{qm2pi.knotations} 

% section notation (end)

\input{qm2pi.process.calculi} 

% section concurrent_process_calculi_and_spatial_logics_ (end)
    
%\input{qm2pi.knots2pi} 

%\input{qm2pi.trefoil} 

%\input{qm2pi.mainthm} 

% subsection basic_interpretation (end)

%\input{qm2pi.rho.presentation} 
\subsection{The syntax and semantics of the notation system}\label{sub:the_syntax_and_semantics_of_the_notation_system} % (fold)

We now summarize a technical presentation of the calculus that
embodies our theory of dynamics. The typical presentation of such a
calculus follows the style of giving generators and relations on
them. The grammar, below, describing term constructors, freely
generates the set of processes, $\Proc$. This set is then quotiented
by a relation known as structural congruence and it is over this set
that the notion of dynamics is expressed. This presentation is
essentially that of \cite{MeredithR05} with the addition of
polyadicity and summation. For readability we have relegated some of
the technical subtleties to an appendix.

\subsubsection{Process grammar}\label{subsub:process_grammar}

\begin{mathpar}
  \inferrule* [lab=synchronization] {} {{M} \bc \pzero \;|\; x?F \;|\; x!C }
  \and
  \inferrule* [lab=abstraction] {} {{F} \bc (x)P}
  \and
  \inferrule* [lab=concretion] {} {{C} \bc \langle Q \rangle}
  \and
  \inferrule* [lab=process] {} {{P,Q} \bc M \;| \;P|Q \;|\; @{x}}
  \and
  \inferrule* [lab=name] {} {{x} \bc \quotep{P}}
\end{mathpar} 

Note that $\vec{x}$ (resp. $\vec{P}$) denotes a vector of names
(resp. processes) of length $|\vec{x}|$ (resp. $|\vec{P}|$). We adopt
the following useful abbreviations.

\begin{mathpar}
   x?(\vec{y}).P := x.(\vec{y})P \and  x\clift{\vec{P}} := x.\clift{\vec{P}}
   \and x!(y) := \lift{x}{\dropn{y}}
   \and \Pi_{i=0}^{n-1}P_i := P_0 | \ldots | P_{n-1}
\end{mathpar}

\subsubsection{Structural congruence}

\paragraph{Free and bound names and alpha-equivalence.} At the
core of structural equivalence is alpha-equivalence which identifies
process that are the same up to a change of variable. Formally, we
recognize the distinction between free and bound names. The free names
of a process, $\freenames{P}$, may be calculated recursively as
follows:

\begin{mathpar}
\freenames{\pzero} := \emptyset
  \and \\
  \freenames{x?(y).P} := \{ x \} \cup (\freenames{P} \setminus \{ y \})
  \and 
  \freenames{x!\langle P \rangle} := \{ x \} \cup \{ P \} 
  \and \\
  \freenames{P|Q} := \freenames{P} \cup \freenames{Q}
  \and \\
  \freenames{@{x}} := \{ x \}
\end{mathpar}

$\pi$
$\quotep{\pi}$

$\freenames{-} : \pi \to \mathcal{P}(\quotep{\pi})$

\begin{eqnarray*}
  \freenames{\pzero} & := & \emptyset \\
  \freenames{x?(y).P} & := & \{ x \} \cup (\freenames{P} \setminus \{ y \}) \\
  \freenames{x!\langle P \rangle} & := & \{ x \} \cup \{ P \} \\
  \freenames{P|Q} & := & \freenames{P} \cup \freenames{Q} \\
  \freenames{\dropn{x}} & := & \{ x \}
\end{eqnarray*}

The bound names of a process, $\boundnames{P}$, are those names occurring in $P$
that are not free. For example, in $x?(y).0$, the name $x$ is free, while $y$ is bound.

\begin{mathpar}
  \inferrule* [lab=monoidal-laws] {} { P|Q \equiv Q|P \and P|0 \equiv P \and P|(Q|R) \equiv (P|Q)|R }
\end{mathpar}

\begin{mathpar}
  \inferrule* [lab=alpha-equivalence] {} { (x)P \equiv (y)P\{y/x\} \and y \not\in \freenames{P} }
\end{mathpar}

\begin{definition}
Then two processes, $P,Q$, are alpha-equivalent if $P = Q\{\vec{y}/\vec{x}\}$ for
some $\vec{x} \in \boundnames{Q},\vec{y} \in \boundnames{P}$, where $Q\{\vec{y}/\vec{x}\}$
denotes the capture-avoiding substitution of $\vec{y}$ for $\vec{x}$ in $Q$.
\end{definition}

\begin{definition}
  The {\em structural congruence} \cite{SangiorgiWalker} , $\equiv$,
  between processes is the least congruence containing
  alpha-equivalence, satisfying the abelian monoid laws
  (associativity, commutativity and $\pzero$ as identity) for parallel
  composition $|$ and for summation $+$.
\end{definition}

\subsection{Name equivalence}

We take name equivalence, written $\nameeq$, to be the smallest
equivalence relation generated by the following rules.

\begin{mathpar}
\inferrule*[lab=Quote-drop]
{ }
{ \quotep{@{x}} \nameeq x }

\inferrule*[lab=Struct-equiv]
{ P \scong Q }
{ \quotep{P} \nameeq \quotep{Q} }
\end{mathpar}

The astute reader will have noticed that the mutual recursion of names
and processes imposes a mutual recursion on alpha-equivalence and
structural equivalence via name-equivalence. Fortunately, all of this
works out pleasantly and we may calculate in the natural way, free of
concern. The reader interested in the details is referred to the
appendix \ref{appendix:rho_details}.

\subsection{Substitution}

We use $\Proc$ for the set of processes, $\QProc$ for the set of
names, and $\id{\{}\vec{y} / \vec{x} \id{\}}$ to denote partial maps,
$s : \QProc \rightarrow \QProc$. A map, $s$ lifts, uniquely, to a map
on process terms, $\widehat{s} : \Proc \rightarrow \Proc$ by the
following equations.

\begin{mathpar}
  (0) \psubstp{Q}{P} := 0 \\
  (R \juxtap S) \psubstp{Q}{P}
  :=    
  (R)\psubstp{Q}{P} \juxtap (S) \psubstp{Q}{P} \\
  (x?(y).R) \psubstp{Q}{P}    
  :=    
  (x)\substp{Q}{P} (z)\concat( (R \psubstn{z}{y}) \psubstp{Q}{P} ) \\
  (\lift{x}{R}) \psubstp{Q}{P}  
  :=
  \lift{(x)\substp{Q}{P}}{ R \psubstp{Q}{P} } \\
%   (\dropn{x})  \psubstp{Q}{P}       
%   := 
%   \left\{ 
%     \begin{array}{ccc} 
%       \dropn{\quotep{Q}} & & x \nameeq \quotep{P} \\
%       \dropn{x} & & otherwise \\
%     \end{array}
%   \right. 
  (\dropn{x})  \psubstp{Q}{P}       
  := 
  \left\{ 
    \begin{array}{ccc} 
      Q & & x \nameeq \quotep{P} \\
      \dropn{x} & & otherwise \\
    \end{array}
  \right.
\end{mathpar}
 

where

\begin{eqnarray}
  (x)\id{\{} \lpquote Q \rpquote / \lpquote P \rpquote \id{\}}            = 
  \left\{ 
    \begin{array}{ccc}
      \lpquote Q \rpquote & & x \nameeq \lpquote P \rpquote \\
      x & & otherwise \\
    \end{array}
  \right. \nonumber
\end{eqnarray}

and $z$ is chosen distinct from $\quotep{P}$, $\quotep{Q}$, the free
names in $Q$, and all the names in $R$. Our $\alpha$-equivalence will
be built in the standard way from this substitution.

\begin{remark}\label{rem:no_self_referential_names}
  One consequence of these definitions is that $\forall P. \quotep{P}
  \not\in \freenames{P}$.
\end{remark}

\subsection{ Dynamic quote: an example }

Anticipating something of what's to come, consider applying the
substitution, $\widehat{\id{\{}u / z \id{\}}}$, to the following pair
of processes, $\lift{w}{y!(z)}$ and $w[ \lpquote y!(z) \rpquote ]$.

\begin{eqnarray}
	\lift{w}{y!(z)}\widehat{\id{\{}u / z \id{\}}}
		& = &
		\lift{w}{y!(u)} \nonumber\\
	w[ \lpquote y!(z) \rpquote ] \widehat{ \id{\{}u / z \id{\}} }
		& = &
		w[ \lpquote y!(z) \rpquote ] \nonumber
\end{eqnarray}

Because the body of the process between quotes is impervious to
substitution, we get radically different answers. In fact, by
examining the first process in an input context,
e.g. $x?(z).\lift{w}{y!(z)}$, we see that the process under the lift
operator may be shaped by prefixed inputs binding a name inside it. In
this sense, the lift operator will be seen as a way to dynamically
construct processes before reifying them as names.

Finally equipped with these standard features we can present the
dynamics of the calculus.

\subsubsection{Operational semantics} 

Finally, we introduce the computational dynamics. What marks these
algebras as distinct from other more traditionally studied algebraic
structures, e.g. vector spaces or polynomial rings, is the manner in
which dynamics is captured. In traditional structures, dynamics is typically
expressed through morphisms between such structures, as in linear maps
between vector spaces or morphisms between rings. In algebras
associated with the semantics of computation, the dynamics is
expressed as part of the algebraic structure itself, through a
reduction reduction relation typically denoted by $\red$. Below, we
give a recursive presentation of this relation for the calculus used
in the encoding.

$\red \subseteq \pi \times \pi$
$\red : \pi \to \mathcal{P}(\pi)$

\begin{mathpar}
  \inferrule* [lab=Comm] { \textsf{match}( x_{src}, x_{trgt} ) } { x_{trgt}?(y)P \; | \; x_{src}!\langle {Q} \rangle \red P\{\quotep{Q}/y}\} }
  \and \\
  \inferrule* [lab=Par] {{P} \red {P}'} {{{P} | {Q}} \red {{P}' | {Q}}}
  \and
  \inferrule* [lab=Equiv]{{{P} \scong {P}'} \andalso {{P}' \red {Q}'} \andalso {{Q}' \scong {Q}}}{{P} \red {Q}}
\end{mathpar}

\begin{eqnarray*}
  match_{\equiv} (\quotep{P},\quotep{Q}) & := & P \equiv Q \\
  match_{\dagger}(\quotep{P},\quotep{Q}) & := & \forall R. P|Q \red^{*} R => R \red^{*} 0 \\
  match_{K}(\quotep{P},\quotep{Q}) & := & K \mbox{ for some context } K
\end{eqnarray*}

$u?(x)P | u!\langle Q \rangle \red P\{\quotep{Q}/x\}$

%We write $\wred$ for $\red^*$, and $P\red$ if $\exists Q $ such that $ P \red Q$.
We write $P\red$ if $\exists Q $ such that $ P \red Q$ and $P\not\red$, otherwise.

\section{Replication}

As mentioned before, it is known that replication (and hence
recursion) can be implemented in a higher-order process algebra
\cite{SangiorgiWalker}. As our first example of calculation with the
machinery thus far presented we give the construction explicitly in
the {\rhoc}.

\begin{eqnarray}
	D_{x} & := & \prefix{x}{y}{(\binpar{\outputp{x}{y}}{@{y}})} \nonumber\\
	\bangp_{x}{P} & := & \binpar{{x}!\langle{\binpar{D_{x}}{P}}\rangle}{D_{x}} \nonumber
\end{eqnarray}

\begin{eqnarray}
	\bangp_{x}{P} & & \nonumber\\
	=
	& {x}!\langle{(\prefix{x}{y}{(\outputp{x}{y} | @{y})) | P}}\rangle 
	      | \prefix{x}{y}{(\outputp{x}{y} | @{y})} & \nonumber\\
	\red
	& (\outputp{x}{y} | @{y})\substn{\quotep{(\prefix{x}{y}{(@{y} | \outputp{x}{y})) | P}}}{y} & \nonumber\\
	=
	& \outputp{x}{\quotep{(\prefix{x}{y}{(\outputp{x}{y} | @{y})) | P}}}
	  | {(\prefix{x}{y}{(\outputp{x}{y} | @{y})) | P}} & \nonumber\\
	\red
	& \ldots & \nonumber\\
	\red^*
	& P | P | \ldots & \nonumber
\end{eqnarray}

Of course, this encoding, as an implementation, runs away, unfolding
$\bangp{P}$ eagerly. A lazier and more implementable replication
operator, restricted to input-guarded processes, may be obtained as follows.

\begin{eqnarray}
\bangp{\prefix{u}{v}{P}} 
	:= 
	\binpar{\lift{x}{\prefix{u}{v}{(\binpar{D(x)}{P})}}}{D(x)} \nonumber
\end{eqnarray}

\begin{remark}
  Note that the lazier definition still does not deal with summation
  or mixed summation (i.e. sums over input and output). The reader is
  invited to construct definitions of replication that deal with these
  features. 

  Further, the definitions are parameterized in a name, $x$. Can you,
  gentle reader, make a definition that eliminates this parameter and
  guarantees no accidental interaction between the replication
  machinery and the process being replicated -- i.e. no accidental
  sharing of names used by the process to get its work done and the
  name(s) used by the replication to effect copying. This latter
  revision of the definition of replication is crucial to obtaining
  the expected identity $!!P \sim !P$.
\end{remark}

\begin{remark}\label{rem:paradoxical_combinator}
  The reader familiar with the lambda calculus will have noticed the
  similarity between $D$ and the paradoxical combinator.

  [Ed. note: the existence of this seems to suggest we have to be more
  restrictive on the set of processes and names we admit if we are to
  support no-cloning.]
\end{remark}

\subsubsection{Bisimulation}

The computational dynamics gives rise to another kind of equivalence,
the equivalence of computational behavior. As previously mentioned
this is typically captured \emph{via} some form of bisimulation.

% The notion we use in this paper is weak barbed bisimulation
% \cite{milner91polyadicpi}.

The notion we use in this paper is derived from weak barbed
bisimulation \cite{milner91polyadicpi}. 

\begin{definition}
An \emph{observation relation}, $\downarrow_{\mathcal N}$, over a set
of names, $\mathcal N$, is the smallest relation satisfying the rules
below.

\infrule[Out-barb]{y \in {\mathcal N}, \; x \nameeq y}
		  {\outputp{x}{v} \downarrow_{\mathcal N} x}
\infrule[Par-barb]{\mbox{$P\downarrow_{\mathcal N} x$ or $Q\downarrow_{\mathcal N} x$}}
		  {\binpar{P}{Q} \downarrow_{\mathcal N} x}

We write $P \Downarrow_{\mathcal N} x$ if there is $Q$ such that 
$P \wred Q$ and $Q \downarrow_{\mathcal N} x$.
\end{definition}

\begin{definition}
%\label{def.bbisim}
An  ${\mathcal N}$-\emph{barbed bisimulation} over a set of names, ${\mathcal N}$, is a symmetric binary relation 
${\mathcal S}_{\mathcal N}$ between agents such that $P\rel{S}_{\mathcal N}Q$ implies:
\begin{enumerate}
\item If $P \red P'$ then $Q \wred Q'$ and $P'\rel{S}_{\mathcal N} Q'$.
\item If $P\downarrow_{\mathcal N} x$, then $Q\Downarrow_{\mathcal N} x$.
\end{enumerate}
$P$ is ${\mathcal N}$-barbed bisimilar to $Q$, written
$P \wbbisim_{\mathcal N} Q$, if $P \rel{S}_{\mathcal N} Q$ for some ${\mathcal N}$-barbed bisimulation ${\mathcal S}_{\mathcal N}$.
\end{definition}

$\mathcal{R} \subseteq \pi \times \pi$

$P \mathcal{R} Q => \forall P'. P \red P' \Rightarrow \exists Q'. Q \red Q', P' \mathcal{R} Q'$

$P \vdash x \Rightarrow Q \vdash x$

\begin{mathpar}
  \inferrule*[lab=Out-barb]{x \nameeq y}{{y}!\langle{Q}\rangle \vdash x}
  \and
  \inferrule*[lab=Par-barb]{\mbox{$P\vdash x$ or $Q\vdash x$}}{\binpar{P}{Q} \vdash x}
\end{mathpar}

\subsubsection{Contexts}

One of the principle advantages of computational calculi like the
$\pi$-calculus is a well-defined notion of context,
contextual-equivalence and a correlation between
contextual-equivalence and notions of bisimulation. The notion of
context allows the decomposition of a process into (sub-)process and
its syntactic environment, its context. Thus, a context may be
thought of as a process with a ``hole'' (written $\Box$) in it. The
application of a context $M$ to a process $P$, written $M[P]$, is
tantamount to filling the hole in $M$ with $P$. In this paper we do
not need the full weight of this theory, but do make use of the notion
of context in the proof the main theorem. 

\begin{mathpar}
  \inferrule* [lab=summation] {} {{M_{M},M_{N}} \bc \Box \;|\; x.M_{A} \;|\; M_{M}+M_{N}}
  \and
  \inferrule* [lab=agent] {} {{M_{A}} \bc (\vec{x})M_{P} \;| \; \clift{P_0,\ldots,M_{P},\ldots,P_N}}
  \and \\
  \inferrule* [lab=process] {} {{M_{P}} \bc M_{N} \;| \;P|M_{P} }
\end{mathpar} 

\begin{mathpar}
  \inferrule* [lab=sychronization] {} {M_{N} \bc \Box \;|\; x?M_{F} \;|\; x!M_{C}}
  \and
  \inferrule* [lab=abstraction] {} {{M_{F}} \bc (x)M_{P} }
  \and
  \inferrule* [lab=concretion] {} {{M_{C}} \bc \langle M_{P} \rangle }
  \and \\
  \inferrule* [lab=process] {} {{M_{P}} \bc M_{N} \;| \;P|M_{P} }
\end{mathpar}

\begin{definition}[contextual application] Given a context $M$, and
  process $P$, we define the \emph{contextual application}, $M[P] :=
  M\{P/\Box\}$. That is, the contextual application of M to P is the
  substitution of $P$ for $\Box$ in $M$.
\end{definition}

$\meaningof{-} : L \to \mathcal{P}(\pi)$

\begin{mathpar}
  \inferrule* [lab=collection] {} {\meaningof{true} = \pi, \and \meaningof{~E} = \pi \setminus \meaningof{E}, \and \meaningof{E_{1} \& E_{2}} = \meaningof{E_{1}} \cap \meaningof{E_{2}}}
\end{mathpar}

\begin{mathpar}
  \inferrule* [lab=structure] {} {\meaningof{0} = \{ P \in \pi | P \equiv 0 \}, \and \\ \meaningof{E_1 | E_2} = \{ P \in \pi | P \equiv P_{1} | P_{2}, P_{1} \in \meaningof{E_{1}}, P_{2} \in \meaningof{E_2}\} }
\end{mathpar}

\begin{mathpar}
 \inferrule* [lab=behavior] {} {\meaningof{\langle a?b \rangle E} = \{ P \in \pi | P \equiv Q | u?(y)P', \\ \and \\\\ \and \\ \;\;\; u \in \meaningof{a}, \forall z.P'\{z/y\} \in \meaningof{E\{z/b\}}\}, \and \\ \meaningof{a!E} = \{ P \in \pi | P \equiv Q | x!\langle P' \rangle, x \in \meaningof{a} P' \in \meaningof{E}\} }
\end{mathpar}

\begin{mathpar}
 \inferrule* [lab=nominal] {} {\meaningof{\quotep{E}} = \{ \quotep{P} \in \quotep{\pi} | P \in \meaningof{E} \}, \and \meaningof{\quotep{P}} = \{ \quotep{Q} \in \quotep{\pi} | P \equiv Q \} \and \\ \meaningof{@\quotep{E}} = \{ P \in \pi | P \equiv @x, x \in \meaningof{E} \}}
\end{mathpar}

\begin{eqnarray*}
  \\
  \meaningof{-} : TS \to ST
\end{eqnarray*}

\begin{eqnarray*}
  \\
  L : TS \to ST
\end{eqnarray*}

\begin{eqnarray*}
  \\
  P \models E \iff P \in \meaningof{E}
\end{eqnarray*}

\begin{eqnarray*}
  P \approx_{L} Q \iff \forall E \in L. P \models E \iff Q \models E
\end{eqnarray*}

\begin{eqnarray*}
  P \approx_{K} Q
\end{eqnarray*}

\begin{eqnarray*}
  P \approx Q
\end{eqnarray*}

$\approx_{K} = \approx = \approx_{L}$

\subsubsection{Contextual duality}

Note that contexts extend the quotation operation to a family of
operations from processes to names. Given a context, $M$, we can
define a \emph{nominal context}, $\quotep{M}$ by $\quotep{M}[P] :=
\quotep{M[P]}$. To foreshadow what is to come we observe that these
operations enjoy a duality with processes very much like the duality
between vectors and maps from vectors to scalars.

Further, because the calculus is essentially higher-order, we have a
correspondence between contexts and processes. More specifically,
given a name $x$ and a context $M$ we can construct $M^{*}_{x}$ such
that 

\begin{mathpar}
  M^{*}_{x} | \lift{x}{P} \red M[P]
\end{mathpar}

namely,

\begin{mathpar}
  M^{*}_{x} := x?(u).M[\dropn{u}]
\end{mathpar}

The dependence of $M^{*}_{x}$ on a name makes it an abstraction, 

\begin{mathpar}
  M^{*} := (x)x?(u).M[\dropn{u}]
\end{mathpar}

\subsection{Additional notation}

It will sometimes be convenient to denote the process a name
quotes. We already have the notation $x = \quotep{P}$, but it will be
convenient to introduce an alternate notation, $\procn{x}$, when we
want to emphasize the connection to the use of the name. Note that, by
virtue of name equivalence, $\quotep{\procn{x}} \nameeq x$; so, the
notation is consistent with previous definitions.

Further, because names have structure it is possible to effect
substitutions on the basis of that structure. This means we need to
upgrade our notation for substitutions, which we accomplish by
adapting comprehension notation. Thus,

\begin{mathpar}
  P\{ y / x : x \in S \}
\end{mathpar}

is interpreted to mean the process derived from P by replacing (in a
capture-avoiding manner) each occurrence of $x$ in $S$ by $y$. For example,

\begin{mathpar}
  P\{ \quotep{\procn{x}|\procn{x}} / x : x \in \freenames{P} \}
\end{mathpar}

will replace each (occurrence) of a free name $x$ in $P$ by
$\quotep{\procn{x}|\procn{x}}$.

Also, we will avail ourselves of the notation $x^{L}$ and $x^{R}$ to
denote injections of a name into disjoint copies of the name
space. There are numerous ways to accomplish this. One example can be
found in \cite{MeredithR05}. This notation overloads to vectors of
names: $\vec{x}^{\pi} := (x_{i}^{\pi} \; : \; 0 \leq i < |\vec{x}| )$ where $\pi \in \{L,R\}$.

We also use $P^{\Box} := P|\Box$.

In \cite{MeredithR05} an interpretation of the new operator is
given. It turns out that there are several possible interpretations
all enjoying the requisite algebraic properties of the operator (see
\cite{milner91polyadicpi}). We will therefore make liberal use of
$(\nu\; \vec{x})P$.

% subsection the_syntax_and_semantics_of_the_notation_system (end)   

\input{qm2pi.qmops} 

\input{qm2pi.sterngerlach} 

\input{qm2pi.metric} 

% section concurrent_process_calculi (end)

%\input{qm2pi.proofsketch}

% section proof sketch (end)

%\input{qm2pi.slviaknots} 

% section spatial logic via knots (end)

\input{qm2pi.conclusion}

% section conclusion (end)

%\input{qm2pi.dtcodes} 

% section wiring algorithm (end)

\input{qm2pi.ack} 

% section acknowledgments (end)

\newpage


\bibliographystyle{plain}   
\bibliography{../../biblios/main.bib}

\input{qm2pi.rhodetails}

\end{document}

 

% section wiring algorithm (end)

\documentclass[12pt]{llncs}
%\documentclass{jktr}

\usepackage[pdftex]{hyperref}                   
\usepackage {listings}
\usepackage {mathpartir}
\usepackage{bcprules}
%\usepackage{listings}
                       
\usepackage{graphicx} 
%\usepackage[margins=2.5cm,nohead,nofoot]{geometry}
%\usepackage{geometry}
\usepackage{amsfonts}
\usepackage{amstext}
\usepackage{latexsym}
\usepackage{amssymb}
\usepackage{color}


%\include{myPreamble}
\include{qm2pi.local} 

%\ifpdf
%\usepackage[pdftex]{graphicx}
%\else
%\usepackage{graphicx}
%\fi

 % \ifpdf
%  \usepackage{pdfsync}
%  \if


%\title{Brief Article}
%\author{David F. Snyder}
%\author{L.G. Meredith}

%\address{Dept. of Math., Texas State University--San Marcos, San Marcos, TX 78666}
       
\pagestyle{empty}


\begin{document}

\lstset{language=[Objective]Caml,frame=shadowbox}

\input{qm2pi.front}

% section front matter (end)

\input{qm2pi.intro} 
 
% section introduction (end)

% \input{qm2pi.knotations} 

% section notation (end)

\input{qm2pi.process.calculi} 

% section concurrent_process_calculi_and_spatial_logics_ (end)
    
%\input{qm2pi.knots2pi} 

%\input{qm2pi.trefoil} 

%\input{qm2pi.mainthm} 

% subsection basic_interpretation (end)

%\input{qm2pi.rho.presentation} 
\subsection{The syntax and semantics of the notation system}\label{sub:the_syntax_and_semantics_of_the_notation_system} % (fold)

We now summarize a technical presentation of the calculus that
embodies our theory of dynamics. The typical presentation of such a
calculus follows the style of giving generators and relations on
them. The grammar, below, describing term constructors, freely
generates the set of processes, $\Proc$. This set is then quotiented
by a relation known as structural congruence and it is over this set
that the notion of dynamics is expressed. This presentation is
essentially that of \cite{MeredithR05} with the addition of
polyadicity and summation. For readability we have relegated some of
the technical subtleties to an appendix.

\subsubsection{Process grammar}\label{subsub:process_grammar}

\begin{mathpar}
  \inferrule* [lab=synchronization] {} {{M} \bc \pzero \;|\; x?F \;|\; x!C }
  \and
  \inferrule* [lab=abstraction] {} {{F} \bc (x)P}
  \and
  \inferrule* [lab=concretion] {} {{C} \bc \langle Q \rangle}
  \and
  \inferrule* [lab=process] {} {{P,Q} \bc M \;| \;P|Q \;|\; @{x}}
  \and
  \inferrule* [lab=name] {} {{x} \bc \quotep{P}}
\end{mathpar} 

Note that $\vec{x}$ (resp. $\vec{P}$) denotes a vector of names
(resp. processes) of length $|\vec{x}|$ (resp. $|\vec{P}|$). We adopt
the following useful abbreviations.

\begin{mathpar}
   x?(\vec{y}).P := x.(\vec{y})P \and  x\clift{\vec{P}} := x.\clift{\vec{P}}
   \and x!(y) := \lift{x}{\dropn{y}}
   \and \Pi_{i=0}^{n-1}P_i := P_0 | \ldots | P_{n-1}
\end{mathpar}

\subsubsection{Structural congruence}

\paragraph{Free and bound names and alpha-equivalence.} At the
core of structural equivalence is alpha-equivalence which identifies
process that are the same up to a change of variable. Formally, we
recognize the distinction between free and bound names. The free names
of a process, $\freenames{P}$, may be calculated recursively as
follows:

\begin{mathpar}
\freenames{\pzero} := \emptyset
  \and \\
  \freenames{x?(y).P} := \{ x \} \cup (\freenames{P} \setminus \{ y \})
  \and 
  \freenames{x!\langle P \rangle} := \{ x \} \cup \{ P \} 
  \and \\
  \freenames{P|Q} := \freenames{P} \cup \freenames{Q}
  \and \\
  \freenames{@{x}} := \{ x \}
\end{mathpar}

$\pi$
$\quotep{\pi}$

$\freenames{-} : \pi \to \mathcal{P}(\quotep{\pi})$

\begin{eqnarray*}
  \freenames{\pzero} & := & \emptyset \\
  \freenames{x?(y).P} & := & \{ x \} \cup (\freenames{P} \setminus \{ y \}) \\
  \freenames{x!\langle P \rangle} & := & \{ x \} \cup \{ P \} \\
  \freenames{P|Q} & := & \freenames{P} \cup \freenames{Q} \\
  \freenames{\dropn{x}} & := & \{ x \}
\end{eqnarray*}

The bound names of a process, $\boundnames{P}$, are those names occurring in $P$
that are not free. For example, in $x?(y).0$, the name $x$ is free, while $y$ is bound.

\begin{mathpar}
  \inferrule* [lab=monoidal-laws] {} { P|Q \equiv Q|P \and P|0 \equiv P \and P|(Q|R) \equiv (P|Q)|R }
\end{mathpar}

\begin{mathpar}
  \inferrule* [lab=alpha-equivalence] {} { (x)P \equiv (y)P\{y/x\} \and y \not\in \freenames{P} }
\end{mathpar}

\begin{definition}
Then two processes, $P,Q$, are alpha-equivalent if $P = Q\{\vec{y}/\vec{x}\}$ for
some $\vec{x} \in \boundnames{Q},\vec{y} \in \boundnames{P}$, where $Q\{\vec{y}/\vec{x}\}$
denotes the capture-avoiding substitution of $\vec{y}$ for $\vec{x}$ in $Q$.
\end{definition}

\begin{definition}
  The {\em structural congruence} \cite{SangiorgiWalker} , $\equiv$,
  between processes is the least congruence containing
  alpha-equivalence, satisfying the abelian monoid laws
  (associativity, commutativity and $\pzero$ as identity) for parallel
  composition $|$ and for summation $+$.
\end{definition}

\subsection{Name equivalence}

We take name equivalence, written $\nameeq$, to be the smallest
equivalence relation generated by the following rules.

\begin{mathpar}
\inferrule*[lab=Quote-drop]
{ }
{ \quotep{@{x}} \nameeq x }

\inferrule*[lab=Struct-equiv]
{ P \scong Q }
{ \quotep{P} \nameeq \quotep{Q} }
\end{mathpar}

The astute reader will have noticed that the mutual recursion of names
and processes imposes a mutual recursion on alpha-equivalence and
structural equivalence via name-equivalence. Fortunately, all of this
works out pleasantly and we may calculate in the natural way, free of
concern. The reader interested in the details is referred to the
appendix \ref{appendix:rho_details}.

\subsection{Substitution}

We use $\Proc$ for the set of processes, $\QProc$ for the set of
names, and $\id{\{}\vec{y} / \vec{x} \id{\}}$ to denote partial maps,
$s : \QProc \rightarrow \QProc$. A map, $s$ lifts, uniquely, to a map
on process terms, $\widehat{s} : \Proc \rightarrow \Proc$ by the
following equations.

\begin{mathpar}
  (0) \psubstp{Q}{P} := 0 \\
  (R \juxtap S) \psubstp{Q}{P}
  :=    
  (R)\psubstp{Q}{P} \juxtap (S) \psubstp{Q}{P} \\
  (x?(y).R) \psubstp{Q}{P}    
  :=    
  (x)\substp{Q}{P} (z)\concat( (R \psubstn{z}{y}) \psubstp{Q}{P} ) \\
  (\lift{x}{R}) \psubstp{Q}{P}  
  :=
  \lift{(x)\substp{Q}{P}}{ R \psubstp{Q}{P} } \\
%   (\dropn{x})  \psubstp{Q}{P}       
%   := 
%   \left\{ 
%     \begin{array}{ccc} 
%       \dropn{\quotep{Q}} & & x \nameeq \quotep{P} \\
%       \dropn{x} & & otherwise \\
%     \end{array}
%   \right. 
  (\dropn{x})  \psubstp{Q}{P}       
  := 
  \left\{ 
    \begin{array}{ccc} 
      Q & & x \nameeq \quotep{P} \\
      \dropn{x} & & otherwise \\
    \end{array}
  \right.
\end{mathpar}
 

where

\begin{eqnarray}
  (x)\id{\{} \lpquote Q \rpquote / \lpquote P \rpquote \id{\}}            = 
  \left\{ 
    \begin{array}{ccc}
      \lpquote Q \rpquote & & x \nameeq \lpquote P \rpquote \\
      x & & otherwise \\
    \end{array}
  \right. \nonumber
\end{eqnarray}

and $z$ is chosen distinct from $\quotep{P}$, $\quotep{Q}$, the free
names in $Q$, and all the names in $R$. Our $\alpha$-equivalence will
be built in the standard way from this substitution.

\begin{remark}\label{rem:no_self_referential_names}
  One consequence of these definitions is that $\forall P. \quotep{P}
  \not\in \freenames{P}$.
\end{remark}

\subsection{ Dynamic quote: an example }

Anticipating something of what's to come, consider applying the
substitution, $\widehat{\id{\{}u / z \id{\}}}$, to the following pair
of processes, $\lift{w}{y!(z)}$ and $w[ \lpquote y!(z) \rpquote ]$.

\begin{eqnarray}
	\lift{w}{y!(z)}\widehat{\id{\{}u / z \id{\}}}
		& = &
		\lift{w}{y!(u)} \nonumber\\
	w[ \lpquote y!(z) \rpquote ] \widehat{ \id{\{}u / z \id{\}} }
		& = &
		w[ \lpquote y!(z) \rpquote ] \nonumber
\end{eqnarray}

Because the body of the process between quotes is impervious to
substitution, we get radically different answers. In fact, by
examining the first process in an input context,
e.g. $x?(z).\lift{w}{y!(z)}$, we see that the process under the lift
operator may be shaped by prefixed inputs binding a name inside it. In
this sense, the lift operator will be seen as a way to dynamically
construct processes before reifying them as names.

Finally equipped with these standard features we can present the
dynamics of the calculus.

\subsubsection{Operational semantics} 

Finally, we introduce the computational dynamics. What marks these
algebras as distinct from other more traditionally studied algebraic
structures, e.g. vector spaces or polynomial rings, is the manner in
which dynamics is captured. In traditional structures, dynamics is typically
expressed through morphisms between such structures, as in linear maps
between vector spaces or morphisms between rings. In algebras
associated with the semantics of computation, the dynamics is
expressed as part of the algebraic structure itself, through a
reduction reduction relation typically denoted by $\red$. Below, we
give a recursive presentation of this relation for the calculus used
in the encoding.

$\red \subseteq \pi \times \pi$
$\red : \pi \to \mathcal{P}(\pi)$

\begin{mathpar}
  \inferrule* [lab=Comm] { \textsf{match}( x_{src}, x_{trgt} ) } { x_{trgt}?(y)P \; | \; x_{src}!\langle {Q} \rangle \red P\{\quotep{Q}/y}\} }
  \and \\
  \inferrule* [lab=Par] {{P} \red {P}'} {{{P} | {Q}} \red {{P}' | {Q}}}
  \and
  \inferrule* [lab=Equiv]{{{P} \scong {P}'} \andalso {{P}' \red {Q}'} \andalso {{Q}' \scong {Q}}}{{P} \red {Q}}
\end{mathpar}

\begin{eqnarray*}
  match_{\equiv} (\quotep{P},\quotep{Q}) & := & P \equiv Q \\
  match_{\dagger}(\quotep{P},\quotep{Q}) & := & \forall R. P|Q \red^{*} R => R \red^{*} 0 \\
  match_{K}(\quotep{P},\quotep{Q}) & := & K \mbox{ for some context } K
\end{eqnarray*}

$u?(x)P | u!\langle Q \rangle \red P\{\quotep{Q}/x\}$

%We write $\wred$ for $\red^*$, and $P\red$ if $\exists Q $ such that $ P \red Q$.
We write $P\red$ if $\exists Q $ such that $ P \red Q$ and $P\not\red$, otherwise.

\section{Replication}

As mentioned before, it is known that replication (and hence
recursion) can be implemented in a higher-order process algebra
\cite{SangiorgiWalker}. As our first example of calculation with the
machinery thus far presented we give the construction explicitly in
the {\rhoc}.

\begin{eqnarray}
	D_{x} & := & \prefix{x}{y}{(\binpar{\outputp{x}{y}}{@{y}})} \nonumber\\
	\bangp_{x}{P} & := & \binpar{{x}!\langle{\binpar{D_{x}}{P}}\rangle}{D_{x}} \nonumber
\end{eqnarray}

\begin{eqnarray}
	\bangp_{x}{P} & & \nonumber\\
	=
	& {x}!\langle{(\prefix{x}{y}{(\outputp{x}{y} | @{y})) | P}}\rangle 
	      | \prefix{x}{y}{(\outputp{x}{y} | @{y})} & \nonumber\\
	\red
	& (\outputp{x}{y} | @{y})\substn{\quotep{(\prefix{x}{y}{(@{y} | \outputp{x}{y})) | P}}}{y} & \nonumber\\
	=
	& \outputp{x}{\quotep{(\prefix{x}{y}{(\outputp{x}{y} | @{y})) | P}}}
	  | {(\prefix{x}{y}{(\outputp{x}{y} | @{y})) | P}} & \nonumber\\
	\red
	& \ldots & \nonumber\\
	\red^*
	& P | P | \ldots & \nonumber
\end{eqnarray}

Of course, this encoding, as an implementation, runs away, unfolding
$\bangp{P}$ eagerly. A lazier and more implementable replication
operator, restricted to input-guarded processes, may be obtained as follows.

\begin{eqnarray}
\bangp{\prefix{u}{v}{P}} 
	:= 
	\binpar{\lift{x}{\prefix{u}{v}{(\binpar{D(x)}{P})}}}{D(x)} \nonumber
\end{eqnarray}

\begin{remark}
  Note that the lazier definition still does not deal with summation
  or mixed summation (i.e. sums over input and output). The reader is
  invited to construct definitions of replication that deal with these
  features. 

  Further, the definitions are parameterized in a name, $x$. Can you,
  gentle reader, make a definition that eliminates this parameter and
  guarantees no accidental interaction between the replication
  machinery and the process being replicated -- i.e. no accidental
  sharing of names used by the process to get its work done and the
  name(s) used by the replication to effect copying. This latter
  revision of the definition of replication is crucial to obtaining
  the expected identity $!!P \sim !P$.
\end{remark}

\begin{remark}\label{rem:paradoxical_combinator}
  The reader familiar with the lambda calculus will have noticed the
  similarity between $D$ and the paradoxical combinator.

  [Ed. note: the existence of this seems to suggest we have to be more
  restrictive on the set of processes and names we admit if we are to
  support no-cloning.]
\end{remark}

\subsubsection{Bisimulation}

The computational dynamics gives rise to another kind of equivalence,
the equivalence of computational behavior. As previously mentioned
this is typically captured \emph{via} some form of bisimulation.

% The notion we use in this paper is weak barbed bisimulation
% \cite{milner91polyadicpi}.

The notion we use in this paper is derived from weak barbed
bisimulation \cite{milner91polyadicpi}. 

\begin{definition}
An \emph{observation relation}, $\downarrow_{\mathcal N}$, over a set
of names, $\mathcal N$, is the smallest relation satisfying the rules
below.

\infrule[Out-barb]{y \in {\mathcal N}, \; x \nameeq y}
		  {\outputp{x}{v} \downarrow_{\mathcal N} x}
\infrule[Par-barb]{\mbox{$P\downarrow_{\mathcal N} x$ or $Q\downarrow_{\mathcal N} x$}}
		  {\binpar{P}{Q} \downarrow_{\mathcal N} x}

We write $P \Downarrow_{\mathcal N} x$ if there is $Q$ such that 
$P \wred Q$ and $Q \downarrow_{\mathcal N} x$.
\end{definition}

\begin{definition}
%\label{def.bbisim}
An  ${\mathcal N}$-\emph{barbed bisimulation} over a set of names, ${\mathcal N}$, is a symmetric binary relation 
${\mathcal S}_{\mathcal N}$ between agents such that $P\rel{S}_{\mathcal N}Q$ implies:
\begin{enumerate}
\item If $P \red P'$ then $Q \wred Q'$ and $P'\rel{S}_{\mathcal N} Q'$.
\item If $P\downarrow_{\mathcal N} x$, then $Q\Downarrow_{\mathcal N} x$.
\end{enumerate}
$P$ is ${\mathcal N}$-barbed bisimilar to $Q$, written
$P \wbbisim_{\mathcal N} Q$, if $P \rel{S}_{\mathcal N} Q$ for some ${\mathcal N}$-barbed bisimulation ${\mathcal S}_{\mathcal N}$.
\end{definition}

$\mathcal{R} \subseteq \pi \times \pi$

$P \mathcal{R} Q => \forall P'. P \red P' \Rightarrow \exists Q'. Q \red Q', P' \mathcal{R} Q'$

$P \vdash x \Rightarrow Q \vdash x$

\begin{mathpar}
  \inferrule*[lab=Out-barb]{x \nameeq y}{{y}!\langle{Q}\rangle \vdash x}
  \and
  \inferrule*[lab=Par-barb]{\mbox{$P\vdash x$ or $Q\vdash x$}}{\binpar{P}{Q} \vdash x}
\end{mathpar}

\subsubsection{Contexts}

One of the principle advantages of computational calculi like the
$\pi$-calculus is a well-defined notion of context,
contextual-equivalence and a correlation between
contextual-equivalence and notions of bisimulation. The notion of
context allows the decomposition of a process into (sub-)process and
its syntactic environment, its context. Thus, a context may be
thought of as a process with a ``hole'' (written $\Box$) in it. The
application of a context $M$ to a process $P$, written $M[P]$, is
tantamount to filling the hole in $M$ with $P$. In this paper we do
not need the full weight of this theory, but do make use of the notion
of context in the proof the main theorem. 

\begin{mathpar}
  \inferrule* [lab=summation] {} {{M_{M},M_{N}} \bc \Box \;|\; x.M_{A} \;|\; M_{M}+M_{N}}
  \and
  \inferrule* [lab=agent] {} {{M_{A}} \bc (\vec{x})M_{P} \;| \; \clift{P_0,\ldots,M_{P},\ldots,P_N}}
  \and \\
  \inferrule* [lab=process] {} {{M_{P}} \bc M_{N} \;| \;P|M_{P} }
\end{mathpar} 

\begin{mathpar}
  \inferrule* [lab=sychronization] {} {M_{N} \bc \Box \;|\; x?M_{F} \;|\; x!M_{C}}
  \and
  \inferrule* [lab=abstraction] {} {{M_{F}} \bc (x)M_{P} }
  \and
  \inferrule* [lab=concretion] {} {{M_{C}} \bc \langle M_{P} \rangle }
  \and \\
  \inferrule* [lab=process] {} {{M_{P}} \bc M_{N} \;| \;P|M_{P} }
\end{mathpar}

\begin{definition}[contextual application] Given a context $M$, and
  process $P$, we define the \emph{contextual application}, $M[P] :=
  M\{P/\Box\}$. That is, the contextual application of M to P is the
  substitution of $P$ for $\Box$ in $M$.
\end{definition}

$\meaningof{-} : L \to \mathcal{P}(\pi)$

\begin{mathpar}
  \inferrule* [lab=collection] {} {\meaningof{true} = \pi, \and \meaningof{~E} = \pi \setminus \meaningof{E}, \and \meaningof{E_{1} \& E_{2}} = \meaningof{E_{1}} \cap \meaningof{E_{2}}}
\end{mathpar}

\begin{mathpar}
  \inferrule* [lab=structure] {} {\meaningof{0} = \{ P \in \pi | P \equiv 0 \}, \and \\ \meaningof{E_1 | E_2} = \{ P \in \pi | P \equiv P_{1} | P_{2}, P_{1} \in \meaningof{E_{1}}, P_{2} \in \meaningof{E_2}\} }
\end{mathpar}

\begin{mathpar}
 \inferrule* [lab=behavior] {} {\meaningof{\langle a?b \rangle E} = \{ P \in \pi | P \equiv Q | u?(y)P', \\ \and \\\\ \and \\ \;\;\; u \in \meaningof{a}, \forall z.P'\{z/y\} \in \meaningof{E\{z/b\}}\}, \and \\ \meaningof{a!E} = \{ P \in \pi | P \equiv Q | x!\langle P' \rangle, x \in \meaningof{a} P' \in \meaningof{E}\} }
\end{mathpar}

\begin{mathpar}
 \inferrule* [lab=nominal] {} {\meaningof{\quotep{E}} = \{ \quotep{P} \in \quotep{\pi} | P \in \meaningof{E} \}, \and \meaningof{\quotep{P}} = \{ \quotep{Q} \in \quotep{\pi} | P \equiv Q \} \and \\ \meaningof{@\quotep{E}} = \{ P \in \pi | P \equiv @x, x \in \meaningof{E} \}}
\end{mathpar}

\begin{eqnarray*}
  \\
  \meaningof{-} : TS \to ST
\end{eqnarray*}

\begin{eqnarray*}
  \\
  L : TS \to ST
\end{eqnarray*}

\begin{eqnarray*}
  \\
  P \models E \iff P \in \meaningof{E}
\end{eqnarray*}

\begin{eqnarray*}
  P \approx_{L} Q \iff \forall E \in L. P \models E \iff Q \models E
\end{eqnarray*}

\begin{eqnarray*}
  P \approx_{K} Q
\end{eqnarray*}

\begin{eqnarray*}
  P \approx Q
\end{eqnarray*}

$\approx_{K} = \approx = \approx_{L}$

\subsubsection{Contextual duality}

Note that contexts extend the quotation operation to a family of
operations from processes to names. Given a context, $M$, we can
define a \emph{nominal context}, $\quotep{M}$ by $\quotep{M}[P] :=
\quotep{M[P]}$. To foreshadow what is to come we observe that these
operations enjoy a duality with processes very much like the duality
between vectors and maps from vectors to scalars.

Further, because the calculus is essentially higher-order, we have a
correspondence between contexts and processes. More specifically,
given a name $x$ and a context $M$ we can construct $M^{*}_{x}$ such
that 

\begin{mathpar}
  M^{*}_{x} | \lift{x}{P} \red M[P]
\end{mathpar}

namely,

\begin{mathpar}
  M^{*}_{x} := x?(u).M[\dropn{u}]
\end{mathpar}

The dependence of $M^{*}_{x}$ on a name makes it an abstraction, 

\begin{mathpar}
  M^{*} := (x)x?(u).M[\dropn{u}]
\end{mathpar}

\subsection{Additional notation}

It will sometimes be convenient to denote the process a name
quotes. We already have the notation $x = \quotep{P}$, but it will be
convenient to introduce an alternate notation, $\procn{x}$, when we
want to emphasize the connection to the use of the name. Note that, by
virtue of name equivalence, $\quotep{\procn{x}} \nameeq x$; so, the
notation is consistent with previous definitions.

Further, because names have structure it is possible to effect
substitutions on the basis of that structure. This means we need to
upgrade our notation for substitutions, which we accomplish by
adapting comprehension notation. Thus,

\begin{mathpar}
  P\{ y / x : x \in S \}
\end{mathpar}

is interpreted to mean the process derived from P by replacing (in a
capture-avoiding manner) each occurrence of $x$ in $S$ by $y$. For example,

\begin{mathpar}
  P\{ \quotep{\procn{x}|\procn{x}} / x : x \in \freenames{P} \}
\end{mathpar}

will replace each (occurrence) of a free name $x$ in $P$ by
$\quotep{\procn{x}|\procn{x}}$.

Also, we will avail ourselves of the notation $x^{L}$ and $x^{R}$ to
denote injections of a name into disjoint copies of the name
space. There are numerous ways to accomplish this. One example can be
found in \cite{MeredithR05}. This notation overloads to vectors of
names: $\vec{x}^{\pi} := (x_{i}^{\pi} \; : \; 0 \leq i < |\vec{x}| )$ where $\pi \in \{L,R\}$.

We also use $P^{\Box} := P|\Box$.

In \cite{MeredithR05} an interpretation of the new operator is
given. It turns out that there are several possible interpretations
all enjoying the requisite algebraic properties of the operator (see
\cite{milner91polyadicpi}). We will therefore make liberal use of
$(\nu\; \vec{x})P$.

% subsection the_syntax_and_semantics_of_the_notation_system (end)   

\input{qm2pi.qmops} 

\input{qm2pi.sterngerlach} 

\input{qm2pi.metric} 

% section concurrent_process_calculi (end)

%\input{qm2pi.proofsketch}

% section proof sketch (end)

%\input{qm2pi.slviaknots} 

% section spatial logic via knots (end)

\input{qm2pi.conclusion}

% section conclusion (end)

%\input{qm2pi.dtcodes} 

% section wiring algorithm (end)

\input{qm2pi.ack} 

% section acknowledgments (end)

\newpage


\bibliographystyle{plain}   
\bibliography{../../biblios/main.bib}

\input{qm2pi.rhodetails}

\end{document}

 

% section acknowledgments (end)

\newpage


\bibliographystyle{plain}   
\bibliography{../../biblios/main.bib}

\documentclass[12pt]{llncs}
%\documentclass{jktr}

\usepackage[pdftex]{hyperref}                   
\usepackage {listings}
\usepackage {mathpartir}
\usepackage{bcprules}
%\usepackage{listings}
                       
\usepackage{graphicx} 
%\usepackage[margins=2.5cm,nohead,nofoot]{geometry}
%\usepackage{geometry}
\usepackage{amsfonts}
\usepackage{amstext}
\usepackage{latexsym}
\usepackage{amssymb}
\usepackage{color}


%\include{myPreamble}
\include{qm2pi.local} 

%\ifpdf
%\usepackage[pdftex]{graphicx}
%\else
%\usepackage{graphicx}
%\fi

 % \ifpdf
%  \usepackage{pdfsync}
%  \if


%\title{Brief Article}
%\author{David F. Snyder}
%\author{L.G. Meredith}

%\address{Dept. of Math., Texas State University--San Marcos, San Marcos, TX 78666}
       
\pagestyle{empty}


\begin{document}

\lstset{language=[Objective]Caml,frame=shadowbox}

\input{qm2pi.front}

% section front matter (end)

\input{qm2pi.intro} 
 
% section introduction (end)

% \input{qm2pi.knotations} 

% section notation (end)

\input{qm2pi.process.calculi} 

% section concurrent_process_calculi_and_spatial_logics_ (end)
    
%\input{qm2pi.knots2pi} 

%\input{qm2pi.trefoil} 

%\input{qm2pi.mainthm} 

% subsection basic_interpretation (end)

%\input{qm2pi.rho.presentation} 
\subsection{The syntax and semantics of the notation system}\label{sub:the_syntax_and_semantics_of_the_notation_system} % (fold)

We now summarize a technical presentation of the calculus that
embodies our theory of dynamics. The typical presentation of such a
calculus follows the style of giving generators and relations on
them. The grammar, below, describing term constructors, freely
generates the set of processes, $\Proc$. This set is then quotiented
by a relation known as structural congruence and it is over this set
that the notion of dynamics is expressed. This presentation is
essentially that of \cite{MeredithR05} with the addition of
polyadicity and summation. For readability we have relegated some of
the technical subtleties to an appendix.

\subsubsection{Process grammar}\label{subsub:process_grammar}

\begin{mathpar}
  \inferrule* [lab=synchronization] {} {{M} \bc \pzero \;|\; x?F \;|\; x!C }
  \and
  \inferrule* [lab=abstraction] {} {{F} \bc (x)P}
  \and
  \inferrule* [lab=concretion] {} {{C} \bc \langle Q \rangle}
  \and
  \inferrule* [lab=process] {} {{P,Q} \bc M \;| \;P|Q \;|\; @{x}}
  \and
  \inferrule* [lab=name] {} {{x} \bc \quotep{P}}
\end{mathpar} 

Note that $\vec{x}$ (resp. $\vec{P}$) denotes a vector of names
(resp. processes) of length $|\vec{x}|$ (resp. $|\vec{P}|$). We adopt
the following useful abbreviations.

\begin{mathpar}
   x?(\vec{y}).P := x.(\vec{y})P \and  x\clift{\vec{P}} := x.\clift{\vec{P}}
   \and x!(y) := \lift{x}{\dropn{y}}
   \and \Pi_{i=0}^{n-1}P_i := P_0 | \ldots | P_{n-1}
\end{mathpar}

\subsubsection{Structural congruence}

\paragraph{Free and bound names and alpha-equivalence.} At the
core of structural equivalence is alpha-equivalence which identifies
process that are the same up to a change of variable. Formally, we
recognize the distinction between free and bound names. The free names
of a process, $\freenames{P}$, may be calculated recursively as
follows:

\begin{mathpar}
\freenames{\pzero} := \emptyset
  \and \\
  \freenames{x?(y).P} := \{ x \} \cup (\freenames{P} \setminus \{ y \})
  \and 
  \freenames{x!\langle P \rangle} := \{ x \} \cup \{ P \} 
  \and \\
  \freenames{P|Q} := \freenames{P} \cup \freenames{Q}
  \and \\
  \freenames{@{x}} := \{ x \}
\end{mathpar}

$\pi$
$\quotep{\pi}$

$\freenames{-} : \pi \to \mathcal{P}(\quotep{\pi})$

\begin{eqnarray*}
  \freenames{\pzero} & := & \emptyset \\
  \freenames{x?(y).P} & := & \{ x \} \cup (\freenames{P} \setminus \{ y \}) \\
  \freenames{x!\langle P \rangle} & := & \{ x \} \cup \{ P \} \\
  \freenames{P|Q} & := & \freenames{P} \cup \freenames{Q} \\
  \freenames{\dropn{x}} & := & \{ x \}
\end{eqnarray*}

The bound names of a process, $\boundnames{P}$, are those names occurring in $P$
that are not free. For example, in $x?(y).0$, the name $x$ is free, while $y$ is bound.

\begin{mathpar}
  \inferrule* [lab=monoidal-laws] {} { P|Q \equiv Q|P \and P|0 \equiv P \and P|(Q|R) \equiv (P|Q)|R }
\end{mathpar}

\begin{mathpar}
  \inferrule* [lab=alpha-equivalence] {} { (x)P \equiv (y)P\{y/x\} \and y \not\in \freenames{P} }
\end{mathpar}

\begin{definition}
Then two processes, $P,Q$, are alpha-equivalent if $P = Q\{\vec{y}/\vec{x}\}$ for
some $\vec{x} \in \boundnames{Q},\vec{y} \in \boundnames{P}$, where $Q\{\vec{y}/\vec{x}\}$
denotes the capture-avoiding substitution of $\vec{y}$ for $\vec{x}$ in $Q$.
\end{definition}

\begin{definition}
  The {\em structural congruence} \cite{SangiorgiWalker} , $\equiv$,
  between processes is the least congruence containing
  alpha-equivalence, satisfying the abelian monoid laws
  (associativity, commutativity and $\pzero$ as identity) for parallel
  composition $|$ and for summation $+$.
\end{definition}

\subsection{Name equivalence}

We take name equivalence, written $\nameeq$, to be the smallest
equivalence relation generated by the following rules.

\begin{mathpar}
\inferrule*[lab=Quote-drop]
{ }
{ \quotep{@{x}} \nameeq x }

\inferrule*[lab=Struct-equiv]
{ P \scong Q }
{ \quotep{P} \nameeq \quotep{Q} }
\end{mathpar}

The astute reader will have noticed that the mutual recursion of names
and processes imposes a mutual recursion on alpha-equivalence and
structural equivalence via name-equivalence. Fortunately, all of this
works out pleasantly and we may calculate in the natural way, free of
concern. The reader interested in the details is referred to the
appendix \ref{appendix:rho_details}.

\subsection{Substitution}

We use $\Proc$ for the set of processes, $\QProc$ for the set of
names, and $\id{\{}\vec{y} / \vec{x} \id{\}}$ to denote partial maps,
$s : \QProc \rightarrow \QProc$. A map, $s$ lifts, uniquely, to a map
on process terms, $\widehat{s} : \Proc \rightarrow \Proc$ by the
following equations.

\begin{mathpar}
  (0) \psubstp{Q}{P} := 0 \\
  (R \juxtap S) \psubstp{Q}{P}
  :=    
  (R)\psubstp{Q}{P} \juxtap (S) \psubstp{Q}{P} \\
  (x?(y).R) \psubstp{Q}{P}    
  :=    
  (x)\substp{Q}{P} (z)\concat( (R \psubstn{z}{y}) \psubstp{Q}{P} ) \\
  (\lift{x}{R}) \psubstp{Q}{P}  
  :=
  \lift{(x)\substp{Q}{P}}{ R \psubstp{Q}{P} } \\
%   (\dropn{x})  \psubstp{Q}{P}       
%   := 
%   \left\{ 
%     \begin{array}{ccc} 
%       \dropn{\quotep{Q}} & & x \nameeq \quotep{P} \\
%       \dropn{x} & & otherwise \\
%     \end{array}
%   \right. 
  (\dropn{x})  \psubstp{Q}{P}       
  := 
  \left\{ 
    \begin{array}{ccc} 
      Q & & x \nameeq \quotep{P} \\
      \dropn{x} & & otherwise \\
    \end{array}
  \right.
\end{mathpar}
 

where

\begin{eqnarray}
  (x)\id{\{} \lpquote Q \rpquote / \lpquote P \rpquote \id{\}}            = 
  \left\{ 
    \begin{array}{ccc}
      \lpquote Q \rpquote & & x \nameeq \lpquote P \rpquote \\
      x & & otherwise \\
    \end{array}
  \right. \nonumber
\end{eqnarray}

and $z$ is chosen distinct from $\quotep{P}$, $\quotep{Q}$, the free
names in $Q$, and all the names in $R$. Our $\alpha$-equivalence will
be built in the standard way from this substitution.

\begin{remark}\label{rem:no_self_referential_names}
  One consequence of these definitions is that $\forall P. \quotep{P}
  \not\in \freenames{P}$.
\end{remark}

\subsection{ Dynamic quote: an example }

Anticipating something of what's to come, consider applying the
substitution, $\widehat{\id{\{}u / z \id{\}}}$, to the following pair
of processes, $\lift{w}{y!(z)}$ and $w[ \lpquote y!(z) \rpquote ]$.

\begin{eqnarray}
	\lift{w}{y!(z)}\widehat{\id{\{}u / z \id{\}}}
		& = &
		\lift{w}{y!(u)} \nonumber\\
	w[ \lpquote y!(z) \rpquote ] \widehat{ \id{\{}u / z \id{\}} }
		& = &
		w[ \lpquote y!(z) \rpquote ] \nonumber
\end{eqnarray}

Because the body of the process between quotes is impervious to
substitution, we get radically different answers. In fact, by
examining the first process in an input context,
e.g. $x?(z).\lift{w}{y!(z)}$, we see that the process under the lift
operator may be shaped by prefixed inputs binding a name inside it. In
this sense, the lift operator will be seen as a way to dynamically
construct processes before reifying them as names.

Finally equipped with these standard features we can present the
dynamics of the calculus.

\subsubsection{Operational semantics} 

Finally, we introduce the computational dynamics. What marks these
algebras as distinct from other more traditionally studied algebraic
structures, e.g. vector spaces or polynomial rings, is the manner in
which dynamics is captured. In traditional structures, dynamics is typically
expressed through morphisms between such structures, as in linear maps
between vector spaces or morphisms between rings. In algebras
associated with the semantics of computation, the dynamics is
expressed as part of the algebraic structure itself, through a
reduction reduction relation typically denoted by $\red$. Below, we
give a recursive presentation of this relation for the calculus used
in the encoding.

$\red \subseteq \pi \times \pi$
$\red : \pi \to \mathcal{P}(\pi)$

\begin{mathpar}
  \inferrule* [lab=Comm] { \textsf{match}( x_{src}, x_{trgt} ) } { x_{trgt}?(y)P \; | \; x_{src}!\langle {Q} \rangle \red P\{\quotep{Q}/y}\} }
  \and \\
  \inferrule* [lab=Par] {{P} \red {P}'} {{{P} | {Q}} \red {{P}' | {Q}}}
  \and
  \inferrule* [lab=Equiv]{{{P} \scong {P}'} \andalso {{P}' \red {Q}'} \andalso {{Q}' \scong {Q}}}{{P} \red {Q}}
\end{mathpar}

\begin{eqnarray*}
  match_{\equiv} (\quotep{P},\quotep{Q}) & := & P \equiv Q \\
  match_{\dagger}(\quotep{P},\quotep{Q}) & := & \forall R. P|Q \red^{*} R => R \red^{*} 0 \\
  match_{K}(\quotep{P},\quotep{Q}) & := & K \mbox{ for some context } K
\end{eqnarray*}

$u?(x)P | u!\langle Q \rangle \red P\{\quotep{Q}/x\}$

%We write $\wred$ for $\red^*$, and $P\red$ if $\exists Q $ such that $ P \red Q$.
We write $P\red$ if $\exists Q $ such that $ P \red Q$ and $P\not\red$, otherwise.

\section{Replication}

As mentioned before, it is known that replication (and hence
recursion) can be implemented in a higher-order process algebra
\cite{SangiorgiWalker}. As our first example of calculation with the
machinery thus far presented we give the construction explicitly in
the {\rhoc}.

\begin{eqnarray}
	D_{x} & := & \prefix{x}{y}{(\binpar{\outputp{x}{y}}{@{y}})} \nonumber\\
	\bangp_{x}{P} & := & \binpar{{x}!\langle{\binpar{D_{x}}{P}}\rangle}{D_{x}} \nonumber
\end{eqnarray}

\begin{eqnarray}
	\bangp_{x}{P} & & \nonumber\\
	=
	& {x}!\langle{(\prefix{x}{y}{(\outputp{x}{y} | @{y})) | P}}\rangle 
	      | \prefix{x}{y}{(\outputp{x}{y} | @{y})} & \nonumber\\
	\red
	& (\outputp{x}{y} | @{y})\substn{\quotep{(\prefix{x}{y}{(@{y} | \outputp{x}{y})) | P}}}{y} & \nonumber\\
	=
	& \outputp{x}{\quotep{(\prefix{x}{y}{(\outputp{x}{y} | @{y})) | P}}}
	  | {(\prefix{x}{y}{(\outputp{x}{y} | @{y})) | P}} & \nonumber\\
	\red
	& \ldots & \nonumber\\
	\red^*
	& P | P | \ldots & \nonumber
\end{eqnarray}

Of course, this encoding, as an implementation, runs away, unfolding
$\bangp{P}$ eagerly. A lazier and more implementable replication
operator, restricted to input-guarded processes, may be obtained as follows.

\begin{eqnarray}
\bangp{\prefix{u}{v}{P}} 
	:= 
	\binpar{\lift{x}{\prefix{u}{v}{(\binpar{D(x)}{P})}}}{D(x)} \nonumber
\end{eqnarray}

\begin{remark}
  Note that the lazier definition still does not deal with summation
  or mixed summation (i.e. sums over input and output). The reader is
  invited to construct definitions of replication that deal with these
  features. 

  Further, the definitions are parameterized in a name, $x$. Can you,
  gentle reader, make a definition that eliminates this parameter and
  guarantees no accidental interaction between the replication
  machinery and the process being replicated -- i.e. no accidental
  sharing of names used by the process to get its work done and the
  name(s) used by the replication to effect copying. This latter
  revision of the definition of replication is crucial to obtaining
  the expected identity $!!P \sim !P$.
\end{remark}

\begin{remark}\label{rem:paradoxical_combinator}
  The reader familiar with the lambda calculus will have noticed the
  similarity between $D$ and the paradoxical combinator.

  [Ed. note: the existence of this seems to suggest we have to be more
  restrictive on the set of processes and names we admit if we are to
  support no-cloning.]
\end{remark}

\subsubsection{Bisimulation}

The computational dynamics gives rise to another kind of equivalence,
the equivalence of computational behavior. As previously mentioned
this is typically captured \emph{via} some form of bisimulation.

% The notion we use in this paper is weak barbed bisimulation
% \cite{milner91polyadicpi}.

The notion we use in this paper is derived from weak barbed
bisimulation \cite{milner91polyadicpi}. 

\begin{definition}
An \emph{observation relation}, $\downarrow_{\mathcal N}$, over a set
of names, $\mathcal N$, is the smallest relation satisfying the rules
below.

\infrule[Out-barb]{y \in {\mathcal N}, \; x \nameeq y}
		  {\outputp{x}{v} \downarrow_{\mathcal N} x}
\infrule[Par-barb]{\mbox{$P\downarrow_{\mathcal N} x$ or $Q\downarrow_{\mathcal N} x$}}
		  {\binpar{P}{Q} \downarrow_{\mathcal N} x}

We write $P \Downarrow_{\mathcal N} x$ if there is $Q$ such that 
$P \wred Q$ and $Q \downarrow_{\mathcal N} x$.
\end{definition}

\begin{definition}
%\label{def.bbisim}
An  ${\mathcal N}$-\emph{barbed bisimulation} over a set of names, ${\mathcal N}$, is a symmetric binary relation 
${\mathcal S}_{\mathcal N}$ between agents such that $P\rel{S}_{\mathcal N}Q$ implies:
\begin{enumerate}
\item If $P \red P'$ then $Q \wred Q'$ and $P'\rel{S}_{\mathcal N} Q'$.
\item If $P\downarrow_{\mathcal N} x$, then $Q\Downarrow_{\mathcal N} x$.
\end{enumerate}
$P$ is ${\mathcal N}$-barbed bisimilar to $Q$, written
$P \wbbisim_{\mathcal N} Q$, if $P \rel{S}_{\mathcal N} Q$ for some ${\mathcal N}$-barbed bisimulation ${\mathcal S}_{\mathcal N}$.
\end{definition}

$\mathcal{R} \subseteq \pi \times \pi$

$P \mathcal{R} Q => \forall P'. P \red P' \Rightarrow \exists Q'. Q \red Q', P' \mathcal{R} Q'$

$P \vdash x \Rightarrow Q \vdash x$

\begin{mathpar}
  \inferrule*[lab=Out-barb]{x \nameeq y}{{y}!\langle{Q}\rangle \vdash x}
  \and
  \inferrule*[lab=Par-barb]{\mbox{$P\vdash x$ or $Q\vdash x$}}{\binpar{P}{Q} \vdash x}
\end{mathpar}

\subsubsection{Contexts}

One of the principle advantages of computational calculi like the
$\pi$-calculus is a well-defined notion of context,
contextual-equivalence and a correlation between
contextual-equivalence and notions of bisimulation. The notion of
context allows the decomposition of a process into (sub-)process and
its syntactic environment, its context. Thus, a context may be
thought of as a process with a ``hole'' (written $\Box$) in it. The
application of a context $M$ to a process $P$, written $M[P]$, is
tantamount to filling the hole in $M$ with $P$. In this paper we do
not need the full weight of this theory, but do make use of the notion
of context in the proof the main theorem. 

\begin{mathpar}
  \inferrule* [lab=summation] {} {{M_{M},M_{N}} \bc \Box \;|\; x.M_{A} \;|\; M_{M}+M_{N}}
  \and
  \inferrule* [lab=agent] {} {{M_{A}} \bc (\vec{x})M_{P} \;| \; \clift{P_0,\ldots,M_{P},\ldots,P_N}}
  \and \\
  \inferrule* [lab=process] {} {{M_{P}} \bc M_{N} \;| \;P|M_{P} }
\end{mathpar} 

\begin{mathpar}
  \inferrule* [lab=sychronization] {} {M_{N} \bc \Box \;|\; x?M_{F} \;|\; x!M_{C}}
  \and
  \inferrule* [lab=abstraction] {} {{M_{F}} \bc (x)M_{P} }
  \and
  \inferrule* [lab=concretion] {} {{M_{C}} \bc \langle M_{P} \rangle }
  \and \\
  \inferrule* [lab=process] {} {{M_{P}} \bc M_{N} \;| \;P|M_{P} }
\end{mathpar}

\begin{definition}[contextual application] Given a context $M$, and
  process $P$, we define the \emph{contextual application}, $M[P] :=
  M\{P/\Box\}$. That is, the contextual application of M to P is the
  substitution of $P$ for $\Box$ in $M$.
\end{definition}

$\meaningof{-} : L \to \mathcal{P}(\pi)$

\begin{mathpar}
  \inferrule* [lab=collection] {} {\meaningof{true} = \pi, \and \meaningof{~E} = \pi \setminus \meaningof{E}, \and \meaningof{E_{1} \& E_{2}} = \meaningof{E_{1}} \cap \meaningof{E_{2}}}
\end{mathpar}

\begin{mathpar}
  \inferrule* [lab=structure] {} {\meaningof{0} = \{ P \in \pi | P \equiv 0 \}, \and \\ \meaningof{E_1 | E_2} = \{ P \in \pi | P \equiv P_{1} | P_{2}, P_{1} \in \meaningof{E_{1}}, P_{2} \in \meaningof{E_2}\} }
\end{mathpar}

\begin{mathpar}
 \inferrule* [lab=behavior] {} {\meaningof{\langle a?b \rangle E} = \{ P \in \pi | P \equiv Q | u?(y)P', \\ \and \\\\ \and \\ \;\;\; u \in \meaningof{a}, \forall z.P'\{z/y\} \in \meaningof{E\{z/b\}}\}, \and \\ \meaningof{a!E} = \{ P \in \pi | P \equiv Q | x!\langle P' \rangle, x \in \meaningof{a} P' \in \meaningof{E}\} }
\end{mathpar}

\begin{mathpar}
 \inferrule* [lab=nominal] {} {\meaningof{\quotep{E}} = \{ \quotep{P} \in \quotep{\pi} | P \in \meaningof{E} \}, \and \meaningof{\quotep{P}} = \{ \quotep{Q} \in \quotep{\pi} | P \equiv Q \} \and \\ \meaningof{@\quotep{E}} = \{ P \in \pi | P \equiv @x, x \in \meaningof{E} \}}
\end{mathpar}

\begin{eqnarray*}
  \\
  \meaningof{-} : TS \to ST
\end{eqnarray*}

\begin{eqnarray*}
  \\
  L : TS \to ST
\end{eqnarray*}

\begin{eqnarray*}
  \\
  P \models E \iff P \in \meaningof{E}
\end{eqnarray*}

\begin{eqnarray*}
  P \approx_{L} Q \iff \forall E \in L. P \models E \iff Q \models E
\end{eqnarray*}

\begin{eqnarray*}
  P \approx_{K} Q
\end{eqnarray*}

\begin{eqnarray*}
  P \approx Q
\end{eqnarray*}

$\approx_{K} = \approx = \approx_{L}$

\subsubsection{Contextual duality}

Note that contexts extend the quotation operation to a family of
operations from processes to names. Given a context, $M$, we can
define a \emph{nominal context}, $\quotep{M}$ by $\quotep{M}[P] :=
\quotep{M[P]}$. To foreshadow what is to come we observe that these
operations enjoy a duality with processes very much like the duality
between vectors and maps from vectors to scalars.

Further, because the calculus is essentially higher-order, we have a
correspondence between contexts and processes. More specifically,
given a name $x$ and a context $M$ we can construct $M^{*}_{x}$ such
that 

\begin{mathpar}
  M^{*}_{x} | \lift{x}{P} \red M[P]
\end{mathpar}

namely,

\begin{mathpar}
  M^{*}_{x} := x?(u).M[\dropn{u}]
\end{mathpar}

The dependence of $M^{*}_{x}$ on a name makes it an abstraction, 

\begin{mathpar}
  M^{*} := (x)x?(u).M[\dropn{u}]
\end{mathpar}

\subsection{Additional notation}

It will sometimes be convenient to denote the process a name
quotes. We already have the notation $x = \quotep{P}$, but it will be
convenient to introduce an alternate notation, $\procn{x}$, when we
want to emphasize the connection to the use of the name. Note that, by
virtue of name equivalence, $\quotep{\procn{x}} \nameeq x$; so, the
notation is consistent with previous definitions.

Further, because names have structure it is possible to effect
substitutions on the basis of that structure. This means we need to
upgrade our notation for substitutions, which we accomplish by
adapting comprehension notation. Thus,

\begin{mathpar}
  P\{ y / x : x \in S \}
\end{mathpar}

is interpreted to mean the process derived from P by replacing (in a
capture-avoiding manner) each occurrence of $x$ in $S$ by $y$. For example,

\begin{mathpar}
  P\{ \quotep{\procn{x}|\procn{x}} / x : x \in \freenames{P} \}
\end{mathpar}

will replace each (occurrence) of a free name $x$ in $P$ by
$\quotep{\procn{x}|\procn{x}}$.

Also, we will avail ourselves of the notation $x^{L}$ and $x^{R}$ to
denote injections of a name into disjoint copies of the name
space. There are numerous ways to accomplish this. One example can be
found in \cite{MeredithR05}. This notation overloads to vectors of
names: $\vec{x}^{\pi} := (x_{i}^{\pi} \; : \; 0 \leq i < |\vec{x}| )$ where $\pi \in \{L,R\}$.

We also use $P^{\Box} := P|\Box$.

In \cite{MeredithR05} an interpretation of the new operator is
given. It turns out that there are several possible interpretations
all enjoying the requisite algebraic properties of the operator (see
\cite{milner91polyadicpi}). We will therefore make liberal use of
$(\nu\; \vec{x})P$.

% subsection the_syntax_and_semantics_of_the_notation_system (end)   

\input{qm2pi.qmops} 

\input{qm2pi.sterngerlach} 

\input{qm2pi.metric} 

% section concurrent_process_calculi (end)

%\input{qm2pi.proofsketch}

% section proof sketch (end)

%\input{qm2pi.slviaknots} 

% section spatial logic via knots (end)

\input{qm2pi.conclusion}

% section conclusion (end)

%\input{qm2pi.dtcodes} 

% section wiring algorithm (end)

\input{qm2pi.ack} 

% section acknowledgments (end)

\newpage


\bibliographystyle{plain}   
\bibliography{../../biblios/main.bib}

\input{qm2pi.rhodetails}

\end{document}



\end{document}

 

% section concurrent_process_calculi (end)

%\documentclass[12pt]{llncs}
%\documentclass{jktr}

\usepackage[pdftex]{hyperref}                   
\usepackage {listings}
\usepackage {mathpartir}
\usepackage{bcprules}
%\usepackage{listings}
                       
\usepackage{graphicx} 
%\usepackage[margins=2.5cm,nohead,nofoot]{geometry}
%\usepackage{geometry}
\usepackage{amsfonts}
\usepackage{amstext}
\usepackage{latexsym}
\usepackage{amssymb}
\usepackage{color}


%\include{myPreamble}
\documentclass[12pt]{llncs}
%\documentclass{jktr}

\usepackage[pdftex]{hyperref}                   
\usepackage {listings}
\usepackage {mathpartir}
\usepackage{bcprules}
%\usepackage{listings}
                       
\usepackage{graphicx} 
%\usepackage[margins=2.5cm,nohead,nofoot]{geometry}
%\usepackage{geometry}
\usepackage{amsfonts}
\usepackage{amstext}
\usepackage{latexsym}
\usepackage{amssymb}
\usepackage{color}


%\include{myPreamble}
\include{qm2pi.local} 

%\ifpdf
%\usepackage[pdftex]{graphicx}
%\else
%\usepackage{graphicx}
%\fi

 % \ifpdf
%  \usepackage{pdfsync}
%  \if


%\title{Brief Article}
%\author{David F. Snyder}
%\author{L.G. Meredith}

%\address{Dept. of Math., Texas State University--San Marcos, San Marcos, TX 78666}
       
\pagestyle{empty}


\begin{document}

\lstset{language=[Objective]Caml,frame=shadowbox}

\input{qm2pi.front}

% section front matter (end)

\input{qm2pi.intro} 
 
% section introduction (end)

% \input{qm2pi.knotations} 

% section notation (end)

\input{qm2pi.process.calculi} 

% section concurrent_process_calculi_and_spatial_logics_ (end)
    
%\input{qm2pi.knots2pi} 

%\input{qm2pi.trefoil} 

%\input{qm2pi.mainthm} 

% subsection basic_interpretation (end)

%\input{qm2pi.rho.presentation} 
\subsection{The syntax and semantics of the notation system}\label{sub:the_syntax_and_semantics_of_the_notation_system} % (fold)

We now summarize a technical presentation of the calculus that
embodies our theory of dynamics. The typical presentation of such a
calculus follows the style of giving generators and relations on
them. The grammar, below, describing term constructors, freely
generates the set of processes, $\Proc$. This set is then quotiented
by a relation known as structural congruence and it is over this set
that the notion of dynamics is expressed. This presentation is
essentially that of \cite{MeredithR05} with the addition of
polyadicity and summation. For readability we have relegated some of
the technical subtleties to an appendix.

\subsubsection{Process grammar}\label{subsub:process_grammar}

\begin{mathpar}
  \inferrule* [lab=synchronization] {} {{M} \bc \pzero \;|\; x?F \;|\; x!C }
  \and
  \inferrule* [lab=abstraction] {} {{F} \bc (x)P}
  \and
  \inferrule* [lab=concretion] {} {{C} \bc \langle Q \rangle}
  \and
  \inferrule* [lab=process] {} {{P,Q} \bc M \;| \;P|Q \;|\; @{x}}
  \and
  \inferrule* [lab=name] {} {{x} \bc \quotep{P}}
\end{mathpar} 

Note that $\vec{x}$ (resp. $\vec{P}$) denotes a vector of names
(resp. processes) of length $|\vec{x}|$ (resp. $|\vec{P}|$). We adopt
the following useful abbreviations.

\begin{mathpar}
   x?(\vec{y}).P := x.(\vec{y})P \and  x\clift{\vec{P}} := x.\clift{\vec{P}}
   \and x!(y) := \lift{x}{\dropn{y}}
   \and \Pi_{i=0}^{n-1}P_i := P_0 | \ldots | P_{n-1}
\end{mathpar}

\subsubsection{Structural congruence}

\paragraph{Free and bound names and alpha-equivalence.} At the
core of structural equivalence is alpha-equivalence which identifies
process that are the same up to a change of variable. Formally, we
recognize the distinction between free and bound names. The free names
of a process, $\freenames{P}$, may be calculated recursively as
follows:

\begin{mathpar}
\freenames{\pzero} := \emptyset
  \and \\
  \freenames{x?(y).P} := \{ x \} \cup (\freenames{P} \setminus \{ y \})
  \and 
  \freenames{x!\langle P \rangle} := \{ x \} \cup \{ P \} 
  \and \\
  \freenames{P|Q} := \freenames{P} \cup \freenames{Q}
  \and \\
  \freenames{@{x}} := \{ x \}
\end{mathpar}

$\pi$
$\quotep{\pi}$

$\freenames{-} : \pi \to \mathcal{P}(\quotep{\pi})$

\begin{eqnarray*}
  \freenames{\pzero} & := & \emptyset \\
  \freenames{x?(y).P} & := & \{ x \} \cup (\freenames{P} \setminus \{ y \}) \\
  \freenames{x!\langle P \rangle} & := & \{ x \} \cup \{ P \} \\
  \freenames{P|Q} & := & \freenames{P} \cup \freenames{Q} \\
  \freenames{\dropn{x}} & := & \{ x \}
\end{eqnarray*}

The bound names of a process, $\boundnames{P}$, are those names occurring in $P$
that are not free. For example, in $x?(y).0$, the name $x$ is free, while $y$ is bound.

\begin{mathpar}
  \inferrule* [lab=monoidal-laws] {} { P|Q \equiv Q|P \and P|0 \equiv P \and P|(Q|R) \equiv (P|Q)|R }
\end{mathpar}

\begin{mathpar}
  \inferrule* [lab=alpha-equivalence] {} { (x)P \equiv (y)P\{y/x\} \and y \not\in \freenames{P} }
\end{mathpar}

\begin{definition}
Then two processes, $P,Q$, are alpha-equivalent if $P = Q\{\vec{y}/\vec{x}\}$ for
some $\vec{x} \in \boundnames{Q},\vec{y} \in \boundnames{P}$, where $Q\{\vec{y}/\vec{x}\}$
denotes the capture-avoiding substitution of $\vec{y}$ for $\vec{x}$ in $Q$.
\end{definition}

\begin{definition}
  The {\em structural congruence} \cite{SangiorgiWalker} , $\equiv$,
  between processes is the least congruence containing
  alpha-equivalence, satisfying the abelian monoid laws
  (associativity, commutativity and $\pzero$ as identity) for parallel
  composition $|$ and for summation $+$.
\end{definition}

\subsection{Name equivalence}

We take name equivalence, written $\nameeq$, to be the smallest
equivalence relation generated by the following rules.

\begin{mathpar}
\inferrule*[lab=Quote-drop]
{ }
{ \quotep{@{x}} \nameeq x }

\inferrule*[lab=Struct-equiv]
{ P \scong Q }
{ \quotep{P} \nameeq \quotep{Q} }
\end{mathpar}

The astute reader will have noticed that the mutual recursion of names
and processes imposes a mutual recursion on alpha-equivalence and
structural equivalence via name-equivalence. Fortunately, all of this
works out pleasantly and we may calculate in the natural way, free of
concern. The reader interested in the details is referred to the
appendix \ref{appendix:rho_details}.

\subsection{Substitution}

We use $\Proc$ for the set of processes, $\QProc$ for the set of
names, and $\id{\{}\vec{y} / \vec{x} \id{\}}$ to denote partial maps,
$s : \QProc \rightarrow \QProc$. A map, $s$ lifts, uniquely, to a map
on process terms, $\widehat{s} : \Proc \rightarrow \Proc$ by the
following equations.

\begin{mathpar}
  (0) \psubstp{Q}{P} := 0 \\
  (R \juxtap S) \psubstp{Q}{P}
  :=    
  (R)\psubstp{Q}{P} \juxtap (S) \psubstp{Q}{P} \\
  (x?(y).R) \psubstp{Q}{P}    
  :=    
  (x)\substp{Q}{P} (z)\concat( (R \psubstn{z}{y}) \psubstp{Q}{P} ) \\
  (\lift{x}{R}) \psubstp{Q}{P}  
  :=
  \lift{(x)\substp{Q}{P}}{ R \psubstp{Q}{P} } \\
%   (\dropn{x})  \psubstp{Q}{P}       
%   := 
%   \left\{ 
%     \begin{array}{ccc} 
%       \dropn{\quotep{Q}} & & x \nameeq \quotep{P} \\
%       \dropn{x} & & otherwise \\
%     \end{array}
%   \right. 
  (\dropn{x})  \psubstp{Q}{P}       
  := 
  \left\{ 
    \begin{array}{ccc} 
      Q & & x \nameeq \quotep{P} \\
      \dropn{x} & & otherwise \\
    \end{array}
  \right.
\end{mathpar}
 

where

\begin{eqnarray}
  (x)\id{\{} \lpquote Q \rpquote / \lpquote P \rpquote \id{\}}            = 
  \left\{ 
    \begin{array}{ccc}
      \lpquote Q \rpquote & & x \nameeq \lpquote P \rpquote \\
      x & & otherwise \\
    \end{array}
  \right. \nonumber
\end{eqnarray}

and $z$ is chosen distinct from $\quotep{P}$, $\quotep{Q}$, the free
names in $Q$, and all the names in $R$. Our $\alpha$-equivalence will
be built in the standard way from this substitution.

\begin{remark}\label{rem:no_self_referential_names}
  One consequence of these definitions is that $\forall P. \quotep{P}
  \not\in \freenames{P}$.
\end{remark}

\subsection{ Dynamic quote: an example }

Anticipating something of what's to come, consider applying the
substitution, $\widehat{\id{\{}u / z \id{\}}}$, to the following pair
of processes, $\lift{w}{y!(z)}$ and $w[ \lpquote y!(z) \rpquote ]$.

\begin{eqnarray}
	\lift{w}{y!(z)}\widehat{\id{\{}u / z \id{\}}}
		& = &
		\lift{w}{y!(u)} \nonumber\\
	w[ \lpquote y!(z) \rpquote ] \widehat{ \id{\{}u / z \id{\}} }
		& = &
		w[ \lpquote y!(z) \rpquote ] \nonumber
\end{eqnarray}

Because the body of the process between quotes is impervious to
substitution, we get radically different answers. In fact, by
examining the first process in an input context,
e.g. $x?(z).\lift{w}{y!(z)}$, we see that the process under the lift
operator may be shaped by prefixed inputs binding a name inside it. In
this sense, the lift operator will be seen as a way to dynamically
construct processes before reifying them as names.

Finally equipped with these standard features we can present the
dynamics of the calculus.

\subsubsection{Operational semantics} 

Finally, we introduce the computational dynamics. What marks these
algebras as distinct from other more traditionally studied algebraic
structures, e.g. vector spaces or polynomial rings, is the manner in
which dynamics is captured. In traditional structures, dynamics is typically
expressed through morphisms between such structures, as in linear maps
between vector spaces or morphisms between rings. In algebras
associated with the semantics of computation, the dynamics is
expressed as part of the algebraic structure itself, through a
reduction reduction relation typically denoted by $\red$. Below, we
give a recursive presentation of this relation for the calculus used
in the encoding.

$\red \subseteq \pi \times \pi$
$\red : \pi \to \mathcal{P}(\pi)$

\begin{mathpar}
  \inferrule* [lab=Comm] { \textsf{match}( x_{src}, x_{trgt} ) } { x_{trgt}?(y)P \; | \; x_{src}!\langle {Q} \rangle \red P\{\quotep{Q}/y}\} }
  \and \\
  \inferrule* [lab=Par] {{P} \red {P}'} {{{P} | {Q}} \red {{P}' | {Q}}}
  \and
  \inferrule* [lab=Equiv]{{{P} \scong {P}'} \andalso {{P}' \red {Q}'} \andalso {{Q}' \scong {Q}}}{{P} \red {Q}}
\end{mathpar}

\begin{eqnarray*}
  match_{\equiv} (\quotep{P},\quotep{Q}) & := & P \equiv Q \\
  match_{\dagger}(\quotep{P},\quotep{Q}) & := & \forall R. P|Q \red^{*} R => R \red^{*} 0 \\
  match_{K}(\quotep{P},\quotep{Q}) & := & K \mbox{ for some context } K
\end{eqnarray*}

$u?(x)P | u!\langle Q \rangle \red P\{\quotep{Q}/x\}$

%We write $\wred$ for $\red^*$, and $P\red$ if $\exists Q $ such that $ P \red Q$.
We write $P\red$ if $\exists Q $ such that $ P \red Q$ and $P\not\red$, otherwise.

\section{Replication}

As mentioned before, it is known that replication (and hence
recursion) can be implemented in a higher-order process algebra
\cite{SangiorgiWalker}. As our first example of calculation with the
machinery thus far presented we give the construction explicitly in
the {\rhoc}.

\begin{eqnarray}
	D_{x} & := & \prefix{x}{y}{(\binpar{\outputp{x}{y}}{@{y}})} \nonumber\\
	\bangp_{x}{P} & := & \binpar{{x}!\langle{\binpar{D_{x}}{P}}\rangle}{D_{x}} \nonumber
\end{eqnarray}

\begin{eqnarray}
	\bangp_{x}{P} & & \nonumber\\
	=
	& {x}!\langle{(\prefix{x}{y}{(\outputp{x}{y} | @{y})) | P}}\rangle 
	      | \prefix{x}{y}{(\outputp{x}{y} | @{y})} & \nonumber\\
	\red
	& (\outputp{x}{y} | @{y})\substn{\quotep{(\prefix{x}{y}{(@{y} | \outputp{x}{y})) | P}}}{y} & \nonumber\\
	=
	& \outputp{x}{\quotep{(\prefix{x}{y}{(\outputp{x}{y} | @{y})) | P}}}
	  | {(\prefix{x}{y}{(\outputp{x}{y} | @{y})) | P}} & \nonumber\\
	\red
	& \ldots & \nonumber\\
	\red^*
	& P | P | \ldots & \nonumber
\end{eqnarray}

Of course, this encoding, as an implementation, runs away, unfolding
$\bangp{P}$ eagerly. A lazier and more implementable replication
operator, restricted to input-guarded processes, may be obtained as follows.

\begin{eqnarray}
\bangp{\prefix{u}{v}{P}} 
	:= 
	\binpar{\lift{x}{\prefix{u}{v}{(\binpar{D(x)}{P})}}}{D(x)} \nonumber
\end{eqnarray}

\begin{remark}
  Note that the lazier definition still does not deal with summation
  or mixed summation (i.e. sums over input and output). The reader is
  invited to construct definitions of replication that deal with these
  features. 

  Further, the definitions are parameterized in a name, $x$. Can you,
  gentle reader, make a definition that eliminates this parameter and
  guarantees no accidental interaction between the replication
  machinery and the process being replicated -- i.e. no accidental
  sharing of names used by the process to get its work done and the
  name(s) used by the replication to effect copying. This latter
  revision of the definition of replication is crucial to obtaining
  the expected identity $!!P \sim !P$.
\end{remark}

\begin{remark}\label{rem:paradoxical_combinator}
  The reader familiar with the lambda calculus will have noticed the
  similarity between $D$ and the paradoxical combinator.

  [Ed. note: the existence of this seems to suggest we have to be more
  restrictive on the set of processes and names we admit if we are to
  support no-cloning.]
\end{remark}

\subsubsection{Bisimulation}

The computational dynamics gives rise to another kind of equivalence,
the equivalence of computational behavior. As previously mentioned
this is typically captured \emph{via} some form of bisimulation.

% The notion we use in this paper is weak barbed bisimulation
% \cite{milner91polyadicpi}.

The notion we use in this paper is derived from weak barbed
bisimulation \cite{milner91polyadicpi}. 

\begin{definition}
An \emph{observation relation}, $\downarrow_{\mathcal N}$, over a set
of names, $\mathcal N$, is the smallest relation satisfying the rules
below.

\infrule[Out-barb]{y \in {\mathcal N}, \; x \nameeq y}
		  {\outputp{x}{v} \downarrow_{\mathcal N} x}
\infrule[Par-barb]{\mbox{$P\downarrow_{\mathcal N} x$ or $Q\downarrow_{\mathcal N} x$}}
		  {\binpar{P}{Q} \downarrow_{\mathcal N} x}

We write $P \Downarrow_{\mathcal N} x$ if there is $Q$ such that 
$P \wred Q$ and $Q \downarrow_{\mathcal N} x$.
\end{definition}

\begin{definition}
%\label{def.bbisim}
An  ${\mathcal N}$-\emph{barbed bisimulation} over a set of names, ${\mathcal N}$, is a symmetric binary relation 
${\mathcal S}_{\mathcal N}$ between agents such that $P\rel{S}_{\mathcal N}Q$ implies:
\begin{enumerate}
\item If $P \red P'$ then $Q \wred Q'$ and $P'\rel{S}_{\mathcal N} Q'$.
\item If $P\downarrow_{\mathcal N} x$, then $Q\Downarrow_{\mathcal N} x$.
\end{enumerate}
$P$ is ${\mathcal N}$-barbed bisimilar to $Q$, written
$P \wbbisim_{\mathcal N} Q$, if $P \rel{S}_{\mathcal N} Q$ for some ${\mathcal N}$-barbed bisimulation ${\mathcal S}_{\mathcal N}$.
\end{definition}

$\mathcal{R} \subseteq \pi \times \pi$

$P \mathcal{R} Q => \forall P'. P \red P' \Rightarrow \exists Q'. Q \red Q', P' \mathcal{R} Q'$

$P \vdash x \Rightarrow Q \vdash x$

\begin{mathpar}
  \inferrule*[lab=Out-barb]{x \nameeq y}{{y}!\langle{Q}\rangle \vdash x}
  \and
  \inferrule*[lab=Par-barb]{\mbox{$P\vdash x$ or $Q\vdash x$}}{\binpar{P}{Q} \vdash x}
\end{mathpar}

\subsubsection{Contexts}

One of the principle advantages of computational calculi like the
$\pi$-calculus is a well-defined notion of context,
contextual-equivalence and a correlation between
contextual-equivalence and notions of bisimulation. The notion of
context allows the decomposition of a process into (sub-)process and
its syntactic environment, its context. Thus, a context may be
thought of as a process with a ``hole'' (written $\Box$) in it. The
application of a context $M$ to a process $P$, written $M[P]$, is
tantamount to filling the hole in $M$ with $P$. In this paper we do
not need the full weight of this theory, but do make use of the notion
of context in the proof the main theorem. 

\begin{mathpar}
  \inferrule* [lab=summation] {} {{M_{M},M_{N}} \bc \Box \;|\; x.M_{A} \;|\; M_{M}+M_{N}}
  \and
  \inferrule* [lab=agent] {} {{M_{A}} \bc (\vec{x})M_{P} \;| \; \clift{P_0,\ldots,M_{P},\ldots,P_N}}
  \and \\
  \inferrule* [lab=process] {} {{M_{P}} \bc M_{N} \;| \;P|M_{P} }
\end{mathpar} 

\begin{mathpar}
  \inferrule* [lab=sychronization] {} {M_{N} \bc \Box \;|\; x?M_{F} \;|\; x!M_{C}}
  \and
  \inferrule* [lab=abstraction] {} {{M_{F}} \bc (x)M_{P} }
  \and
  \inferrule* [lab=concretion] {} {{M_{C}} \bc \langle M_{P} \rangle }
  \and \\
  \inferrule* [lab=process] {} {{M_{P}} \bc M_{N} \;| \;P|M_{P} }
\end{mathpar}

\begin{definition}[contextual application] Given a context $M$, and
  process $P$, we define the \emph{contextual application}, $M[P] :=
  M\{P/\Box\}$. That is, the contextual application of M to P is the
  substitution of $P$ for $\Box$ in $M$.
\end{definition}

$\meaningof{-} : L \to \mathcal{P}(\pi)$

\begin{mathpar}
  \inferrule* [lab=collection] {} {\meaningof{true} = \pi, \and \meaningof{~E} = \pi \setminus \meaningof{E}, \and \meaningof{E_{1} \& E_{2}} = \meaningof{E_{1}} \cap \meaningof{E_{2}}}
\end{mathpar}

\begin{mathpar}
  \inferrule* [lab=structure] {} {\meaningof{0} = \{ P \in \pi | P \equiv 0 \}, \and \\ \meaningof{E_1 | E_2} = \{ P \in \pi | P \equiv P_{1} | P_{2}, P_{1} \in \meaningof{E_{1}}, P_{2} \in \meaningof{E_2}\} }
\end{mathpar}

\begin{mathpar}
 \inferrule* [lab=behavior] {} {\meaningof{\langle a?b \rangle E} = \{ P \in \pi | P \equiv Q | u?(y)P', \\ \and \\\\ \and \\ \;\;\; u \in \meaningof{a}, \forall z.P'\{z/y\} \in \meaningof{E\{z/b\}}\}, \and \\ \meaningof{a!E} = \{ P \in \pi | P \equiv Q | x!\langle P' \rangle, x \in \meaningof{a} P' \in \meaningof{E}\} }
\end{mathpar}

\begin{mathpar}
 \inferrule* [lab=nominal] {} {\meaningof{\quotep{E}} = \{ \quotep{P} \in \quotep{\pi} | P \in \meaningof{E} \}, \and \meaningof{\quotep{P}} = \{ \quotep{Q} \in \quotep{\pi} | P \equiv Q \} \and \\ \meaningof{@\quotep{E}} = \{ P \in \pi | P \equiv @x, x \in \meaningof{E} \}}
\end{mathpar}

\begin{eqnarray*}
  \\
  \meaningof{-} : TS \to ST
\end{eqnarray*}

\begin{eqnarray*}
  \\
  L : TS \to ST
\end{eqnarray*}

\begin{eqnarray*}
  \\
  P \models E \iff P \in \meaningof{E}
\end{eqnarray*}

\begin{eqnarray*}
  P \approx_{L} Q \iff \forall E \in L. P \models E \iff Q \models E
\end{eqnarray*}

\begin{eqnarray*}
  P \approx_{K} Q
\end{eqnarray*}

\begin{eqnarray*}
  P \approx Q
\end{eqnarray*}

$\approx_{K} = \approx = \approx_{L}$

\subsubsection{Contextual duality}

Note that contexts extend the quotation operation to a family of
operations from processes to names. Given a context, $M$, we can
define a \emph{nominal context}, $\quotep{M}$ by $\quotep{M}[P] :=
\quotep{M[P]}$. To foreshadow what is to come we observe that these
operations enjoy a duality with processes very much like the duality
between vectors and maps from vectors to scalars.

Further, because the calculus is essentially higher-order, we have a
correspondence between contexts and processes. More specifically,
given a name $x$ and a context $M$ we can construct $M^{*}_{x}$ such
that 

\begin{mathpar}
  M^{*}_{x} | \lift{x}{P} \red M[P]
\end{mathpar}

namely,

\begin{mathpar}
  M^{*}_{x} := x?(u).M[\dropn{u}]
\end{mathpar}

The dependence of $M^{*}_{x}$ on a name makes it an abstraction, 

\begin{mathpar}
  M^{*} := (x)x?(u).M[\dropn{u}]
\end{mathpar}

\subsection{Additional notation}

It will sometimes be convenient to denote the process a name
quotes. We already have the notation $x = \quotep{P}$, but it will be
convenient to introduce an alternate notation, $\procn{x}$, when we
want to emphasize the connection to the use of the name. Note that, by
virtue of name equivalence, $\quotep{\procn{x}} \nameeq x$; so, the
notation is consistent with previous definitions.

Further, because names have structure it is possible to effect
substitutions on the basis of that structure. This means we need to
upgrade our notation for substitutions, which we accomplish by
adapting comprehension notation. Thus,

\begin{mathpar}
  P\{ y / x : x \in S \}
\end{mathpar}

is interpreted to mean the process derived from P by replacing (in a
capture-avoiding manner) each occurrence of $x$ in $S$ by $y$. For example,

\begin{mathpar}
  P\{ \quotep{\procn{x}|\procn{x}} / x : x \in \freenames{P} \}
\end{mathpar}

will replace each (occurrence) of a free name $x$ in $P$ by
$\quotep{\procn{x}|\procn{x}}$.

Also, we will avail ourselves of the notation $x^{L}$ and $x^{R}$ to
denote injections of a name into disjoint copies of the name
space. There are numerous ways to accomplish this. One example can be
found in \cite{MeredithR05}. This notation overloads to vectors of
names: $\vec{x}^{\pi} := (x_{i}^{\pi} \; : \; 0 \leq i < |\vec{x}| )$ where $\pi \in \{L,R\}$.

We also use $P^{\Box} := P|\Box$.

In \cite{MeredithR05} an interpretation of the new operator is
given. It turns out that there are several possible interpretations
all enjoying the requisite algebraic properties of the operator (see
\cite{milner91polyadicpi}). We will therefore make liberal use of
$(\nu\; \vec{x})P$.

% subsection the_syntax_and_semantics_of_the_notation_system (end)   

\input{qm2pi.qmops} 

\input{qm2pi.sterngerlach} 

\input{qm2pi.metric} 

% section concurrent_process_calculi (end)

%\input{qm2pi.proofsketch}

% section proof sketch (end)

%\input{qm2pi.slviaknots} 

% section spatial logic via knots (end)

\input{qm2pi.conclusion}

% section conclusion (end)

%\input{qm2pi.dtcodes} 

% section wiring algorithm (end)

\input{qm2pi.ack} 

% section acknowledgments (end)

\newpage


\bibliographystyle{plain}   
\bibliography{../../biblios/main.bib}

\input{qm2pi.rhodetails}

\end{document}

 

%\ifpdf
%\usepackage[pdftex]{graphicx}
%\else
%\usepackage{graphicx}
%\fi

 % \ifpdf
%  \usepackage{pdfsync}
%  \if


%\title{Brief Article}
%\author{David F. Snyder}
%\author{L.G. Meredith}

%\address{Dept. of Math., Texas State University--San Marcos, San Marcos, TX 78666}
       
\pagestyle{empty}


\begin{document}

\lstset{language=[Objective]Caml,frame=shadowbox}

\documentclass[12pt]{llncs}
%\documentclass{jktr}

\usepackage[pdftex]{hyperref}                   
\usepackage {listings}
\usepackage {mathpartir}
\usepackage{bcprules}
%\usepackage{listings}
                       
\usepackage{graphicx} 
%\usepackage[margins=2.5cm,nohead,nofoot]{geometry}
%\usepackage{geometry}
\usepackage{amsfonts}
\usepackage{amstext}
\usepackage{latexsym}
\usepackage{amssymb}
\usepackage{color}


%\include{myPreamble}
\include{qm2pi.local} 

%\ifpdf
%\usepackage[pdftex]{graphicx}
%\else
%\usepackage{graphicx}
%\fi

 % \ifpdf
%  \usepackage{pdfsync}
%  \if


%\title{Brief Article}
%\author{David F. Snyder}
%\author{L.G. Meredith}

%\address{Dept. of Math., Texas State University--San Marcos, San Marcos, TX 78666}
       
\pagestyle{empty}


\begin{document}

\lstset{language=[Objective]Caml,frame=shadowbox}

\input{qm2pi.front}

% section front matter (end)

\input{qm2pi.intro} 
 
% section introduction (end)

% \input{qm2pi.knotations} 

% section notation (end)

\input{qm2pi.process.calculi} 

% section concurrent_process_calculi_and_spatial_logics_ (end)
    
%\input{qm2pi.knots2pi} 

%\input{qm2pi.trefoil} 

%\input{qm2pi.mainthm} 

% subsection basic_interpretation (end)

%\input{qm2pi.rho.presentation} 
\subsection{The syntax and semantics of the notation system}\label{sub:the_syntax_and_semantics_of_the_notation_system} % (fold)

We now summarize a technical presentation of the calculus that
embodies our theory of dynamics. The typical presentation of such a
calculus follows the style of giving generators and relations on
them. The grammar, below, describing term constructors, freely
generates the set of processes, $\Proc$. This set is then quotiented
by a relation known as structural congruence and it is over this set
that the notion of dynamics is expressed. This presentation is
essentially that of \cite{MeredithR05} with the addition of
polyadicity and summation. For readability we have relegated some of
the technical subtleties to an appendix.

\subsubsection{Process grammar}\label{subsub:process_grammar}

\begin{mathpar}
  \inferrule* [lab=synchronization] {} {{M} \bc \pzero \;|\; x?F \;|\; x!C }
  \and
  \inferrule* [lab=abstraction] {} {{F} \bc (x)P}
  \and
  \inferrule* [lab=concretion] {} {{C} \bc \langle Q \rangle}
  \and
  \inferrule* [lab=process] {} {{P,Q} \bc M \;| \;P|Q \;|\; @{x}}
  \and
  \inferrule* [lab=name] {} {{x} \bc \quotep{P}}
\end{mathpar} 

Note that $\vec{x}$ (resp. $\vec{P}$) denotes a vector of names
(resp. processes) of length $|\vec{x}|$ (resp. $|\vec{P}|$). We adopt
the following useful abbreviations.

\begin{mathpar}
   x?(\vec{y}).P := x.(\vec{y})P \and  x\clift{\vec{P}} := x.\clift{\vec{P}}
   \and x!(y) := \lift{x}{\dropn{y}}
   \and \Pi_{i=0}^{n-1}P_i := P_0 | \ldots | P_{n-1}
\end{mathpar}

\subsubsection{Structural congruence}

\paragraph{Free and bound names and alpha-equivalence.} At the
core of structural equivalence is alpha-equivalence which identifies
process that are the same up to a change of variable. Formally, we
recognize the distinction between free and bound names. The free names
of a process, $\freenames{P}$, may be calculated recursively as
follows:

\begin{mathpar}
\freenames{\pzero} := \emptyset
  \and \\
  \freenames{x?(y).P} := \{ x \} \cup (\freenames{P} \setminus \{ y \})
  \and 
  \freenames{x!\langle P \rangle} := \{ x \} \cup \{ P \} 
  \and \\
  \freenames{P|Q} := \freenames{P} \cup \freenames{Q}
  \and \\
  \freenames{@{x}} := \{ x \}
\end{mathpar}

$\pi$
$\quotep{\pi}$

$\freenames{-} : \pi \to \mathcal{P}(\quotep{\pi})$

\begin{eqnarray*}
  \freenames{\pzero} & := & \emptyset \\
  \freenames{x?(y).P} & := & \{ x \} \cup (\freenames{P} \setminus \{ y \}) \\
  \freenames{x!\langle P \rangle} & := & \{ x \} \cup \{ P \} \\
  \freenames{P|Q} & := & \freenames{P} \cup \freenames{Q} \\
  \freenames{\dropn{x}} & := & \{ x \}
\end{eqnarray*}

The bound names of a process, $\boundnames{P}$, are those names occurring in $P$
that are not free. For example, in $x?(y).0$, the name $x$ is free, while $y$ is bound.

\begin{mathpar}
  \inferrule* [lab=monoidal-laws] {} { P|Q \equiv Q|P \and P|0 \equiv P \and P|(Q|R) \equiv (P|Q)|R }
\end{mathpar}

\begin{mathpar}
  \inferrule* [lab=alpha-equivalence] {} { (x)P \equiv (y)P\{y/x\} \and y \not\in \freenames{P} }
\end{mathpar}

\begin{definition}
Then two processes, $P,Q$, are alpha-equivalent if $P = Q\{\vec{y}/\vec{x}\}$ for
some $\vec{x} \in \boundnames{Q},\vec{y} \in \boundnames{P}$, where $Q\{\vec{y}/\vec{x}\}$
denotes the capture-avoiding substitution of $\vec{y}$ for $\vec{x}$ in $Q$.
\end{definition}

\begin{definition}
  The {\em structural congruence} \cite{SangiorgiWalker} , $\equiv$,
  between processes is the least congruence containing
  alpha-equivalence, satisfying the abelian monoid laws
  (associativity, commutativity and $\pzero$ as identity) for parallel
  composition $|$ and for summation $+$.
\end{definition}

\subsection{Name equivalence}

We take name equivalence, written $\nameeq$, to be the smallest
equivalence relation generated by the following rules.

\begin{mathpar}
\inferrule*[lab=Quote-drop]
{ }
{ \quotep{@{x}} \nameeq x }

\inferrule*[lab=Struct-equiv]
{ P \scong Q }
{ \quotep{P} \nameeq \quotep{Q} }
\end{mathpar}

The astute reader will have noticed that the mutual recursion of names
and processes imposes a mutual recursion on alpha-equivalence and
structural equivalence via name-equivalence. Fortunately, all of this
works out pleasantly and we may calculate in the natural way, free of
concern. The reader interested in the details is referred to the
appendix \ref{appendix:rho_details}.

\subsection{Substitution}

We use $\Proc$ for the set of processes, $\QProc$ for the set of
names, and $\id{\{}\vec{y} / \vec{x} \id{\}}$ to denote partial maps,
$s : \QProc \rightarrow \QProc$. A map, $s$ lifts, uniquely, to a map
on process terms, $\widehat{s} : \Proc \rightarrow \Proc$ by the
following equations.

\begin{mathpar}
  (0) \psubstp{Q}{P} := 0 \\
  (R \juxtap S) \psubstp{Q}{P}
  :=    
  (R)\psubstp{Q}{P} \juxtap (S) \psubstp{Q}{P} \\
  (x?(y).R) \psubstp{Q}{P}    
  :=    
  (x)\substp{Q}{P} (z)\concat( (R \psubstn{z}{y}) \psubstp{Q}{P} ) \\
  (\lift{x}{R}) \psubstp{Q}{P}  
  :=
  \lift{(x)\substp{Q}{P}}{ R \psubstp{Q}{P} } \\
%   (\dropn{x})  \psubstp{Q}{P}       
%   := 
%   \left\{ 
%     \begin{array}{ccc} 
%       \dropn{\quotep{Q}} & & x \nameeq \quotep{P} \\
%       \dropn{x} & & otherwise \\
%     \end{array}
%   \right. 
  (\dropn{x})  \psubstp{Q}{P}       
  := 
  \left\{ 
    \begin{array}{ccc} 
      Q & & x \nameeq \quotep{P} \\
      \dropn{x} & & otherwise \\
    \end{array}
  \right.
\end{mathpar}
 

where

\begin{eqnarray}
  (x)\id{\{} \lpquote Q \rpquote / \lpquote P \rpquote \id{\}}            = 
  \left\{ 
    \begin{array}{ccc}
      \lpquote Q \rpquote & & x \nameeq \lpquote P \rpquote \\
      x & & otherwise \\
    \end{array}
  \right. \nonumber
\end{eqnarray}

and $z$ is chosen distinct from $\quotep{P}$, $\quotep{Q}$, the free
names in $Q$, and all the names in $R$. Our $\alpha$-equivalence will
be built in the standard way from this substitution.

\begin{remark}\label{rem:no_self_referential_names}
  One consequence of these definitions is that $\forall P. \quotep{P}
  \not\in \freenames{P}$.
\end{remark}

\subsection{ Dynamic quote: an example }

Anticipating something of what's to come, consider applying the
substitution, $\widehat{\id{\{}u / z \id{\}}}$, to the following pair
of processes, $\lift{w}{y!(z)}$ and $w[ \lpquote y!(z) \rpquote ]$.

\begin{eqnarray}
	\lift{w}{y!(z)}\widehat{\id{\{}u / z \id{\}}}
		& = &
		\lift{w}{y!(u)} \nonumber\\
	w[ \lpquote y!(z) \rpquote ] \widehat{ \id{\{}u / z \id{\}} }
		& = &
		w[ \lpquote y!(z) \rpquote ] \nonumber
\end{eqnarray}

Because the body of the process between quotes is impervious to
substitution, we get radically different answers. In fact, by
examining the first process in an input context,
e.g. $x?(z).\lift{w}{y!(z)}$, we see that the process under the lift
operator may be shaped by prefixed inputs binding a name inside it. In
this sense, the lift operator will be seen as a way to dynamically
construct processes before reifying them as names.

Finally equipped with these standard features we can present the
dynamics of the calculus.

\subsubsection{Operational semantics} 

Finally, we introduce the computational dynamics. What marks these
algebras as distinct from other more traditionally studied algebraic
structures, e.g. vector spaces or polynomial rings, is the manner in
which dynamics is captured. In traditional structures, dynamics is typically
expressed through morphisms between such structures, as in linear maps
between vector spaces or morphisms between rings. In algebras
associated with the semantics of computation, the dynamics is
expressed as part of the algebraic structure itself, through a
reduction reduction relation typically denoted by $\red$. Below, we
give a recursive presentation of this relation for the calculus used
in the encoding.

$\red \subseteq \pi \times \pi$
$\red : \pi \to \mathcal{P}(\pi)$

\begin{mathpar}
  \inferrule* [lab=Comm] { \textsf{match}( x_{src}, x_{trgt} ) } { x_{trgt}?(y)P \; | \; x_{src}!\langle {Q} \rangle \red P\{\quotep{Q}/y}\} }
  \and \\
  \inferrule* [lab=Par] {{P} \red {P}'} {{{P} | {Q}} \red {{P}' | {Q}}}
  \and
  \inferrule* [lab=Equiv]{{{P} \scong {P}'} \andalso {{P}' \red {Q}'} \andalso {{Q}' \scong {Q}}}{{P} \red {Q}}
\end{mathpar}

\begin{eqnarray*}
  match_{\equiv} (\quotep{P},\quotep{Q}) & := & P \equiv Q \\
  match_{\dagger}(\quotep{P},\quotep{Q}) & := & \forall R. P|Q \red^{*} R => R \red^{*} 0 \\
  match_{K}(\quotep{P},\quotep{Q}) & := & K \mbox{ for some context } K
\end{eqnarray*}

$u?(x)P | u!\langle Q \rangle \red P\{\quotep{Q}/x\}$

%We write $\wred$ for $\red^*$, and $P\red$ if $\exists Q $ such that $ P \red Q$.
We write $P\red$ if $\exists Q $ such that $ P \red Q$ and $P\not\red$, otherwise.

\section{Replication}

As mentioned before, it is known that replication (and hence
recursion) can be implemented in a higher-order process algebra
\cite{SangiorgiWalker}. As our first example of calculation with the
machinery thus far presented we give the construction explicitly in
the {\rhoc}.

\begin{eqnarray}
	D_{x} & := & \prefix{x}{y}{(\binpar{\outputp{x}{y}}{@{y}})} \nonumber\\
	\bangp_{x}{P} & := & \binpar{{x}!\langle{\binpar{D_{x}}{P}}\rangle}{D_{x}} \nonumber
\end{eqnarray}

\begin{eqnarray}
	\bangp_{x}{P} & & \nonumber\\
	=
	& {x}!\langle{(\prefix{x}{y}{(\outputp{x}{y} | @{y})) | P}}\rangle 
	      | \prefix{x}{y}{(\outputp{x}{y} | @{y})} & \nonumber\\
	\red
	& (\outputp{x}{y} | @{y})\substn{\quotep{(\prefix{x}{y}{(@{y} | \outputp{x}{y})) | P}}}{y} & \nonumber\\
	=
	& \outputp{x}{\quotep{(\prefix{x}{y}{(\outputp{x}{y} | @{y})) | P}}}
	  | {(\prefix{x}{y}{(\outputp{x}{y} | @{y})) | P}} & \nonumber\\
	\red
	& \ldots & \nonumber\\
	\red^*
	& P | P | \ldots & \nonumber
\end{eqnarray}

Of course, this encoding, as an implementation, runs away, unfolding
$\bangp{P}$ eagerly. A lazier and more implementable replication
operator, restricted to input-guarded processes, may be obtained as follows.

\begin{eqnarray}
\bangp{\prefix{u}{v}{P}} 
	:= 
	\binpar{\lift{x}{\prefix{u}{v}{(\binpar{D(x)}{P})}}}{D(x)} \nonumber
\end{eqnarray}

\begin{remark}
  Note that the lazier definition still does not deal with summation
  or mixed summation (i.e. sums over input and output). The reader is
  invited to construct definitions of replication that deal with these
  features. 

  Further, the definitions are parameterized in a name, $x$. Can you,
  gentle reader, make a definition that eliminates this parameter and
  guarantees no accidental interaction between the replication
  machinery and the process being replicated -- i.e. no accidental
  sharing of names used by the process to get its work done and the
  name(s) used by the replication to effect copying. This latter
  revision of the definition of replication is crucial to obtaining
  the expected identity $!!P \sim !P$.
\end{remark}

\begin{remark}\label{rem:paradoxical_combinator}
  The reader familiar with the lambda calculus will have noticed the
  similarity between $D$ and the paradoxical combinator.

  [Ed. note: the existence of this seems to suggest we have to be more
  restrictive on the set of processes and names we admit if we are to
  support no-cloning.]
\end{remark}

\subsubsection{Bisimulation}

The computational dynamics gives rise to another kind of equivalence,
the equivalence of computational behavior. As previously mentioned
this is typically captured \emph{via} some form of bisimulation.

% The notion we use in this paper is weak barbed bisimulation
% \cite{milner91polyadicpi}.

The notion we use in this paper is derived from weak barbed
bisimulation \cite{milner91polyadicpi}. 

\begin{definition}
An \emph{observation relation}, $\downarrow_{\mathcal N}$, over a set
of names, $\mathcal N$, is the smallest relation satisfying the rules
below.

\infrule[Out-barb]{y \in {\mathcal N}, \; x \nameeq y}
		  {\outputp{x}{v} \downarrow_{\mathcal N} x}
\infrule[Par-barb]{\mbox{$P\downarrow_{\mathcal N} x$ or $Q\downarrow_{\mathcal N} x$}}
		  {\binpar{P}{Q} \downarrow_{\mathcal N} x}

We write $P \Downarrow_{\mathcal N} x$ if there is $Q$ such that 
$P \wred Q$ and $Q \downarrow_{\mathcal N} x$.
\end{definition}

\begin{definition}
%\label{def.bbisim}
An  ${\mathcal N}$-\emph{barbed bisimulation} over a set of names, ${\mathcal N}$, is a symmetric binary relation 
${\mathcal S}_{\mathcal N}$ between agents such that $P\rel{S}_{\mathcal N}Q$ implies:
\begin{enumerate}
\item If $P \red P'$ then $Q \wred Q'$ and $P'\rel{S}_{\mathcal N} Q'$.
\item If $P\downarrow_{\mathcal N} x$, then $Q\Downarrow_{\mathcal N} x$.
\end{enumerate}
$P$ is ${\mathcal N}$-barbed bisimilar to $Q$, written
$P \wbbisim_{\mathcal N} Q$, if $P \rel{S}_{\mathcal N} Q$ for some ${\mathcal N}$-barbed bisimulation ${\mathcal S}_{\mathcal N}$.
\end{definition}

$\mathcal{R} \subseteq \pi \times \pi$

$P \mathcal{R} Q => \forall P'. P \red P' \Rightarrow \exists Q'. Q \red Q', P' \mathcal{R} Q'$

$P \vdash x \Rightarrow Q \vdash x$

\begin{mathpar}
  \inferrule*[lab=Out-barb]{x \nameeq y}{{y}!\langle{Q}\rangle \vdash x}
  \and
  \inferrule*[lab=Par-barb]{\mbox{$P\vdash x$ or $Q\vdash x$}}{\binpar{P}{Q} \vdash x}
\end{mathpar}

\subsubsection{Contexts}

One of the principle advantages of computational calculi like the
$\pi$-calculus is a well-defined notion of context,
contextual-equivalence and a correlation between
contextual-equivalence and notions of bisimulation. The notion of
context allows the decomposition of a process into (sub-)process and
its syntactic environment, its context. Thus, a context may be
thought of as a process with a ``hole'' (written $\Box$) in it. The
application of a context $M$ to a process $P$, written $M[P]$, is
tantamount to filling the hole in $M$ with $P$. In this paper we do
not need the full weight of this theory, but do make use of the notion
of context in the proof the main theorem. 

\begin{mathpar}
  \inferrule* [lab=summation] {} {{M_{M},M_{N}} \bc \Box \;|\; x.M_{A} \;|\; M_{M}+M_{N}}
  \and
  \inferrule* [lab=agent] {} {{M_{A}} \bc (\vec{x})M_{P} \;| \; \clift{P_0,\ldots,M_{P},\ldots,P_N}}
  \and \\
  \inferrule* [lab=process] {} {{M_{P}} \bc M_{N} \;| \;P|M_{P} }
\end{mathpar} 

\begin{mathpar}
  \inferrule* [lab=sychronization] {} {M_{N} \bc \Box \;|\; x?M_{F} \;|\; x!M_{C}}
  \and
  \inferrule* [lab=abstraction] {} {{M_{F}} \bc (x)M_{P} }
  \and
  \inferrule* [lab=concretion] {} {{M_{C}} \bc \langle M_{P} \rangle }
  \and \\
  \inferrule* [lab=process] {} {{M_{P}} \bc M_{N} \;| \;P|M_{P} }
\end{mathpar}

\begin{definition}[contextual application] Given a context $M$, and
  process $P$, we define the \emph{contextual application}, $M[P] :=
  M\{P/\Box\}$. That is, the contextual application of M to P is the
  substitution of $P$ for $\Box$ in $M$.
\end{definition}

$\meaningof{-} : L \to \mathcal{P}(\pi)$

\begin{mathpar}
  \inferrule* [lab=collection] {} {\meaningof{true} = \pi, \and \meaningof{~E} = \pi \setminus \meaningof{E}, \and \meaningof{E_{1} \& E_{2}} = \meaningof{E_{1}} \cap \meaningof{E_{2}}}
\end{mathpar}

\begin{mathpar}
  \inferrule* [lab=structure] {} {\meaningof{0} = \{ P \in \pi | P \equiv 0 \}, \and \\ \meaningof{E_1 | E_2} = \{ P \in \pi | P \equiv P_{1} | P_{2}, P_{1} \in \meaningof{E_{1}}, P_{2} \in \meaningof{E_2}\} }
\end{mathpar}

\begin{mathpar}
 \inferrule* [lab=behavior] {} {\meaningof{\langle a?b \rangle E} = \{ P \in \pi | P \equiv Q | u?(y)P', \\ \and \\\\ \and \\ \;\;\; u \in \meaningof{a}, \forall z.P'\{z/y\} \in \meaningof{E\{z/b\}}\}, \and \\ \meaningof{a!E} = \{ P \in \pi | P \equiv Q | x!\langle P' \rangle, x \in \meaningof{a} P' \in \meaningof{E}\} }
\end{mathpar}

\begin{mathpar}
 \inferrule* [lab=nominal] {} {\meaningof{\quotep{E}} = \{ \quotep{P} \in \quotep{\pi} | P \in \meaningof{E} \}, \and \meaningof{\quotep{P}} = \{ \quotep{Q} \in \quotep{\pi} | P \equiv Q \} \and \\ \meaningof{@\quotep{E}} = \{ P \in \pi | P \equiv @x, x \in \meaningof{E} \}}
\end{mathpar}

\begin{eqnarray*}
  \\
  \meaningof{-} : TS \to ST
\end{eqnarray*}

\begin{eqnarray*}
  \\
  L : TS \to ST
\end{eqnarray*}

\begin{eqnarray*}
  \\
  P \models E \iff P \in \meaningof{E}
\end{eqnarray*}

\begin{eqnarray*}
  P \approx_{L} Q \iff \forall E \in L. P \models E \iff Q \models E
\end{eqnarray*}

\begin{eqnarray*}
  P \approx_{K} Q
\end{eqnarray*}

\begin{eqnarray*}
  P \approx Q
\end{eqnarray*}

$\approx_{K} = \approx = \approx_{L}$

\subsubsection{Contextual duality}

Note that contexts extend the quotation operation to a family of
operations from processes to names. Given a context, $M$, we can
define a \emph{nominal context}, $\quotep{M}$ by $\quotep{M}[P] :=
\quotep{M[P]}$. To foreshadow what is to come we observe that these
operations enjoy a duality with processes very much like the duality
between vectors and maps from vectors to scalars.

Further, because the calculus is essentially higher-order, we have a
correspondence between contexts and processes. More specifically,
given a name $x$ and a context $M$ we can construct $M^{*}_{x}$ such
that 

\begin{mathpar}
  M^{*}_{x} | \lift{x}{P} \red M[P]
\end{mathpar}

namely,

\begin{mathpar}
  M^{*}_{x} := x?(u).M[\dropn{u}]
\end{mathpar}

The dependence of $M^{*}_{x}$ on a name makes it an abstraction, 

\begin{mathpar}
  M^{*} := (x)x?(u).M[\dropn{u}]
\end{mathpar}

\subsection{Additional notation}

It will sometimes be convenient to denote the process a name
quotes. We already have the notation $x = \quotep{P}$, but it will be
convenient to introduce an alternate notation, $\procn{x}$, when we
want to emphasize the connection to the use of the name. Note that, by
virtue of name equivalence, $\quotep{\procn{x}} \nameeq x$; so, the
notation is consistent with previous definitions.

Further, because names have structure it is possible to effect
substitutions on the basis of that structure. This means we need to
upgrade our notation for substitutions, which we accomplish by
adapting comprehension notation. Thus,

\begin{mathpar}
  P\{ y / x : x \in S \}
\end{mathpar}

is interpreted to mean the process derived from P by replacing (in a
capture-avoiding manner) each occurrence of $x$ in $S$ by $y$. For example,

\begin{mathpar}
  P\{ \quotep{\procn{x}|\procn{x}} / x : x \in \freenames{P} \}
\end{mathpar}

will replace each (occurrence) of a free name $x$ in $P$ by
$\quotep{\procn{x}|\procn{x}}$.

Also, we will avail ourselves of the notation $x^{L}$ and $x^{R}$ to
denote injections of a name into disjoint copies of the name
space. There are numerous ways to accomplish this. One example can be
found in \cite{MeredithR05}. This notation overloads to vectors of
names: $\vec{x}^{\pi} := (x_{i}^{\pi} \; : \; 0 \leq i < |\vec{x}| )$ where $\pi \in \{L,R\}$.

We also use $P^{\Box} := P|\Box$.

In \cite{MeredithR05} an interpretation of the new operator is
given. It turns out that there are several possible interpretations
all enjoying the requisite algebraic properties of the operator (see
\cite{milner91polyadicpi}). We will therefore make liberal use of
$(\nu\; \vec{x})P$.

% subsection the_syntax_and_semantics_of_the_notation_system (end)   

\input{qm2pi.qmops} 

\input{qm2pi.sterngerlach} 

\input{qm2pi.metric} 

% section concurrent_process_calculi (end)

%\input{qm2pi.proofsketch}

% section proof sketch (end)

%\input{qm2pi.slviaknots} 

% section spatial logic via knots (end)

\input{qm2pi.conclusion}

% section conclusion (end)

%\input{qm2pi.dtcodes} 

% section wiring algorithm (end)

\input{qm2pi.ack} 

% section acknowledgments (end)

\newpage


\bibliographystyle{plain}   
\bibliography{../../biblios/main.bib}

\input{qm2pi.rhodetails}

\end{document}



% section front matter (end)

\section{Introduction}\label{sec:introduction} % (fold)
In this draft of the material i am going to have to dispense with the
usual writing conventions adopted in papers on these topics. i'm going
to have adopt whatever tone i need at the time i'm writing up the
calculations. Sometimes this may be very conversational; others it may
be the barest mathematical grunts; others still it may be that i have
lifted text from one of my other papers because the exposition of some
point was better said there. i hope that my readers are not unduly put
out by this decision. i'm not doing this to flout convention or be
rebellious. i find these calculations very technically challenging. To
keep everything going technically, something has to give; i have to
let go of some cognitive burden. So, the academic writing style --
with all of its trade-offs in terms of facilitating technical
communication -- is what i'm letting go of. Perhaps subsequent drafts
can be tightened and polished, but for now, i'm going to speak as if
we were sitting together in a coffee shop with a laptop, wifi and a
pad of paper and a pencil.

So, here's what i have to say. We -- you and i, comfortably ensconced
in our coffee shop and well-equipped with our tools -- can realize and
carry out the calculations of quantum mechanics over a very different
formal theory of dynamics, a formal theory of dynamics that
corresponds to a theory of concurrent computation with
\emph{reflection}. It has the advantage that the underlying theory is
already `quantized', but supports analogues all of the continuuous
operations. Strikingly, this underlying theory has recently been
connected with a notion of metric that we can show, by calculating
together, coincides with the metric induced by the inner product.

There are a lot of reasons why you might be interested in seeing
calculations of this form. Here's why i'm interested. For the past
several centuries there has been no competitor to the ``Newtonian''
account of dynamics. As a result the predominant share of accounts of
dynamical systems and situations have had to be formulated in terms of
the Newtonian machinery. i view this as an intellectually dangerous
position to occupy. Everything, despite it's intrinsic shape, turns
into a nail to be hit with this hammer. Recently, however, the theory
of computation has matured to the point where we have candidates for
theories of dynamics that offer very different perspective on
reasoning about dynamical systems and situations. Testing these
candidates against very successful accounts of dynamical situations,
like quantum mechanics, is going to give us some sense of how mature
they are and some measure of the quality of these accounts of
dynamics.

\subsection{Summary of contributions and outline of paper}

So, we're going to develop an interpretation of the operations of
quantum mechanics normally interpreted by Hilbert spaces and
operators. We're going to do this over a theory of computation. Note
that this is very different than the usual quantum computation program
which develops notions of computation over quantum mechanics. Rather,
we are developing a story that aligns with Wheeler's slogan: It from
Bit. To do this we will first provide an account of the theory of
computation at play here. Then we will dive into a calculation-driven
interpretation of the operations of quantum mechanics.

The reason we take this approach is that -- until very recently --
there hasn't been an axiomatic account of quantum mechanics. As a
result there has been no sharp delineation of the mathematical theory
supporting interpretation of the physical theory and the physical
theory, itself. So, ambient features of the maths are free to be
exploited (or supressed) without a real accounting of their physical
relevance. There is no sharp statement ``here's the physical theory''
qua \emph{theory} and ``here's the mathematical interpretation''
enabling a judgment of how faithful the interpretation is -- apart
from experimental observation. When there is an axiomatic account we
can judge how well a given mathematical formalism supports an
interpretation of the axioms, independent of
experimentation. Likewise, we can judge how well we have captured our
physical evidence and experience with our axiomatics, independent of
any specific mathematical implementation, with accidental detail that
may or may not have physical significance. 

In lieu of a fully fleshed out and vetted axiomatic account of quantum
mechanics, interpreting the operational notions in service of modeling
physical systems will have to suffice. In other words, we are not in
the business of providing a model of Hilbert spaces and operators. We
are in the business of providing a model of quantum mechanics because
we are motivated by testing our notions of dynamics against physical
theory; and, the predictive calculations of the physical theory must
serve as the best formulation -- shy of a fully fleshed out axiomatic
account -- of the physical theory itself (as they have for scientific
theories since time immemorial). Put another way, despite a
whole-hearted commitment to an It-from-Bit ontology, we are firmly
aligned with the shut-up-and-calculate camp as the best way to obtain
results either from the physical perspective or as a quality assurance
measure of our fledgling theory of dynamics.

In detail, we present a reflective process calculus. Then we develop
intuitive correspondences between the notions available in this
calculus and the usual physical notions supporting quantum mechanical
calculations. Thus, 

\begin{table}[htp]
  \center{
    \fbox{
      \begin{tabular}{c|c}
        quantum mechanics & process calculus \\
        \hline
        scalar & name \\
        state vector & process \\
        dual & contextual duals \\
        matrix & formal sums of process-context-dual pairs \\
        orthogonality & process annihilation \\
        inner product & execution-formula + quoting
      \end{tabular}
    }
  }
  \caption{QM - process calculi correspondences}
\end{table}

Then we tighten up these intuitions to operational definitions. We
employ the Dirac notation as the best proxy we can find for an
abstract syntax of the quantum mechanical notions. The definitions we
develop put us in contact with equational constraints coming from the
theory that we demonstrate the definitions and calculations satisfy.

This puts us in a position to shut up and calculate for the
Stern-Gerlach experimental set up, showing how these predictive
calculations become calculations on processes in our theory of a
reflective process calculus.

Penultimately, we demonstrate that the notion of metric coming from
the inner product coincides with the notion of metric available from
the theory of bisimulation. This demonstration gives us the right to
think of space as arising from behavior. Finally, we consider where we
might go from the new vantage point we have obtained.

% section introduction (end) 
 
% section introduction (end)

% \documentclass[12pt]{llncs}
%\documentclass{jktr}

\usepackage[pdftex]{hyperref}                   
\usepackage {listings}
\usepackage {mathpartir}
\usepackage{bcprules}
%\usepackage{listings}
                       
\usepackage{graphicx} 
%\usepackage[margins=2.5cm,nohead,nofoot]{geometry}
%\usepackage{geometry}
\usepackage{amsfonts}
\usepackage{amstext}
\usepackage{latexsym}
\usepackage{amssymb}
\usepackage{color}


%\include{myPreamble}
\include{qm2pi.local} 

%\ifpdf
%\usepackage[pdftex]{graphicx}
%\else
%\usepackage{graphicx}
%\fi

 % \ifpdf
%  \usepackage{pdfsync}
%  \if


%\title{Brief Article}
%\author{David F. Snyder}
%\author{L.G. Meredith}

%\address{Dept. of Math., Texas State University--San Marcos, San Marcos, TX 78666}
       
\pagestyle{empty}


\begin{document}

\lstset{language=[Objective]Caml,frame=shadowbox}

\input{qm2pi.front}

% section front matter (end)

\input{qm2pi.intro} 
 
% section introduction (end)

% \input{qm2pi.knotations} 

% section notation (end)

\input{qm2pi.process.calculi} 

% section concurrent_process_calculi_and_spatial_logics_ (end)
    
%\input{qm2pi.knots2pi} 

%\input{qm2pi.trefoil} 

%\input{qm2pi.mainthm} 

% subsection basic_interpretation (end)

%\input{qm2pi.rho.presentation} 
\subsection{The syntax and semantics of the notation system}\label{sub:the_syntax_and_semantics_of_the_notation_system} % (fold)

We now summarize a technical presentation of the calculus that
embodies our theory of dynamics. The typical presentation of such a
calculus follows the style of giving generators and relations on
them. The grammar, below, describing term constructors, freely
generates the set of processes, $\Proc$. This set is then quotiented
by a relation known as structural congruence and it is over this set
that the notion of dynamics is expressed. This presentation is
essentially that of \cite{MeredithR05} with the addition of
polyadicity and summation. For readability we have relegated some of
the technical subtleties to an appendix.

\subsubsection{Process grammar}\label{subsub:process_grammar}

\begin{mathpar}
  \inferrule* [lab=synchronization] {} {{M} \bc \pzero \;|\; x?F \;|\; x!C }
  \and
  \inferrule* [lab=abstraction] {} {{F} \bc (x)P}
  \and
  \inferrule* [lab=concretion] {} {{C} \bc \langle Q \rangle}
  \and
  \inferrule* [lab=process] {} {{P,Q} \bc M \;| \;P|Q \;|\; @{x}}
  \and
  \inferrule* [lab=name] {} {{x} \bc \quotep{P}}
\end{mathpar} 

Note that $\vec{x}$ (resp. $\vec{P}$) denotes a vector of names
(resp. processes) of length $|\vec{x}|$ (resp. $|\vec{P}|$). We adopt
the following useful abbreviations.

\begin{mathpar}
   x?(\vec{y}).P := x.(\vec{y})P \and  x\clift{\vec{P}} := x.\clift{\vec{P}}
   \and x!(y) := \lift{x}{\dropn{y}}
   \and \Pi_{i=0}^{n-1}P_i := P_0 | \ldots | P_{n-1}
\end{mathpar}

\subsubsection{Structural congruence}

\paragraph{Free and bound names and alpha-equivalence.} At the
core of structural equivalence is alpha-equivalence which identifies
process that are the same up to a change of variable. Formally, we
recognize the distinction between free and bound names. The free names
of a process, $\freenames{P}$, may be calculated recursively as
follows:

\begin{mathpar}
\freenames{\pzero} := \emptyset
  \and \\
  \freenames{x?(y).P} := \{ x \} \cup (\freenames{P} \setminus \{ y \})
  \and 
  \freenames{x!\langle P \rangle} := \{ x \} \cup \{ P \} 
  \and \\
  \freenames{P|Q} := \freenames{P} \cup \freenames{Q}
  \and \\
  \freenames{@{x}} := \{ x \}
\end{mathpar}

$\pi$
$\quotep{\pi}$

$\freenames{-} : \pi \to \mathcal{P}(\quotep{\pi})$

\begin{eqnarray*}
  \freenames{\pzero} & := & \emptyset \\
  \freenames{x?(y).P} & := & \{ x \} \cup (\freenames{P} \setminus \{ y \}) \\
  \freenames{x!\langle P \rangle} & := & \{ x \} \cup \{ P \} \\
  \freenames{P|Q} & := & \freenames{P} \cup \freenames{Q} \\
  \freenames{\dropn{x}} & := & \{ x \}
\end{eqnarray*}

The bound names of a process, $\boundnames{P}$, are those names occurring in $P$
that are not free. For example, in $x?(y).0$, the name $x$ is free, while $y$ is bound.

\begin{mathpar}
  \inferrule* [lab=monoidal-laws] {} { P|Q \equiv Q|P \and P|0 \equiv P \and P|(Q|R) \equiv (P|Q)|R }
\end{mathpar}

\begin{mathpar}
  \inferrule* [lab=alpha-equivalence] {} { (x)P \equiv (y)P\{y/x\} \and y \not\in \freenames{P} }
\end{mathpar}

\begin{definition}
Then two processes, $P,Q$, are alpha-equivalent if $P = Q\{\vec{y}/\vec{x}\}$ for
some $\vec{x} \in \boundnames{Q},\vec{y} \in \boundnames{P}$, where $Q\{\vec{y}/\vec{x}\}$
denotes the capture-avoiding substitution of $\vec{y}$ for $\vec{x}$ in $Q$.
\end{definition}

\begin{definition}
  The {\em structural congruence} \cite{SangiorgiWalker} , $\equiv$,
  between processes is the least congruence containing
  alpha-equivalence, satisfying the abelian monoid laws
  (associativity, commutativity and $\pzero$ as identity) for parallel
  composition $|$ and for summation $+$.
\end{definition}

\subsection{Name equivalence}

We take name equivalence, written $\nameeq$, to be the smallest
equivalence relation generated by the following rules.

\begin{mathpar}
\inferrule*[lab=Quote-drop]
{ }
{ \quotep{@{x}} \nameeq x }

\inferrule*[lab=Struct-equiv]
{ P \scong Q }
{ \quotep{P} \nameeq \quotep{Q} }
\end{mathpar}

The astute reader will have noticed that the mutual recursion of names
and processes imposes a mutual recursion on alpha-equivalence and
structural equivalence via name-equivalence. Fortunately, all of this
works out pleasantly and we may calculate in the natural way, free of
concern. The reader interested in the details is referred to the
appendix \ref{appendix:rho_details}.

\subsection{Substitution}

We use $\Proc$ for the set of processes, $\QProc$ for the set of
names, and $\id{\{}\vec{y} / \vec{x} \id{\}}$ to denote partial maps,
$s : \QProc \rightarrow \QProc$. A map, $s$ lifts, uniquely, to a map
on process terms, $\widehat{s} : \Proc \rightarrow \Proc$ by the
following equations.

\begin{mathpar}
  (0) \psubstp{Q}{P} := 0 \\
  (R \juxtap S) \psubstp{Q}{P}
  :=    
  (R)\psubstp{Q}{P} \juxtap (S) \psubstp{Q}{P} \\
  (x?(y).R) \psubstp{Q}{P}    
  :=    
  (x)\substp{Q}{P} (z)\concat( (R \psubstn{z}{y}) \psubstp{Q}{P} ) \\
  (\lift{x}{R}) \psubstp{Q}{P}  
  :=
  \lift{(x)\substp{Q}{P}}{ R \psubstp{Q}{P} } \\
%   (\dropn{x})  \psubstp{Q}{P}       
%   := 
%   \left\{ 
%     \begin{array}{ccc} 
%       \dropn{\quotep{Q}} & & x \nameeq \quotep{P} \\
%       \dropn{x} & & otherwise \\
%     \end{array}
%   \right. 
  (\dropn{x})  \psubstp{Q}{P}       
  := 
  \left\{ 
    \begin{array}{ccc} 
      Q & & x \nameeq \quotep{P} \\
      \dropn{x} & & otherwise \\
    \end{array}
  \right.
\end{mathpar}
 

where

\begin{eqnarray}
  (x)\id{\{} \lpquote Q \rpquote / \lpquote P \rpquote \id{\}}            = 
  \left\{ 
    \begin{array}{ccc}
      \lpquote Q \rpquote & & x \nameeq \lpquote P \rpquote \\
      x & & otherwise \\
    \end{array}
  \right. \nonumber
\end{eqnarray}

and $z$ is chosen distinct from $\quotep{P}$, $\quotep{Q}$, the free
names in $Q$, and all the names in $R$. Our $\alpha$-equivalence will
be built in the standard way from this substitution.

\begin{remark}\label{rem:no_self_referential_names}
  One consequence of these definitions is that $\forall P. \quotep{P}
  \not\in \freenames{P}$.
\end{remark}

\subsection{ Dynamic quote: an example }

Anticipating something of what's to come, consider applying the
substitution, $\widehat{\id{\{}u / z \id{\}}}$, to the following pair
of processes, $\lift{w}{y!(z)}$ and $w[ \lpquote y!(z) \rpquote ]$.

\begin{eqnarray}
	\lift{w}{y!(z)}\widehat{\id{\{}u / z \id{\}}}
		& = &
		\lift{w}{y!(u)} \nonumber\\
	w[ \lpquote y!(z) \rpquote ] \widehat{ \id{\{}u / z \id{\}} }
		& = &
		w[ \lpquote y!(z) \rpquote ] \nonumber
\end{eqnarray}

Because the body of the process between quotes is impervious to
substitution, we get radically different answers. In fact, by
examining the first process in an input context,
e.g. $x?(z).\lift{w}{y!(z)}$, we see that the process under the lift
operator may be shaped by prefixed inputs binding a name inside it. In
this sense, the lift operator will be seen as a way to dynamically
construct processes before reifying them as names.

Finally equipped with these standard features we can present the
dynamics of the calculus.

\subsubsection{Operational semantics} 

Finally, we introduce the computational dynamics. What marks these
algebras as distinct from other more traditionally studied algebraic
structures, e.g. vector spaces or polynomial rings, is the manner in
which dynamics is captured. In traditional structures, dynamics is typically
expressed through morphisms between such structures, as in linear maps
between vector spaces or morphisms between rings. In algebras
associated with the semantics of computation, the dynamics is
expressed as part of the algebraic structure itself, through a
reduction reduction relation typically denoted by $\red$. Below, we
give a recursive presentation of this relation for the calculus used
in the encoding.

$\red \subseteq \pi \times \pi$
$\red : \pi \to \mathcal{P}(\pi)$

\begin{mathpar}
  \inferrule* [lab=Comm] { \textsf{match}( x_{src}, x_{trgt} ) } { x_{trgt}?(y)P \; | \; x_{src}!\langle {Q} \rangle \red P\{\quotep{Q}/y}\} }
  \and \\
  \inferrule* [lab=Par] {{P} \red {P}'} {{{P} | {Q}} \red {{P}' | {Q}}}
  \and
  \inferrule* [lab=Equiv]{{{P} \scong {P}'} \andalso {{P}' \red {Q}'} \andalso {{Q}' \scong {Q}}}{{P} \red {Q}}
\end{mathpar}

\begin{eqnarray*}
  match_{\equiv} (\quotep{P},\quotep{Q}) & := & P \equiv Q \\
  match_{\dagger}(\quotep{P},\quotep{Q}) & := & \forall R. P|Q \red^{*} R => R \red^{*} 0 \\
  match_{K}(\quotep{P},\quotep{Q}) & := & K \mbox{ for some context } K
\end{eqnarray*}

$u?(x)P | u!\langle Q \rangle \red P\{\quotep{Q}/x\}$

%We write $\wred$ for $\red^*$, and $P\red$ if $\exists Q $ such that $ P \red Q$.
We write $P\red$ if $\exists Q $ such that $ P \red Q$ and $P\not\red$, otherwise.

\section{Replication}

As mentioned before, it is known that replication (and hence
recursion) can be implemented in a higher-order process algebra
\cite{SangiorgiWalker}. As our first example of calculation with the
machinery thus far presented we give the construction explicitly in
the {\rhoc}.

\begin{eqnarray}
	D_{x} & := & \prefix{x}{y}{(\binpar{\outputp{x}{y}}{@{y}})} \nonumber\\
	\bangp_{x}{P} & := & \binpar{{x}!\langle{\binpar{D_{x}}{P}}\rangle}{D_{x}} \nonumber
\end{eqnarray}

\begin{eqnarray}
	\bangp_{x}{P} & & \nonumber\\
	=
	& {x}!\langle{(\prefix{x}{y}{(\outputp{x}{y} | @{y})) | P}}\rangle 
	      | \prefix{x}{y}{(\outputp{x}{y} | @{y})} & \nonumber\\
	\red
	& (\outputp{x}{y} | @{y})\substn{\quotep{(\prefix{x}{y}{(@{y} | \outputp{x}{y})) | P}}}{y} & \nonumber\\
	=
	& \outputp{x}{\quotep{(\prefix{x}{y}{(\outputp{x}{y} | @{y})) | P}}}
	  | {(\prefix{x}{y}{(\outputp{x}{y} | @{y})) | P}} & \nonumber\\
	\red
	& \ldots & \nonumber\\
	\red^*
	& P | P | \ldots & \nonumber
\end{eqnarray}

Of course, this encoding, as an implementation, runs away, unfolding
$\bangp{P}$ eagerly. A lazier and more implementable replication
operator, restricted to input-guarded processes, may be obtained as follows.

\begin{eqnarray}
\bangp{\prefix{u}{v}{P}} 
	:= 
	\binpar{\lift{x}{\prefix{u}{v}{(\binpar{D(x)}{P})}}}{D(x)} \nonumber
\end{eqnarray}

\begin{remark}
  Note that the lazier definition still does not deal with summation
  or mixed summation (i.e. sums over input and output). The reader is
  invited to construct definitions of replication that deal with these
  features. 

  Further, the definitions are parameterized in a name, $x$. Can you,
  gentle reader, make a definition that eliminates this parameter and
  guarantees no accidental interaction between the replication
  machinery and the process being replicated -- i.e. no accidental
  sharing of names used by the process to get its work done and the
  name(s) used by the replication to effect copying. This latter
  revision of the definition of replication is crucial to obtaining
  the expected identity $!!P \sim !P$.
\end{remark}

\begin{remark}\label{rem:paradoxical_combinator}
  The reader familiar with the lambda calculus will have noticed the
  similarity between $D$ and the paradoxical combinator.

  [Ed. note: the existence of this seems to suggest we have to be more
  restrictive on the set of processes and names we admit if we are to
  support no-cloning.]
\end{remark}

\subsubsection{Bisimulation}

The computational dynamics gives rise to another kind of equivalence,
the equivalence of computational behavior. As previously mentioned
this is typically captured \emph{via} some form of bisimulation.

% The notion we use in this paper is weak barbed bisimulation
% \cite{milner91polyadicpi}.

The notion we use in this paper is derived from weak barbed
bisimulation \cite{milner91polyadicpi}. 

\begin{definition}
An \emph{observation relation}, $\downarrow_{\mathcal N}$, over a set
of names, $\mathcal N$, is the smallest relation satisfying the rules
below.

\infrule[Out-barb]{y \in {\mathcal N}, \; x \nameeq y}
		  {\outputp{x}{v} \downarrow_{\mathcal N} x}
\infrule[Par-barb]{\mbox{$P\downarrow_{\mathcal N} x$ or $Q\downarrow_{\mathcal N} x$}}
		  {\binpar{P}{Q} \downarrow_{\mathcal N} x}

We write $P \Downarrow_{\mathcal N} x$ if there is $Q$ such that 
$P \wred Q$ and $Q \downarrow_{\mathcal N} x$.
\end{definition}

\begin{definition}
%\label{def.bbisim}
An  ${\mathcal N}$-\emph{barbed bisimulation} over a set of names, ${\mathcal N}$, is a symmetric binary relation 
${\mathcal S}_{\mathcal N}$ between agents such that $P\rel{S}_{\mathcal N}Q$ implies:
\begin{enumerate}
\item If $P \red P'$ then $Q \wred Q'$ and $P'\rel{S}_{\mathcal N} Q'$.
\item If $P\downarrow_{\mathcal N} x$, then $Q\Downarrow_{\mathcal N} x$.
\end{enumerate}
$P$ is ${\mathcal N}$-barbed bisimilar to $Q$, written
$P \wbbisim_{\mathcal N} Q$, if $P \rel{S}_{\mathcal N} Q$ for some ${\mathcal N}$-barbed bisimulation ${\mathcal S}_{\mathcal N}$.
\end{definition}

$\mathcal{R} \subseteq \pi \times \pi$

$P \mathcal{R} Q => \forall P'. P \red P' \Rightarrow \exists Q'. Q \red Q', P' \mathcal{R} Q'$

$P \vdash x \Rightarrow Q \vdash x$

\begin{mathpar}
  \inferrule*[lab=Out-barb]{x \nameeq y}{{y}!\langle{Q}\rangle \vdash x}
  \and
  \inferrule*[lab=Par-barb]{\mbox{$P\vdash x$ or $Q\vdash x$}}{\binpar{P}{Q} \vdash x}
\end{mathpar}

\subsubsection{Contexts}

One of the principle advantages of computational calculi like the
$\pi$-calculus is a well-defined notion of context,
contextual-equivalence and a correlation between
contextual-equivalence and notions of bisimulation. The notion of
context allows the decomposition of a process into (sub-)process and
its syntactic environment, its context. Thus, a context may be
thought of as a process with a ``hole'' (written $\Box$) in it. The
application of a context $M$ to a process $P$, written $M[P]$, is
tantamount to filling the hole in $M$ with $P$. In this paper we do
not need the full weight of this theory, but do make use of the notion
of context in the proof the main theorem. 

\begin{mathpar}
  \inferrule* [lab=summation] {} {{M_{M},M_{N}} \bc \Box \;|\; x.M_{A} \;|\; M_{M}+M_{N}}
  \and
  \inferrule* [lab=agent] {} {{M_{A}} \bc (\vec{x})M_{P} \;| \; \clift{P_0,\ldots,M_{P},\ldots,P_N}}
  \and \\
  \inferrule* [lab=process] {} {{M_{P}} \bc M_{N} \;| \;P|M_{P} }
\end{mathpar} 

\begin{mathpar}
  \inferrule* [lab=sychronization] {} {M_{N} \bc \Box \;|\; x?M_{F} \;|\; x!M_{C}}
  \and
  \inferrule* [lab=abstraction] {} {{M_{F}} \bc (x)M_{P} }
  \and
  \inferrule* [lab=concretion] {} {{M_{C}} \bc \langle M_{P} \rangle }
  \and \\
  \inferrule* [lab=process] {} {{M_{P}} \bc M_{N} \;| \;P|M_{P} }
\end{mathpar}

\begin{definition}[contextual application] Given a context $M$, and
  process $P$, we define the \emph{contextual application}, $M[P] :=
  M\{P/\Box\}$. That is, the contextual application of M to P is the
  substitution of $P$ for $\Box$ in $M$.
\end{definition}

$\meaningof{-} : L \to \mathcal{P}(\pi)$

\begin{mathpar}
  \inferrule* [lab=collection] {} {\meaningof{true} = \pi, \and \meaningof{~E} = \pi \setminus \meaningof{E}, \and \meaningof{E_{1} \& E_{2}} = \meaningof{E_{1}} \cap \meaningof{E_{2}}}
\end{mathpar}

\begin{mathpar}
  \inferrule* [lab=structure] {} {\meaningof{0} = \{ P \in \pi | P \equiv 0 \}, \and \\ \meaningof{E_1 | E_2} = \{ P \in \pi | P \equiv P_{1} | P_{2}, P_{1} \in \meaningof{E_{1}}, P_{2} \in \meaningof{E_2}\} }
\end{mathpar}

\begin{mathpar}
 \inferrule* [lab=behavior] {} {\meaningof{\langle a?b \rangle E} = \{ P \in \pi | P \equiv Q | u?(y)P', \\ \and \\\\ \and \\ \;\;\; u \in \meaningof{a}, \forall z.P'\{z/y\} \in \meaningof{E\{z/b\}}\}, \and \\ \meaningof{a!E} = \{ P \in \pi | P \equiv Q | x!\langle P' \rangle, x \in \meaningof{a} P' \in \meaningof{E}\} }
\end{mathpar}

\begin{mathpar}
 \inferrule* [lab=nominal] {} {\meaningof{\quotep{E}} = \{ \quotep{P} \in \quotep{\pi} | P \in \meaningof{E} \}, \and \meaningof{\quotep{P}} = \{ \quotep{Q} \in \quotep{\pi} | P \equiv Q \} \and \\ \meaningof{@\quotep{E}} = \{ P \in \pi | P \equiv @x, x \in \meaningof{E} \}}
\end{mathpar}

\begin{eqnarray*}
  \\
  \meaningof{-} : TS \to ST
\end{eqnarray*}

\begin{eqnarray*}
  \\
  L : TS \to ST
\end{eqnarray*}

\begin{eqnarray*}
  \\
  P \models E \iff P \in \meaningof{E}
\end{eqnarray*}

\begin{eqnarray*}
  P \approx_{L} Q \iff \forall E \in L. P \models E \iff Q \models E
\end{eqnarray*}

\begin{eqnarray*}
  P \approx_{K} Q
\end{eqnarray*}

\begin{eqnarray*}
  P \approx Q
\end{eqnarray*}

$\approx_{K} = \approx = \approx_{L}$

\subsubsection{Contextual duality}

Note that contexts extend the quotation operation to a family of
operations from processes to names. Given a context, $M$, we can
define a \emph{nominal context}, $\quotep{M}$ by $\quotep{M}[P] :=
\quotep{M[P]}$. To foreshadow what is to come we observe that these
operations enjoy a duality with processes very much like the duality
between vectors and maps from vectors to scalars.

Further, because the calculus is essentially higher-order, we have a
correspondence between contexts and processes. More specifically,
given a name $x$ and a context $M$ we can construct $M^{*}_{x}$ such
that 

\begin{mathpar}
  M^{*}_{x} | \lift{x}{P} \red M[P]
\end{mathpar}

namely,

\begin{mathpar}
  M^{*}_{x} := x?(u).M[\dropn{u}]
\end{mathpar}

The dependence of $M^{*}_{x}$ on a name makes it an abstraction, 

\begin{mathpar}
  M^{*} := (x)x?(u).M[\dropn{u}]
\end{mathpar}

\subsection{Additional notation}

It will sometimes be convenient to denote the process a name
quotes. We already have the notation $x = \quotep{P}$, but it will be
convenient to introduce an alternate notation, $\procn{x}$, when we
want to emphasize the connection to the use of the name. Note that, by
virtue of name equivalence, $\quotep{\procn{x}} \nameeq x$; so, the
notation is consistent with previous definitions.

Further, because names have structure it is possible to effect
substitutions on the basis of that structure. This means we need to
upgrade our notation for substitutions, which we accomplish by
adapting comprehension notation. Thus,

\begin{mathpar}
  P\{ y / x : x \in S \}
\end{mathpar}

is interpreted to mean the process derived from P by replacing (in a
capture-avoiding manner) each occurrence of $x$ in $S$ by $y$. For example,

\begin{mathpar}
  P\{ \quotep{\procn{x}|\procn{x}} / x : x \in \freenames{P} \}
\end{mathpar}

will replace each (occurrence) of a free name $x$ in $P$ by
$\quotep{\procn{x}|\procn{x}}$.

Also, we will avail ourselves of the notation $x^{L}$ and $x^{R}$ to
denote injections of a name into disjoint copies of the name
space. There are numerous ways to accomplish this. One example can be
found in \cite{MeredithR05}. This notation overloads to vectors of
names: $\vec{x}^{\pi} := (x_{i}^{\pi} \; : \; 0 \leq i < |\vec{x}| )$ where $\pi \in \{L,R\}$.

We also use $P^{\Box} := P|\Box$.

In \cite{MeredithR05} an interpretation of the new operator is
given. It turns out that there are several possible interpretations
all enjoying the requisite algebraic properties of the operator (see
\cite{milner91polyadicpi}). We will therefore make liberal use of
$(\nu\; \vec{x})P$.

% subsection the_syntax_and_semantics_of_the_notation_system (end)   

\input{qm2pi.qmops} 

\input{qm2pi.sterngerlach} 

\input{qm2pi.metric} 

% section concurrent_process_calculi (end)

%\input{qm2pi.proofsketch}

% section proof sketch (end)

%\input{qm2pi.slviaknots} 

% section spatial logic via knots (end)

\input{qm2pi.conclusion}

% section conclusion (end)

%\input{qm2pi.dtcodes} 

% section wiring algorithm (end)

\input{qm2pi.ack} 

% section acknowledgments (end)

\newpage


\bibliographystyle{plain}   
\bibliography{../../biblios/main.bib}

\input{qm2pi.rhodetails}

\end{document}

 

% section notation (end)

\input{qm2pi.process.calculi} 

% section concurrent_process_calculi_and_spatial_logics_ (end)
    
%\documentclass[12pt]{llncs}
%\documentclass{jktr}

\usepackage[pdftex]{hyperref}                   
\usepackage {listings}
\usepackage {mathpartir}
\usepackage{bcprules}
%\usepackage{listings}
                       
\usepackage{graphicx} 
%\usepackage[margins=2.5cm,nohead,nofoot]{geometry}
%\usepackage{geometry}
\usepackage{amsfonts}
\usepackage{amstext}
\usepackage{latexsym}
\usepackage{amssymb}
\usepackage{color}


%\include{myPreamble}
\include{qm2pi.local} 

%\ifpdf
%\usepackage[pdftex]{graphicx}
%\else
%\usepackage{graphicx}
%\fi

 % \ifpdf
%  \usepackage{pdfsync}
%  \if


%\title{Brief Article}
%\author{David F. Snyder}
%\author{L.G. Meredith}

%\address{Dept. of Math., Texas State University--San Marcos, San Marcos, TX 78666}
       
\pagestyle{empty}


\begin{document}

\lstset{language=[Objective]Caml,frame=shadowbox}

\input{qm2pi.front}

% section front matter (end)

\input{qm2pi.intro} 
 
% section introduction (end)

% \input{qm2pi.knotations} 

% section notation (end)

\input{qm2pi.process.calculi} 

% section concurrent_process_calculi_and_spatial_logics_ (end)
    
%\input{qm2pi.knots2pi} 

%\input{qm2pi.trefoil} 

%\input{qm2pi.mainthm} 

% subsection basic_interpretation (end)

%\input{qm2pi.rho.presentation} 
\subsection{The syntax and semantics of the notation system}\label{sub:the_syntax_and_semantics_of_the_notation_system} % (fold)

We now summarize a technical presentation of the calculus that
embodies our theory of dynamics. The typical presentation of such a
calculus follows the style of giving generators and relations on
them. The grammar, below, describing term constructors, freely
generates the set of processes, $\Proc$. This set is then quotiented
by a relation known as structural congruence and it is over this set
that the notion of dynamics is expressed. This presentation is
essentially that of \cite{MeredithR05} with the addition of
polyadicity and summation. For readability we have relegated some of
the technical subtleties to an appendix.

\subsubsection{Process grammar}\label{subsub:process_grammar}

\begin{mathpar}
  \inferrule* [lab=synchronization] {} {{M} \bc \pzero \;|\; x?F \;|\; x!C }
  \and
  \inferrule* [lab=abstraction] {} {{F} \bc (x)P}
  \and
  \inferrule* [lab=concretion] {} {{C} \bc \langle Q \rangle}
  \and
  \inferrule* [lab=process] {} {{P,Q} \bc M \;| \;P|Q \;|\; @{x}}
  \and
  \inferrule* [lab=name] {} {{x} \bc \quotep{P}}
\end{mathpar} 

Note that $\vec{x}$ (resp. $\vec{P}$) denotes a vector of names
(resp. processes) of length $|\vec{x}|$ (resp. $|\vec{P}|$). We adopt
the following useful abbreviations.

\begin{mathpar}
   x?(\vec{y}).P := x.(\vec{y})P \and  x\clift{\vec{P}} := x.\clift{\vec{P}}
   \and x!(y) := \lift{x}{\dropn{y}}
   \and \Pi_{i=0}^{n-1}P_i := P_0 | \ldots | P_{n-1}
\end{mathpar}

\subsubsection{Structural congruence}

\paragraph{Free and bound names and alpha-equivalence.} At the
core of structural equivalence is alpha-equivalence which identifies
process that are the same up to a change of variable. Formally, we
recognize the distinction between free and bound names. The free names
of a process, $\freenames{P}$, may be calculated recursively as
follows:

\begin{mathpar}
\freenames{\pzero} := \emptyset
  \and \\
  \freenames{x?(y).P} := \{ x \} \cup (\freenames{P} \setminus \{ y \})
  \and 
  \freenames{x!\langle P \rangle} := \{ x \} \cup \{ P \} 
  \and \\
  \freenames{P|Q} := \freenames{P} \cup \freenames{Q}
  \and \\
  \freenames{@{x}} := \{ x \}
\end{mathpar}

$\pi$
$\quotep{\pi}$

$\freenames{-} : \pi \to \mathcal{P}(\quotep{\pi})$

\begin{eqnarray*}
  \freenames{\pzero} & := & \emptyset \\
  \freenames{x?(y).P} & := & \{ x \} \cup (\freenames{P} \setminus \{ y \}) \\
  \freenames{x!\langle P \rangle} & := & \{ x \} \cup \{ P \} \\
  \freenames{P|Q} & := & \freenames{P} \cup \freenames{Q} \\
  \freenames{\dropn{x}} & := & \{ x \}
\end{eqnarray*}

The bound names of a process, $\boundnames{P}$, are those names occurring in $P$
that are not free. For example, in $x?(y).0$, the name $x$ is free, while $y$ is bound.

\begin{mathpar}
  \inferrule* [lab=monoidal-laws] {} { P|Q \equiv Q|P \and P|0 \equiv P \and P|(Q|R) \equiv (P|Q)|R }
\end{mathpar}

\begin{mathpar}
  \inferrule* [lab=alpha-equivalence] {} { (x)P \equiv (y)P\{y/x\} \and y \not\in \freenames{P} }
\end{mathpar}

\begin{definition}
Then two processes, $P,Q$, are alpha-equivalent if $P = Q\{\vec{y}/\vec{x}\}$ for
some $\vec{x} \in \boundnames{Q},\vec{y} \in \boundnames{P}$, where $Q\{\vec{y}/\vec{x}\}$
denotes the capture-avoiding substitution of $\vec{y}$ for $\vec{x}$ in $Q$.
\end{definition}

\begin{definition}
  The {\em structural congruence} \cite{SangiorgiWalker} , $\equiv$,
  between processes is the least congruence containing
  alpha-equivalence, satisfying the abelian monoid laws
  (associativity, commutativity and $\pzero$ as identity) for parallel
  composition $|$ and for summation $+$.
\end{definition}

\subsection{Name equivalence}

We take name equivalence, written $\nameeq$, to be the smallest
equivalence relation generated by the following rules.

\begin{mathpar}
\inferrule*[lab=Quote-drop]
{ }
{ \quotep{@{x}} \nameeq x }

\inferrule*[lab=Struct-equiv]
{ P \scong Q }
{ \quotep{P} \nameeq \quotep{Q} }
\end{mathpar}

The astute reader will have noticed that the mutual recursion of names
and processes imposes a mutual recursion on alpha-equivalence and
structural equivalence via name-equivalence. Fortunately, all of this
works out pleasantly and we may calculate in the natural way, free of
concern. The reader interested in the details is referred to the
appendix \ref{appendix:rho_details}.

\subsection{Substitution}

We use $\Proc$ for the set of processes, $\QProc$ for the set of
names, and $\id{\{}\vec{y} / \vec{x} \id{\}}$ to denote partial maps,
$s : \QProc \rightarrow \QProc$. A map, $s$ lifts, uniquely, to a map
on process terms, $\widehat{s} : \Proc \rightarrow \Proc$ by the
following equations.

\begin{mathpar}
  (0) \psubstp{Q}{P} := 0 \\
  (R \juxtap S) \psubstp{Q}{P}
  :=    
  (R)\psubstp{Q}{P} \juxtap (S) \psubstp{Q}{P} \\
  (x?(y).R) \psubstp{Q}{P}    
  :=    
  (x)\substp{Q}{P} (z)\concat( (R \psubstn{z}{y}) \psubstp{Q}{P} ) \\
  (\lift{x}{R}) \psubstp{Q}{P}  
  :=
  \lift{(x)\substp{Q}{P}}{ R \psubstp{Q}{P} } \\
%   (\dropn{x})  \psubstp{Q}{P}       
%   := 
%   \left\{ 
%     \begin{array}{ccc} 
%       \dropn{\quotep{Q}} & & x \nameeq \quotep{P} \\
%       \dropn{x} & & otherwise \\
%     \end{array}
%   \right. 
  (\dropn{x})  \psubstp{Q}{P}       
  := 
  \left\{ 
    \begin{array}{ccc} 
      Q & & x \nameeq \quotep{P} \\
      \dropn{x} & & otherwise \\
    \end{array}
  \right.
\end{mathpar}
 

where

\begin{eqnarray}
  (x)\id{\{} \lpquote Q \rpquote / \lpquote P \rpquote \id{\}}            = 
  \left\{ 
    \begin{array}{ccc}
      \lpquote Q \rpquote & & x \nameeq \lpquote P \rpquote \\
      x & & otherwise \\
    \end{array}
  \right. \nonumber
\end{eqnarray}

and $z$ is chosen distinct from $\quotep{P}$, $\quotep{Q}$, the free
names in $Q$, and all the names in $R$. Our $\alpha$-equivalence will
be built in the standard way from this substitution.

\begin{remark}\label{rem:no_self_referential_names}
  One consequence of these definitions is that $\forall P. \quotep{P}
  \not\in \freenames{P}$.
\end{remark}

\subsection{ Dynamic quote: an example }

Anticipating something of what's to come, consider applying the
substitution, $\widehat{\id{\{}u / z \id{\}}}$, to the following pair
of processes, $\lift{w}{y!(z)}$ and $w[ \lpquote y!(z) \rpquote ]$.

\begin{eqnarray}
	\lift{w}{y!(z)}\widehat{\id{\{}u / z \id{\}}}
		& = &
		\lift{w}{y!(u)} \nonumber\\
	w[ \lpquote y!(z) \rpquote ] \widehat{ \id{\{}u / z \id{\}} }
		& = &
		w[ \lpquote y!(z) \rpquote ] \nonumber
\end{eqnarray}

Because the body of the process between quotes is impervious to
substitution, we get radically different answers. In fact, by
examining the first process in an input context,
e.g. $x?(z).\lift{w}{y!(z)}$, we see that the process under the lift
operator may be shaped by prefixed inputs binding a name inside it. In
this sense, the lift operator will be seen as a way to dynamically
construct processes before reifying them as names.

Finally equipped with these standard features we can present the
dynamics of the calculus.

\subsubsection{Operational semantics} 

Finally, we introduce the computational dynamics. What marks these
algebras as distinct from other more traditionally studied algebraic
structures, e.g. vector spaces or polynomial rings, is the manner in
which dynamics is captured. In traditional structures, dynamics is typically
expressed through morphisms between such structures, as in linear maps
between vector spaces or morphisms between rings. In algebras
associated with the semantics of computation, the dynamics is
expressed as part of the algebraic structure itself, through a
reduction reduction relation typically denoted by $\red$. Below, we
give a recursive presentation of this relation for the calculus used
in the encoding.

$\red \subseteq \pi \times \pi$
$\red : \pi \to \mathcal{P}(\pi)$

\begin{mathpar}
  \inferrule* [lab=Comm] { \textsf{match}( x_{src}, x_{trgt} ) } { x_{trgt}?(y)P \; | \; x_{src}!\langle {Q} \rangle \red P\{\quotep{Q}/y}\} }
  \and \\
  \inferrule* [lab=Par] {{P} \red {P}'} {{{P} | {Q}} \red {{P}' | {Q}}}
  \and
  \inferrule* [lab=Equiv]{{{P} \scong {P}'} \andalso {{P}' \red {Q}'} \andalso {{Q}' \scong {Q}}}{{P} \red {Q}}
\end{mathpar}

\begin{eqnarray*}
  match_{\equiv} (\quotep{P},\quotep{Q}) & := & P \equiv Q \\
  match_{\dagger}(\quotep{P},\quotep{Q}) & := & \forall R. P|Q \red^{*} R => R \red^{*} 0 \\
  match_{K}(\quotep{P},\quotep{Q}) & := & K \mbox{ for some context } K
\end{eqnarray*}

$u?(x)P | u!\langle Q \rangle \red P\{\quotep{Q}/x\}$

%We write $\wred$ for $\red^*$, and $P\red$ if $\exists Q $ such that $ P \red Q$.
We write $P\red$ if $\exists Q $ such that $ P \red Q$ and $P\not\red$, otherwise.

\section{Replication}

As mentioned before, it is known that replication (and hence
recursion) can be implemented in a higher-order process algebra
\cite{SangiorgiWalker}. As our first example of calculation with the
machinery thus far presented we give the construction explicitly in
the {\rhoc}.

\begin{eqnarray}
	D_{x} & := & \prefix{x}{y}{(\binpar{\outputp{x}{y}}{@{y}})} \nonumber\\
	\bangp_{x}{P} & := & \binpar{{x}!\langle{\binpar{D_{x}}{P}}\rangle}{D_{x}} \nonumber
\end{eqnarray}

\begin{eqnarray}
	\bangp_{x}{P} & & \nonumber\\
	=
	& {x}!\langle{(\prefix{x}{y}{(\outputp{x}{y} | @{y})) | P}}\rangle 
	      | \prefix{x}{y}{(\outputp{x}{y} | @{y})} & \nonumber\\
	\red
	& (\outputp{x}{y} | @{y})\substn{\quotep{(\prefix{x}{y}{(@{y} | \outputp{x}{y})) | P}}}{y} & \nonumber\\
	=
	& \outputp{x}{\quotep{(\prefix{x}{y}{(\outputp{x}{y} | @{y})) | P}}}
	  | {(\prefix{x}{y}{(\outputp{x}{y} | @{y})) | P}} & \nonumber\\
	\red
	& \ldots & \nonumber\\
	\red^*
	& P | P | \ldots & \nonumber
\end{eqnarray}

Of course, this encoding, as an implementation, runs away, unfolding
$\bangp{P}$ eagerly. A lazier and more implementable replication
operator, restricted to input-guarded processes, may be obtained as follows.

\begin{eqnarray}
\bangp{\prefix{u}{v}{P}} 
	:= 
	\binpar{\lift{x}{\prefix{u}{v}{(\binpar{D(x)}{P})}}}{D(x)} \nonumber
\end{eqnarray}

\begin{remark}
  Note that the lazier definition still does not deal with summation
  or mixed summation (i.e. sums over input and output). The reader is
  invited to construct definitions of replication that deal with these
  features. 

  Further, the definitions are parameterized in a name, $x$. Can you,
  gentle reader, make a definition that eliminates this parameter and
  guarantees no accidental interaction between the replication
  machinery and the process being replicated -- i.e. no accidental
  sharing of names used by the process to get its work done and the
  name(s) used by the replication to effect copying. This latter
  revision of the definition of replication is crucial to obtaining
  the expected identity $!!P \sim !P$.
\end{remark}

\begin{remark}\label{rem:paradoxical_combinator}
  The reader familiar with the lambda calculus will have noticed the
  similarity between $D$ and the paradoxical combinator.

  [Ed. note: the existence of this seems to suggest we have to be more
  restrictive on the set of processes and names we admit if we are to
  support no-cloning.]
\end{remark}

\subsubsection{Bisimulation}

The computational dynamics gives rise to another kind of equivalence,
the equivalence of computational behavior. As previously mentioned
this is typically captured \emph{via} some form of bisimulation.

% The notion we use in this paper is weak barbed bisimulation
% \cite{milner91polyadicpi}.

The notion we use in this paper is derived from weak barbed
bisimulation \cite{milner91polyadicpi}. 

\begin{definition}
An \emph{observation relation}, $\downarrow_{\mathcal N}$, over a set
of names, $\mathcal N$, is the smallest relation satisfying the rules
below.

\infrule[Out-barb]{y \in {\mathcal N}, \; x \nameeq y}
		  {\outputp{x}{v} \downarrow_{\mathcal N} x}
\infrule[Par-barb]{\mbox{$P\downarrow_{\mathcal N} x$ or $Q\downarrow_{\mathcal N} x$}}
		  {\binpar{P}{Q} \downarrow_{\mathcal N} x}

We write $P \Downarrow_{\mathcal N} x$ if there is $Q$ such that 
$P \wred Q$ and $Q \downarrow_{\mathcal N} x$.
\end{definition}

\begin{definition}
%\label{def.bbisim}
An  ${\mathcal N}$-\emph{barbed bisimulation} over a set of names, ${\mathcal N}$, is a symmetric binary relation 
${\mathcal S}_{\mathcal N}$ between agents such that $P\rel{S}_{\mathcal N}Q$ implies:
\begin{enumerate}
\item If $P \red P'$ then $Q \wred Q'$ and $P'\rel{S}_{\mathcal N} Q'$.
\item If $P\downarrow_{\mathcal N} x$, then $Q\Downarrow_{\mathcal N} x$.
\end{enumerate}
$P$ is ${\mathcal N}$-barbed bisimilar to $Q$, written
$P \wbbisim_{\mathcal N} Q$, if $P \rel{S}_{\mathcal N} Q$ for some ${\mathcal N}$-barbed bisimulation ${\mathcal S}_{\mathcal N}$.
\end{definition}

$\mathcal{R} \subseteq \pi \times \pi$

$P \mathcal{R} Q => \forall P'. P \red P' \Rightarrow \exists Q'. Q \red Q', P' \mathcal{R} Q'$

$P \vdash x \Rightarrow Q \vdash x$

\begin{mathpar}
  \inferrule*[lab=Out-barb]{x \nameeq y}{{y}!\langle{Q}\rangle \vdash x}
  \and
  \inferrule*[lab=Par-barb]{\mbox{$P\vdash x$ or $Q\vdash x$}}{\binpar{P}{Q} \vdash x}
\end{mathpar}

\subsubsection{Contexts}

One of the principle advantages of computational calculi like the
$\pi$-calculus is a well-defined notion of context,
contextual-equivalence and a correlation between
contextual-equivalence and notions of bisimulation. The notion of
context allows the decomposition of a process into (sub-)process and
its syntactic environment, its context. Thus, a context may be
thought of as a process with a ``hole'' (written $\Box$) in it. The
application of a context $M$ to a process $P$, written $M[P]$, is
tantamount to filling the hole in $M$ with $P$. In this paper we do
not need the full weight of this theory, but do make use of the notion
of context in the proof the main theorem. 

\begin{mathpar}
  \inferrule* [lab=summation] {} {{M_{M},M_{N}} \bc \Box \;|\; x.M_{A} \;|\; M_{M}+M_{N}}
  \and
  \inferrule* [lab=agent] {} {{M_{A}} \bc (\vec{x})M_{P} \;| \; \clift{P_0,\ldots,M_{P},\ldots,P_N}}
  \and \\
  \inferrule* [lab=process] {} {{M_{P}} \bc M_{N} \;| \;P|M_{P} }
\end{mathpar} 

\begin{mathpar}
  \inferrule* [lab=sychronization] {} {M_{N} \bc \Box \;|\; x?M_{F} \;|\; x!M_{C}}
  \and
  \inferrule* [lab=abstraction] {} {{M_{F}} \bc (x)M_{P} }
  \and
  \inferrule* [lab=concretion] {} {{M_{C}} \bc \langle M_{P} \rangle }
  \and \\
  \inferrule* [lab=process] {} {{M_{P}} \bc M_{N} \;| \;P|M_{P} }
\end{mathpar}

\begin{definition}[contextual application] Given a context $M$, and
  process $P$, we define the \emph{contextual application}, $M[P] :=
  M\{P/\Box\}$. That is, the contextual application of M to P is the
  substitution of $P$ for $\Box$ in $M$.
\end{definition}

$\meaningof{-} : L \to \mathcal{P}(\pi)$

\begin{mathpar}
  \inferrule* [lab=collection] {} {\meaningof{true} = \pi, \and \meaningof{~E} = \pi \setminus \meaningof{E}, \and \meaningof{E_{1} \& E_{2}} = \meaningof{E_{1}} \cap \meaningof{E_{2}}}
\end{mathpar}

\begin{mathpar}
  \inferrule* [lab=structure] {} {\meaningof{0} = \{ P \in \pi | P \equiv 0 \}, \and \\ \meaningof{E_1 | E_2} = \{ P \in \pi | P \equiv P_{1} | P_{2}, P_{1} \in \meaningof{E_{1}}, P_{2} \in \meaningof{E_2}\} }
\end{mathpar}

\begin{mathpar}
 \inferrule* [lab=behavior] {} {\meaningof{\langle a?b \rangle E} = \{ P \in \pi | P \equiv Q | u?(y)P', \\ \and \\\\ \and \\ \;\;\; u \in \meaningof{a}, \forall z.P'\{z/y\} \in \meaningof{E\{z/b\}}\}, \and \\ \meaningof{a!E} = \{ P \in \pi | P \equiv Q | x!\langle P' \rangle, x \in \meaningof{a} P' \in \meaningof{E}\} }
\end{mathpar}

\begin{mathpar}
 \inferrule* [lab=nominal] {} {\meaningof{\quotep{E}} = \{ \quotep{P} \in \quotep{\pi} | P \in \meaningof{E} \}, \and \meaningof{\quotep{P}} = \{ \quotep{Q} \in \quotep{\pi} | P \equiv Q \} \and \\ \meaningof{@\quotep{E}} = \{ P \in \pi | P \equiv @x, x \in \meaningof{E} \}}
\end{mathpar}

\begin{eqnarray*}
  \\
  \meaningof{-} : TS \to ST
\end{eqnarray*}

\begin{eqnarray*}
  \\
  L : TS \to ST
\end{eqnarray*}

\begin{eqnarray*}
  \\
  P \models E \iff P \in \meaningof{E}
\end{eqnarray*}

\begin{eqnarray*}
  P \approx_{L} Q \iff \forall E \in L. P \models E \iff Q \models E
\end{eqnarray*}

\begin{eqnarray*}
  P \approx_{K} Q
\end{eqnarray*}

\begin{eqnarray*}
  P \approx Q
\end{eqnarray*}

$\approx_{K} = \approx = \approx_{L}$

\subsubsection{Contextual duality}

Note that contexts extend the quotation operation to a family of
operations from processes to names. Given a context, $M$, we can
define a \emph{nominal context}, $\quotep{M}$ by $\quotep{M}[P] :=
\quotep{M[P]}$. To foreshadow what is to come we observe that these
operations enjoy a duality with processes very much like the duality
between vectors and maps from vectors to scalars.

Further, because the calculus is essentially higher-order, we have a
correspondence between contexts and processes. More specifically,
given a name $x$ and a context $M$ we can construct $M^{*}_{x}$ such
that 

\begin{mathpar}
  M^{*}_{x} | \lift{x}{P} \red M[P]
\end{mathpar}

namely,

\begin{mathpar}
  M^{*}_{x} := x?(u).M[\dropn{u}]
\end{mathpar}

The dependence of $M^{*}_{x}$ on a name makes it an abstraction, 

\begin{mathpar}
  M^{*} := (x)x?(u).M[\dropn{u}]
\end{mathpar}

\subsection{Additional notation}

It will sometimes be convenient to denote the process a name
quotes. We already have the notation $x = \quotep{P}$, but it will be
convenient to introduce an alternate notation, $\procn{x}$, when we
want to emphasize the connection to the use of the name. Note that, by
virtue of name equivalence, $\quotep{\procn{x}} \nameeq x$; so, the
notation is consistent with previous definitions.

Further, because names have structure it is possible to effect
substitutions on the basis of that structure. This means we need to
upgrade our notation for substitutions, which we accomplish by
adapting comprehension notation. Thus,

\begin{mathpar}
  P\{ y / x : x \in S \}
\end{mathpar}

is interpreted to mean the process derived from P by replacing (in a
capture-avoiding manner) each occurrence of $x$ in $S$ by $y$. For example,

\begin{mathpar}
  P\{ \quotep{\procn{x}|\procn{x}} / x : x \in \freenames{P} \}
\end{mathpar}

will replace each (occurrence) of a free name $x$ in $P$ by
$\quotep{\procn{x}|\procn{x}}$.

Also, we will avail ourselves of the notation $x^{L}$ and $x^{R}$ to
denote injections of a name into disjoint copies of the name
space. There are numerous ways to accomplish this. One example can be
found in \cite{MeredithR05}. This notation overloads to vectors of
names: $\vec{x}^{\pi} := (x_{i}^{\pi} \; : \; 0 \leq i < |\vec{x}| )$ where $\pi \in \{L,R\}$.

We also use $P^{\Box} := P|\Box$.

In \cite{MeredithR05} an interpretation of the new operator is
given. It turns out that there are several possible interpretations
all enjoying the requisite algebraic properties of the operator (see
\cite{milner91polyadicpi}). We will therefore make liberal use of
$(\nu\; \vec{x})P$.

% subsection the_syntax_and_semantics_of_the_notation_system (end)   

\input{qm2pi.qmops} 

\input{qm2pi.sterngerlach} 

\input{qm2pi.metric} 

% section concurrent_process_calculi (end)

%\input{qm2pi.proofsketch}

% section proof sketch (end)

%\input{qm2pi.slviaknots} 

% section spatial logic via knots (end)

\input{qm2pi.conclusion}

% section conclusion (end)

%\input{qm2pi.dtcodes} 

% section wiring algorithm (end)

\input{qm2pi.ack} 

% section acknowledgments (end)

\newpage


\bibliographystyle{plain}   
\bibliography{../../biblios/main.bib}

\input{qm2pi.rhodetails}

\end{document}

 

%\documentclass[12pt]{llncs}
%\documentclass{jktr}

\usepackage[pdftex]{hyperref}                   
\usepackage {listings}
\usepackage {mathpartir}
\usepackage{bcprules}
%\usepackage{listings}
                       
\usepackage{graphicx} 
%\usepackage[margins=2.5cm,nohead,nofoot]{geometry}
%\usepackage{geometry}
\usepackage{amsfonts}
\usepackage{amstext}
\usepackage{latexsym}
\usepackage{amssymb}
\usepackage{color}


%\include{myPreamble}
\include{qm2pi.local} 

%\ifpdf
%\usepackage[pdftex]{graphicx}
%\else
%\usepackage{graphicx}
%\fi

 % \ifpdf
%  \usepackage{pdfsync}
%  \if


%\title{Brief Article}
%\author{David F. Snyder}
%\author{L.G. Meredith}

%\address{Dept. of Math., Texas State University--San Marcos, San Marcos, TX 78666}
       
\pagestyle{empty}


\begin{document}

\lstset{language=[Objective]Caml,frame=shadowbox}

\input{qm2pi.front}

% section front matter (end)

\input{qm2pi.intro} 
 
% section introduction (end)

% \input{qm2pi.knotations} 

% section notation (end)

\input{qm2pi.process.calculi} 

% section concurrent_process_calculi_and_spatial_logics_ (end)
    
%\input{qm2pi.knots2pi} 

%\input{qm2pi.trefoil} 

%\input{qm2pi.mainthm} 

% subsection basic_interpretation (end)

%\input{qm2pi.rho.presentation} 
\subsection{The syntax and semantics of the notation system}\label{sub:the_syntax_and_semantics_of_the_notation_system} % (fold)

We now summarize a technical presentation of the calculus that
embodies our theory of dynamics. The typical presentation of such a
calculus follows the style of giving generators and relations on
them. The grammar, below, describing term constructors, freely
generates the set of processes, $\Proc$. This set is then quotiented
by a relation known as structural congruence and it is over this set
that the notion of dynamics is expressed. This presentation is
essentially that of \cite{MeredithR05} with the addition of
polyadicity and summation. For readability we have relegated some of
the technical subtleties to an appendix.

\subsubsection{Process grammar}\label{subsub:process_grammar}

\begin{mathpar}
  \inferrule* [lab=synchronization] {} {{M} \bc \pzero \;|\; x?F \;|\; x!C }
  \and
  \inferrule* [lab=abstraction] {} {{F} \bc (x)P}
  \and
  \inferrule* [lab=concretion] {} {{C} \bc \langle Q \rangle}
  \and
  \inferrule* [lab=process] {} {{P,Q} \bc M \;| \;P|Q \;|\; @{x}}
  \and
  \inferrule* [lab=name] {} {{x} \bc \quotep{P}}
\end{mathpar} 

Note that $\vec{x}$ (resp. $\vec{P}$) denotes a vector of names
(resp. processes) of length $|\vec{x}|$ (resp. $|\vec{P}|$). We adopt
the following useful abbreviations.

\begin{mathpar}
   x?(\vec{y}).P := x.(\vec{y})P \and  x\clift{\vec{P}} := x.\clift{\vec{P}}
   \and x!(y) := \lift{x}{\dropn{y}}
   \and \Pi_{i=0}^{n-1}P_i := P_0 | \ldots | P_{n-1}
\end{mathpar}

\subsubsection{Structural congruence}

\paragraph{Free and bound names and alpha-equivalence.} At the
core of structural equivalence is alpha-equivalence which identifies
process that are the same up to a change of variable. Formally, we
recognize the distinction between free and bound names. The free names
of a process, $\freenames{P}$, may be calculated recursively as
follows:

\begin{mathpar}
\freenames{\pzero} := \emptyset
  \and \\
  \freenames{x?(y).P} := \{ x \} \cup (\freenames{P} \setminus \{ y \})
  \and 
  \freenames{x!\langle P \rangle} := \{ x \} \cup \{ P \} 
  \and \\
  \freenames{P|Q} := \freenames{P} \cup \freenames{Q}
  \and \\
  \freenames{@{x}} := \{ x \}
\end{mathpar}

$\pi$
$\quotep{\pi}$

$\freenames{-} : \pi \to \mathcal{P}(\quotep{\pi})$

\begin{eqnarray*}
  \freenames{\pzero} & := & \emptyset \\
  \freenames{x?(y).P} & := & \{ x \} \cup (\freenames{P} \setminus \{ y \}) \\
  \freenames{x!\langle P \rangle} & := & \{ x \} \cup \{ P \} \\
  \freenames{P|Q} & := & \freenames{P} \cup \freenames{Q} \\
  \freenames{\dropn{x}} & := & \{ x \}
\end{eqnarray*}

The bound names of a process, $\boundnames{P}$, are those names occurring in $P$
that are not free. For example, in $x?(y).0$, the name $x$ is free, while $y$ is bound.

\begin{mathpar}
  \inferrule* [lab=monoidal-laws] {} { P|Q \equiv Q|P \and P|0 \equiv P \and P|(Q|R) \equiv (P|Q)|R }
\end{mathpar}

\begin{mathpar}
  \inferrule* [lab=alpha-equivalence] {} { (x)P \equiv (y)P\{y/x\} \and y \not\in \freenames{P} }
\end{mathpar}

\begin{definition}
Then two processes, $P,Q$, are alpha-equivalent if $P = Q\{\vec{y}/\vec{x}\}$ for
some $\vec{x} \in \boundnames{Q},\vec{y} \in \boundnames{P}$, where $Q\{\vec{y}/\vec{x}\}$
denotes the capture-avoiding substitution of $\vec{y}$ for $\vec{x}$ in $Q$.
\end{definition}

\begin{definition}
  The {\em structural congruence} \cite{SangiorgiWalker} , $\equiv$,
  between processes is the least congruence containing
  alpha-equivalence, satisfying the abelian monoid laws
  (associativity, commutativity and $\pzero$ as identity) for parallel
  composition $|$ and for summation $+$.
\end{definition}

\subsection{Name equivalence}

We take name equivalence, written $\nameeq$, to be the smallest
equivalence relation generated by the following rules.

\begin{mathpar}
\inferrule*[lab=Quote-drop]
{ }
{ \quotep{@{x}} \nameeq x }

\inferrule*[lab=Struct-equiv]
{ P \scong Q }
{ \quotep{P} \nameeq \quotep{Q} }
\end{mathpar}

The astute reader will have noticed that the mutual recursion of names
and processes imposes a mutual recursion on alpha-equivalence and
structural equivalence via name-equivalence. Fortunately, all of this
works out pleasantly and we may calculate in the natural way, free of
concern. The reader interested in the details is referred to the
appendix \ref{appendix:rho_details}.

\subsection{Substitution}

We use $\Proc$ for the set of processes, $\QProc$ for the set of
names, and $\id{\{}\vec{y} / \vec{x} \id{\}}$ to denote partial maps,
$s : \QProc \rightarrow \QProc$. A map, $s$ lifts, uniquely, to a map
on process terms, $\widehat{s} : \Proc \rightarrow \Proc$ by the
following equations.

\begin{mathpar}
  (0) \psubstp{Q}{P} := 0 \\
  (R \juxtap S) \psubstp{Q}{P}
  :=    
  (R)\psubstp{Q}{P} \juxtap (S) \psubstp{Q}{P} \\
  (x?(y).R) \psubstp{Q}{P}    
  :=    
  (x)\substp{Q}{P} (z)\concat( (R \psubstn{z}{y}) \psubstp{Q}{P} ) \\
  (\lift{x}{R}) \psubstp{Q}{P}  
  :=
  \lift{(x)\substp{Q}{P}}{ R \psubstp{Q}{P} } \\
%   (\dropn{x})  \psubstp{Q}{P}       
%   := 
%   \left\{ 
%     \begin{array}{ccc} 
%       \dropn{\quotep{Q}} & & x \nameeq \quotep{P} \\
%       \dropn{x} & & otherwise \\
%     \end{array}
%   \right. 
  (\dropn{x})  \psubstp{Q}{P}       
  := 
  \left\{ 
    \begin{array}{ccc} 
      Q & & x \nameeq \quotep{P} \\
      \dropn{x} & & otherwise \\
    \end{array}
  \right.
\end{mathpar}
 

where

\begin{eqnarray}
  (x)\id{\{} \lpquote Q \rpquote / \lpquote P \rpquote \id{\}}            = 
  \left\{ 
    \begin{array}{ccc}
      \lpquote Q \rpquote & & x \nameeq \lpquote P \rpquote \\
      x & & otherwise \\
    \end{array}
  \right. \nonumber
\end{eqnarray}

and $z$ is chosen distinct from $\quotep{P}$, $\quotep{Q}$, the free
names in $Q$, and all the names in $R$. Our $\alpha$-equivalence will
be built in the standard way from this substitution.

\begin{remark}\label{rem:no_self_referential_names}
  One consequence of these definitions is that $\forall P. \quotep{P}
  \not\in \freenames{P}$.
\end{remark}

\subsection{ Dynamic quote: an example }

Anticipating something of what's to come, consider applying the
substitution, $\widehat{\id{\{}u / z \id{\}}}$, to the following pair
of processes, $\lift{w}{y!(z)}$ and $w[ \lpquote y!(z) \rpquote ]$.

\begin{eqnarray}
	\lift{w}{y!(z)}\widehat{\id{\{}u / z \id{\}}}
		& = &
		\lift{w}{y!(u)} \nonumber\\
	w[ \lpquote y!(z) \rpquote ] \widehat{ \id{\{}u / z \id{\}} }
		& = &
		w[ \lpquote y!(z) \rpquote ] \nonumber
\end{eqnarray}

Because the body of the process between quotes is impervious to
substitution, we get radically different answers. In fact, by
examining the first process in an input context,
e.g. $x?(z).\lift{w}{y!(z)}$, we see that the process under the lift
operator may be shaped by prefixed inputs binding a name inside it. In
this sense, the lift operator will be seen as a way to dynamically
construct processes before reifying them as names.

Finally equipped with these standard features we can present the
dynamics of the calculus.

\subsubsection{Operational semantics} 

Finally, we introduce the computational dynamics. What marks these
algebras as distinct from other more traditionally studied algebraic
structures, e.g. vector spaces or polynomial rings, is the manner in
which dynamics is captured. In traditional structures, dynamics is typically
expressed through morphisms between such structures, as in linear maps
between vector spaces or morphisms between rings. In algebras
associated with the semantics of computation, the dynamics is
expressed as part of the algebraic structure itself, through a
reduction reduction relation typically denoted by $\red$. Below, we
give a recursive presentation of this relation for the calculus used
in the encoding.

$\red \subseteq \pi \times \pi$
$\red : \pi \to \mathcal{P}(\pi)$

\begin{mathpar}
  \inferrule* [lab=Comm] { \textsf{match}( x_{src}, x_{trgt} ) } { x_{trgt}?(y)P \; | \; x_{src}!\langle {Q} \rangle \red P\{\quotep{Q}/y}\} }
  \and \\
  \inferrule* [lab=Par] {{P} \red {P}'} {{{P} | {Q}} \red {{P}' | {Q}}}
  \and
  \inferrule* [lab=Equiv]{{{P} \scong {P}'} \andalso {{P}' \red {Q}'} \andalso {{Q}' \scong {Q}}}{{P} \red {Q}}
\end{mathpar}

\begin{eqnarray*}
  match_{\equiv} (\quotep{P},\quotep{Q}) & := & P \equiv Q \\
  match_{\dagger}(\quotep{P},\quotep{Q}) & := & \forall R. P|Q \red^{*} R => R \red^{*} 0 \\
  match_{K}(\quotep{P},\quotep{Q}) & := & K \mbox{ for some context } K
\end{eqnarray*}

$u?(x)P | u!\langle Q \rangle \red P\{\quotep{Q}/x\}$

%We write $\wred$ for $\red^*$, and $P\red$ if $\exists Q $ such that $ P \red Q$.
We write $P\red$ if $\exists Q $ such that $ P \red Q$ and $P\not\red$, otherwise.

\section{Replication}

As mentioned before, it is known that replication (and hence
recursion) can be implemented in a higher-order process algebra
\cite{SangiorgiWalker}. As our first example of calculation with the
machinery thus far presented we give the construction explicitly in
the {\rhoc}.

\begin{eqnarray}
	D_{x} & := & \prefix{x}{y}{(\binpar{\outputp{x}{y}}{@{y}})} \nonumber\\
	\bangp_{x}{P} & := & \binpar{{x}!\langle{\binpar{D_{x}}{P}}\rangle}{D_{x}} \nonumber
\end{eqnarray}

\begin{eqnarray}
	\bangp_{x}{P} & & \nonumber\\
	=
	& {x}!\langle{(\prefix{x}{y}{(\outputp{x}{y} | @{y})) | P}}\rangle 
	      | \prefix{x}{y}{(\outputp{x}{y} | @{y})} & \nonumber\\
	\red
	& (\outputp{x}{y} | @{y})\substn{\quotep{(\prefix{x}{y}{(@{y} | \outputp{x}{y})) | P}}}{y} & \nonumber\\
	=
	& \outputp{x}{\quotep{(\prefix{x}{y}{(\outputp{x}{y} | @{y})) | P}}}
	  | {(\prefix{x}{y}{(\outputp{x}{y} | @{y})) | P}} & \nonumber\\
	\red
	& \ldots & \nonumber\\
	\red^*
	& P | P | \ldots & \nonumber
\end{eqnarray}

Of course, this encoding, as an implementation, runs away, unfolding
$\bangp{P}$ eagerly. A lazier and more implementable replication
operator, restricted to input-guarded processes, may be obtained as follows.

\begin{eqnarray}
\bangp{\prefix{u}{v}{P}} 
	:= 
	\binpar{\lift{x}{\prefix{u}{v}{(\binpar{D(x)}{P})}}}{D(x)} \nonumber
\end{eqnarray}

\begin{remark}
  Note that the lazier definition still does not deal with summation
  or mixed summation (i.e. sums over input and output). The reader is
  invited to construct definitions of replication that deal with these
  features. 

  Further, the definitions are parameterized in a name, $x$. Can you,
  gentle reader, make a definition that eliminates this parameter and
  guarantees no accidental interaction between the replication
  machinery and the process being replicated -- i.e. no accidental
  sharing of names used by the process to get its work done and the
  name(s) used by the replication to effect copying. This latter
  revision of the definition of replication is crucial to obtaining
  the expected identity $!!P \sim !P$.
\end{remark}

\begin{remark}\label{rem:paradoxical_combinator}
  The reader familiar with the lambda calculus will have noticed the
  similarity between $D$ and the paradoxical combinator.

  [Ed. note: the existence of this seems to suggest we have to be more
  restrictive on the set of processes and names we admit if we are to
  support no-cloning.]
\end{remark}

\subsubsection{Bisimulation}

The computational dynamics gives rise to another kind of equivalence,
the equivalence of computational behavior. As previously mentioned
this is typically captured \emph{via} some form of bisimulation.

% The notion we use in this paper is weak barbed bisimulation
% \cite{milner91polyadicpi}.

The notion we use in this paper is derived from weak barbed
bisimulation \cite{milner91polyadicpi}. 

\begin{definition}
An \emph{observation relation}, $\downarrow_{\mathcal N}$, over a set
of names, $\mathcal N$, is the smallest relation satisfying the rules
below.

\infrule[Out-barb]{y \in {\mathcal N}, \; x \nameeq y}
		  {\outputp{x}{v} \downarrow_{\mathcal N} x}
\infrule[Par-barb]{\mbox{$P\downarrow_{\mathcal N} x$ or $Q\downarrow_{\mathcal N} x$}}
		  {\binpar{P}{Q} \downarrow_{\mathcal N} x}

We write $P \Downarrow_{\mathcal N} x$ if there is $Q$ such that 
$P \wred Q$ and $Q \downarrow_{\mathcal N} x$.
\end{definition}

\begin{definition}
%\label{def.bbisim}
An  ${\mathcal N}$-\emph{barbed bisimulation} over a set of names, ${\mathcal N}$, is a symmetric binary relation 
${\mathcal S}_{\mathcal N}$ between agents such that $P\rel{S}_{\mathcal N}Q$ implies:
\begin{enumerate}
\item If $P \red P'$ then $Q \wred Q'$ and $P'\rel{S}_{\mathcal N} Q'$.
\item If $P\downarrow_{\mathcal N} x$, then $Q\Downarrow_{\mathcal N} x$.
\end{enumerate}
$P$ is ${\mathcal N}$-barbed bisimilar to $Q$, written
$P \wbbisim_{\mathcal N} Q$, if $P \rel{S}_{\mathcal N} Q$ for some ${\mathcal N}$-barbed bisimulation ${\mathcal S}_{\mathcal N}$.
\end{definition}

$\mathcal{R} \subseteq \pi \times \pi$

$P \mathcal{R} Q => \forall P'. P \red P' \Rightarrow \exists Q'. Q \red Q', P' \mathcal{R} Q'$

$P \vdash x \Rightarrow Q \vdash x$

\begin{mathpar}
  \inferrule*[lab=Out-barb]{x \nameeq y}{{y}!\langle{Q}\rangle \vdash x}
  \and
  \inferrule*[lab=Par-barb]{\mbox{$P\vdash x$ or $Q\vdash x$}}{\binpar{P}{Q} \vdash x}
\end{mathpar}

\subsubsection{Contexts}

One of the principle advantages of computational calculi like the
$\pi$-calculus is a well-defined notion of context,
contextual-equivalence and a correlation between
contextual-equivalence and notions of bisimulation. The notion of
context allows the decomposition of a process into (sub-)process and
its syntactic environment, its context. Thus, a context may be
thought of as a process with a ``hole'' (written $\Box$) in it. The
application of a context $M$ to a process $P$, written $M[P]$, is
tantamount to filling the hole in $M$ with $P$. In this paper we do
not need the full weight of this theory, but do make use of the notion
of context in the proof the main theorem. 

\begin{mathpar}
  \inferrule* [lab=summation] {} {{M_{M},M_{N}} \bc \Box \;|\; x.M_{A} \;|\; M_{M}+M_{N}}
  \and
  \inferrule* [lab=agent] {} {{M_{A}} \bc (\vec{x})M_{P} \;| \; \clift{P_0,\ldots,M_{P},\ldots,P_N}}
  \and \\
  \inferrule* [lab=process] {} {{M_{P}} \bc M_{N} \;| \;P|M_{P} }
\end{mathpar} 

\begin{mathpar}
  \inferrule* [lab=sychronization] {} {M_{N} \bc \Box \;|\; x?M_{F} \;|\; x!M_{C}}
  \and
  \inferrule* [lab=abstraction] {} {{M_{F}} \bc (x)M_{P} }
  \and
  \inferrule* [lab=concretion] {} {{M_{C}} \bc \langle M_{P} \rangle }
  \and \\
  \inferrule* [lab=process] {} {{M_{P}} \bc M_{N} \;| \;P|M_{P} }
\end{mathpar}

\begin{definition}[contextual application] Given a context $M$, and
  process $P$, we define the \emph{contextual application}, $M[P] :=
  M\{P/\Box\}$. That is, the contextual application of M to P is the
  substitution of $P$ for $\Box$ in $M$.
\end{definition}

$\meaningof{-} : L \to \mathcal{P}(\pi)$

\begin{mathpar}
  \inferrule* [lab=collection] {} {\meaningof{true} = \pi, \and \meaningof{~E} = \pi \setminus \meaningof{E}, \and \meaningof{E_{1} \& E_{2}} = \meaningof{E_{1}} \cap \meaningof{E_{2}}}
\end{mathpar}

\begin{mathpar}
  \inferrule* [lab=structure] {} {\meaningof{0} = \{ P \in \pi | P \equiv 0 \}, \and \\ \meaningof{E_1 | E_2} = \{ P \in \pi | P \equiv P_{1} | P_{2}, P_{1} \in \meaningof{E_{1}}, P_{2} \in \meaningof{E_2}\} }
\end{mathpar}

\begin{mathpar}
 \inferrule* [lab=behavior] {} {\meaningof{\langle a?b \rangle E} = \{ P \in \pi | P \equiv Q | u?(y)P', \\ \and \\\\ \and \\ \;\;\; u \in \meaningof{a}, \forall z.P'\{z/y\} \in \meaningof{E\{z/b\}}\}, \and \\ \meaningof{a!E} = \{ P \in \pi | P \equiv Q | x!\langle P' \rangle, x \in \meaningof{a} P' \in \meaningof{E}\} }
\end{mathpar}

\begin{mathpar}
 \inferrule* [lab=nominal] {} {\meaningof{\quotep{E}} = \{ \quotep{P} \in \quotep{\pi} | P \in \meaningof{E} \}, \and \meaningof{\quotep{P}} = \{ \quotep{Q} \in \quotep{\pi} | P \equiv Q \} \and \\ \meaningof{@\quotep{E}} = \{ P \in \pi | P \equiv @x, x \in \meaningof{E} \}}
\end{mathpar}

\begin{eqnarray*}
  \\
  \meaningof{-} : TS \to ST
\end{eqnarray*}

\begin{eqnarray*}
  \\
  L : TS \to ST
\end{eqnarray*}

\begin{eqnarray*}
  \\
  P \models E \iff P \in \meaningof{E}
\end{eqnarray*}

\begin{eqnarray*}
  P \approx_{L} Q \iff \forall E \in L. P \models E \iff Q \models E
\end{eqnarray*}

\begin{eqnarray*}
  P \approx_{K} Q
\end{eqnarray*}

\begin{eqnarray*}
  P \approx Q
\end{eqnarray*}

$\approx_{K} = \approx = \approx_{L}$

\subsubsection{Contextual duality}

Note that contexts extend the quotation operation to a family of
operations from processes to names. Given a context, $M$, we can
define a \emph{nominal context}, $\quotep{M}$ by $\quotep{M}[P] :=
\quotep{M[P]}$. To foreshadow what is to come we observe that these
operations enjoy a duality with processes very much like the duality
between vectors and maps from vectors to scalars.

Further, because the calculus is essentially higher-order, we have a
correspondence between contexts and processes. More specifically,
given a name $x$ and a context $M$ we can construct $M^{*}_{x}$ such
that 

\begin{mathpar}
  M^{*}_{x} | \lift{x}{P} \red M[P]
\end{mathpar}

namely,

\begin{mathpar}
  M^{*}_{x} := x?(u).M[\dropn{u}]
\end{mathpar}

The dependence of $M^{*}_{x}$ on a name makes it an abstraction, 

\begin{mathpar}
  M^{*} := (x)x?(u).M[\dropn{u}]
\end{mathpar}

\subsection{Additional notation}

It will sometimes be convenient to denote the process a name
quotes. We already have the notation $x = \quotep{P}$, but it will be
convenient to introduce an alternate notation, $\procn{x}$, when we
want to emphasize the connection to the use of the name. Note that, by
virtue of name equivalence, $\quotep{\procn{x}} \nameeq x$; so, the
notation is consistent with previous definitions.

Further, because names have structure it is possible to effect
substitutions on the basis of that structure. This means we need to
upgrade our notation for substitutions, which we accomplish by
adapting comprehension notation. Thus,

\begin{mathpar}
  P\{ y / x : x \in S \}
\end{mathpar}

is interpreted to mean the process derived from P by replacing (in a
capture-avoiding manner) each occurrence of $x$ in $S$ by $y$. For example,

\begin{mathpar}
  P\{ \quotep{\procn{x}|\procn{x}} / x : x \in \freenames{P} \}
\end{mathpar}

will replace each (occurrence) of a free name $x$ in $P$ by
$\quotep{\procn{x}|\procn{x}}$.

Also, we will avail ourselves of the notation $x^{L}$ and $x^{R}$ to
denote injections of a name into disjoint copies of the name
space. There are numerous ways to accomplish this. One example can be
found in \cite{MeredithR05}. This notation overloads to vectors of
names: $\vec{x}^{\pi} := (x_{i}^{\pi} \; : \; 0 \leq i < |\vec{x}| )$ where $\pi \in \{L,R\}$.

We also use $P^{\Box} := P|\Box$.

In \cite{MeredithR05} an interpretation of the new operator is
given. It turns out that there are several possible interpretations
all enjoying the requisite algebraic properties of the operator (see
\cite{milner91polyadicpi}). We will therefore make liberal use of
$(\nu\; \vec{x})P$.

% subsection the_syntax_and_semantics_of_the_notation_system (end)   

\input{qm2pi.qmops} 

\input{qm2pi.sterngerlach} 

\input{qm2pi.metric} 

% section concurrent_process_calculi (end)

%\input{qm2pi.proofsketch}

% section proof sketch (end)

%\input{qm2pi.slviaknots} 

% section spatial logic via knots (end)

\input{qm2pi.conclusion}

% section conclusion (end)

%\input{qm2pi.dtcodes} 

% section wiring algorithm (end)

\input{qm2pi.ack} 

% section acknowledgments (end)

\newpage


\bibliographystyle{plain}   
\bibliography{../../biblios/main.bib}

\input{qm2pi.rhodetails}

\end{document}

 

%\documentclass[12pt]{llncs}
%\documentclass{jktr}

\usepackage[pdftex]{hyperref}                   
\usepackage {listings}
\usepackage {mathpartir}
\usepackage{bcprules}
%\usepackage{listings}
                       
\usepackage{graphicx} 
%\usepackage[margins=2.5cm,nohead,nofoot]{geometry}
%\usepackage{geometry}
\usepackage{amsfonts}
\usepackage{amstext}
\usepackage{latexsym}
\usepackage{amssymb}
\usepackage{color}


%\include{myPreamble}
\include{qm2pi.local} 

%\ifpdf
%\usepackage[pdftex]{graphicx}
%\else
%\usepackage{graphicx}
%\fi

 % \ifpdf
%  \usepackage{pdfsync}
%  \if


%\title{Brief Article}
%\author{David F. Snyder}
%\author{L.G. Meredith}

%\address{Dept. of Math., Texas State University--San Marcos, San Marcos, TX 78666}
       
\pagestyle{empty}


\begin{document}

\lstset{language=[Objective]Caml,frame=shadowbox}

\input{qm2pi.front}

% section front matter (end)

\input{qm2pi.intro} 
 
% section introduction (end)

% \input{qm2pi.knotations} 

% section notation (end)

\input{qm2pi.process.calculi} 

% section concurrent_process_calculi_and_spatial_logics_ (end)
    
%\input{qm2pi.knots2pi} 

%\input{qm2pi.trefoil} 

%\input{qm2pi.mainthm} 

% subsection basic_interpretation (end)

%\input{qm2pi.rho.presentation} 
\subsection{The syntax and semantics of the notation system}\label{sub:the_syntax_and_semantics_of_the_notation_system} % (fold)

We now summarize a technical presentation of the calculus that
embodies our theory of dynamics. The typical presentation of such a
calculus follows the style of giving generators and relations on
them. The grammar, below, describing term constructors, freely
generates the set of processes, $\Proc$. This set is then quotiented
by a relation known as structural congruence and it is over this set
that the notion of dynamics is expressed. This presentation is
essentially that of \cite{MeredithR05} with the addition of
polyadicity and summation. For readability we have relegated some of
the technical subtleties to an appendix.

\subsubsection{Process grammar}\label{subsub:process_grammar}

\begin{mathpar}
  \inferrule* [lab=synchronization] {} {{M} \bc \pzero \;|\; x?F \;|\; x!C }
  \and
  \inferrule* [lab=abstraction] {} {{F} \bc (x)P}
  \and
  \inferrule* [lab=concretion] {} {{C} \bc \langle Q \rangle}
  \and
  \inferrule* [lab=process] {} {{P,Q} \bc M \;| \;P|Q \;|\; @{x}}
  \and
  \inferrule* [lab=name] {} {{x} \bc \quotep{P}}
\end{mathpar} 

Note that $\vec{x}$ (resp. $\vec{P}$) denotes a vector of names
(resp. processes) of length $|\vec{x}|$ (resp. $|\vec{P}|$). We adopt
the following useful abbreviations.

\begin{mathpar}
   x?(\vec{y}).P := x.(\vec{y})P \and  x\clift{\vec{P}} := x.\clift{\vec{P}}
   \and x!(y) := \lift{x}{\dropn{y}}
   \and \Pi_{i=0}^{n-1}P_i := P_0 | \ldots | P_{n-1}
\end{mathpar}

\subsubsection{Structural congruence}

\paragraph{Free and bound names and alpha-equivalence.} At the
core of structural equivalence is alpha-equivalence which identifies
process that are the same up to a change of variable. Formally, we
recognize the distinction between free and bound names. The free names
of a process, $\freenames{P}$, may be calculated recursively as
follows:

\begin{mathpar}
\freenames{\pzero} := \emptyset
  \and \\
  \freenames{x?(y).P} := \{ x \} \cup (\freenames{P} \setminus \{ y \})
  \and 
  \freenames{x!\langle P \rangle} := \{ x \} \cup \{ P \} 
  \and \\
  \freenames{P|Q} := \freenames{P} \cup \freenames{Q}
  \and \\
  \freenames{@{x}} := \{ x \}
\end{mathpar}

$\pi$
$\quotep{\pi}$

$\freenames{-} : \pi \to \mathcal{P}(\quotep{\pi})$

\begin{eqnarray*}
  \freenames{\pzero} & := & \emptyset \\
  \freenames{x?(y).P} & := & \{ x \} \cup (\freenames{P} \setminus \{ y \}) \\
  \freenames{x!\langle P \rangle} & := & \{ x \} \cup \{ P \} \\
  \freenames{P|Q} & := & \freenames{P} \cup \freenames{Q} \\
  \freenames{\dropn{x}} & := & \{ x \}
\end{eqnarray*}

The bound names of a process, $\boundnames{P}$, are those names occurring in $P$
that are not free. For example, in $x?(y).0$, the name $x$ is free, while $y$ is bound.

\begin{mathpar}
  \inferrule* [lab=monoidal-laws] {} { P|Q \equiv Q|P \and P|0 \equiv P \and P|(Q|R) \equiv (P|Q)|R }
\end{mathpar}

\begin{mathpar}
  \inferrule* [lab=alpha-equivalence] {} { (x)P \equiv (y)P\{y/x\} \and y \not\in \freenames{P} }
\end{mathpar}

\begin{definition}
Then two processes, $P,Q$, are alpha-equivalent if $P = Q\{\vec{y}/\vec{x}\}$ for
some $\vec{x} \in \boundnames{Q},\vec{y} \in \boundnames{P}$, where $Q\{\vec{y}/\vec{x}\}$
denotes the capture-avoiding substitution of $\vec{y}$ for $\vec{x}$ in $Q$.
\end{definition}

\begin{definition}
  The {\em structural congruence} \cite{SangiorgiWalker} , $\equiv$,
  between processes is the least congruence containing
  alpha-equivalence, satisfying the abelian monoid laws
  (associativity, commutativity and $\pzero$ as identity) for parallel
  composition $|$ and for summation $+$.
\end{definition}

\subsection{Name equivalence}

We take name equivalence, written $\nameeq$, to be the smallest
equivalence relation generated by the following rules.

\begin{mathpar}
\inferrule*[lab=Quote-drop]
{ }
{ \quotep{@{x}} \nameeq x }

\inferrule*[lab=Struct-equiv]
{ P \scong Q }
{ \quotep{P} \nameeq \quotep{Q} }
\end{mathpar}

The astute reader will have noticed that the mutual recursion of names
and processes imposes a mutual recursion on alpha-equivalence and
structural equivalence via name-equivalence. Fortunately, all of this
works out pleasantly and we may calculate in the natural way, free of
concern. The reader interested in the details is referred to the
appendix \ref{appendix:rho_details}.

\subsection{Substitution}

We use $\Proc$ for the set of processes, $\QProc$ for the set of
names, and $\id{\{}\vec{y} / \vec{x} \id{\}}$ to denote partial maps,
$s : \QProc \rightarrow \QProc$. A map, $s$ lifts, uniquely, to a map
on process terms, $\widehat{s} : \Proc \rightarrow \Proc$ by the
following equations.

\begin{mathpar}
  (0) \psubstp{Q}{P} := 0 \\
  (R \juxtap S) \psubstp{Q}{P}
  :=    
  (R)\psubstp{Q}{P} \juxtap (S) \psubstp{Q}{P} \\
  (x?(y).R) \psubstp{Q}{P}    
  :=    
  (x)\substp{Q}{P} (z)\concat( (R \psubstn{z}{y}) \psubstp{Q}{P} ) \\
  (\lift{x}{R}) \psubstp{Q}{P}  
  :=
  \lift{(x)\substp{Q}{P}}{ R \psubstp{Q}{P} } \\
%   (\dropn{x})  \psubstp{Q}{P}       
%   := 
%   \left\{ 
%     \begin{array}{ccc} 
%       \dropn{\quotep{Q}} & & x \nameeq \quotep{P} \\
%       \dropn{x} & & otherwise \\
%     \end{array}
%   \right. 
  (\dropn{x})  \psubstp{Q}{P}       
  := 
  \left\{ 
    \begin{array}{ccc} 
      Q & & x \nameeq \quotep{P} \\
      \dropn{x} & & otherwise \\
    \end{array}
  \right.
\end{mathpar}
 

where

\begin{eqnarray}
  (x)\id{\{} \lpquote Q \rpquote / \lpquote P \rpquote \id{\}}            = 
  \left\{ 
    \begin{array}{ccc}
      \lpquote Q \rpquote & & x \nameeq \lpquote P \rpquote \\
      x & & otherwise \\
    \end{array}
  \right. \nonumber
\end{eqnarray}

and $z$ is chosen distinct from $\quotep{P}$, $\quotep{Q}$, the free
names in $Q$, and all the names in $R$. Our $\alpha$-equivalence will
be built in the standard way from this substitution.

\begin{remark}\label{rem:no_self_referential_names}
  One consequence of these definitions is that $\forall P. \quotep{P}
  \not\in \freenames{P}$.
\end{remark}

\subsection{ Dynamic quote: an example }

Anticipating something of what's to come, consider applying the
substitution, $\widehat{\id{\{}u / z \id{\}}}$, to the following pair
of processes, $\lift{w}{y!(z)}$ and $w[ \lpquote y!(z) \rpquote ]$.

\begin{eqnarray}
	\lift{w}{y!(z)}\widehat{\id{\{}u / z \id{\}}}
		& = &
		\lift{w}{y!(u)} \nonumber\\
	w[ \lpquote y!(z) \rpquote ] \widehat{ \id{\{}u / z \id{\}} }
		& = &
		w[ \lpquote y!(z) \rpquote ] \nonumber
\end{eqnarray}

Because the body of the process between quotes is impervious to
substitution, we get radically different answers. In fact, by
examining the first process in an input context,
e.g. $x?(z).\lift{w}{y!(z)}$, we see that the process under the lift
operator may be shaped by prefixed inputs binding a name inside it. In
this sense, the lift operator will be seen as a way to dynamically
construct processes before reifying them as names.

Finally equipped with these standard features we can present the
dynamics of the calculus.

\subsubsection{Operational semantics} 

Finally, we introduce the computational dynamics. What marks these
algebras as distinct from other more traditionally studied algebraic
structures, e.g. vector spaces or polynomial rings, is the manner in
which dynamics is captured. In traditional structures, dynamics is typically
expressed through morphisms between such structures, as in linear maps
between vector spaces or morphisms between rings. In algebras
associated with the semantics of computation, the dynamics is
expressed as part of the algebraic structure itself, through a
reduction reduction relation typically denoted by $\red$. Below, we
give a recursive presentation of this relation for the calculus used
in the encoding.

$\red \subseteq \pi \times \pi$
$\red : \pi \to \mathcal{P}(\pi)$

\begin{mathpar}
  \inferrule* [lab=Comm] { \textsf{match}( x_{src}, x_{trgt} ) } { x_{trgt}?(y)P \; | \; x_{src}!\langle {Q} \rangle \red P\{\quotep{Q}/y}\} }
  \and \\
  \inferrule* [lab=Par] {{P} \red {P}'} {{{P} | {Q}} \red {{P}' | {Q}}}
  \and
  \inferrule* [lab=Equiv]{{{P} \scong {P}'} \andalso {{P}' \red {Q}'} \andalso {{Q}' \scong {Q}}}{{P} \red {Q}}
\end{mathpar}

\begin{eqnarray*}
  match_{\equiv} (\quotep{P},\quotep{Q}) & := & P \equiv Q \\
  match_{\dagger}(\quotep{P},\quotep{Q}) & := & \forall R. P|Q \red^{*} R => R \red^{*} 0 \\
  match_{K}(\quotep{P},\quotep{Q}) & := & K \mbox{ for some context } K
\end{eqnarray*}

$u?(x)P | u!\langle Q \rangle \red P\{\quotep{Q}/x\}$

%We write $\wred$ for $\red^*$, and $P\red$ if $\exists Q $ such that $ P \red Q$.
We write $P\red$ if $\exists Q $ such that $ P \red Q$ and $P\not\red$, otherwise.

\section{Replication}

As mentioned before, it is known that replication (and hence
recursion) can be implemented in a higher-order process algebra
\cite{SangiorgiWalker}. As our first example of calculation with the
machinery thus far presented we give the construction explicitly in
the {\rhoc}.

\begin{eqnarray}
	D_{x} & := & \prefix{x}{y}{(\binpar{\outputp{x}{y}}{@{y}})} \nonumber\\
	\bangp_{x}{P} & := & \binpar{{x}!\langle{\binpar{D_{x}}{P}}\rangle}{D_{x}} \nonumber
\end{eqnarray}

\begin{eqnarray}
	\bangp_{x}{P} & & \nonumber\\
	=
	& {x}!\langle{(\prefix{x}{y}{(\outputp{x}{y} | @{y})) | P}}\rangle 
	      | \prefix{x}{y}{(\outputp{x}{y} | @{y})} & \nonumber\\
	\red
	& (\outputp{x}{y} | @{y})\substn{\quotep{(\prefix{x}{y}{(@{y} | \outputp{x}{y})) | P}}}{y} & \nonumber\\
	=
	& \outputp{x}{\quotep{(\prefix{x}{y}{(\outputp{x}{y} | @{y})) | P}}}
	  | {(\prefix{x}{y}{(\outputp{x}{y} | @{y})) | P}} & \nonumber\\
	\red
	& \ldots & \nonumber\\
	\red^*
	& P | P | \ldots & \nonumber
\end{eqnarray}

Of course, this encoding, as an implementation, runs away, unfolding
$\bangp{P}$ eagerly. A lazier and more implementable replication
operator, restricted to input-guarded processes, may be obtained as follows.

\begin{eqnarray}
\bangp{\prefix{u}{v}{P}} 
	:= 
	\binpar{\lift{x}{\prefix{u}{v}{(\binpar{D(x)}{P})}}}{D(x)} \nonumber
\end{eqnarray}

\begin{remark}
  Note that the lazier definition still does not deal with summation
  or mixed summation (i.e. sums over input and output). The reader is
  invited to construct definitions of replication that deal with these
  features. 

  Further, the definitions are parameterized in a name, $x$. Can you,
  gentle reader, make a definition that eliminates this parameter and
  guarantees no accidental interaction between the replication
  machinery and the process being replicated -- i.e. no accidental
  sharing of names used by the process to get its work done and the
  name(s) used by the replication to effect copying. This latter
  revision of the definition of replication is crucial to obtaining
  the expected identity $!!P \sim !P$.
\end{remark}

\begin{remark}\label{rem:paradoxical_combinator}
  The reader familiar with the lambda calculus will have noticed the
  similarity between $D$ and the paradoxical combinator.

  [Ed. note: the existence of this seems to suggest we have to be more
  restrictive on the set of processes and names we admit if we are to
  support no-cloning.]
\end{remark}

\subsubsection{Bisimulation}

The computational dynamics gives rise to another kind of equivalence,
the equivalence of computational behavior. As previously mentioned
this is typically captured \emph{via} some form of bisimulation.

% The notion we use in this paper is weak barbed bisimulation
% \cite{milner91polyadicpi}.

The notion we use in this paper is derived from weak barbed
bisimulation \cite{milner91polyadicpi}. 

\begin{definition}
An \emph{observation relation}, $\downarrow_{\mathcal N}$, over a set
of names, $\mathcal N$, is the smallest relation satisfying the rules
below.

\infrule[Out-barb]{y \in {\mathcal N}, \; x \nameeq y}
		  {\outputp{x}{v} \downarrow_{\mathcal N} x}
\infrule[Par-barb]{\mbox{$P\downarrow_{\mathcal N} x$ or $Q\downarrow_{\mathcal N} x$}}
		  {\binpar{P}{Q} \downarrow_{\mathcal N} x}

We write $P \Downarrow_{\mathcal N} x$ if there is $Q$ such that 
$P \wred Q$ and $Q \downarrow_{\mathcal N} x$.
\end{definition}

\begin{definition}
%\label{def.bbisim}
An  ${\mathcal N}$-\emph{barbed bisimulation} over a set of names, ${\mathcal N}$, is a symmetric binary relation 
${\mathcal S}_{\mathcal N}$ between agents such that $P\rel{S}_{\mathcal N}Q$ implies:
\begin{enumerate}
\item If $P \red P'$ then $Q \wred Q'$ and $P'\rel{S}_{\mathcal N} Q'$.
\item If $P\downarrow_{\mathcal N} x$, then $Q\Downarrow_{\mathcal N} x$.
\end{enumerate}
$P$ is ${\mathcal N}$-barbed bisimilar to $Q$, written
$P \wbbisim_{\mathcal N} Q$, if $P \rel{S}_{\mathcal N} Q$ for some ${\mathcal N}$-barbed bisimulation ${\mathcal S}_{\mathcal N}$.
\end{definition}

$\mathcal{R} \subseteq \pi \times \pi$

$P \mathcal{R} Q => \forall P'. P \red P' \Rightarrow \exists Q'. Q \red Q', P' \mathcal{R} Q'$

$P \vdash x \Rightarrow Q \vdash x$

\begin{mathpar}
  \inferrule*[lab=Out-barb]{x \nameeq y}{{y}!\langle{Q}\rangle \vdash x}
  \and
  \inferrule*[lab=Par-barb]{\mbox{$P\vdash x$ or $Q\vdash x$}}{\binpar{P}{Q} \vdash x}
\end{mathpar}

\subsubsection{Contexts}

One of the principle advantages of computational calculi like the
$\pi$-calculus is a well-defined notion of context,
contextual-equivalence and a correlation between
contextual-equivalence and notions of bisimulation. The notion of
context allows the decomposition of a process into (sub-)process and
its syntactic environment, its context. Thus, a context may be
thought of as a process with a ``hole'' (written $\Box$) in it. The
application of a context $M$ to a process $P$, written $M[P]$, is
tantamount to filling the hole in $M$ with $P$. In this paper we do
not need the full weight of this theory, but do make use of the notion
of context in the proof the main theorem. 

\begin{mathpar}
  \inferrule* [lab=summation] {} {{M_{M},M_{N}} \bc \Box \;|\; x.M_{A} \;|\; M_{M}+M_{N}}
  \and
  \inferrule* [lab=agent] {} {{M_{A}} \bc (\vec{x})M_{P} \;| \; \clift{P_0,\ldots,M_{P},\ldots,P_N}}
  \and \\
  \inferrule* [lab=process] {} {{M_{P}} \bc M_{N} \;| \;P|M_{P} }
\end{mathpar} 

\begin{mathpar}
  \inferrule* [lab=sychronization] {} {M_{N} \bc \Box \;|\; x?M_{F} \;|\; x!M_{C}}
  \and
  \inferrule* [lab=abstraction] {} {{M_{F}} \bc (x)M_{P} }
  \and
  \inferrule* [lab=concretion] {} {{M_{C}} \bc \langle M_{P} \rangle }
  \and \\
  \inferrule* [lab=process] {} {{M_{P}} \bc M_{N} \;| \;P|M_{P} }
\end{mathpar}

\begin{definition}[contextual application] Given a context $M$, and
  process $P$, we define the \emph{contextual application}, $M[P] :=
  M\{P/\Box\}$. That is, the contextual application of M to P is the
  substitution of $P$ for $\Box$ in $M$.
\end{definition}

$\meaningof{-} : L \to \mathcal{P}(\pi)$

\begin{mathpar}
  \inferrule* [lab=collection] {} {\meaningof{true} = \pi, \and \meaningof{~E} = \pi \setminus \meaningof{E}, \and \meaningof{E_{1} \& E_{2}} = \meaningof{E_{1}} \cap \meaningof{E_{2}}}
\end{mathpar}

\begin{mathpar}
  \inferrule* [lab=structure] {} {\meaningof{0} = \{ P \in \pi | P \equiv 0 \}, \and \\ \meaningof{E_1 | E_2} = \{ P \in \pi | P \equiv P_{1} | P_{2}, P_{1} \in \meaningof{E_{1}}, P_{2} \in \meaningof{E_2}\} }
\end{mathpar}

\begin{mathpar}
 \inferrule* [lab=behavior] {} {\meaningof{\langle a?b \rangle E} = \{ P \in \pi | P \equiv Q | u?(y)P', \\ \and \\\\ \and \\ \;\;\; u \in \meaningof{a}, \forall z.P'\{z/y\} \in \meaningof{E\{z/b\}}\}, \and \\ \meaningof{a!E} = \{ P \in \pi | P \equiv Q | x!\langle P' \rangle, x \in \meaningof{a} P' \in \meaningof{E}\} }
\end{mathpar}

\begin{mathpar}
 \inferrule* [lab=nominal] {} {\meaningof{\quotep{E}} = \{ \quotep{P} \in \quotep{\pi} | P \in \meaningof{E} \}, \and \meaningof{\quotep{P}} = \{ \quotep{Q} \in \quotep{\pi} | P \equiv Q \} \and \\ \meaningof{@\quotep{E}} = \{ P \in \pi | P \equiv @x, x \in \meaningof{E} \}}
\end{mathpar}

\begin{eqnarray*}
  \\
  \meaningof{-} : TS \to ST
\end{eqnarray*}

\begin{eqnarray*}
  \\
  L : TS \to ST
\end{eqnarray*}

\begin{eqnarray*}
  \\
  P \models E \iff P \in \meaningof{E}
\end{eqnarray*}

\begin{eqnarray*}
  P \approx_{L} Q \iff \forall E \in L. P \models E \iff Q \models E
\end{eqnarray*}

\begin{eqnarray*}
  P \approx_{K} Q
\end{eqnarray*}

\begin{eqnarray*}
  P \approx Q
\end{eqnarray*}

$\approx_{K} = \approx = \approx_{L}$

\subsubsection{Contextual duality}

Note that contexts extend the quotation operation to a family of
operations from processes to names. Given a context, $M$, we can
define a \emph{nominal context}, $\quotep{M}$ by $\quotep{M}[P] :=
\quotep{M[P]}$. To foreshadow what is to come we observe that these
operations enjoy a duality with processes very much like the duality
between vectors and maps from vectors to scalars.

Further, because the calculus is essentially higher-order, we have a
correspondence between contexts and processes. More specifically,
given a name $x$ and a context $M$ we can construct $M^{*}_{x}$ such
that 

\begin{mathpar}
  M^{*}_{x} | \lift{x}{P} \red M[P]
\end{mathpar}

namely,

\begin{mathpar}
  M^{*}_{x} := x?(u).M[\dropn{u}]
\end{mathpar}

The dependence of $M^{*}_{x}$ on a name makes it an abstraction, 

\begin{mathpar}
  M^{*} := (x)x?(u).M[\dropn{u}]
\end{mathpar}

\subsection{Additional notation}

It will sometimes be convenient to denote the process a name
quotes. We already have the notation $x = \quotep{P}$, but it will be
convenient to introduce an alternate notation, $\procn{x}$, when we
want to emphasize the connection to the use of the name. Note that, by
virtue of name equivalence, $\quotep{\procn{x}} \nameeq x$; so, the
notation is consistent with previous definitions.

Further, because names have structure it is possible to effect
substitutions on the basis of that structure. This means we need to
upgrade our notation for substitutions, which we accomplish by
adapting comprehension notation. Thus,

\begin{mathpar}
  P\{ y / x : x \in S \}
\end{mathpar}

is interpreted to mean the process derived from P by replacing (in a
capture-avoiding manner) each occurrence of $x$ in $S$ by $y$. For example,

\begin{mathpar}
  P\{ \quotep{\procn{x}|\procn{x}} / x : x \in \freenames{P} \}
\end{mathpar}

will replace each (occurrence) of a free name $x$ in $P$ by
$\quotep{\procn{x}|\procn{x}}$.

Also, we will avail ourselves of the notation $x^{L}$ and $x^{R}$ to
denote injections of a name into disjoint copies of the name
space. There are numerous ways to accomplish this. One example can be
found in \cite{MeredithR05}. This notation overloads to vectors of
names: $\vec{x}^{\pi} := (x_{i}^{\pi} \; : \; 0 \leq i < |\vec{x}| )$ where $\pi \in \{L,R\}$.

We also use $P^{\Box} := P|\Box$.

In \cite{MeredithR05} an interpretation of the new operator is
given. It turns out that there are several possible interpretations
all enjoying the requisite algebraic properties of the operator (see
\cite{milner91polyadicpi}). We will therefore make liberal use of
$(\nu\; \vec{x})P$.

% subsection the_syntax_and_semantics_of_the_notation_system (end)   

\input{qm2pi.qmops} 

\input{qm2pi.sterngerlach} 

\input{qm2pi.metric} 

% section concurrent_process_calculi (end)

%\input{qm2pi.proofsketch}

% section proof sketch (end)

%\input{qm2pi.slviaknots} 

% section spatial logic via knots (end)

\input{qm2pi.conclusion}

% section conclusion (end)

%\input{qm2pi.dtcodes} 

% section wiring algorithm (end)

\input{qm2pi.ack} 

% section acknowledgments (end)

\newpage


\bibliographystyle{plain}   
\bibliography{../../biblios/main.bib}

\input{qm2pi.rhodetails}

\end{document}

 

% subsection basic_interpretation (end)

%\input{qm2pi.rho.presentation} 
\subsection{The syntax and semantics of the notation system}\label{sub:the_syntax_and_semantics_of_the_notation_system} % (fold)

We now summarize a technical presentation of the calculus that
embodies our theory of dynamics. The typical presentation of such a
calculus follows the style of giving generators and relations on
them. The grammar, below, describing term constructors, freely
generates the set of processes, $\Proc$. This set is then quotiented
by a relation known as structural congruence and it is over this set
that the notion of dynamics is expressed. This presentation is
essentially that of \cite{MeredithR05} with the addition of
polyadicity and summation. For readability we have relegated some of
the technical subtleties to an appendix.

\subsubsection{Process grammar}\label{subsub:process_grammar}

\begin{mathpar}
  \inferrule* [lab=synchronization] {} {{M} \bc \pzero \;|\; x?F \;|\; x!C }
  \and
  \inferrule* [lab=abstraction] {} {{F} \bc (x)P}
  \and
  \inferrule* [lab=concretion] {} {{C} \bc \langle Q \rangle}
  \and
  \inferrule* [lab=process] {} {{P,Q} \bc M \;| \;P|Q \;|\; @{x}}
  \and
  \inferrule* [lab=name] {} {{x} \bc \quotep{P}}
\end{mathpar} 

Note that $\vec{x}$ (resp. $\vec{P}$) denotes a vector of names
(resp. processes) of length $|\vec{x}|$ (resp. $|\vec{P}|$). We adopt
the following useful abbreviations.

\begin{mathpar}
   x?(\vec{y}).P := x.(\vec{y})P \and  x\clift{\vec{P}} := x.\clift{\vec{P}}
   \and x!(y) := \lift{x}{\dropn{y}}
   \and \Pi_{i=0}^{n-1}P_i := P_0 | \ldots | P_{n-1}
\end{mathpar}

\subsubsection{Structural congruence}

\paragraph{Free and bound names and alpha-equivalence.} At the
core of structural equivalence is alpha-equivalence which identifies
process that are the same up to a change of variable. Formally, we
recognize the distinction between free and bound names. The free names
of a process, $\freenames{P}$, may be calculated recursively as
follows:

\begin{mathpar}
\freenames{\pzero} := \emptyset
  \and \\
  \freenames{x?(y).P} := \{ x \} \cup (\freenames{P} \setminus \{ y \})
  \and 
  \freenames{x!\langle P \rangle} := \{ x \} \cup \{ P \} 
  \and \\
  \freenames{P|Q} := \freenames{P} \cup \freenames{Q}
  \and \\
  \freenames{@{x}} := \{ x \}
\end{mathpar}

$\pi$
$\quotep{\pi}$

$\freenames{-} : \pi \to \mathcal{P}(\quotep{\pi})$

\begin{eqnarray*}
  \freenames{\pzero} & := & \emptyset \\
  \freenames{x?(y).P} & := & \{ x \} \cup (\freenames{P} \setminus \{ y \}) \\
  \freenames{x!\langle P \rangle} & := & \{ x \} \cup \{ P \} \\
  \freenames{P|Q} & := & \freenames{P} \cup \freenames{Q} \\
  \freenames{\dropn{x}} & := & \{ x \}
\end{eqnarray*}

The bound names of a process, $\boundnames{P}$, are those names occurring in $P$
that are not free. For example, in $x?(y).0$, the name $x$ is free, while $y$ is bound.

\begin{mathpar}
  \inferrule* [lab=monoidal-laws] {} { P|Q \equiv Q|P \and P|0 \equiv P \and P|(Q|R) \equiv (P|Q)|R }
\end{mathpar}

\begin{mathpar}
  \inferrule* [lab=alpha-equivalence] {} { (x)P \equiv (y)P\{y/x\} \and y \not\in \freenames{P} }
\end{mathpar}

\begin{definition}
Then two processes, $P,Q$, are alpha-equivalent if $P = Q\{\vec{y}/\vec{x}\}$ for
some $\vec{x} \in \boundnames{Q},\vec{y} \in \boundnames{P}$, where $Q\{\vec{y}/\vec{x}\}$
denotes the capture-avoiding substitution of $\vec{y}$ for $\vec{x}$ in $Q$.
\end{definition}

\begin{definition}
  The {\em structural congruence} \cite{SangiorgiWalker} , $\equiv$,
  between processes is the least congruence containing
  alpha-equivalence, satisfying the abelian monoid laws
  (associativity, commutativity and $\pzero$ as identity) for parallel
  composition $|$ and for summation $+$.
\end{definition}

\subsection{Name equivalence}

We take name equivalence, written $\nameeq$, to be the smallest
equivalence relation generated by the following rules.

\begin{mathpar}
\inferrule*[lab=Quote-drop]
{ }
{ \quotep{@{x}} \nameeq x }

\inferrule*[lab=Struct-equiv]
{ P \scong Q }
{ \quotep{P} \nameeq \quotep{Q} }
\end{mathpar}

The astute reader will have noticed that the mutual recursion of names
and processes imposes a mutual recursion on alpha-equivalence and
structural equivalence via name-equivalence. Fortunately, all of this
works out pleasantly and we may calculate in the natural way, free of
concern. The reader interested in the details is referred to the
appendix \ref{appendix:rho_details}.

\subsection{Substitution}

We use $\Proc$ for the set of processes, $\QProc$ for the set of
names, and $\id{\{}\vec{y} / \vec{x} \id{\}}$ to denote partial maps,
$s : \QProc \rightarrow \QProc$. A map, $s$ lifts, uniquely, to a map
on process terms, $\widehat{s} : \Proc \rightarrow \Proc$ by the
following equations.

\begin{mathpar}
  (0) \psubstp{Q}{P} := 0 \\
  (R \juxtap S) \psubstp{Q}{P}
  :=    
  (R)\psubstp{Q}{P} \juxtap (S) \psubstp{Q}{P} \\
  (x?(y).R) \psubstp{Q}{P}    
  :=    
  (x)\substp{Q}{P} (z)\concat( (R \psubstn{z}{y}) \psubstp{Q}{P} ) \\
  (\lift{x}{R}) \psubstp{Q}{P}  
  :=
  \lift{(x)\substp{Q}{P}}{ R \psubstp{Q}{P} } \\
%   (\dropn{x})  \psubstp{Q}{P}       
%   := 
%   \left\{ 
%     \begin{array}{ccc} 
%       \dropn{\quotep{Q}} & & x \nameeq \quotep{P} \\
%       \dropn{x} & & otherwise \\
%     \end{array}
%   \right. 
  (\dropn{x})  \psubstp{Q}{P}       
  := 
  \left\{ 
    \begin{array}{ccc} 
      Q & & x \nameeq \quotep{P} \\
      \dropn{x} & & otherwise \\
    \end{array}
  \right.
\end{mathpar}
 

where

\begin{eqnarray}
  (x)\id{\{} \lpquote Q \rpquote / \lpquote P \rpquote \id{\}}            = 
  \left\{ 
    \begin{array}{ccc}
      \lpquote Q \rpquote & & x \nameeq \lpquote P \rpquote \\
      x & & otherwise \\
    \end{array}
  \right. \nonumber
\end{eqnarray}

and $z$ is chosen distinct from $\quotep{P}$, $\quotep{Q}$, the free
names in $Q$, and all the names in $R$. Our $\alpha$-equivalence will
be built in the standard way from this substitution.

\begin{remark}\label{rem:no_self_referential_names}
  One consequence of these definitions is that $\forall P. \quotep{P}
  \not\in \freenames{P}$.
\end{remark}

\subsection{ Dynamic quote: an example }

Anticipating something of what's to come, consider applying the
substitution, $\widehat{\id{\{}u / z \id{\}}}$, to the following pair
of processes, $\lift{w}{y!(z)}$ and $w[ \lpquote y!(z) \rpquote ]$.

\begin{eqnarray}
	\lift{w}{y!(z)}\widehat{\id{\{}u / z \id{\}}}
		& = &
		\lift{w}{y!(u)} \nonumber\\
	w[ \lpquote y!(z) \rpquote ] \widehat{ \id{\{}u / z \id{\}} }
		& = &
		w[ \lpquote y!(z) \rpquote ] \nonumber
\end{eqnarray}

Because the body of the process between quotes is impervious to
substitution, we get radically different answers. In fact, by
examining the first process in an input context,
e.g. $x?(z).\lift{w}{y!(z)}$, we see that the process under the lift
operator may be shaped by prefixed inputs binding a name inside it. In
this sense, the lift operator will be seen as a way to dynamically
construct processes before reifying them as names.

Finally equipped with these standard features we can present the
dynamics of the calculus.

\subsubsection{Operational semantics} 

Finally, we introduce the computational dynamics. What marks these
algebras as distinct from other more traditionally studied algebraic
structures, e.g. vector spaces or polynomial rings, is the manner in
which dynamics is captured. In traditional structures, dynamics is typically
expressed through morphisms between such structures, as in linear maps
between vector spaces or morphisms between rings. In algebras
associated with the semantics of computation, the dynamics is
expressed as part of the algebraic structure itself, through a
reduction reduction relation typically denoted by $\red$. Below, we
give a recursive presentation of this relation for the calculus used
in the encoding.

$\red \subseteq \pi \times \pi$
$\red : \pi \to \mathcal{P}(\pi)$

\begin{mathpar}
  \inferrule* [lab=Comm] { \textsf{match}( x_{src}, x_{trgt} ) } { x_{trgt}?(y)P \; | \; x_{src}!\langle {Q} \rangle \red P\{\quotep{Q}/y}\} }
  \and \\
  \inferrule* [lab=Par] {{P} \red {P}'} {{{P} | {Q}} \red {{P}' | {Q}}}
  \and
  \inferrule* [lab=Equiv]{{{P} \scong {P}'} \andalso {{P}' \red {Q}'} \andalso {{Q}' \scong {Q}}}{{P} \red {Q}}
\end{mathpar}

\begin{eqnarray*}
  match_{\equiv} (\quotep{P},\quotep{Q}) & := & P \equiv Q \\
  match_{\dagger}(\quotep{P},\quotep{Q}) & := & \forall R. P|Q \red^{*} R => R \red^{*} 0 \\
  match_{K}(\quotep{P},\quotep{Q}) & := & K \mbox{ for some context } K
\end{eqnarray*}

$u?(x)P | u!\langle Q \rangle \red P\{\quotep{Q}/x\}$

%We write $\wred$ for $\red^*$, and $P\red$ if $\exists Q $ such that $ P \red Q$.
We write $P\red$ if $\exists Q $ such that $ P \red Q$ and $P\not\red$, otherwise.

\section{Replication}

As mentioned before, it is known that replication (and hence
recursion) can be implemented in a higher-order process algebra
\cite{SangiorgiWalker}. As our first example of calculation with the
machinery thus far presented we give the construction explicitly in
the {\rhoc}.

\begin{eqnarray}
	D_{x} & := & \prefix{x}{y}{(\binpar{\outputp{x}{y}}{@{y}})} \nonumber\\
	\bangp_{x}{P} & := & \binpar{{x}!\langle{\binpar{D_{x}}{P}}\rangle}{D_{x}} \nonumber
\end{eqnarray}

\begin{eqnarray}
	\bangp_{x}{P} & & \nonumber\\
	=
	& {x}!\langle{(\prefix{x}{y}{(\outputp{x}{y} | @{y})) | P}}\rangle 
	      | \prefix{x}{y}{(\outputp{x}{y} | @{y})} & \nonumber\\
	\red
	& (\outputp{x}{y} | @{y})\substn{\quotep{(\prefix{x}{y}{(@{y} | \outputp{x}{y})) | P}}}{y} & \nonumber\\
	=
	& \outputp{x}{\quotep{(\prefix{x}{y}{(\outputp{x}{y} | @{y})) | P}}}
	  | {(\prefix{x}{y}{(\outputp{x}{y} | @{y})) | P}} & \nonumber\\
	\red
	& \ldots & \nonumber\\
	\red^*
	& P | P | \ldots & \nonumber
\end{eqnarray}

Of course, this encoding, as an implementation, runs away, unfolding
$\bangp{P}$ eagerly. A lazier and more implementable replication
operator, restricted to input-guarded processes, may be obtained as follows.

\begin{eqnarray}
\bangp{\prefix{u}{v}{P}} 
	:= 
	\binpar{\lift{x}{\prefix{u}{v}{(\binpar{D(x)}{P})}}}{D(x)} \nonumber
\end{eqnarray}

\begin{remark}
  Note that the lazier definition still does not deal with summation
  or mixed summation (i.e. sums over input and output). The reader is
  invited to construct definitions of replication that deal with these
  features. 

  Further, the definitions are parameterized in a name, $x$. Can you,
  gentle reader, make a definition that eliminates this parameter and
  guarantees no accidental interaction between the replication
  machinery and the process being replicated -- i.e. no accidental
  sharing of names used by the process to get its work done and the
  name(s) used by the replication to effect copying. This latter
  revision of the definition of replication is crucial to obtaining
  the expected identity $!!P \sim !P$.
\end{remark}

\begin{remark}\label{rem:paradoxical_combinator}
  The reader familiar with the lambda calculus will have noticed the
  similarity between $D$ and the paradoxical combinator.

  [Ed. note: the existence of this seems to suggest we have to be more
  restrictive on the set of processes and names we admit if we are to
  support no-cloning.]
\end{remark}

\subsubsection{Bisimulation}

The computational dynamics gives rise to another kind of equivalence,
the equivalence of computational behavior. As previously mentioned
this is typically captured \emph{via} some form of bisimulation.

% The notion we use in this paper is weak barbed bisimulation
% \cite{milner91polyadicpi}.

The notion we use in this paper is derived from weak barbed
bisimulation \cite{milner91polyadicpi}. 

\begin{definition}
An \emph{observation relation}, $\downarrow_{\mathcal N}$, over a set
of names, $\mathcal N$, is the smallest relation satisfying the rules
below.

\infrule[Out-barb]{y \in {\mathcal N}, \; x \nameeq y}
		  {\outputp{x}{v} \downarrow_{\mathcal N} x}
\infrule[Par-barb]{\mbox{$P\downarrow_{\mathcal N} x$ or $Q\downarrow_{\mathcal N} x$}}
		  {\binpar{P}{Q} \downarrow_{\mathcal N} x}

We write $P \Downarrow_{\mathcal N} x$ if there is $Q$ such that 
$P \wred Q$ and $Q \downarrow_{\mathcal N} x$.
\end{definition}

\begin{definition}
%\label{def.bbisim}
An  ${\mathcal N}$-\emph{barbed bisimulation} over a set of names, ${\mathcal N}$, is a symmetric binary relation 
${\mathcal S}_{\mathcal N}$ between agents such that $P\rel{S}_{\mathcal N}Q$ implies:
\begin{enumerate}
\item If $P \red P'$ then $Q \wred Q'$ and $P'\rel{S}_{\mathcal N} Q'$.
\item If $P\downarrow_{\mathcal N} x$, then $Q\Downarrow_{\mathcal N} x$.
\end{enumerate}
$P$ is ${\mathcal N}$-barbed bisimilar to $Q$, written
$P \wbbisim_{\mathcal N} Q$, if $P \rel{S}_{\mathcal N} Q$ for some ${\mathcal N}$-barbed bisimulation ${\mathcal S}_{\mathcal N}$.
\end{definition}

$\mathcal{R} \subseteq \pi \times \pi$

$P \mathcal{R} Q => \forall P'. P \red P' \Rightarrow \exists Q'. Q \red Q', P' \mathcal{R} Q'$

$P \vdash x \Rightarrow Q \vdash x$

\begin{mathpar}
  \inferrule*[lab=Out-barb]{x \nameeq y}{{y}!\langle{Q}\rangle \vdash x}
  \and
  \inferrule*[lab=Par-barb]{\mbox{$P\vdash x$ or $Q\vdash x$}}{\binpar{P}{Q} \vdash x}
\end{mathpar}

\subsubsection{Contexts}

One of the principle advantages of computational calculi like the
$\pi$-calculus is a well-defined notion of context,
contextual-equivalence and a correlation between
contextual-equivalence and notions of bisimulation. The notion of
context allows the decomposition of a process into (sub-)process and
its syntactic environment, its context. Thus, a context may be
thought of as a process with a ``hole'' (written $\Box$) in it. The
application of a context $M$ to a process $P$, written $M[P]$, is
tantamount to filling the hole in $M$ with $P$. In this paper we do
not need the full weight of this theory, but do make use of the notion
of context in the proof the main theorem. 

\begin{mathpar}
  \inferrule* [lab=summation] {} {{M_{M},M_{N}} \bc \Box \;|\; x.M_{A} \;|\; M_{M}+M_{N}}
  \and
  \inferrule* [lab=agent] {} {{M_{A}} \bc (\vec{x})M_{P} \;| \; \clift{P_0,\ldots,M_{P},\ldots,P_N}}
  \and \\
  \inferrule* [lab=process] {} {{M_{P}} \bc M_{N} \;| \;P|M_{P} }
\end{mathpar} 

\begin{mathpar}
  \inferrule* [lab=sychronization] {} {M_{N} \bc \Box \;|\; x?M_{F} \;|\; x!M_{C}}
  \and
  \inferrule* [lab=abstraction] {} {{M_{F}} \bc (x)M_{P} }
  \and
  \inferrule* [lab=concretion] {} {{M_{C}} \bc \langle M_{P} \rangle }
  \and \\
  \inferrule* [lab=process] {} {{M_{P}} \bc M_{N} \;| \;P|M_{P} }
\end{mathpar}

\begin{definition}[contextual application] Given a context $M$, and
  process $P$, we define the \emph{contextual application}, $M[P] :=
  M\{P/\Box\}$. That is, the contextual application of M to P is the
  substitution of $P$ for $\Box$ in $M$.
\end{definition}

$\meaningof{-} : L \to \mathcal{P}(\pi)$

\begin{mathpar}
  \inferrule* [lab=collection] {} {\meaningof{true} = \pi, \and \meaningof{~E} = \pi \setminus \meaningof{E}, \and \meaningof{E_{1} \& E_{2}} = \meaningof{E_{1}} \cap \meaningof{E_{2}}}
\end{mathpar}

\begin{mathpar}
  \inferrule* [lab=structure] {} {\meaningof{0} = \{ P \in \pi | P \equiv 0 \}, \and \\ \meaningof{E_1 | E_2} = \{ P \in \pi | P \equiv P_{1} | P_{2}, P_{1} \in \meaningof{E_{1}}, P_{2} \in \meaningof{E_2}\} }
\end{mathpar}

\begin{mathpar}
 \inferrule* [lab=behavior] {} {\meaningof{\langle a?b \rangle E} = \{ P \in \pi | P \equiv Q | u?(y)P', \\ \and \\\\ \and \\ \;\;\; u \in \meaningof{a}, \forall z.P'\{z/y\} \in \meaningof{E\{z/b\}}\}, \and \\ \meaningof{a!E} = \{ P \in \pi | P \equiv Q | x!\langle P' \rangle, x \in \meaningof{a} P' \in \meaningof{E}\} }
\end{mathpar}

\begin{mathpar}
 \inferrule* [lab=nominal] {} {\meaningof{\quotep{E}} = \{ \quotep{P} \in \quotep{\pi} | P \in \meaningof{E} \}, \and \meaningof{\quotep{P}} = \{ \quotep{Q} \in \quotep{\pi} | P \equiv Q \} \and \\ \meaningof{@\quotep{E}} = \{ P \in \pi | P \equiv @x, x \in \meaningof{E} \}}
\end{mathpar}

\begin{eqnarray*}
  \\
  \meaningof{-} : TS \to ST
\end{eqnarray*}

\begin{eqnarray*}
  \\
  L : TS \to ST
\end{eqnarray*}

\begin{eqnarray*}
  \\
  P \models E \iff P \in \meaningof{E}
\end{eqnarray*}

\begin{eqnarray*}
  P \approx_{L} Q \iff \forall E \in L. P \models E \iff Q \models E
\end{eqnarray*}

\begin{eqnarray*}
  P \approx_{K} Q
\end{eqnarray*}

\begin{eqnarray*}
  P \approx Q
\end{eqnarray*}

$\approx_{K} = \approx = \approx_{L}$

\subsubsection{Contextual duality}

Note that contexts extend the quotation operation to a family of
operations from processes to names. Given a context, $M$, we can
define a \emph{nominal context}, $\quotep{M}$ by $\quotep{M}[P] :=
\quotep{M[P]}$. To foreshadow what is to come we observe that these
operations enjoy a duality with processes very much like the duality
between vectors and maps from vectors to scalars.

Further, because the calculus is essentially higher-order, we have a
correspondence between contexts and processes. More specifically,
given a name $x$ and a context $M$ we can construct $M^{*}_{x}$ such
that 

\begin{mathpar}
  M^{*}_{x} | \lift{x}{P} \red M[P]
\end{mathpar}

namely,

\begin{mathpar}
  M^{*}_{x} := x?(u).M[\dropn{u}]
\end{mathpar}

The dependence of $M^{*}_{x}$ on a name makes it an abstraction, 

\begin{mathpar}
  M^{*} := (x)x?(u).M[\dropn{u}]
\end{mathpar}

\subsection{Additional notation}

It will sometimes be convenient to denote the process a name
quotes. We already have the notation $x = \quotep{P}$, but it will be
convenient to introduce an alternate notation, $\procn{x}$, when we
want to emphasize the connection to the use of the name. Note that, by
virtue of name equivalence, $\quotep{\procn{x}} \nameeq x$; so, the
notation is consistent with previous definitions.

Further, because names have structure it is possible to effect
substitutions on the basis of that structure. This means we need to
upgrade our notation for substitutions, which we accomplish by
adapting comprehension notation. Thus,

\begin{mathpar}
  P\{ y / x : x \in S \}
\end{mathpar}

is interpreted to mean the process derived from P by replacing (in a
capture-avoiding manner) each occurrence of $x$ in $S$ by $y$. For example,

\begin{mathpar}
  P\{ \quotep{\procn{x}|\procn{x}} / x : x \in \freenames{P} \}
\end{mathpar}

will replace each (occurrence) of a free name $x$ in $P$ by
$\quotep{\procn{x}|\procn{x}}$.

Also, we will avail ourselves of the notation $x^{L}$ and $x^{R}$ to
denote injections of a name into disjoint copies of the name
space. There are numerous ways to accomplish this. One example can be
found in \cite{MeredithR05}. This notation overloads to vectors of
names: $\vec{x}^{\pi} := (x_{i}^{\pi} \; : \; 0 \leq i < |\vec{x}| )$ where $\pi \in \{L,R\}$.

We also use $P^{\Box} := P|\Box$.

In \cite{MeredithR05} an interpretation of the new operator is
given. It turns out that there are several possible interpretations
all enjoying the requisite algebraic properties of the operator (see
\cite{milner91polyadicpi}). We will therefore make liberal use of
$(\nu\; \vec{x})P$.

% subsection the_syntax_and_semantics_of_the_notation_system (end)   

\section{Interpretation of QM}
\subsection{Supporting definitions}
\subsubsection{Multiplication}
\begin{mathpar}
  \quotep{Q} \cdot \quotep{R} := \quotep{Q|R}
  \and \\
  \quotep{Q} \cdot P := P\{ \quotep{Q|R} / \quotep{R} : \quotep{R} \in \freenames{P} \}
\end{mathpar}

\paragraph{Discussion}
The first line needs little explanation. The second line says that
each free name of the process is replaced with the multiplication of
that name by the scalar. Multiplication of a scalar (name) by a state
(process) results in a process all the names of which have been `moved
over' by parallel composition with the process the scalar
quotes. There is a subtlety that the bound names have to be
manipulated so that multiplied names aren't accidentally
captured. There are many ways to achieve this.

\begin{remark}\label{rem:multiplication_identities}
  The reader is invited to verify that for all $x,y,z \in \QProc$ and $P \in \Proc$
  \begin{mathpar}
    x \cdot \quotep{0} \equiv x 
    \and
    x \cdot y \equiv y \cdot x
    \and
    x \cdot (y \cdot z) \equiv (x \cdot y) \cdot z
    \and \\
    \quotep{0} \cdot P \equiv P
    \and \\
    x \cdot (y \cdot P) \equiv (x \cdot y) \cdot P
    \and \\
    x \cdot (P|Q) \equiv (x \cdot P) | (x \cdot Q)
    \and \\    
  \end{mathpar}
\end{remark}

\subsubsection{Tensor product}

We define a tensor product on processes by structural induction.

\paragraph{Tensor of sums} First note that all summations, including
$\pzero$ and sequence, can be written $\Sigma_{i} x_{i}.A_{i} +
\Sigma_{j} x_{j}.C_{j}$, where we have grouped input-guarded processes
together and output-guarded processes together.

Thus, we can define the tensor product of two summations, $N_{1}\otimes N_{2}$, where

\begin{mathpar}
  N_{1} := \Sigma_{i} x_{i}.A_{i} + \Sigma_{j} x_{j}.C_{j}
  \and
  N_{2} := \Sigma_{i'} y_{i'}.B_{i'} + \Sigma_{j'} y_{j'}.D_{j'} 
\end{mathpar}

as follows.

\begin{mathpar}
  \Sigma_{i} x_{i}.A_{i} + \Sigma_{j} x_{j}.C_{j} \otimes \Sigma_{i'}
  y_{i'}.B_{i'} + \Sigma_{j'} y_{j'}.D_{j'} 
  \and \\
  := \; \Sigma_{i} \Sigma_{i'} \quotep{\stackrel{\vee}{x_{i}}| \stackrel{\vee}{y_{i'}}}.(A_{i}\otimes B_{i'}) \; | \; \Sigma_{i'} \Sigma_{i} \quotep{\stackrel{\vee}{y_{i'}}|\stackrel{\vee}{x_{i}}}.(B_{i'}\otimes A_{i})
  \and
  \;\; | \;\; \Sigma_{j} \Sigma_{j'} \quotep{\stackrel{\vee}{x_{j}}|\stackrel{\vee}{y_{j'}}}.(A_{j}\otimes B_{j'}) \; | \; \Sigma_{j'} \Sigma_{j} \quotep{\stackrel{\vee}{y_{j'}}|\stackrel{\vee}{x_{j}}}.(B_{j'}\otimes A_{j})
\end{mathpar}

\begin{remark}
  Do we need to $x^{L}$ and $y^{R}$ for this construction as well?
\end{remark}

\paragraph{Tensor of parallel compositions} Next, we distribute tensor
over par.

\begin{mathpar}
  P_{1}|P_{2} \otimes Q_{1}|Q_{2} := (P_{1} \otimes Q_{1}) | (P_{1}
  \otimes Q_{2}) | (P_{2} \otimes Q_{1}) | (P_{2} \otimes Q_{2})
\end{mathpar}

\paragraph{Tensor with dropped names} We treat tensor of a
process with a dropped name as parallel composition.

\begin{mathpar}
  P \otimes \dropn{x} := P | \dropn{x}
\end{mathpar}

\paragraph{Tensor of agents}

Finally, we need to define tensor on agents. Note that the definition
of tensor on normal products only tensors inputs with inputs and
outputs with outputs. Thus, we only have to define the operation on
``homogeneous'' pairings.

\begin{mathpar}
  (\vec{x})P \otimes (\vec{y})Q
  \and \\
  := (x_{0}^{L}|y_{0}^{R},\ldots,x_{0}^{L}|y_{n}^{R},\ldots,x_{m}^{L}|y_{0}^{R},\ldots,x_{m}^{L}|y_{n}^R)(P\{ \vec{x}^{L}/\vec{x}\} \otimes Q \{ \vec{y}^{R}/\vec{y}\})
  \and \\
  \clift{\vec{P}} \otimes \clift{\vec{Q}}
  \and \\
  := \clift{P_{0}\otimes Q_{0},\ldots,P_{0}\otimes Q_{n},\ldots,P_{m}\otimes Q_{0},\ldots,P_{m}\otimes Q_{n}}
\end{mathpar}

\begin{remark}
  Observe that arities of tensored abstractions matches arities of
  tensored concretions if the original arities matched. Note also that
  the length of the arities corresponds to the increase in dimension
  we see in ordinary vector space tensor product.
\end{remark}

\begin{remark}
  Operationally, this definition distributes the tensor down to
  components ``linked'' by summation. Tensor over summation is
  intriguing in that it mixes names. Moreover, as a consequence of the
  way it mixes names we have the identities for all $x \in \QProc$ and
  $P,Q \in \Proc$

  \begin{mathpar}
    (x \cdot P) \otimes Q \equiv x \cdot (P \otimes Q) \equiv P \otimes (x \cdot Q)
    \and
    P \otimes \pzero \equiv P
  \end{mathpar}

  that the reader is invited to verify.
\end{remark}

\subsubsection{Annihilation}
\begin{mathpar}
  P^{\perp} := \{ Q | \forall R. P|Q \red^{*} R \Rightarrow R \red^{*} \pzero \}
  \and \\
  P^{\underline{\perp}} := \Sigma_{Q \in P^{\perp}} \quotep{Q}?(y).(\dropn{y}|Q) | \Sigma_{Q \in P^{\perp}} \quotep{Q}\clift{\Box}
\end{mathpar}

\paragraph{Discussion} The reader will note that $P^{\perp}$ is a
\emph{set} of processes, while $P^{\underline{\perp}}$ is a
\emph{context}. We call the set $P^{\perp}$ the \emph{annihilators} of
$P$. The parallel composition of a process in the annihilators of $P$
with $P$ will result in a process, the state space of which has all
paths eventually leading to $\pzero$. Execution may endure loops; but
under reasonable conditions of fairness (naturally guaranteed under
most notions of bisimulation) such a composite process cannot get
stuck in such a loop and will, eventually pop out and terminate.

The context $P^{\underline{\perp}}$ is ready and willing to ``take the
$P$ out of'' the process to which it is applied. It will effectively
transmit the code of the process to which it is applied to one of the
annihilators and run the process against it.

\subsubsection{Evaluation}
We fix $M$ a domain of fully abstract interpretation with an equality
coincident with bisimulation. We take $\meaningof{\cdot} : \Proc \to
M$ to be the map interpreting processes and $\nmeaningof{\cdot} : \M
\to Proc$ to be the map running the other way. Then we define

\begin{mathpar}
  \int P := \nmeaningof{\meaningof{P}}
\end{mathpar}

\paragraph{Discussion}
There are many fully abstract interpretations of Milner's
$\pi$-calculus. Any of them can be used as a basis for interpreting
the reflective calculus here. Equipped with such a domain it is
largely a matter of grinding through to check that the Yoneda
construction for the normalization-by-evaluation program can be
extended to this setting.

\begin{remark}
  The reader is invited to verify that $\int (P^{\underline{\perp}}[P]) = 0$.
\end{remark}

\subsection{Quantum mechanics}

Table \ref{tbl:core_qm_op_defns} gives the core operational definitions

\begin{table}[htp]\label{tbl:core_qm_op_defns}
  \center{
    \fbox{
      \begin{tabular}{c|c}
        quantum mechanics & process calculus \\
        \hline
        scalar & $x := \quotep{P}$ \\
        state vector & $\state{P} := P$ \\
        dual & $\state{P}^{*} := \event{P^{\underline{\perp}}} := \quotep{P^{\underline{\perp}}}[-]$ \\
        matrix & $ \Sigma_{\alpha} \state{P_{\alpha}}x_{\alpha}\event{Q_{\alpha}}$ \\
        vector addition & $\state{P} + \state{Q} := \state{P | Q}$ \\
        tensor product & $\state{P} \otimes \state{Q} := \state{P \otimes Q}$ \\
        inner product & $\innerprod{P}{Q} := \quotep{\int P^{\underline{\perp}}[Q]}$ \\
      \end{tabular}
    }
  }
  \caption{QM - operational definitions}
\end{table}

where

\begin{mathpar}
  \prmatrix{P}{Q} := \fprmatrix{P}{\quotep{\pzero}}{Q}
  \and
  \fprmatrix{P}{x}{Q} := (\state{P},x,\event{Q})
  \and
  (\fprmatrix{P}{x}{Q})(\state{R}) := x \cdot \innerprod{Q}{R} \cdot \state{P}
  \and
  (\fprmatrix{P}{x}{Q})(\event{R}) := x \cdot \innerprod{R}{P} \cdot \event{Q}
\end{mathpar}

\paragraph{Discussion}
As promised: vectors (aka states) are represented as processes; duals
as contextual duals; inner product definition should be compared with
standard inner product definition for ....

\begin{remark}
  Assuming $\int (P^{\underline{\perp}}[P]) = 0$, the reader is
  invited to verify that $(\fprmatrix{P}{x}{P})(\state{P}) = x \cdot \state{P}$.
\end{remark}

\begin{remark}
  The reader is invited to verify that $\innerprod{P}{Q}$ could
  equally well have been written $\quotep{\int \stackrel{\vee}{x}}$
  where $x = \event{P^{\underline{\perp}}}(Q)$.

  One of the motivations for this remark is that there is another way
  to factor these operations. We could package up evaluation in the dual:

  \begin{mathpar}
    \state{P}^{*} := \event{\int P^{\underline{\perp}}} := \quotep{\int P^{\underline{\perp}}}[-]
  \end{mathpar}

  and then have inner product defined by
  
  \begin{mathpar}
    \innerprod{P}{Q} := \event{P}(Q)
  \end{mathpar}

  Hopefully, experience with the calculations will provide guidance on
  the best factoring.
\end{remark}

\begin{remark}
  Assuming $\int (P^{\underline{\perp}}[P]) = 0$, the reader is
  invited to verify that $\forall P,Q. (\prmatrix{0}{Q})(\state{0}) =
  \state{0}$ and dually $(\prmatrix{P}{0})(\event{0}) = \event{0}$.
\end{remark}

\begin{remark}
  i'm a little worried that i don't (yet) have proper support for
  complex conjugacy. But, the observation above may give us a
  clue. According to Abramsky, it must be the case that the scalars
  are iso to the homset of the identity for the tensor -- which the
  observation above characterizes. 

  For now, we will simply bookmark the notion with $\overline{x}$.
\end{remark}

\subsubsection{Adjointness}

We need to give a definition of $(\cdot)^{\dagger}$ for matrices. The
obvious candidate definition is
\begin{mathpar}
(\Sigma_{\alpha}\fprmatrix{P_{\alpha}}{x_{\alpha}}{Q_{\alpha}})^{\dagger}
= \Sigma_{\alpha}\fprmatrix{(Q_{\alpha}^{\underline{\perp}})^{*}}{\overline{x}_{\alpha}}{P_{\alpha}^{\underline{\perp}}} 
\end{mathpar}

But, $(Q_{\alpha}^{\underline{\perp}})^{*}$ requires a name along
which to communicate the process to achieve the context application.

\subsubsection{Basis for a basis}
If processes label states and ``addition'' of states (a.k.a. vector
addition) is interpreted as parallel composition, what corresponds to
notions of linear independence and basis? Here, we recall that Yoshida
has developed a set of \emph{combinators} for an asynchronous verison
of Milner's $\pi$-calculus. These are a finite set of processes such
any process can be expressed as parallel composition of these
combinators together with liberal uses of the new operator and
replication. We can simply give a translation of these into the
present calculus and have reasonable expectation that the property
carries over. That is, that the resultant set allows to express all
processes via parallel composition. Note, however, that there is no
new operator or replication in this calculus. As a result, we expect
that the corresponding set is actually infinite. That is, we expect
that the space is actually infinite dimensional.

\begin{remark}
  The attentive reader may be a bit concerned. Certainly, the
  collection $S$, $K$ and $I$ is a finite set of
  combinators. Shouldn't we expect to see a finite set of combinators
  for an effectively equivalent system? i am very sympathetic to this
  critique and feel it warrants full attention. On the other hand, i
  also have in mind the following analogy. The natural numbers, as a
  monoid under addition, has exactly $1$ generator, while the natural
  numbers, as a monoid under multiplication, has countably many
  generators (the primes). We observe that the application of the
  lambda calculus is much less resource sensitive than the parallel
  composition of the $\pi$-calculus. Could it be the case that we have
  an analogy of the form
  
  \begin{mathpar}
    m + n : MN :: m*n : M|N
  \end{mathpar}

  giving a similar blow up in the set of ``primes''?  This is such a
  wonderful thought that, even if it's not true, i think it's worth
  writing down.
\end{remark}
 

\documentclass[12pt]{llncs}
%\documentclass{jktr}

\usepackage[pdftex]{hyperref}                   
\usepackage {listings}
\usepackage {mathpartir}
\usepackage{bcprules}
%\usepackage{listings}
                       
\usepackage{graphicx} 
%\usepackage[margins=2.5cm,nohead,nofoot]{geometry}
%\usepackage{geometry}
\usepackage{amsfonts}
\usepackage{amstext}
\usepackage{latexsym}
\usepackage{amssymb}
\usepackage{color}


%\include{myPreamble}
\include{qm2pi.local} 

%\ifpdf
%\usepackage[pdftex]{graphicx}
%\else
%\usepackage{graphicx}
%\fi

 % \ifpdf
%  \usepackage{pdfsync}
%  \if


%\title{Brief Article}
%\author{David F. Snyder}
%\author{L.G. Meredith}

%\address{Dept. of Math., Texas State University--San Marcos, San Marcos, TX 78666}
       
\pagestyle{empty}


\begin{document}

\lstset{language=[Objective]Caml,frame=shadowbox}

\input{qm2pi.front}

% section front matter (end)

\input{qm2pi.intro} 
 
% section introduction (end)

% \input{qm2pi.knotations} 

% section notation (end)

\input{qm2pi.process.calculi} 

% section concurrent_process_calculi_and_spatial_logics_ (end)
    
%\input{qm2pi.knots2pi} 

%\input{qm2pi.trefoil} 

%\input{qm2pi.mainthm} 

% subsection basic_interpretation (end)

%\input{qm2pi.rho.presentation} 
\subsection{The syntax and semantics of the notation system}\label{sub:the_syntax_and_semantics_of_the_notation_system} % (fold)

We now summarize a technical presentation of the calculus that
embodies our theory of dynamics. The typical presentation of such a
calculus follows the style of giving generators and relations on
them. The grammar, below, describing term constructors, freely
generates the set of processes, $\Proc$. This set is then quotiented
by a relation known as structural congruence and it is over this set
that the notion of dynamics is expressed. This presentation is
essentially that of \cite{MeredithR05} with the addition of
polyadicity and summation. For readability we have relegated some of
the technical subtleties to an appendix.

\subsubsection{Process grammar}\label{subsub:process_grammar}

\begin{mathpar}
  \inferrule* [lab=synchronization] {} {{M} \bc \pzero \;|\; x?F \;|\; x!C }
  \and
  \inferrule* [lab=abstraction] {} {{F} \bc (x)P}
  \and
  \inferrule* [lab=concretion] {} {{C} \bc \langle Q \rangle}
  \and
  \inferrule* [lab=process] {} {{P,Q} \bc M \;| \;P|Q \;|\; @{x}}
  \and
  \inferrule* [lab=name] {} {{x} \bc \quotep{P}}
\end{mathpar} 

Note that $\vec{x}$ (resp. $\vec{P}$) denotes a vector of names
(resp. processes) of length $|\vec{x}|$ (resp. $|\vec{P}|$). We adopt
the following useful abbreviations.

\begin{mathpar}
   x?(\vec{y}).P := x.(\vec{y})P \and  x\clift{\vec{P}} := x.\clift{\vec{P}}
   \and x!(y) := \lift{x}{\dropn{y}}
   \and \Pi_{i=0}^{n-1}P_i := P_0 | \ldots | P_{n-1}
\end{mathpar}

\subsubsection{Structural congruence}

\paragraph{Free and bound names and alpha-equivalence.} At the
core of structural equivalence is alpha-equivalence which identifies
process that are the same up to a change of variable. Formally, we
recognize the distinction between free and bound names. The free names
of a process, $\freenames{P}$, may be calculated recursively as
follows:

\begin{mathpar}
\freenames{\pzero} := \emptyset
  \and \\
  \freenames{x?(y).P} := \{ x \} \cup (\freenames{P} \setminus \{ y \})
  \and 
  \freenames{x!\langle P \rangle} := \{ x \} \cup \{ P \} 
  \and \\
  \freenames{P|Q} := \freenames{P} \cup \freenames{Q}
  \and \\
  \freenames{@{x}} := \{ x \}
\end{mathpar}

$\pi$
$\quotep{\pi}$

$\freenames{-} : \pi \to \mathcal{P}(\quotep{\pi})$

\begin{eqnarray*}
  \freenames{\pzero} & := & \emptyset \\
  \freenames{x?(y).P} & := & \{ x \} \cup (\freenames{P} \setminus \{ y \}) \\
  \freenames{x!\langle P \rangle} & := & \{ x \} \cup \{ P \} \\
  \freenames{P|Q} & := & \freenames{P} \cup \freenames{Q} \\
  \freenames{\dropn{x}} & := & \{ x \}
\end{eqnarray*}

The bound names of a process, $\boundnames{P}$, are those names occurring in $P$
that are not free. For example, in $x?(y).0$, the name $x$ is free, while $y$ is bound.

\begin{mathpar}
  \inferrule* [lab=monoidal-laws] {} { P|Q \equiv Q|P \and P|0 \equiv P \and P|(Q|R) \equiv (P|Q)|R }
\end{mathpar}

\begin{mathpar}
  \inferrule* [lab=alpha-equivalence] {} { (x)P \equiv (y)P\{y/x\} \and y \not\in \freenames{P} }
\end{mathpar}

\begin{definition}
Then two processes, $P,Q$, are alpha-equivalent if $P = Q\{\vec{y}/\vec{x}\}$ for
some $\vec{x} \in \boundnames{Q},\vec{y} \in \boundnames{P}$, where $Q\{\vec{y}/\vec{x}\}$
denotes the capture-avoiding substitution of $\vec{y}$ for $\vec{x}$ in $Q$.
\end{definition}

\begin{definition}
  The {\em structural congruence} \cite{SangiorgiWalker} , $\equiv$,
  between processes is the least congruence containing
  alpha-equivalence, satisfying the abelian monoid laws
  (associativity, commutativity and $\pzero$ as identity) for parallel
  composition $|$ and for summation $+$.
\end{definition}

\subsection{Name equivalence}

We take name equivalence, written $\nameeq$, to be the smallest
equivalence relation generated by the following rules.

\begin{mathpar}
\inferrule*[lab=Quote-drop]
{ }
{ \quotep{@{x}} \nameeq x }

\inferrule*[lab=Struct-equiv]
{ P \scong Q }
{ \quotep{P} \nameeq \quotep{Q} }
\end{mathpar}

The astute reader will have noticed that the mutual recursion of names
and processes imposes a mutual recursion on alpha-equivalence and
structural equivalence via name-equivalence. Fortunately, all of this
works out pleasantly and we may calculate in the natural way, free of
concern. The reader interested in the details is referred to the
appendix \ref{appendix:rho_details}.

\subsection{Substitution}

We use $\Proc$ for the set of processes, $\QProc$ for the set of
names, and $\id{\{}\vec{y} / \vec{x} \id{\}}$ to denote partial maps,
$s : \QProc \rightarrow \QProc$. A map, $s$ lifts, uniquely, to a map
on process terms, $\widehat{s} : \Proc \rightarrow \Proc$ by the
following equations.

\begin{mathpar}
  (0) \psubstp{Q}{P} := 0 \\
  (R \juxtap S) \psubstp{Q}{P}
  :=    
  (R)\psubstp{Q}{P} \juxtap (S) \psubstp{Q}{P} \\
  (x?(y).R) \psubstp{Q}{P}    
  :=    
  (x)\substp{Q}{P} (z)\concat( (R \psubstn{z}{y}) \psubstp{Q}{P} ) \\
  (\lift{x}{R}) \psubstp{Q}{P}  
  :=
  \lift{(x)\substp{Q}{P}}{ R \psubstp{Q}{P} } \\
%   (\dropn{x})  \psubstp{Q}{P}       
%   := 
%   \left\{ 
%     \begin{array}{ccc} 
%       \dropn{\quotep{Q}} & & x \nameeq \quotep{P} \\
%       \dropn{x} & & otherwise \\
%     \end{array}
%   \right. 
  (\dropn{x})  \psubstp{Q}{P}       
  := 
  \left\{ 
    \begin{array}{ccc} 
      Q & & x \nameeq \quotep{P} \\
      \dropn{x} & & otherwise \\
    \end{array}
  \right.
\end{mathpar}
 

where

\begin{eqnarray}
  (x)\id{\{} \lpquote Q \rpquote / \lpquote P \rpquote \id{\}}            = 
  \left\{ 
    \begin{array}{ccc}
      \lpquote Q \rpquote & & x \nameeq \lpquote P \rpquote \\
      x & & otherwise \\
    \end{array}
  \right. \nonumber
\end{eqnarray}

and $z$ is chosen distinct from $\quotep{P}$, $\quotep{Q}$, the free
names in $Q$, and all the names in $R$. Our $\alpha$-equivalence will
be built in the standard way from this substitution.

\begin{remark}\label{rem:no_self_referential_names}
  One consequence of these definitions is that $\forall P. \quotep{P}
  \not\in \freenames{P}$.
\end{remark}

\subsection{ Dynamic quote: an example }

Anticipating something of what's to come, consider applying the
substitution, $\widehat{\id{\{}u / z \id{\}}}$, to the following pair
of processes, $\lift{w}{y!(z)}$ and $w[ \lpquote y!(z) \rpquote ]$.

\begin{eqnarray}
	\lift{w}{y!(z)}\widehat{\id{\{}u / z \id{\}}}
		& = &
		\lift{w}{y!(u)} \nonumber\\
	w[ \lpquote y!(z) \rpquote ] \widehat{ \id{\{}u / z \id{\}} }
		& = &
		w[ \lpquote y!(z) \rpquote ] \nonumber
\end{eqnarray}

Because the body of the process between quotes is impervious to
substitution, we get radically different answers. In fact, by
examining the first process in an input context,
e.g. $x?(z).\lift{w}{y!(z)}$, we see that the process under the lift
operator may be shaped by prefixed inputs binding a name inside it. In
this sense, the lift operator will be seen as a way to dynamically
construct processes before reifying them as names.

Finally equipped with these standard features we can present the
dynamics of the calculus.

\subsubsection{Operational semantics} 

Finally, we introduce the computational dynamics. What marks these
algebras as distinct from other more traditionally studied algebraic
structures, e.g. vector spaces or polynomial rings, is the manner in
which dynamics is captured. In traditional structures, dynamics is typically
expressed through morphisms between such structures, as in linear maps
between vector spaces or morphisms between rings. In algebras
associated with the semantics of computation, the dynamics is
expressed as part of the algebraic structure itself, through a
reduction reduction relation typically denoted by $\red$. Below, we
give a recursive presentation of this relation for the calculus used
in the encoding.

$\red \subseteq \pi \times \pi$
$\red : \pi \to \mathcal{P}(\pi)$

\begin{mathpar}
  \inferrule* [lab=Comm] { \textsf{match}( x_{src}, x_{trgt} ) } { x_{trgt}?(y)P \; | \; x_{src}!\langle {Q} \rangle \red P\{\quotep{Q}/y}\} }
  \and \\
  \inferrule* [lab=Par] {{P} \red {P}'} {{{P} | {Q}} \red {{P}' | {Q}}}
  \and
  \inferrule* [lab=Equiv]{{{P} \scong {P}'} \andalso {{P}' \red {Q}'} \andalso {{Q}' \scong {Q}}}{{P} \red {Q}}
\end{mathpar}

\begin{eqnarray*}
  match_{\equiv} (\quotep{P},\quotep{Q}) & := & P \equiv Q \\
  match_{\dagger}(\quotep{P},\quotep{Q}) & := & \forall R. P|Q \red^{*} R => R \red^{*} 0 \\
  match_{K}(\quotep{P},\quotep{Q}) & := & K \mbox{ for some context } K
\end{eqnarray*}

$u?(x)P | u!\langle Q \rangle \red P\{\quotep{Q}/x\}$

%We write $\wred$ for $\red^*$, and $P\red$ if $\exists Q $ such that $ P \red Q$.
We write $P\red$ if $\exists Q $ such that $ P \red Q$ and $P\not\red$, otherwise.

\section{Replication}

As mentioned before, it is known that replication (and hence
recursion) can be implemented in a higher-order process algebra
\cite{SangiorgiWalker}. As our first example of calculation with the
machinery thus far presented we give the construction explicitly in
the {\rhoc}.

\begin{eqnarray}
	D_{x} & := & \prefix{x}{y}{(\binpar{\outputp{x}{y}}{@{y}})} \nonumber\\
	\bangp_{x}{P} & := & \binpar{{x}!\langle{\binpar{D_{x}}{P}}\rangle}{D_{x}} \nonumber
\end{eqnarray}

\begin{eqnarray}
	\bangp_{x}{P} & & \nonumber\\
	=
	& {x}!\langle{(\prefix{x}{y}{(\outputp{x}{y} | @{y})) | P}}\rangle 
	      | \prefix{x}{y}{(\outputp{x}{y} | @{y})} & \nonumber\\
	\red
	& (\outputp{x}{y} | @{y})\substn{\quotep{(\prefix{x}{y}{(@{y} | \outputp{x}{y})) | P}}}{y} & \nonumber\\
	=
	& \outputp{x}{\quotep{(\prefix{x}{y}{(\outputp{x}{y} | @{y})) | P}}}
	  | {(\prefix{x}{y}{(\outputp{x}{y} | @{y})) | P}} & \nonumber\\
	\red
	& \ldots & \nonumber\\
	\red^*
	& P | P | \ldots & \nonumber
\end{eqnarray}

Of course, this encoding, as an implementation, runs away, unfolding
$\bangp{P}$ eagerly. A lazier and more implementable replication
operator, restricted to input-guarded processes, may be obtained as follows.

\begin{eqnarray}
\bangp{\prefix{u}{v}{P}} 
	:= 
	\binpar{\lift{x}{\prefix{u}{v}{(\binpar{D(x)}{P})}}}{D(x)} \nonumber
\end{eqnarray}

\begin{remark}
  Note that the lazier definition still does not deal with summation
  or mixed summation (i.e. sums over input and output). The reader is
  invited to construct definitions of replication that deal with these
  features. 

  Further, the definitions are parameterized in a name, $x$. Can you,
  gentle reader, make a definition that eliminates this parameter and
  guarantees no accidental interaction between the replication
  machinery and the process being replicated -- i.e. no accidental
  sharing of names used by the process to get its work done and the
  name(s) used by the replication to effect copying. This latter
  revision of the definition of replication is crucial to obtaining
  the expected identity $!!P \sim !P$.
\end{remark}

\begin{remark}\label{rem:paradoxical_combinator}
  The reader familiar with the lambda calculus will have noticed the
  similarity between $D$ and the paradoxical combinator.

  [Ed. note: the existence of this seems to suggest we have to be more
  restrictive on the set of processes and names we admit if we are to
  support no-cloning.]
\end{remark}

\subsubsection{Bisimulation}

The computational dynamics gives rise to another kind of equivalence,
the equivalence of computational behavior. As previously mentioned
this is typically captured \emph{via} some form of bisimulation.

% The notion we use in this paper is weak barbed bisimulation
% \cite{milner91polyadicpi}.

The notion we use in this paper is derived from weak barbed
bisimulation \cite{milner91polyadicpi}. 

\begin{definition}
An \emph{observation relation}, $\downarrow_{\mathcal N}$, over a set
of names, $\mathcal N$, is the smallest relation satisfying the rules
below.

\infrule[Out-barb]{y \in {\mathcal N}, \; x \nameeq y}
		  {\outputp{x}{v} \downarrow_{\mathcal N} x}
\infrule[Par-barb]{\mbox{$P\downarrow_{\mathcal N} x$ or $Q\downarrow_{\mathcal N} x$}}
		  {\binpar{P}{Q} \downarrow_{\mathcal N} x}

We write $P \Downarrow_{\mathcal N} x$ if there is $Q$ such that 
$P \wred Q$ and $Q \downarrow_{\mathcal N} x$.
\end{definition}

\begin{definition}
%\label{def.bbisim}
An  ${\mathcal N}$-\emph{barbed bisimulation} over a set of names, ${\mathcal N}$, is a symmetric binary relation 
${\mathcal S}_{\mathcal N}$ between agents such that $P\rel{S}_{\mathcal N}Q$ implies:
\begin{enumerate}
\item If $P \red P'$ then $Q \wred Q'$ and $P'\rel{S}_{\mathcal N} Q'$.
\item If $P\downarrow_{\mathcal N} x$, then $Q\Downarrow_{\mathcal N} x$.
\end{enumerate}
$P$ is ${\mathcal N}$-barbed bisimilar to $Q$, written
$P \wbbisim_{\mathcal N} Q$, if $P \rel{S}_{\mathcal N} Q$ for some ${\mathcal N}$-barbed bisimulation ${\mathcal S}_{\mathcal N}$.
\end{definition}

$\mathcal{R} \subseteq \pi \times \pi$

$P \mathcal{R} Q => \forall P'. P \red P' \Rightarrow \exists Q'. Q \red Q', P' \mathcal{R} Q'$

$P \vdash x \Rightarrow Q \vdash x$

\begin{mathpar}
  \inferrule*[lab=Out-barb]{x \nameeq y}{{y}!\langle{Q}\rangle \vdash x}
  \and
  \inferrule*[lab=Par-barb]{\mbox{$P\vdash x$ or $Q\vdash x$}}{\binpar{P}{Q} \vdash x}
\end{mathpar}

\subsubsection{Contexts}

One of the principle advantages of computational calculi like the
$\pi$-calculus is a well-defined notion of context,
contextual-equivalence and a correlation between
contextual-equivalence and notions of bisimulation. The notion of
context allows the decomposition of a process into (sub-)process and
its syntactic environment, its context. Thus, a context may be
thought of as a process with a ``hole'' (written $\Box$) in it. The
application of a context $M$ to a process $P$, written $M[P]$, is
tantamount to filling the hole in $M$ with $P$. In this paper we do
not need the full weight of this theory, but do make use of the notion
of context in the proof the main theorem. 

\begin{mathpar}
  \inferrule* [lab=summation] {} {{M_{M},M_{N}} \bc \Box \;|\; x.M_{A} \;|\; M_{M}+M_{N}}
  \and
  \inferrule* [lab=agent] {} {{M_{A}} \bc (\vec{x})M_{P} \;| \; \clift{P_0,\ldots,M_{P},\ldots,P_N}}
  \and \\
  \inferrule* [lab=process] {} {{M_{P}} \bc M_{N} \;| \;P|M_{P} }
\end{mathpar} 

\begin{mathpar}
  \inferrule* [lab=sychronization] {} {M_{N} \bc \Box \;|\; x?M_{F} \;|\; x!M_{C}}
  \and
  \inferrule* [lab=abstraction] {} {{M_{F}} \bc (x)M_{P} }
  \and
  \inferrule* [lab=concretion] {} {{M_{C}} \bc \langle M_{P} \rangle }
  \and \\
  \inferrule* [lab=process] {} {{M_{P}} \bc M_{N} \;| \;P|M_{P} }
\end{mathpar}

\begin{definition}[contextual application] Given a context $M$, and
  process $P$, we define the \emph{contextual application}, $M[P] :=
  M\{P/\Box\}$. That is, the contextual application of M to P is the
  substitution of $P$ for $\Box$ in $M$.
\end{definition}

$\meaningof{-} : L \to \mathcal{P}(\pi)$

\begin{mathpar}
  \inferrule* [lab=collection] {} {\meaningof{true} = \pi, \and \meaningof{~E} = \pi \setminus \meaningof{E}, \and \meaningof{E_{1} \& E_{2}} = \meaningof{E_{1}} \cap \meaningof{E_{2}}}
\end{mathpar}

\begin{mathpar}
  \inferrule* [lab=structure] {} {\meaningof{0} = \{ P \in \pi | P \equiv 0 \}, \and \\ \meaningof{E_1 | E_2} = \{ P \in \pi | P \equiv P_{1} | P_{2}, P_{1} \in \meaningof{E_{1}}, P_{2} \in \meaningof{E_2}\} }
\end{mathpar}

\begin{mathpar}
 \inferrule* [lab=behavior] {} {\meaningof{\langle a?b \rangle E} = \{ P \in \pi | P \equiv Q | u?(y)P', \\ \and \\\\ \and \\ \;\;\; u \in \meaningof{a}, \forall z.P'\{z/y\} \in \meaningof{E\{z/b\}}\}, \and \\ \meaningof{a!E} = \{ P \in \pi | P \equiv Q | x!\langle P' \rangle, x \in \meaningof{a} P' \in \meaningof{E}\} }
\end{mathpar}

\begin{mathpar}
 \inferrule* [lab=nominal] {} {\meaningof{\quotep{E}} = \{ \quotep{P} \in \quotep{\pi} | P \in \meaningof{E} \}, \and \meaningof{\quotep{P}} = \{ \quotep{Q} \in \quotep{\pi} | P \equiv Q \} \and \\ \meaningof{@\quotep{E}} = \{ P \in \pi | P \equiv @x, x \in \meaningof{E} \}}
\end{mathpar}

\begin{eqnarray*}
  \\
  \meaningof{-} : TS \to ST
\end{eqnarray*}

\begin{eqnarray*}
  \\
  L : TS \to ST
\end{eqnarray*}

\begin{eqnarray*}
  \\
  P \models E \iff P \in \meaningof{E}
\end{eqnarray*}

\begin{eqnarray*}
  P \approx_{L} Q \iff \forall E \in L. P \models E \iff Q \models E
\end{eqnarray*}

\begin{eqnarray*}
  P \approx_{K} Q
\end{eqnarray*}

\begin{eqnarray*}
  P \approx Q
\end{eqnarray*}

$\approx_{K} = \approx = \approx_{L}$

\subsubsection{Contextual duality}

Note that contexts extend the quotation operation to a family of
operations from processes to names. Given a context, $M$, we can
define a \emph{nominal context}, $\quotep{M}$ by $\quotep{M}[P] :=
\quotep{M[P]}$. To foreshadow what is to come we observe that these
operations enjoy a duality with processes very much like the duality
between vectors and maps from vectors to scalars.

Further, because the calculus is essentially higher-order, we have a
correspondence between contexts and processes. More specifically,
given a name $x$ and a context $M$ we can construct $M^{*}_{x}$ such
that 

\begin{mathpar}
  M^{*}_{x} | \lift{x}{P} \red M[P]
\end{mathpar}

namely,

\begin{mathpar}
  M^{*}_{x} := x?(u).M[\dropn{u}]
\end{mathpar}

The dependence of $M^{*}_{x}$ on a name makes it an abstraction, 

\begin{mathpar}
  M^{*} := (x)x?(u).M[\dropn{u}]
\end{mathpar}

\subsection{Additional notation}

It will sometimes be convenient to denote the process a name
quotes. We already have the notation $x = \quotep{P}$, but it will be
convenient to introduce an alternate notation, $\procn{x}$, when we
want to emphasize the connection to the use of the name. Note that, by
virtue of name equivalence, $\quotep{\procn{x}} \nameeq x$; so, the
notation is consistent with previous definitions.

Further, because names have structure it is possible to effect
substitutions on the basis of that structure. This means we need to
upgrade our notation for substitutions, which we accomplish by
adapting comprehension notation. Thus,

\begin{mathpar}
  P\{ y / x : x \in S \}
\end{mathpar}

is interpreted to mean the process derived from P by replacing (in a
capture-avoiding manner) each occurrence of $x$ in $S$ by $y$. For example,

\begin{mathpar}
  P\{ \quotep{\procn{x}|\procn{x}} / x : x \in \freenames{P} \}
\end{mathpar}

will replace each (occurrence) of a free name $x$ in $P$ by
$\quotep{\procn{x}|\procn{x}}$.

Also, we will avail ourselves of the notation $x^{L}$ and $x^{R}$ to
denote injections of a name into disjoint copies of the name
space. There are numerous ways to accomplish this. One example can be
found in \cite{MeredithR05}. This notation overloads to vectors of
names: $\vec{x}^{\pi} := (x_{i}^{\pi} \; : \; 0 \leq i < |\vec{x}| )$ where $\pi \in \{L,R\}$.

We also use $P^{\Box} := P|\Box$.

In \cite{MeredithR05} an interpretation of the new operator is
given. It turns out that there are several possible interpretations
all enjoying the requisite algebraic properties of the operator (see
\cite{milner91polyadicpi}). We will therefore make liberal use of
$(\nu\; \vec{x})P$.

% subsection the_syntax_and_semantics_of_the_notation_system (end)   

\input{qm2pi.qmops} 

\input{qm2pi.sterngerlach} 

\input{qm2pi.metric} 

% section concurrent_process_calculi (end)

%\input{qm2pi.proofsketch}

% section proof sketch (end)

%\input{qm2pi.slviaknots} 

% section spatial logic via knots (end)

\input{qm2pi.conclusion}

% section conclusion (end)

%\input{qm2pi.dtcodes} 

% section wiring algorithm (end)

\input{qm2pi.ack} 

% section acknowledgments (end)

\newpage


\bibliographystyle{plain}   
\bibliography{../../biblios/main.bib}

\input{qm2pi.rhodetails}

\end{document}

 

\documentclass[12pt]{llncs}
%\documentclass{jktr}

\usepackage[pdftex]{hyperref}                   
\usepackage {listings}
\usepackage {mathpartir}
\usepackage{bcprules}
%\usepackage{listings}
                       
\usepackage{graphicx} 
%\usepackage[margins=2.5cm,nohead,nofoot]{geometry}
%\usepackage{geometry}
\usepackage{amsfonts}
\usepackage{amstext}
\usepackage{latexsym}
\usepackage{amssymb}
\usepackage{color}


%\include{myPreamble}
\include{qm2pi.local} 

%\ifpdf
%\usepackage[pdftex]{graphicx}
%\else
%\usepackage{graphicx}
%\fi

 % \ifpdf
%  \usepackage{pdfsync}
%  \if


%\title{Brief Article}
%\author{David F. Snyder}
%\author{L.G. Meredith}

%\address{Dept. of Math., Texas State University--San Marcos, San Marcos, TX 78666}
       
\pagestyle{empty}


\begin{document}

\lstset{language=[Objective]Caml,frame=shadowbox}

\input{qm2pi.front}

% section front matter (end)

\input{qm2pi.intro} 
 
% section introduction (end)

% \input{qm2pi.knotations} 

% section notation (end)

\input{qm2pi.process.calculi} 

% section concurrent_process_calculi_and_spatial_logics_ (end)
    
%\input{qm2pi.knots2pi} 

%\input{qm2pi.trefoil} 

%\input{qm2pi.mainthm} 

% subsection basic_interpretation (end)

%\input{qm2pi.rho.presentation} 
\subsection{The syntax and semantics of the notation system}\label{sub:the_syntax_and_semantics_of_the_notation_system} % (fold)

We now summarize a technical presentation of the calculus that
embodies our theory of dynamics. The typical presentation of such a
calculus follows the style of giving generators and relations on
them. The grammar, below, describing term constructors, freely
generates the set of processes, $\Proc$. This set is then quotiented
by a relation known as structural congruence and it is over this set
that the notion of dynamics is expressed. This presentation is
essentially that of \cite{MeredithR05} with the addition of
polyadicity and summation. For readability we have relegated some of
the technical subtleties to an appendix.

\subsubsection{Process grammar}\label{subsub:process_grammar}

\begin{mathpar}
  \inferrule* [lab=synchronization] {} {{M} \bc \pzero \;|\; x?F \;|\; x!C }
  \and
  \inferrule* [lab=abstraction] {} {{F} \bc (x)P}
  \and
  \inferrule* [lab=concretion] {} {{C} \bc \langle Q \rangle}
  \and
  \inferrule* [lab=process] {} {{P,Q} \bc M \;| \;P|Q \;|\; @{x}}
  \and
  \inferrule* [lab=name] {} {{x} \bc \quotep{P}}
\end{mathpar} 

Note that $\vec{x}$ (resp. $\vec{P}$) denotes a vector of names
(resp. processes) of length $|\vec{x}|$ (resp. $|\vec{P}|$). We adopt
the following useful abbreviations.

\begin{mathpar}
   x?(\vec{y}).P := x.(\vec{y})P \and  x\clift{\vec{P}} := x.\clift{\vec{P}}
   \and x!(y) := \lift{x}{\dropn{y}}
   \and \Pi_{i=0}^{n-1}P_i := P_0 | \ldots | P_{n-1}
\end{mathpar}

\subsubsection{Structural congruence}

\paragraph{Free and bound names and alpha-equivalence.} At the
core of structural equivalence is alpha-equivalence which identifies
process that are the same up to a change of variable. Formally, we
recognize the distinction between free and bound names. The free names
of a process, $\freenames{P}$, may be calculated recursively as
follows:

\begin{mathpar}
\freenames{\pzero} := \emptyset
  \and \\
  \freenames{x?(y).P} := \{ x \} \cup (\freenames{P} \setminus \{ y \})
  \and 
  \freenames{x!\langle P \rangle} := \{ x \} \cup \{ P \} 
  \and \\
  \freenames{P|Q} := \freenames{P} \cup \freenames{Q}
  \and \\
  \freenames{@{x}} := \{ x \}
\end{mathpar}

$\pi$
$\quotep{\pi}$

$\freenames{-} : \pi \to \mathcal{P}(\quotep{\pi})$

\begin{eqnarray*}
  \freenames{\pzero} & := & \emptyset \\
  \freenames{x?(y).P} & := & \{ x \} \cup (\freenames{P} \setminus \{ y \}) \\
  \freenames{x!\langle P \rangle} & := & \{ x \} \cup \{ P \} \\
  \freenames{P|Q} & := & \freenames{P} \cup \freenames{Q} \\
  \freenames{\dropn{x}} & := & \{ x \}
\end{eqnarray*}

The bound names of a process, $\boundnames{P}$, are those names occurring in $P$
that are not free. For example, in $x?(y).0$, the name $x$ is free, while $y$ is bound.

\begin{mathpar}
  \inferrule* [lab=monoidal-laws] {} { P|Q \equiv Q|P \and P|0 \equiv P \and P|(Q|R) \equiv (P|Q)|R }
\end{mathpar}

\begin{mathpar}
  \inferrule* [lab=alpha-equivalence] {} { (x)P \equiv (y)P\{y/x\} \and y \not\in \freenames{P} }
\end{mathpar}

\begin{definition}
Then two processes, $P,Q$, are alpha-equivalent if $P = Q\{\vec{y}/\vec{x}\}$ for
some $\vec{x} \in \boundnames{Q},\vec{y} \in \boundnames{P}$, where $Q\{\vec{y}/\vec{x}\}$
denotes the capture-avoiding substitution of $\vec{y}$ for $\vec{x}$ in $Q$.
\end{definition}

\begin{definition}
  The {\em structural congruence} \cite{SangiorgiWalker} , $\equiv$,
  between processes is the least congruence containing
  alpha-equivalence, satisfying the abelian monoid laws
  (associativity, commutativity and $\pzero$ as identity) for parallel
  composition $|$ and for summation $+$.
\end{definition}

\subsection{Name equivalence}

We take name equivalence, written $\nameeq$, to be the smallest
equivalence relation generated by the following rules.

\begin{mathpar}
\inferrule*[lab=Quote-drop]
{ }
{ \quotep{@{x}} \nameeq x }

\inferrule*[lab=Struct-equiv]
{ P \scong Q }
{ \quotep{P} \nameeq \quotep{Q} }
\end{mathpar}

The astute reader will have noticed that the mutual recursion of names
and processes imposes a mutual recursion on alpha-equivalence and
structural equivalence via name-equivalence. Fortunately, all of this
works out pleasantly and we may calculate in the natural way, free of
concern. The reader interested in the details is referred to the
appendix \ref{appendix:rho_details}.

\subsection{Substitution}

We use $\Proc$ for the set of processes, $\QProc$ for the set of
names, and $\id{\{}\vec{y} / \vec{x} \id{\}}$ to denote partial maps,
$s : \QProc \rightarrow \QProc$. A map, $s$ lifts, uniquely, to a map
on process terms, $\widehat{s} : \Proc \rightarrow \Proc$ by the
following equations.

\begin{mathpar}
  (0) \psubstp{Q}{P} := 0 \\
  (R \juxtap S) \psubstp{Q}{P}
  :=    
  (R)\psubstp{Q}{P} \juxtap (S) \psubstp{Q}{P} \\
  (x?(y).R) \psubstp{Q}{P}    
  :=    
  (x)\substp{Q}{P} (z)\concat( (R \psubstn{z}{y}) \psubstp{Q}{P} ) \\
  (\lift{x}{R}) \psubstp{Q}{P}  
  :=
  \lift{(x)\substp{Q}{P}}{ R \psubstp{Q}{P} } \\
%   (\dropn{x})  \psubstp{Q}{P}       
%   := 
%   \left\{ 
%     \begin{array}{ccc} 
%       \dropn{\quotep{Q}} & & x \nameeq \quotep{P} \\
%       \dropn{x} & & otherwise \\
%     \end{array}
%   \right. 
  (\dropn{x})  \psubstp{Q}{P}       
  := 
  \left\{ 
    \begin{array}{ccc} 
      Q & & x \nameeq \quotep{P} \\
      \dropn{x} & & otherwise \\
    \end{array}
  \right.
\end{mathpar}
 

where

\begin{eqnarray}
  (x)\id{\{} \lpquote Q \rpquote / \lpquote P \rpquote \id{\}}            = 
  \left\{ 
    \begin{array}{ccc}
      \lpquote Q \rpquote & & x \nameeq \lpquote P \rpquote \\
      x & & otherwise \\
    \end{array}
  \right. \nonumber
\end{eqnarray}

and $z$ is chosen distinct from $\quotep{P}$, $\quotep{Q}$, the free
names in $Q$, and all the names in $R$. Our $\alpha$-equivalence will
be built in the standard way from this substitution.

\begin{remark}\label{rem:no_self_referential_names}
  One consequence of these definitions is that $\forall P. \quotep{P}
  \not\in \freenames{P}$.
\end{remark}

\subsection{ Dynamic quote: an example }

Anticipating something of what's to come, consider applying the
substitution, $\widehat{\id{\{}u / z \id{\}}}$, to the following pair
of processes, $\lift{w}{y!(z)}$ and $w[ \lpquote y!(z) \rpquote ]$.

\begin{eqnarray}
	\lift{w}{y!(z)}\widehat{\id{\{}u / z \id{\}}}
		& = &
		\lift{w}{y!(u)} \nonumber\\
	w[ \lpquote y!(z) \rpquote ] \widehat{ \id{\{}u / z \id{\}} }
		& = &
		w[ \lpquote y!(z) \rpquote ] \nonumber
\end{eqnarray}

Because the body of the process between quotes is impervious to
substitution, we get radically different answers. In fact, by
examining the first process in an input context,
e.g. $x?(z).\lift{w}{y!(z)}$, we see that the process under the lift
operator may be shaped by prefixed inputs binding a name inside it. In
this sense, the lift operator will be seen as a way to dynamically
construct processes before reifying them as names.

Finally equipped with these standard features we can present the
dynamics of the calculus.

\subsubsection{Operational semantics} 

Finally, we introduce the computational dynamics. What marks these
algebras as distinct from other more traditionally studied algebraic
structures, e.g. vector spaces or polynomial rings, is the manner in
which dynamics is captured. In traditional structures, dynamics is typically
expressed through morphisms between such structures, as in linear maps
between vector spaces or morphisms between rings. In algebras
associated with the semantics of computation, the dynamics is
expressed as part of the algebraic structure itself, through a
reduction reduction relation typically denoted by $\red$. Below, we
give a recursive presentation of this relation for the calculus used
in the encoding.

$\red \subseteq \pi \times \pi$
$\red : \pi \to \mathcal{P}(\pi)$

\begin{mathpar}
  \inferrule* [lab=Comm] { \textsf{match}( x_{src}, x_{trgt} ) } { x_{trgt}?(y)P \; | \; x_{src}!\langle {Q} \rangle \red P\{\quotep{Q}/y}\} }
  \and \\
  \inferrule* [lab=Par] {{P} \red {P}'} {{{P} | {Q}} \red {{P}' | {Q}}}
  \and
  \inferrule* [lab=Equiv]{{{P} \scong {P}'} \andalso {{P}' \red {Q}'} \andalso {{Q}' \scong {Q}}}{{P} \red {Q}}
\end{mathpar}

\begin{eqnarray*}
  match_{\equiv} (\quotep{P},\quotep{Q}) & := & P \equiv Q \\
  match_{\dagger}(\quotep{P},\quotep{Q}) & := & \forall R. P|Q \red^{*} R => R \red^{*} 0 \\
  match_{K}(\quotep{P},\quotep{Q}) & := & K \mbox{ for some context } K
\end{eqnarray*}

$u?(x)P | u!\langle Q \rangle \red P\{\quotep{Q}/x\}$

%We write $\wred$ for $\red^*$, and $P\red$ if $\exists Q $ such that $ P \red Q$.
We write $P\red$ if $\exists Q $ such that $ P \red Q$ and $P\not\red$, otherwise.

\section{Replication}

As mentioned before, it is known that replication (and hence
recursion) can be implemented in a higher-order process algebra
\cite{SangiorgiWalker}. As our first example of calculation with the
machinery thus far presented we give the construction explicitly in
the {\rhoc}.

\begin{eqnarray}
	D_{x} & := & \prefix{x}{y}{(\binpar{\outputp{x}{y}}{@{y}})} \nonumber\\
	\bangp_{x}{P} & := & \binpar{{x}!\langle{\binpar{D_{x}}{P}}\rangle}{D_{x}} \nonumber
\end{eqnarray}

\begin{eqnarray}
	\bangp_{x}{P} & & \nonumber\\
	=
	& {x}!\langle{(\prefix{x}{y}{(\outputp{x}{y} | @{y})) | P}}\rangle 
	      | \prefix{x}{y}{(\outputp{x}{y} | @{y})} & \nonumber\\
	\red
	& (\outputp{x}{y} | @{y})\substn{\quotep{(\prefix{x}{y}{(@{y} | \outputp{x}{y})) | P}}}{y} & \nonumber\\
	=
	& \outputp{x}{\quotep{(\prefix{x}{y}{(\outputp{x}{y} | @{y})) | P}}}
	  | {(\prefix{x}{y}{(\outputp{x}{y} | @{y})) | P}} & \nonumber\\
	\red
	& \ldots & \nonumber\\
	\red^*
	& P | P | \ldots & \nonumber
\end{eqnarray}

Of course, this encoding, as an implementation, runs away, unfolding
$\bangp{P}$ eagerly. A lazier and more implementable replication
operator, restricted to input-guarded processes, may be obtained as follows.

\begin{eqnarray}
\bangp{\prefix{u}{v}{P}} 
	:= 
	\binpar{\lift{x}{\prefix{u}{v}{(\binpar{D(x)}{P})}}}{D(x)} \nonumber
\end{eqnarray}

\begin{remark}
  Note that the lazier definition still does not deal with summation
  or mixed summation (i.e. sums over input and output). The reader is
  invited to construct definitions of replication that deal with these
  features. 

  Further, the definitions are parameterized in a name, $x$. Can you,
  gentle reader, make a definition that eliminates this parameter and
  guarantees no accidental interaction between the replication
  machinery and the process being replicated -- i.e. no accidental
  sharing of names used by the process to get its work done and the
  name(s) used by the replication to effect copying. This latter
  revision of the definition of replication is crucial to obtaining
  the expected identity $!!P \sim !P$.
\end{remark}

\begin{remark}\label{rem:paradoxical_combinator}
  The reader familiar with the lambda calculus will have noticed the
  similarity between $D$ and the paradoxical combinator.

  [Ed. note: the existence of this seems to suggest we have to be more
  restrictive on the set of processes and names we admit if we are to
  support no-cloning.]
\end{remark}

\subsubsection{Bisimulation}

The computational dynamics gives rise to another kind of equivalence,
the equivalence of computational behavior. As previously mentioned
this is typically captured \emph{via} some form of bisimulation.

% The notion we use in this paper is weak barbed bisimulation
% \cite{milner91polyadicpi}.

The notion we use in this paper is derived from weak barbed
bisimulation \cite{milner91polyadicpi}. 

\begin{definition}
An \emph{observation relation}, $\downarrow_{\mathcal N}$, over a set
of names, $\mathcal N$, is the smallest relation satisfying the rules
below.

\infrule[Out-barb]{y \in {\mathcal N}, \; x \nameeq y}
		  {\outputp{x}{v} \downarrow_{\mathcal N} x}
\infrule[Par-barb]{\mbox{$P\downarrow_{\mathcal N} x$ or $Q\downarrow_{\mathcal N} x$}}
		  {\binpar{P}{Q} \downarrow_{\mathcal N} x}

We write $P \Downarrow_{\mathcal N} x$ if there is $Q$ such that 
$P \wred Q$ and $Q \downarrow_{\mathcal N} x$.
\end{definition}

\begin{definition}
%\label{def.bbisim}
An  ${\mathcal N}$-\emph{barbed bisimulation} over a set of names, ${\mathcal N}$, is a symmetric binary relation 
${\mathcal S}_{\mathcal N}$ between agents such that $P\rel{S}_{\mathcal N}Q$ implies:
\begin{enumerate}
\item If $P \red P'$ then $Q \wred Q'$ and $P'\rel{S}_{\mathcal N} Q'$.
\item If $P\downarrow_{\mathcal N} x$, then $Q\Downarrow_{\mathcal N} x$.
\end{enumerate}
$P$ is ${\mathcal N}$-barbed bisimilar to $Q$, written
$P \wbbisim_{\mathcal N} Q$, if $P \rel{S}_{\mathcal N} Q$ for some ${\mathcal N}$-barbed bisimulation ${\mathcal S}_{\mathcal N}$.
\end{definition}

$\mathcal{R} \subseteq \pi \times \pi$

$P \mathcal{R} Q => \forall P'. P \red P' \Rightarrow \exists Q'. Q \red Q', P' \mathcal{R} Q'$

$P \vdash x \Rightarrow Q \vdash x$

\begin{mathpar}
  \inferrule*[lab=Out-barb]{x \nameeq y}{{y}!\langle{Q}\rangle \vdash x}
  \and
  \inferrule*[lab=Par-barb]{\mbox{$P\vdash x$ or $Q\vdash x$}}{\binpar{P}{Q} \vdash x}
\end{mathpar}

\subsubsection{Contexts}

One of the principle advantages of computational calculi like the
$\pi$-calculus is a well-defined notion of context,
contextual-equivalence and a correlation between
contextual-equivalence and notions of bisimulation. The notion of
context allows the decomposition of a process into (sub-)process and
its syntactic environment, its context. Thus, a context may be
thought of as a process with a ``hole'' (written $\Box$) in it. The
application of a context $M$ to a process $P$, written $M[P]$, is
tantamount to filling the hole in $M$ with $P$. In this paper we do
not need the full weight of this theory, but do make use of the notion
of context in the proof the main theorem. 

\begin{mathpar}
  \inferrule* [lab=summation] {} {{M_{M},M_{N}} \bc \Box \;|\; x.M_{A} \;|\; M_{M}+M_{N}}
  \and
  \inferrule* [lab=agent] {} {{M_{A}} \bc (\vec{x})M_{P} \;| \; \clift{P_0,\ldots,M_{P},\ldots,P_N}}
  \and \\
  \inferrule* [lab=process] {} {{M_{P}} \bc M_{N} \;| \;P|M_{P} }
\end{mathpar} 

\begin{mathpar}
  \inferrule* [lab=sychronization] {} {M_{N} \bc \Box \;|\; x?M_{F} \;|\; x!M_{C}}
  \and
  \inferrule* [lab=abstraction] {} {{M_{F}} \bc (x)M_{P} }
  \and
  \inferrule* [lab=concretion] {} {{M_{C}} \bc \langle M_{P} \rangle }
  \and \\
  \inferrule* [lab=process] {} {{M_{P}} \bc M_{N} \;| \;P|M_{P} }
\end{mathpar}

\begin{definition}[contextual application] Given a context $M$, and
  process $P$, we define the \emph{contextual application}, $M[P] :=
  M\{P/\Box\}$. That is, the contextual application of M to P is the
  substitution of $P$ for $\Box$ in $M$.
\end{definition}

$\meaningof{-} : L \to \mathcal{P}(\pi)$

\begin{mathpar}
  \inferrule* [lab=collection] {} {\meaningof{true} = \pi, \and \meaningof{~E} = \pi \setminus \meaningof{E}, \and \meaningof{E_{1} \& E_{2}} = \meaningof{E_{1}} \cap \meaningof{E_{2}}}
\end{mathpar}

\begin{mathpar}
  \inferrule* [lab=structure] {} {\meaningof{0} = \{ P \in \pi | P \equiv 0 \}, \and \\ \meaningof{E_1 | E_2} = \{ P \in \pi | P \equiv P_{1} | P_{2}, P_{1} \in \meaningof{E_{1}}, P_{2} \in \meaningof{E_2}\} }
\end{mathpar}

\begin{mathpar}
 \inferrule* [lab=behavior] {} {\meaningof{\langle a?b \rangle E} = \{ P \in \pi | P \equiv Q | u?(y)P', \\ \and \\\\ \and \\ \;\;\; u \in \meaningof{a}, \forall z.P'\{z/y\} \in \meaningof{E\{z/b\}}\}, \and \\ \meaningof{a!E} = \{ P \in \pi | P \equiv Q | x!\langle P' \rangle, x \in \meaningof{a} P' \in \meaningof{E}\} }
\end{mathpar}

\begin{mathpar}
 \inferrule* [lab=nominal] {} {\meaningof{\quotep{E}} = \{ \quotep{P} \in \quotep{\pi} | P \in \meaningof{E} \}, \and \meaningof{\quotep{P}} = \{ \quotep{Q} \in \quotep{\pi} | P \equiv Q \} \and \\ \meaningof{@\quotep{E}} = \{ P \in \pi | P \equiv @x, x \in \meaningof{E} \}}
\end{mathpar}

\begin{eqnarray*}
  \\
  \meaningof{-} : TS \to ST
\end{eqnarray*}

\begin{eqnarray*}
  \\
  L : TS \to ST
\end{eqnarray*}

\begin{eqnarray*}
  \\
  P \models E \iff P \in \meaningof{E}
\end{eqnarray*}

\begin{eqnarray*}
  P \approx_{L} Q \iff \forall E \in L. P \models E \iff Q \models E
\end{eqnarray*}

\begin{eqnarray*}
  P \approx_{K} Q
\end{eqnarray*}

\begin{eqnarray*}
  P \approx Q
\end{eqnarray*}

$\approx_{K} = \approx = \approx_{L}$

\subsubsection{Contextual duality}

Note that contexts extend the quotation operation to a family of
operations from processes to names. Given a context, $M$, we can
define a \emph{nominal context}, $\quotep{M}$ by $\quotep{M}[P] :=
\quotep{M[P]}$. To foreshadow what is to come we observe that these
operations enjoy a duality with processes very much like the duality
between vectors and maps from vectors to scalars.

Further, because the calculus is essentially higher-order, we have a
correspondence between contexts and processes. More specifically,
given a name $x$ and a context $M$ we can construct $M^{*}_{x}$ such
that 

\begin{mathpar}
  M^{*}_{x} | \lift{x}{P} \red M[P]
\end{mathpar}

namely,

\begin{mathpar}
  M^{*}_{x} := x?(u).M[\dropn{u}]
\end{mathpar}

The dependence of $M^{*}_{x}$ on a name makes it an abstraction, 

\begin{mathpar}
  M^{*} := (x)x?(u).M[\dropn{u}]
\end{mathpar}

\subsection{Additional notation}

It will sometimes be convenient to denote the process a name
quotes. We already have the notation $x = \quotep{P}$, but it will be
convenient to introduce an alternate notation, $\procn{x}$, when we
want to emphasize the connection to the use of the name. Note that, by
virtue of name equivalence, $\quotep{\procn{x}} \nameeq x$; so, the
notation is consistent with previous definitions.

Further, because names have structure it is possible to effect
substitutions on the basis of that structure. This means we need to
upgrade our notation for substitutions, which we accomplish by
adapting comprehension notation. Thus,

\begin{mathpar}
  P\{ y / x : x \in S \}
\end{mathpar}

is interpreted to mean the process derived from P by replacing (in a
capture-avoiding manner) each occurrence of $x$ in $S$ by $y$. For example,

\begin{mathpar}
  P\{ \quotep{\procn{x}|\procn{x}} / x : x \in \freenames{P} \}
\end{mathpar}

will replace each (occurrence) of a free name $x$ in $P$ by
$\quotep{\procn{x}|\procn{x}}$.

Also, we will avail ourselves of the notation $x^{L}$ and $x^{R}$ to
denote injections of a name into disjoint copies of the name
space. There are numerous ways to accomplish this. One example can be
found in \cite{MeredithR05}. This notation overloads to vectors of
names: $\vec{x}^{\pi} := (x_{i}^{\pi} \; : \; 0 \leq i < |\vec{x}| )$ where $\pi \in \{L,R\}$.

We also use $P^{\Box} := P|\Box$.

In \cite{MeredithR05} an interpretation of the new operator is
given. It turns out that there are several possible interpretations
all enjoying the requisite algebraic properties of the operator (see
\cite{milner91polyadicpi}). We will therefore make liberal use of
$(\nu\; \vec{x})P$.

% subsection the_syntax_and_semantics_of_the_notation_system (end)   

\input{qm2pi.qmops} 

\input{qm2pi.sterngerlach} 

\input{qm2pi.metric} 

% section concurrent_process_calculi (end)

%\input{qm2pi.proofsketch}

% section proof sketch (end)

%\input{qm2pi.slviaknots} 

% section spatial logic via knots (end)

\input{qm2pi.conclusion}

% section conclusion (end)

%\input{qm2pi.dtcodes} 

% section wiring algorithm (end)

\input{qm2pi.ack} 

% section acknowledgments (end)

\newpage


\bibliographystyle{plain}   
\bibliography{../../biblios/main.bib}

\input{qm2pi.rhodetails}

\end{document}

 

% section concurrent_process_calculi (end)

%\documentclass[12pt]{llncs}
%\documentclass{jktr}

\usepackage[pdftex]{hyperref}                   
\usepackage {listings}
\usepackage {mathpartir}
\usepackage{bcprules}
%\usepackage{listings}
                       
\usepackage{graphicx} 
%\usepackage[margins=2.5cm,nohead,nofoot]{geometry}
%\usepackage{geometry}
\usepackage{amsfonts}
\usepackage{amstext}
\usepackage{latexsym}
\usepackage{amssymb}
\usepackage{color}


%\include{myPreamble}
\include{qm2pi.local} 

%\ifpdf
%\usepackage[pdftex]{graphicx}
%\else
%\usepackage{graphicx}
%\fi

 % \ifpdf
%  \usepackage{pdfsync}
%  \if


%\title{Brief Article}
%\author{David F. Snyder}
%\author{L.G. Meredith}

%\address{Dept. of Math., Texas State University--San Marcos, San Marcos, TX 78666}
       
\pagestyle{empty}


\begin{document}

\lstset{language=[Objective]Caml,frame=shadowbox}

\input{qm2pi.front}

% section front matter (end)

\input{qm2pi.intro} 
 
% section introduction (end)

% \input{qm2pi.knotations} 

% section notation (end)

\input{qm2pi.process.calculi} 

% section concurrent_process_calculi_and_spatial_logics_ (end)
    
%\input{qm2pi.knots2pi} 

%\input{qm2pi.trefoil} 

%\input{qm2pi.mainthm} 

% subsection basic_interpretation (end)

%\input{qm2pi.rho.presentation} 
\subsection{The syntax and semantics of the notation system}\label{sub:the_syntax_and_semantics_of_the_notation_system} % (fold)

We now summarize a technical presentation of the calculus that
embodies our theory of dynamics. The typical presentation of such a
calculus follows the style of giving generators and relations on
them. The grammar, below, describing term constructors, freely
generates the set of processes, $\Proc$. This set is then quotiented
by a relation known as structural congruence and it is over this set
that the notion of dynamics is expressed. This presentation is
essentially that of \cite{MeredithR05} with the addition of
polyadicity and summation. For readability we have relegated some of
the technical subtleties to an appendix.

\subsubsection{Process grammar}\label{subsub:process_grammar}

\begin{mathpar}
  \inferrule* [lab=synchronization] {} {{M} \bc \pzero \;|\; x?F \;|\; x!C }
  \and
  \inferrule* [lab=abstraction] {} {{F} \bc (x)P}
  \and
  \inferrule* [lab=concretion] {} {{C} \bc \langle Q \rangle}
  \and
  \inferrule* [lab=process] {} {{P,Q} \bc M \;| \;P|Q \;|\; @{x}}
  \and
  \inferrule* [lab=name] {} {{x} \bc \quotep{P}}
\end{mathpar} 

Note that $\vec{x}$ (resp. $\vec{P}$) denotes a vector of names
(resp. processes) of length $|\vec{x}|$ (resp. $|\vec{P}|$). We adopt
the following useful abbreviations.

\begin{mathpar}
   x?(\vec{y}).P := x.(\vec{y})P \and  x\clift{\vec{P}} := x.\clift{\vec{P}}
   \and x!(y) := \lift{x}{\dropn{y}}
   \and \Pi_{i=0}^{n-1}P_i := P_0 | \ldots | P_{n-1}
\end{mathpar}

\subsubsection{Structural congruence}

\paragraph{Free and bound names and alpha-equivalence.} At the
core of structural equivalence is alpha-equivalence which identifies
process that are the same up to a change of variable. Formally, we
recognize the distinction between free and bound names. The free names
of a process, $\freenames{P}$, may be calculated recursively as
follows:

\begin{mathpar}
\freenames{\pzero} := \emptyset
  \and \\
  \freenames{x?(y).P} := \{ x \} \cup (\freenames{P} \setminus \{ y \})
  \and 
  \freenames{x!\langle P \rangle} := \{ x \} \cup \{ P \} 
  \and \\
  \freenames{P|Q} := \freenames{P} \cup \freenames{Q}
  \and \\
  \freenames{@{x}} := \{ x \}
\end{mathpar}

$\pi$
$\quotep{\pi}$

$\freenames{-} : \pi \to \mathcal{P}(\quotep{\pi})$

\begin{eqnarray*}
  \freenames{\pzero} & := & \emptyset \\
  \freenames{x?(y).P} & := & \{ x \} \cup (\freenames{P} \setminus \{ y \}) \\
  \freenames{x!\langle P \rangle} & := & \{ x \} \cup \{ P \} \\
  \freenames{P|Q} & := & \freenames{P} \cup \freenames{Q} \\
  \freenames{\dropn{x}} & := & \{ x \}
\end{eqnarray*}

The bound names of a process, $\boundnames{P}$, are those names occurring in $P$
that are not free. For example, in $x?(y).0$, the name $x$ is free, while $y$ is bound.

\begin{mathpar}
  \inferrule* [lab=monoidal-laws] {} { P|Q \equiv Q|P \and P|0 \equiv P \and P|(Q|R) \equiv (P|Q)|R }
\end{mathpar}

\begin{mathpar}
  \inferrule* [lab=alpha-equivalence] {} { (x)P \equiv (y)P\{y/x\} \and y \not\in \freenames{P} }
\end{mathpar}

\begin{definition}
Then two processes, $P,Q$, are alpha-equivalent if $P = Q\{\vec{y}/\vec{x}\}$ for
some $\vec{x} \in \boundnames{Q},\vec{y} \in \boundnames{P}$, where $Q\{\vec{y}/\vec{x}\}$
denotes the capture-avoiding substitution of $\vec{y}$ for $\vec{x}$ in $Q$.
\end{definition}

\begin{definition}
  The {\em structural congruence} \cite{SangiorgiWalker} , $\equiv$,
  between processes is the least congruence containing
  alpha-equivalence, satisfying the abelian monoid laws
  (associativity, commutativity and $\pzero$ as identity) for parallel
  composition $|$ and for summation $+$.
\end{definition}

\subsection{Name equivalence}

We take name equivalence, written $\nameeq$, to be the smallest
equivalence relation generated by the following rules.

\begin{mathpar}
\inferrule*[lab=Quote-drop]
{ }
{ \quotep{@{x}} \nameeq x }

\inferrule*[lab=Struct-equiv]
{ P \scong Q }
{ \quotep{P} \nameeq \quotep{Q} }
\end{mathpar}

The astute reader will have noticed that the mutual recursion of names
and processes imposes a mutual recursion on alpha-equivalence and
structural equivalence via name-equivalence. Fortunately, all of this
works out pleasantly and we may calculate in the natural way, free of
concern. The reader interested in the details is referred to the
appendix \ref{appendix:rho_details}.

\subsection{Substitution}

We use $\Proc$ for the set of processes, $\QProc$ for the set of
names, and $\id{\{}\vec{y} / \vec{x} \id{\}}$ to denote partial maps,
$s : \QProc \rightarrow \QProc$. A map, $s$ lifts, uniquely, to a map
on process terms, $\widehat{s} : \Proc \rightarrow \Proc$ by the
following equations.

\begin{mathpar}
  (0) \psubstp{Q}{P} := 0 \\
  (R \juxtap S) \psubstp{Q}{P}
  :=    
  (R)\psubstp{Q}{P} \juxtap (S) \psubstp{Q}{P} \\
  (x?(y).R) \psubstp{Q}{P}    
  :=    
  (x)\substp{Q}{P} (z)\concat( (R \psubstn{z}{y}) \psubstp{Q}{P} ) \\
  (\lift{x}{R}) \psubstp{Q}{P}  
  :=
  \lift{(x)\substp{Q}{P}}{ R \psubstp{Q}{P} } \\
%   (\dropn{x})  \psubstp{Q}{P}       
%   := 
%   \left\{ 
%     \begin{array}{ccc} 
%       \dropn{\quotep{Q}} & & x \nameeq \quotep{P} \\
%       \dropn{x} & & otherwise \\
%     \end{array}
%   \right. 
  (\dropn{x})  \psubstp{Q}{P}       
  := 
  \left\{ 
    \begin{array}{ccc} 
      Q & & x \nameeq \quotep{P} \\
      \dropn{x} & & otherwise \\
    \end{array}
  \right.
\end{mathpar}
 

where

\begin{eqnarray}
  (x)\id{\{} \lpquote Q \rpquote / \lpquote P \rpquote \id{\}}            = 
  \left\{ 
    \begin{array}{ccc}
      \lpquote Q \rpquote & & x \nameeq \lpquote P \rpquote \\
      x & & otherwise \\
    \end{array}
  \right. \nonumber
\end{eqnarray}

and $z$ is chosen distinct from $\quotep{P}$, $\quotep{Q}$, the free
names in $Q$, and all the names in $R$. Our $\alpha$-equivalence will
be built in the standard way from this substitution.

\begin{remark}\label{rem:no_self_referential_names}
  One consequence of these definitions is that $\forall P. \quotep{P}
  \not\in \freenames{P}$.
\end{remark}

\subsection{ Dynamic quote: an example }

Anticipating something of what's to come, consider applying the
substitution, $\widehat{\id{\{}u / z \id{\}}}$, to the following pair
of processes, $\lift{w}{y!(z)}$ and $w[ \lpquote y!(z) \rpquote ]$.

\begin{eqnarray}
	\lift{w}{y!(z)}\widehat{\id{\{}u / z \id{\}}}
		& = &
		\lift{w}{y!(u)} \nonumber\\
	w[ \lpquote y!(z) \rpquote ] \widehat{ \id{\{}u / z \id{\}} }
		& = &
		w[ \lpquote y!(z) \rpquote ] \nonumber
\end{eqnarray}

Because the body of the process between quotes is impervious to
substitution, we get radically different answers. In fact, by
examining the first process in an input context,
e.g. $x?(z).\lift{w}{y!(z)}$, we see that the process under the lift
operator may be shaped by prefixed inputs binding a name inside it. In
this sense, the lift operator will be seen as a way to dynamically
construct processes before reifying them as names.

Finally equipped with these standard features we can present the
dynamics of the calculus.

\subsubsection{Operational semantics} 

Finally, we introduce the computational dynamics. What marks these
algebras as distinct from other more traditionally studied algebraic
structures, e.g. vector spaces or polynomial rings, is the manner in
which dynamics is captured. In traditional structures, dynamics is typically
expressed through morphisms between such structures, as in linear maps
between vector spaces or morphisms between rings. In algebras
associated with the semantics of computation, the dynamics is
expressed as part of the algebraic structure itself, through a
reduction reduction relation typically denoted by $\red$. Below, we
give a recursive presentation of this relation for the calculus used
in the encoding.

$\red \subseteq \pi \times \pi$
$\red : \pi \to \mathcal{P}(\pi)$

\begin{mathpar}
  \inferrule* [lab=Comm] { \textsf{match}( x_{src}, x_{trgt} ) } { x_{trgt}?(y)P \; | \; x_{src}!\langle {Q} \rangle \red P\{\quotep{Q}/y}\} }
  \and \\
  \inferrule* [lab=Par] {{P} \red {P}'} {{{P} | {Q}} \red {{P}' | {Q}}}
  \and
  \inferrule* [lab=Equiv]{{{P} \scong {P}'} \andalso {{P}' \red {Q}'} \andalso {{Q}' \scong {Q}}}{{P} \red {Q}}
\end{mathpar}

\begin{eqnarray*}
  match_{\equiv} (\quotep{P},\quotep{Q}) & := & P \equiv Q \\
  match_{\dagger}(\quotep{P},\quotep{Q}) & := & \forall R. P|Q \red^{*} R => R \red^{*} 0 \\
  match_{K}(\quotep{P},\quotep{Q}) & := & K \mbox{ for some context } K
\end{eqnarray*}

$u?(x)P | u!\langle Q \rangle \red P\{\quotep{Q}/x\}$

%We write $\wred$ for $\red^*$, and $P\red$ if $\exists Q $ such that $ P \red Q$.
We write $P\red$ if $\exists Q $ such that $ P \red Q$ and $P\not\red$, otherwise.

\section{Replication}

As mentioned before, it is known that replication (and hence
recursion) can be implemented in a higher-order process algebra
\cite{SangiorgiWalker}. As our first example of calculation with the
machinery thus far presented we give the construction explicitly in
the {\rhoc}.

\begin{eqnarray}
	D_{x} & := & \prefix{x}{y}{(\binpar{\outputp{x}{y}}{@{y}})} \nonumber\\
	\bangp_{x}{P} & := & \binpar{{x}!\langle{\binpar{D_{x}}{P}}\rangle}{D_{x}} \nonumber
\end{eqnarray}

\begin{eqnarray}
	\bangp_{x}{P} & & \nonumber\\
	=
	& {x}!\langle{(\prefix{x}{y}{(\outputp{x}{y} | @{y})) | P}}\rangle 
	      | \prefix{x}{y}{(\outputp{x}{y} | @{y})} & \nonumber\\
	\red
	& (\outputp{x}{y} | @{y})\substn{\quotep{(\prefix{x}{y}{(@{y} | \outputp{x}{y})) | P}}}{y} & \nonumber\\
	=
	& \outputp{x}{\quotep{(\prefix{x}{y}{(\outputp{x}{y} | @{y})) | P}}}
	  | {(\prefix{x}{y}{(\outputp{x}{y} | @{y})) | P}} & \nonumber\\
	\red
	& \ldots & \nonumber\\
	\red^*
	& P | P | \ldots & \nonumber
\end{eqnarray}

Of course, this encoding, as an implementation, runs away, unfolding
$\bangp{P}$ eagerly. A lazier and more implementable replication
operator, restricted to input-guarded processes, may be obtained as follows.

\begin{eqnarray}
\bangp{\prefix{u}{v}{P}} 
	:= 
	\binpar{\lift{x}{\prefix{u}{v}{(\binpar{D(x)}{P})}}}{D(x)} \nonumber
\end{eqnarray}

\begin{remark}
  Note that the lazier definition still does not deal with summation
  or mixed summation (i.e. sums over input and output). The reader is
  invited to construct definitions of replication that deal with these
  features. 

  Further, the definitions are parameterized in a name, $x$. Can you,
  gentle reader, make a definition that eliminates this parameter and
  guarantees no accidental interaction between the replication
  machinery and the process being replicated -- i.e. no accidental
  sharing of names used by the process to get its work done and the
  name(s) used by the replication to effect copying. This latter
  revision of the definition of replication is crucial to obtaining
  the expected identity $!!P \sim !P$.
\end{remark}

\begin{remark}\label{rem:paradoxical_combinator}
  The reader familiar with the lambda calculus will have noticed the
  similarity between $D$ and the paradoxical combinator.

  [Ed. note: the existence of this seems to suggest we have to be more
  restrictive on the set of processes and names we admit if we are to
  support no-cloning.]
\end{remark}

\subsubsection{Bisimulation}

The computational dynamics gives rise to another kind of equivalence,
the equivalence of computational behavior. As previously mentioned
this is typically captured \emph{via} some form of bisimulation.

% The notion we use in this paper is weak barbed bisimulation
% \cite{milner91polyadicpi}.

The notion we use in this paper is derived from weak barbed
bisimulation \cite{milner91polyadicpi}. 

\begin{definition}
An \emph{observation relation}, $\downarrow_{\mathcal N}$, over a set
of names, $\mathcal N$, is the smallest relation satisfying the rules
below.

\infrule[Out-barb]{y \in {\mathcal N}, \; x \nameeq y}
		  {\outputp{x}{v} \downarrow_{\mathcal N} x}
\infrule[Par-barb]{\mbox{$P\downarrow_{\mathcal N} x$ or $Q\downarrow_{\mathcal N} x$}}
		  {\binpar{P}{Q} \downarrow_{\mathcal N} x}

We write $P \Downarrow_{\mathcal N} x$ if there is $Q$ such that 
$P \wred Q$ and $Q \downarrow_{\mathcal N} x$.
\end{definition}

\begin{definition}
%\label{def.bbisim}
An  ${\mathcal N}$-\emph{barbed bisimulation} over a set of names, ${\mathcal N}$, is a symmetric binary relation 
${\mathcal S}_{\mathcal N}$ between agents such that $P\rel{S}_{\mathcal N}Q$ implies:
\begin{enumerate}
\item If $P \red P'$ then $Q \wred Q'$ and $P'\rel{S}_{\mathcal N} Q'$.
\item If $P\downarrow_{\mathcal N} x$, then $Q\Downarrow_{\mathcal N} x$.
\end{enumerate}
$P$ is ${\mathcal N}$-barbed bisimilar to $Q$, written
$P \wbbisim_{\mathcal N} Q$, if $P \rel{S}_{\mathcal N} Q$ for some ${\mathcal N}$-barbed bisimulation ${\mathcal S}_{\mathcal N}$.
\end{definition}

$\mathcal{R} \subseteq \pi \times \pi$

$P \mathcal{R} Q => \forall P'. P \red P' \Rightarrow \exists Q'. Q \red Q', P' \mathcal{R} Q'$

$P \vdash x \Rightarrow Q \vdash x$

\begin{mathpar}
  \inferrule*[lab=Out-barb]{x \nameeq y}{{y}!\langle{Q}\rangle \vdash x}
  \and
  \inferrule*[lab=Par-barb]{\mbox{$P\vdash x$ or $Q\vdash x$}}{\binpar{P}{Q} \vdash x}
\end{mathpar}

\subsubsection{Contexts}

One of the principle advantages of computational calculi like the
$\pi$-calculus is a well-defined notion of context,
contextual-equivalence and a correlation between
contextual-equivalence and notions of bisimulation. The notion of
context allows the decomposition of a process into (sub-)process and
its syntactic environment, its context. Thus, a context may be
thought of as a process with a ``hole'' (written $\Box$) in it. The
application of a context $M$ to a process $P$, written $M[P]$, is
tantamount to filling the hole in $M$ with $P$. In this paper we do
not need the full weight of this theory, but do make use of the notion
of context in the proof the main theorem. 

\begin{mathpar}
  \inferrule* [lab=summation] {} {{M_{M},M_{N}} \bc \Box \;|\; x.M_{A} \;|\; M_{M}+M_{N}}
  \and
  \inferrule* [lab=agent] {} {{M_{A}} \bc (\vec{x})M_{P} \;| \; \clift{P_0,\ldots,M_{P},\ldots,P_N}}
  \and \\
  \inferrule* [lab=process] {} {{M_{P}} \bc M_{N} \;| \;P|M_{P} }
\end{mathpar} 

\begin{mathpar}
  \inferrule* [lab=sychronization] {} {M_{N} \bc \Box \;|\; x?M_{F} \;|\; x!M_{C}}
  \and
  \inferrule* [lab=abstraction] {} {{M_{F}} \bc (x)M_{P} }
  \and
  \inferrule* [lab=concretion] {} {{M_{C}} \bc \langle M_{P} \rangle }
  \and \\
  \inferrule* [lab=process] {} {{M_{P}} \bc M_{N} \;| \;P|M_{P} }
\end{mathpar}

\begin{definition}[contextual application] Given a context $M$, and
  process $P$, we define the \emph{contextual application}, $M[P] :=
  M\{P/\Box\}$. That is, the contextual application of M to P is the
  substitution of $P$ for $\Box$ in $M$.
\end{definition}

$\meaningof{-} : L \to \mathcal{P}(\pi)$

\begin{mathpar}
  \inferrule* [lab=collection] {} {\meaningof{true} = \pi, \and \meaningof{~E} = \pi \setminus \meaningof{E}, \and \meaningof{E_{1} \& E_{2}} = \meaningof{E_{1}} \cap \meaningof{E_{2}}}
\end{mathpar}

\begin{mathpar}
  \inferrule* [lab=structure] {} {\meaningof{0} = \{ P \in \pi | P \equiv 0 \}, \and \\ \meaningof{E_1 | E_2} = \{ P \in \pi | P \equiv P_{1} | P_{2}, P_{1} \in \meaningof{E_{1}}, P_{2} \in \meaningof{E_2}\} }
\end{mathpar}

\begin{mathpar}
 \inferrule* [lab=behavior] {} {\meaningof{\langle a?b \rangle E} = \{ P \in \pi | P \equiv Q | u?(y)P', \\ \and \\\\ \and \\ \;\;\; u \in \meaningof{a}, \forall z.P'\{z/y\} \in \meaningof{E\{z/b\}}\}, \and \\ \meaningof{a!E} = \{ P \in \pi | P \equiv Q | x!\langle P' \rangle, x \in \meaningof{a} P' \in \meaningof{E}\} }
\end{mathpar}

\begin{mathpar}
 \inferrule* [lab=nominal] {} {\meaningof{\quotep{E}} = \{ \quotep{P} \in \quotep{\pi} | P \in \meaningof{E} \}, \and \meaningof{\quotep{P}} = \{ \quotep{Q} \in \quotep{\pi} | P \equiv Q \} \and \\ \meaningof{@\quotep{E}} = \{ P \in \pi | P \equiv @x, x \in \meaningof{E} \}}
\end{mathpar}

\begin{eqnarray*}
  \\
  \meaningof{-} : TS \to ST
\end{eqnarray*}

\begin{eqnarray*}
  \\
  L : TS \to ST
\end{eqnarray*}

\begin{eqnarray*}
  \\
  P \models E \iff P \in \meaningof{E}
\end{eqnarray*}

\begin{eqnarray*}
  P \approx_{L} Q \iff \forall E \in L. P \models E \iff Q \models E
\end{eqnarray*}

\begin{eqnarray*}
  P \approx_{K} Q
\end{eqnarray*}

\begin{eqnarray*}
  P \approx Q
\end{eqnarray*}

$\approx_{K} = \approx = \approx_{L}$

\subsubsection{Contextual duality}

Note that contexts extend the quotation operation to a family of
operations from processes to names. Given a context, $M$, we can
define a \emph{nominal context}, $\quotep{M}$ by $\quotep{M}[P] :=
\quotep{M[P]}$. To foreshadow what is to come we observe that these
operations enjoy a duality with processes very much like the duality
between vectors and maps from vectors to scalars.

Further, because the calculus is essentially higher-order, we have a
correspondence between contexts and processes. More specifically,
given a name $x$ and a context $M$ we can construct $M^{*}_{x}$ such
that 

\begin{mathpar}
  M^{*}_{x} | \lift{x}{P} \red M[P]
\end{mathpar}

namely,

\begin{mathpar}
  M^{*}_{x} := x?(u).M[\dropn{u}]
\end{mathpar}

The dependence of $M^{*}_{x}$ on a name makes it an abstraction, 

\begin{mathpar}
  M^{*} := (x)x?(u).M[\dropn{u}]
\end{mathpar}

\subsection{Additional notation}

It will sometimes be convenient to denote the process a name
quotes. We already have the notation $x = \quotep{P}$, but it will be
convenient to introduce an alternate notation, $\procn{x}$, when we
want to emphasize the connection to the use of the name. Note that, by
virtue of name equivalence, $\quotep{\procn{x}} \nameeq x$; so, the
notation is consistent with previous definitions.

Further, because names have structure it is possible to effect
substitutions on the basis of that structure. This means we need to
upgrade our notation for substitutions, which we accomplish by
adapting comprehension notation. Thus,

\begin{mathpar}
  P\{ y / x : x \in S \}
\end{mathpar}

is interpreted to mean the process derived from P by replacing (in a
capture-avoiding manner) each occurrence of $x$ in $S$ by $y$. For example,

\begin{mathpar}
  P\{ \quotep{\procn{x}|\procn{x}} / x : x \in \freenames{P} \}
\end{mathpar}

will replace each (occurrence) of a free name $x$ in $P$ by
$\quotep{\procn{x}|\procn{x}}$.

Also, we will avail ourselves of the notation $x^{L}$ and $x^{R}$ to
denote injections of a name into disjoint copies of the name
space. There are numerous ways to accomplish this. One example can be
found in \cite{MeredithR05}. This notation overloads to vectors of
names: $\vec{x}^{\pi} := (x_{i}^{\pi} \; : \; 0 \leq i < |\vec{x}| )$ where $\pi \in \{L,R\}$.

We also use $P^{\Box} := P|\Box$.

In \cite{MeredithR05} an interpretation of the new operator is
given. It turns out that there are several possible interpretations
all enjoying the requisite algebraic properties of the operator (see
\cite{milner91polyadicpi}). We will therefore make liberal use of
$(\nu\; \vec{x})P$.

% subsection the_syntax_and_semantics_of_the_notation_system (end)   

\input{qm2pi.qmops} 

\input{qm2pi.sterngerlach} 

\input{qm2pi.metric} 

% section concurrent_process_calculi (end)

%\input{qm2pi.proofsketch}

% section proof sketch (end)

%\input{qm2pi.slviaknots} 

% section spatial logic via knots (end)

\input{qm2pi.conclusion}

% section conclusion (end)

%\input{qm2pi.dtcodes} 

% section wiring algorithm (end)

\input{qm2pi.ack} 

% section acknowledgments (end)

\newpage


\bibliographystyle{plain}   
\bibliography{../../biblios/main.bib}

\input{qm2pi.rhodetails}

\end{document}



% section proof sketch (end)

%\section{Unlikely characters: spatial logic for
  knots}\label{sub:characteristic_formulae} % (fold)

Associated to the mobile process calculi are a family of logics known
as the Hennessy-Milner logics. These logics typically enjoy a
semantics interpreting formulae as sets of processes that when
factored through the encoding outlined above allows an identification
of classes of knots with logical formulae. In the context of this
encoding the sub-family known as the spatial logics \cite{CairesC03}
\cite{CairesC04} \cite{Caires04} are of particular interest providing
several important features for expressing and reasoning about
properties (i.e. classes) of knots. We hint here at how this may be done.

%\begin{description}
%\item [structural connectives] 
\subsubsection{Structural connectives} The spatial logics enjoy
structural connectives corresponding, at the logical level, to the
parallel composition ($P | Q$) and new name ($(\nu \; x)P$)
connectives for processes. As illustrated in the examples below, these
connectives are extremely expressive given the shape of our encoding.
%\item [decideable satisfaction]

\subsubsection{Decideable satisfaction}
In \cite{Caires04} the satisfaction relation is shown to be decideable
for a rich class of processes. It further turns out that the image of
the our encoding is a proper subset of that class. This result
provides the basis for an algorithm by which to search for knots
enjoying a given property.
%\item [characteristic formulae]

\subsubsection{Characteristic formulae}
In the same paper \cite{Caires04} , Caires presents a means of calculating
characteristic formulae, selecting equivalence classes of processes
up to a pre--specified depth limit on the support set of names. Composed with our
encoding, this characteristic formula can be used to select
characteristic formulae for knots.
%\end{description}

\subsubsection{Spatial logic formulae}

The grammar below (segmented for comprehension) summarizes the syntax
of spatial logic formulae. We employ illustrative examples in the
sequel to provide an intuitive understanding of their meaning
referring the reader to \cite{Caires04} for a more detailed explication
of the semantics.

\begin{mathpar}
  \inferrule* [lab=boolean] {} {{A,B} \bc T \;|\; \neg A \;|\; A \wedge B \;|\; \eta = \eta'}
  \and
  \inferrule* [lab=spatial] {} {|\; \pzero \;|\; A | B \;|\; x \text{\textregistered} A \;|\; \forall x . A \;|\;  H x . A}
  \and
  \inferrule* [lab=behavioral] {} {|\; \alpha . A}
  \and 
  \inferrule* [lab=recursion] {} {|\; X(\vec{u}) \;|\; \mu X(\vec{u}) . A}
  \and
  \inferrule* [lab=action] {} {\alpha \bc \langle x?(\vec{y}) \rangle \;|\; \langle x!(\vec{y}) \rangle \;|\; \langle \tau \rangle}
  \and 
  \inferrule* [lab=name] {} {\eta \bc x \;|\; \tau}
\end{mathpar} 

% subsection characteristic_formulae (end)   	 

\subsection{Example formulae}\label{sub:example_formulae_} % (fold)

\subsubsection{Crossing as formula.}
% 
% \begin{align*}
%   \frac{d}{dx} \sin x &= \cos x 
%   & \frac{d}{dx} e^x &= e^x \\
%   \frac{d}{dx} \cos x &= - \sin x 
%   & \frac{d}{dx} \log x &= \frac{1}{x} \\
% \end{align*} 

\begin{align*}
 \mu C(x_{0},x_{1},y_{0},y_{1},u).&(\langle x_{0}?(z) \rangle(\langle u! \rangle\langle y_{1}!z \rangle C(x_{0},x_{1},y_{0},y_{1},u)) & \\
  & \wedge \langle y_{1}?(z) \rangle (\langle u! \rangle \langle x_{0}!z \rangle C(x_{0},x_{1},y_{0},y_{1},u)) & \\
  & \wedge \langle x_{1}?(z) \rangle (\langle u? \rangle \langle y_{0}!z \rangle C(x_{0},x_{1},y_{0},y_{1},u)) & \\
  & \wedge \langle y_{0}?(z) \rangle (\langle u? \rangle \langle x_{1}!z \rangle C(x_{0},x_{1},y_{0},y_{1},u))) &
\end{align*}

The lexicographical similarity between the shape of this formulae and
the shape of definition of the process representing a crossing reveals
the intuitive meaning of this formulae. It describes the capabilities
of a process that has the right to represent a crossing. For example
it picks out processes that may perform an input on the port $x_0$ in
its initial menu of capabilities. What differentiates the formula
from the process, however, is that the crossing process is the
smallest candidate to satisfy the formula. Infinitely many other
processes -- with internal behavior hidden behind this interface, so
to speak -- also satisfy this formula. Even this simple formula,
then, can be seen to open a new view onto knots, providing a
computational interpretation of \emph{virtual} knots.

Note that this formula is derived by hand. A similar formula can be
derived by employing Caires' calculation of characteristic formula
\cite{Caires04} to the process representing a crossing. In light of
this discussion, we let
$\meaningof{C}_{\phi}(x0,x1,y0,y1,u)$ denote a formula specifying the
dynamics we wish to capture of a crossing. To guarantee we preserve
the shape of the interface and minimal semantics we demand that
$\meaningof{C}_{\phi}(x0,x1,y0,y1,u) \Rightarrow
\textbf{C}(x0,x1,y0,y1,u)$ where $\textbf{C}(x0,x1,y0,y1,u)$ denotes
the formula above.
                            
\subsubsection{Crossing number constraints.}
The moral content of the context lemma (Lemma \ref{context}) is that the notion of
``locality'' in the Reidemeister moves is effectively captured by the
parallel composition operator of the process calculus. This intuition
extends through the logic. Given a formula,
$\meaningof{C}_{\phi}(x0,x1,y0,y1,u)$, we can use the structural
connectives to specify constraints on crossing numbers, such as at
least $n$ crossings, or exactly $n$ crossings.
\begin{mathpar}
  \inferrule* [lab=at-least-n] {} { K^{\geq n}_{\phi}(\vec{xs},\vec{ys}) := \Pi_{i=0}^{n-1} Hu . \meaningof{C}_{\phi}(xs_i,ys_i,u) | T }
  \and 
  \inferrule* [lab=exactly-n] {} { K^{= n}_{\phi}(\vec{xs},\vec{ys}) := \Pi_{i=0}^{n-1} Hu . \meaningof{C}_{\phi}(xs_i,ys_i,u) | \neg (\forall x_0,y_0,x_1,y_1,u . \meaningof{C}_{\phi}(x_0,y_0,x_1,y_1,u) | T) }
\end{mathpar}

To round out this section, recall that the encoding of an $n$-crossing
knot decomposes into a parallel composition of $n$ \emph{copies} of a
crossing process together with a wiring harness. To specify different
knot classes with the same crossing number amounts to specifying
logical constraints on the wiring harness. In the interest of space,
we defer examples to a forthcoming paper. Suffice it to say that both
the conditions ``alternating knot'' and ``contains the tangle
corresponding to 5/3'' are expressible. For example, it is possible to
calculate the characteristic formula of a process corresponding to the
tangle 5/3 and conjoin it into the classifying formula via the
composition connective of the logic.

Finally, we wish to observe that it is entirely within reason to
contemplate a more domain-specific version of spatial logic tailored
to the shape of processes in the image of the encoding. Such a
domain-specific logic would have a better claim to the title formal
language of knot properties.

% subsection example_formulae_ (end)

% section knots_as_processes (end) 

% section spatial logic via knots (end)

\section{Conclusions and future work}

\paragraph{Testing physical space}
You, gentle reader, may wonder why of all the theorems to be proved
given this set up we pick the one above. In some sense it's hardly
central to quantum mechanics. We see it as central in the sense that
it firmly establishes a notion of physical space arising from a notion
of the equivalence of behavior. Relating bisimulation to a metric is a
big step forward, but one is faced with interpreting the relationship
of that metric space to something more physical. Quantum mechanical
notions of ``physical'' space are still far from intuitive, but by
relating this idea of distance as testing to calculations that predict
physical circumstances we are making a not insignificant step forward
toward an understanding of the physical space we inhabit as
essentially dynamic.

\paragraph{Effectivity and simulation}
One of the observations we have yet to make is that the entire program
spelled out here is effective. We have built various interpreters for
the reflective calculus at work in this interpretation. In principle,
then, we can simulate quantum mechanics on a computer. The place where
the simulation may lose fidelity is the infinitely branching summation
for the annihilator.

In this connection i also want to point out that the evaluation style
calculation of the inner product puts the non-determinism of the
summation right at the heart of measurement. This suggests that
Milner's original reduction-based formulation of the dynamics of his
calculi in terms of sums was not just notationally suggestive of a
notion of measure-and-continue but captured some significant part of
the physics.

\paragraph{Quantum continuations}
In light of this last observation i want to point out that the
predominant account of quantum mechanics is missing a key aspect of a
truly compositional story of the physical situation. In a real lab,
when a measurement is made the observation can be made to feed into
another device that then makes another measurement conditioned on the
results of the first. This means that after the superposition was
collapsed the entire experimental set up remained in
superposition. While QM offers a means of writing this down it doesn't
quite line up well with the well-trodden formulation of computation
and continuation that we see so succinctly expressed in Milner's
calculi. This suggests that there might be advantages to this account
of dynamics waiting to be explored.

\paragraph{Quantum logic}
In this connection, we also note that by virtue of having the
Hennessy-Milner construction, we can pull the construction through the
interpretation of QM. This gives us a natural candidate for a quantum
logic that enjoys an extremely tight connection with it's domain of
interpretation, making the construction much less ad hoc (rather it is
the image of functor!).

\paragraph{Quantum probabiity}
i have questions about the basis of the interpretation of inner
product as probability amplitude. In particular, using which
axiomatization of probability theory does the notion of probability
amplitude earn the right to be so dubbed? In other words, where is the
proof that the operation for calculating a probability amplitude (and
then squaring) satisfies the axioms of what it means to calculate a
probability? Even if such a proof exists (i have yet to find it in the
literature), i wonder if it might not be possible to turn things on
their heads. Can we view the calculation of the probability amplitude
as an axiomatization of probability? If so, then the definition we
give for calculating probability amplitude may provide the basis for
an \emph{effective} theory of probability.

\paragraph{Quantum vs ``biological'' information}
Finally, i want to conclude with a more philosophical observation. At
a recent workshop in which QM was a predominant topic i noticed
something about quantum information. The speaker was giving a riveting
discussion of axiomatic QM and showing how properties of ``no
cloning'' and ``no deleting'' emerged as consequences of the
axiomatization. Theorems of this form are necessary to give us a sense
of confidence that our axioms characterize the physical theory. What
struck me, though, was that if quantum information is neither erasable
nor replicable it is markedly different from \emph{life}. Two of the
things we know about life is that

\begin{itemize}
  \item it ends;
  \item to gain some measure of persistence, to transcend it's
    finitude it is imminently copyable.
\end{itemize}

Both of these qualities are summarized succinctly in the aphorism: all
flesh is grass. For me these two kinds of ``information'' -- call them
quantum and biological -- are end points on a spectrum of strategies
for persistence. At one end, we have those curious entities that enjoy
uniqueness and permanence; at the other, we have those who in the face
of a certain end and an uncertain present make a go of passing
something on. To me one of the more remarkable aspects of the latter
strategy is that in the presence of noise (and certain features of
copying) we get a kind of dynamism, a chance for improvement against a
given persistent condition.

% subsection other_calculi_other_bisimulations_and_geometry_as_behavior (end)




% section conclusion (end)

%\documentclass[12pt]{llncs}
%\documentclass{jktr}

\usepackage[pdftex]{hyperref}                   
\usepackage {listings}
\usepackage {mathpartir}
\usepackage{bcprules}
%\usepackage{listings}
                       
\usepackage{graphicx} 
%\usepackage[margins=2.5cm,nohead,nofoot]{geometry}
%\usepackage{geometry}
\usepackage{amsfonts}
\usepackage{amstext}
\usepackage{latexsym}
\usepackage{amssymb}
\usepackage{color}


%\include{myPreamble}
\include{qm2pi.local} 

%\ifpdf
%\usepackage[pdftex]{graphicx}
%\else
%\usepackage{graphicx}
%\fi

 % \ifpdf
%  \usepackage{pdfsync}
%  \if


%\title{Brief Article}
%\author{David F. Snyder}
%\author{L.G. Meredith}

%\address{Dept. of Math., Texas State University--San Marcos, San Marcos, TX 78666}
       
\pagestyle{empty}


\begin{document}

\lstset{language=[Objective]Caml,frame=shadowbox}

\input{qm2pi.front}

% section front matter (end)

\input{qm2pi.intro} 
 
% section introduction (end)

% \input{qm2pi.knotations} 

% section notation (end)

\input{qm2pi.process.calculi} 

% section concurrent_process_calculi_and_spatial_logics_ (end)
    
%\input{qm2pi.knots2pi} 

%\input{qm2pi.trefoil} 

%\input{qm2pi.mainthm} 

% subsection basic_interpretation (end)

%\input{qm2pi.rho.presentation} 
\subsection{The syntax and semantics of the notation system}\label{sub:the_syntax_and_semantics_of_the_notation_system} % (fold)

We now summarize a technical presentation of the calculus that
embodies our theory of dynamics. The typical presentation of such a
calculus follows the style of giving generators and relations on
them. The grammar, below, describing term constructors, freely
generates the set of processes, $\Proc$. This set is then quotiented
by a relation known as structural congruence and it is over this set
that the notion of dynamics is expressed. This presentation is
essentially that of \cite{MeredithR05} with the addition of
polyadicity and summation. For readability we have relegated some of
the technical subtleties to an appendix.

\subsubsection{Process grammar}\label{subsub:process_grammar}

\begin{mathpar}
  \inferrule* [lab=synchronization] {} {{M} \bc \pzero \;|\; x?F \;|\; x!C }
  \and
  \inferrule* [lab=abstraction] {} {{F} \bc (x)P}
  \and
  \inferrule* [lab=concretion] {} {{C} \bc \langle Q \rangle}
  \and
  \inferrule* [lab=process] {} {{P,Q} \bc M \;| \;P|Q \;|\; @{x}}
  \and
  \inferrule* [lab=name] {} {{x} \bc \quotep{P}}
\end{mathpar} 

Note that $\vec{x}$ (resp. $\vec{P}$) denotes a vector of names
(resp. processes) of length $|\vec{x}|$ (resp. $|\vec{P}|$). We adopt
the following useful abbreviations.

\begin{mathpar}
   x?(\vec{y}).P := x.(\vec{y})P \and  x\clift{\vec{P}} := x.\clift{\vec{P}}
   \and x!(y) := \lift{x}{\dropn{y}}
   \and \Pi_{i=0}^{n-1}P_i := P_0 | \ldots | P_{n-1}
\end{mathpar}

\subsubsection{Structural congruence}

\paragraph{Free and bound names and alpha-equivalence.} At the
core of structural equivalence is alpha-equivalence which identifies
process that are the same up to a change of variable. Formally, we
recognize the distinction between free and bound names. The free names
of a process, $\freenames{P}$, may be calculated recursively as
follows:

\begin{mathpar}
\freenames{\pzero} := \emptyset
  \and \\
  \freenames{x?(y).P} := \{ x \} \cup (\freenames{P} \setminus \{ y \})
  \and 
  \freenames{x!\langle P \rangle} := \{ x \} \cup \{ P \} 
  \and \\
  \freenames{P|Q} := \freenames{P} \cup \freenames{Q}
  \and \\
  \freenames{@{x}} := \{ x \}
\end{mathpar}

$\pi$
$\quotep{\pi}$

$\freenames{-} : \pi \to \mathcal{P}(\quotep{\pi})$

\begin{eqnarray*}
  \freenames{\pzero} & := & \emptyset \\
  \freenames{x?(y).P} & := & \{ x \} \cup (\freenames{P} \setminus \{ y \}) \\
  \freenames{x!\langle P \rangle} & := & \{ x \} \cup \{ P \} \\
  \freenames{P|Q} & := & \freenames{P} \cup \freenames{Q} \\
  \freenames{\dropn{x}} & := & \{ x \}
\end{eqnarray*}

The bound names of a process, $\boundnames{P}$, are those names occurring in $P$
that are not free. For example, in $x?(y).0$, the name $x$ is free, while $y$ is bound.

\begin{mathpar}
  \inferrule* [lab=monoidal-laws] {} { P|Q \equiv Q|P \and P|0 \equiv P \and P|(Q|R) \equiv (P|Q)|R }
\end{mathpar}

\begin{mathpar}
  \inferrule* [lab=alpha-equivalence] {} { (x)P \equiv (y)P\{y/x\} \and y \not\in \freenames{P} }
\end{mathpar}

\begin{definition}
Then two processes, $P,Q$, are alpha-equivalent if $P = Q\{\vec{y}/\vec{x}\}$ for
some $\vec{x} \in \boundnames{Q},\vec{y} \in \boundnames{P}$, where $Q\{\vec{y}/\vec{x}\}$
denotes the capture-avoiding substitution of $\vec{y}$ for $\vec{x}$ in $Q$.
\end{definition}

\begin{definition}
  The {\em structural congruence} \cite{SangiorgiWalker} , $\equiv$,
  between processes is the least congruence containing
  alpha-equivalence, satisfying the abelian monoid laws
  (associativity, commutativity and $\pzero$ as identity) for parallel
  composition $|$ and for summation $+$.
\end{definition}

\subsection{Name equivalence}

We take name equivalence, written $\nameeq$, to be the smallest
equivalence relation generated by the following rules.

\begin{mathpar}
\inferrule*[lab=Quote-drop]
{ }
{ \quotep{@{x}} \nameeq x }

\inferrule*[lab=Struct-equiv]
{ P \scong Q }
{ \quotep{P} \nameeq \quotep{Q} }
\end{mathpar}

The astute reader will have noticed that the mutual recursion of names
and processes imposes a mutual recursion on alpha-equivalence and
structural equivalence via name-equivalence. Fortunately, all of this
works out pleasantly and we may calculate in the natural way, free of
concern. The reader interested in the details is referred to the
appendix \ref{appendix:rho_details}.

\subsection{Substitution}

We use $\Proc$ for the set of processes, $\QProc$ for the set of
names, and $\id{\{}\vec{y} / \vec{x} \id{\}}$ to denote partial maps,
$s : \QProc \rightarrow \QProc$. A map, $s$ lifts, uniquely, to a map
on process terms, $\widehat{s} : \Proc \rightarrow \Proc$ by the
following equations.

\begin{mathpar}
  (0) \psubstp{Q}{P} := 0 \\
  (R \juxtap S) \psubstp{Q}{P}
  :=    
  (R)\psubstp{Q}{P} \juxtap (S) \psubstp{Q}{P} \\
  (x?(y).R) \psubstp{Q}{P}    
  :=    
  (x)\substp{Q}{P} (z)\concat( (R \psubstn{z}{y}) \psubstp{Q}{P} ) \\
  (\lift{x}{R}) \psubstp{Q}{P}  
  :=
  \lift{(x)\substp{Q}{P}}{ R \psubstp{Q}{P} } \\
%   (\dropn{x})  \psubstp{Q}{P}       
%   := 
%   \left\{ 
%     \begin{array}{ccc} 
%       \dropn{\quotep{Q}} & & x \nameeq \quotep{P} \\
%       \dropn{x} & & otherwise \\
%     \end{array}
%   \right. 
  (\dropn{x})  \psubstp{Q}{P}       
  := 
  \left\{ 
    \begin{array}{ccc} 
      Q & & x \nameeq \quotep{P} \\
      \dropn{x} & & otherwise \\
    \end{array}
  \right.
\end{mathpar}
 

where

\begin{eqnarray}
  (x)\id{\{} \lpquote Q \rpquote / \lpquote P \rpquote \id{\}}            = 
  \left\{ 
    \begin{array}{ccc}
      \lpquote Q \rpquote & & x \nameeq \lpquote P \rpquote \\
      x & & otherwise \\
    \end{array}
  \right. \nonumber
\end{eqnarray}

and $z$ is chosen distinct from $\quotep{P}$, $\quotep{Q}$, the free
names in $Q$, and all the names in $R$. Our $\alpha$-equivalence will
be built in the standard way from this substitution.

\begin{remark}\label{rem:no_self_referential_names}
  One consequence of these definitions is that $\forall P. \quotep{P}
  \not\in \freenames{P}$.
\end{remark}

\subsection{ Dynamic quote: an example }

Anticipating something of what's to come, consider applying the
substitution, $\widehat{\id{\{}u / z \id{\}}}$, to the following pair
of processes, $\lift{w}{y!(z)}$ and $w[ \lpquote y!(z) \rpquote ]$.

\begin{eqnarray}
	\lift{w}{y!(z)}\widehat{\id{\{}u / z \id{\}}}
		& = &
		\lift{w}{y!(u)} \nonumber\\
	w[ \lpquote y!(z) \rpquote ] \widehat{ \id{\{}u / z \id{\}} }
		& = &
		w[ \lpquote y!(z) \rpquote ] \nonumber
\end{eqnarray}

Because the body of the process between quotes is impervious to
substitution, we get radically different answers. In fact, by
examining the first process in an input context,
e.g. $x?(z).\lift{w}{y!(z)}$, we see that the process under the lift
operator may be shaped by prefixed inputs binding a name inside it. In
this sense, the lift operator will be seen as a way to dynamically
construct processes before reifying them as names.

Finally equipped with these standard features we can present the
dynamics of the calculus.

\subsubsection{Operational semantics} 

Finally, we introduce the computational dynamics. What marks these
algebras as distinct from other more traditionally studied algebraic
structures, e.g. vector spaces or polynomial rings, is the manner in
which dynamics is captured. In traditional structures, dynamics is typically
expressed through morphisms between such structures, as in linear maps
between vector spaces or morphisms between rings. In algebras
associated with the semantics of computation, the dynamics is
expressed as part of the algebraic structure itself, through a
reduction reduction relation typically denoted by $\red$. Below, we
give a recursive presentation of this relation for the calculus used
in the encoding.

$\red \subseteq \pi \times \pi$
$\red : \pi \to \mathcal{P}(\pi)$

\begin{mathpar}
  \inferrule* [lab=Comm] { \textsf{match}( x_{src}, x_{trgt} ) } { x_{trgt}?(y)P \; | \; x_{src}!\langle {Q} \rangle \red P\{\quotep{Q}/y}\} }
  \and \\
  \inferrule* [lab=Par] {{P} \red {P}'} {{{P} | {Q}} \red {{P}' | {Q}}}
  \and
  \inferrule* [lab=Equiv]{{{P} \scong {P}'} \andalso {{P}' \red {Q}'} \andalso {{Q}' \scong {Q}}}{{P} \red {Q}}
\end{mathpar}

\begin{eqnarray*}
  match_{\equiv} (\quotep{P},\quotep{Q}) & := & P \equiv Q \\
  match_{\dagger}(\quotep{P},\quotep{Q}) & := & \forall R. P|Q \red^{*} R => R \red^{*} 0 \\
  match_{K}(\quotep{P},\quotep{Q}) & := & K \mbox{ for some context } K
\end{eqnarray*}

$u?(x)P | u!\langle Q \rangle \red P\{\quotep{Q}/x\}$

%We write $\wred$ for $\red^*$, and $P\red$ if $\exists Q $ such that $ P \red Q$.
We write $P\red$ if $\exists Q $ such that $ P \red Q$ and $P\not\red$, otherwise.

\section{Replication}

As mentioned before, it is known that replication (and hence
recursion) can be implemented in a higher-order process algebra
\cite{SangiorgiWalker}. As our first example of calculation with the
machinery thus far presented we give the construction explicitly in
the {\rhoc}.

\begin{eqnarray}
	D_{x} & := & \prefix{x}{y}{(\binpar{\outputp{x}{y}}{@{y}})} \nonumber\\
	\bangp_{x}{P} & := & \binpar{{x}!\langle{\binpar{D_{x}}{P}}\rangle}{D_{x}} \nonumber
\end{eqnarray}

\begin{eqnarray}
	\bangp_{x}{P} & & \nonumber\\
	=
	& {x}!\langle{(\prefix{x}{y}{(\outputp{x}{y} | @{y})) | P}}\rangle 
	      | \prefix{x}{y}{(\outputp{x}{y} | @{y})} & \nonumber\\
	\red
	& (\outputp{x}{y} | @{y})\substn{\quotep{(\prefix{x}{y}{(@{y} | \outputp{x}{y})) | P}}}{y} & \nonumber\\
	=
	& \outputp{x}{\quotep{(\prefix{x}{y}{(\outputp{x}{y} | @{y})) | P}}}
	  | {(\prefix{x}{y}{(\outputp{x}{y} | @{y})) | P}} & \nonumber\\
	\red
	& \ldots & \nonumber\\
	\red^*
	& P | P | \ldots & \nonumber
\end{eqnarray}

Of course, this encoding, as an implementation, runs away, unfolding
$\bangp{P}$ eagerly. A lazier and more implementable replication
operator, restricted to input-guarded processes, may be obtained as follows.

\begin{eqnarray}
\bangp{\prefix{u}{v}{P}} 
	:= 
	\binpar{\lift{x}{\prefix{u}{v}{(\binpar{D(x)}{P})}}}{D(x)} \nonumber
\end{eqnarray}

\begin{remark}
  Note that the lazier definition still does not deal with summation
  or mixed summation (i.e. sums over input and output). The reader is
  invited to construct definitions of replication that deal with these
  features. 

  Further, the definitions are parameterized in a name, $x$. Can you,
  gentle reader, make a definition that eliminates this parameter and
  guarantees no accidental interaction between the replication
  machinery and the process being replicated -- i.e. no accidental
  sharing of names used by the process to get its work done and the
  name(s) used by the replication to effect copying. This latter
  revision of the definition of replication is crucial to obtaining
  the expected identity $!!P \sim !P$.
\end{remark}

\begin{remark}\label{rem:paradoxical_combinator}
  The reader familiar with the lambda calculus will have noticed the
  similarity between $D$ and the paradoxical combinator.

  [Ed. note: the existence of this seems to suggest we have to be more
  restrictive on the set of processes and names we admit if we are to
  support no-cloning.]
\end{remark}

\subsubsection{Bisimulation}

The computational dynamics gives rise to another kind of equivalence,
the equivalence of computational behavior. As previously mentioned
this is typically captured \emph{via} some form of bisimulation.

% The notion we use in this paper is weak barbed bisimulation
% \cite{milner91polyadicpi}.

The notion we use in this paper is derived from weak barbed
bisimulation \cite{milner91polyadicpi}. 

\begin{definition}
An \emph{observation relation}, $\downarrow_{\mathcal N}$, over a set
of names, $\mathcal N$, is the smallest relation satisfying the rules
below.

\infrule[Out-barb]{y \in {\mathcal N}, \; x \nameeq y}
		  {\outputp{x}{v} \downarrow_{\mathcal N} x}
\infrule[Par-barb]{\mbox{$P\downarrow_{\mathcal N} x$ or $Q\downarrow_{\mathcal N} x$}}
		  {\binpar{P}{Q} \downarrow_{\mathcal N} x}

We write $P \Downarrow_{\mathcal N} x$ if there is $Q$ such that 
$P \wred Q$ and $Q \downarrow_{\mathcal N} x$.
\end{definition}

\begin{definition}
%\label{def.bbisim}
An  ${\mathcal N}$-\emph{barbed bisimulation} over a set of names, ${\mathcal N}$, is a symmetric binary relation 
${\mathcal S}_{\mathcal N}$ between agents such that $P\rel{S}_{\mathcal N}Q$ implies:
\begin{enumerate}
\item If $P \red P'$ then $Q \wred Q'$ and $P'\rel{S}_{\mathcal N} Q'$.
\item If $P\downarrow_{\mathcal N} x$, then $Q\Downarrow_{\mathcal N} x$.
\end{enumerate}
$P$ is ${\mathcal N}$-barbed bisimilar to $Q$, written
$P \wbbisim_{\mathcal N} Q$, if $P \rel{S}_{\mathcal N} Q$ for some ${\mathcal N}$-barbed bisimulation ${\mathcal S}_{\mathcal N}$.
\end{definition}

$\mathcal{R} \subseteq \pi \times \pi$

$P \mathcal{R} Q => \forall P'. P \red P' \Rightarrow \exists Q'. Q \red Q', P' \mathcal{R} Q'$

$P \vdash x \Rightarrow Q \vdash x$

\begin{mathpar}
  \inferrule*[lab=Out-barb]{x \nameeq y}{{y}!\langle{Q}\rangle \vdash x}
  \and
  \inferrule*[lab=Par-barb]{\mbox{$P\vdash x$ or $Q\vdash x$}}{\binpar{P}{Q} \vdash x}
\end{mathpar}

\subsubsection{Contexts}

One of the principle advantages of computational calculi like the
$\pi$-calculus is a well-defined notion of context,
contextual-equivalence and a correlation between
contextual-equivalence and notions of bisimulation. The notion of
context allows the decomposition of a process into (sub-)process and
its syntactic environment, its context. Thus, a context may be
thought of as a process with a ``hole'' (written $\Box$) in it. The
application of a context $M$ to a process $P$, written $M[P]$, is
tantamount to filling the hole in $M$ with $P$. In this paper we do
not need the full weight of this theory, but do make use of the notion
of context in the proof the main theorem. 

\begin{mathpar}
  \inferrule* [lab=summation] {} {{M_{M},M_{N}} \bc \Box \;|\; x.M_{A} \;|\; M_{M}+M_{N}}
  \and
  \inferrule* [lab=agent] {} {{M_{A}} \bc (\vec{x})M_{P} \;| \; \clift{P_0,\ldots,M_{P},\ldots,P_N}}
  \and \\
  \inferrule* [lab=process] {} {{M_{P}} \bc M_{N} \;| \;P|M_{P} }
\end{mathpar} 

\begin{mathpar}
  \inferrule* [lab=sychronization] {} {M_{N} \bc \Box \;|\; x?M_{F} \;|\; x!M_{C}}
  \and
  \inferrule* [lab=abstraction] {} {{M_{F}} \bc (x)M_{P} }
  \and
  \inferrule* [lab=concretion] {} {{M_{C}} \bc \langle M_{P} \rangle }
  \and \\
  \inferrule* [lab=process] {} {{M_{P}} \bc M_{N} \;| \;P|M_{P} }
\end{mathpar}

\begin{definition}[contextual application] Given a context $M$, and
  process $P$, we define the \emph{contextual application}, $M[P] :=
  M\{P/\Box\}$. That is, the contextual application of M to P is the
  substitution of $P$ for $\Box$ in $M$.
\end{definition}

$\meaningof{-} : L \to \mathcal{P}(\pi)$

\begin{mathpar}
  \inferrule* [lab=collection] {} {\meaningof{true} = \pi, \and \meaningof{~E} = \pi \setminus \meaningof{E}, \and \meaningof{E_{1} \& E_{2}} = \meaningof{E_{1}} \cap \meaningof{E_{2}}}
\end{mathpar}

\begin{mathpar}
  \inferrule* [lab=structure] {} {\meaningof{0} = \{ P \in \pi | P \equiv 0 \}, \and \\ \meaningof{E_1 | E_2} = \{ P \in \pi | P \equiv P_{1} | P_{2}, P_{1} \in \meaningof{E_{1}}, P_{2} \in \meaningof{E_2}\} }
\end{mathpar}

\begin{mathpar}
 \inferrule* [lab=behavior] {} {\meaningof{\langle a?b \rangle E} = \{ P \in \pi | P \equiv Q | u?(y)P', \\ \and \\\\ \and \\ \;\;\; u \in \meaningof{a}, \forall z.P'\{z/y\} \in \meaningof{E\{z/b\}}\}, \and \\ \meaningof{a!E} = \{ P \in \pi | P \equiv Q | x!\langle P' \rangle, x \in \meaningof{a} P' \in \meaningof{E}\} }
\end{mathpar}

\begin{mathpar}
 \inferrule* [lab=nominal] {} {\meaningof{\quotep{E}} = \{ \quotep{P} \in \quotep{\pi} | P \in \meaningof{E} \}, \and \meaningof{\quotep{P}} = \{ \quotep{Q} \in \quotep{\pi} | P \equiv Q \} \and \\ \meaningof{@\quotep{E}} = \{ P \in \pi | P \equiv @x, x \in \meaningof{E} \}}
\end{mathpar}

\begin{eqnarray*}
  \\
  \meaningof{-} : TS \to ST
\end{eqnarray*}

\begin{eqnarray*}
  \\
  L : TS \to ST
\end{eqnarray*}

\begin{eqnarray*}
  \\
  P \models E \iff P \in \meaningof{E}
\end{eqnarray*}

\begin{eqnarray*}
  P \approx_{L} Q \iff \forall E \in L. P \models E \iff Q \models E
\end{eqnarray*}

\begin{eqnarray*}
  P \approx_{K} Q
\end{eqnarray*}

\begin{eqnarray*}
  P \approx Q
\end{eqnarray*}

$\approx_{K} = \approx = \approx_{L}$

\subsubsection{Contextual duality}

Note that contexts extend the quotation operation to a family of
operations from processes to names. Given a context, $M$, we can
define a \emph{nominal context}, $\quotep{M}$ by $\quotep{M}[P] :=
\quotep{M[P]}$. To foreshadow what is to come we observe that these
operations enjoy a duality with processes very much like the duality
between vectors and maps from vectors to scalars.

Further, because the calculus is essentially higher-order, we have a
correspondence between contexts and processes. More specifically,
given a name $x$ and a context $M$ we can construct $M^{*}_{x}$ such
that 

\begin{mathpar}
  M^{*}_{x} | \lift{x}{P} \red M[P]
\end{mathpar}

namely,

\begin{mathpar}
  M^{*}_{x} := x?(u).M[\dropn{u}]
\end{mathpar}

The dependence of $M^{*}_{x}$ on a name makes it an abstraction, 

\begin{mathpar}
  M^{*} := (x)x?(u).M[\dropn{u}]
\end{mathpar}

\subsection{Additional notation}

It will sometimes be convenient to denote the process a name
quotes. We already have the notation $x = \quotep{P}$, but it will be
convenient to introduce an alternate notation, $\procn{x}$, when we
want to emphasize the connection to the use of the name. Note that, by
virtue of name equivalence, $\quotep{\procn{x}} \nameeq x$; so, the
notation is consistent with previous definitions.

Further, because names have structure it is possible to effect
substitutions on the basis of that structure. This means we need to
upgrade our notation for substitutions, which we accomplish by
adapting comprehension notation. Thus,

\begin{mathpar}
  P\{ y / x : x \in S \}
\end{mathpar}

is interpreted to mean the process derived from P by replacing (in a
capture-avoiding manner) each occurrence of $x$ in $S$ by $y$. For example,

\begin{mathpar}
  P\{ \quotep{\procn{x}|\procn{x}} / x : x \in \freenames{P} \}
\end{mathpar}

will replace each (occurrence) of a free name $x$ in $P$ by
$\quotep{\procn{x}|\procn{x}}$.

Also, we will avail ourselves of the notation $x^{L}$ and $x^{R}$ to
denote injections of a name into disjoint copies of the name
space. There are numerous ways to accomplish this. One example can be
found in \cite{MeredithR05}. This notation overloads to vectors of
names: $\vec{x}^{\pi} := (x_{i}^{\pi} \; : \; 0 \leq i < |\vec{x}| )$ where $\pi \in \{L,R\}$.

We also use $P^{\Box} := P|\Box$.

In \cite{MeredithR05} an interpretation of the new operator is
given. It turns out that there are several possible interpretations
all enjoying the requisite algebraic properties of the operator (see
\cite{milner91polyadicpi}). We will therefore make liberal use of
$(\nu\; \vec{x})P$.

% subsection the_syntax_and_semantics_of_the_notation_system (end)   

\input{qm2pi.qmops} 

\input{qm2pi.sterngerlach} 

\input{qm2pi.metric} 

% section concurrent_process_calculi (end)

%\input{qm2pi.proofsketch}

% section proof sketch (end)

%\input{qm2pi.slviaknots} 

% section spatial logic via knots (end)

\input{qm2pi.conclusion}

% section conclusion (end)

%\input{qm2pi.dtcodes} 

% section wiring algorithm (end)

\input{qm2pi.ack} 

% section acknowledgments (end)

\newpage


\bibliographystyle{plain}   
\bibliography{../../biblios/main.bib}

\input{qm2pi.rhodetails}

\end{document}

 

% section wiring algorithm (end)

\documentclass[12pt]{llncs}
%\documentclass{jktr}

\usepackage[pdftex]{hyperref}                   
\usepackage {listings}
\usepackage {mathpartir}
\usepackage{bcprules}
%\usepackage{listings}
                       
\usepackage{graphicx} 
%\usepackage[margins=2.5cm,nohead,nofoot]{geometry}
%\usepackage{geometry}
\usepackage{amsfonts}
\usepackage{amstext}
\usepackage{latexsym}
\usepackage{amssymb}
\usepackage{color}


%\include{myPreamble}
\include{qm2pi.local} 

%\ifpdf
%\usepackage[pdftex]{graphicx}
%\else
%\usepackage{graphicx}
%\fi

 % \ifpdf
%  \usepackage{pdfsync}
%  \if


%\title{Brief Article}
%\author{David F. Snyder}
%\author{L.G. Meredith}

%\address{Dept. of Math., Texas State University--San Marcos, San Marcos, TX 78666}
       
\pagestyle{empty}


\begin{document}

\lstset{language=[Objective]Caml,frame=shadowbox}

\input{qm2pi.front}

% section front matter (end)

\input{qm2pi.intro} 
 
% section introduction (end)

% \input{qm2pi.knotations} 

% section notation (end)

\input{qm2pi.process.calculi} 

% section concurrent_process_calculi_and_spatial_logics_ (end)
    
%\input{qm2pi.knots2pi} 

%\input{qm2pi.trefoil} 

%\input{qm2pi.mainthm} 

% subsection basic_interpretation (end)

%\input{qm2pi.rho.presentation} 
\subsection{The syntax and semantics of the notation system}\label{sub:the_syntax_and_semantics_of_the_notation_system} % (fold)

We now summarize a technical presentation of the calculus that
embodies our theory of dynamics. The typical presentation of such a
calculus follows the style of giving generators and relations on
them. The grammar, below, describing term constructors, freely
generates the set of processes, $\Proc$. This set is then quotiented
by a relation known as structural congruence and it is over this set
that the notion of dynamics is expressed. This presentation is
essentially that of \cite{MeredithR05} with the addition of
polyadicity and summation. For readability we have relegated some of
the technical subtleties to an appendix.

\subsubsection{Process grammar}\label{subsub:process_grammar}

\begin{mathpar}
  \inferrule* [lab=synchronization] {} {{M} \bc \pzero \;|\; x?F \;|\; x!C }
  \and
  \inferrule* [lab=abstraction] {} {{F} \bc (x)P}
  \and
  \inferrule* [lab=concretion] {} {{C} \bc \langle Q \rangle}
  \and
  \inferrule* [lab=process] {} {{P,Q} \bc M \;| \;P|Q \;|\; @{x}}
  \and
  \inferrule* [lab=name] {} {{x} \bc \quotep{P}}
\end{mathpar} 

Note that $\vec{x}$ (resp. $\vec{P}$) denotes a vector of names
(resp. processes) of length $|\vec{x}|$ (resp. $|\vec{P}|$). We adopt
the following useful abbreviations.

\begin{mathpar}
   x?(\vec{y}).P := x.(\vec{y})P \and  x\clift{\vec{P}} := x.\clift{\vec{P}}
   \and x!(y) := \lift{x}{\dropn{y}}
   \and \Pi_{i=0}^{n-1}P_i := P_0 | \ldots | P_{n-1}
\end{mathpar}

\subsubsection{Structural congruence}

\paragraph{Free and bound names and alpha-equivalence.} At the
core of structural equivalence is alpha-equivalence which identifies
process that are the same up to a change of variable. Formally, we
recognize the distinction between free and bound names. The free names
of a process, $\freenames{P}$, may be calculated recursively as
follows:

\begin{mathpar}
\freenames{\pzero} := \emptyset
  \and \\
  \freenames{x?(y).P} := \{ x \} \cup (\freenames{P} \setminus \{ y \})
  \and 
  \freenames{x!\langle P \rangle} := \{ x \} \cup \{ P \} 
  \and \\
  \freenames{P|Q} := \freenames{P} \cup \freenames{Q}
  \and \\
  \freenames{@{x}} := \{ x \}
\end{mathpar}

$\pi$
$\quotep{\pi}$

$\freenames{-} : \pi \to \mathcal{P}(\quotep{\pi})$

\begin{eqnarray*}
  \freenames{\pzero} & := & \emptyset \\
  \freenames{x?(y).P} & := & \{ x \} \cup (\freenames{P} \setminus \{ y \}) \\
  \freenames{x!\langle P \rangle} & := & \{ x \} \cup \{ P \} \\
  \freenames{P|Q} & := & \freenames{P} \cup \freenames{Q} \\
  \freenames{\dropn{x}} & := & \{ x \}
\end{eqnarray*}

The bound names of a process, $\boundnames{P}$, are those names occurring in $P$
that are not free. For example, in $x?(y).0$, the name $x$ is free, while $y$ is bound.

\begin{mathpar}
  \inferrule* [lab=monoidal-laws] {} { P|Q \equiv Q|P \and P|0 \equiv P \and P|(Q|R) \equiv (P|Q)|R }
\end{mathpar}

\begin{mathpar}
  \inferrule* [lab=alpha-equivalence] {} { (x)P \equiv (y)P\{y/x\} \and y \not\in \freenames{P} }
\end{mathpar}

\begin{definition}
Then two processes, $P,Q$, are alpha-equivalent if $P = Q\{\vec{y}/\vec{x}\}$ for
some $\vec{x} \in \boundnames{Q},\vec{y} \in \boundnames{P}$, where $Q\{\vec{y}/\vec{x}\}$
denotes the capture-avoiding substitution of $\vec{y}$ for $\vec{x}$ in $Q$.
\end{definition}

\begin{definition}
  The {\em structural congruence} \cite{SangiorgiWalker} , $\equiv$,
  between processes is the least congruence containing
  alpha-equivalence, satisfying the abelian monoid laws
  (associativity, commutativity and $\pzero$ as identity) for parallel
  composition $|$ and for summation $+$.
\end{definition}

\subsection{Name equivalence}

We take name equivalence, written $\nameeq$, to be the smallest
equivalence relation generated by the following rules.

\begin{mathpar}
\inferrule*[lab=Quote-drop]
{ }
{ \quotep{@{x}} \nameeq x }

\inferrule*[lab=Struct-equiv]
{ P \scong Q }
{ \quotep{P} \nameeq \quotep{Q} }
\end{mathpar}

The astute reader will have noticed that the mutual recursion of names
and processes imposes a mutual recursion on alpha-equivalence and
structural equivalence via name-equivalence. Fortunately, all of this
works out pleasantly and we may calculate in the natural way, free of
concern. The reader interested in the details is referred to the
appendix \ref{appendix:rho_details}.

\subsection{Substitution}

We use $\Proc$ for the set of processes, $\QProc$ for the set of
names, and $\id{\{}\vec{y} / \vec{x} \id{\}}$ to denote partial maps,
$s : \QProc \rightarrow \QProc$. A map, $s$ lifts, uniquely, to a map
on process terms, $\widehat{s} : \Proc \rightarrow \Proc$ by the
following equations.

\begin{mathpar}
  (0) \psubstp{Q}{P} := 0 \\
  (R \juxtap S) \psubstp{Q}{P}
  :=    
  (R)\psubstp{Q}{P} \juxtap (S) \psubstp{Q}{P} \\
  (x?(y).R) \psubstp{Q}{P}    
  :=    
  (x)\substp{Q}{P} (z)\concat( (R \psubstn{z}{y}) \psubstp{Q}{P} ) \\
  (\lift{x}{R}) \psubstp{Q}{P}  
  :=
  \lift{(x)\substp{Q}{P}}{ R \psubstp{Q}{P} } \\
%   (\dropn{x})  \psubstp{Q}{P}       
%   := 
%   \left\{ 
%     \begin{array}{ccc} 
%       \dropn{\quotep{Q}} & & x \nameeq \quotep{P} \\
%       \dropn{x} & & otherwise \\
%     \end{array}
%   \right. 
  (\dropn{x})  \psubstp{Q}{P}       
  := 
  \left\{ 
    \begin{array}{ccc} 
      Q & & x \nameeq \quotep{P} \\
      \dropn{x} & & otherwise \\
    \end{array}
  \right.
\end{mathpar}
 

where

\begin{eqnarray}
  (x)\id{\{} \lpquote Q \rpquote / \lpquote P \rpquote \id{\}}            = 
  \left\{ 
    \begin{array}{ccc}
      \lpquote Q \rpquote & & x \nameeq \lpquote P \rpquote \\
      x & & otherwise \\
    \end{array}
  \right. \nonumber
\end{eqnarray}

and $z$ is chosen distinct from $\quotep{P}$, $\quotep{Q}$, the free
names in $Q$, and all the names in $R$. Our $\alpha$-equivalence will
be built in the standard way from this substitution.

\begin{remark}\label{rem:no_self_referential_names}
  One consequence of these definitions is that $\forall P. \quotep{P}
  \not\in \freenames{P}$.
\end{remark}

\subsection{ Dynamic quote: an example }

Anticipating something of what's to come, consider applying the
substitution, $\widehat{\id{\{}u / z \id{\}}}$, to the following pair
of processes, $\lift{w}{y!(z)}$ and $w[ \lpquote y!(z) \rpquote ]$.

\begin{eqnarray}
	\lift{w}{y!(z)}\widehat{\id{\{}u / z \id{\}}}
		& = &
		\lift{w}{y!(u)} \nonumber\\
	w[ \lpquote y!(z) \rpquote ] \widehat{ \id{\{}u / z \id{\}} }
		& = &
		w[ \lpquote y!(z) \rpquote ] \nonumber
\end{eqnarray}

Because the body of the process between quotes is impervious to
substitution, we get radically different answers. In fact, by
examining the first process in an input context,
e.g. $x?(z).\lift{w}{y!(z)}$, we see that the process under the lift
operator may be shaped by prefixed inputs binding a name inside it. In
this sense, the lift operator will be seen as a way to dynamically
construct processes before reifying them as names.

Finally equipped with these standard features we can present the
dynamics of the calculus.

\subsubsection{Operational semantics} 

Finally, we introduce the computational dynamics. What marks these
algebras as distinct from other more traditionally studied algebraic
structures, e.g. vector spaces or polynomial rings, is the manner in
which dynamics is captured. In traditional structures, dynamics is typically
expressed through morphisms between such structures, as in linear maps
between vector spaces or morphisms between rings. In algebras
associated with the semantics of computation, the dynamics is
expressed as part of the algebraic structure itself, through a
reduction reduction relation typically denoted by $\red$. Below, we
give a recursive presentation of this relation for the calculus used
in the encoding.

$\red \subseteq \pi \times \pi$
$\red : \pi \to \mathcal{P}(\pi)$

\begin{mathpar}
  \inferrule* [lab=Comm] { \textsf{match}( x_{src}, x_{trgt} ) } { x_{trgt}?(y)P \; | \; x_{src}!\langle {Q} \rangle \red P\{\quotep{Q}/y}\} }
  \and \\
  \inferrule* [lab=Par] {{P} \red {P}'} {{{P} | {Q}} \red {{P}' | {Q}}}
  \and
  \inferrule* [lab=Equiv]{{{P} \scong {P}'} \andalso {{P}' \red {Q}'} \andalso {{Q}' \scong {Q}}}{{P} \red {Q}}
\end{mathpar}

\begin{eqnarray*}
  match_{\equiv} (\quotep{P},\quotep{Q}) & := & P \equiv Q \\
  match_{\dagger}(\quotep{P},\quotep{Q}) & := & \forall R. P|Q \red^{*} R => R \red^{*} 0 \\
  match_{K}(\quotep{P},\quotep{Q}) & := & K \mbox{ for some context } K
\end{eqnarray*}

$u?(x)P | u!\langle Q \rangle \red P\{\quotep{Q}/x\}$

%We write $\wred$ for $\red^*$, and $P\red$ if $\exists Q $ such that $ P \red Q$.
We write $P\red$ if $\exists Q $ such that $ P \red Q$ and $P\not\red$, otherwise.

\section{Replication}

As mentioned before, it is known that replication (and hence
recursion) can be implemented in a higher-order process algebra
\cite{SangiorgiWalker}. As our first example of calculation with the
machinery thus far presented we give the construction explicitly in
the {\rhoc}.

\begin{eqnarray}
	D_{x} & := & \prefix{x}{y}{(\binpar{\outputp{x}{y}}{@{y}})} \nonumber\\
	\bangp_{x}{P} & := & \binpar{{x}!\langle{\binpar{D_{x}}{P}}\rangle}{D_{x}} \nonumber
\end{eqnarray}

\begin{eqnarray}
	\bangp_{x}{P} & & \nonumber\\
	=
	& {x}!\langle{(\prefix{x}{y}{(\outputp{x}{y} | @{y})) | P}}\rangle 
	      | \prefix{x}{y}{(\outputp{x}{y} | @{y})} & \nonumber\\
	\red
	& (\outputp{x}{y} | @{y})\substn{\quotep{(\prefix{x}{y}{(@{y} | \outputp{x}{y})) | P}}}{y} & \nonumber\\
	=
	& \outputp{x}{\quotep{(\prefix{x}{y}{(\outputp{x}{y} | @{y})) | P}}}
	  | {(\prefix{x}{y}{(\outputp{x}{y} | @{y})) | P}} & \nonumber\\
	\red
	& \ldots & \nonumber\\
	\red^*
	& P | P | \ldots & \nonumber
\end{eqnarray}

Of course, this encoding, as an implementation, runs away, unfolding
$\bangp{P}$ eagerly. A lazier and more implementable replication
operator, restricted to input-guarded processes, may be obtained as follows.

\begin{eqnarray}
\bangp{\prefix{u}{v}{P}} 
	:= 
	\binpar{\lift{x}{\prefix{u}{v}{(\binpar{D(x)}{P})}}}{D(x)} \nonumber
\end{eqnarray}

\begin{remark}
  Note that the lazier definition still does not deal with summation
  or mixed summation (i.e. sums over input and output). The reader is
  invited to construct definitions of replication that deal with these
  features. 

  Further, the definitions are parameterized in a name, $x$. Can you,
  gentle reader, make a definition that eliminates this parameter and
  guarantees no accidental interaction between the replication
  machinery and the process being replicated -- i.e. no accidental
  sharing of names used by the process to get its work done and the
  name(s) used by the replication to effect copying. This latter
  revision of the definition of replication is crucial to obtaining
  the expected identity $!!P \sim !P$.
\end{remark}

\begin{remark}\label{rem:paradoxical_combinator}
  The reader familiar with the lambda calculus will have noticed the
  similarity between $D$ and the paradoxical combinator.

  [Ed. note: the existence of this seems to suggest we have to be more
  restrictive on the set of processes and names we admit if we are to
  support no-cloning.]
\end{remark}

\subsubsection{Bisimulation}

The computational dynamics gives rise to another kind of equivalence,
the equivalence of computational behavior. As previously mentioned
this is typically captured \emph{via} some form of bisimulation.

% The notion we use in this paper is weak barbed bisimulation
% \cite{milner91polyadicpi}.

The notion we use in this paper is derived from weak barbed
bisimulation \cite{milner91polyadicpi}. 

\begin{definition}
An \emph{observation relation}, $\downarrow_{\mathcal N}$, over a set
of names, $\mathcal N$, is the smallest relation satisfying the rules
below.

\infrule[Out-barb]{y \in {\mathcal N}, \; x \nameeq y}
		  {\outputp{x}{v} \downarrow_{\mathcal N} x}
\infrule[Par-barb]{\mbox{$P\downarrow_{\mathcal N} x$ or $Q\downarrow_{\mathcal N} x$}}
		  {\binpar{P}{Q} \downarrow_{\mathcal N} x}

We write $P \Downarrow_{\mathcal N} x$ if there is $Q$ such that 
$P \wred Q$ and $Q \downarrow_{\mathcal N} x$.
\end{definition}

\begin{definition}
%\label{def.bbisim}
An  ${\mathcal N}$-\emph{barbed bisimulation} over a set of names, ${\mathcal N}$, is a symmetric binary relation 
${\mathcal S}_{\mathcal N}$ between agents such that $P\rel{S}_{\mathcal N}Q$ implies:
\begin{enumerate}
\item If $P \red P'$ then $Q \wred Q'$ and $P'\rel{S}_{\mathcal N} Q'$.
\item If $P\downarrow_{\mathcal N} x$, then $Q\Downarrow_{\mathcal N} x$.
\end{enumerate}
$P$ is ${\mathcal N}$-barbed bisimilar to $Q$, written
$P \wbbisim_{\mathcal N} Q$, if $P \rel{S}_{\mathcal N} Q$ for some ${\mathcal N}$-barbed bisimulation ${\mathcal S}_{\mathcal N}$.
\end{definition}

$\mathcal{R} \subseteq \pi \times \pi$

$P \mathcal{R} Q => \forall P'. P \red P' \Rightarrow \exists Q'. Q \red Q', P' \mathcal{R} Q'$

$P \vdash x \Rightarrow Q \vdash x$

\begin{mathpar}
  \inferrule*[lab=Out-barb]{x \nameeq y}{{y}!\langle{Q}\rangle \vdash x}
  \and
  \inferrule*[lab=Par-barb]{\mbox{$P\vdash x$ or $Q\vdash x$}}{\binpar{P}{Q} \vdash x}
\end{mathpar}

\subsubsection{Contexts}

One of the principle advantages of computational calculi like the
$\pi$-calculus is a well-defined notion of context,
contextual-equivalence and a correlation between
contextual-equivalence and notions of bisimulation. The notion of
context allows the decomposition of a process into (sub-)process and
its syntactic environment, its context. Thus, a context may be
thought of as a process with a ``hole'' (written $\Box$) in it. The
application of a context $M$ to a process $P$, written $M[P]$, is
tantamount to filling the hole in $M$ with $P$. In this paper we do
not need the full weight of this theory, but do make use of the notion
of context in the proof the main theorem. 

\begin{mathpar}
  \inferrule* [lab=summation] {} {{M_{M},M_{N}} \bc \Box \;|\; x.M_{A} \;|\; M_{M}+M_{N}}
  \and
  \inferrule* [lab=agent] {} {{M_{A}} \bc (\vec{x})M_{P} \;| \; \clift{P_0,\ldots,M_{P},\ldots,P_N}}
  \and \\
  \inferrule* [lab=process] {} {{M_{P}} \bc M_{N} \;| \;P|M_{P} }
\end{mathpar} 

\begin{mathpar}
  \inferrule* [lab=sychronization] {} {M_{N} \bc \Box \;|\; x?M_{F} \;|\; x!M_{C}}
  \and
  \inferrule* [lab=abstraction] {} {{M_{F}} \bc (x)M_{P} }
  \and
  \inferrule* [lab=concretion] {} {{M_{C}} \bc \langle M_{P} \rangle }
  \and \\
  \inferrule* [lab=process] {} {{M_{P}} \bc M_{N} \;| \;P|M_{P} }
\end{mathpar}

\begin{definition}[contextual application] Given a context $M$, and
  process $P$, we define the \emph{contextual application}, $M[P] :=
  M\{P/\Box\}$. That is, the contextual application of M to P is the
  substitution of $P$ for $\Box$ in $M$.
\end{definition}

$\meaningof{-} : L \to \mathcal{P}(\pi)$

\begin{mathpar}
  \inferrule* [lab=collection] {} {\meaningof{true} = \pi, \and \meaningof{~E} = \pi \setminus \meaningof{E}, \and \meaningof{E_{1} \& E_{2}} = \meaningof{E_{1}} \cap \meaningof{E_{2}}}
\end{mathpar}

\begin{mathpar}
  \inferrule* [lab=structure] {} {\meaningof{0} = \{ P \in \pi | P \equiv 0 \}, \and \\ \meaningof{E_1 | E_2} = \{ P \in \pi | P \equiv P_{1} | P_{2}, P_{1} \in \meaningof{E_{1}}, P_{2} \in \meaningof{E_2}\} }
\end{mathpar}

\begin{mathpar}
 \inferrule* [lab=behavior] {} {\meaningof{\langle a?b \rangle E} = \{ P \in \pi | P \equiv Q | u?(y)P', \\ \and \\\\ \and \\ \;\;\; u \in \meaningof{a}, \forall z.P'\{z/y\} \in \meaningof{E\{z/b\}}\}, \and \\ \meaningof{a!E} = \{ P \in \pi | P \equiv Q | x!\langle P' \rangle, x \in \meaningof{a} P' \in \meaningof{E}\} }
\end{mathpar}

\begin{mathpar}
 \inferrule* [lab=nominal] {} {\meaningof{\quotep{E}} = \{ \quotep{P} \in \quotep{\pi} | P \in \meaningof{E} \}, \and \meaningof{\quotep{P}} = \{ \quotep{Q} \in \quotep{\pi} | P \equiv Q \} \and \\ \meaningof{@\quotep{E}} = \{ P \in \pi | P \equiv @x, x \in \meaningof{E} \}}
\end{mathpar}

\begin{eqnarray*}
  \\
  \meaningof{-} : TS \to ST
\end{eqnarray*}

\begin{eqnarray*}
  \\
  L : TS \to ST
\end{eqnarray*}

\begin{eqnarray*}
  \\
  P \models E \iff P \in \meaningof{E}
\end{eqnarray*}

\begin{eqnarray*}
  P \approx_{L} Q \iff \forall E \in L. P \models E \iff Q \models E
\end{eqnarray*}

\begin{eqnarray*}
  P \approx_{K} Q
\end{eqnarray*}

\begin{eqnarray*}
  P \approx Q
\end{eqnarray*}

$\approx_{K} = \approx = \approx_{L}$

\subsubsection{Contextual duality}

Note that contexts extend the quotation operation to a family of
operations from processes to names. Given a context, $M$, we can
define a \emph{nominal context}, $\quotep{M}$ by $\quotep{M}[P] :=
\quotep{M[P]}$. To foreshadow what is to come we observe that these
operations enjoy a duality with processes very much like the duality
between vectors and maps from vectors to scalars.

Further, because the calculus is essentially higher-order, we have a
correspondence between contexts and processes. More specifically,
given a name $x$ and a context $M$ we can construct $M^{*}_{x}$ such
that 

\begin{mathpar}
  M^{*}_{x} | \lift{x}{P} \red M[P]
\end{mathpar}

namely,

\begin{mathpar}
  M^{*}_{x} := x?(u).M[\dropn{u}]
\end{mathpar}

The dependence of $M^{*}_{x}$ on a name makes it an abstraction, 

\begin{mathpar}
  M^{*} := (x)x?(u).M[\dropn{u}]
\end{mathpar}

\subsection{Additional notation}

It will sometimes be convenient to denote the process a name
quotes. We already have the notation $x = \quotep{P}$, but it will be
convenient to introduce an alternate notation, $\procn{x}$, when we
want to emphasize the connection to the use of the name. Note that, by
virtue of name equivalence, $\quotep{\procn{x}} \nameeq x$; so, the
notation is consistent with previous definitions.

Further, because names have structure it is possible to effect
substitutions on the basis of that structure. This means we need to
upgrade our notation for substitutions, which we accomplish by
adapting comprehension notation. Thus,

\begin{mathpar}
  P\{ y / x : x \in S \}
\end{mathpar}

is interpreted to mean the process derived from P by replacing (in a
capture-avoiding manner) each occurrence of $x$ in $S$ by $y$. For example,

\begin{mathpar}
  P\{ \quotep{\procn{x}|\procn{x}} / x : x \in \freenames{P} \}
\end{mathpar}

will replace each (occurrence) of a free name $x$ in $P$ by
$\quotep{\procn{x}|\procn{x}}$.

Also, we will avail ourselves of the notation $x^{L}$ and $x^{R}$ to
denote injections of a name into disjoint copies of the name
space. There are numerous ways to accomplish this. One example can be
found in \cite{MeredithR05}. This notation overloads to vectors of
names: $\vec{x}^{\pi} := (x_{i}^{\pi} \; : \; 0 \leq i < |\vec{x}| )$ where $\pi \in \{L,R\}$.

We also use $P^{\Box} := P|\Box$.

In \cite{MeredithR05} an interpretation of the new operator is
given. It turns out that there are several possible interpretations
all enjoying the requisite algebraic properties of the operator (see
\cite{milner91polyadicpi}). We will therefore make liberal use of
$(\nu\; \vec{x})P$.

% subsection the_syntax_and_semantics_of_the_notation_system (end)   

\input{qm2pi.qmops} 

\input{qm2pi.sterngerlach} 

\input{qm2pi.metric} 

% section concurrent_process_calculi (end)

%\input{qm2pi.proofsketch}

% section proof sketch (end)

%\input{qm2pi.slviaknots} 

% section spatial logic via knots (end)

\input{qm2pi.conclusion}

% section conclusion (end)

%\input{qm2pi.dtcodes} 

% section wiring algorithm (end)

\input{qm2pi.ack} 

% section acknowledgments (end)

\newpage


\bibliographystyle{plain}   
\bibliography{../../biblios/main.bib}

\input{qm2pi.rhodetails}

\end{document}

 

% section acknowledgments (end)

\newpage


\bibliographystyle{plain}   
\bibliography{../../biblios/main.bib}

\documentclass[12pt]{llncs}
%\documentclass{jktr}

\usepackage[pdftex]{hyperref}                   
\usepackage {listings}
\usepackage {mathpartir}
\usepackage{bcprules}
%\usepackage{listings}
                       
\usepackage{graphicx} 
%\usepackage[margins=2.5cm,nohead,nofoot]{geometry}
%\usepackage{geometry}
\usepackage{amsfonts}
\usepackage{amstext}
\usepackage{latexsym}
\usepackage{amssymb}
\usepackage{color}


%\include{myPreamble}
\include{qm2pi.local} 

%\ifpdf
%\usepackage[pdftex]{graphicx}
%\else
%\usepackage{graphicx}
%\fi

 % \ifpdf
%  \usepackage{pdfsync}
%  \if


%\title{Brief Article}
%\author{David F. Snyder}
%\author{L.G. Meredith}

%\address{Dept. of Math., Texas State University--San Marcos, San Marcos, TX 78666}
       
\pagestyle{empty}


\begin{document}

\lstset{language=[Objective]Caml,frame=shadowbox}

\input{qm2pi.front}

% section front matter (end)

\input{qm2pi.intro} 
 
% section introduction (end)

% \input{qm2pi.knotations} 

% section notation (end)

\input{qm2pi.process.calculi} 

% section concurrent_process_calculi_and_spatial_logics_ (end)
    
%\input{qm2pi.knots2pi} 

%\input{qm2pi.trefoil} 

%\input{qm2pi.mainthm} 

% subsection basic_interpretation (end)

%\input{qm2pi.rho.presentation} 
\subsection{The syntax and semantics of the notation system}\label{sub:the_syntax_and_semantics_of_the_notation_system} % (fold)

We now summarize a technical presentation of the calculus that
embodies our theory of dynamics. The typical presentation of such a
calculus follows the style of giving generators and relations on
them. The grammar, below, describing term constructors, freely
generates the set of processes, $\Proc$. This set is then quotiented
by a relation known as structural congruence and it is over this set
that the notion of dynamics is expressed. This presentation is
essentially that of \cite{MeredithR05} with the addition of
polyadicity and summation. For readability we have relegated some of
the technical subtleties to an appendix.

\subsubsection{Process grammar}\label{subsub:process_grammar}

\begin{mathpar}
  \inferrule* [lab=synchronization] {} {{M} \bc \pzero \;|\; x?F \;|\; x!C }
  \and
  \inferrule* [lab=abstraction] {} {{F} \bc (x)P}
  \and
  \inferrule* [lab=concretion] {} {{C} \bc \langle Q \rangle}
  \and
  \inferrule* [lab=process] {} {{P,Q} \bc M \;| \;P|Q \;|\; @{x}}
  \and
  \inferrule* [lab=name] {} {{x} \bc \quotep{P}}
\end{mathpar} 

Note that $\vec{x}$ (resp. $\vec{P}$) denotes a vector of names
(resp. processes) of length $|\vec{x}|$ (resp. $|\vec{P}|$). We adopt
the following useful abbreviations.

\begin{mathpar}
   x?(\vec{y}).P := x.(\vec{y})P \and  x\clift{\vec{P}} := x.\clift{\vec{P}}
   \and x!(y) := \lift{x}{\dropn{y}}
   \and \Pi_{i=0}^{n-1}P_i := P_0 | \ldots | P_{n-1}
\end{mathpar}

\subsubsection{Structural congruence}

\paragraph{Free and bound names and alpha-equivalence.} At the
core of structural equivalence is alpha-equivalence which identifies
process that are the same up to a change of variable. Formally, we
recognize the distinction between free and bound names. The free names
of a process, $\freenames{P}$, may be calculated recursively as
follows:

\begin{mathpar}
\freenames{\pzero} := \emptyset
  \and \\
  \freenames{x?(y).P} := \{ x \} \cup (\freenames{P} \setminus \{ y \})
  \and 
  \freenames{x!\langle P \rangle} := \{ x \} \cup \{ P \} 
  \and \\
  \freenames{P|Q} := \freenames{P} \cup \freenames{Q}
  \and \\
  \freenames{@{x}} := \{ x \}
\end{mathpar}

$\pi$
$\quotep{\pi}$

$\freenames{-} : \pi \to \mathcal{P}(\quotep{\pi})$

\begin{eqnarray*}
  \freenames{\pzero} & := & \emptyset \\
  \freenames{x?(y).P} & := & \{ x \} \cup (\freenames{P} \setminus \{ y \}) \\
  \freenames{x!\langle P \rangle} & := & \{ x \} \cup \{ P \} \\
  \freenames{P|Q} & := & \freenames{P} \cup \freenames{Q} \\
  \freenames{\dropn{x}} & := & \{ x \}
\end{eqnarray*}

The bound names of a process, $\boundnames{P}$, are those names occurring in $P$
that are not free. For example, in $x?(y).0$, the name $x$ is free, while $y$ is bound.

\begin{mathpar}
  \inferrule* [lab=monoidal-laws] {} { P|Q \equiv Q|P \and P|0 \equiv P \and P|(Q|R) \equiv (P|Q)|R }
\end{mathpar}

\begin{mathpar}
  \inferrule* [lab=alpha-equivalence] {} { (x)P \equiv (y)P\{y/x\} \and y \not\in \freenames{P} }
\end{mathpar}

\begin{definition}
Then two processes, $P,Q$, are alpha-equivalent if $P = Q\{\vec{y}/\vec{x}\}$ for
some $\vec{x} \in \boundnames{Q},\vec{y} \in \boundnames{P}$, where $Q\{\vec{y}/\vec{x}\}$
denotes the capture-avoiding substitution of $\vec{y}$ for $\vec{x}$ in $Q$.
\end{definition}

\begin{definition}
  The {\em structural congruence} \cite{SangiorgiWalker} , $\equiv$,
  between processes is the least congruence containing
  alpha-equivalence, satisfying the abelian monoid laws
  (associativity, commutativity and $\pzero$ as identity) for parallel
  composition $|$ and for summation $+$.
\end{definition}

\subsection{Name equivalence}

We take name equivalence, written $\nameeq$, to be the smallest
equivalence relation generated by the following rules.

\begin{mathpar}
\inferrule*[lab=Quote-drop]
{ }
{ \quotep{@{x}} \nameeq x }

\inferrule*[lab=Struct-equiv]
{ P \scong Q }
{ \quotep{P} \nameeq \quotep{Q} }
\end{mathpar}

The astute reader will have noticed that the mutual recursion of names
and processes imposes a mutual recursion on alpha-equivalence and
structural equivalence via name-equivalence. Fortunately, all of this
works out pleasantly and we may calculate in the natural way, free of
concern. The reader interested in the details is referred to the
appendix \ref{appendix:rho_details}.

\subsection{Substitution}

We use $\Proc$ for the set of processes, $\QProc$ for the set of
names, and $\id{\{}\vec{y} / \vec{x} \id{\}}$ to denote partial maps,
$s : \QProc \rightarrow \QProc$. A map, $s$ lifts, uniquely, to a map
on process terms, $\widehat{s} : \Proc \rightarrow \Proc$ by the
following equations.

\begin{mathpar}
  (0) \psubstp{Q}{P} := 0 \\
  (R \juxtap S) \psubstp{Q}{P}
  :=    
  (R)\psubstp{Q}{P} \juxtap (S) \psubstp{Q}{P} \\
  (x?(y).R) \psubstp{Q}{P}    
  :=    
  (x)\substp{Q}{P} (z)\concat( (R \psubstn{z}{y}) \psubstp{Q}{P} ) \\
  (\lift{x}{R}) \psubstp{Q}{P}  
  :=
  \lift{(x)\substp{Q}{P}}{ R \psubstp{Q}{P} } \\
%   (\dropn{x})  \psubstp{Q}{P}       
%   := 
%   \left\{ 
%     \begin{array}{ccc} 
%       \dropn{\quotep{Q}} & & x \nameeq \quotep{P} \\
%       \dropn{x} & & otherwise \\
%     \end{array}
%   \right. 
  (\dropn{x})  \psubstp{Q}{P}       
  := 
  \left\{ 
    \begin{array}{ccc} 
      Q & & x \nameeq \quotep{P} \\
      \dropn{x} & & otherwise \\
    \end{array}
  \right.
\end{mathpar}
 

where

\begin{eqnarray}
  (x)\id{\{} \lpquote Q \rpquote / \lpquote P \rpquote \id{\}}            = 
  \left\{ 
    \begin{array}{ccc}
      \lpquote Q \rpquote & & x \nameeq \lpquote P \rpquote \\
      x & & otherwise \\
    \end{array}
  \right. \nonumber
\end{eqnarray}

and $z$ is chosen distinct from $\quotep{P}$, $\quotep{Q}$, the free
names in $Q$, and all the names in $R$. Our $\alpha$-equivalence will
be built in the standard way from this substitution.

\begin{remark}\label{rem:no_self_referential_names}
  One consequence of these definitions is that $\forall P. \quotep{P}
  \not\in \freenames{P}$.
\end{remark}

\subsection{ Dynamic quote: an example }

Anticipating something of what's to come, consider applying the
substitution, $\widehat{\id{\{}u / z \id{\}}}$, to the following pair
of processes, $\lift{w}{y!(z)}$ and $w[ \lpquote y!(z) \rpquote ]$.

\begin{eqnarray}
	\lift{w}{y!(z)}\widehat{\id{\{}u / z \id{\}}}
		& = &
		\lift{w}{y!(u)} \nonumber\\
	w[ \lpquote y!(z) \rpquote ] \widehat{ \id{\{}u / z \id{\}} }
		& = &
		w[ \lpquote y!(z) \rpquote ] \nonumber
\end{eqnarray}

Because the body of the process between quotes is impervious to
substitution, we get radically different answers. In fact, by
examining the first process in an input context,
e.g. $x?(z).\lift{w}{y!(z)}$, we see that the process under the lift
operator may be shaped by prefixed inputs binding a name inside it. In
this sense, the lift operator will be seen as a way to dynamically
construct processes before reifying them as names.

Finally equipped with these standard features we can present the
dynamics of the calculus.

\subsubsection{Operational semantics} 

Finally, we introduce the computational dynamics. What marks these
algebras as distinct from other more traditionally studied algebraic
structures, e.g. vector spaces or polynomial rings, is the manner in
which dynamics is captured. In traditional structures, dynamics is typically
expressed through morphisms between such structures, as in linear maps
between vector spaces or morphisms between rings. In algebras
associated with the semantics of computation, the dynamics is
expressed as part of the algebraic structure itself, through a
reduction reduction relation typically denoted by $\red$. Below, we
give a recursive presentation of this relation for the calculus used
in the encoding.

$\red \subseteq \pi \times \pi$
$\red : \pi \to \mathcal{P}(\pi)$

\begin{mathpar}
  \inferrule* [lab=Comm] { \textsf{match}( x_{src}, x_{trgt} ) } { x_{trgt}?(y)P \; | \; x_{src}!\langle {Q} \rangle \red P\{\quotep{Q}/y}\} }
  \and \\
  \inferrule* [lab=Par] {{P} \red {P}'} {{{P} | {Q}} \red {{P}' | {Q}}}
  \and
  \inferrule* [lab=Equiv]{{{P} \scong {P}'} \andalso {{P}' \red {Q}'} \andalso {{Q}' \scong {Q}}}{{P} \red {Q}}
\end{mathpar}

\begin{eqnarray*}
  match_{\equiv} (\quotep{P},\quotep{Q}) & := & P \equiv Q \\
  match_{\dagger}(\quotep{P},\quotep{Q}) & := & \forall R. P|Q \red^{*} R => R \red^{*} 0 \\
  match_{K}(\quotep{P},\quotep{Q}) & := & K \mbox{ for some context } K
\end{eqnarray*}

$u?(x)P | u!\langle Q \rangle \red P\{\quotep{Q}/x\}$

%We write $\wred$ for $\red^*$, and $P\red$ if $\exists Q $ such that $ P \red Q$.
We write $P\red$ if $\exists Q $ such that $ P \red Q$ and $P\not\red$, otherwise.

\section{Replication}

As mentioned before, it is known that replication (and hence
recursion) can be implemented in a higher-order process algebra
\cite{SangiorgiWalker}. As our first example of calculation with the
machinery thus far presented we give the construction explicitly in
the {\rhoc}.

\begin{eqnarray}
	D_{x} & := & \prefix{x}{y}{(\binpar{\outputp{x}{y}}{@{y}})} \nonumber\\
	\bangp_{x}{P} & := & \binpar{{x}!\langle{\binpar{D_{x}}{P}}\rangle}{D_{x}} \nonumber
\end{eqnarray}

\begin{eqnarray}
	\bangp_{x}{P} & & \nonumber\\
	=
	& {x}!\langle{(\prefix{x}{y}{(\outputp{x}{y} | @{y})) | P}}\rangle 
	      | \prefix{x}{y}{(\outputp{x}{y} | @{y})} & \nonumber\\
	\red
	& (\outputp{x}{y} | @{y})\substn{\quotep{(\prefix{x}{y}{(@{y} | \outputp{x}{y})) | P}}}{y} & \nonumber\\
	=
	& \outputp{x}{\quotep{(\prefix{x}{y}{(\outputp{x}{y} | @{y})) | P}}}
	  | {(\prefix{x}{y}{(\outputp{x}{y} | @{y})) | P}} & \nonumber\\
	\red
	& \ldots & \nonumber\\
	\red^*
	& P | P | \ldots & \nonumber
\end{eqnarray}

Of course, this encoding, as an implementation, runs away, unfolding
$\bangp{P}$ eagerly. A lazier and more implementable replication
operator, restricted to input-guarded processes, may be obtained as follows.

\begin{eqnarray}
\bangp{\prefix{u}{v}{P}} 
	:= 
	\binpar{\lift{x}{\prefix{u}{v}{(\binpar{D(x)}{P})}}}{D(x)} \nonumber
\end{eqnarray}

\begin{remark}
  Note that the lazier definition still does not deal with summation
  or mixed summation (i.e. sums over input and output). The reader is
  invited to construct definitions of replication that deal with these
  features. 

  Further, the definitions are parameterized in a name, $x$. Can you,
  gentle reader, make a definition that eliminates this parameter and
  guarantees no accidental interaction between the replication
  machinery and the process being replicated -- i.e. no accidental
  sharing of names used by the process to get its work done and the
  name(s) used by the replication to effect copying. This latter
  revision of the definition of replication is crucial to obtaining
  the expected identity $!!P \sim !P$.
\end{remark}

\begin{remark}\label{rem:paradoxical_combinator}
  The reader familiar with the lambda calculus will have noticed the
  similarity between $D$ and the paradoxical combinator.

  [Ed. note: the existence of this seems to suggest we have to be more
  restrictive on the set of processes and names we admit if we are to
  support no-cloning.]
\end{remark}

\subsubsection{Bisimulation}

The computational dynamics gives rise to another kind of equivalence,
the equivalence of computational behavior. As previously mentioned
this is typically captured \emph{via} some form of bisimulation.

% The notion we use in this paper is weak barbed bisimulation
% \cite{milner91polyadicpi}.

The notion we use in this paper is derived from weak barbed
bisimulation \cite{milner91polyadicpi}. 

\begin{definition}
An \emph{observation relation}, $\downarrow_{\mathcal N}$, over a set
of names, $\mathcal N$, is the smallest relation satisfying the rules
below.

\infrule[Out-barb]{y \in {\mathcal N}, \; x \nameeq y}
		  {\outputp{x}{v} \downarrow_{\mathcal N} x}
\infrule[Par-barb]{\mbox{$P\downarrow_{\mathcal N} x$ or $Q\downarrow_{\mathcal N} x$}}
		  {\binpar{P}{Q} \downarrow_{\mathcal N} x}

We write $P \Downarrow_{\mathcal N} x$ if there is $Q$ such that 
$P \wred Q$ and $Q \downarrow_{\mathcal N} x$.
\end{definition}

\begin{definition}
%\label{def.bbisim}
An  ${\mathcal N}$-\emph{barbed bisimulation} over a set of names, ${\mathcal N}$, is a symmetric binary relation 
${\mathcal S}_{\mathcal N}$ between agents such that $P\rel{S}_{\mathcal N}Q$ implies:
\begin{enumerate}
\item If $P \red P'$ then $Q \wred Q'$ and $P'\rel{S}_{\mathcal N} Q'$.
\item If $P\downarrow_{\mathcal N} x$, then $Q\Downarrow_{\mathcal N} x$.
\end{enumerate}
$P$ is ${\mathcal N}$-barbed bisimilar to $Q$, written
$P \wbbisim_{\mathcal N} Q$, if $P \rel{S}_{\mathcal N} Q$ for some ${\mathcal N}$-barbed bisimulation ${\mathcal S}_{\mathcal N}$.
\end{definition}

$\mathcal{R} \subseteq \pi \times \pi$

$P \mathcal{R} Q => \forall P'. P \red P' \Rightarrow \exists Q'. Q \red Q', P' \mathcal{R} Q'$

$P \vdash x \Rightarrow Q \vdash x$

\begin{mathpar}
  \inferrule*[lab=Out-barb]{x \nameeq y}{{y}!\langle{Q}\rangle \vdash x}
  \and
  \inferrule*[lab=Par-barb]{\mbox{$P\vdash x$ or $Q\vdash x$}}{\binpar{P}{Q} \vdash x}
\end{mathpar}

\subsubsection{Contexts}

One of the principle advantages of computational calculi like the
$\pi$-calculus is a well-defined notion of context,
contextual-equivalence and a correlation between
contextual-equivalence and notions of bisimulation. The notion of
context allows the decomposition of a process into (sub-)process and
its syntactic environment, its context. Thus, a context may be
thought of as a process with a ``hole'' (written $\Box$) in it. The
application of a context $M$ to a process $P$, written $M[P]$, is
tantamount to filling the hole in $M$ with $P$. In this paper we do
not need the full weight of this theory, but do make use of the notion
of context in the proof the main theorem. 

\begin{mathpar}
  \inferrule* [lab=summation] {} {{M_{M},M_{N}} \bc \Box \;|\; x.M_{A} \;|\; M_{M}+M_{N}}
  \and
  \inferrule* [lab=agent] {} {{M_{A}} \bc (\vec{x})M_{P} \;| \; \clift{P_0,\ldots,M_{P},\ldots,P_N}}
  \and \\
  \inferrule* [lab=process] {} {{M_{P}} \bc M_{N} \;| \;P|M_{P} }
\end{mathpar} 

\begin{mathpar}
  \inferrule* [lab=sychronization] {} {M_{N} \bc \Box \;|\; x?M_{F} \;|\; x!M_{C}}
  \and
  \inferrule* [lab=abstraction] {} {{M_{F}} \bc (x)M_{P} }
  \and
  \inferrule* [lab=concretion] {} {{M_{C}} \bc \langle M_{P} \rangle }
  \and \\
  \inferrule* [lab=process] {} {{M_{P}} \bc M_{N} \;| \;P|M_{P} }
\end{mathpar}

\begin{definition}[contextual application] Given a context $M$, and
  process $P$, we define the \emph{contextual application}, $M[P] :=
  M\{P/\Box\}$. That is, the contextual application of M to P is the
  substitution of $P$ for $\Box$ in $M$.
\end{definition}

$\meaningof{-} : L \to \mathcal{P}(\pi)$

\begin{mathpar}
  \inferrule* [lab=collection] {} {\meaningof{true} = \pi, \and \meaningof{~E} = \pi \setminus \meaningof{E}, \and \meaningof{E_{1} \& E_{2}} = \meaningof{E_{1}} \cap \meaningof{E_{2}}}
\end{mathpar}

\begin{mathpar}
  \inferrule* [lab=structure] {} {\meaningof{0} = \{ P \in \pi | P \equiv 0 \}, \and \\ \meaningof{E_1 | E_2} = \{ P \in \pi | P \equiv P_{1} | P_{2}, P_{1} \in \meaningof{E_{1}}, P_{2} \in \meaningof{E_2}\} }
\end{mathpar}

\begin{mathpar}
 \inferrule* [lab=behavior] {} {\meaningof{\langle a?b \rangle E} = \{ P \in \pi | P \equiv Q | u?(y)P', \\ \and \\\\ \and \\ \;\;\; u \in \meaningof{a}, \forall z.P'\{z/y\} \in \meaningof{E\{z/b\}}\}, \and \\ \meaningof{a!E} = \{ P \in \pi | P \equiv Q | x!\langle P' \rangle, x \in \meaningof{a} P' \in \meaningof{E}\} }
\end{mathpar}

\begin{mathpar}
 \inferrule* [lab=nominal] {} {\meaningof{\quotep{E}} = \{ \quotep{P} \in \quotep{\pi} | P \in \meaningof{E} \}, \and \meaningof{\quotep{P}} = \{ \quotep{Q} \in \quotep{\pi} | P \equiv Q \} \and \\ \meaningof{@\quotep{E}} = \{ P \in \pi | P \equiv @x, x \in \meaningof{E} \}}
\end{mathpar}

\begin{eqnarray*}
  \\
  \meaningof{-} : TS \to ST
\end{eqnarray*}

\begin{eqnarray*}
  \\
  L : TS \to ST
\end{eqnarray*}

\begin{eqnarray*}
  \\
  P \models E \iff P \in \meaningof{E}
\end{eqnarray*}

\begin{eqnarray*}
  P \approx_{L} Q \iff \forall E \in L. P \models E \iff Q \models E
\end{eqnarray*}

\begin{eqnarray*}
  P \approx_{K} Q
\end{eqnarray*}

\begin{eqnarray*}
  P \approx Q
\end{eqnarray*}

$\approx_{K} = \approx = \approx_{L}$

\subsubsection{Contextual duality}

Note that contexts extend the quotation operation to a family of
operations from processes to names. Given a context, $M$, we can
define a \emph{nominal context}, $\quotep{M}$ by $\quotep{M}[P] :=
\quotep{M[P]}$. To foreshadow what is to come we observe that these
operations enjoy a duality with processes very much like the duality
between vectors and maps from vectors to scalars.

Further, because the calculus is essentially higher-order, we have a
correspondence between contexts and processes. More specifically,
given a name $x$ and a context $M$ we can construct $M^{*}_{x}$ such
that 

\begin{mathpar}
  M^{*}_{x} | \lift{x}{P} \red M[P]
\end{mathpar}

namely,

\begin{mathpar}
  M^{*}_{x} := x?(u).M[\dropn{u}]
\end{mathpar}

The dependence of $M^{*}_{x}$ on a name makes it an abstraction, 

\begin{mathpar}
  M^{*} := (x)x?(u).M[\dropn{u}]
\end{mathpar}

\subsection{Additional notation}

It will sometimes be convenient to denote the process a name
quotes. We already have the notation $x = \quotep{P}$, but it will be
convenient to introduce an alternate notation, $\procn{x}$, when we
want to emphasize the connection to the use of the name. Note that, by
virtue of name equivalence, $\quotep{\procn{x}} \nameeq x$; so, the
notation is consistent with previous definitions.

Further, because names have structure it is possible to effect
substitutions on the basis of that structure. This means we need to
upgrade our notation for substitutions, which we accomplish by
adapting comprehension notation. Thus,

\begin{mathpar}
  P\{ y / x : x \in S \}
\end{mathpar}

is interpreted to mean the process derived from P by replacing (in a
capture-avoiding manner) each occurrence of $x$ in $S$ by $y$. For example,

\begin{mathpar}
  P\{ \quotep{\procn{x}|\procn{x}} / x : x \in \freenames{P} \}
\end{mathpar}

will replace each (occurrence) of a free name $x$ in $P$ by
$\quotep{\procn{x}|\procn{x}}$.

Also, we will avail ourselves of the notation $x^{L}$ and $x^{R}$ to
denote injections of a name into disjoint copies of the name
space. There are numerous ways to accomplish this. One example can be
found in \cite{MeredithR05}. This notation overloads to vectors of
names: $\vec{x}^{\pi} := (x_{i}^{\pi} \; : \; 0 \leq i < |\vec{x}| )$ where $\pi \in \{L,R\}$.

We also use $P^{\Box} := P|\Box$.

In \cite{MeredithR05} an interpretation of the new operator is
given. It turns out that there are several possible interpretations
all enjoying the requisite algebraic properties of the operator (see
\cite{milner91polyadicpi}). We will therefore make liberal use of
$(\nu\; \vec{x})P$.

% subsection the_syntax_and_semantics_of_the_notation_system (end)   

\input{qm2pi.qmops} 

\input{qm2pi.sterngerlach} 

\input{qm2pi.metric} 

% section concurrent_process_calculi (end)

%\input{qm2pi.proofsketch}

% section proof sketch (end)

%\input{qm2pi.slviaknots} 

% section spatial logic via knots (end)

\input{qm2pi.conclusion}

% section conclusion (end)

%\input{qm2pi.dtcodes} 

% section wiring algorithm (end)

\input{qm2pi.ack} 

% section acknowledgments (end)

\newpage


\bibliographystyle{plain}   
\bibliography{../../biblios/main.bib}

\input{qm2pi.rhodetails}

\end{document}



\end{document}



% section proof sketch (end)

%\section{Unlikely characters: spatial logic for
  knots}\label{sub:characteristic_formulae} % (fold)

Associated to the mobile process calculi are a family of logics known
as the Hennessy-Milner logics. These logics typically enjoy a
semantics interpreting formulae as sets of processes that when
factored through the encoding outlined above allows an identification
of classes of knots with logical formulae. In the context of this
encoding the sub-family known as the spatial logics \cite{CairesC03}
\cite{CairesC04} \cite{Caires04} are of particular interest providing
several important features for expressing and reasoning about
properties (i.e. classes) of knots. We hint here at how this may be done.

%\begin{description}
%\item [structural connectives] 
\subsubsection{Structural connectives} The spatial logics enjoy
structural connectives corresponding, at the logical level, to the
parallel composition ($P | Q$) and new name ($(\nu \; x)P$)
connectives for processes. As illustrated in the examples below, these
connectives are extremely expressive given the shape of our encoding.
%\item [decideable satisfaction]

\subsubsection{Decideable satisfaction}
In \cite{Caires04} the satisfaction relation is shown to be decideable
for a rich class of processes. It further turns out that the image of
the our encoding is a proper subset of that class. This result
provides the basis for an algorithm by which to search for knots
enjoying a given property.
%\item [characteristic formulae]

\subsubsection{Characteristic formulae}
In the same paper \cite{Caires04} , Caires presents a means of calculating
characteristic formulae, selecting equivalence classes of processes
up to a pre--specified depth limit on the support set of names. Composed with our
encoding, this characteristic formula can be used to select
characteristic formulae for knots.
%\end{description}

\subsubsection{Spatial logic formulae}

The grammar below (segmented for comprehension) summarizes the syntax
of spatial logic formulae. We employ illustrative examples in the
sequel to provide an intuitive understanding of their meaning
referring the reader to \cite{Caires04} for a more detailed explication
of the semantics.

\begin{mathpar}
  \inferrule* [lab=boolean] {} {{A,B} \bc T \;|\; \neg A \;|\; A \wedge B \;|\; \eta = \eta'}
  \and
  \inferrule* [lab=spatial] {} {|\; \pzero \;|\; A | B \;|\; x \text{\textregistered} A \;|\; \forall x . A \;|\;  H x . A}
  \and
  \inferrule* [lab=behavioral] {} {|\; \alpha . A}
  \and 
  \inferrule* [lab=recursion] {} {|\; X(\vec{u}) \;|\; \mu X(\vec{u}) . A}
  \and
  \inferrule* [lab=action] {} {\alpha \bc \langle x?(\vec{y}) \rangle \;|\; \langle x!(\vec{y}) \rangle \;|\; \langle \tau \rangle}
  \and 
  \inferrule* [lab=name] {} {\eta \bc x \;|\; \tau}
\end{mathpar} 

% subsection characteristic_formulae (end)   	 

\subsection{Example formulae}\label{sub:example_formulae_} % (fold)

\subsubsection{Crossing as formula.}
% 
% \begin{align*}
%   \frac{d}{dx} \sin x &= \cos x 
%   & \frac{d}{dx} e^x &= e^x \\
%   \frac{d}{dx} \cos x &= - \sin x 
%   & \frac{d}{dx} \log x &= \frac{1}{x} \\
% \end{align*} 

\begin{align*}
 \mu C(x_{0},x_{1},y_{0},y_{1},u).&(\langle x_{0}?(z) \rangle(\langle u! \rangle\langle y_{1}!z \rangle C(x_{0},x_{1},y_{0},y_{1},u)) & \\
  & \wedge \langle y_{1}?(z) \rangle (\langle u! \rangle \langle x_{0}!z \rangle C(x_{0},x_{1},y_{0},y_{1},u)) & \\
  & \wedge \langle x_{1}?(z) \rangle (\langle u? \rangle \langle y_{0}!z \rangle C(x_{0},x_{1},y_{0},y_{1},u)) & \\
  & \wedge \langle y_{0}?(z) \rangle (\langle u? \rangle \langle x_{1}!z \rangle C(x_{0},x_{1},y_{0},y_{1},u))) &
\end{align*}

The lexicographical similarity between the shape of this formulae and
the shape of definition of the process representing a crossing reveals
the intuitive meaning of this formulae. It describes the capabilities
of a process that has the right to represent a crossing. For example
it picks out processes that may perform an input on the port $x_0$ in
its initial menu of capabilities. What differentiates the formula
from the process, however, is that the crossing process is the
smallest candidate to satisfy the formula. Infinitely many other
processes -- with internal behavior hidden behind this interface, so
to speak -- also satisfy this formula. Even this simple formula,
then, can be seen to open a new view onto knots, providing a
computational interpretation of \emph{virtual} knots.

Note that this formula is derived by hand. A similar formula can be
derived by employing Caires' calculation of characteristic formula
\cite{Caires04} to the process representing a crossing. In light of
this discussion, we let
$\meaningof{C}_{\phi}(x0,x1,y0,y1,u)$ denote a formula specifying the
dynamics we wish to capture of a crossing. To guarantee we preserve
the shape of the interface and minimal semantics we demand that
$\meaningof{C}_{\phi}(x0,x1,y0,y1,u) \Rightarrow
\textbf{C}(x0,x1,y0,y1,u)$ where $\textbf{C}(x0,x1,y0,y1,u)$ denotes
the formula above.
                            
\subsubsection{Crossing number constraints.}
The moral content of the context lemma (Lemma \ref{context}) is that the notion of
``locality'' in the Reidemeister moves is effectively captured by the
parallel composition operator of the process calculus. This intuition
extends through the logic. Given a formula,
$\meaningof{C}_{\phi}(x0,x1,y0,y1,u)$, we can use the structural
connectives to specify constraints on crossing numbers, such as at
least $n$ crossings, or exactly $n$ crossings.
\begin{mathpar}
  \inferrule* [lab=at-least-n] {} { K^{\geq n}_{\phi}(\vec{xs},\vec{ys}) := \Pi_{i=0}^{n-1} Hu . \meaningof{C}_{\phi}(xs_i,ys_i,u) | T }
  \and 
  \inferrule* [lab=exactly-n] {} { K^{= n}_{\phi}(\vec{xs},\vec{ys}) := \Pi_{i=0}^{n-1} Hu . \meaningof{C}_{\phi}(xs_i,ys_i,u) | \neg (\forall x_0,y_0,x_1,y_1,u . \meaningof{C}_{\phi}(x_0,y_0,x_1,y_1,u) | T) }
\end{mathpar}

To round out this section, recall that the encoding of an $n$-crossing
knot decomposes into a parallel composition of $n$ \emph{copies} of a
crossing process together with a wiring harness. To specify different
knot classes with the same crossing number amounts to specifying
logical constraints on the wiring harness. In the interest of space,
we defer examples to a forthcoming paper. Suffice it to say that both
the conditions ``alternating knot'' and ``contains the tangle
corresponding to 5/3'' are expressible. For example, it is possible to
calculate the characteristic formula of a process corresponding to the
tangle 5/3 and conjoin it into the classifying formula via the
composition connective of the logic.

Finally, we wish to observe that it is entirely within reason to
contemplate a more domain-specific version of spatial logic tailored
to the shape of processes in the image of the encoding. Such a
domain-specific logic would have a better claim to the title formal
language of knot properties.

% subsection example_formulae_ (end)

% section knots_as_processes (end) 

% section spatial logic via knots (end)

\section{Conclusions and future work}

\paragraph{Testing physical space}
You, gentle reader, may wonder why of all the theorems to be proved
given this set up we pick the one above. In some sense it's hardly
central to quantum mechanics. We see it as central in the sense that
it firmly establishes a notion of physical space arising from a notion
of the equivalence of behavior. Relating bisimulation to a metric is a
big step forward, but one is faced with interpreting the relationship
of that metric space to something more physical. Quantum mechanical
notions of ``physical'' space are still far from intuitive, but by
relating this idea of distance as testing to calculations that predict
physical circumstances we are making a not insignificant step forward
toward an understanding of the physical space we inhabit as
essentially dynamic.

\paragraph{Effectivity and simulation}
One of the observations we have yet to make is that the entire program
spelled out here is effective. We have built various interpreters for
the reflective calculus at work in this interpretation. In principle,
then, we can simulate quantum mechanics on a computer. The place where
the simulation may lose fidelity is the infinitely branching summation
for the annihilator.

In this connection i also want to point out that the evaluation style
calculation of the inner product puts the non-determinism of the
summation right at the heart of measurement. This suggests that
Milner's original reduction-based formulation of the dynamics of his
calculi in terms of sums was not just notationally suggestive of a
notion of measure-and-continue but captured some significant part of
the physics.

\paragraph{Quantum continuations}
In light of this last observation i want to point out that the
predominant account of quantum mechanics is missing a key aspect of a
truly compositional story of the physical situation. In a real lab,
when a measurement is made the observation can be made to feed into
another device that then makes another measurement conditioned on the
results of the first. This means that after the superposition was
collapsed the entire experimental set up remained in
superposition. While QM offers a means of writing this down it doesn't
quite line up well with the well-trodden formulation of computation
and continuation that we see so succinctly expressed in Milner's
calculi. This suggests that there might be advantages to this account
of dynamics waiting to be explored.

\paragraph{Quantum logic}
In this connection, we also note that by virtue of having the
Hennessy-Milner construction, we can pull the construction through the
interpretation of QM. This gives us a natural candidate for a quantum
logic that enjoys an extremely tight connection with it's domain of
interpretation, making the construction much less ad hoc (rather it is
the image of functor!).

\paragraph{Quantum probabiity}
i have questions about the basis of the interpretation of inner
product as probability amplitude. In particular, using which
axiomatization of probability theory does the notion of probability
amplitude earn the right to be so dubbed? In other words, where is the
proof that the operation for calculating a probability amplitude (and
then squaring) satisfies the axioms of what it means to calculate a
probability? Even if such a proof exists (i have yet to find it in the
literature), i wonder if it might not be possible to turn things on
their heads. Can we view the calculation of the probability amplitude
as an axiomatization of probability? If so, then the definition we
give for calculating probability amplitude may provide the basis for
an \emph{effective} theory of probability.

\paragraph{Quantum vs ``biological'' information}
Finally, i want to conclude with a more philosophical observation. At
a recent workshop in which QM was a predominant topic i noticed
something about quantum information. The speaker was giving a riveting
discussion of axiomatic QM and showing how properties of ``no
cloning'' and ``no deleting'' emerged as consequences of the
axiomatization. Theorems of this form are necessary to give us a sense
of confidence that our axioms characterize the physical theory. What
struck me, though, was that if quantum information is neither erasable
nor replicable it is markedly different from \emph{life}. Two of the
things we know about life is that

\begin{itemize}
  \item it ends;
  \item to gain some measure of persistence, to transcend it's
    finitude it is imminently copyable.
\end{itemize}

Both of these qualities are summarized succinctly in the aphorism: all
flesh is grass. For me these two kinds of ``information'' -- call them
quantum and biological -- are end points on a spectrum of strategies
for persistence. At one end, we have those curious entities that enjoy
uniqueness and permanence; at the other, we have those who in the face
of a certain end and an uncertain present make a go of passing
something on. To me one of the more remarkable aspects of the latter
strategy is that in the presence of noise (and certain features of
copying) we get a kind of dynamism, a chance for improvement against a
given persistent condition.

% subsection other_calculi_other_bisimulations_and_geometry_as_behavior (end)




% section conclusion (end)

%\documentclass[12pt]{llncs}
%\documentclass{jktr}

\usepackage[pdftex]{hyperref}                   
\usepackage {listings}
\usepackage {mathpartir}
\usepackage{bcprules}
%\usepackage{listings}
                       
\usepackage{graphicx} 
%\usepackage[margins=2.5cm,nohead,nofoot]{geometry}
%\usepackage{geometry}
\usepackage{amsfonts}
\usepackage{amstext}
\usepackage{latexsym}
\usepackage{amssymb}
\usepackage{color}


%\include{myPreamble}
\documentclass[12pt]{llncs}
%\documentclass{jktr}

\usepackage[pdftex]{hyperref}                   
\usepackage {listings}
\usepackage {mathpartir}
\usepackage{bcprules}
%\usepackage{listings}
                       
\usepackage{graphicx} 
%\usepackage[margins=2.5cm,nohead,nofoot]{geometry}
%\usepackage{geometry}
\usepackage{amsfonts}
\usepackage{amstext}
\usepackage{latexsym}
\usepackage{amssymb}
\usepackage{color}


%\include{myPreamble}
\include{qm2pi.local} 

%\ifpdf
%\usepackage[pdftex]{graphicx}
%\else
%\usepackage{graphicx}
%\fi

 % \ifpdf
%  \usepackage{pdfsync}
%  \if


%\title{Brief Article}
%\author{David F. Snyder}
%\author{L.G. Meredith}

%\address{Dept. of Math., Texas State University--San Marcos, San Marcos, TX 78666}
       
\pagestyle{empty}


\begin{document}

\lstset{language=[Objective]Caml,frame=shadowbox}

\input{qm2pi.front}

% section front matter (end)

\input{qm2pi.intro} 
 
% section introduction (end)

% \input{qm2pi.knotations} 

% section notation (end)

\input{qm2pi.process.calculi} 

% section concurrent_process_calculi_and_spatial_logics_ (end)
    
%\input{qm2pi.knots2pi} 

%\input{qm2pi.trefoil} 

%\input{qm2pi.mainthm} 

% subsection basic_interpretation (end)

%\input{qm2pi.rho.presentation} 
\subsection{The syntax and semantics of the notation system}\label{sub:the_syntax_and_semantics_of_the_notation_system} % (fold)

We now summarize a technical presentation of the calculus that
embodies our theory of dynamics. The typical presentation of such a
calculus follows the style of giving generators and relations on
them. The grammar, below, describing term constructors, freely
generates the set of processes, $\Proc$. This set is then quotiented
by a relation known as structural congruence and it is over this set
that the notion of dynamics is expressed. This presentation is
essentially that of \cite{MeredithR05} with the addition of
polyadicity and summation. For readability we have relegated some of
the technical subtleties to an appendix.

\subsubsection{Process grammar}\label{subsub:process_grammar}

\begin{mathpar}
  \inferrule* [lab=synchronization] {} {{M} \bc \pzero \;|\; x?F \;|\; x!C }
  \and
  \inferrule* [lab=abstraction] {} {{F} \bc (x)P}
  \and
  \inferrule* [lab=concretion] {} {{C} \bc \langle Q \rangle}
  \and
  \inferrule* [lab=process] {} {{P,Q} \bc M \;| \;P|Q \;|\; @{x}}
  \and
  \inferrule* [lab=name] {} {{x} \bc \quotep{P}}
\end{mathpar} 

Note that $\vec{x}$ (resp. $\vec{P}$) denotes a vector of names
(resp. processes) of length $|\vec{x}|$ (resp. $|\vec{P}|$). We adopt
the following useful abbreviations.

\begin{mathpar}
   x?(\vec{y}).P := x.(\vec{y})P \and  x\clift{\vec{P}} := x.\clift{\vec{P}}
   \and x!(y) := \lift{x}{\dropn{y}}
   \and \Pi_{i=0}^{n-1}P_i := P_0 | \ldots | P_{n-1}
\end{mathpar}

\subsubsection{Structural congruence}

\paragraph{Free and bound names and alpha-equivalence.} At the
core of structural equivalence is alpha-equivalence which identifies
process that are the same up to a change of variable. Formally, we
recognize the distinction between free and bound names. The free names
of a process, $\freenames{P}$, may be calculated recursively as
follows:

\begin{mathpar}
\freenames{\pzero} := \emptyset
  \and \\
  \freenames{x?(y).P} := \{ x \} \cup (\freenames{P} \setminus \{ y \})
  \and 
  \freenames{x!\langle P \rangle} := \{ x \} \cup \{ P \} 
  \and \\
  \freenames{P|Q} := \freenames{P} \cup \freenames{Q}
  \and \\
  \freenames{@{x}} := \{ x \}
\end{mathpar}

$\pi$
$\quotep{\pi}$

$\freenames{-} : \pi \to \mathcal{P}(\quotep{\pi})$

\begin{eqnarray*}
  \freenames{\pzero} & := & \emptyset \\
  \freenames{x?(y).P} & := & \{ x \} \cup (\freenames{P} \setminus \{ y \}) \\
  \freenames{x!\langle P \rangle} & := & \{ x \} \cup \{ P \} \\
  \freenames{P|Q} & := & \freenames{P} \cup \freenames{Q} \\
  \freenames{\dropn{x}} & := & \{ x \}
\end{eqnarray*}

The bound names of a process, $\boundnames{P}$, are those names occurring in $P$
that are not free. For example, in $x?(y).0$, the name $x$ is free, while $y$ is bound.

\begin{mathpar}
  \inferrule* [lab=monoidal-laws] {} { P|Q \equiv Q|P \and P|0 \equiv P \and P|(Q|R) \equiv (P|Q)|R }
\end{mathpar}

\begin{mathpar}
  \inferrule* [lab=alpha-equivalence] {} { (x)P \equiv (y)P\{y/x\} \and y \not\in \freenames{P} }
\end{mathpar}

\begin{definition}
Then two processes, $P,Q$, are alpha-equivalent if $P = Q\{\vec{y}/\vec{x}\}$ for
some $\vec{x} \in \boundnames{Q},\vec{y} \in \boundnames{P}$, where $Q\{\vec{y}/\vec{x}\}$
denotes the capture-avoiding substitution of $\vec{y}$ for $\vec{x}$ in $Q$.
\end{definition}

\begin{definition}
  The {\em structural congruence} \cite{SangiorgiWalker} , $\equiv$,
  between processes is the least congruence containing
  alpha-equivalence, satisfying the abelian monoid laws
  (associativity, commutativity and $\pzero$ as identity) for parallel
  composition $|$ and for summation $+$.
\end{definition}

\subsection{Name equivalence}

We take name equivalence, written $\nameeq$, to be the smallest
equivalence relation generated by the following rules.

\begin{mathpar}
\inferrule*[lab=Quote-drop]
{ }
{ \quotep{@{x}} \nameeq x }

\inferrule*[lab=Struct-equiv]
{ P \scong Q }
{ \quotep{P} \nameeq \quotep{Q} }
\end{mathpar}

The astute reader will have noticed that the mutual recursion of names
and processes imposes a mutual recursion on alpha-equivalence and
structural equivalence via name-equivalence. Fortunately, all of this
works out pleasantly and we may calculate in the natural way, free of
concern. The reader interested in the details is referred to the
appendix \ref{appendix:rho_details}.

\subsection{Substitution}

We use $\Proc$ for the set of processes, $\QProc$ for the set of
names, and $\id{\{}\vec{y} / \vec{x} \id{\}}$ to denote partial maps,
$s : \QProc \rightarrow \QProc$. A map, $s$ lifts, uniquely, to a map
on process terms, $\widehat{s} : \Proc \rightarrow \Proc$ by the
following equations.

\begin{mathpar}
  (0) \psubstp{Q}{P} := 0 \\
  (R \juxtap S) \psubstp{Q}{P}
  :=    
  (R)\psubstp{Q}{P} \juxtap (S) \psubstp{Q}{P} \\
  (x?(y).R) \psubstp{Q}{P}    
  :=    
  (x)\substp{Q}{P} (z)\concat( (R \psubstn{z}{y}) \psubstp{Q}{P} ) \\
  (\lift{x}{R}) \psubstp{Q}{P}  
  :=
  \lift{(x)\substp{Q}{P}}{ R \psubstp{Q}{P} } \\
%   (\dropn{x})  \psubstp{Q}{P}       
%   := 
%   \left\{ 
%     \begin{array}{ccc} 
%       \dropn{\quotep{Q}} & & x \nameeq \quotep{P} \\
%       \dropn{x} & & otherwise \\
%     \end{array}
%   \right. 
  (\dropn{x})  \psubstp{Q}{P}       
  := 
  \left\{ 
    \begin{array}{ccc} 
      Q & & x \nameeq \quotep{P} \\
      \dropn{x} & & otherwise \\
    \end{array}
  \right.
\end{mathpar}
 

where

\begin{eqnarray}
  (x)\id{\{} \lpquote Q \rpquote / \lpquote P \rpquote \id{\}}            = 
  \left\{ 
    \begin{array}{ccc}
      \lpquote Q \rpquote & & x \nameeq \lpquote P \rpquote \\
      x & & otherwise \\
    \end{array}
  \right. \nonumber
\end{eqnarray}

and $z$ is chosen distinct from $\quotep{P}$, $\quotep{Q}$, the free
names in $Q$, and all the names in $R$. Our $\alpha$-equivalence will
be built in the standard way from this substitution.

\begin{remark}\label{rem:no_self_referential_names}
  One consequence of these definitions is that $\forall P. \quotep{P}
  \not\in \freenames{P}$.
\end{remark}

\subsection{ Dynamic quote: an example }

Anticipating something of what's to come, consider applying the
substitution, $\widehat{\id{\{}u / z \id{\}}}$, to the following pair
of processes, $\lift{w}{y!(z)}$ and $w[ \lpquote y!(z) \rpquote ]$.

\begin{eqnarray}
	\lift{w}{y!(z)}\widehat{\id{\{}u / z \id{\}}}
		& = &
		\lift{w}{y!(u)} \nonumber\\
	w[ \lpquote y!(z) \rpquote ] \widehat{ \id{\{}u / z \id{\}} }
		& = &
		w[ \lpquote y!(z) \rpquote ] \nonumber
\end{eqnarray}

Because the body of the process between quotes is impervious to
substitution, we get radically different answers. In fact, by
examining the first process in an input context,
e.g. $x?(z).\lift{w}{y!(z)}$, we see that the process under the lift
operator may be shaped by prefixed inputs binding a name inside it. In
this sense, the lift operator will be seen as a way to dynamically
construct processes before reifying them as names.

Finally equipped with these standard features we can present the
dynamics of the calculus.

\subsubsection{Operational semantics} 

Finally, we introduce the computational dynamics. What marks these
algebras as distinct from other more traditionally studied algebraic
structures, e.g. vector spaces or polynomial rings, is the manner in
which dynamics is captured. In traditional structures, dynamics is typically
expressed through morphisms between such structures, as in linear maps
between vector spaces or morphisms between rings. In algebras
associated with the semantics of computation, the dynamics is
expressed as part of the algebraic structure itself, through a
reduction reduction relation typically denoted by $\red$. Below, we
give a recursive presentation of this relation for the calculus used
in the encoding.

$\red \subseteq \pi \times \pi$
$\red : \pi \to \mathcal{P}(\pi)$

\begin{mathpar}
  \inferrule* [lab=Comm] { \textsf{match}( x_{src}, x_{trgt} ) } { x_{trgt}?(y)P \; | \; x_{src}!\langle {Q} \rangle \red P\{\quotep{Q}/y}\} }
  \and \\
  \inferrule* [lab=Par] {{P} \red {P}'} {{{P} | {Q}} \red {{P}' | {Q}}}
  \and
  \inferrule* [lab=Equiv]{{{P} \scong {P}'} \andalso {{P}' \red {Q}'} \andalso {{Q}' \scong {Q}}}{{P} \red {Q}}
\end{mathpar}

\begin{eqnarray*}
  match_{\equiv} (\quotep{P},\quotep{Q}) & := & P \equiv Q \\
  match_{\dagger}(\quotep{P},\quotep{Q}) & := & \forall R. P|Q \red^{*} R => R \red^{*} 0 \\
  match_{K}(\quotep{P},\quotep{Q}) & := & K \mbox{ for some context } K
\end{eqnarray*}

$u?(x)P | u!\langle Q \rangle \red P\{\quotep{Q}/x\}$

%We write $\wred$ for $\red^*$, and $P\red$ if $\exists Q $ such that $ P \red Q$.
We write $P\red$ if $\exists Q $ such that $ P \red Q$ and $P\not\red$, otherwise.

\section{Replication}

As mentioned before, it is known that replication (and hence
recursion) can be implemented in a higher-order process algebra
\cite{SangiorgiWalker}. As our first example of calculation with the
machinery thus far presented we give the construction explicitly in
the {\rhoc}.

\begin{eqnarray}
	D_{x} & := & \prefix{x}{y}{(\binpar{\outputp{x}{y}}{@{y}})} \nonumber\\
	\bangp_{x}{P} & := & \binpar{{x}!\langle{\binpar{D_{x}}{P}}\rangle}{D_{x}} \nonumber
\end{eqnarray}

\begin{eqnarray}
	\bangp_{x}{P} & & \nonumber\\
	=
	& {x}!\langle{(\prefix{x}{y}{(\outputp{x}{y} | @{y})) | P}}\rangle 
	      | \prefix{x}{y}{(\outputp{x}{y} | @{y})} & \nonumber\\
	\red
	& (\outputp{x}{y} | @{y})\substn{\quotep{(\prefix{x}{y}{(@{y} | \outputp{x}{y})) | P}}}{y} & \nonumber\\
	=
	& \outputp{x}{\quotep{(\prefix{x}{y}{(\outputp{x}{y} | @{y})) | P}}}
	  | {(\prefix{x}{y}{(\outputp{x}{y} | @{y})) | P}} & \nonumber\\
	\red
	& \ldots & \nonumber\\
	\red^*
	& P | P | \ldots & \nonumber
\end{eqnarray}

Of course, this encoding, as an implementation, runs away, unfolding
$\bangp{P}$ eagerly. A lazier and more implementable replication
operator, restricted to input-guarded processes, may be obtained as follows.

\begin{eqnarray}
\bangp{\prefix{u}{v}{P}} 
	:= 
	\binpar{\lift{x}{\prefix{u}{v}{(\binpar{D(x)}{P})}}}{D(x)} \nonumber
\end{eqnarray}

\begin{remark}
  Note that the lazier definition still does not deal with summation
  or mixed summation (i.e. sums over input and output). The reader is
  invited to construct definitions of replication that deal with these
  features. 

  Further, the definitions are parameterized in a name, $x$. Can you,
  gentle reader, make a definition that eliminates this parameter and
  guarantees no accidental interaction between the replication
  machinery and the process being replicated -- i.e. no accidental
  sharing of names used by the process to get its work done and the
  name(s) used by the replication to effect copying. This latter
  revision of the definition of replication is crucial to obtaining
  the expected identity $!!P \sim !P$.
\end{remark}

\begin{remark}\label{rem:paradoxical_combinator}
  The reader familiar with the lambda calculus will have noticed the
  similarity between $D$ and the paradoxical combinator.

  [Ed. note: the existence of this seems to suggest we have to be more
  restrictive on the set of processes and names we admit if we are to
  support no-cloning.]
\end{remark}

\subsubsection{Bisimulation}

The computational dynamics gives rise to another kind of equivalence,
the equivalence of computational behavior. As previously mentioned
this is typically captured \emph{via} some form of bisimulation.

% The notion we use in this paper is weak barbed bisimulation
% \cite{milner91polyadicpi}.

The notion we use in this paper is derived from weak barbed
bisimulation \cite{milner91polyadicpi}. 

\begin{definition}
An \emph{observation relation}, $\downarrow_{\mathcal N}$, over a set
of names, $\mathcal N$, is the smallest relation satisfying the rules
below.

\infrule[Out-barb]{y \in {\mathcal N}, \; x \nameeq y}
		  {\outputp{x}{v} \downarrow_{\mathcal N} x}
\infrule[Par-barb]{\mbox{$P\downarrow_{\mathcal N} x$ or $Q\downarrow_{\mathcal N} x$}}
		  {\binpar{P}{Q} \downarrow_{\mathcal N} x}

We write $P \Downarrow_{\mathcal N} x$ if there is $Q$ such that 
$P \wred Q$ and $Q \downarrow_{\mathcal N} x$.
\end{definition}

\begin{definition}
%\label{def.bbisim}
An  ${\mathcal N}$-\emph{barbed bisimulation} over a set of names, ${\mathcal N}$, is a symmetric binary relation 
${\mathcal S}_{\mathcal N}$ between agents such that $P\rel{S}_{\mathcal N}Q$ implies:
\begin{enumerate}
\item If $P \red P'$ then $Q \wred Q'$ and $P'\rel{S}_{\mathcal N} Q'$.
\item If $P\downarrow_{\mathcal N} x$, then $Q\Downarrow_{\mathcal N} x$.
\end{enumerate}
$P$ is ${\mathcal N}$-barbed bisimilar to $Q$, written
$P \wbbisim_{\mathcal N} Q$, if $P \rel{S}_{\mathcal N} Q$ for some ${\mathcal N}$-barbed bisimulation ${\mathcal S}_{\mathcal N}$.
\end{definition}

$\mathcal{R} \subseteq \pi \times \pi$

$P \mathcal{R} Q => \forall P'. P \red P' \Rightarrow \exists Q'. Q \red Q', P' \mathcal{R} Q'$

$P \vdash x \Rightarrow Q \vdash x$

\begin{mathpar}
  \inferrule*[lab=Out-barb]{x \nameeq y}{{y}!\langle{Q}\rangle \vdash x}
  \and
  \inferrule*[lab=Par-barb]{\mbox{$P\vdash x$ or $Q\vdash x$}}{\binpar{P}{Q} \vdash x}
\end{mathpar}

\subsubsection{Contexts}

One of the principle advantages of computational calculi like the
$\pi$-calculus is a well-defined notion of context,
contextual-equivalence and a correlation between
contextual-equivalence and notions of bisimulation. The notion of
context allows the decomposition of a process into (sub-)process and
its syntactic environment, its context. Thus, a context may be
thought of as a process with a ``hole'' (written $\Box$) in it. The
application of a context $M$ to a process $P$, written $M[P]$, is
tantamount to filling the hole in $M$ with $P$. In this paper we do
not need the full weight of this theory, but do make use of the notion
of context in the proof the main theorem. 

\begin{mathpar}
  \inferrule* [lab=summation] {} {{M_{M},M_{N}} \bc \Box \;|\; x.M_{A} \;|\; M_{M}+M_{N}}
  \and
  \inferrule* [lab=agent] {} {{M_{A}} \bc (\vec{x})M_{P} \;| \; \clift{P_0,\ldots,M_{P},\ldots,P_N}}
  \and \\
  \inferrule* [lab=process] {} {{M_{P}} \bc M_{N} \;| \;P|M_{P} }
\end{mathpar} 

\begin{mathpar}
  \inferrule* [lab=sychronization] {} {M_{N} \bc \Box \;|\; x?M_{F} \;|\; x!M_{C}}
  \and
  \inferrule* [lab=abstraction] {} {{M_{F}} \bc (x)M_{P} }
  \and
  \inferrule* [lab=concretion] {} {{M_{C}} \bc \langle M_{P} \rangle }
  \and \\
  \inferrule* [lab=process] {} {{M_{P}} \bc M_{N} \;| \;P|M_{P} }
\end{mathpar}

\begin{definition}[contextual application] Given a context $M$, and
  process $P$, we define the \emph{contextual application}, $M[P] :=
  M\{P/\Box\}$. That is, the contextual application of M to P is the
  substitution of $P$ for $\Box$ in $M$.
\end{definition}

$\meaningof{-} : L \to \mathcal{P}(\pi)$

\begin{mathpar}
  \inferrule* [lab=collection] {} {\meaningof{true} = \pi, \and \meaningof{~E} = \pi \setminus \meaningof{E}, \and \meaningof{E_{1} \& E_{2}} = \meaningof{E_{1}} \cap \meaningof{E_{2}}}
\end{mathpar}

\begin{mathpar}
  \inferrule* [lab=structure] {} {\meaningof{0} = \{ P \in \pi | P \equiv 0 \}, \and \\ \meaningof{E_1 | E_2} = \{ P \in \pi | P \equiv P_{1} | P_{2}, P_{1} \in \meaningof{E_{1}}, P_{2} \in \meaningof{E_2}\} }
\end{mathpar}

\begin{mathpar}
 \inferrule* [lab=behavior] {} {\meaningof{\langle a?b \rangle E} = \{ P \in \pi | P \equiv Q | u?(y)P', \\ \and \\\\ \and \\ \;\;\; u \in \meaningof{a}, \forall z.P'\{z/y\} \in \meaningof{E\{z/b\}}\}, \and \\ \meaningof{a!E} = \{ P \in \pi | P \equiv Q | x!\langle P' \rangle, x \in \meaningof{a} P' \in \meaningof{E}\} }
\end{mathpar}

\begin{mathpar}
 \inferrule* [lab=nominal] {} {\meaningof{\quotep{E}} = \{ \quotep{P} \in \quotep{\pi} | P \in \meaningof{E} \}, \and \meaningof{\quotep{P}} = \{ \quotep{Q} \in \quotep{\pi} | P \equiv Q \} \and \\ \meaningof{@\quotep{E}} = \{ P \in \pi | P \equiv @x, x \in \meaningof{E} \}}
\end{mathpar}

\begin{eqnarray*}
  \\
  \meaningof{-} : TS \to ST
\end{eqnarray*}

\begin{eqnarray*}
  \\
  L : TS \to ST
\end{eqnarray*}

\begin{eqnarray*}
  \\
  P \models E \iff P \in \meaningof{E}
\end{eqnarray*}

\begin{eqnarray*}
  P \approx_{L} Q \iff \forall E \in L. P \models E \iff Q \models E
\end{eqnarray*}

\begin{eqnarray*}
  P \approx_{K} Q
\end{eqnarray*}

\begin{eqnarray*}
  P \approx Q
\end{eqnarray*}

$\approx_{K} = \approx = \approx_{L}$

\subsubsection{Contextual duality}

Note that contexts extend the quotation operation to a family of
operations from processes to names. Given a context, $M$, we can
define a \emph{nominal context}, $\quotep{M}$ by $\quotep{M}[P] :=
\quotep{M[P]}$. To foreshadow what is to come we observe that these
operations enjoy a duality with processes very much like the duality
between vectors and maps from vectors to scalars.

Further, because the calculus is essentially higher-order, we have a
correspondence between contexts and processes. More specifically,
given a name $x$ and a context $M$ we can construct $M^{*}_{x}$ such
that 

\begin{mathpar}
  M^{*}_{x} | \lift{x}{P} \red M[P]
\end{mathpar}

namely,

\begin{mathpar}
  M^{*}_{x} := x?(u).M[\dropn{u}]
\end{mathpar}

The dependence of $M^{*}_{x}$ on a name makes it an abstraction, 

\begin{mathpar}
  M^{*} := (x)x?(u).M[\dropn{u}]
\end{mathpar}

\subsection{Additional notation}

It will sometimes be convenient to denote the process a name
quotes. We already have the notation $x = \quotep{P}$, but it will be
convenient to introduce an alternate notation, $\procn{x}$, when we
want to emphasize the connection to the use of the name. Note that, by
virtue of name equivalence, $\quotep{\procn{x}} \nameeq x$; so, the
notation is consistent with previous definitions.

Further, because names have structure it is possible to effect
substitutions on the basis of that structure. This means we need to
upgrade our notation for substitutions, which we accomplish by
adapting comprehension notation. Thus,

\begin{mathpar}
  P\{ y / x : x \in S \}
\end{mathpar}

is interpreted to mean the process derived from P by replacing (in a
capture-avoiding manner) each occurrence of $x$ in $S$ by $y$. For example,

\begin{mathpar}
  P\{ \quotep{\procn{x}|\procn{x}} / x : x \in \freenames{P} \}
\end{mathpar}

will replace each (occurrence) of a free name $x$ in $P$ by
$\quotep{\procn{x}|\procn{x}}$.

Also, we will avail ourselves of the notation $x^{L}$ and $x^{R}$ to
denote injections of a name into disjoint copies of the name
space. There are numerous ways to accomplish this. One example can be
found in \cite{MeredithR05}. This notation overloads to vectors of
names: $\vec{x}^{\pi} := (x_{i}^{\pi} \; : \; 0 \leq i < |\vec{x}| )$ where $\pi \in \{L,R\}$.

We also use $P^{\Box} := P|\Box$.

In \cite{MeredithR05} an interpretation of the new operator is
given. It turns out that there are several possible interpretations
all enjoying the requisite algebraic properties of the operator (see
\cite{milner91polyadicpi}). We will therefore make liberal use of
$(\nu\; \vec{x})P$.

% subsection the_syntax_and_semantics_of_the_notation_system (end)   

\input{qm2pi.qmops} 

\input{qm2pi.sterngerlach} 

\input{qm2pi.metric} 

% section concurrent_process_calculi (end)

%\input{qm2pi.proofsketch}

% section proof sketch (end)

%\input{qm2pi.slviaknots} 

% section spatial logic via knots (end)

\input{qm2pi.conclusion}

% section conclusion (end)

%\input{qm2pi.dtcodes} 

% section wiring algorithm (end)

\input{qm2pi.ack} 

% section acknowledgments (end)

\newpage


\bibliographystyle{plain}   
\bibliography{../../biblios/main.bib}

\input{qm2pi.rhodetails}

\end{document}

 

%\ifpdf
%\usepackage[pdftex]{graphicx}
%\else
%\usepackage{graphicx}
%\fi

 % \ifpdf
%  \usepackage{pdfsync}
%  \if


%\title{Brief Article}
%\author{David F. Snyder}
%\author{L.G. Meredith}

%\address{Dept. of Math., Texas State University--San Marcos, San Marcos, TX 78666}
       
\pagestyle{empty}


\begin{document}

\lstset{language=[Objective]Caml,frame=shadowbox}

\documentclass[12pt]{llncs}
%\documentclass{jktr}

\usepackage[pdftex]{hyperref}                   
\usepackage {listings}
\usepackage {mathpartir}
\usepackage{bcprules}
%\usepackage{listings}
                       
\usepackage{graphicx} 
%\usepackage[margins=2.5cm,nohead,nofoot]{geometry}
%\usepackage{geometry}
\usepackage{amsfonts}
\usepackage{amstext}
\usepackage{latexsym}
\usepackage{amssymb}
\usepackage{color}


%\include{myPreamble}
\include{qm2pi.local} 

%\ifpdf
%\usepackage[pdftex]{graphicx}
%\else
%\usepackage{graphicx}
%\fi

 % \ifpdf
%  \usepackage{pdfsync}
%  \if


%\title{Brief Article}
%\author{David F. Snyder}
%\author{L.G. Meredith}

%\address{Dept. of Math., Texas State University--San Marcos, San Marcos, TX 78666}
       
\pagestyle{empty}


\begin{document}

\lstset{language=[Objective]Caml,frame=shadowbox}

\input{qm2pi.front}

% section front matter (end)

\input{qm2pi.intro} 
 
% section introduction (end)

% \input{qm2pi.knotations} 

% section notation (end)

\input{qm2pi.process.calculi} 

% section concurrent_process_calculi_and_spatial_logics_ (end)
    
%\input{qm2pi.knots2pi} 

%\input{qm2pi.trefoil} 

%\input{qm2pi.mainthm} 

% subsection basic_interpretation (end)

%\input{qm2pi.rho.presentation} 
\subsection{The syntax and semantics of the notation system}\label{sub:the_syntax_and_semantics_of_the_notation_system} % (fold)

We now summarize a technical presentation of the calculus that
embodies our theory of dynamics. The typical presentation of such a
calculus follows the style of giving generators and relations on
them. The grammar, below, describing term constructors, freely
generates the set of processes, $\Proc$. This set is then quotiented
by a relation known as structural congruence and it is over this set
that the notion of dynamics is expressed. This presentation is
essentially that of \cite{MeredithR05} with the addition of
polyadicity and summation. For readability we have relegated some of
the technical subtleties to an appendix.

\subsubsection{Process grammar}\label{subsub:process_grammar}

\begin{mathpar}
  \inferrule* [lab=synchronization] {} {{M} \bc \pzero \;|\; x?F \;|\; x!C }
  \and
  \inferrule* [lab=abstraction] {} {{F} \bc (x)P}
  \and
  \inferrule* [lab=concretion] {} {{C} \bc \langle Q \rangle}
  \and
  \inferrule* [lab=process] {} {{P,Q} \bc M \;| \;P|Q \;|\; @{x}}
  \and
  \inferrule* [lab=name] {} {{x} \bc \quotep{P}}
\end{mathpar} 

Note that $\vec{x}$ (resp. $\vec{P}$) denotes a vector of names
(resp. processes) of length $|\vec{x}|$ (resp. $|\vec{P}|$). We adopt
the following useful abbreviations.

\begin{mathpar}
   x?(\vec{y}).P := x.(\vec{y})P \and  x\clift{\vec{P}} := x.\clift{\vec{P}}
   \and x!(y) := \lift{x}{\dropn{y}}
   \and \Pi_{i=0}^{n-1}P_i := P_0 | \ldots | P_{n-1}
\end{mathpar}

\subsubsection{Structural congruence}

\paragraph{Free and bound names and alpha-equivalence.} At the
core of structural equivalence is alpha-equivalence which identifies
process that are the same up to a change of variable. Formally, we
recognize the distinction between free and bound names. The free names
of a process, $\freenames{P}$, may be calculated recursively as
follows:

\begin{mathpar}
\freenames{\pzero} := \emptyset
  \and \\
  \freenames{x?(y).P} := \{ x \} \cup (\freenames{P} \setminus \{ y \})
  \and 
  \freenames{x!\langle P \rangle} := \{ x \} \cup \{ P \} 
  \and \\
  \freenames{P|Q} := \freenames{P} \cup \freenames{Q}
  \and \\
  \freenames{@{x}} := \{ x \}
\end{mathpar}

$\pi$
$\quotep{\pi}$

$\freenames{-} : \pi \to \mathcal{P}(\quotep{\pi})$

\begin{eqnarray*}
  \freenames{\pzero} & := & \emptyset \\
  \freenames{x?(y).P} & := & \{ x \} \cup (\freenames{P} \setminus \{ y \}) \\
  \freenames{x!\langle P \rangle} & := & \{ x \} \cup \{ P \} \\
  \freenames{P|Q} & := & \freenames{P} \cup \freenames{Q} \\
  \freenames{\dropn{x}} & := & \{ x \}
\end{eqnarray*}

The bound names of a process, $\boundnames{P}$, are those names occurring in $P$
that are not free. For example, in $x?(y).0$, the name $x$ is free, while $y$ is bound.

\begin{mathpar}
  \inferrule* [lab=monoidal-laws] {} { P|Q \equiv Q|P \and P|0 \equiv P \and P|(Q|R) \equiv (P|Q)|R }
\end{mathpar}

\begin{mathpar}
  \inferrule* [lab=alpha-equivalence] {} { (x)P \equiv (y)P\{y/x\} \and y \not\in \freenames{P} }
\end{mathpar}

\begin{definition}
Then two processes, $P,Q$, are alpha-equivalent if $P = Q\{\vec{y}/\vec{x}\}$ for
some $\vec{x} \in \boundnames{Q},\vec{y} \in \boundnames{P}$, where $Q\{\vec{y}/\vec{x}\}$
denotes the capture-avoiding substitution of $\vec{y}$ for $\vec{x}$ in $Q$.
\end{definition}

\begin{definition}
  The {\em structural congruence} \cite{SangiorgiWalker} , $\equiv$,
  between processes is the least congruence containing
  alpha-equivalence, satisfying the abelian monoid laws
  (associativity, commutativity and $\pzero$ as identity) for parallel
  composition $|$ and for summation $+$.
\end{definition}

\subsection{Name equivalence}

We take name equivalence, written $\nameeq$, to be the smallest
equivalence relation generated by the following rules.

\begin{mathpar}
\inferrule*[lab=Quote-drop]
{ }
{ \quotep{@{x}} \nameeq x }

\inferrule*[lab=Struct-equiv]
{ P \scong Q }
{ \quotep{P} \nameeq \quotep{Q} }
\end{mathpar}

The astute reader will have noticed that the mutual recursion of names
and processes imposes a mutual recursion on alpha-equivalence and
structural equivalence via name-equivalence. Fortunately, all of this
works out pleasantly and we may calculate in the natural way, free of
concern. The reader interested in the details is referred to the
appendix \ref{appendix:rho_details}.

\subsection{Substitution}

We use $\Proc$ for the set of processes, $\QProc$ for the set of
names, and $\id{\{}\vec{y} / \vec{x} \id{\}}$ to denote partial maps,
$s : \QProc \rightarrow \QProc$. A map, $s$ lifts, uniquely, to a map
on process terms, $\widehat{s} : \Proc \rightarrow \Proc$ by the
following equations.

\begin{mathpar}
  (0) \psubstp{Q}{P} := 0 \\
  (R \juxtap S) \psubstp{Q}{P}
  :=    
  (R)\psubstp{Q}{P} \juxtap (S) \psubstp{Q}{P} \\
  (x?(y).R) \psubstp{Q}{P}    
  :=    
  (x)\substp{Q}{P} (z)\concat( (R \psubstn{z}{y}) \psubstp{Q}{P} ) \\
  (\lift{x}{R}) \psubstp{Q}{P}  
  :=
  \lift{(x)\substp{Q}{P}}{ R \psubstp{Q}{P} } \\
%   (\dropn{x})  \psubstp{Q}{P}       
%   := 
%   \left\{ 
%     \begin{array}{ccc} 
%       \dropn{\quotep{Q}} & & x \nameeq \quotep{P} \\
%       \dropn{x} & & otherwise \\
%     \end{array}
%   \right. 
  (\dropn{x})  \psubstp{Q}{P}       
  := 
  \left\{ 
    \begin{array}{ccc} 
      Q & & x \nameeq \quotep{P} \\
      \dropn{x} & & otherwise \\
    \end{array}
  \right.
\end{mathpar}
 

where

\begin{eqnarray}
  (x)\id{\{} \lpquote Q \rpquote / \lpquote P \rpquote \id{\}}            = 
  \left\{ 
    \begin{array}{ccc}
      \lpquote Q \rpquote & & x \nameeq \lpquote P \rpquote \\
      x & & otherwise \\
    \end{array}
  \right. \nonumber
\end{eqnarray}

and $z$ is chosen distinct from $\quotep{P}$, $\quotep{Q}$, the free
names in $Q$, and all the names in $R$. Our $\alpha$-equivalence will
be built in the standard way from this substitution.

\begin{remark}\label{rem:no_self_referential_names}
  One consequence of these definitions is that $\forall P. \quotep{P}
  \not\in \freenames{P}$.
\end{remark}

\subsection{ Dynamic quote: an example }

Anticipating something of what's to come, consider applying the
substitution, $\widehat{\id{\{}u / z \id{\}}}$, to the following pair
of processes, $\lift{w}{y!(z)}$ and $w[ \lpquote y!(z) \rpquote ]$.

\begin{eqnarray}
	\lift{w}{y!(z)}\widehat{\id{\{}u / z \id{\}}}
		& = &
		\lift{w}{y!(u)} \nonumber\\
	w[ \lpquote y!(z) \rpquote ] \widehat{ \id{\{}u / z \id{\}} }
		& = &
		w[ \lpquote y!(z) \rpquote ] \nonumber
\end{eqnarray}

Because the body of the process between quotes is impervious to
substitution, we get radically different answers. In fact, by
examining the first process in an input context,
e.g. $x?(z).\lift{w}{y!(z)}$, we see that the process under the lift
operator may be shaped by prefixed inputs binding a name inside it. In
this sense, the lift operator will be seen as a way to dynamically
construct processes before reifying them as names.

Finally equipped with these standard features we can present the
dynamics of the calculus.

\subsubsection{Operational semantics} 

Finally, we introduce the computational dynamics. What marks these
algebras as distinct from other more traditionally studied algebraic
structures, e.g. vector spaces or polynomial rings, is the manner in
which dynamics is captured. In traditional structures, dynamics is typically
expressed through morphisms between such structures, as in linear maps
between vector spaces or morphisms between rings. In algebras
associated with the semantics of computation, the dynamics is
expressed as part of the algebraic structure itself, through a
reduction reduction relation typically denoted by $\red$. Below, we
give a recursive presentation of this relation for the calculus used
in the encoding.

$\red \subseteq \pi \times \pi$
$\red : \pi \to \mathcal{P}(\pi)$

\begin{mathpar}
  \inferrule* [lab=Comm] { \textsf{match}( x_{src}, x_{trgt} ) } { x_{trgt}?(y)P \; | \; x_{src}!\langle {Q} \rangle \red P\{\quotep{Q}/y}\} }
  \and \\
  \inferrule* [lab=Par] {{P} \red {P}'} {{{P} | {Q}} \red {{P}' | {Q}}}
  \and
  \inferrule* [lab=Equiv]{{{P} \scong {P}'} \andalso {{P}' \red {Q}'} \andalso {{Q}' \scong {Q}}}{{P} \red {Q}}
\end{mathpar}

\begin{eqnarray*}
  match_{\equiv} (\quotep{P},\quotep{Q}) & := & P \equiv Q \\
  match_{\dagger}(\quotep{P},\quotep{Q}) & := & \forall R. P|Q \red^{*} R => R \red^{*} 0 \\
  match_{K}(\quotep{P},\quotep{Q}) & := & K \mbox{ for some context } K
\end{eqnarray*}

$u?(x)P | u!\langle Q \rangle \red P\{\quotep{Q}/x\}$

%We write $\wred$ for $\red^*$, and $P\red$ if $\exists Q $ such that $ P \red Q$.
We write $P\red$ if $\exists Q $ such that $ P \red Q$ and $P\not\red$, otherwise.

\section{Replication}

As mentioned before, it is known that replication (and hence
recursion) can be implemented in a higher-order process algebra
\cite{SangiorgiWalker}. As our first example of calculation with the
machinery thus far presented we give the construction explicitly in
the {\rhoc}.

\begin{eqnarray}
	D_{x} & := & \prefix{x}{y}{(\binpar{\outputp{x}{y}}{@{y}})} \nonumber\\
	\bangp_{x}{P} & := & \binpar{{x}!\langle{\binpar{D_{x}}{P}}\rangle}{D_{x}} \nonumber
\end{eqnarray}

\begin{eqnarray}
	\bangp_{x}{P} & & \nonumber\\
	=
	& {x}!\langle{(\prefix{x}{y}{(\outputp{x}{y} | @{y})) | P}}\rangle 
	      | \prefix{x}{y}{(\outputp{x}{y} | @{y})} & \nonumber\\
	\red
	& (\outputp{x}{y} | @{y})\substn{\quotep{(\prefix{x}{y}{(@{y} | \outputp{x}{y})) | P}}}{y} & \nonumber\\
	=
	& \outputp{x}{\quotep{(\prefix{x}{y}{(\outputp{x}{y} | @{y})) | P}}}
	  | {(\prefix{x}{y}{(\outputp{x}{y} | @{y})) | P}} & \nonumber\\
	\red
	& \ldots & \nonumber\\
	\red^*
	& P | P | \ldots & \nonumber
\end{eqnarray}

Of course, this encoding, as an implementation, runs away, unfolding
$\bangp{P}$ eagerly. A lazier and more implementable replication
operator, restricted to input-guarded processes, may be obtained as follows.

\begin{eqnarray}
\bangp{\prefix{u}{v}{P}} 
	:= 
	\binpar{\lift{x}{\prefix{u}{v}{(\binpar{D(x)}{P})}}}{D(x)} \nonumber
\end{eqnarray}

\begin{remark}
  Note that the lazier definition still does not deal with summation
  or mixed summation (i.e. sums over input and output). The reader is
  invited to construct definitions of replication that deal with these
  features. 

  Further, the definitions are parameterized in a name, $x$. Can you,
  gentle reader, make a definition that eliminates this parameter and
  guarantees no accidental interaction between the replication
  machinery and the process being replicated -- i.e. no accidental
  sharing of names used by the process to get its work done and the
  name(s) used by the replication to effect copying. This latter
  revision of the definition of replication is crucial to obtaining
  the expected identity $!!P \sim !P$.
\end{remark}

\begin{remark}\label{rem:paradoxical_combinator}
  The reader familiar with the lambda calculus will have noticed the
  similarity between $D$ and the paradoxical combinator.

  [Ed. note: the existence of this seems to suggest we have to be more
  restrictive on the set of processes and names we admit if we are to
  support no-cloning.]
\end{remark}

\subsubsection{Bisimulation}

The computational dynamics gives rise to another kind of equivalence,
the equivalence of computational behavior. As previously mentioned
this is typically captured \emph{via} some form of bisimulation.

% The notion we use in this paper is weak barbed bisimulation
% \cite{milner91polyadicpi}.

The notion we use in this paper is derived from weak barbed
bisimulation \cite{milner91polyadicpi}. 

\begin{definition}
An \emph{observation relation}, $\downarrow_{\mathcal N}$, over a set
of names, $\mathcal N$, is the smallest relation satisfying the rules
below.

\infrule[Out-barb]{y \in {\mathcal N}, \; x \nameeq y}
		  {\outputp{x}{v} \downarrow_{\mathcal N} x}
\infrule[Par-barb]{\mbox{$P\downarrow_{\mathcal N} x$ or $Q\downarrow_{\mathcal N} x$}}
		  {\binpar{P}{Q} \downarrow_{\mathcal N} x}

We write $P \Downarrow_{\mathcal N} x$ if there is $Q$ such that 
$P \wred Q$ and $Q \downarrow_{\mathcal N} x$.
\end{definition}

\begin{definition}
%\label{def.bbisim}
An  ${\mathcal N}$-\emph{barbed bisimulation} over a set of names, ${\mathcal N}$, is a symmetric binary relation 
${\mathcal S}_{\mathcal N}$ between agents such that $P\rel{S}_{\mathcal N}Q$ implies:
\begin{enumerate}
\item If $P \red P'$ then $Q \wred Q'$ and $P'\rel{S}_{\mathcal N} Q'$.
\item If $P\downarrow_{\mathcal N} x$, then $Q\Downarrow_{\mathcal N} x$.
\end{enumerate}
$P$ is ${\mathcal N}$-barbed bisimilar to $Q$, written
$P \wbbisim_{\mathcal N} Q$, if $P \rel{S}_{\mathcal N} Q$ for some ${\mathcal N}$-barbed bisimulation ${\mathcal S}_{\mathcal N}$.
\end{definition}

$\mathcal{R} \subseteq \pi \times \pi$

$P \mathcal{R} Q => \forall P'. P \red P' \Rightarrow \exists Q'. Q \red Q', P' \mathcal{R} Q'$

$P \vdash x \Rightarrow Q \vdash x$

\begin{mathpar}
  \inferrule*[lab=Out-barb]{x \nameeq y}{{y}!\langle{Q}\rangle \vdash x}
  \and
  \inferrule*[lab=Par-barb]{\mbox{$P\vdash x$ or $Q\vdash x$}}{\binpar{P}{Q} \vdash x}
\end{mathpar}

\subsubsection{Contexts}

One of the principle advantages of computational calculi like the
$\pi$-calculus is a well-defined notion of context,
contextual-equivalence and a correlation between
contextual-equivalence and notions of bisimulation. The notion of
context allows the decomposition of a process into (sub-)process and
its syntactic environment, its context. Thus, a context may be
thought of as a process with a ``hole'' (written $\Box$) in it. The
application of a context $M$ to a process $P$, written $M[P]$, is
tantamount to filling the hole in $M$ with $P$. In this paper we do
not need the full weight of this theory, but do make use of the notion
of context in the proof the main theorem. 

\begin{mathpar}
  \inferrule* [lab=summation] {} {{M_{M},M_{N}} \bc \Box \;|\; x.M_{A} \;|\; M_{M}+M_{N}}
  \and
  \inferrule* [lab=agent] {} {{M_{A}} \bc (\vec{x})M_{P} \;| \; \clift{P_0,\ldots,M_{P},\ldots,P_N}}
  \and \\
  \inferrule* [lab=process] {} {{M_{P}} \bc M_{N} \;| \;P|M_{P} }
\end{mathpar} 

\begin{mathpar}
  \inferrule* [lab=sychronization] {} {M_{N} \bc \Box \;|\; x?M_{F} \;|\; x!M_{C}}
  \and
  \inferrule* [lab=abstraction] {} {{M_{F}} \bc (x)M_{P} }
  \and
  \inferrule* [lab=concretion] {} {{M_{C}} \bc \langle M_{P} \rangle }
  \and \\
  \inferrule* [lab=process] {} {{M_{P}} \bc M_{N} \;| \;P|M_{P} }
\end{mathpar}

\begin{definition}[contextual application] Given a context $M$, and
  process $P$, we define the \emph{contextual application}, $M[P] :=
  M\{P/\Box\}$. That is, the contextual application of M to P is the
  substitution of $P$ for $\Box$ in $M$.
\end{definition}

$\meaningof{-} : L \to \mathcal{P}(\pi)$

\begin{mathpar}
  \inferrule* [lab=collection] {} {\meaningof{true} = \pi, \and \meaningof{~E} = \pi \setminus \meaningof{E}, \and \meaningof{E_{1} \& E_{2}} = \meaningof{E_{1}} \cap \meaningof{E_{2}}}
\end{mathpar}

\begin{mathpar}
  \inferrule* [lab=structure] {} {\meaningof{0} = \{ P \in \pi | P \equiv 0 \}, \and \\ \meaningof{E_1 | E_2} = \{ P \in \pi | P \equiv P_{1} | P_{2}, P_{1} \in \meaningof{E_{1}}, P_{2} \in \meaningof{E_2}\} }
\end{mathpar}

\begin{mathpar}
 \inferrule* [lab=behavior] {} {\meaningof{\langle a?b \rangle E} = \{ P \in \pi | P \equiv Q | u?(y)P', \\ \and \\\\ \and \\ \;\;\; u \in \meaningof{a}, \forall z.P'\{z/y\} \in \meaningof{E\{z/b\}}\}, \and \\ \meaningof{a!E} = \{ P \in \pi | P \equiv Q | x!\langle P' \rangle, x \in \meaningof{a} P' \in \meaningof{E}\} }
\end{mathpar}

\begin{mathpar}
 \inferrule* [lab=nominal] {} {\meaningof{\quotep{E}} = \{ \quotep{P} \in \quotep{\pi} | P \in \meaningof{E} \}, \and \meaningof{\quotep{P}} = \{ \quotep{Q} \in \quotep{\pi} | P \equiv Q \} \and \\ \meaningof{@\quotep{E}} = \{ P \in \pi | P \equiv @x, x \in \meaningof{E} \}}
\end{mathpar}

\begin{eqnarray*}
  \\
  \meaningof{-} : TS \to ST
\end{eqnarray*}

\begin{eqnarray*}
  \\
  L : TS \to ST
\end{eqnarray*}

\begin{eqnarray*}
  \\
  P \models E \iff P \in \meaningof{E}
\end{eqnarray*}

\begin{eqnarray*}
  P \approx_{L} Q \iff \forall E \in L. P \models E \iff Q \models E
\end{eqnarray*}

\begin{eqnarray*}
  P \approx_{K} Q
\end{eqnarray*}

\begin{eqnarray*}
  P \approx Q
\end{eqnarray*}

$\approx_{K} = \approx = \approx_{L}$

\subsubsection{Contextual duality}

Note that contexts extend the quotation operation to a family of
operations from processes to names. Given a context, $M$, we can
define a \emph{nominal context}, $\quotep{M}$ by $\quotep{M}[P] :=
\quotep{M[P]}$. To foreshadow what is to come we observe that these
operations enjoy a duality with processes very much like the duality
between vectors and maps from vectors to scalars.

Further, because the calculus is essentially higher-order, we have a
correspondence between contexts and processes. More specifically,
given a name $x$ and a context $M$ we can construct $M^{*}_{x}$ such
that 

\begin{mathpar}
  M^{*}_{x} | \lift{x}{P} \red M[P]
\end{mathpar}

namely,

\begin{mathpar}
  M^{*}_{x} := x?(u).M[\dropn{u}]
\end{mathpar}

The dependence of $M^{*}_{x}$ on a name makes it an abstraction, 

\begin{mathpar}
  M^{*} := (x)x?(u).M[\dropn{u}]
\end{mathpar}

\subsection{Additional notation}

It will sometimes be convenient to denote the process a name
quotes. We already have the notation $x = \quotep{P}$, but it will be
convenient to introduce an alternate notation, $\procn{x}$, when we
want to emphasize the connection to the use of the name. Note that, by
virtue of name equivalence, $\quotep{\procn{x}} \nameeq x$; so, the
notation is consistent with previous definitions.

Further, because names have structure it is possible to effect
substitutions on the basis of that structure. This means we need to
upgrade our notation for substitutions, which we accomplish by
adapting comprehension notation. Thus,

\begin{mathpar}
  P\{ y / x : x \in S \}
\end{mathpar}

is interpreted to mean the process derived from P by replacing (in a
capture-avoiding manner) each occurrence of $x$ in $S$ by $y$. For example,

\begin{mathpar}
  P\{ \quotep{\procn{x}|\procn{x}} / x : x \in \freenames{P} \}
\end{mathpar}

will replace each (occurrence) of a free name $x$ in $P$ by
$\quotep{\procn{x}|\procn{x}}$.

Also, we will avail ourselves of the notation $x^{L}$ and $x^{R}$ to
denote injections of a name into disjoint copies of the name
space. There are numerous ways to accomplish this. One example can be
found in \cite{MeredithR05}. This notation overloads to vectors of
names: $\vec{x}^{\pi} := (x_{i}^{\pi} \; : \; 0 \leq i < |\vec{x}| )$ where $\pi \in \{L,R\}$.

We also use $P^{\Box} := P|\Box$.

In \cite{MeredithR05} an interpretation of the new operator is
given. It turns out that there are several possible interpretations
all enjoying the requisite algebraic properties of the operator (see
\cite{milner91polyadicpi}). We will therefore make liberal use of
$(\nu\; \vec{x})P$.

% subsection the_syntax_and_semantics_of_the_notation_system (end)   

\input{qm2pi.qmops} 

\input{qm2pi.sterngerlach} 

\input{qm2pi.metric} 

% section concurrent_process_calculi (end)

%\input{qm2pi.proofsketch}

% section proof sketch (end)

%\input{qm2pi.slviaknots} 

% section spatial logic via knots (end)

\input{qm2pi.conclusion}

% section conclusion (end)

%\input{qm2pi.dtcodes} 

% section wiring algorithm (end)

\input{qm2pi.ack} 

% section acknowledgments (end)

\newpage


\bibliographystyle{plain}   
\bibliography{../../biblios/main.bib}

\input{qm2pi.rhodetails}

\end{document}



% section front matter (end)

\section{Introduction}\label{sec:introduction} % (fold)
In this draft of the material i am going to have to dispense with the
usual writing conventions adopted in papers on these topics. i'm going
to have adopt whatever tone i need at the time i'm writing up the
calculations. Sometimes this may be very conversational; others it may
be the barest mathematical grunts; others still it may be that i have
lifted text from one of my other papers because the exposition of some
point was better said there. i hope that my readers are not unduly put
out by this decision. i'm not doing this to flout convention or be
rebellious. i find these calculations very technically challenging. To
keep everything going technically, something has to give; i have to
let go of some cognitive burden. So, the academic writing style --
with all of its trade-offs in terms of facilitating technical
communication -- is what i'm letting go of. Perhaps subsequent drafts
can be tightened and polished, but for now, i'm going to speak as if
we were sitting together in a coffee shop with a laptop, wifi and a
pad of paper and a pencil.

So, here's what i have to say. We -- you and i, comfortably ensconced
in our coffee shop and well-equipped with our tools -- can realize and
carry out the calculations of quantum mechanics over a very different
formal theory of dynamics, a formal theory of dynamics that
corresponds to a theory of concurrent computation with
\emph{reflection}. It has the advantage that the underlying theory is
already `quantized', but supports analogues all of the continuuous
operations. Strikingly, this underlying theory has recently been
connected with a notion of metric that we can show, by calculating
together, coincides with the metric induced by the inner product.

There are a lot of reasons why you might be interested in seeing
calculations of this form. Here's why i'm interested. For the past
several centuries there has been no competitor to the ``Newtonian''
account of dynamics. As a result the predominant share of accounts of
dynamical systems and situations have had to be formulated in terms of
the Newtonian machinery. i view this as an intellectually dangerous
position to occupy. Everything, despite it's intrinsic shape, turns
into a nail to be hit with this hammer. Recently, however, the theory
of computation has matured to the point where we have candidates for
theories of dynamics that offer very different perspective on
reasoning about dynamical systems and situations. Testing these
candidates against very successful accounts of dynamical situations,
like quantum mechanics, is going to give us some sense of how mature
they are and some measure of the quality of these accounts of
dynamics.

\subsection{Summary of contributions and outline of paper}

So, we're going to develop an interpretation of the operations of
quantum mechanics normally interpreted by Hilbert spaces and
operators. We're going to do this over a theory of computation. Note
that this is very different than the usual quantum computation program
which develops notions of computation over quantum mechanics. Rather,
we are developing a story that aligns with Wheeler's slogan: It from
Bit. To do this we will first provide an account of the theory of
computation at play here. Then we will dive into a calculation-driven
interpretation of the operations of quantum mechanics.

The reason we take this approach is that -- until very recently --
there hasn't been an axiomatic account of quantum mechanics. As a
result there has been no sharp delineation of the mathematical theory
supporting interpretation of the physical theory and the physical
theory, itself. So, ambient features of the maths are free to be
exploited (or supressed) without a real accounting of their physical
relevance. There is no sharp statement ``here's the physical theory''
qua \emph{theory} and ``here's the mathematical interpretation''
enabling a judgment of how faithful the interpretation is -- apart
from experimental observation. When there is an axiomatic account we
can judge how well a given mathematical formalism supports an
interpretation of the axioms, independent of
experimentation. Likewise, we can judge how well we have captured our
physical evidence and experience with our axiomatics, independent of
any specific mathematical implementation, with accidental detail that
may or may not have physical significance. 

In lieu of a fully fleshed out and vetted axiomatic account of quantum
mechanics, interpreting the operational notions in service of modeling
physical systems will have to suffice. In other words, we are not in
the business of providing a model of Hilbert spaces and operators. We
are in the business of providing a model of quantum mechanics because
we are motivated by testing our notions of dynamics against physical
theory; and, the predictive calculations of the physical theory must
serve as the best formulation -- shy of a fully fleshed out axiomatic
account -- of the physical theory itself (as they have for scientific
theories since time immemorial). Put another way, despite a
whole-hearted commitment to an It-from-Bit ontology, we are firmly
aligned with the shut-up-and-calculate camp as the best way to obtain
results either from the physical perspective or as a quality assurance
measure of our fledgling theory of dynamics.

In detail, we present a reflective process calculus. Then we develop
intuitive correspondences between the notions available in this
calculus and the usual physical notions supporting quantum mechanical
calculations. Thus, 

\begin{table}[htp]
  \center{
    \fbox{
      \begin{tabular}{c|c}
        quantum mechanics & process calculus \\
        \hline
        scalar & name \\
        state vector & process \\
        dual & contextual duals \\
        matrix & formal sums of process-context-dual pairs \\
        orthogonality & process annihilation \\
        inner product & execution-formula + quoting
      \end{tabular}
    }
  }
  \caption{QM - process calculi correspondences}
\end{table}

Then we tighten up these intuitions to operational definitions. We
employ the Dirac notation as the best proxy we can find for an
abstract syntax of the quantum mechanical notions. The definitions we
develop put us in contact with equational constraints coming from the
theory that we demonstrate the definitions and calculations satisfy.

This puts us in a position to shut up and calculate for the
Stern-Gerlach experimental set up, showing how these predictive
calculations become calculations on processes in our theory of a
reflective process calculus.

Penultimately, we demonstrate that the notion of metric coming from
the inner product coincides with the notion of metric available from
the theory of bisimulation. This demonstration gives us the right to
think of space as arising from behavior. Finally, we consider where we
might go from the new vantage point we have obtained.

% section introduction (end) 
 
% section introduction (end)

% \documentclass[12pt]{llncs}
%\documentclass{jktr}

\usepackage[pdftex]{hyperref}                   
\usepackage {listings}
\usepackage {mathpartir}
\usepackage{bcprules}
%\usepackage{listings}
                       
\usepackage{graphicx} 
%\usepackage[margins=2.5cm,nohead,nofoot]{geometry}
%\usepackage{geometry}
\usepackage{amsfonts}
\usepackage{amstext}
\usepackage{latexsym}
\usepackage{amssymb}
\usepackage{color}


%\include{myPreamble}
\include{qm2pi.local} 

%\ifpdf
%\usepackage[pdftex]{graphicx}
%\else
%\usepackage{graphicx}
%\fi

 % \ifpdf
%  \usepackage{pdfsync}
%  \if


%\title{Brief Article}
%\author{David F. Snyder}
%\author{L.G. Meredith}

%\address{Dept. of Math., Texas State University--San Marcos, San Marcos, TX 78666}
       
\pagestyle{empty}


\begin{document}

\lstset{language=[Objective]Caml,frame=shadowbox}

\input{qm2pi.front}

% section front matter (end)

\input{qm2pi.intro} 
 
% section introduction (end)

% \input{qm2pi.knotations} 

% section notation (end)

\input{qm2pi.process.calculi} 

% section concurrent_process_calculi_and_spatial_logics_ (end)
    
%\input{qm2pi.knots2pi} 

%\input{qm2pi.trefoil} 

%\input{qm2pi.mainthm} 

% subsection basic_interpretation (end)

%\input{qm2pi.rho.presentation} 
\subsection{The syntax and semantics of the notation system}\label{sub:the_syntax_and_semantics_of_the_notation_system} % (fold)

We now summarize a technical presentation of the calculus that
embodies our theory of dynamics. The typical presentation of such a
calculus follows the style of giving generators and relations on
them. The grammar, below, describing term constructors, freely
generates the set of processes, $\Proc$. This set is then quotiented
by a relation known as structural congruence and it is over this set
that the notion of dynamics is expressed. This presentation is
essentially that of \cite{MeredithR05} with the addition of
polyadicity and summation. For readability we have relegated some of
the technical subtleties to an appendix.

\subsubsection{Process grammar}\label{subsub:process_grammar}

\begin{mathpar}
  \inferrule* [lab=synchronization] {} {{M} \bc \pzero \;|\; x?F \;|\; x!C }
  \and
  \inferrule* [lab=abstraction] {} {{F} \bc (x)P}
  \and
  \inferrule* [lab=concretion] {} {{C} \bc \langle Q \rangle}
  \and
  \inferrule* [lab=process] {} {{P,Q} \bc M \;| \;P|Q \;|\; @{x}}
  \and
  \inferrule* [lab=name] {} {{x} \bc \quotep{P}}
\end{mathpar} 

Note that $\vec{x}$ (resp. $\vec{P}$) denotes a vector of names
(resp. processes) of length $|\vec{x}|$ (resp. $|\vec{P}|$). We adopt
the following useful abbreviations.

\begin{mathpar}
   x?(\vec{y}).P := x.(\vec{y})P \and  x\clift{\vec{P}} := x.\clift{\vec{P}}
   \and x!(y) := \lift{x}{\dropn{y}}
   \and \Pi_{i=0}^{n-1}P_i := P_0 | \ldots | P_{n-1}
\end{mathpar}

\subsubsection{Structural congruence}

\paragraph{Free and bound names and alpha-equivalence.} At the
core of structural equivalence is alpha-equivalence which identifies
process that are the same up to a change of variable. Formally, we
recognize the distinction between free and bound names. The free names
of a process, $\freenames{P}$, may be calculated recursively as
follows:

\begin{mathpar}
\freenames{\pzero} := \emptyset
  \and \\
  \freenames{x?(y).P} := \{ x \} \cup (\freenames{P} \setminus \{ y \})
  \and 
  \freenames{x!\langle P \rangle} := \{ x \} \cup \{ P \} 
  \and \\
  \freenames{P|Q} := \freenames{P} \cup \freenames{Q}
  \and \\
  \freenames{@{x}} := \{ x \}
\end{mathpar}

$\pi$
$\quotep{\pi}$

$\freenames{-} : \pi \to \mathcal{P}(\quotep{\pi})$

\begin{eqnarray*}
  \freenames{\pzero} & := & \emptyset \\
  \freenames{x?(y).P} & := & \{ x \} \cup (\freenames{P} \setminus \{ y \}) \\
  \freenames{x!\langle P \rangle} & := & \{ x \} \cup \{ P \} \\
  \freenames{P|Q} & := & \freenames{P} \cup \freenames{Q} \\
  \freenames{\dropn{x}} & := & \{ x \}
\end{eqnarray*}

The bound names of a process, $\boundnames{P}$, are those names occurring in $P$
that are not free. For example, in $x?(y).0$, the name $x$ is free, while $y$ is bound.

\begin{mathpar}
  \inferrule* [lab=monoidal-laws] {} { P|Q \equiv Q|P \and P|0 \equiv P \and P|(Q|R) \equiv (P|Q)|R }
\end{mathpar}

\begin{mathpar}
  \inferrule* [lab=alpha-equivalence] {} { (x)P \equiv (y)P\{y/x\} \and y \not\in \freenames{P} }
\end{mathpar}

\begin{definition}
Then two processes, $P,Q$, are alpha-equivalent if $P = Q\{\vec{y}/\vec{x}\}$ for
some $\vec{x} \in \boundnames{Q},\vec{y} \in \boundnames{P}$, where $Q\{\vec{y}/\vec{x}\}$
denotes the capture-avoiding substitution of $\vec{y}$ for $\vec{x}$ in $Q$.
\end{definition}

\begin{definition}
  The {\em structural congruence} \cite{SangiorgiWalker} , $\equiv$,
  between processes is the least congruence containing
  alpha-equivalence, satisfying the abelian monoid laws
  (associativity, commutativity and $\pzero$ as identity) for parallel
  composition $|$ and for summation $+$.
\end{definition}

\subsection{Name equivalence}

We take name equivalence, written $\nameeq$, to be the smallest
equivalence relation generated by the following rules.

\begin{mathpar}
\inferrule*[lab=Quote-drop]
{ }
{ \quotep{@{x}} \nameeq x }

\inferrule*[lab=Struct-equiv]
{ P \scong Q }
{ \quotep{P} \nameeq \quotep{Q} }
\end{mathpar}

The astute reader will have noticed that the mutual recursion of names
and processes imposes a mutual recursion on alpha-equivalence and
structural equivalence via name-equivalence. Fortunately, all of this
works out pleasantly and we may calculate in the natural way, free of
concern. The reader interested in the details is referred to the
appendix \ref{appendix:rho_details}.

\subsection{Substitution}

We use $\Proc$ for the set of processes, $\QProc$ for the set of
names, and $\id{\{}\vec{y} / \vec{x} \id{\}}$ to denote partial maps,
$s : \QProc \rightarrow \QProc$. A map, $s$ lifts, uniquely, to a map
on process terms, $\widehat{s} : \Proc \rightarrow \Proc$ by the
following equations.

\begin{mathpar}
  (0) \psubstp{Q}{P} := 0 \\
  (R \juxtap S) \psubstp{Q}{P}
  :=    
  (R)\psubstp{Q}{P} \juxtap (S) \psubstp{Q}{P} \\
  (x?(y).R) \psubstp{Q}{P}    
  :=    
  (x)\substp{Q}{P} (z)\concat( (R \psubstn{z}{y}) \psubstp{Q}{P} ) \\
  (\lift{x}{R}) \psubstp{Q}{P}  
  :=
  \lift{(x)\substp{Q}{P}}{ R \psubstp{Q}{P} } \\
%   (\dropn{x})  \psubstp{Q}{P}       
%   := 
%   \left\{ 
%     \begin{array}{ccc} 
%       \dropn{\quotep{Q}} & & x \nameeq \quotep{P} \\
%       \dropn{x} & & otherwise \\
%     \end{array}
%   \right. 
  (\dropn{x})  \psubstp{Q}{P}       
  := 
  \left\{ 
    \begin{array}{ccc} 
      Q & & x \nameeq \quotep{P} \\
      \dropn{x} & & otherwise \\
    \end{array}
  \right.
\end{mathpar}
 

where

\begin{eqnarray}
  (x)\id{\{} \lpquote Q \rpquote / \lpquote P \rpquote \id{\}}            = 
  \left\{ 
    \begin{array}{ccc}
      \lpquote Q \rpquote & & x \nameeq \lpquote P \rpquote \\
      x & & otherwise \\
    \end{array}
  \right. \nonumber
\end{eqnarray}

and $z$ is chosen distinct from $\quotep{P}$, $\quotep{Q}$, the free
names in $Q$, and all the names in $R$. Our $\alpha$-equivalence will
be built in the standard way from this substitution.

\begin{remark}\label{rem:no_self_referential_names}
  One consequence of these definitions is that $\forall P. \quotep{P}
  \not\in \freenames{P}$.
\end{remark}

\subsection{ Dynamic quote: an example }

Anticipating something of what's to come, consider applying the
substitution, $\widehat{\id{\{}u / z \id{\}}}$, to the following pair
of processes, $\lift{w}{y!(z)}$ and $w[ \lpquote y!(z) \rpquote ]$.

\begin{eqnarray}
	\lift{w}{y!(z)}\widehat{\id{\{}u / z \id{\}}}
		& = &
		\lift{w}{y!(u)} \nonumber\\
	w[ \lpquote y!(z) \rpquote ] \widehat{ \id{\{}u / z \id{\}} }
		& = &
		w[ \lpquote y!(z) \rpquote ] \nonumber
\end{eqnarray}

Because the body of the process between quotes is impervious to
substitution, we get radically different answers. In fact, by
examining the first process in an input context,
e.g. $x?(z).\lift{w}{y!(z)}$, we see that the process under the lift
operator may be shaped by prefixed inputs binding a name inside it. In
this sense, the lift operator will be seen as a way to dynamically
construct processes before reifying them as names.

Finally equipped with these standard features we can present the
dynamics of the calculus.

\subsubsection{Operational semantics} 

Finally, we introduce the computational dynamics. What marks these
algebras as distinct from other more traditionally studied algebraic
structures, e.g. vector spaces or polynomial rings, is the manner in
which dynamics is captured. In traditional structures, dynamics is typically
expressed through morphisms between such structures, as in linear maps
between vector spaces or morphisms between rings. In algebras
associated with the semantics of computation, the dynamics is
expressed as part of the algebraic structure itself, through a
reduction reduction relation typically denoted by $\red$. Below, we
give a recursive presentation of this relation for the calculus used
in the encoding.

$\red \subseteq \pi \times \pi$
$\red : \pi \to \mathcal{P}(\pi)$

\begin{mathpar}
  \inferrule* [lab=Comm] { \textsf{match}( x_{src}, x_{trgt} ) } { x_{trgt}?(y)P \; | \; x_{src}!\langle {Q} \rangle \red P\{\quotep{Q}/y}\} }
  \and \\
  \inferrule* [lab=Par] {{P} \red {P}'} {{{P} | {Q}} \red {{P}' | {Q}}}
  \and
  \inferrule* [lab=Equiv]{{{P} \scong {P}'} \andalso {{P}' \red {Q}'} \andalso {{Q}' \scong {Q}}}{{P} \red {Q}}
\end{mathpar}

\begin{eqnarray*}
  match_{\equiv} (\quotep{P},\quotep{Q}) & := & P \equiv Q \\
  match_{\dagger}(\quotep{P},\quotep{Q}) & := & \forall R. P|Q \red^{*} R => R \red^{*} 0 \\
  match_{K}(\quotep{P},\quotep{Q}) & := & K \mbox{ for some context } K
\end{eqnarray*}

$u?(x)P | u!\langle Q \rangle \red P\{\quotep{Q}/x\}$

%We write $\wred$ for $\red^*$, and $P\red$ if $\exists Q $ such that $ P \red Q$.
We write $P\red$ if $\exists Q $ such that $ P \red Q$ and $P\not\red$, otherwise.

\section{Replication}

As mentioned before, it is known that replication (and hence
recursion) can be implemented in a higher-order process algebra
\cite{SangiorgiWalker}. As our first example of calculation with the
machinery thus far presented we give the construction explicitly in
the {\rhoc}.

\begin{eqnarray}
	D_{x} & := & \prefix{x}{y}{(\binpar{\outputp{x}{y}}{@{y}})} \nonumber\\
	\bangp_{x}{P} & := & \binpar{{x}!\langle{\binpar{D_{x}}{P}}\rangle}{D_{x}} \nonumber
\end{eqnarray}

\begin{eqnarray}
	\bangp_{x}{P} & & \nonumber\\
	=
	& {x}!\langle{(\prefix{x}{y}{(\outputp{x}{y} | @{y})) | P}}\rangle 
	      | \prefix{x}{y}{(\outputp{x}{y} | @{y})} & \nonumber\\
	\red
	& (\outputp{x}{y} | @{y})\substn{\quotep{(\prefix{x}{y}{(@{y} | \outputp{x}{y})) | P}}}{y} & \nonumber\\
	=
	& \outputp{x}{\quotep{(\prefix{x}{y}{(\outputp{x}{y} | @{y})) | P}}}
	  | {(\prefix{x}{y}{(\outputp{x}{y} | @{y})) | P}} & \nonumber\\
	\red
	& \ldots & \nonumber\\
	\red^*
	& P | P | \ldots & \nonumber
\end{eqnarray}

Of course, this encoding, as an implementation, runs away, unfolding
$\bangp{P}$ eagerly. A lazier and more implementable replication
operator, restricted to input-guarded processes, may be obtained as follows.

\begin{eqnarray}
\bangp{\prefix{u}{v}{P}} 
	:= 
	\binpar{\lift{x}{\prefix{u}{v}{(\binpar{D(x)}{P})}}}{D(x)} \nonumber
\end{eqnarray}

\begin{remark}
  Note that the lazier definition still does not deal with summation
  or mixed summation (i.e. sums over input and output). The reader is
  invited to construct definitions of replication that deal with these
  features. 

  Further, the definitions are parameterized in a name, $x$. Can you,
  gentle reader, make a definition that eliminates this parameter and
  guarantees no accidental interaction between the replication
  machinery and the process being replicated -- i.e. no accidental
  sharing of names used by the process to get its work done and the
  name(s) used by the replication to effect copying. This latter
  revision of the definition of replication is crucial to obtaining
  the expected identity $!!P \sim !P$.
\end{remark}

\begin{remark}\label{rem:paradoxical_combinator}
  The reader familiar with the lambda calculus will have noticed the
  similarity between $D$ and the paradoxical combinator.

  [Ed. note: the existence of this seems to suggest we have to be more
  restrictive on the set of processes and names we admit if we are to
  support no-cloning.]
\end{remark}

\subsubsection{Bisimulation}

The computational dynamics gives rise to another kind of equivalence,
the equivalence of computational behavior. As previously mentioned
this is typically captured \emph{via} some form of bisimulation.

% The notion we use in this paper is weak barbed bisimulation
% \cite{milner91polyadicpi}.

The notion we use in this paper is derived from weak barbed
bisimulation \cite{milner91polyadicpi}. 

\begin{definition}
An \emph{observation relation}, $\downarrow_{\mathcal N}$, over a set
of names, $\mathcal N$, is the smallest relation satisfying the rules
below.

\infrule[Out-barb]{y \in {\mathcal N}, \; x \nameeq y}
		  {\outputp{x}{v} \downarrow_{\mathcal N} x}
\infrule[Par-barb]{\mbox{$P\downarrow_{\mathcal N} x$ or $Q\downarrow_{\mathcal N} x$}}
		  {\binpar{P}{Q} \downarrow_{\mathcal N} x}

We write $P \Downarrow_{\mathcal N} x$ if there is $Q$ such that 
$P \wred Q$ and $Q \downarrow_{\mathcal N} x$.
\end{definition}

\begin{definition}
%\label{def.bbisim}
An  ${\mathcal N}$-\emph{barbed bisimulation} over a set of names, ${\mathcal N}$, is a symmetric binary relation 
${\mathcal S}_{\mathcal N}$ between agents such that $P\rel{S}_{\mathcal N}Q$ implies:
\begin{enumerate}
\item If $P \red P'$ then $Q \wred Q'$ and $P'\rel{S}_{\mathcal N} Q'$.
\item If $P\downarrow_{\mathcal N} x$, then $Q\Downarrow_{\mathcal N} x$.
\end{enumerate}
$P$ is ${\mathcal N}$-barbed bisimilar to $Q$, written
$P \wbbisim_{\mathcal N} Q$, if $P \rel{S}_{\mathcal N} Q$ for some ${\mathcal N}$-barbed bisimulation ${\mathcal S}_{\mathcal N}$.
\end{definition}

$\mathcal{R} \subseteq \pi \times \pi$

$P \mathcal{R} Q => \forall P'. P \red P' \Rightarrow \exists Q'. Q \red Q', P' \mathcal{R} Q'$

$P \vdash x \Rightarrow Q \vdash x$

\begin{mathpar}
  \inferrule*[lab=Out-barb]{x \nameeq y}{{y}!\langle{Q}\rangle \vdash x}
  \and
  \inferrule*[lab=Par-barb]{\mbox{$P\vdash x$ or $Q\vdash x$}}{\binpar{P}{Q} \vdash x}
\end{mathpar}

\subsubsection{Contexts}

One of the principle advantages of computational calculi like the
$\pi$-calculus is a well-defined notion of context,
contextual-equivalence and a correlation between
contextual-equivalence and notions of bisimulation. The notion of
context allows the decomposition of a process into (sub-)process and
its syntactic environment, its context. Thus, a context may be
thought of as a process with a ``hole'' (written $\Box$) in it. The
application of a context $M$ to a process $P$, written $M[P]$, is
tantamount to filling the hole in $M$ with $P$. In this paper we do
not need the full weight of this theory, but do make use of the notion
of context in the proof the main theorem. 

\begin{mathpar}
  \inferrule* [lab=summation] {} {{M_{M},M_{N}} \bc \Box \;|\; x.M_{A} \;|\; M_{M}+M_{N}}
  \and
  \inferrule* [lab=agent] {} {{M_{A}} \bc (\vec{x})M_{P} \;| \; \clift{P_0,\ldots,M_{P},\ldots,P_N}}
  \and \\
  \inferrule* [lab=process] {} {{M_{P}} \bc M_{N} \;| \;P|M_{P} }
\end{mathpar} 

\begin{mathpar}
  \inferrule* [lab=sychronization] {} {M_{N} \bc \Box \;|\; x?M_{F} \;|\; x!M_{C}}
  \and
  \inferrule* [lab=abstraction] {} {{M_{F}} \bc (x)M_{P} }
  \and
  \inferrule* [lab=concretion] {} {{M_{C}} \bc \langle M_{P} \rangle }
  \and \\
  \inferrule* [lab=process] {} {{M_{P}} \bc M_{N} \;| \;P|M_{P} }
\end{mathpar}

\begin{definition}[contextual application] Given a context $M$, and
  process $P$, we define the \emph{contextual application}, $M[P] :=
  M\{P/\Box\}$. That is, the contextual application of M to P is the
  substitution of $P$ for $\Box$ in $M$.
\end{definition}

$\meaningof{-} : L \to \mathcal{P}(\pi)$

\begin{mathpar}
  \inferrule* [lab=collection] {} {\meaningof{true} = \pi, \and \meaningof{~E} = \pi \setminus \meaningof{E}, \and \meaningof{E_{1} \& E_{2}} = \meaningof{E_{1}} \cap \meaningof{E_{2}}}
\end{mathpar}

\begin{mathpar}
  \inferrule* [lab=structure] {} {\meaningof{0} = \{ P \in \pi | P \equiv 0 \}, \and \\ \meaningof{E_1 | E_2} = \{ P \in \pi | P \equiv P_{1} | P_{2}, P_{1} \in \meaningof{E_{1}}, P_{2} \in \meaningof{E_2}\} }
\end{mathpar}

\begin{mathpar}
 \inferrule* [lab=behavior] {} {\meaningof{\langle a?b \rangle E} = \{ P \in \pi | P \equiv Q | u?(y)P', \\ \and \\\\ \and \\ \;\;\; u \in \meaningof{a}, \forall z.P'\{z/y\} \in \meaningof{E\{z/b\}}\}, \and \\ \meaningof{a!E} = \{ P \in \pi | P \equiv Q | x!\langle P' \rangle, x \in \meaningof{a} P' \in \meaningof{E}\} }
\end{mathpar}

\begin{mathpar}
 \inferrule* [lab=nominal] {} {\meaningof{\quotep{E}} = \{ \quotep{P} \in \quotep{\pi} | P \in \meaningof{E} \}, \and \meaningof{\quotep{P}} = \{ \quotep{Q} \in \quotep{\pi} | P \equiv Q \} \and \\ \meaningof{@\quotep{E}} = \{ P \in \pi | P \equiv @x, x \in \meaningof{E} \}}
\end{mathpar}

\begin{eqnarray*}
  \\
  \meaningof{-} : TS \to ST
\end{eqnarray*}

\begin{eqnarray*}
  \\
  L : TS \to ST
\end{eqnarray*}

\begin{eqnarray*}
  \\
  P \models E \iff P \in \meaningof{E}
\end{eqnarray*}

\begin{eqnarray*}
  P \approx_{L} Q \iff \forall E \in L. P \models E \iff Q \models E
\end{eqnarray*}

\begin{eqnarray*}
  P \approx_{K} Q
\end{eqnarray*}

\begin{eqnarray*}
  P \approx Q
\end{eqnarray*}

$\approx_{K} = \approx = \approx_{L}$

\subsubsection{Contextual duality}

Note that contexts extend the quotation operation to a family of
operations from processes to names. Given a context, $M$, we can
define a \emph{nominal context}, $\quotep{M}$ by $\quotep{M}[P] :=
\quotep{M[P]}$. To foreshadow what is to come we observe that these
operations enjoy a duality with processes very much like the duality
between vectors and maps from vectors to scalars.

Further, because the calculus is essentially higher-order, we have a
correspondence between contexts and processes. More specifically,
given a name $x$ and a context $M$ we can construct $M^{*}_{x}$ such
that 

\begin{mathpar}
  M^{*}_{x} | \lift{x}{P} \red M[P]
\end{mathpar}

namely,

\begin{mathpar}
  M^{*}_{x} := x?(u).M[\dropn{u}]
\end{mathpar}

The dependence of $M^{*}_{x}$ on a name makes it an abstraction, 

\begin{mathpar}
  M^{*} := (x)x?(u).M[\dropn{u}]
\end{mathpar}

\subsection{Additional notation}

It will sometimes be convenient to denote the process a name
quotes. We already have the notation $x = \quotep{P}$, but it will be
convenient to introduce an alternate notation, $\procn{x}$, when we
want to emphasize the connection to the use of the name. Note that, by
virtue of name equivalence, $\quotep{\procn{x}} \nameeq x$; so, the
notation is consistent with previous definitions.

Further, because names have structure it is possible to effect
substitutions on the basis of that structure. This means we need to
upgrade our notation for substitutions, which we accomplish by
adapting comprehension notation. Thus,

\begin{mathpar}
  P\{ y / x : x \in S \}
\end{mathpar}

is interpreted to mean the process derived from P by replacing (in a
capture-avoiding manner) each occurrence of $x$ in $S$ by $y$. For example,

\begin{mathpar}
  P\{ \quotep{\procn{x}|\procn{x}} / x : x \in \freenames{P} \}
\end{mathpar}

will replace each (occurrence) of a free name $x$ in $P$ by
$\quotep{\procn{x}|\procn{x}}$.

Also, we will avail ourselves of the notation $x^{L}$ and $x^{R}$ to
denote injections of a name into disjoint copies of the name
space. There are numerous ways to accomplish this. One example can be
found in \cite{MeredithR05}. This notation overloads to vectors of
names: $\vec{x}^{\pi} := (x_{i}^{\pi} \; : \; 0 \leq i < |\vec{x}| )$ where $\pi \in \{L,R\}$.

We also use $P^{\Box} := P|\Box$.

In \cite{MeredithR05} an interpretation of the new operator is
given. It turns out that there are several possible interpretations
all enjoying the requisite algebraic properties of the operator (see
\cite{milner91polyadicpi}). We will therefore make liberal use of
$(\nu\; \vec{x})P$.

% subsection the_syntax_and_semantics_of_the_notation_system (end)   

\input{qm2pi.qmops} 

\input{qm2pi.sterngerlach} 

\input{qm2pi.metric} 

% section concurrent_process_calculi (end)

%\input{qm2pi.proofsketch}

% section proof sketch (end)

%\input{qm2pi.slviaknots} 

% section spatial logic via knots (end)

\input{qm2pi.conclusion}

% section conclusion (end)

%\input{qm2pi.dtcodes} 

% section wiring algorithm (end)

\input{qm2pi.ack} 

% section acknowledgments (end)

\newpage


\bibliographystyle{plain}   
\bibliography{../../biblios/main.bib}

\input{qm2pi.rhodetails}

\end{document}

 

% section notation (end)

\input{qm2pi.process.calculi} 

% section concurrent_process_calculi_and_spatial_logics_ (end)
    
%\documentclass[12pt]{llncs}
%\documentclass{jktr}

\usepackage[pdftex]{hyperref}                   
\usepackage {listings}
\usepackage {mathpartir}
\usepackage{bcprules}
%\usepackage{listings}
                       
\usepackage{graphicx} 
%\usepackage[margins=2.5cm,nohead,nofoot]{geometry}
%\usepackage{geometry}
\usepackage{amsfonts}
\usepackage{amstext}
\usepackage{latexsym}
\usepackage{amssymb}
\usepackage{color}


%\include{myPreamble}
\include{qm2pi.local} 

%\ifpdf
%\usepackage[pdftex]{graphicx}
%\else
%\usepackage{graphicx}
%\fi

 % \ifpdf
%  \usepackage{pdfsync}
%  \if


%\title{Brief Article}
%\author{David F. Snyder}
%\author{L.G. Meredith}

%\address{Dept. of Math., Texas State University--San Marcos, San Marcos, TX 78666}
       
\pagestyle{empty}


\begin{document}

\lstset{language=[Objective]Caml,frame=shadowbox}

\input{qm2pi.front}

% section front matter (end)

\input{qm2pi.intro} 
 
% section introduction (end)

% \input{qm2pi.knotations} 

% section notation (end)

\input{qm2pi.process.calculi} 

% section concurrent_process_calculi_and_spatial_logics_ (end)
    
%\input{qm2pi.knots2pi} 

%\input{qm2pi.trefoil} 

%\input{qm2pi.mainthm} 

% subsection basic_interpretation (end)

%\input{qm2pi.rho.presentation} 
\subsection{The syntax and semantics of the notation system}\label{sub:the_syntax_and_semantics_of_the_notation_system} % (fold)

We now summarize a technical presentation of the calculus that
embodies our theory of dynamics. The typical presentation of such a
calculus follows the style of giving generators and relations on
them. The grammar, below, describing term constructors, freely
generates the set of processes, $\Proc$. This set is then quotiented
by a relation known as structural congruence and it is over this set
that the notion of dynamics is expressed. This presentation is
essentially that of \cite{MeredithR05} with the addition of
polyadicity and summation. For readability we have relegated some of
the technical subtleties to an appendix.

\subsubsection{Process grammar}\label{subsub:process_grammar}

\begin{mathpar}
  \inferrule* [lab=synchronization] {} {{M} \bc \pzero \;|\; x?F \;|\; x!C }
  \and
  \inferrule* [lab=abstraction] {} {{F} \bc (x)P}
  \and
  \inferrule* [lab=concretion] {} {{C} \bc \langle Q \rangle}
  \and
  \inferrule* [lab=process] {} {{P,Q} \bc M \;| \;P|Q \;|\; @{x}}
  \and
  \inferrule* [lab=name] {} {{x} \bc \quotep{P}}
\end{mathpar} 

Note that $\vec{x}$ (resp. $\vec{P}$) denotes a vector of names
(resp. processes) of length $|\vec{x}|$ (resp. $|\vec{P}|$). We adopt
the following useful abbreviations.

\begin{mathpar}
   x?(\vec{y}).P := x.(\vec{y})P \and  x\clift{\vec{P}} := x.\clift{\vec{P}}
   \and x!(y) := \lift{x}{\dropn{y}}
   \and \Pi_{i=0}^{n-1}P_i := P_0 | \ldots | P_{n-1}
\end{mathpar}

\subsubsection{Structural congruence}

\paragraph{Free and bound names and alpha-equivalence.} At the
core of structural equivalence is alpha-equivalence which identifies
process that are the same up to a change of variable. Formally, we
recognize the distinction between free and bound names. The free names
of a process, $\freenames{P}$, may be calculated recursively as
follows:

\begin{mathpar}
\freenames{\pzero} := \emptyset
  \and \\
  \freenames{x?(y).P} := \{ x \} \cup (\freenames{P} \setminus \{ y \})
  \and 
  \freenames{x!\langle P \rangle} := \{ x \} \cup \{ P \} 
  \and \\
  \freenames{P|Q} := \freenames{P} \cup \freenames{Q}
  \and \\
  \freenames{@{x}} := \{ x \}
\end{mathpar}

$\pi$
$\quotep{\pi}$

$\freenames{-} : \pi \to \mathcal{P}(\quotep{\pi})$

\begin{eqnarray*}
  \freenames{\pzero} & := & \emptyset \\
  \freenames{x?(y).P} & := & \{ x \} \cup (\freenames{P} \setminus \{ y \}) \\
  \freenames{x!\langle P \rangle} & := & \{ x \} \cup \{ P \} \\
  \freenames{P|Q} & := & \freenames{P} \cup \freenames{Q} \\
  \freenames{\dropn{x}} & := & \{ x \}
\end{eqnarray*}

The bound names of a process, $\boundnames{P}$, are those names occurring in $P$
that are not free. For example, in $x?(y).0$, the name $x$ is free, while $y$ is bound.

\begin{mathpar}
  \inferrule* [lab=monoidal-laws] {} { P|Q \equiv Q|P \and P|0 \equiv P \and P|(Q|R) \equiv (P|Q)|R }
\end{mathpar}

\begin{mathpar}
  \inferrule* [lab=alpha-equivalence] {} { (x)P \equiv (y)P\{y/x\} \and y \not\in \freenames{P} }
\end{mathpar}

\begin{definition}
Then two processes, $P,Q$, are alpha-equivalent if $P = Q\{\vec{y}/\vec{x}\}$ for
some $\vec{x} \in \boundnames{Q},\vec{y} \in \boundnames{P}$, where $Q\{\vec{y}/\vec{x}\}$
denotes the capture-avoiding substitution of $\vec{y}$ for $\vec{x}$ in $Q$.
\end{definition}

\begin{definition}
  The {\em structural congruence} \cite{SangiorgiWalker} , $\equiv$,
  between processes is the least congruence containing
  alpha-equivalence, satisfying the abelian monoid laws
  (associativity, commutativity and $\pzero$ as identity) for parallel
  composition $|$ and for summation $+$.
\end{definition}

\subsection{Name equivalence}

We take name equivalence, written $\nameeq$, to be the smallest
equivalence relation generated by the following rules.

\begin{mathpar}
\inferrule*[lab=Quote-drop]
{ }
{ \quotep{@{x}} \nameeq x }

\inferrule*[lab=Struct-equiv]
{ P \scong Q }
{ \quotep{P} \nameeq \quotep{Q} }
\end{mathpar}

The astute reader will have noticed that the mutual recursion of names
and processes imposes a mutual recursion on alpha-equivalence and
structural equivalence via name-equivalence. Fortunately, all of this
works out pleasantly and we may calculate in the natural way, free of
concern. The reader interested in the details is referred to the
appendix \ref{appendix:rho_details}.

\subsection{Substitution}

We use $\Proc$ for the set of processes, $\QProc$ for the set of
names, and $\id{\{}\vec{y} / \vec{x} \id{\}}$ to denote partial maps,
$s : \QProc \rightarrow \QProc$. A map, $s$ lifts, uniquely, to a map
on process terms, $\widehat{s} : \Proc \rightarrow \Proc$ by the
following equations.

\begin{mathpar}
  (0) \psubstp{Q}{P} := 0 \\
  (R \juxtap S) \psubstp{Q}{P}
  :=    
  (R)\psubstp{Q}{P} \juxtap (S) \psubstp{Q}{P} \\
  (x?(y).R) \psubstp{Q}{P}    
  :=    
  (x)\substp{Q}{P} (z)\concat( (R \psubstn{z}{y}) \psubstp{Q}{P} ) \\
  (\lift{x}{R}) \psubstp{Q}{P}  
  :=
  \lift{(x)\substp{Q}{P}}{ R \psubstp{Q}{P} } \\
%   (\dropn{x})  \psubstp{Q}{P}       
%   := 
%   \left\{ 
%     \begin{array}{ccc} 
%       \dropn{\quotep{Q}} & & x \nameeq \quotep{P} \\
%       \dropn{x} & & otherwise \\
%     \end{array}
%   \right. 
  (\dropn{x})  \psubstp{Q}{P}       
  := 
  \left\{ 
    \begin{array}{ccc} 
      Q & & x \nameeq \quotep{P} \\
      \dropn{x} & & otherwise \\
    \end{array}
  \right.
\end{mathpar}
 

where

\begin{eqnarray}
  (x)\id{\{} \lpquote Q \rpquote / \lpquote P \rpquote \id{\}}            = 
  \left\{ 
    \begin{array}{ccc}
      \lpquote Q \rpquote & & x \nameeq \lpquote P \rpquote \\
      x & & otherwise \\
    \end{array}
  \right. \nonumber
\end{eqnarray}

and $z$ is chosen distinct from $\quotep{P}$, $\quotep{Q}$, the free
names in $Q$, and all the names in $R$. Our $\alpha$-equivalence will
be built in the standard way from this substitution.

\begin{remark}\label{rem:no_self_referential_names}
  One consequence of these definitions is that $\forall P. \quotep{P}
  \not\in \freenames{P}$.
\end{remark}

\subsection{ Dynamic quote: an example }

Anticipating something of what's to come, consider applying the
substitution, $\widehat{\id{\{}u / z \id{\}}}$, to the following pair
of processes, $\lift{w}{y!(z)}$ and $w[ \lpquote y!(z) \rpquote ]$.

\begin{eqnarray}
	\lift{w}{y!(z)}\widehat{\id{\{}u / z \id{\}}}
		& = &
		\lift{w}{y!(u)} \nonumber\\
	w[ \lpquote y!(z) \rpquote ] \widehat{ \id{\{}u / z \id{\}} }
		& = &
		w[ \lpquote y!(z) \rpquote ] \nonumber
\end{eqnarray}

Because the body of the process between quotes is impervious to
substitution, we get radically different answers. In fact, by
examining the first process in an input context,
e.g. $x?(z).\lift{w}{y!(z)}$, we see that the process under the lift
operator may be shaped by prefixed inputs binding a name inside it. In
this sense, the lift operator will be seen as a way to dynamically
construct processes before reifying them as names.

Finally equipped with these standard features we can present the
dynamics of the calculus.

\subsubsection{Operational semantics} 

Finally, we introduce the computational dynamics. What marks these
algebras as distinct from other more traditionally studied algebraic
structures, e.g. vector spaces or polynomial rings, is the manner in
which dynamics is captured. In traditional structures, dynamics is typically
expressed through morphisms between such structures, as in linear maps
between vector spaces or morphisms between rings. In algebras
associated with the semantics of computation, the dynamics is
expressed as part of the algebraic structure itself, through a
reduction reduction relation typically denoted by $\red$. Below, we
give a recursive presentation of this relation for the calculus used
in the encoding.

$\red \subseteq \pi \times \pi$
$\red : \pi \to \mathcal{P}(\pi)$

\begin{mathpar}
  \inferrule* [lab=Comm] { \textsf{match}( x_{src}, x_{trgt} ) } { x_{trgt}?(y)P \; | \; x_{src}!\langle {Q} \rangle \red P\{\quotep{Q}/y}\} }
  \and \\
  \inferrule* [lab=Par] {{P} \red {P}'} {{{P} | {Q}} \red {{P}' | {Q}}}
  \and
  \inferrule* [lab=Equiv]{{{P} \scong {P}'} \andalso {{P}' \red {Q}'} \andalso {{Q}' \scong {Q}}}{{P} \red {Q}}
\end{mathpar}

\begin{eqnarray*}
  match_{\equiv} (\quotep{P},\quotep{Q}) & := & P \equiv Q \\
  match_{\dagger}(\quotep{P},\quotep{Q}) & := & \forall R. P|Q \red^{*} R => R \red^{*} 0 \\
  match_{K}(\quotep{P},\quotep{Q}) & := & K \mbox{ for some context } K
\end{eqnarray*}

$u?(x)P | u!\langle Q \rangle \red P\{\quotep{Q}/x\}$

%We write $\wred$ for $\red^*$, and $P\red$ if $\exists Q $ such that $ P \red Q$.
We write $P\red$ if $\exists Q $ such that $ P \red Q$ and $P\not\red$, otherwise.

\section{Replication}

As mentioned before, it is known that replication (and hence
recursion) can be implemented in a higher-order process algebra
\cite{SangiorgiWalker}. As our first example of calculation with the
machinery thus far presented we give the construction explicitly in
the {\rhoc}.

\begin{eqnarray}
	D_{x} & := & \prefix{x}{y}{(\binpar{\outputp{x}{y}}{@{y}})} \nonumber\\
	\bangp_{x}{P} & := & \binpar{{x}!\langle{\binpar{D_{x}}{P}}\rangle}{D_{x}} \nonumber
\end{eqnarray}

\begin{eqnarray}
	\bangp_{x}{P} & & \nonumber\\
	=
	& {x}!\langle{(\prefix{x}{y}{(\outputp{x}{y} | @{y})) | P}}\rangle 
	      | \prefix{x}{y}{(\outputp{x}{y} | @{y})} & \nonumber\\
	\red
	& (\outputp{x}{y} | @{y})\substn{\quotep{(\prefix{x}{y}{(@{y} | \outputp{x}{y})) | P}}}{y} & \nonumber\\
	=
	& \outputp{x}{\quotep{(\prefix{x}{y}{(\outputp{x}{y} | @{y})) | P}}}
	  | {(\prefix{x}{y}{(\outputp{x}{y} | @{y})) | P}} & \nonumber\\
	\red
	& \ldots & \nonumber\\
	\red^*
	& P | P | \ldots & \nonumber
\end{eqnarray}

Of course, this encoding, as an implementation, runs away, unfolding
$\bangp{P}$ eagerly. A lazier and more implementable replication
operator, restricted to input-guarded processes, may be obtained as follows.

\begin{eqnarray}
\bangp{\prefix{u}{v}{P}} 
	:= 
	\binpar{\lift{x}{\prefix{u}{v}{(\binpar{D(x)}{P})}}}{D(x)} \nonumber
\end{eqnarray}

\begin{remark}
  Note that the lazier definition still does not deal with summation
  or mixed summation (i.e. sums over input and output). The reader is
  invited to construct definitions of replication that deal with these
  features. 

  Further, the definitions are parameterized in a name, $x$. Can you,
  gentle reader, make a definition that eliminates this parameter and
  guarantees no accidental interaction between the replication
  machinery and the process being replicated -- i.e. no accidental
  sharing of names used by the process to get its work done and the
  name(s) used by the replication to effect copying. This latter
  revision of the definition of replication is crucial to obtaining
  the expected identity $!!P \sim !P$.
\end{remark}

\begin{remark}\label{rem:paradoxical_combinator}
  The reader familiar with the lambda calculus will have noticed the
  similarity between $D$ and the paradoxical combinator.

  [Ed. note: the existence of this seems to suggest we have to be more
  restrictive on the set of processes and names we admit if we are to
  support no-cloning.]
\end{remark}

\subsubsection{Bisimulation}

The computational dynamics gives rise to another kind of equivalence,
the equivalence of computational behavior. As previously mentioned
this is typically captured \emph{via} some form of bisimulation.

% The notion we use in this paper is weak barbed bisimulation
% \cite{milner91polyadicpi}.

The notion we use in this paper is derived from weak barbed
bisimulation \cite{milner91polyadicpi}. 

\begin{definition}
An \emph{observation relation}, $\downarrow_{\mathcal N}$, over a set
of names, $\mathcal N$, is the smallest relation satisfying the rules
below.

\infrule[Out-barb]{y \in {\mathcal N}, \; x \nameeq y}
		  {\outputp{x}{v} \downarrow_{\mathcal N} x}
\infrule[Par-barb]{\mbox{$P\downarrow_{\mathcal N} x$ or $Q\downarrow_{\mathcal N} x$}}
		  {\binpar{P}{Q} \downarrow_{\mathcal N} x}

We write $P \Downarrow_{\mathcal N} x$ if there is $Q$ such that 
$P \wred Q$ and $Q \downarrow_{\mathcal N} x$.
\end{definition}

\begin{definition}
%\label{def.bbisim}
An  ${\mathcal N}$-\emph{barbed bisimulation} over a set of names, ${\mathcal N}$, is a symmetric binary relation 
${\mathcal S}_{\mathcal N}$ between agents such that $P\rel{S}_{\mathcal N}Q$ implies:
\begin{enumerate}
\item If $P \red P'$ then $Q \wred Q'$ and $P'\rel{S}_{\mathcal N} Q'$.
\item If $P\downarrow_{\mathcal N} x$, then $Q\Downarrow_{\mathcal N} x$.
\end{enumerate}
$P$ is ${\mathcal N}$-barbed bisimilar to $Q$, written
$P \wbbisim_{\mathcal N} Q$, if $P \rel{S}_{\mathcal N} Q$ for some ${\mathcal N}$-barbed bisimulation ${\mathcal S}_{\mathcal N}$.
\end{definition}

$\mathcal{R} \subseteq \pi \times \pi$

$P \mathcal{R} Q => \forall P'. P \red P' \Rightarrow \exists Q'. Q \red Q', P' \mathcal{R} Q'$

$P \vdash x \Rightarrow Q \vdash x$

\begin{mathpar}
  \inferrule*[lab=Out-barb]{x \nameeq y}{{y}!\langle{Q}\rangle \vdash x}
  \and
  \inferrule*[lab=Par-barb]{\mbox{$P\vdash x$ or $Q\vdash x$}}{\binpar{P}{Q} \vdash x}
\end{mathpar}

\subsubsection{Contexts}

One of the principle advantages of computational calculi like the
$\pi$-calculus is a well-defined notion of context,
contextual-equivalence and a correlation between
contextual-equivalence and notions of bisimulation. The notion of
context allows the decomposition of a process into (sub-)process and
its syntactic environment, its context. Thus, a context may be
thought of as a process with a ``hole'' (written $\Box$) in it. The
application of a context $M$ to a process $P$, written $M[P]$, is
tantamount to filling the hole in $M$ with $P$. In this paper we do
not need the full weight of this theory, but do make use of the notion
of context in the proof the main theorem. 

\begin{mathpar}
  \inferrule* [lab=summation] {} {{M_{M},M_{N}} \bc \Box \;|\; x.M_{A} \;|\; M_{M}+M_{N}}
  \and
  \inferrule* [lab=agent] {} {{M_{A}} \bc (\vec{x})M_{P} \;| \; \clift{P_0,\ldots,M_{P},\ldots,P_N}}
  \and \\
  \inferrule* [lab=process] {} {{M_{P}} \bc M_{N} \;| \;P|M_{P} }
\end{mathpar} 

\begin{mathpar}
  \inferrule* [lab=sychronization] {} {M_{N} \bc \Box \;|\; x?M_{F} \;|\; x!M_{C}}
  \and
  \inferrule* [lab=abstraction] {} {{M_{F}} \bc (x)M_{P} }
  \and
  \inferrule* [lab=concretion] {} {{M_{C}} \bc \langle M_{P} \rangle }
  \and \\
  \inferrule* [lab=process] {} {{M_{P}} \bc M_{N} \;| \;P|M_{P} }
\end{mathpar}

\begin{definition}[contextual application] Given a context $M$, and
  process $P$, we define the \emph{contextual application}, $M[P] :=
  M\{P/\Box\}$. That is, the contextual application of M to P is the
  substitution of $P$ for $\Box$ in $M$.
\end{definition}

$\meaningof{-} : L \to \mathcal{P}(\pi)$

\begin{mathpar}
  \inferrule* [lab=collection] {} {\meaningof{true} = \pi, \and \meaningof{~E} = \pi \setminus \meaningof{E}, \and \meaningof{E_{1} \& E_{2}} = \meaningof{E_{1}} \cap \meaningof{E_{2}}}
\end{mathpar}

\begin{mathpar}
  \inferrule* [lab=structure] {} {\meaningof{0} = \{ P \in \pi | P \equiv 0 \}, \and \\ \meaningof{E_1 | E_2} = \{ P \in \pi | P \equiv P_{1} | P_{2}, P_{1} \in \meaningof{E_{1}}, P_{2} \in \meaningof{E_2}\} }
\end{mathpar}

\begin{mathpar}
 \inferrule* [lab=behavior] {} {\meaningof{\langle a?b \rangle E} = \{ P \in \pi | P \equiv Q | u?(y)P', \\ \and \\\\ \and \\ \;\;\; u \in \meaningof{a}, \forall z.P'\{z/y\} \in \meaningof{E\{z/b\}}\}, \and \\ \meaningof{a!E} = \{ P \in \pi | P \equiv Q | x!\langle P' \rangle, x \in \meaningof{a} P' \in \meaningof{E}\} }
\end{mathpar}

\begin{mathpar}
 \inferrule* [lab=nominal] {} {\meaningof{\quotep{E}} = \{ \quotep{P} \in \quotep{\pi} | P \in \meaningof{E} \}, \and \meaningof{\quotep{P}} = \{ \quotep{Q} \in \quotep{\pi} | P \equiv Q \} \and \\ \meaningof{@\quotep{E}} = \{ P \in \pi | P \equiv @x, x \in \meaningof{E} \}}
\end{mathpar}

\begin{eqnarray*}
  \\
  \meaningof{-} : TS \to ST
\end{eqnarray*}

\begin{eqnarray*}
  \\
  L : TS \to ST
\end{eqnarray*}

\begin{eqnarray*}
  \\
  P \models E \iff P \in \meaningof{E}
\end{eqnarray*}

\begin{eqnarray*}
  P \approx_{L} Q \iff \forall E \in L. P \models E \iff Q \models E
\end{eqnarray*}

\begin{eqnarray*}
  P \approx_{K} Q
\end{eqnarray*}

\begin{eqnarray*}
  P \approx Q
\end{eqnarray*}

$\approx_{K} = \approx = \approx_{L}$

\subsubsection{Contextual duality}

Note that contexts extend the quotation operation to a family of
operations from processes to names. Given a context, $M$, we can
define a \emph{nominal context}, $\quotep{M}$ by $\quotep{M}[P] :=
\quotep{M[P]}$. To foreshadow what is to come we observe that these
operations enjoy a duality with processes very much like the duality
between vectors and maps from vectors to scalars.

Further, because the calculus is essentially higher-order, we have a
correspondence between contexts and processes. More specifically,
given a name $x$ and a context $M$ we can construct $M^{*}_{x}$ such
that 

\begin{mathpar}
  M^{*}_{x} | \lift{x}{P} \red M[P]
\end{mathpar}

namely,

\begin{mathpar}
  M^{*}_{x} := x?(u).M[\dropn{u}]
\end{mathpar}

The dependence of $M^{*}_{x}$ on a name makes it an abstraction, 

\begin{mathpar}
  M^{*} := (x)x?(u).M[\dropn{u}]
\end{mathpar}

\subsection{Additional notation}

It will sometimes be convenient to denote the process a name
quotes. We already have the notation $x = \quotep{P}$, but it will be
convenient to introduce an alternate notation, $\procn{x}$, when we
want to emphasize the connection to the use of the name. Note that, by
virtue of name equivalence, $\quotep{\procn{x}} \nameeq x$; so, the
notation is consistent with previous definitions.

Further, because names have structure it is possible to effect
substitutions on the basis of that structure. This means we need to
upgrade our notation for substitutions, which we accomplish by
adapting comprehension notation. Thus,

\begin{mathpar}
  P\{ y / x : x \in S \}
\end{mathpar}

is interpreted to mean the process derived from P by replacing (in a
capture-avoiding manner) each occurrence of $x$ in $S$ by $y$. For example,

\begin{mathpar}
  P\{ \quotep{\procn{x}|\procn{x}} / x : x \in \freenames{P} \}
\end{mathpar}

will replace each (occurrence) of a free name $x$ in $P$ by
$\quotep{\procn{x}|\procn{x}}$.

Also, we will avail ourselves of the notation $x^{L}$ and $x^{R}$ to
denote injections of a name into disjoint copies of the name
space. There are numerous ways to accomplish this. One example can be
found in \cite{MeredithR05}. This notation overloads to vectors of
names: $\vec{x}^{\pi} := (x_{i}^{\pi} \; : \; 0 \leq i < |\vec{x}| )$ where $\pi \in \{L,R\}$.

We also use $P^{\Box} := P|\Box$.

In \cite{MeredithR05} an interpretation of the new operator is
given. It turns out that there are several possible interpretations
all enjoying the requisite algebraic properties of the operator (see
\cite{milner91polyadicpi}). We will therefore make liberal use of
$(\nu\; \vec{x})P$.

% subsection the_syntax_and_semantics_of_the_notation_system (end)   

\input{qm2pi.qmops} 

\input{qm2pi.sterngerlach} 

\input{qm2pi.metric} 

% section concurrent_process_calculi (end)

%\input{qm2pi.proofsketch}

% section proof sketch (end)

%\input{qm2pi.slviaknots} 

% section spatial logic via knots (end)

\input{qm2pi.conclusion}

% section conclusion (end)

%\input{qm2pi.dtcodes} 

% section wiring algorithm (end)

\input{qm2pi.ack} 

% section acknowledgments (end)

\newpage


\bibliographystyle{plain}   
\bibliography{../../biblios/main.bib}

\input{qm2pi.rhodetails}

\end{document}

 

%\documentclass[12pt]{llncs}
%\documentclass{jktr}

\usepackage[pdftex]{hyperref}                   
\usepackage {listings}
\usepackage {mathpartir}
\usepackage{bcprules}
%\usepackage{listings}
                       
\usepackage{graphicx} 
%\usepackage[margins=2.5cm,nohead,nofoot]{geometry}
%\usepackage{geometry}
\usepackage{amsfonts}
\usepackage{amstext}
\usepackage{latexsym}
\usepackage{amssymb}
\usepackage{color}


%\include{myPreamble}
\include{qm2pi.local} 

%\ifpdf
%\usepackage[pdftex]{graphicx}
%\else
%\usepackage{graphicx}
%\fi

 % \ifpdf
%  \usepackage{pdfsync}
%  \if


%\title{Brief Article}
%\author{David F. Snyder}
%\author{L.G. Meredith}

%\address{Dept. of Math., Texas State University--San Marcos, San Marcos, TX 78666}
       
\pagestyle{empty}


\begin{document}

\lstset{language=[Objective]Caml,frame=shadowbox}

\input{qm2pi.front}

% section front matter (end)

\input{qm2pi.intro} 
 
% section introduction (end)

% \input{qm2pi.knotations} 

% section notation (end)

\input{qm2pi.process.calculi} 

% section concurrent_process_calculi_and_spatial_logics_ (end)
    
%\input{qm2pi.knots2pi} 

%\input{qm2pi.trefoil} 

%\input{qm2pi.mainthm} 

% subsection basic_interpretation (end)

%\input{qm2pi.rho.presentation} 
\subsection{The syntax and semantics of the notation system}\label{sub:the_syntax_and_semantics_of_the_notation_system} % (fold)

We now summarize a technical presentation of the calculus that
embodies our theory of dynamics. The typical presentation of such a
calculus follows the style of giving generators and relations on
them. The grammar, below, describing term constructors, freely
generates the set of processes, $\Proc$. This set is then quotiented
by a relation known as structural congruence and it is over this set
that the notion of dynamics is expressed. This presentation is
essentially that of \cite{MeredithR05} with the addition of
polyadicity and summation. For readability we have relegated some of
the technical subtleties to an appendix.

\subsubsection{Process grammar}\label{subsub:process_grammar}

\begin{mathpar}
  \inferrule* [lab=synchronization] {} {{M} \bc \pzero \;|\; x?F \;|\; x!C }
  \and
  \inferrule* [lab=abstraction] {} {{F} \bc (x)P}
  \and
  \inferrule* [lab=concretion] {} {{C} \bc \langle Q \rangle}
  \and
  \inferrule* [lab=process] {} {{P,Q} \bc M \;| \;P|Q \;|\; @{x}}
  \and
  \inferrule* [lab=name] {} {{x} \bc \quotep{P}}
\end{mathpar} 

Note that $\vec{x}$ (resp. $\vec{P}$) denotes a vector of names
(resp. processes) of length $|\vec{x}|$ (resp. $|\vec{P}|$). We adopt
the following useful abbreviations.

\begin{mathpar}
   x?(\vec{y}).P := x.(\vec{y})P \and  x\clift{\vec{P}} := x.\clift{\vec{P}}
   \and x!(y) := \lift{x}{\dropn{y}}
   \and \Pi_{i=0}^{n-1}P_i := P_0 | \ldots | P_{n-1}
\end{mathpar}

\subsubsection{Structural congruence}

\paragraph{Free and bound names and alpha-equivalence.} At the
core of structural equivalence is alpha-equivalence which identifies
process that are the same up to a change of variable. Formally, we
recognize the distinction between free and bound names. The free names
of a process, $\freenames{P}$, may be calculated recursively as
follows:

\begin{mathpar}
\freenames{\pzero} := \emptyset
  \and \\
  \freenames{x?(y).P} := \{ x \} \cup (\freenames{P} \setminus \{ y \})
  \and 
  \freenames{x!\langle P \rangle} := \{ x \} \cup \{ P \} 
  \and \\
  \freenames{P|Q} := \freenames{P} \cup \freenames{Q}
  \and \\
  \freenames{@{x}} := \{ x \}
\end{mathpar}

$\pi$
$\quotep{\pi}$

$\freenames{-} : \pi \to \mathcal{P}(\quotep{\pi})$

\begin{eqnarray*}
  \freenames{\pzero} & := & \emptyset \\
  \freenames{x?(y).P} & := & \{ x \} \cup (\freenames{P} \setminus \{ y \}) \\
  \freenames{x!\langle P \rangle} & := & \{ x \} \cup \{ P \} \\
  \freenames{P|Q} & := & \freenames{P} \cup \freenames{Q} \\
  \freenames{\dropn{x}} & := & \{ x \}
\end{eqnarray*}

The bound names of a process, $\boundnames{P}$, are those names occurring in $P$
that are not free. For example, in $x?(y).0$, the name $x$ is free, while $y$ is bound.

\begin{mathpar}
  \inferrule* [lab=monoidal-laws] {} { P|Q \equiv Q|P \and P|0 \equiv P \and P|(Q|R) \equiv (P|Q)|R }
\end{mathpar}

\begin{mathpar}
  \inferrule* [lab=alpha-equivalence] {} { (x)P \equiv (y)P\{y/x\} \and y \not\in \freenames{P} }
\end{mathpar}

\begin{definition}
Then two processes, $P,Q$, are alpha-equivalent if $P = Q\{\vec{y}/\vec{x}\}$ for
some $\vec{x} \in \boundnames{Q},\vec{y} \in \boundnames{P}$, where $Q\{\vec{y}/\vec{x}\}$
denotes the capture-avoiding substitution of $\vec{y}$ for $\vec{x}$ in $Q$.
\end{definition}

\begin{definition}
  The {\em structural congruence} \cite{SangiorgiWalker} , $\equiv$,
  between processes is the least congruence containing
  alpha-equivalence, satisfying the abelian monoid laws
  (associativity, commutativity and $\pzero$ as identity) for parallel
  composition $|$ and for summation $+$.
\end{definition}

\subsection{Name equivalence}

We take name equivalence, written $\nameeq$, to be the smallest
equivalence relation generated by the following rules.

\begin{mathpar}
\inferrule*[lab=Quote-drop]
{ }
{ \quotep{@{x}} \nameeq x }

\inferrule*[lab=Struct-equiv]
{ P \scong Q }
{ \quotep{P} \nameeq \quotep{Q} }
\end{mathpar}

The astute reader will have noticed that the mutual recursion of names
and processes imposes a mutual recursion on alpha-equivalence and
structural equivalence via name-equivalence. Fortunately, all of this
works out pleasantly and we may calculate in the natural way, free of
concern. The reader interested in the details is referred to the
appendix \ref{appendix:rho_details}.

\subsection{Substitution}

We use $\Proc$ for the set of processes, $\QProc$ for the set of
names, and $\id{\{}\vec{y} / \vec{x} \id{\}}$ to denote partial maps,
$s : \QProc \rightarrow \QProc$. A map, $s$ lifts, uniquely, to a map
on process terms, $\widehat{s} : \Proc \rightarrow \Proc$ by the
following equations.

\begin{mathpar}
  (0) \psubstp{Q}{P} := 0 \\
  (R \juxtap S) \psubstp{Q}{P}
  :=    
  (R)\psubstp{Q}{P} \juxtap (S) \psubstp{Q}{P} \\
  (x?(y).R) \psubstp{Q}{P}    
  :=    
  (x)\substp{Q}{P} (z)\concat( (R \psubstn{z}{y}) \psubstp{Q}{P} ) \\
  (\lift{x}{R}) \psubstp{Q}{P}  
  :=
  \lift{(x)\substp{Q}{P}}{ R \psubstp{Q}{P} } \\
%   (\dropn{x})  \psubstp{Q}{P}       
%   := 
%   \left\{ 
%     \begin{array}{ccc} 
%       \dropn{\quotep{Q}} & & x \nameeq \quotep{P} \\
%       \dropn{x} & & otherwise \\
%     \end{array}
%   \right. 
  (\dropn{x})  \psubstp{Q}{P}       
  := 
  \left\{ 
    \begin{array}{ccc} 
      Q & & x \nameeq \quotep{P} \\
      \dropn{x} & & otherwise \\
    \end{array}
  \right.
\end{mathpar}
 

where

\begin{eqnarray}
  (x)\id{\{} \lpquote Q \rpquote / \lpquote P \rpquote \id{\}}            = 
  \left\{ 
    \begin{array}{ccc}
      \lpquote Q \rpquote & & x \nameeq \lpquote P \rpquote \\
      x & & otherwise \\
    \end{array}
  \right. \nonumber
\end{eqnarray}

and $z$ is chosen distinct from $\quotep{P}$, $\quotep{Q}$, the free
names in $Q$, and all the names in $R$. Our $\alpha$-equivalence will
be built in the standard way from this substitution.

\begin{remark}\label{rem:no_self_referential_names}
  One consequence of these definitions is that $\forall P. \quotep{P}
  \not\in \freenames{P}$.
\end{remark}

\subsection{ Dynamic quote: an example }

Anticipating something of what's to come, consider applying the
substitution, $\widehat{\id{\{}u / z \id{\}}}$, to the following pair
of processes, $\lift{w}{y!(z)}$ and $w[ \lpquote y!(z) \rpquote ]$.

\begin{eqnarray}
	\lift{w}{y!(z)}\widehat{\id{\{}u / z \id{\}}}
		& = &
		\lift{w}{y!(u)} \nonumber\\
	w[ \lpquote y!(z) \rpquote ] \widehat{ \id{\{}u / z \id{\}} }
		& = &
		w[ \lpquote y!(z) \rpquote ] \nonumber
\end{eqnarray}

Because the body of the process between quotes is impervious to
substitution, we get radically different answers. In fact, by
examining the first process in an input context,
e.g. $x?(z).\lift{w}{y!(z)}$, we see that the process under the lift
operator may be shaped by prefixed inputs binding a name inside it. In
this sense, the lift operator will be seen as a way to dynamically
construct processes before reifying them as names.

Finally equipped with these standard features we can present the
dynamics of the calculus.

\subsubsection{Operational semantics} 

Finally, we introduce the computational dynamics. What marks these
algebras as distinct from other more traditionally studied algebraic
structures, e.g. vector spaces or polynomial rings, is the manner in
which dynamics is captured. In traditional structures, dynamics is typically
expressed through morphisms between such structures, as in linear maps
between vector spaces or morphisms between rings. In algebras
associated with the semantics of computation, the dynamics is
expressed as part of the algebraic structure itself, through a
reduction reduction relation typically denoted by $\red$. Below, we
give a recursive presentation of this relation for the calculus used
in the encoding.

$\red \subseteq \pi \times \pi$
$\red : \pi \to \mathcal{P}(\pi)$

\begin{mathpar}
  \inferrule* [lab=Comm] { \textsf{match}( x_{src}, x_{trgt} ) } { x_{trgt}?(y)P \; | \; x_{src}!\langle {Q} \rangle \red P\{\quotep{Q}/y}\} }
  \and \\
  \inferrule* [lab=Par] {{P} \red {P}'} {{{P} | {Q}} \red {{P}' | {Q}}}
  \and
  \inferrule* [lab=Equiv]{{{P} \scong {P}'} \andalso {{P}' \red {Q}'} \andalso {{Q}' \scong {Q}}}{{P} \red {Q}}
\end{mathpar}

\begin{eqnarray*}
  match_{\equiv} (\quotep{P},\quotep{Q}) & := & P \equiv Q \\
  match_{\dagger}(\quotep{P},\quotep{Q}) & := & \forall R. P|Q \red^{*} R => R \red^{*} 0 \\
  match_{K}(\quotep{P},\quotep{Q}) & := & K \mbox{ for some context } K
\end{eqnarray*}

$u?(x)P | u!\langle Q \rangle \red P\{\quotep{Q}/x\}$

%We write $\wred$ for $\red^*$, and $P\red$ if $\exists Q $ such that $ P \red Q$.
We write $P\red$ if $\exists Q $ such that $ P \red Q$ and $P\not\red$, otherwise.

\section{Replication}

As mentioned before, it is known that replication (and hence
recursion) can be implemented in a higher-order process algebra
\cite{SangiorgiWalker}. As our first example of calculation with the
machinery thus far presented we give the construction explicitly in
the {\rhoc}.

\begin{eqnarray}
	D_{x} & := & \prefix{x}{y}{(\binpar{\outputp{x}{y}}{@{y}})} \nonumber\\
	\bangp_{x}{P} & := & \binpar{{x}!\langle{\binpar{D_{x}}{P}}\rangle}{D_{x}} \nonumber
\end{eqnarray}

\begin{eqnarray}
	\bangp_{x}{P} & & \nonumber\\
	=
	& {x}!\langle{(\prefix{x}{y}{(\outputp{x}{y} | @{y})) | P}}\rangle 
	      | \prefix{x}{y}{(\outputp{x}{y} | @{y})} & \nonumber\\
	\red
	& (\outputp{x}{y} | @{y})\substn{\quotep{(\prefix{x}{y}{(@{y} | \outputp{x}{y})) | P}}}{y} & \nonumber\\
	=
	& \outputp{x}{\quotep{(\prefix{x}{y}{(\outputp{x}{y} | @{y})) | P}}}
	  | {(\prefix{x}{y}{(\outputp{x}{y} | @{y})) | P}} & \nonumber\\
	\red
	& \ldots & \nonumber\\
	\red^*
	& P | P | \ldots & \nonumber
\end{eqnarray}

Of course, this encoding, as an implementation, runs away, unfolding
$\bangp{P}$ eagerly. A lazier and more implementable replication
operator, restricted to input-guarded processes, may be obtained as follows.

\begin{eqnarray}
\bangp{\prefix{u}{v}{P}} 
	:= 
	\binpar{\lift{x}{\prefix{u}{v}{(\binpar{D(x)}{P})}}}{D(x)} \nonumber
\end{eqnarray}

\begin{remark}
  Note that the lazier definition still does not deal with summation
  or mixed summation (i.e. sums over input and output). The reader is
  invited to construct definitions of replication that deal with these
  features. 

  Further, the definitions are parameterized in a name, $x$. Can you,
  gentle reader, make a definition that eliminates this parameter and
  guarantees no accidental interaction between the replication
  machinery and the process being replicated -- i.e. no accidental
  sharing of names used by the process to get its work done and the
  name(s) used by the replication to effect copying. This latter
  revision of the definition of replication is crucial to obtaining
  the expected identity $!!P \sim !P$.
\end{remark}

\begin{remark}\label{rem:paradoxical_combinator}
  The reader familiar with the lambda calculus will have noticed the
  similarity between $D$ and the paradoxical combinator.

  [Ed. note: the existence of this seems to suggest we have to be more
  restrictive on the set of processes and names we admit if we are to
  support no-cloning.]
\end{remark}

\subsubsection{Bisimulation}

The computational dynamics gives rise to another kind of equivalence,
the equivalence of computational behavior. As previously mentioned
this is typically captured \emph{via} some form of bisimulation.

% The notion we use in this paper is weak barbed bisimulation
% \cite{milner91polyadicpi}.

The notion we use in this paper is derived from weak barbed
bisimulation \cite{milner91polyadicpi}. 

\begin{definition}
An \emph{observation relation}, $\downarrow_{\mathcal N}$, over a set
of names, $\mathcal N$, is the smallest relation satisfying the rules
below.

\infrule[Out-barb]{y \in {\mathcal N}, \; x \nameeq y}
		  {\outputp{x}{v} \downarrow_{\mathcal N} x}
\infrule[Par-barb]{\mbox{$P\downarrow_{\mathcal N} x$ or $Q\downarrow_{\mathcal N} x$}}
		  {\binpar{P}{Q} \downarrow_{\mathcal N} x}

We write $P \Downarrow_{\mathcal N} x$ if there is $Q$ such that 
$P \wred Q$ and $Q \downarrow_{\mathcal N} x$.
\end{definition}

\begin{definition}
%\label{def.bbisim}
An  ${\mathcal N}$-\emph{barbed bisimulation} over a set of names, ${\mathcal N}$, is a symmetric binary relation 
${\mathcal S}_{\mathcal N}$ between agents such that $P\rel{S}_{\mathcal N}Q$ implies:
\begin{enumerate}
\item If $P \red P'$ then $Q \wred Q'$ and $P'\rel{S}_{\mathcal N} Q'$.
\item If $P\downarrow_{\mathcal N} x$, then $Q\Downarrow_{\mathcal N} x$.
\end{enumerate}
$P$ is ${\mathcal N}$-barbed bisimilar to $Q$, written
$P \wbbisim_{\mathcal N} Q$, if $P \rel{S}_{\mathcal N} Q$ for some ${\mathcal N}$-barbed bisimulation ${\mathcal S}_{\mathcal N}$.
\end{definition}

$\mathcal{R} \subseteq \pi \times \pi$

$P \mathcal{R} Q => \forall P'. P \red P' \Rightarrow \exists Q'. Q \red Q', P' \mathcal{R} Q'$

$P \vdash x \Rightarrow Q \vdash x$

\begin{mathpar}
  \inferrule*[lab=Out-barb]{x \nameeq y}{{y}!\langle{Q}\rangle \vdash x}
  \and
  \inferrule*[lab=Par-barb]{\mbox{$P\vdash x$ or $Q\vdash x$}}{\binpar{P}{Q} \vdash x}
\end{mathpar}

\subsubsection{Contexts}

One of the principle advantages of computational calculi like the
$\pi$-calculus is a well-defined notion of context,
contextual-equivalence and a correlation between
contextual-equivalence and notions of bisimulation. The notion of
context allows the decomposition of a process into (sub-)process and
its syntactic environment, its context. Thus, a context may be
thought of as a process with a ``hole'' (written $\Box$) in it. The
application of a context $M$ to a process $P$, written $M[P]$, is
tantamount to filling the hole in $M$ with $P$. In this paper we do
not need the full weight of this theory, but do make use of the notion
of context in the proof the main theorem. 

\begin{mathpar}
  \inferrule* [lab=summation] {} {{M_{M},M_{N}} \bc \Box \;|\; x.M_{A} \;|\; M_{M}+M_{N}}
  \and
  \inferrule* [lab=agent] {} {{M_{A}} \bc (\vec{x})M_{P} \;| \; \clift{P_0,\ldots,M_{P},\ldots,P_N}}
  \and \\
  \inferrule* [lab=process] {} {{M_{P}} \bc M_{N} \;| \;P|M_{P} }
\end{mathpar} 

\begin{mathpar}
  \inferrule* [lab=sychronization] {} {M_{N} \bc \Box \;|\; x?M_{F} \;|\; x!M_{C}}
  \and
  \inferrule* [lab=abstraction] {} {{M_{F}} \bc (x)M_{P} }
  \and
  \inferrule* [lab=concretion] {} {{M_{C}} \bc \langle M_{P} \rangle }
  \and \\
  \inferrule* [lab=process] {} {{M_{P}} \bc M_{N} \;| \;P|M_{P} }
\end{mathpar}

\begin{definition}[contextual application] Given a context $M$, and
  process $P$, we define the \emph{contextual application}, $M[P] :=
  M\{P/\Box\}$. That is, the contextual application of M to P is the
  substitution of $P$ for $\Box$ in $M$.
\end{definition}

$\meaningof{-} : L \to \mathcal{P}(\pi)$

\begin{mathpar}
  \inferrule* [lab=collection] {} {\meaningof{true} = \pi, \and \meaningof{~E} = \pi \setminus \meaningof{E}, \and \meaningof{E_{1} \& E_{2}} = \meaningof{E_{1}} \cap \meaningof{E_{2}}}
\end{mathpar}

\begin{mathpar}
  \inferrule* [lab=structure] {} {\meaningof{0} = \{ P \in \pi | P \equiv 0 \}, \and \\ \meaningof{E_1 | E_2} = \{ P \in \pi | P \equiv P_{1} | P_{2}, P_{1} \in \meaningof{E_{1}}, P_{2} \in \meaningof{E_2}\} }
\end{mathpar}

\begin{mathpar}
 \inferrule* [lab=behavior] {} {\meaningof{\langle a?b \rangle E} = \{ P \in \pi | P \equiv Q | u?(y)P', \\ \and \\\\ \and \\ \;\;\; u \in \meaningof{a}, \forall z.P'\{z/y\} \in \meaningof{E\{z/b\}}\}, \and \\ \meaningof{a!E} = \{ P \in \pi | P \equiv Q | x!\langle P' \rangle, x \in \meaningof{a} P' \in \meaningof{E}\} }
\end{mathpar}

\begin{mathpar}
 \inferrule* [lab=nominal] {} {\meaningof{\quotep{E}} = \{ \quotep{P} \in \quotep{\pi} | P \in \meaningof{E} \}, \and \meaningof{\quotep{P}} = \{ \quotep{Q} \in \quotep{\pi} | P \equiv Q \} \and \\ \meaningof{@\quotep{E}} = \{ P \in \pi | P \equiv @x, x \in \meaningof{E} \}}
\end{mathpar}

\begin{eqnarray*}
  \\
  \meaningof{-} : TS \to ST
\end{eqnarray*}

\begin{eqnarray*}
  \\
  L : TS \to ST
\end{eqnarray*}

\begin{eqnarray*}
  \\
  P \models E \iff P \in \meaningof{E}
\end{eqnarray*}

\begin{eqnarray*}
  P \approx_{L} Q \iff \forall E \in L. P \models E \iff Q \models E
\end{eqnarray*}

\begin{eqnarray*}
  P \approx_{K} Q
\end{eqnarray*}

\begin{eqnarray*}
  P \approx Q
\end{eqnarray*}

$\approx_{K} = \approx = \approx_{L}$

\subsubsection{Contextual duality}

Note that contexts extend the quotation operation to a family of
operations from processes to names. Given a context, $M$, we can
define a \emph{nominal context}, $\quotep{M}$ by $\quotep{M}[P] :=
\quotep{M[P]}$. To foreshadow what is to come we observe that these
operations enjoy a duality with processes very much like the duality
between vectors and maps from vectors to scalars.

Further, because the calculus is essentially higher-order, we have a
correspondence between contexts and processes. More specifically,
given a name $x$ and a context $M$ we can construct $M^{*}_{x}$ such
that 

\begin{mathpar}
  M^{*}_{x} | \lift{x}{P} \red M[P]
\end{mathpar}

namely,

\begin{mathpar}
  M^{*}_{x} := x?(u).M[\dropn{u}]
\end{mathpar}

The dependence of $M^{*}_{x}$ on a name makes it an abstraction, 

\begin{mathpar}
  M^{*} := (x)x?(u).M[\dropn{u}]
\end{mathpar}

\subsection{Additional notation}

It will sometimes be convenient to denote the process a name
quotes. We already have the notation $x = \quotep{P}$, but it will be
convenient to introduce an alternate notation, $\procn{x}$, when we
want to emphasize the connection to the use of the name. Note that, by
virtue of name equivalence, $\quotep{\procn{x}} \nameeq x$; so, the
notation is consistent with previous definitions.

Further, because names have structure it is possible to effect
substitutions on the basis of that structure. This means we need to
upgrade our notation for substitutions, which we accomplish by
adapting comprehension notation. Thus,

\begin{mathpar}
  P\{ y / x : x \in S \}
\end{mathpar}

is interpreted to mean the process derived from P by replacing (in a
capture-avoiding manner) each occurrence of $x$ in $S$ by $y$. For example,

\begin{mathpar}
  P\{ \quotep{\procn{x}|\procn{x}} / x : x \in \freenames{P} \}
\end{mathpar}

will replace each (occurrence) of a free name $x$ in $P$ by
$\quotep{\procn{x}|\procn{x}}$.

Also, we will avail ourselves of the notation $x^{L}$ and $x^{R}$ to
denote injections of a name into disjoint copies of the name
space. There are numerous ways to accomplish this. One example can be
found in \cite{MeredithR05}. This notation overloads to vectors of
names: $\vec{x}^{\pi} := (x_{i}^{\pi} \; : \; 0 \leq i < |\vec{x}| )$ where $\pi \in \{L,R\}$.

We also use $P^{\Box} := P|\Box$.

In \cite{MeredithR05} an interpretation of the new operator is
given. It turns out that there are several possible interpretations
all enjoying the requisite algebraic properties of the operator (see
\cite{milner91polyadicpi}). We will therefore make liberal use of
$(\nu\; \vec{x})P$.

% subsection the_syntax_and_semantics_of_the_notation_system (end)   

\input{qm2pi.qmops} 

\input{qm2pi.sterngerlach} 

\input{qm2pi.metric} 

% section concurrent_process_calculi (end)

%\input{qm2pi.proofsketch}

% section proof sketch (end)

%\input{qm2pi.slviaknots} 

% section spatial logic via knots (end)

\input{qm2pi.conclusion}

% section conclusion (end)

%\input{qm2pi.dtcodes} 

% section wiring algorithm (end)

\input{qm2pi.ack} 

% section acknowledgments (end)

\newpage


\bibliographystyle{plain}   
\bibliography{../../biblios/main.bib}

\input{qm2pi.rhodetails}

\end{document}

 

%\documentclass[12pt]{llncs}
%\documentclass{jktr}

\usepackage[pdftex]{hyperref}                   
\usepackage {listings}
\usepackage {mathpartir}
\usepackage{bcprules}
%\usepackage{listings}
                       
\usepackage{graphicx} 
%\usepackage[margins=2.5cm,nohead,nofoot]{geometry}
%\usepackage{geometry}
\usepackage{amsfonts}
\usepackage{amstext}
\usepackage{latexsym}
\usepackage{amssymb}
\usepackage{color}


%\include{myPreamble}
\include{qm2pi.local} 

%\ifpdf
%\usepackage[pdftex]{graphicx}
%\else
%\usepackage{graphicx}
%\fi

 % \ifpdf
%  \usepackage{pdfsync}
%  \if


%\title{Brief Article}
%\author{David F. Snyder}
%\author{L.G. Meredith}

%\address{Dept. of Math., Texas State University--San Marcos, San Marcos, TX 78666}
       
\pagestyle{empty}


\begin{document}

\lstset{language=[Objective]Caml,frame=shadowbox}

\input{qm2pi.front}

% section front matter (end)

\input{qm2pi.intro} 
 
% section introduction (end)

% \input{qm2pi.knotations} 

% section notation (end)

\input{qm2pi.process.calculi} 

% section concurrent_process_calculi_and_spatial_logics_ (end)
    
%\input{qm2pi.knots2pi} 

%\input{qm2pi.trefoil} 

%\input{qm2pi.mainthm} 

% subsection basic_interpretation (end)

%\input{qm2pi.rho.presentation} 
\subsection{The syntax and semantics of the notation system}\label{sub:the_syntax_and_semantics_of_the_notation_system} % (fold)

We now summarize a technical presentation of the calculus that
embodies our theory of dynamics. The typical presentation of such a
calculus follows the style of giving generators and relations on
them. The grammar, below, describing term constructors, freely
generates the set of processes, $\Proc$. This set is then quotiented
by a relation known as structural congruence and it is over this set
that the notion of dynamics is expressed. This presentation is
essentially that of \cite{MeredithR05} with the addition of
polyadicity and summation. For readability we have relegated some of
the technical subtleties to an appendix.

\subsubsection{Process grammar}\label{subsub:process_grammar}

\begin{mathpar}
  \inferrule* [lab=synchronization] {} {{M} \bc \pzero \;|\; x?F \;|\; x!C }
  \and
  \inferrule* [lab=abstraction] {} {{F} \bc (x)P}
  \and
  \inferrule* [lab=concretion] {} {{C} \bc \langle Q \rangle}
  \and
  \inferrule* [lab=process] {} {{P,Q} \bc M \;| \;P|Q \;|\; @{x}}
  \and
  \inferrule* [lab=name] {} {{x} \bc \quotep{P}}
\end{mathpar} 

Note that $\vec{x}$ (resp. $\vec{P}$) denotes a vector of names
(resp. processes) of length $|\vec{x}|$ (resp. $|\vec{P}|$). We adopt
the following useful abbreviations.

\begin{mathpar}
   x?(\vec{y}).P := x.(\vec{y})P \and  x\clift{\vec{P}} := x.\clift{\vec{P}}
   \and x!(y) := \lift{x}{\dropn{y}}
   \and \Pi_{i=0}^{n-1}P_i := P_0 | \ldots | P_{n-1}
\end{mathpar}

\subsubsection{Structural congruence}

\paragraph{Free and bound names and alpha-equivalence.} At the
core of structural equivalence is alpha-equivalence which identifies
process that are the same up to a change of variable. Formally, we
recognize the distinction between free and bound names. The free names
of a process, $\freenames{P}$, may be calculated recursively as
follows:

\begin{mathpar}
\freenames{\pzero} := \emptyset
  \and \\
  \freenames{x?(y).P} := \{ x \} \cup (\freenames{P} \setminus \{ y \})
  \and 
  \freenames{x!\langle P \rangle} := \{ x \} \cup \{ P \} 
  \and \\
  \freenames{P|Q} := \freenames{P} \cup \freenames{Q}
  \and \\
  \freenames{@{x}} := \{ x \}
\end{mathpar}

$\pi$
$\quotep{\pi}$

$\freenames{-} : \pi \to \mathcal{P}(\quotep{\pi})$

\begin{eqnarray*}
  \freenames{\pzero} & := & \emptyset \\
  \freenames{x?(y).P} & := & \{ x \} \cup (\freenames{P} \setminus \{ y \}) \\
  \freenames{x!\langle P \rangle} & := & \{ x \} \cup \{ P \} \\
  \freenames{P|Q} & := & \freenames{P} \cup \freenames{Q} \\
  \freenames{\dropn{x}} & := & \{ x \}
\end{eqnarray*}

The bound names of a process, $\boundnames{P}$, are those names occurring in $P$
that are not free. For example, in $x?(y).0$, the name $x$ is free, while $y$ is bound.

\begin{mathpar}
  \inferrule* [lab=monoidal-laws] {} { P|Q \equiv Q|P \and P|0 \equiv P \and P|(Q|R) \equiv (P|Q)|R }
\end{mathpar}

\begin{mathpar}
  \inferrule* [lab=alpha-equivalence] {} { (x)P \equiv (y)P\{y/x\} \and y \not\in \freenames{P} }
\end{mathpar}

\begin{definition}
Then two processes, $P,Q$, are alpha-equivalent if $P = Q\{\vec{y}/\vec{x}\}$ for
some $\vec{x} \in \boundnames{Q},\vec{y} \in \boundnames{P}$, where $Q\{\vec{y}/\vec{x}\}$
denotes the capture-avoiding substitution of $\vec{y}$ for $\vec{x}$ in $Q$.
\end{definition}

\begin{definition}
  The {\em structural congruence} \cite{SangiorgiWalker} , $\equiv$,
  between processes is the least congruence containing
  alpha-equivalence, satisfying the abelian monoid laws
  (associativity, commutativity and $\pzero$ as identity) for parallel
  composition $|$ and for summation $+$.
\end{definition}

\subsection{Name equivalence}

We take name equivalence, written $\nameeq$, to be the smallest
equivalence relation generated by the following rules.

\begin{mathpar}
\inferrule*[lab=Quote-drop]
{ }
{ \quotep{@{x}} \nameeq x }

\inferrule*[lab=Struct-equiv]
{ P \scong Q }
{ \quotep{P} \nameeq \quotep{Q} }
\end{mathpar}

The astute reader will have noticed that the mutual recursion of names
and processes imposes a mutual recursion on alpha-equivalence and
structural equivalence via name-equivalence. Fortunately, all of this
works out pleasantly and we may calculate in the natural way, free of
concern. The reader interested in the details is referred to the
appendix \ref{appendix:rho_details}.

\subsection{Substitution}

We use $\Proc$ for the set of processes, $\QProc$ for the set of
names, and $\id{\{}\vec{y} / \vec{x} \id{\}}$ to denote partial maps,
$s : \QProc \rightarrow \QProc$. A map, $s$ lifts, uniquely, to a map
on process terms, $\widehat{s} : \Proc \rightarrow \Proc$ by the
following equations.

\begin{mathpar}
  (0) \psubstp{Q}{P} := 0 \\
  (R \juxtap S) \psubstp{Q}{P}
  :=    
  (R)\psubstp{Q}{P} \juxtap (S) \psubstp{Q}{P} \\
  (x?(y).R) \psubstp{Q}{P}    
  :=    
  (x)\substp{Q}{P} (z)\concat( (R \psubstn{z}{y}) \psubstp{Q}{P} ) \\
  (\lift{x}{R}) \psubstp{Q}{P}  
  :=
  \lift{(x)\substp{Q}{P}}{ R \psubstp{Q}{P} } \\
%   (\dropn{x})  \psubstp{Q}{P}       
%   := 
%   \left\{ 
%     \begin{array}{ccc} 
%       \dropn{\quotep{Q}} & & x \nameeq \quotep{P} \\
%       \dropn{x} & & otherwise \\
%     \end{array}
%   \right. 
  (\dropn{x})  \psubstp{Q}{P}       
  := 
  \left\{ 
    \begin{array}{ccc} 
      Q & & x \nameeq \quotep{P} \\
      \dropn{x} & & otherwise \\
    \end{array}
  \right.
\end{mathpar}
 

where

\begin{eqnarray}
  (x)\id{\{} \lpquote Q \rpquote / \lpquote P \rpquote \id{\}}            = 
  \left\{ 
    \begin{array}{ccc}
      \lpquote Q \rpquote & & x \nameeq \lpquote P \rpquote \\
      x & & otherwise \\
    \end{array}
  \right. \nonumber
\end{eqnarray}

and $z$ is chosen distinct from $\quotep{P}$, $\quotep{Q}$, the free
names in $Q$, and all the names in $R$. Our $\alpha$-equivalence will
be built in the standard way from this substitution.

\begin{remark}\label{rem:no_self_referential_names}
  One consequence of these definitions is that $\forall P. \quotep{P}
  \not\in \freenames{P}$.
\end{remark}

\subsection{ Dynamic quote: an example }

Anticipating something of what's to come, consider applying the
substitution, $\widehat{\id{\{}u / z \id{\}}}$, to the following pair
of processes, $\lift{w}{y!(z)}$ and $w[ \lpquote y!(z) \rpquote ]$.

\begin{eqnarray}
	\lift{w}{y!(z)}\widehat{\id{\{}u / z \id{\}}}
		& = &
		\lift{w}{y!(u)} \nonumber\\
	w[ \lpquote y!(z) \rpquote ] \widehat{ \id{\{}u / z \id{\}} }
		& = &
		w[ \lpquote y!(z) \rpquote ] \nonumber
\end{eqnarray}

Because the body of the process between quotes is impervious to
substitution, we get radically different answers. In fact, by
examining the first process in an input context,
e.g. $x?(z).\lift{w}{y!(z)}$, we see that the process under the lift
operator may be shaped by prefixed inputs binding a name inside it. In
this sense, the lift operator will be seen as a way to dynamically
construct processes before reifying them as names.

Finally equipped with these standard features we can present the
dynamics of the calculus.

\subsubsection{Operational semantics} 

Finally, we introduce the computational dynamics. What marks these
algebras as distinct from other more traditionally studied algebraic
structures, e.g. vector spaces or polynomial rings, is the manner in
which dynamics is captured. In traditional structures, dynamics is typically
expressed through morphisms between such structures, as in linear maps
between vector spaces or morphisms between rings. In algebras
associated with the semantics of computation, the dynamics is
expressed as part of the algebraic structure itself, through a
reduction reduction relation typically denoted by $\red$. Below, we
give a recursive presentation of this relation for the calculus used
in the encoding.

$\red \subseteq \pi \times \pi$
$\red : \pi \to \mathcal{P}(\pi)$

\begin{mathpar}
  \inferrule* [lab=Comm] { \textsf{match}( x_{src}, x_{trgt} ) } { x_{trgt}?(y)P \; | \; x_{src}!\langle {Q} \rangle \red P\{\quotep{Q}/y}\} }
  \and \\
  \inferrule* [lab=Par] {{P} \red {P}'} {{{P} | {Q}} \red {{P}' | {Q}}}
  \and
  \inferrule* [lab=Equiv]{{{P} \scong {P}'} \andalso {{P}' \red {Q}'} \andalso {{Q}' \scong {Q}}}{{P} \red {Q}}
\end{mathpar}

\begin{eqnarray*}
  match_{\equiv} (\quotep{P},\quotep{Q}) & := & P \equiv Q \\
  match_{\dagger}(\quotep{P},\quotep{Q}) & := & \forall R. P|Q \red^{*} R => R \red^{*} 0 \\
  match_{K}(\quotep{P},\quotep{Q}) & := & K \mbox{ for some context } K
\end{eqnarray*}

$u?(x)P | u!\langle Q \rangle \red P\{\quotep{Q}/x\}$

%We write $\wred$ for $\red^*$, and $P\red$ if $\exists Q $ such that $ P \red Q$.
We write $P\red$ if $\exists Q $ such that $ P \red Q$ and $P\not\red$, otherwise.

\section{Replication}

As mentioned before, it is known that replication (and hence
recursion) can be implemented in a higher-order process algebra
\cite{SangiorgiWalker}. As our first example of calculation with the
machinery thus far presented we give the construction explicitly in
the {\rhoc}.

\begin{eqnarray}
	D_{x} & := & \prefix{x}{y}{(\binpar{\outputp{x}{y}}{@{y}})} \nonumber\\
	\bangp_{x}{P} & := & \binpar{{x}!\langle{\binpar{D_{x}}{P}}\rangle}{D_{x}} \nonumber
\end{eqnarray}

\begin{eqnarray}
	\bangp_{x}{P} & & \nonumber\\
	=
	& {x}!\langle{(\prefix{x}{y}{(\outputp{x}{y} | @{y})) | P}}\rangle 
	      | \prefix{x}{y}{(\outputp{x}{y} | @{y})} & \nonumber\\
	\red
	& (\outputp{x}{y} | @{y})\substn{\quotep{(\prefix{x}{y}{(@{y} | \outputp{x}{y})) | P}}}{y} & \nonumber\\
	=
	& \outputp{x}{\quotep{(\prefix{x}{y}{(\outputp{x}{y} | @{y})) | P}}}
	  | {(\prefix{x}{y}{(\outputp{x}{y} | @{y})) | P}} & \nonumber\\
	\red
	& \ldots & \nonumber\\
	\red^*
	& P | P | \ldots & \nonumber
\end{eqnarray}

Of course, this encoding, as an implementation, runs away, unfolding
$\bangp{P}$ eagerly. A lazier and more implementable replication
operator, restricted to input-guarded processes, may be obtained as follows.

\begin{eqnarray}
\bangp{\prefix{u}{v}{P}} 
	:= 
	\binpar{\lift{x}{\prefix{u}{v}{(\binpar{D(x)}{P})}}}{D(x)} \nonumber
\end{eqnarray}

\begin{remark}
  Note that the lazier definition still does not deal with summation
  or mixed summation (i.e. sums over input and output). The reader is
  invited to construct definitions of replication that deal with these
  features. 

  Further, the definitions are parameterized in a name, $x$. Can you,
  gentle reader, make a definition that eliminates this parameter and
  guarantees no accidental interaction between the replication
  machinery and the process being replicated -- i.e. no accidental
  sharing of names used by the process to get its work done and the
  name(s) used by the replication to effect copying. This latter
  revision of the definition of replication is crucial to obtaining
  the expected identity $!!P \sim !P$.
\end{remark}

\begin{remark}\label{rem:paradoxical_combinator}
  The reader familiar with the lambda calculus will have noticed the
  similarity between $D$ and the paradoxical combinator.

  [Ed. note: the existence of this seems to suggest we have to be more
  restrictive on the set of processes and names we admit if we are to
  support no-cloning.]
\end{remark}

\subsubsection{Bisimulation}

The computational dynamics gives rise to another kind of equivalence,
the equivalence of computational behavior. As previously mentioned
this is typically captured \emph{via} some form of bisimulation.

% The notion we use in this paper is weak barbed bisimulation
% \cite{milner91polyadicpi}.

The notion we use in this paper is derived from weak barbed
bisimulation \cite{milner91polyadicpi}. 

\begin{definition}
An \emph{observation relation}, $\downarrow_{\mathcal N}$, over a set
of names, $\mathcal N$, is the smallest relation satisfying the rules
below.

\infrule[Out-barb]{y \in {\mathcal N}, \; x \nameeq y}
		  {\outputp{x}{v} \downarrow_{\mathcal N} x}
\infrule[Par-barb]{\mbox{$P\downarrow_{\mathcal N} x$ or $Q\downarrow_{\mathcal N} x$}}
		  {\binpar{P}{Q} \downarrow_{\mathcal N} x}

We write $P \Downarrow_{\mathcal N} x$ if there is $Q$ such that 
$P \wred Q$ and $Q \downarrow_{\mathcal N} x$.
\end{definition}

\begin{definition}
%\label{def.bbisim}
An  ${\mathcal N}$-\emph{barbed bisimulation} over a set of names, ${\mathcal N}$, is a symmetric binary relation 
${\mathcal S}_{\mathcal N}$ between agents such that $P\rel{S}_{\mathcal N}Q$ implies:
\begin{enumerate}
\item If $P \red P'$ then $Q \wred Q'$ and $P'\rel{S}_{\mathcal N} Q'$.
\item If $P\downarrow_{\mathcal N} x$, then $Q\Downarrow_{\mathcal N} x$.
\end{enumerate}
$P$ is ${\mathcal N}$-barbed bisimilar to $Q$, written
$P \wbbisim_{\mathcal N} Q$, if $P \rel{S}_{\mathcal N} Q$ for some ${\mathcal N}$-barbed bisimulation ${\mathcal S}_{\mathcal N}$.
\end{definition}

$\mathcal{R} \subseteq \pi \times \pi$

$P \mathcal{R} Q => \forall P'. P \red P' \Rightarrow \exists Q'. Q \red Q', P' \mathcal{R} Q'$

$P \vdash x \Rightarrow Q \vdash x$

\begin{mathpar}
  \inferrule*[lab=Out-barb]{x \nameeq y}{{y}!\langle{Q}\rangle \vdash x}
  \and
  \inferrule*[lab=Par-barb]{\mbox{$P\vdash x$ or $Q\vdash x$}}{\binpar{P}{Q} \vdash x}
\end{mathpar}

\subsubsection{Contexts}

One of the principle advantages of computational calculi like the
$\pi$-calculus is a well-defined notion of context,
contextual-equivalence and a correlation between
contextual-equivalence and notions of bisimulation. The notion of
context allows the decomposition of a process into (sub-)process and
its syntactic environment, its context. Thus, a context may be
thought of as a process with a ``hole'' (written $\Box$) in it. The
application of a context $M$ to a process $P$, written $M[P]$, is
tantamount to filling the hole in $M$ with $P$. In this paper we do
not need the full weight of this theory, but do make use of the notion
of context in the proof the main theorem. 

\begin{mathpar}
  \inferrule* [lab=summation] {} {{M_{M},M_{N}} \bc \Box \;|\; x.M_{A} \;|\; M_{M}+M_{N}}
  \and
  \inferrule* [lab=agent] {} {{M_{A}} \bc (\vec{x})M_{P} \;| \; \clift{P_0,\ldots,M_{P},\ldots,P_N}}
  \and \\
  \inferrule* [lab=process] {} {{M_{P}} \bc M_{N} \;| \;P|M_{P} }
\end{mathpar} 

\begin{mathpar}
  \inferrule* [lab=sychronization] {} {M_{N} \bc \Box \;|\; x?M_{F} \;|\; x!M_{C}}
  \and
  \inferrule* [lab=abstraction] {} {{M_{F}} \bc (x)M_{P} }
  \and
  \inferrule* [lab=concretion] {} {{M_{C}} \bc \langle M_{P} \rangle }
  \and \\
  \inferrule* [lab=process] {} {{M_{P}} \bc M_{N} \;| \;P|M_{P} }
\end{mathpar}

\begin{definition}[contextual application] Given a context $M$, and
  process $P$, we define the \emph{contextual application}, $M[P] :=
  M\{P/\Box\}$. That is, the contextual application of M to P is the
  substitution of $P$ for $\Box$ in $M$.
\end{definition}

$\meaningof{-} : L \to \mathcal{P}(\pi)$

\begin{mathpar}
  \inferrule* [lab=collection] {} {\meaningof{true} = \pi, \and \meaningof{~E} = \pi \setminus \meaningof{E}, \and \meaningof{E_{1} \& E_{2}} = \meaningof{E_{1}} \cap \meaningof{E_{2}}}
\end{mathpar}

\begin{mathpar}
  \inferrule* [lab=structure] {} {\meaningof{0} = \{ P \in \pi | P \equiv 0 \}, \and \\ \meaningof{E_1 | E_2} = \{ P \in \pi | P \equiv P_{1} | P_{2}, P_{1} \in \meaningof{E_{1}}, P_{2} \in \meaningof{E_2}\} }
\end{mathpar}

\begin{mathpar}
 \inferrule* [lab=behavior] {} {\meaningof{\langle a?b \rangle E} = \{ P \in \pi | P \equiv Q | u?(y)P', \\ \and \\\\ \and \\ \;\;\; u \in \meaningof{a}, \forall z.P'\{z/y\} \in \meaningof{E\{z/b\}}\}, \and \\ \meaningof{a!E} = \{ P \in \pi | P \equiv Q | x!\langle P' \rangle, x \in \meaningof{a} P' \in \meaningof{E}\} }
\end{mathpar}

\begin{mathpar}
 \inferrule* [lab=nominal] {} {\meaningof{\quotep{E}} = \{ \quotep{P} \in \quotep{\pi} | P \in \meaningof{E} \}, \and \meaningof{\quotep{P}} = \{ \quotep{Q} \in \quotep{\pi} | P \equiv Q \} \and \\ \meaningof{@\quotep{E}} = \{ P \in \pi | P \equiv @x, x \in \meaningof{E} \}}
\end{mathpar}

\begin{eqnarray*}
  \\
  \meaningof{-} : TS \to ST
\end{eqnarray*}

\begin{eqnarray*}
  \\
  L : TS \to ST
\end{eqnarray*}

\begin{eqnarray*}
  \\
  P \models E \iff P \in \meaningof{E}
\end{eqnarray*}

\begin{eqnarray*}
  P \approx_{L} Q \iff \forall E \in L. P \models E \iff Q \models E
\end{eqnarray*}

\begin{eqnarray*}
  P \approx_{K} Q
\end{eqnarray*}

\begin{eqnarray*}
  P \approx Q
\end{eqnarray*}

$\approx_{K} = \approx = \approx_{L}$

\subsubsection{Contextual duality}

Note that contexts extend the quotation operation to a family of
operations from processes to names. Given a context, $M$, we can
define a \emph{nominal context}, $\quotep{M}$ by $\quotep{M}[P] :=
\quotep{M[P]}$. To foreshadow what is to come we observe that these
operations enjoy a duality with processes very much like the duality
between vectors and maps from vectors to scalars.

Further, because the calculus is essentially higher-order, we have a
correspondence between contexts and processes. More specifically,
given a name $x$ and a context $M$ we can construct $M^{*}_{x}$ such
that 

\begin{mathpar}
  M^{*}_{x} | \lift{x}{P} \red M[P]
\end{mathpar}

namely,

\begin{mathpar}
  M^{*}_{x} := x?(u).M[\dropn{u}]
\end{mathpar}

The dependence of $M^{*}_{x}$ on a name makes it an abstraction, 

\begin{mathpar}
  M^{*} := (x)x?(u).M[\dropn{u}]
\end{mathpar}

\subsection{Additional notation}

It will sometimes be convenient to denote the process a name
quotes. We already have the notation $x = \quotep{P}$, but it will be
convenient to introduce an alternate notation, $\procn{x}$, when we
want to emphasize the connection to the use of the name. Note that, by
virtue of name equivalence, $\quotep{\procn{x}} \nameeq x$; so, the
notation is consistent with previous definitions.

Further, because names have structure it is possible to effect
substitutions on the basis of that structure. This means we need to
upgrade our notation for substitutions, which we accomplish by
adapting comprehension notation. Thus,

\begin{mathpar}
  P\{ y / x : x \in S \}
\end{mathpar}

is interpreted to mean the process derived from P by replacing (in a
capture-avoiding manner) each occurrence of $x$ in $S$ by $y$. For example,

\begin{mathpar}
  P\{ \quotep{\procn{x}|\procn{x}} / x : x \in \freenames{P} \}
\end{mathpar}

will replace each (occurrence) of a free name $x$ in $P$ by
$\quotep{\procn{x}|\procn{x}}$.

Also, we will avail ourselves of the notation $x^{L}$ and $x^{R}$ to
denote injections of a name into disjoint copies of the name
space. There are numerous ways to accomplish this. One example can be
found in \cite{MeredithR05}. This notation overloads to vectors of
names: $\vec{x}^{\pi} := (x_{i}^{\pi} \; : \; 0 \leq i < |\vec{x}| )$ where $\pi \in \{L,R\}$.

We also use $P^{\Box} := P|\Box$.

In \cite{MeredithR05} an interpretation of the new operator is
given. It turns out that there are several possible interpretations
all enjoying the requisite algebraic properties of the operator (see
\cite{milner91polyadicpi}). We will therefore make liberal use of
$(\nu\; \vec{x})P$.

% subsection the_syntax_and_semantics_of_the_notation_system (end)   

\input{qm2pi.qmops} 

\input{qm2pi.sterngerlach} 

\input{qm2pi.metric} 

% section concurrent_process_calculi (end)

%\input{qm2pi.proofsketch}

% section proof sketch (end)

%\input{qm2pi.slviaknots} 

% section spatial logic via knots (end)

\input{qm2pi.conclusion}

% section conclusion (end)

%\input{qm2pi.dtcodes} 

% section wiring algorithm (end)

\input{qm2pi.ack} 

% section acknowledgments (end)

\newpage


\bibliographystyle{plain}   
\bibliography{../../biblios/main.bib}

\input{qm2pi.rhodetails}

\end{document}

 

% subsection basic_interpretation (end)

%\input{qm2pi.rho.presentation} 
\subsection{The syntax and semantics of the notation system}\label{sub:the_syntax_and_semantics_of_the_notation_system} % (fold)

We now summarize a technical presentation of the calculus that
embodies our theory of dynamics. The typical presentation of such a
calculus follows the style of giving generators and relations on
them. The grammar, below, describing term constructors, freely
generates the set of processes, $\Proc$. This set is then quotiented
by a relation known as structural congruence and it is over this set
that the notion of dynamics is expressed. This presentation is
essentially that of \cite{MeredithR05} with the addition of
polyadicity and summation. For readability we have relegated some of
the technical subtleties to an appendix.

\subsubsection{Process grammar}\label{subsub:process_grammar}

\begin{mathpar}
  \inferrule* [lab=synchronization] {} {{M} \bc \pzero \;|\; x?F \;|\; x!C }
  \and
  \inferrule* [lab=abstraction] {} {{F} \bc (x)P}
  \and
  \inferrule* [lab=concretion] {} {{C} \bc \langle Q \rangle}
  \and
  \inferrule* [lab=process] {} {{P,Q} \bc M \;| \;P|Q \;|\; @{x}}
  \and
  \inferrule* [lab=name] {} {{x} \bc \quotep{P}}
\end{mathpar} 

Note that $\vec{x}$ (resp. $\vec{P}$) denotes a vector of names
(resp. processes) of length $|\vec{x}|$ (resp. $|\vec{P}|$). We adopt
the following useful abbreviations.

\begin{mathpar}
   x?(\vec{y}).P := x.(\vec{y})P \and  x\clift{\vec{P}} := x.\clift{\vec{P}}
   \and x!(y) := \lift{x}{\dropn{y}}
   \and \Pi_{i=0}^{n-1}P_i := P_0 | \ldots | P_{n-1}
\end{mathpar}

\subsubsection{Structural congruence}

\paragraph{Free and bound names and alpha-equivalence.} At the
core of structural equivalence is alpha-equivalence which identifies
process that are the same up to a change of variable. Formally, we
recognize the distinction between free and bound names. The free names
of a process, $\freenames{P}$, may be calculated recursively as
follows:

\begin{mathpar}
\freenames{\pzero} := \emptyset
  \and \\
  \freenames{x?(y).P} := \{ x \} \cup (\freenames{P} \setminus \{ y \})
  \and 
  \freenames{x!\langle P \rangle} := \{ x \} \cup \{ P \} 
  \and \\
  \freenames{P|Q} := \freenames{P} \cup \freenames{Q}
  \and \\
  \freenames{@{x}} := \{ x \}
\end{mathpar}

$\pi$
$\quotep{\pi}$

$\freenames{-} : \pi \to \mathcal{P}(\quotep{\pi})$

\begin{eqnarray*}
  \freenames{\pzero} & := & \emptyset \\
  \freenames{x?(y).P} & := & \{ x \} \cup (\freenames{P} \setminus \{ y \}) \\
  \freenames{x!\langle P \rangle} & := & \{ x \} \cup \{ P \} \\
  \freenames{P|Q} & := & \freenames{P} \cup \freenames{Q} \\
  \freenames{\dropn{x}} & := & \{ x \}
\end{eqnarray*}

The bound names of a process, $\boundnames{P}$, are those names occurring in $P$
that are not free. For example, in $x?(y).0$, the name $x$ is free, while $y$ is bound.

\begin{mathpar}
  \inferrule* [lab=monoidal-laws] {} { P|Q \equiv Q|P \and P|0 \equiv P \and P|(Q|R) \equiv (P|Q)|R }
\end{mathpar}

\begin{mathpar}
  \inferrule* [lab=alpha-equivalence] {} { (x)P \equiv (y)P\{y/x\} \and y \not\in \freenames{P} }
\end{mathpar}

\begin{definition}
Then two processes, $P,Q$, are alpha-equivalent if $P = Q\{\vec{y}/\vec{x}\}$ for
some $\vec{x} \in \boundnames{Q},\vec{y} \in \boundnames{P}$, where $Q\{\vec{y}/\vec{x}\}$
denotes the capture-avoiding substitution of $\vec{y}$ for $\vec{x}$ in $Q$.
\end{definition}

\begin{definition}
  The {\em structural congruence} \cite{SangiorgiWalker} , $\equiv$,
  between processes is the least congruence containing
  alpha-equivalence, satisfying the abelian monoid laws
  (associativity, commutativity and $\pzero$ as identity) for parallel
  composition $|$ and for summation $+$.
\end{definition}

\subsection{Name equivalence}

We take name equivalence, written $\nameeq$, to be the smallest
equivalence relation generated by the following rules.

\begin{mathpar}
\inferrule*[lab=Quote-drop]
{ }
{ \quotep{@{x}} \nameeq x }

\inferrule*[lab=Struct-equiv]
{ P \scong Q }
{ \quotep{P} \nameeq \quotep{Q} }
\end{mathpar}

The astute reader will have noticed that the mutual recursion of names
and processes imposes a mutual recursion on alpha-equivalence and
structural equivalence via name-equivalence. Fortunately, all of this
works out pleasantly and we may calculate in the natural way, free of
concern. The reader interested in the details is referred to the
appendix \ref{appendix:rho_details}.

\subsection{Substitution}

We use $\Proc$ for the set of processes, $\QProc$ for the set of
names, and $\id{\{}\vec{y} / \vec{x} \id{\}}$ to denote partial maps,
$s : \QProc \rightarrow \QProc$. A map, $s$ lifts, uniquely, to a map
on process terms, $\widehat{s} : \Proc \rightarrow \Proc$ by the
following equations.

\begin{mathpar}
  (0) \psubstp{Q}{P} := 0 \\
  (R \juxtap S) \psubstp{Q}{P}
  :=    
  (R)\psubstp{Q}{P} \juxtap (S) \psubstp{Q}{P} \\
  (x?(y).R) \psubstp{Q}{P}    
  :=    
  (x)\substp{Q}{P} (z)\concat( (R \psubstn{z}{y}) \psubstp{Q}{P} ) \\
  (\lift{x}{R}) \psubstp{Q}{P}  
  :=
  \lift{(x)\substp{Q}{P}}{ R \psubstp{Q}{P} } \\
%   (\dropn{x})  \psubstp{Q}{P}       
%   := 
%   \left\{ 
%     \begin{array}{ccc} 
%       \dropn{\quotep{Q}} & & x \nameeq \quotep{P} \\
%       \dropn{x} & & otherwise \\
%     \end{array}
%   \right. 
  (\dropn{x})  \psubstp{Q}{P}       
  := 
  \left\{ 
    \begin{array}{ccc} 
      Q & & x \nameeq \quotep{P} \\
      \dropn{x} & & otherwise \\
    \end{array}
  \right.
\end{mathpar}
 

where

\begin{eqnarray}
  (x)\id{\{} \lpquote Q \rpquote / \lpquote P \rpquote \id{\}}            = 
  \left\{ 
    \begin{array}{ccc}
      \lpquote Q \rpquote & & x \nameeq \lpquote P \rpquote \\
      x & & otherwise \\
    \end{array}
  \right. \nonumber
\end{eqnarray}

and $z$ is chosen distinct from $\quotep{P}$, $\quotep{Q}$, the free
names in $Q$, and all the names in $R$. Our $\alpha$-equivalence will
be built in the standard way from this substitution.

\begin{remark}\label{rem:no_self_referential_names}
  One consequence of these definitions is that $\forall P. \quotep{P}
  \not\in \freenames{P}$.
\end{remark}

\subsection{ Dynamic quote: an example }

Anticipating something of what's to come, consider applying the
substitution, $\widehat{\id{\{}u / z \id{\}}}$, to the following pair
of processes, $\lift{w}{y!(z)}$ and $w[ \lpquote y!(z) \rpquote ]$.

\begin{eqnarray}
	\lift{w}{y!(z)}\widehat{\id{\{}u / z \id{\}}}
		& = &
		\lift{w}{y!(u)} \nonumber\\
	w[ \lpquote y!(z) \rpquote ] \widehat{ \id{\{}u / z \id{\}} }
		& = &
		w[ \lpquote y!(z) \rpquote ] \nonumber
\end{eqnarray}

Because the body of the process between quotes is impervious to
substitution, we get radically different answers. In fact, by
examining the first process in an input context,
e.g. $x?(z).\lift{w}{y!(z)}$, we see that the process under the lift
operator may be shaped by prefixed inputs binding a name inside it. In
this sense, the lift operator will be seen as a way to dynamically
construct processes before reifying them as names.

Finally equipped with these standard features we can present the
dynamics of the calculus.

\subsubsection{Operational semantics} 

Finally, we introduce the computational dynamics. What marks these
algebras as distinct from other more traditionally studied algebraic
structures, e.g. vector spaces or polynomial rings, is the manner in
which dynamics is captured. In traditional structures, dynamics is typically
expressed through morphisms between such structures, as in linear maps
between vector spaces or morphisms between rings. In algebras
associated with the semantics of computation, the dynamics is
expressed as part of the algebraic structure itself, through a
reduction reduction relation typically denoted by $\red$. Below, we
give a recursive presentation of this relation for the calculus used
in the encoding.

$\red \subseteq \pi \times \pi$
$\red : \pi \to \mathcal{P}(\pi)$

\begin{mathpar}
  \inferrule* [lab=Comm] { \textsf{match}( x_{src}, x_{trgt} ) } { x_{trgt}?(y)P \; | \; x_{src}!\langle {Q} \rangle \red P\{\quotep{Q}/y}\} }
  \and \\
  \inferrule* [lab=Par] {{P} \red {P}'} {{{P} | {Q}} \red {{P}' | {Q}}}
  \and
  \inferrule* [lab=Equiv]{{{P} \scong {P}'} \andalso {{P}' \red {Q}'} \andalso {{Q}' \scong {Q}}}{{P} \red {Q}}
\end{mathpar}

\begin{eqnarray*}
  match_{\equiv} (\quotep{P},\quotep{Q}) & := & P \equiv Q \\
  match_{\dagger}(\quotep{P},\quotep{Q}) & := & \forall R. P|Q \red^{*} R => R \red^{*} 0 \\
  match_{K}(\quotep{P},\quotep{Q}) & := & K \mbox{ for some context } K
\end{eqnarray*}

$u?(x)P | u!\langle Q \rangle \red P\{\quotep{Q}/x\}$

%We write $\wred$ for $\red^*$, and $P\red$ if $\exists Q $ such that $ P \red Q$.
We write $P\red$ if $\exists Q $ such that $ P \red Q$ and $P\not\red$, otherwise.

\section{Replication}

As mentioned before, it is known that replication (and hence
recursion) can be implemented in a higher-order process algebra
\cite{SangiorgiWalker}. As our first example of calculation with the
machinery thus far presented we give the construction explicitly in
the {\rhoc}.

\begin{eqnarray}
	D_{x} & := & \prefix{x}{y}{(\binpar{\outputp{x}{y}}{@{y}})} \nonumber\\
	\bangp_{x}{P} & := & \binpar{{x}!\langle{\binpar{D_{x}}{P}}\rangle}{D_{x}} \nonumber
\end{eqnarray}

\begin{eqnarray}
	\bangp_{x}{P} & & \nonumber\\
	=
	& {x}!\langle{(\prefix{x}{y}{(\outputp{x}{y} | @{y})) | P}}\rangle 
	      | \prefix{x}{y}{(\outputp{x}{y} | @{y})} & \nonumber\\
	\red
	& (\outputp{x}{y} | @{y})\substn{\quotep{(\prefix{x}{y}{(@{y} | \outputp{x}{y})) | P}}}{y} & \nonumber\\
	=
	& \outputp{x}{\quotep{(\prefix{x}{y}{(\outputp{x}{y} | @{y})) | P}}}
	  | {(\prefix{x}{y}{(\outputp{x}{y} | @{y})) | P}} & \nonumber\\
	\red
	& \ldots & \nonumber\\
	\red^*
	& P | P | \ldots & \nonumber
\end{eqnarray}

Of course, this encoding, as an implementation, runs away, unfolding
$\bangp{P}$ eagerly. A lazier and more implementable replication
operator, restricted to input-guarded processes, may be obtained as follows.

\begin{eqnarray}
\bangp{\prefix{u}{v}{P}} 
	:= 
	\binpar{\lift{x}{\prefix{u}{v}{(\binpar{D(x)}{P})}}}{D(x)} \nonumber
\end{eqnarray}

\begin{remark}
  Note that the lazier definition still does not deal with summation
  or mixed summation (i.e. sums over input and output). The reader is
  invited to construct definitions of replication that deal with these
  features. 

  Further, the definitions are parameterized in a name, $x$. Can you,
  gentle reader, make a definition that eliminates this parameter and
  guarantees no accidental interaction between the replication
  machinery and the process being replicated -- i.e. no accidental
  sharing of names used by the process to get its work done and the
  name(s) used by the replication to effect copying. This latter
  revision of the definition of replication is crucial to obtaining
  the expected identity $!!P \sim !P$.
\end{remark}

\begin{remark}\label{rem:paradoxical_combinator}
  The reader familiar with the lambda calculus will have noticed the
  similarity between $D$ and the paradoxical combinator.

  [Ed. note: the existence of this seems to suggest we have to be more
  restrictive on the set of processes and names we admit if we are to
  support no-cloning.]
\end{remark}

\subsubsection{Bisimulation}

The computational dynamics gives rise to another kind of equivalence,
the equivalence of computational behavior. As previously mentioned
this is typically captured \emph{via} some form of bisimulation.

% The notion we use in this paper is weak barbed bisimulation
% \cite{milner91polyadicpi}.

The notion we use in this paper is derived from weak barbed
bisimulation \cite{milner91polyadicpi}. 

\begin{definition}
An \emph{observation relation}, $\downarrow_{\mathcal N}$, over a set
of names, $\mathcal N$, is the smallest relation satisfying the rules
below.

\infrule[Out-barb]{y \in {\mathcal N}, \; x \nameeq y}
		  {\outputp{x}{v} \downarrow_{\mathcal N} x}
\infrule[Par-barb]{\mbox{$P\downarrow_{\mathcal N} x$ or $Q\downarrow_{\mathcal N} x$}}
		  {\binpar{P}{Q} \downarrow_{\mathcal N} x}

We write $P \Downarrow_{\mathcal N} x$ if there is $Q$ such that 
$P \wred Q$ and $Q \downarrow_{\mathcal N} x$.
\end{definition}

\begin{definition}
%\label{def.bbisim}
An  ${\mathcal N}$-\emph{barbed bisimulation} over a set of names, ${\mathcal N}$, is a symmetric binary relation 
${\mathcal S}_{\mathcal N}$ between agents such that $P\rel{S}_{\mathcal N}Q$ implies:
\begin{enumerate}
\item If $P \red P'$ then $Q \wred Q'$ and $P'\rel{S}_{\mathcal N} Q'$.
\item If $P\downarrow_{\mathcal N} x$, then $Q\Downarrow_{\mathcal N} x$.
\end{enumerate}
$P$ is ${\mathcal N}$-barbed bisimilar to $Q$, written
$P \wbbisim_{\mathcal N} Q$, if $P \rel{S}_{\mathcal N} Q$ for some ${\mathcal N}$-barbed bisimulation ${\mathcal S}_{\mathcal N}$.
\end{definition}

$\mathcal{R} \subseteq \pi \times \pi$

$P \mathcal{R} Q => \forall P'. P \red P' \Rightarrow \exists Q'. Q \red Q', P' \mathcal{R} Q'$

$P \vdash x \Rightarrow Q \vdash x$

\begin{mathpar}
  \inferrule*[lab=Out-barb]{x \nameeq y}{{y}!\langle{Q}\rangle \vdash x}
  \and
  \inferrule*[lab=Par-barb]{\mbox{$P\vdash x$ or $Q\vdash x$}}{\binpar{P}{Q} \vdash x}
\end{mathpar}

\subsubsection{Contexts}

One of the principle advantages of computational calculi like the
$\pi$-calculus is a well-defined notion of context,
contextual-equivalence and a correlation between
contextual-equivalence and notions of bisimulation. The notion of
context allows the decomposition of a process into (sub-)process and
its syntactic environment, its context. Thus, a context may be
thought of as a process with a ``hole'' (written $\Box$) in it. The
application of a context $M$ to a process $P$, written $M[P]$, is
tantamount to filling the hole in $M$ with $P$. In this paper we do
not need the full weight of this theory, but do make use of the notion
of context in the proof the main theorem. 

\begin{mathpar}
  \inferrule* [lab=summation] {} {{M_{M},M_{N}} \bc \Box \;|\; x.M_{A} \;|\; M_{M}+M_{N}}
  \and
  \inferrule* [lab=agent] {} {{M_{A}} \bc (\vec{x})M_{P} \;| \; \clift{P_0,\ldots,M_{P},\ldots,P_N}}
  \and \\
  \inferrule* [lab=process] {} {{M_{P}} \bc M_{N} \;| \;P|M_{P} }
\end{mathpar} 

\begin{mathpar}
  \inferrule* [lab=sychronization] {} {M_{N} \bc \Box \;|\; x?M_{F} \;|\; x!M_{C}}
  \and
  \inferrule* [lab=abstraction] {} {{M_{F}} \bc (x)M_{P} }
  \and
  \inferrule* [lab=concretion] {} {{M_{C}} \bc \langle M_{P} \rangle }
  \and \\
  \inferrule* [lab=process] {} {{M_{P}} \bc M_{N} \;| \;P|M_{P} }
\end{mathpar}

\begin{definition}[contextual application] Given a context $M$, and
  process $P$, we define the \emph{contextual application}, $M[P] :=
  M\{P/\Box\}$. That is, the contextual application of M to P is the
  substitution of $P$ for $\Box$ in $M$.
\end{definition}

$\meaningof{-} : L \to \mathcal{P}(\pi)$

\begin{mathpar}
  \inferrule* [lab=collection] {} {\meaningof{true} = \pi, \and \meaningof{~E} = \pi \setminus \meaningof{E}, \and \meaningof{E_{1} \& E_{2}} = \meaningof{E_{1}} \cap \meaningof{E_{2}}}
\end{mathpar}

\begin{mathpar}
  \inferrule* [lab=structure] {} {\meaningof{0} = \{ P \in \pi | P \equiv 0 \}, \and \\ \meaningof{E_1 | E_2} = \{ P \in \pi | P \equiv P_{1} | P_{2}, P_{1} \in \meaningof{E_{1}}, P_{2} \in \meaningof{E_2}\} }
\end{mathpar}

\begin{mathpar}
 \inferrule* [lab=behavior] {} {\meaningof{\langle a?b \rangle E} = \{ P \in \pi | P \equiv Q | u?(y)P', \\ \and \\\\ \and \\ \;\;\; u \in \meaningof{a}, \forall z.P'\{z/y\} \in \meaningof{E\{z/b\}}\}, \and \\ \meaningof{a!E} = \{ P \in \pi | P \equiv Q | x!\langle P' \rangle, x \in \meaningof{a} P' \in \meaningof{E}\} }
\end{mathpar}

\begin{mathpar}
 \inferrule* [lab=nominal] {} {\meaningof{\quotep{E}} = \{ \quotep{P} \in \quotep{\pi} | P \in \meaningof{E} \}, \and \meaningof{\quotep{P}} = \{ \quotep{Q} \in \quotep{\pi} | P \equiv Q \} \and \\ \meaningof{@\quotep{E}} = \{ P \in \pi | P \equiv @x, x \in \meaningof{E} \}}
\end{mathpar}

\begin{eqnarray*}
  \\
  \meaningof{-} : TS \to ST
\end{eqnarray*}

\begin{eqnarray*}
  \\
  L : TS \to ST
\end{eqnarray*}

\begin{eqnarray*}
  \\
  P \models E \iff P \in \meaningof{E}
\end{eqnarray*}

\begin{eqnarray*}
  P \approx_{L} Q \iff \forall E \in L. P \models E \iff Q \models E
\end{eqnarray*}

\begin{eqnarray*}
  P \approx_{K} Q
\end{eqnarray*}

\begin{eqnarray*}
  P \approx Q
\end{eqnarray*}

$\approx_{K} = \approx = \approx_{L}$

\subsubsection{Contextual duality}

Note that contexts extend the quotation operation to a family of
operations from processes to names. Given a context, $M$, we can
define a \emph{nominal context}, $\quotep{M}$ by $\quotep{M}[P] :=
\quotep{M[P]}$. To foreshadow what is to come we observe that these
operations enjoy a duality with processes very much like the duality
between vectors and maps from vectors to scalars.

Further, because the calculus is essentially higher-order, we have a
correspondence between contexts and processes. More specifically,
given a name $x$ and a context $M$ we can construct $M^{*}_{x}$ such
that 

\begin{mathpar}
  M^{*}_{x} | \lift{x}{P} \red M[P]
\end{mathpar}

namely,

\begin{mathpar}
  M^{*}_{x} := x?(u).M[\dropn{u}]
\end{mathpar}

The dependence of $M^{*}_{x}$ on a name makes it an abstraction, 

\begin{mathpar}
  M^{*} := (x)x?(u).M[\dropn{u}]
\end{mathpar}

\subsection{Additional notation}

It will sometimes be convenient to denote the process a name
quotes. We already have the notation $x = \quotep{P}$, but it will be
convenient to introduce an alternate notation, $\procn{x}$, when we
want to emphasize the connection to the use of the name. Note that, by
virtue of name equivalence, $\quotep{\procn{x}} \nameeq x$; so, the
notation is consistent with previous definitions.

Further, because names have structure it is possible to effect
substitutions on the basis of that structure. This means we need to
upgrade our notation for substitutions, which we accomplish by
adapting comprehension notation. Thus,

\begin{mathpar}
  P\{ y / x : x \in S \}
\end{mathpar}

is interpreted to mean the process derived from P by replacing (in a
capture-avoiding manner) each occurrence of $x$ in $S$ by $y$. For example,

\begin{mathpar}
  P\{ \quotep{\procn{x}|\procn{x}} / x : x \in \freenames{P} \}
\end{mathpar}

will replace each (occurrence) of a free name $x$ in $P$ by
$\quotep{\procn{x}|\procn{x}}$.

Also, we will avail ourselves of the notation $x^{L}$ and $x^{R}$ to
denote injections of a name into disjoint copies of the name
space. There are numerous ways to accomplish this. One example can be
found in \cite{MeredithR05}. This notation overloads to vectors of
names: $\vec{x}^{\pi} := (x_{i}^{\pi} \; : \; 0 \leq i < |\vec{x}| )$ where $\pi \in \{L,R\}$.

We also use $P^{\Box} := P|\Box$.

In \cite{MeredithR05} an interpretation of the new operator is
given. It turns out that there are several possible interpretations
all enjoying the requisite algebraic properties of the operator (see
\cite{milner91polyadicpi}). We will therefore make liberal use of
$(\nu\; \vec{x})P$.

% subsection the_syntax_and_semantics_of_the_notation_system (end)   

\section{Interpretation of QM}
\subsection{Supporting definitions}
\subsubsection{Multiplication}
\begin{mathpar}
  \quotep{Q} \cdot \quotep{R} := \quotep{Q|R}
  \and \\
  \quotep{Q} \cdot P := P\{ \quotep{Q|R} / \quotep{R} : \quotep{R} \in \freenames{P} \}
\end{mathpar}

\paragraph{Discussion}
The first line needs little explanation. The second line says that
each free name of the process is replaced with the multiplication of
that name by the scalar. Multiplication of a scalar (name) by a state
(process) results in a process all the names of which have been `moved
over' by parallel composition with the process the scalar
quotes. There is a subtlety that the bound names have to be
manipulated so that multiplied names aren't accidentally
captured. There are many ways to achieve this.

\begin{remark}\label{rem:multiplication_identities}
  The reader is invited to verify that for all $x,y,z \in \QProc$ and $P \in \Proc$
  \begin{mathpar}
    x \cdot \quotep{0} \equiv x 
    \and
    x \cdot y \equiv y \cdot x
    \and
    x \cdot (y \cdot z) \equiv (x \cdot y) \cdot z
    \and \\
    \quotep{0} \cdot P \equiv P
    \and \\
    x \cdot (y \cdot P) \equiv (x \cdot y) \cdot P
    \and \\
    x \cdot (P|Q) \equiv (x \cdot P) | (x \cdot Q)
    \and \\    
  \end{mathpar}
\end{remark}

\subsubsection{Tensor product}

We define a tensor product on processes by structural induction.

\paragraph{Tensor of sums} First note that all summations, including
$\pzero$ and sequence, can be written $\Sigma_{i} x_{i}.A_{i} +
\Sigma_{j} x_{j}.C_{j}$, where we have grouped input-guarded processes
together and output-guarded processes together.

Thus, we can define the tensor product of two summations, $N_{1}\otimes N_{2}$, where

\begin{mathpar}
  N_{1} := \Sigma_{i} x_{i}.A_{i} + \Sigma_{j} x_{j}.C_{j}
  \and
  N_{2} := \Sigma_{i'} y_{i'}.B_{i'} + \Sigma_{j'} y_{j'}.D_{j'} 
\end{mathpar}

as follows.

\begin{mathpar}
  \Sigma_{i} x_{i}.A_{i} + \Sigma_{j} x_{j}.C_{j} \otimes \Sigma_{i'}
  y_{i'}.B_{i'} + \Sigma_{j'} y_{j'}.D_{j'} 
  \and \\
  := \; \Sigma_{i} \Sigma_{i'} \quotep{\stackrel{\vee}{x_{i}}| \stackrel{\vee}{y_{i'}}}.(A_{i}\otimes B_{i'}) \; | \; \Sigma_{i'} \Sigma_{i} \quotep{\stackrel{\vee}{y_{i'}}|\stackrel{\vee}{x_{i}}}.(B_{i'}\otimes A_{i})
  \and
  \;\; | \;\; \Sigma_{j} \Sigma_{j'} \quotep{\stackrel{\vee}{x_{j}}|\stackrel{\vee}{y_{j'}}}.(A_{j}\otimes B_{j'}) \; | \; \Sigma_{j'} \Sigma_{j} \quotep{\stackrel{\vee}{y_{j'}}|\stackrel{\vee}{x_{j}}}.(B_{j'}\otimes A_{j})
\end{mathpar}

\begin{remark}
  Do we need to $x^{L}$ and $y^{R}$ for this construction as well?
\end{remark}

\paragraph{Tensor of parallel compositions} Next, we distribute tensor
over par.

\begin{mathpar}
  P_{1}|P_{2} \otimes Q_{1}|Q_{2} := (P_{1} \otimes Q_{1}) | (P_{1}
  \otimes Q_{2}) | (P_{2} \otimes Q_{1}) | (P_{2} \otimes Q_{2})
\end{mathpar}

\paragraph{Tensor with dropped names} We treat tensor of a
process with a dropped name as parallel composition.

\begin{mathpar}
  P \otimes \dropn{x} := P | \dropn{x}
\end{mathpar}

\paragraph{Tensor of agents}

Finally, we need to define tensor on agents. Note that the definition
of tensor on normal products only tensors inputs with inputs and
outputs with outputs. Thus, we only have to define the operation on
``homogeneous'' pairings.

\begin{mathpar}
  (\vec{x})P \otimes (\vec{y})Q
  \and \\
  := (x_{0}^{L}|y_{0}^{R},\ldots,x_{0}^{L}|y_{n}^{R},\ldots,x_{m}^{L}|y_{0}^{R},\ldots,x_{m}^{L}|y_{n}^R)(P\{ \vec{x}^{L}/\vec{x}\} \otimes Q \{ \vec{y}^{R}/\vec{y}\})
  \and \\
  \clift{\vec{P}} \otimes \clift{\vec{Q}}
  \and \\
  := \clift{P_{0}\otimes Q_{0},\ldots,P_{0}\otimes Q_{n},\ldots,P_{m}\otimes Q_{0},\ldots,P_{m}\otimes Q_{n}}
\end{mathpar}

\begin{remark}
  Observe that arities of tensored abstractions matches arities of
  tensored concretions if the original arities matched. Note also that
  the length of the arities corresponds to the increase in dimension
  we see in ordinary vector space tensor product.
\end{remark}

\begin{remark}
  Operationally, this definition distributes the tensor down to
  components ``linked'' by summation. Tensor over summation is
  intriguing in that it mixes names. Moreover, as a consequence of the
  way it mixes names we have the identities for all $x \in \QProc$ and
  $P,Q \in \Proc$

  \begin{mathpar}
    (x \cdot P) \otimes Q \equiv x \cdot (P \otimes Q) \equiv P \otimes (x \cdot Q)
    \and
    P \otimes \pzero \equiv P
  \end{mathpar}

  that the reader is invited to verify.
\end{remark}

\subsubsection{Annihilation}
\begin{mathpar}
  P^{\perp} := \{ Q | \forall R. P|Q \red^{*} R \Rightarrow R \red^{*} \pzero \}
  \and \\
  P^{\underline{\perp}} := \Sigma_{Q \in P^{\perp}} \quotep{Q}?(y).(\dropn{y}|Q) | \Sigma_{Q \in P^{\perp}} \quotep{Q}\clift{\Box}
\end{mathpar}

\paragraph{Discussion} The reader will note that $P^{\perp}$ is a
\emph{set} of processes, while $P^{\underline{\perp}}$ is a
\emph{context}. We call the set $P^{\perp}$ the \emph{annihilators} of
$P$. The parallel composition of a process in the annihilators of $P$
with $P$ will result in a process, the state space of which has all
paths eventually leading to $\pzero$. Execution may endure loops; but
under reasonable conditions of fairness (naturally guaranteed under
most notions of bisimulation) such a composite process cannot get
stuck in such a loop and will, eventually pop out and terminate.

The context $P^{\underline{\perp}}$ is ready and willing to ``take the
$P$ out of'' the process to which it is applied. It will effectively
transmit the code of the process to which it is applied to one of the
annihilators and run the process against it.

\subsubsection{Evaluation}
We fix $M$ a domain of fully abstract interpretation with an equality
coincident with bisimulation. We take $\meaningof{\cdot} : \Proc \to
M$ to be the map interpreting processes and $\nmeaningof{\cdot} : \M
\to Proc$ to be the map running the other way. Then we define

\begin{mathpar}
  \int P := \nmeaningof{\meaningof{P}}
\end{mathpar}

\paragraph{Discussion}
There are many fully abstract interpretations of Milner's
$\pi$-calculus. Any of them can be used as a basis for interpreting
the reflective calculus here. Equipped with such a domain it is
largely a matter of grinding through to check that the Yoneda
construction for the normalization-by-evaluation program can be
extended to this setting.

\begin{remark}
  The reader is invited to verify that $\int (P^{\underline{\perp}}[P]) = 0$.
\end{remark}

\subsection{Quantum mechanics}

Table \ref{tbl:core_qm_op_defns} gives the core operational definitions

\begin{table}[htp]\label{tbl:core_qm_op_defns}
  \center{
    \fbox{
      \begin{tabular}{c|c}
        quantum mechanics & process calculus \\
        \hline
        scalar & $x := \quotep{P}$ \\
        state vector & $\state{P} := P$ \\
        dual & $\state{P}^{*} := \event{P^{\underline{\perp}}} := \quotep{P^{\underline{\perp}}}[-]$ \\
        matrix & $ \Sigma_{\alpha} \state{P_{\alpha}}x_{\alpha}\event{Q_{\alpha}}$ \\
        vector addition & $\state{P} + \state{Q} := \state{P | Q}$ \\
        tensor product & $\state{P} \otimes \state{Q} := \state{P \otimes Q}$ \\
        inner product & $\innerprod{P}{Q} := \quotep{\int P^{\underline{\perp}}[Q]}$ \\
      \end{tabular}
    }
  }
  \caption{QM - operational definitions}
\end{table}

where

\begin{mathpar}
  \prmatrix{P}{Q} := \fprmatrix{P}{\quotep{\pzero}}{Q}
  \and
  \fprmatrix{P}{x}{Q} := (\state{P},x,\event{Q})
  \and
  (\fprmatrix{P}{x}{Q})(\state{R}) := x \cdot \innerprod{Q}{R} \cdot \state{P}
  \and
  (\fprmatrix{P}{x}{Q})(\event{R}) := x \cdot \innerprod{R}{P} \cdot \event{Q}
\end{mathpar}

\paragraph{Discussion}
As promised: vectors (aka states) are represented as processes; duals
as contextual duals; inner product definition should be compared with
standard inner product definition for ....

\begin{remark}
  Assuming $\int (P^{\underline{\perp}}[P]) = 0$, the reader is
  invited to verify that $(\fprmatrix{P}{x}{P})(\state{P}) = x \cdot \state{P}$.
\end{remark}

\begin{remark}
  The reader is invited to verify that $\innerprod{P}{Q}$ could
  equally well have been written $\quotep{\int \stackrel{\vee}{x}}$
  where $x = \event{P^{\underline{\perp}}}(Q)$.

  One of the motivations for this remark is that there is another way
  to factor these operations. We could package up evaluation in the dual:

  \begin{mathpar}
    \state{P}^{*} := \event{\int P^{\underline{\perp}}} := \quotep{\int P^{\underline{\perp}}}[-]
  \end{mathpar}

  and then have inner product defined by
  
  \begin{mathpar}
    \innerprod{P}{Q} := \event{P}(Q)
  \end{mathpar}

  Hopefully, experience with the calculations will provide guidance on
  the best factoring.
\end{remark}

\begin{remark}
  Assuming $\int (P^{\underline{\perp}}[P]) = 0$, the reader is
  invited to verify that $\forall P,Q. (\prmatrix{0}{Q})(\state{0}) =
  \state{0}$ and dually $(\prmatrix{P}{0})(\event{0}) = \event{0}$.
\end{remark}

\begin{remark}
  i'm a little worried that i don't (yet) have proper support for
  complex conjugacy. But, the observation above may give us a
  clue. According to Abramsky, it must be the case that the scalars
  are iso to the homset of the identity for the tensor -- which the
  observation above characterizes. 

  For now, we will simply bookmark the notion with $\overline{x}$.
\end{remark}

\subsubsection{Adjointness}

We need to give a definition of $(\cdot)^{\dagger}$ for matrices. The
obvious candidate definition is
\begin{mathpar}
(\Sigma_{\alpha}\fprmatrix{P_{\alpha}}{x_{\alpha}}{Q_{\alpha}})^{\dagger}
= \Sigma_{\alpha}\fprmatrix{(Q_{\alpha}^{\underline{\perp}})^{*}}{\overline{x}_{\alpha}}{P_{\alpha}^{\underline{\perp}}} 
\end{mathpar}

But, $(Q_{\alpha}^{\underline{\perp}})^{*}$ requires a name along
which to communicate the process to achieve the context application.

\subsubsection{Basis for a basis}
If processes label states and ``addition'' of states (a.k.a. vector
addition) is interpreted as parallel composition, what corresponds to
notions of linear independence and basis? Here, we recall that Yoshida
has developed a set of \emph{combinators} for an asynchronous verison
of Milner's $\pi$-calculus. These are a finite set of processes such
any process can be expressed as parallel composition of these
combinators together with liberal uses of the new operator and
replication. We can simply give a translation of these into the
present calculus and have reasonable expectation that the property
carries over. That is, that the resultant set allows to express all
processes via parallel composition. Note, however, that there is no
new operator or replication in this calculus. As a result, we expect
that the corresponding set is actually infinite. That is, we expect
that the space is actually infinite dimensional.

\begin{remark}
  The attentive reader may be a bit concerned. Certainly, the
  collection $S$, $K$ and $I$ is a finite set of
  combinators. Shouldn't we expect to see a finite set of combinators
  for an effectively equivalent system? i am very sympathetic to this
  critique and feel it warrants full attention. On the other hand, i
  also have in mind the following analogy. The natural numbers, as a
  monoid under addition, has exactly $1$ generator, while the natural
  numbers, as a monoid under multiplication, has countably many
  generators (the primes). We observe that the application of the
  lambda calculus is much less resource sensitive than the parallel
  composition of the $\pi$-calculus. Could it be the case that we have
  an analogy of the form
  
  \begin{mathpar}
    m + n : MN :: m*n : M|N
  \end{mathpar}

  giving a similar blow up in the set of ``primes''?  This is such a
  wonderful thought that, even if it's not true, i think it's worth
  writing down.
\end{remark}
 

\documentclass[12pt]{llncs}
%\documentclass{jktr}

\usepackage[pdftex]{hyperref}                   
\usepackage {listings}
\usepackage {mathpartir}
\usepackage{bcprules}
%\usepackage{listings}
                       
\usepackage{graphicx} 
%\usepackage[margins=2.5cm,nohead,nofoot]{geometry}
%\usepackage{geometry}
\usepackage{amsfonts}
\usepackage{amstext}
\usepackage{latexsym}
\usepackage{amssymb}
\usepackage{color}


%\include{myPreamble}
\include{qm2pi.local} 

%\ifpdf
%\usepackage[pdftex]{graphicx}
%\else
%\usepackage{graphicx}
%\fi

 % \ifpdf
%  \usepackage{pdfsync}
%  \if


%\title{Brief Article}
%\author{David F. Snyder}
%\author{L.G. Meredith}

%\address{Dept. of Math., Texas State University--San Marcos, San Marcos, TX 78666}
       
\pagestyle{empty}


\begin{document}

\lstset{language=[Objective]Caml,frame=shadowbox}

\input{qm2pi.front}

% section front matter (end)

\input{qm2pi.intro} 
 
% section introduction (end)

% \input{qm2pi.knotations} 

% section notation (end)

\input{qm2pi.process.calculi} 

% section concurrent_process_calculi_and_spatial_logics_ (end)
    
%\input{qm2pi.knots2pi} 

%\input{qm2pi.trefoil} 

%\input{qm2pi.mainthm} 

% subsection basic_interpretation (end)

%\input{qm2pi.rho.presentation} 
\subsection{The syntax and semantics of the notation system}\label{sub:the_syntax_and_semantics_of_the_notation_system} % (fold)

We now summarize a technical presentation of the calculus that
embodies our theory of dynamics. The typical presentation of such a
calculus follows the style of giving generators and relations on
them. The grammar, below, describing term constructors, freely
generates the set of processes, $\Proc$. This set is then quotiented
by a relation known as structural congruence and it is over this set
that the notion of dynamics is expressed. This presentation is
essentially that of \cite{MeredithR05} with the addition of
polyadicity and summation. For readability we have relegated some of
the technical subtleties to an appendix.

\subsubsection{Process grammar}\label{subsub:process_grammar}

\begin{mathpar}
  \inferrule* [lab=synchronization] {} {{M} \bc \pzero \;|\; x?F \;|\; x!C }
  \and
  \inferrule* [lab=abstraction] {} {{F} \bc (x)P}
  \and
  \inferrule* [lab=concretion] {} {{C} \bc \langle Q \rangle}
  \and
  \inferrule* [lab=process] {} {{P,Q} \bc M \;| \;P|Q \;|\; @{x}}
  \and
  \inferrule* [lab=name] {} {{x} \bc \quotep{P}}
\end{mathpar} 

Note that $\vec{x}$ (resp. $\vec{P}$) denotes a vector of names
(resp. processes) of length $|\vec{x}|$ (resp. $|\vec{P}|$). We adopt
the following useful abbreviations.

\begin{mathpar}
   x?(\vec{y}).P := x.(\vec{y})P \and  x\clift{\vec{P}} := x.\clift{\vec{P}}
   \and x!(y) := \lift{x}{\dropn{y}}
   \and \Pi_{i=0}^{n-1}P_i := P_0 | \ldots | P_{n-1}
\end{mathpar}

\subsubsection{Structural congruence}

\paragraph{Free and bound names and alpha-equivalence.} At the
core of structural equivalence is alpha-equivalence which identifies
process that are the same up to a change of variable. Formally, we
recognize the distinction between free and bound names. The free names
of a process, $\freenames{P}$, may be calculated recursively as
follows:

\begin{mathpar}
\freenames{\pzero} := \emptyset
  \and \\
  \freenames{x?(y).P} := \{ x \} \cup (\freenames{P} \setminus \{ y \})
  \and 
  \freenames{x!\langle P \rangle} := \{ x \} \cup \{ P \} 
  \and \\
  \freenames{P|Q} := \freenames{P} \cup \freenames{Q}
  \and \\
  \freenames{@{x}} := \{ x \}
\end{mathpar}

$\pi$
$\quotep{\pi}$

$\freenames{-} : \pi \to \mathcal{P}(\quotep{\pi})$

\begin{eqnarray*}
  \freenames{\pzero} & := & \emptyset \\
  \freenames{x?(y).P} & := & \{ x \} \cup (\freenames{P} \setminus \{ y \}) \\
  \freenames{x!\langle P \rangle} & := & \{ x \} \cup \{ P \} \\
  \freenames{P|Q} & := & \freenames{P} \cup \freenames{Q} \\
  \freenames{\dropn{x}} & := & \{ x \}
\end{eqnarray*}

The bound names of a process, $\boundnames{P}$, are those names occurring in $P$
that are not free. For example, in $x?(y).0$, the name $x$ is free, while $y$ is bound.

\begin{mathpar}
  \inferrule* [lab=monoidal-laws] {} { P|Q \equiv Q|P \and P|0 \equiv P \and P|(Q|R) \equiv (P|Q)|R }
\end{mathpar}

\begin{mathpar}
  \inferrule* [lab=alpha-equivalence] {} { (x)P \equiv (y)P\{y/x\} \and y \not\in \freenames{P} }
\end{mathpar}

\begin{definition}
Then two processes, $P,Q$, are alpha-equivalent if $P = Q\{\vec{y}/\vec{x}\}$ for
some $\vec{x} \in \boundnames{Q},\vec{y} \in \boundnames{P}$, where $Q\{\vec{y}/\vec{x}\}$
denotes the capture-avoiding substitution of $\vec{y}$ for $\vec{x}$ in $Q$.
\end{definition}

\begin{definition}
  The {\em structural congruence} \cite{SangiorgiWalker} , $\equiv$,
  between processes is the least congruence containing
  alpha-equivalence, satisfying the abelian monoid laws
  (associativity, commutativity and $\pzero$ as identity) for parallel
  composition $|$ and for summation $+$.
\end{definition}

\subsection{Name equivalence}

We take name equivalence, written $\nameeq$, to be the smallest
equivalence relation generated by the following rules.

\begin{mathpar}
\inferrule*[lab=Quote-drop]
{ }
{ \quotep{@{x}} \nameeq x }

\inferrule*[lab=Struct-equiv]
{ P \scong Q }
{ \quotep{P} \nameeq \quotep{Q} }
\end{mathpar}

The astute reader will have noticed that the mutual recursion of names
and processes imposes a mutual recursion on alpha-equivalence and
structural equivalence via name-equivalence. Fortunately, all of this
works out pleasantly and we may calculate in the natural way, free of
concern. The reader interested in the details is referred to the
appendix \ref{appendix:rho_details}.

\subsection{Substitution}

We use $\Proc$ for the set of processes, $\QProc$ for the set of
names, and $\id{\{}\vec{y} / \vec{x} \id{\}}$ to denote partial maps,
$s : \QProc \rightarrow \QProc$. A map, $s$ lifts, uniquely, to a map
on process terms, $\widehat{s} : \Proc \rightarrow \Proc$ by the
following equations.

\begin{mathpar}
  (0) \psubstp{Q}{P} := 0 \\
  (R \juxtap S) \psubstp{Q}{P}
  :=    
  (R)\psubstp{Q}{P} \juxtap (S) \psubstp{Q}{P} \\
  (x?(y).R) \psubstp{Q}{P}    
  :=    
  (x)\substp{Q}{P} (z)\concat( (R \psubstn{z}{y}) \psubstp{Q}{P} ) \\
  (\lift{x}{R}) \psubstp{Q}{P}  
  :=
  \lift{(x)\substp{Q}{P}}{ R \psubstp{Q}{P} } \\
%   (\dropn{x})  \psubstp{Q}{P}       
%   := 
%   \left\{ 
%     \begin{array}{ccc} 
%       \dropn{\quotep{Q}} & & x \nameeq \quotep{P} \\
%       \dropn{x} & & otherwise \\
%     \end{array}
%   \right. 
  (\dropn{x})  \psubstp{Q}{P}       
  := 
  \left\{ 
    \begin{array}{ccc} 
      Q & & x \nameeq \quotep{P} \\
      \dropn{x} & & otherwise \\
    \end{array}
  \right.
\end{mathpar}
 

where

\begin{eqnarray}
  (x)\id{\{} \lpquote Q \rpquote / \lpquote P \rpquote \id{\}}            = 
  \left\{ 
    \begin{array}{ccc}
      \lpquote Q \rpquote & & x \nameeq \lpquote P \rpquote \\
      x & & otherwise \\
    \end{array}
  \right. \nonumber
\end{eqnarray}

and $z$ is chosen distinct from $\quotep{P}$, $\quotep{Q}$, the free
names in $Q$, and all the names in $R$. Our $\alpha$-equivalence will
be built in the standard way from this substitution.

\begin{remark}\label{rem:no_self_referential_names}
  One consequence of these definitions is that $\forall P. \quotep{P}
  \not\in \freenames{P}$.
\end{remark}

\subsection{ Dynamic quote: an example }

Anticipating something of what's to come, consider applying the
substitution, $\widehat{\id{\{}u / z \id{\}}}$, to the following pair
of processes, $\lift{w}{y!(z)}$ and $w[ \lpquote y!(z) \rpquote ]$.

\begin{eqnarray}
	\lift{w}{y!(z)}\widehat{\id{\{}u / z \id{\}}}
		& = &
		\lift{w}{y!(u)} \nonumber\\
	w[ \lpquote y!(z) \rpquote ] \widehat{ \id{\{}u / z \id{\}} }
		& = &
		w[ \lpquote y!(z) \rpquote ] \nonumber
\end{eqnarray}

Because the body of the process between quotes is impervious to
substitution, we get radically different answers. In fact, by
examining the first process in an input context,
e.g. $x?(z).\lift{w}{y!(z)}$, we see that the process under the lift
operator may be shaped by prefixed inputs binding a name inside it. In
this sense, the lift operator will be seen as a way to dynamically
construct processes before reifying them as names.

Finally equipped with these standard features we can present the
dynamics of the calculus.

\subsubsection{Operational semantics} 

Finally, we introduce the computational dynamics. What marks these
algebras as distinct from other more traditionally studied algebraic
structures, e.g. vector spaces or polynomial rings, is the manner in
which dynamics is captured. In traditional structures, dynamics is typically
expressed through morphisms between such structures, as in linear maps
between vector spaces or morphisms between rings. In algebras
associated with the semantics of computation, the dynamics is
expressed as part of the algebraic structure itself, through a
reduction reduction relation typically denoted by $\red$. Below, we
give a recursive presentation of this relation for the calculus used
in the encoding.

$\red \subseteq \pi \times \pi$
$\red : \pi \to \mathcal{P}(\pi)$

\begin{mathpar}
  \inferrule* [lab=Comm] { \textsf{match}( x_{src}, x_{trgt} ) } { x_{trgt}?(y)P \; | \; x_{src}!\langle {Q} \rangle \red P\{\quotep{Q}/y}\} }
  \and \\
  \inferrule* [lab=Par] {{P} \red {P}'} {{{P} | {Q}} \red {{P}' | {Q}}}
  \and
  \inferrule* [lab=Equiv]{{{P} \scong {P}'} \andalso {{P}' \red {Q}'} \andalso {{Q}' \scong {Q}}}{{P} \red {Q}}
\end{mathpar}

\begin{eqnarray*}
  match_{\equiv} (\quotep{P},\quotep{Q}) & := & P \equiv Q \\
  match_{\dagger}(\quotep{P},\quotep{Q}) & := & \forall R. P|Q \red^{*} R => R \red^{*} 0 \\
  match_{K}(\quotep{P},\quotep{Q}) & := & K \mbox{ for some context } K
\end{eqnarray*}

$u?(x)P | u!\langle Q \rangle \red P\{\quotep{Q}/x\}$

%We write $\wred$ for $\red^*$, and $P\red$ if $\exists Q $ such that $ P \red Q$.
We write $P\red$ if $\exists Q $ such that $ P \red Q$ and $P\not\red$, otherwise.

\section{Replication}

As mentioned before, it is known that replication (and hence
recursion) can be implemented in a higher-order process algebra
\cite{SangiorgiWalker}. As our first example of calculation with the
machinery thus far presented we give the construction explicitly in
the {\rhoc}.

\begin{eqnarray}
	D_{x} & := & \prefix{x}{y}{(\binpar{\outputp{x}{y}}{@{y}})} \nonumber\\
	\bangp_{x}{P} & := & \binpar{{x}!\langle{\binpar{D_{x}}{P}}\rangle}{D_{x}} \nonumber
\end{eqnarray}

\begin{eqnarray}
	\bangp_{x}{P} & & \nonumber\\
	=
	& {x}!\langle{(\prefix{x}{y}{(\outputp{x}{y} | @{y})) | P}}\rangle 
	      | \prefix{x}{y}{(\outputp{x}{y} | @{y})} & \nonumber\\
	\red
	& (\outputp{x}{y} | @{y})\substn{\quotep{(\prefix{x}{y}{(@{y} | \outputp{x}{y})) | P}}}{y} & \nonumber\\
	=
	& \outputp{x}{\quotep{(\prefix{x}{y}{(\outputp{x}{y} | @{y})) | P}}}
	  | {(\prefix{x}{y}{(\outputp{x}{y} | @{y})) | P}} & \nonumber\\
	\red
	& \ldots & \nonumber\\
	\red^*
	& P | P | \ldots & \nonumber
\end{eqnarray}

Of course, this encoding, as an implementation, runs away, unfolding
$\bangp{P}$ eagerly. A lazier and more implementable replication
operator, restricted to input-guarded processes, may be obtained as follows.

\begin{eqnarray}
\bangp{\prefix{u}{v}{P}} 
	:= 
	\binpar{\lift{x}{\prefix{u}{v}{(\binpar{D(x)}{P})}}}{D(x)} \nonumber
\end{eqnarray}

\begin{remark}
  Note that the lazier definition still does not deal with summation
  or mixed summation (i.e. sums over input and output). The reader is
  invited to construct definitions of replication that deal with these
  features. 

  Further, the definitions are parameterized in a name, $x$. Can you,
  gentle reader, make a definition that eliminates this parameter and
  guarantees no accidental interaction between the replication
  machinery and the process being replicated -- i.e. no accidental
  sharing of names used by the process to get its work done and the
  name(s) used by the replication to effect copying. This latter
  revision of the definition of replication is crucial to obtaining
  the expected identity $!!P \sim !P$.
\end{remark}

\begin{remark}\label{rem:paradoxical_combinator}
  The reader familiar with the lambda calculus will have noticed the
  similarity between $D$ and the paradoxical combinator.

  [Ed. note: the existence of this seems to suggest we have to be more
  restrictive on the set of processes and names we admit if we are to
  support no-cloning.]
\end{remark}

\subsubsection{Bisimulation}

The computational dynamics gives rise to another kind of equivalence,
the equivalence of computational behavior. As previously mentioned
this is typically captured \emph{via} some form of bisimulation.

% The notion we use in this paper is weak barbed bisimulation
% \cite{milner91polyadicpi}.

The notion we use in this paper is derived from weak barbed
bisimulation \cite{milner91polyadicpi}. 

\begin{definition}
An \emph{observation relation}, $\downarrow_{\mathcal N}$, over a set
of names, $\mathcal N$, is the smallest relation satisfying the rules
below.

\infrule[Out-barb]{y \in {\mathcal N}, \; x \nameeq y}
		  {\outputp{x}{v} \downarrow_{\mathcal N} x}
\infrule[Par-barb]{\mbox{$P\downarrow_{\mathcal N} x$ or $Q\downarrow_{\mathcal N} x$}}
		  {\binpar{P}{Q} \downarrow_{\mathcal N} x}

We write $P \Downarrow_{\mathcal N} x$ if there is $Q$ such that 
$P \wred Q$ and $Q \downarrow_{\mathcal N} x$.
\end{definition}

\begin{definition}
%\label{def.bbisim}
An  ${\mathcal N}$-\emph{barbed bisimulation} over a set of names, ${\mathcal N}$, is a symmetric binary relation 
${\mathcal S}_{\mathcal N}$ between agents such that $P\rel{S}_{\mathcal N}Q$ implies:
\begin{enumerate}
\item If $P \red P'$ then $Q \wred Q'$ and $P'\rel{S}_{\mathcal N} Q'$.
\item If $P\downarrow_{\mathcal N} x$, then $Q\Downarrow_{\mathcal N} x$.
\end{enumerate}
$P$ is ${\mathcal N}$-barbed bisimilar to $Q$, written
$P \wbbisim_{\mathcal N} Q$, if $P \rel{S}_{\mathcal N} Q$ for some ${\mathcal N}$-barbed bisimulation ${\mathcal S}_{\mathcal N}$.
\end{definition}

$\mathcal{R} \subseteq \pi \times \pi$

$P \mathcal{R} Q => \forall P'. P \red P' \Rightarrow \exists Q'. Q \red Q', P' \mathcal{R} Q'$

$P \vdash x \Rightarrow Q \vdash x$

\begin{mathpar}
  \inferrule*[lab=Out-barb]{x \nameeq y}{{y}!\langle{Q}\rangle \vdash x}
  \and
  \inferrule*[lab=Par-barb]{\mbox{$P\vdash x$ or $Q\vdash x$}}{\binpar{P}{Q} \vdash x}
\end{mathpar}

\subsubsection{Contexts}

One of the principle advantages of computational calculi like the
$\pi$-calculus is a well-defined notion of context,
contextual-equivalence and a correlation between
contextual-equivalence and notions of bisimulation. The notion of
context allows the decomposition of a process into (sub-)process and
its syntactic environment, its context. Thus, a context may be
thought of as a process with a ``hole'' (written $\Box$) in it. The
application of a context $M$ to a process $P$, written $M[P]$, is
tantamount to filling the hole in $M$ with $P$. In this paper we do
not need the full weight of this theory, but do make use of the notion
of context in the proof the main theorem. 

\begin{mathpar}
  \inferrule* [lab=summation] {} {{M_{M},M_{N}} \bc \Box \;|\; x.M_{A} \;|\; M_{M}+M_{N}}
  \and
  \inferrule* [lab=agent] {} {{M_{A}} \bc (\vec{x})M_{P} \;| \; \clift{P_0,\ldots,M_{P},\ldots,P_N}}
  \and \\
  \inferrule* [lab=process] {} {{M_{P}} \bc M_{N} \;| \;P|M_{P} }
\end{mathpar} 

\begin{mathpar}
  \inferrule* [lab=sychronization] {} {M_{N} \bc \Box \;|\; x?M_{F} \;|\; x!M_{C}}
  \and
  \inferrule* [lab=abstraction] {} {{M_{F}} \bc (x)M_{P} }
  \and
  \inferrule* [lab=concretion] {} {{M_{C}} \bc \langle M_{P} \rangle }
  \and \\
  \inferrule* [lab=process] {} {{M_{P}} \bc M_{N} \;| \;P|M_{P} }
\end{mathpar}

\begin{definition}[contextual application] Given a context $M$, and
  process $P$, we define the \emph{contextual application}, $M[P] :=
  M\{P/\Box\}$. That is, the contextual application of M to P is the
  substitution of $P$ for $\Box$ in $M$.
\end{definition}

$\meaningof{-} : L \to \mathcal{P}(\pi)$

\begin{mathpar}
  \inferrule* [lab=collection] {} {\meaningof{true} = \pi, \and \meaningof{~E} = \pi \setminus \meaningof{E}, \and \meaningof{E_{1} \& E_{2}} = \meaningof{E_{1}} \cap \meaningof{E_{2}}}
\end{mathpar}

\begin{mathpar}
  \inferrule* [lab=structure] {} {\meaningof{0} = \{ P \in \pi | P \equiv 0 \}, \and \\ \meaningof{E_1 | E_2} = \{ P \in \pi | P \equiv P_{1} | P_{2}, P_{1} \in \meaningof{E_{1}}, P_{2} \in \meaningof{E_2}\} }
\end{mathpar}

\begin{mathpar}
 \inferrule* [lab=behavior] {} {\meaningof{\langle a?b \rangle E} = \{ P \in \pi | P \equiv Q | u?(y)P', \\ \and \\\\ \and \\ \;\;\; u \in \meaningof{a}, \forall z.P'\{z/y\} \in \meaningof{E\{z/b\}}\}, \and \\ \meaningof{a!E} = \{ P \in \pi | P \equiv Q | x!\langle P' \rangle, x \in \meaningof{a} P' \in \meaningof{E}\} }
\end{mathpar}

\begin{mathpar}
 \inferrule* [lab=nominal] {} {\meaningof{\quotep{E}} = \{ \quotep{P} \in \quotep{\pi} | P \in \meaningof{E} \}, \and \meaningof{\quotep{P}} = \{ \quotep{Q} \in \quotep{\pi} | P \equiv Q \} \and \\ \meaningof{@\quotep{E}} = \{ P \in \pi | P \equiv @x, x \in \meaningof{E} \}}
\end{mathpar}

\begin{eqnarray*}
  \\
  \meaningof{-} : TS \to ST
\end{eqnarray*}

\begin{eqnarray*}
  \\
  L : TS \to ST
\end{eqnarray*}

\begin{eqnarray*}
  \\
  P \models E \iff P \in \meaningof{E}
\end{eqnarray*}

\begin{eqnarray*}
  P \approx_{L} Q \iff \forall E \in L. P \models E \iff Q \models E
\end{eqnarray*}

\begin{eqnarray*}
  P \approx_{K} Q
\end{eqnarray*}

\begin{eqnarray*}
  P \approx Q
\end{eqnarray*}

$\approx_{K} = \approx = \approx_{L}$

\subsubsection{Contextual duality}

Note that contexts extend the quotation operation to a family of
operations from processes to names. Given a context, $M$, we can
define a \emph{nominal context}, $\quotep{M}$ by $\quotep{M}[P] :=
\quotep{M[P]}$. To foreshadow what is to come we observe that these
operations enjoy a duality with processes very much like the duality
between vectors and maps from vectors to scalars.

Further, because the calculus is essentially higher-order, we have a
correspondence between contexts and processes. More specifically,
given a name $x$ and a context $M$ we can construct $M^{*}_{x}$ such
that 

\begin{mathpar}
  M^{*}_{x} | \lift{x}{P} \red M[P]
\end{mathpar}

namely,

\begin{mathpar}
  M^{*}_{x} := x?(u).M[\dropn{u}]
\end{mathpar}

The dependence of $M^{*}_{x}$ on a name makes it an abstraction, 

\begin{mathpar}
  M^{*} := (x)x?(u).M[\dropn{u}]
\end{mathpar}

\subsection{Additional notation}

It will sometimes be convenient to denote the process a name
quotes. We already have the notation $x = \quotep{P}$, but it will be
convenient to introduce an alternate notation, $\procn{x}$, when we
want to emphasize the connection to the use of the name. Note that, by
virtue of name equivalence, $\quotep{\procn{x}} \nameeq x$; so, the
notation is consistent with previous definitions.

Further, because names have structure it is possible to effect
substitutions on the basis of that structure. This means we need to
upgrade our notation for substitutions, which we accomplish by
adapting comprehension notation. Thus,

\begin{mathpar}
  P\{ y / x : x \in S \}
\end{mathpar}

is interpreted to mean the process derived from P by replacing (in a
capture-avoiding manner) each occurrence of $x$ in $S$ by $y$. For example,

\begin{mathpar}
  P\{ \quotep{\procn{x}|\procn{x}} / x : x \in \freenames{P} \}
\end{mathpar}

will replace each (occurrence) of a free name $x$ in $P$ by
$\quotep{\procn{x}|\procn{x}}$.

Also, we will avail ourselves of the notation $x^{L}$ and $x^{R}$ to
denote injections of a name into disjoint copies of the name
space. There are numerous ways to accomplish this. One example can be
found in \cite{MeredithR05}. This notation overloads to vectors of
names: $\vec{x}^{\pi} := (x_{i}^{\pi} \; : \; 0 \leq i < |\vec{x}| )$ where $\pi \in \{L,R\}$.

We also use $P^{\Box} := P|\Box$.

In \cite{MeredithR05} an interpretation of the new operator is
given. It turns out that there are several possible interpretations
all enjoying the requisite algebraic properties of the operator (see
\cite{milner91polyadicpi}). We will therefore make liberal use of
$(\nu\; \vec{x})P$.

% subsection the_syntax_and_semantics_of_the_notation_system (end)   

\input{qm2pi.qmops} 

\input{qm2pi.sterngerlach} 

\input{qm2pi.metric} 

% section concurrent_process_calculi (end)

%\input{qm2pi.proofsketch}

% section proof sketch (end)

%\input{qm2pi.slviaknots} 

% section spatial logic via knots (end)

\input{qm2pi.conclusion}

% section conclusion (end)

%\input{qm2pi.dtcodes} 

% section wiring algorithm (end)

\input{qm2pi.ack} 

% section acknowledgments (end)

\newpage


\bibliographystyle{plain}   
\bibliography{../../biblios/main.bib}

\input{qm2pi.rhodetails}

\end{document}

 

\documentclass[12pt]{llncs}
%\documentclass{jktr}

\usepackage[pdftex]{hyperref}                   
\usepackage {listings}
\usepackage {mathpartir}
\usepackage{bcprules}
%\usepackage{listings}
                       
\usepackage{graphicx} 
%\usepackage[margins=2.5cm,nohead,nofoot]{geometry}
%\usepackage{geometry}
\usepackage{amsfonts}
\usepackage{amstext}
\usepackage{latexsym}
\usepackage{amssymb}
\usepackage{color}


%\include{myPreamble}
\include{qm2pi.local} 

%\ifpdf
%\usepackage[pdftex]{graphicx}
%\else
%\usepackage{graphicx}
%\fi

 % \ifpdf
%  \usepackage{pdfsync}
%  \if


%\title{Brief Article}
%\author{David F. Snyder}
%\author{L.G. Meredith}

%\address{Dept. of Math., Texas State University--San Marcos, San Marcos, TX 78666}
       
\pagestyle{empty}


\begin{document}

\lstset{language=[Objective]Caml,frame=shadowbox}

\input{qm2pi.front}

% section front matter (end)

\input{qm2pi.intro} 
 
% section introduction (end)

% \input{qm2pi.knotations} 

% section notation (end)

\input{qm2pi.process.calculi} 

% section concurrent_process_calculi_and_spatial_logics_ (end)
    
%\input{qm2pi.knots2pi} 

%\input{qm2pi.trefoil} 

%\input{qm2pi.mainthm} 

% subsection basic_interpretation (end)

%\input{qm2pi.rho.presentation} 
\subsection{The syntax and semantics of the notation system}\label{sub:the_syntax_and_semantics_of_the_notation_system} % (fold)

We now summarize a technical presentation of the calculus that
embodies our theory of dynamics. The typical presentation of such a
calculus follows the style of giving generators and relations on
them. The grammar, below, describing term constructors, freely
generates the set of processes, $\Proc$. This set is then quotiented
by a relation known as structural congruence and it is over this set
that the notion of dynamics is expressed. This presentation is
essentially that of \cite{MeredithR05} with the addition of
polyadicity and summation. For readability we have relegated some of
the technical subtleties to an appendix.

\subsubsection{Process grammar}\label{subsub:process_grammar}

\begin{mathpar}
  \inferrule* [lab=synchronization] {} {{M} \bc \pzero \;|\; x?F \;|\; x!C }
  \and
  \inferrule* [lab=abstraction] {} {{F} \bc (x)P}
  \and
  \inferrule* [lab=concretion] {} {{C} \bc \langle Q \rangle}
  \and
  \inferrule* [lab=process] {} {{P,Q} \bc M \;| \;P|Q \;|\; @{x}}
  \and
  \inferrule* [lab=name] {} {{x} \bc \quotep{P}}
\end{mathpar} 

Note that $\vec{x}$ (resp. $\vec{P}$) denotes a vector of names
(resp. processes) of length $|\vec{x}|$ (resp. $|\vec{P}|$). We adopt
the following useful abbreviations.

\begin{mathpar}
   x?(\vec{y}).P := x.(\vec{y})P \and  x\clift{\vec{P}} := x.\clift{\vec{P}}
   \and x!(y) := \lift{x}{\dropn{y}}
   \and \Pi_{i=0}^{n-1}P_i := P_0 | \ldots | P_{n-1}
\end{mathpar}

\subsubsection{Structural congruence}

\paragraph{Free and bound names and alpha-equivalence.} At the
core of structural equivalence is alpha-equivalence which identifies
process that are the same up to a change of variable. Formally, we
recognize the distinction between free and bound names. The free names
of a process, $\freenames{P}$, may be calculated recursively as
follows:

\begin{mathpar}
\freenames{\pzero} := \emptyset
  \and \\
  \freenames{x?(y).P} := \{ x \} \cup (\freenames{P} \setminus \{ y \})
  \and 
  \freenames{x!\langle P \rangle} := \{ x \} \cup \{ P \} 
  \and \\
  \freenames{P|Q} := \freenames{P} \cup \freenames{Q}
  \and \\
  \freenames{@{x}} := \{ x \}
\end{mathpar}

$\pi$
$\quotep{\pi}$

$\freenames{-} : \pi \to \mathcal{P}(\quotep{\pi})$

\begin{eqnarray*}
  \freenames{\pzero} & := & \emptyset \\
  \freenames{x?(y).P} & := & \{ x \} \cup (\freenames{P} \setminus \{ y \}) \\
  \freenames{x!\langle P \rangle} & := & \{ x \} \cup \{ P \} \\
  \freenames{P|Q} & := & \freenames{P} \cup \freenames{Q} \\
  \freenames{\dropn{x}} & := & \{ x \}
\end{eqnarray*}

The bound names of a process, $\boundnames{P}$, are those names occurring in $P$
that are not free. For example, in $x?(y).0$, the name $x$ is free, while $y$ is bound.

\begin{mathpar}
  \inferrule* [lab=monoidal-laws] {} { P|Q \equiv Q|P \and P|0 \equiv P \and P|(Q|R) \equiv (P|Q)|R }
\end{mathpar}

\begin{mathpar}
  \inferrule* [lab=alpha-equivalence] {} { (x)P \equiv (y)P\{y/x\} \and y \not\in \freenames{P} }
\end{mathpar}

\begin{definition}
Then two processes, $P,Q$, are alpha-equivalent if $P = Q\{\vec{y}/\vec{x}\}$ for
some $\vec{x} \in \boundnames{Q},\vec{y} \in \boundnames{P}$, where $Q\{\vec{y}/\vec{x}\}$
denotes the capture-avoiding substitution of $\vec{y}$ for $\vec{x}$ in $Q$.
\end{definition}

\begin{definition}
  The {\em structural congruence} \cite{SangiorgiWalker} , $\equiv$,
  between processes is the least congruence containing
  alpha-equivalence, satisfying the abelian monoid laws
  (associativity, commutativity and $\pzero$ as identity) for parallel
  composition $|$ and for summation $+$.
\end{definition}

\subsection{Name equivalence}

We take name equivalence, written $\nameeq$, to be the smallest
equivalence relation generated by the following rules.

\begin{mathpar}
\inferrule*[lab=Quote-drop]
{ }
{ \quotep{@{x}} \nameeq x }

\inferrule*[lab=Struct-equiv]
{ P \scong Q }
{ \quotep{P} \nameeq \quotep{Q} }
\end{mathpar}

The astute reader will have noticed that the mutual recursion of names
and processes imposes a mutual recursion on alpha-equivalence and
structural equivalence via name-equivalence. Fortunately, all of this
works out pleasantly and we may calculate in the natural way, free of
concern. The reader interested in the details is referred to the
appendix \ref{appendix:rho_details}.

\subsection{Substitution}

We use $\Proc$ for the set of processes, $\QProc$ for the set of
names, and $\id{\{}\vec{y} / \vec{x} \id{\}}$ to denote partial maps,
$s : \QProc \rightarrow \QProc$. A map, $s$ lifts, uniquely, to a map
on process terms, $\widehat{s} : \Proc \rightarrow \Proc$ by the
following equations.

\begin{mathpar}
  (0) \psubstp{Q}{P} := 0 \\
  (R \juxtap S) \psubstp{Q}{P}
  :=    
  (R)\psubstp{Q}{P} \juxtap (S) \psubstp{Q}{P} \\
  (x?(y).R) \psubstp{Q}{P}    
  :=    
  (x)\substp{Q}{P} (z)\concat( (R \psubstn{z}{y}) \psubstp{Q}{P} ) \\
  (\lift{x}{R}) \psubstp{Q}{P}  
  :=
  \lift{(x)\substp{Q}{P}}{ R \psubstp{Q}{P} } \\
%   (\dropn{x})  \psubstp{Q}{P}       
%   := 
%   \left\{ 
%     \begin{array}{ccc} 
%       \dropn{\quotep{Q}} & & x \nameeq \quotep{P} \\
%       \dropn{x} & & otherwise \\
%     \end{array}
%   \right. 
  (\dropn{x})  \psubstp{Q}{P}       
  := 
  \left\{ 
    \begin{array}{ccc} 
      Q & & x \nameeq \quotep{P} \\
      \dropn{x} & & otherwise \\
    \end{array}
  \right.
\end{mathpar}
 

where

\begin{eqnarray}
  (x)\id{\{} \lpquote Q \rpquote / \lpquote P \rpquote \id{\}}            = 
  \left\{ 
    \begin{array}{ccc}
      \lpquote Q \rpquote & & x \nameeq \lpquote P \rpquote \\
      x & & otherwise \\
    \end{array}
  \right. \nonumber
\end{eqnarray}

and $z$ is chosen distinct from $\quotep{P}$, $\quotep{Q}$, the free
names in $Q$, and all the names in $R$. Our $\alpha$-equivalence will
be built in the standard way from this substitution.

\begin{remark}\label{rem:no_self_referential_names}
  One consequence of these definitions is that $\forall P. \quotep{P}
  \not\in \freenames{P}$.
\end{remark}

\subsection{ Dynamic quote: an example }

Anticipating something of what's to come, consider applying the
substitution, $\widehat{\id{\{}u / z \id{\}}}$, to the following pair
of processes, $\lift{w}{y!(z)}$ and $w[ \lpquote y!(z) \rpquote ]$.

\begin{eqnarray}
	\lift{w}{y!(z)}\widehat{\id{\{}u / z \id{\}}}
		& = &
		\lift{w}{y!(u)} \nonumber\\
	w[ \lpquote y!(z) \rpquote ] \widehat{ \id{\{}u / z \id{\}} }
		& = &
		w[ \lpquote y!(z) \rpquote ] \nonumber
\end{eqnarray}

Because the body of the process between quotes is impervious to
substitution, we get radically different answers. In fact, by
examining the first process in an input context,
e.g. $x?(z).\lift{w}{y!(z)}$, we see that the process under the lift
operator may be shaped by prefixed inputs binding a name inside it. In
this sense, the lift operator will be seen as a way to dynamically
construct processes before reifying them as names.

Finally equipped with these standard features we can present the
dynamics of the calculus.

\subsubsection{Operational semantics} 

Finally, we introduce the computational dynamics. What marks these
algebras as distinct from other more traditionally studied algebraic
structures, e.g. vector spaces or polynomial rings, is the manner in
which dynamics is captured. In traditional structures, dynamics is typically
expressed through morphisms between such structures, as in linear maps
between vector spaces or morphisms between rings. In algebras
associated with the semantics of computation, the dynamics is
expressed as part of the algebraic structure itself, through a
reduction reduction relation typically denoted by $\red$. Below, we
give a recursive presentation of this relation for the calculus used
in the encoding.

$\red \subseteq \pi \times \pi$
$\red : \pi \to \mathcal{P}(\pi)$

\begin{mathpar}
  \inferrule* [lab=Comm] { \textsf{match}( x_{src}, x_{trgt} ) } { x_{trgt}?(y)P \; | \; x_{src}!\langle {Q} \rangle \red P\{\quotep{Q}/y}\} }
  \and \\
  \inferrule* [lab=Par] {{P} \red {P}'} {{{P} | {Q}} \red {{P}' | {Q}}}
  \and
  \inferrule* [lab=Equiv]{{{P} \scong {P}'} \andalso {{P}' \red {Q}'} \andalso {{Q}' \scong {Q}}}{{P} \red {Q}}
\end{mathpar}

\begin{eqnarray*}
  match_{\equiv} (\quotep{P},\quotep{Q}) & := & P \equiv Q \\
  match_{\dagger}(\quotep{P},\quotep{Q}) & := & \forall R. P|Q \red^{*} R => R \red^{*} 0 \\
  match_{K}(\quotep{P},\quotep{Q}) & := & K \mbox{ for some context } K
\end{eqnarray*}

$u?(x)P | u!\langle Q \rangle \red P\{\quotep{Q}/x\}$

%We write $\wred$ for $\red^*$, and $P\red$ if $\exists Q $ such that $ P \red Q$.
We write $P\red$ if $\exists Q $ such that $ P \red Q$ and $P\not\red$, otherwise.

\section{Replication}

As mentioned before, it is known that replication (and hence
recursion) can be implemented in a higher-order process algebra
\cite{SangiorgiWalker}. As our first example of calculation with the
machinery thus far presented we give the construction explicitly in
the {\rhoc}.

\begin{eqnarray}
	D_{x} & := & \prefix{x}{y}{(\binpar{\outputp{x}{y}}{@{y}})} \nonumber\\
	\bangp_{x}{P} & := & \binpar{{x}!\langle{\binpar{D_{x}}{P}}\rangle}{D_{x}} \nonumber
\end{eqnarray}

\begin{eqnarray}
	\bangp_{x}{P} & & \nonumber\\
	=
	& {x}!\langle{(\prefix{x}{y}{(\outputp{x}{y} | @{y})) | P}}\rangle 
	      | \prefix{x}{y}{(\outputp{x}{y} | @{y})} & \nonumber\\
	\red
	& (\outputp{x}{y} | @{y})\substn{\quotep{(\prefix{x}{y}{(@{y} | \outputp{x}{y})) | P}}}{y} & \nonumber\\
	=
	& \outputp{x}{\quotep{(\prefix{x}{y}{(\outputp{x}{y} | @{y})) | P}}}
	  | {(\prefix{x}{y}{(\outputp{x}{y} | @{y})) | P}} & \nonumber\\
	\red
	& \ldots & \nonumber\\
	\red^*
	& P | P | \ldots & \nonumber
\end{eqnarray}

Of course, this encoding, as an implementation, runs away, unfolding
$\bangp{P}$ eagerly. A lazier and more implementable replication
operator, restricted to input-guarded processes, may be obtained as follows.

\begin{eqnarray}
\bangp{\prefix{u}{v}{P}} 
	:= 
	\binpar{\lift{x}{\prefix{u}{v}{(\binpar{D(x)}{P})}}}{D(x)} \nonumber
\end{eqnarray}

\begin{remark}
  Note that the lazier definition still does not deal with summation
  or mixed summation (i.e. sums over input and output). The reader is
  invited to construct definitions of replication that deal with these
  features. 

  Further, the definitions are parameterized in a name, $x$. Can you,
  gentle reader, make a definition that eliminates this parameter and
  guarantees no accidental interaction between the replication
  machinery and the process being replicated -- i.e. no accidental
  sharing of names used by the process to get its work done and the
  name(s) used by the replication to effect copying. This latter
  revision of the definition of replication is crucial to obtaining
  the expected identity $!!P \sim !P$.
\end{remark}

\begin{remark}\label{rem:paradoxical_combinator}
  The reader familiar with the lambda calculus will have noticed the
  similarity between $D$ and the paradoxical combinator.

  [Ed. note: the existence of this seems to suggest we have to be more
  restrictive on the set of processes and names we admit if we are to
  support no-cloning.]
\end{remark}

\subsubsection{Bisimulation}

The computational dynamics gives rise to another kind of equivalence,
the equivalence of computational behavior. As previously mentioned
this is typically captured \emph{via} some form of bisimulation.

% The notion we use in this paper is weak barbed bisimulation
% \cite{milner91polyadicpi}.

The notion we use in this paper is derived from weak barbed
bisimulation \cite{milner91polyadicpi}. 

\begin{definition}
An \emph{observation relation}, $\downarrow_{\mathcal N}$, over a set
of names, $\mathcal N$, is the smallest relation satisfying the rules
below.

\infrule[Out-barb]{y \in {\mathcal N}, \; x \nameeq y}
		  {\outputp{x}{v} \downarrow_{\mathcal N} x}
\infrule[Par-barb]{\mbox{$P\downarrow_{\mathcal N} x$ or $Q\downarrow_{\mathcal N} x$}}
		  {\binpar{P}{Q} \downarrow_{\mathcal N} x}

We write $P \Downarrow_{\mathcal N} x$ if there is $Q$ such that 
$P \wred Q$ and $Q \downarrow_{\mathcal N} x$.
\end{definition}

\begin{definition}
%\label{def.bbisim}
An  ${\mathcal N}$-\emph{barbed bisimulation} over a set of names, ${\mathcal N}$, is a symmetric binary relation 
${\mathcal S}_{\mathcal N}$ between agents such that $P\rel{S}_{\mathcal N}Q$ implies:
\begin{enumerate}
\item If $P \red P'$ then $Q \wred Q'$ and $P'\rel{S}_{\mathcal N} Q'$.
\item If $P\downarrow_{\mathcal N} x$, then $Q\Downarrow_{\mathcal N} x$.
\end{enumerate}
$P$ is ${\mathcal N}$-barbed bisimilar to $Q$, written
$P \wbbisim_{\mathcal N} Q$, if $P \rel{S}_{\mathcal N} Q$ for some ${\mathcal N}$-barbed bisimulation ${\mathcal S}_{\mathcal N}$.
\end{definition}

$\mathcal{R} \subseteq \pi \times \pi$

$P \mathcal{R} Q => \forall P'. P \red P' \Rightarrow \exists Q'. Q \red Q', P' \mathcal{R} Q'$

$P \vdash x \Rightarrow Q \vdash x$

\begin{mathpar}
  \inferrule*[lab=Out-barb]{x \nameeq y}{{y}!\langle{Q}\rangle \vdash x}
  \and
  \inferrule*[lab=Par-barb]{\mbox{$P\vdash x$ or $Q\vdash x$}}{\binpar{P}{Q} \vdash x}
\end{mathpar}

\subsubsection{Contexts}

One of the principle advantages of computational calculi like the
$\pi$-calculus is a well-defined notion of context,
contextual-equivalence and a correlation between
contextual-equivalence and notions of bisimulation. The notion of
context allows the decomposition of a process into (sub-)process and
its syntactic environment, its context. Thus, a context may be
thought of as a process with a ``hole'' (written $\Box$) in it. The
application of a context $M$ to a process $P$, written $M[P]$, is
tantamount to filling the hole in $M$ with $P$. In this paper we do
not need the full weight of this theory, but do make use of the notion
of context in the proof the main theorem. 

\begin{mathpar}
  \inferrule* [lab=summation] {} {{M_{M},M_{N}} \bc \Box \;|\; x.M_{A} \;|\; M_{M}+M_{N}}
  \and
  \inferrule* [lab=agent] {} {{M_{A}} \bc (\vec{x})M_{P} \;| \; \clift{P_0,\ldots,M_{P},\ldots,P_N}}
  \and \\
  \inferrule* [lab=process] {} {{M_{P}} \bc M_{N} \;| \;P|M_{P} }
\end{mathpar} 

\begin{mathpar}
  \inferrule* [lab=sychronization] {} {M_{N} \bc \Box \;|\; x?M_{F} \;|\; x!M_{C}}
  \and
  \inferrule* [lab=abstraction] {} {{M_{F}} \bc (x)M_{P} }
  \and
  \inferrule* [lab=concretion] {} {{M_{C}} \bc \langle M_{P} \rangle }
  \and \\
  \inferrule* [lab=process] {} {{M_{P}} \bc M_{N} \;| \;P|M_{P} }
\end{mathpar}

\begin{definition}[contextual application] Given a context $M$, and
  process $P$, we define the \emph{contextual application}, $M[P] :=
  M\{P/\Box\}$. That is, the contextual application of M to P is the
  substitution of $P$ for $\Box$ in $M$.
\end{definition}

$\meaningof{-} : L \to \mathcal{P}(\pi)$

\begin{mathpar}
  \inferrule* [lab=collection] {} {\meaningof{true} = \pi, \and \meaningof{~E} = \pi \setminus \meaningof{E}, \and \meaningof{E_{1} \& E_{2}} = \meaningof{E_{1}} \cap \meaningof{E_{2}}}
\end{mathpar}

\begin{mathpar}
  \inferrule* [lab=structure] {} {\meaningof{0} = \{ P \in \pi | P \equiv 0 \}, \and \\ \meaningof{E_1 | E_2} = \{ P \in \pi | P \equiv P_{1} | P_{2}, P_{1} \in \meaningof{E_{1}}, P_{2} \in \meaningof{E_2}\} }
\end{mathpar}

\begin{mathpar}
 \inferrule* [lab=behavior] {} {\meaningof{\langle a?b \rangle E} = \{ P \in \pi | P \equiv Q | u?(y)P', \\ \and \\\\ \and \\ \;\;\; u \in \meaningof{a}, \forall z.P'\{z/y\} \in \meaningof{E\{z/b\}}\}, \and \\ \meaningof{a!E} = \{ P \in \pi | P \equiv Q | x!\langle P' \rangle, x \in \meaningof{a} P' \in \meaningof{E}\} }
\end{mathpar}

\begin{mathpar}
 \inferrule* [lab=nominal] {} {\meaningof{\quotep{E}} = \{ \quotep{P} \in \quotep{\pi} | P \in \meaningof{E} \}, \and \meaningof{\quotep{P}} = \{ \quotep{Q} \in \quotep{\pi} | P \equiv Q \} \and \\ \meaningof{@\quotep{E}} = \{ P \in \pi | P \equiv @x, x \in \meaningof{E} \}}
\end{mathpar}

\begin{eqnarray*}
  \\
  \meaningof{-} : TS \to ST
\end{eqnarray*}

\begin{eqnarray*}
  \\
  L : TS \to ST
\end{eqnarray*}

\begin{eqnarray*}
  \\
  P \models E \iff P \in \meaningof{E}
\end{eqnarray*}

\begin{eqnarray*}
  P \approx_{L} Q \iff \forall E \in L. P \models E \iff Q \models E
\end{eqnarray*}

\begin{eqnarray*}
  P \approx_{K} Q
\end{eqnarray*}

\begin{eqnarray*}
  P \approx Q
\end{eqnarray*}

$\approx_{K} = \approx = \approx_{L}$

\subsubsection{Contextual duality}

Note that contexts extend the quotation operation to a family of
operations from processes to names. Given a context, $M$, we can
define a \emph{nominal context}, $\quotep{M}$ by $\quotep{M}[P] :=
\quotep{M[P]}$. To foreshadow what is to come we observe that these
operations enjoy a duality with processes very much like the duality
between vectors and maps from vectors to scalars.

Further, because the calculus is essentially higher-order, we have a
correspondence between contexts and processes. More specifically,
given a name $x$ and a context $M$ we can construct $M^{*}_{x}$ such
that 

\begin{mathpar}
  M^{*}_{x} | \lift{x}{P} \red M[P]
\end{mathpar}

namely,

\begin{mathpar}
  M^{*}_{x} := x?(u).M[\dropn{u}]
\end{mathpar}

The dependence of $M^{*}_{x}$ on a name makes it an abstraction, 

\begin{mathpar}
  M^{*} := (x)x?(u).M[\dropn{u}]
\end{mathpar}

\subsection{Additional notation}

It will sometimes be convenient to denote the process a name
quotes. We already have the notation $x = \quotep{P}$, but it will be
convenient to introduce an alternate notation, $\procn{x}$, when we
want to emphasize the connection to the use of the name. Note that, by
virtue of name equivalence, $\quotep{\procn{x}} \nameeq x$; so, the
notation is consistent with previous definitions.

Further, because names have structure it is possible to effect
substitutions on the basis of that structure. This means we need to
upgrade our notation for substitutions, which we accomplish by
adapting comprehension notation. Thus,

\begin{mathpar}
  P\{ y / x : x \in S \}
\end{mathpar}

is interpreted to mean the process derived from P by replacing (in a
capture-avoiding manner) each occurrence of $x$ in $S$ by $y$. For example,

\begin{mathpar}
  P\{ \quotep{\procn{x}|\procn{x}} / x : x \in \freenames{P} \}
\end{mathpar}

will replace each (occurrence) of a free name $x$ in $P$ by
$\quotep{\procn{x}|\procn{x}}$.

Also, we will avail ourselves of the notation $x^{L}$ and $x^{R}$ to
denote injections of a name into disjoint copies of the name
space. There are numerous ways to accomplish this. One example can be
found in \cite{MeredithR05}. This notation overloads to vectors of
names: $\vec{x}^{\pi} := (x_{i}^{\pi} \; : \; 0 \leq i < |\vec{x}| )$ where $\pi \in \{L,R\}$.

We also use $P^{\Box} := P|\Box$.

In \cite{MeredithR05} an interpretation of the new operator is
given. It turns out that there are several possible interpretations
all enjoying the requisite algebraic properties of the operator (see
\cite{milner91polyadicpi}). We will therefore make liberal use of
$(\nu\; \vec{x})P$.

% subsection the_syntax_and_semantics_of_the_notation_system (end)   

\input{qm2pi.qmops} 

\input{qm2pi.sterngerlach} 

\input{qm2pi.metric} 

% section concurrent_process_calculi (end)

%\input{qm2pi.proofsketch}

% section proof sketch (end)

%\input{qm2pi.slviaknots} 

% section spatial logic via knots (end)

\input{qm2pi.conclusion}

% section conclusion (end)

%\input{qm2pi.dtcodes} 

% section wiring algorithm (end)

\input{qm2pi.ack} 

% section acknowledgments (end)

\newpage


\bibliographystyle{plain}   
\bibliography{../../biblios/main.bib}

\input{qm2pi.rhodetails}

\end{document}

 

% section concurrent_process_calculi (end)

%\documentclass[12pt]{llncs}
%\documentclass{jktr}

\usepackage[pdftex]{hyperref}                   
\usepackage {listings}
\usepackage {mathpartir}
\usepackage{bcprules}
%\usepackage{listings}
                       
\usepackage{graphicx} 
%\usepackage[margins=2.5cm,nohead,nofoot]{geometry}
%\usepackage{geometry}
\usepackage{amsfonts}
\usepackage{amstext}
\usepackage{latexsym}
\usepackage{amssymb}
\usepackage{color}


%\include{myPreamble}
\include{qm2pi.local} 

%\ifpdf
%\usepackage[pdftex]{graphicx}
%\else
%\usepackage{graphicx}
%\fi

 % \ifpdf
%  \usepackage{pdfsync}
%  \if


%\title{Brief Article}
%\author{David F. Snyder}
%\author{L.G. Meredith}

%\address{Dept. of Math., Texas State University--San Marcos, San Marcos, TX 78666}
       
\pagestyle{empty}


\begin{document}

\lstset{language=[Objective]Caml,frame=shadowbox}

\input{qm2pi.front}

% section front matter (end)

\input{qm2pi.intro} 
 
% section introduction (end)

% \input{qm2pi.knotations} 

% section notation (end)

\input{qm2pi.process.calculi} 

% section concurrent_process_calculi_and_spatial_logics_ (end)
    
%\input{qm2pi.knots2pi} 

%\input{qm2pi.trefoil} 

%\input{qm2pi.mainthm} 

% subsection basic_interpretation (end)

%\input{qm2pi.rho.presentation} 
\subsection{The syntax and semantics of the notation system}\label{sub:the_syntax_and_semantics_of_the_notation_system} % (fold)

We now summarize a technical presentation of the calculus that
embodies our theory of dynamics. The typical presentation of such a
calculus follows the style of giving generators and relations on
them. The grammar, below, describing term constructors, freely
generates the set of processes, $\Proc$. This set is then quotiented
by a relation known as structural congruence and it is over this set
that the notion of dynamics is expressed. This presentation is
essentially that of \cite{MeredithR05} with the addition of
polyadicity and summation. For readability we have relegated some of
the technical subtleties to an appendix.

\subsubsection{Process grammar}\label{subsub:process_grammar}

\begin{mathpar}
  \inferrule* [lab=synchronization] {} {{M} \bc \pzero \;|\; x?F \;|\; x!C }
  \and
  \inferrule* [lab=abstraction] {} {{F} \bc (x)P}
  \and
  \inferrule* [lab=concretion] {} {{C} \bc \langle Q \rangle}
  \and
  \inferrule* [lab=process] {} {{P,Q} \bc M \;| \;P|Q \;|\; @{x}}
  \and
  \inferrule* [lab=name] {} {{x} \bc \quotep{P}}
\end{mathpar} 

Note that $\vec{x}$ (resp. $\vec{P}$) denotes a vector of names
(resp. processes) of length $|\vec{x}|$ (resp. $|\vec{P}|$). We adopt
the following useful abbreviations.

\begin{mathpar}
   x?(\vec{y}).P := x.(\vec{y})P \and  x\clift{\vec{P}} := x.\clift{\vec{P}}
   \and x!(y) := \lift{x}{\dropn{y}}
   \and \Pi_{i=0}^{n-1}P_i := P_0 | \ldots | P_{n-1}
\end{mathpar}

\subsubsection{Structural congruence}

\paragraph{Free and bound names and alpha-equivalence.} At the
core of structural equivalence is alpha-equivalence which identifies
process that are the same up to a change of variable. Formally, we
recognize the distinction between free and bound names. The free names
of a process, $\freenames{P}$, may be calculated recursively as
follows:

\begin{mathpar}
\freenames{\pzero} := \emptyset
  \and \\
  \freenames{x?(y).P} := \{ x \} \cup (\freenames{P} \setminus \{ y \})
  \and 
  \freenames{x!\langle P \rangle} := \{ x \} \cup \{ P \} 
  \and \\
  \freenames{P|Q} := \freenames{P} \cup \freenames{Q}
  \and \\
  \freenames{@{x}} := \{ x \}
\end{mathpar}

$\pi$
$\quotep{\pi}$

$\freenames{-} : \pi \to \mathcal{P}(\quotep{\pi})$

\begin{eqnarray*}
  \freenames{\pzero} & := & \emptyset \\
  \freenames{x?(y).P} & := & \{ x \} \cup (\freenames{P} \setminus \{ y \}) \\
  \freenames{x!\langle P \rangle} & := & \{ x \} \cup \{ P \} \\
  \freenames{P|Q} & := & \freenames{P} \cup \freenames{Q} \\
  \freenames{\dropn{x}} & := & \{ x \}
\end{eqnarray*}

The bound names of a process, $\boundnames{P}$, are those names occurring in $P$
that are not free. For example, in $x?(y).0$, the name $x$ is free, while $y$ is bound.

\begin{mathpar}
  \inferrule* [lab=monoidal-laws] {} { P|Q \equiv Q|P \and P|0 \equiv P \and P|(Q|R) \equiv (P|Q)|R }
\end{mathpar}

\begin{mathpar}
  \inferrule* [lab=alpha-equivalence] {} { (x)P \equiv (y)P\{y/x\} \and y \not\in \freenames{P} }
\end{mathpar}

\begin{definition}
Then two processes, $P,Q$, are alpha-equivalent if $P = Q\{\vec{y}/\vec{x}\}$ for
some $\vec{x} \in \boundnames{Q},\vec{y} \in \boundnames{P}$, where $Q\{\vec{y}/\vec{x}\}$
denotes the capture-avoiding substitution of $\vec{y}$ for $\vec{x}$ in $Q$.
\end{definition}

\begin{definition}
  The {\em structural congruence} \cite{SangiorgiWalker} , $\equiv$,
  between processes is the least congruence containing
  alpha-equivalence, satisfying the abelian monoid laws
  (associativity, commutativity and $\pzero$ as identity) for parallel
  composition $|$ and for summation $+$.
\end{definition}

\subsection{Name equivalence}

We take name equivalence, written $\nameeq$, to be the smallest
equivalence relation generated by the following rules.

\begin{mathpar}
\inferrule*[lab=Quote-drop]
{ }
{ \quotep{@{x}} \nameeq x }

\inferrule*[lab=Struct-equiv]
{ P \scong Q }
{ \quotep{P} \nameeq \quotep{Q} }
\end{mathpar}

The astute reader will have noticed that the mutual recursion of names
and processes imposes a mutual recursion on alpha-equivalence and
structural equivalence via name-equivalence. Fortunately, all of this
works out pleasantly and we may calculate in the natural way, free of
concern. The reader interested in the details is referred to the
appendix \ref{appendix:rho_details}.

\subsection{Substitution}

We use $\Proc$ for the set of processes, $\QProc$ for the set of
names, and $\id{\{}\vec{y} / \vec{x} \id{\}}$ to denote partial maps,
$s : \QProc \rightarrow \QProc$. A map, $s$ lifts, uniquely, to a map
on process terms, $\widehat{s} : \Proc \rightarrow \Proc$ by the
following equations.

\begin{mathpar}
  (0) \psubstp{Q}{P} := 0 \\
  (R \juxtap S) \psubstp{Q}{P}
  :=    
  (R)\psubstp{Q}{P} \juxtap (S) \psubstp{Q}{P} \\
  (x?(y).R) \psubstp{Q}{P}    
  :=    
  (x)\substp{Q}{P} (z)\concat( (R \psubstn{z}{y}) \psubstp{Q}{P} ) \\
  (\lift{x}{R}) \psubstp{Q}{P}  
  :=
  \lift{(x)\substp{Q}{P}}{ R \psubstp{Q}{P} } \\
%   (\dropn{x})  \psubstp{Q}{P}       
%   := 
%   \left\{ 
%     \begin{array}{ccc} 
%       \dropn{\quotep{Q}} & & x \nameeq \quotep{P} \\
%       \dropn{x} & & otherwise \\
%     \end{array}
%   \right. 
  (\dropn{x})  \psubstp{Q}{P}       
  := 
  \left\{ 
    \begin{array}{ccc} 
      Q & & x \nameeq \quotep{P} \\
      \dropn{x} & & otherwise \\
    \end{array}
  \right.
\end{mathpar}
 

where

\begin{eqnarray}
  (x)\id{\{} \lpquote Q \rpquote / \lpquote P \rpquote \id{\}}            = 
  \left\{ 
    \begin{array}{ccc}
      \lpquote Q \rpquote & & x \nameeq \lpquote P \rpquote \\
      x & & otherwise \\
    \end{array}
  \right. \nonumber
\end{eqnarray}

and $z$ is chosen distinct from $\quotep{P}$, $\quotep{Q}$, the free
names in $Q$, and all the names in $R$. Our $\alpha$-equivalence will
be built in the standard way from this substitution.

\begin{remark}\label{rem:no_self_referential_names}
  One consequence of these definitions is that $\forall P. \quotep{P}
  \not\in \freenames{P}$.
\end{remark}

\subsection{ Dynamic quote: an example }

Anticipating something of what's to come, consider applying the
substitution, $\widehat{\id{\{}u / z \id{\}}}$, to the following pair
of processes, $\lift{w}{y!(z)}$ and $w[ \lpquote y!(z) \rpquote ]$.

\begin{eqnarray}
	\lift{w}{y!(z)}\widehat{\id{\{}u / z \id{\}}}
		& = &
		\lift{w}{y!(u)} \nonumber\\
	w[ \lpquote y!(z) \rpquote ] \widehat{ \id{\{}u / z \id{\}} }
		& = &
		w[ \lpquote y!(z) \rpquote ] \nonumber
\end{eqnarray}

Because the body of the process between quotes is impervious to
substitution, we get radically different answers. In fact, by
examining the first process in an input context,
e.g. $x?(z).\lift{w}{y!(z)}$, we see that the process under the lift
operator may be shaped by prefixed inputs binding a name inside it. In
this sense, the lift operator will be seen as a way to dynamically
construct processes before reifying them as names.

Finally equipped with these standard features we can present the
dynamics of the calculus.

\subsubsection{Operational semantics} 

Finally, we introduce the computational dynamics. What marks these
algebras as distinct from other more traditionally studied algebraic
structures, e.g. vector spaces or polynomial rings, is the manner in
which dynamics is captured. In traditional structures, dynamics is typically
expressed through morphisms between such structures, as in linear maps
between vector spaces or morphisms between rings. In algebras
associated with the semantics of computation, the dynamics is
expressed as part of the algebraic structure itself, through a
reduction reduction relation typically denoted by $\red$. Below, we
give a recursive presentation of this relation for the calculus used
in the encoding.

$\red \subseteq \pi \times \pi$
$\red : \pi \to \mathcal{P}(\pi)$

\begin{mathpar}
  \inferrule* [lab=Comm] { \textsf{match}( x_{src}, x_{trgt} ) } { x_{trgt}?(y)P \; | \; x_{src}!\langle {Q} \rangle \red P\{\quotep{Q}/y}\} }
  \and \\
  \inferrule* [lab=Par] {{P} \red {P}'} {{{P} | {Q}} \red {{P}' | {Q}}}
  \and
  \inferrule* [lab=Equiv]{{{P} \scong {P}'} \andalso {{P}' \red {Q}'} \andalso {{Q}' \scong {Q}}}{{P} \red {Q}}
\end{mathpar}

\begin{eqnarray*}
  match_{\equiv} (\quotep{P},\quotep{Q}) & := & P \equiv Q \\
  match_{\dagger}(\quotep{P},\quotep{Q}) & := & \forall R. P|Q \red^{*} R => R \red^{*} 0 \\
  match_{K}(\quotep{P},\quotep{Q}) & := & K \mbox{ for some context } K
\end{eqnarray*}

$u?(x)P | u!\langle Q \rangle \red P\{\quotep{Q}/x\}$

%We write $\wred$ for $\red^*$, and $P\red$ if $\exists Q $ such that $ P \red Q$.
We write $P\red$ if $\exists Q $ such that $ P \red Q$ and $P\not\red$, otherwise.

\section{Replication}

As mentioned before, it is known that replication (and hence
recursion) can be implemented in a higher-order process algebra
\cite{SangiorgiWalker}. As our first example of calculation with the
machinery thus far presented we give the construction explicitly in
the {\rhoc}.

\begin{eqnarray}
	D_{x} & := & \prefix{x}{y}{(\binpar{\outputp{x}{y}}{@{y}})} \nonumber\\
	\bangp_{x}{P} & := & \binpar{{x}!\langle{\binpar{D_{x}}{P}}\rangle}{D_{x}} \nonumber
\end{eqnarray}

\begin{eqnarray}
	\bangp_{x}{P} & & \nonumber\\
	=
	& {x}!\langle{(\prefix{x}{y}{(\outputp{x}{y} | @{y})) | P}}\rangle 
	      | \prefix{x}{y}{(\outputp{x}{y} | @{y})} & \nonumber\\
	\red
	& (\outputp{x}{y} | @{y})\substn{\quotep{(\prefix{x}{y}{(@{y} | \outputp{x}{y})) | P}}}{y} & \nonumber\\
	=
	& \outputp{x}{\quotep{(\prefix{x}{y}{(\outputp{x}{y} | @{y})) | P}}}
	  | {(\prefix{x}{y}{(\outputp{x}{y} | @{y})) | P}} & \nonumber\\
	\red
	& \ldots & \nonumber\\
	\red^*
	& P | P | \ldots & \nonumber
\end{eqnarray}

Of course, this encoding, as an implementation, runs away, unfolding
$\bangp{P}$ eagerly. A lazier and more implementable replication
operator, restricted to input-guarded processes, may be obtained as follows.

\begin{eqnarray}
\bangp{\prefix{u}{v}{P}} 
	:= 
	\binpar{\lift{x}{\prefix{u}{v}{(\binpar{D(x)}{P})}}}{D(x)} \nonumber
\end{eqnarray}

\begin{remark}
  Note that the lazier definition still does not deal with summation
  or mixed summation (i.e. sums over input and output). The reader is
  invited to construct definitions of replication that deal with these
  features. 

  Further, the definitions are parameterized in a name, $x$. Can you,
  gentle reader, make a definition that eliminates this parameter and
  guarantees no accidental interaction between the replication
  machinery and the process being replicated -- i.e. no accidental
  sharing of names used by the process to get its work done and the
  name(s) used by the replication to effect copying. This latter
  revision of the definition of replication is crucial to obtaining
  the expected identity $!!P \sim !P$.
\end{remark}

\begin{remark}\label{rem:paradoxical_combinator}
  The reader familiar with the lambda calculus will have noticed the
  similarity between $D$ and the paradoxical combinator.

  [Ed. note: the existence of this seems to suggest we have to be more
  restrictive on the set of processes and names we admit if we are to
  support no-cloning.]
\end{remark}

\subsubsection{Bisimulation}

The computational dynamics gives rise to another kind of equivalence,
the equivalence of computational behavior. As previously mentioned
this is typically captured \emph{via} some form of bisimulation.

% The notion we use in this paper is weak barbed bisimulation
% \cite{milner91polyadicpi}.

The notion we use in this paper is derived from weak barbed
bisimulation \cite{milner91polyadicpi}. 

\begin{definition}
An \emph{observation relation}, $\downarrow_{\mathcal N}$, over a set
of names, $\mathcal N$, is the smallest relation satisfying the rules
below.

\infrule[Out-barb]{y \in {\mathcal N}, \; x \nameeq y}
		  {\outputp{x}{v} \downarrow_{\mathcal N} x}
\infrule[Par-barb]{\mbox{$P\downarrow_{\mathcal N} x$ or $Q\downarrow_{\mathcal N} x$}}
		  {\binpar{P}{Q} \downarrow_{\mathcal N} x}

We write $P \Downarrow_{\mathcal N} x$ if there is $Q$ such that 
$P \wred Q$ and $Q \downarrow_{\mathcal N} x$.
\end{definition}

\begin{definition}
%\label{def.bbisim}
An  ${\mathcal N}$-\emph{barbed bisimulation} over a set of names, ${\mathcal N}$, is a symmetric binary relation 
${\mathcal S}_{\mathcal N}$ between agents such that $P\rel{S}_{\mathcal N}Q$ implies:
\begin{enumerate}
\item If $P \red P'$ then $Q \wred Q'$ and $P'\rel{S}_{\mathcal N} Q'$.
\item If $P\downarrow_{\mathcal N} x$, then $Q\Downarrow_{\mathcal N} x$.
\end{enumerate}
$P$ is ${\mathcal N}$-barbed bisimilar to $Q$, written
$P \wbbisim_{\mathcal N} Q$, if $P \rel{S}_{\mathcal N} Q$ for some ${\mathcal N}$-barbed bisimulation ${\mathcal S}_{\mathcal N}$.
\end{definition}

$\mathcal{R} \subseteq \pi \times \pi$

$P \mathcal{R} Q => \forall P'. P \red P' \Rightarrow \exists Q'. Q \red Q', P' \mathcal{R} Q'$

$P \vdash x \Rightarrow Q \vdash x$

\begin{mathpar}
  \inferrule*[lab=Out-barb]{x \nameeq y}{{y}!\langle{Q}\rangle \vdash x}
  \and
  \inferrule*[lab=Par-barb]{\mbox{$P\vdash x$ or $Q\vdash x$}}{\binpar{P}{Q} \vdash x}
\end{mathpar}

\subsubsection{Contexts}

One of the principle advantages of computational calculi like the
$\pi$-calculus is a well-defined notion of context,
contextual-equivalence and a correlation between
contextual-equivalence and notions of bisimulation. The notion of
context allows the decomposition of a process into (sub-)process and
its syntactic environment, its context. Thus, a context may be
thought of as a process with a ``hole'' (written $\Box$) in it. The
application of a context $M$ to a process $P$, written $M[P]$, is
tantamount to filling the hole in $M$ with $P$. In this paper we do
not need the full weight of this theory, but do make use of the notion
of context in the proof the main theorem. 

\begin{mathpar}
  \inferrule* [lab=summation] {} {{M_{M},M_{N}} \bc \Box \;|\; x.M_{A} \;|\; M_{M}+M_{N}}
  \and
  \inferrule* [lab=agent] {} {{M_{A}} \bc (\vec{x})M_{P} \;| \; \clift{P_0,\ldots,M_{P},\ldots,P_N}}
  \and \\
  \inferrule* [lab=process] {} {{M_{P}} \bc M_{N} \;| \;P|M_{P} }
\end{mathpar} 

\begin{mathpar}
  \inferrule* [lab=sychronization] {} {M_{N} \bc \Box \;|\; x?M_{F} \;|\; x!M_{C}}
  \and
  \inferrule* [lab=abstraction] {} {{M_{F}} \bc (x)M_{P} }
  \and
  \inferrule* [lab=concretion] {} {{M_{C}} \bc \langle M_{P} \rangle }
  \and \\
  \inferrule* [lab=process] {} {{M_{P}} \bc M_{N} \;| \;P|M_{P} }
\end{mathpar}

\begin{definition}[contextual application] Given a context $M$, and
  process $P$, we define the \emph{contextual application}, $M[P] :=
  M\{P/\Box\}$. That is, the contextual application of M to P is the
  substitution of $P$ for $\Box$ in $M$.
\end{definition}

$\meaningof{-} : L \to \mathcal{P}(\pi)$

\begin{mathpar}
  \inferrule* [lab=collection] {} {\meaningof{true} = \pi, \and \meaningof{~E} = \pi \setminus \meaningof{E}, \and \meaningof{E_{1} \& E_{2}} = \meaningof{E_{1}} \cap \meaningof{E_{2}}}
\end{mathpar}

\begin{mathpar}
  \inferrule* [lab=structure] {} {\meaningof{0} = \{ P \in \pi | P \equiv 0 \}, \and \\ \meaningof{E_1 | E_2} = \{ P \in \pi | P \equiv P_{1} | P_{2}, P_{1} \in \meaningof{E_{1}}, P_{2} \in \meaningof{E_2}\} }
\end{mathpar}

\begin{mathpar}
 \inferrule* [lab=behavior] {} {\meaningof{\langle a?b \rangle E} = \{ P \in \pi | P \equiv Q | u?(y)P', \\ \and \\\\ \and \\ \;\;\; u \in \meaningof{a}, \forall z.P'\{z/y\} \in \meaningof{E\{z/b\}}\}, \and \\ \meaningof{a!E} = \{ P \in \pi | P \equiv Q | x!\langle P' \rangle, x \in \meaningof{a} P' \in \meaningof{E}\} }
\end{mathpar}

\begin{mathpar}
 \inferrule* [lab=nominal] {} {\meaningof{\quotep{E}} = \{ \quotep{P} \in \quotep{\pi} | P \in \meaningof{E} \}, \and \meaningof{\quotep{P}} = \{ \quotep{Q} \in \quotep{\pi} | P \equiv Q \} \and \\ \meaningof{@\quotep{E}} = \{ P \in \pi | P \equiv @x, x \in \meaningof{E} \}}
\end{mathpar}

\begin{eqnarray*}
  \\
  \meaningof{-} : TS \to ST
\end{eqnarray*}

\begin{eqnarray*}
  \\
  L : TS \to ST
\end{eqnarray*}

\begin{eqnarray*}
  \\
  P \models E \iff P \in \meaningof{E}
\end{eqnarray*}

\begin{eqnarray*}
  P \approx_{L} Q \iff \forall E \in L. P \models E \iff Q \models E
\end{eqnarray*}

\begin{eqnarray*}
  P \approx_{K} Q
\end{eqnarray*}

\begin{eqnarray*}
  P \approx Q
\end{eqnarray*}

$\approx_{K} = \approx = \approx_{L}$

\subsubsection{Contextual duality}

Note that contexts extend the quotation operation to a family of
operations from processes to names. Given a context, $M$, we can
define a \emph{nominal context}, $\quotep{M}$ by $\quotep{M}[P] :=
\quotep{M[P]}$. To foreshadow what is to come we observe that these
operations enjoy a duality with processes very much like the duality
between vectors and maps from vectors to scalars.

Further, because the calculus is essentially higher-order, we have a
correspondence between contexts and processes. More specifically,
given a name $x$ and a context $M$ we can construct $M^{*}_{x}$ such
that 

\begin{mathpar}
  M^{*}_{x} | \lift{x}{P} \red M[P]
\end{mathpar}

namely,

\begin{mathpar}
  M^{*}_{x} := x?(u).M[\dropn{u}]
\end{mathpar}

The dependence of $M^{*}_{x}$ on a name makes it an abstraction, 

\begin{mathpar}
  M^{*} := (x)x?(u).M[\dropn{u}]
\end{mathpar}

\subsection{Additional notation}

It will sometimes be convenient to denote the process a name
quotes. We already have the notation $x = \quotep{P}$, but it will be
convenient to introduce an alternate notation, $\procn{x}$, when we
want to emphasize the connection to the use of the name. Note that, by
virtue of name equivalence, $\quotep{\procn{x}} \nameeq x$; so, the
notation is consistent with previous definitions.

Further, because names have structure it is possible to effect
substitutions on the basis of that structure. This means we need to
upgrade our notation for substitutions, which we accomplish by
adapting comprehension notation. Thus,

\begin{mathpar}
  P\{ y / x : x \in S \}
\end{mathpar}

is interpreted to mean the process derived from P by replacing (in a
capture-avoiding manner) each occurrence of $x$ in $S$ by $y$. For example,

\begin{mathpar}
  P\{ \quotep{\procn{x}|\procn{x}} / x : x \in \freenames{P} \}
\end{mathpar}

will replace each (occurrence) of a free name $x$ in $P$ by
$\quotep{\procn{x}|\procn{x}}$.

Also, we will avail ourselves of the notation $x^{L}$ and $x^{R}$ to
denote injections of a name into disjoint copies of the name
space. There are numerous ways to accomplish this. One example can be
found in \cite{MeredithR05}. This notation overloads to vectors of
names: $\vec{x}^{\pi} := (x_{i}^{\pi} \; : \; 0 \leq i < |\vec{x}| )$ where $\pi \in \{L,R\}$.

We also use $P^{\Box} := P|\Box$.

In \cite{MeredithR05} an interpretation of the new operator is
given. It turns out that there are several possible interpretations
all enjoying the requisite algebraic properties of the operator (see
\cite{milner91polyadicpi}). We will therefore make liberal use of
$(\nu\; \vec{x})P$.

% subsection the_syntax_and_semantics_of_the_notation_system (end)   

\input{qm2pi.qmops} 

\input{qm2pi.sterngerlach} 

\input{qm2pi.metric} 

% section concurrent_process_calculi (end)

%\input{qm2pi.proofsketch}

% section proof sketch (end)

%\input{qm2pi.slviaknots} 

% section spatial logic via knots (end)

\input{qm2pi.conclusion}

% section conclusion (end)

%\input{qm2pi.dtcodes} 

% section wiring algorithm (end)

\input{qm2pi.ack} 

% section acknowledgments (end)

\newpage


\bibliographystyle{plain}   
\bibliography{../../biblios/main.bib}

\input{qm2pi.rhodetails}

\end{document}



% section proof sketch (end)

%\section{Unlikely characters: spatial logic for
  knots}\label{sub:characteristic_formulae} % (fold)

Associated to the mobile process calculi are a family of logics known
as the Hennessy-Milner logics. These logics typically enjoy a
semantics interpreting formulae as sets of processes that when
factored through the encoding outlined above allows an identification
of classes of knots with logical formulae. In the context of this
encoding the sub-family known as the spatial logics \cite{CairesC03}
\cite{CairesC04} \cite{Caires04} are of particular interest providing
several important features for expressing and reasoning about
properties (i.e. classes) of knots. We hint here at how this may be done.

%\begin{description}
%\item [structural connectives] 
\subsubsection{Structural connectives} The spatial logics enjoy
structural connectives corresponding, at the logical level, to the
parallel composition ($P | Q$) and new name ($(\nu \; x)P$)
connectives for processes. As illustrated in the examples below, these
connectives are extremely expressive given the shape of our encoding.
%\item [decideable satisfaction]

\subsubsection{Decideable satisfaction}
In \cite{Caires04} the satisfaction relation is shown to be decideable
for a rich class of processes. It further turns out that the image of
the our encoding is a proper subset of that class. This result
provides the basis for an algorithm by which to search for knots
enjoying a given property.
%\item [characteristic formulae]

\subsubsection{Characteristic formulae}
In the same paper \cite{Caires04} , Caires presents a means of calculating
characteristic formulae, selecting equivalence classes of processes
up to a pre--specified depth limit on the support set of names. Composed with our
encoding, this characteristic formula can be used to select
characteristic formulae for knots.
%\end{description}

\subsubsection{Spatial logic formulae}

The grammar below (segmented for comprehension) summarizes the syntax
of spatial logic formulae. We employ illustrative examples in the
sequel to provide an intuitive understanding of their meaning
referring the reader to \cite{Caires04} for a more detailed explication
of the semantics.

\begin{mathpar}
  \inferrule* [lab=boolean] {} {{A,B} \bc T \;|\; \neg A \;|\; A \wedge B \;|\; \eta = \eta'}
  \and
  \inferrule* [lab=spatial] {} {|\; \pzero \;|\; A | B \;|\; x \text{\textregistered} A \;|\; \forall x . A \;|\;  H x . A}
  \and
  \inferrule* [lab=behavioral] {} {|\; \alpha . A}
  \and 
  \inferrule* [lab=recursion] {} {|\; X(\vec{u}) \;|\; \mu X(\vec{u}) . A}
  \and
  \inferrule* [lab=action] {} {\alpha \bc \langle x?(\vec{y}) \rangle \;|\; \langle x!(\vec{y}) \rangle \;|\; \langle \tau \rangle}
  \and 
  \inferrule* [lab=name] {} {\eta \bc x \;|\; \tau}
\end{mathpar} 

% subsection characteristic_formulae (end)   	 

\subsection{Example formulae}\label{sub:example_formulae_} % (fold)

\subsubsection{Crossing as formula.}
% 
% \begin{align*}
%   \frac{d}{dx} \sin x &= \cos x 
%   & \frac{d}{dx} e^x &= e^x \\
%   \frac{d}{dx} \cos x &= - \sin x 
%   & \frac{d}{dx} \log x &= \frac{1}{x} \\
% \end{align*} 

\begin{align*}
 \mu C(x_{0},x_{1},y_{0},y_{1},u).&(\langle x_{0}?(z) \rangle(\langle u! \rangle\langle y_{1}!z \rangle C(x_{0},x_{1},y_{0},y_{1},u)) & \\
  & \wedge \langle y_{1}?(z) \rangle (\langle u! \rangle \langle x_{0}!z \rangle C(x_{0},x_{1},y_{0},y_{1},u)) & \\
  & \wedge \langle x_{1}?(z) \rangle (\langle u? \rangle \langle y_{0}!z \rangle C(x_{0},x_{1},y_{0},y_{1},u)) & \\
  & \wedge \langle y_{0}?(z) \rangle (\langle u? \rangle \langle x_{1}!z \rangle C(x_{0},x_{1},y_{0},y_{1},u))) &
\end{align*}

The lexicographical similarity between the shape of this formulae and
the shape of definition of the process representing a crossing reveals
the intuitive meaning of this formulae. It describes the capabilities
of a process that has the right to represent a crossing. For example
it picks out processes that may perform an input on the port $x_0$ in
its initial menu of capabilities. What differentiates the formula
from the process, however, is that the crossing process is the
smallest candidate to satisfy the formula. Infinitely many other
processes -- with internal behavior hidden behind this interface, so
to speak -- also satisfy this formula. Even this simple formula,
then, can be seen to open a new view onto knots, providing a
computational interpretation of \emph{virtual} knots.

Note that this formula is derived by hand. A similar formula can be
derived by employing Caires' calculation of characteristic formula
\cite{Caires04} to the process representing a crossing. In light of
this discussion, we let
$\meaningof{C}_{\phi}(x0,x1,y0,y1,u)$ denote a formula specifying the
dynamics we wish to capture of a crossing. To guarantee we preserve
the shape of the interface and minimal semantics we demand that
$\meaningof{C}_{\phi}(x0,x1,y0,y1,u) \Rightarrow
\textbf{C}(x0,x1,y0,y1,u)$ where $\textbf{C}(x0,x1,y0,y1,u)$ denotes
the formula above.
                            
\subsubsection{Crossing number constraints.}
The moral content of the context lemma (Lemma \ref{context}) is that the notion of
``locality'' in the Reidemeister moves is effectively captured by the
parallel composition operator of the process calculus. This intuition
extends through the logic. Given a formula,
$\meaningof{C}_{\phi}(x0,x1,y0,y1,u)$, we can use the structural
connectives to specify constraints on crossing numbers, such as at
least $n$ crossings, or exactly $n$ crossings.
\begin{mathpar}
  \inferrule* [lab=at-least-n] {} { K^{\geq n}_{\phi}(\vec{xs},\vec{ys}) := \Pi_{i=0}^{n-1} Hu . \meaningof{C}_{\phi}(xs_i,ys_i,u) | T }
  \and 
  \inferrule* [lab=exactly-n] {} { K^{= n}_{\phi}(\vec{xs},\vec{ys}) := \Pi_{i=0}^{n-1} Hu . \meaningof{C}_{\phi}(xs_i,ys_i,u) | \neg (\forall x_0,y_0,x_1,y_1,u . \meaningof{C}_{\phi}(x_0,y_0,x_1,y_1,u) | T) }
\end{mathpar}

To round out this section, recall that the encoding of an $n$-crossing
knot decomposes into a parallel composition of $n$ \emph{copies} of a
crossing process together with a wiring harness. To specify different
knot classes with the same crossing number amounts to specifying
logical constraints on the wiring harness. In the interest of space,
we defer examples to a forthcoming paper. Suffice it to say that both
the conditions ``alternating knot'' and ``contains the tangle
corresponding to 5/3'' are expressible. For example, it is possible to
calculate the characteristic formula of a process corresponding to the
tangle 5/3 and conjoin it into the classifying formula via the
composition connective of the logic.

Finally, we wish to observe that it is entirely within reason to
contemplate a more domain-specific version of spatial logic tailored
to the shape of processes in the image of the encoding. Such a
domain-specific logic would have a better claim to the title formal
language of knot properties.

% subsection example_formulae_ (end)

% section knots_as_processes (end) 

% section spatial logic via knots (end)

\section{Conclusions and future work}

\paragraph{Testing physical space}
You, gentle reader, may wonder why of all the theorems to be proved
given this set up we pick the one above. In some sense it's hardly
central to quantum mechanics. We see it as central in the sense that
it firmly establishes a notion of physical space arising from a notion
of the equivalence of behavior. Relating bisimulation to a metric is a
big step forward, but one is faced with interpreting the relationship
of that metric space to something more physical. Quantum mechanical
notions of ``physical'' space are still far from intuitive, but by
relating this idea of distance as testing to calculations that predict
physical circumstances we are making a not insignificant step forward
toward an understanding of the physical space we inhabit as
essentially dynamic.

\paragraph{Effectivity and simulation}
One of the observations we have yet to make is that the entire program
spelled out here is effective. We have built various interpreters for
the reflective calculus at work in this interpretation. In principle,
then, we can simulate quantum mechanics on a computer. The place where
the simulation may lose fidelity is the infinitely branching summation
for the annihilator.

In this connection i also want to point out that the evaluation style
calculation of the inner product puts the non-determinism of the
summation right at the heart of measurement. This suggests that
Milner's original reduction-based formulation of the dynamics of his
calculi in terms of sums was not just notationally suggestive of a
notion of measure-and-continue but captured some significant part of
the physics.

\paragraph{Quantum continuations}
In light of this last observation i want to point out that the
predominant account of quantum mechanics is missing a key aspect of a
truly compositional story of the physical situation. In a real lab,
when a measurement is made the observation can be made to feed into
another device that then makes another measurement conditioned on the
results of the first. This means that after the superposition was
collapsed the entire experimental set up remained in
superposition. While QM offers a means of writing this down it doesn't
quite line up well with the well-trodden formulation of computation
and continuation that we see so succinctly expressed in Milner's
calculi. This suggests that there might be advantages to this account
of dynamics waiting to be explored.

\paragraph{Quantum logic}
In this connection, we also note that by virtue of having the
Hennessy-Milner construction, we can pull the construction through the
interpretation of QM. This gives us a natural candidate for a quantum
logic that enjoys an extremely tight connection with it's domain of
interpretation, making the construction much less ad hoc (rather it is
the image of functor!).

\paragraph{Quantum probabiity}
i have questions about the basis of the interpretation of inner
product as probability amplitude. In particular, using which
axiomatization of probability theory does the notion of probability
amplitude earn the right to be so dubbed? In other words, where is the
proof that the operation for calculating a probability amplitude (and
then squaring) satisfies the axioms of what it means to calculate a
probability? Even if such a proof exists (i have yet to find it in the
literature), i wonder if it might not be possible to turn things on
their heads. Can we view the calculation of the probability amplitude
as an axiomatization of probability? If so, then the definition we
give for calculating probability amplitude may provide the basis for
an \emph{effective} theory of probability.

\paragraph{Quantum vs ``biological'' information}
Finally, i want to conclude with a more philosophical observation. At
a recent workshop in which QM was a predominant topic i noticed
something about quantum information. The speaker was giving a riveting
discussion of axiomatic QM and showing how properties of ``no
cloning'' and ``no deleting'' emerged as consequences of the
axiomatization. Theorems of this form are necessary to give us a sense
of confidence that our axioms characterize the physical theory. What
struck me, though, was that if quantum information is neither erasable
nor replicable it is markedly different from \emph{life}. Two of the
things we know about life is that

\begin{itemize}
  \item it ends;
  \item to gain some measure of persistence, to transcend it's
    finitude it is imminently copyable.
\end{itemize}

Both of these qualities are summarized succinctly in the aphorism: all
flesh is grass. For me these two kinds of ``information'' -- call them
quantum and biological -- are end points on a spectrum of strategies
for persistence. At one end, we have those curious entities that enjoy
uniqueness and permanence; at the other, we have those who in the face
of a certain end and an uncertain present make a go of passing
something on. To me one of the more remarkable aspects of the latter
strategy is that in the presence of noise (and certain features of
copying) we get a kind of dynamism, a chance for improvement against a
given persistent condition.

% subsection other_calculi_other_bisimulations_and_geometry_as_behavior (end)




% section conclusion (end)

%\documentclass[12pt]{llncs}
%\documentclass{jktr}

\usepackage[pdftex]{hyperref}                   
\usepackage {listings}
\usepackage {mathpartir}
\usepackage{bcprules}
%\usepackage{listings}
                       
\usepackage{graphicx} 
%\usepackage[margins=2.5cm,nohead,nofoot]{geometry}
%\usepackage{geometry}
\usepackage{amsfonts}
\usepackage{amstext}
\usepackage{latexsym}
\usepackage{amssymb}
\usepackage{color}


%\include{myPreamble}
\include{qm2pi.local} 

%\ifpdf
%\usepackage[pdftex]{graphicx}
%\else
%\usepackage{graphicx}
%\fi

 % \ifpdf
%  \usepackage{pdfsync}
%  \if


%\title{Brief Article}
%\author{David F. Snyder}
%\author{L.G. Meredith}

%\address{Dept. of Math., Texas State University--San Marcos, San Marcos, TX 78666}
       
\pagestyle{empty}


\begin{document}

\lstset{language=[Objective]Caml,frame=shadowbox}

\input{qm2pi.front}

% section front matter (end)

\input{qm2pi.intro} 
 
% section introduction (end)

% \input{qm2pi.knotations} 

% section notation (end)

\input{qm2pi.process.calculi} 

% section concurrent_process_calculi_and_spatial_logics_ (end)
    
%\input{qm2pi.knots2pi} 

%\input{qm2pi.trefoil} 

%\input{qm2pi.mainthm} 

% subsection basic_interpretation (end)

%\input{qm2pi.rho.presentation} 
\subsection{The syntax and semantics of the notation system}\label{sub:the_syntax_and_semantics_of_the_notation_system} % (fold)

We now summarize a technical presentation of the calculus that
embodies our theory of dynamics. The typical presentation of such a
calculus follows the style of giving generators and relations on
them. The grammar, below, describing term constructors, freely
generates the set of processes, $\Proc$. This set is then quotiented
by a relation known as structural congruence and it is over this set
that the notion of dynamics is expressed. This presentation is
essentially that of \cite{MeredithR05} with the addition of
polyadicity and summation. For readability we have relegated some of
the technical subtleties to an appendix.

\subsubsection{Process grammar}\label{subsub:process_grammar}

\begin{mathpar}
  \inferrule* [lab=synchronization] {} {{M} \bc \pzero \;|\; x?F \;|\; x!C }
  \and
  \inferrule* [lab=abstraction] {} {{F} \bc (x)P}
  \and
  \inferrule* [lab=concretion] {} {{C} \bc \langle Q \rangle}
  \and
  \inferrule* [lab=process] {} {{P,Q} \bc M \;| \;P|Q \;|\; @{x}}
  \and
  \inferrule* [lab=name] {} {{x} \bc \quotep{P}}
\end{mathpar} 

Note that $\vec{x}$ (resp. $\vec{P}$) denotes a vector of names
(resp. processes) of length $|\vec{x}|$ (resp. $|\vec{P}|$). We adopt
the following useful abbreviations.

\begin{mathpar}
   x?(\vec{y}).P := x.(\vec{y})P \and  x\clift{\vec{P}} := x.\clift{\vec{P}}
   \and x!(y) := \lift{x}{\dropn{y}}
   \and \Pi_{i=0}^{n-1}P_i := P_0 | \ldots | P_{n-1}
\end{mathpar}

\subsubsection{Structural congruence}

\paragraph{Free and bound names and alpha-equivalence.} At the
core of structural equivalence is alpha-equivalence which identifies
process that are the same up to a change of variable. Formally, we
recognize the distinction between free and bound names. The free names
of a process, $\freenames{P}$, may be calculated recursively as
follows:

\begin{mathpar}
\freenames{\pzero} := \emptyset
  \and \\
  \freenames{x?(y).P} := \{ x \} \cup (\freenames{P} \setminus \{ y \})
  \and 
  \freenames{x!\langle P \rangle} := \{ x \} \cup \{ P \} 
  \and \\
  \freenames{P|Q} := \freenames{P} \cup \freenames{Q}
  \and \\
  \freenames{@{x}} := \{ x \}
\end{mathpar}

$\pi$
$\quotep{\pi}$

$\freenames{-} : \pi \to \mathcal{P}(\quotep{\pi})$

\begin{eqnarray*}
  \freenames{\pzero} & := & \emptyset \\
  \freenames{x?(y).P} & := & \{ x \} \cup (\freenames{P} \setminus \{ y \}) \\
  \freenames{x!\langle P \rangle} & := & \{ x \} \cup \{ P \} \\
  \freenames{P|Q} & := & \freenames{P} \cup \freenames{Q} \\
  \freenames{\dropn{x}} & := & \{ x \}
\end{eqnarray*}

The bound names of a process, $\boundnames{P}$, are those names occurring in $P$
that are not free. For example, in $x?(y).0$, the name $x$ is free, while $y$ is bound.

\begin{mathpar}
  \inferrule* [lab=monoidal-laws] {} { P|Q \equiv Q|P \and P|0 \equiv P \and P|(Q|R) \equiv (P|Q)|R }
\end{mathpar}

\begin{mathpar}
  \inferrule* [lab=alpha-equivalence] {} { (x)P \equiv (y)P\{y/x\} \and y \not\in \freenames{P} }
\end{mathpar}

\begin{definition}
Then two processes, $P,Q$, are alpha-equivalent if $P = Q\{\vec{y}/\vec{x}\}$ for
some $\vec{x} \in \boundnames{Q},\vec{y} \in \boundnames{P}$, where $Q\{\vec{y}/\vec{x}\}$
denotes the capture-avoiding substitution of $\vec{y}$ for $\vec{x}$ in $Q$.
\end{definition}

\begin{definition}
  The {\em structural congruence} \cite{SangiorgiWalker} , $\equiv$,
  between processes is the least congruence containing
  alpha-equivalence, satisfying the abelian monoid laws
  (associativity, commutativity and $\pzero$ as identity) for parallel
  composition $|$ and for summation $+$.
\end{definition}

\subsection{Name equivalence}

We take name equivalence, written $\nameeq$, to be the smallest
equivalence relation generated by the following rules.

\begin{mathpar}
\inferrule*[lab=Quote-drop]
{ }
{ \quotep{@{x}} \nameeq x }

\inferrule*[lab=Struct-equiv]
{ P \scong Q }
{ \quotep{P} \nameeq \quotep{Q} }
\end{mathpar}

The astute reader will have noticed that the mutual recursion of names
and processes imposes a mutual recursion on alpha-equivalence and
structural equivalence via name-equivalence. Fortunately, all of this
works out pleasantly and we may calculate in the natural way, free of
concern. The reader interested in the details is referred to the
appendix \ref{appendix:rho_details}.

\subsection{Substitution}

We use $\Proc$ for the set of processes, $\QProc$ for the set of
names, and $\id{\{}\vec{y} / \vec{x} \id{\}}$ to denote partial maps,
$s : \QProc \rightarrow \QProc$. A map, $s$ lifts, uniquely, to a map
on process terms, $\widehat{s} : \Proc \rightarrow \Proc$ by the
following equations.

\begin{mathpar}
  (0) \psubstp{Q}{P} := 0 \\
  (R \juxtap S) \psubstp{Q}{P}
  :=    
  (R)\psubstp{Q}{P} \juxtap (S) \psubstp{Q}{P} \\
  (x?(y).R) \psubstp{Q}{P}    
  :=    
  (x)\substp{Q}{P} (z)\concat( (R \psubstn{z}{y}) \psubstp{Q}{P} ) \\
  (\lift{x}{R}) \psubstp{Q}{P}  
  :=
  \lift{(x)\substp{Q}{P}}{ R \psubstp{Q}{P} } \\
%   (\dropn{x})  \psubstp{Q}{P}       
%   := 
%   \left\{ 
%     \begin{array}{ccc} 
%       \dropn{\quotep{Q}} & & x \nameeq \quotep{P} \\
%       \dropn{x} & & otherwise \\
%     \end{array}
%   \right. 
  (\dropn{x})  \psubstp{Q}{P}       
  := 
  \left\{ 
    \begin{array}{ccc} 
      Q & & x \nameeq \quotep{P} \\
      \dropn{x} & & otherwise \\
    \end{array}
  \right.
\end{mathpar}
 

where

\begin{eqnarray}
  (x)\id{\{} \lpquote Q \rpquote / \lpquote P \rpquote \id{\}}            = 
  \left\{ 
    \begin{array}{ccc}
      \lpquote Q \rpquote & & x \nameeq \lpquote P \rpquote \\
      x & & otherwise \\
    \end{array}
  \right. \nonumber
\end{eqnarray}

and $z$ is chosen distinct from $\quotep{P}$, $\quotep{Q}$, the free
names in $Q$, and all the names in $R$. Our $\alpha$-equivalence will
be built in the standard way from this substitution.

\begin{remark}\label{rem:no_self_referential_names}
  One consequence of these definitions is that $\forall P. \quotep{P}
  \not\in \freenames{P}$.
\end{remark}

\subsection{ Dynamic quote: an example }

Anticipating something of what's to come, consider applying the
substitution, $\widehat{\id{\{}u / z \id{\}}}$, to the following pair
of processes, $\lift{w}{y!(z)}$ and $w[ \lpquote y!(z) \rpquote ]$.

\begin{eqnarray}
	\lift{w}{y!(z)}\widehat{\id{\{}u / z \id{\}}}
		& = &
		\lift{w}{y!(u)} \nonumber\\
	w[ \lpquote y!(z) \rpquote ] \widehat{ \id{\{}u / z \id{\}} }
		& = &
		w[ \lpquote y!(z) \rpquote ] \nonumber
\end{eqnarray}

Because the body of the process between quotes is impervious to
substitution, we get radically different answers. In fact, by
examining the first process in an input context,
e.g. $x?(z).\lift{w}{y!(z)}$, we see that the process under the lift
operator may be shaped by prefixed inputs binding a name inside it. In
this sense, the lift operator will be seen as a way to dynamically
construct processes before reifying them as names.

Finally equipped with these standard features we can present the
dynamics of the calculus.

\subsubsection{Operational semantics} 

Finally, we introduce the computational dynamics. What marks these
algebras as distinct from other more traditionally studied algebraic
structures, e.g. vector spaces or polynomial rings, is the manner in
which dynamics is captured. In traditional structures, dynamics is typically
expressed through morphisms between such structures, as in linear maps
between vector spaces or morphisms between rings. In algebras
associated with the semantics of computation, the dynamics is
expressed as part of the algebraic structure itself, through a
reduction reduction relation typically denoted by $\red$. Below, we
give a recursive presentation of this relation for the calculus used
in the encoding.

$\red \subseteq \pi \times \pi$
$\red : \pi \to \mathcal{P}(\pi)$

\begin{mathpar}
  \inferrule* [lab=Comm] { \textsf{match}( x_{src}, x_{trgt} ) } { x_{trgt}?(y)P \; | \; x_{src}!\langle {Q} \rangle \red P\{\quotep{Q}/y}\} }
  \and \\
  \inferrule* [lab=Par] {{P} \red {P}'} {{{P} | {Q}} \red {{P}' | {Q}}}
  \and
  \inferrule* [lab=Equiv]{{{P} \scong {P}'} \andalso {{P}' \red {Q}'} \andalso {{Q}' \scong {Q}}}{{P} \red {Q}}
\end{mathpar}

\begin{eqnarray*}
  match_{\equiv} (\quotep{P},\quotep{Q}) & := & P \equiv Q \\
  match_{\dagger}(\quotep{P},\quotep{Q}) & := & \forall R. P|Q \red^{*} R => R \red^{*} 0 \\
  match_{K}(\quotep{P},\quotep{Q}) & := & K \mbox{ for some context } K
\end{eqnarray*}

$u?(x)P | u!\langle Q \rangle \red P\{\quotep{Q}/x\}$

%We write $\wred$ for $\red^*$, and $P\red$ if $\exists Q $ such that $ P \red Q$.
We write $P\red$ if $\exists Q $ such that $ P \red Q$ and $P\not\red$, otherwise.

\section{Replication}

As mentioned before, it is known that replication (and hence
recursion) can be implemented in a higher-order process algebra
\cite{SangiorgiWalker}. As our first example of calculation with the
machinery thus far presented we give the construction explicitly in
the {\rhoc}.

\begin{eqnarray}
	D_{x} & := & \prefix{x}{y}{(\binpar{\outputp{x}{y}}{@{y}})} \nonumber\\
	\bangp_{x}{P} & := & \binpar{{x}!\langle{\binpar{D_{x}}{P}}\rangle}{D_{x}} \nonumber
\end{eqnarray}

\begin{eqnarray}
	\bangp_{x}{P} & & \nonumber\\
	=
	& {x}!\langle{(\prefix{x}{y}{(\outputp{x}{y} | @{y})) | P}}\rangle 
	      | \prefix{x}{y}{(\outputp{x}{y} | @{y})} & \nonumber\\
	\red
	& (\outputp{x}{y} | @{y})\substn{\quotep{(\prefix{x}{y}{(@{y} | \outputp{x}{y})) | P}}}{y} & \nonumber\\
	=
	& \outputp{x}{\quotep{(\prefix{x}{y}{(\outputp{x}{y} | @{y})) | P}}}
	  | {(\prefix{x}{y}{(\outputp{x}{y} | @{y})) | P}} & \nonumber\\
	\red
	& \ldots & \nonumber\\
	\red^*
	& P | P | \ldots & \nonumber
\end{eqnarray}

Of course, this encoding, as an implementation, runs away, unfolding
$\bangp{P}$ eagerly. A lazier and more implementable replication
operator, restricted to input-guarded processes, may be obtained as follows.

\begin{eqnarray}
\bangp{\prefix{u}{v}{P}} 
	:= 
	\binpar{\lift{x}{\prefix{u}{v}{(\binpar{D(x)}{P})}}}{D(x)} \nonumber
\end{eqnarray}

\begin{remark}
  Note that the lazier definition still does not deal with summation
  or mixed summation (i.e. sums over input and output). The reader is
  invited to construct definitions of replication that deal with these
  features. 

  Further, the definitions are parameterized in a name, $x$. Can you,
  gentle reader, make a definition that eliminates this parameter and
  guarantees no accidental interaction between the replication
  machinery and the process being replicated -- i.e. no accidental
  sharing of names used by the process to get its work done and the
  name(s) used by the replication to effect copying. This latter
  revision of the definition of replication is crucial to obtaining
  the expected identity $!!P \sim !P$.
\end{remark}

\begin{remark}\label{rem:paradoxical_combinator}
  The reader familiar with the lambda calculus will have noticed the
  similarity between $D$ and the paradoxical combinator.

  [Ed. note: the existence of this seems to suggest we have to be more
  restrictive on the set of processes and names we admit if we are to
  support no-cloning.]
\end{remark}

\subsubsection{Bisimulation}

The computational dynamics gives rise to another kind of equivalence,
the equivalence of computational behavior. As previously mentioned
this is typically captured \emph{via} some form of bisimulation.

% The notion we use in this paper is weak barbed bisimulation
% \cite{milner91polyadicpi}.

The notion we use in this paper is derived from weak barbed
bisimulation \cite{milner91polyadicpi}. 

\begin{definition}
An \emph{observation relation}, $\downarrow_{\mathcal N}$, over a set
of names, $\mathcal N$, is the smallest relation satisfying the rules
below.

\infrule[Out-barb]{y \in {\mathcal N}, \; x \nameeq y}
		  {\outputp{x}{v} \downarrow_{\mathcal N} x}
\infrule[Par-barb]{\mbox{$P\downarrow_{\mathcal N} x$ or $Q\downarrow_{\mathcal N} x$}}
		  {\binpar{P}{Q} \downarrow_{\mathcal N} x}

We write $P \Downarrow_{\mathcal N} x$ if there is $Q$ such that 
$P \wred Q$ and $Q \downarrow_{\mathcal N} x$.
\end{definition}

\begin{definition}
%\label{def.bbisim}
An  ${\mathcal N}$-\emph{barbed bisimulation} over a set of names, ${\mathcal N}$, is a symmetric binary relation 
${\mathcal S}_{\mathcal N}$ between agents such that $P\rel{S}_{\mathcal N}Q$ implies:
\begin{enumerate}
\item If $P \red P'$ then $Q \wred Q'$ and $P'\rel{S}_{\mathcal N} Q'$.
\item If $P\downarrow_{\mathcal N} x$, then $Q\Downarrow_{\mathcal N} x$.
\end{enumerate}
$P$ is ${\mathcal N}$-barbed bisimilar to $Q$, written
$P \wbbisim_{\mathcal N} Q$, if $P \rel{S}_{\mathcal N} Q$ for some ${\mathcal N}$-barbed bisimulation ${\mathcal S}_{\mathcal N}$.
\end{definition}

$\mathcal{R} \subseteq \pi \times \pi$

$P \mathcal{R} Q => \forall P'. P \red P' \Rightarrow \exists Q'. Q \red Q', P' \mathcal{R} Q'$

$P \vdash x \Rightarrow Q \vdash x$

\begin{mathpar}
  \inferrule*[lab=Out-barb]{x \nameeq y}{{y}!\langle{Q}\rangle \vdash x}
  \and
  \inferrule*[lab=Par-barb]{\mbox{$P\vdash x$ or $Q\vdash x$}}{\binpar{P}{Q} \vdash x}
\end{mathpar}

\subsubsection{Contexts}

One of the principle advantages of computational calculi like the
$\pi$-calculus is a well-defined notion of context,
contextual-equivalence and a correlation between
contextual-equivalence and notions of bisimulation. The notion of
context allows the decomposition of a process into (sub-)process and
its syntactic environment, its context. Thus, a context may be
thought of as a process with a ``hole'' (written $\Box$) in it. The
application of a context $M$ to a process $P$, written $M[P]$, is
tantamount to filling the hole in $M$ with $P$. In this paper we do
not need the full weight of this theory, but do make use of the notion
of context in the proof the main theorem. 

\begin{mathpar}
  \inferrule* [lab=summation] {} {{M_{M},M_{N}} \bc \Box \;|\; x.M_{A} \;|\; M_{M}+M_{N}}
  \and
  \inferrule* [lab=agent] {} {{M_{A}} \bc (\vec{x})M_{P} \;| \; \clift{P_0,\ldots,M_{P},\ldots,P_N}}
  \and \\
  \inferrule* [lab=process] {} {{M_{P}} \bc M_{N} \;| \;P|M_{P} }
\end{mathpar} 

\begin{mathpar}
  \inferrule* [lab=sychronization] {} {M_{N} \bc \Box \;|\; x?M_{F} \;|\; x!M_{C}}
  \and
  \inferrule* [lab=abstraction] {} {{M_{F}} \bc (x)M_{P} }
  \and
  \inferrule* [lab=concretion] {} {{M_{C}} \bc \langle M_{P} \rangle }
  \and \\
  \inferrule* [lab=process] {} {{M_{P}} \bc M_{N} \;| \;P|M_{P} }
\end{mathpar}

\begin{definition}[contextual application] Given a context $M$, and
  process $P$, we define the \emph{contextual application}, $M[P] :=
  M\{P/\Box\}$. That is, the contextual application of M to P is the
  substitution of $P$ for $\Box$ in $M$.
\end{definition}

$\meaningof{-} : L \to \mathcal{P}(\pi)$

\begin{mathpar}
  \inferrule* [lab=collection] {} {\meaningof{true} = \pi, \and \meaningof{~E} = \pi \setminus \meaningof{E}, \and \meaningof{E_{1} \& E_{2}} = \meaningof{E_{1}} \cap \meaningof{E_{2}}}
\end{mathpar}

\begin{mathpar}
  \inferrule* [lab=structure] {} {\meaningof{0} = \{ P \in \pi | P \equiv 0 \}, \and \\ \meaningof{E_1 | E_2} = \{ P \in \pi | P \equiv P_{1} | P_{2}, P_{1} \in \meaningof{E_{1}}, P_{2} \in \meaningof{E_2}\} }
\end{mathpar}

\begin{mathpar}
 \inferrule* [lab=behavior] {} {\meaningof{\langle a?b \rangle E} = \{ P \in \pi | P \equiv Q | u?(y)P', \\ \and \\\\ \and \\ \;\;\; u \in \meaningof{a}, \forall z.P'\{z/y\} \in \meaningof{E\{z/b\}}\}, \and \\ \meaningof{a!E} = \{ P \in \pi | P \equiv Q | x!\langle P' \rangle, x \in \meaningof{a} P' \in \meaningof{E}\} }
\end{mathpar}

\begin{mathpar}
 \inferrule* [lab=nominal] {} {\meaningof{\quotep{E}} = \{ \quotep{P} \in \quotep{\pi} | P \in \meaningof{E} \}, \and \meaningof{\quotep{P}} = \{ \quotep{Q} \in \quotep{\pi} | P \equiv Q \} \and \\ \meaningof{@\quotep{E}} = \{ P \in \pi | P \equiv @x, x \in \meaningof{E} \}}
\end{mathpar}

\begin{eqnarray*}
  \\
  \meaningof{-} : TS \to ST
\end{eqnarray*}

\begin{eqnarray*}
  \\
  L : TS \to ST
\end{eqnarray*}

\begin{eqnarray*}
  \\
  P \models E \iff P \in \meaningof{E}
\end{eqnarray*}

\begin{eqnarray*}
  P \approx_{L} Q \iff \forall E \in L. P \models E \iff Q \models E
\end{eqnarray*}

\begin{eqnarray*}
  P \approx_{K} Q
\end{eqnarray*}

\begin{eqnarray*}
  P \approx Q
\end{eqnarray*}

$\approx_{K} = \approx = \approx_{L}$

\subsubsection{Contextual duality}

Note that contexts extend the quotation operation to a family of
operations from processes to names. Given a context, $M$, we can
define a \emph{nominal context}, $\quotep{M}$ by $\quotep{M}[P] :=
\quotep{M[P]}$. To foreshadow what is to come we observe that these
operations enjoy a duality with processes very much like the duality
between vectors and maps from vectors to scalars.

Further, because the calculus is essentially higher-order, we have a
correspondence between contexts and processes. More specifically,
given a name $x$ and a context $M$ we can construct $M^{*}_{x}$ such
that 

\begin{mathpar}
  M^{*}_{x} | \lift{x}{P} \red M[P]
\end{mathpar}

namely,

\begin{mathpar}
  M^{*}_{x} := x?(u).M[\dropn{u}]
\end{mathpar}

The dependence of $M^{*}_{x}$ on a name makes it an abstraction, 

\begin{mathpar}
  M^{*} := (x)x?(u).M[\dropn{u}]
\end{mathpar}

\subsection{Additional notation}

It will sometimes be convenient to denote the process a name
quotes. We already have the notation $x = \quotep{P}$, but it will be
convenient to introduce an alternate notation, $\procn{x}$, when we
want to emphasize the connection to the use of the name. Note that, by
virtue of name equivalence, $\quotep{\procn{x}} \nameeq x$; so, the
notation is consistent with previous definitions.

Further, because names have structure it is possible to effect
substitutions on the basis of that structure. This means we need to
upgrade our notation for substitutions, which we accomplish by
adapting comprehension notation. Thus,

\begin{mathpar}
  P\{ y / x : x \in S \}
\end{mathpar}

is interpreted to mean the process derived from P by replacing (in a
capture-avoiding manner) each occurrence of $x$ in $S$ by $y$. For example,

\begin{mathpar}
  P\{ \quotep{\procn{x}|\procn{x}} / x : x \in \freenames{P} \}
\end{mathpar}

will replace each (occurrence) of a free name $x$ in $P$ by
$\quotep{\procn{x}|\procn{x}}$.

Also, we will avail ourselves of the notation $x^{L}$ and $x^{R}$ to
denote injections of a name into disjoint copies of the name
space. There are numerous ways to accomplish this. One example can be
found in \cite{MeredithR05}. This notation overloads to vectors of
names: $\vec{x}^{\pi} := (x_{i}^{\pi} \; : \; 0 \leq i < |\vec{x}| )$ where $\pi \in \{L,R\}$.

We also use $P^{\Box} := P|\Box$.

In \cite{MeredithR05} an interpretation of the new operator is
given. It turns out that there are several possible interpretations
all enjoying the requisite algebraic properties of the operator (see
\cite{milner91polyadicpi}). We will therefore make liberal use of
$(\nu\; \vec{x})P$.

% subsection the_syntax_and_semantics_of_the_notation_system (end)   

\input{qm2pi.qmops} 

\input{qm2pi.sterngerlach} 

\input{qm2pi.metric} 

% section concurrent_process_calculi (end)

%\input{qm2pi.proofsketch}

% section proof sketch (end)

%\input{qm2pi.slviaknots} 

% section spatial logic via knots (end)

\input{qm2pi.conclusion}

% section conclusion (end)

%\input{qm2pi.dtcodes} 

% section wiring algorithm (end)

\input{qm2pi.ack} 

% section acknowledgments (end)

\newpage


\bibliographystyle{plain}   
\bibliography{../../biblios/main.bib}

\input{qm2pi.rhodetails}

\end{document}

 

% section wiring algorithm (end)

\documentclass[12pt]{llncs}
%\documentclass{jktr}

\usepackage[pdftex]{hyperref}                   
\usepackage {listings}
\usepackage {mathpartir}
\usepackage{bcprules}
%\usepackage{listings}
                       
\usepackage{graphicx} 
%\usepackage[margins=2.5cm,nohead,nofoot]{geometry}
%\usepackage{geometry}
\usepackage{amsfonts}
\usepackage{amstext}
\usepackage{latexsym}
\usepackage{amssymb}
\usepackage{color}


%\include{myPreamble}
\include{qm2pi.local} 

%\ifpdf
%\usepackage[pdftex]{graphicx}
%\else
%\usepackage{graphicx}
%\fi

 % \ifpdf
%  \usepackage{pdfsync}
%  \if


%\title{Brief Article}
%\author{David F. Snyder}
%\author{L.G. Meredith}

%\address{Dept. of Math., Texas State University--San Marcos, San Marcos, TX 78666}
       
\pagestyle{empty}


\begin{document}

\lstset{language=[Objective]Caml,frame=shadowbox}

\input{qm2pi.front}

% section front matter (end)

\input{qm2pi.intro} 
 
% section introduction (end)

% \input{qm2pi.knotations} 

% section notation (end)

\input{qm2pi.process.calculi} 

% section concurrent_process_calculi_and_spatial_logics_ (end)
    
%\input{qm2pi.knots2pi} 

%\input{qm2pi.trefoil} 

%\input{qm2pi.mainthm} 

% subsection basic_interpretation (end)

%\input{qm2pi.rho.presentation} 
\subsection{The syntax and semantics of the notation system}\label{sub:the_syntax_and_semantics_of_the_notation_system} % (fold)

We now summarize a technical presentation of the calculus that
embodies our theory of dynamics. The typical presentation of such a
calculus follows the style of giving generators and relations on
them. The grammar, below, describing term constructors, freely
generates the set of processes, $\Proc$. This set is then quotiented
by a relation known as structural congruence and it is over this set
that the notion of dynamics is expressed. This presentation is
essentially that of \cite{MeredithR05} with the addition of
polyadicity and summation. For readability we have relegated some of
the technical subtleties to an appendix.

\subsubsection{Process grammar}\label{subsub:process_grammar}

\begin{mathpar}
  \inferrule* [lab=synchronization] {} {{M} \bc \pzero \;|\; x?F \;|\; x!C }
  \and
  \inferrule* [lab=abstraction] {} {{F} \bc (x)P}
  \and
  \inferrule* [lab=concretion] {} {{C} \bc \langle Q \rangle}
  \and
  \inferrule* [lab=process] {} {{P,Q} \bc M \;| \;P|Q \;|\; @{x}}
  \and
  \inferrule* [lab=name] {} {{x} \bc \quotep{P}}
\end{mathpar} 

Note that $\vec{x}$ (resp. $\vec{P}$) denotes a vector of names
(resp. processes) of length $|\vec{x}|$ (resp. $|\vec{P}|$). We adopt
the following useful abbreviations.

\begin{mathpar}
   x?(\vec{y}).P := x.(\vec{y})P \and  x\clift{\vec{P}} := x.\clift{\vec{P}}
   \and x!(y) := \lift{x}{\dropn{y}}
   \and \Pi_{i=0}^{n-1}P_i := P_0 | \ldots | P_{n-1}
\end{mathpar}

\subsubsection{Structural congruence}

\paragraph{Free and bound names and alpha-equivalence.} At the
core of structural equivalence is alpha-equivalence which identifies
process that are the same up to a change of variable. Formally, we
recognize the distinction between free and bound names. The free names
of a process, $\freenames{P}$, may be calculated recursively as
follows:

\begin{mathpar}
\freenames{\pzero} := \emptyset
  \and \\
  \freenames{x?(y).P} := \{ x \} \cup (\freenames{P} \setminus \{ y \})
  \and 
  \freenames{x!\langle P \rangle} := \{ x \} \cup \{ P \} 
  \and \\
  \freenames{P|Q} := \freenames{P} \cup \freenames{Q}
  \and \\
  \freenames{@{x}} := \{ x \}
\end{mathpar}

$\pi$
$\quotep{\pi}$

$\freenames{-} : \pi \to \mathcal{P}(\quotep{\pi})$

\begin{eqnarray*}
  \freenames{\pzero} & := & \emptyset \\
  \freenames{x?(y).P} & := & \{ x \} \cup (\freenames{P} \setminus \{ y \}) \\
  \freenames{x!\langle P \rangle} & := & \{ x \} \cup \{ P \} \\
  \freenames{P|Q} & := & \freenames{P} \cup \freenames{Q} \\
  \freenames{\dropn{x}} & := & \{ x \}
\end{eqnarray*}

The bound names of a process, $\boundnames{P}$, are those names occurring in $P$
that are not free. For example, in $x?(y).0$, the name $x$ is free, while $y$ is bound.

\begin{mathpar}
  \inferrule* [lab=monoidal-laws] {} { P|Q \equiv Q|P \and P|0 \equiv P \and P|(Q|R) \equiv (P|Q)|R }
\end{mathpar}

\begin{mathpar}
  \inferrule* [lab=alpha-equivalence] {} { (x)P \equiv (y)P\{y/x\} \and y \not\in \freenames{P} }
\end{mathpar}

\begin{definition}
Then two processes, $P,Q$, are alpha-equivalent if $P = Q\{\vec{y}/\vec{x}\}$ for
some $\vec{x} \in \boundnames{Q},\vec{y} \in \boundnames{P}$, where $Q\{\vec{y}/\vec{x}\}$
denotes the capture-avoiding substitution of $\vec{y}$ for $\vec{x}$ in $Q$.
\end{definition}

\begin{definition}
  The {\em structural congruence} \cite{SangiorgiWalker} , $\equiv$,
  between processes is the least congruence containing
  alpha-equivalence, satisfying the abelian monoid laws
  (associativity, commutativity and $\pzero$ as identity) for parallel
  composition $|$ and for summation $+$.
\end{definition}

\subsection{Name equivalence}

We take name equivalence, written $\nameeq$, to be the smallest
equivalence relation generated by the following rules.

\begin{mathpar}
\inferrule*[lab=Quote-drop]
{ }
{ \quotep{@{x}} \nameeq x }

\inferrule*[lab=Struct-equiv]
{ P \scong Q }
{ \quotep{P} \nameeq \quotep{Q} }
\end{mathpar}

The astute reader will have noticed that the mutual recursion of names
and processes imposes a mutual recursion on alpha-equivalence and
structural equivalence via name-equivalence. Fortunately, all of this
works out pleasantly and we may calculate in the natural way, free of
concern. The reader interested in the details is referred to the
appendix \ref{appendix:rho_details}.

\subsection{Substitution}

We use $\Proc$ for the set of processes, $\QProc$ for the set of
names, and $\id{\{}\vec{y} / \vec{x} \id{\}}$ to denote partial maps,
$s : \QProc \rightarrow \QProc$. A map, $s$ lifts, uniquely, to a map
on process terms, $\widehat{s} : \Proc \rightarrow \Proc$ by the
following equations.

\begin{mathpar}
  (0) \psubstp{Q}{P} := 0 \\
  (R \juxtap S) \psubstp{Q}{P}
  :=    
  (R)\psubstp{Q}{P} \juxtap (S) \psubstp{Q}{P} \\
  (x?(y).R) \psubstp{Q}{P}    
  :=    
  (x)\substp{Q}{P} (z)\concat( (R \psubstn{z}{y}) \psubstp{Q}{P} ) \\
  (\lift{x}{R}) \psubstp{Q}{P}  
  :=
  \lift{(x)\substp{Q}{P}}{ R \psubstp{Q}{P} } \\
%   (\dropn{x})  \psubstp{Q}{P}       
%   := 
%   \left\{ 
%     \begin{array}{ccc} 
%       \dropn{\quotep{Q}} & & x \nameeq \quotep{P} \\
%       \dropn{x} & & otherwise \\
%     \end{array}
%   \right. 
  (\dropn{x})  \psubstp{Q}{P}       
  := 
  \left\{ 
    \begin{array}{ccc} 
      Q & & x \nameeq \quotep{P} \\
      \dropn{x} & & otherwise \\
    \end{array}
  \right.
\end{mathpar}
 

where

\begin{eqnarray}
  (x)\id{\{} \lpquote Q \rpquote / \lpquote P \rpquote \id{\}}            = 
  \left\{ 
    \begin{array}{ccc}
      \lpquote Q \rpquote & & x \nameeq \lpquote P \rpquote \\
      x & & otherwise \\
    \end{array}
  \right. \nonumber
\end{eqnarray}

and $z$ is chosen distinct from $\quotep{P}$, $\quotep{Q}$, the free
names in $Q$, and all the names in $R$. Our $\alpha$-equivalence will
be built in the standard way from this substitution.

\begin{remark}\label{rem:no_self_referential_names}
  One consequence of these definitions is that $\forall P. \quotep{P}
  \not\in \freenames{P}$.
\end{remark}

\subsection{ Dynamic quote: an example }

Anticipating something of what's to come, consider applying the
substitution, $\widehat{\id{\{}u / z \id{\}}}$, to the following pair
of processes, $\lift{w}{y!(z)}$ and $w[ \lpquote y!(z) \rpquote ]$.

\begin{eqnarray}
	\lift{w}{y!(z)}\widehat{\id{\{}u / z \id{\}}}
		& = &
		\lift{w}{y!(u)} \nonumber\\
	w[ \lpquote y!(z) \rpquote ] \widehat{ \id{\{}u / z \id{\}} }
		& = &
		w[ \lpquote y!(z) \rpquote ] \nonumber
\end{eqnarray}

Because the body of the process between quotes is impervious to
substitution, we get radically different answers. In fact, by
examining the first process in an input context,
e.g. $x?(z).\lift{w}{y!(z)}$, we see that the process under the lift
operator may be shaped by prefixed inputs binding a name inside it. In
this sense, the lift operator will be seen as a way to dynamically
construct processes before reifying them as names.

Finally equipped with these standard features we can present the
dynamics of the calculus.

\subsubsection{Operational semantics} 

Finally, we introduce the computational dynamics. What marks these
algebras as distinct from other more traditionally studied algebraic
structures, e.g. vector spaces or polynomial rings, is the manner in
which dynamics is captured. In traditional structures, dynamics is typically
expressed through morphisms between such structures, as in linear maps
between vector spaces or morphisms between rings. In algebras
associated with the semantics of computation, the dynamics is
expressed as part of the algebraic structure itself, through a
reduction reduction relation typically denoted by $\red$. Below, we
give a recursive presentation of this relation for the calculus used
in the encoding.

$\red \subseteq \pi \times \pi$
$\red : \pi \to \mathcal{P}(\pi)$

\begin{mathpar}
  \inferrule* [lab=Comm] { \textsf{match}( x_{src}, x_{trgt} ) } { x_{trgt}?(y)P \; | \; x_{src}!\langle {Q} \rangle \red P\{\quotep{Q}/y}\} }
  \and \\
  \inferrule* [lab=Par] {{P} \red {P}'} {{{P} | {Q}} \red {{P}' | {Q}}}
  \and
  \inferrule* [lab=Equiv]{{{P} \scong {P}'} \andalso {{P}' \red {Q}'} \andalso {{Q}' \scong {Q}}}{{P} \red {Q}}
\end{mathpar}

\begin{eqnarray*}
  match_{\equiv} (\quotep{P},\quotep{Q}) & := & P \equiv Q \\
  match_{\dagger}(\quotep{P},\quotep{Q}) & := & \forall R. P|Q \red^{*} R => R \red^{*} 0 \\
  match_{K}(\quotep{P},\quotep{Q}) & := & K \mbox{ for some context } K
\end{eqnarray*}

$u?(x)P | u!\langle Q \rangle \red P\{\quotep{Q}/x\}$

%We write $\wred$ for $\red^*$, and $P\red$ if $\exists Q $ such that $ P \red Q$.
We write $P\red$ if $\exists Q $ such that $ P \red Q$ and $P\not\red$, otherwise.

\section{Replication}

As mentioned before, it is known that replication (and hence
recursion) can be implemented in a higher-order process algebra
\cite{SangiorgiWalker}. As our first example of calculation with the
machinery thus far presented we give the construction explicitly in
the {\rhoc}.

\begin{eqnarray}
	D_{x} & := & \prefix{x}{y}{(\binpar{\outputp{x}{y}}{@{y}})} \nonumber\\
	\bangp_{x}{P} & := & \binpar{{x}!\langle{\binpar{D_{x}}{P}}\rangle}{D_{x}} \nonumber
\end{eqnarray}

\begin{eqnarray}
	\bangp_{x}{P} & & \nonumber\\
	=
	& {x}!\langle{(\prefix{x}{y}{(\outputp{x}{y} | @{y})) | P}}\rangle 
	      | \prefix{x}{y}{(\outputp{x}{y} | @{y})} & \nonumber\\
	\red
	& (\outputp{x}{y} | @{y})\substn{\quotep{(\prefix{x}{y}{(@{y} | \outputp{x}{y})) | P}}}{y} & \nonumber\\
	=
	& \outputp{x}{\quotep{(\prefix{x}{y}{(\outputp{x}{y} | @{y})) | P}}}
	  | {(\prefix{x}{y}{(\outputp{x}{y} | @{y})) | P}} & \nonumber\\
	\red
	& \ldots & \nonumber\\
	\red^*
	& P | P | \ldots & \nonumber
\end{eqnarray}

Of course, this encoding, as an implementation, runs away, unfolding
$\bangp{P}$ eagerly. A lazier and more implementable replication
operator, restricted to input-guarded processes, may be obtained as follows.

\begin{eqnarray}
\bangp{\prefix{u}{v}{P}} 
	:= 
	\binpar{\lift{x}{\prefix{u}{v}{(\binpar{D(x)}{P})}}}{D(x)} \nonumber
\end{eqnarray}

\begin{remark}
  Note that the lazier definition still does not deal with summation
  or mixed summation (i.e. sums over input and output). The reader is
  invited to construct definitions of replication that deal with these
  features. 

  Further, the definitions are parameterized in a name, $x$. Can you,
  gentle reader, make a definition that eliminates this parameter and
  guarantees no accidental interaction between the replication
  machinery and the process being replicated -- i.e. no accidental
  sharing of names used by the process to get its work done and the
  name(s) used by the replication to effect copying. This latter
  revision of the definition of replication is crucial to obtaining
  the expected identity $!!P \sim !P$.
\end{remark}

\begin{remark}\label{rem:paradoxical_combinator}
  The reader familiar with the lambda calculus will have noticed the
  similarity between $D$ and the paradoxical combinator.

  [Ed. note: the existence of this seems to suggest we have to be more
  restrictive on the set of processes and names we admit if we are to
  support no-cloning.]
\end{remark}

\subsubsection{Bisimulation}

The computational dynamics gives rise to another kind of equivalence,
the equivalence of computational behavior. As previously mentioned
this is typically captured \emph{via} some form of bisimulation.

% The notion we use in this paper is weak barbed bisimulation
% \cite{milner91polyadicpi}.

The notion we use in this paper is derived from weak barbed
bisimulation \cite{milner91polyadicpi}. 

\begin{definition}
An \emph{observation relation}, $\downarrow_{\mathcal N}$, over a set
of names, $\mathcal N$, is the smallest relation satisfying the rules
below.

\infrule[Out-barb]{y \in {\mathcal N}, \; x \nameeq y}
		  {\outputp{x}{v} \downarrow_{\mathcal N} x}
\infrule[Par-barb]{\mbox{$P\downarrow_{\mathcal N} x$ or $Q\downarrow_{\mathcal N} x$}}
		  {\binpar{P}{Q} \downarrow_{\mathcal N} x}

We write $P \Downarrow_{\mathcal N} x$ if there is $Q$ such that 
$P \wred Q$ and $Q \downarrow_{\mathcal N} x$.
\end{definition}

\begin{definition}
%\label{def.bbisim}
An  ${\mathcal N}$-\emph{barbed bisimulation} over a set of names, ${\mathcal N}$, is a symmetric binary relation 
${\mathcal S}_{\mathcal N}$ between agents such that $P\rel{S}_{\mathcal N}Q$ implies:
\begin{enumerate}
\item If $P \red P'$ then $Q \wred Q'$ and $P'\rel{S}_{\mathcal N} Q'$.
\item If $P\downarrow_{\mathcal N} x$, then $Q\Downarrow_{\mathcal N} x$.
\end{enumerate}
$P$ is ${\mathcal N}$-barbed bisimilar to $Q$, written
$P \wbbisim_{\mathcal N} Q$, if $P \rel{S}_{\mathcal N} Q$ for some ${\mathcal N}$-barbed bisimulation ${\mathcal S}_{\mathcal N}$.
\end{definition}

$\mathcal{R} \subseteq \pi \times \pi$

$P \mathcal{R} Q => \forall P'. P \red P' \Rightarrow \exists Q'. Q \red Q', P' \mathcal{R} Q'$

$P \vdash x \Rightarrow Q \vdash x$

\begin{mathpar}
  \inferrule*[lab=Out-barb]{x \nameeq y}{{y}!\langle{Q}\rangle \vdash x}
  \and
  \inferrule*[lab=Par-barb]{\mbox{$P\vdash x$ or $Q\vdash x$}}{\binpar{P}{Q} \vdash x}
\end{mathpar}

\subsubsection{Contexts}

One of the principle advantages of computational calculi like the
$\pi$-calculus is a well-defined notion of context,
contextual-equivalence and a correlation between
contextual-equivalence and notions of bisimulation. The notion of
context allows the decomposition of a process into (sub-)process and
its syntactic environment, its context. Thus, a context may be
thought of as a process with a ``hole'' (written $\Box$) in it. The
application of a context $M$ to a process $P$, written $M[P]$, is
tantamount to filling the hole in $M$ with $P$. In this paper we do
not need the full weight of this theory, but do make use of the notion
of context in the proof the main theorem. 

\begin{mathpar}
  \inferrule* [lab=summation] {} {{M_{M},M_{N}} \bc \Box \;|\; x.M_{A} \;|\; M_{M}+M_{N}}
  \and
  \inferrule* [lab=agent] {} {{M_{A}} \bc (\vec{x})M_{P} \;| \; \clift{P_0,\ldots,M_{P},\ldots,P_N}}
  \and \\
  \inferrule* [lab=process] {} {{M_{P}} \bc M_{N} \;| \;P|M_{P} }
\end{mathpar} 

\begin{mathpar}
  \inferrule* [lab=sychronization] {} {M_{N} \bc \Box \;|\; x?M_{F} \;|\; x!M_{C}}
  \and
  \inferrule* [lab=abstraction] {} {{M_{F}} \bc (x)M_{P} }
  \and
  \inferrule* [lab=concretion] {} {{M_{C}} \bc \langle M_{P} \rangle }
  \and \\
  \inferrule* [lab=process] {} {{M_{P}} \bc M_{N} \;| \;P|M_{P} }
\end{mathpar}

\begin{definition}[contextual application] Given a context $M$, and
  process $P$, we define the \emph{contextual application}, $M[P] :=
  M\{P/\Box\}$. That is, the contextual application of M to P is the
  substitution of $P$ for $\Box$ in $M$.
\end{definition}

$\meaningof{-} : L \to \mathcal{P}(\pi)$

\begin{mathpar}
  \inferrule* [lab=collection] {} {\meaningof{true} = \pi, \and \meaningof{~E} = \pi \setminus \meaningof{E}, \and \meaningof{E_{1} \& E_{2}} = \meaningof{E_{1}} \cap \meaningof{E_{2}}}
\end{mathpar}

\begin{mathpar}
  \inferrule* [lab=structure] {} {\meaningof{0} = \{ P \in \pi | P \equiv 0 \}, \and \\ \meaningof{E_1 | E_2} = \{ P \in \pi | P \equiv P_{1} | P_{2}, P_{1} \in \meaningof{E_{1}}, P_{2} \in \meaningof{E_2}\} }
\end{mathpar}

\begin{mathpar}
 \inferrule* [lab=behavior] {} {\meaningof{\langle a?b \rangle E} = \{ P \in \pi | P \equiv Q | u?(y)P', \\ \and \\\\ \and \\ \;\;\; u \in \meaningof{a}, \forall z.P'\{z/y\} \in \meaningof{E\{z/b\}}\}, \and \\ \meaningof{a!E} = \{ P \in \pi | P \equiv Q | x!\langle P' \rangle, x \in \meaningof{a} P' \in \meaningof{E}\} }
\end{mathpar}

\begin{mathpar}
 \inferrule* [lab=nominal] {} {\meaningof{\quotep{E}} = \{ \quotep{P} \in \quotep{\pi} | P \in \meaningof{E} \}, \and \meaningof{\quotep{P}} = \{ \quotep{Q} \in \quotep{\pi} | P \equiv Q \} \and \\ \meaningof{@\quotep{E}} = \{ P \in \pi | P \equiv @x, x \in \meaningof{E} \}}
\end{mathpar}

\begin{eqnarray*}
  \\
  \meaningof{-} : TS \to ST
\end{eqnarray*}

\begin{eqnarray*}
  \\
  L : TS \to ST
\end{eqnarray*}

\begin{eqnarray*}
  \\
  P \models E \iff P \in \meaningof{E}
\end{eqnarray*}

\begin{eqnarray*}
  P \approx_{L} Q \iff \forall E \in L. P \models E \iff Q \models E
\end{eqnarray*}

\begin{eqnarray*}
  P \approx_{K} Q
\end{eqnarray*}

\begin{eqnarray*}
  P \approx Q
\end{eqnarray*}

$\approx_{K} = \approx = \approx_{L}$

\subsubsection{Contextual duality}

Note that contexts extend the quotation operation to a family of
operations from processes to names. Given a context, $M$, we can
define a \emph{nominal context}, $\quotep{M}$ by $\quotep{M}[P] :=
\quotep{M[P]}$. To foreshadow what is to come we observe that these
operations enjoy a duality with processes very much like the duality
between vectors and maps from vectors to scalars.

Further, because the calculus is essentially higher-order, we have a
correspondence between contexts and processes. More specifically,
given a name $x$ and a context $M$ we can construct $M^{*}_{x}$ such
that 

\begin{mathpar}
  M^{*}_{x} | \lift{x}{P} \red M[P]
\end{mathpar}

namely,

\begin{mathpar}
  M^{*}_{x} := x?(u).M[\dropn{u}]
\end{mathpar}

The dependence of $M^{*}_{x}$ on a name makes it an abstraction, 

\begin{mathpar}
  M^{*} := (x)x?(u).M[\dropn{u}]
\end{mathpar}

\subsection{Additional notation}

It will sometimes be convenient to denote the process a name
quotes. We already have the notation $x = \quotep{P}$, but it will be
convenient to introduce an alternate notation, $\procn{x}$, when we
want to emphasize the connection to the use of the name. Note that, by
virtue of name equivalence, $\quotep{\procn{x}} \nameeq x$; so, the
notation is consistent with previous definitions.

Further, because names have structure it is possible to effect
substitutions on the basis of that structure. This means we need to
upgrade our notation for substitutions, which we accomplish by
adapting comprehension notation. Thus,

\begin{mathpar}
  P\{ y / x : x \in S \}
\end{mathpar}

is interpreted to mean the process derived from P by replacing (in a
capture-avoiding manner) each occurrence of $x$ in $S$ by $y$. For example,

\begin{mathpar}
  P\{ \quotep{\procn{x}|\procn{x}} / x : x \in \freenames{P} \}
\end{mathpar}

will replace each (occurrence) of a free name $x$ in $P$ by
$\quotep{\procn{x}|\procn{x}}$.

Also, we will avail ourselves of the notation $x^{L}$ and $x^{R}$ to
denote injections of a name into disjoint copies of the name
space. There are numerous ways to accomplish this. One example can be
found in \cite{MeredithR05}. This notation overloads to vectors of
names: $\vec{x}^{\pi} := (x_{i}^{\pi} \; : \; 0 \leq i < |\vec{x}| )$ where $\pi \in \{L,R\}$.

We also use $P^{\Box} := P|\Box$.

In \cite{MeredithR05} an interpretation of the new operator is
given. It turns out that there are several possible interpretations
all enjoying the requisite algebraic properties of the operator (see
\cite{milner91polyadicpi}). We will therefore make liberal use of
$(\nu\; \vec{x})P$.

% subsection the_syntax_and_semantics_of_the_notation_system (end)   

\input{qm2pi.qmops} 

\input{qm2pi.sterngerlach} 

\input{qm2pi.metric} 

% section concurrent_process_calculi (end)

%\input{qm2pi.proofsketch}

% section proof sketch (end)

%\input{qm2pi.slviaknots} 

% section spatial logic via knots (end)

\input{qm2pi.conclusion}

% section conclusion (end)

%\input{qm2pi.dtcodes} 

% section wiring algorithm (end)

\input{qm2pi.ack} 

% section acknowledgments (end)

\newpage


\bibliographystyle{plain}   
\bibliography{../../biblios/main.bib}

\input{qm2pi.rhodetails}

\end{document}

 

% section acknowledgments (end)

\newpage


\bibliographystyle{plain}   
\bibliography{../../biblios/main.bib}

\documentclass[12pt]{llncs}
%\documentclass{jktr}

\usepackage[pdftex]{hyperref}                   
\usepackage {listings}
\usepackage {mathpartir}
\usepackage{bcprules}
%\usepackage{listings}
                       
\usepackage{graphicx} 
%\usepackage[margins=2.5cm,nohead,nofoot]{geometry}
%\usepackage{geometry}
\usepackage{amsfonts}
\usepackage{amstext}
\usepackage{latexsym}
\usepackage{amssymb}
\usepackage{color}


%\include{myPreamble}
\include{qm2pi.local} 

%\ifpdf
%\usepackage[pdftex]{graphicx}
%\else
%\usepackage{graphicx}
%\fi

 % \ifpdf
%  \usepackage{pdfsync}
%  \if


%\title{Brief Article}
%\author{David F. Snyder}
%\author{L.G. Meredith}

%\address{Dept. of Math., Texas State University--San Marcos, San Marcos, TX 78666}
       
\pagestyle{empty}


\begin{document}

\lstset{language=[Objective]Caml,frame=shadowbox}

\input{qm2pi.front}

% section front matter (end)

\input{qm2pi.intro} 
 
% section introduction (end)

% \input{qm2pi.knotations} 

% section notation (end)

\input{qm2pi.process.calculi} 

% section concurrent_process_calculi_and_spatial_logics_ (end)
    
%\input{qm2pi.knots2pi} 

%\input{qm2pi.trefoil} 

%\input{qm2pi.mainthm} 

% subsection basic_interpretation (end)

%\input{qm2pi.rho.presentation} 
\subsection{The syntax and semantics of the notation system}\label{sub:the_syntax_and_semantics_of_the_notation_system} % (fold)

We now summarize a technical presentation of the calculus that
embodies our theory of dynamics. The typical presentation of such a
calculus follows the style of giving generators and relations on
them. The grammar, below, describing term constructors, freely
generates the set of processes, $\Proc$. This set is then quotiented
by a relation known as structural congruence and it is over this set
that the notion of dynamics is expressed. This presentation is
essentially that of \cite{MeredithR05} with the addition of
polyadicity and summation. For readability we have relegated some of
the technical subtleties to an appendix.

\subsubsection{Process grammar}\label{subsub:process_grammar}

\begin{mathpar}
  \inferrule* [lab=synchronization] {} {{M} \bc \pzero \;|\; x?F \;|\; x!C }
  \and
  \inferrule* [lab=abstraction] {} {{F} \bc (x)P}
  \and
  \inferrule* [lab=concretion] {} {{C} \bc \langle Q \rangle}
  \and
  \inferrule* [lab=process] {} {{P,Q} \bc M \;| \;P|Q \;|\; @{x}}
  \and
  \inferrule* [lab=name] {} {{x} \bc \quotep{P}}
\end{mathpar} 

Note that $\vec{x}$ (resp. $\vec{P}$) denotes a vector of names
(resp. processes) of length $|\vec{x}|$ (resp. $|\vec{P}|$). We adopt
the following useful abbreviations.

\begin{mathpar}
   x?(\vec{y}).P := x.(\vec{y})P \and  x\clift{\vec{P}} := x.\clift{\vec{P}}
   \and x!(y) := \lift{x}{\dropn{y}}
   \and \Pi_{i=0}^{n-1}P_i := P_0 | \ldots | P_{n-1}
\end{mathpar}

\subsubsection{Structural congruence}

\paragraph{Free and bound names and alpha-equivalence.} At the
core of structural equivalence is alpha-equivalence which identifies
process that are the same up to a change of variable. Formally, we
recognize the distinction between free and bound names. The free names
of a process, $\freenames{P}$, may be calculated recursively as
follows:

\begin{mathpar}
\freenames{\pzero} := \emptyset
  \and \\
  \freenames{x?(y).P} := \{ x \} \cup (\freenames{P} \setminus \{ y \})
  \and 
  \freenames{x!\langle P \rangle} := \{ x \} \cup \{ P \} 
  \and \\
  \freenames{P|Q} := \freenames{P} \cup \freenames{Q}
  \and \\
  \freenames{@{x}} := \{ x \}
\end{mathpar}

$\pi$
$\quotep{\pi}$

$\freenames{-} : \pi \to \mathcal{P}(\quotep{\pi})$

\begin{eqnarray*}
  \freenames{\pzero} & := & \emptyset \\
  \freenames{x?(y).P} & := & \{ x \} \cup (\freenames{P} \setminus \{ y \}) \\
  \freenames{x!\langle P \rangle} & := & \{ x \} \cup \{ P \} \\
  \freenames{P|Q} & := & \freenames{P} \cup \freenames{Q} \\
  \freenames{\dropn{x}} & := & \{ x \}
\end{eqnarray*}

The bound names of a process, $\boundnames{P}$, are those names occurring in $P$
that are not free. For example, in $x?(y).0$, the name $x$ is free, while $y$ is bound.

\begin{mathpar}
  \inferrule* [lab=monoidal-laws] {} { P|Q \equiv Q|P \and P|0 \equiv P \and P|(Q|R) \equiv (P|Q)|R }
\end{mathpar}

\begin{mathpar}
  \inferrule* [lab=alpha-equivalence] {} { (x)P \equiv (y)P\{y/x\} \and y \not\in \freenames{P} }
\end{mathpar}

\begin{definition}
Then two processes, $P,Q$, are alpha-equivalent if $P = Q\{\vec{y}/\vec{x}\}$ for
some $\vec{x} \in \boundnames{Q},\vec{y} \in \boundnames{P}$, where $Q\{\vec{y}/\vec{x}\}$
denotes the capture-avoiding substitution of $\vec{y}$ for $\vec{x}$ in $Q$.
\end{definition}

\begin{definition}
  The {\em structural congruence} \cite{SangiorgiWalker} , $\equiv$,
  between processes is the least congruence containing
  alpha-equivalence, satisfying the abelian monoid laws
  (associativity, commutativity and $\pzero$ as identity) for parallel
  composition $|$ and for summation $+$.
\end{definition}

\subsection{Name equivalence}

We take name equivalence, written $\nameeq$, to be the smallest
equivalence relation generated by the following rules.

\begin{mathpar}
\inferrule*[lab=Quote-drop]
{ }
{ \quotep{@{x}} \nameeq x }

\inferrule*[lab=Struct-equiv]
{ P \scong Q }
{ \quotep{P} \nameeq \quotep{Q} }
\end{mathpar}

The astute reader will have noticed that the mutual recursion of names
and processes imposes a mutual recursion on alpha-equivalence and
structural equivalence via name-equivalence. Fortunately, all of this
works out pleasantly and we may calculate in the natural way, free of
concern. The reader interested in the details is referred to the
appendix \ref{appendix:rho_details}.

\subsection{Substitution}

We use $\Proc$ for the set of processes, $\QProc$ for the set of
names, and $\id{\{}\vec{y} / \vec{x} \id{\}}$ to denote partial maps,
$s : \QProc \rightarrow \QProc$. A map, $s$ lifts, uniquely, to a map
on process terms, $\widehat{s} : \Proc \rightarrow \Proc$ by the
following equations.

\begin{mathpar}
  (0) \psubstp{Q}{P} := 0 \\
  (R \juxtap S) \psubstp{Q}{P}
  :=    
  (R)\psubstp{Q}{P} \juxtap (S) \psubstp{Q}{P} \\
  (x?(y).R) \psubstp{Q}{P}    
  :=    
  (x)\substp{Q}{P} (z)\concat( (R \psubstn{z}{y}) \psubstp{Q}{P} ) \\
  (\lift{x}{R}) \psubstp{Q}{P}  
  :=
  \lift{(x)\substp{Q}{P}}{ R \psubstp{Q}{P} } \\
%   (\dropn{x})  \psubstp{Q}{P}       
%   := 
%   \left\{ 
%     \begin{array}{ccc} 
%       \dropn{\quotep{Q}} & & x \nameeq \quotep{P} \\
%       \dropn{x} & & otherwise \\
%     \end{array}
%   \right. 
  (\dropn{x})  \psubstp{Q}{P}       
  := 
  \left\{ 
    \begin{array}{ccc} 
      Q & & x \nameeq \quotep{P} \\
      \dropn{x} & & otherwise \\
    \end{array}
  \right.
\end{mathpar}
 

where

\begin{eqnarray}
  (x)\id{\{} \lpquote Q \rpquote / \lpquote P \rpquote \id{\}}            = 
  \left\{ 
    \begin{array}{ccc}
      \lpquote Q \rpquote & & x \nameeq \lpquote P \rpquote \\
      x & & otherwise \\
    \end{array}
  \right. \nonumber
\end{eqnarray}

and $z$ is chosen distinct from $\quotep{P}$, $\quotep{Q}$, the free
names in $Q$, and all the names in $R$. Our $\alpha$-equivalence will
be built in the standard way from this substitution.

\begin{remark}\label{rem:no_self_referential_names}
  One consequence of these definitions is that $\forall P. \quotep{P}
  \not\in \freenames{P}$.
\end{remark}

\subsection{ Dynamic quote: an example }

Anticipating something of what's to come, consider applying the
substitution, $\widehat{\id{\{}u / z \id{\}}}$, to the following pair
of processes, $\lift{w}{y!(z)}$ and $w[ \lpquote y!(z) \rpquote ]$.

\begin{eqnarray}
	\lift{w}{y!(z)}\widehat{\id{\{}u / z \id{\}}}
		& = &
		\lift{w}{y!(u)} \nonumber\\
	w[ \lpquote y!(z) \rpquote ] \widehat{ \id{\{}u / z \id{\}} }
		& = &
		w[ \lpquote y!(z) \rpquote ] \nonumber
\end{eqnarray}

Because the body of the process between quotes is impervious to
substitution, we get radically different answers. In fact, by
examining the first process in an input context,
e.g. $x?(z).\lift{w}{y!(z)}$, we see that the process under the lift
operator may be shaped by prefixed inputs binding a name inside it. In
this sense, the lift operator will be seen as a way to dynamically
construct processes before reifying them as names.

Finally equipped with these standard features we can present the
dynamics of the calculus.

\subsubsection{Operational semantics} 

Finally, we introduce the computational dynamics. What marks these
algebras as distinct from other more traditionally studied algebraic
structures, e.g. vector spaces or polynomial rings, is the manner in
which dynamics is captured. In traditional structures, dynamics is typically
expressed through morphisms between such structures, as in linear maps
between vector spaces or morphisms between rings. In algebras
associated with the semantics of computation, the dynamics is
expressed as part of the algebraic structure itself, through a
reduction reduction relation typically denoted by $\red$. Below, we
give a recursive presentation of this relation for the calculus used
in the encoding.

$\red \subseteq \pi \times \pi$
$\red : \pi \to \mathcal{P}(\pi)$

\begin{mathpar}
  \inferrule* [lab=Comm] { \textsf{match}( x_{src}, x_{trgt} ) } { x_{trgt}?(y)P \; | \; x_{src}!\langle {Q} \rangle \red P\{\quotep{Q}/y}\} }
  \and \\
  \inferrule* [lab=Par] {{P} \red {P}'} {{{P} | {Q}} \red {{P}' | {Q}}}
  \and
  \inferrule* [lab=Equiv]{{{P} \scong {P}'} \andalso {{P}' \red {Q}'} \andalso {{Q}' \scong {Q}}}{{P} \red {Q}}
\end{mathpar}

\begin{eqnarray*}
  match_{\equiv} (\quotep{P},\quotep{Q}) & := & P \equiv Q \\
  match_{\dagger}(\quotep{P},\quotep{Q}) & := & \forall R. P|Q \red^{*} R => R \red^{*} 0 \\
  match_{K}(\quotep{P},\quotep{Q}) & := & K \mbox{ for some context } K
\end{eqnarray*}

$u?(x)P | u!\langle Q \rangle \red P\{\quotep{Q}/x\}$

%We write $\wred$ for $\red^*$, and $P\red$ if $\exists Q $ such that $ P \red Q$.
We write $P\red$ if $\exists Q $ such that $ P \red Q$ and $P\not\red$, otherwise.

\section{Replication}

As mentioned before, it is known that replication (and hence
recursion) can be implemented in a higher-order process algebra
\cite{SangiorgiWalker}. As our first example of calculation with the
machinery thus far presented we give the construction explicitly in
the {\rhoc}.

\begin{eqnarray}
	D_{x} & := & \prefix{x}{y}{(\binpar{\outputp{x}{y}}{@{y}})} \nonumber\\
	\bangp_{x}{P} & := & \binpar{{x}!\langle{\binpar{D_{x}}{P}}\rangle}{D_{x}} \nonumber
\end{eqnarray}

\begin{eqnarray}
	\bangp_{x}{P} & & \nonumber\\
	=
	& {x}!\langle{(\prefix{x}{y}{(\outputp{x}{y} | @{y})) | P}}\rangle 
	      | \prefix{x}{y}{(\outputp{x}{y} | @{y})} & \nonumber\\
	\red
	& (\outputp{x}{y} | @{y})\substn{\quotep{(\prefix{x}{y}{(@{y} | \outputp{x}{y})) | P}}}{y} & \nonumber\\
	=
	& \outputp{x}{\quotep{(\prefix{x}{y}{(\outputp{x}{y} | @{y})) | P}}}
	  | {(\prefix{x}{y}{(\outputp{x}{y} | @{y})) | P}} & \nonumber\\
	\red
	& \ldots & \nonumber\\
	\red^*
	& P | P | \ldots & \nonumber
\end{eqnarray}

Of course, this encoding, as an implementation, runs away, unfolding
$\bangp{P}$ eagerly. A lazier and more implementable replication
operator, restricted to input-guarded processes, may be obtained as follows.

\begin{eqnarray}
\bangp{\prefix{u}{v}{P}} 
	:= 
	\binpar{\lift{x}{\prefix{u}{v}{(\binpar{D(x)}{P})}}}{D(x)} \nonumber
\end{eqnarray}

\begin{remark}
  Note that the lazier definition still does not deal with summation
  or mixed summation (i.e. sums over input and output). The reader is
  invited to construct definitions of replication that deal with these
  features. 

  Further, the definitions are parameterized in a name, $x$. Can you,
  gentle reader, make a definition that eliminates this parameter and
  guarantees no accidental interaction between the replication
  machinery and the process being replicated -- i.e. no accidental
  sharing of names used by the process to get its work done and the
  name(s) used by the replication to effect copying. This latter
  revision of the definition of replication is crucial to obtaining
  the expected identity $!!P \sim !P$.
\end{remark}

\begin{remark}\label{rem:paradoxical_combinator}
  The reader familiar with the lambda calculus will have noticed the
  similarity between $D$ and the paradoxical combinator.

  [Ed. note: the existence of this seems to suggest we have to be more
  restrictive on the set of processes and names we admit if we are to
  support no-cloning.]
\end{remark}

\subsubsection{Bisimulation}

The computational dynamics gives rise to another kind of equivalence,
the equivalence of computational behavior. As previously mentioned
this is typically captured \emph{via} some form of bisimulation.

% The notion we use in this paper is weak barbed bisimulation
% \cite{milner91polyadicpi}.

The notion we use in this paper is derived from weak barbed
bisimulation \cite{milner91polyadicpi}. 

\begin{definition}
An \emph{observation relation}, $\downarrow_{\mathcal N}$, over a set
of names, $\mathcal N$, is the smallest relation satisfying the rules
below.

\infrule[Out-barb]{y \in {\mathcal N}, \; x \nameeq y}
		  {\outputp{x}{v} \downarrow_{\mathcal N} x}
\infrule[Par-barb]{\mbox{$P\downarrow_{\mathcal N} x$ or $Q\downarrow_{\mathcal N} x$}}
		  {\binpar{P}{Q} \downarrow_{\mathcal N} x}

We write $P \Downarrow_{\mathcal N} x$ if there is $Q$ such that 
$P \wred Q$ and $Q \downarrow_{\mathcal N} x$.
\end{definition}

\begin{definition}
%\label{def.bbisim}
An  ${\mathcal N}$-\emph{barbed bisimulation} over a set of names, ${\mathcal N}$, is a symmetric binary relation 
${\mathcal S}_{\mathcal N}$ between agents such that $P\rel{S}_{\mathcal N}Q$ implies:
\begin{enumerate}
\item If $P \red P'$ then $Q \wred Q'$ and $P'\rel{S}_{\mathcal N} Q'$.
\item If $P\downarrow_{\mathcal N} x$, then $Q\Downarrow_{\mathcal N} x$.
\end{enumerate}
$P$ is ${\mathcal N}$-barbed bisimilar to $Q$, written
$P \wbbisim_{\mathcal N} Q$, if $P \rel{S}_{\mathcal N} Q$ for some ${\mathcal N}$-barbed bisimulation ${\mathcal S}_{\mathcal N}$.
\end{definition}

$\mathcal{R} \subseteq \pi \times \pi$

$P \mathcal{R} Q => \forall P'. P \red P' \Rightarrow \exists Q'. Q \red Q', P' \mathcal{R} Q'$

$P \vdash x \Rightarrow Q \vdash x$

\begin{mathpar}
  \inferrule*[lab=Out-barb]{x \nameeq y}{{y}!\langle{Q}\rangle \vdash x}
  \and
  \inferrule*[lab=Par-barb]{\mbox{$P\vdash x$ or $Q\vdash x$}}{\binpar{P}{Q} \vdash x}
\end{mathpar}

\subsubsection{Contexts}

One of the principle advantages of computational calculi like the
$\pi$-calculus is a well-defined notion of context,
contextual-equivalence and a correlation between
contextual-equivalence and notions of bisimulation. The notion of
context allows the decomposition of a process into (sub-)process and
its syntactic environment, its context. Thus, a context may be
thought of as a process with a ``hole'' (written $\Box$) in it. The
application of a context $M$ to a process $P$, written $M[P]$, is
tantamount to filling the hole in $M$ with $P$. In this paper we do
not need the full weight of this theory, but do make use of the notion
of context in the proof the main theorem. 

\begin{mathpar}
  \inferrule* [lab=summation] {} {{M_{M},M_{N}} \bc \Box \;|\; x.M_{A} \;|\; M_{M}+M_{N}}
  \and
  \inferrule* [lab=agent] {} {{M_{A}} \bc (\vec{x})M_{P} \;| \; \clift{P_0,\ldots,M_{P},\ldots,P_N}}
  \and \\
  \inferrule* [lab=process] {} {{M_{P}} \bc M_{N} \;| \;P|M_{P} }
\end{mathpar} 

\begin{mathpar}
  \inferrule* [lab=sychronization] {} {M_{N} \bc \Box \;|\; x?M_{F} \;|\; x!M_{C}}
  \and
  \inferrule* [lab=abstraction] {} {{M_{F}} \bc (x)M_{P} }
  \and
  \inferrule* [lab=concretion] {} {{M_{C}} \bc \langle M_{P} \rangle }
  \and \\
  \inferrule* [lab=process] {} {{M_{P}} \bc M_{N} \;| \;P|M_{P} }
\end{mathpar}

\begin{definition}[contextual application] Given a context $M$, and
  process $P$, we define the \emph{contextual application}, $M[P] :=
  M\{P/\Box\}$. That is, the contextual application of M to P is the
  substitution of $P$ for $\Box$ in $M$.
\end{definition}

$\meaningof{-} : L \to \mathcal{P}(\pi)$

\begin{mathpar}
  \inferrule* [lab=collection] {} {\meaningof{true} = \pi, \and \meaningof{~E} = \pi \setminus \meaningof{E}, \and \meaningof{E_{1} \& E_{2}} = \meaningof{E_{1}} \cap \meaningof{E_{2}}}
\end{mathpar}

\begin{mathpar}
  \inferrule* [lab=structure] {} {\meaningof{0} = \{ P \in \pi | P \equiv 0 \}, \and \\ \meaningof{E_1 | E_2} = \{ P \in \pi | P \equiv P_{1} | P_{2}, P_{1} \in \meaningof{E_{1}}, P_{2} \in \meaningof{E_2}\} }
\end{mathpar}

\begin{mathpar}
 \inferrule* [lab=behavior] {} {\meaningof{\langle a?b \rangle E} = \{ P \in \pi | P \equiv Q | u?(y)P', \\ \and \\\\ \and \\ \;\;\; u \in \meaningof{a}, \forall z.P'\{z/y\} \in \meaningof{E\{z/b\}}\}, \and \\ \meaningof{a!E} = \{ P \in \pi | P \equiv Q | x!\langle P' \rangle, x \in \meaningof{a} P' \in \meaningof{E}\} }
\end{mathpar}

\begin{mathpar}
 \inferrule* [lab=nominal] {} {\meaningof{\quotep{E}} = \{ \quotep{P} \in \quotep{\pi} | P \in \meaningof{E} \}, \and \meaningof{\quotep{P}} = \{ \quotep{Q} \in \quotep{\pi} | P \equiv Q \} \and \\ \meaningof{@\quotep{E}} = \{ P \in \pi | P \equiv @x, x \in \meaningof{E} \}}
\end{mathpar}

\begin{eqnarray*}
  \\
  \meaningof{-} : TS \to ST
\end{eqnarray*}

\begin{eqnarray*}
  \\
  L : TS \to ST
\end{eqnarray*}

\begin{eqnarray*}
  \\
  P \models E \iff P \in \meaningof{E}
\end{eqnarray*}

\begin{eqnarray*}
  P \approx_{L} Q \iff \forall E \in L. P \models E \iff Q \models E
\end{eqnarray*}

\begin{eqnarray*}
  P \approx_{K} Q
\end{eqnarray*}

\begin{eqnarray*}
  P \approx Q
\end{eqnarray*}

$\approx_{K} = \approx = \approx_{L}$

\subsubsection{Contextual duality}

Note that contexts extend the quotation operation to a family of
operations from processes to names. Given a context, $M$, we can
define a \emph{nominal context}, $\quotep{M}$ by $\quotep{M}[P] :=
\quotep{M[P]}$. To foreshadow what is to come we observe that these
operations enjoy a duality with processes very much like the duality
between vectors and maps from vectors to scalars.

Further, because the calculus is essentially higher-order, we have a
correspondence between contexts and processes. More specifically,
given a name $x$ and a context $M$ we can construct $M^{*}_{x}$ such
that 

\begin{mathpar}
  M^{*}_{x} | \lift{x}{P} \red M[P]
\end{mathpar}

namely,

\begin{mathpar}
  M^{*}_{x} := x?(u).M[\dropn{u}]
\end{mathpar}

The dependence of $M^{*}_{x}$ on a name makes it an abstraction, 

\begin{mathpar}
  M^{*} := (x)x?(u).M[\dropn{u}]
\end{mathpar}

\subsection{Additional notation}

It will sometimes be convenient to denote the process a name
quotes. We already have the notation $x = \quotep{P}$, but it will be
convenient to introduce an alternate notation, $\procn{x}$, when we
want to emphasize the connection to the use of the name. Note that, by
virtue of name equivalence, $\quotep{\procn{x}} \nameeq x$; so, the
notation is consistent with previous definitions.

Further, because names have structure it is possible to effect
substitutions on the basis of that structure. This means we need to
upgrade our notation for substitutions, which we accomplish by
adapting comprehension notation. Thus,

\begin{mathpar}
  P\{ y / x : x \in S \}
\end{mathpar}

is interpreted to mean the process derived from P by replacing (in a
capture-avoiding manner) each occurrence of $x$ in $S$ by $y$. For example,

\begin{mathpar}
  P\{ \quotep{\procn{x}|\procn{x}} / x : x \in \freenames{P} \}
\end{mathpar}

will replace each (occurrence) of a free name $x$ in $P$ by
$\quotep{\procn{x}|\procn{x}}$.

Also, we will avail ourselves of the notation $x^{L}$ and $x^{R}$ to
denote injections of a name into disjoint copies of the name
space. There are numerous ways to accomplish this. One example can be
found in \cite{MeredithR05}. This notation overloads to vectors of
names: $\vec{x}^{\pi} := (x_{i}^{\pi} \; : \; 0 \leq i < |\vec{x}| )$ where $\pi \in \{L,R\}$.

We also use $P^{\Box} := P|\Box$.

In \cite{MeredithR05} an interpretation of the new operator is
given. It turns out that there are several possible interpretations
all enjoying the requisite algebraic properties of the operator (see
\cite{milner91polyadicpi}). We will therefore make liberal use of
$(\nu\; \vec{x})P$.

% subsection the_syntax_and_semantics_of_the_notation_system (end)   

\input{qm2pi.qmops} 

\input{qm2pi.sterngerlach} 

\input{qm2pi.metric} 

% section concurrent_process_calculi (end)

%\input{qm2pi.proofsketch}

% section proof sketch (end)

%\input{qm2pi.slviaknots} 

% section spatial logic via knots (end)

\input{qm2pi.conclusion}

% section conclusion (end)

%\input{qm2pi.dtcodes} 

% section wiring algorithm (end)

\input{qm2pi.ack} 

% section acknowledgments (end)

\newpage


\bibliographystyle{plain}   
\bibliography{../../biblios/main.bib}

\input{qm2pi.rhodetails}

\end{document}



\end{document}

 

% section wiring algorithm (end)

\documentclass[12pt]{llncs}
%\documentclass{jktr}

\usepackage[pdftex]{hyperref}                   
\usepackage {listings}
\usepackage {mathpartir}
\usepackage{bcprules}
%\usepackage{listings}
                       
\usepackage{graphicx} 
%\usepackage[margins=2.5cm,nohead,nofoot]{geometry}
%\usepackage{geometry}
\usepackage{amsfonts}
\usepackage{amstext}
\usepackage{latexsym}
\usepackage{amssymb}
\usepackage{color}


%\include{myPreamble}
\documentclass[12pt]{llncs}
%\documentclass{jktr}

\usepackage[pdftex]{hyperref}                   
\usepackage {listings}
\usepackage {mathpartir}
\usepackage{bcprules}
%\usepackage{listings}
                       
\usepackage{graphicx} 
%\usepackage[margins=2.5cm,nohead,nofoot]{geometry}
%\usepackage{geometry}
\usepackage{amsfonts}
\usepackage{amstext}
\usepackage{latexsym}
\usepackage{amssymb}
\usepackage{color}


%\include{myPreamble}
\include{qm2pi.local} 

%\ifpdf
%\usepackage[pdftex]{graphicx}
%\else
%\usepackage{graphicx}
%\fi

 % \ifpdf
%  \usepackage{pdfsync}
%  \if


%\title{Brief Article}
%\author{David F. Snyder}
%\author{L.G. Meredith}

%\address{Dept. of Math., Texas State University--San Marcos, San Marcos, TX 78666}
       
\pagestyle{empty}


\begin{document}

\lstset{language=[Objective]Caml,frame=shadowbox}

\input{qm2pi.front}

% section front matter (end)

\input{qm2pi.intro} 
 
% section introduction (end)

% \input{qm2pi.knotations} 

% section notation (end)

\input{qm2pi.process.calculi} 

% section concurrent_process_calculi_and_spatial_logics_ (end)
    
%\input{qm2pi.knots2pi} 

%\input{qm2pi.trefoil} 

%\input{qm2pi.mainthm} 

% subsection basic_interpretation (end)

%\input{qm2pi.rho.presentation} 
\subsection{The syntax and semantics of the notation system}\label{sub:the_syntax_and_semantics_of_the_notation_system} % (fold)

We now summarize a technical presentation of the calculus that
embodies our theory of dynamics. The typical presentation of such a
calculus follows the style of giving generators and relations on
them. The grammar, below, describing term constructors, freely
generates the set of processes, $\Proc$. This set is then quotiented
by a relation known as structural congruence and it is over this set
that the notion of dynamics is expressed. This presentation is
essentially that of \cite{MeredithR05} with the addition of
polyadicity and summation. For readability we have relegated some of
the technical subtleties to an appendix.

\subsubsection{Process grammar}\label{subsub:process_grammar}

\begin{mathpar}
  \inferrule* [lab=synchronization] {} {{M} \bc \pzero \;|\; x?F \;|\; x!C }
  \and
  \inferrule* [lab=abstraction] {} {{F} \bc (x)P}
  \and
  \inferrule* [lab=concretion] {} {{C} \bc \langle Q \rangle}
  \and
  \inferrule* [lab=process] {} {{P,Q} \bc M \;| \;P|Q \;|\; @{x}}
  \and
  \inferrule* [lab=name] {} {{x} \bc \quotep{P}}
\end{mathpar} 

Note that $\vec{x}$ (resp. $\vec{P}$) denotes a vector of names
(resp. processes) of length $|\vec{x}|$ (resp. $|\vec{P}|$). We adopt
the following useful abbreviations.

\begin{mathpar}
   x?(\vec{y}).P := x.(\vec{y})P \and  x\clift{\vec{P}} := x.\clift{\vec{P}}
   \and x!(y) := \lift{x}{\dropn{y}}
   \and \Pi_{i=0}^{n-1}P_i := P_0 | \ldots | P_{n-1}
\end{mathpar}

\subsubsection{Structural congruence}

\paragraph{Free and bound names and alpha-equivalence.} At the
core of structural equivalence is alpha-equivalence which identifies
process that are the same up to a change of variable. Formally, we
recognize the distinction between free and bound names. The free names
of a process, $\freenames{P}$, may be calculated recursively as
follows:

\begin{mathpar}
\freenames{\pzero} := \emptyset
  \and \\
  \freenames{x?(y).P} := \{ x \} \cup (\freenames{P} \setminus \{ y \})
  \and 
  \freenames{x!\langle P \rangle} := \{ x \} \cup \{ P \} 
  \and \\
  \freenames{P|Q} := \freenames{P} \cup \freenames{Q}
  \and \\
  \freenames{@{x}} := \{ x \}
\end{mathpar}

$\pi$
$\quotep{\pi}$

$\freenames{-} : \pi \to \mathcal{P}(\quotep{\pi})$

\begin{eqnarray*}
  \freenames{\pzero} & := & \emptyset \\
  \freenames{x?(y).P} & := & \{ x \} \cup (\freenames{P} \setminus \{ y \}) \\
  \freenames{x!\langle P \rangle} & := & \{ x \} \cup \{ P \} \\
  \freenames{P|Q} & := & \freenames{P} \cup \freenames{Q} \\
  \freenames{\dropn{x}} & := & \{ x \}
\end{eqnarray*}

The bound names of a process, $\boundnames{P}$, are those names occurring in $P$
that are not free. For example, in $x?(y).0$, the name $x$ is free, while $y$ is bound.

\begin{mathpar}
  \inferrule* [lab=monoidal-laws] {} { P|Q \equiv Q|P \and P|0 \equiv P \and P|(Q|R) \equiv (P|Q)|R }
\end{mathpar}

\begin{mathpar}
  \inferrule* [lab=alpha-equivalence] {} { (x)P \equiv (y)P\{y/x\} \and y \not\in \freenames{P} }
\end{mathpar}

\begin{definition}
Then two processes, $P,Q$, are alpha-equivalent if $P = Q\{\vec{y}/\vec{x}\}$ for
some $\vec{x} \in \boundnames{Q},\vec{y} \in \boundnames{P}$, where $Q\{\vec{y}/\vec{x}\}$
denotes the capture-avoiding substitution of $\vec{y}$ for $\vec{x}$ in $Q$.
\end{definition}

\begin{definition}
  The {\em structural congruence} \cite{SangiorgiWalker} , $\equiv$,
  between processes is the least congruence containing
  alpha-equivalence, satisfying the abelian monoid laws
  (associativity, commutativity and $\pzero$ as identity) for parallel
  composition $|$ and for summation $+$.
\end{definition}

\subsection{Name equivalence}

We take name equivalence, written $\nameeq$, to be the smallest
equivalence relation generated by the following rules.

\begin{mathpar}
\inferrule*[lab=Quote-drop]
{ }
{ \quotep{@{x}} \nameeq x }

\inferrule*[lab=Struct-equiv]
{ P \scong Q }
{ \quotep{P} \nameeq \quotep{Q} }
\end{mathpar}

The astute reader will have noticed that the mutual recursion of names
and processes imposes a mutual recursion on alpha-equivalence and
structural equivalence via name-equivalence. Fortunately, all of this
works out pleasantly and we may calculate in the natural way, free of
concern. The reader interested in the details is referred to the
appendix \ref{appendix:rho_details}.

\subsection{Substitution}

We use $\Proc$ for the set of processes, $\QProc$ for the set of
names, and $\id{\{}\vec{y} / \vec{x} \id{\}}$ to denote partial maps,
$s : \QProc \rightarrow \QProc$. A map, $s$ lifts, uniquely, to a map
on process terms, $\widehat{s} : \Proc \rightarrow \Proc$ by the
following equations.

\begin{mathpar}
  (0) \psubstp{Q}{P} := 0 \\
  (R \juxtap S) \psubstp{Q}{P}
  :=    
  (R)\psubstp{Q}{P} \juxtap (S) \psubstp{Q}{P} \\
  (x?(y).R) \psubstp{Q}{P}    
  :=    
  (x)\substp{Q}{P} (z)\concat( (R \psubstn{z}{y}) \psubstp{Q}{P} ) \\
  (\lift{x}{R}) \psubstp{Q}{P}  
  :=
  \lift{(x)\substp{Q}{P}}{ R \psubstp{Q}{P} } \\
%   (\dropn{x})  \psubstp{Q}{P}       
%   := 
%   \left\{ 
%     \begin{array}{ccc} 
%       \dropn{\quotep{Q}} & & x \nameeq \quotep{P} \\
%       \dropn{x} & & otherwise \\
%     \end{array}
%   \right. 
  (\dropn{x})  \psubstp{Q}{P}       
  := 
  \left\{ 
    \begin{array}{ccc} 
      Q & & x \nameeq \quotep{P} \\
      \dropn{x} & & otherwise \\
    \end{array}
  \right.
\end{mathpar}
 

where

\begin{eqnarray}
  (x)\id{\{} \lpquote Q \rpquote / \lpquote P \rpquote \id{\}}            = 
  \left\{ 
    \begin{array}{ccc}
      \lpquote Q \rpquote & & x \nameeq \lpquote P \rpquote \\
      x & & otherwise \\
    \end{array}
  \right. \nonumber
\end{eqnarray}

and $z$ is chosen distinct from $\quotep{P}$, $\quotep{Q}$, the free
names in $Q$, and all the names in $R$. Our $\alpha$-equivalence will
be built in the standard way from this substitution.

\begin{remark}\label{rem:no_self_referential_names}
  One consequence of these definitions is that $\forall P. \quotep{P}
  \not\in \freenames{P}$.
\end{remark}

\subsection{ Dynamic quote: an example }

Anticipating something of what's to come, consider applying the
substitution, $\widehat{\id{\{}u / z \id{\}}}$, to the following pair
of processes, $\lift{w}{y!(z)}$ and $w[ \lpquote y!(z) \rpquote ]$.

\begin{eqnarray}
	\lift{w}{y!(z)}\widehat{\id{\{}u / z \id{\}}}
		& = &
		\lift{w}{y!(u)} \nonumber\\
	w[ \lpquote y!(z) \rpquote ] \widehat{ \id{\{}u / z \id{\}} }
		& = &
		w[ \lpquote y!(z) \rpquote ] \nonumber
\end{eqnarray}

Because the body of the process between quotes is impervious to
substitution, we get radically different answers. In fact, by
examining the first process in an input context,
e.g. $x?(z).\lift{w}{y!(z)}$, we see that the process under the lift
operator may be shaped by prefixed inputs binding a name inside it. In
this sense, the lift operator will be seen as a way to dynamically
construct processes before reifying them as names.

Finally equipped with these standard features we can present the
dynamics of the calculus.

\subsubsection{Operational semantics} 

Finally, we introduce the computational dynamics. What marks these
algebras as distinct from other more traditionally studied algebraic
structures, e.g. vector spaces or polynomial rings, is the manner in
which dynamics is captured. In traditional structures, dynamics is typically
expressed through morphisms between such structures, as in linear maps
between vector spaces or morphisms between rings. In algebras
associated with the semantics of computation, the dynamics is
expressed as part of the algebraic structure itself, through a
reduction reduction relation typically denoted by $\red$. Below, we
give a recursive presentation of this relation for the calculus used
in the encoding.

$\red \subseteq \pi \times \pi$
$\red : \pi \to \mathcal{P}(\pi)$

\begin{mathpar}
  \inferrule* [lab=Comm] { \textsf{match}( x_{src}, x_{trgt} ) } { x_{trgt}?(y)P \; | \; x_{src}!\langle {Q} \rangle \red P\{\quotep{Q}/y}\} }
  \and \\
  \inferrule* [lab=Par] {{P} \red {P}'} {{{P} | {Q}} \red {{P}' | {Q}}}
  \and
  \inferrule* [lab=Equiv]{{{P} \scong {P}'} \andalso {{P}' \red {Q}'} \andalso {{Q}' \scong {Q}}}{{P} \red {Q}}
\end{mathpar}

\begin{eqnarray*}
  match_{\equiv} (\quotep{P},\quotep{Q}) & := & P \equiv Q \\
  match_{\dagger}(\quotep{P},\quotep{Q}) & := & \forall R. P|Q \red^{*} R => R \red^{*} 0 \\
  match_{K}(\quotep{P},\quotep{Q}) & := & K \mbox{ for some context } K
\end{eqnarray*}

$u?(x)P | u!\langle Q \rangle \red P\{\quotep{Q}/x\}$

%We write $\wred$ for $\red^*$, and $P\red$ if $\exists Q $ such that $ P \red Q$.
We write $P\red$ if $\exists Q $ such that $ P \red Q$ and $P\not\red$, otherwise.

\section{Replication}

As mentioned before, it is known that replication (and hence
recursion) can be implemented in a higher-order process algebra
\cite{SangiorgiWalker}. As our first example of calculation with the
machinery thus far presented we give the construction explicitly in
the {\rhoc}.

\begin{eqnarray}
	D_{x} & := & \prefix{x}{y}{(\binpar{\outputp{x}{y}}{@{y}})} \nonumber\\
	\bangp_{x}{P} & := & \binpar{{x}!\langle{\binpar{D_{x}}{P}}\rangle}{D_{x}} \nonumber
\end{eqnarray}

\begin{eqnarray}
	\bangp_{x}{P} & & \nonumber\\
	=
	& {x}!\langle{(\prefix{x}{y}{(\outputp{x}{y} | @{y})) | P}}\rangle 
	      | \prefix{x}{y}{(\outputp{x}{y} | @{y})} & \nonumber\\
	\red
	& (\outputp{x}{y} | @{y})\substn{\quotep{(\prefix{x}{y}{(@{y} | \outputp{x}{y})) | P}}}{y} & \nonumber\\
	=
	& \outputp{x}{\quotep{(\prefix{x}{y}{(\outputp{x}{y} | @{y})) | P}}}
	  | {(\prefix{x}{y}{(\outputp{x}{y} | @{y})) | P}} & \nonumber\\
	\red
	& \ldots & \nonumber\\
	\red^*
	& P | P | \ldots & \nonumber
\end{eqnarray}

Of course, this encoding, as an implementation, runs away, unfolding
$\bangp{P}$ eagerly. A lazier and more implementable replication
operator, restricted to input-guarded processes, may be obtained as follows.

\begin{eqnarray}
\bangp{\prefix{u}{v}{P}} 
	:= 
	\binpar{\lift{x}{\prefix{u}{v}{(\binpar{D(x)}{P})}}}{D(x)} \nonumber
\end{eqnarray}

\begin{remark}
  Note that the lazier definition still does not deal with summation
  or mixed summation (i.e. sums over input and output). The reader is
  invited to construct definitions of replication that deal with these
  features. 

  Further, the definitions are parameterized in a name, $x$. Can you,
  gentle reader, make a definition that eliminates this parameter and
  guarantees no accidental interaction between the replication
  machinery and the process being replicated -- i.e. no accidental
  sharing of names used by the process to get its work done and the
  name(s) used by the replication to effect copying. This latter
  revision of the definition of replication is crucial to obtaining
  the expected identity $!!P \sim !P$.
\end{remark}

\begin{remark}\label{rem:paradoxical_combinator}
  The reader familiar with the lambda calculus will have noticed the
  similarity between $D$ and the paradoxical combinator.

  [Ed. note: the existence of this seems to suggest we have to be more
  restrictive on the set of processes and names we admit if we are to
  support no-cloning.]
\end{remark}

\subsubsection{Bisimulation}

The computational dynamics gives rise to another kind of equivalence,
the equivalence of computational behavior. As previously mentioned
this is typically captured \emph{via} some form of bisimulation.

% The notion we use in this paper is weak barbed bisimulation
% \cite{milner91polyadicpi}.

The notion we use in this paper is derived from weak barbed
bisimulation \cite{milner91polyadicpi}. 

\begin{definition}
An \emph{observation relation}, $\downarrow_{\mathcal N}$, over a set
of names, $\mathcal N$, is the smallest relation satisfying the rules
below.

\infrule[Out-barb]{y \in {\mathcal N}, \; x \nameeq y}
		  {\outputp{x}{v} \downarrow_{\mathcal N} x}
\infrule[Par-barb]{\mbox{$P\downarrow_{\mathcal N} x$ or $Q\downarrow_{\mathcal N} x$}}
		  {\binpar{P}{Q} \downarrow_{\mathcal N} x}

We write $P \Downarrow_{\mathcal N} x$ if there is $Q$ such that 
$P \wred Q$ and $Q \downarrow_{\mathcal N} x$.
\end{definition}

\begin{definition}
%\label{def.bbisim}
An  ${\mathcal N}$-\emph{barbed bisimulation} over a set of names, ${\mathcal N}$, is a symmetric binary relation 
${\mathcal S}_{\mathcal N}$ between agents such that $P\rel{S}_{\mathcal N}Q$ implies:
\begin{enumerate}
\item If $P \red P'$ then $Q \wred Q'$ and $P'\rel{S}_{\mathcal N} Q'$.
\item If $P\downarrow_{\mathcal N} x$, then $Q\Downarrow_{\mathcal N} x$.
\end{enumerate}
$P$ is ${\mathcal N}$-barbed bisimilar to $Q$, written
$P \wbbisim_{\mathcal N} Q$, if $P \rel{S}_{\mathcal N} Q$ for some ${\mathcal N}$-barbed bisimulation ${\mathcal S}_{\mathcal N}$.
\end{definition}

$\mathcal{R} \subseteq \pi \times \pi$

$P \mathcal{R} Q => \forall P'. P \red P' \Rightarrow \exists Q'. Q \red Q', P' \mathcal{R} Q'$

$P \vdash x \Rightarrow Q \vdash x$

\begin{mathpar}
  \inferrule*[lab=Out-barb]{x \nameeq y}{{y}!\langle{Q}\rangle \vdash x}
  \and
  \inferrule*[lab=Par-barb]{\mbox{$P\vdash x$ or $Q\vdash x$}}{\binpar{P}{Q} \vdash x}
\end{mathpar}

\subsubsection{Contexts}

One of the principle advantages of computational calculi like the
$\pi$-calculus is a well-defined notion of context,
contextual-equivalence and a correlation between
contextual-equivalence and notions of bisimulation. The notion of
context allows the decomposition of a process into (sub-)process and
its syntactic environment, its context. Thus, a context may be
thought of as a process with a ``hole'' (written $\Box$) in it. The
application of a context $M$ to a process $P$, written $M[P]$, is
tantamount to filling the hole in $M$ with $P$. In this paper we do
not need the full weight of this theory, but do make use of the notion
of context in the proof the main theorem. 

\begin{mathpar}
  \inferrule* [lab=summation] {} {{M_{M},M_{N}} \bc \Box \;|\; x.M_{A} \;|\; M_{M}+M_{N}}
  \and
  \inferrule* [lab=agent] {} {{M_{A}} \bc (\vec{x})M_{P} \;| \; \clift{P_0,\ldots,M_{P},\ldots,P_N}}
  \and \\
  \inferrule* [lab=process] {} {{M_{P}} \bc M_{N} \;| \;P|M_{P} }
\end{mathpar} 

\begin{mathpar}
  \inferrule* [lab=sychronization] {} {M_{N} \bc \Box \;|\; x?M_{F} \;|\; x!M_{C}}
  \and
  \inferrule* [lab=abstraction] {} {{M_{F}} \bc (x)M_{P} }
  \and
  \inferrule* [lab=concretion] {} {{M_{C}} \bc \langle M_{P} \rangle }
  \and \\
  \inferrule* [lab=process] {} {{M_{P}} \bc M_{N} \;| \;P|M_{P} }
\end{mathpar}

\begin{definition}[contextual application] Given a context $M$, and
  process $P$, we define the \emph{contextual application}, $M[P] :=
  M\{P/\Box\}$. That is, the contextual application of M to P is the
  substitution of $P$ for $\Box$ in $M$.
\end{definition}

$\meaningof{-} : L \to \mathcal{P}(\pi)$

\begin{mathpar}
  \inferrule* [lab=collection] {} {\meaningof{true} = \pi, \and \meaningof{~E} = \pi \setminus \meaningof{E}, \and \meaningof{E_{1} \& E_{2}} = \meaningof{E_{1}} \cap \meaningof{E_{2}}}
\end{mathpar}

\begin{mathpar}
  \inferrule* [lab=structure] {} {\meaningof{0} = \{ P \in \pi | P \equiv 0 \}, \and \\ \meaningof{E_1 | E_2} = \{ P \in \pi | P \equiv P_{1} | P_{2}, P_{1} \in \meaningof{E_{1}}, P_{2} \in \meaningof{E_2}\} }
\end{mathpar}

\begin{mathpar}
 \inferrule* [lab=behavior] {} {\meaningof{\langle a?b \rangle E} = \{ P \in \pi | P \equiv Q | u?(y)P', \\ \and \\\\ \and \\ \;\;\; u \in \meaningof{a}, \forall z.P'\{z/y\} \in \meaningof{E\{z/b\}}\}, \and \\ \meaningof{a!E} = \{ P \in \pi | P \equiv Q | x!\langle P' \rangle, x \in \meaningof{a} P' \in \meaningof{E}\} }
\end{mathpar}

\begin{mathpar}
 \inferrule* [lab=nominal] {} {\meaningof{\quotep{E}} = \{ \quotep{P} \in \quotep{\pi} | P \in \meaningof{E} \}, \and \meaningof{\quotep{P}} = \{ \quotep{Q} \in \quotep{\pi} | P \equiv Q \} \and \\ \meaningof{@\quotep{E}} = \{ P \in \pi | P \equiv @x, x \in \meaningof{E} \}}
\end{mathpar}

\begin{eqnarray*}
  \\
  \meaningof{-} : TS \to ST
\end{eqnarray*}

\begin{eqnarray*}
  \\
  L : TS \to ST
\end{eqnarray*}

\begin{eqnarray*}
  \\
  P \models E \iff P \in \meaningof{E}
\end{eqnarray*}

\begin{eqnarray*}
  P \approx_{L} Q \iff \forall E \in L. P \models E \iff Q \models E
\end{eqnarray*}

\begin{eqnarray*}
  P \approx_{K} Q
\end{eqnarray*}

\begin{eqnarray*}
  P \approx Q
\end{eqnarray*}

$\approx_{K} = \approx = \approx_{L}$

\subsubsection{Contextual duality}

Note that contexts extend the quotation operation to a family of
operations from processes to names. Given a context, $M$, we can
define a \emph{nominal context}, $\quotep{M}$ by $\quotep{M}[P] :=
\quotep{M[P]}$. To foreshadow what is to come we observe that these
operations enjoy a duality with processes very much like the duality
between vectors and maps from vectors to scalars.

Further, because the calculus is essentially higher-order, we have a
correspondence between contexts and processes. More specifically,
given a name $x$ and a context $M$ we can construct $M^{*}_{x}$ such
that 

\begin{mathpar}
  M^{*}_{x} | \lift{x}{P} \red M[P]
\end{mathpar}

namely,

\begin{mathpar}
  M^{*}_{x} := x?(u).M[\dropn{u}]
\end{mathpar}

The dependence of $M^{*}_{x}$ on a name makes it an abstraction, 

\begin{mathpar}
  M^{*} := (x)x?(u).M[\dropn{u}]
\end{mathpar}

\subsection{Additional notation}

It will sometimes be convenient to denote the process a name
quotes. We already have the notation $x = \quotep{P}$, but it will be
convenient to introduce an alternate notation, $\procn{x}$, when we
want to emphasize the connection to the use of the name. Note that, by
virtue of name equivalence, $\quotep{\procn{x}} \nameeq x$; so, the
notation is consistent with previous definitions.

Further, because names have structure it is possible to effect
substitutions on the basis of that structure. This means we need to
upgrade our notation for substitutions, which we accomplish by
adapting comprehension notation. Thus,

\begin{mathpar}
  P\{ y / x : x \in S \}
\end{mathpar}

is interpreted to mean the process derived from P by replacing (in a
capture-avoiding manner) each occurrence of $x$ in $S$ by $y$. For example,

\begin{mathpar}
  P\{ \quotep{\procn{x}|\procn{x}} / x : x \in \freenames{P} \}
\end{mathpar}

will replace each (occurrence) of a free name $x$ in $P$ by
$\quotep{\procn{x}|\procn{x}}$.

Also, we will avail ourselves of the notation $x^{L}$ and $x^{R}$ to
denote injections of a name into disjoint copies of the name
space. There are numerous ways to accomplish this. One example can be
found in \cite{MeredithR05}. This notation overloads to vectors of
names: $\vec{x}^{\pi} := (x_{i}^{\pi} \; : \; 0 \leq i < |\vec{x}| )$ where $\pi \in \{L,R\}$.

We also use $P^{\Box} := P|\Box$.

In \cite{MeredithR05} an interpretation of the new operator is
given. It turns out that there are several possible interpretations
all enjoying the requisite algebraic properties of the operator (see
\cite{milner91polyadicpi}). We will therefore make liberal use of
$(\nu\; \vec{x})P$.

% subsection the_syntax_and_semantics_of_the_notation_system (end)   

\input{qm2pi.qmops} 

\input{qm2pi.sterngerlach} 

\input{qm2pi.metric} 

% section concurrent_process_calculi (end)

%\input{qm2pi.proofsketch}

% section proof sketch (end)

%\input{qm2pi.slviaknots} 

% section spatial logic via knots (end)

\input{qm2pi.conclusion}

% section conclusion (end)

%\input{qm2pi.dtcodes} 

% section wiring algorithm (end)

\input{qm2pi.ack} 

% section acknowledgments (end)

\newpage


\bibliographystyle{plain}   
\bibliography{../../biblios/main.bib}

\input{qm2pi.rhodetails}

\end{document}

 

%\ifpdf
%\usepackage[pdftex]{graphicx}
%\else
%\usepackage{graphicx}
%\fi

 % \ifpdf
%  \usepackage{pdfsync}
%  \if


%\title{Brief Article}
%\author{David F. Snyder}
%\author{L.G. Meredith}

%\address{Dept. of Math., Texas State University--San Marcos, San Marcos, TX 78666}
       
\pagestyle{empty}


\begin{document}

\lstset{language=[Objective]Caml,frame=shadowbox}

\documentclass[12pt]{llncs}
%\documentclass{jktr}

\usepackage[pdftex]{hyperref}                   
\usepackage {listings}
\usepackage {mathpartir}
\usepackage{bcprules}
%\usepackage{listings}
                       
\usepackage{graphicx} 
%\usepackage[margins=2.5cm,nohead,nofoot]{geometry}
%\usepackage{geometry}
\usepackage{amsfonts}
\usepackage{amstext}
\usepackage{latexsym}
\usepackage{amssymb}
\usepackage{color}


%\include{myPreamble}
\include{qm2pi.local} 

%\ifpdf
%\usepackage[pdftex]{graphicx}
%\else
%\usepackage{graphicx}
%\fi

 % \ifpdf
%  \usepackage{pdfsync}
%  \if


%\title{Brief Article}
%\author{David F. Snyder}
%\author{L.G. Meredith}

%\address{Dept. of Math., Texas State University--San Marcos, San Marcos, TX 78666}
       
\pagestyle{empty}


\begin{document}

\lstset{language=[Objective]Caml,frame=shadowbox}

\input{qm2pi.front}

% section front matter (end)

\input{qm2pi.intro} 
 
% section introduction (end)

% \input{qm2pi.knotations} 

% section notation (end)

\input{qm2pi.process.calculi} 

% section concurrent_process_calculi_and_spatial_logics_ (end)
    
%\input{qm2pi.knots2pi} 

%\input{qm2pi.trefoil} 

%\input{qm2pi.mainthm} 

% subsection basic_interpretation (end)

%\input{qm2pi.rho.presentation} 
\subsection{The syntax and semantics of the notation system}\label{sub:the_syntax_and_semantics_of_the_notation_system} % (fold)

We now summarize a technical presentation of the calculus that
embodies our theory of dynamics. The typical presentation of such a
calculus follows the style of giving generators and relations on
them. The grammar, below, describing term constructors, freely
generates the set of processes, $\Proc$. This set is then quotiented
by a relation known as structural congruence and it is over this set
that the notion of dynamics is expressed. This presentation is
essentially that of \cite{MeredithR05} with the addition of
polyadicity and summation. For readability we have relegated some of
the technical subtleties to an appendix.

\subsubsection{Process grammar}\label{subsub:process_grammar}

\begin{mathpar}
  \inferrule* [lab=synchronization] {} {{M} \bc \pzero \;|\; x?F \;|\; x!C }
  \and
  \inferrule* [lab=abstraction] {} {{F} \bc (x)P}
  \and
  \inferrule* [lab=concretion] {} {{C} \bc \langle Q \rangle}
  \and
  \inferrule* [lab=process] {} {{P,Q} \bc M \;| \;P|Q \;|\; @{x}}
  \and
  \inferrule* [lab=name] {} {{x} \bc \quotep{P}}
\end{mathpar} 

Note that $\vec{x}$ (resp. $\vec{P}$) denotes a vector of names
(resp. processes) of length $|\vec{x}|$ (resp. $|\vec{P}|$). We adopt
the following useful abbreviations.

\begin{mathpar}
   x?(\vec{y}).P := x.(\vec{y})P \and  x\clift{\vec{P}} := x.\clift{\vec{P}}
   \and x!(y) := \lift{x}{\dropn{y}}
   \and \Pi_{i=0}^{n-1}P_i := P_0 | \ldots | P_{n-1}
\end{mathpar}

\subsubsection{Structural congruence}

\paragraph{Free and bound names and alpha-equivalence.} At the
core of structural equivalence is alpha-equivalence which identifies
process that are the same up to a change of variable. Formally, we
recognize the distinction between free and bound names. The free names
of a process, $\freenames{P}$, may be calculated recursively as
follows:

\begin{mathpar}
\freenames{\pzero} := \emptyset
  \and \\
  \freenames{x?(y).P} := \{ x \} \cup (\freenames{P} \setminus \{ y \})
  \and 
  \freenames{x!\langle P \rangle} := \{ x \} \cup \{ P \} 
  \and \\
  \freenames{P|Q} := \freenames{P} \cup \freenames{Q}
  \and \\
  \freenames{@{x}} := \{ x \}
\end{mathpar}

$\pi$
$\quotep{\pi}$

$\freenames{-} : \pi \to \mathcal{P}(\quotep{\pi})$

\begin{eqnarray*}
  \freenames{\pzero} & := & \emptyset \\
  \freenames{x?(y).P} & := & \{ x \} \cup (\freenames{P} \setminus \{ y \}) \\
  \freenames{x!\langle P \rangle} & := & \{ x \} \cup \{ P \} \\
  \freenames{P|Q} & := & \freenames{P} \cup \freenames{Q} \\
  \freenames{\dropn{x}} & := & \{ x \}
\end{eqnarray*}

The bound names of a process, $\boundnames{P}$, are those names occurring in $P$
that are not free. For example, in $x?(y).0$, the name $x$ is free, while $y$ is bound.

\begin{mathpar}
  \inferrule* [lab=monoidal-laws] {} { P|Q \equiv Q|P \and P|0 \equiv P \and P|(Q|R) \equiv (P|Q)|R }
\end{mathpar}

\begin{mathpar}
  \inferrule* [lab=alpha-equivalence] {} { (x)P \equiv (y)P\{y/x\} \and y \not\in \freenames{P} }
\end{mathpar}

\begin{definition}
Then two processes, $P,Q$, are alpha-equivalent if $P = Q\{\vec{y}/\vec{x}\}$ for
some $\vec{x} \in \boundnames{Q},\vec{y} \in \boundnames{P}$, where $Q\{\vec{y}/\vec{x}\}$
denotes the capture-avoiding substitution of $\vec{y}$ for $\vec{x}$ in $Q$.
\end{definition}

\begin{definition}
  The {\em structural congruence} \cite{SangiorgiWalker} , $\equiv$,
  between processes is the least congruence containing
  alpha-equivalence, satisfying the abelian monoid laws
  (associativity, commutativity and $\pzero$ as identity) for parallel
  composition $|$ and for summation $+$.
\end{definition}

\subsection{Name equivalence}

We take name equivalence, written $\nameeq$, to be the smallest
equivalence relation generated by the following rules.

\begin{mathpar}
\inferrule*[lab=Quote-drop]
{ }
{ \quotep{@{x}} \nameeq x }

\inferrule*[lab=Struct-equiv]
{ P \scong Q }
{ \quotep{P} \nameeq \quotep{Q} }
\end{mathpar}

The astute reader will have noticed that the mutual recursion of names
and processes imposes a mutual recursion on alpha-equivalence and
structural equivalence via name-equivalence. Fortunately, all of this
works out pleasantly and we may calculate in the natural way, free of
concern. The reader interested in the details is referred to the
appendix \ref{appendix:rho_details}.

\subsection{Substitution}

We use $\Proc$ for the set of processes, $\QProc$ for the set of
names, and $\id{\{}\vec{y} / \vec{x} \id{\}}$ to denote partial maps,
$s : \QProc \rightarrow \QProc$. A map, $s$ lifts, uniquely, to a map
on process terms, $\widehat{s} : \Proc \rightarrow \Proc$ by the
following equations.

\begin{mathpar}
  (0) \psubstp{Q}{P} := 0 \\
  (R \juxtap S) \psubstp{Q}{P}
  :=    
  (R)\psubstp{Q}{P} \juxtap (S) \psubstp{Q}{P} \\
  (x?(y).R) \psubstp{Q}{P}    
  :=    
  (x)\substp{Q}{P} (z)\concat( (R \psubstn{z}{y}) \psubstp{Q}{P} ) \\
  (\lift{x}{R}) \psubstp{Q}{P}  
  :=
  \lift{(x)\substp{Q}{P}}{ R \psubstp{Q}{P} } \\
%   (\dropn{x})  \psubstp{Q}{P}       
%   := 
%   \left\{ 
%     \begin{array}{ccc} 
%       \dropn{\quotep{Q}} & & x \nameeq \quotep{P} \\
%       \dropn{x} & & otherwise \\
%     \end{array}
%   \right. 
  (\dropn{x})  \psubstp{Q}{P}       
  := 
  \left\{ 
    \begin{array}{ccc} 
      Q & & x \nameeq \quotep{P} \\
      \dropn{x} & & otherwise \\
    \end{array}
  \right.
\end{mathpar}
 

where

\begin{eqnarray}
  (x)\id{\{} \lpquote Q \rpquote / \lpquote P \rpquote \id{\}}            = 
  \left\{ 
    \begin{array}{ccc}
      \lpquote Q \rpquote & & x \nameeq \lpquote P \rpquote \\
      x & & otherwise \\
    \end{array}
  \right. \nonumber
\end{eqnarray}

and $z$ is chosen distinct from $\quotep{P}$, $\quotep{Q}$, the free
names in $Q$, and all the names in $R$. Our $\alpha$-equivalence will
be built in the standard way from this substitution.

\begin{remark}\label{rem:no_self_referential_names}
  One consequence of these definitions is that $\forall P. \quotep{P}
  \not\in \freenames{P}$.
\end{remark}

\subsection{ Dynamic quote: an example }

Anticipating something of what's to come, consider applying the
substitution, $\widehat{\id{\{}u / z \id{\}}}$, to the following pair
of processes, $\lift{w}{y!(z)}$ and $w[ \lpquote y!(z) \rpquote ]$.

\begin{eqnarray}
	\lift{w}{y!(z)}\widehat{\id{\{}u / z \id{\}}}
		& = &
		\lift{w}{y!(u)} \nonumber\\
	w[ \lpquote y!(z) \rpquote ] \widehat{ \id{\{}u / z \id{\}} }
		& = &
		w[ \lpquote y!(z) \rpquote ] \nonumber
\end{eqnarray}

Because the body of the process between quotes is impervious to
substitution, we get radically different answers. In fact, by
examining the first process in an input context,
e.g. $x?(z).\lift{w}{y!(z)}$, we see that the process under the lift
operator may be shaped by prefixed inputs binding a name inside it. In
this sense, the lift operator will be seen as a way to dynamically
construct processes before reifying them as names.

Finally equipped with these standard features we can present the
dynamics of the calculus.

\subsubsection{Operational semantics} 

Finally, we introduce the computational dynamics. What marks these
algebras as distinct from other more traditionally studied algebraic
structures, e.g. vector spaces or polynomial rings, is the manner in
which dynamics is captured. In traditional structures, dynamics is typically
expressed through morphisms between such structures, as in linear maps
between vector spaces or morphisms between rings. In algebras
associated with the semantics of computation, the dynamics is
expressed as part of the algebraic structure itself, through a
reduction reduction relation typically denoted by $\red$. Below, we
give a recursive presentation of this relation for the calculus used
in the encoding.

$\red \subseteq \pi \times \pi$
$\red : \pi \to \mathcal{P}(\pi)$

\begin{mathpar}
  \inferrule* [lab=Comm] { \textsf{match}( x_{src}, x_{trgt} ) } { x_{trgt}?(y)P \; | \; x_{src}!\langle {Q} \rangle \red P\{\quotep{Q}/y}\} }
  \and \\
  \inferrule* [lab=Par] {{P} \red {P}'} {{{P} | {Q}} \red {{P}' | {Q}}}
  \and
  \inferrule* [lab=Equiv]{{{P} \scong {P}'} \andalso {{P}' \red {Q}'} \andalso {{Q}' \scong {Q}}}{{P} \red {Q}}
\end{mathpar}

\begin{eqnarray*}
  match_{\equiv} (\quotep{P},\quotep{Q}) & := & P \equiv Q \\
  match_{\dagger}(\quotep{P},\quotep{Q}) & := & \forall R. P|Q \red^{*} R => R \red^{*} 0 \\
  match_{K}(\quotep{P},\quotep{Q}) & := & K \mbox{ for some context } K
\end{eqnarray*}

$u?(x)P | u!\langle Q \rangle \red P\{\quotep{Q}/x\}$

%We write $\wred$ for $\red^*$, and $P\red$ if $\exists Q $ such that $ P \red Q$.
We write $P\red$ if $\exists Q $ such that $ P \red Q$ and $P\not\red$, otherwise.

\section{Replication}

As mentioned before, it is known that replication (and hence
recursion) can be implemented in a higher-order process algebra
\cite{SangiorgiWalker}. As our first example of calculation with the
machinery thus far presented we give the construction explicitly in
the {\rhoc}.

\begin{eqnarray}
	D_{x} & := & \prefix{x}{y}{(\binpar{\outputp{x}{y}}{@{y}})} \nonumber\\
	\bangp_{x}{P} & := & \binpar{{x}!\langle{\binpar{D_{x}}{P}}\rangle}{D_{x}} \nonumber
\end{eqnarray}

\begin{eqnarray}
	\bangp_{x}{P} & & \nonumber\\
	=
	& {x}!\langle{(\prefix{x}{y}{(\outputp{x}{y} | @{y})) | P}}\rangle 
	      | \prefix{x}{y}{(\outputp{x}{y} | @{y})} & \nonumber\\
	\red
	& (\outputp{x}{y} | @{y})\substn{\quotep{(\prefix{x}{y}{(@{y} | \outputp{x}{y})) | P}}}{y} & \nonumber\\
	=
	& \outputp{x}{\quotep{(\prefix{x}{y}{(\outputp{x}{y} | @{y})) | P}}}
	  | {(\prefix{x}{y}{(\outputp{x}{y} | @{y})) | P}} & \nonumber\\
	\red
	& \ldots & \nonumber\\
	\red^*
	& P | P | \ldots & \nonumber
\end{eqnarray}

Of course, this encoding, as an implementation, runs away, unfolding
$\bangp{P}$ eagerly. A lazier and more implementable replication
operator, restricted to input-guarded processes, may be obtained as follows.

\begin{eqnarray}
\bangp{\prefix{u}{v}{P}} 
	:= 
	\binpar{\lift{x}{\prefix{u}{v}{(\binpar{D(x)}{P})}}}{D(x)} \nonumber
\end{eqnarray}

\begin{remark}
  Note that the lazier definition still does not deal with summation
  or mixed summation (i.e. sums over input and output). The reader is
  invited to construct definitions of replication that deal with these
  features. 

  Further, the definitions are parameterized in a name, $x$. Can you,
  gentle reader, make a definition that eliminates this parameter and
  guarantees no accidental interaction between the replication
  machinery and the process being replicated -- i.e. no accidental
  sharing of names used by the process to get its work done and the
  name(s) used by the replication to effect copying. This latter
  revision of the definition of replication is crucial to obtaining
  the expected identity $!!P \sim !P$.
\end{remark}

\begin{remark}\label{rem:paradoxical_combinator}
  The reader familiar with the lambda calculus will have noticed the
  similarity between $D$ and the paradoxical combinator.

  [Ed. note: the existence of this seems to suggest we have to be more
  restrictive on the set of processes and names we admit if we are to
  support no-cloning.]
\end{remark}

\subsubsection{Bisimulation}

The computational dynamics gives rise to another kind of equivalence,
the equivalence of computational behavior. As previously mentioned
this is typically captured \emph{via} some form of bisimulation.

% The notion we use in this paper is weak barbed bisimulation
% \cite{milner91polyadicpi}.

The notion we use in this paper is derived from weak barbed
bisimulation \cite{milner91polyadicpi}. 

\begin{definition}
An \emph{observation relation}, $\downarrow_{\mathcal N}$, over a set
of names, $\mathcal N$, is the smallest relation satisfying the rules
below.

\infrule[Out-barb]{y \in {\mathcal N}, \; x \nameeq y}
		  {\outputp{x}{v} \downarrow_{\mathcal N} x}
\infrule[Par-barb]{\mbox{$P\downarrow_{\mathcal N} x$ or $Q\downarrow_{\mathcal N} x$}}
		  {\binpar{P}{Q} \downarrow_{\mathcal N} x}

We write $P \Downarrow_{\mathcal N} x$ if there is $Q$ such that 
$P \wred Q$ and $Q \downarrow_{\mathcal N} x$.
\end{definition}

\begin{definition}
%\label{def.bbisim}
An  ${\mathcal N}$-\emph{barbed bisimulation} over a set of names, ${\mathcal N}$, is a symmetric binary relation 
${\mathcal S}_{\mathcal N}$ between agents such that $P\rel{S}_{\mathcal N}Q$ implies:
\begin{enumerate}
\item If $P \red P'$ then $Q \wred Q'$ and $P'\rel{S}_{\mathcal N} Q'$.
\item If $P\downarrow_{\mathcal N} x$, then $Q\Downarrow_{\mathcal N} x$.
\end{enumerate}
$P$ is ${\mathcal N}$-barbed bisimilar to $Q$, written
$P \wbbisim_{\mathcal N} Q$, if $P \rel{S}_{\mathcal N} Q$ for some ${\mathcal N}$-barbed bisimulation ${\mathcal S}_{\mathcal N}$.
\end{definition}

$\mathcal{R} \subseteq \pi \times \pi$

$P \mathcal{R} Q => \forall P'. P \red P' \Rightarrow \exists Q'. Q \red Q', P' \mathcal{R} Q'$

$P \vdash x \Rightarrow Q \vdash x$

\begin{mathpar}
  \inferrule*[lab=Out-barb]{x \nameeq y}{{y}!\langle{Q}\rangle \vdash x}
  \and
  \inferrule*[lab=Par-barb]{\mbox{$P\vdash x$ or $Q\vdash x$}}{\binpar{P}{Q} \vdash x}
\end{mathpar}

\subsubsection{Contexts}

One of the principle advantages of computational calculi like the
$\pi$-calculus is a well-defined notion of context,
contextual-equivalence and a correlation between
contextual-equivalence and notions of bisimulation. The notion of
context allows the decomposition of a process into (sub-)process and
its syntactic environment, its context. Thus, a context may be
thought of as a process with a ``hole'' (written $\Box$) in it. The
application of a context $M$ to a process $P$, written $M[P]$, is
tantamount to filling the hole in $M$ with $P$. In this paper we do
not need the full weight of this theory, but do make use of the notion
of context in the proof the main theorem. 

\begin{mathpar}
  \inferrule* [lab=summation] {} {{M_{M},M_{N}} \bc \Box \;|\; x.M_{A} \;|\; M_{M}+M_{N}}
  \and
  \inferrule* [lab=agent] {} {{M_{A}} \bc (\vec{x})M_{P} \;| \; \clift{P_0,\ldots,M_{P},\ldots,P_N}}
  \and \\
  \inferrule* [lab=process] {} {{M_{P}} \bc M_{N} \;| \;P|M_{P} }
\end{mathpar} 

\begin{mathpar}
  \inferrule* [lab=sychronization] {} {M_{N} \bc \Box \;|\; x?M_{F} \;|\; x!M_{C}}
  \and
  \inferrule* [lab=abstraction] {} {{M_{F}} \bc (x)M_{P} }
  \and
  \inferrule* [lab=concretion] {} {{M_{C}} \bc \langle M_{P} \rangle }
  \and \\
  \inferrule* [lab=process] {} {{M_{P}} \bc M_{N} \;| \;P|M_{P} }
\end{mathpar}

\begin{definition}[contextual application] Given a context $M$, and
  process $P$, we define the \emph{contextual application}, $M[P] :=
  M\{P/\Box\}$. That is, the contextual application of M to P is the
  substitution of $P$ for $\Box$ in $M$.
\end{definition}

$\meaningof{-} : L \to \mathcal{P}(\pi)$

\begin{mathpar}
  \inferrule* [lab=collection] {} {\meaningof{true} = \pi, \and \meaningof{~E} = \pi \setminus \meaningof{E}, \and \meaningof{E_{1} \& E_{2}} = \meaningof{E_{1}} \cap \meaningof{E_{2}}}
\end{mathpar}

\begin{mathpar}
  \inferrule* [lab=structure] {} {\meaningof{0} = \{ P \in \pi | P \equiv 0 \}, \and \\ \meaningof{E_1 | E_2} = \{ P \in \pi | P \equiv P_{1} | P_{2}, P_{1} \in \meaningof{E_{1}}, P_{2} \in \meaningof{E_2}\} }
\end{mathpar}

\begin{mathpar}
 \inferrule* [lab=behavior] {} {\meaningof{\langle a?b \rangle E} = \{ P \in \pi | P \equiv Q | u?(y)P', \\ \and \\\\ \and \\ \;\;\; u \in \meaningof{a}, \forall z.P'\{z/y\} \in \meaningof{E\{z/b\}}\}, \and \\ \meaningof{a!E} = \{ P \in \pi | P \equiv Q | x!\langle P' \rangle, x \in \meaningof{a} P' \in \meaningof{E}\} }
\end{mathpar}

\begin{mathpar}
 \inferrule* [lab=nominal] {} {\meaningof{\quotep{E}} = \{ \quotep{P} \in \quotep{\pi} | P \in \meaningof{E} \}, \and \meaningof{\quotep{P}} = \{ \quotep{Q} \in \quotep{\pi} | P \equiv Q \} \and \\ \meaningof{@\quotep{E}} = \{ P \in \pi | P \equiv @x, x \in \meaningof{E} \}}
\end{mathpar}

\begin{eqnarray*}
  \\
  \meaningof{-} : TS \to ST
\end{eqnarray*}

\begin{eqnarray*}
  \\
  L : TS \to ST
\end{eqnarray*}

\begin{eqnarray*}
  \\
  P \models E \iff P \in \meaningof{E}
\end{eqnarray*}

\begin{eqnarray*}
  P \approx_{L} Q \iff \forall E \in L. P \models E \iff Q \models E
\end{eqnarray*}

\begin{eqnarray*}
  P \approx_{K} Q
\end{eqnarray*}

\begin{eqnarray*}
  P \approx Q
\end{eqnarray*}

$\approx_{K} = \approx = \approx_{L}$

\subsubsection{Contextual duality}

Note that contexts extend the quotation operation to a family of
operations from processes to names. Given a context, $M$, we can
define a \emph{nominal context}, $\quotep{M}$ by $\quotep{M}[P] :=
\quotep{M[P]}$. To foreshadow what is to come we observe that these
operations enjoy a duality with processes very much like the duality
between vectors and maps from vectors to scalars.

Further, because the calculus is essentially higher-order, we have a
correspondence between contexts and processes. More specifically,
given a name $x$ and a context $M$ we can construct $M^{*}_{x}$ such
that 

\begin{mathpar}
  M^{*}_{x} | \lift{x}{P} \red M[P]
\end{mathpar}

namely,

\begin{mathpar}
  M^{*}_{x} := x?(u).M[\dropn{u}]
\end{mathpar}

The dependence of $M^{*}_{x}$ on a name makes it an abstraction, 

\begin{mathpar}
  M^{*} := (x)x?(u).M[\dropn{u}]
\end{mathpar}

\subsection{Additional notation}

It will sometimes be convenient to denote the process a name
quotes. We already have the notation $x = \quotep{P}$, but it will be
convenient to introduce an alternate notation, $\procn{x}$, when we
want to emphasize the connection to the use of the name. Note that, by
virtue of name equivalence, $\quotep{\procn{x}} \nameeq x$; so, the
notation is consistent with previous definitions.

Further, because names have structure it is possible to effect
substitutions on the basis of that structure. This means we need to
upgrade our notation for substitutions, which we accomplish by
adapting comprehension notation. Thus,

\begin{mathpar}
  P\{ y / x : x \in S \}
\end{mathpar}

is interpreted to mean the process derived from P by replacing (in a
capture-avoiding manner) each occurrence of $x$ in $S$ by $y$. For example,

\begin{mathpar}
  P\{ \quotep{\procn{x}|\procn{x}} / x : x \in \freenames{P} \}
\end{mathpar}

will replace each (occurrence) of a free name $x$ in $P$ by
$\quotep{\procn{x}|\procn{x}}$.

Also, we will avail ourselves of the notation $x^{L}$ and $x^{R}$ to
denote injections of a name into disjoint copies of the name
space. There are numerous ways to accomplish this. One example can be
found in \cite{MeredithR05}. This notation overloads to vectors of
names: $\vec{x}^{\pi} := (x_{i}^{\pi} \; : \; 0 \leq i < |\vec{x}| )$ where $\pi \in \{L,R\}$.

We also use $P^{\Box} := P|\Box$.

In \cite{MeredithR05} an interpretation of the new operator is
given. It turns out that there are several possible interpretations
all enjoying the requisite algebraic properties of the operator (see
\cite{milner91polyadicpi}). We will therefore make liberal use of
$(\nu\; \vec{x})P$.

% subsection the_syntax_and_semantics_of_the_notation_system (end)   

\input{qm2pi.qmops} 

\input{qm2pi.sterngerlach} 

\input{qm2pi.metric} 

% section concurrent_process_calculi (end)

%\input{qm2pi.proofsketch}

% section proof sketch (end)

%\input{qm2pi.slviaknots} 

% section spatial logic via knots (end)

\input{qm2pi.conclusion}

% section conclusion (end)

%\input{qm2pi.dtcodes} 

% section wiring algorithm (end)

\input{qm2pi.ack} 

% section acknowledgments (end)

\newpage


\bibliographystyle{plain}   
\bibliography{../../biblios/main.bib}

\input{qm2pi.rhodetails}

\end{document}



% section front matter (end)

\section{Introduction}\label{sec:introduction} % (fold)
In this draft of the material i am going to have to dispense with the
usual writing conventions adopted in papers on these topics. i'm going
to have adopt whatever tone i need at the time i'm writing up the
calculations. Sometimes this may be very conversational; others it may
be the barest mathematical grunts; others still it may be that i have
lifted text from one of my other papers because the exposition of some
point was better said there. i hope that my readers are not unduly put
out by this decision. i'm not doing this to flout convention or be
rebellious. i find these calculations very technically challenging. To
keep everything going technically, something has to give; i have to
let go of some cognitive burden. So, the academic writing style --
with all of its trade-offs in terms of facilitating technical
communication -- is what i'm letting go of. Perhaps subsequent drafts
can be tightened and polished, but for now, i'm going to speak as if
we were sitting together in a coffee shop with a laptop, wifi and a
pad of paper and a pencil.

So, here's what i have to say. We -- you and i, comfortably ensconced
in our coffee shop and well-equipped with our tools -- can realize and
carry out the calculations of quantum mechanics over a very different
formal theory of dynamics, a formal theory of dynamics that
corresponds to a theory of concurrent computation with
\emph{reflection}. It has the advantage that the underlying theory is
already `quantized', but supports analogues all of the continuuous
operations. Strikingly, this underlying theory has recently been
connected with a notion of metric that we can show, by calculating
together, coincides with the metric induced by the inner product.

There are a lot of reasons why you might be interested in seeing
calculations of this form. Here's why i'm interested. For the past
several centuries there has been no competitor to the ``Newtonian''
account of dynamics. As a result the predominant share of accounts of
dynamical systems and situations have had to be formulated in terms of
the Newtonian machinery. i view this as an intellectually dangerous
position to occupy. Everything, despite it's intrinsic shape, turns
into a nail to be hit with this hammer. Recently, however, the theory
of computation has matured to the point where we have candidates for
theories of dynamics that offer very different perspective on
reasoning about dynamical systems and situations. Testing these
candidates against very successful accounts of dynamical situations,
like quantum mechanics, is going to give us some sense of how mature
they are and some measure of the quality of these accounts of
dynamics.

\subsection{Summary of contributions and outline of paper}

So, we're going to develop an interpretation of the operations of
quantum mechanics normally interpreted by Hilbert spaces and
operators. We're going to do this over a theory of computation. Note
that this is very different than the usual quantum computation program
which develops notions of computation over quantum mechanics. Rather,
we are developing a story that aligns with Wheeler's slogan: It from
Bit. To do this we will first provide an account of the theory of
computation at play here. Then we will dive into a calculation-driven
interpretation of the operations of quantum mechanics.

The reason we take this approach is that -- until very recently --
there hasn't been an axiomatic account of quantum mechanics. As a
result there has been no sharp delineation of the mathematical theory
supporting interpretation of the physical theory and the physical
theory, itself. So, ambient features of the maths are free to be
exploited (or supressed) without a real accounting of their physical
relevance. There is no sharp statement ``here's the physical theory''
qua \emph{theory} and ``here's the mathematical interpretation''
enabling a judgment of how faithful the interpretation is -- apart
from experimental observation. When there is an axiomatic account we
can judge how well a given mathematical formalism supports an
interpretation of the axioms, independent of
experimentation. Likewise, we can judge how well we have captured our
physical evidence and experience with our axiomatics, independent of
any specific mathematical implementation, with accidental detail that
may or may not have physical significance. 

In lieu of a fully fleshed out and vetted axiomatic account of quantum
mechanics, interpreting the operational notions in service of modeling
physical systems will have to suffice. In other words, we are not in
the business of providing a model of Hilbert spaces and operators. We
are in the business of providing a model of quantum mechanics because
we are motivated by testing our notions of dynamics against physical
theory; and, the predictive calculations of the physical theory must
serve as the best formulation -- shy of a fully fleshed out axiomatic
account -- of the physical theory itself (as they have for scientific
theories since time immemorial). Put another way, despite a
whole-hearted commitment to an It-from-Bit ontology, we are firmly
aligned with the shut-up-and-calculate camp as the best way to obtain
results either from the physical perspective or as a quality assurance
measure of our fledgling theory of dynamics.

In detail, we present a reflective process calculus. Then we develop
intuitive correspondences between the notions available in this
calculus and the usual physical notions supporting quantum mechanical
calculations. Thus, 

\begin{table}[htp]
  \center{
    \fbox{
      \begin{tabular}{c|c}
        quantum mechanics & process calculus \\
        \hline
        scalar & name \\
        state vector & process \\
        dual & contextual duals \\
        matrix & formal sums of process-context-dual pairs \\
        orthogonality & process annihilation \\
        inner product & execution-formula + quoting
      \end{tabular}
    }
  }
  \caption{QM - process calculi correspondences}
\end{table}

Then we tighten up these intuitions to operational definitions. We
employ the Dirac notation as the best proxy we can find for an
abstract syntax of the quantum mechanical notions. The definitions we
develop put us in contact with equational constraints coming from the
theory that we demonstrate the definitions and calculations satisfy.

This puts us in a position to shut up and calculate for the
Stern-Gerlach experimental set up, showing how these predictive
calculations become calculations on processes in our theory of a
reflective process calculus.

Penultimately, we demonstrate that the notion of metric coming from
the inner product coincides with the notion of metric available from
the theory of bisimulation. This demonstration gives us the right to
think of space as arising from behavior. Finally, we consider where we
might go from the new vantage point we have obtained.

% section introduction (end) 
 
% section introduction (end)

% \documentclass[12pt]{llncs}
%\documentclass{jktr}

\usepackage[pdftex]{hyperref}                   
\usepackage {listings}
\usepackage {mathpartir}
\usepackage{bcprules}
%\usepackage{listings}
                       
\usepackage{graphicx} 
%\usepackage[margins=2.5cm,nohead,nofoot]{geometry}
%\usepackage{geometry}
\usepackage{amsfonts}
\usepackage{amstext}
\usepackage{latexsym}
\usepackage{amssymb}
\usepackage{color}


%\include{myPreamble}
\include{qm2pi.local} 

%\ifpdf
%\usepackage[pdftex]{graphicx}
%\else
%\usepackage{graphicx}
%\fi

 % \ifpdf
%  \usepackage{pdfsync}
%  \if


%\title{Brief Article}
%\author{David F. Snyder}
%\author{L.G. Meredith}

%\address{Dept. of Math., Texas State University--San Marcos, San Marcos, TX 78666}
       
\pagestyle{empty}


\begin{document}

\lstset{language=[Objective]Caml,frame=shadowbox}

\input{qm2pi.front}

% section front matter (end)

\input{qm2pi.intro} 
 
% section introduction (end)

% \input{qm2pi.knotations} 

% section notation (end)

\input{qm2pi.process.calculi} 

% section concurrent_process_calculi_and_spatial_logics_ (end)
    
%\input{qm2pi.knots2pi} 

%\input{qm2pi.trefoil} 

%\input{qm2pi.mainthm} 

% subsection basic_interpretation (end)

%\input{qm2pi.rho.presentation} 
\subsection{The syntax and semantics of the notation system}\label{sub:the_syntax_and_semantics_of_the_notation_system} % (fold)

We now summarize a technical presentation of the calculus that
embodies our theory of dynamics. The typical presentation of such a
calculus follows the style of giving generators and relations on
them. The grammar, below, describing term constructors, freely
generates the set of processes, $\Proc$. This set is then quotiented
by a relation known as structural congruence and it is over this set
that the notion of dynamics is expressed. This presentation is
essentially that of \cite{MeredithR05} with the addition of
polyadicity and summation. For readability we have relegated some of
the technical subtleties to an appendix.

\subsubsection{Process grammar}\label{subsub:process_grammar}

\begin{mathpar}
  \inferrule* [lab=synchronization] {} {{M} \bc \pzero \;|\; x?F \;|\; x!C }
  \and
  \inferrule* [lab=abstraction] {} {{F} \bc (x)P}
  \and
  \inferrule* [lab=concretion] {} {{C} \bc \langle Q \rangle}
  \and
  \inferrule* [lab=process] {} {{P,Q} \bc M \;| \;P|Q \;|\; @{x}}
  \and
  \inferrule* [lab=name] {} {{x} \bc \quotep{P}}
\end{mathpar} 

Note that $\vec{x}$ (resp. $\vec{P}$) denotes a vector of names
(resp. processes) of length $|\vec{x}|$ (resp. $|\vec{P}|$). We adopt
the following useful abbreviations.

\begin{mathpar}
   x?(\vec{y}).P := x.(\vec{y})P \and  x\clift{\vec{P}} := x.\clift{\vec{P}}
   \and x!(y) := \lift{x}{\dropn{y}}
   \and \Pi_{i=0}^{n-1}P_i := P_0 | \ldots | P_{n-1}
\end{mathpar}

\subsubsection{Structural congruence}

\paragraph{Free and bound names and alpha-equivalence.} At the
core of structural equivalence is alpha-equivalence which identifies
process that are the same up to a change of variable. Formally, we
recognize the distinction between free and bound names. The free names
of a process, $\freenames{P}$, may be calculated recursively as
follows:

\begin{mathpar}
\freenames{\pzero} := \emptyset
  \and \\
  \freenames{x?(y).P} := \{ x \} \cup (\freenames{P} \setminus \{ y \})
  \and 
  \freenames{x!\langle P \rangle} := \{ x \} \cup \{ P \} 
  \and \\
  \freenames{P|Q} := \freenames{P} \cup \freenames{Q}
  \and \\
  \freenames{@{x}} := \{ x \}
\end{mathpar}

$\pi$
$\quotep{\pi}$

$\freenames{-} : \pi \to \mathcal{P}(\quotep{\pi})$

\begin{eqnarray*}
  \freenames{\pzero} & := & \emptyset \\
  \freenames{x?(y).P} & := & \{ x \} \cup (\freenames{P} \setminus \{ y \}) \\
  \freenames{x!\langle P \rangle} & := & \{ x \} \cup \{ P \} \\
  \freenames{P|Q} & := & \freenames{P} \cup \freenames{Q} \\
  \freenames{\dropn{x}} & := & \{ x \}
\end{eqnarray*}

The bound names of a process, $\boundnames{P}$, are those names occurring in $P$
that are not free. For example, in $x?(y).0$, the name $x$ is free, while $y$ is bound.

\begin{mathpar}
  \inferrule* [lab=monoidal-laws] {} { P|Q \equiv Q|P \and P|0 \equiv P \and P|(Q|R) \equiv (P|Q)|R }
\end{mathpar}

\begin{mathpar}
  \inferrule* [lab=alpha-equivalence] {} { (x)P \equiv (y)P\{y/x\} \and y \not\in \freenames{P} }
\end{mathpar}

\begin{definition}
Then two processes, $P,Q$, are alpha-equivalent if $P = Q\{\vec{y}/\vec{x}\}$ for
some $\vec{x} \in \boundnames{Q},\vec{y} \in \boundnames{P}$, where $Q\{\vec{y}/\vec{x}\}$
denotes the capture-avoiding substitution of $\vec{y}$ for $\vec{x}$ in $Q$.
\end{definition}

\begin{definition}
  The {\em structural congruence} \cite{SangiorgiWalker} , $\equiv$,
  between processes is the least congruence containing
  alpha-equivalence, satisfying the abelian monoid laws
  (associativity, commutativity and $\pzero$ as identity) for parallel
  composition $|$ and for summation $+$.
\end{definition}

\subsection{Name equivalence}

We take name equivalence, written $\nameeq$, to be the smallest
equivalence relation generated by the following rules.

\begin{mathpar}
\inferrule*[lab=Quote-drop]
{ }
{ \quotep{@{x}} \nameeq x }

\inferrule*[lab=Struct-equiv]
{ P \scong Q }
{ \quotep{P} \nameeq \quotep{Q} }
\end{mathpar}

The astute reader will have noticed that the mutual recursion of names
and processes imposes a mutual recursion on alpha-equivalence and
structural equivalence via name-equivalence. Fortunately, all of this
works out pleasantly and we may calculate in the natural way, free of
concern. The reader interested in the details is referred to the
appendix \ref{appendix:rho_details}.

\subsection{Substitution}

We use $\Proc$ for the set of processes, $\QProc$ for the set of
names, and $\id{\{}\vec{y} / \vec{x} \id{\}}$ to denote partial maps,
$s : \QProc \rightarrow \QProc$. A map, $s$ lifts, uniquely, to a map
on process terms, $\widehat{s} : \Proc \rightarrow \Proc$ by the
following equations.

\begin{mathpar}
  (0) \psubstp{Q}{P} := 0 \\
  (R \juxtap S) \psubstp{Q}{P}
  :=    
  (R)\psubstp{Q}{P} \juxtap (S) \psubstp{Q}{P} \\
  (x?(y).R) \psubstp{Q}{P}    
  :=    
  (x)\substp{Q}{P} (z)\concat( (R \psubstn{z}{y}) \psubstp{Q}{P} ) \\
  (\lift{x}{R}) \psubstp{Q}{P}  
  :=
  \lift{(x)\substp{Q}{P}}{ R \psubstp{Q}{P} } \\
%   (\dropn{x})  \psubstp{Q}{P}       
%   := 
%   \left\{ 
%     \begin{array}{ccc} 
%       \dropn{\quotep{Q}} & & x \nameeq \quotep{P} \\
%       \dropn{x} & & otherwise \\
%     \end{array}
%   \right. 
  (\dropn{x})  \psubstp{Q}{P}       
  := 
  \left\{ 
    \begin{array}{ccc} 
      Q & & x \nameeq \quotep{P} \\
      \dropn{x} & & otherwise \\
    \end{array}
  \right.
\end{mathpar}
 

where

\begin{eqnarray}
  (x)\id{\{} \lpquote Q \rpquote / \lpquote P \rpquote \id{\}}            = 
  \left\{ 
    \begin{array}{ccc}
      \lpquote Q \rpquote & & x \nameeq \lpquote P \rpquote \\
      x & & otherwise \\
    \end{array}
  \right. \nonumber
\end{eqnarray}

and $z$ is chosen distinct from $\quotep{P}$, $\quotep{Q}$, the free
names in $Q$, and all the names in $R$. Our $\alpha$-equivalence will
be built in the standard way from this substitution.

\begin{remark}\label{rem:no_self_referential_names}
  One consequence of these definitions is that $\forall P. \quotep{P}
  \not\in \freenames{P}$.
\end{remark}

\subsection{ Dynamic quote: an example }

Anticipating something of what's to come, consider applying the
substitution, $\widehat{\id{\{}u / z \id{\}}}$, to the following pair
of processes, $\lift{w}{y!(z)}$ and $w[ \lpquote y!(z) \rpquote ]$.

\begin{eqnarray}
	\lift{w}{y!(z)}\widehat{\id{\{}u / z \id{\}}}
		& = &
		\lift{w}{y!(u)} \nonumber\\
	w[ \lpquote y!(z) \rpquote ] \widehat{ \id{\{}u / z \id{\}} }
		& = &
		w[ \lpquote y!(z) \rpquote ] \nonumber
\end{eqnarray}

Because the body of the process between quotes is impervious to
substitution, we get radically different answers. In fact, by
examining the first process in an input context,
e.g. $x?(z).\lift{w}{y!(z)}$, we see that the process under the lift
operator may be shaped by prefixed inputs binding a name inside it. In
this sense, the lift operator will be seen as a way to dynamically
construct processes before reifying them as names.

Finally equipped with these standard features we can present the
dynamics of the calculus.

\subsubsection{Operational semantics} 

Finally, we introduce the computational dynamics. What marks these
algebras as distinct from other more traditionally studied algebraic
structures, e.g. vector spaces or polynomial rings, is the manner in
which dynamics is captured. In traditional structures, dynamics is typically
expressed through morphisms between such structures, as in linear maps
between vector spaces or morphisms between rings. In algebras
associated with the semantics of computation, the dynamics is
expressed as part of the algebraic structure itself, through a
reduction reduction relation typically denoted by $\red$. Below, we
give a recursive presentation of this relation for the calculus used
in the encoding.

$\red \subseteq \pi \times \pi$
$\red : \pi \to \mathcal{P}(\pi)$

\begin{mathpar}
  \inferrule* [lab=Comm] { \textsf{match}( x_{src}, x_{trgt} ) } { x_{trgt}?(y)P \; | \; x_{src}!\langle {Q} \rangle \red P\{\quotep{Q}/y}\} }
  \and \\
  \inferrule* [lab=Par] {{P} \red {P}'} {{{P} | {Q}} \red {{P}' | {Q}}}
  \and
  \inferrule* [lab=Equiv]{{{P} \scong {P}'} \andalso {{P}' \red {Q}'} \andalso {{Q}' \scong {Q}}}{{P} \red {Q}}
\end{mathpar}

\begin{eqnarray*}
  match_{\equiv} (\quotep{P},\quotep{Q}) & := & P \equiv Q \\
  match_{\dagger}(\quotep{P},\quotep{Q}) & := & \forall R. P|Q \red^{*} R => R \red^{*} 0 \\
  match_{K}(\quotep{P},\quotep{Q}) & := & K \mbox{ for some context } K
\end{eqnarray*}

$u?(x)P | u!\langle Q \rangle \red P\{\quotep{Q}/x\}$

%We write $\wred$ for $\red^*$, and $P\red$ if $\exists Q $ such that $ P \red Q$.
We write $P\red$ if $\exists Q $ such that $ P \red Q$ and $P\not\red$, otherwise.

\section{Replication}

As mentioned before, it is known that replication (and hence
recursion) can be implemented in a higher-order process algebra
\cite{SangiorgiWalker}. As our first example of calculation with the
machinery thus far presented we give the construction explicitly in
the {\rhoc}.

\begin{eqnarray}
	D_{x} & := & \prefix{x}{y}{(\binpar{\outputp{x}{y}}{@{y}})} \nonumber\\
	\bangp_{x}{P} & := & \binpar{{x}!\langle{\binpar{D_{x}}{P}}\rangle}{D_{x}} \nonumber
\end{eqnarray}

\begin{eqnarray}
	\bangp_{x}{P} & & \nonumber\\
	=
	& {x}!\langle{(\prefix{x}{y}{(\outputp{x}{y} | @{y})) | P}}\rangle 
	      | \prefix{x}{y}{(\outputp{x}{y} | @{y})} & \nonumber\\
	\red
	& (\outputp{x}{y} | @{y})\substn{\quotep{(\prefix{x}{y}{(@{y} | \outputp{x}{y})) | P}}}{y} & \nonumber\\
	=
	& \outputp{x}{\quotep{(\prefix{x}{y}{(\outputp{x}{y} | @{y})) | P}}}
	  | {(\prefix{x}{y}{(\outputp{x}{y} | @{y})) | P}} & \nonumber\\
	\red
	& \ldots & \nonumber\\
	\red^*
	& P | P | \ldots & \nonumber
\end{eqnarray}

Of course, this encoding, as an implementation, runs away, unfolding
$\bangp{P}$ eagerly. A lazier and more implementable replication
operator, restricted to input-guarded processes, may be obtained as follows.

\begin{eqnarray}
\bangp{\prefix{u}{v}{P}} 
	:= 
	\binpar{\lift{x}{\prefix{u}{v}{(\binpar{D(x)}{P})}}}{D(x)} \nonumber
\end{eqnarray}

\begin{remark}
  Note that the lazier definition still does not deal with summation
  or mixed summation (i.e. sums over input and output). The reader is
  invited to construct definitions of replication that deal with these
  features. 

  Further, the definitions are parameterized in a name, $x$. Can you,
  gentle reader, make a definition that eliminates this parameter and
  guarantees no accidental interaction between the replication
  machinery and the process being replicated -- i.e. no accidental
  sharing of names used by the process to get its work done and the
  name(s) used by the replication to effect copying. This latter
  revision of the definition of replication is crucial to obtaining
  the expected identity $!!P \sim !P$.
\end{remark}

\begin{remark}\label{rem:paradoxical_combinator}
  The reader familiar with the lambda calculus will have noticed the
  similarity between $D$ and the paradoxical combinator.

  [Ed. note: the existence of this seems to suggest we have to be more
  restrictive on the set of processes and names we admit if we are to
  support no-cloning.]
\end{remark}

\subsubsection{Bisimulation}

The computational dynamics gives rise to another kind of equivalence,
the equivalence of computational behavior. As previously mentioned
this is typically captured \emph{via} some form of bisimulation.

% The notion we use in this paper is weak barbed bisimulation
% \cite{milner91polyadicpi}.

The notion we use in this paper is derived from weak barbed
bisimulation \cite{milner91polyadicpi}. 

\begin{definition}
An \emph{observation relation}, $\downarrow_{\mathcal N}$, over a set
of names, $\mathcal N$, is the smallest relation satisfying the rules
below.

\infrule[Out-barb]{y \in {\mathcal N}, \; x \nameeq y}
		  {\outputp{x}{v} \downarrow_{\mathcal N} x}
\infrule[Par-barb]{\mbox{$P\downarrow_{\mathcal N} x$ or $Q\downarrow_{\mathcal N} x$}}
		  {\binpar{P}{Q} \downarrow_{\mathcal N} x}

We write $P \Downarrow_{\mathcal N} x$ if there is $Q$ such that 
$P \wred Q$ and $Q \downarrow_{\mathcal N} x$.
\end{definition}

\begin{definition}
%\label{def.bbisim}
An  ${\mathcal N}$-\emph{barbed bisimulation} over a set of names, ${\mathcal N}$, is a symmetric binary relation 
${\mathcal S}_{\mathcal N}$ between agents such that $P\rel{S}_{\mathcal N}Q$ implies:
\begin{enumerate}
\item If $P \red P'$ then $Q \wred Q'$ and $P'\rel{S}_{\mathcal N} Q'$.
\item If $P\downarrow_{\mathcal N} x$, then $Q\Downarrow_{\mathcal N} x$.
\end{enumerate}
$P$ is ${\mathcal N}$-barbed bisimilar to $Q$, written
$P \wbbisim_{\mathcal N} Q$, if $P \rel{S}_{\mathcal N} Q$ for some ${\mathcal N}$-barbed bisimulation ${\mathcal S}_{\mathcal N}$.
\end{definition}

$\mathcal{R} \subseteq \pi \times \pi$

$P \mathcal{R} Q => \forall P'. P \red P' \Rightarrow \exists Q'. Q \red Q', P' \mathcal{R} Q'$

$P \vdash x \Rightarrow Q \vdash x$

\begin{mathpar}
  \inferrule*[lab=Out-barb]{x \nameeq y}{{y}!\langle{Q}\rangle \vdash x}
  \and
  \inferrule*[lab=Par-barb]{\mbox{$P\vdash x$ or $Q\vdash x$}}{\binpar{P}{Q} \vdash x}
\end{mathpar}

\subsubsection{Contexts}

One of the principle advantages of computational calculi like the
$\pi$-calculus is a well-defined notion of context,
contextual-equivalence and a correlation between
contextual-equivalence and notions of bisimulation. The notion of
context allows the decomposition of a process into (sub-)process and
its syntactic environment, its context. Thus, a context may be
thought of as a process with a ``hole'' (written $\Box$) in it. The
application of a context $M$ to a process $P$, written $M[P]$, is
tantamount to filling the hole in $M$ with $P$. In this paper we do
not need the full weight of this theory, but do make use of the notion
of context in the proof the main theorem. 

\begin{mathpar}
  \inferrule* [lab=summation] {} {{M_{M},M_{N}} \bc \Box \;|\; x.M_{A} \;|\; M_{M}+M_{N}}
  \and
  \inferrule* [lab=agent] {} {{M_{A}} \bc (\vec{x})M_{P} \;| \; \clift{P_0,\ldots,M_{P},\ldots,P_N}}
  \and \\
  \inferrule* [lab=process] {} {{M_{P}} \bc M_{N} \;| \;P|M_{P} }
\end{mathpar} 

\begin{mathpar}
  \inferrule* [lab=sychronization] {} {M_{N} \bc \Box \;|\; x?M_{F} \;|\; x!M_{C}}
  \and
  \inferrule* [lab=abstraction] {} {{M_{F}} \bc (x)M_{P} }
  \and
  \inferrule* [lab=concretion] {} {{M_{C}} \bc \langle M_{P} \rangle }
  \and \\
  \inferrule* [lab=process] {} {{M_{P}} \bc M_{N} \;| \;P|M_{P} }
\end{mathpar}

\begin{definition}[contextual application] Given a context $M$, and
  process $P$, we define the \emph{contextual application}, $M[P] :=
  M\{P/\Box\}$. That is, the contextual application of M to P is the
  substitution of $P$ for $\Box$ in $M$.
\end{definition}

$\meaningof{-} : L \to \mathcal{P}(\pi)$

\begin{mathpar}
  \inferrule* [lab=collection] {} {\meaningof{true} = \pi, \and \meaningof{~E} = \pi \setminus \meaningof{E}, \and \meaningof{E_{1} \& E_{2}} = \meaningof{E_{1}} \cap \meaningof{E_{2}}}
\end{mathpar}

\begin{mathpar}
  \inferrule* [lab=structure] {} {\meaningof{0} = \{ P \in \pi | P \equiv 0 \}, \and \\ \meaningof{E_1 | E_2} = \{ P \in \pi | P \equiv P_{1} | P_{2}, P_{1} \in \meaningof{E_{1}}, P_{2} \in \meaningof{E_2}\} }
\end{mathpar}

\begin{mathpar}
 \inferrule* [lab=behavior] {} {\meaningof{\langle a?b \rangle E} = \{ P \in \pi | P \equiv Q | u?(y)P', \\ \and \\\\ \and \\ \;\;\; u \in \meaningof{a}, \forall z.P'\{z/y\} \in \meaningof{E\{z/b\}}\}, \and \\ \meaningof{a!E} = \{ P \in \pi | P \equiv Q | x!\langle P' \rangle, x \in \meaningof{a} P' \in \meaningof{E}\} }
\end{mathpar}

\begin{mathpar}
 \inferrule* [lab=nominal] {} {\meaningof{\quotep{E}} = \{ \quotep{P} \in \quotep{\pi} | P \in \meaningof{E} \}, \and \meaningof{\quotep{P}} = \{ \quotep{Q} \in \quotep{\pi} | P \equiv Q \} \and \\ \meaningof{@\quotep{E}} = \{ P \in \pi | P \equiv @x, x \in \meaningof{E} \}}
\end{mathpar}

\begin{eqnarray*}
  \\
  \meaningof{-} : TS \to ST
\end{eqnarray*}

\begin{eqnarray*}
  \\
  L : TS \to ST
\end{eqnarray*}

\begin{eqnarray*}
  \\
  P \models E \iff P \in \meaningof{E}
\end{eqnarray*}

\begin{eqnarray*}
  P \approx_{L} Q \iff \forall E \in L. P \models E \iff Q \models E
\end{eqnarray*}

\begin{eqnarray*}
  P \approx_{K} Q
\end{eqnarray*}

\begin{eqnarray*}
  P \approx Q
\end{eqnarray*}

$\approx_{K} = \approx = \approx_{L}$

\subsubsection{Contextual duality}

Note that contexts extend the quotation operation to a family of
operations from processes to names. Given a context, $M$, we can
define a \emph{nominal context}, $\quotep{M}$ by $\quotep{M}[P] :=
\quotep{M[P]}$. To foreshadow what is to come we observe that these
operations enjoy a duality with processes very much like the duality
between vectors and maps from vectors to scalars.

Further, because the calculus is essentially higher-order, we have a
correspondence between contexts and processes. More specifically,
given a name $x$ and a context $M$ we can construct $M^{*}_{x}$ such
that 

\begin{mathpar}
  M^{*}_{x} | \lift{x}{P} \red M[P]
\end{mathpar}

namely,

\begin{mathpar}
  M^{*}_{x} := x?(u).M[\dropn{u}]
\end{mathpar}

The dependence of $M^{*}_{x}$ on a name makes it an abstraction, 

\begin{mathpar}
  M^{*} := (x)x?(u).M[\dropn{u}]
\end{mathpar}

\subsection{Additional notation}

It will sometimes be convenient to denote the process a name
quotes. We already have the notation $x = \quotep{P}$, but it will be
convenient to introduce an alternate notation, $\procn{x}$, when we
want to emphasize the connection to the use of the name. Note that, by
virtue of name equivalence, $\quotep{\procn{x}} \nameeq x$; so, the
notation is consistent with previous definitions.

Further, because names have structure it is possible to effect
substitutions on the basis of that structure. This means we need to
upgrade our notation for substitutions, which we accomplish by
adapting comprehension notation. Thus,

\begin{mathpar}
  P\{ y / x : x \in S \}
\end{mathpar}

is interpreted to mean the process derived from P by replacing (in a
capture-avoiding manner) each occurrence of $x$ in $S$ by $y$. For example,

\begin{mathpar}
  P\{ \quotep{\procn{x}|\procn{x}} / x : x \in \freenames{P} \}
\end{mathpar}

will replace each (occurrence) of a free name $x$ in $P$ by
$\quotep{\procn{x}|\procn{x}}$.

Also, we will avail ourselves of the notation $x^{L}$ and $x^{R}$ to
denote injections of a name into disjoint copies of the name
space. There are numerous ways to accomplish this. One example can be
found in \cite{MeredithR05}. This notation overloads to vectors of
names: $\vec{x}^{\pi} := (x_{i}^{\pi} \; : \; 0 \leq i < |\vec{x}| )$ where $\pi \in \{L,R\}$.

We also use $P^{\Box} := P|\Box$.

In \cite{MeredithR05} an interpretation of the new operator is
given. It turns out that there are several possible interpretations
all enjoying the requisite algebraic properties of the operator (see
\cite{milner91polyadicpi}). We will therefore make liberal use of
$(\nu\; \vec{x})P$.

% subsection the_syntax_and_semantics_of_the_notation_system (end)   

\input{qm2pi.qmops} 

\input{qm2pi.sterngerlach} 

\input{qm2pi.metric} 

% section concurrent_process_calculi (end)

%\input{qm2pi.proofsketch}

% section proof sketch (end)

%\input{qm2pi.slviaknots} 

% section spatial logic via knots (end)

\input{qm2pi.conclusion}

% section conclusion (end)

%\input{qm2pi.dtcodes} 

% section wiring algorithm (end)

\input{qm2pi.ack} 

% section acknowledgments (end)

\newpage


\bibliographystyle{plain}   
\bibliography{../../biblios/main.bib}

\input{qm2pi.rhodetails}

\end{document}

 

% section notation (end)

\input{qm2pi.process.calculi} 

% section concurrent_process_calculi_and_spatial_logics_ (end)
    
%\documentclass[12pt]{llncs}
%\documentclass{jktr}

\usepackage[pdftex]{hyperref}                   
\usepackage {listings}
\usepackage {mathpartir}
\usepackage{bcprules}
%\usepackage{listings}
                       
\usepackage{graphicx} 
%\usepackage[margins=2.5cm,nohead,nofoot]{geometry}
%\usepackage{geometry}
\usepackage{amsfonts}
\usepackage{amstext}
\usepackage{latexsym}
\usepackage{amssymb}
\usepackage{color}


%\include{myPreamble}
\include{qm2pi.local} 

%\ifpdf
%\usepackage[pdftex]{graphicx}
%\else
%\usepackage{graphicx}
%\fi

 % \ifpdf
%  \usepackage{pdfsync}
%  \if


%\title{Brief Article}
%\author{David F. Snyder}
%\author{L.G. Meredith}

%\address{Dept. of Math., Texas State University--San Marcos, San Marcos, TX 78666}
       
\pagestyle{empty}


\begin{document}

\lstset{language=[Objective]Caml,frame=shadowbox}

\input{qm2pi.front}

% section front matter (end)

\input{qm2pi.intro} 
 
% section introduction (end)

% \input{qm2pi.knotations} 

% section notation (end)

\input{qm2pi.process.calculi} 

% section concurrent_process_calculi_and_spatial_logics_ (end)
    
%\input{qm2pi.knots2pi} 

%\input{qm2pi.trefoil} 

%\input{qm2pi.mainthm} 

% subsection basic_interpretation (end)

%\input{qm2pi.rho.presentation} 
\subsection{The syntax and semantics of the notation system}\label{sub:the_syntax_and_semantics_of_the_notation_system} % (fold)

We now summarize a technical presentation of the calculus that
embodies our theory of dynamics. The typical presentation of such a
calculus follows the style of giving generators and relations on
them. The grammar, below, describing term constructors, freely
generates the set of processes, $\Proc$. This set is then quotiented
by a relation known as structural congruence and it is over this set
that the notion of dynamics is expressed. This presentation is
essentially that of \cite{MeredithR05} with the addition of
polyadicity and summation. For readability we have relegated some of
the technical subtleties to an appendix.

\subsubsection{Process grammar}\label{subsub:process_grammar}

\begin{mathpar}
  \inferrule* [lab=synchronization] {} {{M} \bc \pzero \;|\; x?F \;|\; x!C }
  \and
  \inferrule* [lab=abstraction] {} {{F} \bc (x)P}
  \and
  \inferrule* [lab=concretion] {} {{C} \bc \langle Q \rangle}
  \and
  \inferrule* [lab=process] {} {{P,Q} \bc M \;| \;P|Q \;|\; @{x}}
  \and
  \inferrule* [lab=name] {} {{x} \bc \quotep{P}}
\end{mathpar} 

Note that $\vec{x}$ (resp. $\vec{P}$) denotes a vector of names
(resp. processes) of length $|\vec{x}|$ (resp. $|\vec{P}|$). We adopt
the following useful abbreviations.

\begin{mathpar}
   x?(\vec{y}).P := x.(\vec{y})P \and  x\clift{\vec{P}} := x.\clift{\vec{P}}
   \and x!(y) := \lift{x}{\dropn{y}}
   \and \Pi_{i=0}^{n-1}P_i := P_0 | \ldots | P_{n-1}
\end{mathpar}

\subsubsection{Structural congruence}

\paragraph{Free and bound names and alpha-equivalence.} At the
core of structural equivalence is alpha-equivalence which identifies
process that are the same up to a change of variable. Formally, we
recognize the distinction between free and bound names. The free names
of a process, $\freenames{P}$, may be calculated recursively as
follows:

\begin{mathpar}
\freenames{\pzero} := \emptyset
  \and \\
  \freenames{x?(y).P} := \{ x \} \cup (\freenames{P} \setminus \{ y \})
  \and 
  \freenames{x!\langle P \rangle} := \{ x \} \cup \{ P \} 
  \and \\
  \freenames{P|Q} := \freenames{P} \cup \freenames{Q}
  \and \\
  \freenames{@{x}} := \{ x \}
\end{mathpar}

$\pi$
$\quotep{\pi}$

$\freenames{-} : \pi \to \mathcal{P}(\quotep{\pi})$

\begin{eqnarray*}
  \freenames{\pzero} & := & \emptyset \\
  \freenames{x?(y).P} & := & \{ x \} \cup (\freenames{P} \setminus \{ y \}) \\
  \freenames{x!\langle P \rangle} & := & \{ x \} \cup \{ P \} \\
  \freenames{P|Q} & := & \freenames{P} \cup \freenames{Q} \\
  \freenames{\dropn{x}} & := & \{ x \}
\end{eqnarray*}

The bound names of a process, $\boundnames{P}$, are those names occurring in $P$
that are not free. For example, in $x?(y).0$, the name $x$ is free, while $y$ is bound.

\begin{mathpar}
  \inferrule* [lab=monoidal-laws] {} { P|Q \equiv Q|P \and P|0 \equiv P \and P|(Q|R) \equiv (P|Q)|R }
\end{mathpar}

\begin{mathpar}
  \inferrule* [lab=alpha-equivalence] {} { (x)P \equiv (y)P\{y/x\} \and y \not\in \freenames{P} }
\end{mathpar}

\begin{definition}
Then two processes, $P,Q$, are alpha-equivalent if $P = Q\{\vec{y}/\vec{x}\}$ for
some $\vec{x} \in \boundnames{Q},\vec{y} \in \boundnames{P}$, where $Q\{\vec{y}/\vec{x}\}$
denotes the capture-avoiding substitution of $\vec{y}$ for $\vec{x}$ in $Q$.
\end{definition}

\begin{definition}
  The {\em structural congruence} \cite{SangiorgiWalker} , $\equiv$,
  between processes is the least congruence containing
  alpha-equivalence, satisfying the abelian monoid laws
  (associativity, commutativity and $\pzero$ as identity) for parallel
  composition $|$ and for summation $+$.
\end{definition}

\subsection{Name equivalence}

We take name equivalence, written $\nameeq$, to be the smallest
equivalence relation generated by the following rules.

\begin{mathpar}
\inferrule*[lab=Quote-drop]
{ }
{ \quotep{@{x}} \nameeq x }

\inferrule*[lab=Struct-equiv]
{ P \scong Q }
{ \quotep{P} \nameeq \quotep{Q} }
\end{mathpar}

The astute reader will have noticed that the mutual recursion of names
and processes imposes a mutual recursion on alpha-equivalence and
structural equivalence via name-equivalence. Fortunately, all of this
works out pleasantly and we may calculate in the natural way, free of
concern. The reader interested in the details is referred to the
appendix \ref{appendix:rho_details}.

\subsection{Substitution}

We use $\Proc$ for the set of processes, $\QProc$ for the set of
names, and $\id{\{}\vec{y} / \vec{x} \id{\}}$ to denote partial maps,
$s : \QProc \rightarrow \QProc$. A map, $s$ lifts, uniquely, to a map
on process terms, $\widehat{s} : \Proc \rightarrow \Proc$ by the
following equations.

\begin{mathpar}
  (0) \psubstp{Q}{P} := 0 \\
  (R \juxtap S) \psubstp{Q}{P}
  :=    
  (R)\psubstp{Q}{P} \juxtap (S) \psubstp{Q}{P} \\
  (x?(y).R) \psubstp{Q}{P}    
  :=    
  (x)\substp{Q}{P} (z)\concat( (R \psubstn{z}{y}) \psubstp{Q}{P} ) \\
  (\lift{x}{R}) \psubstp{Q}{P}  
  :=
  \lift{(x)\substp{Q}{P}}{ R \psubstp{Q}{P} } \\
%   (\dropn{x})  \psubstp{Q}{P}       
%   := 
%   \left\{ 
%     \begin{array}{ccc} 
%       \dropn{\quotep{Q}} & & x \nameeq \quotep{P} \\
%       \dropn{x} & & otherwise \\
%     \end{array}
%   \right. 
  (\dropn{x})  \psubstp{Q}{P}       
  := 
  \left\{ 
    \begin{array}{ccc} 
      Q & & x \nameeq \quotep{P} \\
      \dropn{x} & & otherwise \\
    \end{array}
  \right.
\end{mathpar}
 

where

\begin{eqnarray}
  (x)\id{\{} \lpquote Q \rpquote / \lpquote P \rpquote \id{\}}            = 
  \left\{ 
    \begin{array}{ccc}
      \lpquote Q \rpquote & & x \nameeq \lpquote P \rpquote \\
      x & & otherwise \\
    \end{array}
  \right. \nonumber
\end{eqnarray}

and $z$ is chosen distinct from $\quotep{P}$, $\quotep{Q}$, the free
names in $Q$, and all the names in $R$. Our $\alpha$-equivalence will
be built in the standard way from this substitution.

\begin{remark}\label{rem:no_self_referential_names}
  One consequence of these definitions is that $\forall P. \quotep{P}
  \not\in \freenames{P}$.
\end{remark}

\subsection{ Dynamic quote: an example }

Anticipating something of what's to come, consider applying the
substitution, $\widehat{\id{\{}u / z \id{\}}}$, to the following pair
of processes, $\lift{w}{y!(z)}$ and $w[ \lpquote y!(z) \rpquote ]$.

\begin{eqnarray}
	\lift{w}{y!(z)}\widehat{\id{\{}u / z \id{\}}}
		& = &
		\lift{w}{y!(u)} \nonumber\\
	w[ \lpquote y!(z) \rpquote ] \widehat{ \id{\{}u / z \id{\}} }
		& = &
		w[ \lpquote y!(z) \rpquote ] \nonumber
\end{eqnarray}

Because the body of the process between quotes is impervious to
substitution, we get radically different answers. In fact, by
examining the first process in an input context,
e.g. $x?(z).\lift{w}{y!(z)}$, we see that the process under the lift
operator may be shaped by prefixed inputs binding a name inside it. In
this sense, the lift operator will be seen as a way to dynamically
construct processes before reifying them as names.

Finally equipped with these standard features we can present the
dynamics of the calculus.

\subsubsection{Operational semantics} 

Finally, we introduce the computational dynamics. What marks these
algebras as distinct from other more traditionally studied algebraic
structures, e.g. vector spaces or polynomial rings, is the manner in
which dynamics is captured. In traditional structures, dynamics is typically
expressed through morphisms between such structures, as in linear maps
between vector spaces or morphisms between rings. In algebras
associated with the semantics of computation, the dynamics is
expressed as part of the algebraic structure itself, through a
reduction reduction relation typically denoted by $\red$. Below, we
give a recursive presentation of this relation for the calculus used
in the encoding.

$\red \subseteq \pi \times \pi$
$\red : \pi \to \mathcal{P}(\pi)$

\begin{mathpar}
  \inferrule* [lab=Comm] { \textsf{match}( x_{src}, x_{trgt} ) } { x_{trgt}?(y)P \; | \; x_{src}!\langle {Q} \rangle \red P\{\quotep{Q}/y}\} }
  \and \\
  \inferrule* [lab=Par] {{P} \red {P}'} {{{P} | {Q}} \red {{P}' | {Q}}}
  \and
  \inferrule* [lab=Equiv]{{{P} \scong {P}'} \andalso {{P}' \red {Q}'} \andalso {{Q}' \scong {Q}}}{{P} \red {Q}}
\end{mathpar}

\begin{eqnarray*}
  match_{\equiv} (\quotep{P},\quotep{Q}) & := & P \equiv Q \\
  match_{\dagger}(\quotep{P},\quotep{Q}) & := & \forall R. P|Q \red^{*} R => R \red^{*} 0 \\
  match_{K}(\quotep{P},\quotep{Q}) & := & K \mbox{ for some context } K
\end{eqnarray*}

$u?(x)P | u!\langle Q \rangle \red P\{\quotep{Q}/x\}$

%We write $\wred$ for $\red^*$, and $P\red$ if $\exists Q $ such that $ P \red Q$.
We write $P\red$ if $\exists Q $ such that $ P \red Q$ and $P\not\red$, otherwise.

\section{Replication}

As mentioned before, it is known that replication (and hence
recursion) can be implemented in a higher-order process algebra
\cite{SangiorgiWalker}. As our first example of calculation with the
machinery thus far presented we give the construction explicitly in
the {\rhoc}.

\begin{eqnarray}
	D_{x} & := & \prefix{x}{y}{(\binpar{\outputp{x}{y}}{@{y}})} \nonumber\\
	\bangp_{x}{P} & := & \binpar{{x}!\langle{\binpar{D_{x}}{P}}\rangle}{D_{x}} \nonumber
\end{eqnarray}

\begin{eqnarray}
	\bangp_{x}{P} & & \nonumber\\
	=
	& {x}!\langle{(\prefix{x}{y}{(\outputp{x}{y} | @{y})) | P}}\rangle 
	      | \prefix{x}{y}{(\outputp{x}{y} | @{y})} & \nonumber\\
	\red
	& (\outputp{x}{y} | @{y})\substn{\quotep{(\prefix{x}{y}{(@{y} | \outputp{x}{y})) | P}}}{y} & \nonumber\\
	=
	& \outputp{x}{\quotep{(\prefix{x}{y}{(\outputp{x}{y} | @{y})) | P}}}
	  | {(\prefix{x}{y}{(\outputp{x}{y} | @{y})) | P}} & \nonumber\\
	\red
	& \ldots & \nonumber\\
	\red^*
	& P | P | \ldots & \nonumber
\end{eqnarray}

Of course, this encoding, as an implementation, runs away, unfolding
$\bangp{P}$ eagerly. A lazier and more implementable replication
operator, restricted to input-guarded processes, may be obtained as follows.

\begin{eqnarray}
\bangp{\prefix{u}{v}{P}} 
	:= 
	\binpar{\lift{x}{\prefix{u}{v}{(\binpar{D(x)}{P})}}}{D(x)} \nonumber
\end{eqnarray}

\begin{remark}
  Note that the lazier definition still does not deal with summation
  or mixed summation (i.e. sums over input and output). The reader is
  invited to construct definitions of replication that deal with these
  features. 

  Further, the definitions are parameterized in a name, $x$. Can you,
  gentle reader, make a definition that eliminates this parameter and
  guarantees no accidental interaction between the replication
  machinery and the process being replicated -- i.e. no accidental
  sharing of names used by the process to get its work done and the
  name(s) used by the replication to effect copying. This latter
  revision of the definition of replication is crucial to obtaining
  the expected identity $!!P \sim !P$.
\end{remark}

\begin{remark}\label{rem:paradoxical_combinator}
  The reader familiar with the lambda calculus will have noticed the
  similarity between $D$ and the paradoxical combinator.

  [Ed. note: the existence of this seems to suggest we have to be more
  restrictive on the set of processes and names we admit if we are to
  support no-cloning.]
\end{remark}

\subsubsection{Bisimulation}

The computational dynamics gives rise to another kind of equivalence,
the equivalence of computational behavior. As previously mentioned
this is typically captured \emph{via} some form of bisimulation.

% The notion we use in this paper is weak barbed bisimulation
% \cite{milner91polyadicpi}.

The notion we use in this paper is derived from weak barbed
bisimulation \cite{milner91polyadicpi}. 

\begin{definition}
An \emph{observation relation}, $\downarrow_{\mathcal N}$, over a set
of names, $\mathcal N$, is the smallest relation satisfying the rules
below.

\infrule[Out-barb]{y \in {\mathcal N}, \; x \nameeq y}
		  {\outputp{x}{v} \downarrow_{\mathcal N} x}
\infrule[Par-barb]{\mbox{$P\downarrow_{\mathcal N} x$ or $Q\downarrow_{\mathcal N} x$}}
		  {\binpar{P}{Q} \downarrow_{\mathcal N} x}

We write $P \Downarrow_{\mathcal N} x$ if there is $Q$ such that 
$P \wred Q$ and $Q \downarrow_{\mathcal N} x$.
\end{definition}

\begin{definition}
%\label{def.bbisim}
An  ${\mathcal N}$-\emph{barbed bisimulation} over a set of names, ${\mathcal N}$, is a symmetric binary relation 
${\mathcal S}_{\mathcal N}$ between agents such that $P\rel{S}_{\mathcal N}Q$ implies:
\begin{enumerate}
\item If $P \red P'$ then $Q \wred Q'$ and $P'\rel{S}_{\mathcal N} Q'$.
\item If $P\downarrow_{\mathcal N} x$, then $Q\Downarrow_{\mathcal N} x$.
\end{enumerate}
$P$ is ${\mathcal N}$-barbed bisimilar to $Q$, written
$P \wbbisim_{\mathcal N} Q$, if $P \rel{S}_{\mathcal N} Q$ for some ${\mathcal N}$-barbed bisimulation ${\mathcal S}_{\mathcal N}$.
\end{definition}

$\mathcal{R} \subseteq \pi \times \pi$

$P \mathcal{R} Q => \forall P'. P \red P' \Rightarrow \exists Q'. Q \red Q', P' \mathcal{R} Q'$

$P \vdash x \Rightarrow Q \vdash x$

\begin{mathpar}
  \inferrule*[lab=Out-barb]{x \nameeq y}{{y}!\langle{Q}\rangle \vdash x}
  \and
  \inferrule*[lab=Par-barb]{\mbox{$P\vdash x$ or $Q\vdash x$}}{\binpar{P}{Q} \vdash x}
\end{mathpar}

\subsubsection{Contexts}

One of the principle advantages of computational calculi like the
$\pi$-calculus is a well-defined notion of context,
contextual-equivalence and a correlation between
contextual-equivalence and notions of bisimulation. The notion of
context allows the decomposition of a process into (sub-)process and
its syntactic environment, its context. Thus, a context may be
thought of as a process with a ``hole'' (written $\Box$) in it. The
application of a context $M$ to a process $P$, written $M[P]$, is
tantamount to filling the hole in $M$ with $P$. In this paper we do
not need the full weight of this theory, but do make use of the notion
of context in the proof the main theorem. 

\begin{mathpar}
  \inferrule* [lab=summation] {} {{M_{M},M_{N}} \bc \Box \;|\; x.M_{A} \;|\; M_{M}+M_{N}}
  \and
  \inferrule* [lab=agent] {} {{M_{A}} \bc (\vec{x})M_{P} \;| \; \clift{P_0,\ldots,M_{P},\ldots,P_N}}
  \and \\
  \inferrule* [lab=process] {} {{M_{P}} \bc M_{N} \;| \;P|M_{P} }
\end{mathpar} 

\begin{mathpar}
  \inferrule* [lab=sychronization] {} {M_{N} \bc \Box \;|\; x?M_{F} \;|\; x!M_{C}}
  \and
  \inferrule* [lab=abstraction] {} {{M_{F}} \bc (x)M_{P} }
  \and
  \inferrule* [lab=concretion] {} {{M_{C}} \bc \langle M_{P} \rangle }
  \and \\
  \inferrule* [lab=process] {} {{M_{P}} \bc M_{N} \;| \;P|M_{P} }
\end{mathpar}

\begin{definition}[contextual application] Given a context $M$, and
  process $P$, we define the \emph{contextual application}, $M[P] :=
  M\{P/\Box\}$. That is, the contextual application of M to P is the
  substitution of $P$ for $\Box$ in $M$.
\end{definition}

$\meaningof{-} : L \to \mathcal{P}(\pi)$

\begin{mathpar}
  \inferrule* [lab=collection] {} {\meaningof{true} = \pi, \and \meaningof{~E} = \pi \setminus \meaningof{E}, \and \meaningof{E_{1} \& E_{2}} = \meaningof{E_{1}} \cap \meaningof{E_{2}}}
\end{mathpar}

\begin{mathpar}
  \inferrule* [lab=structure] {} {\meaningof{0} = \{ P \in \pi | P \equiv 0 \}, \and \\ \meaningof{E_1 | E_2} = \{ P \in \pi | P \equiv P_{1} | P_{2}, P_{1} \in \meaningof{E_{1}}, P_{2} \in \meaningof{E_2}\} }
\end{mathpar}

\begin{mathpar}
 \inferrule* [lab=behavior] {} {\meaningof{\langle a?b \rangle E} = \{ P \in \pi | P \equiv Q | u?(y)P', \\ \and \\\\ \and \\ \;\;\; u \in \meaningof{a}, \forall z.P'\{z/y\} \in \meaningof{E\{z/b\}}\}, \and \\ \meaningof{a!E} = \{ P \in \pi | P \equiv Q | x!\langle P' \rangle, x \in \meaningof{a} P' \in \meaningof{E}\} }
\end{mathpar}

\begin{mathpar}
 \inferrule* [lab=nominal] {} {\meaningof{\quotep{E}} = \{ \quotep{P} \in \quotep{\pi} | P \in \meaningof{E} \}, \and \meaningof{\quotep{P}} = \{ \quotep{Q} \in \quotep{\pi} | P \equiv Q \} \and \\ \meaningof{@\quotep{E}} = \{ P \in \pi | P \equiv @x, x \in \meaningof{E} \}}
\end{mathpar}

\begin{eqnarray*}
  \\
  \meaningof{-} : TS \to ST
\end{eqnarray*}

\begin{eqnarray*}
  \\
  L : TS \to ST
\end{eqnarray*}

\begin{eqnarray*}
  \\
  P \models E \iff P \in \meaningof{E}
\end{eqnarray*}

\begin{eqnarray*}
  P \approx_{L} Q \iff \forall E \in L. P \models E \iff Q \models E
\end{eqnarray*}

\begin{eqnarray*}
  P \approx_{K} Q
\end{eqnarray*}

\begin{eqnarray*}
  P \approx Q
\end{eqnarray*}

$\approx_{K} = \approx = \approx_{L}$

\subsubsection{Contextual duality}

Note that contexts extend the quotation operation to a family of
operations from processes to names. Given a context, $M$, we can
define a \emph{nominal context}, $\quotep{M}$ by $\quotep{M}[P] :=
\quotep{M[P]}$. To foreshadow what is to come we observe that these
operations enjoy a duality with processes very much like the duality
between vectors and maps from vectors to scalars.

Further, because the calculus is essentially higher-order, we have a
correspondence between contexts and processes. More specifically,
given a name $x$ and a context $M$ we can construct $M^{*}_{x}$ such
that 

\begin{mathpar}
  M^{*}_{x} | \lift{x}{P} \red M[P]
\end{mathpar}

namely,

\begin{mathpar}
  M^{*}_{x} := x?(u).M[\dropn{u}]
\end{mathpar}

The dependence of $M^{*}_{x}$ on a name makes it an abstraction, 

\begin{mathpar}
  M^{*} := (x)x?(u).M[\dropn{u}]
\end{mathpar}

\subsection{Additional notation}

It will sometimes be convenient to denote the process a name
quotes. We already have the notation $x = \quotep{P}$, but it will be
convenient to introduce an alternate notation, $\procn{x}$, when we
want to emphasize the connection to the use of the name. Note that, by
virtue of name equivalence, $\quotep{\procn{x}} \nameeq x$; so, the
notation is consistent with previous definitions.

Further, because names have structure it is possible to effect
substitutions on the basis of that structure. This means we need to
upgrade our notation for substitutions, which we accomplish by
adapting comprehension notation. Thus,

\begin{mathpar}
  P\{ y / x : x \in S \}
\end{mathpar}

is interpreted to mean the process derived from P by replacing (in a
capture-avoiding manner) each occurrence of $x$ in $S$ by $y$. For example,

\begin{mathpar}
  P\{ \quotep{\procn{x}|\procn{x}} / x : x \in \freenames{P} \}
\end{mathpar}

will replace each (occurrence) of a free name $x$ in $P$ by
$\quotep{\procn{x}|\procn{x}}$.

Also, we will avail ourselves of the notation $x^{L}$ and $x^{R}$ to
denote injections of a name into disjoint copies of the name
space. There are numerous ways to accomplish this. One example can be
found in \cite{MeredithR05}. This notation overloads to vectors of
names: $\vec{x}^{\pi} := (x_{i}^{\pi} \; : \; 0 \leq i < |\vec{x}| )$ where $\pi \in \{L,R\}$.

We also use $P^{\Box} := P|\Box$.

In \cite{MeredithR05} an interpretation of the new operator is
given. It turns out that there are several possible interpretations
all enjoying the requisite algebraic properties of the operator (see
\cite{milner91polyadicpi}). We will therefore make liberal use of
$(\nu\; \vec{x})P$.

% subsection the_syntax_and_semantics_of_the_notation_system (end)   

\input{qm2pi.qmops} 

\input{qm2pi.sterngerlach} 

\input{qm2pi.metric} 

% section concurrent_process_calculi (end)

%\input{qm2pi.proofsketch}

% section proof sketch (end)

%\input{qm2pi.slviaknots} 

% section spatial logic via knots (end)

\input{qm2pi.conclusion}

% section conclusion (end)

%\input{qm2pi.dtcodes} 

% section wiring algorithm (end)

\input{qm2pi.ack} 

% section acknowledgments (end)

\newpage


\bibliographystyle{plain}   
\bibliography{../../biblios/main.bib}

\input{qm2pi.rhodetails}

\end{document}

 

%\documentclass[12pt]{llncs}
%\documentclass{jktr}

\usepackage[pdftex]{hyperref}                   
\usepackage {listings}
\usepackage {mathpartir}
\usepackage{bcprules}
%\usepackage{listings}
                       
\usepackage{graphicx} 
%\usepackage[margins=2.5cm,nohead,nofoot]{geometry}
%\usepackage{geometry}
\usepackage{amsfonts}
\usepackage{amstext}
\usepackage{latexsym}
\usepackage{amssymb}
\usepackage{color}


%\include{myPreamble}
\include{qm2pi.local} 

%\ifpdf
%\usepackage[pdftex]{graphicx}
%\else
%\usepackage{graphicx}
%\fi

 % \ifpdf
%  \usepackage{pdfsync}
%  \if


%\title{Brief Article}
%\author{David F. Snyder}
%\author{L.G. Meredith}

%\address{Dept. of Math., Texas State University--San Marcos, San Marcos, TX 78666}
       
\pagestyle{empty}


\begin{document}

\lstset{language=[Objective]Caml,frame=shadowbox}

\input{qm2pi.front}

% section front matter (end)

\input{qm2pi.intro} 
 
% section introduction (end)

% \input{qm2pi.knotations} 

% section notation (end)

\input{qm2pi.process.calculi} 

% section concurrent_process_calculi_and_spatial_logics_ (end)
    
%\input{qm2pi.knots2pi} 

%\input{qm2pi.trefoil} 

%\input{qm2pi.mainthm} 

% subsection basic_interpretation (end)

%\input{qm2pi.rho.presentation} 
\subsection{The syntax and semantics of the notation system}\label{sub:the_syntax_and_semantics_of_the_notation_system} % (fold)

We now summarize a technical presentation of the calculus that
embodies our theory of dynamics. The typical presentation of such a
calculus follows the style of giving generators and relations on
them. The grammar, below, describing term constructors, freely
generates the set of processes, $\Proc$. This set is then quotiented
by a relation known as structural congruence and it is over this set
that the notion of dynamics is expressed. This presentation is
essentially that of \cite{MeredithR05} with the addition of
polyadicity and summation. For readability we have relegated some of
the technical subtleties to an appendix.

\subsubsection{Process grammar}\label{subsub:process_grammar}

\begin{mathpar}
  \inferrule* [lab=synchronization] {} {{M} \bc \pzero \;|\; x?F \;|\; x!C }
  \and
  \inferrule* [lab=abstraction] {} {{F} \bc (x)P}
  \and
  \inferrule* [lab=concretion] {} {{C} \bc \langle Q \rangle}
  \and
  \inferrule* [lab=process] {} {{P,Q} \bc M \;| \;P|Q \;|\; @{x}}
  \and
  \inferrule* [lab=name] {} {{x} \bc \quotep{P}}
\end{mathpar} 

Note that $\vec{x}$ (resp. $\vec{P}$) denotes a vector of names
(resp. processes) of length $|\vec{x}|$ (resp. $|\vec{P}|$). We adopt
the following useful abbreviations.

\begin{mathpar}
   x?(\vec{y}).P := x.(\vec{y})P \and  x\clift{\vec{P}} := x.\clift{\vec{P}}
   \and x!(y) := \lift{x}{\dropn{y}}
   \and \Pi_{i=0}^{n-1}P_i := P_0 | \ldots | P_{n-1}
\end{mathpar}

\subsubsection{Structural congruence}

\paragraph{Free and bound names and alpha-equivalence.} At the
core of structural equivalence is alpha-equivalence which identifies
process that are the same up to a change of variable. Formally, we
recognize the distinction between free and bound names. The free names
of a process, $\freenames{P}$, may be calculated recursively as
follows:

\begin{mathpar}
\freenames{\pzero} := \emptyset
  \and \\
  \freenames{x?(y).P} := \{ x \} \cup (\freenames{P} \setminus \{ y \})
  \and 
  \freenames{x!\langle P \rangle} := \{ x \} \cup \{ P \} 
  \and \\
  \freenames{P|Q} := \freenames{P} \cup \freenames{Q}
  \and \\
  \freenames{@{x}} := \{ x \}
\end{mathpar}

$\pi$
$\quotep{\pi}$

$\freenames{-} : \pi \to \mathcal{P}(\quotep{\pi})$

\begin{eqnarray*}
  \freenames{\pzero} & := & \emptyset \\
  \freenames{x?(y).P} & := & \{ x \} \cup (\freenames{P} \setminus \{ y \}) \\
  \freenames{x!\langle P \rangle} & := & \{ x \} \cup \{ P \} \\
  \freenames{P|Q} & := & \freenames{P} \cup \freenames{Q} \\
  \freenames{\dropn{x}} & := & \{ x \}
\end{eqnarray*}

The bound names of a process, $\boundnames{P}$, are those names occurring in $P$
that are not free. For example, in $x?(y).0$, the name $x$ is free, while $y$ is bound.

\begin{mathpar}
  \inferrule* [lab=monoidal-laws] {} { P|Q \equiv Q|P \and P|0 \equiv P \and P|(Q|R) \equiv (P|Q)|R }
\end{mathpar}

\begin{mathpar}
  \inferrule* [lab=alpha-equivalence] {} { (x)P \equiv (y)P\{y/x\} \and y \not\in \freenames{P} }
\end{mathpar}

\begin{definition}
Then two processes, $P,Q$, are alpha-equivalent if $P = Q\{\vec{y}/\vec{x}\}$ for
some $\vec{x} \in \boundnames{Q},\vec{y} \in \boundnames{P}$, where $Q\{\vec{y}/\vec{x}\}$
denotes the capture-avoiding substitution of $\vec{y}$ for $\vec{x}$ in $Q$.
\end{definition}

\begin{definition}
  The {\em structural congruence} \cite{SangiorgiWalker} , $\equiv$,
  between processes is the least congruence containing
  alpha-equivalence, satisfying the abelian monoid laws
  (associativity, commutativity and $\pzero$ as identity) for parallel
  composition $|$ and for summation $+$.
\end{definition}

\subsection{Name equivalence}

We take name equivalence, written $\nameeq$, to be the smallest
equivalence relation generated by the following rules.

\begin{mathpar}
\inferrule*[lab=Quote-drop]
{ }
{ \quotep{@{x}} \nameeq x }

\inferrule*[lab=Struct-equiv]
{ P \scong Q }
{ \quotep{P} \nameeq \quotep{Q} }
\end{mathpar}

The astute reader will have noticed that the mutual recursion of names
and processes imposes a mutual recursion on alpha-equivalence and
structural equivalence via name-equivalence. Fortunately, all of this
works out pleasantly and we may calculate in the natural way, free of
concern. The reader interested in the details is referred to the
appendix \ref{appendix:rho_details}.

\subsection{Substitution}

We use $\Proc$ for the set of processes, $\QProc$ for the set of
names, and $\id{\{}\vec{y} / \vec{x} \id{\}}$ to denote partial maps,
$s : \QProc \rightarrow \QProc$. A map, $s$ lifts, uniquely, to a map
on process terms, $\widehat{s} : \Proc \rightarrow \Proc$ by the
following equations.

\begin{mathpar}
  (0) \psubstp{Q}{P} := 0 \\
  (R \juxtap S) \psubstp{Q}{P}
  :=    
  (R)\psubstp{Q}{P} \juxtap (S) \psubstp{Q}{P} \\
  (x?(y).R) \psubstp{Q}{P}    
  :=    
  (x)\substp{Q}{P} (z)\concat( (R \psubstn{z}{y}) \psubstp{Q}{P} ) \\
  (\lift{x}{R}) \psubstp{Q}{P}  
  :=
  \lift{(x)\substp{Q}{P}}{ R \psubstp{Q}{P} } \\
%   (\dropn{x})  \psubstp{Q}{P}       
%   := 
%   \left\{ 
%     \begin{array}{ccc} 
%       \dropn{\quotep{Q}} & & x \nameeq \quotep{P} \\
%       \dropn{x} & & otherwise \\
%     \end{array}
%   \right. 
  (\dropn{x})  \psubstp{Q}{P}       
  := 
  \left\{ 
    \begin{array}{ccc} 
      Q & & x \nameeq \quotep{P} \\
      \dropn{x} & & otherwise \\
    \end{array}
  \right.
\end{mathpar}
 

where

\begin{eqnarray}
  (x)\id{\{} \lpquote Q \rpquote / \lpquote P \rpquote \id{\}}            = 
  \left\{ 
    \begin{array}{ccc}
      \lpquote Q \rpquote & & x \nameeq \lpquote P \rpquote \\
      x & & otherwise \\
    \end{array}
  \right. \nonumber
\end{eqnarray}

and $z$ is chosen distinct from $\quotep{P}$, $\quotep{Q}$, the free
names in $Q$, and all the names in $R$. Our $\alpha$-equivalence will
be built in the standard way from this substitution.

\begin{remark}\label{rem:no_self_referential_names}
  One consequence of these definitions is that $\forall P. \quotep{P}
  \not\in \freenames{P}$.
\end{remark}

\subsection{ Dynamic quote: an example }

Anticipating something of what's to come, consider applying the
substitution, $\widehat{\id{\{}u / z \id{\}}}$, to the following pair
of processes, $\lift{w}{y!(z)}$ and $w[ \lpquote y!(z) \rpquote ]$.

\begin{eqnarray}
	\lift{w}{y!(z)}\widehat{\id{\{}u / z \id{\}}}
		& = &
		\lift{w}{y!(u)} \nonumber\\
	w[ \lpquote y!(z) \rpquote ] \widehat{ \id{\{}u / z \id{\}} }
		& = &
		w[ \lpquote y!(z) \rpquote ] \nonumber
\end{eqnarray}

Because the body of the process between quotes is impervious to
substitution, we get radically different answers. In fact, by
examining the first process in an input context,
e.g. $x?(z).\lift{w}{y!(z)}$, we see that the process under the lift
operator may be shaped by prefixed inputs binding a name inside it. In
this sense, the lift operator will be seen as a way to dynamically
construct processes before reifying them as names.

Finally equipped with these standard features we can present the
dynamics of the calculus.

\subsubsection{Operational semantics} 

Finally, we introduce the computational dynamics. What marks these
algebras as distinct from other more traditionally studied algebraic
structures, e.g. vector spaces or polynomial rings, is the manner in
which dynamics is captured. In traditional structures, dynamics is typically
expressed through morphisms between such structures, as in linear maps
between vector spaces or morphisms between rings. In algebras
associated with the semantics of computation, the dynamics is
expressed as part of the algebraic structure itself, through a
reduction reduction relation typically denoted by $\red$. Below, we
give a recursive presentation of this relation for the calculus used
in the encoding.

$\red \subseteq \pi \times \pi$
$\red : \pi \to \mathcal{P}(\pi)$

\begin{mathpar}
  \inferrule* [lab=Comm] { \textsf{match}( x_{src}, x_{trgt} ) } { x_{trgt}?(y)P \; | \; x_{src}!\langle {Q} \rangle \red P\{\quotep{Q}/y}\} }
  \and \\
  \inferrule* [lab=Par] {{P} \red {P}'} {{{P} | {Q}} \red {{P}' | {Q}}}
  \and
  \inferrule* [lab=Equiv]{{{P} \scong {P}'} \andalso {{P}' \red {Q}'} \andalso {{Q}' \scong {Q}}}{{P} \red {Q}}
\end{mathpar}

\begin{eqnarray*}
  match_{\equiv} (\quotep{P},\quotep{Q}) & := & P \equiv Q \\
  match_{\dagger}(\quotep{P},\quotep{Q}) & := & \forall R. P|Q \red^{*} R => R \red^{*} 0 \\
  match_{K}(\quotep{P},\quotep{Q}) & := & K \mbox{ for some context } K
\end{eqnarray*}

$u?(x)P | u!\langle Q \rangle \red P\{\quotep{Q}/x\}$

%We write $\wred$ for $\red^*$, and $P\red$ if $\exists Q $ such that $ P \red Q$.
We write $P\red$ if $\exists Q $ such that $ P \red Q$ and $P\not\red$, otherwise.

\section{Replication}

As mentioned before, it is known that replication (and hence
recursion) can be implemented in a higher-order process algebra
\cite{SangiorgiWalker}. As our first example of calculation with the
machinery thus far presented we give the construction explicitly in
the {\rhoc}.

\begin{eqnarray}
	D_{x} & := & \prefix{x}{y}{(\binpar{\outputp{x}{y}}{@{y}})} \nonumber\\
	\bangp_{x}{P} & := & \binpar{{x}!\langle{\binpar{D_{x}}{P}}\rangle}{D_{x}} \nonumber
\end{eqnarray}

\begin{eqnarray}
	\bangp_{x}{P} & & \nonumber\\
	=
	& {x}!\langle{(\prefix{x}{y}{(\outputp{x}{y} | @{y})) | P}}\rangle 
	      | \prefix{x}{y}{(\outputp{x}{y} | @{y})} & \nonumber\\
	\red
	& (\outputp{x}{y} | @{y})\substn{\quotep{(\prefix{x}{y}{(@{y} | \outputp{x}{y})) | P}}}{y} & \nonumber\\
	=
	& \outputp{x}{\quotep{(\prefix{x}{y}{(\outputp{x}{y} | @{y})) | P}}}
	  | {(\prefix{x}{y}{(\outputp{x}{y} | @{y})) | P}} & \nonumber\\
	\red
	& \ldots & \nonumber\\
	\red^*
	& P | P | \ldots & \nonumber
\end{eqnarray}

Of course, this encoding, as an implementation, runs away, unfolding
$\bangp{P}$ eagerly. A lazier and more implementable replication
operator, restricted to input-guarded processes, may be obtained as follows.

\begin{eqnarray}
\bangp{\prefix{u}{v}{P}} 
	:= 
	\binpar{\lift{x}{\prefix{u}{v}{(\binpar{D(x)}{P})}}}{D(x)} \nonumber
\end{eqnarray}

\begin{remark}
  Note that the lazier definition still does not deal with summation
  or mixed summation (i.e. sums over input and output). The reader is
  invited to construct definitions of replication that deal with these
  features. 

  Further, the definitions are parameterized in a name, $x$. Can you,
  gentle reader, make a definition that eliminates this parameter and
  guarantees no accidental interaction between the replication
  machinery and the process being replicated -- i.e. no accidental
  sharing of names used by the process to get its work done and the
  name(s) used by the replication to effect copying. This latter
  revision of the definition of replication is crucial to obtaining
  the expected identity $!!P \sim !P$.
\end{remark}

\begin{remark}\label{rem:paradoxical_combinator}
  The reader familiar with the lambda calculus will have noticed the
  similarity between $D$ and the paradoxical combinator.

  [Ed. note: the existence of this seems to suggest we have to be more
  restrictive on the set of processes and names we admit if we are to
  support no-cloning.]
\end{remark}

\subsubsection{Bisimulation}

The computational dynamics gives rise to another kind of equivalence,
the equivalence of computational behavior. As previously mentioned
this is typically captured \emph{via} some form of bisimulation.

% The notion we use in this paper is weak barbed bisimulation
% \cite{milner91polyadicpi}.

The notion we use in this paper is derived from weak barbed
bisimulation \cite{milner91polyadicpi}. 

\begin{definition}
An \emph{observation relation}, $\downarrow_{\mathcal N}$, over a set
of names, $\mathcal N$, is the smallest relation satisfying the rules
below.

\infrule[Out-barb]{y \in {\mathcal N}, \; x \nameeq y}
		  {\outputp{x}{v} \downarrow_{\mathcal N} x}
\infrule[Par-barb]{\mbox{$P\downarrow_{\mathcal N} x$ or $Q\downarrow_{\mathcal N} x$}}
		  {\binpar{P}{Q} \downarrow_{\mathcal N} x}

We write $P \Downarrow_{\mathcal N} x$ if there is $Q$ such that 
$P \wred Q$ and $Q \downarrow_{\mathcal N} x$.
\end{definition}

\begin{definition}
%\label{def.bbisim}
An  ${\mathcal N}$-\emph{barbed bisimulation} over a set of names, ${\mathcal N}$, is a symmetric binary relation 
${\mathcal S}_{\mathcal N}$ between agents such that $P\rel{S}_{\mathcal N}Q$ implies:
\begin{enumerate}
\item If $P \red P'$ then $Q \wred Q'$ and $P'\rel{S}_{\mathcal N} Q'$.
\item If $P\downarrow_{\mathcal N} x$, then $Q\Downarrow_{\mathcal N} x$.
\end{enumerate}
$P$ is ${\mathcal N}$-barbed bisimilar to $Q$, written
$P \wbbisim_{\mathcal N} Q$, if $P \rel{S}_{\mathcal N} Q$ for some ${\mathcal N}$-barbed bisimulation ${\mathcal S}_{\mathcal N}$.
\end{definition}

$\mathcal{R} \subseteq \pi \times \pi$

$P \mathcal{R} Q => \forall P'. P \red P' \Rightarrow \exists Q'. Q \red Q', P' \mathcal{R} Q'$

$P \vdash x \Rightarrow Q \vdash x$

\begin{mathpar}
  \inferrule*[lab=Out-barb]{x \nameeq y}{{y}!\langle{Q}\rangle \vdash x}
  \and
  \inferrule*[lab=Par-barb]{\mbox{$P\vdash x$ or $Q\vdash x$}}{\binpar{P}{Q} \vdash x}
\end{mathpar}

\subsubsection{Contexts}

One of the principle advantages of computational calculi like the
$\pi$-calculus is a well-defined notion of context,
contextual-equivalence and a correlation between
contextual-equivalence and notions of bisimulation. The notion of
context allows the decomposition of a process into (sub-)process and
its syntactic environment, its context. Thus, a context may be
thought of as a process with a ``hole'' (written $\Box$) in it. The
application of a context $M$ to a process $P$, written $M[P]$, is
tantamount to filling the hole in $M$ with $P$. In this paper we do
not need the full weight of this theory, but do make use of the notion
of context in the proof the main theorem. 

\begin{mathpar}
  \inferrule* [lab=summation] {} {{M_{M},M_{N}} \bc \Box \;|\; x.M_{A} \;|\; M_{M}+M_{N}}
  \and
  \inferrule* [lab=agent] {} {{M_{A}} \bc (\vec{x})M_{P} \;| \; \clift{P_0,\ldots,M_{P},\ldots,P_N}}
  \and \\
  \inferrule* [lab=process] {} {{M_{P}} \bc M_{N} \;| \;P|M_{P} }
\end{mathpar} 

\begin{mathpar}
  \inferrule* [lab=sychronization] {} {M_{N} \bc \Box \;|\; x?M_{F} \;|\; x!M_{C}}
  \and
  \inferrule* [lab=abstraction] {} {{M_{F}} \bc (x)M_{P} }
  \and
  \inferrule* [lab=concretion] {} {{M_{C}} \bc \langle M_{P} \rangle }
  \and \\
  \inferrule* [lab=process] {} {{M_{P}} \bc M_{N} \;| \;P|M_{P} }
\end{mathpar}

\begin{definition}[contextual application] Given a context $M$, and
  process $P$, we define the \emph{contextual application}, $M[P] :=
  M\{P/\Box\}$. That is, the contextual application of M to P is the
  substitution of $P$ for $\Box$ in $M$.
\end{definition}

$\meaningof{-} : L \to \mathcal{P}(\pi)$

\begin{mathpar}
  \inferrule* [lab=collection] {} {\meaningof{true} = \pi, \and \meaningof{~E} = \pi \setminus \meaningof{E}, \and \meaningof{E_{1} \& E_{2}} = \meaningof{E_{1}} \cap \meaningof{E_{2}}}
\end{mathpar}

\begin{mathpar}
  \inferrule* [lab=structure] {} {\meaningof{0} = \{ P \in \pi | P \equiv 0 \}, \and \\ \meaningof{E_1 | E_2} = \{ P \in \pi | P \equiv P_{1} | P_{2}, P_{1} \in \meaningof{E_{1}}, P_{2} \in \meaningof{E_2}\} }
\end{mathpar}

\begin{mathpar}
 \inferrule* [lab=behavior] {} {\meaningof{\langle a?b \rangle E} = \{ P \in \pi | P \equiv Q | u?(y)P', \\ \and \\\\ \and \\ \;\;\; u \in \meaningof{a}, \forall z.P'\{z/y\} \in \meaningof{E\{z/b\}}\}, \and \\ \meaningof{a!E} = \{ P \in \pi | P \equiv Q | x!\langle P' \rangle, x \in \meaningof{a} P' \in \meaningof{E}\} }
\end{mathpar}

\begin{mathpar}
 \inferrule* [lab=nominal] {} {\meaningof{\quotep{E}} = \{ \quotep{P} \in \quotep{\pi} | P \in \meaningof{E} \}, \and \meaningof{\quotep{P}} = \{ \quotep{Q} \in \quotep{\pi} | P \equiv Q \} \and \\ \meaningof{@\quotep{E}} = \{ P \in \pi | P \equiv @x, x \in \meaningof{E} \}}
\end{mathpar}

\begin{eqnarray*}
  \\
  \meaningof{-} : TS \to ST
\end{eqnarray*}

\begin{eqnarray*}
  \\
  L : TS \to ST
\end{eqnarray*}

\begin{eqnarray*}
  \\
  P \models E \iff P \in \meaningof{E}
\end{eqnarray*}

\begin{eqnarray*}
  P \approx_{L} Q \iff \forall E \in L. P \models E \iff Q \models E
\end{eqnarray*}

\begin{eqnarray*}
  P \approx_{K} Q
\end{eqnarray*}

\begin{eqnarray*}
  P \approx Q
\end{eqnarray*}

$\approx_{K} = \approx = \approx_{L}$

\subsubsection{Contextual duality}

Note that contexts extend the quotation operation to a family of
operations from processes to names. Given a context, $M$, we can
define a \emph{nominal context}, $\quotep{M}$ by $\quotep{M}[P] :=
\quotep{M[P]}$. To foreshadow what is to come we observe that these
operations enjoy a duality with processes very much like the duality
between vectors and maps from vectors to scalars.

Further, because the calculus is essentially higher-order, we have a
correspondence between contexts and processes. More specifically,
given a name $x$ and a context $M$ we can construct $M^{*}_{x}$ such
that 

\begin{mathpar}
  M^{*}_{x} | \lift{x}{P} \red M[P]
\end{mathpar}

namely,

\begin{mathpar}
  M^{*}_{x} := x?(u).M[\dropn{u}]
\end{mathpar}

The dependence of $M^{*}_{x}$ on a name makes it an abstraction, 

\begin{mathpar}
  M^{*} := (x)x?(u).M[\dropn{u}]
\end{mathpar}

\subsection{Additional notation}

It will sometimes be convenient to denote the process a name
quotes. We already have the notation $x = \quotep{P}$, but it will be
convenient to introduce an alternate notation, $\procn{x}$, when we
want to emphasize the connection to the use of the name. Note that, by
virtue of name equivalence, $\quotep{\procn{x}} \nameeq x$; so, the
notation is consistent with previous definitions.

Further, because names have structure it is possible to effect
substitutions on the basis of that structure. This means we need to
upgrade our notation for substitutions, which we accomplish by
adapting comprehension notation. Thus,

\begin{mathpar}
  P\{ y / x : x \in S \}
\end{mathpar}

is interpreted to mean the process derived from P by replacing (in a
capture-avoiding manner) each occurrence of $x$ in $S$ by $y$. For example,

\begin{mathpar}
  P\{ \quotep{\procn{x}|\procn{x}} / x : x \in \freenames{P} \}
\end{mathpar}

will replace each (occurrence) of a free name $x$ in $P$ by
$\quotep{\procn{x}|\procn{x}}$.

Also, we will avail ourselves of the notation $x^{L}$ and $x^{R}$ to
denote injections of a name into disjoint copies of the name
space. There are numerous ways to accomplish this. One example can be
found in \cite{MeredithR05}. This notation overloads to vectors of
names: $\vec{x}^{\pi} := (x_{i}^{\pi} \; : \; 0 \leq i < |\vec{x}| )$ where $\pi \in \{L,R\}$.

We also use $P^{\Box} := P|\Box$.

In \cite{MeredithR05} an interpretation of the new operator is
given. It turns out that there are several possible interpretations
all enjoying the requisite algebraic properties of the operator (see
\cite{milner91polyadicpi}). We will therefore make liberal use of
$(\nu\; \vec{x})P$.

% subsection the_syntax_and_semantics_of_the_notation_system (end)   

\input{qm2pi.qmops} 

\input{qm2pi.sterngerlach} 

\input{qm2pi.metric} 

% section concurrent_process_calculi (end)

%\input{qm2pi.proofsketch}

% section proof sketch (end)

%\input{qm2pi.slviaknots} 

% section spatial logic via knots (end)

\input{qm2pi.conclusion}

% section conclusion (end)

%\input{qm2pi.dtcodes} 

% section wiring algorithm (end)

\input{qm2pi.ack} 

% section acknowledgments (end)

\newpage


\bibliographystyle{plain}   
\bibliography{../../biblios/main.bib}

\input{qm2pi.rhodetails}

\end{document}

 

%\documentclass[12pt]{llncs}
%\documentclass{jktr}

\usepackage[pdftex]{hyperref}                   
\usepackage {listings}
\usepackage {mathpartir}
\usepackage{bcprules}
%\usepackage{listings}
                       
\usepackage{graphicx} 
%\usepackage[margins=2.5cm,nohead,nofoot]{geometry}
%\usepackage{geometry}
\usepackage{amsfonts}
\usepackage{amstext}
\usepackage{latexsym}
\usepackage{amssymb}
\usepackage{color}


%\include{myPreamble}
\include{qm2pi.local} 

%\ifpdf
%\usepackage[pdftex]{graphicx}
%\else
%\usepackage{graphicx}
%\fi

 % \ifpdf
%  \usepackage{pdfsync}
%  \if


%\title{Brief Article}
%\author{David F. Snyder}
%\author{L.G. Meredith}

%\address{Dept. of Math., Texas State University--San Marcos, San Marcos, TX 78666}
       
\pagestyle{empty}


\begin{document}

\lstset{language=[Objective]Caml,frame=shadowbox}

\input{qm2pi.front}

% section front matter (end)

\input{qm2pi.intro} 
 
% section introduction (end)

% \input{qm2pi.knotations} 

% section notation (end)

\input{qm2pi.process.calculi} 

% section concurrent_process_calculi_and_spatial_logics_ (end)
    
%\input{qm2pi.knots2pi} 

%\input{qm2pi.trefoil} 

%\input{qm2pi.mainthm} 

% subsection basic_interpretation (end)

%\input{qm2pi.rho.presentation} 
\subsection{The syntax and semantics of the notation system}\label{sub:the_syntax_and_semantics_of_the_notation_system} % (fold)

We now summarize a technical presentation of the calculus that
embodies our theory of dynamics. The typical presentation of such a
calculus follows the style of giving generators and relations on
them. The grammar, below, describing term constructors, freely
generates the set of processes, $\Proc$. This set is then quotiented
by a relation known as structural congruence and it is over this set
that the notion of dynamics is expressed. This presentation is
essentially that of \cite{MeredithR05} with the addition of
polyadicity and summation. For readability we have relegated some of
the technical subtleties to an appendix.

\subsubsection{Process grammar}\label{subsub:process_grammar}

\begin{mathpar}
  \inferrule* [lab=synchronization] {} {{M} \bc \pzero \;|\; x?F \;|\; x!C }
  \and
  \inferrule* [lab=abstraction] {} {{F} \bc (x)P}
  \and
  \inferrule* [lab=concretion] {} {{C} \bc \langle Q \rangle}
  \and
  \inferrule* [lab=process] {} {{P,Q} \bc M \;| \;P|Q \;|\; @{x}}
  \and
  \inferrule* [lab=name] {} {{x} \bc \quotep{P}}
\end{mathpar} 

Note that $\vec{x}$ (resp. $\vec{P}$) denotes a vector of names
(resp. processes) of length $|\vec{x}|$ (resp. $|\vec{P}|$). We adopt
the following useful abbreviations.

\begin{mathpar}
   x?(\vec{y}).P := x.(\vec{y})P \and  x\clift{\vec{P}} := x.\clift{\vec{P}}
   \and x!(y) := \lift{x}{\dropn{y}}
   \and \Pi_{i=0}^{n-1}P_i := P_0 | \ldots | P_{n-1}
\end{mathpar}

\subsubsection{Structural congruence}

\paragraph{Free and bound names and alpha-equivalence.} At the
core of structural equivalence is alpha-equivalence which identifies
process that are the same up to a change of variable. Formally, we
recognize the distinction between free and bound names. The free names
of a process, $\freenames{P}$, may be calculated recursively as
follows:

\begin{mathpar}
\freenames{\pzero} := \emptyset
  \and \\
  \freenames{x?(y).P} := \{ x \} \cup (\freenames{P} \setminus \{ y \})
  \and 
  \freenames{x!\langle P \rangle} := \{ x \} \cup \{ P \} 
  \and \\
  \freenames{P|Q} := \freenames{P} \cup \freenames{Q}
  \and \\
  \freenames{@{x}} := \{ x \}
\end{mathpar}

$\pi$
$\quotep{\pi}$

$\freenames{-} : \pi \to \mathcal{P}(\quotep{\pi})$

\begin{eqnarray*}
  \freenames{\pzero} & := & \emptyset \\
  \freenames{x?(y).P} & := & \{ x \} \cup (\freenames{P} \setminus \{ y \}) \\
  \freenames{x!\langle P \rangle} & := & \{ x \} \cup \{ P \} \\
  \freenames{P|Q} & := & \freenames{P} \cup \freenames{Q} \\
  \freenames{\dropn{x}} & := & \{ x \}
\end{eqnarray*}

The bound names of a process, $\boundnames{P}$, are those names occurring in $P$
that are not free. For example, in $x?(y).0$, the name $x$ is free, while $y$ is bound.

\begin{mathpar}
  \inferrule* [lab=monoidal-laws] {} { P|Q \equiv Q|P \and P|0 \equiv P \and P|(Q|R) \equiv (P|Q)|R }
\end{mathpar}

\begin{mathpar}
  \inferrule* [lab=alpha-equivalence] {} { (x)P \equiv (y)P\{y/x\} \and y \not\in \freenames{P} }
\end{mathpar}

\begin{definition}
Then two processes, $P,Q$, are alpha-equivalent if $P = Q\{\vec{y}/\vec{x}\}$ for
some $\vec{x} \in \boundnames{Q},\vec{y} \in \boundnames{P}$, where $Q\{\vec{y}/\vec{x}\}$
denotes the capture-avoiding substitution of $\vec{y}$ for $\vec{x}$ in $Q$.
\end{definition}

\begin{definition}
  The {\em structural congruence} \cite{SangiorgiWalker} , $\equiv$,
  between processes is the least congruence containing
  alpha-equivalence, satisfying the abelian monoid laws
  (associativity, commutativity and $\pzero$ as identity) for parallel
  composition $|$ and for summation $+$.
\end{definition}

\subsection{Name equivalence}

We take name equivalence, written $\nameeq$, to be the smallest
equivalence relation generated by the following rules.

\begin{mathpar}
\inferrule*[lab=Quote-drop]
{ }
{ \quotep{@{x}} \nameeq x }

\inferrule*[lab=Struct-equiv]
{ P \scong Q }
{ \quotep{P} \nameeq \quotep{Q} }
\end{mathpar}

The astute reader will have noticed that the mutual recursion of names
and processes imposes a mutual recursion on alpha-equivalence and
structural equivalence via name-equivalence. Fortunately, all of this
works out pleasantly and we may calculate in the natural way, free of
concern. The reader interested in the details is referred to the
appendix \ref{appendix:rho_details}.

\subsection{Substitution}

We use $\Proc$ for the set of processes, $\QProc$ for the set of
names, and $\id{\{}\vec{y} / \vec{x} \id{\}}$ to denote partial maps,
$s : \QProc \rightarrow \QProc$. A map, $s$ lifts, uniquely, to a map
on process terms, $\widehat{s} : \Proc \rightarrow \Proc$ by the
following equations.

\begin{mathpar}
  (0) \psubstp{Q}{P} := 0 \\
  (R \juxtap S) \psubstp{Q}{P}
  :=    
  (R)\psubstp{Q}{P} \juxtap (S) \psubstp{Q}{P} \\
  (x?(y).R) \psubstp{Q}{P}    
  :=    
  (x)\substp{Q}{P} (z)\concat( (R \psubstn{z}{y}) \psubstp{Q}{P} ) \\
  (\lift{x}{R}) \psubstp{Q}{P}  
  :=
  \lift{(x)\substp{Q}{P}}{ R \psubstp{Q}{P} } \\
%   (\dropn{x})  \psubstp{Q}{P}       
%   := 
%   \left\{ 
%     \begin{array}{ccc} 
%       \dropn{\quotep{Q}} & & x \nameeq \quotep{P} \\
%       \dropn{x} & & otherwise \\
%     \end{array}
%   \right. 
  (\dropn{x})  \psubstp{Q}{P}       
  := 
  \left\{ 
    \begin{array}{ccc} 
      Q & & x \nameeq \quotep{P} \\
      \dropn{x} & & otherwise \\
    \end{array}
  \right.
\end{mathpar}
 

where

\begin{eqnarray}
  (x)\id{\{} \lpquote Q \rpquote / \lpquote P \rpquote \id{\}}            = 
  \left\{ 
    \begin{array}{ccc}
      \lpquote Q \rpquote & & x \nameeq \lpquote P \rpquote \\
      x & & otherwise \\
    \end{array}
  \right. \nonumber
\end{eqnarray}

and $z$ is chosen distinct from $\quotep{P}$, $\quotep{Q}$, the free
names in $Q$, and all the names in $R$. Our $\alpha$-equivalence will
be built in the standard way from this substitution.

\begin{remark}\label{rem:no_self_referential_names}
  One consequence of these definitions is that $\forall P. \quotep{P}
  \not\in \freenames{P}$.
\end{remark}

\subsection{ Dynamic quote: an example }

Anticipating something of what's to come, consider applying the
substitution, $\widehat{\id{\{}u / z \id{\}}}$, to the following pair
of processes, $\lift{w}{y!(z)}$ and $w[ \lpquote y!(z) \rpquote ]$.

\begin{eqnarray}
	\lift{w}{y!(z)}\widehat{\id{\{}u / z \id{\}}}
		& = &
		\lift{w}{y!(u)} \nonumber\\
	w[ \lpquote y!(z) \rpquote ] \widehat{ \id{\{}u / z \id{\}} }
		& = &
		w[ \lpquote y!(z) \rpquote ] \nonumber
\end{eqnarray}

Because the body of the process between quotes is impervious to
substitution, we get radically different answers. In fact, by
examining the first process in an input context,
e.g. $x?(z).\lift{w}{y!(z)}$, we see that the process under the lift
operator may be shaped by prefixed inputs binding a name inside it. In
this sense, the lift operator will be seen as a way to dynamically
construct processes before reifying them as names.

Finally equipped with these standard features we can present the
dynamics of the calculus.

\subsubsection{Operational semantics} 

Finally, we introduce the computational dynamics. What marks these
algebras as distinct from other more traditionally studied algebraic
structures, e.g. vector spaces or polynomial rings, is the manner in
which dynamics is captured. In traditional structures, dynamics is typically
expressed through morphisms between such structures, as in linear maps
between vector spaces or morphisms between rings. In algebras
associated with the semantics of computation, the dynamics is
expressed as part of the algebraic structure itself, through a
reduction reduction relation typically denoted by $\red$. Below, we
give a recursive presentation of this relation for the calculus used
in the encoding.

$\red \subseteq \pi \times \pi$
$\red : \pi \to \mathcal{P}(\pi)$

\begin{mathpar}
  \inferrule* [lab=Comm] { \textsf{match}( x_{src}, x_{trgt} ) } { x_{trgt}?(y)P \; | \; x_{src}!\langle {Q} \rangle \red P\{\quotep{Q}/y}\} }
  \and \\
  \inferrule* [lab=Par] {{P} \red {P}'} {{{P} | {Q}} \red {{P}' | {Q}}}
  \and
  \inferrule* [lab=Equiv]{{{P} \scong {P}'} \andalso {{P}' \red {Q}'} \andalso {{Q}' \scong {Q}}}{{P} \red {Q}}
\end{mathpar}

\begin{eqnarray*}
  match_{\equiv} (\quotep{P},\quotep{Q}) & := & P \equiv Q \\
  match_{\dagger}(\quotep{P},\quotep{Q}) & := & \forall R. P|Q \red^{*} R => R \red^{*} 0 \\
  match_{K}(\quotep{P},\quotep{Q}) & := & K \mbox{ for some context } K
\end{eqnarray*}

$u?(x)P | u!\langle Q \rangle \red P\{\quotep{Q}/x\}$

%We write $\wred$ for $\red^*$, and $P\red$ if $\exists Q $ such that $ P \red Q$.
We write $P\red$ if $\exists Q $ such that $ P \red Q$ and $P\not\red$, otherwise.

\section{Replication}

As mentioned before, it is known that replication (and hence
recursion) can be implemented in a higher-order process algebra
\cite{SangiorgiWalker}. As our first example of calculation with the
machinery thus far presented we give the construction explicitly in
the {\rhoc}.

\begin{eqnarray}
	D_{x} & := & \prefix{x}{y}{(\binpar{\outputp{x}{y}}{@{y}})} \nonumber\\
	\bangp_{x}{P} & := & \binpar{{x}!\langle{\binpar{D_{x}}{P}}\rangle}{D_{x}} \nonumber
\end{eqnarray}

\begin{eqnarray}
	\bangp_{x}{P} & & \nonumber\\
	=
	& {x}!\langle{(\prefix{x}{y}{(\outputp{x}{y} | @{y})) | P}}\rangle 
	      | \prefix{x}{y}{(\outputp{x}{y} | @{y})} & \nonumber\\
	\red
	& (\outputp{x}{y} | @{y})\substn{\quotep{(\prefix{x}{y}{(@{y} | \outputp{x}{y})) | P}}}{y} & \nonumber\\
	=
	& \outputp{x}{\quotep{(\prefix{x}{y}{(\outputp{x}{y} | @{y})) | P}}}
	  | {(\prefix{x}{y}{(\outputp{x}{y} | @{y})) | P}} & \nonumber\\
	\red
	& \ldots & \nonumber\\
	\red^*
	& P | P | \ldots & \nonumber
\end{eqnarray}

Of course, this encoding, as an implementation, runs away, unfolding
$\bangp{P}$ eagerly. A lazier and more implementable replication
operator, restricted to input-guarded processes, may be obtained as follows.

\begin{eqnarray}
\bangp{\prefix{u}{v}{P}} 
	:= 
	\binpar{\lift{x}{\prefix{u}{v}{(\binpar{D(x)}{P})}}}{D(x)} \nonumber
\end{eqnarray}

\begin{remark}
  Note that the lazier definition still does not deal with summation
  or mixed summation (i.e. sums over input and output). The reader is
  invited to construct definitions of replication that deal with these
  features. 

  Further, the definitions are parameterized in a name, $x$. Can you,
  gentle reader, make a definition that eliminates this parameter and
  guarantees no accidental interaction between the replication
  machinery and the process being replicated -- i.e. no accidental
  sharing of names used by the process to get its work done and the
  name(s) used by the replication to effect copying. This latter
  revision of the definition of replication is crucial to obtaining
  the expected identity $!!P \sim !P$.
\end{remark}

\begin{remark}\label{rem:paradoxical_combinator}
  The reader familiar with the lambda calculus will have noticed the
  similarity between $D$ and the paradoxical combinator.

  [Ed. note: the existence of this seems to suggest we have to be more
  restrictive on the set of processes and names we admit if we are to
  support no-cloning.]
\end{remark}

\subsubsection{Bisimulation}

The computational dynamics gives rise to another kind of equivalence,
the equivalence of computational behavior. As previously mentioned
this is typically captured \emph{via} some form of bisimulation.

% The notion we use in this paper is weak barbed bisimulation
% \cite{milner91polyadicpi}.

The notion we use in this paper is derived from weak barbed
bisimulation \cite{milner91polyadicpi}. 

\begin{definition}
An \emph{observation relation}, $\downarrow_{\mathcal N}$, over a set
of names, $\mathcal N$, is the smallest relation satisfying the rules
below.

\infrule[Out-barb]{y \in {\mathcal N}, \; x \nameeq y}
		  {\outputp{x}{v} \downarrow_{\mathcal N} x}
\infrule[Par-barb]{\mbox{$P\downarrow_{\mathcal N} x$ or $Q\downarrow_{\mathcal N} x$}}
		  {\binpar{P}{Q} \downarrow_{\mathcal N} x}

We write $P \Downarrow_{\mathcal N} x$ if there is $Q$ such that 
$P \wred Q$ and $Q \downarrow_{\mathcal N} x$.
\end{definition}

\begin{definition}
%\label{def.bbisim}
An  ${\mathcal N}$-\emph{barbed bisimulation} over a set of names, ${\mathcal N}$, is a symmetric binary relation 
${\mathcal S}_{\mathcal N}$ between agents such that $P\rel{S}_{\mathcal N}Q$ implies:
\begin{enumerate}
\item If $P \red P'$ then $Q \wred Q'$ and $P'\rel{S}_{\mathcal N} Q'$.
\item If $P\downarrow_{\mathcal N} x$, then $Q\Downarrow_{\mathcal N} x$.
\end{enumerate}
$P$ is ${\mathcal N}$-barbed bisimilar to $Q$, written
$P \wbbisim_{\mathcal N} Q$, if $P \rel{S}_{\mathcal N} Q$ for some ${\mathcal N}$-barbed bisimulation ${\mathcal S}_{\mathcal N}$.
\end{definition}

$\mathcal{R} \subseteq \pi \times \pi$

$P \mathcal{R} Q => \forall P'. P \red P' \Rightarrow \exists Q'. Q \red Q', P' \mathcal{R} Q'$

$P \vdash x \Rightarrow Q \vdash x$

\begin{mathpar}
  \inferrule*[lab=Out-barb]{x \nameeq y}{{y}!\langle{Q}\rangle \vdash x}
  \and
  \inferrule*[lab=Par-barb]{\mbox{$P\vdash x$ or $Q\vdash x$}}{\binpar{P}{Q} \vdash x}
\end{mathpar}

\subsubsection{Contexts}

One of the principle advantages of computational calculi like the
$\pi$-calculus is a well-defined notion of context,
contextual-equivalence and a correlation between
contextual-equivalence and notions of bisimulation. The notion of
context allows the decomposition of a process into (sub-)process and
its syntactic environment, its context. Thus, a context may be
thought of as a process with a ``hole'' (written $\Box$) in it. The
application of a context $M$ to a process $P$, written $M[P]$, is
tantamount to filling the hole in $M$ with $P$. In this paper we do
not need the full weight of this theory, but do make use of the notion
of context in the proof the main theorem. 

\begin{mathpar}
  \inferrule* [lab=summation] {} {{M_{M},M_{N}} \bc \Box \;|\; x.M_{A} \;|\; M_{M}+M_{N}}
  \and
  \inferrule* [lab=agent] {} {{M_{A}} \bc (\vec{x})M_{P} \;| \; \clift{P_0,\ldots,M_{P},\ldots,P_N}}
  \and \\
  \inferrule* [lab=process] {} {{M_{P}} \bc M_{N} \;| \;P|M_{P} }
\end{mathpar} 

\begin{mathpar}
  \inferrule* [lab=sychronization] {} {M_{N} \bc \Box \;|\; x?M_{F} \;|\; x!M_{C}}
  \and
  \inferrule* [lab=abstraction] {} {{M_{F}} \bc (x)M_{P} }
  \and
  \inferrule* [lab=concretion] {} {{M_{C}} \bc \langle M_{P} \rangle }
  \and \\
  \inferrule* [lab=process] {} {{M_{P}} \bc M_{N} \;| \;P|M_{P} }
\end{mathpar}

\begin{definition}[contextual application] Given a context $M$, and
  process $P$, we define the \emph{contextual application}, $M[P] :=
  M\{P/\Box\}$. That is, the contextual application of M to P is the
  substitution of $P$ for $\Box$ in $M$.
\end{definition}

$\meaningof{-} : L \to \mathcal{P}(\pi)$

\begin{mathpar}
  \inferrule* [lab=collection] {} {\meaningof{true} = \pi, \and \meaningof{~E} = \pi \setminus \meaningof{E}, \and \meaningof{E_{1} \& E_{2}} = \meaningof{E_{1}} \cap \meaningof{E_{2}}}
\end{mathpar}

\begin{mathpar}
  \inferrule* [lab=structure] {} {\meaningof{0} = \{ P \in \pi | P \equiv 0 \}, \and \\ \meaningof{E_1 | E_2} = \{ P \in \pi | P \equiv P_{1} | P_{2}, P_{1} \in \meaningof{E_{1}}, P_{2} \in \meaningof{E_2}\} }
\end{mathpar}

\begin{mathpar}
 \inferrule* [lab=behavior] {} {\meaningof{\langle a?b \rangle E} = \{ P \in \pi | P \equiv Q | u?(y)P', \\ \and \\\\ \and \\ \;\;\; u \in \meaningof{a}, \forall z.P'\{z/y\} \in \meaningof{E\{z/b\}}\}, \and \\ \meaningof{a!E} = \{ P \in \pi | P \equiv Q | x!\langle P' \rangle, x \in \meaningof{a} P' \in \meaningof{E}\} }
\end{mathpar}

\begin{mathpar}
 \inferrule* [lab=nominal] {} {\meaningof{\quotep{E}} = \{ \quotep{P} \in \quotep{\pi} | P \in \meaningof{E} \}, \and \meaningof{\quotep{P}} = \{ \quotep{Q} \in \quotep{\pi} | P \equiv Q \} \and \\ \meaningof{@\quotep{E}} = \{ P \in \pi | P \equiv @x, x \in \meaningof{E} \}}
\end{mathpar}

\begin{eqnarray*}
  \\
  \meaningof{-} : TS \to ST
\end{eqnarray*}

\begin{eqnarray*}
  \\
  L : TS \to ST
\end{eqnarray*}

\begin{eqnarray*}
  \\
  P \models E \iff P \in \meaningof{E}
\end{eqnarray*}

\begin{eqnarray*}
  P \approx_{L} Q \iff \forall E \in L. P \models E \iff Q \models E
\end{eqnarray*}

\begin{eqnarray*}
  P \approx_{K} Q
\end{eqnarray*}

\begin{eqnarray*}
  P \approx Q
\end{eqnarray*}

$\approx_{K} = \approx = \approx_{L}$

\subsubsection{Contextual duality}

Note that contexts extend the quotation operation to a family of
operations from processes to names. Given a context, $M$, we can
define a \emph{nominal context}, $\quotep{M}$ by $\quotep{M}[P] :=
\quotep{M[P]}$. To foreshadow what is to come we observe that these
operations enjoy a duality with processes very much like the duality
between vectors and maps from vectors to scalars.

Further, because the calculus is essentially higher-order, we have a
correspondence between contexts and processes. More specifically,
given a name $x$ and a context $M$ we can construct $M^{*}_{x}$ such
that 

\begin{mathpar}
  M^{*}_{x} | \lift{x}{P} \red M[P]
\end{mathpar}

namely,

\begin{mathpar}
  M^{*}_{x} := x?(u).M[\dropn{u}]
\end{mathpar}

The dependence of $M^{*}_{x}$ on a name makes it an abstraction, 

\begin{mathpar}
  M^{*} := (x)x?(u).M[\dropn{u}]
\end{mathpar}

\subsection{Additional notation}

It will sometimes be convenient to denote the process a name
quotes. We already have the notation $x = \quotep{P}$, but it will be
convenient to introduce an alternate notation, $\procn{x}$, when we
want to emphasize the connection to the use of the name. Note that, by
virtue of name equivalence, $\quotep{\procn{x}} \nameeq x$; so, the
notation is consistent with previous definitions.

Further, because names have structure it is possible to effect
substitutions on the basis of that structure. This means we need to
upgrade our notation for substitutions, which we accomplish by
adapting comprehension notation. Thus,

\begin{mathpar}
  P\{ y / x : x \in S \}
\end{mathpar}

is interpreted to mean the process derived from P by replacing (in a
capture-avoiding manner) each occurrence of $x$ in $S$ by $y$. For example,

\begin{mathpar}
  P\{ \quotep{\procn{x}|\procn{x}} / x : x \in \freenames{P} \}
\end{mathpar}

will replace each (occurrence) of a free name $x$ in $P$ by
$\quotep{\procn{x}|\procn{x}}$.

Also, we will avail ourselves of the notation $x^{L}$ and $x^{R}$ to
denote injections of a name into disjoint copies of the name
space. There are numerous ways to accomplish this. One example can be
found in \cite{MeredithR05}. This notation overloads to vectors of
names: $\vec{x}^{\pi} := (x_{i}^{\pi} \; : \; 0 \leq i < |\vec{x}| )$ where $\pi \in \{L,R\}$.

We also use $P^{\Box} := P|\Box$.

In \cite{MeredithR05} an interpretation of the new operator is
given. It turns out that there are several possible interpretations
all enjoying the requisite algebraic properties of the operator (see
\cite{milner91polyadicpi}). We will therefore make liberal use of
$(\nu\; \vec{x})P$.

% subsection the_syntax_and_semantics_of_the_notation_system (end)   

\input{qm2pi.qmops} 

\input{qm2pi.sterngerlach} 

\input{qm2pi.metric} 

% section concurrent_process_calculi (end)

%\input{qm2pi.proofsketch}

% section proof sketch (end)

%\input{qm2pi.slviaknots} 

% section spatial logic via knots (end)

\input{qm2pi.conclusion}

% section conclusion (end)

%\input{qm2pi.dtcodes} 

% section wiring algorithm (end)

\input{qm2pi.ack} 

% section acknowledgments (end)

\newpage


\bibliographystyle{plain}   
\bibliography{../../biblios/main.bib}

\input{qm2pi.rhodetails}

\end{document}

 

% subsection basic_interpretation (end)

%\input{qm2pi.rho.presentation} 
\subsection{The syntax and semantics of the notation system}\label{sub:the_syntax_and_semantics_of_the_notation_system} % (fold)

We now summarize a technical presentation of the calculus that
embodies our theory of dynamics. The typical presentation of such a
calculus follows the style of giving generators and relations on
them. The grammar, below, describing term constructors, freely
generates the set of processes, $\Proc$. This set is then quotiented
by a relation known as structural congruence and it is over this set
that the notion of dynamics is expressed. This presentation is
essentially that of \cite{MeredithR05} with the addition of
polyadicity and summation. For readability we have relegated some of
the technical subtleties to an appendix.

\subsubsection{Process grammar}\label{subsub:process_grammar}

\begin{mathpar}
  \inferrule* [lab=synchronization] {} {{M} \bc \pzero \;|\; x?F \;|\; x!C }
  \and
  \inferrule* [lab=abstraction] {} {{F} \bc (x)P}
  \and
  \inferrule* [lab=concretion] {} {{C} \bc \langle Q \rangle}
  \and
  \inferrule* [lab=process] {} {{P,Q} \bc M \;| \;P|Q \;|\; @{x}}
  \and
  \inferrule* [lab=name] {} {{x} \bc \quotep{P}}
\end{mathpar} 

Note that $\vec{x}$ (resp. $\vec{P}$) denotes a vector of names
(resp. processes) of length $|\vec{x}|$ (resp. $|\vec{P}|$). We adopt
the following useful abbreviations.

\begin{mathpar}
   x?(\vec{y}).P := x.(\vec{y})P \and  x\clift{\vec{P}} := x.\clift{\vec{P}}
   \and x!(y) := \lift{x}{\dropn{y}}
   \and \Pi_{i=0}^{n-1}P_i := P_0 | \ldots | P_{n-1}
\end{mathpar}

\subsubsection{Structural congruence}

\paragraph{Free and bound names and alpha-equivalence.} At the
core of structural equivalence is alpha-equivalence which identifies
process that are the same up to a change of variable. Formally, we
recognize the distinction between free and bound names. The free names
of a process, $\freenames{P}$, may be calculated recursively as
follows:

\begin{mathpar}
\freenames{\pzero} := \emptyset
  \and \\
  \freenames{x?(y).P} := \{ x \} \cup (\freenames{P} \setminus \{ y \})
  \and 
  \freenames{x!\langle P \rangle} := \{ x \} \cup \{ P \} 
  \and \\
  \freenames{P|Q} := \freenames{P} \cup \freenames{Q}
  \and \\
  \freenames{@{x}} := \{ x \}
\end{mathpar}

$\pi$
$\quotep{\pi}$

$\freenames{-} : \pi \to \mathcal{P}(\quotep{\pi})$

\begin{eqnarray*}
  \freenames{\pzero} & := & \emptyset \\
  \freenames{x?(y).P} & := & \{ x \} \cup (\freenames{P} \setminus \{ y \}) \\
  \freenames{x!\langle P \rangle} & := & \{ x \} \cup \{ P \} \\
  \freenames{P|Q} & := & \freenames{P} \cup \freenames{Q} \\
  \freenames{\dropn{x}} & := & \{ x \}
\end{eqnarray*}

The bound names of a process, $\boundnames{P}$, are those names occurring in $P$
that are not free. For example, in $x?(y).0$, the name $x$ is free, while $y$ is bound.

\begin{mathpar}
  \inferrule* [lab=monoidal-laws] {} { P|Q \equiv Q|P \and P|0 \equiv P \and P|(Q|R) \equiv (P|Q)|R }
\end{mathpar}

\begin{mathpar}
  \inferrule* [lab=alpha-equivalence] {} { (x)P \equiv (y)P\{y/x\} \and y \not\in \freenames{P} }
\end{mathpar}

\begin{definition}
Then two processes, $P,Q$, are alpha-equivalent if $P = Q\{\vec{y}/\vec{x}\}$ for
some $\vec{x} \in \boundnames{Q},\vec{y} \in \boundnames{P}$, where $Q\{\vec{y}/\vec{x}\}$
denotes the capture-avoiding substitution of $\vec{y}$ for $\vec{x}$ in $Q$.
\end{definition}

\begin{definition}
  The {\em structural congruence} \cite{SangiorgiWalker} , $\equiv$,
  between processes is the least congruence containing
  alpha-equivalence, satisfying the abelian monoid laws
  (associativity, commutativity and $\pzero$ as identity) for parallel
  composition $|$ and for summation $+$.
\end{definition}

\subsection{Name equivalence}

We take name equivalence, written $\nameeq$, to be the smallest
equivalence relation generated by the following rules.

\begin{mathpar}
\inferrule*[lab=Quote-drop]
{ }
{ \quotep{@{x}} \nameeq x }

\inferrule*[lab=Struct-equiv]
{ P \scong Q }
{ \quotep{P} \nameeq \quotep{Q} }
\end{mathpar}

The astute reader will have noticed that the mutual recursion of names
and processes imposes a mutual recursion on alpha-equivalence and
structural equivalence via name-equivalence. Fortunately, all of this
works out pleasantly and we may calculate in the natural way, free of
concern. The reader interested in the details is referred to the
appendix \ref{appendix:rho_details}.

\subsection{Substitution}

We use $\Proc$ for the set of processes, $\QProc$ for the set of
names, and $\id{\{}\vec{y} / \vec{x} \id{\}}$ to denote partial maps,
$s : \QProc \rightarrow \QProc$. A map, $s$ lifts, uniquely, to a map
on process terms, $\widehat{s} : \Proc \rightarrow \Proc$ by the
following equations.

\begin{mathpar}
  (0) \psubstp{Q}{P} := 0 \\
  (R \juxtap S) \psubstp{Q}{P}
  :=    
  (R)\psubstp{Q}{P} \juxtap (S) \psubstp{Q}{P} \\
  (x?(y).R) \psubstp{Q}{P}    
  :=    
  (x)\substp{Q}{P} (z)\concat( (R \psubstn{z}{y}) \psubstp{Q}{P} ) \\
  (\lift{x}{R}) \psubstp{Q}{P}  
  :=
  \lift{(x)\substp{Q}{P}}{ R \psubstp{Q}{P} } \\
%   (\dropn{x})  \psubstp{Q}{P}       
%   := 
%   \left\{ 
%     \begin{array}{ccc} 
%       \dropn{\quotep{Q}} & & x \nameeq \quotep{P} \\
%       \dropn{x} & & otherwise \\
%     \end{array}
%   \right. 
  (\dropn{x})  \psubstp{Q}{P}       
  := 
  \left\{ 
    \begin{array}{ccc} 
      Q & & x \nameeq \quotep{P} \\
      \dropn{x} & & otherwise \\
    \end{array}
  \right.
\end{mathpar}
 

where

\begin{eqnarray}
  (x)\id{\{} \lpquote Q \rpquote / \lpquote P \rpquote \id{\}}            = 
  \left\{ 
    \begin{array}{ccc}
      \lpquote Q \rpquote & & x \nameeq \lpquote P \rpquote \\
      x & & otherwise \\
    \end{array}
  \right. \nonumber
\end{eqnarray}

and $z$ is chosen distinct from $\quotep{P}$, $\quotep{Q}$, the free
names in $Q$, and all the names in $R$. Our $\alpha$-equivalence will
be built in the standard way from this substitution.

\begin{remark}\label{rem:no_self_referential_names}
  One consequence of these definitions is that $\forall P. \quotep{P}
  \not\in \freenames{P}$.
\end{remark}

\subsection{ Dynamic quote: an example }

Anticipating something of what's to come, consider applying the
substitution, $\widehat{\id{\{}u / z \id{\}}}$, to the following pair
of processes, $\lift{w}{y!(z)}$ and $w[ \lpquote y!(z) \rpquote ]$.

\begin{eqnarray}
	\lift{w}{y!(z)}\widehat{\id{\{}u / z \id{\}}}
		& = &
		\lift{w}{y!(u)} \nonumber\\
	w[ \lpquote y!(z) \rpquote ] \widehat{ \id{\{}u / z \id{\}} }
		& = &
		w[ \lpquote y!(z) \rpquote ] \nonumber
\end{eqnarray}

Because the body of the process between quotes is impervious to
substitution, we get radically different answers. In fact, by
examining the first process in an input context,
e.g. $x?(z).\lift{w}{y!(z)}$, we see that the process under the lift
operator may be shaped by prefixed inputs binding a name inside it. In
this sense, the lift operator will be seen as a way to dynamically
construct processes before reifying them as names.

Finally equipped with these standard features we can present the
dynamics of the calculus.

\subsubsection{Operational semantics} 

Finally, we introduce the computational dynamics. What marks these
algebras as distinct from other more traditionally studied algebraic
structures, e.g. vector spaces or polynomial rings, is the manner in
which dynamics is captured. In traditional structures, dynamics is typically
expressed through morphisms between such structures, as in linear maps
between vector spaces or morphisms between rings. In algebras
associated with the semantics of computation, the dynamics is
expressed as part of the algebraic structure itself, through a
reduction reduction relation typically denoted by $\red$. Below, we
give a recursive presentation of this relation for the calculus used
in the encoding.

$\red \subseteq \pi \times \pi$
$\red : \pi \to \mathcal{P}(\pi)$

\begin{mathpar}
  \inferrule* [lab=Comm] { \textsf{match}( x_{src}, x_{trgt} ) } { x_{trgt}?(y)P \; | \; x_{src}!\langle {Q} \rangle \red P\{\quotep{Q}/y}\} }
  \and \\
  \inferrule* [lab=Par] {{P} \red {P}'} {{{P} | {Q}} \red {{P}' | {Q}}}
  \and
  \inferrule* [lab=Equiv]{{{P} \scong {P}'} \andalso {{P}' \red {Q}'} \andalso {{Q}' \scong {Q}}}{{P} \red {Q}}
\end{mathpar}

\begin{eqnarray*}
  match_{\equiv} (\quotep{P},\quotep{Q}) & := & P \equiv Q \\
  match_{\dagger}(\quotep{P},\quotep{Q}) & := & \forall R. P|Q \red^{*} R => R \red^{*} 0 \\
  match_{K}(\quotep{P},\quotep{Q}) & := & K \mbox{ for some context } K
\end{eqnarray*}

$u?(x)P | u!\langle Q \rangle \red P\{\quotep{Q}/x\}$

%We write $\wred$ for $\red^*$, and $P\red$ if $\exists Q $ such that $ P \red Q$.
We write $P\red$ if $\exists Q $ such that $ P \red Q$ and $P\not\red$, otherwise.

\section{Replication}

As mentioned before, it is known that replication (and hence
recursion) can be implemented in a higher-order process algebra
\cite{SangiorgiWalker}. As our first example of calculation with the
machinery thus far presented we give the construction explicitly in
the {\rhoc}.

\begin{eqnarray}
	D_{x} & := & \prefix{x}{y}{(\binpar{\outputp{x}{y}}{@{y}})} \nonumber\\
	\bangp_{x}{P} & := & \binpar{{x}!\langle{\binpar{D_{x}}{P}}\rangle}{D_{x}} \nonumber
\end{eqnarray}

\begin{eqnarray}
	\bangp_{x}{P} & & \nonumber\\
	=
	& {x}!\langle{(\prefix{x}{y}{(\outputp{x}{y} | @{y})) | P}}\rangle 
	      | \prefix{x}{y}{(\outputp{x}{y} | @{y})} & \nonumber\\
	\red
	& (\outputp{x}{y} | @{y})\substn{\quotep{(\prefix{x}{y}{(@{y} | \outputp{x}{y})) | P}}}{y} & \nonumber\\
	=
	& \outputp{x}{\quotep{(\prefix{x}{y}{(\outputp{x}{y} | @{y})) | P}}}
	  | {(\prefix{x}{y}{(\outputp{x}{y} | @{y})) | P}} & \nonumber\\
	\red
	& \ldots & \nonumber\\
	\red^*
	& P | P | \ldots & \nonumber
\end{eqnarray}

Of course, this encoding, as an implementation, runs away, unfolding
$\bangp{P}$ eagerly. A lazier and more implementable replication
operator, restricted to input-guarded processes, may be obtained as follows.

\begin{eqnarray}
\bangp{\prefix{u}{v}{P}} 
	:= 
	\binpar{\lift{x}{\prefix{u}{v}{(\binpar{D(x)}{P})}}}{D(x)} \nonumber
\end{eqnarray}

\begin{remark}
  Note that the lazier definition still does not deal with summation
  or mixed summation (i.e. sums over input and output). The reader is
  invited to construct definitions of replication that deal with these
  features. 

  Further, the definitions are parameterized in a name, $x$. Can you,
  gentle reader, make a definition that eliminates this parameter and
  guarantees no accidental interaction between the replication
  machinery and the process being replicated -- i.e. no accidental
  sharing of names used by the process to get its work done and the
  name(s) used by the replication to effect copying. This latter
  revision of the definition of replication is crucial to obtaining
  the expected identity $!!P \sim !P$.
\end{remark}

\begin{remark}\label{rem:paradoxical_combinator}
  The reader familiar with the lambda calculus will have noticed the
  similarity between $D$ and the paradoxical combinator.

  [Ed. note: the existence of this seems to suggest we have to be more
  restrictive on the set of processes and names we admit if we are to
  support no-cloning.]
\end{remark}

\subsubsection{Bisimulation}

The computational dynamics gives rise to another kind of equivalence,
the equivalence of computational behavior. As previously mentioned
this is typically captured \emph{via} some form of bisimulation.

% The notion we use in this paper is weak barbed bisimulation
% \cite{milner91polyadicpi}.

The notion we use in this paper is derived from weak barbed
bisimulation \cite{milner91polyadicpi}. 

\begin{definition}
An \emph{observation relation}, $\downarrow_{\mathcal N}$, over a set
of names, $\mathcal N$, is the smallest relation satisfying the rules
below.

\infrule[Out-barb]{y \in {\mathcal N}, \; x \nameeq y}
		  {\outputp{x}{v} \downarrow_{\mathcal N} x}
\infrule[Par-barb]{\mbox{$P\downarrow_{\mathcal N} x$ or $Q\downarrow_{\mathcal N} x$}}
		  {\binpar{P}{Q} \downarrow_{\mathcal N} x}

We write $P \Downarrow_{\mathcal N} x$ if there is $Q$ such that 
$P \wred Q$ and $Q \downarrow_{\mathcal N} x$.
\end{definition}

\begin{definition}
%\label{def.bbisim}
An  ${\mathcal N}$-\emph{barbed bisimulation} over a set of names, ${\mathcal N}$, is a symmetric binary relation 
${\mathcal S}_{\mathcal N}$ between agents such that $P\rel{S}_{\mathcal N}Q$ implies:
\begin{enumerate}
\item If $P \red P'$ then $Q \wred Q'$ and $P'\rel{S}_{\mathcal N} Q'$.
\item If $P\downarrow_{\mathcal N} x$, then $Q\Downarrow_{\mathcal N} x$.
\end{enumerate}
$P$ is ${\mathcal N}$-barbed bisimilar to $Q$, written
$P \wbbisim_{\mathcal N} Q$, if $P \rel{S}_{\mathcal N} Q$ for some ${\mathcal N}$-barbed bisimulation ${\mathcal S}_{\mathcal N}$.
\end{definition}

$\mathcal{R} \subseteq \pi \times \pi$

$P \mathcal{R} Q => \forall P'. P \red P' \Rightarrow \exists Q'. Q \red Q', P' \mathcal{R} Q'$

$P \vdash x \Rightarrow Q \vdash x$

\begin{mathpar}
  \inferrule*[lab=Out-barb]{x \nameeq y}{{y}!\langle{Q}\rangle \vdash x}
  \and
  \inferrule*[lab=Par-barb]{\mbox{$P\vdash x$ or $Q\vdash x$}}{\binpar{P}{Q} \vdash x}
\end{mathpar}

\subsubsection{Contexts}

One of the principle advantages of computational calculi like the
$\pi$-calculus is a well-defined notion of context,
contextual-equivalence and a correlation between
contextual-equivalence and notions of bisimulation. The notion of
context allows the decomposition of a process into (sub-)process and
its syntactic environment, its context. Thus, a context may be
thought of as a process with a ``hole'' (written $\Box$) in it. The
application of a context $M$ to a process $P$, written $M[P]$, is
tantamount to filling the hole in $M$ with $P$. In this paper we do
not need the full weight of this theory, but do make use of the notion
of context in the proof the main theorem. 

\begin{mathpar}
  \inferrule* [lab=summation] {} {{M_{M},M_{N}} \bc \Box \;|\; x.M_{A} \;|\; M_{M}+M_{N}}
  \and
  \inferrule* [lab=agent] {} {{M_{A}} \bc (\vec{x})M_{P} \;| \; \clift{P_0,\ldots,M_{P},\ldots,P_N}}
  \and \\
  \inferrule* [lab=process] {} {{M_{P}} \bc M_{N} \;| \;P|M_{P} }
\end{mathpar} 

\begin{mathpar}
  \inferrule* [lab=sychronization] {} {M_{N} \bc \Box \;|\; x?M_{F} \;|\; x!M_{C}}
  \and
  \inferrule* [lab=abstraction] {} {{M_{F}} \bc (x)M_{P} }
  \and
  \inferrule* [lab=concretion] {} {{M_{C}} \bc \langle M_{P} \rangle }
  \and \\
  \inferrule* [lab=process] {} {{M_{P}} \bc M_{N} \;| \;P|M_{P} }
\end{mathpar}

\begin{definition}[contextual application] Given a context $M$, and
  process $P$, we define the \emph{contextual application}, $M[P] :=
  M\{P/\Box\}$. That is, the contextual application of M to P is the
  substitution of $P$ for $\Box$ in $M$.
\end{definition}

$\meaningof{-} : L \to \mathcal{P}(\pi)$

\begin{mathpar}
  \inferrule* [lab=collection] {} {\meaningof{true} = \pi, \and \meaningof{~E} = \pi \setminus \meaningof{E}, \and \meaningof{E_{1} \& E_{2}} = \meaningof{E_{1}} \cap \meaningof{E_{2}}}
\end{mathpar}

\begin{mathpar}
  \inferrule* [lab=structure] {} {\meaningof{0} = \{ P \in \pi | P \equiv 0 \}, \and \\ \meaningof{E_1 | E_2} = \{ P \in \pi | P \equiv P_{1} | P_{2}, P_{1} \in \meaningof{E_{1}}, P_{2} \in \meaningof{E_2}\} }
\end{mathpar}

\begin{mathpar}
 \inferrule* [lab=behavior] {} {\meaningof{\langle a?b \rangle E} = \{ P \in \pi | P \equiv Q | u?(y)P', \\ \and \\\\ \and \\ \;\;\; u \in \meaningof{a}, \forall z.P'\{z/y\} \in \meaningof{E\{z/b\}}\}, \and \\ \meaningof{a!E} = \{ P \in \pi | P \equiv Q | x!\langle P' \rangle, x \in \meaningof{a} P' \in \meaningof{E}\} }
\end{mathpar}

\begin{mathpar}
 \inferrule* [lab=nominal] {} {\meaningof{\quotep{E}} = \{ \quotep{P} \in \quotep{\pi} | P \in \meaningof{E} \}, \and \meaningof{\quotep{P}} = \{ \quotep{Q} \in \quotep{\pi} | P \equiv Q \} \and \\ \meaningof{@\quotep{E}} = \{ P \in \pi | P \equiv @x, x \in \meaningof{E} \}}
\end{mathpar}

\begin{eqnarray*}
  \\
  \meaningof{-} : TS \to ST
\end{eqnarray*}

\begin{eqnarray*}
  \\
  L : TS \to ST
\end{eqnarray*}

\begin{eqnarray*}
  \\
  P \models E \iff P \in \meaningof{E}
\end{eqnarray*}

\begin{eqnarray*}
  P \approx_{L} Q \iff \forall E \in L. P \models E \iff Q \models E
\end{eqnarray*}

\begin{eqnarray*}
  P \approx_{K} Q
\end{eqnarray*}

\begin{eqnarray*}
  P \approx Q
\end{eqnarray*}

$\approx_{K} = \approx = \approx_{L}$

\subsubsection{Contextual duality}

Note that contexts extend the quotation operation to a family of
operations from processes to names. Given a context, $M$, we can
define a \emph{nominal context}, $\quotep{M}$ by $\quotep{M}[P] :=
\quotep{M[P]}$. To foreshadow what is to come we observe that these
operations enjoy a duality with processes very much like the duality
between vectors and maps from vectors to scalars.

Further, because the calculus is essentially higher-order, we have a
correspondence between contexts and processes. More specifically,
given a name $x$ and a context $M$ we can construct $M^{*}_{x}$ such
that 

\begin{mathpar}
  M^{*}_{x} | \lift{x}{P} \red M[P]
\end{mathpar}

namely,

\begin{mathpar}
  M^{*}_{x} := x?(u).M[\dropn{u}]
\end{mathpar}

The dependence of $M^{*}_{x}$ on a name makes it an abstraction, 

\begin{mathpar}
  M^{*} := (x)x?(u).M[\dropn{u}]
\end{mathpar}

\subsection{Additional notation}

It will sometimes be convenient to denote the process a name
quotes. We already have the notation $x = \quotep{P}$, but it will be
convenient to introduce an alternate notation, $\procn{x}$, when we
want to emphasize the connection to the use of the name. Note that, by
virtue of name equivalence, $\quotep{\procn{x}} \nameeq x$; so, the
notation is consistent with previous definitions.

Further, because names have structure it is possible to effect
substitutions on the basis of that structure. This means we need to
upgrade our notation for substitutions, which we accomplish by
adapting comprehension notation. Thus,

\begin{mathpar}
  P\{ y / x : x \in S \}
\end{mathpar}

is interpreted to mean the process derived from P by replacing (in a
capture-avoiding manner) each occurrence of $x$ in $S$ by $y$. For example,

\begin{mathpar}
  P\{ \quotep{\procn{x}|\procn{x}} / x : x \in \freenames{P} \}
\end{mathpar}

will replace each (occurrence) of a free name $x$ in $P$ by
$\quotep{\procn{x}|\procn{x}}$.

Also, we will avail ourselves of the notation $x^{L}$ and $x^{R}$ to
denote injections of a name into disjoint copies of the name
space. There are numerous ways to accomplish this. One example can be
found in \cite{MeredithR05}. This notation overloads to vectors of
names: $\vec{x}^{\pi} := (x_{i}^{\pi} \; : \; 0 \leq i < |\vec{x}| )$ where $\pi \in \{L,R\}$.

We also use $P^{\Box} := P|\Box$.

In \cite{MeredithR05} an interpretation of the new operator is
given. It turns out that there are several possible interpretations
all enjoying the requisite algebraic properties of the operator (see
\cite{milner91polyadicpi}). We will therefore make liberal use of
$(\nu\; \vec{x})P$.

% subsection the_syntax_and_semantics_of_the_notation_system (end)   

\section{Interpretation of QM}
\subsection{Supporting definitions}
\subsubsection{Multiplication}
\begin{mathpar}
  \quotep{Q} \cdot \quotep{R} := \quotep{Q|R}
  \and \\
  \quotep{Q} \cdot P := P\{ \quotep{Q|R} / \quotep{R} : \quotep{R} \in \freenames{P} \}
\end{mathpar}

\paragraph{Discussion}
The first line needs little explanation. The second line says that
each free name of the process is replaced with the multiplication of
that name by the scalar. Multiplication of a scalar (name) by a state
(process) results in a process all the names of which have been `moved
over' by parallel composition with the process the scalar
quotes. There is a subtlety that the bound names have to be
manipulated so that multiplied names aren't accidentally
captured. There are many ways to achieve this.

\begin{remark}\label{rem:multiplication_identities}
  The reader is invited to verify that for all $x,y,z \in \QProc$ and $P \in \Proc$
  \begin{mathpar}
    x \cdot \quotep{0} \equiv x 
    \and
    x \cdot y \equiv y \cdot x
    \and
    x \cdot (y \cdot z) \equiv (x \cdot y) \cdot z
    \and \\
    \quotep{0} \cdot P \equiv P
    \and \\
    x \cdot (y \cdot P) \equiv (x \cdot y) \cdot P
    \and \\
    x \cdot (P|Q) \equiv (x \cdot P) | (x \cdot Q)
    \and \\    
  \end{mathpar}
\end{remark}

\subsubsection{Tensor product}

We define a tensor product on processes by structural induction.

\paragraph{Tensor of sums} First note that all summations, including
$\pzero$ and sequence, can be written $\Sigma_{i} x_{i}.A_{i} +
\Sigma_{j} x_{j}.C_{j}$, where we have grouped input-guarded processes
together and output-guarded processes together.

Thus, we can define the tensor product of two summations, $N_{1}\otimes N_{2}$, where

\begin{mathpar}
  N_{1} := \Sigma_{i} x_{i}.A_{i} + \Sigma_{j} x_{j}.C_{j}
  \and
  N_{2} := \Sigma_{i'} y_{i'}.B_{i'} + \Sigma_{j'} y_{j'}.D_{j'} 
\end{mathpar}

as follows.

\begin{mathpar}
  \Sigma_{i} x_{i}.A_{i} + \Sigma_{j} x_{j}.C_{j} \otimes \Sigma_{i'}
  y_{i'}.B_{i'} + \Sigma_{j'} y_{j'}.D_{j'} 
  \and \\
  := \; \Sigma_{i} \Sigma_{i'} \quotep{\stackrel{\vee}{x_{i}}| \stackrel{\vee}{y_{i'}}}.(A_{i}\otimes B_{i'}) \; | \; \Sigma_{i'} \Sigma_{i} \quotep{\stackrel{\vee}{y_{i'}}|\stackrel{\vee}{x_{i}}}.(B_{i'}\otimes A_{i})
  \and
  \;\; | \;\; \Sigma_{j} \Sigma_{j'} \quotep{\stackrel{\vee}{x_{j}}|\stackrel{\vee}{y_{j'}}}.(A_{j}\otimes B_{j'}) \; | \; \Sigma_{j'} \Sigma_{j} \quotep{\stackrel{\vee}{y_{j'}}|\stackrel{\vee}{x_{j}}}.(B_{j'}\otimes A_{j})
\end{mathpar}

\begin{remark}
  Do we need to $x^{L}$ and $y^{R}$ for this construction as well?
\end{remark}

\paragraph{Tensor of parallel compositions} Next, we distribute tensor
over par.

\begin{mathpar}
  P_{1}|P_{2} \otimes Q_{1}|Q_{2} := (P_{1} \otimes Q_{1}) | (P_{1}
  \otimes Q_{2}) | (P_{2} \otimes Q_{1}) | (P_{2} \otimes Q_{2})
\end{mathpar}

\paragraph{Tensor with dropped names} We treat tensor of a
process with a dropped name as parallel composition.

\begin{mathpar}
  P \otimes \dropn{x} := P | \dropn{x}
\end{mathpar}

\paragraph{Tensor of agents}

Finally, we need to define tensor on agents. Note that the definition
of tensor on normal products only tensors inputs with inputs and
outputs with outputs. Thus, we only have to define the operation on
``homogeneous'' pairings.

\begin{mathpar}
  (\vec{x})P \otimes (\vec{y})Q
  \and \\
  := (x_{0}^{L}|y_{0}^{R},\ldots,x_{0}^{L}|y_{n}^{R},\ldots,x_{m}^{L}|y_{0}^{R},\ldots,x_{m}^{L}|y_{n}^R)(P\{ \vec{x}^{L}/\vec{x}\} \otimes Q \{ \vec{y}^{R}/\vec{y}\})
  \and \\
  \clift{\vec{P}} \otimes \clift{\vec{Q}}
  \and \\
  := \clift{P_{0}\otimes Q_{0},\ldots,P_{0}\otimes Q_{n},\ldots,P_{m}\otimes Q_{0},\ldots,P_{m}\otimes Q_{n}}
\end{mathpar}

\begin{remark}
  Observe that arities of tensored abstractions matches arities of
  tensored concretions if the original arities matched. Note also that
  the length of the arities corresponds to the increase in dimension
  we see in ordinary vector space tensor product.
\end{remark}

\begin{remark}
  Operationally, this definition distributes the tensor down to
  components ``linked'' by summation. Tensor over summation is
  intriguing in that it mixes names. Moreover, as a consequence of the
  way it mixes names we have the identities for all $x \in \QProc$ and
  $P,Q \in \Proc$

  \begin{mathpar}
    (x \cdot P) \otimes Q \equiv x \cdot (P \otimes Q) \equiv P \otimes (x \cdot Q)
    \and
    P \otimes \pzero \equiv P
  \end{mathpar}

  that the reader is invited to verify.
\end{remark}

\subsubsection{Annihilation}
\begin{mathpar}
  P^{\perp} := \{ Q | \forall R. P|Q \red^{*} R \Rightarrow R \red^{*} \pzero \}
  \and \\
  P^{\underline{\perp}} := \Sigma_{Q \in P^{\perp}} \quotep{Q}?(y).(\dropn{y}|Q) | \Sigma_{Q \in P^{\perp}} \quotep{Q}\clift{\Box}
\end{mathpar}

\paragraph{Discussion} The reader will note that $P^{\perp}$ is a
\emph{set} of processes, while $P^{\underline{\perp}}$ is a
\emph{context}. We call the set $P^{\perp}$ the \emph{annihilators} of
$P$. The parallel composition of a process in the annihilators of $P$
with $P$ will result in a process, the state space of which has all
paths eventually leading to $\pzero$. Execution may endure loops; but
under reasonable conditions of fairness (naturally guaranteed under
most notions of bisimulation) such a composite process cannot get
stuck in such a loop and will, eventually pop out and terminate.

The context $P^{\underline{\perp}}$ is ready and willing to ``take the
$P$ out of'' the process to which it is applied. It will effectively
transmit the code of the process to which it is applied to one of the
annihilators and run the process against it.

\subsubsection{Evaluation}
We fix $M$ a domain of fully abstract interpretation with an equality
coincident with bisimulation. We take $\meaningof{\cdot} : \Proc \to
M$ to be the map interpreting processes and $\nmeaningof{\cdot} : \M
\to Proc$ to be the map running the other way. Then we define

\begin{mathpar}
  \int P := \nmeaningof{\meaningof{P}}
\end{mathpar}

\paragraph{Discussion}
There are many fully abstract interpretations of Milner's
$\pi$-calculus. Any of them can be used as a basis for interpreting
the reflective calculus here. Equipped with such a domain it is
largely a matter of grinding through to check that the Yoneda
construction for the normalization-by-evaluation program can be
extended to this setting.

\begin{remark}
  The reader is invited to verify that $\int (P^{\underline{\perp}}[P]) = 0$.
\end{remark}

\subsection{Quantum mechanics}

Table \ref{tbl:core_qm_op_defns} gives the core operational definitions

\begin{table}[htp]\label{tbl:core_qm_op_defns}
  \center{
    \fbox{
      \begin{tabular}{c|c}
        quantum mechanics & process calculus \\
        \hline
        scalar & $x := \quotep{P}$ \\
        state vector & $\state{P} := P$ \\
        dual & $\state{P}^{*} := \event{P^{\underline{\perp}}} := \quotep{P^{\underline{\perp}}}[-]$ \\
        matrix & $ \Sigma_{\alpha} \state{P_{\alpha}}x_{\alpha}\event{Q_{\alpha}}$ \\
        vector addition & $\state{P} + \state{Q} := \state{P | Q}$ \\
        tensor product & $\state{P} \otimes \state{Q} := \state{P \otimes Q}$ \\
        inner product & $\innerprod{P}{Q} := \quotep{\int P^{\underline{\perp}}[Q]}$ \\
      \end{tabular}
    }
  }
  \caption{QM - operational definitions}
\end{table}

where

\begin{mathpar}
  \prmatrix{P}{Q} := \fprmatrix{P}{\quotep{\pzero}}{Q}
  \and
  \fprmatrix{P}{x}{Q} := (\state{P},x,\event{Q})
  \and
  (\fprmatrix{P}{x}{Q})(\state{R}) := x \cdot \innerprod{Q}{R} \cdot \state{P}
  \and
  (\fprmatrix{P}{x}{Q})(\event{R}) := x \cdot \innerprod{R}{P} \cdot \event{Q}
\end{mathpar}

\paragraph{Discussion}
As promised: vectors (aka states) are represented as processes; duals
as contextual duals; inner product definition should be compared with
standard inner product definition for ....

\begin{remark}
  Assuming $\int (P^{\underline{\perp}}[P]) = 0$, the reader is
  invited to verify that $(\fprmatrix{P}{x}{P})(\state{P}) = x \cdot \state{P}$.
\end{remark}

\begin{remark}
  The reader is invited to verify that $\innerprod{P}{Q}$ could
  equally well have been written $\quotep{\int \stackrel{\vee}{x}}$
  where $x = \event{P^{\underline{\perp}}}(Q)$.

  One of the motivations for this remark is that there is another way
  to factor these operations. We could package up evaluation in the dual:

  \begin{mathpar}
    \state{P}^{*} := \event{\int P^{\underline{\perp}}} := \quotep{\int P^{\underline{\perp}}}[-]
  \end{mathpar}

  and then have inner product defined by
  
  \begin{mathpar}
    \innerprod{P}{Q} := \event{P}(Q)
  \end{mathpar}

  Hopefully, experience with the calculations will provide guidance on
  the best factoring.
\end{remark}

\begin{remark}
  Assuming $\int (P^{\underline{\perp}}[P]) = 0$, the reader is
  invited to verify that $\forall P,Q. (\prmatrix{0}{Q})(\state{0}) =
  \state{0}$ and dually $(\prmatrix{P}{0})(\event{0}) = \event{0}$.
\end{remark}

\begin{remark}
  i'm a little worried that i don't (yet) have proper support for
  complex conjugacy. But, the observation above may give us a
  clue. According to Abramsky, it must be the case that the scalars
  are iso to the homset of the identity for the tensor -- which the
  observation above characterizes. 

  For now, we will simply bookmark the notion with $\overline{x}$.
\end{remark}

\subsubsection{Adjointness}

We need to give a definition of $(\cdot)^{\dagger}$ for matrices. The
obvious candidate definition is
\begin{mathpar}
(\Sigma_{\alpha}\fprmatrix{P_{\alpha}}{x_{\alpha}}{Q_{\alpha}})^{\dagger}
= \Sigma_{\alpha}\fprmatrix{(Q_{\alpha}^{\underline{\perp}})^{*}}{\overline{x}_{\alpha}}{P_{\alpha}^{\underline{\perp}}} 
\end{mathpar}

But, $(Q_{\alpha}^{\underline{\perp}})^{*}$ requires a name along
which to communicate the process to achieve the context application.

\subsubsection{Basis for a basis}
If processes label states and ``addition'' of states (a.k.a. vector
addition) is interpreted as parallel composition, what corresponds to
notions of linear independence and basis? Here, we recall that Yoshida
has developed a set of \emph{combinators} for an asynchronous verison
of Milner's $\pi$-calculus. These are a finite set of processes such
any process can be expressed as parallel composition of these
combinators together with liberal uses of the new operator and
replication. We can simply give a translation of these into the
present calculus and have reasonable expectation that the property
carries over. That is, that the resultant set allows to express all
processes via parallel composition. Note, however, that there is no
new operator or replication in this calculus. As a result, we expect
that the corresponding set is actually infinite. That is, we expect
that the space is actually infinite dimensional.

\begin{remark}
  The attentive reader may be a bit concerned. Certainly, the
  collection $S$, $K$ and $I$ is a finite set of
  combinators. Shouldn't we expect to see a finite set of combinators
  for an effectively equivalent system? i am very sympathetic to this
  critique and feel it warrants full attention. On the other hand, i
  also have in mind the following analogy. The natural numbers, as a
  monoid under addition, has exactly $1$ generator, while the natural
  numbers, as a monoid under multiplication, has countably many
  generators (the primes). We observe that the application of the
  lambda calculus is much less resource sensitive than the parallel
  composition of the $\pi$-calculus. Could it be the case that we have
  an analogy of the form
  
  \begin{mathpar}
    m + n : MN :: m*n : M|N
  \end{mathpar}

  giving a similar blow up in the set of ``primes''?  This is such a
  wonderful thought that, even if it's not true, i think it's worth
  writing down.
\end{remark}
 

\documentclass[12pt]{llncs}
%\documentclass{jktr}

\usepackage[pdftex]{hyperref}                   
\usepackage {listings}
\usepackage {mathpartir}
\usepackage{bcprules}
%\usepackage{listings}
                       
\usepackage{graphicx} 
%\usepackage[margins=2.5cm,nohead,nofoot]{geometry}
%\usepackage{geometry}
\usepackage{amsfonts}
\usepackage{amstext}
\usepackage{latexsym}
\usepackage{amssymb}
\usepackage{color}


%\include{myPreamble}
\include{qm2pi.local} 

%\ifpdf
%\usepackage[pdftex]{graphicx}
%\else
%\usepackage{graphicx}
%\fi

 % \ifpdf
%  \usepackage{pdfsync}
%  \if


%\title{Brief Article}
%\author{David F. Snyder}
%\author{L.G. Meredith}

%\address{Dept. of Math., Texas State University--San Marcos, San Marcos, TX 78666}
       
\pagestyle{empty}


\begin{document}

\lstset{language=[Objective]Caml,frame=shadowbox}

\input{qm2pi.front}

% section front matter (end)

\input{qm2pi.intro} 
 
% section introduction (end)

% \input{qm2pi.knotations} 

% section notation (end)

\input{qm2pi.process.calculi} 

% section concurrent_process_calculi_and_spatial_logics_ (end)
    
%\input{qm2pi.knots2pi} 

%\input{qm2pi.trefoil} 

%\input{qm2pi.mainthm} 

% subsection basic_interpretation (end)

%\input{qm2pi.rho.presentation} 
\subsection{The syntax and semantics of the notation system}\label{sub:the_syntax_and_semantics_of_the_notation_system} % (fold)

We now summarize a technical presentation of the calculus that
embodies our theory of dynamics. The typical presentation of such a
calculus follows the style of giving generators and relations on
them. The grammar, below, describing term constructors, freely
generates the set of processes, $\Proc$. This set is then quotiented
by a relation known as structural congruence and it is over this set
that the notion of dynamics is expressed. This presentation is
essentially that of \cite{MeredithR05} with the addition of
polyadicity and summation. For readability we have relegated some of
the technical subtleties to an appendix.

\subsubsection{Process grammar}\label{subsub:process_grammar}

\begin{mathpar}
  \inferrule* [lab=synchronization] {} {{M} \bc \pzero \;|\; x?F \;|\; x!C }
  \and
  \inferrule* [lab=abstraction] {} {{F} \bc (x)P}
  \and
  \inferrule* [lab=concretion] {} {{C} \bc \langle Q \rangle}
  \and
  \inferrule* [lab=process] {} {{P,Q} \bc M \;| \;P|Q \;|\; @{x}}
  \and
  \inferrule* [lab=name] {} {{x} \bc \quotep{P}}
\end{mathpar} 

Note that $\vec{x}$ (resp. $\vec{P}$) denotes a vector of names
(resp. processes) of length $|\vec{x}|$ (resp. $|\vec{P}|$). We adopt
the following useful abbreviations.

\begin{mathpar}
   x?(\vec{y}).P := x.(\vec{y})P \and  x\clift{\vec{P}} := x.\clift{\vec{P}}
   \and x!(y) := \lift{x}{\dropn{y}}
   \and \Pi_{i=0}^{n-1}P_i := P_0 | \ldots | P_{n-1}
\end{mathpar}

\subsubsection{Structural congruence}

\paragraph{Free and bound names and alpha-equivalence.} At the
core of structural equivalence is alpha-equivalence which identifies
process that are the same up to a change of variable. Formally, we
recognize the distinction between free and bound names. The free names
of a process, $\freenames{P}$, may be calculated recursively as
follows:

\begin{mathpar}
\freenames{\pzero} := \emptyset
  \and \\
  \freenames{x?(y).P} := \{ x \} \cup (\freenames{P} \setminus \{ y \})
  \and 
  \freenames{x!\langle P \rangle} := \{ x \} \cup \{ P \} 
  \and \\
  \freenames{P|Q} := \freenames{P} \cup \freenames{Q}
  \and \\
  \freenames{@{x}} := \{ x \}
\end{mathpar}

$\pi$
$\quotep{\pi}$

$\freenames{-} : \pi \to \mathcal{P}(\quotep{\pi})$

\begin{eqnarray*}
  \freenames{\pzero} & := & \emptyset \\
  \freenames{x?(y).P} & := & \{ x \} \cup (\freenames{P} \setminus \{ y \}) \\
  \freenames{x!\langle P \rangle} & := & \{ x \} \cup \{ P \} \\
  \freenames{P|Q} & := & \freenames{P} \cup \freenames{Q} \\
  \freenames{\dropn{x}} & := & \{ x \}
\end{eqnarray*}

The bound names of a process, $\boundnames{P}$, are those names occurring in $P$
that are not free. For example, in $x?(y).0$, the name $x$ is free, while $y$ is bound.

\begin{mathpar}
  \inferrule* [lab=monoidal-laws] {} { P|Q \equiv Q|P \and P|0 \equiv P \and P|(Q|R) \equiv (P|Q)|R }
\end{mathpar}

\begin{mathpar}
  \inferrule* [lab=alpha-equivalence] {} { (x)P \equiv (y)P\{y/x\} \and y \not\in \freenames{P} }
\end{mathpar}

\begin{definition}
Then two processes, $P,Q$, are alpha-equivalent if $P = Q\{\vec{y}/\vec{x}\}$ for
some $\vec{x} \in \boundnames{Q},\vec{y} \in \boundnames{P}$, where $Q\{\vec{y}/\vec{x}\}$
denotes the capture-avoiding substitution of $\vec{y}$ for $\vec{x}$ in $Q$.
\end{definition}

\begin{definition}
  The {\em structural congruence} \cite{SangiorgiWalker} , $\equiv$,
  between processes is the least congruence containing
  alpha-equivalence, satisfying the abelian monoid laws
  (associativity, commutativity and $\pzero$ as identity) for parallel
  composition $|$ and for summation $+$.
\end{definition}

\subsection{Name equivalence}

We take name equivalence, written $\nameeq$, to be the smallest
equivalence relation generated by the following rules.

\begin{mathpar}
\inferrule*[lab=Quote-drop]
{ }
{ \quotep{@{x}} \nameeq x }

\inferrule*[lab=Struct-equiv]
{ P \scong Q }
{ \quotep{P} \nameeq \quotep{Q} }
\end{mathpar}

The astute reader will have noticed that the mutual recursion of names
and processes imposes a mutual recursion on alpha-equivalence and
structural equivalence via name-equivalence. Fortunately, all of this
works out pleasantly and we may calculate in the natural way, free of
concern. The reader interested in the details is referred to the
appendix \ref{appendix:rho_details}.

\subsection{Substitution}

We use $\Proc$ for the set of processes, $\QProc$ for the set of
names, and $\id{\{}\vec{y} / \vec{x} \id{\}}$ to denote partial maps,
$s : \QProc \rightarrow \QProc$. A map, $s$ lifts, uniquely, to a map
on process terms, $\widehat{s} : \Proc \rightarrow \Proc$ by the
following equations.

\begin{mathpar}
  (0) \psubstp{Q}{P} := 0 \\
  (R \juxtap S) \psubstp{Q}{P}
  :=    
  (R)\psubstp{Q}{P} \juxtap (S) \psubstp{Q}{P} \\
  (x?(y).R) \psubstp{Q}{P}    
  :=    
  (x)\substp{Q}{P} (z)\concat( (R \psubstn{z}{y}) \psubstp{Q}{P} ) \\
  (\lift{x}{R}) \psubstp{Q}{P}  
  :=
  \lift{(x)\substp{Q}{P}}{ R \psubstp{Q}{P} } \\
%   (\dropn{x})  \psubstp{Q}{P}       
%   := 
%   \left\{ 
%     \begin{array}{ccc} 
%       \dropn{\quotep{Q}} & & x \nameeq \quotep{P} \\
%       \dropn{x} & & otherwise \\
%     \end{array}
%   \right. 
  (\dropn{x})  \psubstp{Q}{P}       
  := 
  \left\{ 
    \begin{array}{ccc} 
      Q & & x \nameeq \quotep{P} \\
      \dropn{x} & & otherwise \\
    \end{array}
  \right.
\end{mathpar}
 

where

\begin{eqnarray}
  (x)\id{\{} \lpquote Q \rpquote / \lpquote P \rpquote \id{\}}            = 
  \left\{ 
    \begin{array}{ccc}
      \lpquote Q \rpquote & & x \nameeq \lpquote P \rpquote \\
      x & & otherwise \\
    \end{array}
  \right. \nonumber
\end{eqnarray}

and $z$ is chosen distinct from $\quotep{P}$, $\quotep{Q}$, the free
names in $Q$, and all the names in $R$. Our $\alpha$-equivalence will
be built in the standard way from this substitution.

\begin{remark}\label{rem:no_self_referential_names}
  One consequence of these definitions is that $\forall P. \quotep{P}
  \not\in \freenames{P}$.
\end{remark}

\subsection{ Dynamic quote: an example }

Anticipating something of what's to come, consider applying the
substitution, $\widehat{\id{\{}u / z \id{\}}}$, to the following pair
of processes, $\lift{w}{y!(z)}$ and $w[ \lpquote y!(z) \rpquote ]$.

\begin{eqnarray}
	\lift{w}{y!(z)}\widehat{\id{\{}u / z \id{\}}}
		& = &
		\lift{w}{y!(u)} \nonumber\\
	w[ \lpquote y!(z) \rpquote ] \widehat{ \id{\{}u / z \id{\}} }
		& = &
		w[ \lpquote y!(z) \rpquote ] \nonumber
\end{eqnarray}

Because the body of the process between quotes is impervious to
substitution, we get radically different answers. In fact, by
examining the first process in an input context,
e.g. $x?(z).\lift{w}{y!(z)}$, we see that the process under the lift
operator may be shaped by prefixed inputs binding a name inside it. In
this sense, the lift operator will be seen as a way to dynamically
construct processes before reifying them as names.

Finally equipped with these standard features we can present the
dynamics of the calculus.

\subsubsection{Operational semantics} 

Finally, we introduce the computational dynamics. What marks these
algebras as distinct from other more traditionally studied algebraic
structures, e.g. vector spaces or polynomial rings, is the manner in
which dynamics is captured. In traditional structures, dynamics is typically
expressed through morphisms between such structures, as in linear maps
between vector spaces or morphisms between rings. In algebras
associated with the semantics of computation, the dynamics is
expressed as part of the algebraic structure itself, through a
reduction reduction relation typically denoted by $\red$. Below, we
give a recursive presentation of this relation for the calculus used
in the encoding.

$\red \subseteq \pi \times \pi$
$\red : \pi \to \mathcal{P}(\pi)$

\begin{mathpar}
  \inferrule* [lab=Comm] { \textsf{match}( x_{src}, x_{trgt} ) } { x_{trgt}?(y)P \; | \; x_{src}!\langle {Q} \rangle \red P\{\quotep{Q}/y}\} }
  \and \\
  \inferrule* [lab=Par] {{P} \red {P}'} {{{P} | {Q}} \red {{P}' | {Q}}}
  \and
  \inferrule* [lab=Equiv]{{{P} \scong {P}'} \andalso {{P}' \red {Q}'} \andalso {{Q}' \scong {Q}}}{{P} \red {Q}}
\end{mathpar}

\begin{eqnarray*}
  match_{\equiv} (\quotep{P},\quotep{Q}) & := & P \equiv Q \\
  match_{\dagger}(\quotep{P},\quotep{Q}) & := & \forall R. P|Q \red^{*} R => R \red^{*} 0 \\
  match_{K}(\quotep{P},\quotep{Q}) & := & K \mbox{ for some context } K
\end{eqnarray*}

$u?(x)P | u!\langle Q \rangle \red P\{\quotep{Q}/x\}$

%We write $\wred$ for $\red^*$, and $P\red$ if $\exists Q $ such that $ P \red Q$.
We write $P\red$ if $\exists Q $ such that $ P \red Q$ and $P\not\red$, otherwise.

\section{Replication}

As mentioned before, it is known that replication (and hence
recursion) can be implemented in a higher-order process algebra
\cite{SangiorgiWalker}. As our first example of calculation with the
machinery thus far presented we give the construction explicitly in
the {\rhoc}.

\begin{eqnarray}
	D_{x} & := & \prefix{x}{y}{(\binpar{\outputp{x}{y}}{@{y}})} \nonumber\\
	\bangp_{x}{P} & := & \binpar{{x}!\langle{\binpar{D_{x}}{P}}\rangle}{D_{x}} \nonumber
\end{eqnarray}

\begin{eqnarray}
	\bangp_{x}{P} & & \nonumber\\
	=
	& {x}!\langle{(\prefix{x}{y}{(\outputp{x}{y} | @{y})) | P}}\rangle 
	      | \prefix{x}{y}{(\outputp{x}{y} | @{y})} & \nonumber\\
	\red
	& (\outputp{x}{y} | @{y})\substn{\quotep{(\prefix{x}{y}{(@{y} | \outputp{x}{y})) | P}}}{y} & \nonumber\\
	=
	& \outputp{x}{\quotep{(\prefix{x}{y}{(\outputp{x}{y} | @{y})) | P}}}
	  | {(\prefix{x}{y}{(\outputp{x}{y} | @{y})) | P}} & \nonumber\\
	\red
	& \ldots & \nonumber\\
	\red^*
	& P | P | \ldots & \nonumber
\end{eqnarray}

Of course, this encoding, as an implementation, runs away, unfolding
$\bangp{P}$ eagerly. A lazier and more implementable replication
operator, restricted to input-guarded processes, may be obtained as follows.

\begin{eqnarray}
\bangp{\prefix{u}{v}{P}} 
	:= 
	\binpar{\lift{x}{\prefix{u}{v}{(\binpar{D(x)}{P})}}}{D(x)} \nonumber
\end{eqnarray}

\begin{remark}
  Note that the lazier definition still does not deal with summation
  or mixed summation (i.e. sums over input and output). The reader is
  invited to construct definitions of replication that deal with these
  features. 

  Further, the definitions are parameterized in a name, $x$. Can you,
  gentle reader, make a definition that eliminates this parameter and
  guarantees no accidental interaction between the replication
  machinery and the process being replicated -- i.e. no accidental
  sharing of names used by the process to get its work done and the
  name(s) used by the replication to effect copying. This latter
  revision of the definition of replication is crucial to obtaining
  the expected identity $!!P \sim !P$.
\end{remark}

\begin{remark}\label{rem:paradoxical_combinator}
  The reader familiar with the lambda calculus will have noticed the
  similarity between $D$ and the paradoxical combinator.

  [Ed. note: the existence of this seems to suggest we have to be more
  restrictive on the set of processes and names we admit if we are to
  support no-cloning.]
\end{remark}

\subsubsection{Bisimulation}

The computational dynamics gives rise to another kind of equivalence,
the equivalence of computational behavior. As previously mentioned
this is typically captured \emph{via} some form of bisimulation.

% The notion we use in this paper is weak barbed bisimulation
% \cite{milner91polyadicpi}.

The notion we use in this paper is derived from weak barbed
bisimulation \cite{milner91polyadicpi}. 

\begin{definition}
An \emph{observation relation}, $\downarrow_{\mathcal N}$, over a set
of names, $\mathcal N$, is the smallest relation satisfying the rules
below.

\infrule[Out-barb]{y \in {\mathcal N}, \; x \nameeq y}
		  {\outputp{x}{v} \downarrow_{\mathcal N} x}
\infrule[Par-barb]{\mbox{$P\downarrow_{\mathcal N} x$ or $Q\downarrow_{\mathcal N} x$}}
		  {\binpar{P}{Q} \downarrow_{\mathcal N} x}

We write $P \Downarrow_{\mathcal N} x$ if there is $Q$ such that 
$P \wred Q$ and $Q \downarrow_{\mathcal N} x$.
\end{definition}

\begin{definition}
%\label{def.bbisim}
An  ${\mathcal N}$-\emph{barbed bisimulation} over a set of names, ${\mathcal N}$, is a symmetric binary relation 
${\mathcal S}_{\mathcal N}$ between agents such that $P\rel{S}_{\mathcal N}Q$ implies:
\begin{enumerate}
\item If $P \red P'$ then $Q \wred Q'$ and $P'\rel{S}_{\mathcal N} Q'$.
\item If $P\downarrow_{\mathcal N} x$, then $Q\Downarrow_{\mathcal N} x$.
\end{enumerate}
$P$ is ${\mathcal N}$-barbed bisimilar to $Q$, written
$P \wbbisim_{\mathcal N} Q$, if $P \rel{S}_{\mathcal N} Q$ for some ${\mathcal N}$-barbed bisimulation ${\mathcal S}_{\mathcal N}$.
\end{definition}

$\mathcal{R} \subseteq \pi \times \pi$

$P \mathcal{R} Q => \forall P'. P \red P' \Rightarrow \exists Q'. Q \red Q', P' \mathcal{R} Q'$

$P \vdash x \Rightarrow Q \vdash x$

\begin{mathpar}
  \inferrule*[lab=Out-barb]{x \nameeq y}{{y}!\langle{Q}\rangle \vdash x}
  \and
  \inferrule*[lab=Par-barb]{\mbox{$P\vdash x$ or $Q\vdash x$}}{\binpar{P}{Q} \vdash x}
\end{mathpar}

\subsubsection{Contexts}

One of the principle advantages of computational calculi like the
$\pi$-calculus is a well-defined notion of context,
contextual-equivalence and a correlation between
contextual-equivalence and notions of bisimulation. The notion of
context allows the decomposition of a process into (sub-)process and
its syntactic environment, its context. Thus, a context may be
thought of as a process with a ``hole'' (written $\Box$) in it. The
application of a context $M$ to a process $P$, written $M[P]$, is
tantamount to filling the hole in $M$ with $P$. In this paper we do
not need the full weight of this theory, but do make use of the notion
of context in the proof the main theorem. 

\begin{mathpar}
  \inferrule* [lab=summation] {} {{M_{M},M_{N}} \bc \Box \;|\; x.M_{A} \;|\; M_{M}+M_{N}}
  \and
  \inferrule* [lab=agent] {} {{M_{A}} \bc (\vec{x})M_{P} \;| \; \clift{P_0,\ldots,M_{P},\ldots,P_N}}
  \and \\
  \inferrule* [lab=process] {} {{M_{P}} \bc M_{N} \;| \;P|M_{P} }
\end{mathpar} 

\begin{mathpar}
  \inferrule* [lab=sychronization] {} {M_{N} \bc \Box \;|\; x?M_{F} \;|\; x!M_{C}}
  \and
  \inferrule* [lab=abstraction] {} {{M_{F}} \bc (x)M_{P} }
  \and
  \inferrule* [lab=concretion] {} {{M_{C}} \bc \langle M_{P} \rangle }
  \and \\
  \inferrule* [lab=process] {} {{M_{P}} \bc M_{N} \;| \;P|M_{P} }
\end{mathpar}

\begin{definition}[contextual application] Given a context $M$, and
  process $P$, we define the \emph{contextual application}, $M[P] :=
  M\{P/\Box\}$. That is, the contextual application of M to P is the
  substitution of $P$ for $\Box$ in $M$.
\end{definition}

$\meaningof{-} : L \to \mathcal{P}(\pi)$

\begin{mathpar}
  \inferrule* [lab=collection] {} {\meaningof{true} = \pi, \and \meaningof{~E} = \pi \setminus \meaningof{E}, \and \meaningof{E_{1} \& E_{2}} = \meaningof{E_{1}} \cap \meaningof{E_{2}}}
\end{mathpar}

\begin{mathpar}
  \inferrule* [lab=structure] {} {\meaningof{0} = \{ P \in \pi | P \equiv 0 \}, \and \\ \meaningof{E_1 | E_2} = \{ P \in \pi | P \equiv P_{1} | P_{2}, P_{1} \in \meaningof{E_{1}}, P_{2} \in \meaningof{E_2}\} }
\end{mathpar}

\begin{mathpar}
 \inferrule* [lab=behavior] {} {\meaningof{\langle a?b \rangle E} = \{ P \in \pi | P \equiv Q | u?(y)P', \\ \and \\\\ \and \\ \;\;\; u \in \meaningof{a}, \forall z.P'\{z/y\} \in \meaningof{E\{z/b\}}\}, \and \\ \meaningof{a!E} = \{ P \in \pi | P \equiv Q | x!\langle P' \rangle, x \in \meaningof{a} P' \in \meaningof{E}\} }
\end{mathpar}

\begin{mathpar}
 \inferrule* [lab=nominal] {} {\meaningof{\quotep{E}} = \{ \quotep{P} \in \quotep{\pi} | P \in \meaningof{E} \}, \and \meaningof{\quotep{P}} = \{ \quotep{Q} \in \quotep{\pi} | P \equiv Q \} \and \\ \meaningof{@\quotep{E}} = \{ P \in \pi | P \equiv @x, x \in \meaningof{E} \}}
\end{mathpar}

\begin{eqnarray*}
  \\
  \meaningof{-} : TS \to ST
\end{eqnarray*}

\begin{eqnarray*}
  \\
  L : TS \to ST
\end{eqnarray*}

\begin{eqnarray*}
  \\
  P \models E \iff P \in \meaningof{E}
\end{eqnarray*}

\begin{eqnarray*}
  P \approx_{L} Q \iff \forall E \in L. P \models E \iff Q \models E
\end{eqnarray*}

\begin{eqnarray*}
  P \approx_{K} Q
\end{eqnarray*}

\begin{eqnarray*}
  P \approx Q
\end{eqnarray*}

$\approx_{K} = \approx = \approx_{L}$

\subsubsection{Contextual duality}

Note that contexts extend the quotation operation to a family of
operations from processes to names. Given a context, $M$, we can
define a \emph{nominal context}, $\quotep{M}$ by $\quotep{M}[P] :=
\quotep{M[P]}$. To foreshadow what is to come we observe that these
operations enjoy a duality with processes very much like the duality
between vectors and maps from vectors to scalars.

Further, because the calculus is essentially higher-order, we have a
correspondence between contexts and processes. More specifically,
given a name $x$ and a context $M$ we can construct $M^{*}_{x}$ such
that 

\begin{mathpar}
  M^{*}_{x} | \lift{x}{P} \red M[P]
\end{mathpar}

namely,

\begin{mathpar}
  M^{*}_{x} := x?(u).M[\dropn{u}]
\end{mathpar}

The dependence of $M^{*}_{x}$ on a name makes it an abstraction, 

\begin{mathpar}
  M^{*} := (x)x?(u).M[\dropn{u}]
\end{mathpar}

\subsection{Additional notation}

It will sometimes be convenient to denote the process a name
quotes. We already have the notation $x = \quotep{P}$, but it will be
convenient to introduce an alternate notation, $\procn{x}$, when we
want to emphasize the connection to the use of the name. Note that, by
virtue of name equivalence, $\quotep{\procn{x}} \nameeq x$; so, the
notation is consistent with previous definitions.

Further, because names have structure it is possible to effect
substitutions on the basis of that structure. This means we need to
upgrade our notation for substitutions, which we accomplish by
adapting comprehension notation. Thus,

\begin{mathpar}
  P\{ y / x : x \in S \}
\end{mathpar}

is interpreted to mean the process derived from P by replacing (in a
capture-avoiding manner) each occurrence of $x$ in $S$ by $y$. For example,

\begin{mathpar}
  P\{ \quotep{\procn{x}|\procn{x}} / x : x \in \freenames{P} \}
\end{mathpar}

will replace each (occurrence) of a free name $x$ in $P$ by
$\quotep{\procn{x}|\procn{x}}$.

Also, we will avail ourselves of the notation $x^{L}$ and $x^{R}$ to
denote injections of a name into disjoint copies of the name
space. There are numerous ways to accomplish this. One example can be
found in \cite{MeredithR05}. This notation overloads to vectors of
names: $\vec{x}^{\pi} := (x_{i}^{\pi} \; : \; 0 \leq i < |\vec{x}| )$ where $\pi \in \{L,R\}$.

We also use $P^{\Box} := P|\Box$.

In \cite{MeredithR05} an interpretation of the new operator is
given. It turns out that there are several possible interpretations
all enjoying the requisite algebraic properties of the operator (see
\cite{milner91polyadicpi}). We will therefore make liberal use of
$(\nu\; \vec{x})P$.

% subsection the_syntax_and_semantics_of_the_notation_system (end)   

\input{qm2pi.qmops} 

\input{qm2pi.sterngerlach} 

\input{qm2pi.metric} 

% section concurrent_process_calculi (end)

%\input{qm2pi.proofsketch}

% section proof sketch (end)

%\input{qm2pi.slviaknots} 

% section spatial logic via knots (end)

\input{qm2pi.conclusion}

% section conclusion (end)

%\input{qm2pi.dtcodes} 

% section wiring algorithm (end)

\input{qm2pi.ack} 

% section acknowledgments (end)

\newpage


\bibliographystyle{plain}   
\bibliography{../../biblios/main.bib}

\input{qm2pi.rhodetails}

\end{document}

 

\documentclass[12pt]{llncs}
%\documentclass{jktr}

\usepackage[pdftex]{hyperref}                   
\usepackage {listings}
\usepackage {mathpartir}
\usepackage{bcprules}
%\usepackage{listings}
                       
\usepackage{graphicx} 
%\usepackage[margins=2.5cm,nohead,nofoot]{geometry}
%\usepackage{geometry}
\usepackage{amsfonts}
\usepackage{amstext}
\usepackage{latexsym}
\usepackage{amssymb}
\usepackage{color}


%\include{myPreamble}
\include{qm2pi.local} 

%\ifpdf
%\usepackage[pdftex]{graphicx}
%\else
%\usepackage{graphicx}
%\fi

 % \ifpdf
%  \usepackage{pdfsync}
%  \if


%\title{Brief Article}
%\author{David F. Snyder}
%\author{L.G. Meredith}

%\address{Dept. of Math., Texas State University--San Marcos, San Marcos, TX 78666}
       
\pagestyle{empty}


\begin{document}

\lstset{language=[Objective]Caml,frame=shadowbox}

\input{qm2pi.front}

% section front matter (end)

\input{qm2pi.intro} 
 
% section introduction (end)

% \input{qm2pi.knotations} 

% section notation (end)

\input{qm2pi.process.calculi} 

% section concurrent_process_calculi_and_spatial_logics_ (end)
    
%\input{qm2pi.knots2pi} 

%\input{qm2pi.trefoil} 

%\input{qm2pi.mainthm} 

% subsection basic_interpretation (end)

%\input{qm2pi.rho.presentation} 
\subsection{The syntax and semantics of the notation system}\label{sub:the_syntax_and_semantics_of_the_notation_system} % (fold)

We now summarize a technical presentation of the calculus that
embodies our theory of dynamics. The typical presentation of such a
calculus follows the style of giving generators and relations on
them. The grammar, below, describing term constructors, freely
generates the set of processes, $\Proc$. This set is then quotiented
by a relation known as structural congruence and it is over this set
that the notion of dynamics is expressed. This presentation is
essentially that of \cite{MeredithR05} with the addition of
polyadicity and summation. For readability we have relegated some of
the technical subtleties to an appendix.

\subsubsection{Process grammar}\label{subsub:process_grammar}

\begin{mathpar}
  \inferrule* [lab=synchronization] {} {{M} \bc \pzero \;|\; x?F \;|\; x!C }
  \and
  \inferrule* [lab=abstraction] {} {{F} \bc (x)P}
  \and
  \inferrule* [lab=concretion] {} {{C} \bc \langle Q \rangle}
  \and
  \inferrule* [lab=process] {} {{P,Q} \bc M \;| \;P|Q \;|\; @{x}}
  \and
  \inferrule* [lab=name] {} {{x} \bc \quotep{P}}
\end{mathpar} 

Note that $\vec{x}$ (resp. $\vec{P}$) denotes a vector of names
(resp. processes) of length $|\vec{x}|$ (resp. $|\vec{P}|$). We adopt
the following useful abbreviations.

\begin{mathpar}
   x?(\vec{y}).P := x.(\vec{y})P \and  x\clift{\vec{P}} := x.\clift{\vec{P}}
   \and x!(y) := \lift{x}{\dropn{y}}
   \and \Pi_{i=0}^{n-1}P_i := P_0 | \ldots | P_{n-1}
\end{mathpar}

\subsubsection{Structural congruence}

\paragraph{Free and bound names and alpha-equivalence.} At the
core of structural equivalence is alpha-equivalence which identifies
process that are the same up to a change of variable. Formally, we
recognize the distinction between free and bound names. The free names
of a process, $\freenames{P}$, may be calculated recursively as
follows:

\begin{mathpar}
\freenames{\pzero} := \emptyset
  \and \\
  \freenames{x?(y).P} := \{ x \} \cup (\freenames{P} \setminus \{ y \})
  \and 
  \freenames{x!\langle P \rangle} := \{ x \} \cup \{ P \} 
  \and \\
  \freenames{P|Q} := \freenames{P} \cup \freenames{Q}
  \and \\
  \freenames{@{x}} := \{ x \}
\end{mathpar}

$\pi$
$\quotep{\pi}$

$\freenames{-} : \pi \to \mathcal{P}(\quotep{\pi})$

\begin{eqnarray*}
  \freenames{\pzero} & := & \emptyset \\
  \freenames{x?(y).P} & := & \{ x \} \cup (\freenames{P} \setminus \{ y \}) \\
  \freenames{x!\langle P \rangle} & := & \{ x \} \cup \{ P \} \\
  \freenames{P|Q} & := & \freenames{P} \cup \freenames{Q} \\
  \freenames{\dropn{x}} & := & \{ x \}
\end{eqnarray*}

The bound names of a process, $\boundnames{P}$, are those names occurring in $P$
that are not free. For example, in $x?(y).0$, the name $x$ is free, while $y$ is bound.

\begin{mathpar}
  \inferrule* [lab=monoidal-laws] {} { P|Q \equiv Q|P \and P|0 \equiv P \and P|(Q|R) \equiv (P|Q)|R }
\end{mathpar}

\begin{mathpar}
  \inferrule* [lab=alpha-equivalence] {} { (x)P \equiv (y)P\{y/x\} \and y \not\in \freenames{P} }
\end{mathpar}

\begin{definition}
Then two processes, $P,Q$, are alpha-equivalent if $P = Q\{\vec{y}/\vec{x}\}$ for
some $\vec{x} \in \boundnames{Q},\vec{y} \in \boundnames{P}$, where $Q\{\vec{y}/\vec{x}\}$
denotes the capture-avoiding substitution of $\vec{y}$ for $\vec{x}$ in $Q$.
\end{definition}

\begin{definition}
  The {\em structural congruence} \cite{SangiorgiWalker} , $\equiv$,
  between processes is the least congruence containing
  alpha-equivalence, satisfying the abelian monoid laws
  (associativity, commutativity and $\pzero$ as identity) for parallel
  composition $|$ and for summation $+$.
\end{definition}

\subsection{Name equivalence}

We take name equivalence, written $\nameeq$, to be the smallest
equivalence relation generated by the following rules.

\begin{mathpar}
\inferrule*[lab=Quote-drop]
{ }
{ \quotep{@{x}} \nameeq x }

\inferrule*[lab=Struct-equiv]
{ P \scong Q }
{ \quotep{P} \nameeq \quotep{Q} }
\end{mathpar}

The astute reader will have noticed that the mutual recursion of names
and processes imposes a mutual recursion on alpha-equivalence and
structural equivalence via name-equivalence. Fortunately, all of this
works out pleasantly and we may calculate in the natural way, free of
concern. The reader interested in the details is referred to the
appendix \ref{appendix:rho_details}.

\subsection{Substitution}

We use $\Proc$ for the set of processes, $\QProc$ for the set of
names, and $\id{\{}\vec{y} / \vec{x} \id{\}}$ to denote partial maps,
$s : \QProc \rightarrow \QProc$. A map, $s$ lifts, uniquely, to a map
on process terms, $\widehat{s} : \Proc \rightarrow \Proc$ by the
following equations.

\begin{mathpar}
  (0) \psubstp{Q}{P} := 0 \\
  (R \juxtap S) \psubstp{Q}{P}
  :=    
  (R)\psubstp{Q}{P} \juxtap (S) \psubstp{Q}{P} \\
  (x?(y).R) \psubstp{Q}{P}    
  :=    
  (x)\substp{Q}{P} (z)\concat( (R \psubstn{z}{y}) \psubstp{Q}{P} ) \\
  (\lift{x}{R}) \psubstp{Q}{P}  
  :=
  \lift{(x)\substp{Q}{P}}{ R \psubstp{Q}{P} } \\
%   (\dropn{x})  \psubstp{Q}{P}       
%   := 
%   \left\{ 
%     \begin{array}{ccc} 
%       \dropn{\quotep{Q}} & & x \nameeq \quotep{P} \\
%       \dropn{x} & & otherwise \\
%     \end{array}
%   \right. 
  (\dropn{x})  \psubstp{Q}{P}       
  := 
  \left\{ 
    \begin{array}{ccc} 
      Q & & x \nameeq \quotep{P} \\
      \dropn{x} & & otherwise \\
    \end{array}
  \right.
\end{mathpar}
 

where

\begin{eqnarray}
  (x)\id{\{} \lpquote Q \rpquote / \lpquote P \rpquote \id{\}}            = 
  \left\{ 
    \begin{array}{ccc}
      \lpquote Q \rpquote & & x \nameeq \lpquote P \rpquote \\
      x & & otherwise \\
    \end{array}
  \right. \nonumber
\end{eqnarray}

and $z$ is chosen distinct from $\quotep{P}$, $\quotep{Q}$, the free
names in $Q$, and all the names in $R$. Our $\alpha$-equivalence will
be built in the standard way from this substitution.

\begin{remark}\label{rem:no_self_referential_names}
  One consequence of these definitions is that $\forall P. \quotep{P}
  \not\in \freenames{P}$.
\end{remark}

\subsection{ Dynamic quote: an example }

Anticipating something of what's to come, consider applying the
substitution, $\widehat{\id{\{}u / z \id{\}}}$, to the following pair
of processes, $\lift{w}{y!(z)}$ and $w[ \lpquote y!(z) \rpquote ]$.

\begin{eqnarray}
	\lift{w}{y!(z)}\widehat{\id{\{}u / z \id{\}}}
		& = &
		\lift{w}{y!(u)} \nonumber\\
	w[ \lpquote y!(z) \rpquote ] \widehat{ \id{\{}u / z \id{\}} }
		& = &
		w[ \lpquote y!(z) \rpquote ] \nonumber
\end{eqnarray}

Because the body of the process between quotes is impervious to
substitution, we get radically different answers. In fact, by
examining the first process in an input context,
e.g. $x?(z).\lift{w}{y!(z)}$, we see that the process under the lift
operator may be shaped by prefixed inputs binding a name inside it. In
this sense, the lift operator will be seen as a way to dynamically
construct processes before reifying them as names.

Finally equipped with these standard features we can present the
dynamics of the calculus.

\subsubsection{Operational semantics} 

Finally, we introduce the computational dynamics. What marks these
algebras as distinct from other more traditionally studied algebraic
structures, e.g. vector spaces or polynomial rings, is the manner in
which dynamics is captured. In traditional structures, dynamics is typically
expressed through morphisms between such structures, as in linear maps
between vector spaces or morphisms between rings. In algebras
associated with the semantics of computation, the dynamics is
expressed as part of the algebraic structure itself, through a
reduction reduction relation typically denoted by $\red$. Below, we
give a recursive presentation of this relation for the calculus used
in the encoding.

$\red \subseteq \pi \times \pi$
$\red : \pi \to \mathcal{P}(\pi)$

\begin{mathpar}
  \inferrule* [lab=Comm] { \textsf{match}( x_{src}, x_{trgt} ) } { x_{trgt}?(y)P \; | \; x_{src}!\langle {Q} \rangle \red P\{\quotep{Q}/y}\} }
  \and \\
  \inferrule* [lab=Par] {{P} \red {P}'} {{{P} | {Q}} \red {{P}' | {Q}}}
  \and
  \inferrule* [lab=Equiv]{{{P} \scong {P}'} \andalso {{P}' \red {Q}'} \andalso {{Q}' \scong {Q}}}{{P} \red {Q}}
\end{mathpar}

\begin{eqnarray*}
  match_{\equiv} (\quotep{P},\quotep{Q}) & := & P \equiv Q \\
  match_{\dagger}(\quotep{P},\quotep{Q}) & := & \forall R. P|Q \red^{*} R => R \red^{*} 0 \\
  match_{K}(\quotep{P},\quotep{Q}) & := & K \mbox{ for some context } K
\end{eqnarray*}

$u?(x)P | u!\langle Q \rangle \red P\{\quotep{Q}/x\}$

%We write $\wred$ for $\red^*$, and $P\red$ if $\exists Q $ such that $ P \red Q$.
We write $P\red$ if $\exists Q $ such that $ P \red Q$ and $P\not\red$, otherwise.

\section{Replication}

As mentioned before, it is known that replication (and hence
recursion) can be implemented in a higher-order process algebra
\cite{SangiorgiWalker}. As our first example of calculation with the
machinery thus far presented we give the construction explicitly in
the {\rhoc}.

\begin{eqnarray}
	D_{x} & := & \prefix{x}{y}{(\binpar{\outputp{x}{y}}{@{y}})} \nonumber\\
	\bangp_{x}{P} & := & \binpar{{x}!\langle{\binpar{D_{x}}{P}}\rangle}{D_{x}} \nonumber
\end{eqnarray}

\begin{eqnarray}
	\bangp_{x}{P} & & \nonumber\\
	=
	& {x}!\langle{(\prefix{x}{y}{(\outputp{x}{y} | @{y})) | P}}\rangle 
	      | \prefix{x}{y}{(\outputp{x}{y} | @{y})} & \nonumber\\
	\red
	& (\outputp{x}{y} | @{y})\substn{\quotep{(\prefix{x}{y}{(@{y} | \outputp{x}{y})) | P}}}{y} & \nonumber\\
	=
	& \outputp{x}{\quotep{(\prefix{x}{y}{(\outputp{x}{y} | @{y})) | P}}}
	  | {(\prefix{x}{y}{(\outputp{x}{y} | @{y})) | P}} & \nonumber\\
	\red
	& \ldots & \nonumber\\
	\red^*
	& P | P | \ldots & \nonumber
\end{eqnarray}

Of course, this encoding, as an implementation, runs away, unfolding
$\bangp{P}$ eagerly. A lazier and more implementable replication
operator, restricted to input-guarded processes, may be obtained as follows.

\begin{eqnarray}
\bangp{\prefix{u}{v}{P}} 
	:= 
	\binpar{\lift{x}{\prefix{u}{v}{(\binpar{D(x)}{P})}}}{D(x)} \nonumber
\end{eqnarray}

\begin{remark}
  Note that the lazier definition still does not deal with summation
  or mixed summation (i.e. sums over input and output). The reader is
  invited to construct definitions of replication that deal with these
  features. 

  Further, the definitions are parameterized in a name, $x$. Can you,
  gentle reader, make a definition that eliminates this parameter and
  guarantees no accidental interaction between the replication
  machinery and the process being replicated -- i.e. no accidental
  sharing of names used by the process to get its work done and the
  name(s) used by the replication to effect copying. This latter
  revision of the definition of replication is crucial to obtaining
  the expected identity $!!P \sim !P$.
\end{remark}

\begin{remark}\label{rem:paradoxical_combinator}
  The reader familiar with the lambda calculus will have noticed the
  similarity between $D$ and the paradoxical combinator.

  [Ed. note: the existence of this seems to suggest we have to be more
  restrictive on the set of processes and names we admit if we are to
  support no-cloning.]
\end{remark}

\subsubsection{Bisimulation}

The computational dynamics gives rise to another kind of equivalence,
the equivalence of computational behavior. As previously mentioned
this is typically captured \emph{via} some form of bisimulation.

% The notion we use in this paper is weak barbed bisimulation
% \cite{milner91polyadicpi}.

The notion we use in this paper is derived from weak barbed
bisimulation \cite{milner91polyadicpi}. 

\begin{definition}
An \emph{observation relation}, $\downarrow_{\mathcal N}$, over a set
of names, $\mathcal N$, is the smallest relation satisfying the rules
below.

\infrule[Out-barb]{y \in {\mathcal N}, \; x \nameeq y}
		  {\outputp{x}{v} \downarrow_{\mathcal N} x}
\infrule[Par-barb]{\mbox{$P\downarrow_{\mathcal N} x$ or $Q\downarrow_{\mathcal N} x$}}
		  {\binpar{P}{Q} \downarrow_{\mathcal N} x}

We write $P \Downarrow_{\mathcal N} x$ if there is $Q$ such that 
$P \wred Q$ and $Q \downarrow_{\mathcal N} x$.
\end{definition}

\begin{definition}
%\label{def.bbisim}
An  ${\mathcal N}$-\emph{barbed bisimulation} over a set of names, ${\mathcal N}$, is a symmetric binary relation 
${\mathcal S}_{\mathcal N}$ between agents such that $P\rel{S}_{\mathcal N}Q$ implies:
\begin{enumerate}
\item If $P \red P'$ then $Q \wred Q'$ and $P'\rel{S}_{\mathcal N} Q'$.
\item If $P\downarrow_{\mathcal N} x$, then $Q\Downarrow_{\mathcal N} x$.
\end{enumerate}
$P$ is ${\mathcal N}$-barbed bisimilar to $Q$, written
$P \wbbisim_{\mathcal N} Q$, if $P \rel{S}_{\mathcal N} Q$ for some ${\mathcal N}$-barbed bisimulation ${\mathcal S}_{\mathcal N}$.
\end{definition}

$\mathcal{R} \subseteq \pi \times \pi$

$P \mathcal{R} Q => \forall P'. P \red P' \Rightarrow \exists Q'. Q \red Q', P' \mathcal{R} Q'$

$P \vdash x \Rightarrow Q \vdash x$

\begin{mathpar}
  \inferrule*[lab=Out-barb]{x \nameeq y}{{y}!\langle{Q}\rangle \vdash x}
  \and
  \inferrule*[lab=Par-barb]{\mbox{$P\vdash x$ or $Q\vdash x$}}{\binpar{P}{Q} \vdash x}
\end{mathpar}

\subsubsection{Contexts}

One of the principle advantages of computational calculi like the
$\pi$-calculus is a well-defined notion of context,
contextual-equivalence and a correlation between
contextual-equivalence and notions of bisimulation. The notion of
context allows the decomposition of a process into (sub-)process and
its syntactic environment, its context. Thus, a context may be
thought of as a process with a ``hole'' (written $\Box$) in it. The
application of a context $M$ to a process $P$, written $M[P]$, is
tantamount to filling the hole in $M$ with $P$. In this paper we do
not need the full weight of this theory, but do make use of the notion
of context in the proof the main theorem. 

\begin{mathpar}
  \inferrule* [lab=summation] {} {{M_{M},M_{N}} \bc \Box \;|\; x.M_{A} \;|\; M_{M}+M_{N}}
  \and
  \inferrule* [lab=agent] {} {{M_{A}} \bc (\vec{x})M_{P} \;| \; \clift{P_0,\ldots,M_{P},\ldots,P_N}}
  \and \\
  \inferrule* [lab=process] {} {{M_{P}} \bc M_{N} \;| \;P|M_{P} }
\end{mathpar} 

\begin{mathpar}
  \inferrule* [lab=sychronization] {} {M_{N} \bc \Box \;|\; x?M_{F} \;|\; x!M_{C}}
  \and
  \inferrule* [lab=abstraction] {} {{M_{F}} \bc (x)M_{P} }
  \and
  \inferrule* [lab=concretion] {} {{M_{C}} \bc \langle M_{P} \rangle }
  \and \\
  \inferrule* [lab=process] {} {{M_{P}} \bc M_{N} \;| \;P|M_{P} }
\end{mathpar}

\begin{definition}[contextual application] Given a context $M$, and
  process $P$, we define the \emph{contextual application}, $M[P] :=
  M\{P/\Box\}$. That is, the contextual application of M to P is the
  substitution of $P$ for $\Box$ in $M$.
\end{definition}

$\meaningof{-} : L \to \mathcal{P}(\pi)$

\begin{mathpar}
  \inferrule* [lab=collection] {} {\meaningof{true} = \pi, \and \meaningof{~E} = \pi \setminus \meaningof{E}, \and \meaningof{E_{1} \& E_{2}} = \meaningof{E_{1}} \cap \meaningof{E_{2}}}
\end{mathpar}

\begin{mathpar}
  \inferrule* [lab=structure] {} {\meaningof{0} = \{ P \in \pi | P \equiv 0 \}, \and \\ \meaningof{E_1 | E_2} = \{ P \in \pi | P \equiv P_{1} | P_{2}, P_{1} \in \meaningof{E_{1}}, P_{2} \in \meaningof{E_2}\} }
\end{mathpar}

\begin{mathpar}
 \inferrule* [lab=behavior] {} {\meaningof{\langle a?b \rangle E} = \{ P \in \pi | P \equiv Q | u?(y)P', \\ \and \\\\ \and \\ \;\;\; u \in \meaningof{a}, \forall z.P'\{z/y\} \in \meaningof{E\{z/b\}}\}, \and \\ \meaningof{a!E} = \{ P \in \pi | P \equiv Q | x!\langle P' \rangle, x \in \meaningof{a} P' \in \meaningof{E}\} }
\end{mathpar}

\begin{mathpar}
 \inferrule* [lab=nominal] {} {\meaningof{\quotep{E}} = \{ \quotep{P} \in \quotep{\pi} | P \in \meaningof{E} \}, \and \meaningof{\quotep{P}} = \{ \quotep{Q} \in \quotep{\pi} | P \equiv Q \} \and \\ \meaningof{@\quotep{E}} = \{ P \in \pi | P \equiv @x, x \in \meaningof{E} \}}
\end{mathpar}

\begin{eqnarray*}
  \\
  \meaningof{-} : TS \to ST
\end{eqnarray*}

\begin{eqnarray*}
  \\
  L : TS \to ST
\end{eqnarray*}

\begin{eqnarray*}
  \\
  P \models E \iff P \in \meaningof{E}
\end{eqnarray*}

\begin{eqnarray*}
  P \approx_{L} Q \iff \forall E \in L. P \models E \iff Q \models E
\end{eqnarray*}

\begin{eqnarray*}
  P \approx_{K} Q
\end{eqnarray*}

\begin{eqnarray*}
  P \approx Q
\end{eqnarray*}

$\approx_{K} = \approx = \approx_{L}$

\subsubsection{Contextual duality}

Note that contexts extend the quotation operation to a family of
operations from processes to names. Given a context, $M$, we can
define a \emph{nominal context}, $\quotep{M}$ by $\quotep{M}[P] :=
\quotep{M[P]}$. To foreshadow what is to come we observe that these
operations enjoy a duality with processes very much like the duality
between vectors and maps from vectors to scalars.

Further, because the calculus is essentially higher-order, we have a
correspondence between contexts and processes. More specifically,
given a name $x$ and a context $M$ we can construct $M^{*}_{x}$ such
that 

\begin{mathpar}
  M^{*}_{x} | \lift{x}{P} \red M[P]
\end{mathpar}

namely,

\begin{mathpar}
  M^{*}_{x} := x?(u).M[\dropn{u}]
\end{mathpar}

The dependence of $M^{*}_{x}$ on a name makes it an abstraction, 

\begin{mathpar}
  M^{*} := (x)x?(u).M[\dropn{u}]
\end{mathpar}

\subsection{Additional notation}

It will sometimes be convenient to denote the process a name
quotes. We already have the notation $x = \quotep{P}$, but it will be
convenient to introduce an alternate notation, $\procn{x}$, when we
want to emphasize the connection to the use of the name. Note that, by
virtue of name equivalence, $\quotep{\procn{x}} \nameeq x$; so, the
notation is consistent with previous definitions.

Further, because names have structure it is possible to effect
substitutions on the basis of that structure. This means we need to
upgrade our notation for substitutions, which we accomplish by
adapting comprehension notation. Thus,

\begin{mathpar}
  P\{ y / x : x \in S \}
\end{mathpar}

is interpreted to mean the process derived from P by replacing (in a
capture-avoiding manner) each occurrence of $x$ in $S$ by $y$. For example,

\begin{mathpar}
  P\{ \quotep{\procn{x}|\procn{x}} / x : x \in \freenames{P} \}
\end{mathpar}

will replace each (occurrence) of a free name $x$ in $P$ by
$\quotep{\procn{x}|\procn{x}}$.

Also, we will avail ourselves of the notation $x^{L}$ and $x^{R}$ to
denote injections of a name into disjoint copies of the name
space. There are numerous ways to accomplish this. One example can be
found in \cite{MeredithR05}. This notation overloads to vectors of
names: $\vec{x}^{\pi} := (x_{i}^{\pi} \; : \; 0 \leq i < |\vec{x}| )$ where $\pi \in \{L,R\}$.

We also use $P^{\Box} := P|\Box$.

In \cite{MeredithR05} an interpretation of the new operator is
given. It turns out that there are several possible interpretations
all enjoying the requisite algebraic properties of the operator (see
\cite{milner91polyadicpi}). We will therefore make liberal use of
$(\nu\; \vec{x})P$.

% subsection the_syntax_and_semantics_of_the_notation_system (end)   

\input{qm2pi.qmops} 

\input{qm2pi.sterngerlach} 

\input{qm2pi.metric} 

% section concurrent_process_calculi (end)

%\input{qm2pi.proofsketch}

% section proof sketch (end)

%\input{qm2pi.slviaknots} 

% section spatial logic via knots (end)

\input{qm2pi.conclusion}

% section conclusion (end)

%\input{qm2pi.dtcodes} 

% section wiring algorithm (end)

\input{qm2pi.ack} 

% section acknowledgments (end)

\newpage


\bibliographystyle{plain}   
\bibliography{../../biblios/main.bib}

\input{qm2pi.rhodetails}

\end{document}

 

% section concurrent_process_calculi (end)

%\documentclass[12pt]{llncs}
%\documentclass{jktr}

\usepackage[pdftex]{hyperref}                   
\usepackage {listings}
\usepackage {mathpartir}
\usepackage{bcprules}
%\usepackage{listings}
                       
\usepackage{graphicx} 
%\usepackage[margins=2.5cm,nohead,nofoot]{geometry}
%\usepackage{geometry}
\usepackage{amsfonts}
\usepackage{amstext}
\usepackage{latexsym}
\usepackage{amssymb}
\usepackage{color}


%\include{myPreamble}
\include{qm2pi.local} 

%\ifpdf
%\usepackage[pdftex]{graphicx}
%\else
%\usepackage{graphicx}
%\fi

 % \ifpdf
%  \usepackage{pdfsync}
%  \if


%\title{Brief Article}
%\author{David F. Snyder}
%\author{L.G. Meredith}

%\address{Dept. of Math., Texas State University--San Marcos, San Marcos, TX 78666}
       
\pagestyle{empty}


\begin{document}

\lstset{language=[Objective]Caml,frame=shadowbox}

\input{qm2pi.front}

% section front matter (end)

\input{qm2pi.intro} 
 
% section introduction (end)

% \input{qm2pi.knotations} 

% section notation (end)

\input{qm2pi.process.calculi} 

% section concurrent_process_calculi_and_spatial_logics_ (end)
    
%\input{qm2pi.knots2pi} 

%\input{qm2pi.trefoil} 

%\input{qm2pi.mainthm} 

% subsection basic_interpretation (end)

%\input{qm2pi.rho.presentation} 
\subsection{The syntax and semantics of the notation system}\label{sub:the_syntax_and_semantics_of_the_notation_system} % (fold)

We now summarize a technical presentation of the calculus that
embodies our theory of dynamics. The typical presentation of such a
calculus follows the style of giving generators and relations on
them. The grammar, below, describing term constructors, freely
generates the set of processes, $\Proc$. This set is then quotiented
by a relation known as structural congruence and it is over this set
that the notion of dynamics is expressed. This presentation is
essentially that of \cite{MeredithR05} with the addition of
polyadicity and summation. For readability we have relegated some of
the technical subtleties to an appendix.

\subsubsection{Process grammar}\label{subsub:process_grammar}

\begin{mathpar}
  \inferrule* [lab=synchronization] {} {{M} \bc \pzero \;|\; x?F \;|\; x!C }
  \and
  \inferrule* [lab=abstraction] {} {{F} \bc (x)P}
  \and
  \inferrule* [lab=concretion] {} {{C} \bc \langle Q \rangle}
  \and
  \inferrule* [lab=process] {} {{P,Q} \bc M \;| \;P|Q \;|\; @{x}}
  \and
  \inferrule* [lab=name] {} {{x} \bc \quotep{P}}
\end{mathpar} 

Note that $\vec{x}$ (resp. $\vec{P}$) denotes a vector of names
(resp. processes) of length $|\vec{x}|$ (resp. $|\vec{P}|$). We adopt
the following useful abbreviations.

\begin{mathpar}
   x?(\vec{y}).P := x.(\vec{y})P \and  x\clift{\vec{P}} := x.\clift{\vec{P}}
   \and x!(y) := \lift{x}{\dropn{y}}
   \and \Pi_{i=0}^{n-1}P_i := P_0 | \ldots | P_{n-1}
\end{mathpar}

\subsubsection{Structural congruence}

\paragraph{Free and bound names and alpha-equivalence.} At the
core of structural equivalence is alpha-equivalence which identifies
process that are the same up to a change of variable. Formally, we
recognize the distinction between free and bound names. The free names
of a process, $\freenames{P}$, may be calculated recursively as
follows:

\begin{mathpar}
\freenames{\pzero} := \emptyset
  \and \\
  \freenames{x?(y).P} := \{ x \} \cup (\freenames{P} \setminus \{ y \})
  \and 
  \freenames{x!\langle P \rangle} := \{ x \} \cup \{ P \} 
  \and \\
  \freenames{P|Q} := \freenames{P} \cup \freenames{Q}
  \and \\
  \freenames{@{x}} := \{ x \}
\end{mathpar}

$\pi$
$\quotep{\pi}$

$\freenames{-} : \pi \to \mathcal{P}(\quotep{\pi})$

\begin{eqnarray*}
  \freenames{\pzero} & := & \emptyset \\
  \freenames{x?(y).P} & := & \{ x \} \cup (\freenames{P} \setminus \{ y \}) \\
  \freenames{x!\langle P \rangle} & := & \{ x \} \cup \{ P \} \\
  \freenames{P|Q} & := & \freenames{P} \cup \freenames{Q} \\
  \freenames{\dropn{x}} & := & \{ x \}
\end{eqnarray*}

The bound names of a process, $\boundnames{P}$, are those names occurring in $P$
that are not free. For example, in $x?(y).0$, the name $x$ is free, while $y$ is bound.

\begin{mathpar}
  \inferrule* [lab=monoidal-laws] {} { P|Q \equiv Q|P \and P|0 \equiv P \and P|(Q|R) \equiv (P|Q)|R }
\end{mathpar}

\begin{mathpar}
  \inferrule* [lab=alpha-equivalence] {} { (x)P \equiv (y)P\{y/x\} \and y \not\in \freenames{P} }
\end{mathpar}

\begin{definition}
Then two processes, $P,Q$, are alpha-equivalent if $P = Q\{\vec{y}/\vec{x}\}$ for
some $\vec{x} \in \boundnames{Q},\vec{y} \in \boundnames{P}$, where $Q\{\vec{y}/\vec{x}\}$
denotes the capture-avoiding substitution of $\vec{y}$ for $\vec{x}$ in $Q$.
\end{definition}

\begin{definition}
  The {\em structural congruence} \cite{SangiorgiWalker} , $\equiv$,
  between processes is the least congruence containing
  alpha-equivalence, satisfying the abelian monoid laws
  (associativity, commutativity and $\pzero$ as identity) for parallel
  composition $|$ and for summation $+$.
\end{definition}

\subsection{Name equivalence}

We take name equivalence, written $\nameeq$, to be the smallest
equivalence relation generated by the following rules.

\begin{mathpar}
\inferrule*[lab=Quote-drop]
{ }
{ \quotep{@{x}} \nameeq x }

\inferrule*[lab=Struct-equiv]
{ P \scong Q }
{ \quotep{P} \nameeq \quotep{Q} }
\end{mathpar}

The astute reader will have noticed that the mutual recursion of names
and processes imposes a mutual recursion on alpha-equivalence and
structural equivalence via name-equivalence. Fortunately, all of this
works out pleasantly and we may calculate in the natural way, free of
concern. The reader interested in the details is referred to the
appendix \ref{appendix:rho_details}.

\subsection{Substitution}

We use $\Proc$ for the set of processes, $\QProc$ for the set of
names, and $\id{\{}\vec{y} / \vec{x} \id{\}}$ to denote partial maps,
$s : \QProc \rightarrow \QProc$. A map, $s$ lifts, uniquely, to a map
on process terms, $\widehat{s} : \Proc \rightarrow \Proc$ by the
following equations.

\begin{mathpar}
  (0) \psubstp{Q}{P} := 0 \\
  (R \juxtap S) \psubstp{Q}{P}
  :=    
  (R)\psubstp{Q}{P} \juxtap (S) \psubstp{Q}{P} \\
  (x?(y).R) \psubstp{Q}{P}    
  :=    
  (x)\substp{Q}{P} (z)\concat( (R \psubstn{z}{y}) \psubstp{Q}{P} ) \\
  (\lift{x}{R}) \psubstp{Q}{P}  
  :=
  \lift{(x)\substp{Q}{P}}{ R \psubstp{Q}{P} } \\
%   (\dropn{x})  \psubstp{Q}{P}       
%   := 
%   \left\{ 
%     \begin{array}{ccc} 
%       \dropn{\quotep{Q}} & & x \nameeq \quotep{P} \\
%       \dropn{x} & & otherwise \\
%     \end{array}
%   \right. 
  (\dropn{x})  \psubstp{Q}{P}       
  := 
  \left\{ 
    \begin{array}{ccc} 
      Q & & x \nameeq \quotep{P} \\
      \dropn{x} & & otherwise \\
    \end{array}
  \right.
\end{mathpar}
 

where

\begin{eqnarray}
  (x)\id{\{} \lpquote Q \rpquote / \lpquote P \rpquote \id{\}}            = 
  \left\{ 
    \begin{array}{ccc}
      \lpquote Q \rpquote & & x \nameeq \lpquote P \rpquote \\
      x & & otherwise \\
    \end{array}
  \right. \nonumber
\end{eqnarray}

and $z$ is chosen distinct from $\quotep{P}$, $\quotep{Q}$, the free
names in $Q$, and all the names in $R$. Our $\alpha$-equivalence will
be built in the standard way from this substitution.

\begin{remark}\label{rem:no_self_referential_names}
  One consequence of these definitions is that $\forall P. \quotep{P}
  \not\in \freenames{P}$.
\end{remark}

\subsection{ Dynamic quote: an example }

Anticipating something of what's to come, consider applying the
substitution, $\widehat{\id{\{}u / z \id{\}}}$, to the following pair
of processes, $\lift{w}{y!(z)}$ and $w[ \lpquote y!(z) \rpquote ]$.

\begin{eqnarray}
	\lift{w}{y!(z)}\widehat{\id{\{}u / z \id{\}}}
		& = &
		\lift{w}{y!(u)} \nonumber\\
	w[ \lpquote y!(z) \rpquote ] \widehat{ \id{\{}u / z \id{\}} }
		& = &
		w[ \lpquote y!(z) \rpquote ] \nonumber
\end{eqnarray}

Because the body of the process between quotes is impervious to
substitution, we get radically different answers. In fact, by
examining the first process in an input context,
e.g. $x?(z).\lift{w}{y!(z)}$, we see that the process under the lift
operator may be shaped by prefixed inputs binding a name inside it. In
this sense, the lift operator will be seen as a way to dynamically
construct processes before reifying them as names.

Finally equipped with these standard features we can present the
dynamics of the calculus.

\subsubsection{Operational semantics} 

Finally, we introduce the computational dynamics. What marks these
algebras as distinct from other more traditionally studied algebraic
structures, e.g. vector spaces or polynomial rings, is the manner in
which dynamics is captured. In traditional structures, dynamics is typically
expressed through morphisms between such structures, as in linear maps
between vector spaces or morphisms between rings. In algebras
associated with the semantics of computation, the dynamics is
expressed as part of the algebraic structure itself, through a
reduction reduction relation typically denoted by $\red$. Below, we
give a recursive presentation of this relation for the calculus used
in the encoding.

$\red \subseteq \pi \times \pi$
$\red : \pi \to \mathcal{P}(\pi)$

\begin{mathpar}
  \inferrule* [lab=Comm] { \textsf{match}( x_{src}, x_{trgt} ) } { x_{trgt}?(y)P \; | \; x_{src}!\langle {Q} \rangle \red P\{\quotep{Q}/y}\} }
  \and \\
  \inferrule* [lab=Par] {{P} \red {P}'} {{{P} | {Q}} \red {{P}' | {Q}}}
  \and
  \inferrule* [lab=Equiv]{{{P} \scong {P}'} \andalso {{P}' \red {Q}'} \andalso {{Q}' \scong {Q}}}{{P} \red {Q}}
\end{mathpar}

\begin{eqnarray*}
  match_{\equiv} (\quotep{P},\quotep{Q}) & := & P \equiv Q \\
  match_{\dagger}(\quotep{P},\quotep{Q}) & := & \forall R. P|Q \red^{*} R => R \red^{*} 0 \\
  match_{K}(\quotep{P},\quotep{Q}) & := & K \mbox{ for some context } K
\end{eqnarray*}

$u?(x)P | u!\langle Q \rangle \red P\{\quotep{Q}/x\}$

%We write $\wred$ for $\red^*$, and $P\red$ if $\exists Q $ such that $ P \red Q$.
We write $P\red$ if $\exists Q $ such that $ P \red Q$ and $P\not\red$, otherwise.

\section{Replication}

As mentioned before, it is known that replication (and hence
recursion) can be implemented in a higher-order process algebra
\cite{SangiorgiWalker}. As our first example of calculation with the
machinery thus far presented we give the construction explicitly in
the {\rhoc}.

\begin{eqnarray}
	D_{x} & := & \prefix{x}{y}{(\binpar{\outputp{x}{y}}{@{y}})} \nonumber\\
	\bangp_{x}{P} & := & \binpar{{x}!\langle{\binpar{D_{x}}{P}}\rangle}{D_{x}} \nonumber
\end{eqnarray}

\begin{eqnarray}
	\bangp_{x}{P} & & \nonumber\\
	=
	& {x}!\langle{(\prefix{x}{y}{(\outputp{x}{y} | @{y})) | P}}\rangle 
	      | \prefix{x}{y}{(\outputp{x}{y} | @{y})} & \nonumber\\
	\red
	& (\outputp{x}{y} | @{y})\substn{\quotep{(\prefix{x}{y}{(@{y} | \outputp{x}{y})) | P}}}{y} & \nonumber\\
	=
	& \outputp{x}{\quotep{(\prefix{x}{y}{(\outputp{x}{y} | @{y})) | P}}}
	  | {(\prefix{x}{y}{(\outputp{x}{y} | @{y})) | P}} & \nonumber\\
	\red
	& \ldots & \nonumber\\
	\red^*
	& P | P | \ldots & \nonumber
\end{eqnarray}

Of course, this encoding, as an implementation, runs away, unfolding
$\bangp{P}$ eagerly. A lazier and more implementable replication
operator, restricted to input-guarded processes, may be obtained as follows.

\begin{eqnarray}
\bangp{\prefix{u}{v}{P}} 
	:= 
	\binpar{\lift{x}{\prefix{u}{v}{(\binpar{D(x)}{P})}}}{D(x)} \nonumber
\end{eqnarray}

\begin{remark}
  Note that the lazier definition still does not deal with summation
  or mixed summation (i.e. sums over input and output). The reader is
  invited to construct definitions of replication that deal with these
  features. 

  Further, the definitions are parameterized in a name, $x$. Can you,
  gentle reader, make a definition that eliminates this parameter and
  guarantees no accidental interaction between the replication
  machinery and the process being replicated -- i.e. no accidental
  sharing of names used by the process to get its work done and the
  name(s) used by the replication to effect copying. This latter
  revision of the definition of replication is crucial to obtaining
  the expected identity $!!P \sim !P$.
\end{remark}

\begin{remark}\label{rem:paradoxical_combinator}
  The reader familiar with the lambda calculus will have noticed the
  similarity between $D$ and the paradoxical combinator.

  [Ed. note: the existence of this seems to suggest we have to be more
  restrictive on the set of processes and names we admit if we are to
  support no-cloning.]
\end{remark}

\subsubsection{Bisimulation}

The computational dynamics gives rise to another kind of equivalence,
the equivalence of computational behavior. As previously mentioned
this is typically captured \emph{via} some form of bisimulation.

% The notion we use in this paper is weak barbed bisimulation
% \cite{milner91polyadicpi}.

The notion we use in this paper is derived from weak barbed
bisimulation \cite{milner91polyadicpi}. 

\begin{definition}
An \emph{observation relation}, $\downarrow_{\mathcal N}$, over a set
of names, $\mathcal N$, is the smallest relation satisfying the rules
below.

\infrule[Out-barb]{y \in {\mathcal N}, \; x \nameeq y}
		  {\outputp{x}{v} \downarrow_{\mathcal N} x}
\infrule[Par-barb]{\mbox{$P\downarrow_{\mathcal N} x$ or $Q\downarrow_{\mathcal N} x$}}
		  {\binpar{P}{Q} \downarrow_{\mathcal N} x}

We write $P \Downarrow_{\mathcal N} x$ if there is $Q$ such that 
$P \wred Q$ and $Q \downarrow_{\mathcal N} x$.
\end{definition}

\begin{definition}
%\label{def.bbisim}
An  ${\mathcal N}$-\emph{barbed bisimulation} over a set of names, ${\mathcal N}$, is a symmetric binary relation 
${\mathcal S}_{\mathcal N}$ between agents such that $P\rel{S}_{\mathcal N}Q$ implies:
\begin{enumerate}
\item If $P \red P'$ then $Q \wred Q'$ and $P'\rel{S}_{\mathcal N} Q'$.
\item If $P\downarrow_{\mathcal N} x$, then $Q\Downarrow_{\mathcal N} x$.
\end{enumerate}
$P$ is ${\mathcal N}$-barbed bisimilar to $Q$, written
$P \wbbisim_{\mathcal N} Q$, if $P \rel{S}_{\mathcal N} Q$ for some ${\mathcal N}$-barbed bisimulation ${\mathcal S}_{\mathcal N}$.
\end{definition}

$\mathcal{R} \subseteq \pi \times \pi$

$P \mathcal{R} Q => \forall P'. P \red P' \Rightarrow \exists Q'. Q \red Q', P' \mathcal{R} Q'$

$P \vdash x \Rightarrow Q \vdash x$

\begin{mathpar}
  \inferrule*[lab=Out-barb]{x \nameeq y}{{y}!\langle{Q}\rangle \vdash x}
  \and
  \inferrule*[lab=Par-barb]{\mbox{$P\vdash x$ or $Q\vdash x$}}{\binpar{P}{Q} \vdash x}
\end{mathpar}

\subsubsection{Contexts}

One of the principle advantages of computational calculi like the
$\pi$-calculus is a well-defined notion of context,
contextual-equivalence and a correlation between
contextual-equivalence and notions of bisimulation. The notion of
context allows the decomposition of a process into (sub-)process and
its syntactic environment, its context. Thus, a context may be
thought of as a process with a ``hole'' (written $\Box$) in it. The
application of a context $M$ to a process $P$, written $M[P]$, is
tantamount to filling the hole in $M$ with $P$. In this paper we do
not need the full weight of this theory, but do make use of the notion
of context in the proof the main theorem. 

\begin{mathpar}
  \inferrule* [lab=summation] {} {{M_{M},M_{N}} \bc \Box \;|\; x.M_{A} \;|\; M_{M}+M_{N}}
  \and
  \inferrule* [lab=agent] {} {{M_{A}} \bc (\vec{x})M_{P} \;| \; \clift{P_0,\ldots,M_{P},\ldots,P_N}}
  \and \\
  \inferrule* [lab=process] {} {{M_{P}} \bc M_{N} \;| \;P|M_{P} }
\end{mathpar} 

\begin{mathpar}
  \inferrule* [lab=sychronization] {} {M_{N} \bc \Box \;|\; x?M_{F} \;|\; x!M_{C}}
  \and
  \inferrule* [lab=abstraction] {} {{M_{F}} \bc (x)M_{P} }
  \and
  \inferrule* [lab=concretion] {} {{M_{C}} \bc \langle M_{P} \rangle }
  \and \\
  \inferrule* [lab=process] {} {{M_{P}} \bc M_{N} \;| \;P|M_{P} }
\end{mathpar}

\begin{definition}[contextual application] Given a context $M$, and
  process $P$, we define the \emph{contextual application}, $M[P] :=
  M\{P/\Box\}$. That is, the contextual application of M to P is the
  substitution of $P$ for $\Box$ in $M$.
\end{definition}

$\meaningof{-} : L \to \mathcal{P}(\pi)$

\begin{mathpar}
  \inferrule* [lab=collection] {} {\meaningof{true} = \pi, \and \meaningof{~E} = \pi \setminus \meaningof{E}, \and \meaningof{E_{1} \& E_{2}} = \meaningof{E_{1}} \cap \meaningof{E_{2}}}
\end{mathpar}

\begin{mathpar}
  \inferrule* [lab=structure] {} {\meaningof{0} = \{ P \in \pi | P \equiv 0 \}, \and \\ \meaningof{E_1 | E_2} = \{ P \in \pi | P \equiv P_{1} | P_{2}, P_{1} \in \meaningof{E_{1}}, P_{2} \in \meaningof{E_2}\} }
\end{mathpar}

\begin{mathpar}
 \inferrule* [lab=behavior] {} {\meaningof{\langle a?b \rangle E} = \{ P \in \pi | P \equiv Q | u?(y)P', \\ \and \\\\ \and \\ \;\;\; u \in \meaningof{a}, \forall z.P'\{z/y\} \in \meaningof{E\{z/b\}}\}, \and \\ \meaningof{a!E} = \{ P \in \pi | P \equiv Q | x!\langle P' \rangle, x \in \meaningof{a} P' \in \meaningof{E}\} }
\end{mathpar}

\begin{mathpar}
 \inferrule* [lab=nominal] {} {\meaningof{\quotep{E}} = \{ \quotep{P} \in \quotep{\pi} | P \in \meaningof{E} \}, \and \meaningof{\quotep{P}} = \{ \quotep{Q} \in \quotep{\pi} | P \equiv Q \} \and \\ \meaningof{@\quotep{E}} = \{ P \in \pi | P \equiv @x, x \in \meaningof{E} \}}
\end{mathpar}

\begin{eqnarray*}
  \\
  \meaningof{-} : TS \to ST
\end{eqnarray*}

\begin{eqnarray*}
  \\
  L : TS \to ST
\end{eqnarray*}

\begin{eqnarray*}
  \\
  P \models E \iff P \in \meaningof{E}
\end{eqnarray*}

\begin{eqnarray*}
  P \approx_{L} Q \iff \forall E \in L. P \models E \iff Q \models E
\end{eqnarray*}

\begin{eqnarray*}
  P \approx_{K} Q
\end{eqnarray*}

\begin{eqnarray*}
  P \approx Q
\end{eqnarray*}

$\approx_{K} = \approx = \approx_{L}$

\subsubsection{Contextual duality}

Note that contexts extend the quotation operation to a family of
operations from processes to names. Given a context, $M$, we can
define a \emph{nominal context}, $\quotep{M}$ by $\quotep{M}[P] :=
\quotep{M[P]}$. To foreshadow what is to come we observe that these
operations enjoy a duality with processes very much like the duality
between vectors and maps from vectors to scalars.

Further, because the calculus is essentially higher-order, we have a
correspondence between contexts and processes. More specifically,
given a name $x$ and a context $M$ we can construct $M^{*}_{x}$ such
that 

\begin{mathpar}
  M^{*}_{x} | \lift{x}{P} \red M[P]
\end{mathpar}

namely,

\begin{mathpar}
  M^{*}_{x} := x?(u).M[\dropn{u}]
\end{mathpar}

The dependence of $M^{*}_{x}$ on a name makes it an abstraction, 

\begin{mathpar}
  M^{*} := (x)x?(u).M[\dropn{u}]
\end{mathpar}

\subsection{Additional notation}

It will sometimes be convenient to denote the process a name
quotes. We already have the notation $x = \quotep{P}$, but it will be
convenient to introduce an alternate notation, $\procn{x}$, when we
want to emphasize the connection to the use of the name. Note that, by
virtue of name equivalence, $\quotep{\procn{x}} \nameeq x$; so, the
notation is consistent with previous definitions.

Further, because names have structure it is possible to effect
substitutions on the basis of that structure. This means we need to
upgrade our notation for substitutions, which we accomplish by
adapting comprehension notation. Thus,

\begin{mathpar}
  P\{ y / x : x \in S \}
\end{mathpar}

is interpreted to mean the process derived from P by replacing (in a
capture-avoiding manner) each occurrence of $x$ in $S$ by $y$. For example,

\begin{mathpar}
  P\{ \quotep{\procn{x}|\procn{x}} / x : x \in \freenames{P} \}
\end{mathpar}

will replace each (occurrence) of a free name $x$ in $P$ by
$\quotep{\procn{x}|\procn{x}}$.

Also, we will avail ourselves of the notation $x^{L}$ and $x^{R}$ to
denote injections of a name into disjoint copies of the name
space. There are numerous ways to accomplish this. One example can be
found in \cite{MeredithR05}. This notation overloads to vectors of
names: $\vec{x}^{\pi} := (x_{i}^{\pi} \; : \; 0 \leq i < |\vec{x}| )$ where $\pi \in \{L,R\}$.

We also use $P^{\Box} := P|\Box$.

In \cite{MeredithR05} an interpretation of the new operator is
given. It turns out that there are several possible interpretations
all enjoying the requisite algebraic properties of the operator (see
\cite{milner91polyadicpi}). We will therefore make liberal use of
$(\nu\; \vec{x})P$.

% subsection the_syntax_and_semantics_of_the_notation_system (end)   

\input{qm2pi.qmops} 

\input{qm2pi.sterngerlach} 

\input{qm2pi.metric} 

% section concurrent_process_calculi (end)

%\input{qm2pi.proofsketch}

% section proof sketch (end)

%\input{qm2pi.slviaknots} 

% section spatial logic via knots (end)

\input{qm2pi.conclusion}

% section conclusion (end)

%\input{qm2pi.dtcodes} 

% section wiring algorithm (end)

\input{qm2pi.ack} 

% section acknowledgments (end)

\newpage


\bibliographystyle{plain}   
\bibliography{../../biblios/main.bib}

\input{qm2pi.rhodetails}

\end{document}



% section proof sketch (end)

%\section{Unlikely characters: spatial logic for
  knots}\label{sub:characteristic_formulae} % (fold)

Associated to the mobile process calculi are a family of logics known
as the Hennessy-Milner logics. These logics typically enjoy a
semantics interpreting formulae as sets of processes that when
factored through the encoding outlined above allows an identification
of classes of knots with logical formulae. In the context of this
encoding the sub-family known as the spatial logics \cite{CairesC03}
\cite{CairesC04} \cite{Caires04} are of particular interest providing
several important features for expressing and reasoning about
properties (i.e. classes) of knots. We hint here at how this may be done.

%\begin{description}
%\item [structural connectives] 
\subsubsection{Structural connectives} The spatial logics enjoy
structural connectives corresponding, at the logical level, to the
parallel composition ($P | Q$) and new name ($(\nu \; x)P$)
connectives for processes. As illustrated in the examples below, these
connectives are extremely expressive given the shape of our encoding.
%\item [decideable satisfaction]

\subsubsection{Decideable satisfaction}
In \cite{Caires04} the satisfaction relation is shown to be decideable
for a rich class of processes. It further turns out that the image of
the our encoding is a proper subset of that class. This result
provides the basis for an algorithm by which to search for knots
enjoying a given property.
%\item [characteristic formulae]

\subsubsection{Characteristic formulae}
In the same paper \cite{Caires04} , Caires presents a means of calculating
characteristic formulae, selecting equivalence classes of processes
up to a pre--specified depth limit on the support set of names. Composed with our
encoding, this characteristic formula can be used to select
characteristic formulae for knots.
%\end{description}

\subsubsection{Spatial logic formulae}

The grammar below (segmented for comprehension) summarizes the syntax
of spatial logic formulae. We employ illustrative examples in the
sequel to provide an intuitive understanding of their meaning
referring the reader to \cite{Caires04} for a more detailed explication
of the semantics.

\begin{mathpar}
  \inferrule* [lab=boolean] {} {{A,B} \bc T \;|\; \neg A \;|\; A \wedge B \;|\; \eta = \eta'}
  \and
  \inferrule* [lab=spatial] {} {|\; \pzero \;|\; A | B \;|\; x \text{\textregistered} A \;|\; \forall x . A \;|\;  H x . A}
  \and
  \inferrule* [lab=behavioral] {} {|\; \alpha . A}
  \and 
  \inferrule* [lab=recursion] {} {|\; X(\vec{u}) \;|\; \mu X(\vec{u}) . A}
  \and
  \inferrule* [lab=action] {} {\alpha \bc \langle x?(\vec{y}) \rangle \;|\; \langle x!(\vec{y}) \rangle \;|\; \langle \tau \rangle}
  \and 
  \inferrule* [lab=name] {} {\eta \bc x \;|\; \tau}
\end{mathpar} 

% subsection characteristic_formulae (end)   	 

\subsection{Example formulae}\label{sub:example_formulae_} % (fold)

\subsubsection{Crossing as formula.}
% 
% \begin{align*}
%   \frac{d}{dx} \sin x &= \cos x 
%   & \frac{d}{dx} e^x &= e^x \\
%   \frac{d}{dx} \cos x &= - \sin x 
%   & \frac{d}{dx} \log x &= \frac{1}{x} \\
% \end{align*} 

\begin{align*}
 \mu C(x_{0},x_{1},y_{0},y_{1},u).&(\langle x_{0}?(z) \rangle(\langle u! \rangle\langle y_{1}!z \rangle C(x_{0},x_{1},y_{0},y_{1},u)) & \\
  & \wedge \langle y_{1}?(z) \rangle (\langle u! \rangle \langle x_{0}!z \rangle C(x_{0},x_{1},y_{0},y_{1},u)) & \\
  & \wedge \langle x_{1}?(z) \rangle (\langle u? \rangle \langle y_{0}!z \rangle C(x_{0},x_{1},y_{0},y_{1},u)) & \\
  & \wedge \langle y_{0}?(z) \rangle (\langle u? \rangle \langle x_{1}!z \rangle C(x_{0},x_{1},y_{0},y_{1},u))) &
\end{align*}

The lexicographical similarity between the shape of this formulae and
the shape of definition of the process representing a crossing reveals
the intuitive meaning of this formulae. It describes the capabilities
of a process that has the right to represent a crossing. For example
it picks out processes that may perform an input on the port $x_0$ in
its initial menu of capabilities. What differentiates the formula
from the process, however, is that the crossing process is the
smallest candidate to satisfy the formula. Infinitely many other
processes -- with internal behavior hidden behind this interface, so
to speak -- also satisfy this formula. Even this simple formula,
then, can be seen to open a new view onto knots, providing a
computational interpretation of \emph{virtual} knots.

Note that this formula is derived by hand. A similar formula can be
derived by employing Caires' calculation of characteristic formula
\cite{Caires04} to the process representing a crossing. In light of
this discussion, we let
$\meaningof{C}_{\phi}(x0,x1,y0,y1,u)$ denote a formula specifying the
dynamics we wish to capture of a crossing. To guarantee we preserve
the shape of the interface and minimal semantics we demand that
$\meaningof{C}_{\phi}(x0,x1,y0,y1,u) \Rightarrow
\textbf{C}(x0,x1,y0,y1,u)$ where $\textbf{C}(x0,x1,y0,y1,u)$ denotes
the formula above.
                            
\subsubsection{Crossing number constraints.}
The moral content of the context lemma (Lemma \ref{context}) is that the notion of
``locality'' in the Reidemeister moves is effectively captured by the
parallel composition operator of the process calculus. This intuition
extends through the logic. Given a formula,
$\meaningof{C}_{\phi}(x0,x1,y0,y1,u)$, we can use the structural
connectives to specify constraints on crossing numbers, such as at
least $n$ crossings, or exactly $n$ crossings.
\begin{mathpar}
  \inferrule* [lab=at-least-n] {} { K^{\geq n}_{\phi}(\vec{xs},\vec{ys}) := \Pi_{i=0}^{n-1} Hu . \meaningof{C}_{\phi}(xs_i,ys_i,u) | T }
  \and 
  \inferrule* [lab=exactly-n] {} { K^{= n}_{\phi}(\vec{xs},\vec{ys}) := \Pi_{i=0}^{n-1} Hu . \meaningof{C}_{\phi}(xs_i,ys_i,u) | \neg (\forall x_0,y_0,x_1,y_1,u . \meaningof{C}_{\phi}(x_0,y_0,x_1,y_1,u) | T) }
\end{mathpar}

To round out this section, recall that the encoding of an $n$-crossing
knot decomposes into a parallel composition of $n$ \emph{copies} of a
crossing process together with a wiring harness. To specify different
knot classes with the same crossing number amounts to specifying
logical constraints on the wiring harness. In the interest of space,
we defer examples to a forthcoming paper. Suffice it to say that both
the conditions ``alternating knot'' and ``contains the tangle
corresponding to 5/3'' are expressible. For example, it is possible to
calculate the characteristic formula of a process corresponding to the
tangle 5/3 and conjoin it into the classifying formula via the
composition connective of the logic.

Finally, we wish to observe that it is entirely within reason to
contemplate a more domain-specific version of spatial logic tailored
to the shape of processes in the image of the encoding. Such a
domain-specific logic would have a better claim to the title formal
language of knot properties.

% subsection example_formulae_ (end)

% section knots_as_processes (end) 

% section spatial logic via knots (end)

\section{Conclusions and future work}

\paragraph{Testing physical space}
You, gentle reader, may wonder why of all the theorems to be proved
given this set up we pick the one above. In some sense it's hardly
central to quantum mechanics. We see it as central in the sense that
it firmly establishes a notion of physical space arising from a notion
of the equivalence of behavior. Relating bisimulation to a metric is a
big step forward, but one is faced with interpreting the relationship
of that metric space to something more physical. Quantum mechanical
notions of ``physical'' space are still far from intuitive, but by
relating this idea of distance as testing to calculations that predict
physical circumstances we are making a not insignificant step forward
toward an understanding of the physical space we inhabit as
essentially dynamic.

\paragraph{Effectivity and simulation}
One of the observations we have yet to make is that the entire program
spelled out here is effective. We have built various interpreters for
the reflective calculus at work in this interpretation. In principle,
then, we can simulate quantum mechanics on a computer. The place where
the simulation may lose fidelity is the infinitely branching summation
for the annihilator.

In this connection i also want to point out that the evaluation style
calculation of the inner product puts the non-determinism of the
summation right at the heart of measurement. This suggests that
Milner's original reduction-based formulation of the dynamics of his
calculi in terms of sums was not just notationally suggestive of a
notion of measure-and-continue but captured some significant part of
the physics.

\paragraph{Quantum continuations}
In light of this last observation i want to point out that the
predominant account of quantum mechanics is missing a key aspect of a
truly compositional story of the physical situation. In a real lab,
when a measurement is made the observation can be made to feed into
another device that then makes another measurement conditioned on the
results of the first. This means that after the superposition was
collapsed the entire experimental set up remained in
superposition. While QM offers a means of writing this down it doesn't
quite line up well with the well-trodden formulation of computation
and continuation that we see so succinctly expressed in Milner's
calculi. This suggests that there might be advantages to this account
of dynamics waiting to be explored.

\paragraph{Quantum logic}
In this connection, we also note that by virtue of having the
Hennessy-Milner construction, we can pull the construction through the
interpretation of QM. This gives us a natural candidate for a quantum
logic that enjoys an extremely tight connection with it's domain of
interpretation, making the construction much less ad hoc (rather it is
the image of functor!).

\paragraph{Quantum probabiity}
i have questions about the basis of the interpretation of inner
product as probability amplitude. In particular, using which
axiomatization of probability theory does the notion of probability
amplitude earn the right to be so dubbed? In other words, where is the
proof that the operation for calculating a probability amplitude (and
then squaring) satisfies the axioms of what it means to calculate a
probability? Even if such a proof exists (i have yet to find it in the
literature), i wonder if it might not be possible to turn things on
their heads. Can we view the calculation of the probability amplitude
as an axiomatization of probability? If so, then the definition we
give for calculating probability amplitude may provide the basis for
an \emph{effective} theory of probability.

\paragraph{Quantum vs ``biological'' information}
Finally, i want to conclude with a more philosophical observation. At
a recent workshop in which QM was a predominant topic i noticed
something about quantum information. The speaker was giving a riveting
discussion of axiomatic QM and showing how properties of ``no
cloning'' and ``no deleting'' emerged as consequences of the
axiomatization. Theorems of this form are necessary to give us a sense
of confidence that our axioms characterize the physical theory. What
struck me, though, was that if quantum information is neither erasable
nor replicable it is markedly different from \emph{life}. Two of the
things we know about life is that

\begin{itemize}
  \item it ends;
  \item to gain some measure of persistence, to transcend it's
    finitude it is imminently copyable.
\end{itemize}

Both of these qualities are summarized succinctly in the aphorism: all
flesh is grass. For me these two kinds of ``information'' -- call them
quantum and biological -- are end points on a spectrum of strategies
for persistence. At one end, we have those curious entities that enjoy
uniqueness and permanence; at the other, we have those who in the face
of a certain end and an uncertain present make a go of passing
something on. To me one of the more remarkable aspects of the latter
strategy is that in the presence of noise (and certain features of
copying) we get a kind of dynamism, a chance for improvement against a
given persistent condition.

% subsection other_calculi_other_bisimulations_and_geometry_as_behavior (end)




% section conclusion (end)

%\documentclass[12pt]{llncs}
%\documentclass{jktr}

\usepackage[pdftex]{hyperref}                   
\usepackage {listings}
\usepackage {mathpartir}
\usepackage{bcprules}
%\usepackage{listings}
                       
\usepackage{graphicx} 
%\usepackage[margins=2.5cm,nohead,nofoot]{geometry}
%\usepackage{geometry}
\usepackage{amsfonts}
\usepackage{amstext}
\usepackage{latexsym}
\usepackage{amssymb}
\usepackage{color}


%\include{myPreamble}
\include{qm2pi.local} 

%\ifpdf
%\usepackage[pdftex]{graphicx}
%\else
%\usepackage{graphicx}
%\fi

 % \ifpdf
%  \usepackage{pdfsync}
%  \if


%\title{Brief Article}
%\author{David F. Snyder}
%\author{L.G. Meredith}

%\address{Dept. of Math., Texas State University--San Marcos, San Marcos, TX 78666}
       
\pagestyle{empty}


\begin{document}

\lstset{language=[Objective]Caml,frame=shadowbox}

\input{qm2pi.front}

% section front matter (end)

\input{qm2pi.intro} 
 
% section introduction (end)

% \input{qm2pi.knotations} 

% section notation (end)

\input{qm2pi.process.calculi} 

% section concurrent_process_calculi_and_spatial_logics_ (end)
    
%\input{qm2pi.knots2pi} 

%\input{qm2pi.trefoil} 

%\input{qm2pi.mainthm} 

% subsection basic_interpretation (end)

%\input{qm2pi.rho.presentation} 
\subsection{The syntax and semantics of the notation system}\label{sub:the_syntax_and_semantics_of_the_notation_system} % (fold)

We now summarize a technical presentation of the calculus that
embodies our theory of dynamics. The typical presentation of such a
calculus follows the style of giving generators and relations on
them. The grammar, below, describing term constructors, freely
generates the set of processes, $\Proc$. This set is then quotiented
by a relation known as structural congruence and it is over this set
that the notion of dynamics is expressed. This presentation is
essentially that of \cite{MeredithR05} with the addition of
polyadicity and summation. For readability we have relegated some of
the technical subtleties to an appendix.

\subsubsection{Process grammar}\label{subsub:process_grammar}

\begin{mathpar}
  \inferrule* [lab=synchronization] {} {{M} \bc \pzero \;|\; x?F \;|\; x!C }
  \and
  \inferrule* [lab=abstraction] {} {{F} \bc (x)P}
  \and
  \inferrule* [lab=concretion] {} {{C} \bc \langle Q \rangle}
  \and
  \inferrule* [lab=process] {} {{P,Q} \bc M \;| \;P|Q \;|\; @{x}}
  \and
  \inferrule* [lab=name] {} {{x} \bc \quotep{P}}
\end{mathpar} 

Note that $\vec{x}$ (resp. $\vec{P}$) denotes a vector of names
(resp. processes) of length $|\vec{x}|$ (resp. $|\vec{P}|$). We adopt
the following useful abbreviations.

\begin{mathpar}
   x?(\vec{y}).P := x.(\vec{y})P \and  x\clift{\vec{P}} := x.\clift{\vec{P}}
   \and x!(y) := \lift{x}{\dropn{y}}
   \and \Pi_{i=0}^{n-1}P_i := P_0 | \ldots | P_{n-1}
\end{mathpar}

\subsubsection{Structural congruence}

\paragraph{Free and bound names and alpha-equivalence.} At the
core of structural equivalence is alpha-equivalence which identifies
process that are the same up to a change of variable. Formally, we
recognize the distinction between free and bound names. The free names
of a process, $\freenames{P}$, may be calculated recursively as
follows:

\begin{mathpar}
\freenames{\pzero} := \emptyset
  \and \\
  \freenames{x?(y).P} := \{ x \} \cup (\freenames{P} \setminus \{ y \})
  \and 
  \freenames{x!\langle P \rangle} := \{ x \} \cup \{ P \} 
  \and \\
  \freenames{P|Q} := \freenames{P} \cup \freenames{Q}
  \and \\
  \freenames{@{x}} := \{ x \}
\end{mathpar}

$\pi$
$\quotep{\pi}$

$\freenames{-} : \pi \to \mathcal{P}(\quotep{\pi})$

\begin{eqnarray*}
  \freenames{\pzero} & := & \emptyset \\
  \freenames{x?(y).P} & := & \{ x \} \cup (\freenames{P} \setminus \{ y \}) \\
  \freenames{x!\langle P \rangle} & := & \{ x \} \cup \{ P \} \\
  \freenames{P|Q} & := & \freenames{P} \cup \freenames{Q} \\
  \freenames{\dropn{x}} & := & \{ x \}
\end{eqnarray*}

The bound names of a process, $\boundnames{P}$, are those names occurring in $P$
that are not free. For example, in $x?(y).0$, the name $x$ is free, while $y$ is bound.

\begin{mathpar}
  \inferrule* [lab=monoidal-laws] {} { P|Q \equiv Q|P \and P|0 \equiv P \and P|(Q|R) \equiv (P|Q)|R }
\end{mathpar}

\begin{mathpar}
  \inferrule* [lab=alpha-equivalence] {} { (x)P \equiv (y)P\{y/x\} \and y \not\in \freenames{P} }
\end{mathpar}

\begin{definition}
Then two processes, $P,Q$, are alpha-equivalent if $P = Q\{\vec{y}/\vec{x}\}$ for
some $\vec{x} \in \boundnames{Q},\vec{y} \in \boundnames{P}$, where $Q\{\vec{y}/\vec{x}\}$
denotes the capture-avoiding substitution of $\vec{y}$ for $\vec{x}$ in $Q$.
\end{definition}

\begin{definition}
  The {\em structural congruence} \cite{SangiorgiWalker} , $\equiv$,
  between processes is the least congruence containing
  alpha-equivalence, satisfying the abelian monoid laws
  (associativity, commutativity and $\pzero$ as identity) for parallel
  composition $|$ and for summation $+$.
\end{definition}

\subsection{Name equivalence}

We take name equivalence, written $\nameeq$, to be the smallest
equivalence relation generated by the following rules.

\begin{mathpar}
\inferrule*[lab=Quote-drop]
{ }
{ \quotep{@{x}} \nameeq x }

\inferrule*[lab=Struct-equiv]
{ P \scong Q }
{ \quotep{P} \nameeq \quotep{Q} }
\end{mathpar}

The astute reader will have noticed that the mutual recursion of names
and processes imposes a mutual recursion on alpha-equivalence and
structural equivalence via name-equivalence. Fortunately, all of this
works out pleasantly and we may calculate in the natural way, free of
concern. The reader interested in the details is referred to the
appendix \ref{appendix:rho_details}.

\subsection{Substitution}

We use $\Proc$ for the set of processes, $\QProc$ for the set of
names, and $\id{\{}\vec{y} / \vec{x} \id{\}}$ to denote partial maps,
$s : \QProc \rightarrow \QProc$. A map, $s$ lifts, uniquely, to a map
on process terms, $\widehat{s} : \Proc \rightarrow \Proc$ by the
following equations.

\begin{mathpar}
  (0) \psubstp{Q}{P} := 0 \\
  (R \juxtap S) \psubstp{Q}{P}
  :=    
  (R)\psubstp{Q}{P} \juxtap (S) \psubstp{Q}{P} \\
  (x?(y).R) \psubstp{Q}{P}    
  :=    
  (x)\substp{Q}{P} (z)\concat( (R \psubstn{z}{y}) \psubstp{Q}{P} ) \\
  (\lift{x}{R}) \psubstp{Q}{P}  
  :=
  \lift{(x)\substp{Q}{P}}{ R \psubstp{Q}{P} } \\
%   (\dropn{x})  \psubstp{Q}{P}       
%   := 
%   \left\{ 
%     \begin{array}{ccc} 
%       \dropn{\quotep{Q}} & & x \nameeq \quotep{P} \\
%       \dropn{x} & & otherwise \\
%     \end{array}
%   \right. 
  (\dropn{x})  \psubstp{Q}{P}       
  := 
  \left\{ 
    \begin{array}{ccc} 
      Q & & x \nameeq \quotep{P} \\
      \dropn{x} & & otherwise \\
    \end{array}
  \right.
\end{mathpar}
 

where

\begin{eqnarray}
  (x)\id{\{} \lpquote Q \rpquote / \lpquote P \rpquote \id{\}}            = 
  \left\{ 
    \begin{array}{ccc}
      \lpquote Q \rpquote & & x \nameeq \lpquote P \rpquote \\
      x & & otherwise \\
    \end{array}
  \right. \nonumber
\end{eqnarray}

and $z$ is chosen distinct from $\quotep{P}$, $\quotep{Q}$, the free
names in $Q$, and all the names in $R$. Our $\alpha$-equivalence will
be built in the standard way from this substitution.

\begin{remark}\label{rem:no_self_referential_names}
  One consequence of these definitions is that $\forall P. \quotep{P}
  \not\in \freenames{P}$.
\end{remark}

\subsection{ Dynamic quote: an example }

Anticipating something of what's to come, consider applying the
substitution, $\widehat{\id{\{}u / z \id{\}}}$, to the following pair
of processes, $\lift{w}{y!(z)}$ and $w[ \lpquote y!(z) \rpquote ]$.

\begin{eqnarray}
	\lift{w}{y!(z)}\widehat{\id{\{}u / z \id{\}}}
		& = &
		\lift{w}{y!(u)} \nonumber\\
	w[ \lpquote y!(z) \rpquote ] \widehat{ \id{\{}u / z \id{\}} }
		& = &
		w[ \lpquote y!(z) \rpquote ] \nonumber
\end{eqnarray}

Because the body of the process between quotes is impervious to
substitution, we get radically different answers. In fact, by
examining the first process in an input context,
e.g. $x?(z).\lift{w}{y!(z)}$, we see that the process under the lift
operator may be shaped by prefixed inputs binding a name inside it. In
this sense, the lift operator will be seen as a way to dynamically
construct processes before reifying them as names.

Finally equipped with these standard features we can present the
dynamics of the calculus.

\subsubsection{Operational semantics} 

Finally, we introduce the computational dynamics. What marks these
algebras as distinct from other more traditionally studied algebraic
structures, e.g. vector spaces or polynomial rings, is the manner in
which dynamics is captured. In traditional structures, dynamics is typically
expressed through morphisms between such structures, as in linear maps
between vector spaces or morphisms between rings. In algebras
associated with the semantics of computation, the dynamics is
expressed as part of the algebraic structure itself, through a
reduction reduction relation typically denoted by $\red$. Below, we
give a recursive presentation of this relation for the calculus used
in the encoding.

$\red \subseteq \pi \times \pi$
$\red : \pi \to \mathcal{P}(\pi)$

\begin{mathpar}
  \inferrule* [lab=Comm] { \textsf{match}( x_{src}, x_{trgt} ) } { x_{trgt}?(y)P \; | \; x_{src}!\langle {Q} \rangle \red P\{\quotep{Q}/y}\} }
  \and \\
  \inferrule* [lab=Par] {{P} \red {P}'} {{{P} | {Q}} \red {{P}' | {Q}}}
  \and
  \inferrule* [lab=Equiv]{{{P} \scong {P}'} \andalso {{P}' \red {Q}'} \andalso {{Q}' \scong {Q}}}{{P} \red {Q}}
\end{mathpar}

\begin{eqnarray*}
  match_{\equiv} (\quotep{P},\quotep{Q}) & := & P \equiv Q \\
  match_{\dagger}(\quotep{P},\quotep{Q}) & := & \forall R. P|Q \red^{*} R => R \red^{*} 0 \\
  match_{K}(\quotep{P},\quotep{Q}) & := & K \mbox{ for some context } K
\end{eqnarray*}

$u?(x)P | u!\langle Q \rangle \red P\{\quotep{Q}/x\}$

%We write $\wred$ for $\red^*$, and $P\red$ if $\exists Q $ such that $ P \red Q$.
We write $P\red$ if $\exists Q $ such that $ P \red Q$ and $P\not\red$, otherwise.

\section{Replication}

As mentioned before, it is known that replication (and hence
recursion) can be implemented in a higher-order process algebra
\cite{SangiorgiWalker}. As our first example of calculation with the
machinery thus far presented we give the construction explicitly in
the {\rhoc}.

\begin{eqnarray}
	D_{x} & := & \prefix{x}{y}{(\binpar{\outputp{x}{y}}{@{y}})} \nonumber\\
	\bangp_{x}{P} & := & \binpar{{x}!\langle{\binpar{D_{x}}{P}}\rangle}{D_{x}} \nonumber
\end{eqnarray}

\begin{eqnarray}
	\bangp_{x}{P} & & \nonumber\\
	=
	& {x}!\langle{(\prefix{x}{y}{(\outputp{x}{y} | @{y})) | P}}\rangle 
	      | \prefix{x}{y}{(\outputp{x}{y} | @{y})} & \nonumber\\
	\red
	& (\outputp{x}{y} | @{y})\substn{\quotep{(\prefix{x}{y}{(@{y} | \outputp{x}{y})) | P}}}{y} & \nonumber\\
	=
	& \outputp{x}{\quotep{(\prefix{x}{y}{(\outputp{x}{y} | @{y})) | P}}}
	  | {(\prefix{x}{y}{(\outputp{x}{y} | @{y})) | P}} & \nonumber\\
	\red
	& \ldots & \nonumber\\
	\red^*
	& P | P | \ldots & \nonumber
\end{eqnarray}

Of course, this encoding, as an implementation, runs away, unfolding
$\bangp{P}$ eagerly. A lazier and more implementable replication
operator, restricted to input-guarded processes, may be obtained as follows.

\begin{eqnarray}
\bangp{\prefix{u}{v}{P}} 
	:= 
	\binpar{\lift{x}{\prefix{u}{v}{(\binpar{D(x)}{P})}}}{D(x)} \nonumber
\end{eqnarray}

\begin{remark}
  Note that the lazier definition still does not deal with summation
  or mixed summation (i.e. sums over input and output). The reader is
  invited to construct definitions of replication that deal with these
  features. 

  Further, the definitions are parameterized in a name, $x$. Can you,
  gentle reader, make a definition that eliminates this parameter and
  guarantees no accidental interaction between the replication
  machinery and the process being replicated -- i.e. no accidental
  sharing of names used by the process to get its work done and the
  name(s) used by the replication to effect copying. This latter
  revision of the definition of replication is crucial to obtaining
  the expected identity $!!P \sim !P$.
\end{remark}

\begin{remark}\label{rem:paradoxical_combinator}
  The reader familiar with the lambda calculus will have noticed the
  similarity between $D$ and the paradoxical combinator.

  [Ed. note: the existence of this seems to suggest we have to be more
  restrictive on the set of processes and names we admit if we are to
  support no-cloning.]
\end{remark}

\subsubsection{Bisimulation}

The computational dynamics gives rise to another kind of equivalence,
the equivalence of computational behavior. As previously mentioned
this is typically captured \emph{via} some form of bisimulation.

% The notion we use in this paper is weak barbed bisimulation
% \cite{milner91polyadicpi}.

The notion we use in this paper is derived from weak barbed
bisimulation \cite{milner91polyadicpi}. 

\begin{definition}
An \emph{observation relation}, $\downarrow_{\mathcal N}$, over a set
of names, $\mathcal N$, is the smallest relation satisfying the rules
below.

\infrule[Out-barb]{y \in {\mathcal N}, \; x \nameeq y}
		  {\outputp{x}{v} \downarrow_{\mathcal N} x}
\infrule[Par-barb]{\mbox{$P\downarrow_{\mathcal N} x$ or $Q\downarrow_{\mathcal N} x$}}
		  {\binpar{P}{Q} \downarrow_{\mathcal N} x}

We write $P \Downarrow_{\mathcal N} x$ if there is $Q$ such that 
$P \wred Q$ and $Q \downarrow_{\mathcal N} x$.
\end{definition}

\begin{definition}
%\label{def.bbisim}
An  ${\mathcal N}$-\emph{barbed bisimulation} over a set of names, ${\mathcal N}$, is a symmetric binary relation 
${\mathcal S}_{\mathcal N}$ between agents such that $P\rel{S}_{\mathcal N}Q$ implies:
\begin{enumerate}
\item If $P \red P'$ then $Q \wred Q'$ and $P'\rel{S}_{\mathcal N} Q'$.
\item If $P\downarrow_{\mathcal N} x$, then $Q\Downarrow_{\mathcal N} x$.
\end{enumerate}
$P$ is ${\mathcal N}$-barbed bisimilar to $Q$, written
$P \wbbisim_{\mathcal N} Q$, if $P \rel{S}_{\mathcal N} Q$ for some ${\mathcal N}$-barbed bisimulation ${\mathcal S}_{\mathcal N}$.
\end{definition}

$\mathcal{R} \subseteq \pi \times \pi$

$P \mathcal{R} Q => \forall P'. P \red P' \Rightarrow \exists Q'. Q \red Q', P' \mathcal{R} Q'$

$P \vdash x \Rightarrow Q \vdash x$

\begin{mathpar}
  \inferrule*[lab=Out-barb]{x \nameeq y}{{y}!\langle{Q}\rangle \vdash x}
  \and
  \inferrule*[lab=Par-barb]{\mbox{$P\vdash x$ or $Q\vdash x$}}{\binpar{P}{Q} \vdash x}
\end{mathpar}

\subsubsection{Contexts}

One of the principle advantages of computational calculi like the
$\pi$-calculus is a well-defined notion of context,
contextual-equivalence and a correlation between
contextual-equivalence and notions of bisimulation. The notion of
context allows the decomposition of a process into (sub-)process and
its syntactic environment, its context. Thus, a context may be
thought of as a process with a ``hole'' (written $\Box$) in it. The
application of a context $M$ to a process $P$, written $M[P]$, is
tantamount to filling the hole in $M$ with $P$. In this paper we do
not need the full weight of this theory, but do make use of the notion
of context in the proof the main theorem. 

\begin{mathpar}
  \inferrule* [lab=summation] {} {{M_{M},M_{N}} \bc \Box \;|\; x.M_{A} \;|\; M_{M}+M_{N}}
  \and
  \inferrule* [lab=agent] {} {{M_{A}} \bc (\vec{x})M_{P} \;| \; \clift{P_0,\ldots,M_{P},\ldots,P_N}}
  \and \\
  \inferrule* [lab=process] {} {{M_{P}} \bc M_{N} \;| \;P|M_{P} }
\end{mathpar} 

\begin{mathpar}
  \inferrule* [lab=sychronization] {} {M_{N} \bc \Box \;|\; x?M_{F} \;|\; x!M_{C}}
  \and
  \inferrule* [lab=abstraction] {} {{M_{F}} \bc (x)M_{P} }
  \and
  \inferrule* [lab=concretion] {} {{M_{C}} \bc \langle M_{P} \rangle }
  \and \\
  \inferrule* [lab=process] {} {{M_{P}} \bc M_{N} \;| \;P|M_{P} }
\end{mathpar}

\begin{definition}[contextual application] Given a context $M$, and
  process $P$, we define the \emph{contextual application}, $M[P] :=
  M\{P/\Box\}$. That is, the contextual application of M to P is the
  substitution of $P$ for $\Box$ in $M$.
\end{definition}

$\meaningof{-} : L \to \mathcal{P}(\pi)$

\begin{mathpar}
  \inferrule* [lab=collection] {} {\meaningof{true} = \pi, \and \meaningof{~E} = \pi \setminus \meaningof{E}, \and \meaningof{E_{1} \& E_{2}} = \meaningof{E_{1}} \cap \meaningof{E_{2}}}
\end{mathpar}

\begin{mathpar}
  \inferrule* [lab=structure] {} {\meaningof{0} = \{ P \in \pi | P \equiv 0 \}, \and \\ \meaningof{E_1 | E_2} = \{ P \in \pi | P \equiv P_{1} | P_{2}, P_{1} \in \meaningof{E_{1}}, P_{2} \in \meaningof{E_2}\} }
\end{mathpar}

\begin{mathpar}
 \inferrule* [lab=behavior] {} {\meaningof{\langle a?b \rangle E} = \{ P \in \pi | P \equiv Q | u?(y)P', \\ \and \\\\ \and \\ \;\;\; u \in \meaningof{a}, \forall z.P'\{z/y\} \in \meaningof{E\{z/b\}}\}, \and \\ \meaningof{a!E} = \{ P \in \pi | P \equiv Q | x!\langle P' \rangle, x \in \meaningof{a} P' \in \meaningof{E}\} }
\end{mathpar}

\begin{mathpar}
 \inferrule* [lab=nominal] {} {\meaningof{\quotep{E}} = \{ \quotep{P} \in \quotep{\pi} | P \in \meaningof{E} \}, \and \meaningof{\quotep{P}} = \{ \quotep{Q} \in \quotep{\pi} | P \equiv Q \} \and \\ \meaningof{@\quotep{E}} = \{ P \in \pi | P \equiv @x, x \in \meaningof{E} \}}
\end{mathpar}

\begin{eqnarray*}
  \\
  \meaningof{-} : TS \to ST
\end{eqnarray*}

\begin{eqnarray*}
  \\
  L : TS \to ST
\end{eqnarray*}

\begin{eqnarray*}
  \\
  P \models E \iff P \in \meaningof{E}
\end{eqnarray*}

\begin{eqnarray*}
  P \approx_{L} Q \iff \forall E \in L. P \models E \iff Q \models E
\end{eqnarray*}

\begin{eqnarray*}
  P \approx_{K} Q
\end{eqnarray*}

\begin{eqnarray*}
  P \approx Q
\end{eqnarray*}

$\approx_{K} = \approx = \approx_{L}$

\subsubsection{Contextual duality}

Note that contexts extend the quotation operation to a family of
operations from processes to names. Given a context, $M$, we can
define a \emph{nominal context}, $\quotep{M}$ by $\quotep{M}[P] :=
\quotep{M[P]}$. To foreshadow what is to come we observe that these
operations enjoy a duality with processes very much like the duality
between vectors and maps from vectors to scalars.

Further, because the calculus is essentially higher-order, we have a
correspondence between contexts and processes. More specifically,
given a name $x$ and a context $M$ we can construct $M^{*}_{x}$ such
that 

\begin{mathpar}
  M^{*}_{x} | \lift{x}{P} \red M[P]
\end{mathpar}

namely,

\begin{mathpar}
  M^{*}_{x} := x?(u).M[\dropn{u}]
\end{mathpar}

The dependence of $M^{*}_{x}$ on a name makes it an abstraction, 

\begin{mathpar}
  M^{*} := (x)x?(u).M[\dropn{u}]
\end{mathpar}

\subsection{Additional notation}

It will sometimes be convenient to denote the process a name
quotes. We already have the notation $x = \quotep{P}$, but it will be
convenient to introduce an alternate notation, $\procn{x}$, when we
want to emphasize the connection to the use of the name. Note that, by
virtue of name equivalence, $\quotep{\procn{x}} \nameeq x$; so, the
notation is consistent with previous definitions.

Further, because names have structure it is possible to effect
substitutions on the basis of that structure. This means we need to
upgrade our notation for substitutions, which we accomplish by
adapting comprehension notation. Thus,

\begin{mathpar}
  P\{ y / x : x \in S \}
\end{mathpar}

is interpreted to mean the process derived from P by replacing (in a
capture-avoiding manner) each occurrence of $x$ in $S$ by $y$. For example,

\begin{mathpar}
  P\{ \quotep{\procn{x}|\procn{x}} / x : x \in \freenames{P} \}
\end{mathpar}

will replace each (occurrence) of a free name $x$ in $P$ by
$\quotep{\procn{x}|\procn{x}}$.

Also, we will avail ourselves of the notation $x^{L}$ and $x^{R}$ to
denote injections of a name into disjoint copies of the name
space. There are numerous ways to accomplish this. One example can be
found in \cite{MeredithR05}. This notation overloads to vectors of
names: $\vec{x}^{\pi} := (x_{i}^{\pi} \; : \; 0 \leq i < |\vec{x}| )$ where $\pi \in \{L,R\}$.

We also use $P^{\Box} := P|\Box$.

In \cite{MeredithR05} an interpretation of the new operator is
given. It turns out that there are several possible interpretations
all enjoying the requisite algebraic properties of the operator (see
\cite{milner91polyadicpi}). We will therefore make liberal use of
$(\nu\; \vec{x})P$.

% subsection the_syntax_and_semantics_of_the_notation_system (end)   

\input{qm2pi.qmops} 

\input{qm2pi.sterngerlach} 

\input{qm2pi.metric} 

% section concurrent_process_calculi (end)

%\input{qm2pi.proofsketch}

% section proof sketch (end)

%\input{qm2pi.slviaknots} 

% section spatial logic via knots (end)

\input{qm2pi.conclusion}

% section conclusion (end)

%\input{qm2pi.dtcodes} 

% section wiring algorithm (end)

\input{qm2pi.ack} 

% section acknowledgments (end)

\newpage


\bibliographystyle{plain}   
\bibliography{../../biblios/main.bib}

\input{qm2pi.rhodetails}

\end{document}

 

% section wiring algorithm (end)

\documentclass[12pt]{llncs}
%\documentclass{jktr}

\usepackage[pdftex]{hyperref}                   
\usepackage {listings}
\usepackage {mathpartir}
\usepackage{bcprules}
%\usepackage{listings}
                       
\usepackage{graphicx} 
%\usepackage[margins=2.5cm,nohead,nofoot]{geometry}
%\usepackage{geometry}
\usepackage{amsfonts}
\usepackage{amstext}
\usepackage{latexsym}
\usepackage{amssymb}
\usepackage{color}


%\include{myPreamble}
\include{qm2pi.local} 

%\ifpdf
%\usepackage[pdftex]{graphicx}
%\else
%\usepackage{graphicx}
%\fi

 % \ifpdf
%  \usepackage{pdfsync}
%  \if


%\title{Brief Article}
%\author{David F. Snyder}
%\author{L.G. Meredith}

%\address{Dept. of Math., Texas State University--San Marcos, San Marcos, TX 78666}
       
\pagestyle{empty}


\begin{document}

\lstset{language=[Objective]Caml,frame=shadowbox}

\input{qm2pi.front}

% section front matter (end)

\input{qm2pi.intro} 
 
% section introduction (end)

% \input{qm2pi.knotations} 

% section notation (end)

\input{qm2pi.process.calculi} 

% section concurrent_process_calculi_and_spatial_logics_ (end)
    
%\input{qm2pi.knots2pi} 

%\input{qm2pi.trefoil} 

%\input{qm2pi.mainthm} 

% subsection basic_interpretation (end)

%\input{qm2pi.rho.presentation} 
\subsection{The syntax and semantics of the notation system}\label{sub:the_syntax_and_semantics_of_the_notation_system} % (fold)

We now summarize a technical presentation of the calculus that
embodies our theory of dynamics. The typical presentation of such a
calculus follows the style of giving generators and relations on
them. The grammar, below, describing term constructors, freely
generates the set of processes, $\Proc$. This set is then quotiented
by a relation known as structural congruence and it is over this set
that the notion of dynamics is expressed. This presentation is
essentially that of \cite{MeredithR05} with the addition of
polyadicity and summation. For readability we have relegated some of
the technical subtleties to an appendix.

\subsubsection{Process grammar}\label{subsub:process_grammar}

\begin{mathpar}
  \inferrule* [lab=synchronization] {} {{M} \bc \pzero \;|\; x?F \;|\; x!C }
  \and
  \inferrule* [lab=abstraction] {} {{F} \bc (x)P}
  \and
  \inferrule* [lab=concretion] {} {{C} \bc \langle Q \rangle}
  \and
  \inferrule* [lab=process] {} {{P,Q} \bc M \;| \;P|Q \;|\; @{x}}
  \and
  \inferrule* [lab=name] {} {{x} \bc \quotep{P}}
\end{mathpar} 

Note that $\vec{x}$ (resp. $\vec{P}$) denotes a vector of names
(resp. processes) of length $|\vec{x}|$ (resp. $|\vec{P}|$). We adopt
the following useful abbreviations.

\begin{mathpar}
   x?(\vec{y}).P := x.(\vec{y})P \and  x\clift{\vec{P}} := x.\clift{\vec{P}}
   \and x!(y) := \lift{x}{\dropn{y}}
   \and \Pi_{i=0}^{n-1}P_i := P_0 | \ldots | P_{n-1}
\end{mathpar}

\subsubsection{Structural congruence}

\paragraph{Free and bound names and alpha-equivalence.} At the
core of structural equivalence is alpha-equivalence which identifies
process that are the same up to a change of variable. Formally, we
recognize the distinction between free and bound names. The free names
of a process, $\freenames{P}$, may be calculated recursively as
follows:

\begin{mathpar}
\freenames{\pzero} := \emptyset
  \and \\
  \freenames{x?(y).P} := \{ x \} \cup (\freenames{P} \setminus \{ y \})
  \and 
  \freenames{x!\langle P \rangle} := \{ x \} \cup \{ P \} 
  \and \\
  \freenames{P|Q} := \freenames{P} \cup \freenames{Q}
  \and \\
  \freenames{@{x}} := \{ x \}
\end{mathpar}

$\pi$
$\quotep{\pi}$

$\freenames{-} : \pi \to \mathcal{P}(\quotep{\pi})$

\begin{eqnarray*}
  \freenames{\pzero} & := & \emptyset \\
  \freenames{x?(y).P} & := & \{ x \} \cup (\freenames{P} \setminus \{ y \}) \\
  \freenames{x!\langle P \rangle} & := & \{ x \} \cup \{ P \} \\
  \freenames{P|Q} & := & \freenames{P} \cup \freenames{Q} \\
  \freenames{\dropn{x}} & := & \{ x \}
\end{eqnarray*}

The bound names of a process, $\boundnames{P}$, are those names occurring in $P$
that are not free. For example, in $x?(y).0$, the name $x$ is free, while $y$ is bound.

\begin{mathpar}
  \inferrule* [lab=monoidal-laws] {} { P|Q \equiv Q|P \and P|0 \equiv P \and P|(Q|R) \equiv (P|Q)|R }
\end{mathpar}

\begin{mathpar}
  \inferrule* [lab=alpha-equivalence] {} { (x)P \equiv (y)P\{y/x\} \and y \not\in \freenames{P} }
\end{mathpar}

\begin{definition}
Then two processes, $P,Q$, are alpha-equivalent if $P = Q\{\vec{y}/\vec{x}\}$ for
some $\vec{x} \in \boundnames{Q},\vec{y} \in \boundnames{P}$, where $Q\{\vec{y}/\vec{x}\}$
denotes the capture-avoiding substitution of $\vec{y}$ for $\vec{x}$ in $Q$.
\end{definition}

\begin{definition}
  The {\em structural congruence} \cite{SangiorgiWalker} , $\equiv$,
  between processes is the least congruence containing
  alpha-equivalence, satisfying the abelian monoid laws
  (associativity, commutativity and $\pzero$ as identity) for parallel
  composition $|$ and for summation $+$.
\end{definition}

\subsection{Name equivalence}

We take name equivalence, written $\nameeq$, to be the smallest
equivalence relation generated by the following rules.

\begin{mathpar}
\inferrule*[lab=Quote-drop]
{ }
{ \quotep{@{x}} \nameeq x }

\inferrule*[lab=Struct-equiv]
{ P \scong Q }
{ \quotep{P} \nameeq \quotep{Q} }
\end{mathpar}

The astute reader will have noticed that the mutual recursion of names
and processes imposes a mutual recursion on alpha-equivalence and
structural equivalence via name-equivalence. Fortunately, all of this
works out pleasantly and we may calculate in the natural way, free of
concern. The reader interested in the details is referred to the
appendix \ref{appendix:rho_details}.

\subsection{Substitution}

We use $\Proc$ for the set of processes, $\QProc$ for the set of
names, and $\id{\{}\vec{y} / \vec{x} \id{\}}$ to denote partial maps,
$s : \QProc \rightarrow \QProc$. A map, $s$ lifts, uniquely, to a map
on process terms, $\widehat{s} : \Proc \rightarrow \Proc$ by the
following equations.

\begin{mathpar}
  (0) \psubstp{Q}{P} := 0 \\
  (R \juxtap S) \psubstp{Q}{P}
  :=    
  (R)\psubstp{Q}{P} \juxtap (S) \psubstp{Q}{P} \\
  (x?(y).R) \psubstp{Q}{P}    
  :=    
  (x)\substp{Q}{P} (z)\concat( (R \psubstn{z}{y}) \psubstp{Q}{P} ) \\
  (\lift{x}{R}) \psubstp{Q}{P}  
  :=
  \lift{(x)\substp{Q}{P}}{ R \psubstp{Q}{P} } \\
%   (\dropn{x})  \psubstp{Q}{P}       
%   := 
%   \left\{ 
%     \begin{array}{ccc} 
%       \dropn{\quotep{Q}} & & x \nameeq \quotep{P} \\
%       \dropn{x} & & otherwise \\
%     \end{array}
%   \right. 
  (\dropn{x})  \psubstp{Q}{P}       
  := 
  \left\{ 
    \begin{array}{ccc} 
      Q & & x \nameeq \quotep{P} \\
      \dropn{x} & & otherwise \\
    \end{array}
  \right.
\end{mathpar}
 

where

\begin{eqnarray}
  (x)\id{\{} \lpquote Q \rpquote / \lpquote P \rpquote \id{\}}            = 
  \left\{ 
    \begin{array}{ccc}
      \lpquote Q \rpquote & & x \nameeq \lpquote P \rpquote \\
      x & & otherwise \\
    \end{array}
  \right. \nonumber
\end{eqnarray}

and $z$ is chosen distinct from $\quotep{P}$, $\quotep{Q}$, the free
names in $Q$, and all the names in $R$. Our $\alpha$-equivalence will
be built in the standard way from this substitution.

\begin{remark}\label{rem:no_self_referential_names}
  One consequence of these definitions is that $\forall P. \quotep{P}
  \not\in \freenames{P}$.
\end{remark}

\subsection{ Dynamic quote: an example }

Anticipating something of what's to come, consider applying the
substitution, $\widehat{\id{\{}u / z \id{\}}}$, to the following pair
of processes, $\lift{w}{y!(z)}$ and $w[ \lpquote y!(z) \rpquote ]$.

\begin{eqnarray}
	\lift{w}{y!(z)}\widehat{\id{\{}u / z \id{\}}}
		& = &
		\lift{w}{y!(u)} \nonumber\\
	w[ \lpquote y!(z) \rpquote ] \widehat{ \id{\{}u / z \id{\}} }
		& = &
		w[ \lpquote y!(z) \rpquote ] \nonumber
\end{eqnarray}

Because the body of the process between quotes is impervious to
substitution, we get radically different answers. In fact, by
examining the first process in an input context,
e.g. $x?(z).\lift{w}{y!(z)}$, we see that the process under the lift
operator may be shaped by prefixed inputs binding a name inside it. In
this sense, the lift operator will be seen as a way to dynamically
construct processes before reifying them as names.

Finally equipped with these standard features we can present the
dynamics of the calculus.

\subsubsection{Operational semantics} 

Finally, we introduce the computational dynamics. What marks these
algebras as distinct from other more traditionally studied algebraic
structures, e.g. vector spaces or polynomial rings, is the manner in
which dynamics is captured. In traditional structures, dynamics is typically
expressed through morphisms between such structures, as in linear maps
between vector spaces or morphisms between rings. In algebras
associated with the semantics of computation, the dynamics is
expressed as part of the algebraic structure itself, through a
reduction reduction relation typically denoted by $\red$. Below, we
give a recursive presentation of this relation for the calculus used
in the encoding.

$\red \subseteq \pi \times \pi$
$\red : \pi \to \mathcal{P}(\pi)$

\begin{mathpar}
  \inferrule* [lab=Comm] { \textsf{match}( x_{src}, x_{trgt} ) } { x_{trgt}?(y)P \; | \; x_{src}!\langle {Q} \rangle \red P\{\quotep{Q}/y}\} }
  \and \\
  \inferrule* [lab=Par] {{P} \red {P}'} {{{P} | {Q}} \red {{P}' | {Q}}}
  \and
  \inferrule* [lab=Equiv]{{{P} \scong {P}'} \andalso {{P}' \red {Q}'} \andalso {{Q}' \scong {Q}}}{{P} \red {Q}}
\end{mathpar}

\begin{eqnarray*}
  match_{\equiv} (\quotep{P},\quotep{Q}) & := & P \equiv Q \\
  match_{\dagger}(\quotep{P},\quotep{Q}) & := & \forall R. P|Q \red^{*} R => R \red^{*} 0 \\
  match_{K}(\quotep{P},\quotep{Q}) & := & K \mbox{ for some context } K
\end{eqnarray*}

$u?(x)P | u!\langle Q \rangle \red P\{\quotep{Q}/x\}$

%We write $\wred$ for $\red^*$, and $P\red$ if $\exists Q $ such that $ P \red Q$.
We write $P\red$ if $\exists Q $ such that $ P \red Q$ and $P\not\red$, otherwise.

\section{Replication}

As mentioned before, it is known that replication (and hence
recursion) can be implemented in a higher-order process algebra
\cite{SangiorgiWalker}. As our first example of calculation with the
machinery thus far presented we give the construction explicitly in
the {\rhoc}.

\begin{eqnarray}
	D_{x} & := & \prefix{x}{y}{(\binpar{\outputp{x}{y}}{@{y}})} \nonumber\\
	\bangp_{x}{P} & := & \binpar{{x}!\langle{\binpar{D_{x}}{P}}\rangle}{D_{x}} \nonumber
\end{eqnarray}

\begin{eqnarray}
	\bangp_{x}{P} & & \nonumber\\
	=
	& {x}!\langle{(\prefix{x}{y}{(\outputp{x}{y} | @{y})) | P}}\rangle 
	      | \prefix{x}{y}{(\outputp{x}{y} | @{y})} & \nonumber\\
	\red
	& (\outputp{x}{y} | @{y})\substn{\quotep{(\prefix{x}{y}{(@{y} | \outputp{x}{y})) | P}}}{y} & \nonumber\\
	=
	& \outputp{x}{\quotep{(\prefix{x}{y}{(\outputp{x}{y} | @{y})) | P}}}
	  | {(\prefix{x}{y}{(\outputp{x}{y} | @{y})) | P}} & \nonumber\\
	\red
	& \ldots & \nonumber\\
	\red^*
	& P | P | \ldots & \nonumber
\end{eqnarray}

Of course, this encoding, as an implementation, runs away, unfolding
$\bangp{P}$ eagerly. A lazier and more implementable replication
operator, restricted to input-guarded processes, may be obtained as follows.

\begin{eqnarray}
\bangp{\prefix{u}{v}{P}} 
	:= 
	\binpar{\lift{x}{\prefix{u}{v}{(\binpar{D(x)}{P})}}}{D(x)} \nonumber
\end{eqnarray}

\begin{remark}
  Note that the lazier definition still does not deal with summation
  or mixed summation (i.e. sums over input and output). The reader is
  invited to construct definitions of replication that deal with these
  features. 

  Further, the definitions are parameterized in a name, $x$. Can you,
  gentle reader, make a definition that eliminates this parameter and
  guarantees no accidental interaction between the replication
  machinery and the process being replicated -- i.e. no accidental
  sharing of names used by the process to get its work done and the
  name(s) used by the replication to effect copying. This latter
  revision of the definition of replication is crucial to obtaining
  the expected identity $!!P \sim !P$.
\end{remark}

\begin{remark}\label{rem:paradoxical_combinator}
  The reader familiar with the lambda calculus will have noticed the
  similarity between $D$ and the paradoxical combinator.

  [Ed. note: the existence of this seems to suggest we have to be more
  restrictive on the set of processes and names we admit if we are to
  support no-cloning.]
\end{remark}

\subsubsection{Bisimulation}

The computational dynamics gives rise to another kind of equivalence,
the equivalence of computational behavior. As previously mentioned
this is typically captured \emph{via} some form of bisimulation.

% The notion we use in this paper is weak barbed bisimulation
% \cite{milner91polyadicpi}.

The notion we use in this paper is derived from weak barbed
bisimulation \cite{milner91polyadicpi}. 

\begin{definition}
An \emph{observation relation}, $\downarrow_{\mathcal N}$, over a set
of names, $\mathcal N$, is the smallest relation satisfying the rules
below.

\infrule[Out-barb]{y \in {\mathcal N}, \; x \nameeq y}
		  {\outputp{x}{v} \downarrow_{\mathcal N} x}
\infrule[Par-barb]{\mbox{$P\downarrow_{\mathcal N} x$ or $Q\downarrow_{\mathcal N} x$}}
		  {\binpar{P}{Q} \downarrow_{\mathcal N} x}

We write $P \Downarrow_{\mathcal N} x$ if there is $Q$ such that 
$P \wred Q$ and $Q \downarrow_{\mathcal N} x$.
\end{definition}

\begin{definition}
%\label{def.bbisim}
An  ${\mathcal N}$-\emph{barbed bisimulation} over a set of names, ${\mathcal N}$, is a symmetric binary relation 
${\mathcal S}_{\mathcal N}$ between agents such that $P\rel{S}_{\mathcal N}Q$ implies:
\begin{enumerate}
\item If $P \red P'$ then $Q \wred Q'$ and $P'\rel{S}_{\mathcal N} Q'$.
\item If $P\downarrow_{\mathcal N} x$, then $Q\Downarrow_{\mathcal N} x$.
\end{enumerate}
$P$ is ${\mathcal N}$-barbed bisimilar to $Q$, written
$P \wbbisim_{\mathcal N} Q$, if $P \rel{S}_{\mathcal N} Q$ for some ${\mathcal N}$-barbed bisimulation ${\mathcal S}_{\mathcal N}$.
\end{definition}

$\mathcal{R} \subseteq \pi \times \pi$

$P \mathcal{R} Q => \forall P'. P \red P' \Rightarrow \exists Q'. Q \red Q', P' \mathcal{R} Q'$

$P \vdash x \Rightarrow Q \vdash x$

\begin{mathpar}
  \inferrule*[lab=Out-barb]{x \nameeq y}{{y}!\langle{Q}\rangle \vdash x}
  \and
  \inferrule*[lab=Par-barb]{\mbox{$P\vdash x$ or $Q\vdash x$}}{\binpar{P}{Q} \vdash x}
\end{mathpar}

\subsubsection{Contexts}

One of the principle advantages of computational calculi like the
$\pi$-calculus is a well-defined notion of context,
contextual-equivalence and a correlation between
contextual-equivalence and notions of bisimulation. The notion of
context allows the decomposition of a process into (sub-)process and
its syntactic environment, its context. Thus, a context may be
thought of as a process with a ``hole'' (written $\Box$) in it. The
application of a context $M$ to a process $P$, written $M[P]$, is
tantamount to filling the hole in $M$ with $P$. In this paper we do
not need the full weight of this theory, but do make use of the notion
of context in the proof the main theorem. 

\begin{mathpar}
  \inferrule* [lab=summation] {} {{M_{M},M_{N}} \bc \Box \;|\; x.M_{A} \;|\; M_{M}+M_{N}}
  \and
  \inferrule* [lab=agent] {} {{M_{A}} \bc (\vec{x})M_{P} \;| \; \clift{P_0,\ldots,M_{P},\ldots,P_N}}
  \and \\
  \inferrule* [lab=process] {} {{M_{P}} \bc M_{N} \;| \;P|M_{P} }
\end{mathpar} 

\begin{mathpar}
  \inferrule* [lab=sychronization] {} {M_{N} \bc \Box \;|\; x?M_{F} \;|\; x!M_{C}}
  \and
  \inferrule* [lab=abstraction] {} {{M_{F}} \bc (x)M_{P} }
  \and
  \inferrule* [lab=concretion] {} {{M_{C}} \bc \langle M_{P} \rangle }
  \and \\
  \inferrule* [lab=process] {} {{M_{P}} \bc M_{N} \;| \;P|M_{P} }
\end{mathpar}

\begin{definition}[contextual application] Given a context $M$, and
  process $P$, we define the \emph{contextual application}, $M[P] :=
  M\{P/\Box\}$. That is, the contextual application of M to P is the
  substitution of $P$ for $\Box$ in $M$.
\end{definition}

$\meaningof{-} : L \to \mathcal{P}(\pi)$

\begin{mathpar}
  \inferrule* [lab=collection] {} {\meaningof{true} = \pi, \and \meaningof{~E} = \pi \setminus \meaningof{E}, \and \meaningof{E_{1} \& E_{2}} = \meaningof{E_{1}} \cap \meaningof{E_{2}}}
\end{mathpar}

\begin{mathpar}
  \inferrule* [lab=structure] {} {\meaningof{0} = \{ P \in \pi | P \equiv 0 \}, \and \\ \meaningof{E_1 | E_2} = \{ P \in \pi | P \equiv P_{1} | P_{2}, P_{1} \in \meaningof{E_{1}}, P_{2} \in \meaningof{E_2}\} }
\end{mathpar}

\begin{mathpar}
 \inferrule* [lab=behavior] {} {\meaningof{\langle a?b \rangle E} = \{ P \in \pi | P \equiv Q | u?(y)P', \\ \and \\\\ \and \\ \;\;\; u \in \meaningof{a}, \forall z.P'\{z/y\} \in \meaningof{E\{z/b\}}\}, \and \\ \meaningof{a!E} = \{ P \in \pi | P \equiv Q | x!\langle P' \rangle, x \in \meaningof{a} P' \in \meaningof{E}\} }
\end{mathpar}

\begin{mathpar}
 \inferrule* [lab=nominal] {} {\meaningof{\quotep{E}} = \{ \quotep{P} \in \quotep{\pi} | P \in \meaningof{E} \}, \and \meaningof{\quotep{P}} = \{ \quotep{Q} \in \quotep{\pi} | P \equiv Q \} \and \\ \meaningof{@\quotep{E}} = \{ P \in \pi | P \equiv @x, x \in \meaningof{E} \}}
\end{mathpar}

\begin{eqnarray*}
  \\
  \meaningof{-} : TS \to ST
\end{eqnarray*}

\begin{eqnarray*}
  \\
  L : TS \to ST
\end{eqnarray*}

\begin{eqnarray*}
  \\
  P \models E \iff P \in \meaningof{E}
\end{eqnarray*}

\begin{eqnarray*}
  P \approx_{L} Q \iff \forall E \in L. P \models E \iff Q \models E
\end{eqnarray*}

\begin{eqnarray*}
  P \approx_{K} Q
\end{eqnarray*}

\begin{eqnarray*}
  P \approx Q
\end{eqnarray*}

$\approx_{K} = \approx = \approx_{L}$

\subsubsection{Contextual duality}

Note that contexts extend the quotation operation to a family of
operations from processes to names. Given a context, $M$, we can
define a \emph{nominal context}, $\quotep{M}$ by $\quotep{M}[P] :=
\quotep{M[P]}$. To foreshadow what is to come we observe that these
operations enjoy a duality with processes very much like the duality
between vectors and maps from vectors to scalars.

Further, because the calculus is essentially higher-order, we have a
correspondence between contexts and processes. More specifically,
given a name $x$ and a context $M$ we can construct $M^{*}_{x}$ such
that 

\begin{mathpar}
  M^{*}_{x} | \lift{x}{P} \red M[P]
\end{mathpar}

namely,

\begin{mathpar}
  M^{*}_{x} := x?(u).M[\dropn{u}]
\end{mathpar}

The dependence of $M^{*}_{x}$ on a name makes it an abstraction, 

\begin{mathpar}
  M^{*} := (x)x?(u).M[\dropn{u}]
\end{mathpar}

\subsection{Additional notation}

It will sometimes be convenient to denote the process a name
quotes. We already have the notation $x = \quotep{P}$, but it will be
convenient to introduce an alternate notation, $\procn{x}$, when we
want to emphasize the connection to the use of the name. Note that, by
virtue of name equivalence, $\quotep{\procn{x}} \nameeq x$; so, the
notation is consistent with previous definitions.

Further, because names have structure it is possible to effect
substitutions on the basis of that structure. This means we need to
upgrade our notation for substitutions, which we accomplish by
adapting comprehension notation. Thus,

\begin{mathpar}
  P\{ y / x : x \in S \}
\end{mathpar}

is interpreted to mean the process derived from P by replacing (in a
capture-avoiding manner) each occurrence of $x$ in $S$ by $y$. For example,

\begin{mathpar}
  P\{ \quotep{\procn{x}|\procn{x}} / x : x \in \freenames{P} \}
\end{mathpar}

will replace each (occurrence) of a free name $x$ in $P$ by
$\quotep{\procn{x}|\procn{x}}$.

Also, we will avail ourselves of the notation $x^{L}$ and $x^{R}$ to
denote injections of a name into disjoint copies of the name
space. There are numerous ways to accomplish this. One example can be
found in \cite{MeredithR05}. This notation overloads to vectors of
names: $\vec{x}^{\pi} := (x_{i}^{\pi} \; : \; 0 \leq i < |\vec{x}| )$ where $\pi \in \{L,R\}$.

We also use $P^{\Box} := P|\Box$.

In \cite{MeredithR05} an interpretation of the new operator is
given. It turns out that there are several possible interpretations
all enjoying the requisite algebraic properties of the operator (see
\cite{milner91polyadicpi}). We will therefore make liberal use of
$(\nu\; \vec{x})P$.

% subsection the_syntax_and_semantics_of_the_notation_system (end)   

\input{qm2pi.qmops} 

\input{qm2pi.sterngerlach} 

\input{qm2pi.metric} 

% section concurrent_process_calculi (end)

%\input{qm2pi.proofsketch}

% section proof sketch (end)

%\input{qm2pi.slviaknots} 

% section spatial logic via knots (end)

\input{qm2pi.conclusion}

% section conclusion (end)

%\input{qm2pi.dtcodes} 

% section wiring algorithm (end)

\input{qm2pi.ack} 

% section acknowledgments (end)

\newpage


\bibliographystyle{plain}   
\bibliography{../../biblios/main.bib}

\input{qm2pi.rhodetails}

\end{document}

 

% section acknowledgments (end)

\newpage


\bibliographystyle{plain}   
\bibliography{../../biblios/main.bib}

\documentclass[12pt]{llncs}
%\documentclass{jktr}

\usepackage[pdftex]{hyperref}                   
\usepackage {listings}
\usepackage {mathpartir}
\usepackage{bcprules}
%\usepackage{listings}
                       
\usepackage{graphicx} 
%\usepackage[margins=2.5cm,nohead,nofoot]{geometry}
%\usepackage{geometry}
\usepackage{amsfonts}
\usepackage{amstext}
\usepackage{latexsym}
\usepackage{amssymb}
\usepackage{color}


%\include{myPreamble}
\include{qm2pi.local} 

%\ifpdf
%\usepackage[pdftex]{graphicx}
%\else
%\usepackage{graphicx}
%\fi

 % \ifpdf
%  \usepackage{pdfsync}
%  \if


%\title{Brief Article}
%\author{David F. Snyder}
%\author{L.G. Meredith}

%\address{Dept. of Math., Texas State University--San Marcos, San Marcos, TX 78666}
       
\pagestyle{empty}


\begin{document}

\lstset{language=[Objective]Caml,frame=shadowbox}

\input{qm2pi.front}

% section front matter (end)

\input{qm2pi.intro} 
 
% section introduction (end)

% \input{qm2pi.knotations} 

% section notation (end)

\input{qm2pi.process.calculi} 

% section concurrent_process_calculi_and_spatial_logics_ (end)
    
%\input{qm2pi.knots2pi} 

%\input{qm2pi.trefoil} 

%\input{qm2pi.mainthm} 

% subsection basic_interpretation (end)

%\input{qm2pi.rho.presentation} 
\subsection{The syntax and semantics of the notation system}\label{sub:the_syntax_and_semantics_of_the_notation_system} % (fold)

We now summarize a technical presentation of the calculus that
embodies our theory of dynamics. The typical presentation of such a
calculus follows the style of giving generators and relations on
them. The grammar, below, describing term constructors, freely
generates the set of processes, $\Proc$. This set is then quotiented
by a relation known as structural congruence and it is over this set
that the notion of dynamics is expressed. This presentation is
essentially that of \cite{MeredithR05} with the addition of
polyadicity and summation. For readability we have relegated some of
the technical subtleties to an appendix.

\subsubsection{Process grammar}\label{subsub:process_grammar}

\begin{mathpar}
  \inferrule* [lab=synchronization] {} {{M} \bc \pzero \;|\; x?F \;|\; x!C }
  \and
  \inferrule* [lab=abstraction] {} {{F} \bc (x)P}
  \and
  \inferrule* [lab=concretion] {} {{C} \bc \langle Q \rangle}
  \and
  \inferrule* [lab=process] {} {{P,Q} \bc M \;| \;P|Q \;|\; @{x}}
  \and
  \inferrule* [lab=name] {} {{x} \bc \quotep{P}}
\end{mathpar} 

Note that $\vec{x}$ (resp. $\vec{P}$) denotes a vector of names
(resp. processes) of length $|\vec{x}|$ (resp. $|\vec{P}|$). We adopt
the following useful abbreviations.

\begin{mathpar}
   x?(\vec{y}).P := x.(\vec{y})P \and  x\clift{\vec{P}} := x.\clift{\vec{P}}
   \and x!(y) := \lift{x}{\dropn{y}}
   \and \Pi_{i=0}^{n-1}P_i := P_0 | \ldots | P_{n-1}
\end{mathpar}

\subsubsection{Structural congruence}

\paragraph{Free and bound names and alpha-equivalence.} At the
core of structural equivalence is alpha-equivalence which identifies
process that are the same up to a change of variable. Formally, we
recognize the distinction between free and bound names. The free names
of a process, $\freenames{P}$, may be calculated recursively as
follows:

\begin{mathpar}
\freenames{\pzero} := \emptyset
  \and \\
  \freenames{x?(y).P} := \{ x \} \cup (\freenames{P} \setminus \{ y \})
  \and 
  \freenames{x!\langle P \rangle} := \{ x \} \cup \{ P \} 
  \and \\
  \freenames{P|Q} := \freenames{P} \cup \freenames{Q}
  \and \\
  \freenames{@{x}} := \{ x \}
\end{mathpar}

$\pi$
$\quotep{\pi}$

$\freenames{-} : \pi \to \mathcal{P}(\quotep{\pi})$

\begin{eqnarray*}
  \freenames{\pzero} & := & \emptyset \\
  \freenames{x?(y).P} & := & \{ x \} \cup (\freenames{P} \setminus \{ y \}) \\
  \freenames{x!\langle P \rangle} & := & \{ x \} \cup \{ P \} \\
  \freenames{P|Q} & := & \freenames{P} \cup \freenames{Q} \\
  \freenames{\dropn{x}} & := & \{ x \}
\end{eqnarray*}

The bound names of a process, $\boundnames{P}$, are those names occurring in $P$
that are not free. For example, in $x?(y).0$, the name $x$ is free, while $y$ is bound.

\begin{mathpar}
  \inferrule* [lab=monoidal-laws] {} { P|Q \equiv Q|P \and P|0 \equiv P \and P|(Q|R) \equiv (P|Q)|R }
\end{mathpar}

\begin{mathpar}
  \inferrule* [lab=alpha-equivalence] {} { (x)P \equiv (y)P\{y/x\} \and y \not\in \freenames{P} }
\end{mathpar}

\begin{definition}
Then two processes, $P,Q$, are alpha-equivalent if $P = Q\{\vec{y}/\vec{x}\}$ for
some $\vec{x} \in \boundnames{Q},\vec{y} \in \boundnames{P}$, where $Q\{\vec{y}/\vec{x}\}$
denotes the capture-avoiding substitution of $\vec{y}$ for $\vec{x}$ in $Q$.
\end{definition}

\begin{definition}
  The {\em structural congruence} \cite{SangiorgiWalker} , $\equiv$,
  between processes is the least congruence containing
  alpha-equivalence, satisfying the abelian monoid laws
  (associativity, commutativity and $\pzero$ as identity) for parallel
  composition $|$ and for summation $+$.
\end{definition}

\subsection{Name equivalence}

We take name equivalence, written $\nameeq$, to be the smallest
equivalence relation generated by the following rules.

\begin{mathpar}
\inferrule*[lab=Quote-drop]
{ }
{ \quotep{@{x}} \nameeq x }

\inferrule*[lab=Struct-equiv]
{ P \scong Q }
{ \quotep{P} \nameeq \quotep{Q} }
\end{mathpar}

The astute reader will have noticed that the mutual recursion of names
and processes imposes a mutual recursion on alpha-equivalence and
structural equivalence via name-equivalence. Fortunately, all of this
works out pleasantly and we may calculate in the natural way, free of
concern. The reader interested in the details is referred to the
appendix \ref{appendix:rho_details}.

\subsection{Substitution}

We use $\Proc$ for the set of processes, $\QProc$ for the set of
names, and $\id{\{}\vec{y} / \vec{x} \id{\}}$ to denote partial maps,
$s : \QProc \rightarrow \QProc$. A map, $s$ lifts, uniquely, to a map
on process terms, $\widehat{s} : \Proc \rightarrow \Proc$ by the
following equations.

\begin{mathpar}
  (0) \psubstp{Q}{P} := 0 \\
  (R \juxtap S) \psubstp{Q}{P}
  :=    
  (R)\psubstp{Q}{P} \juxtap (S) \psubstp{Q}{P} \\
  (x?(y).R) \psubstp{Q}{P}    
  :=    
  (x)\substp{Q}{P} (z)\concat( (R \psubstn{z}{y}) \psubstp{Q}{P} ) \\
  (\lift{x}{R}) \psubstp{Q}{P}  
  :=
  \lift{(x)\substp{Q}{P}}{ R \psubstp{Q}{P} } \\
%   (\dropn{x})  \psubstp{Q}{P}       
%   := 
%   \left\{ 
%     \begin{array}{ccc} 
%       \dropn{\quotep{Q}} & & x \nameeq \quotep{P} \\
%       \dropn{x} & & otherwise \\
%     \end{array}
%   \right. 
  (\dropn{x})  \psubstp{Q}{P}       
  := 
  \left\{ 
    \begin{array}{ccc} 
      Q & & x \nameeq \quotep{P} \\
      \dropn{x} & & otherwise \\
    \end{array}
  \right.
\end{mathpar}
 

where

\begin{eqnarray}
  (x)\id{\{} \lpquote Q \rpquote / \lpquote P \rpquote \id{\}}            = 
  \left\{ 
    \begin{array}{ccc}
      \lpquote Q \rpquote & & x \nameeq \lpquote P \rpquote \\
      x & & otherwise \\
    \end{array}
  \right. \nonumber
\end{eqnarray}

and $z$ is chosen distinct from $\quotep{P}$, $\quotep{Q}$, the free
names in $Q$, and all the names in $R$. Our $\alpha$-equivalence will
be built in the standard way from this substitution.

\begin{remark}\label{rem:no_self_referential_names}
  One consequence of these definitions is that $\forall P. \quotep{P}
  \not\in \freenames{P}$.
\end{remark}

\subsection{ Dynamic quote: an example }

Anticipating something of what's to come, consider applying the
substitution, $\widehat{\id{\{}u / z \id{\}}}$, to the following pair
of processes, $\lift{w}{y!(z)}$ and $w[ \lpquote y!(z) \rpquote ]$.

\begin{eqnarray}
	\lift{w}{y!(z)}\widehat{\id{\{}u / z \id{\}}}
		& = &
		\lift{w}{y!(u)} \nonumber\\
	w[ \lpquote y!(z) \rpquote ] \widehat{ \id{\{}u / z \id{\}} }
		& = &
		w[ \lpquote y!(z) \rpquote ] \nonumber
\end{eqnarray}

Because the body of the process between quotes is impervious to
substitution, we get radically different answers. In fact, by
examining the first process in an input context,
e.g. $x?(z).\lift{w}{y!(z)}$, we see that the process under the lift
operator may be shaped by prefixed inputs binding a name inside it. In
this sense, the lift operator will be seen as a way to dynamically
construct processes before reifying them as names.

Finally equipped with these standard features we can present the
dynamics of the calculus.

\subsubsection{Operational semantics} 

Finally, we introduce the computational dynamics. What marks these
algebras as distinct from other more traditionally studied algebraic
structures, e.g. vector spaces or polynomial rings, is the manner in
which dynamics is captured. In traditional structures, dynamics is typically
expressed through morphisms between such structures, as in linear maps
between vector spaces or morphisms between rings. In algebras
associated with the semantics of computation, the dynamics is
expressed as part of the algebraic structure itself, through a
reduction reduction relation typically denoted by $\red$. Below, we
give a recursive presentation of this relation for the calculus used
in the encoding.

$\red \subseteq \pi \times \pi$
$\red : \pi \to \mathcal{P}(\pi)$

\begin{mathpar}
  \inferrule* [lab=Comm] { \textsf{match}( x_{src}, x_{trgt} ) } { x_{trgt}?(y)P \; | \; x_{src}!\langle {Q} \rangle \red P\{\quotep{Q}/y}\} }
  \and \\
  \inferrule* [lab=Par] {{P} \red {P}'} {{{P} | {Q}} \red {{P}' | {Q}}}
  \and
  \inferrule* [lab=Equiv]{{{P} \scong {P}'} \andalso {{P}' \red {Q}'} \andalso {{Q}' \scong {Q}}}{{P} \red {Q}}
\end{mathpar}

\begin{eqnarray*}
  match_{\equiv} (\quotep{P},\quotep{Q}) & := & P \equiv Q \\
  match_{\dagger}(\quotep{P},\quotep{Q}) & := & \forall R. P|Q \red^{*} R => R \red^{*} 0 \\
  match_{K}(\quotep{P},\quotep{Q}) & := & K \mbox{ for some context } K
\end{eqnarray*}

$u?(x)P | u!\langle Q \rangle \red P\{\quotep{Q}/x\}$

%We write $\wred$ for $\red^*$, and $P\red$ if $\exists Q $ such that $ P \red Q$.
We write $P\red$ if $\exists Q $ such that $ P \red Q$ and $P\not\red$, otherwise.

\section{Replication}

As mentioned before, it is known that replication (and hence
recursion) can be implemented in a higher-order process algebra
\cite{SangiorgiWalker}. As our first example of calculation with the
machinery thus far presented we give the construction explicitly in
the {\rhoc}.

\begin{eqnarray}
	D_{x} & := & \prefix{x}{y}{(\binpar{\outputp{x}{y}}{@{y}})} \nonumber\\
	\bangp_{x}{P} & := & \binpar{{x}!\langle{\binpar{D_{x}}{P}}\rangle}{D_{x}} \nonumber
\end{eqnarray}

\begin{eqnarray}
	\bangp_{x}{P} & & \nonumber\\
	=
	& {x}!\langle{(\prefix{x}{y}{(\outputp{x}{y} | @{y})) | P}}\rangle 
	      | \prefix{x}{y}{(\outputp{x}{y} | @{y})} & \nonumber\\
	\red
	& (\outputp{x}{y} | @{y})\substn{\quotep{(\prefix{x}{y}{(@{y} | \outputp{x}{y})) | P}}}{y} & \nonumber\\
	=
	& \outputp{x}{\quotep{(\prefix{x}{y}{(\outputp{x}{y} | @{y})) | P}}}
	  | {(\prefix{x}{y}{(\outputp{x}{y} | @{y})) | P}} & \nonumber\\
	\red
	& \ldots & \nonumber\\
	\red^*
	& P | P | \ldots & \nonumber
\end{eqnarray}

Of course, this encoding, as an implementation, runs away, unfolding
$\bangp{P}$ eagerly. A lazier and more implementable replication
operator, restricted to input-guarded processes, may be obtained as follows.

\begin{eqnarray}
\bangp{\prefix{u}{v}{P}} 
	:= 
	\binpar{\lift{x}{\prefix{u}{v}{(\binpar{D(x)}{P})}}}{D(x)} \nonumber
\end{eqnarray}

\begin{remark}
  Note that the lazier definition still does not deal with summation
  or mixed summation (i.e. sums over input and output). The reader is
  invited to construct definitions of replication that deal with these
  features. 

  Further, the definitions are parameterized in a name, $x$. Can you,
  gentle reader, make a definition that eliminates this parameter and
  guarantees no accidental interaction between the replication
  machinery and the process being replicated -- i.e. no accidental
  sharing of names used by the process to get its work done and the
  name(s) used by the replication to effect copying. This latter
  revision of the definition of replication is crucial to obtaining
  the expected identity $!!P \sim !P$.
\end{remark}

\begin{remark}\label{rem:paradoxical_combinator}
  The reader familiar with the lambda calculus will have noticed the
  similarity between $D$ and the paradoxical combinator.

  [Ed. note: the existence of this seems to suggest we have to be more
  restrictive on the set of processes and names we admit if we are to
  support no-cloning.]
\end{remark}

\subsubsection{Bisimulation}

The computational dynamics gives rise to another kind of equivalence,
the equivalence of computational behavior. As previously mentioned
this is typically captured \emph{via} some form of bisimulation.

% The notion we use in this paper is weak barbed bisimulation
% \cite{milner91polyadicpi}.

The notion we use in this paper is derived from weak barbed
bisimulation \cite{milner91polyadicpi}. 

\begin{definition}
An \emph{observation relation}, $\downarrow_{\mathcal N}$, over a set
of names, $\mathcal N$, is the smallest relation satisfying the rules
below.

\infrule[Out-barb]{y \in {\mathcal N}, \; x \nameeq y}
		  {\outputp{x}{v} \downarrow_{\mathcal N} x}
\infrule[Par-barb]{\mbox{$P\downarrow_{\mathcal N} x$ or $Q\downarrow_{\mathcal N} x$}}
		  {\binpar{P}{Q} \downarrow_{\mathcal N} x}

We write $P \Downarrow_{\mathcal N} x$ if there is $Q$ such that 
$P \wred Q$ and $Q \downarrow_{\mathcal N} x$.
\end{definition}

\begin{definition}
%\label{def.bbisim}
An  ${\mathcal N}$-\emph{barbed bisimulation} over a set of names, ${\mathcal N}$, is a symmetric binary relation 
${\mathcal S}_{\mathcal N}$ between agents such that $P\rel{S}_{\mathcal N}Q$ implies:
\begin{enumerate}
\item If $P \red P'$ then $Q \wred Q'$ and $P'\rel{S}_{\mathcal N} Q'$.
\item If $P\downarrow_{\mathcal N} x$, then $Q\Downarrow_{\mathcal N} x$.
\end{enumerate}
$P$ is ${\mathcal N}$-barbed bisimilar to $Q$, written
$P \wbbisim_{\mathcal N} Q$, if $P \rel{S}_{\mathcal N} Q$ for some ${\mathcal N}$-barbed bisimulation ${\mathcal S}_{\mathcal N}$.
\end{definition}

$\mathcal{R} \subseteq \pi \times \pi$

$P \mathcal{R} Q => \forall P'. P \red P' \Rightarrow \exists Q'. Q \red Q', P' \mathcal{R} Q'$

$P \vdash x \Rightarrow Q \vdash x$

\begin{mathpar}
  \inferrule*[lab=Out-barb]{x \nameeq y}{{y}!\langle{Q}\rangle \vdash x}
  \and
  \inferrule*[lab=Par-barb]{\mbox{$P\vdash x$ or $Q\vdash x$}}{\binpar{P}{Q} \vdash x}
\end{mathpar}

\subsubsection{Contexts}

One of the principle advantages of computational calculi like the
$\pi$-calculus is a well-defined notion of context,
contextual-equivalence and a correlation between
contextual-equivalence and notions of bisimulation. The notion of
context allows the decomposition of a process into (sub-)process and
its syntactic environment, its context. Thus, a context may be
thought of as a process with a ``hole'' (written $\Box$) in it. The
application of a context $M$ to a process $P$, written $M[P]$, is
tantamount to filling the hole in $M$ with $P$. In this paper we do
not need the full weight of this theory, but do make use of the notion
of context in the proof the main theorem. 

\begin{mathpar}
  \inferrule* [lab=summation] {} {{M_{M},M_{N}} \bc \Box \;|\; x.M_{A} \;|\; M_{M}+M_{N}}
  \and
  \inferrule* [lab=agent] {} {{M_{A}} \bc (\vec{x})M_{P} \;| \; \clift{P_0,\ldots,M_{P},\ldots,P_N}}
  \and \\
  \inferrule* [lab=process] {} {{M_{P}} \bc M_{N} \;| \;P|M_{P} }
\end{mathpar} 

\begin{mathpar}
  \inferrule* [lab=sychronization] {} {M_{N} \bc \Box \;|\; x?M_{F} \;|\; x!M_{C}}
  \and
  \inferrule* [lab=abstraction] {} {{M_{F}} \bc (x)M_{P} }
  \and
  \inferrule* [lab=concretion] {} {{M_{C}} \bc \langle M_{P} \rangle }
  \and \\
  \inferrule* [lab=process] {} {{M_{P}} \bc M_{N} \;| \;P|M_{P} }
\end{mathpar}

\begin{definition}[contextual application] Given a context $M$, and
  process $P$, we define the \emph{contextual application}, $M[P] :=
  M\{P/\Box\}$. That is, the contextual application of M to P is the
  substitution of $P$ for $\Box$ in $M$.
\end{definition}

$\meaningof{-} : L \to \mathcal{P}(\pi)$

\begin{mathpar}
  \inferrule* [lab=collection] {} {\meaningof{true} = \pi, \and \meaningof{~E} = \pi \setminus \meaningof{E}, \and \meaningof{E_{1} \& E_{2}} = \meaningof{E_{1}} \cap \meaningof{E_{2}}}
\end{mathpar}

\begin{mathpar}
  \inferrule* [lab=structure] {} {\meaningof{0} = \{ P \in \pi | P \equiv 0 \}, \and \\ \meaningof{E_1 | E_2} = \{ P \in \pi | P \equiv P_{1} | P_{2}, P_{1} \in \meaningof{E_{1}}, P_{2} \in \meaningof{E_2}\} }
\end{mathpar}

\begin{mathpar}
 \inferrule* [lab=behavior] {} {\meaningof{\langle a?b \rangle E} = \{ P \in \pi | P \equiv Q | u?(y)P', \\ \and \\\\ \and \\ \;\;\; u \in \meaningof{a}, \forall z.P'\{z/y\} \in \meaningof{E\{z/b\}}\}, \and \\ \meaningof{a!E} = \{ P \in \pi | P \equiv Q | x!\langle P' \rangle, x \in \meaningof{a} P' \in \meaningof{E}\} }
\end{mathpar}

\begin{mathpar}
 \inferrule* [lab=nominal] {} {\meaningof{\quotep{E}} = \{ \quotep{P} \in \quotep{\pi} | P \in \meaningof{E} \}, \and \meaningof{\quotep{P}} = \{ \quotep{Q} \in \quotep{\pi} | P \equiv Q \} \and \\ \meaningof{@\quotep{E}} = \{ P \in \pi | P \equiv @x, x \in \meaningof{E} \}}
\end{mathpar}

\begin{eqnarray*}
  \\
  \meaningof{-} : TS \to ST
\end{eqnarray*}

\begin{eqnarray*}
  \\
  L : TS \to ST
\end{eqnarray*}

\begin{eqnarray*}
  \\
  P \models E \iff P \in \meaningof{E}
\end{eqnarray*}

\begin{eqnarray*}
  P \approx_{L} Q \iff \forall E \in L. P \models E \iff Q \models E
\end{eqnarray*}

\begin{eqnarray*}
  P \approx_{K} Q
\end{eqnarray*}

\begin{eqnarray*}
  P \approx Q
\end{eqnarray*}

$\approx_{K} = \approx = \approx_{L}$

\subsubsection{Contextual duality}

Note that contexts extend the quotation operation to a family of
operations from processes to names. Given a context, $M$, we can
define a \emph{nominal context}, $\quotep{M}$ by $\quotep{M}[P] :=
\quotep{M[P]}$. To foreshadow what is to come we observe that these
operations enjoy a duality with processes very much like the duality
between vectors and maps from vectors to scalars.

Further, because the calculus is essentially higher-order, we have a
correspondence between contexts and processes. More specifically,
given a name $x$ and a context $M$ we can construct $M^{*}_{x}$ such
that 

\begin{mathpar}
  M^{*}_{x} | \lift{x}{P} \red M[P]
\end{mathpar}

namely,

\begin{mathpar}
  M^{*}_{x} := x?(u).M[\dropn{u}]
\end{mathpar}

The dependence of $M^{*}_{x}$ on a name makes it an abstraction, 

\begin{mathpar}
  M^{*} := (x)x?(u).M[\dropn{u}]
\end{mathpar}

\subsection{Additional notation}

It will sometimes be convenient to denote the process a name
quotes. We already have the notation $x = \quotep{P}$, but it will be
convenient to introduce an alternate notation, $\procn{x}$, when we
want to emphasize the connection to the use of the name. Note that, by
virtue of name equivalence, $\quotep{\procn{x}} \nameeq x$; so, the
notation is consistent with previous definitions.

Further, because names have structure it is possible to effect
substitutions on the basis of that structure. This means we need to
upgrade our notation for substitutions, which we accomplish by
adapting comprehension notation. Thus,

\begin{mathpar}
  P\{ y / x : x \in S \}
\end{mathpar}

is interpreted to mean the process derived from P by replacing (in a
capture-avoiding manner) each occurrence of $x$ in $S$ by $y$. For example,

\begin{mathpar}
  P\{ \quotep{\procn{x}|\procn{x}} / x : x \in \freenames{P} \}
\end{mathpar}

will replace each (occurrence) of a free name $x$ in $P$ by
$\quotep{\procn{x}|\procn{x}}$.

Also, we will avail ourselves of the notation $x^{L}$ and $x^{R}$ to
denote injections of a name into disjoint copies of the name
space. There are numerous ways to accomplish this. One example can be
found in \cite{MeredithR05}. This notation overloads to vectors of
names: $\vec{x}^{\pi} := (x_{i}^{\pi} \; : \; 0 \leq i < |\vec{x}| )$ where $\pi \in \{L,R\}$.

We also use $P^{\Box} := P|\Box$.

In \cite{MeredithR05} an interpretation of the new operator is
given. It turns out that there are several possible interpretations
all enjoying the requisite algebraic properties of the operator (see
\cite{milner91polyadicpi}). We will therefore make liberal use of
$(\nu\; \vec{x})P$.

% subsection the_syntax_and_semantics_of_the_notation_system (end)   

\input{qm2pi.qmops} 

\input{qm2pi.sterngerlach} 

\input{qm2pi.metric} 

% section concurrent_process_calculi (end)

%\input{qm2pi.proofsketch}

% section proof sketch (end)

%\input{qm2pi.slviaknots} 

% section spatial logic via knots (end)

\input{qm2pi.conclusion}

% section conclusion (end)

%\input{qm2pi.dtcodes} 

% section wiring algorithm (end)

\input{qm2pi.ack} 

% section acknowledgments (end)

\newpage


\bibliographystyle{plain}   
\bibliography{../../biblios/main.bib}

\input{qm2pi.rhodetails}

\end{document}



\end{document}

 

% section acknowledgments (end)

\newpage


\bibliographystyle{plain}   
\bibliography{../../biblios/main.bib}

\documentclass[12pt]{llncs}
%\documentclass{jktr}

\usepackage[pdftex]{hyperref}                   
\usepackage {listings}
\usepackage {mathpartir}
\usepackage{bcprules}
%\usepackage{listings}
                       
\usepackage{graphicx} 
%\usepackage[margins=2.5cm,nohead,nofoot]{geometry}
%\usepackage{geometry}
\usepackage{amsfonts}
\usepackage{amstext}
\usepackage{latexsym}
\usepackage{amssymb}
\usepackage{color}


%\include{myPreamble}
\documentclass[12pt]{llncs}
%\documentclass{jktr}

\usepackage[pdftex]{hyperref}                   
\usepackage {listings}
\usepackage {mathpartir}
\usepackage{bcprules}
%\usepackage{listings}
                       
\usepackage{graphicx} 
%\usepackage[margins=2.5cm,nohead,nofoot]{geometry}
%\usepackage{geometry}
\usepackage{amsfonts}
\usepackage{amstext}
\usepackage{latexsym}
\usepackage{amssymb}
\usepackage{color}


%\include{myPreamble}
\include{qm2pi.local} 

%\ifpdf
%\usepackage[pdftex]{graphicx}
%\else
%\usepackage{graphicx}
%\fi

 % \ifpdf
%  \usepackage{pdfsync}
%  \if


%\title{Brief Article}
%\author{David F. Snyder}
%\author{L.G. Meredith}

%\address{Dept. of Math., Texas State University--San Marcos, San Marcos, TX 78666}
       
\pagestyle{empty}


\begin{document}

\lstset{language=[Objective]Caml,frame=shadowbox}

\input{qm2pi.front}

% section front matter (end)

\input{qm2pi.intro} 
 
% section introduction (end)

% \input{qm2pi.knotations} 

% section notation (end)

\input{qm2pi.process.calculi} 

% section concurrent_process_calculi_and_spatial_logics_ (end)
    
%\input{qm2pi.knots2pi} 

%\input{qm2pi.trefoil} 

%\input{qm2pi.mainthm} 

% subsection basic_interpretation (end)

%\input{qm2pi.rho.presentation} 
\subsection{The syntax and semantics of the notation system}\label{sub:the_syntax_and_semantics_of_the_notation_system} % (fold)

We now summarize a technical presentation of the calculus that
embodies our theory of dynamics. The typical presentation of such a
calculus follows the style of giving generators and relations on
them. The grammar, below, describing term constructors, freely
generates the set of processes, $\Proc$. This set is then quotiented
by a relation known as structural congruence and it is over this set
that the notion of dynamics is expressed. This presentation is
essentially that of \cite{MeredithR05} with the addition of
polyadicity and summation. For readability we have relegated some of
the technical subtleties to an appendix.

\subsubsection{Process grammar}\label{subsub:process_grammar}

\begin{mathpar}
  \inferrule* [lab=synchronization] {} {{M} \bc \pzero \;|\; x?F \;|\; x!C }
  \and
  \inferrule* [lab=abstraction] {} {{F} \bc (x)P}
  \and
  \inferrule* [lab=concretion] {} {{C} \bc \langle Q \rangle}
  \and
  \inferrule* [lab=process] {} {{P,Q} \bc M \;| \;P|Q \;|\; @{x}}
  \and
  \inferrule* [lab=name] {} {{x} \bc \quotep{P}}
\end{mathpar} 

Note that $\vec{x}$ (resp. $\vec{P}$) denotes a vector of names
(resp. processes) of length $|\vec{x}|$ (resp. $|\vec{P}|$). We adopt
the following useful abbreviations.

\begin{mathpar}
   x?(\vec{y}).P := x.(\vec{y})P \and  x\clift{\vec{P}} := x.\clift{\vec{P}}
   \and x!(y) := \lift{x}{\dropn{y}}
   \and \Pi_{i=0}^{n-1}P_i := P_0 | \ldots | P_{n-1}
\end{mathpar}

\subsubsection{Structural congruence}

\paragraph{Free and bound names and alpha-equivalence.} At the
core of structural equivalence is alpha-equivalence which identifies
process that are the same up to a change of variable. Formally, we
recognize the distinction between free and bound names. The free names
of a process, $\freenames{P}$, may be calculated recursively as
follows:

\begin{mathpar}
\freenames{\pzero} := \emptyset
  \and \\
  \freenames{x?(y).P} := \{ x \} \cup (\freenames{P} \setminus \{ y \})
  \and 
  \freenames{x!\langle P \rangle} := \{ x \} \cup \{ P \} 
  \and \\
  \freenames{P|Q} := \freenames{P} \cup \freenames{Q}
  \and \\
  \freenames{@{x}} := \{ x \}
\end{mathpar}

$\pi$
$\quotep{\pi}$

$\freenames{-} : \pi \to \mathcal{P}(\quotep{\pi})$

\begin{eqnarray*}
  \freenames{\pzero} & := & \emptyset \\
  \freenames{x?(y).P} & := & \{ x \} \cup (\freenames{P} \setminus \{ y \}) \\
  \freenames{x!\langle P \rangle} & := & \{ x \} \cup \{ P \} \\
  \freenames{P|Q} & := & \freenames{P} \cup \freenames{Q} \\
  \freenames{\dropn{x}} & := & \{ x \}
\end{eqnarray*}

The bound names of a process, $\boundnames{P}$, are those names occurring in $P$
that are not free. For example, in $x?(y).0$, the name $x$ is free, while $y$ is bound.

\begin{mathpar}
  \inferrule* [lab=monoidal-laws] {} { P|Q \equiv Q|P \and P|0 \equiv P \and P|(Q|R) \equiv (P|Q)|R }
\end{mathpar}

\begin{mathpar}
  \inferrule* [lab=alpha-equivalence] {} { (x)P \equiv (y)P\{y/x\} \and y \not\in \freenames{P} }
\end{mathpar}

\begin{definition}
Then two processes, $P,Q$, are alpha-equivalent if $P = Q\{\vec{y}/\vec{x}\}$ for
some $\vec{x} \in \boundnames{Q},\vec{y} \in \boundnames{P}$, where $Q\{\vec{y}/\vec{x}\}$
denotes the capture-avoiding substitution of $\vec{y}$ for $\vec{x}$ in $Q$.
\end{definition}

\begin{definition}
  The {\em structural congruence} \cite{SangiorgiWalker} , $\equiv$,
  between processes is the least congruence containing
  alpha-equivalence, satisfying the abelian monoid laws
  (associativity, commutativity and $\pzero$ as identity) for parallel
  composition $|$ and for summation $+$.
\end{definition}

\subsection{Name equivalence}

We take name equivalence, written $\nameeq$, to be the smallest
equivalence relation generated by the following rules.

\begin{mathpar}
\inferrule*[lab=Quote-drop]
{ }
{ \quotep{@{x}} \nameeq x }

\inferrule*[lab=Struct-equiv]
{ P \scong Q }
{ \quotep{P} \nameeq \quotep{Q} }
\end{mathpar}

The astute reader will have noticed that the mutual recursion of names
and processes imposes a mutual recursion on alpha-equivalence and
structural equivalence via name-equivalence. Fortunately, all of this
works out pleasantly and we may calculate in the natural way, free of
concern. The reader interested in the details is referred to the
appendix \ref{appendix:rho_details}.

\subsection{Substitution}

We use $\Proc$ for the set of processes, $\QProc$ for the set of
names, and $\id{\{}\vec{y} / \vec{x} \id{\}}$ to denote partial maps,
$s : \QProc \rightarrow \QProc$. A map, $s$ lifts, uniquely, to a map
on process terms, $\widehat{s} : \Proc \rightarrow \Proc$ by the
following equations.

\begin{mathpar}
  (0) \psubstp{Q}{P} := 0 \\
  (R \juxtap S) \psubstp{Q}{P}
  :=    
  (R)\psubstp{Q}{P} \juxtap (S) \psubstp{Q}{P} \\
  (x?(y).R) \psubstp{Q}{P}    
  :=    
  (x)\substp{Q}{P} (z)\concat( (R \psubstn{z}{y}) \psubstp{Q}{P} ) \\
  (\lift{x}{R}) \psubstp{Q}{P}  
  :=
  \lift{(x)\substp{Q}{P}}{ R \psubstp{Q}{P} } \\
%   (\dropn{x})  \psubstp{Q}{P}       
%   := 
%   \left\{ 
%     \begin{array}{ccc} 
%       \dropn{\quotep{Q}} & & x \nameeq \quotep{P} \\
%       \dropn{x} & & otherwise \\
%     \end{array}
%   \right. 
  (\dropn{x})  \psubstp{Q}{P}       
  := 
  \left\{ 
    \begin{array}{ccc} 
      Q & & x \nameeq \quotep{P} \\
      \dropn{x} & & otherwise \\
    \end{array}
  \right.
\end{mathpar}
 

where

\begin{eqnarray}
  (x)\id{\{} \lpquote Q \rpquote / \lpquote P \rpquote \id{\}}            = 
  \left\{ 
    \begin{array}{ccc}
      \lpquote Q \rpquote & & x \nameeq \lpquote P \rpquote \\
      x & & otherwise \\
    \end{array}
  \right. \nonumber
\end{eqnarray}

and $z$ is chosen distinct from $\quotep{P}$, $\quotep{Q}$, the free
names in $Q$, and all the names in $R$. Our $\alpha$-equivalence will
be built in the standard way from this substitution.

\begin{remark}\label{rem:no_self_referential_names}
  One consequence of these definitions is that $\forall P. \quotep{P}
  \not\in \freenames{P}$.
\end{remark}

\subsection{ Dynamic quote: an example }

Anticipating something of what's to come, consider applying the
substitution, $\widehat{\id{\{}u / z \id{\}}}$, to the following pair
of processes, $\lift{w}{y!(z)}$ and $w[ \lpquote y!(z) \rpquote ]$.

\begin{eqnarray}
	\lift{w}{y!(z)}\widehat{\id{\{}u / z \id{\}}}
		& = &
		\lift{w}{y!(u)} \nonumber\\
	w[ \lpquote y!(z) \rpquote ] \widehat{ \id{\{}u / z \id{\}} }
		& = &
		w[ \lpquote y!(z) \rpquote ] \nonumber
\end{eqnarray}

Because the body of the process between quotes is impervious to
substitution, we get radically different answers. In fact, by
examining the first process in an input context,
e.g. $x?(z).\lift{w}{y!(z)}$, we see that the process under the lift
operator may be shaped by prefixed inputs binding a name inside it. In
this sense, the lift operator will be seen as a way to dynamically
construct processes before reifying them as names.

Finally equipped with these standard features we can present the
dynamics of the calculus.

\subsubsection{Operational semantics} 

Finally, we introduce the computational dynamics. What marks these
algebras as distinct from other more traditionally studied algebraic
structures, e.g. vector spaces or polynomial rings, is the manner in
which dynamics is captured. In traditional structures, dynamics is typically
expressed through morphisms between such structures, as in linear maps
between vector spaces or morphisms between rings. In algebras
associated with the semantics of computation, the dynamics is
expressed as part of the algebraic structure itself, through a
reduction reduction relation typically denoted by $\red$. Below, we
give a recursive presentation of this relation for the calculus used
in the encoding.

$\red \subseteq \pi \times \pi$
$\red : \pi \to \mathcal{P}(\pi)$

\begin{mathpar}
  \inferrule* [lab=Comm] { \textsf{match}( x_{src}, x_{trgt} ) } { x_{trgt}?(y)P \; | \; x_{src}!\langle {Q} \rangle \red P\{\quotep{Q}/y}\} }
  \and \\
  \inferrule* [lab=Par] {{P} \red {P}'} {{{P} | {Q}} \red {{P}' | {Q}}}
  \and
  \inferrule* [lab=Equiv]{{{P} \scong {P}'} \andalso {{P}' \red {Q}'} \andalso {{Q}' \scong {Q}}}{{P} \red {Q}}
\end{mathpar}

\begin{eqnarray*}
  match_{\equiv} (\quotep{P},\quotep{Q}) & := & P \equiv Q \\
  match_{\dagger}(\quotep{P},\quotep{Q}) & := & \forall R. P|Q \red^{*} R => R \red^{*} 0 \\
  match_{K}(\quotep{P},\quotep{Q}) & := & K \mbox{ for some context } K
\end{eqnarray*}

$u?(x)P | u!\langle Q \rangle \red P\{\quotep{Q}/x\}$

%We write $\wred$ for $\red^*$, and $P\red$ if $\exists Q $ such that $ P \red Q$.
We write $P\red$ if $\exists Q $ such that $ P \red Q$ and $P\not\red$, otherwise.

\section{Replication}

As mentioned before, it is known that replication (and hence
recursion) can be implemented in a higher-order process algebra
\cite{SangiorgiWalker}. As our first example of calculation with the
machinery thus far presented we give the construction explicitly in
the {\rhoc}.

\begin{eqnarray}
	D_{x} & := & \prefix{x}{y}{(\binpar{\outputp{x}{y}}{@{y}})} \nonumber\\
	\bangp_{x}{P} & := & \binpar{{x}!\langle{\binpar{D_{x}}{P}}\rangle}{D_{x}} \nonumber
\end{eqnarray}

\begin{eqnarray}
	\bangp_{x}{P} & & \nonumber\\
	=
	& {x}!\langle{(\prefix{x}{y}{(\outputp{x}{y} | @{y})) | P}}\rangle 
	      | \prefix{x}{y}{(\outputp{x}{y} | @{y})} & \nonumber\\
	\red
	& (\outputp{x}{y} | @{y})\substn{\quotep{(\prefix{x}{y}{(@{y} | \outputp{x}{y})) | P}}}{y} & \nonumber\\
	=
	& \outputp{x}{\quotep{(\prefix{x}{y}{(\outputp{x}{y} | @{y})) | P}}}
	  | {(\prefix{x}{y}{(\outputp{x}{y} | @{y})) | P}} & \nonumber\\
	\red
	& \ldots & \nonumber\\
	\red^*
	& P | P | \ldots & \nonumber
\end{eqnarray}

Of course, this encoding, as an implementation, runs away, unfolding
$\bangp{P}$ eagerly. A lazier and more implementable replication
operator, restricted to input-guarded processes, may be obtained as follows.

\begin{eqnarray}
\bangp{\prefix{u}{v}{P}} 
	:= 
	\binpar{\lift{x}{\prefix{u}{v}{(\binpar{D(x)}{P})}}}{D(x)} \nonumber
\end{eqnarray}

\begin{remark}
  Note that the lazier definition still does not deal with summation
  or mixed summation (i.e. sums over input and output). The reader is
  invited to construct definitions of replication that deal with these
  features. 

  Further, the definitions are parameterized in a name, $x$. Can you,
  gentle reader, make a definition that eliminates this parameter and
  guarantees no accidental interaction between the replication
  machinery and the process being replicated -- i.e. no accidental
  sharing of names used by the process to get its work done and the
  name(s) used by the replication to effect copying. This latter
  revision of the definition of replication is crucial to obtaining
  the expected identity $!!P \sim !P$.
\end{remark}

\begin{remark}\label{rem:paradoxical_combinator}
  The reader familiar with the lambda calculus will have noticed the
  similarity between $D$ and the paradoxical combinator.

  [Ed. note: the existence of this seems to suggest we have to be more
  restrictive on the set of processes and names we admit if we are to
  support no-cloning.]
\end{remark}

\subsubsection{Bisimulation}

The computational dynamics gives rise to another kind of equivalence,
the equivalence of computational behavior. As previously mentioned
this is typically captured \emph{via} some form of bisimulation.

% The notion we use in this paper is weak barbed bisimulation
% \cite{milner91polyadicpi}.

The notion we use in this paper is derived from weak barbed
bisimulation \cite{milner91polyadicpi}. 

\begin{definition}
An \emph{observation relation}, $\downarrow_{\mathcal N}$, over a set
of names, $\mathcal N$, is the smallest relation satisfying the rules
below.

\infrule[Out-barb]{y \in {\mathcal N}, \; x \nameeq y}
		  {\outputp{x}{v} \downarrow_{\mathcal N} x}
\infrule[Par-barb]{\mbox{$P\downarrow_{\mathcal N} x$ or $Q\downarrow_{\mathcal N} x$}}
		  {\binpar{P}{Q} \downarrow_{\mathcal N} x}

We write $P \Downarrow_{\mathcal N} x$ if there is $Q$ such that 
$P \wred Q$ and $Q \downarrow_{\mathcal N} x$.
\end{definition}

\begin{definition}
%\label{def.bbisim}
An  ${\mathcal N}$-\emph{barbed bisimulation} over a set of names, ${\mathcal N}$, is a symmetric binary relation 
${\mathcal S}_{\mathcal N}$ between agents such that $P\rel{S}_{\mathcal N}Q$ implies:
\begin{enumerate}
\item If $P \red P'$ then $Q \wred Q'$ and $P'\rel{S}_{\mathcal N} Q'$.
\item If $P\downarrow_{\mathcal N} x$, then $Q\Downarrow_{\mathcal N} x$.
\end{enumerate}
$P$ is ${\mathcal N}$-barbed bisimilar to $Q$, written
$P \wbbisim_{\mathcal N} Q$, if $P \rel{S}_{\mathcal N} Q$ for some ${\mathcal N}$-barbed bisimulation ${\mathcal S}_{\mathcal N}$.
\end{definition}

$\mathcal{R} \subseteq \pi \times \pi$

$P \mathcal{R} Q => \forall P'. P \red P' \Rightarrow \exists Q'. Q \red Q', P' \mathcal{R} Q'$

$P \vdash x \Rightarrow Q \vdash x$

\begin{mathpar}
  \inferrule*[lab=Out-barb]{x \nameeq y}{{y}!\langle{Q}\rangle \vdash x}
  \and
  \inferrule*[lab=Par-barb]{\mbox{$P\vdash x$ or $Q\vdash x$}}{\binpar{P}{Q} \vdash x}
\end{mathpar}

\subsubsection{Contexts}

One of the principle advantages of computational calculi like the
$\pi$-calculus is a well-defined notion of context,
contextual-equivalence and a correlation between
contextual-equivalence and notions of bisimulation. The notion of
context allows the decomposition of a process into (sub-)process and
its syntactic environment, its context. Thus, a context may be
thought of as a process with a ``hole'' (written $\Box$) in it. The
application of a context $M$ to a process $P$, written $M[P]$, is
tantamount to filling the hole in $M$ with $P$. In this paper we do
not need the full weight of this theory, but do make use of the notion
of context in the proof the main theorem. 

\begin{mathpar}
  \inferrule* [lab=summation] {} {{M_{M},M_{N}} \bc \Box \;|\; x.M_{A} \;|\; M_{M}+M_{N}}
  \and
  \inferrule* [lab=agent] {} {{M_{A}} \bc (\vec{x})M_{P} \;| \; \clift{P_0,\ldots,M_{P},\ldots,P_N}}
  \and \\
  \inferrule* [lab=process] {} {{M_{P}} \bc M_{N} \;| \;P|M_{P} }
\end{mathpar} 

\begin{mathpar}
  \inferrule* [lab=sychronization] {} {M_{N} \bc \Box \;|\; x?M_{F} \;|\; x!M_{C}}
  \and
  \inferrule* [lab=abstraction] {} {{M_{F}} \bc (x)M_{P} }
  \and
  \inferrule* [lab=concretion] {} {{M_{C}} \bc \langle M_{P} \rangle }
  \and \\
  \inferrule* [lab=process] {} {{M_{P}} \bc M_{N} \;| \;P|M_{P} }
\end{mathpar}

\begin{definition}[contextual application] Given a context $M$, and
  process $P$, we define the \emph{contextual application}, $M[P] :=
  M\{P/\Box\}$. That is, the contextual application of M to P is the
  substitution of $P$ for $\Box$ in $M$.
\end{definition}

$\meaningof{-} : L \to \mathcal{P}(\pi)$

\begin{mathpar}
  \inferrule* [lab=collection] {} {\meaningof{true} = \pi, \and \meaningof{~E} = \pi \setminus \meaningof{E}, \and \meaningof{E_{1} \& E_{2}} = \meaningof{E_{1}} \cap \meaningof{E_{2}}}
\end{mathpar}

\begin{mathpar}
  \inferrule* [lab=structure] {} {\meaningof{0} = \{ P \in \pi | P \equiv 0 \}, \and \\ \meaningof{E_1 | E_2} = \{ P \in \pi | P \equiv P_{1} | P_{2}, P_{1} \in \meaningof{E_{1}}, P_{2} \in \meaningof{E_2}\} }
\end{mathpar}

\begin{mathpar}
 \inferrule* [lab=behavior] {} {\meaningof{\langle a?b \rangle E} = \{ P \in \pi | P \equiv Q | u?(y)P', \\ \and \\\\ \and \\ \;\;\; u \in \meaningof{a}, \forall z.P'\{z/y\} \in \meaningof{E\{z/b\}}\}, \and \\ \meaningof{a!E} = \{ P \in \pi | P \equiv Q | x!\langle P' \rangle, x \in \meaningof{a} P' \in \meaningof{E}\} }
\end{mathpar}

\begin{mathpar}
 \inferrule* [lab=nominal] {} {\meaningof{\quotep{E}} = \{ \quotep{P} \in \quotep{\pi} | P \in \meaningof{E} \}, \and \meaningof{\quotep{P}} = \{ \quotep{Q} \in \quotep{\pi} | P \equiv Q \} \and \\ \meaningof{@\quotep{E}} = \{ P \in \pi | P \equiv @x, x \in \meaningof{E} \}}
\end{mathpar}

\begin{eqnarray*}
  \\
  \meaningof{-} : TS \to ST
\end{eqnarray*}

\begin{eqnarray*}
  \\
  L : TS \to ST
\end{eqnarray*}

\begin{eqnarray*}
  \\
  P \models E \iff P \in \meaningof{E}
\end{eqnarray*}

\begin{eqnarray*}
  P \approx_{L} Q \iff \forall E \in L. P \models E \iff Q \models E
\end{eqnarray*}

\begin{eqnarray*}
  P \approx_{K} Q
\end{eqnarray*}

\begin{eqnarray*}
  P \approx Q
\end{eqnarray*}

$\approx_{K} = \approx = \approx_{L}$

\subsubsection{Contextual duality}

Note that contexts extend the quotation operation to a family of
operations from processes to names. Given a context, $M$, we can
define a \emph{nominal context}, $\quotep{M}$ by $\quotep{M}[P] :=
\quotep{M[P]}$. To foreshadow what is to come we observe that these
operations enjoy a duality with processes very much like the duality
between vectors and maps from vectors to scalars.

Further, because the calculus is essentially higher-order, we have a
correspondence between contexts and processes. More specifically,
given a name $x$ and a context $M$ we can construct $M^{*}_{x}$ such
that 

\begin{mathpar}
  M^{*}_{x} | \lift{x}{P} \red M[P]
\end{mathpar}

namely,

\begin{mathpar}
  M^{*}_{x} := x?(u).M[\dropn{u}]
\end{mathpar}

The dependence of $M^{*}_{x}$ on a name makes it an abstraction, 

\begin{mathpar}
  M^{*} := (x)x?(u).M[\dropn{u}]
\end{mathpar}

\subsection{Additional notation}

It will sometimes be convenient to denote the process a name
quotes. We already have the notation $x = \quotep{P}$, but it will be
convenient to introduce an alternate notation, $\procn{x}$, when we
want to emphasize the connection to the use of the name. Note that, by
virtue of name equivalence, $\quotep{\procn{x}} \nameeq x$; so, the
notation is consistent with previous definitions.

Further, because names have structure it is possible to effect
substitutions on the basis of that structure. This means we need to
upgrade our notation for substitutions, which we accomplish by
adapting comprehension notation. Thus,

\begin{mathpar}
  P\{ y / x : x \in S \}
\end{mathpar}

is interpreted to mean the process derived from P by replacing (in a
capture-avoiding manner) each occurrence of $x$ in $S$ by $y$. For example,

\begin{mathpar}
  P\{ \quotep{\procn{x}|\procn{x}} / x : x \in \freenames{P} \}
\end{mathpar}

will replace each (occurrence) of a free name $x$ in $P$ by
$\quotep{\procn{x}|\procn{x}}$.

Also, we will avail ourselves of the notation $x^{L}$ and $x^{R}$ to
denote injections of a name into disjoint copies of the name
space. There are numerous ways to accomplish this. One example can be
found in \cite{MeredithR05}. This notation overloads to vectors of
names: $\vec{x}^{\pi} := (x_{i}^{\pi} \; : \; 0 \leq i < |\vec{x}| )$ where $\pi \in \{L,R\}$.

We also use $P^{\Box} := P|\Box$.

In \cite{MeredithR05} an interpretation of the new operator is
given. It turns out that there are several possible interpretations
all enjoying the requisite algebraic properties of the operator (see
\cite{milner91polyadicpi}). We will therefore make liberal use of
$(\nu\; \vec{x})P$.

% subsection the_syntax_and_semantics_of_the_notation_system (end)   

\input{qm2pi.qmops} 

\input{qm2pi.sterngerlach} 

\input{qm2pi.metric} 

% section concurrent_process_calculi (end)

%\input{qm2pi.proofsketch}

% section proof sketch (end)

%\input{qm2pi.slviaknots} 

% section spatial logic via knots (end)

\input{qm2pi.conclusion}

% section conclusion (end)

%\input{qm2pi.dtcodes} 

% section wiring algorithm (end)

\input{qm2pi.ack} 

% section acknowledgments (end)

\newpage


\bibliographystyle{plain}   
\bibliography{../../biblios/main.bib}

\input{qm2pi.rhodetails}

\end{document}

 

%\ifpdf
%\usepackage[pdftex]{graphicx}
%\else
%\usepackage{graphicx}
%\fi

 % \ifpdf
%  \usepackage{pdfsync}
%  \if


%\title{Brief Article}
%\author{David F. Snyder}
%\author{L.G. Meredith}

%\address{Dept. of Math., Texas State University--San Marcos, San Marcos, TX 78666}
       
\pagestyle{empty}


\begin{document}

\lstset{language=[Objective]Caml,frame=shadowbox}

\documentclass[12pt]{llncs}
%\documentclass{jktr}

\usepackage[pdftex]{hyperref}                   
\usepackage {listings}
\usepackage {mathpartir}
\usepackage{bcprules}
%\usepackage{listings}
                       
\usepackage{graphicx} 
%\usepackage[margins=2.5cm,nohead,nofoot]{geometry}
%\usepackage{geometry}
\usepackage{amsfonts}
\usepackage{amstext}
\usepackage{latexsym}
\usepackage{amssymb}
\usepackage{color}


%\include{myPreamble}
\include{qm2pi.local} 

%\ifpdf
%\usepackage[pdftex]{graphicx}
%\else
%\usepackage{graphicx}
%\fi

 % \ifpdf
%  \usepackage{pdfsync}
%  \if


%\title{Brief Article}
%\author{David F. Snyder}
%\author{L.G. Meredith}

%\address{Dept. of Math., Texas State University--San Marcos, San Marcos, TX 78666}
       
\pagestyle{empty}


\begin{document}

\lstset{language=[Objective]Caml,frame=shadowbox}

\input{qm2pi.front}

% section front matter (end)

\input{qm2pi.intro} 
 
% section introduction (end)

% \input{qm2pi.knotations} 

% section notation (end)

\input{qm2pi.process.calculi} 

% section concurrent_process_calculi_and_spatial_logics_ (end)
    
%\input{qm2pi.knots2pi} 

%\input{qm2pi.trefoil} 

%\input{qm2pi.mainthm} 

% subsection basic_interpretation (end)

%\input{qm2pi.rho.presentation} 
\subsection{The syntax and semantics of the notation system}\label{sub:the_syntax_and_semantics_of_the_notation_system} % (fold)

We now summarize a technical presentation of the calculus that
embodies our theory of dynamics. The typical presentation of such a
calculus follows the style of giving generators and relations on
them. The grammar, below, describing term constructors, freely
generates the set of processes, $\Proc$. This set is then quotiented
by a relation known as structural congruence and it is over this set
that the notion of dynamics is expressed. This presentation is
essentially that of \cite{MeredithR05} with the addition of
polyadicity and summation. For readability we have relegated some of
the technical subtleties to an appendix.

\subsubsection{Process grammar}\label{subsub:process_grammar}

\begin{mathpar}
  \inferrule* [lab=synchronization] {} {{M} \bc \pzero \;|\; x?F \;|\; x!C }
  \and
  \inferrule* [lab=abstraction] {} {{F} \bc (x)P}
  \and
  \inferrule* [lab=concretion] {} {{C} \bc \langle Q \rangle}
  \and
  \inferrule* [lab=process] {} {{P,Q} \bc M \;| \;P|Q \;|\; @{x}}
  \and
  \inferrule* [lab=name] {} {{x} \bc \quotep{P}}
\end{mathpar} 

Note that $\vec{x}$ (resp. $\vec{P}$) denotes a vector of names
(resp. processes) of length $|\vec{x}|$ (resp. $|\vec{P}|$). We adopt
the following useful abbreviations.

\begin{mathpar}
   x?(\vec{y}).P := x.(\vec{y})P \and  x\clift{\vec{P}} := x.\clift{\vec{P}}
   \and x!(y) := \lift{x}{\dropn{y}}
   \and \Pi_{i=0}^{n-1}P_i := P_0 | \ldots | P_{n-1}
\end{mathpar}

\subsubsection{Structural congruence}

\paragraph{Free and bound names and alpha-equivalence.} At the
core of structural equivalence is alpha-equivalence which identifies
process that are the same up to a change of variable. Formally, we
recognize the distinction between free and bound names. The free names
of a process, $\freenames{P}$, may be calculated recursively as
follows:

\begin{mathpar}
\freenames{\pzero} := \emptyset
  \and \\
  \freenames{x?(y).P} := \{ x \} \cup (\freenames{P} \setminus \{ y \})
  \and 
  \freenames{x!\langle P \rangle} := \{ x \} \cup \{ P \} 
  \and \\
  \freenames{P|Q} := \freenames{P} \cup \freenames{Q}
  \and \\
  \freenames{@{x}} := \{ x \}
\end{mathpar}

$\pi$
$\quotep{\pi}$

$\freenames{-} : \pi \to \mathcal{P}(\quotep{\pi})$

\begin{eqnarray*}
  \freenames{\pzero} & := & \emptyset \\
  \freenames{x?(y).P} & := & \{ x \} \cup (\freenames{P} \setminus \{ y \}) \\
  \freenames{x!\langle P \rangle} & := & \{ x \} \cup \{ P \} \\
  \freenames{P|Q} & := & \freenames{P} \cup \freenames{Q} \\
  \freenames{\dropn{x}} & := & \{ x \}
\end{eqnarray*}

The bound names of a process, $\boundnames{P}$, are those names occurring in $P$
that are not free. For example, in $x?(y).0$, the name $x$ is free, while $y$ is bound.

\begin{mathpar}
  \inferrule* [lab=monoidal-laws] {} { P|Q \equiv Q|P \and P|0 \equiv P \and P|(Q|R) \equiv (P|Q)|R }
\end{mathpar}

\begin{mathpar}
  \inferrule* [lab=alpha-equivalence] {} { (x)P \equiv (y)P\{y/x\} \and y \not\in \freenames{P} }
\end{mathpar}

\begin{definition}
Then two processes, $P,Q$, are alpha-equivalent if $P = Q\{\vec{y}/\vec{x}\}$ for
some $\vec{x} \in \boundnames{Q},\vec{y} \in \boundnames{P}$, where $Q\{\vec{y}/\vec{x}\}$
denotes the capture-avoiding substitution of $\vec{y}$ for $\vec{x}$ in $Q$.
\end{definition}

\begin{definition}
  The {\em structural congruence} \cite{SangiorgiWalker} , $\equiv$,
  between processes is the least congruence containing
  alpha-equivalence, satisfying the abelian monoid laws
  (associativity, commutativity and $\pzero$ as identity) for parallel
  composition $|$ and for summation $+$.
\end{definition}

\subsection{Name equivalence}

We take name equivalence, written $\nameeq$, to be the smallest
equivalence relation generated by the following rules.

\begin{mathpar}
\inferrule*[lab=Quote-drop]
{ }
{ \quotep{@{x}} \nameeq x }

\inferrule*[lab=Struct-equiv]
{ P \scong Q }
{ \quotep{P} \nameeq \quotep{Q} }
\end{mathpar}

The astute reader will have noticed that the mutual recursion of names
and processes imposes a mutual recursion on alpha-equivalence and
structural equivalence via name-equivalence. Fortunately, all of this
works out pleasantly and we may calculate in the natural way, free of
concern. The reader interested in the details is referred to the
appendix \ref{appendix:rho_details}.

\subsection{Substitution}

We use $\Proc$ for the set of processes, $\QProc$ for the set of
names, and $\id{\{}\vec{y} / \vec{x} \id{\}}$ to denote partial maps,
$s : \QProc \rightarrow \QProc$. A map, $s$ lifts, uniquely, to a map
on process terms, $\widehat{s} : \Proc \rightarrow \Proc$ by the
following equations.

\begin{mathpar}
  (0) \psubstp{Q}{P} := 0 \\
  (R \juxtap S) \psubstp{Q}{P}
  :=    
  (R)\psubstp{Q}{P} \juxtap (S) \psubstp{Q}{P} \\
  (x?(y).R) \psubstp{Q}{P}    
  :=    
  (x)\substp{Q}{P} (z)\concat( (R \psubstn{z}{y}) \psubstp{Q}{P} ) \\
  (\lift{x}{R}) \psubstp{Q}{P}  
  :=
  \lift{(x)\substp{Q}{P}}{ R \psubstp{Q}{P} } \\
%   (\dropn{x})  \psubstp{Q}{P}       
%   := 
%   \left\{ 
%     \begin{array}{ccc} 
%       \dropn{\quotep{Q}} & & x \nameeq \quotep{P} \\
%       \dropn{x} & & otherwise \\
%     \end{array}
%   \right. 
  (\dropn{x})  \psubstp{Q}{P}       
  := 
  \left\{ 
    \begin{array}{ccc} 
      Q & & x \nameeq \quotep{P} \\
      \dropn{x} & & otherwise \\
    \end{array}
  \right.
\end{mathpar}
 

where

\begin{eqnarray}
  (x)\id{\{} \lpquote Q \rpquote / \lpquote P \rpquote \id{\}}            = 
  \left\{ 
    \begin{array}{ccc}
      \lpquote Q \rpquote & & x \nameeq \lpquote P \rpquote \\
      x & & otherwise \\
    \end{array}
  \right. \nonumber
\end{eqnarray}

and $z$ is chosen distinct from $\quotep{P}$, $\quotep{Q}$, the free
names in $Q$, and all the names in $R$. Our $\alpha$-equivalence will
be built in the standard way from this substitution.

\begin{remark}\label{rem:no_self_referential_names}
  One consequence of these definitions is that $\forall P. \quotep{P}
  \not\in \freenames{P}$.
\end{remark}

\subsection{ Dynamic quote: an example }

Anticipating something of what's to come, consider applying the
substitution, $\widehat{\id{\{}u / z \id{\}}}$, to the following pair
of processes, $\lift{w}{y!(z)}$ and $w[ \lpquote y!(z) \rpquote ]$.

\begin{eqnarray}
	\lift{w}{y!(z)}\widehat{\id{\{}u / z \id{\}}}
		& = &
		\lift{w}{y!(u)} \nonumber\\
	w[ \lpquote y!(z) \rpquote ] \widehat{ \id{\{}u / z \id{\}} }
		& = &
		w[ \lpquote y!(z) \rpquote ] \nonumber
\end{eqnarray}

Because the body of the process between quotes is impervious to
substitution, we get radically different answers. In fact, by
examining the first process in an input context,
e.g. $x?(z).\lift{w}{y!(z)}$, we see that the process under the lift
operator may be shaped by prefixed inputs binding a name inside it. In
this sense, the lift operator will be seen as a way to dynamically
construct processes before reifying them as names.

Finally equipped with these standard features we can present the
dynamics of the calculus.

\subsubsection{Operational semantics} 

Finally, we introduce the computational dynamics. What marks these
algebras as distinct from other more traditionally studied algebraic
structures, e.g. vector spaces or polynomial rings, is the manner in
which dynamics is captured. In traditional structures, dynamics is typically
expressed through morphisms between such structures, as in linear maps
between vector spaces or morphisms between rings. In algebras
associated with the semantics of computation, the dynamics is
expressed as part of the algebraic structure itself, through a
reduction reduction relation typically denoted by $\red$. Below, we
give a recursive presentation of this relation for the calculus used
in the encoding.

$\red \subseteq \pi \times \pi$
$\red : \pi \to \mathcal{P}(\pi)$

\begin{mathpar}
  \inferrule* [lab=Comm] { \textsf{match}( x_{src}, x_{trgt} ) } { x_{trgt}?(y)P \; | \; x_{src}!\langle {Q} \rangle \red P\{\quotep{Q}/y}\} }
  \and \\
  \inferrule* [lab=Par] {{P} \red {P}'} {{{P} | {Q}} \red {{P}' | {Q}}}
  \and
  \inferrule* [lab=Equiv]{{{P} \scong {P}'} \andalso {{P}' \red {Q}'} \andalso {{Q}' \scong {Q}}}{{P} \red {Q}}
\end{mathpar}

\begin{eqnarray*}
  match_{\equiv} (\quotep{P},\quotep{Q}) & := & P \equiv Q \\
  match_{\dagger}(\quotep{P},\quotep{Q}) & := & \forall R. P|Q \red^{*} R => R \red^{*} 0 \\
  match_{K}(\quotep{P},\quotep{Q}) & := & K \mbox{ for some context } K
\end{eqnarray*}

$u?(x)P | u!\langle Q \rangle \red P\{\quotep{Q}/x\}$

%We write $\wred$ for $\red^*$, and $P\red$ if $\exists Q $ such that $ P \red Q$.
We write $P\red$ if $\exists Q $ such that $ P \red Q$ and $P\not\red$, otherwise.

\section{Replication}

As mentioned before, it is known that replication (and hence
recursion) can be implemented in a higher-order process algebra
\cite{SangiorgiWalker}. As our first example of calculation with the
machinery thus far presented we give the construction explicitly in
the {\rhoc}.

\begin{eqnarray}
	D_{x} & := & \prefix{x}{y}{(\binpar{\outputp{x}{y}}{@{y}})} \nonumber\\
	\bangp_{x}{P} & := & \binpar{{x}!\langle{\binpar{D_{x}}{P}}\rangle}{D_{x}} \nonumber
\end{eqnarray}

\begin{eqnarray}
	\bangp_{x}{P} & & \nonumber\\
	=
	& {x}!\langle{(\prefix{x}{y}{(\outputp{x}{y} | @{y})) | P}}\rangle 
	      | \prefix{x}{y}{(\outputp{x}{y} | @{y})} & \nonumber\\
	\red
	& (\outputp{x}{y} | @{y})\substn{\quotep{(\prefix{x}{y}{(@{y} | \outputp{x}{y})) | P}}}{y} & \nonumber\\
	=
	& \outputp{x}{\quotep{(\prefix{x}{y}{(\outputp{x}{y} | @{y})) | P}}}
	  | {(\prefix{x}{y}{(\outputp{x}{y} | @{y})) | P}} & \nonumber\\
	\red
	& \ldots & \nonumber\\
	\red^*
	& P | P | \ldots & \nonumber
\end{eqnarray}

Of course, this encoding, as an implementation, runs away, unfolding
$\bangp{P}$ eagerly. A lazier and more implementable replication
operator, restricted to input-guarded processes, may be obtained as follows.

\begin{eqnarray}
\bangp{\prefix{u}{v}{P}} 
	:= 
	\binpar{\lift{x}{\prefix{u}{v}{(\binpar{D(x)}{P})}}}{D(x)} \nonumber
\end{eqnarray}

\begin{remark}
  Note that the lazier definition still does not deal with summation
  or mixed summation (i.e. sums over input and output). The reader is
  invited to construct definitions of replication that deal with these
  features. 

  Further, the definitions are parameterized in a name, $x$. Can you,
  gentle reader, make a definition that eliminates this parameter and
  guarantees no accidental interaction between the replication
  machinery and the process being replicated -- i.e. no accidental
  sharing of names used by the process to get its work done and the
  name(s) used by the replication to effect copying. This latter
  revision of the definition of replication is crucial to obtaining
  the expected identity $!!P \sim !P$.
\end{remark}

\begin{remark}\label{rem:paradoxical_combinator}
  The reader familiar with the lambda calculus will have noticed the
  similarity between $D$ and the paradoxical combinator.

  [Ed. note: the existence of this seems to suggest we have to be more
  restrictive on the set of processes and names we admit if we are to
  support no-cloning.]
\end{remark}

\subsubsection{Bisimulation}

The computational dynamics gives rise to another kind of equivalence,
the equivalence of computational behavior. As previously mentioned
this is typically captured \emph{via} some form of bisimulation.

% The notion we use in this paper is weak barbed bisimulation
% \cite{milner91polyadicpi}.

The notion we use in this paper is derived from weak barbed
bisimulation \cite{milner91polyadicpi}. 

\begin{definition}
An \emph{observation relation}, $\downarrow_{\mathcal N}$, over a set
of names, $\mathcal N$, is the smallest relation satisfying the rules
below.

\infrule[Out-barb]{y \in {\mathcal N}, \; x \nameeq y}
		  {\outputp{x}{v} \downarrow_{\mathcal N} x}
\infrule[Par-barb]{\mbox{$P\downarrow_{\mathcal N} x$ or $Q\downarrow_{\mathcal N} x$}}
		  {\binpar{P}{Q} \downarrow_{\mathcal N} x}

We write $P \Downarrow_{\mathcal N} x$ if there is $Q$ such that 
$P \wred Q$ and $Q \downarrow_{\mathcal N} x$.
\end{definition}

\begin{definition}
%\label{def.bbisim}
An  ${\mathcal N}$-\emph{barbed bisimulation} over a set of names, ${\mathcal N}$, is a symmetric binary relation 
${\mathcal S}_{\mathcal N}$ between agents such that $P\rel{S}_{\mathcal N}Q$ implies:
\begin{enumerate}
\item If $P \red P'$ then $Q \wred Q'$ and $P'\rel{S}_{\mathcal N} Q'$.
\item If $P\downarrow_{\mathcal N} x$, then $Q\Downarrow_{\mathcal N} x$.
\end{enumerate}
$P$ is ${\mathcal N}$-barbed bisimilar to $Q$, written
$P \wbbisim_{\mathcal N} Q$, if $P \rel{S}_{\mathcal N} Q$ for some ${\mathcal N}$-barbed bisimulation ${\mathcal S}_{\mathcal N}$.
\end{definition}

$\mathcal{R} \subseteq \pi \times \pi$

$P \mathcal{R} Q => \forall P'. P \red P' \Rightarrow \exists Q'. Q \red Q', P' \mathcal{R} Q'$

$P \vdash x \Rightarrow Q \vdash x$

\begin{mathpar}
  \inferrule*[lab=Out-barb]{x \nameeq y}{{y}!\langle{Q}\rangle \vdash x}
  \and
  \inferrule*[lab=Par-barb]{\mbox{$P\vdash x$ or $Q\vdash x$}}{\binpar{P}{Q} \vdash x}
\end{mathpar}

\subsubsection{Contexts}

One of the principle advantages of computational calculi like the
$\pi$-calculus is a well-defined notion of context,
contextual-equivalence and a correlation between
contextual-equivalence and notions of bisimulation. The notion of
context allows the decomposition of a process into (sub-)process and
its syntactic environment, its context. Thus, a context may be
thought of as a process with a ``hole'' (written $\Box$) in it. The
application of a context $M$ to a process $P$, written $M[P]$, is
tantamount to filling the hole in $M$ with $P$. In this paper we do
not need the full weight of this theory, but do make use of the notion
of context in the proof the main theorem. 

\begin{mathpar}
  \inferrule* [lab=summation] {} {{M_{M},M_{N}} \bc \Box \;|\; x.M_{A} \;|\; M_{M}+M_{N}}
  \and
  \inferrule* [lab=agent] {} {{M_{A}} \bc (\vec{x})M_{P} \;| \; \clift{P_0,\ldots,M_{P},\ldots,P_N}}
  \and \\
  \inferrule* [lab=process] {} {{M_{P}} \bc M_{N} \;| \;P|M_{P} }
\end{mathpar} 

\begin{mathpar}
  \inferrule* [lab=sychronization] {} {M_{N} \bc \Box \;|\; x?M_{F} \;|\; x!M_{C}}
  \and
  \inferrule* [lab=abstraction] {} {{M_{F}} \bc (x)M_{P} }
  \and
  \inferrule* [lab=concretion] {} {{M_{C}} \bc \langle M_{P} \rangle }
  \and \\
  \inferrule* [lab=process] {} {{M_{P}} \bc M_{N} \;| \;P|M_{P} }
\end{mathpar}

\begin{definition}[contextual application] Given a context $M$, and
  process $P$, we define the \emph{contextual application}, $M[P] :=
  M\{P/\Box\}$. That is, the contextual application of M to P is the
  substitution of $P$ for $\Box$ in $M$.
\end{definition}

$\meaningof{-} : L \to \mathcal{P}(\pi)$

\begin{mathpar}
  \inferrule* [lab=collection] {} {\meaningof{true} = \pi, \and \meaningof{~E} = \pi \setminus \meaningof{E}, \and \meaningof{E_{1} \& E_{2}} = \meaningof{E_{1}} \cap \meaningof{E_{2}}}
\end{mathpar}

\begin{mathpar}
  \inferrule* [lab=structure] {} {\meaningof{0} = \{ P \in \pi | P \equiv 0 \}, \and \\ \meaningof{E_1 | E_2} = \{ P \in \pi | P \equiv P_{1} | P_{2}, P_{1} \in \meaningof{E_{1}}, P_{2} \in \meaningof{E_2}\} }
\end{mathpar}

\begin{mathpar}
 \inferrule* [lab=behavior] {} {\meaningof{\langle a?b \rangle E} = \{ P \in \pi | P \equiv Q | u?(y)P', \\ \and \\\\ \and \\ \;\;\; u \in \meaningof{a}, \forall z.P'\{z/y\} \in \meaningof{E\{z/b\}}\}, \and \\ \meaningof{a!E} = \{ P \in \pi | P \equiv Q | x!\langle P' \rangle, x \in \meaningof{a} P' \in \meaningof{E}\} }
\end{mathpar}

\begin{mathpar}
 \inferrule* [lab=nominal] {} {\meaningof{\quotep{E}} = \{ \quotep{P} \in \quotep{\pi} | P \in \meaningof{E} \}, \and \meaningof{\quotep{P}} = \{ \quotep{Q} \in \quotep{\pi} | P \equiv Q \} \and \\ \meaningof{@\quotep{E}} = \{ P \in \pi | P \equiv @x, x \in \meaningof{E} \}}
\end{mathpar}

\begin{eqnarray*}
  \\
  \meaningof{-} : TS \to ST
\end{eqnarray*}

\begin{eqnarray*}
  \\
  L : TS \to ST
\end{eqnarray*}

\begin{eqnarray*}
  \\
  P \models E \iff P \in \meaningof{E}
\end{eqnarray*}

\begin{eqnarray*}
  P \approx_{L} Q \iff \forall E \in L. P \models E \iff Q \models E
\end{eqnarray*}

\begin{eqnarray*}
  P \approx_{K} Q
\end{eqnarray*}

\begin{eqnarray*}
  P \approx Q
\end{eqnarray*}

$\approx_{K} = \approx = \approx_{L}$

\subsubsection{Contextual duality}

Note that contexts extend the quotation operation to a family of
operations from processes to names. Given a context, $M$, we can
define a \emph{nominal context}, $\quotep{M}$ by $\quotep{M}[P] :=
\quotep{M[P]}$. To foreshadow what is to come we observe that these
operations enjoy a duality with processes very much like the duality
between vectors and maps from vectors to scalars.

Further, because the calculus is essentially higher-order, we have a
correspondence between contexts and processes. More specifically,
given a name $x$ and a context $M$ we can construct $M^{*}_{x}$ such
that 

\begin{mathpar}
  M^{*}_{x} | \lift{x}{P} \red M[P]
\end{mathpar}

namely,

\begin{mathpar}
  M^{*}_{x} := x?(u).M[\dropn{u}]
\end{mathpar}

The dependence of $M^{*}_{x}$ on a name makes it an abstraction, 

\begin{mathpar}
  M^{*} := (x)x?(u).M[\dropn{u}]
\end{mathpar}

\subsection{Additional notation}

It will sometimes be convenient to denote the process a name
quotes. We already have the notation $x = \quotep{P}$, but it will be
convenient to introduce an alternate notation, $\procn{x}$, when we
want to emphasize the connection to the use of the name. Note that, by
virtue of name equivalence, $\quotep{\procn{x}} \nameeq x$; so, the
notation is consistent with previous definitions.

Further, because names have structure it is possible to effect
substitutions on the basis of that structure. This means we need to
upgrade our notation for substitutions, which we accomplish by
adapting comprehension notation. Thus,

\begin{mathpar}
  P\{ y / x : x \in S \}
\end{mathpar}

is interpreted to mean the process derived from P by replacing (in a
capture-avoiding manner) each occurrence of $x$ in $S$ by $y$. For example,

\begin{mathpar}
  P\{ \quotep{\procn{x}|\procn{x}} / x : x \in \freenames{P} \}
\end{mathpar}

will replace each (occurrence) of a free name $x$ in $P$ by
$\quotep{\procn{x}|\procn{x}}$.

Also, we will avail ourselves of the notation $x^{L}$ and $x^{R}$ to
denote injections of a name into disjoint copies of the name
space. There are numerous ways to accomplish this. One example can be
found in \cite{MeredithR05}. This notation overloads to vectors of
names: $\vec{x}^{\pi} := (x_{i}^{\pi} \; : \; 0 \leq i < |\vec{x}| )$ where $\pi \in \{L,R\}$.

We also use $P^{\Box} := P|\Box$.

In \cite{MeredithR05} an interpretation of the new operator is
given. It turns out that there are several possible interpretations
all enjoying the requisite algebraic properties of the operator (see
\cite{milner91polyadicpi}). We will therefore make liberal use of
$(\nu\; \vec{x})P$.

% subsection the_syntax_and_semantics_of_the_notation_system (end)   

\input{qm2pi.qmops} 

\input{qm2pi.sterngerlach} 

\input{qm2pi.metric} 

% section concurrent_process_calculi (end)

%\input{qm2pi.proofsketch}

% section proof sketch (end)

%\input{qm2pi.slviaknots} 

% section spatial logic via knots (end)

\input{qm2pi.conclusion}

% section conclusion (end)

%\input{qm2pi.dtcodes} 

% section wiring algorithm (end)

\input{qm2pi.ack} 

% section acknowledgments (end)

\newpage


\bibliographystyle{plain}   
\bibliography{../../biblios/main.bib}

\input{qm2pi.rhodetails}

\end{document}



% section front matter (end)

\section{Introduction}\label{sec:introduction} % (fold)
In this draft of the material i am going to have to dispense with the
usual writing conventions adopted in papers on these topics. i'm going
to have adopt whatever tone i need at the time i'm writing up the
calculations. Sometimes this may be very conversational; others it may
be the barest mathematical grunts; others still it may be that i have
lifted text from one of my other papers because the exposition of some
point was better said there. i hope that my readers are not unduly put
out by this decision. i'm not doing this to flout convention or be
rebellious. i find these calculations very technically challenging. To
keep everything going technically, something has to give; i have to
let go of some cognitive burden. So, the academic writing style --
with all of its trade-offs in terms of facilitating technical
communication -- is what i'm letting go of. Perhaps subsequent drafts
can be tightened and polished, but for now, i'm going to speak as if
we were sitting together in a coffee shop with a laptop, wifi and a
pad of paper and a pencil.

So, here's what i have to say. We -- you and i, comfortably ensconced
in our coffee shop and well-equipped with our tools -- can realize and
carry out the calculations of quantum mechanics over a very different
formal theory of dynamics, a formal theory of dynamics that
corresponds to a theory of concurrent computation with
\emph{reflection}. It has the advantage that the underlying theory is
already `quantized', but supports analogues all of the continuuous
operations. Strikingly, this underlying theory has recently been
connected with a notion of metric that we can show, by calculating
together, coincides with the metric induced by the inner product.

There are a lot of reasons why you might be interested in seeing
calculations of this form. Here's why i'm interested. For the past
several centuries there has been no competitor to the ``Newtonian''
account of dynamics. As a result the predominant share of accounts of
dynamical systems and situations have had to be formulated in terms of
the Newtonian machinery. i view this as an intellectually dangerous
position to occupy. Everything, despite it's intrinsic shape, turns
into a nail to be hit with this hammer. Recently, however, the theory
of computation has matured to the point where we have candidates for
theories of dynamics that offer very different perspective on
reasoning about dynamical systems and situations. Testing these
candidates against very successful accounts of dynamical situations,
like quantum mechanics, is going to give us some sense of how mature
they are and some measure of the quality of these accounts of
dynamics.

\subsection{Summary of contributions and outline of paper}

So, we're going to develop an interpretation of the operations of
quantum mechanics normally interpreted by Hilbert spaces and
operators. We're going to do this over a theory of computation. Note
that this is very different than the usual quantum computation program
which develops notions of computation over quantum mechanics. Rather,
we are developing a story that aligns with Wheeler's slogan: It from
Bit. To do this we will first provide an account of the theory of
computation at play here. Then we will dive into a calculation-driven
interpretation of the operations of quantum mechanics.

The reason we take this approach is that -- until very recently --
there hasn't been an axiomatic account of quantum mechanics. As a
result there has been no sharp delineation of the mathematical theory
supporting interpretation of the physical theory and the physical
theory, itself. So, ambient features of the maths are free to be
exploited (or supressed) without a real accounting of their physical
relevance. There is no sharp statement ``here's the physical theory''
qua \emph{theory} and ``here's the mathematical interpretation''
enabling a judgment of how faithful the interpretation is -- apart
from experimental observation. When there is an axiomatic account we
can judge how well a given mathematical formalism supports an
interpretation of the axioms, independent of
experimentation. Likewise, we can judge how well we have captured our
physical evidence and experience with our axiomatics, independent of
any specific mathematical implementation, with accidental detail that
may or may not have physical significance. 

In lieu of a fully fleshed out and vetted axiomatic account of quantum
mechanics, interpreting the operational notions in service of modeling
physical systems will have to suffice. In other words, we are not in
the business of providing a model of Hilbert spaces and operators. We
are in the business of providing a model of quantum mechanics because
we are motivated by testing our notions of dynamics against physical
theory; and, the predictive calculations of the physical theory must
serve as the best formulation -- shy of a fully fleshed out axiomatic
account -- of the physical theory itself (as they have for scientific
theories since time immemorial). Put another way, despite a
whole-hearted commitment to an It-from-Bit ontology, we are firmly
aligned with the shut-up-and-calculate camp as the best way to obtain
results either from the physical perspective or as a quality assurance
measure of our fledgling theory of dynamics.

In detail, we present a reflective process calculus. Then we develop
intuitive correspondences between the notions available in this
calculus and the usual physical notions supporting quantum mechanical
calculations. Thus, 

\begin{table}[htp]
  \center{
    \fbox{
      \begin{tabular}{c|c}
        quantum mechanics & process calculus \\
        \hline
        scalar & name \\
        state vector & process \\
        dual & contextual duals \\
        matrix & formal sums of process-context-dual pairs \\
        orthogonality & process annihilation \\
        inner product & execution-formula + quoting
      \end{tabular}
    }
  }
  \caption{QM - process calculi correspondences}
\end{table}

Then we tighten up these intuitions to operational definitions. We
employ the Dirac notation as the best proxy we can find for an
abstract syntax of the quantum mechanical notions. The definitions we
develop put us in contact with equational constraints coming from the
theory that we demonstrate the definitions and calculations satisfy.

This puts us in a position to shut up and calculate for the
Stern-Gerlach experimental set up, showing how these predictive
calculations become calculations on processes in our theory of a
reflective process calculus.

Penultimately, we demonstrate that the notion of metric coming from
the inner product coincides with the notion of metric available from
the theory of bisimulation. This demonstration gives us the right to
think of space as arising from behavior. Finally, we consider where we
might go from the new vantage point we have obtained.

% section introduction (end) 
 
% section introduction (end)

% \documentclass[12pt]{llncs}
%\documentclass{jktr}

\usepackage[pdftex]{hyperref}                   
\usepackage {listings}
\usepackage {mathpartir}
\usepackage{bcprules}
%\usepackage{listings}
                       
\usepackage{graphicx} 
%\usepackage[margins=2.5cm,nohead,nofoot]{geometry}
%\usepackage{geometry}
\usepackage{amsfonts}
\usepackage{amstext}
\usepackage{latexsym}
\usepackage{amssymb}
\usepackage{color}


%\include{myPreamble}
\include{qm2pi.local} 

%\ifpdf
%\usepackage[pdftex]{graphicx}
%\else
%\usepackage{graphicx}
%\fi

 % \ifpdf
%  \usepackage{pdfsync}
%  \if


%\title{Brief Article}
%\author{David F. Snyder}
%\author{L.G. Meredith}

%\address{Dept. of Math., Texas State University--San Marcos, San Marcos, TX 78666}
       
\pagestyle{empty}


\begin{document}

\lstset{language=[Objective]Caml,frame=shadowbox}

\input{qm2pi.front}

% section front matter (end)

\input{qm2pi.intro} 
 
% section introduction (end)

% \input{qm2pi.knotations} 

% section notation (end)

\input{qm2pi.process.calculi} 

% section concurrent_process_calculi_and_spatial_logics_ (end)
    
%\input{qm2pi.knots2pi} 

%\input{qm2pi.trefoil} 

%\input{qm2pi.mainthm} 

% subsection basic_interpretation (end)

%\input{qm2pi.rho.presentation} 
\subsection{The syntax and semantics of the notation system}\label{sub:the_syntax_and_semantics_of_the_notation_system} % (fold)

We now summarize a technical presentation of the calculus that
embodies our theory of dynamics. The typical presentation of such a
calculus follows the style of giving generators and relations on
them. The grammar, below, describing term constructors, freely
generates the set of processes, $\Proc$. This set is then quotiented
by a relation known as structural congruence and it is over this set
that the notion of dynamics is expressed. This presentation is
essentially that of \cite{MeredithR05} with the addition of
polyadicity and summation. For readability we have relegated some of
the technical subtleties to an appendix.

\subsubsection{Process grammar}\label{subsub:process_grammar}

\begin{mathpar}
  \inferrule* [lab=synchronization] {} {{M} \bc \pzero \;|\; x?F \;|\; x!C }
  \and
  \inferrule* [lab=abstraction] {} {{F} \bc (x)P}
  \and
  \inferrule* [lab=concretion] {} {{C} \bc \langle Q \rangle}
  \and
  \inferrule* [lab=process] {} {{P,Q} \bc M \;| \;P|Q \;|\; @{x}}
  \and
  \inferrule* [lab=name] {} {{x} \bc \quotep{P}}
\end{mathpar} 

Note that $\vec{x}$ (resp. $\vec{P}$) denotes a vector of names
(resp. processes) of length $|\vec{x}|$ (resp. $|\vec{P}|$). We adopt
the following useful abbreviations.

\begin{mathpar}
   x?(\vec{y}).P := x.(\vec{y})P \and  x\clift{\vec{P}} := x.\clift{\vec{P}}
   \and x!(y) := \lift{x}{\dropn{y}}
   \and \Pi_{i=0}^{n-1}P_i := P_0 | \ldots | P_{n-1}
\end{mathpar}

\subsubsection{Structural congruence}

\paragraph{Free and bound names and alpha-equivalence.} At the
core of structural equivalence is alpha-equivalence which identifies
process that are the same up to a change of variable. Formally, we
recognize the distinction between free and bound names. The free names
of a process, $\freenames{P}$, may be calculated recursively as
follows:

\begin{mathpar}
\freenames{\pzero} := \emptyset
  \and \\
  \freenames{x?(y).P} := \{ x \} \cup (\freenames{P} \setminus \{ y \})
  \and 
  \freenames{x!\langle P \rangle} := \{ x \} \cup \{ P \} 
  \and \\
  \freenames{P|Q} := \freenames{P} \cup \freenames{Q}
  \and \\
  \freenames{@{x}} := \{ x \}
\end{mathpar}

$\pi$
$\quotep{\pi}$

$\freenames{-} : \pi \to \mathcal{P}(\quotep{\pi})$

\begin{eqnarray*}
  \freenames{\pzero} & := & \emptyset \\
  \freenames{x?(y).P} & := & \{ x \} \cup (\freenames{P} \setminus \{ y \}) \\
  \freenames{x!\langle P \rangle} & := & \{ x \} \cup \{ P \} \\
  \freenames{P|Q} & := & \freenames{P} \cup \freenames{Q} \\
  \freenames{\dropn{x}} & := & \{ x \}
\end{eqnarray*}

The bound names of a process, $\boundnames{P}$, are those names occurring in $P$
that are not free. For example, in $x?(y).0$, the name $x$ is free, while $y$ is bound.

\begin{mathpar}
  \inferrule* [lab=monoidal-laws] {} { P|Q \equiv Q|P \and P|0 \equiv P \and P|(Q|R) \equiv (P|Q)|R }
\end{mathpar}

\begin{mathpar}
  \inferrule* [lab=alpha-equivalence] {} { (x)P \equiv (y)P\{y/x\} \and y \not\in \freenames{P} }
\end{mathpar}

\begin{definition}
Then two processes, $P,Q$, are alpha-equivalent if $P = Q\{\vec{y}/\vec{x}\}$ for
some $\vec{x} \in \boundnames{Q},\vec{y} \in \boundnames{P}$, where $Q\{\vec{y}/\vec{x}\}$
denotes the capture-avoiding substitution of $\vec{y}$ for $\vec{x}$ in $Q$.
\end{definition}

\begin{definition}
  The {\em structural congruence} \cite{SangiorgiWalker} , $\equiv$,
  between processes is the least congruence containing
  alpha-equivalence, satisfying the abelian monoid laws
  (associativity, commutativity and $\pzero$ as identity) for parallel
  composition $|$ and for summation $+$.
\end{definition}

\subsection{Name equivalence}

We take name equivalence, written $\nameeq$, to be the smallest
equivalence relation generated by the following rules.

\begin{mathpar}
\inferrule*[lab=Quote-drop]
{ }
{ \quotep{@{x}} \nameeq x }

\inferrule*[lab=Struct-equiv]
{ P \scong Q }
{ \quotep{P} \nameeq \quotep{Q} }
\end{mathpar}

The astute reader will have noticed that the mutual recursion of names
and processes imposes a mutual recursion on alpha-equivalence and
structural equivalence via name-equivalence. Fortunately, all of this
works out pleasantly and we may calculate in the natural way, free of
concern. The reader interested in the details is referred to the
appendix \ref{appendix:rho_details}.

\subsection{Substitution}

We use $\Proc$ for the set of processes, $\QProc$ for the set of
names, and $\id{\{}\vec{y} / \vec{x} \id{\}}$ to denote partial maps,
$s : \QProc \rightarrow \QProc$. A map, $s$ lifts, uniquely, to a map
on process terms, $\widehat{s} : \Proc \rightarrow \Proc$ by the
following equations.

\begin{mathpar}
  (0) \psubstp{Q}{P} := 0 \\
  (R \juxtap S) \psubstp{Q}{P}
  :=    
  (R)\psubstp{Q}{P} \juxtap (S) \psubstp{Q}{P} \\
  (x?(y).R) \psubstp{Q}{P}    
  :=    
  (x)\substp{Q}{P} (z)\concat( (R \psubstn{z}{y}) \psubstp{Q}{P} ) \\
  (\lift{x}{R}) \psubstp{Q}{P}  
  :=
  \lift{(x)\substp{Q}{P}}{ R \psubstp{Q}{P} } \\
%   (\dropn{x})  \psubstp{Q}{P}       
%   := 
%   \left\{ 
%     \begin{array}{ccc} 
%       \dropn{\quotep{Q}} & & x \nameeq \quotep{P} \\
%       \dropn{x} & & otherwise \\
%     \end{array}
%   \right. 
  (\dropn{x})  \psubstp{Q}{P}       
  := 
  \left\{ 
    \begin{array}{ccc} 
      Q & & x \nameeq \quotep{P} \\
      \dropn{x} & & otherwise \\
    \end{array}
  \right.
\end{mathpar}
 

where

\begin{eqnarray}
  (x)\id{\{} \lpquote Q \rpquote / \lpquote P \rpquote \id{\}}            = 
  \left\{ 
    \begin{array}{ccc}
      \lpquote Q \rpquote & & x \nameeq \lpquote P \rpquote \\
      x & & otherwise \\
    \end{array}
  \right. \nonumber
\end{eqnarray}

and $z$ is chosen distinct from $\quotep{P}$, $\quotep{Q}$, the free
names in $Q$, and all the names in $R$. Our $\alpha$-equivalence will
be built in the standard way from this substitution.

\begin{remark}\label{rem:no_self_referential_names}
  One consequence of these definitions is that $\forall P. \quotep{P}
  \not\in \freenames{P}$.
\end{remark}

\subsection{ Dynamic quote: an example }

Anticipating something of what's to come, consider applying the
substitution, $\widehat{\id{\{}u / z \id{\}}}$, to the following pair
of processes, $\lift{w}{y!(z)}$ and $w[ \lpquote y!(z) \rpquote ]$.

\begin{eqnarray}
	\lift{w}{y!(z)}\widehat{\id{\{}u / z \id{\}}}
		& = &
		\lift{w}{y!(u)} \nonumber\\
	w[ \lpquote y!(z) \rpquote ] \widehat{ \id{\{}u / z \id{\}} }
		& = &
		w[ \lpquote y!(z) \rpquote ] \nonumber
\end{eqnarray}

Because the body of the process between quotes is impervious to
substitution, we get radically different answers. In fact, by
examining the first process in an input context,
e.g. $x?(z).\lift{w}{y!(z)}$, we see that the process under the lift
operator may be shaped by prefixed inputs binding a name inside it. In
this sense, the lift operator will be seen as a way to dynamically
construct processes before reifying them as names.

Finally equipped with these standard features we can present the
dynamics of the calculus.

\subsubsection{Operational semantics} 

Finally, we introduce the computational dynamics. What marks these
algebras as distinct from other more traditionally studied algebraic
structures, e.g. vector spaces or polynomial rings, is the manner in
which dynamics is captured. In traditional structures, dynamics is typically
expressed through morphisms between such structures, as in linear maps
between vector spaces or morphisms between rings. In algebras
associated with the semantics of computation, the dynamics is
expressed as part of the algebraic structure itself, through a
reduction reduction relation typically denoted by $\red$. Below, we
give a recursive presentation of this relation for the calculus used
in the encoding.

$\red \subseteq \pi \times \pi$
$\red : \pi \to \mathcal{P}(\pi)$

\begin{mathpar}
  \inferrule* [lab=Comm] { \textsf{match}( x_{src}, x_{trgt} ) } { x_{trgt}?(y)P \; | \; x_{src}!\langle {Q} \rangle \red P\{\quotep{Q}/y}\} }
  \and \\
  \inferrule* [lab=Par] {{P} \red {P}'} {{{P} | {Q}} \red {{P}' | {Q}}}
  \and
  \inferrule* [lab=Equiv]{{{P} \scong {P}'} \andalso {{P}' \red {Q}'} \andalso {{Q}' \scong {Q}}}{{P} \red {Q}}
\end{mathpar}

\begin{eqnarray*}
  match_{\equiv} (\quotep{P},\quotep{Q}) & := & P \equiv Q \\
  match_{\dagger}(\quotep{P},\quotep{Q}) & := & \forall R. P|Q \red^{*} R => R \red^{*} 0 \\
  match_{K}(\quotep{P},\quotep{Q}) & := & K \mbox{ for some context } K
\end{eqnarray*}

$u?(x)P | u!\langle Q \rangle \red P\{\quotep{Q}/x\}$

%We write $\wred$ for $\red^*$, and $P\red$ if $\exists Q $ such that $ P \red Q$.
We write $P\red$ if $\exists Q $ such that $ P \red Q$ and $P\not\red$, otherwise.

\section{Replication}

As mentioned before, it is known that replication (and hence
recursion) can be implemented in a higher-order process algebra
\cite{SangiorgiWalker}. As our first example of calculation with the
machinery thus far presented we give the construction explicitly in
the {\rhoc}.

\begin{eqnarray}
	D_{x} & := & \prefix{x}{y}{(\binpar{\outputp{x}{y}}{@{y}})} \nonumber\\
	\bangp_{x}{P} & := & \binpar{{x}!\langle{\binpar{D_{x}}{P}}\rangle}{D_{x}} \nonumber
\end{eqnarray}

\begin{eqnarray}
	\bangp_{x}{P} & & \nonumber\\
	=
	& {x}!\langle{(\prefix{x}{y}{(\outputp{x}{y} | @{y})) | P}}\rangle 
	      | \prefix{x}{y}{(\outputp{x}{y} | @{y})} & \nonumber\\
	\red
	& (\outputp{x}{y} | @{y})\substn{\quotep{(\prefix{x}{y}{(@{y} | \outputp{x}{y})) | P}}}{y} & \nonumber\\
	=
	& \outputp{x}{\quotep{(\prefix{x}{y}{(\outputp{x}{y} | @{y})) | P}}}
	  | {(\prefix{x}{y}{(\outputp{x}{y} | @{y})) | P}} & \nonumber\\
	\red
	& \ldots & \nonumber\\
	\red^*
	& P | P | \ldots & \nonumber
\end{eqnarray}

Of course, this encoding, as an implementation, runs away, unfolding
$\bangp{P}$ eagerly. A lazier and more implementable replication
operator, restricted to input-guarded processes, may be obtained as follows.

\begin{eqnarray}
\bangp{\prefix{u}{v}{P}} 
	:= 
	\binpar{\lift{x}{\prefix{u}{v}{(\binpar{D(x)}{P})}}}{D(x)} \nonumber
\end{eqnarray}

\begin{remark}
  Note that the lazier definition still does not deal with summation
  or mixed summation (i.e. sums over input and output). The reader is
  invited to construct definitions of replication that deal with these
  features. 

  Further, the definitions are parameterized in a name, $x$. Can you,
  gentle reader, make a definition that eliminates this parameter and
  guarantees no accidental interaction between the replication
  machinery and the process being replicated -- i.e. no accidental
  sharing of names used by the process to get its work done and the
  name(s) used by the replication to effect copying. This latter
  revision of the definition of replication is crucial to obtaining
  the expected identity $!!P \sim !P$.
\end{remark}

\begin{remark}\label{rem:paradoxical_combinator}
  The reader familiar with the lambda calculus will have noticed the
  similarity between $D$ and the paradoxical combinator.

  [Ed. note: the existence of this seems to suggest we have to be more
  restrictive on the set of processes and names we admit if we are to
  support no-cloning.]
\end{remark}

\subsubsection{Bisimulation}

The computational dynamics gives rise to another kind of equivalence,
the equivalence of computational behavior. As previously mentioned
this is typically captured \emph{via} some form of bisimulation.

% The notion we use in this paper is weak barbed bisimulation
% \cite{milner91polyadicpi}.

The notion we use in this paper is derived from weak barbed
bisimulation \cite{milner91polyadicpi}. 

\begin{definition}
An \emph{observation relation}, $\downarrow_{\mathcal N}$, over a set
of names, $\mathcal N$, is the smallest relation satisfying the rules
below.

\infrule[Out-barb]{y \in {\mathcal N}, \; x \nameeq y}
		  {\outputp{x}{v} \downarrow_{\mathcal N} x}
\infrule[Par-barb]{\mbox{$P\downarrow_{\mathcal N} x$ or $Q\downarrow_{\mathcal N} x$}}
		  {\binpar{P}{Q} \downarrow_{\mathcal N} x}

We write $P \Downarrow_{\mathcal N} x$ if there is $Q$ such that 
$P \wred Q$ and $Q \downarrow_{\mathcal N} x$.
\end{definition}

\begin{definition}
%\label{def.bbisim}
An  ${\mathcal N}$-\emph{barbed bisimulation} over a set of names, ${\mathcal N}$, is a symmetric binary relation 
${\mathcal S}_{\mathcal N}$ between agents such that $P\rel{S}_{\mathcal N}Q$ implies:
\begin{enumerate}
\item If $P \red P'$ then $Q \wred Q'$ and $P'\rel{S}_{\mathcal N} Q'$.
\item If $P\downarrow_{\mathcal N} x$, then $Q\Downarrow_{\mathcal N} x$.
\end{enumerate}
$P$ is ${\mathcal N}$-barbed bisimilar to $Q$, written
$P \wbbisim_{\mathcal N} Q$, if $P \rel{S}_{\mathcal N} Q$ for some ${\mathcal N}$-barbed bisimulation ${\mathcal S}_{\mathcal N}$.
\end{definition}

$\mathcal{R} \subseteq \pi \times \pi$

$P \mathcal{R} Q => \forall P'. P \red P' \Rightarrow \exists Q'. Q \red Q', P' \mathcal{R} Q'$

$P \vdash x \Rightarrow Q \vdash x$

\begin{mathpar}
  \inferrule*[lab=Out-barb]{x \nameeq y}{{y}!\langle{Q}\rangle \vdash x}
  \and
  \inferrule*[lab=Par-barb]{\mbox{$P\vdash x$ or $Q\vdash x$}}{\binpar{P}{Q} \vdash x}
\end{mathpar}

\subsubsection{Contexts}

One of the principle advantages of computational calculi like the
$\pi$-calculus is a well-defined notion of context,
contextual-equivalence and a correlation between
contextual-equivalence and notions of bisimulation. The notion of
context allows the decomposition of a process into (sub-)process and
its syntactic environment, its context. Thus, a context may be
thought of as a process with a ``hole'' (written $\Box$) in it. The
application of a context $M$ to a process $P$, written $M[P]$, is
tantamount to filling the hole in $M$ with $P$. In this paper we do
not need the full weight of this theory, but do make use of the notion
of context in the proof the main theorem. 

\begin{mathpar}
  \inferrule* [lab=summation] {} {{M_{M},M_{N}} \bc \Box \;|\; x.M_{A} \;|\; M_{M}+M_{N}}
  \and
  \inferrule* [lab=agent] {} {{M_{A}} \bc (\vec{x})M_{P} \;| \; \clift{P_0,\ldots,M_{P},\ldots,P_N}}
  \and \\
  \inferrule* [lab=process] {} {{M_{P}} \bc M_{N} \;| \;P|M_{P} }
\end{mathpar} 

\begin{mathpar}
  \inferrule* [lab=sychronization] {} {M_{N} \bc \Box \;|\; x?M_{F} \;|\; x!M_{C}}
  \and
  \inferrule* [lab=abstraction] {} {{M_{F}} \bc (x)M_{P} }
  \and
  \inferrule* [lab=concretion] {} {{M_{C}} \bc \langle M_{P} \rangle }
  \and \\
  \inferrule* [lab=process] {} {{M_{P}} \bc M_{N} \;| \;P|M_{P} }
\end{mathpar}

\begin{definition}[contextual application] Given a context $M$, and
  process $P$, we define the \emph{contextual application}, $M[P] :=
  M\{P/\Box\}$. That is, the contextual application of M to P is the
  substitution of $P$ for $\Box$ in $M$.
\end{definition}

$\meaningof{-} : L \to \mathcal{P}(\pi)$

\begin{mathpar}
  \inferrule* [lab=collection] {} {\meaningof{true} = \pi, \and \meaningof{~E} = \pi \setminus \meaningof{E}, \and \meaningof{E_{1} \& E_{2}} = \meaningof{E_{1}} \cap \meaningof{E_{2}}}
\end{mathpar}

\begin{mathpar}
  \inferrule* [lab=structure] {} {\meaningof{0} = \{ P \in \pi | P \equiv 0 \}, \and \\ \meaningof{E_1 | E_2} = \{ P \in \pi | P \equiv P_{1} | P_{2}, P_{1} \in \meaningof{E_{1}}, P_{2} \in \meaningof{E_2}\} }
\end{mathpar}

\begin{mathpar}
 \inferrule* [lab=behavior] {} {\meaningof{\langle a?b \rangle E} = \{ P \in \pi | P \equiv Q | u?(y)P', \\ \and \\\\ \and \\ \;\;\; u \in \meaningof{a}, \forall z.P'\{z/y\} \in \meaningof{E\{z/b\}}\}, \and \\ \meaningof{a!E} = \{ P \in \pi | P \equiv Q | x!\langle P' \rangle, x \in \meaningof{a} P' \in \meaningof{E}\} }
\end{mathpar}

\begin{mathpar}
 \inferrule* [lab=nominal] {} {\meaningof{\quotep{E}} = \{ \quotep{P} \in \quotep{\pi} | P \in \meaningof{E} \}, \and \meaningof{\quotep{P}} = \{ \quotep{Q} \in \quotep{\pi} | P \equiv Q \} \and \\ \meaningof{@\quotep{E}} = \{ P \in \pi | P \equiv @x, x \in \meaningof{E} \}}
\end{mathpar}

\begin{eqnarray*}
  \\
  \meaningof{-} : TS \to ST
\end{eqnarray*}

\begin{eqnarray*}
  \\
  L : TS \to ST
\end{eqnarray*}

\begin{eqnarray*}
  \\
  P \models E \iff P \in \meaningof{E}
\end{eqnarray*}

\begin{eqnarray*}
  P \approx_{L} Q \iff \forall E \in L. P \models E \iff Q \models E
\end{eqnarray*}

\begin{eqnarray*}
  P \approx_{K} Q
\end{eqnarray*}

\begin{eqnarray*}
  P \approx Q
\end{eqnarray*}

$\approx_{K} = \approx = \approx_{L}$

\subsubsection{Contextual duality}

Note that contexts extend the quotation operation to a family of
operations from processes to names. Given a context, $M$, we can
define a \emph{nominal context}, $\quotep{M}$ by $\quotep{M}[P] :=
\quotep{M[P]}$. To foreshadow what is to come we observe that these
operations enjoy a duality with processes very much like the duality
between vectors and maps from vectors to scalars.

Further, because the calculus is essentially higher-order, we have a
correspondence between contexts and processes. More specifically,
given a name $x$ and a context $M$ we can construct $M^{*}_{x}$ such
that 

\begin{mathpar}
  M^{*}_{x} | \lift{x}{P} \red M[P]
\end{mathpar}

namely,

\begin{mathpar}
  M^{*}_{x} := x?(u).M[\dropn{u}]
\end{mathpar}

The dependence of $M^{*}_{x}$ on a name makes it an abstraction, 

\begin{mathpar}
  M^{*} := (x)x?(u).M[\dropn{u}]
\end{mathpar}

\subsection{Additional notation}

It will sometimes be convenient to denote the process a name
quotes. We already have the notation $x = \quotep{P}$, but it will be
convenient to introduce an alternate notation, $\procn{x}$, when we
want to emphasize the connection to the use of the name. Note that, by
virtue of name equivalence, $\quotep{\procn{x}} \nameeq x$; so, the
notation is consistent with previous definitions.

Further, because names have structure it is possible to effect
substitutions on the basis of that structure. This means we need to
upgrade our notation for substitutions, which we accomplish by
adapting comprehension notation. Thus,

\begin{mathpar}
  P\{ y / x : x \in S \}
\end{mathpar}

is interpreted to mean the process derived from P by replacing (in a
capture-avoiding manner) each occurrence of $x$ in $S$ by $y$. For example,

\begin{mathpar}
  P\{ \quotep{\procn{x}|\procn{x}} / x : x \in \freenames{P} \}
\end{mathpar}

will replace each (occurrence) of a free name $x$ in $P$ by
$\quotep{\procn{x}|\procn{x}}$.

Also, we will avail ourselves of the notation $x^{L}$ and $x^{R}$ to
denote injections of a name into disjoint copies of the name
space. There are numerous ways to accomplish this. One example can be
found in \cite{MeredithR05}. This notation overloads to vectors of
names: $\vec{x}^{\pi} := (x_{i}^{\pi} \; : \; 0 \leq i < |\vec{x}| )$ where $\pi \in \{L,R\}$.

We also use $P^{\Box} := P|\Box$.

In \cite{MeredithR05} an interpretation of the new operator is
given. It turns out that there are several possible interpretations
all enjoying the requisite algebraic properties of the operator (see
\cite{milner91polyadicpi}). We will therefore make liberal use of
$(\nu\; \vec{x})P$.

% subsection the_syntax_and_semantics_of_the_notation_system (end)   

\input{qm2pi.qmops} 

\input{qm2pi.sterngerlach} 

\input{qm2pi.metric} 

% section concurrent_process_calculi (end)

%\input{qm2pi.proofsketch}

% section proof sketch (end)

%\input{qm2pi.slviaknots} 

% section spatial logic via knots (end)

\input{qm2pi.conclusion}

% section conclusion (end)

%\input{qm2pi.dtcodes} 

% section wiring algorithm (end)

\input{qm2pi.ack} 

% section acknowledgments (end)

\newpage


\bibliographystyle{plain}   
\bibliography{../../biblios/main.bib}

\input{qm2pi.rhodetails}

\end{document}

 

% section notation (end)

\input{qm2pi.process.calculi} 

% section concurrent_process_calculi_and_spatial_logics_ (end)
    
%\documentclass[12pt]{llncs}
%\documentclass{jktr}

\usepackage[pdftex]{hyperref}                   
\usepackage {listings}
\usepackage {mathpartir}
\usepackage{bcprules}
%\usepackage{listings}
                       
\usepackage{graphicx} 
%\usepackage[margins=2.5cm,nohead,nofoot]{geometry}
%\usepackage{geometry}
\usepackage{amsfonts}
\usepackage{amstext}
\usepackage{latexsym}
\usepackage{amssymb}
\usepackage{color}


%\include{myPreamble}
\include{qm2pi.local} 

%\ifpdf
%\usepackage[pdftex]{graphicx}
%\else
%\usepackage{graphicx}
%\fi

 % \ifpdf
%  \usepackage{pdfsync}
%  \if


%\title{Brief Article}
%\author{David F. Snyder}
%\author{L.G. Meredith}

%\address{Dept. of Math., Texas State University--San Marcos, San Marcos, TX 78666}
       
\pagestyle{empty}


\begin{document}

\lstset{language=[Objective]Caml,frame=shadowbox}

\input{qm2pi.front}

% section front matter (end)

\input{qm2pi.intro} 
 
% section introduction (end)

% \input{qm2pi.knotations} 

% section notation (end)

\input{qm2pi.process.calculi} 

% section concurrent_process_calculi_and_spatial_logics_ (end)
    
%\input{qm2pi.knots2pi} 

%\input{qm2pi.trefoil} 

%\input{qm2pi.mainthm} 

% subsection basic_interpretation (end)

%\input{qm2pi.rho.presentation} 
\subsection{The syntax and semantics of the notation system}\label{sub:the_syntax_and_semantics_of_the_notation_system} % (fold)

We now summarize a technical presentation of the calculus that
embodies our theory of dynamics. The typical presentation of such a
calculus follows the style of giving generators and relations on
them. The grammar, below, describing term constructors, freely
generates the set of processes, $\Proc$. This set is then quotiented
by a relation known as structural congruence and it is over this set
that the notion of dynamics is expressed. This presentation is
essentially that of \cite{MeredithR05} with the addition of
polyadicity and summation. For readability we have relegated some of
the technical subtleties to an appendix.

\subsubsection{Process grammar}\label{subsub:process_grammar}

\begin{mathpar}
  \inferrule* [lab=synchronization] {} {{M} \bc \pzero \;|\; x?F \;|\; x!C }
  \and
  \inferrule* [lab=abstraction] {} {{F} \bc (x)P}
  \and
  \inferrule* [lab=concretion] {} {{C} \bc \langle Q \rangle}
  \and
  \inferrule* [lab=process] {} {{P,Q} \bc M \;| \;P|Q \;|\; @{x}}
  \and
  \inferrule* [lab=name] {} {{x} \bc \quotep{P}}
\end{mathpar} 

Note that $\vec{x}$ (resp. $\vec{P}$) denotes a vector of names
(resp. processes) of length $|\vec{x}|$ (resp. $|\vec{P}|$). We adopt
the following useful abbreviations.

\begin{mathpar}
   x?(\vec{y}).P := x.(\vec{y})P \and  x\clift{\vec{P}} := x.\clift{\vec{P}}
   \and x!(y) := \lift{x}{\dropn{y}}
   \and \Pi_{i=0}^{n-1}P_i := P_0 | \ldots | P_{n-1}
\end{mathpar}

\subsubsection{Structural congruence}

\paragraph{Free and bound names and alpha-equivalence.} At the
core of structural equivalence is alpha-equivalence which identifies
process that are the same up to a change of variable. Formally, we
recognize the distinction between free and bound names. The free names
of a process, $\freenames{P}$, may be calculated recursively as
follows:

\begin{mathpar}
\freenames{\pzero} := \emptyset
  \and \\
  \freenames{x?(y).P} := \{ x \} \cup (\freenames{P} \setminus \{ y \})
  \and 
  \freenames{x!\langle P \rangle} := \{ x \} \cup \{ P \} 
  \and \\
  \freenames{P|Q} := \freenames{P} \cup \freenames{Q}
  \and \\
  \freenames{@{x}} := \{ x \}
\end{mathpar}

$\pi$
$\quotep{\pi}$

$\freenames{-} : \pi \to \mathcal{P}(\quotep{\pi})$

\begin{eqnarray*}
  \freenames{\pzero} & := & \emptyset \\
  \freenames{x?(y).P} & := & \{ x \} \cup (\freenames{P} \setminus \{ y \}) \\
  \freenames{x!\langle P \rangle} & := & \{ x \} \cup \{ P \} \\
  \freenames{P|Q} & := & \freenames{P} \cup \freenames{Q} \\
  \freenames{\dropn{x}} & := & \{ x \}
\end{eqnarray*}

The bound names of a process, $\boundnames{P}$, are those names occurring in $P$
that are not free. For example, in $x?(y).0$, the name $x$ is free, while $y$ is bound.

\begin{mathpar}
  \inferrule* [lab=monoidal-laws] {} { P|Q \equiv Q|P \and P|0 \equiv P \and P|(Q|R) \equiv (P|Q)|R }
\end{mathpar}

\begin{mathpar}
  \inferrule* [lab=alpha-equivalence] {} { (x)P \equiv (y)P\{y/x\} \and y \not\in \freenames{P} }
\end{mathpar}

\begin{definition}
Then two processes, $P,Q$, are alpha-equivalent if $P = Q\{\vec{y}/\vec{x}\}$ for
some $\vec{x} \in \boundnames{Q},\vec{y} \in \boundnames{P}$, where $Q\{\vec{y}/\vec{x}\}$
denotes the capture-avoiding substitution of $\vec{y}$ for $\vec{x}$ in $Q$.
\end{definition}

\begin{definition}
  The {\em structural congruence} \cite{SangiorgiWalker} , $\equiv$,
  between processes is the least congruence containing
  alpha-equivalence, satisfying the abelian monoid laws
  (associativity, commutativity and $\pzero$ as identity) for parallel
  composition $|$ and for summation $+$.
\end{definition}

\subsection{Name equivalence}

We take name equivalence, written $\nameeq$, to be the smallest
equivalence relation generated by the following rules.

\begin{mathpar}
\inferrule*[lab=Quote-drop]
{ }
{ \quotep{@{x}} \nameeq x }

\inferrule*[lab=Struct-equiv]
{ P \scong Q }
{ \quotep{P} \nameeq \quotep{Q} }
\end{mathpar}

The astute reader will have noticed that the mutual recursion of names
and processes imposes a mutual recursion on alpha-equivalence and
structural equivalence via name-equivalence. Fortunately, all of this
works out pleasantly and we may calculate in the natural way, free of
concern. The reader interested in the details is referred to the
appendix \ref{appendix:rho_details}.

\subsection{Substitution}

We use $\Proc$ for the set of processes, $\QProc$ for the set of
names, and $\id{\{}\vec{y} / \vec{x} \id{\}}$ to denote partial maps,
$s : \QProc \rightarrow \QProc$. A map, $s$ lifts, uniquely, to a map
on process terms, $\widehat{s} : \Proc \rightarrow \Proc$ by the
following equations.

\begin{mathpar}
  (0) \psubstp{Q}{P} := 0 \\
  (R \juxtap S) \psubstp{Q}{P}
  :=    
  (R)\psubstp{Q}{P} \juxtap (S) \psubstp{Q}{P} \\
  (x?(y).R) \psubstp{Q}{P}    
  :=    
  (x)\substp{Q}{P} (z)\concat( (R \psubstn{z}{y}) \psubstp{Q}{P} ) \\
  (\lift{x}{R}) \psubstp{Q}{P}  
  :=
  \lift{(x)\substp{Q}{P}}{ R \psubstp{Q}{P} } \\
%   (\dropn{x})  \psubstp{Q}{P}       
%   := 
%   \left\{ 
%     \begin{array}{ccc} 
%       \dropn{\quotep{Q}} & & x \nameeq \quotep{P} \\
%       \dropn{x} & & otherwise \\
%     \end{array}
%   \right. 
  (\dropn{x})  \psubstp{Q}{P}       
  := 
  \left\{ 
    \begin{array}{ccc} 
      Q & & x \nameeq \quotep{P} \\
      \dropn{x} & & otherwise \\
    \end{array}
  \right.
\end{mathpar}
 

where

\begin{eqnarray}
  (x)\id{\{} \lpquote Q \rpquote / \lpquote P \rpquote \id{\}}            = 
  \left\{ 
    \begin{array}{ccc}
      \lpquote Q \rpquote & & x \nameeq \lpquote P \rpquote \\
      x & & otherwise \\
    \end{array}
  \right. \nonumber
\end{eqnarray}

and $z$ is chosen distinct from $\quotep{P}$, $\quotep{Q}$, the free
names in $Q$, and all the names in $R$. Our $\alpha$-equivalence will
be built in the standard way from this substitution.

\begin{remark}\label{rem:no_self_referential_names}
  One consequence of these definitions is that $\forall P. \quotep{P}
  \not\in \freenames{P}$.
\end{remark}

\subsection{ Dynamic quote: an example }

Anticipating something of what's to come, consider applying the
substitution, $\widehat{\id{\{}u / z \id{\}}}$, to the following pair
of processes, $\lift{w}{y!(z)}$ and $w[ \lpquote y!(z) \rpquote ]$.

\begin{eqnarray}
	\lift{w}{y!(z)}\widehat{\id{\{}u / z \id{\}}}
		& = &
		\lift{w}{y!(u)} \nonumber\\
	w[ \lpquote y!(z) \rpquote ] \widehat{ \id{\{}u / z \id{\}} }
		& = &
		w[ \lpquote y!(z) \rpquote ] \nonumber
\end{eqnarray}

Because the body of the process between quotes is impervious to
substitution, we get radically different answers. In fact, by
examining the first process in an input context,
e.g. $x?(z).\lift{w}{y!(z)}$, we see that the process under the lift
operator may be shaped by prefixed inputs binding a name inside it. In
this sense, the lift operator will be seen as a way to dynamically
construct processes before reifying them as names.

Finally equipped with these standard features we can present the
dynamics of the calculus.

\subsubsection{Operational semantics} 

Finally, we introduce the computational dynamics. What marks these
algebras as distinct from other more traditionally studied algebraic
structures, e.g. vector spaces or polynomial rings, is the manner in
which dynamics is captured. In traditional structures, dynamics is typically
expressed through morphisms between such structures, as in linear maps
between vector spaces or morphisms between rings. In algebras
associated with the semantics of computation, the dynamics is
expressed as part of the algebraic structure itself, through a
reduction reduction relation typically denoted by $\red$. Below, we
give a recursive presentation of this relation for the calculus used
in the encoding.

$\red \subseteq \pi \times \pi$
$\red : \pi \to \mathcal{P}(\pi)$

\begin{mathpar}
  \inferrule* [lab=Comm] { \textsf{match}( x_{src}, x_{trgt} ) } { x_{trgt}?(y)P \; | \; x_{src}!\langle {Q} \rangle \red P\{\quotep{Q}/y}\} }
  \and \\
  \inferrule* [lab=Par] {{P} \red {P}'} {{{P} | {Q}} \red {{P}' | {Q}}}
  \and
  \inferrule* [lab=Equiv]{{{P} \scong {P}'} \andalso {{P}' \red {Q}'} \andalso {{Q}' \scong {Q}}}{{P} \red {Q}}
\end{mathpar}

\begin{eqnarray*}
  match_{\equiv} (\quotep{P},\quotep{Q}) & := & P \equiv Q \\
  match_{\dagger}(\quotep{P},\quotep{Q}) & := & \forall R. P|Q \red^{*} R => R \red^{*} 0 \\
  match_{K}(\quotep{P},\quotep{Q}) & := & K \mbox{ for some context } K
\end{eqnarray*}

$u?(x)P | u!\langle Q \rangle \red P\{\quotep{Q}/x\}$

%We write $\wred$ for $\red^*$, and $P\red$ if $\exists Q $ such that $ P \red Q$.
We write $P\red$ if $\exists Q $ such that $ P \red Q$ and $P\not\red$, otherwise.

\section{Replication}

As mentioned before, it is known that replication (and hence
recursion) can be implemented in a higher-order process algebra
\cite{SangiorgiWalker}. As our first example of calculation with the
machinery thus far presented we give the construction explicitly in
the {\rhoc}.

\begin{eqnarray}
	D_{x} & := & \prefix{x}{y}{(\binpar{\outputp{x}{y}}{@{y}})} \nonumber\\
	\bangp_{x}{P} & := & \binpar{{x}!\langle{\binpar{D_{x}}{P}}\rangle}{D_{x}} \nonumber
\end{eqnarray}

\begin{eqnarray}
	\bangp_{x}{P} & & \nonumber\\
	=
	& {x}!\langle{(\prefix{x}{y}{(\outputp{x}{y} | @{y})) | P}}\rangle 
	      | \prefix{x}{y}{(\outputp{x}{y} | @{y})} & \nonumber\\
	\red
	& (\outputp{x}{y} | @{y})\substn{\quotep{(\prefix{x}{y}{(@{y} | \outputp{x}{y})) | P}}}{y} & \nonumber\\
	=
	& \outputp{x}{\quotep{(\prefix{x}{y}{(\outputp{x}{y} | @{y})) | P}}}
	  | {(\prefix{x}{y}{(\outputp{x}{y} | @{y})) | P}} & \nonumber\\
	\red
	& \ldots & \nonumber\\
	\red^*
	& P | P | \ldots & \nonumber
\end{eqnarray}

Of course, this encoding, as an implementation, runs away, unfolding
$\bangp{P}$ eagerly. A lazier and more implementable replication
operator, restricted to input-guarded processes, may be obtained as follows.

\begin{eqnarray}
\bangp{\prefix{u}{v}{P}} 
	:= 
	\binpar{\lift{x}{\prefix{u}{v}{(\binpar{D(x)}{P})}}}{D(x)} \nonumber
\end{eqnarray}

\begin{remark}
  Note that the lazier definition still does not deal with summation
  or mixed summation (i.e. sums over input and output). The reader is
  invited to construct definitions of replication that deal with these
  features. 

  Further, the definitions are parameterized in a name, $x$. Can you,
  gentle reader, make a definition that eliminates this parameter and
  guarantees no accidental interaction between the replication
  machinery and the process being replicated -- i.e. no accidental
  sharing of names used by the process to get its work done and the
  name(s) used by the replication to effect copying. This latter
  revision of the definition of replication is crucial to obtaining
  the expected identity $!!P \sim !P$.
\end{remark}

\begin{remark}\label{rem:paradoxical_combinator}
  The reader familiar with the lambda calculus will have noticed the
  similarity between $D$ and the paradoxical combinator.

  [Ed. note: the existence of this seems to suggest we have to be more
  restrictive on the set of processes and names we admit if we are to
  support no-cloning.]
\end{remark}

\subsubsection{Bisimulation}

The computational dynamics gives rise to another kind of equivalence,
the equivalence of computational behavior. As previously mentioned
this is typically captured \emph{via} some form of bisimulation.

% The notion we use in this paper is weak barbed bisimulation
% \cite{milner91polyadicpi}.

The notion we use in this paper is derived from weak barbed
bisimulation \cite{milner91polyadicpi}. 

\begin{definition}
An \emph{observation relation}, $\downarrow_{\mathcal N}$, over a set
of names, $\mathcal N$, is the smallest relation satisfying the rules
below.

\infrule[Out-barb]{y \in {\mathcal N}, \; x \nameeq y}
		  {\outputp{x}{v} \downarrow_{\mathcal N} x}
\infrule[Par-barb]{\mbox{$P\downarrow_{\mathcal N} x$ or $Q\downarrow_{\mathcal N} x$}}
		  {\binpar{P}{Q} \downarrow_{\mathcal N} x}

We write $P \Downarrow_{\mathcal N} x$ if there is $Q$ such that 
$P \wred Q$ and $Q \downarrow_{\mathcal N} x$.
\end{definition}

\begin{definition}
%\label{def.bbisim}
An  ${\mathcal N}$-\emph{barbed bisimulation} over a set of names, ${\mathcal N}$, is a symmetric binary relation 
${\mathcal S}_{\mathcal N}$ between agents such that $P\rel{S}_{\mathcal N}Q$ implies:
\begin{enumerate}
\item If $P \red P'$ then $Q \wred Q'$ and $P'\rel{S}_{\mathcal N} Q'$.
\item If $P\downarrow_{\mathcal N} x$, then $Q\Downarrow_{\mathcal N} x$.
\end{enumerate}
$P$ is ${\mathcal N}$-barbed bisimilar to $Q$, written
$P \wbbisim_{\mathcal N} Q$, if $P \rel{S}_{\mathcal N} Q$ for some ${\mathcal N}$-barbed bisimulation ${\mathcal S}_{\mathcal N}$.
\end{definition}

$\mathcal{R} \subseteq \pi \times \pi$

$P \mathcal{R} Q => \forall P'. P \red P' \Rightarrow \exists Q'. Q \red Q', P' \mathcal{R} Q'$

$P \vdash x \Rightarrow Q \vdash x$

\begin{mathpar}
  \inferrule*[lab=Out-barb]{x \nameeq y}{{y}!\langle{Q}\rangle \vdash x}
  \and
  \inferrule*[lab=Par-barb]{\mbox{$P\vdash x$ or $Q\vdash x$}}{\binpar{P}{Q} \vdash x}
\end{mathpar}

\subsubsection{Contexts}

One of the principle advantages of computational calculi like the
$\pi$-calculus is a well-defined notion of context,
contextual-equivalence and a correlation between
contextual-equivalence and notions of bisimulation. The notion of
context allows the decomposition of a process into (sub-)process and
its syntactic environment, its context. Thus, a context may be
thought of as a process with a ``hole'' (written $\Box$) in it. The
application of a context $M$ to a process $P$, written $M[P]$, is
tantamount to filling the hole in $M$ with $P$. In this paper we do
not need the full weight of this theory, but do make use of the notion
of context in the proof the main theorem. 

\begin{mathpar}
  \inferrule* [lab=summation] {} {{M_{M},M_{N}} \bc \Box \;|\; x.M_{A} \;|\; M_{M}+M_{N}}
  \and
  \inferrule* [lab=agent] {} {{M_{A}} \bc (\vec{x})M_{P} \;| \; \clift{P_0,\ldots,M_{P},\ldots,P_N}}
  \and \\
  \inferrule* [lab=process] {} {{M_{P}} \bc M_{N} \;| \;P|M_{P} }
\end{mathpar} 

\begin{mathpar}
  \inferrule* [lab=sychronization] {} {M_{N} \bc \Box \;|\; x?M_{F} \;|\; x!M_{C}}
  \and
  \inferrule* [lab=abstraction] {} {{M_{F}} \bc (x)M_{P} }
  \and
  \inferrule* [lab=concretion] {} {{M_{C}} \bc \langle M_{P} \rangle }
  \and \\
  \inferrule* [lab=process] {} {{M_{P}} \bc M_{N} \;| \;P|M_{P} }
\end{mathpar}

\begin{definition}[contextual application] Given a context $M$, and
  process $P$, we define the \emph{contextual application}, $M[P] :=
  M\{P/\Box\}$. That is, the contextual application of M to P is the
  substitution of $P$ for $\Box$ in $M$.
\end{definition}

$\meaningof{-} : L \to \mathcal{P}(\pi)$

\begin{mathpar}
  \inferrule* [lab=collection] {} {\meaningof{true} = \pi, \and \meaningof{~E} = \pi \setminus \meaningof{E}, \and \meaningof{E_{1} \& E_{2}} = \meaningof{E_{1}} \cap \meaningof{E_{2}}}
\end{mathpar}

\begin{mathpar}
  \inferrule* [lab=structure] {} {\meaningof{0} = \{ P \in \pi | P \equiv 0 \}, \and \\ \meaningof{E_1 | E_2} = \{ P \in \pi | P \equiv P_{1} | P_{2}, P_{1} \in \meaningof{E_{1}}, P_{2} \in \meaningof{E_2}\} }
\end{mathpar}

\begin{mathpar}
 \inferrule* [lab=behavior] {} {\meaningof{\langle a?b \rangle E} = \{ P \in \pi | P \equiv Q | u?(y)P', \\ \and \\\\ \and \\ \;\;\; u \in \meaningof{a}, \forall z.P'\{z/y\} \in \meaningof{E\{z/b\}}\}, \and \\ \meaningof{a!E} = \{ P \in \pi | P \equiv Q | x!\langle P' \rangle, x \in \meaningof{a} P' \in \meaningof{E}\} }
\end{mathpar}

\begin{mathpar}
 \inferrule* [lab=nominal] {} {\meaningof{\quotep{E}} = \{ \quotep{P} \in \quotep{\pi} | P \in \meaningof{E} \}, \and \meaningof{\quotep{P}} = \{ \quotep{Q} \in \quotep{\pi} | P \equiv Q \} \and \\ \meaningof{@\quotep{E}} = \{ P \in \pi | P \equiv @x, x \in \meaningof{E} \}}
\end{mathpar}

\begin{eqnarray*}
  \\
  \meaningof{-} : TS \to ST
\end{eqnarray*}

\begin{eqnarray*}
  \\
  L : TS \to ST
\end{eqnarray*}

\begin{eqnarray*}
  \\
  P \models E \iff P \in \meaningof{E}
\end{eqnarray*}

\begin{eqnarray*}
  P \approx_{L} Q \iff \forall E \in L. P \models E \iff Q \models E
\end{eqnarray*}

\begin{eqnarray*}
  P \approx_{K} Q
\end{eqnarray*}

\begin{eqnarray*}
  P \approx Q
\end{eqnarray*}

$\approx_{K} = \approx = \approx_{L}$

\subsubsection{Contextual duality}

Note that contexts extend the quotation operation to a family of
operations from processes to names. Given a context, $M$, we can
define a \emph{nominal context}, $\quotep{M}$ by $\quotep{M}[P] :=
\quotep{M[P]}$. To foreshadow what is to come we observe that these
operations enjoy a duality with processes very much like the duality
between vectors and maps from vectors to scalars.

Further, because the calculus is essentially higher-order, we have a
correspondence between contexts and processes. More specifically,
given a name $x$ and a context $M$ we can construct $M^{*}_{x}$ such
that 

\begin{mathpar}
  M^{*}_{x} | \lift{x}{P} \red M[P]
\end{mathpar}

namely,

\begin{mathpar}
  M^{*}_{x} := x?(u).M[\dropn{u}]
\end{mathpar}

The dependence of $M^{*}_{x}$ on a name makes it an abstraction, 

\begin{mathpar}
  M^{*} := (x)x?(u).M[\dropn{u}]
\end{mathpar}

\subsection{Additional notation}

It will sometimes be convenient to denote the process a name
quotes. We already have the notation $x = \quotep{P}$, but it will be
convenient to introduce an alternate notation, $\procn{x}$, when we
want to emphasize the connection to the use of the name. Note that, by
virtue of name equivalence, $\quotep{\procn{x}} \nameeq x$; so, the
notation is consistent with previous definitions.

Further, because names have structure it is possible to effect
substitutions on the basis of that structure. This means we need to
upgrade our notation for substitutions, which we accomplish by
adapting comprehension notation. Thus,

\begin{mathpar}
  P\{ y / x : x \in S \}
\end{mathpar}

is interpreted to mean the process derived from P by replacing (in a
capture-avoiding manner) each occurrence of $x$ in $S$ by $y$. For example,

\begin{mathpar}
  P\{ \quotep{\procn{x}|\procn{x}} / x : x \in \freenames{P} \}
\end{mathpar}

will replace each (occurrence) of a free name $x$ in $P$ by
$\quotep{\procn{x}|\procn{x}}$.

Also, we will avail ourselves of the notation $x^{L}$ and $x^{R}$ to
denote injections of a name into disjoint copies of the name
space. There are numerous ways to accomplish this. One example can be
found in \cite{MeredithR05}. This notation overloads to vectors of
names: $\vec{x}^{\pi} := (x_{i}^{\pi} \; : \; 0 \leq i < |\vec{x}| )$ where $\pi \in \{L,R\}$.

We also use $P^{\Box} := P|\Box$.

In \cite{MeredithR05} an interpretation of the new operator is
given. It turns out that there are several possible interpretations
all enjoying the requisite algebraic properties of the operator (see
\cite{milner91polyadicpi}). We will therefore make liberal use of
$(\nu\; \vec{x})P$.

% subsection the_syntax_and_semantics_of_the_notation_system (end)   

\input{qm2pi.qmops} 

\input{qm2pi.sterngerlach} 

\input{qm2pi.metric} 

% section concurrent_process_calculi (end)

%\input{qm2pi.proofsketch}

% section proof sketch (end)

%\input{qm2pi.slviaknots} 

% section spatial logic via knots (end)

\input{qm2pi.conclusion}

% section conclusion (end)

%\input{qm2pi.dtcodes} 

% section wiring algorithm (end)

\input{qm2pi.ack} 

% section acknowledgments (end)

\newpage


\bibliographystyle{plain}   
\bibliography{../../biblios/main.bib}

\input{qm2pi.rhodetails}

\end{document}

 

%\documentclass[12pt]{llncs}
%\documentclass{jktr}

\usepackage[pdftex]{hyperref}                   
\usepackage {listings}
\usepackage {mathpartir}
\usepackage{bcprules}
%\usepackage{listings}
                       
\usepackage{graphicx} 
%\usepackage[margins=2.5cm,nohead,nofoot]{geometry}
%\usepackage{geometry}
\usepackage{amsfonts}
\usepackage{amstext}
\usepackage{latexsym}
\usepackage{amssymb}
\usepackage{color}


%\include{myPreamble}
\include{qm2pi.local} 

%\ifpdf
%\usepackage[pdftex]{graphicx}
%\else
%\usepackage{graphicx}
%\fi

 % \ifpdf
%  \usepackage{pdfsync}
%  \if


%\title{Brief Article}
%\author{David F. Snyder}
%\author{L.G. Meredith}

%\address{Dept. of Math., Texas State University--San Marcos, San Marcos, TX 78666}
       
\pagestyle{empty}


\begin{document}

\lstset{language=[Objective]Caml,frame=shadowbox}

\input{qm2pi.front}

% section front matter (end)

\input{qm2pi.intro} 
 
% section introduction (end)

% \input{qm2pi.knotations} 

% section notation (end)

\input{qm2pi.process.calculi} 

% section concurrent_process_calculi_and_spatial_logics_ (end)
    
%\input{qm2pi.knots2pi} 

%\input{qm2pi.trefoil} 

%\input{qm2pi.mainthm} 

% subsection basic_interpretation (end)

%\input{qm2pi.rho.presentation} 
\subsection{The syntax and semantics of the notation system}\label{sub:the_syntax_and_semantics_of_the_notation_system} % (fold)

We now summarize a technical presentation of the calculus that
embodies our theory of dynamics. The typical presentation of such a
calculus follows the style of giving generators and relations on
them. The grammar, below, describing term constructors, freely
generates the set of processes, $\Proc$. This set is then quotiented
by a relation known as structural congruence and it is over this set
that the notion of dynamics is expressed. This presentation is
essentially that of \cite{MeredithR05} with the addition of
polyadicity and summation. For readability we have relegated some of
the technical subtleties to an appendix.

\subsubsection{Process grammar}\label{subsub:process_grammar}

\begin{mathpar}
  \inferrule* [lab=synchronization] {} {{M} \bc \pzero \;|\; x?F \;|\; x!C }
  \and
  \inferrule* [lab=abstraction] {} {{F} \bc (x)P}
  \and
  \inferrule* [lab=concretion] {} {{C} \bc \langle Q \rangle}
  \and
  \inferrule* [lab=process] {} {{P,Q} \bc M \;| \;P|Q \;|\; @{x}}
  \and
  \inferrule* [lab=name] {} {{x} \bc \quotep{P}}
\end{mathpar} 

Note that $\vec{x}$ (resp. $\vec{P}$) denotes a vector of names
(resp. processes) of length $|\vec{x}|$ (resp. $|\vec{P}|$). We adopt
the following useful abbreviations.

\begin{mathpar}
   x?(\vec{y}).P := x.(\vec{y})P \and  x\clift{\vec{P}} := x.\clift{\vec{P}}
   \and x!(y) := \lift{x}{\dropn{y}}
   \and \Pi_{i=0}^{n-1}P_i := P_0 | \ldots | P_{n-1}
\end{mathpar}

\subsubsection{Structural congruence}

\paragraph{Free and bound names and alpha-equivalence.} At the
core of structural equivalence is alpha-equivalence which identifies
process that are the same up to a change of variable. Formally, we
recognize the distinction between free and bound names. The free names
of a process, $\freenames{P}$, may be calculated recursively as
follows:

\begin{mathpar}
\freenames{\pzero} := \emptyset
  \and \\
  \freenames{x?(y).P} := \{ x \} \cup (\freenames{P} \setminus \{ y \})
  \and 
  \freenames{x!\langle P \rangle} := \{ x \} \cup \{ P \} 
  \and \\
  \freenames{P|Q} := \freenames{P} \cup \freenames{Q}
  \and \\
  \freenames{@{x}} := \{ x \}
\end{mathpar}

$\pi$
$\quotep{\pi}$

$\freenames{-} : \pi \to \mathcal{P}(\quotep{\pi})$

\begin{eqnarray*}
  \freenames{\pzero} & := & \emptyset \\
  \freenames{x?(y).P} & := & \{ x \} \cup (\freenames{P} \setminus \{ y \}) \\
  \freenames{x!\langle P \rangle} & := & \{ x \} \cup \{ P \} \\
  \freenames{P|Q} & := & \freenames{P} \cup \freenames{Q} \\
  \freenames{\dropn{x}} & := & \{ x \}
\end{eqnarray*}

The bound names of a process, $\boundnames{P}$, are those names occurring in $P$
that are not free. For example, in $x?(y).0$, the name $x$ is free, while $y$ is bound.

\begin{mathpar}
  \inferrule* [lab=monoidal-laws] {} { P|Q \equiv Q|P \and P|0 \equiv P \and P|(Q|R) \equiv (P|Q)|R }
\end{mathpar}

\begin{mathpar}
  \inferrule* [lab=alpha-equivalence] {} { (x)P \equiv (y)P\{y/x\} \and y \not\in \freenames{P} }
\end{mathpar}

\begin{definition}
Then two processes, $P,Q$, are alpha-equivalent if $P = Q\{\vec{y}/\vec{x}\}$ for
some $\vec{x} \in \boundnames{Q},\vec{y} \in \boundnames{P}$, where $Q\{\vec{y}/\vec{x}\}$
denotes the capture-avoiding substitution of $\vec{y}$ for $\vec{x}$ in $Q$.
\end{definition}

\begin{definition}
  The {\em structural congruence} \cite{SangiorgiWalker} , $\equiv$,
  between processes is the least congruence containing
  alpha-equivalence, satisfying the abelian monoid laws
  (associativity, commutativity and $\pzero$ as identity) for parallel
  composition $|$ and for summation $+$.
\end{definition}

\subsection{Name equivalence}

We take name equivalence, written $\nameeq$, to be the smallest
equivalence relation generated by the following rules.

\begin{mathpar}
\inferrule*[lab=Quote-drop]
{ }
{ \quotep{@{x}} \nameeq x }

\inferrule*[lab=Struct-equiv]
{ P \scong Q }
{ \quotep{P} \nameeq \quotep{Q} }
\end{mathpar}

The astute reader will have noticed that the mutual recursion of names
and processes imposes a mutual recursion on alpha-equivalence and
structural equivalence via name-equivalence. Fortunately, all of this
works out pleasantly and we may calculate in the natural way, free of
concern. The reader interested in the details is referred to the
appendix \ref{appendix:rho_details}.

\subsection{Substitution}

We use $\Proc$ for the set of processes, $\QProc$ for the set of
names, and $\id{\{}\vec{y} / \vec{x} \id{\}}$ to denote partial maps,
$s : \QProc \rightarrow \QProc$. A map, $s$ lifts, uniquely, to a map
on process terms, $\widehat{s} : \Proc \rightarrow \Proc$ by the
following equations.

\begin{mathpar}
  (0) \psubstp{Q}{P} := 0 \\
  (R \juxtap S) \psubstp{Q}{P}
  :=    
  (R)\psubstp{Q}{P} \juxtap (S) \psubstp{Q}{P} \\
  (x?(y).R) \psubstp{Q}{P}    
  :=    
  (x)\substp{Q}{P} (z)\concat( (R \psubstn{z}{y}) \psubstp{Q}{P} ) \\
  (\lift{x}{R}) \psubstp{Q}{P}  
  :=
  \lift{(x)\substp{Q}{P}}{ R \psubstp{Q}{P} } \\
%   (\dropn{x})  \psubstp{Q}{P}       
%   := 
%   \left\{ 
%     \begin{array}{ccc} 
%       \dropn{\quotep{Q}} & & x \nameeq \quotep{P} \\
%       \dropn{x} & & otherwise \\
%     \end{array}
%   \right. 
  (\dropn{x})  \psubstp{Q}{P}       
  := 
  \left\{ 
    \begin{array}{ccc} 
      Q & & x \nameeq \quotep{P} \\
      \dropn{x} & & otherwise \\
    \end{array}
  \right.
\end{mathpar}
 

where

\begin{eqnarray}
  (x)\id{\{} \lpquote Q \rpquote / \lpquote P \rpquote \id{\}}            = 
  \left\{ 
    \begin{array}{ccc}
      \lpquote Q \rpquote & & x \nameeq \lpquote P \rpquote \\
      x & & otherwise \\
    \end{array}
  \right. \nonumber
\end{eqnarray}

and $z$ is chosen distinct from $\quotep{P}$, $\quotep{Q}$, the free
names in $Q$, and all the names in $R$. Our $\alpha$-equivalence will
be built in the standard way from this substitution.

\begin{remark}\label{rem:no_self_referential_names}
  One consequence of these definitions is that $\forall P. \quotep{P}
  \not\in \freenames{P}$.
\end{remark}

\subsection{ Dynamic quote: an example }

Anticipating something of what's to come, consider applying the
substitution, $\widehat{\id{\{}u / z \id{\}}}$, to the following pair
of processes, $\lift{w}{y!(z)}$ and $w[ \lpquote y!(z) \rpquote ]$.

\begin{eqnarray}
	\lift{w}{y!(z)}\widehat{\id{\{}u / z \id{\}}}
		& = &
		\lift{w}{y!(u)} \nonumber\\
	w[ \lpquote y!(z) \rpquote ] \widehat{ \id{\{}u / z \id{\}} }
		& = &
		w[ \lpquote y!(z) \rpquote ] \nonumber
\end{eqnarray}

Because the body of the process between quotes is impervious to
substitution, we get radically different answers. In fact, by
examining the first process in an input context,
e.g. $x?(z).\lift{w}{y!(z)}$, we see that the process under the lift
operator may be shaped by prefixed inputs binding a name inside it. In
this sense, the lift operator will be seen as a way to dynamically
construct processes before reifying them as names.

Finally equipped with these standard features we can present the
dynamics of the calculus.

\subsubsection{Operational semantics} 

Finally, we introduce the computational dynamics. What marks these
algebras as distinct from other more traditionally studied algebraic
structures, e.g. vector spaces or polynomial rings, is the manner in
which dynamics is captured. In traditional structures, dynamics is typically
expressed through morphisms between such structures, as in linear maps
between vector spaces or morphisms between rings. In algebras
associated with the semantics of computation, the dynamics is
expressed as part of the algebraic structure itself, through a
reduction reduction relation typically denoted by $\red$. Below, we
give a recursive presentation of this relation for the calculus used
in the encoding.

$\red \subseteq \pi \times \pi$
$\red : \pi \to \mathcal{P}(\pi)$

\begin{mathpar}
  \inferrule* [lab=Comm] { \textsf{match}( x_{src}, x_{trgt} ) } { x_{trgt}?(y)P \; | \; x_{src}!\langle {Q} \rangle \red P\{\quotep{Q}/y}\} }
  \and \\
  \inferrule* [lab=Par] {{P} \red {P}'} {{{P} | {Q}} \red {{P}' | {Q}}}
  \and
  \inferrule* [lab=Equiv]{{{P} \scong {P}'} \andalso {{P}' \red {Q}'} \andalso {{Q}' \scong {Q}}}{{P} \red {Q}}
\end{mathpar}

\begin{eqnarray*}
  match_{\equiv} (\quotep{P},\quotep{Q}) & := & P \equiv Q \\
  match_{\dagger}(\quotep{P},\quotep{Q}) & := & \forall R. P|Q \red^{*} R => R \red^{*} 0 \\
  match_{K}(\quotep{P},\quotep{Q}) & := & K \mbox{ for some context } K
\end{eqnarray*}

$u?(x)P | u!\langle Q \rangle \red P\{\quotep{Q}/x\}$

%We write $\wred$ for $\red^*$, and $P\red$ if $\exists Q $ such that $ P \red Q$.
We write $P\red$ if $\exists Q $ such that $ P \red Q$ and $P\not\red$, otherwise.

\section{Replication}

As mentioned before, it is known that replication (and hence
recursion) can be implemented in a higher-order process algebra
\cite{SangiorgiWalker}. As our first example of calculation with the
machinery thus far presented we give the construction explicitly in
the {\rhoc}.

\begin{eqnarray}
	D_{x} & := & \prefix{x}{y}{(\binpar{\outputp{x}{y}}{@{y}})} \nonumber\\
	\bangp_{x}{P} & := & \binpar{{x}!\langle{\binpar{D_{x}}{P}}\rangle}{D_{x}} \nonumber
\end{eqnarray}

\begin{eqnarray}
	\bangp_{x}{P} & & \nonumber\\
	=
	& {x}!\langle{(\prefix{x}{y}{(\outputp{x}{y} | @{y})) | P}}\rangle 
	      | \prefix{x}{y}{(\outputp{x}{y} | @{y})} & \nonumber\\
	\red
	& (\outputp{x}{y} | @{y})\substn{\quotep{(\prefix{x}{y}{(@{y} | \outputp{x}{y})) | P}}}{y} & \nonumber\\
	=
	& \outputp{x}{\quotep{(\prefix{x}{y}{(\outputp{x}{y} | @{y})) | P}}}
	  | {(\prefix{x}{y}{(\outputp{x}{y} | @{y})) | P}} & \nonumber\\
	\red
	& \ldots & \nonumber\\
	\red^*
	& P | P | \ldots & \nonumber
\end{eqnarray}

Of course, this encoding, as an implementation, runs away, unfolding
$\bangp{P}$ eagerly. A lazier and more implementable replication
operator, restricted to input-guarded processes, may be obtained as follows.

\begin{eqnarray}
\bangp{\prefix{u}{v}{P}} 
	:= 
	\binpar{\lift{x}{\prefix{u}{v}{(\binpar{D(x)}{P})}}}{D(x)} \nonumber
\end{eqnarray}

\begin{remark}
  Note that the lazier definition still does not deal with summation
  or mixed summation (i.e. sums over input and output). The reader is
  invited to construct definitions of replication that deal with these
  features. 

  Further, the definitions are parameterized in a name, $x$. Can you,
  gentle reader, make a definition that eliminates this parameter and
  guarantees no accidental interaction between the replication
  machinery and the process being replicated -- i.e. no accidental
  sharing of names used by the process to get its work done and the
  name(s) used by the replication to effect copying. This latter
  revision of the definition of replication is crucial to obtaining
  the expected identity $!!P \sim !P$.
\end{remark}

\begin{remark}\label{rem:paradoxical_combinator}
  The reader familiar with the lambda calculus will have noticed the
  similarity between $D$ and the paradoxical combinator.

  [Ed. note: the existence of this seems to suggest we have to be more
  restrictive on the set of processes and names we admit if we are to
  support no-cloning.]
\end{remark}

\subsubsection{Bisimulation}

The computational dynamics gives rise to another kind of equivalence,
the equivalence of computational behavior. As previously mentioned
this is typically captured \emph{via} some form of bisimulation.

% The notion we use in this paper is weak barbed bisimulation
% \cite{milner91polyadicpi}.

The notion we use in this paper is derived from weak barbed
bisimulation \cite{milner91polyadicpi}. 

\begin{definition}
An \emph{observation relation}, $\downarrow_{\mathcal N}$, over a set
of names, $\mathcal N$, is the smallest relation satisfying the rules
below.

\infrule[Out-barb]{y \in {\mathcal N}, \; x \nameeq y}
		  {\outputp{x}{v} \downarrow_{\mathcal N} x}
\infrule[Par-barb]{\mbox{$P\downarrow_{\mathcal N} x$ or $Q\downarrow_{\mathcal N} x$}}
		  {\binpar{P}{Q} \downarrow_{\mathcal N} x}

We write $P \Downarrow_{\mathcal N} x$ if there is $Q$ such that 
$P \wred Q$ and $Q \downarrow_{\mathcal N} x$.
\end{definition}

\begin{definition}
%\label{def.bbisim}
An  ${\mathcal N}$-\emph{barbed bisimulation} over a set of names, ${\mathcal N}$, is a symmetric binary relation 
${\mathcal S}_{\mathcal N}$ between agents such that $P\rel{S}_{\mathcal N}Q$ implies:
\begin{enumerate}
\item If $P \red P'$ then $Q \wred Q'$ and $P'\rel{S}_{\mathcal N} Q'$.
\item If $P\downarrow_{\mathcal N} x$, then $Q\Downarrow_{\mathcal N} x$.
\end{enumerate}
$P$ is ${\mathcal N}$-barbed bisimilar to $Q$, written
$P \wbbisim_{\mathcal N} Q$, if $P \rel{S}_{\mathcal N} Q$ for some ${\mathcal N}$-barbed bisimulation ${\mathcal S}_{\mathcal N}$.
\end{definition}

$\mathcal{R} \subseteq \pi \times \pi$

$P \mathcal{R} Q => \forall P'. P \red P' \Rightarrow \exists Q'. Q \red Q', P' \mathcal{R} Q'$

$P \vdash x \Rightarrow Q \vdash x$

\begin{mathpar}
  \inferrule*[lab=Out-barb]{x \nameeq y}{{y}!\langle{Q}\rangle \vdash x}
  \and
  \inferrule*[lab=Par-barb]{\mbox{$P\vdash x$ or $Q\vdash x$}}{\binpar{P}{Q} \vdash x}
\end{mathpar}

\subsubsection{Contexts}

One of the principle advantages of computational calculi like the
$\pi$-calculus is a well-defined notion of context,
contextual-equivalence and a correlation between
contextual-equivalence and notions of bisimulation. The notion of
context allows the decomposition of a process into (sub-)process and
its syntactic environment, its context. Thus, a context may be
thought of as a process with a ``hole'' (written $\Box$) in it. The
application of a context $M$ to a process $P$, written $M[P]$, is
tantamount to filling the hole in $M$ with $P$. In this paper we do
not need the full weight of this theory, but do make use of the notion
of context in the proof the main theorem. 

\begin{mathpar}
  \inferrule* [lab=summation] {} {{M_{M},M_{N}} \bc \Box \;|\; x.M_{A} \;|\; M_{M}+M_{N}}
  \and
  \inferrule* [lab=agent] {} {{M_{A}} \bc (\vec{x})M_{P} \;| \; \clift{P_0,\ldots,M_{P},\ldots,P_N}}
  \and \\
  \inferrule* [lab=process] {} {{M_{P}} \bc M_{N} \;| \;P|M_{P} }
\end{mathpar} 

\begin{mathpar}
  \inferrule* [lab=sychronization] {} {M_{N} \bc \Box \;|\; x?M_{F} \;|\; x!M_{C}}
  \and
  \inferrule* [lab=abstraction] {} {{M_{F}} \bc (x)M_{P} }
  \and
  \inferrule* [lab=concretion] {} {{M_{C}} \bc \langle M_{P} \rangle }
  \and \\
  \inferrule* [lab=process] {} {{M_{P}} \bc M_{N} \;| \;P|M_{P} }
\end{mathpar}

\begin{definition}[contextual application] Given a context $M$, and
  process $P$, we define the \emph{contextual application}, $M[P] :=
  M\{P/\Box\}$. That is, the contextual application of M to P is the
  substitution of $P$ for $\Box$ in $M$.
\end{definition}

$\meaningof{-} : L \to \mathcal{P}(\pi)$

\begin{mathpar}
  \inferrule* [lab=collection] {} {\meaningof{true} = \pi, \and \meaningof{~E} = \pi \setminus \meaningof{E}, \and \meaningof{E_{1} \& E_{2}} = \meaningof{E_{1}} \cap \meaningof{E_{2}}}
\end{mathpar}

\begin{mathpar}
  \inferrule* [lab=structure] {} {\meaningof{0} = \{ P \in \pi | P \equiv 0 \}, \and \\ \meaningof{E_1 | E_2} = \{ P \in \pi | P \equiv P_{1} | P_{2}, P_{1} \in \meaningof{E_{1}}, P_{2} \in \meaningof{E_2}\} }
\end{mathpar}

\begin{mathpar}
 \inferrule* [lab=behavior] {} {\meaningof{\langle a?b \rangle E} = \{ P \in \pi | P \equiv Q | u?(y)P', \\ \and \\\\ \and \\ \;\;\; u \in \meaningof{a}, \forall z.P'\{z/y\} \in \meaningof{E\{z/b\}}\}, \and \\ \meaningof{a!E} = \{ P \in \pi | P \equiv Q | x!\langle P' \rangle, x \in \meaningof{a} P' \in \meaningof{E}\} }
\end{mathpar}

\begin{mathpar}
 \inferrule* [lab=nominal] {} {\meaningof{\quotep{E}} = \{ \quotep{P} \in \quotep{\pi} | P \in \meaningof{E} \}, \and \meaningof{\quotep{P}} = \{ \quotep{Q} \in \quotep{\pi} | P \equiv Q \} \and \\ \meaningof{@\quotep{E}} = \{ P \in \pi | P \equiv @x, x \in \meaningof{E} \}}
\end{mathpar}

\begin{eqnarray*}
  \\
  \meaningof{-} : TS \to ST
\end{eqnarray*}

\begin{eqnarray*}
  \\
  L : TS \to ST
\end{eqnarray*}

\begin{eqnarray*}
  \\
  P \models E \iff P \in \meaningof{E}
\end{eqnarray*}

\begin{eqnarray*}
  P \approx_{L} Q \iff \forall E \in L. P \models E \iff Q \models E
\end{eqnarray*}

\begin{eqnarray*}
  P \approx_{K} Q
\end{eqnarray*}

\begin{eqnarray*}
  P \approx Q
\end{eqnarray*}

$\approx_{K} = \approx = \approx_{L}$

\subsubsection{Contextual duality}

Note that contexts extend the quotation operation to a family of
operations from processes to names. Given a context, $M$, we can
define a \emph{nominal context}, $\quotep{M}$ by $\quotep{M}[P] :=
\quotep{M[P]}$. To foreshadow what is to come we observe that these
operations enjoy a duality with processes very much like the duality
between vectors and maps from vectors to scalars.

Further, because the calculus is essentially higher-order, we have a
correspondence between contexts and processes. More specifically,
given a name $x$ and a context $M$ we can construct $M^{*}_{x}$ such
that 

\begin{mathpar}
  M^{*}_{x} | \lift{x}{P} \red M[P]
\end{mathpar}

namely,

\begin{mathpar}
  M^{*}_{x} := x?(u).M[\dropn{u}]
\end{mathpar}

The dependence of $M^{*}_{x}$ on a name makes it an abstraction, 

\begin{mathpar}
  M^{*} := (x)x?(u).M[\dropn{u}]
\end{mathpar}

\subsection{Additional notation}

It will sometimes be convenient to denote the process a name
quotes. We already have the notation $x = \quotep{P}$, but it will be
convenient to introduce an alternate notation, $\procn{x}$, when we
want to emphasize the connection to the use of the name. Note that, by
virtue of name equivalence, $\quotep{\procn{x}} \nameeq x$; so, the
notation is consistent with previous definitions.

Further, because names have structure it is possible to effect
substitutions on the basis of that structure. This means we need to
upgrade our notation for substitutions, which we accomplish by
adapting comprehension notation. Thus,

\begin{mathpar}
  P\{ y / x : x \in S \}
\end{mathpar}

is interpreted to mean the process derived from P by replacing (in a
capture-avoiding manner) each occurrence of $x$ in $S$ by $y$. For example,

\begin{mathpar}
  P\{ \quotep{\procn{x}|\procn{x}} / x : x \in \freenames{P} \}
\end{mathpar}

will replace each (occurrence) of a free name $x$ in $P$ by
$\quotep{\procn{x}|\procn{x}}$.

Also, we will avail ourselves of the notation $x^{L}$ and $x^{R}$ to
denote injections of a name into disjoint copies of the name
space. There are numerous ways to accomplish this. One example can be
found in \cite{MeredithR05}. This notation overloads to vectors of
names: $\vec{x}^{\pi} := (x_{i}^{\pi} \; : \; 0 \leq i < |\vec{x}| )$ where $\pi \in \{L,R\}$.

We also use $P^{\Box} := P|\Box$.

In \cite{MeredithR05} an interpretation of the new operator is
given. It turns out that there are several possible interpretations
all enjoying the requisite algebraic properties of the operator (see
\cite{milner91polyadicpi}). We will therefore make liberal use of
$(\nu\; \vec{x})P$.

% subsection the_syntax_and_semantics_of_the_notation_system (end)   

\input{qm2pi.qmops} 

\input{qm2pi.sterngerlach} 

\input{qm2pi.metric} 

% section concurrent_process_calculi (end)

%\input{qm2pi.proofsketch}

% section proof sketch (end)

%\input{qm2pi.slviaknots} 

% section spatial logic via knots (end)

\input{qm2pi.conclusion}

% section conclusion (end)

%\input{qm2pi.dtcodes} 

% section wiring algorithm (end)

\input{qm2pi.ack} 

% section acknowledgments (end)

\newpage


\bibliographystyle{plain}   
\bibliography{../../biblios/main.bib}

\input{qm2pi.rhodetails}

\end{document}

 

%\documentclass[12pt]{llncs}
%\documentclass{jktr}

\usepackage[pdftex]{hyperref}                   
\usepackage {listings}
\usepackage {mathpartir}
\usepackage{bcprules}
%\usepackage{listings}
                       
\usepackage{graphicx} 
%\usepackage[margins=2.5cm,nohead,nofoot]{geometry}
%\usepackage{geometry}
\usepackage{amsfonts}
\usepackage{amstext}
\usepackage{latexsym}
\usepackage{amssymb}
\usepackage{color}


%\include{myPreamble}
\include{qm2pi.local} 

%\ifpdf
%\usepackage[pdftex]{graphicx}
%\else
%\usepackage{graphicx}
%\fi

 % \ifpdf
%  \usepackage{pdfsync}
%  \if


%\title{Brief Article}
%\author{David F. Snyder}
%\author{L.G. Meredith}

%\address{Dept. of Math., Texas State University--San Marcos, San Marcos, TX 78666}
       
\pagestyle{empty}


\begin{document}

\lstset{language=[Objective]Caml,frame=shadowbox}

\input{qm2pi.front}

% section front matter (end)

\input{qm2pi.intro} 
 
% section introduction (end)

% \input{qm2pi.knotations} 

% section notation (end)

\input{qm2pi.process.calculi} 

% section concurrent_process_calculi_and_spatial_logics_ (end)
    
%\input{qm2pi.knots2pi} 

%\input{qm2pi.trefoil} 

%\input{qm2pi.mainthm} 

% subsection basic_interpretation (end)

%\input{qm2pi.rho.presentation} 
\subsection{The syntax and semantics of the notation system}\label{sub:the_syntax_and_semantics_of_the_notation_system} % (fold)

We now summarize a technical presentation of the calculus that
embodies our theory of dynamics. The typical presentation of such a
calculus follows the style of giving generators and relations on
them. The grammar, below, describing term constructors, freely
generates the set of processes, $\Proc$. This set is then quotiented
by a relation known as structural congruence and it is over this set
that the notion of dynamics is expressed. This presentation is
essentially that of \cite{MeredithR05} with the addition of
polyadicity and summation. For readability we have relegated some of
the technical subtleties to an appendix.

\subsubsection{Process grammar}\label{subsub:process_grammar}

\begin{mathpar}
  \inferrule* [lab=synchronization] {} {{M} \bc \pzero \;|\; x?F \;|\; x!C }
  \and
  \inferrule* [lab=abstraction] {} {{F} \bc (x)P}
  \and
  \inferrule* [lab=concretion] {} {{C} \bc \langle Q \rangle}
  \and
  \inferrule* [lab=process] {} {{P,Q} \bc M \;| \;P|Q \;|\; @{x}}
  \and
  \inferrule* [lab=name] {} {{x} \bc \quotep{P}}
\end{mathpar} 

Note that $\vec{x}$ (resp. $\vec{P}$) denotes a vector of names
(resp. processes) of length $|\vec{x}|$ (resp. $|\vec{P}|$). We adopt
the following useful abbreviations.

\begin{mathpar}
   x?(\vec{y}).P := x.(\vec{y})P \and  x\clift{\vec{P}} := x.\clift{\vec{P}}
   \and x!(y) := \lift{x}{\dropn{y}}
   \and \Pi_{i=0}^{n-1}P_i := P_0 | \ldots | P_{n-1}
\end{mathpar}

\subsubsection{Structural congruence}

\paragraph{Free and bound names and alpha-equivalence.} At the
core of structural equivalence is alpha-equivalence which identifies
process that are the same up to a change of variable. Formally, we
recognize the distinction between free and bound names. The free names
of a process, $\freenames{P}$, may be calculated recursively as
follows:

\begin{mathpar}
\freenames{\pzero} := \emptyset
  \and \\
  \freenames{x?(y).P} := \{ x \} \cup (\freenames{P} \setminus \{ y \})
  \and 
  \freenames{x!\langle P \rangle} := \{ x \} \cup \{ P \} 
  \and \\
  \freenames{P|Q} := \freenames{P} \cup \freenames{Q}
  \and \\
  \freenames{@{x}} := \{ x \}
\end{mathpar}

$\pi$
$\quotep{\pi}$

$\freenames{-} : \pi \to \mathcal{P}(\quotep{\pi})$

\begin{eqnarray*}
  \freenames{\pzero} & := & \emptyset \\
  \freenames{x?(y).P} & := & \{ x \} \cup (\freenames{P} \setminus \{ y \}) \\
  \freenames{x!\langle P \rangle} & := & \{ x \} \cup \{ P \} \\
  \freenames{P|Q} & := & \freenames{P} \cup \freenames{Q} \\
  \freenames{\dropn{x}} & := & \{ x \}
\end{eqnarray*}

The bound names of a process, $\boundnames{P}$, are those names occurring in $P$
that are not free. For example, in $x?(y).0$, the name $x$ is free, while $y$ is bound.

\begin{mathpar}
  \inferrule* [lab=monoidal-laws] {} { P|Q \equiv Q|P \and P|0 \equiv P \and P|(Q|R) \equiv (P|Q)|R }
\end{mathpar}

\begin{mathpar}
  \inferrule* [lab=alpha-equivalence] {} { (x)P \equiv (y)P\{y/x\} \and y \not\in \freenames{P} }
\end{mathpar}

\begin{definition}
Then two processes, $P,Q$, are alpha-equivalent if $P = Q\{\vec{y}/\vec{x}\}$ for
some $\vec{x} \in \boundnames{Q},\vec{y} \in \boundnames{P}$, where $Q\{\vec{y}/\vec{x}\}$
denotes the capture-avoiding substitution of $\vec{y}$ for $\vec{x}$ in $Q$.
\end{definition}

\begin{definition}
  The {\em structural congruence} \cite{SangiorgiWalker} , $\equiv$,
  between processes is the least congruence containing
  alpha-equivalence, satisfying the abelian monoid laws
  (associativity, commutativity and $\pzero$ as identity) for parallel
  composition $|$ and for summation $+$.
\end{definition}

\subsection{Name equivalence}

We take name equivalence, written $\nameeq$, to be the smallest
equivalence relation generated by the following rules.

\begin{mathpar}
\inferrule*[lab=Quote-drop]
{ }
{ \quotep{@{x}} \nameeq x }

\inferrule*[lab=Struct-equiv]
{ P \scong Q }
{ \quotep{P} \nameeq \quotep{Q} }
\end{mathpar}

The astute reader will have noticed that the mutual recursion of names
and processes imposes a mutual recursion on alpha-equivalence and
structural equivalence via name-equivalence. Fortunately, all of this
works out pleasantly and we may calculate in the natural way, free of
concern. The reader interested in the details is referred to the
appendix \ref{appendix:rho_details}.

\subsection{Substitution}

We use $\Proc$ for the set of processes, $\QProc$ for the set of
names, and $\id{\{}\vec{y} / \vec{x} \id{\}}$ to denote partial maps,
$s : \QProc \rightarrow \QProc$. A map, $s$ lifts, uniquely, to a map
on process terms, $\widehat{s} : \Proc \rightarrow \Proc$ by the
following equations.

\begin{mathpar}
  (0) \psubstp{Q}{P} := 0 \\
  (R \juxtap S) \psubstp{Q}{P}
  :=    
  (R)\psubstp{Q}{P} \juxtap (S) \psubstp{Q}{P} \\
  (x?(y).R) \psubstp{Q}{P}    
  :=    
  (x)\substp{Q}{P} (z)\concat( (R \psubstn{z}{y}) \psubstp{Q}{P} ) \\
  (\lift{x}{R}) \psubstp{Q}{P}  
  :=
  \lift{(x)\substp{Q}{P}}{ R \psubstp{Q}{P} } \\
%   (\dropn{x})  \psubstp{Q}{P}       
%   := 
%   \left\{ 
%     \begin{array}{ccc} 
%       \dropn{\quotep{Q}} & & x \nameeq \quotep{P} \\
%       \dropn{x} & & otherwise \\
%     \end{array}
%   \right. 
  (\dropn{x})  \psubstp{Q}{P}       
  := 
  \left\{ 
    \begin{array}{ccc} 
      Q & & x \nameeq \quotep{P} \\
      \dropn{x} & & otherwise \\
    \end{array}
  \right.
\end{mathpar}
 

where

\begin{eqnarray}
  (x)\id{\{} \lpquote Q \rpquote / \lpquote P \rpquote \id{\}}            = 
  \left\{ 
    \begin{array}{ccc}
      \lpquote Q \rpquote & & x \nameeq \lpquote P \rpquote \\
      x & & otherwise \\
    \end{array}
  \right. \nonumber
\end{eqnarray}

and $z$ is chosen distinct from $\quotep{P}$, $\quotep{Q}$, the free
names in $Q$, and all the names in $R$. Our $\alpha$-equivalence will
be built in the standard way from this substitution.

\begin{remark}\label{rem:no_self_referential_names}
  One consequence of these definitions is that $\forall P. \quotep{P}
  \not\in \freenames{P}$.
\end{remark}

\subsection{ Dynamic quote: an example }

Anticipating something of what's to come, consider applying the
substitution, $\widehat{\id{\{}u / z \id{\}}}$, to the following pair
of processes, $\lift{w}{y!(z)}$ and $w[ \lpquote y!(z) \rpquote ]$.

\begin{eqnarray}
	\lift{w}{y!(z)}\widehat{\id{\{}u / z \id{\}}}
		& = &
		\lift{w}{y!(u)} \nonumber\\
	w[ \lpquote y!(z) \rpquote ] \widehat{ \id{\{}u / z \id{\}} }
		& = &
		w[ \lpquote y!(z) \rpquote ] \nonumber
\end{eqnarray}

Because the body of the process between quotes is impervious to
substitution, we get radically different answers. In fact, by
examining the first process in an input context,
e.g. $x?(z).\lift{w}{y!(z)}$, we see that the process under the lift
operator may be shaped by prefixed inputs binding a name inside it. In
this sense, the lift operator will be seen as a way to dynamically
construct processes before reifying them as names.

Finally equipped with these standard features we can present the
dynamics of the calculus.

\subsubsection{Operational semantics} 

Finally, we introduce the computational dynamics. What marks these
algebras as distinct from other more traditionally studied algebraic
structures, e.g. vector spaces or polynomial rings, is the manner in
which dynamics is captured. In traditional structures, dynamics is typically
expressed through morphisms between such structures, as in linear maps
between vector spaces or morphisms between rings. In algebras
associated with the semantics of computation, the dynamics is
expressed as part of the algebraic structure itself, through a
reduction reduction relation typically denoted by $\red$. Below, we
give a recursive presentation of this relation for the calculus used
in the encoding.

$\red \subseteq \pi \times \pi$
$\red : \pi \to \mathcal{P}(\pi)$

\begin{mathpar}
  \inferrule* [lab=Comm] { \textsf{match}( x_{src}, x_{trgt} ) } { x_{trgt}?(y)P \; | \; x_{src}!\langle {Q} \rangle \red P\{\quotep{Q}/y}\} }
  \and \\
  \inferrule* [lab=Par] {{P} \red {P}'} {{{P} | {Q}} \red {{P}' | {Q}}}
  \and
  \inferrule* [lab=Equiv]{{{P} \scong {P}'} \andalso {{P}' \red {Q}'} \andalso {{Q}' \scong {Q}}}{{P} \red {Q}}
\end{mathpar}

\begin{eqnarray*}
  match_{\equiv} (\quotep{P},\quotep{Q}) & := & P \equiv Q \\
  match_{\dagger}(\quotep{P},\quotep{Q}) & := & \forall R. P|Q \red^{*} R => R \red^{*} 0 \\
  match_{K}(\quotep{P},\quotep{Q}) & := & K \mbox{ for some context } K
\end{eqnarray*}

$u?(x)P | u!\langle Q \rangle \red P\{\quotep{Q}/x\}$

%We write $\wred$ for $\red^*$, and $P\red$ if $\exists Q $ such that $ P \red Q$.
We write $P\red$ if $\exists Q $ such that $ P \red Q$ and $P\not\red$, otherwise.

\section{Replication}

As mentioned before, it is known that replication (and hence
recursion) can be implemented in a higher-order process algebra
\cite{SangiorgiWalker}. As our first example of calculation with the
machinery thus far presented we give the construction explicitly in
the {\rhoc}.

\begin{eqnarray}
	D_{x} & := & \prefix{x}{y}{(\binpar{\outputp{x}{y}}{@{y}})} \nonumber\\
	\bangp_{x}{P} & := & \binpar{{x}!\langle{\binpar{D_{x}}{P}}\rangle}{D_{x}} \nonumber
\end{eqnarray}

\begin{eqnarray}
	\bangp_{x}{P} & & \nonumber\\
	=
	& {x}!\langle{(\prefix{x}{y}{(\outputp{x}{y} | @{y})) | P}}\rangle 
	      | \prefix{x}{y}{(\outputp{x}{y} | @{y})} & \nonumber\\
	\red
	& (\outputp{x}{y} | @{y})\substn{\quotep{(\prefix{x}{y}{(@{y} | \outputp{x}{y})) | P}}}{y} & \nonumber\\
	=
	& \outputp{x}{\quotep{(\prefix{x}{y}{(\outputp{x}{y} | @{y})) | P}}}
	  | {(\prefix{x}{y}{(\outputp{x}{y} | @{y})) | P}} & \nonumber\\
	\red
	& \ldots & \nonumber\\
	\red^*
	& P | P | \ldots & \nonumber
\end{eqnarray}

Of course, this encoding, as an implementation, runs away, unfolding
$\bangp{P}$ eagerly. A lazier and more implementable replication
operator, restricted to input-guarded processes, may be obtained as follows.

\begin{eqnarray}
\bangp{\prefix{u}{v}{P}} 
	:= 
	\binpar{\lift{x}{\prefix{u}{v}{(\binpar{D(x)}{P})}}}{D(x)} \nonumber
\end{eqnarray}

\begin{remark}
  Note that the lazier definition still does not deal with summation
  or mixed summation (i.e. sums over input and output). The reader is
  invited to construct definitions of replication that deal with these
  features. 

  Further, the definitions are parameterized in a name, $x$. Can you,
  gentle reader, make a definition that eliminates this parameter and
  guarantees no accidental interaction between the replication
  machinery and the process being replicated -- i.e. no accidental
  sharing of names used by the process to get its work done and the
  name(s) used by the replication to effect copying. This latter
  revision of the definition of replication is crucial to obtaining
  the expected identity $!!P \sim !P$.
\end{remark}

\begin{remark}\label{rem:paradoxical_combinator}
  The reader familiar with the lambda calculus will have noticed the
  similarity between $D$ and the paradoxical combinator.

  [Ed. note: the existence of this seems to suggest we have to be more
  restrictive on the set of processes and names we admit if we are to
  support no-cloning.]
\end{remark}

\subsubsection{Bisimulation}

The computational dynamics gives rise to another kind of equivalence,
the equivalence of computational behavior. As previously mentioned
this is typically captured \emph{via} some form of bisimulation.

% The notion we use in this paper is weak barbed bisimulation
% \cite{milner91polyadicpi}.

The notion we use in this paper is derived from weak barbed
bisimulation \cite{milner91polyadicpi}. 

\begin{definition}
An \emph{observation relation}, $\downarrow_{\mathcal N}$, over a set
of names, $\mathcal N$, is the smallest relation satisfying the rules
below.

\infrule[Out-barb]{y \in {\mathcal N}, \; x \nameeq y}
		  {\outputp{x}{v} \downarrow_{\mathcal N} x}
\infrule[Par-barb]{\mbox{$P\downarrow_{\mathcal N} x$ or $Q\downarrow_{\mathcal N} x$}}
		  {\binpar{P}{Q} \downarrow_{\mathcal N} x}

We write $P \Downarrow_{\mathcal N} x$ if there is $Q$ such that 
$P \wred Q$ and $Q \downarrow_{\mathcal N} x$.
\end{definition}

\begin{definition}
%\label{def.bbisim}
An  ${\mathcal N}$-\emph{barbed bisimulation} over a set of names, ${\mathcal N}$, is a symmetric binary relation 
${\mathcal S}_{\mathcal N}$ between agents such that $P\rel{S}_{\mathcal N}Q$ implies:
\begin{enumerate}
\item If $P \red P'$ then $Q \wred Q'$ and $P'\rel{S}_{\mathcal N} Q'$.
\item If $P\downarrow_{\mathcal N} x$, then $Q\Downarrow_{\mathcal N} x$.
\end{enumerate}
$P$ is ${\mathcal N}$-barbed bisimilar to $Q$, written
$P \wbbisim_{\mathcal N} Q$, if $P \rel{S}_{\mathcal N} Q$ for some ${\mathcal N}$-barbed bisimulation ${\mathcal S}_{\mathcal N}$.
\end{definition}

$\mathcal{R} \subseteq \pi \times \pi$

$P \mathcal{R} Q => \forall P'. P \red P' \Rightarrow \exists Q'. Q \red Q', P' \mathcal{R} Q'$

$P \vdash x \Rightarrow Q \vdash x$

\begin{mathpar}
  \inferrule*[lab=Out-barb]{x \nameeq y}{{y}!\langle{Q}\rangle \vdash x}
  \and
  \inferrule*[lab=Par-barb]{\mbox{$P\vdash x$ or $Q\vdash x$}}{\binpar{P}{Q} \vdash x}
\end{mathpar}

\subsubsection{Contexts}

One of the principle advantages of computational calculi like the
$\pi$-calculus is a well-defined notion of context,
contextual-equivalence and a correlation between
contextual-equivalence and notions of bisimulation. The notion of
context allows the decomposition of a process into (sub-)process and
its syntactic environment, its context. Thus, a context may be
thought of as a process with a ``hole'' (written $\Box$) in it. The
application of a context $M$ to a process $P$, written $M[P]$, is
tantamount to filling the hole in $M$ with $P$. In this paper we do
not need the full weight of this theory, but do make use of the notion
of context in the proof the main theorem. 

\begin{mathpar}
  \inferrule* [lab=summation] {} {{M_{M},M_{N}} \bc \Box \;|\; x.M_{A} \;|\; M_{M}+M_{N}}
  \and
  \inferrule* [lab=agent] {} {{M_{A}} \bc (\vec{x})M_{P} \;| \; \clift{P_0,\ldots,M_{P},\ldots,P_N}}
  \and \\
  \inferrule* [lab=process] {} {{M_{P}} \bc M_{N} \;| \;P|M_{P} }
\end{mathpar} 

\begin{mathpar}
  \inferrule* [lab=sychronization] {} {M_{N} \bc \Box \;|\; x?M_{F} \;|\; x!M_{C}}
  \and
  \inferrule* [lab=abstraction] {} {{M_{F}} \bc (x)M_{P} }
  \and
  \inferrule* [lab=concretion] {} {{M_{C}} \bc \langle M_{P} \rangle }
  \and \\
  \inferrule* [lab=process] {} {{M_{P}} \bc M_{N} \;| \;P|M_{P} }
\end{mathpar}

\begin{definition}[contextual application] Given a context $M$, and
  process $P$, we define the \emph{contextual application}, $M[P] :=
  M\{P/\Box\}$. That is, the contextual application of M to P is the
  substitution of $P$ for $\Box$ in $M$.
\end{definition}

$\meaningof{-} : L \to \mathcal{P}(\pi)$

\begin{mathpar}
  \inferrule* [lab=collection] {} {\meaningof{true} = \pi, \and \meaningof{~E} = \pi \setminus \meaningof{E}, \and \meaningof{E_{1} \& E_{2}} = \meaningof{E_{1}} \cap \meaningof{E_{2}}}
\end{mathpar}

\begin{mathpar}
  \inferrule* [lab=structure] {} {\meaningof{0} = \{ P \in \pi | P \equiv 0 \}, \and \\ \meaningof{E_1 | E_2} = \{ P \in \pi | P \equiv P_{1} | P_{2}, P_{1} \in \meaningof{E_{1}}, P_{2} \in \meaningof{E_2}\} }
\end{mathpar}

\begin{mathpar}
 \inferrule* [lab=behavior] {} {\meaningof{\langle a?b \rangle E} = \{ P \in \pi | P \equiv Q | u?(y)P', \\ \and \\\\ \and \\ \;\;\; u \in \meaningof{a}, \forall z.P'\{z/y\} \in \meaningof{E\{z/b\}}\}, \and \\ \meaningof{a!E} = \{ P \in \pi | P \equiv Q | x!\langle P' \rangle, x \in \meaningof{a} P' \in \meaningof{E}\} }
\end{mathpar}

\begin{mathpar}
 \inferrule* [lab=nominal] {} {\meaningof{\quotep{E}} = \{ \quotep{P} \in \quotep{\pi} | P \in \meaningof{E} \}, \and \meaningof{\quotep{P}} = \{ \quotep{Q} \in \quotep{\pi} | P \equiv Q \} \and \\ \meaningof{@\quotep{E}} = \{ P \in \pi | P \equiv @x, x \in \meaningof{E} \}}
\end{mathpar}

\begin{eqnarray*}
  \\
  \meaningof{-} : TS \to ST
\end{eqnarray*}

\begin{eqnarray*}
  \\
  L : TS \to ST
\end{eqnarray*}

\begin{eqnarray*}
  \\
  P \models E \iff P \in \meaningof{E}
\end{eqnarray*}

\begin{eqnarray*}
  P \approx_{L} Q \iff \forall E \in L. P \models E \iff Q \models E
\end{eqnarray*}

\begin{eqnarray*}
  P \approx_{K} Q
\end{eqnarray*}

\begin{eqnarray*}
  P \approx Q
\end{eqnarray*}

$\approx_{K} = \approx = \approx_{L}$

\subsubsection{Contextual duality}

Note that contexts extend the quotation operation to a family of
operations from processes to names. Given a context, $M$, we can
define a \emph{nominal context}, $\quotep{M}$ by $\quotep{M}[P] :=
\quotep{M[P]}$. To foreshadow what is to come we observe that these
operations enjoy a duality with processes very much like the duality
between vectors and maps from vectors to scalars.

Further, because the calculus is essentially higher-order, we have a
correspondence between contexts and processes. More specifically,
given a name $x$ and a context $M$ we can construct $M^{*}_{x}$ such
that 

\begin{mathpar}
  M^{*}_{x} | \lift{x}{P} \red M[P]
\end{mathpar}

namely,

\begin{mathpar}
  M^{*}_{x} := x?(u).M[\dropn{u}]
\end{mathpar}

The dependence of $M^{*}_{x}$ on a name makes it an abstraction, 

\begin{mathpar}
  M^{*} := (x)x?(u).M[\dropn{u}]
\end{mathpar}

\subsection{Additional notation}

It will sometimes be convenient to denote the process a name
quotes. We already have the notation $x = \quotep{P}$, but it will be
convenient to introduce an alternate notation, $\procn{x}$, when we
want to emphasize the connection to the use of the name. Note that, by
virtue of name equivalence, $\quotep{\procn{x}} \nameeq x$; so, the
notation is consistent with previous definitions.

Further, because names have structure it is possible to effect
substitutions on the basis of that structure. This means we need to
upgrade our notation for substitutions, which we accomplish by
adapting comprehension notation. Thus,

\begin{mathpar}
  P\{ y / x : x \in S \}
\end{mathpar}

is interpreted to mean the process derived from P by replacing (in a
capture-avoiding manner) each occurrence of $x$ in $S$ by $y$. For example,

\begin{mathpar}
  P\{ \quotep{\procn{x}|\procn{x}} / x : x \in \freenames{P} \}
\end{mathpar}

will replace each (occurrence) of a free name $x$ in $P$ by
$\quotep{\procn{x}|\procn{x}}$.

Also, we will avail ourselves of the notation $x^{L}$ and $x^{R}$ to
denote injections of a name into disjoint copies of the name
space. There are numerous ways to accomplish this. One example can be
found in \cite{MeredithR05}. This notation overloads to vectors of
names: $\vec{x}^{\pi} := (x_{i}^{\pi} \; : \; 0 \leq i < |\vec{x}| )$ where $\pi \in \{L,R\}$.

We also use $P^{\Box} := P|\Box$.

In \cite{MeredithR05} an interpretation of the new operator is
given. It turns out that there are several possible interpretations
all enjoying the requisite algebraic properties of the operator (see
\cite{milner91polyadicpi}). We will therefore make liberal use of
$(\nu\; \vec{x})P$.

% subsection the_syntax_and_semantics_of_the_notation_system (end)   

\input{qm2pi.qmops} 

\input{qm2pi.sterngerlach} 

\input{qm2pi.metric} 

% section concurrent_process_calculi (end)

%\input{qm2pi.proofsketch}

% section proof sketch (end)

%\input{qm2pi.slviaknots} 

% section spatial logic via knots (end)

\input{qm2pi.conclusion}

% section conclusion (end)

%\input{qm2pi.dtcodes} 

% section wiring algorithm (end)

\input{qm2pi.ack} 

% section acknowledgments (end)

\newpage


\bibliographystyle{plain}   
\bibliography{../../biblios/main.bib}

\input{qm2pi.rhodetails}

\end{document}

 

% subsection basic_interpretation (end)

%\input{qm2pi.rho.presentation} 
\subsection{The syntax and semantics of the notation system}\label{sub:the_syntax_and_semantics_of_the_notation_system} % (fold)

We now summarize a technical presentation of the calculus that
embodies our theory of dynamics. The typical presentation of such a
calculus follows the style of giving generators and relations on
them. The grammar, below, describing term constructors, freely
generates the set of processes, $\Proc$. This set is then quotiented
by a relation known as structural congruence and it is over this set
that the notion of dynamics is expressed. This presentation is
essentially that of \cite{MeredithR05} with the addition of
polyadicity and summation. For readability we have relegated some of
the technical subtleties to an appendix.

\subsubsection{Process grammar}\label{subsub:process_grammar}

\begin{mathpar}
  \inferrule* [lab=synchronization] {} {{M} \bc \pzero \;|\; x?F \;|\; x!C }
  \and
  \inferrule* [lab=abstraction] {} {{F} \bc (x)P}
  \and
  \inferrule* [lab=concretion] {} {{C} \bc \langle Q \rangle}
  \and
  \inferrule* [lab=process] {} {{P,Q} \bc M \;| \;P|Q \;|\; @{x}}
  \and
  \inferrule* [lab=name] {} {{x} \bc \quotep{P}}
\end{mathpar} 

Note that $\vec{x}$ (resp. $\vec{P}$) denotes a vector of names
(resp. processes) of length $|\vec{x}|$ (resp. $|\vec{P}|$). We adopt
the following useful abbreviations.

\begin{mathpar}
   x?(\vec{y}).P := x.(\vec{y})P \and  x\clift{\vec{P}} := x.\clift{\vec{P}}
   \and x!(y) := \lift{x}{\dropn{y}}
   \and \Pi_{i=0}^{n-1}P_i := P_0 | \ldots | P_{n-1}
\end{mathpar}

\subsubsection{Structural congruence}

\paragraph{Free and bound names and alpha-equivalence.} At the
core of structural equivalence is alpha-equivalence which identifies
process that are the same up to a change of variable. Formally, we
recognize the distinction between free and bound names. The free names
of a process, $\freenames{P}$, may be calculated recursively as
follows:

\begin{mathpar}
\freenames{\pzero} := \emptyset
  \and \\
  \freenames{x?(y).P} := \{ x \} \cup (\freenames{P} \setminus \{ y \})
  \and 
  \freenames{x!\langle P \rangle} := \{ x \} \cup \{ P \} 
  \and \\
  \freenames{P|Q} := \freenames{P} \cup \freenames{Q}
  \and \\
  \freenames{@{x}} := \{ x \}
\end{mathpar}

$\pi$
$\quotep{\pi}$

$\freenames{-} : \pi \to \mathcal{P}(\quotep{\pi})$

\begin{eqnarray*}
  \freenames{\pzero} & := & \emptyset \\
  \freenames{x?(y).P} & := & \{ x \} \cup (\freenames{P} \setminus \{ y \}) \\
  \freenames{x!\langle P \rangle} & := & \{ x \} \cup \{ P \} \\
  \freenames{P|Q} & := & \freenames{P} \cup \freenames{Q} \\
  \freenames{\dropn{x}} & := & \{ x \}
\end{eqnarray*}

The bound names of a process, $\boundnames{P}$, are those names occurring in $P$
that are not free. For example, in $x?(y).0$, the name $x$ is free, while $y$ is bound.

\begin{mathpar}
  \inferrule* [lab=monoidal-laws] {} { P|Q \equiv Q|P \and P|0 \equiv P \and P|(Q|R) \equiv (P|Q)|R }
\end{mathpar}

\begin{mathpar}
  \inferrule* [lab=alpha-equivalence] {} { (x)P \equiv (y)P\{y/x\} \and y \not\in \freenames{P} }
\end{mathpar}

\begin{definition}
Then two processes, $P,Q$, are alpha-equivalent if $P = Q\{\vec{y}/\vec{x}\}$ for
some $\vec{x} \in \boundnames{Q},\vec{y} \in \boundnames{P}$, where $Q\{\vec{y}/\vec{x}\}$
denotes the capture-avoiding substitution of $\vec{y}$ for $\vec{x}$ in $Q$.
\end{definition}

\begin{definition}
  The {\em structural congruence} \cite{SangiorgiWalker} , $\equiv$,
  between processes is the least congruence containing
  alpha-equivalence, satisfying the abelian monoid laws
  (associativity, commutativity and $\pzero$ as identity) for parallel
  composition $|$ and for summation $+$.
\end{definition}

\subsection{Name equivalence}

We take name equivalence, written $\nameeq$, to be the smallest
equivalence relation generated by the following rules.

\begin{mathpar}
\inferrule*[lab=Quote-drop]
{ }
{ \quotep{@{x}} \nameeq x }

\inferrule*[lab=Struct-equiv]
{ P \scong Q }
{ \quotep{P} \nameeq \quotep{Q} }
\end{mathpar}

The astute reader will have noticed that the mutual recursion of names
and processes imposes a mutual recursion on alpha-equivalence and
structural equivalence via name-equivalence. Fortunately, all of this
works out pleasantly and we may calculate in the natural way, free of
concern. The reader interested in the details is referred to the
appendix \ref{appendix:rho_details}.

\subsection{Substitution}

We use $\Proc$ for the set of processes, $\QProc$ for the set of
names, and $\id{\{}\vec{y} / \vec{x} \id{\}}$ to denote partial maps,
$s : \QProc \rightarrow \QProc$. A map, $s$ lifts, uniquely, to a map
on process terms, $\widehat{s} : \Proc \rightarrow \Proc$ by the
following equations.

\begin{mathpar}
  (0) \psubstp{Q}{P} := 0 \\
  (R \juxtap S) \psubstp{Q}{P}
  :=    
  (R)\psubstp{Q}{P} \juxtap (S) \psubstp{Q}{P} \\
  (x?(y).R) \psubstp{Q}{P}    
  :=    
  (x)\substp{Q}{P} (z)\concat( (R \psubstn{z}{y}) \psubstp{Q}{P} ) \\
  (\lift{x}{R}) \psubstp{Q}{P}  
  :=
  \lift{(x)\substp{Q}{P}}{ R \psubstp{Q}{P} } \\
%   (\dropn{x})  \psubstp{Q}{P}       
%   := 
%   \left\{ 
%     \begin{array}{ccc} 
%       \dropn{\quotep{Q}} & & x \nameeq \quotep{P} \\
%       \dropn{x} & & otherwise \\
%     \end{array}
%   \right. 
  (\dropn{x})  \psubstp{Q}{P}       
  := 
  \left\{ 
    \begin{array}{ccc} 
      Q & & x \nameeq \quotep{P} \\
      \dropn{x} & & otherwise \\
    \end{array}
  \right.
\end{mathpar}
 

where

\begin{eqnarray}
  (x)\id{\{} \lpquote Q \rpquote / \lpquote P \rpquote \id{\}}            = 
  \left\{ 
    \begin{array}{ccc}
      \lpquote Q \rpquote & & x \nameeq \lpquote P \rpquote \\
      x & & otherwise \\
    \end{array}
  \right. \nonumber
\end{eqnarray}

and $z$ is chosen distinct from $\quotep{P}$, $\quotep{Q}$, the free
names in $Q$, and all the names in $R$. Our $\alpha$-equivalence will
be built in the standard way from this substitution.

\begin{remark}\label{rem:no_self_referential_names}
  One consequence of these definitions is that $\forall P. \quotep{P}
  \not\in \freenames{P}$.
\end{remark}

\subsection{ Dynamic quote: an example }

Anticipating something of what's to come, consider applying the
substitution, $\widehat{\id{\{}u / z \id{\}}}$, to the following pair
of processes, $\lift{w}{y!(z)}$ and $w[ \lpquote y!(z) \rpquote ]$.

\begin{eqnarray}
	\lift{w}{y!(z)}\widehat{\id{\{}u / z \id{\}}}
		& = &
		\lift{w}{y!(u)} \nonumber\\
	w[ \lpquote y!(z) \rpquote ] \widehat{ \id{\{}u / z \id{\}} }
		& = &
		w[ \lpquote y!(z) \rpquote ] \nonumber
\end{eqnarray}

Because the body of the process between quotes is impervious to
substitution, we get radically different answers. In fact, by
examining the first process in an input context,
e.g. $x?(z).\lift{w}{y!(z)}$, we see that the process under the lift
operator may be shaped by prefixed inputs binding a name inside it. In
this sense, the lift operator will be seen as a way to dynamically
construct processes before reifying them as names.

Finally equipped with these standard features we can present the
dynamics of the calculus.

\subsubsection{Operational semantics} 

Finally, we introduce the computational dynamics. What marks these
algebras as distinct from other more traditionally studied algebraic
structures, e.g. vector spaces or polynomial rings, is the manner in
which dynamics is captured. In traditional structures, dynamics is typically
expressed through morphisms between such structures, as in linear maps
between vector spaces or morphisms between rings. In algebras
associated with the semantics of computation, the dynamics is
expressed as part of the algebraic structure itself, through a
reduction reduction relation typically denoted by $\red$. Below, we
give a recursive presentation of this relation for the calculus used
in the encoding.

$\red \subseteq \pi \times \pi$
$\red : \pi \to \mathcal{P}(\pi)$

\begin{mathpar}
  \inferrule* [lab=Comm] { \textsf{match}( x_{src}, x_{trgt} ) } { x_{trgt}?(y)P \; | \; x_{src}!\langle {Q} \rangle \red P\{\quotep{Q}/y}\} }
  \and \\
  \inferrule* [lab=Par] {{P} \red {P}'} {{{P} | {Q}} \red {{P}' | {Q}}}
  \and
  \inferrule* [lab=Equiv]{{{P} \scong {P}'} \andalso {{P}' \red {Q}'} \andalso {{Q}' \scong {Q}}}{{P} \red {Q}}
\end{mathpar}

\begin{eqnarray*}
  match_{\equiv} (\quotep{P},\quotep{Q}) & := & P \equiv Q \\
  match_{\dagger}(\quotep{P},\quotep{Q}) & := & \forall R. P|Q \red^{*} R => R \red^{*} 0 \\
  match_{K}(\quotep{P},\quotep{Q}) & := & K \mbox{ for some context } K
\end{eqnarray*}

$u?(x)P | u!\langle Q \rangle \red P\{\quotep{Q}/x\}$

%We write $\wred$ for $\red^*$, and $P\red$ if $\exists Q $ such that $ P \red Q$.
We write $P\red$ if $\exists Q $ such that $ P \red Q$ and $P\not\red$, otherwise.

\section{Replication}

As mentioned before, it is known that replication (and hence
recursion) can be implemented in a higher-order process algebra
\cite{SangiorgiWalker}. As our first example of calculation with the
machinery thus far presented we give the construction explicitly in
the {\rhoc}.

\begin{eqnarray}
	D_{x} & := & \prefix{x}{y}{(\binpar{\outputp{x}{y}}{@{y}})} \nonumber\\
	\bangp_{x}{P} & := & \binpar{{x}!\langle{\binpar{D_{x}}{P}}\rangle}{D_{x}} \nonumber
\end{eqnarray}

\begin{eqnarray}
	\bangp_{x}{P} & & \nonumber\\
	=
	& {x}!\langle{(\prefix{x}{y}{(\outputp{x}{y} | @{y})) | P}}\rangle 
	      | \prefix{x}{y}{(\outputp{x}{y} | @{y})} & \nonumber\\
	\red
	& (\outputp{x}{y} | @{y})\substn{\quotep{(\prefix{x}{y}{(@{y} | \outputp{x}{y})) | P}}}{y} & \nonumber\\
	=
	& \outputp{x}{\quotep{(\prefix{x}{y}{(\outputp{x}{y} | @{y})) | P}}}
	  | {(\prefix{x}{y}{(\outputp{x}{y} | @{y})) | P}} & \nonumber\\
	\red
	& \ldots & \nonumber\\
	\red^*
	& P | P | \ldots & \nonumber
\end{eqnarray}

Of course, this encoding, as an implementation, runs away, unfolding
$\bangp{P}$ eagerly. A lazier and more implementable replication
operator, restricted to input-guarded processes, may be obtained as follows.

\begin{eqnarray}
\bangp{\prefix{u}{v}{P}} 
	:= 
	\binpar{\lift{x}{\prefix{u}{v}{(\binpar{D(x)}{P})}}}{D(x)} \nonumber
\end{eqnarray}

\begin{remark}
  Note that the lazier definition still does not deal with summation
  or mixed summation (i.e. sums over input and output). The reader is
  invited to construct definitions of replication that deal with these
  features. 

  Further, the definitions are parameterized in a name, $x$. Can you,
  gentle reader, make a definition that eliminates this parameter and
  guarantees no accidental interaction between the replication
  machinery and the process being replicated -- i.e. no accidental
  sharing of names used by the process to get its work done and the
  name(s) used by the replication to effect copying. This latter
  revision of the definition of replication is crucial to obtaining
  the expected identity $!!P \sim !P$.
\end{remark}

\begin{remark}\label{rem:paradoxical_combinator}
  The reader familiar with the lambda calculus will have noticed the
  similarity between $D$ and the paradoxical combinator.

  [Ed. note: the existence of this seems to suggest we have to be more
  restrictive on the set of processes and names we admit if we are to
  support no-cloning.]
\end{remark}

\subsubsection{Bisimulation}

The computational dynamics gives rise to another kind of equivalence,
the equivalence of computational behavior. As previously mentioned
this is typically captured \emph{via} some form of bisimulation.

% The notion we use in this paper is weak barbed bisimulation
% \cite{milner91polyadicpi}.

The notion we use in this paper is derived from weak barbed
bisimulation \cite{milner91polyadicpi}. 

\begin{definition}
An \emph{observation relation}, $\downarrow_{\mathcal N}$, over a set
of names, $\mathcal N$, is the smallest relation satisfying the rules
below.

\infrule[Out-barb]{y \in {\mathcal N}, \; x \nameeq y}
		  {\outputp{x}{v} \downarrow_{\mathcal N} x}
\infrule[Par-barb]{\mbox{$P\downarrow_{\mathcal N} x$ or $Q\downarrow_{\mathcal N} x$}}
		  {\binpar{P}{Q} \downarrow_{\mathcal N} x}

We write $P \Downarrow_{\mathcal N} x$ if there is $Q$ such that 
$P \wred Q$ and $Q \downarrow_{\mathcal N} x$.
\end{definition}

\begin{definition}
%\label{def.bbisim}
An  ${\mathcal N}$-\emph{barbed bisimulation} over a set of names, ${\mathcal N}$, is a symmetric binary relation 
${\mathcal S}_{\mathcal N}$ between agents such that $P\rel{S}_{\mathcal N}Q$ implies:
\begin{enumerate}
\item If $P \red P'$ then $Q \wred Q'$ and $P'\rel{S}_{\mathcal N} Q'$.
\item If $P\downarrow_{\mathcal N} x$, then $Q\Downarrow_{\mathcal N} x$.
\end{enumerate}
$P$ is ${\mathcal N}$-barbed bisimilar to $Q$, written
$P \wbbisim_{\mathcal N} Q$, if $P \rel{S}_{\mathcal N} Q$ for some ${\mathcal N}$-barbed bisimulation ${\mathcal S}_{\mathcal N}$.
\end{definition}

$\mathcal{R} \subseteq \pi \times \pi$

$P \mathcal{R} Q => \forall P'. P \red P' \Rightarrow \exists Q'. Q \red Q', P' \mathcal{R} Q'$

$P \vdash x \Rightarrow Q \vdash x$

\begin{mathpar}
  \inferrule*[lab=Out-barb]{x \nameeq y}{{y}!\langle{Q}\rangle \vdash x}
  \and
  \inferrule*[lab=Par-barb]{\mbox{$P\vdash x$ or $Q\vdash x$}}{\binpar{P}{Q} \vdash x}
\end{mathpar}

\subsubsection{Contexts}

One of the principle advantages of computational calculi like the
$\pi$-calculus is a well-defined notion of context,
contextual-equivalence and a correlation between
contextual-equivalence and notions of bisimulation. The notion of
context allows the decomposition of a process into (sub-)process and
its syntactic environment, its context. Thus, a context may be
thought of as a process with a ``hole'' (written $\Box$) in it. The
application of a context $M$ to a process $P$, written $M[P]$, is
tantamount to filling the hole in $M$ with $P$. In this paper we do
not need the full weight of this theory, but do make use of the notion
of context in the proof the main theorem. 

\begin{mathpar}
  \inferrule* [lab=summation] {} {{M_{M},M_{N}} \bc \Box \;|\; x.M_{A} \;|\; M_{M}+M_{N}}
  \and
  \inferrule* [lab=agent] {} {{M_{A}} \bc (\vec{x})M_{P} \;| \; \clift{P_0,\ldots,M_{P},\ldots,P_N}}
  \and \\
  \inferrule* [lab=process] {} {{M_{P}} \bc M_{N} \;| \;P|M_{P} }
\end{mathpar} 

\begin{mathpar}
  \inferrule* [lab=sychronization] {} {M_{N} \bc \Box \;|\; x?M_{F} \;|\; x!M_{C}}
  \and
  \inferrule* [lab=abstraction] {} {{M_{F}} \bc (x)M_{P} }
  \and
  \inferrule* [lab=concretion] {} {{M_{C}} \bc \langle M_{P} \rangle }
  \and \\
  \inferrule* [lab=process] {} {{M_{P}} \bc M_{N} \;| \;P|M_{P} }
\end{mathpar}

\begin{definition}[contextual application] Given a context $M$, and
  process $P$, we define the \emph{contextual application}, $M[P] :=
  M\{P/\Box\}$. That is, the contextual application of M to P is the
  substitution of $P$ for $\Box$ in $M$.
\end{definition}

$\meaningof{-} : L \to \mathcal{P}(\pi)$

\begin{mathpar}
  \inferrule* [lab=collection] {} {\meaningof{true} = \pi, \and \meaningof{~E} = \pi \setminus \meaningof{E}, \and \meaningof{E_{1} \& E_{2}} = \meaningof{E_{1}} \cap \meaningof{E_{2}}}
\end{mathpar}

\begin{mathpar}
  \inferrule* [lab=structure] {} {\meaningof{0} = \{ P \in \pi | P \equiv 0 \}, \and \\ \meaningof{E_1 | E_2} = \{ P \in \pi | P \equiv P_{1} | P_{2}, P_{1} \in \meaningof{E_{1}}, P_{2} \in \meaningof{E_2}\} }
\end{mathpar}

\begin{mathpar}
 \inferrule* [lab=behavior] {} {\meaningof{\langle a?b \rangle E} = \{ P \in \pi | P \equiv Q | u?(y)P', \\ \and \\\\ \and \\ \;\;\; u \in \meaningof{a}, \forall z.P'\{z/y\} \in \meaningof{E\{z/b\}}\}, \and \\ \meaningof{a!E} = \{ P \in \pi | P \equiv Q | x!\langle P' \rangle, x \in \meaningof{a} P' \in \meaningof{E}\} }
\end{mathpar}

\begin{mathpar}
 \inferrule* [lab=nominal] {} {\meaningof{\quotep{E}} = \{ \quotep{P} \in \quotep{\pi} | P \in \meaningof{E} \}, \and \meaningof{\quotep{P}} = \{ \quotep{Q} \in \quotep{\pi} | P \equiv Q \} \and \\ \meaningof{@\quotep{E}} = \{ P \in \pi | P \equiv @x, x \in \meaningof{E} \}}
\end{mathpar}

\begin{eqnarray*}
  \\
  \meaningof{-} : TS \to ST
\end{eqnarray*}

\begin{eqnarray*}
  \\
  L : TS \to ST
\end{eqnarray*}

\begin{eqnarray*}
  \\
  P \models E \iff P \in \meaningof{E}
\end{eqnarray*}

\begin{eqnarray*}
  P \approx_{L} Q \iff \forall E \in L. P \models E \iff Q \models E
\end{eqnarray*}

\begin{eqnarray*}
  P \approx_{K} Q
\end{eqnarray*}

\begin{eqnarray*}
  P \approx Q
\end{eqnarray*}

$\approx_{K} = \approx = \approx_{L}$

\subsubsection{Contextual duality}

Note that contexts extend the quotation operation to a family of
operations from processes to names. Given a context, $M$, we can
define a \emph{nominal context}, $\quotep{M}$ by $\quotep{M}[P] :=
\quotep{M[P]}$. To foreshadow what is to come we observe that these
operations enjoy a duality with processes very much like the duality
between vectors and maps from vectors to scalars.

Further, because the calculus is essentially higher-order, we have a
correspondence between contexts and processes. More specifically,
given a name $x$ and a context $M$ we can construct $M^{*}_{x}$ such
that 

\begin{mathpar}
  M^{*}_{x} | \lift{x}{P} \red M[P]
\end{mathpar}

namely,

\begin{mathpar}
  M^{*}_{x} := x?(u).M[\dropn{u}]
\end{mathpar}

The dependence of $M^{*}_{x}$ on a name makes it an abstraction, 

\begin{mathpar}
  M^{*} := (x)x?(u).M[\dropn{u}]
\end{mathpar}

\subsection{Additional notation}

It will sometimes be convenient to denote the process a name
quotes. We already have the notation $x = \quotep{P}$, but it will be
convenient to introduce an alternate notation, $\procn{x}$, when we
want to emphasize the connection to the use of the name. Note that, by
virtue of name equivalence, $\quotep{\procn{x}} \nameeq x$; so, the
notation is consistent with previous definitions.

Further, because names have structure it is possible to effect
substitutions on the basis of that structure. This means we need to
upgrade our notation for substitutions, which we accomplish by
adapting comprehension notation. Thus,

\begin{mathpar}
  P\{ y / x : x \in S \}
\end{mathpar}

is interpreted to mean the process derived from P by replacing (in a
capture-avoiding manner) each occurrence of $x$ in $S$ by $y$. For example,

\begin{mathpar}
  P\{ \quotep{\procn{x}|\procn{x}} / x : x \in \freenames{P} \}
\end{mathpar}

will replace each (occurrence) of a free name $x$ in $P$ by
$\quotep{\procn{x}|\procn{x}}$.

Also, we will avail ourselves of the notation $x^{L}$ and $x^{R}$ to
denote injections of a name into disjoint copies of the name
space. There are numerous ways to accomplish this. One example can be
found in \cite{MeredithR05}. This notation overloads to vectors of
names: $\vec{x}^{\pi} := (x_{i}^{\pi} \; : \; 0 \leq i < |\vec{x}| )$ where $\pi \in \{L,R\}$.

We also use $P^{\Box} := P|\Box$.

In \cite{MeredithR05} an interpretation of the new operator is
given. It turns out that there are several possible interpretations
all enjoying the requisite algebraic properties of the operator (see
\cite{milner91polyadicpi}). We will therefore make liberal use of
$(\nu\; \vec{x})P$.

% subsection the_syntax_and_semantics_of_the_notation_system (end)   

\section{Interpretation of QM}
\subsection{Supporting definitions}
\subsubsection{Multiplication}
\begin{mathpar}
  \quotep{Q} \cdot \quotep{R} := \quotep{Q|R}
  \and \\
  \quotep{Q} \cdot P := P\{ \quotep{Q|R} / \quotep{R} : \quotep{R} \in \freenames{P} \}
\end{mathpar}

\paragraph{Discussion}
The first line needs little explanation. The second line says that
each free name of the process is replaced with the multiplication of
that name by the scalar. Multiplication of a scalar (name) by a state
(process) results in a process all the names of which have been `moved
over' by parallel composition with the process the scalar
quotes. There is a subtlety that the bound names have to be
manipulated so that multiplied names aren't accidentally
captured. There are many ways to achieve this.

\begin{remark}\label{rem:multiplication_identities}
  The reader is invited to verify that for all $x,y,z \in \QProc$ and $P \in \Proc$
  \begin{mathpar}
    x \cdot \quotep{0} \equiv x 
    \and
    x \cdot y \equiv y \cdot x
    \and
    x \cdot (y \cdot z) \equiv (x \cdot y) \cdot z
    \and \\
    \quotep{0} \cdot P \equiv P
    \and \\
    x \cdot (y \cdot P) \equiv (x \cdot y) \cdot P
    \and \\
    x \cdot (P|Q) \equiv (x \cdot P) | (x \cdot Q)
    \and \\    
  \end{mathpar}
\end{remark}

\subsubsection{Tensor product}

We define a tensor product on processes by structural induction.

\paragraph{Tensor of sums} First note that all summations, including
$\pzero$ and sequence, can be written $\Sigma_{i} x_{i}.A_{i} +
\Sigma_{j} x_{j}.C_{j}$, where we have grouped input-guarded processes
together and output-guarded processes together.

Thus, we can define the tensor product of two summations, $N_{1}\otimes N_{2}$, where

\begin{mathpar}
  N_{1} := \Sigma_{i} x_{i}.A_{i} + \Sigma_{j} x_{j}.C_{j}
  \and
  N_{2} := \Sigma_{i'} y_{i'}.B_{i'} + \Sigma_{j'} y_{j'}.D_{j'} 
\end{mathpar}

as follows.

\begin{mathpar}
  \Sigma_{i} x_{i}.A_{i} + \Sigma_{j} x_{j}.C_{j} \otimes \Sigma_{i'}
  y_{i'}.B_{i'} + \Sigma_{j'} y_{j'}.D_{j'} 
  \and \\
  := \; \Sigma_{i} \Sigma_{i'} \quotep{\stackrel{\vee}{x_{i}}| \stackrel{\vee}{y_{i'}}}.(A_{i}\otimes B_{i'}) \; | \; \Sigma_{i'} \Sigma_{i} \quotep{\stackrel{\vee}{y_{i'}}|\stackrel{\vee}{x_{i}}}.(B_{i'}\otimes A_{i})
  \and
  \;\; | \;\; \Sigma_{j} \Sigma_{j'} \quotep{\stackrel{\vee}{x_{j}}|\stackrel{\vee}{y_{j'}}}.(A_{j}\otimes B_{j'}) \; | \; \Sigma_{j'} \Sigma_{j} \quotep{\stackrel{\vee}{y_{j'}}|\stackrel{\vee}{x_{j}}}.(B_{j'}\otimes A_{j})
\end{mathpar}

\begin{remark}
  Do we need to $x^{L}$ and $y^{R}$ for this construction as well?
\end{remark}

\paragraph{Tensor of parallel compositions} Next, we distribute tensor
over par.

\begin{mathpar}
  P_{1}|P_{2} \otimes Q_{1}|Q_{2} := (P_{1} \otimes Q_{1}) | (P_{1}
  \otimes Q_{2}) | (P_{2} \otimes Q_{1}) | (P_{2} \otimes Q_{2})
\end{mathpar}

\paragraph{Tensor with dropped names} We treat tensor of a
process with a dropped name as parallel composition.

\begin{mathpar}
  P \otimes \dropn{x} := P | \dropn{x}
\end{mathpar}

\paragraph{Tensor of agents}

Finally, we need to define tensor on agents. Note that the definition
of tensor on normal products only tensors inputs with inputs and
outputs with outputs. Thus, we only have to define the operation on
``homogeneous'' pairings.

\begin{mathpar}
  (\vec{x})P \otimes (\vec{y})Q
  \and \\
  := (x_{0}^{L}|y_{0}^{R},\ldots,x_{0}^{L}|y_{n}^{R},\ldots,x_{m}^{L}|y_{0}^{R},\ldots,x_{m}^{L}|y_{n}^R)(P\{ \vec{x}^{L}/\vec{x}\} \otimes Q \{ \vec{y}^{R}/\vec{y}\})
  \and \\
  \clift{\vec{P}} \otimes \clift{\vec{Q}}
  \and \\
  := \clift{P_{0}\otimes Q_{0},\ldots,P_{0}\otimes Q_{n},\ldots,P_{m}\otimes Q_{0},\ldots,P_{m}\otimes Q_{n}}
\end{mathpar}

\begin{remark}
  Observe that arities of tensored abstractions matches arities of
  tensored concretions if the original arities matched. Note also that
  the length of the arities corresponds to the increase in dimension
  we see in ordinary vector space tensor product.
\end{remark}

\begin{remark}
  Operationally, this definition distributes the tensor down to
  components ``linked'' by summation. Tensor over summation is
  intriguing in that it mixes names. Moreover, as a consequence of the
  way it mixes names we have the identities for all $x \in \QProc$ and
  $P,Q \in \Proc$

  \begin{mathpar}
    (x \cdot P) \otimes Q \equiv x \cdot (P \otimes Q) \equiv P \otimes (x \cdot Q)
    \and
    P \otimes \pzero \equiv P
  \end{mathpar}

  that the reader is invited to verify.
\end{remark}

\subsubsection{Annihilation}
\begin{mathpar}
  P^{\perp} := \{ Q | \forall R. P|Q \red^{*} R \Rightarrow R \red^{*} \pzero \}
  \and \\
  P^{\underline{\perp}} := \Sigma_{Q \in P^{\perp}} \quotep{Q}?(y).(\dropn{y}|Q) | \Sigma_{Q \in P^{\perp}} \quotep{Q}\clift{\Box}
\end{mathpar}

\paragraph{Discussion} The reader will note that $P^{\perp}$ is a
\emph{set} of processes, while $P^{\underline{\perp}}$ is a
\emph{context}. We call the set $P^{\perp}$ the \emph{annihilators} of
$P$. The parallel composition of a process in the annihilators of $P$
with $P$ will result in a process, the state space of which has all
paths eventually leading to $\pzero$. Execution may endure loops; but
under reasonable conditions of fairness (naturally guaranteed under
most notions of bisimulation) such a composite process cannot get
stuck in such a loop and will, eventually pop out and terminate.

The context $P^{\underline{\perp}}$ is ready and willing to ``take the
$P$ out of'' the process to which it is applied. It will effectively
transmit the code of the process to which it is applied to one of the
annihilators and run the process against it.

\subsubsection{Evaluation}
We fix $M$ a domain of fully abstract interpretation with an equality
coincident with bisimulation. We take $\meaningof{\cdot} : \Proc \to
M$ to be the map interpreting processes and $\nmeaningof{\cdot} : \M
\to Proc$ to be the map running the other way. Then we define

\begin{mathpar}
  \int P := \nmeaningof{\meaningof{P}}
\end{mathpar}

\paragraph{Discussion}
There are many fully abstract interpretations of Milner's
$\pi$-calculus. Any of them can be used as a basis for interpreting
the reflective calculus here. Equipped with such a domain it is
largely a matter of grinding through to check that the Yoneda
construction for the normalization-by-evaluation program can be
extended to this setting.

\begin{remark}
  The reader is invited to verify that $\int (P^{\underline{\perp}}[P]) = 0$.
\end{remark}

\subsection{Quantum mechanics}

Table \ref{tbl:core_qm_op_defns} gives the core operational definitions

\begin{table}[htp]\label{tbl:core_qm_op_defns}
  \center{
    \fbox{
      \begin{tabular}{c|c}
        quantum mechanics & process calculus \\
        \hline
        scalar & $x := \quotep{P}$ \\
        state vector & $\state{P} := P$ \\
        dual & $\state{P}^{*} := \event{P^{\underline{\perp}}} := \quotep{P^{\underline{\perp}}}[-]$ \\
        matrix & $ \Sigma_{\alpha} \state{P_{\alpha}}x_{\alpha}\event{Q_{\alpha}}$ \\
        vector addition & $\state{P} + \state{Q} := \state{P | Q}$ \\
        tensor product & $\state{P} \otimes \state{Q} := \state{P \otimes Q}$ \\
        inner product & $\innerprod{P}{Q} := \quotep{\int P^{\underline{\perp}}[Q]}$ \\
      \end{tabular}
    }
  }
  \caption{QM - operational definitions}
\end{table}

where

\begin{mathpar}
  \prmatrix{P}{Q} := \fprmatrix{P}{\quotep{\pzero}}{Q}
  \and
  \fprmatrix{P}{x}{Q} := (\state{P},x,\event{Q})
  \and
  (\fprmatrix{P}{x}{Q})(\state{R}) := x \cdot \innerprod{Q}{R} \cdot \state{P}
  \and
  (\fprmatrix{P}{x}{Q})(\event{R}) := x \cdot \innerprod{R}{P} \cdot \event{Q}
\end{mathpar}

\paragraph{Discussion}
As promised: vectors (aka states) are represented as processes; duals
as contextual duals; inner product definition should be compared with
standard inner product definition for ....

\begin{remark}
  Assuming $\int (P^{\underline{\perp}}[P]) = 0$, the reader is
  invited to verify that $(\fprmatrix{P}{x}{P})(\state{P}) = x \cdot \state{P}$.
\end{remark}

\begin{remark}
  The reader is invited to verify that $\innerprod{P}{Q}$ could
  equally well have been written $\quotep{\int \stackrel{\vee}{x}}$
  where $x = \event{P^{\underline{\perp}}}(Q)$.

  One of the motivations for this remark is that there is another way
  to factor these operations. We could package up evaluation in the dual:

  \begin{mathpar}
    \state{P}^{*} := \event{\int P^{\underline{\perp}}} := \quotep{\int P^{\underline{\perp}}}[-]
  \end{mathpar}

  and then have inner product defined by
  
  \begin{mathpar}
    \innerprod{P}{Q} := \event{P}(Q)
  \end{mathpar}

  Hopefully, experience with the calculations will provide guidance on
  the best factoring.
\end{remark}

\begin{remark}
  Assuming $\int (P^{\underline{\perp}}[P]) = 0$, the reader is
  invited to verify that $\forall P,Q. (\prmatrix{0}{Q})(\state{0}) =
  \state{0}$ and dually $(\prmatrix{P}{0})(\event{0}) = \event{0}$.
\end{remark}

\begin{remark}
  i'm a little worried that i don't (yet) have proper support for
  complex conjugacy. But, the observation above may give us a
  clue. According to Abramsky, it must be the case that the scalars
  are iso to the homset of the identity for the tensor -- which the
  observation above characterizes. 

  For now, we will simply bookmark the notion with $\overline{x}$.
\end{remark}

\subsubsection{Adjointness}

We need to give a definition of $(\cdot)^{\dagger}$ for matrices. The
obvious candidate definition is
\begin{mathpar}
(\Sigma_{\alpha}\fprmatrix{P_{\alpha}}{x_{\alpha}}{Q_{\alpha}})^{\dagger}
= \Sigma_{\alpha}\fprmatrix{(Q_{\alpha}^{\underline{\perp}})^{*}}{\overline{x}_{\alpha}}{P_{\alpha}^{\underline{\perp}}} 
\end{mathpar}

But, $(Q_{\alpha}^{\underline{\perp}})^{*}$ requires a name along
which to communicate the process to achieve the context application.

\subsubsection{Basis for a basis}
If processes label states and ``addition'' of states (a.k.a. vector
addition) is interpreted as parallel composition, what corresponds to
notions of linear independence and basis? Here, we recall that Yoshida
has developed a set of \emph{combinators} for an asynchronous verison
of Milner's $\pi$-calculus. These are a finite set of processes such
any process can be expressed as parallel composition of these
combinators together with liberal uses of the new operator and
replication. We can simply give a translation of these into the
present calculus and have reasonable expectation that the property
carries over. That is, that the resultant set allows to express all
processes via parallel composition. Note, however, that there is no
new operator or replication in this calculus. As a result, we expect
that the corresponding set is actually infinite. That is, we expect
that the space is actually infinite dimensional.

\begin{remark}
  The attentive reader may be a bit concerned. Certainly, the
  collection $S$, $K$ and $I$ is a finite set of
  combinators. Shouldn't we expect to see a finite set of combinators
  for an effectively equivalent system? i am very sympathetic to this
  critique and feel it warrants full attention. On the other hand, i
  also have in mind the following analogy. The natural numbers, as a
  monoid under addition, has exactly $1$ generator, while the natural
  numbers, as a monoid under multiplication, has countably many
  generators (the primes). We observe that the application of the
  lambda calculus is much less resource sensitive than the parallel
  composition of the $\pi$-calculus. Could it be the case that we have
  an analogy of the form
  
  \begin{mathpar}
    m + n : MN :: m*n : M|N
  \end{mathpar}

  giving a similar blow up in the set of ``primes''?  This is such a
  wonderful thought that, even if it's not true, i think it's worth
  writing down.
\end{remark}
 

\documentclass[12pt]{llncs}
%\documentclass{jktr}

\usepackage[pdftex]{hyperref}                   
\usepackage {listings}
\usepackage {mathpartir}
\usepackage{bcprules}
%\usepackage{listings}
                       
\usepackage{graphicx} 
%\usepackage[margins=2.5cm,nohead,nofoot]{geometry}
%\usepackage{geometry}
\usepackage{amsfonts}
\usepackage{amstext}
\usepackage{latexsym}
\usepackage{amssymb}
\usepackage{color}


%\include{myPreamble}
\include{qm2pi.local} 

%\ifpdf
%\usepackage[pdftex]{graphicx}
%\else
%\usepackage{graphicx}
%\fi

 % \ifpdf
%  \usepackage{pdfsync}
%  \if


%\title{Brief Article}
%\author{David F. Snyder}
%\author{L.G. Meredith}

%\address{Dept. of Math., Texas State University--San Marcos, San Marcos, TX 78666}
       
\pagestyle{empty}


\begin{document}

\lstset{language=[Objective]Caml,frame=shadowbox}

\input{qm2pi.front}

% section front matter (end)

\input{qm2pi.intro} 
 
% section introduction (end)

% \input{qm2pi.knotations} 

% section notation (end)

\input{qm2pi.process.calculi} 

% section concurrent_process_calculi_and_spatial_logics_ (end)
    
%\input{qm2pi.knots2pi} 

%\input{qm2pi.trefoil} 

%\input{qm2pi.mainthm} 

% subsection basic_interpretation (end)

%\input{qm2pi.rho.presentation} 
\subsection{The syntax and semantics of the notation system}\label{sub:the_syntax_and_semantics_of_the_notation_system} % (fold)

We now summarize a technical presentation of the calculus that
embodies our theory of dynamics. The typical presentation of such a
calculus follows the style of giving generators and relations on
them. The grammar, below, describing term constructors, freely
generates the set of processes, $\Proc$. This set is then quotiented
by a relation known as structural congruence and it is over this set
that the notion of dynamics is expressed. This presentation is
essentially that of \cite{MeredithR05} with the addition of
polyadicity and summation. For readability we have relegated some of
the technical subtleties to an appendix.

\subsubsection{Process grammar}\label{subsub:process_grammar}

\begin{mathpar}
  \inferrule* [lab=synchronization] {} {{M} \bc \pzero \;|\; x?F \;|\; x!C }
  \and
  \inferrule* [lab=abstraction] {} {{F} \bc (x)P}
  \and
  \inferrule* [lab=concretion] {} {{C} \bc \langle Q \rangle}
  \and
  \inferrule* [lab=process] {} {{P,Q} \bc M \;| \;P|Q \;|\; @{x}}
  \and
  \inferrule* [lab=name] {} {{x} \bc \quotep{P}}
\end{mathpar} 

Note that $\vec{x}$ (resp. $\vec{P}$) denotes a vector of names
(resp. processes) of length $|\vec{x}|$ (resp. $|\vec{P}|$). We adopt
the following useful abbreviations.

\begin{mathpar}
   x?(\vec{y}).P := x.(\vec{y})P \and  x\clift{\vec{P}} := x.\clift{\vec{P}}
   \and x!(y) := \lift{x}{\dropn{y}}
   \and \Pi_{i=0}^{n-1}P_i := P_0 | \ldots | P_{n-1}
\end{mathpar}

\subsubsection{Structural congruence}

\paragraph{Free and bound names and alpha-equivalence.} At the
core of structural equivalence is alpha-equivalence which identifies
process that are the same up to a change of variable. Formally, we
recognize the distinction between free and bound names. The free names
of a process, $\freenames{P}$, may be calculated recursively as
follows:

\begin{mathpar}
\freenames{\pzero} := \emptyset
  \and \\
  \freenames{x?(y).P} := \{ x \} \cup (\freenames{P} \setminus \{ y \})
  \and 
  \freenames{x!\langle P \rangle} := \{ x \} \cup \{ P \} 
  \and \\
  \freenames{P|Q} := \freenames{P} \cup \freenames{Q}
  \and \\
  \freenames{@{x}} := \{ x \}
\end{mathpar}

$\pi$
$\quotep{\pi}$

$\freenames{-} : \pi \to \mathcal{P}(\quotep{\pi})$

\begin{eqnarray*}
  \freenames{\pzero} & := & \emptyset \\
  \freenames{x?(y).P} & := & \{ x \} \cup (\freenames{P} \setminus \{ y \}) \\
  \freenames{x!\langle P \rangle} & := & \{ x \} \cup \{ P \} \\
  \freenames{P|Q} & := & \freenames{P} \cup \freenames{Q} \\
  \freenames{\dropn{x}} & := & \{ x \}
\end{eqnarray*}

The bound names of a process, $\boundnames{P}$, are those names occurring in $P$
that are not free. For example, in $x?(y).0$, the name $x$ is free, while $y$ is bound.

\begin{mathpar}
  \inferrule* [lab=monoidal-laws] {} { P|Q \equiv Q|P \and P|0 \equiv P \and P|(Q|R) \equiv (P|Q)|R }
\end{mathpar}

\begin{mathpar}
  \inferrule* [lab=alpha-equivalence] {} { (x)P \equiv (y)P\{y/x\} \and y \not\in \freenames{P} }
\end{mathpar}

\begin{definition}
Then two processes, $P,Q$, are alpha-equivalent if $P = Q\{\vec{y}/\vec{x}\}$ for
some $\vec{x} \in \boundnames{Q},\vec{y} \in \boundnames{P}$, where $Q\{\vec{y}/\vec{x}\}$
denotes the capture-avoiding substitution of $\vec{y}$ for $\vec{x}$ in $Q$.
\end{definition}

\begin{definition}
  The {\em structural congruence} \cite{SangiorgiWalker} , $\equiv$,
  between processes is the least congruence containing
  alpha-equivalence, satisfying the abelian monoid laws
  (associativity, commutativity and $\pzero$ as identity) for parallel
  composition $|$ and for summation $+$.
\end{definition}

\subsection{Name equivalence}

We take name equivalence, written $\nameeq$, to be the smallest
equivalence relation generated by the following rules.

\begin{mathpar}
\inferrule*[lab=Quote-drop]
{ }
{ \quotep{@{x}} \nameeq x }

\inferrule*[lab=Struct-equiv]
{ P \scong Q }
{ \quotep{P} \nameeq \quotep{Q} }
\end{mathpar}

The astute reader will have noticed that the mutual recursion of names
and processes imposes a mutual recursion on alpha-equivalence and
structural equivalence via name-equivalence. Fortunately, all of this
works out pleasantly and we may calculate in the natural way, free of
concern. The reader interested in the details is referred to the
appendix \ref{appendix:rho_details}.

\subsection{Substitution}

We use $\Proc$ for the set of processes, $\QProc$ for the set of
names, and $\id{\{}\vec{y} / \vec{x} \id{\}}$ to denote partial maps,
$s : \QProc \rightarrow \QProc$. A map, $s$ lifts, uniquely, to a map
on process terms, $\widehat{s} : \Proc \rightarrow \Proc$ by the
following equations.

\begin{mathpar}
  (0) \psubstp{Q}{P} := 0 \\
  (R \juxtap S) \psubstp{Q}{P}
  :=    
  (R)\psubstp{Q}{P} \juxtap (S) \psubstp{Q}{P} \\
  (x?(y).R) \psubstp{Q}{P}    
  :=    
  (x)\substp{Q}{P} (z)\concat( (R \psubstn{z}{y}) \psubstp{Q}{P} ) \\
  (\lift{x}{R}) \psubstp{Q}{P}  
  :=
  \lift{(x)\substp{Q}{P}}{ R \psubstp{Q}{P} } \\
%   (\dropn{x})  \psubstp{Q}{P}       
%   := 
%   \left\{ 
%     \begin{array}{ccc} 
%       \dropn{\quotep{Q}} & & x \nameeq \quotep{P} \\
%       \dropn{x} & & otherwise \\
%     \end{array}
%   \right. 
  (\dropn{x})  \psubstp{Q}{P}       
  := 
  \left\{ 
    \begin{array}{ccc} 
      Q & & x \nameeq \quotep{P} \\
      \dropn{x} & & otherwise \\
    \end{array}
  \right.
\end{mathpar}
 

where

\begin{eqnarray}
  (x)\id{\{} \lpquote Q \rpquote / \lpquote P \rpquote \id{\}}            = 
  \left\{ 
    \begin{array}{ccc}
      \lpquote Q \rpquote & & x \nameeq \lpquote P \rpquote \\
      x & & otherwise \\
    \end{array}
  \right. \nonumber
\end{eqnarray}

and $z$ is chosen distinct from $\quotep{P}$, $\quotep{Q}$, the free
names in $Q$, and all the names in $R$. Our $\alpha$-equivalence will
be built in the standard way from this substitution.

\begin{remark}\label{rem:no_self_referential_names}
  One consequence of these definitions is that $\forall P. \quotep{P}
  \not\in \freenames{P}$.
\end{remark}

\subsection{ Dynamic quote: an example }

Anticipating something of what's to come, consider applying the
substitution, $\widehat{\id{\{}u / z \id{\}}}$, to the following pair
of processes, $\lift{w}{y!(z)}$ and $w[ \lpquote y!(z) \rpquote ]$.

\begin{eqnarray}
	\lift{w}{y!(z)}\widehat{\id{\{}u / z \id{\}}}
		& = &
		\lift{w}{y!(u)} \nonumber\\
	w[ \lpquote y!(z) \rpquote ] \widehat{ \id{\{}u / z \id{\}} }
		& = &
		w[ \lpquote y!(z) \rpquote ] \nonumber
\end{eqnarray}

Because the body of the process between quotes is impervious to
substitution, we get radically different answers. In fact, by
examining the first process in an input context,
e.g. $x?(z).\lift{w}{y!(z)}$, we see that the process under the lift
operator may be shaped by prefixed inputs binding a name inside it. In
this sense, the lift operator will be seen as a way to dynamically
construct processes before reifying them as names.

Finally equipped with these standard features we can present the
dynamics of the calculus.

\subsubsection{Operational semantics} 

Finally, we introduce the computational dynamics. What marks these
algebras as distinct from other more traditionally studied algebraic
structures, e.g. vector spaces or polynomial rings, is the manner in
which dynamics is captured. In traditional structures, dynamics is typically
expressed through morphisms between such structures, as in linear maps
between vector spaces or morphisms between rings. In algebras
associated with the semantics of computation, the dynamics is
expressed as part of the algebraic structure itself, through a
reduction reduction relation typically denoted by $\red$. Below, we
give a recursive presentation of this relation for the calculus used
in the encoding.

$\red \subseteq \pi \times \pi$
$\red : \pi \to \mathcal{P}(\pi)$

\begin{mathpar}
  \inferrule* [lab=Comm] { \textsf{match}( x_{src}, x_{trgt} ) } { x_{trgt}?(y)P \; | \; x_{src}!\langle {Q} \rangle \red P\{\quotep{Q}/y}\} }
  \and \\
  \inferrule* [lab=Par] {{P} \red {P}'} {{{P} | {Q}} \red {{P}' | {Q}}}
  \and
  \inferrule* [lab=Equiv]{{{P} \scong {P}'} \andalso {{P}' \red {Q}'} \andalso {{Q}' \scong {Q}}}{{P} \red {Q}}
\end{mathpar}

\begin{eqnarray*}
  match_{\equiv} (\quotep{P},\quotep{Q}) & := & P \equiv Q \\
  match_{\dagger}(\quotep{P},\quotep{Q}) & := & \forall R. P|Q \red^{*} R => R \red^{*} 0 \\
  match_{K}(\quotep{P},\quotep{Q}) & := & K \mbox{ for some context } K
\end{eqnarray*}

$u?(x)P | u!\langle Q \rangle \red P\{\quotep{Q}/x\}$

%We write $\wred$ for $\red^*$, and $P\red$ if $\exists Q $ such that $ P \red Q$.
We write $P\red$ if $\exists Q $ such that $ P \red Q$ and $P\not\red$, otherwise.

\section{Replication}

As mentioned before, it is known that replication (and hence
recursion) can be implemented in a higher-order process algebra
\cite{SangiorgiWalker}. As our first example of calculation with the
machinery thus far presented we give the construction explicitly in
the {\rhoc}.

\begin{eqnarray}
	D_{x} & := & \prefix{x}{y}{(\binpar{\outputp{x}{y}}{@{y}})} \nonumber\\
	\bangp_{x}{P} & := & \binpar{{x}!\langle{\binpar{D_{x}}{P}}\rangle}{D_{x}} \nonumber
\end{eqnarray}

\begin{eqnarray}
	\bangp_{x}{P} & & \nonumber\\
	=
	& {x}!\langle{(\prefix{x}{y}{(\outputp{x}{y} | @{y})) | P}}\rangle 
	      | \prefix{x}{y}{(\outputp{x}{y} | @{y})} & \nonumber\\
	\red
	& (\outputp{x}{y} | @{y})\substn{\quotep{(\prefix{x}{y}{(@{y} | \outputp{x}{y})) | P}}}{y} & \nonumber\\
	=
	& \outputp{x}{\quotep{(\prefix{x}{y}{(\outputp{x}{y} | @{y})) | P}}}
	  | {(\prefix{x}{y}{(\outputp{x}{y} | @{y})) | P}} & \nonumber\\
	\red
	& \ldots & \nonumber\\
	\red^*
	& P | P | \ldots & \nonumber
\end{eqnarray}

Of course, this encoding, as an implementation, runs away, unfolding
$\bangp{P}$ eagerly. A lazier and more implementable replication
operator, restricted to input-guarded processes, may be obtained as follows.

\begin{eqnarray}
\bangp{\prefix{u}{v}{P}} 
	:= 
	\binpar{\lift{x}{\prefix{u}{v}{(\binpar{D(x)}{P})}}}{D(x)} \nonumber
\end{eqnarray}

\begin{remark}
  Note that the lazier definition still does not deal with summation
  or mixed summation (i.e. sums over input and output). The reader is
  invited to construct definitions of replication that deal with these
  features. 

  Further, the definitions are parameterized in a name, $x$. Can you,
  gentle reader, make a definition that eliminates this parameter and
  guarantees no accidental interaction between the replication
  machinery and the process being replicated -- i.e. no accidental
  sharing of names used by the process to get its work done and the
  name(s) used by the replication to effect copying. This latter
  revision of the definition of replication is crucial to obtaining
  the expected identity $!!P \sim !P$.
\end{remark}

\begin{remark}\label{rem:paradoxical_combinator}
  The reader familiar with the lambda calculus will have noticed the
  similarity between $D$ and the paradoxical combinator.

  [Ed. note: the existence of this seems to suggest we have to be more
  restrictive on the set of processes and names we admit if we are to
  support no-cloning.]
\end{remark}

\subsubsection{Bisimulation}

The computational dynamics gives rise to another kind of equivalence,
the equivalence of computational behavior. As previously mentioned
this is typically captured \emph{via} some form of bisimulation.

% The notion we use in this paper is weak barbed bisimulation
% \cite{milner91polyadicpi}.

The notion we use in this paper is derived from weak barbed
bisimulation \cite{milner91polyadicpi}. 

\begin{definition}
An \emph{observation relation}, $\downarrow_{\mathcal N}$, over a set
of names, $\mathcal N$, is the smallest relation satisfying the rules
below.

\infrule[Out-barb]{y \in {\mathcal N}, \; x \nameeq y}
		  {\outputp{x}{v} \downarrow_{\mathcal N} x}
\infrule[Par-barb]{\mbox{$P\downarrow_{\mathcal N} x$ or $Q\downarrow_{\mathcal N} x$}}
		  {\binpar{P}{Q} \downarrow_{\mathcal N} x}

We write $P \Downarrow_{\mathcal N} x$ if there is $Q$ such that 
$P \wred Q$ and $Q \downarrow_{\mathcal N} x$.
\end{definition}

\begin{definition}
%\label{def.bbisim}
An  ${\mathcal N}$-\emph{barbed bisimulation} over a set of names, ${\mathcal N}$, is a symmetric binary relation 
${\mathcal S}_{\mathcal N}$ between agents such that $P\rel{S}_{\mathcal N}Q$ implies:
\begin{enumerate}
\item If $P \red P'$ then $Q \wred Q'$ and $P'\rel{S}_{\mathcal N} Q'$.
\item If $P\downarrow_{\mathcal N} x$, then $Q\Downarrow_{\mathcal N} x$.
\end{enumerate}
$P$ is ${\mathcal N}$-barbed bisimilar to $Q$, written
$P \wbbisim_{\mathcal N} Q$, if $P \rel{S}_{\mathcal N} Q$ for some ${\mathcal N}$-barbed bisimulation ${\mathcal S}_{\mathcal N}$.
\end{definition}

$\mathcal{R} \subseteq \pi \times \pi$

$P \mathcal{R} Q => \forall P'. P \red P' \Rightarrow \exists Q'. Q \red Q', P' \mathcal{R} Q'$

$P \vdash x \Rightarrow Q \vdash x$

\begin{mathpar}
  \inferrule*[lab=Out-barb]{x \nameeq y}{{y}!\langle{Q}\rangle \vdash x}
  \and
  \inferrule*[lab=Par-barb]{\mbox{$P\vdash x$ or $Q\vdash x$}}{\binpar{P}{Q} \vdash x}
\end{mathpar}

\subsubsection{Contexts}

One of the principle advantages of computational calculi like the
$\pi$-calculus is a well-defined notion of context,
contextual-equivalence and a correlation between
contextual-equivalence and notions of bisimulation. The notion of
context allows the decomposition of a process into (sub-)process and
its syntactic environment, its context. Thus, a context may be
thought of as a process with a ``hole'' (written $\Box$) in it. The
application of a context $M$ to a process $P$, written $M[P]$, is
tantamount to filling the hole in $M$ with $P$. In this paper we do
not need the full weight of this theory, but do make use of the notion
of context in the proof the main theorem. 

\begin{mathpar}
  \inferrule* [lab=summation] {} {{M_{M},M_{N}} \bc \Box \;|\; x.M_{A} \;|\; M_{M}+M_{N}}
  \and
  \inferrule* [lab=agent] {} {{M_{A}} \bc (\vec{x})M_{P} \;| \; \clift{P_0,\ldots,M_{P},\ldots,P_N}}
  \and \\
  \inferrule* [lab=process] {} {{M_{P}} \bc M_{N} \;| \;P|M_{P} }
\end{mathpar} 

\begin{mathpar}
  \inferrule* [lab=sychronization] {} {M_{N} \bc \Box \;|\; x?M_{F} \;|\; x!M_{C}}
  \and
  \inferrule* [lab=abstraction] {} {{M_{F}} \bc (x)M_{P} }
  \and
  \inferrule* [lab=concretion] {} {{M_{C}} \bc \langle M_{P} \rangle }
  \and \\
  \inferrule* [lab=process] {} {{M_{P}} \bc M_{N} \;| \;P|M_{P} }
\end{mathpar}

\begin{definition}[contextual application] Given a context $M$, and
  process $P$, we define the \emph{contextual application}, $M[P] :=
  M\{P/\Box\}$. That is, the contextual application of M to P is the
  substitution of $P$ for $\Box$ in $M$.
\end{definition}

$\meaningof{-} : L \to \mathcal{P}(\pi)$

\begin{mathpar}
  \inferrule* [lab=collection] {} {\meaningof{true} = \pi, \and \meaningof{~E} = \pi \setminus \meaningof{E}, \and \meaningof{E_{1} \& E_{2}} = \meaningof{E_{1}} \cap \meaningof{E_{2}}}
\end{mathpar}

\begin{mathpar}
  \inferrule* [lab=structure] {} {\meaningof{0} = \{ P \in \pi | P \equiv 0 \}, \and \\ \meaningof{E_1 | E_2} = \{ P \in \pi | P \equiv P_{1} | P_{2}, P_{1} \in \meaningof{E_{1}}, P_{2} \in \meaningof{E_2}\} }
\end{mathpar}

\begin{mathpar}
 \inferrule* [lab=behavior] {} {\meaningof{\langle a?b \rangle E} = \{ P \in \pi | P \equiv Q | u?(y)P', \\ \and \\\\ \and \\ \;\;\; u \in \meaningof{a}, \forall z.P'\{z/y\} \in \meaningof{E\{z/b\}}\}, \and \\ \meaningof{a!E} = \{ P \in \pi | P \equiv Q | x!\langle P' \rangle, x \in \meaningof{a} P' \in \meaningof{E}\} }
\end{mathpar}

\begin{mathpar}
 \inferrule* [lab=nominal] {} {\meaningof{\quotep{E}} = \{ \quotep{P} \in \quotep{\pi} | P \in \meaningof{E} \}, \and \meaningof{\quotep{P}} = \{ \quotep{Q} \in \quotep{\pi} | P \equiv Q \} \and \\ \meaningof{@\quotep{E}} = \{ P \in \pi | P \equiv @x, x \in \meaningof{E} \}}
\end{mathpar}

\begin{eqnarray*}
  \\
  \meaningof{-} : TS \to ST
\end{eqnarray*}

\begin{eqnarray*}
  \\
  L : TS \to ST
\end{eqnarray*}

\begin{eqnarray*}
  \\
  P \models E \iff P \in \meaningof{E}
\end{eqnarray*}

\begin{eqnarray*}
  P \approx_{L} Q \iff \forall E \in L. P \models E \iff Q \models E
\end{eqnarray*}

\begin{eqnarray*}
  P \approx_{K} Q
\end{eqnarray*}

\begin{eqnarray*}
  P \approx Q
\end{eqnarray*}

$\approx_{K} = \approx = \approx_{L}$

\subsubsection{Contextual duality}

Note that contexts extend the quotation operation to a family of
operations from processes to names. Given a context, $M$, we can
define a \emph{nominal context}, $\quotep{M}$ by $\quotep{M}[P] :=
\quotep{M[P]}$. To foreshadow what is to come we observe that these
operations enjoy a duality with processes very much like the duality
between vectors and maps from vectors to scalars.

Further, because the calculus is essentially higher-order, we have a
correspondence between contexts and processes. More specifically,
given a name $x$ and a context $M$ we can construct $M^{*}_{x}$ such
that 

\begin{mathpar}
  M^{*}_{x} | \lift{x}{P} \red M[P]
\end{mathpar}

namely,

\begin{mathpar}
  M^{*}_{x} := x?(u).M[\dropn{u}]
\end{mathpar}

The dependence of $M^{*}_{x}$ on a name makes it an abstraction, 

\begin{mathpar}
  M^{*} := (x)x?(u).M[\dropn{u}]
\end{mathpar}

\subsection{Additional notation}

It will sometimes be convenient to denote the process a name
quotes. We already have the notation $x = \quotep{P}$, but it will be
convenient to introduce an alternate notation, $\procn{x}$, when we
want to emphasize the connection to the use of the name. Note that, by
virtue of name equivalence, $\quotep{\procn{x}} \nameeq x$; so, the
notation is consistent with previous definitions.

Further, because names have structure it is possible to effect
substitutions on the basis of that structure. This means we need to
upgrade our notation for substitutions, which we accomplish by
adapting comprehension notation. Thus,

\begin{mathpar}
  P\{ y / x : x \in S \}
\end{mathpar}

is interpreted to mean the process derived from P by replacing (in a
capture-avoiding manner) each occurrence of $x$ in $S$ by $y$. For example,

\begin{mathpar}
  P\{ \quotep{\procn{x}|\procn{x}} / x : x \in \freenames{P} \}
\end{mathpar}

will replace each (occurrence) of a free name $x$ in $P$ by
$\quotep{\procn{x}|\procn{x}}$.

Also, we will avail ourselves of the notation $x^{L}$ and $x^{R}$ to
denote injections of a name into disjoint copies of the name
space. There are numerous ways to accomplish this. One example can be
found in \cite{MeredithR05}. This notation overloads to vectors of
names: $\vec{x}^{\pi} := (x_{i}^{\pi} \; : \; 0 \leq i < |\vec{x}| )$ where $\pi \in \{L,R\}$.

We also use $P^{\Box} := P|\Box$.

In \cite{MeredithR05} an interpretation of the new operator is
given. It turns out that there are several possible interpretations
all enjoying the requisite algebraic properties of the operator (see
\cite{milner91polyadicpi}). We will therefore make liberal use of
$(\nu\; \vec{x})P$.

% subsection the_syntax_and_semantics_of_the_notation_system (end)   

\input{qm2pi.qmops} 

\input{qm2pi.sterngerlach} 

\input{qm2pi.metric} 

% section concurrent_process_calculi (end)

%\input{qm2pi.proofsketch}

% section proof sketch (end)

%\input{qm2pi.slviaknots} 

% section spatial logic via knots (end)

\input{qm2pi.conclusion}

% section conclusion (end)

%\input{qm2pi.dtcodes} 

% section wiring algorithm (end)

\input{qm2pi.ack} 

% section acknowledgments (end)

\newpage


\bibliographystyle{plain}   
\bibliography{../../biblios/main.bib}

\input{qm2pi.rhodetails}

\end{document}

 

\documentclass[12pt]{llncs}
%\documentclass{jktr}

\usepackage[pdftex]{hyperref}                   
\usepackage {listings}
\usepackage {mathpartir}
\usepackage{bcprules}
%\usepackage{listings}
                       
\usepackage{graphicx} 
%\usepackage[margins=2.5cm,nohead,nofoot]{geometry}
%\usepackage{geometry}
\usepackage{amsfonts}
\usepackage{amstext}
\usepackage{latexsym}
\usepackage{amssymb}
\usepackage{color}


%\include{myPreamble}
\include{qm2pi.local} 

%\ifpdf
%\usepackage[pdftex]{graphicx}
%\else
%\usepackage{graphicx}
%\fi

 % \ifpdf
%  \usepackage{pdfsync}
%  \if


%\title{Brief Article}
%\author{David F. Snyder}
%\author{L.G. Meredith}

%\address{Dept. of Math., Texas State University--San Marcos, San Marcos, TX 78666}
       
\pagestyle{empty}


\begin{document}

\lstset{language=[Objective]Caml,frame=shadowbox}

\input{qm2pi.front}

% section front matter (end)

\input{qm2pi.intro} 
 
% section introduction (end)

% \input{qm2pi.knotations} 

% section notation (end)

\input{qm2pi.process.calculi} 

% section concurrent_process_calculi_and_spatial_logics_ (end)
    
%\input{qm2pi.knots2pi} 

%\input{qm2pi.trefoil} 

%\input{qm2pi.mainthm} 

% subsection basic_interpretation (end)

%\input{qm2pi.rho.presentation} 
\subsection{The syntax and semantics of the notation system}\label{sub:the_syntax_and_semantics_of_the_notation_system} % (fold)

We now summarize a technical presentation of the calculus that
embodies our theory of dynamics. The typical presentation of such a
calculus follows the style of giving generators and relations on
them. The grammar, below, describing term constructors, freely
generates the set of processes, $\Proc$. This set is then quotiented
by a relation known as structural congruence and it is over this set
that the notion of dynamics is expressed. This presentation is
essentially that of \cite{MeredithR05} with the addition of
polyadicity and summation. For readability we have relegated some of
the technical subtleties to an appendix.

\subsubsection{Process grammar}\label{subsub:process_grammar}

\begin{mathpar}
  \inferrule* [lab=synchronization] {} {{M} \bc \pzero \;|\; x?F \;|\; x!C }
  \and
  \inferrule* [lab=abstraction] {} {{F} \bc (x)P}
  \and
  \inferrule* [lab=concretion] {} {{C} \bc \langle Q \rangle}
  \and
  \inferrule* [lab=process] {} {{P,Q} \bc M \;| \;P|Q \;|\; @{x}}
  \and
  \inferrule* [lab=name] {} {{x} \bc \quotep{P}}
\end{mathpar} 

Note that $\vec{x}$ (resp. $\vec{P}$) denotes a vector of names
(resp. processes) of length $|\vec{x}|$ (resp. $|\vec{P}|$). We adopt
the following useful abbreviations.

\begin{mathpar}
   x?(\vec{y}).P := x.(\vec{y})P \and  x\clift{\vec{P}} := x.\clift{\vec{P}}
   \and x!(y) := \lift{x}{\dropn{y}}
   \and \Pi_{i=0}^{n-1}P_i := P_0 | \ldots | P_{n-1}
\end{mathpar}

\subsubsection{Structural congruence}

\paragraph{Free and bound names and alpha-equivalence.} At the
core of structural equivalence is alpha-equivalence which identifies
process that are the same up to a change of variable. Formally, we
recognize the distinction between free and bound names. The free names
of a process, $\freenames{P}$, may be calculated recursively as
follows:

\begin{mathpar}
\freenames{\pzero} := \emptyset
  \and \\
  \freenames{x?(y).P} := \{ x \} \cup (\freenames{P} \setminus \{ y \})
  \and 
  \freenames{x!\langle P \rangle} := \{ x \} \cup \{ P \} 
  \and \\
  \freenames{P|Q} := \freenames{P} \cup \freenames{Q}
  \and \\
  \freenames{@{x}} := \{ x \}
\end{mathpar}

$\pi$
$\quotep{\pi}$

$\freenames{-} : \pi \to \mathcal{P}(\quotep{\pi})$

\begin{eqnarray*}
  \freenames{\pzero} & := & \emptyset \\
  \freenames{x?(y).P} & := & \{ x \} \cup (\freenames{P} \setminus \{ y \}) \\
  \freenames{x!\langle P \rangle} & := & \{ x \} \cup \{ P \} \\
  \freenames{P|Q} & := & \freenames{P} \cup \freenames{Q} \\
  \freenames{\dropn{x}} & := & \{ x \}
\end{eqnarray*}

The bound names of a process, $\boundnames{P}$, are those names occurring in $P$
that are not free. For example, in $x?(y).0$, the name $x$ is free, while $y$ is bound.

\begin{mathpar}
  \inferrule* [lab=monoidal-laws] {} { P|Q \equiv Q|P \and P|0 \equiv P \and P|(Q|R) \equiv (P|Q)|R }
\end{mathpar}

\begin{mathpar}
  \inferrule* [lab=alpha-equivalence] {} { (x)P \equiv (y)P\{y/x\} \and y \not\in \freenames{P} }
\end{mathpar}

\begin{definition}
Then two processes, $P,Q$, are alpha-equivalent if $P = Q\{\vec{y}/\vec{x}\}$ for
some $\vec{x} \in \boundnames{Q},\vec{y} \in \boundnames{P}$, where $Q\{\vec{y}/\vec{x}\}$
denotes the capture-avoiding substitution of $\vec{y}$ for $\vec{x}$ in $Q$.
\end{definition}

\begin{definition}
  The {\em structural congruence} \cite{SangiorgiWalker} , $\equiv$,
  between processes is the least congruence containing
  alpha-equivalence, satisfying the abelian monoid laws
  (associativity, commutativity and $\pzero$ as identity) for parallel
  composition $|$ and for summation $+$.
\end{definition}

\subsection{Name equivalence}

We take name equivalence, written $\nameeq$, to be the smallest
equivalence relation generated by the following rules.

\begin{mathpar}
\inferrule*[lab=Quote-drop]
{ }
{ \quotep{@{x}} \nameeq x }

\inferrule*[lab=Struct-equiv]
{ P \scong Q }
{ \quotep{P} \nameeq \quotep{Q} }
\end{mathpar}

The astute reader will have noticed that the mutual recursion of names
and processes imposes a mutual recursion on alpha-equivalence and
structural equivalence via name-equivalence. Fortunately, all of this
works out pleasantly and we may calculate in the natural way, free of
concern. The reader interested in the details is referred to the
appendix \ref{appendix:rho_details}.

\subsection{Substitution}

We use $\Proc$ for the set of processes, $\QProc$ for the set of
names, and $\id{\{}\vec{y} / \vec{x} \id{\}}$ to denote partial maps,
$s : \QProc \rightarrow \QProc$. A map, $s$ lifts, uniquely, to a map
on process terms, $\widehat{s} : \Proc \rightarrow \Proc$ by the
following equations.

\begin{mathpar}
  (0) \psubstp{Q}{P} := 0 \\
  (R \juxtap S) \psubstp{Q}{P}
  :=    
  (R)\psubstp{Q}{P} \juxtap (S) \psubstp{Q}{P} \\
  (x?(y).R) \psubstp{Q}{P}    
  :=    
  (x)\substp{Q}{P} (z)\concat( (R \psubstn{z}{y}) \psubstp{Q}{P} ) \\
  (\lift{x}{R}) \psubstp{Q}{P}  
  :=
  \lift{(x)\substp{Q}{P}}{ R \psubstp{Q}{P} } \\
%   (\dropn{x})  \psubstp{Q}{P}       
%   := 
%   \left\{ 
%     \begin{array}{ccc} 
%       \dropn{\quotep{Q}} & & x \nameeq \quotep{P} \\
%       \dropn{x} & & otherwise \\
%     \end{array}
%   \right. 
  (\dropn{x})  \psubstp{Q}{P}       
  := 
  \left\{ 
    \begin{array}{ccc} 
      Q & & x \nameeq \quotep{P} \\
      \dropn{x} & & otherwise \\
    \end{array}
  \right.
\end{mathpar}
 

where

\begin{eqnarray}
  (x)\id{\{} \lpquote Q \rpquote / \lpquote P \rpquote \id{\}}            = 
  \left\{ 
    \begin{array}{ccc}
      \lpquote Q \rpquote & & x \nameeq \lpquote P \rpquote \\
      x & & otherwise \\
    \end{array}
  \right. \nonumber
\end{eqnarray}

and $z$ is chosen distinct from $\quotep{P}$, $\quotep{Q}$, the free
names in $Q$, and all the names in $R$. Our $\alpha$-equivalence will
be built in the standard way from this substitution.

\begin{remark}\label{rem:no_self_referential_names}
  One consequence of these definitions is that $\forall P. \quotep{P}
  \not\in \freenames{P}$.
\end{remark}

\subsection{ Dynamic quote: an example }

Anticipating something of what's to come, consider applying the
substitution, $\widehat{\id{\{}u / z \id{\}}}$, to the following pair
of processes, $\lift{w}{y!(z)}$ and $w[ \lpquote y!(z) \rpquote ]$.

\begin{eqnarray}
	\lift{w}{y!(z)}\widehat{\id{\{}u / z \id{\}}}
		& = &
		\lift{w}{y!(u)} \nonumber\\
	w[ \lpquote y!(z) \rpquote ] \widehat{ \id{\{}u / z \id{\}} }
		& = &
		w[ \lpquote y!(z) \rpquote ] \nonumber
\end{eqnarray}

Because the body of the process between quotes is impervious to
substitution, we get radically different answers. In fact, by
examining the first process in an input context,
e.g. $x?(z).\lift{w}{y!(z)}$, we see that the process under the lift
operator may be shaped by prefixed inputs binding a name inside it. In
this sense, the lift operator will be seen as a way to dynamically
construct processes before reifying them as names.

Finally equipped with these standard features we can present the
dynamics of the calculus.

\subsubsection{Operational semantics} 

Finally, we introduce the computational dynamics. What marks these
algebras as distinct from other more traditionally studied algebraic
structures, e.g. vector spaces or polynomial rings, is the manner in
which dynamics is captured. In traditional structures, dynamics is typically
expressed through morphisms between such structures, as in linear maps
between vector spaces or morphisms between rings. In algebras
associated with the semantics of computation, the dynamics is
expressed as part of the algebraic structure itself, through a
reduction reduction relation typically denoted by $\red$. Below, we
give a recursive presentation of this relation for the calculus used
in the encoding.

$\red \subseteq \pi \times \pi$
$\red : \pi \to \mathcal{P}(\pi)$

\begin{mathpar}
  \inferrule* [lab=Comm] { \textsf{match}( x_{src}, x_{trgt} ) } { x_{trgt}?(y)P \; | \; x_{src}!\langle {Q} \rangle \red P\{\quotep{Q}/y}\} }
  \and \\
  \inferrule* [lab=Par] {{P} \red {P}'} {{{P} | {Q}} \red {{P}' | {Q}}}
  \and
  \inferrule* [lab=Equiv]{{{P} \scong {P}'} \andalso {{P}' \red {Q}'} \andalso {{Q}' \scong {Q}}}{{P} \red {Q}}
\end{mathpar}

\begin{eqnarray*}
  match_{\equiv} (\quotep{P},\quotep{Q}) & := & P \equiv Q \\
  match_{\dagger}(\quotep{P},\quotep{Q}) & := & \forall R. P|Q \red^{*} R => R \red^{*} 0 \\
  match_{K}(\quotep{P},\quotep{Q}) & := & K \mbox{ for some context } K
\end{eqnarray*}

$u?(x)P | u!\langle Q \rangle \red P\{\quotep{Q}/x\}$

%We write $\wred$ for $\red^*$, and $P\red$ if $\exists Q $ such that $ P \red Q$.
We write $P\red$ if $\exists Q $ such that $ P \red Q$ and $P\not\red$, otherwise.

\section{Replication}

As mentioned before, it is known that replication (and hence
recursion) can be implemented in a higher-order process algebra
\cite{SangiorgiWalker}. As our first example of calculation with the
machinery thus far presented we give the construction explicitly in
the {\rhoc}.

\begin{eqnarray}
	D_{x} & := & \prefix{x}{y}{(\binpar{\outputp{x}{y}}{@{y}})} \nonumber\\
	\bangp_{x}{P} & := & \binpar{{x}!\langle{\binpar{D_{x}}{P}}\rangle}{D_{x}} \nonumber
\end{eqnarray}

\begin{eqnarray}
	\bangp_{x}{P} & & \nonumber\\
	=
	& {x}!\langle{(\prefix{x}{y}{(\outputp{x}{y} | @{y})) | P}}\rangle 
	      | \prefix{x}{y}{(\outputp{x}{y} | @{y})} & \nonumber\\
	\red
	& (\outputp{x}{y} | @{y})\substn{\quotep{(\prefix{x}{y}{(@{y} | \outputp{x}{y})) | P}}}{y} & \nonumber\\
	=
	& \outputp{x}{\quotep{(\prefix{x}{y}{(\outputp{x}{y} | @{y})) | P}}}
	  | {(\prefix{x}{y}{(\outputp{x}{y} | @{y})) | P}} & \nonumber\\
	\red
	& \ldots & \nonumber\\
	\red^*
	& P | P | \ldots & \nonumber
\end{eqnarray}

Of course, this encoding, as an implementation, runs away, unfolding
$\bangp{P}$ eagerly. A lazier and more implementable replication
operator, restricted to input-guarded processes, may be obtained as follows.

\begin{eqnarray}
\bangp{\prefix{u}{v}{P}} 
	:= 
	\binpar{\lift{x}{\prefix{u}{v}{(\binpar{D(x)}{P})}}}{D(x)} \nonumber
\end{eqnarray}

\begin{remark}
  Note that the lazier definition still does not deal with summation
  or mixed summation (i.e. sums over input and output). The reader is
  invited to construct definitions of replication that deal with these
  features. 

  Further, the definitions are parameterized in a name, $x$. Can you,
  gentle reader, make a definition that eliminates this parameter and
  guarantees no accidental interaction between the replication
  machinery and the process being replicated -- i.e. no accidental
  sharing of names used by the process to get its work done and the
  name(s) used by the replication to effect copying. This latter
  revision of the definition of replication is crucial to obtaining
  the expected identity $!!P \sim !P$.
\end{remark}

\begin{remark}\label{rem:paradoxical_combinator}
  The reader familiar with the lambda calculus will have noticed the
  similarity between $D$ and the paradoxical combinator.

  [Ed. note: the existence of this seems to suggest we have to be more
  restrictive on the set of processes and names we admit if we are to
  support no-cloning.]
\end{remark}

\subsubsection{Bisimulation}

The computational dynamics gives rise to another kind of equivalence,
the equivalence of computational behavior. As previously mentioned
this is typically captured \emph{via} some form of bisimulation.

% The notion we use in this paper is weak barbed bisimulation
% \cite{milner91polyadicpi}.

The notion we use in this paper is derived from weak barbed
bisimulation \cite{milner91polyadicpi}. 

\begin{definition}
An \emph{observation relation}, $\downarrow_{\mathcal N}$, over a set
of names, $\mathcal N$, is the smallest relation satisfying the rules
below.

\infrule[Out-barb]{y \in {\mathcal N}, \; x \nameeq y}
		  {\outputp{x}{v} \downarrow_{\mathcal N} x}
\infrule[Par-barb]{\mbox{$P\downarrow_{\mathcal N} x$ or $Q\downarrow_{\mathcal N} x$}}
		  {\binpar{P}{Q} \downarrow_{\mathcal N} x}

We write $P \Downarrow_{\mathcal N} x$ if there is $Q$ such that 
$P \wred Q$ and $Q \downarrow_{\mathcal N} x$.
\end{definition}

\begin{definition}
%\label{def.bbisim}
An  ${\mathcal N}$-\emph{barbed bisimulation} over a set of names, ${\mathcal N}$, is a symmetric binary relation 
${\mathcal S}_{\mathcal N}$ between agents such that $P\rel{S}_{\mathcal N}Q$ implies:
\begin{enumerate}
\item If $P \red P'$ then $Q \wred Q'$ and $P'\rel{S}_{\mathcal N} Q'$.
\item If $P\downarrow_{\mathcal N} x$, then $Q\Downarrow_{\mathcal N} x$.
\end{enumerate}
$P$ is ${\mathcal N}$-barbed bisimilar to $Q$, written
$P \wbbisim_{\mathcal N} Q$, if $P \rel{S}_{\mathcal N} Q$ for some ${\mathcal N}$-barbed bisimulation ${\mathcal S}_{\mathcal N}$.
\end{definition}

$\mathcal{R} \subseteq \pi \times \pi$

$P \mathcal{R} Q => \forall P'. P \red P' \Rightarrow \exists Q'. Q \red Q', P' \mathcal{R} Q'$

$P \vdash x \Rightarrow Q \vdash x$

\begin{mathpar}
  \inferrule*[lab=Out-barb]{x \nameeq y}{{y}!\langle{Q}\rangle \vdash x}
  \and
  \inferrule*[lab=Par-barb]{\mbox{$P\vdash x$ or $Q\vdash x$}}{\binpar{P}{Q} \vdash x}
\end{mathpar}

\subsubsection{Contexts}

One of the principle advantages of computational calculi like the
$\pi$-calculus is a well-defined notion of context,
contextual-equivalence and a correlation between
contextual-equivalence and notions of bisimulation. The notion of
context allows the decomposition of a process into (sub-)process and
its syntactic environment, its context. Thus, a context may be
thought of as a process with a ``hole'' (written $\Box$) in it. The
application of a context $M$ to a process $P$, written $M[P]$, is
tantamount to filling the hole in $M$ with $P$. In this paper we do
not need the full weight of this theory, but do make use of the notion
of context in the proof the main theorem. 

\begin{mathpar}
  \inferrule* [lab=summation] {} {{M_{M},M_{N}} \bc \Box \;|\; x.M_{A} \;|\; M_{M}+M_{N}}
  \and
  \inferrule* [lab=agent] {} {{M_{A}} \bc (\vec{x})M_{P} \;| \; \clift{P_0,\ldots,M_{P},\ldots,P_N}}
  \and \\
  \inferrule* [lab=process] {} {{M_{P}} \bc M_{N} \;| \;P|M_{P} }
\end{mathpar} 

\begin{mathpar}
  \inferrule* [lab=sychronization] {} {M_{N} \bc \Box \;|\; x?M_{F} \;|\; x!M_{C}}
  \and
  \inferrule* [lab=abstraction] {} {{M_{F}} \bc (x)M_{P} }
  \and
  \inferrule* [lab=concretion] {} {{M_{C}} \bc \langle M_{P} \rangle }
  \and \\
  \inferrule* [lab=process] {} {{M_{P}} \bc M_{N} \;| \;P|M_{P} }
\end{mathpar}

\begin{definition}[contextual application] Given a context $M$, and
  process $P$, we define the \emph{contextual application}, $M[P] :=
  M\{P/\Box\}$. That is, the contextual application of M to P is the
  substitution of $P$ for $\Box$ in $M$.
\end{definition}

$\meaningof{-} : L \to \mathcal{P}(\pi)$

\begin{mathpar}
  \inferrule* [lab=collection] {} {\meaningof{true} = \pi, \and \meaningof{~E} = \pi \setminus \meaningof{E}, \and \meaningof{E_{1} \& E_{2}} = \meaningof{E_{1}} \cap \meaningof{E_{2}}}
\end{mathpar}

\begin{mathpar}
  \inferrule* [lab=structure] {} {\meaningof{0} = \{ P \in \pi | P \equiv 0 \}, \and \\ \meaningof{E_1 | E_2} = \{ P \in \pi | P \equiv P_{1} | P_{2}, P_{1} \in \meaningof{E_{1}}, P_{2} \in \meaningof{E_2}\} }
\end{mathpar}

\begin{mathpar}
 \inferrule* [lab=behavior] {} {\meaningof{\langle a?b \rangle E} = \{ P \in \pi | P \equiv Q | u?(y)P', \\ \and \\\\ \and \\ \;\;\; u \in \meaningof{a}, \forall z.P'\{z/y\} \in \meaningof{E\{z/b\}}\}, \and \\ \meaningof{a!E} = \{ P \in \pi | P \equiv Q | x!\langle P' \rangle, x \in \meaningof{a} P' \in \meaningof{E}\} }
\end{mathpar}

\begin{mathpar}
 \inferrule* [lab=nominal] {} {\meaningof{\quotep{E}} = \{ \quotep{P} \in \quotep{\pi} | P \in \meaningof{E} \}, \and \meaningof{\quotep{P}} = \{ \quotep{Q} \in \quotep{\pi} | P \equiv Q \} \and \\ \meaningof{@\quotep{E}} = \{ P \in \pi | P \equiv @x, x \in \meaningof{E} \}}
\end{mathpar}

\begin{eqnarray*}
  \\
  \meaningof{-} : TS \to ST
\end{eqnarray*}

\begin{eqnarray*}
  \\
  L : TS \to ST
\end{eqnarray*}

\begin{eqnarray*}
  \\
  P \models E \iff P \in \meaningof{E}
\end{eqnarray*}

\begin{eqnarray*}
  P \approx_{L} Q \iff \forall E \in L. P \models E \iff Q \models E
\end{eqnarray*}

\begin{eqnarray*}
  P \approx_{K} Q
\end{eqnarray*}

\begin{eqnarray*}
  P \approx Q
\end{eqnarray*}

$\approx_{K} = \approx = \approx_{L}$

\subsubsection{Contextual duality}

Note that contexts extend the quotation operation to a family of
operations from processes to names. Given a context, $M$, we can
define a \emph{nominal context}, $\quotep{M}$ by $\quotep{M}[P] :=
\quotep{M[P]}$. To foreshadow what is to come we observe that these
operations enjoy a duality with processes very much like the duality
between vectors and maps from vectors to scalars.

Further, because the calculus is essentially higher-order, we have a
correspondence between contexts and processes. More specifically,
given a name $x$ and a context $M$ we can construct $M^{*}_{x}$ such
that 

\begin{mathpar}
  M^{*}_{x} | \lift{x}{P} \red M[P]
\end{mathpar}

namely,

\begin{mathpar}
  M^{*}_{x} := x?(u).M[\dropn{u}]
\end{mathpar}

The dependence of $M^{*}_{x}$ on a name makes it an abstraction, 

\begin{mathpar}
  M^{*} := (x)x?(u).M[\dropn{u}]
\end{mathpar}

\subsection{Additional notation}

It will sometimes be convenient to denote the process a name
quotes. We already have the notation $x = \quotep{P}$, but it will be
convenient to introduce an alternate notation, $\procn{x}$, when we
want to emphasize the connection to the use of the name. Note that, by
virtue of name equivalence, $\quotep{\procn{x}} \nameeq x$; so, the
notation is consistent with previous definitions.

Further, because names have structure it is possible to effect
substitutions on the basis of that structure. This means we need to
upgrade our notation for substitutions, which we accomplish by
adapting comprehension notation. Thus,

\begin{mathpar}
  P\{ y / x : x \in S \}
\end{mathpar}

is interpreted to mean the process derived from P by replacing (in a
capture-avoiding manner) each occurrence of $x$ in $S$ by $y$. For example,

\begin{mathpar}
  P\{ \quotep{\procn{x}|\procn{x}} / x : x \in \freenames{P} \}
\end{mathpar}

will replace each (occurrence) of a free name $x$ in $P$ by
$\quotep{\procn{x}|\procn{x}}$.

Also, we will avail ourselves of the notation $x^{L}$ and $x^{R}$ to
denote injections of a name into disjoint copies of the name
space. There are numerous ways to accomplish this. One example can be
found in \cite{MeredithR05}. This notation overloads to vectors of
names: $\vec{x}^{\pi} := (x_{i}^{\pi} \; : \; 0 \leq i < |\vec{x}| )$ where $\pi \in \{L,R\}$.

We also use $P^{\Box} := P|\Box$.

In \cite{MeredithR05} an interpretation of the new operator is
given. It turns out that there are several possible interpretations
all enjoying the requisite algebraic properties of the operator (see
\cite{milner91polyadicpi}). We will therefore make liberal use of
$(\nu\; \vec{x})P$.

% subsection the_syntax_and_semantics_of_the_notation_system (end)   

\input{qm2pi.qmops} 

\input{qm2pi.sterngerlach} 

\input{qm2pi.metric} 

% section concurrent_process_calculi (end)

%\input{qm2pi.proofsketch}

% section proof sketch (end)

%\input{qm2pi.slviaknots} 

% section spatial logic via knots (end)

\input{qm2pi.conclusion}

% section conclusion (end)

%\input{qm2pi.dtcodes} 

% section wiring algorithm (end)

\input{qm2pi.ack} 

% section acknowledgments (end)

\newpage


\bibliographystyle{plain}   
\bibliography{../../biblios/main.bib}

\input{qm2pi.rhodetails}

\end{document}

 

% section concurrent_process_calculi (end)

%\documentclass[12pt]{llncs}
%\documentclass{jktr}

\usepackage[pdftex]{hyperref}                   
\usepackage {listings}
\usepackage {mathpartir}
\usepackage{bcprules}
%\usepackage{listings}
                       
\usepackage{graphicx} 
%\usepackage[margins=2.5cm,nohead,nofoot]{geometry}
%\usepackage{geometry}
\usepackage{amsfonts}
\usepackage{amstext}
\usepackage{latexsym}
\usepackage{amssymb}
\usepackage{color}


%\include{myPreamble}
\include{qm2pi.local} 

%\ifpdf
%\usepackage[pdftex]{graphicx}
%\else
%\usepackage{graphicx}
%\fi

 % \ifpdf
%  \usepackage{pdfsync}
%  \if


%\title{Brief Article}
%\author{David F. Snyder}
%\author{L.G. Meredith}

%\address{Dept. of Math., Texas State University--San Marcos, San Marcos, TX 78666}
       
\pagestyle{empty}


\begin{document}

\lstset{language=[Objective]Caml,frame=shadowbox}

\input{qm2pi.front}

% section front matter (end)

\input{qm2pi.intro} 
 
% section introduction (end)

% \input{qm2pi.knotations} 

% section notation (end)

\input{qm2pi.process.calculi} 

% section concurrent_process_calculi_and_spatial_logics_ (end)
    
%\input{qm2pi.knots2pi} 

%\input{qm2pi.trefoil} 

%\input{qm2pi.mainthm} 

% subsection basic_interpretation (end)

%\input{qm2pi.rho.presentation} 
\subsection{The syntax and semantics of the notation system}\label{sub:the_syntax_and_semantics_of_the_notation_system} % (fold)

We now summarize a technical presentation of the calculus that
embodies our theory of dynamics. The typical presentation of such a
calculus follows the style of giving generators and relations on
them. The grammar, below, describing term constructors, freely
generates the set of processes, $\Proc$. This set is then quotiented
by a relation known as structural congruence and it is over this set
that the notion of dynamics is expressed. This presentation is
essentially that of \cite{MeredithR05} with the addition of
polyadicity and summation. For readability we have relegated some of
the technical subtleties to an appendix.

\subsubsection{Process grammar}\label{subsub:process_grammar}

\begin{mathpar}
  \inferrule* [lab=synchronization] {} {{M} \bc \pzero \;|\; x?F \;|\; x!C }
  \and
  \inferrule* [lab=abstraction] {} {{F} \bc (x)P}
  \and
  \inferrule* [lab=concretion] {} {{C} \bc \langle Q \rangle}
  \and
  \inferrule* [lab=process] {} {{P,Q} \bc M \;| \;P|Q \;|\; @{x}}
  \and
  \inferrule* [lab=name] {} {{x} \bc \quotep{P}}
\end{mathpar} 

Note that $\vec{x}$ (resp. $\vec{P}$) denotes a vector of names
(resp. processes) of length $|\vec{x}|$ (resp. $|\vec{P}|$). We adopt
the following useful abbreviations.

\begin{mathpar}
   x?(\vec{y}).P := x.(\vec{y})P \and  x\clift{\vec{P}} := x.\clift{\vec{P}}
   \and x!(y) := \lift{x}{\dropn{y}}
   \and \Pi_{i=0}^{n-1}P_i := P_0 | \ldots | P_{n-1}
\end{mathpar}

\subsubsection{Structural congruence}

\paragraph{Free and bound names and alpha-equivalence.} At the
core of structural equivalence is alpha-equivalence which identifies
process that are the same up to a change of variable. Formally, we
recognize the distinction between free and bound names. The free names
of a process, $\freenames{P}$, may be calculated recursively as
follows:

\begin{mathpar}
\freenames{\pzero} := \emptyset
  \and \\
  \freenames{x?(y).P} := \{ x \} \cup (\freenames{P} \setminus \{ y \})
  \and 
  \freenames{x!\langle P \rangle} := \{ x \} \cup \{ P \} 
  \and \\
  \freenames{P|Q} := \freenames{P} \cup \freenames{Q}
  \and \\
  \freenames{@{x}} := \{ x \}
\end{mathpar}

$\pi$
$\quotep{\pi}$

$\freenames{-} : \pi \to \mathcal{P}(\quotep{\pi})$

\begin{eqnarray*}
  \freenames{\pzero} & := & \emptyset \\
  \freenames{x?(y).P} & := & \{ x \} \cup (\freenames{P} \setminus \{ y \}) \\
  \freenames{x!\langle P \rangle} & := & \{ x \} \cup \{ P \} \\
  \freenames{P|Q} & := & \freenames{P} \cup \freenames{Q} \\
  \freenames{\dropn{x}} & := & \{ x \}
\end{eqnarray*}

The bound names of a process, $\boundnames{P}$, are those names occurring in $P$
that are not free. For example, in $x?(y).0$, the name $x$ is free, while $y$ is bound.

\begin{mathpar}
  \inferrule* [lab=monoidal-laws] {} { P|Q \equiv Q|P \and P|0 \equiv P \and P|(Q|R) \equiv (P|Q)|R }
\end{mathpar}

\begin{mathpar}
  \inferrule* [lab=alpha-equivalence] {} { (x)P \equiv (y)P\{y/x\} \and y \not\in \freenames{P} }
\end{mathpar}

\begin{definition}
Then two processes, $P,Q$, are alpha-equivalent if $P = Q\{\vec{y}/\vec{x}\}$ for
some $\vec{x} \in \boundnames{Q},\vec{y} \in \boundnames{P}$, where $Q\{\vec{y}/\vec{x}\}$
denotes the capture-avoiding substitution of $\vec{y}$ for $\vec{x}$ in $Q$.
\end{definition}

\begin{definition}
  The {\em structural congruence} \cite{SangiorgiWalker} , $\equiv$,
  between processes is the least congruence containing
  alpha-equivalence, satisfying the abelian monoid laws
  (associativity, commutativity and $\pzero$ as identity) for parallel
  composition $|$ and for summation $+$.
\end{definition}

\subsection{Name equivalence}

We take name equivalence, written $\nameeq$, to be the smallest
equivalence relation generated by the following rules.

\begin{mathpar}
\inferrule*[lab=Quote-drop]
{ }
{ \quotep{@{x}} \nameeq x }

\inferrule*[lab=Struct-equiv]
{ P \scong Q }
{ \quotep{P} \nameeq \quotep{Q} }
\end{mathpar}

The astute reader will have noticed that the mutual recursion of names
and processes imposes a mutual recursion on alpha-equivalence and
structural equivalence via name-equivalence. Fortunately, all of this
works out pleasantly and we may calculate in the natural way, free of
concern. The reader interested in the details is referred to the
appendix \ref{appendix:rho_details}.

\subsection{Substitution}

We use $\Proc$ for the set of processes, $\QProc$ for the set of
names, and $\id{\{}\vec{y} / \vec{x} \id{\}}$ to denote partial maps,
$s : \QProc \rightarrow \QProc$. A map, $s$ lifts, uniquely, to a map
on process terms, $\widehat{s} : \Proc \rightarrow \Proc$ by the
following equations.

\begin{mathpar}
  (0) \psubstp{Q}{P} := 0 \\
  (R \juxtap S) \psubstp{Q}{P}
  :=    
  (R)\psubstp{Q}{P} \juxtap (S) \psubstp{Q}{P} \\
  (x?(y).R) \psubstp{Q}{P}    
  :=    
  (x)\substp{Q}{P} (z)\concat( (R \psubstn{z}{y}) \psubstp{Q}{P} ) \\
  (\lift{x}{R}) \psubstp{Q}{P}  
  :=
  \lift{(x)\substp{Q}{P}}{ R \psubstp{Q}{P} } \\
%   (\dropn{x})  \psubstp{Q}{P}       
%   := 
%   \left\{ 
%     \begin{array}{ccc} 
%       \dropn{\quotep{Q}} & & x \nameeq \quotep{P} \\
%       \dropn{x} & & otherwise \\
%     \end{array}
%   \right. 
  (\dropn{x})  \psubstp{Q}{P}       
  := 
  \left\{ 
    \begin{array}{ccc} 
      Q & & x \nameeq \quotep{P} \\
      \dropn{x} & & otherwise \\
    \end{array}
  \right.
\end{mathpar}
 

where

\begin{eqnarray}
  (x)\id{\{} \lpquote Q \rpquote / \lpquote P \rpquote \id{\}}            = 
  \left\{ 
    \begin{array}{ccc}
      \lpquote Q \rpquote & & x \nameeq \lpquote P \rpquote \\
      x & & otherwise \\
    \end{array}
  \right. \nonumber
\end{eqnarray}

and $z$ is chosen distinct from $\quotep{P}$, $\quotep{Q}$, the free
names in $Q$, and all the names in $R$. Our $\alpha$-equivalence will
be built in the standard way from this substitution.

\begin{remark}\label{rem:no_self_referential_names}
  One consequence of these definitions is that $\forall P. \quotep{P}
  \not\in \freenames{P}$.
\end{remark}

\subsection{ Dynamic quote: an example }

Anticipating something of what's to come, consider applying the
substitution, $\widehat{\id{\{}u / z \id{\}}}$, to the following pair
of processes, $\lift{w}{y!(z)}$ and $w[ \lpquote y!(z) \rpquote ]$.

\begin{eqnarray}
	\lift{w}{y!(z)}\widehat{\id{\{}u / z \id{\}}}
		& = &
		\lift{w}{y!(u)} \nonumber\\
	w[ \lpquote y!(z) \rpquote ] \widehat{ \id{\{}u / z \id{\}} }
		& = &
		w[ \lpquote y!(z) \rpquote ] \nonumber
\end{eqnarray}

Because the body of the process between quotes is impervious to
substitution, we get radically different answers. In fact, by
examining the first process in an input context,
e.g. $x?(z).\lift{w}{y!(z)}$, we see that the process under the lift
operator may be shaped by prefixed inputs binding a name inside it. In
this sense, the lift operator will be seen as a way to dynamically
construct processes before reifying them as names.

Finally equipped with these standard features we can present the
dynamics of the calculus.

\subsubsection{Operational semantics} 

Finally, we introduce the computational dynamics. What marks these
algebras as distinct from other more traditionally studied algebraic
structures, e.g. vector spaces or polynomial rings, is the manner in
which dynamics is captured. In traditional structures, dynamics is typically
expressed through morphisms between such structures, as in linear maps
between vector spaces or morphisms between rings. In algebras
associated with the semantics of computation, the dynamics is
expressed as part of the algebraic structure itself, through a
reduction reduction relation typically denoted by $\red$. Below, we
give a recursive presentation of this relation for the calculus used
in the encoding.

$\red \subseteq \pi \times \pi$
$\red : \pi \to \mathcal{P}(\pi)$

\begin{mathpar}
  \inferrule* [lab=Comm] { \textsf{match}( x_{src}, x_{trgt} ) } { x_{trgt}?(y)P \; | \; x_{src}!\langle {Q} \rangle \red P\{\quotep{Q}/y}\} }
  \and \\
  \inferrule* [lab=Par] {{P} \red {P}'} {{{P} | {Q}} \red {{P}' | {Q}}}
  \and
  \inferrule* [lab=Equiv]{{{P} \scong {P}'} \andalso {{P}' \red {Q}'} \andalso {{Q}' \scong {Q}}}{{P} \red {Q}}
\end{mathpar}

\begin{eqnarray*}
  match_{\equiv} (\quotep{P},\quotep{Q}) & := & P \equiv Q \\
  match_{\dagger}(\quotep{P},\quotep{Q}) & := & \forall R. P|Q \red^{*} R => R \red^{*} 0 \\
  match_{K}(\quotep{P},\quotep{Q}) & := & K \mbox{ for some context } K
\end{eqnarray*}

$u?(x)P | u!\langle Q \rangle \red P\{\quotep{Q}/x\}$

%We write $\wred$ for $\red^*$, and $P\red$ if $\exists Q $ such that $ P \red Q$.
We write $P\red$ if $\exists Q $ such that $ P \red Q$ and $P\not\red$, otherwise.

\section{Replication}

As mentioned before, it is known that replication (and hence
recursion) can be implemented in a higher-order process algebra
\cite{SangiorgiWalker}. As our first example of calculation with the
machinery thus far presented we give the construction explicitly in
the {\rhoc}.

\begin{eqnarray}
	D_{x} & := & \prefix{x}{y}{(\binpar{\outputp{x}{y}}{@{y}})} \nonumber\\
	\bangp_{x}{P} & := & \binpar{{x}!\langle{\binpar{D_{x}}{P}}\rangle}{D_{x}} \nonumber
\end{eqnarray}

\begin{eqnarray}
	\bangp_{x}{P} & & \nonumber\\
	=
	& {x}!\langle{(\prefix{x}{y}{(\outputp{x}{y} | @{y})) | P}}\rangle 
	      | \prefix{x}{y}{(\outputp{x}{y} | @{y})} & \nonumber\\
	\red
	& (\outputp{x}{y} | @{y})\substn{\quotep{(\prefix{x}{y}{(@{y} | \outputp{x}{y})) | P}}}{y} & \nonumber\\
	=
	& \outputp{x}{\quotep{(\prefix{x}{y}{(\outputp{x}{y} | @{y})) | P}}}
	  | {(\prefix{x}{y}{(\outputp{x}{y} | @{y})) | P}} & \nonumber\\
	\red
	& \ldots & \nonumber\\
	\red^*
	& P | P | \ldots & \nonumber
\end{eqnarray}

Of course, this encoding, as an implementation, runs away, unfolding
$\bangp{P}$ eagerly. A lazier and more implementable replication
operator, restricted to input-guarded processes, may be obtained as follows.

\begin{eqnarray}
\bangp{\prefix{u}{v}{P}} 
	:= 
	\binpar{\lift{x}{\prefix{u}{v}{(\binpar{D(x)}{P})}}}{D(x)} \nonumber
\end{eqnarray}

\begin{remark}
  Note that the lazier definition still does not deal with summation
  or mixed summation (i.e. sums over input and output). The reader is
  invited to construct definitions of replication that deal with these
  features. 

  Further, the definitions are parameterized in a name, $x$. Can you,
  gentle reader, make a definition that eliminates this parameter and
  guarantees no accidental interaction between the replication
  machinery and the process being replicated -- i.e. no accidental
  sharing of names used by the process to get its work done and the
  name(s) used by the replication to effect copying. This latter
  revision of the definition of replication is crucial to obtaining
  the expected identity $!!P \sim !P$.
\end{remark}

\begin{remark}\label{rem:paradoxical_combinator}
  The reader familiar with the lambda calculus will have noticed the
  similarity between $D$ and the paradoxical combinator.

  [Ed. note: the existence of this seems to suggest we have to be more
  restrictive on the set of processes and names we admit if we are to
  support no-cloning.]
\end{remark}

\subsubsection{Bisimulation}

The computational dynamics gives rise to another kind of equivalence,
the equivalence of computational behavior. As previously mentioned
this is typically captured \emph{via} some form of bisimulation.

% The notion we use in this paper is weak barbed bisimulation
% \cite{milner91polyadicpi}.

The notion we use in this paper is derived from weak barbed
bisimulation \cite{milner91polyadicpi}. 

\begin{definition}
An \emph{observation relation}, $\downarrow_{\mathcal N}$, over a set
of names, $\mathcal N$, is the smallest relation satisfying the rules
below.

\infrule[Out-barb]{y \in {\mathcal N}, \; x \nameeq y}
		  {\outputp{x}{v} \downarrow_{\mathcal N} x}
\infrule[Par-barb]{\mbox{$P\downarrow_{\mathcal N} x$ or $Q\downarrow_{\mathcal N} x$}}
		  {\binpar{P}{Q} \downarrow_{\mathcal N} x}

We write $P \Downarrow_{\mathcal N} x$ if there is $Q$ such that 
$P \wred Q$ and $Q \downarrow_{\mathcal N} x$.
\end{definition}

\begin{definition}
%\label{def.bbisim}
An  ${\mathcal N}$-\emph{barbed bisimulation} over a set of names, ${\mathcal N}$, is a symmetric binary relation 
${\mathcal S}_{\mathcal N}$ between agents such that $P\rel{S}_{\mathcal N}Q$ implies:
\begin{enumerate}
\item If $P \red P'$ then $Q \wred Q'$ and $P'\rel{S}_{\mathcal N} Q'$.
\item If $P\downarrow_{\mathcal N} x$, then $Q\Downarrow_{\mathcal N} x$.
\end{enumerate}
$P$ is ${\mathcal N}$-barbed bisimilar to $Q$, written
$P \wbbisim_{\mathcal N} Q$, if $P \rel{S}_{\mathcal N} Q$ for some ${\mathcal N}$-barbed bisimulation ${\mathcal S}_{\mathcal N}$.
\end{definition}

$\mathcal{R} \subseteq \pi \times \pi$

$P \mathcal{R} Q => \forall P'. P \red P' \Rightarrow \exists Q'. Q \red Q', P' \mathcal{R} Q'$

$P \vdash x \Rightarrow Q \vdash x$

\begin{mathpar}
  \inferrule*[lab=Out-barb]{x \nameeq y}{{y}!\langle{Q}\rangle \vdash x}
  \and
  \inferrule*[lab=Par-barb]{\mbox{$P\vdash x$ or $Q\vdash x$}}{\binpar{P}{Q} \vdash x}
\end{mathpar}

\subsubsection{Contexts}

One of the principle advantages of computational calculi like the
$\pi$-calculus is a well-defined notion of context,
contextual-equivalence and a correlation between
contextual-equivalence and notions of bisimulation. The notion of
context allows the decomposition of a process into (sub-)process and
its syntactic environment, its context. Thus, a context may be
thought of as a process with a ``hole'' (written $\Box$) in it. The
application of a context $M$ to a process $P$, written $M[P]$, is
tantamount to filling the hole in $M$ with $P$. In this paper we do
not need the full weight of this theory, but do make use of the notion
of context in the proof the main theorem. 

\begin{mathpar}
  \inferrule* [lab=summation] {} {{M_{M},M_{N}} \bc \Box \;|\; x.M_{A} \;|\; M_{M}+M_{N}}
  \and
  \inferrule* [lab=agent] {} {{M_{A}} \bc (\vec{x})M_{P} \;| \; \clift{P_0,\ldots,M_{P},\ldots,P_N}}
  \and \\
  \inferrule* [lab=process] {} {{M_{P}} \bc M_{N} \;| \;P|M_{P} }
\end{mathpar} 

\begin{mathpar}
  \inferrule* [lab=sychronization] {} {M_{N} \bc \Box \;|\; x?M_{F} \;|\; x!M_{C}}
  \and
  \inferrule* [lab=abstraction] {} {{M_{F}} \bc (x)M_{P} }
  \and
  \inferrule* [lab=concretion] {} {{M_{C}} \bc \langle M_{P} \rangle }
  \and \\
  \inferrule* [lab=process] {} {{M_{P}} \bc M_{N} \;| \;P|M_{P} }
\end{mathpar}

\begin{definition}[contextual application] Given a context $M$, and
  process $P$, we define the \emph{contextual application}, $M[P] :=
  M\{P/\Box\}$. That is, the contextual application of M to P is the
  substitution of $P$ for $\Box$ in $M$.
\end{definition}

$\meaningof{-} : L \to \mathcal{P}(\pi)$

\begin{mathpar}
  \inferrule* [lab=collection] {} {\meaningof{true} = \pi, \and \meaningof{~E} = \pi \setminus \meaningof{E}, \and \meaningof{E_{1} \& E_{2}} = \meaningof{E_{1}} \cap \meaningof{E_{2}}}
\end{mathpar}

\begin{mathpar}
  \inferrule* [lab=structure] {} {\meaningof{0} = \{ P \in \pi | P \equiv 0 \}, \and \\ \meaningof{E_1 | E_2} = \{ P \in \pi | P \equiv P_{1} | P_{2}, P_{1} \in \meaningof{E_{1}}, P_{2} \in \meaningof{E_2}\} }
\end{mathpar}

\begin{mathpar}
 \inferrule* [lab=behavior] {} {\meaningof{\langle a?b \rangle E} = \{ P \in \pi | P \equiv Q | u?(y)P', \\ \and \\\\ \and \\ \;\;\; u \in \meaningof{a}, \forall z.P'\{z/y\} \in \meaningof{E\{z/b\}}\}, \and \\ \meaningof{a!E} = \{ P \in \pi | P \equiv Q | x!\langle P' \rangle, x \in \meaningof{a} P' \in \meaningof{E}\} }
\end{mathpar}

\begin{mathpar}
 \inferrule* [lab=nominal] {} {\meaningof{\quotep{E}} = \{ \quotep{P} \in \quotep{\pi} | P \in \meaningof{E} \}, \and \meaningof{\quotep{P}} = \{ \quotep{Q} \in \quotep{\pi} | P \equiv Q \} \and \\ \meaningof{@\quotep{E}} = \{ P \in \pi | P \equiv @x, x \in \meaningof{E} \}}
\end{mathpar}

\begin{eqnarray*}
  \\
  \meaningof{-} : TS \to ST
\end{eqnarray*}

\begin{eqnarray*}
  \\
  L : TS \to ST
\end{eqnarray*}

\begin{eqnarray*}
  \\
  P \models E \iff P \in \meaningof{E}
\end{eqnarray*}

\begin{eqnarray*}
  P \approx_{L} Q \iff \forall E \in L. P \models E \iff Q \models E
\end{eqnarray*}

\begin{eqnarray*}
  P \approx_{K} Q
\end{eqnarray*}

\begin{eqnarray*}
  P \approx Q
\end{eqnarray*}

$\approx_{K} = \approx = \approx_{L}$

\subsubsection{Contextual duality}

Note that contexts extend the quotation operation to a family of
operations from processes to names. Given a context, $M$, we can
define a \emph{nominal context}, $\quotep{M}$ by $\quotep{M}[P] :=
\quotep{M[P]}$. To foreshadow what is to come we observe that these
operations enjoy a duality with processes very much like the duality
between vectors and maps from vectors to scalars.

Further, because the calculus is essentially higher-order, we have a
correspondence between contexts and processes. More specifically,
given a name $x$ and a context $M$ we can construct $M^{*}_{x}$ such
that 

\begin{mathpar}
  M^{*}_{x} | \lift{x}{P} \red M[P]
\end{mathpar}

namely,

\begin{mathpar}
  M^{*}_{x} := x?(u).M[\dropn{u}]
\end{mathpar}

The dependence of $M^{*}_{x}$ on a name makes it an abstraction, 

\begin{mathpar}
  M^{*} := (x)x?(u).M[\dropn{u}]
\end{mathpar}

\subsection{Additional notation}

It will sometimes be convenient to denote the process a name
quotes. We already have the notation $x = \quotep{P}$, but it will be
convenient to introduce an alternate notation, $\procn{x}$, when we
want to emphasize the connection to the use of the name. Note that, by
virtue of name equivalence, $\quotep{\procn{x}} \nameeq x$; so, the
notation is consistent with previous definitions.

Further, because names have structure it is possible to effect
substitutions on the basis of that structure. This means we need to
upgrade our notation for substitutions, which we accomplish by
adapting comprehension notation. Thus,

\begin{mathpar}
  P\{ y / x : x \in S \}
\end{mathpar}

is interpreted to mean the process derived from P by replacing (in a
capture-avoiding manner) each occurrence of $x$ in $S$ by $y$. For example,

\begin{mathpar}
  P\{ \quotep{\procn{x}|\procn{x}} / x : x \in \freenames{P} \}
\end{mathpar}

will replace each (occurrence) of a free name $x$ in $P$ by
$\quotep{\procn{x}|\procn{x}}$.

Also, we will avail ourselves of the notation $x^{L}$ and $x^{R}$ to
denote injections of a name into disjoint copies of the name
space. There are numerous ways to accomplish this. One example can be
found in \cite{MeredithR05}. This notation overloads to vectors of
names: $\vec{x}^{\pi} := (x_{i}^{\pi} \; : \; 0 \leq i < |\vec{x}| )$ where $\pi \in \{L,R\}$.

We also use $P^{\Box} := P|\Box$.

In \cite{MeredithR05} an interpretation of the new operator is
given. It turns out that there are several possible interpretations
all enjoying the requisite algebraic properties of the operator (see
\cite{milner91polyadicpi}). We will therefore make liberal use of
$(\nu\; \vec{x})P$.

% subsection the_syntax_and_semantics_of_the_notation_system (end)   

\input{qm2pi.qmops} 

\input{qm2pi.sterngerlach} 

\input{qm2pi.metric} 

% section concurrent_process_calculi (end)

%\input{qm2pi.proofsketch}

% section proof sketch (end)

%\input{qm2pi.slviaknots} 

% section spatial logic via knots (end)

\input{qm2pi.conclusion}

% section conclusion (end)

%\input{qm2pi.dtcodes} 

% section wiring algorithm (end)

\input{qm2pi.ack} 

% section acknowledgments (end)

\newpage


\bibliographystyle{plain}   
\bibliography{../../biblios/main.bib}

\input{qm2pi.rhodetails}

\end{document}



% section proof sketch (end)

%\section{Unlikely characters: spatial logic for
  knots}\label{sub:characteristic_formulae} % (fold)

Associated to the mobile process calculi are a family of logics known
as the Hennessy-Milner logics. These logics typically enjoy a
semantics interpreting formulae as sets of processes that when
factored through the encoding outlined above allows an identification
of classes of knots with logical formulae. In the context of this
encoding the sub-family known as the spatial logics \cite{CairesC03}
\cite{CairesC04} \cite{Caires04} are of particular interest providing
several important features for expressing and reasoning about
properties (i.e. classes) of knots. We hint here at how this may be done.

%\begin{description}
%\item [structural connectives] 
\subsubsection{Structural connectives} The spatial logics enjoy
structural connectives corresponding, at the logical level, to the
parallel composition ($P | Q$) and new name ($(\nu \; x)P$)
connectives for processes. As illustrated in the examples below, these
connectives are extremely expressive given the shape of our encoding.
%\item [decideable satisfaction]

\subsubsection{Decideable satisfaction}
In \cite{Caires04} the satisfaction relation is shown to be decideable
for a rich class of processes. It further turns out that the image of
the our encoding is a proper subset of that class. This result
provides the basis for an algorithm by which to search for knots
enjoying a given property.
%\item [characteristic formulae]

\subsubsection{Characteristic formulae}
In the same paper \cite{Caires04} , Caires presents a means of calculating
characteristic formulae, selecting equivalence classes of processes
up to a pre--specified depth limit on the support set of names. Composed with our
encoding, this characteristic formula can be used to select
characteristic formulae for knots.
%\end{description}

\subsubsection{Spatial logic formulae}

The grammar below (segmented for comprehension) summarizes the syntax
of spatial logic formulae. We employ illustrative examples in the
sequel to provide an intuitive understanding of their meaning
referring the reader to \cite{Caires04} for a more detailed explication
of the semantics.

\begin{mathpar}
  \inferrule* [lab=boolean] {} {{A,B} \bc T \;|\; \neg A \;|\; A \wedge B \;|\; \eta = \eta'}
  \and
  \inferrule* [lab=spatial] {} {|\; \pzero \;|\; A | B \;|\; x \text{\textregistered} A \;|\; \forall x . A \;|\;  H x . A}
  \and
  \inferrule* [lab=behavioral] {} {|\; \alpha . A}
  \and 
  \inferrule* [lab=recursion] {} {|\; X(\vec{u}) \;|\; \mu X(\vec{u}) . A}
  \and
  \inferrule* [lab=action] {} {\alpha \bc \langle x?(\vec{y}) \rangle \;|\; \langle x!(\vec{y}) \rangle \;|\; \langle \tau \rangle}
  \and 
  \inferrule* [lab=name] {} {\eta \bc x \;|\; \tau}
\end{mathpar} 

% subsection characteristic_formulae (end)   	 

\subsection{Example formulae}\label{sub:example_formulae_} % (fold)

\subsubsection{Crossing as formula.}
% 
% \begin{align*}
%   \frac{d}{dx} \sin x &= \cos x 
%   & \frac{d}{dx} e^x &= e^x \\
%   \frac{d}{dx} \cos x &= - \sin x 
%   & \frac{d}{dx} \log x &= \frac{1}{x} \\
% \end{align*} 

\begin{align*}
 \mu C(x_{0},x_{1},y_{0},y_{1},u).&(\langle x_{0}?(z) \rangle(\langle u! \rangle\langle y_{1}!z \rangle C(x_{0},x_{1},y_{0},y_{1},u)) & \\
  & \wedge \langle y_{1}?(z) \rangle (\langle u! \rangle \langle x_{0}!z \rangle C(x_{0},x_{1},y_{0},y_{1},u)) & \\
  & \wedge \langle x_{1}?(z) \rangle (\langle u? \rangle \langle y_{0}!z \rangle C(x_{0},x_{1},y_{0},y_{1},u)) & \\
  & \wedge \langle y_{0}?(z) \rangle (\langle u? \rangle \langle x_{1}!z \rangle C(x_{0},x_{1},y_{0},y_{1},u))) &
\end{align*}

The lexicographical similarity between the shape of this formulae and
the shape of definition of the process representing a crossing reveals
the intuitive meaning of this formulae. It describes the capabilities
of a process that has the right to represent a crossing. For example
it picks out processes that may perform an input on the port $x_0$ in
its initial menu of capabilities. What differentiates the formula
from the process, however, is that the crossing process is the
smallest candidate to satisfy the formula. Infinitely many other
processes -- with internal behavior hidden behind this interface, so
to speak -- also satisfy this formula. Even this simple formula,
then, can be seen to open a new view onto knots, providing a
computational interpretation of \emph{virtual} knots.

Note that this formula is derived by hand. A similar formula can be
derived by employing Caires' calculation of characteristic formula
\cite{Caires04} to the process representing a crossing. In light of
this discussion, we let
$\meaningof{C}_{\phi}(x0,x1,y0,y1,u)$ denote a formula specifying the
dynamics we wish to capture of a crossing. To guarantee we preserve
the shape of the interface and minimal semantics we demand that
$\meaningof{C}_{\phi}(x0,x1,y0,y1,u) \Rightarrow
\textbf{C}(x0,x1,y0,y1,u)$ where $\textbf{C}(x0,x1,y0,y1,u)$ denotes
the formula above.
                            
\subsubsection{Crossing number constraints.}
The moral content of the context lemma (Lemma \ref{context}) is that the notion of
``locality'' in the Reidemeister moves is effectively captured by the
parallel composition operator of the process calculus. This intuition
extends through the logic. Given a formula,
$\meaningof{C}_{\phi}(x0,x1,y0,y1,u)$, we can use the structural
connectives to specify constraints on crossing numbers, such as at
least $n$ crossings, or exactly $n$ crossings.
\begin{mathpar}
  \inferrule* [lab=at-least-n] {} { K^{\geq n}_{\phi}(\vec{xs},\vec{ys}) := \Pi_{i=0}^{n-1} Hu . \meaningof{C}_{\phi}(xs_i,ys_i,u) | T }
  \and 
  \inferrule* [lab=exactly-n] {} { K^{= n}_{\phi}(\vec{xs},\vec{ys}) := \Pi_{i=0}^{n-1} Hu . \meaningof{C}_{\phi}(xs_i,ys_i,u) | \neg (\forall x_0,y_0,x_1,y_1,u . \meaningof{C}_{\phi}(x_0,y_0,x_1,y_1,u) | T) }
\end{mathpar}

To round out this section, recall that the encoding of an $n$-crossing
knot decomposes into a parallel composition of $n$ \emph{copies} of a
crossing process together with a wiring harness. To specify different
knot classes with the same crossing number amounts to specifying
logical constraints on the wiring harness. In the interest of space,
we defer examples to a forthcoming paper. Suffice it to say that both
the conditions ``alternating knot'' and ``contains the tangle
corresponding to 5/3'' are expressible. For example, it is possible to
calculate the characteristic formula of a process corresponding to the
tangle 5/3 and conjoin it into the classifying formula via the
composition connective of the logic.

Finally, we wish to observe that it is entirely within reason to
contemplate a more domain-specific version of spatial logic tailored
to the shape of processes in the image of the encoding. Such a
domain-specific logic would have a better claim to the title formal
language of knot properties.

% subsection example_formulae_ (end)

% section knots_as_processes (end) 

% section spatial logic via knots (end)

\section{Conclusions and future work}

\paragraph{Testing physical space}
You, gentle reader, may wonder why of all the theorems to be proved
given this set up we pick the one above. In some sense it's hardly
central to quantum mechanics. We see it as central in the sense that
it firmly establishes a notion of physical space arising from a notion
of the equivalence of behavior. Relating bisimulation to a metric is a
big step forward, but one is faced with interpreting the relationship
of that metric space to something more physical. Quantum mechanical
notions of ``physical'' space are still far from intuitive, but by
relating this idea of distance as testing to calculations that predict
physical circumstances we are making a not insignificant step forward
toward an understanding of the physical space we inhabit as
essentially dynamic.

\paragraph{Effectivity and simulation}
One of the observations we have yet to make is that the entire program
spelled out here is effective. We have built various interpreters for
the reflective calculus at work in this interpretation. In principle,
then, we can simulate quantum mechanics on a computer. The place where
the simulation may lose fidelity is the infinitely branching summation
for the annihilator.

In this connection i also want to point out that the evaluation style
calculation of the inner product puts the non-determinism of the
summation right at the heart of measurement. This suggests that
Milner's original reduction-based formulation of the dynamics of his
calculi in terms of sums was not just notationally suggestive of a
notion of measure-and-continue but captured some significant part of
the physics.

\paragraph{Quantum continuations}
In light of this last observation i want to point out that the
predominant account of quantum mechanics is missing a key aspect of a
truly compositional story of the physical situation. In a real lab,
when a measurement is made the observation can be made to feed into
another device that then makes another measurement conditioned on the
results of the first. This means that after the superposition was
collapsed the entire experimental set up remained in
superposition. While QM offers a means of writing this down it doesn't
quite line up well with the well-trodden formulation of computation
and continuation that we see so succinctly expressed in Milner's
calculi. This suggests that there might be advantages to this account
of dynamics waiting to be explored.

\paragraph{Quantum logic}
In this connection, we also note that by virtue of having the
Hennessy-Milner construction, we can pull the construction through the
interpretation of QM. This gives us a natural candidate for a quantum
logic that enjoys an extremely tight connection with it's domain of
interpretation, making the construction much less ad hoc (rather it is
the image of functor!).

\paragraph{Quantum probabiity}
i have questions about the basis of the interpretation of inner
product as probability amplitude. In particular, using which
axiomatization of probability theory does the notion of probability
amplitude earn the right to be so dubbed? In other words, where is the
proof that the operation for calculating a probability amplitude (and
then squaring) satisfies the axioms of what it means to calculate a
probability? Even if such a proof exists (i have yet to find it in the
literature), i wonder if it might not be possible to turn things on
their heads. Can we view the calculation of the probability amplitude
as an axiomatization of probability? If so, then the definition we
give for calculating probability amplitude may provide the basis for
an \emph{effective} theory of probability.

\paragraph{Quantum vs ``biological'' information}
Finally, i want to conclude with a more philosophical observation. At
a recent workshop in which QM was a predominant topic i noticed
something about quantum information. The speaker was giving a riveting
discussion of axiomatic QM and showing how properties of ``no
cloning'' and ``no deleting'' emerged as consequences of the
axiomatization. Theorems of this form are necessary to give us a sense
of confidence that our axioms characterize the physical theory. What
struck me, though, was that if quantum information is neither erasable
nor replicable it is markedly different from \emph{life}. Two of the
things we know about life is that

\begin{itemize}
  \item it ends;
  \item to gain some measure of persistence, to transcend it's
    finitude it is imminently copyable.
\end{itemize}

Both of these qualities are summarized succinctly in the aphorism: all
flesh is grass. For me these two kinds of ``information'' -- call them
quantum and biological -- are end points on a spectrum of strategies
for persistence. At one end, we have those curious entities that enjoy
uniqueness and permanence; at the other, we have those who in the face
of a certain end and an uncertain present make a go of passing
something on. To me one of the more remarkable aspects of the latter
strategy is that in the presence of noise (and certain features of
copying) we get a kind of dynamism, a chance for improvement against a
given persistent condition.

% subsection other_calculi_other_bisimulations_and_geometry_as_behavior (end)




% section conclusion (end)

%\documentclass[12pt]{llncs}
%\documentclass{jktr}

\usepackage[pdftex]{hyperref}                   
\usepackage {listings}
\usepackage {mathpartir}
\usepackage{bcprules}
%\usepackage{listings}
                       
\usepackage{graphicx} 
%\usepackage[margins=2.5cm,nohead,nofoot]{geometry}
%\usepackage{geometry}
\usepackage{amsfonts}
\usepackage{amstext}
\usepackage{latexsym}
\usepackage{amssymb}
\usepackage{color}


%\include{myPreamble}
\include{qm2pi.local} 

%\ifpdf
%\usepackage[pdftex]{graphicx}
%\else
%\usepackage{graphicx}
%\fi

 % \ifpdf
%  \usepackage{pdfsync}
%  \if


%\title{Brief Article}
%\author{David F. Snyder}
%\author{L.G. Meredith}

%\address{Dept. of Math., Texas State University--San Marcos, San Marcos, TX 78666}
       
\pagestyle{empty}


\begin{document}

\lstset{language=[Objective]Caml,frame=shadowbox}

\input{qm2pi.front}

% section front matter (end)

\input{qm2pi.intro} 
 
% section introduction (end)

% \input{qm2pi.knotations} 

% section notation (end)

\input{qm2pi.process.calculi} 

% section concurrent_process_calculi_and_spatial_logics_ (end)
    
%\input{qm2pi.knots2pi} 

%\input{qm2pi.trefoil} 

%\input{qm2pi.mainthm} 

% subsection basic_interpretation (end)

%\input{qm2pi.rho.presentation} 
\subsection{The syntax and semantics of the notation system}\label{sub:the_syntax_and_semantics_of_the_notation_system} % (fold)

We now summarize a technical presentation of the calculus that
embodies our theory of dynamics. The typical presentation of such a
calculus follows the style of giving generators and relations on
them. The grammar, below, describing term constructors, freely
generates the set of processes, $\Proc$. This set is then quotiented
by a relation known as structural congruence and it is over this set
that the notion of dynamics is expressed. This presentation is
essentially that of \cite{MeredithR05} with the addition of
polyadicity and summation. For readability we have relegated some of
the technical subtleties to an appendix.

\subsubsection{Process grammar}\label{subsub:process_grammar}

\begin{mathpar}
  \inferrule* [lab=synchronization] {} {{M} \bc \pzero \;|\; x?F \;|\; x!C }
  \and
  \inferrule* [lab=abstraction] {} {{F} \bc (x)P}
  \and
  \inferrule* [lab=concretion] {} {{C} \bc \langle Q \rangle}
  \and
  \inferrule* [lab=process] {} {{P,Q} \bc M \;| \;P|Q \;|\; @{x}}
  \and
  \inferrule* [lab=name] {} {{x} \bc \quotep{P}}
\end{mathpar} 

Note that $\vec{x}$ (resp. $\vec{P}$) denotes a vector of names
(resp. processes) of length $|\vec{x}|$ (resp. $|\vec{P}|$). We adopt
the following useful abbreviations.

\begin{mathpar}
   x?(\vec{y}).P := x.(\vec{y})P \and  x\clift{\vec{P}} := x.\clift{\vec{P}}
   \and x!(y) := \lift{x}{\dropn{y}}
   \and \Pi_{i=0}^{n-1}P_i := P_0 | \ldots | P_{n-1}
\end{mathpar}

\subsubsection{Structural congruence}

\paragraph{Free and bound names and alpha-equivalence.} At the
core of structural equivalence is alpha-equivalence which identifies
process that are the same up to a change of variable. Formally, we
recognize the distinction between free and bound names. The free names
of a process, $\freenames{P}$, may be calculated recursively as
follows:

\begin{mathpar}
\freenames{\pzero} := \emptyset
  \and \\
  \freenames{x?(y).P} := \{ x \} \cup (\freenames{P} \setminus \{ y \})
  \and 
  \freenames{x!\langle P \rangle} := \{ x \} \cup \{ P \} 
  \and \\
  \freenames{P|Q} := \freenames{P} \cup \freenames{Q}
  \and \\
  \freenames{@{x}} := \{ x \}
\end{mathpar}

$\pi$
$\quotep{\pi}$

$\freenames{-} : \pi \to \mathcal{P}(\quotep{\pi})$

\begin{eqnarray*}
  \freenames{\pzero} & := & \emptyset \\
  \freenames{x?(y).P} & := & \{ x \} \cup (\freenames{P} \setminus \{ y \}) \\
  \freenames{x!\langle P \rangle} & := & \{ x \} \cup \{ P \} \\
  \freenames{P|Q} & := & \freenames{P} \cup \freenames{Q} \\
  \freenames{\dropn{x}} & := & \{ x \}
\end{eqnarray*}

The bound names of a process, $\boundnames{P}$, are those names occurring in $P$
that are not free. For example, in $x?(y).0$, the name $x$ is free, while $y$ is bound.

\begin{mathpar}
  \inferrule* [lab=monoidal-laws] {} { P|Q \equiv Q|P \and P|0 \equiv P \and P|(Q|R) \equiv (P|Q)|R }
\end{mathpar}

\begin{mathpar}
  \inferrule* [lab=alpha-equivalence] {} { (x)P \equiv (y)P\{y/x\} \and y \not\in \freenames{P} }
\end{mathpar}

\begin{definition}
Then two processes, $P,Q$, are alpha-equivalent if $P = Q\{\vec{y}/\vec{x}\}$ for
some $\vec{x} \in \boundnames{Q},\vec{y} \in \boundnames{P}$, where $Q\{\vec{y}/\vec{x}\}$
denotes the capture-avoiding substitution of $\vec{y}$ for $\vec{x}$ in $Q$.
\end{definition}

\begin{definition}
  The {\em structural congruence} \cite{SangiorgiWalker} , $\equiv$,
  between processes is the least congruence containing
  alpha-equivalence, satisfying the abelian monoid laws
  (associativity, commutativity and $\pzero$ as identity) for parallel
  composition $|$ and for summation $+$.
\end{definition}

\subsection{Name equivalence}

We take name equivalence, written $\nameeq$, to be the smallest
equivalence relation generated by the following rules.

\begin{mathpar}
\inferrule*[lab=Quote-drop]
{ }
{ \quotep{@{x}} \nameeq x }

\inferrule*[lab=Struct-equiv]
{ P \scong Q }
{ \quotep{P} \nameeq \quotep{Q} }
\end{mathpar}

The astute reader will have noticed that the mutual recursion of names
and processes imposes a mutual recursion on alpha-equivalence and
structural equivalence via name-equivalence. Fortunately, all of this
works out pleasantly and we may calculate in the natural way, free of
concern. The reader interested in the details is referred to the
appendix \ref{appendix:rho_details}.

\subsection{Substitution}

We use $\Proc$ for the set of processes, $\QProc$ for the set of
names, and $\id{\{}\vec{y} / \vec{x} \id{\}}$ to denote partial maps,
$s : \QProc \rightarrow \QProc$. A map, $s$ lifts, uniquely, to a map
on process terms, $\widehat{s} : \Proc \rightarrow \Proc$ by the
following equations.

\begin{mathpar}
  (0) \psubstp{Q}{P} := 0 \\
  (R \juxtap S) \psubstp{Q}{P}
  :=    
  (R)\psubstp{Q}{P} \juxtap (S) \psubstp{Q}{P} \\
  (x?(y).R) \psubstp{Q}{P}    
  :=    
  (x)\substp{Q}{P} (z)\concat( (R \psubstn{z}{y}) \psubstp{Q}{P} ) \\
  (\lift{x}{R}) \psubstp{Q}{P}  
  :=
  \lift{(x)\substp{Q}{P}}{ R \psubstp{Q}{P} } \\
%   (\dropn{x})  \psubstp{Q}{P}       
%   := 
%   \left\{ 
%     \begin{array}{ccc} 
%       \dropn{\quotep{Q}} & & x \nameeq \quotep{P} \\
%       \dropn{x} & & otherwise \\
%     \end{array}
%   \right. 
  (\dropn{x})  \psubstp{Q}{P}       
  := 
  \left\{ 
    \begin{array}{ccc} 
      Q & & x \nameeq \quotep{P} \\
      \dropn{x} & & otherwise \\
    \end{array}
  \right.
\end{mathpar}
 

where

\begin{eqnarray}
  (x)\id{\{} \lpquote Q \rpquote / \lpquote P \rpquote \id{\}}            = 
  \left\{ 
    \begin{array}{ccc}
      \lpquote Q \rpquote & & x \nameeq \lpquote P \rpquote \\
      x & & otherwise \\
    \end{array}
  \right. \nonumber
\end{eqnarray}

and $z$ is chosen distinct from $\quotep{P}$, $\quotep{Q}$, the free
names in $Q$, and all the names in $R$. Our $\alpha$-equivalence will
be built in the standard way from this substitution.

\begin{remark}\label{rem:no_self_referential_names}
  One consequence of these definitions is that $\forall P. \quotep{P}
  \not\in \freenames{P}$.
\end{remark}

\subsection{ Dynamic quote: an example }

Anticipating something of what's to come, consider applying the
substitution, $\widehat{\id{\{}u / z \id{\}}}$, to the following pair
of processes, $\lift{w}{y!(z)}$ and $w[ \lpquote y!(z) \rpquote ]$.

\begin{eqnarray}
	\lift{w}{y!(z)}\widehat{\id{\{}u / z \id{\}}}
		& = &
		\lift{w}{y!(u)} \nonumber\\
	w[ \lpquote y!(z) \rpquote ] \widehat{ \id{\{}u / z \id{\}} }
		& = &
		w[ \lpquote y!(z) \rpquote ] \nonumber
\end{eqnarray}

Because the body of the process between quotes is impervious to
substitution, we get radically different answers. In fact, by
examining the first process in an input context,
e.g. $x?(z).\lift{w}{y!(z)}$, we see that the process under the lift
operator may be shaped by prefixed inputs binding a name inside it. In
this sense, the lift operator will be seen as a way to dynamically
construct processes before reifying them as names.

Finally equipped with these standard features we can present the
dynamics of the calculus.

\subsubsection{Operational semantics} 

Finally, we introduce the computational dynamics. What marks these
algebras as distinct from other more traditionally studied algebraic
structures, e.g. vector spaces or polynomial rings, is the manner in
which dynamics is captured. In traditional structures, dynamics is typically
expressed through morphisms between such structures, as in linear maps
between vector spaces or morphisms between rings. In algebras
associated with the semantics of computation, the dynamics is
expressed as part of the algebraic structure itself, through a
reduction reduction relation typically denoted by $\red$. Below, we
give a recursive presentation of this relation for the calculus used
in the encoding.

$\red \subseteq \pi \times \pi$
$\red : \pi \to \mathcal{P}(\pi)$

\begin{mathpar}
  \inferrule* [lab=Comm] { \textsf{match}( x_{src}, x_{trgt} ) } { x_{trgt}?(y)P \; | \; x_{src}!\langle {Q} \rangle \red P\{\quotep{Q}/y}\} }
  \and \\
  \inferrule* [lab=Par] {{P} \red {P}'} {{{P} | {Q}} \red {{P}' | {Q}}}
  \and
  \inferrule* [lab=Equiv]{{{P} \scong {P}'} \andalso {{P}' \red {Q}'} \andalso {{Q}' \scong {Q}}}{{P} \red {Q}}
\end{mathpar}

\begin{eqnarray*}
  match_{\equiv} (\quotep{P},\quotep{Q}) & := & P \equiv Q \\
  match_{\dagger}(\quotep{P},\quotep{Q}) & := & \forall R. P|Q \red^{*} R => R \red^{*} 0 \\
  match_{K}(\quotep{P},\quotep{Q}) & := & K \mbox{ for some context } K
\end{eqnarray*}

$u?(x)P | u!\langle Q \rangle \red P\{\quotep{Q}/x\}$

%We write $\wred$ for $\red^*$, and $P\red$ if $\exists Q $ such that $ P \red Q$.
We write $P\red$ if $\exists Q $ such that $ P \red Q$ and $P\not\red$, otherwise.

\section{Replication}

As mentioned before, it is known that replication (and hence
recursion) can be implemented in a higher-order process algebra
\cite{SangiorgiWalker}. As our first example of calculation with the
machinery thus far presented we give the construction explicitly in
the {\rhoc}.

\begin{eqnarray}
	D_{x} & := & \prefix{x}{y}{(\binpar{\outputp{x}{y}}{@{y}})} \nonumber\\
	\bangp_{x}{P} & := & \binpar{{x}!\langle{\binpar{D_{x}}{P}}\rangle}{D_{x}} \nonumber
\end{eqnarray}

\begin{eqnarray}
	\bangp_{x}{P} & & \nonumber\\
	=
	& {x}!\langle{(\prefix{x}{y}{(\outputp{x}{y} | @{y})) | P}}\rangle 
	      | \prefix{x}{y}{(\outputp{x}{y} | @{y})} & \nonumber\\
	\red
	& (\outputp{x}{y} | @{y})\substn{\quotep{(\prefix{x}{y}{(@{y} | \outputp{x}{y})) | P}}}{y} & \nonumber\\
	=
	& \outputp{x}{\quotep{(\prefix{x}{y}{(\outputp{x}{y} | @{y})) | P}}}
	  | {(\prefix{x}{y}{(\outputp{x}{y} | @{y})) | P}} & \nonumber\\
	\red
	& \ldots & \nonumber\\
	\red^*
	& P | P | \ldots & \nonumber
\end{eqnarray}

Of course, this encoding, as an implementation, runs away, unfolding
$\bangp{P}$ eagerly. A lazier and more implementable replication
operator, restricted to input-guarded processes, may be obtained as follows.

\begin{eqnarray}
\bangp{\prefix{u}{v}{P}} 
	:= 
	\binpar{\lift{x}{\prefix{u}{v}{(\binpar{D(x)}{P})}}}{D(x)} \nonumber
\end{eqnarray}

\begin{remark}
  Note that the lazier definition still does not deal with summation
  or mixed summation (i.e. sums over input and output). The reader is
  invited to construct definitions of replication that deal with these
  features. 

  Further, the definitions are parameterized in a name, $x$. Can you,
  gentle reader, make a definition that eliminates this parameter and
  guarantees no accidental interaction between the replication
  machinery and the process being replicated -- i.e. no accidental
  sharing of names used by the process to get its work done and the
  name(s) used by the replication to effect copying. This latter
  revision of the definition of replication is crucial to obtaining
  the expected identity $!!P \sim !P$.
\end{remark}

\begin{remark}\label{rem:paradoxical_combinator}
  The reader familiar with the lambda calculus will have noticed the
  similarity between $D$ and the paradoxical combinator.

  [Ed. note: the existence of this seems to suggest we have to be more
  restrictive on the set of processes and names we admit if we are to
  support no-cloning.]
\end{remark}

\subsubsection{Bisimulation}

The computational dynamics gives rise to another kind of equivalence,
the equivalence of computational behavior. As previously mentioned
this is typically captured \emph{via} some form of bisimulation.

% The notion we use in this paper is weak barbed bisimulation
% \cite{milner91polyadicpi}.

The notion we use in this paper is derived from weak barbed
bisimulation \cite{milner91polyadicpi}. 

\begin{definition}
An \emph{observation relation}, $\downarrow_{\mathcal N}$, over a set
of names, $\mathcal N$, is the smallest relation satisfying the rules
below.

\infrule[Out-barb]{y \in {\mathcal N}, \; x \nameeq y}
		  {\outputp{x}{v} \downarrow_{\mathcal N} x}
\infrule[Par-barb]{\mbox{$P\downarrow_{\mathcal N} x$ or $Q\downarrow_{\mathcal N} x$}}
		  {\binpar{P}{Q} \downarrow_{\mathcal N} x}

We write $P \Downarrow_{\mathcal N} x$ if there is $Q$ such that 
$P \wred Q$ and $Q \downarrow_{\mathcal N} x$.
\end{definition}

\begin{definition}
%\label{def.bbisim}
An  ${\mathcal N}$-\emph{barbed bisimulation} over a set of names, ${\mathcal N}$, is a symmetric binary relation 
${\mathcal S}_{\mathcal N}$ between agents such that $P\rel{S}_{\mathcal N}Q$ implies:
\begin{enumerate}
\item If $P \red P'$ then $Q \wred Q'$ and $P'\rel{S}_{\mathcal N} Q'$.
\item If $P\downarrow_{\mathcal N} x$, then $Q\Downarrow_{\mathcal N} x$.
\end{enumerate}
$P$ is ${\mathcal N}$-barbed bisimilar to $Q$, written
$P \wbbisim_{\mathcal N} Q$, if $P \rel{S}_{\mathcal N} Q$ for some ${\mathcal N}$-barbed bisimulation ${\mathcal S}_{\mathcal N}$.
\end{definition}

$\mathcal{R} \subseteq \pi \times \pi$

$P \mathcal{R} Q => \forall P'. P \red P' \Rightarrow \exists Q'. Q \red Q', P' \mathcal{R} Q'$

$P \vdash x \Rightarrow Q \vdash x$

\begin{mathpar}
  \inferrule*[lab=Out-barb]{x \nameeq y}{{y}!\langle{Q}\rangle \vdash x}
  \and
  \inferrule*[lab=Par-barb]{\mbox{$P\vdash x$ or $Q\vdash x$}}{\binpar{P}{Q} \vdash x}
\end{mathpar}

\subsubsection{Contexts}

One of the principle advantages of computational calculi like the
$\pi$-calculus is a well-defined notion of context,
contextual-equivalence and a correlation between
contextual-equivalence and notions of bisimulation. The notion of
context allows the decomposition of a process into (sub-)process and
its syntactic environment, its context. Thus, a context may be
thought of as a process with a ``hole'' (written $\Box$) in it. The
application of a context $M$ to a process $P$, written $M[P]$, is
tantamount to filling the hole in $M$ with $P$. In this paper we do
not need the full weight of this theory, but do make use of the notion
of context in the proof the main theorem. 

\begin{mathpar}
  \inferrule* [lab=summation] {} {{M_{M},M_{N}} \bc \Box \;|\; x.M_{A} \;|\; M_{M}+M_{N}}
  \and
  \inferrule* [lab=agent] {} {{M_{A}} \bc (\vec{x})M_{P} \;| \; \clift{P_0,\ldots,M_{P},\ldots,P_N}}
  \and \\
  \inferrule* [lab=process] {} {{M_{P}} \bc M_{N} \;| \;P|M_{P} }
\end{mathpar} 

\begin{mathpar}
  \inferrule* [lab=sychronization] {} {M_{N} \bc \Box \;|\; x?M_{F} \;|\; x!M_{C}}
  \and
  \inferrule* [lab=abstraction] {} {{M_{F}} \bc (x)M_{P} }
  \and
  \inferrule* [lab=concretion] {} {{M_{C}} \bc \langle M_{P} \rangle }
  \and \\
  \inferrule* [lab=process] {} {{M_{P}} \bc M_{N} \;| \;P|M_{P} }
\end{mathpar}

\begin{definition}[contextual application] Given a context $M$, and
  process $P$, we define the \emph{contextual application}, $M[P] :=
  M\{P/\Box\}$. That is, the contextual application of M to P is the
  substitution of $P$ for $\Box$ in $M$.
\end{definition}

$\meaningof{-} : L \to \mathcal{P}(\pi)$

\begin{mathpar}
  \inferrule* [lab=collection] {} {\meaningof{true} = \pi, \and \meaningof{~E} = \pi \setminus \meaningof{E}, \and \meaningof{E_{1} \& E_{2}} = \meaningof{E_{1}} \cap \meaningof{E_{2}}}
\end{mathpar}

\begin{mathpar}
  \inferrule* [lab=structure] {} {\meaningof{0} = \{ P \in \pi | P \equiv 0 \}, \and \\ \meaningof{E_1 | E_2} = \{ P \in \pi | P \equiv P_{1} | P_{2}, P_{1} \in \meaningof{E_{1}}, P_{2} \in \meaningof{E_2}\} }
\end{mathpar}

\begin{mathpar}
 \inferrule* [lab=behavior] {} {\meaningof{\langle a?b \rangle E} = \{ P \in \pi | P \equiv Q | u?(y)P', \\ \and \\\\ \and \\ \;\;\; u \in \meaningof{a}, \forall z.P'\{z/y\} \in \meaningof{E\{z/b\}}\}, \and \\ \meaningof{a!E} = \{ P \in \pi | P \equiv Q | x!\langle P' \rangle, x \in \meaningof{a} P' \in \meaningof{E}\} }
\end{mathpar}

\begin{mathpar}
 \inferrule* [lab=nominal] {} {\meaningof{\quotep{E}} = \{ \quotep{P} \in \quotep{\pi} | P \in \meaningof{E} \}, \and \meaningof{\quotep{P}} = \{ \quotep{Q} \in \quotep{\pi} | P \equiv Q \} \and \\ \meaningof{@\quotep{E}} = \{ P \in \pi | P \equiv @x, x \in \meaningof{E} \}}
\end{mathpar}

\begin{eqnarray*}
  \\
  \meaningof{-} : TS \to ST
\end{eqnarray*}

\begin{eqnarray*}
  \\
  L : TS \to ST
\end{eqnarray*}

\begin{eqnarray*}
  \\
  P \models E \iff P \in \meaningof{E}
\end{eqnarray*}

\begin{eqnarray*}
  P \approx_{L} Q \iff \forall E \in L. P \models E \iff Q \models E
\end{eqnarray*}

\begin{eqnarray*}
  P \approx_{K} Q
\end{eqnarray*}

\begin{eqnarray*}
  P \approx Q
\end{eqnarray*}

$\approx_{K} = \approx = \approx_{L}$

\subsubsection{Contextual duality}

Note that contexts extend the quotation operation to a family of
operations from processes to names. Given a context, $M$, we can
define a \emph{nominal context}, $\quotep{M}$ by $\quotep{M}[P] :=
\quotep{M[P]}$. To foreshadow what is to come we observe that these
operations enjoy a duality with processes very much like the duality
between vectors and maps from vectors to scalars.

Further, because the calculus is essentially higher-order, we have a
correspondence between contexts and processes. More specifically,
given a name $x$ and a context $M$ we can construct $M^{*}_{x}$ such
that 

\begin{mathpar}
  M^{*}_{x} | \lift{x}{P} \red M[P]
\end{mathpar}

namely,

\begin{mathpar}
  M^{*}_{x} := x?(u).M[\dropn{u}]
\end{mathpar}

The dependence of $M^{*}_{x}$ on a name makes it an abstraction, 

\begin{mathpar}
  M^{*} := (x)x?(u).M[\dropn{u}]
\end{mathpar}

\subsection{Additional notation}

It will sometimes be convenient to denote the process a name
quotes. We already have the notation $x = \quotep{P}$, but it will be
convenient to introduce an alternate notation, $\procn{x}$, when we
want to emphasize the connection to the use of the name. Note that, by
virtue of name equivalence, $\quotep{\procn{x}} \nameeq x$; so, the
notation is consistent with previous definitions.

Further, because names have structure it is possible to effect
substitutions on the basis of that structure. This means we need to
upgrade our notation for substitutions, which we accomplish by
adapting comprehension notation. Thus,

\begin{mathpar}
  P\{ y / x : x \in S \}
\end{mathpar}

is interpreted to mean the process derived from P by replacing (in a
capture-avoiding manner) each occurrence of $x$ in $S$ by $y$. For example,

\begin{mathpar}
  P\{ \quotep{\procn{x}|\procn{x}} / x : x \in \freenames{P} \}
\end{mathpar}

will replace each (occurrence) of a free name $x$ in $P$ by
$\quotep{\procn{x}|\procn{x}}$.

Also, we will avail ourselves of the notation $x^{L}$ and $x^{R}$ to
denote injections of a name into disjoint copies of the name
space. There are numerous ways to accomplish this. One example can be
found in \cite{MeredithR05}. This notation overloads to vectors of
names: $\vec{x}^{\pi} := (x_{i}^{\pi} \; : \; 0 \leq i < |\vec{x}| )$ where $\pi \in \{L,R\}$.

We also use $P^{\Box} := P|\Box$.

In \cite{MeredithR05} an interpretation of the new operator is
given. It turns out that there are several possible interpretations
all enjoying the requisite algebraic properties of the operator (see
\cite{milner91polyadicpi}). We will therefore make liberal use of
$(\nu\; \vec{x})P$.

% subsection the_syntax_and_semantics_of_the_notation_system (end)   

\input{qm2pi.qmops} 

\input{qm2pi.sterngerlach} 

\input{qm2pi.metric} 

% section concurrent_process_calculi (end)

%\input{qm2pi.proofsketch}

% section proof sketch (end)

%\input{qm2pi.slviaknots} 

% section spatial logic via knots (end)

\input{qm2pi.conclusion}

% section conclusion (end)

%\input{qm2pi.dtcodes} 

% section wiring algorithm (end)

\input{qm2pi.ack} 

% section acknowledgments (end)

\newpage


\bibliographystyle{plain}   
\bibliography{../../biblios/main.bib}

\input{qm2pi.rhodetails}

\end{document}

 

% section wiring algorithm (end)

\documentclass[12pt]{llncs}
%\documentclass{jktr}

\usepackage[pdftex]{hyperref}                   
\usepackage {listings}
\usepackage {mathpartir}
\usepackage{bcprules}
%\usepackage{listings}
                       
\usepackage{graphicx} 
%\usepackage[margins=2.5cm,nohead,nofoot]{geometry}
%\usepackage{geometry}
\usepackage{amsfonts}
\usepackage{amstext}
\usepackage{latexsym}
\usepackage{amssymb}
\usepackage{color}


%\include{myPreamble}
\include{qm2pi.local} 

%\ifpdf
%\usepackage[pdftex]{graphicx}
%\else
%\usepackage{graphicx}
%\fi

 % \ifpdf
%  \usepackage{pdfsync}
%  \if


%\title{Brief Article}
%\author{David F. Snyder}
%\author{L.G. Meredith}

%\address{Dept. of Math., Texas State University--San Marcos, San Marcos, TX 78666}
       
\pagestyle{empty}


\begin{document}

\lstset{language=[Objective]Caml,frame=shadowbox}

\input{qm2pi.front}

% section front matter (end)

\input{qm2pi.intro} 
 
% section introduction (end)

% \input{qm2pi.knotations} 

% section notation (end)

\input{qm2pi.process.calculi} 

% section concurrent_process_calculi_and_spatial_logics_ (end)
    
%\input{qm2pi.knots2pi} 

%\input{qm2pi.trefoil} 

%\input{qm2pi.mainthm} 

% subsection basic_interpretation (end)

%\input{qm2pi.rho.presentation} 
\subsection{The syntax and semantics of the notation system}\label{sub:the_syntax_and_semantics_of_the_notation_system} % (fold)

We now summarize a technical presentation of the calculus that
embodies our theory of dynamics. The typical presentation of such a
calculus follows the style of giving generators and relations on
them. The grammar, below, describing term constructors, freely
generates the set of processes, $\Proc$. This set is then quotiented
by a relation known as structural congruence and it is over this set
that the notion of dynamics is expressed. This presentation is
essentially that of \cite{MeredithR05} with the addition of
polyadicity and summation. For readability we have relegated some of
the technical subtleties to an appendix.

\subsubsection{Process grammar}\label{subsub:process_grammar}

\begin{mathpar}
  \inferrule* [lab=synchronization] {} {{M} \bc \pzero \;|\; x?F \;|\; x!C }
  \and
  \inferrule* [lab=abstraction] {} {{F} \bc (x)P}
  \and
  \inferrule* [lab=concretion] {} {{C} \bc \langle Q \rangle}
  \and
  \inferrule* [lab=process] {} {{P,Q} \bc M \;| \;P|Q \;|\; @{x}}
  \and
  \inferrule* [lab=name] {} {{x} \bc \quotep{P}}
\end{mathpar} 

Note that $\vec{x}$ (resp. $\vec{P}$) denotes a vector of names
(resp. processes) of length $|\vec{x}|$ (resp. $|\vec{P}|$). We adopt
the following useful abbreviations.

\begin{mathpar}
   x?(\vec{y}).P := x.(\vec{y})P \and  x\clift{\vec{P}} := x.\clift{\vec{P}}
   \and x!(y) := \lift{x}{\dropn{y}}
   \and \Pi_{i=0}^{n-1}P_i := P_0 | \ldots | P_{n-1}
\end{mathpar}

\subsubsection{Structural congruence}

\paragraph{Free and bound names and alpha-equivalence.} At the
core of structural equivalence is alpha-equivalence which identifies
process that are the same up to a change of variable. Formally, we
recognize the distinction between free and bound names. The free names
of a process, $\freenames{P}$, may be calculated recursively as
follows:

\begin{mathpar}
\freenames{\pzero} := \emptyset
  \and \\
  \freenames{x?(y).P} := \{ x \} \cup (\freenames{P} \setminus \{ y \})
  \and 
  \freenames{x!\langle P \rangle} := \{ x \} \cup \{ P \} 
  \and \\
  \freenames{P|Q} := \freenames{P} \cup \freenames{Q}
  \and \\
  \freenames{@{x}} := \{ x \}
\end{mathpar}

$\pi$
$\quotep{\pi}$

$\freenames{-} : \pi \to \mathcal{P}(\quotep{\pi})$

\begin{eqnarray*}
  \freenames{\pzero} & := & \emptyset \\
  \freenames{x?(y).P} & := & \{ x \} \cup (\freenames{P} \setminus \{ y \}) \\
  \freenames{x!\langle P \rangle} & := & \{ x \} \cup \{ P \} \\
  \freenames{P|Q} & := & \freenames{P} \cup \freenames{Q} \\
  \freenames{\dropn{x}} & := & \{ x \}
\end{eqnarray*}

The bound names of a process, $\boundnames{P}$, are those names occurring in $P$
that are not free. For example, in $x?(y).0$, the name $x$ is free, while $y$ is bound.

\begin{mathpar}
  \inferrule* [lab=monoidal-laws] {} { P|Q \equiv Q|P \and P|0 \equiv P \and P|(Q|R) \equiv (P|Q)|R }
\end{mathpar}

\begin{mathpar}
  \inferrule* [lab=alpha-equivalence] {} { (x)P \equiv (y)P\{y/x\} \and y \not\in \freenames{P} }
\end{mathpar}

\begin{definition}
Then two processes, $P,Q$, are alpha-equivalent if $P = Q\{\vec{y}/\vec{x}\}$ for
some $\vec{x} \in \boundnames{Q},\vec{y} \in \boundnames{P}$, where $Q\{\vec{y}/\vec{x}\}$
denotes the capture-avoiding substitution of $\vec{y}$ for $\vec{x}$ in $Q$.
\end{definition}

\begin{definition}
  The {\em structural congruence} \cite{SangiorgiWalker} , $\equiv$,
  between processes is the least congruence containing
  alpha-equivalence, satisfying the abelian monoid laws
  (associativity, commutativity and $\pzero$ as identity) for parallel
  composition $|$ and for summation $+$.
\end{definition}

\subsection{Name equivalence}

We take name equivalence, written $\nameeq$, to be the smallest
equivalence relation generated by the following rules.

\begin{mathpar}
\inferrule*[lab=Quote-drop]
{ }
{ \quotep{@{x}} \nameeq x }

\inferrule*[lab=Struct-equiv]
{ P \scong Q }
{ \quotep{P} \nameeq \quotep{Q} }
\end{mathpar}

The astute reader will have noticed that the mutual recursion of names
and processes imposes a mutual recursion on alpha-equivalence and
structural equivalence via name-equivalence. Fortunately, all of this
works out pleasantly and we may calculate in the natural way, free of
concern. The reader interested in the details is referred to the
appendix \ref{appendix:rho_details}.

\subsection{Substitution}

We use $\Proc$ for the set of processes, $\QProc$ for the set of
names, and $\id{\{}\vec{y} / \vec{x} \id{\}}$ to denote partial maps,
$s : \QProc \rightarrow \QProc$. A map, $s$ lifts, uniquely, to a map
on process terms, $\widehat{s} : \Proc \rightarrow \Proc$ by the
following equations.

\begin{mathpar}
  (0) \psubstp{Q}{P} := 0 \\
  (R \juxtap S) \psubstp{Q}{P}
  :=    
  (R)\psubstp{Q}{P} \juxtap (S) \psubstp{Q}{P} \\
  (x?(y).R) \psubstp{Q}{P}    
  :=    
  (x)\substp{Q}{P} (z)\concat( (R \psubstn{z}{y}) \psubstp{Q}{P} ) \\
  (\lift{x}{R}) \psubstp{Q}{P}  
  :=
  \lift{(x)\substp{Q}{P}}{ R \psubstp{Q}{P} } \\
%   (\dropn{x})  \psubstp{Q}{P}       
%   := 
%   \left\{ 
%     \begin{array}{ccc} 
%       \dropn{\quotep{Q}} & & x \nameeq \quotep{P} \\
%       \dropn{x} & & otherwise \\
%     \end{array}
%   \right. 
  (\dropn{x})  \psubstp{Q}{P}       
  := 
  \left\{ 
    \begin{array}{ccc} 
      Q & & x \nameeq \quotep{P} \\
      \dropn{x} & & otherwise \\
    \end{array}
  \right.
\end{mathpar}
 

where

\begin{eqnarray}
  (x)\id{\{} \lpquote Q \rpquote / \lpquote P \rpquote \id{\}}            = 
  \left\{ 
    \begin{array}{ccc}
      \lpquote Q \rpquote & & x \nameeq \lpquote P \rpquote \\
      x & & otherwise \\
    \end{array}
  \right. \nonumber
\end{eqnarray}

and $z$ is chosen distinct from $\quotep{P}$, $\quotep{Q}$, the free
names in $Q$, and all the names in $R$. Our $\alpha$-equivalence will
be built in the standard way from this substitution.

\begin{remark}\label{rem:no_self_referential_names}
  One consequence of these definitions is that $\forall P. \quotep{P}
  \not\in \freenames{P}$.
\end{remark}

\subsection{ Dynamic quote: an example }

Anticipating something of what's to come, consider applying the
substitution, $\widehat{\id{\{}u / z \id{\}}}$, to the following pair
of processes, $\lift{w}{y!(z)}$ and $w[ \lpquote y!(z) \rpquote ]$.

\begin{eqnarray}
	\lift{w}{y!(z)}\widehat{\id{\{}u / z \id{\}}}
		& = &
		\lift{w}{y!(u)} \nonumber\\
	w[ \lpquote y!(z) \rpquote ] \widehat{ \id{\{}u / z \id{\}} }
		& = &
		w[ \lpquote y!(z) \rpquote ] \nonumber
\end{eqnarray}

Because the body of the process between quotes is impervious to
substitution, we get radically different answers. In fact, by
examining the first process in an input context,
e.g. $x?(z).\lift{w}{y!(z)}$, we see that the process under the lift
operator may be shaped by prefixed inputs binding a name inside it. In
this sense, the lift operator will be seen as a way to dynamically
construct processes before reifying them as names.

Finally equipped with these standard features we can present the
dynamics of the calculus.

\subsubsection{Operational semantics} 

Finally, we introduce the computational dynamics. What marks these
algebras as distinct from other more traditionally studied algebraic
structures, e.g. vector spaces or polynomial rings, is the manner in
which dynamics is captured. In traditional structures, dynamics is typically
expressed through morphisms between such structures, as in linear maps
between vector spaces or morphisms between rings. In algebras
associated with the semantics of computation, the dynamics is
expressed as part of the algebraic structure itself, through a
reduction reduction relation typically denoted by $\red$. Below, we
give a recursive presentation of this relation for the calculus used
in the encoding.

$\red \subseteq \pi \times \pi$
$\red : \pi \to \mathcal{P}(\pi)$

\begin{mathpar}
  \inferrule* [lab=Comm] { \textsf{match}( x_{src}, x_{trgt} ) } { x_{trgt}?(y)P \; | \; x_{src}!\langle {Q} \rangle \red P\{\quotep{Q}/y}\} }
  \and \\
  \inferrule* [lab=Par] {{P} \red {P}'} {{{P} | {Q}} \red {{P}' | {Q}}}
  \and
  \inferrule* [lab=Equiv]{{{P} \scong {P}'} \andalso {{P}' \red {Q}'} \andalso {{Q}' \scong {Q}}}{{P} \red {Q}}
\end{mathpar}

\begin{eqnarray*}
  match_{\equiv} (\quotep{P},\quotep{Q}) & := & P \equiv Q \\
  match_{\dagger}(\quotep{P},\quotep{Q}) & := & \forall R. P|Q \red^{*} R => R \red^{*} 0 \\
  match_{K}(\quotep{P},\quotep{Q}) & := & K \mbox{ for some context } K
\end{eqnarray*}

$u?(x)P | u!\langle Q \rangle \red P\{\quotep{Q}/x\}$

%We write $\wred$ for $\red^*$, and $P\red$ if $\exists Q $ such that $ P \red Q$.
We write $P\red$ if $\exists Q $ such that $ P \red Q$ and $P\not\red$, otherwise.

\section{Replication}

As mentioned before, it is known that replication (and hence
recursion) can be implemented in a higher-order process algebra
\cite{SangiorgiWalker}. As our first example of calculation with the
machinery thus far presented we give the construction explicitly in
the {\rhoc}.

\begin{eqnarray}
	D_{x} & := & \prefix{x}{y}{(\binpar{\outputp{x}{y}}{@{y}})} \nonumber\\
	\bangp_{x}{P} & := & \binpar{{x}!\langle{\binpar{D_{x}}{P}}\rangle}{D_{x}} \nonumber
\end{eqnarray}

\begin{eqnarray}
	\bangp_{x}{P} & & \nonumber\\
	=
	& {x}!\langle{(\prefix{x}{y}{(\outputp{x}{y} | @{y})) | P}}\rangle 
	      | \prefix{x}{y}{(\outputp{x}{y} | @{y})} & \nonumber\\
	\red
	& (\outputp{x}{y} | @{y})\substn{\quotep{(\prefix{x}{y}{(@{y} | \outputp{x}{y})) | P}}}{y} & \nonumber\\
	=
	& \outputp{x}{\quotep{(\prefix{x}{y}{(\outputp{x}{y} | @{y})) | P}}}
	  | {(\prefix{x}{y}{(\outputp{x}{y} | @{y})) | P}} & \nonumber\\
	\red
	& \ldots & \nonumber\\
	\red^*
	& P | P | \ldots & \nonumber
\end{eqnarray}

Of course, this encoding, as an implementation, runs away, unfolding
$\bangp{P}$ eagerly. A lazier and more implementable replication
operator, restricted to input-guarded processes, may be obtained as follows.

\begin{eqnarray}
\bangp{\prefix{u}{v}{P}} 
	:= 
	\binpar{\lift{x}{\prefix{u}{v}{(\binpar{D(x)}{P})}}}{D(x)} \nonumber
\end{eqnarray}

\begin{remark}
  Note that the lazier definition still does not deal with summation
  or mixed summation (i.e. sums over input and output). The reader is
  invited to construct definitions of replication that deal with these
  features. 

  Further, the definitions are parameterized in a name, $x$. Can you,
  gentle reader, make a definition that eliminates this parameter and
  guarantees no accidental interaction between the replication
  machinery and the process being replicated -- i.e. no accidental
  sharing of names used by the process to get its work done and the
  name(s) used by the replication to effect copying. This latter
  revision of the definition of replication is crucial to obtaining
  the expected identity $!!P \sim !P$.
\end{remark}

\begin{remark}\label{rem:paradoxical_combinator}
  The reader familiar with the lambda calculus will have noticed the
  similarity between $D$ and the paradoxical combinator.

  [Ed. note: the existence of this seems to suggest we have to be more
  restrictive on the set of processes and names we admit if we are to
  support no-cloning.]
\end{remark}

\subsubsection{Bisimulation}

The computational dynamics gives rise to another kind of equivalence,
the equivalence of computational behavior. As previously mentioned
this is typically captured \emph{via} some form of bisimulation.

% The notion we use in this paper is weak barbed bisimulation
% \cite{milner91polyadicpi}.

The notion we use in this paper is derived from weak barbed
bisimulation \cite{milner91polyadicpi}. 

\begin{definition}
An \emph{observation relation}, $\downarrow_{\mathcal N}$, over a set
of names, $\mathcal N$, is the smallest relation satisfying the rules
below.

\infrule[Out-barb]{y \in {\mathcal N}, \; x \nameeq y}
		  {\outputp{x}{v} \downarrow_{\mathcal N} x}
\infrule[Par-barb]{\mbox{$P\downarrow_{\mathcal N} x$ or $Q\downarrow_{\mathcal N} x$}}
		  {\binpar{P}{Q} \downarrow_{\mathcal N} x}

We write $P \Downarrow_{\mathcal N} x$ if there is $Q$ such that 
$P \wred Q$ and $Q \downarrow_{\mathcal N} x$.
\end{definition}

\begin{definition}
%\label{def.bbisim}
An  ${\mathcal N}$-\emph{barbed bisimulation} over a set of names, ${\mathcal N}$, is a symmetric binary relation 
${\mathcal S}_{\mathcal N}$ between agents such that $P\rel{S}_{\mathcal N}Q$ implies:
\begin{enumerate}
\item If $P \red P'$ then $Q \wred Q'$ and $P'\rel{S}_{\mathcal N} Q'$.
\item If $P\downarrow_{\mathcal N} x$, then $Q\Downarrow_{\mathcal N} x$.
\end{enumerate}
$P$ is ${\mathcal N}$-barbed bisimilar to $Q$, written
$P \wbbisim_{\mathcal N} Q$, if $P \rel{S}_{\mathcal N} Q$ for some ${\mathcal N}$-barbed bisimulation ${\mathcal S}_{\mathcal N}$.
\end{definition}

$\mathcal{R} \subseteq \pi \times \pi$

$P \mathcal{R} Q => \forall P'. P \red P' \Rightarrow \exists Q'. Q \red Q', P' \mathcal{R} Q'$

$P \vdash x \Rightarrow Q \vdash x$

\begin{mathpar}
  \inferrule*[lab=Out-barb]{x \nameeq y}{{y}!\langle{Q}\rangle \vdash x}
  \and
  \inferrule*[lab=Par-barb]{\mbox{$P\vdash x$ or $Q\vdash x$}}{\binpar{P}{Q} \vdash x}
\end{mathpar}

\subsubsection{Contexts}

One of the principle advantages of computational calculi like the
$\pi$-calculus is a well-defined notion of context,
contextual-equivalence and a correlation between
contextual-equivalence and notions of bisimulation. The notion of
context allows the decomposition of a process into (sub-)process and
its syntactic environment, its context. Thus, a context may be
thought of as a process with a ``hole'' (written $\Box$) in it. The
application of a context $M$ to a process $P$, written $M[P]$, is
tantamount to filling the hole in $M$ with $P$. In this paper we do
not need the full weight of this theory, but do make use of the notion
of context in the proof the main theorem. 

\begin{mathpar}
  \inferrule* [lab=summation] {} {{M_{M},M_{N}} \bc \Box \;|\; x.M_{A} \;|\; M_{M}+M_{N}}
  \and
  \inferrule* [lab=agent] {} {{M_{A}} \bc (\vec{x})M_{P} \;| \; \clift{P_0,\ldots,M_{P},\ldots,P_N}}
  \and \\
  \inferrule* [lab=process] {} {{M_{P}} \bc M_{N} \;| \;P|M_{P} }
\end{mathpar} 

\begin{mathpar}
  \inferrule* [lab=sychronization] {} {M_{N} \bc \Box \;|\; x?M_{F} \;|\; x!M_{C}}
  \and
  \inferrule* [lab=abstraction] {} {{M_{F}} \bc (x)M_{P} }
  \and
  \inferrule* [lab=concretion] {} {{M_{C}} \bc \langle M_{P} \rangle }
  \and \\
  \inferrule* [lab=process] {} {{M_{P}} \bc M_{N} \;| \;P|M_{P} }
\end{mathpar}

\begin{definition}[contextual application] Given a context $M$, and
  process $P$, we define the \emph{contextual application}, $M[P] :=
  M\{P/\Box\}$. That is, the contextual application of M to P is the
  substitution of $P$ for $\Box$ in $M$.
\end{definition}

$\meaningof{-} : L \to \mathcal{P}(\pi)$

\begin{mathpar}
  \inferrule* [lab=collection] {} {\meaningof{true} = \pi, \and \meaningof{~E} = \pi \setminus \meaningof{E}, \and \meaningof{E_{1} \& E_{2}} = \meaningof{E_{1}} \cap \meaningof{E_{2}}}
\end{mathpar}

\begin{mathpar}
  \inferrule* [lab=structure] {} {\meaningof{0} = \{ P \in \pi | P \equiv 0 \}, \and \\ \meaningof{E_1 | E_2} = \{ P \in \pi | P \equiv P_{1} | P_{2}, P_{1} \in \meaningof{E_{1}}, P_{2} \in \meaningof{E_2}\} }
\end{mathpar}

\begin{mathpar}
 \inferrule* [lab=behavior] {} {\meaningof{\langle a?b \rangle E} = \{ P \in \pi | P \equiv Q | u?(y)P', \\ \and \\\\ \and \\ \;\;\; u \in \meaningof{a}, \forall z.P'\{z/y\} \in \meaningof{E\{z/b\}}\}, \and \\ \meaningof{a!E} = \{ P \in \pi | P \equiv Q | x!\langle P' \rangle, x \in \meaningof{a} P' \in \meaningof{E}\} }
\end{mathpar}

\begin{mathpar}
 \inferrule* [lab=nominal] {} {\meaningof{\quotep{E}} = \{ \quotep{P} \in \quotep{\pi} | P \in \meaningof{E} \}, \and \meaningof{\quotep{P}} = \{ \quotep{Q} \in \quotep{\pi} | P \equiv Q \} \and \\ \meaningof{@\quotep{E}} = \{ P \in \pi | P \equiv @x, x \in \meaningof{E} \}}
\end{mathpar}

\begin{eqnarray*}
  \\
  \meaningof{-} : TS \to ST
\end{eqnarray*}

\begin{eqnarray*}
  \\
  L : TS \to ST
\end{eqnarray*}

\begin{eqnarray*}
  \\
  P \models E \iff P \in \meaningof{E}
\end{eqnarray*}

\begin{eqnarray*}
  P \approx_{L} Q \iff \forall E \in L. P \models E \iff Q \models E
\end{eqnarray*}

\begin{eqnarray*}
  P \approx_{K} Q
\end{eqnarray*}

\begin{eqnarray*}
  P \approx Q
\end{eqnarray*}

$\approx_{K} = \approx = \approx_{L}$

\subsubsection{Contextual duality}

Note that contexts extend the quotation operation to a family of
operations from processes to names. Given a context, $M$, we can
define a \emph{nominal context}, $\quotep{M}$ by $\quotep{M}[P] :=
\quotep{M[P]}$. To foreshadow what is to come we observe that these
operations enjoy a duality with processes very much like the duality
between vectors and maps from vectors to scalars.

Further, because the calculus is essentially higher-order, we have a
correspondence between contexts and processes. More specifically,
given a name $x$ and a context $M$ we can construct $M^{*}_{x}$ such
that 

\begin{mathpar}
  M^{*}_{x} | \lift{x}{P} \red M[P]
\end{mathpar}

namely,

\begin{mathpar}
  M^{*}_{x} := x?(u).M[\dropn{u}]
\end{mathpar}

The dependence of $M^{*}_{x}$ on a name makes it an abstraction, 

\begin{mathpar}
  M^{*} := (x)x?(u).M[\dropn{u}]
\end{mathpar}

\subsection{Additional notation}

It will sometimes be convenient to denote the process a name
quotes. We already have the notation $x = \quotep{P}$, but it will be
convenient to introduce an alternate notation, $\procn{x}$, when we
want to emphasize the connection to the use of the name. Note that, by
virtue of name equivalence, $\quotep{\procn{x}} \nameeq x$; so, the
notation is consistent with previous definitions.

Further, because names have structure it is possible to effect
substitutions on the basis of that structure. This means we need to
upgrade our notation for substitutions, which we accomplish by
adapting comprehension notation. Thus,

\begin{mathpar}
  P\{ y / x : x \in S \}
\end{mathpar}

is interpreted to mean the process derived from P by replacing (in a
capture-avoiding manner) each occurrence of $x$ in $S$ by $y$. For example,

\begin{mathpar}
  P\{ \quotep{\procn{x}|\procn{x}} / x : x \in \freenames{P} \}
\end{mathpar}

will replace each (occurrence) of a free name $x$ in $P$ by
$\quotep{\procn{x}|\procn{x}}$.

Also, we will avail ourselves of the notation $x^{L}$ and $x^{R}$ to
denote injections of a name into disjoint copies of the name
space. There are numerous ways to accomplish this. One example can be
found in \cite{MeredithR05}. This notation overloads to vectors of
names: $\vec{x}^{\pi} := (x_{i}^{\pi} \; : \; 0 \leq i < |\vec{x}| )$ where $\pi \in \{L,R\}$.

We also use $P^{\Box} := P|\Box$.

In \cite{MeredithR05} an interpretation of the new operator is
given. It turns out that there are several possible interpretations
all enjoying the requisite algebraic properties of the operator (see
\cite{milner91polyadicpi}). We will therefore make liberal use of
$(\nu\; \vec{x})P$.

% subsection the_syntax_and_semantics_of_the_notation_system (end)   

\input{qm2pi.qmops} 

\input{qm2pi.sterngerlach} 

\input{qm2pi.metric} 

% section concurrent_process_calculi (end)

%\input{qm2pi.proofsketch}

% section proof sketch (end)

%\input{qm2pi.slviaknots} 

% section spatial logic via knots (end)

\input{qm2pi.conclusion}

% section conclusion (end)

%\input{qm2pi.dtcodes} 

% section wiring algorithm (end)

\input{qm2pi.ack} 

% section acknowledgments (end)

\newpage


\bibliographystyle{plain}   
\bibliography{../../biblios/main.bib}

\input{qm2pi.rhodetails}

\end{document}

 

% section acknowledgments (end)

\newpage


\bibliographystyle{plain}   
\bibliography{../../biblios/main.bib}

\documentclass[12pt]{llncs}
%\documentclass{jktr}

\usepackage[pdftex]{hyperref}                   
\usepackage {listings}
\usepackage {mathpartir}
\usepackage{bcprules}
%\usepackage{listings}
                       
\usepackage{graphicx} 
%\usepackage[margins=2.5cm,nohead,nofoot]{geometry}
%\usepackage{geometry}
\usepackage{amsfonts}
\usepackage{amstext}
\usepackage{latexsym}
\usepackage{amssymb}
\usepackage{color}


%\include{myPreamble}
\include{qm2pi.local} 

%\ifpdf
%\usepackage[pdftex]{graphicx}
%\else
%\usepackage{graphicx}
%\fi

 % \ifpdf
%  \usepackage{pdfsync}
%  \if


%\title{Brief Article}
%\author{David F. Snyder}
%\author{L.G. Meredith}

%\address{Dept. of Math., Texas State University--San Marcos, San Marcos, TX 78666}
       
\pagestyle{empty}


\begin{document}

\lstset{language=[Objective]Caml,frame=shadowbox}

\input{qm2pi.front}

% section front matter (end)

\input{qm2pi.intro} 
 
% section introduction (end)

% \input{qm2pi.knotations} 

% section notation (end)

\input{qm2pi.process.calculi} 

% section concurrent_process_calculi_and_spatial_logics_ (end)
    
%\input{qm2pi.knots2pi} 

%\input{qm2pi.trefoil} 

%\input{qm2pi.mainthm} 

% subsection basic_interpretation (end)

%\input{qm2pi.rho.presentation} 
\subsection{The syntax and semantics of the notation system}\label{sub:the_syntax_and_semantics_of_the_notation_system} % (fold)

We now summarize a technical presentation of the calculus that
embodies our theory of dynamics. The typical presentation of such a
calculus follows the style of giving generators and relations on
them. The grammar, below, describing term constructors, freely
generates the set of processes, $\Proc$. This set is then quotiented
by a relation known as structural congruence and it is over this set
that the notion of dynamics is expressed. This presentation is
essentially that of \cite{MeredithR05} with the addition of
polyadicity and summation. For readability we have relegated some of
the technical subtleties to an appendix.

\subsubsection{Process grammar}\label{subsub:process_grammar}

\begin{mathpar}
  \inferrule* [lab=synchronization] {} {{M} \bc \pzero \;|\; x?F \;|\; x!C }
  \and
  \inferrule* [lab=abstraction] {} {{F} \bc (x)P}
  \and
  \inferrule* [lab=concretion] {} {{C} \bc \langle Q \rangle}
  \and
  \inferrule* [lab=process] {} {{P,Q} \bc M \;| \;P|Q \;|\; @{x}}
  \and
  \inferrule* [lab=name] {} {{x} \bc \quotep{P}}
\end{mathpar} 

Note that $\vec{x}$ (resp. $\vec{P}$) denotes a vector of names
(resp. processes) of length $|\vec{x}|$ (resp. $|\vec{P}|$). We adopt
the following useful abbreviations.

\begin{mathpar}
   x?(\vec{y}).P := x.(\vec{y})P \and  x\clift{\vec{P}} := x.\clift{\vec{P}}
   \and x!(y) := \lift{x}{\dropn{y}}
   \and \Pi_{i=0}^{n-1}P_i := P_0 | \ldots | P_{n-1}
\end{mathpar}

\subsubsection{Structural congruence}

\paragraph{Free and bound names and alpha-equivalence.} At the
core of structural equivalence is alpha-equivalence which identifies
process that are the same up to a change of variable. Formally, we
recognize the distinction between free and bound names. The free names
of a process, $\freenames{P}$, may be calculated recursively as
follows:

\begin{mathpar}
\freenames{\pzero} := \emptyset
  \and \\
  \freenames{x?(y).P} := \{ x \} \cup (\freenames{P} \setminus \{ y \})
  \and 
  \freenames{x!\langle P \rangle} := \{ x \} \cup \{ P \} 
  \and \\
  \freenames{P|Q} := \freenames{P} \cup \freenames{Q}
  \and \\
  \freenames{@{x}} := \{ x \}
\end{mathpar}

$\pi$
$\quotep{\pi}$

$\freenames{-} : \pi \to \mathcal{P}(\quotep{\pi})$

\begin{eqnarray*}
  \freenames{\pzero} & := & \emptyset \\
  \freenames{x?(y).P} & := & \{ x \} \cup (\freenames{P} \setminus \{ y \}) \\
  \freenames{x!\langle P \rangle} & := & \{ x \} \cup \{ P \} \\
  \freenames{P|Q} & := & \freenames{P} \cup \freenames{Q} \\
  \freenames{\dropn{x}} & := & \{ x \}
\end{eqnarray*}

The bound names of a process, $\boundnames{P}$, are those names occurring in $P$
that are not free. For example, in $x?(y).0$, the name $x$ is free, while $y$ is bound.

\begin{mathpar}
  \inferrule* [lab=monoidal-laws] {} { P|Q \equiv Q|P \and P|0 \equiv P \and P|(Q|R) \equiv (P|Q)|R }
\end{mathpar}

\begin{mathpar}
  \inferrule* [lab=alpha-equivalence] {} { (x)P \equiv (y)P\{y/x\} \and y \not\in \freenames{P} }
\end{mathpar}

\begin{definition}
Then two processes, $P,Q$, are alpha-equivalent if $P = Q\{\vec{y}/\vec{x}\}$ for
some $\vec{x} \in \boundnames{Q},\vec{y} \in \boundnames{P}$, where $Q\{\vec{y}/\vec{x}\}$
denotes the capture-avoiding substitution of $\vec{y}$ for $\vec{x}$ in $Q$.
\end{definition}

\begin{definition}
  The {\em structural congruence} \cite{SangiorgiWalker} , $\equiv$,
  between processes is the least congruence containing
  alpha-equivalence, satisfying the abelian monoid laws
  (associativity, commutativity and $\pzero$ as identity) for parallel
  composition $|$ and for summation $+$.
\end{definition}

\subsection{Name equivalence}

We take name equivalence, written $\nameeq$, to be the smallest
equivalence relation generated by the following rules.

\begin{mathpar}
\inferrule*[lab=Quote-drop]
{ }
{ \quotep{@{x}} \nameeq x }

\inferrule*[lab=Struct-equiv]
{ P \scong Q }
{ \quotep{P} \nameeq \quotep{Q} }
\end{mathpar}

The astute reader will have noticed that the mutual recursion of names
and processes imposes a mutual recursion on alpha-equivalence and
structural equivalence via name-equivalence. Fortunately, all of this
works out pleasantly and we may calculate in the natural way, free of
concern. The reader interested in the details is referred to the
appendix \ref{appendix:rho_details}.

\subsection{Substitution}

We use $\Proc$ for the set of processes, $\QProc$ for the set of
names, and $\id{\{}\vec{y} / \vec{x} \id{\}}$ to denote partial maps,
$s : \QProc \rightarrow \QProc$. A map, $s$ lifts, uniquely, to a map
on process terms, $\widehat{s} : \Proc \rightarrow \Proc$ by the
following equations.

\begin{mathpar}
  (0) \psubstp{Q}{P} := 0 \\
  (R \juxtap S) \psubstp{Q}{P}
  :=    
  (R)\psubstp{Q}{P} \juxtap (S) \psubstp{Q}{P} \\
  (x?(y).R) \psubstp{Q}{P}    
  :=    
  (x)\substp{Q}{P} (z)\concat( (R \psubstn{z}{y}) \psubstp{Q}{P} ) \\
  (\lift{x}{R}) \psubstp{Q}{P}  
  :=
  \lift{(x)\substp{Q}{P}}{ R \psubstp{Q}{P} } \\
%   (\dropn{x})  \psubstp{Q}{P}       
%   := 
%   \left\{ 
%     \begin{array}{ccc} 
%       \dropn{\quotep{Q}} & & x \nameeq \quotep{P} \\
%       \dropn{x} & & otherwise \\
%     \end{array}
%   \right. 
  (\dropn{x})  \psubstp{Q}{P}       
  := 
  \left\{ 
    \begin{array}{ccc} 
      Q & & x \nameeq \quotep{P} \\
      \dropn{x} & & otherwise \\
    \end{array}
  \right.
\end{mathpar}
 

where

\begin{eqnarray}
  (x)\id{\{} \lpquote Q \rpquote / \lpquote P \rpquote \id{\}}            = 
  \left\{ 
    \begin{array}{ccc}
      \lpquote Q \rpquote & & x \nameeq \lpquote P \rpquote \\
      x & & otherwise \\
    \end{array}
  \right. \nonumber
\end{eqnarray}

and $z$ is chosen distinct from $\quotep{P}$, $\quotep{Q}$, the free
names in $Q$, and all the names in $R$. Our $\alpha$-equivalence will
be built in the standard way from this substitution.

\begin{remark}\label{rem:no_self_referential_names}
  One consequence of these definitions is that $\forall P. \quotep{P}
  \not\in \freenames{P}$.
\end{remark}

\subsection{ Dynamic quote: an example }

Anticipating something of what's to come, consider applying the
substitution, $\widehat{\id{\{}u / z \id{\}}}$, to the following pair
of processes, $\lift{w}{y!(z)}$ and $w[ \lpquote y!(z) \rpquote ]$.

\begin{eqnarray}
	\lift{w}{y!(z)}\widehat{\id{\{}u / z \id{\}}}
		& = &
		\lift{w}{y!(u)} \nonumber\\
	w[ \lpquote y!(z) \rpquote ] \widehat{ \id{\{}u / z \id{\}} }
		& = &
		w[ \lpquote y!(z) \rpquote ] \nonumber
\end{eqnarray}

Because the body of the process between quotes is impervious to
substitution, we get radically different answers. In fact, by
examining the first process in an input context,
e.g. $x?(z).\lift{w}{y!(z)}$, we see that the process under the lift
operator may be shaped by prefixed inputs binding a name inside it. In
this sense, the lift operator will be seen as a way to dynamically
construct processes before reifying them as names.

Finally equipped with these standard features we can present the
dynamics of the calculus.

\subsubsection{Operational semantics} 

Finally, we introduce the computational dynamics. What marks these
algebras as distinct from other more traditionally studied algebraic
structures, e.g. vector spaces or polynomial rings, is the manner in
which dynamics is captured. In traditional structures, dynamics is typically
expressed through morphisms between such structures, as in linear maps
between vector spaces or morphisms between rings. In algebras
associated with the semantics of computation, the dynamics is
expressed as part of the algebraic structure itself, through a
reduction reduction relation typically denoted by $\red$. Below, we
give a recursive presentation of this relation for the calculus used
in the encoding.

$\red \subseteq \pi \times \pi$
$\red : \pi \to \mathcal{P}(\pi)$

\begin{mathpar}
  \inferrule* [lab=Comm] { \textsf{match}( x_{src}, x_{trgt} ) } { x_{trgt}?(y)P \; | \; x_{src}!\langle {Q} \rangle \red P\{\quotep{Q}/y}\} }
  \and \\
  \inferrule* [lab=Par] {{P} \red {P}'} {{{P} | {Q}} \red {{P}' | {Q}}}
  \and
  \inferrule* [lab=Equiv]{{{P} \scong {P}'} \andalso {{P}' \red {Q}'} \andalso {{Q}' \scong {Q}}}{{P} \red {Q}}
\end{mathpar}

\begin{eqnarray*}
  match_{\equiv} (\quotep{P},\quotep{Q}) & := & P \equiv Q \\
  match_{\dagger}(\quotep{P},\quotep{Q}) & := & \forall R. P|Q \red^{*} R => R \red^{*} 0 \\
  match_{K}(\quotep{P},\quotep{Q}) & := & K \mbox{ for some context } K
\end{eqnarray*}

$u?(x)P | u!\langle Q \rangle \red P\{\quotep{Q}/x\}$

%We write $\wred$ for $\red^*$, and $P\red$ if $\exists Q $ such that $ P \red Q$.
We write $P\red$ if $\exists Q $ such that $ P \red Q$ and $P\not\red$, otherwise.

\section{Replication}

As mentioned before, it is known that replication (and hence
recursion) can be implemented in a higher-order process algebra
\cite{SangiorgiWalker}. As our first example of calculation with the
machinery thus far presented we give the construction explicitly in
the {\rhoc}.

\begin{eqnarray}
	D_{x} & := & \prefix{x}{y}{(\binpar{\outputp{x}{y}}{@{y}})} \nonumber\\
	\bangp_{x}{P} & := & \binpar{{x}!\langle{\binpar{D_{x}}{P}}\rangle}{D_{x}} \nonumber
\end{eqnarray}

\begin{eqnarray}
	\bangp_{x}{P} & & \nonumber\\
	=
	& {x}!\langle{(\prefix{x}{y}{(\outputp{x}{y} | @{y})) | P}}\rangle 
	      | \prefix{x}{y}{(\outputp{x}{y} | @{y})} & \nonumber\\
	\red
	& (\outputp{x}{y} | @{y})\substn{\quotep{(\prefix{x}{y}{(@{y} | \outputp{x}{y})) | P}}}{y} & \nonumber\\
	=
	& \outputp{x}{\quotep{(\prefix{x}{y}{(\outputp{x}{y} | @{y})) | P}}}
	  | {(\prefix{x}{y}{(\outputp{x}{y} | @{y})) | P}} & \nonumber\\
	\red
	& \ldots & \nonumber\\
	\red^*
	& P | P | \ldots & \nonumber
\end{eqnarray}

Of course, this encoding, as an implementation, runs away, unfolding
$\bangp{P}$ eagerly. A lazier and more implementable replication
operator, restricted to input-guarded processes, may be obtained as follows.

\begin{eqnarray}
\bangp{\prefix{u}{v}{P}} 
	:= 
	\binpar{\lift{x}{\prefix{u}{v}{(\binpar{D(x)}{P})}}}{D(x)} \nonumber
\end{eqnarray}

\begin{remark}
  Note that the lazier definition still does not deal with summation
  or mixed summation (i.e. sums over input and output). The reader is
  invited to construct definitions of replication that deal with these
  features. 

  Further, the definitions are parameterized in a name, $x$. Can you,
  gentle reader, make a definition that eliminates this parameter and
  guarantees no accidental interaction between the replication
  machinery and the process being replicated -- i.e. no accidental
  sharing of names used by the process to get its work done and the
  name(s) used by the replication to effect copying. This latter
  revision of the definition of replication is crucial to obtaining
  the expected identity $!!P \sim !P$.
\end{remark}

\begin{remark}\label{rem:paradoxical_combinator}
  The reader familiar with the lambda calculus will have noticed the
  similarity between $D$ and the paradoxical combinator.

  [Ed. note: the existence of this seems to suggest we have to be more
  restrictive on the set of processes and names we admit if we are to
  support no-cloning.]
\end{remark}

\subsubsection{Bisimulation}

The computational dynamics gives rise to another kind of equivalence,
the equivalence of computational behavior. As previously mentioned
this is typically captured \emph{via} some form of bisimulation.

% The notion we use in this paper is weak barbed bisimulation
% \cite{milner91polyadicpi}.

The notion we use in this paper is derived from weak barbed
bisimulation \cite{milner91polyadicpi}. 

\begin{definition}
An \emph{observation relation}, $\downarrow_{\mathcal N}$, over a set
of names, $\mathcal N$, is the smallest relation satisfying the rules
below.

\infrule[Out-barb]{y \in {\mathcal N}, \; x \nameeq y}
		  {\outputp{x}{v} \downarrow_{\mathcal N} x}
\infrule[Par-barb]{\mbox{$P\downarrow_{\mathcal N} x$ or $Q\downarrow_{\mathcal N} x$}}
		  {\binpar{P}{Q} \downarrow_{\mathcal N} x}

We write $P \Downarrow_{\mathcal N} x$ if there is $Q$ such that 
$P \wred Q$ and $Q \downarrow_{\mathcal N} x$.
\end{definition}

\begin{definition}
%\label{def.bbisim}
An  ${\mathcal N}$-\emph{barbed bisimulation} over a set of names, ${\mathcal N}$, is a symmetric binary relation 
${\mathcal S}_{\mathcal N}$ between agents such that $P\rel{S}_{\mathcal N}Q$ implies:
\begin{enumerate}
\item If $P \red P'$ then $Q \wred Q'$ and $P'\rel{S}_{\mathcal N} Q'$.
\item If $P\downarrow_{\mathcal N} x$, then $Q\Downarrow_{\mathcal N} x$.
\end{enumerate}
$P$ is ${\mathcal N}$-barbed bisimilar to $Q$, written
$P \wbbisim_{\mathcal N} Q$, if $P \rel{S}_{\mathcal N} Q$ for some ${\mathcal N}$-barbed bisimulation ${\mathcal S}_{\mathcal N}$.
\end{definition}

$\mathcal{R} \subseteq \pi \times \pi$

$P \mathcal{R} Q => \forall P'. P \red P' \Rightarrow \exists Q'. Q \red Q', P' \mathcal{R} Q'$

$P \vdash x \Rightarrow Q \vdash x$

\begin{mathpar}
  \inferrule*[lab=Out-barb]{x \nameeq y}{{y}!\langle{Q}\rangle \vdash x}
  \and
  \inferrule*[lab=Par-barb]{\mbox{$P\vdash x$ or $Q\vdash x$}}{\binpar{P}{Q} \vdash x}
\end{mathpar}

\subsubsection{Contexts}

One of the principle advantages of computational calculi like the
$\pi$-calculus is a well-defined notion of context,
contextual-equivalence and a correlation between
contextual-equivalence and notions of bisimulation. The notion of
context allows the decomposition of a process into (sub-)process and
its syntactic environment, its context. Thus, a context may be
thought of as a process with a ``hole'' (written $\Box$) in it. The
application of a context $M$ to a process $P$, written $M[P]$, is
tantamount to filling the hole in $M$ with $P$. In this paper we do
not need the full weight of this theory, but do make use of the notion
of context in the proof the main theorem. 

\begin{mathpar}
  \inferrule* [lab=summation] {} {{M_{M},M_{N}} \bc \Box \;|\; x.M_{A} \;|\; M_{M}+M_{N}}
  \and
  \inferrule* [lab=agent] {} {{M_{A}} \bc (\vec{x})M_{P} \;| \; \clift{P_0,\ldots,M_{P},\ldots,P_N}}
  \and \\
  \inferrule* [lab=process] {} {{M_{P}} \bc M_{N} \;| \;P|M_{P} }
\end{mathpar} 

\begin{mathpar}
  \inferrule* [lab=sychronization] {} {M_{N} \bc \Box \;|\; x?M_{F} \;|\; x!M_{C}}
  \and
  \inferrule* [lab=abstraction] {} {{M_{F}} \bc (x)M_{P} }
  \and
  \inferrule* [lab=concretion] {} {{M_{C}} \bc \langle M_{P} \rangle }
  \and \\
  \inferrule* [lab=process] {} {{M_{P}} \bc M_{N} \;| \;P|M_{P} }
\end{mathpar}

\begin{definition}[contextual application] Given a context $M$, and
  process $P$, we define the \emph{contextual application}, $M[P] :=
  M\{P/\Box\}$. That is, the contextual application of M to P is the
  substitution of $P$ for $\Box$ in $M$.
\end{definition}

$\meaningof{-} : L \to \mathcal{P}(\pi)$

\begin{mathpar}
  \inferrule* [lab=collection] {} {\meaningof{true} = \pi, \and \meaningof{~E} = \pi \setminus \meaningof{E}, \and \meaningof{E_{1} \& E_{2}} = \meaningof{E_{1}} \cap \meaningof{E_{2}}}
\end{mathpar}

\begin{mathpar}
  \inferrule* [lab=structure] {} {\meaningof{0} = \{ P \in \pi | P \equiv 0 \}, \and \\ \meaningof{E_1 | E_2} = \{ P \in \pi | P \equiv P_{1} | P_{2}, P_{1} \in \meaningof{E_{1}}, P_{2} \in \meaningof{E_2}\} }
\end{mathpar}

\begin{mathpar}
 \inferrule* [lab=behavior] {} {\meaningof{\langle a?b \rangle E} = \{ P \in \pi | P \equiv Q | u?(y)P', \\ \and \\\\ \and \\ \;\;\; u \in \meaningof{a}, \forall z.P'\{z/y\} \in \meaningof{E\{z/b\}}\}, \and \\ \meaningof{a!E} = \{ P \in \pi | P \equiv Q | x!\langle P' \rangle, x \in \meaningof{a} P' \in \meaningof{E}\} }
\end{mathpar}

\begin{mathpar}
 \inferrule* [lab=nominal] {} {\meaningof{\quotep{E}} = \{ \quotep{P} \in \quotep{\pi} | P \in \meaningof{E} \}, \and \meaningof{\quotep{P}} = \{ \quotep{Q} \in \quotep{\pi} | P \equiv Q \} \and \\ \meaningof{@\quotep{E}} = \{ P \in \pi | P \equiv @x, x \in \meaningof{E} \}}
\end{mathpar}

\begin{eqnarray*}
  \\
  \meaningof{-} : TS \to ST
\end{eqnarray*}

\begin{eqnarray*}
  \\
  L : TS \to ST
\end{eqnarray*}

\begin{eqnarray*}
  \\
  P \models E \iff P \in \meaningof{E}
\end{eqnarray*}

\begin{eqnarray*}
  P \approx_{L} Q \iff \forall E \in L. P \models E \iff Q \models E
\end{eqnarray*}

\begin{eqnarray*}
  P \approx_{K} Q
\end{eqnarray*}

\begin{eqnarray*}
  P \approx Q
\end{eqnarray*}

$\approx_{K} = \approx = \approx_{L}$

\subsubsection{Contextual duality}

Note that contexts extend the quotation operation to a family of
operations from processes to names. Given a context, $M$, we can
define a \emph{nominal context}, $\quotep{M}$ by $\quotep{M}[P] :=
\quotep{M[P]}$. To foreshadow what is to come we observe that these
operations enjoy a duality with processes very much like the duality
between vectors and maps from vectors to scalars.

Further, because the calculus is essentially higher-order, we have a
correspondence between contexts and processes. More specifically,
given a name $x$ and a context $M$ we can construct $M^{*}_{x}$ such
that 

\begin{mathpar}
  M^{*}_{x} | \lift{x}{P} \red M[P]
\end{mathpar}

namely,

\begin{mathpar}
  M^{*}_{x} := x?(u).M[\dropn{u}]
\end{mathpar}

The dependence of $M^{*}_{x}$ on a name makes it an abstraction, 

\begin{mathpar}
  M^{*} := (x)x?(u).M[\dropn{u}]
\end{mathpar}

\subsection{Additional notation}

It will sometimes be convenient to denote the process a name
quotes. We already have the notation $x = \quotep{P}$, but it will be
convenient to introduce an alternate notation, $\procn{x}$, when we
want to emphasize the connection to the use of the name. Note that, by
virtue of name equivalence, $\quotep{\procn{x}} \nameeq x$; so, the
notation is consistent with previous definitions.

Further, because names have structure it is possible to effect
substitutions on the basis of that structure. This means we need to
upgrade our notation for substitutions, which we accomplish by
adapting comprehension notation. Thus,

\begin{mathpar}
  P\{ y / x : x \in S \}
\end{mathpar}

is interpreted to mean the process derived from P by replacing (in a
capture-avoiding manner) each occurrence of $x$ in $S$ by $y$. For example,

\begin{mathpar}
  P\{ \quotep{\procn{x}|\procn{x}} / x : x \in \freenames{P} \}
\end{mathpar}

will replace each (occurrence) of a free name $x$ in $P$ by
$\quotep{\procn{x}|\procn{x}}$.

Also, we will avail ourselves of the notation $x^{L}$ and $x^{R}$ to
denote injections of a name into disjoint copies of the name
space. There are numerous ways to accomplish this. One example can be
found in \cite{MeredithR05}. This notation overloads to vectors of
names: $\vec{x}^{\pi} := (x_{i}^{\pi} \; : \; 0 \leq i < |\vec{x}| )$ where $\pi \in \{L,R\}$.

We also use $P^{\Box} := P|\Box$.

In \cite{MeredithR05} an interpretation of the new operator is
given. It turns out that there are several possible interpretations
all enjoying the requisite algebraic properties of the operator (see
\cite{milner91polyadicpi}). We will therefore make liberal use of
$(\nu\; \vec{x})P$.

% subsection the_syntax_and_semantics_of_the_notation_system (end)   

\input{qm2pi.qmops} 

\input{qm2pi.sterngerlach} 

\input{qm2pi.metric} 

% section concurrent_process_calculi (end)

%\input{qm2pi.proofsketch}

% section proof sketch (end)

%\input{qm2pi.slviaknots} 

% section spatial logic via knots (end)

\input{qm2pi.conclusion}

% section conclusion (end)

%\input{qm2pi.dtcodes} 

% section wiring algorithm (end)

\input{qm2pi.ack} 

% section acknowledgments (end)

\newpage


\bibliographystyle{plain}   
\bibliography{../../biblios/main.bib}

\input{qm2pi.rhodetails}

\end{document}



\end{document}



\end{document}

 

% section wiring algorithm (end)

\documentclass[12pt]{llncs}
%\documentclass{jktr}

\usepackage[pdftex]{hyperref}                   
\usepackage {listings}
\usepackage {mathpartir}
\usepackage{bcprules}
%\usepackage{listings}
                       
\usepackage{graphicx} 
%\usepackage[margins=2.5cm,nohead,nofoot]{geometry}
%\usepackage{geometry}
\usepackage{amsfonts}
\usepackage{amstext}
\usepackage{latexsym}
\usepackage{amssymb}
\usepackage{color}


%\include{myPreamble}
\documentclass[12pt]{llncs}
%\documentclass{jktr}

\usepackage[pdftex]{hyperref}                   
\usepackage {listings}
\usepackage {mathpartir}
\usepackage{bcprules}
%\usepackage{listings}
                       
\usepackage{graphicx} 
%\usepackage[margins=2.5cm,nohead,nofoot]{geometry}
%\usepackage{geometry}
\usepackage{amsfonts}
\usepackage{amstext}
\usepackage{latexsym}
\usepackage{amssymb}
\usepackage{color}


%\include{myPreamble}
\documentclass[12pt]{llncs}
%\documentclass{jktr}

\usepackage[pdftex]{hyperref}                   
\usepackage {listings}
\usepackage {mathpartir}
\usepackage{bcprules}
%\usepackage{listings}
                       
\usepackage{graphicx} 
%\usepackage[margins=2.5cm,nohead,nofoot]{geometry}
%\usepackage{geometry}
\usepackage{amsfonts}
\usepackage{amstext}
\usepackage{latexsym}
\usepackage{amssymb}
\usepackage{color}


%\include{myPreamble}
\include{qm2pi.local} 

%\ifpdf
%\usepackage[pdftex]{graphicx}
%\else
%\usepackage{graphicx}
%\fi

 % \ifpdf
%  \usepackage{pdfsync}
%  \if


%\title{Brief Article}
%\author{David F. Snyder}
%\author{L.G. Meredith}

%\address{Dept. of Math., Texas State University--San Marcos, San Marcos, TX 78666}
       
\pagestyle{empty}


\begin{document}

\lstset{language=[Objective]Caml,frame=shadowbox}

\input{qm2pi.front}

% section front matter (end)

\input{qm2pi.intro} 
 
% section introduction (end)

% \input{qm2pi.knotations} 

% section notation (end)

\input{qm2pi.process.calculi} 

% section concurrent_process_calculi_and_spatial_logics_ (end)
    
%\input{qm2pi.knots2pi} 

%\input{qm2pi.trefoil} 

%\input{qm2pi.mainthm} 

% subsection basic_interpretation (end)

%\input{qm2pi.rho.presentation} 
\subsection{The syntax and semantics of the notation system}\label{sub:the_syntax_and_semantics_of_the_notation_system} % (fold)

We now summarize a technical presentation of the calculus that
embodies our theory of dynamics. The typical presentation of such a
calculus follows the style of giving generators and relations on
them. The grammar, below, describing term constructors, freely
generates the set of processes, $\Proc$. This set is then quotiented
by a relation known as structural congruence and it is over this set
that the notion of dynamics is expressed. This presentation is
essentially that of \cite{MeredithR05} with the addition of
polyadicity and summation. For readability we have relegated some of
the technical subtleties to an appendix.

\subsubsection{Process grammar}\label{subsub:process_grammar}

\begin{mathpar}
  \inferrule* [lab=synchronization] {} {{M} \bc \pzero \;|\; x?F \;|\; x!C }
  \and
  \inferrule* [lab=abstraction] {} {{F} \bc (x)P}
  \and
  \inferrule* [lab=concretion] {} {{C} \bc \langle Q \rangle}
  \and
  \inferrule* [lab=process] {} {{P,Q} \bc M \;| \;P|Q \;|\; @{x}}
  \and
  \inferrule* [lab=name] {} {{x} \bc \quotep{P}}
\end{mathpar} 

Note that $\vec{x}$ (resp. $\vec{P}$) denotes a vector of names
(resp. processes) of length $|\vec{x}|$ (resp. $|\vec{P}|$). We adopt
the following useful abbreviations.

\begin{mathpar}
   x?(\vec{y}).P := x.(\vec{y})P \and  x\clift{\vec{P}} := x.\clift{\vec{P}}
   \and x!(y) := \lift{x}{\dropn{y}}
   \and \Pi_{i=0}^{n-1}P_i := P_0 | \ldots | P_{n-1}
\end{mathpar}

\subsubsection{Structural congruence}

\paragraph{Free and bound names and alpha-equivalence.} At the
core of structural equivalence is alpha-equivalence which identifies
process that are the same up to a change of variable. Formally, we
recognize the distinction between free and bound names. The free names
of a process, $\freenames{P}$, may be calculated recursively as
follows:

\begin{mathpar}
\freenames{\pzero} := \emptyset
  \and \\
  \freenames{x?(y).P} := \{ x \} \cup (\freenames{P} \setminus \{ y \})
  \and 
  \freenames{x!\langle P \rangle} := \{ x \} \cup \{ P \} 
  \and \\
  \freenames{P|Q} := \freenames{P} \cup \freenames{Q}
  \and \\
  \freenames{@{x}} := \{ x \}
\end{mathpar}

$\pi$
$\quotep{\pi}$

$\freenames{-} : \pi \to \mathcal{P}(\quotep{\pi})$

\begin{eqnarray*}
  \freenames{\pzero} & := & \emptyset \\
  \freenames{x?(y).P} & := & \{ x \} \cup (\freenames{P} \setminus \{ y \}) \\
  \freenames{x!\langle P \rangle} & := & \{ x \} \cup \{ P \} \\
  \freenames{P|Q} & := & \freenames{P} \cup \freenames{Q} \\
  \freenames{\dropn{x}} & := & \{ x \}
\end{eqnarray*}

The bound names of a process, $\boundnames{P}$, are those names occurring in $P$
that are not free. For example, in $x?(y).0$, the name $x$ is free, while $y$ is bound.

\begin{mathpar}
  \inferrule* [lab=monoidal-laws] {} { P|Q \equiv Q|P \and P|0 \equiv P \and P|(Q|R) \equiv (P|Q)|R }
\end{mathpar}

\begin{mathpar}
  \inferrule* [lab=alpha-equivalence] {} { (x)P \equiv (y)P\{y/x\} \and y \not\in \freenames{P} }
\end{mathpar}

\begin{definition}
Then two processes, $P,Q$, are alpha-equivalent if $P = Q\{\vec{y}/\vec{x}\}$ for
some $\vec{x} \in \boundnames{Q},\vec{y} \in \boundnames{P}$, where $Q\{\vec{y}/\vec{x}\}$
denotes the capture-avoiding substitution of $\vec{y}$ for $\vec{x}$ in $Q$.
\end{definition}

\begin{definition}
  The {\em structural congruence} \cite{SangiorgiWalker} , $\equiv$,
  between processes is the least congruence containing
  alpha-equivalence, satisfying the abelian monoid laws
  (associativity, commutativity and $\pzero$ as identity) for parallel
  composition $|$ and for summation $+$.
\end{definition}

\subsection{Name equivalence}

We take name equivalence, written $\nameeq$, to be the smallest
equivalence relation generated by the following rules.

\begin{mathpar}
\inferrule*[lab=Quote-drop]
{ }
{ \quotep{@{x}} \nameeq x }

\inferrule*[lab=Struct-equiv]
{ P \scong Q }
{ \quotep{P} \nameeq \quotep{Q} }
\end{mathpar}

The astute reader will have noticed that the mutual recursion of names
and processes imposes a mutual recursion on alpha-equivalence and
structural equivalence via name-equivalence. Fortunately, all of this
works out pleasantly and we may calculate in the natural way, free of
concern. The reader interested in the details is referred to the
appendix \ref{appendix:rho_details}.

\subsection{Substitution}

We use $\Proc$ for the set of processes, $\QProc$ for the set of
names, and $\id{\{}\vec{y} / \vec{x} \id{\}}$ to denote partial maps,
$s : \QProc \rightarrow \QProc$. A map, $s$ lifts, uniquely, to a map
on process terms, $\widehat{s} : \Proc \rightarrow \Proc$ by the
following equations.

\begin{mathpar}
  (0) \psubstp{Q}{P} := 0 \\
  (R \juxtap S) \psubstp{Q}{P}
  :=    
  (R)\psubstp{Q}{P} \juxtap (S) \psubstp{Q}{P} \\
  (x?(y).R) \psubstp{Q}{P}    
  :=    
  (x)\substp{Q}{P} (z)\concat( (R \psubstn{z}{y}) \psubstp{Q}{P} ) \\
  (\lift{x}{R}) \psubstp{Q}{P}  
  :=
  \lift{(x)\substp{Q}{P}}{ R \psubstp{Q}{P} } \\
%   (\dropn{x})  \psubstp{Q}{P}       
%   := 
%   \left\{ 
%     \begin{array}{ccc} 
%       \dropn{\quotep{Q}} & & x \nameeq \quotep{P} \\
%       \dropn{x} & & otherwise \\
%     \end{array}
%   \right. 
  (\dropn{x})  \psubstp{Q}{P}       
  := 
  \left\{ 
    \begin{array}{ccc} 
      Q & & x \nameeq \quotep{P} \\
      \dropn{x} & & otherwise \\
    \end{array}
  \right.
\end{mathpar}
 

where

\begin{eqnarray}
  (x)\id{\{} \lpquote Q \rpquote / \lpquote P \rpquote \id{\}}            = 
  \left\{ 
    \begin{array}{ccc}
      \lpquote Q \rpquote & & x \nameeq \lpquote P \rpquote \\
      x & & otherwise \\
    \end{array}
  \right. \nonumber
\end{eqnarray}

and $z$ is chosen distinct from $\quotep{P}$, $\quotep{Q}$, the free
names in $Q$, and all the names in $R$. Our $\alpha$-equivalence will
be built in the standard way from this substitution.

\begin{remark}\label{rem:no_self_referential_names}
  One consequence of these definitions is that $\forall P. \quotep{P}
  \not\in \freenames{P}$.
\end{remark}

\subsection{ Dynamic quote: an example }

Anticipating something of what's to come, consider applying the
substitution, $\widehat{\id{\{}u / z \id{\}}}$, to the following pair
of processes, $\lift{w}{y!(z)}$ and $w[ \lpquote y!(z) \rpquote ]$.

\begin{eqnarray}
	\lift{w}{y!(z)}\widehat{\id{\{}u / z \id{\}}}
		& = &
		\lift{w}{y!(u)} \nonumber\\
	w[ \lpquote y!(z) \rpquote ] \widehat{ \id{\{}u / z \id{\}} }
		& = &
		w[ \lpquote y!(z) \rpquote ] \nonumber
\end{eqnarray}

Because the body of the process between quotes is impervious to
substitution, we get radically different answers. In fact, by
examining the first process in an input context,
e.g. $x?(z).\lift{w}{y!(z)}$, we see that the process under the lift
operator may be shaped by prefixed inputs binding a name inside it. In
this sense, the lift operator will be seen as a way to dynamically
construct processes before reifying them as names.

Finally equipped with these standard features we can present the
dynamics of the calculus.

\subsubsection{Operational semantics} 

Finally, we introduce the computational dynamics. What marks these
algebras as distinct from other more traditionally studied algebraic
structures, e.g. vector spaces or polynomial rings, is the manner in
which dynamics is captured. In traditional structures, dynamics is typically
expressed through morphisms between such structures, as in linear maps
between vector spaces or morphisms between rings. In algebras
associated with the semantics of computation, the dynamics is
expressed as part of the algebraic structure itself, through a
reduction reduction relation typically denoted by $\red$. Below, we
give a recursive presentation of this relation for the calculus used
in the encoding.

$\red \subseteq \pi \times \pi$
$\red : \pi \to \mathcal{P}(\pi)$

\begin{mathpar}
  \inferrule* [lab=Comm] { \textsf{match}( x_{src}, x_{trgt} ) } { x_{trgt}?(y)P \; | \; x_{src}!\langle {Q} \rangle \red P\{\quotep{Q}/y}\} }
  \and \\
  \inferrule* [lab=Par] {{P} \red {P}'} {{{P} | {Q}} \red {{P}' | {Q}}}
  \and
  \inferrule* [lab=Equiv]{{{P} \scong {P}'} \andalso {{P}' \red {Q}'} \andalso {{Q}' \scong {Q}}}{{P} \red {Q}}
\end{mathpar}

\begin{eqnarray*}
  match_{\equiv} (\quotep{P},\quotep{Q}) & := & P \equiv Q \\
  match_{\dagger}(\quotep{P},\quotep{Q}) & := & \forall R. P|Q \red^{*} R => R \red^{*} 0 \\
  match_{K}(\quotep{P},\quotep{Q}) & := & K \mbox{ for some context } K
\end{eqnarray*}

$u?(x)P | u!\langle Q \rangle \red P\{\quotep{Q}/x\}$

%We write $\wred$ for $\red^*$, and $P\red$ if $\exists Q $ such that $ P \red Q$.
We write $P\red$ if $\exists Q $ such that $ P \red Q$ and $P\not\red$, otherwise.

\section{Replication}

As mentioned before, it is known that replication (and hence
recursion) can be implemented in a higher-order process algebra
\cite{SangiorgiWalker}. As our first example of calculation with the
machinery thus far presented we give the construction explicitly in
the {\rhoc}.

\begin{eqnarray}
	D_{x} & := & \prefix{x}{y}{(\binpar{\outputp{x}{y}}{@{y}})} \nonumber\\
	\bangp_{x}{P} & := & \binpar{{x}!\langle{\binpar{D_{x}}{P}}\rangle}{D_{x}} \nonumber
\end{eqnarray}

\begin{eqnarray}
	\bangp_{x}{P} & & \nonumber\\
	=
	& {x}!\langle{(\prefix{x}{y}{(\outputp{x}{y} | @{y})) | P}}\rangle 
	      | \prefix{x}{y}{(\outputp{x}{y} | @{y})} & \nonumber\\
	\red
	& (\outputp{x}{y} | @{y})\substn{\quotep{(\prefix{x}{y}{(@{y} | \outputp{x}{y})) | P}}}{y} & \nonumber\\
	=
	& \outputp{x}{\quotep{(\prefix{x}{y}{(\outputp{x}{y} | @{y})) | P}}}
	  | {(\prefix{x}{y}{(\outputp{x}{y} | @{y})) | P}} & \nonumber\\
	\red
	& \ldots & \nonumber\\
	\red^*
	& P | P | \ldots & \nonumber
\end{eqnarray}

Of course, this encoding, as an implementation, runs away, unfolding
$\bangp{P}$ eagerly. A lazier and more implementable replication
operator, restricted to input-guarded processes, may be obtained as follows.

\begin{eqnarray}
\bangp{\prefix{u}{v}{P}} 
	:= 
	\binpar{\lift{x}{\prefix{u}{v}{(\binpar{D(x)}{P})}}}{D(x)} \nonumber
\end{eqnarray}

\begin{remark}
  Note that the lazier definition still does not deal with summation
  or mixed summation (i.e. sums over input and output). The reader is
  invited to construct definitions of replication that deal with these
  features. 

  Further, the definitions are parameterized in a name, $x$. Can you,
  gentle reader, make a definition that eliminates this parameter and
  guarantees no accidental interaction between the replication
  machinery and the process being replicated -- i.e. no accidental
  sharing of names used by the process to get its work done and the
  name(s) used by the replication to effect copying. This latter
  revision of the definition of replication is crucial to obtaining
  the expected identity $!!P \sim !P$.
\end{remark}

\begin{remark}\label{rem:paradoxical_combinator}
  The reader familiar with the lambda calculus will have noticed the
  similarity between $D$ and the paradoxical combinator.

  [Ed. note: the existence of this seems to suggest we have to be more
  restrictive on the set of processes and names we admit if we are to
  support no-cloning.]
\end{remark}

\subsubsection{Bisimulation}

The computational dynamics gives rise to another kind of equivalence,
the equivalence of computational behavior. As previously mentioned
this is typically captured \emph{via} some form of bisimulation.

% The notion we use in this paper is weak barbed bisimulation
% \cite{milner91polyadicpi}.

The notion we use in this paper is derived from weak barbed
bisimulation \cite{milner91polyadicpi}. 

\begin{definition}
An \emph{observation relation}, $\downarrow_{\mathcal N}$, over a set
of names, $\mathcal N$, is the smallest relation satisfying the rules
below.

\infrule[Out-barb]{y \in {\mathcal N}, \; x \nameeq y}
		  {\outputp{x}{v} \downarrow_{\mathcal N} x}
\infrule[Par-barb]{\mbox{$P\downarrow_{\mathcal N} x$ or $Q\downarrow_{\mathcal N} x$}}
		  {\binpar{P}{Q} \downarrow_{\mathcal N} x}

We write $P \Downarrow_{\mathcal N} x$ if there is $Q$ such that 
$P \wred Q$ and $Q \downarrow_{\mathcal N} x$.
\end{definition}

\begin{definition}
%\label{def.bbisim}
An  ${\mathcal N}$-\emph{barbed bisimulation} over a set of names, ${\mathcal N}$, is a symmetric binary relation 
${\mathcal S}_{\mathcal N}$ between agents such that $P\rel{S}_{\mathcal N}Q$ implies:
\begin{enumerate}
\item If $P \red P'$ then $Q \wred Q'$ and $P'\rel{S}_{\mathcal N} Q'$.
\item If $P\downarrow_{\mathcal N} x$, then $Q\Downarrow_{\mathcal N} x$.
\end{enumerate}
$P$ is ${\mathcal N}$-barbed bisimilar to $Q$, written
$P \wbbisim_{\mathcal N} Q$, if $P \rel{S}_{\mathcal N} Q$ for some ${\mathcal N}$-barbed bisimulation ${\mathcal S}_{\mathcal N}$.
\end{definition}

$\mathcal{R} \subseteq \pi \times \pi$

$P \mathcal{R} Q => \forall P'. P \red P' \Rightarrow \exists Q'. Q \red Q', P' \mathcal{R} Q'$

$P \vdash x \Rightarrow Q \vdash x$

\begin{mathpar}
  \inferrule*[lab=Out-barb]{x \nameeq y}{{y}!\langle{Q}\rangle \vdash x}
  \and
  \inferrule*[lab=Par-barb]{\mbox{$P\vdash x$ or $Q\vdash x$}}{\binpar{P}{Q} \vdash x}
\end{mathpar}

\subsubsection{Contexts}

One of the principle advantages of computational calculi like the
$\pi$-calculus is a well-defined notion of context,
contextual-equivalence and a correlation between
contextual-equivalence and notions of bisimulation. The notion of
context allows the decomposition of a process into (sub-)process and
its syntactic environment, its context. Thus, a context may be
thought of as a process with a ``hole'' (written $\Box$) in it. The
application of a context $M$ to a process $P$, written $M[P]$, is
tantamount to filling the hole in $M$ with $P$. In this paper we do
not need the full weight of this theory, but do make use of the notion
of context in the proof the main theorem. 

\begin{mathpar}
  \inferrule* [lab=summation] {} {{M_{M},M_{N}} \bc \Box \;|\; x.M_{A} \;|\; M_{M}+M_{N}}
  \and
  \inferrule* [lab=agent] {} {{M_{A}} \bc (\vec{x})M_{P} \;| \; \clift{P_0,\ldots,M_{P},\ldots,P_N}}
  \and \\
  \inferrule* [lab=process] {} {{M_{P}} \bc M_{N} \;| \;P|M_{P} }
\end{mathpar} 

\begin{mathpar}
  \inferrule* [lab=sychronization] {} {M_{N} \bc \Box \;|\; x?M_{F} \;|\; x!M_{C}}
  \and
  \inferrule* [lab=abstraction] {} {{M_{F}} \bc (x)M_{P} }
  \and
  \inferrule* [lab=concretion] {} {{M_{C}} \bc \langle M_{P} \rangle }
  \and \\
  \inferrule* [lab=process] {} {{M_{P}} \bc M_{N} \;| \;P|M_{P} }
\end{mathpar}

\begin{definition}[contextual application] Given a context $M$, and
  process $P$, we define the \emph{contextual application}, $M[P] :=
  M\{P/\Box\}$. That is, the contextual application of M to P is the
  substitution of $P$ for $\Box$ in $M$.
\end{definition}

$\meaningof{-} : L \to \mathcal{P}(\pi)$

\begin{mathpar}
  \inferrule* [lab=collection] {} {\meaningof{true} = \pi, \and \meaningof{~E} = \pi \setminus \meaningof{E}, \and \meaningof{E_{1} \& E_{2}} = \meaningof{E_{1}} \cap \meaningof{E_{2}}}
\end{mathpar}

\begin{mathpar}
  \inferrule* [lab=structure] {} {\meaningof{0} = \{ P \in \pi | P \equiv 0 \}, \and \\ \meaningof{E_1 | E_2} = \{ P \in \pi | P \equiv P_{1} | P_{2}, P_{1} \in \meaningof{E_{1}}, P_{2} \in \meaningof{E_2}\} }
\end{mathpar}

\begin{mathpar}
 \inferrule* [lab=behavior] {} {\meaningof{\langle a?b \rangle E} = \{ P \in \pi | P \equiv Q | u?(y)P', \\ \and \\\\ \and \\ \;\;\; u \in \meaningof{a}, \forall z.P'\{z/y\} \in \meaningof{E\{z/b\}}\}, \and \\ \meaningof{a!E} = \{ P \in \pi | P \equiv Q | x!\langle P' \rangle, x \in \meaningof{a} P' \in \meaningof{E}\} }
\end{mathpar}

\begin{mathpar}
 \inferrule* [lab=nominal] {} {\meaningof{\quotep{E}} = \{ \quotep{P} \in \quotep{\pi} | P \in \meaningof{E} \}, \and \meaningof{\quotep{P}} = \{ \quotep{Q} \in \quotep{\pi} | P \equiv Q \} \and \\ \meaningof{@\quotep{E}} = \{ P \in \pi | P \equiv @x, x \in \meaningof{E} \}}
\end{mathpar}

\begin{eqnarray*}
  \\
  \meaningof{-} : TS \to ST
\end{eqnarray*}

\begin{eqnarray*}
  \\
  L : TS \to ST
\end{eqnarray*}

\begin{eqnarray*}
  \\
  P \models E \iff P \in \meaningof{E}
\end{eqnarray*}

\begin{eqnarray*}
  P \approx_{L} Q \iff \forall E \in L. P \models E \iff Q \models E
\end{eqnarray*}

\begin{eqnarray*}
  P \approx_{K} Q
\end{eqnarray*}

\begin{eqnarray*}
  P \approx Q
\end{eqnarray*}

$\approx_{K} = \approx = \approx_{L}$

\subsubsection{Contextual duality}

Note that contexts extend the quotation operation to a family of
operations from processes to names. Given a context, $M$, we can
define a \emph{nominal context}, $\quotep{M}$ by $\quotep{M}[P] :=
\quotep{M[P]}$. To foreshadow what is to come we observe that these
operations enjoy a duality with processes very much like the duality
between vectors and maps from vectors to scalars.

Further, because the calculus is essentially higher-order, we have a
correspondence between contexts and processes. More specifically,
given a name $x$ and a context $M$ we can construct $M^{*}_{x}$ such
that 

\begin{mathpar}
  M^{*}_{x} | \lift{x}{P} \red M[P]
\end{mathpar}

namely,

\begin{mathpar}
  M^{*}_{x} := x?(u).M[\dropn{u}]
\end{mathpar}

The dependence of $M^{*}_{x}$ on a name makes it an abstraction, 

\begin{mathpar}
  M^{*} := (x)x?(u).M[\dropn{u}]
\end{mathpar}

\subsection{Additional notation}

It will sometimes be convenient to denote the process a name
quotes. We already have the notation $x = \quotep{P}$, but it will be
convenient to introduce an alternate notation, $\procn{x}$, when we
want to emphasize the connection to the use of the name. Note that, by
virtue of name equivalence, $\quotep{\procn{x}} \nameeq x$; so, the
notation is consistent with previous definitions.

Further, because names have structure it is possible to effect
substitutions on the basis of that structure. This means we need to
upgrade our notation for substitutions, which we accomplish by
adapting comprehension notation. Thus,

\begin{mathpar}
  P\{ y / x : x \in S \}
\end{mathpar}

is interpreted to mean the process derived from P by replacing (in a
capture-avoiding manner) each occurrence of $x$ in $S$ by $y$. For example,

\begin{mathpar}
  P\{ \quotep{\procn{x}|\procn{x}} / x : x \in \freenames{P} \}
\end{mathpar}

will replace each (occurrence) of a free name $x$ in $P$ by
$\quotep{\procn{x}|\procn{x}}$.

Also, we will avail ourselves of the notation $x^{L}$ and $x^{R}$ to
denote injections of a name into disjoint copies of the name
space. There are numerous ways to accomplish this. One example can be
found in \cite{MeredithR05}. This notation overloads to vectors of
names: $\vec{x}^{\pi} := (x_{i}^{\pi} \; : \; 0 \leq i < |\vec{x}| )$ where $\pi \in \{L,R\}$.

We also use $P^{\Box} := P|\Box$.

In \cite{MeredithR05} an interpretation of the new operator is
given. It turns out that there are several possible interpretations
all enjoying the requisite algebraic properties of the operator (see
\cite{milner91polyadicpi}). We will therefore make liberal use of
$(\nu\; \vec{x})P$.

% subsection the_syntax_and_semantics_of_the_notation_system (end)   

\input{qm2pi.qmops} 

\input{qm2pi.sterngerlach} 

\input{qm2pi.metric} 

% section concurrent_process_calculi (end)

%\input{qm2pi.proofsketch}

% section proof sketch (end)

%\input{qm2pi.slviaknots} 

% section spatial logic via knots (end)

\input{qm2pi.conclusion}

% section conclusion (end)

%\input{qm2pi.dtcodes} 

% section wiring algorithm (end)

\input{qm2pi.ack} 

% section acknowledgments (end)

\newpage


\bibliographystyle{plain}   
\bibliography{../../biblios/main.bib}

\input{qm2pi.rhodetails}

\end{document}

 

%\ifpdf
%\usepackage[pdftex]{graphicx}
%\else
%\usepackage{graphicx}
%\fi

 % \ifpdf
%  \usepackage{pdfsync}
%  \if


%\title{Brief Article}
%\author{David F. Snyder}
%\author{L.G. Meredith}

%\address{Dept. of Math., Texas State University--San Marcos, San Marcos, TX 78666}
       
\pagestyle{empty}


\begin{document}

\lstset{language=[Objective]Caml,frame=shadowbox}

\documentclass[12pt]{llncs}
%\documentclass{jktr}

\usepackage[pdftex]{hyperref}                   
\usepackage {listings}
\usepackage {mathpartir}
\usepackage{bcprules}
%\usepackage{listings}
                       
\usepackage{graphicx} 
%\usepackage[margins=2.5cm,nohead,nofoot]{geometry}
%\usepackage{geometry}
\usepackage{amsfonts}
\usepackage{amstext}
\usepackage{latexsym}
\usepackage{amssymb}
\usepackage{color}


%\include{myPreamble}
\include{qm2pi.local} 

%\ifpdf
%\usepackage[pdftex]{graphicx}
%\else
%\usepackage{graphicx}
%\fi

 % \ifpdf
%  \usepackage{pdfsync}
%  \if


%\title{Brief Article}
%\author{David F. Snyder}
%\author{L.G. Meredith}

%\address{Dept. of Math., Texas State University--San Marcos, San Marcos, TX 78666}
       
\pagestyle{empty}


\begin{document}

\lstset{language=[Objective]Caml,frame=shadowbox}

\input{qm2pi.front}

% section front matter (end)

\input{qm2pi.intro} 
 
% section introduction (end)

% \input{qm2pi.knotations} 

% section notation (end)

\input{qm2pi.process.calculi} 

% section concurrent_process_calculi_and_spatial_logics_ (end)
    
%\input{qm2pi.knots2pi} 

%\input{qm2pi.trefoil} 

%\input{qm2pi.mainthm} 

% subsection basic_interpretation (end)

%\input{qm2pi.rho.presentation} 
\subsection{The syntax and semantics of the notation system}\label{sub:the_syntax_and_semantics_of_the_notation_system} % (fold)

We now summarize a technical presentation of the calculus that
embodies our theory of dynamics. The typical presentation of such a
calculus follows the style of giving generators and relations on
them. The grammar, below, describing term constructors, freely
generates the set of processes, $\Proc$. This set is then quotiented
by a relation known as structural congruence and it is over this set
that the notion of dynamics is expressed. This presentation is
essentially that of \cite{MeredithR05} with the addition of
polyadicity and summation. For readability we have relegated some of
the technical subtleties to an appendix.

\subsubsection{Process grammar}\label{subsub:process_grammar}

\begin{mathpar}
  \inferrule* [lab=synchronization] {} {{M} \bc \pzero \;|\; x?F \;|\; x!C }
  \and
  \inferrule* [lab=abstraction] {} {{F} \bc (x)P}
  \and
  \inferrule* [lab=concretion] {} {{C} \bc \langle Q \rangle}
  \and
  \inferrule* [lab=process] {} {{P,Q} \bc M \;| \;P|Q \;|\; @{x}}
  \and
  \inferrule* [lab=name] {} {{x} \bc \quotep{P}}
\end{mathpar} 

Note that $\vec{x}$ (resp. $\vec{P}$) denotes a vector of names
(resp. processes) of length $|\vec{x}|$ (resp. $|\vec{P}|$). We adopt
the following useful abbreviations.

\begin{mathpar}
   x?(\vec{y}).P := x.(\vec{y})P \and  x\clift{\vec{P}} := x.\clift{\vec{P}}
   \and x!(y) := \lift{x}{\dropn{y}}
   \and \Pi_{i=0}^{n-1}P_i := P_0 | \ldots | P_{n-1}
\end{mathpar}

\subsubsection{Structural congruence}

\paragraph{Free and bound names and alpha-equivalence.} At the
core of structural equivalence is alpha-equivalence which identifies
process that are the same up to a change of variable. Formally, we
recognize the distinction between free and bound names. The free names
of a process, $\freenames{P}$, may be calculated recursively as
follows:

\begin{mathpar}
\freenames{\pzero} := \emptyset
  \and \\
  \freenames{x?(y).P} := \{ x \} \cup (\freenames{P} \setminus \{ y \})
  \and 
  \freenames{x!\langle P \rangle} := \{ x \} \cup \{ P \} 
  \and \\
  \freenames{P|Q} := \freenames{P} \cup \freenames{Q}
  \and \\
  \freenames{@{x}} := \{ x \}
\end{mathpar}

$\pi$
$\quotep{\pi}$

$\freenames{-} : \pi \to \mathcal{P}(\quotep{\pi})$

\begin{eqnarray*}
  \freenames{\pzero} & := & \emptyset \\
  \freenames{x?(y).P} & := & \{ x \} \cup (\freenames{P} \setminus \{ y \}) \\
  \freenames{x!\langle P \rangle} & := & \{ x \} \cup \{ P \} \\
  \freenames{P|Q} & := & \freenames{P} \cup \freenames{Q} \\
  \freenames{\dropn{x}} & := & \{ x \}
\end{eqnarray*}

The bound names of a process, $\boundnames{P}$, are those names occurring in $P$
that are not free. For example, in $x?(y).0$, the name $x$ is free, while $y$ is bound.

\begin{mathpar}
  \inferrule* [lab=monoidal-laws] {} { P|Q \equiv Q|P \and P|0 \equiv P \and P|(Q|R) \equiv (P|Q)|R }
\end{mathpar}

\begin{mathpar}
  \inferrule* [lab=alpha-equivalence] {} { (x)P \equiv (y)P\{y/x\} \and y \not\in \freenames{P} }
\end{mathpar}

\begin{definition}
Then two processes, $P,Q$, are alpha-equivalent if $P = Q\{\vec{y}/\vec{x}\}$ for
some $\vec{x} \in \boundnames{Q},\vec{y} \in \boundnames{P}$, where $Q\{\vec{y}/\vec{x}\}$
denotes the capture-avoiding substitution of $\vec{y}$ for $\vec{x}$ in $Q$.
\end{definition}

\begin{definition}
  The {\em structural congruence} \cite{SangiorgiWalker} , $\equiv$,
  between processes is the least congruence containing
  alpha-equivalence, satisfying the abelian monoid laws
  (associativity, commutativity and $\pzero$ as identity) for parallel
  composition $|$ and for summation $+$.
\end{definition}

\subsection{Name equivalence}

We take name equivalence, written $\nameeq$, to be the smallest
equivalence relation generated by the following rules.

\begin{mathpar}
\inferrule*[lab=Quote-drop]
{ }
{ \quotep{@{x}} \nameeq x }

\inferrule*[lab=Struct-equiv]
{ P \scong Q }
{ \quotep{P} \nameeq \quotep{Q} }
\end{mathpar}

The astute reader will have noticed that the mutual recursion of names
and processes imposes a mutual recursion on alpha-equivalence and
structural equivalence via name-equivalence. Fortunately, all of this
works out pleasantly and we may calculate in the natural way, free of
concern. The reader interested in the details is referred to the
appendix \ref{appendix:rho_details}.

\subsection{Substitution}

We use $\Proc$ for the set of processes, $\QProc$ for the set of
names, and $\id{\{}\vec{y} / \vec{x} \id{\}}$ to denote partial maps,
$s : \QProc \rightarrow \QProc$. A map, $s$ lifts, uniquely, to a map
on process terms, $\widehat{s} : \Proc \rightarrow \Proc$ by the
following equations.

\begin{mathpar}
  (0) \psubstp{Q}{P} := 0 \\
  (R \juxtap S) \psubstp{Q}{P}
  :=    
  (R)\psubstp{Q}{P} \juxtap (S) \psubstp{Q}{P} \\
  (x?(y).R) \psubstp{Q}{P}    
  :=    
  (x)\substp{Q}{P} (z)\concat( (R \psubstn{z}{y}) \psubstp{Q}{P} ) \\
  (\lift{x}{R}) \psubstp{Q}{P}  
  :=
  \lift{(x)\substp{Q}{P}}{ R \psubstp{Q}{P} } \\
%   (\dropn{x})  \psubstp{Q}{P}       
%   := 
%   \left\{ 
%     \begin{array}{ccc} 
%       \dropn{\quotep{Q}} & & x \nameeq \quotep{P} \\
%       \dropn{x} & & otherwise \\
%     \end{array}
%   \right. 
  (\dropn{x})  \psubstp{Q}{P}       
  := 
  \left\{ 
    \begin{array}{ccc} 
      Q & & x \nameeq \quotep{P} \\
      \dropn{x} & & otherwise \\
    \end{array}
  \right.
\end{mathpar}
 

where

\begin{eqnarray}
  (x)\id{\{} \lpquote Q \rpquote / \lpquote P \rpquote \id{\}}            = 
  \left\{ 
    \begin{array}{ccc}
      \lpquote Q \rpquote & & x \nameeq \lpquote P \rpquote \\
      x & & otherwise \\
    \end{array}
  \right. \nonumber
\end{eqnarray}

and $z$ is chosen distinct from $\quotep{P}$, $\quotep{Q}$, the free
names in $Q$, and all the names in $R$. Our $\alpha$-equivalence will
be built in the standard way from this substitution.

\begin{remark}\label{rem:no_self_referential_names}
  One consequence of these definitions is that $\forall P. \quotep{P}
  \not\in \freenames{P}$.
\end{remark}

\subsection{ Dynamic quote: an example }

Anticipating something of what's to come, consider applying the
substitution, $\widehat{\id{\{}u / z \id{\}}}$, to the following pair
of processes, $\lift{w}{y!(z)}$ and $w[ \lpquote y!(z) \rpquote ]$.

\begin{eqnarray}
	\lift{w}{y!(z)}\widehat{\id{\{}u / z \id{\}}}
		& = &
		\lift{w}{y!(u)} \nonumber\\
	w[ \lpquote y!(z) \rpquote ] \widehat{ \id{\{}u / z \id{\}} }
		& = &
		w[ \lpquote y!(z) \rpquote ] \nonumber
\end{eqnarray}

Because the body of the process between quotes is impervious to
substitution, we get radically different answers. In fact, by
examining the first process in an input context,
e.g. $x?(z).\lift{w}{y!(z)}$, we see that the process under the lift
operator may be shaped by prefixed inputs binding a name inside it. In
this sense, the lift operator will be seen as a way to dynamically
construct processes before reifying them as names.

Finally equipped with these standard features we can present the
dynamics of the calculus.

\subsubsection{Operational semantics} 

Finally, we introduce the computational dynamics. What marks these
algebras as distinct from other more traditionally studied algebraic
structures, e.g. vector spaces or polynomial rings, is the manner in
which dynamics is captured. In traditional structures, dynamics is typically
expressed through morphisms between such structures, as in linear maps
between vector spaces or morphisms between rings. In algebras
associated with the semantics of computation, the dynamics is
expressed as part of the algebraic structure itself, through a
reduction reduction relation typically denoted by $\red$. Below, we
give a recursive presentation of this relation for the calculus used
in the encoding.

$\red \subseteq \pi \times \pi$
$\red : \pi \to \mathcal{P}(\pi)$

\begin{mathpar}
  \inferrule* [lab=Comm] { \textsf{match}( x_{src}, x_{trgt} ) } { x_{trgt}?(y)P \; | \; x_{src}!\langle {Q} \rangle \red P\{\quotep{Q}/y}\} }
  \and \\
  \inferrule* [lab=Par] {{P} \red {P}'} {{{P} | {Q}} \red {{P}' | {Q}}}
  \and
  \inferrule* [lab=Equiv]{{{P} \scong {P}'} \andalso {{P}' \red {Q}'} \andalso {{Q}' \scong {Q}}}{{P} \red {Q}}
\end{mathpar}

\begin{eqnarray*}
  match_{\equiv} (\quotep{P},\quotep{Q}) & := & P \equiv Q \\
  match_{\dagger}(\quotep{P},\quotep{Q}) & := & \forall R. P|Q \red^{*} R => R \red^{*} 0 \\
  match_{K}(\quotep{P},\quotep{Q}) & := & K \mbox{ for some context } K
\end{eqnarray*}

$u?(x)P | u!\langle Q \rangle \red P\{\quotep{Q}/x\}$

%We write $\wred$ for $\red^*$, and $P\red$ if $\exists Q $ such that $ P \red Q$.
We write $P\red$ if $\exists Q $ such that $ P \red Q$ and $P\not\red$, otherwise.

\section{Replication}

As mentioned before, it is known that replication (and hence
recursion) can be implemented in a higher-order process algebra
\cite{SangiorgiWalker}. As our first example of calculation with the
machinery thus far presented we give the construction explicitly in
the {\rhoc}.

\begin{eqnarray}
	D_{x} & := & \prefix{x}{y}{(\binpar{\outputp{x}{y}}{@{y}})} \nonumber\\
	\bangp_{x}{P} & := & \binpar{{x}!\langle{\binpar{D_{x}}{P}}\rangle}{D_{x}} \nonumber
\end{eqnarray}

\begin{eqnarray}
	\bangp_{x}{P} & & \nonumber\\
	=
	& {x}!\langle{(\prefix{x}{y}{(\outputp{x}{y} | @{y})) | P}}\rangle 
	      | \prefix{x}{y}{(\outputp{x}{y} | @{y})} & \nonumber\\
	\red
	& (\outputp{x}{y} | @{y})\substn{\quotep{(\prefix{x}{y}{(@{y} | \outputp{x}{y})) | P}}}{y} & \nonumber\\
	=
	& \outputp{x}{\quotep{(\prefix{x}{y}{(\outputp{x}{y} | @{y})) | P}}}
	  | {(\prefix{x}{y}{(\outputp{x}{y} | @{y})) | P}} & \nonumber\\
	\red
	& \ldots & \nonumber\\
	\red^*
	& P | P | \ldots & \nonumber
\end{eqnarray}

Of course, this encoding, as an implementation, runs away, unfolding
$\bangp{P}$ eagerly. A lazier and more implementable replication
operator, restricted to input-guarded processes, may be obtained as follows.

\begin{eqnarray}
\bangp{\prefix{u}{v}{P}} 
	:= 
	\binpar{\lift{x}{\prefix{u}{v}{(\binpar{D(x)}{P})}}}{D(x)} \nonumber
\end{eqnarray}

\begin{remark}
  Note that the lazier definition still does not deal with summation
  or mixed summation (i.e. sums over input and output). The reader is
  invited to construct definitions of replication that deal with these
  features. 

  Further, the definitions are parameterized in a name, $x$. Can you,
  gentle reader, make a definition that eliminates this parameter and
  guarantees no accidental interaction between the replication
  machinery and the process being replicated -- i.e. no accidental
  sharing of names used by the process to get its work done and the
  name(s) used by the replication to effect copying. This latter
  revision of the definition of replication is crucial to obtaining
  the expected identity $!!P \sim !P$.
\end{remark}

\begin{remark}\label{rem:paradoxical_combinator}
  The reader familiar with the lambda calculus will have noticed the
  similarity between $D$ and the paradoxical combinator.

  [Ed. note: the existence of this seems to suggest we have to be more
  restrictive on the set of processes and names we admit if we are to
  support no-cloning.]
\end{remark}

\subsubsection{Bisimulation}

The computational dynamics gives rise to another kind of equivalence,
the equivalence of computational behavior. As previously mentioned
this is typically captured \emph{via} some form of bisimulation.

% The notion we use in this paper is weak barbed bisimulation
% \cite{milner91polyadicpi}.

The notion we use in this paper is derived from weak barbed
bisimulation \cite{milner91polyadicpi}. 

\begin{definition}
An \emph{observation relation}, $\downarrow_{\mathcal N}$, over a set
of names, $\mathcal N$, is the smallest relation satisfying the rules
below.

\infrule[Out-barb]{y \in {\mathcal N}, \; x \nameeq y}
		  {\outputp{x}{v} \downarrow_{\mathcal N} x}
\infrule[Par-barb]{\mbox{$P\downarrow_{\mathcal N} x$ or $Q\downarrow_{\mathcal N} x$}}
		  {\binpar{P}{Q} \downarrow_{\mathcal N} x}

We write $P \Downarrow_{\mathcal N} x$ if there is $Q$ such that 
$P \wred Q$ and $Q \downarrow_{\mathcal N} x$.
\end{definition}

\begin{definition}
%\label{def.bbisim}
An  ${\mathcal N}$-\emph{barbed bisimulation} over a set of names, ${\mathcal N}$, is a symmetric binary relation 
${\mathcal S}_{\mathcal N}$ between agents such that $P\rel{S}_{\mathcal N}Q$ implies:
\begin{enumerate}
\item If $P \red P'$ then $Q \wred Q'$ and $P'\rel{S}_{\mathcal N} Q'$.
\item If $P\downarrow_{\mathcal N} x$, then $Q\Downarrow_{\mathcal N} x$.
\end{enumerate}
$P$ is ${\mathcal N}$-barbed bisimilar to $Q$, written
$P \wbbisim_{\mathcal N} Q$, if $P \rel{S}_{\mathcal N} Q$ for some ${\mathcal N}$-barbed bisimulation ${\mathcal S}_{\mathcal N}$.
\end{definition}

$\mathcal{R} \subseteq \pi \times \pi$

$P \mathcal{R} Q => \forall P'. P \red P' \Rightarrow \exists Q'. Q \red Q', P' \mathcal{R} Q'$

$P \vdash x \Rightarrow Q \vdash x$

\begin{mathpar}
  \inferrule*[lab=Out-barb]{x \nameeq y}{{y}!\langle{Q}\rangle \vdash x}
  \and
  \inferrule*[lab=Par-barb]{\mbox{$P\vdash x$ or $Q\vdash x$}}{\binpar{P}{Q} \vdash x}
\end{mathpar}

\subsubsection{Contexts}

One of the principle advantages of computational calculi like the
$\pi$-calculus is a well-defined notion of context,
contextual-equivalence and a correlation between
contextual-equivalence and notions of bisimulation. The notion of
context allows the decomposition of a process into (sub-)process and
its syntactic environment, its context. Thus, a context may be
thought of as a process with a ``hole'' (written $\Box$) in it. The
application of a context $M$ to a process $P$, written $M[P]$, is
tantamount to filling the hole in $M$ with $P$. In this paper we do
not need the full weight of this theory, but do make use of the notion
of context in the proof the main theorem. 

\begin{mathpar}
  \inferrule* [lab=summation] {} {{M_{M},M_{N}} \bc \Box \;|\; x.M_{A} \;|\; M_{M}+M_{N}}
  \and
  \inferrule* [lab=agent] {} {{M_{A}} \bc (\vec{x})M_{P} \;| \; \clift{P_0,\ldots,M_{P},\ldots,P_N}}
  \and \\
  \inferrule* [lab=process] {} {{M_{P}} \bc M_{N} \;| \;P|M_{P} }
\end{mathpar} 

\begin{mathpar}
  \inferrule* [lab=sychronization] {} {M_{N} \bc \Box \;|\; x?M_{F} \;|\; x!M_{C}}
  \and
  \inferrule* [lab=abstraction] {} {{M_{F}} \bc (x)M_{P} }
  \and
  \inferrule* [lab=concretion] {} {{M_{C}} \bc \langle M_{P} \rangle }
  \and \\
  \inferrule* [lab=process] {} {{M_{P}} \bc M_{N} \;| \;P|M_{P} }
\end{mathpar}

\begin{definition}[contextual application] Given a context $M$, and
  process $P$, we define the \emph{contextual application}, $M[P] :=
  M\{P/\Box\}$. That is, the contextual application of M to P is the
  substitution of $P$ for $\Box$ in $M$.
\end{definition}

$\meaningof{-} : L \to \mathcal{P}(\pi)$

\begin{mathpar}
  \inferrule* [lab=collection] {} {\meaningof{true} = \pi, \and \meaningof{~E} = \pi \setminus \meaningof{E}, \and \meaningof{E_{1} \& E_{2}} = \meaningof{E_{1}} \cap \meaningof{E_{2}}}
\end{mathpar}

\begin{mathpar}
  \inferrule* [lab=structure] {} {\meaningof{0} = \{ P \in \pi | P \equiv 0 \}, \and \\ \meaningof{E_1 | E_2} = \{ P \in \pi | P \equiv P_{1} | P_{2}, P_{1} \in \meaningof{E_{1}}, P_{2} \in \meaningof{E_2}\} }
\end{mathpar}

\begin{mathpar}
 \inferrule* [lab=behavior] {} {\meaningof{\langle a?b \rangle E} = \{ P \in \pi | P \equiv Q | u?(y)P', \\ \and \\\\ \and \\ \;\;\; u \in \meaningof{a}, \forall z.P'\{z/y\} \in \meaningof{E\{z/b\}}\}, \and \\ \meaningof{a!E} = \{ P \in \pi | P \equiv Q | x!\langle P' \rangle, x \in \meaningof{a} P' \in \meaningof{E}\} }
\end{mathpar}

\begin{mathpar}
 \inferrule* [lab=nominal] {} {\meaningof{\quotep{E}} = \{ \quotep{P} \in \quotep{\pi} | P \in \meaningof{E} \}, \and \meaningof{\quotep{P}} = \{ \quotep{Q} \in \quotep{\pi} | P \equiv Q \} \and \\ \meaningof{@\quotep{E}} = \{ P \in \pi | P \equiv @x, x \in \meaningof{E} \}}
\end{mathpar}

\begin{eqnarray*}
  \\
  \meaningof{-} : TS \to ST
\end{eqnarray*}

\begin{eqnarray*}
  \\
  L : TS \to ST
\end{eqnarray*}

\begin{eqnarray*}
  \\
  P \models E \iff P \in \meaningof{E}
\end{eqnarray*}

\begin{eqnarray*}
  P \approx_{L} Q \iff \forall E \in L. P \models E \iff Q \models E
\end{eqnarray*}

\begin{eqnarray*}
  P \approx_{K} Q
\end{eqnarray*}

\begin{eqnarray*}
  P \approx Q
\end{eqnarray*}

$\approx_{K} = \approx = \approx_{L}$

\subsubsection{Contextual duality}

Note that contexts extend the quotation operation to a family of
operations from processes to names. Given a context, $M$, we can
define a \emph{nominal context}, $\quotep{M}$ by $\quotep{M}[P] :=
\quotep{M[P]}$. To foreshadow what is to come we observe that these
operations enjoy a duality with processes very much like the duality
between vectors and maps from vectors to scalars.

Further, because the calculus is essentially higher-order, we have a
correspondence between contexts and processes. More specifically,
given a name $x$ and a context $M$ we can construct $M^{*}_{x}$ such
that 

\begin{mathpar}
  M^{*}_{x} | \lift{x}{P} \red M[P]
\end{mathpar}

namely,

\begin{mathpar}
  M^{*}_{x} := x?(u).M[\dropn{u}]
\end{mathpar}

The dependence of $M^{*}_{x}$ on a name makes it an abstraction, 

\begin{mathpar}
  M^{*} := (x)x?(u).M[\dropn{u}]
\end{mathpar}

\subsection{Additional notation}

It will sometimes be convenient to denote the process a name
quotes. We already have the notation $x = \quotep{P}$, but it will be
convenient to introduce an alternate notation, $\procn{x}$, when we
want to emphasize the connection to the use of the name. Note that, by
virtue of name equivalence, $\quotep{\procn{x}} \nameeq x$; so, the
notation is consistent with previous definitions.

Further, because names have structure it is possible to effect
substitutions on the basis of that structure. This means we need to
upgrade our notation for substitutions, which we accomplish by
adapting comprehension notation. Thus,

\begin{mathpar}
  P\{ y / x : x \in S \}
\end{mathpar}

is interpreted to mean the process derived from P by replacing (in a
capture-avoiding manner) each occurrence of $x$ in $S$ by $y$. For example,

\begin{mathpar}
  P\{ \quotep{\procn{x}|\procn{x}} / x : x \in \freenames{P} \}
\end{mathpar}

will replace each (occurrence) of a free name $x$ in $P$ by
$\quotep{\procn{x}|\procn{x}}$.

Also, we will avail ourselves of the notation $x^{L}$ and $x^{R}$ to
denote injections of a name into disjoint copies of the name
space. There are numerous ways to accomplish this. One example can be
found in \cite{MeredithR05}. This notation overloads to vectors of
names: $\vec{x}^{\pi} := (x_{i}^{\pi} \; : \; 0 \leq i < |\vec{x}| )$ where $\pi \in \{L,R\}$.

We also use $P^{\Box} := P|\Box$.

In \cite{MeredithR05} an interpretation of the new operator is
given. It turns out that there are several possible interpretations
all enjoying the requisite algebraic properties of the operator (see
\cite{milner91polyadicpi}). We will therefore make liberal use of
$(\nu\; \vec{x})P$.

% subsection the_syntax_and_semantics_of_the_notation_system (end)   

\input{qm2pi.qmops} 

\input{qm2pi.sterngerlach} 

\input{qm2pi.metric} 

% section concurrent_process_calculi (end)

%\input{qm2pi.proofsketch}

% section proof sketch (end)

%\input{qm2pi.slviaknots} 

% section spatial logic via knots (end)

\input{qm2pi.conclusion}

% section conclusion (end)

%\input{qm2pi.dtcodes} 

% section wiring algorithm (end)

\input{qm2pi.ack} 

% section acknowledgments (end)

\newpage


\bibliographystyle{plain}   
\bibliography{../../biblios/main.bib}

\input{qm2pi.rhodetails}

\end{document}



% section front matter (end)

\section{Introduction}\label{sec:introduction} % (fold)
In this draft of the material i am going to have to dispense with the
usual writing conventions adopted in papers on these topics. i'm going
to have adopt whatever tone i need at the time i'm writing up the
calculations. Sometimes this may be very conversational; others it may
be the barest mathematical grunts; others still it may be that i have
lifted text from one of my other papers because the exposition of some
point was better said there. i hope that my readers are not unduly put
out by this decision. i'm not doing this to flout convention or be
rebellious. i find these calculations very technically challenging. To
keep everything going technically, something has to give; i have to
let go of some cognitive burden. So, the academic writing style --
with all of its trade-offs in terms of facilitating technical
communication -- is what i'm letting go of. Perhaps subsequent drafts
can be tightened and polished, but for now, i'm going to speak as if
we were sitting together in a coffee shop with a laptop, wifi and a
pad of paper and a pencil.

So, here's what i have to say. We -- you and i, comfortably ensconced
in our coffee shop and well-equipped with our tools -- can realize and
carry out the calculations of quantum mechanics over a very different
formal theory of dynamics, a formal theory of dynamics that
corresponds to a theory of concurrent computation with
\emph{reflection}. It has the advantage that the underlying theory is
already `quantized', but supports analogues all of the continuuous
operations. Strikingly, this underlying theory has recently been
connected with a notion of metric that we can show, by calculating
together, coincides with the metric induced by the inner product.

There are a lot of reasons why you might be interested in seeing
calculations of this form. Here's why i'm interested. For the past
several centuries there has been no competitor to the ``Newtonian''
account of dynamics. As a result the predominant share of accounts of
dynamical systems and situations have had to be formulated in terms of
the Newtonian machinery. i view this as an intellectually dangerous
position to occupy. Everything, despite it's intrinsic shape, turns
into a nail to be hit with this hammer. Recently, however, the theory
of computation has matured to the point where we have candidates for
theories of dynamics that offer very different perspective on
reasoning about dynamical systems and situations. Testing these
candidates against very successful accounts of dynamical situations,
like quantum mechanics, is going to give us some sense of how mature
they are and some measure of the quality of these accounts of
dynamics.

\subsection{Summary of contributions and outline of paper}

So, we're going to develop an interpretation of the operations of
quantum mechanics normally interpreted by Hilbert spaces and
operators. We're going to do this over a theory of computation. Note
that this is very different than the usual quantum computation program
which develops notions of computation over quantum mechanics. Rather,
we are developing a story that aligns with Wheeler's slogan: It from
Bit. To do this we will first provide an account of the theory of
computation at play here. Then we will dive into a calculation-driven
interpretation of the operations of quantum mechanics.

The reason we take this approach is that -- until very recently --
there hasn't been an axiomatic account of quantum mechanics. As a
result there has been no sharp delineation of the mathematical theory
supporting interpretation of the physical theory and the physical
theory, itself. So, ambient features of the maths are free to be
exploited (or supressed) without a real accounting of their physical
relevance. There is no sharp statement ``here's the physical theory''
qua \emph{theory} and ``here's the mathematical interpretation''
enabling a judgment of how faithful the interpretation is -- apart
from experimental observation. When there is an axiomatic account we
can judge how well a given mathematical formalism supports an
interpretation of the axioms, independent of
experimentation. Likewise, we can judge how well we have captured our
physical evidence and experience with our axiomatics, independent of
any specific mathematical implementation, with accidental detail that
may or may not have physical significance. 

In lieu of a fully fleshed out and vetted axiomatic account of quantum
mechanics, interpreting the operational notions in service of modeling
physical systems will have to suffice. In other words, we are not in
the business of providing a model of Hilbert spaces and operators. We
are in the business of providing a model of quantum mechanics because
we are motivated by testing our notions of dynamics against physical
theory; and, the predictive calculations of the physical theory must
serve as the best formulation -- shy of a fully fleshed out axiomatic
account -- of the physical theory itself (as they have for scientific
theories since time immemorial). Put another way, despite a
whole-hearted commitment to an It-from-Bit ontology, we are firmly
aligned with the shut-up-and-calculate camp as the best way to obtain
results either from the physical perspective or as a quality assurance
measure of our fledgling theory of dynamics.

In detail, we present a reflective process calculus. Then we develop
intuitive correspondences between the notions available in this
calculus and the usual physical notions supporting quantum mechanical
calculations. Thus, 

\begin{table}[htp]
  \center{
    \fbox{
      \begin{tabular}{c|c}
        quantum mechanics & process calculus \\
        \hline
        scalar & name \\
        state vector & process \\
        dual & contextual duals \\
        matrix & formal sums of process-context-dual pairs \\
        orthogonality & process annihilation \\
        inner product & execution-formula + quoting
      \end{tabular}
    }
  }
  \caption{QM - process calculi correspondences}
\end{table}

Then we tighten up these intuitions to operational definitions. We
employ the Dirac notation as the best proxy we can find for an
abstract syntax of the quantum mechanical notions. The definitions we
develop put us in contact with equational constraints coming from the
theory that we demonstrate the definitions and calculations satisfy.

This puts us in a position to shut up and calculate for the
Stern-Gerlach experimental set up, showing how these predictive
calculations become calculations on processes in our theory of a
reflective process calculus.

Penultimately, we demonstrate that the notion of metric coming from
the inner product coincides with the notion of metric available from
the theory of bisimulation. This demonstration gives us the right to
think of space as arising from behavior. Finally, we consider where we
might go from the new vantage point we have obtained.

% section introduction (end) 
 
% section introduction (end)

% \documentclass[12pt]{llncs}
%\documentclass{jktr}

\usepackage[pdftex]{hyperref}                   
\usepackage {listings}
\usepackage {mathpartir}
\usepackage{bcprules}
%\usepackage{listings}
                       
\usepackage{graphicx} 
%\usepackage[margins=2.5cm,nohead,nofoot]{geometry}
%\usepackage{geometry}
\usepackage{amsfonts}
\usepackage{amstext}
\usepackage{latexsym}
\usepackage{amssymb}
\usepackage{color}


%\include{myPreamble}
\include{qm2pi.local} 

%\ifpdf
%\usepackage[pdftex]{graphicx}
%\else
%\usepackage{graphicx}
%\fi

 % \ifpdf
%  \usepackage{pdfsync}
%  \if


%\title{Brief Article}
%\author{David F. Snyder}
%\author{L.G. Meredith}

%\address{Dept. of Math., Texas State University--San Marcos, San Marcos, TX 78666}
       
\pagestyle{empty}


\begin{document}

\lstset{language=[Objective]Caml,frame=shadowbox}

\input{qm2pi.front}

% section front matter (end)

\input{qm2pi.intro} 
 
% section introduction (end)

% \input{qm2pi.knotations} 

% section notation (end)

\input{qm2pi.process.calculi} 

% section concurrent_process_calculi_and_spatial_logics_ (end)
    
%\input{qm2pi.knots2pi} 

%\input{qm2pi.trefoil} 

%\input{qm2pi.mainthm} 

% subsection basic_interpretation (end)

%\input{qm2pi.rho.presentation} 
\subsection{The syntax and semantics of the notation system}\label{sub:the_syntax_and_semantics_of_the_notation_system} % (fold)

We now summarize a technical presentation of the calculus that
embodies our theory of dynamics. The typical presentation of such a
calculus follows the style of giving generators and relations on
them. The grammar, below, describing term constructors, freely
generates the set of processes, $\Proc$. This set is then quotiented
by a relation known as structural congruence and it is over this set
that the notion of dynamics is expressed. This presentation is
essentially that of \cite{MeredithR05} with the addition of
polyadicity and summation. For readability we have relegated some of
the technical subtleties to an appendix.

\subsubsection{Process grammar}\label{subsub:process_grammar}

\begin{mathpar}
  \inferrule* [lab=synchronization] {} {{M} \bc \pzero \;|\; x?F \;|\; x!C }
  \and
  \inferrule* [lab=abstraction] {} {{F} \bc (x)P}
  \and
  \inferrule* [lab=concretion] {} {{C} \bc \langle Q \rangle}
  \and
  \inferrule* [lab=process] {} {{P,Q} \bc M \;| \;P|Q \;|\; @{x}}
  \and
  \inferrule* [lab=name] {} {{x} \bc \quotep{P}}
\end{mathpar} 

Note that $\vec{x}$ (resp. $\vec{P}$) denotes a vector of names
(resp. processes) of length $|\vec{x}|$ (resp. $|\vec{P}|$). We adopt
the following useful abbreviations.

\begin{mathpar}
   x?(\vec{y}).P := x.(\vec{y})P \and  x\clift{\vec{P}} := x.\clift{\vec{P}}
   \and x!(y) := \lift{x}{\dropn{y}}
   \and \Pi_{i=0}^{n-1}P_i := P_0 | \ldots | P_{n-1}
\end{mathpar}

\subsubsection{Structural congruence}

\paragraph{Free and bound names and alpha-equivalence.} At the
core of structural equivalence is alpha-equivalence which identifies
process that are the same up to a change of variable. Formally, we
recognize the distinction between free and bound names. The free names
of a process, $\freenames{P}$, may be calculated recursively as
follows:

\begin{mathpar}
\freenames{\pzero} := \emptyset
  \and \\
  \freenames{x?(y).P} := \{ x \} \cup (\freenames{P} \setminus \{ y \})
  \and 
  \freenames{x!\langle P \rangle} := \{ x \} \cup \{ P \} 
  \and \\
  \freenames{P|Q} := \freenames{P} \cup \freenames{Q}
  \and \\
  \freenames{@{x}} := \{ x \}
\end{mathpar}

$\pi$
$\quotep{\pi}$

$\freenames{-} : \pi \to \mathcal{P}(\quotep{\pi})$

\begin{eqnarray*}
  \freenames{\pzero} & := & \emptyset \\
  \freenames{x?(y).P} & := & \{ x \} \cup (\freenames{P} \setminus \{ y \}) \\
  \freenames{x!\langle P \rangle} & := & \{ x \} \cup \{ P \} \\
  \freenames{P|Q} & := & \freenames{P} \cup \freenames{Q} \\
  \freenames{\dropn{x}} & := & \{ x \}
\end{eqnarray*}

The bound names of a process, $\boundnames{P}$, are those names occurring in $P$
that are not free. For example, in $x?(y).0$, the name $x$ is free, while $y$ is bound.

\begin{mathpar}
  \inferrule* [lab=monoidal-laws] {} { P|Q \equiv Q|P \and P|0 \equiv P \and P|(Q|R) \equiv (P|Q)|R }
\end{mathpar}

\begin{mathpar}
  \inferrule* [lab=alpha-equivalence] {} { (x)P \equiv (y)P\{y/x\} \and y \not\in \freenames{P} }
\end{mathpar}

\begin{definition}
Then two processes, $P,Q$, are alpha-equivalent if $P = Q\{\vec{y}/\vec{x}\}$ for
some $\vec{x} \in \boundnames{Q},\vec{y} \in \boundnames{P}$, where $Q\{\vec{y}/\vec{x}\}$
denotes the capture-avoiding substitution of $\vec{y}$ for $\vec{x}$ in $Q$.
\end{definition}

\begin{definition}
  The {\em structural congruence} \cite{SangiorgiWalker} , $\equiv$,
  between processes is the least congruence containing
  alpha-equivalence, satisfying the abelian monoid laws
  (associativity, commutativity and $\pzero$ as identity) for parallel
  composition $|$ and for summation $+$.
\end{definition}

\subsection{Name equivalence}

We take name equivalence, written $\nameeq$, to be the smallest
equivalence relation generated by the following rules.

\begin{mathpar}
\inferrule*[lab=Quote-drop]
{ }
{ \quotep{@{x}} \nameeq x }

\inferrule*[lab=Struct-equiv]
{ P \scong Q }
{ \quotep{P} \nameeq \quotep{Q} }
\end{mathpar}

The astute reader will have noticed that the mutual recursion of names
and processes imposes a mutual recursion on alpha-equivalence and
structural equivalence via name-equivalence. Fortunately, all of this
works out pleasantly and we may calculate in the natural way, free of
concern. The reader interested in the details is referred to the
appendix \ref{appendix:rho_details}.

\subsection{Substitution}

We use $\Proc$ for the set of processes, $\QProc$ for the set of
names, and $\id{\{}\vec{y} / \vec{x} \id{\}}$ to denote partial maps,
$s : \QProc \rightarrow \QProc$. A map, $s$ lifts, uniquely, to a map
on process terms, $\widehat{s} : \Proc \rightarrow \Proc$ by the
following equations.

\begin{mathpar}
  (0) \psubstp{Q}{P} := 0 \\
  (R \juxtap S) \psubstp{Q}{P}
  :=    
  (R)\psubstp{Q}{P} \juxtap (S) \psubstp{Q}{P} \\
  (x?(y).R) \psubstp{Q}{P}    
  :=    
  (x)\substp{Q}{P} (z)\concat( (R \psubstn{z}{y}) \psubstp{Q}{P} ) \\
  (\lift{x}{R}) \psubstp{Q}{P}  
  :=
  \lift{(x)\substp{Q}{P}}{ R \psubstp{Q}{P} } \\
%   (\dropn{x})  \psubstp{Q}{P}       
%   := 
%   \left\{ 
%     \begin{array}{ccc} 
%       \dropn{\quotep{Q}} & & x \nameeq \quotep{P} \\
%       \dropn{x} & & otherwise \\
%     \end{array}
%   \right. 
  (\dropn{x})  \psubstp{Q}{P}       
  := 
  \left\{ 
    \begin{array}{ccc} 
      Q & & x \nameeq \quotep{P} \\
      \dropn{x} & & otherwise \\
    \end{array}
  \right.
\end{mathpar}
 

where

\begin{eqnarray}
  (x)\id{\{} \lpquote Q \rpquote / \lpquote P \rpquote \id{\}}            = 
  \left\{ 
    \begin{array}{ccc}
      \lpquote Q \rpquote & & x \nameeq \lpquote P \rpquote \\
      x & & otherwise \\
    \end{array}
  \right. \nonumber
\end{eqnarray}

and $z$ is chosen distinct from $\quotep{P}$, $\quotep{Q}$, the free
names in $Q$, and all the names in $R$. Our $\alpha$-equivalence will
be built in the standard way from this substitution.

\begin{remark}\label{rem:no_self_referential_names}
  One consequence of these definitions is that $\forall P. \quotep{P}
  \not\in \freenames{P}$.
\end{remark}

\subsection{ Dynamic quote: an example }

Anticipating something of what's to come, consider applying the
substitution, $\widehat{\id{\{}u / z \id{\}}}$, to the following pair
of processes, $\lift{w}{y!(z)}$ and $w[ \lpquote y!(z) \rpquote ]$.

\begin{eqnarray}
	\lift{w}{y!(z)}\widehat{\id{\{}u / z \id{\}}}
		& = &
		\lift{w}{y!(u)} \nonumber\\
	w[ \lpquote y!(z) \rpquote ] \widehat{ \id{\{}u / z \id{\}} }
		& = &
		w[ \lpquote y!(z) \rpquote ] \nonumber
\end{eqnarray}

Because the body of the process between quotes is impervious to
substitution, we get radically different answers. In fact, by
examining the first process in an input context,
e.g. $x?(z).\lift{w}{y!(z)}$, we see that the process under the lift
operator may be shaped by prefixed inputs binding a name inside it. In
this sense, the lift operator will be seen as a way to dynamically
construct processes before reifying them as names.

Finally equipped with these standard features we can present the
dynamics of the calculus.

\subsubsection{Operational semantics} 

Finally, we introduce the computational dynamics. What marks these
algebras as distinct from other more traditionally studied algebraic
structures, e.g. vector spaces or polynomial rings, is the manner in
which dynamics is captured. In traditional structures, dynamics is typically
expressed through morphisms between such structures, as in linear maps
between vector spaces or morphisms between rings. In algebras
associated with the semantics of computation, the dynamics is
expressed as part of the algebraic structure itself, through a
reduction reduction relation typically denoted by $\red$. Below, we
give a recursive presentation of this relation for the calculus used
in the encoding.

$\red \subseteq \pi \times \pi$
$\red : \pi \to \mathcal{P}(\pi)$

\begin{mathpar}
  \inferrule* [lab=Comm] { \textsf{match}( x_{src}, x_{trgt} ) } { x_{trgt}?(y)P \; | \; x_{src}!\langle {Q} \rangle \red P\{\quotep{Q}/y}\} }
  \and \\
  \inferrule* [lab=Par] {{P} \red {P}'} {{{P} | {Q}} \red {{P}' | {Q}}}
  \and
  \inferrule* [lab=Equiv]{{{P} \scong {P}'} \andalso {{P}' \red {Q}'} \andalso {{Q}' \scong {Q}}}{{P} \red {Q}}
\end{mathpar}

\begin{eqnarray*}
  match_{\equiv} (\quotep{P},\quotep{Q}) & := & P \equiv Q \\
  match_{\dagger}(\quotep{P},\quotep{Q}) & := & \forall R. P|Q \red^{*} R => R \red^{*} 0 \\
  match_{K}(\quotep{P},\quotep{Q}) & := & K \mbox{ for some context } K
\end{eqnarray*}

$u?(x)P | u!\langle Q \rangle \red P\{\quotep{Q}/x\}$

%We write $\wred$ for $\red^*$, and $P\red$ if $\exists Q $ such that $ P \red Q$.
We write $P\red$ if $\exists Q $ such that $ P \red Q$ and $P\not\red$, otherwise.

\section{Replication}

As mentioned before, it is known that replication (and hence
recursion) can be implemented in a higher-order process algebra
\cite{SangiorgiWalker}. As our first example of calculation with the
machinery thus far presented we give the construction explicitly in
the {\rhoc}.

\begin{eqnarray}
	D_{x} & := & \prefix{x}{y}{(\binpar{\outputp{x}{y}}{@{y}})} \nonumber\\
	\bangp_{x}{P} & := & \binpar{{x}!\langle{\binpar{D_{x}}{P}}\rangle}{D_{x}} \nonumber
\end{eqnarray}

\begin{eqnarray}
	\bangp_{x}{P} & & \nonumber\\
	=
	& {x}!\langle{(\prefix{x}{y}{(\outputp{x}{y} | @{y})) | P}}\rangle 
	      | \prefix{x}{y}{(\outputp{x}{y} | @{y})} & \nonumber\\
	\red
	& (\outputp{x}{y} | @{y})\substn{\quotep{(\prefix{x}{y}{(@{y} | \outputp{x}{y})) | P}}}{y} & \nonumber\\
	=
	& \outputp{x}{\quotep{(\prefix{x}{y}{(\outputp{x}{y} | @{y})) | P}}}
	  | {(\prefix{x}{y}{(\outputp{x}{y} | @{y})) | P}} & \nonumber\\
	\red
	& \ldots & \nonumber\\
	\red^*
	& P | P | \ldots & \nonumber
\end{eqnarray}

Of course, this encoding, as an implementation, runs away, unfolding
$\bangp{P}$ eagerly. A lazier and more implementable replication
operator, restricted to input-guarded processes, may be obtained as follows.

\begin{eqnarray}
\bangp{\prefix{u}{v}{P}} 
	:= 
	\binpar{\lift{x}{\prefix{u}{v}{(\binpar{D(x)}{P})}}}{D(x)} \nonumber
\end{eqnarray}

\begin{remark}
  Note that the lazier definition still does not deal with summation
  or mixed summation (i.e. sums over input and output). The reader is
  invited to construct definitions of replication that deal with these
  features. 

  Further, the definitions are parameterized in a name, $x$. Can you,
  gentle reader, make a definition that eliminates this parameter and
  guarantees no accidental interaction between the replication
  machinery and the process being replicated -- i.e. no accidental
  sharing of names used by the process to get its work done and the
  name(s) used by the replication to effect copying. This latter
  revision of the definition of replication is crucial to obtaining
  the expected identity $!!P \sim !P$.
\end{remark}

\begin{remark}\label{rem:paradoxical_combinator}
  The reader familiar with the lambda calculus will have noticed the
  similarity between $D$ and the paradoxical combinator.

  [Ed. note: the existence of this seems to suggest we have to be more
  restrictive on the set of processes and names we admit if we are to
  support no-cloning.]
\end{remark}

\subsubsection{Bisimulation}

The computational dynamics gives rise to another kind of equivalence,
the equivalence of computational behavior. As previously mentioned
this is typically captured \emph{via} some form of bisimulation.

% The notion we use in this paper is weak barbed bisimulation
% \cite{milner91polyadicpi}.

The notion we use in this paper is derived from weak barbed
bisimulation \cite{milner91polyadicpi}. 

\begin{definition}
An \emph{observation relation}, $\downarrow_{\mathcal N}$, over a set
of names, $\mathcal N$, is the smallest relation satisfying the rules
below.

\infrule[Out-barb]{y \in {\mathcal N}, \; x \nameeq y}
		  {\outputp{x}{v} \downarrow_{\mathcal N} x}
\infrule[Par-barb]{\mbox{$P\downarrow_{\mathcal N} x$ or $Q\downarrow_{\mathcal N} x$}}
		  {\binpar{P}{Q} \downarrow_{\mathcal N} x}

We write $P \Downarrow_{\mathcal N} x$ if there is $Q$ such that 
$P \wred Q$ and $Q \downarrow_{\mathcal N} x$.
\end{definition}

\begin{definition}
%\label{def.bbisim}
An  ${\mathcal N}$-\emph{barbed bisimulation} over a set of names, ${\mathcal N}$, is a symmetric binary relation 
${\mathcal S}_{\mathcal N}$ between agents such that $P\rel{S}_{\mathcal N}Q$ implies:
\begin{enumerate}
\item If $P \red P'$ then $Q \wred Q'$ and $P'\rel{S}_{\mathcal N} Q'$.
\item If $P\downarrow_{\mathcal N} x$, then $Q\Downarrow_{\mathcal N} x$.
\end{enumerate}
$P$ is ${\mathcal N}$-barbed bisimilar to $Q$, written
$P \wbbisim_{\mathcal N} Q$, if $P \rel{S}_{\mathcal N} Q$ for some ${\mathcal N}$-barbed bisimulation ${\mathcal S}_{\mathcal N}$.
\end{definition}

$\mathcal{R} \subseteq \pi \times \pi$

$P \mathcal{R} Q => \forall P'. P \red P' \Rightarrow \exists Q'. Q \red Q', P' \mathcal{R} Q'$

$P \vdash x \Rightarrow Q \vdash x$

\begin{mathpar}
  \inferrule*[lab=Out-barb]{x \nameeq y}{{y}!\langle{Q}\rangle \vdash x}
  \and
  \inferrule*[lab=Par-barb]{\mbox{$P\vdash x$ or $Q\vdash x$}}{\binpar{P}{Q} \vdash x}
\end{mathpar}

\subsubsection{Contexts}

One of the principle advantages of computational calculi like the
$\pi$-calculus is a well-defined notion of context,
contextual-equivalence and a correlation between
contextual-equivalence and notions of bisimulation. The notion of
context allows the decomposition of a process into (sub-)process and
its syntactic environment, its context. Thus, a context may be
thought of as a process with a ``hole'' (written $\Box$) in it. The
application of a context $M$ to a process $P$, written $M[P]$, is
tantamount to filling the hole in $M$ with $P$. In this paper we do
not need the full weight of this theory, but do make use of the notion
of context in the proof the main theorem. 

\begin{mathpar}
  \inferrule* [lab=summation] {} {{M_{M},M_{N}} \bc \Box \;|\; x.M_{A} \;|\; M_{M}+M_{N}}
  \and
  \inferrule* [lab=agent] {} {{M_{A}} \bc (\vec{x})M_{P} \;| \; \clift{P_0,\ldots,M_{P},\ldots,P_N}}
  \and \\
  \inferrule* [lab=process] {} {{M_{P}} \bc M_{N} \;| \;P|M_{P} }
\end{mathpar} 

\begin{mathpar}
  \inferrule* [lab=sychronization] {} {M_{N} \bc \Box \;|\; x?M_{F} \;|\; x!M_{C}}
  \and
  \inferrule* [lab=abstraction] {} {{M_{F}} \bc (x)M_{P} }
  \and
  \inferrule* [lab=concretion] {} {{M_{C}} \bc \langle M_{P} \rangle }
  \and \\
  \inferrule* [lab=process] {} {{M_{P}} \bc M_{N} \;| \;P|M_{P} }
\end{mathpar}

\begin{definition}[contextual application] Given a context $M$, and
  process $P$, we define the \emph{contextual application}, $M[P] :=
  M\{P/\Box\}$. That is, the contextual application of M to P is the
  substitution of $P$ for $\Box$ in $M$.
\end{definition}

$\meaningof{-} : L \to \mathcal{P}(\pi)$

\begin{mathpar}
  \inferrule* [lab=collection] {} {\meaningof{true} = \pi, \and \meaningof{~E} = \pi \setminus \meaningof{E}, \and \meaningof{E_{1} \& E_{2}} = \meaningof{E_{1}} \cap \meaningof{E_{2}}}
\end{mathpar}

\begin{mathpar}
  \inferrule* [lab=structure] {} {\meaningof{0} = \{ P \in \pi | P \equiv 0 \}, \and \\ \meaningof{E_1 | E_2} = \{ P \in \pi | P \equiv P_{1} | P_{2}, P_{1} \in \meaningof{E_{1}}, P_{2} \in \meaningof{E_2}\} }
\end{mathpar}

\begin{mathpar}
 \inferrule* [lab=behavior] {} {\meaningof{\langle a?b \rangle E} = \{ P \in \pi | P \equiv Q | u?(y)P', \\ \and \\\\ \and \\ \;\;\; u \in \meaningof{a}, \forall z.P'\{z/y\} \in \meaningof{E\{z/b\}}\}, \and \\ \meaningof{a!E} = \{ P \in \pi | P \equiv Q | x!\langle P' \rangle, x \in \meaningof{a} P' \in \meaningof{E}\} }
\end{mathpar}

\begin{mathpar}
 \inferrule* [lab=nominal] {} {\meaningof{\quotep{E}} = \{ \quotep{P} \in \quotep{\pi} | P \in \meaningof{E} \}, \and \meaningof{\quotep{P}} = \{ \quotep{Q} \in \quotep{\pi} | P \equiv Q \} \and \\ \meaningof{@\quotep{E}} = \{ P \in \pi | P \equiv @x, x \in \meaningof{E} \}}
\end{mathpar}

\begin{eqnarray*}
  \\
  \meaningof{-} : TS \to ST
\end{eqnarray*}

\begin{eqnarray*}
  \\
  L : TS \to ST
\end{eqnarray*}

\begin{eqnarray*}
  \\
  P \models E \iff P \in \meaningof{E}
\end{eqnarray*}

\begin{eqnarray*}
  P \approx_{L} Q \iff \forall E \in L. P \models E \iff Q \models E
\end{eqnarray*}

\begin{eqnarray*}
  P \approx_{K} Q
\end{eqnarray*}

\begin{eqnarray*}
  P \approx Q
\end{eqnarray*}

$\approx_{K} = \approx = \approx_{L}$

\subsubsection{Contextual duality}

Note that contexts extend the quotation operation to a family of
operations from processes to names. Given a context, $M$, we can
define a \emph{nominal context}, $\quotep{M}$ by $\quotep{M}[P] :=
\quotep{M[P]}$. To foreshadow what is to come we observe that these
operations enjoy a duality with processes very much like the duality
between vectors and maps from vectors to scalars.

Further, because the calculus is essentially higher-order, we have a
correspondence between contexts and processes. More specifically,
given a name $x$ and a context $M$ we can construct $M^{*}_{x}$ such
that 

\begin{mathpar}
  M^{*}_{x} | \lift{x}{P} \red M[P]
\end{mathpar}

namely,

\begin{mathpar}
  M^{*}_{x} := x?(u).M[\dropn{u}]
\end{mathpar}

The dependence of $M^{*}_{x}$ on a name makes it an abstraction, 

\begin{mathpar}
  M^{*} := (x)x?(u).M[\dropn{u}]
\end{mathpar}

\subsection{Additional notation}

It will sometimes be convenient to denote the process a name
quotes. We already have the notation $x = \quotep{P}$, but it will be
convenient to introduce an alternate notation, $\procn{x}$, when we
want to emphasize the connection to the use of the name. Note that, by
virtue of name equivalence, $\quotep{\procn{x}} \nameeq x$; so, the
notation is consistent with previous definitions.

Further, because names have structure it is possible to effect
substitutions on the basis of that structure. This means we need to
upgrade our notation for substitutions, which we accomplish by
adapting comprehension notation. Thus,

\begin{mathpar}
  P\{ y / x : x \in S \}
\end{mathpar}

is interpreted to mean the process derived from P by replacing (in a
capture-avoiding manner) each occurrence of $x$ in $S$ by $y$. For example,

\begin{mathpar}
  P\{ \quotep{\procn{x}|\procn{x}} / x : x \in \freenames{P} \}
\end{mathpar}

will replace each (occurrence) of a free name $x$ in $P$ by
$\quotep{\procn{x}|\procn{x}}$.

Also, we will avail ourselves of the notation $x^{L}$ and $x^{R}$ to
denote injections of a name into disjoint copies of the name
space. There are numerous ways to accomplish this. One example can be
found in \cite{MeredithR05}. This notation overloads to vectors of
names: $\vec{x}^{\pi} := (x_{i}^{\pi} \; : \; 0 \leq i < |\vec{x}| )$ where $\pi \in \{L,R\}$.

We also use $P^{\Box} := P|\Box$.

In \cite{MeredithR05} an interpretation of the new operator is
given. It turns out that there are several possible interpretations
all enjoying the requisite algebraic properties of the operator (see
\cite{milner91polyadicpi}). We will therefore make liberal use of
$(\nu\; \vec{x})P$.

% subsection the_syntax_and_semantics_of_the_notation_system (end)   

\input{qm2pi.qmops} 

\input{qm2pi.sterngerlach} 

\input{qm2pi.metric} 

% section concurrent_process_calculi (end)

%\input{qm2pi.proofsketch}

% section proof sketch (end)

%\input{qm2pi.slviaknots} 

% section spatial logic via knots (end)

\input{qm2pi.conclusion}

% section conclusion (end)

%\input{qm2pi.dtcodes} 

% section wiring algorithm (end)

\input{qm2pi.ack} 

% section acknowledgments (end)

\newpage


\bibliographystyle{plain}   
\bibliography{../../biblios/main.bib}

\input{qm2pi.rhodetails}

\end{document}

 

% section notation (end)

\input{qm2pi.process.calculi} 

% section concurrent_process_calculi_and_spatial_logics_ (end)
    
%\documentclass[12pt]{llncs}
%\documentclass{jktr}

\usepackage[pdftex]{hyperref}                   
\usepackage {listings}
\usepackage {mathpartir}
\usepackage{bcprules}
%\usepackage{listings}
                       
\usepackage{graphicx} 
%\usepackage[margins=2.5cm,nohead,nofoot]{geometry}
%\usepackage{geometry}
\usepackage{amsfonts}
\usepackage{amstext}
\usepackage{latexsym}
\usepackage{amssymb}
\usepackage{color}


%\include{myPreamble}
\include{qm2pi.local} 

%\ifpdf
%\usepackage[pdftex]{graphicx}
%\else
%\usepackage{graphicx}
%\fi

 % \ifpdf
%  \usepackage{pdfsync}
%  \if


%\title{Brief Article}
%\author{David F. Snyder}
%\author{L.G. Meredith}

%\address{Dept. of Math., Texas State University--San Marcos, San Marcos, TX 78666}
       
\pagestyle{empty}


\begin{document}

\lstset{language=[Objective]Caml,frame=shadowbox}

\input{qm2pi.front}

% section front matter (end)

\input{qm2pi.intro} 
 
% section introduction (end)

% \input{qm2pi.knotations} 

% section notation (end)

\input{qm2pi.process.calculi} 

% section concurrent_process_calculi_and_spatial_logics_ (end)
    
%\input{qm2pi.knots2pi} 

%\input{qm2pi.trefoil} 

%\input{qm2pi.mainthm} 

% subsection basic_interpretation (end)

%\input{qm2pi.rho.presentation} 
\subsection{The syntax and semantics of the notation system}\label{sub:the_syntax_and_semantics_of_the_notation_system} % (fold)

We now summarize a technical presentation of the calculus that
embodies our theory of dynamics. The typical presentation of such a
calculus follows the style of giving generators and relations on
them. The grammar, below, describing term constructors, freely
generates the set of processes, $\Proc$. This set is then quotiented
by a relation known as structural congruence and it is over this set
that the notion of dynamics is expressed. This presentation is
essentially that of \cite{MeredithR05} with the addition of
polyadicity and summation. For readability we have relegated some of
the technical subtleties to an appendix.

\subsubsection{Process grammar}\label{subsub:process_grammar}

\begin{mathpar}
  \inferrule* [lab=synchronization] {} {{M} \bc \pzero \;|\; x?F \;|\; x!C }
  \and
  \inferrule* [lab=abstraction] {} {{F} \bc (x)P}
  \and
  \inferrule* [lab=concretion] {} {{C} \bc \langle Q \rangle}
  \and
  \inferrule* [lab=process] {} {{P,Q} \bc M \;| \;P|Q \;|\; @{x}}
  \and
  \inferrule* [lab=name] {} {{x} \bc \quotep{P}}
\end{mathpar} 

Note that $\vec{x}$ (resp. $\vec{P}$) denotes a vector of names
(resp. processes) of length $|\vec{x}|$ (resp. $|\vec{P}|$). We adopt
the following useful abbreviations.

\begin{mathpar}
   x?(\vec{y}).P := x.(\vec{y})P \and  x\clift{\vec{P}} := x.\clift{\vec{P}}
   \and x!(y) := \lift{x}{\dropn{y}}
   \and \Pi_{i=0}^{n-1}P_i := P_0 | \ldots | P_{n-1}
\end{mathpar}

\subsubsection{Structural congruence}

\paragraph{Free and bound names and alpha-equivalence.} At the
core of structural equivalence is alpha-equivalence which identifies
process that are the same up to a change of variable. Formally, we
recognize the distinction between free and bound names. The free names
of a process, $\freenames{P}$, may be calculated recursively as
follows:

\begin{mathpar}
\freenames{\pzero} := \emptyset
  \and \\
  \freenames{x?(y).P} := \{ x \} \cup (\freenames{P} \setminus \{ y \})
  \and 
  \freenames{x!\langle P \rangle} := \{ x \} \cup \{ P \} 
  \and \\
  \freenames{P|Q} := \freenames{P} \cup \freenames{Q}
  \and \\
  \freenames{@{x}} := \{ x \}
\end{mathpar}

$\pi$
$\quotep{\pi}$

$\freenames{-} : \pi \to \mathcal{P}(\quotep{\pi})$

\begin{eqnarray*}
  \freenames{\pzero} & := & \emptyset \\
  \freenames{x?(y).P} & := & \{ x \} \cup (\freenames{P} \setminus \{ y \}) \\
  \freenames{x!\langle P \rangle} & := & \{ x \} \cup \{ P \} \\
  \freenames{P|Q} & := & \freenames{P} \cup \freenames{Q} \\
  \freenames{\dropn{x}} & := & \{ x \}
\end{eqnarray*}

The bound names of a process, $\boundnames{P}$, are those names occurring in $P$
that are not free. For example, in $x?(y).0$, the name $x$ is free, while $y$ is bound.

\begin{mathpar}
  \inferrule* [lab=monoidal-laws] {} { P|Q \equiv Q|P \and P|0 \equiv P \and P|(Q|R) \equiv (P|Q)|R }
\end{mathpar}

\begin{mathpar}
  \inferrule* [lab=alpha-equivalence] {} { (x)P \equiv (y)P\{y/x\} \and y \not\in \freenames{P} }
\end{mathpar}

\begin{definition}
Then two processes, $P,Q$, are alpha-equivalent if $P = Q\{\vec{y}/\vec{x}\}$ for
some $\vec{x} \in \boundnames{Q},\vec{y} \in \boundnames{P}$, where $Q\{\vec{y}/\vec{x}\}$
denotes the capture-avoiding substitution of $\vec{y}$ for $\vec{x}$ in $Q$.
\end{definition}

\begin{definition}
  The {\em structural congruence} \cite{SangiorgiWalker} , $\equiv$,
  between processes is the least congruence containing
  alpha-equivalence, satisfying the abelian monoid laws
  (associativity, commutativity and $\pzero$ as identity) for parallel
  composition $|$ and for summation $+$.
\end{definition}

\subsection{Name equivalence}

We take name equivalence, written $\nameeq$, to be the smallest
equivalence relation generated by the following rules.

\begin{mathpar}
\inferrule*[lab=Quote-drop]
{ }
{ \quotep{@{x}} \nameeq x }

\inferrule*[lab=Struct-equiv]
{ P \scong Q }
{ \quotep{P} \nameeq \quotep{Q} }
\end{mathpar}

The astute reader will have noticed that the mutual recursion of names
and processes imposes a mutual recursion on alpha-equivalence and
structural equivalence via name-equivalence. Fortunately, all of this
works out pleasantly and we may calculate in the natural way, free of
concern. The reader interested in the details is referred to the
appendix \ref{appendix:rho_details}.

\subsection{Substitution}

We use $\Proc$ for the set of processes, $\QProc$ for the set of
names, and $\id{\{}\vec{y} / \vec{x} \id{\}}$ to denote partial maps,
$s : \QProc \rightarrow \QProc$. A map, $s$ lifts, uniquely, to a map
on process terms, $\widehat{s} : \Proc \rightarrow \Proc$ by the
following equations.

\begin{mathpar}
  (0) \psubstp{Q}{P} := 0 \\
  (R \juxtap S) \psubstp{Q}{P}
  :=    
  (R)\psubstp{Q}{P} \juxtap (S) \psubstp{Q}{P} \\
  (x?(y).R) \psubstp{Q}{P}    
  :=    
  (x)\substp{Q}{P} (z)\concat( (R \psubstn{z}{y}) \psubstp{Q}{P} ) \\
  (\lift{x}{R}) \psubstp{Q}{P}  
  :=
  \lift{(x)\substp{Q}{P}}{ R \psubstp{Q}{P} } \\
%   (\dropn{x})  \psubstp{Q}{P}       
%   := 
%   \left\{ 
%     \begin{array}{ccc} 
%       \dropn{\quotep{Q}} & & x \nameeq \quotep{P} \\
%       \dropn{x} & & otherwise \\
%     \end{array}
%   \right. 
  (\dropn{x})  \psubstp{Q}{P}       
  := 
  \left\{ 
    \begin{array}{ccc} 
      Q & & x \nameeq \quotep{P} \\
      \dropn{x} & & otherwise \\
    \end{array}
  \right.
\end{mathpar}
 

where

\begin{eqnarray}
  (x)\id{\{} \lpquote Q \rpquote / \lpquote P \rpquote \id{\}}            = 
  \left\{ 
    \begin{array}{ccc}
      \lpquote Q \rpquote & & x \nameeq \lpquote P \rpquote \\
      x & & otherwise \\
    \end{array}
  \right. \nonumber
\end{eqnarray}

and $z$ is chosen distinct from $\quotep{P}$, $\quotep{Q}$, the free
names in $Q$, and all the names in $R$. Our $\alpha$-equivalence will
be built in the standard way from this substitution.

\begin{remark}\label{rem:no_self_referential_names}
  One consequence of these definitions is that $\forall P. \quotep{P}
  \not\in \freenames{P}$.
\end{remark}

\subsection{ Dynamic quote: an example }

Anticipating something of what's to come, consider applying the
substitution, $\widehat{\id{\{}u / z \id{\}}}$, to the following pair
of processes, $\lift{w}{y!(z)}$ and $w[ \lpquote y!(z) \rpquote ]$.

\begin{eqnarray}
	\lift{w}{y!(z)}\widehat{\id{\{}u / z \id{\}}}
		& = &
		\lift{w}{y!(u)} \nonumber\\
	w[ \lpquote y!(z) \rpquote ] \widehat{ \id{\{}u / z \id{\}} }
		& = &
		w[ \lpquote y!(z) \rpquote ] \nonumber
\end{eqnarray}

Because the body of the process between quotes is impervious to
substitution, we get radically different answers. In fact, by
examining the first process in an input context,
e.g. $x?(z).\lift{w}{y!(z)}$, we see that the process under the lift
operator may be shaped by prefixed inputs binding a name inside it. In
this sense, the lift operator will be seen as a way to dynamically
construct processes before reifying them as names.

Finally equipped with these standard features we can present the
dynamics of the calculus.

\subsubsection{Operational semantics} 

Finally, we introduce the computational dynamics. What marks these
algebras as distinct from other more traditionally studied algebraic
structures, e.g. vector spaces or polynomial rings, is the manner in
which dynamics is captured. In traditional structures, dynamics is typically
expressed through morphisms between such structures, as in linear maps
between vector spaces or morphisms between rings. In algebras
associated with the semantics of computation, the dynamics is
expressed as part of the algebraic structure itself, through a
reduction reduction relation typically denoted by $\red$. Below, we
give a recursive presentation of this relation for the calculus used
in the encoding.

$\red \subseteq \pi \times \pi$
$\red : \pi \to \mathcal{P}(\pi)$

\begin{mathpar}
  \inferrule* [lab=Comm] { \textsf{match}( x_{src}, x_{trgt} ) } { x_{trgt}?(y)P \; | \; x_{src}!\langle {Q} \rangle \red P\{\quotep{Q}/y}\} }
  \and \\
  \inferrule* [lab=Par] {{P} \red {P}'} {{{P} | {Q}} \red {{P}' | {Q}}}
  \and
  \inferrule* [lab=Equiv]{{{P} \scong {P}'} \andalso {{P}' \red {Q}'} \andalso {{Q}' \scong {Q}}}{{P} \red {Q}}
\end{mathpar}

\begin{eqnarray*}
  match_{\equiv} (\quotep{P},\quotep{Q}) & := & P \equiv Q \\
  match_{\dagger}(\quotep{P},\quotep{Q}) & := & \forall R. P|Q \red^{*} R => R \red^{*} 0 \\
  match_{K}(\quotep{P},\quotep{Q}) & := & K \mbox{ for some context } K
\end{eqnarray*}

$u?(x)P | u!\langle Q \rangle \red P\{\quotep{Q}/x\}$

%We write $\wred$ for $\red^*$, and $P\red$ if $\exists Q $ such that $ P \red Q$.
We write $P\red$ if $\exists Q $ such that $ P \red Q$ and $P\not\red$, otherwise.

\section{Replication}

As mentioned before, it is known that replication (and hence
recursion) can be implemented in a higher-order process algebra
\cite{SangiorgiWalker}. As our first example of calculation with the
machinery thus far presented we give the construction explicitly in
the {\rhoc}.

\begin{eqnarray}
	D_{x} & := & \prefix{x}{y}{(\binpar{\outputp{x}{y}}{@{y}})} \nonumber\\
	\bangp_{x}{P} & := & \binpar{{x}!\langle{\binpar{D_{x}}{P}}\rangle}{D_{x}} \nonumber
\end{eqnarray}

\begin{eqnarray}
	\bangp_{x}{P} & & \nonumber\\
	=
	& {x}!\langle{(\prefix{x}{y}{(\outputp{x}{y} | @{y})) | P}}\rangle 
	      | \prefix{x}{y}{(\outputp{x}{y} | @{y})} & \nonumber\\
	\red
	& (\outputp{x}{y} | @{y})\substn{\quotep{(\prefix{x}{y}{(@{y} | \outputp{x}{y})) | P}}}{y} & \nonumber\\
	=
	& \outputp{x}{\quotep{(\prefix{x}{y}{(\outputp{x}{y} | @{y})) | P}}}
	  | {(\prefix{x}{y}{(\outputp{x}{y} | @{y})) | P}} & \nonumber\\
	\red
	& \ldots & \nonumber\\
	\red^*
	& P | P | \ldots & \nonumber
\end{eqnarray}

Of course, this encoding, as an implementation, runs away, unfolding
$\bangp{P}$ eagerly. A lazier and more implementable replication
operator, restricted to input-guarded processes, may be obtained as follows.

\begin{eqnarray}
\bangp{\prefix{u}{v}{P}} 
	:= 
	\binpar{\lift{x}{\prefix{u}{v}{(\binpar{D(x)}{P})}}}{D(x)} \nonumber
\end{eqnarray}

\begin{remark}
  Note that the lazier definition still does not deal with summation
  or mixed summation (i.e. sums over input and output). The reader is
  invited to construct definitions of replication that deal with these
  features. 

  Further, the definitions are parameterized in a name, $x$. Can you,
  gentle reader, make a definition that eliminates this parameter and
  guarantees no accidental interaction between the replication
  machinery and the process being replicated -- i.e. no accidental
  sharing of names used by the process to get its work done and the
  name(s) used by the replication to effect copying. This latter
  revision of the definition of replication is crucial to obtaining
  the expected identity $!!P \sim !P$.
\end{remark}

\begin{remark}\label{rem:paradoxical_combinator}
  The reader familiar with the lambda calculus will have noticed the
  similarity between $D$ and the paradoxical combinator.

  [Ed. note: the existence of this seems to suggest we have to be more
  restrictive on the set of processes and names we admit if we are to
  support no-cloning.]
\end{remark}

\subsubsection{Bisimulation}

The computational dynamics gives rise to another kind of equivalence,
the equivalence of computational behavior. As previously mentioned
this is typically captured \emph{via} some form of bisimulation.

% The notion we use in this paper is weak barbed bisimulation
% \cite{milner91polyadicpi}.

The notion we use in this paper is derived from weak barbed
bisimulation \cite{milner91polyadicpi}. 

\begin{definition}
An \emph{observation relation}, $\downarrow_{\mathcal N}$, over a set
of names, $\mathcal N$, is the smallest relation satisfying the rules
below.

\infrule[Out-barb]{y \in {\mathcal N}, \; x \nameeq y}
		  {\outputp{x}{v} \downarrow_{\mathcal N} x}
\infrule[Par-barb]{\mbox{$P\downarrow_{\mathcal N} x$ or $Q\downarrow_{\mathcal N} x$}}
		  {\binpar{P}{Q} \downarrow_{\mathcal N} x}

We write $P \Downarrow_{\mathcal N} x$ if there is $Q$ such that 
$P \wred Q$ and $Q \downarrow_{\mathcal N} x$.
\end{definition}

\begin{definition}
%\label{def.bbisim}
An  ${\mathcal N}$-\emph{barbed bisimulation} over a set of names, ${\mathcal N}$, is a symmetric binary relation 
${\mathcal S}_{\mathcal N}$ between agents such that $P\rel{S}_{\mathcal N}Q$ implies:
\begin{enumerate}
\item If $P \red P'$ then $Q \wred Q'$ and $P'\rel{S}_{\mathcal N} Q'$.
\item If $P\downarrow_{\mathcal N} x$, then $Q\Downarrow_{\mathcal N} x$.
\end{enumerate}
$P$ is ${\mathcal N}$-barbed bisimilar to $Q$, written
$P \wbbisim_{\mathcal N} Q$, if $P \rel{S}_{\mathcal N} Q$ for some ${\mathcal N}$-barbed bisimulation ${\mathcal S}_{\mathcal N}$.
\end{definition}

$\mathcal{R} \subseteq \pi \times \pi$

$P \mathcal{R} Q => \forall P'. P \red P' \Rightarrow \exists Q'. Q \red Q', P' \mathcal{R} Q'$

$P \vdash x \Rightarrow Q \vdash x$

\begin{mathpar}
  \inferrule*[lab=Out-barb]{x \nameeq y}{{y}!\langle{Q}\rangle \vdash x}
  \and
  \inferrule*[lab=Par-barb]{\mbox{$P\vdash x$ or $Q\vdash x$}}{\binpar{P}{Q} \vdash x}
\end{mathpar}

\subsubsection{Contexts}

One of the principle advantages of computational calculi like the
$\pi$-calculus is a well-defined notion of context,
contextual-equivalence and a correlation between
contextual-equivalence and notions of bisimulation. The notion of
context allows the decomposition of a process into (sub-)process and
its syntactic environment, its context. Thus, a context may be
thought of as a process with a ``hole'' (written $\Box$) in it. The
application of a context $M$ to a process $P$, written $M[P]$, is
tantamount to filling the hole in $M$ with $P$. In this paper we do
not need the full weight of this theory, but do make use of the notion
of context in the proof the main theorem. 

\begin{mathpar}
  \inferrule* [lab=summation] {} {{M_{M},M_{N}} \bc \Box \;|\; x.M_{A} \;|\; M_{M}+M_{N}}
  \and
  \inferrule* [lab=agent] {} {{M_{A}} \bc (\vec{x})M_{P} \;| \; \clift{P_0,\ldots,M_{P},\ldots,P_N}}
  \and \\
  \inferrule* [lab=process] {} {{M_{P}} \bc M_{N} \;| \;P|M_{P} }
\end{mathpar} 

\begin{mathpar}
  \inferrule* [lab=sychronization] {} {M_{N} \bc \Box \;|\; x?M_{F} \;|\; x!M_{C}}
  \and
  \inferrule* [lab=abstraction] {} {{M_{F}} \bc (x)M_{P} }
  \and
  \inferrule* [lab=concretion] {} {{M_{C}} \bc \langle M_{P} \rangle }
  \and \\
  \inferrule* [lab=process] {} {{M_{P}} \bc M_{N} \;| \;P|M_{P} }
\end{mathpar}

\begin{definition}[contextual application] Given a context $M$, and
  process $P$, we define the \emph{contextual application}, $M[P] :=
  M\{P/\Box\}$. That is, the contextual application of M to P is the
  substitution of $P$ for $\Box$ in $M$.
\end{definition}

$\meaningof{-} : L \to \mathcal{P}(\pi)$

\begin{mathpar}
  \inferrule* [lab=collection] {} {\meaningof{true} = \pi, \and \meaningof{~E} = \pi \setminus \meaningof{E}, \and \meaningof{E_{1} \& E_{2}} = \meaningof{E_{1}} \cap \meaningof{E_{2}}}
\end{mathpar}

\begin{mathpar}
  \inferrule* [lab=structure] {} {\meaningof{0} = \{ P \in \pi | P \equiv 0 \}, \and \\ \meaningof{E_1 | E_2} = \{ P \in \pi | P \equiv P_{1} | P_{2}, P_{1} \in \meaningof{E_{1}}, P_{2} \in \meaningof{E_2}\} }
\end{mathpar}

\begin{mathpar}
 \inferrule* [lab=behavior] {} {\meaningof{\langle a?b \rangle E} = \{ P \in \pi | P \equiv Q | u?(y)P', \\ \and \\\\ \and \\ \;\;\; u \in \meaningof{a}, \forall z.P'\{z/y\} \in \meaningof{E\{z/b\}}\}, \and \\ \meaningof{a!E} = \{ P \in \pi | P \equiv Q | x!\langle P' \rangle, x \in \meaningof{a} P' \in \meaningof{E}\} }
\end{mathpar}

\begin{mathpar}
 \inferrule* [lab=nominal] {} {\meaningof{\quotep{E}} = \{ \quotep{P} \in \quotep{\pi} | P \in \meaningof{E} \}, \and \meaningof{\quotep{P}} = \{ \quotep{Q} \in \quotep{\pi} | P \equiv Q \} \and \\ \meaningof{@\quotep{E}} = \{ P \in \pi | P \equiv @x, x \in \meaningof{E} \}}
\end{mathpar}

\begin{eqnarray*}
  \\
  \meaningof{-} : TS \to ST
\end{eqnarray*}

\begin{eqnarray*}
  \\
  L : TS \to ST
\end{eqnarray*}

\begin{eqnarray*}
  \\
  P \models E \iff P \in \meaningof{E}
\end{eqnarray*}

\begin{eqnarray*}
  P \approx_{L} Q \iff \forall E \in L. P \models E \iff Q \models E
\end{eqnarray*}

\begin{eqnarray*}
  P \approx_{K} Q
\end{eqnarray*}

\begin{eqnarray*}
  P \approx Q
\end{eqnarray*}

$\approx_{K} = \approx = \approx_{L}$

\subsubsection{Contextual duality}

Note that contexts extend the quotation operation to a family of
operations from processes to names. Given a context, $M$, we can
define a \emph{nominal context}, $\quotep{M}$ by $\quotep{M}[P] :=
\quotep{M[P]}$. To foreshadow what is to come we observe that these
operations enjoy a duality with processes very much like the duality
between vectors and maps from vectors to scalars.

Further, because the calculus is essentially higher-order, we have a
correspondence between contexts and processes. More specifically,
given a name $x$ and a context $M$ we can construct $M^{*}_{x}$ such
that 

\begin{mathpar}
  M^{*}_{x} | \lift{x}{P} \red M[P]
\end{mathpar}

namely,

\begin{mathpar}
  M^{*}_{x} := x?(u).M[\dropn{u}]
\end{mathpar}

The dependence of $M^{*}_{x}$ on a name makes it an abstraction, 

\begin{mathpar}
  M^{*} := (x)x?(u).M[\dropn{u}]
\end{mathpar}

\subsection{Additional notation}

It will sometimes be convenient to denote the process a name
quotes. We already have the notation $x = \quotep{P}$, but it will be
convenient to introduce an alternate notation, $\procn{x}$, when we
want to emphasize the connection to the use of the name. Note that, by
virtue of name equivalence, $\quotep{\procn{x}} \nameeq x$; so, the
notation is consistent with previous definitions.

Further, because names have structure it is possible to effect
substitutions on the basis of that structure. This means we need to
upgrade our notation for substitutions, which we accomplish by
adapting comprehension notation. Thus,

\begin{mathpar}
  P\{ y / x : x \in S \}
\end{mathpar}

is interpreted to mean the process derived from P by replacing (in a
capture-avoiding manner) each occurrence of $x$ in $S$ by $y$. For example,

\begin{mathpar}
  P\{ \quotep{\procn{x}|\procn{x}} / x : x \in \freenames{P} \}
\end{mathpar}

will replace each (occurrence) of a free name $x$ in $P$ by
$\quotep{\procn{x}|\procn{x}}$.

Also, we will avail ourselves of the notation $x^{L}$ and $x^{R}$ to
denote injections of a name into disjoint copies of the name
space. There are numerous ways to accomplish this. One example can be
found in \cite{MeredithR05}. This notation overloads to vectors of
names: $\vec{x}^{\pi} := (x_{i}^{\pi} \; : \; 0 \leq i < |\vec{x}| )$ where $\pi \in \{L,R\}$.

We also use $P^{\Box} := P|\Box$.

In \cite{MeredithR05} an interpretation of the new operator is
given. It turns out that there are several possible interpretations
all enjoying the requisite algebraic properties of the operator (see
\cite{milner91polyadicpi}). We will therefore make liberal use of
$(\nu\; \vec{x})P$.

% subsection the_syntax_and_semantics_of_the_notation_system (end)   

\input{qm2pi.qmops} 

\input{qm2pi.sterngerlach} 

\input{qm2pi.metric} 

% section concurrent_process_calculi (end)

%\input{qm2pi.proofsketch}

% section proof sketch (end)

%\input{qm2pi.slviaknots} 

% section spatial logic via knots (end)

\input{qm2pi.conclusion}

% section conclusion (end)

%\input{qm2pi.dtcodes} 

% section wiring algorithm (end)

\input{qm2pi.ack} 

% section acknowledgments (end)

\newpage


\bibliographystyle{plain}   
\bibliography{../../biblios/main.bib}

\input{qm2pi.rhodetails}

\end{document}

 

%\documentclass[12pt]{llncs}
%\documentclass{jktr}

\usepackage[pdftex]{hyperref}                   
\usepackage {listings}
\usepackage {mathpartir}
\usepackage{bcprules}
%\usepackage{listings}
                       
\usepackage{graphicx} 
%\usepackage[margins=2.5cm,nohead,nofoot]{geometry}
%\usepackage{geometry}
\usepackage{amsfonts}
\usepackage{amstext}
\usepackage{latexsym}
\usepackage{amssymb}
\usepackage{color}


%\include{myPreamble}
\include{qm2pi.local} 

%\ifpdf
%\usepackage[pdftex]{graphicx}
%\else
%\usepackage{graphicx}
%\fi

 % \ifpdf
%  \usepackage{pdfsync}
%  \if


%\title{Brief Article}
%\author{David F. Snyder}
%\author{L.G. Meredith}

%\address{Dept. of Math., Texas State University--San Marcos, San Marcos, TX 78666}
       
\pagestyle{empty}


\begin{document}

\lstset{language=[Objective]Caml,frame=shadowbox}

\input{qm2pi.front}

% section front matter (end)

\input{qm2pi.intro} 
 
% section introduction (end)

% \input{qm2pi.knotations} 

% section notation (end)

\input{qm2pi.process.calculi} 

% section concurrent_process_calculi_and_spatial_logics_ (end)
    
%\input{qm2pi.knots2pi} 

%\input{qm2pi.trefoil} 

%\input{qm2pi.mainthm} 

% subsection basic_interpretation (end)

%\input{qm2pi.rho.presentation} 
\subsection{The syntax and semantics of the notation system}\label{sub:the_syntax_and_semantics_of_the_notation_system} % (fold)

We now summarize a technical presentation of the calculus that
embodies our theory of dynamics. The typical presentation of such a
calculus follows the style of giving generators and relations on
them. The grammar, below, describing term constructors, freely
generates the set of processes, $\Proc$. This set is then quotiented
by a relation known as structural congruence and it is over this set
that the notion of dynamics is expressed. This presentation is
essentially that of \cite{MeredithR05} with the addition of
polyadicity and summation. For readability we have relegated some of
the technical subtleties to an appendix.

\subsubsection{Process grammar}\label{subsub:process_grammar}

\begin{mathpar}
  \inferrule* [lab=synchronization] {} {{M} \bc \pzero \;|\; x?F \;|\; x!C }
  \and
  \inferrule* [lab=abstraction] {} {{F} \bc (x)P}
  \and
  \inferrule* [lab=concretion] {} {{C} \bc \langle Q \rangle}
  \and
  \inferrule* [lab=process] {} {{P,Q} \bc M \;| \;P|Q \;|\; @{x}}
  \and
  \inferrule* [lab=name] {} {{x} \bc \quotep{P}}
\end{mathpar} 

Note that $\vec{x}$ (resp. $\vec{P}$) denotes a vector of names
(resp. processes) of length $|\vec{x}|$ (resp. $|\vec{P}|$). We adopt
the following useful abbreviations.

\begin{mathpar}
   x?(\vec{y}).P := x.(\vec{y})P \and  x\clift{\vec{P}} := x.\clift{\vec{P}}
   \and x!(y) := \lift{x}{\dropn{y}}
   \and \Pi_{i=0}^{n-1}P_i := P_0 | \ldots | P_{n-1}
\end{mathpar}

\subsubsection{Structural congruence}

\paragraph{Free and bound names and alpha-equivalence.} At the
core of structural equivalence is alpha-equivalence which identifies
process that are the same up to a change of variable. Formally, we
recognize the distinction between free and bound names. The free names
of a process, $\freenames{P}$, may be calculated recursively as
follows:

\begin{mathpar}
\freenames{\pzero} := \emptyset
  \and \\
  \freenames{x?(y).P} := \{ x \} \cup (\freenames{P} \setminus \{ y \})
  \and 
  \freenames{x!\langle P \rangle} := \{ x \} \cup \{ P \} 
  \and \\
  \freenames{P|Q} := \freenames{P} \cup \freenames{Q}
  \and \\
  \freenames{@{x}} := \{ x \}
\end{mathpar}

$\pi$
$\quotep{\pi}$

$\freenames{-} : \pi \to \mathcal{P}(\quotep{\pi})$

\begin{eqnarray*}
  \freenames{\pzero} & := & \emptyset \\
  \freenames{x?(y).P} & := & \{ x \} \cup (\freenames{P} \setminus \{ y \}) \\
  \freenames{x!\langle P \rangle} & := & \{ x \} \cup \{ P \} \\
  \freenames{P|Q} & := & \freenames{P} \cup \freenames{Q} \\
  \freenames{\dropn{x}} & := & \{ x \}
\end{eqnarray*}

The bound names of a process, $\boundnames{P}$, are those names occurring in $P$
that are not free. For example, in $x?(y).0$, the name $x$ is free, while $y$ is bound.

\begin{mathpar}
  \inferrule* [lab=monoidal-laws] {} { P|Q \equiv Q|P \and P|0 \equiv P \and P|(Q|R) \equiv (P|Q)|R }
\end{mathpar}

\begin{mathpar}
  \inferrule* [lab=alpha-equivalence] {} { (x)P \equiv (y)P\{y/x\} \and y \not\in \freenames{P} }
\end{mathpar}

\begin{definition}
Then two processes, $P,Q$, are alpha-equivalent if $P = Q\{\vec{y}/\vec{x}\}$ for
some $\vec{x} \in \boundnames{Q},\vec{y} \in \boundnames{P}$, where $Q\{\vec{y}/\vec{x}\}$
denotes the capture-avoiding substitution of $\vec{y}$ for $\vec{x}$ in $Q$.
\end{definition}

\begin{definition}
  The {\em structural congruence} \cite{SangiorgiWalker} , $\equiv$,
  between processes is the least congruence containing
  alpha-equivalence, satisfying the abelian monoid laws
  (associativity, commutativity and $\pzero$ as identity) for parallel
  composition $|$ and for summation $+$.
\end{definition}

\subsection{Name equivalence}

We take name equivalence, written $\nameeq$, to be the smallest
equivalence relation generated by the following rules.

\begin{mathpar}
\inferrule*[lab=Quote-drop]
{ }
{ \quotep{@{x}} \nameeq x }

\inferrule*[lab=Struct-equiv]
{ P \scong Q }
{ \quotep{P} \nameeq \quotep{Q} }
\end{mathpar}

The astute reader will have noticed that the mutual recursion of names
and processes imposes a mutual recursion on alpha-equivalence and
structural equivalence via name-equivalence. Fortunately, all of this
works out pleasantly and we may calculate in the natural way, free of
concern. The reader interested in the details is referred to the
appendix \ref{appendix:rho_details}.

\subsection{Substitution}

We use $\Proc$ for the set of processes, $\QProc$ for the set of
names, and $\id{\{}\vec{y} / \vec{x} \id{\}}$ to denote partial maps,
$s : \QProc \rightarrow \QProc$. A map, $s$ lifts, uniquely, to a map
on process terms, $\widehat{s} : \Proc \rightarrow \Proc$ by the
following equations.

\begin{mathpar}
  (0) \psubstp{Q}{P} := 0 \\
  (R \juxtap S) \psubstp{Q}{P}
  :=    
  (R)\psubstp{Q}{P} \juxtap (S) \psubstp{Q}{P} \\
  (x?(y).R) \psubstp{Q}{P}    
  :=    
  (x)\substp{Q}{P} (z)\concat( (R \psubstn{z}{y}) \psubstp{Q}{P} ) \\
  (\lift{x}{R}) \psubstp{Q}{P}  
  :=
  \lift{(x)\substp{Q}{P}}{ R \psubstp{Q}{P} } \\
%   (\dropn{x})  \psubstp{Q}{P}       
%   := 
%   \left\{ 
%     \begin{array}{ccc} 
%       \dropn{\quotep{Q}} & & x \nameeq \quotep{P} \\
%       \dropn{x} & & otherwise \\
%     \end{array}
%   \right. 
  (\dropn{x})  \psubstp{Q}{P}       
  := 
  \left\{ 
    \begin{array}{ccc} 
      Q & & x \nameeq \quotep{P} \\
      \dropn{x} & & otherwise \\
    \end{array}
  \right.
\end{mathpar}
 

where

\begin{eqnarray}
  (x)\id{\{} \lpquote Q \rpquote / \lpquote P \rpquote \id{\}}            = 
  \left\{ 
    \begin{array}{ccc}
      \lpquote Q \rpquote & & x \nameeq \lpquote P \rpquote \\
      x & & otherwise \\
    \end{array}
  \right. \nonumber
\end{eqnarray}

and $z$ is chosen distinct from $\quotep{P}$, $\quotep{Q}$, the free
names in $Q$, and all the names in $R$. Our $\alpha$-equivalence will
be built in the standard way from this substitution.

\begin{remark}\label{rem:no_self_referential_names}
  One consequence of these definitions is that $\forall P. \quotep{P}
  \not\in \freenames{P}$.
\end{remark}

\subsection{ Dynamic quote: an example }

Anticipating something of what's to come, consider applying the
substitution, $\widehat{\id{\{}u / z \id{\}}}$, to the following pair
of processes, $\lift{w}{y!(z)}$ and $w[ \lpquote y!(z) \rpquote ]$.

\begin{eqnarray}
	\lift{w}{y!(z)}\widehat{\id{\{}u / z \id{\}}}
		& = &
		\lift{w}{y!(u)} \nonumber\\
	w[ \lpquote y!(z) \rpquote ] \widehat{ \id{\{}u / z \id{\}} }
		& = &
		w[ \lpquote y!(z) \rpquote ] \nonumber
\end{eqnarray}

Because the body of the process between quotes is impervious to
substitution, we get radically different answers. In fact, by
examining the first process in an input context,
e.g. $x?(z).\lift{w}{y!(z)}$, we see that the process under the lift
operator may be shaped by prefixed inputs binding a name inside it. In
this sense, the lift operator will be seen as a way to dynamically
construct processes before reifying them as names.

Finally equipped with these standard features we can present the
dynamics of the calculus.

\subsubsection{Operational semantics} 

Finally, we introduce the computational dynamics. What marks these
algebras as distinct from other more traditionally studied algebraic
structures, e.g. vector spaces or polynomial rings, is the manner in
which dynamics is captured. In traditional structures, dynamics is typically
expressed through morphisms between such structures, as in linear maps
between vector spaces or morphisms between rings. In algebras
associated with the semantics of computation, the dynamics is
expressed as part of the algebraic structure itself, through a
reduction reduction relation typically denoted by $\red$. Below, we
give a recursive presentation of this relation for the calculus used
in the encoding.

$\red \subseteq \pi \times \pi$
$\red : \pi \to \mathcal{P}(\pi)$

\begin{mathpar}
  \inferrule* [lab=Comm] { \textsf{match}( x_{src}, x_{trgt} ) } { x_{trgt}?(y)P \; | \; x_{src}!\langle {Q} \rangle \red P\{\quotep{Q}/y}\} }
  \and \\
  \inferrule* [lab=Par] {{P} \red {P}'} {{{P} | {Q}} \red {{P}' | {Q}}}
  \and
  \inferrule* [lab=Equiv]{{{P} \scong {P}'} \andalso {{P}' \red {Q}'} \andalso {{Q}' \scong {Q}}}{{P} \red {Q}}
\end{mathpar}

\begin{eqnarray*}
  match_{\equiv} (\quotep{P},\quotep{Q}) & := & P \equiv Q \\
  match_{\dagger}(\quotep{P},\quotep{Q}) & := & \forall R. P|Q \red^{*} R => R \red^{*} 0 \\
  match_{K}(\quotep{P},\quotep{Q}) & := & K \mbox{ for some context } K
\end{eqnarray*}

$u?(x)P | u!\langle Q \rangle \red P\{\quotep{Q}/x\}$

%We write $\wred$ for $\red^*$, and $P\red$ if $\exists Q $ such that $ P \red Q$.
We write $P\red$ if $\exists Q $ such that $ P \red Q$ and $P\not\red$, otherwise.

\section{Replication}

As mentioned before, it is known that replication (and hence
recursion) can be implemented in a higher-order process algebra
\cite{SangiorgiWalker}. As our first example of calculation with the
machinery thus far presented we give the construction explicitly in
the {\rhoc}.

\begin{eqnarray}
	D_{x} & := & \prefix{x}{y}{(\binpar{\outputp{x}{y}}{@{y}})} \nonumber\\
	\bangp_{x}{P} & := & \binpar{{x}!\langle{\binpar{D_{x}}{P}}\rangle}{D_{x}} \nonumber
\end{eqnarray}

\begin{eqnarray}
	\bangp_{x}{P} & & \nonumber\\
	=
	& {x}!\langle{(\prefix{x}{y}{(\outputp{x}{y} | @{y})) | P}}\rangle 
	      | \prefix{x}{y}{(\outputp{x}{y} | @{y})} & \nonumber\\
	\red
	& (\outputp{x}{y} | @{y})\substn{\quotep{(\prefix{x}{y}{(@{y} | \outputp{x}{y})) | P}}}{y} & \nonumber\\
	=
	& \outputp{x}{\quotep{(\prefix{x}{y}{(\outputp{x}{y} | @{y})) | P}}}
	  | {(\prefix{x}{y}{(\outputp{x}{y} | @{y})) | P}} & \nonumber\\
	\red
	& \ldots & \nonumber\\
	\red^*
	& P | P | \ldots & \nonumber
\end{eqnarray}

Of course, this encoding, as an implementation, runs away, unfolding
$\bangp{P}$ eagerly. A lazier and more implementable replication
operator, restricted to input-guarded processes, may be obtained as follows.

\begin{eqnarray}
\bangp{\prefix{u}{v}{P}} 
	:= 
	\binpar{\lift{x}{\prefix{u}{v}{(\binpar{D(x)}{P})}}}{D(x)} \nonumber
\end{eqnarray}

\begin{remark}
  Note that the lazier definition still does not deal with summation
  or mixed summation (i.e. sums over input and output). The reader is
  invited to construct definitions of replication that deal with these
  features. 

  Further, the definitions are parameterized in a name, $x$. Can you,
  gentle reader, make a definition that eliminates this parameter and
  guarantees no accidental interaction between the replication
  machinery and the process being replicated -- i.e. no accidental
  sharing of names used by the process to get its work done and the
  name(s) used by the replication to effect copying. This latter
  revision of the definition of replication is crucial to obtaining
  the expected identity $!!P \sim !P$.
\end{remark}

\begin{remark}\label{rem:paradoxical_combinator}
  The reader familiar with the lambda calculus will have noticed the
  similarity between $D$ and the paradoxical combinator.

  [Ed. note: the existence of this seems to suggest we have to be more
  restrictive on the set of processes and names we admit if we are to
  support no-cloning.]
\end{remark}

\subsubsection{Bisimulation}

The computational dynamics gives rise to another kind of equivalence,
the equivalence of computational behavior. As previously mentioned
this is typically captured \emph{via} some form of bisimulation.

% The notion we use in this paper is weak barbed bisimulation
% \cite{milner91polyadicpi}.

The notion we use in this paper is derived from weak barbed
bisimulation \cite{milner91polyadicpi}. 

\begin{definition}
An \emph{observation relation}, $\downarrow_{\mathcal N}$, over a set
of names, $\mathcal N$, is the smallest relation satisfying the rules
below.

\infrule[Out-barb]{y \in {\mathcal N}, \; x \nameeq y}
		  {\outputp{x}{v} \downarrow_{\mathcal N} x}
\infrule[Par-barb]{\mbox{$P\downarrow_{\mathcal N} x$ or $Q\downarrow_{\mathcal N} x$}}
		  {\binpar{P}{Q} \downarrow_{\mathcal N} x}

We write $P \Downarrow_{\mathcal N} x$ if there is $Q$ such that 
$P \wred Q$ and $Q \downarrow_{\mathcal N} x$.
\end{definition}

\begin{definition}
%\label{def.bbisim}
An  ${\mathcal N}$-\emph{barbed bisimulation} over a set of names, ${\mathcal N}$, is a symmetric binary relation 
${\mathcal S}_{\mathcal N}$ between agents such that $P\rel{S}_{\mathcal N}Q$ implies:
\begin{enumerate}
\item If $P \red P'$ then $Q \wred Q'$ and $P'\rel{S}_{\mathcal N} Q'$.
\item If $P\downarrow_{\mathcal N} x$, then $Q\Downarrow_{\mathcal N} x$.
\end{enumerate}
$P$ is ${\mathcal N}$-barbed bisimilar to $Q$, written
$P \wbbisim_{\mathcal N} Q$, if $P \rel{S}_{\mathcal N} Q$ for some ${\mathcal N}$-barbed bisimulation ${\mathcal S}_{\mathcal N}$.
\end{definition}

$\mathcal{R} \subseteq \pi \times \pi$

$P \mathcal{R} Q => \forall P'. P \red P' \Rightarrow \exists Q'. Q \red Q', P' \mathcal{R} Q'$

$P \vdash x \Rightarrow Q \vdash x$

\begin{mathpar}
  \inferrule*[lab=Out-barb]{x \nameeq y}{{y}!\langle{Q}\rangle \vdash x}
  \and
  \inferrule*[lab=Par-barb]{\mbox{$P\vdash x$ or $Q\vdash x$}}{\binpar{P}{Q} \vdash x}
\end{mathpar}

\subsubsection{Contexts}

One of the principle advantages of computational calculi like the
$\pi$-calculus is a well-defined notion of context,
contextual-equivalence and a correlation between
contextual-equivalence and notions of bisimulation. The notion of
context allows the decomposition of a process into (sub-)process and
its syntactic environment, its context. Thus, a context may be
thought of as a process with a ``hole'' (written $\Box$) in it. The
application of a context $M$ to a process $P$, written $M[P]$, is
tantamount to filling the hole in $M$ with $P$. In this paper we do
not need the full weight of this theory, but do make use of the notion
of context in the proof the main theorem. 

\begin{mathpar}
  \inferrule* [lab=summation] {} {{M_{M},M_{N}} \bc \Box \;|\; x.M_{A} \;|\; M_{M}+M_{N}}
  \and
  \inferrule* [lab=agent] {} {{M_{A}} \bc (\vec{x})M_{P} \;| \; \clift{P_0,\ldots,M_{P},\ldots,P_N}}
  \and \\
  \inferrule* [lab=process] {} {{M_{P}} \bc M_{N} \;| \;P|M_{P} }
\end{mathpar} 

\begin{mathpar}
  \inferrule* [lab=sychronization] {} {M_{N} \bc \Box \;|\; x?M_{F} \;|\; x!M_{C}}
  \and
  \inferrule* [lab=abstraction] {} {{M_{F}} \bc (x)M_{P} }
  \and
  \inferrule* [lab=concretion] {} {{M_{C}} \bc \langle M_{P} \rangle }
  \and \\
  \inferrule* [lab=process] {} {{M_{P}} \bc M_{N} \;| \;P|M_{P} }
\end{mathpar}

\begin{definition}[contextual application] Given a context $M$, and
  process $P$, we define the \emph{contextual application}, $M[P] :=
  M\{P/\Box\}$. That is, the contextual application of M to P is the
  substitution of $P$ for $\Box$ in $M$.
\end{definition}

$\meaningof{-} : L \to \mathcal{P}(\pi)$

\begin{mathpar}
  \inferrule* [lab=collection] {} {\meaningof{true} = \pi, \and \meaningof{~E} = \pi \setminus \meaningof{E}, \and \meaningof{E_{1} \& E_{2}} = \meaningof{E_{1}} \cap \meaningof{E_{2}}}
\end{mathpar}

\begin{mathpar}
  \inferrule* [lab=structure] {} {\meaningof{0} = \{ P \in \pi | P \equiv 0 \}, \and \\ \meaningof{E_1 | E_2} = \{ P \in \pi | P \equiv P_{1} | P_{2}, P_{1} \in \meaningof{E_{1}}, P_{2} \in \meaningof{E_2}\} }
\end{mathpar}

\begin{mathpar}
 \inferrule* [lab=behavior] {} {\meaningof{\langle a?b \rangle E} = \{ P \in \pi | P \equiv Q | u?(y)P', \\ \and \\\\ \and \\ \;\;\; u \in \meaningof{a}, \forall z.P'\{z/y\} \in \meaningof{E\{z/b\}}\}, \and \\ \meaningof{a!E} = \{ P \in \pi | P \equiv Q | x!\langle P' \rangle, x \in \meaningof{a} P' \in \meaningof{E}\} }
\end{mathpar}

\begin{mathpar}
 \inferrule* [lab=nominal] {} {\meaningof{\quotep{E}} = \{ \quotep{P} \in \quotep{\pi} | P \in \meaningof{E} \}, \and \meaningof{\quotep{P}} = \{ \quotep{Q} \in \quotep{\pi} | P \equiv Q \} \and \\ \meaningof{@\quotep{E}} = \{ P \in \pi | P \equiv @x, x \in \meaningof{E} \}}
\end{mathpar}

\begin{eqnarray*}
  \\
  \meaningof{-} : TS \to ST
\end{eqnarray*}

\begin{eqnarray*}
  \\
  L : TS \to ST
\end{eqnarray*}

\begin{eqnarray*}
  \\
  P \models E \iff P \in \meaningof{E}
\end{eqnarray*}

\begin{eqnarray*}
  P \approx_{L} Q \iff \forall E \in L. P \models E \iff Q \models E
\end{eqnarray*}

\begin{eqnarray*}
  P \approx_{K} Q
\end{eqnarray*}

\begin{eqnarray*}
  P \approx Q
\end{eqnarray*}

$\approx_{K} = \approx = \approx_{L}$

\subsubsection{Contextual duality}

Note that contexts extend the quotation operation to a family of
operations from processes to names. Given a context, $M$, we can
define a \emph{nominal context}, $\quotep{M}$ by $\quotep{M}[P] :=
\quotep{M[P]}$. To foreshadow what is to come we observe that these
operations enjoy a duality with processes very much like the duality
between vectors and maps from vectors to scalars.

Further, because the calculus is essentially higher-order, we have a
correspondence between contexts and processes. More specifically,
given a name $x$ and a context $M$ we can construct $M^{*}_{x}$ such
that 

\begin{mathpar}
  M^{*}_{x} | \lift{x}{P} \red M[P]
\end{mathpar}

namely,

\begin{mathpar}
  M^{*}_{x} := x?(u).M[\dropn{u}]
\end{mathpar}

The dependence of $M^{*}_{x}$ on a name makes it an abstraction, 

\begin{mathpar}
  M^{*} := (x)x?(u).M[\dropn{u}]
\end{mathpar}

\subsection{Additional notation}

It will sometimes be convenient to denote the process a name
quotes. We already have the notation $x = \quotep{P}$, but it will be
convenient to introduce an alternate notation, $\procn{x}$, when we
want to emphasize the connection to the use of the name. Note that, by
virtue of name equivalence, $\quotep{\procn{x}} \nameeq x$; so, the
notation is consistent with previous definitions.

Further, because names have structure it is possible to effect
substitutions on the basis of that structure. This means we need to
upgrade our notation for substitutions, which we accomplish by
adapting comprehension notation. Thus,

\begin{mathpar}
  P\{ y / x : x \in S \}
\end{mathpar}

is interpreted to mean the process derived from P by replacing (in a
capture-avoiding manner) each occurrence of $x$ in $S$ by $y$. For example,

\begin{mathpar}
  P\{ \quotep{\procn{x}|\procn{x}} / x : x \in \freenames{P} \}
\end{mathpar}

will replace each (occurrence) of a free name $x$ in $P$ by
$\quotep{\procn{x}|\procn{x}}$.

Also, we will avail ourselves of the notation $x^{L}$ and $x^{R}$ to
denote injections of a name into disjoint copies of the name
space. There are numerous ways to accomplish this. One example can be
found in \cite{MeredithR05}. This notation overloads to vectors of
names: $\vec{x}^{\pi} := (x_{i}^{\pi} \; : \; 0 \leq i < |\vec{x}| )$ where $\pi \in \{L,R\}$.

We also use $P^{\Box} := P|\Box$.

In \cite{MeredithR05} an interpretation of the new operator is
given. It turns out that there are several possible interpretations
all enjoying the requisite algebraic properties of the operator (see
\cite{milner91polyadicpi}). We will therefore make liberal use of
$(\nu\; \vec{x})P$.

% subsection the_syntax_and_semantics_of_the_notation_system (end)   

\input{qm2pi.qmops} 

\input{qm2pi.sterngerlach} 

\input{qm2pi.metric} 

% section concurrent_process_calculi (end)

%\input{qm2pi.proofsketch}

% section proof sketch (end)

%\input{qm2pi.slviaknots} 

% section spatial logic via knots (end)

\input{qm2pi.conclusion}

% section conclusion (end)

%\input{qm2pi.dtcodes} 

% section wiring algorithm (end)

\input{qm2pi.ack} 

% section acknowledgments (end)

\newpage


\bibliographystyle{plain}   
\bibliography{../../biblios/main.bib}

\input{qm2pi.rhodetails}

\end{document}

 

%\documentclass[12pt]{llncs}
%\documentclass{jktr}

\usepackage[pdftex]{hyperref}                   
\usepackage {listings}
\usepackage {mathpartir}
\usepackage{bcprules}
%\usepackage{listings}
                       
\usepackage{graphicx} 
%\usepackage[margins=2.5cm,nohead,nofoot]{geometry}
%\usepackage{geometry}
\usepackage{amsfonts}
\usepackage{amstext}
\usepackage{latexsym}
\usepackage{amssymb}
\usepackage{color}


%\include{myPreamble}
\include{qm2pi.local} 

%\ifpdf
%\usepackage[pdftex]{graphicx}
%\else
%\usepackage{graphicx}
%\fi

 % \ifpdf
%  \usepackage{pdfsync}
%  \if


%\title{Brief Article}
%\author{David F. Snyder}
%\author{L.G. Meredith}

%\address{Dept. of Math., Texas State University--San Marcos, San Marcos, TX 78666}
       
\pagestyle{empty}


\begin{document}

\lstset{language=[Objective]Caml,frame=shadowbox}

\input{qm2pi.front}

% section front matter (end)

\input{qm2pi.intro} 
 
% section introduction (end)

% \input{qm2pi.knotations} 

% section notation (end)

\input{qm2pi.process.calculi} 

% section concurrent_process_calculi_and_spatial_logics_ (end)
    
%\input{qm2pi.knots2pi} 

%\input{qm2pi.trefoil} 

%\input{qm2pi.mainthm} 

% subsection basic_interpretation (end)

%\input{qm2pi.rho.presentation} 
\subsection{The syntax and semantics of the notation system}\label{sub:the_syntax_and_semantics_of_the_notation_system} % (fold)

We now summarize a technical presentation of the calculus that
embodies our theory of dynamics. The typical presentation of such a
calculus follows the style of giving generators and relations on
them. The grammar, below, describing term constructors, freely
generates the set of processes, $\Proc$. This set is then quotiented
by a relation known as structural congruence and it is over this set
that the notion of dynamics is expressed. This presentation is
essentially that of \cite{MeredithR05} with the addition of
polyadicity and summation. For readability we have relegated some of
the technical subtleties to an appendix.

\subsubsection{Process grammar}\label{subsub:process_grammar}

\begin{mathpar}
  \inferrule* [lab=synchronization] {} {{M} \bc \pzero \;|\; x?F \;|\; x!C }
  \and
  \inferrule* [lab=abstraction] {} {{F} \bc (x)P}
  \and
  \inferrule* [lab=concretion] {} {{C} \bc \langle Q \rangle}
  \and
  \inferrule* [lab=process] {} {{P,Q} \bc M \;| \;P|Q \;|\; @{x}}
  \and
  \inferrule* [lab=name] {} {{x} \bc \quotep{P}}
\end{mathpar} 

Note that $\vec{x}$ (resp. $\vec{P}$) denotes a vector of names
(resp. processes) of length $|\vec{x}|$ (resp. $|\vec{P}|$). We adopt
the following useful abbreviations.

\begin{mathpar}
   x?(\vec{y}).P := x.(\vec{y})P \and  x\clift{\vec{P}} := x.\clift{\vec{P}}
   \and x!(y) := \lift{x}{\dropn{y}}
   \and \Pi_{i=0}^{n-1}P_i := P_0 | \ldots | P_{n-1}
\end{mathpar}

\subsubsection{Structural congruence}

\paragraph{Free and bound names and alpha-equivalence.} At the
core of structural equivalence is alpha-equivalence which identifies
process that are the same up to a change of variable. Formally, we
recognize the distinction between free and bound names. The free names
of a process, $\freenames{P}$, may be calculated recursively as
follows:

\begin{mathpar}
\freenames{\pzero} := \emptyset
  \and \\
  \freenames{x?(y).P} := \{ x \} \cup (\freenames{P} \setminus \{ y \})
  \and 
  \freenames{x!\langle P \rangle} := \{ x \} \cup \{ P \} 
  \and \\
  \freenames{P|Q} := \freenames{P} \cup \freenames{Q}
  \and \\
  \freenames{@{x}} := \{ x \}
\end{mathpar}

$\pi$
$\quotep{\pi}$

$\freenames{-} : \pi \to \mathcal{P}(\quotep{\pi})$

\begin{eqnarray*}
  \freenames{\pzero} & := & \emptyset \\
  \freenames{x?(y).P} & := & \{ x \} \cup (\freenames{P} \setminus \{ y \}) \\
  \freenames{x!\langle P \rangle} & := & \{ x \} \cup \{ P \} \\
  \freenames{P|Q} & := & \freenames{P} \cup \freenames{Q} \\
  \freenames{\dropn{x}} & := & \{ x \}
\end{eqnarray*}

The bound names of a process, $\boundnames{P}$, are those names occurring in $P$
that are not free. For example, in $x?(y).0$, the name $x$ is free, while $y$ is bound.

\begin{mathpar}
  \inferrule* [lab=monoidal-laws] {} { P|Q \equiv Q|P \and P|0 \equiv P \and P|(Q|R) \equiv (P|Q)|R }
\end{mathpar}

\begin{mathpar}
  \inferrule* [lab=alpha-equivalence] {} { (x)P \equiv (y)P\{y/x\} \and y \not\in \freenames{P} }
\end{mathpar}

\begin{definition}
Then two processes, $P,Q$, are alpha-equivalent if $P = Q\{\vec{y}/\vec{x}\}$ for
some $\vec{x} \in \boundnames{Q},\vec{y} \in \boundnames{P}$, where $Q\{\vec{y}/\vec{x}\}$
denotes the capture-avoiding substitution of $\vec{y}$ for $\vec{x}$ in $Q$.
\end{definition}

\begin{definition}
  The {\em structural congruence} \cite{SangiorgiWalker} , $\equiv$,
  between processes is the least congruence containing
  alpha-equivalence, satisfying the abelian monoid laws
  (associativity, commutativity and $\pzero$ as identity) for parallel
  composition $|$ and for summation $+$.
\end{definition}

\subsection{Name equivalence}

We take name equivalence, written $\nameeq$, to be the smallest
equivalence relation generated by the following rules.

\begin{mathpar}
\inferrule*[lab=Quote-drop]
{ }
{ \quotep{@{x}} \nameeq x }

\inferrule*[lab=Struct-equiv]
{ P \scong Q }
{ \quotep{P} \nameeq \quotep{Q} }
\end{mathpar}

The astute reader will have noticed that the mutual recursion of names
and processes imposes a mutual recursion on alpha-equivalence and
structural equivalence via name-equivalence. Fortunately, all of this
works out pleasantly and we may calculate in the natural way, free of
concern. The reader interested in the details is referred to the
appendix \ref{appendix:rho_details}.

\subsection{Substitution}

We use $\Proc$ for the set of processes, $\QProc$ for the set of
names, and $\id{\{}\vec{y} / \vec{x} \id{\}}$ to denote partial maps,
$s : \QProc \rightarrow \QProc$. A map, $s$ lifts, uniquely, to a map
on process terms, $\widehat{s} : \Proc \rightarrow \Proc$ by the
following equations.

\begin{mathpar}
  (0) \psubstp{Q}{P} := 0 \\
  (R \juxtap S) \psubstp{Q}{P}
  :=    
  (R)\psubstp{Q}{P} \juxtap (S) \psubstp{Q}{P} \\
  (x?(y).R) \psubstp{Q}{P}    
  :=    
  (x)\substp{Q}{P} (z)\concat( (R \psubstn{z}{y}) \psubstp{Q}{P} ) \\
  (\lift{x}{R}) \psubstp{Q}{P}  
  :=
  \lift{(x)\substp{Q}{P}}{ R \psubstp{Q}{P} } \\
%   (\dropn{x})  \psubstp{Q}{P}       
%   := 
%   \left\{ 
%     \begin{array}{ccc} 
%       \dropn{\quotep{Q}} & & x \nameeq \quotep{P} \\
%       \dropn{x} & & otherwise \\
%     \end{array}
%   \right. 
  (\dropn{x})  \psubstp{Q}{P}       
  := 
  \left\{ 
    \begin{array}{ccc} 
      Q & & x \nameeq \quotep{P} \\
      \dropn{x} & & otherwise \\
    \end{array}
  \right.
\end{mathpar}
 

where

\begin{eqnarray}
  (x)\id{\{} \lpquote Q \rpquote / \lpquote P \rpquote \id{\}}            = 
  \left\{ 
    \begin{array}{ccc}
      \lpquote Q \rpquote & & x \nameeq \lpquote P \rpquote \\
      x & & otherwise \\
    \end{array}
  \right. \nonumber
\end{eqnarray}

and $z$ is chosen distinct from $\quotep{P}$, $\quotep{Q}$, the free
names in $Q$, and all the names in $R$. Our $\alpha$-equivalence will
be built in the standard way from this substitution.

\begin{remark}\label{rem:no_self_referential_names}
  One consequence of these definitions is that $\forall P. \quotep{P}
  \not\in \freenames{P}$.
\end{remark}

\subsection{ Dynamic quote: an example }

Anticipating something of what's to come, consider applying the
substitution, $\widehat{\id{\{}u / z \id{\}}}$, to the following pair
of processes, $\lift{w}{y!(z)}$ and $w[ \lpquote y!(z) \rpquote ]$.

\begin{eqnarray}
	\lift{w}{y!(z)}\widehat{\id{\{}u / z \id{\}}}
		& = &
		\lift{w}{y!(u)} \nonumber\\
	w[ \lpquote y!(z) \rpquote ] \widehat{ \id{\{}u / z \id{\}} }
		& = &
		w[ \lpquote y!(z) \rpquote ] \nonumber
\end{eqnarray}

Because the body of the process between quotes is impervious to
substitution, we get radically different answers. In fact, by
examining the first process in an input context,
e.g. $x?(z).\lift{w}{y!(z)}$, we see that the process under the lift
operator may be shaped by prefixed inputs binding a name inside it. In
this sense, the lift operator will be seen as a way to dynamically
construct processes before reifying them as names.

Finally equipped with these standard features we can present the
dynamics of the calculus.

\subsubsection{Operational semantics} 

Finally, we introduce the computational dynamics. What marks these
algebras as distinct from other more traditionally studied algebraic
structures, e.g. vector spaces or polynomial rings, is the manner in
which dynamics is captured. In traditional structures, dynamics is typically
expressed through morphisms between such structures, as in linear maps
between vector spaces or morphisms between rings. In algebras
associated with the semantics of computation, the dynamics is
expressed as part of the algebraic structure itself, through a
reduction reduction relation typically denoted by $\red$. Below, we
give a recursive presentation of this relation for the calculus used
in the encoding.

$\red \subseteq \pi \times \pi$
$\red : \pi \to \mathcal{P}(\pi)$

\begin{mathpar}
  \inferrule* [lab=Comm] { \textsf{match}( x_{src}, x_{trgt} ) } { x_{trgt}?(y)P \; | \; x_{src}!\langle {Q} \rangle \red P\{\quotep{Q}/y}\} }
  \and \\
  \inferrule* [lab=Par] {{P} \red {P}'} {{{P} | {Q}} \red {{P}' | {Q}}}
  \and
  \inferrule* [lab=Equiv]{{{P} \scong {P}'} \andalso {{P}' \red {Q}'} \andalso {{Q}' \scong {Q}}}{{P} \red {Q}}
\end{mathpar}

\begin{eqnarray*}
  match_{\equiv} (\quotep{P},\quotep{Q}) & := & P \equiv Q \\
  match_{\dagger}(\quotep{P},\quotep{Q}) & := & \forall R. P|Q \red^{*} R => R \red^{*} 0 \\
  match_{K}(\quotep{P},\quotep{Q}) & := & K \mbox{ for some context } K
\end{eqnarray*}

$u?(x)P | u!\langle Q \rangle \red P\{\quotep{Q}/x\}$

%We write $\wred$ for $\red^*$, and $P\red$ if $\exists Q $ such that $ P \red Q$.
We write $P\red$ if $\exists Q $ such that $ P \red Q$ and $P\not\red$, otherwise.

\section{Replication}

As mentioned before, it is known that replication (and hence
recursion) can be implemented in a higher-order process algebra
\cite{SangiorgiWalker}. As our first example of calculation with the
machinery thus far presented we give the construction explicitly in
the {\rhoc}.

\begin{eqnarray}
	D_{x} & := & \prefix{x}{y}{(\binpar{\outputp{x}{y}}{@{y}})} \nonumber\\
	\bangp_{x}{P} & := & \binpar{{x}!\langle{\binpar{D_{x}}{P}}\rangle}{D_{x}} \nonumber
\end{eqnarray}

\begin{eqnarray}
	\bangp_{x}{P} & & \nonumber\\
	=
	& {x}!\langle{(\prefix{x}{y}{(\outputp{x}{y} | @{y})) | P}}\rangle 
	      | \prefix{x}{y}{(\outputp{x}{y} | @{y})} & \nonumber\\
	\red
	& (\outputp{x}{y} | @{y})\substn{\quotep{(\prefix{x}{y}{(@{y} | \outputp{x}{y})) | P}}}{y} & \nonumber\\
	=
	& \outputp{x}{\quotep{(\prefix{x}{y}{(\outputp{x}{y} | @{y})) | P}}}
	  | {(\prefix{x}{y}{(\outputp{x}{y} | @{y})) | P}} & \nonumber\\
	\red
	& \ldots & \nonumber\\
	\red^*
	& P | P | \ldots & \nonumber
\end{eqnarray}

Of course, this encoding, as an implementation, runs away, unfolding
$\bangp{P}$ eagerly. A lazier and more implementable replication
operator, restricted to input-guarded processes, may be obtained as follows.

\begin{eqnarray}
\bangp{\prefix{u}{v}{P}} 
	:= 
	\binpar{\lift{x}{\prefix{u}{v}{(\binpar{D(x)}{P})}}}{D(x)} \nonumber
\end{eqnarray}

\begin{remark}
  Note that the lazier definition still does not deal with summation
  or mixed summation (i.e. sums over input and output). The reader is
  invited to construct definitions of replication that deal with these
  features. 

  Further, the definitions are parameterized in a name, $x$. Can you,
  gentle reader, make a definition that eliminates this parameter and
  guarantees no accidental interaction between the replication
  machinery and the process being replicated -- i.e. no accidental
  sharing of names used by the process to get its work done and the
  name(s) used by the replication to effect copying. This latter
  revision of the definition of replication is crucial to obtaining
  the expected identity $!!P \sim !P$.
\end{remark}

\begin{remark}\label{rem:paradoxical_combinator}
  The reader familiar with the lambda calculus will have noticed the
  similarity between $D$ and the paradoxical combinator.

  [Ed. note: the existence of this seems to suggest we have to be more
  restrictive on the set of processes and names we admit if we are to
  support no-cloning.]
\end{remark}

\subsubsection{Bisimulation}

The computational dynamics gives rise to another kind of equivalence,
the equivalence of computational behavior. As previously mentioned
this is typically captured \emph{via} some form of bisimulation.

% The notion we use in this paper is weak barbed bisimulation
% \cite{milner91polyadicpi}.

The notion we use in this paper is derived from weak barbed
bisimulation \cite{milner91polyadicpi}. 

\begin{definition}
An \emph{observation relation}, $\downarrow_{\mathcal N}$, over a set
of names, $\mathcal N$, is the smallest relation satisfying the rules
below.

\infrule[Out-barb]{y \in {\mathcal N}, \; x \nameeq y}
		  {\outputp{x}{v} \downarrow_{\mathcal N} x}
\infrule[Par-barb]{\mbox{$P\downarrow_{\mathcal N} x$ or $Q\downarrow_{\mathcal N} x$}}
		  {\binpar{P}{Q} \downarrow_{\mathcal N} x}

We write $P \Downarrow_{\mathcal N} x$ if there is $Q$ such that 
$P \wred Q$ and $Q \downarrow_{\mathcal N} x$.
\end{definition}

\begin{definition}
%\label{def.bbisim}
An  ${\mathcal N}$-\emph{barbed bisimulation} over a set of names, ${\mathcal N}$, is a symmetric binary relation 
${\mathcal S}_{\mathcal N}$ between agents such that $P\rel{S}_{\mathcal N}Q$ implies:
\begin{enumerate}
\item If $P \red P'$ then $Q \wred Q'$ and $P'\rel{S}_{\mathcal N} Q'$.
\item If $P\downarrow_{\mathcal N} x$, then $Q\Downarrow_{\mathcal N} x$.
\end{enumerate}
$P$ is ${\mathcal N}$-barbed bisimilar to $Q$, written
$P \wbbisim_{\mathcal N} Q$, if $P \rel{S}_{\mathcal N} Q$ for some ${\mathcal N}$-barbed bisimulation ${\mathcal S}_{\mathcal N}$.
\end{definition}

$\mathcal{R} \subseteq \pi \times \pi$

$P \mathcal{R} Q => \forall P'. P \red P' \Rightarrow \exists Q'. Q \red Q', P' \mathcal{R} Q'$

$P \vdash x \Rightarrow Q \vdash x$

\begin{mathpar}
  \inferrule*[lab=Out-barb]{x \nameeq y}{{y}!\langle{Q}\rangle \vdash x}
  \and
  \inferrule*[lab=Par-barb]{\mbox{$P\vdash x$ or $Q\vdash x$}}{\binpar{P}{Q} \vdash x}
\end{mathpar}

\subsubsection{Contexts}

One of the principle advantages of computational calculi like the
$\pi$-calculus is a well-defined notion of context,
contextual-equivalence and a correlation between
contextual-equivalence and notions of bisimulation. The notion of
context allows the decomposition of a process into (sub-)process and
its syntactic environment, its context. Thus, a context may be
thought of as a process with a ``hole'' (written $\Box$) in it. The
application of a context $M$ to a process $P$, written $M[P]$, is
tantamount to filling the hole in $M$ with $P$. In this paper we do
not need the full weight of this theory, but do make use of the notion
of context in the proof the main theorem. 

\begin{mathpar}
  \inferrule* [lab=summation] {} {{M_{M},M_{N}} \bc \Box \;|\; x.M_{A} \;|\; M_{M}+M_{N}}
  \and
  \inferrule* [lab=agent] {} {{M_{A}} \bc (\vec{x})M_{P} \;| \; \clift{P_0,\ldots,M_{P},\ldots,P_N}}
  \and \\
  \inferrule* [lab=process] {} {{M_{P}} \bc M_{N} \;| \;P|M_{P} }
\end{mathpar} 

\begin{mathpar}
  \inferrule* [lab=sychronization] {} {M_{N} \bc \Box \;|\; x?M_{F} \;|\; x!M_{C}}
  \and
  \inferrule* [lab=abstraction] {} {{M_{F}} \bc (x)M_{P} }
  \and
  \inferrule* [lab=concretion] {} {{M_{C}} \bc \langle M_{P} \rangle }
  \and \\
  \inferrule* [lab=process] {} {{M_{P}} \bc M_{N} \;| \;P|M_{P} }
\end{mathpar}

\begin{definition}[contextual application] Given a context $M$, and
  process $P$, we define the \emph{contextual application}, $M[P] :=
  M\{P/\Box\}$. That is, the contextual application of M to P is the
  substitution of $P$ for $\Box$ in $M$.
\end{definition}

$\meaningof{-} : L \to \mathcal{P}(\pi)$

\begin{mathpar}
  \inferrule* [lab=collection] {} {\meaningof{true} = \pi, \and \meaningof{~E} = \pi \setminus \meaningof{E}, \and \meaningof{E_{1} \& E_{2}} = \meaningof{E_{1}} \cap \meaningof{E_{2}}}
\end{mathpar}

\begin{mathpar}
  \inferrule* [lab=structure] {} {\meaningof{0} = \{ P \in \pi | P \equiv 0 \}, \and \\ \meaningof{E_1 | E_2} = \{ P \in \pi | P \equiv P_{1} | P_{2}, P_{1} \in \meaningof{E_{1}}, P_{2} \in \meaningof{E_2}\} }
\end{mathpar}

\begin{mathpar}
 \inferrule* [lab=behavior] {} {\meaningof{\langle a?b \rangle E} = \{ P \in \pi | P \equiv Q | u?(y)P', \\ \and \\\\ \and \\ \;\;\; u \in \meaningof{a}, \forall z.P'\{z/y\} \in \meaningof{E\{z/b\}}\}, \and \\ \meaningof{a!E} = \{ P \in \pi | P \equiv Q | x!\langle P' \rangle, x \in \meaningof{a} P' \in \meaningof{E}\} }
\end{mathpar}

\begin{mathpar}
 \inferrule* [lab=nominal] {} {\meaningof{\quotep{E}} = \{ \quotep{P} \in \quotep{\pi} | P \in \meaningof{E} \}, \and \meaningof{\quotep{P}} = \{ \quotep{Q} \in \quotep{\pi} | P \equiv Q \} \and \\ \meaningof{@\quotep{E}} = \{ P \in \pi | P \equiv @x, x \in \meaningof{E} \}}
\end{mathpar}

\begin{eqnarray*}
  \\
  \meaningof{-} : TS \to ST
\end{eqnarray*}

\begin{eqnarray*}
  \\
  L : TS \to ST
\end{eqnarray*}

\begin{eqnarray*}
  \\
  P \models E \iff P \in \meaningof{E}
\end{eqnarray*}

\begin{eqnarray*}
  P \approx_{L} Q \iff \forall E \in L. P \models E \iff Q \models E
\end{eqnarray*}

\begin{eqnarray*}
  P \approx_{K} Q
\end{eqnarray*}

\begin{eqnarray*}
  P \approx Q
\end{eqnarray*}

$\approx_{K} = \approx = \approx_{L}$

\subsubsection{Contextual duality}

Note that contexts extend the quotation operation to a family of
operations from processes to names. Given a context, $M$, we can
define a \emph{nominal context}, $\quotep{M}$ by $\quotep{M}[P] :=
\quotep{M[P]}$. To foreshadow what is to come we observe that these
operations enjoy a duality with processes very much like the duality
between vectors and maps from vectors to scalars.

Further, because the calculus is essentially higher-order, we have a
correspondence between contexts and processes. More specifically,
given a name $x$ and a context $M$ we can construct $M^{*}_{x}$ such
that 

\begin{mathpar}
  M^{*}_{x} | \lift{x}{P} \red M[P]
\end{mathpar}

namely,

\begin{mathpar}
  M^{*}_{x} := x?(u).M[\dropn{u}]
\end{mathpar}

The dependence of $M^{*}_{x}$ on a name makes it an abstraction, 

\begin{mathpar}
  M^{*} := (x)x?(u).M[\dropn{u}]
\end{mathpar}

\subsection{Additional notation}

It will sometimes be convenient to denote the process a name
quotes. We already have the notation $x = \quotep{P}$, but it will be
convenient to introduce an alternate notation, $\procn{x}$, when we
want to emphasize the connection to the use of the name. Note that, by
virtue of name equivalence, $\quotep{\procn{x}} \nameeq x$; so, the
notation is consistent with previous definitions.

Further, because names have structure it is possible to effect
substitutions on the basis of that structure. This means we need to
upgrade our notation for substitutions, which we accomplish by
adapting comprehension notation. Thus,

\begin{mathpar}
  P\{ y / x : x \in S \}
\end{mathpar}

is interpreted to mean the process derived from P by replacing (in a
capture-avoiding manner) each occurrence of $x$ in $S$ by $y$. For example,

\begin{mathpar}
  P\{ \quotep{\procn{x}|\procn{x}} / x : x \in \freenames{P} \}
\end{mathpar}

will replace each (occurrence) of a free name $x$ in $P$ by
$\quotep{\procn{x}|\procn{x}}$.

Also, we will avail ourselves of the notation $x^{L}$ and $x^{R}$ to
denote injections of a name into disjoint copies of the name
space. There are numerous ways to accomplish this. One example can be
found in \cite{MeredithR05}. This notation overloads to vectors of
names: $\vec{x}^{\pi} := (x_{i}^{\pi} \; : \; 0 \leq i < |\vec{x}| )$ where $\pi \in \{L,R\}$.

We also use $P^{\Box} := P|\Box$.

In \cite{MeredithR05} an interpretation of the new operator is
given. It turns out that there are several possible interpretations
all enjoying the requisite algebraic properties of the operator (see
\cite{milner91polyadicpi}). We will therefore make liberal use of
$(\nu\; \vec{x})P$.

% subsection the_syntax_and_semantics_of_the_notation_system (end)   

\input{qm2pi.qmops} 

\input{qm2pi.sterngerlach} 

\input{qm2pi.metric} 

% section concurrent_process_calculi (end)

%\input{qm2pi.proofsketch}

% section proof sketch (end)

%\input{qm2pi.slviaknots} 

% section spatial logic via knots (end)

\input{qm2pi.conclusion}

% section conclusion (end)

%\input{qm2pi.dtcodes} 

% section wiring algorithm (end)

\input{qm2pi.ack} 

% section acknowledgments (end)

\newpage


\bibliographystyle{plain}   
\bibliography{../../biblios/main.bib}

\input{qm2pi.rhodetails}

\end{document}

 

% subsection basic_interpretation (end)

%\input{qm2pi.rho.presentation} 
\subsection{The syntax and semantics of the notation system}\label{sub:the_syntax_and_semantics_of_the_notation_system} % (fold)

We now summarize a technical presentation of the calculus that
embodies our theory of dynamics. The typical presentation of such a
calculus follows the style of giving generators and relations on
them. The grammar, below, describing term constructors, freely
generates the set of processes, $\Proc$. This set is then quotiented
by a relation known as structural congruence and it is over this set
that the notion of dynamics is expressed. This presentation is
essentially that of \cite{MeredithR05} with the addition of
polyadicity and summation. For readability we have relegated some of
the technical subtleties to an appendix.

\subsubsection{Process grammar}\label{subsub:process_grammar}

\begin{mathpar}
  \inferrule* [lab=synchronization] {} {{M} \bc \pzero \;|\; x?F \;|\; x!C }
  \and
  \inferrule* [lab=abstraction] {} {{F} \bc (x)P}
  \and
  \inferrule* [lab=concretion] {} {{C} \bc \langle Q \rangle}
  \and
  \inferrule* [lab=process] {} {{P,Q} \bc M \;| \;P|Q \;|\; @{x}}
  \and
  \inferrule* [lab=name] {} {{x} \bc \quotep{P}}
\end{mathpar} 

Note that $\vec{x}$ (resp. $\vec{P}$) denotes a vector of names
(resp. processes) of length $|\vec{x}|$ (resp. $|\vec{P}|$). We adopt
the following useful abbreviations.

\begin{mathpar}
   x?(\vec{y}).P := x.(\vec{y})P \and  x\clift{\vec{P}} := x.\clift{\vec{P}}
   \and x!(y) := \lift{x}{\dropn{y}}
   \and \Pi_{i=0}^{n-1}P_i := P_0 | \ldots | P_{n-1}
\end{mathpar}

\subsubsection{Structural congruence}

\paragraph{Free and bound names and alpha-equivalence.} At the
core of structural equivalence is alpha-equivalence which identifies
process that are the same up to a change of variable. Formally, we
recognize the distinction between free and bound names. The free names
of a process, $\freenames{P}$, may be calculated recursively as
follows:

\begin{mathpar}
\freenames{\pzero} := \emptyset
  \and \\
  \freenames{x?(y).P} := \{ x \} \cup (\freenames{P} \setminus \{ y \})
  \and 
  \freenames{x!\langle P \rangle} := \{ x \} \cup \{ P \} 
  \and \\
  \freenames{P|Q} := \freenames{P} \cup \freenames{Q}
  \and \\
  \freenames{@{x}} := \{ x \}
\end{mathpar}

$\pi$
$\quotep{\pi}$

$\freenames{-} : \pi \to \mathcal{P}(\quotep{\pi})$

\begin{eqnarray*}
  \freenames{\pzero} & := & \emptyset \\
  \freenames{x?(y).P} & := & \{ x \} \cup (\freenames{P} \setminus \{ y \}) \\
  \freenames{x!\langle P \rangle} & := & \{ x \} \cup \{ P \} \\
  \freenames{P|Q} & := & \freenames{P} \cup \freenames{Q} \\
  \freenames{\dropn{x}} & := & \{ x \}
\end{eqnarray*}

The bound names of a process, $\boundnames{P}$, are those names occurring in $P$
that are not free. For example, in $x?(y).0$, the name $x$ is free, while $y$ is bound.

\begin{mathpar}
  \inferrule* [lab=monoidal-laws] {} { P|Q \equiv Q|P \and P|0 \equiv P \and P|(Q|R) \equiv (P|Q)|R }
\end{mathpar}

\begin{mathpar}
  \inferrule* [lab=alpha-equivalence] {} { (x)P \equiv (y)P\{y/x\} \and y \not\in \freenames{P} }
\end{mathpar}

\begin{definition}
Then two processes, $P,Q$, are alpha-equivalent if $P = Q\{\vec{y}/\vec{x}\}$ for
some $\vec{x} \in \boundnames{Q},\vec{y} \in \boundnames{P}$, where $Q\{\vec{y}/\vec{x}\}$
denotes the capture-avoiding substitution of $\vec{y}$ for $\vec{x}$ in $Q$.
\end{definition}

\begin{definition}
  The {\em structural congruence} \cite{SangiorgiWalker} , $\equiv$,
  between processes is the least congruence containing
  alpha-equivalence, satisfying the abelian monoid laws
  (associativity, commutativity and $\pzero$ as identity) for parallel
  composition $|$ and for summation $+$.
\end{definition}

\subsection{Name equivalence}

We take name equivalence, written $\nameeq$, to be the smallest
equivalence relation generated by the following rules.

\begin{mathpar}
\inferrule*[lab=Quote-drop]
{ }
{ \quotep{@{x}} \nameeq x }

\inferrule*[lab=Struct-equiv]
{ P \scong Q }
{ \quotep{P} \nameeq \quotep{Q} }
\end{mathpar}

The astute reader will have noticed that the mutual recursion of names
and processes imposes a mutual recursion on alpha-equivalence and
structural equivalence via name-equivalence. Fortunately, all of this
works out pleasantly and we may calculate in the natural way, free of
concern. The reader interested in the details is referred to the
appendix \ref{appendix:rho_details}.

\subsection{Substitution}

We use $\Proc$ for the set of processes, $\QProc$ for the set of
names, and $\id{\{}\vec{y} / \vec{x} \id{\}}$ to denote partial maps,
$s : \QProc \rightarrow \QProc$. A map, $s$ lifts, uniquely, to a map
on process terms, $\widehat{s} : \Proc \rightarrow \Proc$ by the
following equations.

\begin{mathpar}
  (0) \psubstp{Q}{P} := 0 \\
  (R \juxtap S) \psubstp{Q}{P}
  :=    
  (R)\psubstp{Q}{P} \juxtap (S) \psubstp{Q}{P} \\
  (x?(y).R) \psubstp{Q}{P}    
  :=    
  (x)\substp{Q}{P} (z)\concat( (R \psubstn{z}{y}) \psubstp{Q}{P} ) \\
  (\lift{x}{R}) \psubstp{Q}{P}  
  :=
  \lift{(x)\substp{Q}{P}}{ R \psubstp{Q}{P} } \\
%   (\dropn{x})  \psubstp{Q}{P}       
%   := 
%   \left\{ 
%     \begin{array}{ccc} 
%       \dropn{\quotep{Q}} & & x \nameeq \quotep{P} \\
%       \dropn{x} & & otherwise \\
%     \end{array}
%   \right. 
  (\dropn{x})  \psubstp{Q}{P}       
  := 
  \left\{ 
    \begin{array}{ccc} 
      Q & & x \nameeq \quotep{P} \\
      \dropn{x} & & otherwise \\
    \end{array}
  \right.
\end{mathpar}
 

where

\begin{eqnarray}
  (x)\id{\{} \lpquote Q \rpquote / \lpquote P \rpquote \id{\}}            = 
  \left\{ 
    \begin{array}{ccc}
      \lpquote Q \rpquote & & x \nameeq \lpquote P \rpquote \\
      x & & otherwise \\
    \end{array}
  \right. \nonumber
\end{eqnarray}

and $z$ is chosen distinct from $\quotep{P}$, $\quotep{Q}$, the free
names in $Q$, and all the names in $R$. Our $\alpha$-equivalence will
be built in the standard way from this substitution.

\begin{remark}\label{rem:no_self_referential_names}
  One consequence of these definitions is that $\forall P. \quotep{P}
  \not\in \freenames{P}$.
\end{remark}

\subsection{ Dynamic quote: an example }

Anticipating something of what's to come, consider applying the
substitution, $\widehat{\id{\{}u / z \id{\}}}$, to the following pair
of processes, $\lift{w}{y!(z)}$ and $w[ \lpquote y!(z) \rpquote ]$.

\begin{eqnarray}
	\lift{w}{y!(z)}\widehat{\id{\{}u / z \id{\}}}
		& = &
		\lift{w}{y!(u)} \nonumber\\
	w[ \lpquote y!(z) \rpquote ] \widehat{ \id{\{}u / z \id{\}} }
		& = &
		w[ \lpquote y!(z) \rpquote ] \nonumber
\end{eqnarray}

Because the body of the process between quotes is impervious to
substitution, we get radically different answers. In fact, by
examining the first process in an input context,
e.g. $x?(z).\lift{w}{y!(z)}$, we see that the process under the lift
operator may be shaped by prefixed inputs binding a name inside it. In
this sense, the lift operator will be seen as a way to dynamically
construct processes before reifying them as names.

Finally equipped with these standard features we can present the
dynamics of the calculus.

\subsubsection{Operational semantics} 

Finally, we introduce the computational dynamics. What marks these
algebras as distinct from other more traditionally studied algebraic
structures, e.g. vector spaces or polynomial rings, is the manner in
which dynamics is captured. In traditional structures, dynamics is typically
expressed through morphisms between such structures, as in linear maps
between vector spaces or morphisms between rings. In algebras
associated with the semantics of computation, the dynamics is
expressed as part of the algebraic structure itself, through a
reduction reduction relation typically denoted by $\red$. Below, we
give a recursive presentation of this relation for the calculus used
in the encoding.

$\red \subseteq \pi \times \pi$
$\red : \pi \to \mathcal{P}(\pi)$

\begin{mathpar}
  \inferrule* [lab=Comm] { \textsf{match}( x_{src}, x_{trgt} ) } { x_{trgt}?(y)P \; | \; x_{src}!\langle {Q} \rangle \red P\{\quotep{Q}/y}\} }
  \and \\
  \inferrule* [lab=Par] {{P} \red {P}'} {{{P} | {Q}} \red {{P}' | {Q}}}
  \and
  \inferrule* [lab=Equiv]{{{P} \scong {P}'} \andalso {{P}' \red {Q}'} \andalso {{Q}' \scong {Q}}}{{P} \red {Q}}
\end{mathpar}

\begin{eqnarray*}
  match_{\equiv} (\quotep{P},\quotep{Q}) & := & P \equiv Q \\
  match_{\dagger}(\quotep{P},\quotep{Q}) & := & \forall R. P|Q \red^{*} R => R \red^{*} 0 \\
  match_{K}(\quotep{P},\quotep{Q}) & := & K \mbox{ for some context } K
\end{eqnarray*}

$u?(x)P | u!\langle Q \rangle \red P\{\quotep{Q}/x\}$

%We write $\wred$ for $\red^*$, and $P\red$ if $\exists Q $ such that $ P \red Q$.
We write $P\red$ if $\exists Q $ such that $ P \red Q$ and $P\not\red$, otherwise.

\section{Replication}

As mentioned before, it is known that replication (and hence
recursion) can be implemented in a higher-order process algebra
\cite{SangiorgiWalker}. As our first example of calculation with the
machinery thus far presented we give the construction explicitly in
the {\rhoc}.

\begin{eqnarray}
	D_{x} & := & \prefix{x}{y}{(\binpar{\outputp{x}{y}}{@{y}})} \nonumber\\
	\bangp_{x}{P} & := & \binpar{{x}!\langle{\binpar{D_{x}}{P}}\rangle}{D_{x}} \nonumber
\end{eqnarray}

\begin{eqnarray}
	\bangp_{x}{P} & & \nonumber\\
	=
	& {x}!\langle{(\prefix{x}{y}{(\outputp{x}{y} | @{y})) | P}}\rangle 
	      | \prefix{x}{y}{(\outputp{x}{y} | @{y})} & \nonumber\\
	\red
	& (\outputp{x}{y} | @{y})\substn{\quotep{(\prefix{x}{y}{(@{y} | \outputp{x}{y})) | P}}}{y} & \nonumber\\
	=
	& \outputp{x}{\quotep{(\prefix{x}{y}{(\outputp{x}{y} | @{y})) | P}}}
	  | {(\prefix{x}{y}{(\outputp{x}{y} | @{y})) | P}} & \nonumber\\
	\red
	& \ldots & \nonumber\\
	\red^*
	& P | P | \ldots & \nonumber
\end{eqnarray}

Of course, this encoding, as an implementation, runs away, unfolding
$\bangp{P}$ eagerly. A lazier and more implementable replication
operator, restricted to input-guarded processes, may be obtained as follows.

\begin{eqnarray}
\bangp{\prefix{u}{v}{P}} 
	:= 
	\binpar{\lift{x}{\prefix{u}{v}{(\binpar{D(x)}{P})}}}{D(x)} \nonumber
\end{eqnarray}

\begin{remark}
  Note that the lazier definition still does not deal with summation
  or mixed summation (i.e. sums over input and output). The reader is
  invited to construct definitions of replication that deal with these
  features. 

  Further, the definitions are parameterized in a name, $x$. Can you,
  gentle reader, make a definition that eliminates this parameter and
  guarantees no accidental interaction between the replication
  machinery and the process being replicated -- i.e. no accidental
  sharing of names used by the process to get its work done and the
  name(s) used by the replication to effect copying. This latter
  revision of the definition of replication is crucial to obtaining
  the expected identity $!!P \sim !P$.
\end{remark}

\begin{remark}\label{rem:paradoxical_combinator}
  The reader familiar with the lambda calculus will have noticed the
  similarity between $D$ and the paradoxical combinator.

  [Ed. note: the existence of this seems to suggest we have to be more
  restrictive on the set of processes and names we admit if we are to
  support no-cloning.]
\end{remark}

\subsubsection{Bisimulation}

The computational dynamics gives rise to another kind of equivalence,
the equivalence of computational behavior. As previously mentioned
this is typically captured \emph{via} some form of bisimulation.

% The notion we use in this paper is weak barbed bisimulation
% \cite{milner91polyadicpi}.

The notion we use in this paper is derived from weak barbed
bisimulation \cite{milner91polyadicpi}. 

\begin{definition}
An \emph{observation relation}, $\downarrow_{\mathcal N}$, over a set
of names, $\mathcal N$, is the smallest relation satisfying the rules
below.

\infrule[Out-barb]{y \in {\mathcal N}, \; x \nameeq y}
		  {\outputp{x}{v} \downarrow_{\mathcal N} x}
\infrule[Par-barb]{\mbox{$P\downarrow_{\mathcal N} x$ or $Q\downarrow_{\mathcal N} x$}}
		  {\binpar{P}{Q} \downarrow_{\mathcal N} x}

We write $P \Downarrow_{\mathcal N} x$ if there is $Q$ such that 
$P \wred Q$ and $Q \downarrow_{\mathcal N} x$.
\end{definition}

\begin{definition}
%\label{def.bbisim}
An  ${\mathcal N}$-\emph{barbed bisimulation} over a set of names, ${\mathcal N}$, is a symmetric binary relation 
${\mathcal S}_{\mathcal N}$ between agents such that $P\rel{S}_{\mathcal N}Q$ implies:
\begin{enumerate}
\item If $P \red P'$ then $Q \wred Q'$ and $P'\rel{S}_{\mathcal N} Q'$.
\item If $P\downarrow_{\mathcal N} x$, then $Q\Downarrow_{\mathcal N} x$.
\end{enumerate}
$P$ is ${\mathcal N}$-barbed bisimilar to $Q$, written
$P \wbbisim_{\mathcal N} Q$, if $P \rel{S}_{\mathcal N} Q$ for some ${\mathcal N}$-barbed bisimulation ${\mathcal S}_{\mathcal N}$.
\end{definition}

$\mathcal{R} \subseteq \pi \times \pi$

$P \mathcal{R} Q => \forall P'. P \red P' \Rightarrow \exists Q'. Q \red Q', P' \mathcal{R} Q'$

$P \vdash x \Rightarrow Q \vdash x$

\begin{mathpar}
  \inferrule*[lab=Out-barb]{x \nameeq y}{{y}!\langle{Q}\rangle \vdash x}
  \and
  \inferrule*[lab=Par-barb]{\mbox{$P\vdash x$ or $Q\vdash x$}}{\binpar{P}{Q} \vdash x}
\end{mathpar}

\subsubsection{Contexts}

One of the principle advantages of computational calculi like the
$\pi$-calculus is a well-defined notion of context,
contextual-equivalence and a correlation between
contextual-equivalence and notions of bisimulation. The notion of
context allows the decomposition of a process into (sub-)process and
its syntactic environment, its context. Thus, a context may be
thought of as a process with a ``hole'' (written $\Box$) in it. The
application of a context $M$ to a process $P$, written $M[P]$, is
tantamount to filling the hole in $M$ with $P$. In this paper we do
not need the full weight of this theory, but do make use of the notion
of context in the proof the main theorem. 

\begin{mathpar}
  \inferrule* [lab=summation] {} {{M_{M},M_{N}} \bc \Box \;|\; x.M_{A} \;|\; M_{M}+M_{N}}
  \and
  \inferrule* [lab=agent] {} {{M_{A}} \bc (\vec{x})M_{P} \;| \; \clift{P_0,\ldots,M_{P},\ldots,P_N}}
  \and \\
  \inferrule* [lab=process] {} {{M_{P}} \bc M_{N} \;| \;P|M_{P} }
\end{mathpar} 

\begin{mathpar}
  \inferrule* [lab=sychronization] {} {M_{N} \bc \Box \;|\; x?M_{F} \;|\; x!M_{C}}
  \and
  \inferrule* [lab=abstraction] {} {{M_{F}} \bc (x)M_{P} }
  \and
  \inferrule* [lab=concretion] {} {{M_{C}} \bc \langle M_{P} \rangle }
  \and \\
  \inferrule* [lab=process] {} {{M_{P}} \bc M_{N} \;| \;P|M_{P} }
\end{mathpar}

\begin{definition}[contextual application] Given a context $M$, and
  process $P$, we define the \emph{contextual application}, $M[P] :=
  M\{P/\Box\}$. That is, the contextual application of M to P is the
  substitution of $P$ for $\Box$ in $M$.
\end{definition}

$\meaningof{-} : L \to \mathcal{P}(\pi)$

\begin{mathpar}
  \inferrule* [lab=collection] {} {\meaningof{true} = \pi, \and \meaningof{~E} = \pi \setminus \meaningof{E}, \and \meaningof{E_{1} \& E_{2}} = \meaningof{E_{1}} \cap \meaningof{E_{2}}}
\end{mathpar}

\begin{mathpar}
  \inferrule* [lab=structure] {} {\meaningof{0} = \{ P \in \pi | P \equiv 0 \}, \and \\ \meaningof{E_1 | E_2} = \{ P \in \pi | P \equiv P_{1} | P_{2}, P_{1} \in \meaningof{E_{1}}, P_{2} \in \meaningof{E_2}\} }
\end{mathpar}

\begin{mathpar}
 \inferrule* [lab=behavior] {} {\meaningof{\langle a?b \rangle E} = \{ P \in \pi | P \equiv Q | u?(y)P', \\ \and \\\\ \and \\ \;\;\; u \in \meaningof{a}, \forall z.P'\{z/y\} \in \meaningof{E\{z/b\}}\}, \and \\ \meaningof{a!E} = \{ P \in \pi | P \equiv Q | x!\langle P' \rangle, x \in \meaningof{a} P' \in \meaningof{E}\} }
\end{mathpar}

\begin{mathpar}
 \inferrule* [lab=nominal] {} {\meaningof{\quotep{E}} = \{ \quotep{P} \in \quotep{\pi} | P \in \meaningof{E} \}, \and \meaningof{\quotep{P}} = \{ \quotep{Q} \in \quotep{\pi} | P \equiv Q \} \and \\ \meaningof{@\quotep{E}} = \{ P \in \pi | P \equiv @x, x \in \meaningof{E} \}}
\end{mathpar}

\begin{eqnarray*}
  \\
  \meaningof{-} : TS \to ST
\end{eqnarray*}

\begin{eqnarray*}
  \\
  L : TS \to ST
\end{eqnarray*}

\begin{eqnarray*}
  \\
  P \models E \iff P \in \meaningof{E}
\end{eqnarray*}

\begin{eqnarray*}
  P \approx_{L} Q \iff \forall E \in L. P \models E \iff Q \models E
\end{eqnarray*}

\begin{eqnarray*}
  P \approx_{K} Q
\end{eqnarray*}

\begin{eqnarray*}
  P \approx Q
\end{eqnarray*}

$\approx_{K} = \approx = \approx_{L}$

\subsubsection{Contextual duality}

Note that contexts extend the quotation operation to a family of
operations from processes to names. Given a context, $M$, we can
define a \emph{nominal context}, $\quotep{M}$ by $\quotep{M}[P] :=
\quotep{M[P]}$. To foreshadow what is to come we observe that these
operations enjoy a duality with processes very much like the duality
between vectors and maps from vectors to scalars.

Further, because the calculus is essentially higher-order, we have a
correspondence between contexts and processes. More specifically,
given a name $x$ and a context $M$ we can construct $M^{*}_{x}$ such
that 

\begin{mathpar}
  M^{*}_{x} | \lift{x}{P} \red M[P]
\end{mathpar}

namely,

\begin{mathpar}
  M^{*}_{x} := x?(u).M[\dropn{u}]
\end{mathpar}

The dependence of $M^{*}_{x}$ on a name makes it an abstraction, 

\begin{mathpar}
  M^{*} := (x)x?(u).M[\dropn{u}]
\end{mathpar}

\subsection{Additional notation}

It will sometimes be convenient to denote the process a name
quotes. We already have the notation $x = \quotep{P}$, but it will be
convenient to introduce an alternate notation, $\procn{x}$, when we
want to emphasize the connection to the use of the name. Note that, by
virtue of name equivalence, $\quotep{\procn{x}} \nameeq x$; so, the
notation is consistent with previous definitions.

Further, because names have structure it is possible to effect
substitutions on the basis of that structure. This means we need to
upgrade our notation for substitutions, which we accomplish by
adapting comprehension notation. Thus,

\begin{mathpar}
  P\{ y / x : x \in S \}
\end{mathpar}

is interpreted to mean the process derived from P by replacing (in a
capture-avoiding manner) each occurrence of $x$ in $S$ by $y$. For example,

\begin{mathpar}
  P\{ \quotep{\procn{x}|\procn{x}} / x : x \in \freenames{P} \}
\end{mathpar}

will replace each (occurrence) of a free name $x$ in $P$ by
$\quotep{\procn{x}|\procn{x}}$.

Also, we will avail ourselves of the notation $x^{L}$ and $x^{R}$ to
denote injections of a name into disjoint copies of the name
space. There are numerous ways to accomplish this. One example can be
found in \cite{MeredithR05}. This notation overloads to vectors of
names: $\vec{x}^{\pi} := (x_{i}^{\pi} \; : \; 0 \leq i < |\vec{x}| )$ where $\pi \in \{L,R\}$.

We also use $P^{\Box} := P|\Box$.

In \cite{MeredithR05} an interpretation of the new operator is
given. It turns out that there are several possible interpretations
all enjoying the requisite algebraic properties of the operator (see
\cite{milner91polyadicpi}). We will therefore make liberal use of
$(\nu\; \vec{x})P$.

% subsection the_syntax_and_semantics_of_the_notation_system (end)   

\section{Interpretation of QM}
\subsection{Supporting definitions}
\subsubsection{Multiplication}
\begin{mathpar}
  \quotep{Q} \cdot \quotep{R} := \quotep{Q|R}
  \and \\
  \quotep{Q} \cdot P := P\{ \quotep{Q|R} / \quotep{R} : \quotep{R} \in \freenames{P} \}
\end{mathpar}

\paragraph{Discussion}
The first line needs little explanation. The second line says that
each free name of the process is replaced with the multiplication of
that name by the scalar. Multiplication of a scalar (name) by a state
(process) results in a process all the names of which have been `moved
over' by parallel composition with the process the scalar
quotes. There is a subtlety that the bound names have to be
manipulated so that multiplied names aren't accidentally
captured. There are many ways to achieve this.

\begin{remark}\label{rem:multiplication_identities}
  The reader is invited to verify that for all $x,y,z \in \QProc$ and $P \in \Proc$
  \begin{mathpar}
    x \cdot \quotep{0} \equiv x 
    \and
    x \cdot y \equiv y \cdot x
    \and
    x \cdot (y \cdot z) \equiv (x \cdot y) \cdot z
    \and \\
    \quotep{0} \cdot P \equiv P
    \and \\
    x \cdot (y \cdot P) \equiv (x \cdot y) \cdot P
    \and \\
    x \cdot (P|Q) \equiv (x \cdot P) | (x \cdot Q)
    \and \\    
  \end{mathpar}
\end{remark}

\subsubsection{Tensor product}

We define a tensor product on processes by structural induction.

\paragraph{Tensor of sums} First note that all summations, including
$\pzero$ and sequence, can be written $\Sigma_{i} x_{i}.A_{i} +
\Sigma_{j} x_{j}.C_{j}$, where we have grouped input-guarded processes
together and output-guarded processes together.

Thus, we can define the tensor product of two summations, $N_{1}\otimes N_{2}$, where

\begin{mathpar}
  N_{1} := \Sigma_{i} x_{i}.A_{i} + \Sigma_{j} x_{j}.C_{j}
  \and
  N_{2} := \Sigma_{i'} y_{i'}.B_{i'} + \Sigma_{j'} y_{j'}.D_{j'} 
\end{mathpar}

as follows.

\begin{mathpar}
  \Sigma_{i} x_{i}.A_{i} + \Sigma_{j} x_{j}.C_{j} \otimes \Sigma_{i'}
  y_{i'}.B_{i'} + \Sigma_{j'} y_{j'}.D_{j'} 
  \and \\
  := \; \Sigma_{i} \Sigma_{i'} \quotep{\stackrel{\vee}{x_{i}}| \stackrel{\vee}{y_{i'}}}.(A_{i}\otimes B_{i'}) \; | \; \Sigma_{i'} \Sigma_{i} \quotep{\stackrel{\vee}{y_{i'}}|\stackrel{\vee}{x_{i}}}.(B_{i'}\otimes A_{i})
  \and
  \;\; | \;\; \Sigma_{j} \Sigma_{j'} \quotep{\stackrel{\vee}{x_{j}}|\stackrel{\vee}{y_{j'}}}.(A_{j}\otimes B_{j'}) \; | \; \Sigma_{j'} \Sigma_{j} \quotep{\stackrel{\vee}{y_{j'}}|\stackrel{\vee}{x_{j}}}.(B_{j'}\otimes A_{j})
\end{mathpar}

\begin{remark}
  Do we need to $x^{L}$ and $y^{R}$ for this construction as well?
\end{remark}

\paragraph{Tensor of parallel compositions} Next, we distribute tensor
over par.

\begin{mathpar}
  P_{1}|P_{2} \otimes Q_{1}|Q_{2} := (P_{1} \otimes Q_{1}) | (P_{1}
  \otimes Q_{2}) | (P_{2} \otimes Q_{1}) | (P_{2} \otimes Q_{2})
\end{mathpar}

\paragraph{Tensor with dropped names} We treat tensor of a
process with a dropped name as parallel composition.

\begin{mathpar}
  P \otimes \dropn{x} := P | \dropn{x}
\end{mathpar}

\paragraph{Tensor of agents}

Finally, we need to define tensor on agents. Note that the definition
of tensor on normal products only tensors inputs with inputs and
outputs with outputs. Thus, we only have to define the operation on
``homogeneous'' pairings.

\begin{mathpar}
  (\vec{x})P \otimes (\vec{y})Q
  \and \\
  := (x_{0}^{L}|y_{0}^{R},\ldots,x_{0}^{L}|y_{n}^{R},\ldots,x_{m}^{L}|y_{0}^{R},\ldots,x_{m}^{L}|y_{n}^R)(P\{ \vec{x}^{L}/\vec{x}\} \otimes Q \{ \vec{y}^{R}/\vec{y}\})
  \and \\
  \clift{\vec{P}} \otimes \clift{\vec{Q}}
  \and \\
  := \clift{P_{0}\otimes Q_{0},\ldots,P_{0}\otimes Q_{n},\ldots,P_{m}\otimes Q_{0},\ldots,P_{m}\otimes Q_{n}}
\end{mathpar}

\begin{remark}
  Observe that arities of tensored abstractions matches arities of
  tensored concretions if the original arities matched. Note also that
  the length of the arities corresponds to the increase in dimension
  we see in ordinary vector space tensor product.
\end{remark}

\begin{remark}
  Operationally, this definition distributes the tensor down to
  components ``linked'' by summation. Tensor over summation is
  intriguing in that it mixes names. Moreover, as a consequence of the
  way it mixes names we have the identities for all $x \in \QProc$ and
  $P,Q \in \Proc$

  \begin{mathpar}
    (x \cdot P) \otimes Q \equiv x \cdot (P \otimes Q) \equiv P \otimes (x \cdot Q)
    \and
    P \otimes \pzero \equiv P
  \end{mathpar}

  that the reader is invited to verify.
\end{remark}

\subsubsection{Annihilation}
\begin{mathpar}
  P^{\perp} := \{ Q | \forall R. P|Q \red^{*} R \Rightarrow R \red^{*} \pzero \}
  \and \\
  P^{\underline{\perp}} := \Sigma_{Q \in P^{\perp}} \quotep{Q}?(y).(\dropn{y}|Q) | \Sigma_{Q \in P^{\perp}} \quotep{Q}\clift{\Box}
\end{mathpar}

\paragraph{Discussion} The reader will note that $P^{\perp}$ is a
\emph{set} of processes, while $P^{\underline{\perp}}$ is a
\emph{context}. We call the set $P^{\perp}$ the \emph{annihilators} of
$P$. The parallel composition of a process in the annihilators of $P$
with $P$ will result in a process, the state space of which has all
paths eventually leading to $\pzero$. Execution may endure loops; but
under reasonable conditions of fairness (naturally guaranteed under
most notions of bisimulation) such a composite process cannot get
stuck in such a loop and will, eventually pop out and terminate.

The context $P^{\underline{\perp}}$ is ready and willing to ``take the
$P$ out of'' the process to which it is applied. It will effectively
transmit the code of the process to which it is applied to one of the
annihilators and run the process against it.

\subsubsection{Evaluation}
We fix $M$ a domain of fully abstract interpretation with an equality
coincident with bisimulation. We take $\meaningof{\cdot} : \Proc \to
M$ to be the map interpreting processes and $\nmeaningof{\cdot} : \M
\to Proc$ to be the map running the other way. Then we define

\begin{mathpar}
  \int P := \nmeaningof{\meaningof{P}}
\end{mathpar}

\paragraph{Discussion}
There are many fully abstract interpretations of Milner's
$\pi$-calculus. Any of them can be used as a basis for interpreting
the reflective calculus here. Equipped with such a domain it is
largely a matter of grinding through to check that the Yoneda
construction for the normalization-by-evaluation program can be
extended to this setting.

\begin{remark}
  The reader is invited to verify that $\int (P^{\underline{\perp}}[P]) = 0$.
\end{remark}

\subsection{Quantum mechanics}

Table \ref{tbl:core_qm_op_defns} gives the core operational definitions

\begin{table}[htp]\label{tbl:core_qm_op_defns}
  \center{
    \fbox{
      \begin{tabular}{c|c}
        quantum mechanics & process calculus \\
        \hline
        scalar & $x := \quotep{P}$ \\
        state vector & $\state{P} := P$ \\
        dual & $\state{P}^{*} := \event{P^{\underline{\perp}}} := \quotep{P^{\underline{\perp}}}[-]$ \\
        matrix & $ \Sigma_{\alpha} \state{P_{\alpha}}x_{\alpha}\event{Q_{\alpha}}$ \\
        vector addition & $\state{P} + \state{Q} := \state{P | Q}$ \\
        tensor product & $\state{P} \otimes \state{Q} := \state{P \otimes Q}$ \\
        inner product & $\innerprod{P}{Q} := \quotep{\int P^{\underline{\perp}}[Q]}$ \\
      \end{tabular}
    }
  }
  \caption{QM - operational definitions}
\end{table}

where

\begin{mathpar}
  \prmatrix{P}{Q} := \fprmatrix{P}{\quotep{\pzero}}{Q}
  \and
  \fprmatrix{P}{x}{Q} := (\state{P},x,\event{Q})
  \and
  (\fprmatrix{P}{x}{Q})(\state{R}) := x \cdot \innerprod{Q}{R} \cdot \state{P}
  \and
  (\fprmatrix{P}{x}{Q})(\event{R}) := x \cdot \innerprod{R}{P} \cdot \event{Q}
\end{mathpar}

\paragraph{Discussion}
As promised: vectors (aka states) are represented as processes; duals
as contextual duals; inner product definition should be compared with
standard inner product definition for ....

\begin{remark}
  Assuming $\int (P^{\underline{\perp}}[P]) = 0$, the reader is
  invited to verify that $(\fprmatrix{P}{x}{P})(\state{P}) = x \cdot \state{P}$.
\end{remark}

\begin{remark}
  The reader is invited to verify that $\innerprod{P}{Q}$ could
  equally well have been written $\quotep{\int \stackrel{\vee}{x}}$
  where $x = \event{P^{\underline{\perp}}}(Q)$.

  One of the motivations for this remark is that there is another way
  to factor these operations. We could package up evaluation in the dual:

  \begin{mathpar}
    \state{P}^{*} := \event{\int P^{\underline{\perp}}} := \quotep{\int P^{\underline{\perp}}}[-]
  \end{mathpar}

  and then have inner product defined by
  
  \begin{mathpar}
    \innerprod{P}{Q} := \event{P}(Q)
  \end{mathpar}

  Hopefully, experience with the calculations will provide guidance on
  the best factoring.
\end{remark}

\begin{remark}
  Assuming $\int (P^{\underline{\perp}}[P]) = 0$, the reader is
  invited to verify that $\forall P,Q. (\prmatrix{0}{Q})(\state{0}) =
  \state{0}$ and dually $(\prmatrix{P}{0})(\event{0}) = \event{0}$.
\end{remark}

\begin{remark}
  i'm a little worried that i don't (yet) have proper support for
  complex conjugacy. But, the observation above may give us a
  clue. According to Abramsky, it must be the case that the scalars
  are iso to the homset of the identity for the tensor -- which the
  observation above characterizes. 

  For now, we will simply bookmark the notion with $\overline{x}$.
\end{remark}

\subsubsection{Adjointness}

We need to give a definition of $(\cdot)^{\dagger}$ for matrices. The
obvious candidate definition is
\begin{mathpar}
(\Sigma_{\alpha}\fprmatrix{P_{\alpha}}{x_{\alpha}}{Q_{\alpha}})^{\dagger}
= \Sigma_{\alpha}\fprmatrix{(Q_{\alpha}^{\underline{\perp}})^{*}}{\overline{x}_{\alpha}}{P_{\alpha}^{\underline{\perp}}} 
\end{mathpar}

But, $(Q_{\alpha}^{\underline{\perp}})^{*}$ requires a name along
which to communicate the process to achieve the context application.

\subsubsection{Basis for a basis}
If processes label states and ``addition'' of states (a.k.a. vector
addition) is interpreted as parallel composition, what corresponds to
notions of linear independence and basis? Here, we recall that Yoshida
has developed a set of \emph{combinators} for an asynchronous verison
of Milner's $\pi$-calculus. These are a finite set of processes such
any process can be expressed as parallel composition of these
combinators together with liberal uses of the new operator and
replication. We can simply give a translation of these into the
present calculus and have reasonable expectation that the property
carries over. That is, that the resultant set allows to express all
processes via parallel composition. Note, however, that there is no
new operator or replication in this calculus. As a result, we expect
that the corresponding set is actually infinite. That is, we expect
that the space is actually infinite dimensional.

\begin{remark}
  The attentive reader may be a bit concerned. Certainly, the
  collection $S$, $K$ and $I$ is a finite set of
  combinators. Shouldn't we expect to see a finite set of combinators
  for an effectively equivalent system? i am very sympathetic to this
  critique and feel it warrants full attention. On the other hand, i
  also have in mind the following analogy. The natural numbers, as a
  monoid under addition, has exactly $1$ generator, while the natural
  numbers, as a monoid under multiplication, has countably many
  generators (the primes). We observe that the application of the
  lambda calculus is much less resource sensitive than the parallel
  composition of the $\pi$-calculus. Could it be the case that we have
  an analogy of the form
  
  \begin{mathpar}
    m + n : MN :: m*n : M|N
  \end{mathpar}

  giving a similar blow up in the set of ``primes''?  This is such a
  wonderful thought that, even if it's not true, i think it's worth
  writing down.
\end{remark}
 

\documentclass[12pt]{llncs}
%\documentclass{jktr}

\usepackage[pdftex]{hyperref}                   
\usepackage {listings}
\usepackage {mathpartir}
\usepackage{bcprules}
%\usepackage{listings}
                       
\usepackage{graphicx} 
%\usepackage[margins=2.5cm,nohead,nofoot]{geometry}
%\usepackage{geometry}
\usepackage{amsfonts}
\usepackage{amstext}
\usepackage{latexsym}
\usepackage{amssymb}
\usepackage{color}


%\include{myPreamble}
\include{qm2pi.local} 

%\ifpdf
%\usepackage[pdftex]{graphicx}
%\else
%\usepackage{graphicx}
%\fi

 % \ifpdf
%  \usepackage{pdfsync}
%  \if


%\title{Brief Article}
%\author{David F. Snyder}
%\author{L.G. Meredith}

%\address{Dept. of Math., Texas State University--San Marcos, San Marcos, TX 78666}
       
\pagestyle{empty}


\begin{document}

\lstset{language=[Objective]Caml,frame=shadowbox}

\input{qm2pi.front}

% section front matter (end)

\input{qm2pi.intro} 
 
% section introduction (end)

% \input{qm2pi.knotations} 

% section notation (end)

\input{qm2pi.process.calculi} 

% section concurrent_process_calculi_and_spatial_logics_ (end)
    
%\input{qm2pi.knots2pi} 

%\input{qm2pi.trefoil} 

%\input{qm2pi.mainthm} 

% subsection basic_interpretation (end)

%\input{qm2pi.rho.presentation} 
\subsection{The syntax and semantics of the notation system}\label{sub:the_syntax_and_semantics_of_the_notation_system} % (fold)

We now summarize a technical presentation of the calculus that
embodies our theory of dynamics. The typical presentation of such a
calculus follows the style of giving generators and relations on
them. The grammar, below, describing term constructors, freely
generates the set of processes, $\Proc$. This set is then quotiented
by a relation known as structural congruence and it is over this set
that the notion of dynamics is expressed. This presentation is
essentially that of \cite{MeredithR05} with the addition of
polyadicity and summation. For readability we have relegated some of
the technical subtleties to an appendix.

\subsubsection{Process grammar}\label{subsub:process_grammar}

\begin{mathpar}
  \inferrule* [lab=synchronization] {} {{M} \bc \pzero \;|\; x?F \;|\; x!C }
  \and
  \inferrule* [lab=abstraction] {} {{F} \bc (x)P}
  \and
  \inferrule* [lab=concretion] {} {{C} \bc \langle Q \rangle}
  \and
  \inferrule* [lab=process] {} {{P,Q} \bc M \;| \;P|Q \;|\; @{x}}
  \and
  \inferrule* [lab=name] {} {{x} \bc \quotep{P}}
\end{mathpar} 

Note that $\vec{x}$ (resp. $\vec{P}$) denotes a vector of names
(resp. processes) of length $|\vec{x}|$ (resp. $|\vec{P}|$). We adopt
the following useful abbreviations.

\begin{mathpar}
   x?(\vec{y}).P := x.(\vec{y})P \and  x\clift{\vec{P}} := x.\clift{\vec{P}}
   \and x!(y) := \lift{x}{\dropn{y}}
   \and \Pi_{i=0}^{n-1}P_i := P_0 | \ldots | P_{n-1}
\end{mathpar}

\subsubsection{Structural congruence}

\paragraph{Free and bound names and alpha-equivalence.} At the
core of structural equivalence is alpha-equivalence which identifies
process that are the same up to a change of variable. Formally, we
recognize the distinction between free and bound names. The free names
of a process, $\freenames{P}$, may be calculated recursively as
follows:

\begin{mathpar}
\freenames{\pzero} := \emptyset
  \and \\
  \freenames{x?(y).P} := \{ x \} \cup (\freenames{P} \setminus \{ y \})
  \and 
  \freenames{x!\langle P \rangle} := \{ x \} \cup \{ P \} 
  \and \\
  \freenames{P|Q} := \freenames{P} \cup \freenames{Q}
  \and \\
  \freenames{@{x}} := \{ x \}
\end{mathpar}

$\pi$
$\quotep{\pi}$

$\freenames{-} : \pi \to \mathcal{P}(\quotep{\pi})$

\begin{eqnarray*}
  \freenames{\pzero} & := & \emptyset \\
  \freenames{x?(y).P} & := & \{ x \} \cup (\freenames{P} \setminus \{ y \}) \\
  \freenames{x!\langle P \rangle} & := & \{ x \} \cup \{ P \} \\
  \freenames{P|Q} & := & \freenames{P} \cup \freenames{Q} \\
  \freenames{\dropn{x}} & := & \{ x \}
\end{eqnarray*}

The bound names of a process, $\boundnames{P}$, are those names occurring in $P$
that are not free. For example, in $x?(y).0$, the name $x$ is free, while $y$ is bound.

\begin{mathpar}
  \inferrule* [lab=monoidal-laws] {} { P|Q \equiv Q|P \and P|0 \equiv P \and P|(Q|R) \equiv (P|Q)|R }
\end{mathpar}

\begin{mathpar}
  \inferrule* [lab=alpha-equivalence] {} { (x)P \equiv (y)P\{y/x\} \and y \not\in \freenames{P} }
\end{mathpar}

\begin{definition}
Then two processes, $P,Q$, are alpha-equivalent if $P = Q\{\vec{y}/\vec{x}\}$ for
some $\vec{x} \in \boundnames{Q},\vec{y} \in \boundnames{P}$, where $Q\{\vec{y}/\vec{x}\}$
denotes the capture-avoiding substitution of $\vec{y}$ for $\vec{x}$ in $Q$.
\end{definition}

\begin{definition}
  The {\em structural congruence} \cite{SangiorgiWalker} , $\equiv$,
  between processes is the least congruence containing
  alpha-equivalence, satisfying the abelian monoid laws
  (associativity, commutativity and $\pzero$ as identity) for parallel
  composition $|$ and for summation $+$.
\end{definition}

\subsection{Name equivalence}

We take name equivalence, written $\nameeq$, to be the smallest
equivalence relation generated by the following rules.

\begin{mathpar}
\inferrule*[lab=Quote-drop]
{ }
{ \quotep{@{x}} \nameeq x }

\inferrule*[lab=Struct-equiv]
{ P \scong Q }
{ \quotep{P} \nameeq \quotep{Q} }
\end{mathpar}

The astute reader will have noticed that the mutual recursion of names
and processes imposes a mutual recursion on alpha-equivalence and
structural equivalence via name-equivalence. Fortunately, all of this
works out pleasantly and we may calculate in the natural way, free of
concern. The reader interested in the details is referred to the
appendix \ref{appendix:rho_details}.

\subsection{Substitution}

We use $\Proc$ for the set of processes, $\QProc$ for the set of
names, and $\id{\{}\vec{y} / \vec{x} \id{\}}$ to denote partial maps,
$s : \QProc \rightarrow \QProc$. A map, $s$ lifts, uniquely, to a map
on process terms, $\widehat{s} : \Proc \rightarrow \Proc$ by the
following equations.

\begin{mathpar}
  (0) \psubstp{Q}{P} := 0 \\
  (R \juxtap S) \psubstp{Q}{P}
  :=    
  (R)\psubstp{Q}{P} \juxtap (S) \psubstp{Q}{P} \\
  (x?(y).R) \psubstp{Q}{P}    
  :=    
  (x)\substp{Q}{P} (z)\concat( (R \psubstn{z}{y}) \psubstp{Q}{P} ) \\
  (\lift{x}{R}) \psubstp{Q}{P}  
  :=
  \lift{(x)\substp{Q}{P}}{ R \psubstp{Q}{P} } \\
%   (\dropn{x})  \psubstp{Q}{P}       
%   := 
%   \left\{ 
%     \begin{array}{ccc} 
%       \dropn{\quotep{Q}} & & x \nameeq \quotep{P} \\
%       \dropn{x} & & otherwise \\
%     \end{array}
%   \right. 
  (\dropn{x})  \psubstp{Q}{P}       
  := 
  \left\{ 
    \begin{array}{ccc} 
      Q & & x \nameeq \quotep{P} \\
      \dropn{x} & & otherwise \\
    \end{array}
  \right.
\end{mathpar}
 

where

\begin{eqnarray}
  (x)\id{\{} \lpquote Q \rpquote / \lpquote P \rpquote \id{\}}            = 
  \left\{ 
    \begin{array}{ccc}
      \lpquote Q \rpquote & & x \nameeq \lpquote P \rpquote \\
      x & & otherwise \\
    \end{array}
  \right. \nonumber
\end{eqnarray}

and $z$ is chosen distinct from $\quotep{P}$, $\quotep{Q}$, the free
names in $Q$, and all the names in $R$. Our $\alpha$-equivalence will
be built in the standard way from this substitution.

\begin{remark}\label{rem:no_self_referential_names}
  One consequence of these definitions is that $\forall P. \quotep{P}
  \not\in \freenames{P}$.
\end{remark}

\subsection{ Dynamic quote: an example }

Anticipating something of what's to come, consider applying the
substitution, $\widehat{\id{\{}u / z \id{\}}}$, to the following pair
of processes, $\lift{w}{y!(z)}$ and $w[ \lpquote y!(z) \rpquote ]$.

\begin{eqnarray}
	\lift{w}{y!(z)}\widehat{\id{\{}u / z \id{\}}}
		& = &
		\lift{w}{y!(u)} \nonumber\\
	w[ \lpquote y!(z) \rpquote ] \widehat{ \id{\{}u / z \id{\}} }
		& = &
		w[ \lpquote y!(z) \rpquote ] \nonumber
\end{eqnarray}

Because the body of the process between quotes is impervious to
substitution, we get radically different answers. In fact, by
examining the first process in an input context,
e.g. $x?(z).\lift{w}{y!(z)}$, we see that the process under the lift
operator may be shaped by prefixed inputs binding a name inside it. In
this sense, the lift operator will be seen as a way to dynamically
construct processes before reifying them as names.

Finally equipped with these standard features we can present the
dynamics of the calculus.

\subsubsection{Operational semantics} 

Finally, we introduce the computational dynamics. What marks these
algebras as distinct from other more traditionally studied algebraic
structures, e.g. vector spaces or polynomial rings, is the manner in
which dynamics is captured. In traditional structures, dynamics is typically
expressed through morphisms between such structures, as in linear maps
between vector spaces or morphisms between rings. In algebras
associated with the semantics of computation, the dynamics is
expressed as part of the algebraic structure itself, through a
reduction reduction relation typically denoted by $\red$. Below, we
give a recursive presentation of this relation for the calculus used
in the encoding.

$\red \subseteq \pi \times \pi$
$\red : \pi \to \mathcal{P}(\pi)$

\begin{mathpar}
  \inferrule* [lab=Comm] { \textsf{match}( x_{src}, x_{trgt} ) } { x_{trgt}?(y)P \; | \; x_{src}!\langle {Q} \rangle \red P\{\quotep{Q}/y}\} }
  \and \\
  \inferrule* [lab=Par] {{P} \red {P}'} {{{P} | {Q}} \red {{P}' | {Q}}}
  \and
  \inferrule* [lab=Equiv]{{{P} \scong {P}'} \andalso {{P}' \red {Q}'} \andalso {{Q}' \scong {Q}}}{{P} \red {Q}}
\end{mathpar}

\begin{eqnarray*}
  match_{\equiv} (\quotep{P},\quotep{Q}) & := & P \equiv Q \\
  match_{\dagger}(\quotep{P},\quotep{Q}) & := & \forall R. P|Q \red^{*} R => R \red^{*} 0 \\
  match_{K}(\quotep{P},\quotep{Q}) & := & K \mbox{ for some context } K
\end{eqnarray*}

$u?(x)P | u!\langle Q \rangle \red P\{\quotep{Q}/x\}$

%We write $\wred$ for $\red^*$, and $P\red$ if $\exists Q $ such that $ P \red Q$.
We write $P\red$ if $\exists Q $ such that $ P \red Q$ and $P\not\red$, otherwise.

\section{Replication}

As mentioned before, it is known that replication (and hence
recursion) can be implemented in a higher-order process algebra
\cite{SangiorgiWalker}. As our first example of calculation with the
machinery thus far presented we give the construction explicitly in
the {\rhoc}.

\begin{eqnarray}
	D_{x} & := & \prefix{x}{y}{(\binpar{\outputp{x}{y}}{@{y}})} \nonumber\\
	\bangp_{x}{P} & := & \binpar{{x}!\langle{\binpar{D_{x}}{P}}\rangle}{D_{x}} \nonumber
\end{eqnarray}

\begin{eqnarray}
	\bangp_{x}{P} & & \nonumber\\
	=
	& {x}!\langle{(\prefix{x}{y}{(\outputp{x}{y} | @{y})) | P}}\rangle 
	      | \prefix{x}{y}{(\outputp{x}{y} | @{y})} & \nonumber\\
	\red
	& (\outputp{x}{y} | @{y})\substn{\quotep{(\prefix{x}{y}{(@{y} | \outputp{x}{y})) | P}}}{y} & \nonumber\\
	=
	& \outputp{x}{\quotep{(\prefix{x}{y}{(\outputp{x}{y} | @{y})) | P}}}
	  | {(\prefix{x}{y}{(\outputp{x}{y} | @{y})) | P}} & \nonumber\\
	\red
	& \ldots & \nonumber\\
	\red^*
	& P | P | \ldots & \nonumber
\end{eqnarray}

Of course, this encoding, as an implementation, runs away, unfolding
$\bangp{P}$ eagerly. A lazier and more implementable replication
operator, restricted to input-guarded processes, may be obtained as follows.

\begin{eqnarray}
\bangp{\prefix{u}{v}{P}} 
	:= 
	\binpar{\lift{x}{\prefix{u}{v}{(\binpar{D(x)}{P})}}}{D(x)} \nonumber
\end{eqnarray}

\begin{remark}
  Note that the lazier definition still does not deal with summation
  or mixed summation (i.e. sums over input and output). The reader is
  invited to construct definitions of replication that deal with these
  features. 

  Further, the definitions are parameterized in a name, $x$. Can you,
  gentle reader, make a definition that eliminates this parameter and
  guarantees no accidental interaction between the replication
  machinery and the process being replicated -- i.e. no accidental
  sharing of names used by the process to get its work done and the
  name(s) used by the replication to effect copying. This latter
  revision of the definition of replication is crucial to obtaining
  the expected identity $!!P \sim !P$.
\end{remark}

\begin{remark}\label{rem:paradoxical_combinator}
  The reader familiar with the lambda calculus will have noticed the
  similarity between $D$ and the paradoxical combinator.

  [Ed. note: the existence of this seems to suggest we have to be more
  restrictive on the set of processes and names we admit if we are to
  support no-cloning.]
\end{remark}

\subsubsection{Bisimulation}

The computational dynamics gives rise to another kind of equivalence,
the equivalence of computational behavior. As previously mentioned
this is typically captured \emph{via} some form of bisimulation.

% The notion we use in this paper is weak barbed bisimulation
% \cite{milner91polyadicpi}.

The notion we use in this paper is derived from weak barbed
bisimulation \cite{milner91polyadicpi}. 

\begin{definition}
An \emph{observation relation}, $\downarrow_{\mathcal N}$, over a set
of names, $\mathcal N$, is the smallest relation satisfying the rules
below.

\infrule[Out-barb]{y \in {\mathcal N}, \; x \nameeq y}
		  {\outputp{x}{v} \downarrow_{\mathcal N} x}
\infrule[Par-barb]{\mbox{$P\downarrow_{\mathcal N} x$ or $Q\downarrow_{\mathcal N} x$}}
		  {\binpar{P}{Q} \downarrow_{\mathcal N} x}

We write $P \Downarrow_{\mathcal N} x$ if there is $Q$ such that 
$P \wred Q$ and $Q \downarrow_{\mathcal N} x$.
\end{definition}

\begin{definition}
%\label{def.bbisim}
An  ${\mathcal N}$-\emph{barbed bisimulation} over a set of names, ${\mathcal N}$, is a symmetric binary relation 
${\mathcal S}_{\mathcal N}$ between agents such that $P\rel{S}_{\mathcal N}Q$ implies:
\begin{enumerate}
\item If $P \red P'$ then $Q \wred Q'$ and $P'\rel{S}_{\mathcal N} Q'$.
\item If $P\downarrow_{\mathcal N} x$, then $Q\Downarrow_{\mathcal N} x$.
\end{enumerate}
$P$ is ${\mathcal N}$-barbed bisimilar to $Q$, written
$P \wbbisim_{\mathcal N} Q$, if $P \rel{S}_{\mathcal N} Q$ for some ${\mathcal N}$-barbed bisimulation ${\mathcal S}_{\mathcal N}$.
\end{definition}

$\mathcal{R} \subseteq \pi \times \pi$

$P \mathcal{R} Q => \forall P'. P \red P' \Rightarrow \exists Q'. Q \red Q', P' \mathcal{R} Q'$

$P \vdash x \Rightarrow Q \vdash x$

\begin{mathpar}
  \inferrule*[lab=Out-barb]{x \nameeq y}{{y}!\langle{Q}\rangle \vdash x}
  \and
  \inferrule*[lab=Par-barb]{\mbox{$P\vdash x$ or $Q\vdash x$}}{\binpar{P}{Q} \vdash x}
\end{mathpar}

\subsubsection{Contexts}

One of the principle advantages of computational calculi like the
$\pi$-calculus is a well-defined notion of context,
contextual-equivalence and a correlation between
contextual-equivalence and notions of bisimulation. The notion of
context allows the decomposition of a process into (sub-)process and
its syntactic environment, its context. Thus, a context may be
thought of as a process with a ``hole'' (written $\Box$) in it. The
application of a context $M$ to a process $P$, written $M[P]$, is
tantamount to filling the hole in $M$ with $P$. In this paper we do
not need the full weight of this theory, but do make use of the notion
of context in the proof the main theorem. 

\begin{mathpar}
  \inferrule* [lab=summation] {} {{M_{M},M_{N}} \bc \Box \;|\; x.M_{A} \;|\; M_{M}+M_{N}}
  \and
  \inferrule* [lab=agent] {} {{M_{A}} \bc (\vec{x})M_{P} \;| \; \clift{P_0,\ldots,M_{P},\ldots,P_N}}
  \and \\
  \inferrule* [lab=process] {} {{M_{P}} \bc M_{N} \;| \;P|M_{P} }
\end{mathpar} 

\begin{mathpar}
  \inferrule* [lab=sychronization] {} {M_{N} \bc \Box \;|\; x?M_{F} \;|\; x!M_{C}}
  \and
  \inferrule* [lab=abstraction] {} {{M_{F}} \bc (x)M_{P} }
  \and
  \inferrule* [lab=concretion] {} {{M_{C}} \bc \langle M_{P} \rangle }
  \and \\
  \inferrule* [lab=process] {} {{M_{P}} \bc M_{N} \;| \;P|M_{P} }
\end{mathpar}

\begin{definition}[contextual application] Given a context $M$, and
  process $P$, we define the \emph{contextual application}, $M[P] :=
  M\{P/\Box\}$. That is, the contextual application of M to P is the
  substitution of $P$ for $\Box$ in $M$.
\end{definition}

$\meaningof{-} : L \to \mathcal{P}(\pi)$

\begin{mathpar}
  \inferrule* [lab=collection] {} {\meaningof{true} = \pi, \and \meaningof{~E} = \pi \setminus \meaningof{E}, \and \meaningof{E_{1} \& E_{2}} = \meaningof{E_{1}} \cap \meaningof{E_{2}}}
\end{mathpar}

\begin{mathpar}
  \inferrule* [lab=structure] {} {\meaningof{0} = \{ P \in \pi | P \equiv 0 \}, \and \\ \meaningof{E_1 | E_2} = \{ P \in \pi | P \equiv P_{1} | P_{2}, P_{1} \in \meaningof{E_{1}}, P_{2} \in \meaningof{E_2}\} }
\end{mathpar}

\begin{mathpar}
 \inferrule* [lab=behavior] {} {\meaningof{\langle a?b \rangle E} = \{ P \in \pi | P \equiv Q | u?(y)P', \\ \and \\\\ \and \\ \;\;\; u \in \meaningof{a}, \forall z.P'\{z/y\} \in \meaningof{E\{z/b\}}\}, \and \\ \meaningof{a!E} = \{ P \in \pi | P \equiv Q | x!\langle P' \rangle, x \in \meaningof{a} P' \in \meaningof{E}\} }
\end{mathpar}

\begin{mathpar}
 \inferrule* [lab=nominal] {} {\meaningof{\quotep{E}} = \{ \quotep{P} \in \quotep{\pi} | P \in \meaningof{E} \}, \and \meaningof{\quotep{P}} = \{ \quotep{Q} \in \quotep{\pi} | P \equiv Q \} \and \\ \meaningof{@\quotep{E}} = \{ P \in \pi | P \equiv @x, x \in \meaningof{E} \}}
\end{mathpar}

\begin{eqnarray*}
  \\
  \meaningof{-} : TS \to ST
\end{eqnarray*}

\begin{eqnarray*}
  \\
  L : TS \to ST
\end{eqnarray*}

\begin{eqnarray*}
  \\
  P \models E \iff P \in \meaningof{E}
\end{eqnarray*}

\begin{eqnarray*}
  P \approx_{L} Q \iff \forall E \in L. P \models E \iff Q \models E
\end{eqnarray*}

\begin{eqnarray*}
  P \approx_{K} Q
\end{eqnarray*}

\begin{eqnarray*}
  P \approx Q
\end{eqnarray*}

$\approx_{K} = \approx = \approx_{L}$

\subsubsection{Contextual duality}

Note that contexts extend the quotation operation to a family of
operations from processes to names. Given a context, $M$, we can
define a \emph{nominal context}, $\quotep{M}$ by $\quotep{M}[P] :=
\quotep{M[P]}$. To foreshadow what is to come we observe that these
operations enjoy a duality with processes very much like the duality
between vectors and maps from vectors to scalars.

Further, because the calculus is essentially higher-order, we have a
correspondence between contexts and processes. More specifically,
given a name $x$ and a context $M$ we can construct $M^{*}_{x}$ such
that 

\begin{mathpar}
  M^{*}_{x} | \lift{x}{P} \red M[P]
\end{mathpar}

namely,

\begin{mathpar}
  M^{*}_{x} := x?(u).M[\dropn{u}]
\end{mathpar}

The dependence of $M^{*}_{x}$ on a name makes it an abstraction, 

\begin{mathpar}
  M^{*} := (x)x?(u).M[\dropn{u}]
\end{mathpar}

\subsection{Additional notation}

It will sometimes be convenient to denote the process a name
quotes. We already have the notation $x = \quotep{P}$, but it will be
convenient to introduce an alternate notation, $\procn{x}$, when we
want to emphasize the connection to the use of the name. Note that, by
virtue of name equivalence, $\quotep{\procn{x}} \nameeq x$; so, the
notation is consistent with previous definitions.

Further, because names have structure it is possible to effect
substitutions on the basis of that structure. This means we need to
upgrade our notation for substitutions, which we accomplish by
adapting comprehension notation. Thus,

\begin{mathpar}
  P\{ y / x : x \in S \}
\end{mathpar}

is interpreted to mean the process derived from P by replacing (in a
capture-avoiding manner) each occurrence of $x$ in $S$ by $y$. For example,

\begin{mathpar}
  P\{ \quotep{\procn{x}|\procn{x}} / x : x \in \freenames{P} \}
\end{mathpar}

will replace each (occurrence) of a free name $x$ in $P$ by
$\quotep{\procn{x}|\procn{x}}$.

Also, we will avail ourselves of the notation $x^{L}$ and $x^{R}$ to
denote injections of a name into disjoint copies of the name
space. There are numerous ways to accomplish this. One example can be
found in \cite{MeredithR05}. This notation overloads to vectors of
names: $\vec{x}^{\pi} := (x_{i}^{\pi} \; : \; 0 \leq i < |\vec{x}| )$ where $\pi \in \{L,R\}$.

We also use $P^{\Box} := P|\Box$.

In \cite{MeredithR05} an interpretation of the new operator is
given. It turns out that there are several possible interpretations
all enjoying the requisite algebraic properties of the operator (see
\cite{milner91polyadicpi}). We will therefore make liberal use of
$(\nu\; \vec{x})P$.

% subsection the_syntax_and_semantics_of_the_notation_system (end)   

\input{qm2pi.qmops} 

\input{qm2pi.sterngerlach} 

\input{qm2pi.metric} 

% section concurrent_process_calculi (end)

%\input{qm2pi.proofsketch}

% section proof sketch (end)

%\input{qm2pi.slviaknots} 

% section spatial logic via knots (end)

\input{qm2pi.conclusion}

% section conclusion (end)

%\input{qm2pi.dtcodes} 

% section wiring algorithm (end)

\input{qm2pi.ack} 

% section acknowledgments (end)

\newpage


\bibliographystyle{plain}   
\bibliography{../../biblios/main.bib}

\input{qm2pi.rhodetails}

\end{document}

 

\documentclass[12pt]{llncs}
%\documentclass{jktr}

\usepackage[pdftex]{hyperref}                   
\usepackage {listings}
\usepackage {mathpartir}
\usepackage{bcprules}
%\usepackage{listings}
                       
\usepackage{graphicx} 
%\usepackage[margins=2.5cm,nohead,nofoot]{geometry}
%\usepackage{geometry}
\usepackage{amsfonts}
\usepackage{amstext}
\usepackage{latexsym}
\usepackage{amssymb}
\usepackage{color}


%\include{myPreamble}
\include{qm2pi.local} 

%\ifpdf
%\usepackage[pdftex]{graphicx}
%\else
%\usepackage{graphicx}
%\fi

 % \ifpdf
%  \usepackage{pdfsync}
%  \if


%\title{Brief Article}
%\author{David F. Snyder}
%\author{L.G. Meredith}

%\address{Dept. of Math., Texas State University--San Marcos, San Marcos, TX 78666}
       
\pagestyle{empty}


\begin{document}

\lstset{language=[Objective]Caml,frame=shadowbox}

\input{qm2pi.front}

% section front matter (end)

\input{qm2pi.intro} 
 
% section introduction (end)

% \input{qm2pi.knotations} 

% section notation (end)

\input{qm2pi.process.calculi} 

% section concurrent_process_calculi_and_spatial_logics_ (end)
    
%\input{qm2pi.knots2pi} 

%\input{qm2pi.trefoil} 

%\input{qm2pi.mainthm} 

% subsection basic_interpretation (end)

%\input{qm2pi.rho.presentation} 
\subsection{The syntax and semantics of the notation system}\label{sub:the_syntax_and_semantics_of_the_notation_system} % (fold)

We now summarize a technical presentation of the calculus that
embodies our theory of dynamics. The typical presentation of such a
calculus follows the style of giving generators and relations on
them. The grammar, below, describing term constructors, freely
generates the set of processes, $\Proc$. This set is then quotiented
by a relation known as structural congruence and it is over this set
that the notion of dynamics is expressed. This presentation is
essentially that of \cite{MeredithR05} with the addition of
polyadicity and summation. For readability we have relegated some of
the technical subtleties to an appendix.

\subsubsection{Process grammar}\label{subsub:process_grammar}

\begin{mathpar}
  \inferrule* [lab=synchronization] {} {{M} \bc \pzero \;|\; x?F \;|\; x!C }
  \and
  \inferrule* [lab=abstraction] {} {{F} \bc (x)P}
  \and
  \inferrule* [lab=concretion] {} {{C} \bc \langle Q \rangle}
  \and
  \inferrule* [lab=process] {} {{P,Q} \bc M \;| \;P|Q \;|\; @{x}}
  \and
  \inferrule* [lab=name] {} {{x} \bc \quotep{P}}
\end{mathpar} 

Note that $\vec{x}$ (resp. $\vec{P}$) denotes a vector of names
(resp. processes) of length $|\vec{x}|$ (resp. $|\vec{P}|$). We adopt
the following useful abbreviations.

\begin{mathpar}
   x?(\vec{y}).P := x.(\vec{y})P \and  x\clift{\vec{P}} := x.\clift{\vec{P}}
   \and x!(y) := \lift{x}{\dropn{y}}
   \and \Pi_{i=0}^{n-1}P_i := P_0 | \ldots | P_{n-1}
\end{mathpar}

\subsubsection{Structural congruence}

\paragraph{Free and bound names and alpha-equivalence.} At the
core of structural equivalence is alpha-equivalence which identifies
process that are the same up to a change of variable. Formally, we
recognize the distinction between free and bound names. The free names
of a process, $\freenames{P}$, may be calculated recursively as
follows:

\begin{mathpar}
\freenames{\pzero} := \emptyset
  \and \\
  \freenames{x?(y).P} := \{ x \} \cup (\freenames{P} \setminus \{ y \})
  \and 
  \freenames{x!\langle P \rangle} := \{ x \} \cup \{ P \} 
  \and \\
  \freenames{P|Q} := \freenames{P} \cup \freenames{Q}
  \and \\
  \freenames{@{x}} := \{ x \}
\end{mathpar}

$\pi$
$\quotep{\pi}$

$\freenames{-} : \pi \to \mathcal{P}(\quotep{\pi})$

\begin{eqnarray*}
  \freenames{\pzero} & := & \emptyset \\
  \freenames{x?(y).P} & := & \{ x \} \cup (\freenames{P} \setminus \{ y \}) \\
  \freenames{x!\langle P \rangle} & := & \{ x \} \cup \{ P \} \\
  \freenames{P|Q} & := & \freenames{P} \cup \freenames{Q} \\
  \freenames{\dropn{x}} & := & \{ x \}
\end{eqnarray*}

The bound names of a process, $\boundnames{P}$, are those names occurring in $P$
that are not free. For example, in $x?(y).0$, the name $x$ is free, while $y$ is bound.

\begin{mathpar}
  \inferrule* [lab=monoidal-laws] {} { P|Q \equiv Q|P \and P|0 \equiv P \and P|(Q|R) \equiv (P|Q)|R }
\end{mathpar}

\begin{mathpar}
  \inferrule* [lab=alpha-equivalence] {} { (x)P \equiv (y)P\{y/x\} \and y \not\in \freenames{P} }
\end{mathpar}

\begin{definition}
Then two processes, $P,Q$, are alpha-equivalent if $P = Q\{\vec{y}/\vec{x}\}$ for
some $\vec{x} \in \boundnames{Q},\vec{y} \in \boundnames{P}$, where $Q\{\vec{y}/\vec{x}\}$
denotes the capture-avoiding substitution of $\vec{y}$ for $\vec{x}$ in $Q$.
\end{definition}

\begin{definition}
  The {\em structural congruence} \cite{SangiorgiWalker} , $\equiv$,
  between processes is the least congruence containing
  alpha-equivalence, satisfying the abelian monoid laws
  (associativity, commutativity and $\pzero$ as identity) for parallel
  composition $|$ and for summation $+$.
\end{definition}

\subsection{Name equivalence}

We take name equivalence, written $\nameeq$, to be the smallest
equivalence relation generated by the following rules.

\begin{mathpar}
\inferrule*[lab=Quote-drop]
{ }
{ \quotep{@{x}} \nameeq x }

\inferrule*[lab=Struct-equiv]
{ P \scong Q }
{ \quotep{P} \nameeq \quotep{Q} }
\end{mathpar}

The astute reader will have noticed that the mutual recursion of names
and processes imposes a mutual recursion on alpha-equivalence and
structural equivalence via name-equivalence. Fortunately, all of this
works out pleasantly and we may calculate in the natural way, free of
concern. The reader interested in the details is referred to the
appendix \ref{appendix:rho_details}.

\subsection{Substitution}

We use $\Proc$ for the set of processes, $\QProc$ for the set of
names, and $\id{\{}\vec{y} / \vec{x} \id{\}}$ to denote partial maps,
$s : \QProc \rightarrow \QProc$. A map, $s$ lifts, uniquely, to a map
on process terms, $\widehat{s} : \Proc \rightarrow \Proc$ by the
following equations.

\begin{mathpar}
  (0) \psubstp{Q}{P} := 0 \\
  (R \juxtap S) \psubstp{Q}{P}
  :=    
  (R)\psubstp{Q}{P} \juxtap (S) \psubstp{Q}{P} \\
  (x?(y).R) \psubstp{Q}{P}    
  :=    
  (x)\substp{Q}{P} (z)\concat( (R \psubstn{z}{y}) \psubstp{Q}{P} ) \\
  (\lift{x}{R}) \psubstp{Q}{P}  
  :=
  \lift{(x)\substp{Q}{P}}{ R \psubstp{Q}{P} } \\
%   (\dropn{x})  \psubstp{Q}{P}       
%   := 
%   \left\{ 
%     \begin{array}{ccc} 
%       \dropn{\quotep{Q}} & & x \nameeq \quotep{P} \\
%       \dropn{x} & & otherwise \\
%     \end{array}
%   \right. 
  (\dropn{x})  \psubstp{Q}{P}       
  := 
  \left\{ 
    \begin{array}{ccc} 
      Q & & x \nameeq \quotep{P} \\
      \dropn{x} & & otherwise \\
    \end{array}
  \right.
\end{mathpar}
 

where

\begin{eqnarray}
  (x)\id{\{} \lpquote Q \rpquote / \lpquote P \rpquote \id{\}}            = 
  \left\{ 
    \begin{array}{ccc}
      \lpquote Q \rpquote & & x \nameeq \lpquote P \rpquote \\
      x & & otherwise \\
    \end{array}
  \right. \nonumber
\end{eqnarray}

and $z$ is chosen distinct from $\quotep{P}$, $\quotep{Q}$, the free
names in $Q$, and all the names in $R$. Our $\alpha$-equivalence will
be built in the standard way from this substitution.

\begin{remark}\label{rem:no_self_referential_names}
  One consequence of these definitions is that $\forall P. \quotep{P}
  \not\in \freenames{P}$.
\end{remark}

\subsection{ Dynamic quote: an example }

Anticipating something of what's to come, consider applying the
substitution, $\widehat{\id{\{}u / z \id{\}}}$, to the following pair
of processes, $\lift{w}{y!(z)}$ and $w[ \lpquote y!(z) \rpquote ]$.

\begin{eqnarray}
	\lift{w}{y!(z)}\widehat{\id{\{}u / z \id{\}}}
		& = &
		\lift{w}{y!(u)} \nonumber\\
	w[ \lpquote y!(z) \rpquote ] \widehat{ \id{\{}u / z \id{\}} }
		& = &
		w[ \lpquote y!(z) \rpquote ] \nonumber
\end{eqnarray}

Because the body of the process between quotes is impervious to
substitution, we get radically different answers. In fact, by
examining the first process in an input context,
e.g. $x?(z).\lift{w}{y!(z)}$, we see that the process under the lift
operator may be shaped by prefixed inputs binding a name inside it. In
this sense, the lift operator will be seen as a way to dynamically
construct processes before reifying them as names.

Finally equipped with these standard features we can present the
dynamics of the calculus.

\subsubsection{Operational semantics} 

Finally, we introduce the computational dynamics. What marks these
algebras as distinct from other more traditionally studied algebraic
structures, e.g. vector spaces or polynomial rings, is the manner in
which dynamics is captured. In traditional structures, dynamics is typically
expressed through morphisms between such structures, as in linear maps
between vector spaces or morphisms between rings. In algebras
associated with the semantics of computation, the dynamics is
expressed as part of the algebraic structure itself, through a
reduction reduction relation typically denoted by $\red$. Below, we
give a recursive presentation of this relation for the calculus used
in the encoding.

$\red \subseteq \pi \times \pi$
$\red : \pi \to \mathcal{P}(\pi)$

\begin{mathpar}
  \inferrule* [lab=Comm] { \textsf{match}( x_{src}, x_{trgt} ) } { x_{trgt}?(y)P \; | \; x_{src}!\langle {Q} \rangle \red P\{\quotep{Q}/y}\} }
  \and \\
  \inferrule* [lab=Par] {{P} \red {P}'} {{{P} | {Q}} \red {{P}' | {Q}}}
  \and
  \inferrule* [lab=Equiv]{{{P} \scong {P}'} \andalso {{P}' \red {Q}'} \andalso {{Q}' \scong {Q}}}{{P} \red {Q}}
\end{mathpar}

\begin{eqnarray*}
  match_{\equiv} (\quotep{P},\quotep{Q}) & := & P \equiv Q \\
  match_{\dagger}(\quotep{P},\quotep{Q}) & := & \forall R. P|Q \red^{*} R => R \red^{*} 0 \\
  match_{K}(\quotep{P},\quotep{Q}) & := & K \mbox{ for some context } K
\end{eqnarray*}

$u?(x)P | u!\langle Q \rangle \red P\{\quotep{Q}/x\}$

%We write $\wred$ for $\red^*$, and $P\red$ if $\exists Q $ such that $ P \red Q$.
We write $P\red$ if $\exists Q $ such that $ P \red Q$ and $P\not\red$, otherwise.

\section{Replication}

As mentioned before, it is known that replication (and hence
recursion) can be implemented in a higher-order process algebra
\cite{SangiorgiWalker}. As our first example of calculation with the
machinery thus far presented we give the construction explicitly in
the {\rhoc}.

\begin{eqnarray}
	D_{x} & := & \prefix{x}{y}{(\binpar{\outputp{x}{y}}{@{y}})} \nonumber\\
	\bangp_{x}{P} & := & \binpar{{x}!\langle{\binpar{D_{x}}{P}}\rangle}{D_{x}} \nonumber
\end{eqnarray}

\begin{eqnarray}
	\bangp_{x}{P} & & \nonumber\\
	=
	& {x}!\langle{(\prefix{x}{y}{(\outputp{x}{y} | @{y})) | P}}\rangle 
	      | \prefix{x}{y}{(\outputp{x}{y} | @{y})} & \nonumber\\
	\red
	& (\outputp{x}{y} | @{y})\substn{\quotep{(\prefix{x}{y}{(@{y} | \outputp{x}{y})) | P}}}{y} & \nonumber\\
	=
	& \outputp{x}{\quotep{(\prefix{x}{y}{(\outputp{x}{y} | @{y})) | P}}}
	  | {(\prefix{x}{y}{(\outputp{x}{y} | @{y})) | P}} & \nonumber\\
	\red
	& \ldots & \nonumber\\
	\red^*
	& P | P | \ldots & \nonumber
\end{eqnarray}

Of course, this encoding, as an implementation, runs away, unfolding
$\bangp{P}$ eagerly. A lazier and more implementable replication
operator, restricted to input-guarded processes, may be obtained as follows.

\begin{eqnarray}
\bangp{\prefix{u}{v}{P}} 
	:= 
	\binpar{\lift{x}{\prefix{u}{v}{(\binpar{D(x)}{P})}}}{D(x)} \nonumber
\end{eqnarray}

\begin{remark}
  Note that the lazier definition still does not deal with summation
  or mixed summation (i.e. sums over input and output). The reader is
  invited to construct definitions of replication that deal with these
  features. 

  Further, the definitions are parameterized in a name, $x$. Can you,
  gentle reader, make a definition that eliminates this parameter and
  guarantees no accidental interaction between the replication
  machinery and the process being replicated -- i.e. no accidental
  sharing of names used by the process to get its work done and the
  name(s) used by the replication to effect copying. This latter
  revision of the definition of replication is crucial to obtaining
  the expected identity $!!P \sim !P$.
\end{remark}

\begin{remark}\label{rem:paradoxical_combinator}
  The reader familiar with the lambda calculus will have noticed the
  similarity between $D$ and the paradoxical combinator.

  [Ed. note: the existence of this seems to suggest we have to be more
  restrictive on the set of processes and names we admit if we are to
  support no-cloning.]
\end{remark}

\subsubsection{Bisimulation}

The computational dynamics gives rise to another kind of equivalence,
the equivalence of computational behavior. As previously mentioned
this is typically captured \emph{via} some form of bisimulation.

% The notion we use in this paper is weak barbed bisimulation
% \cite{milner91polyadicpi}.

The notion we use in this paper is derived from weak barbed
bisimulation \cite{milner91polyadicpi}. 

\begin{definition}
An \emph{observation relation}, $\downarrow_{\mathcal N}$, over a set
of names, $\mathcal N$, is the smallest relation satisfying the rules
below.

\infrule[Out-barb]{y \in {\mathcal N}, \; x \nameeq y}
		  {\outputp{x}{v} \downarrow_{\mathcal N} x}
\infrule[Par-barb]{\mbox{$P\downarrow_{\mathcal N} x$ or $Q\downarrow_{\mathcal N} x$}}
		  {\binpar{P}{Q} \downarrow_{\mathcal N} x}

We write $P \Downarrow_{\mathcal N} x$ if there is $Q$ such that 
$P \wred Q$ and $Q \downarrow_{\mathcal N} x$.
\end{definition}

\begin{definition}
%\label{def.bbisim}
An  ${\mathcal N}$-\emph{barbed bisimulation} over a set of names, ${\mathcal N}$, is a symmetric binary relation 
${\mathcal S}_{\mathcal N}$ between agents such that $P\rel{S}_{\mathcal N}Q$ implies:
\begin{enumerate}
\item If $P \red P'$ then $Q \wred Q'$ and $P'\rel{S}_{\mathcal N} Q'$.
\item If $P\downarrow_{\mathcal N} x$, then $Q\Downarrow_{\mathcal N} x$.
\end{enumerate}
$P$ is ${\mathcal N}$-barbed bisimilar to $Q$, written
$P \wbbisim_{\mathcal N} Q$, if $P \rel{S}_{\mathcal N} Q$ for some ${\mathcal N}$-barbed bisimulation ${\mathcal S}_{\mathcal N}$.
\end{definition}

$\mathcal{R} \subseteq \pi \times \pi$

$P \mathcal{R} Q => \forall P'. P \red P' \Rightarrow \exists Q'. Q \red Q', P' \mathcal{R} Q'$

$P \vdash x \Rightarrow Q \vdash x$

\begin{mathpar}
  \inferrule*[lab=Out-barb]{x \nameeq y}{{y}!\langle{Q}\rangle \vdash x}
  \and
  \inferrule*[lab=Par-barb]{\mbox{$P\vdash x$ or $Q\vdash x$}}{\binpar{P}{Q} \vdash x}
\end{mathpar}

\subsubsection{Contexts}

One of the principle advantages of computational calculi like the
$\pi$-calculus is a well-defined notion of context,
contextual-equivalence and a correlation between
contextual-equivalence and notions of bisimulation. The notion of
context allows the decomposition of a process into (sub-)process and
its syntactic environment, its context. Thus, a context may be
thought of as a process with a ``hole'' (written $\Box$) in it. The
application of a context $M$ to a process $P$, written $M[P]$, is
tantamount to filling the hole in $M$ with $P$. In this paper we do
not need the full weight of this theory, but do make use of the notion
of context in the proof the main theorem. 

\begin{mathpar}
  \inferrule* [lab=summation] {} {{M_{M},M_{N}} \bc \Box \;|\; x.M_{A} \;|\; M_{M}+M_{N}}
  \and
  \inferrule* [lab=agent] {} {{M_{A}} \bc (\vec{x})M_{P} \;| \; \clift{P_0,\ldots,M_{P},\ldots,P_N}}
  \and \\
  \inferrule* [lab=process] {} {{M_{P}} \bc M_{N} \;| \;P|M_{P} }
\end{mathpar} 

\begin{mathpar}
  \inferrule* [lab=sychronization] {} {M_{N} \bc \Box \;|\; x?M_{F} \;|\; x!M_{C}}
  \and
  \inferrule* [lab=abstraction] {} {{M_{F}} \bc (x)M_{P} }
  \and
  \inferrule* [lab=concretion] {} {{M_{C}} \bc \langle M_{P} \rangle }
  \and \\
  \inferrule* [lab=process] {} {{M_{P}} \bc M_{N} \;| \;P|M_{P} }
\end{mathpar}

\begin{definition}[contextual application] Given a context $M$, and
  process $P$, we define the \emph{contextual application}, $M[P] :=
  M\{P/\Box\}$. That is, the contextual application of M to P is the
  substitution of $P$ for $\Box$ in $M$.
\end{definition}

$\meaningof{-} : L \to \mathcal{P}(\pi)$

\begin{mathpar}
  \inferrule* [lab=collection] {} {\meaningof{true} = \pi, \and \meaningof{~E} = \pi \setminus \meaningof{E}, \and \meaningof{E_{1} \& E_{2}} = \meaningof{E_{1}} \cap \meaningof{E_{2}}}
\end{mathpar}

\begin{mathpar}
  \inferrule* [lab=structure] {} {\meaningof{0} = \{ P \in \pi | P \equiv 0 \}, \and \\ \meaningof{E_1 | E_2} = \{ P \in \pi | P \equiv P_{1} | P_{2}, P_{1} \in \meaningof{E_{1}}, P_{2} \in \meaningof{E_2}\} }
\end{mathpar}

\begin{mathpar}
 \inferrule* [lab=behavior] {} {\meaningof{\langle a?b \rangle E} = \{ P \in \pi | P \equiv Q | u?(y)P', \\ \and \\\\ \and \\ \;\;\; u \in \meaningof{a}, \forall z.P'\{z/y\} \in \meaningof{E\{z/b\}}\}, \and \\ \meaningof{a!E} = \{ P \in \pi | P \equiv Q | x!\langle P' \rangle, x \in \meaningof{a} P' \in \meaningof{E}\} }
\end{mathpar}

\begin{mathpar}
 \inferrule* [lab=nominal] {} {\meaningof{\quotep{E}} = \{ \quotep{P} \in \quotep{\pi} | P \in \meaningof{E} \}, \and \meaningof{\quotep{P}} = \{ \quotep{Q} \in \quotep{\pi} | P \equiv Q \} \and \\ \meaningof{@\quotep{E}} = \{ P \in \pi | P \equiv @x, x \in \meaningof{E} \}}
\end{mathpar}

\begin{eqnarray*}
  \\
  \meaningof{-} : TS \to ST
\end{eqnarray*}

\begin{eqnarray*}
  \\
  L : TS \to ST
\end{eqnarray*}

\begin{eqnarray*}
  \\
  P \models E \iff P \in \meaningof{E}
\end{eqnarray*}

\begin{eqnarray*}
  P \approx_{L} Q \iff \forall E \in L. P \models E \iff Q \models E
\end{eqnarray*}

\begin{eqnarray*}
  P \approx_{K} Q
\end{eqnarray*}

\begin{eqnarray*}
  P \approx Q
\end{eqnarray*}

$\approx_{K} = \approx = \approx_{L}$

\subsubsection{Contextual duality}

Note that contexts extend the quotation operation to a family of
operations from processes to names. Given a context, $M$, we can
define a \emph{nominal context}, $\quotep{M}$ by $\quotep{M}[P] :=
\quotep{M[P]}$. To foreshadow what is to come we observe that these
operations enjoy a duality with processes very much like the duality
between vectors and maps from vectors to scalars.

Further, because the calculus is essentially higher-order, we have a
correspondence between contexts and processes. More specifically,
given a name $x$ and a context $M$ we can construct $M^{*}_{x}$ such
that 

\begin{mathpar}
  M^{*}_{x} | \lift{x}{P} \red M[P]
\end{mathpar}

namely,

\begin{mathpar}
  M^{*}_{x} := x?(u).M[\dropn{u}]
\end{mathpar}

The dependence of $M^{*}_{x}$ on a name makes it an abstraction, 

\begin{mathpar}
  M^{*} := (x)x?(u).M[\dropn{u}]
\end{mathpar}

\subsection{Additional notation}

It will sometimes be convenient to denote the process a name
quotes. We already have the notation $x = \quotep{P}$, but it will be
convenient to introduce an alternate notation, $\procn{x}$, when we
want to emphasize the connection to the use of the name. Note that, by
virtue of name equivalence, $\quotep{\procn{x}} \nameeq x$; so, the
notation is consistent with previous definitions.

Further, because names have structure it is possible to effect
substitutions on the basis of that structure. This means we need to
upgrade our notation for substitutions, which we accomplish by
adapting comprehension notation. Thus,

\begin{mathpar}
  P\{ y / x : x \in S \}
\end{mathpar}

is interpreted to mean the process derived from P by replacing (in a
capture-avoiding manner) each occurrence of $x$ in $S$ by $y$. For example,

\begin{mathpar}
  P\{ \quotep{\procn{x}|\procn{x}} / x : x \in \freenames{P} \}
\end{mathpar}

will replace each (occurrence) of a free name $x$ in $P$ by
$\quotep{\procn{x}|\procn{x}}$.

Also, we will avail ourselves of the notation $x^{L}$ and $x^{R}$ to
denote injections of a name into disjoint copies of the name
space. There are numerous ways to accomplish this. One example can be
found in \cite{MeredithR05}. This notation overloads to vectors of
names: $\vec{x}^{\pi} := (x_{i}^{\pi} \; : \; 0 \leq i < |\vec{x}| )$ where $\pi \in \{L,R\}$.

We also use $P^{\Box} := P|\Box$.

In \cite{MeredithR05} an interpretation of the new operator is
given. It turns out that there are several possible interpretations
all enjoying the requisite algebraic properties of the operator (see
\cite{milner91polyadicpi}). We will therefore make liberal use of
$(\nu\; \vec{x})P$.

% subsection the_syntax_and_semantics_of_the_notation_system (end)   

\input{qm2pi.qmops} 

\input{qm2pi.sterngerlach} 

\input{qm2pi.metric} 

% section concurrent_process_calculi (end)

%\input{qm2pi.proofsketch}

% section proof sketch (end)

%\input{qm2pi.slviaknots} 

% section spatial logic via knots (end)

\input{qm2pi.conclusion}

% section conclusion (end)

%\input{qm2pi.dtcodes} 

% section wiring algorithm (end)

\input{qm2pi.ack} 

% section acknowledgments (end)

\newpage


\bibliographystyle{plain}   
\bibliography{../../biblios/main.bib}

\input{qm2pi.rhodetails}

\end{document}

 

% section concurrent_process_calculi (end)

%\documentclass[12pt]{llncs}
%\documentclass{jktr}

\usepackage[pdftex]{hyperref}                   
\usepackage {listings}
\usepackage {mathpartir}
\usepackage{bcprules}
%\usepackage{listings}
                       
\usepackage{graphicx} 
%\usepackage[margins=2.5cm,nohead,nofoot]{geometry}
%\usepackage{geometry}
\usepackage{amsfonts}
\usepackage{amstext}
\usepackage{latexsym}
\usepackage{amssymb}
\usepackage{color}


%\include{myPreamble}
\include{qm2pi.local} 

%\ifpdf
%\usepackage[pdftex]{graphicx}
%\else
%\usepackage{graphicx}
%\fi

 % \ifpdf
%  \usepackage{pdfsync}
%  \if


%\title{Brief Article}
%\author{David F. Snyder}
%\author{L.G. Meredith}

%\address{Dept. of Math., Texas State University--San Marcos, San Marcos, TX 78666}
       
\pagestyle{empty}


\begin{document}

\lstset{language=[Objective]Caml,frame=shadowbox}

\input{qm2pi.front}

% section front matter (end)

\input{qm2pi.intro} 
 
% section introduction (end)

% \input{qm2pi.knotations} 

% section notation (end)

\input{qm2pi.process.calculi} 

% section concurrent_process_calculi_and_spatial_logics_ (end)
    
%\input{qm2pi.knots2pi} 

%\input{qm2pi.trefoil} 

%\input{qm2pi.mainthm} 

% subsection basic_interpretation (end)

%\input{qm2pi.rho.presentation} 
\subsection{The syntax and semantics of the notation system}\label{sub:the_syntax_and_semantics_of_the_notation_system} % (fold)

We now summarize a technical presentation of the calculus that
embodies our theory of dynamics. The typical presentation of such a
calculus follows the style of giving generators and relations on
them. The grammar, below, describing term constructors, freely
generates the set of processes, $\Proc$. This set is then quotiented
by a relation known as structural congruence and it is over this set
that the notion of dynamics is expressed. This presentation is
essentially that of \cite{MeredithR05} with the addition of
polyadicity and summation. For readability we have relegated some of
the technical subtleties to an appendix.

\subsubsection{Process grammar}\label{subsub:process_grammar}

\begin{mathpar}
  \inferrule* [lab=synchronization] {} {{M} \bc \pzero \;|\; x?F \;|\; x!C }
  \and
  \inferrule* [lab=abstraction] {} {{F} \bc (x)P}
  \and
  \inferrule* [lab=concretion] {} {{C} \bc \langle Q \rangle}
  \and
  \inferrule* [lab=process] {} {{P,Q} \bc M \;| \;P|Q \;|\; @{x}}
  \and
  \inferrule* [lab=name] {} {{x} \bc \quotep{P}}
\end{mathpar} 

Note that $\vec{x}$ (resp. $\vec{P}$) denotes a vector of names
(resp. processes) of length $|\vec{x}|$ (resp. $|\vec{P}|$). We adopt
the following useful abbreviations.

\begin{mathpar}
   x?(\vec{y}).P := x.(\vec{y})P \and  x\clift{\vec{P}} := x.\clift{\vec{P}}
   \and x!(y) := \lift{x}{\dropn{y}}
   \and \Pi_{i=0}^{n-1}P_i := P_0 | \ldots | P_{n-1}
\end{mathpar}

\subsubsection{Structural congruence}

\paragraph{Free and bound names and alpha-equivalence.} At the
core of structural equivalence is alpha-equivalence which identifies
process that are the same up to a change of variable. Formally, we
recognize the distinction between free and bound names. The free names
of a process, $\freenames{P}$, may be calculated recursively as
follows:

\begin{mathpar}
\freenames{\pzero} := \emptyset
  \and \\
  \freenames{x?(y).P} := \{ x \} \cup (\freenames{P} \setminus \{ y \})
  \and 
  \freenames{x!\langle P \rangle} := \{ x \} \cup \{ P \} 
  \and \\
  \freenames{P|Q} := \freenames{P} \cup \freenames{Q}
  \and \\
  \freenames{@{x}} := \{ x \}
\end{mathpar}

$\pi$
$\quotep{\pi}$

$\freenames{-} : \pi \to \mathcal{P}(\quotep{\pi})$

\begin{eqnarray*}
  \freenames{\pzero} & := & \emptyset \\
  \freenames{x?(y).P} & := & \{ x \} \cup (\freenames{P} \setminus \{ y \}) \\
  \freenames{x!\langle P \rangle} & := & \{ x \} \cup \{ P \} \\
  \freenames{P|Q} & := & \freenames{P} \cup \freenames{Q} \\
  \freenames{\dropn{x}} & := & \{ x \}
\end{eqnarray*}

The bound names of a process, $\boundnames{P}$, are those names occurring in $P$
that are not free. For example, in $x?(y).0$, the name $x$ is free, while $y$ is bound.

\begin{mathpar}
  \inferrule* [lab=monoidal-laws] {} { P|Q \equiv Q|P \and P|0 \equiv P \and P|(Q|R) \equiv (P|Q)|R }
\end{mathpar}

\begin{mathpar}
  \inferrule* [lab=alpha-equivalence] {} { (x)P \equiv (y)P\{y/x\} \and y \not\in \freenames{P} }
\end{mathpar}

\begin{definition}
Then two processes, $P,Q$, are alpha-equivalent if $P = Q\{\vec{y}/\vec{x}\}$ for
some $\vec{x} \in \boundnames{Q},\vec{y} \in \boundnames{P}$, where $Q\{\vec{y}/\vec{x}\}$
denotes the capture-avoiding substitution of $\vec{y}$ for $\vec{x}$ in $Q$.
\end{definition}

\begin{definition}
  The {\em structural congruence} \cite{SangiorgiWalker} , $\equiv$,
  between processes is the least congruence containing
  alpha-equivalence, satisfying the abelian monoid laws
  (associativity, commutativity and $\pzero$ as identity) for parallel
  composition $|$ and for summation $+$.
\end{definition}

\subsection{Name equivalence}

We take name equivalence, written $\nameeq$, to be the smallest
equivalence relation generated by the following rules.

\begin{mathpar}
\inferrule*[lab=Quote-drop]
{ }
{ \quotep{@{x}} \nameeq x }

\inferrule*[lab=Struct-equiv]
{ P \scong Q }
{ \quotep{P} \nameeq \quotep{Q} }
\end{mathpar}

The astute reader will have noticed that the mutual recursion of names
and processes imposes a mutual recursion on alpha-equivalence and
structural equivalence via name-equivalence. Fortunately, all of this
works out pleasantly and we may calculate in the natural way, free of
concern. The reader interested in the details is referred to the
appendix \ref{appendix:rho_details}.

\subsection{Substitution}

We use $\Proc$ for the set of processes, $\QProc$ for the set of
names, and $\id{\{}\vec{y} / \vec{x} \id{\}}$ to denote partial maps,
$s : \QProc \rightarrow \QProc$. A map, $s$ lifts, uniquely, to a map
on process terms, $\widehat{s} : \Proc \rightarrow \Proc$ by the
following equations.

\begin{mathpar}
  (0) \psubstp{Q}{P} := 0 \\
  (R \juxtap S) \psubstp{Q}{P}
  :=    
  (R)\psubstp{Q}{P} \juxtap (S) \psubstp{Q}{P} \\
  (x?(y).R) \psubstp{Q}{P}    
  :=    
  (x)\substp{Q}{P} (z)\concat( (R \psubstn{z}{y}) \psubstp{Q}{P} ) \\
  (\lift{x}{R}) \psubstp{Q}{P}  
  :=
  \lift{(x)\substp{Q}{P}}{ R \psubstp{Q}{P} } \\
%   (\dropn{x})  \psubstp{Q}{P}       
%   := 
%   \left\{ 
%     \begin{array}{ccc} 
%       \dropn{\quotep{Q}} & & x \nameeq \quotep{P} \\
%       \dropn{x} & & otherwise \\
%     \end{array}
%   \right. 
  (\dropn{x})  \psubstp{Q}{P}       
  := 
  \left\{ 
    \begin{array}{ccc} 
      Q & & x \nameeq \quotep{P} \\
      \dropn{x} & & otherwise \\
    \end{array}
  \right.
\end{mathpar}
 

where

\begin{eqnarray}
  (x)\id{\{} \lpquote Q \rpquote / \lpquote P \rpquote \id{\}}            = 
  \left\{ 
    \begin{array}{ccc}
      \lpquote Q \rpquote & & x \nameeq \lpquote P \rpquote \\
      x & & otherwise \\
    \end{array}
  \right. \nonumber
\end{eqnarray}

and $z$ is chosen distinct from $\quotep{P}$, $\quotep{Q}$, the free
names in $Q$, and all the names in $R$. Our $\alpha$-equivalence will
be built in the standard way from this substitution.

\begin{remark}\label{rem:no_self_referential_names}
  One consequence of these definitions is that $\forall P. \quotep{P}
  \not\in \freenames{P}$.
\end{remark}

\subsection{ Dynamic quote: an example }

Anticipating something of what's to come, consider applying the
substitution, $\widehat{\id{\{}u / z \id{\}}}$, to the following pair
of processes, $\lift{w}{y!(z)}$ and $w[ \lpquote y!(z) \rpquote ]$.

\begin{eqnarray}
	\lift{w}{y!(z)}\widehat{\id{\{}u / z \id{\}}}
		& = &
		\lift{w}{y!(u)} \nonumber\\
	w[ \lpquote y!(z) \rpquote ] \widehat{ \id{\{}u / z \id{\}} }
		& = &
		w[ \lpquote y!(z) \rpquote ] \nonumber
\end{eqnarray}

Because the body of the process between quotes is impervious to
substitution, we get radically different answers. In fact, by
examining the first process in an input context,
e.g. $x?(z).\lift{w}{y!(z)}$, we see that the process under the lift
operator may be shaped by prefixed inputs binding a name inside it. In
this sense, the lift operator will be seen as a way to dynamically
construct processes before reifying them as names.

Finally equipped with these standard features we can present the
dynamics of the calculus.

\subsubsection{Operational semantics} 

Finally, we introduce the computational dynamics. What marks these
algebras as distinct from other more traditionally studied algebraic
structures, e.g. vector spaces or polynomial rings, is the manner in
which dynamics is captured. In traditional structures, dynamics is typically
expressed through morphisms between such structures, as in linear maps
between vector spaces or morphisms between rings. In algebras
associated with the semantics of computation, the dynamics is
expressed as part of the algebraic structure itself, through a
reduction reduction relation typically denoted by $\red$. Below, we
give a recursive presentation of this relation for the calculus used
in the encoding.

$\red \subseteq \pi \times \pi$
$\red : \pi \to \mathcal{P}(\pi)$

\begin{mathpar}
  \inferrule* [lab=Comm] { \textsf{match}( x_{src}, x_{trgt} ) } { x_{trgt}?(y)P \; | \; x_{src}!\langle {Q} \rangle \red P\{\quotep{Q}/y}\} }
  \and \\
  \inferrule* [lab=Par] {{P} \red {P}'} {{{P} | {Q}} \red {{P}' | {Q}}}
  \and
  \inferrule* [lab=Equiv]{{{P} \scong {P}'} \andalso {{P}' \red {Q}'} \andalso {{Q}' \scong {Q}}}{{P} \red {Q}}
\end{mathpar}

\begin{eqnarray*}
  match_{\equiv} (\quotep{P},\quotep{Q}) & := & P \equiv Q \\
  match_{\dagger}(\quotep{P},\quotep{Q}) & := & \forall R. P|Q \red^{*} R => R \red^{*} 0 \\
  match_{K}(\quotep{P},\quotep{Q}) & := & K \mbox{ for some context } K
\end{eqnarray*}

$u?(x)P | u!\langle Q \rangle \red P\{\quotep{Q}/x\}$

%We write $\wred$ for $\red^*$, and $P\red$ if $\exists Q $ such that $ P \red Q$.
We write $P\red$ if $\exists Q $ such that $ P \red Q$ and $P\not\red$, otherwise.

\section{Replication}

As mentioned before, it is known that replication (and hence
recursion) can be implemented in a higher-order process algebra
\cite{SangiorgiWalker}. As our first example of calculation with the
machinery thus far presented we give the construction explicitly in
the {\rhoc}.

\begin{eqnarray}
	D_{x} & := & \prefix{x}{y}{(\binpar{\outputp{x}{y}}{@{y}})} \nonumber\\
	\bangp_{x}{P} & := & \binpar{{x}!\langle{\binpar{D_{x}}{P}}\rangle}{D_{x}} \nonumber
\end{eqnarray}

\begin{eqnarray}
	\bangp_{x}{P} & & \nonumber\\
	=
	& {x}!\langle{(\prefix{x}{y}{(\outputp{x}{y} | @{y})) | P}}\rangle 
	      | \prefix{x}{y}{(\outputp{x}{y} | @{y})} & \nonumber\\
	\red
	& (\outputp{x}{y} | @{y})\substn{\quotep{(\prefix{x}{y}{(@{y} | \outputp{x}{y})) | P}}}{y} & \nonumber\\
	=
	& \outputp{x}{\quotep{(\prefix{x}{y}{(\outputp{x}{y} | @{y})) | P}}}
	  | {(\prefix{x}{y}{(\outputp{x}{y} | @{y})) | P}} & \nonumber\\
	\red
	& \ldots & \nonumber\\
	\red^*
	& P | P | \ldots & \nonumber
\end{eqnarray}

Of course, this encoding, as an implementation, runs away, unfolding
$\bangp{P}$ eagerly. A lazier and more implementable replication
operator, restricted to input-guarded processes, may be obtained as follows.

\begin{eqnarray}
\bangp{\prefix{u}{v}{P}} 
	:= 
	\binpar{\lift{x}{\prefix{u}{v}{(\binpar{D(x)}{P})}}}{D(x)} \nonumber
\end{eqnarray}

\begin{remark}
  Note that the lazier definition still does not deal with summation
  or mixed summation (i.e. sums over input and output). The reader is
  invited to construct definitions of replication that deal with these
  features. 

  Further, the definitions are parameterized in a name, $x$. Can you,
  gentle reader, make a definition that eliminates this parameter and
  guarantees no accidental interaction between the replication
  machinery and the process being replicated -- i.e. no accidental
  sharing of names used by the process to get its work done and the
  name(s) used by the replication to effect copying. This latter
  revision of the definition of replication is crucial to obtaining
  the expected identity $!!P \sim !P$.
\end{remark}

\begin{remark}\label{rem:paradoxical_combinator}
  The reader familiar with the lambda calculus will have noticed the
  similarity between $D$ and the paradoxical combinator.

  [Ed. note: the existence of this seems to suggest we have to be more
  restrictive on the set of processes and names we admit if we are to
  support no-cloning.]
\end{remark}

\subsubsection{Bisimulation}

The computational dynamics gives rise to another kind of equivalence,
the equivalence of computational behavior. As previously mentioned
this is typically captured \emph{via} some form of bisimulation.

% The notion we use in this paper is weak barbed bisimulation
% \cite{milner91polyadicpi}.

The notion we use in this paper is derived from weak barbed
bisimulation \cite{milner91polyadicpi}. 

\begin{definition}
An \emph{observation relation}, $\downarrow_{\mathcal N}$, over a set
of names, $\mathcal N$, is the smallest relation satisfying the rules
below.

\infrule[Out-barb]{y \in {\mathcal N}, \; x \nameeq y}
		  {\outputp{x}{v} \downarrow_{\mathcal N} x}
\infrule[Par-barb]{\mbox{$P\downarrow_{\mathcal N} x$ or $Q\downarrow_{\mathcal N} x$}}
		  {\binpar{P}{Q} \downarrow_{\mathcal N} x}

We write $P \Downarrow_{\mathcal N} x$ if there is $Q$ such that 
$P \wred Q$ and $Q \downarrow_{\mathcal N} x$.
\end{definition}

\begin{definition}
%\label{def.bbisim}
An  ${\mathcal N}$-\emph{barbed bisimulation} over a set of names, ${\mathcal N}$, is a symmetric binary relation 
${\mathcal S}_{\mathcal N}$ between agents such that $P\rel{S}_{\mathcal N}Q$ implies:
\begin{enumerate}
\item If $P \red P'$ then $Q \wred Q'$ and $P'\rel{S}_{\mathcal N} Q'$.
\item If $P\downarrow_{\mathcal N} x$, then $Q\Downarrow_{\mathcal N} x$.
\end{enumerate}
$P$ is ${\mathcal N}$-barbed bisimilar to $Q$, written
$P \wbbisim_{\mathcal N} Q$, if $P \rel{S}_{\mathcal N} Q$ for some ${\mathcal N}$-barbed bisimulation ${\mathcal S}_{\mathcal N}$.
\end{definition}

$\mathcal{R} \subseteq \pi \times \pi$

$P \mathcal{R} Q => \forall P'. P \red P' \Rightarrow \exists Q'. Q \red Q', P' \mathcal{R} Q'$

$P \vdash x \Rightarrow Q \vdash x$

\begin{mathpar}
  \inferrule*[lab=Out-barb]{x \nameeq y}{{y}!\langle{Q}\rangle \vdash x}
  \and
  \inferrule*[lab=Par-barb]{\mbox{$P\vdash x$ or $Q\vdash x$}}{\binpar{P}{Q} \vdash x}
\end{mathpar}

\subsubsection{Contexts}

One of the principle advantages of computational calculi like the
$\pi$-calculus is a well-defined notion of context,
contextual-equivalence and a correlation between
contextual-equivalence and notions of bisimulation. The notion of
context allows the decomposition of a process into (sub-)process and
its syntactic environment, its context. Thus, a context may be
thought of as a process with a ``hole'' (written $\Box$) in it. The
application of a context $M$ to a process $P$, written $M[P]$, is
tantamount to filling the hole in $M$ with $P$. In this paper we do
not need the full weight of this theory, but do make use of the notion
of context in the proof the main theorem. 

\begin{mathpar}
  \inferrule* [lab=summation] {} {{M_{M},M_{N}} \bc \Box \;|\; x.M_{A} \;|\; M_{M}+M_{N}}
  \and
  \inferrule* [lab=agent] {} {{M_{A}} \bc (\vec{x})M_{P} \;| \; \clift{P_0,\ldots,M_{P},\ldots,P_N}}
  \and \\
  \inferrule* [lab=process] {} {{M_{P}} \bc M_{N} \;| \;P|M_{P} }
\end{mathpar} 

\begin{mathpar}
  \inferrule* [lab=sychronization] {} {M_{N} \bc \Box \;|\; x?M_{F} \;|\; x!M_{C}}
  \and
  \inferrule* [lab=abstraction] {} {{M_{F}} \bc (x)M_{P} }
  \and
  \inferrule* [lab=concretion] {} {{M_{C}} \bc \langle M_{P} \rangle }
  \and \\
  \inferrule* [lab=process] {} {{M_{P}} \bc M_{N} \;| \;P|M_{P} }
\end{mathpar}

\begin{definition}[contextual application] Given a context $M$, and
  process $P$, we define the \emph{contextual application}, $M[P] :=
  M\{P/\Box\}$. That is, the contextual application of M to P is the
  substitution of $P$ for $\Box$ in $M$.
\end{definition}

$\meaningof{-} : L \to \mathcal{P}(\pi)$

\begin{mathpar}
  \inferrule* [lab=collection] {} {\meaningof{true} = \pi, \and \meaningof{~E} = \pi \setminus \meaningof{E}, \and \meaningof{E_{1} \& E_{2}} = \meaningof{E_{1}} \cap \meaningof{E_{2}}}
\end{mathpar}

\begin{mathpar}
  \inferrule* [lab=structure] {} {\meaningof{0} = \{ P \in \pi | P \equiv 0 \}, \and \\ \meaningof{E_1 | E_2} = \{ P \in \pi | P \equiv P_{1} | P_{2}, P_{1} \in \meaningof{E_{1}}, P_{2} \in \meaningof{E_2}\} }
\end{mathpar}

\begin{mathpar}
 \inferrule* [lab=behavior] {} {\meaningof{\langle a?b \rangle E} = \{ P \in \pi | P \equiv Q | u?(y)P', \\ \and \\\\ \and \\ \;\;\; u \in \meaningof{a}, \forall z.P'\{z/y\} \in \meaningof{E\{z/b\}}\}, \and \\ \meaningof{a!E} = \{ P \in \pi | P \equiv Q | x!\langle P' \rangle, x \in \meaningof{a} P' \in \meaningof{E}\} }
\end{mathpar}

\begin{mathpar}
 \inferrule* [lab=nominal] {} {\meaningof{\quotep{E}} = \{ \quotep{P} \in \quotep{\pi} | P \in \meaningof{E} \}, \and \meaningof{\quotep{P}} = \{ \quotep{Q} \in \quotep{\pi} | P \equiv Q \} \and \\ \meaningof{@\quotep{E}} = \{ P \in \pi | P \equiv @x, x \in \meaningof{E} \}}
\end{mathpar}

\begin{eqnarray*}
  \\
  \meaningof{-} : TS \to ST
\end{eqnarray*}

\begin{eqnarray*}
  \\
  L : TS \to ST
\end{eqnarray*}

\begin{eqnarray*}
  \\
  P \models E \iff P \in \meaningof{E}
\end{eqnarray*}

\begin{eqnarray*}
  P \approx_{L} Q \iff \forall E \in L. P \models E \iff Q \models E
\end{eqnarray*}

\begin{eqnarray*}
  P \approx_{K} Q
\end{eqnarray*}

\begin{eqnarray*}
  P \approx Q
\end{eqnarray*}

$\approx_{K} = \approx = \approx_{L}$

\subsubsection{Contextual duality}

Note that contexts extend the quotation operation to a family of
operations from processes to names. Given a context, $M$, we can
define a \emph{nominal context}, $\quotep{M}$ by $\quotep{M}[P] :=
\quotep{M[P]}$. To foreshadow what is to come we observe that these
operations enjoy a duality with processes very much like the duality
between vectors and maps from vectors to scalars.

Further, because the calculus is essentially higher-order, we have a
correspondence between contexts and processes. More specifically,
given a name $x$ and a context $M$ we can construct $M^{*}_{x}$ such
that 

\begin{mathpar}
  M^{*}_{x} | \lift{x}{P} \red M[P]
\end{mathpar}

namely,

\begin{mathpar}
  M^{*}_{x} := x?(u).M[\dropn{u}]
\end{mathpar}

The dependence of $M^{*}_{x}$ on a name makes it an abstraction, 

\begin{mathpar}
  M^{*} := (x)x?(u).M[\dropn{u}]
\end{mathpar}

\subsection{Additional notation}

It will sometimes be convenient to denote the process a name
quotes. We already have the notation $x = \quotep{P}$, but it will be
convenient to introduce an alternate notation, $\procn{x}$, when we
want to emphasize the connection to the use of the name. Note that, by
virtue of name equivalence, $\quotep{\procn{x}} \nameeq x$; so, the
notation is consistent with previous definitions.

Further, because names have structure it is possible to effect
substitutions on the basis of that structure. This means we need to
upgrade our notation for substitutions, which we accomplish by
adapting comprehension notation. Thus,

\begin{mathpar}
  P\{ y / x : x \in S \}
\end{mathpar}

is interpreted to mean the process derived from P by replacing (in a
capture-avoiding manner) each occurrence of $x$ in $S$ by $y$. For example,

\begin{mathpar}
  P\{ \quotep{\procn{x}|\procn{x}} / x : x \in \freenames{P} \}
\end{mathpar}

will replace each (occurrence) of a free name $x$ in $P$ by
$\quotep{\procn{x}|\procn{x}}$.

Also, we will avail ourselves of the notation $x^{L}$ and $x^{R}$ to
denote injections of a name into disjoint copies of the name
space. There are numerous ways to accomplish this. One example can be
found in \cite{MeredithR05}. This notation overloads to vectors of
names: $\vec{x}^{\pi} := (x_{i}^{\pi} \; : \; 0 \leq i < |\vec{x}| )$ where $\pi \in \{L,R\}$.

We also use $P^{\Box} := P|\Box$.

In \cite{MeredithR05} an interpretation of the new operator is
given. It turns out that there are several possible interpretations
all enjoying the requisite algebraic properties of the operator (see
\cite{milner91polyadicpi}). We will therefore make liberal use of
$(\nu\; \vec{x})P$.

% subsection the_syntax_and_semantics_of_the_notation_system (end)   

\input{qm2pi.qmops} 

\input{qm2pi.sterngerlach} 

\input{qm2pi.metric} 

% section concurrent_process_calculi (end)

%\input{qm2pi.proofsketch}

% section proof sketch (end)

%\input{qm2pi.slviaknots} 

% section spatial logic via knots (end)

\input{qm2pi.conclusion}

% section conclusion (end)

%\input{qm2pi.dtcodes} 

% section wiring algorithm (end)

\input{qm2pi.ack} 

% section acknowledgments (end)

\newpage


\bibliographystyle{plain}   
\bibliography{../../biblios/main.bib}

\input{qm2pi.rhodetails}

\end{document}



% section proof sketch (end)

%\section{Unlikely characters: spatial logic for
  knots}\label{sub:characteristic_formulae} % (fold)

Associated to the mobile process calculi are a family of logics known
as the Hennessy-Milner logics. These logics typically enjoy a
semantics interpreting formulae as sets of processes that when
factored through the encoding outlined above allows an identification
of classes of knots with logical formulae. In the context of this
encoding the sub-family known as the spatial logics \cite{CairesC03}
\cite{CairesC04} \cite{Caires04} are of particular interest providing
several important features for expressing and reasoning about
properties (i.e. classes) of knots. We hint here at how this may be done.

%\begin{description}
%\item [structural connectives] 
\subsubsection{Structural connectives} The spatial logics enjoy
structural connectives corresponding, at the logical level, to the
parallel composition ($P | Q$) and new name ($(\nu \; x)P$)
connectives for processes. As illustrated in the examples below, these
connectives are extremely expressive given the shape of our encoding.
%\item [decideable satisfaction]

\subsubsection{Decideable satisfaction}
In \cite{Caires04} the satisfaction relation is shown to be decideable
for a rich class of processes. It further turns out that the image of
the our encoding is a proper subset of that class. This result
provides the basis for an algorithm by which to search for knots
enjoying a given property.
%\item [characteristic formulae]

\subsubsection{Characteristic formulae}
In the same paper \cite{Caires04} , Caires presents a means of calculating
characteristic formulae, selecting equivalence classes of processes
up to a pre--specified depth limit on the support set of names. Composed with our
encoding, this characteristic formula can be used to select
characteristic formulae for knots.
%\end{description}

\subsubsection{Spatial logic formulae}

The grammar below (segmented for comprehension) summarizes the syntax
of spatial logic formulae. We employ illustrative examples in the
sequel to provide an intuitive understanding of their meaning
referring the reader to \cite{Caires04} for a more detailed explication
of the semantics.

\begin{mathpar}
  \inferrule* [lab=boolean] {} {{A,B} \bc T \;|\; \neg A \;|\; A \wedge B \;|\; \eta = \eta'}
  \and
  \inferrule* [lab=spatial] {} {|\; \pzero \;|\; A | B \;|\; x \text{\textregistered} A \;|\; \forall x . A \;|\;  H x . A}
  \and
  \inferrule* [lab=behavioral] {} {|\; \alpha . A}
  \and 
  \inferrule* [lab=recursion] {} {|\; X(\vec{u}) \;|\; \mu X(\vec{u}) . A}
  \and
  \inferrule* [lab=action] {} {\alpha \bc \langle x?(\vec{y}) \rangle \;|\; \langle x!(\vec{y}) \rangle \;|\; \langle \tau \rangle}
  \and 
  \inferrule* [lab=name] {} {\eta \bc x \;|\; \tau}
\end{mathpar} 

% subsection characteristic_formulae (end)   	 

\subsection{Example formulae}\label{sub:example_formulae_} % (fold)

\subsubsection{Crossing as formula.}
% 
% \begin{align*}
%   \frac{d}{dx} \sin x &= \cos x 
%   & \frac{d}{dx} e^x &= e^x \\
%   \frac{d}{dx} \cos x &= - \sin x 
%   & \frac{d}{dx} \log x &= \frac{1}{x} \\
% \end{align*} 

\begin{align*}
 \mu C(x_{0},x_{1},y_{0},y_{1},u).&(\langle x_{0}?(z) \rangle(\langle u! \rangle\langle y_{1}!z \rangle C(x_{0},x_{1},y_{0},y_{1},u)) & \\
  & \wedge \langle y_{1}?(z) \rangle (\langle u! \rangle \langle x_{0}!z \rangle C(x_{0},x_{1},y_{0},y_{1},u)) & \\
  & \wedge \langle x_{1}?(z) \rangle (\langle u? \rangle \langle y_{0}!z \rangle C(x_{0},x_{1},y_{0},y_{1},u)) & \\
  & \wedge \langle y_{0}?(z) \rangle (\langle u? \rangle \langle x_{1}!z \rangle C(x_{0},x_{1},y_{0},y_{1},u))) &
\end{align*}

The lexicographical similarity between the shape of this formulae and
the shape of definition of the process representing a crossing reveals
the intuitive meaning of this formulae. It describes the capabilities
of a process that has the right to represent a crossing. For example
it picks out processes that may perform an input on the port $x_0$ in
its initial menu of capabilities. What differentiates the formula
from the process, however, is that the crossing process is the
smallest candidate to satisfy the formula. Infinitely many other
processes -- with internal behavior hidden behind this interface, so
to speak -- also satisfy this formula. Even this simple formula,
then, can be seen to open a new view onto knots, providing a
computational interpretation of \emph{virtual} knots.

Note that this formula is derived by hand. A similar formula can be
derived by employing Caires' calculation of characteristic formula
\cite{Caires04} to the process representing a crossing. In light of
this discussion, we let
$\meaningof{C}_{\phi}(x0,x1,y0,y1,u)$ denote a formula specifying the
dynamics we wish to capture of a crossing. To guarantee we preserve
the shape of the interface and minimal semantics we demand that
$\meaningof{C}_{\phi}(x0,x1,y0,y1,u) \Rightarrow
\textbf{C}(x0,x1,y0,y1,u)$ where $\textbf{C}(x0,x1,y0,y1,u)$ denotes
the formula above.
                            
\subsubsection{Crossing number constraints.}
The moral content of the context lemma (Lemma \ref{context}) is that the notion of
``locality'' in the Reidemeister moves is effectively captured by the
parallel composition operator of the process calculus. This intuition
extends through the logic. Given a formula,
$\meaningof{C}_{\phi}(x0,x1,y0,y1,u)$, we can use the structural
connectives to specify constraints on crossing numbers, such as at
least $n$ crossings, or exactly $n$ crossings.
\begin{mathpar}
  \inferrule* [lab=at-least-n] {} { K^{\geq n}_{\phi}(\vec{xs},\vec{ys}) := \Pi_{i=0}^{n-1} Hu . \meaningof{C}_{\phi}(xs_i,ys_i,u) | T }
  \and 
  \inferrule* [lab=exactly-n] {} { K^{= n}_{\phi}(\vec{xs},\vec{ys}) := \Pi_{i=0}^{n-1} Hu . \meaningof{C}_{\phi}(xs_i,ys_i,u) | \neg (\forall x_0,y_0,x_1,y_1,u . \meaningof{C}_{\phi}(x_0,y_0,x_1,y_1,u) | T) }
\end{mathpar}

To round out this section, recall that the encoding of an $n$-crossing
knot decomposes into a parallel composition of $n$ \emph{copies} of a
crossing process together with a wiring harness. To specify different
knot classes with the same crossing number amounts to specifying
logical constraints on the wiring harness. In the interest of space,
we defer examples to a forthcoming paper. Suffice it to say that both
the conditions ``alternating knot'' and ``contains the tangle
corresponding to 5/3'' are expressible. For example, it is possible to
calculate the characteristic formula of a process corresponding to the
tangle 5/3 and conjoin it into the classifying formula via the
composition connective of the logic.

Finally, we wish to observe that it is entirely within reason to
contemplate a more domain-specific version of spatial logic tailored
to the shape of processes in the image of the encoding. Such a
domain-specific logic would have a better claim to the title formal
language of knot properties.

% subsection example_formulae_ (end)

% section knots_as_processes (end) 

% section spatial logic via knots (end)

\section{Conclusions and future work}

\paragraph{Testing physical space}
You, gentle reader, may wonder why of all the theorems to be proved
given this set up we pick the one above. In some sense it's hardly
central to quantum mechanics. We see it as central in the sense that
it firmly establishes a notion of physical space arising from a notion
of the equivalence of behavior. Relating bisimulation to a metric is a
big step forward, but one is faced with interpreting the relationship
of that metric space to something more physical. Quantum mechanical
notions of ``physical'' space are still far from intuitive, but by
relating this idea of distance as testing to calculations that predict
physical circumstances we are making a not insignificant step forward
toward an understanding of the physical space we inhabit as
essentially dynamic.

\paragraph{Effectivity and simulation}
One of the observations we have yet to make is that the entire program
spelled out here is effective. We have built various interpreters for
the reflective calculus at work in this interpretation. In principle,
then, we can simulate quantum mechanics on a computer. The place where
the simulation may lose fidelity is the infinitely branching summation
for the annihilator.

In this connection i also want to point out that the evaluation style
calculation of the inner product puts the non-determinism of the
summation right at the heart of measurement. This suggests that
Milner's original reduction-based formulation of the dynamics of his
calculi in terms of sums was not just notationally suggestive of a
notion of measure-and-continue but captured some significant part of
the physics.

\paragraph{Quantum continuations}
In light of this last observation i want to point out that the
predominant account of quantum mechanics is missing a key aspect of a
truly compositional story of the physical situation. In a real lab,
when a measurement is made the observation can be made to feed into
another device that then makes another measurement conditioned on the
results of the first. This means that after the superposition was
collapsed the entire experimental set up remained in
superposition. While QM offers a means of writing this down it doesn't
quite line up well with the well-trodden formulation of computation
and continuation that we see so succinctly expressed in Milner's
calculi. This suggests that there might be advantages to this account
of dynamics waiting to be explored.

\paragraph{Quantum logic}
In this connection, we also note that by virtue of having the
Hennessy-Milner construction, we can pull the construction through the
interpretation of QM. This gives us a natural candidate for a quantum
logic that enjoys an extremely tight connection with it's domain of
interpretation, making the construction much less ad hoc (rather it is
the image of functor!).

\paragraph{Quantum probabiity}
i have questions about the basis of the interpretation of inner
product as probability amplitude. In particular, using which
axiomatization of probability theory does the notion of probability
amplitude earn the right to be so dubbed? In other words, where is the
proof that the operation for calculating a probability amplitude (and
then squaring) satisfies the axioms of what it means to calculate a
probability? Even if such a proof exists (i have yet to find it in the
literature), i wonder if it might not be possible to turn things on
their heads. Can we view the calculation of the probability amplitude
as an axiomatization of probability? If so, then the definition we
give for calculating probability amplitude may provide the basis for
an \emph{effective} theory of probability.

\paragraph{Quantum vs ``biological'' information}
Finally, i want to conclude with a more philosophical observation. At
a recent workshop in which QM was a predominant topic i noticed
something about quantum information. The speaker was giving a riveting
discussion of axiomatic QM and showing how properties of ``no
cloning'' and ``no deleting'' emerged as consequences of the
axiomatization. Theorems of this form are necessary to give us a sense
of confidence that our axioms characterize the physical theory. What
struck me, though, was that if quantum information is neither erasable
nor replicable it is markedly different from \emph{life}. Two of the
things we know about life is that

\begin{itemize}
  \item it ends;
  \item to gain some measure of persistence, to transcend it's
    finitude it is imminently copyable.
\end{itemize}

Both of these qualities are summarized succinctly in the aphorism: all
flesh is grass. For me these two kinds of ``information'' -- call them
quantum and biological -- are end points on a spectrum of strategies
for persistence. At one end, we have those curious entities that enjoy
uniqueness and permanence; at the other, we have those who in the face
of a certain end and an uncertain present make a go of passing
something on. To me one of the more remarkable aspects of the latter
strategy is that in the presence of noise (and certain features of
copying) we get a kind of dynamism, a chance for improvement against a
given persistent condition.

% subsection other_calculi_other_bisimulations_and_geometry_as_behavior (end)




% section conclusion (end)

%\documentclass[12pt]{llncs}
%\documentclass{jktr}

\usepackage[pdftex]{hyperref}                   
\usepackage {listings}
\usepackage {mathpartir}
\usepackage{bcprules}
%\usepackage{listings}
                       
\usepackage{graphicx} 
%\usepackage[margins=2.5cm,nohead,nofoot]{geometry}
%\usepackage{geometry}
\usepackage{amsfonts}
\usepackage{amstext}
\usepackage{latexsym}
\usepackage{amssymb}
\usepackage{color}


%\include{myPreamble}
\include{qm2pi.local} 

%\ifpdf
%\usepackage[pdftex]{graphicx}
%\else
%\usepackage{graphicx}
%\fi

 % \ifpdf
%  \usepackage{pdfsync}
%  \if


%\title{Brief Article}
%\author{David F. Snyder}
%\author{L.G. Meredith}

%\address{Dept. of Math., Texas State University--San Marcos, San Marcos, TX 78666}
       
\pagestyle{empty}


\begin{document}

\lstset{language=[Objective]Caml,frame=shadowbox}

\input{qm2pi.front}

% section front matter (end)

\input{qm2pi.intro} 
 
% section introduction (end)

% \input{qm2pi.knotations} 

% section notation (end)

\input{qm2pi.process.calculi} 

% section concurrent_process_calculi_and_spatial_logics_ (end)
    
%\input{qm2pi.knots2pi} 

%\input{qm2pi.trefoil} 

%\input{qm2pi.mainthm} 

% subsection basic_interpretation (end)

%\input{qm2pi.rho.presentation} 
\subsection{The syntax and semantics of the notation system}\label{sub:the_syntax_and_semantics_of_the_notation_system} % (fold)

We now summarize a technical presentation of the calculus that
embodies our theory of dynamics. The typical presentation of such a
calculus follows the style of giving generators and relations on
them. The grammar, below, describing term constructors, freely
generates the set of processes, $\Proc$. This set is then quotiented
by a relation known as structural congruence and it is over this set
that the notion of dynamics is expressed. This presentation is
essentially that of \cite{MeredithR05} with the addition of
polyadicity and summation. For readability we have relegated some of
the technical subtleties to an appendix.

\subsubsection{Process grammar}\label{subsub:process_grammar}

\begin{mathpar}
  \inferrule* [lab=synchronization] {} {{M} \bc \pzero \;|\; x?F \;|\; x!C }
  \and
  \inferrule* [lab=abstraction] {} {{F} \bc (x)P}
  \and
  \inferrule* [lab=concretion] {} {{C} \bc \langle Q \rangle}
  \and
  \inferrule* [lab=process] {} {{P,Q} \bc M \;| \;P|Q \;|\; @{x}}
  \and
  \inferrule* [lab=name] {} {{x} \bc \quotep{P}}
\end{mathpar} 

Note that $\vec{x}$ (resp. $\vec{P}$) denotes a vector of names
(resp. processes) of length $|\vec{x}|$ (resp. $|\vec{P}|$). We adopt
the following useful abbreviations.

\begin{mathpar}
   x?(\vec{y}).P := x.(\vec{y})P \and  x\clift{\vec{P}} := x.\clift{\vec{P}}
   \and x!(y) := \lift{x}{\dropn{y}}
   \and \Pi_{i=0}^{n-1}P_i := P_0 | \ldots | P_{n-1}
\end{mathpar}

\subsubsection{Structural congruence}

\paragraph{Free and bound names and alpha-equivalence.} At the
core of structural equivalence is alpha-equivalence which identifies
process that are the same up to a change of variable. Formally, we
recognize the distinction between free and bound names. The free names
of a process, $\freenames{P}$, may be calculated recursively as
follows:

\begin{mathpar}
\freenames{\pzero} := \emptyset
  \and \\
  \freenames{x?(y).P} := \{ x \} \cup (\freenames{P} \setminus \{ y \})
  \and 
  \freenames{x!\langle P \rangle} := \{ x \} \cup \{ P \} 
  \and \\
  \freenames{P|Q} := \freenames{P} \cup \freenames{Q}
  \and \\
  \freenames{@{x}} := \{ x \}
\end{mathpar}

$\pi$
$\quotep{\pi}$

$\freenames{-} : \pi \to \mathcal{P}(\quotep{\pi})$

\begin{eqnarray*}
  \freenames{\pzero} & := & \emptyset \\
  \freenames{x?(y).P} & := & \{ x \} \cup (\freenames{P} \setminus \{ y \}) \\
  \freenames{x!\langle P \rangle} & := & \{ x \} \cup \{ P \} \\
  \freenames{P|Q} & := & \freenames{P} \cup \freenames{Q} \\
  \freenames{\dropn{x}} & := & \{ x \}
\end{eqnarray*}

The bound names of a process, $\boundnames{P}$, are those names occurring in $P$
that are not free. For example, in $x?(y).0$, the name $x$ is free, while $y$ is bound.

\begin{mathpar}
  \inferrule* [lab=monoidal-laws] {} { P|Q \equiv Q|P \and P|0 \equiv P \and P|(Q|R) \equiv (P|Q)|R }
\end{mathpar}

\begin{mathpar}
  \inferrule* [lab=alpha-equivalence] {} { (x)P \equiv (y)P\{y/x\} \and y \not\in \freenames{P} }
\end{mathpar}

\begin{definition}
Then two processes, $P,Q$, are alpha-equivalent if $P = Q\{\vec{y}/\vec{x}\}$ for
some $\vec{x} \in \boundnames{Q},\vec{y} \in \boundnames{P}$, where $Q\{\vec{y}/\vec{x}\}$
denotes the capture-avoiding substitution of $\vec{y}$ for $\vec{x}$ in $Q$.
\end{definition}

\begin{definition}
  The {\em structural congruence} \cite{SangiorgiWalker} , $\equiv$,
  between processes is the least congruence containing
  alpha-equivalence, satisfying the abelian monoid laws
  (associativity, commutativity and $\pzero$ as identity) for parallel
  composition $|$ and for summation $+$.
\end{definition}

\subsection{Name equivalence}

We take name equivalence, written $\nameeq$, to be the smallest
equivalence relation generated by the following rules.

\begin{mathpar}
\inferrule*[lab=Quote-drop]
{ }
{ \quotep{@{x}} \nameeq x }

\inferrule*[lab=Struct-equiv]
{ P \scong Q }
{ \quotep{P} \nameeq \quotep{Q} }
\end{mathpar}

The astute reader will have noticed that the mutual recursion of names
and processes imposes a mutual recursion on alpha-equivalence and
structural equivalence via name-equivalence. Fortunately, all of this
works out pleasantly and we may calculate in the natural way, free of
concern. The reader interested in the details is referred to the
appendix \ref{appendix:rho_details}.

\subsection{Substitution}

We use $\Proc$ for the set of processes, $\QProc$ for the set of
names, and $\id{\{}\vec{y} / \vec{x} \id{\}}$ to denote partial maps,
$s : \QProc \rightarrow \QProc$. A map, $s$ lifts, uniquely, to a map
on process terms, $\widehat{s} : \Proc \rightarrow \Proc$ by the
following equations.

\begin{mathpar}
  (0) \psubstp{Q}{P} := 0 \\
  (R \juxtap S) \psubstp{Q}{P}
  :=    
  (R)\psubstp{Q}{P} \juxtap (S) \psubstp{Q}{P} \\
  (x?(y).R) \psubstp{Q}{P}    
  :=    
  (x)\substp{Q}{P} (z)\concat( (R \psubstn{z}{y}) \psubstp{Q}{P} ) \\
  (\lift{x}{R}) \psubstp{Q}{P}  
  :=
  \lift{(x)\substp{Q}{P}}{ R \psubstp{Q}{P} } \\
%   (\dropn{x})  \psubstp{Q}{P}       
%   := 
%   \left\{ 
%     \begin{array}{ccc} 
%       \dropn{\quotep{Q}} & & x \nameeq \quotep{P} \\
%       \dropn{x} & & otherwise \\
%     \end{array}
%   \right. 
  (\dropn{x})  \psubstp{Q}{P}       
  := 
  \left\{ 
    \begin{array}{ccc} 
      Q & & x \nameeq \quotep{P} \\
      \dropn{x} & & otherwise \\
    \end{array}
  \right.
\end{mathpar}
 

where

\begin{eqnarray}
  (x)\id{\{} \lpquote Q \rpquote / \lpquote P \rpquote \id{\}}            = 
  \left\{ 
    \begin{array}{ccc}
      \lpquote Q \rpquote & & x \nameeq \lpquote P \rpquote \\
      x & & otherwise \\
    \end{array}
  \right. \nonumber
\end{eqnarray}

and $z$ is chosen distinct from $\quotep{P}$, $\quotep{Q}$, the free
names in $Q$, and all the names in $R$. Our $\alpha$-equivalence will
be built in the standard way from this substitution.

\begin{remark}\label{rem:no_self_referential_names}
  One consequence of these definitions is that $\forall P. \quotep{P}
  \not\in \freenames{P}$.
\end{remark}

\subsection{ Dynamic quote: an example }

Anticipating something of what's to come, consider applying the
substitution, $\widehat{\id{\{}u / z \id{\}}}$, to the following pair
of processes, $\lift{w}{y!(z)}$ and $w[ \lpquote y!(z) \rpquote ]$.

\begin{eqnarray}
	\lift{w}{y!(z)}\widehat{\id{\{}u / z \id{\}}}
		& = &
		\lift{w}{y!(u)} \nonumber\\
	w[ \lpquote y!(z) \rpquote ] \widehat{ \id{\{}u / z \id{\}} }
		& = &
		w[ \lpquote y!(z) \rpquote ] \nonumber
\end{eqnarray}

Because the body of the process between quotes is impervious to
substitution, we get radically different answers. In fact, by
examining the first process in an input context,
e.g. $x?(z).\lift{w}{y!(z)}$, we see that the process under the lift
operator may be shaped by prefixed inputs binding a name inside it. In
this sense, the lift operator will be seen as a way to dynamically
construct processes before reifying them as names.

Finally equipped with these standard features we can present the
dynamics of the calculus.

\subsubsection{Operational semantics} 

Finally, we introduce the computational dynamics. What marks these
algebras as distinct from other more traditionally studied algebraic
structures, e.g. vector spaces or polynomial rings, is the manner in
which dynamics is captured. In traditional structures, dynamics is typically
expressed through morphisms between such structures, as in linear maps
between vector spaces or morphisms between rings. In algebras
associated with the semantics of computation, the dynamics is
expressed as part of the algebraic structure itself, through a
reduction reduction relation typically denoted by $\red$. Below, we
give a recursive presentation of this relation for the calculus used
in the encoding.

$\red \subseteq \pi \times \pi$
$\red : \pi \to \mathcal{P}(\pi)$

\begin{mathpar}
  \inferrule* [lab=Comm] { \textsf{match}( x_{src}, x_{trgt} ) } { x_{trgt}?(y)P \; | \; x_{src}!\langle {Q} \rangle \red P\{\quotep{Q}/y}\} }
  \and \\
  \inferrule* [lab=Par] {{P} \red {P}'} {{{P} | {Q}} \red {{P}' | {Q}}}
  \and
  \inferrule* [lab=Equiv]{{{P} \scong {P}'} \andalso {{P}' \red {Q}'} \andalso {{Q}' \scong {Q}}}{{P} \red {Q}}
\end{mathpar}

\begin{eqnarray*}
  match_{\equiv} (\quotep{P},\quotep{Q}) & := & P \equiv Q \\
  match_{\dagger}(\quotep{P},\quotep{Q}) & := & \forall R. P|Q \red^{*} R => R \red^{*} 0 \\
  match_{K}(\quotep{P},\quotep{Q}) & := & K \mbox{ for some context } K
\end{eqnarray*}

$u?(x)P | u!\langle Q \rangle \red P\{\quotep{Q}/x\}$

%We write $\wred$ for $\red^*$, and $P\red$ if $\exists Q $ such that $ P \red Q$.
We write $P\red$ if $\exists Q $ such that $ P \red Q$ and $P\not\red$, otherwise.

\section{Replication}

As mentioned before, it is known that replication (and hence
recursion) can be implemented in a higher-order process algebra
\cite{SangiorgiWalker}. As our first example of calculation with the
machinery thus far presented we give the construction explicitly in
the {\rhoc}.

\begin{eqnarray}
	D_{x} & := & \prefix{x}{y}{(\binpar{\outputp{x}{y}}{@{y}})} \nonumber\\
	\bangp_{x}{P} & := & \binpar{{x}!\langle{\binpar{D_{x}}{P}}\rangle}{D_{x}} \nonumber
\end{eqnarray}

\begin{eqnarray}
	\bangp_{x}{P} & & \nonumber\\
	=
	& {x}!\langle{(\prefix{x}{y}{(\outputp{x}{y} | @{y})) | P}}\rangle 
	      | \prefix{x}{y}{(\outputp{x}{y} | @{y})} & \nonumber\\
	\red
	& (\outputp{x}{y} | @{y})\substn{\quotep{(\prefix{x}{y}{(@{y} | \outputp{x}{y})) | P}}}{y} & \nonumber\\
	=
	& \outputp{x}{\quotep{(\prefix{x}{y}{(\outputp{x}{y} | @{y})) | P}}}
	  | {(\prefix{x}{y}{(\outputp{x}{y} | @{y})) | P}} & \nonumber\\
	\red
	& \ldots & \nonumber\\
	\red^*
	& P | P | \ldots & \nonumber
\end{eqnarray}

Of course, this encoding, as an implementation, runs away, unfolding
$\bangp{P}$ eagerly. A lazier and more implementable replication
operator, restricted to input-guarded processes, may be obtained as follows.

\begin{eqnarray}
\bangp{\prefix{u}{v}{P}} 
	:= 
	\binpar{\lift{x}{\prefix{u}{v}{(\binpar{D(x)}{P})}}}{D(x)} \nonumber
\end{eqnarray}

\begin{remark}
  Note that the lazier definition still does not deal with summation
  or mixed summation (i.e. sums over input and output). The reader is
  invited to construct definitions of replication that deal with these
  features. 

  Further, the definitions are parameterized in a name, $x$. Can you,
  gentle reader, make a definition that eliminates this parameter and
  guarantees no accidental interaction between the replication
  machinery and the process being replicated -- i.e. no accidental
  sharing of names used by the process to get its work done and the
  name(s) used by the replication to effect copying. This latter
  revision of the definition of replication is crucial to obtaining
  the expected identity $!!P \sim !P$.
\end{remark}

\begin{remark}\label{rem:paradoxical_combinator}
  The reader familiar with the lambda calculus will have noticed the
  similarity between $D$ and the paradoxical combinator.

  [Ed. note: the existence of this seems to suggest we have to be more
  restrictive on the set of processes and names we admit if we are to
  support no-cloning.]
\end{remark}

\subsubsection{Bisimulation}

The computational dynamics gives rise to another kind of equivalence,
the equivalence of computational behavior. As previously mentioned
this is typically captured \emph{via} some form of bisimulation.

% The notion we use in this paper is weak barbed bisimulation
% \cite{milner91polyadicpi}.

The notion we use in this paper is derived from weak barbed
bisimulation \cite{milner91polyadicpi}. 

\begin{definition}
An \emph{observation relation}, $\downarrow_{\mathcal N}$, over a set
of names, $\mathcal N$, is the smallest relation satisfying the rules
below.

\infrule[Out-barb]{y \in {\mathcal N}, \; x \nameeq y}
		  {\outputp{x}{v} \downarrow_{\mathcal N} x}
\infrule[Par-barb]{\mbox{$P\downarrow_{\mathcal N} x$ or $Q\downarrow_{\mathcal N} x$}}
		  {\binpar{P}{Q} \downarrow_{\mathcal N} x}

We write $P \Downarrow_{\mathcal N} x$ if there is $Q$ such that 
$P \wred Q$ and $Q \downarrow_{\mathcal N} x$.
\end{definition}

\begin{definition}
%\label{def.bbisim}
An  ${\mathcal N}$-\emph{barbed bisimulation} over a set of names, ${\mathcal N}$, is a symmetric binary relation 
${\mathcal S}_{\mathcal N}$ between agents such that $P\rel{S}_{\mathcal N}Q$ implies:
\begin{enumerate}
\item If $P \red P'$ then $Q \wred Q'$ and $P'\rel{S}_{\mathcal N} Q'$.
\item If $P\downarrow_{\mathcal N} x$, then $Q\Downarrow_{\mathcal N} x$.
\end{enumerate}
$P$ is ${\mathcal N}$-barbed bisimilar to $Q$, written
$P \wbbisim_{\mathcal N} Q$, if $P \rel{S}_{\mathcal N} Q$ for some ${\mathcal N}$-barbed bisimulation ${\mathcal S}_{\mathcal N}$.
\end{definition}

$\mathcal{R} \subseteq \pi \times \pi$

$P \mathcal{R} Q => \forall P'. P \red P' \Rightarrow \exists Q'. Q \red Q', P' \mathcal{R} Q'$

$P \vdash x \Rightarrow Q \vdash x$

\begin{mathpar}
  \inferrule*[lab=Out-barb]{x \nameeq y}{{y}!\langle{Q}\rangle \vdash x}
  \and
  \inferrule*[lab=Par-barb]{\mbox{$P\vdash x$ or $Q\vdash x$}}{\binpar{P}{Q} \vdash x}
\end{mathpar}

\subsubsection{Contexts}

One of the principle advantages of computational calculi like the
$\pi$-calculus is a well-defined notion of context,
contextual-equivalence and a correlation between
contextual-equivalence and notions of bisimulation. The notion of
context allows the decomposition of a process into (sub-)process and
its syntactic environment, its context. Thus, a context may be
thought of as a process with a ``hole'' (written $\Box$) in it. The
application of a context $M$ to a process $P$, written $M[P]$, is
tantamount to filling the hole in $M$ with $P$. In this paper we do
not need the full weight of this theory, but do make use of the notion
of context in the proof the main theorem. 

\begin{mathpar}
  \inferrule* [lab=summation] {} {{M_{M},M_{N}} \bc \Box \;|\; x.M_{A} \;|\; M_{M}+M_{N}}
  \and
  \inferrule* [lab=agent] {} {{M_{A}} \bc (\vec{x})M_{P} \;| \; \clift{P_0,\ldots,M_{P},\ldots,P_N}}
  \and \\
  \inferrule* [lab=process] {} {{M_{P}} \bc M_{N} \;| \;P|M_{P} }
\end{mathpar} 

\begin{mathpar}
  \inferrule* [lab=sychronization] {} {M_{N} \bc \Box \;|\; x?M_{F} \;|\; x!M_{C}}
  \and
  \inferrule* [lab=abstraction] {} {{M_{F}} \bc (x)M_{P} }
  \and
  \inferrule* [lab=concretion] {} {{M_{C}} \bc \langle M_{P} \rangle }
  \and \\
  \inferrule* [lab=process] {} {{M_{P}} \bc M_{N} \;| \;P|M_{P} }
\end{mathpar}

\begin{definition}[contextual application] Given a context $M$, and
  process $P$, we define the \emph{contextual application}, $M[P] :=
  M\{P/\Box\}$. That is, the contextual application of M to P is the
  substitution of $P$ for $\Box$ in $M$.
\end{definition}

$\meaningof{-} : L \to \mathcal{P}(\pi)$

\begin{mathpar}
  \inferrule* [lab=collection] {} {\meaningof{true} = \pi, \and \meaningof{~E} = \pi \setminus \meaningof{E}, \and \meaningof{E_{1} \& E_{2}} = \meaningof{E_{1}} \cap \meaningof{E_{2}}}
\end{mathpar}

\begin{mathpar}
  \inferrule* [lab=structure] {} {\meaningof{0} = \{ P \in \pi | P \equiv 0 \}, \and \\ \meaningof{E_1 | E_2} = \{ P \in \pi | P \equiv P_{1} | P_{2}, P_{1} \in \meaningof{E_{1}}, P_{2} \in \meaningof{E_2}\} }
\end{mathpar}

\begin{mathpar}
 \inferrule* [lab=behavior] {} {\meaningof{\langle a?b \rangle E} = \{ P \in \pi | P \equiv Q | u?(y)P', \\ \and \\\\ \and \\ \;\;\; u \in \meaningof{a}, \forall z.P'\{z/y\} \in \meaningof{E\{z/b\}}\}, \and \\ \meaningof{a!E} = \{ P \in \pi | P \equiv Q | x!\langle P' \rangle, x \in \meaningof{a} P' \in \meaningof{E}\} }
\end{mathpar}

\begin{mathpar}
 \inferrule* [lab=nominal] {} {\meaningof{\quotep{E}} = \{ \quotep{P} \in \quotep{\pi} | P \in \meaningof{E} \}, \and \meaningof{\quotep{P}} = \{ \quotep{Q} \in \quotep{\pi} | P \equiv Q \} \and \\ \meaningof{@\quotep{E}} = \{ P \in \pi | P \equiv @x, x \in \meaningof{E} \}}
\end{mathpar}

\begin{eqnarray*}
  \\
  \meaningof{-} : TS \to ST
\end{eqnarray*}

\begin{eqnarray*}
  \\
  L : TS \to ST
\end{eqnarray*}

\begin{eqnarray*}
  \\
  P \models E \iff P \in \meaningof{E}
\end{eqnarray*}

\begin{eqnarray*}
  P \approx_{L} Q \iff \forall E \in L. P \models E \iff Q \models E
\end{eqnarray*}

\begin{eqnarray*}
  P \approx_{K} Q
\end{eqnarray*}

\begin{eqnarray*}
  P \approx Q
\end{eqnarray*}

$\approx_{K} = \approx = \approx_{L}$

\subsubsection{Contextual duality}

Note that contexts extend the quotation operation to a family of
operations from processes to names. Given a context, $M$, we can
define a \emph{nominal context}, $\quotep{M}$ by $\quotep{M}[P] :=
\quotep{M[P]}$. To foreshadow what is to come we observe that these
operations enjoy a duality with processes very much like the duality
between vectors and maps from vectors to scalars.

Further, because the calculus is essentially higher-order, we have a
correspondence between contexts and processes. More specifically,
given a name $x$ and a context $M$ we can construct $M^{*}_{x}$ such
that 

\begin{mathpar}
  M^{*}_{x} | \lift{x}{P} \red M[P]
\end{mathpar}

namely,

\begin{mathpar}
  M^{*}_{x} := x?(u).M[\dropn{u}]
\end{mathpar}

The dependence of $M^{*}_{x}$ on a name makes it an abstraction, 

\begin{mathpar}
  M^{*} := (x)x?(u).M[\dropn{u}]
\end{mathpar}

\subsection{Additional notation}

It will sometimes be convenient to denote the process a name
quotes. We already have the notation $x = \quotep{P}$, but it will be
convenient to introduce an alternate notation, $\procn{x}$, when we
want to emphasize the connection to the use of the name. Note that, by
virtue of name equivalence, $\quotep{\procn{x}} \nameeq x$; so, the
notation is consistent with previous definitions.

Further, because names have structure it is possible to effect
substitutions on the basis of that structure. This means we need to
upgrade our notation for substitutions, which we accomplish by
adapting comprehension notation. Thus,

\begin{mathpar}
  P\{ y / x : x \in S \}
\end{mathpar}

is interpreted to mean the process derived from P by replacing (in a
capture-avoiding manner) each occurrence of $x$ in $S$ by $y$. For example,

\begin{mathpar}
  P\{ \quotep{\procn{x}|\procn{x}} / x : x \in \freenames{P} \}
\end{mathpar}

will replace each (occurrence) of a free name $x$ in $P$ by
$\quotep{\procn{x}|\procn{x}}$.

Also, we will avail ourselves of the notation $x^{L}$ and $x^{R}$ to
denote injections of a name into disjoint copies of the name
space. There are numerous ways to accomplish this. One example can be
found in \cite{MeredithR05}. This notation overloads to vectors of
names: $\vec{x}^{\pi} := (x_{i}^{\pi} \; : \; 0 \leq i < |\vec{x}| )$ where $\pi \in \{L,R\}$.

We also use $P^{\Box} := P|\Box$.

In \cite{MeredithR05} an interpretation of the new operator is
given. It turns out that there are several possible interpretations
all enjoying the requisite algebraic properties of the operator (see
\cite{milner91polyadicpi}). We will therefore make liberal use of
$(\nu\; \vec{x})P$.

% subsection the_syntax_and_semantics_of_the_notation_system (end)   

\input{qm2pi.qmops} 

\input{qm2pi.sterngerlach} 

\input{qm2pi.metric} 

% section concurrent_process_calculi (end)

%\input{qm2pi.proofsketch}

% section proof sketch (end)

%\input{qm2pi.slviaknots} 

% section spatial logic via knots (end)

\input{qm2pi.conclusion}

% section conclusion (end)

%\input{qm2pi.dtcodes} 

% section wiring algorithm (end)

\input{qm2pi.ack} 

% section acknowledgments (end)

\newpage


\bibliographystyle{plain}   
\bibliography{../../biblios/main.bib}

\input{qm2pi.rhodetails}

\end{document}

 

% section wiring algorithm (end)

\documentclass[12pt]{llncs}
%\documentclass{jktr}

\usepackage[pdftex]{hyperref}                   
\usepackage {listings}
\usepackage {mathpartir}
\usepackage{bcprules}
%\usepackage{listings}
                       
\usepackage{graphicx} 
%\usepackage[margins=2.5cm,nohead,nofoot]{geometry}
%\usepackage{geometry}
\usepackage{amsfonts}
\usepackage{amstext}
\usepackage{latexsym}
\usepackage{amssymb}
\usepackage{color}


%\include{myPreamble}
\include{qm2pi.local} 

%\ifpdf
%\usepackage[pdftex]{graphicx}
%\else
%\usepackage{graphicx}
%\fi

 % \ifpdf
%  \usepackage{pdfsync}
%  \if


%\title{Brief Article}
%\author{David F. Snyder}
%\author{L.G. Meredith}

%\address{Dept. of Math., Texas State University--San Marcos, San Marcos, TX 78666}
       
\pagestyle{empty}


\begin{document}

\lstset{language=[Objective]Caml,frame=shadowbox}

\input{qm2pi.front}

% section front matter (end)

\input{qm2pi.intro} 
 
% section introduction (end)

% \input{qm2pi.knotations} 

% section notation (end)

\input{qm2pi.process.calculi} 

% section concurrent_process_calculi_and_spatial_logics_ (end)
    
%\input{qm2pi.knots2pi} 

%\input{qm2pi.trefoil} 

%\input{qm2pi.mainthm} 

% subsection basic_interpretation (end)

%\input{qm2pi.rho.presentation} 
\subsection{The syntax and semantics of the notation system}\label{sub:the_syntax_and_semantics_of_the_notation_system} % (fold)

We now summarize a technical presentation of the calculus that
embodies our theory of dynamics. The typical presentation of such a
calculus follows the style of giving generators and relations on
them. The grammar, below, describing term constructors, freely
generates the set of processes, $\Proc$. This set is then quotiented
by a relation known as structural congruence and it is over this set
that the notion of dynamics is expressed. This presentation is
essentially that of \cite{MeredithR05} with the addition of
polyadicity and summation. For readability we have relegated some of
the technical subtleties to an appendix.

\subsubsection{Process grammar}\label{subsub:process_grammar}

\begin{mathpar}
  \inferrule* [lab=synchronization] {} {{M} \bc \pzero \;|\; x?F \;|\; x!C }
  \and
  \inferrule* [lab=abstraction] {} {{F} \bc (x)P}
  \and
  \inferrule* [lab=concretion] {} {{C} \bc \langle Q \rangle}
  \and
  \inferrule* [lab=process] {} {{P,Q} \bc M \;| \;P|Q \;|\; @{x}}
  \and
  \inferrule* [lab=name] {} {{x} \bc \quotep{P}}
\end{mathpar} 

Note that $\vec{x}$ (resp. $\vec{P}$) denotes a vector of names
(resp. processes) of length $|\vec{x}|$ (resp. $|\vec{P}|$). We adopt
the following useful abbreviations.

\begin{mathpar}
   x?(\vec{y}).P := x.(\vec{y})P \and  x\clift{\vec{P}} := x.\clift{\vec{P}}
   \and x!(y) := \lift{x}{\dropn{y}}
   \and \Pi_{i=0}^{n-1}P_i := P_0 | \ldots | P_{n-1}
\end{mathpar}

\subsubsection{Structural congruence}

\paragraph{Free and bound names and alpha-equivalence.} At the
core of structural equivalence is alpha-equivalence which identifies
process that are the same up to a change of variable. Formally, we
recognize the distinction between free and bound names. The free names
of a process, $\freenames{P}$, may be calculated recursively as
follows:

\begin{mathpar}
\freenames{\pzero} := \emptyset
  \and \\
  \freenames{x?(y).P} := \{ x \} \cup (\freenames{P} \setminus \{ y \})
  \and 
  \freenames{x!\langle P \rangle} := \{ x \} \cup \{ P \} 
  \and \\
  \freenames{P|Q} := \freenames{P} \cup \freenames{Q}
  \and \\
  \freenames{@{x}} := \{ x \}
\end{mathpar}

$\pi$
$\quotep{\pi}$

$\freenames{-} : \pi \to \mathcal{P}(\quotep{\pi})$

\begin{eqnarray*}
  \freenames{\pzero} & := & \emptyset \\
  \freenames{x?(y).P} & := & \{ x \} \cup (\freenames{P} \setminus \{ y \}) \\
  \freenames{x!\langle P \rangle} & := & \{ x \} \cup \{ P \} \\
  \freenames{P|Q} & := & \freenames{P} \cup \freenames{Q} \\
  \freenames{\dropn{x}} & := & \{ x \}
\end{eqnarray*}

The bound names of a process, $\boundnames{P}$, are those names occurring in $P$
that are not free. For example, in $x?(y).0$, the name $x$ is free, while $y$ is bound.

\begin{mathpar}
  \inferrule* [lab=monoidal-laws] {} { P|Q \equiv Q|P \and P|0 \equiv P \and P|(Q|R) \equiv (P|Q)|R }
\end{mathpar}

\begin{mathpar}
  \inferrule* [lab=alpha-equivalence] {} { (x)P \equiv (y)P\{y/x\} \and y \not\in \freenames{P} }
\end{mathpar}

\begin{definition}
Then two processes, $P,Q$, are alpha-equivalent if $P = Q\{\vec{y}/\vec{x}\}$ for
some $\vec{x} \in \boundnames{Q},\vec{y} \in \boundnames{P}$, where $Q\{\vec{y}/\vec{x}\}$
denotes the capture-avoiding substitution of $\vec{y}$ for $\vec{x}$ in $Q$.
\end{definition}

\begin{definition}
  The {\em structural congruence} \cite{SangiorgiWalker} , $\equiv$,
  between processes is the least congruence containing
  alpha-equivalence, satisfying the abelian monoid laws
  (associativity, commutativity and $\pzero$ as identity) for parallel
  composition $|$ and for summation $+$.
\end{definition}

\subsection{Name equivalence}

We take name equivalence, written $\nameeq$, to be the smallest
equivalence relation generated by the following rules.

\begin{mathpar}
\inferrule*[lab=Quote-drop]
{ }
{ \quotep{@{x}} \nameeq x }

\inferrule*[lab=Struct-equiv]
{ P \scong Q }
{ \quotep{P} \nameeq \quotep{Q} }
\end{mathpar}

The astute reader will have noticed that the mutual recursion of names
and processes imposes a mutual recursion on alpha-equivalence and
structural equivalence via name-equivalence. Fortunately, all of this
works out pleasantly and we may calculate in the natural way, free of
concern. The reader interested in the details is referred to the
appendix \ref{appendix:rho_details}.

\subsection{Substitution}

We use $\Proc$ for the set of processes, $\QProc$ for the set of
names, and $\id{\{}\vec{y} / \vec{x} \id{\}}$ to denote partial maps,
$s : \QProc \rightarrow \QProc$. A map, $s$ lifts, uniquely, to a map
on process terms, $\widehat{s} : \Proc \rightarrow \Proc$ by the
following equations.

\begin{mathpar}
  (0) \psubstp{Q}{P} := 0 \\
  (R \juxtap S) \psubstp{Q}{P}
  :=    
  (R)\psubstp{Q}{P} \juxtap (S) \psubstp{Q}{P} \\
  (x?(y).R) \psubstp{Q}{P}    
  :=    
  (x)\substp{Q}{P} (z)\concat( (R \psubstn{z}{y}) \psubstp{Q}{P} ) \\
  (\lift{x}{R}) \psubstp{Q}{P}  
  :=
  \lift{(x)\substp{Q}{P}}{ R \psubstp{Q}{P} } \\
%   (\dropn{x})  \psubstp{Q}{P}       
%   := 
%   \left\{ 
%     \begin{array}{ccc} 
%       \dropn{\quotep{Q}} & & x \nameeq \quotep{P} \\
%       \dropn{x} & & otherwise \\
%     \end{array}
%   \right. 
  (\dropn{x})  \psubstp{Q}{P}       
  := 
  \left\{ 
    \begin{array}{ccc} 
      Q & & x \nameeq \quotep{P} \\
      \dropn{x} & & otherwise \\
    \end{array}
  \right.
\end{mathpar}
 

where

\begin{eqnarray}
  (x)\id{\{} \lpquote Q \rpquote / \lpquote P \rpquote \id{\}}            = 
  \left\{ 
    \begin{array}{ccc}
      \lpquote Q \rpquote & & x \nameeq \lpquote P \rpquote \\
      x & & otherwise \\
    \end{array}
  \right. \nonumber
\end{eqnarray}

and $z$ is chosen distinct from $\quotep{P}$, $\quotep{Q}$, the free
names in $Q$, and all the names in $R$. Our $\alpha$-equivalence will
be built in the standard way from this substitution.

\begin{remark}\label{rem:no_self_referential_names}
  One consequence of these definitions is that $\forall P. \quotep{P}
  \not\in \freenames{P}$.
\end{remark}

\subsection{ Dynamic quote: an example }

Anticipating something of what's to come, consider applying the
substitution, $\widehat{\id{\{}u / z \id{\}}}$, to the following pair
of processes, $\lift{w}{y!(z)}$ and $w[ \lpquote y!(z) \rpquote ]$.

\begin{eqnarray}
	\lift{w}{y!(z)}\widehat{\id{\{}u / z \id{\}}}
		& = &
		\lift{w}{y!(u)} \nonumber\\
	w[ \lpquote y!(z) \rpquote ] \widehat{ \id{\{}u / z \id{\}} }
		& = &
		w[ \lpquote y!(z) \rpquote ] \nonumber
\end{eqnarray}

Because the body of the process between quotes is impervious to
substitution, we get radically different answers. In fact, by
examining the first process in an input context,
e.g. $x?(z).\lift{w}{y!(z)}$, we see that the process under the lift
operator may be shaped by prefixed inputs binding a name inside it. In
this sense, the lift operator will be seen as a way to dynamically
construct processes before reifying them as names.

Finally equipped with these standard features we can present the
dynamics of the calculus.

\subsubsection{Operational semantics} 

Finally, we introduce the computational dynamics. What marks these
algebras as distinct from other more traditionally studied algebraic
structures, e.g. vector spaces or polynomial rings, is the manner in
which dynamics is captured. In traditional structures, dynamics is typically
expressed through morphisms between such structures, as in linear maps
between vector spaces or morphisms between rings. In algebras
associated with the semantics of computation, the dynamics is
expressed as part of the algebraic structure itself, through a
reduction reduction relation typically denoted by $\red$. Below, we
give a recursive presentation of this relation for the calculus used
in the encoding.

$\red \subseteq \pi \times \pi$
$\red : \pi \to \mathcal{P}(\pi)$

\begin{mathpar}
  \inferrule* [lab=Comm] { \textsf{match}( x_{src}, x_{trgt} ) } { x_{trgt}?(y)P \; | \; x_{src}!\langle {Q} \rangle \red P\{\quotep{Q}/y}\} }
  \and \\
  \inferrule* [lab=Par] {{P} \red {P}'} {{{P} | {Q}} \red {{P}' | {Q}}}
  \and
  \inferrule* [lab=Equiv]{{{P} \scong {P}'} \andalso {{P}' \red {Q}'} \andalso {{Q}' \scong {Q}}}{{P} \red {Q}}
\end{mathpar}

\begin{eqnarray*}
  match_{\equiv} (\quotep{P},\quotep{Q}) & := & P \equiv Q \\
  match_{\dagger}(\quotep{P},\quotep{Q}) & := & \forall R. P|Q \red^{*} R => R \red^{*} 0 \\
  match_{K}(\quotep{P},\quotep{Q}) & := & K \mbox{ for some context } K
\end{eqnarray*}

$u?(x)P | u!\langle Q \rangle \red P\{\quotep{Q}/x\}$

%We write $\wred$ for $\red^*$, and $P\red$ if $\exists Q $ such that $ P \red Q$.
We write $P\red$ if $\exists Q $ such that $ P \red Q$ and $P\not\red$, otherwise.

\section{Replication}

As mentioned before, it is known that replication (and hence
recursion) can be implemented in a higher-order process algebra
\cite{SangiorgiWalker}. As our first example of calculation with the
machinery thus far presented we give the construction explicitly in
the {\rhoc}.

\begin{eqnarray}
	D_{x} & := & \prefix{x}{y}{(\binpar{\outputp{x}{y}}{@{y}})} \nonumber\\
	\bangp_{x}{P} & := & \binpar{{x}!\langle{\binpar{D_{x}}{P}}\rangle}{D_{x}} \nonumber
\end{eqnarray}

\begin{eqnarray}
	\bangp_{x}{P} & & \nonumber\\
	=
	& {x}!\langle{(\prefix{x}{y}{(\outputp{x}{y} | @{y})) | P}}\rangle 
	      | \prefix{x}{y}{(\outputp{x}{y} | @{y})} & \nonumber\\
	\red
	& (\outputp{x}{y} | @{y})\substn{\quotep{(\prefix{x}{y}{(@{y} | \outputp{x}{y})) | P}}}{y} & \nonumber\\
	=
	& \outputp{x}{\quotep{(\prefix{x}{y}{(\outputp{x}{y} | @{y})) | P}}}
	  | {(\prefix{x}{y}{(\outputp{x}{y} | @{y})) | P}} & \nonumber\\
	\red
	& \ldots & \nonumber\\
	\red^*
	& P | P | \ldots & \nonumber
\end{eqnarray}

Of course, this encoding, as an implementation, runs away, unfolding
$\bangp{P}$ eagerly. A lazier and more implementable replication
operator, restricted to input-guarded processes, may be obtained as follows.

\begin{eqnarray}
\bangp{\prefix{u}{v}{P}} 
	:= 
	\binpar{\lift{x}{\prefix{u}{v}{(\binpar{D(x)}{P})}}}{D(x)} \nonumber
\end{eqnarray}

\begin{remark}
  Note that the lazier definition still does not deal with summation
  or mixed summation (i.e. sums over input and output). The reader is
  invited to construct definitions of replication that deal with these
  features. 

  Further, the definitions are parameterized in a name, $x$. Can you,
  gentle reader, make a definition that eliminates this parameter and
  guarantees no accidental interaction between the replication
  machinery and the process being replicated -- i.e. no accidental
  sharing of names used by the process to get its work done and the
  name(s) used by the replication to effect copying. This latter
  revision of the definition of replication is crucial to obtaining
  the expected identity $!!P \sim !P$.
\end{remark}

\begin{remark}\label{rem:paradoxical_combinator}
  The reader familiar with the lambda calculus will have noticed the
  similarity between $D$ and the paradoxical combinator.

  [Ed. note: the existence of this seems to suggest we have to be more
  restrictive on the set of processes and names we admit if we are to
  support no-cloning.]
\end{remark}

\subsubsection{Bisimulation}

The computational dynamics gives rise to another kind of equivalence,
the equivalence of computational behavior. As previously mentioned
this is typically captured \emph{via} some form of bisimulation.

% The notion we use in this paper is weak barbed bisimulation
% \cite{milner91polyadicpi}.

The notion we use in this paper is derived from weak barbed
bisimulation \cite{milner91polyadicpi}. 

\begin{definition}
An \emph{observation relation}, $\downarrow_{\mathcal N}$, over a set
of names, $\mathcal N$, is the smallest relation satisfying the rules
below.

\infrule[Out-barb]{y \in {\mathcal N}, \; x \nameeq y}
		  {\outputp{x}{v} \downarrow_{\mathcal N} x}
\infrule[Par-barb]{\mbox{$P\downarrow_{\mathcal N} x$ or $Q\downarrow_{\mathcal N} x$}}
		  {\binpar{P}{Q} \downarrow_{\mathcal N} x}

We write $P \Downarrow_{\mathcal N} x$ if there is $Q$ such that 
$P \wred Q$ and $Q \downarrow_{\mathcal N} x$.
\end{definition}

\begin{definition}
%\label{def.bbisim}
An  ${\mathcal N}$-\emph{barbed bisimulation} over a set of names, ${\mathcal N}$, is a symmetric binary relation 
${\mathcal S}_{\mathcal N}$ between agents such that $P\rel{S}_{\mathcal N}Q$ implies:
\begin{enumerate}
\item If $P \red P'$ then $Q \wred Q'$ and $P'\rel{S}_{\mathcal N} Q'$.
\item If $P\downarrow_{\mathcal N} x$, then $Q\Downarrow_{\mathcal N} x$.
\end{enumerate}
$P$ is ${\mathcal N}$-barbed bisimilar to $Q$, written
$P \wbbisim_{\mathcal N} Q$, if $P \rel{S}_{\mathcal N} Q$ for some ${\mathcal N}$-barbed bisimulation ${\mathcal S}_{\mathcal N}$.
\end{definition}

$\mathcal{R} \subseteq \pi \times \pi$

$P \mathcal{R} Q => \forall P'. P \red P' \Rightarrow \exists Q'. Q \red Q', P' \mathcal{R} Q'$

$P \vdash x \Rightarrow Q \vdash x$

\begin{mathpar}
  \inferrule*[lab=Out-barb]{x \nameeq y}{{y}!\langle{Q}\rangle \vdash x}
  \and
  \inferrule*[lab=Par-barb]{\mbox{$P\vdash x$ or $Q\vdash x$}}{\binpar{P}{Q} \vdash x}
\end{mathpar}

\subsubsection{Contexts}

One of the principle advantages of computational calculi like the
$\pi$-calculus is a well-defined notion of context,
contextual-equivalence and a correlation between
contextual-equivalence and notions of bisimulation. The notion of
context allows the decomposition of a process into (sub-)process and
its syntactic environment, its context. Thus, a context may be
thought of as a process with a ``hole'' (written $\Box$) in it. The
application of a context $M$ to a process $P$, written $M[P]$, is
tantamount to filling the hole in $M$ with $P$. In this paper we do
not need the full weight of this theory, but do make use of the notion
of context in the proof the main theorem. 

\begin{mathpar}
  \inferrule* [lab=summation] {} {{M_{M},M_{N}} \bc \Box \;|\; x.M_{A} \;|\; M_{M}+M_{N}}
  \and
  \inferrule* [lab=agent] {} {{M_{A}} \bc (\vec{x})M_{P} \;| \; \clift{P_0,\ldots,M_{P},\ldots,P_N}}
  \and \\
  \inferrule* [lab=process] {} {{M_{P}} \bc M_{N} \;| \;P|M_{P} }
\end{mathpar} 

\begin{mathpar}
  \inferrule* [lab=sychronization] {} {M_{N} \bc \Box \;|\; x?M_{F} \;|\; x!M_{C}}
  \and
  \inferrule* [lab=abstraction] {} {{M_{F}} \bc (x)M_{P} }
  \and
  \inferrule* [lab=concretion] {} {{M_{C}} \bc \langle M_{P} \rangle }
  \and \\
  \inferrule* [lab=process] {} {{M_{P}} \bc M_{N} \;| \;P|M_{P} }
\end{mathpar}

\begin{definition}[contextual application] Given a context $M$, and
  process $P$, we define the \emph{contextual application}, $M[P] :=
  M\{P/\Box\}$. That is, the contextual application of M to P is the
  substitution of $P$ for $\Box$ in $M$.
\end{definition}

$\meaningof{-} : L \to \mathcal{P}(\pi)$

\begin{mathpar}
  \inferrule* [lab=collection] {} {\meaningof{true} = \pi, \and \meaningof{~E} = \pi \setminus \meaningof{E}, \and \meaningof{E_{1} \& E_{2}} = \meaningof{E_{1}} \cap \meaningof{E_{2}}}
\end{mathpar}

\begin{mathpar}
  \inferrule* [lab=structure] {} {\meaningof{0} = \{ P \in \pi | P \equiv 0 \}, \and \\ \meaningof{E_1 | E_2} = \{ P \in \pi | P \equiv P_{1} | P_{2}, P_{1} \in \meaningof{E_{1}}, P_{2} \in \meaningof{E_2}\} }
\end{mathpar}

\begin{mathpar}
 \inferrule* [lab=behavior] {} {\meaningof{\langle a?b \rangle E} = \{ P \in \pi | P \equiv Q | u?(y)P', \\ \and \\\\ \and \\ \;\;\; u \in \meaningof{a}, \forall z.P'\{z/y\} \in \meaningof{E\{z/b\}}\}, \and \\ \meaningof{a!E} = \{ P \in \pi | P \equiv Q | x!\langle P' \rangle, x \in \meaningof{a} P' \in \meaningof{E}\} }
\end{mathpar}

\begin{mathpar}
 \inferrule* [lab=nominal] {} {\meaningof{\quotep{E}} = \{ \quotep{P} \in \quotep{\pi} | P \in \meaningof{E} \}, \and \meaningof{\quotep{P}} = \{ \quotep{Q} \in \quotep{\pi} | P \equiv Q \} \and \\ \meaningof{@\quotep{E}} = \{ P \in \pi | P \equiv @x, x \in \meaningof{E} \}}
\end{mathpar}

\begin{eqnarray*}
  \\
  \meaningof{-} : TS \to ST
\end{eqnarray*}

\begin{eqnarray*}
  \\
  L : TS \to ST
\end{eqnarray*}

\begin{eqnarray*}
  \\
  P \models E \iff P \in \meaningof{E}
\end{eqnarray*}

\begin{eqnarray*}
  P \approx_{L} Q \iff \forall E \in L. P \models E \iff Q \models E
\end{eqnarray*}

\begin{eqnarray*}
  P \approx_{K} Q
\end{eqnarray*}

\begin{eqnarray*}
  P \approx Q
\end{eqnarray*}

$\approx_{K} = \approx = \approx_{L}$

\subsubsection{Contextual duality}

Note that contexts extend the quotation operation to a family of
operations from processes to names. Given a context, $M$, we can
define a \emph{nominal context}, $\quotep{M}$ by $\quotep{M}[P] :=
\quotep{M[P]}$. To foreshadow what is to come we observe that these
operations enjoy a duality with processes very much like the duality
between vectors and maps from vectors to scalars.

Further, because the calculus is essentially higher-order, we have a
correspondence between contexts and processes. More specifically,
given a name $x$ and a context $M$ we can construct $M^{*}_{x}$ such
that 

\begin{mathpar}
  M^{*}_{x} | \lift{x}{P} \red M[P]
\end{mathpar}

namely,

\begin{mathpar}
  M^{*}_{x} := x?(u).M[\dropn{u}]
\end{mathpar}

The dependence of $M^{*}_{x}$ on a name makes it an abstraction, 

\begin{mathpar}
  M^{*} := (x)x?(u).M[\dropn{u}]
\end{mathpar}

\subsection{Additional notation}

It will sometimes be convenient to denote the process a name
quotes. We already have the notation $x = \quotep{P}$, but it will be
convenient to introduce an alternate notation, $\procn{x}$, when we
want to emphasize the connection to the use of the name. Note that, by
virtue of name equivalence, $\quotep{\procn{x}} \nameeq x$; so, the
notation is consistent with previous definitions.

Further, because names have structure it is possible to effect
substitutions on the basis of that structure. This means we need to
upgrade our notation for substitutions, which we accomplish by
adapting comprehension notation. Thus,

\begin{mathpar}
  P\{ y / x : x \in S \}
\end{mathpar}

is interpreted to mean the process derived from P by replacing (in a
capture-avoiding manner) each occurrence of $x$ in $S$ by $y$. For example,

\begin{mathpar}
  P\{ \quotep{\procn{x}|\procn{x}} / x : x \in \freenames{P} \}
\end{mathpar}

will replace each (occurrence) of a free name $x$ in $P$ by
$\quotep{\procn{x}|\procn{x}}$.

Also, we will avail ourselves of the notation $x^{L}$ and $x^{R}$ to
denote injections of a name into disjoint copies of the name
space. There are numerous ways to accomplish this. One example can be
found in \cite{MeredithR05}. This notation overloads to vectors of
names: $\vec{x}^{\pi} := (x_{i}^{\pi} \; : \; 0 \leq i < |\vec{x}| )$ where $\pi \in \{L,R\}$.

We also use $P^{\Box} := P|\Box$.

In \cite{MeredithR05} an interpretation of the new operator is
given. It turns out that there are several possible interpretations
all enjoying the requisite algebraic properties of the operator (see
\cite{milner91polyadicpi}). We will therefore make liberal use of
$(\nu\; \vec{x})P$.

% subsection the_syntax_and_semantics_of_the_notation_system (end)   

\input{qm2pi.qmops} 

\input{qm2pi.sterngerlach} 

\input{qm2pi.metric} 

% section concurrent_process_calculi (end)

%\input{qm2pi.proofsketch}

% section proof sketch (end)

%\input{qm2pi.slviaknots} 

% section spatial logic via knots (end)

\input{qm2pi.conclusion}

% section conclusion (end)

%\input{qm2pi.dtcodes} 

% section wiring algorithm (end)

\input{qm2pi.ack} 

% section acknowledgments (end)

\newpage


\bibliographystyle{plain}   
\bibliography{../../biblios/main.bib}

\input{qm2pi.rhodetails}

\end{document}

 

% section acknowledgments (end)

\newpage


\bibliographystyle{plain}   
\bibliography{../../biblios/main.bib}

\documentclass[12pt]{llncs}
%\documentclass{jktr}

\usepackage[pdftex]{hyperref}                   
\usepackage {listings}
\usepackage {mathpartir}
\usepackage{bcprules}
%\usepackage{listings}
                       
\usepackage{graphicx} 
%\usepackage[margins=2.5cm,nohead,nofoot]{geometry}
%\usepackage{geometry}
\usepackage{amsfonts}
\usepackage{amstext}
\usepackage{latexsym}
\usepackage{amssymb}
\usepackage{color}


%\include{myPreamble}
\include{qm2pi.local} 

%\ifpdf
%\usepackage[pdftex]{graphicx}
%\else
%\usepackage{graphicx}
%\fi

 % \ifpdf
%  \usepackage{pdfsync}
%  \if


%\title{Brief Article}
%\author{David F. Snyder}
%\author{L.G. Meredith}

%\address{Dept. of Math., Texas State University--San Marcos, San Marcos, TX 78666}
       
\pagestyle{empty}


\begin{document}

\lstset{language=[Objective]Caml,frame=shadowbox}

\input{qm2pi.front}

% section front matter (end)

\input{qm2pi.intro} 
 
% section introduction (end)

% \input{qm2pi.knotations} 

% section notation (end)

\input{qm2pi.process.calculi} 

% section concurrent_process_calculi_and_spatial_logics_ (end)
    
%\input{qm2pi.knots2pi} 

%\input{qm2pi.trefoil} 

%\input{qm2pi.mainthm} 

% subsection basic_interpretation (end)

%\input{qm2pi.rho.presentation} 
\subsection{The syntax and semantics of the notation system}\label{sub:the_syntax_and_semantics_of_the_notation_system} % (fold)

We now summarize a technical presentation of the calculus that
embodies our theory of dynamics. The typical presentation of such a
calculus follows the style of giving generators and relations on
them. The grammar, below, describing term constructors, freely
generates the set of processes, $\Proc$. This set is then quotiented
by a relation known as structural congruence and it is over this set
that the notion of dynamics is expressed. This presentation is
essentially that of \cite{MeredithR05} with the addition of
polyadicity and summation. For readability we have relegated some of
the technical subtleties to an appendix.

\subsubsection{Process grammar}\label{subsub:process_grammar}

\begin{mathpar}
  \inferrule* [lab=synchronization] {} {{M} \bc \pzero \;|\; x?F \;|\; x!C }
  \and
  \inferrule* [lab=abstraction] {} {{F} \bc (x)P}
  \and
  \inferrule* [lab=concretion] {} {{C} \bc \langle Q \rangle}
  \and
  \inferrule* [lab=process] {} {{P,Q} \bc M \;| \;P|Q \;|\; @{x}}
  \and
  \inferrule* [lab=name] {} {{x} \bc \quotep{P}}
\end{mathpar} 

Note that $\vec{x}$ (resp. $\vec{P}$) denotes a vector of names
(resp. processes) of length $|\vec{x}|$ (resp. $|\vec{P}|$). We adopt
the following useful abbreviations.

\begin{mathpar}
   x?(\vec{y}).P := x.(\vec{y})P \and  x\clift{\vec{P}} := x.\clift{\vec{P}}
   \and x!(y) := \lift{x}{\dropn{y}}
   \and \Pi_{i=0}^{n-1}P_i := P_0 | \ldots | P_{n-1}
\end{mathpar}

\subsubsection{Structural congruence}

\paragraph{Free and bound names and alpha-equivalence.} At the
core of structural equivalence is alpha-equivalence which identifies
process that are the same up to a change of variable. Formally, we
recognize the distinction between free and bound names. The free names
of a process, $\freenames{P}$, may be calculated recursively as
follows:

\begin{mathpar}
\freenames{\pzero} := \emptyset
  \and \\
  \freenames{x?(y).P} := \{ x \} \cup (\freenames{P} \setminus \{ y \})
  \and 
  \freenames{x!\langle P \rangle} := \{ x \} \cup \{ P \} 
  \and \\
  \freenames{P|Q} := \freenames{P} \cup \freenames{Q}
  \and \\
  \freenames{@{x}} := \{ x \}
\end{mathpar}

$\pi$
$\quotep{\pi}$

$\freenames{-} : \pi \to \mathcal{P}(\quotep{\pi})$

\begin{eqnarray*}
  \freenames{\pzero} & := & \emptyset \\
  \freenames{x?(y).P} & := & \{ x \} \cup (\freenames{P} \setminus \{ y \}) \\
  \freenames{x!\langle P \rangle} & := & \{ x \} \cup \{ P \} \\
  \freenames{P|Q} & := & \freenames{P} \cup \freenames{Q} \\
  \freenames{\dropn{x}} & := & \{ x \}
\end{eqnarray*}

The bound names of a process, $\boundnames{P}$, are those names occurring in $P$
that are not free. For example, in $x?(y).0$, the name $x$ is free, while $y$ is bound.

\begin{mathpar}
  \inferrule* [lab=monoidal-laws] {} { P|Q \equiv Q|P \and P|0 \equiv P \and P|(Q|R) \equiv (P|Q)|R }
\end{mathpar}

\begin{mathpar}
  \inferrule* [lab=alpha-equivalence] {} { (x)P \equiv (y)P\{y/x\} \and y \not\in \freenames{P} }
\end{mathpar}

\begin{definition}
Then two processes, $P,Q$, are alpha-equivalent if $P = Q\{\vec{y}/\vec{x}\}$ for
some $\vec{x} \in \boundnames{Q},\vec{y} \in \boundnames{P}$, where $Q\{\vec{y}/\vec{x}\}$
denotes the capture-avoiding substitution of $\vec{y}$ for $\vec{x}$ in $Q$.
\end{definition}

\begin{definition}
  The {\em structural congruence} \cite{SangiorgiWalker} , $\equiv$,
  between processes is the least congruence containing
  alpha-equivalence, satisfying the abelian monoid laws
  (associativity, commutativity and $\pzero$ as identity) for parallel
  composition $|$ and for summation $+$.
\end{definition}

\subsection{Name equivalence}

We take name equivalence, written $\nameeq$, to be the smallest
equivalence relation generated by the following rules.

\begin{mathpar}
\inferrule*[lab=Quote-drop]
{ }
{ \quotep{@{x}} \nameeq x }

\inferrule*[lab=Struct-equiv]
{ P \scong Q }
{ \quotep{P} \nameeq \quotep{Q} }
\end{mathpar}

The astute reader will have noticed that the mutual recursion of names
and processes imposes a mutual recursion on alpha-equivalence and
structural equivalence via name-equivalence. Fortunately, all of this
works out pleasantly and we may calculate in the natural way, free of
concern. The reader interested in the details is referred to the
appendix \ref{appendix:rho_details}.

\subsection{Substitution}

We use $\Proc$ for the set of processes, $\QProc$ for the set of
names, and $\id{\{}\vec{y} / \vec{x} \id{\}}$ to denote partial maps,
$s : \QProc \rightarrow \QProc$. A map, $s$ lifts, uniquely, to a map
on process terms, $\widehat{s} : \Proc \rightarrow \Proc$ by the
following equations.

\begin{mathpar}
  (0) \psubstp{Q}{P} := 0 \\
  (R \juxtap S) \psubstp{Q}{P}
  :=    
  (R)\psubstp{Q}{P} \juxtap (S) \psubstp{Q}{P} \\
  (x?(y).R) \psubstp{Q}{P}    
  :=    
  (x)\substp{Q}{P} (z)\concat( (R \psubstn{z}{y}) \psubstp{Q}{P} ) \\
  (\lift{x}{R}) \psubstp{Q}{P}  
  :=
  \lift{(x)\substp{Q}{P}}{ R \psubstp{Q}{P} } \\
%   (\dropn{x})  \psubstp{Q}{P}       
%   := 
%   \left\{ 
%     \begin{array}{ccc} 
%       \dropn{\quotep{Q}} & & x \nameeq \quotep{P} \\
%       \dropn{x} & & otherwise \\
%     \end{array}
%   \right. 
  (\dropn{x})  \psubstp{Q}{P}       
  := 
  \left\{ 
    \begin{array}{ccc} 
      Q & & x \nameeq \quotep{P} \\
      \dropn{x} & & otherwise \\
    \end{array}
  \right.
\end{mathpar}
 

where

\begin{eqnarray}
  (x)\id{\{} \lpquote Q \rpquote / \lpquote P \rpquote \id{\}}            = 
  \left\{ 
    \begin{array}{ccc}
      \lpquote Q \rpquote & & x \nameeq \lpquote P \rpquote \\
      x & & otherwise \\
    \end{array}
  \right. \nonumber
\end{eqnarray}

and $z$ is chosen distinct from $\quotep{P}$, $\quotep{Q}$, the free
names in $Q$, and all the names in $R$. Our $\alpha$-equivalence will
be built in the standard way from this substitution.

\begin{remark}\label{rem:no_self_referential_names}
  One consequence of these definitions is that $\forall P. \quotep{P}
  \not\in \freenames{P}$.
\end{remark}

\subsection{ Dynamic quote: an example }

Anticipating something of what's to come, consider applying the
substitution, $\widehat{\id{\{}u / z \id{\}}}$, to the following pair
of processes, $\lift{w}{y!(z)}$ and $w[ \lpquote y!(z) \rpquote ]$.

\begin{eqnarray}
	\lift{w}{y!(z)}\widehat{\id{\{}u / z \id{\}}}
		& = &
		\lift{w}{y!(u)} \nonumber\\
	w[ \lpquote y!(z) \rpquote ] \widehat{ \id{\{}u / z \id{\}} }
		& = &
		w[ \lpquote y!(z) \rpquote ] \nonumber
\end{eqnarray}

Because the body of the process between quotes is impervious to
substitution, we get radically different answers. In fact, by
examining the first process in an input context,
e.g. $x?(z).\lift{w}{y!(z)}$, we see that the process under the lift
operator may be shaped by prefixed inputs binding a name inside it. In
this sense, the lift operator will be seen as a way to dynamically
construct processes before reifying them as names.

Finally equipped with these standard features we can present the
dynamics of the calculus.

\subsubsection{Operational semantics} 

Finally, we introduce the computational dynamics. What marks these
algebras as distinct from other more traditionally studied algebraic
structures, e.g. vector spaces or polynomial rings, is the manner in
which dynamics is captured. In traditional structures, dynamics is typically
expressed through morphisms between such structures, as in linear maps
between vector spaces or morphisms between rings. In algebras
associated with the semantics of computation, the dynamics is
expressed as part of the algebraic structure itself, through a
reduction reduction relation typically denoted by $\red$. Below, we
give a recursive presentation of this relation for the calculus used
in the encoding.

$\red \subseteq \pi \times \pi$
$\red : \pi \to \mathcal{P}(\pi)$

\begin{mathpar}
  \inferrule* [lab=Comm] { \textsf{match}( x_{src}, x_{trgt} ) } { x_{trgt}?(y)P \; | \; x_{src}!\langle {Q} \rangle \red P\{\quotep{Q}/y}\} }
  \and \\
  \inferrule* [lab=Par] {{P} \red {P}'} {{{P} | {Q}} \red {{P}' | {Q}}}
  \and
  \inferrule* [lab=Equiv]{{{P} \scong {P}'} \andalso {{P}' \red {Q}'} \andalso {{Q}' \scong {Q}}}{{P} \red {Q}}
\end{mathpar}

\begin{eqnarray*}
  match_{\equiv} (\quotep{P},\quotep{Q}) & := & P \equiv Q \\
  match_{\dagger}(\quotep{P},\quotep{Q}) & := & \forall R. P|Q \red^{*} R => R \red^{*} 0 \\
  match_{K}(\quotep{P},\quotep{Q}) & := & K \mbox{ for some context } K
\end{eqnarray*}

$u?(x)P | u!\langle Q \rangle \red P\{\quotep{Q}/x\}$

%We write $\wred$ for $\red^*$, and $P\red$ if $\exists Q $ such that $ P \red Q$.
We write $P\red$ if $\exists Q $ such that $ P \red Q$ and $P\not\red$, otherwise.

\section{Replication}

As mentioned before, it is known that replication (and hence
recursion) can be implemented in a higher-order process algebra
\cite{SangiorgiWalker}. As our first example of calculation with the
machinery thus far presented we give the construction explicitly in
the {\rhoc}.

\begin{eqnarray}
	D_{x} & := & \prefix{x}{y}{(\binpar{\outputp{x}{y}}{@{y}})} \nonumber\\
	\bangp_{x}{P} & := & \binpar{{x}!\langle{\binpar{D_{x}}{P}}\rangle}{D_{x}} \nonumber
\end{eqnarray}

\begin{eqnarray}
	\bangp_{x}{P} & & \nonumber\\
	=
	& {x}!\langle{(\prefix{x}{y}{(\outputp{x}{y} | @{y})) | P}}\rangle 
	      | \prefix{x}{y}{(\outputp{x}{y} | @{y})} & \nonumber\\
	\red
	& (\outputp{x}{y} | @{y})\substn{\quotep{(\prefix{x}{y}{(@{y} | \outputp{x}{y})) | P}}}{y} & \nonumber\\
	=
	& \outputp{x}{\quotep{(\prefix{x}{y}{(\outputp{x}{y} | @{y})) | P}}}
	  | {(\prefix{x}{y}{(\outputp{x}{y} | @{y})) | P}} & \nonumber\\
	\red
	& \ldots & \nonumber\\
	\red^*
	& P | P | \ldots & \nonumber
\end{eqnarray}

Of course, this encoding, as an implementation, runs away, unfolding
$\bangp{P}$ eagerly. A lazier and more implementable replication
operator, restricted to input-guarded processes, may be obtained as follows.

\begin{eqnarray}
\bangp{\prefix{u}{v}{P}} 
	:= 
	\binpar{\lift{x}{\prefix{u}{v}{(\binpar{D(x)}{P})}}}{D(x)} \nonumber
\end{eqnarray}

\begin{remark}
  Note that the lazier definition still does not deal with summation
  or mixed summation (i.e. sums over input and output). The reader is
  invited to construct definitions of replication that deal with these
  features. 

  Further, the definitions are parameterized in a name, $x$. Can you,
  gentle reader, make a definition that eliminates this parameter and
  guarantees no accidental interaction between the replication
  machinery and the process being replicated -- i.e. no accidental
  sharing of names used by the process to get its work done and the
  name(s) used by the replication to effect copying. This latter
  revision of the definition of replication is crucial to obtaining
  the expected identity $!!P \sim !P$.
\end{remark}

\begin{remark}\label{rem:paradoxical_combinator}
  The reader familiar with the lambda calculus will have noticed the
  similarity between $D$ and the paradoxical combinator.

  [Ed. note: the existence of this seems to suggest we have to be more
  restrictive on the set of processes and names we admit if we are to
  support no-cloning.]
\end{remark}

\subsubsection{Bisimulation}

The computational dynamics gives rise to another kind of equivalence,
the equivalence of computational behavior. As previously mentioned
this is typically captured \emph{via} some form of bisimulation.

% The notion we use in this paper is weak barbed bisimulation
% \cite{milner91polyadicpi}.

The notion we use in this paper is derived from weak barbed
bisimulation \cite{milner91polyadicpi}. 

\begin{definition}
An \emph{observation relation}, $\downarrow_{\mathcal N}$, over a set
of names, $\mathcal N$, is the smallest relation satisfying the rules
below.

\infrule[Out-barb]{y \in {\mathcal N}, \; x \nameeq y}
		  {\outputp{x}{v} \downarrow_{\mathcal N} x}
\infrule[Par-barb]{\mbox{$P\downarrow_{\mathcal N} x$ or $Q\downarrow_{\mathcal N} x$}}
		  {\binpar{P}{Q} \downarrow_{\mathcal N} x}

We write $P \Downarrow_{\mathcal N} x$ if there is $Q$ such that 
$P \wred Q$ and $Q \downarrow_{\mathcal N} x$.
\end{definition}

\begin{definition}
%\label{def.bbisim}
An  ${\mathcal N}$-\emph{barbed bisimulation} over a set of names, ${\mathcal N}$, is a symmetric binary relation 
${\mathcal S}_{\mathcal N}$ between agents such that $P\rel{S}_{\mathcal N}Q$ implies:
\begin{enumerate}
\item If $P \red P'$ then $Q \wred Q'$ and $P'\rel{S}_{\mathcal N} Q'$.
\item If $P\downarrow_{\mathcal N} x$, then $Q\Downarrow_{\mathcal N} x$.
\end{enumerate}
$P$ is ${\mathcal N}$-barbed bisimilar to $Q$, written
$P \wbbisim_{\mathcal N} Q$, if $P \rel{S}_{\mathcal N} Q$ for some ${\mathcal N}$-barbed bisimulation ${\mathcal S}_{\mathcal N}$.
\end{definition}

$\mathcal{R} \subseteq \pi \times \pi$

$P \mathcal{R} Q => \forall P'. P \red P' \Rightarrow \exists Q'. Q \red Q', P' \mathcal{R} Q'$

$P \vdash x \Rightarrow Q \vdash x$

\begin{mathpar}
  \inferrule*[lab=Out-barb]{x \nameeq y}{{y}!\langle{Q}\rangle \vdash x}
  \and
  \inferrule*[lab=Par-barb]{\mbox{$P\vdash x$ or $Q\vdash x$}}{\binpar{P}{Q} \vdash x}
\end{mathpar}

\subsubsection{Contexts}

One of the principle advantages of computational calculi like the
$\pi$-calculus is a well-defined notion of context,
contextual-equivalence and a correlation between
contextual-equivalence and notions of bisimulation. The notion of
context allows the decomposition of a process into (sub-)process and
its syntactic environment, its context. Thus, a context may be
thought of as a process with a ``hole'' (written $\Box$) in it. The
application of a context $M$ to a process $P$, written $M[P]$, is
tantamount to filling the hole in $M$ with $P$. In this paper we do
not need the full weight of this theory, but do make use of the notion
of context in the proof the main theorem. 

\begin{mathpar}
  \inferrule* [lab=summation] {} {{M_{M},M_{N}} \bc \Box \;|\; x.M_{A} \;|\; M_{M}+M_{N}}
  \and
  \inferrule* [lab=agent] {} {{M_{A}} \bc (\vec{x})M_{P} \;| \; \clift{P_0,\ldots,M_{P},\ldots,P_N}}
  \and \\
  \inferrule* [lab=process] {} {{M_{P}} \bc M_{N} \;| \;P|M_{P} }
\end{mathpar} 

\begin{mathpar}
  \inferrule* [lab=sychronization] {} {M_{N} \bc \Box \;|\; x?M_{F} \;|\; x!M_{C}}
  \and
  \inferrule* [lab=abstraction] {} {{M_{F}} \bc (x)M_{P} }
  \and
  \inferrule* [lab=concretion] {} {{M_{C}} \bc \langle M_{P} \rangle }
  \and \\
  \inferrule* [lab=process] {} {{M_{P}} \bc M_{N} \;| \;P|M_{P} }
\end{mathpar}

\begin{definition}[contextual application] Given a context $M$, and
  process $P$, we define the \emph{contextual application}, $M[P] :=
  M\{P/\Box\}$. That is, the contextual application of M to P is the
  substitution of $P$ for $\Box$ in $M$.
\end{definition}

$\meaningof{-} : L \to \mathcal{P}(\pi)$

\begin{mathpar}
  \inferrule* [lab=collection] {} {\meaningof{true} = \pi, \and \meaningof{~E} = \pi \setminus \meaningof{E}, \and \meaningof{E_{1} \& E_{2}} = \meaningof{E_{1}} \cap \meaningof{E_{2}}}
\end{mathpar}

\begin{mathpar}
  \inferrule* [lab=structure] {} {\meaningof{0} = \{ P \in \pi | P \equiv 0 \}, \and \\ \meaningof{E_1 | E_2} = \{ P \in \pi | P \equiv P_{1} | P_{2}, P_{1} \in \meaningof{E_{1}}, P_{2} \in \meaningof{E_2}\} }
\end{mathpar}

\begin{mathpar}
 \inferrule* [lab=behavior] {} {\meaningof{\langle a?b \rangle E} = \{ P \in \pi | P \equiv Q | u?(y)P', \\ \and \\\\ \and \\ \;\;\; u \in \meaningof{a}, \forall z.P'\{z/y\} \in \meaningof{E\{z/b\}}\}, \and \\ \meaningof{a!E} = \{ P \in \pi | P \equiv Q | x!\langle P' \rangle, x \in \meaningof{a} P' \in \meaningof{E}\} }
\end{mathpar}

\begin{mathpar}
 \inferrule* [lab=nominal] {} {\meaningof{\quotep{E}} = \{ \quotep{P} \in \quotep{\pi} | P \in \meaningof{E} \}, \and \meaningof{\quotep{P}} = \{ \quotep{Q} \in \quotep{\pi} | P \equiv Q \} \and \\ \meaningof{@\quotep{E}} = \{ P \in \pi | P \equiv @x, x \in \meaningof{E} \}}
\end{mathpar}

\begin{eqnarray*}
  \\
  \meaningof{-} : TS \to ST
\end{eqnarray*}

\begin{eqnarray*}
  \\
  L : TS \to ST
\end{eqnarray*}

\begin{eqnarray*}
  \\
  P \models E \iff P \in \meaningof{E}
\end{eqnarray*}

\begin{eqnarray*}
  P \approx_{L} Q \iff \forall E \in L. P \models E \iff Q \models E
\end{eqnarray*}

\begin{eqnarray*}
  P \approx_{K} Q
\end{eqnarray*}

\begin{eqnarray*}
  P \approx Q
\end{eqnarray*}

$\approx_{K} = \approx = \approx_{L}$

\subsubsection{Contextual duality}

Note that contexts extend the quotation operation to a family of
operations from processes to names. Given a context, $M$, we can
define a \emph{nominal context}, $\quotep{M}$ by $\quotep{M}[P] :=
\quotep{M[P]}$. To foreshadow what is to come we observe that these
operations enjoy a duality with processes very much like the duality
between vectors and maps from vectors to scalars.

Further, because the calculus is essentially higher-order, we have a
correspondence between contexts and processes. More specifically,
given a name $x$ and a context $M$ we can construct $M^{*}_{x}$ such
that 

\begin{mathpar}
  M^{*}_{x} | \lift{x}{P} \red M[P]
\end{mathpar}

namely,

\begin{mathpar}
  M^{*}_{x} := x?(u).M[\dropn{u}]
\end{mathpar}

The dependence of $M^{*}_{x}$ on a name makes it an abstraction, 

\begin{mathpar}
  M^{*} := (x)x?(u).M[\dropn{u}]
\end{mathpar}

\subsection{Additional notation}

It will sometimes be convenient to denote the process a name
quotes. We already have the notation $x = \quotep{P}$, but it will be
convenient to introduce an alternate notation, $\procn{x}$, when we
want to emphasize the connection to the use of the name. Note that, by
virtue of name equivalence, $\quotep{\procn{x}} \nameeq x$; so, the
notation is consistent with previous definitions.

Further, because names have structure it is possible to effect
substitutions on the basis of that structure. This means we need to
upgrade our notation for substitutions, which we accomplish by
adapting comprehension notation. Thus,

\begin{mathpar}
  P\{ y / x : x \in S \}
\end{mathpar}

is interpreted to mean the process derived from P by replacing (in a
capture-avoiding manner) each occurrence of $x$ in $S$ by $y$. For example,

\begin{mathpar}
  P\{ \quotep{\procn{x}|\procn{x}} / x : x \in \freenames{P} \}
\end{mathpar}

will replace each (occurrence) of a free name $x$ in $P$ by
$\quotep{\procn{x}|\procn{x}}$.

Also, we will avail ourselves of the notation $x^{L}$ and $x^{R}$ to
denote injections of a name into disjoint copies of the name
space. There are numerous ways to accomplish this. One example can be
found in \cite{MeredithR05}. This notation overloads to vectors of
names: $\vec{x}^{\pi} := (x_{i}^{\pi} \; : \; 0 \leq i < |\vec{x}| )$ where $\pi \in \{L,R\}$.

We also use $P^{\Box} := P|\Box$.

In \cite{MeredithR05} an interpretation of the new operator is
given. It turns out that there are several possible interpretations
all enjoying the requisite algebraic properties of the operator (see
\cite{milner91polyadicpi}). We will therefore make liberal use of
$(\nu\; \vec{x})P$.

% subsection the_syntax_and_semantics_of_the_notation_system (end)   

\input{qm2pi.qmops} 

\input{qm2pi.sterngerlach} 

\input{qm2pi.metric} 

% section concurrent_process_calculi (end)

%\input{qm2pi.proofsketch}

% section proof sketch (end)

%\input{qm2pi.slviaknots} 

% section spatial logic via knots (end)

\input{qm2pi.conclusion}

% section conclusion (end)

%\input{qm2pi.dtcodes} 

% section wiring algorithm (end)

\input{qm2pi.ack} 

% section acknowledgments (end)

\newpage


\bibliographystyle{plain}   
\bibliography{../../biblios/main.bib}

\input{qm2pi.rhodetails}

\end{document}



\end{document}

 

%\ifpdf
%\usepackage[pdftex]{graphicx}
%\else
%\usepackage{graphicx}
%\fi

 % \ifpdf
%  \usepackage{pdfsync}
%  \if


%\title{Brief Article}
%\author{David F. Snyder}
%\author{L.G. Meredith}

%\address{Dept. of Math., Texas State University--San Marcos, San Marcos, TX 78666}
       
\pagestyle{empty}


\begin{document}

\lstset{language=[Objective]Caml,frame=shadowbox}

\documentclass[12pt]{llncs}
%\documentclass{jktr}

\usepackage[pdftex]{hyperref}                   
\usepackage {listings}
\usepackage {mathpartir}
\usepackage{bcprules}
%\usepackage{listings}
                       
\usepackage{graphicx} 
%\usepackage[margins=2.5cm,nohead,nofoot]{geometry}
%\usepackage{geometry}
\usepackage{amsfonts}
\usepackage{amstext}
\usepackage{latexsym}
\usepackage{amssymb}
\usepackage{color}


%\include{myPreamble}
\documentclass[12pt]{llncs}
%\documentclass{jktr}

\usepackage[pdftex]{hyperref}                   
\usepackage {listings}
\usepackage {mathpartir}
\usepackage{bcprules}
%\usepackage{listings}
                       
\usepackage{graphicx} 
%\usepackage[margins=2.5cm,nohead,nofoot]{geometry}
%\usepackage{geometry}
\usepackage{amsfonts}
\usepackage{amstext}
\usepackage{latexsym}
\usepackage{amssymb}
\usepackage{color}


%\include{myPreamble}
\include{qm2pi.local} 

%\ifpdf
%\usepackage[pdftex]{graphicx}
%\else
%\usepackage{graphicx}
%\fi

 % \ifpdf
%  \usepackage{pdfsync}
%  \if


%\title{Brief Article}
%\author{David F. Snyder}
%\author{L.G. Meredith}

%\address{Dept. of Math., Texas State University--San Marcos, San Marcos, TX 78666}
       
\pagestyle{empty}


\begin{document}

\lstset{language=[Objective]Caml,frame=shadowbox}

\input{qm2pi.front}

% section front matter (end)

\input{qm2pi.intro} 
 
% section introduction (end)

% \input{qm2pi.knotations} 

% section notation (end)

\input{qm2pi.process.calculi} 

% section concurrent_process_calculi_and_spatial_logics_ (end)
    
%\input{qm2pi.knots2pi} 

%\input{qm2pi.trefoil} 

%\input{qm2pi.mainthm} 

% subsection basic_interpretation (end)

%\input{qm2pi.rho.presentation} 
\subsection{The syntax and semantics of the notation system}\label{sub:the_syntax_and_semantics_of_the_notation_system} % (fold)

We now summarize a technical presentation of the calculus that
embodies our theory of dynamics. The typical presentation of such a
calculus follows the style of giving generators and relations on
them. The grammar, below, describing term constructors, freely
generates the set of processes, $\Proc$. This set is then quotiented
by a relation known as structural congruence and it is over this set
that the notion of dynamics is expressed. This presentation is
essentially that of \cite{MeredithR05} with the addition of
polyadicity and summation. For readability we have relegated some of
the technical subtleties to an appendix.

\subsubsection{Process grammar}\label{subsub:process_grammar}

\begin{mathpar}
  \inferrule* [lab=synchronization] {} {{M} \bc \pzero \;|\; x?F \;|\; x!C }
  \and
  \inferrule* [lab=abstraction] {} {{F} \bc (x)P}
  \and
  \inferrule* [lab=concretion] {} {{C} \bc \langle Q \rangle}
  \and
  \inferrule* [lab=process] {} {{P,Q} \bc M \;| \;P|Q \;|\; @{x}}
  \and
  \inferrule* [lab=name] {} {{x} \bc \quotep{P}}
\end{mathpar} 

Note that $\vec{x}$ (resp. $\vec{P}$) denotes a vector of names
(resp. processes) of length $|\vec{x}|$ (resp. $|\vec{P}|$). We adopt
the following useful abbreviations.

\begin{mathpar}
   x?(\vec{y}).P := x.(\vec{y})P \and  x\clift{\vec{P}} := x.\clift{\vec{P}}
   \and x!(y) := \lift{x}{\dropn{y}}
   \and \Pi_{i=0}^{n-1}P_i := P_0 | \ldots | P_{n-1}
\end{mathpar}

\subsubsection{Structural congruence}

\paragraph{Free and bound names and alpha-equivalence.} At the
core of structural equivalence is alpha-equivalence which identifies
process that are the same up to a change of variable. Formally, we
recognize the distinction between free and bound names. The free names
of a process, $\freenames{P}$, may be calculated recursively as
follows:

\begin{mathpar}
\freenames{\pzero} := \emptyset
  \and \\
  \freenames{x?(y).P} := \{ x \} \cup (\freenames{P} \setminus \{ y \})
  \and 
  \freenames{x!\langle P \rangle} := \{ x \} \cup \{ P \} 
  \and \\
  \freenames{P|Q} := \freenames{P} \cup \freenames{Q}
  \and \\
  \freenames{@{x}} := \{ x \}
\end{mathpar}

$\pi$
$\quotep{\pi}$

$\freenames{-} : \pi \to \mathcal{P}(\quotep{\pi})$

\begin{eqnarray*}
  \freenames{\pzero} & := & \emptyset \\
  \freenames{x?(y).P} & := & \{ x \} \cup (\freenames{P} \setminus \{ y \}) \\
  \freenames{x!\langle P \rangle} & := & \{ x \} \cup \{ P \} \\
  \freenames{P|Q} & := & \freenames{P} \cup \freenames{Q} \\
  \freenames{\dropn{x}} & := & \{ x \}
\end{eqnarray*}

The bound names of a process, $\boundnames{P}$, are those names occurring in $P$
that are not free. For example, in $x?(y).0$, the name $x$ is free, while $y$ is bound.

\begin{mathpar}
  \inferrule* [lab=monoidal-laws] {} { P|Q \equiv Q|P \and P|0 \equiv P \and P|(Q|R) \equiv (P|Q)|R }
\end{mathpar}

\begin{mathpar}
  \inferrule* [lab=alpha-equivalence] {} { (x)P \equiv (y)P\{y/x\} \and y \not\in \freenames{P} }
\end{mathpar}

\begin{definition}
Then two processes, $P,Q$, are alpha-equivalent if $P = Q\{\vec{y}/\vec{x}\}$ for
some $\vec{x} \in \boundnames{Q},\vec{y} \in \boundnames{P}$, where $Q\{\vec{y}/\vec{x}\}$
denotes the capture-avoiding substitution of $\vec{y}$ for $\vec{x}$ in $Q$.
\end{definition}

\begin{definition}
  The {\em structural congruence} \cite{SangiorgiWalker} , $\equiv$,
  between processes is the least congruence containing
  alpha-equivalence, satisfying the abelian monoid laws
  (associativity, commutativity and $\pzero$ as identity) for parallel
  composition $|$ and for summation $+$.
\end{definition}

\subsection{Name equivalence}

We take name equivalence, written $\nameeq$, to be the smallest
equivalence relation generated by the following rules.

\begin{mathpar}
\inferrule*[lab=Quote-drop]
{ }
{ \quotep{@{x}} \nameeq x }

\inferrule*[lab=Struct-equiv]
{ P \scong Q }
{ \quotep{P} \nameeq \quotep{Q} }
\end{mathpar}

The astute reader will have noticed that the mutual recursion of names
and processes imposes a mutual recursion on alpha-equivalence and
structural equivalence via name-equivalence. Fortunately, all of this
works out pleasantly and we may calculate in the natural way, free of
concern. The reader interested in the details is referred to the
appendix \ref{appendix:rho_details}.

\subsection{Substitution}

We use $\Proc$ for the set of processes, $\QProc$ for the set of
names, and $\id{\{}\vec{y} / \vec{x} \id{\}}$ to denote partial maps,
$s : \QProc \rightarrow \QProc$. A map, $s$ lifts, uniquely, to a map
on process terms, $\widehat{s} : \Proc \rightarrow \Proc$ by the
following equations.

\begin{mathpar}
  (0) \psubstp{Q}{P} := 0 \\
  (R \juxtap S) \psubstp{Q}{P}
  :=    
  (R)\psubstp{Q}{P} \juxtap (S) \psubstp{Q}{P} \\
  (x?(y).R) \psubstp{Q}{P}    
  :=    
  (x)\substp{Q}{P} (z)\concat( (R \psubstn{z}{y}) \psubstp{Q}{P} ) \\
  (\lift{x}{R}) \psubstp{Q}{P}  
  :=
  \lift{(x)\substp{Q}{P}}{ R \psubstp{Q}{P} } \\
%   (\dropn{x})  \psubstp{Q}{P}       
%   := 
%   \left\{ 
%     \begin{array}{ccc} 
%       \dropn{\quotep{Q}} & & x \nameeq \quotep{P} \\
%       \dropn{x} & & otherwise \\
%     \end{array}
%   \right. 
  (\dropn{x})  \psubstp{Q}{P}       
  := 
  \left\{ 
    \begin{array}{ccc} 
      Q & & x \nameeq \quotep{P} \\
      \dropn{x} & & otherwise \\
    \end{array}
  \right.
\end{mathpar}
 

where

\begin{eqnarray}
  (x)\id{\{} \lpquote Q \rpquote / \lpquote P \rpquote \id{\}}            = 
  \left\{ 
    \begin{array}{ccc}
      \lpquote Q \rpquote & & x \nameeq \lpquote P \rpquote \\
      x & & otherwise \\
    \end{array}
  \right. \nonumber
\end{eqnarray}

and $z$ is chosen distinct from $\quotep{P}$, $\quotep{Q}$, the free
names in $Q$, and all the names in $R$. Our $\alpha$-equivalence will
be built in the standard way from this substitution.

\begin{remark}\label{rem:no_self_referential_names}
  One consequence of these definitions is that $\forall P. \quotep{P}
  \not\in \freenames{P}$.
\end{remark}

\subsection{ Dynamic quote: an example }

Anticipating something of what's to come, consider applying the
substitution, $\widehat{\id{\{}u / z \id{\}}}$, to the following pair
of processes, $\lift{w}{y!(z)}$ and $w[ \lpquote y!(z) \rpquote ]$.

\begin{eqnarray}
	\lift{w}{y!(z)}\widehat{\id{\{}u / z \id{\}}}
		& = &
		\lift{w}{y!(u)} \nonumber\\
	w[ \lpquote y!(z) \rpquote ] \widehat{ \id{\{}u / z \id{\}} }
		& = &
		w[ \lpquote y!(z) \rpquote ] \nonumber
\end{eqnarray}

Because the body of the process between quotes is impervious to
substitution, we get radically different answers. In fact, by
examining the first process in an input context,
e.g. $x?(z).\lift{w}{y!(z)}$, we see that the process under the lift
operator may be shaped by prefixed inputs binding a name inside it. In
this sense, the lift operator will be seen as a way to dynamically
construct processes before reifying them as names.

Finally equipped with these standard features we can present the
dynamics of the calculus.

\subsubsection{Operational semantics} 

Finally, we introduce the computational dynamics. What marks these
algebras as distinct from other more traditionally studied algebraic
structures, e.g. vector spaces or polynomial rings, is the manner in
which dynamics is captured. In traditional structures, dynamics is typically
expressed through morphisms between such structures, as in linear maps
between vector spaces or morphisms between rings. In algebras
associated with the semantics of computation, the dynamics is
expressed as part of the algebraic structure itself, through a
reduction reduction relation typically denoted by $\red$. Below, we
give a recursive presentation of this relation for the calculus used
in the encoding.

$\red \subseteq \pi \times \pi$
$\red : \pi \to \mathcal{P}(\pi)$

\begin{mathpar}
  \inferrule* [lab=Comm] { \textsf{match}( x_{src}, x_{trgt} ) } { x_{trgt}?(y)P \; | \; x_{src}!\langle {Q} \rangle \red P\{\quotep{Q}/y}\} }
  \and \\
  \inferrule* [lab=Par] {{P} \red {P}'} {{{P} | {Q}} \red {{P}' | {Q}}}
  \and
  \inferrule* [lab=Equiv]{{{P} \scong {P}'} \andalso {{P}' \red {Q}'} \andalso {{Q}' \scong {Q}}}{{P} \red {Q}}
\end{mathpar}

\begin{eqnarray*}
  match_{\equiv} (\quotep{P},\quotep{Q}) & := & P \equiv Q \\
  match_{\dagger}(\quotep{P},\quotep{Q}) & := & \forall R. P|Q \red^{*} R => R \red^{*} 0 \\
  match_{K}(\quotep{P},\quotep{Q}) & := & K \mbox{ for some context } K
\end{eqnarray*}

$u?(x)P | u!\langle Q \rangle \red P\{\quotep{Q}/x\}$

%We write $\wred$ for $\red^*$, and $P\red$ if $\exists Q $ such that $ P \red Q$.
We write $P\red$ if $\exists Q $ such that $ P \red Q$ and $P\not\red$, otherwise.

\section{Replication}

As mentioned before, it is known that replication (and hence
recursion) can be implemented in a higher-order process algebra
\cite{SangiorgiWalker}. As our first example of calculation with the
machinery thus far presented we give the construction explicitly in
the {\rhoc}.

\begin{eqnarray}
	D_{x} & := & \prefix{x}{y}{(\binpar{\outputp{x}{y}}{@{y}})} \nonumber\\
	\bangp_{x}{P} & := & \binpar{{x}!\langle{\binpar{D_{x}}{P}}\rangle}{D_{x}} \nonumber
\end{eqnarray}

\begin{eqnarray}
	\bangp_{x}{P} & & \nonumber\\
	=
	& {x}!\langle{(\prefix{x}{y}{(\outputp{x}{y} | @{y})) | P}}\rangle 
	      | \prefix{x}{y}{(\outputp{x}{y} | @{y})} & \nonumber\\
	\red
	& (\outputp{x}{y} | @{y})\substn{\quotep{(\prefix{x}{y}{(@{y} | \outputp{x}{y})) | P}}}{y} & \nonumber\\
	=
	& \outputp{x}{\quotep{(\prefix{x}{y}{(\outputp{x}{y} | @{y})) | P}}}
	  | {(\prefix{x}{y}{(\outputp{x}{y} | @{y})) | P}} & \nonumber\\
	\red
	& \ldots & \nonumber\\
	\red^*
	& P | P | \ldots & \nonumber
\end{eqnarray}

Of course, this encoding, as an implementation, runs away, unfolding
$\bangp{P}$ eagerly. A lazier and more implementable replication
operator, restricted to input-guarded processes, may be obtained as follows.

\begin{eqnarray}
\bangp{\prefix{u}{v}{P}} 
	:= 
	\binpar{\lift{x}{\prefix{u}{v}{(\binpar{D(x)}{P})}}}{D(x)} \nonumber
\end{eqnarray}

\begin{remark}
  Note that the lazier definition still does not deal with summation
  or mixed summation (i.e. sums over input and output). The reader is
  invited to construct definitions of replication that deal with these
  features. 

  Further, the definitions are parameterized in a name, $x$. Can you,
  gentle reader, make a definition that eliminates this parameter and
  guarantees no accidental interaction between the replication
  machinery and the process being replicated -- i.e. no accidental
  sharing of names used by the process to get its work done and the
  name(s) used by the replication to effect copying. This latter
  revision of the definition of replication is crucial to obtaining
  the expected identity $!!P \sim !P$.
\end{remark}

\begin{remark}\label{rem:paradoxical_combinator}
  The reader familiar with the lambda calculus will have noticed the
  similarity between $D$ and the paradoxical combinator.

  [Ed. note: the existence of this seems to suggest we have to be more
  restrictive on the set of processes and names we admit if we are to
  support no-cloning.]
\end{remark}

\subsubsection{Bisimulation}

The computational dynamics gives rise to another kind of equivalence,
the equivalence of computational behavior. As previously mentioned
this is typically captured \emph{via} some form of bisimulation.

% The notion we use in this paper is weak barbed bisimulation
% \cite{milner91polyadicpi}.

The notion we use in this paper is derived from weak barbed
bisimulation \cite{milner91polyadicpi}. 

\begin{definition}
An \emph{observation relation}, $\downarrow_{\mathcal N}$, over a set
of names, $\mathcal N$, is the smallest relation satisfying the rules
below.

\infrule[Out-barb]{y \in {\mathcal N}, \; x \nameeq y}
		  {\outputp{x}{v} \downarrow_{\mathcal N} x}
\infrule[Par-barb]{\mbox{$P\downarrow_{\mathcal N} x$ or $Q\downarrow_{\mathcal N} x$}}
		  {\binpar{P}{Q} \downarrow_{\mathcal N} x}

We write $P \Downarrow_{\mathcal N} x$ if there is $Q$ such that 
$P \wred Q$ and $Q \downarrow_{\mathcal N} x$.
\end{definition}

\begin{definition}
%\label{def.bbisim}
An  ${\mathcal N}$-\emph{barbed bisimulation} over a set of names, ${\mathcal N}$, is a symmetric binary relation 
${\mathcal S}_{\mathcal N}$ between agents such that $P\rel{S}_{\mathcal N}Q$ implies:
\begin{enumerate}
\item If $P \red P'$ then $Q \wred Q'$ and $P'\rel{S}_{\mathcal N} Q'$.
\item If $P\downarrow_{\mathcal N} x$, then $Q\Downarrow_{\mathcal N} x$.
\end{enumerate}
$P$ is ${\mathcal N}$-barbed bisimilar to $Q$, written
$P \wbbisim_{\mathcal N} Q$, if $P \rel{S}_{\mathcal N} Q$ for some ${\mathcal N}$-barbed bisimulation ${\mathcal S}_{\mathcal N}$.
\end{definition}

$\mathcal{R} \subseteq \pi \times \pi$

$P \mathcal{R} Q => \forall P'. P \red P' \Rightarrow \exists Q'. Q \red Q', P' \mathcal{R} Q'$

$P \vdash x \Rightarrow Q \vdash x$

\begin{mathpar}
  \inferrule*[lab=Out-barb]{x \nameeq y}{{y}!\langle{Q}\rangle \vdash x}
  \and
  \inferrule*[lab=Par-barb]{\mbox{$P\vdash x$ or $Q\vdash x$}}{\binpar{P}{Q} \vdash x}
\end{mathpar}

\subsubsection{Contexts}

One of the principle advantages of computational calculi like the
$\pi$-calculus is a well-defined notion of context,
contextual-equivalence and a correlation between
contextual-equivalence and notions of bisimulation. The notion of
context allows the decomposition of a process into (sub-)process and
its syntactic environment, its context. Thus, a context may be
thought of as a process with a ``hole'' (written $\Box$) in it. The
application of a context $M$ to a process $P$, written $M[P]$, is
tantamount to filling the hole in $M$ with $P$. In this paper we do
not need the full weight of this theory, but do make use of the notion
of context in the proof the main theorem. 

\begin{mathpar}
  \inferrule* [lab=summation] {} {{M_{M},M_{N}} \bc \Box \;|\; x.M_{A} \;|\; M_{M}+M_{N}}
  \and
  \inferrule* [lab=agent] {} {{M_{A}} \bc (\vec{x})M_{P} \;| \; \clift{P_0,\ldots,M_{P},\ldots,P_N}}
  \and \\
  \inferrule* [lab=process] {} {{M_{P}} \bc M_{N} \;| \;P|M_{P} }
\end{mathpar} 

\begin{mathpar}
  \inferrule* [lab=sychronization] {} {M_{N} \bc \Box \;|\; x?M_{F} \;|\; x!M_{C}}
  \and
  \inferrule* [lab=abstraction] {} {{M_{F}} \bc (x)M_{P} }
  \and
  \inferrule* [lab=concretion] {} {{M_{C}} \bc \langle M_{P} \rangle }
  \and \\
  \inferrule* [lab=process] {} {{M_{P}} \bc M_{N} \;| \;P|M_{P} }
\end{mathpar}

\begin{definition}[contextual application] Given a context $M$, and
  process $P$, we define the \emph{contextual application}, $M[P] :=
  M\{P/\Box\}$. That is, the contextual application of M to P is the
  substitution of $P$ for $\Box$ in $M$.
\end{definition}

$\meaningof{-} : L \to \mathcal{P}(\pi)$

\begin{mathpar}
  \inferrule* [lab=collection] {} {\meaningof{true} = \pi, \and \meaningof{~E} = \pi \setminus \meaningof{E}, \and \meaningof{E_{1} \& E_{2}} = \meaningof{E_{1}} \cap \meaningof{E_{2}}}
\end{mathpar}

\begin{mathpar}
  \inferrule* [lab=structure] {} {\meaningof{0} = \{ P \in \pi | P \equiv 0 \}, \and \\ \meaningof{E_1 | E_2} = \{ P \in \pi | P \equiv P_{1} | P_{2}, P_{1} \in \meaningof{E_{1}}, P_{2} \in \meaningof{E_2}\} }
\end{mathpar}

\begin{mathpar}
 \inferrule* [lab=behavior] {} {\meaningof{\langle a?b \rangle E} = \{ P \in \pi | P \equiv Q | u?(y)P', \\ \and \\\\ \and \\ \;\;\; u \in \meaningof{a}, \forall z.P'\{z/y\} \in \meaningof{E\{z/b\}}\}, \and \\ \meaningof{a!E} = \{ P \in \pi | P \equiv Q | x!\langle P' \rangle, x \in \meaningof{a} P' \in \meaningof{E}\} }
\end{mathpar}

\begin{mathpar}
 \inferrule* [lab=nominal] {} {\meaningof{\quotep{E}} = \{ \quotep{P} \in \quotep{\pi} | P \in \meaningof{E} \}, \and \meaningof{\quotep{P}} = \{ \quotep{Q} \in \quotep{\pi} | P \equiv Q \} \and \\ \meaningof{@\quotep{E}} = \{ P \in \pi | P \equiv @x, x \in \meaningof{E} \}}
\end{mathpar}

\begin{eqnarray*}
  \\
  \meaningof{-} : TS \to ST
\end{eqnarray*}

\begin{eqnarray*}
  \\
  L : TS \to ST
\end{eqnarray*}

\begin{eqnarray*}
  \\
  P \models E \iff P \in \meaningof{E}
\end{eqnarray*}

\begin{eqnarray*}
  P \approx_{L} Q \iff \forall E \in L. P \models E \iff Q \models E
\end{eqnarray*}

\begin{eqnarray*}
  P \approx_{K} Q
\end{eqnarray*}

\begin{eqnarray*}
  P \approx Q
\end{eqnarray*}

$\approx_{K} = \approx = \approx_{L}$

\subsubsection{Contextual duality}

Note that contexts extend the quotation operation to a family of
operations from processes to names. Given a context, $M$, we can
define a \emph{nominal context}, $\quotep{M}$ by $\quotep{M}[P] :=
\quotep{M[P]}$. To foreshadow what is to come we observe that these
operations enjoy a duality with processes very much like the duality
between vectors and maps from vectors to scalars.

Further, because the calculus is essentially higher-order, we have a
correspondence between contexts and processes. More specifically,
given a name $x$ and a context $M$ we can construct $M^{*}_{x}$ such
that 

\begin{mathpar}
  M^{*}_{x} | \lift{x}{P} \red M[P]
\end{mathpar}

namely,

\begin{mathpar}
  M^{*}_{x} := x?(u).M[\dropn{u}]
\end{mathpar}

The dependence of $M^{*}_{x}$ on a name makes it an abstraction, 

\begin{mathpar}
  M^{*} := (x)x?(u).M[\dropn{u}]
\end{mathpar}

\subsection{Additional notation}

It will sometimes be convenient to denote the process a name
quotes. We already have the notation $x = \quotep{P}$, but it will be
convenient to introduce an alternate notation, $\procn{x}$, when we
want to emphasize the connection to the use of the name. Note that, by
virtue of name equivalence, $\quotep{\procn{x}} \nameeq x$; so, the
notation is consistent with previous definitions.

Further, because names have structure it is possible to effect
substitutions on the basis of that structure. This means we need to
upgrade our notation for substitutions, which we accomplish by
adapting comprehension notation. Thus,

\begin{mathpar}
  P\{ y / x : x \in S \}
\end{mathpar}

is interpreted to mean the process derived from P by replacing (in a
capture-avoiding manner) each occurrence of $x$ in $S$ by $y$. For example,

\begin{mathpar}
  P\{ \quotep{\procn{x}|\procn{x}} / x : x \in \freenames{P} \}
\end{mathpar}

will replace each (occurrence) of a free name $x$ in $P$ by
$\quotep{\procn{x}|\procn{x}}$.

Also, we will avail ourselves of the notation $x^{L}$ and $x^{R}$ to
denote injections of a name into disjoint copies of the name
space. There are numerous ways to accomplish this. One example can be
found in \cite{MeredithR05}. This notation overloads to vectors of
names: $\vec{x}^{\pi} := (x_{i}^{\pi} \; : \; 0 \leq i < |\vec{x}| )$ where $\pi \in \{L,R\}$.

We also use $P^{\Box} := P|\Box$.

In \cite{MeredithR05} an interpretation of the new operator is
given. It turns out that there are several possible interpretations
all enjoying the requisite algebraic properties of the operator (see
\cite{milner91polyadicpi}). We will therefore make liberal use of
$(\nu\; \vec{x})P$.

% subsection the_syntax_and_semantics_of_the_notation_system (end)   

\input{qm2pi.qmops} 

\input{qm2pi.sterngerlach} 

\input{qm2pi.metric} 

% section concurrent_process_calculi (end)

%\input{qm2pi.proofsketch}

% section proof sketch (end)

%\input{qm2pi.slviaknots} 

% section spatial logic via knots (end)

\input{qm2pi.conclusion}

% section conclusion (end)

%\input{qm2pi.dtcodes} 

% section wiring algorithm (end)

\input{qm2pi.ack} 

% section acknowledgments (end)

\newpage


\bibliographystyle{plain}   
\bibliography{../../biblios/main.bib}

\input{qm2pi.rhodetails}

\end{document}

 

%\ifpdf
%\usepackage[pdftex]{graphicx}
%\else
%\usepackage{graphicx}
%\fi

 % \ifpdf
%  \usepackage{pdfsync}
%  \if


%\title{Brief Article}
%\author{David F. Snyder}
%\author{L.G. Meredith}

%\address{Dept. of Math., Texas State University--San Marcos, San Marcos, TX 78666}
       
\pagestyle{empty}


\begin{document}

\lstset{language=[Objective]Caml,frame=shadowbox}

\documentclass[12pt]{llncs}
%\documentclass{jktr}

\usepackage[pdftex]{hyperref}                   
\usepackage {listings}
\usepackage {mathpartir}
\usepackage{bcprules}
%\usepackage{listings}
                       
\usepackage{graphicx} 
%\usepackage[margins=2.5cm,nohead,nofoot]{geometry}
%\usepackage{geometry}
\usepackage{amsfonts}
\usepackage{amstext}
\usepackage{latexsym}
\usepackage{amssymb}
\usepackage{color}


%\include{myPreamble}
\include{qm2pi.local} 

%\ifpdf
%\usepackage[pdftex]{graphicx}
%\else
%\usepackage{graphicx}
%\fi

 % \ifpdf
%  \usepackage{pdfsync}
%  \if


%\title{Brief Article}
%\author{David F. Snyder}
%\author{L.G. Meredith}

%\address{Dept. of Math., Texas State University--San Marcos, San Marcos, TX 78666}
       
\pagestyle{empty}


\begin{document}

\lstset{language=[Objective]Caml,frame=shadowbox}

\input{qm2pi.front}

% section front matter (end)

\input{qm2pi.intro} 
 
% section introduction (end)

% \input{qm2pi.knotations} 

% section notation (end)

\input{qm2pi.process.calculi} 

% section concurrent_process_calculi_and_spatial_logics_ (end)
    
%\input{qm2pi.knots2pi} 

%\input{qm2pi.trefoil} 

%\input{qm2pi.mainthm} 

% subsection basic_interpretation (end)

%\input{qm2pi.rho.presentation} 
\subsection{The syntax and semantics of the notation system}\label{sub:the_syntax_and_semantics_of_the_notation_system} % (fold)

We now summarize a technical presentation of the calculus that
embodies our theory of dynamics. The typical presentation of such a
calculus follows the style of giving generators and relations on
them. The grammar, below, describing term constructors, freely
generates the set of processes, $\Proc$. This set is then quotiented
by a relation known as structural congruence and it is over this set
that the notion of dynamics is expressed. This presentation is
essentially that of \cite{MeredithR05} with the addition of
polyadicity and summation. For readability we have relegated some of
the technical subtleties to an appendix.

\subsubsection{Process grammar}\label{subsub:process_grammar}

\begin{mathpar}
  \inferrule* [lab=synchronization] {} {{M} \bc \pzero \;|\; x?F \;|\; x!C }
  \and
  \inferrule* [lab=abstraction] {} {{F} \bc (x)P}
  \and
  \inferrule* [lab=concretion] {} {{C} \bc \langle Q \rangle}
  \and
  \inferrule* [lab=process] {} {{P,Q} \bc M \;| \;P|Q \;|\; @{x}}
  \and
  \inferrule* [lab=name] {} {{x} \bc \quotep{P}}
\end{mathpar} 

Note that $\vec{x}$ (resp. $\vec{P}$) denotes a vector of names
(resp. processes) of length $|\vec{x}|$ (resp. $|\vec{P}|$). We adopt
the following useful abbreviations.

\begin{mathpar}
   x?(\vec{y}).P := x.(\vec{y})P \and  x\clift{\vec{P}} := x.\clift{\vec{P}}
   \and x!(y) := \lift{x}{\dropn{y}}
   \and \Pi_{i=0}^{n-1}P_i := P_0 | \ldots | P_{n-1}
\end{mathpar}

\subsubsection{Structural congruence}

\paragraph{Free and bound names and alpha-equivalence.} At the
core of structural equivalence is alpha-equivalence which identifies
process that are the same up to a change of variable. Formally, we
recognize the distinction between free and bound names. The free names
of a process, $\freenames{P}$, may be calculated recursively as
follows:

\begin{mathpar}
\freenames{\pzero} := \emptyset
  \and \\
  \freenames{x?(y).P} := \{ x \} \cup (\freenames{P} \setminus \{ y \})
  \and 
  \freenames{x!\langle P \rangle} := \{ x \} \cup \{ P \} 
  \and \\
  \freenames{P|Q} := \freenames{P} \cup \freenames{Q}
  \and \\
  \freenames{@{x}} := \{ x \}
\end{mathpar}

$\pi$
$\quotep{\pi}$

$\freenames{-} : \pi \to \mathcal{P}(\quotep{\pi})$

\begin{eqnarray*}
  \freenames{\pzero} & := & \emptyset \\
  \freenames{x?(y).P} & := & \{ x \} \cup (\freenames{P} \setminus \{ y \}) \\
  \freenames{x!\langle P \rangle} & := & \{ x \} \cup \{ P \} \\
  \freenames{P|Q} & := & \freenames{P} \cup \freenames{Q} \\
  \freenames{\dropn{x}} & := & \{ x \}
\end{eqnarray*}

The bound names of a process, $\boundnames{P}$, are those names occurring in $P$
that are not free. For example, in $x?(y).0$, the name $x$ is free, while $y$ is bound.

\begin{mathpar}
  \inferrule* [lab=monoidal-laws] {} { P|Q \equiv Q|P \and P|0 \equiv P \and P|(Q|R) \equiv (P|Q)|R }
\end{mathpar}

\begin{mathpar}
  \inferrule* [lab=alpha-equivalence] {} { (x)P \equiv (y)P\{y/x\} \and y \not\in \freenames{P} }
\end{mathpar}

\begin{definition}
Then two processes, $P,Q$, are alpha-equivalent if $P = Q\{\vec{y}/\vec{x}\}$ for
some $\vec{x} \in \boundnames{Q},\vec{y} \in \boundnames{P}$, where $Q\{\vec{y}/\vec{x}\}$
denotes the capture-avoiding substitution of $\vec{y}$ for $\vec{x}$ in $Q$.
\end{definition}

\begin{definition}
  The {\em structural congruence} \cite{SangiorgiWalker} , $\equiv$,
  between processes is the least congruence containing
  alpha-equivalence, satisfying the abelian monoid laws
  (associativity, commutativity and $\pzero$ as identity) for parallel
  composition $|$ and for summation $+$.
\end{definition}

\subsection{Name equivalence}

We take name equivalence, written $\nameeq$, to be the smallest
equivalence relation generated by the following rules.

\begin{mathpar}
\inferrule*[lab=Quote-drop]
{ }
{ \quotep{@{x}} \nameeq x }

\inferrule*[lab=Struct-equiv]
{ P \scong Q }
{ \quotep{P} \nameeq \quotep{Q} }
\end{mathpar}

The astute reader will have noticed that the mutual recursion of names
and processes imposes a mutual recursion on alpha-equivalence and
structural equivalence via name-equivalence. Fortunately, all of this
works out pleasantly and we may calculate in the natural way, free of
concern. The reader interested in the details is referred to the
appendix \ref{appendix:rho_details}.

\subsection{Substitution}

We use $\Proc$ for the set of processes, $\QProc$ for the set of
names, and $\id{\{}\vec{y} / \vec{x} \id{\}}$ to denote partial maps,
$s : \QProc \rightarrow \QProc$. A map, $s$ lifts, uniquely, to a map
on process terms, $\widehat{s} : \Proc \rightarrow \Proc$ by the
following equations.

\begin{mathpar}
  (0) \psubstp{Q}{P} := 0 \\
  (R \juxtap S) \psubstp{Q}{P}
  :=    
  (R)\psubstp{Q}{P} \juxtap (S) \psubstp{Q}{P} \\
  (x?(y).R) \psubstp{Q}{P}    
  :=    
  (x)\substp{Q}{P} (z)\concat( (R \psubstn{z}{y}) \psubstp{Q}{P} ) \\
  (\lift{x}{R}) \psubstp{Q}{P}  
  :=
  \lift{(x)\substp{Q}{P}}{ R \psubstp{Q}{P} } \\
%   (\dropn{x})  \psubstp{Q}{P}       
%   := 
%   \left\{ 
%     \begin{array}{ccc} 
%       \dropn{\quotep{Q}} & & x \nameeq \quotep{P} \\
%       \dropn{x} & & otherwise \\
%     \end{array}
%   \right. 
  (\dropn{x})  \psubstp{Q}{P}       
  := 
  \left\{ 
    \begin{array}{ccc} 
      Q & & x \nameeq \quotep{P} \\
      \dropn{x} & & otherwise \\
    \end{array}
  \right.
\end{mathpar}
 

where

\begin{eqnarray}
  (x)\id{\{} \lpquote Q \rpquote / \lpquote P \rpquote \id{\}}            = 
  \left\{ 
    \begin{array}{ccc}
      \lpquote Q \rpquote & & x \nameeq \lpquote P \rpquote \\
      x & & otherwise \\
    \end{array}
  \right. \nonumber
\end{eqnarray}

and $z$ is chosen distinct from $\quotep{P}$, $\quotep{Q}$, the free
names in $Q$, and all the names in $R$. Our $\alpha$-equivalence will
be built in the standard way from this substitution.

\begin{remark}\label{rem:no_self_referential_names}
  One consequence of these definitions is that $\forall P. \quotep{P}
  \not\in \freenames{P}$.
\end{remark}

\subsection{ Dynamic quote: an example }

Anticipating something of what's to come, consider applying the
substitution, $\widehat{\id{\{}u / z \id{\}}}$, to the following pair
of processes, $\lift{w}{y!(z)}$ and $w[ \lpquote y!(z) \rpquote ]$.

\begin{eqnarray}
	\lift{w}{y!(z)}\widehat{\id{\{}u / z \id{\}}}
		& = &
		\lift{w}{y!(u)} \nonumber\\
	w[ \lpquote y!(z) \rpquote ] \widehat{ \id{\{}u / z \id{\}} }
		& = &
		w[ \lpquote y!(z) \rpquote ] \nonumber
\end{eqnarray}

Because the body of the process between quotes is impervious to
substitution, we get radically different answers. In fact, by
examining the first process in an input context,
e.g. $x?(z).\lift{w}{y!(z)}$, we see that the process under the lift
operator may be shaped by prefixed inputs binding a name inside it. In
this sense, the lift operator will be seen as a way to dynamically
construct processes before reifying them as names.

Finally equipped with these standard features we can present the
dynamics of the calculus.

\subsubsection{Operational semantics} 

Finally, we introduce the computational dynamics. What marks these
algebras as distinct from other more traditionally studied algebraic
structures, e.g. vector spaces or polynomial rings, is the manner in
which dynamics is captured. In traditional structures, dynamics is typically
expressed through morphisms between such structures, as in linear maps
between vector spaces or morphisms between rings. In algebras
associated with the semantics of computation, the dynamics is
expressed as part of the algebraic structure itself, through a
reduction reduction relation typically denoted by $\red$. Below, we
give a recursive presentation of this relation for the calculus used
in the encoding.

$\red \subseteq \pi \times \pi$
$\red : \pi \to \mathcal{P}(\pi)$

\begin{mathpar}
  \inferrule* [lab=Comm] { \textsf{match}( x_{src}, x_{trgt} ) } { x_{trgt}?(y)P \; | \; x_{src}!\langle {Q} \rangle \red P\{\quotep{Q}/y}\} }
  \and \\
  \inferrule* [lab=Par] {{P} \red {P}'} {{{P} | {Q}} \red {{P}' | {Q}}}
  \and
  \inferrule* [lab=Equiv]{{{P} \scong {P}'} \andalso {{P}' \red {Q}'} \andalso {{Q}' \scong {Q}}}{{P} \red {Q}}
\end{mathpar}

\begin{eqnarray*}
  match_{\equiv} (\quotep{P},\quotep{Q}) & := & P \equiv Q \\
  match_{\dagger}(\quotep{P},\quotep{Q}) & := & \forall R. P|Q \red^{*} R => R \red^{*} 0 \\
  match_{K}(\quotep{P},\quotep{Q}) & := & K \mbox{ for some context } K
\end{eqnarray*}

$u?(x)P | u!\langle Q \rangle \red P\{\quotep{Q}/x\}$

%We write $\wred$ for $\red^*$, and $P\red$ if $\exists Q $ such that $ P \red Q$.
We write $P\red$ if $\exists Q $ such that $ P \red Q$ and $P\not\red$, otherwise.

\section{Replication}

As mentioned before, it is known that replication (and hence
recursion) can be implemented in a higher-order process algebra
\cite{SangiorgiWalker}. As our first example of calculation with the
machinery thus far presented we give the construction explicitly in
the {\rhoc}.

\begin{eqnarray}
	D_{x} & := & \prefix{x}{y}{(\binpar{\outputp{x}{y}}{@{y}})} \nonumber\\
	\bangp_{x}{P} & := & \binpar{{x}!\langle{\binpar{D_{x}}{P}}\rangle}{D_{x}} \nonumber
\end{eqnarray}

\begin{eqnarray}
	\bangp_{x}{P} & & \nonumber\\
	=
	& {x}!\langle{(\prefix{x}{y}{(\outputp{x}{y} | @{y})) | P}}\rangle 
	      | \prefix{x}{y}{(\outputp{x}{y} | @{y})} & \nonumber\\
	\red
	& (\outputp{x}{y} | @{y})\substn{\quotep{(\prefix{x}{y}{(@{y} | \outputp{x}{y})) | P}}}{y} & \nonumber\\
	=
	& \outputp{x}{\quotep{(\prefix{x}{y}{(\outputp{x}{y} | @{y})) | P}}}
	  | {(\prefix{x}{y}{(\outputp{x}{y} | @{y})) | P}} & \nonumber\\
	\red
	& \ldots & \nonumber\\
	\red^*
	& P | P | \ldots & \nonumber
\end{eqnarray}

Of course, this encoding, as an implementation, runs away, unfolding
$\bangp{P}$ eagerly. A lazier and more implementable replication
operator, restricted to input-guarded processes, may be obtained as follows.

\begin{eqnarray}
\bangp{\prefix{u}{v}{P}} 
	:= 
	\binpar{\lift{x}{\prefix{u}{v}{(\binpar{D(x)}{P})}}}{D(x)} \nonumber
\end{eqnarray}

\begin{remark}
  Note that the lazier definition still does not deal with summation
  or mixed summation (i.e. sums over input and output). The reader is
  invited to construct definitions of replication that deal with these
  features. 

  Further, the definitions are parameterized in a name, $x$. Can you,
  gentle reader, make a definition that eliminates this parameter and
  guarantees no accidental interaction between the replication
  machinery and the process being replicated -- i.e. no accidental
  sharing of names used by the process to get its work done and the
  name(s) used by the replication to effect copying. This latter
  revision of the definition of replication is crucial to obtaining
  the expected identity $!!P \sim !P$.
\end{remark}

\begin{remark}\label{rem:paradoxical_combinator}
  The reader familiar with the lambda calculus will have noticed the
  similarity between $D$ and the paradoxical combinator.

  [Ed. note: the existence of this seems to suggest we have to be more
  restrictive on the set of processes and names we admit if we are to
  support no-cloning.]
\end{remark}

\subsubsection{Bisimulation}

The computational dynamics gives rise to another kind of equivalence,
the equivalence of computational behavior. As previously mentioned
this is typically captured \emph{via} some form of bisimulation.

% The notion we use in this paper is weak barbed bisimulation
% \cite{milner91polyadicpi}.

The notion we use in this paper is derived from weak barbed
bisimulation \cite{milner91polyadicpi}. 

\begin{definition}
An \emph{observation relation}, $\downarrow_{\mathcal N}$, over a set
of names, $\mathcal N$, is the smallest relation satisfying the rules
below.

\infrule[Out-barb]{y \in {\mathcal N}, \; x \nameeq y}
		  {\outputp{x}{v} \downarrow_{\mathcal N} x}
\infrule[Par-barb]{\mbox{$P\downarrow_{\mathcal N} x$ or $Q\downarrow_{\mathcal N} x$}}
		  {\binpar{P}{Q} \downarrow_{\mathcal N} x}

We write $P \Downarrow_{\mathcal N} x$ if there is $Q$ such that 
$P \wred Q$ and $Q \downarrow_{\mathcal N} x$.
\end{definition}

\begin{definition}
%\label{def.bbisim}
An  ${\mathcal N}$-\emph{barbed bisimulation} over a set of names, ${\mathcal N}$, is a symmetric binary relation 
${\mathcal S}_{\mathcal N}$ between agents such that $P\rel{S}_{\mathcal N}Q$ implies:
\begin{enumerate}
\item If $P \red P'$ then $Q \wred Q'$ and $P'\rel{S}_{\mathcal N} Q'$.
\item If $P\downarrow_{\mathcal N} x$, then $Q\Downarrow_{\mathcal N} x$.
\end{enumerate}
$P$ is ${\mathcal N}$-barbed bisimilar to $Q$, written
$P \wbbisim_{\mathcal N} Q$, if $P \rel{S}_{\mathcal N} Q$ for some ${\mathcal N}$-barbed bisimulation ${\mathcal S}_{\mathcal N}$.
\end{definition}

$\mathcal{R} \subseteq \pi \times \pi$

$P \mathcal{R} Q => \forall P'. P \red P' \Rightarrow \exists Q'. Q \red Q', P' \mathcal{R} Q'$

$P \vdash x \Rightarrow Q \vdash x$

\begin{mathpar}
  \inferrule*[lab=Out-barb]{x \nameeq y}{{y}!\langle{Q}\rangle \vdash x}
  \and
  \inferrule*[lab=Par-barb]{\mbox{$P\vdash x$ or $Q\vdash x$}}{\binpar{P}{Q} \vdash x}
\end{mathpar}

\subsubsection{Contexts}

One of the principle advantages of computational calculi like the
$\pi$-calculus is a well-defined notion of context,
contextual-equivalence and a correlation between
contextual-equivalence and notions of bisimulation. The notion of
context allows the decomposition of a process into (sub-)process and
its syntactic environment, its context. Thus, a context may be
thought of as a process with a ``hole'' (written $\Box$) in it. The
application of a context $M$ to a process $P$, written $M[P]$, is
tantamount to filling the hole in $M$ with $P$. In this paper we do
not need the full weight of this theory, but do make use of the notion
of context in the proof the main theorem. 

\begin{mathpar}
  \inferrule* [lab=summation] {} {{M_{M},M_{N}} \bc \Box \;|\; x.M_{A} \;|\; M_{M}+M_{N}}
  \and
  \inferrule* [lab=agent] {} {{M_{A}} \bc (\vec{x})M_{P} \;| \; \clift{P_0,\ldots,M_{P},\ldots,P_N}}
  \and \\
  \inferrule* [lab=process] {} {{M_{P}} \bc M_{N} \;| \;P|M_{P} }
\end{mathpar} 

\begin{mathpar}
  \inferrule* [lab=sychronization] {} {M_{N} \bc \Box \;|\; x?M_{F} \;|\; x!M_{C}}
  \and
  \inferrule* [lab=abstraction] {} {{M_{F}} \bc (x)M_{P} }
  \and
  \inferrule* [lab=concretion] {} {{M_{C}} \bc \langle M_{P} \rangle }
  \and \\
  \inferrule* [lab=process] {} {{M_{P}} \bc M_{N} \;| \;P|M_{P} }
\end{mathpar}

\begin{definition}[contextual application] Given a context $M$, and
  process $P$, we define the \emph{contextual application}, $M[P] :=
  M\{P/\Box\}$. That is, the contextual application of M to P is the
  substitution of $P$ for $\Box$ in $M$.
\end{definition}

$\meaningof{-} : L \to \mathcal{P}(\pi)$

\begin{mathpar}
  \inferrule* [lab=collection] {} {\meaningof{true} = \pi, \and \meaningof{~E} = \pi \setminus \meaningof{E}, \and \meaningof{E_{1} \& E_{2}} = \meaningof{E_{1}} \cap \meaningof{E_{2}}}
\end{mathpar}

\begin{mathpar}
  \inferrule* [lab=structure] {} {\meaningof{0} = \{ P \in \pi | P \equiv 0 \}, \and \\ \meaningof{E_1 | E_2} = \{ P \in \pi | P \equiv P_{1} | P_{2}, P_{1} \in \meaningof{E_{1}}, P_{2} \in \meaningof{E_2}\} }
\end{mathpar}

\begin{mathpar}
 \inferrule* [lab=behavior] {} {\meaningof{\langle a?b \rangle E} = \{ P \in \pi | P \equiv Q | u?(y)P', \\ \and \\\\ \and \\ \;\;\; u \in \meaningof{a}, \forall z.P'\{z/y\} \in \meaningof{E\{z/b\}}\}, \and \\ \meaningof{a!E} = \{ P \in \pi | P \equiv Q | x!\langle P' \rangle, x \in \meaningof{a} P' \in \meaningof{E}\} }
\end{mathpar}

\begin{mathpar}
 \inferrule* [lab=nominal] {} {\meaningof{\quotep{E}} = \{ \quotep{P} \in \quotep{\pi} | P \in \meaningof{E} \}, \and \meaningof{\quotep{P}} = \{ \quotep{Q} \in \quotep{\pi} | P \equiv Q \} \and \\ \meaningof{@\quotep{E}} = \{ P \in \pi | P \equiv @x, x \in \meaningof{E} \}}
\end{mathpar}

\begin{eqnarray*}
  \\
  \meaningof{-} : TS \to ST
\end{eqnarray*}

\begin{eqnarray*}
  \\
  L : TS \to ST
\end{eqnarray*}

\begin{eqnarray*}
  \\
  P \models E \iff P \in \meaningof{E}
\end{eqnarray*}

\begin{eqnarray*}
  P \approx_{L} Q \iff \forall E \in L. P \models E \iff Q \models E
\end{eqnarray*}

\begin{eqnarray*}
  P \approx_{K} Q
\end{eqnarray*}

\begin{eqnarray*}
  P \approx Q
\end{eqnarray*}

$\approx_{K} = \approx = \approx_{L}$

\subsubsection{Contextual duality}

Note that contexts extend the quotation operation to a family of
operations from processes to names. Given a context, $M$, we can
define a \emph{nominal context}, $\quotep{M}$ by $\quotep{M}[P] :=
\quotep{M[P]}$. To foreshadow what is to come we observe that these
operations enjoy a duality with processes very much like the duality
between vectors and maps from vectors to scalars.

Further, because the calculus is essentially higher-order, we have a
correspondence between contexts and processes. More specifically,
given a name $x$ and a context $M$ we can construct $M^{*}_{x}$ such
that 

\begin{mathpar}
  M^{*}_{x} | \lift{x}{P} \red M[P]
\end{mathpar}

namely,

\begin{mathpar}
  M^{*}_{x} := x?(u).M[\dropn{u}]
\end{mathpar}

The dependence of $M^{*}_{x}$ on a name makes it an abstraction, 

\begin{mathpar}
  M^{*} := (x)x?(u).M[\dropn{u}]
\end{mathpar}

\subsection{Additional notation}

It will sometimes be convenient to denote the process a name
quotes. We already have the notation $x = \quotep{P}$, but it will be
convenient to introduce an alternate notation, $\procn{x}$, when we
want to emphasize the connection to the use of the name. Note that, by
virtue of name equivalence, $\quotep{\procn{x}} \nameeq x$; so, the
notation is consistent with previous definitions.

Further, because names have structure it is possible to effect
substitutions on the basis of that structure. This means we need to
upgrade our notation for substitutions, which we accomplish by
adapting comprehension notation. Thus,

\begin{mathpar}
  P\{ y / x : x \in S \}
\end{mathpar}

is interpreted to mean the process derived from P by replacing (in a
capture-avoiding manner) each occurrence of $x$ in $S$ by $y$. For example,

\begin{mathpar}
  P\{ \quotep{\procn{x}|\procn{x}} / x : x \in \freenames{P} \}
\end{mathpar}

will replace each (occurrence) of a free name $x$ in $P$ by
$\quotep{\procn{x}|\procn{x}}$.

Also, we will avail ourselves of the notation $x^{L}$ and $x^{R}$ to
denote injections of a name into disjoint copies of the name
space. There are numerous ways to accomplish this. One example can be
found in \cite{MeredithR05}. This notation overloads to vectors of
names: $\vec{x}^{\pi} := (x_{i}^{\pi} \; : \; 0 \leq i < |\vec{x}| )$ where $\pi \in \{L,R\}$.

We also use $P^{\Box} := P|\Box$.

In \cite{MeredithR05} an interpretation of the new operator is
given. It turns out that there are several possible interpretations
all enjoying the requisite algebraic properties of the operator (see
\cite{milner91polyadicpi}). We will therefore make liberal use of
$(\nu\; \vec{x})P$.

% subsection the_syntax_and_semantics_of_the_notation_system (end)   

\input{qm2pi.qmops} 

\input{qm2pi.sterngerlach} 

\input{qm2pi.metric} 

% section concurrent_process_calculi (end)

%\input{qm2pi.proofsketch}

% section proof sketch (end)

%\input{qm2pi.slviaknots} 

% section spatial logic via knots (end)

\input{qm2pi.conclusion}

% section conclusion (end)

%\input{qm2pi.dtcodes} 

% section wiring algorithm (end)

\input{qm2pi.ack} 

% section acknowledgments (end)

\newpage


\bibliographystyle{plain}   
\bibliography{../../biblios/main.bib}

\input{qm2pi.rhodetails}

\end{document}



% section front matter (end)

\section{Introduction}\label{sec:introduction} % (fold)
In this draft of the material i am going to have to dispense with the
usual writing conventions adopted in papers on these topics. i'm going
to have adopt whatever tone i need at the time i'm writing up the
calculations. Sometimes this may be very conversational; others it may
be the barest mathematical grunts; others still it may be that i have
lifted text from one of my other papers because the exposition of some
point was better said there. i hope that my readers are not unduly put
out by this decision. i'm not doing this to flout convention or be
rebellious. i find these calculations very technically challenging. To
keep everything going technically, something has to give; i have to
let go of some cognitive burden. So, the academic writing style --
with all of its trade-offs in terms of facilitating technical
communication -- is what i'm letting go of. Perhaps subsequent drafts
can be tightened and polished, but for now, i'm going to speak as if
we were sitting together in a coffee shop with a laptop, wifi and a
pad of paper and a pencil.

So, here's what i have to say. We -- you and i, comfortably ensconced
in our coffee shop and well-equipped with our tools -- can realize and
carry out the calculations of quantum mechanics over a very different
formal theory of dynamics, a formal theory of dynamics that
corresponds to a theory of concurrent computation with
\emph{reflection}. It has the advantage that the underlying theory is
already `quantized', but supports analogues all of the continuuous
operations. Strikingly, this underlying theory has recently been
connected with a notion of metric that we can show, by calculating
together, coincides with the metric induced by the inner product.

There are a lot of reasons why you might be interested in seeing
calculations of this form. Here's why i'm interested. For the past
several centuries there has been no competitor to the ``Newtonian''
account of dynamics. As a result the predominant share of accounts of
dynamical systems and situations have had to be formulated in terms of
the Newtonian machinery. i view this as an intellectually dangerous
position to occupy. Everything, despite it's intrinsic shape, turns
into a nail to be hit with this hammer. Recently, however, the theory
of computation has matured to the point where we have candidates for
theories of dynamics that offer very different perspective on
reasoning about dynamical systems and situations. Testing these
candidates against very successful accounts of dynamical situations,
like quantum mechanics, is going to give us some sense of how mature
they are and some measure of the quality of these accounts of
dynamics.

\subsection{Summary of contributions and outline of paper}

So, we're going to develop an interpretation of the operations of
quantum mechanics normally interpreted by Hilbert spaces and
operators. We're going to do this over a theory of computation. Note
that this is very different than the usual quantum computation program
which develops notions of computation over quantum mechanics. Rather,
we are developing a story that aligns with Wheeler's slogan: It from
Bit. To do this we will first provide an account of the theory of
computation at play here. Then we will dive into a calculation-driven
interpretation of the operations of quantum mechanics.

The reason we take this approach is that -- until very recently --
there hasn't been an axiomatic account of quantum mechanics. As a
result there has been no sharp delineation of the mathematical theory
supporting interpretation of the physical theory and the physical
theory, itself. So, ambient features of the maths are free to be
exploited (or supressed) without a real accounting of their physical
relevance. There is no sharp statement ``here's the physical theory''
qua \emph{theory} and ``here's the mathematical interpretation''
enabling a judgment of how faithful the interpretation is -- apart
from experimental observation. When there is an axiomatic account we
can judge how well a given mathematical formalism supports an
interpretation of the axioms, independent of
experimentation. Likewise, we can judge how well we have captured our
physical evidence and experience with our axiomatics, independent of
any specific mathematical implementation, with accidental detail that
may or may not have physical significance. 

In lieu of a fully fleshed out and vetted axiomatic account of quantum
mechanics, interpreting the operational notions in service of modeling
physical systems will have to suffice. In other words, we are not in
the business of providing a model of Hilbert spaces and operators. We
are in the business of providing a model of quantum mechanics because
we are motivated by testing our notions of dynamics against physical
theory; and, the predictive calculations of the physical theory must
serve as the best formulation -- shy of a fully fleshed out axiomatic
account -- of the physical theory itself (as they have for scientific
theories since time immemorial). Put another way, despite a
whole-hearted commitment to an It-from-Bit ontology, we are firmly
aligned with the shut-up-and-calculate camp as the best way to obtain
results either from the physical perspective or as a quality assurance
measure of our fledgling theory of dynamics.

In detail, we present a reflective process calculus. Then we develop
intuitive correspondences between the notions available in this
calculus and the usual physical notions supporting quantum mechanical
calculations. Thus, 

\begin{table}[htp]
  \center{
    \fbox{
      \begin{tabular}{c|c}
        quantum mechanics & process calculus \\
        \hline
        scalar & name \\
        state vector & process \\
        dual & contextual duals \\
        matrix & formal sums of process-context-dual pairs \\
        orthogonality & process annihilation \\
        inner product & execution-formula + quoting
      \end{tabular}
    }
  }
  \caption{QM - process calculi correspondences}
\end{table}

Then we tighten up these intuitions to operational definitions. We
employ the Dirac notation as the best proxy we can find for an
abstract syntax of the quantum mechanical notions. The definitions we
develop put us in contact with equational constraints coming from the
theory that we demonstrate the definitions and calculations satisfy.

This puts us in a position to shut up and calculate for the
Stern-Gerlach experimental set up, showing how these predictive
calculations become calculations on processes in our theory of a
reflective process calculus.

Penultimately, we demonstrate that the notion of metric coming from
the inner product coincides with the notion of metric available from
the theory of bisimulation. This demonstration gives us the right to
think of space as arising from behavior. Finally, we consider where we
might go from the new vantage point we have obtained.

% section introduction (end) 
 
% section introduction (end)

% \documentclass[12pt]{llncs}
%\documentclass{jktr}

\usepackage[pdftex]{hyperref}                   
\usepackage {listings}
\usepackage {mathpartir}
\usepackage{bcprules}
%\usepackage{listings}
                       
\usepackage{graphicx} 
%\usepackage[margins=2.5cm,nohead,nofoot]{geometry}
%\usepackage{geometry}
\usepackage{amsfonts}
\usepackage{amstext}
\usepackage{latexsym}
\usepackage{amssymb}
\usepackage{color}


%\include{myPreamble}
\include{qm2pi.local} 

%\ifpdf
%\usepackage[pdftex]{graphicx}
%\else
%\usepackage{graphicx}
%\fi

 % \ifpdf
%  \usepackage{pdfsync}
%  \if


%\title{Brief Article}
%\author{David F. Snyder}
%\author{L.G. Meredith}

%\address{Dept. of Math., Texas State University--San Marcos, San Marcos, TX 78666}
       
\pagestyle{empty}


\begin{document}

\lstset{language=[Objective]Caml,frame=shadowbox}

\input{qm2pi.front}

% section front matter (end)

\input{qm2pi.intro} 
 
% section introduction (end)

% \input{qm2pi.knotations} 

% section notation (end)

\input{qm2pi.process.calculi} 

% section concurrent_process_calculi_and_spatial_logics_ (end)
    
%\input{qm2pi.knots2pi} 

%\input{qm2pi.trefoil} 

%\input{qm2pi.mainthm} 

% subsection basic_interpretation (end)

%\input{qm2pi.rho.presentation} 
\subsection{The syntax and semantics of the notation system}\label{sub:the_syntax_and_semantics_of_the_notation_system} % (fold)

We now summarize a technical presentation of the calculus that
embodies our theory of dynamics. The typical presentation of such a
calculus follows the style of giving generators and relations on
them. The grammar, below, describing term constructors, freely
generates the set of processes, $\Proc$. This set is then quotiented
by a relation known as structural congruence and it is over this set
that the notion of dynamics is expressed. This presentation is
essentially that of \cite{MeredithR05} with the addition of
polyadicity and summation. For readability we have relegated some of
the technical subtleties to an appendix.

\subsubsection{Process grammar}\label{subsub:process_grammar}

\begin{mathpar}
  \inferrule* [lab=synchronization] {} {{M} \bc \pzero \;|\; x?F \;|\; x!C }
  \and
  \inferrule* [lab=abstraction] {} {{F} \bc (x)P}
  \and
  \inferrule* [lab=concretion] {} {{C} \bc \langle Q \rangle}
  \and
  \inferrule* [lab=process] {} {{P,Q} \bc M \;| \;P|Q \;|\; @{x}}
  \and
  \inferrule* [lab=name] {} {{x} \bc \quotep{P}}
\end{mathpar} 

Note that $\vec{x}$ (resp. $\vec{P}$) denotes a vector of names
(resp. processes) of length $|\vec{x}|$ (resp. $|\vec{P}|$). We adopt
the following useful abbreviations.

\begin{mathpar}
   x?(\vec{y}).P := x.(\vec{y})P \and  x\clift{\vec{P}} := x.\clift{\vec{P}}
   \and x!(y) := \lift{x}{\dropn{y}}
   \and \Pi_{i=0}^{n-1}P_i := P_0 | \ldots | P_{n-1}
\end{mathpar}

\subsubsection{Structural congruence}

\paragraph{Free and bound names and alpha-equivalence.} At the
core of structural equivalence is alpha-equivalence which identifies
process that are the same up to a change of variable. Formally, we
recognize the distinction between free and bound names. The free names
of a process, $\freenames{P}$, may be calculated recursively as
follows:

\begin{mathpar}
\freenames{\pzero} := \emptyset
  \and \\
  \freenames{x?(y).P} := \{ x \} \cup (\freenames{P} \setminus \{ y \})
  \and 
  \freenames{x!\langle P \rangle} := \{ x \} \cup \{ P \} 
  \and \\
  \freenames{P|Q} := \freenames{P} \cup \freenames{Q}
  \and \\
  \freenames{@{x}} := \{ x \}
\end{mathpar}

$\pi$
$\quotep{\pi}$

$\freenames{-} : \pi \to \mathcal{P}(\quotep{\pi})$

\begin{eqnarray*}
  \freenames{\pzero} & := & \emptyset \\
  \freenames{x?(y).P} & := & \{ x \} \cup (\freenames{P} \setminus \{ y \}) \\
  \freenames{x!\langle P \rangle} & := & \{ x \} \cup \{ P \} \\
  \freenames{P|Q} & := & \freenames{P} \cup \freenames{Q} \\
  \freenames{\dropn{x}} & := & \{ x \}
\end{eqnarray*}

The bound names of a process, $\boundnames{P}$, are those names occurring in $P$
that are not free. For example, in $x?(y).0$, the name $x$ is free, while $y$ is bound.

\begin{mathpar}
  \inferrule* [lab=monoidal-laws] {} { P|Q \equiv Q|P \and P|0 \equiv P \and P|(Q|R) \equiv (P|Q)|R }
\end{mathpar}

\begin{mathpar}
  \inferrule* [lab=alpha-equivalence] {} { (x)P \equiv (y)P\{y/x\} \and y \not\in \freenames{P} }
\end{mathpar}

\begin{definition}
Then two processes, $P,Q$, are alpha-equivalent if $P = Q\{\vec{y}/\vec{x}\}$ for
some $\vec{x} \in \boundnames{Q},\vec{y} \in \boundnames{P}$, where $Q\{\vec{y}/\vec{x}\}$
denotes the capture-avoiding substitution of $\vec{y}$ for $\vec{x}$ in $Q$.
\end{definition}

\begin{definition}
  The {\em structural congruence} \cite{SangiorgiWalker} , $\equiv$,
  between processes is the least congruence containing
  alpha-equivalence, satisfying the abelian monoid laws
  (associativity, commutativity and $\pzero$ as identity) for parallel
  composition $|$ and for summation $+$.
\end{definition}

\subsection{Name equivalence}

We take name equivalence, written $\nameeq$, to be the smallest
equivalence relation generated by the following rules.

\begin{mathpar}
\inferrule*[lab=Quote-drop]
{ }
{ \quotep{@{x}} \nameeq x }

\inferrule*[lab=Struct-equiv]
{ P \scong Q }
{ \quotep{P} \nameeq \quotep{Q} }
\end{mathpar}

The astute reader will have noticed that the mutual recursion of names
and processes imposes a mutual recursion on alpha-equivalence and
structural equivalence via name-equivalence. Fortunately, all of this
works out pleasantly and we may calculate in the natural way, free of
concern. The reader interested in the details is referred to the
appendix \ref{appendix:rho_details}.

\subsection{Substitution}

We use $\Proc$ for the set of processes, $\QProc$ for the set of
names, and $\id{\{}\vec{y} / \vec{x} \id{\}}$ to denote partial maps,
$s : \QProc \rightarrow \QProc$. A map, $s$ lifts, uniquely, to a map
on process terms, $\widehat{s} : \Proc \rightarrow \Proc$ by the
following equations.

\begin{mathpar}
  (0) \psubstp{Q}{P} := 0 \\
  (R \juxtap S) \psubstp{Q}{P}
  :=    
  (R)\psubstp{Q}{P} \juxtap (S) \psubstp{Q}{P} \\
  (x?(y).R) \psubstp{Q}{P}    
  :=    
  (x)\substp{Q}{P} (z)\concat( (R \psubstn{z}{y}) \psubstp{Q}{P} ) \\
  (\lift{x}{R}) \psubstp{Q}{P}  
  :=
  \lift{(x)\substp{Q}{P}}{ R \psubstp{Q}{P} } \\
%   (\dropn{x})  \psubstp{Q}{P}       
%   := 
%   \left\{ 
%     \begin{array}{ccc} 
%       \dropn{\quotep{Q}} & & x \nameeq \quotep{P} \\
%       \dropn{x} & & otherwise \\
%     \end{array}
%   \right. 
  (\dropn{x})  \psubstp{Q}{P}       
  := 
  \left\{ 
    \begin{array}{ccc} 
      Q & & x \nameeq \quotep{P} \\
      \dropn{x} & & otherwise \\
    \end{array}
  \right.
\end{mathpar}
 

where

\begin{eqnarray}
  (x)\id{\{} \lpquote Q \rpquote / \lpquote P \rpquote \id{\}}            = 
  \left\{ 
    \begin{array}{ccc}
      \lpquote Q \rpquote & & x \nameeq \lpquote P \rpquote \\
      x & & otherwise \\
    \end{array}
  \right. \nonumber
\end{eqnarray}

and $z$ is chosen distinct from $\quotep{P}$, $\quotep{Q}$, the free
names in $Q$, and all the names in $R$. Our $\alpha$-equivalence will
be built in the standard way from this substitution.

\begin{remark}\label{rem:no_self_referential_names}
  One consequence of these definitions is that $\forall P. \quotep{P}
  \not\in \freenames{P}$.
\end{remark}

\subsection{ Dynamic quote: an example }

Anticipating something of what's to come, consider applying the
substitution, $\widehat{\id{\{}u / z \id{\}}}$, to the following pair
of processes, $\lift{w}{y!(z)}$ and $w[ \lpquote y!(z) \rpquote ]$.

\begin{eqnarray}
	\lift{w}{y!(z)}\widehat{\id{\{}u / z \id{\}}}
		& = &
		\lift{w}{y!(u)} \nonumber\\
	w[ \lpquote y!(z) \rpquote ] \widehat{ \id{\{}u / z \id{\}} }
		& = &
		w[ \lpquote y!(z) \rpquote ] \nonumber
\end{eqnarray}

Because the body of the process between quotes is impervious to
substitution, we get radically different answers. In fact, by
examining the first process in an input context,
e.g. $x?(z).\lift{w}{y!(z)}$, we see that the process under the lift
operator may be shaped by prefixed inputs binding a name inside it. In
this sense, the lift operator will be seen as a way to dynamically
construct processes before reifying them as names.

Finally equipped with these standard features we can present the
dynamics of the calculus.

\subsubsection{Operational semantics} 

Finally, we introduce the computational dynamics. What marks these
algebras as distinct from other more traditionally studied algebraic
structures, e.g. vector spaces or polynomial rings, is the manner in
which dynamics is captured. In traditional structures, dynamics is typically
expressed through morphisms between such structures, as in linear maps
between vector spaces or morphisms between rings. In algebras
associated with the semantics of computation, the dynamics is
expressed as part of the algebraic structure itself, through a
reduction reduction relation typically denoted by $\red$. Below, we
give a recursive presentation of this relation for the calculus used
in the encoding.

$\red \subseteq \pi \times \pi$
$\red : \pi \to \mathcal{P}(\pi)$

\begin{mathpar}
  \inferrule* [lab=Comm] { \textsf{match}( x_{src}, x_{trgt} ) } { x_{trgt}?(y)P \; | \; x_{src}!\langle {Q} \rangle \red P\{\quotep{Q}/y}\} }
  \and \\
  \inferrule* [lab=Par] {{P} \red {P}'} {{{P} | {Q}} \red {{P}' | {Q}}}
  \and
  \inferrule* [lab=Equiv]{{{P} \scong {P}'} \andalso {{P}' \red {Q}'} \andalso {{Q}' \scong {Q}}}{{P} \red {Q}}
\end{mathpar}

\begin{eqnarray*}
  match_{\equiv} (\quotep{P},\quotep{Q}) & := & P \equiv Q \\
  match_{\dagger}(\quotep{P},\quotep{Q}) & := & \forall R. P|Q \red^{*} R => R \red^{*} 0 \\
  match_{K}(\quotep{P},\quotep{Q}) & := & K \mbox{ for some context } K
\end{eqnarray*}

$u?(x)P | u!\langle Q \rangle \red P\{\quotep{Q}/x\}$

%We write $\wred$ for $\red^*$, and $P\red$ if $\exists Q $ such that $ P \red Q$.
We write $P\red$ if $\exists Q $ such that $ P \red Q$ and $P\not\red$, otherwise.

\section{Replication}

As mentioned before, it is known that replication (and hence
recursion) can be implemented in a higher-order process algebra
\cite{SangiorgiWalker}. As our first example of calculation with the
machinery thus far presented we give the construction explicitly in
the {\rhoc}.

\begin{eqnarray}
	D_{x} & := & \prefix{x}{y}{(\binpar{\outputp{x}{y}}{@{y}})} \nonumber\\
	\bangp_{x}{P} & := & \binpar{{x}!\langle{\binpar{D_{x}}{P}}\rangle}{D_{x}} \nonumber
\end{eqnarray}

\begin{eqnarray}
	\bangp_{x}{P} & & \nonumber\\
	=
	& {x}!\langle{(\prefix{x}{y}{(\outputp{x}{y} | @{y})) | P}}\rangle 
	      | \prefix{x}{y}{(\outputp{x}{y} | @{y})} & \nonumber\\
	\red
	& (\outputp{x}{y} | @{y})\substn{\quotep{(\prefix{x}{y}{(@{y} | \outputp{x}{y})) | P}}}{y} & \nonumber\\
	=
	& \outputp{x}{\quotep{(\prefix{x}{y}{(\outputp{x}{y} | @{y})) | P}}}
	  | {(\prefix{x}{y}{(\outputp{x}{y} | @{y})) | P}} & \nonumber\\
	\red
	& \ldots & \nonumber\\
	\red^*
	& P | P | \ldots & \nonumber
\end{eqnarray}

Of course, this encoding, as an implementation, runs away, unfolding
$\bangp{P}$ eagerly. A lazier and more implementable replication
operator, restricted to input-guarded processes, may be obtained as follows.

\begin{eqnarray}
\bangp{\prefix{u}{v}{P}} 
	:= 
	\binpar{\lift{x}{\prefix{u}{v}{(\binpar{D(x)}{P})}}}{D(x)} \nonumber
\end{eqnarray}

\begin{remark}
  Note that the lazier definition still does not deal with summation
  or mixed summation (i.e. sums over input and output). The reader is
  invited to construct definitions of replication that deal with these
  features. 

  Further, the definitions are parameterized in a name, $x$. Can you,
  gentle reader, make a definition that eliminates this parameter and
  guarantees no accidental interaction between the replication
  machinery and the process being replicated -- i.e. no accidental
  sharing of names used by the process to get its work done and the
  name(s) used by the replication to effect copying. This latter
  revision of the definition of replication is crucial to obtaining
  the expected identity $!!P \sim !P$.
\end{remark}

\begin{remark}\label{rem:paradoxical_combinator}
  The reader familiar with the lambda calculus will have noticed the
  similarity between $D$ and the paradoxical combinator.

  [Ed. note: the existence of this seems to suggest we have to be more
  restrictive on the set of processes and names we admit if we are to
  support no-cloning.]
\end{remark}

\subsubsection{Bisimulation}

The computational dynamics gives rise to another kind of equivalence,
the equivalence of computational behavior. As previously mentioned
this is typically captured \emph{via} some form of bisimulation.

% The notion we use in this paper is weak barbed bisimulation
% \cite{milner91polyadicpi}.

The notion we use in this paper is derived from weak barbed
bisimulation \cite{milner91polyadicpi}. 

\begin{definition}
An \emph{observation relation}, $\downarrow_{\mathcal N}$, over a set
of names, $\mathcal N$, is the smallest relation satisfying the rules
below.

\infrule[Out-barb]{y \in {\mathcal N}, \; x \nameeq y}
		  {\outputp{x}{v} \downarrow_{\mathcal N} x}
\infrule[Par-barb]{\mbox{$P\downarrow_{\mathcal N} x$ or $Q\downarrow_{\mathcal N} x$}}
		  {\binpar{P}{Q} \downarrow_{\mathcal N} x}

We write $P \Downarrow_{\mathcal N} x$ if there is $Q$ such that 
$P \wred Q$ and $Q \downarrow_{\mathcal N} x$.
\end{definition}

\begin{definition}
%\label{def.bbisim}
An  ${\mathcal N}$-\emph{barbed bisimulation} over a set of names, ${\mathcal N}$, is a symmetric binary relation 
${\mathcal S}_{\mathcal N}$ between agents such that $P\rel{S}_{\mathcal N}Q$ implies:
\begin{enumerate}
\item If $P \red P'$ then $Q \wred Q'$ and $P'\rel{S}_{\mathcal N} Q'$.
\item If $P\downarrow_{\mathcal N} x$, then $Q\Downarrow_{\mathcal N} x$.
\end{enumerate}
$P$ is ${\mathcal N}$-barbed bisimilar to $Q$, written
$P \wbbisim_{\mathcal N} Q$, if $P \rel{S}_{\mathcal N} Q$ for some ${\mathcal N}$-barbed bisimulation ${\mathcal S}_{\mathcal N}$.
\end{definition}

$\mathcal{R} \subseteq \pi \times \pi$

$P \mathcal{R} Q => \forall P'. P \red P' \Rightarrow \exists Q'. Q \red Q', P' \mathcal{R} Q'$

$P \vdash x \Rightarrow Q \vdash x$

\begin{mathpar}
  \inferrule*[lab=Out-barb]{x \nameeq y}{{y}!\langle{Q}\rangle \vdash x}
  \and
  \inferrule*[lab=Par-barb]{\mbox{$P\vdash x$ or $Q\vdash x$}}{\binpar{P}{Q} \vdash x}
\end{mathpar}

\subsubsection{Contexts}

One of the principle advantages of computational calculi like the
$\pi$-calculus is a well-defined notion of context,
contextual-equivalence and a correlation between
contextual-equivalence and notions of bisimulation. The notion of
context allows the decomposition of a process into (sub-)process and
its syntactic environment, its context. Thus, a context may be
thought of as a process with a ``hole'' (written $\Box$) in it. The
application of a context $M$ to a process $P$, written $M[P]$, is
tantamount to filling the hole in $M$ with $P$. In this paper we do
not need the full weight of this theory, but do make use of the notion
of context in the proof the main theorem. 

\begin{mathpar}
  \inferrule* [lab=summation] {} {{M_{M},M_{N}} \bc \Box \;|\; x.M_{A} \;|\; M_{M}+M_{N}}
  \and
  \inferrule* [lab=agent] {} {{M_{A}} \bc (\vec{x})M_{P} \;| \; \clift{P_0,\ldots,M_{P},\ldots,P_N}}
  \and \\
  \inferrule* [lab=process] {} {{M_{P}} \bc M_{N} \;| \;P|M_{P} }
\end{mathpar} 

\begin{mathpar}
  \inferrule* [lab=sychronization] {} {M_{N} \bc \Box \;|\; x?M_{F} \;|\; x!M_{C}}
  \and
  \inferrule* [lab=abstraction] {} {{M_{F}} \bc (x)M_{P} }
  \and
  \inferrule* [lab=concretion] {} {{M_{C}} \bc \langle M_{P} \rangle }
  \and \\
  \inferrule* [lab=process] {} {{M_{P}} \bc M_{N} \;| \;P|M_{P} }
\end{mathpar}

\begin{definition}[contextual application] Given a context $M$, and
  process $P$, we define the \emph{contextual application}, $M[P] :=
  M\{P/\Box\}$. That is, the contextual application of M to P is the
  substitution of $P$ for $\Box$ in $M$.
\end{definition}

$\meaningof{-} : L \to \mathcal{P}(\pi)$

\begin{mathpar}
  \inferrule* [lab=collection] {} {\meaningof{true} = \pi, \and \meaningof{~E} = \pi \setminus \meaningof{E}, \and \meaningof{E_{1} \& E_{2}} = \meaningof{E_{1}} \cap \meaningof{E_{2}}}
\end{mathpar}

\begin{mathpar}
  \inferrule* [lab=structure] {} {\meaningof{0} = \{ P \in \pi | P \equiv 0 \}, \and \\ \meaningof{E_1 | E_2} = \{ P \in \pi | P \equiv P_{1} | P_{2}, P_{1} \in \meaningof{E_{1}}, P_{2} \in \meaningof{E_2}\} }
\end{mathpar}

\begin{mathpar}
 \inferrule* [lab=behavior] {} {\meaningof{\langle a?b \rangle E} = \{ P \in \pi | P \equiv Q | u?(y)P', \\ \and \\\\ \and \\ \;\;\; u \in \meaningof{a}, \forall z.P'\{z/y\} \in \meaningof{E\{z/b\}}\}, \and \\ \meaningof{a!E} = \{ P \in \pi | P \equiv Q | x!\langle P' \rangle, x \in \meaningof{a} P' \in \meaningof{E}\} }
\end{mathpar}

\begin{mathpar}
 \inferrule* [lab=nominal] {} {\meaningof{\quotep{E}} = \{ \quotep{P} \in \quotep{\pi} | P \in \meaningof{E} \}, \and \meaningof{\quotep{P}} = \{ \quotep{Q} \in \quotep{\pi} | P \equiv Q \} \and \\ \meaningof{@\quotep{E}} = \{ P \in \pi | P \equiv @x, x \in \meaningof{E} \}}
\end{mathpar}

\begin{eqnarray*}
  \\
  \meaningof{-} : TS \to ST
\end{eqnarray*}

\begin{eqnarray*}
  \\
  L : TS \to ST
\end{eqnarray*}

\begin{eqnarray*}
  \\
  P \models E \iff P \in \meaningof{E}
\end{eqnarray*}

\begin{eqnarray*}
  P \approx_{L} Q \iff \forall E \in L. P \models E \iff Q \models E
\end{eqnarray*}

\begin{eqnarray*}
  P \approx_{K} Q
\end{eqnarray*}

\begin{eqnarray*}
  P \approx Q
\end{eqnarray*}

$\approx_{K} = \approx = \approx_{L}$

\subsubsection{Contextual duality}

Note that contexts extend the quotation operation to a family of
operations from processes to names. Given a context, $M$, we can
define a \emph{nominal context}, $\quotep{M}$ by $\quotep{M}[P] :=
\quotep{M[P]}$. To foreshadow what is to come we observe that these
operations enjoy a duality with processes very much like the duality
between vectors and maps from vectors to scalars.

Further, because the calculus is essentially higher-order, we have a
correspondence between contexts and processes. More specifically,
given a name $x$ and a context $M$ we can construct $M^{*}_{x}$ such
that 

\begin{mathpar}
  M^{*}_{x} | \lift{x}{P} \red M[P]
\end{mathpar}

namely,

\begin{mathpar}
  M^{*}_{x} := x?(u).M[\dropn{u}]
\end{mathpar}

The dependence of $M^{*}_{x}$ on a name makes it an abstraction, 

\begin{mathpar}
  M^{*} := (x)x?(u).M[\dropn{u}]
\end{mathpar}

\subsection{Additional notation}

It will sometimes be convenient to denote the process a name
quotes. We already have the notation $x = \quotep{P}$, but it will be
convenient to introduce an alternate notation, $\procn{x}$, when we
want to emphasize the connection to the use of the name. Note that, by
virtue of name equivalence, $\quotep{\procn{x}} \nameeq x$; so, the
notation is consistent with previous definitions.

Further, because names have structure it is possible to effect
substitutions on the basis of that structure. This means we need to
upgrade our notation for substitutions, which we accomplish by
adapting comprehension notation. Thus,

\begin{mathpar}
  P\{ y / x : x \in S \}
\end{mathpar}

is interpreted to mean the process derived from P by replacing (in a
capture-avoiding manner) each occurrence of $x$ in $S$ by $y$. For example,

\begin{mathpar}
  P\{ \quotep{\procn{x}|\procn{x}} / x : x \in \freenames{P} \}
\end{mathpar}

will replace each (occurrence) of a free name $x$ in $P$ by
$\quotep{\procn{x}|\procn{x}}$.

Also, we will avail ourselves of the notation $x^{L}$ and $x^{R}$ to
denote injections of a name into disjoint copies of the name
space. There are numerous ways to accomplish this. One example can be
found in \cite{MeredithR05}. This notation overloads to vectors of
names: $\vec{x}^{\pi} := (x_{i}^{\pi} \; : \; 0 \leq i < |\vec{x}| )$ where $\pi \in \{L,R\}$.

We also use $P^{\Box} := P|\Box$.

In \cite{MeredithR05} an interpretation of the new operator is
given. It turns out that there are several possible interpretations
all enjoying the requisite algebraic properties of the operator (see
\cite{milner91polyadicpi}). We will therefore make liberal use of
$(\nu\; \vec{x})P$.

% subsection the_syntax_and_semantics_of_the_notation_system (end)   

\input{qm2pi.qmops} 

\input{qm2pi.sterngerlach} 

\input{qm2pi.metric} 

% section concurrent_process_calculi (end)

%\input{qm2pi.proofsketch}

% section proof sketch (end)

%\input{qm2pi.slviaknots} 

% section spatial logic via knots (end)

\input{qm2pi.conclusion}

% section conclusion (end)

%\input{qm2pi.dtcodes} 

% section wiring algorithm (end)

\input{qm2pi.ack} 

% section acknowledgments (end)

\newpage


\bibliographystyle{plain}   
\bibliography{../../biblios/main.bib}

\input{qm2pi.rhodetails}

\end{document}

 

% section notation (end)

\input{qm2pi.process.calculi} 

% section concurrent_process_calculi_and_spatial_logics_ (end)
    
%\documentclass[12pt]{llncs}
%\documentclass{jktr}

\usepackage[pdftex]{hyperref}                   
\usepackage {listings}
\usepackage {mathpartir}
\usepackage{bcprules}
%\usepackage{listings}
                       
\usepackage{graphicx} 
%\usepackage[margins=2.5cm,nohead,nofoot]{geometry}
%\usepackage{geometry}
\usepackage{amsfonts}
\usepackage{amstext}
\usepackage{latexsym}
\usepackage{amssymb}
\usepackage{color}


%\include{myPreamble}
\include{qm2pi.local} 

%\ifpdf
%\usepackage[pdftex]{graphicx}
%\else
%\usepackage{graphicx}
%\fi

 % \ifpdf
%  \usepackage{pdfsync}
%  \if


%\title{Brief Article}
%\author{David F. Snyder}
%\author{L.G. Meredith}

%\address{Dept. of Math., Texas State University--San Marcos, San Marcos, TX 78666}
       
\pagestyle{empty}


\begin{document}

\lstset{language=[Objective]Caml,frame=shadowbox}

\input{qm2pi.front}

% section front matter (end)

\input{qm2pi.intro} 
 
% section introduction (end)

% \input{qm2pi.knotations} 

% section notation (end)

\input{qm2pi.process.calculi} 

% section concurrent_process_calculi_and_spatial_logics_ (end)
    
%\input{qm2pi.knots2pi} 

%\input{qm2pi.trefoil} 

%\input{qm2pi.mainthm} 

% subsection basic_interpretation (end)

%\input{qm2pi.rho.presentation} 
\subsection{The syntax and semantics of the notation system}\label{sub:the_syntax_and_semantics_of_the_notation_system} % (fold)

We now summarize a technical presentation of the calculus that
embodies our theory of dynamics. The typical presentation of such a
calculus follows the style of giving generators and relations on
them. The grammar, below, describing term constructors, freely
generates the set of processes, $\Proc$. This set is then quotiented
by a relation known as structural congruence and it is over this set
that the notion of dynamics is expressed. This presentation is
essentially that of \cite{MeredithR05} with the addition of
polyadicity and summation. For readability we have relegated some of
the technical subtleties to an appendix.

\subsubsection{Process grammar}\label{subsub:process_grammar}

\begin{mathpar}
  \inferrule* [lab=synchronization] {} {{M} \bc \pzero \;|\; x?F \;|\; x!C }
  \and
  \inferrule* [lab=abstraction] {} {{F} \bc (x)P}
  \and
  \inferrule* [lab=concretion] {} {{C} \bc \langle Q \rangle}
  \and
  \inferrule* [lab=process] {} {{P,Q} \bc M \;| \;P|Q \;|\; @{x}}
  \and
  \inferrule* [lab=name] {} {{x} \bc \quotep{P}}
\end{mathpar} 

Note that $\vec{x}$ (resp. $\vec{P}$) denotes a vector of names
(resp. processes) of length $|\vec{x}|$ (resp. $|\vec{P}|$). We adopt
the following useful abbreviations.

\begin{mathpar}
   x?(\vec{y}).P := x.(\vec{y})P \and  x\clift{\vec{P}} := x.\clift{\vec{P}}
   \and x!(y) := \lift{x}{\dropn{y}}
   \and \Pi_{i=0}^{n-1}P_i := P_0 | \ldots | P_{n-1}
\end{mathpar}

\subsubsection{Structural congruence}

\paragraph{Free and bound names and alpha-equivalence.} At the
core of structural equivalence is alpha-equivalence which identifies
process that are the same up to a change of variable. Formally, we
recognize the distinction between free and bound names. The free names
of a process, $\freenames{P}$, may be calculated recursively as
follows:

\begin{mathpar}
\freenames{\pzero} := \emptyset
  \and \\
  \freenames{x?(y).P} := \{ x \} \cup (\freenames{P} \setminus \{ y \})
  \and 
  \freenames{x!\langle P \rangle} := \{ x \} \cup \{ P \} 
  \and \\
  \freenames{P|Q} := \freenames{P} \cup \freenames{Q}
  \and \\
  \freenames{@{x}} := \{ x \}
\end{mathpar}

$\pi$
$\quotep{\pi}$

$\freenames{-} : \pi \to \mathcal{P}(\quotep{\pi})$

\begin{eqnarray*}
  \freenames{\pzero} & := & \emptyset \\
  \freenames{x?(y).P} & := & \{ x \} \cup (\freenames{P} \setminus \{ y \}) \\
  \freenames{x!\langle P \rangle} & := & \{ x \} \cup \{ P \} \\
  \freenames{P|Q} & := & \freenames{P} \cup \freenames{Q} \\
  \freenames{\dropn{x}} & := & \{ x \}
\end{eqnarray*}

The bound names of a process, $\boundnames{P}$, are those names occurring in $P$
that are not free. For example, in $x?(y).0$, the name $x$ is free, while $y$ is bound.

\begin{mathpar}
  \inferrule* [lab=monoidal-laws] {} { P|Q \equiv Q|P \and P|0 \equiv P \and P|(Q|R) \equiv (P|Q)|R }
\end{mathpar}

\begin{mathpar}
  \inferrule* [lab=alpha-equivalence] {} { (x)P \equiv (y)P\{y/x\} \and y \not\in \freenames{P} }
\end{mathpar}

\begin{definition}
Then two processes, $P,Q$, are alpha-equivalent if $P = Q\{\vec{y}/\vec{x}\}$ for
some $\vec{x} \in \boundnames{Q},\vec{y} \in \boundnames{P}$, where $Q\{\vec{y}/\vec{x}\}$
denotes the capture-avoiding substitution of $\vec{y}$ for $\vec{x}$ in $Q$.
\end{definition}

\begin{definition}
  The {\em structural congruence} \cite{SangiorgiWalker} , $\equiv$,
  between processes is the least congruence containing
  alpha-equivalence, satisfying the abelian monoid laws
  (associativity, commutativity and $\pzero$ as identity) for parallel
  composition $|$ and for summation $+$.
\end{definition}

\subsection{Name equivalence}

We take name equivalence, written $\nameeq$, to be the smallest
equivalence relation generated by the following rules.

\begin{mathpar}
\inferrule*[lab=Quote-drop]
{ }
{ \quotep{@{x}} \nameeq x }

\inferrule*[lab=Struct-equiv]
{ P \scong Q }
{ \quotep{P} \nameeq \quotep{Q} }
\end{mathpar}

The astute reader will have noticed that the mutual recursion of names
and processes imposes a mutual recursion on alpha-equivalence and
structural equivalence via name-equivalence. Fortunately, all of this
works out pleasantly and we may calculate in the natural way, free of
concern. The reader interested in the details is referred to the
appendix \ref{appendix:rho_details}.

\subsection{Substitution}

We use $\Proc$ for the set of processes, $\QProc$ for the set of
names, and $\id{\{}\vec{y} / \vec{x} \id{\}}$ to denote partial maps,
$s : \QProc \rightarrow \QProc$. A map, $s$ lifts, uniquely, to a map
on process terms, $\widehat{s} : \Proc \rightarrow \Proc$ by the
following equations.

\begin{mathpar}
  (0) \psubstp{Q}{P} := 0 \\
  (R \juxtap S) \psubstp{Q}{P}
  :=    
  (R)\psubstp{Q}{P} \juxtap (S) \psubstp{Q}{P} \\
  (x?(y).R) \psubstp{Q}{P}    
  :=    
  (x)\substp{Q}{P} (z)\concat( (R \psubstn{z}{y}) \psubstp{Q}{P} ) \\
  (\lift{x}{R}) \psubstp{Q}{P}  
  :=
  \lift{(x)\substp{Q}{P}}{ R \psubstp{Q}{P} } \\
%   (\dropn{x})  \psubstp{Q}{P}       
%   := 
%   \left\{ 
%     \begin{array}{ccc} 
%       \dropn{\quotep{Q}} & & x \nameeq \quotep{P} \\
%       \dropn{x} & & otherwise \\
%     \end{array}
%   \right. 
  (\dropn{x})  \psubstp{Q}{P}       
  := 
  \left\{ 
    \begin{array}{ccc} 
      Q & & x \nameeq \quotep{P} \\
      \dropn{x} & & otherwise \\
    \end{array}
  \right.
\end{mathpar}
 

where

\begin{eqnarray}
  (x)\id{\{} \lpquote Q \rpquote / \lpquote P \rpquote \id{\}}            = 
  \left\{ 
    \begin{array}{ccc}
      \lpquote Q \rpquote & & x \nameeq \lpquote P \rpquote \\
      x & & otherwise \\
    \end{array}
  \right. \nonumber
\end{eqnarray}

and $z$ is chosen distinct from $\quotep{P}$, $\quotep{Q}$, the free
names in $Q$, and all the names in $R$. Our $\alpha$-equivalence will
be built in the standard way from this substitution.

\begin{remark}\label{rem:no_self_referential_names}
  One consequence of these definitions is that $\forall P. \quotep{P}
  \not\in \freenames{P}$.
\end{remark}

\subsection{ Dynamic quote: an example }

Anticipating something of what's to come, consider applying the
substitution, $\widehat{\id{\{}u / z \id{\}}}$, to the following pair
of processes, $\lift{w}{y!(z)}$ and $w[ \lpquote y!(z) \rpquote ]$.

\begin{eqnarray}
	\lift{w}{y!(z)}\widehat{\id{\{}u / z \id{\}}}
		& = &
		\lift{w}{y!(u)} \nonumber\\
	w[ \lpquote y!(z) \rpquote ] \widehat{ \id{\{}u / z \id{\}} }
		& = &
		w[ \lpquote y!(z) \rpquote ] \nonumber
\end{eqnarray}

Because the body of the process between quotes is impervious to
substitution, we get radically different answers. In fact, by
examining the first process in an input context,
e.g. $x?(z).\lift{w}{y!(z)}$, we see that the process under the lift
operator may be shaped by prefixed inputs binding a name inside it. In
this sense, the lift operator will be seen as a way to dynamically
construct processes before reifying them as names.

Finally equipped with these standard features we can present the
dynamics of the calculus.

\subsubsection{Operational semantics} 

Finally, we introduce the computational dynamics. What marks these
algebras as distinct from other more traditionally studied algebraic
structures, e.g. vector spaces or polynomial rings, is the manner in
which dynamics is captured. In traditional structures, dynamics is typically
expressed through morphisms between such structures, as in linear maps
between vector spaces or morphisms between rings. In algebras
associated with the semantics of computation, the dynamics is
expressed as part of the algebraic structure itself, through a
reduction reduction relation typically denoted by $\red$. Below, we
give a recursive presentation of this relation for the calculus used
in the encoding.

$\red \subseteq \pi \times \pi$
$\red : \pi \to \mathcal{P}(\pi)$

\begin{mathpar}
  \inferrule* [lab=Comm] { \textsf{match}( x_{src}, x_{trgt} ) } { x_{trgt}?(y)P \; | \; x_{src}!\langle {Q} \rangle \red P\{\quotep{Q}/y}\} }
  \and \\
  \inferrule* [lab=Par] {{P} \red {P}'} {{{P} | {Q}} \red {{P}' | {Q}}}
  \and
  \inferrule* [lab=Equiv]{{{P} \scong {P}'} \andalso {{P}' \red {Q}'} \andalso {{Q}' \scong {Q}}}{{P} \red {Q}}
\end{mathpar}

\begin{eqnarray*}
  match_{\equiv} (\quotep{P},\quotep{Q}) & := & P \equiv Q \\
  match_{\dagger}(\quotep{P},\quotep{Q}) & := & \forall R. P|Q \red^{*} R => R \red^{*} 0 \\
  match_{K}(\quotep{P},\quotep{Q}) & := & K \mbox{ for some context } K
\end{eqnarray*}

$u?(x)P | u!\langle Q \rangle \red P\{\quotep{Q}/x\}$

%We write $\wred$ for $\red^*$, and $P\red$ if $\exists Q $ such that $ P \red Q$.
We write $P\red$ if $\exists Q $ such that $ P \red Q$ and $P\not\red$, otherwise.

\section{Replication}

As mentioned before, it is known that replication (and hence
recursion) can be implemented in a higher-order process algebra
\cite{SangiorgiWalker}. As our first example of calculation with the
machinery thus far presented we give the construction explicitly in
the {\rhoc}.

\begin{eqnarray}
	D_{x} & := & \prefix{x}{y}{(\binpar{\outputp{x}{y}}{@{y}})} \nonumber\\
	\bangp_{x}{P} & := & \binpar{{x}!\langle{\binpar{D_{x}}{P}}\rangle}{D_{x}} \nonumber
\end{eqnarray}

\begin{eqnarray}
	\bangp_{x}{P} & & \nonumber\\
	=
	& {x}!\langle{(\prefix{x}{y}{(\outputp{x}{y} | @{y})) | P}}\rangle 
	      | \prefix{x}{y}{(\outputp{x}{y} | @{y})} & \nonumber\\
	\red
	& (\outputp{x}{y} | @{y})\substn{\quotep{(\prefix{x}{y}{(@{y} | \outputp{x}{y})) | P}}}{y} & \nonumber\\
	=
	& \outputp{x}{\quotep{(\prefix{x}{y}{(\outputp{x}{y} | @{y})) | P}}}
	  | {(\prefix{x}{y}{(\outputp{x}{y} | @{y})) | P}} & \nonumber\\
	\red
	& \ldots & \nonumber\\
	\red^*
	& P | P | \ldots & \nonumber
\end{eqnarray}

Of course, this encoding, as an implementation, runs away, unfolding
$\bangp{P}$ eagerly. A lazier and more implementable replication
operator, restricted to input-guarded processes, may be obtained as follows.

\begin{eqnarray}
\bangp{\prefix{u}{v}{P}} 
	:= 
	\binpar{\lift{x}{\prefix{u}{v}{(\binpar{D(x)}{P})}}}{D(x)} \nonumber
\end{eqnarray}

\begin{remark}
  Note that the lazier definition still does not deal with summation
  or mixed summation (i.e. sums over input and output). The reader is
  invited to construct definitions of replication that deal with these
  features. 

  Further, the definitions are parameterized in a name, $x$. Can you,
  gentle reader, make a definition that eliminates this parameter and
  guarantees no accidental interaction between the replication
  machinery and the process being replicated -- i.e. no accidental
  sharing of names used by the process to get its work done and the
  name(s) used by the replication to effect copying. This latter
  revision of the definition of replication is crucial to obtaining
  the expected identity $!!P \sim !P$.
\end{remark}

\begin{remark}\label{rem:paradoxical_combinator}
  The reader familiar with the lambda calculus will have noticed the
  similarity between $D$ and the paradoxical combinator.

  [Ed. note: the existence of this seems to suggest we have to be more
  restrictive on the set of processes and names we admit if we are to
  support no-cloning.]
\end{remark}

\subsubsection{Bisimulation}

The computational dynamics gives rise to another kind of equivalence,
the equivalence of computational behavior. As previously mentioned
this is typically captured \emph{via} some form of bisimulation.

% The notion we use in this paper is weak barbed bisimulation
% \cite{milner91polyadicpi}.

The notion we use in this paper is derived from weak barbed
bisimulation \cite{milner91polyadicpi}. 

\begin{definition}
An \emph{observation relation}, $\downarrow_{\mathcal N}$, over a set
of names, $\mathcal N$, is the smallest relation satisfying the rules
below.

\infrule[Out-barb]{y \in {\mathcal N}, \; x \nameeq y}
		  {\outputp{x}{v} \downarrow_{\mathcal N} x}
\infrule[Par-barb]{\mbox{$P\downarrow_{\mathcal N} x$ or $Q\downarrow_{\mathcal N} x$}}
		  {\binpar{P}{Q} \downarrow_{\mathcal N} x}

We write $P \Downarrow_{\mathcal N} x$ if there is $Q$ such that 
$P \wred Q$ and $Q \downarrow_{\mathcal N} x$.
\end{definition}

\begin{definition}
%\label{def.bbisim}
An  ${\mathcal N}$-\emph{barbed bisimulation} over a set of names, ${\mathcal N}$, is a symmetric binary relation 
${\mathcal S}_{\mathcal N}$ between agents such that $P\rel{S}_{\mathcal N}Q$ implies:
\begin{enumerate}
\item If $P \red P'$ then $Q \wred Q'$ and $P'\rel{S}_{\mathcal N} Q'$.
\item If $P\downarrow_{\mathcal N} x$, then $Q\Downarrow_{\mathcal N} x$.
\end{enumerate}
$P$ is ${\mathcal N}$-barbed bisimilar to $Q$, written
$P \wbbisim_{\mathcal N} Q$, if $P \rel{S}_{\mathcal N} Q$ for some ${\mathcal N}$-barbed bisimulation ${\mathcal S}_{\mathcal N}$.
\end{definition}

$\mathcal{R} \subseteq \pi \times \pi$

$P \mathcal{R} Q => \forall P'. P \red P' \Rightarrow \exists Q'. Q \red Q', P' \mathcal{R} Q'$

$P \vdash x \Rightarrow Q \vdash x$

\begin{mathpar}
  \inferrule*[lab=Out-barb]{x \nameeq y}{{y}!\langle{Q}\rangle \vdash x}
  \and
  \inferrule*[lab=Par-barb]{\mbox{$P\vdash x$ or $Q\vdash x$}}{\binpar{P}{Q} \vdash x}
\end{mathpar}

\subsubsection{Contexts}

One of the principle advantages of computational calculi like the
$\pi$-calculus is a well-defined notion of context,
contextual-equivalence and a correlation between
contextual-equivalence and notions of bisimulation. The notion of
context allows the decomposition of a process into (sub-)process and
its syntactic environment, its context. Thus, a context may be
thought of as a process with a ``hole'' (written $\Box$) in it. The
application of a context $M$ to a process $P$, written $M[P]$, is
tantamount to filling the hole in $M$ with $P$. In this paper we do
not need the full weight of this theory, but do make use of the notion
of context in the proof the main theorem. 

\begin{mathpar}
  \inferrule* [lab=summation] {} {{M_{M},M_{N}} \bc \Box \;|\; x.M_{A} \;|\; M_{M}+M_{N}}
  \and
  \inferrule* [lab=agent] {} {{M_{A}} \bc (\vec{x})M_{P} \;| \; \clift{P_0,\ldots,M_{P},\ldots,P_N}}
  \and \\
  \inferrule* [lab=process] {} {{M_{P}} \bc M_{N} \;| \;P|M_{P} }
\end{mathpar} 

\begin{mathpar}
  \inferrule* [lab=sychronization] {} {M_{N} \bc \Box \;|\; x?M_{F} \;|\; x!M_{C}}
  \and
  \inferrule* [lab=abstraction] {} {{M_{F}} \bc (x)M_{P} }
  \and
  \inferrule* [lab=concretion] {} {{M_{C}} \bc \langle M_{P} \rangle }
  \and \\
  \inferrule* [lab=process] {} {{M_{P}} \bc M_{N} \;| \;P|M_{P} }
\end{mathpar}

\begin{definition}[contextual application] Given a context $M$, and
  process $P$, we define the \emph{contextual application}, $M[P] :=
  M\{P/\Box\}$. That is, the contextual application of M to P is the
  substitution of $P$ for $\Box$ in $M$.
\end{definition}

$\meaningof{-} : L \to \mathcal{P}(\pi)$

\begin{mathpar}
  \inferrule* [lab=collection] {} {\meaningof{true} = \pi, \and \meaningof{~E} = \pi \setminus \meaningof{E}, \and \meaningof{E_{1} \& E_{2}} = \meaningof{E_{1}} \cap \meaningof{E_{2}}}
\end{mathpar}

\begin{mathpar}
  \inferrule* [lab=structure] {} {\meaningof{0} = \{ P \in \pi | P \equiv 0 \}, \and \\ \meaningof{E_1 | E_2} = \{ P \in \pi | P \equiv P_{1} | P_{2}, P_{1} \in \meaningof{E_{1}}, P_{2} \in \meaningof{E_2}\} }
\end{mathpar}

\begin{mathpar}
 \inferrule* [lab=behavior] {} {\meaningof{\langle a?b \rangle E} = \{ P \in \pi | P \equiv Q | u?(y)P', \\ \and \\\\ \and \\ \;\;\; u \in \meaningof{a}, \forall z.P'\{z/y\} \in \meaningof{E\{z/b\}}\}, \and \\ \meaningof{a!E} = \{ P \in \pi | P \equiv Q | x!\langle P' \rangle, x \in \meaningof{a} P' \in \meaningof{E}\} }
\end{mathpar}

\begin{mathpar}
 \inferrule* [lab=nominal] {} {\meaningof{\quotep{E}} = \{ \quotep{P} \in \quotep{\pi} | P \in \meaningof{E} \}, \and \meaningof{\quotep{P}} = \{ \quotep{Q} \in \quotep{\pi} | P \equiv Q \} \and \\ \meaningof{@\quotep{E}} = \{ P \in \pi | P \equiv @x, x \in \meaningof{E} \}}
\end{mathpar}

\begin{eqnarray*}
  \\
  \meaningof{-} : TS \to ST
\end{eqnarray*}

\begin{eqnarray*}
  \\
  L : TS \to ST
\end{eqnarray*}

\begin{eqnarray*}
  \\
  P \models E \iff P \in \meaningof{E}
\end{eqnarray*}

\begin{eqnarray*}
  P \approx_{L} Q \iff \forall E \in L. P \models E \iff Q \models E
\end{eqnarray*}

\begin{eqnarray*}
  P \approx_{K} Q
\end{eqnarray*}

\begin{eqnarray*}
  P \approx Q
\end{eqnarray*}

$\approx_{K} = \approx = \approx_{L}$

\subsubsection{Contextual duality}

Note that contexts extend the quotation operation to a family of
operations from processes to names. Given a context, $M$, we can
define a \emph{nominal context}, $\quotep{M}$ by $\quotep{M}[P] :=
\quotep{M[P]}$. To foreshadow what is to come we observe that these
operations enjoy a duality with processes very much like the duality
between vectors and maps from vectors to scalars.

Further, because the calculus is essentially higher-order, we have a
correspondence between contexts and processes. More specifically,
given a name $x$ and a context $M$ we can construct $M^{*}_{x}$ such
that 

\begin{mathpar}
  M^{*}_{x} | \lift{x}{P} \red M[P]
\end{mathpar}

namely,

\begin{mathpar}
  M^{*}_{x} := x?(u).M[\dropn{u}]
\end{mathpar}

The dependence of $M^{*}_{x}$ on a name makes it an abstraction, 

\begin{mathpar}
  M^{*} := (x)x?(u).M[\dropn{u}]
\end{mathpar}

\subsection{Additional notation}

It will sometimes be convenient to denote the process a name
quotes. We already have the notation $x = \quotep{P}$, but it will be
convenient to introduce an alternate notation, $\procn{x}$, when we
want to emphasize the connection to the use of the name. Note that, by
virtue of name equivalence, $\quotep{\procn{x}} \nameeq x$; so, the
notation is consistent with previous definitions.

Further, because names have structure it is possible to effect
substitutions on the basis of that structure. This means we need to
upgrade our notation for substitutions, which we accomplish by
adapting comprehension notation. Thus,

\begin{mathpar}
  P\{ y / x : x \in S \}
\end{mathpar}

is interpreted to mean the process derived from P by replacing (in a
capture-avoiding manner) each occurrence of $x$ in $S$ by $y$. For example,

\begin{mathpar}
  P\{ \quotep{\procn{x}|\procn{x}} / x : x \in \freenames{P} \}
\end{mathpar}

will replace each (occurrence) of a free name $x$ in $P$ by
$\quotep{\procn{x}|\procn{x}}$.

Also, we will avail ourselves of the notation $x^{L}$ and $x^{R}$ to
denote injections of a name into disjoint copies of the name
space. There are numerous ways to accomplish this. One example can be
found in \cite{MeredithR05}. This notation overloads to vectors of
names: $\vec{x}^{\pi} := (x_{i}^{\pi} \; : \; 0 \leq i < |\vec{x}| )$ where $\pi \in \{L,R\}$.

We also use $P^{\Box} := P|\Box$.

In \cite{MeredithR05} an interpretation of the new operator is
given. It turns out that there are several possible interpretations
all enjoying the requisite algebraic properties of the operator (see
\cite{milner91polyadicpi}). We will therefore make liberal use of
$(\nu\; \vec{x})P$.

% subsection the_syntax_and_semantics_of_the_notation_system (end)   

\input{qm2pi.qmops} 

\input{qm2pi.sterngerlach} 

\input{qm2pi.metric} 

% section concurrent_process_calculi (end)

%\input{qm2pi.proofsketch}

% section proof sketch (end)

%\input{qm2pi.slviaknots} 

% section spatial logic via knots (end)

\input{qm2pi.conclusion}

% section conclusion (end)

%\input{qm2pi.dtcodes} 

% section wiring algorithm (end)

\input{qm2pi.ack} 

% section acknowledgments (end)

\newpage


\bibliographystyle{plain}   
\bibliography{../../biblios/main.bib}

\input{qm2pi.rhodetails}

\end{document}

 

%\documentclass[12pt]{llncs}
%\documentclass{jktr}

\usepackage[pdftex]{hyperref}                   
\usepackage {listings}
\usepackage {mathpartir}
\usepackage{bcprules}
%\usepackage{listings}
                       
\usepackage{graphicx} 
%\usepackage[margins=2.5cm,nohead,nofoot]{geometry}
%\usepackage{geometry}
\usepackage{amsfonts}
\usepackage{amstext}
\usepackage{latexsym}
\usepackage{amssymb}
\usepackage{color}


%\include{myPreamble}
\include{qm2pi.local} 

%\ifpdf
%\usepackage[pdftex]{graphicx}
%\else
%\usepackage{graphicx}
%\fi

 % \ifpdf
%  \usepackage{pdfsync}
%  \if


%\title{Brief Article}
%\author{David F. Snyder}
%\author{L.G. Meredith}

%\address{Dept. of Math., Texas State University--San Marcos, San Marcos, TX 78666}
       
\pagestyle{empty}


\begin{document}

\lstset{language=[Objective]Caml,frame=shadowbox}

\input{qm2pi.front}

% section front matter (end)

\input{qm2pi.intro} 
 
% section introduction (end)

% \input{qm2pi.knotations} 

% section notation (end)

\input{qm2pi.process.calculi} 

% section concurrent_process_calculi_and_spatial_logics_ (end)
    
%\input{qm2pi.knots2pi} 

%\input{qm2pi.trefoil} 

%\input{qm2pi.mainthm} 

% subsection basic_interpretation (end)

%\input{qm2pi.rho.presentation} 
\subsection{The syntax and semantics of the notation system}\label{sub:the_syntax_and_semantics_of_the_notation_system} % (fold)

We now summarize a technical presentation of the calculus that
embodies our theory of dynamics. The typical presentation of such a
calculus follows the style of giving generators and relations on
them. The grammar, below, describing term constructors, freely
generates the set of processes, $\Proc$. This set is then quotiented
by a relation known as structural congruence and it is over this set
that the notion of dynamics is expressed. This presentation is
essentially that of \cite{MeredithR05} with the addition of
polyadicity and summation. For readability we have relegated some of
the technical subtleties to an appendix.

\subsubsection{Process grammar}\label{subsub:process_grammar}

\begin{mathpar}
  \inferrule* [lab=synchronization] {} {{M} \bc \pzero \;|\; x?F \;|\; x!C }
  \and
  \inferrule* [lab=abstraction] {} {{F} \bc (x)P}
  \and
  \inferrule* [lab=concretion] {} {{C} \bc \langle Q \rangle}
  \and
  \inferrule* [lab=process] {} {{P,Q} \bc M \;| \;P|Q \;|\; @{x}}
  \and
  \inferrule* [lab=name] {} {{x} \bc \quotep{P}}
\end{mathpar} 

Note that $\vec{x}$ (resp. $\vec{P}$) denotes a vector of names
(resp. processes) of length $|\vec{x}|$ (resp. $|\vec{P}|$). We adopt
the following useful abbreviations.

\begin{mathpar}
   x?(\vec{y}).P := x.(\vec{y})P \and  x\clift{\vec{P}} := x.\clift{\vec{P}}
   \and x!(y) := \lift{x}{\dropn{y}}
   \and \Pi_{i=0}^{n-1}P_i := P_0 | \ldots | P_{n-1}
\end{mathpar}

\subsubsection{Structural congruence}

\paragraph{Free and bound names and alpha-equivalence.} At the
core of structural equivalence is alpha-equivalence which identifies
process that are the same up to a change of variable. Formally, we
recognize the distinction between free and bound names. The free names
of a process, $\freenames{P}$, may be calculated recursively as
follows:

\begin{mathpar}
\freenames{\pzero} := \emptyset
  \and \\
  \freenames{x?(y).P} := \{ x \} \cup (\freenames{P} \setminus \{ y \})
  \and 
  \freenames{x!\langle P \rangle} := \{ x \} \cup \{ P \} 
  \and \\
  \freenames{P|Q} := \freenames{P} \cup \freenames{Q}
  \and \\
  \freenames{@{x}} := \{ x \}
\end{mathpar}

$\pi$
$\quotep{\pi}$

$\freenames{-} : \pi \to \mathcal{P}(\quotep{\pi})$

\begin{eqnarray*}
  \freenames{\pzero} & := & \emptyset \\
  \freenames{x?(y).P} & := & \{ x \} \cup (\freenames{P} \setminus \{ y \}) \\
  \freenames{x!\langle P \rangle} & := & \{ x \} \cup \{ P \} \\
  \freenames{P|Q} & := & \freenames{P} \cup \freenames{Q} \\
  \freenames{\dropn{x}} & := & \{ x \}
\end{eqnarray*}

The bound names of a process, $\boundnames{P}$, are those names occurring in $P$
that are not free. For example, in $x?(y).0$, the name $x$ is free, while $y$ is bound.

\begin{mathpar}
  \inferrule* [lab=monoidal-laws] {} { P|Q \equiv Q|P \and P|0 \equiv P \and P|(Q|R) \equiv (P|Q)|R }
\end{mathpar}

\begin{mathpar}
  \inferrule* [lab=alpha-equivalence] {} { (x)P \equiv (y)P\{y/x\} \and y \not\in \freenames{P} }
\end{mathpar}

\begin{definition}
Then two processes, $P,Q$, are alpha-equivalent if $P = Q\{\vec{y}/\vec{x}\}$ for
some $\vec{x} \in \boundnames{Q},\vec{y} \in \boundnames{P}$, where $Q\{\vec{y}/\vec{x}\}$
denotes the capture-avoiding substitution of $\vec{y}$ for $\vec{x}$ in $Q$.
\end{definition}

\begin{definition}
  The {\em structural congruence} \cite{SangiorgiWalker} , $\equiv$,
  between processes is the least congruence containing
  alpha-equivalence, satisfying the abelian monoid laws
  (associativity, commutativity and $\pzero$ as identity) for parallel
  composition $|$ and for summation $+$.
\end{definition}

\subsection{Name equivalence}

We take name equivalence, written $\nameeq$, to be the smallest
equivalence relation generated by the following rules.

\begin{mathpar}
\inferrule*[lab=Quote-drop]
{ }
{ \quotep{@{x}} \nameeq x }

\inferrule*[lab=Struct-equiv]
{ P \scong Q }
{ \quotep{P} \nameeq \quotep{Q} }
\end{mathpar}

The astute reader will have noticed that the mutual recursion of names
and processes imposes a mutual recursion on alpha-equivalence and
structural equivalence via name-equivalence. Fortunately, all of this
works out pleasantly and we may calculate in the natural way, free of
concern. The reader interested in the details is referred to the
appendix \ref{appendix:rho_details}.

\subsection{Substitution}

We use $\Proc$ for the set of processes, $\QProc$ for the set of
names, and $\id{\{}\vec{y} / \vec{x} \id{\}}$ to denote partial maps,
$s : \QProc \rightarrow \QProc$. A map, $s$ lifts, uniquely, to a map
on process terms, $\widehat{s} : \Proc \rightarrow \Proc$ by the
following equations.

\begin{mathpar}
  (0) \psubstp{Q}{P} := 0 \\
  (R \juxtap S) \psubstp{Q}{P}
  :=    
  (R)\psubstp{Q}{P} \juxtap (S) \psubstp{Q}{P} \\
  (x?(y).R) \psubstp{Q}{P}    
  :=    
  (x)\substp{Q}{P} (z)\concat( (R \psubstn{z}{y}) \psubstp{Q}{P} ) \\
  (\lift{x}{R}) \psubstp{Q}{P}  
  :=
  \lift{(x)\substp{Q}{P}}{ R \psubstp{Q}{P} } \\
%   (\dropn{x})  \psubstp{Q}{P}       
%   := 
%   \left\{ 
%     \begin{array}{ccc} 
%       \dropn{\quotep{Q}} & & x \nameeq \quotep{P} \\
%       \dropn{x} & & otherwise \\
%     \end{array}
%   \right. 
  (\dropn{x})  \psubstp{Q}{P}       
  := 
  \left\{ 
    \begin{array}{ccc} 
      Q & & x \nameeq \quotep{P} \\
      \dropn{x} & & otherwise \\
    \end{array}
  \right.
\end{mathpar}
 

where

\begin{eqnarray}
  (x)\id{\{} \lpquote Q \rpquote / \lpquote P \rpquote \id{\}}            = 
  \left\{ 
    \begin{array}{ccc}
      \lpquote Q \rpquote & & x \nameeq \lpquote P \rpquote \\
      x & & otherwise \\
    \end{array}
  \right. \nonumber
\end{eqnarray}

and $z$ is chosen distinct from $\quotep{P}$, $\quotep{Q}$, the free
names in $Q$, and all the names in $R$. Our $\alpha$-equivalence will
be built in the standard way from this substitution.

\begin{remark}\label{rem:no_self_referential_names}
  One consequence of these definitions is that $\forall P. \quotep{P}
  \not\in \freenames{P}$.
\end{remark}

\subsection{ Dynamic quote: an example }

Anticipating something of what's to come, consider applying the
substitution, $\widehat{\id{\{}u / z \id{\}}}$, to the following pair
of processes, $\lift{w}{y!(z)}$ and $w[ \lpquote y!(z) \rpquote ]$.

\begin{eqnarray}
	\lift{w}{y!(z)}\widehat{\id{\{}u / z \id{\}}}
		& = &
		\lift{w}{y!(u)} \nonumber\\
	w[ \lpquote y!(z) \rpquote ] \widehat{ \id{\{}u / z \id{\}} }
		& = &
		w[ \lpquote y!(z) \rpquote ] \nonumber
\end{eqnarray}

Because the body of the process between quotes is impervious to
substitution, we get radically different answers. In fact, by
examining the first process in an input context,
e.g. $x?(z).\lift{w}{y!(z)}$, we see that the process under the lift
operator may be shaped by prefixed inputs binding a name inside it. In
this sense, the lift operator will be seen as a way to dynamically
construct processes before reifying them as names.

Finally equipped with these standard features we can present the
dynamics of the calculus.

\subsubsection{Operational semantics} 

Finally, we introduce the computational dynamics. What marks these
algebras as distinct from other more traditionally studied algebraic
structures, e.g. vector spaces or polynomial rings, is the manner in
which dynamics is captured. In traditional structures, dynamics is typically
expressed through morphisms between such structures, as in linear maps
between vector spaces or morphisms between rings. In algebras
associated with the semantics of computation, the dynamics is
expressed as part of the algebraic structure itself, through a
reduction reduction relation typically denoted by $\red$. Below, we
give a recursive presentation of this relation for the calculus used
in the encoding.

$\red \subseteq \pi \times \pi$
$\red : \pi \to \mathcal{P}(\pi)$

\begin{mathpar}
  \inferrule* [lab=Comm] { \textsf{match}( x_{src}, x_{trgt} ) } { x_{trgt}?(y)P \; | \; x_{src}!\langle {Q} \rangle \red P\{\quotep{Q}/y}\} }
  \and \\
  \inferrule* [lab=Par] {{P} \red {P}'} {{{P} | {Q}} \red {{P}' | {Q}}}
  \and
  \inferrule* [lab=Equiv]{{{P} \scong {P}'} \andalso {{P}' \red {Q}'} \andalso {{Q}' \scong {Q}}}{{P} \red {Q}}
\end{mathpar}

\begin{eqnarray*}
  match_{\equiv} (\quotep{P},\quotep{Q}) & := & P \equiv Q \\
  match_{\dagger}(\quotep{P},\quotep{Q}) & := & \forall R. P|Q \red^{*} R => R \red^{*} 0 \\
  match_{K}(\quotep{P},\quotep{Q}) & := & K \mbox{ for some context } K
\end{eqnarray*}

$u?(x)P | u!\langle Q \rangle \red P\{\quotep{Q}/x\}$

%We write $\wred$ for $\red^*$, and $P\red$ if $\exists Q $ such that $ P \red Q$.
We write $P\red$ if $\exists Q $ such that $ P \red Q$ and $P\not\red$, otherwise.

\section{Replication}

As mentioned before, it is known that replication (and hence
recursion) can be implemented in a higher-order process algebra
\cite{SangiorgiWalker}. As our first example of calculation with the
machinery thus far presented we give the construction explicitly in
the {\rhoc}.

\begin{eqnarray}
	D_{x} & := & \prefix{x}{y}{(\binpar{\outputp{x}{y}}{@{y}})} \nonumber\\
	\bangp_{x}{P} & := & \binpar{{x}!\langle{\binpar{D_{x}}{P}}\rangle}{D_{x}} \nonumber
\end{eqnarray}

\begin{eqnarray}
	\bangp_{x}{P} & & \nonumber\\
	=
	& {x}!\langle{(\prefix{x}{y}{(\outputp{x}{y} | @{y})) | P}}\rangle 
	      | \prefix{x}{y}{(\outputp{x}{y} | @{y})} & \nonumber\\
	\red
	& (\outputp{x}{y} | @{y})\substn{\quotep{(\prefix{x}{y}{(@{y} | \outputp{x}{y})) | P}}}{y} & \nonumber\\
	=
	& \outputp{x}{\quotep{(\prefix{x}{y}{(\outputp{x}{y} | @{y})) | P}}}
	  | {(\prefix{x}{y}{(\outputp{x}{y} | @{y})) | P}} & \nonumber\\
	\red
	& \ldots & \nonumber\\
	\red^*
	& P | P | \ldots & \nonumber
\end{eqnarray}

Of course, this encoding, as an implementation, runs away, unfolding
$\bangp{P}$ eagerly. A lazier and more implementable replication
operator, restricted to input-guarded processes, may be obtained as follows.

\begin{eqnarray}
\bangp{\prefix{u}{v}{P}} 
	:= 
	\binpar{\lift{x}{\prefix{u}{v}{(\binpar{D(x)}{P})}}}{D(x)} \nonumber
\end{eqnarray}

\begin{remark}
  Note that the lazier definition still does not deal with summation
  or mixed summation (i.e. sums over input and output). The reader is
  invited to construct definitions of replication that deal with these
  features. 

  Further, the definitions are parameterized in a name, $x$. Can you,
  gentle reader, make a definition that eliminates this parameter and
  guarantees no accidental interaction between the replication
  machinery and the process being replicated -- i.e. no accidental
  sharing of names used by the process to get its work done and the
  name(s) used by the replication to effect copying. This latter
  revision of the definition of replication is crucial to obtaining
  the expected identity $!!P \sim !P$.
\end{remark}

\begin{remark}\label{rem:paradoxical_combinator}
  The reader familiar with the lambda calculus will have noticed the
  similarity between $D$ and the paradoxical combinator.

  [Ed. note: the existence of this seems to suggest we have to be more
  restrictive on the set of processes and names we admit if we are to
  support no-cloning.]
\end{remark}

\subsubsection{Bisimulation}

The computational dynamics gives rise to another kind of equivalence,
the equivalence of computational behavior. As previously mentioned
this is typically captured \emph{via} some form of bisimulation.

% The notion we use in this paper is weak barbed bisimulation
% \cite{milner91polyadicpi}.

The notion we use in this paper is derived from weak barbed
bisimulation \cite{milner91polyadicpi}. 

\begin{definition}
An \emph{observation relation}, $\downarrow_{\mathcal N}$, over a set
of names, $\mathcal N$, is the smallest relation satisfying the rules
below.

\infrule[Out-barb]{y \in {\mathcal N}, \; x \nameeq y}
		  {\outputp{x}{v} \downarrow_{\mathcal N} x}
\infrule[Par-barb]{\mbox{$P\downarrow_{\mathcal N} x$ or $Q\downarrow_{\mathcal N} x$}}
		  {\binpar{P}{Q} \downarrow_{\mathcal N} x}

We write $P \Downarrow_{\mathcal N} x$ if there is $Q$ such that 
$P \wred Q$ and $Q \downarrow_{\mathcal N} x$.
\end{definition}

\begin{definition}
%\label{def.bbisim}
An  ${\mathcal N}$-\emph{barbed bisimulation} over a set of names, ${\mathcal N}$, is a symmetric binary relation 
${\mathcal S}_{\mathcal N}$ between agents such that $P\rel{S}_{\mathcal N}Q$ implies:
\begin{enumerate}
\item If $P \red P'$ then $Q \wred Q'$ and $P'\rel{S}_{\mathcal N} Q'$.
\item If $P\downarrow_{\mathcal N} x$, then $Q\Downarrow_{\mathcal N} x$.
\end{enumerate}
$P$ is ${\mathcal N}$-barbed bisimilar to $Q$, written
$P \wbbisim_{\mathcal N} Q$, if $P \rel{S}_{\mathcal N} Q$ for some ${\mathcal N}$-barbed bisimulation ${\mathcal S}_{\mathcal N}$.
\end{definition}

$\mathcal{R} \subseteq \pi \times \pi$

$P \mathcal{R} Q => \forall P'. P \red P' \Rightarrow \exists Q'. Q \red Q', P' \mathcal{R} Q'$

$P \vdash x \Rightarrow Q \vdash x$

\begin{mathpar}
  \inferrule*[lab=Out-barb]{x \nameeq y}{{y}!\langle{Q}\rangle \vdash x}
  \and
  \inferrule*[lab=Par-barb]{\mbox{$P\vdash x$ or $Q\vdash x$}}{\binpar{P}{Q} \vdash x}
\end{mathpar}

\subsubsection{Contexts}

One of the principle advantages of computational calculi like the
$\pi$-calculus is a well-defined notion of context,
contextual-equivalence and a correlation between
contextual-equivalence and notions of bisimulation. The notion of
context allows the decomposition of a process into (sub-)process and
its syntactic environment, its context. Thus, a context may be
thought of as a process with a ``hole'' (written $\Box$) in it. The
application of a context $M$ to a process $P$, written $M[P]$, is
tantamount to filling the hole in $M$ with $P$. In this paper we do
not need the full weight of this theory, but do make use of the notion
of context in the proof the main theorem. 

\begin{mathpar}
  \inferrule* [lab=summation] {} {{M_{M},M_{N}} \bc \Box \;|\; x.M_{A} \;|\; M_{M}+M_{N}}
  \and
  \inferrule* [lab=agent] {} {{M_{A}} \bc (\vec{x})M_{P} \;| \; \clift{P_0,\ldots,M_{P},\ldots,P_N}}
  \and \\
  \inferrule* [lab=process] {} {{M_{P}} \bc M_{N} \;| \;P|M_{P} }
\end{mathpar} 

\begin{mathpar}
  \inferrule* [lab=sychronization] {} {M_{N} \bc \Box \;|\; x?M_{F} \;|\; x!M_{C}}
  \and
  \inferrule* [lab=abstraction] {} {{M_{F}} \bc (x)M_{P} }
  \and
  \inferrule* [lab=concretion] {} {{M_{C}} \bc \langle M_{P} \rangle }
  \and \\
  \inferrule* [lab=process] {} {{M_{P}} \bc M_{N} \;| \;P|M_{P} }
\end{mathpar}

\begin{definition}[contextual application] Given a context $M$, and
  process $P$, we define the \emph{contextual application}, $M[P] :=
  M\{P/\Box\}$. That is, the contextual application of M to P is the
  substitution of $P$ for $\Box$ in $M$.
\end{definition}

$\meaningof{-} : L \to \mathcal{P}(\pi)$

\begin{mathpar}
  \inferrule* [lab=collection] {} {\meaningof{true} = \pi, \and \meaningof{~E} = \pi \setminus \meaningof{E}, \and \meaningof{E_{1} \& E_{2}} = \meaningof{E_{1}} \cap \meaningof{E_{2}}}
\end{mathpar}

\begin{mathpar}
  \inferrule* [lab=structure] {} {\meaningof{0} = \{ P \in \pi | P \equiv 0 \}, \and \\ \meaningof{E_1 | E_2} = \{ P \in \pi | P \equiv P_{1} | P_{2}, P_{1} \in \meaningof{E_{1}}, P_{2} \in \meaningof{E_2}\} }
\end{mathpar}

\begin{mathpar}
 \inferrule* [lab=behavior] {} {\meaningof{\langle a?b \rangle E} = \{ P \in \pi | P \equiv Q | u?(y)P', \\ \and \\\\ \and \\ \;\;\; u \in \meaningof{a}, \forall z.P'\{z/y\} \in \meaningof{E\{z/b\}}\}, \and \\ \meaningof{a!E} = \{ P \in \pi | P \equiv Q | x!\langle P' \rangle, x \in \meaningof{a} P' \in \meaningof{E}\} }
\end{mathpar}

\begin{mathpar}
 \inferrule* [lab=nominal] {} {\meaningof{\quotep{E}} = \{ \quotep{P} \in \quotep{\pi} | P \in \meaningof{E} \}, \and \meaningof{\quotep{P}} = \{ \quotep{Q} \in \quotep{\pi} | P \equiv Q \} \and \\ \meaningof{@\quotep{E}} = \{ P \in \pi | P \equiv @x, x \in \meaningof{E} \}}
\end{mathpar}

\begin{eqnarray*}
  \\
  \meaningof{-} : TS \to ST
\end{eqnarray*}

\begin{eqnarray*}
  \\
  L : TS \to ST
\end{eqnarray*}

\begin{eqnarray*}
  \\
  P \models E \iff P \in \meaningof{E}
\end{eqnarray*}

\begin{eqnarray*}
  P \approx_{L} Q \iff \forall E \in L. P \models E \iff Q \models E
\end{eqnarray*}

\begin{eqnarray*}
  P \approx_{K} Q
\end{eqnarray*}

\begin{eqnarray*}
  P \approx Q
\end{eqnarray*}

$\approx_{K} = \approx = \approx_{L}$

\subsubsection{Contextual duality}

Note that contexts extend the quotation operation to a family of
operations from processes to names. Given a context, $M$, we can
define a \emph{nominal context}, $\quotep{M}$ by $\quotep{M}[P] :=
\quotep{M[P]}$. To foreshadow what is to come we observe that these
operations enjoy a duality with processes very much like the duality
between vectors and maps from vectors to scalars.

Further, because the calculus is essentially higher-order, we have a
correspondence between contexts and processes. More specifically,
given a name $x$ and a context $M$ we can construct $M^{*}_{x}$ such
that 

\begin{mathpar}
  M^{*}_{x} | \lift{x}{P} \red M[P]
\end{mathpar}

namely,

\begin{mathpar}
  M^{*}_{x} := x?(u).M[\dropn{u}]
\end{mathpar}

The dependence of $M^{*}_{x}$ on a name makes it an abstraction, 

\begin{mathpar}
  M^{*} := (x)x?(u).M[\dropn{u}]
\end{mathpar}

\subsection{Additional notation}

It will sometimes be convenient to denote the process a name
quotes. We already have the notation $x = \quotep{P}$, but it will be
convenient to introduce an alternate notation, $\procn{x}$, when we
want to emphasize the connection to the use of the name. Note that, by
virtue of name equivalence, $\quotep{\procn{x}} \nameeq x$; so, the
notation is consistent with previous definitions.

Further, because names have structure it is possible to effect
substitutions on the basis of that structure. This means we need to
upgrade our notation for substitutions, which we accomplish by
adapting comprehension notation. Thus,

\begin{mathpar}
  P\{ y / x : x \in S \}
\end{mathpar}

is interpreted to mean the process derived from P by replacing (in a
capture-avoiding manner) each occurrence of $x$ in $S$ by $y$. For example,

\begin{mathpar}
  P\{ \quotep{\procn{x}|\procn{x}} / x : x \in \freenames{P} \}
\end{mathpar}

will replace each (occurrence) of a free name $x$ in $P$ by
$\quotep{\procn{x}|\procn{x}}$.

Also, we will avail ourselves of the notation $x^{L}$ and $x^{R}$ to
denote injections of a name into disjoint copies of the name
space. There are numerous ways to accomplish this. One example can be
found in \cite{MeredithR05}. This notation overloads to vectors of
names: $\vec{x}^{\pi} := (x_{i}^{\pi} \; : \; 0 \leq i < |\vec{x}| )$ where $\pi \in \{L,R\}$.

We also use $P^{\Box} := P|\Box$.

In \cite{MeredithR05} an interpretation of the new operator is
given. It turns out that there are several possible interpretations
all enjoying the requisite algebraic properties of the operator (see
\cite{milner91polyadicpi}). We will therefore make liberal use of
$(\nu\; \vec{x})P$.

% subsection the_syntax_and_semantics_of_the_notation_system (end)   

\input{qm2pi.qmops} 

\input{qm2pi.sterngerlach} 

\input{qm2pi.metric} 

% section concurrent_process_calculi (end)

%\input{qm2pi.proofsketch}

% section proof sketch (end)

%\input{qm2pi.slviaknots} 

% section spatial logic via knots (end)

\input{qm2pi.conclusion}

% section conclusion (end)

%\input{qm2pi.dtcodes} 

% section wiring algorithm (end)

\input{qm2pi.ack} 

% section acknowledgments (end)

\newpage


\bibliographystyle{plain}   
\bibliography{../../biblios/main.bib}

\input{qm2pi.rhodetails}

\end{document}

 

%\documentclass[12pt]{llncs}
%\documentclass{jktr}

\usepackage[pdftex]{hyperref}                   
\usepackage {listings}
\usepackage {mathpartir}
\usepackage{bcprules}
%\usepackage{listings}
                       
\usepackage{graphicx} 
%\usepackage[margins=2.5cm,nohead,nofoot]{geometry}
%\usepackage{geometry}
\usepackage{amsfonts}
\usepackage{amstext}
\usepackage{latexsym}
\usepackage{amssymb}
\usepackage{color}


%\include{myPreamble}
\include{qm2pi.local} 

%\ifpdf
%\usepackage[pdftex]{graphicx}
%\else
%\usepackage{graphicx}
%\fi

 % \ifpdf
%  \usepackage{pdfsync}
%  \if


%\title{Brief Article}
%\author{David F. Snyder}
%\author{L.G. Meredith}

%\address{Dept. of Math., Texas State University--San Marcos, San Marcos, TX 78666}
       
\pagestyle{empty}


\begin{document}

\lstset{language=[Objective]Caml,frame=shadowbox}

\input{qm2pi.front}

% section front matter (end)

\input{qm2pi.intro} 
 
% section introduction (end)

% \input{qm2pi.knotations} 

% section notation (end)

\input{qm2pi.process.calculi} 

% section concurrent_process_calculi_and_spatial_logics_ (end)
    
%\input{qm2pi.knots2pi} 

%\input{qm2pi.trefoil} 

%\input{qm2pi.mainthm} 

% subsection basic_interpretation (end)

%\input{qm2pi.rho.presentation} 
\subsection{The syntax and semantics of the notation system}\label{sub:the_syntax_and_semantics_of_the_notation_system} % (fold)

We now summarize a technical presentation of the calculus that
embodies our theory of dynamics. The typical presentation of such a
calculus follows the style of giving generators and relations on
them. The grammar, below, describing term constructors, freely
generates the set of processes, $\Proc$. This set is then quotiented
by a relation known as structural congruence and it is over this set
that the notion of dynamics is expressed. This presentation is
essentially that of \cite{MeredithR05} with the addition of
polyadicity and summation. For readability we have relegated some of
the technical subtleties to an appendix.

\subsubsection{Process grammar}\label{subsub:process_grammar}

\begin{mathpar}
  \inferrule* [lab=synchronization] {} {{M} \bc \pzero \;|\; x?F \;|\; x!C }
  \and
  \inferrule* [lab=abstraction] {} {{F} \bc (x)P}
  \and
  \inferrule* [lab=concretion] {} {{C} \bc \langle Q \rangle}
  \and
  \inferrule* [lab=process] {} {{P,Q} \bc M \;| \;P|Q \;|\; @{x}}
  \and
  \inferrule* [lab=name] {} {{x} \bc \quotep{P}}
\end{mathpar} 

Note that $\vec{x}$ (resp. $\vec{P}$) denotes a vector of names
(resp. processes) of length $|\vec{x}|$ (resp. $|\vec{P}|$). We adopt
the following useful abbreviations.

\begin{mathpar}
   x?(\vec{y}).P := x.(\vec{y})P \and  x\clift{\vec{P}} := x.\clift{\vec{P}}
   \and x!(y) := \lift{x}{\dropn{y}}
   \and \Pi_{i=0}^{n-1}P_i := P_0 | \ldots | P_{n-1}
\end{mathpar}

\subsubsection{Structural congruence}

\paragraph{Free and bound names and alpha-equivalence.} At the
core of structural equivalence is alpha-equivalence which identifies
process that are the same up to a change of variable. Formally, we
recognize the distinction between free and bound names. The free names
of a process, $\freenames{P}$, may be calculated recursively as
follows:

\begin{mathpar}
\freenames{\pzero} := \emptyset
  \and \\
  \freenames{x?(y).P} := \{ x \} \cup (\freenames{P} \setminus \{ y \})
  \and 
  \freenames{x!\langle P \rangle} := \{ x \} \cup \{ P \} 
  \and \\
  \freenames{P|Q} := \freenames{P} \cup \freenames{Q}
  \and \\
  \freenames{@{x}} := \{ x \}
\end{mathpar}

$\pi$
$\quotep{\pi}$

$\freenames{-} : \pi \to \mathcal{P}(\quotep{\pi})$

\begin{eqnarray*}
  \freenames{\pzero} & := & \emptyset \\
  \freenames{x?(y).P} & := & \{ x \} \cup (\freenames{P} \setminus \{ y \}) \\
  \freenames{x!\langle P \rangle} & := & \{ x \} \cup \{ P \} \\
  \freenames{P|Q} & := & \freenames{P} \cup \freenames{Q} \\
  \freenames{\dropn{x}} & := & \{ x \}
\end{eqnarray*}

The bound names of a process, $\boundnames{P}$, are those names occurring in $P$
that are not free. For example, in $x?(y).0$, the name $x$ is free, while $y$ is bound.

\begin{mathpar}
  \inferrule* [lab=monoidal-laws] {} { P|Q \equiv Q|P \and P|0 \equiv P \and P|(Q|R) \equiv (P|Q)|R }
\end{mathpar}

\begin{mathpar}
  \inferrule* [lab=alpha-equivalence] {} { (x)P \equiv (y)P\{y/x\} \and y \not\in \freenames{P} }
\end{mathpar}

\begin{definition}
Then two processes, $P,Q$, are alpha-equivalent if $P = Q\{\vec{y}/\vec{x}\}$ for
some $\vec{x} \in \boundnames{Q},\vec{y} \in \boundnames{P}$, where $Q\{\vec{y}/\vec{x}\}$
denotes the capture-avoiding substitution of $\vec{y}$ for $\vec{x}$ in $Q$.
\end{definition}

\begin{definition}
  The {\em structural congruence} \cite{SangiorgiWalker} , $\equiv$,
  between processes is the least congruence containing
  alpha-equivalence, satisfying the abelian monoid laws
  (associativity, commutativity and $\pzero$ as identity) for parallel
  composition $|$ and for summation $+$.
\end{definition}

\subsection{Name equivalence}

We take name equivalence, written $\nameeq$, to be the smallest
equivalence relation generated by the following rules.

\begin{mathpar}
\inferrule*[lab=Quote-drop]
{ }
{ \quotep{@{x}} \nameeq x }

\inferrule*[lab=Struct-equiv]
{ P \scong Q }
{ \quotep{P} \nameeq \quotep{Q} }
\end{mathpar}

The astute reader will have noticed that the mutual recursion of names
and processes imposes a mutual recursion on alpha-equivalence and
structural equivalence via name-equivalence. Fortunately, all of this
works out pleasantly and we may calculate in the natural way, free of
concern. The reader interested in the details is referred to the
appendix \ref{appendix:rho_details}.

\subsection{Substitution}

We use $\Proc$ for the set of processes, $\QProc$ for the set of
names, and $\id{\{}\vec{y} / \vec{x} \id{\}}$ to denote partial maps,
$s : \QProc \rightarrow \QProc$. A map, $s$ lifts, uniquely, to a map
on process terms, $\widehat{s} : \Proc \rightarrow \Proc$ by the
following equations.

\begin{mathpar}
  (0) \psubstp{Q}{P} := 0 \\
  (R \juxtap S) \psubstp{Q}{P}
  :=    
  (R)\psubstp{Q}{P} \juxtap (S) \psubstp{Q}{P} \\
  (x?(y).R) \psubstp{Q}{P}    
  :=    
  (x)\substp{Q}{P} (z)\concat( (R \psubstn{z}{y}) \psubstp{Q}{P} ) \\
  (\lift{x}{R}) \psubstp{Q}{P}  
  :=
  \lift{(x)\substp{Q}{P}}{ R \psubstp{Q}{P} } \\
%   (\dropn{x})  \psubstp{Q}{P}       
%   := 
%   \left\{ 
%     \begin{array}{ccc} 
%       \dropn{\quotep{Q}} & & x \nameeq \quotep{P} \\
%       \dropn{x} & & otherwise \\
%     \end{array}
%   \right. 
  (\dropn{x})  \psubstp{Q}{P}       
  := 
  \left\{ 
    \begin{array}{ccc} 
      Q & & x \nameeq \quotep{P} \\
      \dropn{x} & & otherwise \\
    \end{array}
  \right.
\end{mathpar}
 

where

\begin{eqnarray}
  (x)\id{\{} \lpquote Q \rpquote / \lpquote P \rpquote \id{\}}            = 
  \left\{ 
    \begin{array}{ccc}
      \lpquote Q \rpquote & & x \nameeq \lpquote P \rpquote \\
      x & & otherwise \\
    \end{array}
  \right. \nonumber
\end{eqnarray}

and $z$ is chosen distinct from $\quotep{P}$, $\quotep{Q}$, the free
names in $Q$, and all the names in $R$. Our $\alpha$-equivalence will
be built in the standard way from this substitution.

\begin{remark}\label{rem:no_self_referential_names}
  One consequence of these definitions is that $\forall P. \quotep{P}
  \not\in \freenames{P}$.
\end{remark}

\subsection{ Dynamic quote: an example }

Anticipating something of what's to come, consider applying the
substitution, $\widehat{\id{\{}u / z \id{\}}}$, to the following pair
of processes, $\lift{w}{y!(z)}$ and $w[ \lpquote y!(z) \rpquote ]$.

\begin{eqnarray}
	\lift{w}{y!(z)}\widehat{\id{\{}u / z \id{\}}}
		& = &
		\lift{w}{y!(u)} \nonumber\\
	w[ \lpquote y!(z) \rpquote ] \widehat{ \id{\{}u / z \id{\}} }
		& = &
		w[ \lpquote y!(z) \rpquote ] \nonumber
\end{eqnarray}

Because the body of the process between quotes is impervious to
substitution, we get radically different answers. In fact, by
examining the first process in an input context,
e.g. $x?(z).\lift{w}{y!(z)}$, we see that the process under the lift
operator may be shaped by prefixed inputs binding a name inside it. In
this sense, the lift operator will be seen as a way to dynamically
construct processes before reifying them as names.

Finally equipped with these standard features we can present the
dynamics of the calculus.

\subsubsection{Operational semantics} 

Finally, we introduce the computational dynamics. What marks these
algebras as distinct from other more traditionally studied algebraic
structures, e.g. vector spaces or polynomial rings, is the manner in
which dynamics is captured. In traditional structures, dynamics is typically
expressed through morphisms between such structures, as in linear maps
between vector spaces or morphisms between rings. In algebras
associated with the semantics of computation, the dynamics is
expressed as part of the algebraic structure itself, through a
reduction reduction relation typically denoted by $\red$. Below, we
give a recursive presentation of this relation for the calculus used
in the encoding.

$\red \subseteq \pi \times \pi$
$\red : \pi \to \mathcal{P}(\pi)$

\begin{mathpar}
  \inferrule* [lab=Comm] { \textsf{match}( x_{src}, x_{trgt} ) } { x_{trgt}?(y)P \; | \; x_{src}!\langle {Q} \rangle \red P\{\quotep{Q}/y}\} }
  \and \\
  \inferrule* [lab=Par] {{P} \red {P}'} {{{P} | {Q}} \red {{P}' | {Q}}}
  \and
  \inferrule* [lab=Equiv]{{{P} \scong {P}'} \andalso {{P}' \red {Q}'} \andalso {{Q}' \scong {Q}}}{{P} \red {Q}}
\end{mathpar}

\begin{eqnarray*}
  match_{\equiv} (\quotep{P},\quotep{Q}) & := & P \equiv Q \\
  match_{\dagger}(\quotep{P},\quotep{Q}) & := & \forall R. P|Q \red^{*} R => R \red^{*} 0 \\
  match_{K}(\quotep{P},\quotep{Q}) & := & K \mbox{ for some context } K
\end{eqnarray*}

$u?(x)P | u!\langle Q \rangle \red P\{\quotep{Q}/x\}$

%We write $\wred$ for $\red^*$, and $P\red$ if $\exists Q $ such that $ P \red Q$.
We write $P\red$ if $\exists Q $ such that $ P \red Q$ and $P\not\red$, otherwise.

\section{Replication}

As mentioned before, it is known that replication (and hence
recursion) can be implemented in a higher-order process algebra
\cite{SangiorgiWalker}. As our first example of calculation with the
machinery thus far presented we give the construction explicitly in
the {\rhoc}.

\begin{eqnarray}
	D_{x} & := & \prefix{x}{y}{(\binpar{\outputp{x}{y}}{@{y}})} \nonumber\\
	\bangp_{x}{P} & := & \binpar{{x}!\langle{\binpar{D_{x}}{P}}\rangle}{D_{x}} \nonumber
\end{eqnarray}

\begin{eqnarray}
	\bangp_{x}{P} & & \nonumber\\
	=
	& {x}!\langle{(\prefix{x}{y}{(\outputp{x}{y} | @{y})) | P}}\rangle 
	      | \prefix{x}{y}{(\outputp{x}{y} | @{y})} & \nonumber\\
	\red
	& (\outputp{x}{y} | @{y})\substn{\quotep{(\prefix{x}{y}{(@{y} | \outputp{x}{y})) | P}}}{y} & \nonumber\\
	=
	& \outputp{x}{\quotep{(\prefix{x}{y}{(\outputp{x}{y} | @{y})) | P}}}
	  | {(\prefix{x}{y}{(\outputp{x}{y} | @{y})) | P}} & \nonumber\\
	\red
	& \ldots & \nonumber\\
	\red^*
	& P | P | \ldots & \nonumber
\end{eqnarray}

Of course, this encoding, as an implementation, runs away, unfolding
$\bangp{P}$ eagerly. A lazier and more implementable replication
operator, restricted to input-guarded processes, may be obtained as follows.

\begin{eqnarray}
\bangp{\prefix{u}{v}{P}} 
	:= 
	\binpar{\lift{x}{\prefix{u}{v}{(\binpar{D(x)}{P})}}}{D(x)} \nonumber
\end{eqnarray}

\begin{remark}
  Note that the lazier definition still does not deal with summation
  or mixed summation (i.e. sums over input and output). The reader is
  invited to construct definitions of replication that deal with these
  features. 

  Further, the definitions are parameterized in a name, $x$. Can you,
  gentle reader, make a definition that eliminates this parameter and
  guarantees no accidental interaction between the replication
  machinery and the process being replicated -- i.e. no accidental
  sharing of names used by the process to get its work done and the
  name(s) used by the replication to effect copying. This latter
  revision of the definition of replication is crucial to obtaining
  the expected identity $!!P \sim !P$.
\end{remark}

\begin{remark}\label{rem:paradoxical_combinator}
  The reader familiar with the lambda calculus will have noticed the
  similarity between $D$ and the paradoxical combinator.

  [Ed. note: the existence of this seems to suggest we have to be more
  restrictive on the set of processes and names we admit if we are to
  support no-cloning.]
\end{remark}

\subsubsection{Bisimulation}

The computational dynamics gives rise to another kind of equivalence,
the equivalence of computational behavior. As previously mentioned
this is typically captured \emph{via} some form of bisimulation.

% The notion we use in this paper is weak barbed bisimulation
% \cite{milner91polyadicpi}.

The notion we use in this paper is derived from weak barbed
bisimulation \cite{milner91polyadicpi}. 

\begin{definition}
An \emph{observation relation}, $\downarrow_{\mathcal N}$, over a set
of names, $\mathcal N$, is the smallest relation satisfying the rules
below.

\infrule[Out-barb]{y \in {\mathcal N}, \; x \nameeq y}
		  {\outputp{x}{v} \downarrow_{\mathcal N} x}
\infrule[Par-barb]{\mbox{$P\downarrow_{\mathcal N} x$ or $Q\downarrow_{\mathcal N} x$}}
		  {\binpar{P}{Q} \downarrow_{\mathcal N} x}

We write $P \Downarrow_{\mathcal N} x$ if there is $Q$ such that 
$P \wred Q$ and $Q \downarrow_{\mathcal N} x$.
\end{definition}

\begin{definition}
%\label{def.bbisim}
An  ${\mathcal N}$-\emph{barbed bisimulation} over a set of names, ${\mathcal N}$, is a symmetric binary relation 
${\mathcal S}_{\mathcal N}$ between agents such that $P\rel{S}_{\mathcal N}Q$ implies:
\begin{enumerate}
\item If $P \red P'$ then $Q \wred Q'$ and $P'\rel{S}_{\mathcal N} Q'$.
\item If $P\downarrow_{\mathcal N} x$, then $Q\Downarrow_{\mathcal N} x$.
\end{enumerate}
$P$ is ${\mathcal N}$-barbed bisimilar to $Q$, written
$P \wbbisim_{\mathcal N} Q$, if $P \rel{S}_{\mathcal N} Q$ for some ${\mathcal N}$-barbed bisimulation ${\mathcal S}_{\mathcal N}$.
\end{definition}

$\mathcal{R} \subseteq \pi \times \pi$

$P \mathcal{R} Q => \forall P'. P \red P' \Rightarrow \exists Q'. Q \red Q', P' \mathcal{R} Q'$

$P \vdash x \Rightarrow Q \vdash x$

\begin{mathpar}
  \inferrule*[lab=Out-barb]{x \nameeq y}{{y}!\langle{Q}\rangle \vdash x}
  \and
  \inferrule*[lab=Par-barb]{\mbox{$P\vdash x$ or $Q\vdash x$}}{\binpar{P}{Q} \vdash x}
\end{mathpar}

\subsubsection{Contexts}

One of the principle advantages of computational calculi like the
$\pi$-calculus is a well-defined notion of context,
contextual-equivalence and a correlation between
contextual-equivalence and notions of bisimulation. The notion of
context allows the decomposition of a process into (sub-)process and
its syntactic environment, its context. Thus, a context may be
thought of as a process with a ``hole'' (written $\Box$) in it. The
application of a context $M$ to a process $P$, written $M[P]$, is
tantamount to filling the hole in $M$ with $P$. In this paper we do
not need the full weight of this theory, but do make use of the notion
of context in the proof the main theorem. 

\begin{mathpar}
  \inferrule* [lab=summation] {} {{M_{M},M_{N}} \bc \Box \;|\; x.M_{A} \;|\; M_{M}+M_{N}}
  \and
  \inferrule* [lab=agent] {} {{M_{A}} \bc (\vec{x})M_{P} \;| \; \clift{P_0,\ldots,M_{P},\ldots,P_N}}
  \and \\
  \inferrule* [lab=process] {} {{M_{P}} \bc M_{N} \;| \;P|M_{P} }
\end{mathpar} 

\begin{mathpar}
  \inferrule* [lab=sychronization] {} {M_{N} \bc \Box \;|\; x?M_{F} \;|\; x!M_{C}}
  \and
  \inferrule* [lab=abstraction] {} {{M_{F}} \bc (x)M_{P} }
  \and
  \inferrule* [lab=concretion] {} {{M_{C}} \bc \langle M_{P} \rangle }
  \and \\
  \inferrule* [lab=process] {} {{M_{P}} \bc M_{N} \;| \;P|M_{P} }
\end{mathpar}

\begin{definition}[contextual application] Given a context $M$, and
  process $P$, we define the \emph{contextual application}, $M[P] :=
  M\{P/\Box\}$. That is, the contextual application of M to P is the
  substitution of $P$ for $\Box$ in $M$.
\end{definition}

$\meaningof{-} : L \to \mathcal{P}(\pi)$

\begin{mathpar}
  \inferrule* [lab=collection] {} {\meaningof{true} = \pi, \and \meaningof{~E} = \pi \setminus \meaningof{E}, \and \meaningof{E_{1} \& E_{2}} = \meaningof{E_{1}} \cap \meaningof{E_{2}}}
\end{mathpar}

\begin{mathpar}
  \inferrule* [lab=structure] {} {\meaningof{0} = \{ P \in \pi | P \equiv 0 \}, \and \\ \meaningof{E_1 | E_2} = \{ P \in \pi | P \equiv P_{1} | P_{2}, P_{1} \in \meaningof{E_{1}}, P_{2} \in \meaningof{E_2}\} }
\end{mathpar}

\begin{mathpar}
 \inferrule* [lab=behavior] {} {\meaningof{\langle a?b \rangle E} = \{ P \in \pi | P \equiv Q | u?(y)P', \\ \and \\\\ \and \\ \;\;\; u \in \meaningof{a}, \forall z.P'\{z/y\} \in \meaningof{E\{z/b\}}\}, \and \\ \meaningof{a!E} = \{ P \in \pi | P \equiv Q | x!\langle P' \rangle, x \in \meaningof{a} P' \in \meaningof{E}\} }
\end{mathpar}

\begin{mathpar}
 \inferrule* [lab=nominal] {} {\meaningof{\quotep{E}} = \{ \quotep{P} \in \quotep{\pi} | P \in \meaningof{E} \}, \and \meaningof{\quotep{P}} = \{ \quotep{Q} \in \quotep{\pi} | P \equiv Q \} \and \\ \meaningof{@\quotep{E}} = \{ P \in \pi | P \equiv @x, x \in \meaningof{E} \}}
\end{mathpar}

\begin{eqnarray*}
  \\
  \meaningof{-} : TS \to ST
\end{eqnarray*}

\begin{eqnarray*}
  \\
  L : TS \to ST
\end{eqnarray*}

\begin{eqnarray*}
  \\
  P \models E \iff P \in \meaningof{E}
\end{eqnarray*}

\begin{eqnarray*}
  P \approx_{L} Q \iff \forall E \in L. P \models E \iff Q \models E
\end{eqnarray*}

\begin{eqnarray*}
  P \approx_{K} Q
\end{eqnarray*}

\begin{eqnarray*}
  P \approx Q
\end{eqnarray*}

$\approx_{K} = \approx = \approx_{L}$

\subsubsection{Contextual duality}

Note that contexts extend the quotation operation to a family of
operations from processes to names. Given a context, $M$, we can
define a \emph{nominal context}, $\quotep{M}$ by $\quotep{M}[P] :=
\quotep{M[P]}$. To foreshadow what is to come we observe that these
operations enjoy a duality with processes very much like the duality
between vectors and maps from vectors to scalars.

Further, because the calculus is essentially higher-order, we have a
correspondence between contexts and processes. More specifically,
given a name $x$ and a context $M$ we can construct $M^{*}_{x}$ such
that 

\begin{mathpar}
  M^{*}_{x} | \lift{x}{P} \red M[P]
\end{mathpar}

namely,

\begin{mathpar}
  M^{*}_{x} := x?(u).M[\dropn{u}]
\end{mathpar}

The dependence of $M^{*}_{x}$ on a name makes it an abstraction, 

\begin{mathpar}
  M^{*} := (x)x?(u).M[\dropn{u}]
\end{mathpar}

\subsection{Additional notation}

It will sometimes be convenient to denote the process a name
quotes. We already have the notation $x = \quotep{P}$, but it will be
convenient to introduce an alternate notation, $\procn{x}$, when we
want to emphasize the connection to the use of the name. Note that, by
virtue of name equivalence, $\quotep{\procn{x}} \nameeq x$; so, the
notation is consistent with previous definitions.

Further, because names have structure it is possible to effect
substitutions on the basis of that structure. This means we need to
upgrade our notation for substitutions, which we accomplish by
adapting comprehension notation. Thus,

\begin{mathpar}
  P\{ y / x : x \in S \}
\end{mathpar}

is interpreted to mean the process derived from P by replacing (in a
capture-avoiding manner) each occurrence of $x$ in $S$ by $y$. For example,

\begin{mathpar}
  P\{ \quotep{\procn{x}|\procn{x}} / x : x \in \freenames{P} \}
\end{mathpar}

will replace each (occurrence) of a free name $x$ in $P$ by
$\quotep{\procn{x}|\procn{x}}$.

Also, we will avail ourselves of the notation $x^{L}$ and $x^{R}$ to
denote injections of a name into disjoint copies of the name
space. There are numerous ways to accomplish this. One example can be
found in \cite{MeredithR05}. This notation overloads to vectors of
names: $\vec{x}^{\pi} := (x_{i}^{\pi} \; : \; 0 \leq i < |\vec{x}| )$ where $\pi \in \{L,R\}$.

We also use $P^{\Box} := P|\Box$.

In \cite{MeredithR05} an interpretation of the new operator is
given. It turns out that there are several possible interpretations
all enjoying the requisite algebraic properties of the operator (see
\cite{milner91polyadicpi}). We will therefore make liberal use of
$(\nu\; \vec{x})P$.

% subsection the_syntax_and_semantics_of_the_notation_system (end)   

\input{qm2pi.qmops} 

\input{qm2pi.sterngerlach} 

\input{qm2pi.metric} 

% section concurrent_process_calculi (end)

%\input{qm2pi.proofsketch}

% section proof sketch (end)

%\input{qm2pi.slviaknots} 

% section spatial logic via knots (end)

\input{qm2pi.conclusion}

% section conclusion (end)

%\input{qm2pi.dtcodes} 

% section wiring algorithm (end)

\input{qm2pi.ack} 

% section acknowledgments (end)

\newpage


\bibliographystyle{plain}   
\bibliography{../../biblios/main.bib}

\input{qm2pi.rhodetails}

\end{document}

 

% subsection basic_interpretation (end)

%\input{qm2pi.rho.presentation} 
\subsection{The syntax and semantics of the notation system}\label{sub:the_syntax_and_semantics_of_the_notation_system} % (fold)

We now summarize a technical presentation of the calculus that
embodies our theory of dynamics. The typical presentation of such a
calculus follows the style of giving generators and relations on
them. The grammar, below, describing term constructors, freely
generates the set of processes, $\Proc$. This set is then quotiented
by a relation known as structural congruence and it is over this set
that the notion of dynamics is expressed. This presentation is
essentially that of \cite{MeredithR05} with the addition of
polyadicity and summation. For readability we have relegated some of
the technical subtleties to an appendix.

\subsubsection{Process grammar}\label{subsub:process_grammar}

\begin{mathpar}
  \inferrule* [lab=synchronization] {} {{M} \bc \pzero \;|\; x?F \;|\; x!C }
  \and
  \inferrule* [lab=abstraction] {} {{F} \bc (x)P}
  \and
  \inferrule* [lab=concretion] {} {{C} \bc \langle Q \rangle}
  \and
  \inferrule* [lab=process] {} {{P,Q} \bc M \;| \;P|Q \;|\; @{x}}
  \and
  \inferrule* [lab=name] {} {{x} \bc \quotep{P}}
\end{mathpar} 

Note that $\vec{x}$ (resp. $\vec{P}$) denotes a vector of names
(resp. processes) of length $|\vec{x}|$ (resp. $|\vec{P}|$). We adopt
the following useful abbreviations.

\begin{mathpar}
   x?(\vec{y}).P := x.(\vec{y})P \and  x\clift{\vec{P}} := x.\clift{\vec{P}}
   \and x!(y) := \lift{x}{\dropn{y}}
   \and \Pi_{i=0}^{n-1}P_i := P_0 | \ldots | P_{n-1}
\end{mathpar}

\subsubsection{Structural congruence}

\paragraph{Free and bound names and alpha-equivalence.} At the
core of structural equivalence is alpha-equivalence which identifies
process that are the same up to a change of variable. Formally, we
recognize the distinction between free and bound names. The free names
of a process, $\freenames{P}$, may be calculated recursively as
follows:

\begin{mathpar}
\freenames{\pzero} := \emptyset
  \and \\
  \freenames{x?(y).P} := \{ x \} \cup (\freenames{P} \setminus \{ y \})
  \and 
  \freenames{x!\langle P \rangle} := \{ x \} \cup \{ P \} 
  \and \\
  \freenames{P|Q} := \freenames{P} \cup \freenames{Q}
  \and \\
  \freenames{@{x}} := \{ x \}
\end{mathpar}

$\pi$
$\quotep{\pi}$

$\freenames{-} : \pi \to \mathcal{P}(\quotep{\pi})$

\begin{eqnarray*}
  \freenames{\pzero} & := & \emptyset \\
  \freenames{x?(y).P} & := & \{ x \} \cup (\freenames{P} \setminus \{ y \}) \\
  \freenames{x!\langle P \rangle} & := & \{ x \} \cup \{ P \} \\
  \freenames{P|Q} & := & \freenames{P} \cup \freenames{Q} \\
  \freenames{\dropn{x}} & := & \{ x \}
\end{eqnarray*}

The bound names of a process, $\boundnames{P}$, are those names occurring in $P$
that are not free. For example, in $x?(y).0$, the name $x$ is free, while $y$ is bound.

\begin{mathpar}
  \inferrule* [lab=monoidal-laws] {} { P|Q \equiv Q|P \and P|0 \equiv P \and P|(Q|R) \equiv (P|Q)|R }
\end{mathpar}

\begin{mathpar}
  \inferrule* [lab=alpha-equivalence] {} { (x)P \equiv (y)P\{y/x\} \and y \not\in \freenames{P} }
\end{mathpar}

\begin{definition}
Then two processes, $P,Q$, are alpha-equivalent if $P = Q\{\vec{y}/\vec{x}\}$ for
some $\vec{x} \in \boundnames{Q},\vec{y} \in \boundnames{P}$, where $Q\{\vec{y}/\vec{x}\}$
denotes the capture-avoiding substitution of $\vec{y}$ for $\vec{x}$ in $Q$.
\end{definition}

\begin{definition}
  The {\em structural congruence} \cite{SangiorgiWalker} , $\equiv$,
  between processes is the least congruence containing
  alpha-equivalence, satisfying the abelian monoid laws
  (associativity, commutativity and $\pzero$ as identity) for parallel
  composition $|$ and for summation $+$.
\end{definition}

\subsection{Name equivalence}

We take name equivalence, written $\nameeq$, to be the smallest
equivalence relation generated by the following rules.

\begin{mathpar}
\inferrule*[lab=Quote-drop]
{ }
{ \quotep{@{x}} \nameeq x }

\inferrule*[lab=Struct-equiv]
{ P \scong Q }
{ \quotep{P} \nameeq \quotep{Q} }
\end{mathpar}

The astute reader will have noticed that the mutual recursion of names
and processes imposes a mutual recursion on alpha-equivalence and
structural equivalence via name-equivalence. Fortunately, all of this
works out pleasantly and we may calculate in the natural way, free of
concern. The reader interested in the details is referred to the
appendix \ref{appendix:rho_details}.

\subsection{Substitution}

We use $\Proc$ for the set of processes, $\QProc$ for the set of
names, and $\id{\{}\vec{y} / \vec{x} \id{\}}$ to denote partial maps,
$s : \QProc \rightarrow \QProc$. A map, $s$ lifts, uniquely, to a map
on process terms, $\widehat{s} : \Proc \rightarrow \Proc$ by the
following equations.

\begin{mathpar}
  (0) \psubstp{Q}{P} := 0 \\
  (R \juxtap S) \psubstp{Q}{P}
  :=    
  (R)\psubstp{Q}{P} \juxtap (S) \psubstp{Q}{P} \\
  (x?(y).R) \psubstp{Q}{P}    
  :=    
  (x)\substp{Q}{P} (z)\concat( (R \psubstn{z}{y}) \psubstp{Q}{P} ) \\
  (\lift{x}{R}) \psubstp{Q}{P}  
  :=
  \lift{(x)\substp{Q}{P}}{ R \psubstp{Q}{P} } \\
%   (\dropn{x})  \psubstp{Q}{P}       
%   := 
%   \left\{ 
%     \begin{array}{ccc} 
%       \dropn{\quotep{Q}} & & x \nameeq \quotep{P} \\
%       \dropn{x} & & otherwise \\
%     \end{array}
%   \right. 
  (\dropn{x})  \psubstp{Q}{P}       
  := 
  \left\{ 
    \begin{array}{ccc} 
      Q & & x \nameeq \quotep{P} \\
      \dropn{x} & & otherwise \\
    \end{array}
  \right.
\end{mathpar}
 

where

\begin{eqnarray}
  (x)\id{\{} \lpquote Q \rpquote / \lpquote P \rpquote \id{\}}            = 
  \left\{ 
    \begin{array}{ccc}
      \lpquote Q \rpquote & & x \nameeq \lpquote P \rpquote \\
      x & & otherwise \\
    \end{array}
  \right. \nonumber
\end{eqnarray}

and $z$ is chosen distinct from $\quotep{P}$, $\quotep{Q}$, the free
names in $Q$, and all the names in $R$. Our $\alpha$-equivalence will
be built in the standard way from this substitution.

\begin{remark}\label{rem:no_self_referential_names}
  One consequence of these definitions is that $\forall P. \quotep{P}
  \not\in \freenames{P}$.
\end{remark}

\subsection{ Dynamic quote: an example }

Anticipating something of what's to come, consider applying the
substitution, $\widehat{\id{\{}u / z \id{\}}}$, to the following pair
of processes, $\lift{w}{y!(z)}$ and $w[ \lpquote y!(z) \rpquote ]$.

\begin{eqnarray}
	\lift{w}{y!(z)}\widehat{\id{\{}u / z \id{\}}}
		& = &
		\lift{w}{y!(u)} \nonumber\\
	w[ \lpquote y!(z) \rpquote ] \widehat{ \id{\{}u / z \id{\}} }
		& = &
		w[ \lpquote y!(z) \rpquote ] \nonumber
\end{eqnarray}

Because the body of the process between quotes is impervious to
substitution, we get radically different answers. In fact, by
examining the first process in an input context,
e.g. $x?(z).\lift{w}{y!(z)}$, we see that the process under the lift
operator may be shaped by prefixed inputs binding a name inside it. In
this sense, the lift operator will be seen as a way to dynamically
construct processes before reifying them as names.

Finally equipped with these standard features we can present the
dynamics of the calculus.

\subsubsection{Operational semantics} 

Finally, we introduce the computational dynamics. What marks these
algebras as distinct from other more traditionally studied algebraic
structures, e.g. vector spaces or polynomial rings, is the manner in
which dynamics is captured. In traditional structures, dynamics is typically
expressed through morphisms between such structures, as in linear maps
between vector spaces or morphisms between rings. In algebras
associated with the semantics of computation, the dynamics is
expressed as part of the algebraic structure itself, through a
reduction reduction relation typically denoted by $\red$. Below, we
give a recursive presentation of this relation for the calculus used
in the encoding.

$\red \subseteq \pi \times \pi$
$\red : \pi \to \mathcal{P}(\pi)$

\begin{mathpar}
  \inferrule* [lab=Comm] { \textsf{match}( x_{src}, x_{trgt} ) } { x_{trgt}?(y)P \; | \; x_{src}!\langle {Q} \rangle \red P\{\quotep{Q}/y}\} }
  \and \\
  \inferrule* [lab=Par] {{P} \red {P}'} {{{P} | {Q}} \red {{P}' | {Q}}}
  \and
  \inferrule* [lab=Equiv]{{{P} \scong {P}'} \andalso {{P}' \red {Q}'} \andalso {{Q}' \scong {Q}}}{{P} \red {Q}}
\end{mathpar}

\begin{eqnarray*}
  match_{\equiv} (\quotep{P},\quotep{Q}) & := & P \equiv Q \\
  match_{\dagger}(\quotep{P},\quotep{Q}) & := & \forall R. P|Q \red^{*} R => R \red^{*} 0 \\
  match_{K}(\quotep{P},\quotep{Q}) & := & K \mbox{ for some context } K
\end{eqnarray*}

$u?(x)P | u!\langle Q \rangle \red P\{\quotep{Q}/x\}$

%We write $\wred$ for $\red^*$, and $P\red$ if $\exists Q $ such that $ P \red Q$.
We write $P\red$ if $\exists Q $ such that $ P \red Q$ and $P\not\red$, otherwise.

\section{Replication}

As mentioned before, it is known that replication (and hence
recursion) can be implemented in a higher-order process algebra
\cite{SangiorgiWalker}. As our first example of calculation with the
machinery thus far presented we give the construction explicitly in
the {\rhoc}.

\begin{eqnarray}
	D_{x} & := & \prefix{x}{y}{(\binpar{\outputp{x}{y}}{@{y}})} \nonumber\\
	\bangp_{x}{P} & := & \binpar{{x}!\langle{\binpar{D_{x}}{P}}\rangle}{D_{x}} \nonumber
\end{eqnarray}

\begin{eqnarray}
	\bangp_{x}{P} & & \nonumber\\
	=
	& {x}!\langle{(\prefix{x}{y}{(\outputp{x}{y} | @{y})) | P}}\rangle 
	      | \prefix{x}{y}{(\outputp{x}{y} | @{y})} & \nonumber\\
	\red
	& (\outputp{x}{y} | @{y})\substn{\quotep{(\prefix{x}{y}{(@{y} | \outputp{x}{y})) | P}}}{y} & \nonumber\\
	=
	& \outputp{x}{\quotep{(\prefix{x}{y}{(\outputp{x}{y} | @{y})) | P}}}
	  | {(\prefix{x}{y}{(\outputp{x}{y} | @{y})) | P}} & \nonumber\\
	\red
	& \ldots & \nonumber\\
	\red^*
	& P | P | \ldots & \nonumber
\end{eqnarray}

Of course, this encoding, as an implementation, runs away, unfolding
$\bangp{P}$ eagerly. A lazier and more implementable replication
operator, restricted to input-guarded processes, may be obtained as follows.

\begin{eqnarray}
\bangp{\prefix{u}{v}{P}} 
	:= 
	\binpar{\lift{x}{\prefix{u}{v}{(\binpar{D(x)}{P})}}}{D(x)} \nonumber
\end{eqnarray}

\begin{remark}
  Note that the lazier definition still does not deal with summation
  or mixed summation (i.e. sums over input and output). The reader is
  invited to construct definitions of replication that deal with these
  features. 

  Further, the definitions are parameterized in a name, $x$. Can you,
  gentle reader, make a definition that eliminates this parameter and
  guarantees no accidental interaction between the replication
  machinery and the process being replicated -- i.e. no accidental
  sharing of names used by the process to get its work done and the
  name(s) used by the replication to effect copying. This latter
  revision of the definition of replication is crucial to obtaining
  the expected identity $!!P \sim !P$.
\end{remark}

\begin{remark}\label{rem:paradoxical_combinator}
  The reader familiar with the lambda calculus will have noticed the
  similarity between $D$ and the paradoxical combinator.

  [Ed. note: the existence of this seems to suggest we have to be more
  restrictive on the set of processes and names we admit if we are to
  support no-cloning.]
\end{remark}

\subsubsection{Bisimulation}

The computational dynamics gives rise to another kind of equivalence,
the equivalence of computational behavior. As previously mentioned
this is typically captured \emph{via} some form of bisimulation.

% The notion we use in this paper is weak barbed bisimulation
% \cite{milner91polyadicpi}.

The notion we use in this paper is derived from weak barbed
bisimulation \cite{milner91polyadicpi}. 

\begin{definition}
An \emph{observation relation}, $\downarrow_{\mathcal N}$, over a set
of names, $\mathcal N$, is the smallest relation satisfying the rules
below.

\infrule[Out-barb]{y \in {\mathcal N}, \; x \nameeq y}
		  {\outputp{x}{v} \downarrow_{\mathcal N} x}
\infrule[Par-barb]{\mbox{$P\downarrow_{\mathcal N} x$ or $Q\downarrow_{\mathcal N} x$}}
		  {\binpar{P}{Q} \downarrow_{\mathcal N} x}

We write $P \Downarrow_{\mathcal N} x$ if there is $Q$ such that 
$P \wred Q$ and $Q \downarrow_{\mathcal N} x$.
\end{definition}

\begin{definition}
%\label{def.bbisim}
An  ${\mathcal N}$-\emph{barbed bisimulation} over a set of names, ${\mathcal N}$, is a symmetric binary relation 
${\mathcal S}_{\mathcal N}$ between agents such that $P\rel{S}_{\mathcal N}Q$ implies:
\begin{enumerate}
\item If $P \red P'$ then $Q \wred Q'$ and $P'\rel{S}_{\mathcal N} Q'$.
\item If $P\downarrow_{\mathcal N} x$, then $Q\Downarrow_{\mathcal N} x$.
\end{enumerate}
$P$ is ${\mathcal N}$-barbed bisimilar to $Q$, written
$P \wbbisim_{\mathcal N} Q$, if $P \rel{S}_{\mathcal N} Q$ for some ${\mathcal N}$-barbed bisimulation ${\mathcal S}_{\mathcal N}$.
\end{definition}

$\mathcal{R} \subseteq \pi \times \pi$

$P \mathcal{R} Q => \forall P'. P \red P' \Rightarrow \exists Q'. Q \red Q', P' \mathcal{R} Q'$

$P \vdash x \Rightarrow Q \vdash x$

\begin{mathpar}
  \inferrule*[lab=Out-barb]{x \nameeq y}{{y}!\langle{Q}\rangle \vdash x}
  \and
  \inferrule*[lab=Par-barb]{\mbox{$P\vdash x$ or $Q\vdash x$}}{\binpar{P}{Q} \vdash x}
\end{mathpar}

\subsubsection{Contexts}

One of the principle advantages of computational calculi like the
$\pi$-calculus is a well-defined notion of context,
contextual-equivalence and a correlation between
contextual-equivalence and notions of bisimulation. The notion of
context allows the decomposition of a process into (sub-)process and
its syntactic environment, its context. Thus, a context may be
thought of as a process with a ``hole'' (written $\Box$) in it. The
application of a context $M$ to a process $P$, written $M[P]$, is
tantamount to filling the hole in $M$ with $P$. In this paper we do
not need the full weight of this theory, but do make use of the notion
of context in the proof the main theorem. 

\begin{mathpar}
  \inferrule* [lab=summation] {} {{M_{M},M_{N}} \bc \Box \;|\; x.M_{A} \;|\; M_{M}+M_{N}}
  \and
  \inferrule* [lab=agent] {} {{M_{A}} \bc (\vec{x})M_{P} \;| \; \clift{P_0,\ldots,M_{P},\ldots,P_N}}
  \and \\
  \inferrule* [lab=process] {} {{M_{P}} \bc M_{N} \;| \;P|M_{P} }
\end{mathpar} 

\begin{mathpar}
  \inferrule* [lab=sychronization] {} {M_{N} \bc \Box \;|\; x?M_{F} \;|\; x!M_{C}}
  \and
  \inferrule* [lab=abstraction] {} {{M_{F}} \bc (x)M_{P} }
  \and
  \inferrule* [lab=concretion] {} {{M_{C}} \bc \langle M_{P} \rangle }
  \and \\
  \inferrule* [lab=process] {} {{M_{P}} \bc M_{N} \;| \;P|M_{P} }
\end{mathpar}

\begin{definition}[contextual application] Given a context $M$, and
  process $P$, we define the \emph{contextual application}, $M[P] :=
  M\{P/\Box\}$. That is, the contextual application of M to P is the
  substitution of $P$ for $\Box$ in $M$.
\end{definition}

$\meaningof{-} : L \to \mathcal{P}(\pi)$

\begin{mathpar}
  \inferrule* [lab=collection] {} {\meaningof{true} = \pi, \and \meaningof{~E} = \pi \setminus \meaningof{E}, \and \meaningof{E_{1} \& E_{2}} = \meaningof{E_{1}} \cap \meaningof{E_{2}}}
\end{mathpar}

\begin{mathpar}
  \inferrule* [lab=structure] {} {\meaningof{0} = \{ P \in \pi | P \equiv 0 \}, \and \\ \meaningof{E_1 | E_2} = \{ P \in \pi | P \equiv P_{1} | P_{2}, P_{1} \in \meaningof{E_{1}}, P_{2} \in \meaningof{E_2}\} }
\end{mathpar}

\begin{mathpar}
 \inferrule* [lab=behavior] {} {\meaningof{\langle a?b \rangle E} = \{ P \in \pi | P \equiv Q | u?(y)P', \\ \and \\\\ \and \\ \;\;\; u \in \meaningof{a}, \forall z.P'\{z/y\} \in \meaningof{E\{z/b\}}\}, \and \\ \meaningof{a!E} = \{ P \in \pi | P \equiv Q | x!\langle P' \rangle, x \in \meaningof{a} P' \in \meaningof{E}\} }
\end{mathpar}

\begin{mathpar}
 \inferrule* [lab=nominal] {} {\meaningof{\quotep{E}} = \{ \quotep{P} \in \quotep{\pi} | P \in \meaningof{E} \}, \and \meaningof{\quotep{P}} = \{ \quotep{Q} \in \quotep{\pi} | P \equiv Q \} \and \\ \meaningof{@\quotep{E}} = \{ P \in \pi | P \equiv @x, x \in \meaningof{E} \}}
\end{mathpar}

\begin{eqnarray*}
  \\
  \meaningof{-} : TS \to ST
\end{eqnarray*}

\begin{eqnarray*}
  \\
  L : TS \to ST
\end{eqnarray*}

\begin{eqnarray*}
  \\
  P \models E \iff P \in \meaningof{E}
\end{eqnarray*}

\begin{eqnarray*}
  P \approx_{L} Q \iff \forall E \in L. P \models E \iff Q \models E
\end{eqnarray*}

\begin{eqnarray*}
  P \approx_{K} Q
\end{eqnarray*}

\begin{eqnarray*}
  P \approx Q
\end{eqnarray*}

$\approx_{K} = \approx = \approx_{L}$

\subsubsection{Contextual duality}

Note that contexts extend the quotation operation to a family of
operations from processes to names. Given a context, $M$, we can
define a \emph{nominal context}, $\quotep{M}$ by $\quotep{M}[P] :=
\quotep{M[P]}$. To foreshadow what is to come we observe that these
operations enjoy a duality with processes very much like the duality
between vectors and maps from vectors to scalars.

Further, because the calculus is essentially higher-order, we have a
correspondence between contexts and processes. More specifically,
given a name $x$ and a context $M$ we can construct $M^{*}_{x}$ such
that 

\begin{mathpar}
  M^{*}_{x} | \lift{x}{P} \red M[P]
\end{mathpar}

namely,

\begin{mathpar}
  M^{*}_{x} := x?(u).M[\dropn{u}]
\end{mathpar}

The dependence of $M^{*}_{x}$ on a name makes it an abstraction, 

\begin{mathpar}
  M^{*} := (x)x?(u).M[\dropn{u}]
\end{mathpar}

\subsection{Additional notation}

It will sometimes be convenient to denote the process a name
quotes. We already have the notation $x = \quotep{P}$, but it will be
convenient to introduce an alternate notation, $\procn{x}$, when we
want to emphasize the connection to the use of the name. Note that, by
virtue of name equivalence, $\quotep{\procn{x}} \nameeq x$; so, the
notation is consistent with previous definitions.

Further, because names have structure it is possible to effect
substitutions on the basis of that structure. This means we need to
upgrade our notation for substitutions, which we accomplish by
adapting comprehension notation. Thus,

\begin{mathpar}
  P\{ y / x : x \in S \}
\end{mathpar}

is interpreted to mean the process derived from P by replacing (in a
capture-avoiding manner) each occurrence of $x$ in $S$ by $y$. For example,

\begin{mathpar}
  P\{ \quotep{\procn{x}|\procn{x}} / x : x \in \freenames{P} \}
\end{mathpar}

will replace each (occurrence) of a free name $x$ in $P$ by
$\quotep{\procn{x}|\procn{x}}$.

Also, we will avail ourselves of the notation $x^{L}$ and $x^{R}$ to
denote injections of a name into disjoint copies of the name
space. There are numerous ways to accomplish this. One example can be
found in \cite{MeredithR05}. This notation overloads to vectors of
names: $\vec{x}^{\pi} := (x_{i}^{\pi} \; : \; 0 \leq i < |\vec{x}| )$ where $\pi \in \{L,R\}$.

We also use $P^{\Box} := P|\Box$.

In \cite{MeredithR05} an interpretation of the new operator is
given. It turns out that there are several possible interpretations
all enjoying the requisite algebraic properties of the operator (see
\cite{milner91polyadicpi}). We will therefore make liberal use of
$(\nu\; \vec{x})P$.

% subsection the_syntax_and_semantics_of_the_notation_system (end)   

\section{Interpretation of QM}
\subsection{Supporting definitions}
\subsubsection{Multiplication}
\begin{mathpar}
  \quotep{Q} \cdot \quotep{R} := \quotep{Q|R}
  \and \\
  \quotep{Q} \cdot P := P\{ \quotep{Q|R} / \quotep{R} : \quotep{R} \in \freenames{P} \}
\end{mathpar}

\paragraph{Discussion}
The first line needs little explanation. The second line says that
each free name of the process is replaced with the multiplication of
that name by the scalar. Multiplication of a scalar (name) by a state
(process) results in a process all the names of which have been `moved
over' by parallel composition with the process the scalar
quotes. There is a subtlety that the bound names have to be
manipulated so that multiplied names aren't accidentally
captured. There are many ways to achieve this.

\begin{remark}\label{rem:multiplication_identities}
  The reader is invited to verify that for all $x,y,z \in \QProc$ and $P \in \Proc$
  \begin{mathpar}
    x \cdot \quotep{0} \equiv x 
    \and
    x \cdot y \equiv y \cdot x
    \and
    x \cdot (y \cdot z) \equiv (x \cdot y) \cdot z
    \and \\
    \quotep{0} \cdot P \equiv P
    \and \\
    x \cdot (y \cdot P) \equiv (x \cdot y) \cdot P
    \and \\
    x \cdot (P|Q) \equiv (x \cdot P) | (x \cdot Q)
    \and \\    
  \end{mathpar}
\end{remark}

\subsubsection{Tensor product}

We define a tensor product on processes by structural induction.

\paragraph{Tensor of sums} First note that all summations, including
$\pzero$ and sequence, can be written $\Sigma_{i} x_{i}.A_{i} +
\Sigma_{j} x_{j}.C_{j}$, where we have grouped input-guarded processes
together and output-guarded processes together.

Thus, we can define the tensor product of two summations, $N_{1}\otimes N_{2}$, where

\begin{mathpar}
  N_{1} := \Sigma_{i} x_{i}.A_{i} + \Sigma_{j} x_{j}.C_{j}
  \and
  N_{2} := \Sigma_{i'} y_{i'}.B_{i'} + \Sigma_{j'} y_{j'}.D_{j'} 
\end{mathpar}

as follows.

\begin{mathpar}
  \Sigma_{i} x_{i}.A_{i} + \Sigma_{j} x_{j}.C_{j} \otimes \Sigma_{i'}
  y_{i'}.B_{i'} + \Sigma_{j'} y_{j'}.D_{j'} 
  \and \\
  := \; \Sigma_{i} \Sigma_{i'} \quotep{\stackrel{\vee}{x_{i}}| \stackrel{\vee}{y_{i'}}}.(A_{i}\otimes B_{i'}) \; | \; \Sigma_{i'} \Sigma_{i} \quotep{\stackrel{\vee}{y_{i'}}|\stackrel{\vee}{x_{i}}}.(B_{i'}\otimes A_{i})
  \and
  \;\; | \;\; \Sigma_{j} \Sigma_{j'} \quotep{\stackrel{\vee}{x_{j}}|\stackrel{\vee}{y_{j'}}}.(A_{j}\otimes B_{j'}) \; | \; \Sigma_{j'} \Sigma_{j} \quotep{\stackrel{\vee}{y_{j'}}|\stackrel{\vee}{x_{j}}}.(B_{j'}\otimes A_{j})
\end{mathpar}

\begin{remark}
  Do we need to $x^{L}$ and $y^{R}$ for this construction as well?
\end{remark}

\paragraph{Tensor of parallel compositions} Next, we distribute tensor
over par.

\begin{mathpar}
  P_{1}|P_{2} \otimes Q_{1}|Q_{2} := (P_{1} \otimes Q_{1}) | (P_{1}
  \otimes Q_{2}) | (P_{2} \otimes Q_{1}) | (P_{2} \otimes Q_{2})
\end{mathpar}

\paragraph{Tensor with dropped names} We treat tensor of a
process with a dropped name as parallel composition.

\begin{mathpar}
  P \otimes \dropn{x} := P | \dropn{x}
\end{mathpar}

\paragraph{Tensor of agents}

Finally, we need to define tensor on agents. Note that the definition
of tensor on normal products only tensors inputs with inputs and
outputs with outputs. Thus, we only have to define the operation on
``homogeneous'' pairings.

\begin{mathpar}
  (\vec{x})P \otimes (\vec{y})Q
  \and \\
  := (x_{0}^{L}|y_{0}^{R},\ldots,x_{0}^{L}|y_{n}^{R},\ldots,x_{m}^{L}|y_{0}^{R},\ldots,x_{m}^{L}|y_{n}^R)(P\{ \vec{x}^{L}/\vec{x}\} \otimes Q \{ \vec{y}^{R}/\vec{y}\})
  \and \\
  \clift{\vec{P}} \otimes \clift{\vec{Q}}
  \and \\
  := \clift{P_{0}\otimes Q_{0},\ldots,P_{0}\otimes Q_{n},\ldots,P_{m}\otimes Q_{0},\ldots,P_{m}\otimes Q_{n}}
\end{mathpar}

\begin{remark}
  Observe that arities of tensored abstractions matches arities of
  tensored concretions if the original arities matched. Note also that
  the length of the arities corresponds to the increase in dimension
  we see in ordinary vector space tensor product.
\end{remark}

\begin{remark}
  Operationally, this definition distributes the tensor down to
  components ``linked'' by summation. Tensor over summation is
  intriguing in that it mixes names. Moreover, as a consequence of the
  way it mixes names we have the identities for all $x \in \QProc$ and
  $P,Q \in \Proc$

  \begin{mathpar}
    (x \cdot P) \otimes Q \equiv x \cdot (P \otimes Q) \equiv P \otimes (x \cdot Q)
    \and
    P \otimes \pzero \equiv P
  \end{mathpar}

  that the reader is invited to verify.
\end{remark}

\subsubsection{Annihilation}
\begin{mathpar}
  P^{\perp} := \{ Q | \forall R. P|Q \red^{*} R \Rightarrow R \red^{*} \pzero \}
  \and \\
  P^{\underline{\perp}} := \Sigma_{Q \in P^{\perp}} \quotep{Q}?(y).(\dropn{y}|Q) | \Sigma_{Q \in P^{\perp}} \quotep{Q}\clift{\Box}
\end{mathpar}

\paragraph{Discussion} The reader will note that $P^{\perp}$ is a
\emph{set} of processes, while $P^{\underline{\perp}}$ is a
\emph{context}. We call the set $P^{\perp}$ the \emph{annihilators} of
$P$. The parallel composition of a process in the annihilators of $P$
with $P$ will result in a process, the state space of which has all
paths eventually leading to $\pzero$. Execution may endure loops; but
under reasonable conditions of fairness (naturally guaranteed under
most notions of bisimulation) such a composite process cannot get
stuck in such a loop and will, eventually pop out and terminate.

The context $P^{\underline{\perp}}$ is ready and willing to ``take the
$P$ out of'' the process to which it is applied. It will effectively
transmit the code of the process to which it is applied to one of the
annihilators and run the process against it.

\subsubsection{Evaluation}
We fix $M$ a domain of fully abstract interpretation with an equality
coincident with bisimulation. We take $\meaningof{\cdot} : \Proc \to
M$ to be the map interpreting processes and $\nmeaningof{\cdot} : \M
\to Proc$ to be the map running the other way. Then we define

\begin{mathpar}
  \int P := \nmeaningof{\meaningof{P}}
\end{mathpar}

\paragraph{Discussion}
There are many fully abstract interpretations of Milner's
$\pi$-calculus. Any of them can be used as a basis for interpreting
the reflective calculus here. Equipped with such a domain it is
largely a matter of grinding through to check that the Yoneda
construction for the normalization-by-evaluation program can be
extended to this setting.

\begin{remark}
  The reader is invited to verify that $\int (P^{\underline{\perp}}[P]) = 0$.
\end{remark}

\subsection{Quantum mechanics}

Table \ref{tbl:core_qm_op_defns} gives the core operational definitions

\begin{table}[htp]\label{tbl:core_qm_op_defns}
  \center{
    \fbox{
      \begin{tabular}{c|c}
        quantum mechanics & process calculus \\
        \hline
        scalar & $x := \quotep{P}$ \\
        state vector & $\state{P} := P$ \\
        dual & $\state{P}^{*} := \event{P^{\underline{\perp}}} := \quotep{P^{\underline{\perp}}}[-]$ \\
        matrix & $ \Sigma_{\alpha} \state{P_{\alpha}}x_{\alpha}\event{Q_{\alpha}}$ \\
        vector addition & $\state{P} + \state{Q} := \state{P | Q}$ \\
        tensor product & $\state{P} \otimes \state{Q} := \state{P \otimes Q}$ \\
        inner product & $\innerprod{P}{Q} := \quotep{\int P^{\underline{\perp}}[Q]}$ \\
      \end{tabular}
    }
  }
  \caption{QM - operational definitions}
\end{table}

where

\begin{mathpar}
  \prmatrix{P}{Q} := \fprmatrix{P}{\quotep{\pzero}}{Q}
  \and
  \fprmatrix{P}{x}{Q} := (\state{P},x,\event{Q})
  \and
  (\fprmatrix{P}{x}{Q})(\state{R}) := x \cdot \innerprod{Q}{R} \cdot \state{P}
  \and
  (\fprmatrix{P}{x}{Q})(\event{R}) := x \cdot \innerprod{R}{P} \cdot \event{Q}
\end{mathpar}

\paragraph{Discussion}
As promised: vectors (aka states) are represented as processes; duals
as contextual duals; inner product definition should be compared with
standard inner product definition for ....

\begin{remark}
  Assuming $\int (P^{\underline{\perp}}[P]) = 0$, the reader is
  invited to verify that $(\fprmatrix{P}{x}{P})(\state{P}) = x \cdot \state{P}$.
\end{remark}

\begin{remark}
  The reader is invited to verify that $\innerprod{P}{Q}$ could
  equally well have been written $\quotep{\int \stackrel{\vee}{x}}$
  where $x = \event{P^{\underline{\perp}}}(Q)$.

  One of the motivations for this remark is that there is another way
  to factor these operations. We could package up evaluation in the dual:

  \begin{mathpar}
    \state{P}^{*} := \event{\int P^{\underline{\perp}}} := \quotep{\int P^{\underline{\perp}}}[-]
  \end{mathpar}

  and then have inner product defined by
  
  \begin{mathpar}
    \innerprod{P}{Q} := \event{P}(Q)
  \end{mathpar}

  Hopefully, experience with the calculations will provide guidance on
  the best factoring.
\end{remark}

\begin{remark}
  Assuming $\int (P^{\underline{\perp}}[P]) = 0$, the reader is
  invited to verify that $\forall P,Q. (\prmatrix{0}{Q})(\state{0}) =
  \state{0}$ and dually $(\prmatrix{P}{0})(\event{0}) = \event{0}$.
\end{remark}

\begin{remark}
  i'm a little worried that i don't (yet) have proper support for
  complex conjugacy. But, the observation above may give us a
  clue. According to Abramsky, it must be the case that the scalars
  are iso to the homset of the identity for the tensor -- which the
  observation above characterizes. 

  For now, we will simply bookmark the notion with $\overline{x}$.
\end{remark}

\subsubsection{Adjointness}

We need to give a definition of $(\cdot)^{\dagger}$ for matrices. The
obvious candidate definition is
\begin{mathpar}
(\Sigma_{\alpha}\fprmatrix{P_{\alpha}}{x_{\alpha}}{Q_{\alpha}})^{\dagger}
= \Sigma_{\alpha}\fprmatrix{(Q_{\alpha}^{\underline{\perp}})^{*}}{\overline{x}_{\alpha}}{P_{\alpha}^{\underline{\perp}}} 
\end{mathpar}

But, $(Q_{\alpha}^{\underline{\perp}})^{*}$ requires a name along
which to communicate the process to achieve the context application.

\subsubsection{Basis for a basis}
If processes label states and ``addition'' of states (a.k.a. vector
addition) is interpreted as parallel composition, what corresponds to
notions of linear independence and basis? Here, we recall that Yoshida
has developed a set of \emph{combinators} for an asynchronous verison
of Milner's $\pi$-calculus. These are a finite set of processes such
any process can be expressed as parallel composition of these
combinators together with liberal uses of the new operator and
replication. We can simply give a translation of these into the
present calculus and have reasonable expectation that the property
carries over. That is, that the resultant set allows to express all
processes via parallel composition. Note, however, that there is no
new operator or replication in this calculus. As a result, we expect
that the corresponding set is actually infinite. That is, we expect
that the space is actually infinite dimensional.

\begin{remark}
  The attentive reader may be a bit concerned. Certainly, the
  collection $S$, $K$ and $I$ is a finite set of
  combinators. Shouldn't we expect to see a finite set of combinators
  for an effectively equivalent system? i am very sympathetic to this
  critique and feel it warrants full attention. On the other hand, i
  also have in mind the following analogy. The natural numbers, as a
  monoid under addition, has exactly $1$ generator, while the natural
  numbers, as a monoid under multiplication, has countably many
  generators (the primes). We observe that the application of the
  lambda calculus is much less resource sensitive than the parallel
  composition of the $\pi$-calculus. Could it be the case that we have
  an analogy of the form
  
  \begin{mathpar}
    m + n : MN :: m*n : M|N
  \end{mathpar}

  giving a similar blow up in the set of ``primes''?  This is such a
  wonderful thought that, even if it's not true, i think it's worth
  writing down.
\end{remark}
 

\documentclass[12pt]{llncs}
%\documentclass{jktr}

\usepackage[pdftex]{hyperref}                   
\usepackage {listings}
\usepackage {mathpartir}
\usepackage{bcprules}
%\usepackage{listings}
                       
\usepackage{graphicx} 
%\usepackage[margins=2.5cm,nohead,nofoot]{geometry}
%\usepackage{geometry}
\usepackage{amsfonts}
\usepackage{amstext}
\usepackage{latexsym}
\usepackage{amssymb}
\usepackage{color}


%\include{myPreamble}
\include{qm2pi.local} 

%\ifpdf
%\usepackage[pdftex]{graphicx}
%\else
%\usepackage{graphicx}
%\fi

 % \ifpdf
%  \usepackage{pdfsync}
%  \if


%\title{Brief Article}
%\author{David F. Snyder}
%\author{L.G. Meredith}

%\address{Dept. of Math., Texas State University--San Marcos, San Marcos, TX 78666}
       
\pagestyle{empty}


\begin{document}

\lstset{language=[Objective]Caml,frame=shadowbox}

\input{qm2pi.front}

% section front matter (end)

\input{qm2pi.intro} 
 
% section introduction (end)

% \input{qm2pi.knotations} 

% section notation (end)

\input{qm2pi.process.calculi} 

% section concurrent_process_calculi_and_spatial_logics_ (end)
    
%\input{qm2pi.knots2pi} 

%\input{qm2pi.trefoil} 

%\input{qm2pi.mainthm} 

% subsection basic_interpretation (end)

%\input{qm2pi.rho.presentation} 
\subsection{The syntax and semantics of the notation system}\label{sub:the_syntax_and_semantics_of_the_notation_system} % (fold)

We now summarize a technical presentation of the calculus that
embodies our theory of dynamics. The typical presentation of such a
calculus follows the style of giving generators and relations on
them. The grammar, below, describing term constructors, freely
generates the set of processes, $\Proc$. This set is then quotiented
by a relation known as structural congruence and it is over this set
that the notion of dynamics is expressed. This presentation is
essentially that of \cite{MeredithR05} with the addition of
polyadicity and summation. For readability we have relegated some of
the technical subtleties to an appendix.

\subsubsection{Process grammar}\label{subsub:process_grammar}

\begin{mathpar}
  \inferrule* [lab=synchronization] {} {{M} \bc \pzero \;|\; x?F \;|\; x!C }
  \and
  \inferrule* [lab=abstraction] {} {{F} \bc (x)P}
  \and
  \inferrule* [lab=concretion] {} {{C} \bc \langle Q \rangle}
  \and
  \inferrule* [lab=process] {} {{P,Q} \bc M \;| \;P|Q \;|\; @{x}}
  \and
  \inferrule* [lab=name] {} {{x} \bc \quotep{P}}
\end{mathpar} 

Note that $\vec{x}$ (resp. $\vec{P}$) denotes a vector of names
(resp. processes) of length $|\vec{x}|$ (resp. $|\vec{P}|$). We adopt
the following useful abbreviations.

\begin{mathpar}
   x?(\vec{y}).P := x.(\vec{y})P \and  x\clift{\vec{P}} := x.\clift{\vec{P}}
   \and x!(y) := \lift{x}{\dropn{y}}
   \and \Pi_{i=0}^{n-1}P_i := P_0 | \ldots | P_{n-1}
\end{mathpar}

\subsubsection{Structural congruence}

\paragraph{Free and bound names and alpha-equivalence.} At the
core of structural equivalence is alpha-equivalence which identifies
process that are the same up to a change of variable. Formally, we
recognize the distinction between free and bound names. The free names
of a process, $\freenames{P}$, may be calculated recursively as
follows:

\begin{mathpar}
\freenames{\pzero} := \emptyset
  \and \\
  \freenames{x?(y).P} := \{ x \} \cup (\freenames{P} \setminus \{ y \})
  \and 
  \freenames{x!\langle P \rangle} := \{ x \} \cup \{ P \} 
  \and \\
  \freenames{P|Q} := \freenames{P} \cup \freenames{Q}
  \and \\
  \freenames{@{x}} := \{ x \}
\end{mathpar}

$\pi$
$\quotep{\pi}$

$\freenames{-} : \pi \to \mathcal{P}(\quotep{\pi})$

\begin{eqnarray*}
  \freenames{\pzero} & := & \emptyset \\
  \freenames{x?(y).P} & := & \{ x \} \cup (\freenames{P} \setminus \{ y \}) \\
  \freenames{x!\langle P \rangle} & := & \{ x \} \cup \{ P \} \\
  \freenames{P|Q} & := & \freenames{P} \cup \freenames{Q} \\
  \freenames{\dropn{x}} & := & \{ x \}
\end{eqnarray*}

The bound names of a process, $\boundnames{P}$, are those names occurring in $P$
that are not free. For example, in $x?(y).0$, the name $x$ is free, while $y$ is bound.

\begin{mathpar}
  \inferrule* [lab=monoidal-laws] {} { P|Q \equiv Q|P \and P|0 \equiv P \and P|(Q|R) \equiv (P|Q)|R }
\end{mathpar}

\begin{mathpar}
  \inferrule* [lab=alpha-equivalence] {} { (x)P \equiv (y)P\{y/x\} \and y \not\in \freenames{P} }
\end{mathpar}

\begin{definition}
Then two processes, $P,Q$, are alpha-equivalent if $P = Q\{\vec{y}/\vec{x}\}$ for
some $\vec{x} \in \boundnames{Q},\vec{y} \in \boundnames{P}$, where $Q\{\vec{y}/\vec{x}\}$
denotes the capture-avoiding substitution of $\vec{y}$ for $\vec{x}$ in $Q$.
\end{definition}

\begin{definition}
  The {\em structural congruence} \cite{SangiorgiWalker} , $\equiv$,
  between processes is the least congruence containing
  alpha-equivalence, satisfying the abelian monoid laws
  (associativity, commutativity and $\pzero$ as identity) for parallel
  composition $|$ and for summation $+$.
\end{definition}

\subsection{Name equivalence}

We take name equivalence, written $\nameeq$, to be the smallest
equivalence relation generated by the following rules.

\begin{mathpar}
\inferrule*[lab=Quote-drop]
{ }
{ \quotep{@{x}} \nameeq x }

\inferrule*[lab=Struct-equiv]
{ P \scong Q }
{ \quotep{P} \nameeq \quotep{Q} }
\end{mathpar}

The astute reader will have noticed that the mutual recursion of names
and processes imposes a mutual recursion on alpha-equivalence and
structural equivalence via name-equivalence. Fortunately, all of this
works out pleasantly and we may calculate in the natural way, free of
concern. The reader interested in the details is referred to the
appendix \ref{appendix:rho_details}.

\subsection{Substitution}

We use $\Proc$ for the set of processes, $\QProc$ for the set of
names, and $\id{\{}\vec{y} / \vec{x} \id{\}}$ to denote partial maps,
$s : \QProc \rightarrow \QProc$. A map, $s$ lifts, uniquely, to a map
on process terms, $\widehat{s} : \Proc \rightarrow \Proc$ by the
following equations.

\begin{mathpar}
  (0) \psubstp{Q}{P} := 0 \\
  (R \juxtap S) \psubstp{Q}{P}
  :=    
  (R)\psubstp{Q}{P} \juxtap (S) \psubstp{Q}{P} \\
  (x?(y).R) \psubstp{Q}{P}    
  :=    
  (x)\substp{Q}{P} (z)\concat( (R \psubstn{z}{y}) \psubstp{Q}{P} ) \\
  (\lift{x}{R}) \psubstp{Q}{P}  
  :=
  \lift{(x)\substp{Q}{P}}{ R \psubstp{Q}{P} } \\
%   (\dropn{x})  \psubstp{Q}{P}       
%   := 
%   \left\{ 
%     \begin{array}{ccc} 
%       \dropn{\quotep{Q}} & & x \nameeq \quotep{P} \\
%       \dropn{x} & & otherwise \\
%     \end{array}
%   \right. 
  (\dropn{x})  \psubstp{Q}{P}       
  := 
  \left\{ 
    \begin{array}{ccc} 
      Q & & x \nameeq \quotep{P} \\
      \dropn{x} & & otherwise \\
    \end{array}
  \right.
\end{mathpar}
 

where

\begin{eqnarray}
  (x)\id{\{} \lpquote Q \rpquote / \lpquote P \rpquote \id{\}}            = 
  \left\{ 
    \begin{array}{ccc}
      \lpquote Q \rpquote & & x \nameeq \lpquote P \rpquote \\
      x & & otherwise \\
    \end{array}
  \right. \nonumber
\end{eqnarray}

and $z$ is chosen distinct from $\quotep{P}$, $\quotep{Q}$, the free
names in $Q$, and all the names in $R$. Our $\alpha$-equivalence will
be built in the standard way from this substitution.

\begin{remark}\label{rem:no_self_referential_names}
  One consequence of these definitions is that $\forall P. \quotep{P}
  \not\in \freenames{P}$.
\end{remark}

\subsection{ Dynamic quote: an example }

Anticipating something of what's to come, consider applying the
substitution, $\widehat{\id{\{}u / z \id{\}}}$, to the following pair
of processes, $\lift{w}{y!(z)}$ and $w[ \lpquote y!(z) \rpquote ]$.

\begin{eqnarray}
	\lift{w}{y!(z)}\widehat{\id{\{}u / z \id{\}}}
		& = &
		\lift{w}{y!(u)} \nonumber\\
	w[ \lpquote y!(z) \rpquote ] \widehat{ \id{\{}u / z \id{\}} }
		& = &
		w[ \lpquote y!(z) \rpquote ] \nonumber
\end{eqnarray}

Because the body of the process between quotes is impervious to
substitution, we get radically different answers. In fact, by
examining the first process in an input context,
e.g. $x?(z).\lift{w}{y!(z)}$, we see that the process under the lift
operator may be shaped by prefixed inputs binding a name inside it. In
this sense, the lift operator will be seen as a way to dynamically
construct processes before reifying them as names.

Finally equipped with these standard features we can present the
dynamics of the calculus.

\subsubsection{Operational semantics} 

Finally, we introduce the computational dynamics. What marks these
algebras as distinct from other more traditionally studied algebraic
structures, e.g. vector spaces or polynomial rings, is the manner in
which dynamics is captured. In traditional structures, dynamics is typically
expressed through morphisms between such structures, as in linear maps
between vector spaces or morphisms between rings. In algebras
associated with the semantics of computation, the dynamics is
expressed as part of the algebraic structure itself, through a
reduction reduction relation typically denoted by $\red$. Below, we
give a recursive presentation of this relation for the calculus used
in the encoding.

$\red \subseteq \pi \times \pi$
$\red : \pi \to \mathcal{P}(\pi)$

\begin{mathpar}
  \inferrule* [lab=Comm] { \textsf{match}( x_{src}, x_{trgt} ) } { x_{trgt}?(y)P \; | \; x_{src}!\langle {Q} \rangle \red P\{\quotep{Q}/y}\} }
  \and \\
  \inferrule* [lab=Par] {{P} \red {P}'} {{{P} | {Q}} \red {{P}' | {Q}}}
  \and
  \inferrule* [lab=Equiv]{{{P} \scong {P}'} \andalso {{P}' \red {Q}'} \andalso {{Q}' \scong {Q}}}{{P} \red {Q}}
\end{mathpar}

\begin{eqnarray*}
  match_{\equiv} (\quotep{P},\quotep{Q}) & := & P \equiv Q \\
  match_{\dagger}(\quotep{P},\quotep{Q}) & := & \forall R. P|Q \red^{*} R => R \red^{*} 0 \\
  match_{K}(\quotep{P},\quotep{Q}) & := & K \mbox{ for some context } K
\end{eqnarray*}

$u?(x)P | u!\langle Q \rangle \red P\{\quotep{Q}/x\}$

%We write $\wred$ for $\red^*$, and $P\red$ if $\exists Q $ such that $ P \red Q$.
We write $P\red$ if $\exists Q $ such that $ P \red Q$ and $P\not\red$, otherwise.

\section{Replication}

As mentioned before, it is known that replication (and hence
recursion) can be implemented in a higher-order process algebra
\cite{SangiorgiWalker}. As our first example of calculation with the
machinery thus far presented we give the construction explicitly in
the {\rhoc}.

\begin{eqnarray}
	D_{x} & := & \prefix{x}{y}{(\binpar{\outputp{x}{y}}{@{y}})} \nonumber\\
	\bangp_{x}{P} & := & \binpar{{x}!\langle{\binpar{D_{x}}{P}}\rangle}{D_{x}} \nonumber
\end{eqnarray}

\begin{eqnarray}
	\bangp_{x}{P} & & \nonumber\\
	=
	& {x}!\langle{(\prefix{x}{y}{(\outputp{x}{y} | @{y})) | P}}\rangle 
	      | \prefix{x}{y}{(\outputp{x}{y} | @{y})} & \nonumber\\
	\red
	& (\outputp{x}{y} | @{y})\substn{\quotep{(\prefix{x}{y}{(@{y} | \outputp{x}{y})) | P}}}{y} & \nonumber\\
	=
	& \outputp{x}{\quotep{(\prefix{x}{y}{(\outputp{x}{y} | @{y})) | P}}}
	  | {(\prefix{x}{y}{(\outputp{x}{y} | @{y})) | P}} & \nonumber\\
	\red
	& \ldots & \nonumber\\
	\red^*
	& P | P | \ldots & \nonumber
\end{eqnarray}

Of course, this encoding, as an implementation, runs away, unfolding
$\bangp{P}$ eagerly. A lazier and more implementable replication
operator, restricted to input-guarded processes, may be obtained as follows.

\begin{eqnarray}
\bangp{\prefix{u}{v}{P}} 
	:= 
	\binpar{\lift{x}{\prefix{u}{v}{(\binpar{D(x)}{P})}}}{D(x)} \nonumber
\end{eqnarray}

\begin{remark}
  Note that the lazier definition still does not deal with summation
  or mixed summation (i.e. sums over input and output). The reader is
  invited to construct definitions of replication that deal with these
  features. 

  Further, the definitions are parameterized in a name, $x$. Can you,
  gentle reader, make a definition that eliminates this parameter and
  guarantees no accidental interaction between the replication
  machinery and the process being replicated -- i.e. no accidental
  sharing of names used by the process to get its work done and the
  name(s) used by the replication to effect copying. This latter
  revision of the definition of replication is crucial to obtaining
  the expected identity $!!P \sim !P$.
\end{remark}

\begin{remark}\label{rem:paradoxical_combinator}
  The reader familiar with the lambda calculus will have noticed the
  similarity between $D$ and the paradoxical combinator.

  [Ed. note: the existence of this seems to suggest we have to be more
  restrictive on the set of processes and names we admit if we are to
  support no-cloning.]
\end{remark}

\subsubsection{Bisimulation}

The computational dynamics gives rise to another kind of equivalence,
the equivalence of computational behavior. As previously mentioned
this is typically captured \emph{via} some form of bisimulation.

% The notion we use in this paper is weak barbed bisimulation
% \cite{milner91polyadicpi}.

The notion we use in this paper is derived from weak barbed
bisimulation \cite{milner91polyadicpi}. 

\begin{definition}
An \emph{observation relation}, $\downarrow_{\mathcal N}$, over a set
of names, $\mathcal N$, is the smallest relation satisfying the rules
below.

\infrule[Out-barb]{y \in {\mathcal N}, \; x \nameeq y}
		  {\outputp{x}{v} \downarrow_{\mathcal N} x}
\infrule[Par-barb]{\mbox{$P\downarrow_{\mathcal N} x$ or $Q\downarrow_{\mathcal N} x$}}
		  {\binpar{P}{Q} \downarrow_{\mathcal N} x}

We write $P \Downarrow_{\mathcal N} x$ if there is $Q$ such that 
$P \wred Q$ and $Q \downarrow_{\mathcal N} x$.
\end{definition}

\begin{definition}
%\label{def.bbisim}
An  ${\mathcal N}$-\emph{barbed bisimulation} over a set of names, ${\mathcal N}$, is a symmetric binary relation 
${\mathcal S}_{\mathcal N}$ between agents such that $P\rel{S}_{\mathcal N}Q$ implies:
\begin{enumerate}
\item If $P \red P'$ then $Q \wred Q'$ and $P'\rel{S}_{\mathcal N} Q'$.
\item If $P\downarrow_{\mathcal N} x$, then $Q\Downarrow_{\mathcal N} x$.
\end{enumerate}
$P$ is ${\mathcal N}$-barbed bisimilar to $Q$, written
$P \wbbisim_{\mathcal N} Q$, if $P \rel{S}_{\mathcal N} Q$ for some ${\mathcal N}$-barbed bisimulation ${\mathcal S}_{\mathcal N}$.
\end{definition}

$\mathcal{R} \subseteq \pi \times \pi$

$P \mathcal{R} Q => \forall P'. P \red P' \Rightarrow \exists Q'. Q \red Q', P' \mathcal{R} Q'$

$P \vdash x \Rightarrow Q \vdash x$

\begin{mathpar}
  \inferrule*[lab=Out-barb]{x \nameeq y}{{y}!\langle{Q}\rangle \vdash x}
  \and
  \inferrule*[lab=Par-barb]{\mbox{$P\vdash x$ or $Q\vdash x$}}{\binpar{P}{Q} \vdash x}
\end{mathpar}

\subsubsection{Contexts}

One of the principle advantages of computational calculi like the
$\pi$-calculus is a well-defined notion of context,
contextual-equivalence and a correlation between
contextual-equivalence and notions of bisimulation. The notion of
context allows the decomposition of a process into (sub-)process and
its syntactic environment, its context. Thus, a context may be
thought of as a process with a ``hole'' (written $\Box$) in it. The
application of a context $M$ to a process $P$, written $M[P]$, is
tantamount to filling the hole in $M$ with $P$. In this paper we do
not need the full weight of this theory, but do make use of the notion
of context in the proof the main theorem. 

\begin{mathpar}
  \inferrule* [lab=summation] {} {{M_{M},M_{N}} \bc \Box \;|\; x.M_{A} \;|\; M_{M}+M_{N}}
  \and
  \inferrule* [lab=agent] {} {{M_{A}} \bc (\vec{x})M_{P} \;| \; \clift{P_0,\ldots,M_{P},\ldots,P_N}}
  \and \\
  \inferrule* [lab=process] {} {{M_{P}} \bc M_{N} \;| \;P|M_{P} }
\end{mathpar} 

\begin{mathpar}
  \inferrule* [lab=sychronization] {} {M_{N} \bc \Box \;|\; x?M_{F} \;|\; x!M_{C}}
  \and
  \inferrule* [lab=abstraction] {} {{M_{F}} \bc (x)M_{P} }
  \and
  \inferrule* [lab=concretion] {} {{M_{C}} \bc \langle M_{P} \rangle }
  \and \\
  \inferrule* [lab=process] {} {{M_{P}} \bc M_{N} \;| \;P|M_{P} }
\end{mathpar}

\begin{definition}[contextual application] Given a context $M$, and
  process $P$, we define the \emph{contextual application}, $M[P] :=
  M\{P/\Box\}$. That is, the contextual application of M to P is the
  substitution of $P$ for $\Box$ in $M$.
\end{definition}

$\meaningof{-} : L \to \mathcal{P}(\pi)$

\begin{mathpar}
  \inferrule* [lab=collection] {} {\meaningof{true} = \pi, \and \meaningof{~E} = \pi \setminus \meaningof{E}, \and \meaningof{E_{1} \& E_{2}} = \meaningof{E_{1}} \cap \meaningof{E_{2}}}
\end{mathpar}

\begin{mathpar}
  \inferrule* [lab=structure] {} {\meaningof{0} = \{ P \in \pi | P \equiv 0 \}, \and \\ \meaningof{E_1 | E_2} = \{ P \in \pi | P \equiv P_{1} | P_{2}, P_{1} \in \meaningof{E_{1}}, P_{2} \in \meaningof{E_2}\} }
\end{mathpar}

\begin{mathpar}
 \inferrule* [lab=behavior] {} {\meaningof{\langle a?b \rangle E} = \{ P \in \pi | P \equiv Q | u?(y)P', \\ \and \\\\ \and \\ \;\;\; u \in \meaningof{a}, \forall z.P'\{z/y\} \in \meaningof{E\{z/b\}}\}, \and \\ \meaningof{a!E} = \{ P \in \pi | P \equiv Q | x!\langle P' \rangle, x \in \meaningof{a} P' \in \meaningof{E}\} }
\end{mathpar}

\begin{mathpar}
 \inferrule* [lab=nominal] {} {\meaningof{\quotep{E}} = \{ \quotep{P} \in \quotep{\pi} | P \in \meaningof{E} \}, \and \meaningof{\quotep{P}} = \{ \quotep{Q} \in \quotep{\pi} | P \equiv Q \} \and \\ \meaningof{@\quotep{E}} = \{ P \in \pi | P \equiv @x, x \in \meaningof{E} \}}
\end{mathpar}

\begin{eqnarray*}
  \\
  \meaningof{-} : TS \to ST
\end{eqnarray*}

\begin{eqnarray*}
  \\
  L : TS \to ST
\end{eqnarray*}

\begin{eqnarray*}
  \\
  P \models E \iff P \in \meaningof{E}
\end{eqnarray*}

\begin{eqnarray*}
  P \approx_{L} Q \iff \forall E \in L. P \models E \iff Q \models E
\end{eqnarray*}

\begin{eqnarray*}
  P \approx_{K} Q
\end{eqnarray*}

\begin{eqnarray*}
  P \approx Q
\end{eqnarray*}

$\approx_{K} = \approx = \approx_{L}$

\subsubsection{Contextual duality}

Note that contexts extend the quotation operation to a family of
operations from processes to names. Given a context, $M$, we can
define a \emph{nominal context}, $\quotep{M}$ by $\quotep{M}[P] :=
\quotep{M[P]}$. To foreshadow what is to come we observe that these
operations enjoy a duality with processes very much like the duality
between vectors and maps from vectors to scalars.

Further, because the calculus is essentially higher-order, we have a
correspondence between contexts and processes. More specifically,
given a name $x$ and a context $M$ we can construct $M^{*}_{x}$ such
that 

\begin{mathpar}
  M^{*}_{x} | \lift{x}{P} \red M[P]
\end{mathpar}

namely,

\begin{mathpar}
  M^{*}_{x} := x?(u).M[\dropn{u}]
\end{mathpar}

The dependence of $M^{*}_{x}$ on a name makes it an abstraction, 

\begin{mathpar}
  M^{*} := (x)x?(u).M[\dropn{u}]
\end{mathpar}

\subsection{Additional notation}

It will sometimes be convenient to denote the process a name
quotes. We already have the notation $x = \quotep{P}$, but it will be
convenient to introduce an alternate notation, $\procn{x}$, when we
want to emphasize the connection to the use of the name. Note that, by
virtue of name equivalence, $\quotep{\procn{x}} \nameeq x$; so, the
notation is consistent with previous definitions.

Further, because names have structure it is possible to effect
substitutions on the basis of that structure. This means we need to
upgrade our notation for substitutions, which we accomplish by
adapting comprehension notation. Thus,

\begin{mathpar}
  P\{ y / x : x \in S \}
\end{mathpar}

is interpreted to mean the process derived from P by replacing (in a
capture-avoiding manner) each occurrence of $x$ in $S$ by $y$. For example,

\begin{mathpar}
  P\{ \quotep{\procn{x}|\procn{x}} / x : x \in \freenames{P} \}
\end{mathpar}

will replace each (occurrence) of a free name $x$ in $P$ by
$\quotep{\procn{x}|\procn{x}}$.

Also, we will avail ourselves of the notation $x^{L}$ and $x^{R}$ to
denote injections of a name into disjoint copies of the name
space. There are numerous ways to accomplish this. One example can be
found in \cite{MeredithR05}. This notation overloads to vectors of
names: $\vec{x}^{\pi} := (x_{i}^{\pi} \; : \; 0 \leq i < |\vec{x}| )$ where $\pi \in \{L,R\}$.

We also use $P^{\Box} := P|\Box$.

In \cite{MeredithR05} an interpretation of the new operator is
given. It turns out that there are several possible interpretations
all enjoying the requisite algebraic properties of the operator (see
\cite{milner91polyadicpi}). We will therefore make liberal use of
$(\nu\; \vec{x})P$.

% subsection the_syntax_and_semantics_of_the_notation_system (end)   

\input{qm2pi.qmops} 

\input{qm2pi.sterngerlach} 

\input{qm2pi.metric} 

% section concurrent_process_calculi (end)

%\input{qm2pi.proofsketch}

% section proof sketch (end)

%\input{qm2pi.slviaknots} 

% section spatial logic via knots (end)

\input{qm2pi.conclusion}

% section conclusion (end)

%\input{qm2pi.dtcodes} 

% section wiring algorithm (end)

\input{qm2pi.ack} 

% section acknowledgments (end)

\newpage


\bibliographystyle{plain}   
\bibliography{../../biblios/main.bib}

\input{qm2pi.rhodetails}

\end{document}

 

\documentclass[12pt]{llncs}
%\documentclass{jktr}

\usepackage[pdftex]{hyperref}                   
\usepackage {listings}
\usepackage {mathpartir}
\usepackage{bcprules}
%\usepackage{listings}
                       
\usepackage{graphicx} 
%\usepackage[margins=2.5cm,nohead,nofoot]{geometry}
%\usepackage{geometry}
\usepackage{amsfonts}
\usepackage{amstext}
\usepackage{latexsym}
\usepackage{amssymb}
\usepackage{color}


%\include{myPreamble}
\include{qm2pi.local} 

%\ifpdf
%\usepackage[pdftex]{graphicx}
%\else
%\usepackage{graphicx}
%\fi

 % \ifpdf
%  \usepackage{pdfsync}
%  \if


%\title{Brief Article}
%\author{David F. Snyder}
%\author{L.G. Meredith}

%\address{Dept. of Math., Texas State University--San Marcos, San Marcos, TX 78666}
       
\pagestyle{empty}


\begin{document}

\lstset{language=[Objective]Caml,frame=shadowbox}

\input{qm2pi.front}

% section front matter (end)

\input{qm2pi.intro} 
 
% section introduction (end)

% \input{qm2pi.knotations} 

% section notation (end)

\input{qm2pi.process.calculi} 

% section concurrent_process_calculi_and_spatial_logics_ (end)
    
%\input{qm2pi.knots2pi} 

%\input{qm2pi.trefoil} 

%\input{qm2pi.mainthm} 

% subsection basic_interpretation (end)

%\input{qm2pi.rho.presentation} 
\subsection{The syntax and semantics of the notation system}\label{sub:the_syntax_and_semantics_of_the_notation_system} % (fold)

We now summarize a technical presentation of the calculus that
embodies our theory of dynamics. The typical presentation of such a
calculus follows the style of giving generators and relations on
them. The grammar, below, describing term constructors, freely
generates the set of processes, $\Proc$. This set is then quotiented
by a relation known as structural congruence and it is over this set
that the notion of dynamics is expressed. This presentation is
essentially that of \cite{MeredithR05} with the addition of
polyadicity and summation. For readability we have relegated some of
the technical subtleties to an appendix.

\subsubsection{Process grammar}\label{subsub:process_grammar}

\begin{mathpar}
  \inferrule* [lab=synchronization] {} {{M} \bc \pzero \;|\; x?F \;|\; x!C }
  \and
  \inferrule* [lab=abstraction] {} {{F} \bc (x)P}
  \and
  \inferrule* [lab=concretion] {} {{C} \bc \langle Q \rangle}
  \and
  \inferrule* [lab=process] {} {{P,Q} \bc M \;| \;P|Q \;|\; @{x}}
  \and
  \inferrule* [lab=name] {} {{x} \bc \quotep{P}}
\end{mathpar} 

Note that $\vec{x}$ (resp. $\vec{P}$) denotes a vector of names
(resp. processes) of length $|\vec{x}|$ (resp. $|\vec{P}|$). We adopt
the following useful abbreviations.

\begin{mathpar}
   x?(\vec{y}).P := x.(\vec{y})P \and  x\clift{\vec{P}} := x.\clift{\vec{P}}
   \and x!(y) := \lift{x}{\dropn{y}}
   \and \Pi_{i=0}^{n-1}P_i := P_0 | \ldots | P_{n-1}
\end{mathpar}

\subsubsection{Structural congruence}

\paragraph{Free and bound names and alpha-equivalence.} At the
core of structural equivalence is alpha-equivalence which identifies
process that are the same up to a change of variable. Formally, we
recognize the distinction between free and bound names. The free names
of a process, $\freenames{P}$, may be calculated recursively as
follows:

\begin{mathpar}
\freenames{\pzero} := \emptyset
  \and \\
  \freenames{x?(y).P} := \{ x \} \cup (\freenames{P} \setminus \{ y \})
  \and 
  \freenames{x!\langle P \rangle} := \{ x \} \cup \{ P \} 
  \and \\
  \freenames{P|Q} := \freenames{P} \cup \freenames{Q}
  \and \\
  \freenames{@{x}} := \{ x \}
\end{mathpar}

$\pi$
$\quotep{\pi}$

$\freenames{-} : \pi \to \mathcal{P}(\quotep{\pi})$

\begin{eqnarray*}
  \freenames{\pzero} & := & \emptyset \\
  \freenames{x?(y).P} & := & \{ x \} \cup (\freenames{P} \setminus \{ y \}) \\
  \freenames{x!\langle P \rangle} & := & \{ x \} \cup \{ P \} \\
  \freenames{P|Q} & := & \freenames{P} \cup \freenames{Q} \\
  \freenames{\dropn{x}} & := & \{ x \}
\end{eqnarray*}

The bound names of a process, $\boundnames{P}$, are those names occurring in $P$
that are not free. For example, in $x?(y).0$, the name $x$ is free, while $y$ is bound.

\begin{mathpar}
  \inferrule* [lab=monoidal-laws] {} { P|Q \equiv Q|P \and P|0 \equiv P \and P|(Q|R) \equiv (P|Q)|R }
\end{mathpar}

\begin{mathpar}
  \inferrule* [lab=alpha-equivalence] {} { (x)P \equiv (y)P\{y/x\} \and y \not\in \freenames{P} }
\end{mathpar}

\begin{definition}
Then two processes, $P,Q$, are alpha-equivalent if $P = Q\{\vec{y}/\vec{x}\}$ for
some $\vec{x} \in \boundnames{Q},\vec{y} \in \boundnames{P}$, where $Q\{\vec{y}/\vec{x}\}$
denotes the capture-avoiding substitution of $\vec{y}$ for $\vec{x}$ in $Q$.
\end{definition}

\begin{definition}
  The {\em structural congruence} \cite{SangiorgiWalker} , $\equiv$,
  between processes is the least congruence containing
  alpha-equivalence, satisfying the abelian monoid laws
  (associativity, commutativity and $\pzero$ as identity) for parallel
  composition $|$ and for summation $+$.
\end{definition}

\subsection{Name equivalence}

We take name equivalence, written $\nameeq$, to be the smallest
equivalence relation generated by the following rules.

\begin{mathpar}
\inferrule*[lab=Quote-drop]
{ }
{ \quotep{@{x}} \nameeq x }

\inferrule*[lab=Struct-equiv]
{ P \scong Q }
{ \quotep{P} \nameeq \quotep{Q} }
\end{mathpar}

The astute reader will have noticed that the mutual recursion of names
and processes imposes a mutual recursion on alpha-equivalence and
structural equivalence via name-equivalence. Fortunately, all of this
works out pleasantly and we may calculate in the natural way, free of
concern. The reader interested in the details is referred to the
appendix \ref{appendix:rho_details}.

\subsection{Substitution}

We use $\Proc$ for the set of processes, $\QProc$ for the set of
names, and $\id{\{}\vec{y} / \vec{x} \id{\}}$ to denote partial maps,
$s : \QProc \rightarrow \QProc$. A map, $s$ lifts, uniquely, to a map
on process terms, $\widehat{s} : \Proc \rightarrow \Proc$ by the
following equations.

\begin{mathpar}
  (0) \psubstp{Q}{P} := 0 \\
  (R \juxtap S) \psubstp{Q}{P}
  :=    
  (R)\psubstp{Q}{P} \juxtap (S) \psubstp{Q}{P} \\
  (x?(y).R) \psubstp{Q}{P}    
  :=    
  (x)\substp{Q}{P} (z)\concat( (R \psubstn{z}{y}) \psubstp{Q}{P} ) \\
  (\lift{x}{R}) \psubstp{Q}{P}  
  :=
  \lift{(x)\substp{Q}{P}}{ R \psubstp{Q}{P} } \\
%   (\dropn{x})  \psubstp{Q}{P}       
%   := 
%   \left\{ 
%     \begin{array}{ccc} 
%       \dropn{\quotep{Q}} & & x \nameeq \quotep{P} \\
%       \dropn{x} & & otherwise \\
%     \end{array}
%   \right. 
  (\dropn{x})  \psubstp{Q}{P}       
  := 
  \left\{ 
    \begin{array}{ccc} 
      Q & & x \nameeq \quotep{P} \\
      \dropn{x} & & otherwise \\
    \end{array}
  \right.
\end{mathpar}
 

where

\begin{eqnarray}
  (x)\id{\{} \lpquote Q \rpquote / \lpquote P \rpquote \id{\}}            = 
  \left\{ 
    \begin{array}{ccc}
      \lpquote Q \rpquote & & x \nameeq \lpquote P \rpquote \\
      x & & otherwise \\
    \end{array}
  \right. \nonumber
\end{eqnarray}

and $z$ is chosen distinct from $\quotep{P}$, $\quotep{Q}$, the free
names in $Q$, and all the names in $R$. Our $\alpha$-equivalence will
be built in the standard way from this substitution.

\begin{remark}\label{rem:no_self_referential_names}
  One consequence of these definitions is that $\forall P. \quotep{P}
  \not\in \freenames{P}$.
\end{remark}

\subsection{ Dynamic quote: an example }

Anticipating something of what's to come, consider applying the
substitution, $\widehat{\id{\{}u / z \id{\}}}$, to the following pair
of processes, $\lift{w}{y!(z)}$ and $w[ \lpquote y!(z) \rpquote ]$.

\begin{eqnarray}
	\lift{w}{y!(z)}\widehat{\id{\{}u / z \id{\}}}
		& = &
		\lift{w}{y!(u)} \nonumber\\
	w[ \lpquote y!(z) \rpquote ] \widehat{ \id{\{}u / z \id{\}} }
		& = &
		w[ \lpquote y!(z) \rpquote ] \nonumber
\end{eqnarray}

Because the body of the process between quotes is impervious to
substitution, we get radically different answers. In fact, by
examining the first process in an input context,
e.g. $x?(z).\lift{w}{y!(z)}$, we see that the process under the lift
operator may be shaped by prefixed inputs binding a name inside it. In
this sense, the lift operator will be seen as a way to dynamically
construct processes before reifying them as names.

Finally equipped with these standard features we can present the
dynamics of the calculus.

\subsubsection{Operational semantics} 

Finally, we introduce the computational dynamics. What marks these
algebras as distinct from other more traditionally studied algebraic
structures, e.g. vector spaces or polynomial rings, is the manner in
which dynamics is captured. In traditional structures, dynamics is typically
expressed through morphisms between such structures, as in linear maps
between vector spaces or morphisms between rings. In algebras
associated with the semantics of computation, the dynamics is
expressed as part of the algebraic structure itself, through a
reduction reduction relation typically denoted by $\red$. Below, we
give a recursive presentation of this relation for the calculus used
in the encoding.

$\red \subseteq \pi \times \pi$
$\red : \pi \to \mathcal{P}(\pi)$

\begin{mathpar}
  \inferrule* [lab=Comm] { \textsf{match}( x_{src}, x_{trgt} ) } { x_{trgt}?(y)P \; | \; x_{src}!\langle {Q} \rangle \red P\{\quotep{Q}/y}\} }
  \and \\
  \inferrule* [lab=Par] {{P} \red {P}'} {{{P} | {Q}} \red {{P}' | {Q}}}
  \and
  \inferrule* [lab=Equiv]{{{P} \scong {P}'} \andalso {{P}' \red {Q}'} \andalso {{Q}' \scong {Q}}}{{P} \red {Q}}
\end{mathpar}

\begin{eqnarray*}
  match_{\equiv} (\quotep{P},\quotep{Q}) & := & P \equiv Q \\
  match_{\dagger}(\quotep{P},\quotep{Q}) & := & \forall R. P|Q \red^{*} R => R \red^{*} 0 \\
  match_{K}(\quotep{P},\quotep{Q}) & := & K \mbox{ for some context } K
\end{eqnarray*}

$u?(x)P | u!\langle Q \rangle \red P\{\quotep{Q}/x\}$

%We write $\wred$ for $\red^*$, and $P\red$ if $\exists Q $ such that $ P \red Q$.
We write $P\red$ if $\exists Q $ such that $ P \red Q$ and $P\not\red$, otherwise.

\section{Replication}

As mentioned before, it is known that replication (and hence
recursion) can be implemented in a higher-order process algebra
\cite{SangiorgiWalker}. As our first example of calculation with the
machinery thus far presented we give the construction explicitly in
the {\rhoc}.

\begin{eqnarray}
	D_{x} & := & \prefix{x}{y}{(\binpar{\outputp{x}{y}}{@{y}})} \nonumber\\
	\bangp_{x}{P} & := & \binpar{{x}!\langle{\binpar{D_{x}}{P}}\rangle}{D_{x}} \nonumber
\end{eqnarray}

\begin{eqnarray}
	\bangp_{x}{P} & & \nonumber\\
	=
	& {x}!\langle{(\prefix{x}{y}{(\outputp{x}{y} | @{y})) | P}}\rangle 
	      | \prefix{x}{y}{(\outputp{x}{y} | @{y})} & \nonumber\\
	\red
	& (\outputp{x}{y} | @{y})\substn{\quotep{(\prefix{x}{y}{(@{y} | \outputp{x}{y})) | P}}}{y} & \nonumber\\
	=
	& \outputp{x}{\quotep{(\prefix{x}{y}{(\outputp{x}{y} | @{y})) | P}}}
	  | {(\prefix{x}{y}{(\outputp{x}{y} | @{y})) | P}} & \nonumber\\
	\red
	& \ldots & \nonumber\\
	\red^*
	& P | P | \ldots & \nonumber
\end{eqnarray}

Of course, this encoding, as an implementation, runs away, unfolding
$\bangp{P}$ eagerly. A lazier and more implementable replication
operator, restricted to input-guarded processes, may be obtained as follows.

\begin{eqnarray}
\bangp{\prefix{u}{v}{P}} 
	:= 
	\binpar{\lift{x}{\prefix{u}{v}{(\binpar{D(x)}{P})}}}{D(x)} \nonumber
\end{eqnarray}

\begin{remark}
  Note that the lazier definition still does not deal with summation
  or mixed summation (i.e. sums over input and output). The reader is
  invited to construct definitions of replication that deal with these
  features. 

  Further, the definitions are parameterized in a name, $x$. Can you,
  gentle reader, make a definition that eliminates this parameter and
  guarantees no accidental interaction between the replication
  machinery and the process being replicated -- i.e. no accidental
  sharing of names used by the process to get its work done and the
  name(s) used by the replication to effect copying. This latter
  revision of the definition of replication is crucial to obtaining
  the expected identity $!!P \sim !P$.
\end{remark}

\begin{remark}\label{rem:paradoxical_combinator}
  The reader familiar with the lambda calculus will have noticed the
  similarity between $D$ and the paradoxical combinator.

  [Ed. note: the existence of this seems to suggest we have to be more
  restrictive on the set of processes and names we admit if we are to
  support no-cloning.]
\end{remark}

\subsubsection{Bisimulation}

The computational dynamics gives rise to another kind of equivalence,
the equivalence of computational behavior. As previously mentioned
this is typically captured \emph{via} some form of bisimulation.

% The notion we use in this paper is weak barbed bisimulation
% \cite{milner91polyadicpi}.

The notion we use in this paper is derived from weak barbed
bisimulation \cite{milner91polyadicpi}. 

\begin{definition}
An \emph{observation relation}, $\downarrow_{\mathcal N}$, over a set
of names, $\mathcal N$, is the smallest relation satisfying the rules
below.

\infrule[Out-barb]{y \in {\mathcal N}, \; x \nameeq y}
		  {\outputp{x}{v} \downarrow_{\mathcal N} x}
\infrule[Par-barb]{\mbox{$P\downarrow_{\mathcal N} x$ or $Q\downarrow_{\mathcal N} x$}}
		  {\binpar{P}{Q} \downarrow_{\mathcal N} x}

We write $P \Downarrow_{\mathcal N} x$ if there is $Q$ such that 
$P \wred Q$ and $Q \downarrow_{\mathcal N} x$.
\end{definition}

\begin{definition}
%\label{def.bbisim}
An  ${\mathcal N}$-\emph{barbed bisimulation} over a set of names, ${\mathcal N}$, is a symmetric binary relation 
${\mathcal S}_{\mathcal N}$ between agents such that $P\rel{S}_{\mathcal N}Q$ implies:
\begin{enumerate}
\item If $P \red P'$ then $Q \wred Q'$ and $P'\rel{S}_{\mathcal N} Q'$.
\item If $P\downarrow_{\mathcal N} x$, then $Q\Downarrow_{\mathcal N} x$.
\end{enumerate}
$P$ is ${\mathcal N}$-barbed bisimilar to $Q$, written
$P \wbbisim_{\mathcal N} Q$, if $P \rel{S}_{\mathcal N} Q$ for some ${\mathcal N}$-barbed bisimulation ${\mathcal S}_{\mathcal N}$.
\end{definition}

$\mathcal{R} \subseteq \pi \times \pi$

$P \mathcal{R} Q => \forall P'. P \red P' \Rightarrow \exists Q'. Q \red Q', P' \mathcal{R} Q'$

$P \vdash x \Rightarrow Q \vdash x$

\begin{mathpar}
  \inferrule*[lab=Out-barb]{x \nameeq y}{{y}!\langle{Q}\rangle \vdash x}
  \and
  \inferrule*[lab=Par-barb]{\mbox{$P\vdash x$ or $Q\vdash x$}}{\binpar{P}{Q} \vdash x}
\end{mathpar}

\subsubsection{Contexts}

One of the principle advantages of computational calculi like the
$\pi$-calculus is a well-defined notion of context,
contextual-equivalence and a correlation between
contextual-equivalence and notions of bisimulation. The notion of
context allows the decomposition of a process into (sub-)process and
its syntactic environment, its context. Thus, a context may be
thought of as a process with a ``hole'' (written $\Box$) in it. The
application of a context $M$ to a process $P$, written $M[P]$, is
tantamount to filling the hole in $M$ with $P$. In this paper we do
not need the full weight of this theory, but do make use of the notion
of context in the proof the main theorem. 

\begin{mathpar}
  \inferrule* [lab=summation] {} {{M_{M},M_{N}} \bc \Box \;|\; x.M_{A} \;|\; M_{M}+M_{N}}
  \and
  \inferrule* [lab=agent] {} {{M_{A}} \bc (\vec{x})M_{P} \;| \; \clift{P_0,\ldots,M_{P},\ldots,P_N}}
  \and \\
  \inferrule* [lab=process] {} {{M_{P}} \bc M_{N} \;| \;P|M_{P} }
\end{mathpar} 

\begin{mathpar}
  \inferrule* [lab=sychronization] {} {M_{N} \bc \Box \;|\; x?M_{F} \;|\; x!M_{C}}
  \and
  \inferrule* [lab=abstraction] {} {{M_{F}} \bc (x)M_{P} }
  \and
  \inferrule* [lab=concretion] {} {{M_{C}} \bc \langle M_{P} \rangle }
  \and \\
  \inferrule* [lab=process] {} {{M_{P}} \bc M_{N} \;| \;P|M_{P} }
\end{mathpar}

\begin{definition}[contextual application] Given a context $M$, and
  process $P$, we define the \emph{contextual application}, $M[P] :=
  M\{P/\Box\}$. That is, the contextual application of M to P is the
  substitution of $P$ for $\Box$ in $M$.
\end{definition}

$\meaningof{-} : L \to \mathcal{P}(\pi)$

\begin{mathpar}
  \inferrule* [lab=collection] {} {\meaningof{true} = \pi, \and \meaningof{~E} = \pi \setminus \meaningof{E}, \and \meaningof{E_{1} \& E_{2}} = \meaningof{E_{1}} \cap \meaningof{E_{2}}}
\end{mathpar}

\begin{mathpar}
  \inferrule* [lab=structure] {} {\meaningof{0} = \{ P \in \pi | P \equiv 0 \}, \and \\ \meaningof{E_1 | E_2} = \{ P \in \pi | P \equiv P_{1} | P_{2}, P_{1} \in \meaningof{E_{1}}, P_{2} \in \meaningof{E_2}\} }
\end{mathpar}

\begin{mathpar}
 \inferrule* [lab=behavior] {} {\meaningof{\langle a?b \rangle E} = \{ P \in \pi | P \equiv Q | u?(y)P', \\ \and \\\\ \and \\ \;\;\; u \in \meaningof{a}, \forall z.P'\{z/y\} \in \meaningof{E\{z/b\}}\}, \and \\ \meaningof{a!E} = \{ P \in \pi | P \equiv Q | x!\langle P' \rangle, x \in \meaningof{a} P' \in \meaningof{E}\} }
\end{mathpar}

\begin{mathpar}
 \inferrule* [lab=nominal] {} {\meaningof{\quotep{E}} = \{ \quotep{P} \in \quotep{\pi} | P \in \meaningof{E} \}, \and \meaningof{\quotep{P}} = \{ \quotep{Q} \in \quotep{\pi} | P \equiv Q \} \and \\ \meaningof{@\quotep{E}} = \{ P \in \pi | P \equiv @x, x \in \meaningof{E} \}}
\end{mathpar}

\begin{eqnarray*}
  \\
  \meaningof{-} : TS \to ST
\end{eqnarray*}

\begin{eqnarray*}
  \\
  L : TS \to ST
\end{eqnarray*}

\begin{eqnarray*}
  \\
  P \models E \iff P \in \meaningof{E}
\end{eqnarray*}

\begin{eqnarray*}
  P \approx_{L} Q \iff \forall E \in L. P \models E \iff Q \models E
\end{eqnarray*}

\begin{eqnarray*}
  P \approx_{K} Q
\end{eqnarray*}

\begin{eqnarray*}
  P \approx Q
\end{eqnarray*}

$\approx_{K} = \approx = \approx_{L}$

\subsubsection{Contextual duality}

Note that contexts extend the quotation operation to a family of
operations from processes to names. Given a context, $M$, we can
define a \emph{nominal context}, $\quotep{M}$ by $\quotep{M}[P] :=
\quotep{M[P]}$. To foreshadow what is to come we observe that these
operations enjoy a duality with processes very much like the duality
between vectors and maps from vectors to scalars.

Further, because the calculus is essentially higher-order, we have a
correspondence between contexts and processes. More specifically,
given a name $x$ and a context $M$ we can construct $M^{*}_{x}$ such
that 

\begin{mathpar}
  M^{*}_{x} | \lift{x}{P} \red M[P]
\end{mathpar}

namely,

\begin{mathpar}
  M^{*}_{x} := x?(u).M[\dropn{u}]
\end{mathpar}

The dependence of $M^{*}_{x}$ on a name makes it an abstraction, 

\begin{mathpar}
  M^{*} := (x)x?(u).M[\dropn{u}]
\end{mathpar}

\subsection{Additional notation}

It will sometimes be convenient to denote the process a name
quotes. We already have the notation $x = \quotep{P}$, but it will be
convenient to introduce an alternate notation, $\procn{x}$, when we
want to emphasize the connection to the use of the name. Note that, by
virtue of name equivalence, $\quotep{\procn{x}} \nameeq x$; so, the
notation is consistent with previous definitions.

Further, because names have structure it is possible to effect
substitutions on the basis of that structure. This means we need to
upgrade our notation for substitutions, which we accomplish by
adapting comprehension notation. Thus,

\begin{mathpar}
  P\{ y / x : x \in S \}
\end{mathpar}

is interpreted to mean the process derived from P by replacing (in a
capture-avoiding manner) each occurrence of $x$ in $S$ by $y$. For example,

\begin{mathpar}
  P\{ \quotep{\procn{x}|\procn{x}} / x : x \in \freenames{P} \}
\end{mathpar}

will replace each (occurrence) of a free name $x$ in $P$ by
$\quotep{\procn{x}|\procn{x}}$.

Also, we will avail ourselves of the notation $x^{L}$ and $x^{R}$ to
denote injections of a name into disjoint copies of the name
space. There are numerous ways to accomplish this. One example can be
found in \cite{MeredithR05}. This notation overloads to vectors of
names: $\vec{x}^{\pi} := (x_{i}^{\pi} \; : \; 0 \leq i < |\vec{x}| )$ where $\pi \in \{L,R\}$.

We also use $P^{\Box} := P|\Box$.

In \cite{MeredithR05} an interpretation of the new operator is
given. It turns out that there are several possible interpretations
all enjoying the requisite algebraic properties of the operator (see
\cite{milner91polyadicpi}). We will therefore make liberal use of
$(\nu\; \vec{x})P$.

% subsection the_syntax_and_semantics_of_the_notation_system (end)   

\input{qm2pi.qmops} 

\input{qm2pi.sterngerlach} 

\input{qm2pi.metric} 

% section concurrent_process_calculi (end)

%\input{qm2pi.proofsketch}

% section proof sketch (end)

%\input{qm2pi.slviaknots} 

% section spatial logic via knots (end)

\input{qm2pi.conclusion}

% section conclusion (end)

%\input{qm2pi.dtcodes} 

% section wiring algorithm (end)

\input{qm2pi.ack} 

% section acknowledgments (end)

\newpage


\bibliographystyle{plain}   
\bibliography{../../biblios/main.bib}

\input{qm2pi.rhodetails}

\end{document}

 

% section concurrent_process_calculi (end)

%\documentclass[12pt]{llncs}
%\documentclass{jktr}

\usepackage[pdftex]{hyperref}                   
\usepackage {listings}
\usepackage {mathpartir}
\usepackage{bcprules}
%\usepackage{listings}
                       
\usepackage{graphicx} 
%\usepackage[margins=2.5cm,nohead,nofoot]{geometry}
%\usepackage{geometry}
\usepackage{amsfonts}
\usepackage{amstext}
\usepackage{latexsym}
\usepackage{amssymb}
\usepackage{color}


%\include{myPreamble}
\include{qm2pi.local} 

%\ifpdf
%\usepackage[pdftex]{graphicx}
%\else
%\usepackage{graphicx}
%\fi

 % \ifpdf
%  \usepackage{pdfsync}
%  \if


%\title{Brief Article}
%\author{David F. Snyder}
%\author{L.G. Meredith}

%\address{Dept. of Math., Texas State University--San Marcos, San Marcos, TX 78666}
       
\pagestyle{empty}


\begin{document}

\lstset{language=[Objective]Caml,frame=shadowbox}

\input{qm2pi.front}

% section front matter (end)

\input{qm2pi.intro} 
 
% section introduction (end)

% \input{qm2pi.knotations} 

% section notation (end)

\input{qm2pi.process.calculi} 

% section concurrent_process_calculi_and_spatial_logics_ (end)
    
%\input{qm2pi.knots2pi} 

%\input{qm2pi.trefoil} 

%\input{qm2pi.mainthm} 

% subsection basic_interpretation (end)

%\input{qm2pi.rho.presentation} 
\subsection{The syntax and semantics of the notation system}\label{sub:the_syntax_and_semantics_of_the_notation_system} % (fold)

We now summarize a technical presentation of the calculus that
embodies our theory of dynamics. The typical presentation of such a
calculus follows the style of giving generators and relations on
them. The grammar, below, describing term constructors, freely
generates the set of processes, $\Proc$. This set is then quotiented
by a relation known as structural congruence and it is over this set
that the notion of dynamics is expressed. This presentation is
essentially that of \cite{MeredithR05} with the addition of
polyadicity and summation. For readability we have relegated some of
the technical subtleties to an appendix.

\subsubsection{Process grammar}\label{subsub:process_grammar}

\begin{mathpar}
  \inferrule* [lab=synchronization] {} {{M} \bc \pzero \;|\; x?F \;|\; x!C }
  \and
  \inferrule* [lab=abstraction] {} {{F} \bc (x)P}
  \and
  \inferrule* [lab=concretion] {} {{C} \bc \langle Q \rangle}
  \and
  \inferrule* [lab=process] {} {{P,Q} \bc M \;| \;P|Q \;|\; @{x}}
  \and
  \inferrule* [lab=name] {} {{x} \bc \quotep{P}}
\end{mathpar} 

Note that $\vec{x}$ (resp. $\vec{P}$) denotes a vector of names
(resp. processes) of length $|\vec{x}|$ (resp. $|\vec{P}|$). We adopt
the following useful abbreviations.

\begin{mathpar}
   x?(\vec{y}).P := x.(\vec{y})P \and  x\clift{\vec{P}} := x.\clift{\vec{P}}
   \and x!(y) := \lift{x}{\dropn{y}}
   \and \Pi_{i=0}^{n-1}P_i := P_0 | \ldots | P_{n-1}
\end{mathpar}

\subsubsection{Structural congruence}

\paragraph{Free and bound names and alpha-equivalence.} At the
core of structural equivalence is alpha-equivalence which identifies
process that are the same up to a change of variable. Formally, we
recognize the distinction between free and bound names. The free names
of a process, $\freenames{P}$, may be calculated recursively as
follows:

\begin{mathpar}
\freenames{\pzero} := \emptyset
  \and \\
  \freenames{x?(y).P} := \{ x \} \cup (\freenames{P} \setminus \{ y \})
  \and 
  \freenames{x!\langle P \rangle} := \{ x \} \cup \{ P \} 
  \and \\
  \freenames{P|Q} := \freenames{P} \cup \freenames{Q}
  \and \\
  \freenames{@{x}} := \{ x \}
\end{mathpar}

$\pi$
$\quotep{\pi}$

$\freenames{-} : \pi \to \mathcal{P}(\quotep{\pi})$

\begin{eqnarray*}
  \freenames{\pzero} & := & \emptyset \\
  \freenames{x?(y).P} & := & \{ x \} \cup (\freenames{P} \setminus \{ y \}) \\
  \freenames{x!\langle P \rangle} & := & \{ x \} \cup \{ P \} \\
  \freenames{P|Q} & := & \freenames{P} \cup \freenames{Q} \\
  \freenames{\dropn{x}} & := & \{ x \}
\end{eqnarray*}

The bound names of a process, $\boundnames{P}$, are those names occurring in $P$
that are not free. For example, in $x?(y).0$, the name $x$ is free, while $y$ is bound.

\begin{mathpar}
  \inferrule* [lab=monoidal-laws] {} { P|Q \equiv Q|P \and P|0 \equiv P \and P|(Q|R) \equiv (P|Q)|R }
\end{mathpar}

\begin{mathpar}
  \inferrule* [lab=alpha-equivalence] {} { (x)P \equiv (y)P\{y/x\} \and y \not\in \freenames{P} }
\end{mathpar}

\begin{definition}
Then two processes, $P,Q$, are alpha-equivalent if $P = Q\{\vec{y}/\vec{x}\}$ for
some $\vec{x} \in \boundnames{Q},\vec{y} \in \boundnames{P}$, where $Q\{\vec{y}/\vec{x}\}$
denotes the capture-avoiding substitution of $\vec{y}$ for $\vec{x}$ in $Q$.
\end{definition}

\begin{definition}
  The {\em structural congruence} \cite{SangiorgiWalker} , $\equiv$,
  between processes is the least congruence containing
  alpha-equivalence, satisfying the abelian monoid laws
  (associativity, commutativity and $\pzero$ as identity) for parallel
  composition $|$ and for summation $+$.
\end{definition}

\subsection{Name equivalence}

We take name equivalence, written $\nameeq$, to be the smallest
equivalence relation generated by the following rules.

\begin{mathpar}
\inferrule*[lab=Quote-drop]
{ }
{ \quotep{@{x}} \nameeq x }

\inferrule*[lab=Struct-equiv]
{ P \scong Q }
{ \quotep{P} \nameeq \quotep{Q} }
\end{mathpar}

The astute reader will have noticed that the mutual recursion of names
and processes imposes a mutual recursion on alpha-equivalence and
structural equivalence via name-equivalence. Fortunately, all of this
works out pleasantly and we may calculate in the natural way, free of
concern. The reader interested in the details is referred to the
appendix \ref{appendix:rho_details}.

\subsection{Substitution}

We use $\Proc$ for the set of processes, $\QProc$ for the set of
names, and $\id{\{}\vec{y} / \vec{x} \id{\}}$ to denote partial maps,
$s : \QProc \rightarrow \QProc$. A map, $s$ lifts, uniquely, to a map
on process terms, $\widehat{s} : \Proc \rightarrow \Proc$ by the
following equations.

\begin{mathpar}
  (0) \psubstp{Q}{P} := 0 \\
  (R \juxtap S) \psubstp{Q}{P}
  :=    
  (R)\psubstp{Q}{P} \juxtap (S) \psubstp{Q}{P} \\
  (x?(y).R) \psubstp{Q}{P}    
  :=    
  (x)\substp{Q}{P} (z)\concat( (R \psubstn{z}{y}) \psubstp{Q}{P} ) \\
  (\lift{x}{R}) \psubstp{Q}{P}  
  :=
  \lift{(x)\substp{Q}{P}}{ R \psubstp{Q}{P} } \\
%   (\dropn{x})  \psubstp{Q}{P}       
%   := 
%   \left\{ 
%     \begin{array}{ccc} 
%       \dropn{\quotep{Q}} & & x \nameeq \quotep{P} \\
%       \dropn{x} & & otherwise \\
%     \end{array}
%   \right. 
  (\dropn{x})  \psubstp{Q}{P}       
  := 
  \left\{ 
    \begin{array}{ccc} 
      Q & & x \nameeq \quotep{P} \\
      \dropn{x} & & otherwise \\
    \end{array}
  \right.
\end{mathpar}
 

where

\begin{eqnarray}
  (x)\id{\{} \lpquote Q \rpquote / \lpquote P \rpquote \id{\}}            = 
  \left\{ 
    \begin{array}{ccc}
      \lpquote Q \rpquote & & x \nameeq \lpquote P \rpquote \\
      x & & otherwise \\
    \end{array}
  \right. \nonumber
\end{eqnarray}

and $z$ is chosen distinct from $\quotep{P}$, $\quotep{Q}$, the free
names in $Q$, and all the names in $R$. Our $\alpha$-equivalence will
be built in the standard way from this substitution.

\begin{remark}\label{rem:no_self_referential_names}
  One consequence of these definitions is that $\forall P. \quotep{P}
  \not\in \freenames{P}$.
\end{remark}

\subsection{ Dynamic quote: an example }

Anticipating something of what's to come, consider applying the
substitution, $\widehat{\id{\{}u / z \id{\}}}$, to the following pair
of processes, $\lift{w}{y!(z)}$ and $w[ \lpquote y!(z) \rpquote ]$.

\begin{eqnarray}
	\lift{w}{y!(z)}\widehat{\id{\{}u / z \id{\}}}
		& = &
		\lift{w}{y!(u)} \nonumber\\
	w[ \lpquote y!(z) \rpquote ] \widehat{ \id{\{}u / z \id{\}} }
		& = &
		w[ \lpquote y!(z) \rpquote ] \nonumber
\end{eqnarray}

Because the body of the process between quotes is impervious to
substitution, we get radically different answers. In fact, by
examining the first process in an input context,
e.g. $x?(z).\lift{w}{y!(z)}$, we see that the process under the lift
operator may be shaped by prefixed inputs binding a name inside it. In
this sense, the lift operator will be seen as a way to dynamically
construct processes before reifying them as names.

Finally equipped with these standard features we can present the
dynamics of the calculus.

\subsubsection{Operational semantics} 

Finally, we introduce the computational dynamics. What marks these
algebras as distinct from other more traditionally studied algebraic
structures, e.g. vector spaces or polynomial rings, is the manner in
which dynamics is captured. In traditional structures, dynamics is typically
expressed through morphisms between such structures, as in linear maps
between vector spaces or morphisms between rings. In algebras
associated with the semantics of computation, the dynamics is
expressed as part of the algebraic structure itself, through a
reduction reduction relation typically denoted by $\red$. Below, we
give a recursive presentation of this relation for the calculus used
in the encoding.

$\red \subseteq \pi \times \pi$
$\red : \pi \to \mathcal{P}(\pi)$

\begin{mathpar}
  \inferrule* [lab=Comm] { \textsf{match}( x_{src}, x_{trgt} ) } { x_{trgt}?(y)P \; | \; x_{src}!\langle {Q} \rangle \red P\{\quotep{Q}/y}\} }
  \and \\
  \inferrule* [lab=Par] {{P} \red {P}'} {{{P} | {Q}} \red {{P}' | {Q}}}
  \and
  \inferrule* [lab=Equiv]{{{P} \scong {P}'} \andalso {{P}' \red {Q}'} \andalso {{Q}' \scong {Q}}}{{P} \red {Q}}
\end{mathpar}

\begin{eqnarray*}
  match_{\equiv} (\quotep{P},\quotep{Q}) & := & P \equiv Q \\
  match_{\dagger}(\quotep{P},\quotep{Q}) & := & \forall R. P|Q \red^{*} R => R \red^{*} 0 \\
  match_{K}(\quotep{P},\quotep{Q}) & := & K \mbox{ for some context } K
\end{eqnarray*}

$u?(x)P | u!\langle Q \rangle \red P\{\quotep{Q}/x\}$

%We write $\wred$ for $\red^*$, and $P\red$ if $\exists Q $ such that $ P \red Q$.
We write $P\red$ if $\exists Q $ such that $ P \red Q$ and $P\not\red$, otherwise.

\section{Replication}

As mentioned before, it is known that replication (and hence
recursion) can be implemented in a higher-order process algebra
\cite{SangiorgiWalker}. As our first example of calculation with the
machinery thus far presented we give the construction explicitly in
the {\rhoc}.

\begin{eqnarray}
	D_{x} & := & \prefix{x}{y}{(\binpar{\outputp{x}{y}}{@{y}})} \nonumber\\
	\bangp_{x}{P} & := & \binpar{{x}!\langle{\binpar{D_{x}}{P}}\rangle}{D_{x}} \nonumber
\end{eqnarray}

\begin{eqnarray}
	\bangp_{x}{P} & & \nonumber\\
	=
	& {x}!\langle{(\prefix{x}{y}{(\outputp{x}{y} | @{y})) | P}}\rangle 
	      | \prefix{x}{y}{(\outputp{x}{y} | @{y})} & \nonumber\\
	\red
	& (\outputp{x}{y} | @{y})\substn{\quotep{(\prefix{x}{y}{(@{y} | \outputp{x}{y})) | P}}}{y} & \nonumber\\
	=
	& \outputp{x}{\quotep{(\prefix{x}{y}{(\outputp{x}{y} | @{y})) | P}}}
	  | {(\prefix{x}{y}{(\outputp{x}{y} | @{y})) | P}} & \nonumber\\
	\red
	& \ldots & \nonumber\\
	\red^*
	& P | P | \ldots & \nonumber
\end{eqnarray}

Of course, this encoding, as an implementation, runs away, unfolding
$\bangp{P}$ eagerly. A lazier and more implementable replication
operator, restricted to input-guarded processes, may be obtained as follows.

\begin{eqnarray}
\bangp{\prefix{u}{v}{P}} 
	:= 
	\binpar{\lift{x}{\prefix{u}{v}{(\binpar{D(x)}{P})}}}{D(x)} \nonumber
\end{eqnarray}

\begin{remark}
  Note that the lazier definition still does not deal with summation
  or mixed summation (i.e. sums over input and output). The reader is
  invited to construct definitions of replication that deal with these
  features. 

  Further, the definitions are parameterized in a name, $x$. Can you,
  gentle reader, make a definition that eliminates this parameter and
  guarantees no accidental interaction between the replication
  machinery and the process being replicated -- i.e. no accidental
  sharing of names used by the process to get its work done and the
  name(s) used by the replication to effect copying. This latter
  revision of the definition of replication is crucial to obtaining
  the expected identity $!!P \sim !P$.
\end{remark}

\begin{remark}\label{rem:paradoxical_combinator}
  The reader familiar with the lambda calculus will have noticed the
  similarity between $D$ and the paradoxical combinator.

  [Ed. note: the existence of this seems to suggest we have to be more
  restrictive on the set of processes and names we admit if we are to
  support no-cloning.]
\end{remark}

\subsubsection{Bisimulation}

The computational dynamics gives rise to another kind of equivalence,
the equivalence of computational behavior. As previously mentioned
this is typically captured \emph{via} some form of bisimulation.

% The notion we use in this paper is weak barbed bisimulation
% \cite{milner91polyadicpi}.

The notion we use in this paper is derived from weak barbed
bisimulation \cite{milner91polyadicpi}. 

\begin{definition}
An \emph{observation relation}, $\downarrow_{\mathcal N}$, over a set
of names, $\mathcal N$, is the smallest relation satisfying the rules
below.

\infrule[Out-barb]{y \in {\mathcal N}, \; x \nameeq y}
		  {\outputp{x}{v} \downarrow_{\mathcal N} x}
\infrule[Par-barb]{\mbox{$P\downarrow_{\mathcal N} x$ or $Q\downarrow_{\mathcal N} x$}}
		  {\binpar{P}{Q} \downarrow_{\mathcal N} x}

We write $P \Downarrow_{\mathcal N} x$ if there is $Q$ such that 
$P \wred Q$ and $Q \downarrow_{\mathcal N} x$.
\end{definition}

\begin{definition}
%\label{def.bbisim}
An  ${\mathcal N}$-\emph{barbed bisimulation} over a set of names, ${\mathcal N}$, is a symmetric binary relation 
${\mathcal S}_{\mathcal N}$ between agents such that $P\rel{S}_{\mathcal N}Q$ implies:
\begin{enumerate}
\item If $P \red P'$ then $Q \wred Q'$ and $P'\rel{S}_{\mathcal N} Q'$.
\item If $P\downarrow_{\mathcal N} x$, then $Q\Downarrow_{\mathcal N} x$.
\end{enumerate}
$P$ is ${\mathcal N}$-barbed bisimilar to $Q$, written
$P \wbbisim_{\mathcal N} Q$, if $P \rel{S}_{\mathcal N} Q$ for some ${\mathcal N}$-barbed bisimulation ${\mathcal S}_{\mathcal N}$.
\end{definition}

$\mathcal{R} \subseteq \pi \times \pi$

$P \mathcal{R} Q => \forall P'. P \red P' \Rightarrow \exists Q'. Q \red Q', P' \mathcal{R} Q'$

$P \vdash x \Rightarrow Q \vdash x$

\begin{mathpar}
  \inferrule*[lab=Out-barb]{x \nameeq y}{{y}!\langle{Q}\rangle \vdash x}
  \and
  \inferrule*[lab=Par-barb]{\mbox{$P\vdash x$ or $Q\vdash x$}}{\binpar{P}{Q} \vdash x}
\end{mathpar}

\subsubsection{Contexts}

One of the principle advantages of computational calculi like the
$\pi$-calculus is a well-defined notion of context,
contextual-equivalence and a correlation between
contextual-equivalence and notions of bisimulation. The notion of
context allows the decomposition of a process into (sub-)process and
its syntactic environment, its context. Thus, a context may be
thought of as a process with a ``hole'' (written $\Box$) in it. The
application of a context $M$ to a process $P$, written $M[P]$, is
tantamount to filling the hole in $M$ with $P$. In this paper we do
not need the full weight of this theory, but do make use of the notion
of context in the proof the main theorem. 

\begin{mathpar}
  \inferrule* [lab=summation] {} {{M_{M},M_{N}} \bc \Box \;|\; x.M_{A} \;|\; M_{M}+M_{N}}
  \and
  \inferrule* [lab=agent] {} {{M_{A}} \bc (\vec{x})M_{P} \;| \; \clift{P_0,\ldots,M_{P},\ldots,P_N}}
  \and \\
  \inferrule* [lab=process] {} {{M_{P}} \bc M_{N} \;| \;P|M_{P} }
\end{mathpar} 

\begin{mathpar}
  \inferrule* [lab=sychronization] {} {M_{N} \bc \Box \;|\; x?M_{F} \;|\; x!M_{C}}
  \and
  \inferrule* [lab=abstraction] {} {{M_{F}} \bc (x)M_{P} }
  \and
  \inferrule* [lab=concretion] {} {{M_{C}} \bc \langle M_{P} \rangle }
  \and \\
  \inferrule* [lab=process] {} {{M_{P}} \bc M_{N} \;| \;P|M_{P} }
\end{mathpar}

\begin{definition}[contextual application] Given a context $M$, and
  process $P$, we define the \emph{contextual application}, $M[P] :=
  M\{P/\Box\}$. That is, the contextual application of M to P is the
  substitution of $P$ for $\Box$ in $M$.
\end{definition}

$\meaningof{-} : L \to \mathcal{P}(\pi)$

\begin{mathpar}
  \inferrule* [lab=collection] {} {\meaningof{true} = \pi, \and \meaningof{~E} = \pi \setminus \meaningof{E}, \and \meaningof{E_{1} \& E_{2}} = \meaningof{E_{1}} \cap \meaningof{E_{2}}}
\end{mathpar}

\begin{mathpar}
  \inferrule* [lab=structure] {} {\meaningof{0} = \{ P \in \pi | P \equiv 0 \}, \and \\ \meaningof{E_1 | E_2} = \{ P \in \pi | P \equiv P_{1} | P_{2}, P_{1} \in \meaningof{E_{1}}, P_{2} \in \meaningof{E_2}\} }
\end{mathpar}

\begin{mathpar}
 \inferrule* [lab=behavior] {} {\meaningof{\langle a?b \rangle E} = \{ P \in \pi | P \equiv Q | u?(y)P', \\ \and \\\\ \and \\ \;\;\; u \in \meaningof{a}, \forall z.P'\{z/y\} \in \meaningof{E\{z/b\}}\}, \and \\ \meaningof{a!E} = \{ P \in \pi | P \equiv Q | x!\langle P' \rangle, x \in \meaningof{a} P' \in \meaningof{E}\} }
\end{mathpar}

\begin{mathpar}
 \inferrule* [lab=nominal] {} {\meaningof{\quotep{E}} = \{ \quotep{P} \in \quotep{\pi} | P \in \meaningof{E} \}, \and \meaningof{\quotep{P}} = \{ \quotep{Q} \in \quotep{\pi} | P \equiv Q \} \and \\ \meaningof{@\quotep{E}} = \{ P \in \pi | P \equiv @x, x \in \meaningof{E} \}}
\end{mathpar}

\begin{eqnarray*}
  \\
  \meaningof{-} : TS \to ST
\end{eqnarray*}

\begin{eqnarray*}
  \\
  L : TS \to ST
\end{eqnarray*}

\begin{eqnarray*}
  \\
  P \models E \iff P \in \meaningof{E}
\end{eqnarray*}

\begin{eqnarray*}
  P \approx_{L} Q \iff \forall E \in L. P \models E \iff Q \models E
\end{eqnarray*}

\begin{eqnarray*}
  P \approx_{K} Q
\end{eqnarray*}

\begin{eqnarray*}
  P \approx Q
\end{eqnarray*}

$\approx_{K} = \approx = \approx_{L}$

\subsubsection{Contextual duality}

Note that contexts extend the quotation operation to a family of
operations from processes to names. Given a context, $M$, we can
define a \emph{nominal context}, $\quotep{M}$ by $\quotep{M}[P] :=
\quotep{M[P]}$. To foreshadow what is to come we observe that these
operations enjoy a duality with processes very much like the duality
between vectors and maps from vectors to scalars.

Further, because the calculus is essentially higher-order, we have a
correspondence between contexts and processes. More specifically,
given a name $x$ and a context $M$ we can construct $M^{*}_{x}$ such
that 

\begin{mathpar}
  M^{*}_{x} | \lift{x}{P} \red M[P]
\end{mathpar}

namely,

\begin{mathpar}
  M^{*}_{x} := x?(u).M[\dropn{u}]
\end{mathpar}

The dependence of $M^{*}_{x}$ on a name makes it an abstraction, 

\begin{mathpar}
  M^{*} := (x)x?(u).M[\dropn{u}]
\end{mathpar}

\subsection{Additional notation}

It will sometimes be convenient to denote the process a name
quotes. We already have the notation $x = \quotep{P}$, but it will be
convenient to introduce an alternate notation, $\procn{x}$, when we
want to emphasize the connection to the use of the name. Note that, by
virtue of name equivalence, $\quotep{\procn{x}} \nameeq x$; so, the
notation is consistent with previous definitions.

Further, because names have structure it is possible to effect
substitutions on the basis of that structure. This means we need to
upgrade our notation for substitutions, which we accomplish by
adapting comprehension notation. Thus,

\begin{mathpar}
  P\{ y / x : x \in S \}
\end{mathpar}

is interpreted to mean the process derived from P by replacing (in a
capture-avoiding manner) each occurrence of $x$ in $S$ by $y$. For example,

\begin{mathpar}
  P\{ \quotep{\procn{x}|\procn{x}} / x : x \in \freenames{P} \}
\end{mathpar}

will replace each (occurrence) of a free name $x$ in $P$ by
$\quotep{\procn{x}|\procn{x}}$.

Also, we will avail ourselves of the notation $x^{L}$ and $x^{R}$ to
denote injections of a name into disjoint copies of the name
space. There are numerous ways to accomplish this. One example can be
found in \cite{MeredithR05}. This notation overloads to vectors of
names: $\vec{x}^{\pi} := (x_{i}^{\pi} \; : \; 0 \leq i < |\vec{x}| )$ where $\pi \in \{L,R\}$.

We also use $P^{\Box} := P|\Box$.

In \cite{MeredithR05} an interpretation of the new operator is
given. It turns out that there are several possible interpretations
all enjoying the requisite algebraic properties of the operator (see
\cite{milner91polyadicpi}). We will therefore make liberal use of
$(\nu\; \vec{x})P$.

% subsection the_syntax_and_semantics_of_the_notation_system (end)   

\input{qm2pi.qmops} 

\input{qm2pi.sterngerlach} 

\input{qm2pi.metric} 

% section concurrent_process_calculi (end)

%\input{qm2pi.proofsketch}

% section proof sketch (end)

%\input{qm2pi.slviaknots} 

% section spatial logic via knots (end)

\input{qm2pi.conclusion}

% section conclusion (end)

%\input{qm2pi.dtcodes} 

% section wiring algorithm (end)

\input{qm2pi.ack} 

% section acknowledgments (end)

\newpage


\bibliographystyle{plain}   
\bibliography{../../biblios/main.bib}

\input{qm2pi.rhodetails}

\end{document}



% section proof sketch (end)

%\section{Unlikely characters: spatial logic for
  knots}\label{sub:characteristic_formulae} % (fold)

Associated to the mobile process calculi are a family of logics known
as the Hennessy-Milner logics. These logics typically enjoy a
semantics interpreting formulae as sets of processes that when
factored through the encoding outlined above allows an identification
of classes of knots with logical formulae. In the context of this
encoding the sub-family known as the spatial logics \cite{CairesC03}
\cite{CairesC04} \cite{Caires04} are of particular interest providing
several important features for expressing and reasoning about
properties (i.e. classes) of knots. We hint here at how this may be done.

%\begin{description}
%\item [structural connectives] 
\subsubsection{Structural connectives} The spatial logics enjoy
structural connectives corresponding, at the logical level, to the
parallel composition ($P | Q$) and new name ($(\nu \; x)P$)
connectives for processes. As illustrated in the examples below, these
connectives are extremely expressive given the shape of our encoding.
%\item [decideable satisfaction]

\subsubsection{Decideable satisfaction}
In \cite{Caires04} the satisfaction relation is shown to be decideable
for a rich class of processes. It further turns out that the image of
the our encoding is a proper subset of that class. This result
provides the basis for an algorithm by which to search for knots
enjoying a given property.
%\item [characteristic formulae]

\subsubsection{Characteristic formulae}
In the same paper \cite{Caires04} , Caires presents a means of calculating
characteristic formulae, selecting equivalence classes of processes
up to a pre--specified depth limit on the support set of names. Composed with our
encoding, this characteristic formula can be used to select
characteristic formulae for knots.
%\end{description}

\subsubsection{Spatial logic formulae}

The grammar below (segmented for comprehension) summarizes the syntax
of spatial logic formulae. We employ illustrative examples in the
sequel to provide an intuitive understanding of their meaning
referring the reader to \cite{Caires04} for a more detailed explication
of the semantics.

\begin{mathpar}
  \inferrule* [lab=boolean] {} {{A,B} \bc T \;|\; \neg A \;|\; A \wedge B \;|\; \eta = \eta'}
  \and
  \inferrule* [lab=spatial] {} {|\; \pzero \;|\; A | B \;|\; x \text{\textregistered} A \;|\; \forall x . A \;|\;  H x . A}
  \and
  \inferrule* [lab=behavioral] {} {|\; \alpha . A}
  \and 
  \inferrule* [lab=recursion] {} {|\; X(\vec{u}) \;|\; \mu X(\vec{u}) . A}
  \and
  \inferrule* [lab=action] {} {\alpha \bc \langle x?(\vec{y}) \rangle \;|\; \langle x!(\vec{y}) \rangle \;|\; \langle \tau \rangle}
  \and 
  \inferrule* [lab=name] {} {\eta \bc x \;|\; \tau}
\end{mathpar} 

% subsection characteristic_formulae (end)   	 

\subsection{Example formulae}\label{sub:example_formulae_} % (fold)

\subsubsection{Crossing as formula.}
% 
% \begin{align*}
%   \frac{d}{dx} \sin x &= \cos x 
%   & \frac{d}{dx} e^x &= e^x \\
%   \frac{d}{dx} \cos x &= - \sin x 
%   & \frac{d}{dx} \log x &= \frac{1}{x} \\
% \end{align*} 

\begin{align*}
 \mu C(x_{0},x_{1},y_{0},y_{1},u).&(\langle x_{0}?(z) \rangle(\langle u! \rangle\langle y_{1}!z \rangle C(x_{0},x_{1},y_{0},y_{1},u)) & \\
  & \wedge \langle y_{1}?(z) \rangle (\langle u! \rangle \langle x_{0}!z \rangle C(x_{0},x_{1},y_{0},y_{1},u)) & \\
  & \wedge \langle x_{1}?(z) \rangle (\langle u? \rangle \langle y_{0}!z \rangle C(x_{0},x_{1},y_{0},y_{1},u)) & \\
  & \wedge \langle y_{0}?(z) \rangle (\langle u? \rangle \langle x_{1}!z \rangle C(x_{0},x_{1},y_{0},y_{1},u))) &
\end{align*}

The lexicographical similarity between the shape of this formulae and
the shape of definition of the process representing a crossing reveals
the intuitive meaning of this formulae. It describes the capabilities
of a process that has the right to represent a crossing. For example
it picks out processes that may perform an input on the port $x_0$ in
its initial menu of capabilities. What differentiates the formula
from the process, however, is that the crossing process is the
smallest candidate to satisfy the formula. Infinitely many other
processes -- with internal behavior hidden behind this interface, so
to speak -- also satisfy this formula. Even this simple formula,
then, can be seen to open a new view onto knots, providing a
computational interpretation of \emph{virtual} knots.

Note that this formula is derived by hand. A similar formula can be
derived by employing Caires' calculation of characteristic formula
\cite{Caires04} to the process representing a crossing. In light of
this discussion, we let
$\meaningof{C}_{\phi}(x0,x1,y0,y1,u)$ denote a formula specifying the
dynamics we wish to capture of a crossing. To guarantee we preserve
the shape of the interface and minimal semantics we demand that
$\meaningof{C}_{\phi}(x0,x1,y0,y1,u) \Rightarrow
\textbf{C}(x0,x1,y0,y1,u)$ where $\textbf{C}(x0,x1,y0,y1,u)$ denotes
the formula above.
                            
\subsubsection{Crossing number constraints.}
The moral content of the context lemma (Lemma \ref{context}) is that the notion of
``locality'' in the Reidemeister moves is effectively captured by the
parallel composition operator of the process calculus. This intuition
extends through the logic. Given a formula,
$\meaningof{C}_{\phi}(x0,x1,y0,y1,u)$, we can use the structural
connectives to specify constraints on crossing numbers, such as at
least $n$ crossings, or exactly $n$ crossings.
\begin{mathpar}
  \inferrule* [lab=at-least-n] {} { K^{\geq n}_{\phi}(\vec{xs},\vec{ys}) := \Pi_{i=0}^{n-1} Hu . \meaningof{C}_{\phi}(xs_i,ys_i,u) | T }
  \and 
  \inferrule* [lab=exactly-n] {} { K^{= n}_{\phi}(\vec{xs},\vec{ys}) := \Pi_{i=0}^{n-1} Hu . \meaningof{C}_{\phi}(xs_i,ys_i,u) | \neg (\forall x_0,y_0,x_1,y_1,u . \meaningof{C}_{\phi}(x_0,y_0,x_1,y_1,u) | T) }
\end{mathpar}

To round out this section, recall that the encoding of an $n$-crossing
knot decomposes into a parallel composition of $n$ \emph{copies} of a
crossing process together with a wiring harness. To specify different
knot classes with the same crossing number amounts to specifying
logical constraints on the wiring harness. In the interest of space,
we defer examples to a forthcoming paper. Suffice it to say that both
the conditions ``alternating knot'' and ``contains the tangle
corresponding to 5/3'' are expressible. For example, it is possible to
calculate the characteristic formula of a process corresponding to the
tangle 5/3 and conjoin it into the classifying formula via the
composition connective of the logic.

Finally, we wish to observe that it is entirely within reason to
contemplate a more domain-specific version of spatial logic tailored
to the shape of processes in the image of the encoding. Such a
domain-specific logic would have a better claim to the title formal
language of knot properties.

% subsection example_formulae_ (end)

% section knots_as_processes (end) 

% section spatial logic via knots (end)

\section{Conclusions and future work}

\paragraph{Testing physical space}
You, gentle reader, may wonder why of all the theorems to be proved
given this set up we pick the one above. In some sense it's hardly
central to quantum mechanics. We see it as central in the sense that
it firmly establishes a notion of physical space arising from a notion
of the equivalence of behavior. Relating bisimulation to a metric is a
big step forward, but one is faced with interpreting the relationship
of that metric space to something more physical. Quantum mechanical
notions of ``physical'' space are still far from intuitive, but by
relating this idea of distance as testing to calculations that predict
physical circumstances we are making a not insignificant step forward
toward an understanding of the physical space we inhabit as
essentially dynamic.

\paragraph{Effectivity and simulation}
One of the observations we have yet to make is that the entire program
spelled out here is effective. We have built various interpreters for
the reflective calculus at work in this interpretation. In principle,
then, we can simulate quantum mechanics on a computer. The place where
the simulation may lose fidelity is the infinitely branching summation
for the annihilator.

In this connection i also want to point out that the evaluation style
calculation of the inner product puts the non-determinism of the
summation right at the heart of measurement. This suggests that
Milner's original reduction-based formulation of the dynamics of his
calculi in terms of sums was not just notationally suggestive of a
notion of measure-and-continue but captured some significant part of
the physics.

\paragraph{Quantum continuations}
In light of this last observation i want to point out that the
predominant account of quantum mechanics is missing a key aspect of a
truly compositional story of the physical situation. In a real lab,
when a measurement is made the observation can be made to feed into
another device that then makes another measurement conditioned on the
results of the first. This means that after the superposition was
collapsed the entire experimental set up remained in
superposition. While QM offers a means of writing this down it doesn't
quite line up well with the well-trodden formulation of computation
and continuation that we see so succinctly expressed in Milner's
calculi. This suggests that there might be advantages to this account
of dynamics waiting to be explored.

\paragraph{Quantum logic}
In this connection, we also note that by virtue of having the
Hennessy-Milner construction, we can pull the construction through the
interpretation of QM. This gives us a natural candidate for a quantum
logic that enjoys an extremely tight connection with it's domain of
interpretation, making the construction much less ad hoc (rather it is
the image of functor!).

\paragraph{Quantum probabiity}
i have questions about the basis of the interpretation of inner
product as probability amplitude. In particular, using which
axiomatization of probability theory does the notion of probability
amplitude earn the right to be so dubbed? In other words, where is the
proof that the operation for calculating a probability amplitude (and
then squaring) satisfies the axioms of what it means to calculate a
probability? Even if such a proof exists (i have yet to find it in the
literature), i wonder if it might not be possible to turn things on
their heads. Can we view the calculation of the probability amplitude
as an axiomatization of probability? If so, then the definition we
give for calculating probability amplitude may provide the basis for
an \emph{effective} theory of probability.

\paragraph{Quantum vs ``biological'' information}
Finally, i want to conclude with a more philosophical observation. At
a recent workshop in which QM was a predominant topic i noticed
something about quantum information. The speaker was giving a riveting
discussion of axiomatic QM and showing how properties of ``no
cloning'' and ``no deleting'' emerged as consequences of the
axiomatization. Theorems of this form are necessary to give us a sense
of confidence that our axioms characterize the physical theory. What
struck me, though, was that if quantum information is neither erasable
nor replicable it is markedly different from \emph{life}. Two of the
things we know about life is that

\begin{itemize}
  \item it ends;
  \item to gain some measure of persistence, to transcend it's
    finitude it is imminently copyable.
\end{itemize}

Both of these qualities are summarized succinctly in the aphorism: all
flesh is grass. For me these two kinds of ``information'' -- call them
quantum and biological -- are end points on a spectrum of strategies
for persistence. At one end, we have those curious entities that enjoy
uniqueness and permanence; at the other, we have those who in the face
of a certain end and an uncertain present make a go of passing
something on. To me one of the more remarkable aspects of the latter
strategy is that in the presence of noise (and certain features of
copying) we get a kind of dynamism, a chance for improvement against a
given persistent condition.

% subsection other_calculi_other_bisimulations_and_geometry_as_behavior (end)




% section conclusion (end)

%\documentclass[12pt]{llncs}
%\documentclass{jktr}

\usepackage[pdftex]{hyperref}                   
\usepackage {listings}
\usepackage {mathpartir}
\usepackage{bcprules}
%\usepackage{listings}
                       
\usepackage{graphicx} 
%\usepackage[margins=2.5cm,nohead,nofoot]{geometry}
%\usepackage{geometry}
\usepackage{amsfonts}
\usepackage{amstext}
\usepackage{latexsym}
\usepackage{amssymb}
\usepackage{color}


%\include{myPreamble}
\include{qm2pi.local} 

%\ifpdf
%\usepackage[pdftex]{graphicx}
%\else
%\usepackage{graphicx}
%\fi

 % \ifpdf
%  \usepackage{pdfsync}
%  \if


%\title{Brief Article}
%\author{David F. Snyder}
%\author{L.G. Meredith}

%\address{Dept. of Math., Texas State University--San Marcos, San Marcos, TX 78666}
       
\pagestyle{empty}


\begin{document}

\lstset{language=[Objective]Caml,frame=shadowbox}

\input{qm2pi.front}

% section front matter (end)

\input{qm2pi.intro} 
 
% section introduction (end)

% \input{qm2pi.knotations} 

% section notation (end)

\input{qm2pi.process.calculi} 

% section concurrent_process_calculi_and_spatial_logics_ (end)
    
%\input{qm2pi.knots2pi} 

%\input{qm2pi.trefoil} 

%\input{qm2pi.mainthm} 

% subsection basic_interpretation (end)

%\input{qm2pi.rho.presentation} 
\subsection{The syntax and semantics of the notation system}\label{sub:the_syntax_and_semantics_of_the_notation_system} % (fold)

We now summarize a technical presentation of the calculus that
embodies our theory of dynamics. The typical presentation of such a
calculus follows the style of giving generators and relations on
them. The grammar, below, describing term constructors, freely
generates the set of processes, $\Proc$. This set is then quotiented
by a relation known as structural congruence and it is over this set
that the notion of dynamics is expressed. This presentation is
essentially that of \cite{MeredithR05} with the addition of
polyadicity and summation. For readability we have relegated some of
the technical subtleties to an appendix.

\subsubsection{Process grammar}\label{subsub:process_grammar}

\begin{mathpar}
  \inferrule* [lab=synchronization] {} {{M} \bc \pzero \;|\; x?F \;|\; x!C }
  \and
  \inferrule* [lab=abstraction] {} {{F} \bc (x)P}
  \and
  \inferrule* [lab=concretion] {} {{C} \bc \langle Q \rangle}
  \and
  \inferrule* [lab=process] {} {{P,Q} \bc M \;| \;P|Q \;|\; @{x}}
  \and
  \inferrule* [lab=name] {} {{x} \bc \quotep{P}}
\end{mathpar} 

Note that $\vec{x}$ (resp. $\vec{P}$) denotes a vector of names
(resp. processes) of length $|\vec{x}|$ (resp. $|\vec{P}|$). We adopt
the following useful abbreviations.

\begin{mathpar}
   x?(\vec{y}).P := x.(\vec{y})P \and  x\clift{\vec{P}} := x.\clift{\vec{P}}
   \and x!(y) := \lift{x}{\dropn{y}}
   \and \Pi_{i=0}^{n-1}P_i := P_0 | \ldots | P_{n-1}
\end{mathpar}

\subsubsection{Structural congruence}

\paragraph{Free and bound names and alpha-equivalence.} At the
core of structural equivalence is alpha-equivalence which identifies
process that are the same up to a change of variable. Formally, we
recognize the distinction between free and bound names. The free names
of a process, $\freenames{P}$, may be calculated recursively as
follows:

\begin{mathpar}
\freenames{\pzero} := \emptyset
  \and \\
  \freenames{x?(y).P} := \{ x \} \cup (\freenames{P} \setminus \{ y \})
  \and 
  \freenames{x!\langle P \rangle} := \{ x \} \cup \{ P \} 
  \and \\
  \freenames{P|Q} := \freenames{P} \cup \freenames{Q}
  \and \\
  \freenames{@{x}} := \{ x \}
\end{mathpar}

$\pi$
$\quotep{\pi}$

$\freenames{-} : \pi \to \mathcal{P}(\quotep{\pi})$

\begin{eqnarray*}
  \freenames{\pzero} & := & \emptyset \\
  \freenames{x?(y).P} & := & \{ x \} \cup (\freenames{P} \setminus \{ y \}) \\
  \freenames{x!\langle P \rangle} & := & \{ x \} \cup \{ P \} \\
  \freenames{P|Q} & := & \freenames{P} \cup \freenames{Q} \\
  \freenames{\dropn{x}} & := & \{ x \}
\end{eqnarray*}

The bound names of a process, $\boundnames{P}$, are those names occurring in $P$
that are not free. For example, in $x?(y).0$, the name $x$ is free, while $y$ is bound.

\begin{mathpar}
  \inferrule* [lab=monoidal-laws] {} { P|Q \equiv Q|P \and P|0 \equiv P \and P|(Q|R) \equiv (P|Q)|R }
\end{mathpar}

\begin{mathpar}
  \inferrule* [lab=alpha-equivalence] {} { (x)P \equiv (y)P\{y/x\} \and y \not\in \freenames{P} }
\end{mathpar}

\begin{definition}
Then two processes, $P,Q$, are alpha-equivalent if $P = Q\{\vec{y}/\vec{x}\}$ for
some $\vec{x} \in \boundnames{Q},\vec{y} \in \boundnames{P}$, where $Q\{\vec{y}/\vec{x}\}$
denotes the capture-avoiding substitution of $\vec{y}$ for $\vec{x}$ in $Q$.
\end{definition}

\begin{definition}
  The {\em structural congruence} \cite{SangiorgiWalker} , $\equiv$,
  between processes is the least congruence containing
  alpha-equivalence, satisfying the abelian monoid laws
  (associativity, commutativity and $\pzero$ as identity) for parallel
  composition $|$ and for summation $+$.
\end{definition}

\subsection{Name equivalence}

We take name equivalence, written $\nameeq$, to be the smallest
equivalence relation generated by the following rules.

\begin{mathpar}
\inferrule*[lab=Quote-drop]
{ }
{ \quotep{@{x}} \nameeq x }

\inferrule*[lab=Struct-equiv]
{ P \scong Q }
{ \quotep{P} \nameeq \quotep{Q} }
\end{mathpar}

The astute reader will have noticed that the mutual recursion of names
and processes imposes a mutual recursion on alpha-equivalence and
structural equivalence via name-equivalence. Fortunately, all of this
works out pleasantly and we may calculate in the natural way, free of
concern. The reader interested in the details is referred to the
appendix \ref{appendix:rho_details}.

\subsection{Substitution}

We use $\Proc$ for the set of processes, $\QProc$ for the set of
names, and $\id{\{}\vec{y} / \vec{x} \id{\}}$ to denote partial maps,
$s : \QProc \rightarrow \QProc$. A map, $s$ lifts, uniquely, to a map
on process terms, $\widehat{s} : \Proc \rightarrow \Proc$ by the
following equations.

\begin{mathpar}
  (0) \psubstp{Q}{P} := 0 \\
  (R \juxtap S) \psubstp{Q}{P}
  :=    
  (R)\psubstp{Q}{P} \juxtap (S) \psubstp{Q}{P} \\
  (x?(y).R) \psubstp{Q}{P}    
  :=    
  (x)\substp{Q}{P} (z)\concat( (R \psubstn{z}{y}) \psubstp{Q}{P} ) \\
  (\lift{x}{R}) \psubstp{Q}{P}  
  :=
  \lift{(x)\substp{Q}{P}}{ R \psubstp{Q}{P} } \\
%   (\dropn{x})  \psubstp{Q}{P}       
%   := 
%   \left\{ 
%     \begin{array}{ccc} 
%       \dropn{\quotep{Q}} & & x \nameeq \quotep{P} \\
%       \dropn{x} & & otherwise \\
%     \end{array}
%   \right. 
  (\dropn{x})  \psubstp{Q}{P}       
  := 
  \left\{ 
    \begin{array}{ccc} 
      Q & & x \nameeq \quotep{P} \\
      \dropn{x} & & otherwise \\
    \end{array}
  \right.
\end{mathpar}
 

where

\begin{eqnarray}
  (x)\id{\{} \lpquote Q \rpquote / \lpquote P \rpquote \id{\}}            = 
  \left\{ 
    \begin{array}{ccc}
      \lpquote Q \rpquote & & x \nameeq \lpquote P \rpquote \\
      x & & otherwise \\
    \end{array}
  \right. \nonumber
\end{eqnarray}

and $z$ is chosen distinct from $\quotep{P}$, $\quotep{Q}$, the free
names in $Q$, and all the names in $R$. Our $\alpha$-equivalence will
be built in the standard way from this substitution.

\begin{remark}\label{rem:no_self_referential_names}
  One consequence of these definitions is that $\forall P. \quotep{P}
  \not\in \freenames{P}$.
\end{remark}

\subsection{ Dynamic quote: an example }

Anticipating something of what's to come, consider applying the
substitution, $\widehat{\id{\{}u / z \id{\}}}$, to the following pair
of processes, $\lift{w}{y!(z)}$ and $w[ \lpquote y!(z) \rpquote ]$.

\begin{eqnarray}
	\lift{w}{y!(z)}\widehat{\id{\{}u / z \id{\}}}
		& = &
		\lift{w}{y!(u)} \nonumber\\
	w[ \lpquote y!(z) \rpquote ] \widehat{ \id{\{}u / z \id{\}} }
		& = &
		w[ \lpquote y!(z) \rpquote ] \nonumber
\end{eqnarray}

Because the body of the process between quotes is impervious to
substitution, we get radically different answers. In fact, by
examining the first process in an input context,
e.g. $x?(z).\lift{w}{y!(z)}$, we see that the process under the lift
operator may be shaped by prefixed inputs binding a name inside it. In
this sense, the lift operator will be seen as a way to dynamically
construct processes before reifying them as names.

Finally equipped with these standard features we can present the
dynamics of the calculus.

\subsubsection{Operational semantics} 

Finally, we introduce the computational dynamics. What marks these
algebras as distinct from other more traditionally studied algebraic
structures, e.g. vector spaces or polynomial rings, is the manner in
which dynamics is captured. In traditional structures, dynamics is typically
expressed through morphisms between such structures, as in linear maps
between vector spaces or morphisms between rings. In algebras
associated with the semantics of computation, the dynamics is
expressed as part of the algebraic structure itself, through a
reduction reduction relation typically denoted by $\red$. Below, we
give a recursive presentation of this relation for the calculus used
in the encoding.

$\red \subseteq \pi \times \pi$
$\red : \pi \to \mathcal{P}(\pi)$

\begin{mathpar}
  \inferrule* [lab=Comm] { \textsf{match}( x_{src}, x_{trgt} ) } { x_{trgt}?(y)P \; | \; x_{src}!\langle {Q} \rangle \red P\{\quotep{Q}/y}\} }
  \and \\
  \inferrule* [lab=Par] {{P} \red {P}'} {{{P} | {Q}} \red {{P}' | {Q}}}
  \and
  \inferrule* [lab=Equiv]{{{P} \scong {P}'} \andalso {{P}' \red {Q}'} \andalso {{Q}' \scong {Q}}}{{P} \red {Q}}
\end{mathpar}

\begin{eqnarray*}
  match_{\equiv} (\quotep{P},\quotep{Q}) & := & P \equiv Q \\
  match_{\dagger}(\quotep{P},\quotep{Q}) & := & \forall R. P|Q \red^{*} R => R \red^{*} 0 \\
  match_{K}(\quotep{P},\quotep{Q}) & := & K \mbox{ for some context } K
\end{eqnarray*}

$u?(x)P | u!\langle Q \rangle \red P\{\quotep{Q}/x\}$

%We write $\wred$ for $\red^*$, and $P\red$ if $\exists Q $ such that $ P \red Q$.
We write $P\red$ if $\exists Q $ such that $ P \red Q$ and $P\not\red$, otherwise.

\section{Replication}

As mentioned before, it is known that replication (and hence
recursion) can be implemented in a higher-order process algebra
\cite{SangiorgiWalker}. As our first example of calculation with the
machinery thus far presented we give the construction explicitly in
the {\rhoc}.

\begin{eqnarray}
	D_{x} & := & \prefix{x}{y}{(\binpar{\outputp{x}{y}}{@{y}})} \nonumber\\
	\bangp_{x}{P} & := & \binpar{{x}!\langle{\binpar{D_{x}}{P}}\rangle}{D_{x}} \nonumber
\end{eqnarray}

\begin{eqnarray}
	\bangp_{x}{P} & & \nonumber\\
	=
	& {x}!\langle{(\prefix{x}{y}{(\outputp{x}{y} | @{y})) | P}}\rangle 
	      | \prefix{x}{y}{(\outputp{x}{y} | @{y})} & \nonumber\\
	\red
	& (\outputp{x}{y} | @{y})\substn{\quotep{(\prefix{x}{y}{(@{y} | \outputp{x}{y})) | P}}}{y} & \nonumber\\
	=
	& \outputp{x}{\quotep{(\prefix{x}{y}{(\outputp{x}{y} | @{y})) | P}}}
	  | {(\prefix{x}{y}{(\outputp{x}{y} | @{y})) | P}} & \nonumber\\
	\red
	& \ldots & \nonumber\\
	\red^*
	& P | P | \ldots & \nonumber
\end{eqnarray}

Of course, this encoding, as an implementation, runs away, unfolding
$\bangp{P}$ eagerly. A lazier and more implementable replication
operator, restricted to input-guarded processes, may be obtained as follows.

\begin{eqnarray}
\bangp{\prefix{u}{v}{P}} 
	:= 
	\binpar{\lift{x}{\prefix{u}{v}{(\binpar{D(x)}{P})}}}{D(x)} \nonumber
\end{eqnarray}

\begin{remark}
  Note that the lazier definition still does not deal with summation
  or mixed summation (i.e. sums over input and output). The reader is
  invited to construct definitions of replication that deal with these
  features. 

  Further, the definitions are parameterized in a name, $x$. Can you,
  gentle reader, make a definition that eliminates this parameter and
  guarantees no accidental interaction between the replication
  machinery and the process being replicated -- i.e. no accidental
  sharing of names used by the process to get its work done and the
  name(s) used by the replication to effect copying. This latter
  revision of the definition of replication is crucial to obtaining
  the expected identity $!!P \sim !P$.
\end{remark}

\begin{remark}\label{rem:paradoxical_combinator}
  The reader familiar with the lambda calculus will have noticed the
  similarity between $D$ and the paradoxical combinator.

  [Ed. note: the existence of this seems to suggest we have to be more
  restrictive on the set of processes and names we admit if we are to
  support no-cloning.]
\end{remark}

\subsubsection{Bisimulation}

The computational dynamics gives rise to another kind of equivalence,
the equivalence of computational behavior. As previously mentioned
this is typically captured \emph{via} some form of bisimulation.

% The notion we use in this paper is weak barbed bisimulation
% \cite{milner91polyadicpi}.

The notion we use in this paper is derived from weak barbed
bisimulation \cite{milner91polyadicpi}. 

\begin{definition}
An \emph{observation relation}, $\downarrow_{\mathcal N}$, over a set
of names, $\mathcal N$, is the smallest relation satisfying the rules
below.

\infrule[Out-barb]{y \in {\mathcal N}, \; x \nameeq y}
		  {\outputp{x}{v} \downarrow_{\mathcal N} x}
\infrule[Par-barb]{\mbox{$P\downarrow_{\mathcal N} x$ or $Q\downarrow_{\mathcal N} x$}}
		  {\binpar{P}{Q} \downarrow_{\mathcal N} x}

We write $P \Downarrow_{\mathcal N} x$ if there is $Q$ such that 
$P \wred Q$ and $Q \downarrow_{\mathcal N} x$.
\end{definition}

\begin{definition}
%\label{def.bbisim}
An  ${\mathcal N}$-\emph{barbed bisimulation} over a set of names, ${\mathcal N}$, is a symmetric binary relation 
${\mathcal S}_{\mathcal N}$ between agents such that $P\rel{S}_{\mathcal N}Q$ implies:
\begin{enumerate}
\item If $P \red P'$ then $Q \wred Q'$ and $P'\rel{S}_{\mathcal N} Q'$.
\item If $P\downarrow_{\mathcal N} x$, then $Q\Downarrow_{\mathcal N} x$.
\end{enumerate}
$P$ is ${\mathcal N}$-barbed bisimilar to $Q$, written
$P \wbbisim_{\mathcal N} Q$, if $P \rel{S}_{\mathcal N} Q$ for some ${\mathcal N}$-barbed bisimulation ${\mathcal S}_{\mathcal N}$.
\end{definition}

$\mathcal{R} \subseteq \pi \times \pi$

$P \mathcal{R} Q => \forall P'. P \red P' \Rightarrow \exists Q'. Q \red Q', P' \mathcal{R} Q'$

$P \vdash x \Rightarrow Q \vdash x$

\begin{mathpar}
  \inferrule*[lab=Out-barb]{x \nameeq y}{{y}!\langle{Q}\rangle \vdash x}
  \and
  \inferrule*[lab=Par-barb]{\mbox{$P\vdash x$ or $Q\vdash x$}}{\binpar{P}{Q} \vdash x}
\end{mathpar}

\subsubsection{Contexts}

One of the principle advantages of computational calculi like the
$\pi$-calculus is a well-defined notion of context,
contextual-equivalence and a correlation between
contextual-equivalence and notions of bisimulation. The notion of
context allows the decomposition of a process into (sub-)process and
its syntactic environment, its context. Thus, a context may be
thought of as a process with a ``hole'' (written $\Box$) in it. The
application of a context $M$ to a process $P$, written $M[P]$, is
tantamount to filling the hole in $M$ with $P$. In this paper we do
not need the full weight of this theory, but do make use of the notion
of context in the proof the main theorem. 

\begin{mathpar}
  \inferrule* [lab=summation] {} {{M_{M},M_{N}} \bc \Box \;|\; x.M_{A} \;|\; M_{M}+M_{N}}
  \and
  \inferrule* [lab=agent] {} {{M_{A}} \bc (\vec{x})M_{P} \;| \; \clift{P_0,\ldots,M_{P},\ldots,P_N}}
  \and \\
  \inferrule* [lab=process] {} {{M_{P}} \bc M_{N} \;| \;P|M_{P} }
\end{mathpar} 

\begin{mathpar}
  \inferrule* [lab=sychronization] {} {M_{N} \bc \Box \;|\; x?M_{F} \;|\; x!M_{C}}
  \and
  \inferrule* [lab=abstraction] {} {{M_{F}} \bc (x)M_{P} }
  \and
  \inferrule* [lab=concretion] {} {{M_{C}} \bc \langle M_{P} \rangle }
  \and \\
  \inferrule* [lab=process] {} {{M_{P}} \bc M_{N} \;| \;P|M_{P} }
\end{mathpar}

\begin{definition}[contextual application] Given a context $M$, and
  process $P$, we define the \emph{contextual application}, $M[P] :=
  M\{P/\Box\}$. That is, the contextual application of M to P is the
  substitution of $P$ for $\Box$ in $M$.
\end{definition}

$\meaningof{-} : L \to \mathcal{P}(\pi)$

\begin{mathpar}
  \inferrule* [lab=collection] {} {\meaningof{true} = \pi, \and \meaningof{~E} = \pi \setminus \meaningof{E}, \and \meaningof{E_{1} \& E_{2}} = \meaningof{E_{1}} \cap \meaningof{E_{2}}}
\end{mathpar}

\begin{mathpar}
  \inferrule* [lab=structure] {} {\meaningof{0} = \{ P \in \pi | P \equiv 0 \}, \and \\ \meaningof{E_1 | E_2} = \{ P \in \pi | P \equiv P_{1} | P_{2}, P_{1} \in \meaningof{E_{1}}, P_{2} \in \meaningof{E_2}\} }
\end{mathpar}

\begin{mathpar}
 \inferrule* [lab=behavior] {} {\meaningof{\langle a?b \rangle E} = \{ P \in \pi | P \equiv Q | u?(y)P', \\ \and \\\\ \and \\ \;\;\; u \in \meaningof{a}, \forall z.P'\{z/y\} \in \meaningof{E\{z/b\}}\}, \and \\ \meaningof{a!E} = \{ P \in \pi | P \equiv Q | x!\langle P' \rangle, x \in \meaningof{a} P' \in \meaningof{E}\} }
\end{mathpar}

\begin{mathpar}
 \inferrule* [lab=nominal] {} {\meaningof{\quotep{E}} = \{ \quotep{P} \in \quotep{\pi} | P \in \meaningof{E} \}, \and \meaningof{\quotep{P}} = \{ \quotep{Q} \in \quotep{\pi} | P \equiv Q \} \and \\ \meaningof{@\quotep{E}} = \{ P \in \pi | P \equiv @x, x \in \meaningof{E} \}}
\end{mathpar}

\begin{eqnarray*}
  \\
  \meaningof{-} : TS \to ST
\end{eqnarray*}

\begin{eqnarray*}
  \\
  L : TS \to ST
\end{eqnarray*}

\begin{eqnarray*}
  \\
  P \models E \iff P \in \meaningof{E}
\end{eqnarray*}

\begin{eqnarray*}
  P \approx_{L} Q \iff \forall E \in L. P \models E \iff Q \models E
\end{eqnarray*}

\begin{eqnarray*}
  P \approx_{K} Q
\end{eqnarray*}

\begin{eqnarray*}
  P \approx Q
\end{eqnarray*}

$\approx_{K} = \approx = \approx_{L}$

\subsubsection{Contextual duality}

Note that contexts extend the quotation operation to a family of
operations from processes to names. Given a context, $M$, we can
define a \emph{nominal context}, $\quotep{M}$ by $\quotep{M}[P] :=
\quotep{M[P]}$. To foreshadow what is to come we observe that these
operations enjoy a duality with processes very much like the duality
between vectors and maps from vectors to scalars.

Further, because the calculus is essentially higher-order, we have a
correspondence between contexts and processes. More specifically,
given a name $x$ and a context $M$ we can construct $M^{*}_{x}$ such
that 

\begin{mathpar}
  M^{*}_{x} | \lift{x}{P} \red M[P]
\end{mathpar}

namely,

\begin{mathpar}
  M^{*}_{x} := x?(u).M[\dropn{u}]
\end{mathpar}

The dependence of $M^{*}_{x}$ on a name makes it an abstraction, 

\begin{mathpar}
  M^{*} := (x)x?(u).M[\dropn{u}]
\end{mathpar}

\subsection{Additional notation}

It will sometimes be convenient to denote the process a name
quotes. We already have the notation $x = \quotep{P}$, but it will be
convenient to introduce an alternate notation, $\procn{x}$, when we
want to emphasize the connection to the use of the name. Note that, by
virtue of name equivalence, $\quotep{\procn{x}} \nameeq x$; so, the
notation is consistent with previous definitions.

Further, because names have structure it is possible to effect
substitutions on the basis of that structure. This means we need to
upgrade our notation for substitutions, which we accomplish by
adapting comprehension notation. Thus,

\begin{mathpar}
  P\{ y / x : x \in S \}
\end{mathpar}

is interpreted to mean the process derived from P by replacing (in a
capture-avoiding manner) each occurrence of $x$ in $S$ by $y$. For example,

\begin{mathpar}
  P\{ \quotep{\procn{x}|\procn{x}} / x : x \in \freenames{P} \}
\end{mathpar}

will replace each (occurrence) of a free name $x$ in $P$ by
$\quotep{\procn{x}|\procn{x}}$.

Also, we will avail ourselves of the notation $x^{L}$ and $x^{R}$ to
denote injections of a name into disjoint copies of the name
space. There are numerous ways to accomplish this. One example can be
found in \cite{MeredithR05}. This notation overloads to vectors of
names: $\vec{x}^{\pi} := (x_{i}^{\pi} \; : \; 0 \leq i < |\vec{x}| )$ where $\pi \in \{L,R\}$.

We also use $P^{\Box} := P|\Box$.

In \cite{MeredithR05} an interpretation of the new operator is
given. It turns out that there are several possible interpretations
all enjoying the requisite algebraic properties of the operator (see
\cite{milner91polyadicpi}). We will therefore make liberal use of
$(\nu\; \vec{x})P$.

% subsection the_syntax_and_semantics_of_the_notation_system (end)   

\input{qm2pi.qmops} 

\input{qm2pi.sterngerlach} 

\input{qm2pi.metric} 

% section concurrent_process_calculi (end)

%\input{qm2pi.proofsketch}

% section proof sketch (end)

%\input{qm2pi.slviaknots} 

% section spatial logic via knots (end)

\input{qm2pi.conclusion}

% section conclusion (end)

%\input{qm2pi.dtcodes} 

% section wiring algorithm (end)

\input{qm2pi.ack} 

% section acknowledgments (end)

\newpage


\bibliographystyle{plain}   
\bibliography{../../biblios/main.bib}

\input{qm2pi.rhodetails}

\end{document}

 

% section wiring algorithm (end)

\documentclass[12pt]{llncs}
%\documentclass{jktr}

\usepackage[pdftex]{hyperref}                   
\usepackage {listings}
\usepackage {mathpartir}
\usepackage{bcprules}
%\usepackage{listings}
                       
\usepackage{graphicx} 
%\usepackage[margins=2.5cm,nohead,nofoot]{geometry}
%\usepackage{geometry}
\usepackage{amsfonts}
\usepackage{amstext}
\usepackage{latexsym}
\usepackage{amssymb}
\usepackage{color}


%\include{myPreamble}
\include{qm2pi.local} 

%\ifpdf
%\usepackage[pdftex]{graphicx}
%\else
%\usepackage{graphicx}
%\fi

 % \ifpdf
%  \usepackage{pdfsync}
%  \if


%\title{Brief Article}
%\author{David F. Snyder}
%\author{L.G. Meredith}

%\address{Dept. of Math., Texas State University--San Marcos, San Marcos, TX 78666}
       
\pagestyle{empty}


\begin{document}

\lstset{language=[Objective]Caml,frame=shadowbox}

\input{qm2pi.front}

% section front matter (end)

\input{qm2pi.intro} 
 
% section introduction (end)

% \input{qm2pi.knotations} 

% section notation (end)

\input{qm2pi.process.calculi} 

% section concurrent_process_calculi_and_spatial_logics_ (end)
    
%\input{qm2pi.knots2pi} 

%\input{qm2pi.trefoil} 

%\input{qm2pi.mainthm} 

% subsection basic_interpretation (end)

%\input{qm2pi.rho.presentation} 
\subsection{The syntax and semantics of the notation system}\label{sub:the_syntax_and_semantics_of_the_notation_system} % (fold)

We now summarize a technical presentation of the calculus that
embodies our theory of dynamics. The typical presentation of such a
calculus follows the style of giving generators and relations on
them. The grammar, below, describing term constructors, freely
generates the set of processes, $\Proc$. This set is then quotiented
by a relation known as structural congruence and it is over this set
that the notion of dynamics is expressed. This presentation is
essentially that of \cite{MeredithR05} with the addition of
polyadicity and summation. For readability we have relegated some of
the technical subtleties to an appendix.

\subsubsection{Process grammar}\label{subsub:process_grammar}

\begin{mathpar}
  \inferrule* [lab=synchronization] {} {{M} \bc \pzero \;|\; x?F \;|\; x!C }
  \and
  \inferrule* [lab=abstraction] {} {{F} \bc (x)P}
  \and
  \inferrule* [lab=concretion] {} {{C} \bc \langle Q \rangle}
  \and
  \inferrule* [lab=process] {} {{P,Q} \bc M \;| \;P|Q \;|\; @{x}}
  \and
  \inferrule* [lab=name] {} {{x} \bc \quotep{P}}
\end{mathpar} 

Note that $\vec{x}$ (resp. $\vec{P}$) denotes a vector of names
(resp. processes) of length $|\vec{x}|$ (resp. $|\vec{P}|$). We adopt
the following useful abbreviations.

\begin{mathpar}
   x?(\vec{y}).P := x.(\vec{y})P \and  x\clift{\vec{P}} := x.\clift{\vec{P}}
   \and x!(y) := \lift{x}{\dropn{y}}
   \and \Pi_{i=0}^{n-1}P_i := P_0 | \ldots | P_{n-1}
\end{mathpar}

\subsubsection{Structural congruence}

\paragraph{Free and bound names and alpha-equivalence.} At the
core of structural equivalence is alpha-equivalence which identifies
process that are the same up to a change of variable. Formally, we
recognize the distinction between free and bound names. The free names
of a process, $\freenames{P}$, may be calculated recursively as
follows:

\begin{mathpar}
\freenames{\pzero} := \emptyset
  \and \\
  \freenames{x?(y).P} := \{ x \} \cup (\freenames{P} \setminus \{ y \})
  \and 
  \freenames{x!\langle P \rangle} := \{ x \} \cup \{ P \} 
  \and \\
  \freenames{P|Q} := \freenames{P} \cup \freenames{Q}
  \and \\
  \freenames{@{x}} := \{ x \}
\end{mathpar}

$\pi$
$\quotep{\pi}$

$\freenames{-} : \pi \to \mathcal{P}(\quotep{\pi})$

\begin{eqnarray*}
  \freenames{\pzero} & := & \emptyset \\
  \freenames{x?(y).P} & := & \{ x \} \cup (\freenames{P} \setminus \{ y \}) \\
  \freenames{x!\langle P \rangle} & := & \{ x \} \cup \{ P \} \\
  \freenames{P|Q} & := & \freenames{P} \cup \freenames{Q} \\
  \freenames{\dropn{x}} & := & \{ x \}
\end{eqnarray*}

The bound names of a process, $\boundnames{P}$, are those names occurring in $P$
that are not free. For example, in $x?(y).0$, the name $x$ is free, while $y$ is bound.

\begin{mathpar}
  \inferrule* [lab=monoidal-laws] {} { P|Q \equiv Q|P \and P|0 \equiv P \and P|(Q|R) \equiv (P|Q)|R }
\end{mathpar}

\begin{mathpar}
  \inferrule* [lab=alpha-equivalence] {} { (x)P \equiv (y)P\{y/x\} \and y \not\in \freenames{P} }
\end{mathpar}

\begin{definition}
Then two processes, $P,Q$, are alpha-equivalent if $P = Q\{\vec{y}/\vec{x}\}$ for
some $\vec{x} \in \boundnames{Q},\vec{y} \in \boundnames{P}$, where $Q\{\vec{y}/\vec{x}\}$
denotes the capture-avoiding substitution of $\vec{y}$ for $\vec{x}$ in $Q$.
\end{definition}

\begin{definition}
  The {\em structural congruence} \cite{SangiorgiWalker} , $\equiv$,
  between processes is the least congruence containing
  alpha-equivalence, satisfying the abelian monoid laws
  (associativity, commutativity and $\pzero$ as identity) for parallel
  composition $|$ and for summation $+$.
\end{definition}

\subsection{Name equivalence}

We take name equivalence, written $\nameeq$, to be the smallest
equivalence relation generated by the following rules.

\begin{mathpar}
\inferrule*[lab=Quote-drop]
{ }
{ \quotep{@{x}} \nameeq x }

\inferrule*[lab=Struct-equiv]
{ P \scong Q }
{ \quotep{P} \nameeq \quotep{Q} }
\end{mathpar}

The astute reader will have noticed that the mutual recursion of names
and processes imposes a mutual recursion on alpha-equivalence and
structural equivalence via name-equivalence. Fortunately, all of this
works out pleasantly and we may calculate in the natural way, free of
concern. The reader interested in the details is referred to the
appendix \ref{appendix:rho_details}.

\subsection{Substitution}

We use $\Proc$ for the set of processes, $\QProc$ for the set of
names, and $\id{\{}\vec{y} / \vec{x} \id{\}}$ to denote partial maps,
$s : \QProc \rightarrow \QProc$. A map, $s$ lifts, uniquely, to a map
on process terms, $\widehat{s} : \Proc \rightarrow \Proc$ by the
following equations.

\begin{mathpar}
  (0) \psubstp{Q}{P} := 0 \\
  (R \juxtap S) \psubstp{Q}{P}
  :=    
  (R)\psubstp{Q}{P} \juxtap (S) \psubstp{Q}{P} \\
  (x?(y).R) \psubstp{Q}{P}    
  :=    
  (x)\substp{Q}{P} (z)\concat( (R \psubstn{z}{y}) \psubstp{Q}{P} ) \\
  (\lift{x}{R}) \psubstp{Q}{P}  
  :=
  \lift{(x)\substp{Q}{P}}{ R \psubstp{Q}{P} } \\
%   (\dropn{x})  \psubstp{Q}{P}       
%   := 
%   \left\{ 
%     \begin{array}{ccc} 
%       \dropn{\quotep{Q}} & & x \nameeq \quotep{P} \\
%       \dropn{x} & & otherwise \\
%     \end{array}
%   \right. 
  (\dropn{x})  \psubstp{Q}{P}       
  := 
  \left\{ 
    \begin{array}{ccc} 
      Q & & x \nameeq \quotep{P} \\
      \dropn{x} & & otherwise \\
    \end{array}
  \right.
\end{mathpar}
 

where

\begin{eqnarray}
  (x)\id{\{} \lpquote Q \rpquote / \lpquote P \rpquote \id{\}}            = 
  \left\{ 
    \begin{array}{ccc}
      \lpquote Q \rpquote & & x \nameeq \lpquote P \rpquote \\
      x & & otherwise \\
    \end{array}
  \right. \nonumber
\end{eqnarray}

and $z$ is chosen distinct from $\quotep{P}$, $\quotep{Q}$, the free
names in $Q$, and all the names in $R$. Our $\alpha$-equivalence will
be built in the standard way from this substitution.

\begin{remark}\label{rem:no_self_referential_names}
  One consequence of these definitions is that $\forall P. \quotep{P}
  \not\in \freenames{P}$.
\end{remark}

\subsection{ Dynamic quote: an example }

Anticipating something of what's to come, consider applying the
substitution, $\widehat{\id{\{}u / z \id{\}}}$, to the following pair
of processes, $\lift{w}{y!(z)}$ and $w[ \lpquote y!(z) \rpquote ]$.

\begin{eqnarray}
	\lift{w}{y!(z)}\widehat{\id{\{}u / z \id{\}}}
		& = &
		\lift{w}{y!(u)} \nonumber\\
	w[ \lpquote y!(z) \rpquote ] \widehat{ \id{\{}u / z \id{\}} }
		& = &
		w[ \lpquote y!(z) \rpquote ] \nonumber
\end{eqnarray}

Because the body of the process between quotes is impervious to
substitution, we get radically different answers. In fact, by
examining the first process in an input context,
e.g. $x?(z).\lift{w}{y!(z)}$, we see that the process under the lift
operator may be shaped by prefixed inputs binding a name inside it. In
this sense, the lift operator will be seen as a way to dynamically
construct processes before reifying them as names.

Finally equipped with these standard features we can present the
dynamics of the calculus.

\subsubsection{Operational semantics} 

Finally, we introduce the computational dynamics. What marks these
algebras as distinct from other more traditionally studied algebraic
structures, e.g. vector spaces or polynomial rings, is the manner in
which dynamics is captured. In traditional structures, dynamics is typically
expressed through morphisms between such structures, as in linear maps
between vector spaces or morphisms between rings. In algebras
associated with the semantics of computation, the dynamics is
expressed as part of the algebraic structure itself, through a
reduction reduction relation typically denoted by $\red$. Below, we
give a recursive presentation of this relation for the calculus used
in the encoding.

$\red \subseteq \pi \times \pi$
$\red : \pi \to \mathcal{P}(\pi)$

\begin{mathpar}
  \inferrule* [lab=Comm] { \textsf{match}( x_{src}, x_{trgt} ) } { x_{trgt}?(y)P \; | \; x_{src}!\langle {Q} \rangle \red P\{\quotep{Q}/y}\} }
  \and \\
  \inferrule* [lab=Par] {{P} \red {P}'} {{{P} | {Q}} \red {{P}' | {Q}}}
  \and
  \inferrule* [lab=Equiv]{{{P} \scong {P}'} \andalso {{P}' \red {Q}'} \andalso {{Q}' \scong {Q}}}{{P} \red {Q}}
\end{mathpar}

\begin{eqnarray*}
  match_{\equiv} (\quotep{P},\quotep{Q}) & := & P \equiv Q \\
  match_{\dagger}(\quotep{P},\quotep{Q}) & := & \forall R. P|Q \red^{*} R => R \red^{*} 0 \\
  match_{K}(\quotep{P},\quotep{Q}) & := & K \mbox{ for some context } K
\end{eqnarray*}

$u?(x)P | u!\langle Q \rangle \red P\{\quotep{Q}/x\}$

%We write $\wred$ for $\red^*$, and $P\red$ if $\exists Q $ such that $ P \red Q$.
We write $P\red$ if $\exists Q $ such that $ P \red Q$ and $P\not\red$, otherwise.

\section{Replication}

As mentioned before, it is known that replication (and hence
recursion) can be implemented in a higher-order process algebra
\cite{SangiorgiWalker}. As our first example of calculation with the
machinery thus far presented we give the construction explicitly in
the {\rhoc}.

\begin{eqnarray}
	D_{x} & := & \prefix{x}{y}{(\binpar{\outputp{x}{y}}{@{y}})} \nonumber\\
	\bangp_{x}{P} & := & \binpar{{x}!\langle{\binpar{D_{x}}{P}}\rangle}{D_{x}} \nonumber
\end{eqnarray}

\begin{eqnarray}
	\bangp_{x}{P} & & \nonumber\\
	=
	& {x}!\langle{(\prefix{x}{y}{(\outputp{x}{y} | @{y})) | P}}\rangle 
	      | \prefix{x}{y}{(\outputp{x}{y} | @{y})} & \nonumber\\
	\red
	& (\outputp{x}{y} | @{y})\substn{\quotep{(\prefix{x}{y}{(@{y} | \outputp{x}{y})) | P}}}{y} & \nonumber\\
	=
	& \outputp{x}{\quotep{(\prefix{x}{y}{(\outputp{x}{y} | @{y})) | P}}}
	  | {(\prefix{x}{y}{(\outputp{x}{y} | @{y})) | P}} & \nonumber\\
	\red
	& \ldots & \nonumber\\
	\red^*
	& P | P | \ldots & \nonumber
\end{eqnarray}

Of course, this encoding, as an implementation, runs away, unfolding
$\bangp{P}$ eagerly. A lazier and more implementable replication
operator, restricted to input-guarded processes, may be obtained as follows.

\begin{eqnarray}
\bangp{\prefix{u}{v}{P}} 
	:= 
	\binpar{\lift{x}{\prefix{u}{v}{(\binpar{D(x)}{P})}}}{D(x)} \nonumber
\end{eqnarray}

\begin{remark}
  Note that the lazier definition still does not deal with summation
  or mixed summation (i.e. sums over input and output). The reader is
  invited to construct definitions of replication that deal with these
  features. 

  Further, the definitions are parameterized in a name, $x$. Can you,
  gentle reader, make a definition that eliminates this parameter and
  guarantees no accidental interaction between the replication
  machinery and the process being replicated -- i.e. no accidental
  sharing of names used by the process to get its work done and the
  name(s) used by the replication to effect copying. This latter
  revision of the definition of replication is crucial to obtaining
  the expected identity $!!P \sim !P$.
\end{remark}

\begin{remark}\label{rem:paradoxical_combinator}
  The reader familiar with the lambda calculus will have noticed the
  similarity between $D$ and the paradoxical combinator.

  [Ed. note: the existence of this seems to suggest we have to be more
  restrictive on the set of processes and names we admit if we are to
  support no-cloning.]
\end{remark}

\subsubsection{Bisimulation}

The computational dynamics gives rise to another kind of equivalence,
the equivalence of computational behavior. As previously mentioned
this is typically captured \emph{via} some form of bisimulation.

% The notion we use in this paper is weak barbed bisimulation
% \cite{milner91polyadicpi}.

The notion we use in this paper is derived from weak barbed
bisimulation \cite{milner91polyadicpi}. 

\begin{definition}
An \emph{observation relation}, $\downarrow_{\mathcal N}$, over a set
of names, $\mathcal N$, is the smallest relation satisfying the rules
below.

\infrule[Out-barb]{y \in {\mathcal N}, \; x \nameeq y}
		  {\outputp{x}{v} \downarrow_{\mathcal N} x}
\infrule[Par-barb]{\mbox{$P\downarrow_{\mathcal N} x$ or $Q\downarrow_{\mathcal N} x$}}
		  {\binpar{P}{Q} \downarrow_{\mathcal N} x}

We write $P \Downarrow_{\mathcal N} x$ if there is $Q$ such that 
$P \wred Q$ and $Q \downarrow_{\mathcal N} x$.
\end{definition}

\begin{definition}
%\label{def.bbisim}
An  ${\mathcal N}$-\emph{barbed bisimulation} over a set of names, ${\mathcal N}$, is a symmetric binary relation 
${\mathcal S}_{\mathcal N}$ between agents such that $P\rel{S}_{\mathcal N}Q$ implies:
\begin{enumerate}
\item If $P \red P'$ then $Q \wred Q'$ and $P'\rel{S}_{\mathcal N} Q'$.
\item If $P\downarrow_{\mathcal N} x$, then $Q\Downarrow_{\mathcal N} x$.
\end{enumerate}
$P$ is ${\mathcal N}$-barbed bisimilar to $Q$, written
$P \wbbisim_{\mathcal N} Q$, if $P \rel{S}_{\mathcal N} Q$ for some ${\mathcal N}$-barbed bisimulation ${\mathcal S}_{\mathcal N}$.
\end{definition}

$\mathcal{R} \subseteq \pi \times \pi$

$P \mathcal{R} Q => \forall P'. P \red P' \Rightarrow \exists Q'. Q \red Q', P' \mathcal{R} Q'$

$P \vdash x \Rightarrow Q \vdash x$

\begin{mathpar}
  \inferrule*[lab=Out-barb]{x \nameeq y}{{y}!\langle{Q}\rangle \vdash x}
  \and
  \inferrule*[lab=Par-barb]{\mbox{$P\vdash x$ or $Q\vdash x$}}{\binpar{P}{Q} \vdash x}
\end{mathpar}

\subsubsection{Contexts}

One of the principle advantages of computational calculi like the
$\pi$-calculus is a well-defined notion of context,
contextual-equivalence and a correlation between
contextual-equivalence and notions of bisimulation. The notion of
context allows the decomposition of a process into (sub-)process and
its syntactic environment, its context. Thus, a context may be
thought of as a process with a ``hole'' (written $\Box$) in it. The
application of a context $M$ to a process $P$, written $M[P]$, is
tantamount to filling the hole in $M$ with $P$. In this paper we do
not need the full weight of this theory, but do make use of the notion
of context in the proof the main theorem. 

\begin{mathpar}
  \inferrule* [lab=summation] {} {{M_{M},M_{N}} \bc \Box \;|\; x.M_{A} \;|\; M_{M}+M_{N}}
  \and
  \inferrule* [lab=agent] {} {{M_{A}} \bc (\vec{x})M_{P} \;| \; \clift{P_0,\ldots,M_{P},\ldots,P_N}}
  \and \\
  \inferrule* [lab=process] {} {{M_{P}} \bc M_{N} \;| \;P|M_{P} }
\end{mathpar} 

\begin{mathpar}
  \inferrule* [lab=sychronization] {} {M_{N} \bc \Box \;|\; x?M_{F} \;|\; x!M_{C}}
  \and
  \inferrule* [lab=abstraction] {} {{M_{F}} \bc (x)M_{P} }
  \and
  \inferrule* [lab=concretion] {} {{M_{C}} \bc \langle M_{P} \rangle }
  \and \\
  \inferrule* [lab=process] {} {{M_{P}} \bc M_{N} \;| \;P|M_{P} }
\end{mathpar}

\begin{definition}[contextual application] Given a context $M$, and
  process $P$, we define the \emph{contextual application}, $M[P] :=
  M\{P/\Box\}$. That is, the contextual application of M to P is the
  substitution of $P$ for $\Box$ in $M$.
\end{definition}

$\meaningof{-} : L \to \mathcal{P}(\pi)$

\begin{mathpar}
  \inferrule* [lab=collection] {} {\meaningof{true} = \pi, \and \meaningof{~E} = \pi \setminus \meaningof{E}, \and \meaningof{E_{1} \& E_{2}} = \meaningof{E_{1}} \cap \meaningof{E_{2}}}
\end{mathpar}

\begin{mathpar}
  \inferrule* [lab=structure] {} {\meaningof{0} = \{ P \in \pi | P \equiv 0 \}, \and \\ \meaningof{E_1 | E_2} = \{ P \in \pi | P \equiv P_{1} | P_{2}, P_{1} \in \meaningof{E_{1}}, P_{2} \in \meaningof{E_2}\} }
\end{mathpar}

\begin{mathpar}
 \inferrule* [lab=behavior] {} {\meaningof{\langle a?b \rangle E} = \{ P \in \pi | P \equiv Q | u?(y)P', \\ \and \\\\ \and \\ \;\;\; u \in \meaningof{a}, \forall z.P'\{z/y\} \in \meaningof{E\{z/b\}}\}, \and \\ \meaningof{a!E} = \{ P \in \pi | P \equiv Q | x!\langle P' \rangle, x \in \meaningof{a} P' \in \meaningof{E}\} }
\end{mathpar}

\begin{mathpar}
 \inferrule* [lab=nominal] {} {\meaningof{\quotep{E}} = \{ \quotep{P} \in \quotep{\pi} | P \in \meaningof{E} \}, \and \meaningof{\quotep{P}} = \{ \quotep{Q} \in \quotep{\pi} | P \equiv Q \} \and \\ \meaningof{@\quotep{E}} = \{ P \in \pi | P \equiv @x, x \in \meaningof{E} \}}
\end{mathpar}

\begin{eqnarray*}
  \\
  \meaningof{-} : TS \to ST
\end{eqnarray*}

\begin{eqnarray*}
  \\
  L : TS \to ST
\end{eqnarray*}

\begin{eqnarray*}
  \\
  P \models E \iff P \in \meaningof{E}
\end{eqnarray*}

\begin{eqnarray*}
  P \approx_{L} Q \iff \forall E \in L. P \models E \iff Q \models E
\end{eqnarray*}

\begin{eqnarray*}
  P \approx_{K} Q
\end{eqnarray*}

\begin{eqnarray*}
  P \approx Q
\end{eqnarray*}

$\approx_{K} = \approx = \approx_{L}$

\subsubsection{Contextual duality}

Note that contexts extend the quotation operation to a family of
operations from processes to names. Given a context, $M$, we can
define a \emph{nominal context}, $\quotep{M}$ by $\quotep{M}[P] :=
\quotep{M[P]}$. To foreshadow what is to come we observe that these
operations enjoy a duality with processes very much like the duality
between vectors and maps from vectors to scalars.

Further, because the calculus is essentially higher-order, we have a
correspondence between contexts and processes. More specifically,
given a name $x$ and a context $M$ we can construct $M^{*}_{x}$ such
that 

\begin{mathpar}
  M^{*}_{x} | \lift{x}{P} \red M[P]
\end{mathpar}

namely,

\begin{mathpar}
  M^{*}_{x} := x?(u).M[\dropn{u}]
\end{mathpar}

The dependence of $M^{*}_{x}$ on a name makes it an abstraction, 

\begin{mathpar}
  M^{*} := (x)x?(u).M[\dropn{u}]
\end{mathpar}

\subsection{Additional notation}

It will sometimes be convenient to denote the process a name
quotes. We already have the notation $x = \quotep{P}$, but it will be
convenient to introduce an alternate notation, $\procn{x}$, when we
want to emphasize the connection to the use of the name. Note that, by
virtue of name equivalence, $\quotep{\procn{x}} \nameeq x$; so, the
notation is consistent with previous definitions.

Further, because names have structure it is possible to effect
substitutions on the basis of that structure. This means we need to
upgrade our notation for substitutions, which we accomplish by
adapting comprehension notation. Thus,

\begin{mathpar}
  P\{ y / x : x \in S \}
\end{mathpar}

is interpreted to mean the process derived from P by replacing (in a
capture-avoiding manner) each occurrence of $x$ in $S$ by $y$. For example,

\begin{mathpar}
  P\{ \quotep{\procn{x}|\procn{x}} / x : x \in \freenames{P} \}
\end{mathpar}

will replace each (occurrence) of a free name $x$ in $P$ by
$\quotep{\procn{x}|\procn{x}}$.

Also, we will avail ourselves of the notation $x^{L}$ and $x^{R}$ to
denote injections of a name into disjoint copies of the name
space. There are numerous ways to accomplish this. One example can be
found in \cite{MeredithR05}. This notation overloads to vectors of
names: $\vec{x}^{\pi} := (x_{i}^{\pi} \; : \; 0 \leq i < |\vec{x}| )$ where $\pi \in \{L,R\}$.

We also use $P^{\Box} := P|\Box$.

In \cite{MeredithR05} an interpretation of the new operator is
given. It turns out that there are several possible interpretations
all enjoying the requisite algebraic properties of the operator (see
\cite{milner91polyadicpi}). We will therefore make liberal use of
$(\nu\; \vec{x})P$.

% subsection the_syntax_and_semantics_of_the_notation_system (end)   

\input{qm2pi.qmops} 

\input{qm2pi.sterngerlach} 

\input{qm2pi.metric} 

% section concurrent_process_calculi (end)

%\input{qm2pi.proofsketch}

% section proof sketch (end)

%\input{qm2pi.slviaknots} 

% section spatial logic via knots (end)

\input{qm2pi.conclusion}

% section conclusion (end)

%\input{qm2pi.dtcodes} 

% section wiring algorithm (end)

\input{qm2pi.ack} 

% section acknowledgments (end)

\newpage


\bibliographystyle{plain}   
\bibliography{../../biblios/main.bib}

\input{qm2pi.rhodetails}

\end{document}

 

% section acknowledgments (end)

\newpage


\bibliographystyle{plain}   
\bibliography{../../biblios/main.bib}

\documentclass[12pt]{llncs}
%\documentclass{jktr}

\usepackage[pdftex]{hyperref}                   
\usepackage {listings}
\usepackage {mathpartir}
\usepackage{bcprules}
%\usepackage{listings}
                       
\usepackage{graphicx} 
%\usepackage[margins=2.5cm,nohead,nofoot]{geometry}
%\usepackage{geometry}
\usepackage{amsfonts}
\usepackage{amstext}
\usepackage{latexsym}
\usepackage{amssymb}
\usepackage{color}


%\include{myPreamble}
\include{qm2pi.local} 

%\ifpdf
%\usepackage[pdftex]{graphicx}
%\else
%\usepackage{graphicx}
%\fi

 % \ifpdf
%  \usepackage{pdfsync}
%  \if


%\title{Brief Article}
%\author{David F. Snyder}
%\author{L.G. Meredith}

%\address{Dept. of Math., Texas State University--San Marcos, San Marcos, TX 78666}
       
\pagestyle{empty}


\begin{document}

\lstset{language=[Objective]Caml,frame=shadowbox}

\input{qm2pi.front}

% section front matter (end)

\input{qm2pi.intro} 
 
% section introduction (end)

% \input{qm2pi.knotations} 

% section notation (end)

\input{qm2pi.process.calculi} 

% section concurrent_process_calculi_and_spatial_logics_ (end)
    
%\input{qm2pi.knots2pi} 

%\input{qm2pi.trefoil} 

%\input{qm2pi.mainthm} 

% subsection basic_interpretation (end)

%\input{qm2pi.rho.presentation} 
\subsection{The syntax and semantics of the notation system}\label{sub:the_syntax_and_semantics_of_the_notation_system} % (fold)

We now summarize a technical presentation of the calculus that
embodies our theory of dynamics. The typical presentation of such a
calculus follows the style of giving generators and relations on
them. The grammar, below, describing term constructors, freely
generates the set of processes, $\Proc$. This set is then quotiented
by a relation known as structural congruence and it is over this set
that the notion of dynamics is expressed. This presentation is
essentially that of \cite{MeredithR05} with the addition of
polyadicity and summation. For readability we have relegated some of
the technical subtleties to an appendix.

\subsubsection{Process grammar}\label{subsub:process_grammar}

\begin{mathpar}
  \inferrule* [lab=synchronization] {} {{M} \bc \pzero \;|\; x?F \;|\; x!C }
  \and
  \inferrule* [lab=abstraction] {} {{F} \bc (x)P}
  \and
  \inferrule* [lab=concretion] {} {{C} \bc \langle Q \rangle}
  \and
  \inferrule* [lab=process] {} {{P,Q} \bc M \;| \;P|Q \;|\; @{x}}
  \and
  \inferrule* [lab=name] {} {{x} \bc \quotep{P}}
\end{mathpar} 

Note that $\vec{x}$ (resp. $\vec{P}$) denotes a vector of names
(resp. processes) of length $|\vec{x}|$ (resp. $|\vec{P}|$). We adopt
the following useful abbreviations.

\begin{mathpar}
   x?(\vec{y}).P := x.(\vec{y})P \and  x\clift{\vec{P}} := x.\clift{\vec{P}}
   \and x!(y) := \lift{x}{\dropn{y}}
   \and \Pi_{i=0}^{n-1}P_i := P_0 | \ldots | P_{n-1}
\end{mathpar}

\subsubsection{Structural congruence}

\paragraph{Free and bound names and alpha-equivalence.} At the
core of structural equivalence is alpha-equivalence which identifies
process that are the same up to a change of variable. Formally, we
recognize the distinction between free and bound names. The free names
of a process, $\freenames{P}$, may be calculated recursively as
follows:

\begin{mathpar}
\freenames{\pzero} := \emptyset
  \and \\
  \freenames{x?(y).P} := \{ x \} \cup (\freenames{P} \setminus \{ y \})
  \and 
  \freenames{x!\langle P \rangle} := \{ x \} \cup \{ P \} 
  \and \\
  \freenames{P|Q} := \freenames{P} \cup \freenames{Q}
  \and \\
  \freenames{@{x}} := \{ x \}
\end{mathpar}

$\pi$
$\quotep{\pi}$

$\freenames{-} : \pi \to \mathcal{P}(\quotep{\pi})$

\begin{eqnarray*}
  \freenames{\pzero} & := & \emptyset \\
  \freenames{x?(y).P} & := & \{ x \} \cup (\freenames{P} \setminus \{ y \}) \\
  \freenames{x!\langle P \rangle} & := & \{ x \} \cup \{ P \} \\
  \freenames{P|Q} & := & \freenames{P} \cup \freenames{Q} \\
  \freenames{\dropn{x}} & := & \{ x \}
\end{eqnarray*}

The bound names of a process, $\boundnames{P}$, are those names occurring in $P$
that are not free. For example, in $x?(y).0$, the name $x$ is free, while $y$ is bound.

\begin{mathpar}
  \inferrule* [lab=monoidal-laws] {} { P|Q \equiv Q|P \and P|0 \equiv P \and P|(Q|R) \equiv (P|Q)|R }
\end{mathpar}

\begin{mathpar}
  \inferrule* [lab=alpha-equivalence] {} { (x)P \equiv (y)P\{y/x\} \and y \not\in \freenames{P} }
\end{mathpar}

\begin{definition}
Then two processes, $P,Q$, are alpha-equivalent if $P = Q\{\vec{y}/\vec{x}\}$ for
some $\vec{x} \in \boundnames{Q},\vec{y} \in \boundnames{P}$, where $Q\{\vec{y}/\vec{x}\}$
denotes the capture-avoiding substitution of $\vec{y}$ for $\vec{x}$ in $Q$.
\end{definition}

\begin{definition}
  The {\em structural congruence} \cite{SangiorgiWalker} , $\equiv$,
  between processes is the least congruence containing
  alpha-equivalence, satisfying the abelian monoid laws
  (associativity, commutativity and $\pzero$ as identity) for parallel
  composition $|$ and for summation $+$.
\end{definition}

\subsection{Name equivalence}

We take name equivalence, written $\nameeq$, to be the smallest
equivalence relation generated by the following rules.

\begin{mathpar}
\inferrule*[lab=Quote-drop]
{ }
{ \quotep{@{x}} \nameeq x }

\inferrule*[lab=Struct-equiv]
{ P \scong Q }
{ \quotep{P} \nameeq \quotep{Q} }
\end{mathpar}

The astute reader will have noticed that the mutual recursion of names
and processes imposes a mutual recursion on alpha-equivalence and
structural equivalence via name-equivalence. Fortunately, all of this
works out pleasantly and we may calculate in the natural way, free of
concern. The reader interested in the details is referred to the
appendix \ref{appendix:rho_details}.

\subsection{Substitution}

We use $\Proc$ for the set of processes, $\QProc$ for the set of
names, and $\id{\{}\vec{y} / \vec{x} \id{\}}$ to denote partial maps,
$s : \QProc \rightarrow \QProc$. A map, $s$ lifts, uniquely, to a map
on process terms, $\widehat{s} : \Proc \rightarrow \Proc$ by the
following equations.

\begin{mathpar}
  (0) \psubstp{Q}{P} := 0 \\
  (R \juxtap S) \psubstp{Q}{P}
  :=    
  (R)\psubstp{Q}{P} \juxtap (S) \psubstp{Q}{P} \\
  (x?(y).R) \psubstp{Q}{P}    
  :=    
  (x)\substp{Q}{P} (z)\concat( (R \psubstn{z}{y}) \psubstp{Q}{P} ) \\
  (\lift{x}{R}) \psubstp{Q}{P}  
  :=
  \lift{(x)\substp{Q}{P}}{ R \psubstp{Q}{P} } \\
%   (\dropn{x})  \psubstp{Q}{P}       
%   := 
%   \left\{ 
%     \begin{array}{ccc} 
%       \dropn{\quotep{Q}} & & x \nameeq \quotep{P} \\
%       \dropn{x} & & otherwise \\
%     \end{array}
%   \right. 
  (\dropn{x})  \psubstp{Q}{P}       
  := 
  \left\{ 
    \begin{array}{ccc} 
      Q & & x \nameeq \quotep{P} \\
      \dropn{x} & & otherwise \\
    \end{array}
  \right.
\end{mathpar}
 

where

\begin{eqnarray}
  (x)\id{\{} \lpquote Q \rpquote / \lpquote P \rpquote \id{\}}            = 
  \left\{ 
    \begin{array}{ccc}
      \lpquote Q \rpquote & & x \nameeq \lpquote P \rpquote \\
      x & & otherwise \\
    \end{array}
  \right. \nonumber
\end{eqnarray}

and $z$ is chosen distinct from $\quotep{P}$, $\quotep{Q}$, the free
names in $Q$, and all the names in $R$. Our $\alpha$-equivalence will
be built in the standard way from this substitution.

\begin{remark}\label{rem:no_self_referential_names}
  One consequence of these definitions is that $\forall P. \quotep{P}
  \not\in \freenames{P}$.
\end{remark}

\subsection{ Dynamic quote: an example }

Anticipating something of what's to come, consider applying the
substitution, $\widehat{\id{\{}u / z \id{\}}}$, to the following pair
of processes, $\lift{w}{y!(z)}$ and $w[ \lpquote y!(z) \rpquote ]$.

\begin{eqnarray}
	\lift{w}{y!(z)}\widehat{\id{\{}u / z \id{\}}}
		& = &
		\lift{w}{y!(u)} \nonumber\\
	w[ \lpquote y!(z) \rpquote ] \widehat{ \id{\{}u / z \id{\}} }
		& = &
		w[ \lpquote y!(z) \rpquote ] \nonumber
\end{eqnarray}

Because the body of the process between quotes is impervious to
substitution, we get radically different answers. In fact, by
examining the first process in an input context,
e.g. $x?(z).\lift{w}{y!(z)}$, we see that the process under the lift
operator may be shaped by prefixed inputs binding a name inside it. In
this sense, the lift operator will be seen as a way to dynamically
construct processes before reifying them as names.

Finally equipped with these standard features we can present the
dynamics of the calculus.

\subsubsection{Operational semantics} 

Finally, we introduce the computational dynamics. What marks these
algebras as distinct from other more traditionally studied algebraic
structures, e.g. vector spaces or polynomial rings, is the manner in
which dynamics is captured. In traditional structures, dynamics is typically
expressed through morphisms between such structures, as in linear maps
between vector spaces or morphisms between rings. In algebras
associated with the semantics of computation, the dynamics is
expressed as part of the algebraic structure itself, through a
reduction reduction relation typically denoted by $\red$. Below, we
give a recursive presentation of this relation for the calculus used
in the encoding.

$\red \subseteq \pi \times \pi$
$\red : \pi \to \mathcal{P}(\pi)$

\begin{mathpar}
  \inferrule* [lab=Comm] { \textsf{match}( x_{src}, x_{trgt} ) } { x_{trgt}?(y)P \; | \; x_{src}!\langle {Q} \rangle \red P\{\quotep{Q}/y}\} }
  \and \\
  \inferrule* [lab=Par] {{P} \red {P}'} {{{P} | {Q}} \red {{P}' | {Q}}}
  \and
  \inferrule* [lab=Equiv]{{{P} \scong {P}'} \andalso {{P}' \red {Q}'} \andalso {{Q}' \scong {Q}}}{{P} \red {Q}}
\end{mathpar}

\begin{eqnarray*}
  match_{\equiv} (\quotep{P},\quotep{Q}) & := & P \equiv Q \\
  match_{\dagger}(\quotep{P},\quotep{Q}) & := & \forall R. P|Q \red^{*} R => R \red^{*} 0 \\
  match_{K}(\quotep{P},\quotep{Q}) & := & K \mbox{ for some context } K
\end{eqnarray*}

$u?(x)P | u!\langle Q \rangle \red P\{\quotep{Q}/x\}$

%We write $\wred$ for $\red^*$, and $P\red$ if $\exists Q $ such that $ P \red Q$.
We write $P\red$ if $\exists Q $ such that $ P \red Q$ and $P\not\red$, otherwise.

\section{Replication}

As mentioned before, it is known that replication (and hence
recursion) can be implemented in a higher-order process algebra
\cite{SangiorgiWalker}. As our first example of calculation with the
machinery thus far presented we give the construction explicitly in
the {\rhoc}.

\begin{eqnarray}
	D_{x} & := & \prefix{x}{y}{(\binpar{\outputp{x}{y}}{@{y}})} \nonumber\\
	\bangp_{x}{P} & := & \binpar{{x}!\langle{\binpar{D_{x}}{P}}\rangle}{D_{x}} \nonumber
\end{eqnarray}

\begin{eqnarray}
	\bangp_{x}{P} & & \nonumber\\
	=
	& {x}!\langle{(\prefix{x}{y}{(\outputp{x}{y} | @{y})) | P}}\rangle 
	      | \prefix{x}{y}{(\outputp{x}{y} | @{y})} & \nonumber\\
	\red
	& (\outputp{x}{y} | @{y})\substn{\quotep{(\prefix{x}{y}{(@{y} | \outputp{x}{y})) | P}}}{y} & \nonumber\\
	=
	& \outputp{x}{\quotep{(\prefix{x}{y}{(\outputp{x}{y} | @{y})) | P}}}
	  | {(\prefix{x}{y}{(\outputp{x}{y} | @{y})) | P}} & \nonumber\\
	\red
	& \ldots & \nonumber\\
	\red^*
	& P | P | \ldots & \nonumber
\end{eqnarray}

Of course, this encoding, as an implementation, runs away, unfolding
$\bangp{P}$ eagerly. A lazier and more implementable replication
operator, restricted to input-guarded processes, may be obtained as follows.

\begin{eqnarray}
\bangp{\prefix{u}{v}{P}} 
	:= 
	\binpar{\lift{x}{\prefix{u}{v}{(\binpar{D(x)}{P})}}}{D(x)} \nonumber
\end{eqnarray}

\begin{remark}
  Note that the lazier definition still does not deal with summation
  or mixed summation (i.e. sums over input and output). The reader is
  invited to construct definitions of replication that deal with these
  features. 

  Further, the definitions are parameterized in a name, $x$. Can you,
  gentle reader, make a definition that eliminates this parameter and
  guarantees no accidental interaction between the replication
  machinery and the process being replicated -- i.e. no accidental
  sharing of names used by the process to get its work done and the
  name(s) used by the replication to effect copying. This latter
  revision of the definition of replication is crucial to obtaining
  the expected identity $!!P \sim !P$.
\end{remark}

\begin{remark}\label{rem:paradoxical_combinator}
  The reader familiar with the lambda calculus will have noticed the
  similarity between $D$ and the paradoxical combinator.

  [Ed. note: the existence of this seems to suggest we have to be more
  restrictive on the set of processes and names we admit if we are to
  support no-cloning.]
\end{remark}

\subsubsection{Bisimulation}

The computational dynamics gives rise to another kind of equivalence,
the equivalence of computational behavior. As previously mentioned
this is typically captured \emph{via} some form of bisimulation.

% The notion we use in this paper is weak barbed bisimulation
% \cite{milner91polyadicpi}.

The notion we use in this paper is derived from weak barbed
bisimulation \cite{milner91polyadicpi}. 

\begin{definition}
An \emph{observation relation}, $\downarrow_{\mathcal N}$, over a set
of names, $\mathcal N$, is the smallest relation satisfying the rules
below.

\infrule[Out-barb]{y \in {\mathcal N}, \; x \nameeq y}
		  {\outputp{x}{v} \downarrow_{\mathcal N} x}
\infrule[Par-barb]{\mbox{$P\downarrow_{\mathcal N} x$ or $Q\downarrow_{\mathcal N} x$}}
		  {\binpar{P}{Q} \downarrow_{\mathcal N} x}

We write $P \Downarrow_{\mathcal N} x$ if there is $Q$ such that 
$P \wred Q$ and $Q \downarrow_{\mathcal N} x$.
\end{definition}

\begin{definition}
%\label{def.bbisim}
An  ${\mathcal N}$-\emph{barbed bisimulation} over a set of names, ${\mathcal N}$, is a symmetric binary relation 
${\mathcal S}_{\mathcal N}$ between agents such that $P\rel{S}_{\mathcal N}Q$ implies:
\begin{enumerate}
\item If $P \red P'$ then $Q \wred Q'$ and $P'\rel{S}_{\mathcal N} Q'$.
\item If $P\downarrow_{\mathcal N} x$, then $Q\Downarrow_{\mathcal N} x$.
\end{enumerate}
$P$ is ${\mathcal N}$-barbed bisimilar to $Q$, written
$P \wbbisim_{\mathcal N} Q$, if $P \rel{S}_{\mathcal N} Q$ for some ${\mathcal N}$-barbed bisimulation ${\mathcal S}_{\mathcal N}$.
\end{definition}

$\mathcal{R} \subseteq \pi \times \pi$

$P \mathcal{R} Q => \forall P'. P \red P' \Rightarrow \exists Q'. Q \red Q', P' \mathcal{R} Q'$

$P \vdash x \Rightarrow Q \vdash x$

\begin{mathpar}
  \inferrule*[lab=Out-barb]{x \nameeq y}{{y}!\langle{Q}\rangle \vdash x}
  \and
  \inferrule*[lab=Par-barb]{\mbox{$P\vdash x$ or $Q\vdash x$}}{\binpar{P}{Q} \vdash x}
\end{mathpar}

\subsubsection{Contexts}

One of the principle advantages of computational calculi like the
$\pi$-calculus is a well-defined notion of context,
contextual-equivalence and a correlation between
contextual-equivalence and notions of bisimulation. The notion of
context allows the decomposition of a process into (sub-)process and
its syntactic environment, its context. Thus, a context may be
thought of as a process with a ``hole'' (written $\Box$) in it. The
application of a context $M$ to a process $P$, written $M[P]$, is
tantamount to filling the hole in $M$ with $P$. In this paper we do
not need the full weight of this theory, but do make use of the notion
of context in the proof the main theorem. 

\begin{mathpar}
  \inferrule* [lab=summation] {} {{M_{M},M_{N}} \bc \Box \;|\; x.M_{A} \;|\; M_{M}+M_{N}}
  \and
  \inferrule* [lab=agent] {} {{M_{A}} \bc (\vec{x})M_{P} \;| \; \clift{P_0,\ldots,M_{P},\ldots,P_N}}
  \and \\
  \inferrule* [lab=process] {} {{M_{P}} \bc M_{N} \;| \;P|M_{P} }
\end{mathpar} 

\begin{mathpar}
  \inferrule* [lab=sychronization] {} {M_{N} \bc \Box \;|\; x?M_{F} \;|\; x!M_{C}}
  \and
  \inferrule* [lab=abstraction] {} {{M_{F}} \bc (x)M_{P} }
  \and
  \inferrule* [lab=concretion] {} {{M_{C}} \bc \langle M_{P} \rangle }
  \and \\
  \inferrule* [lab=process] {} {{M_{P}} \bc M_{N} \;| \;P|M_{P} }
\end{mathpar}

\begin{definition}[contextual application] Given a context $M$, and
  process $P$, we define the \emph{contextual application}, $M[P] :=
  M\{P/\Box\}$. That is, the contextual application of M to P is the
  substitution of $P$ for $\Box$ in $M$.
\end{definition}

$\meaningof{-} : L \to \mathcal{P}(\pi)$

\begin{mathpar}
  \inferrule* [lab=collection] {} {\meaningof{true} = \pi, \and \meaningof{~E} = \pi \setminus \meaningof{E}, \and \meaningof{E_{1} \& E_{2}} = \meaningof{E_{1}} \cap \meaningof{E_{2}}}
\end{mathpar}

\begin{mathpar}
  \inferrule* [lab=structure] {} {\meaningof{0} = \{ P \in \pi | P \equiv 0 \}, \and \\ \meaningof{E_1 | E_2} = \{ P \in \pi | P \equiv P_{1} | P_{2}, P_{1} \in \meaningof{E_{1}}, P_{2} \in \meaningof{E_2}\} }
\end{mathpar}

\begin{mathpar}
 \inferrule* [lab=behavior] {} {\meaningof{\langle a?b \rangle E} = \{ P \in \pi | P \equiv Q | u?(y)P', \\ \and \\\\ \and \\ \;\;\; u \in \meaningof{a}, \forall z.P'\{z/y\} \in \meaningof{E\{z/b\}}\}, \and \\ \meaningof{a!E} = \{ P \in \pi | P \equiv Q | x!\langle P' \rangle, x \in \meaningof{a} P' \in \meaningof{E}\} }
\end{mathpar}

\begin{mathpar}
 \inferrule* [lab=nominal] {} {\meaningof{\quotep{E}} = \{ \quotep{P} \in \quotep{\pi} | P \in \meaningof{E} \}, \and \meaningof{\quotep{P}} = \{ \quotep{Q} \in \quotep{\pi} | P \equiv Q \} \and \\ \meaningof{@\quotep{E}} = \{ P \in \pi | P \equiv @x, x \in \meaningof{E} \}}
\end{mathpar}

\begin{eqnarray*}
  \\
  \meaningof{-} : TS \to ST
\end{eqnarray*}

\begin{eqnarray*}
  \\
  L : TS \to ST
\end{eqnarray*}

\begin{eqnarray*}
  \\
  P \models E \iff P \in \meaningof{E}
\end{eqnarray*}

\begin{eqnarray*}
  P \approx_{L} Q \iff \forall E \in L. P \models E \iff Q \models E
\end{eqnarray*}

\begin{eqnarray*}
  P \approx_{K} Q
\end{eqnarray*}

\begin{eqnarray*}
  P \approx Q
\end{eqnarray*}

$\approx_{K} = \approx = \approx_{L}$

\subsubsection{Contextual duality}

Note that contexts extend the quotation operation to a family of
operations from processes to names. Given a context, $M$, we can
define a \emph{nominal context}, $\quotep{M}$ by $\quotep{M}[P] :=
\quotep{M[P]}$. To foreshadow what is to come we observe that these
operations enjoy a duality with processes very much like the duality
between vectors and maps from vectors to scalars.

Further, because the calculus is essentially higher-order, we have a
correspondence between contexts and processes. More specifically,
given a name $x$ and a context $M$ we can construct $M^{*}_{x}$ such
that 

\begin{mathpar}
  M^{*}_{x} | \lift{x}{P} \red M[P]
\end{mathpar}

namely,

\begin{mathpar}
  M^{*}_{x} := x?(u).M[\dropn{u}]
\end{mathpar}

The dependence of $M^{*}_{x}$ on a name makes it an abstraction, 

\begin{mathpar}
  M^{*} := (x)x?(u).M[\dropn{u}]
\end{mathpar}

\subsection{Additional notation}

It will sometimes be convenient to denote the process a name
quotes. We already have the notation $x = \quotep{P}$, but it will be
convenient to introduce an alternate notation, $\procn{x}$, when we
want to emphasize the connection to the use of the name. Note that, by
virtue of name equivalence, $\quotep{\procn{x}} \nameeq x$; so, the
notation is consistent with previous definitions.

Further, because names have structure it is possible to effect
substitutions on the basis of that structure. This means we need to
upgrade our notation for substitutions, which we accomplish by
adapting comprehension notation. Thus,

\begin{mathpar}
  P\{ y / x : x \in S \}
\end{mathpar}

is interpreted to mean the process derived from P by replacing (in a
capture-avoiding manner) each occurrence of $x$ in $S$ by $y$. For example,

\begin{mathpar}
  P\{ \quotep{\procn{x}|\procn{x}} / x : x \in \freenames{P} \}
\end{mathpar}

will replace each (occurrence) of a free name $x$ in $P$ by
$\quotep{\procn{x}|\procn{x}}$.

Also, we will avail ourselves of the notation $x^{L}$ and $x^{R}$ to
denote injections of a name into disjoint copies of the name
space. There are numerous ways to accomplish this. One example can be
found in \cite{MeredithR05}. This notation overloads to vectors of
names: $\vec{x}^{\pi} := (x_{i}^{\pi} \; : \; 0 \leq i < |\vec{x}| )$ where $\pi \in \{L,R\}$.

We also use $P^{\Box} := P|\Box$.

In \cite{MeredithR05} an interpretation of the new operator is
given. It turns out that there are several possible interpretations
all enjoying the requisite algebraic properties of the operator (see
\cite{milner91polyadicpi}). We will therefore make liberal use of
$(\nu\; \vec{x})P$.

% subsection the_syntax_and_semantics_of_the_notation_system (end)   

\input{qm2pi.qmops} 

\input{qm2pi.sterngerlach} 

\input{qm2pi.metric} 

% section concurrent_process_calculi (end)

%\input{qm2pi.proofsketch}

% section proof sketch (end)

%\input{qm2pi.slviaknots} 

% section spatial logic via knots (end)

\input{qm2pi.conclusion}

% section conclusion (end)

%\input{qm2pi.dtcodes} 

% section wiring algorithm (end)

\input{qm2pi.ack} 

% section acknowledgments (end)

\newpage


\bibliographystyle{plain}   
\bibliography{../../biblios/main.bib}

\input{qm2pi.rhodetails}

\end{document}



\end{document}



% section front matter (end)

\section{Introduction}\label{sec:introduction} % (fold)
In this draft of the material i am going to have to dispense with the
usual writing conventions adopted in papers on these topics. i'm going
to have adopt whatever tone i need at the time i'm writing up the
calculations. Sometimes this may be very conversational; others it may
be the barest mathematical grunts; others still it may be that i have
lifted text from one of my other papers because the exposition of some
point was better said there. i hope that my readers are not unduly put
out by this decision. i'm not doing this to flout convention or be
rebellious. i find these calculations very technically challenging. To
keep everything going technically, something has to give; i have to
let go of some cognitive burden. So, the academic writing style --
with all of its trade-offs in terms of facilitating technical
communication -- is what i'm letting go of. Perhaps subsequent drafts
can be tightened and polished, but for now, i'm going to speak as if
we were sitting together in a coffee shop with a laptop, wifi and a
pad of paper and a pencil.

So, here's what i have to say. We -- you and i, comfortably ensconced
in our coffee shop and well-equipped with our tools -- can realize and
carry out the calculations of quantum mechanics over a very different
formal theory of dynamics, a formal theory of dynamics that
corresponds to a theory of concurrent computation with
\emph{reflection}. It has the advantage that the underlying theory is
already `quantized', but supports analogues all of the continuuous
operations. Strikingly, this underlying theory has recently been
connected with a notion of metric that we can show, by calculating
together, coincides with the metric induced by the inner product.

There are a lot of reasons why you might be interested in seeing
calculations of this form. Here's why i'm interested. For the past
several centuries there has been no competitor to the ``Newtonian''
account of dynamics. As a result the predominant share of accounts of
dynamical systems and situations have had to be formulated in terms of
the Newtonian machinery. i view this as an intellectually dangerous
position to occupy. Everything, despite it's intrinsic shape, turns
into a nail to be hit with this hammer. Recently, however, the theory
of computation has matured to the point where we have candidates for
theories of dynamics that offer very different perspective on
reasoning about dynamical systems and situations. Testing these
candidates against very successful accounts of dynamical situations,
like quantum mechanics, is going to give us some sense of how mature
they are and some measure of the quality of these accounts of
dynamics.

\subsection{Summary of contributions and outline of paper}

So, we're going to develop an interpretation of the operations of
quantum mechanics normally interpreted by Hilbert spaces and
operators. We're going to do this over a theory of computation. Note
that this is very different than the usual quantum computation program
which develops notions of computation over quantum mechanics. Rather,
we are developing a story that aligns with Wheeler's slogan: It from
Bit. To do this we will first provide an account of the theory of
computation at play here. Then we will dive into a calculation-driven
interpretation of the operations of quantum mechanics.

The reason we take this approach is that -- until very recently --
there hasn't been an axiomatic account of quantum mechanics. As a
result there has been no sharp delineation of the mathematical theory
supporting interpretation of the physical theory and the physical
theory, itself. So, ambient features of the maths are free to be
exploited (or supressed) without a real accounting of their physical
relevance. There is no sharp statement ``here's the physical theory''
qua \emph{theory} and ``here's the mathematical interpretation''
enabling a judgment of how faithful the interpretation is -- apart
from experimental observation. When there is an axiomatic account we
can judge how well a given mathematical formalism supports an
interpretation of the axioms, independent of
experimentation. Likewise, we can judge how well we have captured our
physical evidence and experience with our axiomatics, independent of
any specific mathematical implementation, with accidental detail that
may or may not have physical significance. 

In lieu of a fully fleshed out and vetted axiomatic account of quantum
mechanics, interpreting the operational notions in service of modeling
physical systems will have to suffice. In other words, we are not in
the business of providing a model of Hilbert spaces and operators. We
are in the business of providing a model of quantum mechanics because
we are motivated by testing our notions of dynamics against physical
theory; and, the predictive calculations of the physical theory must
serve as the best formulation -- shy of a fully fleshed out axiomatic
account -- of the physical theory itself (as they have for scientific
theories since time immemorial). Put another way, despite a
whole-hearted commitment to an It-from-Bit ontology, we are firmly
aligned with the shut-up-and-calculate camp as the best way to obtain
results either from the physical perspective or as a quality assurance
measure of our fledgling theory of dynamics.

In detail, we present a reflective process calculus. Then we develop
intuitive correspondences between the notions available in this
calculus and the usual physical notions supporting quantum mechanical
calculations. Thus, 

\begin{table}[htp]
  \center{
    \fbox{
      \begin{tabular}{c|c}
        quantum mechanics & process calculus \\
        \hline
        scalar & name \\
        state vector & process \\
        dual & contextual duals \\
        matrix & formal sums of process-context-dual pairs \\
        orthogonality & process annihilation \\
        inner product & execution-formula + quoting
      \end{tabular}
    }
  }
  \caption{QM - process calculi correspondences}
\end{table}

Then we tighten up these intuitions to operational definitions. We
employ the Dirac notation as the best proxy we can find for an
abstract syntax of the quantum mechanical notions. The definitions we
develop put us in contact with equational constraints coming from the
theory that we demonstrate the definitions and calculations satisfy.

This puts us in a position to shut up and calculate for the
Stern-Gerlach experimental set up, showing how these predictive
calculations become calculations on processes in our theory of a
reflective process calculus.

Penultimately, we demonstrate that the notion of metric coming from
the inner product coincides with the notion of metric available from
the theory of bisimulation. This demonstration gives us the right to
think of space as arising from behavior. Finally, we consider where we
might go from the new vantage point we have obtained.

% section introduction (end) 
 
% section introduction (end)

% \documentclass[12pt]{llncs}
%\documentclass{jktr}

\usepackage[pdftex]{hyperref}                   
\usepackage {listings}
\usepackage {mathpartir}
\usepackage{bcprules}
%\usepackage{listings}
                       
\usepackage{graphicx} 
%\usepackage[margins=2.5cm,nohead,nofoot]{geometry}
%\usepackage{geometry}
\usepackage{amsfonts}
\usepackage{amstext}
\usepackage{latexsym}
\usepackage{amssymb}
\usepackage{color}


%\include{myPreamble}
\documentclass[12pt]{llncs}
%\documentclass{jktr}

\usepackage[pdftex]{hyperref}                   
\usepackage {listings}
\usepackage {mathpartir}
\usepackage{bcprules}
%\usepackage{listings}
                       
\usepackage{graphicx} 
%\usepackage[margins=2.5cm,nohead,nofoot]{geometry}
%\usepackage{geometry}
\usepackage{amsfonts}
\usepackage{amstext}
\usepackage{latexsym}
\usepackage{amssymb}
\usepackage{color}


%\include{myPreamble}
\include{qm2pi.local} 

%\ifpdf
%\usepackage[pdftex]{graphicx}
%\else
%\usepackage{graphicx}
%\fi

 % \ifpdf
%  \usepackage{pdfsync}
%  \if


%\title{Brief Article}
%\author{David F. Snyder}
%\author{L.G. Meredith}

%\address{Dept. of Math., Texas State University--San Marcos, San Marcos, TX 78666}
       
\pagestyle{empty}


\begin{document}

\lstset{language=[Objective]Caml,frame=shadowbox}

\input{qm2pi.front}

% section front matter (end)

\input{qm2pi.intro} 
 
% section introduction (end)

% \input{qm2pi.knotations} 

% section notation (end)

\input{qm2pi.process.calculi} 

% section concurrent_process_calculi_and_spatial_logics_ (end)
    
%\input{qm2pi.knots2pi} 

%\input{qm2pi.trefoil} 

%\input{qm2pi.mainthm} 

% subsection basic_interpretation (end)

%\input{qm2pi.rho.presentation} 
\subsection{The syntax and semantics of the notation system}\label{sub:the_syntax_and_semantics_of_the_notation_system} % (fold)

We now summarize a technical presentation of the calculus that
embodies our theory of dynamics. The typical presentation of such a
calculus follows the style of giving generators and relations on
them. The grammar, below, describing term constructors, freely
generates the set of processes, $\Proc$. This set is then quotiented
by a relation known as structural congruence and it is over this set
that the notion of dynamics is expressed. This presentation is
essentially that of \cite{MeredithR05} with the addition of
polyadicity and summation. For readability we have relegated some of
the technical subtleties to an appendix.

\subsubsection{Process grammar}\label{subsub:process_grammar}

\begin{mathpar}
  \inferrule* [lab=synchronization] {} {{M} \bc \pzero \;|\; x?F \;|\; x!C }
  \and
  \inferrule* [lab=abstraction] {} {{F} \bc (x)P}
  \and
  \inferrule* [lab=concretion] {} {{C} \bc \langle Q \rangle}
  \and
  \inferrule* [lab=process] {} {{P,Q} \bc M \;| \;P|Q \;|\; @{x}}
  \and
  \inferrule* [lab=name] {} {{x} \bc \quotep{P}}
\end{mathpar} 

Note that $\vec{x}$ (resp. $\vec{P}$) denotes a vector of names
(resp. processes) of length $|\vec{x}|$ (resp. $|\vec{P}|$). We adopt
the following useful abbreviations.

\begin{mathpar}
   x?(\vec{y}).P := x.(\vec{y})P \and  x\clift{\vec{P}} := x.\clift{\vec{P}}
   \and x!(y) := \lift{x}{\dropn{y}}
   \and \Pi_{i=0}^{n-1}P_i := P_0 | \ldots | P_{n-1}
\end{mathpar}

\subsubsection{Structural congruence}

\paragraph{Free and bound names and alpha-equivalence.} At the
core of structural equivalence is alpha-equivalence which identifies
process that are the same up to a change of variable. Formally, we
recognize the distinction between free and bound names. The free names
of a process, $\freenames{P}$, may be calculated recursively as
follows:

\begin{mathpar}
\freenames{\pzero} := \emptyset
  \and \\
  \freenames{x?(y).P} := \{ x \} \cup (\freenames{P} \setminus \{ y \})
  \and 
  \freenames{x!\langle P \rangle} := \{ x \} \cup \{ P \} 
  \and \\
  \freenames{P|Q} := \freenames{P} \cup \freenames{Q}
  \and \\
  \freenames{@{x}} := \{ x \}
\end{mathpar}

$\pi$
$\quotep{\pi}$

$\freenames{-} : \pi \to \mathcal{P}(\quotep{\pi})$

\begin{eqnarray*}
  \freenames{\pzero} & := & \emptyset \\
  \freenames{x?(y).P} & := & \{ x \} \cup (\freenames{P} \setminus \{ y \}) \\
  \freenames{x!\langle P \rangle} & := & \{ x \} \cup \{ P \} \\
  \freenames{P|Q} & := & \freenames{P} \cup \freenames{Q} \\
  \freenames{\dropn{x}} & := & \{ x \}
\end{eqnarray*}

The bound names of a process, $\boundnames{P}$, are those names occurring in $P$
that are not free. For example, in $x?(y).0$, the name $x$ is free, while $y$ is bound.

\begin{mathpar}
  \inferrule* [lab=monoidal-laws] {} { P|Q \equiv Q|P \and P|0 \equiv P \and P|(Q|R) \equiv (P|Q)|R }
\end{mathpar}

\begin{mathpar}
  \inferrule* [lab=alpha-equivalence] {} { (x)P \equiv (y)P\{y/x\} \and y \not\in \freenames{P} }
\end{mathpar}

\begin{definition}
Then two processes, $P,Q$, are alpha-equivalent if $P = Q\{\vec{y}/\vec{x}\}$ for
some $\vec{x} \in \boundnames{Q},\vec{y} \in \boundnames{P}$, where $Q\{\vec{y}/\vec{x}\}$
denotes the capture-avoiding substitution of $\vec{y}$ for $\vec{x}$ in $Q$.
\end{definition}

\begin{definition}
  The {\em structural congruence} \cite{SangiorgiWalker} , $\equiv$,
  between processes is the least congruence containing
  alpha-equivalence, satisfying the abelian monoid laws
  (associativity, commutativity and $\pzero$ as identity) for parallel
  composition $|$ and for summation $+$.
\end{definition}

\subsection{Name equivalence}

We take name equivalence, written $\nameeq$, to be the smallest
equivalence relation generated by the following rules.

\begin{mathpar}
\inferrule*[lab=Quote-drop]
{ }
{ \quotep{@{x}} \nameeq x }

\inferrule*[lab=Struct-equiv]
{ P \scong Q }
{ \quotep{P} \nameeq \quotep{Q} }
\end{mathpar}

The astute reader will have noticed that the mutual recursion of names
and processes imposes a mutual recursion on alpha-equivalence and
structural equivalence via name-equivalence. Fortunately, all of this
works out pleasantly and we may calculate in the natural way, free of
concern. The reader interested in the details is referred to the
appendix \ref{appendix:rho_details}.

\subsection{Substitution}

We use $\Proc$ for the set of processes, $\QProc$ for the set of
names, and $\id{\{}\vec{y} / \vec{x} \id{\}}$ to denote partial maps,
$s : \QProc \rightarrow \QProc$. A map, $s$ lifts, uniquely, to a map
on process terms, $\widehat{s} : \Proc \rightarrow \Proc$ by the
following equations.

\begin{mathpar}
  (0) \psubstp{Q}{P} := 0 \\
  (R \juxtap S) \psubstp{Q}{P}
  :=    
  (R)\psubstp{Q}{P} \juxtap (S) \psubstp{Q}{P} \\
  (x?(y).R) \psubstp{Q}{P}    
  :=    
  (x)\substp{Q}{P} (z)\concat( (R \psubstn{z}{y}) \psubstp{Q}{P} ) \\
  (\lift{x}{R}) \psubstp{Q}{P}  
  :=
  \lift{(x)\substp{Q}{P}}{ R \psubstp{Q}{P} } \\
%   (\dropn{x})  \psubstp{Q}{P}       
%   := 
%   \left\{ 
%     \begin{array}{ccc} 
%       \dropn{\quotep{Q}} & & x \nameeq \quotep{P} \\
%       \dropn{x} & & otherwise \\
%     \end{array}
%   \right. 
  (\dropn{x})  \psubstp{Q}{P}       
  := 
  \left\{ 
    \begin{array}{ccc} 
      Q & & x \nameeq \quotep{P} \\
      \dropn{x} & & otherwise \\
    \end{array}
  \right.
\end{mathpar}
 

where

\begin{eqnarray}
  (x)\id{\{} \lpquote Q \rpquote / \lpquote P \rpquote \id{\}}            = 
  \left\{ 
    \begin{array}{ccc}
      \lpquote Q \rpquote & & x \nameeq \lpquote P \rpquote \\
      x & & otherwise \\
    \end{array}
  \right. \nonumber
\end{eqnarray}

and $z$ is chosen distinct from $\quotep{P}$, $\quotep{Q}$, the free
names in $Q$, and all the names in $R$. Our $\alpha$-equivalence will
be built in the standard way from this substitution.

\begin{remark}\label{rem:no_self_referential_names}
  One consequence of these definitions is that $\forall P. \quotep{P}
  \not\in \freenames{P}$.
\end{remark}

\subsection{ Dynamic quote: an example }

Anticipating something of what's to come, consider applying the
substitution, $\widehat{\id{\{}u / z \id{\}}}$, to the following pair
of processes, $\lift{w}{y!(z)}$ and $w[ \lpquote y!(z) \rpquote ]$.

\begin{eqnarray}
	\lift{w}{y!(z)}\widehat{\id{\{}u / z \id{\}}}
		& = &
		\lift{w}{y!(u)} \nonumber\\
	w[ \lpquote y!(z) \rpquote ] \widehat{ \id{\{}u / z \id{\}} }
		& = &
		w[ \lpquote y!(z) \rpquote ] \nonumber
\end{eqnarray}

Because the body of the process between quotes is impervious to
substitution, we get radically different answers. In fact, by
examining the first process in an input context,
e.g. $x?(z).\lift{w}{y!(z)}$, we see that the process under the lift
operator may be shaped by prefixed inputs binding a name inside it. In
this sense, the lift operator will be seen as a way to dynamically
construct processes before reifying them as names.

Finally equipped with these standard features we can present the
dynamics of the calculus.

\subsubsection{Operational semantics} 

Finally, we introduce the computational dynamics. What marks these
algebras as distinct from other more traditionally studied algebraic
structures, e.g. vector spaces or polynomial rings, is the manner in
which dynamics is captured. In traditional structures, dynamics is typically
expressed through morphisms between such structures, as in linear maps
between vector spaces or morphisms between rings. In algebras
associated with the semantics of computation, the dynamics is
expressed as part of the algebraic structure itself, through a
reduction reduction relation typically denoted by $\red$. Below, we
give a recursive presentation of this relation for the calculus used
in the encoding.

$\red \subseteq \pi \times \pi$
$\red : \pi \to \mathcal{P}(\pi)$

\begin{mathpar}
  \inferrule* [lab=Comm] { \textsf{match}( x_{src}, x_{trgt} ) } { x_{trgt}?(y)P \; | \; x_{src}!\langle {Q} \rangle \red P\{\quotep{Q}/y}\} }
  \and \\
  \inferrule* [lab=Par] {{P} \red {P}'} {{{P} | {Q}} \red {{P}' | {Q}}}
  \and
  \inferrule* [lab=Equiv]{{{P} \scong {P}'} \andalso {{P}' \red {Q}'} \andalso {{Q}' \scong {Q}}}{{P} \red {Q}}
\end{mathpar}

\begin{eqnarray*}
  match_{\equiv} (\quotep{P},\quotep{Q}) & := & P \equiv Q \\
  match_{\dagger}(\quotep{P},\quotep{Q}) & := & \forall R. P|Q \red^{*} R => R \red^{*} 0 \\
  match_{K}(\quotep{P},\quotep{Q}) & := & K \mbox{ for some context } K
\end{eqnarray*}

$u?(x)P | u!\langle Q \rangle \red P\{\quotep{Q}/x\}$

%We write $\wred$ for $\red^*$, and $P\red$ if $\exists Q $ such that $ P \red Q$.
We write $P\red$ if $\exists Q $ such that $ P \red Q$ and $P\not\red$, otherwise.

\section{Replication}

As mentioned before, it is known that replication (and hence
recursion) can be implemented in a higher-order process algebra
\cite{SangiorgiWalker}. As our first example of calculation with the
machinery thus far presented we give the construction explicitly in
the {\rhoc}.

\begin{eqnarray}
	D_{x} & := & \prefix{x}{y}{(\binpar{\outputp{x}{y}}{@{y}})} \nonumber\\
	\bangp_{x}{P} & := & \binpar{{x}!\langle{\binpar{D_{x}}{P}}\rangle}{D_{x}} \nonumber
\end{eqnarray}

\begin{eqnarray}
	\bangp_{x}{P} & & \nonumber\\
	=
	& {x}!\langle{(\prefix{x}{y}{(\outputp{x}{y} | @{y})) | P}}\rangle 
	      | \prefix{x}{y}{(\outputp{x}{y} | @{y})} & \nonumber\\
	\red
	& (\outputp{x}{y} | @{y})\substn{\quotep{(\prefix{x}{y}{(@{y} | \outputp{x}{y})) | P}}}{y} & \nonumber\\
	=
	& \outputp{x}{\quotep{(\prefix{x}{y}{(\outputp{x}{y} | @{y})) | P}}}
	  | {(\prefix{x}{y}{(\outputp{x}{y} | @{y})) | P}} & \nonumber\\
	\red
	& \ldots & \nonumber\\
	\red^*
	& P | P | \ldots & \nonumber
\end{eqnarray}

Of course, this encoding, as an implementation, runs away, unfolding
$\bangp{P}$ eagerly. A lazier and more implementable replication
operator, restricted to input-guarded processes, may be obtained as follows.

\begin{eqnarray}
\bangp{\prefix{u}{v}{P}} 
	:= 
	\binpar{\lift{x}{\prefix{u}{v}{(\binpar{D(x)}{P})}}}{D(x)} \nonumber
\end{eqnarray}

\begin{remark}
  Note that the lazier definition still does not deal with summation
  or mixed summation (i.e. sums over input and output). The reader is
  invited to construct definitions of replication that deal with these
  features. 

  Further, the definitions are parameterized in a name, $x$. Can you,
  gentle reader, make a definition that eliminates this parameter and
  guarantees no accidental interaction between the replication
  machinery and the process being replicated -- i.e. no accidental
  sharing of names used by the process to get its work done and the
  name(s) used by the replication to effect copying. This latter
  revision of the definition of replication is crucial to obtaining
  the expected identity $!!P \sim !P$.
\end{remark}

\begin{remark}\label{rem:paradoxical_combinator}
  The reader familiar with the lambda calculus will have noticed the
  similarity between $D$ and the paradoxical combinator.

  [Ed. note: the existence of this seems to suggest we have to be more
  restrictive on the set of processes and names we admit if we are to
  support no-cloning.]
\end{remark}

\subsubsection{Bisimulation}

The computational dynamics gives rise to another kind of equivalence,
the equivalence of computational behavior. As previously mentioned
this is typically captured \emph{via} some form of bisimulation.

% The notion we use in this paper is weak barbed bisimulation
% \cite{milner91polyadicpi}.

The notion we use in this paper is derived from weak barbed
bisimulation \cite{milner91polyadicpi}. 

\begin{definition}
An \emph{observation relation}, $\downarrow_{\mathcal N}$, over a set
of names, $\mathcal N$, is the smallest relation satisfying the rules
below.

\infrule[Out-barb]{y \in {\mathcal N}, \; x \nameeq y}
		  {\outputp{x}{v} \downarrow_{\mathcal N} x}
\infrule[Par-barb]{\mbox{$P\downarrow_{\mathcal N} x$ or $Q\downarrow_{\mathcal N} x$}}
		  {\binpar{P}{Q} \downarrow_{\mathcal N} x}

We write $P \Downarrow_{\mathcal N} x$ if there is $Q$ such that 
$P \wred Q$ and $Q \downarrow_{\mathcal N} x$.
\end{definition}

\begin{definition}
%\label{def.bbisim}
An  ${\mathcal N}$-\emph{barbed bisimulation} over a set of names, ${\mathcal N}$, is a symmetric binary relation 
${\mathcal S}_{\mathcal N}$ between agents such that $P\rel{S}_{\mathcal N}Q$ implies:
\begin{enumerate}
\item If $P \red P'$ then $Q \wred Q'$ and $P'\rel{S}_{\mathcal N} Q'$.
\item If $P\downarrow_{\mathcal N} x$, then $Q\Downarrow_{\mathcal N} x$.
\end{enumerate}
$P$ is ${\mathcal N}$-barbed bisimilar to $Q$, written
$P \wbbisim_{\mathcal N} Q$, if $P \rel{S}_{\mathcal N} Q$ for some ${\mathcal N}$-barbed bisimulation ${\mathcal S}_{\mathcal N}$.
\end{definition}

$\mathcal{R} \subseteq \pi \times \pi$

$P \mathcal{R} Q => \forall P'. P \red P' \Rightarrow \exists Q'. Q \red Q', P' \mathcal{R} Q'$

$P \vdash x \Rightarrow Q \vdash x$

\begin{mathpar}
  \inferrule*[lab=Out-barb]{x \nameeq y}{{y}!\langle{Q}\rangle \vdash x}
  \and
  \inferrule*[lab=Par-barb]{\mbox{$P\vdash x$ or $Q\vdash x$}}{\binpar{P}{Q} \vdash x}
\end{mathpar}

\subsubsection{Contexts}

One of the principle advantages of computational calculi like the
$\pi$-calculus is a well-defined notion of context,
contextual-equivalence and a correlation between
contextual-equivalence and notions of bisimulation. The notion of
context allows the decomposition of a process into (sub-)process and
its syntactic environment, its context. Thus, a context may be
thought of as a process with a ``hole'' (written $\Box$) in it. The
application of a context $M$ to a process $P$, written $M[P]$, is
tantamount to filling the hole in $M$ with $P$. In this paper we do
not need the full weight of this theory, but do make use of the notion
of context in the proof the main theorem. 

\begin{mathpar}
  \inferrule* [lab=summation] {} {{M_{M},M_{N}} \bc \Box \;|\; x.M_{A} \;|\; M_{M}+M_{N}}
  \and
  \inferrule* [lab=agent] {} {{M_{A}} \bc (\vec{x})M_{P} \;| \; \clift{P_0,\ldots,M_{P},\ldots,P_N}}
  \and \\
  \inferrule* [lab=process] {} {{M_{P}} \bc M_{N} \;| \;P|M_{P} }
\end{mathpar} 

\begin{mathpar}
  \inferrule* [lab=sychronization] {} {M_{N} \bc \Box \;|\; x?M_{F} \;|\; x!M_{C}}
  \and
  \inferrule* [lab=abstraction] {} {{M_{F}} \bc (x)M_{P} }
  \and
  \inferrule* [lab=concretion] {} {{M_{C}} \bc \langle M_{P} \rangle }
  \and \\
  \inferrule* [lab=process] {} {{M_{P}} \bc M_{N} \;| \;P|M_{P} }
\end{mathpar}

\begin{definition}[contextual application] Given a context $M$, and
  process $P$, we define the \emph{contextual application}, $M[P] :=
  M\{P/\Box\}$. That is, the contextual application of M to P is the
  substitution of $P$ for $\Box$ in $M$.
\end{definition}

$\meaningof{-} : L \to \mathcal{P}(\pi)$

\begin{mathpar}
  \inferrule* [lab=collection] {} {\meaningof{true} = \pi, \and \meaningof{~E} = \pi \setminus \meaningof{E}, \and \meaningof{E_{1} \& E_{2}} = \meaningof{E_{1}} \cap \meaningof{E_{2}}}
\end{mathpar}

\begin{mathpar}
  \inferrule* [lab=structure] {} {\meaningof{0} = \{ P \in \pi | P \equiv 0 \}, \and \\ \meaningof{E_1 | E_2} = \{ P \in \pi | P \equiv P_{1} | P_{2}, P_{1} \in \meaningof{E_{1}}, P_{2} \in \meaningof{E_2}\} }
\end{mathpar}

\begin{mathpar}
 \inferrule* [lab=behavior] {} {\meaningof{\langle a?b \rangle E} = \{ P \in \pi | P \equiv Q | u?(y)P', \\ \and \\\\ \and \\ \;\;\; u \in \meaningof{a}, \forall z.P'\{z/y\} \in \meaningof{E\{z/b\}}\}, \and \\ \meaningof{a!E} = \{ P \in \pi | P \equiv Q | x!\langle P' \rangle, x \in \meaningof{a} P' \in \meaningof{E}\} }
\end{mathpar}

\begin{mathpar}
 \inferrule* [lab=nominal] {} {\meaningof{\quotep{E}} = \{ \quotep{P} \in \quotep{\pi} | P \in \meaningof{E} \}, \and \meaningof{\quotep{P}} = \{ \quotep{Q} \in \quotep{\pi} | P \equiv Q \} \and \\ \meaningof{@\quotep{E}} = \{ P \in \pi | P \equiv @x, x \in \meaningof{E} \}}
\end{mathpar}

\begin{eqnarray*}
  \\
  \meaningof{-} : TS \to ST
\end{eqnarray*}

\begin{eqnarray*}
  \\
  L : TS \to ST
\end{eqnarray*}

\begin{eqnarray*}
  \\
  P \models E \iff P \in \meaningof{E}
\end{eqnarray*}

\begin{eqnarray*}
  P \approx_{L} Q \iff \forall E \in L. P \models E \iff Q \models E
\end{eqnarray*}

\begin{eqnarray*}
  P \approx_{K} Q
\end{eqnarray*}

\begin{eqnarray*}
  P \approx Q
\end{eqnarray*}

$\approx_{K} = \approx = \approx_{L}$

\subsubsection{Contextual duality}

Note that contexts extend the quotation operation to a family of
operations from processes to names. Given a context, $M$, we can
define a \emph{nominal context}, $\quotep{M}$ by $\quotep{M}[P] :=
\quotep{M[P]}$. To foreshadow what is to come we observe that these
operations enjoy a duality with processes very much like the duality
between vectors and maps from vectors to scalars.

Further, because the calculus is essentially higher-order, we have a
correspondence between contexts and processes. More specifically,
given a name $x$ and a context $M$ we can construct $M^{*}_{x}$ such
that 

\begin{mathpar}
  M^{*}_{x} | \lift{x}{P} \red M[P]
\end{mathpar}

namely,

\begin{mathpar}
  M^{*}_{x} := x?(u).M[\dropn{u}]
\end{mathpar}

The dependence of $M^{*}_{x}$ on a name makes it an abstraction, 

\begin{mathpar}
  M^{*} := (x)x?(u).M[\dropn{u}]
\end{mathpar}

\subsection{Additional notation}

It will sometimes be convenient to denote the process a name
quotes. We already have the notation $x = \quotep{P}$, but it will be
convenient to introduce an alternate notation, $\procn{x}$, when we
want to emphasize the connection to the use of the name. Note that, by
virtue of name equivalence, $\quotep{\procn{x}} \nameeq x$; so, the
notation is consistent with previous definitions.

Further, because names have structure it is possible to effect
substitutions on the basis of that structure. This means we need to
upgrade our notation for substitutions, which we accomplish by
adapting comprehension notation. Thus,

\begin{mathpar}
  P\{ y / x : x \in S \}
\end{mathpar}

is interpreted to mean the process derived from P by replacing (in a
capture-avoiding manner) each occurrence of $x$ in $S$ by $y$. For example,

\begin{mathpar}
  P\{ \quotep{\procn{x}|\procn{x}} / x : x \in \freenames{P} \}
\end{mathpar}

will replace each (occurrence) of a free name $x$ in $P$ by
$\quotep{\procn{x}|\procn{x}}$.

Also, we will avail ourselves of the notation $x^{L}$ and $x^{R}$ to
denote injections of a name into disjoint copies of the name
space. There are numerous ways to accomplish this. One example can be
found in \cite{MeredithR05}. This notation overloads to vectors of
names: $\vec{x}^{\pi} := (x_{i}^{\pi} \; : \; 0 \leq i < |\vec{x}| )$ where $\pi \in \{L,R\}$.

We also use $P^{\Box} := P|\Box$.

In \cite{MeredithR05} an interpretation of the new operator is
given. It turns out that there are several possible interpretations
all enjoying the requisite algebraic properties of the operator (see
\cite{milner91polyadicpi}). We will therefore make liberal use of
$(\nu\; \vec{x})P$.

% subsection the_syntax_and_semantics_of_the_notation_system (end)   

\input{qm2pi.qmops} 

\input{qm2pi.sterngerlach} 

\input{qm2pi.metric} 

% section concurrent_process_calculi (end)

%\input{qm2pi.proofsketch}

% section proof sketch (end)

%\input{qm2pi.slviaknots} 

% section spatial logic via knots (end)

\input{qm2pi.conclusion}

% section conclusion (end)

%\input{qm2pi.dtcodes} 

% section wiring algorithm (end)

\input{qm2pi.ack} 

% section acknowledgments (end)

\newpage


\bibliographystyle{plain}   
\bibliography{../../biblios/main.bib}

\input{qm2pi.rhodetails}

\end{document}

 

%\ifpdf
%\usepackage[pdftex]{graphicx}
%\else
%\usepackage{graphicx}
%\fi

 % \ifpdf
%  \usepackage{pdfsync}
%  \if


%\title{Brief Article}
%\author{David F. Snyder}
%\author{L.G. Meredith}

%\address{Dept. of Math., Texas State University--San Marcos, San Marcos, TX 78666}
       
\pagestyle{empty}


\begin{document}

\lstset{language=[Objective]Caml,frame=shadowbox}

\documentclass[12pt]{llncs}
%\documentclass{jktr}

\usepackage[pdftex]{hyperref}                   
\usepackage {listings}
\usepackage {mathpartir}
\usepackage{bcprules}
%\usepackage{listings}
                       
\usepackage{graphicx} 
%\usepackage[margins=2.5cm,nohead,nofoot]{geometry}
%\usepackage{geometry}
\usepackage{amsfonts}
\usepackage{amstext}
\usepackage{latexsym}
\usepackage{amssymb}
\usepackage{color}


%\include{myPreamble}
\include{qm2pi.local} 

%\ifpdf
%\usepackage[pdftex]{graphicx}
%\else
%\usepackage{graphicx}
%\fi

 % \ifpdf
%  \usepackage{pdfsync}
%  \if


%\title{Brief Article}
%\author{David F. Snyder}
%\author{L.G. Meredith}

%\address{Dept. of Math., Texas State University--San Marcos, San Marcos, TX 78666}
       
\pagestyle{empty}


\begin{document}

\lstset{language=[Objective]Caml,frame=shadowbox}

\input{qm2pi.front}

% section front matter (end)

\input{qm2pi.intro} 
 
% section introduction (end)

% \input{qm2pi.knotations} 

% section notation (end)

\input{qm2pi.process.calculi} 

% section concurrent_process_calculi_and_spatial_logics_ (end)
    
%\input{qm2pi.knots2pi} 

%\input{qm2pi.trefoil} 

%\input{qm2pi.mainthm} 

% subsection basic_interpretation (end)

%\input{qm2pi.rho.presentation} 
\subsection{The syntax and semantics of the notation system}\label{sub:the_syntax_and_semantics_of_the_notation_system} % (fold)

We now summarize a technical presentation of the calculus that
embodies our theory of dynamics. The typical presentation of such a
calculus follows the style of giving generators and relations on
them. The grammar, below, describing term constructors, freely
generates the set of processes, $\Proc$. This set is then quotiented
by a relation known as structural congruence and it is over this set
that the notion of dynamics is expressed. This presentation is
essentially that of \cite{MeredithR05} with the addition of
polyadicity and summation. For readability we have relegated some of
the technical subtleties to an appendix.

\subsubsection{Process grammar}\label{subsub:process_grammar}

\begin{mathpar}
  \inferrule* [lab=synchronization] {} {{M} \bc \pzero \;|\; x?F \;|\; x!C }
  \and
  \inferrule* [lab=abstraction] {} {{F} \bc (x)P}
  \and
  \inferrule* [lab=concretion] {} {{C} \bc \langle Q \rangle}
  \and
  \inferrule* [lab=process] {} {{P,Q} \bc M \;| \;P|Q \;|\; @{x}}
  \and
  \inferrule* [lab=name] {} {{x} \bc \quotep{P}}
\end{mathpar} 

Note that $\vec{x}$ (resp. $\vec{P}$) denotes a vector of names
(resp. processes) of length $|\vec{x}|$ (resp. $|\vec{P}|$). We adopt
the following useful abbreviations.

\begin{mathpar}
   x?(\vec{y}).P := x.(\vec{y})P \and  x\clift{\vec{P}} := x.\clift{\vec{P}}
   \and x!(y) := \lift{x}{\dropn{y}}
   \and \Pi_{i=0}^{n-1}P_i := P_0 | \ldots | P_{n-1}
\end{mathpar}

\subsubsection{Structural congruence}

\paragraph{Free and bound names and alpha-equivalence.} At the
core of structural equivalence is alpha-equivalence which identifies
process that are the same up to a change of variable. Formally, we
recognize the distinction between free and bound names. The free names
of a process, $\freenames{P}$, may be calculated recursively as
follows:

\begin{mathpar}
\freenames{\pzero} := \emptyset
  \and \\
  \freenames{x?(y).P} := \{ x \} \cup (\freenames{P} \setminus \{ y \})
  \and 
  \freenames{x!\langle P \rangle} := \{ x \} \cup \{ P \} 
  \and \\
  \freenames{P|Q} := \freenames{P} \cup \freenames{Q}
  \and \\
  \freenames{@{x}} := \{ x \}
\end{mathpar}

$\pi$
$\quotep{\pi}$

$\freenames{-} : \pi \to \mathcal{P}(\quotep{\pi})$

\begin{eqnarray*}
  \freenames{\pzero} & := & \emptyset \\
  \freenames{x?(y).P} & := & \{ x \} \cup (\freenames{P} \setminus \{ y \}) \\
  \freenames{x!\langle P \rangle} & := & \{ x \} \cup \{ P \} \\
  \freenames{P|Q} & := & \freenames{P} \cup \freenames{Q} \\
  \freenames{\dropn{x}} & := & \{ x \}
\end{eqnarray*}

The bound names of a process, $\boundnames{P}$, are those names occurring in $P$
that are not free. For example, in $x?(y).0$, the name $x$ is free, while $y$ is bound.

\begin{mathpar}
  \inferrule* [lab=monoidal-laws] {} { P|Q \equiv Q|P \and P|0 \equiv P \and P|(Q|R) \equiv (P|Q)|R }
\end{mathpar}

\begin{mathpar}
  \inferrule* [lab=alpha-equivalence] {} { (x)P \equiv (y)P\{y/x\} \and y \not\in \freenames{P} }
\end{mathpar}

\begin{definition}
Then two processes, $P,Q$, are alpha-equivalent if $P = Q\{\vec{y}/\vec{x}\}$ for
some $\vec{x} \in \boundnames{Q},\vec{y} \in \boundnames{P}$, where $Q\{\vec{y}/\vec{x}\}$
denotes the capture-avoiding substitution of $\vec{y}$ for $\vec{x}$ in $Q$.
\end{definition}

\begin{definition}
  The {\em structural congruence} \cite{SangiorgiWalker} , $\equiv$,
  between processes is the least congruence containing
  alpha-equivalence, satisfying the abelian monoid laws
  (associativity, commutativity and $\pzero$ as identity) for parallel
  composition $|$ and for summation $+$.
\end{definition}

\subsection{Name equivalence}

We take name equivalence, written $\nameeq$, to be the smallest
equivalence relation generated by the following rules.

\begin{mathpar}
\inferrule*[lab=Quote-drop]
{ }
{ \quotep{@{x}} \nameeq x }

\inferrule*[lab=Struct-equiv]
{ P \scong Q }
{ \quotep{P} \nameeq \quotep{Q} }
\end{mathpar}

The astute reader will have noticed that the mutual recursion of names
and processes imposes a mutual recursion on alpha-equivalence and
structural equivalence via name-equivalence. Fortunately, all of this
works out pleasantly and we may calculate in the natural way, free of
concern. The reader interested in the details is referred to the
appendix \ref{appendix:rho_details}.

\subsection{Substitution}

We use $\Proc$ for the set of processes, $\QProc$ for the set of
names, and $\id{\{}\vec{y} / \vec{x} \id{\}}$ to denote partial maps,
$s : \QProc \rightarrow \QProc$. A map, $s$ lifts, uniquely, to a map
on process terms, $\widehat{s} : \Proc \rightarrow \Proc$ by the
following equations.

\begin{mathpar}
  (0) \psubstp{Q}{P} := 0 \\
  (R \juxtap S) \psubstp{Q}{P}
  :=    
  (R)\psubstp{Q}{P} \juxtap (S) \psubstp{Q}{P} \\
  (x?(y).R) \psubstp{Q}{P}    
  :=    
  (x)\substp{Q}{P} (z)\concat( (R \psubstn{z}{y}) \psubstp{Q}{P} ) \\
  (\lift{x}{R}) \psubstp{Q}{P}  
  :=
  \lift{(x)\substp{Q}{P}}{ R \psubstp{Q}{P} } \\
%   (\dropn{x})  \psubstp{Q}{P}       
%   := 
%   \left\{ 
%     \begin{array}{ccc} 
%       \dropn{\quotep{Q}} & & x \nameeq \quotep{P} \\
%       \dropn{x} & & otherwise \\
%     \end{array}
%   \right. 
  (\dropn{x})  \psubstp{Q}{P}       
  := 
  \left\{ 
    \begin{array}{ccc} 
      Q & & x \nameeq \quotep{P} \\
      \dropn{x} & & otherwise \\
    \end{array}
  \right.
\end{mathpar}
 

where

\begin{eqnarray}
  (x)\id{\{} \lpquote Q \rpquote / \lpquote P \rpquote \id{\}}            = 
  \left\{ 
    \begin{array}{ccc}
      \lpquote Q \rpquote & & x \nameeq \lpquote P \rpquote \\
      x & & otherwise \\
    \end{array}
  \right. \nonumber
\end{eqnarray}

and $z$ is chosen distinct from $\quotep{P}$, $\quotep{Q}$, the free
names in $Q$, and all the names in $R$. Our $\alpha$-equivalence will
be built in the standard way from this substitution.

\begin{remark}\label{rem:no_self_referential_names}
  One consequence of these definitions is that $\forall P. \quotep{P}
  \not\in \freenames{P}$.
\end{remark}

\subsection{ Dynamic quote: an example }

Anticipating something of what's to come, consider applying the
substitution, $\widehat{\id{\{}u / z \id{\}}}$, to the following pair
of processes, $\lift{w}{y!(z)}$ and $w[ \lpquote y!(z) \rpquote ]$.

\begin{eqnarray}
	\lift{w}{y!(z)}\widehat{\id{\{}u / z \id{\}}}
		& = &
		\lift{w}{y!(u)} \nonumber\\
	w[ \lpquote y!(z) \rpquote ] \widehat{ \id{\{}u / z \id{\}} }
		& = &
		w[ \lpquote y!(z) \rpquote ] \nonumber
\end{eqnarray}

Because the body of the process between quotes is impervious to
substitution, we get radically different answers. In fact, by
examining the first process in an input context,
e.g. $x?(z).\lift{w}{y!(z)}$, we see that the process under the lift
operator may be shaped by prefixed inputs binding a name inside it. In
this sense, the lift operator will be seen as a way to dynamically
construct processes before reifying them as names.

Finally equipped with these standard features we can present the
dynamics of the calculus.

\subsubsection{Operational semantics} 

Finally, we introduce the computational dynamics. What marks these
algebras as distinct from other more traditionally studied algebraic
structures, e.g. vector spaces or polynomial rings, is the manner in
which dynamics is captured. In traditional structures, dynamics is typically
expressed through morphisms between such structures, as in linear maps
between vector spaces or morphisms between rings. In algebras
associated with the semantics of computation, the dynamics is
expressed as part of the algebraic structure itself, through a
reduction reduction relation typically denoted by $\red$. Below, we
give a recursive presentation of this relation for the calculus used
in the encoding.

$\red \subseteq \pi \times \pi$
$\red : \pi \to \mathcal{P}(\pi)$

\begin{mathpar}
  \inferrule* [lab=Comm] { \textsf{match}( x_{src}, x_{trgt} ) } { x_{trgt}?(y)P \; | \; x_{src}!\langle {Q} \rangle \red P\{\quotep{Q}/y}\} }
  \and \\
  \inferrule* [lab=Par] {{P} \red {P}'} {{{P} | {Q}} \red {{P}' | {Q}}}
  \and
  \inferrule* [lab=Equiv]{{{P} \scong {P}'} \andalso {{P}' \red {Q}'} \andalso {{Q}' \scong {Q}}}{{P} \red {Q}}
\end{mathpar}

\begin{eqnarray*}
  match_{\equiv} (\quotep{P},\quotep{Q}) & := & P \equiv Q \\
  match_{\dagger}(\quotep{P},\quotep{Q}) & := & \forall R. P|Q \red^{*} R => R \red^{*} 0 \\
  match_{K}(\quotep{P},\quotep{Q}) & := & K \mbox{ for some context } K
\end{eqnarray*}

$u?(x)P | u!\langle Q \rangle \red P\{\quotep{Q}/x\}$

%We write $\wred$ for $\red^*$, and $P\red$ if $\exists Q $ such that $ P \red Q$.
We write $P\red$ if $\exists Q $ such that $ P \red Q$ and $P\not\red$, otherwise.

\section{Replication}

As mentioned before, it is known that replication (and hence
recursion) can be implemented in a higher-order process algebra
\cite{SangiorgiWalker}. As our first example of calculation with the
machinery thus far presented we give the construction explicitly in
the {\rhoc}.

\begin{eqnarray}
	D_{x} & := & \prefix{x}{y}{(\binpar{\outputp{x}{y}}{@{y}})} \nonumber\\
	\bangp_{x}{P} & := & \binpar{{x}!\langle{\binpar{D_{x}}{P}}\rangle}{D_{x}} \nonumber
\end{eqnarray}

\begin{eqnarray}
	\bangp_{x}{P} & & \nonumber\\
	=
	& {x}!\langle{(\prefix{x}{y}{(\outputp{x}{y} | @{y})) | P}}\rangle 
	      | \prefix{x}{y}{(\outputp{x}{y} | @{y})} & \nonumber\\
	\red
	& (\outputp{x}{y} | @{y})\substn{\quotep{(\prefix{x}{y}{(@{y} | \outputp{x}{y})) | P}}}{y} & \nonumber\\
	=
	& \outputp{x}{\quotep{(\prefix{x}{y}{(\outputp{x}{y} | @{y})) | P}}}
	  | {(\prefix{x}{y}{(\outputp{x}{y} | @{y})) | P}} & \nonumber\\
	\red
	& \ldots & \nonumber\\
	\red^*
	& P | P | \ldots & \nonumber
\end{eqnarray}

Of course, this encoding, as an implementation, runs away, unfolding
$\bangp{P}$ eagerly. A lazier and more implementable replication
operator, restricted to input-guarded processes, may be obtained as follows.

\begin{eqnarray}
\bangp{\prefix{u}{v}{P}} 
	:= 
	\binpar{\lift{x}{\prefix{u}{v}{(\binpar{D(x)}{P})}}}{D(x)} \nonumber
\end{eqnarray}

\begin{remark}
  Note that the lazier definition still does not deal with summation
  or mixed summation (i.e. sums over input and output). The reader is
  invited to construct definitions of replication that deal with these
  features. 

  Further, the definitions are parameterized in a name, $x$. Can you,
  gentle reader, make a definition that eliminates this parameter and
  guarantees no accidental interaction between the replication
  machinery and the process being replicated -- i.e. no accidental
  sharing of names used by the process to get its work done and the
  name(s) used by the replication to effect copying. This latter
  revision of the definition of replication is crucial to obtaining
  the expected identity $!!P \sim !P$.
\end{remark}

\begin{remark}\label{rem:paradoxical_combinator}
  The reader familiar with the lambda calculus will have noticed the
  similarity between $D$ and the paradoxical combinator.

  [Ed. note: the existence of this seems to suggest we have to be more
  restrictive on the set of processes and names we admit if we are to
  support no-cloning.]
\end{remark}

\subsubsection{Bisimulation}

The computational dynamics gives rise to another kind of equivalence,
the equivalence of computational behavior. As previously mentioned
this is typically captured \emph{via} some form of bisimulation.

% The notion we use in this paper is weak barbed bisimulation
% \cite{milner91polyadicpi}.

The notion we use in this paper is derived from weak barbed
bisimulation \cite{milner91polyadicpi}. 

\begin{definition}
An \emph{observation relation}, $\downarrow_{\mathcal N}$, over a set
of names, $\mathcal N$, is the smallest relation satisfying the rules
below.

\infrule[Out-barb]{y \in {\mathcal N}, \; x \nameeq y}
		  {\outputp{x}{v} \downarrow_{\mathcal N} x}
\infrule[Par-barb]{\mbox{$P\downarrow_{\mathcal N} x$ or $Q\downarrow_{\mathcal N} x$}}
		  {\binpar{P}{Q} \downarrow_{\mathcal N} x}

We write $P \Downarrow_{\mathcal N} x$ if there is $Q$ such that 
$P \wred Q$ and $Q \downarrow_{\mathcal N} x$.
\end{definition}

\begin{definition}
%\label{def.bbisim}
An  ${\mathcal N}$-\emph{barbed bisimulation} over a set of names, ${\mathcal N}$, is a symmetric binary relation 
${\mathcal S}_{\mathcal N}$ between agents such that $P\rel{S}_{\mathcal N}Q$ implies:
\begin{enumerate}
\item If $P \red P'$ then $Q \wred Q'$ and $P'\rel{S}_{\mathcal N} Q'$.
\item If $P\downarrow_{\mathcal N} x$, then $Q\Downarrow_{\mathcal N} x$.
\end{enumerate}
$P$ is ${\mathcal N}$-barbed bisimilar to $Q$, written
$P \wbbisim_{\mathcal N} Q$, if $P \rel{S}_{\mathcal N} Q$ for some ${\mathcal N}$-barbed bisimulation ${\mathcal S}_{\mathcal N}$.
\end{definition}

$\mathcal{R} \subseteq \pi \times \pi$

$P \mathcal{R} Q => \forall P'. P \red P' \Rightarrow \exists Q'. Q \red Q', P' \mathcal{R} Q'$

$P \vdash x \Rightarrow Q \vdash x$

\begin{mathpar}
  \inferrule*[lab=Out-barb]{x \nameeq y}{{y}!\langle{Q}\rangle \vdash x}
  \and
  \inferrule*[lab=Par-barb]{\mbox{$P\vdash x$ or $Q\vdash x$}}{\binpar{P}{Q} \vdash x}
\end{mathpar}

\subsubsection{Contexts}

One of the principle advantages of computational calculi like the
$\pi$-calculus is a well-defined notion of context,
contextual-equivalence and a correlation between
contextual-equivalence and notions of bisimulation. The notion of
context allows the decomposition of a process into (sub-)process and
its syntactic environment, its context. Thus, a context may be
thought of as a process with a ``hole'' (written $\Box$) in it. The
application of a context $M$ to a process $P$, written $M[P]$, is
tantamount to filling the hole in $M$ with $P$. In this paper we do
not need the full weight of this theory, but do make use of the notion
of context in the proof the main theorem. 

\begin{mathpar}
  \inferrule* [lab=summation] {} {{M_{M},M_{N}} \bc \Box \;|\; x.M_{A} \;|\; M_{M}+M_{N}}
  \and
  \inferrule* [lab=agent] {} {{M_{A}} \bc (\vec{x})M_{P} \;| \; \clift{P_0,\ldots,M_{P},\ldots,P_N}}
  \and \\
  \inferrule* [lab=process] {} {{M_{P}} \bc M_{N} \;| \;P|M_{P} }
\end{mathpar} 

\begin{mathpar}
  \inferrule* [lab=sychronization] {} {M_{N} \bc \Box \;|\; x?M_{F} \;|\; x!M_{C}}
  \and
  \inferrule* [lab=abstraction] {} {{M_{F}} \bc (x)M_{P} }
  \and
  \inferrule* [lab=concretion] {} {{M_{C}} \bc \langle M_{P} \rangle }
  \and \\
  \inferrule* [lab=process] {} {{M_{P}} \bc M_{N} \;| \;P|M_{P} }
\end{mathpar}

\begin{definition}[contextual application] Given a context $M$, and
  process $P$, we define the \emph{contextual application}, $M[P] :=
  M\{P/\Box\}$. That is, the contextual application of M to P is the
  substitution of $P$ for $\Box$ in $M$.
\end{definition}

$\meaningof{-} : L \to \mathcal{P}(\pi)$

\begin{mathpar}
  \inferrule* [lab=collection] {} {\meaningof{true} = \pi, \and \meaningof{~E} = \pi \setminus \meaningof{E}, \and \meaningof{E_{1} \& E_{2}} = \meaningof{E_{1}} \cap \meaningof{E_{2}}}
\end{mathpar}

\begin{mathpar}
  \inferrule* [lab=structure] {} {\meaningof{0} = \{ P \in \pi | P \equiv 0 \}, \and \\ \meaningof{E_1 | E_2} = \{ P \in \pi | P \equiv P_{1} | P_{2}, P_{1} \in \meaningof{E_{1}}, P_{2} \in \meaningof{E_2}\} }
\end{mathpar}

\begin{mathpar}
 \inferrule* [lab=behavior] {} {\meaningof{\langle a?b \rangle E} = \{ P \in \pi | P \equiv Q | u?(y)P', \\ \and \\\\ \and \\ \;\;\; u \in \meaningof{a}, \forall z.P'\{z/y\} \in \meaningof{E\{z/b\}}\}, \and \\ \meaningof{a!E} = \{ P \in \pi | P \equiv Q | x!\langle P' \rangle, x \in \meaningof{a} P' \in \meaningof{E}\} }
\end{mathpar}

\begin{mathpar}
 \inferrule* [lab=nominal] {} {\meaningof{\quotep{E}} = \{ \quotep{P} \in \quotep{\pi} | P \in \meaningof{E} \}, \and \meaningof{\quotep{P}} = \{ \quotep{Q} \in \quotep{\pi} | P \equiv Q \} \and \\ \meaningof{@\quotep{E}} = \{ P \in \pi | P \equiv @x, x \in \meaningof{E} \}}
\end{mathpar}

\begin{eqnarray*}
  \\
  \meaningof{-} : TS \to ST
\end{eqnarray*}

\begin{eqnarray*}
  \\
  L : TS \to ST
\end{eqnarray*}

\begin{eqnarray*}
  \\
  P \models E \iff P \in \meaningof{E}
\end{eqnarray*}

\begin{eqnarray*}
  P \approx_{L} Q \iff \forall E \in L. P \models E \iff Q \models E
\end{eqnarray*}

\begin{eqnarray*}
  P \approx_{K} Q
\end{eqnarray*}

\begin{eqnarray*}
  P \approx Q
\end{eqnarray*}

$\approx_{K} = \approx = \approx_{L}$

\subsubsection{Contextual duality}

Note that contexts extend the quotation operation to a family of
operations from processes to names. Given a context, $M$, we can
define a \emph{nominal context}, $\quotep{M}$ by $\quotep{M}[P] :=
\quotep{M[P]}$. To foreshadow what is to come we observe that these
operations enjoy a duality with processes very much like the duality
between vectors and maps from vectors to scalars.

Further, because the calculus is essentially higher-order, we have a
correspondence between contexts and processes. More specifically,
given a name $x$ and a context $M$ we can construct $M^{*}_{x}$ such
that 

\begin{mathpar}
  M^{*}_{x} | \lift{x}{P} \red M[P]
\end{mathpar}

namely,

\begin{mathpar}
  M^{*}_{x} := x?(u).M[\dropn{u}]
\end{mathpar}

The dependence of $M^{*}_{x}$ on a name makes it an abstraction, 

\begin{mathpar}
  M^{*} := (x)x?(u).M[\dropn{u}]
\end{mathpar}

\subsection{Additional notation}

It will sometimes be convenient to denote the process a name
quotes. We already have the notation $x = \quotep{P}$, but it will be
convenient to introduce an alternate notation, $\procn{x}$, when we
want to emphasize the connection to the use of the name. Note that, by
virtue of name equivalence, $\quotep{\procn{x}} \nameeq x$; so, the
notation is consistent with previous definitions.

Further, because names have structure it is possible to effect
substitutions on the basis of that structure. This means we need to
upgrade our notation for substitutions, which we accomplish by
adapting comprehension notation. Thus,

\begin{mathpar}
  P\{ y / x : x \in S \}
\end{mathpar}

is interpreted to mean the process derived from P by replacing (in a
capture-avoiding manner) each occurrence of $x$ in $S$ by $y$. For example,

\begin{mathpar}
  P\{ \quotep{\procn{x}|\procn{x}} / x : x \in \freenames{P} \}
\end{mathpar}

will replace each (occurrence) of a free name $x$ in $P$ by
$\quotep{\procn{x}|\procn{x}}$.

Also, we will avail ourselves of the notation $x^{L}$ and $x^{R}$ to
denote injections of a name into disjoint copies of the name
space. There are numerous ways to accomplish this. One example can be
found in \cite{MeredithR05}. This notation overloads to vectors of
names: $\vec{x}^{\pi} := (x_{i}^{\pi} \; : \; 0 \leq i < |\vec{x}| )$ where $\pi \in \{L,R\}$.

We also use $P^{\Box} := P|\Box$.

In \cite{MeredithR05} an interpretation of the new operator is
given. It turns out that there are several possible interpretations
all enjoying the requisite algebraic properties of the operator (see
\cite{milner91polyadicpi}). We will therefore make liberal use of
$(\nu\; \vec{x})P$.

% subsection the_syntax_and_semantics_of_the_notation_system (end)   

\input{qm2pi.qmops} 

\input{qm2pi.sterngerlach} 

\input{qm2pi.metric} 

% section concurrent_process_calculi (end)

%\input{qm2pi.proofsketch}

% section proof sketch (end)

%\input{qm2pi.slviaknots} 

% section spatial logic via knots (end)

\input{qm2pi.conclusion}

% section conclusion (end)

%\input{qm2pi.dtcodes} 

% section wiring algorithm (end)

\input{qm2pi.ack} 

% section acknowledgments (end)

\newpage


\bibliographystyle{plain}   
\bibliography{../../biblios/main.bib}

\input{qm2pi.rhodetails}

\end{document}



% section front matter (end)

\section{Introduction}\label{sec:introduction} % (fold)
In this draft of the material i am going to have to dispense with the
usual writing conventions adopted in papers on these topics. i'm going
to have adopt whatever tone i need at the time i'm writing up the
calculations. Sometimes this may be very conversational; others it may
be the barest mathematical grunts; others still it may be that i have
lifted text from one of my other papers because the exposition of some
point was better said there. i hope that my readers are not unduly put
out by this decision. i'm not doing this to flout convention or be
rebellious. i find these calculations very technically challenging. To
keep everything going technically, something has to give; i have to
let go of some cognitive burden. So, the academic writing style --
with all of its trade-offs in terms of facilitating technical
communication -- is what i'm letting go of. Perhaps subsequent drafts
can be tightened and polished, but for now, i'm going to speak as if
we were sitting together in a coffee shop with a laptop, wifi and a
pad of paper and a pencil.

So, here's what i have to say. We -- you and i, comfortably ensconced
in our coffee shop and well-equipped with our tools -- can realize and
carry out the calculations of quantum mechanics over a very different
formal theory of dynamics, a formal theory of dynamics that
corresponds to a theory of concurrent computation with
\emph{reflection}. It has the advantage that the underlying theory is
already `quantized', but supports analogues all of the continuuous
operations. Strikingly, this underlying theory has recently been
connected with a notion of metric that we can show, by calculating
together, coincides with the metric induced by the inner product.

There are a lot of reasons why you might be interested in seeing
calculations of this form. Here's why i'm interested. For the past
several centuries there has been no competitor to the ``Newtonian''
account of dynamics. As a result the predominant share of accounts of
dynamical systems and situations have had to be formulated in terms of
the Newtonian machinery. i view this as an intellectually dangerous
position to occupy. Everything, despite it's intrinsic shape, turns
into a nail to be hit with this hammer. Recently, however, the theory
of computation has matured to the point where we have candidates for
theories of dynamics that offer very different perspective on
reasoning about dynamical systems and situations. Testing these
candidates against very successful accounts of dynamical situations,
like quantum mechanics, is going to give us some sense of how mature
they are and some measure of the quality of these accounts of
dynamics.

\subsection{Summary of contributions and outline of paper}

So, we're going to develop an interpretation of the operations of
quantum mechanics normally interpreted by Hilbert spaces and
operators. We're going to do this over a theory of computation. Note
that this is very different than the usual quantum computation program
which develops notions of computation over quantum mechanics. Rather,
we are developing a story that aligns with Wheeler's slogan: It from
Bit. To do this we will first provide an account of the theory of
computation at play here. Then we will dive into a calculation-driven
interpretation of the operations of quantum mechanics.

The reason we take this approach is that -- until very recently --
there hasn't been an axiomatic account of quantum mechanics. As a
result there has been no sharp delineation of the mathematical theory
supporting interpretation of the physical theory and the physical
theory, itself. So, ambient features of the maths are free to be
exploited (or supressed) without a real accounting of their physical
relevance. There is no sharp statement ``here's the physical theory''
qua \emph{theory} and ``here's the mathematical interpretation''
enabling a judgment of how faithful the interpretation is -- apart
from experimental observation. When there is an axiomatic account we
can judge how well a given mathematical formalism supports an
interpretation of the axioms, independent of
experimentation. Likewise, we can judge how well we have captured our
physical evidence and experience with our axiomatics, independent of
any specific mathematical implementation, with accidental detail that
may or may not have physical significance. 

In lieu of a fully fleshed out and vetted axiomatic account of quantum
mechanics, interpreting the operational notions in service of modeling
physical systems will have to suffice. In other words, we are not in
the business of providing a model of Hilbert spaces and operators. We
are in the business of providing a model of quantum mechanics because
we are motivated by testing our notions of dynamics against physical
theory; and, the predictive calculations of the physical theory must
serve as the best formulation -- shy of a fully fleshed out axiomatic
account -- of the physical theory itself (as they have for scientific
theories since time immemorial). Put another way, despite a
whole-hearted commitment to an It-from-Bit ontology, we are firmly
aligned with the shut-up-and-calculate camp as the best way to obtain
results either from the physical perspective or as a quality assurance
measure of our fledgling theory of dynamics.

In detail, we present a reflective process calculus. Then we develop
intuitive correspondences between the notions available in this
calculus and the usual physical notions supporting quantum mechanical
calculations. Thus, 

\begin{table}[htp]
  \center{
    \fbox{
      \begin{tabular}{c|c}
        quantum mechanics & process calculus \\
        \hline
        scalar & name \\
        state vector & process \\
        dual & contextual duals \\
        matrix & formal sums of process-context-dual pairs \\
        orthogonality & process annihilation \\
        inner product & execution-formula + quoting
      \end{tabular}
    }
  }
  \caption{QM - process calculi correspondences}
\end{table}

Then we tighten up these intuitions to operational definitions. We
employ the Dirac notation as the best proxy we can find for an
abstract syntax of the quantum mechanical notions. The definitions we
develop put us in contact with equational constraints coming from the
theory that we demonstrate the definitions and calculations satisfy.

This puts us in a position to shut up and calculate for the
Stern-Gerlach experimental set up, showing how these predictive
calculations become calculations on processes in our theory of a
reflective process calculus.

Penultimately, we demonstrate that the notion of metric coming from
the inner product coincides with the notion of metric available from
the theory of bisimulation. This demonstration gives us the right to
think of space as arising from behavior. Finally, we consider where we
might go from the new vantage point we have obtained.

% section introduction (end) 
 
% section introduction (end)

% \documentclass[12pt]{llncs}
%\documentclass{jktr}

\usepackage[pdftex]{hyperref}                   
\usepackage {listings}
\usepackage {mathpartir}
\usepackage{bcprules}
%\usepackage{listings}
                       
\usepackage{graphicx} 
%\usepackage[margins=2.5cm,nohead,nofoot]{geometry}
%\usepackage{geometry}
\usepackage{amsfonts}
\usepackage{amstext}
\usepackage{latexsym}
\usepackage{amssymb}
\usepackage{color}


%\include{myPreamble}
\include{qm2pi.local} 

%\ifpdf
%\usepackage[pdftex]{graphicx}
%\else
%\usepackage{graphicx}
%\fi

 % \ifpdf
%  \usepackage{pdfsync}
%  \if


%\title{Brief Article}
%\author{David F. Snyder}
%\author{L.G. Meredith}

%\address{Dept. of Math., Texas State University--San Marcos, San Marcos, TX 78666}
       
\pagestyle{empty}


\begin{document}

\lstset{language=[Objective]Caml,frame=shadowbox}

\input{qm2pi.front}

% section front matter (end)

\input{qm2pi.intro} 
 
% section introduction (end)

% \input{qm2pi.knotations} 

% section notation (end)

\input{qm2pi.process.calculi} 

% section concurrent_process_calculi_and_spatial_logics_ (end)
    
%\input{qm2pi.knots2pi} 

%\input{qm2pi.trefoil} 

%\input{qm2pi.mainthm} 

% subsection basic_interpretation (end)

%\input{qm2pi.rho.presentation} 
\subsection{The syntax and semantics of the notation system}\label{sub:the_syntax_and_semantics_of_the_notation_system} % (fold)

We now summarize a technical presentation of the calculus that
embodies our theory of dynamics. The typical presentation of such a
calculus follows the style of giving generators and relations on
them. The grammar, below, describing term constructors, freely
generates the set of processes, $\Proc$. This set is then quotiented
by a relation known as structural congruence and it is over this set
that the notion of dynamics is expressed. This presentation is
essentially that of \cite{MeredithR05} with the addition of
polyadicity and summation. For readability we have relegated some of
the technical subtleties to an appendix.

\subsubsection{Process grammar}\label{subsub:process_grammar}

\begin{mathpar}
  \inferrule* [lab=synchronization] {} {{M} \bc \pzero \;|\; x?F \;|\; x!C }
  \and
  \inferrule* [lab=abstraction] {} {{F} \bc (x)P}
  \and
  \inferrule* [lab=concretion] {} {{C} \bc \langle Q \rangle}
  \and
  \inferrule* [lab=process] {} {{P,Q} \bc M \;| \;P|Q \;|\; @{x}}
  \and
  \inferrule* [lab=name] {} {{x} \bc \quotep{P}}
\end{mathpar} 

Note that $\vec{x}$ (resp. $\vec{P}$) denotes a vector of names
(resp. processes) of length $|\vec{x}|$ (resp. $|\vec{P}|$). We adopt
the following useful abbreviations.

\begin{mathpar}
   x?(\vec{y}).P := x.(\vec{y})P \and  x\clift{\vec{P}} := x.\clift{\vec{P}}
   \and x!(y) := \lift{x}{\dropn{y}}
   \and \Pi_{i=0}^{n-1}P_i := P_0 | \ldots | P_{n-1}
\end{mathpar}

\subsubsection{Structural congruence}

\paragraph{Free and bound names and alpha-equivalence.} At the
core of structural equivalence is alpha-equivalence which identifies
process that are the same up to a change of variable. Formally, we
recognize the distinction between free and bound names. The free names
of a process, $\freenames{P}$, may be calculated recursively as
follows:

\begin{mathpar}
\freenames{\pzero} := \emptyset
  \and \\
  \freenames{x?(y).P} := \{ x \} \cup (\freenames{P} \setminus \{ y \})
  \and 
  \freenames{x!\langle P \rangle} := \{ x \} \cup \{ P \} 
  \and \\
  \freenames{P|Q} := \freenames{P} \cup \freenames{Q}
  \and \\
  \freenames{@{x}} := \{ x \}
\end{mathpar}

$\pi$
$\quotep{\pi}$

$\freenames{-} : \pi \to \mathcal{P}(\quotep{\pi})$

\begin{eqnarray*}
  \freenames{\pzero} & := & \emptyset \\
  \freenames{x?(y).P} & := & \{ x \} \cup (\freenames{P} \setminus \{ y \}) \\
  \freenames{x!\langle P \rangle} & := & \{ x \} \cup \{ P \} \\
  \freenames{P|Q} & := & \freenames{P} \cup \freenames{Q} \\
  \freenames{\dropn{x}} & := & \{ x \}
\end{eqnarray*}

The bound names of a process, $\boundnames{P}$, are those names occurring in $P$
that are not free. For example, in $x?(y).0$, the name $x$ is free, while $y$ is bound.

\begin{mathpar}
  \inferrule* [lab=monoidal-laws] {} { P|Q \equiv Q|P \and P|0 \equiv P \and P|(Q|R) \equiv (P|Q)|R }
\end{mathpar}

\begin{mathpar}
  \inferrule* [lab=alpha-equivalence] {} { (x)P \equiv (y)P\{y/x\} \and y \not\in \freenames{P} }
\end{mathpar}

\begin{definition}
Then two processes, $P,Q$, are alpha-equivalent if $P = Q\{\vec{y}/\vec{x}\}$ for
some $\vec{x} \in \boundnames{Q},\vec{y} \in \boundnames{P}$, where $Q\{\vec{y}/\vec{x}\}$
denotes the capture-avoiding substitution of $\vec{y}$ for $\vec{x}$ in $Q$.
\end{definition}

\begin{definition}
  The {\em structural congruence} \cite{SangiorgiWalker} , $\equiv$,
  between processes is the least congruence containing
  alpha-equivalence, satisfying the abelian monoid laws
  (associativity, commutativity and $\pzero$ as identity) for parallel
  composition $|$ and for summation $+$.
\end{definition}

\subsection{Name equivalence}

We take name equivalence, written $\nameeq$, to be the smallest
equivalence relation generated by the following rules.

\begin{mathpar}
\inferrule*[lab=Quote-drop]
{ }
{ \quotep{@{x}} \nameeq x }

\inferrule*[lab=Struct-equiv]
{ P \scong Q }
{ \quotep{P} \nameeq \quotep{Q} }
\end{mathpar}

The astute reader will have noticed that the mutual recursion of names
and processes imposes a mutual recursion on alpha-equivalence and
structural equivalence via name-equivalence. Fortunately, all of this
works out pleasantly and we may calculate in the natural way, free of
concern. The reader interested in the details is referred to the
appendix \ref{appendix:rho_details}.

\subsection{Substitution}

We use $\Proc$ for the set of processes, $\QProc$ for the set of
names, and $\id{\{}\vec{y} / \vec{x} \id{\}}$ to denote partial maps,
$s : \QProc \rightarrow \QProc$. A map, $s$ lifts, uniquely, to a map
on process terms, $\widehat{s} : \Proc \rightarrow \Proc$ by the
following equations.

\begin{mathpar}
  (0) \psubstp{Q}{P} := 0 \\
  (R \juxtap S) \psubstp{Q}{P}
  :=    
  (R)\psubstp{Q}{P} \juxtap (S) \psubstp{Q}{P} \\
  (x?(y).R) \psubstp{Q}{P}    
  :=    
  (x)\substp{Q}{P} (z)\concat( (R \psubstn{z}{y}) \psubstp{Q}{P} ) \\
  (\lift{x}{R}) \psubstp{Q}{P}  
  :=
  \lift{(x)\substp{Q}{P}}{ R \psubstp{Q}{P} } \\
%   (\dropn{x})  \psubstp{Q}{P}       
%   := 
%   \left\{ 
%     \begin{array}{ccc} 
%       \dropn{\quotep{Q}} & & x \nameeq \quotep{P} \\
%       \dropn{x} & & otherwise \\
%     \end{array}
%   \right. 
  (\dropn{x})  \psubstp{Q}{P}       
  := 
  \left\{ 
    \begin{array}{ccc} 
      Q & & x \nameeq \quotep{P} \\
      \dropn{x} & & otherwise \\
    \end{array}
  \right.
\end{mathpar}
 

where

\begin{eqnarray}
  (x)\id{\{} \lpquote Q \rpquote / \lpquote P \rpquote \id{\}}            = 
  \left\{ 
    \begin{array}{ccc}
      \lpquote Q \rpquote & & x \nameeq \lpquote P \rpquote \\
      x & & otherwise \\
    \end{array}
  \right. \nonumber
\end{eqnarray}

and $z$ is chosen distinct from $\quotep{P}$, $\quotep{Q}$, the free
names in $Q$, and all the names in $R$. Our $\alpha$-equivalence will
be built in the standard way from this substitution.

\begin{remark}\label{rem:no_self_referential_names}
  One consequence of these definitions is that $\forall P. \quotep{P}
  \not\in \freenames{P}$.
\end{remark}

\subsection{ Dynamic quote: an example }

Anticipating something of what's to come, consider applying the
substitution, $\widehat{\id{\{}u / z \id{\}}}$, to the following pair
of processes, $\lift{w}{y!(z)}$ and $w[ \lpquote y!(z) \rpquote ]$.

\begin{eqnarray}
	\lift{w}{y!(z)}\widehat{\id{\{}u / z \id{\}}}
		& = &
		\lift{w}{y!(u)} \nonumber\\
	w[ \lpquote y!(z) \rpquote ] \widehat{ \id{\{}u / z \id{\}} }
		& = &
		w[ \lpquote y!(z) \rpquote ] \nonumber
\end{eqnarray}

Because the body of the process between quotes is impervious to
substitution, we get radically different answers. In fact, by
examining the first process in an input context,
e.g. $x?(z).\lift{w}{y!(z)}$, we see that the process under the lift
operator may be shaped by prefixed inputs binding a name inside it. In
this sense, the lift operator will be seen as a way to dynamically
construct processes before reifying them as names.

Finally equipped with these standard features we can present the
dynamics of the calculus.

\subsubsection{Operational semantics} 

Finally, we introduce the computational dynamics. What marks these
algebras as distinct from other more traditionally studied algebraic
structures, e.g. vector spaces or polynomial rings, is the manner in
which dynamics is captured. In traditional structures, dynamics is typically
expressed through morphisms between such structures, as in linear maps
between vector spaces or morphisms between rings. In algebras
associated with the semantics of computation, the dynamics is
expressed as part of the algebraic structure itself, through a
reduction reduction relation typically denoted by $\red$. Below, we
give a recursive presentation of this relation for the calculus used
in the encoding.

$\red \subseteq \pi \times \pi$
$\red : \pi \to \mathcal{P}(\pi)$

\begin{mathpar}
  \inferrule* [lab=Comm] { \textsf{match}( x_{src}, x_{trgt} ) } { x_{trgt}?(y)P \; | \; x_{src}!\langle {Q} \rangle \red P\{\quotep{Q}/y}\} }
  \and \\
  \inferrule* [lab=Par] {{P} \red {P}'} {{{P} | {Q}} \red {{P}' | {Q}}}
  \and
  \inferrule* [lab=Equiv]{{{P} \scong {P}'} \andalso {{P}' \red {Q}'} \andalso {{Q}' \scong {Q}}}{{P} \red {Q}}
\end{mathpar}

\begin{eqnarray*}
  match_{\equiv} (\quotep{P},\quotep{Q}) & := & P \equiv Q \\
  match_{\dagger}(\quotep{P},\quotep{Q}) & := & \forall R. P|Q \red^{*} R => R \red^{*} 0 \\
  match_{K}(\quotep{P},\quotep{Q}) & := & K \mbox{ for some context } K
\end{eqnarray*}

$u?(x)P | u!\langle Q \rangle \red P\{\quotep{Q}/x\}$

%We write $\wred$ for $\red^*$, and $P\red$ if $\exists Q $ such that $ P \red Q$.
We write $P\red$ if $\exists Q $ such that $ P \red Q$ and $P\not\red$, otherwise.

\section{Replication}

As mentioned before, it is known that replication (and hence
recursion) can be implemented in a higher-order process algebra
\cite{SangiorgiWalker}. As our first example of calculation with the
machinery thus far presented we give the construction explicitly in
the {\rhoc}.

\begin{eqnarray}
	D_{x} & := & \prefix{x}{y}{(\binpar{\outputp{x}{y}}{@{y}})} \nonumber\\
	\bangp_{x}{P} & := & \binpar{{x}!\langle{\binpar{D_{x}}{P}}\rangle}{D_{x}} \nonumber
\end{eqnarray}

\begin{eqnarray}
	\bangp_{x}{P} & & \nonumber\\
	=
	& {x}!\langle{(\prefix{x}{y}{(\outputp{x}{y} | @{y})) | P}}\rangle 
	      | \prefix{x}{y}{(\outputp{x}{y} | @{y})} & \nonumber\\
	\red
	& (\outputp{x}{y} | @{y})\substn{\quotep{(\prefix{x}{y}{(@{y} | \outputp{x}{y})) | P}}}{y} & \nonumber\\
	=
	& \outputp{x}{\quotep{(\prefix{x}{y}{(\outputp{x}{y} | @{y})) | P}}}
	  | {(\prefix{x}{y}{(\outputp{x}{y} | @{y})) | P}} & \nonumber\\
	\red
	& \ldots & \nonumber\\
	\red^*
	& P | P | \ldots & \nonumber
\end{eqnarray}

Of course, this encoding, as an implementation, runs away, unfolding
$\bangp{P}$ eagerly. A lazier and more implementable replication
operator, restricted to input-guarded processes, may be obtained as follows.

\begin{eqnarray}
\bangp{\prefix{u}{v}{P}} 
	:= 
	\binpar{\lift{x}{\prefix{u}{v}{(\binpar{D(x)}{P})}}}{D(x)} \nonumber
\end{eqnarray}

\begin{remark}
  Note that the lazier definition still does not deal with summation
  or mixed summation (i.e. sums over input and output). The reader is
  invited to construct definitions of replication that deal with these
  features. 

  Further, the definitions are parameterized in a name, $x$. Can you,
  gentle reader, make a definition that eliminates this parameter and
  guarantees no accidental interaction between the replication
  machinery and the process being replicated -- i.e. no accidental
  sharing of names used by the process to get its work done and the
  name(s) used by the replication to effect copying. This latter
  revision of the definition of replication is crucial to obtaining
  the expected identity $!!P \sim !P$.
\end{remark}

\begin{remark}\label{rem:paradoxical_combinator}
  The reader familiar with the lambda calculus will have noticed the
  similarity between $D$ and the paradoxical combinator.

  [Ed. note: the existence of this seems to suggest we have to be more
  restrictive on the set of processes and names we admit if we are to
  support no-cloning.]
\end{remark}

\subsubsection{Bisimulation}

The computational dynamics gives rise to another kind of equivalence,
the equivalence of computational behavior. As previously mentioned
this is typically captured \emph{via} some form of bisimulation.

% The notion we use in this paper is weak barbed bisimulation
% \cite{milner91polyadicpi}.

The notion we use in this paper is derived from weak barbed
bisimulation \cite{milner91polyadicpi}. 

\begin{definition}
An \emph{observation relation}, $\downarrow_{\mathcal N}$, over a set
of names, $\mathcal N$, is the smallest relation satisfying the rules
below.

\infrule[Out-barb]{y \in {\mathcal N}, \; x \nameeq y}
		  {\outputp{x}{v} \downarrow_{\mathcal N} x}
\infrule[Par-barb]{\mbox{$P\downarrow_{\mathcal N} x$ or $Q\downarrow_{\mathcal N} x$}}
		  {\binpar{P}{Q} \downarrow_{\mathcal N} x}

We write $P \Downarrow_{\mathcal N} x$ if there is $Q$ such that 
$P \wred Q$ and $Q \downarrow_{\mathcal N} x$.
\end{definition}

\begin{definition}
%\label{def.bbisim}
An  ${\mathcal N}$-\emph{barbed bisimulation} over a set of names, ${\mathcal N}$, is a symmetric binary relation 
${\mathcal S}_{\mathcal N}$ between agents such that $P\rel{S}_{\mathcal N}Q$ implies:
\begin{enumerate}
\item If $P \red P'$ then $Q \wred Q'$ and $P'\rel{S}_{\mathcal N} Q'$.
\item If $P\downarrow_{\mathcal N} x$, then $Q\Downarrow_{\mathcal N} x$.
\end{enumerate}
$P$ is ${\mathcal N}$-barbed bisimilar to $Q$, written
$P \wbbisim_{\mathcal N} Q$, if $P \rel{S}_{\mathcal N} Q$ for some ${\mathcal N}$-barbed bisimulation ${\mathcal S}_{\mathcal N}$.
\end{definition}

$\mathcal{R} \subseteq \pi \times \pi$

$P \mathcal{R} Q => \forall P'. P \red P' \Rightarrow \exists Q'. Q \red Q', P' \mathcal{R} Q'$

$P \vdash x \Rightarrow Q \vdash x$

\begin{mathpar}
  \inferrule*[lab=Out-barb]{x \nameeq y}{{y}!\langle{Q}\rangle \vdash x}
  \and
  \inferrule*[lab=Par-barb]{\mbox{$P\vdash x$ or $Q\vdash x$}}{\binpar{P}{Q} \vdash x}
\end{mathpar}

\subsubsection{Contexts}

One of the principle advantages of computational calculi like the
$\pi$-calculus is a well-defined notion of context,
contextual-equivalence and a correlation between
contextual-equivalence and notions of bisimulation. The notion of
context allows the decomposition of a process into (sub-)process and
its syntactic environment, its context. Thus, a context may be
thought of as a process with a ``hole'' (written $\Box$) in it. The
application of a context $M$ to a process $P$, written $M[P]$, is
tantamount to filling the hole in $M$ with $P$. In this paper we do
not need the full weight of this theory, but do make use of the notion
of context in the proof the main theorem. 

\begin{mathpar}
  \inferrule* [lab=summation] {} {{M_{M},M_{N}} \bc \Box \;|\; x.M_{A} \;|\; M_{M}+M_{N}}
  \and
  \inferrule* [lab=agent] {} {{M_{A}} \bc (\vec{x})M_{P} \;| \; \clift{P_0,\ldots,M_{P},\ldots,P_N}}
  \and \\
  \inferrule* [lab=process] {} {{M_{P}} \bc M_{N} \;| \;P|M_{P} }
\end{mathpar} 

\begin{mathpar}
  \inferrule* [lab=sychronization] {} {M_{N} \bc \Box \;|\; x?M_{F} \;|\; x!M_{C}}
  \and
  \inferrule* [lab=abstraction] {} {{M_{F}} \bc (x)M_{P} }
  \and
  \inferrule* [lab=concretion] {} {{M_{C}} \bc \langle M_{P} \rangle }
  \and \\
  \inferrule* [lab=process] {} {{M_{P}} \bc M_{N} \;| \;P|M_{P} }
\end{mathpar}

\begin{definition}[contextual application] Given a context $M$, and
  process $P$, we define the \emph{contextual application}, $M[P] :=
  M\{P/\Box\}$. That is, the contextual application of M to P is the
  substitution of $P$ for $\Box$ in $M$.
\end{definition}

$\meaningof{-} : L \to \mathcal{P}(\pi)$

\begin{mathpar}
  \inferrule* [lab=collection] {} {\meaningof{true} = \pi, \and \meaningof{~E} = \pi \setminus \meaningof{E}, \and \meaningof{E_{1} \& E_{2}} = \meaningof{E_{1}} \cap \meaningof{E_{2}}}
\end{mathpar}

\begin{mathpar}
  \inferrule* [lab=structure] {} {\meaningof{0} = \{ P \in \pi | P \equiv 0 \}, \and \\ \meaningof{E_1 | E_2} = \{ P \in \pi | P \equiv P_{1} | P_{2}, P_{1} \in \meaningof{E_{1}}, P_{2} \in \meaningof{E_2}\} }
\end{mathpar}

\begin{mathpar}
 \inferrule* [lab=behavior] {} {\meaningof{\langle a?b \rangle E} = \{ P \in \pi | P \equiv Q | u?(y)P', \\ \and \\\\ \and \\ \;\;\; u \in \meaningof{a}, \forall z.P'\{z/y\} \in \meaningof{E\{z/b\}}\}, \and \\ \meaningof{a!E} = \{ P \in \pi | P \equiv Q | x!\langle P' \rangle, x \in \meaningof{a} P' \in \meaningof{E}\} }
\end{mathpar}

\begin{mathpar}
 \inferrule* [lab=nominal] {} {\meaningof{\quotep{E}} = \{ \quotep{P} \in \quotep{\pi} | P \in \meaningof{E} \}, \and \meaningof{\quotep{P}} = \{ \quotep{Q} \in \quotep{\pi} | P \equiv Q \} \and \\ \meaningof{@\quotep{E}} = \{ P \in \pi | P \equiv @x, x \in \meaningof{E} \}}
\end{mathpar}

\begin{eqnarray*}
  \\
  \meaningof{-} : TS \to ST
\end{eqnarray*}

\begin{eqnarray*}
  \\
  L : TS \to ST
\end{eqnarray*}

\begin{eqnarray*}
  \\
  P \models E \iff P \in \meaningof{E}
\end{eqnarray*}

\begin{eqnarray*}
  P \approx_{L} Q \iff \forall E \in L. P \models E \iff Q \models E
\end{eqnarray*}

\begin{eqnarray*}
  P \approx_{K} Q
\end{eqnarray*}

\begin{eqnarray*}
  P \approx Q
\end{eqnarray*}

$\approx_{K} = \approx = \approx_{L}$

\subsubsection{Contextual duality}

Note that contexts extend the quotation operation to a family of
operations from processes to names. Given a context, $M$, we can
define a \emph{nominal context}, $\quotep{M}$ by $\quotep{M}[P] :=
\quotep{M[P]}$. To foreshadow what is to come we observe that these
operations enjoy a duality with processes very much like the duality
between vectors and maps from vectors to scalars.

Further, because the calculus is essentially higher-order, we have a
correspondence between contexts and processes. More specifically,
given a name $x$ and a context $M$ we can construct $M^{*}_{x}$ such
that 

\begin{mathpar}
  M^{*}_{x} | \lift{x}{P} \red M[P]
\end{mathpar}

namely,

\begin{mathpar}
  M^{*}_{x} := x?(u).M[\dropn{u}]
\end{mathpar}

The dependence of $M^{*}_{x}$ on a name makes it an abstraction, 

\begin{mathpar}
  M^{*} := (x)x?(u).M[\dropn{u}]
\end{mathpar}

\subsection{Additional notation}

It will sometimes be convenient to denote the process a name
quotes. We already have the notation $x = \quotep{P}$, but it will be
convenient to introduce an alternate notation, $\procn{x}$, when we
want to emphasize the connection to the use of the name. Note that, by
virtue of name equivalence, $\quotep{\procn{x}} \nameeq x$; so, the
notation is consistent with previous definitions.

Further, because names have structure it is possible to effect
substitutions on the basis of that structure. This means we need to
upgrade our notation for substitutions, which we accomplish by
adapting comprehension notation. Thus,

\begin{mathpar}
  P\{ y / x : x \in S \}
\end{mathpar}

is interpreted to mean the process derived from P by replacing (in a
capture-avoiding manner) each occurrence of $x$ in $S$ by $y$. For example,

\begin{mathpar}
  P\{ \quotep{\procn{x}|\procn{x}} / x : x \in \freenames{P} \}
\end{mathpar}

will replace each (occurrence) of a free name $x$ in $P$ by
$\quotep{\procn{x}|\procn{x}}$.

Also, we will avail ourselves of the notation $x^{L}$ and $x^{R}$ to
denote injections of a name into disjoint copies of the name
space. There are numerous ways to accomplish this. One example can be
found in \cite{MeredithR05}. This notation overloads to vectors of
names: $\vec{x}^{\pi} := (x_{i}^{\pi} \; : \; 0 \leq i < |\vec{x}| )$ where $\pi \in \{L,R\}$.

We also use $P^{\Box} := P|\Box$.

In \cite{MeredithR05} an interpretation of the new operator is
given. It turns out that there are several possible interpretations
all enjoying the requisite algebraic properties of the operator (see
\cite{milner91polyadicpi}). We will therefore make liberal use of
$(\nu\; \vec{x})P$.

% subsection the_syntax_and_semantics_of_the_notation_system (end)   

\input{qm2pi.qmops} 

\input{qm2pi.sterngerlach} 

\input{qm2pi.metric} 

% section concurrent_process_calculi (end)

%\input{qm2pi.proofsketch}

% section proof sketch (end)

%\input{qm2pi.slviaknots} 

% section spatial logic via knots (end)

\input{qm2pi.conclusion}

% section conclusion (end)

%\input{qm2pi.dtcodes} 

% section wiring algorithm (end)

\input{qm2pi.ack} 

% section acknowledgments (end)

\newpage


\bibliographystyle{plain}   
\bibliography{../../biblios/main.bib}

\input{qm2pi.rhodetails}

\end{document}

 

% section notation (end)

\input{qm2pi.process.calculi} 

% section concurrent_process_calculi_and_spatial_logics_ (end)
    
%\documentclass[12pt]{llncs}
%\documentclass{jktr}

\usepackage[pdftex]{hyperref}                   
\usepackage {listings}
\usepackage {mathpartir}
\usepackage{bcprules}
%\usepackage{listings}
                       
\usepackage{graphicx} 
%\usepackage[margins=2.5cm,nohead,nofoot]{geometry}
%\usepackage{geometry}
\usepackage{amsfonts}
\usepackage{amstext}
\usepackage{latexsym}
\usepackage{amssymb}
\usepackage{color}


%\include{myPreamble}
\include{qm2pi.local} 

%\ifpdf
%\usepackage[pdftex]{graphicx}
%\else
%\usepackage{graphicx}
%\fi

 % \ifpdf
%  \usepackage{pdfsync}
%  \if


%\title{Brief Article}
%\author{David F. Snyder}
%\author{L.G. Meredith}

%\address{Dept. of Math., Texas State University--San Marcos, San Marcos, TX 78666}
       
\pagestyle{empty}


\begin{document}

\lstset{language=[Objective]Caml,frame=shadowbox}

\input{qm2pi.front}

% section front matter (end)

\input{qm2pi.intro} 
 
% section introduction (end)

% \input{qm2pi.knotations} 

% section notation (end)

\input{qm2pi.process.calculi} 

% section concurrent_process_calculi_and_spatial_logics_ (end)
    
%\input{qm2pi.knots2pi} 

%\input{qm2pi.trefoil} 

%\input{qm2pi.mainthm} 

% subsection basic_interpretation (end)

%\input{qm2pi.rho.presentation} 
\subsection{The syntax and semantics of the notation system}\label{sub:the_syntax_and_semantics_of_the_notation_system} % (fold)

We now summarize a technical presentation of the calculus that
embodies our theory of dynamics. The typical presentation of such a
calculus follows the style of giving generators and relations on
them. The grammar, below, describing term constructors, freely
generates the set of processes, $\Proc$. This set is then quotiented
by a relation known as structural congruence and it is over this set
that the notion of dynamics is expressed. This presentation is
essentially that of \cite{MeredithR05} with the addition of
polyadicity and summation. For readability we have relegated some of
the technical subtleties to an appendix.

\subsubsection{Process grammar}\label{subsub:process_grammar}

\begin{mathpar}
  \inferrule* [lab=synchronization] {} {{M} \bc \pzero \;|\; x?F \;|\; x!C }
  \and
  \inferrule* [lab=abstraction] {} {{F} \bc (x)P}
  \and
  \inferrule* [lab=concretion] {} {{C} \bc \langle Q \rangle}
  \and
  \inferrule* [lab=process] {} {{P,Q} \bc M \;| \;P|Q \;|\; @{x}}
  \and
  \inferrule* [lab=name] {} {{x} \bc \quotep{P}}
\end{mathpar} 

Note that $\vec{x}$ (resp. $\vec{P}$) denotes a vector of names
(resp. processes) of length $|\vec{x}|$ (resp. $|\vec{P}|$). We adopt
the following useful abbreviations.

\begin{mathpar}
   x?(\vec{y}).P := x.(\vec{y})P \and  x\clift{\vec{P}} := x.\clift{\vec{P}}
   \and x!(y) := \lift{x}{\dropn{y}}
   \and \Pi_{i=0}^{n-1}P_i := P_0 | \ldots | P_{n-1}
\end{mathpar}

\subsubsection{Structural congruence}

\paragraph{Free and bound names and alpha-equivalence.} At the
core of structural equivalence is alpha-equivalence which identifies
process that are the same up to a change of variable. Formally, we
recognize the distinction between free and bound names. The free names
of a process, $\freenames{P}$, may be calculated recursively as
follows:

\begin{mathpar}
\freenames{\pzero} := \emptyset
  \and \\
  \freenames{x?(y).P} := \{ x \} \cup (\freenames{P} \setminus \{ y \})
  \and 
  \freenames{x!\langle P \rangle} := \{ x \} \cup \{ P \} 
  \and \\
  \freenames{P|Q} := \freenames{P} \cup \freenames{Q}
  \and \\
  \freenames{@{x}} := \{ x \}
\end{mathpar}

$\pi$
$\quotep{\pi}$

$\freenames{-} : \pi \to \mathcal{P}(\quotep{\pi})$

\begin{eqnarray*}
  \freenames{\pzero} & := & \emptyset \\
  \freenames{x?(y).P} & := & \{ x \} \cup (\freenames{P} \setminus \{ y \}) \\
  \freenames{x!\langle P \rangle} & := & \{ x \} \cup \{ P \} \\
  \freenames{P|Q} & := & \freenames{P} \cup \freenames{Q} \\
  \freenames{\dropn{x}} & := & \{ x \}
\end{eqnarray*}

The bound names of a process, $\boundnames{P}$, are those names occurring in $P$
that are not free. For example, in $x?(y).0$, the name $x$ is free, while $y$ is bound.

\begin{mathpar}
  \inferrule* [lab=monoidal-laws] {} { P|Q \equiv Q|P \and P|0 \equiv P \and P|(Q|R) \equiv (P|Q)|R }
\end{mathpar}

\begin{mathpar}
  \inferrule* [lab=alpha-equivalence] {} { (x)P \equiv (y)P\{y/x\} \and y \not\in \freenames{P} }
\end{mathpar}

\begin{definition}
Then two processes, $P,Q$, are alpha-equivalent if $P = Q\{\vec{y}/\vec{x}\}$ for
some $\vec{x} \in \boundnames{Q},\vec{y} \in \boundnames{P}$, where $Q\{\vec{y}/\vec{x}\}$
denotes the capture-avoiding substitution of $\vec{y}$ for $\vec{x}$ in $Q$.
\end{definition}

\begin{definition}
  The {\em structural congruence} \cite{SangiorgiWalker} , $\equiv$,
  between processes is the least congruence containing
  alpha-equivalence, satisfying the abelian monoid laws
  (associativity, commutativity and $\pzero$ as identity) for parallel
  composition $|$ and for summation $+$.
\end{definition}

\subsection{Name equivalence}

We take name equivalence, written $\nameeq$, to be the smallest
equivalence relation generated by the following rules.

\begin{mathpar}
\inferrule*[lab=Quote-drop]
{ }
{ \quotep{@{x}} \nameeq x }

\inferrule*[lab=Struct-equiv]
{ P \scong Q }
{ \quotep{P} \nameeq \quotep{Q} }
\end{mathpar}

The astute reader will have noticed that the mutual recursion of names
and processes imposes a mutual recursion on alpha-equivalence and
structural equivalence via name-equivalence. Fortunately, all of this
works out pleasantly and we may calculate in the natural way, free of
concern. The reader interested in the details is referred to the
appendix \ref{appendix:rho_details}.

\subsection{Substitution}

We use $\Proc$ for the set of processes, $\QProc$ for the set of
names, and $\id{\{}\vec{y} / \vec{x} \id{\}}$ to denote partial maps,
$s : \QProc \rightarrow \QProc$. A map, $s$ lifts, uniquely, to a map
on process terms, $\widehat{s} : \Proc \rightarrow \Proc$ by the
following equations.

\begin{mathpar}
  (0) \psubstp{Q}{P} := 0 \\
  (R \juxtap S) \psubstp{Q}{P}
  :=    
  (R)\psubstp{Q}{P} \juxtap (S) \psubstp{Q}{P} \\
  (x?(y).R) \psubstp{Q}{P}    
  :=    
  (x)\substp{Q}{P} (z)\concat( (R \psubstn{z}{y}) \psubstp{Q}{P} ) \\
  (\lift{x}{R}) \psubstp{Q}{P}  
  :=
  \lift{(x)\substp{Q}{P}}{ R \psubstp{Q}{P} } \\
%   (\dropn{x})  \psubstp{Q}{P}       
%   := 
%   \left\{ 
%     \begin{array}{ccc} 
%       \dropn{\quotep{Q}} & & x \nameeq \quotep{P} \\
%       \dropn{x} & & otherwise \\
%     \end{array}
%   \right. 
  (\dropn{x})  \psubstp{Q}{P}       
  := 
  \left\{ 
    \begin{array}{ccc} 
      Q & & x \nameeq \quotep{P} \\
      \dropn{x} & & otherwise \\
    \end{array}
  \right.
\end{mathpar}
 

where

\begin{eqnarray}
  (x)\id{\{} \lpquote Q \rpquote / \lpquote P \rpquote \id{\}}            = 
  \left\{ 
    \begin{array}{ccc}
      \lpquote Q \rpquote & & x \nameeq \lpquote P \rpquote \\
      x & & otherwise \\
    \end{array}
  \right. \nonumber
\end{eqnarray}

and $z$ is chosen distinct from $\quotep{P}$, $\quotep{Q}$, the free
names in $Q$, and all the names in $R$. Our $\alpha$-equivalence will
be built in the standard way from this substitution.

\begin{remark}\label{rem:no_self_referential_names}
  One consequence of these definitions is that $\forall P. \quotep{P}
  \not\in \freenames{P}$.
\end{remark}

\subsection{ Dynamic quote: an example }

Anticipating something of what's to come, consider applying the
substitution, $\widehat{\id{\{}u / z \id{\}}}$, to the following pair
of processes, $\lift{w}{y!(z)}$ and $w[ \lpquote y!(z) \rpquote ]$.

\begin{eqnarray}
	\lift{w}{y!(z)}\widehat{\id{\{}u / z \id{\}}}
		& = &
		\lift{w}{y!(u)} \nonumber\\
	w[ \lpquote y!(z) \rpquote ] \widehat{ \id{\{}u / z \id{\}} }
		& = &
		w[ \lpquote y!(z) \rpquote ] \nonumber
\end{eqnarray}

Because the body of the process between quotes is impervious to
substitution, we get radically different answers. In fact, by
examining the first process in an input context,
e.g. $x?(z).\lift{w}{y!(z)}$, we see that the process under the lift
operator may be shaped by prefixed inputs binding a name inside it. In
this sense, the lift operator will be seen as a way to dynamically
construct processes before reifying them as names.

Finally equipped with these standard features we can present the
dynamics of the calculus.

\subsubsection{Operational semantics} 

Finally, we introduce the computational dynamics. What marks these
algebras as distinct from other more traditionally studied algebraic
structures, e.g. vector spaces or polynomial rings, is the manner in
which dynamics is captured. In traditional structures, dynamics is typically
expressed through morphisms between such structures, as in linear maps
between vector spaces or morphisms between rings. In algebras
associated with the semantics of computation, the dynamics is
expressed as part of the algebraic structure itself, through a
reduction reduction relation typically denoted by $\red$. Below, we
give a recursive presentation of this relation for the calculus used
in the encoding.

$\red \subseteq \pi \times \pi$
$\red : \pi \to \mathcal{P}(\pi)$

\begin{mathpar}
  \inferrule* [lab=Comm] { \textsf{match}( x_{src}, x_{trgt} ) } { x_{trgt}?(y)P \; | \; x_{src}!\langle {Q} \rangle \red P\{\quotep{Q}/y}\} }
  \and \\
  \inferrule* [lab=Par] {{P} \red {P}'} {{{P} | {Q}} \red {{P}' | {Q}}}
  \and
  \inferrule* [lab=Equiv]{{{P} \scong {P}'} \andalso {{P}' \red {Q}'} \andalso {{Q}' \scong {Q}}}{{P} \red {Q}}
\end{mathpar}

\begin{eqnarray*}
  match_{\equiv} (\quotep{P},\quotep{Q}) & := & P \equiv Q \\
  match_{\dagger}(\quotep{P},\quotep{Q}) & := & \forall R. P|Q \red^{*} R => R \red^{*} 0 \\
  match_{K}(\quotep{P},\quotep{Q}) & := & K \mbox{ for some context } K
\end{eqnarray*}

$u?(x)P | u!\langle Q \rangle \red P\{\quotep{Q}/x\}$

%We write $\wred$ for $\red^*$, and $P\red$ if $\exists Q $ such that $ P \red Q$.
We write $P\red$ if $\exists Q $ such that $ P \red Q$ and $P\not\red$, otherwise.

\section{Replication}

As mentioned before, it is known that replication (and hence
recursion) can be implemented in a higher-order process algebra
\cite{SangiorgiWalker}. As our first example of calculation with the
machinery thus far presented we give the construction explicitly in
the {\rhoc}.

\begin{eqnarray}
	D_{x} & := & \prefix{x}{y}{(\binpar{\outputp{x}{y}}{@{y}})} \nonumber\\
	\bangp_{x}{P} & := & \binpar{{x}!\langle{\binpar{D_{x}}{P}}\rangle}{D_{x}} \nonumber
\end{eqnarray}

\begin{eqnarray}
	\bangp_{x}{P} & & \nonumber\\
	=
	& {x}!\langle{(\prefix{x}{y}{(\outputp{x}{y} | @{y})) | P}}\rangle 
	      | \prefix{x}{y}{(\outputp{x}{y} | @{y})} & \nonumber\\
	\red
	& (\outputp{x}{y} | @{y})\substn{\quotep{(\prefix{x}{y}{(@{y} | \outputp{x}{y})) | P}}}{y} & \nonumber\\
	=
	& \outputp{x}{\quotep{(\prefix{x}{y}{(\outputp{x}{y} | @{y})) | P}}}
	  | {(\prefix{x}{y}{(\outputp{x}{y} | @{y})) | P}} & \nonumber\\
	\red
	& \ldots & \nonumber\\
	\red^*
	& P | P | \ldots & \nonumber
\end{eqnarray}

Of course, this encoding, as an implementation, runs away, unfolding
$\bangp{P}$ eagerly. A lazier and more implementable replication
operator, restricted to input-guarded processes, may be obtained as follows.

\begin{eqnarray}
\bangp{\prefix{u}{v}{P}} 
	:= 
	\binpar{\lift{x}{\prefix{u}{v}{(\binpar{D(x)}{P})}}}{D(x)} \nonumber
\end{eqnarray}

\begin{remark}
  Note that the lazier definition still does not deal with summation
  or mixed summation (i.e. sums over input and output). The reader is
  invited to construct definitions of replication that deal with these
  features. 

  Further, the definitions are parameterized in a name, $x$. Can you,
  gentle reader, make a definition that eliminates this parameter and
  guarantees no accidental interaction between the replication
  machinery and the process being replicated -- i.e. no accidental
  sharing of names used by the process to get its work done and the
  name(s) used by the replication to effect copying. This latter
  revision of the definition of replication is crucial to obtaining
  the expected identity $!!P \sim !P$.
\end{remark}

\begin{remark}\label{rem:paradoxical_combinator}
  The reader familiar with the lambda calculus will have noticed the
  similarity between $D$ and the paradoxical combinator.

  [Ed. note: the existence of this seems to suggest we have to be more
  restrictive on the set of processes and names we admit if we are to
  support no-cloning.]
\end{remark}

\subsubsection{Bisimulation}

The computational dynamics gives rise to another kind of equivalence,
the equivalence of computational behavior. As previously mentioned
this is typically captured \emph{via} some form of bisimulation.

% The notion we use in this paper is weak barbed bisimulation
% \cite{milner91polyadicpi}.

The notion we use in this paper is derived from weak barbed
bisimulation \cite{milner91polyadicpi}. 

\begin{definition}
An \emph{observation relation}, $\downarrow_{\mathcal N}$, over a set
of names, $\mathcal N$, is the smallest relation satisfying the rules
below.

\infrule[Out-barb]{y \in {\mathcal N}, \; x \nameeq y}
		  {\outputp{x}{v} \downarrow_{\mathcal N} x}
\infrule[Par-barb]{\mbox{$P\downarrow_{\mathcal N} x$ or $Q\downarrow_{\mathcal N} x$}}
		  {\binpar{P}{Q} \downarrow_{\mathcal N} x}

We write $P \Downarrow_{\mathcal N} x$ if there is $Q$ such that 
$P \wred Q$ and $Q \downarrow_{\mathcal N} x$.
\end{definition}

\begin{definition}
%\label{def.bbisim}
An  ${\mathcal N}$-\emph{barbed bisimulation} over a set of names, ${\mathcal N}$, is a symmetric binary relation 
${\mathcal S}_{\mathcal N}$ between agents such that $P\rel{S}_{\mathcal N}Q$ implies:
\begin{enumerate}
\item If $P \red P'$ then $Q \wred Q'$ and $P'\rel{S}_{\mathcal N} Q'$.
\item If $P\downarrow_{\mathcal N} x$, then $Q\Downarrow_{\mathcal N} x$.
\end{enumerate}
$P$ is ${\mathcal N}$-barbed bisimilar to $Q$, written
$P \wbbisim_{\mathcal N} Q$, if $P \rel{S}_{\mathcal N} Q$ for some ${\mathcal N}$-barbed bisimulation ${\mathcal S}_{\mathcal N}$.
\end{definition}

$\mathcal{R} \subseteq \pi \times \pi$

$P \mathcal{R} Q => \forall P'. P \red P' \Rightarrow \exists Q'. Q \red Q', P' \mathcal{R} Q'$

$P \vdash x \Rightarrow Q \vdash x$

\begin{mathpar}
  \inferrule*[lab=Out-barb]{x \nameeq y}{{y}!\langle{Q}\rangle \vdash x}
  \and
  \inferrule*[lab=Par-barb]{\mbox{$P\vdash x$ or $Q\vdash x$}}{\binpar{P}{Q} \vdash x}
\end{mathpar}

\subsubsection{Contexts}

One of the principle advantages of computational calculi like the
$\pi$-calculus is a well-defined notion of context,
contextual-equivalence and a correlation between
contextual-equivalence and notions of bisimulation. The notion of
context allows the decomposition of a process into (sub-)process and
its syntactic environment, its context. Thus, a context may be
thought of as a process with a ``hole'' (written $\Box$) in it. The
application of a context $M$ to a process $P$, written $M[P]$, is
tantamount to filling the hole in $M$ with $P$. In this paper we do
not need the full weight of this theory, but do make use of the notion
of context in the proof the main theorem. 

\begin{mathpar}
  \inferrule* [lab=summation] {} {{M_{M},M_{N}} \bc \Box \;|\; x.M_{A} \;|\; M_{M}+M_{N}}
  \and
  \inferrule* [lab=agent] {} {{M_{A}} \bc (\vec{x})M_{P} \;| \; \clift{P_0,\ldots,M_{P},\ldots,P_N}}
  \and \\
  \inferrule* [lab=process] {} {{M_{P}} \bc M_{N} \;| \;P|M_{P} }
\end{mathpar} 

\begin{mathpar}
  \inferrule* [lab=sychronization] {} {M_{N} \bc \Box \;|\; x?M_{F} \;|\; x!M_{C}}
  \and
  \inferrule* [lab=abstraction] {} {{M_{F}} \bc (x)M_{P} }
  \and
  \inferrule* [lab=concretion] {} {{M_{C}} \bc \langle M_{P} \rangle }
  \and \\
  \inferrule* [lab=process] {} {{M_{P}} \bc M_{N} \;| \;P|M_{P} }
\end{mathpar}

\begin{definition}[contextual application] Given a context $M$, and
  process $P$, we define the \emph{contextual application}, $M[P] :=
  M\{P/\Box\}$. That is, the contextual application of M to P is the
  substitution of $P$ for $\Box$ in $M$.
\end{definition}

$\meaningof{-} : L \to \mathcal{P}(\pi)$

\begin{mathpar}
  \inferrule* [lab=collection] {} {\meaningof{true} = \pi, \and \meaningof{~E} = \pi \setminus \meaningof{E}, \and \meaningof{E_{1} \& E_{2}} = \meaningof{E_{1}} \cap \meaningof{E_{2}}}
\end{mathpar}

\begin{mathpar}
  \inferrule* [lab=structure] {} {\meaningof{0} = \{ P \in \pi | P \equiv 0 \}, \and \\ \meaningof{E_1 | E_2} = \{ P \in \pi | P \equiv P_{1} | P_{2}, P_{1} \in \meaningof{E_{1}}, P_{2} \in \meaningof{E_2}\} }
\end{mathpar}

\begin{mathpar}
 \inferrule* [lab=behavior] {} {\meaningof{\langle a?b \rangle E} = \{ P \in \pi | P \equiv Q | u?(y)P', \\ \and \\\\ \and \\ \;\;\; u \in \meaningof{a}, \forall z.P'\{z/y\} \in \meaningof{E\{z/b\}}\}, \and \\ \meaningof{a!E} = \{ P \in \pi | P \equiv Q | x!\langle P' \rangle, x \in \meaningof{a} P' \in \meaningof{E}\} }
\end{mathpar}

\begin{mathpar}
 \inferrule* [lab=nominal] {} {\meaningof{\quotep{E}} = \{ \quotep{P} \in \quotep{\pi} | P \in \meaningof{E} \}, \and \meaningof{\quotep{P}} = \{ \quotep{Q} \in \quotep{\pi} | P \equiv Q \} \and \\ \meaningof{@\quotep{E}} = \{ P \in \pi | P \equiv @x, x \in \meaningof{E} \}}
\end{mathpar}

\begin{eqnarray*}
  \\
  \meaningof{-} : TS \to ST
\end{eqnarray*}

\begin{eqnarray*}
  \\
  L : TS \to ST
\end{eqnarray*}

\begin{eqnarray*}
  \\
  P \models E \iff P \in \meaningof{E}
\end{eqnarray*}

\begin{eqnarray*}
  P \approx_{L} Q \iff \forall E \in L. P \models E \iff Q \models E
\end{eqnarray*}

\begin{eqnarray*}
  P \approx_{K} Q
\end{eqnarray*}

\begin{eqnarray*}
  P \approx Q
\end{eqnarray*}

$\approx_{K} = \approx = \approx_{L}$

\subsubsection{Contextual duality}

Note that contexts extend the quotation operation to a family of
operations from processes to names. Given a context, $M$, we can
define a \emph{nominal context}, $\quotep{M}$ by $\quotep{M}[P] :=
\quotep{M[P]}$. To foreshadow what is to come we observe that these
operations enjoy a duality with processes very much like the duality
between vectors and maps from vectors to scalars.

Further, because the calculus is essentially higher-order, we have a
correspondence between contexts and processes. More specifically,
given a name $x$ and a context $M$ we can construct $M^{*}_{x}$ such
that 

\begin{mathpar}
  M^{*}_{x} | \lift{x}{P} \red M[P]
\end{mathpar}

namely,

\begin{mathpar}
  M^{*}_{x} := x?(u).M[\dropn{u}]
\end{mathpar}

The dependence of $M^{*}_{x}$ on a name makes it an abstraction, 

\begin{mathpar}
  M^{*} := (x)x?(u).M[\dropn{u}]
\end{mathpar}

\subsection{Additional notation}

It will sometimes be convenient to denote the process a name
quotes. We already have the notation $x = \quotep{P}$, but it will be
convenient to introduce an alternate notation, $\procn{x}$, when we
want to emphasize the connection to the use of the name. Note that, by
virtue of name equivalence, $\quotep{\procn{x}} \nameeq x$; so, the
notation is consistent with previous definitions.

Further, because names have structure it is possible to effect
substitutions on the basis of that structure. This means we need to
upgrade our notation for substitutions, which we accomplish by
adapting comprehension notation. Thus,

\begin{mathpar}
  P\{ y / x : x \in S \}
\end{mathpar}

is interpreted to mean the process derived from P by replacing (in a
capture-avoiding manner) each occurrence of $x$ in $S$ by $y$. For example,

\begin{mathpar}
  P\{ \quotep{\procn{x}|\procn{x}} / x : x \in \freenames{P} \}
\end{mathpar}

will replace each (occurrence) of a free name $x$ in $P$ by
$\quotep{\procn{x}|\procn{x}}$.

Also, we will avail ourselves of the notation $x^{L}$ and $x^{R}$ to
denote injections of a name into disjoint copies of the name
space. There are numerous ways to accomplish this. One example can be
found in \cite{MeredithR05}. This notation overloads to vectors of
names: $\vec{x}^{\pi} := (x_{i}^{\pi} \; : \; 0 \leq i < |\vec{x}| )$ where $\pi \in \{L,R\}$.

We also use $P^{\Box} := P|\Box$.

In \cite{MeredithR05} an interpretation of the new operator is
given. It turns out that there are several possible interpretations
all enjoying the requisite algebraic properties of the operator (see
\cite{milner91polyadicpi}). We will therefore make liberal use of
$(\nu\; \vec{x})P$.

% subsection the_syntax_and_semantics_of_the_notation_system (end)   

\input{qm2pi.qmops} 

\input{qm2pi.sterngerlach} 

\input{qm2pi.metric} 

% section concurrent_process_calculi (end)

%\input{qm2pi.proofsketch}

% section proof sketch (end)

%\input{qm2pi.slviaknots} 

% section spatial logic via knots (end)

\input{qm2pi.conclusion}

% section conclusion (end)

%\input{qm2pi.dtcodes} 

% section wiring algorithm (end)

\input{qm2pi.ack} 

% section acknowledgments (end)

\newpage


\bibliographystyle{plain}   
\bibliography{../../biblios/main.bib}

\input{qm2pi.rhodetails}

\end{document}

 

%\documentclass[12pt]{llncs}
%\documentclass{jktr}

\usepackage[pdftex]{hyperref}                   
\usepackage {listings}
\usepackage {mathpartir}
\usepackage{bcprules}
%\usepackage{listings}
                       
\usepackage{graphicx} 
%\usepackage[margins=2.5cm,nohead,nofoot]{geometry}
%\usepackage{geometry}
\usepackage{amsfonts}
\usepackage{amstext}
\usepackage{latexsym}
\usepackage{amssymb}
\usepackage{color}


%\include{myPreamble}
\include{qm2pi.local} 

%\ifpdf
%\usepackage[pdftex]{graphicx}
%\else
%\usepackage{graphicx}
%\fi

 % \ifpdf
%  \usepackage{pdfsync}
%  \if


%\title{Brief Article}
%\author{David F. Snyder}
%\author{L.G. Meredith}

%\address{Dept. of Math., Texas State University--San Marcos, San Marcos, TX 78666}
       
\pagestyle{empty}


\begin{document}

\lstset{language=[Objective]Caml,frame=shadowbox}

\input{qm2pi.front}

% section front matter (end)

\input{qm2pi.intro} 
 
% section introduction (end)

% \input{qm2pi.knotations} 

% section notation (end)

\input{qm2pi.process.calculi} 

% section concurrent_process_calculi_and_spatial_logics_ (end)
    
%\input{qm2pi.knots2pi} 

%\input{qm2pi.trefoil} 

%\input{qm2pi.mainthm} 

% subsection basic_interpretation (end)

%\input{qm2pi.rho.presentation} 
\subsection{The syntax and semantics of the notation system}\label{sub:the_syntax_and_semantics_of_the_notation_system} % (fold)

We now summarize a technical presentation of the calculus that
embodies our theory of dynamics. The typical presentation of such a
calculus follows the style of giving generators and relations on
them. The grammar, below, describing term constructors, freely
generates the set of processes, $\Proc$. This set is then quotiented
by a relation known as structural congruence and it is over this set
that the notion of dynamics is expressed. This presentation is
essentially that of \cite{MeredithR05} with the addition of
polyadicity and summation. For readability we have relegated some of
the technical subtleties to an appendix.

\subsubsection{Process grammar}\label{subsub:process_grammar}

\begin{mathpar}
  \inferrule* [lab=synchronization] {} {{M} \bc \pzero \;|\; x?F \;|\; x!C }
  \and
  \inferrule* [lab=abstraction] {} {{F} \bc (x)P}
  \and
  \inferrule* [lab=concretion] {} {{C} \bc \langle Q \rangle}
  \and
  \inferrule* [lab=process] {} {{P,Q} \bc M \;| \;P|Q \;|\; @{x}}
  \and
  \inferrule* [lab=name] {} {{x} \bc \quotep{P}}
\end{mathpar} 

Note that $\vec{x}$ (resp. $\vec{P}$) denotes a vector of names
(resp. processes) of length $|\vec{x}|$ (resp. $|\vec{P}|$). We adopt
the following useful abbreviations.

\begin{mathpar}
   x?(\vec{y}).P := x.(\vec{y})P \and  x\clift{\vec{P}} := x.\clift{\vec{P}}
   \and x!(y) := \lift{x}{\dropn{y}}
   \and \Pi_{i=0}^{n-1}P_i := P_0 | \ldots | P_{n-1}
\end{mathpar}

\subsubsection{Structural congruence}

\paragraph{Free and bound names and alpha-equivalence.} At the
core of structural equivalence is alpha-equivalence which identifies
process that are the same up to a change of variable. Formally, we
recognize the distinction between free and bound names. The free names
of a process, $\freenames{P}$, may be calculated recursively as
follows:

\begin{mathpar}
\freenames{\pzero} := \emptyset
  \and \\
  \freenames{x?(y).P} := \{ x \} \cup (\freenames{P} \setminus \{ y \})
  \and 
  \freenames{x!\langle P \rangle} := \{ x \} \cup \{ P \} 
  \and \\
  \freenames{P|Q} := \freenames{P} \cup \freenames{Q}
  \and \\
  \freenames{@{x}} := \{ x \}
\end{mathpar}

$\pi$
$\quotep{\pi}$

$\freenames{-} : \pi \to \mathcal{P}(\quotep{\pi})$

\begin{eqnarray*}
  \freenames{\pzero} & := & \emptyset \\
  \freenames{x?(y).P} & := & \{ x \} \cup (\freenames{P} \setminus \{ y \}) \\
  \freenames{x!\langle P \rangle} & := & \{ x \} \cup \{ P \} \\
  \freenames{P|Q} & := & \freenames{P} \cup \freenames{Q} \\
  \freenames{\dropn{x}} & := & \{ x \}
\end{eqnarray*}

The bound names of a process, $\boundnames{P}$, are those names occurring in $P$
that are not free. For example, in $x?(y).0$, the name $x$ is free, while $y$ is bound.

\begin{mathpar}
  \inferrule* [lab=monoidal-laws] {} { P|Q \equiv Q|P \and P|0 \equiv P \and P|(Q|R) \equiv (P|Q)|R }
\end{mathpar}

\begin{mathpar}
  \inferrule* [lab=alpha-equivalence] {} { (x)P \equiv (y)P\{y/x\} \and y \not\in \freenames{P} }
\end{mathpar}

\begin{definition}
Then two processes, $P,Q$, are alpha-equivalent if $P = Q\{\vec{y}/\vec{x}\}$ for
some $\vec{x} \in \boundnames{Q},\vec{y} \in \boundnames{P}$, where $Q\{\vec{y}/\vec{x}\}$
denotes the capture-avoiding substitution of $\vec{y}$ for $\vec{x}$ in $Q$.
\end{definition}

\begin{definition}
  The {\em structural congruence} \cite{SangiorgiWalker} , $\equiv$,
  between processes is the least congruence containing
  alpha-equivalence, satisfying the abelian monoid laws
  (associativity, commutativity and $\pzero$ as identity) for parallel
  composition $|$ and for summation $+$.
\end{definition}

\subsection{Name equivalence}

We take name equivalence, written $\nameeq$, to be the smallest
equivalence relation generated by the following rules.

\begin{mathpar}
\inferrule*[lab=Quote-drop]
{ }
{ \quotep{@{x}} \nameeq x }

\inferrule*[lab=Struct-equiv]
{ P \scong Q }
{ \quotep{P} \nameeq \quotep{Q} }
\end{mathpar}

The astute reader will have noticed that the mutual recursion of names
and processes imposes a mutual recursion on alpha-equivalence and
structural equivalence via name-equivalence. Fortunately, all of this
works out pleasantly and we may calculate in the natural way, free of
concern. The reader interested in the details is referred to the
appendix \ref{appendix:rho_details}.

\subsection{Substitution}

We use $\Proc$ for the set of processes, $\QProc$ for the set of
names, and $\id{\{}\vec{y} / \vec{x} \id{\}}$ to denote partial maps,
$s : \QProc \rightarrow \QProc$. A map, $s$ lifts, uniquely, to a map
on process terms, $\widehat{s} : \Proc \rightarrow \Proc$ by the
following equations.

\begin{mathpar}
  (0) \psubstp{Q}{P} := 0 \\
  (R \juxtap S) \psubstp{Q}{P}
  :=    
  (R)\psubstp{Q}{P} \juxtap (S) \psubstp{Q}{P} \\
  (x?(y).R) \psubstp{Q}{P}    
  :=    
  (x)\substp{Q}{P} (z)\concat( (R \psubstn{z}{y}) \psubstp{Q}{P} ) \\
  (\lift{x}{R}) \psubstp{Q}{P}  
  :=
  \lift{(x)\substp{Q}{P}}{ R \psubstp{Q}{P} } \\
%   (\dropn{x})  \psubstp{Q}{P}       
%   := 
%   \left\{ 
%     \begin{array}{ccc} 
%       \dropn{\quotep{Q}} & & x \nameeq \quotep{P} \\
%       \dropn{x} & & otherwise \\
%     \end{array}
%   \right. 
  (\dropn{x})  \psubstp{Q}{P}       
  := 
  \left\{ 
    \begin{array}{ccc} 
      Q & & x \nameeq \quotep{P} \\
      \dropn{x} & & otherwise \\
    \end{array}
  \right.
\end{mathpar}
 

where

\begin{eqnarray}
  (x)\id{\{} \lpquote Q \rpquote / \lpquote P \rpquote \id{\}}            = 
  \left\{ 
    \begin{array}{ccc}
      \lpquote Q \rpquote & & x \nameeq \lpquote P \rpquote \\
      x & & otherwise \\
    \end{array}
  \right. \nonumber
\end{eqnarray}

and $z$ is chosen distinct from $\quotep{P}$, $\quotep{Q}$, the free
names in $Q$, and all the names in $R$. Our $\alpha$-equivalence will
be built in the standard way from this substitution.

\begin{remark}\label{rem:no_self_referential_names}
  One consequence of these definitions is that $\forall P. \quotep{P}
  \not\in \freenames{P}$.
\end{remark}

\subsection{ Dynamic quote: an example }

Anticipating something of what's to come, consider applying the
substitution, $\widehat{\id{\{}u / z \id{\}}}$, to the following pair
of processes, $\lift{w}{y!(z)}$ and $w[ \lpquote y!(z) \rpquote ]$.

\begin{eqnarray}
	\lift{w}{y!(z)}\widehat{\id{\{}u / z \id{\}}}
		& = &
		\lift{w}{y!(u)} \nonumber\\
	w[ \lpquote y!(z) \rpquote ] \widehat{ \id{\{}u / z \id{\}} }
		& = &
		w[ \lpquote y!(z) \rpquote ] \nonumber
\end{eqnarray}

Because the body of the process between quotes is impervious to
substitution, we get radically different answers. In fact, by
examining the first process in an input context,
e.g. $x?(z).\lift{w}{y!(z)}$, we see that the process under the lift
operator may be shaped by prefixed inputs binding a name inside it. In
this sense, the lift operator will be seen as a way to dynamically
construct processes before reifying them as names.

Finally equipped with these standard features we can present the
dynamics of the calculus.

\subsubsection{Operational semantics} 

Finally, we introduce the computational dynamics. What marks these
algebras as distinct from other more traditionally studied algebraic
structures, e.g. vector spaces or polynomial rings, is the manner in
which dynamics is captured. In traditional structures, dynamics is typically
expressed through morphisms between such structures, as in linear maps
between vector spaces or morphisms between rings. In algebras
associated with the semantics of computation, the dynamics is
expressed as part of the algebraic structure itself, through a
reduction reduction relation typically denoted by $\red$. Below, we
give a recursive presentation of this relation for the calculus used
in the encoding.

$\red \subseteq \pi \times \pi$
$\red : \pi \to \mathcal{P}(\pi)$

\begin{mathpar}
  \inferrule* [lab=Comm] { \textsf{match}( x_{src}, x_{trgt} ) } { x_{trgt}?(y)P \; | \; x_{src}!\langle {Q} \rangle \red P\{\quotep{Q}/y}\} }
  \and \\
  \inferrule* [lab=Par] {{P} \red {P}'} {{{P} | {Q}} \red {{P}' | {Q}}}
  \and
  \inferrule* [lab=Equiv]{{{P} \scong {P}'} \andalso {{P}' \red {Q}'} \andalso {{Q}' \scong {Q}}}{{P} \red {Q}}
\end{mathpar}

\begin{eqnarray*}
  match_{\equiv} (\quotep{P},\quotep{Q}) & := & P \equiv Q \\
  match_{\dagger}(\quotep{P},\quotep{Q}) & := & \forall R. P|Q \red^{*} R => R \red^{*} 0 \\
  match_{K}(\quotep{P},\quotep{Q}) & := & K \mbox{ for some context } K
\end{eqnarray*}

$u?(x)P | u!\langle Q \rangle \red P\{\quotep{Q}/x\}$

%We write $\wred$ for $\red^*$, and $P\red$ if $\exists Q $ such that $ P \red Q$.
We write $P\red$ if $\exists Q $ such that $ P \red Q$ and $P\not\red$, otherwise.

\section{Replication}

As mentioned before, it is known that replication (and hence
recursion) can be implemented in a higher-order process algebra
\cite{SangiorgiWalker}. As our first example of calculation with the
machinery thus far presented we give the construction explicitly in
the {\rhoc}.

\begin{eqnarray}
	D_{x} & := & \prefix{x}{y}{(\binpar{\outputp{x}{y}}{@{y}})} \nonumber\\
	\bangp_{x}{P} & := & \binpar{{x}!\langle{\binpar{D_{x}}{P}}\rangle}{D_{x}} \nonumber
\end{eqnarray}

\begin{eqnarray}
	\bangp_{x}{P} & & \nonumber\\
	=
	& {x}!\langle{(\prefix{x}{y}{(\outputp{x}{y} | @{y})) | P}}\rangle 
	      | \prefix{x}{y}{(\outputp{x}{y} | @{y})} & \nonumber\\
	\red
	& (\outputp{x}{y} | @{y})\substn{\quotep{(\prefix{x}{y}{(@{y} | \outputp{x}{y})) | P}}}{y} & \nonumber\\
	=
	& \outputp{x}{\quotep{(\prefix{x}{y}{(\outputp{x}{y} | @{y})) | P}}}
	  | {(\prefix{x}{y}{(\outputp{x}{y} | @{y})) | P}} & \nonumber\\
	\red
	& \ldots & \nonumber\\
	\red^*
	& P | P | \ldots & \nonumber
\end{eqnarray}

Of course, this encoding, as an implementation, runs away, unfolding
$\bangp{P}$ eagerly. A lazier and more implementable replication
operator, restricted to input-guarded processes, may be obtained as follows.

\begin{eqnarray}
\bangp{\prefix{u}{v}{P}} 
	:= 
	\binpar{\lift{x}{\prefix{u}{v}{(\binpar{D(x)}{P})}}}{D(x)} \nonumber
\end{eqnarray}

\begin{remark}
  Note that the lazier definition still does not deal with summation
  or mixed summation (i.e. sums over input and output). The reader is
  invited to construct definitions of replication that deal with these
  features. 

  Further, the definitions are parameterized in a name, $x$. Can you,
  gentle reader, make a definition that eliminates this parameter and
  guarantees no accidental interaction between the replication
  machinery and the process being replicated -- i.e. no accidental
  sharing of names used by the process to get its work done and the
  name(s) used by the replication to effect copying. This latter
  revision of the definition of replication is crucial to obtaining
  the expected identity $!!P \sim !P$.
\end{remark}

\begin{remark}\label{rem:paradoxical_combinator}
  The reader familiar with the lambda calculus will have noticed the
  similarity between $D$ and the paradoxical combinator.

  [Ed. note: the existence of this seems to suggest we have to be more
  restrictive on the set of processes and names we admit if we are to
  support no-cloning.]
\end{remark}

\subsubsection{Bisimulation}

The computational dynamics gives rise to another kind of equivalence,
the equivalence of computational behavior. As previously mentioned
this is typically captured \emph{via} some form of bisimulation.

% The notion we use in this paper is weak barbed bisimulation
% \cite{milner91polyadicpi}.

The notion we use in this paper is derived from weak barbed
bisimulation \cite{milner91polyadicpi}. 

\begin{definition}
An \emph{observation relation}, $\downarrow_{\mathcal N}$, over a set
of names, $\mathcal N$, is the smallest relation satisfying the rules
below.

\infrule[Out-barb]{y \in {\mathcal N}, \; x \nameeq y}
		  {\outputp{x}{v} \downarrow_{\mathcal N} x}
\infrule[Par-barb]{\mbox{$P\downarrow_{\mathcal N} x$ or $Q\downarrow_{\mathcal N} x$}}
		  {\binpar{P}{Q} \downarrow_{\mathcal N} x}

We write $P \Downarrow_{\mathcal N} x$ if there is $Q$ such that 
$P \wred Q$ and $Q \downarrow_{\mathcal N} x$.
\end{definition}

\begin{definition}
%\label{def.bbisim}
An  ${\mathcal N}$-\emph{barbed bisimulation} over a set of names, ${\mathcal N}$, is a symmetric binary relation 
${\mathcal S}_{\mathcal N}$ between agents such that $P\rel{S}_{\mathcal N}Q$ implies:
\begin{enumerate}
\item If $P \red P'$ then $Q \wred Q'$ and $P'\rel{S}_{\mathcal N} Q'$.
\item If $P\downarrow_{\mathcal N} x$, then $Q\Downarrow_{\mathcal N} x$.
\end{enumerate}
$P$ is ${\mathcal N}$-barbed bisimilar to $Q$, written
$P \wbbisim_{\mathcal N} Q$, if $P \rel{S}_{\mathcal N} Q$ for some ${\mathcal N}$-barbed bisimulation ${\mathcal S}_{\mathcal N}$.
\end{definition}

$\mathcal{R} \subseteq \pi \times \pi$

$P \mathcal{R} Q => \forall P'. P \red P' \Rightarrow \exists Q'. Q \red Q', P' \mathcal{R} Q'$

$P \vdash x \Rightarrow Q \vdash x$

\begin{mathpar}
  \inferrule*[lab=Out-barb]{x \nameeq y}{{y}!\langle{Q}\rangle \vdash x}
  \and
  \inferrule*[lab=Par-barb]{\mbox{$P\vdash x$ or $Q\vdash x$}}{\binpar{P}{Q} \vdash x}
\end{mathpar}

\subsubsection{Contexts}

One of the principle advantages of computational calculi like the
$\pi$-calculus is a well-defined notion of context,
contextual-equivalence and a correlation between
contextual-equivalence and notions of bisimulation. The notion of
context allows the decomposition of a process into (sub-)process and
its syntactic environment, its context. Thus, a context may be
thought of as a process with a ``hole'' (written $\Box$) in it. The
application of a context $M$ to a process $P$, written $M[P]$, is
tantamount to filling the hole in $M$ with $P$. In this paper we do
not need the full weight of this theory, but do make use of the notion
of context in the proof the main theorem. 

\begin{mathpar}
  \inferrule* [lab=summation] {} {{M_{M},M_{N}} \bc \Box \;|\; x.M_{A} \;|\; M_{M}+M_{N}}
  \and
  \inferrule* [lab=agent] {} {{M_{A}} \bc (\vec{x})M_{P} \;| \; \clift{P_0,\ldots,M_{P},\ldots,P_N}}
  \and \\
  \inferrule* [lab=process] {} {{M_{P}} \bc M_{N} \;| \;P|M_{P} }
\end{mathpar} 

\begin{mathpar}
  \inferrule* [lab=sychronization] {} {M_{N} \bc \Box \;|\; x?M_{F} \;|\; x!M_{C}}
  \and
  \inferrule* [lab=abstraction] {} {{M_{F}} \bc (x)M_{P} }
  \and
  \inferrule* [lab=concretion] {} {{M_{C}} \bc \langle M_{P} \rangle }
  \and \\
  \inferrule* [lab=process] {} {{M_{P}} \bc M_{N} \;| \;P|M_{P} }
\end{mathpar}

\begin{definition}[contextual application] Given a context $M$, and
  process $P$, we define the \emph{contextual application}, $M[P] :=
  M\{P/\Box\}$. That is, the contextual application of M to P is the
  substitution of $P$ for $\Box$ in $M$.
\end{definition}

$\meaningof{-} : L \to \mathcal{P}(\pi)$

\begin{mathpar}
  \inferrule* [lab=collection] {} {\meaningof{true} = \pi, \and \meaningof{~E} = \pi \setminus \meaningof{E}, \and \meaningof{E_{1} \& E_{2}} = \meaningof{E_{1}} \cap \meaningof{E_{2}}}
\end{mathpar}

\begin{mathpar}
  \inferrule* [lab=structure] {} {\meaningof{0} = \{ P \in \pi | P \equiv 0 \}, \and \\ \meaningof{E_1 | E_2} = \{ P \in \pi | P \equiv P_{1} | P_{2}, P_{1} \in \meaningof{E_{1}}, P_{2} \in \meaningof{E_2}\} }
\end{mathpar}

\begin{mathpar}
 \inferrule* [lab=behavior] {} {\meaningof{\langle a?b \rangle E} = \{ P \in \pi | P \equiv Q | u?(y)P', \\ \and \\\\ \and \\ \;\;\; u \in \meaningof{a}, \forall z.P'\{z/y\} \in \meaningof{E\{z/b\}}\}, \and \\ \meaningof{a!E} = \{ P \in \pi | P \equiv Q | x!\langle P' \rangle, x \in \meaningof{a} P' \in \meaningof{E}\} }
\end{mathpar}

\begin{mathpar}
 \inferrule* [lab=nominal] {} {\meaningof{\quotep{E}} = \{ \quotep{P} \in \quotep{\pi} | P \in \meaningof{E} \}, \and \meaningof{\quotep{P}} = \{ \quotep{Q} \in \quotep{\pi} | P \equiv Q \} \and \\ \meaningof{@\quotep{E}} = \{ P \in \pi | P \equiv @x, x \in \meaningof{E} \}}
\end{mathpar}

\begin{eqnarray*}
  \\
  \meaningof{-} : TS \to ST
\end{eqnarray*}

\begin{eqnarray*}
  \\
  L : TS \to ST
\end{eqnarray*}

\begin{eqnarray*}
  \\
  P \models E \iff P \in \meaningof{E}
\end{eqnarray*}

\begin{eqnarray*}
  P \approx_{L} Q \iff \forall E \in L. P \models E \iff Q \models E
\end{eqnarray*}

\begin{eqnarray*}
  P \approx_{K} Q
\end{eqnarray*}

\begin{eqnarray*}
  P \approx Q
\end{eqnarray*}

$\approx_{K} = \approx = \approx_{L}$

\subsubsection{Contextual duality}

Note that contexts extend the quotation operation to a family of
operations from processes to names. Given a context, $M$, we can
define a \emph{nominal context}, $\quotep{M}$ by $\quotep{M}[P] :=
\quotep{M[P]}$. To foreshadow what is to come we observe that these
operations enjoy a duality with processes very much like the duality
between vectors and maps from vectors to scalars.

Further, because the calculus is essentially higher-order, we have a
correspondence between contexts and processes. More specifically,
given a name $x$ and a context $M$ we can construct $M^{*}_{x}$ such
that 

\begin{mathpar}
  M^{*}_{x} | \lift{x}{P} \red M[P]
\end{mathpar}

namely,

\begin{mathpar}
  M^{*}_{x} := x?(u).M[\dropn{u}]
\end{mathpar}

The dependence of $M^{*}_{x}$ on a name makes it an abstraction, 

\begin{mathpar}
  M^{*} := (x)x?(u).M[\dropn{u}]
\end{mathpar}

\subsection{Additional notation}

It will sometimes be convenient to denote the process a name
quotes. We already have the notation $x = \quotep{P}$, but it will be
convenient to introduce an alternate notation, $\procn{x}$, when we
want to emphasize the connection to the use of the name. Note that, by
virtue of name equivalence, $\quotep{\procn{x}} \nameeq x$; so, the
notation is consistent with previous definitions.

Further, because names have structure it is possible to effect
substitutions on the basis of that structure. This means we need to
upgrade our notation for substitutions, which we accomplish by
adapting comprehension notation. Thus,

\begin{mathpar}
  P\{ y / x : x \in S \}
\end{mathpar}

is interpreted to mean the process derived from P by replacing (in a
capture-avoiding manner) each occurrence of $x$ in $S$ by $y$. For example,

\begin{mathpar}
  P\{ \quotep{\procn{x}|\procn{x}} / x : x \in \freenames{P} \}
\end{mathpar}

will replace each (occurrence) of a free name $x$ in $P$ by
$\quotep{\procn{x}|\procn{x}}$.

Also, we will avail ourselves of the notation $x^{L}$ and $x^{R}$ to
denote injections of a name into disjoint copies of the name
space. There are numerous ways to accomplish this. One example can be
found in \cite{MeredithR05}. This notation overloads to vectors of
names: $\vec{x}^{\pi} := (x_{i}^{\pi} \; : \; 0 \leq i < |\vec{x}| )$ where $\pi \in \{L,R\}$.

We also use $P^{\Box} := P|\Box$.

In \cite{MeredithR05} an interpretation of the new operator is
given. It turns out that there are several possible interpretations
all enjoying the requisite algebraic properties of the operator (see
\cite{milner91polyadicpi}). We will therefore make liberal use of
$(\nu\; \vec{x})P$.

% subsection the_syntax_and_semantics_of_the_notation_system (end)   

\input{qm2pi.qmops} 

\input{qm2pi.sterngerlach} 

\input{qm2pi.metric} 

% section concurrent_process_calculi (end)

%\input{qm2pi.proofsketch}

% section proof sketch (end)

%\input{qm2pi.slviaknots} 

% section spatial logic via knots (end)

\input{qm2pi.conclusion}

% section conclusion (end)

%\input{qm2pi.dtcodes} 

% section wiring algorithm (end)

\input{qm2pi.ack} 

% section acknowledgments (end)

\newpage


\bibliographystyle{plain}   
\bibliography{../../biblios/main.bib}

\input{qm2pi.rhodetails}

\end{document}

 

%\documentclass[12pt]{llncs}
%\documentclass{jktr}

\usepackage[pdftex]{hyperref}                   
\usepackage {listings}
\usepackage {mathpartir}
\usepackage{bcprules}
%\usepackage{listings}
                       
\usepackage{graphicx} 
%\usepackage[margins=2.5cm,nohead,nofoot]{geometry}
%\usepackage{geometry}
\usepackage{amsfonts}
\usepackage{amstext}
\usepackage{latexsym}
\usepackage{amssymb}
\usepackage{color}


%\include{myPreamble}
\include{qm2pi.local} 

%\ifpdf
%\usepackage[pdftex]{graphicx}
%\else
%\usepackage{graphicx}
%\fi

 % \ifpdf
%  \usepackage{pdfsync}
%  \if


%\title{Brief Article}
%\author{David F. Snyder}
%\author{L.G. Meredith}

%\address{Dept. of Math., Texas State University--San Marcos, San Marcos, TX 78666}
       
\pagestyle{empty}


\begin{document}

\lstset{language=[Objective]Caml,frame=shadowbox}

\input{qm2pi.front}

% section front matter (end)

\input{qm2pi.intro} 
 
% section introduction (end)

% \input{qm2pi.knotations} 

% section notation (end)

\input{qm2pi.process.calculi} 

% section concurrent_process_calculi_and_spatial_logics_ (end)
    
%\input{qm2pi.knots2pi} 

%\input{qm2pi.trefoil} 

%\input{qm2pi.mainthm} 

% subsection basic_interpretation (end)

%\input{qm2pi.rho.presentation} 
\subsection{The syntax and semantics of the notation system}\label{sub:the_syntax_and_semantics_of_the_notation_system} % (fold)

We now summarize a technical presentation of the calculus that
embodies our theory of dynamics. The typical presentation of such a
calculus follows the style of giving generators and relations on
them. The grammar, below, describing term constructors, freely
generates the set of processes, $\Proc$. This set is then quotiented
by a relation known as structural congruence and it is over this set
that the notion of dynamics is expressed. This presentation is
essentially that of \cite{MeredithR05} with the addition of
polyadicity and summation. For readability we have relegated some of
the technical subtleties to an appendix.

\subsubsection{Process grammar}\label{subsub:process_grammar}

\begin{mathpar}
  \inferrule* [lab=synchronization] {} {{M} \bc \pzero \;|\; x?F \;|\; x!C }
  \and
  \inferrule* [lab=abstraction] {} {{F} \bc (x)P}
  \and
  \inferrule* [lab=concretion] {} {{C} \bc \langle Q \rangle}
  \and
  \inferrule* [lab=process] {} {{P,Q} \bc M \;| \;P|Q \;|\; @{x}}
  \and
  \inferrule* [lab=name] {} {{x} \bc \quotep{P}}
\end{mathpar} 

Note that $\vec{x}$ (resp. $\vec{P}$) denotes a vector of names
(resp. processes) of length $|\vec{x}|$ (resp. $|\vec{P}|$). We adopt
the following useful abbreviations.

\begin{mathpar}
   x?(\vec{y}).P := x.(\vec{y})P \and  x\clift{\vec{P}} := x.\clift{\vec{P}}
   \and x!(y) := \lift{x}{\dropn{y}}
   \and \Pi_{i=0}^{n-1}P_i := P_0 | \ldots | P_{n-1}
\end{mathpar}

\subsubsection{Structural congruence}

\paragraph{Free and bound names and alpha-equivalence.} At the
core of structural equivalence is alpha-equivalence which identifies
process that are the same up to a change of variable. Formally, we
recognize the distinction between free and bound names. The free names
of a process, $\freenames{P}$, may be calculated recursively as
follows:

\begin{mathpar}
\freenames{\pzero} := \emptyset
  \and \\
  \freenames{x?(y).P} := \{ x \} \cup (\freenames{P} \setminus \{ y \})
  \and 
  \freenames{x!\langle P \rangle} := \{ x \} \cup \{ P \} 
  \and \\
  \freenames{P|Q} := \freenames{P} \cup \freenames{Q}
  \and \\
  \freenames{@{x}} := \{ x \}
\end{mathpar}

$\pi$
$\quotep{\pi}$

$\freenames{-} : \pi \to \mathcal{P}(\quotep{\pi})$

\begin{eqnarray*}
  \freenames{\pzero} & := & \emptyset \\
  \freenames{x?(y).P} & := & \{ x \} \cup (\freenames{P} \setminus \{ y \}) \\
  \freenames{x!\langle P \rangle} & := & \{ x \} \cup \{ P \} \\
  \freenames{P|Q} & := & \freenames{P} \cup \freenames{Q} \\
  \freenames{\dropn{x}} & := & \{ x \}
\end{eqnarray*}

The bound names of a process, $\boundnames{P}$, are those names occurring in $P$
that are not free. For example, in $x?(y).0$, the name $x$ is free, while $y$ is bound.

\begin{mathpar}
  \inferrule* [lab=monoidal-laws] {} { P|Q \equiv Q|P \and P|0 \equiv P \and P|(Q|R) \equiv (P|Q)|R }
\end{mathpar}

\begin{mathpar}
  \inferrule* [lab=alpha-equivalence] {} { (x)P \equiv (y)P\{y/x\} \and y \not\in \freenames{P} }
\end{mathpar}

\begin{definition}
Then two processes, $P,Q$, are alpha-equivalent if $P = Q\{\vec{y}/\vec{x}\}$ for
some $\vec{x} \in \boundnames{Q},\vec{y} \in \boundnames{P}$, where $Q\{\vec{y}/\vec{x}\}$
denotes the capture-avoiding substitution of $\vec{y}$ for $\vec{x}$ in $Q$.
\end{definition}

\begin{definition}
  The {\em structural congruence} \cite{SangiorgiWalker} , $\equiv$,
  between processes is the least congruence containing
  alpha-equivalence, satisfying the abelian monoid laws
  (associativity, commutativity and $\pzero$ as identity) for parallel
  composition $|$ and for summation $+$.
\end{definition}

\subsection{Name equivalence}

We take name equivalence, written $\nameeq$, to be the smallest
equivalence relation generated by the following rules.

\begin{mathpar}
\inferrule*[lab=Quote-drop]
{ }
{ \quotep{@{x}} \nameeq x }

\inferrule*[lab=Struct-equiv]
{ P \scong Q }
{ \quotep{P} \nameeq \quotep{Q} }
\end{mathpar}

The astute reader will have noticed that the mutual recursion of names
and processes imposes a mutual recursion on alpha-equivalence and
structural equivalence via name-equivalence. Fortunately, all of this
works out pleasantly and we may calculate in the natural way, free of
concern. The reader interested in the details is referred to the
appendix \ref{appendix:rho_details}.

\subsection{Substitution}

We use $\Proc$ for the set of processes, $\QProc$ for the set of
names, and $\id{\{}\vec{y} / \vec{x} \id{\}}$ to denote partial maps,
$s : \QProc \rightarrow \QProc$. A map, $s$ lifts, uniquely, to a map
on process terms, $\widehat{s} : \Proc \rightarrow \Proc$ by the
following equations.

\begin{mathpar}
  (0) \psubstp{Q}{P} := 0 \\
  (R \juxtap S) \psubstp{Q}{P}
  :=    
  (R)\psubstp{Q}{P} \juxtap (S) \psubstp{Q}{P} \\
  (x?(y).R) \psubstp{Q}{P}    
  :=    
  (x)\substp{Q}{P} (z)\concat( (R \psubstn{z}{y}) \psubstp{Q}{P} ) \\
  (\lift{x}{R}) \psubstp{Q}{P}  
  :=
  \lift{(x)\substp{Q}{P}}{ R \psubstp{Q}{P} } \\
%   (\dropn{x})  \psubstp{Q}{P}       
%   := 
%   \left\{ 
%     \begin{array}{ccc} 
%       \dropn{\quotep{Q}} & & x \nameeq \quotep{P} \\
%       \dropn{x} & & otherwise \\
%     \end{array}
%   \right. 
  (\dropn{x})  \psubstp{Q}{P}       
  := 
  \left\{ 
    \begin{array}{ccc} 
      Q & & x \nameeq \quotep{P} \\
      \dropn{x} & & otherwise \\
    \end{array}
  \right.
\end{mathpar}
 

where

\begin{eqnarray}
  (x)\id{\{} \lpquote Q \rpquote / \lpquote P \rpquote \id{\}}            = 
  \left\{ 
    \begin{array}{ccc}
      \lpquote Q \rpquote & & x \nameeq \lpquote P \rpquote \\
      x & & otherwise \\
    \end{array}
  \right. \nonumber
\end{eqnarray}

and $z$ is chosen distinct from $\quotep{P}$, $\quotep{Q}$, the free
names in $Q$, and all the names in $R$. Our $\alpha$-equivalence will
be built in the standard way from this substitution.

\begin{remark}\label{rem:no_self_referential_names}
  One consequence of these definitions is that $\forall P. \quotep{P}
  \not\in \freenames{P}$.
\end{remark}

\subsection{ Dynamic quote: an example }

Anticipating something of what's to come, consider applying the
substitution, $\widehat{\id{\{}u / z \id{\}}}$, to the following pair
of processes, $\lift{w}{y!(z)}$ and $w[ \lpquote y!(z) \rpquote ]$.

\begin{eqnarray}
	\lift{w}{y!(z)}\widehat{\id{\{}u / z \id{\}}}
		& = &
		\lift{w}{y!(u)} \nonumber\\
	w[ \lpquote y!(z) \rpquote ] \widehat{ \id{\{}u / z \id{\}} }
		& = &
		w[ \lpquote y!(z) \rpquote ] \nonumber
\end{eqnarray}

Because the body of the process between quotes is impervious to
substitution, we get radically different answers. In fact, by
examining the first process in an input context,
e.g. $x?(z).\lift{w}{y!(z)}$, we see that the process under the lift
operator may be shaped by prefixed inputs binding a name inside it. In
this sense, the lift operator will be seen as a way to dynamically
construct processes before reifying them as names.

Finally equipped with these standard features we can present the
dynamics of the calculus.

\subsubsection{Operational semantics} 

Finally, we introduce the computational dynamics. What marks these
algebras as distinct from other more traditionally studied algebraic
structures, e.g. vector spaces or polynomial rings, is the manner in
which dynamics is captured. In traditional structures, dynamics is typically
expressed through morphisms between such structures, as in linear maps
between vector spaces or morphisms between rings. In algebras
associated with the semantics of computation, the dynamics is
expressed as part of the algebraic structure itself, through a
reduction reduction relation typically denoted by $\red$. Below, we
give a recursive presentation of this relation for the calculus used
in the encoding.

$\red \subseteq \pi \times \pi$
$\red : \pi \to \mathcal{P}(\pi)$

\begin{mathpar}
  \inferrule* [lab=Comm] { \textsf{match}( x_{src}, x_{trgt} ) } { x_{trgt}?(y)P \; | \; x_{src}!\langle {Q} \rangle \red P\{\quotep{Q}/y}\} }
  \and \\
  \inferrule* [lab=Par] {{P} \red {P}'} {{{P} | {Q}} \red {{P}' | {Q}}}
  \and
  \inferrule* [lab=Equiv]{{{P} \scong {P}'} \andalso {{P}' \red {Q}'} \andalso {{Q}' \scong {Q}}}{{P} \red {Q}}
\end{mathpar}

\begin{eqnarray*}
  match_{\equiv} (\quotep{P},\quotep{Q}) & := & P \equiv Q \\
  match_{\dagger}(\quotep{P},\quotep{Q}) & := & \forall R. P|Q \red^{*} R => R \red^{*} 0 \\
  match_{K}(\quotep{P},\quotep{Q}) & := & K \mbox{ for some context } K
\end{eqnarray*}

$u?(x)P | u!\langle Q \rangle \red P\{\quotep{Q}/x\}$

%We write $\wred$ for $\red^*$, and $P\red$ if $\exists Q $ such that $ P \red Q$.
We write $P\red$ if $\exists Q $ such that $ P \red Q$ and $P\not\red$, otherwise.

\section{Replication}

As mentioned before, it is known that replication (and hence
recursion) can be implemented in a higher-order process algebra
\cite{SangiorgiWalker}. As our first example of calculation with the
machinery thus far presented we give the construction explicitly in
the {\rhoc}.

\begin{eqnarray}
	D_{x} & := & \prefix{x}{y}{(\binpar{\outputp{x}{y}}{@{y}})} \nonumber\\
	\bangp_{x}{P} & := & \binpar{{x}!\langle{\binpar{D_{x}}{P}}\rangle}{D_{x}} \nonumber
\end{eqnarray}

\begin{eqnarray}
	\bangp_{x}{P} & & \nonumber\\
	=
	& {x}!\langle{(\prefix{x}{y}{(\outputp{x}{y} | @{y})) | P}}\rangle 
	      | \prefix{x}{y}{(\outputp{x}{y} | @{y})} & \nonumber\\
	\red
	& (\outputp{x}{y} | @{y})\substn{\quotep{(\prefix{x}{y}{(@{y} | \outputp{x}{y})) | P}}}{y} & \nonumber\\
	=
	& \outputp{x}{\quotep{(\prefix{x}{y}{(\outputp{x}{y} | @{y})) | P}}}
	  | {(\prefix{x}{y}{(\outputp{x}{y} | @{y})) | P}} & \nonumber\\
	\red
	& \ldots & \nonumber\\
	\red^*
	& P | P | \ldots & \nonumber
\end{eqnarray}

Of course, this encoding, as an implementation, runs away, unfolding
$\bangp{P}$ eagerly. A lazier and more implementable replication
operator, restricted to input-guarded processes, may be obtained as follows.

\begin{eqnarray}
\bangp{\prefix{u}{v}{P}} 
	:= 
	\binpar{\lift{x}{\prefix{u}{v}{(\binpar{D(x)}{P})}}}{D(x)} \nonumber
\end{eqnarray}

\begin{remark}
  Note that the lazier definition still does not deal with summation
  or mixed summation (i.e. sums over input and output). The reader is
  invited to construct definitions of replication that deal with these
  features. 

  Further, the definitions are parameterized in a name, $x$. Can you,
  gentle reader, make a definition that eliminates this parameter and
  guarantees no accidental interaction between the replication
  machinery and the process being replicated -- i.e. no accidental
  sharing of names used by the process to get its work done and the
  name(s) used by the replication to effect copying. This latter
  revision of the definition of replication is crucial to obtaining
  the expected identity $!!P \sim !P$.
\end{remark}

\begin{remark}\label{rem:paradoxical_combinator}
  The reader familiar with the lambda calculus will have noticed the
  similarity between $D$ and the paradoxical combinator.

  [Ed. note: the existence of this seems to suggest we have to be more
  restrictive on the set of processes and names we admit if we are to
  support no-cloning.]
\end{remark}

\subsubsection{Bisimulation}

The computational dynamics gives rise to another kind of equivalence,
the equivalence of computational behavior. As previously mentioned
this is typically captured \emph{via} some form of bisimulation.

% The notion we use in this paper is weak barbed bisimulation
% \cite{milner91polyadicpi}.

The notion we use in this paper is derived from weak barbed
bisimulation \cite{milner91polyadicpi}. 

\begin{definition}
An \emph{observation relation}, $\downarrow_{\mathcal N}$, over a set
of names, $\mathcal N$, is the smallest relation satisfying the rules
below.

\infrule[Out-barb]{y \in {\mathcal N}, \; x \nameeq y}
		  {\outputp{x}{v} \downarrow_{\mathcal N} x}
\infrule[Par-barb]{\mbox{$P\downarrow_{\mathcal N} x$ or $Q\downarrow_{\mathcal N} x$}}
		  {\binpar{P}{Q} \downarrow_{\mathcal N} x}

We write $P \Downarrow_{\mathcal N} x$ if there is $Q$ such that 
$P \wred Q$ and $Q \downarrow_{\mathcal N} x$.
\end{definition}

\begin{definition}
%\label{def.bbisim}
An  ${\mathcal N}$-\emph{barbed bisimulation} over a set of names, ${\mathcal N}$, is a symmetric binary relation 
${\mathcal S}_{\mathcal N}$ between agents such that $P\rel{S}_{\mathcal N}Q$ implies:
\begin{enumerate}
\item If $P \red P'$ then $Q \wred Q'$ and $P'\rel{S}_{\mathcal N} Q'$.
\item If $P\downarrow_{\mathcal N} x$, then $Q\Downarrow_{\mathcal N} x$.
\end{enumerate}
$P$ is ${\mathcal N}$-barbed bisimilar to $Q$, written
$P \wbbisim_{\mathcal N} Q$, if $P \rel{S}_{\mathcal N} Q$ for some ${\mathcal N}$-barbed bisimulation ${\mathcal S}_{\mathcal N}$.
\end{definition}

$\mathcal{R} \subseteq \pi \times \pi$

$P \mathcal{R} Q => \forall P'. P \red P' \Rightarrow \exists Q'. Q \red Q', P' \mathcal{R} Q'$

$P \vdash x \Rightarrow Q \vdash x$

\begin{mathpar}
  \inferrule*[lab=Out-barb]{x \nameeq y}{{y}!\langle{Q}\rangle \vdash x}
  \and
  \inferrule*[lab=Par-barb]{\mbox{$P\vdash x$ or $Q\vdash x$}}{\binpar{P}{Q} \vdash x}
\end{mathpar}

\subsubsection{Contexts}

One of the principle advantages of computational calculi like the
$\pi$-calculus is a well-defined notion of context,
contextual-equivalence and a correlation between
contextual-equivalence and notions of bisimulation. The notion of
context allows the decomposition of a process into (sub-)process and
its syntactic environment, its context. Thus, a context may be
thought of as a process with a ``hole'' (written $\Box$) in it. The
application of a context $M$ to a process $P$, written $M[P]$, is
tantamount to filling the hole in $M$ with $P$. In this paper we do
not need the full weight of this theory, but do make use of the notion
of context in the proof the main theorem. 

\begin{mathpar}
  \inferrule* [lab=summation] {} {{M_{M},M_{N}} \bc \Box \;|\; x.M_{A} \;|\; M_{M}+M_{N}}
  \and
  \inferrule* [lab=agent] {} {{M_{A}} \bc (\vec{x})M_{P} \;| \; \clift{P_0,\ldots,M_{P},\ldots,P_N}}
  \and \\
  \inferrule* [lab=process] {} {{M_{P}} \bc M_{N} \;| \;P|M_{P} }
\end{mathpar} 

\begin{mathpar}
  \inferrule* [lab=sychronization] {} {M_{N} \bc \Box \;|\; x?M_{F} \;|\; x!M_{C}}
  \and
  \inferrule* [lab=abstraction] {} {{M_{F}} \bc (x)M_{P} }
  \and
  \inferrule* [lab=concretion] {} {{M_{C}} \bc \langle M_{P} \rangle }
  \and \\
  \inferrule* [lab=process] {} {{M_{P}} \bc M_{N} \;| \;P|M_{P} }
\end{mathpar}

\begin{definition}[contextual application] Given a context $M$, and
  process $P$, we define the \emph{contextual application}, $M[P] :=
  M\{P/\Box\}$. That is, the contextual application of M to P is the
  substitution of $P$ for $\Box$ in $M$.
\end{definition}

$\meaningof{-} : L \to \mathcal{P}(\pi)$

\begin{mathpar}
  \inferrule* [lab=collection] {} {\meaningof{true} = \pi, \and \meaningof{~E} = \pi \setminus \meaningof{E}, \and \meaningof{E_{1} \& E_{2}} = \meaningof{E_{1}} \cap \meaningof{E_{2}}}
\end{mathpar}

\begin{mathpar}
  \inferrule* [lab=structure] {} {\meaningof{0} = \{ P \in \pi | P \equiv 0 \}, \and \\ \meaningof{E_1 | E_2} = \{ P \in \pi | P \equiv P_{1} | P_{2}, P_{1} \in \meaningof{E_{1}}, P_{2} \in \meaningof{E_2}\} }
\end{mathpar}

\begin{mathpar}
 \inferrule* [lab=behavior] {} {\meaningof{\langle a?b \rangle E} = \{ P \in \pi | P \equiv Q | u?(y)P', \\ \and \\\\ \and \\ \;\;\; u \in \meaningof{a}, \forall z.P'\{z/y\} \in \meaningof{E\{z/b\}}\}, \and \\ \meaningof{a!E} = \{ P \in \pi | P \equiv Q | x!\langle P' \rangle, x \in \meaningof{a} P' \in \meaningof{E}\} }
\end{mathpar}

\begin{mathpar}
 \inferrule* [lab=nominal] {} {\meaningof{\quotep{E}} = \{ \quotep{P} \in \quotep{\pi} | P \in \meaningof{E} \}, \and \meaningof{\quotep{P}} = \{ \quotep{Q} \in \quotep{\pi} | P \equiv Q \} \and \\ \meaningof{@\quotep{E}} = \{ P \in \pi | P \equiv @x, x \in \meaningof{E} \}}
\end{mathpar}

\begin{eqnarray*}
  \\
  \meaningof{-} : TS \to ST
\end{eqnarray*}

\begin{eqnarray*}
  \\
  L : TS \to ST
\end{eqnarray*}

\begin{eqnarray*}
  \\
  P \models E \iff P \in \meaningof{E}
\end{eqnarray*}

\begin{eqnarray*}
  P \approx_{L} Q \iff \forall E \in L. P \models E \iff Q \models E
\end{eqnarray*}

\begin{eqnarray*}
  P \approx_{K} Q
\end{eqnarray*}

\begin{eqnarray*}
  P \approx Q
\end{eqnarray*}

$\approx_{K} = \approx = \approx_{L}$

\subsubsection{Contextual duality}

Note that contexts extend the quotation operation to a family of
operations from processes to names. Given a context, $M$, we can
define a \emph{nominal context}, $\quotep{M}$ by $\quotep{M}[P] :=
\quotep{M[P]}$. To foreshadow what is to come we observe that these
operations enjoy a duality with processes very much like the duality
between vectors and maps from vectors to scalars.

Further, because the calculus is essentially higher-order, we have a
correspondence between contexts and processes. More specifically,
given a name $x$ and a context $M$ we can construct $M^{*}_{x}$ such
that 

\begin{mathpar}
  M^{*}_{x} | \lift{x}{P} \red M[P]
\end{mathpar}

namely,

\begin{mathpar}
  M^{*}_{x} := x?(u).M[\dropn{u}]
\end{mathpar}

The dependence of $M^{*}_{x}$ on a name makes it an abstraction, 

\begin{mathpar}
  M^{*} := (x)x?(u).M[\dropn{u}]
\end{mathpar}

\subsection{Additional notation}

It will sometimes be convenient to denote the process a name
quotes. We already have the notation $x = \quotep{P}$, but it will be
convenient to introduce an alternate notation, $\procn{x}$, when we
want to emphasize the connection to the use of the name. Note that, by
virtue of name equivalence, $\quotep{\procn{x}} \nameeq x$; so, the
notation is consistent with previous definitions.

Further, because names have structure it is possible to effect
substitutions on the basis of that structure. This means we need to
upgrade our notation for substitutions, which we accomplish by
adapting comprehension notation. Thus,

\begin{mathpar}
  P\{ y / x : x \in S \}
\end{mathpar}

is interpreted to mean the process derived from P by replacing (in a
capture-avoiding manner) each occurrence of $x$ in $S$ by $y$. For example,

\begin{mathpar}
  P\{ \quotep{\procn{x}|\procn{x}} / x : x \in \freenames{P} \}
\end{mathpar}

will replace each (occurrence) of a free name $x$ in $P$ by
$\quotep{\procn{x}|\procn{x}}$.

Also, we will avail ourselves of the notation $x^{L}$ and $x^{R}$ to
denote injections of a name into disjoint copies of the name
space. There are numerous ways to accomplish this. One example can be
found in \cite{MeredithR05}. This notation overloads to vectors of
names: $\vec{x}^{\pi} := (x_{i}^{\pi} \; : \; 0 \leq i < |\vec{x}| )$ where $\pi \in \{L,R\}$.

We also use $P^{\Box} := P|\Box$.

In \cite{MeredithR05} an interpretation of the new operator is
given. It turns out that there are several possible interpretations
all enjoying the requisite algebraic properties of the operator (see
\cite{milner91polyadicpi}). We will therefore make liberal use of
$(\nu\; \vec{x})P$.

% subsection the_syntax_and_semantics_of_the_notation_system (end)   

\input{qm2pi.qmops} 

\input{qm2pi.sterngerlach} 

\input{qm2pi.metric} 

% section concurrent_process_calculi (end)

%\input{qm2pi.proofsketch}

% section proof sketch (end)

%\input{qm2pi.slviaknots} 

% section spatial logic via knots (end)

\input{qm2pi.conclusion}

% section conclusion (end)

%\input{qm2pi.dtcodes} 

% section wiring algorithm (end)

\input{qm2pi.ack} 

% section acknowledgments (end)

\newpage


\bibliographystyle{plain}   
\bibliography{../../biblios/main.bib}

\input{qm2pi.rhodetails}

\end{document}

 

% subsection basic_interpretation (end)

%\input{qm2pi.rho.presentation} 
\subsection{The syntax and semantics of the notation system}\label{sub:the_syntax_and_semantics_of_the_notation_system} % (fold)

We now summarize a technical presentation of the calculus that
embodies our theory of dynamics. The typical presentation of such a
calculus follows the style of giving generators and relations on
them. The grammar, below, describing term constructors, freely
generates the set of processes, $\Proc$. This set is then quotiented
by a relation known as structural congruence and it is over this set
that the notion of dynamics is expressed. This presentation is
essentially that of \cite{MeredithR05} with the addition of
polyadicity and summation. For readability we have relegated some of
the technical subtleties to an appendix.

\subsubsection{Process grammar}\label{subsub:process_grammar}

\begin{mathpar}
  \inferrule* [lab=synchronization] {} {{M} \bc \pzero \;|\; x?F \;|\; x!C }
  \and
  \inferrule* [lab=abstraction] {} {{F} \bc (x)P}
  \and
  \inferrule* [lab=concretion] {} {{C} \bc \langle Q \rangle}
  \and
  \inferrule* [lab=process] {} {{P,Q} \bc M \;| \;P|Q \;|\; @{x}}
  \and
  \inferrule* [lab=name] {} {{x} \bc \quotep{P}}
\end{mathpar} 

Note that $\vec{x}$ (resp. $\vec{P}$) denotes a vector of names
(resp. processes) of length $|\vec{x}|$ (resp. $|\vec{P}|$). We adopt
the following useful abbreviations.

\begin{mathpar}
   x?(\vec{y}).P := x.(\vec{y})P \and  x\clift{\vec{P}} := x.\clift{\vec{P}}
   \and x!(y) := \lift{x}{\dropn{y}}
   \and \Pi_{i=0}^{n-1}P_i := P_0 | \ldots | P_{n-1}
\end{mathpar}

\subsubsection{Structural congruence}

\paragraph{Free and bound names and alpha-equivalence.} At the
core of structural equivalence is alpha-equivalence which identifies
process that are the same up to a change of variable. Formally, we
recognize the distinction between free and bound names. The free names
of a process, $\freenames{P}$, may be calculated recursively as
follows:

\begin{mathpar}
\freenames{\pzero} := \emptyset
  \and \\
  \freenames{x?(y).P} := \{ x \} \cup (\freenames{P} \setminus \{ y \})
  \and 
  \freenames{x!\langle P \rangle} := \{ x \} \cup \{ P \} 
  \and \\
  \freenames{P|Q} := \freenames{P} \cup \freenames{Q}
  \and \\
  \freenames{@{x}} := \{ x \}
\end{mathpar}

$\pi$
$\quotep{\pi}$

$\freenames{-} : \pi \to \mathcal{P}(\quotep{\pi})$

\begin{eqnarray*}
  \freenames{\pzero} & := & \emptyset \\
  \freenames{x?(y).P} & := & \{ x \} \cup (\freenames{P} \setminus \{ y \}) \\
  \freenames{x!\langle P \rangle} & := & \{ x \} \cup \{ P \} \\
  \freenames{P|Q} & := & \freenames{P} \cup \freenames{Q} \\
  \freenames{\dropn{x}} & := & \{ x \}
\end{eqnarray*}

The bound names of a process, $\boundnames{P}$, are those names occurring in $P$
that are not free. For example, in $x?(y).0$, the name $x$ is free, while $y$ is bound.

\begin{mathpar}
  \inferrule* [lab=monoidal-laws] {} { P|Q \equiv Q|P \and P|0 \equiv P \and P|(Q|R) \equiv (P|Q)|R }
\end{mathpar}

\begin{mathpar}
  \inferrule* [lab=alpha-equivalence] {} { (x)P \equiv (y)P\{y/x\} \and y \not\in \freenames{P} }
\end{mathpar}

\begin{definition}
Then two processes, $P,Q$, are alpha-equivalent if $P = Q\{\vec{y}/\vec{x}\}$ for
some $\vec{x} \in \boundnames{Q},\vec{y} \in \boundnames{P}$, where $Q\{\vec{y}/\vec{x}\}$
denotes the capture-avoiding substitution of $\vec{y}$ for $\vec{x}$ in $Q$.
\end{definition}

\begin{definition}
  The {\em structural congruence} \cite{SangiorgiWalker} , $\equiv$,
  between processes is the least congruence containing
  alpha-equivalence, satisfying the abelian monoid laws
  (associativity, commutativity and $\pzero$ as identity) for parallel
  composition $|$ and for summation $+$.
\end{definition}

\subsection{Name equivalence}

We take name equivalence, written $\nameeq$, to be the smallest
equivalence relation generated by the following rules.

\begin{mathpar}
\inferrule*[lab=Quote-drop]
{ }
{ \quotep{@{x}} \nameeq x }

\inferrule*[lab=Struct-equiv]
{ P \scong Q }
{ \quotep{P} \nameeq \quotep{Q} }
\end{mathpar}

The astute reader will have noticed that the mutual recursion of names
and processes imposes a mutual recursion on alpha-equivalence and
structural equivalence via name-equivalence. Fortunately, all of this
works out pleasantly and we may calculate in the natural way, free of
concern. The reader interested in the details is referred to the
appendix \ref{appendix:rho_details}.

\subsection{Substitution}

We use $\Proc$ for the set of processes, $\QProc$ for the set of
names, and $\id{\{}\vec{y} / \vec{x} \id{\}}$ to denote partial maps,
$s : \QProc \rightarrow \QProc$. A map, $s$ lifts, uniquely, to a map
on process terms, $\widehat{s} : \Proc \rightarrow \Proc$ by the
following equations.

\begin{mathpar}
  (0) \psubstp{Q}{P} := 0 \\
  (R \juxtap S) \psubstp{Q}{P}
  :=    
  (R)\psubstp{Q}{P} \juxtap (S) \psubstp{Q}{P} \\
  (x?(y).R) \psubstp{Q}{P}    
  :=    
  (x)\substp{Q}{P} (z)\concat( (R \psubstn{z}{y}) \psubstp{Q}{P} ) \\
  (\lift{x}{R}) \psubstp{Q}{P}  
  :=
  \lift{(x)\substp{Q}{P}}{ R \psubstp{Q}{P} } \\
%   (\dropn{x})  \psubstp{Q}{P}       
%   := 
%   \left\{ 
%     \begin{array}{ccc} 
%       \dropn{\quotep{Q}} & & x \nameeq \quotep{P} \\
%       \dropn{x} & & otherwise \\
%     \end{array}
%   \right. 
  (\dropn{x})  \psubstp{Q}{P}       
  := 
  \left\{ 
    \begin{array}{ccc} 
      Q & & x \nameeq \quotep{P} \\
      \dropn{x} & & otherwise \\
    \end{array}
  \right.
\end{mathpar}
 

where

\begin{eqnarray}
  (x)\id{\{} \lpquote Q \rpquote / \lpquote P \rpquote \id{\}}            = 
  \left\{ 
    \begin{array}{ccc}
      \lpquote Q \rpquote & & x \nameeq \lpquote P \rpquote \\
      x & & otherwise \\
    \end{array}
  \right. \nonumber
\end{eqnarray}

and $z$ is chosen distinct from $\quotep{P}$, $\quotep{Q}$, the free
names in $Q$, and all the names in $R$. Our $\alpha$-equivalence will
be built in the standard way from this substitution.

\begin{remark}\label{rem:no_self_referential_names}
  One consequence of these definitions is that $\forall P. \quotep{P}
  \not\in \freenames{P}$.
\end{remark}

\subsection{ Dynamic quote: an example }

Anticipating something of what's to come, consider applying the
substitution, $\widehat{\id{\{}u / z \id{\}}}$, to the following pair
of processes, $\lift{w}{y!(z)}$ and $w[ \lpquote y!(z) \rpquote ]$.

\begin{eqnarray}
	\lift{w}{y!(z)}\widehat{\id{\{}u / z \id{\}}}
		& = &
		\lift{w}{y!(u)} \nonumber\\
	w[ \lpquote y!(z) \rpquote ] \widehat{ \id{\{}u / z \id{\}} }
		& = &
		w[ \lpquote y!(z) \rpquote ] \nonumber
\end{eqnarray}

Because the body of the process between quotes is impervious to
substitution, we get radically different answers. In fact, by
examining the first process in an input context,
e.g. $x?(z).\lift{w}{y!(z)}$, we see that the process under the lift
operator may be shaped by prefixed inputs binding a name inside it. In
this sense, the lift operator will be seen as a way to dynamically
construct processes before reifying them as names.

Finally equipped with these standard features we can present the
dynamics of the calculus.

\subsubsection{Operational semantics} 

Finally, we introduce the computational dynamics. What marks these
algebras as distinct from other more traditionally studied algebraic
structures, e.g. vector spaces or polynomial rings, is the manner in
which dynamics is captured. In traditional structures, dynamics is typically
expressed through morphisms between such structures, as in linear maps
between vector spaces or morphisms between rings. In algebras
associated with the semantics of computation, the dynamics is
expressed as part of the algebraic structure itself, through a
reduction reduction relation typically denoted by $\red$. Below, we
give a recursive presentation of this relation for the calculus used
in the encoding.

$\red \subseteq \pi \times \pi$
$\red : \pi \to \mathcal{P}(\pi)$

\begin{mathpar}
  \inferrule* [lab=Comm] { \textsf{match}( x_{src}, x_{trgt} ) } { x_{trgt}?(y)P \; | \; x_{src}!\langle {Q} \rangle \red P\{\quotep{Q}/y}\} }
  \and \\
  \inferrule* [lab=Par] {{P} \red {P}'} {{{P} | {Q}} \red {{P}' | {Q}}}
  \and
  \inferrule* [lab=Equiv]{{{P} \scong {P}'} \andalso {{P}' \red {Q}'} \andalso {{Q}' \scong {Q}}}{{P} \red {Q}}
\end{mathpar}

\begin{eqnarray*}
  match_{\equiv} (\quotep{P},\quotep{Q}) & := & P \equiv Q \\
  match_{\dagger}(\quotep{P},\quotep{Q}) & := & \forall R. P|Q \red^{*} R => R \red^{*} 0 \\
  match_{K}(\quotep{P},\quotep{Q}) & := & K \mbox{ for some context } K
\end{eqnarray*}

$u?(x)P | u!\langle Q \rangle \red P\{\quotep{Q}/x\}$

%We write $\wred$ for $\red^*$, and $P\red$ if $\exists Q $ such that $ P \red Q$.
We write $P\red$ if $\exists Q $ such that $ P \red Q$ and $P\not\red$, otherwise.

\section{Replication}

As mentioned before, it is known that replication (and hence
recursion) can be implemented in a higher-order process algebra
\cite{SangiorgiWalker}. As our first example of calculation with the
machinery thus far presented we give the construction explicitly in
the {\rhoc}.

\begin{eqnarray}
	D_{x} & := & \prefix{x}{y}{(\binpar{\outputp{x}{y}}{@{y}})} \nonumber\\
	\bangp_{x}{P} & := & \binpar{{x}!\langle{\binpar{D_{x}}{P}}\rangle}{D_{x}} \nonumber
\end{eqnarray}

\begin{eqnarray}
	\bangp_{x}{P} & & \nonumber\\
	=
	& {x}!\langle{(\prefix{x}{y}{(\outputp{x}{y} | @{y})) | P}}\rangle 
	      | \prefix{x}{y}{(\outputp{x}{y} | @{y})} & \nonumber\\
	\red
	& (\outputp{x}{y} | @{y})\substn{\quotep{(\prefix{x}{y}{(@{y} | \outputp{x}{y})) | P}}}{y} & \nonumber\\
	=
	& \outputp{x}{\quotep{(\prefix{x}{y}{(\outputp{x}{y} | @{y})) | P}}}
	  | {(\prefix{x}{y}{(\outputp{x}{y} | @{y})) | P}} & \nonumber\\
	\red
	& \ldots & \nonumber\\
	\red^*
	& P | P | \ldots & \nonumber
\end{eqnarray}

Of course, this encoding, as an implementation, runs away, unfolding
$\bangp{P}$ eagerly. A lazier and more implementable replication
operator, restricted to input-guarded processes, may be obtained as follows.

\begin{eqnarray}
\bangp{\prefix{u}{v}{P}} 
	:= 
	\binpar{\lift{x}{\prefix{u}{v}{(\binpar{D(x)}{P})}}}{D(x)} \nonumber
\end{eqnarray}

\begin{remark}
  Note that the lazier definition still does not deal with summation
  or mixed summation (i.e. sums over input and output). The reader is
  invited to construct definitions of replication that deal with these
  features. 

  Further, the definitions are parameterized in a name, $x$. Can you,
  gentle reader, make a definition that eliminates this parameter and
  guarantees no accidental interaction between the replication
  machinery and the process being replicated -- i.e. no accidental
  sharing of names used by the process to get its work done and the
  name(s) used by the replication to effect copying. This latter
  revision of the definition of replication is crucial to obtaining
  the expected identity $!!P \sim !P$.
\end{remark}

\begin{remark}\label{rem:paradoxical_combinator}
  The reader familiar with the lambda calculus will have noticed the
  similarity between $D$ and the paradoxical combinator.

  [Ed. note: the existence of this seems to suggest we have to be more
  restrictive on the set of processes and names we admit if we are to
  support no-cloning.]
\end{remark}

\subsubsection{Bisimulation}

The computational dynamics gives rise to another kind of equivalence,
the equivalence of computational behavior. As previously mentioned
this is typically captured \emph{via} some form of bisimulation.

% The notion we use in this paper is weak barbed bisimulation
% \cite{milner91polyadicpi}.

The notion we use in this paper is derived from weak barbed
bisimulation \cite{milner91polyadicpi}. 

\begin{definition}
An \emph{observation relation}, $\downarrow_{\mathcal N}$, over a set
of names, $\mathcal N$, is the smallest relation satisfying the rules
below.

\infrule[Out-barb]{y \in {\mathcal N}, \; x \nameeq y}
		  {\outputp{x}{v} \downarrow_{\mathcal N} x}
\infrule[Par-barb]{\mbox{$P\downarrow_{\mathcal N} x$ or $Q\downarrow_{\mathcal N} x$}}
		  {\binpar{P}{Q} \downarrow_{\mathcal N} x}

We write $P \Downarrow_{\mathcal N} x$ if there is $Q$ such that 
$P \wred Q$ and $Q \downarrow_{\mathcal N} x$.
\end{definition}

\begin{definition}
%\label{def.bbisim}
An  ${\mathcal N}$-\emph{barbed bisimulation} over a set of names, ${\mathcal N}$, is a symmetric binary relation 
${\mathcal S}_{\mathcal N}$ between agents such that $P\rel{S}_{\mathcal N}Q$ implies:
\begin{enumerate}
\item If $P \red P'$ then $Q \wred Q'$ and $P'\rel{S}_{\mathcal N} Q'$.
\item If $P\downarrow_{\mathcal N} x$, then $Q\Downarrow_{\mathcal N} x$.
\end{enumerate}
$P$ is ${\mathcal N}$-barbed bisimilar to $Q$, written
$P \wbbisim_{\mathcal N} Q$, if $P \rel{S}_{\mathcal N} Q$ for some ${\mathcal N}$-barbed bisimulation ${\mathcal S}_{\mathcal N}$.
\end{definition}

$\mathcal{R} \subseteq \pi \times \pi$

$P \mathcal{R} Q => \forall P'. P \red P' \Rightarrow \exists Q'. Q \red Q', P' \mathcal{R} Q'$

$P \vdash x \Rightarrow Q \vdash x$

\begin{mathpar}
  \inferrule*[lab=Out-barb]{x \nameeq y}{{y}!\langle{Q}\rangle \vdash x}
  \and
  \inferrule*[lab=Par-barb]{\mbox{$P\vdash x$ or $Q\vdash x$}}{\binpar{P}{Q} \vdash x}
\end{mathpar}

\subsubsection{Contexts}

One of the principle advantages of computational calculi like the
$\pi$-calculus is a well-defined notion of context,
contextual-equivalence and a correlation between
contextual-equivalence and notions of bisimulation. The notion of
context allows the decomposition of a process into (sub-)process and
its syntactic environment, its context. Thus, a context may be
thought of as a process with a ``hole'' (written $\Box$) in it. The
application of a context $M$ to a process $P$, written $M[P]$, is
tantamount to filling the hole in $M$ with $P$. In this paper we do
not need the full weight of this theory, but do make use of the notion
of context in the proof the main theorem. 

\begin{mathpar}
  \inferrule* [lab=summation] {} {{M_{M},M_{N}} \bc \Box \;|\; x.M_{A} \;|\; M_{M}+M_{N}}
  \and
  \inferrule* [lab=agent] {} {{M_{A}} \bc (\vec{x})M_{P} \;| \; \clift{P_0,\ldots,M_{P},\ldots,P_N}}
  \and \\
  \inferrule* [lab=process] {} {{M_{P}} \bc M_{N} \;| \;P|M_{P} }
\end{mathpar} 

\begin{mathpar}
  \inferrule* [lab=sychronization] {} {M_{N} \bc \Box \;|\; x?M_{F} \;|\; x!M_{C}}
  \and
  \inferrule* [lab=abstraction] {} {{M_{F}} \bc (x)M_{P} }
  \and
  \inferrule* [lab=concretion] {} {{M_{C}} \bc \langle M_{P} \rangle }
  \and \\
  \inferrule* [lab=process] {} {{M_{P}} \bc M_{N} \;| \;P|M_{P} }
\end{mathpar}

\begin{definition}[contextual application] Given a context $M$, and
  process $P$, we define the \emph{contextual application}, $M[P] :=
  M\{P/\Box\}$. That is, the contextual application of M to P is the
  substitution of $P$ for $\Box$ in $M$.
\end{definition}

$\meaningof{-} : L \to \mathcal{P}(\pi)$

\begin{mathpar}
  \inferrule* [lab=collection] {} {\meaningof{true} = \pi, \and \meaningof{~E} = \pi \setminus \meaningof{E}, \and \meaningof{E_{1} \& E_{2}} = \meaningof{E_{1}} \cap \meaningof{E_{2}}}
\end{mathpar}

\begin{mathpar}
  \inferrule* [lab=structure] {} {\meaningof{0} = \{ P \in \pi | P \equiv 0 \}, \and \\ \meaningof{E_1 | E_2} = \{ P \in \pi | P \equiv P_{1} | P_{2}, P_{1} \in \meaningof{E_{1}}, P_{2} \in \meaningof{E_2}\} }
\end{mathpar}

\begin{mathpar}
 \inferrule* [lab=behavior] {} {\meaningof{\langle a?b \rangle E} = \{ P \in \pi | P \equiv Q | u?(y)P', \\ \and \\\\ \and \\ \;\;\; u \in \meaningof{a}, \forall z.P'\{z/y\} \in \meaningof{E\{z/b\}}\}, \and \\ \meaningof{a!E} = \{ P \in \pi | P \equiv Q | x!\langle P' \rangle, x \in \meaningof{a} P' \in \meaningof{E}\} }
\end{mathpar}

\begin{mathpar}
 \inferrule* [lab=nominal] {} {\meaningof{\quotep{E}} = \{ \quotep{P} \in \quotep{\pi} | P \in \meaningof{E} \}, \and \meaningof{\quotep{P}} = \{ \quotep{Q} \in \quotep{\pi} | P \equiv Q \} \and \\ \meaningof{@\quotep{E}} = \{ P \in \pi | P \equiv @x, x \in \meaningof{E} \}}
\end{mathpar}

\begin{eqnarray*}
  \\
  \meaningof{-} : TS \to ST
\end{eqnarray*}

\begin{eqnarray*}
  \\
  L : TS \to ST
\end{eqnarray*}

\begin{eqnarray*}
  \\
  P \models E \iff P \in \meaningof{E}
\end{eqnarray*}

\begin{eqnarray*}
  P \approx_{L} Q \iff \forall E \in L. P \models E \iff Q \models E
\end{eqnarray*}

\begin{eqnarray*}
  P \approx_{K} Q
\end{eqnarray*}

\begin{eqnarray*}
  P \approx Q
\end{eqnarray*}

$\approx_{K} = \approx = \approx_{L}$

\subsubsection{Contextual duality}

Note that contexts extend the quotation operation to a family of
operations from processes to names. Given a context, $M$, we can
define a \emph{nominal context}, $\quotep{M}$ by $\quotep{M}[P] :=
\quotep{M[P]}$. To foreshadow what is to come we observe that these
operations enjoy a duality with processes very much like the duality
between vectors and maps from vectors to scalars.

Further, because the calculus is essentially higher-order, we have a
correspondence between contexts and processes. More specifically,
given a name $x$ and a context $M$ we can construct $M^{*}_{x}$ such
that 

\begin{mathpar}
  M^{*}_{x} | \lift{x}{P} \red M[P]
\end{mathpar}

namely,

\begin{mathpar}
  M^{*}_{x} := x?(u).M[\dropn{u}]
\end{mathpar}

The dependence of $M^{*}_{x}$ on a name makes it an abstraction, 

\begin{mathpar}
  M^{*} := (x)x?(u).M[\dropn{u}]
\end{mathpar}

\subsection{Additional notation}

It will sometimes be convenient to denote the process a name
quotes. We already have the notation $x = \quotep{P}$, but it will be
convenient to introduce an alternate notation, $\procn{x}$, when we
want to emphasize the connection to the use of the name. Note that, by
virtue of name equivalence, $\quotep{\procn{x}} \nameeq x$; so, the
notation is consistent with previous definitions.

Further, because names have structure it is possible to effect
substitutions on the basis of that structure. This means we need to
upgrade our notation for substitutions, which we accomplish by
adapting comprehension notation. Thus,

\begin{mathpar}
  P\{ y / x : x \in S \}
\end{mathpar}

is interpreted to mean the process derived from P by replacing (in a
capture-avoiding manner) each occurrence of $x$ in $S$ by $y$. For example,

\begin{mathpar}
  P\{ \quotep{\procn{x}|\procn{x}} / x : x \in \freenames{P} \}
\end{mathpar}

will replace each (occurrence) of a free name $x$ in $P$ by
$\quotep{\procn{x}|\procn{x}}$.

Also, we will avail ourselves of the notation $x^{L}$ and $x^{R}$ to
denote injections of a name into disjoint copies of the name
space. There are numerous ways to accomplish this. One example can be
found in \cite{MeredithR05}. This notation overloads to vectors of
names: $\vec{x}^{\pi} := (x_{i}^{\pi} \; : \; 0 \leq i < |\vec{x}| )$ where $\pi \in \{L,R\}$.

We also use $P^{\Box} := P|\Box$.

In \cite{MeredithR05} an interpretation of the new operator is
given. It turns out that there are several possible interpretations
all enjoying the requisite algebraic properties of the operator (see
\cite{milner91polyadicpi}). We will therefore make liberal use of
$(\nu\; \vec{x})P$.

% subsection the_syntax_and_semantics_of_the_notation_system (end)   

\section{Interpretation of QM}
\subsection{Supporting definitions}
\subsubsection{Multiplication}
\begin{mathpar}
  \quotep{Q} \cdot \quotep{R} := \quotep{Q|R}
  \and \\
  \quotep{Q} \cdot P := P\{ \quotep{Q|R} / \quotep{R} : \quotep{R} \in \freenames{P} \}
\end{mathpar}

\paragraph{Discussion}
The first line needs little explanation. The second line says that
each free name of the process is replaced with the multiplication of
that name by the scalar. Multiplication of a scalar (name) by a state
(process) results in a process all the names of which have been `moved
over' by parallel composition with the process the scalar
quotes. There is a subtlety that the bound names have to be
manipulated so that multiplied names aren't accidentally
captured. There are many ways to achieve this.

\begin{remark}\label{rem:multiplication_identities}
  The reader is invited to verify that for all $x,y,z \in \QProc$ and $P \in \Proc$
  \begin{mathpar}
    x \cdot \quotep{0} \equiv x 
    \and
    x \cdot y \equiv y \cdot x
    \and
    x \cdot (y \cdot z) \equiv (x \cdot y) \cdot z
    \and \\
    \quotep{0} \cdot P \equiv P
    \and \\
    x \cdot (y \cdot P) \equiv (x \cdot y) \cdot P
    \and \\
    x \cdot (P|Q) \equiv (x \cdot P) | (x \cdot Q)
    \and \\    
  \end{mathpar}
\end{remark}

\subsubsection{Tensor product}

We define a tensor product on processes by structural induction.

\paragraph{Tensor of sums} First note that all summations, including
$\pzero$ and sequence, can be written $\Sigma_{i} x_{i}.A_{i} +
\Sigma_{j} x_{j}.C_{j}$, where we have grouped input-guarded processes
together and output-guarded processes together.

Thus, we can define the tensor product of two summations, $N_{1}\otimes N_{2}$, where

\begin{mathpar}
  N_{1} := \Sigma_{i} x_{i}.A_{i} + \Sigma_{j} x_{j}.C_{j}
  \and
  N_{2} := \Sigma_{i'} y_{i'}.B_{i'} + \Sigma_{j'} y_{j'}.D_{j'} 
\end{mathpar}

as follows.

\begin{mathpar}
  \Sigma_{i} x_{i}.A_{i} + \Sigma_{j} x_{j}.C_{j} \otimes \Sigma_{i'}
  y_{i'}.B_{i'} + \Sigma_{j'} y_{j'}.D_{j'} 
  \and \\
  := \; \Sigma_{i} \Sigma_{i'} \quotep{\stackrel{\vee}{x_{i}}| \stackrel{\vee}{y_{i'}}}.(A_{i}\otimes B_{i'}) \; | \; \Sigma_{i'} \Sigma_{i} \quotep{\stackrel{\vee}{y_{i'}}|\stackrel{\vee}{x_{i}}}.(B_{i'}\otimes A_{i})
  \and
  \;\; | \;\; \Sigma_{j} \Sigma_{j'} \quotep{\stackrel{\vee}{x_{j}}|\stackrel{\vee}{y_{j'}}}.(A_{j}\otimes B_{j'}) \; | \; \Sigma_{j'} \Sigma_{j} \quotep{\stackrel{\vee}{y_{j'}}|\stackrel{\vee}{x_{j}}}.(B_{j'}\otimes A_{j})
\end{mathpar}

\begin{remark}
  Do we need to $x^{L}$ and $y^{R}$ for this construction as well?
\end{remark}

\paragraph{Tensor of parallel compositions} Next, we distribute tensor
over par.

\begin{mathpar}
  P_{1}|P_{2} \otimes Q_{1}|Q_{2} := (P_{1} \otimes Q_{1}) | (P_{1}
  \otimes Q_{2}) | (P_{2} \otimes Q_{1}) | (P_{2} \otimes Q_{2})
\end{mathpar}

\paragraph{Tensor with dropped names} We treat tensor of a
process with a dropped name as parallel composition.

\begin{mathpar}
  P \otimes \dropn{x} := P | \dropn{x}
\end{mathpar}

\paragraph{Tensor of agents}

Finally, we need to define tensor on agents. Note that the definition
of tensor on normal products only tensors inputs with inputs and
outputs with outputs. Thus, we only have to define the operation on
``homogeneous'' pairings.

\begin{mathpar}
  (\vec{x})P \otimes (\vec{y})Q
  \and \\
  := (x_{0}^{L}|y_{0}^{R},\ldots,x_{0}^{L}|y_{n}^{R},\ldots,x_{m}^{L}|y_{0}^{R},\ldots,x_{m}^{L}|y_{n}^R)(P\{ \vec{x}^{L}/\vec{x}\} \otimes Q \{ \vec{y}^{R}/\vec{y}\})
  \and \\
  \clift{\vec{P}} \otimes \clift{\vec{Q}}
  \and \\
  := \clift{P_{0}\otimes Q_{0},\ldots,P_{0}\otimes Q_{n},\ldots,P_{m}\otimes Q_{0},\ldots,P_{m}\otimes Q_{n}}
\end{mathpar}

\begin{remark}
  Observe that arities of tensored abstractions matches arities of
  tensored concretions if the original arities matched. Note also that
  the length of the arities corresponds to the increase in dimension
  we see in ordinary vector space tensor product.
\end{remark}

\begin{remark}
  Operationally, this definition distributes the tensor down to
  components ``linked'' by summation. Tensor over summation is
  intriguing in that it mixes names. Moreover, as a consequence of the
  way it mixes names we have the identities for all $x \in \QProc$ and
  $P,Q \in \Proc$

  \begin{mathpar}
    (x \cdot P) \otimes Q \equiv x \cdot (P \otimes Q) \equiv P \otimes (x \cdot Q)
    \and
    P \otimes \pzero \equiv P
  \end{mathpar}

  that the reader is invited to verify.
\end{remark}

\subsubsection{Annihilation}
\begin{mathpar}
  P^{\perp} := \{ Q | \forall R. P|Q \red^{*} R \Rightarrow R \red^{*} \pzero \}
  \and \\
  P^{\underline{\perp}} := \Sigma_{Q \in P^{\perp}} \quotep{Q}?(y).(\dropn{y}|Q) | \Sigma_{Q \in P^{\perp}} \quotep{Q}\clift{\Box}
\end{mathpar}

\paragraph{Discussion} The reader will note that $P^{\perp}$ is a
\emph{set} of processes, while $P^{\underline{\perp}}$ is a
\emph{context}. We call the set $P^{\perp}$ the \emph{annihilators} of
$P$. The parallel composition of a process in the annihilators of $P$
with $P$ will result in a process, the state space of which has all
paths eventually leading to $\pzero$. Execution may endure loops; but
under reasonable conditions of fairness (naturally guaranteed under
most notions of bisimulation) such a composite process cannot get
stuck in such a loop and will, eventually pop out and terminate.

The context $P^{\underline{\perp}}$ is ready and willing to ``take the
$P$ out of'' the process to which it is applied. It will effectively
transmit the code of the process to which it is applied to one of the
annihilators and run the process against it.

\subsubsection{Evaluation}
We fix $M$ a domain of fully abstract interpretation with an equality
coincident with bisimulation. We take $\meaningof{\cdot} : \Proc \to
M$ to be the map interpreting processes and $\nmeaningof{\cdot} : \M
\to Proc$ to be the map running the other way. Then we define

\begin{mathpar}
  \int P := \nmeaningof{\meaningof{P}}
\end{mathpar}

\paragraph{Discussion}
There are many fully abstract interpretations of Milner's
$\pi$-calculus. Any of them can be used as a basis for interpreting
the reflective calculus here. Equipped with such a domain it is
largely a matter of grinding through to check that the Yoneda
construction for the normalization-by-evaluation program can be
extended to this setting.

\begin{remark}
  The reader is invited to verify that $\int (P^{\underline{\perp}}[P]) = 0$.
\end{remark}

\subsection{Quantum mechanics}

Table \ref{tbl:core_qm_op_defns} gives the core operational definitions

\begin{table}[htp]\label{tbl:core_qm_op_defns}
  \center{
    \fbox{
      \begin{tabular}{c|c}
        quantum mechanics & process calculus \\
        \hline
        scalar & $x := \quotep{P}$ \\
        state vector & $\state{P} := P$ \\
        dual & $\state{P}^{*} := \event{P^{\underline{\perp}}} := \quotep{P^{\underline{\perp}}}[-]$ \\
        matrix & $ \Sigma_{\alpha} \state{P_{\alpha}}x_{\alpha}\event{Q_{\alpha}}$ \\
        vector addition & $\state{P} + \state{Q} := \state{P | Q}$ \\
        tensor product & $\state{P} \otimes \state{Q} := \state{P \otimes Q}$ \\
        inner product & $\innerprod{P}{Q} := \quotep{\int P^{\underline{\perp}}[Q]}$ \\
      \end{tabular}
    }
  }
  \caption{QM - operational definitions}
\end{table}

where

\begin{mathpar}
  \prmatrix{P}{Q} := \fprmatrix{P}{\quotep{\pzero}}{Q}
  \and
  \fprmatrix{P}{x}{Q} := (\state{P},x,\event{Q})
  \and
  (\fprmatrix{P}{x}{Q})(\state{R}) := x \cdot \innerprod{Q}{R} \cdot \state{P}
  \and
  (\fprmatrix{P}{x}{Q})(\event{R}) := x \cdot \innerprod{R}{P} \cdot \event{Q}
\end{mathpar}

\paragraph{Discussion}
As promised: vectors (aka states) are represented as processes; duals
as contextual duals; inner product definition should be compared with
standard inner product definition for ....

\begin{remark}
  Assuming $\int (P^{\underline{\perp}}[P]) = 0$, the reader is
  invited to verify that $(\fprmatrix{P}{x}{P})(\state{P}) = x \cdot \state{P}$.
\end{remark}

\begin{remark}
  The reader is invited to verify that $\innerprod{P}{Q}$ could
  equally well have been written $\quotep{\int \stackrel{\vee}{x}}$
  where $x = \event{P^{\underline{\perp}}}(Q)$.

  One of the motivations for this remark is that there is another way
  to factor these operations. We could package up evaluation in the dual:

  \begin{mathpar}
    \state{P}^{*} := \event{\int P^{\underline{\perp}}} := \quotep{\int P^{\underline{\perp}}}[-]
  \end{mathpar}

  and then have inner product defined by
  
  \begin{mathpar}
    \innerprod{P}{Q} := \event{P}(Q)
  \end{mathpar}

  Hopefully, experience with the calculations will provide guidance on
  the best factoring.
\end{remark}

\begin{remark}
  Assuming $\int (P^{\underline{\perp}}[P]) = 0$, the reader is
  invited to verify that $\forall P,Q. (\prmatrix{0}{Q})(\state{0}) =
  \state{0}$ and dually $(\prmatrix{P}{0})(\event{0}) = \event{0}$.
\end{remark}

\begin{remark}
  i'm a little worried that i don't (yet) have proper support for
  complex conjugacy. But, the observation above may give us a
  clue. According to Abramsky, it must be the case that the scalars
  are iso to the homset of the identity for the tensor -- which the
  observation above characterizes. 

  For now, we will simply bookmark the notion with $\overline{x}$.
\end{remark}

\subsubsection{Adjointness}

We need to give a definition of $(\cdot)^{\dagger}$ for matrices. The
obvious candidate definition is
\begin{mathpar}
(\Sigma_{\alpha}\fprmatrix{P_{\alpha}}{x_{\alpha}}{Q_{\alpha}})^{\dagger}
= \Sigma_{\alpha}\fprmatrix{(Q_{\alpha}^{\underline{\perp}})^{*}}{\overline{x}_{\alpha}}{P_{\alpha}^{\underline{\perp}}} 
\end{mathpar}

But, $(Q_{\alpha}^{\underline{\perp}})^{*}$ requires a name along
which to communicate the process to achieve the context application.

\subsubsection{Basis for a basis}
If processes label states and ``addition'' of states (a.k.a. vector
addition) is interpreted as parallel composition, what corresponds to
notions of linear independence and basis? Here, we recall that Yoshida
has developed a set of \emph{combinators} for an asynchronous verison
of Milner's $\pi$-calculus. These are a finite set of processes such
any process can be expressed as parallel composition of these
combinators together with liberal uses of the new operator and
replication. We can simply give a translation of these into the
present calculus and have reasonable expectation that the property
carries over. That is, that the resultant set allows to express all
processes via parallel composition. Note, however, that there is no
new operator or replication in this calculus. As a result, we expect
that the corresponding set is actually infinite. That is, we expect
that the space is actually infinite dimensional.

\begin{remark}
  The attentive reader may be a bit concerned. Certainly, the
  collection $S$, $K$ and $I$ is a finite set of
  combinators. Shouldn't we expect to see a finite set of combinators
  for an effectively equivalent system? i am very sympathetic to this
  critique and feel it warrants full attention. On the other hand, i
  also have in mind the following analogy. The natural numbers, as a
  monoid under addition, has exactly $1$ generator, while the natural
  numbers, as a monoid under multiplication, has countably many
  generators (the primes). We observe that the application of the
  lambda calculus is much less resource sensitive than the parallel
  composition of the $\pi$-calculus. Could it be the case that we have
  an analogy of the form
  
  \begin{mathpar}
    m + n : MN :: m*n : M|N
  \end{mathpar}

  giving a similar blow up in the set of ``primes''?  This is such a
  wonderful thought that, even if it's not true, i think it's worth
  writing down.
\end{remark}
 

\documentclass[12pt]{llncs}
%\documentclass{jktr}

\usepackage[pdftex]{hyperref}                   
\usepackage {listings}
\usepackage {mathpartir}
\usepackage{bcprules}
%\usepackage{listings}
                       
\usepackage{graphicx} 
%\usepackage[margins=2.5cm,nohead,nofoot]{geometry}
%\usepackage{geometry}
\usepackage{amsfonts}
\usepackage{amstext}
\usepackage{latexsym}
\usepackage{amssymb}
\usepackage{color}


%\include{myPreamble}
\include{qm2pi.local} 

%\ifpdf
%\usepackage[pdftex]{graphicx}
%\else
%\usepackage{graphicx}
%\fi

 % \ifpdf
%  \usepackage{pdfsync}
%  \if


%\title{Brief Article}
%\author{David F. Snyder}
%\author{L.G. Meredith}

%\address{Dept. of Math., Texas State University--San Marcos, San Marcos, TX 78666}
       
\pagestyle{empty}


\begin{document}

\lstset{language=[Objective]Caml,frame=shadowbox}

\input{qm2pi.front}

% section front matter (end)

\input{qm2pi.intro} 
 
% section introduction (end)

% \input{qm2pi.knotations} 

% section notation (end)

\input{qm2pi.process.calculi} 

% section concurrent_process_calculi_and_spatial_logics_ (end)
    
%\input{qm2pi.knots2pi} 

%\input{qm2pi.trefoil} 

%\input{qm2pi.mainthm} 

% subsection basic_interpretation (end)

%\input{qm2pi.rho.presentation} 
\subsection{The syntax and semantics of the notation system}\label{sub:the_syntax_and_semantics_of_the_notation_system} % (fold)

We now summarize a technical presentation of the calculus that
embodies our theory of dynamics. The typical presentation of such a
calculus follows the style of giving generators and relations on
them. The grammar, below, describing term constructors, freely
generates the set of processes, $\Proc$. This set is then quotiented
by a relation known as structural congruence and it is over this set
that the notion of dynamics is expressed. This presentation is
essentially that of \cite{MeredithR05} with the addition of
polyadicity and summation. For readability we have relegated some of
the technical subtleties to an appendix.

\subsubsection{Process grammar}\label{subsub:process_grammar}

\begin{mathpar}
  \inferrule* [lab=synchronization] {} {{M} \bc \pzero \;|\; x?F \;|\; x!C }
  \and
  \inferrule* [lab=abstraction] {} {{F} \bc (x)P}
  \and
  \inferrule* [lab=concretion] {} {{C} \bc \langle Q \rangle}
  \and
  \inferrule* [lab=process] {} {{P,Q} \bc M \;| \;P|Q \;|\; @{x}}
  \and
  \inferrule* [lab=name] {} {{x} \bc \quotep{P}}
\end{mathpar} 

Note that $\vec{x}$ (resp. $\vec{P}$) denotes a vector of names
(resp. processes) of length $|\vec{x}|$ (resp. $|\vec{P}|$). We adopt
the following useful abbreviations.

\begin{mathpar}
   x?(\vec{y}).P := x.(\vec{y})P \and  x\clift{\vec{P}} := x.\clift{\vec{P}}
   \and x!(y) := \lift{x}{\dropn{y}}
   \and \Pi_{i=0}^{n-1}P_i := P_0 | \ldots | P_{n-1}
\end{mathpar}

\subsubsection{Structural congruence}

\paragraph{Free and bound names and alpha-equivalence.} At the
core of structural equivalence is alpha-equivalence which identifies
process that are the same up to a change of variable. Formally, we
recognize the distinction between free and bound names. The free names
of a process, $\freenames{P}$, may be calculated recursively as
follows:

\begin{mathpar}
\freenames{\pzero} := \emptyset
  \and \\
  \freenames{x?(y).P} := \{ x \} \cup (\freenames{P} \setminus \{ y \})
  \and 
  \freenames{x!\langle P \rangle} := \{ x \} \cup \{ P \} 
  \and \\
  \freenames{P|Q} := \freenames{P} \cup \freenames{Q}
  \and \\
  \freenames{@{x}} := \{ x \}
\end{mathpar}

$\pi$
$\quotep{\pi}$

$\freenames{-} : \pi \to \mathcal{P}(\quotep{\pi})$

\begin{eqnarray*}
  \freenames{\pzero} & := & \emptyset \\
  \freenames{x?(y).P} & := & \{ x \} \cup (\freenames{P} \setminus \{ y \}) \\
  \freenames{x!\langle P \rangle} & := & \{ x \} \cup \{ P \} \\
  \freenames{P|Q} & := & \freenames{P} \cup \freenames{Q} \\
  \freenames{\dropn{x}} & := & \{ x \}
\end{eqnarray*}

The bound names of a process, $\boundnames{P}$, are those names occurring in $P$
that are not free. For example, in $x?(y).0$, the name $x$ is free, while $y$ is bound.

\begin{mathpar}
  \inferrule* [lab=monoidal-laws] {} { P|Q \equiv Q|P \and P|0 \equiv P \and P|(Q|R) \equiv (P|Q)|R }
\end{mathpar}

\begin{mathpar}
  \inferrule* [lab=alpha-equivalence] {} { (x)P \equiv (y)P\{y/x\} \and y \not\in \freenames{P} }
\end{mathpar}

\begin{definition}
Then two processes, $P,Q$, are alpha-equivalent if $P = Q\{\vec{y}/\vec{x}\}$ for
some $\vec{x} \in \boundnames{Q},\vec{y} \in \boundnames{P}$, where $Q\{\vec{y}/\vec{x}\}$
denotes the capture-avoiding substitution of $\vec{y}$ for $\vec{x}$ in $Q$.
\end{definition}

\begin{definition}
  The {\em structural congruence} \cite{SangiorgiWalker} , $\equiv$,
  between processes is the least congruence containing
  alpha-equivalence, satisfying the abelian monoid laws
  (associativity, commutativity and $\pzero$ as identity) for parallel
  composition $|$ and for summation $+$.
\end{definition}

\subsection{Name equivalence}

We take name equivalence, written $\nameeq$, to be the smallest
equivalence relation generated by the following rules.

\begin{mathpar}
\inferrule*[lab=Quote-drop]
{ }
{ \quotep{@{x}} \nameeq x }

\inferrule*[lab=Struct-equiv]
{ P \scong Q }
{ \quotep{P} \nameeq \quotep{Q} }
\end{mathpar}

The astute reader will have noticed that the mutual recursion of names
and processes imposes a mutual recursion on alpha-equivalence and
structural equivalence via name-equivalence. Fortunately, all of this
works out pleasantly and we may calculate in the natural way, free of
concern. The reader interested in the details is referred to the
appendix \ref{appendix:rho_details}.

\subsection{Substitution}

We use $\Proc$ for the set of processes, $\QProc$ for the set of
names, and $\id{\{}\vec{y} / \vec{x} \id{\}}$ to denote partial maps,
$s : \QProc \rightarrow \QProc$. A map, $s$ lifts, uniquely, to a map
on process terms, $\widehat{s} : \Proc \rightarrow \Proc$ by the
following equations.

\begin{mathpar}
  (0) \psubstp{Q}{P} := 0 \\
  (R \juxtap S) \psubstp{Q}{P}
  :=    
  (R)\psubstp{Q}{P} \juxtap (S) \psubstp{Q}{P} \\
  (x?(y).R) \psubstp{Q}{P}    
  :=    
  (x)\substp{Q}{P} (z)\concat( (R \psubstn{z}{y}) \psubstp{Q}{P} ) \\
  (\lift{x}{R}) \psubstp{Q}{P}  
  :=
  \lift{(x)\substp{Q}{P}}{ R \psubstp{Q}{P} } \\
%   (\dropn{x})  \psubstp{Q}{P}       
%   := 
%   \left\{ 
%     \begin{array}{ccc} 
%       \dropn{\quotep{Q}} & & x \nameeq \quotep{P} \\
%       \dropn{x} & & otherwise \\
%     \end{array}
%   \right. 
  (\dropn{x})  \psubstp{Q}{P}       
  := 
  \left\{ 
    \begin{array}{ccc} 
      Q & & x \nameeq \quotep{P} \\
      \dropn{x} & & otherwise \\
    \end{array}
  \right.
\end{mathpar}
 

where

\begin{eqnarray}
  (x)\id{\{} \lpquote Q \rpquote / \lpquote P \rpquote \id{\}}            = 
  \left\{ 
    \begin{array}{ccc}
      \lpquote Q \rpquote & & x \nameeq \lpquote P \rpquote \\
      x & & otherwise \\
    \end{array}
  \right. \nonumber
\end{eqnarray}

and $z$ is chosen distinct from $\quotep{P}$, $\quotep{Q}$, the free
names in $Q$, and all the names in $R$. Our $\alpha$-equivalence will
be built in the standard way from this substitution.

\begin{remark}\label{rem:no_self_referential_names}
  One consequence of these definitions is that $\forall P. \quotep{P}
  \not\in \freenames{P}$.
\end{remark}

\subsection{ Dynamic quote: an example }

Anticipating something of what's to come, consider applying the
substitution, $\widehat{\id{\{}u / z \id{\}}}$, to the following pair
of processes, $\lift{w}{y!(z)}$ and $w[ \lpquote y!(z) \rpquote ]$.

\begin{eqnarray}
	\lift{w}{y!(z)}\widehat{\id{\{}u / z \id{\}}}
		& = &
		\lift{w}{y!(u)} \nonumber\\
	w[ \lpquote y!(z) \rpquote ] \widehat{ \id{\{}u / z \id{\}} }
		& = &
		w[ \lpquote y!(z) \rpquote ] \nonumber
\end{eqnarray}

Because the body of the process between quotes is impervious to
substitution, we get radically different answers. In fact, by
examining the first process in an input context,
e.g. $x?(z).\lift{w}{y!(z)}$, we see that the process under the lift
operator may be shaped by prefixed inputs binding a name inside it. In
this sense, the lift operator will be seen as a way to dynamically
construct processes before reifying them as names.

Finally equipped with these standard features we can present the
dynamics of the calculus.

\subsubsection{Operational semantics} 

Finally, we introduce the computational dynamics. What marks these
algebras as distinct from other more traditionally studied algebraic
structures, e.g. vector spaces or polynomial rings, is the manner in
which dynamics is captured. In traditional structures, dynamics is typically
expressed through morphisms between such structures, as in linear maps
between vector spaces or morphisms between rings. In algebras
associated with the semantics of computation, the dynamics is
expressed as part of the algebraic structure itself, through a
reduction reduction relation typically denoted by $\red$. Below, we
give a recursive presentation of this relation for the calculus used
in the encoding.

$\red \subseteq \pi \times \pi$
$\red : \pi \to \mathcal{P}(\pi)$

\begin{mathpar}
  \inferrule* [lab=Comm] { \textsf{match}( x_{src}, x_{trgt} ) } { x_{trgt}?(y)P \; | \; x_{src}!\langle {Q} \rangle \red P\{\quotep{Q}/y}\} }
  \and \\
  \inferrule* [lab=Par] {{P} \red {P}'} {{{P} | {Q}} \red {{P}' | {Q}}}
  \and
  \inferrule* [lab=Equiv]{{{P} \scong {P}'} \andalso {{P}' \red {Q}'} \andalso {{Q}' \scong {Q}}}{{P} \red {Q}}
\end{mathpar}

\begin{eqnarray*}
  match_{\equiv} (\quotep{P},\quotep{Q}) & := & P \equiv Q \\
  match_{\dagger}(\quotep{P},\quotep{Q}) & := & \forall R. P|Q \red^{*} R => R \red^{*} 0 \\
  match_{K}(\quotep{P},\quotep{Q}) & := & K \mbox{ for some context } K
\end{eqnarray*}

$u?(x)P | u!\langle Q \rangle \red P\{\quotep{Q}/x\}$

%We write $\wred$ for $\red^*$, and $P\red$ if $\exists Q $ such that $ P \red Q$.
We write $P\red$ if $\exists Q $ such that $ P \red Q$ and $P\not\red$, otherwise.

\section{Replication}

As mentioned before, it is known that replication (and hence
recursion) can be implemented in a higher-order process algebra
\cite{SangiorgiWalker}. As our first example of calculation with the
machinery thus far presented we give the construction explicitly in
the {\rhoc}.

\begin{eqnarray}
	D_{x} & := & \prefix{x}{y}{(\binpar{\outputp{x}{y}}{@{y}})} \nonumber\\
	\bangp_{x}{P} & := & \binpar{{x}!\langle{\binpar{D_{x}}{P}}\rangle}{D_{x}} \nonumber
\end{eqnarray}

\begin{eqnarray}
	\bangp_{x}{P} & & \nonumber\\
	=
	& {x}!\langle{(\prefix{x}{y}{(\outputp{x}{y} | @{y})) | P}}\rangle 
	      | \prefix{x}{y}{(\outputp{x}{y} | @{y})} & \nonumber\\
	\red
	& (\outputp{x}{y} | @{y})\substn{\quotep{(\prefix{x}{y}{(@{y} | \outputp{x}{y})) | P}}}{y} & \nonumber\\
	=
	& \outputp{x}{\quotep{(\prefix{x}{y}{(\outputp{x}{y} | @{y})) | P}}}
	  | {(\prefix{x}{y}{(\outputp{x}{y} | @{y})) | P}} & \nonumber\\
	\red
	& \ldots & \nonumber\\
	\red^*
	& P | P | \ldots & \nonumber
\end{eqnarray}

Of course, this encoding, as an implementation, runs away, unfolding
$\bangp{P}$ eagerly. A lazier and more implementable replication
operator, restricted to input-guarded processes, may be obtained as follows.

\begin{eqnarray}
\bangp{\prefix{u}{v}{P}} 
	:= 
	\binpar{\lift{x}{\prefix{u}{v}{(\binpar{D(x)}{P})}}}{D(x)} \nonumber
\end{eqnarray}

\begin{remark}
  Note that the lazier definition still does not deal with summation
  or mixed summation (i.e. sums over input and output). The reader is
  invited to construct definitions of replication that deal with these
  features. 

  Further, the definitions are parameterized in a name, $x$. Can you,
  gentle reader, make a definition that eliminates this parameter and
  guarantees no accidental interaction between the replication
  machinery and the process being replicated -- i.e. no accidental
  sharing of names used by the process to get its work done and the
  name(s) used by the replication to effect copying. This latter
  revision of the definition of replication is crucial to obtaining
  the expected identity $!!P \sim !P$.
\end{remark}

\begin{remark}\label{rem:paradoxical_combinator}
  The reader familiar with the lambda calculus will have noticed the
  similarity between $D$ and the paradoxical combinator.

  [Ed. note: the existence of this seems to suggest we have to be more
  restrictive on the set of processes and names we admit if we are to
  support no-cloning.]
\end{remark}

\subsubsection{Bisimulation}

The computational dynamics gives rise to another kind of equivalence,
the equivalence of computational behavior. As previously mentioned
this is typically captured \emph{via} some form of bisimulation.

% The notion we use in this paper is weak barbed bisimulation
% \cite{milner91polyadicpi}.

The notion we use in this paper is derived from weak barbed
bisimulation \cite{milner91polyadicpi}. 

\begin{definition}
An \emph{observation relation}, $\downarrow_{\mathcal N}$, over a set
of names, $\mathcal N$, is the smallest relation satisfying the rules
below.

\infrule[Out-barb]{y \in {\mathcal N}, \; x \nameeq y}
		  {\outputp{x}{v} \downarrow_{\mathcal N} x}
\infrule[Par-barb]{\mbox{$P\downarrow_{\mathcal N} x$ or $Q\downarrow_{\mathcal N} x$}}
		  {\binpar{P}{Q} \downarrow_{\mathcal N} x}

We write $P \Downarrow_{\mathcal N} x$ if there is $Q$ such that 
$P \wred Q$ and $Q \downarrow_{\mathcal N} x$.
\end{definition}

\begin{definition}
%\label{def.bbisim}
An  ${\mathcal N}$-\emph{barbed bisimulation} over a set of names, ${\mathcal N}$, is a symmetric binary relation 
${\mathcal S}_{\mathcal N}$ between agents such that $P\rel{S}_{\mathcal N}Q$ implies:
\begin{enumerate}
\item If $P \red P'$ then $Q \wred Q'$ and $P'\rel{S}_{\mathcal N} Q'$.
\item If $P\downarrow_{\mathcal N} x$, then $Q\Downarrow_{\mathcal N} x$.
\end{enumerate}
$P$ is ${\mathcal N}$-barbed bisimilar to $Q$, written
$P \wbbisim_{\mathcal N} Q$, if $P \rel{S}_{\mathcal N} Q$ for some ${\mathcal N}$-barbed bisimulation ${\mathcal S}_{\mathcal N}$.
\end{definition}

$\mathcal{R} \subseteq \pi \times \pi$

$P \mathcal{R} Q => \forall P'. P \red P' \Rightarrow \exists Q'. Q \red Q', P' \mathcal{R} Q'$

$P \vdash x \Rightarrow Q \vdash x$

\begin{mathpar}
  \inferrule*[lab=Out-barb]{x \nameeq y}{{y}!\langle{Q}\rangle \vdash x}
  \and
  \inferrule*[lab=Par-barb]{\mbox{$P\vdash x$ or $Q\vdash x$}}{\binpar{P}{Q} \vdash x}
\end{mathpar}

\subsubsection{Contexts}

One of the principle advantages of computational calculi like the
$\pi$-calculus is a well-defined notion of context,
contextual-equivalence and a correlation between
contextual-equivalence and notions of bisimulation. The notion of
context allows the decomposition of a process into (sub-)process and
its syntactic environment, its context. Thus, a context may be
thought of as a process with a ``hole'' (written $\Box$) in it. The
application of a context $M$ to a process $P$, written $M[P]$, is
tantamount to filling the hole in $M$ with $P$. In this paper we do
not need the full weight of this theory, but do make use of the notion
of context in the proof the main theorem. 

\begin{mathpar}
  \inferrule* [lab=summation] {} {{M_{M},M_{N}} \bc \Box \;|\; x.M_{A} \;|\; M_{M}+M_{N}}
  \and
  \inferrule* [lab=agent] {} {{M_{A}} \bc (\vec{x})M_{P} \;| \; \clift{P_0,\ldots,M_{P},\ldots,P_N}}
  \and \\
  \inferrule* [lab=process] {} {{M_{P}} \bc M_{N} \;| \;P|M_{P} }
\end{mathpar} 

\begin{mathpar}
  \inferrule* [lab=sychronization] {} {M_{N} \bc \Box \;|\; x?M_{F} \;|\; x!M_{C}}
  \and
  \inferrule* [lab=abstraction] {} {{M_{F}} \bc (x)M_{P} }
  \and
  \inferrule* [lab=concretion] {} {{M_{C}} \bc \langle M_{P} \rangle }
  \and \\
  \inferrule* [lab=process] {} {{M_{P}} \bc M_{N} \;| \;P|M_{P} }
\end{mathpar}

\begin{definition}[contextual application] Given a context $M$, and
  process $P$, we define the \emph{contextual application}, $M[P] :=
  M\{P/\Box\}$. That is, the contextual application of M to P is the
  substitution of $P$ for $\Box$ in $M$.
\end{definition}

$\meaningof{-} : L \to \mathcal{P}(\pi)$

\begin{mathpar}
  \inferrule* [lab=collection] {} {\meaningof{true} = \pi, \and \meaningof{~E} = \pi \setminus \meaningof{E}, \and \meaningof{E_{1} \& E_{2}} = \meaningof{E_{1}} \cap \meaningof{E_{2}}}
\end{mathpar}

\begin{mathpar}
  \inferrule* [lab=structure] {} {\meaningof{0} = \{ P \in \pi | P \equiv 0 \}, \and \\ \meaningof{E_1 | E_2} = \{ P \in \pi | P \equiv P_{1} | P_{2}, P_{1} \in \meaningof{E_{1}}, P_{2} \in \meaningof{E_2}\} }
\end{mathpar}

\begin{mathpar}
 \inferrule* [lab=behavior] {} {\meaningof{\langle a?b \rangle E} = \{ P \in \pi | P \equiv Q | u?(y)P', \\ \and \\\\ \and \\ \;\;\; u \in \meaningof{a}, \forall z.P'\{z/y\} \in \meaningof{E\{z/b\}}\}, \and \\ \meaningof{a!E} = \{ P \in \pi | P \equiv Q | x!\langle P' \rangle, x \in \meaningof{a} P' \in \meaningof{E}\} }
\end{mathpar}

\begin{mathpar}
 \inferrule* [lab=nominal] {} {\meaningof{\quotep{E}} = \{ \quotep{P} \in \quotep{\pi} | P \in \meaningof{E} \}, \and \meaningof{\quotep{P}} = \{ \quotep{Q} \in \quotep{\pi} | P \equiv Q \} \and \\ \meaningof{@\quotep{E}} = \{ P \in \pi | P \equiv @x, x \in \meaningof{E} \}}
\end{mathpar}

\begin{eqnarray*}
  \\
  \meaningof{-} : TS \to ST
\end{eqnarray*}

\begin{eqnarray*}
  \\
  L : TS \to ST
\end{eqnarray*}

\begin{eqnarray*}
  \\
  P \models E \iff P \in \meaningof{E}
\end{eqnarray*}

\begin{eqnarray*}
  P \approx_{L} Q \iff \forall E \in L. P \models E \iff Q \models E
\end{eqnarray*}

\begin{eqnarray*}
  P \approx_{K} Q
\end{eqnarray*}

\begin{eqnarray*}
  P \approx Q
\end{eqnarray*}

$\approx_{K} = \approx = \approx_{L}$

\subsubsection{Contextual duality}

Note that contexts extend the quotation operation to a family of
operations from processes to names. Given a context, $M$, we can
define a \emph{nominal context}, $\quotep{M}$ by $\quotep{M}[P] :=
\quotep{M[P]}$. To foreshadow what is to come we observe that these
operations enjoy a duality with processes very much like the duality
between vectors and maps from vectors to scalars.

Further, because the calculus is essentially higher-order, we have a
correspondence between contexts and processes. More specifically,
given a name $x$ and a context $M$ we can construct $M^{*}_{x}$ such
that 

\begin{mathpar}
  M^{*}_{x} | \lift{x}{P} \red M[P]
\end{mathpar}

namely,

\begin{mathpar}
  M^{*}_{x} := x?(u).M[\dropn{u}]
\end{mathpar}

The dependence of $M^{*}_{x}$ on a name makes it an abstraction, 

\begin{mathpar}
  M^{*} := (x)x?(u).M[\dropn{u}]
\end{mathpar}

\subsection{Additional notation}

It will sometimes be convenient to denote the process a name
quotes. We already have the notation $x = \quotep{P}$, but it will be
convenient to introduce an alternate notation, $\procn{x}$, when we
want to emphasize the connection to the use of the name. Note that, by
virtue of name equivalence, $\quotep{\procn{x}} \nameeq x$; so, the
notation is consistent with previous definitions.

Further, because names have structure it is possible to effect
substitutions on the basis of that structure. This means we need to
upgrade our notation for substitutions, which we accomplish by
adapting comprehension notation. Thus,

\begin{mathpar}
  P\{ y / x : x \in S \}
\end{mathpar}

is interpreted to mean the process derived from P by replacing (in a
capture-avoiding manner) each occurrence of $x$ in $S$ by $y$. For example,

\begin{mathpar}
  P\{ \quotep{\procn{x}|\procn{x}} / x : x \in \freenames{P} \}
\end{mathpar}

will replace each (occurrence) of a free name $x$ in $P$ by
$\quotep{\procn{x}|\procn{x}}$.

Also, we will avail ourselves of the notation $x^{L}$ and $x^{R}$ to
denote injections of a name into disjoint copies of the name
space. There are numerous ways to accomplish this. One example can be
found in \cite{MeredithR05}. This notation overloads to vectors of
names: $\vec{x}^{\pi} := (x_{i}^{\pi} \; : \; 0 \leq i < |\vec{x}| )$ where $\pi \in \{L,R\}$.

We also use $P^{\Box} := P|\Box$.

In \cite{MeredithR05} an interpretation of the new operator is
given. It turns out that there are several possible interpretations
all enjoying the requisite algebraic properties of the operator (see
\cite{milner91polyadicpi}). We will therefore make liberal use of
$(\nu\; \vec{x})P$.

% subsection the_syntax_and_semantics_of_the_notation_system (end)   

\input{qm2pi.qmops} 

\input{qm2pi.sterngerlach} 

\input{qm2pi.metric} 

% section concurrent_process_calculi (end)

%\input{qm2pi.proofsketch}

% section proof sketch (end)

%\input{qm2pi.slviaknots} 

% section spatial logic via knots (end)

\input{qm2pi.conclusion}

% section conclusion (end)

%\input{qm2pi.dtcodes} 

% section wiring algorithm (end)

\input{qm2pi.ack} 

% section acknowledgments (end)

\newpage


\bibliographystyle{plain}   
\bibliography{../../biblios/main.bib}

\input{qm2pi.rhodetails}

\end{document}

 

\documentclass[12pt]{llncs}
%\documentclass{jktr}

\usepackage[pdftex]{hyperref}                   
\usepackage {listings}
\usepackage {mathpartir}
\usepackage{bcprules}
%\usepackage{listings}
                       
\usepackage{graphicx} 
%\usepackage[margins=2.5cm,nohead,nofoot]{geometry}
%\usepackage{geometry}
\usepackage{amsfonts}
\usepackage{amstext}
\usepackage{latexsym}
\usepackage{amssymb}
\usepackage{color}


%\include{myPreamble}
\include{qm2pi.local} 

%\ifpdf
%\usepackage[pdftex]{graphicx}
%\else
%\usepackage{graphicx}
%\fi

 % \ifpdf
%  \usepackage{pdfsync}
%  \if


%\title{Brief Article}
%\author{David F. Snyder}
%\author{L.G. Meredith}

%\address{Dept. of Math., Texas State University--San Marcos, San Marcos, TX 78666}
       
\pagestyle{empty}


\begin{document}

\lstset{language=[Objective]Caml,frame=shadowbox}

\input{qm2pi.front}

% section front matter (end)

\input{qm2pi.intro} 
 
% section introduction (end)

% \input{qm2pi.knotations} 

% section notation (end)

\input{qm2pi.process.calculi} 

% section concurrent_process_calculi_and_spatial_logics_ (end)
    
%\input{qm2pi.knots2pi} 

%\input{qm2pi.trefoil} 

%\input{qm2pi.mainthm} 

% subsection basic_interpretation (end)

%\input{qm2pi.rho.presentation} 
\subsection{The syntax and semantics of the notation system}\label{sub:the_syntax_and_semantics_of_the_notation_system} % (fold)

We now summarize a technical presentation of the calculus that
embodies our theory of dynamics. The typical presentation of such a
calculus follows the style of giving generators and relations on
them. The grammar, below, describing term constructors, freely
generates the set of processes, $\Proc$. This set is then quotiented
by a relation known as structural congruence and it is over this set
that the notion of dynamics is expressed. This presentation is
essentially that of \cite{MeredithR05} with the addition of
polyadicity and summation. For readability we have relegated some of
the technical subtleties to an appendix.

\subsubsection{Process grammar}\label{subsub:process_grammar}

\begin{mathpar}
  \inferrule* [lab=synchronization] {} {{M} \bc \pzero \;|\; x?F \;|\; x!C }
  \and
  \inferrule* [lab=abstraction] {} {{F} \bc (x)P}
  \and
  \inferrule* [lab=concretion] {} {{C} \bc \langle Q \rangle}
  \and
  \inferrule* [lab=process] {} {{P,Q} \bc M \;| \;P|Q \;|\; @{x}}
  \and
  \inferrule* [lab=name] {} {{x} \bc \quotep{P}}
\end{mathpar} 

Note that $\vec{x}$ (resp. $\vec{P}$) denotes a vector of names
(resp. processes) of length $|\vec{x}|$ (resp. $|\vec{P}|$). We adopt
the following useful abbreviations.

\begin{mathpar}
   x?(\vec{y}).P := x.(\vec{y})P \and  x\clift{\vec{P}} := x.\clift{\vec{P}}
   \and x!(y) := \lift{x}{\dropn{y}}
   \and \Pi_{i=0}^{n-1}P_i := P_0 | \ldots | P_{n-1}
\end{mathpar}

\subsubsection{Structural congruence}

\paragraph{Free and bound names and alpha-equivalence.} At the
core of structural equivalence is alpha-equivalence which identifies
process that are the same up to a change of variable. Formally, we
recognize the distinction between free and bound names. The free names
of a process, $\freenames{P}$, may be calculated recursively as
follows:

\begin{mathpar}
\freenames{\pzero} := \emptyset
  \and \\
  \freenames{x?(y).P} := \{ x \} \cup (\freenames{P} \setminus \{ y \})
  \and 
  \freenames{x!\langle P \rangle} := \{ x \} \cup \{ P \} 
  \and \\
  \freenames{P|Q} := \freenames{P} \cup \freenames{Q}
  \and \\
  \freenames{@{x}} := \{ x \}
\end{mathpar}

$\pi$
$\quotep{\pi}$

$\freenames{-} : \pi \to \mathcal{P}(\quotep{\pi})$

\begin{eqnarray*}
  \freenames{\pzero} & := & \emptyset \\
  \freenames{x?(y).P} & := & \{ x \} \cup (\freenames{P} \setminus \{ y \}) \\
  \freenames{x!\langle P \rangle} & := & \{ x \} \cup \{ P \} \\
  \freenames{P|Q} & := & \freenames{P} \cup \freenames{Q} \\
  \freenames{\dropn{x}} & := & \{ x \}
\end{eqnarray*}

The bound names of a process, $\boundnames{P}$, are those names occurring in $P$
that are not free. For example, in $x?(y).0$, the name $x$ is free, while $y$ is bound.

\begin{mathpar}
  \inferrule* [lab=monoidal-laws] {} { P|Q \equiv Q|P \and P|0 \equiv P \and P|(Q|R) \equiv (P|Q)|R }
\end{mathpar}

\begin{mathpar}
  \inferrule* [lab=alpha-equivalence] {} { (x)P \equiv (y)P\{y/x\} \and y \not\in \freenames{P} }
\end{mathpar}

\begin{definition}
Then two processes, $P,Q$, are alpha-equivalent if $P = Q\{\vec{y}/\vec{x}\}$ for
some $\vec{x} \in \boundnames{Q},\vec{y} \in \boundnames{P}$, where $Q\{\vec{y}/\vec{x}\}$
denotes the capture-avoiding substitution of $\vec{y}$ for $\vec{x}$ in $Q$.
\end{definition}

\begin{definition}
  The {\em structural congruence} \cite{SangiorgiWalker} , $\equiv$,
  between processes is the least congruence containing
  alpha-equivalence, satisfying the abelian monoid laws
  (associativity, commutativity and $\pzero$ as identity) for parallel
  composition $|$ and for summation $+$.
\end{definition}

\subsection{Name equivalence}

We take name equivalence, written $\nameeq$, to be the smallest
equivalence relation generated by the following rules.

\begin{mathpar}
\inferrule*[lab=Quote-drop]
{ }
{ \quotep{@{x}} \nameeq x }

\inferrule*[lab=Struct-equiv]
{ P \scong Q }
{ \quotep{P} \nameeq \quotep{Q} }
\end{mathpar}

The astute reader will have noticed that the mutual recursion of names
and processes imposes a mutual recursion on alpha-equivalence and
structural equivalence via name-equivalence. Fortunately, all of this
works out pleasantly and we may calculate in the natural way, free of
concern. The reader interested in the details is referred to the
appendix \ref{appendix:rho_details}.

\subsection{Substitution}

We use $\Proc$ for the set of processes, $\QProc$ for the set of
names, and $\id{\{}\vec{y} / \vec{x} \id{\}}$ to denote partial maps,
$s : \QProc \rightarrow \QProc$. A map, $s$ lifts, uniquely, to a map
on process terms, $\widehat{s} : \Proc \rightarrow \Proc$ by the
following equations.

\begin{mathpar}
  (0) \psubstp{Q}{P} := 0 \\
  (R \juxtap S) \psubstp{Q}{P}
  :=    
  (R)\psubstp{Q}{P} \juxtap (S) \psubstp{Q}{P} \\
  (x?(y).R) \psubstp{Q}{P}    
  :=    
  (x)\substp{Q}{P} (z)\concat( (R \psubstn{z}{y}) \psubstp{Q}{P} ) \\
  (\lift{x}{R}) \psubstp{Q}{P}  
  :=
  \lift{(x)\substp{Q}{P}}{ R \psubstp{Q}{P} } \\
%   (\dropn{x})  \psubstp{Q}{P}       
%   := 
%   \left\{ 
%     \begin{array}{ccc} 
%       \dropn{\quotep{Q}} & & x \nameeq \quotep{P} \\
%       \dropn{x} & & otherwise \\
%     \end{array}
%   \right. 
  (\dropn{x})  \psubstp{Q}{P}       
  := 
  \left\{ 
    \begin{array}{ccc} 
      Q & & x \nameeq \quotep{P} \\
      \dropn{x} & & otherwise \\
    \end{array}
  \right.
\end{mathpar}
 

where

\begin{eqnarray}
  (x)\id{\{} \lpquote Q \rpquote / \lpquote P \rpquote \id{\}}            = 
  \left\{ 
    \begin{array}{ccc}
      \lpquote Q \rpquote & & x \nameeq \lpquote P \rpquote \\
      x & & otherwise \\
    \end{array}
  \right. \nonumber
\end{eqnarray}

and $z$ is chosen distinct from $\quotep{P}$, $\quotep{Q}$, the free
names in $Q$, and all the names in $R$. Our $\alpha$-equivalence will
be built in the standard way from this substitution.

\begin{remark}\label{rem:no_self_referential_names}
  One consequence of these definitions is that $\forall P. \quotep{P}
  \not\in \freenames{P}$.
\end{remark}

\subsection{ Dynamic quote: an example }

Anticipating something of what's to come, consider applying the
substitution, $\widehat{\id{\{}u / z \id{\}}}$, to the following pair
of processes, $\lift{w}{y!(z)}$ and $w[ \lpquote y!(z) \rpquote ]$.

\begin{eqnarray}
	\lift{w}{y!(z)}\widehat{\id{\{}u / z \id{\}}}
		& = &
		\lift{w}{y!(u)} \nonumber\\
	w[ \lpquote y!(z) \rpquote ] \widehat{ \id{\{}u / z \id{\}} }
		& = &
		w[ \lpquote y!(z) \rpquote ] \nonumber
\end{eqnarray}

Because the body of the process between quotes is impervious to
substitution, we get radically different answers. In fact, by
examining the first process in an input context,
e.g. $x?(z).\lift{w}{y!(z)}$, we see that the process under the lift
operator may be shaped by prefixed inputs binding a name inside it. In
this sense, the lift operator will be seen as a way to dynamically
construct processes before reifying them as names.

Finally equipped with these standard features we can present the
dynamics of the calculus.

\subsubsection{Operational semantics} 

Finally, we introduce the computational dynamics. What marks these
algebras as distinct from other more traditionally studied algebraic
structures, e.g. vector spaces or polynomial rings, is the manner in
which dynamics is captured. In traditional structures, dynamics is typically
expressed through morphisms between such structures, as in linear maps
between vector spaces or morphisms between rings. In algebras
associated with the semantics of computation, the dynamics is
expressed as part of the algebraic structure itself, through a
reduction reduction relation typically denoted by $\red$. Below, we
give a recursive presentation of this relation for the calculus used
in the encoding.

$\red \subseteq \pi \times \pi$
$\red : \pi \to \mathcal{P}(\pi)$

\begin{mathpar}
  \inferrule* [lab=Comm] { \textsf{match}( x_{src}, x_{trgt} ) } { x_{trgt}?(y)P \; | \; x_{src}!\langle {Q} \rangle \red P\{\quotep{Q}/y}\} }
  \and \\
  \inferrule* [lab=Par] {{P} \red {P}'} {{{P} | {Q}} \red {{P}' | {Q}}}
  \and
  \inferrule* [lab=Equiv]{{{P} \scong {P}'} \andalso {{P}' \red {Q}'} \andalso {{Q}' \scong {Q}}}{{P} \red {Q}}
\end{mathpar}

\begin{eqnarray*}
  match_{\equiv} (\quotep{P},\quotep{Q}) & := & P \equiv Q \\
  match_{\dagger}(\quotep{P},\quotep{Q}) & := & \forall R. P|Q \red^{*} R => R \red^{*} 0 \\
  match_{K}(\quotep{P},\quotep{Q}) & := & K \mbox{ for some context } K
\end{eqnarray*}

$u?(x)P | u!\langle Q \rangle \red P\{\quotep{Q}/x\}$

%We write $\wred$ for $\red^*$, and $P\red$ if $\exists Q $ such that $ P \red Q$.
We write $P\red$ if $\exists Q $ such that $ P \red Q$ and $P\not\red$, otherwise.

\section{Replication}

As mentioned before, it is known that replication (and hence
recursion) can be implemented in a higher-order process algebra
\cite{SangiorgiWalker}. As our first example of calculation with the
machinery thus far presented we give the construction explicitly in
the {\rhoc}.

\begin{eqnarray}
	D_{x} & := & \prefix{x}{y}{(\binpar{\outputp{x}{y}}{@{y}})} \nonumber\\
	\bangp_{x}{P} & := & \binpar{{x}!\langle{\binpar{D_{x}}{P}}\rangle}{D_{x}} \nonumber
\end{eqnarray}

\begin{eqnarray}
	\bangp_{x}{P} & & \nonumber\\
	=
	& {x}!\langle{(\prefix{x}{y}{(\outputp{x}{y} | @{y})) | P}}\rangle 
	      | \prefix{x}{y}{(\outputp{x}{y} | @{y})} & \nonumber\\
	\red
	& (\outputp{x}{y} | @{y})\substn{\quotep{(\prefix{x}{y}{(@{y} | \outputp{x}{y})) | P}}}{y} & \nonumber\\
	=
	& \outputp{x}{\quotep{(\prefix{x}{y}{(\outputp{x}{y} | @{y})) | P}}}
	  | {(\prefix{x}{y}{(\outputp{x}{y} | @{y})) | P}} & \nonumber\\
	\red
	& \ldots & \nonumber\\
	\red^*
	& P | P | \ldots & \nonumber
\end{eqnarray}

Of course, this encoding, as an implementation, runs away, unfolding
$\bangp{P}$ eagerly. A lazier and more implementable replication
operator, restricted to input-guarded processes, may be obtained as follows.

\begin{eqnarray}
\bangp{\prefix{u}{v}{P}} 
	:= 
	\binpar{\lift{x}{\prefix{u}{v}{(\binpar{D(x)}{P})}}}{D(x)} \nonumber
\end{eqnarray}

\begin{remark}
  Note that the lazier definition still does not deal with summation
  or mixed summation (i.e. sums over input and output). The reader is
  invited to construct definitions of replication that deal with these
  features. 

  Further, the definitions are parameterized in a name, $x$. Can you,
  gentle reader, make a definition that eliminates this parameter and
  guarantees no accidental interaction between the replication
  machinery and the process being replicated -- i.e. no accidental
  sharing of names used by the process to get its work done and the
  name(s) used by the replication to effect copying. This latter
  revision of the definition of replication is crucial to obtaining
  the expected identity $!!P \sim !P$.
\end{remark}

\begin{remark}\label{rem:paradoxical_combinator}
  The reader familiar with the lambda calculus will have noticed the
  similarity between $D$ and the paradoxical combinator.

  [Ed. note: the existence of this seems to suggest we have to be more
  restrictive on the set of processes and names we admit if we are to
  support no-cloning.]
\end{remark}

\subsubsection{Bisimulation}

The computational dynamics gives rise to another kind of equivalence,
the equivalence of computational behavior. As previously mentioned
this is typically captured \emph{via} some form of bisimulation.

% The notion we use in this paper is weak barbed bisimulation
% \cite{milner91polyadicpi}.

The notion we use in this paper is derived from weak barbed
bisimulation \cite{milner91polyadicpi}. 

\begin{definition}
An \emph{observation relation}, $\downarrow_{\mathcal N}$, over a set
of names, $\mathcal N$, is the smallest relation satisfying the rules
below.

\infrule[Out-barb]{y \in {\mathcal N}, \; x \nameeq y}
		  {\outputp{x}{v} \downarrow_{\mathcal N} x}
\infrule[Par-barb]{\mbox{$P\downarrow_{\mathcal N} x$ or $Q\downarrow_{\mathcal N} x$}}
		  {\binpar{P}{Q} \downarrow_{\mathcal N} x}

We write $P \Downarrow_{\mathcal N} x$ if there is $Q$ such that 
$P \wred Q$ and $Q \downarrow_{\mathcal N} x$.
\end{definition}

\begin{definition}
%\label{def.bbisim}
An  ${\mathcal N}$-\emph{barbed bisimulation} over a set of names, ${\mathcal N}$, is a symmetric binary relation 
${\mathcal S}_{\mathcal N}$ between agents such that $P\rel{S}_{\mathcal N}Q$ implies:
\begin{enumerate}
\item If $P \red P'$ then $Q \wred Q'$ and $P'\rel{S}_{\mathcal N} Q'$.
\item If $P\downarrow_{\mathcal N} x$, then $Q\Downarrow_{\mathcal N} x$.
\end{enumerate}
$P$ is ${\mathcal N}$-barbed bisimilar to $Q$, written
$P \wbbisim_{\mathcal N} Q$, if $P \rel{S}_{\mathcal N} Q$ for some ${\mathcal N}$-barbed bisimulation ${\mathcal S}_{\mathcal N}$.
\end{definition}

$\mathcal{R} \subseteq \pi \times \pi$

$P \mathcal{R} Q => \forall P'. P \red P' \Rightarrow \exists Q'. Q \red Q', P' \mathcal{R} Q'$

$P \vdash x \Rightarrow Q \vdash x$

\begin{mathpar}
  \inferrule*[lab=Out-barb]{x \nameeq y}{{y}!\langle{Q}\rangle \vdash x}
  \and
  \inferrule*[lab=Par-barb]{\mbox{$P\vdash x$ or $Q\vdash x$}}{\binpar{P}{Q} \vdash x}
\end{mathpar}

\subsubsection{Contexts}

One of the principle advantages of computational calculi like the
$\pi$-calculus is a well-defined notion of context,
contextual-equivalence and a correlation between
contextual-equivalence and notions of bisimulation. The notion of
context allows the decomposition of a process into (sub-)process and
its syntactic environment, its context. Thus, a context may be
thought of as a process with a ``hole'' (written $\Box$) in it. The
application of a context $M$ to a process $P$, written $M[P]$, is
tantamount to filling the hole in $M$ with $P$. In this paper we do
not need the full weight of this theory, but do make use of the notion
of context in the proof the main theorem. 

\begin{mathpar}
  \inferrule* [lab=summation] {} {{M_{M},M_{N}} \bc \Box \;|\; x.M_{A} \;|\; M_{M}+M_{N}}
  \and
  \inferrule* [lab=agent] {} {{M_{A}} \bc (\vec{x})M_{P} \;| \; \clift{P_0,\ldots,M_{P},\ldots,P_N}}
  \and \\
  \inferrule* [lab=process] {} {{M_{P}} \bc M_{N} \;| \;P|M_{P} }
\end{mathpar} 

\begin{mathpar}
  \inferrule* [lab=sychronization] {} {M_{N} \bc \Box \;|\; x?M_{F} \;|\; x!M_{C}}
  \and
  \inferrule* [lab=abstraction] {} {{M_{F}} \bc (x)M_{P} }
  \and
  \inferrule* [lab=concretion] {} {{M_{C}} \bc \langle M_{P} \rangle }
  \and \\
  \inferrule* [lab=process] {} {{M_{P}} \bc M_{N} \;| \;P|M_{P} }
\end{mathpar}

\begin{definition}[contextual application] Given a context $M$, and
  process $P$, we define the \emph{contextual application}, $M[P] :=
  M\{P/\Box\}$. That is, the contextual application of M to P is the
  substitution of $P$ for $\Box$ in $M$.
\end{definition}

$\meaningof{-} : L \to \mathcal{P}(\pi)$

\begin{mathpar}
  \inferrule* [lab=collection] {} {\meaningof{true} = \pi, \and \meaningof{~E} = \pi \setminus \meaningof{E}, \and \meaningof{E_{1} \& E_{2}} = \meaningof{E_{1}} \cap \meaningof{E_{2}}}
\end{mathpar}

\begin{mathpar}
  \inferrule* [lab=structure] {} {\meaningof{0} = \{ P \in \pi | P \equiv 0 \}, \and \\ \meaningof{E_1 | E_2} = \{ P \in \pi | P \equiv P_{1} | P_{2}, P_{1} \in \meaningof{E_{1}}, P_{2} \in \meaningof{E_2}\} }
\end{mathpar}

\begin{mathpar}
 \inferrule* [lab=behavior] {} {\meaningof{\langle a?b \rangle E} = \{ P \in \pi | P \equiv Q | u?(y)P', \\ \and \\\\ \and \\ \;\;\; u \in \meaningof{a}, \forall z.P'\{z/y\} \in \meaningof{E\{z/b\}}\}, \and \\ \meaningof{a!E} = \{ P \in \pi | P \equiv Q | x!\langle P' \rangle, x \in \meaningof{a} P' \in \meaningof{E}\} }
\end{mathpar}

\begin{mathpar}
 \inferrule* [lab=nominal] {} {\meaningof{\quotep{E}} = \{ \quotep{P} \in \quotep{\pi} | P \in \meaningof{E} \}, \and \meaningof{\quotep{P}} = \{ \quotep{Q} \in \quotep{\pi} | P \equiv Q \} \and \\ \meaningof{@\quotep{E}} = \{ P \in \pi | P \equiv @x, x \in \meaningof{E} \}}
\end{mathpar}

\begin{eqnarray*}
  \\
  \meaningof{-} : TS \to ST
\end{eqnarray*}

\begin{eqnarray*}
  \\
  L : TS \to ST
\end{eqnarray*}

\begin{eqnarray*}
  \\
  P \models E \iff P \in \meaningof{E}
\end{eqnarray*}

\begin{eqnarray*}
  P \approx_{L} Q \iff \forall E \in L. P \models E \iff Q \models E
\end{eqnarray*}

\begin{eqnarray*}
  P \approx_{K} Q
\end{eqnarray*}

\begin{eqnarray*}
  P \approx Q
\end{eqnarray*}

$\approx_{K} = \approx = \approx_{L}$

\subsubsection{Contextual duality}

Note that contexts extend the quotation operation to a family of
operations from processes to names. Given a context, $M$, we can
define a \emph{nominal context}, $\quotep{M}$ by $\quotep{M}[P] :=
\quotep{M[P]}$. To foreshadow what is to come we observe that these
operations enjoy a duality with processes very much like the duality
between vectors and maps from vectors to scalars.

Further, because the calculus is essentially higher-order, we have a
correspondence between contexts and processes. More specifically,
given a name $x$ and a context $M$ we can construct $M^{*}_{x}$ such
that 

\begin{mathpar}
  M^{*}_{x} | \lift{x}{P} \red M[P]
\end{mathpar}

namely,

\begin{mathpar}
  M^{*}_{x} := x?(u).M[\dropn{u}]
\end{mathpar}

The dependence of $M^{*}_{x}$ on a name makes it an abstraction, 

\begin{mathpar}
  M^{*} := (x)x?(u).M[\dropn{u}]
\end{mathpar}

\subsection{Additional notation}

It will sometimes be convenient to denote the process a name
quotes. We already have the notation $x = \quotep{P}$, but it will be
convenient to introduce an alternate notation, $\procn{x}$, when we
want to emphasize the connection to the use of the name. Note that, by
virtue of name equivalence, $\quotep{\procn{x}} \nameeq x$; so, the
notation is consistent with previous definitions.

Further, because names have structure it is possible to effect
substitutions on the basis of that structure. This means we need to
upgrade our notation for substitutions, which we accomplish by
adapting comprehension notation. Thus,

\begin{mathpar}
  P\{ y / x : x \in S \}
\end{mathpar}

is interpreted to mean the process derived from P by replacing (in a
capture-avoiding manner) each occurrence of $x$ in $S$ by $y$. For example,

\begin{mathpar}
  P\{ \quotep{\procn{x}|\procn{x}} / x : x \in \freenames{P} \}
\end{mathpar}

will replace each (occurrence) of a free name $x$ in $P$ by
$\quotep{\procn{x}|\procn{x}}$.

Also, we will avail ourselves of the notation $x^{L}$ and $x^{R}$ to
denote injections of a name into disjoint copies of the name
space. There are numerous ways to accomplish this. One example can be
found in \cite{MeredithR05}. This notation overloads to vectors of
names: $\vec{x}^{\pi} := (x_{i}^{\pi} \; : \; 0 \leq i < |\vec{x}| )$ where $\pi \in \{L,R\}$.

We also use $P^{\Box} := P|\Box$.

In \cite{MeredithR05} an interpretation of the new operator is
given. It turns out that there are several possible interpretations
all enjoying the requisite algebraic properties of the operator (see
\cite{milner91polyadicpi}). We will therefore make liberal use of
$(\nu\; \vec{x})P$.

% subsection the_syntax_and_semantics_of_the_notation_system (end)   

\input{qm2pi.qmops} 

\input{qm2pi.sterngerlach} 

\input{qm2pi.metric} 

% section concurrent_process_calculi (end)

%\input{qm2pi.proofsketch}

% section proof sketch (end)

%\input{qm2pi.slviaknots} 

% section spatial logic via knots (end)

\input{qm2pi.conclusion}

% section conclusion (end)

%\input{qm2pi.dtcodes} 

% section wiring algorithm (end)

\input{qm2pi.ack} 

% section acknowledgments (end)

\newpage


\bibliographystyle{plain}   
\bibliography{../../biblios/main.bib}

\input{qm2pi.rhodetails}

\end{document}

 

% section concurrent_process_calculi (end)

%\documentclass[12pt]{llncs}
%\documentclass{jktr}

\usepackage[pdftex]{hyperref}                   
\usepackage {listings}
\usepackage {mathpartir}
\usepackage{bcprules}
%\usepackage{listings}
                       
\usepackage{graphicx} 
%\usepackage[margins=2.5cm,nohead,nofoot]{geometry}
%\usepackage{geometry}
\usepackage{amsfonts}
\usepackage{amstext}
\usepackage{latexsym}
\usepackage{amssymb}
\usepackage{color}


%\include{myPreamble}
\include{qm2pi.local} 

%\ifpdf
%\usepackage[pdftex]{graphicx}
%\else
%\usepackage{graphicx}
%\fi

 % \ifpdf
%  \usepackage{pdfsync}
%  \if


%\title{Brief Article}
%\author{David F. Snyder}
%\author{L.G. Meredith}

%\address{Dept. of Math., Texas State University--San Marcos, San Marcos, TX 78666}
       
\pagestyle{empty}


\begin{document}

\lstset{language=[Objective]Caml,frame=shadowbox}

\input{qm2pi.front}

% section front matter (end)

\input{qm2pi.intro} 
 
% section introduction (end)

% \input{qm2pi.knotations} 

% section notation (end)

\input{qm2pi.process.calculi} 

% section concurrent_process_calculi_and_spatial_logics_ (end)
    
%\input{qm2pi.knots2pi} 

%\input{qm2pi.trefoil} 

%\input{qm2pi.mainthm} 

% subsection basic_interpretation (end)

%\input{qm2pi.rho.presentation} 
\subsection{The syntax and semantics of the notation system}\label{sub:the_syntax_and_semantics_of_the_notation_system} % (fold)

We now summarize a technical presentation of the calculus that
embodies our theory of dynamics. The typical presentation of such a
calculus follows the style of giving generators and relations on
them. The grammar, below, describing term constructors, freely
generates the set of processes, $\Proc$. This set is then quotiented
by a relation known as structural congruence and it is over this set
that the notion of dynamics is expressed. This presentation is
essentially that of \cite{MeredithR05} with the addition of
polyadicity and summation. For readability we have relegated some of
the technical subtleties to an appendix.

\subsubsection{Process grammar}\label{subsub:process_grammar}

\begin{mathpar}
  \inferrule* [lab=synchronization] {} {{M} \bc \pzero \;|\; x?F \;|\; x!C }
  \and
  \inferrule* [lab=abstraction] {} {{F} \bc (x)P}
  \and
  \inferrule* [lab=concretion] {} {{C} \bc \langle Q \rangle}
  \and
  \inferrule* [lab=process] {} {{P,Q} \bc M \;| \;P|Q \;|\; @{x}}
  \and
  \inferrule* [lab=name] {} {{x} \bc \quotep{P}}
\end{mathpar} 

Note that $\vec{x}$ (resp. $\vec{P}$) denotes a vector of names
(resp. processes) of length $|\vec{x}|$ (resp. $|\vec{P}|$). We adopt
the following useful abbreviations.

\begin{mathpar}
   x?(\vec{y}).P := x.(\vec{y})P \and  x\clift{\vec{P}} := x.\clift{\vec{P}}
   \and x!(y) := \lift{x}{\dropn{y}}
   \and \Pi_{i=0}^{n-1}P_i := P_0 | \ldots | P_{n-1}
\end{mathpar}

\subsubsection{Structural congruence}

\paragraph{Free and bound names and alpha-equivalence.} At the
core of structural equivalence is alpha-equivalence which identifies
process that are the same up to a change of variable. Formally, we
recognize the distinction between free and bound names. The free names
of a process, $\freenames{P}$, may be calculated recursively as
follows:

\begin{mathpar}
\freenames{\pzero} := \emptyset
  \and \\
  \freenames{x?(y).P} := \{ x \} \cup (\freenames{P} \setminus \{ y \})
  \and 
  \freenames{x!\langle P \rangle} := \{ x \} \cup \{ P \} 
  \and \\
  \freenames{P|Q} := \freenames{P} \cup \freenames{Q}
  \and \\
  \freenames{@{x}} := \{ x \}
\end{mathpar}

$\pi$
$\quotep{\pi}$

$\freenames{-} : \pi \to \mathcal{P}(\quotep{\pi})$

\begin{eqnarray*}
  \freenames{\pzero} & := & \emptyset \\
  \freenames{x?(y).P} & := & \{ x \} \cup (\freenames{P} \setminus \{ y \}) \\
  \freenames{x!\langle P \rangle} & := & \{ x \} \cup \{ P \} \\
  \freenames{P|Q} & := & \freenames{P} \cup \freenames{Q} \\
  \freenames{\dropn{x}} & := & \{ x \}
\end{eqnarray*}

The bound names of a process, $\boundnames{P}$, are those names occurring in $P$
that are not free. For example, in $x?(y).0$, the name $x$ is free, while $y$ is bound.

\begin{mathpar}
  \inferrule* [lab=monoidal-laws] {} { P|Q \equiv Q|P \and P|0 \equiv P \and P|(Q|R) \equiv (P|Q)|R }
\end{mathpar}

\begin{mathpar}
  \inferrule* [lab=alpha-equivalence] {} { (x)P \equiv (y)P\{y/x\} \and y \not\in \freenames{P} }
\end{mathpar}

\begin{definition}
Then two processes, $P,Q$, are alpha-equivalent if $P = Q\{\vec{y}/\vec{x}\}$ for
some $\vec{x} \in \boundnames{Q},\vec{y} \in \boundnames{P}$, where $Q\{\vec{y}/\vec{x}\}$
denotes the capture-avoiding substitution of $\vec{y}$ for $\vec{x}$ in $Q$.
\end{definition}

\begin{definition}
  The {\em structural congruence} \cite{SangiorgiWalker} , $\equiv$,
  between processes is the least congruence containing
  alpha-equivalence, satisfying the abelian monoid laws
  (associativity, commutativity and $\pzero$ as identity) for parallel
  composition $|$ and for summation $+$.
\end{definition}

\subsection{Name equivalence}

We take name equivalence, written $\nameeq$, to be the smallest
equivalence relation generated by the following rules.

\begin{mathpar}
\inferrule*[lab=Quote-drop]
{ }
{ \quotep{@{x}} \nameeq x }

\inferrule*[lab=Struct-equiv]
{ P \scong Q }
{ \quotep{P} \nameeq \quotep{Q} }
\end{mathpar}

The astute reader will have noticed that the mutual recursion of names
and processes imposes a mutual recursion on alpha-equivalence and
structural equivalence via name-equivalence. Fortunately, all of this
works out pleasantly and we may calculate in the natural way, free of
concern. The reader interested in the details is referred to the
appendix \ref{appendix:rho_details}.

\subsection{Substitution}

We use $\Proc$ for the set of processes, $\QProc$ for the set of
names, and $\id{\{}\vec{y} / \vec{x} \id{\}}$ to denote partial maps,
$s : \QProc \rightarrow \QProc$. A map, $s$ lifts, uniquely, to a map
on process terms, $\widehat{s} : \Proc \rightarrow \Proc$ by the
following equations.

\begin{mathpar}
  (0) \psubstp{Q}{P} := 0 \\
  (R \juxtap S) \psubstp{Q}{P}
  :=    
  (R)\psubstp{Q}{P} \juxtap (S) \psubstp{Q}{P} \\
  (x?(y).R) \psubstp{Q}{P}    
  :=    
  (x)\substp{Q}{P} (z)\concat( (R \psubstn{z}{y}) \psubstp{Q}{P} ) \\
  (\lift{x}{R}) \psubstp{Q}{P}  
  :=
  \lift{(x)\substp{Q}{P}}{ R \psubstp{Q}{P} } \\
%   (\dropn{x})  \psubstp{Q}{P}       
%   := 
%   \left\{ 
%     \begin{array}{ccc} 
%       \dropn{\quotep{Q}} & & x \nameeq \quotep{P} \\
%       \dropn{x} & & otherwise \\
%     \end{array}
%   \right. 
  (\dropn{x})  \psubstp{Q}{P}       
  := 
  \left\{ 
    \begin{array}{ccc} 
      Q & & x \nameeq \quotep{P} \\
      \dropn{x} & & otherwise \\
    \end{array}
  \right.
\end{mathpar}
 

where

\begin{eqnarray}
  (x)\id{\{} \lpquote Q \rpquote / \lpquote P \rpquote \id{\}}            = 
  \left\{ 
    \begin{array}{ccc}
      \lpquote Q \rpquote & & x \nameeq \lpquote P \rpquote \\
      x & & otherwise \\
    \end{array}
  \right. \nonumber
\end{eqnarray}

and $z$ is chosen distinct from $\quotep{P}$, $\quotep{Q}$, the free
names in $Q$, and all the names in $R$. Our $\alpha$-equivalence will
be built in the standard way from this substitution.

\begin{remark}\label{rem:no_self_referential_names}
  One consequence of these definitions is that $\forall P. \quotep{P}
  \not\in \freenames{P}$.
\end{remark}

\subsection{ Dynamic quote: an example }

Anticipating something of what's to come, consider applying the
substitution, $\widehat{\id{\{}u / z \id{\}}}$, to the following pair
of processes, $\lift{w}{y!(z)}$ and $w[ \lpquote y!(z) \rpquote ]$.

\begin{eqnarray}
	\lift{w}{y!(z)}\widehat{\id{\{}u / z \id{\}}}
		& = &
		\lift{w}{y!(u)} \nonumber\\
	w[ \lpquote y!(z) \rpquote ] \widehat{ \id{\{}u / z \id{\}} }
		& = &
		w[ \lpquote y!(z) \rpquote ] \nonumber
\end{eqnarray}

Because the body of the process between quotes is impervious to
substitution, we get radically different answers. In fact, by
examining the first process in an input context,
e.g. $x?(z).\lift{w}{y!(z)}$, we see that the process under the lift
operator may be shaped by prefixed inputs binding a name inside it. In
this sense, the lift operator will be seen as a way to dynamically
construct processes before reifying them as names.

Finally equipped with these standard features we can present the
dynamics of the calculus.

\subsubsection{Operational semantics} 

Finally, we introduce the computational dynamics. What marks these
algebras as distinct from other more traditionally studied algebraic
structures, e.g. vector spaces or polynomial rings, is the manner in
which dynamics is captured. In traditional structures, dynamics is typically
expressed through morphisms between such structures, as in linear maps
between vector spaces or morphisms between rings. In algebras
associated with the semantics of computation, the dynamics is
expressed as part of the algebraic structure itself, through a
reduction reduction relation typically denoted by $\red$. Below, we
give a recursive presentation of this relation for the calculus used
in the encoding.

$\red \subseteq \pi \times \pi$
$\red : \pi \to \mathcal{P}(\pi)$

\begin{mathpar}
  \inferrule* [lab=Comm] { \textsf{match}( x_{src}, x_{trgt} ) } { x_{trgt}?(y)P \; | \; x_{src}!\langle {Q} \rangle \red P\{\quotep{Q}/y}\} }
  \and \\
  \inferrule* [lab=Par] {{P} \red {P}'} {{{P} | {Q}} \red {{P}' | {Q}}}
  \and
  \inferrule* [lab=Equiv]{{{P} \scong {P}'} \andalso {{P}' \red {Q}'} \andalso {{Q}' \scong {Q}}}{{P} \red {Q}}
\end{mathpar}

\begin{eqnarray*}
  match_{\equiv} (\quotep{P},\quotep{Q}) & := & P \equiv Q \\
  match_{\dagger}(\quotep{P},\quotep{Q}) & := & \forall R. P|Q \red^{*} R => R \red^{*} 0 \\
  match_{K}(\quotep{P},\quotep{Q}) & := & K \mbox{ for some context } K
\end{eqnarray*}

$u?(x)P | u!\langle Q \rangle \red P\{\quotep{Q}/x\}$

%We write $\wred$ for $\red^*$, and $P\red$ if $\exists Q $ such that $ P \red Q$.
We write $P\red$ if $\exists Q $ such that $ P \red Q$ and $P\not\red$, otherwise.

\section{Replication}

As mentioned before, it is known that replication (and hence
recursion) can be implemented in a higher-order process algebra
\cite{SangiorgiWalker}. As our first example of calculation with the
machinery thus far presented we give the construction explicitly in
the {\rhoc}.

\begin{eqnarray}
	D_{x} & := & \prefix{x}{y}{(\binpar{\outputp{x}{y}}{@{y}})} \nonumber\\
	\bangp_{x}{P} & := & \binpar{{x}!\langle{\binpar{D_{x}}{P}}\rangle}{D_{x}} \nonumber
\end{eqnarray}

\begin{eqnarray}
	\bangp_{x}{P} & & \nonumber\\
	=
	& {x}!\langle{(\prefix{x}{y}{(\outputp{x}{y} | @{y})) | P}}\rangle 
	      | \prefix{x}{y}{(\outputp{x}{y} | @{y})} & \nonumber\\
	\red
	& (\outputp{x}{y} | @{y})\substn{\quotep{(\prefix{x}{y}{(@{y} | \outputp{x}{y})) | P}}}{y} & \nonumber\\
	=
	& \outputp{x}{\quotep{(\prefix{x}{y}{(\outputp{x}{y} | @{y})) | P}}}
	  | {(\prefix{x}{y}{(\outputp{x}{y} | @{y})) | P}} & \nonumber\\
	\red
	& \ldots & \nonumber\\
	\red^*
	& P | P | \ldots & \nonumber
\end{eqnarray}

Of course, this encoding, as an implementation, runs away, unfolding
$\bangp{P}$ eagerly. A lazier and more implementable replication
operator, restricted to input-guarded processes, may be obtained as follows.

\begin{eqnarray}
\bangp{\prefix{u}{v}{P}} 
	:= 
	\binpar{\lift{x}{\prefix{u}{v}{(\binpar{D(x)}{P})}}}{D(x)} \nonumber
\end{eqnarray}

\begin{remark}
  Note that the lazier definition still does not deal with summation
  or mixed summation (i.e. sums over input and output). The reader is
  invited to construct definitions of replication that deal with these
  features. 

  Further, the definitions are parameterized in a name, $x$. Can you,
  gentle reader, make a definition that eliminates this parameter and
  guarantees no accidental interaction between the replication
  machinery and the process being replicated -- i.e. no accidental
  sharing of names used by the process to get its work done and the
  name(s) used by the replication to effect copying. This latter
  revision of the definition of replication is crucial to obtaining
  the expected identity $!!P \sim !P$.
\end{remark}

\begin{remark}\label{rem:paradoxical_combinator}
  The reader familiar with the lambda calculus will have noticed the
  similarity between $D$ and the paradoxical combinator.

  [Ed. note: the existence of this seems to suggest we have to be more
  restrictive on the set of processes and names we admit if we are to
  support no-cloning.]
\end{remark}

\subsubsection{Bisimulation}

The computational dynamics gives rise to another kind of equivalence,
the equivalence of computational behavior. As previously mentioned
this is typically captured \emph{via} some form of bisimulation.

% The notion we use in this paper is weak barbed bisimulation
% \cite{milner91polyadicpi}.

The notion we use in this paper is derived from weak barbed
bisimulation \cite{milner91polyadicpi}. 

\begin{definition}
An \emph{observation relation}, $\downarrow_{\mathcal N}$, over a set
of names, $\mathcal N$, is the smallest relation satisfying the rules
below.

\infrule[Out-barb]{y \in {\mathcal N}, \; x \nameeq y}
		  {\outputp{x}{v} \downarrow_{\mathcal N} x}
\infrule[Par-barb]{\mbox{$P\downarrow_{\mathcal N} x$ or $Q\downarrow_{\mathcal N} x$}}
		  {\binpar{P}{Q} \downarrow_{\mathcal N} x}

We write $P \Downarrow_{\mathcal N} x$ if there is $Q$ such that 
$P \wred Q$ and $Q \downarrow_{\mathcal N} x$.
\end{definition}

\begin{definition}
%\label{def.bbisim}
An  ${\mathcal N}$-\emph{barbed bisimulation} over a set of names, ${\mathcal N}$, is a symmetric binary relation 
${\mathcal S}_{\mathcal N}$ between agents such that $P\rel{S}_{\mathcal N}Q$ implies:
\begin{enumerate}
\item If $P \red P'$ then $Q \wred Q'$ and $P'\rel{S}_{\mathcal N} Q'$.
\item If $P\downarrow_{\mathcal N} x$, then $Q\Downarrow_{\mathcal N} x$.
\end{enumerate}
$P$ is ${\mathcal N}$-barbed bisimilar to $Q$, written
$P \wbbisim_{\mathcal N} Q$, if $P \rel{S}_{\mathcal N} Q$ for some ${\mathcal N}$-barbed bisimulation ${\mathcal S}_{\mathcal N}$.
\end{definition}

$\mathcal{R} \subseteq \pi \times \pi$

$P \mathcal{R} Q => \forall P'. P \red P' \Rightarrow \exists Q'. Q \red Q', P' \mathcal{R} Q'$

$P \vdash x \Rightarrow Q \vdash x$

\begin{mathpar}
  \inferrule*[lab=Out-barb]{x \nameeq y}{{y}!\langle{Q}\rangle \vdash x}
  \and
  \inferrule*[lab=Par-barb]{\mbox{$P\vdash x$ or $Q\vdash x$}}{\binpar{P}{Q} \vdash x}
\end{mathpar}

\subsubsection{Contexts}

One of the principle advantages of computational calculi like the
$\pi$-calculus is a well-defined notion of context,
contextual-equivalence and a correlation between
contextual-equivalence and notions of bisimulation. The notion of
context allows the decomposition of a process into (sub-)process and
its syntactic environment, its context. Thus, a context may be
thought of as a process with a ``hole'' (written $\Box$) in it. The
application of a context $M$ to a process $P$, written $M[P]$, is
tantamount to filling the hole in $M$ with $P$. In this paper we do
not need the full weight of this theory, but do make use of the notion
of context in the proof the main theorem. 

\begin{mathpar}
  \inferrule* [lab=summation] {} {{M_{M},M_{N}} \bc \Box \;|\; x.M_{A} \;|\; M_{M}+M_{N}}
  \and
  \inferrule* [lab=agent] {} {{M_{A}} \bc (\vec{x})M_{P} \;| \; \clift{P_0,\ldots,M_{P},\ldots,P_N}}
  \and \\
  \inferrule* [lab=process] {} {{M_{P}} \bc M_{N} \;| \;P|M_{P} }
\end{mathpar} 

\begin{mathpar}
  \inferrule* [lab=sychronization] {} {M_{N} \bc \Box \;|\; x?M_{F} \;|\; x!M_{C}}
  \and
  \inferrule* [lab=abstraction] {} {{M_{F}} \bc (x)M_{P} }
  \and
  \inferrule* [lab=concretion] {} {{M_{C}} \bc \langle M_{P} \rangle }
  \and \\
  \inferrule* [lab=process] {} {{M_{P}} \bc M_{N} \;| \;P|M_{P} }
\end{mathpar}

\begin{definition}[contextual application] Given a context $M$, and
  process $P$, we define the \emph{contextual application}, $M[P] :=
  M\{P/\Box\}$. That is, the contextual application of M to P is the
  substitution of $P$ for $\Box$ in $M$.
\end{definition}

$\meaningof{-} : L \to \mathcal{P}(\pi)$

\begin{mathpar}
  \inferrule* [lab=collection] {} {\meaningof{true} = \pi, \and \meaningof{~E} = \pi \setminus \meaningof{E}, \and \meaningof{E_{1} \& E_{2}} = \meaningof{E_{1}} \cap \meaningof{E_{2}}}
\end{mathpar}

\begin{mathpar}
  \inferrule* [lab=structure] {} {\meaningof{0} = \{ P \in \pi | P \equiv 0 \}, \and \\ \meaningof{E_1 | E_2} = \{ P \in \pi | P \equiv P_{1} | P_{2}, P_{1} \in \meaningof{E_{1}}, P_{2} \in \meaningof{E_2}\} }
\end{mathpar}

\begin{mathpar}
 \inferrule* [lab=behavior] {} {\meaningof{\langle a?b \rangle E} = \{ P \in \pi | P \equiv Q | u?(y)P', \\ \and \\\\ \and \\ \;\;\; u \in \meaningof{a}, \forall z.P'\{z/y\} \in \meaningof{E\{z/b\}}\}, \and \\ \meaningof{a!E} = \{ P \in \pi | P \equiv Q | x!\langle P' \rangle, x \in \meaningof{a} P' \in \meaningof{E}\} }
\end{mathpar}

\begin{mathpar}
 \inferrule* [lab=nominal] {} {\meaningof{\quotep{E}} = \{ \quotep{P} \in \quotep{\pi} | P \in \meaningof{E} \}, \and \meaningof{\quotep{P}} = \{ \quotep{Q} \in \quotep{\pi} | P \equiv Q \} \and \\ \meaningof{@\quotep{E}} = \{ P \in \pi | P \equiv @x, x \in \meaningof{E} \}}
\end{mathpar}

\begin{eqnarray*}
  \\
  \meaningof{-} : TS \to ST
\end{eqnarray*}

\begin{eqnarray*}
  \\
  L : TS \to ST
\end{eqnarray*}

\begin{eqnarray*}
  \\
  P \models E \iff P \in \meaningof{E}
\end{eqnarray*}

\begin{eqnarray*}
  P \approx_{L} Q \iff \forall E \in L. P \models E \iff Q \models E
\end{eqnarray*}

\begin{eqnarray*}
  P \approx_{K} Q
\end{eqnarray*}

\begin{eqnarray*}
  P \approx Q
\end{eqnarray*}

$\approx_{K} = \approx = \approx_{L}$

\subsubsection{Contextual duality}

Note that contexts extend the quotation operation to a family of
operations from processes to names. Given a context, $M$, we can
define a \emph{nominal context}, $\quotep{M}$ by $\quotep{M}[P] :=
\quotep{M[P]}$. To foreshadow what is to come we observe that these
operations enjoy a duality with processes very much like the duality
between vectors and maps from vectors to scalars.

Further, because the calculus is essentially higher-order, we have a
correspondence between contexts and processes. More specifically,
given a name $x$ and a context $M$ we can construct $M^{*}_{x}$ such
that 

\begin{mathpar}
  M^{*}_{x} | \lift{x}{P} \red M[P]
\end{mathpar}

namely,

\begin{mathpar}
  M^{*}_{x} := x?(u).M[\dropn{u}]
\end{mathpar}

The dependence of $M^{*}_{x}$ on a name makes it an abstraction, 

\begin{mathpar}
  M^{*} := (x)x?(u).M[\dropn{u}]
\end{mathpar}

\subsection{Additional notation}

It will sometimes be convenient to denote the process a name
quotes. We already have the notation $x = \quotep{P}$, but it will be
convenient to introduce an alternate notation, $\procn{x}$, when we
want to emphasize the connection to the use of the name. Note that, by
virtue of name equivalence, $\quotep{\procn{x}} \nameeq x$; so, the
notation is consistent with previous definitions.

Further, because names have structure it is possible to effect
substitutions on the basis of that structure. This means we need to
upgrade our notation for substitutions, which we accomplish by
adapting comprehension notation. Thus,

\begin{mathpar}
  P\{ y / x : x \in S \}
\end{mathpar}

is interpreted to mean the process derived from P by replacing (in a
capture-avoiding manner) each occurrence of $x$ in $S$ by $y$. For example,

\begin{mathpar}
  P\{ \quotep{\procn{x}|\procn{x}} / x : x \in \freenames{P} \}
\end{mathpar}

will replace each (occurrence) of a free name $x$ in $P$ by
$\quotep{\procn{x}|\procn{x}}$.

Also, we will avail ourselves of the notation $x^{L}$ and $x^{R}$ to
denote injections of a name into disjoint copies of the name
space. There are numerous ways to accomplish this. One example can be
found in \cite{MeredithR05}. This notation overloads to vectors of
names: $\vec{x}^{\pi} := (x_{i}^{\pi} \; : \; 0 \leq i < |\vec{x}| )$ where $\pi \in \{L,R\}$.

We also use $P^{\Box} := P|\Box$.

In \cite{MeredithR05} an interpretation of the new operator is
given. It turns out that there are several possible interpretations
all enjoying the requisite algebraic properties of the operator (see
\cite{milner91polyadicpi}). We will therefore make liberal use of
$(\nu\; \vec{x})P$.

% subsection the_syntax_and_semantics_of_the_notation_system (end)   

\input{qm2pi.qmops} 

\input{qm2pi.sterngerlach} 

\input{qm2pi.metric} 

% section concurrent_process_calculi (end)

%\input{qm2pi.proofsketch}

% section proof sketch (end)

%\input{qm2pi.slviaknots} 

% section spatial logic via knots (end)

\input{qm2pi.conclusion}

% section conclusion (end)

%\input{qm2pi.dtcodes} 

% section wiring algorithm (end)

\input{qm2pi.ack} 

% section acknowledgments (end)

\newpage


\bibliographystyle{plain}   
\bibliography{../../biblios/main.bib}

\input{qm2pi.rhodetails}

\end{document}



% section proof sketch (end)

%\section{Unlikely characters: spatial logic for
  knots}\label{sub:characteristic_formulae} % (fold)

Associated to the mobile process calculi are a family of logics known
as the Hennessy-Milner logics. These logics typically enjoy a
semantics interpreting formulae as sets of processes that when
factored through the encoding outlined above allows an identification
of classes of knots with logical formulae. In the context of this
encoding the sub-family known as the spatial logics \cite{CairesC03}
\cite{CairesC04} \cite{Caires04} are of particular interest providing
several important features for expressing and reasoning about
properties (i.e. classes) of knots. We hint here at how this may be done.

%\begin{description}
%\item [structural connectives] 
\subsubsection{Structural connectives} The spatial logics enjoy
structural connectives corresponding, at the logical level, to the
parallel composition ($P | Q$) and new name ($(\nu \; x)P$)
connectives for processes. As illustrated in the examples below, these
connectives are extremely expressive given the shape of our encoding.
%\item [decideable satisfaction]

\subsubsection{Decideable satisfaction}
In \cite{Caires04} the satisfaction relation is shown to be decideable
for a rich class of processes. It further turns out that the image of
the our encoding is a proper subset of that class. This result
provides the basis for an algorithm by which to search for knots
enjoying a given property.
%\item [characteristic formulae]

\subsubsection{Characteristic formulae}
In the same paper \cite{Caires04} , Caires presents a means of calculating
characteristic formulae, selecting equivalence classes of processes
up to a pre--specified depth limit on the support set of names. Composed with our
encoding, this characteristic formula can be used to select
characteristic formulae for knots.
%\end{description}

\subsubsection{Spatial logic formulae}

The grammar below (segmented for comprehension) summarizes the syntax
of spatial logic formulae. We employ illustrative examples in the
sequel to provide an intuitive understanding of their meaning
referring the reader to \cite{Caires04} for a more detailed explication
of the semantics.

\begin{mathpar}
  \inferrule* [lab=boolean] {} {{A,B} \bc T \;|\; \neg A \;|\; A \wedge B \;|\; \eta = \eta'}
  \and
  \inferrule* [lab=spatial] {} {|\; \pzero \;|\; A | B \;|\; x \text{\textregistered} A \;|\; \forall x . A \;|\;  H x . A}
  \and
  \inferrule* [lab=behavioral] {} {|\; \alpha . A}
  \and 
  \inferrule* [lab=recursion] {} {|\; X(\vec{u}) \;|\; \mu X(\vec{u}) . A}
  \and
  \inferrule* [lab=action] {} {\alpha \bc \langle x?(\vec{y}) \rangle \;|\; \langle x!(\vec{y}) \rangle \;|\; \langle \tau \rangle}
  \and 
  \inferrule* [lab=name] {} {\eta \bc x \;|\; \tau}
\end{mathpar} 

% subsection characteristic_formulae (end)   	 

\subsection{Example formulae}\label{sub:example_formulae_} % (fold)

\subsubsection{Crossing as formula.}
% 
% \begin{align*}
%   \frac{d}{dx} \sin x &= \cos x 
%   & \frac{d}{dx} e^x &= e^x \\
%   \frac{d}{dx} \cos x &= - \sin x 
%   & \frac{d}{dx} \log x &= \frac{1}{x} \\
% \end{align*} 

\begin{align*}
 \mu C(x_{0},x_{1},y_{0},y_{1},u).&(\langle x_{0}?(z) \rangle(\langle u! \rangle\langle y_{1}!z \rangle C(x_{0},x_{1},y_{0},y_{1},u)) & \\
  & \wedge \langle y_{1}?(z) \rangle (\langle u! \rangle \langle x_{0}!z \rangle C(x_{0},x_{1},y_{0},y_{1},u)) & \\
  & \wedge \langle x_{1}?(z) \rangle (\langle u? \rangle \langle y_{0}!z \rangle C(x_{0},x_{1},y_{0},y_{1},u)) & \\
  & \wedge \langle y_{0}?(z) \rangle (\langle u? \rangle \langle x_{1}!z \rangle C(x_{0},x_{1},y_{0},y_{1},u))) &
\end{align*}

The lexicographical similarity between the shape of this formulae and
the shape of definition of the process representing a crossing reveals
the intuitive meaning of this formulae. It describes the capabilities
of a process that has the right to represent a crossing. For example
it picks out processes that may perform an input on the port $x_0$ in
its initial menu of capabilities. What differentiates the formula
from the process, however, is that the crossing process is the
smallest candidate to satisfy the formula. Infinitely many other
processes -- with internal behavior hidden behind this interface, so
to speak -- also satisfy this formula. Even this simple formula,
then, can be seen to open a new view onto knots, providing a
computational interpretation of \emph{virtual} knots.

Note that this formula is derived by hand. A similar formula can be
derived by employing Caires' calculation of characteristic formula
\cite{Caires04} to the process representing a crossing. In light of
this discussion, we let
$\meaningof{C}_{\phi}(x0,x1,y0,y1,u)$ denote a formula specifying the
dynamics we wish to capture of a crossing. To guarantee we preserve
the shape of the interface and minimal semantics we demand that
$\meaningof{C}_{\phi}(x0,x1,y0,y1,u) \Rightarrow
\textbf{C}(x0,x1,y0,y1,u)$ where $\textbf{C}(x0,x1,y0,y1,u)$ denotes
the formula above.
                            
\subsubsection{Crossing number constraints.}
The moral content of the context lemma (Lemma \ref{context}) is that the notion of
``locality'' in the Reidemeister moves is effectively captured by the
parallel composition operator of the process calculus. This intuition
extends through the logic. Given a formula,
$\meaningof{C}_{\phi}(x0,x1,y0,y1,u)$, we can use the structural
connectives to specify constraints on crossing numbers, such as at
least $n$ crossings, or exactly $n$ crossings.
\begin{mathpar}
  \inferrule* [lab=at-least-n] {} { K^{\geq n}_{\phi}(\vec{xs},\vec{ys}) := \Pi_{i=0}^{n-1} Hu . \meaningof{C}_{\phi}(xs_i,ys_i,u) | T }
  \and 
  \inferrule* [lab=exactly-n] {} { K^{= n}_{\phi}(\vec{xs},\vec{ys}) := \Pi_{i=0}^{n-1} Hu . \meaningof{C}_{\phi}(xs_i,ys_i,u) | \neg (\forall x_0,y_0,x_1,y_1,u . \meaningof{C}_{\phi}(x_0,y_0,x_1,y_1,u) | T) }
\end{mathpar}

To round out this section, recall that the encoding of an $n$-crossing
knot decomposes into a parallel composition of $n$ \emph{copies} of a
crossing process together with a wiring harness. To specify different
knot classes with the same crossing number amounts to specifying
logical constraints on the wiring harness. In the interest of space,
we defer examples to a forthcoming paper. Suffice it to say that both
the conditions ``alternating knot'' and ``contains the tangle
corresponding to 5/3'' are expressible. For example, it is possible to
calculate the characteristic formula of a process corresponding to the
tangle 5/3 and conjoin it into the classifying formula via the
composition connective of the logic.

Finally, we wish to observe that it is entirely within reason to
contemplate a more domain-specific version of spatial logic tailored
to the shape of processes in the image of the encoding. Such a
domain-specific logic would have a better claim to the title formal
language of knot properties.

% subsection example_formulae_ (end)

% section knots_as_processes (end) 

% section spatial logic via knots (end)

\section{Conclusions and future work}

\paragraph{Testing physical space}
You, gentle reader, may wonder why of all the theorems to be proved
given this set up we pick the one above. In some sense it's hardly
central to quantum mechanics. We see it as central in the sense that
it firmly establishes a notion of physical space arising from a notion
of the equivalence of behavior. Relating bisimulation to a metric is a
big step forward, but one is faced with interpreting the relationship
of that metric space to something more physical. Quantum mechanical
notions of ``physical'' space are still far from intuitive, but by
relating this idea of distance as testing to calculations that predict
physical circumstances we are making a not insignificant step forward
toward an understanding of the physical space we inhabit as
essentially dynamic.

\paragraph{Effectivity and simulation}
One of the observations we have yet to make is that the entire program
spelled out here is effective. We have built various interpreters for
the reflective calculus at work in this interpretation. In principle,
then, we can simulate quantum mechanics on a computer. The place where
the simulation may lose fidelity is the infinitely branching summation
for the annihilator.

In this connection i also want to point out that the evaluation style
calculation of the inner product puts the non-determinism of the
summation right at the heart of measurement. This suggests that
Milner's original reduction-based formulation of the dynamics of his
calculi in terms of sums was not just notationally suggestive of a
notion of measure-and-continue but captured some significant part of
the physics.

\paragraph{Quantum continuations}
In light of this last observation i want to point out that the
predominant account of quantum mechanics is missing a key aspect of a
truly compositional story of the physical situation. In a real lab,
when a measurement is made the observation can be made to feed into
another device that then makes another measurement conditioned on the
results of the first. This means that after the superposition was
collapsed the entire experimental set up remained in
superposition. While QM offers a means of writing this down it doesn't
quite line up well with the well-trodden formulation of computation
and continuation that we see so succinctly expressed in Milner's
calculi. This suggests that there might be advantages to this account
of dynamics waiting to be explored.

\paragraph{Quantum logic}
In this connection, we also note that by virtue of having the
Hennessy-Milner construction, we can pull the construction through the
interpretation of QM. This gives us a natural candidate for a quantum
logic that enjoys an extremely tight connection with it's domain of
interpretation, making the construction much less ad hoc (rather it is
the image of functor!).

\paragraph{Quantum probabiity}
i have questions about the basis of the interpretation of inner
product as probability amplitude. In particular, using which
axiomatization of probability theory does the notion of probability
amplitude earn the right to be so dubbed? In other words, where is the
proof that the operation for calculating a probability amplitude (and
then squaring) satisfies the axioms of what it means to calculate a
probability? Even if such a proof exists (i have yet to find it in the
literature), i wonder if it might not be possible to turn things on
their heads. Can we view the calculation of the probability amplitude
as an axiomatization of probability? If so, then the definition we
give for calculating probability amplitude may provide the basis for
an \emph{effective} theory of probability.

\paragraph{Quantum vs ``biological'' information}
Finally, i want to conclude with a more philosophical observation. At
a recent workshop in which QM was a predominant topic i noticed
something about quantum information. The speaker was giving a riveting
discussion of axiomatic QM and showing how properties of ``no
cloning'' and ``no deleting'' emerged as consequences of the
axiomatization. Theorems of this form are necessary to give us a sense
of confidence that our axioms characterize the physical theory. What
struck me, though, was that if quantum information is neither erasable
nor replicable it is markedly different from \emph{life}. Two of the
things we know about life is that

\begin{itemize}
  \item it ends;
  \item to gain some measure of persistence, to transcend it's
    finitude it is imminently copyable.
\end{itemize}

Both of these qualities are summarized succinctly in the aphorism: all
flesh is grass. For me these two kinds of ``information'' -- call them
quantum and biological -- are end points on a spectrum of strategies
for persistence. At one end, we have those curious entities that enjoy
uniqueness and permanence; at the other, we have those who in the face
of a certain end and an uncertain present make a go of passing
something on. To me one of the more remarkable aspects of the latter
strategy is that in the presence of noise (and certain features of
copying) we get a kind of dynamism, a chance for improvement against a
given persistent condition.

% subsection other_calculi_other_bisimulations_and_geometry_as_behavior (end)




% section conclusion (end)

%\documentclass[12pt]{llncs}
%\documentclass{jktr}

\usepackage[pdftex]{hyperref}                   
\usepackage {listings}
\usepackage {mathpartir}
\usepackage{bcprules}
%\usepackage{listings}
                       
\usepackage{graphicx} 
%\usepackage[margins=2.5cm,nohead,nofoot]{geometry}
%\usepackage{geometry}
\usepackage{amsfonts}
\usepackage{amstext}
\usepackage{latexsym}
\usepackage{amssymb}
\usepackage{color}


%\include{myPreamble}
\include{qm2pi.local} 

%\ifpdf
%\usepackage[pdftex]{graphicx}
%\else
%\usepackage{graphicx}
%\fi

 % \ifpdf
%  \usepackage{pdfsync}
%  \if


%\title{Brief Article}
%\author{David F. Snyder}
%\author{L.G. Meredith}

%\address{Dept. of Math., Texas State University--San Marcos, San Marcos, TX 78666}
       
\pagestyle{empty}


\begin{document}

\lstset{language=[Objective]Caml,frame=shadowbox}

\input{qm2pi.front}

% section front matter (end)

\input{qm2pi.intro} 
 
% section introduction (end)

% \input{qm2pi.knotations} 

% section notation (end)

\input{qm2pi.process.calculi} 

% section concurrent_process_calculi_and_spatial_logics_ (end)
    
%\input{qm2pi.knots2pi} 

%\input{qm2pi.trefoil} 

%\input{qm2pi.mainthm} 

% subsection basic_interpretation (end)

%\input{qm2pi.rho.presentation} 
\subsection{The syntax and semantics of the notation system}\label{sub:the_syntax_and_semantics_of_the_notation_system} % (fold)

We now summarize a technical presentation of the calculus that
embodies our theory of dynamics. The typical presentation of such a
calculus follows the style of giving generators and relations on
them. The grammar, below, describing term constructors, freely
generates the set of processes, $\Proc$. This set is then quotiented
by a relation known as structural congruence and it is over this set
that the notion of dynamics is expressed. This presentation is
essentially that of \cite{MeredithR05} with the addition of
polyadicity and summation. For readability we have relegated some of
the technical subtleties to an appendix.

\subsubsection{Process grammar}\label{subsub:process_grammar}

\begin{mathpar}
  \inferrule* [lab=synchronization] {} {{M} \bc \pzero \;|\; x?F \;|\; x!C }
  \and
  \inferrule* [lab=abstraction] {} {{F} \bc (x)P}
  \and
  \inferrule* [lab=concretion] {} {{C} \bc \langle Q \rangle}
  \and
  \inferrule* [lab=process] {} {{P,Q} \bc M \;| \;P|Q \;|\; @{x}}
  \and
  \inferrule* [lab=name] {} {{x} \bc \quotep{P}}
\end{mathpar} 

Note that $\vec{x}$ (resp. $\vec{P}$) denotes a vector of names
(resp. processes) of length $|\vec{x}|$ (resp. $|\vec{P}|$). We adopt
the following useful abbreviations.

\begin{mathpar}
   x?(\vec{y}).P := x.(\vec{y})P \and  x\clift{\vec{P}} := x.\clift{\vec{P}}
   \and x!(y) := \lift{x}{\dropn{y}}
   \and \Pi_{i=0}^{n-1}P_i := P_0 | \ldots | P_{n-1}
\end{mathpar}

\subsubsection{Structural congruence}

\paragraph{Free and bound names and alpha-equivalence.} At the
core of structural equivalence is alpha-equivalence which identifies
process that are the same up to a change of variable. Formally, we
recognize the distinction between free and bound names. The free names
of a process, $\freenames{P}$, may be calculated recursively as
follows:

\begin{mathpar}
\freenames{\pzero} := \emptyset
  \and \\
  \freenames{x?(y).P} := \{ x \} \cup (\freenames{P} \setminus \{ y \})
  \and 
  \freenames{x!\langle P \rangle} := \{ x \} \cup \{ P \} 
  \and \\
  \freenames{P|Q} := \freenames{P} \cup \freenames{Q}
  \and \\
  \freenames{@{x}} := \{ x \}
\end{mathpar}

$\pi$
$\quotep{\pi}$

$\freenames{-} : \pi \to \mathcal{P}(\quotep{\pi})$

\begin{eqnarray*}
  \freenames{\pzero} & := & \emptyset \\
  \freenames{x?(y).P} & := & \{ x \} \cup (\freenames{P} \setminus \{ y \}) \\
  \freenames{x!\langle P \rangle} & := & \{ x \} \cup \{ P \} \\
  \freenames{P|Q} & := & \freenames{P} \cup \freenames{Q} \\
  \freenames{\dropn{x}} & := & \{ x \}
\end{eqnarray*}

The bound names of a process, $\boundnames{P}$, are those names occurring in $P$
that are not free. For example, in $x?(y).0$, the name $x$ is free, while $y$ is bound.

\begin{mathpar}
  \inferrule* [lab=monoidal-laws] {} { P|Q \equiv Q|P \and P|0 \equiv P \and P|(Q|R) \equiv (P|Q)|R }
\end{mathpar}

\begin{mathpar}
  \inferrule* [lab=alpha-equivalence] {} { (x)P \equiv (y)P\{y/x\} \and y \not\in \freenames{P} }
\end{mathpar}

\begin{definition}
Then two processes, $P,Q$, are alpha-equivalent if $P = Q\{\vec{y}/\vec{x}\}$ for
some $\vec{x} \in \boundnames{Q},\vec{y} \in \boundnames{P}$, where $Q\{\vec{y}/\vec{x}\}$
denotes the capture-avoiding substitution of $\vec{y}$ for $\vec{x}$ in $Q$.
\end{definition}

\begin{definition}
  The {\em structural congruence} \cite{SangiorgiWalker} , $\equiv$,
  between processes is the least congruence containing
  alpha-equivalence, satisfying the abelian monoid laws
  (associativity, commutativity and $\pzero$ as identity) for parallel
  composition $|$ and for summation $+$.
\end{definition}

\subsection{Name equivalence}

We take name equivalence, written $\nameeq$, to be the smallest
equivalence relation generated by the following rules.

\begin{mathpar}
\inferrule*[lab=Quote-drop]
{ }
{ \quotep{@{x}} \nameeq x }

\inferrule*[lab=Struct-equiv]
{ P \scong Q }
{ \quotep{P} \nameeq \quotep{Q} }
\end{mathpar}

The astute reader will have noticed that the mutual recursion of names
and processes imposes a mutual recursion on alpha-equivalence and
structural equivalence via name-equivalence. Fortunately, all of this
works out pleasantly and we may calculate in the natural way, free of
concern. The reader interested in the details is referred to the
appendix \ref{appendix:rho_details}.

\subsection{Substitution}

We use $\Proc$ for the set of processes, $\QProc$ for the set of
names, and $\id{\{}\vec{y} / \vec{x} \id{\}}$ to denote partial maps,
$s : \QProc \rightarrow \QProc$. A map, $s$ lifts, uniquely, to a map
on process terms, $\widehat{s} : \Proc \rightarrow \Proc$ by the
following equations.

\begin{mathpar}
  (0) \psubstp{Q}{P} := 0 \\
  (R \juxtap S) \psubstp{Q}{P}
  :=    
  (R)\psubstp{Q}{P} \juxtap (S) \psubstp{Q}{P} \\
  (x?(y).R) \psubstp{Q}{P}    
  :=    
  (x)\substp{Q}{P} (z)\concat( (R \psubstn{z}{y}) \psubstp{Q}{P} ) \\
  (\lift{x}{R}) \psubstp{Q}{P}  
  :=
  \lift{(x)\substp{Q}{P}}{ R \psubstp{Q}{P} } \\
%   (\dropn{x})  \psubstp{Q}{P}       
%   := 
%   \left\{ 
%     \begin{array}{ccc} 
%       \dropn{\quotep{Q}} & & x \nameeq \quotep{P} \\
%       \dropn{x} & & otherwise \\
%     \end{array}
%   \right. 
  (\dropn{x})  \psubstp{Q}{P}       
  := 
  \left\{ 
    \begin{array}{ccc} 
      Q & & x \nameeq \quotep{P} \\
      \dropn{x} & & otherwise \\
    \end{array}
  \right.
\end{mathpar}
 

where

\begin{eqnarray}
  (x)\id{\{} \lpquote Q \rpquote / \lpquote P \rpquote \id{\}}            = 
  \left\{ 
    \begin{array}{ccc}
      \lpquote Q \rpquote & & x \nameeq \lpquote P \rpquote \\
      x & & otherwise \\
    \end{array}
  \right. \nonumber
\end{eqnarray}

and $z$ is chosen distinct from $\quotep{P}$, $\quotep{Q}$, the free
names in $Q$, and all the names in $R$. Our $\alpha$-equivalence will
be built in the standard way from this substitution.

\begin{remark}\label{rem:no_self_referential_names}
  One consequence of these definitions is that $\forall P. \quotep{P}
  \not\in \freenames{P}$.
\end{remark}

\subsection{ Dynamic quote: an example }

Anticipating something of what's to come, consider applying the
substitution, $\widehat{\id{\{}u / z \id{\}}}$, to the following pair
of processes, $\lift{w}{y!(z)}$ and $w[ \lpquote y!(z) \rpquote ]$.

\begin{eqnarray}
	\lift{w}{y!(z)}\widehat{\id{\{}u / z \id{\}}}
		& = &
		\lift{w}{y!(u)} \nonumber\\
	w[ \lpquote y!(z) \rpquote ] \widehat{ \id{\{}u / z \id{\}} }
		& = &
		w[ \lpquote y!(z) \rpquote ] \nonumber
\end{eqnarray}

Because the body of the process between quotes is impervious to
substitution, we get radically different answers. In fact, by
examining the first process in an input context,
e.g. $x?(z).\lift{w}{y!(z)}$, we see that the process under the lift
operator may be shaped by prefixed inputs binding a name inside it. In
this sense, the lift operator will be seen as a way to dynamically
construct processes before reifying them as names.

Finally equipped with these standard features we can present the
dynamics of the calculus.

\subsubsection{Operational semantics} 

Finally, we introduce the computational dynamics. What marks these
algebras as distinct from other more traditionally studied algebraic
structures, e.g. vector spaces or polynomial rings, is the manner in
which dynamics is captured. In traditional structures, dynamics is typically
expressed through morphisms between such structures, as in linear maps
between vector spaces or morphisms between rings. In algebras
associated with the semantics of computation, the dynamics is
expressed as part of the algebraic structure itself, through a
reduction reduction relation typically denoted by $\red$. Below, we
give a recursive presentation of this relation for the calculus used
in the encoding.

$\red \subseteq \pi \times \pi$
$\red : \pi \to \mathcal{P}(\pi)$

\begin{mathpar}
  \inferrule* [lab=Comm] { \textsf{match}( x_{src}, x_{trgt} ) } { x_{trgt}?(y)P \; | \; x_{src}!\langle {Q} \rangle \red P\{\quotep{Q}/y}\} }
  \and \\
  \inferrule* [lab=Par] {{P} \red {P}'} {{{P} | {Q}} \red {{P}' | {Q}}}
  \and
  \inferrule* [lab=Equiv]{{{P} \scong {P}'} \andalso {{P}' \red {Q}'} \andalso {{Q}' \scong {Q}}}{{P} \red {Q}}
\end{mathpar}

\begin{eqnarray*}
  match_{\equiv} (\quotep{P},\quotep{Q}) & := & P \equiv Q \\
  match_{\dagger}(\quotep{P},\quotep{Q}) & := & \forall R. P|Q \red^{*} R => R \red^{*} 0 \\
  match_{K}(\quotep{P},\quotep{Q}) & := & K \mbox{ for some context } K
\end{eqnarray*}

$u?(x)P | u!\langle Q \rangle \red P\{\quotep{Q}/x\}$

%We write $\wred$ for $\red^*$, and $P\red$ if $\exists Q $ such that $ P \red Q$.
We write $P\red$ if $\exists Q $ such that $ P \red Q$ and $P\not\red$, otherwise.

\section{Replication}

As mentioned before, it is known that replication (and hence
recursion) can be implemented in a higher-order process algebra
\cite{SangiorgiWalker}. As our first example of calculation with the
machinery thus far presented we give the construction explicitly in
the {\rhoc}.

\begin{eqnarray}
	D_{x} & := & \prefix{x}{y}{(\binpar{\outputp{x}{y}}{@{y}})} \nonumber\\
	\bangp_{x}{P} & := & \binpar{{x}!\langle{\binpar{D_{x}}{P}}\rangle}{D_{x}} \nonumber
\end{eqnarray}

\begin{eqnarray}
	\bangp_{x}{P} & & \nonumber\\
	=
	& {x}!\langle{(\prefix{x}{y}{(\outputp{x}{y} | @{y})) | P}}\rangle 
	      | \prefix{x}{y}{(\outputp{x}{y} | @{y})} & \nonumber\\
	\red
	& (\outputp{x}{y} | @{y})\substn{\quotep{(\prefix{x}{y}{(@{y} | \outputp{x}{y})) | P}}}{y} & \nonumber\\
	=
	& \outputp{x}{\quotep{(\prefix{x}{y}{(\outputp{x}{y} | @{y})) | P}}}
	  | {(\prefix{x}{y}{(\outputp{x}{y} | @{y})) | P}} & \nonumber\\
	\red
	& \ldots & \nonumber\\
	\red^*
	& P | P | \ldots & \nonumber
\end{eqnarray}

Of course, this encoding, as an implementation, runs away, unfolding
$\bangp{P}$ eagerly. A lazier and more implementable replication
operator, restricted to input-guarded processes, may be obtained as follows.

\begin{eqnarray}
\bangp{\prefix{u}{v}{P}} 
	:= 
	\binpar{\lift{x}{\prefix{u}{v}{(\binpar{D(x)}{P})}}}{D(x)} \nonumber
\end{eqnarray}

\begin{remark}
  Note that the lazier definition still does not deal with summation
  or mixed summation (i.e. sums over input and output). The reader is
  invited to construct definitions of replication that deal with these
  features. 

  Further, the definitions are parameterized in a name, $x$. Can you,
  gentle reader, make a definition that eliminates this parameter and
  guarantees no accidental interaction between the replication
  machinery and the process being replicated -- i.e. no accidental
  sharing of names used by the process to get its work done and the
  name(s) used by the replication to effect copying. This latter
  revision of the definition of replication is crucial to obtaining
  the expected identity $!!P \sim !P$.
\end{remark}

\begin{remark}\label{rem:paradoxical_combinator}
  The reader familiar with the lambda calculus will have noticed the
  similarity between $D$ and the paradoxical combinator.

  [Ed. note: the existence of this seems to suggest we have to be more
  restrictive on the set of processes and names we admit if we are to
  support no-cloning.]
\end{remark}

\subsubsection{Bisimulation}

The computational dynamics gives rise to another kind of equivalence,
the equivalence of computational behavior. As previously mentioned
this is typically captured \emph{via} some form of bisimulation.

% The notion we use in this paper is weak barbed bisimulation
% \cite{milner91polyadicpi}.

The notion we use in this paper is derived from weak barbed
bisimulation \cite{milner91polyadicpi}. 

\begin{definition}
An \emph{observation relation}, $\downarrow_{\mathcal N}$, over a set
of names, $\mathcal N$, is the smallest relation satisfying the rules
below.

\infrule[Out-barb]{y \in {\mathcal N}, \; x \nameeq y}
		  {\outputp{x}{v} \downarrow_{\mathcal N} x}
\infrule[Par-barb]{\mbox{$P\downarrow_{\mathcal N} x$ or $Q\downarrow_{\mathcal N} x$}}
		  {\binpar{P}{Q} \downarrow_{\mathcal N} x}

We write $P \Downarrow_{\mathcal N} x$ if there is $Q$ such that 
$P \wred Q$ and $Q \downarrow_{\mathcal N} x$.
\end{definition}

\begin{definition}
%\label{def.bbisim}
An  ${\mathcal N}$-\emph{barbed bisimulation} over a set of names, ${\mathcal N}$, is a symmetric binary relation 
${\mathcal S}_{\mathcal N}$ between agents such that $P\rel{S}_{\mathcal N}Q$ implies:
\begin{enumerate}
\item If $P \red P'$ then $Q \wred Q'$ and $P'\rel{S}_{\mathcal N} Q'$.
\item If $P\downarrow_{\mathcal N} x$, then $Q\Downarrow_{\mathcal N} x$.
\end{enumerate}
$P$ is ${\mathcal N}$-barbed bisimilar to $Q$, written
$P \wbbisim_{\mathcal N} Q$, if $P \rel{S}_{\mathcal N} Q$ for some ${\mathcal N}$-barbed bisimulation ${\mathcal S}_{\mathcal N}$.
\end{definition}

$\mathcal{R} \subseteq \pi \times \pi$

$P \mathcal{R} Q => \forall P'. P \red P' \Rightarrow \exists Q'. Q \red Q', P' \mathcal{R} Q'$

$P \vdash x \Rightarrow Q \vdash x$

\begin{mathpar}
  \inferrule*[lab=Out-barb]{x \nameeq y}{{y}!\langle{Q}\rangle \vdash x}
  \and
  \inferrule*[lab=Par-barb]{\mbox{$P\vdash x$ or $Q\vdash x$}}{\binpar{P}{Q} \vdash x}
\end{mathpar}

\subsubsection{Contexts}

One of the principle advantages of computational calculi like the
$\pi$-calculus is a well-defined notion of context,
contextual-equivalence and a correlation between
contextual-equivalence and notions of bisimulation. The notion of
context allows the decomposition of a process into (sub-)process and
its syntactic environment, its context. Thus, a context may be
thought of as a process with a ``hole'' (written $\Box$) in it. The
application of a context $M$ to a process $P$, written $M[P]$, is
tantamount to filling the hole in $M$ with $P$. In this paper we do
not need the full weight of this theory, but do make use of the notion
of context in the proof the main theorem. 

\begin{mathpar}
  \inferrule* [lab=summation] {} {{M_{M},M_{N}} \bc \Box \;|\; x.M_{A} \;|\; M_{M}+M_{N}}
  \and
  \inferrule* [lab=agent] {} {{M_{A}} \bc (\vec{x})M_{P} \;| \; \clift{P_0,\ldots,M_{P},\ldots,P_N}}
  \and \\
  \inferrule* [lab=process] {} {{M_{P}} \bc M_{N} \;| \;P|M_{P} }
\end{mathpar} 

\begin{mathpar}
  \inferrule* [lab=sychronization] {} {M_{N} \bc \Box \;|\; x?M_{F} \;|\; x!M_{C}}
  \and
  \inferrule* [lab=abstraction] {} {{M_{F}} \bc (x)M_{P} }
  \and
  \inferrule* [lab=concretion] {} {{M_{C}} \bc \langle M_{P} \rangle }
  \and \\
  \inferrule* [lab=process] {} {{M_{P}} \bc M_{N} \;| \;P|M_{P} }
\end{mathpar}

\begin{definition}[contextual application] Given a context $M$, and
  process $P$, we define the \emph{contextual application}, $M[P] :=
  M\{P/\Box\}$. That is, the contextual application of M to P is the
  substitution of $P$ for $\Box$ in $M$.
\end{definition}

$\meaningof{-} : L \to \mathcal{P}(\pi)$

\begin{mathpar}
  \inferrule* [lab=collection] {} {\meaningof{true} = \pi, \and \meaningof{~E} = \pi \setminus \meaningof{E}, \and \meaningof{E_{1} \& E_{2}} = \meaningof{E_{1}} \cap \meaningof{E_{2}}}
\end{mathpar}

\begin{mathpar}
  \inferrule* [lab=structure] {} {\meaningof{0} = \{ P \in \pi | P \equiv 0 \}, \and \\ \meaningof{E_1 | E_2} = \{ P \in \pi | P \equiv P_{1} | P_{2}, P_{1} \in \meaningof{E_{1}}, P_{2} \in \meaningof{E_2}\} }
\end{mathpar}

\begin{mathpar}
 \inferrule* [lab=behavior] {} {\meaningof{\langle a?b \rangle E} = \{ P \in \pi | P \equiv Q | u?(y)P', \\ \and \\\\ \and \\ \;\;\; u \in \meaningof{a}, \forall z.P'\{z/y\} \in \meaningof{E\{z/b\}}\}, \and \\ \meaningof{a!E} = \{ P \in \pi | P \equiv Q | x!\langle P' \rangle, x \in \meaningof{a} P' \in \meaningof{E}\} }
\end{mathpar}

\begin{mathpar}
 \inferrule* [lab=nominal] {} {\meaningof{\quotep{E}} = \{ \quotep{P} \in \quotep{\pi} | P \in \meaningof{E} \}, \and \meaningof{\quotep{P}} = \{ \quotep{Q} \in \quotep{\pi} | P \equiv Q \} \and \\ \meaningof{@\quotep{E}} = \{ P \in \pi | P \equiv @x, x \in \meaningof{E} \}}
\end{mathpar}

\begin{eqnarray*}
  \\
  \meaningof{-} : TS \to ST
\end{eqnarray*}

\begin{eqnarray*}
  \\
  L : TS \to ST
\end{eqnarray*}

\begin{eqnarray*}
  \\
  P \models E \iff P \in \meaningof{E}
\end{eqnarray*}

\begin{eqnarray*}
  P \approx_{L} Q \iff \forall E \in L. P \models E \iff Q \models E
\end{eqnarray*}

\begin{eqnarray*}
  P \approx_{K} Q
\end{eqnarray*}

\begin{eqnarray*}
  P \approx Q
\end{eqnarray*}

$\approx_{K} = \approx = \approx_{L}$

\subsubsection{Contextual duality}

Note that contexts extend the quotation operation to a family of
operations from processes to names. Given a context, $M$, we can
define a \emph{nominal context}, $\quotep{M}$ by $\quotep{M}[P] :=
\quotep{M[P]}$. To foreshadow what is to come we observe that these
operations enjoy a duality with processes very much like the duality
between vectors and maps from vectors to scalars.

Further, because the calculus is essentially higher-order, we have a
correspondence between contexts and processes. More specifically,
given a name $x$ and a context $M$ we can construct $M^{*}_{x}$ such
that 

\begin{mathpar}
  M^{*}_{x} | \lift{x}{P} \red M[P]
\end{mathpar}

namely,

\begin{mathpar}
  M^{*}_{x} := x?(u).M[\dropn{u}]
\end{mathpar}

The dependence of $M^{*}_{x}$ on a name makes it an abstraction, 

\begin{mathpar}
  M^{*} := (x)x?(u).M[\dropn{u}]
\end{mathpar}

\subsection{Additional notation}

It will sometimes be convenient to denote the process a name
quotes. We already have the notation $x = \quotep{P}$, but it will be
convenient to introduce an alternate notation, $\procn{x}$, when we
want to emphasize the connection to the use of the name. Note that, by
virtue of name equivalence, $\quotep{\procn{x}} \nameeq x$; so, the
notation is consistent with previous definitions.

Further, because names have structure it is possible to effect
substitutions on the basis of that structure. This means we need to
upgrade our notation for substitutions, which we accomplish by
adapting comprehension notation. Thus,

\begin{mathpar}
  P\{ y / x : x \in S \}
\end{mathpar}

is interpreted to mean the process derived from P by replacing (in a
capture-avoiding manner) each occurrence of $x$ in $S$ by $y$. For example,

\begin{mathpar}
  P\{ \quotep{\procn{x}|\procn{x}} / x : x \in \freenames{P} \}
\end{mathpar}

will replace each (occurrence) of a free name $x$ in $P$ by
$\quotep{\procn{x}|\procn{x}}$.

Also, we will avail ourselves of the notation $x^{L}$ and $x^{R}$ to
denote injections of a name into disjoint copies of the name
space. There are numerous ways to accomplish this. One example can be
found in \cite{MeredithR05}. This notation overloads to vectors of
names: $\vec{x}^{\pi} := (x_{i}^{\pi} \; : \; 0 \leq i < |\vec{x}| )$ where $\pi \in \{L,R\}$.

We also use $P^{\Box} := P|\Box$.

In \cite{MeredithR05} an interpretation of the new operator is
given. It turns out that there are several possible interpretations
all enjoying the requisite algebraic properties of the operator (see
\cite{milner91polyadicpi}). We will therefore make liberal use of
$(\nu\; \vec{x})P$.

% subsection the_syntax_and_semantics_of_the_notation_system (end)   

\input{qm2pi.qmops} 

\input{qm2pi.sterngerlach} 

\input{qm2pi.metric} 

% section concurrent_process_calculi (end)

%\input{qm2pi.proofsketch}

% section proof sketch (end)

%\input{qm2pi.slviaknots} 

% section spatial logic via knots (end)

\input{qm2pi.conclusion}

% section conclusion (end)

%\input{qm2pi.dtcodes} 

% section wiring algorithm (end)

\input{qm2pi.ack} 

% section acknowledgments (end)

\newpage


\bibliographystyle{plain}   
\bibliography{../../biblios/main.bib}

\input{qm2pi.rhodetails}

\end{document}

 

% section wiring algorithm (end)

\documentclass[12pt]{llncs}
%\documentclass{jktr}

\usepackage[pdftex]{hyperref}                   
\usepackage {listings}
\usepackage {mathpartir}
\usepackage{bcprules}
%\usepackage{listings}
                       
\usepackage{graphicx} 
%\usepackage[margins=2.5cm,nohead,nofoot]{geometry}
%\usepackage{geometry}
\usepackage{amsfonts}
\usepackage{amstext}
\usepackage{latexsym}
\usepackage{amssymb}
\usepackage{color}


%\include{myPreamble}
\include{qm2pi.local} 

%\ifpdf
%\usepackage[pdftex]{graphicx}
%\else
%\usepackage{graphicx}
%\fi

 % \ifpdf
%  \usepackage{pdfsync}
%  \if


%\title{Brief Article}
%\author{David F. Snyder}
%\author{L.G. Meredith}

%\address{Dept. of Math., Texas State University--San Marcos, San Marcos, TX 78666}
       
\pagestyle{empty}


\begin{document}

\lstset{language=[Objective]Caml,frame=shadowbox}

\input{qm2pi.front}

% section front matter (end)

\input{qm2pi.intro} 
 
% section introduction (end)

% \input{qm2pi.knotations} 

% section notation (end)

\input{qm2pi.process.calculi} 

% section concurrent_process_calculi_and_spatial_logics_ (end)
    
%\input{qm2pi.knots2pi} 

%\input{qm2pi.trefoil} 

%\input{qm2pi.mainthm} 

% subsection basic_interpretation (end)

%\input{qm2pi.rho.presentation} 
\subsection{The syntax and semantics of the notation system}\label{sub:the_syntax_and_semantics_of_the_notation_system} % (fold)

We now summarize a technical presentation of the calculus that
embodies our theory of dynamics. The typical presentation of such a
calculus follows the style of giving generators and relations on
them. The grammar, below, describing term constructors, freely
generates the set of processes, $\Proc$. This set is then quotiented
by a relation known as structural congruence and it is over this set
that the notion of dynamics is expressed. This presentation is
essentially that of \cite{MeredithR05} with the addition of
polyadicity and summation. For readability we have relegated some of
the technical subtleties to an appendix.

\subsubsection{Process grammar}\label{subsub:process_grammar}

\begin{mathpar}
  \inferrule* [lab=synchronization] {} {{M} \bc \pzero \;|\; x?F \;|\; x!C }
  \and
  \inferrule* [lab=abstraction] {} {{F} \bc (x)P}
  \and
  \inferrule* [lab=concretion] {} {{C} \bc \langle Q \rangle}
  \and
  \inferrule* [lab=process] {} {{P,Q} \bc M \;| \;P|Q \;|\; @{x}}
  \and
  \inferrule* [lab=name] {} {{x} \bc \quotep{P}}
\end{mathpar} 

Note that $\vec{x}$ (resp. $\vec{P}$) denotes a vector of names
(resp. processes) of length $|\vec{x}|$ (resp. $|\vec{P}|$). We adopt
the following useful abbreviations.

\begin{mathpar}
   x?(\vec{y}).P := x.(\vec{y})P \and  x\clift{\vec{P}} := x.\clift{\vec{P}}
   \and x!(y) := \lift{x}{\dropn{y}}
   \and \Pi_{i=0}^{n-1}P_i := P_0 | \ldots | P_{n-1}
\end{mathpar}

\subsubsection{Structural congruence}

\paragraph{Free and bound names and alpha-equivalence.} At the
core of structural equivalence is alpha-equivalence which identifies
process that are the same up to a change of variable. Formally, we
recognize the distinction between free and bound names. The free names
of a process, $\freenames{P}$, may be calculated recursively as
follows:

\begin{mathpar}
\freenames{\pzero} := \emptyset
  \and \\
  \freenames{x?(y).P} := \{ x \} \cup (\freenames{P} \setminus \{ y \})
  \and 
  \freenames{x!\langle P \rangle} := \{ x \} \cup \{ P \} 
  \and \\
  \freenames{P|Q} := \freenames{P} \cup \freenames{Q}
  \and \\
  \freenames{@{x}} := \{ x \}
\end{mathpar}

$\pi$
$\quotep{\pi}$

$\freenames{-} : \pi \to \mathcal{P}(\quotep{\pi})$

\begin{eqnarray*}
  \freenames{\pzero} & := & \emptyset \\
  \freenames{x?(y).P} & := & \{ x \} \cup (\freenames{P} \setminus \{ y \}) \\
  \freenames{x!\langle P \rangle} & := & \{ x \} \cup \{ P \} \\
  \freenames{P|Q} & := & \freenames{P} \cup \freenames{Q} \\
  \freenames{\dropn{x}} & := & \{ x \}
\end{eqnarray*}

The bound names of a process, $\boundnames{P}$, are those names occurring in $P$
that are not free. For example, in $x?(y).0$, the name $x$ is free, while $y$ is bound.

\begin{mathpar}
  \inferrule* [lab=monoidal-laws] {} { P|Q \equiv Q|P \and P|0 \equiv P \and P|(Q|R) \equiv (P|Q)|R }
\end{mathpar}

\begin{mathpar}
  \inferrule* [lab=alpha-equivalence] {} { (x)P \equiv (y)P\{y/x\} \and y \not\in \freenames{P} }
\end{mathpar}

\begin{definition}
Then two processes, $P,Q$, are alpha-equivalent if $P = Q\{\vec{y}/\vec{x}\}$ for
some $\vec{x} \in \boundnames{Q},\vec{y} \in \boundnames{P}$, where $Q\{\vec{y}/\vec{x}\}$
denotes the capture-avoiding substitution of $\vec{y}$ for $\vec{x}$ in $Q$.
\end{definition}

\begin{definition}
  The {\em structural congruence} \cite{SangiorgiWalker} , $\equiv$,
  between processes is the least congruence containing
  alpha-equivalence, satisfying the abelian monoid laws
  (associativity, commutativity and $\pzero$ as identity) for parallel
  composition $|$ and for summation $+$.
\end{definition}

\subsection{Name equivalence}

We take name equivalence, written $\nameeq$, to be the smallest
equivalence relation generated by the following rules.

\begin{mathpar}
\inferrule*[lab=Quote-drop]
{ }
{ \quotep{@{x}} \nameeq x }

\inferrule*[lab=Struct-equiv]
{ P \scong Q }
{ \quotep{P} \nameeq \quotep{Q} }
\end{mathpar}

The astute reader will have noticed that the mutual recursion of names
and processes imposes a mutual recursion on alpha-equivalence and
structural equivalence via name-equivalence. Fortunately, all of this
works out pleasantly and we may calculate in the natural way, free of
concern. The reader interested in the details is referred to the
appendix \ref{appendix:rho_details}.

\subsection{Substitution}

We use $\Proc$ for the set of processes, $\QProc$ for the set of
names, and $\id{\{}\vec{y} / \vec{x} \id{\}}$ to denote partial maps,
$s : \QProc \rightarrow \QProc$. A map, $s$ lifts, uniquely, to a map
on process terms, $\widehat{s} : \Proc \rightarrow \Proc$ by the
following equations.

\begin{mathpar}
  (0) \psubstp{Q}{P} := 0 \\
  (R \juxtap S) \psubstp{Q}{P}
  :=    
  (R)\psubstp{Q}{P} \juxtap (S) \psubstp{Q}{P} \\
  (x?(y).R) \psubstp{Q}{P}    
  :=    
  (x)\substp{Q}{P} (z)\concat( (R \psubstn{z}{y}) \psubstp{Q}{P} ) \\
  (\lift{x}{R}) \psubstp{Q}{P}  
  :=
  \lift{(x)\substp{Q}{P}}{ R \psubstp{Q}{P} } \\
%   (\dropn{x})  \psubstp{Q}{P}       
%   := 
%   \left\{ 
%     \begin{array}{ccc} 
%       \dropn{\quotep{Q}} & & x \nameeq \quotep{P} \\
%       \dropn{x} & & otherwise \\
%     \end{array}
%   \right. 
  (\dropn{x})  \psubstp{Q}{P}       
  := 
  \left\{ 
    \begin{array}{ccc} 
      Q & & x \nameeq \quotep{P} \\
      \dropn{x} & & otherwise \\
    \end{array}
  \right.
\end{mathpar}
 

where

\begin{eqnarray}
  (x)\id{\{} \lpquote Q \rpquote / \lpquote P \rpquote \id{\}}            = 
  \left\{ 
    \begin{array}{ccc}
      \lpquote Q \rpquote & & x \nameeq \lpquote P \rpquote \\
      x & & otherwise \\
    \end{array}
  \right. \nonumber
\end{eqnarray}

and $z$ is chosen distinct from $\quotep{P}$, $\quotep{Q}$, the free
names in $Q$, and all the names in $R$. Our $\alpha$-equivalence will
be built in the standard way from this substitution.

\begin{remark}\label{rem:no_self_referential_names}
  One consequence of these definitions is that $\forall P. \quotep{P}
  \not\in \freenames{P}$.
\end{remark}

\subsection{ Dynamic quote: an example }

Anticipating something of what's to come, consider applying the
substitution, $\widehat{\id{\{}u / z \id{\}}}$, to the following pair
of processes, $\lift{w}{y!(z)}$ and $w[ \lpquote y!(z) \rpquote ]$.

\begin{eqnarray}
	\lift{w}{y!(z)}\widehat{\id{\{}u / z \id{\}}}
		& = &
		\lift{w}{y!(u)} \nonumber\\
	w[ \lpquote y!(z) \rpquote ] \widehat{ \id{\{}u / z \id{\}} }
		& = &
		w[ \lpquote y!(z) \rpquote ] \nonumber
\end{eqnarray}

Because the body of the process between quotes is impervious to
substitution, we get radically different answers. In fact, by
examining the first process in an input context,
e.g. $x?(z).\lift{w}{y!(z)}$, we see that the process under the lift
operator may be shaped by prefixed inputs binding a name inside it. In
this sense, the lift operator will be seen as a way to dynamically
construct processes before reifying them as names.

Finally equipped with these standard features we can present the
dynamics of the calculus.

\subsubsection{Operational semantics} 

Finally, we introduce the computational dynamics. What marks these
algebras as distinct from other more traditionally studied algebraic
structures, e.g. vector spaces or polynomial rings, is the manner in
which dynamics is captured. In traditional structures, dynamics is typically
expressed through morphisms between such structures, as in linear maps
between vector spaces or morphisms between rings. In algebras
associated with the semantics of computation, the dynamics is
expressed as part of the algebraic structure itself, through a
reduction reduction relation typically denoted by $\red$. Below, we
give a recursive presentation of this relation for the calculus used
in the encoding.

$\red \subseteq \pi \times \pi$
$\red : \pi \to \mathcal{P}(\pi)$

\begin{mathpar}
  \inferrule* [lab=Comm] { \textsf{match}( x_{src}, x_{trgt} ) } { x_{trgt}?(y)P \; | \; x_{src}!\langle {Q} \rangle \red P\{\quotep{Q}/y}\} }
  \and \\
  \inferrule* [lab=Par] {{P} \red {P}'} {{{P} | {Q}} \red {{P}' | {Q}}}
  \and
  \inferrule* [lab=Equiv]{{{P} \scong {P}'} \andalso {{P}' \red {Q}'} \andalso {{Q}' \scong {Q}}}{{P} \red {Q}}
\end{mathpar}

\begin{eqnarray*}
  match_{\equiv} (\quotep{P},\quotep{Q}) & := & P \equiv Q \\
  match_{\dagger}(\quotep{P},\quotep{Q}) & := & \forall R. P|Q \red^{*} R => R \red^{*} 0 \\
  match_{K}(\quotep{P},\quotep{Q}) & := & K \mbox{ for some context } K
\end{eqnarray*}

$u?(x)P | u!\langle Q \rangle \red P\{\quotep{Q}/x\}$

%We write $\wred$ for $\red^*$, and $P\red$ if $\exists Q $ such that $ P \red Q$.
We write $P\red$ if $\exists Q $ such that $ P \red Q$ and $P\not\red$, otherwise.

\section{Replication}

As mentioned before, it is known that replication (and hence
recursion) can be implemented in a higher-order process algebra
\cite{SangiorgiWalker}. As our first example of calculation with the
machinery thus far presented we give the construction explicitly in
the {\rhoc}.

\begin{eqnarray}
	D_{x} & := & \prefix{x}{y}{(\binpar{\outputp{x}{y}}{@{y}})} \nonumber\\
	\bangp_{x}{P} & := & \binpar{{x}!\langle{\binpar{D_{x}}{P}}\rangle}{D_{x}} \nonumber
\end{eqnarray}

\begin{eqnarray}
	\bangp_{x}{P} & & \nonumber\\
	=
	& {x}!\langle{(\prefix{x}{y}{(\outputp{x}{y} | @{y})) | P}}\rangle 
	      | \prefix{x}{y}{(\outputp{x}{y} | @{y})} & \nonumber\\
	\red
	& (\outputp{x}{y} | @{y})\substn{\quotep{(\prefix{x}{y}{(@{y} | \outputp{x}{y})) | P}}}{y} & \nonumber\\
	=
	& \outputp{x}{\quotep{(\prefix{x}{y}{(\outputp{x}{y} | @{y})) | P}}}
	  | {(\prefix{x}{y}{(\outputp{x}{y} | @{y})) | P}} & \nonumber\\
	\red
	& \ldots & \nonumber\\
	\red^*
	& P | P | \ldots & \nonumber
\end{eqnarray}

Of course, this encoding, as an implementation, runs away, unfolding
$\bangp{P}$ eagerly. A lazier and more implementable replication
operator, restricted to input-guarded processes, may be obtained as follows.

\begin{eqnarray}
\bangp{\prefix{u}{v}{P}} 
	:= 
	\binpar{\lift{x}{\prefix{u}{v}{(\binpar{D(x)}{P})}}}{D(x)} \nonumber
\end{eqnarray}

\begin{remark}
  Note that the lazier definition still does not deal with summation
  or mixed summation (i.e. sums over input and output). The reader is
  invited to construct definitions of replication that deal with these
  features. 

  Further, the definitions are parameterized in a name, $x$. Can you,
  gentle reader, make a definition that eliminates this parameter and
  guarantees no accidental interaction between the replication
  machinery and the process being replicated -- i.e. no accidental
  sharing of names used by the process to get its work done and the
  name(s) used by the replication to effect copying. This latter
  revision of the definition of replication is crucial to obtaining
  the expected identity $!!P \sim !P$.
\end{remark}

\begin{remark}\label{rem:paradoxical_combinator}
  The reader familiar with the lambda calculus will have noticed the
  similarity between $D$ and the paradoxical combinator.

  [Ed. note: the existence of this seems to suggest we have to be more
  restrictive on the set of processes and names we admit if we are to
  support no-cloning.]
\end{remark}

\subsubsection{Bisimulation}

The computational dynamics gives rise to another kind of equivalence,
the equivalence of computational behavior. As previously mentioned
this is typically captured \emph{via} some form of bisimulation.

% The notion we use in this paper is weak barbed bisimulation
% \cite{milner91polyadicpi}.

The notion we use in this paper is derived from weak barbed
bisimulation \cite{milner91polyadicpi}. 

\begin{definition}
An \emph{observation relation}, $\downarrow_{\mathcal N}$, over a set
of names, $\mathcal N$, is the smallest relation satisfying the rules
below.

\infrule[Out-barb]{y \in {\mathcal N}, \; x \nameeq y}
		  {\outputp{x}{v} \downarrow_{\mathcal N} x}
\infrule[Par-barb]{\mbox{$P\downarrow_{\mathcal N} x$ or $Q\downarrow_{\mathcal N} x$}}
		  {\binpar{P}{Q} \downarrow_{\mathcal N} x}

We write $P \Downarrow_{\mathcal N} x$ if there is $Q$ such that 
$P \wred Q$ and $Q \downarrow_{\mathcal N} x$.
\end{definition}

\begin{definition}
%\label{def.bbisim}
An  ${\mathcal N}$-\emph{barbed bisimulation} over a set of names, ${\mathcal N}$, is a symmetric binary relation 
${\mathcal S}_{\mathcal N}$ between agents such that $P\rel{S}_{\mathcal N}Q$ implies:
\begin{enumerate}
\item If $P \red P'$ then $Q \wred Q'$ and $P'\rel{S}_{\mathcal N} Q'$.
\item If $P\downarrow_{\mathcal N} x$, then $Q\Downarrow_{\mathcal N} x$.
\end{enumerate}
$P$ is ${\mathcal N}$-barbed bisimilar to $Q$, written
$P \wbbisim_{\mathcal N} Q$, if $P \rel{S}_{\mathcal N} Q$ for some ${\mathcal N}$-barbed bisimulation ${\mathcal S}_{\mathcal N}$.
\end{definition}

$\mathcal{R} \subseteq \pi \times \pi$

$P \mathcal{R} Q => \forall P'. P \red P' \Rightarrow \exists Q'. Q \red Q', P' \mathcal{R} Q'$

$P \vdash x \Rightarrow Q \vdash x$

\begin{mathpar}
  \inferrule*[lab=Out-barb]{x \nameeq y}{{y}!\langle{Q}\rangle \vdash x}
  \and
  \inferrule*[lab=Par-barb]{\mbox{$P\vdash x$ or $Q\vdash x$}}{\binpar{P}{Q} \vdash x}
\end{mathpar}

\subsubsection{Contexts}

One of the principle advantages of computational calculi like the
$\pi$-calculus is a well-defined notion of context,
contextual-equivalence and a correlation between
contextual-equivalence and notions of bisimulation. The notion of
context allows the decomposition of a process into (sub-)process and
its syntactic environment, its context. Thus, a context may be
thought of as a process with a ``hole'' (written $\Box$) in it. The
application of a context $M$ to a process $P$, written $M[P]$, is
tantamount to filling the hole in $M$ with $P$. In this paper we do
not need the full weight of this theory, but do make use of the notion
of context in the proof the main theorem. 

\begin{mathpar}
  \inferrule* [lab=summation] {} {{M_{M},M_{N}} \bc \Box \;|\; x.M_{A} \;|\; M_{M}+M_{N}}
  \and
  \inferrule* [lab=agent] {} {{M_{A}} \bc (\vec{x})M_{P} \;| \; \clift{P_0,\ldots,M_{P},\ldots,P_N}}
  \and \\
  \inferrule* [lab=process] {} {{M_{P}} \bc M_{N} \;| \;P|M_{P} }
\end{mathpar} 

\begin{mathpar}
  \inferrule* [lab=sychronization] {} {M_{N} \bc \Box \;|\; x?M_{F} \;|\; x!M_{C}}
  \and
  \inferrule* [lab=abstraction] {} {{M_{F}} \bc (x)M_{P} }
  \and
  \inferrule* [lab=concretion] {} {{M_{C}} \bc \langle M_{P} \rangle }
  \and \\
  \inferrule* [lab=process] {} {{M_{P}} \bc M_{N} \;| \;P|M_{P} }
\end{mathpar}

\begin{definition}[contextual application] Given a context $M$, and
  process $P$, we define the \emph{contextual application}, $M[P] :=
  M\{P/\Box\}$. That is, the contextual application of M to P is the
  substitution of $P$ for $\Box$ in $M$.
\end{definition}

$\meaningof{-} : L \to \mathcal{P}(\pi)$

\begin{mathpar}
  \inferrule* [lab=collection] {} {\meaningof{true} = \pi, \and \meaningof{~E} = \pi \setminus \meaningof{E}, \and \meaningof{E_{1} \& E_{2}} = \meaningof{E_{1}} \cap \meaningof{E_{2}}}
\end{mathpar}

\begin{mathpar}
  \inferrule* [lab=structure] {} {\meaningof{0} = \{ P \in \pi | P \equiv 0 \}, \and \\ \meaningof{E_1 | E_2} = \{ P \in \pi | P \equiv P_{1} | P_{2}, P_{1} \in \meaningof{E_{1}}, P_{2} \in \meaningof{E_2}\} }
\end{mathpar}

\begin{mathpar}
 \inferrule* [lab=behavior] {} {\meaningof{\langle a?b \rangle E} = \{ P \in \pi | P \equiv Q | u?(y)P', \\ \and \\\\ \and \\ \;\;\; u \in \meaningof{a}, \forall z.P'\{z/y\} \in \meaningof{E\{z/b\}}\}, \and \\ \meaningof{a!E} = \{ P \in \pi | P \equiv Q | x!\langle P' \rangle, x \in \meaningof{a} P' \in \meaningof{E}\} }
\end{mathpar}

\begin{mathpar}
 \inferrule* [lab=nominal] {} {\meaningof{\quotep{E}} = \{ \quotep{P} \in \quotep{\pi} | P \in \meaningof{E} \}, \and \meaningof{\quotep{P}} = \{ \quotep{Q} \in \quotep{\pi} | P \equiv Q \} \and \\ \meaningof{@\quotep{E}} = \{ P \in \pi | P \equiv @x, x \in \meaningof{E} \}}
\end{mathpar}

\begin{eqnarray*}
  \\
  \meaningof{-} : TS \to ST
\end{eqnarray*}

\begin{eqnarray*}
  \\
  L : TS \to ST
\end{eqnarray*}

\begin{eqnarray*}
  \\
  P \models E \iff P \in \meaningof{E}
\end{eqnarray*}

\begin{eqnarray*}
  P \approx_{L} Q \iff \forall E \in L. P \models E \iff Q \models E
\end{eqnarray*}

\begin{eqnarray*}
  P \approx_{K} Q
\end{eqnarray*}

\begin{eqnarray*}
  P \approx Q
\end{eqnarray*}

$\approx_{K} = \approx = \approx_{L}$

\subsubsection{Contextual duality}

Note that contexts extend the quotation operation to a family of
operations from processes to names. Given a context, $M$, we can
define a \emph{nominal context}, $\quotep{M}$ by $\quotep{M}[P] :=
\quotep{M[P]}$. To foreshadow what is to come we observe that these
operations enjoy a duality with processes very much like the duality
between vectors and maps from vectors to scalars.

Further, because the calculus is essentially higher-order, we have a
correspondence between contexts and processes. More specifically,
given a name $x$ and a context $M$ we can construct $M^{*}_{x}$ such
that 

\begin{mathpar}
  M^{*}_{x} | \lift{x}{P} \red M[P]
\end{mathpar}

namely,

\begin{mathpar}
  M^{*}_{x} := x?(u).M[\dropn{u}]
\end{mathpar}

The dependence of $M^{*}_{x}$ on a name makes it an abstraction, 

\begin{mathpar}
  M^{*} := (x)x?(u).M[\dropn{u}]
\end{mathpar}

\subsection{Additional notation}

It will sometimes be convenient to denote the process a name
quotes. We already have the notation $x = \quotep{P}$, but it will be
convenient to introduce an alternate notation, $\procn{x}$, when we
want to emphasize the connection to the use of the name. Note that, by
virtue of name equivalence, $\quotep{\procn{x}} \nameeq x$; so, the
notation is consistent with previous definitions.

Further, because names have structure it is possible to effect
substitutions on the basis of that structure. This means we need to
upgrade our notation for substitutions, which we accomplish by
adapting comprehension notation. Thus,

\begin{mathpar}
  P\{ y / x : x \in S \}
\end{mathpar}

is interpreted to mean the process derived from P by replacing (in a
capture-avoiding manner) each occurrence of $x$ in $S$ by $y$. For example,

\begin{mathpar}
  P\{ \quotep{\procn{x}|\procn{x}} / x : x \in \freenames{P} \}
\end{mathpar}

will replace each (occurrence) of a free name $x$ in $P$ by
$\quotep{\procn{x}|\procn{x}}$.

Also, we will avail ourselves of the notation $x^{L}$ and $x^{R}$ to
denote injections of a name into disjoint copies of the name
space. There are numerous ways to accomplish this. One example can be
found in \cite{MeredithR05}. This notation overloads to vectors of
names: $\vec{x}^{\pi} := (x_{i}^{\pi} \; : \; 0 \leq i < |\vec{x}| )$ where $\pi \in \{L,R\}$.

We also use $P^{\Box} := P|\Box$.

In \cite{MeredithR05} an interpretation of the new operator is
given. It turns out that there are several possible interpretations
all enjoying the requisite algebraic properties of the operator (see
\cite{milner91polyadicpi}). We will therefore make liberal use of
$(\nu\; \vec{x})P$.

% subsection the_syntax_and_semantics_of_the_notation_system (end)   

\input{qm2pi.qmops} 

\input{qm2pi.sterngerlach} 

\input{qm2pi.metric} 

% section concurrent_process_calculi (end)

%\input{qm2pi.proofsketch}

% section proof sketch (end)

%\input{qm2pi.slviaknots} 

% section spatial logic via knots (end)

\input{qm2pi.conclusion}

% section conclusion (end)

%\input{qm2pi.dtcodes} 

% section wiring algorithm (end)

\input{qm2pi.ack} 

% section acknowledgments (end)

\newpage


\bibliographystyle{plain}   
\bibliography{../../biblios/main.bib}

\input{qm2pi.rhodetails}

\end{document}

 

% section acknowledgments (end)

\newpage


\bibliographystyle{plain}   
\bibliography{../../biblios/main.bib}

\documentclass[12pt]{llncs}
%\documentclass{jktr}

\usepackage[pdftex]{hyperref}                   
\usepackage {listings}
\usepackage {mathpartir}
\usepackage{bcprules}
%\usepackage{listings}
                       
\usepackage{graphicx} 
%\usepackage[margins=2.5cm,nohead,nofoot]{geometry}
%\usepackage{geometry}
\usepackage{amsfonts}
\usepackage{amstext}
\usepackage{latexsym}
\usepackage{amssymb}
\usepackage{color}


%\include{myPreamble}
\include{qm2pi.local} 

%\ifpdf
%\usepackage[pdftex]{graphicx}
%\else
%\usepackage{graphicx}
%\fi

 % \ifpdf
%  \usepackage{pdfsync}
%  \if


%\title{Brief Article}
%\author{David F. Snyder}
%\author{L.G. Meredith}

%\address{Dept. of Math., Texas State University--San Marcos, San Marcos, TX 78666}
       
\pagestyle{empty}


\begin{document}

\lstset{language=[Objective]Caml,frame=shadowbox}

\input{qm2pi.front}

% section front matter (end)

\input{qm2pi.intro} 
 
% section introduction (end)

% \input{qm2pi.knotations} 

% section notation (end)

\input{qm2pi.process.calculi} 

% section concurrent_process_calculi_and_spatial_logics_ (end)
    
%\input{qm2pi.knots2pi} 

%\input{qm2pi.trefoil} 

%\input{qm2pi.mainthm} 

% subsection basic_interpretation (end)

%\input{qm2pi.rho.presentation} 
\subsection{The syntax and semantics of the notation system}\label{sub:the_syntax_and_semantics_of_the_notation_system} % (fold)

We now summarize a technical presentation of the calculus that
embodies our theory of dynamics. The typical presentation of such a
calculus follows the style of giving generators and relations on
them. The grammar, below, describing term constructors, freely
generates the set of processes, $\Proc$. This set is then quotiented
by a relation known as structural congruence and it is over this set
that the notion of dynamics is expressed. This presentation is
essentially that of \cite{MeredithR05} with the addition of
polyadicity and summation. For readability we have relegated some of
the technical subtleties to an appendix.

\subsubsection{Process grammar}\label{subsub:process_grammar}

\begin{mathpar}
  \inferrule* [lab=synchronization] {} {{M} \bc \pzero \;|\; x?F \;|\; x!C }
  \and
  \inferrule* [lab=abstraction] {} {{F} \bc (x)P}
  \and
  \inferrule* [lab=concretion] {} {{C} \bc \langle Q \rangle}
  \and
  \inferrule* [lab=process] {} {{P,Q} \bc M \;| \;P|Q \;|\; @{x}}
  \and
  \inferrule* [lab=name] {} {{x} \bc \quotep{P}}
\end{mathpar} 

Note that $\vec{x}$ (resp. $\vec{P}$) denotes a vector of names
(resp. processes) of length $|\vec{x}|$ (resp. $|\vec{P}|$). We adopt
the following useful abbreviations.

\begin{mathpar}
   x?(\vec{y}).P := x.(\vec{y})P \and  x\clift{\vec{P}} := x.\clift{\vec{P}}
   \and x!(y) := \lift{x}{\dropn{y}}
   \and \Pi_{i=0}^{n-1}P_i := P_0 | \ldots | P_{n-1}
\end{mathpar}

\subsubsection{Structural congruence}

\paragraph{Free and bound names and alpha-equivalence.} At the
core of structural equivalence is alpha-equivalence which identifies
process that are the same up to a change of variable. Formally, we
recognize the distinction between free and bound names. The free names
of a process, $\freenames{P}$, may be calculated recursively as
follows:

\begin{mathpar}
\freenames{\pzero} := \emptyset
  \and \\
  \freenames{x?(y).P} := \{ x \} \cup (\freenames{P} \setminus \{ y \})
  \and 
  \freenames{x!\langle P \rangle} := \{ x \} \cup \{ P \} 
  \and \\
  \freenames{P|Q} := \freenames{P} \cup \freenames{Q}
  \and \\
  \freenames{@{x}} := \{ x \}
\end{mathpar}

$\pi$
$\quotep{\pi}$

$\freenames{-} : \pi \to \mathcal{P}(\quotep{\pi})$

\begin{eqnarray*}
  \freenames{\pzero} & := & \emptyset \\
  \freenames{x?(y).P} & := & \{ x \} \cup (\freenames{P} \setminus \{ y \}) \\
  \freenames{x!\langle P \rangle} & := & \{ x \} \cup \{ P \} \\
  \freenames{P|Q} & := & \freenames{P} \cup \freenames{Q} \\
  \freenames{\dropn{x}} & := & \{ x \}
\end{eqnarray*}

The bound names of a process, $\boundnames{P}$, are those names occurring in $P$
that are not free. For example, in $x?(y).0$, the name $x$ is free, while $y$ is bound.

\begin{mathpar}
  \inferrule* [lab=monoidal-laws] {} { P|Q \equiv Q|P \and P|0 \equiv P \and P|(Q|R) \equiv (P|Q)|R }
\end{mathpar}

\begin{mathpar}
  \inferrule* [lab=alpha-equivalence] {} { (x)P \equiv (y)P\{y/x\} \and y \not\in \freenames{P} }
\end{mathpar}

\begin{definition}
Then two processes, $P,Q$, are alpha-equivalent if $P = Q\{\vec{y}/\vec{x}\}$ for
some $\vec{x} \in \boundnames{Q},\vec{y} \in \boundnames{P}$, where $Q\{\vec{y}/\vec{x}\}$
denotes the capture-avoiding substitution of $\vec{y}$ for $\vec{x}$ in $Q$.
\end{definition}

\begin{definition}
  The {\em structural congruence} \cite{SangiorgiWalker} , $\equiv$,
  between processes is the least congruence containing
  alpha-equivalence, satisfying the abelian monoid laws
  (associativity, commutativity and $\pzero$ as identity) for parallel
  composition $|$ and for summation $+$.
\end{definition}

\subsection{Name equivalence}

We take name equivalence, written $\nameeq$, to be the smallest
equivalence relation generated by the following rules.

\begin{mathpar}
\inferrule*[lab=Quote-drop]
{ }
{ \quotep{@{x}} \nameeq x }

\inferrule*[lab=Struct-equiv]
{ P \scong Q }
{ \quotep{P} \nameeq \quotep{Q} }
\end{mathpar}

The astute reader will have noticed that the mutual recursion of names
and processes imposes a mutual recursion on alpha-equivalence and
structural equivalence via name-equivalence. Fortunately, all of this
works out pleasantly and we may calculate in the natural way, free of
concern. The reader interested in the details is referred to the
appendix \ref{appendix:rho_details}.

\subsection{Substitution}

We use $\Proc$ for the set of processes, $\QProc$ for the set of
names, and $\id{\{}\vec{y} / \vec{x} \id{\}}$ to denote partial maps,
$s : \QProc \rightarrow \QProc$. A map, $s$ lifts, uniquely, to a map
on process terms, $\widehat{s} : \Proc \rightarrow \Proc$ by the
following equations.

\begin{mathpar}
  (0) \psubstp{Q}{P} := 0 \\
  (R \juxtap S) \psubstp{Q}{P}
  :=    
  (R)\psubstp{Q}{P} \juxtap (S) \psubstp{Q}{P} \\
  (x?(y).R) \psubstp{Q}{P}    
  :=    
  (x)\substp{Q}{P} (z)\concat( (R \psubstn{z}{y}) \psubstp{Q}{P} ) \\
  (\lift{x}{R}) \psubstp{Q}{P}  
  :=
  \lift{(x)\substp{Q}{P}}{ R \psubstp{Q}{P} } \\
%   (\dropn{x})  \psubstp{Q}{P}       
%   := 
%   \left\{ 
%     \begin{array}{ccc} 
%       \dropn{\quotep{Q}} & & x \nameeq \quotep{P} \\
%       \dropn{x} & & otherwise \\
%     \end{array}
%   \right. 
  (\dropn{x})  \psubstp{Q}{P}       
  := 
  \left\{ 
    \begin{array}{ccc} 
      Q & & x \nameeq \quotep{P} \\
      \dropn{x} & & otherwise \\
    \end{array}
  \right.
\end{mathpar}
 

where

\begin{eqnarray}
  (x)\id{\{} \lpquote Q \rpquote / \lpquote P \rpquote \id{\}}            = 
  \left\{ 
    \begin{array}{ccc}
      \lpquote Q \rpquote & & x \nameeq \lpquote P \rpquote \\
      x & & otherwise \\
    \end{array}
  \right. \nonumber
\end{eqnarray}

and $z$ is chosen distinct from $\quotep{P}$, $\quotep{Q}$, the free
names in $Q$, and all the names in $R$. Our $\alpha$-equivalence will
be built in the standard way from this substitution.

\begin{remark}\label{rem:no_self_referential_names}
  One consequence of these definitions is that $\forall P. \quotep{P}
  \not\in \freenames{P}$.
\end{remark}

\subsection{ Dynamic quote: an example }

Anticipating something of what's to come, consider applying the
substitution, $\widehat{\id{\{}u / z \id{\}}}$, to the following pair
of processes, $\lift{w}{y!(z)}$ and $w[ \lpquote y!(z) \rpquote ]$.

\begin{eqnarray}
	\lift{w}{y!(z)}\widehat{\id{\{}u / z \id{\}}}
		& = &
		\lift{w}{y!(u)} \nonumber\\
	w[ \lpquote y!(z) \rpquote ] \widehat{ \id{\{}u / z \id{\}} }
		& = &
		w[ \lpquote y!(z) \rpquote ] \nonumber
\end{eqnarray}

Because the body of the process between quotes is impervious to
substitution, we get radically different answers. In fact, by
examining the first process in an input context,
e.g. $x?(z).\lift{w}{y!(z)}$, we see that the process under the lift
operator may be shaped by prefixed inputs binding a name inside it. In
this sense, the lift operator will be seen as a way to dynamically
construct processes before reifying them as names.

Finally equipped with these standard features we can present the
dynamics of the calculus.

\subsubsection{Operational semantics} 

Finally, we introduce the computational dynamics. What marks these
algebras as distinct from other more traditionally studied algebraic
structures, e.g. vector spaces or polynomial rings, is the manner in
which dynamics is captured. In traditional structures, dynamics is typically
expressed through morphisms between such structures, as in linear maps
between vector spaces or morphisms between rings. In algebras
associated with the semantics of computation, the dynamics is
expressed as part of the algebraic structure itself, through a
reduction reduction relation typically denoted by $\red$. Below, we
give a recursive presentation of this relation for the calculus used
in the encoding.

$\red \subseteq \pi \times \pi$
$\red : \pi \to \mathcal{P}(\pi)$

\begin{mathpar}
  \inferrule* [lab=Comm] { \textsf{match}( x_{src}, x_{trgt} ) } { x_{trgt}?(y)P \; | \; x_{src}!\langle {Q} \rangle \red P\{\quotep{Q}/y}\} }
  \and \\
  \inferrule* [lab=Par] {{P} \red {P}'} {{{P} | {Q}} \red {{P}' | {Q}}}
  \and
  \inferrule* [lab=Equiv]{{{P} \scong {P}'} \andalso {{P}' \red {Q}'} \andalso {{Q}' \scong {Q}}}{{P} \red {Q}}
\end{mathpar}

\begin{eqnarray*}
  match_{\equiv} (\quotep{P},\quotep{Q}) & := & P \equiv Q \\
  match_{\dagger}(\quotep{P},\quotep{Q}) & := & \forall R. P|Q \red^{*} R => R \red^{*} 0 \\
  match_{K}(\quotep{P},\quotep{Q}) & := & K \mbox{ for some context } K
\end{eqnarray*}

$u?(x)P | u!\langle Q \rangle \red P\{\quotep{Q}/x\}$

%We write $\wred$ for $\red^*$, and $P\red$ if $\exists Q $ such that $ P \red Q$.
We write $P\red$ if $\exists Q $ such that $ P \red Q$ and $P\not\red$, otherwise.

\section{Replication}

As mentioned before, it is known that replication (and hence
recursion) can be implemented in a higher-order process algebra
\cite{SangiorgiWalker}. As our first example of calculation with the
machinery thus far presented we give the construction explicitly in
the {\rhoc}.

\begin{eqnarray}
	D_{x} & := & \prefix{x}{y}{(\binpar{\outputp{x}{y}}{@{y}})} \nonumber\\
	\bangp_{x}{P} & := & \binpar{{x}!\langle{\binpar{D_{x}}{P}}\rangle}{D_{x}} \nonumber
\end{eqnarray}

\begin{eqnarray}
	\bangp_{x}{P} & & \nonumber\\
	=
	& {x}!\langle{(\prefix{x}{y}{(\outputp{x}{y} | @{y})) | P}}\rangle 
	      | \prefix{x}{y}{(\outputp{x}{y} | @{y})} & \nonumber\\
	\red
	& (\outputp{x}{y} | @{y})\substn{\quotep{(\prefix{x}{y}{(@{y} | \outputp{x}{y})) | P}}}{y} & \nonumber\\
	=
	& \outputp{x}{\quotep{(\prefix{x}{y}{(\outputp{x}{y} | @{y})) | P}}}
	  | {(\prefix{x}{y}{(\outputp{x}{y} | @{y})) | P}} & \nonumber\\
	\red
	& \ldots & \nonumber\\
	\red^*
	& P | P | \ldots & \nonumber
\end{eqnarray}

Of course, this encoding, as an implementation, runs away, unfolding
$\bangp{P}$ eagerly. A lazier and more implementable replication
operator, restricted to input-guarded processes, may be obtained as follows.

\begin{eqnarray}
\bangp{\prefix{u}{v}{P}} 
	:= 
	\binpar{\lift{x}{\prefix{u}{v}{(\binpar{D(x)}{P})}}}{D(x)} \nonumber
\end{eqnarray}

\begin{remark}
  Note that the lazier definition still does not deal with summation
  or mixed summation (i.e. sums over input and output). The reader is
  invited to construct definitions of replication that deal with these
  features. 

  Further, the definitions are parameterized in a name, $x$. Can you,
  gentle reader, make a definition that eliminates this parameter and
  guarantees no accidental interaction between the replication
  machinery and the process being replicated -- i.e. no accidental
  sharing of names used by the process to get its work done and the
  name(s) used by the replication to effect copying. This latter
  revision of the definition of replication is crucial to obtaining
  the expected identity $!!P \sim !P$.
\end{remark}

\begin{remark}\label{rem:paradoxical_combinator}
  The reader familiar with the lambda calculus will have noticed the
  similarity between $D$ and the paradoxical combinator.

  [Ed. note: the existence of this seems to suggest we have to be more
  restrictive on the set of processes and names we admit if we are to
  support no-cloning.]
\end{remark}

\subsubsection{Bisimulation}

The computational dynamics gives rise to another kind of equivalence,
the equivalence of computational behavior. As previously mentioned
this is typically captured \emph{via} some form of bisimulation.

% The notion we use in this paper is weak barbed bisimulation
% \cite{milner91polyadicpi}.

The notion we use in this paper is derived from weak barbed
bisimulation \cite{milner91polyadicpi}. 

\begin{definition}
An \emph{observation relation}, $\downarrow_{\mathcal N}$, over a set
of names, $\mathcal N$, is the smallest relation satisfying the rules
below.

\infrule[Out-barb]{y \in {\mathcal N}, \; x \nameeq y}
		  {\outputp{x}{v} \downarrow_{\mathcal N} x}
\infrule[Par-barb]{\mbox{$P\downarrow_{\mathcal N} x$ or $Q\downarrow_{\mathcal N} x$}}
		  {\binpar{P}{Q} \downarrow_{\mathcal N} x}

We write $P \Downarrow_{\mathcal N} x$ if there is $Q$ such that 
$P \wred Q$ and $Q \downarrow_{\mathcal N} x$.
\end{definition}

\begin{definition}
%\label{def.bbisim}
An  ${\mathcal N}$-\emph{barbed bisimulation} over a set of names, ${\mathcal N}$, is a symmetric binary relation 
${\mathcal S}_{\mathcal N}$ between agents such that $P\rel{S}_{\mathcal N}Q$ implies:
\begin{enumerate}
\item If $P \red P'$ then $Q \wred Q'$ and $P'\rel{S}_{\mathcal N} Q'$.
\item If $P\downarrow_{\mathcal N} x$, then $Q\Downarrow_{\mathcal N} x$.
\end{enumerate}
$P$ is ${\mathcal N}$-barbed bisimilar to $Q$, written
$P \wbbisim_{\mathcal N} Q$, if $P \rel{S}_{\mathcal N} Q$ for some ${\mathcal N}$-barbed bisimulation ${\mathcal S}_{\mathcal N}$.
\end{definition}

$\mathcal{R} \subseteq \pi \times \pi$

$P \mathcal{R} Q => \forall P'. P \red P' \Rightarrow \exists Q'. Q \red Q', P' \mathcal{R} Q'$

$P \vdash x \Rightarrow Q \vdash x$

\begin{mathpar}
  \inferrule*[lab=Out-barb]{x \nameeq y}{{y}!\langle{Q}\rangle \vdash x}
  \and
  \inferrule*[lab=Par-barb]{\mbox{$P\vdash x$ or $Q\vdash x$}}{\binpar{P}{Q} \vdash x}
\end{mathpar}

\subsubsection{Contexts}

One of the principle advantages of computational calculi like the
$\pi$-calculus is a well-defined notion of context,
contextual-equivalence and a correlation between
contextual-equivalence and notions of bisimulation. The notion of
context allows the decomposition of a process into (sub-)process and
its syntactic environment, its context. Thus, a context may be
thought of as a process with a ``hole'' (written $\Box$) in it. The
application of a context $M$ to a process $P$, written $M[P]$, is
tantamount to filling the hole in $M$ with $P$. In this paper we do
not need the full weight of this theory, but do make use of the notion
of context in the proof the main theorem. 

\begin{mathpar}
  \inferrule* [lab=summation] {} {{M_{M},M_{N}} \bc \Box \;|\; x.M_{A} \;|\; M_{M}+M_{N}}
  \and
  \inferrule* [lab=agent] {} {{M_{A}} \bc (\vec{x})M_{P} \;| \; \clift{P_0,\ldots,M_{P},\ldots,P_N}}
  \and \\
  \inferrule* [lab=process] {} {{M_{P}} \bc M_{N} \;| \;P|M_{P} }
\end{mathpar} 

\begin{mathpar}
  \inferrule* [lab=sychronization] {} {M_{N} \bc \Box \;|\; x?M_{F} \;|\; x!M_{C}}
  \and
  \inferrule* [lab=abstraction] {} {{M_{F}} \bc (x)M_{P} }
  \and
  \inferrule* [lab=concretion] {} {{M_{C}} \bc \langle M_{P} \rangle }
  \and \\
  \inferrule* [lab=process] {} {{M_{P}} \bc M_{N} \;| \;P|M_{P} }
\end{mathpar}

\begin{definition}[contextual application] Given a context $M$, and
  process $P$, we define the \emph{contextual application}, $M[P] :=
  M\{P/\Box\}$. That is, the contextual application of M to P is the
  substitution of $P$ for $\Box$ in $M$.
\end{definition}

$\meaningof{-} : L \to \mathcal{P}(\pi)$

\begin{mathpar}
  \inferrule* [lab=collection] {} {\meaningof{true} = \pi, \and \meaningof{~E} = \pi \setminus \meaningof{E}, \and \meaningof{E_{1} \& E_{2}} = \meaningof{E_{1}} \cap \meaningof{E_{2}}}
\end{mathpar}

\begin{mathpar}
  \inferrule* [lab=structure] {} {\meaningof{0} = \{ P \in \pi | P \equiv 0 \}, \and \\ \meaningof{E_1 | E_2} = \{ P \in \pi | P \equiv P_{1} | P_{2}, P_{1} \in \meaningof{E_{1}}, P_{2} \in \meaningof{E_2}\} }
\end{mathpar}

\begin{mathpar}
 \inferrule* [lab=behavior] {} {\meaningof{\langle a?b \rangle E} = \{ P \in \pi | P \equiv Q | u?(y)P', \\ \and \\\\ \and \\ \;\;\; u \in \meaningof{a}, \forall z.P'\{z/y\} \in \meaningof{E\{z/b\}}\}, \and \\ \meaningof{a!E} = \{ P \in \pi | P \equiv Q | x!\langle P' \rangle, x \in \meaningof{a} P' \in \meaningof{E}\} }
\end{mathpar}

\begin{mathpar}
 \inferrule* [lab=nominal] {} {\meaningof{\quotep{E}} = \{ \quotep{P} \in \quotep{\pi} | P \in \meaningof{E} \}, \and \meaningof{\quotep{P}} = \{ \quotep{Q} \in \quotep{\pi} | P \equiv Q \} \and \\ \meaningof{@\quotep{E}} = \{ P \in \pi | P \equiv @x, x \in \meaningof{E} \}}
\end{mathpar}

\begin{eqnarray*}
  \\
  \meaningof{-} : TS \to ST
\end{eqnarray*}

\begin{eqnarray*}
  \\
  L : TS \to ST
\end{eqnarray*}

\begin{eqnarray*}
  \\
  P \models E \iff P \in \meaningof{E}
\end{eqnarray*}

\begin{eqnarray*}
  P \approx_{L} Q \iff \forall E \in L. P \models E \iff Q \models E
\end{eqnarray*}

\begin{eqnarray*}
  P \approx_{K} Q
\end{eqnarray*}

\begin{eqnarray*}
  P \approx Q
\end{eqnarray*}

$\approx_{K} = \approx = \approx_{L}$

\subsubsection{Contextual duality}

Note that contexts extend the quotation operation to a family of
operations from processes to names. Given a context, $M$, we can
define a \emph{nominal context}, $\quotep{M}$ by $\quotep{M}[P] :=
\quotep{M[P]}$. To foreshadow what is to come we observe that these
operations enjoy a duality with processes very much like the duality
between vectors and maps from vectors to scalars.

Further, because the calculus is essentially higher-order, we have a
correspondence between contexts and processes. More specifically,
given a name $x$ and a context $M$ we can construct $M^{*}_{x}$ such
that 

\begin{mathpar}
  M^{*}_{x} | \lift{x}{P} \red M[P]
\end{mathpar}

namely,

\begin{mathpar}
  M^{*}_{x} := x?(u).M[\dropn{u}]
\end{mathpar}

The dependence of $M^{*}_{x}$ on a name makes it an abstraction, 

\begin{mathpar}
  M^{*} := (x)x?(u).M[\dropn{u}]
\end{mathpar}

\subsection{Additional notation}

It will sometimes be convenient to denote the process a name
quotes. We already have the notation $x = \quotep{P}$, but it will be
convenient to introduce an alternate notation, $\procn{x}$, when we
want to emphasize the connection to the use of the name. Note that, by
virtue of name equivalence, $\quotep{\procn{x}} \nameeq x$; so, the
notation is consistent with previous definitions.

Further, because names have structure it is possible to effect
substitutions on the basis of that structure. This means we need to
upgrade our notation for substitutions, which we accomplish by
adapting comprehension notation. Thus,

\begin{mathpar}
  P\{ y / x : x \in S \}
\end{mathpar}

is interpreted to mean the process derived from P by replacing (in a
capture-avoiding manner) each occurrence of $x$ in $S$ by $y$. For example,

\begin{mathpar}
  P\{ \quotep{\procn{x}|\procn{x}} / x : x \in \freenames{P} \}
\end{mathpar}

will replace each (occurrence) of a free name $x$ in $P$ by
$\quotep{\procn{x}|\procn{x}}$.

Also, we will avail ourselves of the notation $x^{L}$ and $x^{R}$ to
denote injections of a name into disjoint copies of the name
space. There are numerous ways to accomplish this. One example can be
found in \cite{MeredithR05}. This notation overloads to vectors of
names: $\vec{x}^{\pi} := (x_{i}^{\pi} \; : \; 0 \leq i < |\vec{x}| )$ where $\pi \in \{L,R\}$.

We also use $P^{\Box} := P|\Box$.

In \cite{MeredithR05} an interpretation of the new operator is
given. It turns out that there are several possible interpretations
all enjoying the requisite algebraic properties of the operator (see
\cite{milner91polyadicpi}). We will therefore make liberal use of
$(\nu\; \vec{x})P$.

% subsection the_syntax_and_semantics_of_the_notation_system (end)   

\input{qm2pi.qmops} 

\input{qm2pi.sterngerlach} 

\input{qm2pi.metric} 

% section concurrent_process_calculi (end)

%\input{qm2pi.proofsketch}

% section proof sketch (end)

%\input{qm2pi.slviaknots} 

% section spatial logic via knots (end)

\input{qm2pi.conclusion}

% section conclusion (end)

%\input{qm2pi.dtcodes} 

% section wiring algorithm (end)

\input{qm2pi.ack} 

% section acknowledgments (end)

\newpage


\bibliographystyle{plain}   
\bibliography{../../biblios/main.bib}

\input{qm2pi.rhodetails}

\end{document}



\end{document}

 

% section notation (end)

\input{qm2pi.process.calculi} 

% section concurrent_process_calculi_and_spatial_logics_ (end)
    
%\documentclass[12pt]{llncs}
%\documentclass{jktr}

\usepackage[pdftex]{hyperref}                   
\usepackage {listings}
\usepackage {mathpartir}
\usepackage{bcprules}
%\usepackage{listings}
                       
\usepackage{graphicx} 
%\usepackage[margins=2.5cm,nohead,nofoot]{geometry}
%\usepackage{geometry}
\usepackage{amsfonts}
\usepackage{amstext}
\usepackage{latexsym}
\usepackage{amssymb}
\usepackage{color}


%\include{myPreamble}
\documentclass[12pt]{llncs}
%\documentclass{jktr}

\usepackage[pdftex]{hyperref}                   
\usepackage {listings}
\usepackage {mathpartir}
\usepackage{bcprules}
%\usepackage{listings}
                       
\usepackage{graphicx} 
%\usepackage[margins=2.5cm,nohead,nofoot]{geometry}
%\usepackage{geometry}
\usepackage{amsfonts}
\usepackage{amstext}
\usepackage{latexsym}
\usepackage{amssymb}
\usepackage{color}


%\include{myPreamble}
\include{qm2pi.local} 

%\ifpdf
%\usepackage[pdftex]{graphicx}
%\else
%\usepackage{graphicx}
%\fi

 % \ifpdf
%  \usepackage{pdfsync}
%  \if


%\title{Brief Article}
%\author{David F. Snyder}
%\author{L.G. Meredith}

%\address{Dept. of Math., Texas State University--San Marcos, San Marcos, TX 78666}
       
\pagestyle{empty}


\begin{document}

\lstset{language=[Objective]Caml,frame=shadowbox}

\input{qm2pi.front}

% section front matter (end)

\input{qm2pi.intro} 
 
% section introduction (end)

% \input{qm2pi.knotations} 

% section notation (end)

\input{qm2pi.process.calculi} 

% section concurrent_process_calculi_and_spatial_logics_ (end)
    
%\input{qm2pi.knots2pi} 

%\input{qm2pi.trefoil} 

%\input{qm2pi.mainthm} 

% subsection basic_interpretation (end)

%\input{qm2pi.rho.presentation} 
\subsection{The syntax and semantics of the notation system}\label{sub:the_syntax_and_semantics_of_the_notation_system} % (fold)

We now summarize a technical presentation of the calculus that
embodies our theory of dynamics. The typical presentation of such a
calculus follows the style of giving generators and relations on
them. The grammar, below, describing term constructors, freely
generates the set of processes, $\Proc$. This set is then quotiented
by a relation known as structural congruence and it is over this set
that the notion of dynamics is expressed. This presentation is
essentially that of \cite{MeredithR05} with the addition of
polyadicity and summation. For readability we have relegated some of
the technical subtleties to an appendix.

\subsubsection{Process grammar}\label{subsub:process_grammar}

\begin{mathpar}
  \inferrule* [lab=synchronization] {} {{M} \bc \pzero \;|\; x?F \;|\; x!C }
  \and
  \inferrule* [lab=abstraction] {} {{F} \bc (x)P}
  \and
  \inferrule* [lab=concretion] {} {{C} \bc \langle Q \rangle}
  \and
  \inferrule* [lab=process] {} {{P,Q} \bc M \;| \;P|Q \;|\; @{x}}
  \and
  \inferrule* [lab=name] {} {{x} \bc \quotep{P}}
\end{mathpar} 

Note that $\vec{x}$ (resp. $\vec{P}$) denotes a vector of names
(resp. processes) of length $|\vec{x}|$ (resp. $|\vec{P}|$). We adopt
the following useful abbreviations.

\begin{mathpar}
   x?(\vec{y}).P := x.(\vec{y})P \and  x\clift{\vec{P}} := x.\clift{\vec{P}}
   \and x!(y) := \lift{x}{\dropn{y}}
   \and \Pi_{i=0}^{n-1}P_i := P_0 | \ldots | P_{n-1}
\end{mathpar}

\subsubsection{Structural congruence}

\paragraph{Free and bound names and alpha-equivalence.} At the
core of structural equivalence is alpha-equivalence which identifies
process that are the same up to a change of variable. Formally, we
recognize the distinction between free and bound names. The free names
of a process, $\freenames{P}$, may be calculated recursively as
follows:

\begin{mathpar}
\freenames{\pzero} := \emptyset
  \and \\
  \freenames{x?(y).P} := \{ x \} \cup (\freenames{P} \setminus \{ y \})
  \and 
  \freenames{x!\langle P \rangle} := \{ x \} \cup \{ P \} 
  \and \\
  \freenames{P|Q} := \freenames{P} \cup \freenames{Q}
  \and \\
  \freenames{@{x}} := \{ x \}
\end{mathpar}

$\pi$
$\quotep{\pi}$

$\freenames{-} : \pi \to \mathcal{P}(\quotep{\pi})$

\begin{eqnarray*}
  \freenames{\pzero} & := & \emptyset \\
  \freenames{x?(y).P} & := & \{ x \} \cup (\freenames{P} \setminus \{ y \}) \\
  \freenames{x!\langle P \rangle} & := & \{ x \} \cup \{ P \} \\
  \freenames{P|Q} & := & \freenames{P} \cup \freenames{Q} \\
  \freenames{\dropn{x}} & := & \{ x \}
\end{eqnarray*}

The bound names of a process, $\boundnames{P}$, are those names occurring in $P$
that are not free. For example, in $x?(y).0$, the name $x$ is free, while $y$ is bound.

\begin{mathpar}
  \inferrule* [lab=monoidal-laws] {} { P|Q \equiv Q|P \and P|0 \equiv P \and P|(Q|R) \equiv (P|Q)|R }
\end{mathpar}

\begin{mathpar}
  \inferrule* [lab=alpha-equivalence] {} { (x)P \equiv (y)P\{y/x\} \and y \not\in \freenames{P} }
\end{mathpar}

\begin{definition}
Then two processes, $P,Q$, are alpha-equivalent if $P = Q\{\vec{y}/\vec{x}\}$ for
some $\vec{x} \in \boundnames{Q},\vec{y} \in \boundnames{P}$, where $Q\{\vec{y}/\vec{x}\}$
denotes the capture-avoiding substitution of $\vec{y}$ for $\vec{x}$ in $Q$.
\end{definition}

\begin{definition}
  The {\em structural congruence} \cite{SangiorgiWalker} , $\equiv$,
  between processes is the least congruence containing
  alpha-equivalence, satisfying the abelian monoid laws
  (associativity, commutativity and $\pzero$ as identity) for parallel
  composition $|$ and for summation $+$.
\end{definition}

\subsection{Name equivalence}

We take name equivalence, written $\nameeq$, to be the smallest
equivalence relation generated by the following rules.

\begin{mathpar}
\inferrule*[lab=Quote-drop]
{ }
{ \quotep{@{x}} \nameeq x }

\inferrule*[lab=Struct-equiv]
{ P \scong Q }
{ \quotep{P} \nameeq \quotep{Q} }
\end{mathpar}

The astute reader will have noticed that the mutual recursion of names
and processes imposes a mutual recursion on alpha-equivalence and
structural equivalence via name-equivalence. Fortunately, all of this
works out pleasantly and we may calculate in the natural way, free of
concern. The reader interested in the details is referred to the
appendix \ref{appendix:rho_details}.

\subsection{Substitution}

We use $\Proc$ for the set of processes, $\QProc$ for the set of
names, and $\id{\{}\vec{y} / \vec{x} \id{\}}$ to denote partial maps,
$s : \QProc \rightarrow \QProc$. A map, $s$ lifts, uniquely, to a map
on process terms, $\widehat{s} : \Proc \rightarrow \Proc$ by the
following equations.

\begin{mathpar}
  (0) \psubstp{Q}{P} := 0 \\
  (R \juxtap S) \psubstp{Q}{P}
  :=    
  (R)\psubstp{Q}{P} \juxtap (S) \psubstp{Q}{P} \\
  (x?(y).R) \psubstp{Q}{P}    
  :=    
  (x)\substp{Q}{P} (z)\concat( (R \psubstn{z}{y}) \psubstp{Q}{P} ) \\
  (\lift{x}{R}) \psubstp{Q}{P}  
  :=
  \lift{(x)\substp{Q}{P}}{ R \psubstp{Q}{P} } \\
%   (\dropn{x})  \psubstp{Q}{P}       
%   := 
%   \left\{ 
%     \begin{array}{ccc} 
%       \dropn{\quotep{Q}} & & x \nameeq \quotep{P} \\
%       \dropn{x} & & otherwise \\
%     \end{array}
%   \right. 
  (\dropn{x})  \psubstp{Q}{P}       
  := 
  \left\{ 
    \begin{array}{ccc} 
      Q & & x \nameeq \quotep{P} \\
      \dropn{x} & & otherwise \\
    \end{array}
  \right.
\end{mathpar}
 

where

\begin{eqnarray}
  (x)\id{\{} \lpquote Q \rpquote / \lpquote P \rpquote \id{\}}            = 
  \left\{ 
    \begin{array}{ccc}
      \lpquote Q \rpquote & & x \nameeq \lpquote P \rpquote \\
      x & & otherwise \\
    \end{array}
  \right. \nonumber
\end{eqnarray}

and $z$ is chosen distinct from $\quotep{P}$, $\quotep{Q}$, the free
names in $Q$, and all the names in $R$. Our $\alpha$-equivalence will
be built in the standard way from this substitution.

\begin{remark}\label{rem:no_self_referential_names}
  One consequence of these definitions is that $\forall P. \quotep{P}
  \not\in \freenames{P}$.
\end{remark}

\subsection{ Dynamic quote: an example }

Anticipating something of what's to come, consider applying the
substitution, $\widehat{\id{\{}u / z \id{\}}}$, to the following pair
of processes, $\lift{w}{y!(z)}$ and $w[ \lpquote y!(z) \rpquote ]$.

\begin{eqnarray}
	\lift{w}{y!(z)}\widehat{\id{\{}u / z \id{\}}}
		& = &
		\lift{w}{y!(u)} \nonumber\\
	w[ \lpquote y!(z) \rpquote ] \widehat{ \id{\{}u / z \id{\}} }
		& = &
		w[ \lpquote y!(z) \rpquote ] \nonumber
\end{eqnarray}

Because the body of the process between quotes is impervious to
substitution, we get radically different answers. In fact, by
examining the first process in an input context,
e.g. $x?(z).\lift{w}{y!(z)}$, we see that the process under the lift
operator may be shaped by prefixed inputs binding a name inside it. In
this sense, the lift operator will be seen as a way to dynamically
construct processes before reifying them as names.

Finally equipped with these standard features we can present the
dynamics of the calculus.

\subsubsection{Operational semantics} 

Finally, we introduce the computational dynamics. What marks these
algebras as distinct from other more traditionally studied algebraic
structures, e.g. vector spaces or polynomial rings, is the manner in
which dynamics is captured. In traditional structures, dynamics is typically
expressed through morphisms between such structures, as in linear maps
between vector spaces or morphisms between rings. In algebras
associated with the semantics of computation, the dynamics is
expressed as part of the algebraic structure itself, through a
reduction reduction relation typically denoted by $\red$. Below, we
give a recursive presentation of this relation for the calculus used
in the encoding.

$\red \subseteq \pi \times \pi$
$\red : \pi \to \mathcal{P}(\pi)$

\begin{mathpar}
  \inferrule* [lab=Comm] { \textsf{match}( x_{src}, x_{trgt} ) } { x_{trgt}?(y)P \; | \; x_{src}!\langle {Q} \rangle \red P\{\quotep{Q}/y}\} }
  \and \\
  \inferrule* [lab=Par] {{P} \red {P}'} {{{P} | {Q}} \red {{P}' | {Q}}}
  \and
  \inferrule* [lab=Equiv]{{{P} \scong {P}'} \andalso {{P}' \red {Q}'} \andalso {{Q}' \scong {Q}}}{{P} \red {Q}}
\end{mathpar}

\begin{eqnarray*}
  match_{\equiv} (\quotep{P},\quotep{Q}) & := & P \equiv Q \\
  match_{\dagger}(\quotep{P},\quotep{Q}) & := & \forall R. P|Q \red^{*} R => R \red^{*} 0 \\
  match_{K}(\quotep{P},\quotep{Q}) & := & K \mbox{ for some context } K
\end{eqnarray*}

$u?(x)P | u!\langle Q \rangle \red P\{\quotep{Q}/x\}$

%We write $\wred$ for $\red^*$, and $P\red$ if $\exists Q $ such that $ P \red Q$.
We write $P\red$ if $\exists Q $ such that $ P \red Q$ and $P\not\red$, otherwise.

\section{Replication}

As mentioned before, it is known that replication (and hence
recursion) can be implemented in a higher-order process algebra
\cite{SangiorgiWalker}. As our first example of calculation with the
machinery thus far presented we give the construction explicitly in
the {\rhoc}.

\begin{eqnarray}
	D_{x} & := & \prefix{x}{y}{(\binpar{\outputp{x}{y}}{@{y}})} \nonumber\\
	\bangp_{x}{P} & := & \binpar{{x}!\langle{\binpar{D_{x}}{P}}\rangle}{D_{x}} \nonumber
\end{eqnarray}

\begin{eqnarray}
	\bangp_{x}{P} & & \nonumber\\
	=
	& {x}!\langle{(\prefix{x}{y}{(\outputp{x}{y} | @{y})) | P}}\rangle 
	      | \prefix{x}{y}{(\outputp{x}{y} | @{y})} & \nonumber\\
	\red
	& (\outputp{x}{y} | @{y})\substn{\quotep{(\prefix{x}{y}{(@{y} | \outputp{x}{y})) | P}}}{y} & \nonumber\\
	=
	& \outputp{x}{\quotep{(\prefix{x}{y}{(\outputp{x}{y} | @{y})) | P}}}
	  | {(\prefix{x}{y}{(\outputp{x}{y} | @{y})) | P}} & \nonumber\\
	\red
	& \ldots & \nonumber\\
	\red^*
	& P | P | \ldots & \nonumber
\end{eqnarray}

Of course, this encoding, as an implementation, runs away, unfolding
$\bangp{P}$ eagerly. A lazier and more implementable replication
operator, restricted to input-guarded processes, may be obtained as follows.

\begin{eqnarray}
\bangp{\prefix{u}{v}{P}} 
	:= 
	\binpar{\lift{x}{\prefix{u}{v}{(\binpar{D(x)}{P})}}}{D(x)} \nonumber
\end{eqnarray}

\begin{remark}
  Note that the lazier definition still does not deal with summation
  or mixed summation (i.e. sums over input and output). The reader is
  invited to construct definitions of replication that deal with these
  features. 

  Further, the definitions are parameterized in a name, $x$. Can you,
  gentle reader, make a definition that eliminates this parameter and
  guarantees no accidental interaction between the replication
  machinery and the process being replicated -- i.e. no accidental
  sharing of names used by the process to get its work done and the
  name(s) used by the replication to effect copying. This latter
  revision of the definition of replication is crucial to obtaining
  the expected identity $!!P \sim !P$.
\end{remark}

\begin{remark}\label{rem:paradoxical_combinator}
  The reader familiar with the lambda calculus will have noticed the
  similarity between $D$ and the paradoxical combinator.

  [Ed. note: the existence of this seems to suggest we have to be more
  restrictive on the set of processes and names we admit if we are to
  support no-cloning.]
\end{remark}

\subsubsection{Bisimulation}

The computational dynamics gives rise to another kind of equivalence,
the equivalence of computational behavior. As previously mentioned
this is typically captured \emph{via} some form of bisimulation.

% The notion we use in this paper is weak barbed bisimulation
% \cite{milner91polyadicpi}.

The notion we use in this paper is derived from weak barbed
bisimulation \cite{milner91polyadicpi}. 

\begin{definition}
An \emph{observation relation}, $\downarrow_{\mathcal N}$, over a set
of names, $\mathcal N$, is the smallest relation satisfying the rules
below.

\infrule[Out-barb]{y \in {\mathcal N}, \; x \nameeq y}
		  {\outputp{x}{v} \downarrow_{\mathcal N} x}
\infrule[Par-barb]{\mbox{$P\downarrow_{\mathcal N} x$ or $Q\downarrow_{\mathcal N} x$}}
		  {\binpar{P}{Q} \downarrow_{\mathcal N} x}

We write $P \Downarrow_{\mathcal N} x$ if there is $Q$ such that 
$P \wred Q$ and $Q \downarrow_{\mathcal N} x$.
\end{definition}

\begin{definition}
%\label{def.bbisim}
An  ${\mathcal N}$-\emph{barbed bisimulation} over a set of names, ${\mathcal N}$, is a symmetric binary relation 
${\mathcal S}_{\mathcal N}$ between agents such that $P\rel{S}_{\mathcal N}Q$ implies:
\begin{enumerate}
\item If $P \red P'$ then $Q \wred Q'$ and $P'\rel{S}_{\mathcal N} Q'$.
\item If $P\downarrow_{\mathcal N} x$, then $Q\Downarrow_{\mathcal N} x$.
\end{enumerate}
$P$ is ${\mathcal N}$-barbed bisimilar to $Q$, written
$P \wbbisim_{\mathcal N} Q$, if $P \rel{S}_{\mathcal N} Q$ for some ${\mathcal N}$-barbed bisimulation ${\mathcal S}_{\mathcal N}$.
\end{definition}

$\mathcal{R} \subseteq \pi \times \pi$

$P \mathcal{R} Q => \forall P'. P \red P' \Rightarrow \exists Q'. Q \red Q', P' \mathcal{R} Q'$

$P \vdash x \Rightarrow Q \vdash x$

\begin{mathpar}
  \inferrule*[lab=Out-barb]{x \nameeq y}{{y}!\langle{Q}\rangle \vdash x}
  \and
  \inferrule*[lab=Par-barb]{\mbox{$P\vdash x$ or $Q\vdash x$}}{\binpar{P}{Q} \vdash x}
\end{mathpar}

\subsubsection{Contexts}

One of the principle advantages of computational calculi like the
$\pi$-calculus is a well-defined notion of context,
contextual-equivalence and a correlation between
contextual-equivalence and notions of bisimulation. The notion of
context allows the decomposition of a process into (sub-)process and
its syntactic environment, its context. Thus, a context may be
thought of as a process with a ``hole'' (written $\Box$) in it. The
application of a context $M$ to a process $P$, written $M[P]$, is
tantamount to filling the hole in $M$ with $P$. In this paper we do
not need the full weight of this theory, but do make use of the notion
of context in the proof the main theorem. 

\begin{mathpar}
  \inferrule* [lab=summation] {} {{M_{M},M_{N}} \bc \Box \;|\; x.M_{A} \;|\; M_{M}+M_{N}}
  \and
  \inferrule* [lab=agent] {} {{M_{A}} \bc (\vec{x})M_{P} \;| \; \clift{P_0,\ldots,M_{P},\ldots,P_N}}
  \and \\
  \inferrule* [lab=process] {} {{M_{P}} \bc M_{N} \;| \;P|M_{P} }
\end{mathpar} 

\begin{mathpar}
  \inferrule* [lab=sychronization] {} {M_{N} \bc \Box \;|\; x?M_{F} \;|\; x!M_{C}}
  \and
  \inferrule* [lab=abstraction] {} {{M_{F}} \bc (x)M_{P} }
  \and
  \inferrule* [lab=concretion] {} {{M_{C}} \bc \langle M_{P} \rangle }
  \and \\
  \inferrule* [lab=process] {} {{M_{P}} \bc M_{N} \;| \;P|M_{P} }
\end{mathpar}

\begin{definition}[contextual application] Given a context $M$, and
  process $P$, we define the \emph{contextual application}, $M[P] :=
  M\{P/\Box\}$. That is, the contextual application of M to P is the
  substitution of $P$ for $\Box$ in $M$.
\end{definition}

$\meaningof{-} : L \to \mathcal{P}(\pi)$

\begin{mathpar}
  \inferrule* [lab=collection] {} {\meaningof{true} = \pi, \and \meaningof{~E} = \pi \setminus \meaningof{E}, \and \meaningof{E_{1} \& E_{2}} = \meaningof{E_{1}} \cap \meaningof{E_{2}}}
\end{mathpar}

\begin{mathpar}
  \inferrule* [lab=structure] {} {\meaningof{0} = \{ P \in \pi | P \equiv 0 \}, \and \\ \meaningof{E_1 | E_2} = \{ P \in \pi | P \equiv P_{1} | P_{2}, P_{1} \in \meaningof{E_{1}}, P_{2} \in \meaningof{E_2}\} }
\end{mathpar}

\begin{mathpar}
 \inferrule* [lab=behavior] {} {\meaningof{\langle a?b \rangle E} = \{ P \in \pi | P \equiv Q | u?(y)P', \\ \and \\\\ \and \\ \;\;\; u \in \meaningof{a}, \forall z.P'\{z/y\} \in \meaningof{E\{z/b\}}\}, \and \\ \meaningof{a!E} = \{ P \in \pi | P \equiv Q | x!\langle P' \rangle, x \in \meaningof{a} P' \in \meaningof{E}\} }
\end{mathpar}

\begin{mathpar}
 \inferrule* [lab=nominal] {} {\meaningof{\quotep{E}} = \{ \quotep{P} \in \quotep{\pi} | P \in \meaningof{E} \}, \and \meaningof{\quotep{P}} = \{ \quotep{Q} \in \quotep{\pi} | P \equiv Q \} \and \\ \meaningof{@\quotep{E}} = \{ P \in \pi | P \equiv @x, x \in \meaningof{E} \}}
\end{mathpar}

\begin{eqnarray*}
  \\
  \meaningof{-} : TS \to ST
\end{eqnarray*}

\begin{eqnarray*}
  \\
  L : TS \to ST
\end{eqnarray*}

\begin{eqnarray*}
  \\
  P \models E \iff P \in \meaningof{E}
\end{eqnarray*}

\begin{eqnarray*}
  P \approx_{L} Q \iff \forall E \in L. P \models E \iff Q \models E
\end{eqnarray*}

\begin{eqnarray*}
  P \approx_{K} Q
\end{eqnarray*}

\begin{eqnarray*}
  P \approx Q
\end{eqnarray*}

$\approx_{K} = \approx = \approx_{L}$

\subsubsection{Contextual duality}

Note that contexts extend the quotation operation to a family of
operations from processes to names. Given a context, $M$, we can
define a \emph{nominal context}, $\quotep{M}$ by $\quotep{M}[P] :=
\quotep{M[P]}$. To foreshadow what is to come we observe that these
operations enjoy a duality with processes very much like the duality
between vectors and maps from vectors to scalars.

Further, because the calculus is essentially higher-order, we have a
correspondence between contexts and processes. More specifically,
given a name $x$ and a context $M$ we can construct $M^{*}_{x}$ such
that 

\begin{mathpar}
  M^{*}_{x} | \lift{x}{P} \red M[P]
\end{mathpar}

namely,

\begin{mathpar}
  M^{*}_{x} := x?(u).M[\dropn{u}]
\end{mathpar}

The dependence of $M^{*}_{x}$ on a name makes it an abstraction, 

\begin{mathpar}
  M^{*} := (x)x?(u).M[\dropn{u}]
\end{mathpar}

\subsection{Additional notation}

It will sometimes be convenient to denote the process a name
quotes. We already have the notation $x = \quotep{P}$, but it will be
convenient to introduce an alternate notation, $\procn{x}$, when we
want to emphasize the connection to the use of the name. Note that, by
virtue of name equivalence, $\quotep{\procn{x}} \nameeq x$; so, the
notation is consistent with previous definitions.

Further, because names have structure it is possible to effect
substitutions on the basis of that structure. This means we need to
upgrade our notation for substitutions, which we accomplish by
adapting comprehension notation. Thus,

\begin{mathpar}
  P\{ y / x : x \in S \}
\end{mathpar}

is interpreted to mean the process derived from P by replacing (in a
capture-avoiding manner) each occurrence of $x$ in $S$ by $y$. For example,

\begin{mathpar}
  P\{ \quotep{\procn{x}|\procn{x}} / x : x \in \freenames{P} \}
\end{mathpar}

will replace each (occurrence) of a free name $x$ in $P$ by
$\quotep{\procn{x}|\procn{x}}$.

Also, we will avail ourselves of the notation $x^{L}$ and $x^{R}$ to
denote injections of a name into disjoint copies of the name
space. There are numerous ways to accomplish this. One example can be
found in \cite{MeredithR05}. This notation overloads to vectors of
names: $\vec{x}^{\pi} := (x_{i}^{\pi} \; : \; 0 \leq i < |\vec{x}| )$ where $\pi \in \{L,R\}$.

We also use $P^{\Box} := P|\Box$.

In \cite{MeredithR05} an interpretation of the new operator is
given. It turns out that there are several possible interpretations
all enjoying the requisite algebraic properties of the operator (see
\cite{milner91polyadicpi}). We will therefore make liberal use of
$(\nu\; \vec{x})P$.

% subsection the_syntax_and_semantics_of_the_notation_system (end)   

\input{qm2pi.qmops} 

\input{qm2pi.sterngerlach} 

\input{qm2pi.metric} 

% section concurrent_process_calculi (end)

%\input{qm2pi.proofsketch}

% section proof sketch (end)

%\input{qm2pi.slviaknots} 

% section spatial logic via knots (end)

\input{qm2pi.conclusion}

% section conclusion (end)

%\input{qm2pi.dtcodes} 

% section wiring algorithm (end)

\input{qm2pi.ack} 

% section acknowledgments (end)

\newpage


\bibliographystyle{plain}   
\bibliography{../../biblios/main.bib}

\input{qm2pi.rhodetails}

\end{document}

 

%\ifpdf
%\usepackage[pdftex]{graphicx}
%\else
%\usepackage{graphicx}
%\fi

 % \ifpdf
%  \usepackage{pdfsync}
%  \if


%\title{Brief Article}
%\author{David F. Snyder}
%\author{L.G. Meredith}

%\address{Dept. of Math., Texas State University--San Marcos, San Marcos, TX 78666}
       
\pagestyle{empty}


\begin{document}

\lstset{language=[Objective]Caml,frame=shadowbox}

\documentclass[12pt]{llncs}
%\documentclass{jktr}

\usepackage[pdftex]{hyperref}                   
\usepackage {listings}
\usepackage {mathpartir}
\usepackage{bcprules}
%\usepackage{listings}
                       
\usepackage{graphicx} 
%\usepackage[margins=2.5cm,nohead,nofoot]{geometry}
%\usepackage{geometry}
\usepackage{amsfonts}
\usepackage{amstext}
\usepackage{latexsym}
\usepackage{amssymb}
\usepackage{color}


%\include{myPreamble}
\include{qm2pi.local} 

%\ifpdf
%\usepackage[pdftex]{graphicx}
%\else
%\usepackage{graphicx}
%\fi

 % \ifpdf
%  \usepackage{pdfsync}
%  \if


%\title{Brief Article}
%\author{David F. Snyder}
%\author{L.G. Meredith}

%\address{Dept. of Math., Texas State University--San Marcos, San Marcos, TX 78666}
       
\pagestyle{empty}


\begin{document}

\lstset{language=[Objective]Caml,frame=shadowbox}

\input{qm2pi.front}

% section front matter (end)

\input{qm2pi.intro} 
 
% section introduction (end)

% \input{qm2pi.knotations} 

% section notation (end)

\input{qm2pi.process.calculi} 

% section concurrent_process_calculi_and_spatial_logics_ (end)
    
%\input{qm2pi.knots2pi} 

%\input{qm2pi.trefoil} 

%\input{qm2pi.mainthm} 

% subsection basic_interpretation (end)

%\input{qm2pi.rho.presentation} 
\subsection{The syntax and semantics of the notation system}\label{sub:the_syntax_and_semantics_of_the_notation_system} % (fold)

We now summarize a technical presentation of the calculus that
embodies our theory of dynamics. The typical presentation of such a
calculus follows the style of giving generators and relations on
them. The grammar, below, describing term constructors, freely
generates the set of processes, $\Proc$. This set is then quotiented
by a relation known as structural congruence and it is over this set
that the notion of dynamics is expressed. This presentation is
essentially that of \cite{MeredithR05} with the addition of
polyadicity and summation. For readability we have relegated some of
the technical subtleties to an appendix.

\subsubsection{Process grammar}\label{subsub:process_grammar}

\begin{mathpar}
  \inferrule* [lab=synchronization] {} {{M} \bc \pzero \;|\; x?F \;|\; x!C }
  \and
  \inferrule* [lab=abstraction] {} {{F} \bc (x)P}
  \and
  \inferrule* [lab=concretion] {} {{C} \bc \langle Q \rangle}
  \and
  \inferrule* [lab=process] {} {{P,Q} \bc M \;| \;P|Q \;|\; @{x}}
  \and
  \inferrule* [lab=name] {} {{x} \bc \quotep{P}}
\end{mathpar} 

Note that $\vec{x}$ (resp. $\vec{P}$) denotes a vector of names
(resp. processes) of length $|\vec{x}|$ (resp. $|\vec{P}|$). We adopt
the following useful abbreviations.

\begin{mathpar}
   x?(\vec{y}).P := x.(\vec{y})P \and  x\clift{\vec{P}} := x.\clift{\vec{P}}
   \and x!(y) := \lift{x}{\dropn{y}}
   \and \Pi_{i=0}^{n-1}P_i := P_0 | \ldots | P_{n-1}
\end{mathpar}

\subsubsection{Structural congruence}

\paragraph{Free and bound names and alpha-equivalence.} At the
core of structural equivalence is alpha-equivalence which identifies
process that are the same up to a change of variable. Formally, we
recognize the distinction between free and bound names. The free names
of a process, $\freenames{P}$, may be calculated recursively as
follows:

\begin{mathpar}
\freenames{\pzero} := \emptyset
  \and \\
  \freenames{x?(y).P} := \{ x \} \cup (\freenames{P} \setminus \{ y \})
  \and 
  \freenames{x!\langle P \rangle} := \{ x \} \cup \{ P \} 
  \and \\
  \freenames{P|Q} := \freenames{P} \cup \freenames{Q}
  \and \\
  \freenames{@{x}} := \{ x \}
\end{mathpar}

$\pi$
$\quotep{\pi}$

$\freenames{-} : \pi \to \mathcal{P}(\quotep{\pi})$

\begin{eqnarray*}
  \freenames{\pzero} & := & \emptyset \\
  \freenames{x?(y).P} & := & \{ x \} \cup (\freenames{P} \setminus \{ y \}) \\
  \freenames{x!\langle P \rangle} & := & \{ x \} \cup \{ P \} \\
  \freenames{P|Q} & := & \freenames{P} \cup \freenames{Q} \\
  \freenames{\dropn{x}} & := & \{ x \}
\end{eqnarray*}

The bound names of a process, $\boundnames{P}$, are those names occurring in $P$
that are not free. For example, in $x?(y).0$, the name $x$ is free, while $y$ is bound.

\begin{mathpar}
  \inferrule* [lab=monoidal-laws] {} { P|Q \equiv Q|P \and P|0 \equiv P \and P|(Q|R) \equiv (P|Q)|R }
\end{mathpar}

\begin{mathpar}
  \inferrule* [lab=alpha-equivalence] {} { (x)P \equiv (y)P\{y/x\} \and y \not\in \freenames{P} }
\end{mathpar}

\begin{definition}
Then two processes, $P,Q$, are alpha-equivalent if $P = Q\{\vec{y}/\vec{x}\}$ for
some $\vec{x} \in \boundnames{Q},\vec{y} \in \boundnames{P}$, where $Q\{\vec{y}/\vec{x}\}$
denotes the capture-avoiding substitution of $\vec{y}$ for $\vec{x}$ in $Q$.
\end{definition}

\begin{definition}
  The {\em structural congruence} \cite{SangiorgiWalker} , $\equiv$,
  between processes is the least congruence containing
  alpha-equivalence, satisfying the abelian monoid laws
  (associativity, commutativity and $\pzero$ as identity) for parallel
  composition $|$ and for summation $+$.
\end{definition}

\subsection{Name equivalence}

We take name equivalence, written $\nameeq$, to be the smallest
equivalence relation generated by the following rules.

\begin{mathpar}
\inferrule*[lab=Quote-drop]
{ }
{ \quotep{@{x}} \nameeq x }

\inferrule*[lab=Struct-equiv]
{ P \scong Q }
{ \quotep{P} \nameeq \quotep{Q} }
\end{mathpar}

The astute reader will have noticed that the mutual recursion of names
and processes imposes a mutual recursion on alpha-equivalence and
structural equivalence via name-equivalence. Fortunately, all of this
works out pleasantly and we may calculate in the natural way, free of
concern. The reader interested in the details is referred to the
appendix \ref{appendix:rho_details}.

\subsection{Substitution}

We use $\Proc$ for the set of processes, $\QProc$ for the set of
names, and $\id{\{}\vec{y} / \vec{x} \id{\}}$ to denote partial maps,
$s : \QProc \rightarrow \QProc$. A map, $s$ lifts, uniquely, to a map
on process terms, $\widehat{s} : \Proc \rightarrow \Proc$ by the
following equations.

\begin{mathpar}
  (0) \psubstp{Q}{P} := 0 \\
  (R \juxtap S) \psubstp{Q}{P}
  :=    
  (R)\psubstp{Q}{P} \juxtap (S) \psubstp{Q}{P} \\
  (x?(y).R) \psubstp{Q}{P}    
  :=    
  (x)\substp{Q}{P} (z)\concat( (R \psubstn{z}{y}) \psubstp{Q}{P} ) \\
  (\lift{x}{R}) \psubstp{Q}{P}  
  :=
  \lift{(x)\substp{Q}{P}}{ R \psubstp{Q}{P} } \\
%   (\dropn{x})  \psubstp{Q}{P}       
%   := 
%   \left\{ 
%     \begin{array}{ccc} 
%       \dropn{\quotep{Q}} & & x \nameeq \quotep{P} \\
%       \dropn{x} & & otherwise \\
%     \end{array}
%   \right. 
  (\dropn{x})  \psubstp{Q}{P}       
  := 
  \left\{ 
    \begin{array}{ccc} 
      Q & & x \nameeq \quotep{P} \\
      \dropn{x} & & otherwise \\
    \end{array}
  \right.
\end{mathpar}
 

where

\begin{eqnarray}
  (x)\id{\{} \lpquote Q \rpquote / \lpquote P \rpquote \id{\}}            = 
  \left\{ 
    \begin{array}{ccc}
      \lpquote Q \rpquote & & x \nameeq \lpquote P \rpquote \\
      x & & otherwise \\
    \end{array}
  \right. \nonumber
\end{eqnarray}

and $z$ is chosen distinct from $\quotep{P}$, $\quotep{Q}$, the free
names in $Q$, and all the names in $R$. Our $\alpha$-equivalence will
be built in the standard way from this substitution.

\begin{remark}\label{rem:no_self_referential_names}
  One consequence of these definitions is that $\forall P. \quotep{P}
  \not\in \freenames{P}$.
\end{remark}

\subsection{ Dynamic quote: an example }

Anticipating something of what's to come, consider applying the
substitution, $\widehat{\id{\{}u / z \id{\}}}$, to the following pair
of processes, $\lift{w}{y!(z)}$ and $w[ \lpquote y!(z) \rpquote ]$.

\begin{eqnarray}
	\lift{w}{y!(z)}\widehat{\id{\{}u / z \id{\}}}
		& = &
		\lift{w}{y!(u)} \nonumber\\
	w[ \lpquote y!(z) \rpquote ] \widehat{ \id{\{}u / z \id{\}} }
		& = &
		w[ \lpquote y!(z) \rpquote ] \nonumber
\end{eqnarray}

Because the body of the process between quotes is impervious to
substitution, we get radically different answers. In fact, by
examining the first process in an input context,
e.g. $x?(z).\lift{w}{y!(z)}$, we see that the process under the lift
operator may be shaped by prefixed inputs binding a name inside it. In
this sense, the lift operator will be seen as a way to dynamically
construct processes before reifying them as names.

Finally equipped with these standard features we can present the
dynamics of the calculus.

\subsubsection{Operational semantics} 

Finally, we introduce the computational dynamics. What marks these
algebras as distinct from other more traditionally studied algebraic
structures, e.g. vector spaces or polynomial rings, is the manner in
which dynamics is captured. In traditional structures, dynamics is typically
expressed through morphisms between such structures, as in linear maps
between vector spaces or morphisms between rings. In algebras
associated with the semantics of computation, the dynamics is
expressed as part of the algebraic structure itself, through a
reduction reduction relation typically denoted by $\red$. Below, we
give a recursive presentation of this relation for the calculus used
in the encoding.

$\red \subseteq \pi \times \pi$
$\red : \pi \to \mathcal{P}(\pi)$

\begin{mathpar}
  \inferrule* [lab=Comm] { \textsf{match}( x_{src}, x_{trgt} ) } { x_{trgt}?(y)P \; | \; x_{src}!\langle {Q} \rangle \red P\{\quotep{Q}/y}\} }
  \and \\
  \inferrule* [lab=Par] {{P} \red {P}'} {{{P} | {Q}} \red {{P}' | {Q}}}
  \and
  \inferrule* [lab=Equiv]{{{P} \scong {P}'} \andalso {{P}' \red {Q}'} \andalso {{Q}' \scong {Q}}}{{P} \red {Q}}
\end{mathpar}

\begin{eqnarray*}
  match_{\equiv} (\quotep{P},\quotep{Q}) & := & P \equiv Q \\
  match_{\dagger}(\quotep{P},\quotep{Q}) & := & \forall R. P|Q \red^{*} R => R \red^{*} 0 \\
  match_{K}(\quotep{P},\quotep{Q}) & := & K \mbox{ for some context } K
\end{eqnarray*}

$u?(x)P | u!\langle Q \rangle \red P\{\quotep{Q}/x\}$

%We write $\wred$ for $\red^*$, and $P\red$ if $\exists Q $ such that $ P \red Q$.
We write $P\red$ if $\exists Q $ such that $ P \red Q$ and $P\not\red$, otherwise.

\section{Replication}

As mentioned before, it is known that replication (and hence
recursion) can be implemented in a higher-order process algebra
\cite{SangiorgiWalker}. As our first example of calculation with the
machinery thus far presented we give the construction explicitly in
the {\rhoc}.

\begin{eqnarray}
	D_{x} & := & \prefix{x}{y}{(\binpar{\outputp{x}{y}}{@{y}})} \nonumber\\
	\bangp_{x}{P} & := & \binpar{{x}!\langle{\binpar{D_{x}}{P}}\rangle}{D_{x}} \nonumber
\end{eqnarray}

\begin{eqnarray}
	\bangp_{x}{P} & & \nonumber\\
	=
	& {x}!\langle{(\prefix{x}{y}{(\outputp{x}{y} | @{y})) | P}}\rangle 
	      | \prefix{x}{y}{(\outputp{x}{y} | @{y})} & \nonumber\\
	\red
	& (\outputp{x}{y} | @{y})\substn{\quotep{(\prefix{x}{y}{(@{y} | \outputp{x}{y})) | P}}}{y} & \nonumber\\
	=
	& \outputp{x}{\quotep{(\prefix{x}{y}{(\outputp{x}{y} | @{y})) | P}}}
	  | {(\prefix{x}{y}{(\outputp{x}{y} | @{y})) | P}} & \nonumber\\
	\red
	& \ldots & \nonumber\\
	\red^*
	& P | P | \ldots & \nonumber
\end{eqnarray}

Of course, this encoding, as an implementation, runs away, unfolding
$\bangp{P}$ eagerly. A lazier and more implementable replication
operator, restricted to input-guarded processes, may be obtained as follows.

\begin{eqnarray}
\bangp{\prefix{u}{v}{P}} 
	:= 
	\binpar{\lift{x}{\prefix{u}{v}{(\binpar{D(x)}{P})}}}{D(x)} \nonumber
\end{eqnarray}

\begin{remark}
  Note that the lazier definition still does not deal with summation
  or mixed summation (i.e. sums over input and output). The reader is
  invited to construct definitions of replication that deal with these
  features. 

  Further, the definitions are parameterized in a name, $x$. Can you,
  gentle reader, make a definition that eliminates this parameter and
  guarantees no accidental interaction between the replication
  machinery and the process being replicated -- i.e. no accidental
  sharing of names used by the process to get its work done and the
  name(s) used by the replication to effect copying. This latter
  revision of the definition of replication is crucial to obtaining
  the expected identity $!!P \sim !P$.
\end{remark}

\begin{remark}\label{rem:paradoxical_combinator}
  The reader familiar with the lambda calculus will have noticed the
  similarity between $D$ and the paradoxical combinator.

  [Ed. note: the existence of this seems to suggest we have to be more
  restrictive on the set of processes and names we admit if we are to
  support no-cloning.]
\end{remark}

\subsubsection{Bisimulation}

The computational dynamics gives rise to another kind of equivalence,
the equivalence of computational behavior. As previously mentioned
this is typically captured \emph{via} some form of bisimulation.

% The notion we use in this paper is weak barbed bisimulation
% \cite{milner91polyadicpi}.

The notion we use in this paper is derived from weak barbed
bisimulation \cite{milner91polyadicpi}. 

\begin{definition}
An \emph{observation relation}, $\downarrow_{\mathcal N}$, over a set
of names, $\mathcal N$, is the smallest relation satisfying the rules
below.

\infrule[Out-barb]{y \in {\mathcal N}, \; x \nameeq y}
		  {\outputp{x}{v} \downarrow_{\mathcal N} x}
\infrule[Par-barb]{\mbox{$P\downarrow_{\mathcal N} x$ or $Q\downarrow_{\mathcal N} x$}}
		  {\binpar{P}{Q} \downarrow_{\mathcal N} x}

We write $P \Downarrow_{\mathcal N} x$ if there is $Q$ such that 
$P \wred Q$ and $Q \downarrow_{\mathcal N} x$.
\end{definition}

\begin{definition}
%\label{def.bbisim}
An  ${\mathcal N}$-\emph{barbed bisimulation} over a set of names, ${\mathcal N}$, is a symmetric binary relation 
${\mathcal S}_{\mathcal N}$ between agents such that $P\rel{S}_{\mathcal N}Q$ implies:
\begin{enumerate}
\item If $P \red P'$ then $Q \wred Q'$ and $P'\rel{S}_{\mathcal N} Q'$.
\item If $P\downarrow_{\mathcal N} x$, then $Q\Downarrow_{\mathcal N} x$.
\end{enumerate}
$P$ is ${\mathcal N}$-barbed bisimilar to $Q$, written
$P \wbbisim_{\mathcal N} Q$, if $P \rel{S}_{\mathcal N} Q$ for some ${\mathcal N}$-barbed bisimulation ${\mathcal S}_{\mathcal N}$.
\end{definition}

$\mathcal{R} \subseteq \pi \times \pi$

$P \mathcal{R} Q => \forall P'. P \red P' \Rightarrow \exists Q'. Q \red Q', P' \mathcal{R} Q'$

$P \vdash x \Rightarrow Q \vdash x$

\begin{mathpar}
  \inferrule*[lab=Out-barb]{x \nameeq y}{{y}!\langle{Q}\rangle \vdash x}
  \and
  \inferrule*[lab=Par-barb]{\mbox{$P\vdash x$ or $Q\vdash x$}}{\binpar{P}{Q} \vdash x}
\end{mathpar}

\subsubsection{Contexts}

One of the principle advantages of computational calculi like the
$\pi$-calculus is a well-defined notion of context,
contextual-equivalence and a correlation between
contextual-equivalence and notions of bisimulation. The notion of
context allows the decomposition of a process into (sub-)process and
its syntactic environment, its context. Thus, a context may be
thought of as a process with a ``hole'' (written $\Box$) in it. The
application of a context $M$ to a process $P$, written $M[P]$, is
tantamount to filling the hole in $M$ with $P$. In this paper we do
not need the full weight of this theory, but do make use of the notion
of context in the proof the main theorem. 

\begin{mathpar}
  \inferrule* [lab=summation] {} {{M_{M},M_{N}} \bc \Box \;|\; x.M_{A} \;|\; M_{M}+M_{N}}
  \and
  \inferrule* [lab=agent] {} {{M_{A}} \bc (\vec{x})M_{P} \;| \; \clift{P_0,\ldots,M_{P},\ldots,P_N}}
  \and \\
  \inferrule* [lab=process] {} {{M_{P}} \bc M_{N} \;| \;P|M_{P} }
\end{mathpar} 

\begin{mathpar}
  \inferrule* [lab=sychronization] {} {M_{N} \bc \Box \;|\; x?M_{F} \;|\; x!M_{C}}
  \and
  \inferrule* [lab=abstraction] {} {{M_{F}} \bc (x)M_{P} }
  \and
  \inferrule* [lab=concretion] {} {{M_{C}} \bc \langle M_{P} \rangle }
  \and \\
  \inferrule* [lab=process] {} {{M_{P}} \bc M_{N} \;| \;P|M_{P} }
\end{mathpar}

\begin{definition}[contextual application] Given a context $M$, and
  process $P$, we define the \emph{contextual application}, $M[P] :=
  M\{P/\Box\}$. That is, the contextual application of M to P is the
  substitution of $P$ for $\Box$ in $M$.
\end{definition}

$\meaningof{-} : L \to \mathcal{P}(\pi)$

\begin{mathpar}
  \inferrule* [lab=collection] {} {\meaningof{true} = \pi, \and \meaningof{~E} = \pi \setminus \meaningof{E}, \and \meaningof{E_{1} \& E_{2}} = \meaningof{E_{1}} \cap \meaningof{E_{2}}}
\end{mathpar}

\begin{mathpar}
  \inferrule* [lab=structure] {} {\meaningof{0} = \{ P \in \pi | P \equiv 0 \}, \and \\ \meaningof{E_1 | E_2} = \{ P \in \pi | P \equiv P_{1} | P_{2}, P_{1} \in \meaningof{E_{1}}, P_{2} \in \meaningof{E_2}\} }
\end{mathpar}

\begin{mathpar}
 \inferrule* [lab=behavior] {} {\meaningof{\langle a?b \rangle E} = \{ P \in \pi | P \equiv Q | u?(y)P', \\ \and \\\\ \and \\ \;\;\; u \in \meaningof{a}, \forall z.P'\{z/y\} \in \meaningof{E\{z/b\}}\}, \and \\ \meaningof{a!E} = \{ P \in \pi | P \equiv Q | x!\langle P' \rangle, x \in \meaningof{a} P' \in \meaningof{E}\} }
\end{mathpar}

\begin{mathpar}
 \inferrule* [lab=nominal] {} {\meaningof{\quotep{E}} = \{ \quotep{P} \in \quotep{\pi} | P \in \meaningof{E} \}, \and \meaningof{\quotep{P}} = \{ \quotep{Q} \in \quotep{\pi} | P \equiv Q \} \and \\ \meaningof{@\quotep{E}} = \{ P \in \pi | P \equiv @x, x \in \meaningof{E} \}}
\end{mathpar}

\begin{eqnarray*}
  \\
  \meaningof{-} : TS \to ST
\end{eqnarray*}

\begin{eqnarray*}
  \\
  L : TS \to ST
\end{eqnarray*}

\begin{eqnarray*}
  \\
  P \models E \iff P \in \meaningof{E}
\end{eqnarray*}

\begin{eqnarray*}
  P \approx_{L} Q \iff \forall E \in L. P \models E \iff Q \models E
\end{eqnarray*}

\begin{eqnarray*}
  P \approx_{K} Q
\end{eqnarray*}

\begin{eqnarray*}
  P \approx Q
\end{eqnarray*}

$\approx_{K} = \approx = \approx_{L}$

\subsubsection{Contextual duality}

Note that contexts extend the quotation operation to a family of
operations from processes to names. Given a context, $M$, we can
define a \emph{nominal context}, $\quotep{M}$ by $\quotep{M}[P] :=
\quotep{M[P]}$. To foreshadow what is to come we observe that these
operations enjoy a duality with processes very much like the duality
between vectors and maps from vectors to scalars.

Further, because the calculus is essentially higher-order, we have a
correspondence between contexts and processes. More specifically,
given a name $x$ and a context $M$ we can construct $M^{*}_{x}$ such
that 

\begin{mathpar}
  M^{*}_{x} | \lift{x}{P} \red M[P]
\end{mathpar}

namely,

\begin{mathpar}
  M^{*}_{x} := x?(u).M[\dropn{u}]
\end{mathpar}

The dependence of $M^{*}_{x}$ on a name makes it an abstraction, 

\begin{mathpar}
  M^{*} := (x)x?(u).M[\dropn{u}]
\end{mathpar}

\subsection{Additional notation}

It will sometimes be convenient to denote the process a name
quotes. We already have the notation $x = \quotep{P}$, but it will be
convenient to introduce an alternate notation, $\procn{x}$, when we
want to emphasize the connection to the use of the name. Note that, by
virtue of name equivalence, $\quotep{\procn{x}} \nameeq x$; so, the
notation is consistent with previous definitions.

Further, because names have structure it is possible to effect
substitutions on the basis of that structure. This means we need to
upgrade our notation for substitutions, which we accomplish by
adapting comprehension notation. Thus,

\begin{mathpar}
  P\{ y / x : x \in S \}
\end{mathpar}

is interpreted to mean the process derived from P by replacing (in a
capture-avoiding manner) each occurrence of $x$ in $S$ by $y$. For example,

\begin{mathpar}
  P\{ \quotep{\procn{x}|\procn{x}} / x : x \in \freenames{P} \}
\end{mathpar}

will replace each (occurrence) of a free name $x$ in $P$ by
$\quotep{\procn{x}|\procn{x}}$.

Also, we will avail ourselves of the notation $x^{L}$ and $x^{R}$ to
denote injections of a name into disjoint copies of the name
space. There are numerous ways to accomplish this. One example can be
found in \cite{MeredithR05}. This notation overloads to vectors of
names: $\vec{x}^{\pi} := (x_{i}^{\pi} \; : \; 0 \leq i < |\vec{x}| )$ where $\pi \in \{L,R\}$.

We also use $P^{\Box} := P|\Box$.

In \cite{MeredithR05} an interpretation of the new operator is
given. It turns out that there are several possible interpretations
all enjoying the requisite algebraic properties of the operator (see
\cite{milner91polyadicpi}). We will therefore make liberal use of
$(\nu\; \vec{x})P$.

% subsection the_syntax_and_semantics_of_the_notation_system (end)   

\input{qm2pi.qmops} 

\input{qm2pi.sterngerlach} 

\input{qm2pi.metric} 

% section concurrent_process_calculi (end)

%\input{qm2pi.proofsketch}

% section proof sketch (end)

%\input{qm2pi.slviaknots} 

% section spatial logic via knots (end)

\input{qm2pi.conclusion}

% section conclusion (end)

%\input{qm2pi.dtcodes} 

% section wiring algorithm (end)

\input{qm2pi.ack} 

% section acknowledgments (end)

\newpage


\bibliographystyle{plain}   
\bibliography{../../biblios/main.bib}

\input{qm2pi.rhodetails}

\end{document}



% section front matter (end)

\section{Introduction}\label{sec:introduction} % (fold)
In this draft of the material i am going to have to dispense with the
usual writing conventions adopted in papers on these topics. i'm going
to have adopt whatever tone i need at the time i'm writing up the
calculations. Sometimes this may be very conversational; others it may
be the barest mathematical grunts; others still it may be that i have
lifted text from one of my other papers because the exposition of some
point was better said there. i hope that my readers are not unduly put
out by this decision. i'm not doing this to flout convention or be
rebellious. i find these calculations very technically challenging. To
keep everything going technically, something has to give; i have to
let go of some cognitive burden. So, the academic writing style --
with all of its trade-offs in terms of facilitating technical
communication -- is what i'm letting go of. Perhaps subsequent drafts
can be tightened and polished, but for now, i'm going to speak as if
we were sitting together in a coffee shop with a laptop, wifi and a
pad of paper and a pencil.

So, here's what i have to say. We -- you and i, comfortably ensconced
in our coffee shop and well-equipped with our tools -- can realize and
carry out the calculations of quantum mechanics over a very different
formal theory of dynamics, a formal theory of dynamics that
corresponds to a theory of concurrent computation with
\emph{reflection}. It has the advantage that the underlying theory is
already `quantized', but supports analogues all of the continuuous
operations. Strikingly, this underlying theory has recently been
connected with a notion of metric that we can show, by calculating
together, coincides with the metric induced by the inner product.

There are a lot of reasons why you might be interested in seeing
calculations of this form. Here's why i'm interested. For the past
several centuries there has been no competitor to the ``Newtonian''
account of dynamics. As a result the predominant share of accounts of
dynamical systems and situations have had to be formulated in terms of
the Newtonian machinery. i view this as an intellectually dangerous
position to occupy. Everything, despite it's intrinsic shape, turns
into a nail to be hit with this hammer. Recently, however, the theory
of computation has matured to the point where we have candidates for
theories of dynamics that offer very different perspective on
reasoning about dynamical systems and situations. Testing these
candidates against very successful accounts of dynamical situations,
like quantum mechanics, is going to give us some sense of how mature
they are and some measure of the quality of these accounts of
dynamics.

\subsection{Summary of contributions and outline of paper}

So, we're going to develop an interpretation of the operations of
quantum mechanics normally interpreted by Hilbert spaces and
operators. We're going to do this over a theory of computation. Note
that this is very different than the usual quantum computation program
which develops notions of computation over quantum mechanics. Rather,
we are developing a story that aligns with Wheeler's slogan: It from
Bit. To do this we will first provide an account of the theory of
computation at play here. Then we will dive into a calculation-driven
interpretation of the operations of quantum mechanics.

The reason we take this approach is that -- until very recently --
there hasn't been an axiomatic account of quantum mechanics. As a
result there has been no sharp delineation of the mathematical theory
supporting interpretation of the physical theory and the physical
theory, itself. So, ambient features of the maths are free to be
exploited (or supressed) without a real accounting of their physical
relevance. There is no sharp statement ``here's the physical theory''
qua \emph{theory} and ``here's the mathematical interpretation''
enabling a judgment of how faithful the interpretation is -- apart
from experimental observation. When there is an axiomatic account we
can judge how well a given mathematical formalism supports an
interpretation of the axioms, independent of
experimentation. Likewise, we can judge how well we have captured our
physical evidence and experience with our axiomatics, independent of
any specific mathematical implementation, with accidental detail that
may or may not have physical significance. 

In lieu of a fully fleshed out and vetted axiomatic account of quantum
mechanics, interpreting the operational notions in service of modeling
physical systems will have to suffice. In other words, we are not in
the business of providing a model of Hilbert spaces and operators. We
are in the business of providing a model of quantum mechanics because
we are motivated by testing our notions of dynamics against physical
theory; and, the predictive calculations of the physical theory must
serve as the best formulation -- shy of a fully fleshed out axiomatic
account -- of the physical theory itself (as they have for scientific
theories since time immemorial). Put another way, despite a
whole-hearted commitment to an It-from-Bit ontology, we are firmly
aligned with the shut-up-and-calculate camp as the best way to obtain
results either from the physical perspective or as a quality assurance
measure of our fledgling theory of dynamics.

In detail, we present a reflective process calculus. Then we develop
intuitive correspondences between the notions available in this
calculus and the usual physical notions supporting quantum mechanical
calculations. Thus, 

\begin{table}[htp]
  \center{
    \fbox{
      \begin{tabular}{c|c}
        quantum mechanics & process calculus \\
        \hline
        scalar & name \\
        state vector & process \\
        dual & contextual duals \\
        matrix & formal sums of process-context-dual pairs \\
        orthogonality & process annihilation \\
        inner product & execution-formula + quoting
      \end{tabular}
    }
  }
  \caption{QM - process calculi correspondences}
\end{table}

Then we tighten up these intuitions to operational definitions. We
employ the Dirac notation as the best proxy we can find for an
abstract syntax of the quantum mechanical notions. The definitions we
develop put us in contact with equational constraints coming from the
theory that we demonstrate the definitions and calculations satisfy.

This puts us in a position to shut up and calculate for the
Stern-Gerlach experimental set up, showing how these predictive
calculations become calculations on processes in our theory of a
reflective process calculus.

Penultimately, we demonstrate that the notion of metric coming from
the inner product coincides with the notion of metric available from
the theory of bisimulation. This demonstration gives us the right to
think of space as arising from behavior. Finally, we consider where we
might go from the new vantage point we have obtained.

% section introduction (end) 
 
% section introduction (end)

% \documentclass[12pt]{llncs}
%\documentclass{jktr}

\usepackage[pdftex]{hyperref}                   
\usepackage {listings}
\usepackage {mathpartir}
\usepackage{bcprules}
%\usepackage{listings}
                       
\usepackage{graphicx} 
%\usepackage[margins=2.5cm,nohead,nofoot]{geometry}
%\usepackage{geometry}
\usepackage{amsfonts}
\usepackage{amstext}
\usepackage{latexsym}
\usepackage{amssymb}
\usepackage{color}


%\include{myPreamble}
\include{qm2pi.local} 

%\ifpdf
%\usepackage[pdftex]{graphicx}
%\else
%\usepackage{graphicx}
%\fi

 % \ifpdf
%  \usepackage{pdfsync}
%  \if


%\title{Brief Article}
%\author{David F. Snyder}
%\author{L.G. Meredith}

%\address{Dept. of Math., Texas State University--San Marcos, San Marcos, TX 78666}
       
\pagestyle{empty}


\begin{document}

\lstset{language=[Objective]Caml,frame=shadowbox}

\input{qm2pi.front}

% section front matter (end)

\input{qm2pi.intro} 
 
% section introduction (end)

% \input{qm2pi.knotations} 

% section notation (end)

\input{qm2pi.process.calculi} 

% section concurrent_process_calculi_and_spatial_logics_ (end)
    
%\input{qm2pi.knots2pi} 

%\input{qm2pi.trefoil} 

%\input{qm2pi.mainthm} 

% subsection basic_interpretation (end)

%\input{qm2pi.rho.presentation} 
\subsection{The syntax and semantics of the notation system}\label{sub:the_syntax_and_semantics_of_the_notation_system} % (fold)

We now summarize a technical presentation of the calculus that
embodies our theory of dynamics. The typical presentation of such a
calculus follows the style of giving generators and relations on
them. The grammar, below, describing term constructors, freely
generates the set of processes, $\Proc$. This set is then quotiented
by a relation known as structural congruence and it is over this set
that the notion of dynamics is expressed. This presentation is
essentially that of \cite{MeredithR05} with the addition of
polyadicity and summation. For readability we have relegated some of
the technical subtleties to an appendix.

\subsubsection{Process grammar}\label{subsub:process_grammar}

\begin{mathpar}
  \inferrule* [lab=synchronization] {} {{M} \bc \pzero \;|\; x?F \;|\; x!C }
  \and
  \inferrule* [lab=abstraction] {} {{F} \bc (x)P}
  \and
  \inferrule* [lab=concretion] {} {{C} \bc \langle Q \rangle}
  \and
  \inferrule* [lab=process] {} {{P,Q} \bc M \;| \;P|Q \;|\; @{x}}
  \and
  \inferrule* [lab=name] {} {{x} \bc \quotep{P}}
\end{mathpar} 

Note that $\vec{x}$ (resp. $\vec{P}$) denotes a vector of names
(resp. processes) of length $|\vec{x}|$ (resp. $|\vec{P}|$). We adopt
the following useful abbreviations.

\begin{mathpar}
   x?(\vec{y}).P := x.(\vec{y})P \and  x\clift{\vec{P}} := x.\clift{\vec{P}}
   \and x!(y) := \lift{x}{\dropn{y}}
   \and \Pi_{i=0}^{n-1}P_i := P_0 | \ldots | P_{n-1}
\end{mathpar}

\subsubsection{Structural congruence}

\paragraph{Free and bound names and alpha-equivalence.} At the
core of structural equivalence is alpha-equivalence which identifies
process that are the same up to a change of variable. Formally, we
recognize the distinction between free and bound names. The free names
of a process, $\freenames{P}$, may be calculated recursively as
follows:

\begin{mathpar}
\freenames{\pzero} := \emptyset
  \and \\
  \freenames{x?(y).P} := \{ x \} \cup (\freenames{P} \setminus \{ y \})
  \and 
  \freenames{x!\langle P \rangle} := \{ x \} \cup \{ P \} 
  \and \\
  \freenames{P|Q} := \freenames{P} \cup \freenames{Q}
  \and \\
  \freenames{@{x}} := \{ x \}
\end{mathpar}

$\pi$
$\quotep{\pi}$

$\freenames{-} : \pi \to \mathcal{P}(\quotep{\pi})$

\begin{eqnarray*}
  \freenames{\pzero} & := & \emptyset \\
  \freenames{x?(y).P} & := & \{ x \} \cup (\freenames{P} \setminus \{ y \}) \\
  \freenames{x!\langle P \rangle} & := & \{ x \} \cup \{ P \} \\
  \freenames{P|Q} & := & \freenames{P} \cup \freenames{Q} \\
  \freenames{\dropn{x}} & := & \{ x \}
\end{eqnarray*}

The bound names of a process, $\boundnames{P}$, are those names occurring in $P$
that are not free. For example, in $x?(y).0$, the name $x$ is free, while $y$ is bound.

\begin{mathpar}
  \inferrule* [lab=monoidal-laws] {} { P|Q \equiv Q|P \and P|0 \equiv P \and P|(Q|R) \equiv (P|Q)|R }
\end{mathpar}

\begin{mathpar}
  \inferrule* [lab=alpha-equivalence] {} { (x)P \equiv (y)P\{y/x\} \and y \not\in \freenames{P} }
\end{mathpar}

\begin{definition}
Then two processes, $P,Q$, are alpha-equivalent if $P = Q\{\vec{y}/\vec{x}\}$ for
some $\vec{x} \in \boundnames{Q},\vec{y} \in \boundnames{P}$, where $Q\{\vec{y}/\vec{x}\}$
denotes the capture-avoiding substitution of $\vec{y}$ for $\vec{x}$ in $Q$.
\end{definition}

\begin{definition}
  The {\em structural congruence} \cite{SangiorgiWalker} , $\equiv$,
  between processes is the least congruence containing
  alpha-equivalence, satisfying the abelian monoid laws
  (associativity, commutativity and $\pzero$ as identity) for parallel
  composition $|$ and for summation $+$.
\end{definition}

\subsection{Name equivalence}

We take name equivalence, written $\nameeq$, to be the smallest
equivalence relation generated by the following rules.

\begin{mathpar}
\inferrule*[lab=Quote-drop]
{ }
{ \quotep{@{x}} \nameeq x }

\inferrule*[lab=Struct-equiv]
{ P \scong Q }
{ \quotep{P} \nameeq \quotep{Q} }
\end{mathpar}

The astute reader will have noticed that the mutual recursion of names
and processes imposes a mutual recursion on alpha-equivalence and
structural equivalence via name-equivalence. Fortunately, all of this
works out pleasantly and we may calculate in the natural way, free of
concern. The reader interested in the details is referred to the
appendix \ref{appendix:rho_details}.

\subsection{Substitution}

We use $\Proc$ for the set of processes, $\QProc$ for the set of
names, and $\id{\{}\vec{y} / \vec{x} \id{\}}$ to denote partial maps,
$s : \QProc \rightarrow \QProc$. A map, $s$ lifts, uniquely, to a map
on process terms, $\widehat{s} : \Proc \rightarrow \Proc$ by the
following equations.

\begin{mathpar}
  (0) \psubstp{Q}{P} := 0 \\
  (R \juxtap S) \psubstp{Q}{P}
  :=    
  (R)\psubstp{Q}{P} \juxtap (S) \psubstp{Q}{P} \\
  (x?(y).R) \psubstp{Q}{P}    
  :=    
  (x)\substp{Q}{P} (z)\concat( (R \psubstn{z}{y}) \psubstp{Q}{P} ) \\
  (\lift{x}{R}) \psubstp{Q}{P}  
  :=
  \lift{(x)\substp{Q}{P}}{ R \psubstp{Q}{P} } \\
%   (\dropn{x})  \psubstp{Q}{P}       
%   := 
%   \left\{ 
%     \begin{array}{ccc} 
%       \dropn{\quotep{Q}} & & x \nameeq \quotep{P} \\
%       \dropn{x} & & otherwise \\
%     \end{array}
%   \right. 
  (\dropn{x})  \psubstp{Q}{P}       
  := 
  \left\{ 
    \begin{array}{ccc} 
      Q & & x \nameeq \quotep{P} \\
      \dropn{x} & & otherwise \\
    \end{array}
  \right.
\end{mathpar}
 

where

\begin{eqnarray}
  (x)\id{\{} \lpquote Q \rpquote / \lpquote P \rpquote \id{\}}            = 
  \left\{ 
    \begin{array}{ccc}
      \lpquote Q \rpquote & & x \nameeq \lpquote P \rpquote \\
      x & & otherwise \\
    \end{array}
  \right. \nonumber
\end{eqnarray}

and $z$ is chosen distinct from $\quotep{P}$, $\quotep{Q}$, the free
names in $Q$, and all the names in $R$. Our $\alpha$-equivalence will
be built in the standard way from this substitution.

\begin{remark}\label{rem:no_self_referential_names}
  One consequence of these definitions is that $\forall P. \quotep{P}
  \not\in \freenames{P}$.
\end{remark}

\subsection{ Dynamic quote: an example }

Anticipating something of what's to come, consider applying the
substitution, $\widehat{\id{\{}u / z \id{\}}}$, to the following pair
of processes, $\lift{w}{y!(z)}$ and $w[ \lpquote y!(z) \rpquote ]$.

\begin{eqnarray}
	\lift{w}{y!(z)}\widehat{\id{\{}u / z \id{\}}}
		& = &
		\lift{w}{y!(u)} \nonumber\\
	w[ \lpquote y!(z) \rpquote ] \widehat{ \id{\{}u / z \id{\}} }
		& = &
		w[ \lpquote y!(z) \rpquote ] \nonumber
\end{eqnarray}

Because the body of the process between quotes is impervious to
substitution, we get radically different answers. In fact, by
examining the first process in an input context,
e.g. $x?(z).\lift{w}{y!(z)}$, we see that the process under the lift
operator may be shaped by prefixed inputs binding a name inside it. In
this sense, the lift operator will be seen as a way to dynamically
construct processes before reifying them as names.

Finally equipped with these standard features we can present the
dynamics of the calculus.

\subsubsection{Operational semantics} 

Finally, we introduce the computational dynamics. What marks these
algebras as distinct from other more traditionally studied algebraic
structures, e.g. vector spaces or polynomial rings, is the manner in
which dynamics is captured. In traditional structures, dynamics is typically
expressed through morphisms between such structures, as in linear maps
between vector spaces or morphisms between rings. In algebras
associated with the semantics of computation, the dynamics is
expressed as part of the algebraic structure itself, through a
reduction reduction relation typically denoted by $\red$. Below, we
give a recursive presentation of this relation for the calculus used
in the encoding.

$\red \subseteq \pi \times \pi$
$\red : \pi \to \mathcal{P}(\pi)$

\begin{mathpar}
  \inferrule* [lab=Comm] { \textsf{match}( x_{src}, x_{trgt} ) } { x_{trgt}?(y)P \; | \; x_{src}!\langle {Q} \rangle \red P\{\quotep{Q}/y}\} }
  \and \\
  \inferrule* [lab=Par] {{P} \red {P}'} {{{P} | {Q}} \red {{P}' | {Q}}}
  \and
  \inferrule* [lab=Equiv]{{{P} \scong {P}'} \andalso {{P}' \red {Q}'} \andalso {{Q}' \scong {Q}}}{{P} \red {Q}}
\end{mathpar}

\begin{eqnarray*}
  match_{\equiv} (\quotep{P},\quotep{Q}) & := & P \equiv Q \\
  match_{\dagger}(\quotep{P},\quotep{Q}) & := & \forall R. P|Q \red^{*} R => R \red^{*} 0 \\
  match_{K}(\quotep{P},\quotep{Q}) & := & K \mbox{ for some context } K
\end{eqnarray*}

$u?(x)P | u!\langle Q \rangle \red P\{\quotep{Q}/x\}$

%We write $\wred$ for $\red^*$, and $P\red$ if $\exists Q $ such that $ P \red Q$.
We write $P\red$ if $\exists Q $ such that $ P \red Q$ and $P\not\red$, otherwise.

\section{Replication}

As mentioned before, it is known that replication (and hence
recursion) can be implemented in a higher-order process algebra
\cite{SangiorgiWalker}. As our first example of calculation with the
machinery thus far presented we give the construction explicitly in
the {\rhoc}.

\begin{eqnarray}
	D_{x} & := & \prefix{x}{y}{(\binpar{\outputp{x}{y}}{@{y}})} \nonumber\\
	\bangp_{x}{P} & := & \binpar{{x}!\langle{\binpar{D_{x}}{P}}\rangle}{D_{x}} \nonumber
\end{eqnarray}

\begin{eqnarray}
	\bangp_{x}{P} & & \nonumber\\
	=
	& {x}!\langle{(\prefix{x}{y}{(\outputp{x}{y} | @{y})) | P}}\rangle 
	      | \prefix{x}{y}{(\outputp{x}{y} | @{y})} & \nonumber\\
	\red
	& (\outputp{x}{y} | @{y})\substn{\quotep{(\prefix{x}{y}{(@{y} | \outputp{x}{y})) | P}}}{y} & \nonumber\\
	=
	& \outputp{x}{\quotep{(\prefix{x}{y}{(\outputp{x}{y} | @{y})) | P}}}
	  | {(\prefix{x}{y}{(\outputp{x}{y} | @{y})) | P}} & \nonumber\\
	\red
	& \ldots & \nonumber\\
	\red^*
	& P | P | \ldots & \nonumber
\end{eqnarray}

Of course, this encoding, as an implementation, runs away, unfolding
$\bangp{P}$ eagerly. A lazier and more implementable replication
operator, restricted to input-guarded processes, may be obtained as follows.

\begin{eqnarray}
\bangp{\prefix{u}{v}{P}} 
	:= 
	\binpar{\lift{x}{\prefix{u}{v}{(\binpar{D(x)}{P})}}}{D(x)} \nonumber
\end{eqnarray}

\begin{remark}
  Note that the lazier definition still does not deal with summation
  or mixed summation (i.e. sums over input and output). The reader is
  invited to construct definitions of replication that deal with these
  features. 

  Further, the definitions are parameterized in a name, $x$. Can you,
  gentle reader, make a definition that eliminates this parameter and
  guarantees no accidental interaction between the replication
  machinery and the process being replicated -- i.e. no accidental
  sharing of names used by the process to get its work done and the
  name(s) used by the replication to effect copying. This latter
  revision of the definition of replication is crucial to obtaining
  the expected identity $!!P \sim !P$.
\end{remark}

\begin{remark}\label{rem:paradoxical_combinator}
  The reader familiar with the lambda calculus will have noticed the
  similarity between $D$ and the paradoxical combinator.

  [Ed. note: the existence of this seems to suggest we have to be more
  restrictive on the set of processes and names we admit if we are to
  support no-cloning.]
\end{remark}

\subsubsection{Bisimulation}

The computational dynamics gives rise to another kind of equivalence,
the equivalence of computational behavior. As previously mentioned
this is typically captured \emph{via} some form of bisimulation.

% The notion we use in this paper is weak barbed bisimulation
% \cite{milner91polyadicpi}.

The notion we use in this paper is derived from weak barbed
bisimulation \cite{milner91polyadicpi}. 

\begin{definition}
An \emph{observation relation}, $\downarrow_{\mathcal N}$, over a set
of names, $\mathcal N$, is the smallest relation satisfying the rules
below.

\infrule[Out-barb]{y \in {\mathcal N}, \; x \nameeq y}
		  {\outputp{x}{v} \downarrow_{\mathcal N} x}
\infrule[Par-barb]{\mbox{$P\downarrow_{\mathcal N} x$ or $Q\downarrow_{\mathcal N} x$}}
		  {\binpar{P}{Q} \downarrow_{\mathcal N} x}

We write $P \Downarrow_{\mathcal N} x$ if there is $Q$ such that 
$P \wred Q$ and $Q \downarrow_{\mathcal N} x$.
\end{definition}

\begin{definition}
%\label{def.bbisim}
An  ${\mathcal N}$-\emph{barbed bisimulation} over a set of names, ${\mathcal N}$, is a symmetric binary relation 
${\mathcal S}_{\mathcal N}$ between agents such that $P\rel{S}_{\mathcal N}Q$ implies:
\begin{enumerate}
\item If $P \red P'$ then $Q \wred Q'$ and $P'\rel{S}_{\mathcal N} Q'$.
\item If $P\downarrow_{\mathcal N} x$, then $Q\Downarrow_{\mathcal N} x$.
\end{enumerate}
$P$ is ${\mathcal N}$-barbed bisimilar to $Q$, written
$P \wbbisim_{\mathcal N} Q$, if $P \rel{S}_{\mathcal N} Q$ for some ${\mathcal N}$-barbed bisimulation ${\mathcal S}_{\mathcal N}$.
\end{definition}

$\mathcal{R} \subseteq \pi \times \pi$

$P \mathcal{R} Q => \forall P'. P \red P' \Rightarrow \exists Q'. Q \red Q', P' \mathcal{R} Q'$

$P \vdash x \Rightarrow Q \vdash x$

\begin{mathpar}
  \inferrule*[lab=Out-barb]{x \nameeq y}{{y}!\langle{Q}\rangle \vdash x}
  \and
  \inferrule*[lab=Par-barb]{\mbox{$P\vdash x$ or $Q\vdash x$}}{\binpar{P}{Q} \vdash x}
\end{mathpar}

\subsubsection{Contexts}

One of the principle advantages of computational calculi like the
$\pi$-calculus is a well-defined notion of context,
contextual-equivalence and a correlation between
contextual-equivalence and notions of bisimulation. The notion of
context allows the decomposition of a process into (sub-)process and
its syntactic environment, its context. Thus, a context may be
thought of as a process with a ``hole'' (written $\Box$) in it. The
application of a context $M$ to a process $P$, written $M[P]$, is
tantamount to filling the hole in $M$ with $P$. In this paper we do
not need the full weight of this theory, but do make use of the notion
of context in the proof the main theorem. 

\begin{mathpar}
  \inferrule* [lab=summation] {} {{M_{M},M_{N}} \bc \Box \;|\; x.M_{A} \;|\; M_{M}+M_{N}}
  \and
  \inferrule* [lab=agent] {} {{M_{A}} \bc (\vec{x})M_{P} \;| \; \clift{P_0,\ldots,M_{P},\ldots,P_N}}
  \and \\
  \inferrule* [lab=process] {} {{M_{P}} \bc M_{N} \;| \;P|M_{P} }
\end{mathpar} 

\begin{mathpar}
  \inferrule* [lab=sychronization] {} {M_{N} \bc \Box \;|\; x?M_{F} \;|\; x!M_{C}}
  \and
  \inferrule* [lab=abstraction] {} {{M_{F}} \bc (x)M_{P} }
  \and
  \inferrule* [lab=concretion] {} {{M_{C}} \bc \langle M_{P} \rangle }
  \and \\
  \inferrule* [lab=process] {} {{M_{P}} \bc M_{N} \;| \;P|M_{P} }
\end{mathpar}

\begin{definition}[contextual application] Given a context $M$, and
  process $P$, we define the \emph{contextual application}, $M[P] :=
  M\{P/\Box\}$. That is, the contextual application of M to P is the
  substitution of $P$ for $\Box$ in $M$.
\end{definition}

$\meaningof{-} : L \to \mathcal{P}(\pi)$

\begin{mathpar}
  \inferrule* [lab=collection] {} {\meaningof{true} = \pi, \and \meaningof{~E} = \pi \setminus \meaningof{E}, \and \meaningof{E_{1} \& E_{2}} = \meaningof{E_{1}} \cap \meaningof{E_{2}}}
\end{mathpar}

\begin{mathpar}
  \inferrule* [lab=structure] {} {\meaningof{0} = \{ P \in \pi | P \equiv 0 \}, \and \\ \meaningof{E_1 | E_2} = \{ P \in \pi | P \equiv P_{1} | P_{2}, P_{1} \in \meaningof{E_{1}}, P_{2} \in \meaningof{E_2}\} }
\end{mathpar}

\begin{mathpar}
 \inferrule* [lab=behavior] {} {\meaningof{\langle a?b \rangle E} = \{ P \in \pi | P \equiv Q | u?(y)P', \\ \and \\\\ \and \\ \;\;\; u \in \meaningof{a}, \forall z.P'\{z/y\} \in \meaningof{E\{z/b\}}\}, \and \\ \meaningof{a!E} = \{ P \in \pi | P \equiv Q | x!\langle P' \rangle, x \in \meaningof{a} P' \in \meaningof{E}\} }
\end{mathpar}

\begin{mathpar}
 \inferrule* [lab=nominal] {} {\meaningof{\quotep{E}} = \{ \quotep{P} \in \quotep{\pi} | P \in \meaningof{E} \}, \and \meaningof{\quotep{P}} = \{ \quotep{Q} \in \quotep{\pi} | P \equiv Q \} \and \\ \meaningof{@\quotep{E}} = \{ P \in \pi | P \equiv @x, x \in \meaningof{E} \}}
\end{mathpar}

\begin{eqnarray*}
  \\
  \meaningof{-} : TS \to ST
\end{eqnarray*}

\begin{eqnarray*}
  \\
  L : TS \to ST
\end{eqnarray*}

\begin{eqnarray*}
  \\
  P \models E \iff P \in \meaningof{E}
\end{eqnarray*}

\begin{eqnarray*}
  P \approx_{L} Q \iff \forall E \in L. P \models E \iff Q \models E
\end{eqnarray*}

\begin{eqnarray*}
  P \approx_{K} Q
\end{eqnarray*}

\begin{eqnarray*}
  P \approx Q
\end{eqnarray*}

$\approx_{K} = \approx = \approx_{L}$

\subsubsection{Contextual duality}

Note that contexts extend the quotation operation to a family of
operations from processes to names. Given a context, $M$, we can
define a \emph{nominal context}, $\quotep{M}$ by $\quotep{M}[P] :=
\quotep{M[P]}$. To foreshadow what is to come we observe that these
operations enjoy a duality with processes very much like the duality
between vectors and maps from vectors to scalars.

Further, because the calculus is essentially higher-order, we have a
correspondence between contexts and processes. More specifically,
given a name $x$ and a context $M$ we can construct $M^{*}_{x}$ such
that 

\begin{mathpar}
  M^{*}_{x} | \lift{x}{P} \red M[P]
\end{mathpar}

namely,

\begin{mathpar}
  M^{*}_{x} := x?(u).M[\dropn{u}]
\end{mathpar}

The dependence of $M^{*}_{x}$ on a name makes it an abstraction, 

\begin{mathpar}
  M^{*} := (x)x?(u).M[\dropn{u}]
\end{mathpar}

\subsection{Additional notation}

It will sometimes be convenient to denote the process a name
quotes. We already have the notation $x = \quotep{P}$, but it will be
convenient to introduce an alternate notation, $\procn{x}$, when we
want to emphasize the connection to the use of the name. Note that, by
virtue of name equivalence, $\quotep{\procn{x}} \nameeq x$; so, the
notation is consistent with previous definitions.

Further, because names have structure it is possible to effect
substitutions on the basis of that structure. This means we need to
upgrade our notation for substitutions, which we accomplish by
adapting comprehension notation. Thus,

\begin{mathpar}
  P\{ y / x : x \in S \}
\end{mathpar}

is interpreted to mean the process derived from P by replacing (in a
capture-avoiding manner) each occurrence of $x$ in $S$ by $y$. For example,

\begin{mathpar}
  P\{ \quotep{\procn{x}|\procn{x}} / x : x \in \freenames{P} \}
\end{mathpar}

will replace each (occurrence) of a free name $x$ in $P$ by
$\quotep{\procn{x}|\procn{x}}$.

Also, we will avail ourselves of the notation $x^{L}$ and $x^{R}$ to
denote injections of a name into disjoint copies of the name
space. There are numerous ways to accomplish this. One example can be
found in \cite{MeredithR05}. This notation overloads to vectors of
names: $\vec{x}^{\pi} := (x_{i}^{\pi} \; : \; 0 \leq i < |\vec{x}| )$ where $\pi \in \{L,R\}$.

We also use $P^{\Box} := P|\Box$.

In \cite{MeredithR05} an interpretation of the new operator is
given. It turns out that there are several possible interpretations
all enjoying the requisite algebraic properties of the operator (see
\cite{milner91polyadicpi}). We will therefore make liberal use of
$(\nu\; \vec{x})P$.

% subsection the_syntax_and_semantics_of_the_notation_system (end)   

\input{qm2pi.qmops} 

\input{qm2pi.sterngerlach} 

\input{qm2pi.metric} 

% section concurrent_process_calculi (end)

%\input{qm2pi.proofsketch}

% section proof sketch (end)

%\input{qm2pi.slviaknots} 

% section spatial logic via knots (end)

\input{qm2pi.conclusion}

% section conclusion (end)

%\input{qm2pi.dtcodes} 

% section wiring algorithm (end)

\input{qm2pi.ack} 

% section acknowledgments (end)

\newpage


\bibliographystyle{plain}   
\bibliography{../../biblios/main.bib}

\input{qm2pi.rhodetails}

\end{document}

 

% section notation (end)

\input{qm2pi.process.calculi} 

% section concurrent_process_calculi_and_spatial_logics_ (end)
    
%\documentclass[12pt]{llncs}
%\documentclass{jktr}

\usepackage[pdftex]{hyperref}                   
\usepackage {listings}
\usepackage {mathpartir}
\usepackage{bcprules}
%\usepackage{listings}
                       
\usepackage{graphicx} 
%\usepackage[margins=2.5cm,nohead,nofoot]{geometry}
%\usepackage{geometry}
\usepackage{amsfonts}
\usepackage{amstext}
\usepackage{latexsym}
\usepackage{amssymb}
\usepackage{color}


%\include{myPreamble}
\include{qm2pi.local} 

%\ifpdf
%\usepackage[pdftex]{graphicx}
%\else
%\usepackage{graphicx}
%\fi

 % \ifpdf
%  \usepackage{pdfsync}
%  \if


%\title{Brief Article}
%\author{David F. Snyder}
%\author{L.G. Meredith}

%\address{Dept. of Math., Texas State University--San Marcos, San Marcos, TX 78666}
       
\pagestyle{empty}


\begin{document}

\lstset{language=[Objective]Caml,frame=shadowbox}

\input{qm2pi.front}

% section front matter (end)

\input{qm2pi.intro} 
 
% section introduction (end)

% \input{qm2pi.knotations} 

% section notation (end)

\input{qm2pi.process.calculi} 

% section concurrent_process_calculi_and_spatial_logics_ (end)
    
%\input{qm2pi.knots2pi} 

%\input{qm2pi.trefoil} 

%\input{qm2pi.mainthm} 

% subsection basic_interpretation (end)

%\input{qm2pi.rho.presentation} 
\subsection{The syntax and semantics of the notation system}\label{sub:the_syntax_and_semantics_of_the_notation_system} % (fold)

We now summarize a technical presentation of the calculus that
embodies our theory of dynamics. The typical presentation of such a
calculus follows the style of giving generators and relations on
them. The grammar, below, describing term constructors, freely
generates the set of processes, $\Proc$. This set is then quotiented
by a relation known as structural congruence and it is over this set
that the notion of dynamics is expressed. This presentation is
essentially that of \cite{MeredithR05} with the addition of
polyadicity and summation. For readability we have relegated some of
the technical subtleties to an appendix.

\subsubsection{Process grammar}\label{subsub:process_grammar}

\begin{mathpar}
  \inferrule* [lab=synchronization] {} {{M} \bc \pzero \;|\; x?F \;|\; x!C }
  \and
  \inferrule* [lab=abstraction] {} {{F} \bc (x)P}
  \and
  \inferrule* [lab=concretion] {} {{C} \bc \langle Q \rangle}
  \and
  \inferrule* [lab=process] {} {{P,Q} \bc M \;| \;P|Q \;|\; @{x}}
  \and
  \inferrule* [lab=name] {} {{x} \bc \quotep{P}}
\end{mathpar} 

Note that $\vec{x}$ (resp. $\vec{P}$) denotes a vector of names
(resp. processes) of length $|\vec{x}|$ (resp. $|\vec{P}|$). We adopt
the following useful abbreviations.

\begin{mathpar}
   x?(\vec{y}).P := x.(\vec{y})P \and  x\clift{\vec{P}} := x.\clift{\vec{P}}
   \and x!(y) := \lift{x}{\dropn{y}}
   \and \Pi_{i=0}^{n-1}P_i := P_0 | \ldots | P_{n-1}
\end{mathpar}

\subsubsection{Structural congruence}

\paragraph{Free and bound names and alpha-equivalence.} At the
core of structural equivalence is alpha-equivalence which identifies
process that are the same up to a change of variable. Formally, we
recognize the distinction between free and bound names. The free names
of a process, $\freenames{P}$, may be calculated recursively as
follows:

\begin{mathpar}
\freenames{\pzero} := \emptyset
  \and \\
  \freenames{x?(y).P} := \{ x \} \cup (\freenames{P} \setminus \{ y \})
  \and 
  \freenames{x!\langle P \rangle} := \{ x \} \cup \{ P \} 
  \and \\
  \freenames{P|Q} := \freenames{P} \cup \freenames{Q}
  \and \\
  \freenames{@{x}} := \{ x \}
\end{mathpar}

$\pi$
$\quotep{\pi}$

$\freenames{-} : \pi \to \mathcal{P}(\quotep{\pi})$

\begin{eqnarray*}
  \freenames{\pzero} & := & \emptyset \\
  \freenames{x?(y).P} & := & \{ x \} \cup (\freenames{P} \setminus \{ y \}) \\
  \freenames{x!\langle P \rangle} & := & \{ x \} \cup \{ P \} \\
  \freenames{P|Q} & := & \freenames{P} \cup \freenames{Q} \\
  \freenames{\dropn{x}} & := & \{ x \}
\end{eqnarray*}

The bound names of a process, $\boundnames{P}$, are those names occurring in $P$
that are not free. For example, in $x?(y).0$, the name $x$ is free, while $y$ is bound.

\begin{mathpar}
  \inferrule* [lab=monoidal-laws] {} { P|Q \equiv Q|P \and P|0 \equiv P \and P|(Q|R) \equiv (P|Q)|R }
\end{mathpar}

\begin{mathpar}
  \inferrule* [lab=alpha-equivalence] {} { (x)P \equiv (y)P\{y/x\} \and y \not\in \freenames{P} }
\end{mathpar}

\begin{definition}
Then two processes, $P,Q$, are alpha-equivalent if $P = Q\{\vec{y}/\vec{x}\}$ for
some $\vec{x} \in \boundnames{Q},\vec{y} \in \boundnames{P}$, where $Q\{\vec{y}/\vec{x}\}$
denotes the capture-avoiding substitution of $\vec{y}$ for $\vec{x}$ in $Q$.
\end{definition}

\begin{definition}
  The {\em structural congruence} \cite{SangiorgiWalker} , $\equiv$,
  between processes is the least congruence containing
  alpha-equivalence, satisfying the abelian monoid laws
  (associativity, commutativity and $\pzero$ as identity) for parallel
  composition $|$ and for summation $+$.
\end{definition}

\subsection{Name equivalence}

We take name equivalence, written $\nameeq$, to be the smallest
equivalence relation generated by the following rules.

\begin{mathpar}
\inferrule*[lab=Quote-drop]
{ }
{ \quotep{@{x}} \nameeq x }

\inferrule*[lab=Struct-equiv]
{ P \scong Q }
{ \quotep{P} \nameeq \quotep{Q} }
\end{mathpar}

The astute reader will have noticed that the mutual recursion of names
and processes imposes a mutual recursion on alpha-equivalence and
structural equivalence via name-equivalence. Fortunately, all of this
works out pleasantly and we may calculate in the natural way, free of
concern. The reader interested in the details is referred to the
appendix \ref{appendix:rho_details}.

\subsection{Substitution}

We use $\Proc$ for the set of processes, $\QProc$ for the set of
names, and $\id{\{}\vec{y} / \vec{x} \id{\}}$ to denote partial maps,
$s : \QProc \rightarrow \QProc$. A map, $s$ lifts, uniquely, to a map
on process terms, $\widehat{s} : \Proc \rightarrow \Proc$ by the
following equations.

\begin{mathpar}
  (0) \psubstp{Q}{P} := 0 \\
  (R \juxtap S) \psubstp{Q}{P}
  :=    
  (R)\psubstp{Q}{P} \juxtap (S) \psubstp{Q}{P} \\
  (x?(y).R) \psubstp{Q}{P}    
  :=    
  (x)\substp{Q}{P} (z)\concat( (R \psubstn{z}{y}) \psubstp{Q}{P} ) \\
  (\lift{x}{R}) \psubstp{Q}{P}  
  :=
  \lift{(x)\substp{Q}{P}}{ R \psubstp{Q}{P} } \\
%   (\dropn{x})  \psubstp{Q}{P}       
%   := 
%   \left\{ 
%     \begin{array}{ccc} 
%       \dropn{\quotep{Q}} & & x \nameeq \quotep{P} \\
%       \dropn{x} & & otherwise \\
%     \end{array}
%   \right. 
  (\dropn{x})  \psubstp{Q}{P}       
  := 
  \left\{ 
    \begin{array}{ccc} 
      Q & & x \nameeq \quotep{P} \\
      \dropn{x} & & otherwise \\
    \end{array}
  \right.
\end{mathpar}
 

where

\begin{eqnarray}
  (x)\id{\{} \lpquote Q \rpquote / \lpquote P \rpquote \id{\}}            = 
  \left\{ 
    \begin{array}{ccc}
      \lpquote Q \rpquote & & x \nameeq \lpquote P \rpquote \\
      x & & otherwise \\
    \end{array}
  \right. \nonumber
\end{eqnarray}

and $z$ is chosen distinct from $\quotep{P}$, $\quotep{Q}$, the free
names in $Q$, and all the names in $R$. Our $\alpha$-equivalence will
be built in the standard way from this substitution.

\begin{remark}\label{rem:no_self_referential_names}
  One consequence of these definitions is that $\forall P. \quotep{P}
  \not\in \freenames{P}$.
\end{remark}

\subsection{ Dynamic quote: an example }

Anticipating something of what's to come, consider applying the
substitution, $\widehat{\id{\{}u / z \id{\}}}$, to the following pair
of processes, $\lift{w}{y!(z)}$ and $w[ \lpquote y!(z) \rpquote ]$.

\begin{eqnarray}
	\lift{w}{y!(z)}\widehat{\id{\{}u / z \id{\}}}
		& = &
		\lift{w}{y!(u)} \nonumber\\
	w[ \lpquote y!(z) \rpquote ] \widehat{ \id{\{}u / z \id{\}} }
		& = &
		w[ \lpquote y!(z) \rpquote ] \nonumber
\end{eqnarray}

Because the body of the process between quotes is impervious to
substitution, we get radically different answers. In fact, by
examining the first process in an input context,
e.g. $x?(z).\lift{w}{y!(z)}$, we see that the process under the lift
operator may be shaped by prefixed inputs binding a name inside it. In
this sense, the lift operator will be seen as a way to dynamically
construct processes before reifying them as names.

Finally equipped with these standard features we can present the
dynamics of the calculus.

\subsubsection{Operational semantics} 

Finally, we introduce the computational dynamics. What marks these
algebras as distinct from other more traditionally studied algebraic
structures, e.g. vector spaces or polynomial rings, is the manner in
which dynamics is captured. In traditional structures, dynamics is typically
expressed through morphisms between such structures, as in linear maps
between vector spaces or morphisms between rings. In algebras
associated with the semantics of computation, the dynamics is
expressed as part of the algebraic structure itself, through a
reduction reduction relation typically denoted by $\red$. Below, we
give a recursive presentation of this relation for the calculus used
in the encoding.

$\red \subseteq \pi \times \pi$
$\red : \pi \to \mathcal{P}(\pi)$

\begin{mathpar}
  \inferrule* [lab=Comm] { \textsf{match}( x_{src}, x_{trgt} ) } { x_{trgt}?(y)P \; | \; x_{src}!\langle {Q} \rangle \red P\{\quotep{Q}/y}\} }
  \and \\
  \inferrule* [lab=Par] {{P} \red {P}'} {{{P} | {Q}} \red {{P}' | {Q}}}
  \and
  \inferrule* [lab=Equiv]{{{P} \scong {P}'} \andalso {{P}' \red {Q}'} \andalso {{Q}' \scong {Q}}}{{P} \red {Q}}
\end{mathpar}

\begin{eqnarray*}
  match_{\equiv} (\quotep{P},\quotep{Q}) & := & P \equiv Q \\
  match_{\dagger}(\quotep{P},\quotep{Q}) & := & \forall R. P|Q \red^{*} R => R \red^{*} 0 \\
  match_{K}(\quotep{P},\quotep{Q}) & := & K \mbox{ for some context } K
\end{eqnarray*}

$u?(x)P | u!\langle Q \rangle \red P\{\quotep{Q}/x\}$

%We write $\wred$ for $\red^*$, and $P\red$ if $\exists Q $ such that $ P \red Q$.
We write $P\red$ if $\exists Q $ such that $ P \red Q$ and $P\not\red$, otherwise.

\section{Replication}

As mentioned before, it is known that replication (and hence
recursion) can be implemented in a higher-order process algebra
\cite{SangiorgiWalker}. As our first example of calculation with the
machinery thus far presented we give the construction explicitly in
the {\rhoc}.

\begin{eqnarray}
	D_{x} & := & \prefix{x}{y}{(\binpar{\outputp{x}{y}}{@{y}})} \nonumber\\
	\bangp_{x}{P} & := & \binpar{{x}!\langle{\binpar{D_{x}}{P}}\rangle}{D_{x}} \nonumber
\end{eqnarray}

\begin{eqnarray}
	\bangp_{x}{P} & & \nonumber\\
	=
	& {x}!\langle{(\prefix{x}{y}{(\outputp{x}{y} | @{y})) | P}}\rangle 
	      | \prefix{x}{y}{(\outputp{x}{y} | @{y})} & \nonumber\\
	\red
	& (\outputp{x}{y} | @{y})\substn{\quotep{(\prefix{x}{y}{(@{y} | \outputp{x}{y})) | P}}}{y} & \nonumber\\
	=
	& \outputp{x}{\quotep{(\prefix{x}{y}{(\outputp{x}{y} | @{y})) | P}}}
	  | {(\prefix{x}{y}{(\outputp{x}{y} | @{y})) | P}} & \nonumber\\
	\red
	& \ldots & \nonumber\\
	\red^*
	& P | P | \ldots & \nonumber
\end{eqnarray}

Of course, this encoding, as an implementation, runs away, unfolding
$\bangp{P}$ eagerly. A lazier and more implementable replication
operator, restricted to input-guarded processes, may be obtained as follows.

\begin{eqnarray}
\bangp{\prefix{u}{v}{P}} 
	:= 
	\binpar{\lift{x}{\prefix{u}{v}{(\binpar{D(x)}{P})}}}{D(x)} \nonumber
\end{eqnarray}

\begin{remark}
  Note that the lazier definition still does not deal with summation
  or mixed summation (i.e. sums over input and output). The reader is
  invited to construct definitions of replication that deal with these
  features. 

  Further, the definitions are parameterized in a name, $x$. Can you,
  gentle reader, make a definition that eliminates this parameter and
  guarantees no accidental interaction between the replication
  machinery and the process being replicated -- i.e. no accidental
  sharing of names used by the process to get its work done and the
  name(s) used by the replication to effect copying. This latter
  revision of the definition of replication is crucial to obtaining
  the expected identity $!!P \sim !P$.
\end{remark}

\begin{remark}\label{rem:paradoxical_combinator}
  The reader familiar with the lambda calculus will have noticed the
  similarity between $D$ and the paradoxical combinator.

  [Ed. note: the existence of this seems to suggest we have to be more
  restrictive on the set of processes and names we admit if we are to
  support no-cloning.]
\end{remark}

\subsubsection{Bisimulation}

The computational dynamics gives rise to another kind of equivalence,
the equivalence of computational behavior. As previously mentioned
this is typically captured \emph{via} some form of bisimulation.

% The notion we use in this paper is weak barbed bisimulation
% \cite{milner91polyadicpi}.

The notion we use in this paper is derived from weak barbed
bisimulation \cite{milner91polyadicpi}. 

\begin{definition}
An \emph{observation relation}, $\downarrow_{\mathcal N}$, over a set
of names, $\mathcal N$, is the smallest relation satisfying the rules
below.

\infrule[Out-barb]{y \in {\mathcal N}, \; x \nameeq y}
		  {\outputp{x}{v} \downarrow_{\mathcal N} x}
\infrule[Par-barb]{\mbox{$P\downarrow_{\mathcal N} x$ or $Q\downarrow_{\mathcal N} x$}}
		  {\binpar{P}{Q} \downarrow_{\mathcal N} x}

We write $P \Downarrow_{\mathcal N} x$ if there is $Q$ such that 
$P \wred Q$ and $Q \downarrow_{\mathcal N} x$.
\end{definition}

\begin{definition}
%\label{def.bbisim}
An  ${\mathcal N}$-\emph{barbed bisimulation} over a set of names, ${\mathcal N}$, is a symmetric binary relation 
${\mathcal S}_{\mathcal N}$ between agents such that $P\rel{S}_{\mathcal N}Q$ implies:
\begin{enumerate}
\item If $P \red P'$ then $Q \wred Q'$ and $P'\rel{S}_{\mathcal N} Q'$.
\item If $P\downarrow_{\mathcal N} x$, then $Q\Downarrow_{\mathcal N} x$.
\end{enumerate}
$P$ is ${\mathcal N}$-barbed bisimilar to $Q$, written
$P \wbbisim_{\mathcal N} Q$, if $P \rel{S}_{\mathcal N} Q$ for some ${\mathcal N}$-barbed bisimulation ${\mathcal S}_{\mathcal N}$.
\end{definition}

$\mathcal{R} \subseteq \pi \times \pi$

$P \mathcal{R} Q => \forall P'. P \red P' \Rightarrow \exists Q'. Q \red Q', P' \mathcal{R} Q'$

$P \vdash x \Rightarrow Q \vdash x$

\begin{mathpar}
  \inferrule*[lab=Out-barb]{x \nameeq y}{{y}!\langle{Q}\rangle \vdash x}
  \and
  \inferrule*[lab=Par-barb]{\mbox{$P\vdash x$ or $Q\vdash x$}}{\binpar{P}{Q} \vdash x}
\end{mathpar}

\subsubsection{Contexts}

One of the principle advantages of computational calculi like the
$\pi$-calculus is a well-defined notion of context,
contextual-equivalence and a correlation between
contextual-equivalence and notions of bisimulation. The notion of
context allows the decomposition of a process into (sub-)process and
its syntactic environment, its context. Thus, a context may be
thought of as a process with a ``hole'' (written $\Box$) in it. The
application of a context $M$ to a process $P$, written $M[P]$, is
tantamount to filling the hole in $M$ with $P$. In this paper we do
not need the full weight of this theory, but do make use of the notion
of context in the proof the main theorem. 

\begin{mathpar}
  \inferrule* [lab=summation] {} {{M_{M},M_{N}} \bc \Box \;|\; x.M_{A} \;|\; M_{M}+M_{N}}
  \and
  \inferrule* [lab=agent] {} {{M_{A}} \bc (\vec{x})M_{P} \;| \; \clift{P_0,\ldots,M_{P},\ldots,P_N}}
  \and \\
  \inferrule* [lab=process] {} {{M_{P}} \bc M_{N} \;| \;P|M_{P} }
\end{mathpar} 

\begin{mathpar}
  \inferrule* [lab=sychronization] {} {M_{N} \bc \Box \;|\; x?M_{F} \;|\; x!M_{C}}
  \and
  \inferrule* [lab=abstraction] {} {{M_{F}} \bc (x)M_{P} }
  \and
  \inferrule* [lab=concretion] {} {{M_{C}} \bc \langle M_{P} \rangle }
  \and \\
  \inferrule* [lab=process] {} {{M_{P}} \bc M_{N} \;| \;P|M_{P} }
\end{mathpar}

\begin{definition}[contextual application] Given a context $M$, and
  process $P$, we define the \emph{contextual application}, $M[P] :=
  M\{P/\Box\}$. That is, the contextual application of M to P is the
  substitution of $P$ for $\Box$ in $M$.
\end{definition}

$\meaningof{-} : L \to \mathcal{P}(\pi)$

\begin{mathpar}
  \inferrule* [lab=collection] {} {\meaningof{true} = \pi, \and \meaningof{~E} = \pi \setminus \meaningof{E}, \and \meaningof{E_{1} \& E_{2}} = \meaningof{E_{1}} \cap \meaningof{E_{2}}}
\end{mathpar}

\begin{mathpar}
  \inferrule* [lab=structure] {} {\meaningof{0} = \{ P \in \pi | P \equiv 0 \}, \and \\ \meaningof{E_1 | E_2} = \{ P \in \pi | P \equiv P_{1} | P_{2}, P_{1} \in \meaningof{E_{1}}, P_{2} \in \meaningof{E_2}\} }
\end{mathpar}

\begin{mathpar}
 \inferrule* [lab=behavior] {} {\meaningof{\langle a?b \rangle E} = \{ P \in \pi | P \equiv Q | u?(y)P', \\ \and \\\\ \and \\ \;\;\; u \in \meaningof{a}, \forall z.P'\{z/y\} \in \meaningof{E\{z/b\}}\}, \and \\ \meaningof{a!E} = \{ P \in \pi | P \equiv Q | x!\langle P' \rangle, x \in \meaningof{a} P' \in \meaningof{E}\} }
\end{mathpar}

\begin{mathpar}
 \inferrule* [lab=nominal] {} {\meaningof{\quotep{E}} = \{ \quotep{P} \in \quotep{\pi} | P \in \meaningof{E} \}, \and \meaningof{\quotep{P}} = \{ \quotep{Q} \in \quotep{\pi} | P \equiv Q \} \and \\ \meaningof{@\quotep{E}} = \{ P \in \pi | P \equiv @x, x \in \meaningof{E} \}}
\end{mathpar}

\begin{eqnarray*}
  \\
  \meaningof{-} : TS \to ST
\end{eqnarray*}

\begin{eqnarray*}
  \\
  L : TS \to ST
\end{eqnarray*}

\begin{eqnarray*}
  \\
  P \models E \iff P \in \meaningof{E}
\end{eqnarray*}

\begin{eqnarray*}
  P \approx_{L} Q \iff \forall E \in L. P \models E \iff Q \models E
\end{eqnarray*}

\begin{eqnarray*}
  P \approx_{K} Q
\end{eqnarray*}

\begin{eqnarray*}
  P \approx Q
\end{eqnarray*}

$\approx_{K} = \approx = \approx_{L}$

\subsubsection{Contextual duality}

Note that contexts extend the quotation operation to a family of
operations from processes to names. Given a context, $M$, we can
define a \emph{nominal context}, $\quotep{M}$ by $\quotep{M}[P] :=
\quotep{M[P]}$. To foreshadow what is to come we observe that these
operations enjoy a duality with processes very much like the duality
between vectors and maps from vectors to scalars.

Further, because the calculus is essentially higher-order, we have a
correspondence between contexts and processes. More specifically,
given a name $x$ and a context $M$ we can construct $M^{*}_{x}$ such
that 

\begin{mathpar}
  M^{*}_{x} | \lift{x}{P} \red M[P]
\end{mathpar}

namely,

\begin{mathpar}
  M^{*}_{x} := x?(u).M[\dropn{u}]
\end{mathpar}

The dependence of $M^{*}_{x}$ on a name makes it an abstraction, 

\begin{mathpar}
  M^{*} := (x)x?(u).M[\dropn{u}]
\end{mathpar}

\subsection{Additional notation}

It will sometimes be convenient to denote the process a name
quotes. We already have the notation $x = \quotep{P}$, but it will be
convenient to introduce an alternate notation, $\procn{x}$, when we
want to emphasize the connection to the use of the name. Note that, by
virtue of name equivalence, $\quotep{\procn{x}} \nameeq x$; so, the
notation is consistent with previous definitions.

Further, because names have structure it is possible to effect
substitutions on the basis of that structure. This means we need to
upgrade our notation for substitutions, which we accomplish by
adapting comprehension notation. Thus,

\begin{mathpar}
  P\{ y / x : x \in S \}
\end{mathpar}

is interpreted to mean the process derived from P by replacing (in a
capture-avoiding manner) each occurrence of $x$ in $S$ by $y$. For example,

\begin{mathpar}
  P\{ \quotep{\procn{x}|\procn{x}} / x : x \in \freenames{P} \}
\end{mathpar}

will replace each (occurrence) of a free name $x$ in $P$ by
$\quotep{\procn{x}|\procn{x}}$.

Also, we will avail ourselves of the notation $x^{L}$ and $x^{R}$ to
denote injections of a name into disjoint copies of the name
space. There are numerous ways to accomplish this. One example can be
found in \cite{MeredithR05}. This notation overloads to vectors of
names: $\vec{x}^{\pi} := (x_{i}^{\pi} \; : \; 0 \leq i < |\vec{x}| )$ where $\pi \in \{L,R\}$.

We also use $P^{\Box} := P|\Box$.

In \cite{MeredithR05} an interpretation of the new operator is
given. It turns out that there are several possible interpretations
all enjoying the requisite algebraic properties of the operator (see
\cite{milner91polyadicpi}). We will therefore make liberal use of
$(\nu\; \vec{x})P$.

% subsection the_syntax_and_semantics_of_the_notation_system (end)   

\input{qm2pi.qmops} 

\input{qm2pi.sterngerlach} 

\input{qm2pi.metric} 

% section concurrent_process_calculi (end)

%\input{qm2pi.proofsketch}

% section proof sketch (end)

%\input{qm2pi.slviaknots} 

% section spatial logic via knots (end)

\input{qm2pi.conclusion}

% section conclusion (end)

%\input{qm2pi.dtcodes} 

% section wiring algorithm (end)

\input{qm2pi.ack} 

% section acknowledgments (end)

\newpage


\bibliographystyle{plain}   
\bibliography{../../biblios/main.bib}

\input{qm2pi.rhodetails}

\end{document}

 

%\documentclass[12pt]{llncs}
%\documentclass{jktr}

\usepackage[pdftex]{hyperref}                   
\usepackage {listings}
\usepackage {mathpartir}
\usepackage{bcprules}
%\usepackage{listings}
                       
\usepackage{graphicx} 
%\usepackage[margins=2.5cm,nohead,nofoot]{geometry}
%\usepackage{geometry}
\usepackage{amsfonts}
\usepackage{amstext}
\usepackage{latexsym}
\usepackage{amssymb}
\usepackage{color}


%\include{myPreamble}
\include{qm2pi.local} 

%\ifpdf
%\usepackage[pdftex]{graphicx}
%\else
%\usepackage{graphicx}
%\fi

 % \ifpdf
%  \usepackage{pdfsync}
%  \if


%\title{Brief Article}
%\author{David F. Snyder}
%\author{L.G. Meredith}

%\address{Dept. of Math., Texas State University--San Marcos, San Marcos, TX 78666}
       
\pagestyle{empty}


\begin{document}

\lstset{language=[Objective]Caml,frame=shadowbox}

\input{qm2pi.front}

% section front matter (end)

\input{qm2pi.intro} 
 
% section introduction (end)

% \input{qm2pi.knotations} 

% section notation (end)

\input{qm2pi.process.calculi} 

% section concurrent_process_calculi_and_spatial_logics_ (end)
    
%\input{qm2pi.knots2pi} 

%\input{qm2pi.trefoil} 

%\input{qm2pi.mainthm} 

% subsection basic_interpretation (end)

%\input{qm2pi.rho.presentation} 
\subsection{The syntax and semantics of the notation system}\label{sub:the_syntax_and_semantics_of_the_notation_system} % (fold)

We now summarize a technical presentation of the calculus that
embodies our theory of dynamics. The typical presentation of such a
calculus follows the style of giving generators and relations on
them. The grammar, below, describing term constructors, freely
generates the set of processes, $\Proc$. This set is then quotiented
by a relation known as structural congruence and it is over this set
that the notion of dynamics is expressed. This presentation is
essentially that of \cite{MeredithR05} with the addition of
polyadicity and summation. For readability we have relegated some of
the technical subtleties to an appendix.

\subsubsection{Process grammar}\label{subsub:process_grammar}

\begin{mathpar}
  \inferrule* [lab=synchronization] {} {{M} \bc \pzero \;|\; x?F \;|\; x!C }
  \and
  \inferrule* [lab=abstraction] {} {{F} \bc (x)P}
  \and
  \inferrule* [lab=concretion] {} {{C} \bc \langle Q \rangle}
  \and
  \inferrule* [lab=process] {} {{P,Q} \bc M \;| \;P|Q \;|\; @{x}}
  \and
  \inferrule* [lab=name] {} {{x} \bc \quotep{P}}
\end{mathpar} 

Note that $\vec{x}$ (resp. $\vec{P}$) denotes a vector of names
(resp. processes) of length $|\vec{x}|$ (resp. $|\vec{P}|$). We adopt
the following useful abbreviations.

\begin{mathpar}
   x?(\vec{y}).P := x.(\vec{y})P \and  x\clift{\vec{P}} := x.\clift{\vec{P}}
   \and x!(y) := \lift{x}{\dropn{y}}
   \and \Pi_{i=0}^{n-1}P_i := P_0 | \ldots | P_{n-1}
\end{mathpar}

\subsubsection{Structural congruence}

\paragraph{Free and bound names and alpha-equivalence.} At the
core of structural equivalence is alpha-equivalence which identifies
process that are the same up to a change of variable. Formally, we
recognize the distinction between free and bound names. The free names
of a process, $\freenames{P}$, may be calculated recursively as
follows:

\begin{mathpar}
\freenames{\pzero} := \emptyset
  \and \\
  \freenames{x?(y).P} := \{ x \} \cup (\freenames{P} \setminus \{ y \})
  \and 
  \freenames{x!\langle P \rangle} := \{ x \} \cup \{ P \} 
  \and \\
  \freenames{P|Q} := \freenames{P} \cup \freenames{Q}
  \and \\
  \freenames{@{x}} := \{ x \}
\end{mathpar}

$\pi$
$\quotep{\pi}$

$\freenames{-} : \pi \to \mathcal{P}(\quotep{\pi})$

\begin{eqnarray*}
  \freenames{\pzero} & := & \emptyset \\
  \freenames{x?(y).P} & := & \{ x \} \cup (\freenames{P} \setminus \{ y \}) \\
  \freenames{x!\langle P \rangle} & := & \{ x \} \cup \{ P \} \\
  \freenames{P|Q} & := & \freenames{P} \cup \freenames{Q} \\
  \freenames{\dropn{x}} & := & \{ x \}
\end{eqnarray*}

The bound names of a process, $\boundnames{P}$, are those names occurring in $P$
that are not free. For example, in $x?(y).0$, the name $x$ is free, while $y$ is bound.

\begin{mathpar}
  \inferrule* [lab=monoidal-laws] {} { P|Q \equiv Q|P \and P|0 \equiv P \and P|(Q|R) \equiv (P|Q)|R }
\end{mathpar}

\begin{mathpar}
  \inferrule* [lab=alpha-equivalence] {} { (x)P \equiv (y)P\{y/x\} \and y \not\in \freenames{P} }
\end{mathpar}

\begin{definition}
Then two processes, $P,Q$, are alpha-equivalent if $P = Q\{\vec{y}/\vec{x}\}$ for
some $\vec{x} \in \boundnames{Q},\vec{y} \in \boundnames{P}$, where $Q\{\vec{y}/\vec{x}\}$
denotes the capture-avoiding substitution of $\vec{y}$ for $\vec{x}$ in $Q$.
\end{definition}

\begin{definition}
  The {\em structural congruence} \cite{SangiorgiWalker} , $\equiv$,
  between processes is the least congruence containing
  alpha-equivalence, satisfying the abelian monoid laws
  (associativity, commutativity and $\pzero$ as identity) for parallel
  composition $|$ and for summation $+$.
\end{definition}

\subsection{Name equivalence}

We take name equivalence, written $\nameeq$, to be the smallest
equivalence relation generated by the following rules.

\begin{mathpar}
\inferrule*[lab=Quote-drop]
{ }
{ \quotep{@{x}} \nameeq x }

\inferrule*[lab=Struct-equiv]
{ P \scong Q }
{ \quotep{P} \nameeq \quotep{Q} }
\end{mathpar}

The astute reader will have noticed that the mutual recursion of names
and processes imposes a mutual recursion on alpha-equivalence and
structural equivalence via name-equivalence. Fortunately, all of this
works out pleasantly and we may calculate in the natural way, free of
concern. The reader interested in the details is referred to the
appendix \ref{appendix:rho_details}.

\subsection{Substitution}

We use $\Proc$ for the set of processes, $\QProc$ for the set of
names, and $\id{\{}\vec{y} / \vec{x} \id{\}}$ to denote partial maps,
$s : \QProc \rightarrow \QProc$. A map, $s$ lifts, uniquely, to a map
on process terms, $\widehat{s} : \Proc \rightarrow \Proc$ by the
following equations.

\begin{mathpar}
  (0) \psubstp{Q}{P} := 0 \\
  (R \juxtap S) \psubstp{Q}{P}
  :=    
  (R)\psubstp{Q}{P} \juxtap (S) \psubstp{Q}{P} \\
  (x?(y).R) \psubstp{Q}{P}    
  :=    
  (x)\substp{Q}{P} (z)\concat( (R \psubstn{z}{y}) \psubstp{Q}{P} ) \\
  (\lift{x}{R}) \psubstp{Q}{P}  
  :=
  \lift{(x)\substp{Q}{P}}{ R \psubstp{Q}{P} } \\
%   (\dropn{x})  \psubstp{Q}{P}       
%   := 
%   \left\{ 
%     \begin{array}{ccc} 
%       \dropn{\quotep{Q}} & & x \nameeq \quotep{P} \\
%       \dropn{x} & & otherwise \\
%     \end{array}
%   \right. 
  (\dropn{x})  \psubstp{Q}{P}       
  := 
  \left\{ 
    \begin{array}{ccc} 
      Q & & x \nameeq \quotep{P} \\
      \dropn{x} & & otherwise \\
    \end{array}
  \right.
\end{mathpar}
 

where

\begin{eqnarray}
  (x)\id{\{} \lpquote Q \rpquote / \lpquote P \rpquote \id{\}}            = 
  \left\{ 
    \begin{array}{ccc}
      \lpquote Q \rpquote & & x \nameeq \lpquote P \rpquote \\
      x & & otherwise \\
    \end{array}
  \right. \nonumber
\end{eqnarray}

and $z$ is chosen distinct from $\quotep{P}$, $\quotep{Q}$, the free
names in $Q$, and all the names in $R$. Our $\alpha$-equivalence will
be built in the standard way from this substitution.

\begin{remark}\label{rem:no_self_referential_names}
  One consequence of these definitions is that $\forall P. \quotep{P}
  \not\in \freenames{P}$.
\end{remark}

\subsection{ Dynamic quote: an example }

Anticipating something of what's to come, consider applying the
substitution, $\widehat{\id{\{}u / z \id{\}}}$, to the following pair
of processes, $\lift{w}{y!(z)}$ and $w[ \lpquote y!(z) \rpquote ]$.

\begin{eqnarray}
	\lift{w}{y!(z)}\widehat{\id{\{}u / z \id{\}}}
		& = &
		\lift{w}{y!(u)} \nonumber\\
	w[ \lpquote y!(z) \rpquote ] \widehat{ \id{\{}u / z \id{\}} }
		& = &
		w[ \lpquote y!(z) \rpquote ] \nonumber
\end{eqnarray}

Because the body of the process between quotes is impervious to
substitution, we get radically different answers. In fact, by
examining the first process in an input context,
e.g. $x?(z).\lift{w}{y!(z)}$, we see that the process under the lift
operator may be shaped by prefixed inputs binding a name inside it. In
this sense, the lift operator will be seen as a way to dynamically
construct processes before reifying them as names.

Finally equipped with these standard features we can present the
dynamics of the calculus.

\subsubsection{Operational semantics} 

Finally, we introduce the computational dynamics. What marks these
algebras as distinct from other more traditionally studied algebraic
structures, e.g. vector spaces or polynomial rings, is the manner in
which dynamics is captured. In traditional structures, dynamics is typically
expressed through morphisms between such structures, as in linear maps
between vector spaces or morphisms between rings. In algebras
associated with the semantics of computation, the dynamics is
expressed as part of the algebraic structure itself, through a
reduction reduction relation typically denoted by $\red$. Below, we
give a recursive presentation of this relation for the calculus used
in the encoding.

$\red \subseteq \pi \times \pi$
$\red : \pi \to \mathcal{P}(\pi)$

\begin{mathpar}
  \inferrule* [lab=Comm] { \textsf{match}( x_{src}, x_{trgt} ) } { x_{trgt}?(y)P \; | \; x_{src}!\langle {Q} \rangle \red P\{\quotep{Q}/y}\} }
  \and \\
  \inferrule* [lab=Par] {{P} \red {P}'} {{{P} | {Q}} \red {{P}' | {Q}}}
  \and
  \inferrule* [lab=Equiv]{{{P} \scong {P}'} \andalso {{P}' \red {Q}'} \andalso {{Q}' \scong {Q}}}{{P} \red {Q}}
\end{mathpar}

\begin{eqnarray*}
  match_{\equiv} (\quotep{P},\quotep{Q}) & := & P \equiv Q \\
  match_{\dagger}(\quotep{P},\quotep{Q}) & := & \forall R. P|Q \red^{*} R => R \red^{*} 0 \\
  match_{K}(\quotep{P},\quotep{Q}) & := & K \mbox{ for some context } K
\end{eqnarray*}

$u?(x)P | u!\langle Q \rangle \red P\{\quotep{Q}/x\}$

%We write $\wred$ for $\red^*$, and $P\red$ if $\exists Q $ such that $ P \red Q$.
We write $P\red$ if $\exists Q $ such that $ P \red Q$ and $P\not\red$, otherwise.

\section{Replication}

As mentioned before, it is known that replication (and hence
recursion) can be implemented in a higher-order process algebra
\cite{SangiorgiWalker}. As our first example of calculation with the
machinery thus far presented we give the construction explicitly in
the {\rhoc}.

\begin{eqnarray}
	D_{x} & := & \prefix{x}{y}{(\binpar{\outputp{x}{y}}{@{y}})} \nonumber\\
	\bangp_{x}{P} & := & \binpar{{x}!\langle{\binpar{D_{x}}{P}}\rangle}{D_{x}} \nonumber
\end{eqnarray}

\begin{eqnarray}
	\bangp_{x}{P} & & \nonumber\\
	=
	& {x}!\langle{(\prefix{x}{y}{(\outputp{x}{y} | @{y})) | P}}\rangle 
	      | \prefix{x}{y}{(\outputp{x}{y} | @{y})} & \nonumber\\
	\red
	& (\outputp{x}{y} | @{y})\substn{\quotep{(\prefix{x}{y}{(@{y} | \outputp{x}{y})) | P}}}{y} & \nonumber\\
	=
	& \outputp{x}{\quotep{(\prefix{x}{y}{(\outputp{x}{y} | @{y})) | P}}}
	  | {(\prefix{x}{y}{(\outputp{x}{y} | @{y})) | P}} & \nonumber\\
	\red
	& \ldots & \nonumber\\
	\red^*
	& P | P | \ldots & \nonumber
\end{eqnarray}

Of course, this encoding, as an implementation, runs away, unfolding
$\bangp{P}$ eagerly. A lazier and more implementable replication
operator, restricted to input-guarded processes, may be obtained as follows.

\begin{eqnarray}
\bangp{\prefix{u}{v}{P}} 
	:= 
	\binpar{\lift{x}{\prefix{u}{v}{(\binpar{D(x)}{P})}}}{D(x)} \nonumber
\end{eqnarray}

\begin{remark}
  Note that the lazier definition still does not deal with summation
  or mixed summation (i.e. sums over input and output). The reader is
  invited to construct definitions of replication that deal with these
  features. 

  Further, the definitions are parameterized in a name, $x$. Can you,
  gentle reader, make a definition that eliminates this parameter and
  guarantees no accidental interaction between the replication
  machinery and the process being replicated -- i.e. no accidental
  sharing of names used by the process to get its work done and the
  name(s) used by the replication to effect copying. This latter
  revision of the definition of replication is crucial to obtaining
  the expected identity $!!P \sim !P$.
\end{remark}

\begin{remark}\label{rem:paradoxical_combinator}
  The reader familiar with the lambda calculus will have noticed the
  similarity between $D$ and the paradoxical combinator.

  [Ed. note: the existence of this seems to suggest we have to be more
  restrictive on the set of processes and names we admit if we are to
  support no-cloning.]
\end{remark}

\subsubsection{Bisimulation}

The computational dynamics gives rise to another kind of equivalence,
the equivalence of computational behavior. As previously mentioned
this is typically captured \emph{via} some form of bisimulation.

% The notion we use in this paper is weak barbed bisimulation
% \cite{milner91polyadicpi}.

The notion we use in this paper is derived from weak barbed
bisimulation \cite{milner91polyadicpi}. 

\begin{definition}
An \emph{observation relation}, $\downarrow_{\mathcal N}$, over a set
of names, $\mathcal N$, is the smallest relation satisfying the rules
below.

\infrule[Out-barb]{y \in {\mathcal N}, \; x \nameeq y}
		  {\outputp{x}{v} \downarrow_{\mathcal N} x}
\infrule[Par-barb]{\mbox{$P\downarrow_{\mathcal N} x$ or $Q\downarrow_{\mathcal N} x$}}
		  {\binpar{P}{Q} \downarrow_{\mathcal N} x}

We write $P \Downarrow_{\mathcal N} x$ if there is $Q$ such that 
$P \wred Q$ and $Q \downarrow_{\mathcal N} x$.
\end{definition}

\begin{definition}
%\label{def.bbisim}
An  ${\mathcal N}$-\emph{barbed bisimulation} over a set of names, ${\mathcal N}$, is a symmetric binary relation 
${\mathcal S}_{\mathcal N}$ between agents such that $P\rel{S}_{\mathcal N}Q$ implies:
\begin{enumerate}
\item If $P \red P'$ then $Q \wred Q'$ and $P'\rel{S}_{\mathcal N} Q'$.
\item If $P\downarrow_{\mathcal N} x$, then $Q\Downarrow_{\mathcal N} x$.
\end{enumerate}
$P$ is ${\mathcal N}$-barbed bisimilar to $Q$, written
$P \wbbisim_{\mathcal N} Q$, if $P \rel{S}_{\mathcal N} Q$ for some ${\mathcal N}$-barbed bisimulation ${\mathcal S}_{\mathcal N}$.
\end{definition}

$\mathcal{R} \subseteq \pi \times \pi$

$P \mathcal{R} Q => \forall P'. P \red P' \Rightarrow \exists Q'. Q \red Q', P' \mathcal{R} Q'$

$P \vdash x \Rightarrow Q \vdash x$

\begin{mathpar}
  \inferrule*[lab=Out-barb]{x \nameeq y}{{y}!\langle{Q}\rangle \vdash x}
  \and
  \inferrule*[lab=Par-barb]{\mbox{$P\vdash x$ or $Q\vdash x$}}{\binpar{P}{Q} \vdash x}
\end{mathpar}

\subsubsection{Contexts}

One of the principle advantages of computational calculi like the
$\pi$-calculus is a well-defined notion of context,
contextual-equivalence and a correlation between
contextual-equivalence and notions of bisimulation. The notion of
context allows the decomposition of a process into (sub-)process and
its syntactic environment, its context. Thus, a context may be
thought of as a process with a ``hole'' (written $\Box$) in it. The
application of a context $M$ to a process $P$, written $M[P]$, is
tantamount to filling the hole in $M$ with $P$. In this paper we do
not need the full weight of this theory, but do make use of the notion
of context in the proof the main theorem. 

\begin{mathpar}
  \inferrule* [lab=summation] {} {{M_{M},M_{N}} \bc \Box \;|\; x.M_{A} \;|\; M_{M}+M_{N}}
  \and
  \inferrule* [lab=agent] {} {{M_{A}} \bc (\vec{x})M_{P} \;| \; \clift{P_0,\ldots,M_{P},\ldots,P_N}}
  \and \\
  \inferrule* [lab=process] {} {{M_{P}} \bc M_{N} \;| \;P|M_{P} }
\end{mathpar} 

\begin{mathpar}
  \inferrule* [lab=sychronization] {} {M_{N} \bc \Box \;|\; x?M_{F} \;|\; x!M_{C}}
  \and
  \inferrule* [lab=abstraction] {} {{M_{F}} \bc (x)M_{P} }
  \and
  \inferrule* [lab=concretion] {} {{M_{C}} \bc \langle M_{P} \rangle }
  \and \\
  \inferrule* [lab=process] {} {{M_{P}} \bc M_{N} \;| \;P|M_{P} }
\end{mathpar}

\begin{definition}[contextual application] Given a context $M$, and
  process $P$, we define the \emph{contextual application}, $M[P] :=
  M\{P/\Box\}$. That is, the contextual application of M to P is the
  substitution of $P$ for $\Box$ in $M$.
\end{definition}

$\meaningof{-} : L \to \mathcal{P}(\pi)$

\begin{mathpar}
  \inferrule* [lab=collection] {} {\meaningof{true} = \pi, \and \meaningof{~E} = \pi \setminus \meaningof{E}, \and \meaningof{E_{1} \& E_{2}} = \meaningof{E_{1}} \cap \meaningof{E_{2}}}
\end{mathpar}

\begin{mathpar}
  \inferrule* [lab=structure] {} {\meaningof{0} = \{ P \in \pi | P \equiv 0 \}, \and \\ \meaningof{E_1 | E_2} = \{ P \in \pi | P \equiv P_{1} | P_{2}, P_{1} \in \meaningof{E_{1}}, P_{2} \in \meaningof{E_2}\} }
\end{mathpar}

\begin{mathpar}
 \inferrule* [lab=behavior] {} {\meaningof{\langle a?b \rangle E} = \{ P \in \pi | P \equiv Q | u?(y)P', \\ \and \\\\ \and \\ \;\;\; u \in \meaningof{a}, \forall z.P'\{z/y\} \in \meaningof{E\{z/b\}}\}, \and \\ \meaningof{a!E} = \{ P \in \pi | P \equiv Q | x!\langle P' \rangle, x \in \meaningof{a} P' \in \meaningof{E}\} }
\end{mathpar}

\begin{mathpar}
 \inferrule* [lab=nominal] {} {\meaningof{\quotep{E}} = \{ \quotep{P} \in \quotep{\pi} | P \in \meaningof{E} \}, \and \meaningof{\quotep{P}} = \{ \quotep{Q} \in \quotep{\pi} | P \equiv Q \} \and \\ \meaningof{@\quotep{E}} = \{ P \in \pi | P \equiv @x, x \in \meaningof{E} \}}
\end{mathpar}

\begin{eqnarray*}
  \\
  \meaningof{-} : TS \to ST
\end{eqnarray*}

\begin{eqnarray*}
  \\
  L : TS \to ST
\end{eqnarray*}

\begin{eqnarray*}
  \\
  P \models E \iff P \in \meaningof{E}
\end{eqnarray*}

\begin{eqnarray*}
  P \approx_{L} Q \iff \forall E \in L. P \models E \iff Q \models E
\end{eqnarray*}

\begin{eqnarray*}
  P \approx_{K} Q
\end{eqnarray*}

\begin{eqnarray*}
  P \approx Q
\end{eqnarray*}

$\approx_{K} = \approx = \approx_{L}$

\subsubsection{Contextual duality}

Note that contexts extend the quotation operation to a family of
operations from processes to names. Given a context, $M$, we can
define a \emph{nominal context}, $\quotep{M}$ by $\quotep{M}[P] :=
\quotep{M[P]}$. To foreshadow what is to come we observe that these
operations enjoy a duality with processes very much like the duality
between vectors and maps from vectors to scalars.

Further, because the calculus is essentially higher-order, we have a
correspondence between contexts and processes. More specifically,
given a name $x$ and a context $M$ we can construct $M^{*}_{x}$ such
that 

\begin{mathpar}
  M^{*}_{x} | \lift{x}{P} \red M[P]
\end{mathpar}

namely,

\begin{mathpar}
  M^{*}_{x} := x?(u).M[\dropn{u}]
\end{mathpar}

The dependence of $M^{*}_{x}$ on a name makes it an abstraction, 

\begin{mathpar}
  M^{*} := (x)x?(u).M[\dropn{u}]
\end{mathpar}

\subsection{Additional notation}

It will sometimes be convenient to denote the process a name
quotes. We already have the notation $x = \quotep{P}$, but it will be
convenient to introduce an alternate notation, $\procn{x}$, when we
want to emphasize the connection to the use of the name. Note that, by
virtue of name equivalence, $\quotep{\procn{x}} \nameeq x$; so, the
notation is consistent with previous definitions.

Further, because names have structure it is possible to effect
substitutions on the basis of that structure. This means we need to
upgrade our notation for substitutions, which we accomplish by
adapting comprehension notation. Thus,

\begin{mathpar}
  P\{ y / x : x \in S \}
\end{mathpar}

is interpreted to mean the process derived from P by replacing (in a
capture-avoiding manner) each occurrence of $x$ in $S$ by $y$. For example,

\begin{mathpar}
  P\{ \quotep{\procn{x}|\procn{x}} / x : x \in \freenames{P} \}
\end{mathpar}

will replace each (occurrence) of a free name $x$ in $P$ by
$\quotep{\procn{x}|\procn{x}}$.

Also, we will avail ourselves of the notation $x^{L}$ and $x^{R}$ to
denote injections of a name into disjoint copies of the name
space. There are numerous ways to accomplish this. One example can be
found in \cite{MeredithR05}. This notation overloads to vectors of
names: $\vec{x}^{\pi} := (x_{i}^{\pi} \; : \; 0 \leq i < |\vec{x}| )$ where $\pi \in \{L,R\}$.

We also use $P^{\Box} := P|\Box$.

In \cite{MeredithR05} an interpretation of the new operator is
given. It turns out that there are several possible interpretations
all enjoying the requisite algebraic properties of the operator (see
\cite{milner91polyadicpi}). We will therefore make liberal use of
$(\nu\; \vec{x})P$.

% subsection the_syntax_and_semantics_of_the_notation_system (end)   

\input{qm2pi.qmops} 

\input{qm2pi.sterngerlach} 

\input{qm2pi.metric} 

% section concurrent_process_calculi (end)

%\input{qm2pi.proofsketch}

% section proof sketch (end)

%\input{qm2pi.slviaknots} 

% section spatial logic via knots (end)

\input{qm2pi.conclusion}

% section conclusion (end)

%\input{qm2pi.dtcodes} 

% section wiring algorithm (end)

\input{qm2pi.ack} 

% section acknowledgments (end)

\newpage


\bibliographystyle{plain}   
\bibliography{../../biblios/main.bib}

\input{qm2pi.rhodetails}

\end{document}

 

%\documentclass[12pt]{llncs}
%\documentclass{jktr}

\usepackage[pdftex]{hyperref}                   
\usepackage {listings}
\usepackage {mathpartir}
\usepackage{bcprules}
%\usepackage{listings}
                       
\usepackage{graphicx} 
%\usepackage[margins=2.5cm,nohead,nofoot]{geometry}
%\usepackage{geometry}
\usepackage{amsfonts}
\usepackage{amstext}
\usepackage{latexsym}
\usepackage{amssymb}
\usepackage{color}


%\include{myPreamble}
\include{qm2pi.local} 

%\ifpdf
%\usepackage[pdftex]{graphicx}
%\else
%\usepackage{graphicx}
%\fi

 % \ifpdf
%  \usepackage{pdfsync}
%  \if


%\title{Brief Article}
%\author{David F. Snyder}
%\author{L.G. Meredith}

%\address{Dept. of Math., Texas State University--San Marcos, San Marcos, TX 78666}
       
\pagestyle{empty}


\begin{document}

\lstset{language=[Objective]Caml,frame=shadowbox}

\input{qm2pi.front}

% section front matter (end)

\input{qm2pi.intro} 
 
% section introduction (end)

% \input{qm2pi.knotations} 

% section notation (end)

\input{qm2pi.process.calculi} 

% section concurrent_process_calculi_and_spatial_logics_ (end)
    
%\input{qm2pi.knots2pi} 

%\input{qm2pi.trefoil} 

%\input{qm2pi.mainthm} 

% subsection basic_interpretation (end)

%\input{qm2pi.rho.presentation} 
\subsection{The syntax and semantics of the notation system}\label{sub:the_syntax_and_semantics_of_the_notation_system} % (fold)

We now summarize a technical presentation of the calculus that
embodies our theory of dynamics. The typical presentation of such a
calculus follows the style of giving generators and relations on
them. The grammar, below, describing term constructors, freely
generates the set of processes, $\Proc$. This set is then quotiented
by a relation known as structural congruence and it is over this set
that the notion of dynamics is expressed. This presentation is
essentially that of \cite{MeredithR05} with the addition of
polyadicity and summation. For readability we have relegated some of
the technical subtleties to an appendix.

\subsubsection{Process grammar}\label{subsub:process_grammar}

\begin{mathpar}
  \inferrule* [lab=synchronization] {} {{M} \bc \pzero \;|\; x?F \;|\; x!C }
  \and
  \inferrule* [lab=abstraction] {} {{F} \bc (x)P}
  \and
  \inferrule* [lab=concretion] {} {{C} \bc \langle Q \rangle}
  \and
  \inferrule* [lab=process] {} {{P,Q} \bc M \;| \;P|Q \;|\; @{x}}
  \and
  \inferrule* [lab=name] {} {{x} \bc \quotep{P}}
\end{mathpar} 

Note that $\vec{x}$ (resp. $\vec{P}$) denotes a vector of names
(resp. processes) of length $|\vec{x}|$ (resp. $|\vec{P}|$). We adopt
the following useful abbreviations.

\begin{mathpar}
   x?(\vec{y}).P := x.(\vec{y})P \and  x\clift{\vec{P}} := x.\clift{\vec{P}}
   \and x!(y) := \lift{x}{\dropn{y}}
   \and \Pi_{i=0}^{n-1}P_i := P_0 | \ldots | P_{n-1}
\end{mathpar}

\subsubsection{Structural congruence}

\paragraph{Free and bound names and alpha-equivalence.} At the
core of structural equivalence is alpha-equivalence which identifies
process that are the same up to a change of variable. Formally, we
recognize the distinction between free and bound names. The free names
of a process, $\freenames{P}$, may be calculated recursively as
follows:

\begin{mathpar}
\freenames{\pzero} := \emptyset
  \and \\
  \freenames{x?(y).P} := \{ x \} \cup (\freenames{P} \setminus \{ y \})
  \and 
  \freenames{x!\langle P \rangle} := \{ x \} \cup \{ P \} 
  \and \\
  \freenames{P|Q} := \freenames{P} \cup \freenames{Q}
  \and \\
  \freenames{@{x}} := \{ x \}
\end{mathpar}

$\pi$
$\quotep{\pi}$

$\freenames{-} : \pi \to \mathcal{P}(\quotep{\pi})$

\begin{eqnarray*}
  \freenames{\pzero} & := & \emptyset \\
  \freenames{x?(y).P} & := & \{ x \} \cup (\freenames{P} \setminus \{ y \}) \\
  \freenames{x!\langle P \rangle} & := & \{ x \} \cup \{ P \} \\
  \freenames{P|Q} & := & \freenames{P} \cup \freenames{Q} \\
  \freenames{\dropn{x}} & := & \{ x \}
\end{eqnarray*}

The bound names of a process, $\boundnames{P}$, are those names occurring in $P$
that are not free. For example, in $x?(y).0$, the name $x$ is free, while $y$ is bound.

\begin{mathpar}
  \inferrule* [lab=monoidal-laws] {} { P|Q \equiv Q|P \and P|0 \equiv P \and P|(Q|R) \equiv (P|Q)|R }
\end{mathpar}

\begin{mathpar}
  \inferrule* [lab=alpha-equivalence] {} { (x)P \equiv (y)P\{y/x\} \and y \not\in \freenames{P} }
\end{mathpar}

\begin{definition}
Then two processes, $P,Q$, are alpha-equivalent if $P = Q\{\vec{y}/\vec{x}\}$ for
some $\vec{x} \in \boundnames{Q},\vec{y} \in \boundnames{P}$, where $Q\{\vec{y}/\vec{x}\}$
denotes the capture-avoiding substitution of $\vec{y}$ for $\vec{x}$ in $Q$.
\end{definition}

\begin{definition}
  The {\em structural congruence} \cite{SangiorgiWalker} , $\equiv$,
  between processes is the least congruence containing
  alpha-equivalence, satisfying the abelian monoid laws
  (associativity, commutativity and $\pzero$ as identity) for parallel
  composition $|$ and for summation $+$.
\end{definition}

\subsection{Name equivalence}

We take name equivalence, written $\nameeq$, to be the smallest
equivalence relation generated by the following rules.

\begin{mathpar}
\inferrule*[lab=Quote-drop]
{ }
{ \quotep{@{x}} \nameeq x }

\inferrule*[lab=Struct-equiv]
{ P \scong Q }
{ \quotep{P} \nameeq \quotep{Q} }
\end{mathpar}

The astute reader will have noticed that the mutual recursion of names
and processes imposes a mutual recursion on alpha-equivalence and
structural equivalence via name-equivalence. Fortunately, all of this
works out pleasantly and we may calculate in the natural way, free of
concern. The reader interested in the details is referred to the
appendix \ref{appendix:rho_details}.

\subsection{Substitution}

We use $\Proc$ for the set of processes, $\QProc$ for the set of
names, and $\id{\{}\vec{y} / \vec{x} \id{\}}$ to denote partial maps,
$s : \QProc \rightarrow \QProc$. A map, $s$ lifts, uniquely, to a map
on process terms, $\widehat{s} : \Proc \rightarrow \Proc$ by the
following equations.

\begin{mathpar}
  (0) \psubstp{Q}{P} := 0 \\
  (R \juxtap S) \psubstp{Q}{P}
  :=    
  (R)\psubstp{Q}{P} \juxtap (S) \psubstp{Q}{P} \\
  (x?(y).R) \psubstp{Q}{P}    
  :=    
  (x)\substp{Q}{P} (z)\concat( (R \psubstn{z}{y}) \psubstp{Q}{P} ) \\
  (\lift{x}{R}) \psubstp{Q}{P}  
  :=
  \lift{(x)\substp{Q}{P}}{ R \psubstp{Q}{P} } \\
%   (\dropn{x})  \psubstp{Q}{P}       
%   := 
%   \left\{ 
%     \begin{array}{ccc} 
%       \dropn{\quotep{Q}} & & x \nameeq \quotep{P} \\
%       \dropn{x} & & otherwise \\
%     \end{array}
%   \right. 
  (\dropn{x})  \psubstp{Q}{P}       
  := 
  \left\{ 
    \begin{array}{ccc} 
      Q & & x \nameeq \quotep{P} \\
      \dropn{x} & & otherwise \\
    \end{array}
  \right.
\end{mathpar}
 

where

\begin{eqnarray}
  (x)\id{\{} \lpquote Q \rpquote / \lpquote P \rpquote \id{\}}            = 
  \left\{ 
    \begin{array}{ccc}
      \lpquote Q \rpquote & & x \nameeq \lpquote P \rpquote \\
      x & & otherwise \\
    \end{array}
  \right. \nonumber
\end{eqnarray}

and $z$ is chosen distinct from $\quotep{P}$, $\quotep{Q}$, the free
names in $Q$, and all the names in $R$. Our $\alpha$-equivalence will
be built in the standard way from this substitution.

\begin{remark}\label{rem:no_self_referential_names}
  One consequence of these definitions is that $\forall P. \quotep{P}
  \not\in \freenames{P}$.
\end{remark}

\subsection{ Dynamic quote: an example }

Anticipating something of what's to come, consider applying the
substitution, $\widehat{\id{\{}u / z \id{\}}}$, to the following pair
of processes, $\lift{w}{y!(z)}$ and $w[ \lpquote y!(z) \rpquote ]$.

\begin{eqnarray}
	\lift{w}{y!(z)}\widehat{\id{\{}u / z \id{\}}}
		& = &
		\lift{w}{y!(u)} \nonumber\\
	w[ \lpquote y!(z) \rpquote ] \widehat{ \id{\{}u / z \id{\}} }
		& = &
		w[ \lpquote y!(z) \rpquote ] \nonumber
\end{eqnarray}

Because the body of the process between quotes is impervious to
substitution, we get radically different answers. In fact, by
examining the first process in an input context,
e.g. $x?(z).\lift{w}{y!(z)}$, we see that the process under the lift
operator may be shaped by prefixed inputs binding a name inside it. In
this sense, the lift operator will be seen as a way to dynamically
construct processes before reifying them as names.

Finally equipped with these standard features we can present the
dynamics of the calculus.

\subsubsection{Operational semantics} 

Finally, we introduce the computational dynamics. What marks these
algebras as distinct from other more traditionally studied algebraic
structures, e.g. vector spaces or polynomial rings, is the manner in
which dynamics is captured. In traditional structures, dynamics is typically
expressed through morphisms between such structures, as in linear maps
between vector spaces or morphisms between rings. In algebras
associated with the semantics of computation, the dynamics is
expressed as part of the algebraic structure itself, through a
reduction reduction relation typically denoted by $\red$. Below, we
give a recursive presentation of this relation for the calculus used
in the encoding.

$\red \subseteq \pi \times \pi$
$\red : \pi \to \mathcal{P}(\pi)$

\begin{mathpar}
  \inferrule* [lab=Comm] { \textsf{match}( x_{src}, x_{trgt} ) } { x_{trgt}?(y)P \; | \; x_{src}!\langle {Q} \rangle \red P\{\quotep{Q}/y}\} }
  \and \\
  \inferrule* [lab=Par] {{P} \red {P}'} {{{P} | {Q}} \red {{P}' | {Q}}}
  \and
  \inferrule* [lab=Equiv]{{{P} \scong {P}'} \andalso {{P}' \red {Q}'} \andalso {{Q}' \scong {Q}}}{{P} \red {Q}}
\end{mathpar}

\begin{eqnarray*}
  match_{\equiv} (\quotep{P},\quotep{Q}) & := & P \equiv Q \\
  match_{\dagger}(\quotep{P},\quotep{Q}) & := & \forall R. P|Q \red^{*} R => R \red^{*} 0 \\
  match_{K}(\quotep{P},\quotep{Q}) & := & K \mbox{ for some context } K
\end{eqnarray*}

$u?(x)P | u!\langle Q \rangle \red P\{\quotep{Q}/x\}$

%We write $\wred$ for $\red^*$, and $P\red$ if $\exists Q $ such that $ P \red Q$.
We write $P\red$ if $\exists Q $ such that $ P \red Q$ and $P\not\red$, otherwise.

\section{Replication}

As mentioned before, it is known that replication (and hence
recursion) can be implemented in a higher-order process algebra
\cite{SangiorgiWalker}. As our first example of calculation with the
machinery thus far presented we give the construction explicitly in
the {\rhoc}.

\begin{eqnarray}
	D_{x} & := & \prefix{x}{y}{(\binpar{\outputp{x}{y}}{@{y}})} \nonumber\\
	\bangp_{x}{P} & := & \binpar{{x}!\langle{\binpar{D_{x}}{P}}\rangle}{D_{x}} \nonumber
\end{eqnarray}

\begin{eqnarray}
	\bangp_{x}{P} & & \nonumber\\
	=
	& {x}!\langle{(\prefix{x}{y}{(\outputp{x}{y} | @{y})) | P}}\rangle 
	      | \prefix{x}{y}{(\outputp{x}{y} | @{y})} & \nonumber\\
	\red
	& (\outputp{x}{y} | @{y})\substn{\quotep{(\prefix{x}{y}{(@{y} | \outputp{x}{y})) | P}}}{y} & \nonumber\\
	=
	& \outputp{x}{\quotep{(\prefix{x}{y}{(\outputp{x}{y} | @{y})) | P}}}
	  | {(\prefix{x}{y}{(\outputp{x}{y} | @{y})) | P}} & \nonumber\\
	\red
	& \ldots & \nonumber\\
	\red^*
	& P | P | \ldots & \nonumber
\end{eqnarray}

Of course, this encoding, as an implementation, runs away, unfolding
$\bangp{P}$ eagerly. A lazier and more implementable replication
operator, restricted to input-guarded processes, may be obtained as follows.

\begin{eqnarray}
\bangp{\prefix{u}{v}{P}} 
	:= 
	\binpar{\lift{x}{\prefix{u}{v}{(\binpar{D(x)}{P})}}}{D(x)} \nonumber
\end{eqnarray}

\begin{remark}
  Note that the lazier definition still does not deal with summation
  or mixed summation (i.e. sums over input and output). The reader is
  invited to construct definitions of replication that deal with these
  features. 

  Further, the definitions are parameterized in a name, $x$. Can you,
  gentle reader, make a definition that eliminates this parameter and
  guarantees no accidental interaction between the replication
  machinery and the process being replicated -- i.e. no accidental
  sharing of names used by the process to get its work done and the
  name(s) used by the replication to effect copying. This latter
  revision of the definition of replication is crucial to obtaining
  the expected identity $!!P \sim !P$.
\end{remark}

\begin{remark}\label{rem:paradoxical_combinator}
  The reader familiar with the lambda calculus will have noticed the
  similarity between $D$ and the paradoxical combinator.

  [Ed. note: the existence of this seems to suggest we have to be more
  restrictive on the set of processes and names we admit if we are to
  support no-cloning.]
\end{remark}

\subsubsection{Bisimulation}

The computational dynamics gives rise to another kind of equivalence,
the equivalence of computational behavior. As previously mentioned
this is typically captured \emph{via} some form of bisimulation.

% The notion we use in this paper is weak barbed bisimulation
% \cite{milner91polyadicpi}.

The notion we use in this paper is derived from weak barbed
bisimulation \cite{milner91polyadicpi}. 

\begin{definition}
An \emph{observation relation}, $\downarrow_{\mathcal N}$, over a set
of names, $\mathcal N$, is the smallest relation satisfying the rules
below.

\infrule[Out-barb]{y \in {\mathcal N}, \; x \nameeq y}
		  {\outputp{x}{v} \downarrow_{\mathcal N} x}
\infrule[Par-barb]{\mbox{$P\downarrow_{\mathcal N} x$ or $Q\downarrow_{\mathcal N} x$}}
		  {\binpar{P}{Q} \downarrow_{\mathcal N} x}

We write $P \Downarrow_{\mathcal N} x$ if there is $Q$ such that 
$P \wred Q$ and $Q \downarrow_{\mathcal N} x$.
\end{definition}

\begin{definition}
%\label{def.bbisim}
An  ${\mathcal N}$-\emph{barbed bisimulation} over a set of names, ${\mathcal N}$, is a symmetric binary relation 
${\mathcal S}_{\mathcal N}$ between agents such that $P\rel{S}_{\mathcal N}Q$ implies:
\begin{enumerate}
\item If $P \red P'$ then $Q \wred Q'$ and $P'\rel{S}_{\mathcal N} Q'$.
\item If $P\downarrow_{\mathcal N} x$, then $Q\Downarrow_{\mathcal N} x$.
\end{enumerate}
$P$ is ${\mathcal N}$-barbed bisimilar to $Q$, written
$P \wbbisim_{\mathcal N} Q$, if $P \rel{S}_{\mathcal N} Q$ for some ${\mathcal N}$-barbed bisimulation ${\mathcal S}_{\mathcal N}$.
\end{definition}

$\mathcal{R} \subseteq \pi \times \pi$

$P \mathcal{R} Q => \forall P'. P \red P' \Rightarrow \exists Q'. Q \red Q', P' \mathcal{R} Q'$

$P \vdash x \Rightarrow Q \vdash x$

\begin{mathpar}
  \inferrule*[lab=Out-barb]{x \nameeq y}{{y}!\langle{Q}\rangle \vdash x}
  \and
  \inferrule*[lab=Par-barb]{\mbox{$P\vdash x$ or $Q\vdash x$}}{\binpar{P}{Q} \vdash x}
\end{mathpar}

\subsubsection{Contexts}

One of the principle advantages of computational calculi like the
$\pi$-calculus is a well-defined notion of context,
contextual-equivalence and a correlation between
contextual-equivalence and notions of bisimulation. The notion of
context allows the decomposition of a process into (sub-)process and
its syntactic environment, its context. Thus, a context may be
thought of as a process with a ``hole'' (written $\Box$) in it. The
application of a context $M$ to a process $P$, written $M[P]$, is
tantamount to filling the hole in $M$ with $P$. In this paper we do
not need the full weight of this theory, but do make use of the notion
of context in the proof the main theorem. 

\begin{mathpar}
  \inferrule* [lab=summation] {} {{M_{M},M_{N}} \bc \Box \;|\; x.M_{A} \;|\; M_{M}+M_{N}}
  \and
  \inferrule* [lab=agent] {} {{M_{A}} \bc (\vec{x})M_{P} \;| \; \clift{P_0,\ldots,M_{P},\ldots,P_N}}
  \and \\
  \inferrule* [lab=process] {} {{M_{P}} \bc M_{N} \;| \;P|M_{P} }
\end{mathpar} 

\begin{mathpar}
  \inferrule* [lab=sychronization] {} {M_{N} \bc \Box \;|\; x?M_{F} \;|\; x!M_{C}}
  \and
  \inferrule* [lab=abstraction] {} {{M_{F}} \bc (x)M_{P} }
  \and
  \inferrule* [lab=concretion] {} {{M_{C}} \bc \langle M_{P} \rangle }
  \and \\
  \inferrule* [lab=process] {} {{M_{P}} \bc M_{N} \;| \;P|M_{P} }
\end{mathpar}

\begin{definition}[contextual application] Given a context $M$, and
  process $P$, we define the \emph{contextual application}, $M[P] :=
  M\{P/\Box\}$. That is, the contextual application of M to P is the
  substitution of $P$ for $\Box$ in $M$.
\end{definition}

$\meaningof{-} : L \to \mathcal{P}(\pi)$

\begin{mathpar}
  \inferrule* [lab=collection] {} {\meaningof{true} = \pi, \and \meaningof{~E} = \pi \setminus \meaningof{E}, \and \meaningof{E_{1} \& E_{2}} = \meaningof{E_{1}} \cap \meaningof{E_{2}}}
\end{mathpar}

\begin{mathpar}
  \inferrule* [lab=structure] {} {\meaningof{0} = \{ P \in \pi | P \equiv 0 \}, \and \\ \meaningof{E_1 | E_2} = \{ P \in \pi | P \equiv P_{1} | P_{2}, P_{1} \in \meaningof{E_{1}}, P_{2} \in \meaningof{E_2}\} }
\end{mathpar}

\begin{mathpar}
 \inferrule* [lab=behavior] {} {\meaningof{\langle a?b \rangle E} = \{ P \in \pi | P \equiv Q | u?(y)P', \\ \and \\\\ \and \\ \;\;\; u \in \meaningof{a}, \forall z.P'\{z/y\} \in \meaningof{E\{z/b\}}\}, \and \\ \meaningof{a!E} = \{ P \in \pi | P \equiv Q | x!\langle P' \rangle, x \in \meaningof{a} P' \in \meaningof{E}\} }
\end{mathpar}

\begin{mathpar}
 \inferrule* [lab=nominal] {} {\meaningof{\quotep{E}} = \{ \quotep{P} \in \quotep{\pi} | P \in \meaningof{E} \}, \and \meaningof{\quotep{P}} = \{ \quotep{Q} \in \quotep{\pi} | P \equiv Q \} \and \\ \meaningof{@\quotep{E}} = \{ P \in \pi | P \equiv @x, x \in \meaningof{E} \}}
\end{mathpar}

\begin{eqnarray*}
  \\
  \meaningof{-} : TS \to ST
\end{eqnarray*}

\begin{eqnarray*}
  \\
  L : TS \to ST
\end{eqnarray*}

\begin{eqnarray*}
  \\
  P \models E \iff P \in \meaningof{E}
\end{eqnarray*}

\begin{eqnarray*}
  P \approx_{L} Q \iff \forall E \in L. P \models E \iff Q \models E
\end{eqnarray*}

\begin{eqnarray*}
  P \approx_{K} Q
\end{eqnarray*}

\begin{eqnarray*}
  P \approx Q
\end{eqnarray*}

$\approx_{K} = \approx = \approx_{L}$

\subsubsection{Contextual duality}

Note that contexts extend the quotation operation to a family of
operations from processes to names. Given a context, $M$, we can
define a \emph{nominal context}, $\quotep{M}$ by $\quotep{M}[P] :=
\quotep{M[P]}$. To foreshadow what is to come we observe that these
operations enjoy a duality with processes very much like the duality
between vectors and maps from vectors to scalars.

Further, because the calculus is essentially higher-order, we have a
correspondence between contexts and processes. More specifically,
given a name $x$ and a context $M$ we can construct $M^{*}_{x}$ such
that 

\begin{mathpar}
  M^{*}_{x} | \lift{x}{P} \red M[P]
\end{mathpar}

namely,

\begin{mathpar}
  M^{*}_{x} := x?(u).M[\dropn{u}]
\end{mathpar}

The dependence of $M^{*}_{x}$ on a name makes it an abstraction, 

\begin{mathpar}
  M^{*} := (x)x?(u).M[\dropn{u}]
\end{mathpar}

\subsection{Additional notation}

It will sometimes be convenient to denote the process a name
quotes. We already have the notation $x = \quotep{P}$, but it will be
convenient to introduce an alternate notation, $\procn{x}$, when we
want to emphasize the connection to the use of the name. Note that, by
virtue of name equivalence, $\quotep{\procn{x}} \nameeq x$; so, the
notation is consistent with previous definitions.

Further, because names have structure it is possible to effect
substitutions on the basis of that structure. This means we need to
upgrade our notation for substitutions, which we accomplish by
adapting comprehension notation. Thus,

\begin{mathpar}
  P\{ y / x : x \in S \}
\end{mathpar}

is interpreted to mean the process derived from P by replacing (in a
capture-avoiding manner) each occurrence of $x$ in $S$ by $y$. For example,

\begin{mathpar}
  P\{ \quotep{\procn{x}|\procn{x}} / x : x \in \freenames{P} \}
\end{mathpar}

will replace each (occurrence) of a free name $x$ in $P$ by
$\quotep{\procn{x}|\procn{x}}$.

Also, we will avail ourselves of the notation $x^{L}$ and $x^{R}$ to
denote injections of a name into disjoint copies of the name
space. There are numerous ways to accomplish this. One example can be
found in \cite{MeredithR05}. This notation overloads to vectors of
names: $\vec{x}^{\pi} := (x_{i}^{\pi} \; : \; 0 \leq i < |\vec{x}| )$ where $\pi \in \{L,R\}$.

We also use $P^{\Box} := P|\Box$.

In \cite{MeredithR05} an interpretation of the new operator is
given. It turns out that there are several possible interpretations
all enjoying the requisite algebraic properties of the operator (see
\cite{milner91polyadicpi}). We will therefore make liberal use of
$(\nu\; \vec{x})P$.

% subsection the_syntax_and_semantics_of_the_notation_system (end)   

\input{qm2pi.qmops} 

\input{qm2pi.sterngerlach} 

\input{qm2pi.metric} 

% section concurrent_process_calculi (end)

%\input{qm2pi.proofsketch}

% section proof sketch (end)

%\input{qm2pi.slviaknots} 

% section spatial logic via knots (end)

\input{qm2pi.conclusion}

% section conclusion (end)

%\input{qm2pi.dtcodes} 

% section wiring algorithm (end)

\input{qm2pi.ack} 

% section acknowledgments (end)

\newpage


\bibliographystyle{plain}   
\bibliography{../../biblios/main.bib}

\input{qm2pi.rhodetails}

\end{document}

 

% subsection basic_interpretation (end)

%\input{qm2pi.rho.presentation} 
\subsection{The syntax and semantics of the notation system}\label{sub:the_syntax_and_semantics_of_the_notation_system} % (fold)

We now summarize a technical presentation of the calculus that
embodies our theory of dynamics. The typical presentation of such a
calculus follows the style of giving generators and relations on
them. The grammar, below, describing term constructors, freely
generates the set of processes, $\Proc$. This set is then quotiented
by a relation known as structural congruence and it is over this set
that the notion of dynamics is expressed. This presentation is
essentially that of \cite{MeredithR05} with the addition of
polyadicity and summation. For readability we have relegated some of
the technical subtleties to an appendix.

\subsubsection{Process grammar}\label{subsub:process_grammar}

\begin{mathpar}
  \inferrule* [lab=synchronization] {} {{M} \bc \pzero \;|\; x?F \;|\; x!C }
  \and
  \inferrule* [lab=abstraction] {} {{F} \bc (x)P}
  \and
  \inferrule* [lab=concretion] {} {{C} \bc \langle Q \rangle}
  \and
  \inferrule* [lab=process] {} {{P,Q} \bc M \;| \;P|Q \;|\; @{x}}
  \and
  \inferrule* [lab=name] {} {{x} \bc \quotep{P}}
\end{mathpar} 

Note that $\vec{x}$ (resp. $\vec{P}$) denotes a vector of names
(resp. processes) of length $|\vec{x}|$ (resp. $|\vec{P}|$). We adopt
the following useful abbreviations.

\begin{mathpar}
   x?(\vec{y}).P := x.(\vec{y})P \and  x\clift{\vec{P}} := x.\clift{\vec{P}}
   \and x!(y) := \lift{x}{\dropn{y}}
   \and \Pi_{i=0}^{n-1}P_i := P_0 | \ldots | P_{n-1}
\end{mathpar}

\subsubsection{Structural congruence}

\paragraph{Free and bound names and alpha-equivalence.} At the
core of structural equivalence is alpha-equivalence which identifies
process that are the same up to a change of variable. Formally, we
recognize the distinction between free and bound names. The free names
of a process, $\freenames{P}$, may be calculated recursively as
follows:

\begin{mathpar}
\freenames{\pzero} := \emptyset
  \and \\
  \freenames{x?(y).P} := \{ x \} \cup (\freenames{P} \setminus \{ y \})
  \and 
  \freenames{x!\langle P \rangle} := \{ x \} \cup \{ P \} 
  \and \\
  \freenames{P|Q} := \freenames{P} \cup \freenames{Q}
  \and \\
  \freenames{@{x}} := \{ x \}
\end{mathpar}

$\pi$
$\quotep{\pi}$

$\freenames{-} : \pi \to \mathcal{P}(\quotep{\pi})$

\begin{eqnarray*}
  \freenames{\pzero} & := & \emptyset \\
  \freenames{x?(y).P} & := & \{ x \} \cup (\freenames{P} \setminus \{ y \}) \\
  \freenames{x!\langle P \rangle} & := & \{ x \} \cup \{ P \} \\
  \freenames{P|Q} & := & \freenames{P} \cup \freenames{Q} \\
  \freenames{\dropn{x}} & := & \{ x \}
\end{eqnarray*}

The bound names of a process, $\boundnames{P}$, are those names occurring in $P$
that are not free. For example, in $x?(y).0$, the name $x$ is free, while $y$ is bound.

\begin{mathpar}
  \inferrule* [lab=monoidal-laws] {} { P|Q \equiv Q|P \and P|0 \equiv P \and P|(Q|R) \equiv (P|Q)|R }
\end{mathpar}

\begin{mathpar}
  \inferrule* [lab=alpha-equivalence] {} { (x)P \equiv (y)P\{y/x\} \and y \not\in \freenames{P} }
\end{mathpar}

\begin{definition}
Then two processes, $P,Q$, are alpha-equivalent if $P = Q\{\vec{y}/\vec{x}\}$ for
some $\vec{x} \in \boundnames{Q},\vec{y} \in \boundnames{P}$, where $Q\{\vec{y}/\vec{x}\}$
denotes the capture-avoiding substitution of $\vec{y}$ for $\vec{x}$ in $Q$.
\end{definition}

\begin{definition}
  The {\em structural congruence} \cite{SangiorgiWalker} , $\equiv$,
  between processes is the least congruence containing
  alpha-equivalence, satisfying the abelian monoid laws
  (associativity, commutativity and $\pzero$ as identity) for parallel
  composition $|$ and for summation $+$.
\end{definition}

\subsection{Name equivalence}

We take name equivalence, written $\nameeq$, to be the smallest
equivalence relation generated by the following rules.

\begin{mathpar}
\inferrule*[lab=Quote-drop]
{ }
{ \quotep{@{x}} \nameeq x }

\inferrule*[lab=Struct-equiv]
{ P \scong Q }
{ \quotep{P} \nameeq \quotep{Q} }
\end{mathpar}

The astute reader will have noticed that the mutual recursion of names
and processes imposes a mutual recursion on alpha-equivalence and
structural equivalence via name-equivalence. Fortunately, all of this
works out pleasantly and we may calculate in the natural way, free of
concern. The reader interested in the details is referred to the
appendix \ref{appendix:rho_details}.

\subsection{Substitution}

We use $\Proc$ for the set of processes, $\QProc$ for the set of
names, and $\id{\{}\vec{y} / \vec{x} \id{\}}$ to denote partial maps,
$s : \QProc \rightarrow \QProc$. A map, $s$ lifts, uniquely, to a map
on process terms, $\widehat{s} : \Proc \rightarrow \Proc$ by the
following equations.

\begin{mathpar}
  (0) \psubstp{Q}{P} := 0 \\
  (R \juxtap S) \psubstp{Q}{P}
  :=    
  (R)\psubstp{Q}{P} \juxtap (S) \psubstp{Q}{P} \\
  (x?(y).R) \psubstp{Q}{P}    
  :=    
  (x)\substp{Q}{P} (z)\concat( (R \psubstn{z}{y}) \psubstp{Q}{P} ) \\
  (\lift{x}{R}) \psubstp{Q}{P}  
  :=
  \lift{(x)\substp{Q}{P}}{ R \psubstp{Q}{P} } \\
%   (\dropn{x})  \psubstp{Q}{P}       
%   := 
%   \left\{ 
%     \begin{array}{ccc} 
%       \dropn{\quotep{Q}} & & x \nameeq \quotep{P} \\
%       \dropn{x} & & otherwise \\
%     \end{array}
%   \right. 
  (\dropn{x})  \psubstp{Q}{P}       
  := 
  \left\{ 
    \begin{array}{ccc} 
      Q & & x \nameeq \quotep{P} \\
      \dropn{x} & & otherwise \\
    \end{array}
  \right.
\end{mathpar}
 

where

\begin{eqnarray}
  (x)\id{\{} \lpquote Q \rpquote / \lpquote P \rpquote \id{\}}            = 
  \left\{ 
    \begin{array}{ccc}
      \lpquote Q \rpquote & & x \nameeq \lpquote P \rpquote \\
      x & & otherwise \\
    \end{array}
  \right. \nonumber
\end{eqnarray}

and $z$ is chosen distinct from $\quotep{P}$, $\quotep{Q}$, the free
names in $Q$, and all the names in $R$. Our $\alpha$-equivalence will
be built in the standard way from this substitution.

\begin{remark}\label{rem:no_self_referential_names}
  One consequence of these definitions is that $\forall P. \quotep{P}
  \not\in \freenames{P}$.
\end{remark}

\subsection{ Dynamic quote: an example }

Anticipating something of what's to come, consider applying the
substitution, $\widehat{\id{\{}u / z \id{\}}}$, to the following pair
of processes, $\lift{w}{y!(z)}$ and $w[ \lpquote y!(z) \rpquote ]$.

\begin{eqnarray}
	\lift{w}{y!(z)}\widehat{\id{\{}u / z \id{\}}}
		& = &
		\lift{w}{y!(u)} \nonumber\\
	w[ \lpquote y!(z) \rpquote ] \widehat{ \id{\{}u / z \id{\}} }
		& = &
		w[ \lpquote y!(z) \rpquote ] \nonumber
\end{eqnarray}

Because the body of the process between quotes is impervious to
substitution, we get radically different answers. In fact, by
examining the first process in an input context,
e.g. $x?(z).\lift{w}{y!(z)}$, we see that the process under the lift
operator may be shaped by prefixed inputs binding a name inside it. In
this sense, the lift operator will be seen as a way to dynamically
construct processes before reifying them as names.

Finally equipped with these standard features we can present the
dynamics of the calculus.

\subsubsection{Operational semantics} 

Finally, we introduce the computational dynamics. What marks these
algebras as distinct from other more traditionally studied algebraic
structures, e.g. vector spaces or polynomial rings, is the manner in
which dynamics is captured. In traditional structures, dynamics is typically
expressed through morphisms between such structures, as in linear maps
between vector spaces or morphisms between rings. In algebras
associated with the semantics of computation, the dynamics is
expressed as part of the algebraic structure itself, through a
reduction reduction relation typically denoted by $\red$. Below, we
give a recursive presentation of this relation for the calculus used
in the encoding.

$\red \subseteq \pi \times \pi$
$\red : \pi \to \mathcal{P}(\pi)$

\begin{mathpar}
  \inferrule* [lab=Comm] { \textsf{match}( x_{src}, x_{trgt} ) } { x_{trgt}?(y)P \; | \; x_{src}!\langle {Q} \rangle \red P\{\quotep{Q}/y}\} }
  \and \\
  \inferrule* [lab=Par] {{P} \red {P}'} {{{P} | {Q}} \red {{P}' | {Q}}}
  \and
  \inferrule* [lab=Equiv]{{{P} \scong {P}'} \andalso {{P}' \red {Q}'} \andalso {{Q}' \scong {Q}}}{{P} \red {Q}}
\end{mathpar}

\begin{eqnarray*}
  match_{\equiv} (\quotep{P},\quotep{Q}) & := & P \equiv Q \\
  match_{\dagger}(\quotep{P},\quotep{Q}) & := & \forall R. P|Q \red^{*} R => R \red^{*} 0 \\
  match_{K}(\quotep{P},\quotep{Q}) & := & K \mbox{ for some context } K
\end{eqnarray*}

$u?(x)P | u!\langle Q \rangle \red P\{\quotep{Q}/x\}$

%We write $\wred$ for $\red^*$, and $P\red$ if $\exists Q $ such that $ P \red Q$.
We write $P\red$ if $\exists Q $ such that $ P \red Q$ and $P\not\red$, otherwise.

\section{Replication}

As mentioned before, it is known that replication (and hence
recursion) can be implemented in a higher-order process algebra
\cite{SangiorgiWalker}. As our first example of calculation with the
machinery thus far presented we give the construction explicitly in
the {\rhoc}.

\begin{eqnarray}
	D_{x} & := & \prefix{x}{y}{(\binpar{\outputp{x}{y}}{@{y}})} \nonumber\\
	\bangp_{x}{P} & := & \binpar{{x}!\langle{\binpar{D_{x}}{P}}\rangle}{D_{x}} \nonumber
\end{eqnarray}

\begin{eqnarray}
	\bangp_{x}{P} & & \nonumber\\
	=
	& {x}!\langle{(\prefix{x}{y}{(\outputp{x}{y} | @{y})) | P}}\rangle 
	      | \prefix{x}{y}{(\outputp{x}{y} | @{y})} & \nonumber\\
	\red
	& (\outputp{x}{y} | @{y})\substn{\quotep{(\prefix{x}{y}{(@{y} | \outputp{x}{y})) | P}}}{y} & \nonumber\\
	=
	& \outputp{x}{\quotep{(\prefix{x}{y}{(\outputp{x}{y} | @{y})) | P}}}
	  | {(\prefix{x}{y}{(\outputp{x}{y} | @{y})) | P}} & \nonumber\\
	\red
	& \ldots & \nonumber\\
	\red^*
	& P | P | \ldots & \nonumber
\end{eqnarray}

Of course, this encoding, as an implementation, runs away, unfolding
$\bangp{P}$ eagerly. A lazier and more implementable replication
operator, restricted to input-guarded processes, may be obtained as follows.

\begin{eqnarray}
\bangp{\prefix{u}{v}{P}} 
	:= 
	\binpar{\lift{x}{\prefix{u}{v}{(\binpar{D(x)}{P})}}}{D(x)} \nonumber
\end{eqnarray}

\begin{remark}
  Note that the lazier definition still does not deal with summation
  or mixed summation (i.e. sums over input and output). The reader is
  invited to construct definitions of replication that deal with these
  features. 

  Further, the definitions are parameterized in a name, $x$. Can you,
  gentle reader, make a definition that eliminates this parameter and
  guarantees no accidental interaction between the replication
  machinery and the process being replicated -- i.e. no accidental
  sharing of names used by the process to get its work done and the
  name(s) used by the replication to effect copying. This latter
  revision of the definition of replication is crucial to obtaining
  the expected identity $!!P \sim !P$.
\end{remark}

\begin{remark}\label{rem:paradoxical_combinator}
  The reader familiar with the lambda calculus will have noticed the
  similarity between $D$ and the paradoxical combinator.

  [Ed. note: the existence of this seems to suggest we have to be more
  restrictive on the set of processes and names we admit if we are to
  support no-cloning.]
\end{remark}

\subsubsection{Bisimulation}

The computational dynamics gives rise to another kind of equivalence,
the equivalence of computational behavior. As previously mentioned
this is typically captured \emph{via} some form of bisimulation.

% The notion we use in this paper is weak barbed bisimulation
% \cite{milner91polyadicpi}.

The notion we use in this paper is derived from weak barbed
bisimulation \cite{milner91polyadicpi}. 

\begin{definition}
An \emph{observation relation}, $\downarrow_{\mathcal N}$, over a set
of names, $\mathcal N$, is the smallest relation satisfying the rules
below.

\infrule[Out-barb]{y \in {\mathcal N}, \; x \nameeq y}
		  {\outputp{x}{v} \downarrow_{\mathcal N} x}
\infrule[Par-barb]{\mbox{$P\downarrow_{\mathcal N} x$ or $Q\downarrow_{\mathcal N} x$}}
		  {\binpar{P}{Q} \downarrow_{\mathcal N} x}

We write $P \Downarrow_{\mathcal N} x$ if there is $Q$ such that 
$P \wred Q$ and $Q \downarrow_{\mathcal N} x$.
\end{definition}

\begin{definition}
%\label{def.bbisim}
An  ${\mathcal N}$-\emph{barbed bisimulation} over a set of names, ${\mathcal N}$, is a symmetric binary relation 
${\mathcal S}_{\mathcal N}$ between agents such that $P\rel{S}_{\mathcal N}Q$ implies:
\begin{enumerate}
\item If $P \red P'$ then $Q \wred Q'$ and $P'\rel{S}_{\mathcal N} Q'$.
\item If $P\downarrow_{\mathcal N} x$, then $Q\Downarrow_{\mathcal N} x$.
\end{enumerate}
$P$ is ${\mathcal N}$-barbed bisimilar to $Q$, written
$P \wbbisim_{\mathcal N} Q$, if $P \rel{S}_{\mathcal N} Q$ for some ${\mathcal N}$-barbed bisimulation ${\mathcal S}_{\mathcal N}$.
\end{definition}

$\mathcal{R} \subseteq \pi \times \pi$

$P \mathcal{R} Q => \forall P'. P \red P' \Rightarrow \exists Q'. Q \red Q', P' \mathcal{R} Q'$

$P \vdash x \Rightarrow Q \vdash x$

\begin{mathpar}
  \inferrule*[lab=Out-barb]{x \nameeq y}{{y}!\langle{Q}\rangle \vdash x}
  \and
  \inferrule*[lab=Par-barb]{\mbox{$P\vdash x$ or $Q\vdash x$}}{\binpar{P}{Q} \vdash x}
\end{mathpar}

\subsubsection{Contexts}

One of the principle advantages of computational calculi like the
$\pi$-calculus is a well-defined notion of context,
contextual-equivalence and a correlation between
contextual-equivalence and notions of bisimulation. The notion of
context allows the decomposition of a process into (sub-)process and
its syntactic environment, its context. Thus, a context may be
thought of as a process with a ``hole'' (written $\Box$) in it. The
application of a context $M$ to a process $P$, written $M[P]$, is
tantamount to filling the hole in $M$ with $P$. In this paper we do
not need the full weight of this theory, but do make use of the notion
of context in the proof the main theorem. 

\begin{mathpar}
  \inferrule* [lab=summation] {} {{M_{M},M_{N}} \bc \Box \;|\; x.M_{A} \;|\; M_{M}+M_{N}}
  \and
  \inferrule* [lab=agent] {} {{M_{A}} \bc (\vec{x})M_{P} \;| \; \clift{P_0,\ldots,M_{P},\ldots,P_N}}
  \and \\
  \inferrule* [lab=process] {} {{M_{P}} \bc M_{N} \;| \;P|M_{P} }
\end{mathpar} 

\begin{mathpar}
  \inferrule* [lab=sychronization] {} {M_{N} \bc \Box \;|\; x?M_{F} \;|\; x!M_{C}}
  \and
  \inferrule* [lab=abstraction] {} {{M_{F}} \bc (x)M_{P} }
  \and
  \inferrule* [lab=concretion] {} {{M_{C}} \bc \langle M_{P} \rangle }
  \and \\
  \inferrule* [lab=process] {} {{M_{P}} \bc M_{N} \;| \;P|M_{P} }
\end{mathpar}

\begin{definition}[contextual application] Given a context $M$, and
  process $P$, we define the \emph{contextual application}, $M[P] :=
  M\{P/\Box\}$. That is, the contextual application of M to P is the
  substitution of $P$ for $\Box$ in $M$.
\end{definition}

$\meaningof{-} : L \to \mathcal{P}(\pi)$

\begin{mathpar}
  \inferrule* [lab=collection] {} {\meaningof{true} = \pi, \and \meaningof{~E} = \pi \setminus \meaningof{E}, \and \meaningof{E_{1} \& E_{2}} = \meaningof{E_{1}} \cap \meaningof{E_{2}}}
\end{mathpar}

\begin{mathpar}
  \inferrule* [lab=structure] {} {\meaningof{0} = \{ P \in \pi | P \equiv 0 \}, \and \\ \meaningof{E_1 | E_2} = \{ P \in \pi | P \equiv P_{1} | P_{2}, P_{1} \in \meaningof{E_{1}}, P_{2} \in \meaningof{E_2}\} }
\end{mathpar}

\begin{mathpar}
 \inferrule* [lab=behavior] {} {\meaningof{\langle a?b \rangle E} = \{ P \in \pi | P \equiv Q | u?(y)P', \\ \and \\\\ \and \\ \;\;\; u \in \meaningof{a}, \forall z.P'\{z/y\} \in \meaningof{E\{z/b\}}\}, \and \\ \meaningof{a!E} = \{ P \in \pi | P \equiv Q | x!\langle P' \rangle, x \in \meaningof{a} P' \in \meaningof{E}\} }
\end{mathpar}

\begin{mathpar}
 \inferrule* [lab=nominal] {} {\meaningof{\quotep{E}} = \{ \quotep{P} \in \quotep{\pi} | P \in \meaningof{E} \}, \and \meaningof{\quotep{P}} = \{ \quotep{Q} \in \quotep{\pi} | P \equiv Q \} \and \\ \meaningof{@\quotep{E}} = \{ P \in \pi | P \equiv @x, x \in \meaningof{E} \}}
\end{mathpar}

\begin{eqnarray*}
  \\
  \meaningof{-} : TS \to ST
\end{eqnarray*}

\begin{eqnarray*}
  \\
  L : TS \to ST
\end{eqnarray*}

\begin{eqnarray*}
  \\
  P \models E \iff P \in \meaningof{E}
\end{eqnarray*}

\begin{eqnarray*}
  P \approx_{L} Q \iff \forall E \in L. P \models E \iff Q \models E
\end{eqnarray*}

\begin{eqnarray*}
  P \approx_{K} Q
\end{eqnarray*}

\begin{eqnarray*}
  P \approx Q
\end{eqnarray*}

$\approx_{K} = \approx = \approx_{L}$

\subsubsection{Contextual duality}

Note that contexts extend the quotation operation to a family of
operations from processes to names. Given a context, $M$, we can
define a \emph{nominal context}, $\quotep{M}$ by $\quotep{M}[P] :=
\quotep{M[P]}$. To foreshadow what is to come we observe that these
operations enjoy a duality with processes very much like the duality
between vectors and maps from vectors to scalars.

Further, because the calculus is essentially higher-order, we have a
correspondence between contexts and processes. More specifically,
given a name $x$ and a context $M$ we can construct $M^{*}_{x}$ such
that 

\begin{mathpar}
  M^{*}_{x} | \lift{x}{P} \red M[P]
\end{mathpar}

namely,

\begin{mathpar}
  M^{*}_{x} := x?(u).M[\dropn{u}]
\end{mathpar}

The dependence of $M^{*}_{x}$ on a name makes it an abstraction, 

\begin{mathpar}
  M^{*} := (x)x?(u).M[\dropn{u}]
\end{mathpar}

\subsection{Additional notation}

It will sometimes be convenient to denote the process a name
quotes. We already have the notation $x = \quotep{P}$, but it will be
convenient to introduce an alternate notation, $\procn{x}$, when we
want to emphasize the connection to the use of the name. Note that, by
virtue of name equivalence, $\quotep{\procn{x}} \nameeq x$; so, the
notation is consistent with previous definitions.

Further, because names have structure it is possible to effect
substitutions on the basis of that structure. This means we need to
upgrade our notation for substitutions, which we accomplish by
adapting comprehension notation. Thus,

\begin{mathpar}
  P\{ y / x : x \in S \}
\end{mathpar}

is interpreted to mean the process derived from P by replacing (in a
capture-avoiding manner) each occurrence of $x$ in $S$ by $y$. For example,

\begin{mathpar}
  P\{ \quotep{\procn{x}|\procn{x}} / x : x \in \freenames{P} \}
\end{mathpar}

will replace each (occurrence) of a free name $x$ in $P$ by
$\quotep{\procn{x}|\procn{x}}$.

Also, we will avail ourselves of the notation $x^{L}$ and $x^{R}$ to
denote injections of a name into disjoint copies of the name
space. There are numerous ways to accomplish this. One example can be
found in \cite{MeredithR05}. This notation overloads to vectors of
names: $\vec{x}^{\pi} := (x_{i}^{\pi} \; : \; 0 \leq i < |\vec{x}| )$ where $\pi \in \{L,R\}$.

We also use $P^{\Box} := P|\Box$.

In \cite{MeredithR05} an interpretation of the new operator is
given. It turns out that there are several possible interpretations
all enjoying the requisite algebraic properties of the operator (see
\cite{milner91polyadicpi}). We will therefore make liberal use of
$(\nu\; \vec{x})P$.

% subsection the_syntax_and_semantics_of_the_notation_system (end)   

\section{Interpretation of QM}
\subsection{Supporting definitions}
\subsubsection{Multiplication}
\begin{mathpar}
  \quotep{Q} \cdot \quotep{R} := \quotep{Q|R}
  \and \\
  \quotep{Q} \cdot P := P\{ \quotep{Q|R} / \quotep{R} : \quotep{R} \in \freenames{P} \}
\end{mathpar}

\paragraph{Discussion}
The first line needs little explanation. The second line says that
each free name of the process is replaced with the multiplication of
that name by the scalar. Multiplication of a scalar (name) by a state
(process) results in a process all the names of which have been `moved
over' by parallel composition with the process the scalar
quotes. There is a subtlety that the bound names have to be
manipulated so that multiplied names aren't accidentally
captured. There are many ways to achieve this.

\begin{remark}\label{rem:multiplication_identities}
  The reader is invited to verify that for all $x,y,z \in \QProc$ and $P \in \Proc$
  \begin{mathpar}
    x \cdot \quotep{0} \equiv x 
    \and
    x \cdot y \equiv y \cdot x
    \and
    x \cdot (y \cdot z) \equiv (x \cdot y) \cdot z
    \and \\
    \quotep{0} \cdot P \equiv P
    \and \\
    x \cdot (y \cdot P) \equiv (x \cdot y) \cdot P
    \and \\
    x \cdot (P|Q) \equiv (x \cdot P) | (x \cdot Q)
    \and \\    
  \end{mathpar}
\end{remark}

\subsubsection{Tensor product}

We define a tensor product on processes by structural induction.

\paragraph{Tensor of sums} First note that all summations, including
$\pzero$ and sequence, can be written $\Sigma_{i} x_{i}.A_{i} +
\Sigma_{j} x_{j}.C_{j}$, where we have grouped input-guarded processes
together and output-guarded processes together.

Thus, we can define the tensor product of two summations, $N_{1}\otimes N_{2}$, where

\begin{mathpar}
  N_{1} := \Sigma_{i} x_{i}.A_{i} + \Sigma_{j} x_{j}.C_{j}
  \and
  N_{2} := \Sigma_{i'} y_{i'}.B_{i'} + \Sigma_{j'} y_{j'}.D_{j'} 
\end{mathpar}

as follows.

\begin{mathpar}
  \Sigma_{i} x_{i}.A_{i} + \Sigma_{j} x_{j}.C_{j} \otimes \Sigma_{i'}
  y_{i'}.B_{i'} + \Sigma_{j'} y_{j'}.D_{j'} 
  \and \\
  := \; \Sigma_{i} \Sigma_{i'} \quotep{\stackrel{\vee}{x_{i}}| \stackrel{\vee}{y_{i'}}}.(A_{i}\otimes B_{i'}) \; | \; \Sigma_{i'} \Sigma_{i} \quotep{\stackrel{\vee}{y_{i'}}|\stackrel{\vee}{x_{i}}}.(B_{i'}\otimes A_{i})
  \and
  \;\; | \;\; \Sigma_{j} \Sigma_{j'} \quotep{\stackrel{\vee}{x_{j}}|\stackrel{\vee}{y_{j'}}}.(A_{j}\otimes B_{j'}) \; | \; \Sigma_{j'} \Sigma_{j} \quotep{\stackrel{\vee}{y_{j'}}|\stackrel{\vee}{x_{j}}}.(B_{j'}\otimes A_{j})
\end{mathpar}

\begin{remark}
  Do we need to $x^{L}$ and $y^{R}$ for this construction as well?
\end{remark}

\paragraph{Tensor of parallel compositions} Next, we distribute tensor
over par.

\begin{mathpar}
  P_{1}|P_{2} \otimes Q_{1}|Q_{2} := (P_{1} \otimes Q_{1}) | (P_{1}
  \otimes Q_{2}) | (P_{2} \otimes Q_{1}) | (P_{2} \otimes Q_{2})
\end{mathpar}

\paragraph{Tensor with dropped names} We treat tensor of a
process with a dropped name as parallel composition.

\begin{mathpar}
  P \otimes \dropn{x} := P | \dropn{x}
\end{mathpar}

\paragraph{Tensor of agents}

Finally, we need to define tensor on agents. Note that the definition
of tensor on normal products only tensors inputs with inputs and
outputs with outputs. Thus, we only have to define the operation on
``homogeneous'' pairings.

\begin{mathpar}
  (\vec{x})P \otimes (\vec{y})Q
  \and \\
  := (x_{0}^{L}|y_{0}^{R},\ldots,x_{0}^{L}|y_{n}^{R},\ldots,x_{m}^{L}|y_{0}^{R},\ldots,x_{m}^{L}|y_{n}^R)(P\{ \vec{x}^{L}/\vec{x}\} \otimes Q \{ \vec{y}^{R}/\vec{y}\})
  \and \\
  \clift{\vec{P}} \otimes \clift{\vec{Q}}
  \and \\
  := \clift{P_{0}\otimes Q_{0},\ldots,P_{0}\otimes Q_{n},\ldots,P_{m}\otimes Q_{0},\ldots,P_{m}\otimes Q_{n}}
\end{mathpar}

\begin{remark}
  Observe that arities of tensored abstractions matches arities of
  tensored concretions if the original arities matched. Note also that
  the length of the arities corresponds to the increase in dimension
  we see in ordinary vector space tensor product.
\end{remark}

\begin{remark}
  Operationally, this definition distributes the tensor down to
  components ``linked'' by summation. Tensor over summation is
  intriguing in that it mixes names. Moreover, as a consequence of the
  way it mixes names we have the identities for all $x \in \QProc$ and
  $P,Q \in \Proc$

  \begin{mathpar}
    (x \cdot P) \otimes Q \equiv x \cdot (P \otimes Q) \equiv P \otimes (x \cdot Q)
    \and
    P \otimes \pzero \equiv P
  \end{mathpar}

  that the reader is invited to verify.
\end{remark}

\subsubsection{Annihilation}
\begin{mathpar}
  P^{\perp} := \{ Q | \forall R. P|Q \red^{*} R \Rightarrow R \red^{*} \pzero \}
  \and \\
  P^{\underline{\perp}} := \Sigma_{Q \in P^{\perp}} \quotep{Q}?(y).(\dropn{y}|Q) | \Sigma_{Q \in P^{\perp}} \quotep{Q}\clift{\Box}
\end{mathpar}

\paragraph{Discussion} The reader will note that $P^{\perp}$ is a
\emph{set} of processes, while $P^{\underline{\perp}}$ is a
\emph{context}. We call the set $P^{\perp}$ the \emph{annihilators} of
$P$. The parallel composition of a process in the annihilators of $P$
with $P$ will result in a process, the state space of which has all
paths eventually leading to $\pzero$. Execution may endure loops; but
under reasonable conditions of fairness (naturally guaranteed under
most notions of bisimulation) such a composite process cannot get
stuck in such a loop and will, eventually pop out and terminate.

The context $P^{\underline{\perp}}$ is ready and willing to ``take the
$P$ out of'' the process to which it is applied. It will effectively
transmit the code of the process to which it is applied to one of the
annihilators and run the process against it.

\subsubsection{Evaluation}
We fix $M$ a domain of fully abstract interpretation with an equality
coincident with bisimulation. We take $\meaningof{\cdot} : \Proc \to
M$ to be the map interpreting processes and $\nmeaningof{\cdot} : \M
\to Proc$ to be the map running the other way. Then we define

\begin{mathpar}
  \int P := \nmeaningof{\meaningof{P}}
\end{mathpar}

\paragraph{Discussion}
There are many fully abstract interpretations of Milner's
$\pi$-calculus. Any of them can be used as a basis for interpreting
the reflective calculus here. Equipped with such a domain it is
largely a matter of grinding through to check that the Yoneda
construction for the normalization-by-evaluation program can be
extended to this setting.

\begin{remark}
  The reader is invited to verify that $\int (P^{\underline{\perp}}[P]) = 0$.
\end{remark}

\subsection{Quantum mechanics}

Table \ref{tbl:core_qm_op_defns} gives the core operational definitions

\begin{table}[htp]\label{tbl:core_qm_op_defns}
  \center{
    \fbox{
      \begin{tabular}{c|c}
        quantum mechanics & process calculus \\
        \hline
        scalar & $x := \quotep{P}$ \\
        state vector & $\state{P} := P$ \\
        dual & $\state{P}^{*} := \event{P^{\underline{\perp}}} := \quotep{P^{\underline{\perp}}}[-]$ \\
        matrix & $ \Sigma_{\alpha} \state{P_{\alpha}}x_{\alpha}\event{Q_{\alpha}}$ \\
        vector addition & $\state{P} + \state{Q} := \state{P | Q}$ \\
        tensor product & $\state{P} \otimes \state{Q} := \state{P \otimes Q}$ \\
        inner product & $\innerprod{P}{Q} := \quotep{\int P^{\underline{\perp}}[Q]}$ \\
      \end{tabular}
    }
  }
  \caption{QM - operational definitions}
\end{table}

where

\begin{mathpar}
  \prmatrix{P}{Q} := \fprmatrix{P}{\quotep{\pzero}}{Q}
  \and
  \fprmatrix{P}{x}{Q} := (\state{P},x,\event{Q})
  \and
  (\fprmatrix{P}{x}{Q})(\state{R}) := x \cdot \innerprod{Q}{R} \cdot \state{P}
  \and
  (\fprmatrix{P}{x}{Q})(\event{R}) := x \cdot \innerprod{R}{P} \cdot \event{Q}
\end{mathpar}

\paragraph{Discussion}
As promised: vectors (aka states) are represented as processes; duals
as contextual duals; inner product definition should be compared with
standard inner product definition for ....

\begin{remark}
  Assuming $\int (P^{\underline{\perp}}[P]) = 0$, the reader is
  invited to verify that $(\fprmatrix{P}{x}{P})(\state{P}) = x \cdot \state{P}$.
\end{remark}

\begin{remark}
  The reader is invited to verify that $\innerprod{P}{Q}$ could
  equally well have been written $\quotep{\int \stackrel{\vee}{x}}$
  where $x = \event{P^{\underline{\perp}}}(Q)$.

  One of the motivations for this remark is that there is another way
  to factor these operations. We could package up evaluation in the dual:

  \begin{mathpar}
    \state{P}^{*} := \event{\int P^{\underline{\perp}}} := \quotep{\int P^{\underline{\perp}}}[-]
  \end{mathpar}

  and then have inner product defined by
  
  \begin{mathpar}
    \innerprod{P}{Q} := \event{P}(Q)
  \end{mathpar}

  Hopefully, experience with the calculations will provide guidance on
  the best factoring.
\end{remark}

\begin{remark}
  Assuming $\int (P^{\underline{\perp}}[P]) = 0$, the reader is
  invited to verify that $\forall P,Q. (\prmatrix{0}{Q})(\state{0}) =
  \state{0}$ and dually $(\prmatrix{P}{0})(\event{0}) = \event{0}$.
\end{remark}

\begin{remark}
  i'm a little worried that i don't (yet) have proper support for
  complex conjugacy. But, the observation above may give us a
  clue. According to Abramsky, it must be the case that the scalars
  are iso to the homset of the identity for the tensor -- which the
  observation above characterizes. 

  For now, we will simply bookmark the notion with $\overline{x}$.
\end{remark}

\subsubsection{Adjointness}

We need to give a definition of $(\cdot)^{\dagger}$ for matrices. The
obvious candidate definition is
\begin{mathpar}
(\Sigma_{\alpha}\fprmatrix{P_{\alpha}}{x_{\alpha}}{Q_{\alpha}})^{\dagger}
= \Sigma_{\alpha}\fprmatrix{(Q_{\alpha}^{\underline{\perp}})^{*}}{\overline{x}_{\alpha}}{P_{\alpha}^{\underline{\perp}}} 
\end{mathpar}

But, $(Q_{\alpha}^{\underline{\perp}})^{*}$ requires a name along
which to communicate the process to achieve the context application.

\subsubsection{Basis for a basis}
If processes label states and ``addition'' of states (a.k.a. vector
addition) is interpreted as parallel composition, what corresponds to
notions of linear independence and basis? Here, we recall that Yoshida
has developed a set of \emph{combinators} for an asynchronous verison
of Milner's $\pi$-calculus. These are a finite set of processes such
any process can be expressed as parallel composition of these
combinators together with liberal uses of the new operator and
replication. We can simply give a translation of these into the
present calculus and have reasonable expectation that the property
carries over. That is, that the resultant set allows to express all
processes via parallel composition. Note, however, that there is no
new operator or replication in this calculus. As a result, we expect
that the corresponding set is actually infinite. That is, we expect
that the space is actually infinite dimensional.

\begin{remark}
  The attentive reader may be a bit concerned. Certainly, the
  collection $S$, $K$ and $I$ is a finite set of
  combinators. Shouldn't we expect to see a finite set of combinators
  for an effectively equivalent system? i am very sympathetic to this
  critique and feel it warrants full attention. On the other hand, i
  also have in mind the following analogy. The natural numbers, as a
  monoid under addition, has exactly $1$ generator, while the natural
  numbers, as a monoid under multiplication, has countably many
  generators (the primes). We observe that the application of the
  lambda calculus is much less resource sensitive than the parallel
  composition of the $\pi$-calculus. Could it be the case that we have
  an analogy of the form
  
  \begin{mathpar}
    m + n : MN :: m*n : M|N
  \end{mathpar}

  giving a similar blow up in the set of ``primes''?  This is such a
  wonderful thought that, even if it's not true, i think it's worth
  writing down.
\end{remark}
 

\documentclass[12pt]{llncs}
%\documentclass{jktr}

\usepackage[pdftex]{hyperref}                   
\usepackage {listings}
\usepackage {mathpartir}
\usepackage{bcprules}
%\usepackage{listings}
                       
\usepackage{graphicx} 
%\usepackage[margins=2.5cm,nohead,nofoot]{geometry}
%\usepackage{geometry}
\usepackage{amsfonts}
\usepackage{amstext}
\usepackage{latexsym}
\usepackage{amssymb}
\usepackage{color}


%\include{myPreamble}
\include{qm2pi.local} 

%\ifpdf
%\usepackage[pdftex]{graphicx}
%\else
%\usepackage{graphicx}
%\fi

 % \ifpdf
%  \usepackage{pdfsync}
%  \if


%\title{Brief Article}
%\author{David F. Snyder}
%\author{L.G. Meredith}

%\address{Dept. of Math., Texas State University--San Marcos, San Marcos, TX 78666}
       
\pagestyle{empty}


\begin{document}

\lstset{language=[Objective]Caml,frame=shadowbox}

\input{qm2pi.front}

% section front matter (end)

\input{qm2pi.intro} 
 
% section introduction (end)

% \input{qm2pi.knotations} 

% section notation (end)

\input{qm2pi.process.calculi} 

% section concurrent_process_calculi_and_spatial_logics_ (end)
    
%\input{qm2pi.knots2pi} 

%\input{qm2pi.trefoil} 

%\input{qm2pi.mainthm} 

% subsection basic_interpretation (end)

%\input{qm2pi.rho.presentation} 
\subsection{The syntax and semantics of the notation system}\label{sub:the_syntax_and_semantics_of_the_notation_system} % (fold)

We now summarize a technical presentation of the calculus that
embodies our theory of dynamics. The typical presentation of such a
calculus follows the style of giving generators and relations on
them. The grammar, below, describing term constructors, freely
generates the set of processes, $\Proc$. This set is then quotiented
by a relation known as structural congruence and it is over this set
that the notion of dynamics is expressed. This presentation is
essentially that of \cite{MeredithR05} with the addition of
polyadicity and summation. For readability we have relegated some of
the technical subtleties to an appendix.

\subsubsection{Process grammar}\label{subsub:process_grammar}

\begin{mathpar}
  \inferrule* [lab=synchronization] {} {{M} \bc \pzero \;|\; x?F \;|\; x!C }
  \and
  \inferrule* [lab=abstraction] {} {{F} \bc (x)P}
  \and
  \inferrule* [lab=concretion] {} {{C} \bc \langle Q \rangle}
  \and
  \inferrule* [lab=process] {} {{P,Q} \bc M \;| \;P|Q \;|\; @{x}}
  \and
  \inferrule* [lab=name] {} {{x} \bc \quotep{P}}
\end{mathpar} 

Note that $\vec{x}$ (resp. $\vec{P}$) denotes a vector of names
(resp. processes) of length $|\vec{x}|$ (resp. $|\vec{P}|$). We adopt
the following useful abbreviations.

\begin{mathpar}
   x?(\vec{y}).P := x.(\vec{y})P \and  x\clift{\vec{P}} := x.\clift{\vec{P}}
   \and x!(y) := \lift{x}{\dropn{y}}
   \and \Pi_{i=0}^{n-1}P_i := P_0 | \ldots | P_{n-1}
\end{mathpar}

\subsubsection{Structural congruence}

\paragraph{Free and bound names and alpha-equivalence.} At the
core of structural equivalence is alpha-equivalence which identifies
process that are the same up to a change of variable. Formally, we
recognize the distinction between free and bound names. The free names
of a process, $\freenames{P}$, may be calculated recursively as
follows:

\begin{mathpar}
\freenames{\pzero} := \emptyset
  \and \\
  \freenames{x?(y).P} := \{ x \} \cup (\freenames{P} \setminus \{ y \})
  \and 
  \freenames{x!\langle P \rangle} := \{ x \} \cup \{ P \} 
  \and \\
  \freenames{P|Q} := \freenames{P} \cup \freenames{Q}
  \and \\
  \freenames{@{x}} := \{ x \}
\end{mathpar}

$\pi$
$\quotep{\pi}$

$\freenames{-} : \pi \to \mathcal{P}(\quotep{\pi})$

\begin{eqnarray*}
  \freenames{\pzero} & := & \emptyset \\
  \freenames{x?(y).P} & := & \{ x \} \cup (\freenames{P} \setminus \{ y \}) \\
  \freenames{x!\langle P \rangle} & := & \{ x \} \cup \{ P \} \\
  \freenames{P|Q} & := & \freenames{P} \cup \freenames{Q} \\
  \freenames{\dropn{x}} & := & \{ x \}
\end{eqnarray*}

The bound names of a process, $\boundnames{P}$, are those names occurring in $P$
that are not free. For example, in $x?(y).0$, the name $x$ is free, while $y$ is bound.

\begin{mathpar}
  \inferrule* [lab=monoidal-laws] {} { P|Q \equiv Q|P \and P|0 \equiv P \and P|(Q|R) \equiv (P|Q)|R }
\end{mathpar}

\begin{mathpar}
  \inferrule* [lab=alpha-equivalence] {} { (x)P \equiv (y)P\{y/x\} \and y \not\in \freenames{P} }
\end{mathpar}

\begin{definition}
Then two processes, $P,Q$, are alpha-equivalent if $P = Q\{\vec{y}/\vec{x}\}$ for
some $\vec{x} \in \boundnames{Q},\vec{y} \in \boundnames{P}$, where $Q\{\vec{y}/\vec{x}\}$
denotes the capture-avoiding substitution of $\vec{y}$ for $\vec{x}$ in $Q$.
\end{definition}

\begin{definition}
  The {\em structural congruence} \cite{SangiorgiWalker} , $\equiv$,
  between processes is the least congruence containing
  alpha-equivalence, satisfying the abelian monoid laws
  (associativity, commutativity and $\pzero$ as identity) for parallel
  composition $|$ and for summation $+$.
\end{definition}

\subsection{Name equivalence}

We take name equivalence, written $\nameeq$, to be the smallest
equivalence relation generated by the following rules.

\begin{mathpar}
\inferrule*[lab=Quote-drop]
{ }
{ \quotep{@{x}} \nameeq x }

\inferrule*[lab=Struct-equiv]
{ P \scong Q }
{ \quotep{P} \nameeq \quotep{Q} }
\end{mathpar}

The astute reader will have noticed that the mutual recursion of names
and processes imposes a mutual recursion on alpha-equivalence and
structural equivalence via name-equivalence. Fortunately, all of this
works out pleasantly and we may calculate in the natural way, free of
concern. The reader interested in the details is referred to the
appendix \ref{appendix:rho_details}.

\subsection{Substitution}

We use $\Proc$ for the set of processes, $\QProc$ for the set of
names, and $\id{\{}\vec{y} / \vec{x} \id{\}}$ to denote partial maps,
$s : \QProc \rightarrow \QProc$. A map, $s$ lifts, uniquely, to a map
on process terms, $\widehat{s} : \Proc \rightarrow \Proc$ by the
following equations.

\begin{mathpar}
  (0) \psubstp{Q}{P} := 0 \\
  (R \juxtap S) \psubstp{Q}{P}
  :=    
  (R)\psubstp{Q}{P} \juxtap (S) \psubstp{Q}{P} \\
  (x?(y).R) \psubstp{Q}{P}    
  :=    
  (x)\substp{Q}{P} (z)\concat( (R \psubstn{z}{y}) \psubstp{Q}{P} ) \\
  (\lift{x}{R}) \psubstp{Q}{P}  
  :=
  \lift{(x)\substp{Q}{P}}{ R \psubstp{Q}{P} } \\
%   (\dropn{x})  \psubstp{Q}{P}       
%   := 
%   \left\{ 
%     \begin{array}{ccc} 
%       \dropn{\quotep{Q}} & & x \nameeq \quotep{P} \\
%       \dropn{x} & & otherwise \\
%     \end{array}
%   \right. 
  (\dropn{x})  \psubstp{Q}{P}       
  := 
  \left\{ 
    \begin{array}{ccc} 
      Q & & x \nameeq \quotep{P} \\
      \dropn{x} & & otherwise \\
    \end{array}
  \right.
\end{mathpar}
 

where

\begin{eqnarray}
  (x)\id{\{} \lpquote Q \rpquote / \lpquote P \rpquote \id{\}}            = 
  \left\{ 
    \begin{array}{ccc}
      \lpquote Q \rpquote & & x \nameeq \lpquote P \rpquote \\
      x & & otherwise \\
    \end{array}
  \right. \nonumber
\end{eqnarray}

and $z$ is chosen distinct from $\quotep{P}$, $\quotep{Q}$, the free
names in $Q$, and all the names in $R$. Our $\alpha$-equivalence will
be built in the standard way from this substitution.

\begin{remark}\label{rem:no_self_referential_names}
  One consequence of these definitions is that $\forall P. \quotep{P}
  \not\in \freenames{P}$.
\end{remark}

\subsection{ Dynamic quote: an example }

Anticipating something of what's to come, consider applying the
substitution, $\widehat{\id{\{}u / z \id{\}}}$, to the following pair
of processes, $\lift{w}{y!(z)}$ and $w[ \lpquote y!(z) \rpquote ]$.

\begin{eqnarray}
	\lift{w}{y!(z)}\widehat{\id{\{}u / z \id{\}}}
		& = &
		\lift{w}{y!(u)} \nonumber\\
	w[ \lpquote y!(z) \rpquote ] \widehat{ \id{\{}u / z \id{\}} }
		& = &
		w[ \lpquote y!(z) \rpquote ] \nonumber
\end{eqnarray}

Because the body of the process between quotes is impervious to
substitution, we get radically different answers. In fact, by
examining the first process in an input context,
e.g. $x?(z).\lift{w}{y!(z)}$, we see that the process under the lift
operator may be shaped by prefixed inputs binding a name inside it. In
this sense, the lift operator will be seen as a way to dynamically
construct processes before reifying them as names.

Finally equipped with these standard features we can present the
dynamics of the calculus.

\subsubsection{Operational semantics} 

Finally, we introduce the computational dynamics. What marks these
algebras as distinct from other more traditionally studied algebraic
structures, e.g. vector spaces or polynomial rings, is the manner in
which dynamics is captured. In traditional structures, dynamics is typically
expressed through morphisms between such structures, as in linear maps
between vector spaces or morphisms between rings. In algebras
associated with the semantics of computation, the dynamics is
expressed as part of the algebraic structure itself, through a
reduction reduction relation typically denoted by $\red$. Below, we
give a recursive presentation of this relation for the calculus used
in the encoding.

$\red \subseteq \pi \times \pi$
$\red : \pi \to \mathcal{P}(\pi)$

\begin{mathpar}
  \inferrule* [lab=Comm] { \textsf{match}( x_{src}, x_{trgt} ) } { x_{trgt}?(y)P \; | \; x_{src}!\langle {Q} \rangle \red P\{\quotep{Q}/y}\} }
  \and \\
  \inferrule* [lab=Par] {{P} \red {P}'} {{{P} | {Q}} \red {{P}' | {Q}}}
  \and
  \inferrule* [lab=Equiv]{{{P} \scong {P}'} \andalso {{P}' \red {Q}'} \andalso {{Q}' \scong {Q}}}{{P} \red {Q}}
\end{mathpar}

\begin{eqnarray*}
  match_{\equiv} (\quotep{P},\quotep{Q}) & := & P \equiv Q \\
  match_{\dagger}(\quotep{P},\quotep{Q}) & := & \forall R. P|Q \red^{*} R => R \red^{*} 0 \\
  match_{K}(\quotep{P},\quotep{Q}) & := & K \mbox{ for some context } K
\end{eqnarray*}

$u?(x)P | u!\langle Q \rangle \red P\{\quotep{Q}/x\}$

%We write $\wred$ for $\red^*$, and $P\red$ if $\exists Q $ such that $ P \red Q$.
We write $P\red$ if $\exists Q $ such that $ P \red Q$ and $P\not\red$, otherwise.

\section{Replication}

As mentioned before, it is known that replication (and hence
recursion) can be implemented in a higher-order process algebra
\cite{SangiorgiWalker}. As our first example of calculation with the
machinery thus far presented we give the construction explicitly in
the {\rhoc}.

\begin{eqnarray}
	D_{x} & := & \prefix{x}{y}{(\binpar{\outputp{x}{y}}{@{y}})} \nonumber\\
	\bangp_{x}{P} & := & \binpar{{x}!\langle{\binpar{D_{x}}{P}}\rangle}{D_{x}} \nonumber
\end{eqnarray}

\begin{eqnarray}
	\bangp_{x}{P} & & \nonumber\\
	=
	& {x}!\langle{(\prefix{x}{y}{(\outputp{x}{y} | @{y})) | P}}\rangle 
	      | \prefix{x}{y}{(\outputp{x}{y} | @{y})} & \nonumber\\
	\red
	& (\outputp{x}{y} | @{y})\substn{\quotep{(\prefix{x}{y}{(@{y} | \outputp{x}{y})) | P}}}{y} & \nonumber\\
	=
	& \outputp{x}{\quotep{(\prefix{x}{y}{(\outputp{x}{y} | @{y})) | P}}}
	  | {(\prefix{x}{y}{(\outputp{x}{y} | @{y})) | P}} & \nonumber\\
	\red
	& \ldots & \nonumber\\
	\red^*
	& P | P | \ldots & \nonumber
\end{eqnarray}

Of course, this encoding, as an implementation, runs away, unfolding
$\bangp{P}$ eagerly. A lazier and more implementable replication
operator, restricted to input-guarded processes, may be obtained as follows.

\begin{eqnarray}
\bangp{\prefix{u}{v}{P}} 
	:= 
	\binpar{\lift{x}{\prefix{u}{v}{(\binpar{D(x)}{P})}}}{D(x)} \nonumber
\end{eqnarray}

\begin{remark}
  Note that the lazier definition still does not deal with summation
  or mixed summation (i.e. sums over input and output). The reader is
  invited to construct definitions of replication that deal with these
  features. 

  Further, the definitions are parameterized in a name, $x$. Can you,
  gentle reader, make a definition that eliminates this parameter and
  guarantees no accidental interaction between the replication
  machinery and the process being replicated -- i.e. no accidental
  sharing of names used by the process to get its work done and the
  name(s) used by the replication to effect copying. This latter
  revision of the definition of replication is crucial to obtaining
  the expected identity $!!P \sim !P$.
\end{remark}

\begin{remark}\label{rem:paradoxical_combinator}
  The reader familiar with the lambda calculus will have noticed the
  similarity between $D$ and the paradoxical combinator.

  [Ed. note: the existence of this seems to suggest we have to be more
  restrictive on the set of processes and names we admit if we are to
  support no-cloning.]
\end{remark}

\subsubsection{Bisimulation}

The computational dynamics gives rise to another kind of equivalence,
the equivalence of computational behavior. As previously mentioned
this is typically captured \emph{via} some form of bisimulation.

% The notion we use in this paper is weak barbed bisimulation
% \cite{milner91polyadicpi}.

The notion we use in this paper is derived from weak barbed
bisimulation \cite{milner91polyadicpi}. 

\begin{definition}
An \emph{observation relation}, $\downarrow_{\mathcal N}$, over a set
of names, $\mathcal N$, is the smallest relation satisfying the rules
below.

\infrule[Out-barb]{y \in {\mathcal N}, \; x \nameeq y}
		  {\outputp{x}{v} \downarrow_{\mathcal N} x}
\infrule[Par-barb]{\mbox{$P\downarrow_{\mathcal N} x$ or $Q\downarrow_{\mathcal N} x$}}
		  {\binpar{P}{Q} \downarrow_{\mathcal N} x}

We write $P \Downarrow_{\mathcal N} x$ if there is $Q$ such that 
$P \wred Q$ and $Q \downarrow_{\mathcal N} x$.
\end{definition}

\begin{definition}
%\label{def.bbisim}
An  ${\mathcal N}$-\emph{barbed bisimulation} over a set of names, ${\mathcal N}$, is a symmetric binary relation 
${\mathcal S}_{\mathcal N}$ between agents such that $P\rel{S}_{\mathcal N}Q$ implies:
\begin{enumerate}
\item If $P \red P'$ then $Q \wred Q'$ and $P'\rel{S}_{\mathcal N} Q'$.
\item If $P\downarrow_{\mathcal N} x$, then $Q\Downarrow_{\mathcal N} x$.
\end{enumerate}
$P$ is ${\mathcal N}$-barbed bisimilar to $Q$, written
$P \wbbisim_{\mathcal N} Q$, if $P \rel{S}_{\mathcal N} Q$ for some ${\mathcal N}$-barbed bisimulation ${\mathcal S}_{\mathcal N}$.
\end{definition}

$\mathcal{R} \subseteq \pi \times \pi$

$P \mathcal{R} Q => \forall P'. P \red P' \Rightarrow \exists Q'. Q \red Q', P' \mathcal{R} Q'$

$P \vdash x \Rightarrow Q \vdash x$

\begin{mathpar}
  \inferrule*[lab=Out-barb]{x \nameeq y}{{y}!\langle{Q}\rangle \vdash x}
  \and
  \inferrule*[lab=Par-barb]{\mbox{$P\vdash x$ or $Q\vdash x$}}{\binpar{P}{Q} \vdash x}
\end{mathpar}

\subsubsection{Contexts}

One of the principle advantages of computational calculi like the
$\pi$-calculus is a well-defined notion of context,
contextual-equivalence and a correlation between
contextual-equivalence and notions of bisimulation. The notion of
context allows the decomposition of a process into (sub-)process and
its syntactic environment, its context. Thus, a context may be
thought of as a process with a ``hole'' (written $\Box$) in it. The
application of a context $M$ to a process $P$, written $M[P]$, is
tantamount to filling the hole in $M$ with $P$. In this paper we do
not need the full weight of this theory, but do make use of the notion
of context in the proof the main theorem. 

\begin{mathpar}
  \inferrule* [lab=summation] {} {{M_{M},M_{N}} \bc \Box \;|\; x.M_{A} \;|\; M_{M}+M_{N}}
  \and
  \inferrule* [lab=agent] {} {{M_{A}} \bc (\vec{x})M_{P} \;| \; \clift{P_0,\ldots,M_{P},\ldots,P_N}}
  \and \\
  \inferrule* [lab=process] {} {{M_{P}} \bc M_{N} \;| \;P|M_{P} }
\end{mathpar} 

\begin{mathpar}
  \inferrule* [lab=sychronization] {} {M_{N} \bc \Box \;|\; x?M_{F} \;|\; x!M_{C}}
  \and
  \inferrule* [lab=abstraction] {} {{M_{F}} \bc (x)M_{P} }
  \and
  \inferrule* [lab=concretion] {} {{M_{C}} \bc \langle M_{P} \rangle }
  \and \\
  \inferrule* [lab=process] {} {{M_{P}} \bc M_{N} \;| \;P|M_{P} }
\end{mathpar}

\begin{definition}[contextual application] Given a context $M$, and
  process $P$, we define the \emph{contextual application}, $M[P] :=
  M\{P/\Box\}$. That is, the contextual application of M to P is the
  substitution of $P$ for $\Box$ in $M$.
\end{definition}

$\meaningof{-} : L \to \mathcal{P}(\pi)$

\begin{mathpar}
  \inferrule* [lab=collection] {} {\meaningof{true} = \pi, \and \meaningof{~E} = \pi \setminus \meaningof{E}, \and \meaningof{E_{1} \& E_{2}} = \meaningof{E_{1}} \cap \meaningof{E_{2}}}
\end{mathpar}

\begin{mathpar}
  \inferrule* [lab=structure] {} {\meaningof{0} = \{ P \in \pi | P \equiv 0 \}, \and \\ \meaningof{E_1 | E_2} = \{ P \in \pi | P \equiv P_{1} | P_{2}, P_{1} \in \meaningof{E_{1}}, P_{2} \in \meaningof{E_2}\} }
\end{mathpar}

\begin{mathpar}
 \inferrule* [lab=behavior] {} {\meaningof{\langle a?b \rangle E} = \{ P \in \pi | P \equiv Q | u?(y)P', \\ \and \\\\ \and \\ \;\;\; u \in \meaningof{a}, \forall z.P'\{z/y\} \in \meaningof{E\{z/b\}}\}, \and \\ \meaningof{a!E} = \{ P \in \pi | P \equiv Q | x!\langle P' \rangle, x \in \meaningof{a} P' \in \meaningof{E}\} }
\end{mathpar}

\begin{mathpar}
 \inferrule* [lab=nominal] {} {\meaningof{\quotep{E}} = \{ \quotep{P} \in \quotep{\pi} | P \in \meaningof{E} \}, \and \meaningof{\quotep{P}} = \{ \quotep{Q} \in \quotep{\pi} | P \equiv Q \} \and \\ \meaningof{@\quotep{E}} = \{ P \in \pi | P \equiv @x, x \in \meaningof{E} \}}
\end{mathpar}

\begin{eqnarray*}
  \\
  \meaningof{-} : TS \to ST
\end{eqnarray*}

\begin{eqnarray*}
  \\
  L : TS \to ST
\end{eqnarray*}

\begin{eqnarray*}
  \\
  P \models E \iff P \in \meaningof{E}
\end{eqnarray*}

\begin{eqnarray*}
  P \approx_{L} Q \iff \forall E \in L. P \models E \iff Q \models E
\end{eqnarray*}

\begin{eqnarray*}
  P \approx_{K} Q
\end{eqnarray*}

\begin{eqnarray*}
  P \approx Q
\end{eqnarray*}

$\approx_{K} = \approx = \approx_{L}$

\subsubsection{Contextual duality}

Note that contexts extend the quotation operation to a family of
operations from processes to names. Given a context, $M$, we can
define a \emph{nominal context}, $\quotep{M}$ by $\quotep{M}[P] :=
\quotep{M[P]}$. To foreshadow what is to come we observe that these
operations enjoy a duality with processes very much like the duality
between vectors and maps from vectors to scalars.

Further, because the calculus is essentially higher-order, we have a
correspondence between contexts and processes. More specifically,
given a name $x$ and a context $M$ we can construct $M^{*}_{x}$ such
that 

\begin{mathpar}
  M^{*}_{x} | \lift{x}{P} \red M[P]
\end{mathpar}

namely,

\begin{mathpar}
  M^{*}_{x} := x?(u).M[\dropn{u}]
\end{mathpar}

The dependence of $M^{*}_{x}$ on a name makes it an abstraction, 

\begin{mathpar}
  M^{*} := (x)x?(u).M[\dropn{u}]
\end{mathpar}

\subsection{Additional notation}

It will sometimes be convenient to denote the process a name
quotes. We already have the notation $x = \quotep{P}$, but it will be
convenient to introduce an alternate notation, $\procn{x}$, when we
want to emphasize the connection to the use of the name. Note that, by
virtue of name equivalence, $\quotep{\procn{x}} \nameeq x$; so, the
notation is consistent with previous definitions.

Further, because names have structure it is possible to effect
substitutions on the basis of that structure. This means we need to
upgrade our notation for substitutions, which we accomplish by
adapting comprehension notation. Thus,

\begin{mathpar}
  P\{ y / x : x \in S \}
\end{mathpar}

is interpreted to mean the process derived from P by replacing (in a
capture-avoiding manner) each occurrence of $x$ in $S$ by $y$. For example,

\begin{mathpar}
  P\{ \quotep{\procn{x}|\procn{x}} / x : x \in \freenames{P} \}
\end{mathpar}

will replace each (occurrence) of a free name $x$ in $P$ by
$\quotep{\procn{x}|\procn{x}}$.

Also, we will avail ourselves of the notation $x^{L}$ and $x^{R}$ to
denote injections of a name into disjoint copies of the name
space. There are numerous ways to accomplish this. One example can be
found in \cite{MeredithR05}. This notation overloads to vectors of
names: $\vec{x}^{\pi} := (x_{i}^{\pi} \; : \; 0 \leq i < |\vec{x}| )$ where $\pi \in \{L,R\}$.

We also use $P^{\Box} := P|\Box$.

In \cite{MeredithR05} an interpretation of the new operator is
given. It turns out that there are several possible interpretations
all enjoying the requisite algebraic properties of the operator (see
\cite{milner91polyadicpi}). We will therefore make liberal use of
$(\nu\; \vec{x})P$.

% subsection the_syntax_and_semantics_of_the_notation_system (end)   

\input{qm2pi.qmops} 

\input{qm2pi.sterngerlach} 

\input{qm2pi.metric} 

% section concurrent_process_calculi (end)

%\input{qm2pi.proofsketch}

% section proof sketch (end)

%\input{qm2pi.slviaknots} 

% section spatial logic via knots (end)

\input{qm2pi.conclusion}

% section conclusion (end)

%\input{qm2pi.dtcodes} 

% section wiring algorithm (end)

\input{qm2pi.ack} 

% section acknowledgments (end)

\newpage


\bibliographystyle{plain}   
\bibliography{../../biblios/main.bib}

\input{qm2pi.rhodetails}

\end{document}

 

\documentclass[12pt]{llncs}
%\documentclass{jktr}

\usepackage[pdftex]{hyperref}                   
\usepackage {listings}
\usepackage {mathpartir}
\usepackage{bcprules}
%\usepackage{listings}
                       
\usepackage{graphicx} 
%\usepackage[margins=2.5cm,nohead,nofoot]{geometry}
%\usepackage{geometry}
\usepackage{amsfonts}
\usepackage{amstext}
\usepackage{latexsym}
\usepackage{amssymb}
\usepackage{color}


%\include{myPreamble}
\include{qm2pi.local} 

%\ifpdf
%\usepackage[pdftex]{graphicx}
%\else
%\usepackage{graphicx}
%\fi

 % \ifpdf
%  \usepackage{pdfsync}
%  \if


%\title{Brief Article}
%\author{David F. Snyder}
%\author{L.G. Meredith}

%\address{Dept. of Math., Texas State University--San Marcos, San Marcos, TX 78666}
       
\pagestyle{empty}


\begin{document}

\lstset{language=[Objective]Caml,frame=shadowbox}

\input{qm2pi.front}

% section front matter (end)

\input{qm2pi.intro} 
 
% section introduction (end)

% \input{qm2pi.knotations} 

% section notation (end)

\input{qm2pi.process.calculi} 

% section concurrent_process_calculi_and_spatial_logics_ (end)
    
%\input{qm2pi.knots2pi} 

%\input{qm2pi.trefoil} 

%\input{qm2pi.mainthm} 

% subsection basic_interpretation (end)

%\input{qm2pi.rho.presentation} 
\subsection{The syntax and semantics of the notation system}\label{sub:the_syntax_and_semantics_of_the_notation_system} % (fold)

We now summarize a technical presentation of the calculus that
embodies our theory of dynamics. The typical presentation of such a
calculus follows the style of giving generators and relations on
them. The grammar, below, describing term constructors, freely
generates the set of processes, $\Proc$. This set is then quotiented
by a relation known as structural congruence and it is over this set
that the notion of dynamics is expressed. This presentation is
essentially that of \cite{MeredithR05} with the addition of
polyadicity and summation. For readability we have relegated some of
the technical subtleties to an appendix.

\subsubsection{Process grammar}\label{subsub:process_grammar}

\begin{mathpar}
  \inferrule* [lab=synchronization] {} {{M} \bc \pzero \;|\; x?F \;|\; x!C }
  \and
  \inferrule* [lab=abstraction] {} {{F} \bc (x)P}
  \and
  \inferrule* [lab=concretion] {} {{C} \bc \langle Q \rangle}
  \and
  \inferrule* [lab=process] {} {{P,Q} \bc M \;| \;P|Q \;|\; @{x}}
  \and
  \inferrule* [lab=name] {} {{x} \bc \quotep{P}}
\end{mathpar} 

Note that $\vec{x}$ (resp. $\vec{P}$) denotes a vector of names
(resp. processes) of length $|\vec{x}|$ (resp. $|\vec{P}|$). We adopt
the following useful abbreviations.

\begin{mathpar}
   x?(\vec{y}).P := x.(\vec{y})P \and  x\clift{\vec{P}} := x.\clift{\vec{P}}
   \and x!(y) := \lift{x}{\dropn{y}}
   \and \Pi_{i=0}^{n-1}P_i := P_0 | \ldots | P_{n-1}
\end{mathpar}

\subsubsection{Structural congruence}

\paragraph{Free and bound names and alpha-equivalence.} At the
core of structural equivalence is alpha-equivalence which identifies
process that are the same up to a change of variable. Formally, we
recognize the distinction between free and bound names. The free names
of a process, $\freenames{P}$, may be calculated recursively as
follows:

\begin{mathpar}
\freenames{\pzero} := \emptyset
  \and \\
  \freenames{x?(y).P} := \{ x \} \cup (\freenames{P} \setminus \{ y \})
  \and 
  \freenames{x!\langle P \rangle} := \{ x \} \cup \{ P \} 
  \and \\
  \freenames{P|Q} := \freenames{P} \cup \freenames{Q}
  \and \\
  \freenames{@{x}} := \{ x \}
\end{mathpar}

$\pi$
$\quotep{\pi}$

$\freenames{-} : \pi \to \mathcal{P}(\quotep{\pi})$

\begin{eqnarray*}
  \freenames{\pzero} & := & \emptyset \\
  \freenames{x?(y).P} & := & \{ x \} \cup (\freenames{P} \setminus \{ y \}) \\
  \freenames{x!\langle P \rangle} & := & \{ x \} \cup \{ P \} \\
  \freenames{P|Q} & := & \freenames{P} \cup \freenames{Q} \\
  \freenames{\dropn{x}} & := & \{ x \}
\end{eqnarray*}

The bound names of a process, $\boundnames{P}$, are those names occurring in $P$
that are not free. For example, in $x?(y).0$, the name $x$ is free, while $y$ is bound.

\begin{mathpar}
  \inferrule* [lab=monoidal-laws] {} { P|Q \equiv Q|P \and P|0 \equiv P \and P|(Q|R) \equiv (P|Q)|R }
\end{mathpar}

\begin{mathpar}
  \inferrule* [lab=alpha-equivalence] {} { (x)P \equiv (y)P\{y/x\} \and y \not\in \freenames{P} }
\end{mathpar}

\begin{definition}
Then two processes, $P,Q$, are alpha-equivalent if $P = Q\{\vec{y}/\vec{x}\}$ for
some $\vec{x} \in \boundnames{Q},\vec{y} \in \boundnames{P}$, where $Q\{\vec{y}/\vec{x}\}$
denotes the capture-avoiding substitution of $\vec{y}$ for $\vec{x}$ in $Q$.
\end{definition}

\begin{definition}
  The {\em structural congruence} \cite{SangiorgiWalker} , $\equiv$,
  between processes is the least congruence containing
  alpha-equivalence, satisfying the abelian monoid laws
  (associativity, commutativity and $\pzero$ as identity) for parallel
  composition $|$ and for summation $+$.
\end{definition}

\subsection{Name equivalence}

We take name equivalence, written $\nameeq$, to be the smallest
equivalence relation generated by the following rules.

\begin{mathpar}
\inferrule*[lab=Quote-drop]
{ }
{ \quotep{@{x}} \nameeq x }

\inferrule*[lab=Struct-equiv]
{ P \scong Q }
{ \quotep{P} \nameeq \quotep{Q} }
\end{mathpar}

The astute reader will have noticed that the mutual recursion of names
and processes imposes a mutual recursion on alpha-equivalence and
structural equivalence via name-equivalence. Fortunately, all of this
works out pleasantly and we may calculate in the natural way, free of
concern. The reader interested in the details is referred to the
appendix \ref{appendix:rho_details}.

\subsection{Substitution}

We use $\Proc$ for the set of processes, $\QProc$ for the set of
names, and $\id{\{}\vec{y} / \vec{x} \id{\}}$ to denote partial maps,
$s : \QProc \rightarrow \QProc$. A map, $s$ lifts, uniquely, to a map
on process terms, $\widehat{s} : \Proc \rightarrow \Proc$ by the
following equations.

\begin{mathpar}
  (0) \psubstp{Q}{P} := 0 \\
  (R \juxtap S) \psubstp{Q}{P}
  :=    
  (R)\psubstp{Q}{P} \juxtap (S) \psubstp{Q}{P} \\
  (x?(y).R) \psubstp{Q}{P}    
  :=    
  (x)\substp{Q}{P} (z)\concat( (R \psubstn{z}{y}) \psubstp{Q}{P} ) \\
  (\lift{x}{R}) \psubstp{Q}{P}  
  :=
  \lift{(x)\substp{Q}{P}}{ R \psubstp{Q}{P} } \\
%   (\dropn{x})  \psubstp{Q}{P}       
%   := 
%   \left\{ 
%     \begin{array}{ccc} 
%       \dropn{\quotep{Q}} & & x \nameeq \quotep{P} \\
%       \dropn{x} & & otherwise \\
%     \end{array}
%   \right. 
  (\dropn{x})  \psubstp{Q}{P}       
  := 
  \left\{ 
    \begin{array}{ccc} 
      Q & & x \nameeq \quotep{P} \\
      \dropn{x} & & otherwise \\
    \end{array}
  \right.
\end{mathpar}
 

where

\begin{eqnarray}
  (x)\id{\{} \lpquote Q \rpquote / \lpquote P \rpquote \id{\}}            = 
  \left\{ 
    \begin{array}{ccc}
      \lpquote Q \rpquote & & x \nameeq \lpquote P \rpquote \\
      x & & otherwise \\
    \end{array}
  \right. \nonumber
\end{eqnarray}

and $z$ is chosen distinct from $\quotep{P}$, $\quotep{Q}$, the free
names in $Q$, and all the names in $R$. Our $\alpha$-equivalence will
be built in the standard way from this substitution.

\begin{remark}\label{rem:no_self_referential_names}
  One consequence of these definitions is that $\forall P. \quotep{P}
  \not\in \freenames{P}$.
\end{remark}

\subsection{ Dynamic quote: an example }

Anticipating something of what's to come, consider applying the
substitution, $\widehat{\id{\{}u / z \id{\}}}$, to the following pair
of processes, $\lift{w}{y!(z)}$ and $w[ \lpquote y!(z) \rpquote ]$.

\begin{eqnarray}
	\lift{w}{y!(z)}\widehat{\id{\{}u / z \id{\}}}
		& = &
		\lift{w}{y!(u)} \nonumber\\
	w[ \lpquote y!(z) \rpquote ] \widehat{ \id{\{}u / z \id{\}} }
		& = &
		w[ \lpquote y!(z) \rpquote ] \nonumber
\end{eqnarray}

Because the body of the process between quotes is impervious to
substitution, we get radically different answers. In fact, by
examining the first process in an input context,
e.g. $x?(z).\lift{w}{y!(z)}$, we see that the process under the lift
operator may be shaped by prefixed inputs binding a name inside it. In
this sense, the lift operator will be seen as a way to dynamically
construct processes before reifying them as names.

Finally equipped with these standard features we can present the
dynamics of the calculus.

\subsubsection{Operational semantics} 

Finally, we introduce the computational dynamics. What marks these
algebras as distinct from other more traditionally studied algebraic
structures, e.g. vector spaces or polynomial rings, is the manner in
which dynamics is captured. In traditional structures, dynamics is typically
expressed through morphisms between such structures, as in linear maps
between vector spaces or morphisms between rings. In algebras
associated with the semantics of computation, the dynamics is
expressed as part of the algebraic structure itself, through a
reduction reduction relation typically denoted by $\red$. Below, we
give a recursive presentation of this relation for the calculus used
in the encoding.

$\red \subseteq \pi \times \pi$
$\red : \pi \to \mathcal{P}(\pi)$

\begin{mathpar}
  \inferrule* [lab=Comm] { \textsf{match}( x_{src}, x_{trgt} ) } { x_{trgt}?(y)P \; | \; x_{src}!\langle {Q} \rangle \red P\{\quotep{Q}/y}\} }
  \and \\
  \inferrule* [lab=Par] {{P} \red {P}'} {{{P} | {Q}} \red {{P}' | {Q}}}
  \and
  \inferrule* [lab=Equiv]{{{P} \scong {P}'} \andalso {{P}' \red {Q}'} \andalso {{Q}' \scong {Q}}}{{P} \red {Q}}
\end{mathpar}

\begin{eqnarray*}
  match_{\equiv} (\quotep{P},\quotep{Q}) & := & P \equiv Q \\
  match_{\dagger}(\quotep{P},\quotep{Q}) & := & \forall R. P|Q \red^{*} R => R \red^{*} 0 \\
  match_{K}(\quotep{P},\quotep{Q}) & := & K \mbox{ for some context } K
\end{eqnarray*}

$u?(x)P | u!\langle Q \rangle \red P\{\quotep{Q}/x\}$

%We write $\wred$ for $\red^*$, and $P\red$ if $\exists Q $ such that $ P \red Q$.
We write $P\red$ if $\exists Q $ such that $ P \red Q$ and $P\not\red$, otherwise.

\section{Replication}

As mentioned before, it is known that replication (and hence
recursion) can be implemented in a higher-order process algebra
\cite{SangiorgiWalker}. As our first example of calculation with the
machinery thus far presented we give the construction explicitly in
the {\rhoc}.

\begin{eqnarray}
	D_{x} & := & \prefix{x}{y}{(\binpar{\outputp{x}{y}}{@{y}})} \nonumber\\
	\bangp_{x}{P} & := & \binpar{{x}!\langle{\binpar{D_{x}}{P}}\rangle}{D_{x}} \nonumber
\end{eqnarray}

\begin{eqnarray}
	\bangp_{x}{P} & & \nonumber\\
	=
	& {x}!\langle{(\prefix{x}{y}{(\outputp{x}{y} | @{y})) | P}}\rangle 
	      | \prefix{x}{y}{(\outputp{x}{y} | @{y})} & \nonumber\\
	\red
	& (\outputp{x}{y} | @{y})\substn{\quotep{(\prefix{x}{y}{(@{y} | \outputp{x}{y})) | P}}}{y} & \nonumber\\
	=
	& \outputp{x}{\quotep{(\prefix{x}{y}{(\outputp{x}{y} | @{y})) | P}}}
	  | {(\prefix{x}{y}{(\outputp{x}{y} | @{y})) | P}} & \nonumber\\
	\red
	& \ldots & \nonumber\\
	\red^*
	& P | P | \ldots & \nonumber
\end{eqnarray}

Of course, this encoding, as an implementation, runs away, unfolding
$\bangp{P}$ eagerly. A lazier and more implementable replication
operator, restricted to input-guarded processes, may be obtained as follows.

\begin{eqnarray}
\bangp{\prefix{u}{v}{P}} 
	:= 
	\binpar{\lift{x}{\prefix{u}{v}{(\binpar{D(x)}{P})}}}{D(x)} \nonumber
\end{eqnarray}

\begin{remark}
  Note that the lazier definition still does not deal with summation
  or mixed summation (i.e. sums over input and output). The reader is
  invited to construct definitions of replication that deal with these
  features. 

  Further, the definitions are parameterized in a name, $x$. Can you,
  gentle reader, make a definition that eliminates this parameter and
  guarantees no accidental interaction between the replication
  machinery and the process being replicated -- i.e. no accidental
  sharing of names used by the process to get its work done and the
  name(s) used by the replication to effect copying. This latter
  revision of the definition of replication is crucial to obtaining
  the expected identity $!!P \sim !P$.
\end{remark}

\begin{remark}\label{rem:paradoxical_combinator}
  The reader familiar with the lambda calculus will have noticed the
  similarity between $D$ and the paradoxical combinator.

  [Ed. note: the existence of this seems to suggest we have to be more
  restrictive on the set of processes and names we admit if we are to
  support no-cloning.]
\end{remark}

\subsubsection{Bisimulation}

The computational dynamics gives rise to another kind of equivalence,
the equivalence of computational behavior. As previously mentioned
this is typically captured \emph{via} some form of bisimulation.

% The notion we use in this paper is weak barbed bisimulation
% \cite{milner91polyadicpi}.

The notion we use in this paper is derived from weak barbed
bisimulation \cite{milner91polyadicpi}. 

\begin{definition}
An \emph{observation relation}, $\downarrow_{\mathcal N}$, over a set
of names, $\mathcal N$, is the smallest relation satisfying the rules
below.

\infrule[Out-barb]{y \in {\mathcal N}, \; x \nameeq y}
		  {\outputp{x}{v} \downarrow_{\mathcal N} x}
\infrule[Par-barb]{\mbox{$P\downarrow_{\mathcal N} x$ or $Q\downarrow_{\mathcal N} x$}}
		  {\binpar{P}{Q} \downarrow_{\mathcal N} x}

We write $P \Downarrow_{\mathcal N} x$ if there is $Q$ such that 
$P \wred Q$ and $Q \downarrow_{\mathcal N} x$.
\end{definition}

\begin{definition}
%\label{def.bbisim}
An  ${\mathcal N}$-\emph{barbed bisimulation} over a set of names, ${\mathcal N}$, is a symmetric binary relation 
${\mathcal S}_{\mathcal N}$ between agents such that $P\rel{S}_{\mathcal N}Q$ implies:
\begin{enumerate}
\item If $P \red P'$ then $Q \wred Q'$ and $P'\rel{S}_{\mathcal N} Q'$.
\item If $P\downarrow_{\mathcal N} x$, then $Q\Downarrow_{\mathcal N} x$.
\end{enumerate}
$P$ is ${\mathcal N}$-barbed bisimilar to $Q$, written
$P \wbbisim_{\mathcal N} Q$, if $P \rel{S}_{\mathcal N} Q$ for some ${\mathcal N}$-barbed bisimulation ${\mathcal S}_{\mathcal N}$.
\end{definition}

$\mathcal{R} \subseteq \pi \times \pi$

$P \mathcal{R} Q => \forall P'. P \red P' \Rightarrow \exists Q'. Q \red Q', P' \mathcal{R} Q'$

$P \vdash x \Rightarrow Q \vdash x$

\begin{mathpar}
  \inferrule*[lab=Out-barb]{x \nameeq y}{{y}!\langle{Q}\rangle \vdash x}
  \and
  \inferrule*[lab=Par-barb]{\mbox{$P\vdash x$ or $Q\vdash x$}}{\binpar{P}{Q} \vdash x}
\end{mathpar}

\subsubsection{Contexts}

One of the principle advantages of computational calculi like the
$\pi$-calculus is a well-defined notion of context,
contextual-equivalence and a correlation between
contextual-equivalence and notions of bisimulation. The notion of
context allows the decomposition of a process into (sub-)process and
its syntactic environment, its context. Thus, a context may be
thought of as a process with a ``hole'' (written $\Box$) in it. The
application of a context $M$ to a process $P$, written $M[P]$, is
tantamount to filling the hole in $M$ with $P$. In this paper we do
not need the full weight of this theory, but do make use of the notion
of context in the proof the main theorem. 

\begin{mathpar}
  \inferrule* [lab=summation] {} {{M_{M},M_{N}} \bc \Box \;|\; x.M_{A} \;|\; M_{M}+M_{N}}
  \and
  \inferrule* [lab=agent] {} {{M_{A}} \bc (\vec{x})M_{P} \;| \; \clift{P_0,\ldots,M_{P},\ldots,P_N}}
  \and \\
  \inferrule* [lab=process] {} {{M_{P}} \bc M_{N} \;| \;P|M_{P} }
\end{mathpar} 

\begin{mathpar}
  \inferrule* [lab=sychronization] {} {M_{N} \bc \Box \;|\; x?M_{F} \;|\; x!M_{C}}
  \and
  \inferrule* [lab=abstraction] {} {{M_{F}} \bc (x)M_{P} }
  \and
  \inferrule* [lab=concretion] {} {{M_{C}} \bc \langle M_{P} \rangle }
  \and \\
  \inferrule* [lab=process] {} {{M_{P}} \bc M_{N} \;| \;P|M_{P} }
\end{mathpar}

\begin{definition}[contextual application] Given a context $M$, and
  process $P$, we define the \emph{contextual application}, $M[P] :=
  M\{P/\Box\}$. That is, the contextual application of M to P is the
  substitution of $P$ for $\Box$ in $M$.
\end{definition}

$\meaningof{-} : L \to \mathcal{P}(\pi)$

\begin{mathpar}
  \inferrule* [lab=collection] {} {\meaningof{true} = \pi, \and \meaningof{~E} = \pi \setminus \meaningof{E}, \and \meaningof{E_{1} \& E_{2}} = \meaningof{E_{1}} \cap \meaningof{E_{2}}}
\end{mathpar}

\begin{mathpar}
  \inferrule* [lab=structure] {} {\meaningof{0} = \{ P \in \pi | P \equiv 0 \}, \and \\ \meaningof{E_1 | E_2} = \{ P \in \pi | P \equiv P_{1} | P_{2}, P_{1} \in \meaningof{E_{1}}, P_{2} \in \meaningof{E_2}\} }
\end{mathpar}

\begin{mathpar}
 \inferrule* [lab=behavior] {} {\meaningof{\langle a?b \rangle E} = \{ P \in \pi | P \equiv Q | u?(y)P', \\ \and \\\\ \and \\ \;\;\; u \in \meaningof{a}, \forall z.P'\{z/y\} \in \meaningof{E\{z/b\}}\}, \and \\ \meaningof{a!E} = \{ P \in \pi | P \equiv Q | x!\langle P' \rangle, x \in \meaningof{a} P' \in \meaningof{E}\} }
\end{mathpar}

\begin{mathpar}
 \inferrule* [lab=nominal] {} {\meaningof{\quotep{E}} = \{ \quotep{P} \in \quotep{\pi} | P \in \meaningof{E} \}, \and \meaningof{\quotep{P}} = \{ \quotep{Q} \in \quotep{\pi} | P \equiv Q \} \and \\ \meaningof{@\quotep{E}} = \{ P \in \pi | P \equiv @x, x \in \meaningof{E} \}}
\end{mathpar}

\begin{eqnarray*}
  \\
  \meaningof{-} : TS \to ST
\end{eqnarray*}

\begin{eqnarray*}
  \\
  L : TS \to ST
\end{eqnarray*}

\begin{eqnarray*}
  \\
  P \models E \iff P \in \meaningof{E}
\end{eqnarray*}

\begin{eqnarray*}
  P \approx_{L} Q \iff \forall E \in L. P \models E \iff Q \models E
\end{eqnarray*}

\begin{eqnarray*}
  P \approx_{K} Q
\end{eqnarray*}

\begin{eqnarray*}
  P \approx Q
\end{eqnarray*}

$\approx_{K} = \approx = \approx_{L}$

\subsubsection{Contextual duality}

Note that contexts extend the quotation operation to a family of
operations from processes to names. Given a context, $M$, we can
define a \emph{nominal context}, $\quotep{M}$ by $\quotep{M}[P] :=
\quotep{M[P]}$. To foreshadow what is to come we observe that these
operations enjoy a duality with processes very much like the duality
between vectors and maps from vectors to scalars.

Further, because the calculus is essentially higher-order, we have a
correspondence between contexts and processes. More specifically,
given a name $x$ and a context $M$ we can construct $M^{*}_{x}$ such
that 

\begin{mathpar}
  M^{*}_{x} | \lift{x}{P} \red M[P]
\end{mathpar}

namely,

\begin{mathpar}
  M^{*}_{x} := x?(u).M[\dropn{u}]
\end{mathpar}

The dependence of $M^{*}_{x}$ on a name makes it an abstraction, 

\begin{mathpar}
  M^{*} := (x)x?(u).M[\dropn{u}]
\end{mathpar}

\subsection{Additional notation}

It will sometimes be convenient to denote the process a name
quotes. We already have the notation $x = \quotep{P}$, but it will be
convenient to introduce an alternate notation, $\procn{x}$, when we
want to emphasize the connection to the use of the name. Note that, by
virtue of name equivalence, $\quotep{\procn{x}} \nameeq x$; so, the
notation is consistent with previous definitions.

Further, because names have structure it is possible to effect
substitutions on the basis of that structure. This means we need to
upgrade our notation for substitutions, which we accomplish by
adapting comprehension notation. Thus,

\begin{mathpar}
  P\{ y / x : x \in S \}
\end{mathpar}

is interpreted to mean the process derived from P by replacing (in a
capture-avoiding manner) each occurrence of $x$ in $S$ by $y$. For example,

\begin{mathpar}
  P\{ \quotep{\procn{x}|\procn{x}} / x : x \in \freenames{P} \}
\end{mathpar}

will replace each (occurrence) of a free name $x$ in $P$ by
$\quotep{\procn{x}|\procn{x}}$.

Also, we will avail ourselves of the notation $x^{L}$ and $x^{R}$ to
denote injections of a name into disjoint copies of the name
space. There are numerous ways to accomplish this. One example can be
found in \cite{MeredithR05}. This notation overloads to vectors of
names: $\vec{x}^{\pi} := (x_{i}^{\pi} \; : \; 0 \leq i < |\vec{x}| )$ where $\pi \in \{L,R\}$.

We also use $P^{\Box} := P|\Box$.

In \cite{MeredithR05} an interpretation of the new operator is
given. It turns out that there are several possible interpretations
all enjoying the requisite algebraic properties of the operator (see
\cite{milner91polyadicpi}). We will therefore make liberal use of
$(\nu\; \vec{x})P$.

% subsection the_syntax_and_semantics_of_the_notation_system (end)   

\input{qm2pi.qmops} 

\input{qm2pi.sterngerlach} 

\input{qm2pi.metric} 

% section concurrent_process_calculi (end)

%\input{qm2pi.proofsketch}

% section proof sketch (end)

%\input{qm2pi.slviaknots} 

% section spatial logic via knots (end)

\input{qm2pi.conclusion}

% section conclusion (end)

%\input{qm2pi.dtcodes} 

% section wiring algorithm (end)

\input{qm2pi.ack} 

% section acknowledgments (end)

\newpage


\bibliographystyle{plain}   
\bibliography{../../biblios/main.bib}

\input{qm2pi.rhodetails}

\end{document}

 

% section concurrent_process_calculi (end)

%\documentclass[12pt]{llncs}
%\documentclass{jktr}

\usepackage[pdftex]{hyperref}                   
\usepackage {listings}
\usepackage {mathpartir}
\usepackage{bcprules}
%\usepackage{listings}
                       
\usepackage{graphicx} 
%\usepackage[margins=2.5cm,nohead,nofoot]{geometry}
%\usepackage{geometry}
\usepackage{amsfonts}
\usepackage{amstext}
\usepackage{latexsym}
\usepackage{amssymb}
\usepackage{color}


%\include{myPreamble}
\include{qm2pi.local} 

%\ifpdf
%\usepackage[pdftex]{graphicx}
%\else
%\usepackage{graphicx}
%\fi

 % \ifpdf
%  \usepackage{pdfsync}
%  \if


%\title{Brief Article}
%\author{David F. Snyder}
%\author{L.G. Meredith}

%\address{Dept. of Math., Texas State University--San Marcos, San Marcos, TX 78666}
       
\pagestyle{empty}


\begin{document}

\lstset{language=[Objective]Caml,frame=shadowbox}

\input{qm2pi.front}

% section front matter (end)

\input{qm2pi.intro} 
 
% section introduction (end)

% \input{qm2pi.knotations} 

% section notation (end)

\input{qm2pi.process.calculi} 

% section concurrent_process_calculi_and_spatial_logics_ (end)
    
%\input{qm2pi.knots2pi} 

%\input{qm2pi.trefoil} 

%\input{qm2pi.mainthm} 

% subsection basic_interpretation (end)

%\input{qm2pi.rho.presentation} 
\subsection{The syntax and semantics of the notation system}\label{sub:the_syntax_and_semantics_of_the_notation_system} % (fold)

We now summarize a technical presentation of the calculus that
embodies our theory of dynamics. The typical presentation of such a
calculus follows the style of giving generators and relations on
them. The grammar, below, describing term constructors, freely
generates the set of processes, $\Proc$. This set is then quotiented
by a relation known as structural congruence and it is over this set
that the notion of dynamics is expressed. This presentation is
essentially that of \cite{MeredithR05} with the addition of
polyadicity and summation. For readability we have relegated some of
the technical subtleties to an appendix.

\subsubsection{Process grammar}\label{subsub:process_grammar}

\begin{mathpar}
  \inferrule* [lab=synchronization] {} {{M} \bc \pzero \;|\; x?F \;|\; x!C }
  \and
  \inferrule* [lab=abstraction] {} {{F} \bc (x)P}
  \and
  \inferrule* [lab=concretion] {} {{C} \bc \langle Q \rangle}
  \and
  \inferrule* [lab=process] {} {{P,Q} \bc M \;| \;P|Q \;|\; @{x}}
  \and
  \inferrule* [lab=name] {} {{x} \bc \quotep{P}}
\end{mathpar} 

Note that $\vec{x}$ (resp. $\vec{P}$) denotes a vector of names
(resp. processes) of length $|\vec{x}|$ (resp. $|\vec{P}|$). We adopt
the following useful abbreviations.

\begin{mathpar}
   x?(\vec{y}).P := x.(\vec{y})P \and  x\clift{\vec{P}} := x.\clift{\vec{P}}
   \and x!(y) := \lift{x}{\dropn{y}}
   \and \Pi_{i=0}^{n-1}P_i := P_0 | \ldots | P_{n-1}
\end{mathpar}

\subsubsection{Structural congruence}

\paragraph{Free and bound names and alpha-equivalence.} At the
core of structural equivalence is alpha-equivalence which identifies
process that are the same up to a change of variable. Formally, we
recognize the distinction between free and bound names. The free names
of a process, $\freenames{P}$, may be calculated recursively as
follows:

\begin{mathpar}
\freenames{\pzero} := \emptyset
  \and \\
  \freenames{x?(y).P} := \{ x \} \cup (\freenames{P} \setminus \{ y \})
  \and 
  \freenames{x!\langle P \rangle} := \{ x \} \cup \{ P \} 
  \and \\
  \freenames{P|Q} := \freenames{P} \cup \freenames{Q}
  \and \\
  \freenames{@{x}} := \{ x \}
\end{mathpar}

$\pi$
$\quotep{\pi}$

$\freenames{-} : \pi \to \mathcal{P}(\quotep{\pi})$

\begin{eqnarray*}
  \freenames{\pzero} & := & \emptyset \\
  \freenames{x?(y).P} & := & \{ x \} \cup (\freenames{P} \setminus \{ y \}) \\
  \freenames{x!\langle P \rangle} & := & \{ x \} \cup \{ P \} \\
  \freenames{P|Q} & := & \freenames{P} \cup \freenames{Q} \\
  \freenames{\dropn{x}} & := & \{ x \}
\end{eqnarray*}

The bound names of a process, $\boundnames{P}$, are those names occurring in $P$
that are not free. For example, in $x?(y).0$, the name $x$ is free, while $y$ is bound.

\begin{mathpar}
  \inferrule* [lab=monoidal-laws] {} { P|Q \equiv Q|P \and P|0 \equiv P \and P|(Q|R) \equiv (P|Q)|R }
\end{mathpar}

\begin{mathpar}
  \inferrule* [lab=alpha-equivalence] {} { (x)P \equiv (y)P\{y/x\} \and y \not\in \freenames{P} }
\end{mathpar}

\begin{definition}
Then two processes, $P,Q$, are alpha-equivalent if $P = Q\{\vec{y}/\vec{x}\}$ for
some $\vec{x} \in \boundnames{Q},\vec{y} \in \boundnames{P}$, where $Q\{\vec{y}/\vec{x}\}$
denotes the capture-avoiding substitution of $\vec{y}$ for $\vec{x}$ in $Q$.
\end{definition}

\begin{definition}
  The {\em structural congruence} \cite{SangiorgiWalker} , $\equiv$,
  between processes is the least congruence containing
  alpha-equivalence, satisfying the abelian monoid laws
  (associativity, commutativity and $\pzero$ as identity) for parallel
  composition $|$ and for summation $+$.
\end{definition}

\subsection{Name equivalence}

We take name equivalence, written $\nameeq$, to be the smallest
equivalence relation generated by the following rules.

\begin{mathpar}
\inferrule*[lab=Quote-drop]
{ }
{ \quotep{@{x}} \nameeq x }

\inferrule*[lab=Struct-equiv]
{ P \scong Q }
{ \quotep{P} \nameeq \quotep{Q} }
\end{mathpar}

The astute reader will have noticed that the mutual recursion of names
and processes imposes a mutual recursion on alpha-equivalence and
structural equivalence via name-equivalence. Fortunately, all of this
works out pleasantly and we may calculate in the natural way, free of
concern. The reader interested in the details is referred to the
appendix \ref{appendix:rho_details}.

\subsection{Substitution}

We use $\Proc$ for the set of processes, $\QProc$ for the set of
names, and $\id{\{}\vec{y} / \vec{x} \id{\}}$ to denote partial maps,
$s : \QProc \rightarrow \QProc$. A map, $s$ lifts, uniquely, to a map
on process terms, $\widehat{s} : \Proc \rightarrow \Proc$ by the
following equations.

\begin{mathpar}
  (0) \psubstp{Q}{P} := 0 \\
  (R \juxtap S) \psubstp{Q}{P}
  :=    
  (R)\psubstp{Q}{P} \juxtap (S) \psubstp{Q}{P} \\
  (x?(y).R) \psubstp{Q}{P}    
  :=    
  (x)\substp{Q}{P} (z)\concat( (R \psubstn{z}{y}) \psubstp{Q}{P} ) \\
  (\lift{x}{R}) \psubstp{Q}{P}  
  :=
  \lift{(x)\substp{Q}{P}}{ R \psubstp{Q}{P} } \\
%   (\dropn{x})  \psubstp{Q}{P}       
%   := 
%   \left\{ 
%     \begin{array}{ccc} 
%       \dropn{\quotep{Q}} & & x \nameeq \quotep{P} \\
%       \dropn{x} & & otherwise \\
%     \end{array}
%   \right. 
  (\dropn{x})  \psubstp{Q}{P}       
  := 
  \left\{ 
    \begin{array}{ccc} 
      Q & & x \nameeq \quotep{P} \\
      \dropn{x} & & otherwise \\
    \end{array}
  \right.
\end{mathpar}
 

where

\begin{eqnarray}
  (x)\id{\{} \lpquote Q \rpquote / \lpquote P \rpquote \id{\}}            = 
  \left\{ 
    \begin{array}{ccc}
      \lpquote Q \rpquote & & x \nameeq \lpquote P \rpquote \\
      x & & otherwise \\
    \end{array}
  \right. \nonumber
\end{eqnarray}

and $z$ is chosen distinct from $\quotep{P}$, $\quotep{Q}$, the free
names in $Q$, and all the names in $R$. Our $\alpha$-equivalence will
be built in the standard way from this substitution.

\begin{remark}\label{rem:no_self_referential_names}
  One consequence of these definitions is that $\forall P. \quotep{P}
  \not\in \freenames{P}$.
\end{remark}

\subsection{ Dynamic quote: an example }

Anticipating something of what's to come, consider applying the
substitution, $\widehat{\id{\{}u / z \id{\}}}$, to the following pair
of processes, $\lift{w}{y!(z)}$ and $w[ \lpquote y!(z) \rpquote ]$.

\begin{eqnarray}
	\lift{w}{y!(z)}\widehat{\id{\{}u / z \id{\}}}
		& = &
		\lift{w}{y!(u)} \nonumber\\
	w[ \lpquote y!(z) \rpquote ] \widehat{ \id{\{}u / z \id{\}} }
		& = &
		w[ \lpquote y!(z) \rpquote ] \nonumber
\end{eqnarray}

Because the body of the process between quotes is impervious to
substitution, we get radically different answers. In fact, by
examining the first process in an input context,
e.g. $x?(z).\lift{w}{y!(z)}$, we see that the process under the lift
operator may be shaped by prefixed inputs binding a name inside it. In
this sense, the lift operator will be seen as a way to dynamically
construct processes before reifying them as names.

Finally equipped with these standard features we can present the
dynamics of the calculus.

\subsubsection{Operational semantics} 

Finally, we introduce the computational dynamics. What marks these
algebras as distinct from other more traditionally studied algebraic
structures, e.g. vector spaces or polynomial rings, is the manner in
which dynamics is captured. In traditional structures, dynamics is typically
expressed through morphisms between such structures, as in linear maps
between vector spaces or morphisms between rings. In algebras
associated with the semantics of computation, the dynamics is
expressed as part of the algebraic structure itself, through a
reduction reduction relation typically denoted by $\red$. Below, we
give a recursive presentation of this relation for the calculus used
in the encoding.

$\red \subseteq \pi \times \pi$
$\red : \pi \to \mathcal{P}(\pi)$

\begin{mathpar}
  \inferrule* [lab=Comm] { \textsf{match}( x_{src}, x_{trgt} ) } { x_{trgt}?(y)P \; | \; x_{src}!\langle {Q} \rangle \red P\{\quotep{Q}/y}\} }
  \and \\
  \inferrule* [lab=Par] {{P} \red {P}'} {{{P} | {Q}} \red {{P}' | {Q}}}
  \and
  \inferrule* [lab=Equiv]{{{P} \scong {P}'} \andalso {{P}' \red {Q}'} \andalso {{Q}' \scong {Q}}}{{P} \red {Q}}
\end{mathpar}

\begin{eqnarray*}
  match_{\equiv} (\quotep{P},\quotep{Q}) & := & P \equiv Q \\
  match_{\dagger}(\quotep{P},\quotep{Q}) & := & \forall R. P|Q \red^{*} R => R \red^{*} 0 \\
  match_{K}(\quotep{P},\quotep{Q}) & := & K \mbox{ for some context } K
\end{eqnarray*}

$u?(x)P | u!\langle Q \rangle \red P\{\quotep{Q}/x\}$

%We write $\wred$ for $\red^*$, and $P\red$ if $\exists Q $ such that $ P \red Q$.
We write $P\red$ if $\exists Q $ such that $ P \red Q$ and $P\not\red$, otherwise.

\section{Replication}

As mentioned before, it is known that replication (and hence
recursion) can be implemented in a higher-order process algebra
\cite{SangiorgiWalker}. As our first example of calculation with the
machinery thus far presented we give the construction explicitly in
the {\rhoc}.

\begin{eqnarray}
	D_{x} & := & \prefix{x}{y}{(\binpar{\outputp{x}{y}}{@{y}})} \nonumber\\
	\bangp_{x}{P} & := & \binpar{{x}!\langle{\binpar{D_{x}}{P}}\rangle}{D_{x}} \nonumber
\end{eqnarray}

\begin{eqnarray}
	\bangp_{x}{P} & & \nonumber\\
	=
	& {x}!\langle{(\prefix{x}{y}{(\outputp{x}{y} | @{y})) | P}}\rangle 
	      | \prefix{x}{y}{(\outputp{x}{y} | @{y})} & \nonumber\\
	\red
	& (\outputp{x}{y} | @{y})\substn{\quotep{(\prefix{x}{y}{(@{y} | \outputp{x}{y})) | P}}}{y} & \nonumber\\
	=
	& \outputp{x}{\quotep{(\prefix{x}{y}{(\outputp{x}{y} | @{y})) | P}}}
	  | {(\prefix{x}{y}{(\outputp{x}{y} | @{y})) | P}} & \nonumber\\
	\red
	& \ldots & \nonumber\\
	\red^*
	& P | P | \ldots & \nonumber
\end{eqnarray}

Of course, this encoding, as an implementation, runs away, unfolding
$\bangp{P}$ eagerly. A lazier and more implementable replication
operator, restricted to input-guarded processes, may be obtained as follows.

\begin{eqnarray}
\bangp{\prefix{u}{v}{P}} 
	:= 
	\binpar{\lift{x}{\prefix{u}{v}{(\binpar{D(x)}{P})}}}{D(x)} \nonumber
\end{eqnarray}

\begin{remark}
  Note that the lazier definition still does not deal with summation
  or mixed summation (i.e. sums over input and output). The reader is
  invited to construct definitions of replication that deal with these
  features. 

  Further, the definitions are parameterized in a name, $x$. Can you,
  gentle reader, make a definition that eliminates this parameter and
  guarantees no accidental interaction between the replication
  machinery and the process being replicated -- i.e. no accidental
  sharing of names used by the process to get its work done and the
  name(s) used by the replication to effect copying. This latter
  revision of the definition of replication is crucial to obtaining
  the expected identity $!!P \sim !P$.
\end{remark}

\begin{remark}\label{rem:paradoxical_combinator}
  The reader familiar with the lambda calculus will have noticed the
  similarity between $D$ and the paradoxical combinator.

  [Ed. note: the existence of this seems to suggest we have to be more
  restrictive on the set of processes and names we admit if we are to
  support no-cloning.]
\end{remark}

\subsubsection{Bisimulation}

The computational dynamics gives rise to another kind of equivalence,
the equivalence of computational behavior. As previously mentioned
this is typically captured \emph{via} some form of bisimulation.

% The notion we use in this paper is weak barbed bisimulation
% \cite{milner91polyadicpi}.

The notion we use in this paper is derived from weak barbed
bisimulation \cite{milner91polyadicpi}. 

\begin{definition}
An \emph{observation relation}, $\downarrow_{\mathcal N}$, over a set
of names, $\mathcal N$, is the smallest relation satisfying the rules
below.

\infrule[Out-barb]{y \in {\mathcal N}, \; x \nameeq y}
		  {\outputp{x}{v} \downarrow_{\mathcal N} x}
\infrule[Par-barb]{\mbox{$P\downarrow_{\mathcal N} x$ or $Q\downarrow_{\mathcal N} x$}}
		  {\binpar{P}{Q} \downarrow_{\mathcal N} x}

We write $P \Downarrow_{\mathcal N} x$ if there is $Q$ such that 
$P \wred Q$ and $Q \downarrow_{\mathcal N} x$.
\end{definition}

\begin{definition}
%\label{def.bbisim}
An  ${\mathcal N}$-\emph{barbed bisimulation} over a set of names, ${\mathcal N}$, is a symmetric binary relation 
${\mathcal S}_{\mathcal N}$ between agents such that $P\rel{S}_{\mathcal N}Q$ implies:
\begin{enumerate}
\item If $P \red P'$ then $Q \wred Q'$ and $P'\rel{S}_{\mathcal N} Q'$.
\item If $P\downarrow_{\mathcal N} x$, then $Q\Downarrow_{\mathcal N} x$.
\end{enumerate}
$P$ is ${\mathcal N}$-barbed bisimilar to $Q$, written
$P \wbbisim_{\mathcal N} Q$, if $P \rel{S}_{\mathcal N} Q$ for some ${\mathcal N}$-barbed bisimulation ${\mathcal S}_{\mathcal N}$.
\end{definition}

$\mathcal{R} \subseteq \pi \times \pi$

$P \mathcal{R} Q => \forall P'. P \red P' \Rightarrow \exists Q'. Q \red Q', P' \mathcal{R} Q'$

$P \vdash x \Rightarrow Q \vdash x$

\begin{mathpar}
  \inferrule*[lab=Out-barb]{x \nameeq y}{{y}!\langle{Q}\rangle \vdash x}
  \and
  \inferrule*[lab=Par-barb]{\mbox{$P\vdash x$ or $Q\vdash x$}}{\binpar{P}{Q} \vdash x}
\end{mathpar}

\subsubsection{Contexts}

One of the principle advantages of computational calculi like the
$\pi$-calculus is a well-defined notion of context,
contextual-equivalence and a correlation between
contextual-equivalence and notions of bisimulation. The notion of
context allows the decomposition of a process into (sub-)process and
its syntactic environment, its context. Thus, a context may be
thought of as a process with a ``hole'' (written $\Box$) in it. The
application of a context $M$ to a process $P$, written $M[P]$, is
tantamount to filling the hole in $M$ with $P$. In this paper we do
not need the full weight of this theory, but do make use of the notion
of context in the proof the main theorem. 

\begin{mathpar}
  \inferrule* [lab=summation] {} {{M_{M},M_{N}} \bc \Box \;|\; x.M_{A} \;|\; M_{M}+M_{N}}
  \and
  \inferrule* [lab=agent] {} {{M_{A}} \bc (\vec{x})M_{P} \;| \; \clift{P_0,\ldots,M_{P},\ldots,P_N}}
  \and \\
  \inferrule* [lab=process] {} {{M_{P}} \bc M_{N} \;| \;P|M_{P} }
\end{mathpar} 

\begin{mathpar}
  \inferrule* [lab=sychronization] {} {M_{N} \bc \Box \;|\; x?M_{F} \;|\; x!M_{C}}
  \and
  \inferrule* [lab=abstraction] {} {{M_{F}} \bc (x)M_{P} }
  \and
  \inferrule* [lab=concretion] {} {{M_{C}} \bc \langle M_{P} \rangle }
  \and \\
  \inferrule* [lab=process] {} {{M_{P}} \bc M_{N} \;| \;P|M_{P} }
\end{mathpar}

\begin{definition}[contextual application] Given a context $M$, and
  process $P$, we define the \emph{contextual application}, $M[P] :=
  M\{P/\Box\}$. That is, the contextual application of M to P is the
  substitution of $P$ for $\Box$ in $M$.
\end{definition}

$\meaningof{-} : L \to \mathcal{P}(\pi)$

\begin{mathpar}
  \inferrule* [lab=collection] {} {\meaningof{true} = \pi, \and \meaningof{~E} = \pi \setminus \meaningof{E}, \and \meaningof{E_{1} \& E_{2}} = \meaningof{E_{1}} \cap \meaningof{E_{2}}}
\end{mathpar}

\begin{mathpar}
  \inferrule* [lab=structure] {} {\meaningof{0} = \{ P \in \pi | P \equiv 0 \}, \and \\ \meaningof{E_1 | E_2} = \{ P \in \pi | P \equiv P_{1} | P_{2}, P_{1} \in \meaningof{E_{1}}, P_{2} \in \meaningof{E_2}\} }
\end{mathpar}

\begin{mathpar}
 \inferrule* [lab=behavior] {} {\meaningof{\langle a?b \rangle E} = \{ P \in \pi | P \equiv Q | u?(y)P', \\ \and \\\\ \and \\ \;\;\; u \in \meaningof{a}, \forall z.P'\{z/y\} \in \meaningof{E\{z/b\}}\}, \and \\ \meaningof{a!E} = \{ P \in \pi | P \equiv Q | x!\langle P' \rangle, x \in \meaningof{a} P' \in \meaningof{E}\} }
\end{mathpar}

\begin{mathpar}
 \inferrule* [lab=nominal] {} {\meaningof{\quotep{E}} = \{ \quotep{P} \in \quotep{\pi} | P \in \meaningof{E} \}, \and \meaningof{\quotep{P}} = \{ \quotep{Q} \in \quotep{\pi} | P \equiv Q \} \and \\ \meaningof{@\quotep{E}} = \{ P \in \pi | P \equiv @x, x \in \meaningof{E} \}}
\end{mathpar}

\begin{eqnarray*}
  \\
  \meaningof{-} : TS \to ST
\end{eqnarray*}

\begin{eqnarray*}
  \\
  L : TS \to ST
\end{eqnarray*}

\begin{eqnarray*}
  \\
  P \models E \iff P \in \meaningof{E}
\end{eqnarray*}

\begin{eqnarray*}
  P \approx_{L} Q \iff \forall E \in L. P \models E \iff Q \models E
\end{eqnarray*}

\begin{eqnarray*}
  P \approx_{K} Q
\end{eqnarray*}

\begin{eqnarray*}
  P \approx Q
\end{eqnarray*}

$\approx_{K} = \approx = \approx_{L}$

\subsubsection{Contextual duality}

Note that contexts extend the quotation operation to a family of
operations from processes to names. Given a context, $M$, we can
define a \emph{nominal context}, $\quotep{M}$ by $\quotep{M}[P] :=
\quotep{M[P]}$. To foreshadow what is to come we observe that these
operations enjoy a duality with processes very much like the duality
between vectors and maps from vectors to scalars.

Further, because the calculus is essentially higher-order, we have a
correspondence between contexts and processes. More specifically,
given a name $x$ and a context $M$ we can construct $M^{*}_{x}$ such
that 

\begin{mathpar}
  M^{*}_{x} | \lift{x}{P} \red M[P]
\end{mathpar}

namely,

\begin{mathpar}
  M^{*}_{x} := x?(u).M[\dropn{u}]
\end{mathpar}

The dependence of $M^{*}_{x}$ on a name makes it an abstraction, 

\begin{mathpar}
  M^{*} := (x)x?(u).M[\dropn{u}]
\end{mathpar}

\subsection{Additional notation}

It will sometimes be convenient to denote the process a name
quotes. We already have the notation $x = \quotep{P}$, but it will be
convenient to introduce an alternate notation, $\procn{x}$, when we
want to emphasize the connection to the use of the name. Note that, by
virtue of name equivalence, $\quotep{\procn{x}} \nameeq x$; so, the
notation is consistent with previous definitions.

Further, because names have structure it is possible to effect
substitutions on the basis of that structure. This means we need to
upgrade our notation for substitutions, which we accomplish by
adapting comprehension notation. Thus,

\begin{mathpar}
  P\{ y / x : x \in S \}
\end{mathpar}

is interpreted to mean the process derived from P by replacing (in a
capture-avoiding manner) each occurrence of $x$ in $S$ by $y$. For example,

\begin{mathpar}
  P\{ \quotep{\procn{x}|\procn{x}} / x : x \in \freenames{P} \}
\end{mathpar}

will replace each (occurrence) of a free name $x$ in $P$ by
$\quotep{\procn{x}|\procn{x}}$.

Also, we will avail ourselves of the notation $x^{L}$ and $x^{R}$ to
denote injections of a name into disjoint copies of the name
space. There are numerous ways to accomplish this. One example can be
found in \cite{MeredithR05}. This notation overloads to vectors of
names: $\vec{x}^{\pi} := (x_{i}^{\pi} \; : \; 0 \leq i < |\vec{x}| )$ where $\pi \in \{L,R\}$.

We also use $P^{\Box} := P|\Box$.

In \cite{MeredithR05} an interpretation of the new operator is
given. It turns out that there are several possible interpretations
all enjoying the requisite algebraic properties of the operator (see
\cite{milner91polyadicpi}). We will therefore make liberal use of
$(\nu\; \vec{x})P$.

% subsection the_syntax_and_semantics_of_the_notation_system (end)   

\input{qm2pi.qmops} 

\input{qm2pi.sterngerlach} 

\input{qm2pi.metric} 

% section concurrent_process_calculi (end)

%\input{qm2pi.proofsketch}

% section proof sketch (end)

%\input{qm2pi.slviaknots} 

% section spatial logic via knots (end)

\input{qm2pi.conclusion}

% section conclusion (end)

%\input{qm2pi.dtcodes} 

% section wiring algorithm (end)

\input{qm2pi.ack} 

% section acknowledgments (end)

\newpage


\bibliographystyle{plain}   
\bibliography{../../biblios/main.bib}

\input{qm2pi.rhodetails}

\end{document}



% section proof sketch (end)

%\section{Unlikely characters: spatial logic for
  knots}\label{sub:characteristic_formulae} % (fold)

Associated to the mobile process calculi are a family of logics known
as the Hennessy-Milner logics. These logics typically enjoy a
semantics interpreting formulae as sets of processes that when
factored through the encoding outlined above allows an identification
of classes of knots with logical formulae. In the context of this
encoding the sub-family known as the spatial logics \cite{CairesC03}
\cite{CairesC04} \cite{Caires04} are of particular interest providing
several important features for expressing and reasoning about
properties (i.e. classes) of knots. We hint here at how this may be done.

%\begin{description}
%\item [structural connectives] 
\subsubsection{Structural connectives} The spatial logics enjoy
structural connectives corresponding, at the logical level, to the
parallel composition ($P | Q$) and new name ($(\nu \; x)P$)
connectives for processes. As illustrated in the examples below, these
connectives are extremely expressive given the shape of our encoding.
%\item [decideable satisfaction]

\subsubsection{Decideable satisfaction}
In \cite{Caires04} the satisfaction relation is shown to be decideable
for a rich class of processes. It further turns out that the image of
the our encoding is a proper subset of that class. This result
provides the basis for an algorithm by which to search for knots
enjoying a given property.
%\item [characteristic formulae]

\subsubsection{Characteristic formulae}
In the same paper \cite{Caires04} , Caires presents a means of calculating
characteristic formulae, selecting equivalence classes of processes
up to a pre--specified depth limit on the support set of names. Composed with our
encoding, this characteristic formula can be used to select
characteristic formulae for knots.
%\end{description}

\subsubsection{Spatial logic formulae}

The grammar below (segmented for comprehension) summarizes the syntax
of spatial logic formulae. We employ illustrative examples in the
sequel to provide an intuitive understanding of their meaning
referring the reader to \cite{Caires04} for a more detailed explication
of the semantics.

\begin{mathpar}
  \inferrule* [lab=boolean] {} {{A,B} \bc T \;|\; \neg A \;|\; A \wedge B \;|\; \eta = \eta'}
  \and
  \inferrule* [lab=spatial] {} {|\; \pzero \;|\; A | B \;|\; x \text{\textregistered} A \;|\; \forall x . A \;|\;  H x . A}
  \and
  \inferrule* [lab=behavioral] {} {|\; \alpha . A}
  \and 
  \inferrule* [lab=recursion] {} {|\; X(\vec{u}) \;|\; \mu X(\vec{u}) . A}
  \and
  \inferrule* [lab=action] {} {\alpha \bc \langle x?(\vec{y}) \rangle \;|\; \langle x!(\vec{y}) \rangle \;|\; \langle \tau \rangle}
  \and 
  \inferrule* [lab=name] {} {\eta \bc x \;|\; \tau}
\end{mathpar} 

% subsection characteristic_formulae (end)   	 

\subsection{Example formulae}\label{sub:example_formulae_} % (fold)

\subsubsection{Crossing as formula.}
% 
% \begin{align*}
%   \frac{d}{dx} \sin x &= \cos x 
%   & \frac{d}{dx} e^x &= e^x \\
%   \frac{d}{dx} \cos x &= - \sin x 
%   & \frac{d}{dx} \log x &= \frac{1}{x} \\
% \end{align*} 

\begin{align*}
 \mu C(x_{0},x_{1},y_{0},y_{1},u).&(\langle x_{0}?(z) \rangle(\langle u! \rangle\langle y_{1}!z \rangle C(x_{0},x_{1},y_{0},y_{1},u)) & \\
  & \wedge \langle y_{1}?(z) \rangle (\langle u! \rangle \langle x_{0}!z \rangle C(x_{0},x_{1},y_{0},y_{1},u)) & \\
  & \wedge \langle x_{1}?(z) \rangle (\langle u? \rangle \langle y_{0}!z \rangle C(x_{0},x_{1},y_{0},y_{1},u)) & \\
  & \wedge \langle y_{0}?(z) \rangle (\langle u? \rangle \langle x_{1}!z \rangle C(x_{0},x_{1},y_{0},y_{1},u))) &
\end{align*}

The lexicographical similarity between the shape of this formulae and
the shape of definition of the process representing a crossing reveals
the intuitive meaning of this formulae. It describes the capabilities
of a process that has the right to represent a crossing. For example
it picks out processes that may perform an input on the port $x_0$ in
its initial menu of capabilities. What differentiates the formula
from the process, however, is that the crossing process is the
smallest candidate to satisfy the formula. Infinitely many other
processes -- with internal behavior hidden behind this interface, so
to speak -- also satisfy this formula. Even this simple formula,
then, can be seen to open a new view onto knots, providing a
computational interpretation of \emph{virtual} knots.

Note that this formula is derived by hand. A similar formula can be
derived by employing Caires' calculation of characteristic formula
\cite{Caires04} to the process representing a crossing. In light of
this discussion, we let
$\meaningof{C}_{\phi}(x0,x1,y0,y1,u)$ denote a formula specifying the
dynamics we wish to capture of a crossing. To guarantee we preserve
the shape of the interface and minimal semantics we demand that
$\meaningof{C}_{\phi}(x0,x1,y0,y1,u) \Rightarrow
\textbf{C}(x0,x1,y0,y1,u)$ where $\textbf{C}(x0,x1,y0,y1,u)$ denotes
the formula above.
                            
\subsubsection{Crossing number constraints.}
The moral content of the context lemma (Lemma \ref{context}) is that the notion of
``locality'' in the Reidemeister moves is effectively captured by the
parallel composition operator of the process calculus. This intuition
extends through the logic. Given a formula,
$\meaningof{C}_{\phi}(x0,x1,y0,y1,u)$, we can use the structural
connectives to specify constraints on crossing numbers, such as at
least $n$ crossings, or exactly $n$ crossings.
\begin{mathpar}
  \inferrule* [lab=at-least-n] {} { K^{\geq n}_{\phi}(\vec{xs},\vec{ys}) := \Pi_{i=0}^{n-1} Hu . \meaningof{C}_{\phi}(xs_i,ys_i,u) | T }
  \and 
  \inferrule* [lab=exactly-n] {} { K^{= n}_{\phi}(\vec{xs},\vec{ys}) := \Pi_{i=0}^{n-1} Hu . \meaningof{C}_{\phi}(xs_i,ys_i,u) | \neg (\forall x_0,y_0,x_1,y_1,u . \meaningof{C}_{\phi}(x_0,y_0,x_1,y_1,u) | T) }
\end{mathpar}

To round out this section, recall that the encoding of an $n$-crossing
knot decomposes into a parallel composition of $n$ \emph{copies} of a
crossing process together with a wiring harness. To specify different
knot classes with the same crossing number amounts to specifying
logical constraints on the wiring harness. In the interest of space,
we defer examples to a forthcoming paper. Suffice it to say that both
the conditions ``alternating knot'' and ``contains the tangle
corresponding to 5/3'' are expressible. For example, it is possible to
calculate the characteristic formula of a process corresponding to the
tangle 5/3 and conjoin it into the classifying formula via the
composition connective of the logic.

Finally, we wish to observe that it is entirely within reason to
contemplate a more domain-specific version of spatial logic tailored
to the shape of processes in the image of the encoding. Such a
domain-specific logic would have a better claim to the title formal
language of knot properties.

% subsection example_formulae_ (end)

% section knots_as_processes (end) 

% section spatial logic via knots (end)

\section{Conclusions and future work}

\paragraph{Testing physical space}
You, gentle reader, may wonder why of all the theorems to be proved
given this set up we pick the one above. In some sense it's hardly
central to quantum mechanics. We see it as central in the sense that
it firmly establishes a notion of physical space arising from a notion
of the equivalence of behavior. Relating bisimulation to a metric is a
big step forward, but one is faced with interpreting the relationship
of that metric space to something more physical. Quantum mechanical
notions of ``physical'' space are still far from intuitive, but by
relating this idea of distance as testing to calculations that predict
physical circumstances we are making a not insignificant step forward
toward an understanding of the physical space we inhabit as
essentially dynamic.

\paragraph{Effectivity and simulation}
One of the observations we have yet to make is that the entire program
spelled out here is effective. We have built various interpreters for
the reflective calculus at work in this interpretation. In principle,
then, we can simulate quantum mechanics on a computer. The place where
the simulation may lose fidelity is the infinitely branching summation
for the annihilator.

In this connection i also want to point out that the evaluation style
calculation of the inner product puts the non-determinism of the
summation right at the heart of measurement. This suggests that
Milner's original reduction-based formulation of the dynamics of his
calculi in terms of sums was not just notationally suggestive of a
notion of measure-and-continue but captured some significant part of
the physics.

\paragraph{Quantum continuations}
In light of this last observation i want to point out that the
predominant account of quantum mechanics is missing a key aspect of a
truly compositional story of the physical situation. In a real lab,
when a measurement is made the observation can be made to feed into
another device that then makes another measurement conditioned on the
results of the first. This means that after the superposition was
collapsed the entire experimental set up remained in
superposition. While QM offers a means of writing this down it doesn't
quite line up well with the well-trodden formulation of computation
and continuation that we see so succinctly expressed in Milner's
calculi. This suggests that there might be advantages to this account
of dynamics waiting to be explored.

\paragraph{Quantum logic}
In this connection, we also note that by virtue of having the
Hennessy-Milner construction, we can pull the construction through the
interpretation of QM. This gives us a natural candidate for a quantum
logic that enjoys an extremely tight connection with it's domain of
interpretation, making the construction much less ad hoc (rather it is
the image of functor!).

\paragraph{Quantum probabiity}
i have questions about the basis of the interpretation of inner
product as probability amplitude. In particular, using which
axiomatization of probability theory does the notion of probability
amplitude earn the right to be so dubbed? In other words, where is the
proof that the operation for calculating a probability amplitude (and
then squaring) satisfies the axioms of what it means to calculate a
probability? Even if such a proof exists (i have yet to find it in the
literature), i wonder if it might not be possible to turn things on
their heads. Can we view the calculation of the probability amplitude
as an axiomatization of probability? If so, then the definition we
give for calculating probability amplitude may provide the basis for
an \emph{effective} theory of probability.

\paragraph{Quantum vs ``biological'' information}
Finally, i want to conclude with a more philosophical observation. At
a recent workshop in which QM was a predominant topic i noticed
something about quantum information. The speaker was giving a riveting
discussion of axiomatic QM and showing how properties of ``no
cloning'' and ``no deleting'' emerged as consequences of the
axiomatization. Theorems of this form are necessary to give us a sense
of confidence that our axioms characterize the physical theory. What
struck me, though, was that if quantum information is neither erasable
nor replicable it is markedly different from \emph{life}. Two of the
things we know about life is that

\begin{itemize}
  \item it ends;
  \item to gain some measure of persistence, to transcend it's
    finitude it is imminently copyable.
\end{itemize}

Both of these qualities are summarized succinctly in the aphorism: all
flesh is grass. For me these two kinds of ``information'' -- call them
quantum and biological -- are end points on a spectrum of strategies
for persistence. At one end, we have those curious entities that enjoy
uniqueness and permanence; at the other, we have those who in the face
of a certain end and an uncertain present make a go of passing
something on. To me one of the more remarkable aspects of the latter
strategy is that in the presence of noise (and certain features of
copying) we get a kind of dynamism, a chance for improvement against a
given persistent condition.

% subsection other_calculi_other_bisimulations_and_geometry_as_behavior (end)




% section conclusion (end)

%\documentclass[12pt]{llncs}
%\documentclass{jktr}

\usepackage[pdftex]{hyperref}                   
\usepackage {listings}
\usepackage {mathpartir}
\usepackage{bcprules}
%\usepackage{listings}
                       
\usepackage{graphicx} 
%\usepackage[margins=2.5cm,nohead,nofoot]{geometry}
%\usepackage{geometry}
\usepackage{amsfonts}
\usepackage{amstext}
\usepackage{latexsym}
\usepackage{amssymb}
\usepackage{color}


%\include{myPreamble}
\include{qm2pi.local} 

%\ifpdf
%\usepackage[pdftex]{graphicx}
%\else
%\usepackage{graphicx}
%\fi

 % \ifpdf
%  \usepackage{pdfsync}
%  \if


%\title{Brief Article}
%\author{David F. Snyder}
%\author{L.G. Meredith}

%\address{Dept. of Math., Texas State University--San Marcos, San Marcos, TX 78666}
       
\pagestyle{empty}


\begin{document}

\lstset{language=[Objective]Caml,frame=shadowbox}

\input{qm2pi.front}

% section front matter (end)

\input{qm2pi.intro} 
 
% section introduction (end)

% \input{qm2pi.knotations} 

% section notation (end)

\input{qm2pi.process.calculi} 

% section concurrent_process_calculi_and_spatial_logics_ (end)
    
%\input{qm2pi.knots2pi} 

%\input{qm2pi.trefoil} 

%\input{qm2pi.mainthm} 

% subsection basic_interpretation (end)

%\input{qm2pi.rho.presentation} 
\subsection{The syntax and semantics of the notation system}\label{sub:the_syntax_and_semantics_of_the_notation_system} % (fold)

We now summarize a technical presentation of the calculus that
embodies our theory of dynamics. The typical presentation of such a
calculus follows the style of giving generators and relations on
them. The grammar, below, describing term constructors, freely
generates the set of processes, $\Proc$. This set is then quotiented
by a relation known as structural congruence and it is over this set
that the notion of dynamics is expressed. This presentation is
essentially that of \cite{MeredithR05} with the addition of
polyadicity and summation. For readability we have relegated some of
the technical subtleties to an appendix.

\subsubsection{Process grammar}\label{subsub:process_grammar}

\begin{mathpar}
  \inferrule* [lab=synchronization] {} {{M} \bc \pzero \;|\; x?F \;|\; x!C }
  \and
  \inferrule* [lab=abstraction] {} {{F} \bc (x)P}
  \and
  \inferrule* [lab=concretion] {} {{C} \bc \langle Q \rangle}
  \and
  \inferrule* [lab=process] {} {{P,Q} \bc M \;| \;P|Q \;|\; @{x}}
  \and
  \inferrule* [lab=name] {} {{x} \bc \quotep{P}}
\end{mathpar} 

Note that $\vec{x}$ (resp. $\vec{P}$) denotes a vector of names
(resp. processes) of length $|\vec{x}|$ (resp. $|\vec{P}|$). We adopt
the following useful abbreviations.

\begin{mathpar}
   x?(\vec{y}).P := x.(\vec{y})P \and  x\clift{\vec{P}} := x.\clift{\vec{P}}
   \and x!(y) := \lift{x}{\dropn{y}}
   \and \Pi_{i=0}^{n-1}P_i := P_0 | \ldots | P_{n-1}
\end{mathpar}

\subsubsection{Structural congruence}

\paragraph{Free and bound names and alpha-equivalence.} At the
core of structural equivalence is alpha-equivalence which identifies
process that are the same up to a change of variable. Formally, we
recognize the distinction between free and bound names. The free names
of a process, $\freenames{P}$, may be calculated recursively as
follows:

\begin{mathpar}
\freenames{\pzero} := \emptyset
  \and \\
  \freenames{x?(y).P} := \{ x \} \cup (\freenames{P} \setminus \{ y \})
  \and 
  \freenames{x!\langle P \rangle} := \{ x \} \cup \{ P \} 
  \and \\
  \freenames{P|Q} := \freenames{P} \cup \freenames{Q}
  \and \\
  \freenames{@{x}} := \{ x \}
\end{mathpar}

$\pi$
$\quotep{\pi}$

$\freenames{-} : \pi \to \mathcal{P}(\quotep{\pi})$

\begin{eqnarray*}
  \freenames{\pzero} & := & \emptyset \\
  \freenames{x?(y).P} & := & \{ x \} \cup (\freenames{P} \setminus \{ y \}) \\
  \freenames{x!\langle P \rangle} & := & \{ x \} \cup \{ P \} \\
  \freenames{P|Q} & := & \freenames{P} \cup \freenames{Q} \\
  \freenames{\dropn{x}} & := & \{ x \}
\end{eqnarray*}

The bound names of a process, $\boundnames{P}$, are those names occurring in $P$
that are not free. For example, in $x?(y).0$, the name $x$ is free, while $y$ is bound.

\begin{mathpar}
  \inferrule* [lab=monoidal-laws] {} { P|Q \equiv Q|P \and P|0 \equiv P \and P|(Q|R) \equiv (P|Q)|R }
\end{mathpar}

\begin{mathpar}
  \inferrule* [lab=alpha-equivalence] {} { (x)P \equiv (y)P\{y/x\} \and y \not\in \freenames{P} }
\end{mathpar}

\begin{definition}
Then two processes, $P,Q$, are alpha-equivalent if $P = Q\{\vec{y}/\vec{x}\}$ for
some $\vec{x} \in \boundnames{Q},\vec{y} \in \boundnames{P}$, where $Q\{\vec{y}/\vec{x}\}$
denotes the capture-avoiding substitution of $\vec{y}$ for $\vec{x}$ in $Q$.
\end{definition}

\begin{definition}
  The {\em structural congruence} \cite{SangiorgiWalker} , $\equiv$,
  between processes is the least congruence containing
  alpha-equivalence, satisfying the abelian monoid laws
  (associativity, commutativity and $\pzero$ as identity) for parallel
  composition $|$ and for summation $+$.
\end{definition}

\subsection{Name equivalence}

We take name equivalence, written $\nameeq$, to be the smallest
equivalence relation generated by the following rules.

\begin{mathpar}
\inferrule*[lab=Quote-drop]
{ }
{ \quotep{@{x}} \nameeq x }

\inferrule*[lab=Struct-equiv]
{ P \scong Q }
{ \quotep{P} \nameeq \quotep{Q} }
\end{mathpar}

The astute reader will have noticed that the mutual recursion of names
and processes imposes a mutual recursion on alpha-equivalence and
structural equivalence via name-equivalence. Fortunately, all of this
works out pleasantly and we may calculate in the natural way, free of
concern. The reader interested in the details is referred to the
appendix \ref{appendix:rho_details}.

\subsection{Substitution}

We use $\Proc$ for the set of processes, $\QProc$ for the set of
names, and $\id{\{}\vec{y} / \vec{x} \id{\}}$ to denote partial maps,
$s : \QProc \rightarrow \QProc$. A map, $s$ lifts, uniquely, to a map
on process terms, $\widehat{s} : \Proc \rightarrow \Proc$ by the
following equations.

\begin{mathpar}
  (0) \psubstp{Q}{P} := 0 \\
  (R \juxtap S) \psubstp{Q}{P}
  :=    
  (R)\psubstp{Q}{P} \juxtap (S) \psubstp{Q}{P} \\
  (x?(y).R) \psubstp{Q}{P}    
  :=    
  (x)\substp{Q}{P} (z)\concat( (R \psubstn{z}{y}) \psubstp{Q}{P} ) \\
  (\lift{x}{R}) \psubstp{Q}{P}  
  :=
  \lift{(x)\substp{Q}{P}}{ R \psubstp{Q}{P} } \\
%   (\dropn{x})  \psubstp{Q}{P}       
%   := 
%   \left\{ 
%     \begin{array}{ccc} 
%       \dropn{\quotep{Q}} & & x \nameeq \quotep{P} \\
%       \dropn{x} & & otherwise \\
%     \end{array}
%   \right. 
  (\dropn{x})  \psubstp{Q}{P}       
  := 
  \left\{ 
    \begin{array}{ccc} 
      Q & & x \nameeq \quotep{P} \\
      \dropn{x} & & otherwise \\
    \end{array}
  \right.
\end{mathpar}
 

where

\begin{eqnarray}
  (x)\id{\{} \lpquote Q \rpquote / \lpquote P \rpquote \id{\}}            = 
  \left\{ 
    \begin{array}{ccc}
      \lpquote Q \rpquote & & x \nameeq \lpquote P \rpquote \\
      x & & otherwise \\
    \end{array}
  \right. \nonumber
\end{eqnarray}

and $z$ is chosen distinct from $\quotep{P}$, $\quotep{Q}$, the free
names in $Q$, and all the names in $R$. Our $\alpha$-equivalence will
be built in the standard way from this substitution.

\begin{remark}\label{rem:no_self_referential_names}
  One consequence of these definitions is that $\forall P. \quotep{P}
  \not\in \freenames{P}$.
\end{remark}

\subsection{ Dynamic quote: an example }

Anticipating something of what's to come, consider applying the
substitution, $\widehat{\id{\{}u / z \id{\}}}$, to the following pair
of processes, $\lift{w}{y!(z)}$ and $w[ \lpquote y!(z) \rpquote ]$.

\begin{eqnarray}
	\lift{w}{y!(z)}\widehat{\id{\{}u / z \id{\}}}
		& = &
		\lift{w}{y!(u)} \nonumber\\
	w[ \lpquote y!(z) \rpquote ] \widehat{ \id{\{}u / z \id{\}} }
		& = &
		w[ \lpquote y!(z) \rpquote ] \nonumber
\end{eqnarray}

Because the body of the process between quotes is impervious to
substitution, we get radically different answers. In fact, by
examining the first process in an input context,
e.g. $x?(z).\lift{w}{y!(z)}$, we see that the process under the lift
operator may be shaped by prefixed inputs binding a name inside it. In
this sense, the lift operator will be seen as a way to dynamically
construct processes before reifying them as names.

Finally equipped with these standard features we can present the
dynamics of the calculus.

\subsubsection{Operational semantics} 

Finally, we introduce the computational dynamics. What marks these
algebras as distinct from other more traditionally studied algebraic
structures, e.g. vector spaces or polynomial rings, is the manner in
which dynamics is captured. In traditional structures, dynamics is typically
expressed through morphisms between such structures, as in linear maps
between vector spaces or morphisms between rings. In algebras
associated with the semantics of computation, the dynamics is
expressed as part of the algebraic structure itself, through a
reduction reduction relation typically denoted by $\red$. Below, we
give a recursive presentation of this relation for the calculus used
in the encoding.

$\red \subseteq \pi \times \pi$
$\red : \pi \to \mathcal{P}(\pi)$

\begin{mathpar}
  \inferrule* [lab=Comm] { \textsf{match}( x_{src}, x_{trgt} ) } { x_{trgt}?(y)P \; | \; x_{src}!\langle {Q} \rangle \red P\{\quotep{Q}/y}\} }
  \and \\
  \inferrule* [lab=Par] {{P} \red {P}'} {{{P} | {Q}} \red {{P}' | {Q}}}
  \and
  \inferrule* [lab=Equiv]{{{P} \scong {P}'} \andalso {{P}' \red {Q}'} \andalso {{Q}' \scong {Q}}}{{P} \red {Q}}
\end{mathpar}

\begin{eqnarray*}
  match_{\equiv} (\quotep{P},\quotep{Q}) & := & P \equiv Q \\
  match_{\dagger}(\quotep{P},\quotep{Q}) & := & \forall R. P|Q \red^{*} R => R \red^{*} 0 \\
  match_{K}(\quotep{P},\quotep{Q}) & := & K \mbox{ for some context } K
\end{eqnarray*}

$u?(x)P | u!\langle Q \rangle \red P\{\quotep{Q}/x\}$

%We write $\wred$ for $\red^*$, and $P\red$ if $\exists Q $ such that $ P \red Q$.
We write $P\red$ if $\exists Q $ such that $ P \red Q$ and $P\not\red$, otherwise.

\section{Replication}

As mentioned before, it is known that replication (and hence
recursion) can be implemented in a higher-order process algebra
\cite{SangiorgiWalker}. As our first example of calculation with the
machinery thus far presented we give the construction explicitly in
the {\rhoc}.

\begin{eqnarray}
	D_{x} & := & \prefix{x}{y}{(\binpar{\outputp{x}{y}}{@{y}})} \nonumber\\
	\bangp_{x}{P} & := & \binpar{{x}!\langle{\binpar{D_{x}}{P}}\rangle}{D_{x}} \nonumber
\end{eqnarray}

\begin{eqnarray}
	\bangp_{x}{P} & & \nonumber\\
	=
	& {x}!\langle{(\prefix{x}{y}{(\outputp{x}{y} | @{y})) | P}}\rangle 
	      | \prefix{x}{y}{(\outputp{x}{y} | @{y})} & \nonumber\\
	\red
	& (\outputp{x}{y} | @{y})\substn{\quotep{(\prefix{x}{y}{(@{y} | \outputp{x}{y})) | P}}}{y} & \nonumber\\
	=
	& \outputp{x}{\quotep{(\prefix{x}{y}{(\outputp{x}{y} | @{y})) | P}}}
	  | {(\prefix{x}{y}{(\outputp{x}{y} | @{y})) | P}} & \nonumber\\
	\red
	& \ldots & \nonumber\\
	\red^*
	& P | P | \ldots & \nonumber
\end{eqnarray}

Of course, this encoding, as an implementation, runs away, unfolding
$\bangp{P}$ eagerly. A lazier and more implementable replication
operator, restricted to input-guarded processes, may be obtained as follows.

\begin{eqnarray}
\bangp{\prefix{u}{v}{P}} 
	:= 
	\binpar{\lift{x}{\prefix{u}{v}{(\binpar{D(x)}{P})}}}{D(x)} \nonumber
\end{eqnarray}

\begin{remark}
  Note that the lazier definition still does not deal with summation
  or mixed summation (i.e. sums over input and output). The reader is
  invited to construct definitions of replication that deal with these
  features. 

  Further, the definitions are parameterized in a name, $x$. Can you,
  gentle reader, make a definition that eliminates this parameter and
  guarantees no accidental interaction between the replication
  machinery and the process being replicated -- i.e. no accidental
  sharing of names used by the process to get its work done and the
  name(s) used by the replication to effect copying. This latter
  revision of the definition of replication is crucial to obtaining
  the expected identity $!!P \sim !P$.
\end{remark}

\begin{remark}\label{rem:paradoxical_combinator}
  The reader familiar with the lambda calculus will have noticed the
  similarity between $D$ and the paradoxical combinator.

  [Ed. note: the existence of this seems to suggest we have to be more
  restrictive on the set of processes and names we admit if we are to
  support no-cloning.]
\end{remark}

\subsubsection{Bisimulation}

The computational dynamics gives rise to another kind of equivalence,
the equivalence of computational behavior. As previously mentioned
this is typically captured \emph{via} some form of bisimulation.

% The notion we use in this paper is weak barbed bisimulation
% \cite{milner91polyadicpi}.

The notion we use in this paper is derived from weak barbed
bisimulation \cite{milner91polyadicpi}. 

\begin{definition}
An \emph{observation relation}, $\downarrow_{\mathcal N}$, over a set
of names, $\mathcal N$, is the smallest relation satisfying the rules
below.

\infrule[Out-barb]{y \in {\mathcal N}, \; x \nameeq y}
		  {\outputp{x}{v} \downarrow_{\mathcal N} x}
\infrule[Par-barb]{\mbox{$P\downarrow_{\mathcal N} x$ or $Q\downarrow_{\mathcal N} x$}}
		  {\binpar{P}{Q} \downarrow_{\mathcal N} x}

We write $P \Downarrow_{\mathcal N} x$ if there is $Q$ such that 
$P \wred Q$ and $Q \downarrow_{\mathcal N} x$.
\end{definition}

\begin{definition}
%\label{def.bbisim}
An  ${\mathcal N}$-\emph{barbed bisimulation} over a set of names, ${\mathcal N}$, is a symmetric binary relation 
${\mathcal S}_{\mathcal N}$ between agents such that $P\rel{S}_{\mathcal N}Q$ implies:
\begin{enumerate}
\item If $P \red P'$ then $Q \wred Q'$ and $P'\rel{S}_{\mathcal N} Q'$.
\item If $P\downarrow_{\mathcal N} x$, then $Q\Downarrow_{\mathcal N} x$.
\end{enumerate}
$P$ is ${\mathcal N}$-barbed bisimilar to $Q$, written
$P \wbbisim_{\mathcal N} Q$, if $P \rel{S}_{\mathcal N} Q$ for some ${\mathcal N}$-barbed bisimulation ${\mathcal S}_{\mathcal N}$.
\end{definition}

$\mathcal{R} \subseteq \pi \times \pi$

$P \mathcal{R} Q => \forall P'. P \red P' \Rightarrow \exists Q'. Q \red Q', P' \mathcal{R} Q'$

$P \vdash x \Rightarrow Q \vdash x$

\begin{mathpar}
  \inferrule*[lab=Out-barb]{x \nameeq y}{{y}!\langle{Q}\rangle \vdash x}
  \and
  \inferrule*[lab=Par-barb]{\mbox{$P\vdash x$ or $Q\vdash x$}}{\binpar{P}{Q} \vdash x}
\end{mathpar}

\subsubsection{Contexts}

One of the principle advantages of computational calculi like the
$\pi$-calculus is a well-defined notion of context,
contextual-equivalence and a correlation between
contextual-equivalence and notions of bisimulation. The notion of
context allows the decomposition of a process into (sub-)process and
its syntactic environment, its context. Thus, a context may be
thought of as a process with a ``hole'' (written $\Box$) in it. The
application of a context $M$ to a process $P$, written $M[P]$, is
tantamount to filling the hole in $M$ with $P$. In this paper we do
not need the full weight of this theory, but do make use of the notion
of context in the proof the main theorem. 

\begin{mathpar}
  \inferrule* [lab=summation] {} {{M_{M},M_{N}} \bc \Box \;|\; x.M_{A} \;|\; M_{M}+M_{N}}
  \and
  \inferrule* [lab=agent] {} {{M_{A}} \bc (\vec{x})M_{P} \;| \; \clift{P_0,\ldots,M_{P},\ldots,P_N}}
  \and \\
  \inferrule* [lab=process] {} {{M_{P}} \bc M_{N} \;| \;P|M_{P} }
\end{mathpar} 

\begin{mathpar}
  \inferrule* [lab=sychronization] {} {M_{N} \bc \Box \;|\; x?M_{F} \;|\; x!M_{C}}
  \and
  \inferrule* [lab=abstraction] {} {{M_{F}} \bc (x)M_{P} }
  \and
  \inferrule* [lab=concretion] {} {{M_{C}} \bc \langle M_{P} \rangle }
  \and \\
  \inferrule* [lab=process] {} {{M_{P}} \bc M_{N} \;| \;P|M_{P} }
\end{mathpar}

\begin{definition}[contextual application] Given a context $M$, and
  process $P$, we define the \emph{contextual application}, $M[P] :=
  M\{P/\Box\}$. That is, the contextual application of M to P is the
  substitution of $P$ for $\Box$ in $M$.
\end{definition}

$\meaningof{-} : L \to \mathcal{P}(\pi)$

\begin{mathpar}
  \inferrule* [lab=collection] {} {\meaningof{true} = \pi, \and \meaningof{~E} = \pi \setminus \meaningof{E}, \and \meaningof{E_{1} \& E_{2}} = \meaningof{E_{1}} \cap \meaningof{E_{2}}}
\end{mathpar}

\begin{mathpar}
  \inferrule* [lab=structure] {} {\meaningof{0} = \{ P \in \pi | P \equiv 0 \}, \and \\ \meaningof{E_1 | E_2} = \{ P \in \pi | P \equiv P_{1} | P_{2}, P_{1} \in \meaningof{E_{1}}, P_{2} \in \meaningof{E_2}\} }
\end{mathpar}

\begin{mathpar}
 \inferrule* [lab=behavior] {} {\meaningof{\langle a?b \rangle E} = \{ P \in \pi | P \equiv Q | u?(y)P', \\ \and \\\\ \and \\ \;\;\; u \in \meaningof{a}, \forall z.P'\{z/y\} \in \meaningof{E\{z/b\}}\}, \and \\ \meaningof{a!E} = \{ P \in \pi | P \equiv Q | x!\langle P' \rangle, x \in \meaningof{a} P' \in \meaningof{E}\} }
\end{mathpar}

\begin{mathpar}
 \inferrule* [lab=nominal] {} {\meaningof{\quotep{E}} = \{ \quotep{P} \in \quotep{\pi} | P \in \meaningof{E} \}, \and \meaningof{\quotep{P}} = \{ \quotep{Q} \in \quotep{\pi} | P \equiv Q \} \and \\ \meaningof{@\quotep{E}} = \{ P \in \pi | P \equiv @x, x \in \meaningof{E} \}}
\end{mathpar}

\begin{eqnarray*}
  \\
  \meaningof{-} : TS \to ST
\end{eqnarray*}

\begin{eqnarray*}
  \\
  L : TS \to ST
\end{eqnarray*}

\begin{eqnarray*}
  \\
  P \models E \iff P \in \meaningof{E}
\end{eqnarray*}

\begin{eqnarray*}
  P \approx_{L} Q \iff \forall E \in L. P \models E \iff Q \models E
\end{eqnarray*}

\begin{eqnarray*}
  P \approx_{K} Q
\end{eqnarray*}

\begin{eqnarray*}
  P \approx Q
\end{eqnarray*}

$\approx_{K} = \approx = \approx_{L}$

\subsubsection{Contextual duality}

Note that contexts extend the quotation operation to a family of
operations from processes to names. Given a context, $M$, we can
define a \emph{nominal context}, $\quotep{M}$ by $\quotep{M}[P] :=
\quotep{M[P]}$. To foreshadow what is to come we observe that these
operations enjoy a duality with processes very much like the duality
between vectors and maps from vectors to scalars.

Further, because the calculus is essentially higher-order, we have a
correspondence between contexts and processes. More specifically,
given a name $x$ and a context $M$ we can construct $M^{*}_{x}$ such
that 

\begin{mathpar}
  M^{*}_{x} | \lift{x}{P} \red M[P]
\end{mathpar}

namely,

\begin{mathpar}
  M^{*}_{x} := x?(u).M[\dropn{u}]
\end{mathpar}

The dependence of $M^{*}_{x}$ on a name makes it an abstraction, 

\begin{mathpar}
  M^{*} := (x)x?(u).M[\dropn{u}]
\end{mathpar}

\subsection{Additional notation}

It will sometimes be convenient to denote the process a name
quotes. We already have the notation $x = \quotep{P}$, but it will be
convenient to introduce an alternate notation, $\procn{x}$, when we
want to emphasize the connection to the use of the name. Note that, by
virtue of name equivalence, $\quotep{\procn{x}} \nameeq x$; so, the
notation is consistent with previous definitions.

Further, because names have structure it is possible to effect
substitutions on the basis of that structure. This means we need to
upgrade our notation for substitutions, which we accomplish by
adapting comprehension notation. Thus,

\begin{mathpar}
  P\{ y / x : x \in S \}
\end{mathpar}

is interpreted to mean the process derived from P by replacing (in a
capture-avoiding manner) each occurrence of $x$ in $S$ by $y$. For example,

\begin{mathpar}
  P\{ \quotep{\procn{x}|\procn{x}} / x : x \in \freenames{P} \}
\end{mathpar}

will replace each (occurrence) of a free name $x$ in $P$ by
$\quotep{\procn{x}|\procn{x}}$.

Also, we will avail ourselves of the notation $x^{L}$ and $x^{R}$ to
denote injections of a name into disjoint copies of the name
space. There are numerous ways to accomplish this. One example can be
found in \cite{MeredithR05}. This notation overloads to vectors of
names: $\vec{x}^{\pi} := (x_{i}^{\pi} \; : \; 0 \leq i < |\vec{x}| )$ where $\pi \in \{L,R\}$.

We also use $P^{\Box} := P|\Box$.

In \cite{MeredithR05} an interpretation of the new operator is
given. It turns out that there are several possible interpretations
all enjoying the requisite algebraic properties of the operator (see
\cite{milner91polyadicpi}). We will therefore make liberal use of
$(\nu\; \vec{x})P$.

% subsection the_syntax_and_semantics_of_the_notation_system (end)   

\input{qm2pi.qmops} 

\input{qm2pi.sterngerlach} 

\input{qm2pi.metric} 

% section concurrent_process_calculi (end)

%\input{qm2pi.proofsketch}

% section proof sketch (end)

%\input{qm2pi.slviaknots} 

% section spatial logic via knots (end)

\input{qm2pi.conclusion}

% section conclusion (end)

%\input{qm2pi.dtcodes} 

% section wiring algorithm (end)

\input{qm2pi.ack} 

% section acknowledgments (end)

\newpage


\bibliographystyle{plain}   
\bibliography{../../biblios/main.bib}

\input{qm2pi.rhodetails}

\end{document}

 

% section wiring algorithm (end)

\documentclass[12pt]{llncs}
%\documentclass{jktr}

\usepackage[pdftex]{hyperref}                   
\usepackage {listings}
\usepackage {mathpartir}
\usepackage{bcprules}
%\usepackage{listings}
                       
\usepackage{graphicx} 
%\usepackage[margins=2.5cm,nohead,nofoot]{geometry}
%\usepackage{geometry}
\usepackage{amsfonts}
\usepackage{amstext}
\usepackage{latexsym}
\usepackage{amssymb}
\usepackage{color}


%\include{myPreamble}
\include{qm2pi.local} 

%\ifpdf
%\usepackage[pdftex]{graphicx}
%\else
%\usepackage{graphicx}
%\fi

 % \ifpdf
%  \usepackage{pdfsync}
%  \if


%\title{Brief Article}
%\author{David F. Snyder}
%\author{L.G. Meredith}

%\address{Dept. of Math., Texas State University--San Marcos, San Marcos, TX 78666}
       
\pagestyle{empty}


\begin{document}

\lstset{language=[Objective]Caml,frame=shadowbox}

\input{qm2pi.front}

% section front matter (end)

\input{qm2pi.intro} 
 
% section introduction (end)

% \input{qm2pi.knotations} 

% section notation (end)

\input{qm2pi.process.calculi} 

% section concurrent_process_calculi_and_spatial_logics_ (end)
    
%\input{qm2pi.knots2pi} 

%\input{qm2pi.trefoil} 

%\input{qm2pi.mainthm} 

% subsection basic_interpretation (end)

%\input{qm2pi.rho.presentation} 
\subsection{The syntax and semantics of the notation system}\label{sub:the_syntax_and_semantics_of_the_notation_system} % (fold)

We now summarize a technical presentation of the calculus that
embodies our theory of dynamics. The typical presentation of such a
calculus follows the style of giving generators and relations on
them. The grammar, below, describing term constructors, freely
generates the set of processes, $\Proc$. This set is then quotiented
by a relation known as structural congruence and it is over this set
that the notion of dynamics is expressed. This presentation is
essentially that of \cite{MeredithR05} with the addition of
polyadicity and summation. For readability we have relegated some of
the technical subtleties to an appendix.

\subsubsection{Process grammar}\label{subsub:process_grammar}

\begin{mathpar}
  \inferrule* [lab=synchronization] {} {{M} \bc \pzero \;|\; x?F \;|\; x!C }
  \and
  \inferrule* [lab=abstraction] {} {{F} \bc (x)P}
  \and
  \inferrule* [lab=concretion] {} {{C} \bc \langle Q \rangle}
  \and
  \inferrule* [lab=process] {} {{P,Q} \bc M \;| \;P|Q \;|\; @{x}}
  \and
  \inferrule* [lab=name] {} {{x} \bc \quotep{P}}
\end{mathpar} 

Note that $\vec{x}$ (resp. $\vec{P}$) denotes a vector of names
(resp. processes) of length $|\vec{x}|$ (resp. $|\vec{P}|$). We adopt
the following useful abbreviations.

\begin{mathpar}
   x?(\vec{y}).P := x.(\vec{y})P \and  x\clift{\vec{P}} := x.\clift{\vec{P}}
   \and x!(y) := \lift{x}{\dropn{y}}
   \and \Pi_{i=0}^{n-1}P_i := P_0 | \ldots | P_{n-1}
\end{mathpar}

\subsubsection{Structural congruence}

\paragraph{Free and bound names and alpha-equivalence.} At the
core of structural equivalence is alpha-equivalence which identifies
process that are the same up to a change of variable. Formally, we
recognize the distinction between free and bound names. The free names
of a process, $\freenames{P}$, may be calculated recursively as
follows:

\begin{mathpar}
\freenames{\pzero} := \emptyset
  \and \\
  \freenames{x?(y).P} := \{ x \} \cup (\freenames{P} \setminus \{ y \})
  \and 
  \freenames{x!\langle P \rangle} := \{ x \} \cup \{ P \} 
  \and \\
  \freenames{P|Q} := \freenames{P} \cup \freenames{Q}
  \and \\
  \freenames{@{x}} := \{ x \}
\end{mathpar}

$\pi$
$\quotep{\pi}$

$\freenames{-} : \pi \to \mathcal{P}(\quotep{\pi})$

\begin{eqnarray*}
  \freenames{\pzero} & := & \emptyset \\
  \freenames{x?(y).P} & := & \{ x \} \cup (\freenames{P} \setminus \{ y \}) \\
  \freenames{x!\langle P \rangle} & := & \{ x \} \cup \{ P \} \\
  \freenames{P|Q} & := & \freenames{P} \cup \freenames{Q} \\
  \freenames{\dropn{x}} & := & \{ x \}
\end{eqnarray*}

The bound names of a process, $\boundnames{P}$, are those names occurring in $P$
that are not free. For example, in $x?(y).0$, the name $x$ is free, while $y$ is bound.

\begin{mathpar}
  \inferrule* [lab=monoidal-laws] {} { P|Q \equiv Q|P \and P|0 \equiv P \and P|(Q|R) \equiv (P|Q)|R }
\end{mathpar}

\begin{mathpar}
  \inferrule* [lab=alpha-equivalence] {} { (x)P \equiv (y)P\{y/x\} \and y \not\in \freenames{P} }
\end{mathpar}

\begin{definition}
Then two processes, $P,Q$, are alpha-equivalent if $P = Q\{\vec{y}/\vec{x}\}$ for
some $\vec{x} \in \boundnames{Q},\vec{y} \in \boundnames{P}$, where $Q\{\vec{y}/\vec{x}\}$
denotes the capture-avoiding substitution of $\vec{y}$ for $\vec{x}$ in $Q$.
\end{definition}

\begin{definition}
  The {\em structural congruence} \cite{SangiorgiWalker} , $\equiv$,
  between processes is the least congruence containing
  alpha-equivalence, satisfying the abelian monoid laws
  (associativity, commutativity and $\pzero$ as identity) for parallel
  composition $|$ and for summation $+$.
\end{definition}

\subsection{Name equivalence}

We take name equivalence, written $\nameeq$, to be the smallest
equivalence relation generated by the following rules.

\begin{mathpar}
\inferrule*[lab=Quote-drop]
{ }
{ \quotep{@{x}} \nameeq x }

\inferrule*[lab=Struct-equiv]
{ P \scong Q }
{ \quotep{P} \nameeq \quotep{Q} }
\end{mathpar}

The astute reader will have noticed that the mutual recursion of names
and processes imposes a mutual recursion on alpha-equivalence and
structural equivalence via name-equivalence. Fortunately, all of this
works out pleasantly and we may calculate in the natural way, free of
concern. The reader interested in the details is referred to the
appendix \ref{appendix:rho_details}.

\subsection{Substitution}

We use $\Proc$ for the set of processes, $\QProc$ for the set of
names, and $\id{\{}\vec{y} / \vec{x} \id{\}}$ to denote partial maps,
$s : \QProc \rightarrow \QProc$. A map, $s$ lifts, uniquely, to a map
on process terms, $\widehat{s} : \Proc \rightarrow \Proc$ by the
following equations.

\begin{mathpar}
  (0) \psubstp{Q}{P} := 0 \\
  (R \juxtap S) \psubstp{Q}{P}
  :=    
  (R)\psubstp{Q}{P} \juxtap (S) \psubstp{Q}{P} \\
  (x?(y).R) \psubstp{Q}{P}    
  :=    
  (x)\substp{Q}{P} (z)\concat( (R \psubstn{z}{y}) \psubstp{Q}{P} ) \\
  (\lift{x}{R}) \psubstp{Q}{P}  
  :=
  \lift{(x)\substp{Q}{P}}{ R \psubstp{Q}{P} } \\
%   (\dropn{x})  \psubstp{Q}{P}       
%   := 
%   \left\{ 
%     \begin{array}{ccc} 
%       \dropn{\quotep{Q}} & & x \nameeq \quotep{P} \\
%       \dropn{x} & & otherwise \\
%     \end{array}
%   \right. 
  (\dropn{x})  \psubstp{Q}{P}       
  := 
  \left\{ 
    \begin{array}{ccc} 
      Q & & x \nameeq \quotep{P} \\
      \dropn{x} & & otherwise \\
    \end{array}
  \right.
\end{mathpar}
 

where

\begin{eqnarray}
  (x)\id{\{} \lpquote Q \rpquote / \lpquote P \rpquote \id{\}}            = 
  \left\{ 
    \begin{array}{ccc}
      \lpquote Q \rpquote & & x \nameeq \lpquote P \rpquote \\
      x & & otherwise \\
    \end{array}
  \right. \nonumber
\end{eqnarray}

and $z$ is chosen distinct from $\quotep{P}$, $\quotep{Q}$, the free
names in $Q$, and all the names in $R$. Our $\alpha$-equivalence will
be built in the standard way from this substitution.

\begin{remark}\label{rem:no_self_referential_names}
  One consequence of these definitions is that $\forall P. \quotep{P}
  \not\in \freenames{P}$.
\end{remark}

\subsection{ Dynamic quote: an example }

Anticipating something of what's to come, consider applying the
substitution, $\widehat{\id{\{}u / z \id{\}}}$, to the following pair
of processes, $\lift{w}{y!(z)}$ and $w[ \lpquote y!(z) \rpquote ]$.

\begin{eqnarray}
	\lift{w}{y!(z)}\widehat{\id{\{}u / z \id{\}}}
		& = &
		\lift{w}{y!(u)} \nonumber\\
	w[ \lpquote y!(z) \rpquote ] \widehat{ \id{\{}u / z \id{\}} }
		& = &
		w[ \lpquote y!(z) \rpquote ] \nonumber
\end{eqnarray}

Because the body of the process between quotes is impervious to
substitution, we get radically different answers. In fact, by
examining the first process in an input context,
e.g. $x?(z).\lift{w}{y!(z)}$, we see that the process under the lift
operator may be shaped by prefixed inputs binding a name inside it. In
this sense, the lift operator will be seen as a way to dynamically
construct processes before reifying them as names.

Finally equipped with these standard features we can present the
dynamics of the calculus.

\subsubsection{Operational semantics} 

Finally, we introduce the computational dynamics. What marks these
algebras as distinct from other more traditionally studied algebraic
structures, e.g. vector spaces or polynomial rings, is the manner in
which dynamics is captured. In traditional structures, dynamics is typically
expressed through morphisms between such structures, as in linear maps
between vector spaces or morphisms between rings. In algebras
associated with the semantics of computation, the dynamics is
expressed as part of the algebraic structure itself, through a
reduction reduction relation typically denoted by $\red$. Below, we
give a recursive presentation of this relation for the calculus used
in the encoding.

$\red \subseteq \pi \times \pi$
$\red : \pi \to \mathcal{P}(\pi)$

\begin{mathpar}
  \inferrule* [lab=Comm] { \textsf{match}( x_{src}, x_{trgt} ) } { x_{trgt}?(y)P \; | \; x_{src}!\langle {Q} \rangle \red P\{\quotep{Q}/y}\} }
  \and \\
  \inferrule* [lab=Par] {{P} \red {P}'} {{{P} | {Q}} \red {{P}' | {Q}}}
  \and
  \inferrule* [lab=Equiv]{{{P} \scong {P}'} \andalso {{P}' \red {Q}'} \andalso {{Q}' \scong {Q}}}{{P} \red {Q}}
\end{mathpar}

\begin{eqnarray*}
  match_{\equiv} (\quotep{P},\quotep{Q}) & := & P \equiv Q \\
  match_{\dagger}(\quotep{P},\quotep{Q}) & := & \forall R. P|Q \red^{*} R => R \red^{*} 0 \\
  match_{K}(\quotep{P},\quotep{Q}) & := & K \mbox{ for some context } K
\end{eqnarray*}

$u?(x)P | u!\langle Q \rangle \red P\{\quotep{Q}/x\}$

%We write $\wred$ for $\red^*$, and $P\red$ if $\exists Q $ such that $ P \red Q$.
We write $P\red$ if $\exists Q $ such that $ P \red Q$ and $P\not\red$, otherwise.

\section{Replication}

As mentioned before, it is known that replication (and hence
recursion) can be implemented in a higher-order process algebra
\cite{SangiorgiWalker}. As our first example of calculation with the
machinery thus far presented we give the construction explicitly in
the {\rhoc}.

\begin{eqnarray}
	D_{x} & := & \prefix{x}{y}{(\binpar{\outputp{x}{y}}{@{y}})} \nonumber\\
	\bangp_{x}{P} & := & \binpar{{x}!\langle{\binpar{D_{x}}{P}}\rangle}{D_{x}} \nonumber
\end{eqnarray}

\begin{eqnarray}
	\bangp_{x}{P} & & \nonumber\\
	=
	& {x}!\langle{(\prefix{x}{y}{(\outputp{x}{y} | @{y})) | P}}\rangle 
	      | \prefix{x}{y}{(\outputp{x}{y} | @{y})} & \nonumber\\
	\red
	& (\outputp{x}{y} | @{y})\substn{\quotep{(\prefix{x}{y}{(@{y} | \outputp{x}{y})) | P}}}{y} & \nonumber\\
	=
	& \outputp{x}{\quotep{(\prefix{x}{y}{(\outputp{x}{y} | @{y})) | P}}}
	  | {(\prefix{x}{y}{(\outputp{x}{y} | @{y})) | P}} & \nonumber\\
	\red
	& \ldots & \nonumber\\
	\red^*
	& P | P | \ldots & \nonumber
\end{eqnarray}

Of course, this encoding, as an implementation, runs away, unfolding
$\bangp{P}$ eagerly. A lazier and more implementable replication
operator, restricted to input-guarded processes, may be obtained as follows.

\begin{eqnarray}
\bangp{\prefix{u}{v}{P}} 
	:= 
	\binpar{\lift{x}{\prefix{u}{v}{(\binpar{D(x)}{P})}}}{D(x)} \nonumber
\end{eqnarray}

\begin{remark}
  Note that the lazier definition still does not deal with summation
  or mixed summation (i.e. sums over input and output). The reader is
  invited to construct definitions of replication that deal with these
  features. 

  Further, the definitions are parameterized in a name, $x$. Can you,
  gentle reader, make a definition that eliminates this parameter and
  guarantees no accidental interaction between the replication
  machinery and the process being replicated -- i.e. no accidental
  sharing of names used by the process to get its work done and the
  name(s) used by the replication to effect copying. This latter
  revision of the definition of replication is crucial to obtaining
  the expected identity $!!P \sim !P$.
\end{remark}

\begin{remark}\label{rem:paradoxical_combinator}
  The reader familiar with the lambda calculus will have noticed the
  similarity between $D$ and the paradoxical combinator.

  [Ed. note: the existence of this seems to suggest we have to be more
  restrictive on the set of processes and names we admit if we are to
  support no-cloning.]
\end{remark}

\subsubsection{Bisimulation}

The computational dynamics gives rise to another kind of equivalence,
the equivalence of computational behavior. As previously mentioned
this is typically captured \emph{via} some form of bisimulation.

% The notion we use in this paper is weak barbed bisimulation
% \cite{milner91polyadicpi}.

The notion we use in this paper is derived from weak barbed
bisimulation \cite{milner91polyadicpi}. 

\begin{definition}
An \emph{observation relation}, $\downarrow_{\mathcal N}$, over a set
of names, $\mathcal N$, is the smallest relation satisfying the rules
below.

\infrule[Out-barb]{y \in {\mathcal N}, \; x \nameeq y}
		  {\outputp{x}{v} \downarrow_{\mathcal N} x}
\infrule[Par-barb]{\mbox{$P\downarrow_{\mathcal N} x$ or $Q\downarrow_{\mathcal N} x$}}
		  {\binpar{P}{Q} \downarrow_{\mathcal N} x}

We write $P \Downarrow_{\mathcal N} x$ if there is $Q$ such that 
$P \wred Q$ and $Q \downarrow_{\mathcal N} x$.
\end{definition}

\begin{definition}
%\label{def.bbisim}
An  ${\mathcal N}$-\emph{barbed bisimulation} over a set of names, ${\mathcal N}$, is a symmetric binary relation 
${\mathcal S}_{\mathcal N}$ between agents such that $P\rel{S}_{\mathcal N}Q$ implies:
\begin{enumerate}
\item If $P \red P'$ then $Q \wred Q'$ and $P'\rel{S}_{\mathcal N} Q'$.
\item If $P\downarrow_{\mathcal N} x$, then $Q\Downarrow_{\mathcal N} x$.
\end{enumerate}
$P$ is ${\mathcal N}$-barbed bisimilar to $Q$, written
$P \wbbisim_{\mathcal N} Q$, if $P \rel{S}_{\mathcal N} Q$ for some ${\mathcal N}$-barbed bisimulation ${\mathcal S}_{\mathcal N}$.
\end{definition}

$\mathcal{R} \subseteq \pi \times \pi$

$P \mathcal{R} Q => \forall P'. P \red P' \Rightarrow \exists Q'. Q \red Q', P' \mathcal{R} Q'$

$P \vdash x \Rightarrow Q \vdash x$

\begin{mathpar}
  \inferrule*[lab=Out-barb]{x \nameeq y}{{y}!\langle{Q}\rangle \vdash x}
  \and
  \inferrule*[lab=Par-barb]{\mbox{$P\vdash x$ or $Q\vdash x$}}{\binpar{P}{Q} \vdash x}
\end{mathpar}

\subsubsection{Contexts}

One of the principle advantages of computational calculi like the
$\pi$-calculus is a well-defined notion of context,
contextual-equivalence and a correlation between
contextual-equivalence and notions of bisimulation. The notion of
context allows the decomposition of a process into (sub-)process and
its syntactic environment, its context. Thus, a context may be
thought of as a process with a ``hole'' (written $\Box$) in it. The
application of a context $M$ to a process $P$, written $M[P]$, is
tantamount to filling the hole in $M$ with $P$. In this paper we do
not need the full weight of this theory, but do make use of the notion
of context in the proof the main theorem. 

\begin{mathpar}
  \inferrule* [lab=summation] {} {{M_{M},M_{N}} \bc \Box \;|\; x.M_{A} \;|\; M_{M}+M_{N}}
  \and
  \inferrule* [lab=agent] {} {{M_{A}} \bc (\vec{x})M_{P} \;| \; \clift{P_0,\ldots,M_{P},\ldots,P_N}}
  \and \\
  \inferrule* [lab=process] {} {{M_{P}} \bc M_{N} \;| \;P|M_{P} }
\end{mathpar} 

\begin{mathpar}
  \inferrule* [lab=sychronization] {} {M_{N} \bc \Box \;|\; x?M_{F} \;|\; x!M_{C}}
  \and
  \inferrule* [lab=abstraction] {} {{M_{F}} \bc (x)M_{P} }
  \and
  \inferrule* [lab=concretion] {} {{M_{C}} \bc \langle M_{P} \rangle }
  \and \\
  \inferrule* [lab=process] {} {{M_{P}} \bc M_{N} \;| \;P|M_{P} }
\end{mathpar}

\begin{definition}[contextual application] Given a context $M$, and
  process $P$, we define the \emph{contextual application}, $M[P] :=
  M\{P/\Box\}$. That is, the contextual application of M to P is the
  substitution of $P$ for $\Box$ in $M$.
\end{definition}

$\meaningof{-} : L \to \mathcal{P}(\pi)$

\begin{mathpar}
  \inferrule* [lab=collection] {} {\meaningof{true} = \pi, \and \meaningof{~E} = \pi \setminus \meaningof{E}, \and \meaningof{E_{1} \& E_{2}} = \meaningof{E_{1}} \cap \meaningof{E_{2}}}
\end{mathpar}

\begin{mathpar}
  \inferrule* [lab=structure] {} {\meaningof{0} = \{ P \in \pi | P \equiv 0 \}, \and \\ \meaningof{E_1 | E_2} = \{ P \in \pi | P \equiv P_{1} | P_{2}, P_{1} \in \meaningof{E_{1}}, P_{2} \in \meaningof{E_2}\} }
\end{mathpar}

\begin{mathpar}
 \inferrule* [lab=behavior] {} {\meaningof{\langle a?b \rangle E} = \{ P \in \pi | P \equiv Q | u?(y)P', \\ \and \\\\ \and \\ \;\;\; u \in \meaningof{a}, \forall z.P'\{z/y\} \in \meaningof{E\{z/b\}}\}, \and \\ \meaningof{a!E} = \{ P \in \pi | P \equiv Q | x!\langle P' \rangle, x \in \meaningof{a} P' \in \meaningof{E}\} }
\end{mathpar}

\begin{mathpar}
 \inferrule* [lab=nominal] {} {\meaningof{\quotep{E}} = \{ \quotep{P} \in \quotep{\pi} | P \in \meaningof{E} \}, \and \meaningof{\quotep{P}} = \{ \quotep{Q} \in \quotep{\pi} | P \equiv Q \} \and \\ \meaningof{@\quotep{E}} = \{ P \in \pi | P \equiv @x, x \in \meaningof{E} \}}
\end{mathpar}

\begin{eqnarray*}
  \\
  \meaningof{-} : TS \to ST
\end{eqnarray*}

\begin{eqnarray*}
  \\
  L : TS \to ST
\end{eqnarray*}

\begin{eqnarray*}
  \\
  P \models E \iff P \in \meaningof{E}
\end{eqnarray*}

\begin{eqnarray*}
  P \approx_{L} Q \iff \forall E \in L. P \models E \iff Q \models E
\end{eqnarray*}

\begin{eqnarray*}
  P \approx_{K} Q
\end{eqnarray*}

\begin{eqnarray*}
  P \approx Q
\end{eqnarray*}

$\approx_{K} = \approx = \approx_{L}$

\subsubsection{Contextual duality}

Note that contexts extend the quotation operation to a family of
operations from processes to names. Given a context, $M$, we can
define a \emph{nominal context}, $\quotep{M}$ by $\quotep{M}[P] :=
\quotep{M[P]}$. To foreshadow what is to come we observe that these
operations enjoy a duality with processes very much like the duality
between vectors and maps from vectors to scalars.

Further, because the calculus is essentially higher-order, we have a
correspondence between contexts and processes. More specifically,
given a name $x$ and a context $M$ we can construct $M^{*}_{x}$ such
that 

\begin{mathpar}
  M^{*}_{x} | \lift{x}{P} \red M[P]
\end{mathpar}

namely,

\begin{mathpar}
  M^{*}_{x} := x?(u).M[\dropn{u}]
\end{mathpar}

The dependence of $M^{*}_{x}$ on a name makes it an abstraction, 

\begin{mathpar}
  M^{*} := (x)x?(u).M[\dropn{u}]
\end{mathpar}

\subsection{Additional notation}

It will sometimes be convenient to denote the process a name
quotes. We already have the notation $x = \quotep{P}$, but it will be
convenient to introduce an alternate notation, $\procn{x}$, when we
want to emphasize the connection to the use of the name. Note that, by
virtue of name equivalence, $\quotep{\procn{x}} \nameeq x$; so, the
notation is consistent with previous definitions.

Further, because names have structure it is possible to effect
substitutions on the basis of that structure. This means we need to
upgrade our notation for substitutions, which we accomplish by
adapting comprehension notation. Thus,

\begin{mathpar}
  P\{ y / x : x \in S \}
\end{mathpar}

is interpreted to mean the process derived from P by replacing (in a
capture-avoiding manner) each occurrence of $x$ in $S$ by $y$. For example,

\begin{mathpar}
  P\{ \quotep{\procn{x}|\procn{x}} / x : x \in \freenames{P} \}
\end{mathpar}

will replace each (occurrence) of a free name $x$ in $P$ by
$\quotep{\procn{x}|\procn{x}}$.

Also, we will avail ourselves of the notation $x^{L}$ and $x^{R}$ to
denote injections of a name into disjoint copies of the name
space. There are numerous ways to accomplish this. One example can be
found in \cite{MeredithR05}. This notation overloads to vectors of
names: $\vec{x}^{\pi} := (x_{i}^{\pi} \; : \; 0 \leq i < |\vec{x}| )$ where $\pi \in \{L,R\}$.

We also use $P^{\Box} := P|\Box$.

In \cite{MeredithR05} an interpretation of the new operator is
given. It turns out that there are several possible interpretations
all enjoying the requisite algebraic properties of the operator (see
\cite{milner91polyadicpi}). We will therefore make liberal use of
$(\nu\; \vec{x})P$.

% subsection the_syntax_and_semantics_of_the_notation_system (end)   

\input{qm2pi.qmops} 

\input{qm2pi.sterngerlach} 

\input{qm2pi.metric} 

% section concurrent_process_calculi (end)

%\input{qm2pi.proofsketch}

% section proof sketch (end)

%\input{qm2pi.slviaknots} 

% section spatial logic via knots (end)

\input{qm2pi.conclusion}

% section conclusion (end)

%\input{qm2pi.dtcodes} 

% section wiring algorithm (end)

\input{qm2pi.ack} 

% section acknowledgments (end)

\newpage


\bibliographystyle{plain}   
\bibliography{../../biblios/main.bib}

\input{qm2pi.rhodetails}

\end{document}

 

% section acknowledgments (end)

\newpage


\bibliographystyle{plain}   
\bibliography{../../biblios/main.bib}

\documentclass[12pt]{llncs}
%\documentclass{jktr}

\usepackage[pdftex]{hyperref}                   
\usepackage {listings}
\usepackage {mathpartir}
\usepackage{bcprules}
%\usepackage{listings}
                       
\usepackage{graphicx} 
%\usepackage[margins=2.5cm,nohead,nofoot]{geometry}
%\usepackage{geometry}
\usepackage{amsfonts}
\usepackage{amstext}
\usepackage{latexsym}
\usepackage{amssymb}
\usepackage{color}


%\include{myPreamble}
\include{qm2pi.local} 

%\ifpdf
%\usepackage[pdftex]{graphicx}
%\else
%\usepackage{graphicx}
%\fi

 % \ifpdf
%  \usepackage{pdfsync}
%  \if


%\title{Brief Article}
%\author{David F. Snyder}
%\author{L.G. Meredith}

%\address{Dept. of Math., Texas State University--San Marcos, San Marcos, TX 78666}
       
\pagestyle{empty}


\begin{document}

\lstset{language=[Objective]Caml,frame=shadowbox}

\input{qm2pi.front}

% section front matter (end)

\input{qm2pi.intro} 
 
% section introduction (end)

% \input{qm2pi.knotations} 

% section notation (end)

\input{qm2pi.process.calculi} 

% section concurrent_process_calculi_and_spatial_logics_ (end)
    
%\input{qm2pi.knots2pi} 

%\input{qm2pi.trefoil} 

%\input{qm2pi.mainthm} 

% subsection basic_interpretation (end)

%\input{qm2pi.rho.presentation} 
\subsection{The syntax and semantics of the notation system}\label{sub:the_syntax_and_semantics_of_the_notation_system} % (fold)

We now summarize a technical presentation of the calculus that
embodies our theory of dynamics. The typical presentation of such a
calculus follows the style of giving generators and relations on
them. The grammar, below, describing term constructors, freely
generates the set of processes, $\Proc$. This set is then quotiented
by a relation known as structural congruence and it is over this set
that the notion of dynamics is expressed. This presentation is
essentially that of \cite{MeredithR05} with the addition of
polyadicity and summation. For readability we have relegated some of
the technical subtleties to an appendix.

\subsubsection{Process grammar}\label{subsub:process_grammar}

\begin{mathpar}
  \inferrule* [lab=synchronization] {} {{M} \bc \pzero \;|\; x?F \;|\; x!C }
  \and
  \inferrule* [lab=abstraction] {} {{F} \bc (x)P}
  \and
  \inferrule* [lab=concretion] {} {{C} \bc \langle Q \rangle}
  \and
  \inferrule* [lab=process] {} {{P,Q} \bc M \;| \;P|Q \;|\; @{x}}
  \and
  \inferrule* [lab=name] {} {{x} \bc \quotep{P}}
\end{mathpar} 

Note that $\vec{x}$ (resp. $\vec{P}$) denotes a vector of names
(resp. processes) of length $|\vec{x}|$ (resp. $|\vec{P}|$). We adopt
the following useful abbreviations.

\begin{mathpar}
   x?(\vec{y}).P := x.(\vec{y})P \and  x\clift{\vec{P}} := x.\clift{\vec{P}}
   \and x!(y) := \lift{x}{\dropn{y}}
   \and \Pi_{i=0}^{n-1}P_i := P_0 | \ldots | P_{n-1}
\end{mathpar}

\subsubsection{Structural congruence}

\paragraph{Free and bound names and alpha-equivalence.} At the
core of structural equivalence is alpha-equivalence which identifies
process that are the same up to a change of variable. Formally, we
recognize the distinction between free and bound names. The free names
of a process, $\freenames{P}$, may be calculated recursively as
follows:

\begin{mathpar}
\freenames{\pzero} := \emptyset
  \and \\
  \freenames{x?(y).P} := \{ x \} \cup (\freenames{P} \setminus \{ y \})
  \and 
  \freenames{x!\langle P \rangle} := \{ x \} \cup \{ P \} 
  \and \\
  \freenames{P|Q} := \freenames{P} \cup \freenames{Q}
  \and \\
  \freenames{@{x}} := \{ x \}
\end{mathpar}

$\pi$
$\quotep{\pi}$

$\freenames{-} : \pi \to \mathcal{P}(\quotep{\pi})$

\begin{eqnarray*}
  \freenames{\pzero} & := & \emptyset \\
  \freenames{x?(y).P} & := & \{ x \} \cup (\freenames{P} \setminus \{ y \}) \\
  \freenames{x!\langle P \rangle} & := & \{ x \} \cup \{ P \} \\
  \freenames{P|Q} & := & \freenames{P} \cup \freenames{Q} \\
  \freenames{\dropn{x}} & := & \{ x \}
\end{eqnarray*}

The bound names of a process, $\boundnames{P}$, are those names occurring in $P$
that are not free. For example, in $x?(y).0$, the name $x$ is free, while $y$ is bound.

\begin{mathpar}
  \inferrule* [lab=monoidal-laws] {} { P|Q \equiv Q|P \and P|0 \equiv P \and P|(Q|R) \equiv (P|Q)|R }
\end{mathpar}

\begin{mathpar}
  \inferrule* [lab=alpha-equivalence] {} { (x)P \equiv (y)P\{y/x\} \and y \not\in \freenames{P} }
\end{mathpar}

\begin{definition}
Then two processes, $P,Q$, are alpha-equivalent if $P = Q\{\vec{y}/\vec{x}\}$ for
some $\vec{x} \in \boundnames{Q},\vec{y} \in \boundnames{P}$, where $Q\{\vec{y}/\vec{x}\}$
denotes the capture-avoiding substitution of $\vec{y}$ for $\vec{x}$ in $Q$.
\end{definition}

\begin{definition}
  The {\em structural congruence} \cite{SangiorgiWalker} , $\equiv$,
  between processes is the least congruence containing
  alpha-equivalence, satisfying the abelian monoid laws
  (associativity, commutativity and $\pzero$ as identity) for parallel
  composition $|$ and for summation $+$.
\end{definition}

\subsection{Name equivalence}

We take name equivalence, written $\nameeq$, to be the smallest
equivalence relation generated by the following rules.

\begin{mathpar}
\inferrule*[lab=Quote-drop]
{ }
{ \quotep{@{x}} \nameeq x }

\inferrule*[lab=Struct-equiv]
{ P \scong Q }
{ \quotep{P} \nameeq \quotep{Q} }
\end{mathpar}

The astute reader will have noticed that the mutual recursion of names
and processes imposes a mutual recursion on alpha-equivalence and
structural equivalence via name-equivalence. Fortunately, all of this
works out pleasantly and we may calculate in the natural way, free of
concern. The reader interested in the details is referred to the
appendix \ref{appendix:rho_details}.

\subsection{Substitution}

We use $\Proc$ for the set of processes, $\QProc$ for the set of
names, and $\id{\{}\vec{y} / \vec{x} \id{\}}$ to denote partial maps,
$s : \QProc \rightarrow \QProc$. A map, $s$ lifts, uniquely, to a map
on process terms, $\widehat{s} : \Proc \rightarrow \Proc$ by the
following equations.

\begin{mathpar}
  (0) \psubstp{Q}{P} := 0 \\
  (R \juxtap S) \psubstp{Q}{P}
  :=    
  (R)\psubstp{Q}{P} \juxtap (S) \psubstp{Q}{P} \\
  (x?(y).R) \psubstp{Q}{P}    
  :=    
  (x)\substp{Q}{P} (z)\concat( (R \psubstn{z}{y}) \psubstp{Q}{P} ) \\
  (\lift{x}{R}) \psubstp{Q}{P}  
  :=
  \lift{(x)\substp{Q}{P}}{ R \psubstp{Q}{P} } \\
%   (\dropn{x})  \psubstp{Q}{P}       
%   := 
%   \left\{ 
%     \begin{array}{ccc} 
%       \dropn{\quotep{Q}} & & x \nameeq \quotep{P} \\
%       \dropn{x} & & otherwise \\
%     \end{array}
%   \right. 
  (\dropn{x})  \psubstp{Q}{P}       
  := 
  \left\{ 
    \begin{array}{ccc} 
      Q & & x \nameeq \quotep{P} \\
      \dropn{x} & & otherwise \\
    \end{array}
  \right.
\end{mathpar}
 

where

\begin{eqnarray}
  (x)\id{\{} \lpquote Q \rpquote / \lpquote P \rpquote \id{\}}            = 
  \left\{ 
    \begin{array}{ccc}
      \lpquote Q \rpquote & & x \nameeq \lpquote P \rpquote \\
      x & & otherwise \\
    \end{array}
  \right. \nonumber
\end{eqnarray}

and $z$ is chosen distinct from $\quotep{P}$, $\quotep{Q}$, the free
names in $Q$, and all the names in $R$. Our $\alpha$-equivalence will
be built in the standard way from this substitution.

\begin{remark}\label{rem:no_self_referential_names}
  One consequence of these definitions is that $\forall P. \quotep{P}
  \not\in \freenames{P}$.
\end{remark}

\subsection{ Dynamic quote: an example }

Anticipating something of what's to come, consider applying the
substitution, $\widehat{\id{\{}u / z \id{\}}}$, to the following pair
of processes, $\lift{w}{y!(z)}$ and $w[ \lpquote y!(z) \rpquote ]$.

\begin{eqnarray}
	\lift{w}{y!(z)}\widehat{\id{\{}u / z \id{\}}}
		& = &
		\lift{w}{y!(u)} \nonumber\\
	w[ \lpquote y!(z) \rpquote ] \widehat{ \id{\{}u / z \id{\}} }
		& = &
		w[ \lpquote y!(z) \rpquote ] \nonumber
\end{eqnarray}

Because the body of the process between quotes is impervious to
substitution, we get radically different answers. In fact, by
examining the first process in an input context,
e.g. $x?(z).\lift{w}{y!(z)}$, we see that the process under the lift
operator may be shaped by prefixed inputs binding a name inside it. In
this sense, the lift operator will be seen as a way to dynamically
construct processes before reifying them as names.

Finally equipped with these standard features we can present the
dynamics of the calculus.

\subsubsection{Operational semantics} 

Finally, we introduce the computational dynamics. What marks these
algebras as distinct from other more traditionally studied algebraic
structures, e.g. vector spaces or polynomial rings, is the manner in
which dynamics is captured. In traditional structures, dynamics is typically
expressed through morphisms between such structures, as in linear maps
between vector spaces or morphisms between rings. In algebras
associated with the semantics of computation, the dynamics is
expressed as part of the algebraic structure itself, through a
reduction reduction relation typically denoted by $\red$. Below, we
give a recursive presentation of this relation for the calculus used
in the encoding.

$\red \subseteq \pi \times \pi$
$\red : \pi \to \mathcal{P}(\pi)$

\begin{mathpar}
  \inferrule* [lab=Comm] { \textsf{match}( x_{src}, x_{trgt} ) } { x_{trgt}?(y)P \; | \; x_{src}!\langle {Q} \rangle \red P\{\quotep{Q}/y}\} }
  \and \\
  \inferrule* [lab=Par] {{P} \red {P}'} {{{P} | {Q}} \red {{P}' | {Q}}}
  \and
  \inferrule* [lab=Equiv]{{{P} \scong {P}'} \andalso {{P}' \red {Q}'} \andalso {{Q}' \scong {Q}}}{{P} \red {Q}}
\end{mathpar}

\begin{eqnarray*}
  match_{\equiv} (\quotep{P},\quotep{Q}) & := & P \equiv Q \\
  match_{\dagger}(\quotep{P},\quotep{Q}) & := & \forall R. P|Q \red^{*} R => R \red^{*} 0 \\
  match_{K}(\quotep{P},\quotep{Q}) & := & K \mbox{ for some context } K
\end{eqnarray*}

$u?(x)P | u!\langle Q \rangle \red P\{\quotep{Q}/x\}$

%We write $\wred$ for $\red^*$, and $P\red$ if $\exists Q $ such that $ P \red Q$.
We write $P\red$ if $\exists Q $ such that $ P \red Q$ and $P\not\red$, otherwise.

\section{Replication}

As mentioned before, it is known that replication (and hence
recursion) can be implemented in a higher-order process algebra
\cite{SangiorgiWalker}. As our first example of calculation with the
machinery thus far presented we give the construction explicitly in
the {\rhoc}.

\begin{eqnarray}
	D_{x} & := & \prefix{x}{y}{(\binpar{\outputp{x}{y}}{@{y}})} \nonumber\\
	\bangp_{x}{P} & := & \binpar{{x}!\langle{\binpar{D_{x}}{P}}\rangle}{D_{x}} \nonumber
\end{eqnarray}

\begin{eqnarray}
	\bangp_{x}{P} & & \nonumber\\
	=
	& {x}!\langle{(\prefix{x}{y}{(\outputp{x}{y} | @{y})) | P}}\rangle 
	      | \prefix{x}{y}{(\outputp{x}{y} | @{y})} & \nonumber\\
	\red
	& (\outputp{x}{y} | @{y})\substn{\quotep{(\prefix{x}{y}{(@{y} | \outputp{x}{y})) | P}}}{y} & \nonumber\\
	=
	& \outputp{x}{\quotep{(\prefix{x}{y}{(\outputp{x}{y} | @{y})) | P}}}
	  | {(\prefix{x}{y}{(\outputp{x}{y} | @{y})) | P}} & \nonumber\\
	\red
	& \ldots & \nonumber\\
	\red^*
	& P | P | \ldots & \nonumber
\end{eqnarray}

Of course, this encoding, as an implementation, runs away, unfolding
$\bangp{P}$ eagerly. A lazier and more implementable replication
operator, restricted to input-guarded processes, may be obtained as follows.

\begin{eqnarray}
\bangp{\prefix{u}{v}{P}} 
	:= 
	\binpar{\lift{x}{\prefix{u}{v}{(\binpar{D(x)}{P})}}}{D(x)} \nonumber
\end{eqnarray}

\begin{remark}
  Note that the lazier definition still does not deal with summation
  or mixed summation (i.e. sums over input and output). The reader is
  invited to construct definitions of replication that deal with these
  features. 

  Further, the definitions are parameterized in a name, $x$. Can you,
  gentle reader, make a definition that eliminates this parameter and
  guarantees no accidental interaction between the replication
  machinery and the process being replicated -- i.e. no accidental
  sharing of names used by the process to get its work done and the
  name(s) used by the replication to effect copying. This latter
  revision of the definition of replication is crucial to obtaining
  the expected identity $!!P \sim !P$.
\end{remark}

\begin{remark}\label{rem:paradoxical_combinator}
  The reader familiar with the lambda calculus will have noticed the
  similarity between $D$ and the paradoxical combinator.

  [Ed. note: the existence of this seems to suggest we have to be more
  restrictive on the set of processes and names we admit if we are to
  support no-cloning.]
\end{remark}

\subsubsection{Bisimulation}

The computational dynamics gives rise to another kind of equivalence,
the equivalence of computational behavior. As previously mentioned
this is typically captured \emph{via} some form of bisimulation.

% The notion we use in this paper is weak barbed bisimulation
% \cite{milner91polyadicpi}.

The notion we use in this paper is derived from weak barbed
bisimulation \cite{milner91polyadicpi}. 

\begin{definition}
An \emph{observation relation}, $\downarrow_{\mathcal N}$, over a set
of names, $\mathcal N$, is the smallest relation satisfying the rules
below.

\infrule[Out-barb]{y \in {\mathcal N}, \; x \nameeq y}
		  {\outputp{x}{v} \downarrow_{\mathcal N} x}
\infrule[Par-barb]{\mbox{$P\downarrow_{\mathcal N} x$ or $Q\downarrow_{\mathcal N} x$}}
		  {\binpar{P}{Q} \downarrow_{\mathcal N} x}

We write $P \Downarrow_{\mathcal N} x$ if there is $Q$ such that 
$P \wred Q$ and $Q \downarrow_{\mathcal N} x$.
\end{definition}

\begin{definition}
%\label{def.bbisim}
An  ${\mathcal N}$-\emph{barbed bisimulation} over a set of names, ${\mathcal N}$, is a symmetric binary relation 
${\mathcal S}_{\mathcal N}$ between agents such that $P\rel{S}_{\mathcal N}Q$ implies:
\begin{enumerate}
\item If $P \red P'$ then $Q \wred Q'$ and $P'\rel{S}_{\mathcal N} Q'$.
\item If $P\downarrow_{\mathcal N} x$, then $Q\Downarrow_{\mathcal N} x$.
\end{enumerate}
$P$ is ${\mathcal N}$-barbed bisimilar to $Q$, written
$P \wbbisim_{\mathcal N} Q$, if $P \rel{S}_{\mathcal N} Q$ for some ${\mathcal N}$-barbed bisimulation ${\mathcal S}_{\mathcal N}$.
\end{definition}

$\mathcal{R} \subseteq \pi \times \pi$

$P \mathcal{R} Q => \forall P'. P \red P' \Rightarrow \exists Q'. Q \red Q', P' \mathcal{R} Q'$

$P \vdash x \Rightarrow Q \vdash x$

\begin{mathpar}
  \inferrule*[lab=Out-barb]{x \nameeq y}{{y}!\langle{Q}\rangle \vdash x}
  \and
  \inferrule*[lab=Par-barb]{\mbox{$P\vdash x$ or $Q\vdash x$}}{\binpar{P}{Q} \vdash x}
\end{mathpar}

\subsubsection{Contexts}

One of the principle advantages of computational calculi like the
$\pi$-calculus is a well-defined notion of context,
contextual-equivalence and a correlation between
contextual-equivalence and notions of bisimulation. The notion of
context allows the decomposition of a process into (sub-)process and
its syntactic environment, its context. Thus, a context may be
thought of as a process with a ``hole'' (written $\Box$) in it. The
application of a context $M$ to a process $P$, written $M[P]$, is
tantamount to filling the hole in $M$ with $P$. In this paper we do
not need the full weight of this theory, but do make use of the notion
of context in the proof the main theorem. 

\begin{mathpar}
  \inferrule* [lab=summation] {} {{M_{M},M_{N}} \bc \Box \;|\; x.M_{A} \;|\; M_{M}+M_{N}}
  \and
  \inferrule* [lab=agent] {} {{M_{A}} \bc (\vec{x})M_{P} \;| \; \clift{P_0,\ldots,M_{P},\ldots,P_N}}
  \and \\
  \inferrule* [lab=process] {} {{M_{P}} \bc M_{N} \;| \;P|M_{P} }
\end{mathpar} 

\begin{mathpar}
  \inferrule* [lab=sychronization] {} {M_{N} \bc \Box \;|\; x?M_{F} \;|\; x!M_{C}}
  \and
  \inferrule* [lab=abstraction] {} {{M_{F}} \bc (x)M_{P} }
  \and
  \inferrule* [lab=concretion] {} {{M_{C}} \bc \langle M_{P} \rangle }
  \and \\
  \inferrule* [lab=process] {} {{M_{P}} \bc M_{N} \;| \;P|M_{P} }
\end{mathpar}

\begin{definition}[contextual application] Given a context $M$, and
  process $P$, we define the \emph{contextual application}, $M[P] :=
  M\{P/\Box\}$. That is, the contextual application of M to P is the
  substitution of $P$ for $\Box$ in $M$.
\end{definition}

$\meaningof{-} : L \to \mathcal{P}(\pi)$

\begin{mathpar}
  \inferrule* [lab=collection] {} {\meaningof{true} = \pi, \and \meaningof{~E} = \pi \setminus \meaningof{E}, \and \meaningof{E_{1} \& E_{2}} = \meaningof{E_{1}} \cap \meaningof{E_{2}}}
\end{mathpar}

\begin{mathpar}
  \inferrule* [lab=structure] {} {\meaningof{0} = \{ P \in \pi | P \equiv 0 \}, \and \\ \meaningof{E_1 | E_2} = \{ P \in \pi | P \equiv P_{1} | P_{2}, P_{1} \in \meaningof{E_{1}}, P_{2} \in \meaningof{E_2}\} }
\end{mathpar}

\begin{mathpar}
 \inferrule* [lab=behavior] {} {\meaningof{\langle a?b \rangle E} = \{ P \in \pi | P \equiv Q | u?(y)P', \\ \and \\\\ \and \\ \;\;\; u \in \meaningof{a}, \forall z.P'\{z/y\} \in \meaningof{E\{z/b\}}\}, \and \\ \meaningof{a!E} = \{ P \in \pi | P \equiv Q | x!\langle P' \rangle, x \in \meaningof{a} P' \in \meaningof{E}\} }
\end{mathpar}

\begin{mathpar}
 \inferrule* [lab=nominal] {} {\meaningof{\quotep{E}} = \{ \quotep{P} \in \quotep{\pi} | P \in \meaningof{E} \}, \and \meaningof{\quotep{P}} = \{ \quotep{Q} \in \quotep{\pi} | P \equiv Q \} \and \\ \meaningof{@\quotep{E}} = \{ P \in \pi | P \equiv @x, x \in \meaningof{E} \}}
\end{mathpar}

\begin{eqnarray*}
  \\
  \meaningof{-} : TS \to ST
\end{eqnarray*}

\begin{eqnarray*}
  \\
  L : TS \to ST
\end{eqnarray*}

\begin{eqnarray*}
  \\
  P \models E \iff P \in \meaningof{E}
\end{eqnarray*}

\begin{eqnarray*}
  P \approx_{L} Q \iff \forall E \in L. P \models E \iff Q \models E
\end{eqnarray*}

\begin{eqnarray*}
  P \approx_{K} Q
\end{eqnarray*}

\begin{eqnarray*}
  P \approx Q
\end{eqnarray*}

$\approx_{K} = \approx = \approx_{L}$

\subsubsection{Contextual duality}

Note that contexts extend the quotation operation to a family of
operations from processes to names. Given a context, $M$, we can
define a \emph{nominal context}, $\quotep{M}$ by $\quotep{M}[P] :=
\quotep{M[P]}$. To foreshadow what is to come we observe that these
operations enjoy a duality with processes very much like the duality
between vectors and maps from vectors to scalars.

Further, because the calculus is essentially higher-order, we have a
correspondence between contexts and processes. More specifically,
given a name $x$ and a context $M$ we can construct $M^{*}_{x}$ such
that 

\begin{mathpar}
  M^{*}_{x} | \lift{x}{P} \red M[P]
\end{mathpar}

namely,

\begin{mathpar}
  M^{*}_{x} := x?(u).M[\dropn{u}]
\end{mathpar}

The dependence of $M^{*}_{x}$ on a name makes it an abstraction, 

\begin{mathpar}
  M^{*} := (x)x?(u).M[\dropn{u}]
\end{mathpar}

\subsection{Additional notation}

It will sometimes be convenient to denote the process a name
quotes. We already have the notation $x = \quotep{P}$, but it will be
convenient to introduce an alternate notation, $\procn{x}$, when we
want to emphasize the connection to the use of the name. Note that, by
virtue of name equivalence, $\quotep{\procn{x}} \nameeq x$; so, the
notation is consistent with previous definitions.

Further, because names have structure it is possible to effect
substitutions on the basis of that structure. This means we need to
upgrade our notation for substitutions, which we accomplish by
adapting comprehension notation. Thus,

\begin{mathpar}
  P\{ y / x : x \in S \}
\end{mathpar}

is interpreted to mean the process derived from P by replacing (in a
capture-avoiding manner) each occurrence of $x$ in $S$ by $y$. For example,

\begin{mathpar}
  P\{ \quotep{\procn{x}|\procn{x}} / x : x \in \freenames{P} \}
\end{mathpar}

will replace each (occurrence) of a free name $x$ in $P$ by
$\quotep{\procn{x}|\procn{x}}$.

Also, we will avail ourselves of the notation $x^{L}$ and $x^{R}$ to
denote injections of a name into disjoint copies of the name
space. There are numerous ways to accomplish this. One example can be
found in \cite{MeredithR05}. This notation overloads to vectors of
names: $\vec{x}^{\pi} := (x_{i}^{\pi} \; : \; 0 \leq i < |\vec{x}| )$ where $\pi \in \{L,R\}$.

We also use $P^{\Box} := P|\Box$.

In \cite{MeredithR05} an interpretation of the new operator is
given. It turns out that there are several possible interpretations
all enjoying the requisite algebraic properties of the operator (see
\cite{milner91polyadicpi}). We will therefore make liberal use of
$(\nu\; \vec{x})P$.

% subsection the_syntax_and_semantics_of_the_notation_system (end)   

\input{qm2pi.qmops} 

\input{qm2pi.sterngerlach} 

\input{qm2pi.metric} 

% section concurrent_process_calculi (end)

%\input{qm2pi.proofsketch}

% section proof sketch (end)

%\input{qm2pi.slviaknots} 

% section spatial logic via knots (end)

\input{qm2pi.conclusion}

% section conclusion (end)

%\input{qm2pi.dtcodes} 

% section wiring algorithm (end)

\input{qm2pi.ack} 

% section acknowledgments (end)

\newpage


\bibliographystyle{plain}   
\bibliography{../../biblios/main.bib}

\input{qm2pi.rhodetails}

\end{document}



\end{document}

 

%\documentclass[12pt]{llncs}
%\documentclass{jktr}

\usepackage[pdftex]{hyperref}                   
\usepackage {listings}
\usepackage {mathpartir}
\usepackage{bcprules}
%\usepackage{listings}
                       
\usepackage{graphicx} 
%\usepackage[margins=2.5cm,nohead,nofoot]{geometry}
%\usepackage{geometry}
\usepackage{amsfonts}
\usepackage{amstext}
\usepackage{latexsym}
\usepackage{amssymb}
\usepackage{color}


%\include{myPreamble}
\documentclass[12pt]{llncs}
%\documentclass{jktr}

\usepackage[pdftex]{hyperref}                   
\usepackage {listings}
\usepackage {mathpartir}
\usepackage{bcprules}
%\usepackage{listings}
                       
\usepackage{graphicx} 
%\usepackage[margins=2.5cm,nohead,nofoot]{geometry}
%\usepackage{geometry}
\usepackage{amsfonts}
\usepackage{amstext}
\usepackage{latexsym}
\usepackage{amssymb}
\usepackage{color}


%\include{myPreamble}
\include{qm2pi.local} 

%\ifpdf
%\usepackage[pdftex]{graphicx}
%\else
%\usepackage{graphicx}
%\fi

 % \ifpdf
%  \usepackage{pdfsync}
%  \if


%\title{Brief Article}
%\author{David F. Snyder}
%\author{L.G. Meredith}

%\address{Dept. of Math., Texas State University--San Marcos, San Marcos, TX 78666}
       
\pagestyle{empty}


\begin{document}

\lstset{language=[Objective]Caml,frame=shadowbox}

\input{qm2pi.front}

% section front matter (end)

\input{qm2pi.intro} 
 
% section introduction (end)

% \input{qm2pi.knotations} 

% section notation (end)

\input{qm2pi.process.calculi} 

% section concurrent_process_calculi_and_spatial_logics_ (end)
    
%\input{qm2pi.knots2pi} 

%\input{qm2pi.trefoil} 

%\input{qm2pi.mainthm} 

% subsection basic_interpretation (end)

%\input{qm2pi.rho.presentation} 
\subsection{The syntax and semantics of the notation system}\label{sub:the_syntax_and_semantics_of_the_notation_system} % (fold)

We now summarize a technical presentation of the calculus that
embodies our theory of dynamics. The typical presentation of such a
calculus follows the style of giving generators and relations on
them. The grammar, below, describing term constructors, freely
generates the set of processes, $\Proc$. This set is then quotiented
by a relation known as structural congruence and it is over this set
that the notion of dynamics is expressed. This presentation is
essentially that of \cite{MeredithR05} with the addition of
polyadicity and summation. For readability we have relegated some of
the technical subtleties to an appendix.

\subsubsection{Process grammar}\label{subsub:process_grammar}

\begin{mathpar}
  \inferrule* [lab=synchronization] {} {{M} \bc \pzero \;|\; x?F \;|\; x!C }
  \and
  \inferrule* [lab=abstraction] {} {{F} \bc (x)P}
  \and
  \inferrule* [lab=concretion] {} {{C} \bc \langle Q \rangle}
  \and
  \inferrule* [lab=process] {} {{P,Q} \bc M \;| \;P|Q \;|\; @{x}}
  \and
  \inferrule* [lab=name] {} {{x} \bc \quotep{P}}
\end{mathpar} 

Note that $\vec{x}$ (resp. $\vec{P}$) denotes a vector of names
(resp. processes) of length $|\vec{x}|$ (resp. $|\vec{P}|$). We adopt
the following useful abbreviations.

\begin{mathpar}
   x?(\vec{y}).P := x.(\vec{y})P \and  x\clift{\vec{P}} := x.\clift{\vec{P}}
   \and x!(y) := \lift{x}{\dropn{y}}
   \and \Pi_{i=0}^{n-1}P_i := P_0 | \ldots | P_{n-1}
\end{mathpar}

\subsubsection{Structural congruence}

\paragraph{Free and bound names and alpha-equivalence.} At the
core of structural equivalence is alpha-equivalence which identifies
process that are the same up to a change of variable. Formally, we
recognize the distinction between free and bound names. The free names
of a process, $\freenames{P}$, may be calculated recursively as
follows:

\begin{mathpar}
\freenames{\pzero} := \emptyset
  \and \\
  \freenames{x?(y).P} := \{ x \} \cup (\freenames{P} \setminus \{ y \})
  \and 
  \freenames{x!\langle P \rangle} := \{ x \} \cup \{ P \} 
  \and \\
  \freenames{P|Q} := \freenames{P} \cup \freenames{Q}
  \and \\
  \freenames{@{x}} := \{ x \}
\end{mathpar}

$\pi$
$\quotep{\pi}$

$\freenames{-} : \pi \to \mathcal{P}(\quotep{\pi})$

\begin{eqnarray*}
  \freenames{\pzero} & := & \emptyset \\
  \freenames{x?(y).P} & := & \{ x \} \cup (\freenames{P} \setminus \{ y \}) \\
  \freenames{x!\langle P \rangle} & := & \{ x \} \cup \{ P \} \\
  \freenames{P|Q} & := & \freenames{P} \cup \freenames{Q} \\
  \freenames{\dropn{x}} & := & \{ x \}
\end{eqnarray*}

The bound names of a process, $\boundnames{P}$, are those names occurring in $P$
that are not free. For example, in $x?(y).0$, the name $x$ is free, while $y$ is bound.

\begin{mathpar}
  \inferrule* [lab=monoidal-laws] {} { P|Q \equiv Q|P \and P|0 \equiv P \and P|(Q|R) \equiv (P|Q)|R }
\end{mathpar}

\begin{mathpar}
  \inferrule* [lab=alpha-equivalence] {} { (x)P \equiv (y)P\{y/x\} \and y \not\in \freenames{P} }
\end{mathpar}

\begin{definition}
Then two processes, $P,Q$, are alpha-equivalent if $P = Q\{\vec{y}/\vec{x}\}$ for
some $\vec{x} \in \boundnames{Q},\vec{y} \in \boundnames{P}$, where $Q\{\vec{y}/\vec{x}\}$
denotes the capture-avoiding substitution of $\vec{y}$ for $\vec{x}$ in $Q$.
\end{definition}

\begin{definition}
  The {\em structural congruence} \cite{SangiorgiWalker} , $\equiv$,
  between processes is the least congruence containing
  alpha-equivalence, satisfying the abelian monoid laws
  (associativity, commutativity and $\pzero$ as identity) for parallel
  composition $|$ and for summation $+$.
\end{definition}

\subsection{Name equivalence}

We take name equivalence, written $\nameeq$, to be the smallest
equivalence relation generated by the following rules.

\begin{mathpar}
\inferrule*[lab=Quote-drop]
{ }
{ \quotep{@{x}} \nameeq x }

\inferrule*[lab=Struct-equiv]
{ P \scong Q }
{ \quotep{P} \nameeq \quotep{Q} }
\end{mathpar}

The astute reader will have noticed that the mutual recursion of names
and processes imposes a mutual recursion on alpha-equivalence and
structural equivalence via name-equivalence. Fortunately, all of this
works out pleasantly and we may calculate in the natural way, free of
concern. The reader interested in the details is referred to the
appendix \ref{appendix:rho_details}.

\subsection{Substitution}

We use $\Proc$ for the set of processes, $\QProc$ for the set of
names, and $\id{\{}\vec{y} / \vec{x} \id{\}}$ to denote partial maps,
$s : \QProc \rightarrow \QProc$. A map, $s$ lifts, uniquely, to a map
on process terms, $\widehat{s} : \Proc \rightarrow \Proc$ by the
following equations.

\begin{mathpar}
  (0) \psubstp{Q}{P} := 0 \\
  (R \juxtap S) \psubstp{Q}{P}
  :=    
  (R)\psubstp{Q}{P} \juxtap (S) \psubstp{Q}{P} \\
  (x?(y).R) \psubstp{Q}{P}    
  :=    
  (x)\substp{Q}{P} (z)\concat( (R \psubstn{z}{y}) \psubstp{Q}{P} ) \\
  (\lift{x}{R}) \psubstp{Q}{P}  
  :=
  \lift{(x)\substp{Q}{P}}{ R \psubstp{Q}{P} } \\
%   (\dropn{x})  \psubstp{Q}{P}       
%   := 
%   \left\{ 
%     \begin{array}{ccc} 
%       \dropn{\quotep{Q}} & & x \nameeq \quotep{P} \\
%       \dropn{x} & & otherwise \\
%     \end{array}
%   \right. 
  (\dropn{x})  \psubstp{Q}{P}       
  := 
  \left\{ 
    \begin{array}{ccc} 
      Q & & x \nameeq \quotep{P} \\
      \dropn{x} & & otherwise \\
    \end{array}
  \right.
\end{mathpar}
 

where

\begin{eqnarray}
  (x)\id{\{} \lpquote Q \rpquote / \lpquote P \rpquote \id{\}}            = 
  \left\{ 
    \begin{array}{ccc}
      \lpquote Q \rpquote & & x \nameeq \lpquote P \rpquote \\
      x & & otherwise \\
    \end{array}
  \right. \nonumber
\end{eqnarray}

and $z$ is chosen distinct from $\quotep{P}$, $\quotep{Q}$, the free
names in $Q$, and all the names in $R$. Our $\alpha$-equivalence will
be built in the standard way from this substitution.

\begin{remark}\label{rem:no_self_referential_names}
  One consequence of these definitions is that $\forall P. \quotep{P}
  \not\in \freenames{P}$.
\end{remark}

\subsection{ Dynamic quote: an example }

Anticipating something of what's to come, consider applying the
substitution, $\widehat{\id{\{}u / z \id{\}}}$, to the following pair
of processes, $\lift{w}{y!(z)}$ and $w[ \lpquote y!(z) \rpquote ]$.

\begin{eqnarray}
	\lift{w}{y!(z)}\widehat{\id{\{}u / z \id{\}}}
		& = &
		\lift{w}{y!(u)} \nonumber\\
	w[ \lpquote y!(z) \rpquote ] \widehat{ \id{\{}u / z \id{\}} }
		& = &
		w[ \lpquote y!(z) \rpquote ] \nonumber
\end{eqnarray}

Because the body of the process between quotes is impervious to
substitution, we get radically different answers. In fact, by
examining the first process in an input context,
e.g. $x?(z).\lift{w}{y!(z)}$, we see that the process under the lift
operator may be shaped by prefixed inputs binding a name inside it. In
this sense, the lift operator will be seen as a way to dynamically
construct processes before reifying them as names.

Finally equipped with these standard features we can present the
dynamics of the calculus.

\subsubsection{Operational semantics} 

Finally, we introduce the computational dynamics. What marks these
algebras as distinct from other more traditionally studied algebraic
structures, e.g. vector spaces or polynomial rings, is the manner in
which dynamics is captured. In traditional structures, dynamics is typically
expressed through morphisms between such structures, as in linear maps
between vector spaces or morphisms between rings. In algebras
associated with the semantics of computation, the dynamics is
expressed as part of the algebraic structure itself, through a
reduction reduction relation typically denoted by $\red$. Below, we
give a recursive presentation of this relation for the calculus used
in the encoding.

$\red \subseteq \pi \times \pi$
$\red : \pi \to \mathcal{P}(\pi)$

\begin{mathpar}
  \inferrule* [lab=Comm] { \textsf{match}( x_{src}, x_{trgt} ) } { x_{trgt}?(y)P \; | \; x_{src}!\langle {Q} \rangle \red P\{\quotep{Q}/y}\} }
  \and \\
  \inferrule* [lab=Par] {{P} \red {P}'} {{{P} | {Q}} \red {{P}' | {Q}}}
  \and
  \inferrule* [lab=Equiv]{{{P} \scong {P}'} \andalso {{P}' \red {Q}'} \andalso {{Q}' \scong {Q}}}{{P} \red {Q}}
\end{mathpar}

\begin{eqnarray*}
  match_{\equiv} (\quotep{P},\quotep{Q}) & := & P \equiv Q \\
  match_{\dagger}(\quotep{P},\quotep{Q}) & := & \forall R. P|Q \red^{*} R => R \red^{*} 0 \\
  match_{K}(\quotep{P},\quotep{Q}) & := & K \mbox{ for some context } K
\end{eqnarray*}

$u?(x)P | u!\langle Q \rangle \red P\{\quotep{Q}/x\}$

%We write $\wred$ for $\red^*$, and $P\red$ if $\exists Q $ such that $ P \red Q$.
We write $P\red$ if $\exists Q $ such that $ P \red Q$ and $P\not\red$, otherwise.

\section{Replication}

As mentioned before, it is known that replication (and hence
recursion) can be implemented in a higher-order process algebra
\cite{SangiorgiWalker}. As our first example of calculation with the
machinery thus far presented we give the construction explicitly in
the {\rhoc}.

\begin{eqnarray}
	D_{x} & := & \prefix{x}{y}{(\binpar{\outputp{x}{y}}{@{y}})} \nonumber\\
	\bangp_{x}{P} & := & \binpar{{x}!\langle{\binpar{D_{x}}{P}}\rangle}{D_{x}} \nonumber
\end{eqnarray}

\begin{eqnarray}
	\bangp_{x}{P} & & \nonumber\\
	=
	& {x}!\langle{(\prefix{x}{y}{(\outputp{x}{y} | @{y})) | P}}\rangle 
	      | \prefix{x}{y}{(\outputp{x}{y} | @{y})} & \nonumber\\
	\red
	& (\outputp{x}{y} | @{y})\substn{\quotep{(\prefix{x}{y}{(@{y} | \outputp{x}{y})) | P}}}{y} & \nonumber\\
	=
	& \outputp{x}{\quotep{(\prefix{x}{y}{(\outputp{x}{y} | @{y})) | P}}}
	  | {(\prefix{x}{y}{(\outputp{x}{y} | @{y})) | P}} & \nonumber\\
	\red
	& \ldots & \nonumber\\
	\red^*
	& P | P | \ldots & \nonumber
\end{eqnarray}

Of course, this encoding, as an implementation, runs away, unfolding
$\bangp{P}$ eagerly. A lazier and more implementable replication
operator, restricted to input-guarded processes, may be obtained as follows.

\begin{eqnarray}
\bangp{\prefix{u}{v}{P}} 
	:= 
	\binpar{\lift{x}{\prefix{u}{v}{(\binpar{D(x)}{P})}}}{D(x)} \nonumber
\end{eqnarray}

\begin{remark}
  Note that the lazier definition still does not deal with summation
  or mixed summation (i.e. sums over input and output). The reader is
  invited to construct definitions of replication that deal with these
  features. 

  Further, the definitions are parameterized in a name, $x$. Can you,
  gentle reader, make a definition that eliminates this parameter and
  guarantees no accidental interaction between the replication
  machinery and the process being replicated -- i.e. no accidental
  sharing of names used by the process to get its work done and the
  name(s) used by the replication to effect copying. This latter
  revision of the definition of replication is crucial to obtaining
  the expected identity $!!P \sim !P$.
\end{remark}

\begin{remark}\label{rem:paradoxical_combinator}
  The reader familiar with the lambda calculus will have noticed the
  similarity between $D$ and the paradoxical combinator.

  [Ed. note: the existence of this seems to suggest we have to be more
  restrictive on the set of processes and names we admit if we are to
  support no-cloning.]
\end{remark}

\subsubsection{Bisimulation}

The computational dynamics gives rise to another kind of equivalence,
the equivalence of computational behavior. As previously mentioned
this is typically captured \emph{via} some form of bisimulation.

% The notion we use in this paper is weak barbed bisimulation
% \cite{milner91polyadicpi}.

The notion we use in this paper is derived from weak barbed
bisimulation \cite{milner91polyadicpi}. 

\begin{definition}
An \emph{observation relation}, $\downarrow_{\mathcal N}$, over a set
of names, $\mathcal N$, is the smallest relation satisfying the rules
below.

\infrule[Out-barb]{y \in {\mathcal N}, \; x \nameeq y}
		  {\outputp{x}{v} \downarrow_{\mathcal N} x}
\infrule[Par-barb]{\mbox{$P\downarrow_{\mathcal N} x$ or $Q\downarrow_{\mathcal N} x$}}
		  {\binpar{P}{Q} \downarrow_{\mathcal N} x}

We write $P \Downarrow_{\mathcal N} x$ if there is $Q$ such that 
$P \wred Q$ and $Q \downarrow_{\mathcal N} x$.
\end{definition}

\begin{definition}
%\label{def.bbisim}
An  ${\mathcal N}$-\emph{barbed bisimulation} over a set of names, ${\mathcal N}$, is a symmetric binary relation 
${\mathcal S}_{\mathcal N}$ between agents such that $P\rel{S}_{\mathcal N}Q$ implies:
\begin{enumerate}
\item If $P \red P'$ then $Q \wred Q'$ and $P'\rel{S}_{\mathcal N} Q'$.
\item If $P\downarrow_{\mathcal N} x$, then $Q\Downarrow_{\mathcal N} x$.
\end{enumerate}
$P$ is ${\mathcal N}$-barbed bisimilar to $Q$, written
$P \wbbisim_{\mathcal N} Q$, if $P \rel{S}_{\mathcal N} Q$ for some ${\mathcal N}$-barbed bisimulation ${\mathcal S}_{\mathcal N}$.
\end{definition}

$\mathcal{R} \subseteq \pi \times \pi$

$P \mathcal{R} Q => \forall P'. P \red P' \Rightarrow \exists Q'. Q \red Q', P' \mathcal{R} Q'$

$P \vdash x \Rightarrow Q \vdash x$

\begin{mathpar}
  \inferrule*[lab=Out-barb]{x \nameeq y}{{y}!\langle{Q}\rangle \vdash x}
  \and
  \inferrule*[lab=Par-barb]{\mbox{$P\vdash x$ or $Q\vdash x$}}{\binpar{P}{Q} \vdash x}
\end{mathpar}

\subsubsection{Contexts}

One of the principle advantages of computational calculi like the
$\pi$-calculus is a well-defined notion of context,
contextual-equivalence and a correlation between
contextual-equivalence and notions of bisimulation. The notion of
context allows the decomposition of a process into (sub-)process and
its syntactic environment, its context. Thus, a context may be
thought of as a process with a ``hole'' (written $\Box$) in it. The
application of a context $M$ to a process $P$, written $M[P]$, is
tantamount to filling the hole in $M$ with $P$. In this paper we do
not need the full weight of this theory, but do make use of the notion
of context in the proof the main theorem. 

\begin{mathpar}
  \inferrule* [lab=summation] {} {{M_{M},M_{N}} \bc \Box \;|\; x.M_{A} \;|\; M_{M}+M_{N}}
  \and
  \inferrule* [lab=agent] {} {{M_{A}} \bc (\vec{x})M_{P} \;| \; \clift{P_0,\ldots,M_{P},\ldots,P_N}}
  \and \\
  \inferrule* [lab=process] {} {{M_{P}} \bc M_{N} \;| \;P|M_{P} }
\end{mathpar} 

\begin{mathpar}
  \inferrule* [lab=sychronization] {} {M_{N} \bc \Box \;|\; x?M_{F} \;|\; x!M_{C}}
  \and
  \inferrule* [lab=abstraction] {} {{M_{F}} \bc (x)M_{P} }
  \and
  \inferrule* [lab=concretion] {} {{M_{C}} \bc \langle M_{P} \rangle }
  \and \\
  \inferrule* [lab=process] {} {{M_{P}} \bc M_{N} \;| \;P|M_{P} }
\end{mathpar}

\begin{definition}[contextual application] Given a context $M$, and
  process $P$, we define the \emph{contextual application}, $M[P] :=
  M\{P/\Box\}$. That is, the contextual application of M to P is the
  substitution of $P$ for $\Box$ in $M$.
\end{definition}

$\meaningof{-} : L \to \mathcal{P}(\pi)$

\begin{mathpar}
  \inferrule* [lab=collection] {} {\meaningof{true} = \pi, \and \meaningof{~E} = \pi \setminus \meaningof{E}, \and \meaningof{E_{1} \& E_{2}} = \meaningof{E_{1}} \cap \meaningof{E_{2}}}
\end{mathpar}

\begin{mathpar}
  \inferrule* [lab=structure] {} {\meaningof{0} = \{ P \in \pi | P \equiv 0 \}, \and \\ \meaningof{E_1 | E_2} = \{ P \in \pi | P \equiv P_{1} | P_{2}, P_{1} \in \meaningof{E_{1}}, P_{2} \in \meaningof{E_2}\} }
\end{mathpar}

\begin{mathpar}
 \inferrule* [lab=behavior] {} {\meaningof{\langle a?b \rangle E} = \{ P \in \pi | P \equiv Q | u?(y)P', \\ \and \\\\ \and \\ \;\;\; u \in \meaningof{a}, \forall z.P'\{z/y\} \in \meaningof{E\{z/b\}}\}, \and \\ \meaningof{a!E} = \{ P \in \pi | P \equiv Q | x!\langle P' \rangle, x \in \meaningof{a} P' \in \meaningof{E}\} }
\end{mathpar}

\begin{mathpar}
 \inferrule* [lab=nominal] {} {\meaningof{\quotep{E}} = \{ \quotep{P} \in \quotep{\pi} | P \in \meaningof{E} \}, \and \meaningof{\quotep{P}} = \{ \quotep{Q} \in \quotep{\pi} | P \equiv Q \} \and \\ \meaningof{@\quotep{E}} = \{ P \in \pi | P \equiv @x, x \in \meaningof{E} \}}
\end{mathpar}

\begin{eqnarray*}
  \\
  \meaningof{-} : TS \to ST
\end{eqnarray*}

\begin{eqnarray*}
  \\
  L : TS \to ST
\end{eqnarray*}

\begin{eqnarray*}
  \\
  P \models E \iff P \in \meaningof{E}
\end{eqnarray*}

\begin{eqnarray*}
  P \approx_{L} Q \iff \forall E \in L. P \models E \iff Q \models E
\end{eqnarray*}

\begin{eqnarray*}
  P \approx_{K} Q
\end{eqnarray*}

\begin{eqnarray*}
  P \approx Q
\end{eqnarray*}

$\approx_{K} = \approx = \approx_{L}$

\subsubsection{Contextual duality}

Note that contexts extend the quotation operation to a family of
operations from processes to names. Given a context, $M$, we can
define a \emph{nominal context}, $\quotep{M}$ by $\quotep{M}[P] :=
\quotep{M[P]}$. To foreshadow what is to come we observe that these
operations enjoy a duality with processes very much like the duality
between vectors and maps from vectors to scalars.

Further, because the calculus is essentially higher-order, we have a
correspondence between contexts and processes. More specifically,
given a name $x$ and a context $M$ we can construct $M^{*}_{x}$ such
that 

\begin{mathpar}
  M^{*}_{x} | \lift{x}{P} \red M[P]
\end{mathpar}

namely,

\begin{mathpar}
  M^{*}_{x} := x?(u).M[\dropn{u}]
\end{mathpar}

The dependence of $M^{*}_{x}$ on a name makes it an abstraction, 

\begin{mathpar}
  M^{*} := (x)x?(u).M[\dropn{u}]
\end{mathpar}

\subsection{Additional notation}

It will sometimes be convenient to denote the process a name
quotes. We already have the notation $x = \quotep{P}$, but it will be
convenient to introduce an alternate notation, $\procn{x}$, when we
want to emphasize the connection to the use of the name. Note that, by
virtue of name equivalence, $\quotep{\procn{x}} \nameeq x$; so, the
notation is consistent with previous definitions.

Further, because names have structure it is possible to effect
substitutions on the basis of that structure. This means we need to
upgrade our notation for substitutions, which we accomplish by
adapting comprehension notation. Thus,

\begin{mathpar}
  P\{ y / x : x \in S \}
\end{mathpar}

is interpreted to mean the process derived from P by replacing (in a
capture-avoiding manner) each occurrence of $x$ in $S$ by $y$. For example,

\begin{mathpar}
  P\{ \quotep{\procn{x}|\procn{x}} / x : x \in \freenames{P} \}
\end{mathpar}

will replace each (occurrence) of a free name $x$ in $P$ by
$\quotep{\procn{x}|\procn{x}}$.

Also, we will avail ourselves of the notation $x^{L}$ and $x^{R}$ to
denote injections of a name into disjoint copies of the name
space. There are numerous ways to accomplish this. One example can be
found in \cite{MeredithR05}. This notation overloads to vectors of
names: $\vec{x}^{\pi} := (x_{i}^{\pi} \; : \; 0 \leq i < |\vec{x}| )$ where $\pi \in \{L,R\}$.

We also use $P^{\Box} := P|\Box$.

In \cite{MeredithR05} an interpretation of the new operator is
given. It turns out that there are several possible interpretations
all enjoying the requisite algebraic properties of the operator (see
\cite{milner91polyadicpi}). We will therefore make liberal use of
$(\nu\; \vec{x})P$.

% subsection the_syntax_and_semantics_of_the_notation_system (end)   

\input{qm2pi.qmops} 

\input{qm2pi.sterngerlach} 

\input{qm2pi.metric} 

% section concurrent_process_calculi (end)

%\input{qm2pi.proofsketch}

% section proof sketch (end)

%\input{qm2pi.slviaknots} 

% section spatial logic via knots (end)

\input{qm2pi.conclusion}

% section conclusion (end)

%\input{qm2pi.dtcodes} 

% section wiring algorithm (end)

\input{qm2pi.ack} 

% section acknowledgments (end)

\newpage


\bibliographystyle{plain}   
\bibliography{../../biblios/main.bib}

\input{qm2pi.rhodetails}

\end{document}

 

%\ifpdf
%\usepackage[pdftex]{graphicx}
%\else
%\usepackage{graphicx}
%\fi

 % \ifpdf
%  \usepackage{pdfsync}
%  \if


%\title{Brief Article}
%\author{David F. Snyder}
%\author{L.G. Meredith}

%\address{Dept. of Math., Texas State University--San Marcos, San Marcos, TX 78666}
       
\pagestyle{empty}


\begin{document}

\lstset{language=[Objective]Caml,frame=shadowbox}

\documentclass[12pt]{llncs}
%\documentclass{jktr}

\usepackage[pdftex]{hyperref}                   
\usepackage {listings}
\usepackage {mathpartir}
\usepackage{bcprules}
%\usepackage{listings}
                       
\usepackage{graphicx} 
%\usepackage[margins=2.5cm,nohead,nofoot]{geometry}
%\usepackage{geometry}
\usepackage{amsfonts}
\usepackage{amstext}
\usepackage{latexsym}
\usepackage{amssymb}
\usepackage{color}


%\include{myPreamble}
\include{qm2pi.local} 

%\ifpdf
%\usepackage[pdftex]{graphicx}
%\else
%\usepackage{graphicx}
%\fi

 % \ifpdf
%  \usepackage{pdfsync}
%  \if


%\title{Brief Article}
%\author{David F. Snyder}
%\author{L.G. Meredith}

%\address{Dept. of Math., Texas State University--San Marcos, San Marcos, TX 78666}
       
\pagestyle{empty}


\begin{document}

\lstset{language=[Objective]Caml,frame=shadowbox}

\input{qm2pi.front}

% section front matter (end)

\input{qm2pi.intro} 
 
% section introduction (end)

% \input{qm2pi.knotations} 

% section notation (end)

\input{qm2pi.process.calculi} 

% section concurrent_process_calculi_and_spatial_logics_ (end)
    
%\input{qm2pi.knots2pi} 

%\input{qm2pi.trefoil} 

%\input{qm2pi.mainthm} 

% subsection basic_interpretation (end)

%\input{qm2pi.rho.presentation} 
\subsection{The syntax and semantics of the notation system}\label{sub:the_syntax_and_semantics_of_the_notation_system} % (fold)

We now summarize a technical presentation of the calculus that
embodies our theory of dynamics. The typical presentation of such a
calculus follows the style of giving generators and relations on
them. The grammar, below, describing term constructors, freely
generates the set of processes, $\Proc$. This set is then quotiented
by a relation known as structural congruence and it is over this set
that the notion of dynamics is expressed. This presentation is
essentially that of \cite{MeredithR05} with the addition of
polyadicity and summation. For readability we have relegated some of
the technical subtleties to an appendix.

\subsubsection{Process grammar}\label{subsub:process_grammar}

\begin{mathpar}
  \inferrule* [lab=synchronization] {} {{M} \bc \pzero \;|\; x?F \;|\; x!C }
  \and
  \inferrule* [lab=abstraction] {} {{F} \bc (x)P}
  \and
  \inferrule* [lab=concretion] {} {{C} \bc \langle Q \rangle}
  \and
  \inferrule* [lab=process] {} {{P,Q} \bc M \;| \;P|Q \;|\; @{x}}
  \and
  \inferrule* [lab=name] {} {{x} \bc \quotep{P}}
\end{mathpar} 

Note that $\vec{x}$ (resp. $\vec{P}$) denotes a vector of names
(resp. processes) of length $|\vec{x}|$ (resp. $|\vec{P}|$). We adopt
the following useful abbreviations.

\begin{mathpar}
   x?(\vec{y}).P := x.(\vec{y})P \and  x\clift{\vec{P}} := x.\clift{\vec{P}}
   \and x!(y) := \lift{x}{\dropn{y}}
   \and \Pi_{i=0}^{n-1}P_i := P_0 | \ldots | P_{n-1}
\end{mathpar}

\subsubsection{Structural congruence}

\paragraph{Free and bound names and alpha-equivalence.} At the
core of structural equivalence is alpha-equivalence which identifies
process that are the same up to a change of variable. Formally, we
recognize the distinction between free and bound names. The free names
of a process, $\freenames{P}$, may be calculated recursively as
follows:

\begin{mathpar}
\freenames{\pzero} := \emptyset
  \and \\
  \freenames{x?(y).P} := \{ x \} \cup (\freenames{P} \setminus \{ y \})
  \and 
  \freenames{x!\langle P \rangle} := \{ x \} \cup \{ P \} 
  \and \\
  \freenames{P|Q} := \freenames{P} \cup \freenames{Q}
  \and \\
  \freenames{@{x}} := \{ x \}
\end{mathpar}

$\pi$
$\quotep{\pi}$

$\freenames{-} : \pi \to \mathcal{P}(\quotep{\pi})$

\begin{eqnarray*}
  \freenames{\pzero} & := & \emptyset \\
  \freenames{x?(y).P} & := & \{ x \} \cup (\freenames{P} \setminus \{ y \}) \\
  \freenames{x!\langle P \rangle} & := & \{ x \} \cup \{ P \} \\
  \freenames{P|Q} & := & \freenames{P} \cup \freenames{Q} \\
  \freenames{\dropn{x}} & := & \{ x \}
\end{eqnarray*}

The bound names of a process, $\boundnames{P}$, are those names occurring in $P$
that are not free. For example, in $x?(y).0$, the name $x$ is free, while $y$ is bound.

\begin{mathpar}
  \inferrule* [lab=monoidal-laws] {} { P|Q \equiv Q|P \and P|0 \equiv P \and P|(Q|R) \equiv (P|Q)|R }
\end{mathpar}

\begin{mathpar}
  \inferrule* [lab=alpha-equivalence] {} { (x)P \equiv (y)P\{y/x\} \and y \not\in \freenames{P} }
\end{mathpar}

\begin{definition}
Then two processes, $P,Q$, are alpha-equivalent if $P = Q\{\vec{y}/\vec{x}\}$ for
some $\vec{x} \in \boundnames{Q},\vec{y} \in \boundnames{P}$, where $Q\{\vec{y}/\vec{x}\}$
denotes the capture-avoiding substitution of $\vec{y}$ for $\vec{x}$ in $Q$.
\end{definition}

\begin{definition}
  The {\em structural congruence} \cite{SangiorgiWalker} , $\equiv$,
  between processes is the least congruence containing
  alpha-equivalence, satisfying the abelian monoid laws
  (associativity, commutativity and $\pzero$ as identity) for parallel
  composition $|$ and for summation $+$.
\end{definition}

\subsection{Name equivalence}

We take name equivalence, written $\nameeq$, to be the smallest
equivalence relation generated by the following rules.

\begin{mathpar}
\inferrule*[lab=Quote-drop]
{ }
{ \quotep{@{x}} \nameeq x }

\inferrule*[lab=Struct-equiv]
{ P \scong Q }
{ \quotep{P} \nameeq \quotep{Q} }
\end{mathpar}

The astute reader will have noticed that the mutual recursion of names
and processes imposes a mutual recursion on alpha-equivalence and
structural equivalence via name-equivalence. Fortunately, all of this
works out pleasantly and we may calculate in the natural way, free of
concern. The reader interested in the details is referred to the
appendix \ref{appendix:rho_details}.

\subsection{Substitution}

We use $\Proc$ for the set of processes, $\QProc$ for the set of
names, and $\id{\{}\vec{y} / \vec{x} \id{\}}$ to denote partial maps,
$s : \QProc \rightarrow \QProc$. A map, $s$ lifts, uniquely, to a map
on process terms, $\widehat{s} : \Proc \rightarrow \Proc$ by the
following equations.

\begin{mathpar}
  (0) \psubstp{Q}{P} := 0 \\
  (R \juxtap S) \psubstp{Q}{P}
  :=    
  (R)\psubstp{Q}{P} \juxtap (S) \psubstp{Q}{P} \\
  (x?(y).R) \psubstp{Q}{P}    
  :=    
  (x)\substp{Q}{P} (z)\concat( (R \psubstn{z}{y}) \psubstp{Q}{P} ) \\
  (\lift{x}{R}) \psubstp{Q}{P}  
  :=
  \lift{(x)\substp{Q}{P}}{ R \psubstp{Q}{P} } \\
%   (\dropn{x})  \psubstp{Q}{P}       
%   := 
%   \left\{ 
%     \begin{array}{ccc} 
%       \dropn{\quotep{Q}} & & x \nameeq \quotep{P} \\
%       \dropn{x} & & otherwise \\
%     \end{array}
%   \right. 
  (\dropn{x})  \psubstp{Q}{P}       
  := 
  \left\{ 
    \begin{array}{ccc} 
      Q & & x \nameeq \quotep{P} \\
      \dropn{x} & & otherwise \\
    \end{array}
  \right.
\end{mathpar}
 

where

\begin{eqnarray}
  (x)\id{\{} \lpquote Q \rpquote / \lpquote P \rpquote \id{\}}            = 
  \left\{ 
    \begin{array}{ccc}
      \lpquote Q \rpquote & & x \nameeq \lpquote P \rpquote \\
      x & & otherwise \\
    \end{array}
  \right. \nonumber
\end{eqnarray}

and $z$ is chosen distinct from $\quotep{P}$, $\quotep{Q}$, the free
names in $Q$, and all the names in $R$. Our $\alpha$-equivalence will
be built in the standard way from this substitution.

\begin{remark}\label{rem:no_self_referential_names}
  One consequence of these definitions is that $\forall P. \quotep{P}
  \not\in \freenames{P}$.
\end{remark}

\subsection{ Dynamic quote: an example }

Anticipating something of what's to come, consider applying the
substitution, $\widehat{\id{\{}u / z \id{\}}}$, to the following pair
of processes, $\lift{w}{y!(z)}$ and $w[ \lpquote y!(z) \rpquote ]$.

\begin{eqnarray}
	\lift{w}{y!(z)}\widehat{\id{\{}u / z \id{\}}}
		& = &
		\lift{w}{y!(u)} \nonumber\\
	w[ \lpquote y!(z) \rpquote ] \widehat{ \id{\{}u / z \id{\}} }
		& = &
		w[ \lpquote y!(z) \rpquote ] \nonumber
\end{eqnarray}

Because the body of the process between quotes is impervious to
substitution, we get radically different answers. In fact, by
examining the first process in an input context,
e.g. $x?(z).\lift{w}{y!(z)}$, we see that the process under the lift
operator may be shaped by prefixed inputs binding a name inside it. In
this sense, the lift operator will be seen as a way to dynamically
construct processes before reifying them as names.

Finally equipped with these standard features we can present the
dynamics of the calculus.

\subsubsection{Operational semantics} 

Finally, we introduce the computational dynamics. What marks these
algebras as distinct from other more traditionally studied algebraic
structures, e.g. vector spaces or polynomial rings, is the manner in
which dynamics is captured. In traditional structures, dynamics is typically
expressed through morphisms between such structures, as in linear maps
between vector spaces or morphisms between rings. In algebras
associated with the semantics of computation, the dynamics is
expressed as part of the algebraic structure itself, through a
reduction reduction relation typically denoted by $\red$. Below, we
give a recursive presentation of this relation for the calculus used
in the encoding.

$\red \subseteq \pi \times \pi$
$\red : \pi \to \mathcal{P}(\pi)$

\begin{mathpar}
  \inferrule* [lab=Comm] { \textsf{match}( x_{src}, x_{trgt} ) } { x_{trgt}?(y)P \; | \; x_{src}!\langle {Q} \rangle \red P\{\quotep{Q}/y}\} }
  \and \\
  \inferrule* [lab=Par] {{P} \red {P}'} {{{P} | {Q}} \red {{P}' | {Q}}}
  \and
  \inferrule* [lab=Equiv]{{{P} \scong {P}'} \andalso {{P}' \red {Q}'} \andalso {{Q}' \scong {Q}}}{{P} \red {Q}}
\end{mathpar}

\begin{eqnarray*}
  match_{\equiv} (\quotep{P},\quotep{Q}) & := & P \equiv Q \\
  match_{\dagger}(\quotep{P},\quotep{Q}) & := & \forall R. P|Q \red^{*} R => R \red^{*} 0 \\
  match_{K}(\quotep{P},\quotep{Q}) & := & K \mbox{ for some context } K
\end{eqnarray*}

$u?(x)P | u!\langle Q \rangle \red P\{\quotep{Q}/x\}$

%We write $\wred$ for $\red^*$, and $P\red$ if $\exists Q $ such that $ P \red Q$.
We write $P\red$ if $\exists Q $ such that $ P \red Q$ and $P\not\red$, otherwise.

\section{Replication}

As mentioned before, it is known that replication (and hence
recursion) can be implemented in a higher-order process algebra
\cite{SangiorgiWalker}. As our first example of calculation with the
machinery thus far presented we give the construction explicitly in
the {\rhoc}.

\begin{eqnarray}
	D_{x} & := & \prefix{x}{y}{(\binpar{\outputp{x}{y}}{@{y}})} \nonumber\\
	\bangp_{x}{P} & := & \binpar{{x}!\langle{\binpar{D_{x}}{P}}\rangle}{D_{x}} \nonumber
\end{eqnarray}

\begin{eqnarray}
	\bangp_{x}{P} & & \nonumber\\
	=
	& {x}!\langle{(\prefix{x}{y}{(\outputp{x}{y} | @{y})) | P}}\rangle 
	      | \prefix{x}{y}{(\outputp{x}{y} | @{y})} & \nonumber\\
	\red
	& (\outputp{x}{y} | @{y})\substn{\quotep{(\prefix{x}{y}{(@{y} | \outputp{x}{y})) | P}}}{y} & \nonumber\\
	=
	& \outputp{x}{\quotep{(\prefix{x}{y}{(\outputp{x}{y} | @{y})) | P}}}
	  | {(\prefix{x}{y}{(\outputp{x}{y} | @{y})) | P}} & \nonumber\\
	\red
	& \ldots & \nonumber\\
	\red^*
	& P | P | \ldots & \nonumber
\end{eqnarray}

Of course, this encoding, as an implementation, runs away, unfolding
$\bangp{P}$ eagerly. A lazier and more implementable replication
operator, restricted to input-guarded processes, may be obtained as follows.

\begin{eqnarray}
\bangp{\prefix{u}{v}{P}} 
	:= 
	\binpar{\lift{x}{\prefix{u}{v}{(\binpar{D(x)}{P})}}}{D(x)} \nonumber
\end{eqnarray}

\begin{remark}
  Note that the lazier definition still does not deal with summation
  or mixed summation (i.e. sums over input and output). The reader is
  invited to construct definitions of replication that deal with these
  features. 

  Further, the definitions are parameterized in a name, $x$. Can you,
  gentle reader, make a definition that eliminates this parameter and
  guarantees no accidental interaction between the replication
  machinery and the process being replicated -- i.e. no accidental
  sharing of names used by the process to get its work done and the
  name(s) used by the replication to effect copying. This latter
  revision of the definition of replication is crucial to obtaining
  the expected identity $!!P \sim !P$.
\end{remark}

\begin{remark}\label{rem:paradoxical_combinator}
  The reader familiar with the lambda calculus will have noticed the
  similarity between $D$ and the paradoxical combinator.

  [Ed. note: the existence of this seems to suggest we have to be more
  restrictive on the set of processes and names we admit if we are to
  support no-cloning.]
\end{remark}

\subsubsection{Bisimulation}

The computational dynamics gives rise to another kind of equivalence,
the equivalence of computational behavior. As previously mentioned
this is typically captured \emph{via} some form of bisimulation.

% The notion we use in this paper is weak barbed bisimulation
% \cite{milner91polyadicpi}.

The notion we use in this paper is derived from weak barbed
bisimulation \cite{milner91polyadicpi}. 

\begin{definition}
An \emph{observation relation}, $\downarrow_{\mathcal N}$, over a set
of names, $\mathcal N$, is the smallest relation satisfying the rules
below.

\infrule[Out-barb]{y \in {\mathcal N}, \; x \nameeq y}
		  {\outputp{x}{v} \downarrow_{\mathcal N} x}
\infrule[Par-barb]{\mbox{$P\downarrow_{\mathcal N} x$ or $Q\downarrow_{\mathcal N} x$}}
		  {\binpar{P}{Q} \downarrow_{\mathcal N} x}

We write $P \Downarrow_{\mathcal N} x$ if there is $Q$ such that 
$P \wred Q$ and $Q \downarrow_{\mathcal N} x$.
\end{definition}

\begin{definition}
%\label{def.bbisim}
An  ${\mathcal N}$-\emph{barbed bisimulation} over a set of names, ${\mathcal N}$, is a symmetric binary relation 
${\mathcal S}_{\mathcal N}$ between agents such that $P\rel{S}_{\mathcal N}Q$ implies:
\begin{enumerate}
\item If $P \red P'$ then $Q \wred Q'$ and $P'\rel{S}_{\mathcal N} Q'$.
\item If $P\downarrow_{\mathcal N} x$, then $Q\Downarrow_{\mathcal N} x$.
\end{enumerate}
$P$ is ${\mathcal N}$-barbed bisimilar to $Q$, written
$P \wbbisim_{\mathcal N} Q$, if $P \rel{S}_{\mathcal N} Q$ for some ${\mathcal N}$-barbed bisimulation ${\mathcal S}_{\mathcal N}$.
\end{definition}

$\mathcal{R} \subseteq \pi \times \pi$

$P \mathcal{R} Q => \forall P'. P \red P' \Rightarrow \exists Q'. Q \red Q', P' \mathcal{R} Q'$

$P \vdash x \Rightarrow Q \vdash x$

\begin{mathpar}
  \inferrule*[lab=Out-barb]{x \nameeq y}{{y}!\langle{Q}\rangle \vdash x}
  \and
  \inferrule*[lab=Par-barb]{\mbox{$P\vdash x$ or $Q\vdash x$}}{\binpar{P}{Q} \vdash x}
\end{mathpar}

\subsubsection{Contexts}

One of the principle advantages of computational calculi like the
$\pi$-calculus is a well-defined notion of context,
contextual-equivalence and a correlation between
contextual-equivalence and notions of bisimulation. The notion of
context allows the decomposition of a process into (sub-)process and
its syntactic environment, its context. Thus, a context may be
thought of as a process with a ``hole'' (written $\Box$) in it. The
application of a context $M$ to a process $P$, written $M[P]$, is
tantamount to filling the hole in $M$ with $P$. In this paper we do
not need the full weight of this theory, but do make use of the notion
of context in the proof the main theorem. 

\begin{mathpar}
  \inferrule* [lab=summation] {} {{M_{M},M_{N}} \bc \Box \;|\; x.M_{A} \;|\; M_{M}+M_{N}}
  \and
  \inferrule* [lab=agent] {} {{M_{A}} \bc (\vec{x})M_{P} \;| \; \clift{P_0,\ldots,M_{P},\ldots,P_N}}
  \and \\
  \inferrule* [lab=process] {} {{M_{P}} \bc M_{N} \;| \;P|M_{P} }
\end{mathpar} 

\begin{mathpar}
  \inferrule* [lab=sychronization] {} {M_{N} \bc \Box \;|\; x?M_{F} \;|\; x!M_{C}}
  \and
  \inferrule* [lab=abstraction] {} {{M_{F}} \bc (x)M_{P} }
  \and
  \inferrule* [lab=concretion] {} {{M_{C}} \bc \langle M_{P} \rangle }
  \and \\
  \inferrule* [lab=process] {} {{M_{P}} \bc M_{N} \;| \;P|M_{P} }
\end{mathpar}

\begin{definition}[contextual application] Given a context $M$, and
  process $P$, we define the \emph{contextual application}, $M[P] :=
  M\{P/\Box\}$. That is, the contextual application of M to P is the
  substitution of $P$ for $\Box$ in $M$.
\end{definition}

$\meaningof{-} : L \to \mathcal{P}(\pi)$

\begin{mathpar}
  \inferrule* [lab=collection] {} {\meaningof{true} = \pi, \and \meaningof{~E} = \pi \setminus \meaningof{E}, \and \meaningof{E_{1} \& E_{2}} = \meaningof{E_{1}} \cap \meaningof{E_{2}}}
\end{mathpar}

\begin{mathpar}
  \inferrule* [lab=structure] {} {\meaningof{0} = \{ P \in \pi | P \equiv 0 \}, \and \\ \meaningof{E_1 | E_2} = \{ P \in \pi | P \equiv P_{1} | P_{2}, P_{1} \in \meaningof{E_{1}}, P_{2} \in \meaningof{E_2}\} }
\end{mathpar}

\begin{mathpar}
 \inferrule* [lab=behavior] {} {\meaningof{\langle a?b \rangle E} = \{ P \in \pi | P \equiv Q | u?(y)P', \\ \and \\\\ \and \\ \;\;\; u \in \meaningof{a}, \forall z.P'\{z/y\} \in \meaningof{E\{z/b\}}\}, \and \\ \meaningof{a!E} = \{ P \in \pi | P \equiv Q | x!\langle P' \rangle, x \in \meaningof{a} P' \in \meaningof{E}\} }
\end{mathpar}

\begin{mathpar}
 \inferrule* [lab=nominal] {} {\meaningof{\quotep{E}} = \{ \quotep{P} \in \quotep{\pi} | P \in \meaningof{E} \}, \and \meaningof{\quotep{P}} = \{ \quotep{Q} \in \quotep{\pi} | P \equiv Q \} \and \\ \meaningof{@\quotep{E}} = \{ P \in \pi | P \equiv @x, x \in \meaningof{E} \}}
\end{mathpar}

\begin{eqnarray*}
  \\
  \meaningof{-} : TS \to ST
\end{eqnarray*}

\begin{eqnarray*}
  \\
  L : TS \to ST
\end{eqnarray*}

\begin{eqnarray*}
  \\
  P \models E \iff P \in \meaningof{E}
\end{eqnarray*}

\begin{eqnarray*}
  P \approx_{L} Q \iff \forall E \in L. P \models E \iff Q \models E
\end{eqnarray*}

\begin{eqnarray*}
  P \approx_{K} Q
\end{eqnarray*}

\begin{eqnarray*}
  P \approx Q
\end{eqnarray*}

$\approx_{K} = \approx = \approx_{L}$

\subsubsection{Contextual duality}

Note that contexts extend the quotation operation to a family of
operations from processes to names. Given a context, $M$, we can
define a \emph{nominal context}, $\quotep{M}$ by $\quotep{M}[P] :=
\quotep{M[P]}$. To foreshadow what is to come we observe that these
operations enjoy a duality with processes very much like the duality
between vectors and maps from vectors to scalars.

Further, because the calculus is essentially higher-order, we have a
correspondence between contexts and processes. More specifically,
given a name $x$ and a context $M$ we can construct $M^{*}_{x}$ such
that 

\begin{mathpar}
  M^{*}_{x} | \lift{x}{P} \red M[P]
\end{mathpar}

namely,

\begin{mathpar}
  M^{*}_{x} := x?(u).M[\dropn{u}]
\end{mathpar}

The dependence of $M^{*}_{x}$ on a name makes it an abstraction, 

\begin{mathpar}
  M^{*} := (x)x?(u).M[\dropn{u}]
\end{mathpar}

\subsection{Additional notation}

It will sometimes be convenient to denote the process a name
quotes. We already have the notation $x = \quotep{P}$, but it will be
convenient to introduce an alternate notation, $\procn{x}$, when we
want to emphasize the connection to the use of the name. Note that, by
virtue of name equivalence, $\quotep{\procn{x}} \nameeq x$; so, the
notation is consistent with previous definitions.

Further, because names have structure it is possible to effect
substitutions on the basis of that structure. This means we need to
upgrade our notation for substitutions, which we accomplish by
adapting comprehension notation. Thus,

\begin{mathpar}
  P\{ y / x : x \in S \}
\end{mathpar}

is interpreted to mean the process derived from P by replacing (in a
capture-avoiding manner) each occurrence of $x$ in $S$ by $y$. For example,

\begin{mathpar}
  P\{ \quotep{\procn{x}|\procn{x}} / x : x \in \freenames{P} \}
\end{mathpar}

will replace each (occurrence) of a free name $x$ in $P$ by
$\quotep{\procn{x}|\procn{x}}$.

Also, we will avail ourselves of the notation $x^{L}$ and $x^{R}$ to
denote injections of a name into disjoint copies of the name
space. There are numerous ways to accomplish this. One example can be
found in \cite{MeredithR05}. This notation overloads to vectors of
names: $\vec{x}^{\pi} := (x_{i}^{\pi} \; : \; 0 \leq i < |\vec{x}| )$ where $\pi \in \{L,R\}$.

We also use $P^{\Box} := P|\Box$.

In \cite{MeredithR05} an interpretation of the new operator is
given. It turns out that there are several possible interpretations
all enjoying the requisite algebraic properties of the operator (see
\cite{milner91polyadicpi}). We will therefore make liberal use of
$(\nu\; \vec{x})P$.

% subsection the_syntax_and_semantics_of_the_notation_system (end)   

\input{qm2pi.qmops} 

\input{qm2pi.sterngerlach} 

\input{qm2pi.metric} 

% section concurrent_process_calculi (end)

%\input{qm2pi.proofsketch}

% section proof sketch (end)

%\input{qm2pi.slviaknots} 

% section spatial logic via knots (end)

\input{qm2pi.conclusion}

% section conclusion (end)

%\input{qm2pi.dtcodes} 

% section wiring algorithm (end)

\input{qm2pi.ack} 

% section acknowledgments (end)

\newpage


\bibliographystyle{plain}   
\bibliography{../../biblios/main.bib}

\input{qm2pi.rhodetails}

\end{document}



% section front matter (end)

\section{Introduction}\label{sec:introduction} % (fold)
In this draft of the material i am going to have to dispense with the
usual writing conventions adopted in papers on these topics. i'm going
to have adopt whatever tone i need at the time i'm writing up the
calculations. Sometimes this may be very conversational; others it may
be the barest mathematical grunts; others still it may be that i have
lifted text from one of my other papers because the exposition of some
point was better said there. i hope that my readers are not unduly put
out by this decision. i'm not doing this to flout convention or be
rebellious. i find these calculations very technically challenging. To
keep everything going technically, something has to give; i have to
let go of some cognitive burden. So, the academic writing style --
with all of its trade-offs in terms of facilitating technical
communication -- is what i'm letting go of. Perhaps subsequent drafts
can be tightened and polished, but for now, i'm going to speak as if
we were sitting together in a coffee shop with a laptop, wifi and a
pad of paper and a pencil.

So, here's what i have to say. We -- you and i, comfortably ensconced
in our coffee shop and well-equipped with our tools -- can realize and
carry out the calculations of quantum mechanics over a very different
formal theory of dynamics, a formal theory of dynamics that
corresponds to a theory of concurrent computation with
\emph{reflection}. It has the advantage that the underlying theory is
already `quantized', but supports analogues all of the continuuous
operations. Strikingly, this underlying theory has recently been
connected with a notion of metric that we can show, by calculating
together, coincides with the metric induced by the inner product.

There are a lot of reasons why you might be interested in seeing
calculations of this form. Here's why i'm interested. For the past
several centuries there has been no competitor to the ``Newtonian''
account of dynamics. As a result the predominant share of accounts of
dynamical systems and situations have had to be formulated in terms of
the Newtonian machinery. i view this as an intellectually dangerous
position to occupy. Everything, despite it's intrinsic shape, turns
into a nail to be hit with this hammer. Recently, however, the theory
of computation has matured to the point where we have candidates for
theories of dynamics that offer very different perspective on
reasoning about dynamical systems and situations. Testing these
candidates against very successful accounts of dynamical situations,
like quantum mechanics, is going to give us some sense of how mature
they are and some measure of the quality of these accounts of
dynamics.

\subsection{Summary of contributions and outline of paper}

So, we're going to develop an interpretation of the operations of
quantum mechanics normally interpreted by Hilbert spaces and
operators. We're going to do this over a theory of computation. Note
that this is very different than the usual quantum computation program
which develops notions of computation over quantum mechanics. Rather,
we are developing a story that aligns with Wheeler's slogan: It from
Bit. To do this we will first provide an account of the theory of
computation at play here. Then we will dive into a calculation-driven
interpretation of the operations of quantum mechanics.

The reason we take this approach is that -- until very recently --
there hasn't been an axiomatic account of quantum mechanics. As a
result there has been no sharp delineation of the mathematical theory
supporting interpretation of the physical theory and the physical
theory, itself. So, ambient features of the maths are free to be
exploited (or supressed) without a real accounting of their physical
relevance. There is no sharp statement ``here's the physical theory''
qua \emph{theory} and ``here's the mathematical interpretation''
enabling a judgment of how faithful the interpretation is -- apart
from experimental observation. When there is an axiomatic account we
can judge how well a given mathematical formalism supports an
interpretation of the axioms, independent of
experimentation. Likewise, we can judge how well we have captured our
physical evidence and experience with our axiomatics, independent of
any specific mathematical implementation, with accidental detail that
may or may not have physical significance. 

In lieu of a fully fleshed out and vetted axiomatic account of quantum
mechanics, interpreting the operational notions in service of modeling
physical systems will have to suffice. In other words, we are not in
the business of providing a model of Hilbert spaces and operators. We
are in the business of providing a model of quantum mechanics because
we are motivated by testing our notions of dynamics against physical
theory; and, the predictive calculations of the physical theory must
serve as the best formulation -- shy of a fully fleshed out axiomatic
account -- of the physical theory itself (as they have for scientific
theories since time immemorial). Put another way, despite a
whole-hearted commitment to an It-from-Bit ontology, we are firmly
aligned with the shut-up-and-calculate camp as the best way to obtain
results either from the physical perspective or as a quality assurance
measure of our fledgling theory of dynamics.

In detail, we present a reflective process calculus. Then we develop
intuitive correspondences between the notions available in this
calculus and the usual physical notions supporting quantum mechanical
calculations. Thus, 

\begin{table}[htp]
  \center{
    \fbox{
      \begin{tabular}{c|c}
        quantum mechanics & process calculus \\
        \hline
        scalar & name \\
        state vector & process \\
        dual & contextual duals \\
        matrix & formal sums of process-context-dual pairs \\
        orthogonality & process annihilation \\
        inner product & execution-formula + quoting
      \end{tabular}
    }
  }
  \caption{QM - process calculi correspondences}
\end{table}

Then we tighten up these intuitions to operational definitions. We
employ the Dirac notation as the best proxy we can find for an
abstract syntax of the quantum mechanical notions. The definitions we
develop put us in contact with equational constraints coming from the
theory that we demonstrate the definitions and calculations satisfy.

This puts us in a position to shut up and calculate for the
Stern-Gerlach experimental set up, showing how these predictive
calculations become calculations on processes in our theory of a
reflective process calculus.

Penultimately, we demonstrate that the notion of metric coming from
the inner product coincides with the notion of metric available from
the theory of bisimulation. This demonstration gives us the right to
think of space as arising from behavior. Finally, we consider where we
might go from the new vantage point we have obtained.

% section introduction (end) 
 
% section introduction (end)

% \documentclass[12pt]{llncs}
%\documentclass{jktr}

\usepackage[pdftex]{hyperref}                   
\usepackage {listings}
\usepackage {mathpartir}
\usepackage{bcprules}
%\usepackage{listings}
                       
\usepackage{graphicx} 
%\usepackage[margins=2.5cm,nohead,nofoot]{geometry}
%\usepackage{geometry}
\usepackage{amsfonts}
\usepackage{amstext}
\usepackage{latexsym}
\usepackage{amssymb}
\usepackage{color}


%\include{myPreamble}
\include{qm2pi.local} 

%\ifpdf
%\usepackage[pdftex]{graphicx}
%\else
%\usepackage{graphicx}
%\fi

 % \ifpdf
%  \usepackage{pdfsync}
%  \if


%\title{Brief Article}
%\author{David F. Snyder}
%\author{L.G. Meredith}

%\address{Dept. of Math., Texas State University--San Marcos, San Marcos, TX 78666}
       
\pagestyle{empty}


\begin{document}

\lstset{language=[Objective]Caml,frame=shadowbox}

\input{qm2pi.front}

% section front matter (end)

\input{qm2pi.intro} 
 
% section introduction (end)

% \input{qm2pi.knotations} 

% section notation (end)

\input{qm2pi.process.calculi} 

% section concurrent_process_calculi_and_spatial_logics_ (end)
    
%\input{qm2pi.knots2pi} 

%\input{qm2pi.trefoil} 

%\input{qm2pi.mainthm} 

% subsection basic_interpretation (end)

%\input{qm2pi.rho.presentation} 
\subsection{The syntax and semantics of the notation system}\label{sub:the_syntax_and_semantics_of_the_notation_system} % (fold)

We now summarize a technical presentation of the calculus that
embodies our theory of dynamics. The typical presentation of such a
calculus follows the style of giving generators and relations on
them. The grammar, below, describing term constructors, freely
generates the set of processes, $\Proc$. This set is then quotiented
by a relation known as structural congruence and it is over this set
that the notion of dynamics is expressed. This presentation is
essentially that of \cite{MeredithR05} with the addition of
polyadicity and summation. For readability we have relegated some of
the technical subtleties to an appendix.

\subsubsection{Process grammar}\label{subsub:process_grammar}

\begin{mathpar}
  \inferrule* [lab=synchronization] {} {{M} \bc \pzero \;|\; x?F \;|\; x!C }
  \and
  \inferrule* [lab=abstraction] {} {{F} \bc (x)P}
  \and
  \inferrule* [lab=concretion] {} {{C} \bc \langle Q \rangle}
  \and
  \inferrule* [lab=process] {} {{P,Q} \bc M \;| \;P|Q \;|\; @{x}}
  \and
  \inferrule* [lab=name] {} {{x} \bc \quotep{P}}
\end{mathpar} 

Note that $\vec{x}$ (resp. $\vec{P}$) denotes a vector of names
(resp. processes) of length $|\vec{x}|$ (resp. $|\vec{P}|$). We adopt
the following useful abbreviations.

\begin{mathpar}
   x?(\vec{y}).P := x.(\vec{y})P \and  x\clift{\vec{P}} := x.\clift{\vec{P}}
   \and x!(y) := \lift{x}{\dropn{y}}
   \and \Pi_{i=0}^{n-1}P_i := P_0 | \ldots | P_{n-1}
\end{mathpar}

\subsubsection{Structural congruence}

\paragraph{Free and bound names and alpha-equivalence.} At the
core of structural equivalence is alpha-equivalence which identifies
process that are the same up to a change of variable. Formally, we
recognize the distinction between free and bound names. The free names
of a process, $\freenames{P}$, may be calculated recursively as
follows:

\begin{mathpar}
\freenames{\pzero} := \emptyset
  \and \\
  \freenames{x?(y).P} := \{ x \} \cup (\freenames{P} \setminus \{ y \})
  \and 
  \freenames{x!\langle P \rangle} := \{ x \} \cup \{ P \} 
  \and \\
  \freenames{P|Q} := \freenames{P} \cup \freenames{Q}
  \and \\
  \freenames{@{x}} := \{ x \}
\end{mathpar}

$\pi$
$\quotep{\pi}$

$\freenames{-} : \pi \to \mathcal{P}(\quotep{\pi})$

\begin{eqnarray*}
  \freenames{\pzero} & := & \emptyset \\
  \freenames{x?(y).P} & := & \{ x \} \cup (\freenames{P} \setminus \{ y \}) \\
  \freenames{x!\langle P \rangle} & := & \{ x \} \cup \{ P \} \\
  \freenames{P|Q} & := & \freenames{P} \cup \freenames{Q} \\
  \freenames{\dropn{x}} & := & \{ x \}
\end{eqnarray*}

The bound names of a process, $\boundnames{P}$, are those names occurring in $P$
that are not free. For example, in $x?(y).0$, the name $x$ is free, while $y$ is bound.

\begin{mathpar}
  \inferrule* [lab=monoidal-laws] {} { P|Q \equiv Q|P \and P|0 \equiv P \and P|(Q|R) \equiv (P|Q)|R }
\end{mathpar}

\begin{mathpar}
  \inferrule* [lab=alpha-equivalence] {} { (x)P \equiv (y)P\{y/x\} \and y \not\in \freenames{P} }
\end{mathpar}

\begin{definition}
Then two processes, $P,Q$, are alpha-equivalent if $P = Q\{\vec{y}/\vec{x}\}$ for
some $\vec{x} \in \boundnames{Q},\vec{y} \in \boundnames{P}$, where $Q\{\vec{y}/\vec{x}\}$
denotes the capture-avoiding substitution of $\vec{y}$ for $\vec{x}$ in $Q$.
\end{definition}

\begin{definition}
  The {\em structural congruence} \cite{SangiorgiWalker} , $\equiv$,
  between processes is the least congruence containing
  alpha-equivalence, satisfying the abelian monoid laws
  (associativity, commutativity and $\pzero$ as identity) for parallel
  composition $|$ and for summation $+$.
\end{definition}

\subsection{Name equivalence}

We take name equivalence, written $\nameeq$, to be the smallest
equivalence relation generated by the following rules.

\begin{mathpar}
\inferrule*[lab=Quote-drop]
{ }
{ \quotep{@{x}} \nameeq x }

\inferrule*[lab=Struct-equiv]
{ P \scong Q }
{ \quotep{P} \nameeq \quotep{Q} }
\end{mathpar}

The astute reader will have noticed that the mutual recursion of names
and processes imposes a mutual recursion on alpha-equivalence and
structural equivalence via name-equivalence. Fortunately, all of this
works out pleasantly and we may calculate in the natural way, free of
concern. The reader interested in the details is referred to the
appendix \ref{appendix:rho_details}.

\subsection{Substitution}

We use $\Proc$ for the set of processes, $\QProc$ for the set of
names, and $\id{\{}\vec{y} / \vec{x} \id{\}}$ to denote partial maps,
$s : \QProc \rightarrow \QProc$. A map, $s$ lifts, uniquely, to a map
on process terms, $\widehat{s} : \Proc \rightarrow \Proc$ by the
following equations.

\begin{mathpar}
  (0) \psubstp{Q}{P} := 0 \\
  (R \juxtap S) \psubstp{Q}{P}
  :=    
  (R)\psubstp{Q}{P} \juxtap (S) \psubstp{Q}{P} \\
  (x?(y).R) \psubstp{Q}{P}    
  :=    
  (x)\substp{Q}{P} (z)\concat( (R \psubstn{z}{y}) \psubstp{Q}{P} ) \\
  (\lift{x}{R}) \psubstp{Q}{P}  
  :=
  \lift{(x)\substp{Q}{P}}{ R \psubstp{Q}{P} } \\
%   (\dropn{x})  \psubstp{Q}{P}       
%   := 
%   \left\{ 
%     \begin{array}{ccc} 
%       \dropn{\quotep{Q}} & & x \nameeq \quotep{P} \\
%       \dropn{x} & & otherwise \\
%     \end{array}
%   \right. 
  (\dropn{x})  \psubstp{Q}{P}       
  := 
  \left\{ 
    \begin{array}{ccc} 
      Q & & x \nameeq \quotep{P} \\
      \dropn{x} & & otherwise \\
    \end{array}
  \right.
\end{mathpar}
 

where

\begin{eqnarray}
  (x)\id{\{} \lpquote Q \rpquote / \lpquote P \rpquote \id{\}}            = 
  \left\{ 
    \begin{array}{ccc}
      \lpquote Q \rpquote & & x \nameeq \lpquote P \rpquote \\
      x & & otherwise \\
    \end{array}
  \right. \nonumber
\end{eqnarray}

and $z$ is chosen distinct from $\quotep{P}$, $\quotep{Q}$, the free
names in $Q$, and all the names in $R$. Our $\alpha$-equivalence will
be built in the standard way from this substitution.

\begin{remark}\label{rem:no_self_referential_names}
  One consequence of these definitions is that $\forall P. \quotep{P}
  \not\in \freenames{P}$.
\end{remark}

\subsection{ Dynamic quote: an example }

Anticipating something of what's to come, consider applying the
substitution, $\widehat{\id{\{}u / z \id{\}}}$, to the following pair
of processes, $\lift{w}{y!(z)}$ and $w[ \lpquote y!(z) \rpquote ]$.

\begin{eqnarray}
	\lift{w}{y!(z)}\widehat{\id{\{}u / z \id{\}}}
		& = &
		\lift{w}{y!(u)} \nonumber\\
	w[ \lpquote y!(z) \rpquote ] \widehat{ \id{\{}u / z \id{\}} }
		& = &
		w[ \lpquote y!(z) \rpquote ] \nonumber
\end{eqnarray}

Because the body of the process between quotes is impervious to
substitution, we get radically different answers. In fact, by
examining the first process in an input context,
e.g. $x?(z).\lift{w}{y!(z)}$, we see that the process under the lift
operator may be shaped by prefixed inputs binding a name inside it. In
this sense, the lift operator will be seen as a way to dynamically
construct processes before reifying them as names.

Finally equipped with these standard features we can present the
dynamics of the calculus.

\subsubsection{Operational semantics} 

Finally, we introduce the computational dynamics. What marks these
algebras as distinct from other more traditionally studied algebraic
structures, e.g. vector spaces or polynomial rings, is the manner in
which dynamics is captured. In traditional structures, dynamics is typically
expressed through morphisms between such structures, as in linear maps
between vector spaces or morphisms between rings. In algebras
associated with the semantics of computation, the dynamics is
expressed as part of the algebraic structure itself, through a
reduction reduction relation typically denoted by $\red$. Below, we
give a recursive presentation of this relation for the calculus used
in the encoding.

$\red \subseteq \pi \times \pi$
$\red : \pi \to \mathcal{P}(\pi)$

\begin{mathpar}
  \inferrule* [lab=Comm] { \textsf{match}( x_{src}, x_{trgt} ) } { x_{trgt}?(y)P \; | \; x_{src}!\langle {Q} \rangle \red P\{\quotep{Q}/y}\} }
  \and \\
  \inferrule* [lab=Par] {{P} \red {P}'} {{{P} | {Q}} \red {{P}' | {Q}}}
  \and
  \inferrule* [lab=Equiv]{{{P} \scong {P}'} \andalso {{P}' \red {Q}'} \andalso {{Q}' \scong {Q}}}{{P} \red {Q}}
\end{mathpar}

\begin{eqnarray*}
  match_{\equiv} (\quotep{P},\quotep{Q}) & := & P \equiv Q \\
  match_{\dagger}(\quotep{P},\quotep{Q}) & := & \forall R. P|Q \red^{*} R => R \red^{*} 0 \\
  match_{K}(\quotep{P},\quotep{Q}) & := & K \mbox{ for some context } K
\end{eqnarray*}

$u?(x)P | u!\langle Q \rangle \red P\{\quotep{Q}/x\}$

%We write $\wred$ for $\red^*$, and $P\red$ if $\exists Q $ such that $ P \red Q$.
We write $P\red$ if $\exists Q $ such that $ P \red Q$ and $P\not\red$, otherwise.

\section{Replication}

As mentioned before, it is known that replication (and hence
recursion) can be implemented in a higher-order process algebra
\cite{SangiorgiWalker}. As our first example of calculation with the
machinery thus far presented we give the construction explicitly in
the {\rhoc}.

\begin{eqnarray}
	D_{x} & := & \prefix{x}{y}{(\binpar{\outputp{x}{y}}{@{y}})} \nonumber\\
	\bangp_{x}{P} & := & \binpar{{x}!\langle{\binpar{D_{x}}{P}}\rangle}{D_{x}} \nonumber
\end{eqnarray}

\begin{eqnarray}
	\bangp_{x}{P} & & \nonumber\\
	=
	& {x}!\langle{(\prefix{x}{y}{(\outputp{x}{y} | @{y})) | P}}\rangle 
	      | \prefix{x}{y}{(\outputp{x}{y} | @{y})} & \nonumber\\
	\red
	& (\outputp{x}{y} | @{y})\substn{\quotep{(\prefix{x}{y}{(@{y} | \outputp{x}{y})) | P}}}{y} & \nonumber\\
	=
	& \outputp{x}{\quotep{(\prefix{x}{y}{(\outputp{x}{y} | @{y})) | P}}}
	  | {(\prefix{x}{y}{(\outputp{x}{y} | @{y})) | P}} & \nonumber\\
	\red
	& \ldots & \nonumber\\
	\red^*
	& P | P | \ldots & \nonumber
\end{eqnarray}

Of course, this encoding, as an implementation, runs away, unfolding
$\bangp{P}$ eagerly. A lazier and more implementable replication
operator, restricted to input-guarded processes, may be obtained as follows.

\begin{eqnarray}
\bangp{\prefix{u}{v}{P}} 
	:= 
	\binpar{\lift{x}{\prefix{u}{v}{(\binpar{D(x)}{P})}}}{D(x)} \nonumber
\end{eqnarray}

\begin{remark}
  Note that the lazier definition still does not deal with summation
  or mixed summation (i.e. sums over input and output). The reader is
  invited to construct definitions of replication that deal with these
  features. 

  Further, the definitions are parameterized in a name, $x$. Can you,
  gentle reader, make a definition that eliminates this parameter and
  guarantees no accidental interaction between the replication
  machinery and the process being replicated -- i.e. no accidental
  sharing of names used by the process to get its work done and the
  name(s) used by the replication to effect copying. This latter
  revision of the definition of replication is crucial to obtaining
  the expected identity $!!P \sim !P$.
\end{remark}

\begin{remark}\label{rem:paradoxical_combinator}
  The reader familiar with the lambda calculus will have noticed the
  similarity between $D$ and the paradoxical combinator.

  [Ed. note: the existence of this seems to suggest we have to be more
  restrictive on the set of processes and names we admit if we are to
  support no-cloning.]
\end{remark}

\subsubsection{Bisimulation}

The computational dynamics gives rise to another kind of equivalence,
the equivalence of computational behavior. As previously mentioned
this is typically captured \emph{via} some form of bisimulation.

% The notion we use in this paper is weak barbed bisimulation
% \cite{milner91polyadicpi}.

The notion we use in this paper is derived from weak barbed
bisimulation \cite{milner91polyadicpi}. 

\begin{definition}
An \emph{observation relation}, $\downarrow_{\mathcal N}$, over a set
of names, $\mathcal N$, is the smallest relation satisfying the rules
below.

\infrule[Out-barb]{y \in {\mathcal N}, \; x \nameeq y}
		  {\outputp{x}{v} \downarrow_{\mathcal N} x}
\infrule[Par-barb]{\mbox{$P\downarrow_{\mathcal N} x$ or $Q\downarrow_{\mathcal N} x$}}
		  {\binpar{P}{Q} \downarrow_{\mathcal N} x}

We write $P \Downarrow_{\mathcal N} x$ if there is $Q$ such that 
$P \wred Q$ and $Q \downarrow_{\mathcal N} x$.
\end{definition}

\begin{definition}
%\label{def.bbisim}
An  ${\mathcal N}$-\emph{barbed bisimulation} over a set of names, ${\mathcal N}$, is a symmetric binary relation 
${\mathcal S}_{\mathcal N}$ between agents such that $P\rel{S}_{\mathcal N}Q$ implies:
\begin{enumerate}
\item If $P \red P'$ then $Q \wred Q'$ and $P'\rel{S}_{\mathcal N} Q'$.
\item If $P\downarrow_{\mathcal N} x$, then $Q\Downarrow_{\mathcal N} x$.
\end{enumerate}
$P$ is ${\mathcal N}$-barbed bisimilar to $Q$, written
$P \wbbisim_{\mathcal N} Q$, if $P \rel{S}_{\mathcal N} Q$ for some ${\mathcal N}$-barbed bisimulation ${\mathcal S}_{\mathcal N}$.
\end{definition}

$\mathcal{R} \subseteq \pi \times \pi$

$P \mathcal{R} Q => \forall P'. P \red P' \Rightarrow \exists Q'. Q \red Q', P' \mathcal{R} Q'$

$P \vdash x \Rightarrow Q \vdash x$

\begin{mathpar}
  \inferrule*[lab=Out-barb]{x \nameeq y}{{y}!\langle{Q}\rangle \vdash x}
  \and
  \inferrule*[lab=Par-barb]{\mbox{$P\vdash x$ or $Q\vdash x$}}{\binpar{P}{Q} \vdash x}
\end{mathpar}

\subsubsection{Contexts}

One of the principle advantages of computational calculi like the
$\pi$-calculus is a well-defined notion of context,
contextual-equivalence and a correlation between
contextual-equivalence and notions of bisimulation. The notion of
context allows the decomposition of a process into (sub-)process and
its syntactic environment, its context. Thus, a context may be
thought of as a process with a ``hole'' (written $\Box$) in it. The
application of a context $M$ to a process $P$, written $M[P]$, is
tantamount to filling the hole in $M$ with $P$. In this paper we do
not need the full weight of this theory, but do make use of the notion
of context in the proof the main theorem. 

\begin{mathpar}
  \inferrule* [lab=summation] {} {{M_{M},M_{N}} \bc \Box \;|\; x.M_{A} \;|\; M_{M}+M_{N}}
  \and
  \inferrule* [lab=agent] {} {{M_{A}} \bc (\vec{x})M_{P} \;| \; \clift{P_0,\ldots,M_{P},\ldots,P_N}}
  \and \\
  \inferrule* [lab=process] {} {{M_{P}} \bc M_{N} \;| \;P|M_{P} }
\end{mathpar} 

\begin{mathpar}
  \inferrule* [lab=sychronization] {} {M_{N} \bc \Box \;|\; x?M_{F} \;|\; x!M_{C}}
  \and
  \inferrule* [lab=abstraction] {} {{M_{F}} \bc (x)M_{P} }
  \and
  \inferrule* [lab=concretion] {} {{M_{C}} \bc \langle M_{P} \rangle }
  \and \\
  \inferrule* [lab=process] {} {{M_{P}} \bc M_{N} \;| \;P|M_{P} }
\end{mathpar}

\begin{definition}[contextual application] Given a context $M$, and
  process $P$, we define the \emph{contextual application}, $M[P] :=
  M\{P/\Box\}$. That is, the contextual application of M to P is the
  substitution of $P$ for $\Box$ in $M$.
\end{definition}

$\meaningof{-} : L \to \mathcal{P}(\pi)$

\begin{mathpar}
  \inferrule* [lab=collection] {} {\meaningof{true} = \pi, \and \meaningof{~E} = \pi \setminus \meaningof{E}, \and \meaningof{E_{1} \& E_{2}} = \meaningof{E_{1}} \cap \meaningof{E_{2}}}
\end{mathpar}

\begin{mathpar}
  \inferrule* [lab=structure] {} {\meaningof{0} = \{ P \in \pi | P \equiv 0 \}, \and \\ \meaningof{E_1 | E_2} = \{ P \in \pi | P \equiv P_{1} | P_{2}, P_{1} \in \meaningof{E_{1}}, P_{2} \in \meaningof{E_2}\} }
\end{mathpar}

\begin{mathpar}
 \inferrule* [lab=behavior] {} {\meaningof{\langle a?b \rangle E} = \{ P \in \pi | P \equiv Q | u?(y)P', \\ \and \\\\ \and \\ \;\;\; u \in \meaningof{a}, \forall z.P'\{z/y\} \in \meaningof{E\{z/b\}}\}, \and \\ \meaningof{a!E} = \{ P \in \pi | P \equiv Q | x!\langle P' \rangle, x \in \meaningof{a} P' \in \meaningof{E}\} }
\end{mathpar}

\begin{mathpar}
 \inferrule* [lab=nominal] {} {\meaningof{\quotep{E}} = \{ \quotep{P} \in \quotep{\pi} | P \in \meaningof{E} \}, \and \meaningof{\quotep{P}} = \{ \quotep{Q} \in \quotep{\pi} | P \equiv Q \} \and \\ \meaningof{@\quotep{E}} = \{ P \in \pi | P \equiv @x, x \in \meaningof{E} \}}
\end{mathpar}

\begin{eqnarray*}
  \\
  \meaningof{-} : TS \to ST
\end{eqnarray*}

\begin{eqnarray*}
  \\
  L : TS \to ST
\end{eqnarray*}

\begin{eqnarray*}
  \\
  P \models E \iff P \in \meaningof{E}
\end{eqnarray*}

\begin{eqnarray*}
  P \approx_{L} Q \iff \forall E \in L. P \models E \iff Q \models E
\end{eqnarray*}

\begin{eqnarray*}
  P \approx_{K} Q
\end{eqnarray*}

\begin{eqnarray*}
  P \approx Q
\end{eqnarray*}

$\approx_{K} = \approx = \approx_{L}$

\subsubsection{Contextual duality}

Note that contexts extend the quotation operation to a family of
operations from processes to names. Given a context, $M$, we can
define a \emph{nominal context}, $\quotep{M}$ by $\quotep{M}[P] :=
\quotep{M[P]}$. To foreshadow what is to come we observe that these
operations enjoy a duality with processes very much like the duality
between vectors and maps from vectors to scalars.

Further, because the calculus is essentially higher-order, we have a
correspondence between contexts and processes. More specifically,
given a name $x$ and a context $M$ we can construct $M^{*}_{x}$ such
that 

\begin{mathpar}
  M^{*}_{x} | \lift{x}{P} \red M[P]
\end{mathpar}

namely,

\begin{mathpar}
  M^{*}_{x} := x?(u).M[\dropn{u}]
\end{mathpar}

The dependence of $M^{*}_{x}$ on a name makes it an abstraction, 

\begin{mathpar}
  M^{*} := (x)x?(u).M[\dropn{u}]
\end{mathpar}

\subsection{Additional notation}

It will sometimes be convenient to denote the process a name
quotes. We already have the notation $x = \quotep{P}$, but it will be
convenient to introduce an alternate notation, $\procn{x}$, when we
want to emphasize the connection to the use of the name. Note that, by
virtue of name equivalence, $\quotep{\procn{x}} \nameeq x$; so, the
notation is consistent with previous definitions.

Further, because names have structure it is possible to effect
substitutions on the basis of that structure. This means we need to
upgrade our notation for substitutions, which we accomplish by
adapting comprehension notation. Thus,

\begin{mathpar}
  P\{ y / x : x \in S \}
\end{mathpar}

is interpreted to mean the process derived from P by replacing (in a
capture-avoiding manner) each occurrence of $x$ in $S$ by $y$. For example,

\begin{mathpar}
  P\{ \quotep{\procn{x}|\procn{x}} / x : x \in \freenames{P} \}
\end{mathpar}

will replace each (occurrence) of a free name $x$ in $P$ by
$\quotep{\procn{x}|\procn{x}}$.

Also, we will avail ourselves of the notation $x^{L}$ and $x^{R}$ to
denote injections of a name into disjoint copies of the name
space. There are numerous ways to accomplish this. One example can be
found in \cite{MeredithR05}. This notation overloads to vectors of
names: $\vec{x}^{\pi} := (x_{i}^{\pi} \; : \; 0 \leq i < |\vec{x}| )$ where $\pi \in \{L,R\}$.

We also use $P^{\Box} := P|\Box$.

In \cite{MeredithR05} an interpretation of the new operator is
given. It turns out that there are several possible interpretations
all enjoying the requisite algebraic properties of the operator (see
\cite{milner91polyadicpi}). We will therefore make liberal use of
$(\nu\; \vec{x})P$.

% subsection the_syntax_and_semantics_of_the_notation_system (end)   

\input{qm2pi.qmops} 

\input{qm2pi.sterngerlach} 

\input{qm2pi.metric} 

% section concurrent_process_calculi (end)

%\input{qm2pi.proofsketch}

% section proof sketch (end)

%\input{qm2pi.slviaknots} 

% section spatial logic via knots (end)

\input{qm2pi.conclusion}

% section conclusion (end)

%\input{qm2pi.dtcodes} 

% section wiring algorithm (end)

\input{qm2pi.ack} 

% section acknowledgments (end)

\newpage


\bibliographystyle{plain}   
\bibliography{../../biblios/main.bib}

\input{qm2pi.rhodetails}

\end{document}

 

% section notation (end)

\input{qm2pi.process.calculi} 

% section concurrent_process_calculi_and_spatial_logics_ (end)
    
%\documentclass[12pt]{llncs}
%\documentclass{jktr}

\usepackage[pdftex]{hyperref}                   
\usepackage {listings}
\usepackage {mathpartir}
\usepackage{bcprules}
%\usepackage{listings}
                       
\usepackage{graphicx} 
%\usepackage[margins=2.5cm,nohead,nofoot]{geometry}
%\usepackage{geometry}
\usepackage{amsfonts}
\usepackage{amstext}
\usepackage{latexsym}
\usepackage{amssymb}
\usepackage{color}


%\include{myPreamble}
\include{qm2pi.local} 

%\ifpdf
%\usepackage[pdftex]{graphicx}
%\else
%\usepackage{graphicx}
%\fi

 % \ifpdf
%  \usepackage{pdfsync}
%  \if


%\title{Brief Article}
%\author{David F. Snyder}
%\author{L.G. Meredith}

%\address{Dept. of Math., Texas State University--San Marcos, San Marcos, TX 78666}
       
\pagestyle{empty}


\begin{document}

\lstset{language=[Objective]Caml,frame=shadowbox}

\input{qm2pi.front}

% section front matter (end)

\input{qm2pi.intro} 
 
% section introduction (end)

% \input{qm2pi.knotations} 

% section notation (end)

\input{qm2pi.process.calculi} 

% section concurrent_process_calculi_and_spatial_logics_ (end)
    
%\input{qm2pi.knots2pi} 

%\input{qm2pi.trefoil} 

%\input{qm2pi.mainthm} 

% subsection basic_interpretation (end)

%\input{qm2pi.rho.presentation} 
\subsection{The syntax and semantics of the notation system}\label{sub:the_syntax_and_semantics_of_the_notation_system} % (fold)

We now summarize a technical presentation of the calculus that
embodies our theory of dynamics. The typical presentation of such a
calculus follows the style of giving generators and relations on
them. The grammar, below, describing term constructors, freely
generates the set of processes, $\Proc$. This set is then quotiented
by a relation known as structural congruence and it is over this set
that the notion of dynamics is expressed. This presentation is
essentially that of \cite{MeredithR05} with the addition of
polyadicity and summation. For readability we have relegated some of
the technical subtleties to an appendix.

\subsubsection{Process grammar}\label{subsub:process_grammar}

\begin{mathpar}
  \inferrule* [lab=synchronization] {} {{M} \bc \pzero \;|\; x?F \;|\; x!C }
  \and
  \inferrule* [lab=abstraction] {} {{F} \bc (x)P}
  \and
  \inferrule* [lab=concretion] {} {{C} \bc \langle Q \rangle}
  \and
  \inferrule* [lab=process] {} {{P,Q} \bc M \;| \;P|Q \;|\; @{x}}
  \and
  \inferrule* [lab=name] {} {{x} \bc \quotep{P}}
\end{mathpar} 

Note that $\vec{x}$ (resp. $\vec{P}$) denotes a vector of names
(resp. processes) of length $|\vec{x}|$ (resp. $|\vec{P}|$). We adopt
the following useful abbreviations.

\begin{mathpar}
   x?(\vec{y}).P := x.(\vec{y})P \and  x\clift{\vec{P}} := x.\clift{\vec{P}}
   \and x!(y) := \lift{x}{\dropn{y}}
   \and \Pi_{i=0}^{n-1}P_i := P_0 | \ldots | P_{n-1}
\end{mathpar}

\subsubsection{Structural congruence}

\paragraph{Free and bound names and alpha-equivalence.} At the
core of structural equivalence is alpha-equivalence which identifies
process that are the same up to a change of variable. Formally, we
recognize the distinction between free and bound names. The free names
of a process, $\freenames{P}$, may be calculated recursively as
follows:

\begin{mathpar}
\freenames{\pzero} := \emptyset
  \and \\
  \freenames{x?(y).P} := \{ x \} \cup (\freenames{P} \setminus \{ y \})
  \and 
  \freenames{x!\langle P \rangle} := \{ x \} \cup \{ P \} 
  \and \\
  \freenames{P|Q} := \freenames{P} \cup \freenames{Q}
  \and \\
  \freenames{@{x}} := \{ x \}
\end{mathpar}

$\pi$
$\quotep{\pi}$

$\freenames{-} : \pi \to \mathcal{P}(\quotep{\pi})$

\begin{eqnarray*}
  \freenames{\pzero} & := & \emptyset \\
  \freenames{x?(y).P} & := & \{ x \} \cup (\freenames{P} \setminus \{ y \}) \\
  \freenames{x!\langle P \rangle} & := & \{ x \} \cup \{ P \} \\
  \freenames{P|Q} & := & \freenames{P} \cup \freenames{Q} \\
  \freenames{\dropn{x}} & := & \{ x \}
\end{eqnarray*}

The bound names of a process, $\boundnames{P}$, are those names occurring in $P$
that are not free. For example, in $x?(y).0$, the name $x$ is free, while $y$ is bound.

\begin{mathpar}
  \inferrule* [lab=monoidal-laws] {} { P|Q \equiv Q|P \and P|0 \equiv P \and P|(Q|R) \equiv (P|Q)|R }
\end{mathpar}

\begin{mathpar}
  \inferrule* [lab=alpha-equivalence] {} { (x)P \equiv (y)P\{y/x\} \and y \not\in \freenames{P} }
\end{mathpar}

\begin{definition}
Then two processes, $P,Q$, are alpha-equivalent if $P = Q\{\vec{y}/\vec{x}\}$ for
some $\vec{x} \in \boundnames{Q},\vec{y} \in \boundnames{P}$, where $Q\{\vec{y}/\vec{x}\}$
denotes the capture-avoiding substitution of $\vec{y}$ for $\vec{x}$ in $Q$.
\end{definition}

\begin{definition}
  The {\em structural congruence} \cite{SangiorgiWalker} , $\equiv$,
  between processes is the least congruence containing
  alpha-equivalence, satisfying the abelian monoid laws
  (associativity, commutativity and $\pzero$ as identity) for parallel
  composition $|$ and for summation $+$.
\end{definition}

\subsection{Name equivalence}

We take name equivalence, written $\nameeq$, to be the smallest
equivalence relation generated by the following rules.

\begin{mathpar}
\inferrule*[lab=Quote-drop]
{ }
{ \quotep{@{x}} \nameeq x }

\inferrule*[lab=Struct-equiv]
{ P \scong Q }
{ \quotep{P} \nameeq \quotep{Q} }
\end{mathpar}

The astute reader will have noticed that the mutual recursion of names
and processes imposes a mutual recursion on alpha-equivalence and
structural equivalence via name-equivalence. Fortunately, all of this
works out pleasantly and we may calculate in the natural way, free of
concern. The reader interested in the details is referred to the
appendix \ref{appendix:rho_details}.

\subsection{Substitution}

We use $\Proc$ for the set of processes, $\QProc$ for the set of
names, and $\id{\{}\vec{y} / \vec{x} \id{\}}$ to denote partial maps,
$s : \QProc \rightarrow \QProc$. A map, $s$ lifts, uniquely, to a map
on process terms, $\widehat{s} : \Proc \rightarrow \Proc$ by the
following equations.

\begin{mathpar}
  (0) \psubstp{Q}{P} := 0 \\
  (R \juxtap S) \psubstp{Q}{P}
  :=    
  (R)\psubstp{Q}{P} \juxtap (S) \psubstp{Q}{P} \\
  (x?(y).R) \psubstp{Q}{P}    
  :=    
  (x)\substp{Q}{P} (z)\concat( (R \psubstn{z}{y}) \psubstp{Q}{P} ) \\
  (\lift{x}{R}) \psubstp{Q}{P}  
  :=
  \lift{(x)\substp{Q}{P}}{ R \psubstp{Q}{P} } \\
%   (\dropn{x})  \psubstp{Q}{P}       
%   := 
%   \left\{ 
%     \begin{array}{ccc} 
%       \dropn{\quotep{Q}} & & x \nameeq \quotep{P} \\
%       \dropn{x} & & otherwise \\
%     \end{array}
%   \right. 
  (\dropn{x})  \psubstp{Q}{P}       
  := 
  \left\{ 
    \begin{array}{ccc} 
      Q & & x \nameeq \quotep{P} \\
      \dropn{x} & & otherwise \\
    \end{array}
  \right.
\end{mathpar}
 

where

\begin{eqnarray}
  (x)\id{\{} \lpquote Q \rpquote / \lpquote P \rpquote \id{\}}            = 
  \left\{ 
    \begin{array}{ccc}
      \lpquote Q \rpquote & & x \nameeq \lpquote P \rpquote \\
      x & & otherwise \\
    \end{array}
  \right. \nonumber
\end{eqnarray}

and $z$ is chosen distinct from $\quotep{P}$, $\quotep{Q}$, the free
names in $Q$, and all the names in $R$. Our $\alpha$-equivalence will
be built in the standard way from this substitution.

\begin{remark}\label{rem:no_self_referential_names}
  One consequence of these definitions is that $\forall P. \quotep{P}
  \not\in \freenames{P}$.
\end{remark}

\subsection{ Dynamic quote: an example }

Anticipating something of what's to come, consider applying the
substitution, $\widehat{\id{\{}u / z \id{\}}}$, to the following pair
of processes, $\lift{w}{y!(z)}$ and $w[ \lpquote y!(z) \rpquote ]$.

\begin{eqnarray}
	\lift{w}{y!(z)}\widehat{\id{\{}u / z \id{\}}}
		& = &
		\lift{w}{y!(u)} \nonumber\\
	w[ \lpquote y!(z) \rpquote ] \widehat{ \id{\{}u / z \id{\}} }
		& = &
		w[ \lpquote y!(z) \rpquote ] \nonumber
\end{eqnarray}

Because the body of the process between quotes is impervious to
substitution, we get radically different answers. In fact, by
examining the first process in an input context,
e.g. $x?(z).\lift{w}{y!(z)}$, we see that the process under the lift
operator may be shaped by prefixed inputs binding a name inside it. In
this sense, the lift operator will be seen as a way to dynamically
construct processes before reifying them as names.

Finally equipped with these standard features we can present the
dynamics of the calculus.

\subsubsection{Operational semantics} 

Finally, we introduce the computational dynamics. What marks these
algebras as distinct from other more traditionally studied algebraic
structures, e.g. vector spaces or polynomial rings, is the manner in
which dynamics is captured. In traditional structures, dynamics is typically
expressed through morphisms between such structures, as in linear maps
between vector spaces or morphisms between rings. In algebras
associated with the semantics of computation, the dynamics is
expressed as part of the algebraic structure itself, through a
reduction reduction relation typically denoted by $\red$. Below, we
give a recursive presentation of this relation for the calculus used
in the encoding.

$\red \subseteq \pi \times \pi$
$\red : \pi \to \mathcal{P}(\pi)$

\begin{mathpar}
  \inferrule* [lab=Comm] { \textsf{match}( x_{src}, x_{trgt} ) } { x_{trgt}?(y)P \; | \; x_{src}!\langle {Q} \rangle \red P\{\quotep{Q}/y}\} }
  \and \\
  \inferrule* [lab=Par] {{P} \red {P}'} {{{P} | {Q}} \red {{P}' | {Q}}}
  \and
  \inferrule* [lab=Equiv]{{{P} \scong {P}'} \andalso {{P}' \red {Q}'} \andalso {{Q}' \scong {Q}}}{{P} \red {Q}}
\end{mathpar}

\begin{eqnarray*}
  match_{\equiv} (\quotep{P},\quotep{Q}) & := & P \equiv Q \\
  match_{\dagger}(\quotep{P},\quotep{Q}) & := & \forall R. P|Q \red^{*} R => R \red^{*} 0 \\
  match_{K}(\quotep{P},\quotep{Q}) & := & K \mbox{ for some context } K
\end{eqnarray*}

$u?(x)P | u!\langle Q \rangle \red P\{\quotep{Q}/x\}$

%We write $\wred$ for $\red^*$, and $P\red$ if $\exists Q $ such that $ P \red Q$.
We write $P\red$ if $\exists Q $ such that $ P \red Q$ and $P\not\red$, otherwise.

\section{Replication}

As mentioned before, it is known that replication (and hence
recursion) can be implemented in a higher-order process algebra
\cite{SangiorgiWalker}. As our first example of calculation with the
machinery thus far presented we give the construction explicitly in
the {\rhoc}.

\begin{eqnarray}
	D_{x} & := & \prefix{x}{y}{(\binpar{\outputp{x}{y}}{@{y}})} \nonumber\\
	\bangp_{x}{P} & := & \binpar{{x}!\langle{\binpar{D_{x}}{P}}\rangle}{D_{x}} \nonumber
\end{eqnarray}

\begin{eqnarray}
	\bangp_{x}{P} & & \nonumber\\
	=
	& {x}!\langle{(\prefix{x}{y}{(\outputp{x}{y} | @{y})) | P}}\rangle 
	      | \prefix{x}{y}{(\outputp{x}{y} | @{y})} & \nonumber\\
	\red
	& (\outputp{x}{y} | @{y})\substn{\quotep{(\prefix{x}{y}{(@{y} | \outputp{x}{y})) | P}}}{y} & \nonumber\\
	=
	& \outputp{x}{\quotep{(\prefix{x}{y}{(\outputp{x}{y} | @{y})) | P}}}
	  | {(\prefix{x}{y}{(\outputp{x}{y} | @{y})) | P}} & \nonumber\\
	\red
	& \ldots & \nonumber\\
	\red^*
	& P | P | \ldots & \nonumber
\end{eqnarray}

Of course, this encoding, as an implementation, runs away, unfolding
$\bangp{P}$ eagerly. A lazier and more implementable replication
operator, restricted to input-guarded processes, may be obtained as follows.

\begin{eqnarray}
\bangp{\prefix{u}{v}{P}} 
	:= 
	\binpar{\lift{x}{\prefix{u}{v}{(\binpar{D(x)}{P})}}}{D(x)} \nonumber
\end{eqnarray}

\begin{remark}
  Note that the lazier definition still does not deal with summation
  or mixed summation (i.e. sums over input and output). The reader is
  invited to construct definitions of replication that deal with these
  features. 

  Further, the definitions are parameterized in a name, $x$. Can you,
  gentle reader, make a definition that eliminates this parameter and
  guarantees no accidental interaction between the replication
  machinery and the process being replicated -- i.e. no accidental
  sharing of names used by the process to get its work done and the
  name(s) used by the replication to effect copying. This latter
  revision of the definition of replication is crucial to obtaining
  the expected identity $!!P \sim !P$.
\end{remark}

\begin{remark}\label{rem:paradoxical_combinator}
  The reader familiar with the lambda calculus will have noticed the
  similarity between $D$ and the paradoxical combinator.

  [Ed. note: the existence of this seems to suggest we have to be more
  restrictive on the set of processes and names we admit if we are to
  support no-cloning.]
\end{remark}

\subsubsection{Bisimulation}

The computational dynamics gives rise to another kind of equivalence,
the equivalence of computational behavior. As previously mentioned
this is typically captured \emph{via} some form of bisimulation.

% The notion we use in this paper is weak barbed bisimulation
% \cite{milner91polyadicpi}.

The notion we use in this paper is derived from weak barbed
bisimulation \cite{milner91polyadicpi}. 

\begin{definition}
An \emph{observation relation}, $\downarrow_{\mathcal N}$, over a set
of names, $\mathcal N$, is the smallest relation satisfying the rules
below.

\infrule[Out-barb]{y \in {\mathcal N}, \; x \nameeq y}
		  {\outputp{x}{v} \downarrow_{\mathcal N} x}
\infrule[Par-barb]{\mbox{$P\downarrow_{\mathcal N} x$ or $Q\downarrow_{\mathcal N} x$}}
		  {\binpar{P}{Q} \downarrow_{\mathcal N} x}

We write $P \Downarrow_{\mathcal N} x$ if there is $Q$ such that 
$P \wred Q$ and $Q \downarrow_{\mathcal N} x$.
\end{definition}

\begin{definition}
%\label{def.bbisim}
An  ${\mathcal N}$-\emph{barbed bisimulation} over a set of names, ${\mathcal N}$, is a symmetric binary relation 
${\mathcal S}_{\mathcal N}$ between agents such that $P\rel{S}_{\mathcal N}Q$ implies:
\begin{enumerate}
\item If $P \red P'$ then $Q \wred Q'$ and $P'\rel{S}_{\mathcal N} Q'$.
\item If $P\downarrow_{\mathcal N} x$, then $Q\Downarrow_{\mathcal N} x$.
\end{enumerate}
$P$ is ${\mathcal N}$-barbed bisimilar to $Q$, written
$P \wbbisim_{\mathcal N} Q$, if $P \rel{S}_{\mathcal N} Q$ for some ${\mathcal N}$-barbed bisimulation ${\mathcal S}_{\mathcal N}$.
\end{definition}

$\mathcal{R} \subseteq \pi \times \pi$

$P \mathcal{R} Q => \forall P'. P \red P' \Rightarrow \exists Q'. Q \red Q', P' \mathcal{R} Q'$

$P \vdash x \Rightarrow Q \vdash x$

\begin{mathpar}
  \inferrule*[lab=Out-barb]{x \nameeq y}{{y}!\langle{Q}\rangle \vdash x}
  \and
  \inferrule*[lab=Par-barb]{\mbox{$P\vdash x$ or $Q\vdash x$}}{\binpar{P}{Q} \vdash x}
\end{mathpar}

\subsubsection{Contexts}

One of the principle advantages of computational calculi like the
$\pi$-calculus is a well-defined notion of context,
contextual-equivalence and a correlation between
contextual-equivalence and notions of bisimulation. The notion of
context allows the decomposition of a process into (sub-)process and
its syntactic environment, its context. Thus, a context may be
thought of as a process with a ``hole'' (written $\Box$) in it. The
application of a context $M$ to a process $P$, written $M[P]$, is
tantamount to filling the hole in $M$ with $P$. In this paper we do
not need the full weight of this theory, but do make use of the notion
of context in the proof the main theorem. 

\begin{mathpar}
  \inferrule* [lab=summation] {} {{M_{M},M_{N}} \bc \Box \;|\; x.M_{A} \;|\; M_{M}+M_{N}}
  \and
  \inferrule* [lab=agent] {} {{M_{A}} \bc (\vec{x})M_{P} \;| \; \clift{P_0,\ldots,M_{P},\ldots,P_N}}
  \and \\
  \inferrule* [lab=process] {} {{M_{P}} \bc M_{N} \;| \;P|M_{P} }
\end{mathpar} 

\begin{mathpar}
  \inferrule* [lab=sychronization] {} {M_{N} \bc \Box \;|\; x?M_{F} \;|\; x!M_{C}}
  \and
  \inferrule* [lab=abstraction] {} {{M_{F}} \bc (x)M_{P} }
  \and
  \inferrule* [lab=concretion] {} {{M_{C}} \bc \langle M_{P} \rangle }
  \and \\
  \inferrule* [lab=process] {} {{M_{P}} \bc M_{N} \;| \;P|M_{P} }
\end{mathpar}

\begin{definition}[contextual application] Given a context $M$, and
  process $P$, we define the \emph{contextual application}, $M[P] :=
  M\{P/\Box\}$. That is, the contextual application of M to P is the
  substitution of $P$ for $\Box$ in $M$.
\end{definition}

$\meaningof{-} : L \to \mathcal{P}(\pi)$

\begin{mathpar}
  \inferrule* [lab=collection] {} {\meaningof{true} = \pi, \and \meaningof{~E} = \pi \setminus \meaningof{E}, \and \meaningof{E_{1} \& E_{2}} = \meaningof{E_{1}} \cap \meaningof{E_{2}}}
\end{mathpar}

\begin{mathpar}
  \inferrule* [lab=structure] {} {\meaningof{0} = \{ P \in \pi | P \equiv 0 \}, \and \\ \meaningof{E_1 | E_2} = \{ P \in \pi | P \equiv P_{1} | P_{2}, P_{1} \in \meaningof{E_{1}}, P_{2} \in \meaningof{E_2}\} }
\end{mathpar}

\begin{mathpar}
 \inferrule* [lab=behavior] {} {\meaningof{\langle a?b \rangle E} = \{ P \in \pi | P \equiv Q | u?(y)P', \\ \and \\\\ \and \\ \;\;\; u \in \meaningof{a}, \forall z.P'\{z/y\} \in \meaningof{E\{z/b\}}\}, \and \\ \meaningof{a!E} = \{ P \in \pi | P \equiv Q | x!\langle P' \rangle, x \in \meaningof{a} P' \in \meaningof{E}\} }
\end{mathpar}

\begin{mathpar}
 \inferrule* [lab=nominal] {} {\meaningof{\quotep{E}} = \{ \quotep{P} \in \quotep{\pi} | P \in \meaningof{E} \}, \and \meaningof{\quotep{P}} = \{ \quotep{Q} \in \quotep{\pi} | P \equiv Q \} \and \\ \meaningof{@\quotep{E}} = \{ P \in \pi | P \equiv @x, x \in \meaningof{E} \}}
\end{mathpar}

\begin{eqnarray*}
  \\
  \meaningof{-} : TS \to ST
\end{eqnarray*}

\begin{eqnarray*}
  \\
  L : TS \to ST
\end{eqnarray*}

\begin{eqnarray*}
  \\
  P \models E \iff P \in \meaningof{E}
\end{eqnarray*}

\begin{eqnarray*}
  P \approx_{L} Q \iff \forall E \in L. P \models E \iff Q \models E
\end{eqnarray*}

\begin{eqnarray*}
  P \approx_{K} Q
\end{eqnarray*}

\begin{eqnarray*}
  P \approx Q
\end{eqnarray*}

$\approx_{K} = \approx = \approx_{L}$

\subsubsection{Contextual duality}

Note that contexts extend the quotation operation to a family of
operations from processes to names. Given a context, $M$, we can
define a \emph{nominal context}, $\quotep{M}$ by $\quotep{M}[P] :=
\quotep{M[P]}$. To foreshadow what is to come we observe that these
operations enjoy a duality with processes very much like the duality
between vectors and maps from vectors to scalars.

Further, because the calculus is essentially higher-order, we have a
correspondence between contexts and processes. More specifically,
given a name $x$ and a context $M$ we can construct $M^{*}_{x}$ such
that 

\begin{mathpar}
  M^{*}_{x} | \lift{x}{P} \red M[P]
\end{mathpar}

namely,

\begin{mathpar}
  M^{*}_{x} := x?(u).M[\dropn{u}]
\end{mathpar}

The dependence of $M^{*}_{x}$ on a name makes it an abstraction, 

\begin{mathpar}
  M^{*} := (x)x?(u).M[\dropn{u}]
\end{mathpar}

\subsection{Additional notation}

It will sometimes be convenient to denote the process a name
quotes. We already have the notation $x = \quotep{P}$, but it will be
convenient to introduce an alternate notation, $\procn{x}$, when we
want to emphasize the connection to the use of the name. Note that, by
virtue of name equivalence, $\quotep{\procn{x}} \nameeq x$; so, the
notation is consistent with previous definitions.

Further, because names have structure it is possible to effect
substitutions on the basis of that structure. This means we need to
upgrade our notation for substitutions, which we accomplish by
adapting comprehension notation. Thus,

\begin{mathpar}
  P\{ y / x : x \in S \}
\end{mathpar}

is interpreted to mean the process derived from P by replacing (in a
capture-avoiding manner) each occurrence of $x$ in $S$ by $y$. For example,

\begin{mathpar}
  P\{ \quotep{\procn{x}|\procn{x}} / x : x \in \freenames{P} \}
\end{mathpar}

will replace each (occurrence) of a free name $x$ in $P$ by
$\quotep{\procn{x}|\procn{x}}$.

Also, we will avail ourselves of the notation $x^{L}$ and $x^{R}$ to
denote injections of a name into disjoint copies of the name
space. There are numerous ways to accomplish this. One example can be
found in \cite{MeredithR05}. This notation overloads to vectors of
names: $\vec{x}^{\pi} := (x_{i}^{\pi} \; : \; 0 \leq i < |\vec{x}| )$ where $\pi \in \{L,R\}$.

We also use $P^{\Box} := P|\Box$.

In \cite{MeredithR05} an interpretation of the new operator is
given. It turns out that there are several possible interpretations
all enjoying the requisite algebraic properties of the operator (see
\cite{milner91polyadicpi}). We will therefore make liberal use of
$(\nu\; \vec{x})P$.

% subsection the_syntax_and_semantics_of_the_notation_system (end)   

\input{qm2pi.qmops} 

\input{qm2pi.sterngerlach} 

\input{qm2pi.metric} 

% section concurrent_process_calculi (end)

%\input{qm2pi.proofsketch}

% section proof sketch (end)

%\input{qm2pi.slviaknots} 

% section spatial logic via knots (end)

\input{qm2pi.conclusion}

% section conclusion (end)

%\input{qm2pi.dtcodes} 

% section wiring algorithm (end)

\input{qm2pi.ack} 

% section acknowledgments (end)

\newpage


\bibliographystyle{plain}   
\bibliography{../../biblios/main.bib}

\input{qm2pi.rhodetails}

\end{document}

 

%\documentclass[12pt]{llncs}
%\documentclass{jktr}

\usepackage[pdftex]{hyperref}                   
\usepackage {listings}
\usepackage {mathpartir}
\usepackage{bcprules}
%\usepackage{listings}
                       
\usepackage{graphicx} 
%\usepackage[margins=2.5cm,nohead,nofoot]{geometry}
%\usepackage{geometry}
\usepackage{amsfonts}
\usepackage{amstext}
\usepackage{latexsym}
\usepackage{amssymb}
\usepackage{color}


%\include{myPreamble}
\include{qm2pi.local} 

%\ifpdf
%\usepackage[pdftex]{graphicx}
%\else
%\usepackage{graphicx}
%\fi

 % \ifpdf
%  \usepackage{pdfsync}
%  \if


%\title{Brief Article}
%\author{David F. Snyder}
%\author{L.G. Meredith}

%\address{Dept. of Math., Texas State University--San Marcos, San Marcos, TX 78666}
       
\pagestyle{empty}


\begin{document}

\lstset{language=[Objective]Caml,frame=shadowbox}

\input{qm2pi.front}

% section front matter (end)

\input{qm2pi.intro} 
 
% section introduction (end)

% \input{qm2pi.knotations} 

% section notation (end)

\input{qm2pi.process.calculi} 

% section concurrent_process_calculi_and_spatial_logics_ (end)
    
%\input{qm2pi.knots2pi} 

%\input{qm2pi.trefoil} 

%\input{qm2pi.mainthm} 

% subsection basic_interpretation (end)

%\input{qm2pi.rho.presentation} 
\subsection{The syntax and semantics of the notation system}\label{sub:the_syntax_and_semantics_of_the_notation_system} % (fold)

We now summarize a technical presentation of the calculus that
embodies our theory of dynamics. The typical presentation of such a
calculus follows the style of giving generators and relations on
them. The grammar, below, describing term constructors, freely
generates the set of processes, $\Proc$. This set is then quotiented
by a relation known as structural congruence and it is over this set
that the notion of dynamics is expressed. This presentation is
essentially that of \cite{MeredithR05} with the addition of
polyadicity and summation. For readability we have relegated some of
the technical subtleties to an appendix.

\subsubsection{Process grammar}\label{subsub:process_grammar}

\begin{mathpar}
  \inferrule* [lab=synchronization] {} {{M} \bc \pzero \;|\; x?F \;|\; x!C }
  \and
  \inferrule* [lab=abstraction] {} {{F} \bc (x)P}
  \and
  \inferrule* [lab=concretion] {} {{C} \bc \langle Q \rangle}
  \and
  \inferrule* [lab=process] {} {{P,Q} \bc M \;| \;P|Q \;|\; @{x}}
  \and
  \inferrule* [lab=name] {} {{x} \bc \quotep{P}}
\end{mathpar} 

Note that $\vec{x}$ (resp. $\vec{P}$) denotes a vector of names
(resp. processes) of length $|\vec{x}|$ (resp. $|\vec{P}|$). We adopt
the following useful abbreviations.

\begin{mathpar}
   x?(\vec{y}).P := x.(\vec{y})P \and  x\clift{\vec{P}} := x.\clift{\vec{P}}
   \and x!(y) := \lift{x}{\dropn{y}}
   \and \Pi_{i=0}^{n-1}P_i := P_0 | \ldots | P_{n-1}
\end{mathpar}

\subsubsection{Structural congruence}

\paragraph{Free and bound names and alpha-equivalence.} At the
core of structural equivalence is alpha-equivalence which identifies
process that are the same up to a change of variable. Formally, we
recognize the distinction between free and bound names. The free names
of a process, $\freenames{P}$, may be calculated recursively as
follows:

\begin{mathpar}
\freenames{\pzero} := \emptyset
  \and \\
  \freenames{x?(y).P} := \{ x \} \cup (\freenames{P} \setminus \{ y \})
  \and 
  \freenames{x!\langle P \rangle} := \{ x \} \cup \{ P \} 
  \and \\
  \freenames{P|Q} := \freenames{P} \cup \freenames{Q}
  \and \\
  \freenames{@{x}} := \{ x \}
\end{mathpar}

$\pi$
$\quotep{\pi}$

$\freenames{-} : \pi \to \mathcal{P}(\quotep{\pi})$

\begin{eqnarray*}
  \freenames{\pzero} & := & \emptyset \\
  \freenames{x?(y).P} & := & \{ x \} \cup (\freenames{P} \setminus \{ y \}) \\
  \freenames{x!\langle P \rangle} & := & \{ x \} \cup \{ P \} \\
  \freenames{P|Q} & := & \freenames{P} \cup \freenames{Q} \\
  \freenames{\dropn{x}} & := & \{ x \}
\end{eqnarray*}

The bound names of a process, $\boundnames{P}$, are those names occurring in $P$
that are not free. For example, in $x?(y).0$, the name $x$ is free, while $y$ is bound.

\begin{mathpar}
  \inferrule* [lab=monoidal-laws] {} { P|Q \equiv Q|P \and P|0 \equiv P \and P|(Q|R) \equiv (P|Q)|R }
\end{mathpar}

\begin{mathpar}
  \inferrule* [lab=alpha-equivalence] {} { (x)P \equiv (y)P\{y/x\} \and y \not\in \freenames{P} }
\end{mathpar}

\begin{definition}
Then two processes, $P,Q$, are alpha-equivalent if $P = Q\{\vec{y}/\vec{x}\}$ for
some $\vec{x} \in \boundnames{Q},\vec{y} \in \boundnames{P}$, where $Q\{\vec{y}/\vec{x}\}$
denotes the capture-avoiding substitution of $\vec{y}$ for $\vec{x}$ in $Q$.
\end{definition}

\begin{definition}
  The {\em structural congruence} \cite{SangiorgiWalker} , $\equiv$,
  between processes is the least congruence containing
  alpha-equivalence, satisfying the abelian monoid laws
  (associativity, commutativity and $\pzero$ as identity) for parallel
  composition $|$ and for summation $+$.
\end{definition}

\subsection{Name equivalence}

We take name equivalence, written $\nameeq$, to be the smallest
equivalence relation generated by the following rules.

\begin{mathpar}
\inferrule*[lab=Quote-drop]
{ }
{ \quotep{@{x}} \nameeq x }

\inferrule*[lab=Struct-equiv]
{ P \scong Q }
{ \quotep{P} \nameeq \quotep{Q} }
\end{mathpar}

The astute reader will have noticed that the mutual recursion of names
and processes imposes a mutual recursion on alpha-equivalence and
structural equivalence via name-equivalence. Fortunately, all of this
works out pleasantly and we may calculate in the natural way, free of
concern. The reader interested in the details is referred to the
appendix \ref{appendix:rho_details}.

\subsection{Substitution}

We use $\Proc$ for the set of processes, $\QProc$ for the set of
names, and $\id{\{}\vec{y} / \vec{x} \id{\}}$ to denote partial maps,
$s : \QProc \rightarrow \QProc$. A map, $s$ lifts, uniquely, to a map
on process terms, $\widehat{s} : \Proc \rightarrow \Proc$ by the
following equations.

\begin{mathpar}
  (0) \psubstp{Q}{P} := 0 \\
  (R \juxtap S) \psubstp{Q}{P}
  :=    
  (R)\psubstp{Q}{P} \juxtap (S) \psubstp{Q}{P} \\
  (x?(y).R) \psubstp{Q}{P}    
  :=    
  (x)\substp{Q}{P} (z)\concat( (R \psubstn{z}{y}) \psubstp{Q}{P} ) \\
  (\lift{x}{R}) \psubstp{Q}{P}  
  :=
  \lift{(x)\substp{Q}{P}}{ R \psubstp{Q}{P} } \\
%   (\dropn{x})  \psubstp{Q}{P}       
%   := 
%   \left\{ 
%     \begin{array}{ccc} 
%       \dropn{\quotep{Q}} & & x \nameeq \quotep{P} \\
%       \dropn{x} & & otherwise \\
%     \end{array}
%   \right. 
  (\dropn{x})  \psubstp{Q}{P}       
  := 
  \left\{ 
    \begin{array}{ccc} 
      Q & & x \nameeq \quotep{P} \\
      \dropn{x} & & otherwise \\
    \end{array}
  \right.
\end{mathpar}
 

where

\begin{eqnarray}
  (x)\id{\{} \lpquote Q \rpquote / \lpquote P \rpquote \id{\}}            = 
  \left\{ 
    \begin{array}{ccc}
      \lpquote Q \rpquote & & x \nameeq \lpquote P \rpquote \\
      x & & otherwise \\
    \end{array}
  \right. \nonumber
\end{eqnarray}

and $z$ is chosen distinct from $\quotep{P}$, $\quotep{Q}$, the free
names in $Q$, and all the names in $R$. Our $\alpha$-equivalence will
be built in the standard way from this substitution.

\begin{remark}\label{rem:no_self_referential_names}
  One consequence of these definitions is that $\forall P. \quotep{P}
  \not\in \freenames{P}$.
\end{remark}

\subsection{ Dynamic quote: an example }

Anticipating something of what's to come, consider applying the
substitution, $\widehat{\id{\{}u / z \id{\}}}$, to the following pair
of processes, $\lift{w}{y!(z)}$ and $w[ \lpquote y!(z) \rpquote ]$.

\begin{eqnarray}
	\lift{w}{y!(z)}\widehat{\id{\{}u / z \id{\}}}
		& = &
		\lift{w}{y!(u)} \nonumber\\
	w[ \lpquote y!(z) \rpquote ] \widehat{ \id{\{}u / z \id{\}} }
		& = &
		w[ \lpquote y!(z) \rpquote ] \nonumber
\end{eqnarray}

Because the body of the process between quotes is impervious to
substitution, we get radically different answers. In fact, by
examining the first process in an input context,
e.g. $x?(z).\lift{w}{y!(z)}$, we see that the process under the lift
operator may be shaped by prefixed inputs binding a name inside it. In
this sense, the lift operator will be seen as a way to dynamically
construct processes before reifying them as names.

Finally equipped with these standard features we can present the
dynamics of the calculus.

\subsubsection{Operational semantics} 

Finally, we introduce the computational dynamics. What marks these
algebras as distinct from other more traditionally studied algebraic
structures, e.g. vector spaces or polynomial rings, is the manner in
which dynamics is captured. In traditional structures, dynamics is typically
expressed through morphisms between such structures, as in linear maps
between vector spaces or morphisms between rings. In algebras
associated with the semantics of computation, the dynamics is
expressed as part of the algebraic structure itself, through a
reduction reduction relation typically denoted by $\red$. Below, we
give a recursive presentation of this relation for the calculus used
in the encoding.

$\red \subseteq \pi \times \pi$
$\red : \pi \to \mathcal{P}(\pi)$

\begin{mathpar}
  \inferrule* [lab=Comm] { \textsf{match}( x_{src}, x_{trgt} ) } { x_{trgt}?(y)P \; | \; x_{src}!\langle {Q} \rangle \red P\{\quotep{Q}/y}\} }
  \and \\
  \inferrule* [lab=Par] {{P} \red {P}'} {{{P} | {Q}} \red {{P}' | {Q}}}
  \and
  \inferrule* [lab=Equiv]{{{P} \scong {P}'} \andalso {{P}' \red {Q}'} \andalso {{Q}' \scong {Q}}}{{P} \red {Q}}
\end{mathpar}

\begin{eqnarray*}
  match_{\equiv} (\quotep{P},\quotep{Q}) & := & P \equiv Q \\
  match_{\dagger}(\quotep{P},\quotep{Q}) & := & \forall R. P|Q \red^{*} R => R \red^{*} 0 \\
  match_{K}(\quotep{P},\quotep{Q}) & := & K \mbox{ for some context } K
\end{eqnarray*}

$u?(x)P | u!\langle Q \rangle \red P\{\quotep{Q}/x\}$

%We write $\wred$ for $\red^*$, and $P\red$ if $\exists Q $ such that $ P \red Q$.
We write $P\red$ if $\exists Q $ such that $ P \red Q$ and $P\not\red$, otherwise.

\section{Replication}

As mentioned before, it is known that replication (and hence
recursion) can be implemented in a higher-order process algebra
\cite{SangiorgiWalker}. As our first example of calculation with the
machinery thus far presented we give the construction explicitly in
the {\rhoc}.

\begin{eqnarray}
	D_{x} & := & \prefix{x}{y}{(\binpar{\outputp{x}{y}}{@{y}})} \nonumber\\
	\bangp_{x}{P} & := & \binpar{{x}!\langle{\binpar{D_{x}}{P}}\rangle}{D_{x}} \nonumber
\end{eqnarray}

\begin{eqnarray}
	\bangp_{x}{P} & & \nonumber\\
	=
	& {x}!\langle{(\prefix{x}{y}{(\outputp{x}{y} | @{y})) | P}}\rangle 
	      | \prefix{x}{y}{(\outputp{x}{y} | @{y})} & \nonumber\\
	\red
	& (\outputp{x}{y} | @{y})\substn{\quotep{(\prefix{x}{y}{(@{y} | \outputp{x}{y})) | P}}}{y} & \nonumber\\
	=
	& \outputp{x}{\quotep{(\prefix{x}{y}{(\outputp{x}{y} | @{y})) | P}}}
	  | {(\prefix{x}{y}{(\outputp{x}{y} | @{y})) | P}} & \nonumber\\
	\red
	& \ldots & \nonumber\\
	\red^*
	& P | P | \ldots & \nonumber
\end{eqnarray}

Of course, this encoding, as an implementation, runs away, unfolding
$\bangp{P}$ eagerly. A lazier and more implementable replication
operator, restricted to input-guarded processes, may be obtained as follows.

\begin{eqnarray}
\bangp{\prefix{u}{v}{P}} 
	:= 
	\binpar{\lift{x}{\prefix{u}{v}{(\binpar{D(x)}{P})}}}{D(x)} \nonumber
\end{eqnarray}

\begin{remark}
  Note that the lazier definition still does not deal with summation
  or mixed summation (i.e. sums over input and output). The reader is
  invited to construct definitions of replication that deal with these
  features. 

  Further, the definitions are parameterized in a name, $x$. Can you,
  gentle reader, make a definition that eliminates this parameter and
  guarantees no accidental interaction between the replication
  machinery and the process being replicated -- i.e. no accidental
  sharing of names used by the process to get its work done and the
  name(s) used by the replication to effect copying. This latter
  revision of the definition of replication is crucial to obtaining
  the expected identity $!!P \sim !P$.
\end{remark}

\begin{remark}\label{rem:paradoxical_combinator}
  The reader familiar with the lambda calculus will have noticed the
  similarity between $D$ and the paradoxical combinator.

  [Ed. note: the existence of this seems to suggest we have to be more
  restrictive on the set of processes and names we admit if we are to
  support no-cloning.]
\end{remark}

\subsubsection{Bisimulation}

The computational dynamics gives rise to another kind of equivalence,
the equivalence of computational behavior. As previously mentioned
this is typically captured \emph{via} some form of bisimulation.

% The notion we use in this paper is weak barbed bisimulation
% \cite{milner91polyadicpi}.

The notion we use in this paper is derived from weak barbed
bisimulation \cite{milner91polyadicpi}. 

\begin{definition}
An \emph{observation relation}, $\downarrow_{\mathcal N}$, over a set
of names, $\mathcal N$, is the smallest relation satisfying the rules
below.

\infrule[Out-barb]{y \in {\mathcal N}, \; x \nameeq y}
		  {\outputp{x}{v} \downarrow_{\mathcal N} x}
\infrule[Par-barb]{\mbox{$P\downarrow_{\mathcal N} x$ or $Q\downarrow_{\mathcal N} x$}}
		  {\binpar{P}{Q} \downarrow_{\mathcal N} x}

We write $P \Downarrow_{\mathcal N} x$ if there is $Q$ such that 
$P \wred Q$ and $Q \downarrow_{\mathcal N} x$.
\end{definition}

\begin{definition}
%\label{def.bbisim}
An  ${\mathcal N}$-\emph{barbed bisimulation} over a set of names, ${\mathcal N}$, is a symmetric binary relation 
${\mathcal S}_{\mathcal N}$ between agents such that $P\rel{S}_{\mathcal N}Q$ implies:
\begin{enumerate}
\item If $P \red P'$ then $Q \wred Q'$ and $P'\rel{S}_{\mathcal N} Q'$.
\item If $P\downarrow_{\mathcal N} x$, then $Q\Downarrow_{\mathcal N} x$.
\end{enumerate}
$P$ is ${\mathcal N}$-barbed bisimilar to $Q$, written
$P \wbbisim_{\mathcal N} Q$, if $P \rel{S}_{\mathcal N} Q$ for some ${\mathcal N}$-barbed bisimulation ${\mathcal S}_{\mathcal N}$.
\end{definition}

$\mathcal{R} \subseteq \pi \times \pi$

$P \mathcal{R} Q => \forall P'. P \red P' \Rightarrow \exists Q'. Q \red Q', P' \mathcal{R} Q'$

$P \vdash x \Rightarrow Q \vdash x$

\begin{mathpar}
  \inferrule*[lab=Out-barb]{x \nameeq y}{{y}!\langle{Q}\rangle \vdash x}
  \and
  \inferrule*[lab=Par-barb]{\mbox{$P\vdash x$ or $Q\vdash x$}}{\binpar{P}{Q} \vdash x}
\end{mathpar}

\subsubsection{Contexts}

One of the principle advantages of computational calculi like the
$\pi$-calculus is a well-defined notion of context,
contextual-equivalence and a correlation between
contextual-equivalence and notions of bisimulation. The notion of
context allows the decomposition of a process into (sub-)process and
its syntactic environment, its context. Thus, a context may be
thought of as a process with a ``hole'' (written $\Box$) in it. The
application of a context $M$ to a process $P$, written $M[P]$, is
tantamount to filling the hole in $M$ with $P$. In this paper we do
not need the full weight of this theory, but do make use of the notion
of context in the proof the main theorem. 

\begin{mathpar}
  \inferrule* [lab=summation] {} {{M_{M},M_{N}} \bc \Box \;|\; x.M_{A} \;|\; M_{M}+M_{N}}
  \and
  \inferrule* [lab=agent] {} {{M_{A}} \bc (\vec{x})M_{P} \;| \; \clift{P_0,\ldots,M_{P},\ldots,P_N}}
  \and \\
  \inferrule* [lab=process] {} {{M_{P}} \bc M_{N} \;| \;P|M_{P} }
\end{mathpar} 

\begin{mathpar}
  \inferrule* [lab=sychronization] {} {M_{N} \bc \Box \;|\; x?M_{F} \;|\; x!M_{C}}
  \and
  \inferrule* [lab=abstraction] {} {{M_{F}} \bc (x)M_{P} }
  \and
  \inferrule* [lab=concretion] {} {{M_{C}} \bc \langle M_{P} \rangle }
  \and \\
  \inferrule* [lab=process] {} {{M_{P}} \bc M_{N} \;| \;P|M_{P} }
\end{mathpar}

\begin{definition}[contextual application] Given a context $M$, and
  process $P$, we define the \emph{contextual application}, $M[P] :=
  M\{P/\Box\}$. That is, the contextual application of M to P is the
  substitution of $P$ for $\Box$ in $M$.
\end{definition}

$\meaningof{-} : L \to \mathcal{P}(\pi)$

\begin{mathpar}
  \inferrule* [lab=collection] {} {\meaningof{true} = \pi, \and \meaningof{~E} = \pi \setminus \meaningof{E}, \and \meaningof{E_{1} \& E_{2}} = \meaningof{E_{1}} \cap \meaningof{E_{2}}}
\end{mathpar}

\begin{mathpar}
  \inferrule* [lab=structure] {} {\meaningof{0} = \{ P \in \pi | P \equiv 0 \}, \and \\ \meaningof{E_1 | E_2} = \{ P \in \pi | P \equiv P_{1} | P_{2}, P_{1} \in \meaningof{E_{1}}, P_{2} \in \meaningof{E_2}\} }
\end{mathpar}

\begin{mathpar}
 \inferrule* [lab=behavior] {} {\meaningof{\langle a?b \rangle E} = \{ P \in \pi | P \equiv Q | u?(y)P', \\ \and \\\\ \and \\ \;\;\; u \in \meaningof{a}, \forall z.P'\{z/y\} \in \meaningof{E\{z/b\}}\}, \and \\ \meaningof{a!E} = \{ P \in \pi | P \equiv Q | x!\langle P' \rangle, x \in \meaningof{a} P' \in \meaningof{E}\} }
\end{mathpar}

\begin{mathpar}
 \inferrule* [lab=nominal] {} {\meaningof{\quotep{E}} = \{ \quotep{P} \in \quotep{\pi} | P \in \meaningof{E} \}, \and \meaningof{\quotep{P}} = \{ \quotep{Q} \in \quotep{\pi} | P \equiv Q \} \and \\ \meaningof{@\quotep{E}} = \{ P \in \pi | P \equiv @x, x \in \meaningof{E} \}}
\end{mathpar}

\begin{eqnarray*}
  \\
  \meaningof{-} : TS \to ST
\end{eqnarray*}

\begin{eqnarray*}
  \\
  L : TS \to ST
\end{eqnarray*}

\begin{eqnarray*}
  \\
  P \models E \iff P \in \meaningof{E}
\end{eqnarray*}

\begin{eqnarray*}
  P \approx_{L} Q \iff \forall E \in L. P \models E \iff Q \models E
\end{eqnarray*}

\begin{eqnarray*}
  P \approx_{K} Q
\end{eqnarray*}

\begin{eqnarray*}
  P \approx Q
\end{eqnarray*}

$\approx_{K} = \approx = \approx_{L}$

\subsubsection{Contextual duality}

Note that contexts extend the quotation operation to a family of
operations from processes to names. Given a context, $M$, we can
define a \emph{nominal context}, $\quotep{M}$ by $\quotep{M}[P] :=
\quotep{M[P]}$. To foreshadow what is to come we observe that these
operations enjoy a duality with processes very much like the duality
between vectors and maps from vectors to scalars.

Further, because the calculus is essentially higher-order, we have a
correspondence between contexts and processes. More specifically,
given a name $x$ and a context $M$ we can construct $M^{*}_{x}$ such
that 

\begin{mathpar}
  M^{*}_{x} | \lift{x}{P} \red M[P]
\end{mathpar}

namely,

\begin{mathpar}
  M^{*}_{x} := x?(u).M[\dropn{u}]
\end{mathpar}

The dependence of $M^{*}_{x}$ on a name makes it an abstraction, 

\begin{mathpar}
  M^{*} := (x)x?(u).M[\dropn{u}]
\end{mathpar}

\subsection{Additional notation}

It will sometimes be convenient to denote the process a name
quotes. We already have the notation $x = \quotep{P}$, but it will be
convenient to introduce an alternate notation, $\procn{x}$, when we
want to emphasize the connection to the use of the name. Note that, by
virtue of name equivalence, $\quotep{\procn{x}} \nameeq x$; so, the
notation is consistent with previous definitions.

Further, because names have structure it is possible to effect
substitutions on the basis of that structure. This means we need to
upgrade our notation for substitutions, which we accomplish by
adapting comprehension notation. Thus,

\begin{mathpar}
  P\{ y / x : x \in S \}
\end{mathpar}

is interpreted to mean the process derived from P by replacing (in a
capture-avoiding manner) each occurrence of $x$ in $S$ by $y$. For example,

\begin{mathpar}
  P\{ \quotep{\procn{x}|\procn{x}} / x : x \in \freenames{P} \}
\end{mathpar}

will replace each (occurrence) of a free name $x$ in $P$ by
$\quotep{\procn{x}|\procn{x}}$.

Also, we will avail ourselves of the notation $x^{L}$ and $x^{R}$ to
denote injections of a name into disjoint copies of the name
space. There are numerous ways to accomplish this. One example can be
found in \cite{MeredithR05}. This notation overloads to vectors of
names: $\vec{x}^{\pi} := (x_{i}^{\pi} \; : \; 0 \leq i < |\vec{x}| )$ where $\pi \in \{L,R\}$.

We also use $P^{\Box} := P|\Box$.

In \cite{MeredithR05} an interpretation of the new operator is
given. It turns out that there are several possible interpretations
all enjoying the requisite algebraic properties of the operator (see
\cite{milner91polyadicpi}). We will therefore make liberal use of
$(\nu\; \vec{x})P$.

% subsection the_syntax_and_semantics_of_the_notation_system (end)   

\input{qm2pi.qmops} 

\input{qm2pi.sterngerlach} 

\input{qm2pi.metric} 

% section concurrent_process_calculi (end)

%\input{qm2pi.proofsketch}

% section proof sketch (end)

%\input{qm2pi.slviaknots} 

% section spatial logic via knots (end)

\input{qm2pi.conclusion}

% section conclusion (end)

%\input{qm2pi.dtcodes} 

% section wiring algorithm (end)

\input{qm2pi.ack} 

% section acknowledgments (end)

\newpage


\bibliographystyle{plain}   
\bibliography{../../biblios/main.bib}

\input{qm2pi.rhodetails}

\end{document}

 

%\documentclass[12pt]{llncs}
%\documentclass{jktr}

\usepackage[pdftex]{hyperref}                   
\usepackage {listings}
\usepackage {mathpartir}
\usepackage{bcprules}
%\usepackage{listings}
                       
\usepackage{graphicx} 
%\usepackage[margins=2.5cm,nohead,nofoot]{geometry}
%\usepackage{geometry}
\usepackage{amsfonts}
\usepackage{amstext}
\usepackage{latexsym}
\usepackage{amssymb}
\usepackage{color}


%\include{myPreamble}
\include{qm2pi.local} 

%\ifpdf
%\usepackage[pdftex]{graphicx}
%\else
%\usepackage{graphicx}
%\fi

 % \ifpdf
%  \usepackage{pdfsync}
%  \if


%\title{Brief Article}
%\author{David F. Snyder}
%\author{L.G. Meredith}

%\address{Dept. of Math., Texas State University--San Marcos, San Marcos, TX 78666}
       
\pagestyle{empty}


\begin{document}

\lstset{language=[Objective]Caml,frame=shadowbox}

\input{qm2pi.front}

% section front matter (end)

\input{qm2pi.intro} 
 
% section introduction (end)

% \input{qm2pi.knotations} 

% section notation (end)

\input{qm2pi.process.calculi} 

% section concurrent_process_calculi_and_spatial_logics_ (end)
    
%\input{qm2pi.knots2pi} 

%\input{qm2pi.trefoil} 

%\input{qm2pi.mainthm} 

% subsection basic_interpretation (end)

%\input{qm2pi.rho.presentation} 
\subsection{The syntax and semantics of the notation system}\label{sub:the_syntax_and_semantics_of_the_notation_system} % (fold)

We now summarize a technical presentation of the calculus that
embodies our theory of dynamics. The typical presentation of such a
calculus follows the style of giving generators and relations on
them. The grammar, below, describing term constructors, freely
generates the set of processes, $\Proc$. This set is then quotiented
by a relation known as structural congruence and it is over this set
that the notion of dynamics is expressed. This presentation is
essentially that of \cite{MeredithR05} with the addition of
polyadicity and summation. For readability we have relegated some of
the technical subtleties to an appendix.

\subsubsection{Process grammar}\label{subsub:process_grammar}

\begin{mathpar}
  \inferrule* [lab=synchronization] {} {{M} \bc \pzero \;|\; x?F \;|\; x!C }
  \and
  \inferrule* [lab=abstraction] {} {{F} \bc (x)P}
  \and
  \inferrule* [lab=concretion] {} {{C} \bc \langle Q \rangle}
  \and
  \inferrule* [lab=process] {} {{P,Q} \bc M \;| \;P|Q \;|\; @{x}}
  \and
  \inferrule* [lab=name] {} {{x} \bc \quotep{P}}
\end{mathpar} 

Note that $\vec{x}$ (resp. $\vec{P}$) denotes a vector of names
(resp. processes) of length $|\vec{x}|$ (resp. $|\vec{P}|$). We adopt
the following useful abbreviations.

\begin{mathpar}
   x?(\vec{y}).P := x.(\vec{y})P \and  x\clift{\vec{P}} := x.\clift{\vec{P}}
   \and x!(y) := \lift{x}{\dropn{y}}
   \and \Pi_{i=0}^{n-1}P_i := P_0 | \ldots | P_{n-1}
\end{mathpar}

\subsubsection{Structural congruence}

\paragraph{Free and bound names and alpha-equivalence.} At the
core of structural equivalence is alpha-equivalence which identifies
process that are the same up to a change of variable. Formally, we
recognize the distinction between free and bound names. The free names
of a process, $\freenames{P}$, may be calculated recursively as
follows:

\begin{mathpar}
\freenames{\pzero} := \emptyset
  \and \\
  \freenames{x?(y).P} := \{ x \} \cup (\freenames{P} \setminus \{ y \})
  \and 
  \freenames{x!\langle P \rangle} := \{ x \} \cup \{ P \} 
  \and \\
  \freenames{P|Q} := \freenames{P} \cup \freenames{Q}
  \and \\
  \freenames{@{x}} := \{ x \}
\end{mathpar}

$\pi$
$\quotep{\pi}$

$\freenames{-} : \pi \to \mathcal{P}(\quotep{\pi})$

\begin{eqnarray*}
  \freenames{\pzero} & := & \emptyset \\
  \freenames{x?(y).P} & := & \{ x \} \cup (\freenames{P} \setminus \{ y \}) \\
  \freenames{x!\langle P \rangle} & := & \{ x \} \cup \{ P \} \\
  \freenames{P|Q} & := & \freenames{P} \cup \freenames{Q} \\
  \freenames{\dropn{x}} & := & \{ x \}
\end{eqnarray*}

The bound names of a process, $\boundnames{P}$, are those names occurring in $P$
that are not free. For example, in $x?(y).0$, the name $x$ is free, while $y$ is bound.

\begin{mathpar}
  \inferrule* [lab=monoidal-laws] {} { P|Q \equiv Q|P \and P|0 \equiv P \and P|(Q|R) \equiv (P|Q)|R }
\end{mathpar}

\begin{mathpar}
  \inferrule* [lab=alpha-equivalence] {} { (x)P \equiv (y)P\{y/x\} \and y \not\in \freenames{P} }
\end{mathpar}

\begin{definition}
Then two processes, $P,Q$, are alpha-equivalent if $P = Q\{\vec{y}/\vec{x}\}$ for
some $\vec{x} \in \boundnames{Q},\vec{y} \in \boundnames{P}$, where $Q\{\vec{y}/\vec{x}\}$
denotes the capture-avoiding substitution of $\vec{y}$ for $\vec{x}$ in $Q$.
\end{definition}

\begin{definition}
  The {\em structural congruence} \cite{SangiorgiWalker} , $\equiv$,
  between processes is the least congruence containing
  alpha-equivalence, satisfying the abelian monoid laws
  (associativity, commutativity and $\pzero$ as identity) for parallel
  composition $|$ and for summation $+$.
\end{definition}

\subsection{Name equivalence}

We take name equivalence, written $\nameeq$, to be the smallest
equivalence relation generated by the following rules.

\begin{mathpar}
\inferrule*[lab=Quote-drop]
{ }
{ \quotep{@{x}} \nameeq x }

\inferrule*[lab=Struct-equiv]
{ P \scong Q }
{ \quotep{P} \nameeq \quotep{Q} }
\end{mathpar}

The astute reader will have noticed that the mutual recursion of names
and processes imposes a mutual recursion on alpha-equivalence and
structural equivalence via name-equivalence. Fortunately, all of this
works out pleasantly and we may calculate in the natural way, free of
concern. The reader interested in the details is referred to the
appendix \ref{appendix:rho_details}.

\subsection{Substitution}

We use $\Proc$ for the set of processes, $\QProc$ for the set of
names, and $\id{\{}\vec{y} / \vec{x} \id{\}}$ to denote partial maps,
$s : \QProc \rightarrow \QProc$. A map, $s$ lifts, uniquely, to a map
on process terms, $\widehat{s} : \Proc \rightarrow \Proc$ by the
following equations.

\begin{mathpar}
  (0) \psubstp{Q}{P} := 0 \\
  (R \juxtap S) \psubstp{Q}{P}
  :=    
  (R)\psubstp{Q}{P} \juxtap (S) \psubstp{Q}{P} \\
  (x?(y).R) \psubstp{Q}{P}    
  :=    
  (x)\substp{Q}{P} (z)\concat( (R \psubstn{z}{y}) \psubstp{Q}{P} ) \\
  (\lift{x}{R}) \psubstp{Q}{P}  
  :=
  \lift{(x)\substp{Q}{P}}{ R \psubstp{Q}{P} } \\
%   (\dropn{x})  \psubstp{Q}{P}       
%   := 
%   \left\{ 
%     \begin{array}{ccc} 
%       \dropn{\quotep{Q}} & & x \nameeq \quotep{P} \\
%       \dropn{x} & & otherwise \\
%     \end{array}
%   \right. 
  (\dropn{x})  \psubstp{Q}{P}       
  := 
  \left\{ 
    \begin{array}{ccc} 
      Q & & x \nameeq \quotep{P} \\
      \dropn{x} & & otherwise \\
    \end{array}
  \right.
\end{mathpar}
 

where

\begin{eqnarray}
  (x)\id{\{} \lpquote Q \rpquote / \lpquote P \rpquote \id{\}}            = 
  \left\{ 
    \begin{array}{ccc}
      \lpquote Q \rpquote & & x \nameeq \lpquote P \rpquote \\
      x & & otherwise \\
    \end{array}
  \right. \nonumber
\end{eqnarray}

and $z$ is chosen distinct from $\quotep{P}$, $\quotep{Q}$, the free
names in $Q$, and all the names in $R$. Our $\alpha$-equivalence will
be built in the standard way from this substitution.

\begin{remark}\label{rem:no_self_referential_names}
  One consequence of these definitions is that $\forall P. \quotep{P}
  \not\in \freenames{P}$.
\end{remark}

\subsection{ Dynamic quote: an example }

Anticipating something of what's to come, consider applying the
substitution, $\widehat{\id{\{}u / z \id{\}}}$, to the following pair
of processes, $\lift{w}{y!(z)}$ and $w[ \lpquote y!(z) \rpquote ]$.

\begin{eqnarray}
	\lift{w}{y!(z)}\widehat{\id{\{}u / z \id{\}}}
		& = &
		\lift{w}{y!(u)} \nonumber\\
	w[ \lpquote y!(z) \rpquote ] \widehat{ \id{\{}u / z \id{\}} }
		& = &
		w[ \lpquote y!(z) \rpquote ] \nonumber
\end{eqnarray}

Because the body of the process between quotes is impervious to
substitution, we get radically different answers. In fact, by
examining the first process in an input context,
e.g. $x?(z).\lift{w}{y!(z)}$, we see that the process under the lift
operator may be shaped by prefixed inputs binding a name inside it. In
this sense, the lift operator will be seen as a way to dynamically
construct processes before reifying them as names.

Finally equipped with these standard features we can present the
dynamics of the calculus.

\subsubsection{Operational semantics} 

Finally, we introduce the computational dynamics. What marks these
algebras as distinct from other more traditionally studied algebraic
structures, e.g. vector spaces or polynomial rings, is the manner in
which dynamics is captured. In traditional structures, dynamics is typically
expressed through morphisms between such structures, as in linear maps
between vector spaces or morphisms between rings. In algebras
associated with the semantics of computation, the dynamics is
expressed as part of the algebraic structure itself, through a
reduction reduction relation typically denoted by $\red$. Below, we
give a recursive presentation of this relation for the calculus used
in the encoding.

$\red \subseteq \pi \times \pi$
$\red : \pi \to \mathcal{P}(\pi)$

\begin{mathpar}
  \inferrule* [lab=Comm] { \textsf{match}( x_{src}, x_{trgt} ) } { x_{trgt}?(y)P \; | \; x_{src}!\langle {Q} \rangle \red P\{\quotep{Q}/y}\} }
  \and \\
  \inferrule* [lab=Par] {{P} \red {P}'} {{{P} | {Q}} \red {{P}' | {Q}}}
  \and
  \inferrule* [lab=Equiv]{{{P} \scong {P}'} \andalso {{P}' \red {Q}'} \andalso {{Q}' \scong {Q}}}{{P} \red {Q}}
\end{mathpar}

\begin{eqnarray*}
  match_{\equiv} (\quotep{P},\quotep{Q}) & := & P \equiv Q \\
  match_{\dagger}(\quotep{P},\quotep{Q}) & := & \forall R. P|Q \red^{*} R => R \red^{*} 0 \\
  match_{K}(\quotep{P},\quotep{Q}) & := & K \mbox{ for some context } K
\end{eqnarray*}

$u?(x)P | u!\langle Q \rangle \red P\{\quotep{Q}/x\}$

%We write $\wred$ for $\red^*$, and $P\red$ if $\exists Q $ such that $ P \red Q$.
We write $P\red$ if $\exists Q $ such that $ P \red Q$ and $P\not\red$, otherwise.

\section{Replication}

As mentioned before, it is known that replication (and hence
recursion) can be implemented in a higher-order process algebra
\cite{SangiorgiWalker}. As our first example of calculation with the
machinery thus far presented we give the construction explicitly in
the {\rhoc}.

\begin{eqnarray}
	D_{x} & := & \prefix{x}{y}{(\binpar{\outputp{x}{y}}{@{y}})} \nonumber\\
	\bangp_{x}{P} & := & \binpar{{x}!\langle{\binpar{D_{x}}{P}}\rangle}{D_{x}} \nonumber
\end{eqnarray}

\begin{eqnarray}
	\bangp_{x}{P} & & \nonumber\\
	=
	& {x}!\langle{(\prefix{x}{y}{(\outputp{x}{y} | @{y})) | P}}\rangle 
	      | \prefix{x}{y}{(\outputp{x}{y} | @{y})} & \nonumber\\
	\red
	& (\outputp{x}{y} | @{y})\substn{\quotep{(\prefix{x}{y}{(@{y} | \outputp{x}{y})) | P}}}{y} & \nonumber\\
	=
	& \outputp{x}{\quotep{(\prefix{x}{y}{(\outputp{x}{y} | @{y})) | P}}}
	  | {(\prefix{x}{y}{(\outputp{x}{y} | @{y})) | P}} & \nonumber\\
	\red
	& \ldots & \nonumber\\
	\red^*
	& P | P | \ldots & \nonumber
\end{eqnarray}

Of course, this encoding, as an implementation, runs away, unfolding
$\bangp{P}$ eagerly. A lazier and more implementable replication
operator, restricted to input-guarded processes, may be obtained as follows.

\begin{eqnarray}
\bangp{\prefix{u}{v}{P}} 
	:= 
	\binpar{\lift{x}{\prefix{u}{v}{(\binpar{D(x)}{P})}}}{D(x)} \nonumber
\end{eqnarray}

\begin{remark}
  Note that the lazier definition still does not deal with summation
  or mixed summation (i.e. sums over input and output). The reader is
  invited to construct definitions of replication that deal with these
  features. 

  Further, the definitions are parameterized in a name, $x$. Can you,
  gentle reader, make a definition that eliminates this parameter and
  guarantees no accidental interaction between the replication
  machinery and the process being replicated -- i.e. no accidental
  sharing of names used by the process to get its work done and the
  name(s) used by the replication to effect copying. This latter
  revision of the definition of replication is crucial to obtaining
  the expected identity $!!P \sim !P$.
\end{remark}

\begin{remark}\label{rem:paradoxical_combinator}
  The reader familiar with the lambda calculus will have noticed the
  similarity between $D$ and the paradoxical combinator.

  [Ed. note: the existence of this seems to suggest we have to be more
  restrictive on the set of processes and names we admit if we are to
  support no-cloning.]
\end{remark}

\subsubsection{Bisimulation}

The computational dynamics gives rise to another kind of equivalence,
the equivalence of computational behavior. As previously mentioned
this is typically captured \emph{via} some form of bisimulation.

% The notion we use in this paper is weak barbed bisimulation
% \cite{milner91polyadicpi}.

The notion we use in this paper is derived from weak barbed
bisimulation \cite{milner91polyadicpi}. 

\begin{definition}
An \emph{observation relation}, $\downarrow_{\mathcal N}$, over a set
of names, $\mathcal N$, is the smallest relation satisfying the rules
below.

\infrule[Out-barb]{y \in {\mathcal N}, \; x \nameeq y}
		  {\outputp{x}{v} \downarrow_{\mathcal N} x}
\infrule[Par-barb]{\mbox{$P\downarrow_{\mathcal N} x$ or $Q\downarrow_{\mathcal N} x$}}
		  {\binpar{P}{Q} \downarrow_{\mathcal N} x}

We write $P \Downarrow_{\mathcal N} x$ if there is $Q$ such that 
$P \wred Q$ and $Q \downarrow_{\mathcal N} x$.
\end{definition}

\begin{definition}
%\label{def.bbisim}
An  ${\mathcal N}$-\emph{barbed bisimulation} over a set of names, ${\mathcal N}$, is a symmetric binary relation 
${\mathcal S}_{\mathcal N}$ between agents such that $P\rel{S}_{\mathcal N}Q$ implies:
\begin{enumerate}
\item If $P \red P'$ then $Q \wred Q'$ and $P'\rel{S}_{\mathcal N} Q'$.
\item If $P\downarrow_{\mathcal N} x$, then $Q\Downarrow_{\mathcal N} x$.
\end{enumerate}
$P$ is ${\mathcal N}$-barbed bisimilar to $Q$, written
$P \wbbisim_{\mathcal N} Q$, if $P \rel{S}_{\mathcal N} Q$ for some ${\mathcal N}$-barbed bisimulation ${\mathcal S}_{\mathcal N}$.
\end{definition}

$\mathcal{R} \subseteq \pi \times \pi$

$P \mathcal{R} Q => \forall P'. P \red P' \Rightarrow \exists Q'. Q \red Q', P' \mathcal{R} Q'$

$P \vdash x \Rightarrow Q \vdash x$

\begin{mathpar}
  \inferrule*[lab=Out-barb]{x \nameeq y}{{y}!\langle{Q}\rangle \vdash x}
  \and
  \inferrule*[lab=Par-barb]{\mbox{$P\vdash x$ or $Q\vdash x$}}{\binpar{P}{Q} \vdash x}
\end{mathpar}

\subsubsection{Contexts}

One of the principle advantages of computational calculi like the
$\pi$-calculus is a well-defined notion of context,
contextual-equivalence and a correlation between
contextual-equivalence and notions of bisimulation. The notion of
context allows the decomposition of a process into (sub-)process and
its syntactic environment, its context. Thus, a context may be
thought of as a process with a ``hole'' (written $\Box$) in it. The
application of a context $M$ to a process $P$, written $M[P]$, is
tantamount to filling the hole in $M$ with $P$. In this paper we do
not need the full weight of this theory, but do make use of the notion
of context in the proof the main theorem. 

\begin{mathpar}
  \inferrule* [lab=summation] {} {{M_{M},M_{N}} \bc \Box \;|\; x.M_{A} \;|\; M_{M}+M_{N}}
  \and
  \inferrule* [lab=agent] {} {{M_{A}} \bc (\vec{x})M_{P} \;| \; \clift{P_0,\ldots,M_{P},\ldots,P_N}}
  \and \\
  \inferrule* [lab=process] {} {{M_{P}} \bc M_{N} \;| \;P|M_{P} }
\end{mathpar} 

\begin{mathpar}
  \inferrule* [lab=sychronization] {} {M_{N} \bc \Box \;|\; x?M_{F} \;|\; x!M_{C}}
  \and
  \inferrule* [lab=abstraction] {} {{M_{F}} \bc (x)M_{P} }
  \and
  \inferrule* [lab=concretion] {} {{M_{C}} \bc \langle M_{P} \rangle }
  \and \\
  \inferrule* [lab=process] {} {{M_{P}} \bc M_{N} \;| \;P|M_{P} }
\end{mathpar}

\begin{definition}[contextual application] Given a context $M$, and
  process $P$, we define the \emph{contextual application}, $M[P] :=
  M\{P/\Box\}$. That is, the contextual application of M to P is the
  substitution of $P$ for $\Box$ in $M$.
\end{definition}

$\meaningof{-} : L \to \mathcal{P}(\pi)$

\begin{mathpar}
  \inferrule* [lab=collection] {} {\meaningof{true} = \pi, \and \meaningof{~E} = \pi \setminus \meaningof{E}, \and \meaningof{E_{1} \& E_{2}} = \meaningof{E_{1}} \cap \meaningof{E_{2}}}
\end{mathpar}

\begin{mathpar}
  \inferrule* [lab=structure] {} {\meaningof{0} = \{ P \in \pi | P \equiv 0 \}, \and \\ \meaningof{E_1 | E_2} = \{ P \in \pi | P \equiv P_{1} | P_{2}, P_{1} \in \meaningof{E_{1}}, P_{2} \in \meaningof{E_2}\} }
\end{mathpar}

\begin{mathpar}
 \inferrule* [lab=behavior] {} {\meaningof{\langle a?b \rangle E} = \{ P \in \pi | P \equiv Q | u?(y)P', \\ \and \\\\ \and \\ \;\;\; u \in \meaningof{a}, \forall z.P'\{z/y\} \in \meaningof{E\{z/b\}}\}, \and \\ \meaningof{a!E} = \{ P \in \pi | P \equiv Q | x!\langle P' \rangle, x \in \meaningof{a} P' \in \meaningof{E}\} }
\end{mathpar}

\begin{mathpar}
 \inferrule* [lab=nominal] {} {\meaningof{\quotep{E}} = \{ \quotep{P} \in \quotep{\pi} | P \in \meaningof{E} \}, \and \meaningof{\quotep{P}} = \{ \quotep{Q} \in \quotep{\pi} | P \equiv Q \} \and \\ \meaningof{@\quotep{E}} = \{ P \in \pi | P \equiv @x, x \in \meaningof{E} \}}
\end{mathpar}

\begin{eqnarray*}
  \\
  \meaningof{-} : TS \to ST
\end{eqnarray*}

\begin{eqnarray*}
  \\
  L : TS \to ST
\end{eqnarray*}

\begin{eqnarray*}
  \\
  P \models E \iff P \in \meaningof{E}
\end{eqnarray*}

\begin{eqnarray*}
  P \approx_{L} Q \iff \forall E \in L. P \models E \iff Q \models E
\end{eqnarray*}

\begin{eqnarray*}
  P \approx_{K} Q
\end{eqnarray*}

\begin{eqnarray*}
  P \approx Q
\end{eqnarray*}

$\approx_{K} = \approx = \approx_{L}$

\subsubsection{Contextual duality}

Note that contexts extend the quotation operation to a family of
operations from processes to names. Given a context, $M$, we can
define a \emph{nominal context}, $\quotep{M}$ by $\quotep{M}[P] :=
\quotep{M[P]}$. To foreshadow what is to come we observe that these
operations enjoy a duality with processes very much like the duality
between vectors and maps from vectors to scalars.

Further, because the calculus is essentially higher-order, we have a
correspondence between contexts and processes. More specifically,
given a name $x$ and a context $M$ we can construct $M^{*}_{x}$ such
that 

\begin{mathpar}
  M^{*}_{x} | \lift{x}{P} \red M[P]
\end{mathpar}

namely,

\begin{mathpar}
  M^{*}_{x} := x?(u).M[\dropn{u}]
\end{mathpar}

The dependence of $M^{*}_{x}$ on a name makes it an abstraction, 

\begin{mathpar}
  M^{*} := (x)x?(u).M[\dropn{u}]
\end{mathpar}

\subsection{Additional notation}

It will sometimes be convenient to denote the process a name
quotes. We already have the notation $x = \quotep{P}$, but it will be
convenient to introduce an alternate notation, $\procn{x}$, when we
want to emphasize the connection to the use of the name. Note that, by
virtue of name equivalence, $\quotep{\procn{x}} \nameeq x$; so, the
notation is consistent with previous definitions.

Further, because names have structure it is possible to effect
substitutions on the basis of that structure. This means we need to
upgrade our notation for substitutions, which we accomplish by
adapting comprehension notation. Thus,

\begin{mathpar}
  P\{ y / x : x \in S \}
\end{mathpar}

is interpreted to mean the process derived from P by replacing (in a
capture-avoiding manner) each occurrence of $x$ in $S$ by $y$. For example,

\begin{mathpar}
  P\{ \quotep{\procn{x}|\procn{x}} / x : x \in \freenames{P} \}
\end{mathpar}

will replace each (occurrence) of a free name $x$ in $P$ by
$\quotep{\procn{x}|\procn{x}}$.

Also, we will avail ourselves of the notation $x^{L}$ and $x^{R}$ to
denote injections of a name into disjoint copies of the name
space. There are numerous ways to accomplish this. One example can be
found in \cite{MeredithR05}. This notation overloads to vectors of
names: $\vec{x}^{\pi} := (x_{i}^{\pi} \; : \; 0 \leq i < |\vec{x}| )$ where $\pi \in \{L,R\}$.

We also use $P^{\Box} := P|\Box$.

In \cite{MeredithR05} an interpretation of the new operator is
given. It turns out that there are several possible interpretations
all enjoying the requisite algebraic properties of the operator (see
\cite{milner91polyadicpi}). We will therefore make liberal use of
$(\nu\; \vec{x})P$.

% subsection the_syntax_and_semantics_of_the_notation_system (end)   

\input{qm2pi.qmops} 

\input{qm2pi.sterngerlach} 

\input{qm2pi.metric} 

% section concurrent_process_calculi (end)

%\input{qm2pi.proofsketch}

% section proof sketch (end)

%\input{qm2pi.slviaknots} 

% section spatial logic via knots (end)

\input{qm2pi.conclusion}

% section conclusion (end)

%\input{qm2pi.dtcodes} 

% section wiring algorithm (end)

\input{qm2pi.ack} 

% section acknowledgments (end)

\newpage


\bibliographystyle{plain}   
\bibliography{../../biblios/main.bib}

\input{qm2pi.rhodetails}

\end{document}

 

% subsection basic_interpretation (end)

%\input{qm2pi.rho.presentation} 
\subsection{The syntax and semantics of the notation system}\label{sub:the_syntax_and_semantics_of_the_notation_system} % (fold)

We now summarize a technical presentation of the calculus that
embodies our theory of dynamics. The typical presentation of such a
calculus follows the style of giving generators and relations on
them. The grammar, below, describing term constructors, freely
generates the set of processes, $\Proc$. This set is then quotiented
by a relation known as structural congruence and it is over this set
that the notion of dynamics is expressed. This presentation is
essentially that of \cite{MeredithR05} with the addition of
polyadicity and summation. For readability we have relegated some of
the technical subtleties to an appendix.

\subsubsection{Process grammar}\label{subsub:process_grammar}

\begin{mathpar}
  \inferrule* [lab=synchronization] {} {{M} \bc \pzero \;|\; x?F \;|\; x!C }
  \and
  \inferrule* [lab=abstraction] {} {{F} \bc (x)P}
  \and
  \inferrule* [lab=concretion] {} {{C} \bc \langle Q \rangle}
  \and
  \inferrule* [lab=process] {} {{P,Q} \bc M \;| \;P|Q \;|\; @{x}}
  \and
  \inferrule* [lab=name] {} {{x} \bc \quotep{P}}
\end{mathpar} 

Note that $\vec{x}$ (resp. $\vec{P}$) denotes a vector of names
(resp. processes) of length $|\vec{x}|$ (resp. $|\vec{P}|$). We adopt
the following useful abbreviations.

\begin{mathpar}
   x?(\vec{y}).P := x.(\vec{y})P \and  x\clift{\vec{P}} := x.\clift{\vec{P}}
   \and x!(y) := \lift{x}{\dropn{y}}
   \and \Pi_{i=0}^{n-1}P_i := P_0 | \ldots | P_{n-1}
\end{mathpar}

\subsubsection{Structural congruence}

\paragraph{Free and bound names and alpha-equivalence.} At the
core of structural equivalence is alpha-equivalence which identifies
process that are the same up to a change of variable. Formally, we
recognize the distinction between free and bound names. The free names
of a process, $\freenames{P}$, may be calculated recursively as
follows:

\begin{mathpar}
\freenames{\pzero} := \emptyset
  \and \\
  \freenames{x?(y).P} := \{ x \} \cup (\freenames{P} \setminus \{ y \})
  \and 
  \freenames{x!\langle P \rangle} := \{ x \} \cup \{ P \} 
  \and \\
  \freenames{P|Q} := \freenames{P} \cup \freenames{Q}
  \and \\
  \freenames{@{x}} := \{ x \}
\end{mathpar}

$\pi$
$\quotep{\pi}$

$\freenames{-} : \pi \to \mathcal{P}(\quotep{\pi})$

\begin{eqnarray*}
  \freenames{\pzero} & := & \emptyset \\
  \freenames{x?(y).P} & := & \{ x \} \cup (\freenames{P} \setminus \{ y \}) \\
  \freenames{x!\langle P \rangle} & := & \{ x \} \cup \{ P \} \\
  \freenames{P|Q} & := & \freenames{P} \cup \freenames{Q} \\
  \freenames{\dropn{x}} & := & \{ x \}
\end{eqnarray*}

The bound names of a process, $\boundnames{P}$, are those names occurring in $P$
that are not free. For example, in $x?(y).0$, the name $x$ is free, while $y$ is bound.

\begin{mathpar}
  \inferrule* [lab=monoidal-laws] {} { P|Q \equiv Q|P \and P|0 \equiv P \and P|(Q|R) \equiv (P|Q)|R }
\end{mathpar}

\begin{mathpar}
  \inferrule* [lab=alpha-equivalence] {} { (x)P \equiv (y)P\{y/x\} \and y \not\in \freenames{P} }
\end{mathpar}

\begin{definition}
Then two processes, $P,Q$, are alpha-equivalent if $P = Q\{\vec{y}/\vec{x}\}$ for
some $\vec{x} \in \boundnames{Q},\vec{y} \in \boundnames{P}$, where $Q\{\vec{y}/\vec{x}\}$
denotes the capture-avoiding substitution of $\vec{y}$ for $\vec{x}$ in $Q$.
\end{definition}

\begin{definition}
  The {\em structural congruence} \cite{SangiorgiWalker} , $\equiv$,
  between processes is the least congruence containing
  alpha-equivalence, satisfying the abelian monoid laws
  (associativity, commutativity and $\pzero$ as identity) for parallel
  composition $|$ and for summation $+$.
\end{definition}

\subsection{Name equivalence}

We take name equivalence, written $\nameeq$, to be the smallest
equivalence relation generated by the following rules.

\begin{mathpar}
\inferrule*[lab=Quote-drop]
{ }
{ \quotep{@{x}} \nameeq x }

\inferrule*[lab=Struct-equiv]
{ P \scong Q }
{ \quotep{P} \nameeq \quotep{Q} }
\end{mathpar}

The astute reader will have noticed that the mutual recursion of names
and processes imposes a mutual recursion on alpha-equivalence and
structural equivalence via name-equivalence. Fortunately, all of this
works out pleasantly and we may calculate in the natural way, free of
concern. The reader interested in the details is referred to the
appendix \ref{appendix:rho_details}.

\subsection{Substitution}

We use $\Proc$ for the set of processes, $\QProc$ for the set of
names, and $\id{\{}\vec{y} / \vec{x} \id{\}}$ to denote partial maps,
$s : \QProc \rightarrow \QProc$. A map, $s$ lifts, uniquely, to a map
on process terms, $\widehat{s} : \Proc \rightarrow \Proc$ by the
following equations.

\begin{mathpar}
  (0) \psubstp{Q}{P} := 0 \\
  (R \juxtap S) \psubstp{Q}{P}
  :=    
  (R)\psubstp{Q}{P} \juxtap (S) \psubstp{Q}{P} \\
  (x?(y).R) \psubstp{Q}{P}    
  :=    
  (x)\substp{Q}{P} (z)\concat( (R \psubstn{z}{y}) \psubstp{Q}{P} ) \\
  (\lift{x}{R}) \psubstp{Q}{P}  
  :=
  \lift{(x)\substp{Q}{P}}{ R \psubstp{Q}{P} } \\
%   (\dropn{x})  \psubstp{Q}{P}       
%   := 
%   \left\{ 
%     \begin{array}{ccc} 
%       \dropn{\quotep{Q}} & & x \nameeq \quotep{P} \\
%       \dropn{x} & & otherwise \\
%     \end{array}
%   \right. 
  (\dropn{x})  \psubstp{Q}{P}       
  := 
  \left\{ 
    \begin{array}{ccc} 
      Q & & x \nameeq \quotep{P} \\
      \dropn{x} & & otherwise \\
    \end{array}
  \right.
\end{mathpar}
 

where

\begin{eqnarray}
  (x)\id{\{} \lpquote Q \rpquote / \lpquote P \rpquote \id{\}}            = 
  \left\{ 
    \begin{array}{ccc}
      \lpquote Q \rpquote & & x \nameeq \lpquote P \rpquote \\
      x & & otherwise \\
    \end{array}
  \right. \nonumber
\end{eqnarray}

and $z$ is chosen distinct from $\quotep{P}$, $\quotep{Q}$, the free
names in $Q$, and all the names in $R$. Our $\alpha$-equivalence will
be built in the standard way from this substitution.

\begin{remark}\label{rem:no_self_referential_names}
  One consequence of these definitions is that $\forall P. \quotep{P}
  \not\in \freenames{P}$.
\end{remark}

\subsection{ Dynamic quote: an example }

Anticipating something of what's to come, consider applying the
substitution, $\widehat{\id{\{}u / z \id{\}}}$, to the following pair
of processes, $\lift{w}{y!(z)}$ and $w[ \lpquote y!(z) \rpquote ]$.

\begin{eqnarray}
	\lift{w}{y!(z)}\widehat{\id{\{}u / z \id{\}}}
		& = &
		\lift{w}{y!(u)} \nonumber\\
	w[ \lpquote y!(z) \rpquote ] \widehat{ \id{\{}u / z \id{\}} }
		& = &
		w[ \lpquote y!(z) \rpquote ] \nonumber
\end{eqnarray}

Because the body of the process between quotes is impervious to
substitution, we get radically different answers. In fact, by
examining the first process in an input context,
e.g. $x?(z).\lift{w}{y!(z)}$, we see that the process under the lift
operator may be shaped by prefixed inputs binding a name inside it. In
this sense, the lift operator will be seen as a way to dynamically
construct processes before reifying them as names.

Finally equipped with these standard features we can present the
dynamics of the calculus.

\subsubsection{Operational semantics} 

Finally, we introduce the computational dynamics. What marks these
algebras as distinct from other more traditionally studied algebraic
structures, e.g. vector spaces or polynomial rings, is the manner in
which dynamics is captured. In traditional structures, dynamics is typically
expressed through morphisms between such structures, as in linear maps
between vector spaces or morphisms between rings. In algebras
associated with the semantics of computation, the dynamics is
expressed as part of the algebraic structure itself, through a
reduction reduction relation typically denoted by $\red$. Below, we
give a recursive presentation of this relation for the calculus used
in the encoding.

$\red \subseteq \pi \times \pi$
$\red : \pi \to \mathcal{P}(\pi)$

\begin{mathpar}
  \inferrule* [lab=Comm] { \textsf{match}( x_{src}, x_{trgt} ) } { x_{trgt}?(y)P \; | \; x_{src}!\langle {Q} \rangle \red P\{\quotep{Q}/y}\} }
  \and \\
  \inferrule* [lab=Par] {{P} \red {P}'} {{{P} | {Q}} \red {{P}' | {Q}}}
  \and
  \inferrule* [lab=Equiv]{{{P} \scong {P}'} \andalso {{P}' \red {Q}'} \andalso {{Q}' \scong {Q}}}{{P} \red {Q}}
\end{mathpar}

\begin{eqnarray*}
  match_{\equiv} (\quotep{P},\quotep{Q}) & := & P \equiv Q \\
  match_{\dagger}(\quotep{P},\quotep{Q}) & := & \forall R. P|Q \red^{*} R => R \red^{*} 0 \\
  match_{K}(\quotep{P},\quotep{Q}) & := & K \mbox{ for some context } K
\end{eqnarray*}

$u?(x)P | u!\langle Q \rangle \red P\{\quotep{Q}/x\}$

%We write $\wred$ for $\red^*$, and $P\red$ if $\exists Q $ such that $ P \red Q$.
We write $P\red$ if $\exists Q $ such that $ P \red Q$ and $P\not\red$, otherwise.

\section{Replication}

As mentioned before, it is known that replication (and hence
recursion) can be implemented in a higher-order process algebra
\cite{SangiorgiWalker}. As our first example of calculation with the
machinery thus far presented we give the construction explicitly in
the {\rhoc}.

\begin{eqnarray}
	D_{x} & := & \prefix{x}{y}{(\binpar{\outputp{x}{y}}{@{y}})} \nonumber\\
	\bangp_{x}{P} & := & \binpar{{x}!\langle{\binpar{D_{x}}{P}}\rangle}{D_{x}} \nonumber
\end{eqnarray}

\begin{eqnarray}
	\bangp_{x}{P} & & \nonumber\\
	=
	& {x}!\langle{(\prefix{x}{y}{(\outputp{x}{y} | @{y})) | P}}\rangle 
	      | \prefix{x}{y}{(\outputp{x}{y} | @{y})} & \nonumber\\
	\red
	& (\outputp{x}{y} | @{y})\substn{\quotep{(\prefix{x}{y}{(@{y} | \outputp{x}{y})) | P}}}{y} & \nonumber\\
	=
	& \outputp{x}{\quotep{(\prefix{x}{y}{(\outputp{x}{y} | @{y})) | P}}}
	  | {(\prefix{x}{y}{(\outputp{x}{y} | @{y})) | P}} & \nonumber\\
	\red
	& \ldots & \nonumber\\
	\red^*
	& P | P | \ldots & \nonumber
\end{eqnarray}

Of course, this encoding, as an implementation, runs away, unfolding
$\bangp{P}$ eagerly. A lazier and more implementable replication
operator, restricted to input-guarded processes, may be obtained as follows.

\begin{eqnarray}
\bangp{\prefix{u}{v}{P}} 
	:= 
	\binpar{\lift{x}{\prefix{u}{v}{(\binpar{D(x)}{P})}}}{D(x)} \nonumber
\end{eqnarray}

\begin{remark}
  Note that the lazier definition still does not deal with summation
  or mixed summation (i.e. sums over input and output). The reader is
  invited to construct definitions of replication that deal with these
  features. 

  Further, the definitions are parameterized in a name, $x$. Can you,
  gentle reader, make a definition that eliminates this parameter and
  guarantees no accidental interaction between the replication
  machinery and the process being replicated -- i.e. no accidental
  sharing of names used by the process to get its work done and the
  name(s) used by the replication to effect copying. This latter
  revision of the definition of replication is crucial to obtaining
  the expected identity $!!P \sim !P$.
\end{remark}

\begin{remark}\label{rem:paradoxical_combinator}
  The reader familiar with the lambda calculus will have noticed the
  similarity between $D$ and the paradoxical combinator.

  [Ed. note: the existence of this seems to suggest we have to be more
  restrictive on the set of processes and names we admit if we are to
  support no-cloning.]
\end{remark}

\subsubsection{Bisimulation}

The computational dynamics gives rise to another kind of equivalence,
the equivalence of computational behavior. As previously mentioned
this is typically captured \emph{via} some form of bisimulation.

% The notion we use in this paper is weak barbed bisimulation
% \cite{milner91polyadicpi}.

The notion we use in this paper is derived from weak barbed
bisimulation \cite{milner91polyadicpi}. 

\begin{definition}
An \emph{observation relation}, $\downarrow_{\mathcal N}$, over a set
of names, $\mathcal N$, is the smallest relation satisfying the rules
below.

\infrule[Out-barb]{y \in {\mathcal N}, \; x \nameeq y}
		  {\outputp{x}{v} \downarrow_{\mathcal N} x}
\infrule[Par-barb]{\mbox{$P\downarrow_{\mathcal N} x$ or $Q\downarrow_{\mathcal N} x$}}
		  {\binpar{P}{Q} \downarrow_{\mathcal N} x}

We write $P \Downarrow_{\mathcal N} x$ if there is $Q$ such that 
$P \wred Q$ and $Q \downarrow_{\mathcal N} x$.
\end{definition}

\begin{definition}
%\label{def.bbisim}
An  ${\mathcal N}$-\emph{barbed bisimulation} over a set of names, ${\mathcal N}$, is a symmetric binary relation 
${\mathcal S}_{\mathcal N}$ between agents such that $P\rel{S}_{\mathcal N}Q$ implies:
\begin{enumerate}
\item If $P \red P'$ then $Q \wred Q'$ and $P'\rel{S}_{\mathcal N} Q'$.
\item If $P\downarrow_{\mathcal N} x$, then $Q\Downarrow_{\mathcal N} x$.
\end{enumerate}
$P$ is ${\mathcal N}$-barbed bisimilar to $Q$, written
$P \wbbisim_{\mathcal N} Q$, if $P \rel{S}_{\mathcal N} Q$ for some ${\mathcal N}$-barbed bisimulation ${\mathcal S}_{\mathcal N}$.
\end{definition}

$\mathcal{R} \subseteq \pi \times \pi$

$P \mathcal{R} Q => \forall P'. P \red P' \Rightarrow \exists Q'. Q \red Q', P' \mathcal{R} Q'$

$P \vdash x \Rightarrow Q \vdash x$

\begin{mathpar}
  \inferrule*[lab=Out-barb]{x \nameeq y}{{y}!\langle{Q}\rangle \vdash x}
  \and
  \inferrule*[lab=Par-barb]{\mbox{$P\vdash x$ or $Q\vdash x$}}{\binpar{P}{Q} \vdash x}
\end{mathpar}

\subsubsection{Contexts}

One of the principle advantages of computational calculi like the
$\pi$-calculus is a well-defined notion of context,
contextual-equivalence and a correlation between
contextual-equivalence and notions of bisimulation. The notion of
context allows the decomposition of a process into (sub-)process and
its syntactic environment, its context. Thus, a context may be
thought of as a process with a ``hole'' (written $\Box$) in it. The
application of a context $M$ to a process $P$, written $M[P]$, is
tantamount to filling the hole in $M$ with $P$. In this paper we do
not need the full weight of this theory, but do make use of the notion
of context in the proof the main theorem. 

\begin{mathpar}
  \inferrule* [lab=summation] {} {{M_{M},M_{N}} \bc \Box \;|\; x.M_{A} \;|\; M_{M}+M_{N}}
  \and
  \inferrule* [lab=agent] {} {{M_{A}} \bc (\vec{x})M_{P} \;| \; \clift{P_0,\ldots,M_{P},\ldots,P_N}}
  \and \\
  \inferrule* [lab=process] {} {{M_{P}} \bc M_{N} \;| \;P|M_{P} }
\end{mathpar} 

\begin{mathpar}
  \inferrule* [lab=sychronization] {} {M_{N} \bc \Box \;|\; x?M_{F} \;|\; x!M_{C}}
  \and
  \inferrule* [lab=abstraction] {} {{M_{F}} \bc (x)M_{P} }
  \and
  \inferrule* [lab=concretion] {} {{M_{C}} \bc \langle M_{P} \rangle }
  \and \\
  \inferrule* [lab=process] {} {{M_{P}} \bc M_{N} \;| \;P|M_{P} }
\end{mathpar}

\begin{definition}[contextual application] Given a context $M$, and
  process $P$, we define the \emph{contextual application}, $M[P] :=
  M\{P/\Box\}$. That is, the contextual application of M to P is the
  substitution of $P$ for $\Box$ in $M$.
\end{definition}

$\meaningof{-} : L \to \mathcal{P}(\pi)$

\begin{mathpar}
  \inferrule* [lab=collection] {} {\meaningof{true} = \pi, \and \meaningof{~E} = \pi \setminus \meaningof{E}, \and \meaningof{E_{1} \& E_{2}} = \meaningof{E_{1}} \cap \meaningof{E_{2}}}
\end{mathpar}

\begin{mathpar}
  \inferrule* [lab=structure] {} {\meaningof{0} = \{ P \in \pi | P \equiv 0 \}, \and \\ \meaningof{E_1 | E_2} = \{ P \in \pi | P \equiv P_{1} | P_{2}, P_{1} \in \meaningof{E_{1}}, P_{2} \in \meaningof{E_2}\} }
\end{mathpar}

\begin{mathpar}
 \inferrule* [lab=behavior] {} {\meaningof{\langle a?b \rangle E} = \{ P \in \pi | P \equiv Q | u?(y)P', \\ \and \\\\ \and \\ \;\;\; u \in \meaningof{a}, \forall z.P'\{z/y\} \in \meaningof{E\{z/b\}}\}, \and \\ \meaningof{a!E} = \{ P \in \pi | P \equiv Q | x!\langle P' \rangle, x \in \meaningof{a} P' \in \meaningof{E}\} }
\end{mathpar}

\begin{mathpar}
 \inferrule* [lab=nominal] {} {\meaningof{\quotep{E}} = \{ \quotep{P} \in \quotep{\pi} | P \in \meaningof{E} \}, \and \meaningof{\quotep{P}} = \{ \quotep{Q} \in \quotep{\pi} | P \equiv Q \} \and \\ \meaningof{@\quotep{E}} = \{ P \in \pi | P \equiv @x, x \in \meaningof{E} \}}
\end{mathpar}

\begin{eqnarray*}
  \\
  \meaningof{-} : TS \to ST
\end{eqnarray*}

\begin{eqnarray*}
  \\
  L : TS \to ST
\end{eqnarray*}

\begin{eqnarray*}
  \\
  P \models E \iff P \in \meaningof{E}
\end{eqnarray*}

\begin{eqnarray*}
  P \approx_{L} Q \iff \forall E \in L. P \models E \iff Q \models E
\end{eqnarray*}

\begin{eqnarray*}
  P \approx_{K} Q
\end{eqnarray*}

\begin{eqnarray*}
  P \approx Q
\end{eqnarray*}

$\approx_{K} = \approx = \approx_{L}$

\subsubsection{Contextual duality}

Note that contexts extend the quotation operation to a family of
operations from processes to names. Given a context, $M$, we can
define a \emph{nominal context}, $\quotep{M}$ by $\quotep{M}[P] :=
\quotep{M[P]}$. To foreshadow what is to come we observe that these
operations enjoy a duality with processes very much like the duality
between vectors and maps from vectors to scalars.

Further, because the calculus is essentially higher-order, we have a
correspondence between contexts and processes. More specifically,
given a name $x$ and a context $M$ we can construct $M^{*}_{x}$ such
that 

\begin{mathpar}
  M^{*}_{x} | \lift{x}{P} \red M[P]
\end{mathpar}

namely,

\begin{mathpar}
  M^{*}_{x} := x?(u).M[\dropn{u}]
\end{mathpar}

The dependence of $M^{*}_{x}$ on a name makes it an abstraction, 

\begin{mathpar}
  M^{*} := (x)x?(u).M[\dropn{u}]
\end{mathpar}

\subsection{Additional notation}

It will sometimes be convenient to denote the process a name
quotes. We already have the notation $x = \quotep{P}$, but it will be
convenient to introduce an alternate notation, $\procn{x}$, when we
want to emphasize the connection to the use of the name. Note that, by
virtue of name equivalence, $\quotep{\procn{x}} \nameeq x$; so, the
notation is consistent with previous definitions.

Further, because names have structure it is possible to effect
substitutions on the basis of that structure. This means we need to
upgrade our notation for substitutions, which we accomplish by
adapting comprehension notation. Thus,

\begin{mathpar}
  P\{ y / x : x \in S \}
\end{mathpar}

is interpreted to mean the process derived from P by replacing (in a
capture-avoiding manner) each occurrence of $x$ in $S$ by $y$. For example,

\begin{mathpar}
  P\{ \quotep{\procn{x}|\procn{x}} / x : x \in \freenames{P} \}
\end{mathpar}

will replace each (occurrence) of a free name $x$ in $P$ by
$\quotep{\procn{x}|\procn{x}}$.

Also, we will avail ourselves of the notation $x^{L}$ and $x^{R}$ to
denote injections of a name into disjoint copies of the name
space. There are numerous ways to accomplish this. One example can be
found in \cite{MeredithR05}. This notation overloads to vectors of
names: $\vec{x}^{\pi} := (x_{i}^{\pi} \; : \; 0 \leq i < |\vec{x}| )$ where $\pi \in \{L,R\}$.

We also use $P^{\Box} := P|\Box$.

In \cite{MeredithR05} an interpretation of the new operator is
given. It turns out that there are several possible interpretations
all enjoying the requisite algebraic properties of the operator (see
\cite{milner91polyadicpi}). We will therefore make liberal use of
$(\nu\; \vec{x})P$.

% subsection the_syntax_and_semantics_of_the_notation_system (end)   

\section{Interpretation of QM}
\subsection{Supporting definitions}
\subsubsection{Multiplication}
\begin{mathpar}
  \quotep{Q} \cdot \quotep{R} := \quotep{Q|R}
  \and \\
  \quotep{Q} \cdot P := P\{ \quotep{Q|R} / \quotep{R} : \quotep{R} \in \freenames{P} \}
\end{mathpar}

\paragraph{Discussion}
The first line needs little explanation. The second line says that
each free name of the process is replaced with the multiplication of
that name by the scalar. Multiplication of a scalar (name) by a state
(process) results in a process all the names of which have been `moved
over' by parallel composition with the process the scalar
quotes. There is a subtlety that the bound names have to be
manipulated so that multiplied names aren't accidentally
captured. There are many ways to achieve this.

\begin{remark}\label{rem:multiplication_identities}
  The reader is invited to verify that for all $x,y,z \in \QProc$ and $P \in \Proc$
  \begin{mathpar}
    x \cdot \quotep{0} \equiv x 
    \and
    x \cdot y \equiv y \cdot x
    \and
    x \cdot (y \cdot z) \equiv (x \cdot y) \cdot z
    \and \\
    \quotep{0} \cdot P \equiv P
    \and \\
    x \cdot (y \cdot P) \equiv (x \cdot y) \cdot P
    \and \\
    x \cdot (P|Q) \equiv (x \cdot P) | (x \cdot Q)
    \and \\    
  \end{mathpar}
\end{remark}

\subsubsection{Tensor product}

We define a tensor product on processes by structural induction.

\paragraph{Tensor of sums} First note that all summations, including
$\pzero$ and sequence, can be written $\Sigma_{i} x_{i}.A_{i} +
\Sigma_{j} x_{j}.C_{j}$, where we have grouped input-guarded processes
together and output-guarded processes together.

Thus, we can define the tensor product of two summations, $N_{1}\otimes N_{2}$, where

\begin{mathpar}
  N_{1} := \Sigma_{i} x_{i}.A_{i} + \Sigma_{j} x_{j}.C_{j}
  \and
  N_{2} := \Sigma_{i'} y_{i'}.B_{i'} + \Sigma_{j'} y_{j'}.D_{j'} 
\end{mathpar}

as follows.

\begin{mathpar}
  \Sigma_{i} x_{i}.A_{i} + \Sigma_{j} x_{j}.C_{j} \otimes \Sigma_{i'}
  y_{i'}.B_{i'} + \Sigma_{j'} y_{j'}.D_{j'} 
  \and \\
  := \; \Sigma_{i} \Sigma_{i'} \quotep{\stackrel{\vee}{x_{i}}| \stackrel{\vee}{y_{i'}}}.(A_{i}\otimes B_{i'}) \; | \; \Sigma_{i'} \Sigma_{i} \quotep{\stackrel{\vee}{y_{i'}}|\stackrel{\vee}{x_{i}}}.(B_{i'}\otimes A_{i})
  \and
  \;\; | \;\; \Sigma_{j} \Sigma_{j'} \quotep{\stackrel{\vee}{x_{j}}|\stackrel{\vee}{y_{j'}}}.(A_{j}\otimes B_{j'}) \; | \; \Sigma_{j'} \Sigma_{j} \quotep{\stackrel{\vee}{y_{j'}}|\stackrel{\vee}{x_{j}}}.(B_{j'}\otimes A_{j})
\end{mathpar}

\begin{remark}
  Do we need to $x^{L}$ and $y^{R}$ for this construction as well?
\end{remark}

\paragraph{Tensor of parallel compositions} Next, we distribute tensor
over par.

\begin{mathpar}
  P_{1}|P_{2} \otimes Q_{1}|Q_{2} := (P_{1} \otimes Q_{1}) | (P_{1}
  \otimes Q_{2}) | (P_{2} \otimes Q_{1}) | (P_{2} \otimes Q_{2})
\end{mathpar}

\paragraph{Tensor with dropped names} We treat tensor of a
process with a dropped name as parallel composition.

\begin{mathpar}
  P \otimes \dropn{x} := P | \dropn{x}
\end{mathpar}

\paragraph{Tensor of agents}

Finally, we need to define tensor on agents. Note that the definition
of tensor on normal products only tensors inputs with inputs and
outputs with outputs. Thus, we only have to define the operation on
``homogeneous'' pairings.

\begin{mathpar}
  (\vec{x})P \otimes (\vec{y})Q
  \and \\
  := (x_{0}^{L}|y_{0}^{R},\ldots,x_{0}^{L}|y_{n}^{R},\ldots,x_{m}^{L}|y_{0}^{R},\ldots,x_{m}^{L}|y_{n}^R)(P\{ \vec{x}^{L}/\vec{x}\} \otimes Q \{ \vec{y}^{R}/\vec{y}\})
  \and \\
  \clift{\vec{P}} \otimes \clift{\vec{Q}}
  \and \\
  := \clift{P_{0}\otimes Q_{0},\ldots,P_{0}\otimes Q_{n},\ldots,P_{m}\otimes Q_{0},\ldots,P_{m}\otimes Q_{n}}
\end{mathpar}

\begin{remark}
  Observe that arities of tensored abstractions matches arities of
  tensored concretions if the original arities matched. Note also that
  the length of the arities corresponds to the increase in dimension
  we see in ordinary vector space tensor product.
\end{remark}

\begin{remark}
  Operationally, this definition distributes the tensor down to
  components ``linked'' by summation. Tensor over summation is
  intriguing in that it mixes names. Moreover, as a consequence of the
  way it mixes names we have the identities for all $x \in \QProc$ and
  $P,Q \in \Proc$

  \begin{mathpar}
    (x \cdot P) \otimes Q \equiv x \cdot (P \otimes Q) \equiv P \otimes (x \cdot Q)
    \and
    P \otimes \pzero \equiv P
  \end{mathpar}

  that the reader is invited to verify.
\end{remark}

\subsubsection{Annihilation}
\begin{mathpar}
  P^{\perp} := \{ Q | \forall R. P|Q \red^{*} R \Rightarrow R \red^{*} \pzero \}
  \and \\
  P^{\underline{\perp}} := \Sigma_{Q \in P^{\perp}} \quotep{Q}?(y).(\dropn{y}|Q) | \Sigma_{Q \in P^{\perp}} \quotep{Q}\clift{\Box}
\end{mathpar}

\paragraph{Discussion} The reader will note that $P^{\perp}$ is a
\emph{set} of processes, while $P^{\underline{\perp}}$ is a
\emph{context}. We call the set $P^{\perp}$ the \emph{annihilators} of
$P$. The parallel composition of a process in the annihilators of $P$
with $P$ will result in a process, the state space of which has all
paths eventually leading to $\pzero$. Execution may endure loops; but
under reasonable conditions of fairness (naturally guaranteed under
most notions of bisimulation) such a composite process cannot get
stuck in such a loop and will, eventually pop out and terminate.

The context $P^{\underline{\perp}}$ is ready and willing to ``take the
$P$ out of'' the process to which it is applied. It will effectively
transmit the code of the process to which it is applied to one of the
annihilators and run the process against it.

\subsubsection{Evaluation}
We fix $M$ a domain of fully abstract interpretation with an equality
coincident with bisimulation. We take $\meaningof{\cdot} : \Proc \to
M$ to be the map interpreting processes and $\nmeaningof{\cdot} : \M
\to Proc$ to be the map running the other way. Then we define

\begin{mathpar}
  \int P := \nmeaningof{\meaningof{P}}
\end{mathpar}

\paragraph{Discussion}
There are many fully abstract interpretations of Milner's
$\pi$-calculus. Any of them can be used as a basis for interpreting
the reflective calculus here. Equipped with such a domain it is
largely a matter of grinding through to check that the Yoneda
construction for the normalization-by-evaluation program can be
extended to this setting.

\begin{remark}
  The reader is invited to verify that $\int (P^{\underline{\perp}}[P]) = 0$.
\end{remark}

\subsection{Quantum mechanics}

Table \ref{tbl:core_qm_op_defns} gives the core operational definitions

\begin{table}[htp]\label{tbl:core_qm_op_defns}
  \center{
    \fbox{
      \begin{tabular}{c|c}
        quantum mechanics & process calculus \\
        \hline
        scalar & $x := \quotep{P}$ \\
        state vector & $\state{P} := P$ \\
        dual & $\state{P}^{*} := \event{P^{\underline{\perp}}} := \quotep{P^{\underline{\perp}}}[-]$ \\
        matrix & $ \Sigma_{\alpha} \state{P_{\alpha}}x_{\alpha}\event{Q_{\alpha}}$ \\
        vector addition & $\state{P} + \state{Q} := \state{P | Q}$ \\
        tensor product & $\state{P} \otimes \state{Q} := \state{P \otimes Q}$ \\
        inner product & $\innerprod{P}{Q} := \quotep{\int P^{\underline{\perp}}[Q]}$ \\
      \end{tabular}
    }
  }
  \caption{QM - operational definitions}
\end{table}

where

\begin{mathpar}
  \prmatrix{P}{Q} := \fprmatrix{P}{\quotep{\pzero}}{Q}
  \and
  \fprmatrix{P}{x}{Q} := (\state{P},x,\event{Q})
  \and
  (\fprmatrix{P}{x}{Q})(\state{R}) := x \cdot \innerprod{Q}{R} \cdot \state{P}
  \and
  (\fprmatrix{P}{x}{Q})(\event{R}) := x \cdot \innerprod{R}{P} \cdot \event{Q}
\end{mathpar}

\paragraph{Discussion}
As promised: vectors (aka states) are represented as processes; duals
as contextual duals; inner product definition should be compared with
standard inner product definition for ....

\begin{remark}
  Assuming $\int (P^{\underline{\perp}}[P]) = 0$, the reader is
  invited to verify that $(\fprmatrix{P}{x}{P})(\state{P}) = x \cdot \state{P}$.
\end{remark}

\begin{remark}
  The reader is invited to verify that $\innerprod{P}{Q}$ could
  equally well have been written $\quotep{\int \stackrel{\vee}{x}}$
  where $x = \event{P^{\underline{\perp}}}(Q)$.

  One of the motivations for this remark is that there is another way
  to factor these operations. We could package up evaluation in the dual:

  \begin{mathpar}
    \state{P}^{*} := \event{\int P^{\underline{\perp}}} := \quotep{\int P^{\underline{\perp}}}[-]
  \end{mathpar}

  and then have inner product defined by
  
  \begin{mathpar}
    \innerprod{P}{Q} := \event{P}(Q)
  \end{mathpar}

  Hopefully, experience with the calculations will provide guidance on
  the best factoring.
\end{remark}

\begin{remark}
  Assuming $\int (P^{\underline{\perp}}[P]) = 0$, the reader is
  invited to verify that $\forall P,Q. (\prmatrix{0}{Q})(\state{0}) =
  \state{0}$ and dually $(\prmatrix{P}{0})(\event{0}) = \event{0}$.
\end{remark}

\begin{remark}
  i'm a little worried that i don't (yet) have proper support for
  complex conjugacy. But, the observation above may give us a
  clue. According to Abramsky, it must be the case that the scalars
  are iso to the homset of the identity for the tensor -- which the
  observation above characterizes. 

  For now, we will simply bookmark the notion with $\overline{x}$.
\end{remark}

\subsubsection{Adjointness}

We need to give a definition of $(\cdot)^{\dagger}$ for matrices. The
obvious candidate definition is
\begin{mathpar}
(\Sigma_{\alpha}\fprmatrix{P_{\alpha}}{x_{\alpha}}{Q_{\alpha}})^{\dagger}
= \Sigma_{\alpha}\fprmatrix{(Q_{\alpha}^{\underline{\perp}})^{*}}{\overline{x}_{\alpha}}{P_{\alpha}^{\underline{\perp}}} 
\end{mathpar}

But, $(Q_{\alpha}^{\underline{\perp}})^{*}$ requires a name along
which to communicate the process to achieve the context application.

\subsubsection{Basis for a basis}
If processes label states and ``addition'' of states (a.k.a. vector
addition) is interpreted as parallel composition, what corresponds to
notions of linear independence and basis? Here, we recall that Yoshida
has developed a set of \emph{combinators} for an asynchronous verison
of Milner's $\pi$-calculus. These are a finite set of processes such
any process can be expressed as parallel composition of these
combinators together with liberal uses of the new operator and
replication. We can simply give a translation of these into the
present calculus and have reasonable expectation that the property
carries over. That is, that the resultant set allows to express all
processes via parallel composition. Note, however, that there is no
new operator or replication in this calculus. As a result, we expect
that the corresponding set is actually infinite. That is, we expect
that the space is actually infinite dimensional.

\begin{remark}
  The attentive reader may be a bit concerned. Certainly, the
  collection $S$, $K$ and $I$ is a finite set of
  combinators. Shouldn't we expect to see a finite set of combinators
  for an effectively equivalent system? i am very sympathetic to this
  critique and feel it warrants full attention. On the other hand, i
  also have in mind the following analogy. The natural numbers, as a
  monoid under addition, has exactly $1$ generator, while the natural
  numbers, as a monoid under multiplication, has countably many
  generators (the primes). We observe that the application of the
  lambda calculus is much less resource sensitive than the parallel
  composition of the $\pi$-calculus. Could it be the case that we have
  an analogy of the form
  
  \begin{mathpar}
    m + n : MN :: m*n : M|N
  \end{mathpar}

  giving a similar blow up in the set of ``primes''?  This is such a
  wonderful thought that, even if it's not true, i think it's worth
  writing down.
\end{remark}
 

\documentclass[12pt]{llncs}
%\documentclass{jktr}

\usepackage[pdftex]{hyperref}                   
\usepackage {listings}
\usepackage {mathpartir}
\usepackage{bcprules}
%\usepackage{listings}
                       
\usepackage{graphicx} 
%\usepackage[margins=2.5cm,nohead,nofoot]{geometry}
%\usepackage{geometry}
\usepackage{amsfonts}
\usepackage{amstext}
\usepackage{latexsym}
\usepackage{amssymb}
\usepackage{color}


%\include{myPreamble}
\include{qm2pi.local} 

%\ifpdf
%\usepackage[pdftex]{graphicx}
%\else
%\usepackage{graphicx}
%\fi

 % \ifpdf
%  \usepackage{pdfsync}
%  \if


%\title{Brief Article}
%\author{David F. Snyder}
%\author{L.G. Meredith}

%\address{Dept. of Math., Texas State University--San Marcos, San Marcos, TX 78666}
       
\pagestyle{empty}


\begin{document}

\lstset{language=[Objective]Caml,frame=shadowbox}

\input{qm2pi.front}

% section front matter (end)

\input{qm2pi.intro} 
 
% section introduction (end)

% \input{qm2pi.knotations} 

% section notation (end)

\input{qm2pi.process.calculi} 

% section concurrent_process_calculi_and_spatial_logics_ (end)
    
%\input{qm2pi.knots2pi} 

%\input{qm2pi.trefoil} 

%\input{qm2pi.mainthm} 

% subsection basic_interpretation (end)

%\input{qm2pi.rho.presentation} 
\subsection{The syntax and semantics of the notation system}\label{sub:the_syntax_and_semantics_of_the_notation_system} % (fold)

We now summarize a technical presentation of the calculus that
embodies our theory of dynamics. The typical presentation of such a
calculus follows the style of giving generators and relations on
them. The grammar, below, describing term constructors, freely
generates the set of processes, $\Proc$. This set is then quotiented
by a relation known as structural congruence and it is over this set
that the notion of dynamics is expressed. This presentation is
essentially that of \cite{MeredithR05} with the addition of
polyadicity and summation. For readability we have relegated some of
the technical subtleties to an appendix.

\subsubsection{Process grammar}\label{subsub:process_grammar}

\begin{mathpar}
  \inferrule* [lab=synchronization] {} {{M} \bc \pzero \;|\; x?F \;|\; x!C }
  \and
  \inferrule* [lab=abstraction] {} {{F} \bc (x)P}
  \and
  \inferrule* [lab=concretion] {} {{C} \bc \langle Q \rangle}
  \and
  \inferrule* [lab=process] {} {{P,Q} \bc M \;| \;P|Q \;|\; @{x}}
  \and
  \inferrule* [lab=name] {} {{x} \bc \quotep{P}}
\end{mathpar} 

Note that $\vec{x}$ (resp. $\vec{P}$) denotes a vector of names
(resp. processes) of length $|\vec{x}|$ (resp. $|\vec{P}|$). We adopt
the following useful abbreviations.

\begin{mathpar}
   x?(\vec{y}).P := x.(\vec{y})P \and  x\clift{\vec{P}} := x.\clift{\vec{P}}
   \and x!(y) := \lift{x}{\dropn{y}}
   \and \Pi_{i=0}^{n-1}P_i := P_0 | \ldots | P_{n-1}
\end{mathpar}

\subsubsection{Structural congruence}

\paragraph{Free and bound names and alpha-equivalence.} At the
core of structural equivalence is alpha-equivalence which identifies
process that are the same up to a change of variable. Formally, we
recognize the distinction between free and bound names. The free names
of a process, $\freenames{P}$, may be calculated recursively as
follows:

\begin{mathpar}
\freenames{\pzero} := \emptyset
  \and \\
  \freenames{x?(y).P} := \{ x \} \cup (\freenames{P} \setminus \{ y \})
  \and 
  \freenames{x!\langle P \rangle} := \{ x \} \cup \{ P \} 
  \and \\
  \freenames{P|Q} := \freenames{P} \cup \freenames{Q}
  \and \\
  \freenames{@{x}} := \{ x \}
\end{mathpar}

$\pi$
$\quotep{\pi}$

$\freenames{-} : \pi \to \mathcal{P}(\quotep{\pi})$

\begin{eqnarray*}
  \freenames{\pzero} & := & \emptyset \\
  \freenames{x?(y).P} & := & \{ x \} \cup (\freenames{P} \setminus \{ y \}) \\
  \freenames{x!\langle P \rangle} & := & \{ x \} \cup \{ P \} \\
  \freenames{P|Q} & := & \freenames{P} \cup \freenames{Q} \\
  \freenames{\dropn{x}} & := & \{ x \}
\end{eqnarray*}

The bound names of a process, $\boundnames{P}$, are those names occurring in $P$
that are not free. For example, in $x?(y).0$, the name $x$ is free, while $y$ is bound.

\begin{mathpar}
  \inferrule* [lab=monoidal-laws] {} { P|Q \equiv Q|P \and P|0 \equiv P \and P|(Q|R) \equiv (P|Q)|R }
\end{mathpar}

\begin{mathpar}
  \inferrule* [lab=alpha-equivalence] {} { (x)P \equiv (y)P\{y/x\} \and y \not\in \freenames{P} }
\end{mathpar}

\begin{definition}
Then two processes, $P,Q$, are alpha-equivalent if $P = Q\{\vec{y}/\vec{x}\}$ for
some $\vec{x} \in \boundnames{Q},\vec{y} \in \boundnames{P}$, where $Q\{\vec{y}/\vec{x}\}$
denotes the capture-avoiding substitution of $\vec{y}$ for $\vec{x}$ in $Q$.
\end{definition}

\begin{definition}
  The {\em structural congruence} \cite{SangiorgiWalker} , $\equiv$,
  between processes is the least congruence containing
  alpha-equivalence, satisfying the abelian monoid laws
  (associativity, commutativity and $\pzero$ as identity) for parallel
  composition $|$ and for summation $+$.
\end{definition}

\subsection{Name equivalence}

We take name equivalence, written $\nameeq$, to be the smallest
equivalence relation generated by the following rules.

\begin{mathpar}
\inferrule*[lab=Quote-drop]
{ }
{ \quotep{@{x}} \nameeq x }

\inferrule*[lab=Struct-equiv]
{ P \scong Q }
{ \quotep{P} \nameeq \quotep{Q} }
\end{mathpar}

The astute reader will have noticed that the mutual recursion of names
and processes imposes a mutual recursion on alpha-equivalence and
structural equivalence via name-equivalence. Fortunately, all of this
works out pleasantly and we may calculate in the natural way, free of
concern. The reader interested in the details is referred to the
appendix \ref{appendix:rho_details}.

\subsection{Substitution}

We use $\Proc$ for the set of processes, $\QProc$ for the set of
names, and $\id{\{}\vec{y} / \vec{x} \id{\}}$ to denote partial maps,
$s : \QProc \rightarrow \QProc$. A map, $s$ lifts, uniquely, to a map
on process terms, $\widehat{s} : \Proc \rightarrow \Proc$ by the
following equations.

\begin{mathpar}
  (0) \psubstp{Q}{P} := 0 \\
  (R \juxtap S) \psubstp{Q}{P}
  :=    
  (R)\psubstp{Q}{P} \juxtap (S) \psubstp{Q}{P} \\
  (x?(y).R) \psubstp{Q}{P}    
  :=    
  (x)\substp{Q}{P} (z)\concat( (R \psubstn{z}{y}) \psubstp{Q}{P} ) \\
  (\lift{x}{R}) \psubstp{Q}{P}  
  :=
  \lift{(x)\substp{Q}{P}}{ R \psubstp{Q}{P} } \\
%   (\dropn{x})  \psubstp{Q}{P}       
%   := 
%   \left\{ 
%     \begin{array}{ccc} 
%       \dropn{\quotep{Q}} & & x \nameeq \quotep{P} \\
%       \dropn{x} & & otherwise \\
%     \end{array}
%   \right. 
  (\dropn{x})  \psubstp{Q}{P}       
  := 
  \left\{ 
    \begin{array}{ccc} 
      Q & & x \nameeq \quotep{P} \\
      \dropn{x} & & otherwise \\
    \end{array}
  \right.
\end{mathpar}
 

where

\begin{eqnarray}
  (x)\id{\{} \lpquote Q \rpquote / \lpquote P \rpquote \id{\}}            = 
  \left\{ 
    \begin{array}{ccc}
      \lpquote Q \rpquote & & x \nameeq \lpquote P \rpquote \\
      x & & otherwise \\
    \end{array}
  \right. \nonumber
\end{eqnarray}

and $z$ is chosen distinct from $\quotep{P}$, $\quotep{Q}$, the free
names in $Q$, and all the names in $R$. Our $\alpha$-equivalence will
be built in the standard way from this substitution.

\begin{remark}\label{rem:no_self_referential_names}
  One consequence of these definitions is that $\forall P. \quotep{P}
  \not\in \freenames{P}$.
\end{remark}

\subsection{ Dynamic quote: an example }

Anticipating something of what's to come, consider applying the
substitution, $\widehat{\id{\{}u / z \id{\}}}$, to the following pair
of processes, $\lift{w}{y!(z)}$ and $w[ \lpquote y!(z) \rpquote ]$.

\begin{eqnarray}
	\lift{w}{y!(z)}\widehat{\id{\{}u / z \id{\}}}
		& = &
		\lift{w}{y!(u)} \nonumber\\
	w[ \lpquote y!(z) \rpquote ] \widehat{ \id{\{}u / z \id{\}} }
		& = &
		w[ \lpquote y!(z) \rpquote ] \nonumber
\end{eqnarray}

Because the body of the process between quotes is impervious to
substitution, we get radically different answers. In fact, by
examining the first process in an input context,
e.g. $x?(z).\lift{w}{y!(z)}$, we see that the process under the lift
operator may be shaped by prefixed inputs binding a name inside it. In
this sense, the lift operator will be seen as a way to dynamically
construct processes before reifying them as names.

Finally equipped with these standard features we can present the
dynamics of the calculus.

\subsubsection{Operational semantics} 

Finally, we introduce the computational dynamics. What marks these
algebras as distinct from other more traditionally studied algebraic
structures, e.g. vector spaces or polynomial rings, is the manner in
which dynamics is captured. In traditional structures, dynamics is typically
expressed through morphisms between such structures, as in linear maps
between vector spaces or morphisms between rings. In algebras
associated with the semantics of computation, the dynamics is
expressed as part of the algebraic structure itself, through a
reduction reduction relation typically denoted by $\red$. Below, we
give a recursive presentation of this relation for the calculus used
in the encoding.

$\red \subseteq \pi \times \pi$
$\red : \pi \to \mathcal{P}(\pi)$

\begin{mathpar}
  \inferrule* [lab=Comm] { \textsf{match}( x_{src}, x_{trgt} ) } { x_{trgt}?(y)P \; | \; x_{src}!\langle {Q} \rangle \red P\{\quotep{Q}/y}\} }
  \and \\
  \inferrule* [lab=Par] {{P} \red {P}'} {{{P} | {Q}} \red {{P}' | {Q}}}
  \and
  \inferrule* [lab=Equiv]{{{P} \scong {P}'} \andalso {{P}' \red {Q}'} \andalso {{Q}' \scong {Q}}}{{P} \red {Q}}
\end{mathpar}

\begin{eqnarray*}
  match_{\equiv} (\quotep{P},\quotep{Q}) & := & P \equiv Q \\
  match_{\dagger}(\quotep{P},\quotep{Q}) & := & \forall R. P|Q \red^{*} R => R \red^{*} 0 \\
  match_{K}(\quotep{P},\quotep{Q}) & := & K \mbox{ for some context } K
\end{eqnarray*}

$u?(x)P | u!\langle Q \rangle \red P\{\quotep{Q}/x\}$

%We write $\wred$ for $\red^*$, and $P\red$ if $\exists Q $ such that $ P \red Q$.
We write $P\red$ if $\exists Q $ such that $ P \red Q$ and $P\not\red$, otherwise.

\section{Replication}

As mentioned before, it is known that replication (and hence
recursion) can be implemented in a higher-order process algebra
\cite{SangiorgiWalker}. As our first example of calculation with the
machinery thus far presented we give the construction explicitly in
the {\rhoc}.

\begin{eqnarray}
	D_{x} & := & \prefix{x}{y}{(\binpar{\outputp{x}{y}}{@{y}})} \nonumber\\
	\bangp_{x}{P} & := & \binpar{{x}!\langle{\binpar{D_{x}}{P}}\rangle}{D_{x}} \nonumber
\end{eqnarray}

\begin{eqnarray}
	\bangp_{x}{P} & & \nonumber\\
	=
	& {x}!\langle{(\prefix{x}{y}{(\outputp{x}{y} | @{y})) | P}}\rangle 
	      | \prefix{x}{y}{(\outputp{x}{y} | @{y})} & \nonumber\\
	\red
	& (\outputp{x}{y} | @{y})\substn{\quotep{(\prefix{x}{y}{(@{y} | \outputp{x}{y})) | P}}}{y} & \nonumber\\
	=
	& \outputp{x}{\quotep{(\prefix{x}{y}{(\outputp{x}{y} | @{y})) | P}}}
	  | {(\prefix{x}{y}{(\outputp{x}{y} | @{y})) | P}} & \nonumber\\
	\red
	& \ldots & \nonumber\\
	\red^*
	& P | P | \ldots & \nonumber
\end{eqnarray}

Of course, this encoding, as an implementation, runs away, unfolding
$\bangp{P}$ eagerly. A lazier and more implementable replication
operator, restricted to input-guarded processes, may be obtained as follows.

\begin{eqnarray}
\bangp{\prefix{u}{v}{P}} 
	:= 
	\binpar{\lift{x}{\prefix{u}{v}{(\binpar{D(x)}{P})}}}{D(x)} \nonumber
\end{eqnarray}

\begin{remark}
  Note that the lazier definition still does not deal with summation
  or mixed summation (i.e. sums over input and output). The reader is
  invited to construct definitions of replication that deal with these
  features. 

  Further, the definitions are parameterized in a name, $x$. Can you,
  gentle reader, make a definition that eliminates this parameter and
  guarantees no accidental interaction between the replication
  machinery and the process being replicated -- i.e. no accidental
  sharing of names used by the process to get its work done and the
  name(s) used by the replication to effect copying. This latter
  revision of the definition of replication is crucial to obtaining
  the expected identity $!!P \sim !P$.
\end{remark}

\begin{remark}\label{rem:paradoxical_combinator}
  The reader familiar with the lambda calculus will have noticed the
  similarity between $D$ and the paradoxical combinator.

  [Ed. note: the existence of this seems to suggest we have to be more
  restrictive on the set of processes and names we admit if we are to
  support no-cloning.]
\end{remark}

\subsubsection{Bisimulation}

The computational dynamics gives rise to another kind of equivalence,
the equivalence of computational behavior. As previously mentioned
this is typically captured \emph{via} some form of bisimulation.

% The notion we use in this paper is weak barbed bisimulation
% \cite{milner91polyadicpi}.

The notion we use in this paper is derived from weak barbed
bisimulation \cite{milner91polyadicpi}. 

\begin{definition}
An \emph{observation relation}, $\downarrow_{\mathcal N}$, over a set
of names, $\mathcal N$, is the smallest relation satisfying the rules
below.

\infrule[Out-barb]{y \in {\mathcal N}, \; x \nameeq y}
		  {\outputp{x}{v} \downarrow_{\mathcal N} x}
\infrule[Par-barb]{\mbox{$P\downarrow_{\mathcal N} x$ or $Q\downarrow_{\mathcal N} x$}}
		  {\binpar{P}{Q} \downarrow_{\mathcal N} x}

We write $P \Downarrow_{\mathcal N} x$ if there is $Q$ such that 
$P \wred Q$ and $Q \downarrow_{\mathcal N} x$.
\end{definition}

\begin{definition}
%\label{def.bbisim}
An  ${\mathcal N}$-\emph{barbed bisimulation} over a set of names, ${\mathcal N}$, is a symmetric binary relation 
${\mathcal S}_{\mathcal N}$ between agents such that $P\rel{S}_{\mathcal N}Q$ implies:
\begin{enumerate}
\item If $P \red P'$ then $Q \wred Q'$ and $P'\rel{S}_{\mathcal N} Q'$.
\item If $P\downarrow_{\mathcal N} x$, then $Q\Downarrow_{\mathcal N} x$.
\end{enumerate}
$P$ is ${\mathcal N}$-barbed bisimilar to $Q$, written
$P \wbbisim_{\mathcal N} Q$, if $P \rel{S}_{\mathcal N} Q$ for some ${\mathcal N}$-barbed bisimulation ${\mathcal S}_{\mathcal N}$.
\end{definition}

$\mathcal{R} \subseteq \pi \times \pi$

$P \mathcal{R} Q => \forall P'. P \red P' \Rightarrow \exists Q'. Q \red Q', P' \mathcal{R} Q'$

$P \vdash x \Rightarrow Q \vdash x$

\begin{mathpar}
  \inferrule*[lab=Out-barb]{x \nameeq y}{{y}!\langle{Q}\rangle \vdash x}
  \and
  \inferrule*[lab=Par-barb]{\mbox{$P\vdash x$ or $Q\vdash x$}}{\binpar{P}{Q} \vdash x}
\end{mathpar}

\subsubsection{Contexts}

One of the principle advantages of computational calculi like the
$\pi$-calculus is a well-defined notion of context,
contextual-equivalence and a correlation between
contextual-equivalence and notions of bisimulation. The notion of
context allows the decomposition of a process into (sub-)process and
its syntactic environment, its context. Thus, a context may be
thought of as a process with a ``hole'' (written $\Box$) in it. The
application of a context $M$ to a process $P$, written $M[P]$, is
tantamount to filling the hole in $M$ with $P$. In this paper we do
not need the full weight of this theory, but do make use of the notion
of context in the proof the main theorem. 

\begin{mathpar}
  \inferrule* [lab=summation] {} {{M_{M},M_{N}} \bc \Box \;|\; x.M_{A} \;|\; M_{M}+M_{N}}
  \and
  \inferrule* [lab=agent] {} {{M_{A}} \bc (\vec{x})M_{P} \;| \; \clift{P_0,\ldots,M_{P},\ldots,P_N}}
  \and \\
  \inferrule* [lab=process] {} {{M_{P}} \bc M_{N} \;| \;P|M_{P} }
\end{mathpar} 

\begin{mathpar}
  \inferrule* [lab=sychronization] {} {M_{N} \bc \Box \;|\; x?M_{F} \;|\; x!M_{C}}
  \and
  \inferrule* [lab=abstraction] {} {{M_{F}} \bc (x)M_{P} }
  \and
  \inferrule* [lab=concretion] {} {{M_{C}} \bc \langle M_{P} \rangle }
  \and \\
  \inferrule* [lab=process] {} {{M_{P}} \bc M_{N} \;| \;P|M_{P} }
\end{mathpar}

\begin{definition}[contextual application] Given a context $M$, and
  process $P$, we define the \emph{contextual application}, $M[P] :=
  M\{P/\Box\}$. That is, the contextual application of M to P is the
  substitution of $P$ for $\Box$ in $M$.
\end{definition}

$\meaningof{-} : L \to \mathcal{P}(\pi)$

\begin{mathpar}
  \inferrule* [lab=collection] {} {\meaningof{true} = \pi, \and \meaningof{~E} = \pi \setminus \meaningof{E}, \and \meaningof{E_{1} \& E_{2}} = \meaningof{E_{1}} \cap \meaningof{E_{2}}}
\end{mathpar}

\begin{mathpar}
  \inferrule* [lab=structure] {} {\meaningof{0} = \{ P \in \pi | P \equiv 0 \}, \and \\ \meaningof{E_1 | E_2} = \{ P \in \pi | P \equiv P_{1} | P_{2}, P_{1} \in \meaningof{E_{1}}, P_{2} \in \meaningof{E_2}\} }
\end{mathpar}

\begin{mathpar}
 \inferrule* [lab=behavior] {} {\meaningof{\langle a?b \rangle E} = \{ P \in \pi | P \equiv Q | u?(y)P', \\ \and \\\\ \and \\ \;\;\; u \in \meaningof{a}, \forall z.P'\{z/y\} \in \meaningof{E\{z/b\}}\}, \and \\ \meaningof{a!E} = \{ P \in \pi | P \equiv Q | x!\langle P' \rangle, x \in \meaningof{a} P' \in \meaningof{E}\} }
\end{mathpar}

\begin{mathpar}
 \inferrule* [lab=nominal] {} {\meaningof{\quotep{E}} = \{ \quotep{P} \in \quotep{\pi} | P \in \meaningof{E} \}, \and \meaningof{\quotep{P}} = \{ \quotep{Q} \in \quotep{\pi} | P \equiv Q \} \and \\ \meaningof{@\quotep{E}} = \{ P \in \pi | P \equiv @x, x \in \meaningof{E} \}}
\end{mathpar}

\begin{eqnarray*}
  \\
  \meaningof{-} : TS \to ST
\end{eqnarray*}

\begin{eqnarray*}
  \\
  L : TS \to ST
\end{eqnarray*}

\begin{eqnarray*}
  \\
  P \models E \iff P \in \meaningof{E}
\end{eqnarray*}

\begin{eqnarray*}
  P \approx_{L} Q \iff \forall E \in L. P \models E \iff Q \models E
\end{eqnarray*}

\begin{eqnarray*}
  P \approx_{K} Q
\end{eqnarray*}

\begin{eqnarray*}
  P \approx Q
\end{eqnarray*}

$\approx_{K} = \approx = \approx_{L}$

\subsubsection{Contextual duality}

Note that contexts extend the quotation operation to a family of
operations from processes to names. Given a context, $M$, we can
define a \emph{nominal context}, $\quotep{M}$ by $\quotep{M}[P] :=
\quotep{M[P]}$. To foreshadow what is to come we observe that these
operations enjoy a duality with processes very much like the duality
between vectors and maps from vectors to scalars.

Further, because the calculus is essentially higher-order, we have a
correspondence between contexts and processes. More specifically,
given a name $x$ and a context $M$ we can construct $M^{*}_{x}$ such
that 

\begin{mathpar}
  M^{*}_{x} | \lift{x}{P} \red M[P]
\end{mathpar}

namely,

\begin{mathpar}
  M^{*}_{x} := x?(u).M[\dropn{u}]
\end{mathpar}

The dependence of $M^{*}_{x}$ on a name makes it an abstraction, 

\begin{mathpar}
  M^{*} := (x)x?(u).M[\dropn{u}]
\end{mathpar}

\subsection{Additional notation}

It will sometimes be convenient to denote the process a name
quotes. We already have the notation $x = \quotep{P}$, but it will be
convenient to introduce an alternate notation, $\procn{x}$, when we
want to emphasize the connection to the use of the name. Note that, by
virtue of name equivalence, $\quotep{\procn{x}} \nameeq x$; so, the
notation is consistent with previous definitions.

Further, because names have structure it is possible to effect
substitutions on the basis of that structure. This means we need to
upgrade our notation for substitutions, which we accomplish by
adapting comprehension notation. Thus,

\begin{mathpar}
  P\{ y / x : x \in S \}
\end{mathpar}

is interpreted to mean the process derived from P by replacing (in a
capture-avoiding manner) each occurrence of $x$ in $S$ by $y$. For example,

\begin{mathpar}
  P\{ \quotep{\procn{x}|\procn{x}} / x : x \in \freenames{P} \}
\end{mathpar}

will replace each (occurrence) of a free name $x$ in $P$ by
$\quotep{\procn{x}|\procn{x}}$.

Also, we will avail ourselves of the notation $x^{L}$ and $x^{R}$ to
denote injections of a name into disjoint copies of the name
space. There are numerous ways to accomplish this. One example can be
found in \cite{MeredithR05}. This notation overloads to vectors of
names: $\vec{x}^{\pi} := (x_{i}^{\pi} \; : \; 0 \leq i < |\vec{x}| )$ where $\pi \in \{L,R\}$.

We also use $P^{\Box} := P|\Box$.

In \cite{MeredithR05} an interpretation of the new operator is
given. It turns out that there are several possible interpretations
all enjoying the requisite algebraic properties of the operator (see
\cite{milner91polyadicpi}). We will therefore make liberal use of
$(\nu\; \vec{x})P$.

% subsection the_syntax_and_semantics_of_the_notation_system (end)   

\input{qm2pi.qmops} 

\input{qm2pi.sterngerlach} 

\input{qm2pi.metric} 

% section concurrent_process_calculi (end)

%\input{qm2pi.proofsketch}

% section proof sketch (end)

%\input{qm2pi.slviaknots} 

% section spatial logic via knots (end)

\input{qm2pi.conclusion}

% section conclusion (end)

%\input{qm2pi.dtcodes} 

% section wiring algorithm (end)

\input{qm2pi.ack} 

% section acknowledgments (end)

\newpage


\bibliographystyle{plain}   
\bibliography{../../biblios/main.bib}

\input{qm2pi.rhodetails}

\end{document}

 

\documentclass[12pt]{llncs}
%\documentclass{jktr}

\usepackage[pdftex]{hyperref}                   
\usepackage {listings}
\usepackage {mathpartir}
\usepackage{bcprules}
%\usepackage{listings}
                       
\usepackage{graphicx} 
%\usepackage[margins=2.5cm,nohead,nofoot]{geometry}
%\usepackage{geometry}
\usepackage{amsfonts}
\usepackage{amstext}
\usepackage{latexsym}
\usepackage{amssymb}
\usepackage{color}


%\include{myPreamble}
\include{qm2pi.local} 

%\ifpdf
%\usepackage[pdftex]{graphicx}
%\else
%\usepackage{graphicx}
%\fi

 % \ifpdf
%  \usepackage{pdfsync}
%  \if


%\title{Brief Article}
%\author{David F. Snyder}
%\author{L.G. Meredith}

%\address{Dept. of Math., Texas State University--San Marcos, San Marcos, TX 78666}
       
\pagestyle{empty}


\begin{document}

\lstset{language=[Objective]Caml,frame=shadowbox}

\input{qm2pi.front}

% section front matter (end)

\input{qm2pi.intro} 
 
% section introduction (end)

% \input{qm2pi.knotations} 

% section notation (end)

\input{qm2pi.process.calculi} 

% section concurrent_process_calculi_and_spatial_logics_ (end)
    
%\input{qm2pi.knots2pi} 

%\input{qm2pi.trefoil} 

%\input{qm2pi.mainthm} 

% subsection basic_interpretation (end)

%\input{qm2pi.rho.presentation} 
\subsection{The syntax and semantics of the notation system}\label{sub:the_syntax_and_semantics_of_the_notation_system} % (fold)

We now summarize a technical presentation of the calculus that
embodies our theory of dynamics. The typical presentation of such a
calculus follows the style of giving generators and relations on
them. The grammar, below, describing term constructors, freely
generates the set of processes, $\Proc$. This set is then quotiented
by a relation known as structural congruence and it is over this set
that the notion of dynamics is expressed. This presentation is
essentially that of \cite{MeredithR05} with the addition of
polyadicity and summation. For readability we have relegated some of
the technical subtleties to an appendix.

\subsubsection{Process grammar}\label{subsub:process_grammar}

\begin{mathpar}
  \inferrule* [lab=synchronization] {} {{M} \bc \pzero \;|\; x?F \;|\; x!C }
  \and
  \inferrule* [lab=abstraction] {} {{F} \bc (x)P}
  \and
  \inferrule* [lab=concretion] {} {{C} \bc \langle Q \rangle}
  \and
  \inferrule* [lab=process] {} {{P,Q} \bc M \;| \;P|Q \;|\; @{x}}
  \and
  \inferrule* [lab=name] {} {{x} \bc \quotep{P}}
\end{mathpar} 

Note that $\vec{x}$ (resp. $\vec{P}$) denotes a vector of names
(resp. processes) of length $|\vec{x}|$ (resp. $|\vec{P}|$). We adopt
the following useful abbreviations.

\begin{mathpar}
   x?(\vec{y}).P := x.(\vec{y})P \and  x\clift{\vec{P}} := x.\clift{\vec{P}}
   \and x!(y) := \lift{x}{\dropn{y}}
   \and \Pi_{i=0}^{n-1}P_i := P_0 | \ldots | P_{n-1}
\end{mathpar}

\subsubsection{Structural congruence}

\paragraph{Free and bound names and alpha-equivalence.} At the
core of structural equivalence is alpha-equivalence which identifies
process that are the same up to a change of variable. Formally, we
recognize the distinction between free and bound names. The free names
of a process, $\freenames{P}$, may be calculated recursively as
follows:

\begin{mathpar}
\freenames{\pzero} := \emptyset
  \and \\
  \freenames{x?(y).P} := \{ x \} \cup (\freenames{P} \setminus \{ y \})
  \and 
  \freenames{x!\langle P \rangle} := \{ x \} \cup \{ P \} 
  \and \\
  \freenames{P|Q} := \freenames{P} \cup \freenames{Q}
  \and \\
  \freenames{@{x}} := \{ x \}
\end{mathpar}

$\pi$
$\quotep{\pi}$

$\freenames{-} : \pi \to \mathcal{P}(\quotep{\pi})$

\begin{eqnarray*}
  \freenames{\pzero} & := & \emptyset \\
  \freenames{x?(y).P} & := & \{ x \} \cup (\freenames{P} \setminus \{ y \}) \\
  \freenames{x!\langle P \rangle} & := & \{ x \} \cup \{ P \} \\
  \freenames{P|Q} & := & \freenames{P} \cup \freenames{Q} \\
  \freenames{\dropn{x}} & := & \{ x \}
\end{eqnarray*}

The bound names of a process, $\boundnames{P}$, are those names occurring in $P$
that are not free. For example, in $x?(y).0$, the name $x$ is free, while $y$ is bound.

\begin{mathpar}
  \inferrule* [lab=monoidal-laws] {} { P|Q \equiv Q|P \and P|0 \equiv P \and P|(Q|R) \equiv (P|Q)|R }
\end{mathpar}

\begin{mathpar}
  \inferrule* [lab=alpha-equivalence] {} { (x)P \equiv (y)P\{y/x\} \and y \not\in \freenames{P} }
\end{mathpar}

\begin{definition}
Then two processes, $P,Q$, are alpha-equivalent if $P = Q\{\vec{y}/\vec{x}\}$ for
some $\vec{x} \in \boundnames{Q},\vec{y} \in \boundnames{P}$, where $Q\{\vec{y}/\vec{x}\}$
denotes the capture-avoiding substitution of $\vec{y}$ for $\vec{x}$ in $Q$.
\end{definition}

\begin{definition}
  The {\em structural congruence} \cite{SangiorgiWalker} , $\equiv$,
  between processes is the least congruence containing
  alpha-equivalence, satisfying the abelian monoid laws
  (associativity, commutativity and $\pzero$ as identity) for parallel
  composition $|$ and for summation $+$.
\end{definition}

\subsection{Name equivalence}

We take name equivalence, written $\nameeq$, to be the smallest
equivalence relation generated by the following rules.

\begin{mathpar}
\inferrule*[lab=Quote-drop]
{ }
{ \quotep{@{x}} \nameeq x }

\inferrule*[lab=Struct-equiv]
{ P \scong Q }
{ \quotep{P} \nameeq \quotep{Q} }
\end{mathpar}

The astute reader will have noticed that the mutual recursion of names
and processes imposes a mutual recursion on alpha-equivalence and
structural equivalence via name-equivalence. Fortunately, all of this
works out pleasantly and we may calculate in the natural way, free of
concern. The reader interested in the details is referred to the
appendix \ref{appendix:rho_details}.

\subsection{Substitution}

We use $\Proc$ for the set of processes, $\QProc$ for the set of
names, and $\id{\{}\vec{y} / \vec{x} \id{\}}$ to denote partial maps,
$s : \QProc \rightarrow \QProc$. A map, $s$ lifts, uniquely, to a map
on process terms, $\widehat{s} : \Proc \rightarrow \Proc$ by the
following equations.

\begin{mathpar}
  (0) \psubstp{Q}{P} := 0 \\
  (R \juxtap S) \psubstp{Q}{P}
  :=    
  (R)\psubstp{Q}{P} \juxtap (S) \psubstp{Q}{P} \\
  (x?(y).R) \psubstp{Q}{P}    
  :=    
  (x)\substp{Q}{P} (z)\concat( (R \psubstn{z}{y}) \psubstp{Q}{P} ) \\
  (\lift{x}{R}) \psubstp{Q}{P}  
  :=
  \lift{(x)\substp{Q}{P}}{ R \psubstp{Q}{P} } \\
%   (\dropn{x})  \psubstp{Q}{P}       
%   := 
%   \left\{ 
%     \begin{array}{ccc} 
%       \dropn{\quotep{Q}} & & x \nameeq \quotep{P} \\
%       \dropn{x} & & otherwise \\
%     \end{array}
%   \right. 
  (\dropn{x})  \psubstp{Q}{P}       
  := 
  \left\{ 
    \begin{array}{ccc} 
      Q & & x \nameeq \quotep{P} \\
      \dropn{x} & & otherwise \\
    \end{array}
  \right.
\end{mathpar}
 

where

\begin{eqnarray}
  (x)\id{\{} \lpquote Q \rpquote / \lpquote P \rpquote \id{\}}            = 
  \left\{ 
    \begin{array}{ccc}
      \lpquote Q \rpquote & & x \nameeq \lpquote P \rpquote \\
      x & & otherwise \\
    \end{array}
  \right. \nonumber
\end{eqnarray}

and $z$ is chosen distinct from $\quotep{P}$, $\quotep{Q}$, the free
names in $Q$, and all the names in $R$. Our $\alpha$-equivalence will
be built in the standard way from this substitution.

\begin{remark}\label{rem:no_self_referential_names}
  One consequence of these definitions is that $\forall P. \quotep{P}
  \not\in \freenames{P}$.
\end{remark}

\subsection{ Dynamic quote: an example }

Anticipating something of what's to come, consider applying the
substitution, $\widehat{\id{\{}u / z \id{\}}}$, to the following pair
of processes, $\lift{w}{y!(z)}$ and $w[ \lpquote y!(z) \rpquote ]$.

\begin{eqnarray}
	\lift{w}{y!(z)}\widehat{\id{\{}u / z \id{\}}}
		& = &
		\lift{w}{y!(u)} \nonumber\\
	w[ \lpquote y!(z) \rpquote ] \widehat{ \id{\{}u / z \id{\}} }
		& = &
		w[ \lpquote y!(z) \rpquote ] \nonumber
\end{eqnarray}

Because the body of the process between quotes is impervious to
substitution, we get radically different answers. In fact, by
examining the first process in an input context,
e.g. $x?(z).\lift{w}{y!(z)}$, we see that the process under the lift
operator may be shaped by prefixed inputs binding a name inside it. In
this sense, the lift operator will be seen as a way to dynamically
construct processes before reifying them as names.

Finally equipped with these standard features we can present the
dynamics of the calculus.

\subsubsection{Operational semantics} 

Finally, we introduce the computational dynamics. What marks these
algebras as distinct from other more traditionally studied algebraic
structures, e.g. vector spaces or polynomial rings, is the manner in
which dynamics is captured. In traditional structures, dynamics is typically
expressed through morphisms between such structures, as in linear maps
between vector spaces or morphisms between rings. In algebras
associated with the semantics of computation, the dynamics is
expressed as part of the algebraic structure itself, through a
reduction reduction relation typically denoted by $\red$. Below, we
give a recursive presentation of this relation for the calculus used
in the encoding.

$\red \subseteq \pi \times \pi$
$\red : \pi \to \mathcal{P}(\pi)$

\begin{mathpar}
  \inferrule* [lab=Comm] { \textsf{match}( x_{src}, x_{trgt} ) } { x_{trgt}?(y)P \; | \; x_{src}!\langle {Q} \rangle \red P\{\quotep{Q}/y}\} }
  \and \\
  \inferrule* [lab=Par] {{P} \red {P}'} {{{P} | {Q}} \red {{P}' | {Q}}}
  \and
  \inferrule* [lab=Equiv]{{{P} \scong {P}'} \andalso {{P}' \red {Q}'} \andalso {{Q}' \scong {Q}}}{{P} \red {Q}}
\end{mathpar}

\begin{eqnarray*}
  match_{\equiv} (\quotep{P},\quotep{Q}) & := & P \equiv Q \\
  match_{\dagger}(\quotep{P},\quotep{Q}) & := & \forall R. P|Q \red^{*} R => R \red^{*} 0 \\
  match_{K}(\quotep{P},\quotep{Q}) & := & K \mbox{ for some context } K
\end{eqnarray*}

$u?(x)P | u!\langle Q \rangle \red P\{\quotep{Q}/x\}$

%We write $\wred$ for $\red^*$, and $P\red$ if $\exists Q $ such that $ P \red Q$.
We write $P\red$ if $\exists Q $ such that $ P \red Q$ and $P\not\red$, otherwise.

\section{Replication}

As mentioned before, it is known that replication (and hence
recursion) can be implemented in a higher-order process algebra
\cite{SangiorgiWalker}. As our first example of calculation with the
machinery thus far presented we give the construction explicitly in
the {\rhoc}.

\begin{eqnarray}
	D_{x} & := & \prefix{x}{y}{(\binpar{\outputp{x}{y}}{@{y}})} \nonumber\\
	\bangp_{x}{P} & := & \binpar{{x}!\langle{\binpar{D_{x}}{P}}\rangle}{D_{x}} \nonumber
\end{eqnarray}

\begin{eqnarray}
	\bangp_{x}{P} & & \nonumber\\
	=
	& {x}!\langle{(\prefix{x}{y}{(\outputp{x}{y} | @{y})) | P}}\rangle 
	      | \prefix{x}{y}{(\outputp{x}{y} | @{y})} & \nonumber\\
	\red
	& (\outputp{x}{y} | @{y})\substn{\quotep{(\prefix{x}{y}{(@{y} | \outputp{x}{y})) | P}}}{y} & \nonumber\\
	=
	& \outputp{x}{\quotep{(\prefix{x}{y}{(\outputp{x}{y} | @{y})) | P}}}
	  | {(\prefix{x}{y}{(\outputp{x}{y} | @{y})) | P}} & \nonumber\\
	\red
	& \ldots & \nonumber\\
	\red^*
	& P | P | \ldots & \nonumber
\end{eqnarray}

Of course, this encoding, as an implementation, runs away, unfolding
$\bangp{P}$ eagerly. A lazier and more implementable replication
operator, restricted to input-guarded processes, may be obtained as follows.

\begin{eqnarray}
\bangp{\prefix{u}{v}{P}} 
	:= 
	\binpar{\lift{x}{\prefix{u}{v}{(\binpar{D(x)}{P})}}}{D(x)} \nonumber
\end{eqnarray}

\begin{remark}
  Note that the lazier definition still does not deal with summation
  or mixed summation (i.e. sums over input and output). The reader is
  invited to construct definitions of replication that deal with these
  features. 

  Further, the definitions are parameterized in a name, $x$. Can you,
  gentle reader, make a definition that eliminates this parameter and
  guarantees no accidental interaction between the replication
  machinery and the process being replicated -- i.e. no accidental
  sharing of names used by the process to get its work done and the
  name(s) used by the replication to effect copying. This latter
  revision of the definition of replication is crucial to obtaining
  the expected identity $!!P \sim !P$.
\end{remark}

\begin{remark}\label{rem:paradoxical_combinator}
  The reader familiar with the lambda calculus will have noticed the
  similarity between $D$ and the paradoxical combinator.

  [Ed. note: the existence of this seems to suggest we have to be more
  restrictive on the set of processes and names we admit if we are to
  support no-cloning.]
\end{remark}

\subsubsection{Bisimulation}

The computational dynamics gives rise to another kind of equivalence,
the equivalence of computational behavior. As previously mentioned
this is typically captured \emph{via} some form of bisimulation.

% The notion we use in this paper is weak barbed bisimulation
% \cite{milner91polyadicpi}.

The notion we use in this paper is derived from weak barbed
bisimulation \cite{milner91polyadicpi}. 

\begin{definition}
An \emph{observation relation}, $\downarrow_{\mathcal N}$, over a set
of names, $\mathcal N$, is the smallest relation satisfying the rules
below.

\infrule[Out-barb]{y \in {\mathcal N}, \; x \nameeq y}
		  {\outputp{x}{v} \downarrow_{\mathcal N} x}
\infrule[Par-barb]{\mbox{$P\downarrow_{\mathcal N} x$ or $Q\downarrow_{\mathcal N} x$}}
		  {\binpar{P}{Q} \downarrow_{\mathcal N} x}

We write $P \Downarrow_{\mathcal N} x$ if there is $Q$ such that 
$P \wred Q$ and $Q \downarrow_{\mathcal N} x$.
\end{definition}

\begin{definition}
%\label{def.bbisim}
An  ${\mathcal N}$-\emph{barbed bisimulation} over a set of names, ${\mathcal N}$, is a symmetric binary relation 
${\mathcal S}_{\mathcal N}$ between agents such that $P\rel{S}_{\mathcal N}Q$ implies:
\begin{enumerate}
\item If $P \red P'$ then $Q \wred Q'$ and $P'\rel{S}_{\mathcal N} Q'$.
\item If $P\downarrow_{\mathcal N} x$, then $Q\Downarrow_{\mathcal N} x$.
\end{enumerate}
$P$ is ${\mathcal N}$-barbed bisimilar to $Q$, written
$P \wbbisim_{\mathcal N} Q$, if $P \rel{S}_{\mathcal N} Q$ for some ${\mathcal N}$-barbed bisimulation ${\mathcal S}_{\mathcal N}$.
\end{definition}

$\mathcal{R} \subseteq \pi \times \pi$

$P \mathcal{R} Q => \forall P'. P \red P' \Rightarrow \exists Q'. Q \red Q', P' \mathcal{R} Q'$

$P \vdash x \Rightarrow Q \vdash x$

\begin{mathpar}
  \inferrule*[lab=Out-barb]{x \nameeq y}{{y}!\langle{Q}\rangle \vdash x}
  \and
  \inferrule*[lab=Par-barb]{\mbox{$P\vdash x$ or $Q\vdash x$}}{\binpar{P}{Q} \vdash x}
\end{mathpar}

\subsubsection{Contexts}

One of the principle advantages of computational calculi like the
$\pi$-calculus is a well-defined notion of context,
contextual-equivalence and a correlation between
contextual-equivalence and notions of bisimulation. The notion of
context allows the decomposition of a process into (sub-)process and
its syntactic environment, its context. Thus, a context may be
thought of as a process with a ``hole'' (written $\Box$) in it. The
application of a context $M$ to a process $P$, written $M[P]$, is
tantamount to filling the hole in $M$ with $P$. In this paper we do
not need the full weight of this theory, but do make use of the notion
of context in the proof the main theorem. 

\begin{mathpar}
  \inferrule* [lab=summation] {} {{M_{M},M_{N}} \bc \Box \;|\; x.M_{A} \;|\; M_{M}+M_{N}}
  \and
  \inferrule* [lab=agent] {} {{M_{A}} \bc (\vec{x})M_{P} \;| \; \clift{P_0,\ldots,M_{P},\ldots,P_N}}
  \and \\
  \inferrule* [lab=process] {} {{M_{P}} \bc M_{N} \;| \;P|M_{P} }
\end{mathpar} 

\begin{mathpar}
  \inferrule* [lab=sychronization] {} {M_{N} \bc \Box \;|\; x?M_{F} \;|\; x!M_{C}}
  \and
  \inferrule* [lab=abstraction] {} {{M_{F}} \bc (x)M_{P} }
  \and
  \inferrule* [lab=concretion] {} {{M_{C}} \bc \langle M_{P} \rangle }
  \and \\
  \inferrule* [lab=process] {} {{M_{P}} \bc M_{N} \;| \;P|M_{P} }
\end{mathpar}

\begin{definition}[contextual application] Given a context $M$, and
  process $P$, we define the \emph{contextual application}, $M[P] :=
  M\{P/\Box\}$. That is, the contextual application of M to P is the
  substitution of $P$ for $\Box$ in $M$.
\end{definition}

$\meaningof{-} : L \to \mathcal{P}(\pi)$

\begin{mathpar}
  \inferrule* [lab=collection] {} {\meaningof{true} = \pi, \and \meaningof{~E} = \pi \setminus \meaningof{E}, \and \meaningof{E_{1} \& E_{2}} = \meaningof{E_{1}} \cap \meaningof{E_{2}}}
\end{mathpar}

\begin{mathpar}
  \inferrule* [lab=structure] {} {\meaningof{0} = \{ P \in \pi | P \equiv 0 \}, \and \\ \meaningof{E_1 | E_2} = \{ P \in \pi | P \equiv P_{1} | P_{2}, P_{1} \in \meaningof{E_{1}}, P_{2} \in \meaningof{E_2}\} }
\end{mathpar}

\begin{mathpar}
 \inferrule* [lab=behavior] {} {\meaningof{\langle a?b \rangle E} = \{ P \in \pi | P \equiv Q | u?(y)P', \\ \and \\\\ \and \\ \;\;\; u \in \meaningof{a}, \forall z.P'\{z/y\} \in \meaningof{E\{z/b\}}\}, \and \\ \meaningof{a!E} = \{ P \in \pi | P \equiv Q | x!\langle P' \rangle, x \in \meaningof{a} P' \in \meaningof{E}\} }
\end{mathpar}

\begin{mathpar}
 \inferrule* [lab=nominal] {} {\meaningof{\quotep{E}} = \{ \quotep{P} \in \quotep{\pi} | P \in \meaningof{E} \}, \and \meaningof{\quotep{P}} = \{ \quotep{Q} \in \quotep{\pi} | P \equiv Q \} \and \\ \meaningof{@\quotep{E}} = \{ P \in \pi | P \equiv @x, x \in \meaningof{E} \}}
\end{mathpar}

\begin{eqnarray*}
  \\
  \meaningof{-} : TS \to ST
\end{eqnarray*}

\begin{eqnarray*}
  \\
  L : TS \to ST
\end{eqnarray*}

\begin{eqnarray*}
  \\
  P \models E \iff P \in \meaningof{E}
\end{eqnarray*}

\begin{eqnarray*}
  P \approx_{L} Q \iff \forall E \in L. P \models E \iff Q \models E
\end{eqnarray*}

\begin{eqnarray*}
  P \approx_{K} Q
\end{eqnarray*}

\begin{eqnarray*}
  P \approx Q
\end{eqnarray*}

$\approx_{K} = \approx = \approx_{L}$

\subsubsection{Contextual duality}

Note that contexts extend the quotation operation to a family of
operations from processes to names. Given a context, $M$, we can
define a \emph{nominal context}, $\quotep{M}$ by $\quotep{M}[P] :=
\quotep{M[P]}$. To foreshadow what is to come we observe that these
operations enjoy a duality with processes very much like the duality
between vectors and maps from vectors to scalars.

Further, because the calculus is essentially higher-order, we have a
correspondence between contexts and processes. More specifically,
given a name $x$ and a context $M$ we can construct $M^{*}_{x}$ such
that 

\begin{mathpar}
  M^{*}_{x} | \lift{x}{P} \red M[P]
\end{mathpar}

namely,

\begin{mathpar}
  M^{*}_{x} := x?(u).M[\dropn{u}]
\end{mathpar}

The dependence of $M^{*}_{x}$ on a name makes it an abstraction, 

\begin{mathpar}
  M^{*} := (x)x?(u).M[\dropn{u}]
\end{mathpar}

\subsection{Additional notation}

It will sometimes be convenient to denote the process a name
quotes. We already have the notation $x = \quotep{P}$, but it will be
convenient to introduce an alternate notation, $\procn{x}$, when we
want to emphasize the connection to the use of the name. Note that, by
virtue of name equivalence, $\quotep{\procn{x}} \nameeq x$; so, the
notation is consistent with previous definitions.

Further, because names have structure it is possible to effect
substitutions on the basis of that structure. This means we need to
upgrade our notation for substitutions, which we accomplish by
adapting comprehension notation. Thus,

\begin{mathpar}
  P\{ y / x : x \in S \}
\end{mathpar}

is interpreted to mean the process derived from P by replacing (in a
capture-avoiding manner) each occurrence of $x$ in $S$ by $y$. For example,

\begin{mathpar}
  P\{ \quotep{\procn{x}|\procn{x}} / x : x \in \freenames{P} \}
\end{mathpar}

will replace each (occurrence) of a free name $x$ in $P$ by
$\quotep{\procn{x}|\procn{x}}$.

Also, we will avail ourselves of the notation $x^{L}$ and $x^{R}$ to
denote injections of a name into disjoint copies of the name
space. There are numerous ways to accomplish this. One example can be
found in \cite{MeredithR05}. This notation overloads to vectors of
names: $\vec{x}^{\pi} := (x_{i}^{\pi} \; : \; 0 \leq i < |\vec{x}| )$ where $\pi \in \{L,R\}$.

We also use $P^{\Box} := P|\Box$.

In \cite{MeredithR05} an interpretation of the new operator is
given. It turns out that there are several possible interpretations
all enjoying the requisite algebraic properties of the operator (see
\cite{milner91polyadicpi}). We will therefore make liberal use of
$(\nu\; \vec{x})P$.

% subsection the_syntax_and_semantics_of_the_notation_system (end)   

\input{qm2pi.qmops} 

\input{qm2pi.sterngerlach} 

\input{qm2pi.metric} 

% section concurrent_process_calculi (end)

%\input{qm2pi.proofsketch}

% section proof sketch (end)

%\input{qm2pi.slviaknots} 

% section spatial logic via knots (end)

\input{qm2pi.conclusion}

% section conclusion (end)

%\input{qm2pi.dtcodes} 

% section wiring algorithm (end)

\input{qm2pi.ack} 

% section acknowledgments (end)

\newpage


\bibliographystyle{plain}   
\bibliography{../../biblios/main.bib}

\input{qm2pi.rhodetails}

\end{document}

 

% section concurrent_process_calculi (end)

%\documentclass[12pt]{llncs}
%\documentclass{jktr}

\usepackage[pdftex]{hyperref}                   
\usepackage {listings}
\usepackage {mathpartir}
\usepackage{bcprules}
%\usepackage{listings}
                       
\usepackage{graphicx} 
%\usepackage[margins=2.5cm,nohead,nofoot]{geometry}
%\usepackage{geometry}
\usepackage{amsfonts}
\usepackage{amstext}
\usepackage{latexsym}
\usepackage{amssymb}
\usepackage{color}


%\include{myPreamble}
\include{qm2pi.local} 

%\ifpdf
%\usepackage[pdftex]{graphicx}
%\else
%\usepackage{graphicx}
%\fi

 % \ifpdf
%  \usepackage{pdfsync}
%  \if


%\title{Brief Article}
%\author{David F. Snyder}
%\author{L.G. Meredith}

%\address{Dept. of Math., Texas State University--San Marcos, San Marcos, TX 78666}
       
\pagestyle{empty}


\begin{document}

\lstset{language=[Objective]Caml,frame=shadowbox}

\input{qm2pi.front}

% section front matter (end)

\input{qm2pi.intro} 
 
% section introduction (end)

% \input{qm2pi.knotations} 

% section notation (end)

\input{qm2pi.process.calculi} 

% section concurrent_process_calculi_and_spatial_logics_ (end)
    
%\input{qm2pi.knots2pi} 

%\input{qm2pi.trefoil} 

%\input{qm2pi.mainthm} 

% subsection basic_interpretation (end)

%\input{qm2pi.rho.presentation} 
\subsection{The syntax and semantics of the notation system}\label{sub:the_syntax_and_semantics_of_the_notation_system} % (fold)

We now summarize a technical presentation of the calculus that
embodies our theory of dynamics. The typical presentation of such a
calculus follows the style of giving generators and relations on
them. The grammar, below, describing term constructors, freely
generates the set of processes, $\Proc$. This set is then quotiented
by a relation known as structural congruence and it is over this set
that the notion of dynamics is expressed. This presentation is
essentially that of \cite{MeredithR05} with the addition of
polyadicity and summation. For readability we have relegated some of
the technical subtleties to an appendix.

\subsubsection{Process grammar}\label{subsub:process_grammar}

\begin{mathpar}
  \inferrule* [lab=synchronization] {} {{M} \bc \pzero \;|\; x?F \;|\; x!C }
  \and
  \inferrule* [lab=abstraction] {} {{F} \bc (x)P}
  \and
  \inferrule* [lab=concretion] {} {{C} \bc \langle Q \rangle}
  \and
  \inferrule* [lab=process] {} {{P,Q} \bc M \;| \;P|Q \;|\; @{x}}
  \and
  \inferrule* [lab=name] {} {{x} \bc \quotep{P}}
\end{mathpar} 

Note that $\vec{x}$ (resp. $\vec{P}$) denotes a vector of names
(resp. processes) of length $|\vec{x}|$ (resp. $|\vec{P}|$). We adopt
the following useful abbreviations.

\begin{mathpar}
   x?(\vec{y}).P := x.(\vec{y})P \and  x\clift{\vec{P}} := x.\clift{\vec{P}}
   \and x!(y) := \lift{x}{\dropn{y}}
   \and \Pi_{i=0}^{n-1}P_i := P_0 | \ldots | P_{n-1}
\end{mathpar}

\subsubsection{Structural congruence}

\paragraph{Free and bound names and alpha-equivalence.} At the
core of structural equivalence is alpha-equivalence which identifies
process that are the same up to a change of variable. Formally, we
recognize the distinction between free and bound names. The free names
of a process, $\freenames{P}$, may be calculated recursively as
follows:

\begin{mathpar}
\freenames{\pzero} := \emptyset
  \and \\
  \freenames{x?(y).P} := \{ x \} \cup (\freenames{P} \setminus \{ y \})
  \and 
  \freenames{x!\langle P \rangle} := \{ x \} \cup \{ P \} 
  \and \\
  \freenames{P|Q} := \freenames{P} \cup \freenames{Q}
  \and \\
  \freenames{@{x}} := \{ x \}
\end{mathpar}

$\pi$
$\quotep{\pi}$

$\freenames{-} : \pi \to \mathcal{P}(\quotep{\pi})$

\begin{eqnarray*}
  \freenames{\pzero} & := & \emptyset \\
  \freenames{x?(y).P} & := & \{ x \} \cup (\freenames{P} \setminus \{ y \}) \\
  \freenames{x!\langle P \rangle} & := & \{ x \} \cup \{ P \} \\
  \freenames{P|Q} & := & \freenames{P} \cup \freenames{Q} \\
  \freenames{\dropn{x}} & := & \{ x \}
\end{eqnarray*}

The bound names of a process, $\boundnames{P}$, are those names occurring in $P$
that are not free. For example, in $x?(y).0$, the name $x$ is free, while $y$ is bound.

\begin{mathpar}
  \inferrule* [lab=monoidal-laws] {} { P|Q \equiv Q|P \and P|0 \equiv P \and P|(Q|R) \equiv (P|Q)|R }
\end{mathpar}

\begin{mathpar}
  \inferrule* [lab=alpha-equivalence] {} { (x)P \equiv (y)P\{y/x\} \and y \not\in \freenames{P} }
\end{mathpar}

\begin{definition}
Then two processes, $P,Q$, are alpha-equivalent if $P = Q\{\vec{y}/\vec{x}\}$ for
some $\vec{x} \in \boundnames{Q},\vec{y} \in \boundnames{P}$, where $Q\{\vec{y}/\vec{x}\}$
denotes the capture-avoiding substitution of $\vec{y}$ for $\vec{x}$ in $Q$.
\end{definition}

\begin{definition}
  The {\em structural congruence} \cite{SangiorgiWalker} , $\equiv$,
  between processes is the least congruence containing
  alpha-equivalence, satisfying the abelian monoid laws
  (associativity, commutativity and $\pzero$ as identity) for parallel
  composition $|$ and for summation $+$.
\end{definition}

\subsection{Name equivalence}

We take name equivalence, written $\nameeq$, to be the smallest
equivalence relation generated by the following rules.

\begin{mathpar}
\inferrule*[lab=Quote-drop]
{ }
{ \quotep{@{x}} \nameeq x }

\inferrule*[lab=Struct-equiv]
{ P \scong Q }
{ \quotep{P} \nameeq \quotep{Q} }
\end{mathpar}

The astute reader will have noticed that the mutual recursion of names
and processes imposes a mutual recursion on alpha-equivalence and
structural equivalence via name-equivalence. Fortunately, all of this
works out pleasantly and we may calculate in the natural way, free of
concern. The reader interested in the details is referred to the
appendix \ref{appendix:rho_details}.

\subsection{Substitution}

We use $\Proc$ for the set of processes, $\QProc$ for the set of
names, and $\id{\{}\vec{y} / \vec{x} \id{\}}$ to denote partial maps,
$s : \QProc \rightarrow \QProc$. A map, $s$ lifts, uniquely, to a map
on process terms, $\widehat{s} : \Proc \rightarrow \Proc$ by the
following equations.

\begin{mathpar}
  (0) \psubstp{Q}{P} := 0 \\
  (R \juxtap S) \psubstp{Q}{P}
  :=    
  (R)\psubstp{Q}{P} \juxtap (S) \psubstp{Q}{P} \\
  (x?(y).R) \psubstp{Q}{P}    
  :=    
  (x)\substp{Q}{P} (z)\concat( (R \psubstn{z}{y}) \psubstp{Q}{P} ) \\
  (\lift{x}{R}) \psubstp{Q}{P}  
  :=
  \lift{(x)\substp{Q}{P}}{ R \psubstp{Q}{P} } \\
%   (\dropn{x})  \psubstp{Q}{P}       
%   := 
%   \left\{ 
%     \begin{array}{ccc} 
%       \dropn{\quotep{Q}} & & x \nameeq \quotep{P} \\
%       \dropn{x} & & otherwise \\
%     \end{array}
%   \right. 
  (\dropn{x})  \psubstp{Q}{P}       
  := 
  \left\{ 
    \begin{array}{ccc} 
      Q & & x \nameeq \quotep{P} \\
      \dropn{x} & & otherwise \\
    \end{array}
  \right.
\end{mathpar}
 

where

\begin{eqnarray}
  (x)\id{\{} \lpquote Q \rpquote / \lpquote P \rpquote \id{\}}            = 
  \left\{ 
    \begin{array}{ccc}
      \lpquote Q \rpquote & & x \nameeq \lpquote P \rpquote \\
      x & & otherwise \\
    \end{array}
  \right. \nonumber
\end{eqnarray}

and $z$ is chosen distinct from $\quotep{P}$, $\quotep{Q}$, the free
names in $Q$, and all the names in $R$. Our $\alpha$-equivalence will
be built in the standard way from this substitution.

\begin{remark}\label{rem:no_self_referential_names}
  One consequence of these definitions is that $\forall P. \quotep{P}
  \not\in \freenames{P}$.
\end{remark}

\subsection{ Dynamic quote: an example }

Anticipating something of what's to come, consider applying the
substitution, $\widehat{\id{\{}u / z \id{\}}}$, to the following pair
of processes, $\lift{w}{y!(z)}$ and $w[ \lpquote y!(z) \rpquote ]$.

\begin{eqnarray}
	\lift{w}{y!(z)}\widehat{\id{\{}u / z \id{\}}}
		& = &
		\lift{w}{y!(u)} \nonumber\\
	w[ \lpquote y!(z) \rpquote ] \widehat{ \id{\{}u / z \id{\}} }
		& = &
		w[ \lpquote y!(z) \rpquote ] \nonumber
\end{eqnarray}

Because the body of the process between quotes is impervious to
substitution, we get radically different answers. In fact, by
examining the first process in an input context,
e.g. $x?(z).\lift{w}{y!(z)}$, we see that the process under the lift
operator may be shaped by prefixed inputs binding a name inside it. In
this sense, the lift operator will be seen as a way to dynamically
construct processes before reifying them as names.

Finally equipped with these standard features we can present the
dynamics of the calculus.

\subsubsection{Operational semantics} 

Finally, we introduce the computational dynamics. What marks these
algebras as distinct from other more traditionally studied algebraic
structures, e.g. vector spaces or polynomial rings, is the manner in
which dynamics is captured. In traditional structures, dynamics is typically
expressed through morphisms between such structures, as in linear maps
between vector spaces or morphisms between rings. In algebras
associated with the semantics of computation, the dynamics is
expressed as part of the algebraic structure itself, through a
reduction reduction relation typically denoted by $\red$. Below, we
give a recursive presentation of this relation for the calculus used
in the encoding.

$\red \subseteq \pi \times \pi$
$\red : \pi \to \mathcal{P}(\pi)$

\begin{mathpar}
  \inferrule* [lab=Comm] { \textsf{match}( x_{src}, x_{trgt} ) } { x_{trgt}?(y)P \; | \; x_{src}!\langle {Q} \rangle \red P\{\quotep{Q}/y}\} }
  \and \\
  \inferrule* [lab=Par] {{P} \red {P}'} {{{P} | {Q}} \red {{P}' | {Q}}}
  \and
  \inferrule* [lab=Equiv]{{{P} \scong {P}'} \andalso {{P}' \red {Q}'} \andalso {{Q}' \scong {Q}}}{{P} \red {Q}}
\end{mathpar}

\begin{eqnarray*}
  match_{\equiv} (\quotep{P},\quotep{Q}) & := & P \equiv Q \\
  match_{\dagger}(\quotep{P},\quotep{Q}) & := & \forall R. P|Q \red^{*} R => R \red^{*} 0 \\
  match_{K}(\quotep{P},\quotep{Q}) & := & K \mbox{ for some context } K
\end{eqnarray*}

$u?(x)P | u!\langle Q \rangle \red P\{\quotep{Q}/x\}$

%We write $\wred$ for $\red^*$, and $P\red$ if $\exists Q $ such that $ P \red Q$.
We write $P\red$ if $\exists Q $ such that $ P \red Q$ and $P\not\red$, otherwise.

\section{Replication}

As mentioned before, it is known that replication (and hence
recursion) can be implemented in a higher-order process algebra
\cite{SangiorgiWalker}. As our first example of calculation with the
machinery thus far presented we give the construction explicitly in
the {\rhoc}.

\begin{eqnarray}
	D_{x} & := & \prefix{x}{y}{(\binpar{\outputp{x}{y}}{@{y}})} \nonumber\\
	\bangp_{x}{P} & := & \binpar{{x}!\langle{\binpar{D_{x}}{P}}\rangle}{D_{x}} \nonumber
\end{eqnarray}

\begin{eqnarray}
	\bangp_{x}{P} & & \nonumber\\
	=
	& {x}!\langle{(\prefix{x}{y}{(\outputp{x}{y} | @{y})) | P}}\rangle 
	      | \prefix{x}{y}{(\outputp{x}{y} | @{y})} & \nonumber\\
	\red
	& (\outputp{x}{y} | @{y})\substn{\quotep{(\prefix{x}{y}{(@{y} | \outputp{x}{y})) | P}}}{y} & \nonumber\\
	=
	& \outputp{x}{\quotep{(\prefix{x}{y}{(\outputp{x}{y} | @{y})) | P}}}
	  | {(\prefix{x}{y}{(\outputp{x}{y} | @{y})) | P}} & \nonumber\\
	\red
	& \ldots & \nonumber\\
	\red^*
	& P | P | \ldots & \nonumber
\end{eqnarray}

Of course, this encoding, as an implementation, runs away, unfolding
$\bangp{P}$ eagerly. A lazier and more implementable replication
operator, restricted to input-guarded processes, may be obtained as follows.

\begin{eqnarray}
\bangp{\prefix{u}{v}{P}} 
	:= 
	\binpar{\lift{x}{\prefix{u}{v}{(\binpar{D(x)}{P})}}}{D(x)} \nonumber
\end{eqnarray}

\begin{remark}
  Note that the lazier definition still does not deal with summation
  or mixed summation (i.e. sums over input and output). The reader is
  invited to construct definitions of replication that deal with these
  features. 

  Further, the definitions are parameterized in a name, $x$. Can you,
  gentle reader, make a definition that eliminates this parameter and
  guarantees no accidental interaction between the replication
  machinery and the process being replicated -- i.e. no accidental
  sharing of names used by the process to get its work done and the
  name(s) used by the replication to effect copying. This latter
  revision of the definition of replication is crucial to obtaining
  the expected identity $!!P \sim !P$.
\end{remark}

\begin{remark}\label{rem:paradoxical_combinator}
  The reader familiar with the lambda calculus will have noticed the
  similarity between $D$ and the paradoxical combinator.

  [Ed. note: the existence of this seems to suggest we have to be more
  restrictive on the set of processes and names we admit if we are to
  support no-cloning.]
\end{remark}

\subsubsection{Bisimulation}

The computational dynamics gives rise to another kind of equivalence,
the equivalence of computational behavior. As previously mentioned
this is typically captured \emph{via} some form of bisimulation.

% The notion we use in this paper is weak barbed bisimulation
% \cite{milner91polyadicpi}.

The notion we use in this paper is derived from weak barbed
bisimulation \cite{milner91polyadicpi}. 

\begin{definition}
An \emph{observation relation}, $\downarrow_{\mathcal N}$, over a set
of names, $\mathcal N$, is the smallest relation satisfying the rules
below.

\infrule[Out-barb]{y \in {\mathcal N}, \; x \nameeq y}
		  {\outputp{x}{v} \downarrow_{\mathcal N} x}
\infrule[Par-barb]{\mbox{$P\downarrow_{\mathcal N} x$ or $Q\downarrow_{\mathcal N} x$}}
		  {\binpar{P}{Q} \downarrow_{\mathcal N} x}

We write $P \Downarrow_{\mathcal N} x$ if there is $Q$ such that 
$P \wred Q$ and $Q \downarrow_{\mathcal N} x$.
\end{definition}

\begin{definition}
%\label{def.bbisim}
An  ${\mathcal N}$-\emph{barbed bisimulation} over a set of names, ${\mathcal N}$, is a symmetric binary relation 
${\mathcal S}_{\mathcal N}$ between agents such that $P\rel{S}_{\mathcal N}Q$ implies:
\begin{enumerate}
\item If $P \red P'$ then $Q \wred Q'$ and $P'\rel{S}_{\mathcal N} Q'$.
\item If $P\downarrow_{\mathcal N} x$, then $Q\Downarrow_{\mathcal N} x$.
\end{enumerate}
$P$ is ${\mathcal N}$-barbed bisimilar to $Q$, written
$P \wbbisim_{\mathcal N} Q$, if $P \rel{S}_{\mathcal N} Q$ for some ${\mathcal N}$-barbed bisimulation ${\mathcal S}_{\mathcal N}$.
\end{definition}

$\mathcal{R} \subseteq \pi \times \pi$

$P \mathcal{R} Q => \forall P'. P \red P' \Rightarrow \exists Q'. Q \red Q', P' \mathcal{R} Q'$

$P \vdash x \Rightarrow Q \vdash x$

\begin{mathpar}
  \inferrule*[lab=Out-barb]{x \nameeq y}{{y}!\langle{Q}\rangle \vdash x}
  \and
  \inferrule*[lab=Par-barb]{\mbox{$P\vdash x$ or $Q\vdash x$}}{\binpar{P}{Q} \vdash x}
\end{mathpar}

\subsubsection{Contexts}

One of the principle advantages of computational calculi like the
$\pi$-calculus is a well-defined notion of context,
contextual-equivalence and a correlation between
contextual-equivalence and notions of bisimulation. The notion of
context allows the decomposition of a process into (sub-)process and
its syntactic environment, its context. Thus, a context may be
thought of as a process with a ``hole'' (written $\Box$) in it. The
application of a context $M$ to a process $P$, written $M[P]$, is
tantamount to filling the hole in $M$ with $P$. In this paper we do
not need the full weight of this theory, but do make use of the notion
of context in the proof the main theorem. 

\begin{mathpar}
  \inferrule* [lab=summation] {} {{M_{M},M_{N}} \bc \Box \;|\; x.M_{A} \;|\; M_{M}+M_{N}}
  \and
  \inferrule* [lab=agent] {} {{M_{A}} \bc (\vec{x})M_{P} \;| \; \clift{P_0,\ldots,M_{P},\ldots,P_N}}
  \and \\
  \inferrule* [lab=process] {} {{M_{P}} \bc M_{N} \;| \;P|M_{P} }
\end{mathpar} 

\begin{mathpar}
  \inferrule* [lab=sychronization] {} {M_{N} \bc \Box \;|\; x?M_{F} \;|\; x!M_{C}}
  \and
  \inferrule* [lab=abstraction] {} {{M_{F}} \bc (x)M_{P} }
  \and
  \inferrule* [lab=concretion] {} {{M_{C}} \bc \langle M_{P} \rangle }
  \and \\
  \inferrule* [lab=process] {} {{M_{P}} \bc M_{N} \;| \;P|M_{P} }
\end{mathpar}

\begin{definition}[contextual application] Given a context $M$, and
  process $P$, we define the \emph{contextual application}, $M[P] :=
  M\{P/\Box\}$. That is, the contextual application of M to P is the
  substitution of $P$ for $\Box$ in $M$.
\end{definition}

$\meaningof{-} : L \to \mathcal{P}(\pi)$

\begin{mathpar}
  \inferrule* [lab=collection] {} {\meaningof{true} = \pi, \and \meaningof{~E} = \pi \setminus \meaningof{E}, \and \meaningof{E_{1} \& E_{2}} = \meaningof{E_{1}} \cap \meaningof{E_{2}}}
\end{mathpar}

\begin{mathpar}
  \inferrule* [lab=structure] {} {\meaningof{0} = \{ P \in \pi | P \equiv 0 \}, \and \\ \meaningof{E_1 | E_2} = \{ P \in \pi | P \equiv P_{1} | P_{2}, P_{1} \in \meaningof{E_{1}}, P_{2} \in \meaningof{E_2}\} }
\end{mathpar}

\begin{mathpar}
 \inferrule* [lab=behavior] {} {\meaningof{\langle a?b \rangle E} = \{ P \in \pi | P \equiv Q | u?(y)P', \\ \and \\\\ \and \\ \;\;\; u \in \meaningof{a}, \forall z.P'\{z/y\} \in \meaningof{E\{z/b\}}\}, \and \\ \meaningof{a!E} = \{ P \in \pi | P \equiv Q | x!\langle P' \rangle, x \in \meaningof{a} P' \in \meaningof{E}\} }
\end{mathpar}

\begin{mathpar}
 \inferrule* [lab=nominal] {} {\meaningof{\quotep{E}} = \{ \quotep{P} \in \quotep{\pi} | P \in \meaningof{E} \}, \and \meaningof{\quotep{P}} = \{ \quotep{Q} \in \quotep{\pi} | P \equiv Q \} \and \\ \meaningof{@\quotep{E}} = \{ P \in \pi | P \equiv @x, x \in \meaningof{E} \}}
\end{mathpar}

\begin{eqnarray*}
  \\
  \meaningof{-} : TS \to ST
\end{eqnarray*}

\begin{eqnarray*}
  \\
  L : TS \to ST
\end{eqnarray*}

\begin{eqnarray*}
  \\
  P \models E \iff P \in \meaningof{E}
\end{eqnarray*}

\begin{eqnarray*}
  P \approx_{L} Q \iff \forall E \in L. P \models E \iff Q \models E
\end{eqnarray*}

\begin{eqnarray*}
  P \approx_{K} Q
\end{eqnarray*}

\begin{eqnarray*}
  P \approx Q
\end{eqnarray*}

$\approx_{K} = \approx = \approx_{L}$

\subsubsection{Contextual duality}

Note that contexts extend the quotation operation to a family of
operations from processes to names. Given a context, $M$, we can
define a \emph{nominal context}, $\quotep{M}$ by $\quotep{M}[P] :=
\quotep{M[P]}$. To foreshadow what is to come we observe that these
operations enjoy a duality with processes very much like the duality
between vectors and maps from vectors to scalars.

Further, because the calculus is essentially higher-order, we have a
correspondence between contexts and processes. More specifically,
given a name $x$ and a context $M$ we can construct $M^{*}_{x}$ such
that 

\begin{mathpar}
  M^{*}_{x} | \lift{x}{P} \red M[P]
\end{mathpar}

namely,

\begin{mathpar}
  M^{*}_{x} := x?(u).M[\dropn{u}]
\end{mathpar}

The dependence of $M^{*}_{x}$ on a name makes it an abstraction, 

\begin{mathpar}
  M^{*} := (x)x?(u).M[\dropn{u}]
\end{mathpar}

\subsection{Additional notation}

It will sometimes be convenient to denote the process a name
quotes. We already have the notation $x = \quotep{P}$, but it will be
convenient to introduce an alternate notation, $\procn{x}$, when we
want to emphasize the connection to the use of the name. Note that, by
virtue of name equivalence, $\quotep{\procn{x}} \nameeq x$; so, the
notation is consistent with previous definitions.

Further, because names have structure it is possible to effect
substitutions on the basis of that structure. This means we need to
upgrade our notation for substitutions, which we accomplish by
adapting comprehension notation. Thus,

\begin{mathpar}
  P\{ y / x : x \in S \}
\end{mathpar}

is interpreted to mean the process derived from P by replacing (in a
capture-avoiding manner) each occurrence of $x$ in $S$ by $y$. For example,

\begin{mathpar}
  P\{ \quotep{\procn{x}|\procn{x}} / x : x \in \freenames{P} \}
\end{mathpar}

will replace each (occurrence) of a free name $x$ in $P$ by
$\quotep{\procn{x}|\procn{x}}$.

Also, we will avail ourselves of the notation $x^{L}$ and $x^{R}$ to
denote injections of a name into disjoint copies of the name
space. There are numerous ways to accomplish this. One example can be
found in \cite{MeredithR05}. This notation overloads to vectors of
names: $\vec{x}^{\pi} := (x_{i}^{\pi} \; : \; 0 \leq i < |\vec{x}| )$ where $\pi \in \{L,R\}$.

We also use $P^{\Box} := P|\Box$.

In \cite{MeredithR05} an interpretation of the new operator is
given. It turns out that there are several possible interpretations
all enjoying the requisite algebraic properties of the operator (see
\cite{milner91polyadicpi}). We will therefore make liberal use of
$(\nu\; \vec{x})P$.

% subsection the_syntax_and_semantics_of_the_notation_system (end)   

\input{qm2pi.qmops} 

\input{qm2pi.sterngerlach} 

\input{qm2pi.metric} 

% section concurrent_process_calculi (end)

%\input{qm2pi.proofsketch}

% section proof sketch (end)

%\input{qm2pi.slviaknots} 

% section spatial logic via knots (end)

\input{qm2pi.conclusion}

% section conclusion (end)

%\input{qm2pi.dtcodes} 

% section wiring algorithm (end)

\input{qm2pi.ack} 

% section acknowledgments (end)

\newpage


\bibliographystyle{plain}   
\bibliography{../../biblios/main.bib}

\input{qm2pi.rhodetails}

\end{document}



% section proof sketch (end)

%\section{Unlikely characters: spatial logic for
  knots}\label{sub:characteristic_formulae} % (fold)

Associated to the mobile process calculi are a family of logics known
as the Hennessy-Milner logics. These logics typically enjoy a
semantics interpreting formulae as sets of processes that when
factored through the encoding outlined above allows an identification
of classes of knots with logical formulae. In the context of this
encoding the sub-family known as the spatial logics \cite{CairesC03}
\cite{CairesC04} \cite{Caires04} are of particular interest providing
several important features for expressing and reasoning about
properties (i.e. classes) of knots. We hint here at how this may be done.

%\begin{description}
%\item [structural connectives] 
\subsubsection{Structural connectives} The spatial logics enjoy
structural connectives corresponding, at the logical level, to the
parallel composition ($P | Q$) and new name ($(\nu \; x)P$)
connectives for processes. As illustrated in the examples below, these
connectives are extremely expressive given the shape of our encoding.
%\item [decideable satisfaction]

\subsubsection{Decideable satisfaction}
In \cite{Caires04} the satisfaction relation is shown to be decideable
for a rich class of processes. It further turns out that the image of
the our encoding is a proper subset of that class. This result
provides the basis for an algorithm by which to search for knots
enjoying a given property.
%\item [characteristic formulae]

\subsubsection{Characteristic formulae}
In the same paper \cite{Caires04} , Caires presents a means of calculating
characteristic formulae, selecting equivalence classes of processes
up to a pre--specified depth limit on the support set of names. Composed with our
encoding, this characteristic formula can be used to select
characteristic formulae for knots.
%\end{description}

\subsubsection{Spatial logic formulae}

The grammar below (segmented for comprehension) summarizes the syntax
of spatial logic formulae. We employ illustrative examples in the
sequel to provide an intuitive understanding of their meaning
referring the reader to \cite{Caires04} for a more detailed explication
of the semantics.

\begin{mathpar}
  \inferrule* [lab=boolean] {} {{A,B} \bc T \;|\; \neg A \;|\; A \wedge B \;|\; \eta = \eta'}
  \and
  \inferrule* [lab=spatial] {} {|\; \pzero \;|\; A | B \;|\; x \text{\textregistered} A \;|\; \forall x . A \;|\;  H x . A}
  \and
  \inferrule* [lab=behavioral] {} {|\; \alpha . A}
  \and 
  \inferrule* [lab=recursion] {} {|\; X(\vec{u}) \;|\; \mu X(\vec{u}) . A}
  \and
  \inferrule* [lab=action] {} {\alpha \bc \langle x?(\vec{y}) \rangle \;|\; \langle x!(\vec{y}) \rangle \;|\; \langle \tau \rangle}
  \and 
  \inferrule* [lab=name] {} {\eta \bc x \;|\; \tau}
\end{mathpar} 

% subsection characteristic_formulae (end)   	 

\subsection{Example formulae}\label{sub:example_formulae_} % (fold)

\subsubsection{Crossing as formula.}
% 
% \begin{align*}
%   \frac{d}{dx} \sin x &= \cos x 
%   & \frac{d}{dx} e^x &= e^x \\
%   \frac{d}{dx} \cos x &= - \sin x 
%   & \frac{d}{dx} \log x &= \frac{1}{x} \\
% \end{align*} 

\begin{align*}
 \mu C(x_{0},x_{1},y_{0},y_{1},u).&(\langle x_{0}?(z) \rangle(\langle u! \rangle\langle y_{1}!z \rangle C(x_{0},x_{1},y_{0},y_{1},u)) & \\
  & \wedge \langle y_{1}?(z) \rangle (\langle u! \rangle \langle x_{0}!z \rangle C(x_{0},x_{1},y_{0},y_{1},u)) & \\
  & \wedge \langle x_{1}?(z) \rangle (\langle u? \rangle \langle y_{0}!z \rangle C(x_{0},x_{1},y_{0},y_{1},u)) & \\
  & \wedge \langle y_{0}?(z) \rangle (\langle u? \rangle \langle x_{1}!z \rangle C(x_{0},x_{1},y_{0},y_{1},u))) &
\end{align*}

The lexicographical similarity between the shape of this formulae and
the shape of definition of the process representing a crossing reveals
the intuitive meaning of this formulae. It describes the capabilities
of a process that has the right to represent a crossing. For example
it picks out processes that may perform an input on the port $x_0$ in
its initial menu of capabilities. What differentiates the formula
from the process, however, is that the crossing process is the
smallest candidate to satisfy the formula. Infinitely many other
processes -- with internal behavior hidden behind this interface, so
to speak -- also satisfy this formula. Even this simple formula,
then, can be seen to open a new view onto knots, providing a
computational interpretation of \emph{virtual} knots.

Note that this formula is derived by hand. A similar formula can be
derived by employing Caires' calculation of characteristic formula
\cite{Caires04} to the process representing a crossing. In light of
this discussion, we let
$\meaningof{C}_{\phi}(x0,x1,y0,y1,u)$ denote a formula specifying the
dynamics we wish to capture of a crossing. To guarantee we preserve
the shape of the interface and minimal semantics we demand that
$\meaningof{C}_{\phi}(x0,x1,y0,y1,u) \Rightarrow
\textbf{C}(x0,x1,y0,y1,u)$ where $\textbf{C}(x0,x1,y0,y1,u)$ denotes
the formula above.
                            
\subsubsection{Crossing number constraints.}
The moral content of the context lemma (Lemma \ref{context}) is that the notion of
``locality'' in the Reidemeister moves is effectively captured by the
parallel composition operator of the process calculus. This intuition
extends through the logic. Given a formula,
$\meaningof{C}_{\phi}(x0,x1,y0,y1,u)$, we can use the structural
connectives to specify constraints on crossing numbers, such as at
least $n$ crossings, or exactly $n$ crossings.
\begin{mathpar}
  \inferrule* [lab=at-least-n] {} { K^{\geq n}_{\phi}(\vec{xs},\vec{ys}) := \Pi_{i=0}^{n-1} Hu . \meaningof{C}_{\phi}(xs_i,ys_i,u) | T }
  \and 
  \inferrule* [lab=exactly-n] {} { K^{= n}_{\phi}(\vec{xs},\vec{ys}) := \Pi_{i=0}^{n-1} Hu . \meaningof{C}_{\phi}(xs_i,ys_i,u) | \neg (\forall x_0,y_0,x_1,y_1,u . \meaningof{C}_{\phi}(x_0,y_0,x_1,y_1,u) | T) }
\end{mathpar}

To round out this section, recall that the encoding of an $n$-crossing
knot decomposes into a parallel composition of $n$ \emph{copies} of a
crossing process together with a wiring harness. To specify different
knot classes with the same crossing number amounts to specifying
logical constraints on the wiring harness. In the interest of space,
we defer examples to a forthcoming paper. Suffice it to say that both
the conditions ``alternating knot'' and ``contains the tangle
corresponding to 5/3'' are expressible. For example, it is possible to
calculate the characteristic formula of a process corresponding to the
tangle 5/3 and conjoin it into the classifying formula via the
composition connective of the logic.

Finally, we wish to observe that it is entirely within reason to
contemplate a more domain-specific version of spatial logic tailored
to the shape of processes in the image of the encoding. Such a
domain-specific logic would have a better claim to the title formal
language of knot properties.

% subsection example_formulae_ (end)

% section knots_as_processes (end) 

% section spatial logic via knots (end)

\section{Conclusions and future work}

\paragraph{Testing physical space}
You, gentle reader, may wonder why of all the theorems to be proved
given this set up we pick the one above. In some sense it's hardly
central to quantum mechanics. We see it as central in the sense that
it firmly establishes a notion of physical space arising from a notion
of the equivalence of behavior. Relating bisimulation to a metric is a
big step forward, but one is faced with interpreting the relationship
of that metric space to something more physical. Quantum mechanical
notions of ``physical'' space are still far from intuitive, but by
relating this idea of distance as testing to calculations that predict
physical circumstances we are making a not insignificant step forward
toward an understanding of the physical space we inhabit as
essentially dynamic.

\paragraph{Effectivity and simulation}
One of the observations we have yet to make is that the entire program
spelled out here is effective. We have built various interpreters for
the reflective calculus at work in this interpretation. In principle,
then, we can simulate quantum mechanics on a computer. The place where
the simulation may lose fidelity is the infinitely branching summation
for the annihilator.

In this connection i also want to point out that the evaluation style
calculation of the inner product puts the non-determinism of the
summation right at the heart of measurement. This suggests that
Milner's original reduction-based formulation of the dynamics of his
calculi in terms of sums was not just notationally suggestive of a
notion of measure-and-continue but captured some significant part of
the physics.

\paragraph{Quantum continuations}
In light of this last observation i want to point out that the
predominant account of quantum mechanics is missing a key aspect of a
truly compositional story of the physical situation. In a real lab,
when a measurement is made the observation can be made to feed into
another device that then makes another measurement conditioned on the
results of the first. This means that after the superposition was
collapsed the entire experimental set up remained in
superposition. While QM offers a means of writing this down it doesn't
quite line up well with the well-trodden formulation of computation
and continuation that we see so succinctly expressed in Milner's
calculi. This suggests that there might be advantages to this account
of dynamics waiting to be explored.

\paragraph{Quantum logic}
In this connection, we also note that by virtue of having the
Hennessy-Milner construction, we can pull the construction through the
interpretation of QM. This gives us a natural candidate for a quantum
logic that enjoys an extremely tight connection with it's domain of
interpretation, making the construction much less ad hoc (rather it is
the image of functor!).

\paragraph{Quantum probabiity}
i have questions about the basis of the interpretation of inner
product as probability amplitude. In particular, using which
axiomatization of probability theory does the notion of probability
amplitude earn the right to be so dubbed? In other words, where is the
proof that the operation for calculating a probability amplitude (and
then squaring) satisfies the axioms of what it means to calculate a
probability? Even if such a proof exists (i have yet to find it in the
literature), i wonder if it might not be possible to turn things on
their heads. Can we view the calculation of the probability amplitude
as an axiomatization of probability? If so, then the definition we
give for calculating probability amplitude may provide the basis for
an \emph{effective} theory of probability.

\paragraph{Quantum vs ``biological'' information}
Finally, i want to conclude with a more philosophical observation. At
a recent workshop in which QM was a predominant topic i noticed
something about quantum information. The speaker was giving a riveting
discussion of axiomatic QM and showing how properties of ``no
cloning'' and ``no deleting'' emerged as consequences of the
axiomatization. Theorems of this form are necessary to give us a sense
of confidence that our axioms characterize the physical theory. What
struck me, though, was that if quantum information is neither erasable
nor replicable it is markedly different from \emph{life}. Two of the
things we know about life is that

\begin{itemize}
  \item it ends;
  \item to gain some measure of persistence, to transcend it's
    finitude it is imminently copyable.
\end{itemize}

Both of these qualities are summarized succinctly in the aphorism: all
flesh is grass. For me these two kinds of ``information'' -- call them
quantum and biological -- are end points on a spectrum of strategies
for persistence. At one end, we have those curious entities that enjoy
uniqueness and permanence; at the other, we have those who in the face
of a certain end and an uncertain present make a go of passing
something on. To me one of the more remarkable aspects of the latter
strategy is that in the presence of noise (and certain features of
copying) we get a kind of dynamism, a chance for improvement against a
given persistent condition.

% subsection other_calculi_other_bisimulations_and_geometry_as_behavior (end)




% section conclusion (end)

%\documentclass[12pt]{llncs}
%\documentclass{jktr}

\usepackage[pdftex]{hyperref}                   
\usepackage {listings}
\usepackage {mathpartir}
\usepackage{bcprules}
%\usepackage{listings}
                       
\usepackage{graphicx} 
%\usepackage[margins=2.5cm,nohead,nofoot]{geometry}
%\usepackage{geometry}
\usepackage{amsfonts}
\usepackage{amstext}
\usepackage{latexsym}
\usepackage{amssymb}
\usepackage{color}


%\include{myPreamble}
\include{qm2pi.local} 

%\ifpdf
%\usepackage[pdftex]{graphicx}
%\else
%\usepackage{graphicx}
%\fi

 % \ifpdf
%  \usepackage{pdfsync}
%  \if


%\title{Brief Article}
%\author{David F. Snyder}
%\author{L.G. Meredith}

%\address{Dept. of Math., Texas State University--San Marcos, San Marcos, TX 78666}
       
\pagestyle{empty}


\begin{document}

\lstset{language=[Objective]Caml,frame=shadowbox}

\input{qm2pi.front}

% section front matter (end)

\input{qm2pi.intro} 
 
% section introduction (end)

% \input{qm2pi.knotations} 

% section notation (end)

\input{qm2pi.process.calculi} 

% section concurrent_process_calculi_and_spatial_logics_ (end)
    
%\input{qm2pi.knots2pi} 

%\input{qm2pi.trefoil} 

%\input{qm2pi.mainthm} 

% subsection basic_interpretation (end)

%\input{qm2pi.rho.presentation} 
\subsection{The syntax and semantics of the notation system}\label{sub:the_syntax_and_semantics_of_the_notation_system} % (fold)

We now summarize a technical presentation of the calculus that
embodies our theory of dynamics. The typical presentation of such a
calculus follows the style of giving generators and relations on
them. The grammar, below, describing term constructors, freely
generates the set of processes, $\Proc$. This set is then quotiented
by a relation known as structural congruence and it is over this set
that the notion of dynamics is expressed. This presentation is
essentially that of \cite{MeredithR05} with the addition of
polyadicity and summation. For readability we have relegated some of
the technical subtleties to an appendix.

\subsubsection{Process grammar}\label{subsub:process_grammar}

\begin{mathpar}
  \inferrule* [lab=synchronization] {} {{M} \bc \pzero \;|\; x?F \;|\; x!C }
  \and
  \inferrule* [lab=abstraction] {} {{F} \bc (x)P}
  \and
  \inferrule* [lab=concretion] {} {{C} \bc \langle Q \rangle}
  \and
  \inferrule* [lab=process] {} {{P,Q} \bc M \;| \;P|Q \;|\; @{x}}
  \and
  \inferrule* [lab=name] {} {{x} \bc \quotep{P}}
\end{mathpar} 

Note that $\vec{x}$ (resp. $\vec{P}$) denotes a vector of names
(resp. processes) of length $|\vec{x}|$ (resp. $|\vec{P}|$). We adopt
the following useful abbreviations.

\begin{mathpar}
   x?(\vec{y}).P := x.(\vec{y})P \and  x\clift{\vec{P}} := x.\clift{\vec{P}}
   \and x!(y) := \lift{x}{\dropn{y}}
   \and \Pi_{i=0}^{n-1}P_i := P_0 | \ldots | P_{n-1}
\end{mathpar}

\subsubsection{Structural congruence}

\paragraph{Free and bound names and alpha-equivalence.} At the
core of structural equivalence is alpha-equivalence which identifies
process that are the same up to a change of variable. Formally, we
recognize the distinction between free and bound names. The free names
of a process, $\freenames{P}$, may be calculated recursively as
follows:

\begin{mathpar}
\freenames{\pzero} := \emptyset
  \and \\
  \freenames{x?(y).P} := \{ x \} \cup (\freenames{P} \setminus \{ y \})
  \and 
  \freenames{x!\langle P \rangle} := \{ x \} \cup \{ P \} 
  \and \\
  \freenames{P|Q} := \freenames{P} \cup \freenames{Q}
  \and \\
  \freenames{@{x}} := \{ x \}
\end{mathpar}

$\pi$
$\quotep{\pi}$

$\freenames{-} : \pi \to \mathcal{P}(\quotep{\pi})$

\begin{eqnarray*}
  \freenames{\pzero} & := & \emptyset \\
  \freenames{x?(y).P} & := & \{ x \} \cup (\freenames{P} \setminus \{ y \}) \\
  \freenames{x!\langle P \rangle} & := & \{ x \} \cup \{ P \} \\
  \freenames{P|Q} & := & \freenames{P} \cup \freenames{Q} \\
  \freenames{\dropn{x}} & := & \{ x \}
\end{eqnarray*}

The bound names of a process, $\boundnames{P}$, are those names occurring in $P$
that are not free. For example, in $x?(y).0$, the name $x$ is free, while $y$ is bound.

\begin{mathpar}
  \inferrule* [lab=monoidal-laws] {} { P|Q \equiv Q|P \and P|0 \equiv P \and P|(Q|R) \equiv (P|Q)|R }
\end{mathpar}

\begin{mathpar}
  \inferrule* [lab=alpha-equivalence] {} { (x)P \equiv (y)P\{y/x\} \and y \not\in \freenames{P} }
\end{mathpar}

\begin{definition}
Then two processes, $P,Q$, are alpha-equivalent if $P = Q\{\vec{y}/\vec{x}\}$ for
some $\vec{x} \in \boundnames{Q},\vec{y} \in \boundnames{P}$, where $Q\{\vec{y}/\vec{x}\}$
denotes the capture-avoiding substitution of $\vec{y}$ for $\vec{x}$ in $Q$.
\end{definition}

\begin{definition}
  The {\em structural congruence} \cite{SangiorgiWalker} , $\equiv$,
  between processes is the least congruence containing
  alpha-equivalence, satisfying the abelian monoid laws
  (associativity, commutativity and $\pzero$ as identity) for parallel
  composition $|$ and for summation $+$.
\end{definition}

\subsection{Name equivalence}

We take name equivalence, written $\nameeq$, to be the smallest
equivalence relation generated by the following rules.

\begin{mathpar}
\inferrule*[lab=Quote-drop]
{ }
{ \quotep{@{x}} \nameeq x }

\inferrule*[lab=Struct-equiv]
{ P \scong Q }
{ \quotep{P} \nameeq \quotep{Q} }
\end{mathpar}

The astute reader will have noticed that the mutual recursion of names
and processes imposes a mutual recursion on alpha-equivalence and
structural equivalence via name-equivalence. Fortunately, all of this
works out pleasantly and we may calculate in the natural way, free of
concern. The reader interested in the details is referred to the
appendix \ref{appendix:rho_details}.

\subsection{Substitution}

We use $\Proc$ for the set of processes, $\QProc$ for the set of
names, and $\id{\{}\vec{y} / \vec{x} \id{\}}$ to denote partial maps,
$s : \QProc \rightarrow \QProc$. A map, $s$ lifts, uniquely, to a map
on process terms, $\widehat{s} : \Proc \rightarrow \Proc$ by the
following equations.

\begin{mathpar}
  (0) \psubstp{Q}{P} := 0 \\
  (R \juxtap S) \psubstp{Q}{P}
  :=    
  (R)\psubstp{Q}{P} \juxtap (S) \psubstp{Q}{P} \\
  (x?(y).R) \psubstp{Q}{P}    
  :=    
  (x)\substp{Q}{P} (z)\concat( (R \psubstn{z}{y}) \psubstp{Q}{P} ) \\
  (\lift{x}{R}) \psubstp{Q}{P}  
  :=
  \lift{(x)\substp{Q}{P}}{ R \psubstp{Q}{P} } \\
%   (\dropn{x})  \psubstp{Q}{P}       
%   := 
%   \left\{ 
%     \begin{array}{ccc} 
%       \dropn{\quotep{Q}} & & x \nameeq \quotep{P} \\
%       \dropn{x} & & otherwise \\
%     \end{array}
%   \right. 
  (\dropn{x})  \psubstp{Q}{P}       
  := 
  \left\{ 
    \begin{array}{ccc} 
      Q & & x \nameeq \quotep{P} \\
      \dropn{x} & & otherwise \\
    \end{array}
  \right.
\end{mathpar}
 

where

\begin{eqnarray}
  (x)\id{\{} \lpquote Q \rpquote / \lpquote P \rpquote \id{\}}            = 
  \left\{ 
    \begin{array}{ccc}
      \lpquote Q \rpquote & & x \nameeq \lpquote P \rpquote \\
      x & & otherwise \\
    \end{array}
  \right. \nonumber
\end{eqnarray}

and $z$ is chosen distinct from $\quotep{P}$, $\quotep{Q}$, the free
names in $Q$, and all the names in $R$. Our $\alpha$-equivalence will
be built in the standard way from this substitution.

\begin{remark}\label{rem:no_self_referential_names}
  One consequence of these definitions is that $\forall P. \quotep{P}
  \not\in \freenames{P}$.
\end{remark}

\subsection{ Dynamic quote: an example }

Anticipating something of what's to come, consider applying the
substitution, $\widehat{\id{\{}u / z \id{\}}}$, to the following pair
of processes, $\lift{w}{y!(z)}$ and $w[ \lpquote y!(z) \rpquote ]$.

\begin{eqnarray}
	\lift{w}{y!(z)}\widehat{\id{\{}u / z \id{\}}}
		& = &
		\lift{w}{y!(u)} \nonumber\\
	w[ \lpquote y!(z) \rpquote ] \widehat{ \id{\{}u / z \id{\}} }
		& = &
		w[ \lpquote y!(z) \rpquote ] \nonumber
\end{eqnarray}

Because the body of the process between quotes is impervious to
substitution, we get radically different answers. In fact, by
examining the first process in an input context,
e.g. $x?(z).\lift{w}{y!(z)}$, we see that the process under the lift
operator may be shaped by prefixed inputs binding a name inside it. In
this sense, the lift operator will be seen as a way to dynamically
construct processes before reifying them as names.

Finally equipped with these standard features we can present the
dynamics of the calculus.

\subsubsection{Operational semantics} 

Finally, we introduce the computational dynamics. What marks these
algebras as distinct from other more traditionally studied algebraic
structures, e.g. vector spaces or polynomial rings, is the manner in
which dynamics is captured. In traditional structures, dynamics is typically
expressed through morphisms between such structures, as in linear maps
between vector spaces or morphisms between rings. In algebras
associated with the semantics of computation, the dynamics is
expressed as part of the algebraic structure itself, through a
reduction reduction relation typically denoted by $\red$. Below, we
give a recursive presentation of this relation for the calculus used
in the encoding.

$\red \subseteq \pi \times \pi$
$\red : \pi \to \mathcal{P}(\pi)$

\begin{mathpar}
  \inferrule* [lab=Comm] { \textsf{match}( x_{src}, x_{trgt} ) } { x_{trgt}?(y)P \; | \; x_{src}!\langle {Q} \rangle \red P\{\quotep{Q}/y}\} }
  \and \\
  \inferrule* [lab=Par] {{P} \red {P}'} {{{P} | {Q}} \red {{P}' | {Q}}}
  \and
  \inferrule* [lab=Equiv]{{{P} \scong {P}'} \andalso {{P}' \red {Q}'} \andalso {{Q}' \scong {Q}}}{{P} \red {Q}}
\end{mathpar}

\begin{eqnarray*}
  match_{\equiv} (\quotep{P},\quotep{Q}) & := & P \equiv Q \\
  match_{\dagger}(\quotep{P},\quotep{Q}) & := & \forall R. P|Q \red^{*} R => R \red^{*} 0 \\
  match_{K}(\quotep{P},\quotep{Q}) & := & K \mbox{ for some context } K
\end{eqnarray*}

$u?(x)P | u!\langle Q \rangle \red P\{\quotep{Q}/x\}$

%We write $\wred$ for $\red^*$, and $P\red$ if $\exists Q $ such that $ P \red Q$.
We write $P\red$ if $\exists Q $ such that $ P \red Q$ and $P\not\red$, otherwise.

\section{Replication}

As mentioned before, it is known that replication (and hence
recursion) can be implemented in a higher-order process algebra
\cite{SangiorgiWalker}. As our first example of calculation with the
machinery thus far presented we give the construction explicitly in
the {\rhoc}.

\begin{eqnarray}
	D_{x} & := & \prefix{x}{y}{(\binpar{\outputp{x}{y}}{@{y}})} \nonumber\\
	\bangp_{x}{P} & := & \binpar{{x}!\langle{\binpar{D_{x}}{P}}\rangle}{D_{x}} \nonumber
\end{eqnarray}

\begin{eqnarray}
	\bangp_{x}{P} & & \nonumber\\
	=
	& {x}!\langle{(\prefix{x}{y}{(\outputp{x}{y} | @{y})) | P}}\rangle 
	      | \prefix{x}{y}{(\outputp{x}{y} | @{y})} & \nonumber\\
	\red
	& (\outputp{x}{y} | @{y})\substn{\quotep{(\prefix{x}{y}{(@{y} | \outputp{x}{y})) | P}}}{y} & \nonumber\\
	=
	& \outputp{x}{\quotep{(\prefix{x}{y}{(\outputp{x}{y} | @{y})) | P}}}
	  | {(\prefix{x}{y}{(\outputp{x}{y} | @{y})) | P}} & \nonumber\\
	\red
	& \ldots & \nonumber\\
	\red^*
	& P | P | \ldots & \nonumber
\end{eqnarray}

Of course, this encoding, as an implementation, runs away, unfolding
$\bangp{P}$ eagerly. A lazier and more implementable replication
operator, restricted to input-guarded processes, may be obtained as follows.

\begin{eqnarray}
\bangp{\prefix{u}{v}{P}} 
	:= 
	\binpar{\lift{x}{\prefix{u}{v}{(\binpar{D(x)}{P})}}}{D(x)} \nonumber
\end{eqnarray}

\begin{remark}
  Note that the lazier definition still does not deal with summation
  or mixed summation (i.e. sums over input and output). The reader is
  invited to construct definitions of replication that deal with these
  features. 

  Further, the definitions are parameterized in a name, $x$. Can you,
  gentle reader, make a definition that eliminates this parameter and
  guarantees no accidental interaction between the replication
  machinery and the process being replicated -- i.e. no accidental
  sharing of names used by the process to get its work done and the
  name(s) used by the replication to effect copying. This latter
  revision of the definition of replication is crucial to obtaining
  the expected identity $!!P \sim !P$.
\end{remark}

\begin{remark}\label{rem:paradoxical_combinator}
  The reader familiar with the lambda calculus will have noticed the
  similarity between $D$ and the paradoxical combinator.

  [Ed. note: the existence of this seems to suggest we have to be more
  restrictive on the set of processes and names we admit if we are to
  support no-cloning.]
\end{remark}

\subsubsection{Bisimulation}

The computational dynamics gives rise to another kind of equivalence,
the equivalence of computational behavior. As previously mentioned
this is typically captured \emph{via} some form of bisimulation.

% The notion we use in this paper is weak barbed bisimulation
% \cite{milner91polyadicpi}.

The notion we use in this paper is derived from weak barbed
bisimulation \cite{milner91polyadicpi}. 

\begin{definition}
An \emph{observation relation}, $\downarrow_{\mathcal N}$, over a set
of names, $\mathcal N$, is the smallest relation satisfying the rules
below.

\infrule[Out-barb]{y \in {\mathcal N}, \; x \nameeq y}
		  {\outputp{x}{v} \downarrow_{\mathcal N} x}
\infrule[Par-barb]{\mbox{$P\downarrow_{\mathcal N} x$ or $Q\downarrow_{\mathcal N} x$}}
		  {\binpar{P}{Q} \downarrow_{\mathcal N} x}

We write $P \Downarrow_{\mathcal N} x$ if there is $Q$ such that 
$P \wred Q$ and $Q \downarrow_{\mathcal N} x$.
\end{definition}

\begin{definition}
%\label{def.bbisim}
An  ${\mathcal N}$-\emph{barbed bisimulation} over a set of names, ${\mathcal N}$, is a symmetric binary relation 
${\mathcal S}_{\mathcal N}$ between agents such that $P\rel{S}_{\mathcal N}Q$ implies:
\begin{enumerate}
\item If $P \red P'$ then $Q \wred Q'$ and $P'\rel{S}_{\mathcal N} Q'$.
\item If $P\downarrow_{\mathcal N} x$, then $Q\Downarrow_{\mathcal N} x$.
\end{enumerate}
$P$ is ${\mathcal N}$-barbed bisimilar to $Q$, written
$P \wbbisim_{\mathcal N} Q$, if $P \rel{S}_{\mathcal N} Q$ for some ${\mathcal N}$-barbed bisimulation ${\mathcal S}_{\mathcal N}$.
\end{definition}

$\mathcal{R} \subseteq \pi \times \pi$

$P \mathcal{R} Q => \forall P'. P \red P' \Rightarrow \exists Q'. Q \red Q', P' \mathcal{R} Q'$

$P \vdash x \Rightarrow Q \vdash x$

\begin{mathpar}
  \inferrule*[lab=Out-barb]{x \nameeq y}{{y}!\langle{Q}\rangle \vdash x}
  \and
  \inferrule*[lab=Par-barb]{\mbox{$P\vdash x$ or $Q\vdash x$}}{\binpar{P}{Q} \vdash x}
\end{mathpar}

\subsubsection{Contexts}

One of the principle advantages of computational calculi like the
$\pi$-calculus is a well-defined notion of context,
contextual-equivalence and a correlation between
contextual-equivalence and notions of bisimulation. The notion of
context allows the decomposition of a process into (sub-)process and
its syntactic environment, its context. Thus, a context may be
thought of as a process with a ``hole'' (written $\Box$) in it. The
application of a context $M$ to a process $P$, written $M[P]$, is
tantamount to filling the hole in $M$ with $P$. In this paper we do
not need the full weight of this theory, but do make use of the notion
of context in the proof the main theorem. 

\begin{mathpar}
  \inferrule* [lab=summation] {} {{M_{M},M_{N}} \bc \Box \;|\; x.M_{A} \;|\; M_{M}+M_{N}}
  \and
  \inferrule* [lab=agent] {} {{M_{A}} \bc (\vec{x})M_{P} \;| \; \clift{P_0,\ldots,M_{P},\ldots,P_N}}
  \and \\
  \inferrule* [lab=process] {} {{M_{P}} \bc M_{N} \;| \;P|M_{P} }
\end{mathpar} 

\begin{mathpar}
  \inferrule* [lab=sychronization] {} {M_{N} \bc \Box \;|\; x?M_{F} \;|\; x!M_{C}}
  \and
  \inferrule* [lab=abstraction] {} {{M_{F}} \bc (x)M_{P} }
  \and
  \inferrule* [lab=concretion] {} {{M_{C}} \bc \langle M_{P} \rangle }
  \and \\
  \inferrule* [lab=process] {} {{M_{P}} \bc M_{N} \;| \;P|M_{P} }
\end{mathpar}

\begin{definition}[contextual application] Given a context $M$, and
  process $P$, we define the \emph{contextual application}, $M[P] :=
  M\{P/\Box\}$. That is, the contextual application of M to P is the
  substitution of $P$ for $\Box$ in $M$.
\end{definition}

$\meaningof{-} : L \to \mathcal{P}(\pi)$

\begin{mathpar}
  \inferrule* [lab=collection] {} {\meaningof{true} = \pi, \and \meaningof{~E} = \pi \setminus \meaningof{E}, \and \meaningof{E_{1} \& E_{2}} = \meaningof{E_{1}} \cap \meaningof{E_{2}}}
\end{mathpar}

\begin{mathpar}
  \inferrule* [lab=structure] {} {\meaningof{0} = \{ P \in \pi | P \equiv 0 \}, \and \\ \meaningof{E_1 | E_2} = \{ P \in \pi | P \equiv P_{1} | P_{2}, P_{1} \in \meaningof{E_{1}}, P_{2} \in \meaningof{E_2}\} }
\end{mathpar}

\begin{mathpar}
 \inferrule* [lab=behavior] {} {\meaningof{\langle a?b \rangle E} = \{ P \in \pi | P \equiv Q | u?(y)P', \\ \and \\\\ \and \\ \;\;\; u \in \meaningof{a}, \forall z.P'\{z/y\} \in \meaningof{E\{z/b\}}\}, \and \\ \meaningof{a!E} = \{ P \in \pi | P \equiv Q | x!\langle P' \rangle, x \in \meaningof{a} P' \in \meaningof{E}\} }
\end{mathpar}

\begin{mathpar}
 \inferrule* [lab=nominal] {} {\meaningof{\quotep{E}} = \{ \quotep{P} \in \quotep{\pi} | P \in \meaningof{E} \}, \and \meaningof{\quotep{P}} = \{ \quotep{Q} \in \quotep{\pi} | P \equiv Q \} \and \\ \meaningof{@\quotep{E}} = \{ P \in \pi | P \equiv @x, x \in \meaningof{E} \}}
\end{mathpar}

\begin{eqnarray*}
  \\
  \meaningof{-} : TS \to ST
\end{eqnarray*}

\begin{eqnarray*}
  \\
  L : TS \to ST
\end{eqnarray*}

\begin{eqnarray*}
  \\
  P \models E \iff P \in \meaningof{E}
\end{eqnarray*}

\begin{eqnarray*}
  P \approx_{L} Q \iff \forall E \in L. P \models E \iff Q \models E
\end{eqnarray*}

\begin{eqnarray*}
  P \approx_{K} Q
\end{eqnarray*}

\begin{eqnarray*}
  P \approx Q
\end{eqnarray*}

$\approx_{K} = \approx = \approx_{L}$

\subsubsection{Contextual duality}

Note that contexts extend the quotation operation to a family of
operations from processes to names. Given a context, $M$, we can
define a \emph{nominal context}, $\quotep{M}$ by $\quotep{M}[P] :=
\quotep{M[P]}$. To foreshadow what is to come we observe that these
operations enjoy a duality with processes very much like the duality
between vectors and maps from vectors to scalars.

Further, because the calculus is essentially higher-order, we have a
correspondence between contexts and processes. More specifically,
given a name $x$ and a context $M$ we can construct $M^{*}_{x}$ such
that 

\begin{mathpar}
  M^{*}_{x} | \lift{x}{P} \red M[P]
\end{mathpar}

namely,

\begin{mathpar}
  M^{*}_{x} := x?(u).M[\dropn{u}]
\end{mathpar}

The dependence of $M^{*}_{x}$ on a name makes it an abstraction, 

\begin{mathpar}
  M^{*} := (x)x?(u).M[\dropn{u}]
\end{mathpar}

\subsection{Additional notation}

It will sometimes be convenient to denote the process a name
quotes. We already have the notation $x = \quotep{P}$, but it will be
convenient to introduce an alternate notation, $\procn{x}$, when we
want to emphasize the connection to the use of the name. Note that, by
virtue of name equivalence, $\quotep{\procn{x}} \nameeq x$; so, the
notation is consistent with previous definitions.

Further, because names have structure it is possible to effect
substitutions on the basis of that structure. This means we need to
upgrade our notation for substitutions, which we accomplish by
adapting comprehension notation. Thus,

\begin{mathpar}
  P\{ y / x : x \in S \}
\end{mathpar}

is interpreted to mean the process derived from P by replacing (in a
capture-avoiding manner) each occurrence of $x$ in $S$ by $y$. For example,

\begin{mathpar}
  P\{ \quotep{\procn{x}|\procn{x}} / x : x \in \freenames{P} \}
\end{mathpar}

will replace each (occurrence) of a free name $x$ in $P$ by
$\quotep{\procn{x}|\procn{x}}$.

Also, we will avail ourselves of the notation $x^{L}$ and $x^{R}$ to
denote injections of a name into disjoint copies of the name
space. There are numerous ways to accomplish this. One example can be
found in \cite{MeredithR05}. This notation overloads to vectors of
names: $\vec{x}^{\pi} := (x_{i}^{\pi} \; : \; 0 \leq i < |\vec{x}| )$ where $\pi \in \{L,R\}$.

We also use $P^{\Box} := P|\Box$.

In \cite{MeredithR05} an interpretation of the new operator is
given. It turns out that there are several possible interpretations
all enjoying the requisite algebraic properties of the operator (see
\cite{milner91polyadicpi}). We will therefore make liberal use of
$(\nu\; \vec{x})P$.

% subsection the_syntax_and_semantics_of_the_notation_system (end)   

\input{qm2pi.qmops} 

\input{qm2pi.sterngerlach} 

\input{qm2pi.metric} 

% section concurrent_process_calculi (end)

%\input{qm2pi.proofsketch}

% section proof sketch (end)

%\input{qm2pi.slviaknots} 

% section spatial logic via knots (end)

\input{qm2pi.conclusion}

% section conclusion (end)

%\input{qm2pi.dtcodes} 

% section wiring algorithm (end)

\input{qm2pi.ack} 

% section acknowledgments (end)

\newpage


\bibliographystyle{plain}   
\bibliography{../../biblios/main.bib}

\input{qm2pi.rhodetails}

\end{document}

 

% section wiring algorithm (end)

\documentclass[12pt]{llncs}
%\documentclass{jktr}

\usepackage[pdftex]{hyperref}                   
\usepackage {listings}
\usepackage {mathpartir}
\usepackage{bcprules}
%\usepackage{listings}
                       
\usepackage{graphicx} 
%\usepackage[margins=2.5cm,nohead,nofoot]{geometry}
%\usepackage{geometry}
\usepackage{amsfonts}
\usepackage{amstext}
\usepackage{latexsym}
\usepackage{amssymb}
\usepackage{color}


%\include{myPreamble}
\include{qm2pi.local} 

%\ifpdf
%\usepackage[pdftex]{graphicx}
%\else
%\usepackage{graphicx}
%\fi

 % \ifpdf
%  \usepackage{pdfsync}
%  \if


%\title{Brief Article}
%\author{David F. Snyder}
%\author{L.G. Meredith}

%\address{Dept. of Math., Texas State University--San Marcos, San Marcos, TX 78666}
       
\pagestyle{empty}


\begin{document}

\lstset{language=[Objective]Caml,frame=shadowbox}

\input{qm2pi.front}

% section front matter (end)

\input{qm2pi.intro} 
 
% section introduction (end)

% \input{qm2pi.knotations} 

% section notation (end)

\input{qm2pi.process.calculi} 

% section concurrent_process_calculi_and_spatial_logics_ (end)
    
%\input{qm2pi.knots2pi} 

%\input{qm2pi.trefoil} 

%\input{qm2pi.mainthm} 

% subsection basic_interpretation (end)

%\input{qm2pi.rho.presentation} 
\subsection{The syntax and semantics of the notation system}\label{sub:the_syntax_and_semantics_of_the_notation_system} % (fold)

We now summarize a technical presentation of the calculus that
embodies our theory of dynamics. The typical presentation of such a
calculus follows the style of giving generators and relations on
them. The grammar, below, describing term constructors, freely
generates the set of processes, $\Proc$. This set is then quotiented
by a relation known as structural congruence and it is over this set
that the notion of dynamics is expressed. This presentation is
essentially that of \cite{MeredithR05} with the addition of
polyadicity and summation. For readability we have relegated some of
the technical subtleties to an appendix.

\subsubsection{Process grammar}\label{subsub:process_grammar}

\begin{mathpar}
  \inferrule* [lab=synchronization] {} {{M} \bc \pzero \;|\; x?F \;|\; x!C }
  \and
  \inferrule* [lab=abstraction] {} {{F} \bc (x)P}
  \and
  \inferrule* [lab=concretion] {} {{C} \bc \langle Q \rangle}
  \and
  \inferrule* [lab=process] {} {{P,Q} \bc M \;| \;P|Q \;|\; @{x}}
  \and
  \inferrule* [lab=name] {} {{x} \bc \quotep{P}}
\end{mathpar} 

Note that $\vec{x}$ (resp. $\vec{P}$) denotes a vector of names
(resp. processes) of length $|\vec{x}|$ (resp. $|\vec{P}|$). We adopt
the following useful abbreviations.

\begin{mathpar}
   x?(\vec{y}).P := x.(\vec{y})P \and  x\clift{\vec{P}} := x.\clift{\vec{P}}
   \and x!(y) := \lift{x}{\dropn{y}}
   \and \Pi_{i=0}^{n-1}P_i := P_0 | \ldots | P_{n-1}
\end{mathpar}

\subsubsection{Structural congruence}

\paragraph{Free and bound names and alpha-equivalence.} At the
core of structural equivalence is alpha-equivalence which identifies
process that are the same up to a change of variable. Formally, we
recognize the distinction between free and bound names. The free names
of a process, $\freenames{P}$, may be calculated recursively as
follows:

\begin{mathpar}
\freenames{\pzero} := \emptyset
  \and \\
  \freenames{x?(y).P} := \{ x \} \cup (\freenames{P} \setminus \{ y \})
  \and 
  \freenames{x!\langle P \rangle} := \{ x \} \cup \{ P \} 
  \and \\
  \freenames{P|Q} := \freenames{P} \cup \freenames{Q}
  \and \\
  \freenames{@{x}} := \{ x \}
\end{mathpar}

$\pi$
$\quotep{\pi}$

$\freenames{-} : \pi \to \mathcal{P}(\quotep{\pi})$

\begin{eqnarray*}
  \freenames{\pzero} & := & \emptyset \\
  \freenames{x?(y).P} & := & \{ x \} \cup (\freenames{P} \setminus \{ y \}) \\
  \freenames{x!\langle P \rangle} & := & \{ x \} \cup \{ P \} \\
  \freenames{P|Q} & := & \freenames{P} \cup \freenames{Q} \\
  \freenames{\dropn{x}} & := & \{ x \}
\end{eqnarray*}

The bound names of a process, $\boundnames{P}$, are those names occurring in $P$
that are not free. For example, in $x?(y).0$, the name $x$ is free, while $y$ is bound.

\begin{mathpar}
  \inferrule* [lab=monoidal-laws] {} { P|Q \equiv Q|P \and P|0 \equiv P \and P|(Q|R) \equiv (P|Q)|R }
\end{mathpar}

\begin{mathpar}
  \inferrule* [lab=alpha-equivalence] {} { (x)P \equiv (y)P\{y/x\} \and y \not\in \freenames{P} }
\end{mathpar}

\begin{definition}
Then two processes, $P,Q$, are alpha-equivalent if $P = Q\{\vec{y}/\vec{x}\}$ for
some $\vec{x} \in \boundnames{Q},\vec{y} \in \boundnames{P}$, where $Q\{\vec{y}/\vec{x}\}$
denotes the capture-avoiding substitution of $\vec{y}$ for $\vec{x}$ in $Q$.
\end{definition}

\begin{definition}
  The {\em structural congruence} \cite{SangiorgiWalker} , $\equiv$,
  between processes is the least congruence containing
  alpha-equivalence, satisfying the abelian monoid laws
  (associativity, commutativity and $\pzero$ as identity) for parallel
  composition $|$ and for summation $+$.
\end{definition}

\subsection{Name equivalence}

We take name equivalence, written $\nameeq$, to be the smallest
equivalence relation generated by the following rules.

\begin{mathpar}
\inferrule*[lab=Quote-drop]
{ }
{ \quotep{@{x}} \nameeq x }

\inferrule*[lab=Struct-equiv]
{ P \scong Q }
{ \quotep{P} \nameeq \quotep{Q} }
\end{mathpar}

The astute reader will have noticed that the mutual recursion of names
and processes imposes a mutual recursion on alpha-equivalence and
structural equivalence via name-equivalence. Fortunately, all of this
works out pleasantly and we may calculate in the natural way, free of
concern. The reader interested in the details is referred to the
appendix \ref{appendix:rho_details}.

\subsection{Substitution}

We use $\Proc$ for the set of processes, $\QProc$ for the set of
names, and $\id{\{}\vec{y} / \vec{x} \id{\}}$ to denote partial maps,
$s : \QProc \rightarrow \QProc$. A map, $s$ lifts, uniquely, to a map
on process terms, $\widehat{s} : \Proc \rightarrow \Proc$ by the
following equations.

\begin{mathpar}
  (0) \psubstp{Q}{P} := 0 \\
  (R \juxtap S) \psubstp{Q}{P}
  :=    
  (R)\psubstp{Q}{P} \juxtap (S) \psubstp{Q}{P} \\
  (x?(y).R) \psubstp{Q}{P}    
  :=    
  (x)\substp{Q}{P} (z)\concat( (R \psubstn{z}{y}) \psubstp{Q}{P} ) \\
  (\lift{x}{R}) \psubstp{Q}{P}  
  :=
  \lift{(x)\substp{Q}{P}}{ R \psubstp{Q}{P} } \\
%   (\dropn{x})  \psubstp{Q}{P}       
%   := 
%   \left\{ 
%     \begin{array}{ccc} 
%       \dropn{\quotep{Q}} & & x \nameeq \quotep{P} \\
%       \dropn{x} & & otherwise \\
%     \end{array}
%   \right. 
  (\dropn{x})  \psubstp{Q}{P}       
  := 
  \left\{ 
    \begin{array}{ccc} 
      Q & & x \nameeq \quotep{P} \\
      \dropn{x} & & otherwise \\
    \end{array}
  \right.
\end{mathpar}
 

where

\begin{eqnarray}
  (x)\id{\{} \lpquote Q \rpquote / \lpquote P \rpquote \id{\}}            = 
  \left\{ 
    \begin{array}{ccc}
      \lpquote Q \rpquote & & x \nameeq \lpquote P \rpquote \\
      x & & otherwise \\
    \end{array}
  \right. \nonumber
\end{eqnarray}

and $z$ is chosen distinct from $\quotep{P}$, $\quotep{Q}$, the free
names in $Q$, and all the names in $R$. Our $\alpha$-equivalence will
be built in the standard way from this substitution.

\begin{remark}\label{rem:no_self_referential_names}
  One consequence of these definitions is that $\forall P. \quotep{P}
  \not\in \freenames{P}$.
\end{remark}

\subsection{ Dynamic quote: an example }

Anticipating something of what's to come, consider applying the
substitution, $\widehat{\id{\{}u / z \id{\}}}$, to the following pair
of processes, $\lift{w}{y!(z)}$ and $w[ \lpquote y!(z) \rpquote ]$.

\begin{eqnarray}
	\lift{w}{y!(z)}\widehat{\id{\{}u / z \id{\}}}
		& = &
		\lift{w}{y!(u)} \nonumber\\
	w[ \lpquote y!(z) \rpquote ] \widehat{ \id{\{}u / z \id{\}} }
		& = &
		w[ \lpquote y!(z) \rpquote ] \nonumber
\end{eqnarray}

Because the body of the process between quotes is impervious to
substitution, we get radically different answers. In fact, by
examining the first process in an input context,
e.g. $x?(z).\lift{w}{y!(z)}$, we see that the process under the lift
operator may be shaped by prefixed inputs binding a name inside it. In
this sense, the lift operator will be seen as a way to dynamically
construct processes before reifying them as names.

Finally equipped with these standard features we can present the
dynamics of the calculus.

\subsubsection{Operational semantics} 

Finally, we introduce the computational dynamics. What marks these
algebras as distinct from other more traditionally studied algebraic
structures, e.g. vector spaces or polynomial rings, is the manner in
which dynamics is captured. In traditional structures, dynamics is typically
expressed through morphisms between such structures, as in linear maps
between vector spaces or morphisms between rings. In algebras
associated with the semantics of computation, the dynamics is
expressed as part of the algebraic structure itself, through a
reduction reduction relation typically denoted by $\red$. Below, we
give a recursive presentation of this relation for the calculus used
in the encoding.

$\red \subseteq \pi \times \pi$
$\red : \pi \to \mathcal{P}(\pi)$

\begin{mathpar}
  \inferrule* [lab=Comm] { \textsf{match}( x_{src}, x_{trgt} ) } { x_{trgt}?(y)P \; | \; x_{src}!\langle {Q} \rangle \red P\{\quotep{Q}/y}\} }
  \and \\
  \inferrule* [lab=Par] {{P} \red {P}'} {{{P} | {Q}} \red {{P}' | {Q}}}
  \and
  \inferrule* [lab=Equiv]{{{P} \scong {P}'} \andalso {{P}' \red {Q}'} \andalso {{Q}' \scong {Q}}}{{P} \red {Q}}
\end{mathpar}

\begin{eqnarray*}
  match_{\equiv} (\quotep{P},\quotep{Q}) & := & P \equiv Q \\
  match_{\dagger}(\quotep{P},\quotep{Q}) & := & \forall R. P|Q \red^{*} R => R \red^{*} 0 \\
  match_{K}(\quotep{P},\quotep{Q}) & := & K \mbox{ for some context } K
\end{eqnarray*}

$u?(x)P | u!\langle Q \rangle \red P\{\quotep{Q}/x\}$

%We write $\wred$ for $\red^*$, and $P\red$ if $\exists Q $ such that $ P \red Q$.
We write $P\red$ if $\exists Q $ such that $ P \red Q$ and $P\not\red$, otherwise.

\section{Replication}

As mentioned before, it is known that replication (and hence
recursion) can be implemented in a higher-order process algebra
\cite{SangiorgiWalker}. As our first example of calculation with the
machinery thus far presented we give the construction explicitly in
the {\rhoc}.

\begin{eqnarray}
	D_{x} & := & \prefix{x}{y}{(\binpar{\outputp{x}{y}}{@{y}})} \nonumber\\
	\bangp_{x}{P} & := & \binpar{{x}!\langle{\binpar{D_{x}}{P}}\rangle}{D_{x}} \nonumber
\end{eqnarray}

\begin{eqnarray}
	\bangp_{x}{P} & & \nonumber\\
	=
	& {x}!\langle{(\prefix{x}{y}{(\outputp{x}{y} | @{y})) | P}}\rangle 
	      | \prefix{x}{y}{(\outputp{x}{y} | @{y})} & \nonumber\\
	\red
	& (\outputp{x}{y} | @{y})\substn{\quotep{(\prefix{x}{y}{(@{y} | \outputp{x}{y})) | P}}}{y} & \nonumber\\
	=
	& \outputp{x}{\quotep{(\prefix{x}{y}{(\outputp{x}{y} | @{y})) | P}}}
	  | {(\prefix{x}{y}{(\outputp{x}{y} | @{y})) | P}} & \nonumber\\
	\red
	& \ldots & \nonumber\\
	\red^*
	& P | P | \ldots & \nonumber
\end{eqnarray}

Of course, this encoding, as an implementation, runs away, unfolding
$\bangp{P}$ eagerly. A lazier and more implementable replication
operator, restricted to input-guarded processes, may be obtained as follows.

\begin{eqnarray}
\bangp{\prefix{u}{v}{P}} 
	:= 
	\binpar{\lift{x}{\prefix{u}{v}{(\binpar{D(x)}{P})}}}{D(x)} \nonumber
\end{eqnarray}

\begin{remark}
  Note that the lazier definition still does not deal with summation
  or mixed summation (i.e. sums over input and output). The reader is
  invited to construct definitions of replication that deal with these
  features. 

  Further, the definitions are parameterized in a name, $x$. Can you,
  gentle reader, make a definition that eliminates this parameter and
  guarantees no accidental interaction between the replication
  machinery and the process being replicated -- i.e. no accidental
  sharing of names used by the process to get its work done and the
  name(s) used by the replication to effect copying. This latter
  revision of the definition of replication is crucial to obtaining
  the expected identity $!!P \sim !P$.
\end{remark}

\begin{remark}\label{rem:paradoxical_combinator}
  The reader familiar with the lambda calculus will have noticed the
  similarity between $D$ and the paradoxical combinator.

  [Ed. note: the existence of this seems to suggest we have to be more
  restrictive on the set of processes and names we admit if we are to
  support no-cloning.]
\end{remark}

\subsubsection{Bisimulation}

The computational dynamics gives rise to another kind of equivalence,
the equivalence of computational behavior. As previously mentioned
this is typically captured \emph{via} some form of bisimulation.

% The notion we use in this paper is weak barbed bisimulation
% \cite{milner91polyadicpi}.

The notion we use in this paper is derived from weak barbed
bisimulation \cite{milner91polyadicpi}. 

\begin{definition}
An \emph{observation relation}, $\downarrow_{\mathcal N}$, over a set
of names, $\mathcal N$, is the smallest relation satisfying the rules
below.

\infrule[Out-barb]{y \in {\mathcal N}, \; x \nameeq y}
		  {\outputp{x}{v} \downarrow_{\mathcal N} x}
\infrule[Par-barb]{\mbox{$P\downarrow_{\mathcal N} x$ or $Q\downarrow_{\mathcal N} x$}}
		  {\binpar{P}{Q} \downarrow_{\mathcal N} x}

We write $P \Downarrow_{\mathcal N} x$ if there is $Q$ such that 
$P \wred Q$ and $Q \downarrow_{\mathcal N} x$.
\end{definition}

\begin{definition}
%\label{def.bbisim}
An  ${\mathcal N}$-\emph{barbed bisimulation} over a set of names, ${\mathcal N}$, is a symmetric binary relation 
${\mathcal S}_{\mathcal N}$ between agents such that $P\rel{S}_{\mathcal N}Q$ implies:
\begin{enumerate}
\item If $P \red P'$ then $Q \wred Q'$ and $P'\rel{S}_{\mathcal N} Q'$.
\item If $P\downarrow_{\mathcal N} x$, then $Q\Downarrow_{\mathcal N} x$.
\end{enumerate}
$P$ is ${\mathcal N}$-barbed bisimilar to $Q$, written
$P \wbbisim_{\mathcal N} Q$, if $P \rel{S}_{\mathcal N} Q$ for some ${\mathcal N}$-barbed bisimulation ${\mathcal S}_{\mathcal N}$.
\end{definition}

$\mathcal{R} \subseteq \pi \times \pi$

$P \mathcal{R} Q => \forall P'. P \red P' \Rightarrow \exists Q'. Q \red Q', P' \mathcal{R} Q'$

$P \vdash x \Rightarrow Q \vdash x$

\begin{mathpar}
  \inferrule*[lab=Out-barb]{x \nameeq y}{{y}!\langle{Q}\rangle \vdash x}
  \and
  \inferrule*[lab=Par-barb]{\mbox{$P\vdash x$ or $Q\vdash x$}}{\binpar{P}{Q} \vdash x}
\end{mathpar}

\subsubsection{Contexts}

One of the principle advantages of computational calculi like the
$\pi$-calculus is a well-defined notion of context,
contextual-equivalence and a correlation between
contextual-equivalence and notions of bisimulation. The notion of
context allows the decomposition of a process into (sub-)process and
its syntactic environment, its context. Thus, a context may be
thought of as a process with a ``hole'' (written $\Box$) in it. The
application of a context $M$ to a process $P$, written $M[P]$, is
tantamount to filling the hole in $M$ with $P$. In this paper we do
not need the full weight of this theory, but do make use of the notion
of context in the proof the main theorem. 

\begin{mathpar}
  \inferrule* [lab=summation] {} {{M_{M},M_{N}} \bc \Box \;|\; x.M_{A} \;|\; M_{M}+M_{N}}
  \and
  \inferrule* [lab=agent] {} {{M_{A}} \bc (\vec{x})M_{P} \;| \; \clift{P_0,\ldots,M_{P},\ldots,P_N}}
  \and \\
  \inferrule* [lab=process] {} {{M_{P}} \bc M_{N} \;| \;P|M_{P} }
\end{mathpar} 

\begin{mathpar}
  \inferrule* [lab=sychronization] {} {M_{N} \bc \Box \;|\; x?M_{F} \;|\; x!M_{C}}
  \and
  \inferrule* [lab=abstraction] {} {{M_{F}} \bc (x)M_{P} }
  \and
  \inferrule* [lab=concretion] {} {{M_{C}} \bc \langle M_{P} \rangle }
  \and \\
  \inferrule* [lab=process] {} {{M_{P}} \bc M_{N} \;| \;P|M_{P} }
\end{mathpar}

\begin{definition}[contextual application] Given a context $M$, and
  process $P$, we define the \emph{contextual application}, $M[P] :=
  M\{P/\Box\}$. That is, the contextual application of M to P is the
  substitution of $P$ for $\Box$ in $M$.
\end{definition}

$\meaningof{-} : L \to \mathcal{P}(\pi)$

\begin{mathpar}
  \inferrule* [lab=collection] {} {\meaningof{true} = \pi, \and \meaningof{~E} = \pi \setminus \meaningof{E}, \and \meaningof{E_{1} \& E_{2}} = \meaningof{E_{1}} \cap \meaningof{E_{2}}}
\end{mathpar}

\begin{mathpar}
  \inferrule* [lab=structure] {} {\meaningof{0} = \{ P \in \pi | P \equiv 0 \}, \and \\ \meaningof{E_1 | E_2} = \{ P \in \pi | P \equiv P_{1} | P_{2}, P_{1} \in \meaningof{E_{1}}, P_{2} \in \meaningof{E_2}\} }
\end{mathpar}

\begin{mathpar}
 \inferrule* [lab=behavior] {} {\meaningof{\langle a?b \rangle E} = \{ P \in \pi | P \equiv Q | u?(y)P', \\ \and \\\\ \and \\ \;\;\; u \in \meaningof{a}, \forall z.P'\{z/y\} \in \meaningof{E\{z/b\}}\}, \and \\ \meaningof{a!E} = \{ P \in \pi | P \equiv Q | x!\langle P' \rangle, x \in \meaningof{a} P' \in \meaningof{E}\} }
\end{mathpar}

\begin{mathpar}
 \inferrule* [lab=nominal] {} {\meaningof{\quotep{E}} = \{ \quotep{P} \in \quotep{\pi} | P \in \meaningof{E} \}, \and \meaningof{\quotep{P}} = \{ \quotep{Q} \in \quotep{\pi} | P \equiv Q \} \and \\ \meaningof{@\quotep{E}} = \{ P \in \pi | P \equiv @x, x \in \meaningof{E} \}}
\end{mathpar}

\begin{eqnarray*}
  \\
  \meaningof{-} : TS \to ST
\end{eqnarray*}

\begin{eqnarray*}
  \\
  L : TS \to ST
\end{eqnarray*}

\begin{eqnarray*}
  \\
  P \models E \iff P \in \meaningof{E}
\end{eqnarray*}

\begin{eqnarray*}
  P \approx_{L} Q \iff \forall E \in L. P \models E \iff Q \models E
\end{eqnarray*}

\begin{eqnarray*}
  P \approx_{K} Q
\end{eqnarray*}

\begin{eqnarray*}
  P \approx Q
\end{eqnarray*}

$\approx_{K} = \approx = \approx_{L}$

\subsubsection{Contextual duality}

Note that contexts extend the quotation operation to a family of
operations from processes to names. Given a context, $M$, we can
define a \emph{nominal context}, $\quotep{M}$ by $\quotep{M}[P] :=
\quotep{M[P]}$. To foreshadow what is to come we observe that these
operations enjoy a duality with processes very much like the duality
between vectors and maps from vectors to scalars.

Further, because the calculus is essentially higher-order, we have a
correspondence between contexts and processes. More specifically,
given a name $x$ and a context $M$ we can construct $M^{*}_{x}$ such
that 

\begin{mathpar}
  M^{*}_{x} | \lift{x}{P} \red M[P]
\end{mathpar}

namely,

\begin{mathpar}
  M^{*}_{x} := x?(u).M[\dropn{u}]
\end{mathpar}

The dependence of $M^{*}_{x}$ on a name makes it an abstraction, 

\begin{mathpar}
  M^{*} := (x)x?(u).M[\dropn{u}]
\end{mathpar}

\subsection{Additional notation}

It will sometimes be convenient to denote the process a name
quotes. We already have the notation $x = \quotep{P}$, but it will be
convenient to introduce an alternate notation, $\procn{x}$, when we
want to emphasize the connection to the use of the name. Note that, by
virtue of name equivalence, $\quotep{\procn{x}} \nameeq x$; so, the
notation is consistent with previous definitions.

Further, because names have structure it is possible to effect
substitutions on the basis of that structure. This means we need to
upgrade our notation for substitutions, which we accomplish by
adapting comprehension notation. Thus,

\begin{mathpar}
  P\{ y / x : x \in S \}
\end{mathpar}

is interpreted to mean the process derived from P by replacing (in a
capture-avoiding manner) each occurrence of $x$ in $S$ by $y$. For example,

\begin{mathpar}
  P\{ \quotep{\procn{x}|\procn{x}} / x : x \in \freenames{P} \}
\end{mathpar}

will replace each (occurrence) of a free name $x$ in $P$ by
$\quotep{\procn{x}|\procn{x}}$.

Also, we will avail ourselves of the notation $x^{L}$ and $x^{R}$ to
denote injections of a name into disjoint copies of the name
space. There are numerous ways to accomplish this. One example can be
found in \cite{MeredithR05}. This notation overloads to vectors of
names: $\vec{x}^{\pi} := (x_{i}^{\pi} \; : \; 0 \leq i < |\vec{x}| )$ where $\pi \in \{L,R\}$.

We also use $P^{\Box} := P|\Box$.

In \cite{MeredithR05} an interpretation of the new operator is
given. It turns out that there are several possible interpretations
all enjoying the requisite algebraic properties of the operator (see
\cite{milner91polyadicpi}). We will therefore make liberal use of
$(\nu\; \vec{x})P$.

% subsection the_syntax_and_semantics_of_the_notation_system (end)   

\input{qm2pi.qmops} 

\input{qm2pi.sterngerlach} 

\input{qm2pi.metric} 

% section concurrent_process_calculi (end)

%\input{qm2pi.proofsketch}

% section proof sketch (end)

%\input{qm2pi.slviaknots} 

% section spatial logic via knots (end)

\input{qm2pi.conclusion}

% section conclusion (end)

%\input{qm2pi.dtcodes} 

% section wiring algorithm (end)

\input{qm2pi.ack} 

% section acknowledgments (end)

\newpage


\bibliographystyle{plain}   
\bibliography{../../biblios/main.bib}

\input{qm2pi.rhodetails}

\end{document}

 

% section acknowledgments (end)

\newpage


\bibliographystyle{plain}   
\bibliography{../../biblios/main.bib}

\documentclass[12pt]{llncs}
%\documentclass{jktr}

\usepackage[pdftex]{hyperref}                   
\usepackage {listings}
\usepackage {mathpartir}
\usepackage{bcprules}
%\usepackage{listings}
                       
\usepackage{graphicx} 
%\usepackage[margins=2.5cm,nohead,nofoot]{geometry}
%\usepackage{geometry}
\usepackage{amsfonts}
\usepackage{amstext}
\usepackage{latexsym}
\usepackage{amssymb}
\usepackage{color}


%\include{myPreamble}
\include{qm2pi.local} 

%\ifpdf
%\usepackage[pdftex]{graphicx}
%\else
%\usepackage{graphicx}
%\fi

 % \ifpdf
%  \usepackage{pdfsync}
%  \if


%\title{Brief Article}
%\author{David F. Snyder}
%\author{L.G. Meredith}

%\address{Dept. of Math., Texas State University--San Marcos, San Marcos, TX 78666}
       
\pagestyle{empty}


\begin{document}

\lstset{language=[Objective]Caml,frame=shadowbox}

\input{qm2pi.front}

% section front matter (end)

\input{qm2pi.intro} 
 
% section introduction (end)

% \input{qm2pi.knotations} 

% section notation (end)

\input{qm2pi.process.calculi} 

% section concurrent_process_calculi_and_spatial_logics_ (end)
    
%\input{qm2pi.knots2pi} 

%\input{qm2pi.trefoil} 

%\input{qm2pi.mainthm} 

% subsection basic_interpretation (end)

%\input{qm2pi.rho.presentation} 
\subsection{The syntax and semantics of the notation system}\label{sub:the_syntax_and_semantics_of_the_notation_system} % (fold)

We now summarize a technical presentation of the calculus that
embodies our theory of dynamics. The typical presentation of such a
calculus follows the style of giving generators and relations on
them. The grammar, below, describing term constructors, freely
generates the set of processes, $\Proc$. This set is then quotiented
by a relation known as structural congruence and it is over this set
that the notion of dynamics is expressed. This presentation is
essentially that of \cite{MeredithR05} with the addition of
polyadicity and summation. For readability we have relegated some of
the technical subtleties to an appendix.

\subsubsection{Process grammar}\label{subsub:process_grammar}

\begin{mathpar}
  \inferrule* [lab=synchronization] {} {{M} \bc \pzero \;|\; x?F \;|\; x!C }
  \and
  \inferrule* [lab=abstraction] {} {{F} \bc (x)P}
  \and
  \inferrule* [lab=concretion] {} {{C} \bc \langle Q \rangle}
  \and
  \inferrule* [lab=process] {} {{P,Q} \bc M \;| \;P|Q \;|\; @{x}}
  \and
  \inferrule* [lab=name] {} {{x} \bc \quotep{P}}
\end{mathpar} 

Note that $\vec{x}$ (resp. $\vec{P}$) denotes a vector of names
(resp. processes) of length $|\vec{x}|$ (resp. $|\vec{P}|$). We adopt
the following useful abbreviations.

\begin{mathpar}
   x?(\vec{y}).P := x.(\vec{y})P \and  x\clift{\vec{P}} := x.\clift{\vec{P}}
   \and x!(y) := \lift{x}{\dropn{y}}
   \and \Pi_{i=0}^{n-1}P_i := P_0 | \ldots | P_{n-1}
\end{mathpar}

\subsubsection{Structural congruence}

\paragraph{Free and bound names and alpha-equivalence.} At the
core of structural equivalence is alpha-equivalence which identifies
process that are the same up to a change of variable. Formally, we
recognize the distinction between free and bound names. The free names
of a process, $\freenames{P}$, may be calculated recursively as
follows:

\begin{mathpar}
\freenames{\pzero} := \emptyset
  \and \\
  \freenames{x?(y).P} := \{ x \} \cup (\freenames{P} \setminus \{ y \})
  \and 
  \freenames{x!\langle P \rangle} := \{ x \} \cup \{ P \} 
  \and \\
  \freenames{P|Q} := \freenames{P} \cup \freenames{Q}
  \and \\
  \freenames{@{x}} := \{ x \}
\end{mathpar}

$\pi$
$\quotep{\pi}$

$\freenames{-} : \pi \to \mathcal{P}(\quotep{\pi})$

\begin{eqnarray*}
  \freenames{\pzero} & := & \emptyset \\
  \freenames{x?(y).P} & := & \{ x \} \cup (\freenames{P} \setminus \{ y \}) \\
  \freenames{x!\langle P \rangle} & := & \{ x \} \cup \{ P \} \\
  \freenames{P|Q} & := & \freenames{P} \cup \freenames{Q} \\
  \freenames{\dropn{x}} & := & \{ x \}
\end{eqnarray*}

The bound names of a process, $\boundnames{P}$, are those names occurring in $P$
that are not free. For example, in $x?(y).0$, the name $x$ is free, while $y$ is bound.

\begin{mathpar}
  \inferrule* [lab=monoidal-laws] {} { P|Q \equiv Q|P \and P|0 \equiv P \and P|(Q|R) \equiv (P|Q)|R }
\end{mathpar}

\begin{mathpar}
  \inferrule* [lab=alpha-equivalence] {} { (x)P \equiv (y)P\{y/x\} \and y \not\in \freenames{P} }
\end{mathpar}

\begin{definition}
Then two processes, $P,Q$, are alpha-equivalent if $P = Q\{\vec{y}/\vec{x}\}$ for
some $\vec{x} \in \boundnames{Q},\vec{y} \in \boundnames{P}$, where $Q\{\vec{y}/\vec{x}\}$
denotes the capture-avoiding substitution of $\vec{y}$ for $\vec{x}$ in $Q$.
\end{definition}

\begin{definition}
  The {\em structural congruence} \cite{SangiorgiWalker} , $\equiv$,
  between processes is the least congruence containing
  alpha-equivalence, satisfying the abelian monoid laws
  (associativity, commutativity and $\pzero$ as identity) for parallel
  composition $|$ and for summation $+$.
\end{definition}

\subsection{Name equivalence}

We take name equivalence, written $\nameeq$, to be the smallest
equivalence relation generated by the following rules.

\begin{mathpar}
\inferrule*[lab=Quote-drop]
{ }
{ \quotep{@{x}} \nameeq x }

\inferrule*[lab=Struct-equiv]
{ P \scong Q }
{ \quotep{P} \nameeq \quotep{Q} }
\end{mathpar}

The astute reader will have noticed that the mutual recursion of names
and processes imposes a mutual recursion on alpha-equivalence and
structural equivalence via name-equivalence. Fortunately, all of this
works out pleasantly and we may calculate in the natural way, free of
concern. The reader interested in the details is referred to the
appendix \ref{appendix:rho_details}.

\subsection{Substitution}

We use $\Proc$ for the set of processes, $\QProc$ for the set of
names, and $\id{\{}\vec{y} / \vec{x} \id{\}}$ to denote partial maps,
$s : \QProc \rightarrow \QProc$. A map, $s$ lifts, uniquely, to a map
on process terms, $\widehat{s} : \Proc \rightarrow \Proc$ by the
following equations.

\begin{mathpar}
  (0) \psubstp{Q}{P} := 0 \\
  (R \juxtap S) \psubstp{Q}{P}
  :=    
  (R)\psubstp{Q}{P} \juxtap (S) \psubstp{Q}{P} \\
  (x?(y).R) \psubstp{Q}{P}    
  :=    
  (x)\substp{Q}{P} (z)\concat( (R \psubstn{z}{y}) \psubstp{Q}{P} ) \\
  (\lift{x}{R}) \psubstp{Q}{P}  
  :=
  \lift{(x)\substp{Q}{P}}{ R \psubstp{Q}{P} } \\
%   (\dropn{x})  \psubstp{Q}{P}       
%   := 
%   \left\{ 
%     \begin{array}{ccc} 
%       \dropn{\quotep{Q}} & & x \nameeq \quotep{P} \\
%       \dropn{x} & & otherwise \\
%     \end{array}
%   \right. 
  (\dropn{x})  \psubstp{Q}{P}       
  := 
  \left\{ 
    \begin{array}{ccc} 
      Q & & x \nameeq \quotep{P} \\
      \dropn{x} & & otherwise \\
    \end{array}
  \right.
\end{mathpar}
 

where

\begin{eqnarray}
  (x)\id{\{} \lpquote Q \rpquote / \lpquote P \rpquote \id{\}}            = 
  \left\{ 
    \begin{array}{ccc}
      \lpquote Q \rpquote & & x \nameeq \lpquote P \rpquote \\
      x & & otherwise \\
    \end{array}
  \right. \nonumber
\end{eqnarray}

and $z$ is chosen distinct from $\quotep{P}$, $\quotep{Q}$, the free
names in $Q$, and all the names in $R$. Our $\alpha$-equivalence will
be built in the standard way from this substitution.

\begin{remark}\label{rem:no_self_referential_names}
  One consequence of these definitions is that $\forall P. \quotep{P}
  \not\in \freenames{P}$.
\end{remark}

\subsection{ Dynamic quote: an example }

Anticipating something of what's to come, consider applying the
substitution, $\widehat{\id{\{}u / z \id{\}}}$, to the following pair
of processes, $\lift{w}{y!(z)}$ and $w[ \lpquote y!(z) \rpquote ]$.

\begin{eqnarray}
	\lift{w}{y!(z)}\widehat{\id{\{}u / z \id{\}}}
		& = &
		\lift{w}{y!(u)} \nonumber\\
	w[ \lpquote y!(z) \rpquote ] \widehat{ \id{\{}u / z \id{\}} }
		& = &
		w[ \lpquote y!(z) \rpquote ] \nonumber
\end{eqnarray}

Because the body of the process between quotes is impervious to
substitution, we get radically different answers. In fact, by
examining the first process in an input context,
e.g. $x?(z).\lift{w}{y!(z)}$, we see that the process under the lift
operator may be shaped by prefixed inputs binding a name inside it. In
this sense, the lift operator will be seen as a way to dynamically
construct processes before reifying them as names.

Finally equipped with these standard features we can present the
dynamics of the calculus.

\subsubsection{Operational semantics} 

Finally, we introduce the computational dynamics. What marks these
algebras as distinct from other more traditionally studied algebraic
structures, e.g. vector spaces or polynomial rings, is the manner in
which dynamics is captured. In traditional structures, dynamics is typically
expressed through morphisms between such structures, as in linear maps
between vector spaces or morphisms between rings. In algebras
associated with the semantics of computation, the dynamics is
expressed as part of the algebraic structure itself, through a
reduction reduction relation typically denoted by $\red$. Below, we
give a recursive presentation of this relation for the calculus used
in the encoding.

$\red \subseteq \pi \times \pi$
$\red : \pi \to \mathcal{P}(\pi)$

\begin{mathpar}
  \inferrule* [lab=Comm] { \textsf{match}( x_{src}, x_{trgt} ) } { x_{trgt}?(y)P \; | \; x_{src}!\langle {Q} \rangle \red P\{\quotep{Q}/y}\} }
  \and \\
  \inferrule* [lab=Par] {{P} \red {P}'} {{{P} | {Q}} \red {{P}' | {Q}}}
  \and
  \inferrule* [lab=Equiv]{{{P} \scong {P}'} \andalso {{P}' \red {Q}'} \andalso {{Q}' \scong {Q}}}{{P} \red {Q}}
\end{mathpar}

\begin{eqnarray*}
  match_{\equiv} (\quotep{P},\quotep{Q}) & := & P \equiv Q \\
  match_{\dagger}(\quotep{P},\quotep{Q}) & := & \forall R. P|Q \red^{*} R => R \red^{*} 0 \\
  match_{K}(\quotep{P},\quotep{Q}) & := & K \mbox{ for some context } K
\end{eqnarray*}

$u?(x)P | u!\langle Q \rangle \red P\{\quotep{Q}/x\}$

%We write $\wred$ for $\red^*$, and $P\red$ if $\exists Q $ such that $ P \red Q$.
We write $P\red$ if $\exists Q $ such that $ P \red Q$ and $P\not\red$, otherwise.

\section{Replication}

As mentioned before, it is known that replication (and hence
recursion) can be implemented in a higher-order process algebra
\cite{SangiorgiWalker}. As our first example of calculation with the
machinery thus far presented we give the construction explicitly in
the {\rhoc}.

\begin{eqnarray}
	D_{x} & := & \prefix{x}{y}{(\binpar{\outputp{x}{y}}{@{y}})} \nonumber\\
	\bangp_{x}{P} & := & \binpar{{x}!\langle{\binpar{D_{x}}{P}}\rangle}{D_{x}} \nonumber
\end{eqnarray}

\begin{eqnarray}
	\bangp_{x}{P} & & \nonumber\\
	=
	& {x}!\langle{(\prefix{x}{y}{(\outputp{x}{y} | @{y})) | P}}\rangle 
	      | \prefix{x}{y}{(\outputp{x}{y} | @{y})} & \nonumber\\
	\red
	& (\outputp{x}{y} | @{y})\substn{\quotep{(\prefix{x}{y}{(@{y} | \outputp{x}{y})) | P}}}{y} & \nonumber\\
	=
	& \outputp{x}{\quotep{(\prefix{x}{y}{(\outputp{x}{y} | @{y})) | P}}}
	  | {(\prefix{x}{y}{(\outputp{x}{y} | @{y})) | P}} & \nonumber\\
	\red
	& \ldots & \nonumber\\
	\red^*
	& P | P | \ldots & \nonumber
\end{eqnarray}

Of course, this encoding, as an implementation, runs away, unfolding
$\bangp{P}$ eagerly. A lazier and more implementable replication
operator, restricted to input-guarded processes, may be obtained as follows.

\begin{eqnarray}
\bangp{\prefix{u}{v}{P}} 
	:= 
	\binpar{\lift{x}{\prefix{u}{v}{(\binpar{D(x)}{P})}}}{D(x)} \nonumber
\end{eqnarray}

\begin{remark}
  Note that the lazier definition still does not deal with summation
  or mixed summation (i.e. sums over input and output). The reader is
  invited to construct definitions of replication that deal with these
  features. 

  Further, the definitions are parameterized in a name, $x$. Can you,
  gentle reader, make a definition that eliminates this parameter and
  guarantees no accidental interaction between the replication
  machinery and the process being replicated -- i.e. no accidental
  sharing of names used by the process to get its work done and the
  name(s) used by the replication to effect copying. This latter
  revision of the definition of replication is crucial to obtaining
  the expected identity $!!P \sim !P$.
\end{remark}

\begin{remark}\label{rem:paradoxical_combinator}
  The reader familiar with the lambda calculus will have noticed the
  similarity between $D$ and the paradoxical combinator.

  [Ed. note: the existence of this seems to suggest we have to be more
  restrictive on the set of processes and names we admit if we are to
  support no-cloning.]
\end{remark}

\subsubsection{Bisimulation}

The computational dynamics gives rise to another kind of equivalence,
the equivalence of computational behavior. As previously mentioned
this is typically captured \emph{via} some form of bisimulation.

% The notion we use in this paper is weak barbed bisimulation
% \cite{milner91polyadicpi}.

The notion we use in this paper is derived from weak barbed
bisimulation \cite{milner91polyadicpi}. 

\begin{definition}
An \emph{observation relation}, $\downarrow_{\mathcal N}$, over a set
of names, $\mathcal N$, is the smallest relation satisfying the rules
below.

\infrule[Out-barb]{y \in {\mathcal N}, \; x \nameeq y}
		  {\outputp{x}{v} \downarrow_{\mathcal N} x}
\infrule[Par-barb]{\mbox{$P\downarrow_{\mathcal N} x$ or $Q\downarrow_{\mathcal N} x$}}
		  {\binpar{P}{Q} \downarrow_{\mathcal N} x}

We write $P \Downarrow_{\mathcal N} x$ if there is $Q$ such that 
$P \wred Q$ and $Q \downarrow_{\mathcal N} x$.
\end{definition}

\begin{definition}
%\label{def.bbisim}
An  ${\mathcal N}$-\emph{barbed bisimulation} over a set of names, ${\mathcal N}$, is a symmetric binary relation 
${\mathcal S}_{\mathcal N}$ between agents such that $P\rel{S}_{\mathcal N}Q$ implies:
\begin{enumerate}
\item If $P \red P'$ then $Q \wred Q'$ and $P'\rel{S}_{\mathcal N} Q'$.
\item If $P\downarrow_{\mathcal N} x$, then $Q\Downarrow_{\mathcal N} x$.
\end{enumerate}
$P$ is ${\mathcal N}$-barbed bisimilar to $Q$, written
$P \wbbisim_{\mathcal N} Q$, if $P \rel{S}_{\mathcal N} Q$ for some ${\mathcal N}$-barbed bisimulation ${\mathcal S}_{\mathcal N}$.
\end{definition}

$\mathcal{R} \subseteq \pi \times \pi$

$P \mathcal{R} Q => \forall P'. P \red P' \Rightarrow \exists Q'. Q \red Q', P' \mathcal{R} Q'$

$P \vdash x \Rightarrow Q \vdash x$

\begin{mathpar}
  \inferrule*[lab=Out-barb]{x \nameeq y}{{y}!\langle{Q}\rangle \vdash x}
  \and
  \inferrule*[lab=Par-barb]{\mbox{$P\vdash x$ or $Q\vdash x$}}{\binpar{P}{Q} \vdash x}
\end{mathpar}

\subsubsection{Contexts}

One of the principle advantages of computational calculi like the
$\pi$-calculus is a well-defined notion of context,
contextual-equivalence and a correlation between
contextual-equivalence and notions of bisimulation. The notion of
context allows the decomposition of a process into (sub-)process and
its syntactic environment, its context. Thus, a context may be
thought of as a process with a ``hole'' (written $\Box$) in it. The
application of a context $M$ to a process $P$, written $M[P]$, is
tantamount to filling the hole in $M$ with $P$. In this paper we do
not need the full weight of this theory, but do make use of the notion
of context in the proof the main theorem. 

\begin{mathpar}
  \inferrule* [lab=summation] {} {{M_{M},M_{N}} \bc \Box \;|\; x.M_{A} \;|\; M_{M}+M_{N}}
  \and
  \inferrule* [lab=agent] {} {{M_{A}} \bc (\vec{x})M_{P} \;| \; \clift{P_0,\ldots,M_{P},\ldots,P_N}}
  \and \\
  \inferrule* [lab=process] {} {{M_{P}} \bc M_{N} \;| \;P|M_{P} }
\end{mathpar} 

\begin{mathpar}
  \inferrule* [lab=sychronization] {} {M_{N} \bc \Box \;|\; x?M_{F} \;|\; x!M_{C}}
  \and
  \inferrule* [lab=abstraction] {} {{M_{F}} \bc (x)M_{P} }
  \and
  \inferrule* [lab=concretion] {} {{M_{C}} \bc \langle M_{P} \rangle }
  \and \\
  \inferrule* [lab=process] {} {{M_{P}} \bc M_{N} \;| \;P|M_{P} }
\end{mathpar}

\begin{definition}[contextual application] Given a context $M$, and
  process $P$, we define the \emph{contextual application}, $M[P] :=
  M\{P/\Box\}$. That is, the contextual application of M to P is the
  substitution of $P$ for $\Box$ in $M$.
\end{definition}

$\meaningof{-} : L \to \mathcal{P}(\pi)$

\begin{mathpar}
  \inferrule* [lab=collection] {} {\meaningof{true} = \pi, \and \meaningof{~E} = \pi \setminus \meaningof{E}, \and \meaningof{E_{1} \& E_{2}} = \meaningof{E_{1}} \cap \meaningof{E_{2}}}
\end{mathpar}

\begin{mathpar}
  \inferrule* [lab=structure] {} {\meaningof{0} = \{ P \in \pi | P \equiv 0 \}, \and \\ \meaningof{E_1 | E_2} = \{ P \in \pi | P \equiv P_{1} | P_{2}, P_{1} \in \meaningof{E_{1}}, P_{2} \in \meaningof{E_2}\} }
\end{mathpar}

\begin{mathpar}
 \inferrule* [lab=behavior] {} {\meaningof{\langle a?b \rangle E} = \{ P \in \pi | P \equiv Q | u?(y)P', \\ \and \\\\ \and \\ \;\;\; u \in \meaningof{a}, \forall z.P'\{z/y\} \in \meaningof{E\{z/b\}}\}, \and \\ \meaningof{a!E} = \{ P \in \pi | P \equiv Q | x!\langle P' \rangle, x \in \meaningof{a} P' \in \meaningof{E}\} }
\end{mathpar}

\begin{mathpar}
 \inferrule* [lab=nominal] {} {\meaningof{\quotep{E}} = \{ \quotep{P} \in \quotep{\pi} | P \in \meaningof{E} \}, \and \meaningof{\quotep{P}} = \{ \quotep{Q} \in \quotep{\pi} | P \equiv Q \} \and \\ \meaningof{@\quotep{E}} = \{ P \in \pi | P \equiv @x, x \in \meaningof{E} \}}
\end{mathpar}

\begin{eqnarray*}
  \\
  \meaningof{-} : TS \to ST
\end{eqnarray*}

\begin{eqnarray*}
  \\
  L : TS \to ST
\end{eqnarray*}

\begin{eqnarray*}
  \\
  P \models E \iff P \in \meaningof{E}
\end{eqnarray*}

\begin{eqnarray*}
  P \approx_{L} Q \iff \forall E \in L. P \models E \iff Q \models E
\end{eqnarray*}

\begin{eqnarray*}
  P \approx_{K} Q
\end{eqnarray*}

\begin{eqnarray*}
  P \approx Q
\end{eqnarray*}

$\approx_{K} = \approx = \approx_{L}$

\subsubsection{Contextual duality}

Note that contexts extend the quotation operation to a family of
operations from processes to names. Given a context, $M$, we can
define a \emph{nominal context}, $\quotep{M}$ by $\quotep{M}[P] :=
\quotep{M[P]}$. To foreshadow what is to come we observe that these
operations enjoy a duality with processes very much like the duality
between vectors and maps from vectors to scalars.

Further, because the calculus is essentially higher-order, we have a
correspondence between contexts and processes. More specifically,
given a name $x$ and a context $M$ we can construct $M^{*}_{x}$ such
that 

\begin{mathpar}
  M^{*}_{x} | \lift{x}{P} \red M[P]
\end{mathpar}

namely,

\begin{mathpar}
  M^{*}_{x} := x?(u).M[\dropn{u}]
\end{mathpar}

The dependence of $M^{*}_{x}$ on a name makes it an abstraction, 

\begin{mathpar}
  M^{*} := (x)x?(u).M[\dropn{u}]
\end{mathpar}

\subsection{Additional notation}

It will sometimes be convenient to denote the process a name
quotes. We already have the notation $x = \quotep{P}$, but it will be
convenient to introduce an alternate notation, $\procn{x}$, when we
want to emphasize the connection to the use of the name. Note that, by
virtue of name equivalence, $\quotep{\procn{x}} \nameeq x$; so, the
notation is consistent with previous definitions.

Further, because names have structure it is possible to effect
substitutions on the basis of that structure. This means we need to
upgrade our notation for substitutions, which we accomplish by
adapting comprehension notation. Thus,

\begin{mathpar}
  P\{ y / x : x \in S \}
\end{mathpar}

is interpreted to mean the process derived from P by replacing (in a
capture-avoiding manner) each occurrence of $x$ in $S$ by $y$. For example,

\begin{mathpar}
  P\{ \quotep{\procn{x}|\procn{x}} / x : x \in \freenames{P} \}
\end{mathpar}

will replace each (occurrence) of a free name $x$ in $P$ by
$\quotep{\procn{x}|\procn{x}}$.

Also, we will avail ourselves of the notation $x^{L}$ and $x^{R}$ to
denote injections of a name into disjoint copies of the name
space. There are numerous ways to accomplish this. One example can be
found in \cite{MeredithR05}. This notation overloads to vectors of
names: $\vec{x}^{\pi} := (x_{i}^{\pi} \; : \; 0 \leq i < |\vec{x}| )$ where $\pi \in \{L,R\}$.

We also use $P^{\Box} := P|\Box$.

In \cite{MeredithR05} an interpretation of the new operator is
given. It turns out that there are several possible interpretations
all enjoying the requisite algebraic properties of the operator (see
\cite{milner91polyadicpi}). We will therefore make liberal use of
$(\nu\; \vec{x})P$.

% subsection the_syntax_and_semantics_of_the_notation_system (end)   

\input{qm2pi.qmops} 

\input{qm2pi.sterngerlach} 

\input{qm2pi.metric} 

% section concurrent_process_calculi (end)

%\input{qm2pi.proofsketch}

% section proof sketch (end)

%\input{qm2pi.slviaknots} 

% section spatial logic via knots (end)

\input{qm2pi.conclusion}

% section conclusion (end)

%\input{qm2pi.dtcodes} 

% section wiring algorithm (end)

\input{qm2pi.ack} 

% section acknowledgments (end)

\newpage


\bibliographystyle{plain}   
\bibliography{../../biblios/main.bib}

\input{qm2pi.rhodetails}

\end{document}



\end{document}

 

%\documentclass[12pt]{llncs}
%\documentclass{jktr}

\usepackage[pdftex]{hyperref}                   
\usepackage {listings}
\usepackage {mathpartir}
\usepackage{bcprules}
%\usepackage{listings}
                       
\usepackage{graphicx} 
%\usepackage[margins=2.5cm,nohead,nofoot]{geometry}
%\usepackage{geometry}
\usepackage{amsfonts}
\usepackage{amstext}
\usepackage{latexsym}
\usepackage{amssymb}
\usepackage{color}


%\include{myPreamble}
\documentclass[12pt]{llncs}
%\documentclass{jktr}

\usepackage[pdftex]{hyperref}                   
\usepackage {listings}
\usepackage {mathpartir}
\usepackage{bcprules}
%\usepackage{listings}
                       
\usepackage{graphicx} 
%\usepackage[margins=2.5cm,nohead,nofoot]{geometry}
%\usepackage{geometry}
\usepackage{amsfonts}
\usepackage{amstext}
\usepackage{latexsym}
\usepackage{amssymb}
\usepackage{color}


%\include{myPreamble}
\include{qm2pi.local} 

%\ifpdf
%\usepackage[pdftex]{graphicx}
%\else
%\usepackage{graphicx}
%\fi

 % \ifpdf
%  \usepackage{pdfsync}
%  \if


%\title{Brief Article}
%\author{David F. Snyder}
%\author{L.G. Meredith}

%\address{Dept. of Math., Texas State University--San Marcos, San Marcos, TX 78666}
       
\pagestyle{empty}


\begin{document}

\lstset{language=[Objective]Caml,frame=shadowbox}

\input{qm2pi.front}

% section front matter (end)

\input{qm2pi.intro} 
 
% section introduction (end)

% \input{qm2pi.knotations} 

% section notation (end)

\input{qm2pi.process.calculi} 

% section concurrent_process_calculi_and_spatial_logics_ (end)
    
%\input{qm2pi.knots2pi} 

%\input{qm2pi.trefoil} 

%\input{qm2pi.mainthm} 

% subsection basic_interpretation (end)

%\input{qm2pi.rho.presentation} 
\subsection{The syntax and semantics of the notation system}\label{sub:the_syntax_and_semantics_of_the_notation_system} % (fold)

We now summarize a technical presentation of the calculus that
embodies our theory of dynamics. The typical presentation of such a
calculus follows the style of giving generators and relations on
them. The grammar, below, describing term constructors, freely
generates the set of processes, $\Proc$. This set is then quotiented
by a relation known as structural congruence and it is over this set
that the notion of dynamics is expressed. This presentation is
essentially that of \cite{MeredithR05} with the addition of
polyadicity and summation. For readability we have relegated some of
the technical subtleties to an appendix.

\subsubsection{Process grammar}\label{subsub:process_grammar}

\begin{mathpar}
  \inferrule* [lab=synchronization] {} {{M} \bc \pzero \;|\; x?F \;|\; x!C }
  \and
  \inferrule* [lab=abstraction] {} {{F} \bc (x)P}
  \and
  \inferrule* [lab=concretion] {} {{C} \bc \langle Q \rangle}
  \and
  \inferrule* [lab=process] {} {{P,Q} \bc M \;| \;P|Q \;|\; @{x}}
  \and
  \inferrule* [lab=name] {} {{x} \bc \quotep{P}}
\end{mathpar} 

Note that $\vec{x}$ (resp. $\vec{P}$) denotes a vector of names
(resp. processes) of length $|\vec{x}|$ (resp. $|\vec{P}|$). We adopt
the following useful abbreviations.

\begin{mathpar}
   x?(\vec{y}).P := x.(\vec{y})P \and  x\clift{\vec{P}} := x.\clift{\vec{P}}
   \and x!(y) := \lift{x}{\dropn{y}}
   \and \Pi_{i=0}^{n-1}P_i := P_0 | \ldots | P_{n-1}
\end{mathpar}

\subsubsection{Structural congruence}

\paragraph{Free and bound names and alpha-equivalence.} At the
core of structural equivalence is alpha-equivalence which identifies
process that are the same up to a change of variable. Formally, we
recognize the distinction between free and bound names. The free names
of a process, $\freenames{P}$, may be calculated recursively as
follows:

\begin{mathpar}
\freenames{\pzero} := \emptyset
  \and \\
  \freenames{x?(y).P} := \{ x \} \cup (\freenames{P} \setminus \{ y \})
  \and 
  \freenames{x!\langle P \rangle} := \{ x \} \cup \{ P \} 
  \and \\
  \freenames{P|Q} := \freenames{P} \cup \freenames{Q}
  \and \\
  \freenames{@{x}} := \{ x \}
\end{mathpar}

$\pi$
$\quotep{\pi}$

$\freenames{-} : \pi \to \mathcal{P}(\quotep{\pi})$

\begin{eqnarray*}
  \freenames{\pzero} & := & \emptyset \\
  \freenames{x?(y).P} & := & \{ x \} \cup (\freenames{P} \setminus \{ y \}) \\
  \freenames{x!\langle P \rangle} & := & \{ x \} \cup \{ P \} \\
  \freenames{P|Q} & := & \freenames{P} \cup \freenames{Q} \\
  \freenames{\dropn{x}} & := & \{ x \}
\end{eqnarray*}

The bound names of a process, $\boundnames{P}$, are those names occurring in $P$
that are not free. For example, in $x?(y).0$, the name $x$ is free, while $y$ is bound.

\begin{mathpar}
  \inferrule* [lab=monoidal-laws] {} { P|Q \equiv Q|P \and P|0 \equiv P \and P|(Q|R) \equiv (P|Q)|R }
\end{mathpar}

\begin{mathpar}
  \inferrule* [lab=alpha-equivalence] {} { (x)P \equiv (y)P\{y/x\} \and y \not\in \freenames{P} }
\end{mathpar}

\begin{definition}
Then two processes, $P,Q$, are alpha-equivalent if $P = Q\{\vec{y}/\vec{x}\}$ for
some $\vec{x} \in \boundnames{Q},\vec{y} \in \boundnames{P}$, where $Q\{\vec{y}/\vec{x}\}$
denotes the capture-avoiding substitution of $\vec{y}$ for $\vec{x}$ in $Q$.
\end{definition}

\begin{definition}
  The {\em structural congruence} \cite{SangiorgiWalker} , $\equiv$,
  between processes is the least congruence containing
  alpha-equivalence, satisfying the abelian monoid laws
  (associativity, commutativity and $\pzero$ as identity) for parallel
  composition $|$ and for summation $+$.
\end{definition}

\subsection{Name equivalence}

We take name equivalence, written $\nameeq$, to be the smallest
equivalence relation generated by the following rules.

\begin{mathpar}
\inferrule*[lab=Quote-drop]
{ }
{ \quotep{@{x}} \nameeq x }

\inferrule*[lab=Struct-equiv]
{ P \scong Q }
{ \quotep{P} \nameeq \quotep{Q} }
\end{mathpar}

The astute reader will have noticed that the mutual recursion of names
and processes imposes a mutual recursion on alpha-equivalence and
structural equivalence via name-equivalence. Fortunately, all of this
works out pleasantly and we may calculate in the natural way, free of
concern. The reader interested in the details is referred to the
appendix \ref{appendix:rho_details}.

\subsection{Substitution}

We use $\Proc$ for the set of processes, $\QProc$ for the set of
names, and $\id{\{}\vec{y} / \vec{x} \id{\}}$ to denote partial maps,
$s : \QProc \rightarrow \QProc$. A map, $s$ lifts, uniquely, to a map
on process terms, $\widehat{s} : \Proc \rightarrow \Proc$ by the
following equations.

\begin{mathpar}
  (0) \psubstp{Q}{P} := 0 \\
  (R \juxtap S) \psubstp{Q}{P}
  :=    
  (R)\psubstp{Q}{P} \juxtap (S) \psubstp{Q}{P} \\
  (x?(y).R) \psubstp{Q}{P}    
  :=    
  (x)\substp{Q}{P} (z)\concat( (R \psubstn{z}{y}) \psubstp{Q}{P} ) \\
  (\lift{x}{R}) \psubstp{Q}{P}  
  :=
  \lift{(x)\substp{Q}{P}}{ R \psubstp{Q}{P} } \\
%   (\dropn{x})  \psubstp{Q}{P}       
%   := 
%   \left\{ 
%     \begin{array}{ccc} 
%       \dropn{\quotep{Q}} & & x \nameeq \quotep{P} \\
%       \dropn{x} & & otherwise \\
%     \end{array}
%   \right. 
  (\dropn{x})  \psubstp{Q}{P}       
  := 
  \left\{ 
    \begin{array}{ccc} 
      Q & & x \nameeq \quotep{P} \\
      \dropn{x} & & otherwise \\
    \end{array}
  \right.
\end{mathpar}
 

where

\begin{eqnarray}
  (x)\id{\{} \lpquote Q \rpquote / \lpquote P \rpquote \id{\}}            = 
  \left\{ 
    \begin{array}{ccc}
      \lpquote Q \rpquote & & x \nameeq \lpquote P \rpquote \\
      x & & otherwise \\
    \end{array}
  \right. \nonumber
\end{eqnarray}

and $z$ is chosen distinct from $\quotep{P}$, $\quotep{Q}$, the free
names in $Q$, and all the names in $R$. Our $\alpha$-equivalence will
be built in the standard way from this substitution.

\begin{remark}\label{rem:no_self_referential_names}
  One consequence of these definitions is that $\forall P. \quotep{P}
  \not\in \freenames{P}$.
\end{remark}

\subsection{ Dynamic quote: an example }

Anticipating something of what's to come, consider applying the
substitution, $\widehat{\id{\{}u / z \id{\}}}$, to the following pair
of processes, $\lift{w}{y!(z)}$ and $w[ \lpquote y!(z) \rpquote ]$.

\begin{eqnarray}
	\lift{w}{y!(z)}\widehat{\id{\{}u / z \id{\}}}
		& = &
		\lift{w}{y!(u)} \nonumber\\
	w[ \lpquote y!(z) \rpquote ] \widehat{ \id{\{}u / z \id{\}} }
		& = &
		w[ \lpquote y!(z) \rpquote ] \nonumber
\end{eqnarray}

Because the body of the process between quotes is impervious to
substitution, we get radically different answers. In fact, by
examining the first process in an input context,
e.g. $x?(z).\lift{w}{y!(z)}$, we see that the process under the lift
operator may be shaped by prefixed inputs binding a name inside it. In
this sense, the lift operator will be seen as a way to dynamically
construct processes before reifying them as names.

Finally equipped with these standard features we can present the
dynamics of the calculus.

\subsubsection{Operational semantics} 

Finally, we introduce the computational dynamics. What marks these
algebras as distinct from other more traditionally studied algebraic
structures, e.g. vector spaces or polynomial rings, is the manner in
which dynamics is captured. In traditional structures, dynamics is typically
expressed through morphisms between such structures, as in linear maps
between vector spaces or morphisms between rings. In algebras
associated with the semantics of computation, the dynamics is
expressed as part of the algebraic structure itself, through a
reduction reduction relation typically denoted by $\red$. Below, we
give a recursive presentation of this relation for the calculus used
in the encoding.

$\red \subseteq \pi \times \pi$
$\red : \pi \to \mathcal{P}(\pi)$

\begin{mathpar}
  \inferrule* [lab=Comm] { \textsf{match}( x_{src}, x_{trgt} ) } { x_{trgt}?(y)P \; | \; x_{src}!\langle {Q} \rangle \red P\{\quotep{Q}/y}\} }
  \and \\
  \inferrule* [lab=Par] {{P} \red {P}'} {{{P} | {Q}} \red {{P}' | {Q}}}
  \and
  \inferrule* [lab=Equiv]{{{P} \scong {P}'} \andalso {{P}' \red {Q}'} \andalso {{Q}' \scong {Q}}}{{P} \red {Q}}
\end{mathpar}

\begin{eqnarray*}
  match_{\equiv} (\quotep{P},\quotep{Q}) & := & P \equiv Q \\
  match_{\dagger}(\quotep{P},\quotep{Q}) & := & \forall R. P|Q \red^{*} R => R \red^{*} 0 \\
  match_{K}(\quotep{P},\quotep{Q}) & := & K \mbox{ for some context } K
\end{eqnarray*}

$u?(x)P | u!\langle Q \rangle \red P\{\quotep{Q}/x\}$

%We write $\wred$ for $\red^*$, and $P\red$ if $\exists Q $ such that $ P \red Q$.
We write $P\red$ if $\exists Q $ such that $ P \red Q$ and $P\not\red$, otherwise.

\section{Replication}

As mentioned before, it is known that replication (and hence
recursion) can be implemented in a higher-order process algebra
\cite{SangiorgiWalker}. As our first example of calculation with the
machinery thus far presented we give the construction explicitly in
the {\rhoc}.

\begin{eqnarray}
	D_{x} & := & \prefix{x}{y}{(\binpar{\outputp{x}{y}}{@{y}})} \nonumber\\
	\bangp_{x}{P} & := & \binpar{{x}!\langle{\binpar{D_{x}}{P}}\rangle}{D_{x}} \nonumber
\end{eqnarray}

\begin{eqnarray}
	\bangp_{x}{P} & & \nonumber\\
	=
	& {x}!\langle{(\prefix{x}{y}{(\outputp{x}{y} | @{y})) | P}}\rangle 
	      | \prefix{x}{y}{(\outputp{x}{y} | @{y})} & \nonumber\\
	\red
	& (\outputp{x}{y} | @{y})\substn{\quotep{(\prefix{x}{y}{(@{y} | \outputp{x}{y})) | P}}}{y} & \nonumber\\
	=
	& \outputp{x}{\quotep{(\prefix{x}{y}{(\outputp{x}{y} | @{y})) | P}}}
	  | {(\prefix{x}{y}{(\outputp{x}{y} | @{y})) | P}} & \nonumber\\
	\red
	& \ldots & \nonumber\\
	\red^*
	& P | P | \ldots & \nonumber
\end{eqnarray}

Of course, this encoding, as an implementation, runs away, unfolding
$\bangp{P}$ eagerly. A lazier and more implementable replication
operator, restricted to input-guarded processes, may be obtained as follows.

\begin{eqnarray}
\bangp{\prefix{u}{v}{P}} 
	:= 
	\binpar{\lift{x}{\prefix{u}{v}{(\binpar{D(x)}{P})}}}{D(x)} \nonumber
\end{eqnarray}

\begin{remark}
  Note that the lazier definition still does not deal with summation
  or mixed summation (i.e. sums over input and output). The reader is
  invited to construct definitions of replication that deal with these
  features. 

  Further, the definitions are parameterized in a name, $x$. Can you,
  gentle reader, make a definition that eliminates this parameter and
  guarantees no accidental interaction between the replication
  machinery and the process being replicated -- i.e. no accidental
  sharing of names used by the process to get its work done and the
  name(s) used by the replication to effect copying. This latter
  revision of the definition of replication is crucial to obtaining
  the expected identity $!!P \sim !P$.
\end{remark}

\begin{remark}\label{rem:paradoxical_combinator}
  The reader familiar with the lambda calculus will have noticed the
  similarity between $D$ and the paradoxical combinator.

  [Ed. note: the existence of this seems to suggest we have to be more
  restrictive on the set of processes and names we admit if we are to
  support no-cloning.]
\end{remark}

\subsubsection{Bisimulation}

The computational dynamics gives rise to another kind of equivalence,
the equivalence of computational behavior. As previously mentioned
this is typically captured \emph{via} some form of bisimulation.

% The notion we use in this paper is weak barbed bisimulation
% \cite{milner91polyadicpi}.

The notion we use in this paper is derived from weak barbed
bisimulation \cite{milner91polyadicpi}. 

\begin{definition}
An \emph{observation relation}, $\downarrow_{\mathcal N}$, over a set
of names, $\mathcal N$, is the smallest relation satisfying the rules
below.

\infrule[Out-barb]{y \in {\mathcal N}, \; x \nameeq y}
		  {\outputp{x}{v} \downarrow_{\mathcal N} x}
\infrule[Par-barb]{\mbox{$P\downarrow_{\mathcal N} x$ or $Q\downarrow_{\mathcal N} x$}}
		  {\binpar{P}{Q} \downarrow_{\mathcal N} x}

We write $P \Downarrow_{\mathcal N} x$ if there is $Q$ such that 
$P \wred Q$ and $Q \downarrow_{\mathcal N} x$.
\end{definition}

\begin{definition}
%\label{def.bbisim}
An  ${\mathcal N}$-\emph{barbed bisimulation} over a set of names, ${\mathcal N}$, is a symmetric binary relation 
${\mathcal S}_{\mathcal N}$ between agents such that $P\rel{S}_{\mathcal N}Q$ implies:
\begin{enumerate}
\item If $P \red P'$ then $Q \wred Q'$ and $P'\rel{S}_{\mathcal N} Q'$.
\item If $P\downarrow_{\mathcal N} x$, then $Q\Downarrow_{\mathcal N} x$.
\end{enumerate}
$P$ is ${\mathcal N}$-barbed bisimilar to $Q$, written
$P \wbbisim_{\mathcal N} Q$, if $P \rel{S}_{\mathcal N} Q$ for some ${\mathcal N}$-barbed bisimulation ${\mathcal S}_{\mathcal N}$.
\end{definition}

$\mathcal{R} \subseteq \pi \times \pi$

$P \mathcal{R} Q => \forall P'. P \red P' \Rightarrow \exists Q'. Q \red Q', P' \mathcal{R} Q'$

$P \vdash x \Rightarrow Q \vdash x$

\begin{mathpar}
  \inferrule*[lab=Out-barb]{x \nameeq y}{{y}!\langle{Q}\rangle \vdash x}
  \and
  \inferrule*[lab=Par-barb]{\mbox{$P\vdash x$ or $Q\vdash x$}}{\binpar{P}{Q} \vdash x}
\end{mathpar}

\subsubsection{Contexts}

One of the principle advantages of computational calculi like the
$\pi$-calculus is a well-defined notion of context,
contextual-equivalence and a correlation between
contextual-equivalence and notions of bisimulation. The notion of
context allows the decomposition of a process into (sub-)process and
its syntactic environment, its context. Thus, a context may be
thought of as a process with a ``hole'' (written $\Box$) in it. The
application of a context $M$ to a process $P$, written $M[P]$, is
tantamount to filling the hole in $M$ with $P$. In this paper we do
not need the full weight of this theory, but do make use of the notion
of context in the proof the main theorem. 

\begin{mathpar}
  \inferrule* [lab=summation] {} {{M_{M},M_{N}} \bc \Box \;|\; x.M_{A} \;|\; M_{M}+M_{N}}
  \and
  \inferrule* [lab=agent] {} {{M_{A}} \bc (\vec{x})M_{P} \;| \; \clift{P_0,\ldots,M_{P},\ldots,P_N}}
  \and \\
  \inferrule* [lab=process] {} {{M_{P}} \bc M_{N} \;| \;P|M_{P} }
\end{mathpar} 

\begin{mathpar}
  \inferrule* [lab=sychronization] {} {M_{N} \bc \Box \;|\; x?M_{F} \;|\; x!M_{C}}
  \and
  \inferrule* [lab=abstraction] {} {{M_{F}} \bc (x)M_{P} }
  \and
  \inferrule* [lab=concretion] {} {{M_{C}} \bc \langle M_{P} \rangle }
  \and \\
  \inferrule* [lab=process] {} {{M_{P}} \bc M_{N} \;| \;P|M_{P} }
\end{mathpar}

\begin{definition}[contextual application] Given a context $M$, and
  process $P$, we define the \emph{contextual application}, $M[P] :=
  M\{P/\Box\}$. That is, the contextual application of M to P is the
  substitution of $P$ for $\Box$ in $M$.
\end{definition}

$\meaningof{-} : L \to \mathcal{P}(\pi)$

\begin{mathpar}
  \inferrule* [lab=collection] {} {\meaningof{true} = \pi, \and \meaningof{~E} = \pi \setminus \meaningof{E}, \and \meaningof{E_{1} \& E_{2}} = \meaningof{E_{1}} \cap \meaningof{E_{2}}}
\end{mathpar}

\begin{mathpar}
  \inferrule* [lab=structure] {} {\meaningof{0} = \{ P \in \pi | P \equiv 0 \}, \and \\ \meaningof{E_1 | E_2} = \{ P \in \pi | P \equiv P_{1} | P_{2}, P_{1} \in \meaningof{E_{1}}, P_{2} \in \meaningof{E_2}\} }
\end{mathpar}

\begin{mathpar}
 \inferrule* [lab=behavior] {} {\meaningof{\langle a?b \rangle E} = \{ P \in \pi | P \equiv Q | u?(y)P', \\ \and \\\\ \and \\ \;\;\; u \in \meaningof{a}, \forall z.P'\{z/y\} \in \meaningof{E\{z/b\}}\}, \and \\ \meaningof{a!E} = \{ P \in \pi | P \equiv Q | x!\langle P' \rangle, x \in \meaningof{a} P' \in \meaningof{E}\} }
\end{mathpar}

\begin{mathpar}
 \inferrule* [lab=nominal] {} {\meaningof{\quotep{E}} = \{ \quotep{P} \in \quotep{\pi} | P \in \meaningof{E} \}, \and \meaningof{\quotep{P}} = \{ \quotep{Q} \in \quotep{\pi} | P \equiv Q \} \and \\ \meaningof{@\quotep{E}} = \{ P \in \pi | P \equiv @x, x \in \meaningof{E} \}}
\end{mathpar}

\begin{eqnarray*}
  \\
  \meaningof{-} : TS \to ST
\end{eqnarray*}

\begin{eqnarray*}
  \\
  L : TS \to ST
\end{eqnarray*}

\begin{eqnarray*}
  \\
  P \models E \iff P \in \meaningof{E}
\end{eqnarray*}

\begin{eqnarray*}
  P \approx_{L} Q \iff \forall E \in L. P \models E \iff Q \models E
\end{eqnarray*}

\begin{eqnarray*}
  P \approx_{K} Q
\end{eqnarray*}

\begin{eqnarray*}
  P \approx Q
\end{eqnarray*}

$\approx_{K} = \approx = \approx_{L}$

\subsubsection{Contextual duality}

Note that contexts extend the quotation operation to a family of
operations from processes to names. Given a context, $M$, we can
define a \emph{nominal context}, $\quotep{M}$ by $\quotep{M}[P] :=
\quotep{M[P]}$. To foreshadow what is to come we observe that these
operations enjoy a duality with processes very much like the duality
between vectors and maps from vectors to scalars.

Further, because the calculus is essentially higher-order, we have a
correspondence between contexts and processes. More specifically,
given a name $x$ and a context $M$ we can construct $M^{*}_{x}$ such
that 

\begin{mathpar}
  M^{*}_{x} | \lift{x}{P} \red M[P]
\end{mathpar}

namely,

\begin{mathpar}
  M^{*}_{x} := x?(u).M[\dropn{u}]
\end{mathpar}

The dependence of $M^{*}_{x}$ on a name makes it an abstraction, 

\begin{mathpar}
  M^{*} := (x)x?(u).M[\dropn{u}]
\end{mathpar}

\subsection{Additional notation}

It will sometimes be convenient to denote the process a name
quotes. We already have the notation $x = \quotep{P}$, but it will be
convenient to introduce an alternate notation, $\procn{x}$, when we
want to emphasize the connection to the use of the name. Note that, by
virtue of name equivalence, $\quotep{\procn{x}} \nameeq x$; so, the
notation is consistent with previous definitions.

Further, because names have structure it is possible to effect
substitutions on the basis of that structure. This means we need to
upgrade our notation for substitutions, which we accomplish by
adapting comprehension notation. Thus,

\begin{mathpar}
  P\{ y / x : x \in S \}
\end{mathpar}

is interpreted to mean the process derived from P by replacing (in a
capture-avoiding manner) each occurrence of $x$ in $S$ by $y$. For example,

\begin{mathpar}
  P\{ \quotep{\procn{x}|\procn{x}} / x : x \in \freenames{P} \}
\end{mathpar}

will replace each (occurrence) of a free name $x$ in $P$ by
$\quotep{\procn{x}|\procn{x}}$.

Also, we will avail ourselves of the notation $x^{L}$ and $x^{R}$ to
denote injections of a name into disjoint copies of the name
space. There are numerous ways to accomplish this. One example can be
found in \cite{MeredithR05}. This notation overloads to vectors of
names: $\vec{x}^{\pi} := (x_{i}^{\pi} \; : \; 0 \leq i < |\vec{x}| )$ where $\pi \in \{L,R\}$.

We also use $P^{\Box} := P|\Box$.

In \cite{MeredithR05} an interpretation of the new operator is
given. It turns out that there are several possible interpretations
all enjoying the requisite algebraic properties of the operator (see
\cite{milner91polyadicpi}). We will therefore make liberal use of
$(\nu\; \vec{x})P$.

% subsection the_syntax_and_semantics_of_the_notation_system (end)   

\input{qm2pi.qmops} 

\input{qm2pi.sterngerlach} 

\input{qm2pi.metric} 

% section concurrent_process_calculi (end)

%\input{qm2pi.proofsketch}

% section proof sketch (end)

%\input{qm2pi.slviaknots} 

% section spatial logic via knots (end)

\input{qm2pi.conclusion}

% section conclusion (end)

%\input{qm2pi.dtcodes} 

% section wiring algorithm (end)

\input{qm2pi.ack} 

% section acknowledgments (end)

\newpage


\bibliographystyle{plain}   
\bibliography{../../biblios/main.bib}

\input{qm2pi.rhodetails}

\end{document}

 

%\ifpdf
%\usepackage[pdftex]{graphicx}
%\else
%\usepackage{graphicx}
%\fi

 % \ifpdf
%  \usepackage{pdfsync}
%  \if


%\title{Brief Article}
%\author{David F. Snyder}
%\author{L.G. Meredith}

%\address{Dept. of Math., Texas State University--San Marcos, San Marcos, TX 78666}
       
\pagestyle{empty}


\begin{document}

\lstset{language=[Objective]Caml,frame=shadowbox}

\documentclass[12pt]{llncs}
%\documentclass{jktr}

\usepackage[pdftex]{hyperref}                   
\usepackage {listings}
\usepackage {mathpartir}
\usepackage{bcprules}
%\usepackage{listings}
                       
\usepackage{graphicx} 
%\usepackage[margins=2.5cm,nohead,nofoot]{geometry}
%\usepackage{geometry}
\usepackage{amsfonts}
\usepackage{amstext}
\usepackage{latexsym}
\usepackage{amssymb}
\usepackage{color}


%\include{myPreamble}
\include{qm2pi.local} 

%\ifpdf
%\usepackage[pdftex]{graphicx}
%\else
%\usepackage{graphicx}
%\fi

 % \ifpdf
%  \usepackage{pdfsync}
%  \if


%\title{Brief Article}
%\author{David F. Snyder}
%\author{L.G. Meredith}

%\address{Dept. of Math., Texas State University--San Marcos, San Marcos, TX 78666}
       
\pagestyle{empty}


\begin{document}

\lstset{language=[Objective]Caml,frame=shadowbox}

\input{qm2pi.front}

% section front matter (end)

\input{qm2pi.intro} 
 
% section introduction (end)

% \input{qm2pi.knotations} 

% section notation (end)

\input{qm2pi.process.calculi} 

% section concurrent_process_calculi_and_spatial_logics_ (end)
    
%\input{qm2pi.knots2pi} 

%\input{qm2pi.trefoil} 

%\input{qm2pi.mainthm} 

% subsection basic_interpretation (end)

%\input{qm2pi.rho.presentation} 
\subsection{The syntax and semantics of the notation system}\label{sub:the_syntax_and_semantics_of_the_notation_system} % (fold)

We now summarize a technical presentation of the calculus that
embodies our theory of dynamics. The typical presentation of such a
calculus follows the style of giving generators and relations on
them. The grammar, below, describing term constructors, freely
generates the set of processes, $\Proc$. This set is then quotiented
by a relation known as structural congruence and it is over this set
that the notion of dynamics is expressed. This presentation is
essentially that of \cite{MeredithR05} with the addition of
polyadicity and summation. For readability we have relegated some of
the technical subtleties to an appendix.

\subsubsection{Process grammar}\label{subsub:process_grammar}

\begin{mathpar}
  \inferrule* [lab=synchronization] {} {{M} \bc \pzero \;|\; x?F \;|\; x!C }
  \and
  \inferrule* [lab=abstraction] {} {{F} \bc (x)P}
  \and
  \inferrule* [lab=concretion] {} {{C} \bc \langle Q \rangle}
  \and
  \inferrule* [lab=process] {} {{P,Q} \bc M \;| \;P|Q \;|\; @{x}}
  \and
  \inferrule* [lab=name] {} {{x} \bc \quotep{P}}
\end{mathpar} 

Note that $\vec{x}$ (resp. $\vec{P}$) denotes a vector of names
(resp. processes) of length $|\vec{x}|$ (resp. $|\vec{P}|$). We adopt
the following useful abbreviations.

\begin{mathpar}
   x?(\vec{y}).P := x.(\vec{y})P \and  x\clift{\vec{P}} := x.\clift{\vec{P}}
   \and x!(y) := \lift{x}{\dropn{y}}
   \and \Pi_{i=0}^{n-1}P_i := P_0 | \ldots | P_{n-1}
\end{mathpar}

\subsubsection{Structural congruence}

\paragraph{Free and bound names and alpha-equivalence.} At the
core of structural equivalence is alpha-equivalence which identifies
process that are the same up to a change of variable. Formally, we
recognize the distinction between free and bound names. The free names
of a process, $\freenames{P}$, may be calculated recursively as
follows:

\begin{mathpar}
\freenames{\pzero} := \emptyset
  \and \\
  \freenames{x?(y).P} := \{ x \} \cup (\freenames{P} \setminus \{ y \})
  \and 
  \freenames{x!\langle P \rangle} := \{ x \} \cup \{ P \} 
  \and \\
  \freenames{P|Q} := \freenames{P} \cup \freenames{Q}
  \and \\
  \freenames{@{x}} := \{ x \}
\end{mathpar}

$\pi$
$\quotep{\pi}$

$\freenames{-} : \pi \to \mathcal{P}(\quotep{\pi})$

\begin{eqnarray*}
  \freenames{\pzero} & := & \emptyset \\
  \freenames{x?(y).P} & := & \{ x \} \cup (\freenames{P} \setminus \{ y \}) \\
  \freenames{x!\langle P \rangle} & := & \{ x \} \cup \{ P \} \\
  \freenames{P|Q} & := & \freenames{P} \cup \freenames{Q} \\
  \freenames{\dropn{x}} & := & \{ x \}
\end{eqnarray*}

The bound names of a process, $\boundnames{P}$, are those names occurring in $P$
that are not free. For example, in $x?(y).0$, the name $x$ is free, while $y$ is bound.

\begin{mathpar}
  \inferrule* [lab=monoidal-laws] {} { P|Q \equiv Q|P \and P|0 \equiv P \and P|(Q|R) \equiv (P|Q)|R }
\end{mathpar}

\begin{mathpar}
  \inferrule* [lab=alpha-equivalence] {} { (x)P \equiv (y)P\{y/x\} \and y \not\in \freenames{P} }
\end{mathpar}

\begin{definition}
Then two processes, $P,Q$, are alpha-equivalent if $P = Q\{\vec{y}/\vec{x}\}$ for
some $\vec{x} \in \boundnames{Q},\vec{y} \in \boundnames{P}$, where $Q\{\vec{y}/\vec{x}\}$
denotes the capture-avoiding substitution of $\vec{y}$ for $\vec{x}$ in $Q$.
\end{definition}

\begin{definition}
  The {\em structural congruence} \cite{SangiorgiWalker} , $\equiv$,
  between processes is the least congruence containing
  alpha-equivalence, satisfying the abelian monoid laws
  (associativity, commutativity and $\pzero$ as identity) for parallel
  composition $|$ and for summation $+$.
\end{definition}

\subsection{Name equivalence}

We take name equivalence, written $\nameeq$, to be the smallest
equivalence relation generated by the following rules.

\begin{mathpar}
\inferrule*[lab=Quote-drop]
{ }
{ \quotep{@{x}} \nameeq x }

\inferrule*[lab=Struct-equiv]
{ P \scong Q }
{ \quotep{P} \nameeq \quotep{Q} }
\end{mathpar}

The astute reader will have noticed that the mutual recursion of names
and processes imposes a mutual recursion on alpha-equivalence and
structural equivalence via name-equivalence. Fortunately, all of this
works out pleasantly and we may calculate in the natural way, free of
concern. The reader interested in the details is referred to the
appendix \ref{appendix:rho_details}.

\subsection{Substitution}

We use $\Proc$ for the set of processes, $\QProc$ for the set of
names, and $\id{\{}\vec{y} / \vec{x} \id{\}}$ to denote partial maps,
$s : \QProc \rightarrow \QProc$. A map, $s$ lifts, uniquely, to a map
on process terms, $\widehat{s} : \Proc \rightarrow \Proc$ by the
following equations.

\begin{mathpar}
  (0) \psubstp{Q}{P} := 0 \\
  (R \juxtap S) \psubstp{Q}{P}
  :=    
  (R)\psubstp{Q}{P} \juxtap (S) \psubstp{Q}{P} \\
  (x?(y).R) \psubstp{Q}{P}    
  :=    
  (x)\substp{Q}{P} (z)\concat( (R \psubstn{z}{y}) \psubstp{Q}{P} ) \\
  (\lift{x}{R}) \psubstp{Q}{P}  
  :=
  \lift{(x)\substp{Q}{P}}{ R \psubstp{Q}{P} } \\
%   (\dropn{x})  \psubstp{Q}{P}       
%   := 
%   \left\{ 
%     \begin{array}{ccc} 
%       \dropn{\quotep{Q}} & & x \nameeq \quotep{P} \\
%       \dropn{x} & & otherwise \\
%     \end{array}
%   \right. 
  (\dropn{x})  \psubstp{Q}{P}       
  := 
  \left\{ 
    \begin{array}{ccc} 
      Q & & x \nameeq \quotep{P} \\
      \dropn{x} & & otherwise \\
    \end{array}
  \right.
\end{mathpar}
 

where

\begin{eqnarray}
  (x)\id{\{} \lpquote Q \rpquote / \lpquote P \rpquote \id{\}}            = 
  \left\{ 
    \begin{array}{ccc}
      \lpquote Q \rpquote & & x \nameeq \lpquote P \rpquote \\
      x & & otherwise \\
    \end{array}
  \right. \nonumber
\end{eqnarray}

and $z$ is chosen distinct from $\quotep{P}$, $\quotep{Q}$, the free
names in $Q$, and all the names in $R$. Our $\alpha$-equivalence will
be built in the standard way from this substitution.

\begin{remark}\label{rem:no_self_referential_names}
  One consequence of these definitions is that $\forall P. \quotep{P}
  \not\in \freenames{P}$.
\end{remark}

\subsection{ Dynamic quote: an example }

Anticipating something of what's to come, consider applying the
substitution, $\widehat{\id{\{}u / z \id{\}}}$, to the following pair
of processes, $\lift{w}{y!(z)}$ and $w[ \lpquote y!(z) \rpquote ]$.

\begin{eqnarray}
	\lift{w}{y!(z)}\widehat{\id{\{}u / z \id{\}}}
		& = &
		\lift{w}{y!(u)} \nonumber\\
	w[ \lpquote y!(z) \rpquote ] \widehat{ \id{\{}u / z \id{\}} }
		& = &
		w[ \lpquote y!(z) \rpquote ] \nonumber
\end{eqnarray}

Because the body of the process between quotes is impervious to
substitution, we get radically different answers. In fact, by
examining the first process in an input context,
e.g. $x?(z).\lift{w}{y!(z)}$, we see that the process under the lift
operator may be shaped by prefixed inputs binding a name inside it. In
this sense, the lift operator will be seen as a way to dynamically
construct processes before reifying them as names.

Finally equipped with these standard features we can present the
dynamics of the calculus.

\subsubsection{Operational semantics} 

Finally, we introduce the computational dynamics. What marks these
algebras as distinct from other more traditionally studied algebraic
structures, e.g. vector spaces or polynomial rings, is the manner in
which dynamics is captured. In traditional structures, dynamics is typically
expressed through morphisms between such structures, as in linear maps
between vector spaces or morphisms between rings. In algebras
associated with the semantics of computation, the dynamics is
expressed as part of the algebraic structure itself, through a
reduction reduction relation typically denoted by $\red$. Below, we
give a recursive presentation of this relation for the calculus used
in the encoding.

$\red \subseteq \pi \times \pi$
$\red : \pi \to \mathcal{P}(\pi)$

\begin{mathpar}
  \inferrule* [lab=Comm] { \textsf{match}( x_{src}, x_{trgt} ) } { x_{trgt}?(y)P \; | \; x_{src}!\langle {Q} \rangle \red P\{\quotep{Q}/y}\} }
  \and \\
  \inferrule* [lab=Par] {{P} \red {P}'} {{{P} | {Q}} \red {{P}' | {Q}}}
  \and
  \inferrule* [lab=Equiv]{{{P} \scong {P}'} \andalso {{P}' \red {Q}'} \andalso {{Q}' \scong {Q}}}{{P} \red {Q}}
\end{mathpar}

\begin{eqnarray*}
  match_{\equiv} (\quotep{P},\quotep{Q}) & := & P \equiv Q \\
  match_{\dagger}(\quotep{P},\quotep{Q}) & := & \forall R. P|Q \red^{*} R => R \red^{*} 0 \\
  match_{K}(\quotep{P},\quotep{Q}) & := & K \mbox{ for some context } K
\end{eqnarray*}

$u?(x)P | u!\langle Q \rangle \red P\{\quotep{Q}/x\}$

%We write $\wred$ for $\red^*$, and $P\red$ if $\exists Q $ such that $ P \red Q$.
We write $P\red$ if $\exists Q $ such that $ P \red Q$ and $P\not\red$, otherwise.

\section{Replication}

As mentioned before, it is known that replication (and hence
recursion) can be implemented in a higher-order process algebra
\cite{SangiorgiWalker}. As our first example of calculation with the
machinery thus far presented we give the construction explicitly in
the {\rhoc}.

\begin{eqnarray}
	D_{x} & := & \prefix{x}{y}{(\binpar{\outputp{x}{y}}{@{y}})} \nonumber\\
	\bangp_{x}{P} & := & \binpar{{x}!\langle{\binpar{D_{x}}{P}}\rangle}{D_{x}} \nonumber
\end{eqnarray}

\begin{eqnarray}
	\bangp_{x}{P} & & \nonumber\\
	=
	& {x}!\langle{(\prefix{x}{y}{(\outputp{x}{y} | @{y})) | P}}\rangle 
	      | \prefix{x}{y}{(\outputp{x}{y} | @{y})} & \nonumber\\
	\red
	& (\outputp{x}{y} | @{y})\substn{\quotep{(\prefix{x}{y}{(@{y} | \outputp{x}{y})) | P}}}{y} & \nonumber\\
	=
	& \outputp{x}{\quotep{(\prefix{x}{y}{(\outputp{x}{y} | @{y})) | P}}}
	  | {(\prefix{x}{y}{(\outputp{x}{y} | @{y})) | P}} & \nonumber\\
	\red
	& \ldots & \nonumber\\
	\red^*
	& P | P | \ldots & \nonumber
\end{eqnarray}

Of course, this encoding, as an implementation, runs away, unfolding
$\bangp{P}$ eagerly. A lazier and more implementable replication
operator, restricted to input-guarded processes, may be obtained as follows.

\begin{eqnarray}
\bangp{\prefix{u}{v}{P}} 
	:= 
	\binpar{\lift{x}{\prefix{u}{v}{(\binpar{D(x)}{P})}}}{D(x)} \nonumber
\end{eqnarray}

\begin{remark}
  Note that the lazier definition still does not deal with summation
  or mixed summation (i.e. sums over input and output). The reader is
  invited to construct definitions of replication that deal with these
  features. 

  Further, the definitions are parameterized in a name, $x$. Can you,
  gentle reader, make a definition that eliminates this parameter and
  guarantees no accidental interaction between the replication
  machinery and the process being replicated -- i.e. no accidental
  sharing of names used by the process to get its work done and the
  name(s) used by the replication to effect copying. This latter
  revision of the definition of replication is crucial to obtaining
  the expected identity $!!P \sim !P$.
\end{remark}

\begin{remark}\label{rem:paradoxical_combinator}
  The reader familiar with the lambda calculus will have noticed the
  similarity between $D$ and the paradoxical combinator.

  [Ed. note: the existence of this seems to suggest we have to be more
  restrictive on the set of processes and names we admit if we are to
  support no-cloning.]
\end{remark}

\subsubsection{Bisimulation}

The computational dynamics gives rise to another kind of equivalence,
the equivalence of computational behavior. As previously mentioned
this is typically captured \emph{via} some form of bisimulation.

% The notion we use in this paper is weak barbed bisimulation
% \cite{milner91polyadicpi}.

The notion we use in this paper is derived from weak barbed
bisimulation \cite{milner91polyadicpi}. 

\begin{definition}
An \emph{observation relation}, $\downarrow_{\mathcal N}$, over a set
of names, $\mathcal N$, is the smallest relation satisfying the rules
below.

\infrule[Out-barb]{y \in {\mathcal N}, \; x \nameeq y}
		  {\outputp{x}{v} \downarrow_{\mathcal N} x}
\infrule[Par-barb]{\mbox{$P\downarrow_{\mathcal N} x$ or $Q\downarrow_{\mathcal N} x$}}
		  {\binpar{P}{Q} \downarrow_{\mathcal N} x}

We write $P \Downarrow_{\mathcal N} x$ if there is $Q$ such that 
$P \wred Q$ and $Q \downarrow_{\mathcal N} x$.
\end{definition}

\begin{definition}
%\label{def.bbisim}
An  ${\mathcal N}$-\emph{barbed bisimulation} over a set of names, ${\mathcal N}$, is a symmetric binary relation 
${\mathcal S}_{\mathcal N}$ between agents such that $P\rel{S}_{\mathcal N}Q$ implies:
\begin{enumerate}
\item If $P \red P'$ then $Q \wred Q'$ and $P'\rel{S}_{\mathcal N} Q'$.
\item If $P\downarrow_{\mathcal N} x$, then $Q\Downarrow_{\mathcal N} x$.
\end{enumerate}
$P$ is ${\mathcal N}$-barbed bisimilar to $Q$, written
$P \wbbisim_{\mathcal N} Q$, if $P \rel{S}_{\mathcal N} Q$ for some ${\mathcal N}$-barbed bisimulation ${\mathcal S}_{\mathcal N}$.
\end{definition}

$\mathcal{R} \subseteq \pi \times \pi$

$P \mathcal{R} Q => \forall P'. P \red P' \Rightarrow \exists Q'. Q \red Q', P' \mathcal{R} Q'$

$P \vdash x \Rightarrow Q \vdash x$

\begin{mathpar}
  \inferrule*[lab=Out-barb]{x \nameeq y}{{y}!\langle{Q}\rangle \vdash x}
  \and
  \inferrule*[lab=Par-barb]{\mbox{$P\vdash x$ or $Q\vdash x$}}{\binpar{P}{Q} \vdash x}
\end{mathpar}

\subsubsection{Contexts}

One of the principle advantages of computational calculi like the
$\pi$-calculus is a well-defined notion of context,
contextual-equivalence and a correlation between
contextual-equivalence and notions of bisimulation. The notion of
context allows the decomposition of a process into (sub-)process and
its syntactic environment, its context. Thus, a context may be
thought of as a process with a ``hole'' (written $\Box$) in it. The
application of a context $M$ to a process $P$, written $M[P]$, is
tantamount to filling the hole in $M$ with $P$. In this paper we do
not need the full weight of this theory, but do make use of the notion
of context in the proof the main theorem. 

\begin{mathpar}
  \inferrule* [lab=summation] {} {{M_{M},M_{N}} \bc \Box \;|\; x.M_{A} \;|\; M_{M}+M_{N}}
  \and
  \inferrule* [lab=agent] {} {{M_{A}} \bc (\vec{x})M_{P} \;| \; \clift{P_0,\ldots,M_{P},\ldots,P_N}}
  \and \\
  \inferrule* [lab=process] {} {{M_{P}} \bc M_{N} \;| \;P|M_{P} }
\end{mathpar} 

\begin{mathpar}
  \inferrule* [lab=sychronization] {} {M_{N} \bc \Box \;|\; x?M_{F} \;|\; x!M_{C}}
  \and
  \inferrule* [lab=abstraction] {} {{M_{F}} \bc (x)M_{P} }
  \and
  \inferrule* [lab=concretion] {} {{M_{C}} \bc \langle M_{P} \rangle }
  \and \\
  \inferrule* [lab=process] {} {{M_{P}} \bc M_{N} \;| \;P|M_{P} }
\end{mathpar}

\begin{definition}[contextual application] Given a context $M$, and
  process $P$, we define the \emph{contextual application}, $M[P] :=
  M\{P/\Box\}$. That is, the contextual application of M to P is the
  substitution of $P$ for $\Box$ in $M$.
\end{definition}

$\meaningof{-} : L \to \mathcal{P}(\pi)$

\begin{mathpar}
  \inferrule* [lab=collection] {} {\meaningof{true} = \pi, \and \meaningof{~E} = \pi \setminus \meaningof{E}, \and \meaningof{E_{1} \& E_{2}} = \meaningof{E_{1}} \cap \meaningof{E_{2}}}
\end{mathpar}

\begin{mathpar}
  \inferrule* [lab=structure] {} {\meaningof{0} = \{ P \in \pi | P \equiv 0 \}, \and \\ \meaningof{E_1 | E_2} = \{ P \in \pi | P \equiv P_{1} | P_{2}, P_{1} \in \meaningof{E_{1}}, P_{2} \in \meaningof{E_2}\} }
\end{mathpar}

\begin{mathpar}
 \inferrule* [lab=behavior] {} {\meaningof{\langle a?b \rangle E} = \{ P \in \pi | P \equiv Q | u?(y)P', \\ \and \\\\ \and \\ \;\;\; u \in \meaningof{a}, \forall z.P'\{z/y\} \in \meaningof{E\{z/b\}}\}, \and \\ \meaningof{a!E} = \{ P \in \pi | P \equiv Q | x!\langle P' \rangle, x \in \meaningof{a} P' \in \meaningof{E}\} }
\end{mathpar}

\begin{mathpar}
 \inferrule* [lab=nominal] {} {\meaningof{\quotep{E}} = \{ \quotep{P} \in \quotep{\pi} | P \in \meaningof{E} \}, \and \meaningof{\quotep{P}} = \{ \quotep{Q} \in \quotep{\pi} | P \equiv Q \} \and \\ \meaningof{@\quotep{E}} = \{ P \in \pi | P \equiv @x, x \in \meaningof{E} \}}
\end{mathpar}

\begin{eqnarray*}
  \\
  \meaningof{-} : TS \to ST
\end{eqnarray*}

\begin{eqnarray*}
  \\
  L : TS \to ST
\end{eqnarray*}

\begin{eqnarray*}
  \\
  P \models E \iff P \in \meaningof{E}
\end{eqnarray*}

\begin{eqnarray*}
  P \approx_{L} Q \iff \forall E \in L. P \models E \iff Q \models E
\end{eqnarray*}

\begin{eqnarray*}
  P \approx_{K} Q
\end{eqnarray*}

\begin{eqnarray*}
  P \approx Q
\end{eqnarray*}

$\approx_{K} = \approx = \approx_{L}$

\subsubsection{Contextual duality}

Note that contexts extend the quotation operation to a family of
operations from processes to names. Given a context, $M$, we can
define a \emph{nominal context}, $\quotep{M}$ by $\quotep{M}[P] :=
\quotep{M[P]}$. To foreshadow what is to come we observe that these
operations enjoy a duality with processes very much like the duality
between vectors and maps from vectors to scalars.

Further, because the calculus is essentially higher-order, we have a
correspondence between contexts and processes. More specifically,
given a name $x$ and a context $M$ we can construct $M^{*}_{x}$ such
that 

\begin{mathpar}
  M^{*}_{x} | \lift{x}{P} \red M[P]
\end{mathpar}

namely,

\begin{mathpar}
  M^{*}_{x} := x?(u).M[\dropn{u}]
\end{mathpar}

The dependence of $M^{*}_{x}$ on a name makes it an abstraction, 

\begin{mathpar}
  M^{*} := (x)x?(u).M[\dropn{u}]
\end{mathpar}

\subsection{Additional notation}

It will sometimes be convenient to denote the process a name
quotes. We already have the notation $x = \quotep{P}$, but it will be
convenient to introduce an alternate notation, $\procn{x}$, when we
want to emphasize the connection to the use of the name. Note that, by
virtue of name equivalence, $\quotep{\procn{x}} \nameeq x$; so, the
notation is consistent with previous definitions.

Further, because names have structure it is possible to effect
substitutions on the basis of that structure. This means we need to
upgrade our notation for substitutions, which we accomplish by
adapting comprehension notation. Thus,

\begin{mathpar}
  P\{ y / x : x \in S \}
\end{mathpar}

is interpreted to mean the process derived from P by replacing (in a
capture-avoiding manner) each occurrence of $x$ in $S$ by $y$. For example,

\begin{mathpar}
  P\{ \quotep{\procn{x}|\procn{x}} / x : x \in \freenames{P} \}
\end{mathpar}

will replace each (occurrence) of a free name $x$ in $P$ by
$\quotep{\procn{x}|\procn{x}}$.

Also, we will avail ourselves of the notation $x^{L}$ and $x^{R}$ to
denote injections of a name into disjoint copies of the name
space. There are numerous ways to accomplish this. One example can be
found in \cite{MeredithR05}. This notation overloads to vectors of
names: $\vec{x}^{\pi} := (x_{i}^{\pi} \; : \; 0 \leq i < |\vec{x}| )$ where $\pi \in \{L,R\}$.

We also use $P^{\Box} := P|\Box$.

In \cite{MeredithR05} an interpretation of the new operator is
given. It turns out that there are several possible interpretations
all enjoying the requisite algebraic properties of the operator (see
\cite{milner91polyadicpi}). We will therefore make liberal use of
$(\nu\; \vec{x})P$.

% subsection the_syntax_and_semantics_of_the_notation_system (end)   

\input{qm2pi.qmops} 

\input{qm2pi.sterngerlach} 

\input{qm2pi.metric} 

% section concurrent_process_calculi (end)

%\input{qm2pi.proofsketch}

% section proof sketch (end)

%\input{qm2pi.slviaknots} 

% section spatial logic via knots (end)

\input{qm2pi.conclusion}

% section conclusion (end)

%\input{qm2pi.dtcodes} 

% section wiring algorithm (end)

\input{qm2pi.ack} 

% section acknowledgments (end)

\newpage


\bibliographystyle{plain}   
\bibliography{../../biblios/main.bib}

\input{qm2pi.rhodetails}

\end{document}



% section front matter (end)

\section{Introduction}\label{sec:introduction} % (fold)
In this draft of the material i am going to have to dispense with the
usual writing conventions adopted in papers on these topics. i'm going
to have adopt whatever tone i need at the time i'm writing up the
calculations. Sometimes this may be very conversational; others it may
be the barest mathematical grunts; others still it may be that i have
lifted text from one of my other papers because the exposition of some
point was better said there. i hope that my readers are not unduly put
out by this decision. i'm not doing this to flout convention or be
rebellious. i find these calculations very technically challenging. To
keep everything going technically, something has to give; i have to
let go of some cognitive burden. So, the academic writing style --
with all of its trade-offs in terms of facilitating technical
communication -- is what i'm letting go of. Perhaps subsequent drafts
can be tightened and polished, but for now, i'm going to speak as if
we were sitting together in a coffee shop with a laptop, wifi and a
pad of paper and a pencil.

So, here's what i have to say. We -- you and i, comfortably ensconced
in our coffee shop and well-equipped with our tools -- can realize and
carry out the calculations of quantum mechanics over a very different
formal theory of dynamics, a formal theory of dynamics that
corresponds to a theory of concurrent computation with
\emph{reflection}. It has the advantage that the underlying theory is
already `quantized', but supports analogues all of the continuuous
operations. Strikingly, this underlying theory has recently been
connected with a notion of metric that we can show, by calculating
together, coincides with the metric induced by the inner product.

There are a lot of reasons why you might be interested in seeing
calculations of this form. Here's why i'm interested. For the past
several centuries there has been no competitor to the ``Newtonian''
account of dynamics. As a result the predominant share of accounts of
dynamical systems and situations have had to be formulated in terms of
the Newtonian machinery. i view this as an intellectually dangerous
position to occupy. Everything, despite it's intrinsic shape, turns
into a nail to be hit with this hammer. Recently, however, the theory
of computation has matured to the point where we have candidates for
theories of dynamics that offer very different perspective on
reasoning about dynamical systems and situations. Testing these
candidates against very successful accounts of dynamical situations,
like quantum mechanics, is going to give us some sense of how mature
they are and some measure of the quality of these accounts of
dynamics.

\subsection{Summary of contributions and outline of paper}

So, we're going to develop an interpretation of the operations of
quantum mechanics normally interpreted by Hilbert spaces and
operators. We're going to do this over a theory of computation. Note
that this is very different than the usual quantum computation program
which develops notions of computation over quantum mechanics. Rather,
we are developing a story that aligns with Wheeler's slogan: It from
Bit. To do this we will first provide an account of the theory of
computation at play here. Then we will dive into a calculation-driven
interpretation of the operations of quantum mechanics.

The reason we take this approach is that -- until very recently --
there hasn't been an axiomatic account of quantum mechanics. As a
result there has been no sharp delineation of the mathematical theory
supporting interpretation of the physical theory and the physical
theory, itself. So, ambient features of the maths are free to be
exploited (or supressed) without a real accounting of their physical
relevance. There is no sharp statement ``here's the physical theory''
qua \emph{theory} and ``here's the mathematical interpretation''
enabling a judgment of how faithful the interpretation is -- apart
from experimental observation. When there is an axiomatic account we
can judge how well a given mathematical formalism supports an
interpretation of the axioms, independent of
experimentation. Likewise, we can judge how well we have captured our
physical evidence and experience with our axiomatics, independent of
any specific mathematical implementation, with accidental detail that
may or may not have physical significance. 

In lieu of a fully fleshed out and vetted axiomatic account of quantum
mechanics, interpreting the operational notions in service of modeling
physical systems will have to suffice. In other words, we are not in
the business of providing a model of Hilbert spaces and operators. We
are in the business of providing a model of quantum mechanics because
we are motivated by testing our notions of dynamics against physical
theory; and, the predictive calculations of the physical theory must
serve as the best formulation -- shy of a fully fleshed out axiomatic
account -- of the physical theory itself (as they have for scientific
theories since time immemorial). Put another way, despite a
whole-hearted commitment to an It-from-Bit ontology, we are firmly
aligned with the shut-up-and-calculate camp as the best way to obtain
results either from the physical perspective or as a quality assurance
measure of our fledgling theory of dynamics.

In detail, we present a reflective process calculus. Then we develop
intuitive correspondences between the notions available in this
calculus and the usual physical notions supporting quantum mechanical
calculations. Thus, 

\begin{table}[htp]
  \center{
    \fbox{
      \begin{tabular}{c|c}
        quantum mechanics & process calculus \\
        \hline
        scalar & name \\
        state vector & process \\
        dual & contextual duals \\
        matrix & formal sums of process-context-dual pairs \\
        orthogonality & process annihilation \\
        inner product & execution-formula + quoting
      \end{tabular}
    }
  }
  \caption{QM - process calculi correspondences}
\end{table}

Then we tighten up these intuitions to operational definitions. We
employ the Dirac notation as the best proxy we can find for an
abstract syntax of the quantum mechanical notions. The definitions we
develop put us in contact with equational constraints coming from the
theory that we demonstrate the definitions and calculations satisfy.

This puts us in a position to shut up and calculate for the
Stern-Gerlach experimental set up, showing how these predictive
calculations become calculations on processes in our theory of a
reflective process calculus.

Penultimately, we demonstrate that the notion of metric coming from
the inner product coincides with the notion of metric available from
the theory of bisimulation. This demonstration gives us the right to
think of space as arising from behavior. Finally, we consider where we
might go from the new vantage point we have obtained.

% section introduction (end) 
 
% section introduction (end)

% \documentclass[12pt]{llncs}
%\documentclass{jktr}

\usepackage[pdftex]{hyperref}                   
\usepackage {listings}
\usepackage {mathpartir}
\usepackage{bcprules}
%\usepackage{listings}
                       
\usepackage{graphicx} 
%\usepackage[margins=2.5cm,nohead,nofoot]{geometry}
%\usepackage{geometry}
\usepackage{amsfonts}
\usepackage{amstext}
\usepackage{latexsym}
\usepackage{amssymb}
\usepackage{color}


%\include{myPreamble}
\include{qm2pi.local} 

%\ifpdf
%\usepackage[pdftex]{graphicx}
%\else
%\usepackage{graphicx}
%\fi

 % \ifpdf
%  \usepackage{pdfsync}
%  \if


%\title{Brief Article}
%\author{David F. Snyder}
%\author{L.G. Meredith}

%\address{Dept. of Math., Texas State University--San Marcos, San Marcos, TX 78666}
       
\pagestyle{empty}


\begin{document}

\lstset{language=[Objective]Caml,frame=shadowbox}

\input{qm2pi.front}

% section front matter (end)

\input{qm2pi.intro} 
 
% section introduction (end)

% \input{qm2pi.knotations} 

% section notation (end)

\input{qm2pi.process.calculi} 

% section concurrent_process_calculi_and_spatial_logics_ (end)
    
%\input{qm2pi.knots2pi} 

%\input{qm2pi.trefoil} 

%\input{qm2pi.mainthm} 

% subsection basic_interpretation (end)

%\input{qm2pi.rho.presentation} 
\subsection{The syntax and semantics of the notation system}\label{sub:the_syntax_and_semantics_of_the_notation_system} % (fold)

We now summarize a technical presentation of the calculus that
embodies our theory of dynamics. The typical presentation of such a
calculus follows the style of giving generators and relations on
them. The grammar, below, describing term constructors, freely
generates the set of processes, $\Proc$. This set is then quotiented
by a relation known as structural congruence and it is over this set
that the notion of dynamics is expressed. This presentation is
essentially that of \cite{MeredithR05} with the addition of
polyadicity and summation. For readability we have relegated some of
the technical subtleties to an appendix.

\subsubsection{Process grammar}\label{subsub:process_grammar}

\begin{mathpar}
  \inferrule* [lab=synchronization] {} {{M} \bc \pzero \;|\; x?F \;|\; x!C }
  \and
  \inferrule* [lab=abstraction] {} {{F} \bc (x)P}
  \and
  \inferrule* [lab=concretion] {} {{C} \bc \langle Q \rangle}
  \and
  \inferrule* [lab=process] {} {{P,Q} \bc M \;| \;P|Q \;|\; @{x}}
  \and
  \inferrule* [lab=name] {} {{x} \bc \quotep{P}}
\end{mathpar} 

Note that $\vec{x}$ (resp. $\vec{P}$) denotes a vector of names
(resp. processes) of length $|\vec{x}|$ (resp. $|\vec{P}|$). We adopt
the following useful abbreviations.

\begin{mathpar}
   x?(\vec{y}).P := x.(\vec{y})P \and  x\clift{\vec{P}} := x.\clift{\vec{P}}
   \and x!(y) := \lift{x}{\dropn{y}}
   \and \Pi_{i=0}^{n-1}P_i := P_0 | \ldots | P_{n-1}
\end{mathpar}

\subsubsection{Structural congruence}

\paragraph{Free and bound names and alpha-equivalence.} At the
core of structural equivalence is alpha-equivalence which identifies
process that are the same up to a change of variable. Formally, we
recognize the distinction between free and bound names. The free names
of a process, $\freenames{P}$, may be calculated recursively as
follows:

\begin{mathpar}
\freenames{\pzero} := \emptyset
  \and \\
  \freenames{x?(y).P} := \{ x \} \cup (\freenames{P} \setminus \{ y \})
  \and 
  \freenames{x!\langle P \rangle} := \{ x \} \cup \{ P \} 
  \and \\
  \freenames{P|Q} := \freenames{P} \cup \freenames{Q}
  \and \\
  \freenames{@{x}} := \{ x \}
\end{mathpar}

$\pi$
$\quotep{\pi}$

$\freenames{-} : \pi \to \mathcal{P}(\quotep{\pi})$

\begin{eqnarray*}
  \freenames{\pzero} & := & \emptyset \\
  \freenames{x?(y).P} & := & \{ x \} \cup (\freenames{P} \setminus \{ y \}) \\
  \freenames{x!\langle P \rangle} & := & \{ x \} \cup \{ P \} \\
  \freenames{P|Q} & := & \freenames{P} \cup \freenames{Q} \\
  \freenames{\dropn{x}} & := & \{ x \}
\end{eqnarray*}

The bound names of a process, $\boundnames{P}$, are those names occurring in $P$
that are not free. For example, in $x?(y).0$, the name $x$ is free, while $y$ is bound.

\begin{mathpar}
  \inferrule* [lab=monoidal-laws] {} { P|Q \equiv Q|P \and P|0 \equiv P \and P|(Q|R) \equiv (P|Q)|R }
\end{mathpar}

\begin{mathpar}
  \inferrule* [lab=alpha-equivalence] {} { (x)P \equiv (y)P\{y/x\} \and y \not\in \freenames{P} }
\end{mathpar}

\begin{definition}
Then two processes, $P,Q$, are alpha-equivalent if $P = Q\{\vec{y}/\vec{x}\}$ for
some $\vec{x} \in \boundnames{Q},\vec{y} \in \boundnames{P}$, where $Q\{\vec{y}/\vec{x}\}$
denotes the capture-avoiding substitution of $\vec{y}$ for $\vec{x}$ in $Q$.
\end{definition}

\begin{definition}
  The {\em structural congruence} \cite{SangiorgiWalker} , $\equiv$,
  between processes is the least congruence containing
  alpha-equivalence, satisfying the abelian monoid laws
  (associativity, commutativity and $\pzero$ as identity) for parallel
  composition $|$ and for summation $+$.
\end{definition}

\subsection{Name equivalence}

We take name equivalence, written $\nameeq$, to be the smallest
equivalence relation generated by the following rules.

\begin{mathpar}
\inferrule*[lab=Quote-drop]
{ }
{ \quotep{@{x}} \nameeq x }

\inferrule*[lab=Struct-equiv]
{ P \scong Q }
{ \quotep{P} \nameeq \quotep{Q} }
\end{mathpar}

The astute reader will have noticed that the mutual recursion of names
and processes imposes a mutual recursion on alpha-equivalence and
structural equivalence via name-equivalence. Fortunately, all of this
works out pleasantly and we may calculate in the natural way, free of
concern. The reader interested in the details is referred to the
appendix \ref{appendix:rho_details}.

\subsection{Substitution}

We use $\Proc$ for the set of processes, $\QProc$ for the set of
names, and $\id{\{}\vec{y} / \vec{x} \id{\}}$ to denote partial maps,
$s : \QProc \rightarrow \QProc$. A map, $s$ lifts, uniquely, to a map
on process terms, $\widehat{s} : \Proc \rightarrow \Proc$ by the
following equations.

\begin{mathpar}
  (0) \psubstp{Q}{P} := 0 \\
  (R \juxtap S) \psubstp{Q}{P}
  :=    
  (R)\psubstp{Q}{P} \juxtap (S) \psubstp{Q}{P} \\
  (x?(y).R) \psubstp{Q}{P}    
  :=    
  (x)\substp{Q}{P} (z)\concat( (R \psubstn{z}{y}) \psubstp{Q}{P} ) \\
  (\lift{x}{R}) \psubstp{Q}{P}  
  :=
  \lift{(x)\substp{Q}{P}}{ R \psubstp{Q}{P} } \\
%   (\dropn{x})  \psubstp{Q}{P}       
%   := 
%   \left\{ 
%     \begin{array}{ccc} 
%       \dropn{\quotep{Q}} & & x \nameeq \quotep{P} \\
%       \dropn{x} & & otherwise \\
%     \end{array}
%   \right. 
  (\dropn{x})  \psubstp{Q}{P}       
  := 
  \left\{ 
    \begin{array}{ccc} 
      Q & & x \nameeq \quotep{P} \\
      \dropn{x} & & otherwise \\
    \end{array}
  \right.
\end{mathpar}
 

where

\begin{eqnarray}
  (x)\id{\{} \lpquote Q \rpquote / \lpquote P \rpquote \id{\}}            = 
  \left\{ 
    \begin{array}{ccc}
      \lpquote Q \rpquote & & x \nameeq \lpquote P \rpquote \\
      x & & otherwise \\
    \end{array}
  \right. \nonumber
\end{eqnarray}

and $z$ is chosen distinct from $\quotep{P}$, $\quotep{Q}$, the free
names in $Q$, and all the names in $R$. Our $\alpha$-equivalence will
be built in the standard way from this substitution.

\begin{remark}\label{rem:no_self_referential_names}
  One consequence of these definitions is that $\forall P. \quotep{P}
  \not\in \freenames{P}$.
\end{remark}

\subsection{ Dynamic quote: an example }

Anticipating something of what's to come, consider applying the
substitution, $\widehat{\id{\{}u / z \id{\}}}$, to the following pair
of processes, $\lift{w}{y!(z)}$ and $w[ \lpquote y!(z) \rpquote ]$.

\begin{eqnarray}
	\lift{w}{y!(z)}\widehat{\id{\{}u / z \id{\}}}
		& = &
		\lift{w}{y!(u)} \nonumber\\
	w[ \lpquote y!(z) \rpquote ] \widehat{ \id{\{}u / z \id{\}} }
		& = &
		w[ \lpquote y!(z) \rpquote ] \nonumber
\end{eqnarray}

Because the body of the process between quotes is impervious to
substitution, we get radically different answers. In fact, by
examining the first process in an input context,
e.g. $x?(z).\lift{w}{y!(z)}$, we see that the process under the lift
operator may be shaped by prefixed inputs binding a name inside it. In
this sense, the lift operator will be seen as a way to dynamically
construct processes before reifying them as names.

Finally equipped with these standard features we can present the
dynamics of the calculus.

\subsubsection{Operational semantics} 

Finally, we introduce the computational dynamics. What marks these
algebras as distinct from other more traditionally studied algebraic
structures, e.g. vector spaces or polynomial rings, is the manner in
which dynamics is captured. In traditional structures, dynamics is typically
expressed through morphisms between such structures, as in linear maps
between vector spaces or morphisms between rings. In algebras
associated with the semantics of computation, the dynamics is
expressed as part of the algebraic structure itself, through a
reduction reduction relation typically denoted by $\red$. Below, we
give a recursive presentation of this relation for the calculus used
in the encoding.

$\red \subseteq \pi \times \pi$
$\red : \pi \to \mathcal{P}(\pi)$

\begin{mathpar}
  \inferrule* [lab=Comm] { \textsf{match}( x_{src}, x_{trgt} ) } { x_{trgt}?(y)P \; | \; x_{src}!\langle {Q} \rangle \red P\{\quotep{Q}/y}\} }
  \and \\
  \inferrule* [lab=Par] {{P} \red {P}'} {{{P} | {Q}} \red {{P}' | {Q}}}
  \and
  \inferrule* [lab=Equiv]{{{P} \scong {P}'} \andalso {{P}' \red {Q}'} \andalso {{Q}' \scong {Q}}}{{P} \red {Q}}
\end{mathpar}

\begin{eqnarray*}
  match_{\equiv} (\quotep{P},\quotep{Q}) & := & P \equiv Q \\
  match_{\dagger}(\quotep{P},\quotep{Q}) & := & \forall R. P|Q \red^{*} R => R \red^{*} 0 \\
  match_{K}(\quotep{P},\quotep{Q}) & := & K \mbox{ for some context } K
\end{eqnarray*}

$u?(x)P | u!\langle Q \rangle \red P\{\quotep{Q}/x\}$

%We write $\wred$ for $\red^*$, and $P\red$ if $\exists Q $ such that $ P \red Q$.
We write $P\red$ if $\exists Q $ such that $ P \red Q$ and $P\not\red$, otherwise.

\section{Replication}

As mentioned before, it is known that replication (and hence
recursion) can be implemented in a higher-order process algebra
\cite{SangiorgiWalker}. As our first example of calculation with the
machinery thus far presented we give the construction explicitly in
the {\rhoc}.

\begin{eqnarray}
	D_{x} & := & \prefix{x}{y}{(\binpar{\outputp{x}{y}}{@{y}})} \nonumber\\
	\bangp_{x}{P} & := & \binpar{{x}!\langle{\binpar{D_{x}}{P}}\rangle}{D_{x}} \nonumber
\end{eqnarray}

\begin{eqnarray}
	\bangp_{x}{P} & & \nonumber\\
	=
	& {x}!\langle{(\prefix{x}{y}{(\outputp{x}{y} | @{y})) | P}}\rangle 
	      | \prefix{x}{y}{(\outputp{x}{y} | @{y})} & \nonumber\\
	\red
	& (\outputp{x}{y} | @{y})\substn{\quotep{(\prefix{x}{y}{(@{y} | \outputp{x}{y})) | P}}}{y} & \nonumber\\
	=
	& \outputp{x}{\quotep{(\prefix{x}{y}{(\outputp{x}{y} | @{y})) | P}}}
	  | {(\prefix{x}{y}{(\outputp{x}{y} | @{y})) | P}} & \nonumber\\
	\red
	& \ldots & \nonumber\\
	\red^*
	& P | P | \ldots & \nonumber
\end{eqnarray}

Of course, this encoding, as an implementation, runs away, unfolding
$\bangp{P}$ eagerly. A lazier and more implementable replication
operator, restricted to input-guarded processes, may be obtained as follows.

\begin{eqnarray}
\bangp{\prefix{u}{v}{P}} 
	:= 
	\binpar{\lift{x}{\prefix{u}{v}{(\binpar{D(x)}{P})}}}{D(x)} \nonumber
\end{eqnarray}

\begin{remark}
  Note that the lazier definition still does not deal with summation
  or mixed summation (i.e. sums over input and output). The reader is
  invited to construct definitions of replication that deal with these
  features. 

  Further, the definitions are parameterized in a name, $x$. Can you,
  gentle reader, make a definition that eliminates this parameter and
  guarantees no accidental interaction between the replication
  machinery and the process being replicated -- i.e. no accidental
  sharing of names used by the process to get its work done and the
  name(s) used by the replication to effect copying. This latter
  revision of the definition of replication is crucial to obtaining
  the expected identity $!!P \sim !P$.
\end{remark}

\begin{remark}\label{rem:paradoxical_combinator}
  The reader familiar with the lambda calculus will have noticed the
  similarity between $D$ and the paradoxical combinator.

  [Ed. note: the existence of this seems to suggest we have to be more
  restrictive on the set of processes and names we admit if we are to
  support no-cloning.]
\end{remark}

\subsubsection{Bisimulation}

The computational dynamics gives rise to another kind of equivalence,
the equivalence of computational behavior. As previously mentioned
this is typically captured \emph{via} some form of bisimulation.

% The notion we use in this paper is weak barbed bisimulation
% \cite{milner91polyadicpi}.

The notion we use in this paper is derived from weak barbed
bisimulation \cite{milner91polyadicpi}. 

\begin{definition}
An \emph{observation relation}, $\downarrow_{\mathcal N}$, over a set
of names, $\mathcal N$, is the smallest relation satisfying the rules
below.

\infrule[Out-barb]{y \in {\mathcal N}, \; x \nameeq y}
		  {\outputp{x}{v} \downarrow_{\mathcal N} x}
\infrule[Par-barb]{\mbox{$P\downarrow_{\mathcal N} x$ or $Q\downarrow_{\mathcal N} x$}}
		  {\binpar{P}{Q} \downarrow_{\mathcal N} x}

We write $P \Downarrow_{\mathcal N} x$ if there is $Q$ such that 
$P \wred Q$ and $Q \downarrow_{\mathcal N} x$.
\end{definition}

\begin{definition}
%\label{def.bbisim}
An  ${\mathcal N}$-\emph{barbed bisimulation} over a set of names, ${\mathcal N}$, is a symmetric binary relation 
${\mathcal S}_{\mathcal N}$ between agents such that $P\rel{S}_{\mathcal N}Q$ implies:
\begin{enumerate}
\item If $P \red P'$ then $Q \wred Q'$ and $P'\rel{S}_{\mathcal N} Q'$.
\item If $P\downarrow_{\mathcal N} x$, then $Q\Downarrow_{\mathcal N} x$.
\end{enumerate}
$P$ is ${\mathcal N}$-barbed bisimilar to $Q$, written
$P \wbbisim_{\mathcal N} Q$, if $P \rel{S}_{\mathcal N} Q$ for some ${\mathcal N}$-barbed bisimulation ${\mathcal S}_{\mathcal N}$.
\end{definition}

$\mathcal{R} \subseteq \pi \times \pi$

$P \mathcal{R} Q => \forall P'. P \red P' \Rightarrow \exists Q'. Q \red Q', P' \mathcal{R} Q'$

$P \vdash x \Rightarrow Q \vdash x$

\begin{mathpar}
  \inferrule*[lab=Out-barb]{x \nameeq y}{{y}!\langle{Q}\rangle \vdash x}
  \and
  \inferrule*[lab=Par-barb]{\mbox{$P\vdash x$ or $Q\vdash x$}}{\binpar{P}{Q} \vdash x}
\end{mathpar}

\subsubsection{Contexts}

One of the principle advantages of computational calculi like the
$\pi$-calculus is a well-defined notion of context,
contextual-equivalence and a correlation between
contextual-equivalence and notions of bisimulation. The notion of
context allows the decomposition of a process into (sub-)process and
its syntactic environment, its context. Thus, a context may be
thought of as a process with a ``hole'' (written $\Box$) in it. The
application of a context $M$ to a process $P$, written $M[P]$, is
tantamount to filling the hole in $M$ with $P$. In this paper we do
not need the full weight of this theory, but do make use of the notion
of context in the proof the main theorem. 

\begin{mathpar}
  \inferrule* [lab=summation] {} {{M_{M},M_{N}} \bc \Box \;|\; x.M_{A} \;|\; M_{M}+M_{N}}
  \and
  \inferrule* [lab=agent] {} {{M_{A}} \bc (\vec{x})M_{P} \;| \; \clift{P_0,\ldots,M_{P},\ldots,P_N}}
  \and \\
  \inferrule* [lab=process] {} {{M_{P}} \bc M_{N} \;| \;P|M_{P} }
\end{mathpar} 

\begin{mathpar}
  \inferrule* [lab=sychronization] {} {M_{N} \bc \Box \;|\; x?M_{F} \;|\; x!M_{C}}
  \and
  \inferrule* [lab=abstraction] {} {{M_{F}} \bc (x)M_{P} }
  \and
  \inferrule* [lab=concretion] {} {{M_{C}} \bc \langle M_{P} \rangle }
  \and \\
  \inferrule* [lab=process] {} {{M_{P}} \bc M_{N} \;| \;P|M_{P} }
\end{mathpar}

\begin{definition}[contextual application] Given a context $M$, and
  process $P$, we define the \emph{contextual application}, $M[P] :=
  M\{P/\Box\}$. That is, the contextual application of M to P is the
  substitution of $P$ for $\Box$ in $M$.
\end{definition}

$\meaningof{-} : L \to \mathcal{P}(\pi)$

\begin{mathpar}
  \inferrule* [lab=collection] {} {\meaningof{true} = \pi, \and \meaningof{~E} = \pi \setminus \meaningof{E}, \and \meaningof{E_{1} \& E_{2}} = \meaningof{E_{1}} \cap \meaningof{E_{2}}}
\end{mathpar}

\begin{mathpar}
  \inferrule* [lab=structure] {} {\meaningof{0} = \{ P \in \pi | P \equiv 0 \}, \and \\ \meaningof{E_1 | E_2} = \{ P \in \pi | P \equiv P_{1} | P_{2}, P_{1} \in \meaningof{E_{1}}, P_{2} \in \meaningof{E_2}\} }
\end{mathpar}

\begin{mathpar}
 \inferrule* [lab=behavior] {} {\meaningof{\langle a?b \rangle E} = \{ P \in \pi | P \equiv Q | u?(y)P', \\ \and \\\\ \and \\ \;\;\; u \in \meaningof{a}, \forall z.P'\{z/y\} \in \meaningof{E\{z/b\}}\}, \and \\ \meaningof{a!E} = \{ P \in \pi | P \equiv Q | x!\langle P' \rangle, x \in \meaningof{a} P' \in \meaningof{E}\} }
\end{mathpar}

\begin{mathpar}
 \inferrule* [lab=nominal] {} {\meaningof{\quotep{E}} = \{ \quotep{P} \in \quotep{\pi} | P \in \meaningof{E} \}, \and \meaningof{\quotep{P}} = \{ \quotep{Q} \in \quotep{\pi} | P \equiv Q \} \and \\ \meaningof{@\quotep{E}} = \{ P \in \pi | P \equiv @x, x \in \meaningof{E} \}}
\end{mathpar}

\begin{eqnarray*}
  \\
  \meaningof{-} : TS \to ST
\end{eqnarray*}

\begin{eqnarray*}
  \\
  L : TS \to ST
\end{eqnarray*}

\begin{eqnarray*}
  \\
  P \models E \iff P \in \meaningof{E}
\end{eqnarray*}

\begin{eqnarray*}
  P \approx_{L} Q \iff \forall E \in L. P \models E \iff Q \models E
\end{eqnarray*}

\begin{eqnarray*}
  P \approx_{K} Q
\end{eqnarray*}

\begin{eqnarray*}
  P \approx Q
\end{eqnarray*}

$\approx_{K} = \approx = \approx_{L}$

\subsubsection{Contextual duality}

Note that contexts extend the quotation operation to a family of
operations from processes to names. Given a context, $M$, we can
define a \emph{nominal context}, $\quotep{M}$ by $\quotep{M}[P] :=
\quotep{M[P]}$. To foreshadow what is to come we observe that these
operations enjoy a duality with processes very much like the duality
between vectors and maps from vectors to scalars.

Further, because the calculus is essentially higher-order, we have a
correspondence between contexts and processes. More specifically,
given a name $x$ and a context $M$ we can construct $M^{*}_{x}$ such
that 

\begin{mathpar}
  M^{*}_{x} | \lift{x}{P} \red M[P]
\end{mathpar}

namely,

\begin{mathpar}
  M^{*}_{x} := x?(u).M[\dropn{u}]
\end{mathpar}

The dependence of $M^{*}_{x}$ on a name makes it an abstraction, 

\begin{mathpar}
  M^{*} := (x)x?(u).M[\dropn{u}]
\end{mathpar}

\subsection{Additional notation}

It will sometimes be convenient to denote the process a name
quotes. We already have the notation $x = \quotep{P}$, but it will be
convenient to introduce an alternate notation, $\procn{x}$, when we
want to emphasize the connection to the use of the name. Note that, by
virtue of name equivalence, $\quotep{\procn{x}} \nameeq x$; so, the
notation is consistent with previous definitions.

Further, because names have structure it is possible to effect
substitutions on the basis of that structure. This means we need to
upgrade our notation for substitutions, which we accomplish by
adapting comprehension notation. Thus,

\begin{mathpar}
  P\{ y / x : x \in S \}
\end{mathpar}

is interpreted to mean the process derived from P by replacing (in a
capture-avoiding manner) each occurrence of $x$ in $S$ by $y$. For example,

\begin{mathpar}
  P\{ \quotep{\procn{x}|\procn{x}} / x : x \in \freenames{P} \}
\end{mathpar}

will replace each (occurrence) of a free name $x$ in $P$ by
$\quotep{\procn{x}|\procn{x}}$.

Also, we will avail ourselves of the notation $x^{L}$ and $x^{R}$ to
denote injections of a name into disjoint copies of the name
space. There are numerous ways to accomplish this. One example can be
found in \cite{MeredithR05}. This notation overloads to vectors of
names: $\vec{x}^{\pi} := (x_{i}^{\pi} \; : \; 0 \leq i < |\vec{x}| )$ where $\pi \in \{L,R\}$.

We also use $P^{\Box} := P|\Box$.

In \cite{MeredithR05} an interpretation of the new operator is
given. It turns out that there are several possible interpretations
all enjoying the requisite algebraic properties of the operator (see
\cite{milner91polyadicpi}). We will therefore make liberal use of
$(\nu\; \vec{x})P$.

% subsection the_syntax_and_semantics_of_the_notation_system (end)   

\input{qm2pi.qmops} 

\input{qm2pi.sterngerlach} 

\input{qm2pi.metric} 

% section concurrent_process_calculi (end)

%\input{qm2pi.proofsketch}

% section proof sketch (end)

%\input{qm2pi.slviaknots} 

% section spatial logic via knots (end)

\input{qm2pi.conclusion}

% section conclusion (end)

%\input{qm2pi.dtcodes} 

% section wiring algorithm (end)

\input{qm2pi.ack} 

% section acknowledgments (end)

\newpage


\bibliographystyle{plain}   
\bibliography{../../biblios/main.bib}

\input{qm2pi.rhodetails}

\end{document}

 

% section notation (end)

\input{qm2pi.process.calculi} 

% section concurrent_process_calculi_and_spatial_logics_ (end)
    
%\documentclass[12pt]{llncs}
%\documentclass{jktr}

\usepackage[pdftex]{hyperref}                   
\usepackage {listings}
\usepackage {mathpartir}
\usepackage{bcprules}
%\usepackage{listings}
                       
\usepackage{graphicx} 
%\usepackage[margins=2.5cm,nohead,nofoot]{geometry}
%\usepackage{geometry}
\usepackage{amsfonts}
\usepackage{amstext}
\usepackage{latexsym}
\usepackage{amssymb}
\usepackage{color}


%\include{myPreamble}
\include{qm2pi.local} 

%\ifpdf
%\usepackage[pdftex]{graphicx}
%\else
%\usepackage{graphicx}
%\fi

 % \ifpdf
%  \usepackage{pdfsync}
%  \if


%\title{Brief Article}
%\author{David F. Snyder}
%\author{L.G. Meredith}

%\address{Dept. of Math., Texas State University--San Marcos, San Marcos, TX 78666}
       
\pagestyle{empty}


\begin{document}

\lstset{language=[Objective]Caml,frame=shadowbox}

\input{qm2pi.front}

% section front matter (end)

\input{qm2pi.intro} 
 
% section introduction (end)

% \input{qm2pi.knotations} 

% section notation (end)

\input{qm2pi.process.calculi} 

% section concurrent_process_calculi_and_spatial_logics_ (end)
    
%\input{qm2pi.knots2pi} 

%\input{qm2pi.trefoil} 

%\input{qm2pi.mainthm} 

% subsection basic_interpretation (end)

%\input{qm2pi.rho.presentation} 
\subsection{The syntax and semantics of the notation system}\label{sub:the_syntax_and_semantics_of_the_notation_system} % (fold)

We now summarize a technical presentation of the calculus that
embodies our theory of dynamics. The typical presentation of such a
calculus follows the style of giving generators and relations on
them. The grammar, below, describing term constructors, freely
generates the set of processes, $\Proc$. This set is then quotiented
by a relation known as structural congruence and it is over this set
that the notion of dynamics is expressed. This presentation is
essentially that of \cite{MeredithR05} with the addition of
polyadicity and summation. For readability we have relegated some of
the technical subtleties to an appendix.

\subsubsection{Process grammar}\label{subsub:process_grammar}

\begin{mathpar}
  \inferrule* [lab=synchronization] {} {{M} \bc \pzero \;|\; x?F \;|\; x!C }
  \and
  \inferrule* [lab=abstraction] {} {{F} \bc (x)P}
  \and
  \inferrule* [lab=concretion] {} {{C} \bc \langle Q \rangle}
  \and
  \inferrule* [lab=process] {} {{P,Q} \bc M \;| \;P|Q \;|\; @{x}}
  \and
  \inferrule* [lab=name] {} {{x} \bc \quotep{P}}
\end{mathpar} 

Note that $\vec{x}$ (resp. $\vec{P}$) denotes a vector of names
(resp. processes) of length $|\vec{x}|$ (resp. $|\vec{P}|$). We adopt
the following useful abbreviations.

\begin{mathpar}
   x?(\vec{y}).P := x.(\vec{y})P \and  x\clift{\vec{P}} := x.\clift{\vec{P}}
   \and x!(y) := \lift{x}{\dropn{y}}
   \and \Pi_{i=0}^{n-1}P_i := P_0 | \ldots | P_{n-1}
\end{mathpar}

\subsubsection{Structural congruence}

\paragraph{Free and bound names and alpha-equivalence.} At the
core of structural equivalence is alpha-equivalence which identifies
process that are the same up to a change of variable. Formally, we
recognize the distinction between free and bound names. The free names
of a process, $\freenames{P}$, may be calculated recursively as
follows:

\begin{mathpar}
\freenames{\pzero} := \emptyset
  \and \\
  \freenames{x?(y).P} := \{ x \} \cup (\freenames{P} \setminus \{ y \})
  \and 
  \freenames{x!\langle P \rangle} := \{ x \} \cup \{ P \} 
  \and \\
  \freenames{P|Q} := \freenames{P} \cup \freenames{Q}
  \and \\
  \freenames{@{x}} := \{ x \}
\end{mathpar}

$\pi$
$\quotep{\pi}$

$\freenames{-} : \pi \to \mathcal{P}(\quotep{\pi})$

\begin{eqnarray*}
  \freenames{\pzero} & := & \emptyset \\
  \freenames{x?(y).P} & := & \{ x \} \cup (\freenames{P} \setminus \{ y \}) \\
  \freenames{x!\langle P \rangle} & := & \{ x \} \cup \{ P \} \\
  \freenames{P|Q} & := & \freenames{P} \cup \freenames{Q} \\
  \freenames{\dropn{x}} & := & \{ x \}
\end{eqnarray*}

The bound names of a process, $\boundnames{P}$, are those names occurring in $P$
that are not free. For example, in $x?(y).0$, the name $x$ is free, while $y$ is bound.

\begin{mathpar}
  \inferrule* [lab=monoidal-laws] {} { P|Q \equiv Q|P \and P|0 \equiv P \and P|(Q|R) \equiv (P|Q)|R }
\end{mathpar}

\begin{mathpar}
  \inferrule* [lab=alpha-equivalence] {} { (x)P \equiv (y)P\{y/x\} \and y \not\in \freenames{P} }
\end{mathpar}

\begin{definition}
Then two processes, $P,Q$, are alpha-equivalent if $P = Q\{\vec{y}/\vec{x}\}$ for
some $\vec{x} \in \boundnames{Q},\vec{y} \in \boundnames{P}$, where $Q\{\vec{y}/\vec{x}\}$
denotes the capture-avoiding substitution of $\vec{y}$ for $\vec{x}$ in $Q$.
\end{definition}

\begin{definition}
  The {\em structural congruence} \cite{SangiorgiWalker} , $\equiv$,
  between processes is the least congruence containing
  alpha-equivalence, satisfying the abelian monoid laws
  (associativity, commutativity and $\pzero$ as identity) for parallel
  composition $|$ and for summation $+$.
\end{definition}

\subsection{Name equivalence}

We take name equivalence, written $\nameeq$, to be the smallest
equivalence relation generated by the following rules.

\begin{mathpar}
\inferrule*[lab=Quote-drop]
{ }
{ \quotep{@{x}} \nameeq x }

\inferrule*[lab=Struct-equiv]
{ P \scong Q }
{ \quotep{P} \nameeq \quotep{Q} }
\end{mathpar}

The astute reader will have noticed that the mutual recursion of names
and processes imposes a mutual recursion on alpha-equivalence and
structural equivalence via name-equivalence. Fortunately, all of this
works out pleasantly and we may calculate in the natural way, free of
concern. The reader interested in the details is referred to the
appendix \ref{appendix:rho_details}.

\subsection{Substitution}

We use $\Proc$ for the set of processes, $\QProc$ for the set of
names, and $\id{\{}\vec{y} / \vec{x} \id{\}}$ to denote partial maps,
$s : \QProc \rightarrow \QProc$. A map, $s$ lifts, uniquely, to a map
on process terms, $\widehat{s} : \Proc \rightarrow \Proc$ by the
following equations.

\begin{mathpar}
  (0) \psubstp{Q}{P} := 0 \\
  (R \juxtap S) \psubstp{Q}{P}
  :=    
  (R)\psubstp{Q}{P} \juxtap (S) \psubstp{Q}{P} \\
  (x?(y).R) \psubstp{Q}{P}    
  :=    
  (x)\substp{Q}{P} (z)\concat( (R \psubstn{z}{y}) \psubstp{Q}{P} ) \\
  (\lift{x}{R}) \psubstp{Q}{P}  
  :=
  \lift{(x)\substp{Q}{P}}{ R \psubstp{Q}{P} } \\
%   (\dropn{x})  \psubstp{Q}{P}       
%   := 
%   \left\{ 
%     \begin{array}{ccc} 
%       \dropn{\quotep{Q}} & & x \nameeq \quotep{P} \\
%       \dropn{x} & & otherwise \\
%     \end{array}
%   \right. 
  (\dropn{x})  \psubstp{Q}{P}       
  := 
  \left\{ 
    \begin{array}{ccc} 
      Q & & x \nameeq \quotep{P} \\
      \dropn{x} & & otherwise \\
    \end{array}
  \right.
\end{mathpar}
 

where

\begin{eqnarray}
  (x)\id{\{} \lpquote Q \rpquote / \lpquote P \rpquote \id{\}}            = 
  \left\{ 
    \begin{array}{ccc}
      \lpquote Q \rpquote & & x \nameeq \lpquote P \rpquote \\
      x & & otherwise \\
    \end{array}
  \right. \nonumber
\end{eqnarray}

and $z$ is chosen distinct from $\quotep{P}$, $\quotep{Q}$, the free
names in $Q$, and all the names in $R$. Our $\alpha$-equivalence will
be built in the standard way from this substitution.

\begin{remark}\label{rem:no_self_referential_names}
  One consequence of these definitions is that $\forall P. \quotep{P}
  \not\in \freenames{P}$.
\end{remark}

\subsection{ Dynamic quote: an example }

Anticipating something of what's to come, consider applying the
substitution, $\widehat{\id{\{}u / z \id{\}}}$, to the following pair
of processes, $\lift{w}{y!(z)}$ and $w[ \lpquote y!(z) \rpquote ]$.

\begin{eqnarray}
	\lift{w}{y!(z)}\widehat{\id{\{}u / z \id{\}}}
		& = &
		\lift{w}{y!(u)} \nonumber\\
	w[ \lpquote y!(z) \rpquote ] \widehat{ \id{\{}u / z \id{\}} }
		& = &
		w[ \lpquote y!(z) \rpquote ] \nonumber
\end{eqnarray}

Because the body of the process between quotes is impervious to
substitution, we get radically different answers. In fact, by
examining the first process in an input context,
e.g. $x?(z).\lift{w}{y!(z)}$, we see that the process under the lift
operator may be shaped by prefixed inputs binding a name inside it. In
this sense, the lift operator will be seen as a way to dynamically
construct processes before reifying them as names.

Finally equipped with these standard features we can present the
dynamics of the calculus.

\subsubsection{Operational semantics} 

Finally, we introduce the computational dynamics. What marks these
algebras as distinct from other more traditionally studied algebraic
structures, e.g. vector spaces or polynomial rings, is the manner in
which dynamics is captured. In traditional structures, dynamics is typically
expressed through morphisms between such structures, as in linear maps
between vector spaces or morphisms between rings. In algebras
associated with the semantics of computation, the dynamics is
expressed as part of the algebraic structure itself, through a
reduction reduction relation typically denoted by $\red$. Below, we
give a recursive presentation of this relation for the calculus used
in the encoding.

$\red \subseteq \pi \times \pi$
$\red : \pi \to \mathcal{P}(\pi)$

\begin{mathpar}
  \inferrule* [lab=Comm] { \textsf{match}( x_{src}, x_{trgt} ) } { x_{trgt}?(y)P \; | \; x_{src}!\langle {Q} \rangle \red P\{\quotep{Q}/y}\} }
  \and \\
  \inferrule* [lab=Par] {{P} \red {P}'} {{{P} | {Q}} \red {{P}' | {Q}}}
  \and
  \inferrule* [lab=Equiv]{{{P} \scong {P}'} \andalso {{P}' \red {Q}'} \andalso {{Q}' \scong {Q}}}{{P} \red {Q}}
\end{mathpar}

\begin{eqnarray*}
  match_{\equiv} (\quotep{P},\quotep{Q}) & := & P \equiv Q \\
  match_{\dagger}(\quotep{P},\quotep{Q}) & := & \forall R. P|Q \red^{*} R => R \red^{*} 0 \\
  match_{K}(\quotep{P},\quotep{Q}) & := & K \mbox{ for some context } K
\end{eqnarray*}

$u?(x)P | u!\langle Q \rangle \red P\{\quotep{Q}/x\}$

%We write $\wred$ for $\red^*$, and $P\red$ if $\exists Q $ such that $ P \red Q$.
We write $P\red$ if $\exists Q $ such that $ P \red Q$ and $P\not\red$, otherwise.

\section{Replication}

As mentioned before, it is known that replication (and hence
recursion) can be implemented in a higher-order process algebra
\cite{SangiorgiWalker}. As our first example of calculation with the
machinery thus far presented we give the construction explicitly in
the {\rhoc}.

\begin{eqnarray}
	D_{x} & := & \prefix{x}{y}{(\binpar{\outputp{x}{y}}{@{y}})} \nonumber\\
	\bangp_{x}{P} & := & \binpar{{x}!\langle{\binpar{D_{x}}{P}}\rangle}{D_{x}} \nonumber
\end{eqnarray}

\begin{eqnarray}
	\bangp_{x}{P} & & \nonumber\\
	=
	& {x}!\langle{(\prefix{x}{y}{(\outputp{x}{y} | @{y})) | P}}\rangle 
	      | \prefix{x}{y}{(\outputp{x}{y} | @{y})} & \nonumber\\
	\red
	& (\outputp{x}{y} | @{y})\substn{\quotep{(\prefix{x}{y}{(@{y} | \outputp{x}{y})) | P}}}{y} & \nonumber\\
	=
	& \outputp{x}{\quotep{(\prefix{x}{y}{(\outputp{x}{y} | @{y})) | P}}}
	  | {(\prefix{x}{y}{(\outputp{x}{y} | @{y})) | P}} & \nonumber\\
	\red
	& \ldots & \nonumber\\
	\red^*
	& P | P | \ldots & \nonumber
\end{eqnarray}

Of course, this encoding, as an implementation, runs away, unfolding
$\bangp{P}$ eagerly. A lazier and more implementable replication
operator, restricted to input-guarded processes, may be obtained as follows.

\begin{eqnarray}
\bangp{\prefix{u}{v}{P}} 
	:= 
	\binpar{\lift{x}{\prefix{u}{v}{(\binpar{D(x)}{P})}}}{D(x)} \nonumber
\end{eqnarray}

\begin{remark}
  Note that the lazier definition still does not deal with summation
  or mixed summation (i.e. sums over input and output). The reader is
  invited to construct definitions of replication that deal with these
  features. 

  Further, the definitions are parameterized in a name, $x$. Can you,
  gentle reader, make a definition that eliminates this parameter and
  guarantees no accidental interaction between the replication
  machinery and the process being replicated -- i.e. no accidental
  sharing of names used by the process to get its work done and the
  name(s) used by the replication to effect copying. This latter
  revision of the definition of replication is crucial to obtaining
  the expected identity $!!P \sim !P$.
\end{remark}

\begin{remark}\label{rem:paradoxical_combinator}
  The reader familiar with the lambda calculus will have noticed the
  similarity between $D$ and the paradoxical combinator.

  [Ed. note: the existence of this seems to suggest we have to be more
  restrictive on the set of processes and names we admit if we are to
  support no-cloning.]
\end{remark}

\subsubsection{Bisimulation}

The computational dynamics gives rise to another kind of equivalence,
the equivalence of computational behavior. As previously mentioned
this is typically captured \emph{via} some form of bisimulation.

% The notion we use in this paper is weak barbed bisimulation
% \cite{milner91polyadicpi}.

The notion we use in this paper is derived from weak barbed
bisimulation \cite{milner91polyadicpi}. 

\begin{definition}
An \emph{observation relation}, $\downarrow_{\mathcal N}$, over a set
of names, $\mathcal N$, is the smallest relation satisfying the rules
below.

\infrule[Out-barb]{y \in {\mathcal N}, \; x \nameeq y}
		  {\outputp{x}{v} \downarrow_{\mathcal N} x}
\infrule[Par-barb]{\mbox{$P\downarrow_{\mathcal N} x$ or $Q\downarrow_{\mathcal N} x$}}
		  {\binpar{P}{Q} \downarrow_{\mathcal N} x}

We write $P \Downarrow_{\mathcal N} x$ if there is $Q$ such that 
$P \wred Q$ and $Q \downarrow_{\mathcal N} x$.
\end{definition}

\begin{definition}
%\label{def.bbisim}
An  ${\mathcal N}$-\emph{barbed bisimulation} over a set of names, ${\mathcal N}$, is a symmetric binary relation 
${\mathcal S}_{\mathcal N}$ between agents such that $P\rel{S}_{\mathcal N}Q$ implies:
\begin{enumerate}
\item If $P \red P'$ then $Q \wred Q'$ and $P'\rel{S}_{\mathcal N} Q'$.
\item If $P\downarrow_{\mathcal N} x$, then $Q\Downarrow_{\mathcal N} x$.
\end{enumerate}
$P$ is ${\mathcal N}$-barbed bisimilar to $Q$, written
$P \wbbisim_{\mathcal N} Q$, if $P \rel{S}_{\mathcal N} Q$ for some ${\mathcal N}$-barbed bisimulation ${\mathcal S}_{\mathcal N}$.
\end{definition}

$\mathcal{R} \subseteq \pi \times \pi$

$P \mathcal{R} Q => \forall P'. P \red P' \Rightarrow \exists Q'. Q \red Q', P' \mathcal{R} Q'$

$P \vdash x \Rightarrow Q \vdash x$

\begin{mathpar}
  \inferrule*[lab=Out-barb]{x \nameeq y}{{y}!\langle{Q}\rangle \vdash x}
  \and
  \inferrule*[lab=Par-barb]{\mbox{$P\vdash x$ or $Q\vdash x$}}{\binpar{P}{Q} \vdash x}
\end{mathpar}

\subsubsection{Contexts}

One of the principle advantages of computational calculi like the
$\pi$-calculus is a well-defined notion of context,
contextual-equivalence and a correlation between
contextual-equivalence and notions of bisimulation. The notion of
context allows the decomposition of a process into (sub-)process and
its syntactic environment, its context. Thus, a context may be
thought of as a process with a ``hole'' (written $\Box$) in it. The
application of a context $M$ to a process $P$, written $M[P]$, is
tantamount to filling the hole in $M$ with $P$. In this paper we do
not need the full weight of this theory, but do make use of the notion
of context in the proof the main theorem. 

\begin{mathpar}
  \inferrule* [lab=summation] {} {{M_{M},M_{N}} \bc \Box \;|\; x.M_{A} \;|\; M_{M}+M_{N}}
  \and
  \inferrule* [lab=agent] {} {{M_{A}} \bc (\vec{x})M_{P} \;| \; \clift{P_0,\ldots,M_{P},\ldots,P_N}}
  \and \\
  \inferrule* [lab=process] {} {{M_{P}} \bc M_{N} \;| \;P|M_{P} }
\end{mathpar} 

\begin{mathpar}
  \inferrule* [lab=sychronization] {} {M_{N} \bc \Box \;|\; x?M_{F} \;|\; x!M_{C}}
  \and
  \inferrule* [lab=abstraction] {} {{M_{F}} \bc (x)M_{P} }
  \and
  \inferrule* [lab=concretion] {} {{M_{C}} \bc \langle M_{P} \rangle }
  \and \\
  \inferrule* [lab=process] {} {{M_{P}} \bc M_{N} \;| \;P|M_{P} }
\end{mathpar}

\begin{definition}[contextual application] Given a context $M$, and
  process $P$, we define the \emph{contextual application}, $M[P] :=
  M\{P/\Box\}$. That is, the contextual application of M to P is the
  substitution of $P$ for $\Box$ in $M$.
\end{definition}

$\meaningof{-} : L \to \mathcal{P}(\pi)$

\begin{mathpar}
  \inferrule* [lab=collection] {} {\meaningof{true} = \pi, \and \meaningof{~E} = \pi \setminus \meaningof{E}, \and \meaningof{E_{1} \& E_{2}} = \meaningof{E_{1}} \cap \meaningof{E_{2}}}
\end{mathpar}

\begin{mathpar}
  \inferrule* [lab=structure] {} {\meaningof{0} = \{ P \in \pi | P \equiv 0 \}, \and \\ \meaningof{E_1 | E_2} = \{ P \in \pi | P \equiv P_{1} | P_{2}, P_{1} \in \meaningof{E_{1}}, P_{2} \in \meaningof{E_2}\} }
\end{mathpar}

\begin{mathpar}
 \inferrule* [lab=behavior] {} {\meaningof{\langle a?b \rangle E} = \{ P \in \pi | P \equiv Q | u?(y)P', \\ \and \\\\ \and \\ \;\;\; u \in \meaningof{a}, \forall z.P'\{z/y\} \in \meaningof{E\{z/b\}}\}, \and \\ \meaningof{a!E} = \{ P \in \pi | P \equiv Q | x!\langle P' \rangle, x \in \meaningof{a} P' \in \meaningof{E}\} }
\end{mathpar}

\begin{mathpar}
 \inferrule* [lab=nominal] {} {\meaningof{\quotep{E}} = \{ \quotep{P} \in \quotep{\pi} | P \in \meaningof{E} \}, \and \meaningof{\quotep{P}} = \{ \quotep{Q} \in \quotep{\pi} | P \equiv Q \} \and \\ \meaningof{@\quotep{E}} = \{ P \in \pi | P \equiv @x, x \in \meaningof{E} \}}
\end{mathpar}

\begin{eqnarray*}
  \\
  \meaningof{-} : TS \to ST
\end{eqnarray*}

\begin{eqnarray*}
  \\
  L : TS \to ST
\end{eqnarray*}

\begin{eqnarray*}
  \\
  P \models E \iff P \in \meaningof{E}
\end{eqnarray*}

\begin{eqnarray*}
  P \approx_{L} Q \iff \forall E \in L. P \models E \iff Q \models E
\end{eqnarray*}

\begin{eqnarray*}
  P \approx_{K} Q
\end{eqnarray*}

\begin{eqnarray*}
  P \approx Q
\end{eqnarray*}

$\approx_{K} = \approx = \approx_{L}$

\subsubsection{Contextual duality}

Note that contexts extend the quotation operation to a family of
operations from processes to names. Given a context, $M$, we can
define a \emph{nominal context}, $\quotep{M}$ by $\quotep{M}[P] :=
\quotep{M[P]}$. To foreshadow what is to come we observe that these
operations enjoy a duality with processes very much like the duality
between vectors and maps from vectors to scalars.

Further, because the calculus is essentially higher-order, we have a
correspondence between contexts and processes. More specifically,
given a name $x$ and a context $M$ we can construct $M^{*}_{x}$ such
that 

\begin{mathpar}
  M^{*}_{x} | \lift{x}{P} \red M[P]
\end{mathpar}

namely,

\begin{mathpar}
  M^{*}_{x} := x?(u).M[\dropn{u}]
\end{mathpar}

The dependence of $M^{*}_{x}$ on a name makes it an abstraction, 

\begin{mathpar}
  M^{*} := (x)x?(u).M[\dropn{u}]
\end{mathpar}

\subsection{Additional notation}

It will sometimes be convenient to denote the process a name
quotes. We already have the notation $x = \quotep{P}$, but it will be
convenient to introduce an alternate notation, $\procn{x}$, when we
want to emphasize the connection to the use of the name. Note that, by
virtue of name equivalence, $\quotep{\procn{x}} \nameeq x$; so, the
notation is consistent with previous definitions.

Further, because names have structure it is possible to effect
substitutions on the basis of that structure. This means we need to
upgrade our notation for substitutions, which we accomplish by
adapting comprehension notation. Thus,

\begin{mathpar}
  P\{ y / x : x \in S \}
\end{mathpar}

is interpreted to mean the process derived from P by replacing (in a
capture-avoiding manner) each occurrence of $x$ in $S$ by $y$. For example,

\begin{mathpar}
  P\{ \quotep{\procn{x}|\procn{x}} / x : x \in \freenames{P} \}
\end{mathpar}

will replace each (occurrence) of a free name $x$ in $P$ by
$\quotep{\procn{x}|\procn{x}}$.

Also, we will avail ourselves of the notation $x^{L}$ and $x^{R}$ to
denote injections of a name into disjoint copies of the name
space. There are numerous ways to accomplish this. One example can be
found in \cite{MeredithR05}. This notation overloads to vectors of
names: $\vec{x}^{\pi} := (x_{i}^{\pi} \; : \; 0 \leq i < |\vec{x}| )$ where $\pi \in \{L,R\}$.

We also use $P^{\Box} := P|\Box$.

In \cite{MeredithR05} an interpretation of the new operator is
given. It turns out that there are several possible interpretations
all enjoying the requisite algebraic properties of the operator (see
\cite{milner91polyadicpi}). We will therefore make liberal use of
$(\nu\; \vec{x})P$.

% subsection the_syntax_and_semantics_of_the_notation_system (end)   

\input{qm2pi.qmops} 

\input{qm2pi.sterngerlach} 

\input{qm2pi.metric} 

% section concurrent_process_calculi (end)

%\input{qm2pi.proofsketch}

% section proof sketch (end)

%\input{qm2pi.slviaknots} 

% section spatial logic via knots (end)

\input{qm2pi.conclusion}

% section conclusion (end)

%\input{qm2pi.dtcodes} 

% section wiring algorithm (end)

\input{qm2pi.ack} 

% section acknowledgments (end)

\newpage


\bibliographystyle{plain}   
\bibliography{../../biblios/main.bib}

\input{qm2pi.rhodetails}

\end{document}

 

%\documentclass[12pt]{llncs}
%\documentclass{jktr}

\usepackage[pdftex]{hyperref}                   
\usepackage {listings}
\usepackage {mathpartir}
\usepackage{bcprules}
%\usepackage{listings}
                       
\usepackage{graphicx} 
%\usepackage[margins=2.5cm,nohead,nofoot]{geometry}
%\usepackage{geometry}
\usepackage{amsfonts}
\usepackage{amstext}
\usepackage{latexsym}
\usepackage{amssymb}
\usepackage{color}


%\include{myPreamble}
\include{qm2pi.local} 

%\ifpdf
%\usepackage[pdftex]{graphicx}
%\else
%\usepackage{graphicx}
%\fi

 % \ifpdf
%  \usepackage{pdfsync}
%  \if


%\title{Brief Article}
%\author{David F. Snyder}
%\author{L.G. Meredith}

%\address{Dept. of Math., Texas State University--San Marcos, San Marcos, TX 78666}
       
\pagestyle{empty}


\begin{document}

\lstset{language=[Objective]Caml,frame=shadowbox}

\input{qm2pi.front}

% section front matter (end)

\input{qm2pi.intro} 
 
% section introduction (end)

% \input{qm2pi.knotations} 

% section notation (end)

\input{qm2pi.process.calculi} 

% section concurrent_process_calculi_and_spatial_logics_ (end)
    
%\input{qm2pi.knots2pi} 

%\input{qm2pi.trefoil} 

%\input{qm2pi.mainthm} 

% subsection basic_interpretation (end)

%\input{qm2pi.rho.presentation} 
\subsection{The syntax and semantics of the notation system}\label{sub:the_syntax_and_semantics_of_the_notation_system} % (fold)

We now summarize a technical presentation of the calculus that
embodies our theory of dynamics. The typical presentation of such a
calculus follows the style of giving generators and relations on
them. The grammar, below, describing term constructors, freely
generates the set of processes, $\Proc$. This set is then quotiented
by a relation known as structural congruence and it is over this set
that the notion of dynamics is expressed. This presentation is
essentially that of \cite{MeredithR05} with the addition of
polyadicity and summation. For readability we have relegated some of
the technical subtleties to an appendix.

\subsubsection{Process grammar}\label{subsub:process_grammar}

\begin{mathpar}
  \inferrule* [lab=synchronization] {} {{M} \bc \pzero \;|\; x?F \;|\; x!C }
  \and
  \inferrule* [lab=abstraction] {} {{F} \bc (x)P}
  \and
  \inferrule* [lab=concretion] {} {{C} \bc \langle Q \rangle}
  \and
  \inferrule* [lab=process] {} {{P,Q} \bc M \;| \;P|Q \;|\; @{x}}
  \and
  \inferrule* [lab=name] {} {{x} \bc \quotep{P}}
\end{mathpar} 

Note that $\vec{x}$ (resp. $\vec{P}$) denotes a vector of names
(resp. processes) of length $|\vec{x}|$ (resp. $|\vec{P}|$). We adopt
the following useful abbreviations.

\begin{mathpar}
   x?(\vec{y}).P := x.(\vec{y})P \and  x\clift{\vec{P}} := x.\clift{\vec{P}}
   \and x!(y) := \lift{x}{\dropn{y}}
   \and \Pi_{i=0}^{n-1}P_i := P_0 | \ldots | P_{n-1}
\end{mathpar}

\subsubsection{Structural congruence}

\paragraph{Free and bound names and alpha-equivalence.} At the
core of structural equivalence is alpha-equivalence which identifies
process that are the same up to a change of variable. Formally, we
recognize the distinction between free and bound names. The free names
of a process, $\freenames{P}$, may be calculated recursively as
follows:

\begin{mathpar}
\freenames{\pzero} := \emptyset
  \and \\
  \freenames{x?(y).P} := \{ x \} \cup (\freenames{P} \setminus \{ y \})
  \and 
  \freenames{x!\langle P \rangle} := \{ x \} \cup \{ P \} 
  \and \\
  \freenames{P|Q} := \freenames{P} \cup \freenames{Q}
  \and \\
  \freenames{@{x}} := \{ x \}
\end{mathpar}

$\pi$
$\quotep{\pi}$

$\freenames{-} : \pi \to \mathcal{P}(\quotep{\pi})$

\begin{eqnarray*}
  \freenames{\pzero} & := & \emptyset \\
  \freenames{x?(y).P} & := & \{ x \} \cup (\freenames{P} \setminus \{ y \}) \\
  \freenames{x!\langle P \rangle} & := & \{ x \} \cup \{ P \} \\
  \freenames{P|Q} & := & \freenames{P} \cup \freenames{Q} \\
  \freenames{\dropn{x}} & := & \{ x \}
\end{eqnarray*}

The bound names of a process, $\boundnames{P}$, are those names occurring in $P$
that are not free. For example, in $x?(y).0$, the name $x$ is free, while $y$ is bound.

\begin{mathpar}
  \inferrule* [lab=monoidal-laws] {} { P|Q \equiv Q|P \and P|0 \equiv P \and P|(Q|R) \equiv (P|Q)|R }
\end{mathpar}

\begin{mathpar}
  \inferrule* [lab=alpha-equivalence] {} { (x)P \equiv (y)P\{y/x\} \and y \not\in \freenames{P} }
\end{mathpar}

\begin{definition}
Then two processes, $P,Q$, are alpha-equivalent if $P = Q\{\vec{y}/\vec{x}\}$ for
some $\vec{x} \in \boundnames{Q},\vec{y} \in \boundnames{P}$, where $Q\{\vec{y}/\vec{x}\}$
denotes the capture-avoiding substitution of $\vec{y}$ for $\vec{x}$ in $Q$.
\end{definition}

\begin{definition}
  The {\em structural congruence} \cite{SangiorgiWalker} , $\equiv$,
  between processes is the least congruence containing
  alpha-equivalence, satisfying the abelian monoid laws
  (associativity, commutativity and $\pzero$ as identity) for parallel
  composition $|$ and for summation $+$.
\end{definition}

\subsection{Name equivalence}

We take name equivalence, written $\nameeq$, to be the smallest
equivalence relation generated by the following rules.

\begin{mathpar}
\inferrule*[lab=Quote-drop]
{ }
{ \quotep{@{x}} \nameeq x }

\inferrule*[lab=Struct-equiv]
{ P \scong Q }
{ \quotep{P} \nameeq \quotep{Q} }
\end{mathpar}

The astute reader will have noticed that the mutual recursion of names
and processes imposes a mutual recursion on alpha-equivalence and
structural equivalence via name-equivalence. Fortunately, all of this
works out pleasantly and we may calculate in the natural way, free of
concern. The reader interested in the details is referred to the
appendix \ref{appendix:rho_details}.

\subsection{Substitution}

We use $\Proc$ for the set of processes, $\QProc$ for the set of
names, and $\id{\{}\vec{y} / \vec{x} \id{\}}$ to denote partial maps,
$s : \QProc \rightarrow \QProc$. A map, $s$ lifts, uniquely, to a map
on process terms, $\widehat{s} : \Proc \rightarrow \Proc$ by the
following equations.

\begin{mathpar}
  (0) \psubstp{Q}{P} := 0 \\
  (R \juxtap S) \psubstp{Q}{P}
  :=    
  (R)\psubstp{Q}{P} \juxtap (S) \psubstp{Q}{P} \\
  (x?(y).R) \psubstp{Q}{P}    
  :=    
  (x)\substp{Q}{P} (z)\concat( (R \psubstn{z}{y}) \psubstp{Q}{P} ) \\
  (\lift{x}{R}) \psubstp{Q}{P}  
  :=
  \lift{(x)\substp{Q}{P}}{ R \psubstp{Q}{P} } \\
%   (\dropn{x})  \psubstp{Q}{P}       
%   := 
%   \left\{ 
%     \begin{array}{ccc} 
%       \dropn{\quotep{Q}} & & x \nameeq \quotep{P} \\
%       \dropn{x} & & otherwise \\
%     \end{array}
%   \right. 
  (\dropn{x})  \psubstp{Q}{P}       
  := 
  \left\{ 
    \begin{array}{ccc} 
      Q & & x \nameeq \quotep{P} \\
      \dropn{x} & & otherwise \\
    \end{array}
  \right.
\end{mathpar}
 

where

\begin{eqnarray}
  (x)\id{\{} \lpquote Q \rpquote / \lpquote P \rpquote \id{\}}            = 
  \left\{ 
    \begin{array}{ccc}
      \lpquote Q \rpquote & & x \nameeq \lpquote P \rpquote \\
      x & & otherwise \\
    \end{array}
  \right. \nonumber
\end{eqnarray}

and $z$ is chosen distinct from $\quotep{P}$, $\quotep{Q}$, the free
names in $Q$, and all the names in $R$. Our $\alpha$-equivalence will
be built in the standard way from this substitution.

\begin{remark}\label{rem:no_self_referential_names}
  One consequence of these definitions is that $\forall P. \quotep{P}
  \not\in \freenames{P}$.
\end{remark}

\subsection{ Dynamic quote: an example }

Anticipating something of what's to come, consider applying the
substitution, $\widehat{\id{\{}u / z \id{\}}}$, to the following pair
of processes, $\lift{w}{y!(z)}$ and $w[ \lpquote y!(z) \rpquote ]$.

\begin{eqnarray}
	\lift{w}{y!(z)}\widehat{\id{\{}u / z \id{\}}}
		& = &
		\lift{w}{y!(u)} \nonumber\\
	w[ \lpquote y!(z) \rpquote ] \widehat{ \id{\{}u / z \id{\}} }
		& = &
		w[ \lpquote y!(z) \rpquote ] \nonumber
\end{eqnarray}

Because the body of the process between quotes is impervious to
substitution, we get radically different answers. In fact, by
examining the first process in an input context,
e.g. $x?(z).\lift{w}{y!(z)}$, we see that the process under the lift
operator may be shaped by prefixed inputs binding a name inside it. In
this sense, the lift operator will be seen as a way to dynamically
construct processes before reifying them as names.

Finally equipped with these standard features we can present the
dynamics of the calculus.

\subsubsection{Operational semantics} 

Finally, we introduce the computational dynamics. What marks these
algebras as distinct from other more traditionally studied algebraic
structures, e.g. vector spaces or polynomial rings, is the manner in
which dynamics is captured. In traditional structures, dynamics is typically
expressed through morphisms between such structures, as in linear maps
between vector spaces or morphisms between rings. In algebras
associated with the semantics of computation, the dynamics is
expressed as part of the algebraic structure itself, through a
reduction reduction relation typically denoted by $\red$. Below, we
give a recursive presentation of this relation for the calculus used
in the encoding.

$\red \subseteq \pi \times \pi$
$\red : \pi \to \mathcal{P}(\pi)$

\begin{mathpar}
  \inferrule* [lab=Comm] { \textsf{match}( x_{src}, x_{trgt} ) } { x_{trgt}?(y)P \; | \; x_{src}!\langle {Q} \rangle \red P\{\quotep{Q}/y}\} }
  \and \\
  \inferrule* [lab=Par] {{P} \red {P}'} {{{P} | {Q}} \red {{P}' | {Q}}}
  \and
  \inferrule* [lab=Equiv]{{{P} \scong {P}'} \andalso {{P}' \red {Q}'} \andalso {{Q}' \scong {Q}}}{{P} \red {Q}}
\end{mathpar}

\begin{eqnarray*}
  match_{\equiv} (\quotep{P},\quotep{Q}) & := & P \equiv Q \\
  match_{\dagger}(\quotep{P},\quotep{Q}) & := & \forall R. P|Q \red^{*} R => R \red^{*} 0 \\
  match_{K}(\quotep{P},\quotep{Q}) & := & K \mbox{ for some context } K
\end{eqnarray*}

$u?(x)P | u!\langle Q \rangle \red P\{\quotep{Q}/x\}$

%We write $\wred$ for $\red^*$, and $P\red$ if $\exists Q $ such that $ P \red Q$.
We write $P\red$ if $\exists Q $ such that $ P \red Q$ and $P\not\red$, otherwise.

\section{Replication}

As mentioned before, it is known that replication (and hence
recursion) can be implemented in a higher-order process algebra
\cite{SangiorgiWalker}. As our first example of calculation with the
machinery thus far presented we give the construction explicitly in
the {\rhoc}.

\begin{eqnarray}
	D_{x} & := & \prefix{x}{y}{(\binpar{\outputp{x}{y}}{@{y}})} \nonumber\\
	\bangp_{x}{P} & := & \binpar{{x}!\langle{\binpar{D_{x}}{P}}\rangle}{D_{x}} \nonumber
\end{eqnarray}

\begin{eqnarray}
	\bangp_{x}{P} & & \nonumber\\
	=
	& {x}!\langle{(\prefix{x}{y}{(\outputp{x}{y} | @{y})) | P}}\rangle 
	      | \prefix{x}{y}{(\outputp{x}{y} | @{y})} & \nonumber\\
	\red
	& (\outputp{x}{y} | @{y})\substn{\quotep{(\prefix{x}{y}{(@{y} | \outputp{x}{y})) | P}}}{y} & \nonumber\\
	=
	& \outputp{x}{\quotep{(\prefix{x}{y}{(\outputp{x}{y} | @{y})) | P}}}
	  | {(\prefix{x}{y}{(\outputp{x}{y} | @{y})) | P}} & \nonumber\\
	\red
	& \ldots & \nonumber\\
	\red^*
	& P | P | \ldots & \nonumber
\end{eqnarray}

Of course, this encoding, as an implementation, runs away, unfolding
$\bangp{P}$ eagerly. A lazier and more implementable replication
operator, restricted to input-guarded processes, may be obtained as follows.

\begin{eqnarray}
\bangp{\prefix{u}{v}{P}} 
	:= 
	\binpar{\lift{x}{\prefix{u}{v}{(\binpar{D(x)}{P})}}}{D(x)} \nonumber
\end{eqnarray}

\begin{remark}
  Note that the lazier definition still does not deal with summation
  or mixed summation (i.e. sums over input and output). The reader is
  invited to construct definitions of replication that deal with these
  features. 

  Further, the definitions are parameterized in a name, $x$. Can you,
  gentle reader, make a definition that eliminates this parameter and
  guarantees no accidental interaction between the replication
  machinery and the process being replicated -- i.e. no accidental
  sharing of names used by the process to get its work done and the
  name(s) used by the replication to effect copying. This latter
  revision of the definition of replication is crucial to obtaining
  the expected identity $!!P \sim !P$.
\end{remark}

\begin{remark}\label{rem:paradoxical_combinator}
  The reader familiar with the lambda calculus will have noticed the
  similarity between $D$ and the paradoxical combinator.

  [Ed. note: the existence of this seems to suggest we have to be more
  restrictive on the set of processes and names we admit if we are to
  support no-cloning.]
\end{remark}

\subsubsection{Bisimulation}

The computational dynamics gives rise to another kind of equivalence,
the equivalence of computational behavior. As previously mentioned
this is typically captured \emph{via} some form of bisimulation.

% The notion we use in this paper is weak barbed bisimulation
% \cite{milner91polyadicpi}.

The notion we use in this paper is derived from weak barbed
bisimulation \cite{milner91polyadicpi}. 

\begin{definition}
An \emph{observation relation}, $\downarrow_{\mathcal N}$, over a set
of names, $\mathcal N$, is the smallest relation satisfying the rules
below.

\infrule[Out-barb]{y \in {\mathcal N}, \; x \nameeq y}
		  {\outputp{x}{v} \downarrow_{\mathcal N} x}
\infrule[Par-barb]{\mbox{$P\downarrow_{\mathcal N} x$ or $Q\downarrow_{\mathcal N} x$}}
		  {\binpar{P}{Q} \downarrow_{\mathcal N} x}

We write $P \Downarrow_{\mathcal N} x$ if there is $Q$ such that 
$P \wred Q$ and $Q \downarrow_{\mathcal N} x$.
\end{definition}

\begin{definition}
%\label{def.bbisim}
An  ${\mathcal N}$-\emph{barbed bisimulation} over a set of names, ${\mathcal N}$, is a symmetric binary relation 
${\mathcal S}_{\mathcal N}$ between agents such that $P\rel{S}_{\mathcal N}Q$ implies:
\begin{enumerate}
\item If $P \red P'$ then $Q \wred Q'$ and $P'\rel{S}_{\mathcal N} Q'$.
\item If $P\downarrow_{\mathcal N} x$, then $Q\Downarrow_{\mathcal N} x$.
\end{enumerate}
$P$ is ${\mathcal N}$-barbed bisimilar to $Q$, written
$P \wbbisim_{\mathcal N} Q$, if $P \rel{S}_{\mathcal N} Q$ for some ${\mathcal N}$-barbed bisimulation ${\mathcal S}_{\mathcal N}$.
\end{definition}

$\mathcal{R} \subseteq \pi \times \pi$

$P \mathcal{R} Q => \forall P'. P \red P' \Rightarrow \exists Q'. Q \red Q', P' \mathcal{R} Q'$

$P \vdash x \Rightarrow Q \vdash x$

\begin{mathpar}
  \inferrule*[lab=Out-barb]{x \nameeq y}{{y}!\langle{Q}\rangle \vdash x}
  \and
  \inferrule*[lab=Par-barb]{\mbox{$P\vdash x$ or $Q\vdash x$}}{\binpar{P}{Q} \vdash x}
\end{mathpar}

\subsubsection{Contexts}

One of the principle advantages of computational calculi like the
$\pi$-calculus is a well-defined notion of context,
contextual-equivalence and a correlation between
contextual-equivalence and notions of bisimulation. The notion of
context allows the decomposition of a process into (sub-)process and
its syntactic environment, its context. Thus, a context may be
thought of as a process with a ``hole'' (written $\Box$) in it. The
application of a context $M$ to a process $P$, written $M[P]$, is
tantamount to filling the hole in $M$ with $P$. In this paper we do
not need the full weight of this theory, but do make use of the notion
of context in the proof the main theorem. 

\begin{mathpar}
  \inferrule* [lab=summation] {} {{M_{M},M_{N}} \bc \Box \;|\; x.M_{A} \;|\; M_{M}+M_{N}}
  \and
  \inferrule* [lab=agent] {} {{M_{A}} \bc (\vec{x})M_{P} \;| \; \clift{P_0,\ldots,M_{P},\ldots,P_N}}
  \and \\
  \inferrule* [lab=process] {} {{M_{P}} \bc M_{N} \;| \;P|M_{P} }
\end{mathpar} 

\begin{mathpar}
  \inferrule* [lab=sychronization] {} {M_{N} \bc \Box \;|\; x?M_{F} \;|\; x!M_{C}}
  \and
  \inferrule* [lab=abstraction] {} {{M_{F}} \bc (x)M_{P} }
  \and
  \inferrule* [lab=concretion] {} {{M_{C}} \bc \langle M_{P} \rangle }
  \and \\
  \inferrule* [lab=process] {} {{M_{P}} \bc M_{N} \;| \;P|M_{P} }
\end{mathpar}

\begin{definition}[contextual application] Given a context $M$, and
  process $P$, we define the \emph{contextual application}, $M[P] :=
  M\{P/\Box\}$. That is, the contextual application of M to P is the
  substitution of $P$ for $\Box$ in $M$.
\end{definition}

$\meaningof{-} : L \to \mathcal{P}(\pi)$

\begin{mathpar}
  \inferrule* [lab=collection] {} {\meaningof{true} = \pi, \and \meaningof{~E} = \pi \setminus \meaningof{E}, \and \meaningof{E_{1} \& E_{2}} = \meaningof{E_{1}} \cap \meaningof{E_{2}}}
\end{mathpar}

\begin{mathpar}
  \inferrule* [lab=structure] {} {\meaningof{0} = \{ P \in \pi | P \equiv 0 \}, \and \\ \meaningof{E_1 | E_2} = \{ P \in \pi | P \equiv P_{1} | P_{2}, P_{1} \in \meaningof{E_{1}}, P_{2} \in \meaningof{E_2}\} }
\end{mathpar}

\begin{mathpar}
 \inferrule* [lab=behavior] {} {\meaningof{\langle a?b \rangle E} = \{ P \in \pi | P \equiv Q | u?(y)P', \\ \and \\\\ \and \\ \;\;\; u \in \meaningof{a}, \forall z.P'\{z/y\} \in \meaningof{E\{z/b\}}\}, \and \\ \meaningof{a!E} = \{ P \in \pi | P \equiv Q | x!\langle P' \rangle, x \in \meaningof{a} P' \in \meaningof{E}\} }
\end{mathpar}

\begin{mathpar}
 \inferrule* [lab=nominal] {} {\meaningof{\quotep{E}} = \{ \quotep{P} \in \quotep{\pi} | P \in \meaningof{E} \}, \and \meaningof{\quotep{P}} = \{ \quotep{Q} \in \quotep{\pi} | P \equiv Q \} \and \\ \meaningof{@\quotep{E}} = \{ P \in \pi | P \equiv @x, x \in \meaningof{E} \}}
\end{mathpar}

\begin{eqnarray*}
  \\
  \meaningof{-} : TS \to ST
\end{eqnarray*}

\begin{eqnarray*}
  \\
  L : TS \to ST
\end{eqnarray*}

\begin{eqnarray*}
  \\
  P \models E \iff P \in \meaningof{E}
\end{eqnarray*}

\begin{eqnarray*}
  P \approx_{L} Q \iff \forall E \in L. P \models E \iff Q \models E
\end{eqnarray*}

\begin{eqnarray*}
  P \approx_{K} Q
\end{eqnarray*}

\begin{eqnarray*}
  P \approx Q
\end{eqnarray*}

$\approx_{K} = \approx = \approx_{L}$

\subsubsection{Contextual duality}

Note that contexts extend the quotation operation to a family of
operations from processes to names. Given a context, $M$, we can
define a \emph{nominal context}, $\quotep{M}$ by $\quotep{M}[P] :=
\quotep{M[P]}$. To foreshadow what is to come we observe that these
operations enjoy a duality with processes very much like the duality
between vectors and maps from vectors to scalars.

Further, because the calculus is essentially higher-order, we have a
correspondence between contexts and processes. More specifically,
given a name $x$ and a context $M$ we can construct $M^{*}_{x}$ such
that 

\begin{mathpar}
  M^{*}_{x} | \lift{x}{P} \red M[P]
\end{mathpar}

namely,

\begin{mathpar}
  M^{*}_{x} := x?(u).M[\dropn{u}]
\end{mathpar}

The dependence of $M^{*}_{x}$ on a name makes it an abstraction, 

\begin{mathpar}
  M^{*} := (x)x?(u).M[\dropn{u}]
\end{mathpar}

\subsection{Additional notation}

It will sometimes be convenient to denote the process a name
quotes. We already have the notation $x = \quotep{P}$, but it will be
convenient to introduce an alternate notation, $\procn{x}$, when we
want to emphasize the connection to the use of the name. Note that, by
virtue of name equivalence, $\quotep{\procn{x}} \nameeq x$; so, the
notation is consistent with previous definitions.

Further, because names have structure it is possible to effect
substitutions on the basis of that structure. This means we need to
upgrade our notation for substitutions, which we accomplish by
adapting comprehension notation. Thus,

\begin{mathpar}
  P\{ y / x : x \in S \}
\end{mathpar}

is interpreted to mean the process derived from P by replacing (in a
capture-avoiding manner) each occurrence of $x$ in $S$ by $y$. For example,

\begin{mathpar}
  P\{ \quotep{\procn{x}|\procn{x}} / x : x \in \freenames{P} \}
\end{mathpar}

will replace each (occurrence) of a free name $x$ in $P$ by
$\quotep{\procn{x}|\procn{x}}$.

Also, we will avail ourselves of the notation $x^{L}$ and $x^{R}$ to
denote injections of a name into disjoint copies of the name
space. There are numerous ways to accomplish this. One example can be
found in \cite{MeredithR05}. This notation overloads to vectors of
names: $\vec{x}^{\pi} := (x_{i}^{\pi} \; : \; 0 \leq i < |\vec{x}| )$ where $\pi \in \{L,R\}$.

We also use $P^{\Box} := P|\Box$.

In \cite{MeredithR05} an interpretation of the new operator is
given. It turns out that there are several possible interpretations
all enjoying the requisite algebraic properties of the operator (see
\cite{milner91polyadicpi}). We will therefore make liberal use of
$(\nu\; \vec{x})P$.

% subsection the_syntax_and_semantics_of_the_notation_system (end)   

\input{qm2pi.qmops} 

\input{qm2pi.sterngerlach} 

\input{qm2pi.metric} 

% section concurrent_process_calculi (end)

%\input{qm2pi.proofsketch}

% section proof sketch (end)

%\input{qm2pi.slviaknots} 

% section spatial logic via knots (end)

\input{qm2pi.conclusion}

% section conclusion (end)

%\input{qm2pi.dtcodes} 

% section wiring algorithm (end)

\input{qm2pi.ack} 

% section acknowledgments (end)

\newpage


\bibliographystyle{plain}   
\bibliography{../../biblios/main.bib}

\input{qm2pi.rhodetails}

\end{document}

 

%\documentclass[12pt]{llncs}
%\documentclass{jktr}

\usepackage[pdftex]{hyperref}                   
\usepackage {listings}
\usepackage {mathpartir}
\usepackage{bcprules}
%\usepackage{listings}
                       
\usepackage{graphicx} 
%\usepackage[margins=2.5cm,nohead,nofoot]{geometry}
%\usepackage{geometry}
\usepackage{amsfonts}
\usepackage{amstext}
\usepackage{latexsym}
\usepackage{amssymb}
\usepackage{color}


%\include{myPreamble}
\include{qm2pi.local} 

%\ifpdf
%\usepackage[pdftex]{graphicx}
%\else
%\usepackage{graphicx}
%\fi

 % \ifpdf
%  \usepackage{pdfsync}
%  \if


%\title{Brief Article}
%\author{David F. Snyder}
%\author{L.G. Meredith}

%\address{Dept. of Math., Texas State University--San Marcos, San Marcos, TX 78666}
       
\pagestyle{empty}


\begin{document}

\lstset{language=[Objective]Caml,frame=shadowbox}

\input{qm2pi.front}

% section front matter (end)

\input{qm2pi.intro} 
 
% section introduction (end)

% \input{qm2pi.knotations} 

% section notation (end)

\input{qm2pi.process.calculi} 

% section concurrent_process_calculi_and_spatial_logics_ (end)
    
%\input{qm2pi.knots2pi} 

%\input{qm2pi.trefoil} 

%\input{qm2pi.mainthm} 

% subsection basic_interpretation (end)

%\input{qm2pi.rho.presentation} 
\subsection{The syntax and semantics of the notation system}\label{sub:the_syntax_and_semantics_of_the_notation_system} % (fold)

We now summarize a technical presentation of the calculus that
embodies our theory of dynamics. The typical presentation of such a
calculus follows the style of giving generators and relations on
them. The grammar, below, describing term constructors, freely
generates the set of processes, $\Proc$. This set is then quotiented
by a relation known as structural congruence and it is over this set
that the notion of dynamics is expressed. This presentation is
essentially that of \cite{MeredithR05} with the addition of
polyadicity and summation. For readability we have relegated some of
the technical subtleties to an appendix.

\subsubsection{Process grammar}\label{subsub:process_grammar}

\begin{mathpar}
  \inferrule* [lab=synchronization] {} {{M} \bc \pzero \;|\; x?F \;|\; x!C }
  \and
  \inferrule* [lab=abstraction] {} {{F} \bc (x)P}
  \and
  \inferrule* [lab=concretion] {} {{C} \bc \langle Q \rangle}
  \and
  \inferrule* [lab=process] {} {{P,Q} \bc M \;| \;P|Q \;|\; @{x}}
  \and
  \inferrule* [lab=name] {} {{x} \bc \quotep{P}}
\end{mathpar} 

Note that $\vec{x}$ (resp. $\vec{P}$) denotes a vector of names
(resp. processes) of length $|\vec{x}|$ (resp. $|\vec{P}|$). We adopt
the following useful abbreviations.

\begin{mathpar}
   x?(\vec{y}).P := x.(\vec{y})P \and  x\clift{\vec{P}} := x.\clift{\vec{P}}
   \and x!(y) := \lift{x}{\dropn{y}}
   \and \Pi_{i=0}^{n-1}P_i := P_0 | \ldots | P_{n-1}
\end{mathpar}

\subsubsection{Structural congruence}

\paragraph{Free and bound names and alpha-equivalence.} At the
core of structural equivalence is alpha-equivalence which identifies
process that are the same up to a change of variable. Formally, we
recognize the distinction between free and bound names. The free names
of a process, $\freenames{P}$, may be calculated recursively as
follows:

\begin{mathpar}
\freenames{\pzero} := \emptyset
  \and \\
  \freenames{x?(y).P} := \{ x \} \cup (\freenames{P} \setminus \{ y \})
  \and 
  \freenames{x!\langle P \rangle} := \{ x \} \cup \{ P \} 
  \and \\
  \freenames{P|Q} := \freenames{P} \cup \freenames{Q}
  \and \\
  \freenames{@{x}} := \{ x \}
\end{mathpar}

$\pi$
$\quotep{\pi}$

$\freenames{-} : \pi \to \mathcal{P}(\quotep{\pi})$

\begin{eqnarray*}
  \freenames{\pzero} & := & \emptyset \\
  \freenames{x?(y).P} & := & \{ x \} \cup (\freenames{P} \setminus \{ y \}) \\
  \freenames{x!\langle P \rangle} & := & \{ x \} \cup \{ P \} \\
  \freenames{P|Q} & := & \freenames{P} \cup \freenames{Q} \\
  \freenames{\dropn{x}} & := & \{ x \}
\end{eqnarray*}

The bound names of a process, $\boundnames{P}$, are those names occurring in $P$
that are not free. For example, in $x?(y).0$, the name $x$ is free, while $y$ is bound.

\begin{mathpar}
  \inferrule* [lab=monoidal-laws] {} { P|Q \equiv Q|P \and P|0 \equiv P \and P|(Q|R) \equiv (P|Q)|R }
\end{mathpar}

\begin{mathpar}
  \inferrule* [lab=alpha-equivalence] {} { (x)P \equiv (y)P\{y/x\} \and y \not\in \freenames{P} }
\end{mathpar}

\begin{definition}
Then two processes, $P,Q$, are alpha-equivalent if $P = Q\{\vec{y}/\vec{x}\}$ for
some $\vec{x} \in \boundnames{Q},\vec{y} \in \boundnames{P}$, where $Q\{\vec{y}/\vec{x}\}$
denotes the capture-avoiding substitution of $\vec{y}$ for $\vec{x}$ in $Q$.
\end{definition}

\begin{definition}
  The {\em structural congruence} \cite{SangiorgiWalker} , $\equiv$,
  between processes is the least congruence containing
  alpha-equivalence, satisfying the abelian monoid laws
  (associativity, commutativity and $\pzero$ as identity) for parallel
  composition $|$ and for summation $+$.
\end{definition}

\subsection{Name equivalence}

We take name equivalence, written $\nameeq$, to be the smallest
equivalence relation generated by the following rules.

\begin{mathpar}
\inferrule*[lab=Quote-drop]
{ }
{ \quotep{@{x}} \nameeq x }

\inferrule*[lab=Struct-equiv]
{ P \scong Q }
{ \quotep{P} \nameeq \quotep{Q} }
\end{mathpar}

The astute reader will have noticed that the mutual recursion of names
and processes imposes a mutual recursion on alpha-equivalence and
structural equivalence via name-equivalence. Fortunately, all of this
works out pleasantly and we may calculate in the natural way, free of
concern. The reader interested in the details is referred to the
appendix \ref{appendix:rho_details}.

\subsection{Substitution}

We use $\Proc$ for the set of processes, $\QProc$ for the set of
names, and $\id{\{}\vec{y} / \vec{x} \id{\}}$ to denote partial maps,
$s : \QProc \rightarrow \QProc$. A map, $s$ lifts, uniquely, to a map
on process terms, $\widehat{s} : \Proc \rightarrow \Proc$ by the
following equations.

\begin{mathpar}
  (0) \psubstp{Q}{P} := 0 \\
  (R \juxtap S) \psubstp{Q}{P}
  :=    
  (R)\psubstp{Q}{P} \juxtap (S) \psubstp{Q}{P} \\
  (x?(y).R) \psubstp{Q}{P}    
  :=    
  (x)\substp{Q}{P} (z)\concat( (R \psubstn{z}{y}) \psubstp{Q}{P} ) \\
  (\lift{x}{R}) \psubstp{Q}{P}  
  :=
  \lift{(x)\substp{Q}{P}}{ R \psubstp{Q}{P} } \\
%   (\dropn{x})  \psubstp{Q}{P}       
%   := 
%   \left\{ 
%     \begin{array}{ccc} 
%       \dropn{\quotep{Q}} & & x \nameeq \quotep{P} \\
%       \dropn{x} & & otherwise \\
%     \end{array}
%   \right. 
  (\dropn{x})  \psubstp{Q}{P}       
  := 
  \left\{ 
    \begin{array}{ccc} 
      Q & & x \nameeq \quotep{P} \\
      \dropn{x} & & otherwise \\
    \end{array}
  \right.
\end{mathpar}
 

where

\begin{eqnarray}
  (x)\id{\{} \lpquote Q \rpquote / \lpquote P \rpquote \id{\}}            = 
  \left\{ 
    \begin{array}{ccc}
      \lpquote Q \rpquote & & x \nameeq \lpquote P \rpquote \\
      x & & otherwise \\
    \end{array}
  \right. \nonumber
\end{eqnarray}

and $z$ is chosen distinct from $\quotep{P}$, $\quotep{Q}$, the free
names in $Q$, and all the names in $R$. Our $\alpha$-equivalence will
be built in the standard way from this substitution.

\begin{remark}\label{rem:no_self_referential_names}
  One consequence of these definitions is that $\forall P. \quotep{P}
  \not\in \freenames{P}$.
\end{remark}

\subsection{ Dynamic quote: an example }

Anticipating something of what's to come, consider applying the
substitution, $\widehat{\id{\{}u / z \id{\}}}$, to the following pair
of processes, $\lift{w}{y!(z)}$ and $w[ \lpquote y!(z) \rpquote ]$.

\begin{eqnarray}
	\lift{w}{y!(z)}\widehat{\id{\{}u / z \id{\}}}
		& = &
		\lift{w}{y!(u)} \nonumber\\
	w[ \lpquote y!(z) \rpquote ] \widehat{ \id{\{}u / z \id{\}} }
		& = &
		w[ \lpquote y!(z) \rpquote ] \nonumber
\end{eqnarray}

Because the body of the process between quotes is impervious to
substitution, we get radically different answers. In fact, by
examining the first process in an input context,
e.g. $x?(z).\lift{w}{y!(z)}$, we see that the process under the lift
operator may be shaped by prefixed inputs binding a name inside it. In
this sense, the lift operator will be seen as a way to dynamically
construct processes before reifying them as names.

Finally equipped with these standard features we can present the
dynamics of the calculus.

\subsubsection{Operational semantics} 

Finally, we introduce the computational dynamics. What marks these
algebras as distinct from other more traditionally studied algebraic
structures, e.g. vector spaces or polynomial rings, is the manner in
which dynamics is captured. In traditional structures, dynamics is typically
expressed through morphisms between such structures, as in linear maps
between vector spaces or morphisms between rings. In algebras
associated with the semantics of computation, the dynamics is
expressed as part of the algebraic structure itself, through a
reduction reduction relation typically denoted by $\red$. Below, we
give a recursive presentation of this relation for the calculus used
in the encoding.

$\red \subseteq \pi \times \pi$
$\red : \pi \to \mathcal{P}(\pi)$

\begin{mathpar}
  \inferrule* [lab=Comm] { \textsf{match}( x_{src}, x_{trgt} ) } { x_{trgt}?(y)P \; | \; x_{src}!\langle {Q} \rangle \red P\{\quotep{Q}/y}\} }
  \and \\
  \inferrule* [lab=Par] {{P} \red {P}'} {{{P} | {Q}} \red {{P}' | {Q}}}
  \and
  \inferrule* [lab=Equiv]{{{P} \scong {P}'} \andalso {{P}' \red {Q}'} \andalso {{Q}' \scong {Q}}}{{P} \red {Q}}
\end{mathpar}

\begin{eqnarray*}
  match_{\equiv} (\quotep{P},\quotep{Q}) & := & P \equiv Q \\
  match_{\dagger}(\quotep{P},\quotep{Q}) & := & \forall R. P|Q \red^{*} R => R \red^{*} 0 \\
  match_{K}(\quotep{P},\quotep{Q}) & := & K \mbox{ for some context } K
\end{eqnarray*}

$u?(x)P | u!\langle Q \rangle \red P\{\quotep{Q}/x\}$

%We write $\wred$ for $\red^*$, and $P\red$ if $\exists Q $ such that $ P \red Q$.
We write $P\red$ if $\exists Q $ such that $ P \red Q$ and $P\not\red$, otherwise.

\section{Replication}

As mentioned before, it is known that replication (and hence
recursion) can be implemented in a higher-order process algebra
\cite{SangiorgiWalker}. As our first example of calculation with the
machinery thus far presented we give the construction explicitly in
the {\rhoc}.

\begin{eqnarray}
	D_{x} & := & \prefix{x}{y}{(\binpar{\outputp{x}{y}}{@{y}})} \nonumber\\
	\bangp_{x}{P} & := & \binpar{{x}!\langle{\binpar{D_{x}}{P}}\rangle}{D_{x}} \nonumber
\end{eqnarray}

\begin{eqnarray}
	\bangp_{x}{P} & & \nonumber\\
	=
	& {x}!\langle{(\prefix{x}{y}{(\outputp{x}{y} | @{y})) | P}}\rangle 
	      | \prefix{x}{y}{(\outputp{x}{y} | @{y})} & \nonumber\\
	\red
	& (\outputp{x}{y} | @{y})\substn{\quotep{(\prefix{x}{y}{(@{y} | \outputp{x}{y})) | P}}}{y} & \nonumber\\
	=
	& \outputp{x}{\quotep{(\prefix{x}{y}{(\outputp{x}{y} | @{y})) | P}}}
	  | {(\prefix{x}{y}{(\outputp{x}{y} | @{y})) | P}} & \nonumber\\
	\red
	& \ldots & \nonumber\\
	\red^*
	& P | P | \ldots & \nonumber
\end{eqnarray}

Of course, this encoding, as an implementation, runs away, unfolding
$\bangp{P}$ eagerly. A lazier and more implementable replication
operator, restricted to input-guarded processes, may be obtained as follows.

\begin{eqnarray}
\bangp{\prefix{u}{v}{P}} 
	:= 
	\binpar{\lift{x}{\prefix{u}{v}{(\binpar{D(x)}{P})}}}{D(x)} \nonumber
\end{eqnarray}

\begin{remark}
  Note that the lazier definition still does not deal with summation
  or mixed summation (i.e. sums over input and output). The reader is
  invited to construct definitions of replication that deal with these
  features. 

  Further, the definitions are parameterized in a name, $x$. Can you,
  gentle reader, make a definition that eliminates this parameter and
  guarantees no accidental interaction between the replication
  machinery and the process being replicated -- i.e. no accidental
  sharing of names used by the process to get its work done and the
  name(s) used by the replication to effect copying. This latter
  revision of the definition of replication is crucial to obtaining
  the expected identity $!!P \sim !P$.
\end{remark}

\begin{remark}\label{rem:paradoxical_combinator}
  The reader familiar with the lambda calculus will have noticed the
  similarity between $D$ and the paradoxical combinator.

  [Ed. note: the existence of this seems to suggest we have to be more
  restrictive on the set of processes and names we admit if we are to
  support no-cloning.]
\end{remark}

\subsubsection{Bisimulation}

The computational dynamics gives rise to another kind of equivalence,
the equivalence of computational behavior. As previously mentioned
this is typically captured \emph{via} some form of bisimulation.

% The notion we use in this paper is weak barbed bisimulation
% \cite{milner91polyadicpi}.

The notion we use in this paper is derived from weak barbed
bisimulation \cite{milner91polyadicpi}. 

\begin{definition}
An \emph{observation relation}, $\downarrow_{\mathcal N}$, over a set
of names, $\mathcal N$, is the smallest relation satisfying the rules
below.

\infrule[Out-barb]{y \in {\mathcal N}, \; x \nameeq y}
		  {\outputp{x}{v} \downarrow_{\mathcal N} x}
\infrule[Par-barb]{\mbox{$P\downarrow_{\mathcal N} x$ or $Q\downarrow_{\mathcal N} x$}}
		  {\binpar{P}{Q} \downarrow_{\mathcal N} x}

We write $P \Downarrow_{\mathcal N} x$ if there is $Q$ such that 
$P \wred Q$ and $Q \downarrow_{\mathcal N} x$.
\end{definition}

\begin{definition}
%\label{def.bbisim}
An  ${\mathcal N}$-\emph{barbed bisimulation} over a set of names, ${\mathcal N}$, is a symmetric binary relation 
${\mathcal S}_{\mathcal N}$ between agents such that $P\rel{S}_{\mathcal N}Q$ implies:
\begin{enumerate}
\item If $P \red P'$ then $Q \wred Q'$ and $P'\rel{S}_{\mathcal N} Q'$.
\item If $P\downarrow_{\mathcal N} x$, then $Q\Downarrow_{\mathcal N} x$.
\end{enumerate}
$P$ is ${\mathcal N}$-barbed bisimilar to $Q$, written
$P \wbbisim_{\mathcal N} Q$, if $P \rel{S}_{\mathcal N} Q$ for some ${\mathcal N}$-barbed bisimulation ${\mathcal S}_{\mathcal N}$.
\end{definition}

$\mathcal{R} \subseteq \pi \times \pi$

$P \mathcal{R} Q => \forall P'. P \red P' \Rightarrow \exists Q'. Q \red Q', P' \mathcal{R} Q'$

$P \vdash x \Rightarrow Q \vdash x$

\begin{mathpar}
  \inferrule*[lab=Out-barb]{x \nameeq y}{{y}!\langle{Q}\rangle \vdash x}
  \and
  \inferrule*[lab=Par-barb]{\mbox{$P\vdash x$ or $Q\vdash x$}}{\binpar{P}{Q} \vdash x}
\end{mathpar}

\subsubsection{Contexts}

One of the principle advantages of computational calculi like the
$\pi$-calculus is a well-defined notion of context,
contextual-equivalence and a correlation between
contextual-equivalence and notions of bisimulation. The notion of
context allows the decomposition of a process into (sub-)process and
its syntactic environment, its context. Thus, a context may be
thought of as a process with a ``hole'' (written $\Box$) in it. The
application of a context $M$ to a process $P$, written $M[P]$, is
tantamount to filling the hole in $M$ with $P$. In this paper we do
not need the full weight of this theory, but do make use of the notion
of context in the proof the main theorem. 

\begin{mathpar}
  \inferrule* [lab=summation] {} {{M_{M},M_{N}} \bc \Box \;|\; x.M_{A} \;|\; M_{M}+M_{N}}
  \and
  \inferrule* [lab=agent] {} {{M_{A}} \bc (\vec{x})M_{P} \;| \; \clift{P_0,\ldots,M_{P},\ldots,P_N}}
  \and \\
  \inferrule* [lab=process] {} {{M_{P}} \bc M_{N} \;| \;P|M_{P} }
\end{mathpar} 

\begin{mathpar}
  \inferrule* [lab=sychronization] {} {M_{N} \bc \Box \;|\; x?M_{F} \;|\; x!M_{C}}
  \and
  \inferrule* [lab=abstraction] {} {{M_{F}} \bc (x)M_{P} }
  \and
  \inferrule* [lab=concretion] {} {{M_{C}} \bc \langle M_{P} \rangle }
  \and \\
  \inferrule* [lab=process] {} {{M_{P}} \bc M_{N} \;| \;P|M_{P} }
\end{mathpar}

\begin{definition}[contextual application] Given a context $M$, and
  process $P$, we define the \emph{contextual application}, $M[P] :=
  M\{P/\Box\}$. That is, the contextual application of M to P is the
  substitution of $P$ for $\Box$ in $M$.
\end{definition}

$\meaningof{-} : L \to \mathcal{P}(\pi)$

\begin{mathpar}
  \inferrule* [lab=collection] {} {\meaningof{true} = \pi, \and \meaningof{~E} = \pi \setminus \meaningof{E}, \and \meaningof{E_{1} \& E_{2}} = \meaningof{E_{1}} \cap \meaningof{E_{2}}}
\end{mathpar}

\begin{mathpar}
  \inferrule* [lab=structure] {} {\meaningof{0} = \{ P \in \pi | P \equiv 0 \}, \and \\ \meaningof{E_1 | E_2} = \{ P \in \pi | P \equiv P_{1} | P_{2}, P_{1} \in \meaningof{E_{1}}, P_{2} \in \meaningof{E_2}\} }
\end{mathpar}

\begin{mathpar}
 \inferrule* [lab=behavior] {} {\meaningof{\langle a?b \rangle E} = \{ P \in \pi | P \equiv Q | u?(y)P', \\ \and \\\\ \and \\ \;\;\; u \in \meaningof{a}, \forall z.P'\{z/y\} \in \meaningof{E\{z/b\}}\}, \and \\ \meaningof{a!E} = \{ P \in \pi | P \equiv Q | x!\langle P' \rangle, x \in \meaningof{a} P' \in \meaningof{E}\} }
\end{mathpar}

\begin{mathpar}
 \inferrule* [lab=nominal] {} {\meaningof{\quotep{E}} = \{ \quotep{P} \in \quotep{\pi} | P \in \meaningof{E} \}, \and \meaningof{\quotep{P}} = \{ \quotep{Q} \in \quotep{\pi} | P \equiv Q \} \and \\ \meaningof{@\quotep{E}} = \{ P \in \pi | P \equiv @x, x \in \meaningof{E} \}}
\end{mathpar}

\begin{eqnarray*}
  \\
  \meaningof{-} : TS \to ST
\end{eqnarray*}

\begin{eqnarray*}
  \\
  L : TS \to ST
\end{eqnarray*}

\begin{eqnarray*}
  \\
  P \models E \iff P \in \meaningof{E}
\end{eqnarray*}

\begin{eqnarray*}
  P \approx_{L} Q \iff \forall E \in L. P \models E \iff Q \models E
\end{eqnarray*}

\begin{eqnarray*}
  P \approx_{K} Q
\end{eqnarray*}

\begin{eqnarray*}
  P \approx Q
\end{eqnarray*}

$\approx_{K} = \approx = \approx_{L}$

\subsubsection{Contextual duality}

Note that contexts extend the quotation operation to a family of
operations from processes to names. Given a context, $M$, we can
define a \emph{nominal context}, $\quotep{M}$ by $\quotep{M}[P] :=
\quotep{M[P]}$. To foreshadow what is to come we observe that these
operations enjoy a duality with processes very much like the duality
between vectors and maps from vectors to scalars.

Further, because the calculus is essentially higher-order, we have a
correspondence between contexts and processes. More specifically,
given a name $x$ and a context $M$ we can construct $M^{*}_{x}$ such
that 

\begin{mathpar}
  M^{*}_{x} | \lift{x}{P} \red M[P]
\end{mathpar}

namely,

\begin{mathpar}
  M^{*}_{x} := x?(u).M[\dropn{u}]
\end{mathpar}

The dependence of $M^{*}_{x}$ on a name makes it an abstraction, 

\begin{mathpar}
  M^{*} := (x)x?(u).M[\dropn{u}]
\end{mathpar}

\subsection{Additional notation}

It will sometimes be convenient to denote the process a name
quotes. We already have the notation $x = \quotep{P}$, but it will be
convenient to introduce an alternate notation, $\procn{x}$, when we
want to emphasize the connection to the use of the name. Note that, by
virtue of name equivalence, $\quotep{\procn{x}} \nameeq x$; so, the
notation is consistent with previous definitions.

Further, because names have structure it is possible to effect
substitutions on the basis of that structure. This means we need to
upgrade our notation for substitutions, which we accomplish by
adapting comprehension notation. Thus,

\begin{mathpar}
  P\{ y / x : x \in S \}
\end{mathpar}

is interpreted to mean the process derived from P by replacing (in a
capture-avoiding manner) each occurrence of $x$ in $S$ by $y$. For example,

\begin{mathpar}
  P\{ \quotep{\procn{x}|\procn{x}} / x : x \in \freenames{P} \}
\end{mathpar}

will replace each (occurrence) of a free name $x$ in $P$ by
$\quotep{\procn{x}|\procn{x}}$.

Also, we will avail ourselves of the notation $x^{L}$ and $x^{R}$ to
denote injections of a name into disjoint copies of the name
space. There are numerous ways to accomplish this. One example can be
found in \cite{MeredithR05}. This notation overloads to vectors of
names: $\vec{x}^{\pi} := (x_{i}^{\pi} \; : \; 0 \leq i < |\vec{x}| )$ where $\pi \in \{L,R\}$.

We also use $P^{\Box} := P|\Box$.

In \cite{MeredithR05} an interpretation of the new operator is
given. It turns out that there are several possible interpretations
all enjoying the requisite algebraic properties of the operator (see
\cite{milner91polyadicpi}). We will therefore make liberal use of
$(\nu\; \vec{x})P$.

% subsection the_syntax_and_semantics_of_the_notation_system (end)   

\input{qm2pi.qmops} 

\input{qm2pi.sterngerlach} 

\input{qm2pi.metric} 

% section concurrent_process_calculi (end)

%\input{qm2pi.proofsketch}

% section proof sketch (end)

%\input{qm2pi.slviaknots} 

% section spatial logic via knots (end)

\input{qm2pi.conclusion}

% section conclusion (end)

%\input{qm2pi.dtcodes} 

% section wiring algorithm (end)

\input{qm2pi.ack} 

% section acknowledgments (end)

\newpage


\bibliographystyle{plain}   
\bibliography{../../biblios/main.bib}

\input{qm2pi.rhodetails}

\end{document}

 

% subsection basic_interpretation (end)

%\input{qm2pi.rho.presentation} 
\subsection{The syntax and semantics of the notation system}\label{sub:the_syntax_and_semantics_of_the_notation_system} % (fold)

We now summarize a technical presentation of the calculus that
embodies our theory of dynamics. The typical presentation of such a
calculus follows the style of giving generators and relations on
them. The grammar, below, describing term constructors, freely
generates the set of processes, $\Proc$. This set is then quotiented
by a relation known as structural congruence and it is over this set
that the notion of dynamics is expressed. This presentation is
essentially that of \cite{MeredithR05} with the addition of
polyadicity and summation. For readability we have relegated some of
the technical subtleties to an appendix.

\subsubsection{Process grammar}\label{subsub:process_grammar}

\begin{mathpar}
  \inferrule* [lab=synchronization] {} {{M} \bc \pzero \;|\; x?F \;|\; x!C }
  \and
  \inferrule* [lab=abstraction] {} {{F} \bc (x)P}
  \and
  \inferrule* [lab=concretion] {} {{C} \bc \langle Q \rangle}
  \and
  \inferrule* [lab=process] {} {{P,Q} \bc M \;| \;P|Q \;|\; @{x}}
  \and
  \inferrule* [lab=name] {} {{x} \bc \quotep{P}}
\end{mathpar} 

Note that $\vec{x}$ (resp. $\vec{P}$) denotes a vector of names
(resp. processes) of length $|\vec{x}|$ (resp. $|\vec{P}|$). We adopt
the following useful abbreviations.

\begin{mathpar}
   x?(\vec{y}).P := x.(\vec{y})P \and  x\clift{\vec{P}} := x.\clift{\vec{P}}
   \and x!(y) := \lift{x}{\dropn{y}}
   \and \Pi_{i=0}^{n-1}P_i := P_0 | \ldots | P_{n-1}
\end{mathpar}

\subsubsection{Structural congruence}

\paragraph{Free and bound names and alpha-equivalence.} At the
core of structural equivalence is alpha-equivalence which identifies
process that are the same up to a change of variable. Formally, we
recognize the distinction between free and bound names. The free names
of a process, $\freenames{P}$, may be calculated recursively as
follows:

\begin{mathpar}
\freenames{\pzero} := \emptyset
  \and \\
  \freenames{x?(y).P} := \{ x \} \cup (\freenames{P} \setminus \{ y \})
  \and 
  \freenames{x!\langle P \rangle} := \{ x \} \cup \{ P \} 
  \and \\
  \freenames{P|Q} := \freenames{P} \cup \freenames{Q}
  \and \\
  \freenames{@{x}} := \{ x \}
\end{mathpar}

$\pi$
$\quotep{\pi}$

$\freenames{-} : \pi \to \mathcal{P}(\quotep{\pi})$

\begin{eqnarray*}
  \freenames{\pzero} & := & \emptyset \\
  \freenames{x?(y).P} & := & \{ x \} \cup (\freenames{P} \setminus \{ y \}) \\
  \freenames{x!\langle P \rangle} & := & \{ x \} \cup \{ P \} \\
  \freenames{P|Q} & := & \freenames{P} \cup \freenames{Q} \\
  \freenames{\dropn{x}} & := & \{ x \}
\end{eqnarray*}

The bound names of a process, $\boundnames{P}$, are those names occurring in $P$
that are not free. For example, in $x?(y).0$, the name $x$ is free, while $y$ is bound.

\begin{mathpar}
  \inferrule* [lab=monoidal-laws] {} { P|Q \equiv Q|P \and P|0 \equiv P \and P|(Q|R) \equiv (P|Q)|R }
\end{mathpar}

\begin{mathpar}
  \inferrule* [lab=alpha-equivalence] {} { (x)P \equiv (y)P\{y/x\} \and y \not\in \freenames{P} }
\end{mathpar}

\begin{definition}
Then two processes, $P,Q$, are alpha-equivalent if $P = Q\{\vec{y}/\vec{x}\}$ for
some $\vec{x} \in \boundnames{Q},\vec{y} \in \boundnames{P}$, where $Q\{\vec{y}/\vec{x}\}$
denotes the capture-avoiding substitution of $\vec{y}$ for $\vec{x}$ in $Q$.
\end{definition}

\begin{definition}
  The {\em structural congruence} \cite{SangiorgiWalker} , $\equiv$,
  between processes is the least congruence containing
  alpha-equivalence, satisfying the abelian monoid laws
  (associativity, commutativity and $\pzero$ as identity) for parallel
  composition $|$ and for summation $+$.
\end{definition}

\subsection{Name equivalence}

We take name equivalence, written $\nameeq$, to be the smallest
equivalence relation generated by the following rules.

\begin{mathpar}
\inferrule*[lab=Quote-drop]
{ }
{ \quotep{@{x}} \nameeq x }

\inferrule*[lab=Struct-equiv]
{ P \scong Q }
{ \quotep{P} \nameeq \quotep{Q} }
\end{mathpar}

The astute reader will have noticed that the mutual recursion of names
and processes imposes a mutual recursion on alpha-equivalence and
structural equivalence via name-equivalence. Fortunately, all of this
works out pleasantly and we may calculate in the natural way, free of
concern. The reader interested in the details is referred to the
appendix \ref{appendix:rho_details}.

\subsection{Substitution}

We use $\Proc$ for the set of processes, $\QProc$ for the set of
names, and $\id{\{}\vec{y} / \vec{x} \id{\}}$ to denote partial maps,
$s : \QProc \rightarrow \QProc$. A map, $s$ lifts, uniquely, to a map
on process terms, $\widehat{s} : \Proc \rightarrow \Proc$ by the
following equations.

\begin{mathpar}
  (0) \psubstp{Q}{P} := 0 \\
  (R \juxtap S) \psubstp{Q}{P}
  :=    
  (R)\psubstp{Q}{P} \juxtap (S) \psubstp{Q}{P} \\
  (x?(y).R) \psubstp{Q}{P}    
  :=    
  (x)\substp{Q}{P} (z)\concat( (R \psubstn{z}{y}) \psubstp{Q}{P} ) \\
  (\lift{x}{R}) \psubstp{Q}{P}  
  :=
  \lift{(x)\substp{Q}{P}}{ R \psubstp{Q}{P} } \\
%   (\dropn{x})  \psubstp{Q}{P}       
%   := 
%   \left\{ 
%     \begin{array}{ccc} 
%       \dropn{\quotep{Q}} & & x \nameeq \quotep{P} \\
%       \dropn{x} & & otherwise \\
%     \end{array}
%   \right. 
  (\dropn{x})  \psubstp{Q}{P}       
  := 
  \left\{ 
    \begin{array}{ccc} 
      Q & & x \nameeq \quotep{P} \\
      \dropn{x} & & otherwise \\
    \end{array}
  \right.
\end{mathpar}
 

where

\begin{eqnarray}
  (x)\id{\{} \lpquote Q \rpquote / \lpquote P \rpquote \id{\}}            = 
  \left\{ 
    \begin{array}{ccc}
      \lpquote Q \rpquote & & x \nameeq \lpquote P \rpquote \\
      x & & otherwise \\
    \end{array}
  \right. \nonumber
\end{eqnarray}

and $z$ is chosen distinct from $\quotep{P}$, $\quotep{Q}$, the free
names in $Q$, and all the names in $R$. Our $\alpha$-equivalence will
be built in the standard way from this substitution.

\begin{remark}\label{rem:no_self_referential_names}
  One consequence of these definitions is that $\forall P. \quotep{P}
  \not\in \freenames{P}$.
\end{remark}

\subsection{ Dynamic quote: an example }

Anticipating something of what's to come, consider applying the
substitution, $\widehat{\id{\{}u / z \id{\}}}$, to the following pair
of processes, $\lift{w}{y!(z)}$ and $w[ \lpquote y!(z) \rpquote ]$.

\begin{eqnarray}
	\lift{w}{y!(z)}\widehat{\id{\{}u / z \id{\}}}
		& = &
		\lift{w}{y!(u)} \nonumber\\
	w[ \lpquote y!(z) \rpquote ] \widehat{ \id{\{}u / z \id{\}} }
		& = &
		w[ \lpquote y!(z) \rpquote ] \nonumber
\end{eqnarray}

Because the body of the process between quotes is impervious to
substitution, we get radically different answers. In fact, by
examining the first process in an input context,
e.g. $x?(z).\lift{w}{y!(z)}$, we see that the process under the lift
operator may be shaped by prefixed inputs binding a name inside it. In
this sense, the lift operator will be seen as a way to dynamically
construct processes before reifying them as names.

Finally equipped with these standard features we can present the
dynamics of the calculus.

\subsubsection{Operational semantics} 

Finally, we introduce the computational dynamics. What marks these
algebras as distinct from other more traditionally studied algebraic
structures, e.g. vector spaces or polynomial rings, is the manner in
which dynamics is captured. In traditional structures, dynamics is typically
expressed through morphisms between such structures, as in linear maps
between vector spaces or morphisms between rings. In algebras
associated with the semantics of computation, the dynamics is
expressed as part of the algebraic structure itself, through a
reduction reduction relation typically denoted by $\red$. Below, we
give a recursive presentation of this relation for the calculus used
in the encoding.

$\red \subseteq \pi \times \pi$
$\red : \pi \to \mathcal{P}(\pi)$

\begin{mathpar}
  \inferrule* [lab=Comm] { \textsf{match}( x_{src}, x_{trgt} ) } { x_{trgt}?(y)P \; | \; x_{src}!\langle {Q} \rangle \red P\{\quotep{Q}/y}\} }
  \and \\
  \inferrule* [lab=Par] {{P} \red {P}'} {{{P} | {Q}} \red {{P}' | {Q}}}
  \and
  \inferrule* [lab=Equiv]{{{P} \scong {P}'} \andalso {{P}' \red {Q}'} \andalso {{Q}' \scong {Q}}}{{P} \red {Q}}
\end{mathpar}

\begin{eqnarray*}
  match_{\equiv} (\quotep{P},\quotep{Q}) & := & P \equiv Q \\
  match_{\dagger}(\quotep{P},\quotep{Q}) & := & \forall R. P|Q \red^{*} R => R \red^{*} 0 \\
  match_{K}(\quotep{P},\quotep{Q}) & := & K \mbox{ for some context } K
\end{eqnarray*}

$u?(x)P | u!\langle Q \rangle \red P\{\quotep{Q}/x\}$

%We write $\wred$ for $\red^*$, and $P\red$ if $\exists Q $ such that $ P \red Q$.
We write $P\red$ if $\exists Q $ such that $ P \red Q$ and $P\not\red$, otherwise.

\section{Replication}

As mentioned before, it is known that replication (and hence
recursion) can be implemented in a higher-order process algebra
\cite{SangiorgiWalker}. As our first example of calculation with the
machinery thus far presented we give the construction explicitly in
the {\rhoc}.

\begin{eqnarray}
	D_{x} & := & \prefix{x}{y}{(\binpar{\outputp{x}{y}}{@{y}})} \nonumber\\
	\bangp_{x}{P} & := & \binpar{{x}!\langle{\binpar{D_{x}}{P}}\rangle}{D_{x}} \nonumber
\end{eqnarray}

\begin{eqnarray}
	\bangp_{x}{P} & & \nonumber\\
	=
	& {x}!\langle{(\prefix{x}{y}{(\outputp{x}{y} | @{y})) | P}}\rangle 
	      | \prefix{x}{y}{(\outputp{x}{y} | @{y})} & \nonumber\\
	\red
	& (\outputp{x}{y} | @{y})\substn{\quotep{(\prefix{x}{y}{(@{y} | \outputp{x}{y})) | P}}}{y} & \nonumber\\
	=
	& \outputp{x}{\quotep{(\prefix{x}{y}{(\outputp{x}{y} | @{y})) | P}}}
	  | {(\prefix{x}{y}{(\outputp{x}{y} | @{y})) | P}} & \nonumber\\
	\red
	& \ldots & \nonumber\\
	\red^*
	& P | P | \ldots & \nonumber
\end{eqnarray}

Of course, this encoding, as an implementation, runs away, unfolding
$\bangp{P}$ eagerly. A lazier and more implementable replication
operator, restricted to input-guarded processes, may be obtained as follows.

\begin{eqnarray}
\bangp{\prefix{u}{v}{P}} 
	:= 
	\binpar{\lift{x}{\prefix{u}{v}{(\binpar{D(x)}{P})}}}{D(x)} \nonumber
\end{eqnarray}

\begin{remark}
  Note that the lazier definition still does not deal with summation
  or mixed summation (i.e. sums over input and output). The reader is
  invited to construct definitions of replication that deal with these
  features. 

  Further, the definitions are parameterized in a name, $x$. Can you,
  gentle reader, make a definition that eliminates this parameter and
  guarantees no accidental interaction between the replication
  machinery and the process being replicated -- i.e. no accidental
  sharing of names used by the process to get its work done and the
  name(s) used by the replication to effect copying. This latter
  revision of the definition of replication is crucial to obtaining
  the expected identity $!!P \sim !P$.
\end{remark}

\begin{remark}\label{rem:paradoxical_combinator}
  The reader familiar with the lambda calculus will have noticed the
  similarity between $D$ and the paradoxical combinator.

  [Ed. note: the existence of this seems to suggest we have to be more
  restrictive on the set of processes and names we admit if we are to
  support no-cloning.]
\end{remark}

\subsubsection{Bisimulation}

The computational dynamics gives rise to another kind of equivalence,
the equivalence of computational behavior. As previously mentioned
this is typically captured \emph{via} some form of bisimulation.

% The notion we use in this paper is weak barbed bisimulation
% \cite{milner91polyadicpi}.

The notion we use in this paper is derived from weak barbed
bisimulation \cite{milner91polyadicpi}. 

\begin{definition}
An \emph{observation relation}, $\downarrow_{\mathcal N}$, over a set
of names, $\mathcal N$, is the smallest relation satisfying the rules
below.

\infrule[Out-barb]{y \in {\mathcal N}, \; x \nameeq y}
		  {\outputp{x}{v} \downarrow_{\mathcal N} x}
\infrule[Par-barb]{\mbox{$P\downarrow_{\mathcal N} x$ or $Q\downarrow_{\mathcal N} x$}}
		  {\binpar{P}{Q} \downarrow_{\mathcal N} x}

We write $P \Downarrow_{\mathcal N} x$ if there is $Q$ such that 
$P \wred Q$ and $Q \downarrow_{\mathcal N} x$.
\end{definition}

\begin{definition}
%\label{def.bbisim}
An  ${\mathcal N}$-\emph{barbed bisimulation} over a set of names, ${\mathcal N}$, is a symmetric binary relation 
${\mathcal S}_{\mathcal N}$ between agents such that $P\rel{S}_{\mathcal N}Q$ implies:
\begin{enumerate}
\item If $P \red P'$ then $Q \wred Q'$ and $P'\rel{S}_{\mathcal N} Q'$.
\item If $P\downarrow_{\mathcal N} x$, then $Q\Downarrow_{\mathcal N} x$.
\end{enumerate}
$P$ is ${\mathcal N}$-barbed bisimilar to $Q$, written
$P \wbbisim_{\mathcal N} Q$, if $P \rel{S}_{\mathcal N} Q$ for some ${\mathcal N}$-barbed bisimulation ${\mathcal S}_{\mathcal N}$.
\end{definition}

$\mathcal{R} \subseteq \pi \times \pi$

$P \mathcal{R} Q => \forall P'. P \red P' \Rightarrow \exists Q'. Q \red Q', P' \mathcal{R} Q'$

$P \vdash x \Rightarrow Q \vdash x$

\begin{mathpar}
  \inferrule*[lab=Out-barb]{x \nameeq y}{{y}!\langle{Q}\rangle \vdash x}
  \and
  \inferrule*[lab=Par-barb]{\mbox{$P\vdash x$ or $Q\vdash x$}}{\binpar{P}{Q} \vdash x}
\end{mathpar}

\subsubsection{Contexts}

One of the principle advantages of computational calculi like the
$\pi$-calculus is a well-defined notion of context,
contextual-equivalence and a correlation between
contextual-equivalence and notions of bisimulation. The notion of
context allows the decomposition of a process into (sub-)process and
its syntactic environment, its context. Thus, a context may be
thought of as a process with a ``hole'' (written $\Box$) in it. The
application of a context $M$ to a process $P$, written $M[P]$, is
tantamount to filling the hole in $M$ with $P$. In this paper we do
not need the full weight of this theory, but do make use of the notion
of context in the proof the main theorem. 

\begin{mathpar}
  \inferrule* [lab=summation] {} {{M_{M},M_{N}} \bc \Box \;|\; x.M_{A} \;|\; M_{M}+M_{N}}
  \and
  \inferrule* [lab=agent] {} {{M_{A}} \bc (\vec{x})M_{P} \;| \; \clift{P_0,\ldots,M_{P},\ldots,P_N}}
  \and \\
  \inferrule* [lab=process] {} {{M_{P}} \bc M_{N} \;| \;P|M_{P} }
\end{mathpar} 

\begin{mathpar}
  \inferrule* [lab=sychronization] {} {M_{N} \bc \Box \;|\; x?M_{F} \;|\; x!M_{C}}
  \and
  \inferrule* [lab=abstraction] {} {{M_{F}} \bc (x)M_{P} }
  \and
  \inferrule* [lab=concretion] {} {{M_{C}} \bc \langle M_{P} \rangle }
  \and \\
  \inferrule* [lab=process] {} {{M_{P}} \bc M_{N} \;| \;P|M_{P} }
\end{mathpar}

\begin{definition}[contextual application] Given a context $M$, and
  process $P$, we define the \emph{contextual application}, $M[P] :=
  M\{P/\Box\}$. That is, the contextual application of M to P is the
  substitution of $P$ for $\Box$ in $M$.
\end{definition}

$\meaningof{-} : L \to \mathcal{P}(\pi)$

\begin{mathpar}
  \inferrule* [lab=collection] {} {\meaningof{true} = \pi, \and \meaningof{~E} = \pi \setminus \meaningof{E}, \and \meaningof{E_{1} \& E_{2}} = \meaningof{E_{1}} \cap \meaningof{E_{2}}}
\end{mathpar}

\begin{mathpar}
  \inferrule* [lab=structure] {} {\meaningof{0} = \{ P \in \pi | P \equiv 0 \}, \and \\ \meaningof{E_1 | E_2} = \{ P \in \pi | P \equiv P_{1} | P_{2}, P_{1} \in \meaningof{E_{1}}, P_{2} \in \meaningof{E_2}\} }
\end{mathpar}

\begin{mathpar}
 \inferrule* [lab=behavior] {} {\meaningof{\langle a?b \rangle E} = \{ P \in \pi | P \equiv Q | u?(y)P', \\ \and \\\\ \and \\ \;\;\; u \in \meaningof{a}, \forall z.P'\{z/y\} \in \meaningof{E\{z/b\}}\}, \and \\ \meaningof{a!E} = \{ P \in \pi | P \equiv Q | x!\langle P' \rangle, x \in \meaningof{a} P' \in \meaningof{E}\} }
\end{mathpar}

\begin{mathpar}
 \inferrule* [lab=nominal] {} {\meaningof{\quotep{E}} = \{ \quotep{P} \in \quotep{\pi} | P \in \meaningof{E} \}, \and \meaningof{\quotep{P}} = \{ \quotep{Q} \in \quotep{\pi} | P \equiv Q \} \and \\ \meaningof{@\quotep{E}} = \{ P \in \pi | P \equiv @x, x \in \meaningof{E} \}}
\end{mathpar}

\begin{eqnarray*}
  \\
  \meaningof{-} : TS \to ST
\end{eqnarray*}

\begin{eqnarray*}
  \\
  L : TS \to ST
\end{eqnarray*}

\begin{eqnarray*}
  \\
  P \models E \iff P \in \meaningof{E}
\end{eqnarray*}

\begin{eqnarray*}
  P \approx_{L} Q \iff \forall E \in L. P \models E \iff Q \models E
\end{eqnarray*}

\begin{eqnarray*}
  P \approx_{K} Q
\end{eqnarray*}

\begin{eqnarray*}
  P \approx Q
\end{eqnarray*}

$\approx_{K} = \approx = \approx_{L}$

\subsubsection{Contextual duality}

Note that contexts extend the quotation operation to a family of
operations from processes to names. Given a context, $M$, we can
define a \emph{nominal context}, $\quotep{M}$ by $\quotep{M}[P] :=
\quotep{M[P]}$. To foreshadow what is to come we observe that these
operations enjoy a duality with processes very much like the duality
between vectors and maps from vectors to scalars.

Further, because the calculus is essentially higher-order, we have a
correspondence between contexts and processes. More specifically,
given a name $x$ and a context $M$ we can construct $M^{*}_{x}$ such
that 

\begin{mathpar}
  M^{*}_{x} | \lift{x}{P} \red M[P]
\end{mathpar}

namely,

\begin{mathpar}
  M^{*}_{x} := x?(u).M[\dropn{u}]
\end{mathpar}

The dependence of $M^{*}_{x}$ on a name makes it an abstraction, 

\begin{mathpar}
  M^{*} := (x)x?(u).M[\dropn{u}]
\end{mathpar}

\subsection{Additional notation}

It will sometimes be convenient to denote the process a name
quotes. We already have the notation $x = \quotep{P}$, but it will be
convenient to introduce an alternate notation, $\procn{x}$, when we
want to emphasize the connection to the use of the name. Note that, by
virtue of name equivalence, $\quotep{\procn{x}} \nameeq x$; so, the
notation is consistent with previous definitions.

Further, because names have structure it is possible to effect
substitutions on the basis of that structure. This means we need to
upgrade our notation for substitutions, which we accomplish by
adapting comprehension notation. Thus,

\begin{mathpar}
  P\{ y / x : x \in S \}
\end{mathpar}

is interpreted to mean the process derived from P by replacing (in a
capture-avoiding manner) each occurrence of $x$ in $S$ by $y$. For example,

\begin{mathpar}
  P\{ \quotep{\procn{x}|\procn{x}} / x : x \in \freenames{P} \}
\end{mathpar}

will replace each (occurrence) of a free name $x$ in $P$ by
$\quotep{\procn{x}|\procn{x}}$.

Also, we will avail ourselves of the notation $x^{L}$ and $x^{R}$ to
denote injections of a name into disjoint copies of the name
space. There are numerous ways to accomplish this. One example can be
found in \cite{MeredithR05}. This notation overloads to vectors of
names: $\vec{x}^{\pi} := (x_{i}^{\pi} \; : \; 0 \leq i < |\vec{x}| )$ where $\pi \in \{L,R\}$.

We also use $P^{\Box} := P|\Box$.

In \cite{MeredithR05} an interpretation of the new operator is
given. It turns out that there are several possible interpretations
all enjoying the requisite algebraic properties of the operator (see
\cite{milner91polyadicpi}). We will therefore make liberal use of
$(\nu\; \vec{x})P$.

% subsection the_syntax_and_semantics_of_the_notation_system (end)   

\section{Interpretation of QM}
\subsection{Supporting definitions}
\subsubsection{Multiplication}
\begin{mathpar}
  \quotep{Q} \cdot \quotep{R} := \quotep{Q|R}
  \and \\
  \quotep{Q} \cdot P := P\{ \quotep{Q|R} / \quotep{R} : \quotep{R} \in \freenames{P} \}
\end{mathpar}

\paragraph{Discussion}
The first line needs little explanation. The second line says that
each free name of the process is replaced with the multiplication of
that name by the scalar. Multiplication of a scalar (name) by a state
(process) results in a process all the names of which have been `moved
over' by parallel composition with the process the scalar
quotes. There is a subtlety that the bound names have to be
manipulated so that multiplied names aren't accidentally
captured. There are many ways to achieve this.

\begin{remark}\label{rem:multiplication_identities}
  The reader is invited to verify that for all $x,y,z \in \QProc$ and $P \in \Proc$
  \begin{mathpar}
    x \cdot \quotep{0} \equiv x 
    \and
    x \cdot y \equiv y \cdot x
    \and
    x \cdot (y \cdot z) \equiv (x \cdot y) \cdot z
    \and \\
    \quotep{0} \cdot P \equiv P
    \and \\
    x \cdot (y \cdot P) \equiv (x \cdot y) \cdot P
    \and \\
    x \cdot (P|Q) \equiv (x \cdot P) | (x \cdot Q)
    \and \\    
  \end{mathpar}
\end{remark}

\subsubsection{Tensor product}

We define a tensor product on processes by structural induction.

\paragraph{Tensor of sums} First note that all summations, including
$\pzero$ and sequence, can be written $\Sigma_{i} x_{i}.A_{i} +
\Sigma_{j} x_{j}.C_{j}$, where we have grouped input-guarded processes
together and output-guarded processes together.

Thus, we can define the tensor product of two summations, $N_{1}\otimes N_{2}$, where

\begin{mathpar}
  N_{1} := \Sigma_{i} x_{i}.A_{i} + \Sigma_{j} x_{j}.C_{j}
  \and
  N_{2} := \Sigma_{i'} y_{i'}.B_{i'} + \Sigma_{j'} y_{j'}.D_{j'} 
\end{mathpar}

as follows.

\begin{mathpar}
  \Sigma_{i} x_{i}.A_{i} + \Sigma_{j} x_{j}.C_{j} \otimes \Sigma_{i'}
  y_{i'}.B_{i'} + \Sigma_{j'} y_{j'}.D_{j'} 
  \and \\
  := \; \Sigma_{i} \Sigma_{i'} \quotep{\stackrel{\vee}{x_{i}}| \stackrel{\vee}{y_{i'}}}.(A_{i}\otimes B_{i'}) \; | \; \Sigma_{i'} \Sigma_{i} \quotep{\stackrel{\vee}{y_{i'}}|\stackrel{\vee}{x_{i}}}.(B_{i'}\otimes A_{i})
  \and
  \;\; | \;\; \Sigma_{j} \Sigma_{j'} \quotep{\stackrel{\vee}{x_{j}}|\stackrel{\vee}{y_{j'}}}.(A_{j}\otimes B_{j'}) \; | \; \Sigma_{j'} \Sigma_{j} \quotep{\stackrel{\vee}{y_{j'}}|\stackrel{\vee}{x_{j}}}.(B_{j'}\otimes A_{j})
\end{mathpar}

\begin{remark}
  Do we need to $x^{L}$ and $y^{R}$ for this construction as well?
\end{remark}

\paragraph{Tensor of parallel compositions} Next, we distribute tensor
over par.

\begin{mathpar}
  P_{1}|P_{2} \otimes Q_{1}|Q_{2} := (P_{1} \otimes Q_{1}) | (P_{1}
  \otimes Q_{2}) | (P_{2} \otimes Q_{1}) | (P_{2} \otimes Q_{2})
\end{mathpar}

\paragraph{Tensor with dropped names} We treat tensor of a
process with a dropped name as parallel composition.

\begin{mathpar}
  P \otimes \dropn{x} := P | \dropn{x}
\end{mathpar}

\paragraph{Tensor of agents}

Finally, we need to define tensor on agents. Note that the definition
of tensor on normal products only tensors inputs with inputs and
outputs with outputs. Thus, we only have to define the operation on
``homogeneous'' pairings.

\begin{mathpar}
  (\vec{x})P \otimes (\vec{y})Q
  \and \\
  := (x_{0}^{L}|y_{0}^{R},\ldots,x_{0}^{L}|y_{n}^{R},\ldots,x_{m}^{L}|y_{0}^{R},\ldots,x_{m}^{L}|y_{n}^R)(P\{ \vec{x}^{L}/\vec{x}\} \otimes Q \{ \vec{y}^{R}/\vec{y}\})
  \and \\
  \clift{\vec{P}} \otimes \clift{\vec{Q}}
  \and \\
  := \clift{P_{0}\otimes Q_{0},\ldots,P_{0}\otimes Q_{n},\ldots,P_{m}\otimes Q_{0},\ldots,P_{m}\otimes Q_{n}}
\end{mathpar}

\begin{remark}
  Observe that arities of tensored abstractions matches arities of
  tensored concretions if the original arities matched. Note also that
  the length of the arities corresponds to the increase in dimension
  we see in ordinary vector space tensor product.
\end{remark}

\begin{remark}
  Operationally, this definition distributes the tensor down to
  components ``linked'' by summation. Tensor over summation is
  intriguing in that it mixes names. Moreover, as a consequence of the
  way it mixes names we have the identities for all $x \in \QProc$ and
  $P,Q \in \Proc$

  \begin{mathpar}
    (x \cdot P) \otimes Q \equiv x \cdot (P \otimes Q) \equiv P \otimes (x \cdot Q)
    \and
    P \otimes \pzero \equiv P
  \end{mathpar}

  that the reader is invited to verify.
\end{remark}

\subsubsection{Annihilation}
\begin{mathpar}
  P^{\perp} := \{ Q | \forall R. P|Q \red^{*} R \Rightarrow R \red^{*} \pzero \}
  \and \\
  P^{\underline{\perp}} := \Sigma_{Q \in P^{\perp}} \quotep{Q}?(y).(\dropn{y}|Q) | \Sigma_{Q \in P^{\perp}} \quotep{Q}\clift{\Box}
\end{mathpar}

\paragraph{Discussion} The reader will note that $P^{\perp}$ is a
\emph{set} of processes, while $P^{\underline{\perp}}$ is a
\emph{context}. We call the set $P^{\perp}$ the \emph{annihilators} of
$P$. The parallel composition of a process in the annihilators of $P$
with $P$ will result in a process, the state space of which has all
paths eventually leading to $\pzero$. Execution may endure loops; but
under reasonable conditions of fairness (naturally guaranteed under
most notions of bisimulation) such a composite process cannot get
stuck in such a loop and will, eventually pop out and terminate.

The context $P^{\underline{\perp}}$ is ready and willing to ``take the
$P$ out of'' the process to which it is applied. It will effectively
transmit the code of the process to which it is applied to one of the
annihilators and run the process against it.

\subsubsection{Evaluation}
We fix $M$ a domain of fully abstract interpretation with an equality
coincident with bisimulation. We take $\meaningof{\cdot} : \Proc \to
M$ to be the map interpreting processes and $\nmeaningof{\cdot} : \M
\to Proc$ to be the map running the other way. Then we define

\begin{mathpar}
  \int P := \nmeaningof{\meaningof{P}}
\end{mathpar}

\paragraph{Discussion}
There are many fully abstract interpretations of Milner's
$\pi$-calculus. Any of them can be used as a basis for interpreting
the reflective calculus here. Equipped with such a domain it is
largely a matter of grinding through to check that the Yoneda
construction for the normalization-by-evaluation program can be
extended to this setting.

\begin{remark}
  The reader is invited to verify that $\int (P^{\underline{\perp}}[P]) = 0$.
\end{remark}

\subsection{Quantum mechanics}

Table \ref{tbl:core_qm_op_defns} gives the core operational definitions

\begin{table}[htp]\label{tbl:core_qm_op_defns}
  \center{
    \fbox{
      \begin{tabular}{c|c}
        quantum mechanics & process calculus \\
        \hline
        scalar & $x := \quotep{P}$ \\
        state vector & $\state{P} := P$ \\
        dual & $\state{P}^{*} := \event{P^{\underline{\perp}}} := \quotep{P^{\underline{\perp}}}[-]$ \\
        matrix & $ \Sigma_{\alpha} \state{P_{\alpha}}x_{\alpha}\event{Q_{\alpha}}$ \\
        vector addition & $\state{P} + \state{Q} := \state{P | Q}$ \\
        tensor product & $\state{P} \otimes \state{Q} := \state{P \otimes Q}$ \\
        inner product & $\innerprod{P}{Q} := \quotep{\int P^{\underline{\perp}}[Q]}$ \\
      \end{tabular}
    }
  }
  \caption{QM - operational definitions}
\end{table}

where

\begin{mathpar}
  \prmatrix{P}{Q} := \fprmatrix{P}{\quotep{\pzero}}{Q}
  \and
  \fprmatrix{P}{x}{Q} := (\state{P},x,\event{Q})
  \and
  (\fprmatrix{P}{x}{Q})(\state{R}) := x \cdot \innerprod{Q}{R} \cdot \state{P}
  \and
  (\fprmatrix{P}{x}{Q})(\event{R}) := x \cdot \innerprod{R}{P} \cdot \event{Q}
\end{mathpar}

\paragraph{Discussion}
As promised: vectors (aka states) are represented as processes; duals
as contextual duals; inner product definition should be compared with
standard inner product definition for ....

\begin{remark}
  Assuming $\int (P^{\underline{\perp}}[P]) = 0$, the reader is
  invited to verify that $(\fprmatrix{P}{x}{P})(\state{P}) = x \cdot \state{P}$.
\end{remark}

\begin{remark}
  The reader is invited to verify that $\innerprod{P}{Q}$ could
  equally well have been written $\quotep{\int \stackrel{\vee}{x}}$
  where $x = \event{P^{\underline{\perp}}}(Q)$.

  One of the motivations for this remark is that there is another way
  to factor these operations. We could package up evaluation in the dual:

  \begin{mathpar}
    \state{P}^{*} := \event{\int P^{\underline{\perp}}} := \quotep{\int P^{\underline{\perp}}}[-]
  \end{mathpar}

  and then have inner product defined by
  
  \begin{mathpar}
    \innerprod{P}{Q} := \event{P}(Q)
  \end{mathpar}

  Hopefully, experience with the calculations will provide guidance on
  the best factoring.
\end{remark}

\begin{remark}
  Assuming $\int (P^{\underline{\perp}}[P]) = 0$, the reader is
  invited to verify that $\forall P,Q. (\prmatrix{0}{Q})(\state{0}) =
  \state{0}$ and dually $(\prmatrix{P}{0})(\event{0}) = \event{0}$.
\end{remark}

\begin{remark}
  i'm a little worried that i don't (yet) have proper support for
  complex conjugacy. But, the observation above may give us a
  clue. According to Abramsky, it must be the case that the scalars
  are iso to the homset of the identity for the tensor -- which the
  observation above characterizes. 

  For now, we will simply bookmark the notion with $\overline{x}$.
\end{remark}

\subsubsection{Adjointness}

We need to give a definition of $(\cdot)^{\dagger}$ for matrices. The
obvious candidate definition is
\begin{mathpar}
(\Sigma_{\alpha}\fprmatrix{P_{\alpha}}{x_{\alpha}}{Q_{\alpha}})^{\dagger}
= \Sigma_{\alpha}\fprmatrix{(Q_{\alpha}^{\underline{\perp}})^{*}}{\overline{x}_{\alpha}}{P_{\alpha}^{\underline{\perp}}} 
\end{mathpar}

But, $(Q_{\alpha}^{\underline{\perp}})^{*}$ requires a name along
which to communicate the process to achieve the context application.

\subsubsection{Basis for a basis}
If processes label states and ``addition'' of states (a.k.a. vector
addition) is interpreted as parallel composition, what corresponds to
notions of linear independence and basis? Here, we recall that Yoshida
has developed a set of \emph{combinators} for an asynchronous verison
of Milner's $\pi$-calculus. These are a finite set of processes such
any process can be expressed as parallel composition of these
combinators together with liberal uses of the new operator and
replication. We can simply give a translation of these into the
present calculus and have reasonable expectation that the property
carries over. That is, that the resultant set allows to express all
processes via parallel composition. Note, however, that there is no
new operator or replication in this calculus. As a result, we expect
that the corresponding set is actually infinite. That is, we expect
that the space is actually infinite dimensional.

\begin{remark}
  The attentive reader may be a bit concerned. Certainly, the
  collection $S$, $K$ and $I$ is a finite set of
  combinators. Shouldn't we expect to see a finite set of combinators
  for an effectively equivalent system? i am very sympathetic to this
  critique and feel it warrants full attention. On the other hand, i
  also have in mind the following analogy. The natural numbers, as a
  monoid under addition, has exactly $1$ generator, while the natural
  numbers, as a monoid under multiplication, has countably many
  generators (the primes). We observe that the application of the
  lambda calculus is much less resource sensitive than the parallel
  composition of the $\pi$-calculus. Could it be the case that we have
  an analogy of the form
  
  \begin{mathpar}
    m + n : MN :: m*n : M|N
  \end{mathpar}

  giving a similar blow up in the set of ``primes''?  This is such a
  wonderful thought that, even if it's not true, i think it's worth
  writing down.
\end{remark}
 

\documentclass[12pt]{llncs}
%\documentclass{jktr}

\usepackage[pdftex]{hyperref}                   
\usepackage {listings}
\usepackage {mathpartir}
\usepackage{bcprules}
%\usepackage{listings}
                       
\usepackage{graphicx} 
%\usepackage[margins=2.5cm,nohead,nofoot]{geometry}
%\usepackage{geometry}
\usepackage{amsfonts}
\usepackage{amstext}
\usepackage{latexsym}
\usepackage{amssymb}
\usepackage{color}


%\include{myPreamble}
\include{qm2pi.local} 

%\ifpdf
%\usepackage[pdftex]{graphicx}
%\else
%\usepackage{graphicx}
%\fi

 % \ifpdf
%  \usepackage{pdfsync}
%  \if


%\title{Brief Article}
%\author{David F. Snyder}
%\author{L.G. Meredith}

%\address{Dept. of Math., Texas State University--San Marcos, San Marcos, TX 78666}
       
\pagestyle{empty}


\begin{document}

\lstset{language=[Objective]Caml,frame=shadowbox}

\input{qm2pi.front}

% section front matter (end)

\input{qm2pi.intro} 
 
% section introduction (end)

% \input{qm2pi.knotations} 

% section notation (end)

\input{qm2pi.process.calculi} 

% section concurrent_process_calculi_and_spatial_logics_ (end)
    
%\input{qm2pi.knots2pi} 

%\input{qm2pi.trefoil} 

%\input{qm2pi.mainthm} 

% subsection basic_interpretation (end)

%\input{qm2pi.rho.presentation} 
\subsection{The syntax and semantics of the notation system}\label{sub:the_syntax_and_semantics_of_the_notation_system} % (fold)

We now summarize a technical presentation of the calculus that
embodies our theory of dynamics. The typical presentation of such a
calculus follows the style of giving generators and relations on
them. The grammar, below, describing term constructors, freely
generates the set of processes, $\Proc$. This set is then quotiented
by a relation known as structural congruence and it is over this set
that the notion of dynamics is expressed. This presentation is
essentially that of \cite{MeredithR05} with the addition of
polyadicity and summation. For readability we have relegated some of
the technical subtleties to an appendix.

\subsubsection{Process grammar}\label{subsub:process_grammar}

\begin{mathpar}
  \inferrule* [lab=synchronization] {} {{M} \bc \pzero \;|\; x?F \;|\; x!C }
  \and
  \inferrule* [lab=abstraction] {} {{F} \bc (x)P}
  \and
  \inferrule* [lab=concretion] {} {{C} \bc \langle Q \rangle}
  \and
  \inferrule* [lab=process] {} {{P,Q} \bc M \;| \;P|Q \;|\; @{x}}
  \and
  \inferrule* [lab=name] {} {{x} \bc \quotep{P}}
\end{mathpar} 

Note that $\vec{x}$ (resp. $\vec{P}$) denotes a vector of names
(resp. processes) of length $|\vec{x}|$ (resp. $|\vec{P}|$). We adopt
the following useful abbreviations.

\begin{mathpar}
   x?(\vec{y}).P := x.(\vec{y})P \and  x\clift{\vec{P}} := x.\clift{\vec{P}}
   \and x!(y) := \lift{x}{\dropn{y}}
   \and \Pi_{i=0}^{n-1}P_i := P_0 | \ldots | P_{n-1}
\end{mathpar}

\subsubsection{Structural congruence}

\paragraph{Free and bound names and alpha-equivalence.} At the
core of structural equivalence is alpha-equivalence which identifies
process that are the same up to a change of variable. Formally, we
recognize the distinction between free and bound names. The free names
of a process, $\freenames{P}$, may be calculated recursively as
follows:

\begin{mathpar}
\freenames{\pzero} := \emptyset
  \and \\
  \freenames{x?(y).P} := \{ x \} \cup (\freenames{P} \setminus \{ y \})
  \and 
  \freenames{x!\langle P \rangle} := \{ x \} \cup \{ P \} 
  \and \\
  \freenames{P|Q} := \freenames{P} \cup \freenames{Q}
  \and \\
  \freenames{@{x}} := \{ x \}
\end{mathpar}

$\pi$
$\quotep{\pi}$

$\freenames{-} : \pi \to \mathcal{P}(\quotep{\pi})$

\begin{eqnarray*}
  \freenames{\pzero} & := & \emptyset \\
  \freenames{x?(y).P} & := & \{ x \} \cup (\freenames{P} \setminus \{ y \}) \\
  \freenames{x!\langle P \rangle} & := & \{ x \} \cup \{ P \} \\
  \freenames{P|Q} & := & \freenames{P} \cup \freenames{Q} \\
  \freenames{\dropn{x}} & := & \{ x \}
\end{eqnarray*}

The bound names of a process, $\boundnames{P}$, are those names occurring in $P$
that are not free. For example, in $x?(y).0$, the name $x$ is free, while $y$ is bound.

\begin{mathpar}
  \inferrule* [lab=monoidal-laws] {} { P|Q \equiv Q|P \and P|0 \equiv P \and P|(Q|R) \equiv (P|Q)|R }
\end{mathpar}

\begin{mathpar}
  \inferrule* [lab=alpha-equivalence] {} { (x)P \equiv (y)P\{y/x\} \and y \not\in \freenames{P} }
\end{mathpar}

\begin{definition}
Then two processes, $P,Q$, are alpha-equivalent if $P = Q\{\vec{y}/\vec{x}\}$ for
some $\vec{x} \in \boundnames{Q},\vec{y} \in \boundnames{P}$, where $Q\{\vec{y}/\vec{x}\}$
denotes the capture-avoiding substitution of $\vec{y}$ for $\vec{x}$ in $Q$.
\end{definition}

\begin{definition}
  The {\em structural congruence} \cite{SangiorgiWalker} , $\equiv$,
  between processes is the least congruence containing
  alpha-equivalence, satisfying the abelian monoid laws
  (associativity, commutativity and $\pzero$ as identity) for parallel
  composition $|$ and for summation $+$.
\end{definition}

\subsection{Name equivalence}

We take name equivalence, written $\nameeq$, to be the smallest
equivalence relation generated by the following rules.

\begin{mathpar}
\inferrule*[lab=Quote-drop]
{ }
{ \quotep{@{x}} \nameeq x }

\inferrule*[lab=Struct-equiv]
{ P \scong Q }
{ \quotep{P} \nameeq \quotep{Q} }
\end{mathpar}

The astute reader will have noticed that the mutual recursion of names
and processes imposes a mutual recursion on alpha-equivalence and
structural equivalence via name-equivalence. Fortunately, all of this
works out pleasantly and we may calculate in the natural way, free of
concern. The reader interested in the details is referred to the
appendix \ref{appendix:rho_details}.

\subsection{Substitution}

We use $\Proc$ for the set of processes, $\QProc$ for the set of
names, and $\id{\{}\vec{y} / \vec{x} \id{\}}$ to denote partial maps,
$s : \QProc \rightarrow \QProc$. A map, $s$ lifts, uniquely, to a map
on process terms, $\widehat{s} : \Proc \rightarrow \Proc$ by the
following equations.

\begin{mathpar}
  (0) \psubstp{Q}{P} := 0 \\
  (R \juxtap S) \psubstp{Q}{P}
  :=    
  (R)\psubstp{Q}{P} \juxtap (S) \psubstp{Q}{P} \\
  (x?(y).R) \psubstp{Q}{P}    
  :=    
  (x)\substp{Q}{P} (z)\concat( (R \psubstn{z}{y}) \psubstp{Q}{P} ) \\
  (\lift{x}{R}) \psubstp{Q}{P}  
  :=
  \lift{(x)\substp{Q}{P}}{ R \psubstp{Q}{P} } \\
%   (\dropn{x})  \psubstp{Q}{P}       
%   := 
%   \left\{ 
%     \begin{array}{ccc} 
%       \dropn{\quotep{Q}} & & x \nameeq \quotep{P} \\
%       \dropn{x} & & otherwise \\
%     \end{array}
%   \right. 
  (\dropn{x})  \psubstp{Q}{P}       
  := 
  \left\{ 
    \begin{array}{ccc} 
      Q & & x \nameeq \quotep{P} \\
      \dropn{x} & & otherwise \\
    \end{array}
  \right.
\end{mathpar}
 

where

\begin{eqnarray}
  (x)\id{\{} \lpquote Q \rpquote / \lpquote P \rpquote \id{\}}            = 
  \left\{ 
    \begin{array}{ccc}
      \lpquote Q \rpquote & & x \nameeq \lpquote P \rpquote \\
      x & & otherwise \\
    \end{array}
  \right. \nonumber
\end{eqnarray}

and $z$ is chosen distinct from $\quotep{P}$, $\quotep{Q}$, the free
names in $Q$, and all the names in $R$. Our $\alpha$-equivalence will
be built in the standard way from this substitution.

\begin{remark}\label{rem:no_self_referential_names}
  One consequence of these definitions is that $\forall P. \quotep{P}
  \not\in \freenames{P}$.
\end{remark}

\subsection{ Dynamic quote: an example }

Anticipating something of what's to come, consider applying the
substitution, $\widehat{\id{\{}u / z \id{\}}}$, to the following pair
of processes, $\lift{w}{y!(z)}$ and $w[ \lpquote y!(z) \rpquote ]$.

\begin{eqnarray}
	\lift{w}{y!(z)}\widehat{\id{\{}u / z \id{\}}}
		& = &
		\lift{w}{y!(u)} \nonumber\\
	w[ \lpquote y!(z) \rpquote ] \widehat{ \id{\{}u / z \id{\}} }
		& = &
		w[ \lpquote y!(z) \rpquote ] \nonumber
\end{eqnarray}

Because the body of the process between quotes is impervious to
substitution, we get radically different answers. In fact, by
examining the first process in an input context,
e.g. $x?(z).\lift{w}{y!(z)}$, we see that the process under the lift
operator may be shaped by prefixed inputs binding a name inside it. In
this sense, the lift operator will be seen as a way to dynamically
construct processes before reifying them as names.

Finally equipped with these standard features we can present the
dynamics of the calculus.

\subsubsection{Operational semantics} 

Finally, we introduce the computational dynamics. What marks these
algebras as distinct from other more traditionally studied algebraic
structures, e.g. vector spaces or polynomial rings, is the manner in
which dynamics is captured. In traditional structures, dynamics is typically
expressed through morphisms between such structures, as in linear maps
between vector spaces or morphisms between rings. In algebras
associated with the semantics of computation, the dynamics is
expressed as part of the algebraic structure itself, through a
reduction reduction relation typically denoted by $\red$. Below, we
give a recursive presentation of this relation for the calculus used
in the encoding.

$\red \subseteq \pi \times \pi$
$\red : \pi \to \mathcal{P}(\pi)$

\begin{mathpar}
  \inferrule* [lab=Comm] { \textsf{match}( x_{src}, x_{trgt} ) } { x_{trgt}?(y)P \; | \; x_{src}!\langle {Q} \rangle \red P\{\quotep{Q}/y}\} }
  \and \\
  \inferrule* [lab=Par] {{P} \red {P}'} {{{P} | {Q}} \red {{P}' | {Q}}}
  \and
  \inferrule* [lab=Equiv]{{{P} \scong {P}'} \andalso {{P}' \red {Q}'} \andalso {{Q}' \scong {Q}}}{{P} \red {Q}}
\end{mathpar}

\begin{eqnarray*}
  match_{\equiv} (\quotep{P},\quotep{Q}) & := & P \equiv Q \\
  match_{\dagger}(\quotep{P},\quotep{Q}) & := & \forall R. P|Q \red^{*} R => R \red^{*} 0 \\
  match_{K}(\quotep{P},\quotep{Q}) & := & K \mbox{ for some context } K
\end{eqnarray*}

$u?(x)P | u!\langle Q \rangle \red P\{\quotep{Q}/x\}$

%We write $\wred$ for $\red^*$, and $P\red$ if $\exists Q $ such that $ P \red Q$.
We write $P\red$ if $\exists Q $ such that $ P \red Q$ and $P\not\red$, otherwise.

\section{Replication}

As mentioned before, it is known that replication (and hence
recursion) can be implemented in a higher-order process algebra
\cite{SangiorgiWalker}. As our first example of calculation with the
machinery thus far presented we give the construction explicitly in
the {\rhoc}.

\begin{eqnarray}
	D_{x} & := & \prefix{x}{y}{(\binpar{\outputp{x}{y}}{@{y}})} \nonumber\\
	\bangp_{x}{P} & := & \binpar{{x}!\langle{\binpar{D_{x}}{P}}\rangle}{D_{x}} \nonumber
\end{eqnarray}

\begin{eqnarray}
	\bangp_{x}{P} & & \nonumber\\
	=
	& {x}!\langle{(\prefix{x}{y}{(\outputp{x}{y} | @{y})) | P}}\rangle 
	      | \prefix{x}{y}{(\outputp{x}{y} | @{y})} & \nonumber\\
	\red
	& (\outputp{x}{y} | @{y})\substn{\quotep{(\prefix{x}{y}{(@{y} | \outputp{x}{y})) | P}}}{y} & \nonumber\\
	=
	& \outputp{x}{\quotep{(\prefix{x}{y}{(\outputp{x}{y} | @{y})) | P}}}
	  | {(\prefix{x}{y}{(\outputp{x}{y} | @{y})) | P}} & \nonumber\\
	\red
	& \ldots & \nonumber\\
	\red^*
	& P | P | \ldots & \nonumber
\end{eqnarray}

Of course, this encoding, as an implementation, runs away, unfolding
$\bangp{P}$ eagerly. A lazier and more implementable replication
operator, restricted to input-guarded processes, may be obtained as follows.

\begin{eqnarray}
\bangp{\prefix{u}{v}{P}} 
	:= 
	\binpar{\lift{x}{\prefix{u}{v}{(\binpar{D(x)}{P})}}}{D(x)} \nonumber
\end{eqnarray}

\begin{remark}
  Note that the lazier definition still does not deal with summation
  or mixed summation (i.e. sums over input and output). The reader is
  invited to construct definitions of replication that deal with these
  features. 

  Further, the definitions are parameterized in a name, $x$. Can you,
  gentle reader, make a definition that eliminates this parameter and
  guarantees no accidental interaction between the replication
  machinery and the process being replicated -- i.e. no accidental
  sharing of names used by the process to get its work done and the
  name(s) used by the replication to effect copying. This latter
  revision of the definition of replication is crucial to obtaining
  the expected identity $!!P \sim !P$.
\end{remark}

\begin{remark}\label{rem:paradoxical_combinator}
  The reader familiar with the lambda calculus will have noticed the
  similarity between $D$ and the paradoxical combinator.

  [Ed. note: the existence of this seems to suggest we have to be more
  restrictive on the set of processes and names we admit if we are to
  support no-cloning.]
\end{remark}

\subsubsection{Bisimulation}

The computational dynamics gives rise to another kind of equivalence,
the equivalence of computational behavior. As previously mentioned
this is typically captured \emph{via} some form of bisimulation.

% The notion we use in this paper is weak barbed bisimulation
% \cite{milner91polyadicpi}.

The notion we use in this paper is derived from weak barbed
bisimulation \cite{milner91polyadicpi}. 

\begin{definition}
An \emph{observation relation}, $\downarrow_{\mathcal N}$, over a set
of names, $\mathcal N$, is the smallest relation satisfying the rules
below.

\infrule[Out-barb]{y \in {\mathcal N}, \; x \nameeq y}
		  {\outputp{x}{v} \downarrow_{\mathcal N} x}
\infrule[Par-barb]{\mbox{$P\downarrow_{\mathcal N} x$ or $Q\downarrow_{\mathcal N} x$}}
		  {\binpar{P}{Q} \downarrow_{\mathcal N} x}

We write $P \Downarrow_{\mathcal N} x$ if there is $Q$ such that 
$P \wred Q$ and $Q \downarrow_{\mathcal N} x$.
\end{definition}

\begin{definition}
%\label{def.bbisim}
An  ${\mathcal N}$-\emph{barbed bisimulation} over a set of names, ${\mathcal N}$, is a symmetric binary relation 
${\mathcal S}_{\mathcal N}$ between agents such that $P\rel{S}_{\mathcal N}Q$ implies:
\begin{enumerate}
\item If $P \red P'$ then $Q \wred Q'$ and $P'\rel{S}_{\mathcal N} Q'$.
\item If $P\downarrow_{\mathcal N} x$, then $Q\Downarrow_{\mathcal N} x$.
\end{enumerate}
$P$ is ${\mathcal N}$-barbed bisimilar to $Q$, written
$P \wbbisim_{\mathcal N} Q$, if $P \rel{S}_{\mathcal N} Q$ for some ${\mathcal N}$-barbed bisimulation ${\mathcal S}_{\mathcal N}$.
\end{definition}

$\mathcal{R} \subseteq \pi \times \pi$

$P \mathcal{R} Q => \forall P'. P \red P' \Rightarrow \exists Q'. Q \red Q', P' \mathcal{R} Q'$

$P \vdash x \Rightarrow Q \vdash x$

\begin{mathpar}
  \inferrule*[lab=Out-barb]{x \nameeq y}{{y}!\langle{Q}\rangle \vdash x}
  \and
  \inferrule*[lab=Par-barb]{\mbox{$P\vdash x$ or $Q\vdash x$}}{\binpar{P}{Q} \vdash x}
\end{mathpar}

\subsubsection{Contexts}

One of the principle advantages of computational calculi like the
$\pi$-calculus is a well-defined notion of context,
contextual-equivalence and a correlation between
contextual-equivalence and notions of bisimulation. The notion of
context allows the decomposition of a process into (sub-)process and
its syntactic environment, its context. Thus, a context may be
thought of as a process with a ``hole'' (written $\Box$) in it. The
application of a context $M$ to a process $P$, written $M[P]$, is
tantamount to filling the hole in $M$ with $P$. In this paper we do
not need the full weight of this theory, but do make use of the notion
of context in the proof the main theorem. 

\begin{mathpar}
  \inferrule* [lab=summation] {} {{M_{M},M_{N}} \bc \Box \;|\; x.M_{A} \;|\; M_{M}+M_{N}}
  \and
  \inferrule* [lab=agent] {} {{M_{A}} \bc (\vec{x})M_{P} \;| \; \clift{P_0,\ldots,M_{P},\ldots,P_N}}
  \and \\
  \inferrule* [lab=process] {} {{M_{P}} \bc M_{N} \;| \;P|M_{P} }
\end{mathpar} 

\begin{mathpar}
  \inferrule* [lab=sychronization] {} {M_{N} \bc \Box \;|\; x?M_{F} \;|\; x!M_{C}}
  \and
  \inferrule* [lab=abstraction] {} {{M_{F}} \bc (x)M_{P} }
  \and
  \inferrule* [lab=concretion] {} {{M_{C}} \bc \langle M_{P} \rangle }
  \and \\
  \inferrule* [lab=process] {} {{M_{P}} \bc M_{N} \;| \;P|M_{P} }
\end{mathpar}

\begin{definition}[contextual application] Given a context $M$, and
  process $P$, we define the \emph{contextual application}, $M[P] :=
  M\{P/\Box\}$. That is, the contextual application of M to P is the
  substitution of $P$ for $\Box$ in $M$.
\end{definition}

$\meaningof{-} : L \to \mathcal{P}(\pi)$

\begin{mathpar}
  \inferrule* [lab=collection] {} {\meaningof{true} = \pi, \and \meaningof{~E} = \pi \setminus \meaningof{E}, \and \meaningof{E_{1} \& E_{2}} = \meaningof{E_{1}} \cap \meaningof{E_{2}}}
\end{mathpar}

\begin{mathpar}
  \inferrule* [lab=structure] {} {\meaningof{0} = \{ P \in \pi | P \equiv 0 \}, \and \\ \meaningof{E_1 | E_2} = \{ P \in \pi | P \equiv P_{1} | P_{2}, P_{1} \in \meaningof{E_{1}}, P_{2} \in \meaningof{E_2}\} }
\end{mathpar}

\begin{mathpar}
 \inferrule* [lab=behavior] {} {\meaningof{\langle a?b \rangle E} = \{ P \in \pi | P \equiv Q | u?(y)P', \\ \and \\\\ \and \\ \;\;\; u \in \meaningof{a}, \forall z.P'\{z/y\} \in \meaningof{E\{z/b\}}\}, \and \\ \meaningof{a!E} = \{ P \in \pi | P \equiv Q | x!\langle P' \rangle, x \in \meaningof{a} P' \in \meaningof{E}\} }
\end{mathpar}

\begin{mathpar}
 \inferrule* [lab=nominal] {} {\meaningof{\quotep{E}} = \{ \quotep{P} \in \quotep{\pi} | P \in \meaningof{E} \}, \and \meaningof{\quotep{P}} = \{ \quotep{Q} \in \quotep{\pi} | P \equiv Q \} \and \\ \meaningof{@\quotep{E}} = \{ P \in \pi | P \equiv @x, x \in \meaningof{E} \}}
\end{mathpar}

\begin{eqnarray*}
  \\
  \meaningof{-} : TS \to ST
\end{eqnarray*}

\begin{eqnarray*}
  \\
  L : TS \to ST
\end{eqnarray*}

\begin{eqnarray*}
  \\
  P \models E \iff P \in \meaningof{E}
\end{eqnarray*}

\begin{eqnarray*}
  P \approx_{L} Q \iff \forall E \in L. P \models E \iff Q \models E
\end{eqnarray*}

\begin{eqnarray*}
  P \approx_{K} Q
\end{eqnarray*}

\begin{eqnarray*}
  P \approx Q
\end{eqnarray*}

$\approx_{K} = \approx = \approx_{L}$

\subsubsection{Contextual duality}

Note that contexts extend the quotation operation to a family of
operations from processes to names. Given a context, $M$, we can
define a \emph{nominal context}, $\quotep{M}$ by $\quotep{M}[P] :=
\quotep{M[P]}$. To foreshadow what is to come we observe that these
operations enjoy a duality with processes very much like the duality
between vectors and maps from vectors to scalars.

Further, because the calculus is essentially higher-order, we have a
correspondence between contexts and processes. More specifically,
given a name $x$ and a context $M$ we can construct $M^{*}_{x}$ such
that 

\begin{mathpar}
  M^{*}_{x} | \lift{x}{P} \red M[P]
\end{mathpar}

namely,

\begin{mathpar}
  M^{*}_{x} := x?(u).M[\dropn{u}]
\end{mathpar}

The dependence of $M^{*}_{x}$ on a name makes it an abstraction, 

\begin{mathpar}
  M^{*} := (x)x?(u).M[\dropn{u}]
\end{mathpar}

\subsection{Additional notation}

It will sometimes be convenient to denote the process a name
quotes. We already have the notation $x = \quotep{P}$, but it will be
convenient to introduce an alternate notation, $\procn{x}$, when we
want to emphasize the connection to the use of the name. Note that, by
virtue of name equivalence, $\quotep{\procn{x}} \nameeq x$; so, the
notation is consistent with previous definitions.

Further, because names have structure it is possible to effect
substitutions on the basis of that structure. This means we need to
upgrade our notation for substitutions, which we accomplish by
adapting comprehension notation. Thus,

\begin{mathpar}
  P\{ y / x : x \in S \}
\end{mathpar}

is interpreted to mean the process derived from P by replacing (in a
capture-avoiding manner) each occurrence of $x$ in $S$ by $y$. For example,

\begin{mathpar}
  P\{ \quotep{\procn{x}|\procn{x}} / x : x \in \freenames{P} \}
\end{mathpar}

will replace each (occurrence) of a free name $x$ in $P$ by
$\quotep{\procn{x}|\procn{x}}$.

Also, we will avail ourselves of the notation $x^{L}$ and $x^{R}$ to
denote injections of a name into disjoint copies of the name
space. There are numerous ways to accomplish this. One example can be
found in \cite{MeredithR05}. This notation overloads to vectors of
names: $\vec{x}^{\pi} := (x_{i}^{\pi} \; : \; 0 \leq i < |\vec{x}| )$ where $\pi \in \{L,R\}$.

We also use $P^{\Box} := P|\Box$.

In \cite{MeredithR05} an interpretation of the new operator is
given. It turns out that there are several possible interpretations
all enjoying the requisite algebraic properties of the operator (see
\cite{milner91polyadicpi}). We will therefore make liberal use of
$(\nu\; \vec{x})P$.

% subsection the_syntax_and_semantics_of_the_notation_system (end)   

\input{qm2pi.qmops} 

\input{qm2pi.sterngerlach} 

\input{qm2pi.metric} 

% section concurrent_process_calculi (end)

%\input{qm2pi.proofsketch}

% section proof sketch (end)

%\input{qm2pi.slviaknots} 

% section spatial logic via knots (end)

\input{qm2pi.conclusion}

% section conclusion (end)

%\input{qm2pi.dtcodes} 

% section wiring algorithm (end)

\input{qm2pi.ack} 

% section acknowledgments (end)

\newpage


\bibliographystyle{plain}   
\bibliography{../../biblios/main.bib}

\input{qm2pi.rhodetails}

\end{document}

 

\documentclass[12pt]{llncs}
%\documentclass{jktr}

\usepackage[pdftex]{hyperref}                   
\usepackage {listings}
\usepackage {mathpartir}
\usepackage{bcprules}
%\usepackage{listings}
                       
\usepackage{graphicx} 
%\usepackage[margins=2.5cm,nohead,nofoot]{geometry}
%\usepackage{geometry}
\usepackage{amsfonts}
\usepackage{amstext}
\usepackage{latexsym}
\usepackage{amssymb}
\usepackage{color}


%\include{myPreamble}
\include{qm2pi.local} 

%\ifpdf
%\usepackage[pdftex]{graphicx}
%\else
%\usepackage{graphicx}
%\fi

 % \ifpdf
%  \usepackage{pdfsync}
%  \if


%\title{Brief Article}
%\author{David F. Snyder}
%\author{L.G. Meredith}

%\address{Dept. of Math., Texas State University--San Marcos, San Marcos, TX 78666}
       
\pagestyle{empty}


\begin{document}

\lstset{language=[Objective]Caml,frame=shadowbox}

\input{qm2pi.front}

% section front matter (end)

\input{qm2pi.intro} 
 
% section introduction (end)

% \input{qm2pi.knotations} 

% section notation (end)

\input{qm2pi.process.calculi} 

% section concurrent_process_calculi_and_spatial_logics_ (end)
    
%\input{qm2pi.knots2pi} 

%\input{qm2pi.trefoil} 

%\input{qm2pi.mainthm} 

% subsection basic_interpretation (end)

%\input{qm2pi.rho.presentation} 
\subsection{The syntax and semantics of the notation system}\label{sub:the_syntax_and_semantics_of_the_notation_system} % (fold)

We now summarize a technical presentation of the calculus that
embodies our theory of dynamics. The typical presentation of such a
calculus follows the style of giving generators and relations on
them. The grammar, below, describing term constructors, freely
generates the set of processes, $\Proc$. This set is then quotiented
by a relation known as structural congruence and it is over this set
that the notion of dynamics is expressed. This presentation is
essentially that of \cite{MeredithR05} with the addition of
polyadicity and summation. For readability we have relegated some of
the technical subtleties to an appendix.

\subsubsection{Process grammar}\label{subsub:process_grammar}

\begin{mathpar}
  \inferrule* [lab=synchronization] {} {{M} \bc \pzero \;|\; x?F \;|\; x!C }
  \and
  \inferrule* [lab=abstraction] {} {{F} \bc (x)P}
  \and
  \inferrule* [lab=concretion] {} {{C} \bc \langle Q \rangle}
  \and
  \inferrule* [lab=process] {} {{P,Q} \bc M \;| \;P|Q \;|\; @{x}}
  \and
  \inferrule* [lab=name] {} {{x} \bc \quotep{P}}
\end{mathpar} 

Note that $\vec{x}$ (resp. $\vec{P}$) denotes a vector of names
(resp. processes) of length $|\vec{x}|$ (resp. $|\vec{P}|$). We adopt
the following useful abbreviations.

\begin{mathpar}
   x?(\vec{y}).P := x.(\vec{y})P \and  x\clift{\vec{P}} := x.\clift{\vec{P}}
   \and x!(y) := \lift{x}{\dropn{y}}
   \and \Pi_{i=0}^{n-1}P_i := P_0 | \ldots | P_{n-1}
\end{mathpar}

\subsubsection{Structural congruence}

\paragraph{Free and bound names and alpha-equivalence.} At the
core of structural equivalence is alpha-equivalence which identifies
process that are the same up to a change of variable. Formally, we
recognize the distinction between free and bound names. The free names
of a process, $\freenames{P}$, may be calculated recursively as
follows:

\begin{mathpar}
\freenames{\pzero} := \emptyset
  \and \\
  \freenames{x?(y).P} := \{ x \} \cup (\freenames{P} \setminus \{ y \})
  \and 
  \freenames{x!\langle P \rangle} := \{ x \} \cup \{ P \} 
  \and \\
  \freenames{P|Q} := \freenames{P} \cup \freenames{Q}
  \and \\
  \freenames{@{x}} := \{ x \}
\end{mathpar}

$\pi$
$\quotep{\pi}$

$\freenames{-} : \pi \to \mathcal{P}(\quotep{\pi})$

\begin{eqnarray*}
  \freenames{\pzero} & := & \emptyset \\
  \freenames{x?(y).P} & := & \{ x \} \cup (\freenames{P} \setminus \{ y \}) \\
  \freenames{x!\langle P \rangle} & := & \{ x \} \cup \{ P \} \\
  \freenames{P|Q} & := & \freenames{P} \cup \freenames{Q} \\
  \freenames{\dropn{x}} & := & \{ x \}
\end{eqnarray*}

The bound names of a process, $\boundnames{P}$, are those names occurring in $P$
that are not free. For example, in $x?(y).0$, the name $x$ is free, while $y$ is bound.

\begin{mathpar}
  \inferrule* [lab=monoidal-laws] {} { P|Q \equiv Q|P \and P|0 \equiv P \and P|(Q|R) \equiv (P|Q)|R }
\end{mathpar}

\begin{mathpar}
  \inferrule* [lab=alpha-equivalence] {} { (x)P \equiv (y)P\{y/x\} \and y \not\in \freenames{P} }
\end{mathpar}

\begin{definition}
Then two processes, $P,Q$, are alpha-equivalent if $P = Q\{\vec{y}/\vec{x}\}$ for
some $\vec{x} \in \boundnames{Q},\vec{y} \in \boundnames{P}$, where $Q\{\vec{y}/\vec{x}\}$
denotes the capture-avoiding substitution of $\vec{y}$ for $\vec{x}$ in $Q$.
\end{definition}

\begin{definition}
  The {\em structural congruence} \cite{SangiorgiWalker} , $\equiv$,
  between processes is the least congruence containing
  alpha-equivalence, satisfying the abelian monoid laws
  (associativity, commutativity and $\pzero$ as identity) for parallel
  composition $|$ and for summation $+$.
\end{definition}

\subsection{Name equivalence}

We take name equivalence, written $\nameeq$, to be the smallest
equivalence relation generated by the following rules.

\begin{mathpar}
\inferrule*[lab=Quote-drop]
{ }
{ \quotep{@{x}} \nameeq x }

\inferrule*[lab=Struct-equiv]
{ P \scong Q }
{ \quotep{P} \nameeq \quotep{Q} }
\end{mathpar}

The astute reader will have noticed that the mutual recursion of names
and processes imposes a mutual recursion on alpha-equivalence and
structural equivalence via name-equivalence. Fortunately, all of this
works out pleasantly and we may calculate in the natural way, free of
concern. The reader interested in the details is referred to the
appendix \ref{appendix:rho_details}.

\subsection{Substitution}

We use $\Proc$ for the set of processes, $\QProc$ for the set of
names, and $\id{\{}\vec{y} / \vec{x} \id{\}}$ to denote partial maps,
$s : \QProc \rightarrow \QProc$. A map, $s$ lifts, uniquely, to a map
on process terms, $\widehat{s} : \Proc \rightarrow \Proc$ by the
following equations.

\begin{mathpar}
  (0) \psubstp{Q}{P} := 0 \\
  (R \juxtap S) \psubstp{Q}{P}
  :=    
  (R)\psubstp{Q}{P} \juxtap (S) \psubstp{Q}{P} \\
  (x?(y).R) \psubstp{Q}{P}    
  :=    
  (x)\substp{Q}{P} (z)\concat( (R \psubstn{z}{y}) \psubstp{Q}{P} ) \\
  (\lift{x}{R}) \psubstp{Q}{P}  
  :=
  \lift{(x)\substp{Q}{P}}{ R \psubstp{Q}{P} } \\
%   (\dropn{x})  \psubstp{Q}{P}       
%   := 
%   \left\{ 
%     \begin{array}{ccc} 
%       \dropn{\quotep{Q}} & & x \nameeq \quotep{P} \\
%       \dropn{x} & & otherwise \\
%     \end{array}
%   \right. 
  (\dropn{x})  \psubstp{Q}{P}       
  := 
  \left\{ 
    \begin{array}{ccc} 
      Q & & x \nameeq \quotep{P} \\
      \dropn{x} & & otherwise \\
    \end{array}
  \right.
\end{mathpar}
 

where

\begin{eqnarray}
  (x)\id{\{} \lpquote Q \rpquote / \lpquote P \rpquote \id{\}}            = 
  \left\{ 
    \begin{array}{ccc}
      \lpquote Q \rpquote & & x \nameeq \lpquote P \rpquote \\
      x & & otherwise \\
    \end{array}
  \right. \nonumber
\end{eqnarray}

and $z$ is chosen distinct from $\quotep{P}$, $\quotep{Q}$, the free
names in $Q$, and all the names in $R$. Our $\alpha$-equivalence will
be built in the standard way from this substitution.

\begin{remark}\label{rem:no_self_referential_names}
  One consequence of these definitions is that $\forall P. \quotep{P}
  \not\in \freenames{P}$.
\end{remark}

\subsection{ Dynamic quote: an example }

Anticipating something of what's to come, consider applying the
substitution, $\widehat{\id{\{}u / z \id{\}}}$, to the following pair
of processes, $\lift{w}{y!(z)}$ and $w[ \lpquote y!(z) \rpquote ]$.

\begin{eqnarray}
	\lift{w}{y!(z)}\widehat{\id{\{}u / z \id{\}}}
		& = &
		\lift{w}{y!(u)} \nonumber\\
	w[ \lpquote y!(z) \rpquote ] \widehat{ \id{\{}u / z \id{\}} }
		& = &
		w[ \lpquote y!(z) \rpquote ] \nonumber
\end{eqnarray}

Because the body of the process between quotes is impervious to
substitution, we get radically different answers. In fact, by
examining the first process in an input context,
e.g. $x?(z).\lift{w}{y!(z)}$, we see that the process under the lift
operator may be shaped by prefixed inputs binding a name inside it. In
this sense, the lift operator will be seen as a way to dynamically
construct processes before reifying them as names.

Finally equipped with these standard features we can present the
dynamics of the calculus.

\subsubsection{Operational semantics} 

Finally, we introduce the computational dynamics. What marks these
algebras as distinct from other more traditionally studied algebraic
structures, e.g. vector spaces or polynomial rings, is the manner in
which dynamics is captured. In traditional structures, dynamics is typically
expressed through morphisms between such structures, as in linear maps
between vector spaces or morphisms between rings. In algebras
associated with the semantics of computation, the dynamics is
expressed as part of the algebraic structure itself, through a
reduction reduction relation typically denoted by $\red$. Below, we
give a recursive presentation of this relation for the calculus used
in the encoding.

$\red \subseteq \pi \times \pi$
$\red : \pi \to \mathcal{P}(\pi)$

\begin{mathpar}
  \inferrule* [lab=Comm] { \textsf{match}( x_{src}, x_{trgt} ) } { x_{trgt}?(y)P \; | \; x_{src}!\langle {Q} \rangle \red P\{\quotep{Q}/y}\} }
  \and \\
  \inferrule* [lab=Par] {{P} \red {P}'} {{{P} | {Q}} \red {{P}' | {Q}}}
  \and
  \inferrule* [lab=Equiv]{{{P} \scong {P}'} \andalso {{P}' \red {Q}'} \andalso {{Q}' \scong {Q}}}{{P} \red {Q}}
\end{mathpar}

\begin{eqnarray*}
  match_{\equiv} (\quotep{P},\quotep{Q}) & := & P \equiv Q \\
  match_{\dagger}(\quotep{P},\quotep{Q}) & := & \forall R. P|Q \red^{*} R => R \red^{*} 0 \\
  match_{K}(\quotep{P},\quotep{Q}) & := & K \mbox{ for some context } K
\end{eqnarray*}

$u?(x)P | u!\langle Q \rangle \red P\{\quotep{Q}/x\}$

%We write $\wred$ for $\red^*$, and $P\red$ if $\exists Q $ such that $ P \red Q$.
We write $P\red$ if $\exists Q $ such that $ P \red Q$ and $P\not\red$, otherwise.

\section{Replication}

As mentioned before, it is known that replication (and hence
recursion) can be implemented in a higher-order process algebra
\cite{SangiorgiWalker}. As our first example of calculation with the
machinery thus far presented we give the construction explicitly in
the {\rhoc}.

\begin{eqnarray}
	D_{x} & := & \prefix{x}{y}{(\binpar{\outputp{x}{y}}{@{y}})} \nonumber\\
	\bangp_{x}{P} & := & \binpar{{x}!\langle{\binpar{D_{x}}{P}}\rangle}{D_{x}} \nonumber
\end{eqnarray}

\begin{eqnarray}
	\bangp_{x}{P} & & \nonumber\\
	=
	& {x}!\langle{(\prefix{x}{y}{(\outputp{x}{y} | @{y})) | P}}\rangle 
	      | \prefix{x}{y}{(\outputp{x}{y} | @{y})} & \nonumber\\
	\red
	& (\outputp{x}{y} | @{y})\substn{\quotep{(\prefix{x}{y}{(@{y} | \outputp{x}{y})) | P}}}{y} & \nonumber\\
	=
	& \outputp{x}{\quotep{(\prefix{x}{y}{(\outputp{x}{y} | @{y})) | P}}}
	  | {(\prefix{x}{y}{(\outputp{x}{y} | @{y})) | P}} & \nonumber\\
	\red
	& \ldots & \nonumber\\
	\red^*
	& P | P | \ldots & \nonumber
\end{eqnarray}

Of course, this encoding, as an implementation, runs away, unfolding
$\bangp{P}$ eagerly. A lazier and more implementable replication
operator, restricted to input-guarded processes, may be obtained as follows.

\begin{eqnarray}
\bangp{\prefix{u}{v}{P}} 
	:= 
	\binpar{\lift{x}{\prefix{u}{v}{(\binpar{D(x)}{P})}}}{D(x)} \nonumber
\end{eqnarray}

\begin{remark}
  Note that the lazier definition still does not deal with summation
  or mixed summation (i.e. sums over input and output). The reader is
  invited to construct definitions of replication that deal with these
  features. 

  Further, the definitions are parameterized in a name, $x$. Can you,
  gentle reader, make a definition that eliminates this parameter and
  guarantees no accidental interaction between the replication
  machinery and the process being replicated -- i.e. no accidental
  sharing of names used by the process to get its work done and the
  name(s) used by the replication to effect copying. This latter
  revision of the definition of replication is crucial to obtaining
  the expected identity $!!P \sim !P$.
\end{remark}

\begin{remark}\label{rem:paradoxical_combinator}
  The reader familiar with the lambda calculus will have noticed the
  similarity between $D$ and the paradoxical combinator.

  [Ed. note: the existence of this seems to suggest we have to be more
  restrictive on the set of processes and names we admit if we are to
  support no-cloning.]
\end{remark}

\subsubsection{Bisimulation}

The computational dynamics gives rise to another kind of equivalence,
the equivalence of computational behavior. As previously mentioned
this is typically captured \emph{via} some form of bisimulation.

% The notion we use in this paper is weak barbed bisimulation
% \cite{milner91polyadicpi}.

The notion we use in this paper is derived from weak barbed
bisimulation \cite{milner91polyadicpi}. 

\begin{definition}
An \emph{observation relation}, $\downarrow_{\mathcal N}$, over a set
of names, $\mathcal N$, is the smallest relation satisfying the rules
below.

\infrule[Out-barb]{y \in {\mathcal N}, \; x \nameeq y}
		  {\outputp{x}{v} \downarrow_{\mathcal N} x}
\infrule[Par-barb]{\mbox{$P\downarrow_{\mathcal N} x$ or $Q\downarrow_{\mathcal N} x$}}
		  {\binpar{P}{Q} \downarrow_{\mathcal N} x}

We write $P \Downarrow_{\mathcal N} x$ if there is $Q$ such that 
$P \wred Q$ and $Q \downarrow_{\mathcal N} x$.
\end{definition}

\begin{definition}
%\label{def.bbisim}
An  ${\mathcal N}$-\emph{barbed bisimulation} over a set of names, ${\mathcal N}$, is a symmetric binary relation 
${\mathcal S}_{\mathcal N}$ between agents such that $P\rel{S}_{\mathcal N}Q$ implies:
\begin{enumerate}
\item If $P \red P'$ then $Q \wred Q'$ and $P'\rel{S}_{\mathcal N} Q'$.
\item If $P\downarrow_{\mathcal N} x$, then $Q\Downarrow_{\mathcal N} x$.
\end{enumerate}
$P$ is ${\mathcal N}$-barbed bisimilar to $Q$, written
$P \wbbisim_{\mathcal N} Q$, if $P \rel{S}_{\mathcal N} Q$ for some ${\mathcal N}$-barbed bisimulation ${\mathcal S}_{\mathcal N}$.
\end{definition}

$\mathcal{R} \subseteq \pi \times \pi$

$P \mathcal{R} Q => \forall P'. P \red P' \Rightarrow \exists Q'. Q \red Q', P' \mathcal{R} Q'$

$P \vdash x \Rightarrow Q \vdash x$

\begin{mathpar}
  \inferrule*[lab=Out-barb]{x \nameeq y}{{y}!\langle{Q}\rangle \vdash x}
  \and
  \inferrule*[lab=Par-barb]{\mbox{$P\vdash x$ or $Q\vdash x$}}{\binpar{P}{Q} \vdash x}
\end{mathpar}

\subsubsection{Contexts}

One of the principle advantages of computational calculi like the
$\pi$-calculus is a well-defined notion of context,
contextual-equivalence and a correlation between
contextual-equivalence and notions of bisimulation. The notion of
context allows the decomposition of a process into (sub-)process and
its syntactic environment, its context. Thus, a context may be
thought of as a process with a ``hole'' (written $\Box$) in it. The
application of a context $M$ to a process $P$, written $M[P]$, is
tantamount to filling the hole in $M$ with $P$. In this paper we do
not need the full weight of this theory, but do make use of the notion
of context in the proof the main theorem. 

\begin{mathpar}
  \inferrule* [lab=summation] {} {{M_{M},M_{N}} \bc \Box \;|\; x.M_{A} \;|\; M_{M}+M_{N}}
  \and
  \inferrule* [lab=agent] {} {{M_{A}} \bc (\vec{x})M_{P} \;| \; \clift{P_0,\ldots,M_{P},\ldots,P_N}}
  \and \\
  \inferrule* [lab=process] {} {{M_{P}} \bc M_{N} \;| \;P|M_{P} }
\end{mathpar} 

\begin{mathpar}
  \inferrule* [lab=sychronization] {} {M_{N} \bc \Box \;|\; x?M_{F} \;|\; x!M_{C}}
  \and
  \inferrule* [lab=abstraction] {} {{M_{F}} \bc (x)M_{P} }
  \and
  \inferrule* [lab=concretion] {} {{M_{C}} \bc \langle M_{P} \rangle }
  \and \\
  \inferrule* [lab=process] {} {{M_{P}} \bc M_{N} \;| \;P|M_{P} }
\end{mathpar}

\begin{definition}[contextual application] Given a context $M$, and
  process $P$, we define the \emph{contextual application}, $M[P] :=
  M\{P/\Box\}$. That is, the contextual application of M to P is the
  substitution of $P$ for $\Box$ in $M$.
\end{definition}

$\meaningof{-} : L \to \mathcal{P}(\pi)$

\begin{mathpar}
  \inferrule* [lab=collection] {} {\meaningof{true} = \pi, \and \meaningof{~E} = \pi \setminus \meaningof{E}, \and \meaningof{E_{1} \& E_{2}} = \meaningof{E_{1}} \cap \meaningof{E_{2}}}
\end{mathpar}

\begin{mathpar}
  \inferrule* [lab=structure] {} {\meaningof{0} = \{ P \in \pi | P \equiv 0 \}, \and \\ \meaningof{E_1 | E_2} = \{ P \in \pi | P \equiv P_{1} | P_{2}, P_{1} \in \meaningof{E_{1}}, P_{2} \in \meaningof{E_2}\} }
\end{mathpar}

\begin{mathpar}
 \inferrule* [lab=behavior] {} {\meaningof{\langle a?b \rangle E} = \{ P \in \pi | P \equiv Q | u?(y)P', \\ \and \\\\ \and \\ \;\;\; u \in \meaningof{a}, \forall z.P'\{z/y\} \in \meaningof{E\{z/b\}}\}, \and \\ \meaningof{a!E} = \{ P \in \pi | P \equiv Q | x!\langle P' \rangle, x \in \meaningof{a} P' \in \meaningof{E}\} }
\end{mathpar}

\begin{mathpar}
 \inferrule* [lab=nominal] {} {\meaningof{\quotep{E}} = \{ \quotep{P} \in \quotep{\pi} | P \in \meaningof{E} \}, \and \meaningof{\quotep{P}} = \{ \quotep{Q} \in \quotep{\pi} | P \equiv Q \} \and \\ \meaningof{@\quotep{E}} = \{ P \in \pi | P \equiv @x, x \in \meaningof{E} \}}
\end{mathpar}

\begin{eqnarray*}
  \\
  \meaningof{-} : TS \to ST
\end{eqnarray*}

\begin{eqnarray*}
  \\
  L : TS \to ST
\end{eqnarray*}

\begin{eqnarray*}
  \\
  P \models E \iff P \in \meaningof{E}
\end{eqnarray*}

\begin{eqnarray*}
  P \approx_{L} Q \iff \forall E \in L. P \models E \iff Q \models E
\end{eqnarray*}

\begin{eqnarray*}
  P \approx_{K} Q
\end{eqnarray*}

\begin{eqnarray*}
  P \approx Q
\end{eqnarray*}

$\approx_{K} = \approx = \approx_{L}$

\subsubsection{Contextual duality}

Note that contexts extend the quotation operation to a family of
operations from processes to names. Given a context, $M$, we can
define a \emph{nominal context}, $\quotep{M}$ by $\quotep{M}[P] :=
\quotep{M[P]}$. To foreshadow what is to come we observe that these
operations enjoy a duality with processes very much like the duality
between vectors and maps from vectors to scalars.

Further, because the calculus is essentially higher-order, we have a
correspondence between contexts and processes. More specifically,
given a name $x$ and a context $M$ we can construct $M^{*}_{x}$ such
that 

\begin{mathpar}
  M^{*}_{x} | \lift{x}{P} \red M[P]
\end{mathpar}

namely,

\begin{mathpar}
  M^{*}_{x} := x?(u).M[\dropn{u}]
\end{mathpar}

The dependence of $M^{*}_{x}$ on a name makes it an abstraction, 

\begin{mathpar}
  M^{*} := (x)x?(u).M[\dropn{u}]
\end{mathpar}

\subsection{Additional notation}

It will sometimes be convenient to denote the process a name
quotes. We already have the notation $x = \quotep{P}$, but it will be
convenient to introduce an alternate notation, $\procn{x}$, when we
want to emphasize the connection to the use of the name. Note that, by
virtue of name equivalence, $\quotep{\procn{x}} \nameeq x$; so, the
notation is consistent with previous definitions.

Further, because names have structure it is possible to effect
substitutions on the basis of that structure. This means we need to
upgrade our notation for substitutions, which we accomplish by
adapting comprehension notation. Thus,

\begin{mathpar}
  P\{ y / x : x \in S \}
\end{mathpar}

is interpreted to mean the process derived from P by replacing (in a
capture-avoiding manner) each occurrence of $x$ in $S$ by $y$. For example,

\begin{mathpar}
  P\{ \quotep{\procn{x}|\procn{x}} / x : x \in \freenames{P} \}
\end{mathpar}

will replace each (occurrence) of a free name $x$ in $P$ by
$\quotep{\procn{x}|\procn{x}}$.

Also, we will avail ourselves of the notation $x^{L}$ and $x^{R}$ to
denote injections of a name into disjoint copies of the name
space. There are numerous ways to accomplish this. One example can be
found in \cite{MeredithR05}. This notation overloads to vectors of
names: $\vec{x}^{\pi} := (x_{i}^{\pi} \; : \; 0 \leq i < |\vec{x}| )$ where $\pi \in \{L,R\}$.

We also use $P^{\Box} := P|\Box$.

In \cite{MeredithR05} an interpretation of the new operator is
given. It turns out that there are several possible interpretations
all enjoying the requisite algebraic properties of the operator (see
\cite{milner91polyadicpi}). We will therefore make liberal use of
$(\nu\; \vec{x})P$.

% subsection the_syntax_and_semantics_of_the_notation_system (end)   

\input{qm2pi.qmops} 

\input{qm2pi.sterngerlach} 

\input{qm2pi.metric} 

% section concurrent_process_calculi (end)

%\input{qm2pi.proofsketch}

% section proof sketch (end)

%\input{qm2pi.slviaknots} 

% section spatial logic via knots (end)

\input{qm2pi.conclusion}

% section conclusion (end)

%\input{qm2pi.dtcodes} 

% section wiring algorithm (end)

\input{qm2pi.ack} 

% section acknowledgments (end)

\newpage


\bibliographystyle{plain}   
\bibliography{../../biblios/main.bib}

\input{qm2pi.rhodetails}

\end{document}

 

% section concurrent_process_calculi (end)

%\documentclass[12pt]{llncs}
%\documentclass{jktr}

\usepackage[pdftex]{hyperref}                   
\usepackage {listings}
\usepackage {mathpartir}
\usepackage{bcprules}
%\usepackage{listings}
                       
\usepackage{graphicx} 
%\usepackage[margins=2.5cm,nohead,nofoot]{geometry}
%\usepackage{geometry}
\usepackage{amsfonts}
\usepackage{amstext}
\usepackage{latexsym}
\usepackage{amssymb}
\usepackage{color}


%\include{myPreamble}
\include{qm2pi.local} 

%\ifpdf
%\usepackage[pdftex]{graphicx}
%\else
%\usepackage{graphicx}
%\fi

 % \ifpdf
%  \usepackage{pdfsync}
%  \if


%\title{Brief Article}
%\author{David F. Snyder}
%\author{L.G. Meredith}

%\address{Dept. of Math., Texas State University--San Marcos, San Marcos, TX 78666}
       
\pagestyle{empty}


\begin{document}

\lstset{language=[Objective]Caml,frame=shadowbox}

\input{qm2pi.front}

% section front matter (end)

\input{qm2pi.intro} 
 
% section introduction (end)

% \input{qm2pi.knotations} 

% section notation (end)

\input{qm2pi.process.calculi} 

% section concurrent_process_calculi_and_spatial_logics_ (end)
    
%\input{qm2pi.knots2pi} 

%\input{qm2pi.trefoil} 

%\input{qm2pi.mainthm} 

% subsection basic_interpretation (end)

%\input{qm2pi.rho.presentation} 
\subsection{The syntax and semantics of the notation system}\label{sub:the_syntax_and_semantics_of_the_notation_system} % (fold)

We now summarize a technical presentation of the calculus that
embodies our theory of dynamics. The typical presentation of such a
calculus follows the style of giving generators and relations on
them. The grammar, below, describing term constructors, freely
generates the set of processes, $\Proc$. This set is then quotiented
by a relation known as structural congruence and it is over this set
that the notion of dynamics is expressed. This presentation is
essentially that of \cite{MeredithR05} with the addition of
polyadicity and summation. For readability we have relegated some of
the technical subtleties to an appendix.

\subsubsection{Process grammar}\label{subsub:process_grammar}

\begin{mathpar}
  \inferrule* [lab=synchronization] {} {{M} \bc \pzero \;|\; x?F \;|\; x!C }
  \and
  \inferrule* [lab=abstraction] {} {{F} \bc (x)P}
  \and
  \inferrule* [lab=concretion] {} {{C} \bc \langle Q \rangle}
  \and
  \inferrule* [lab=process] {} {{P,Q} \bc M \;| \;P|Q \;|\; @{x}}
  \and
  \inferrule* [lab=name] {} {{x} \bc \quotep{P}}
\end{mathpar} 

Note that $\vec{x}$ (resp. $\vec{P}$) denotes a vector of names
(resp. processes) of length $|\vec{x}|$ (resp. $|\vec{P}|$). We adopt
the following useful abbreviations.

\begin{mathpar}
   x?(\vec{y}).P := x.(\vec{y})P \and  x\clift{\vec{P}} := x.\clift{\vec{P}}
   \and x!(y) := \lift{x}{\dropn{y}}
   \and \Pi_{i=0}^{n-1}P_i := P_0 | \ldots | P_{n-1}
\end{mathpar}

\subsubsection{Structural congruence}

\paragraph{Free and bound names and alpha-equivalence.} At the
core of structural equivalence is alpha-equivalence which identifies
process that are the same up to a change of variable. Formally, we
recognize the distinction between free and bound names. The free names
of a process, $\freenames{P}$, may be calculated recursively as
follows:

\begin{mathpar}
\freenames{\pzero} := \emptyset
  \and \\
  \freenames{x?(y).P} := \{ x \} \cup (\freenames{P} \setminus \{ y \})
  \and 
  \freenames{x!\langle P \rangle} := \{ x \} \cup \{ P \} 
  \and \\
  \freenames{P|Q} := \freenames{P} \cup \freenames{Q}
  \and \\
  \freenames{@{x}} := \{ x \}
\end{mathpar}

$\pi$
$\quotep{\pi}$

$\freenames{-} : \pi \to \mathcal{P}(\quotep{\pi})$

\begin{eqnarray*}
  \freenames{\pzero} & := & \emptyset \\
  \freenames{x?(y).P} & := & \{ x \} \cup (\freenames{P} \setminus \{ y \}) \\
  \freenames{x!\langle P \rangle} & := & \{ x \} \cup \{ P \} \\
  \freenames{P|Q} & := & \freenames{P} \cup \freenames{Q} \\
  \freenames{\dropn{x}} & := & \{ x \}
\end{eqnarray*}

The bound names of a process, $\boundnames{P}$, are those names occurring in $P$
that are not free. For example, in $x?(y).0$, the name $x$ is free, while $y$ is bound.

\begin{mathpar}
  \inferrule* [lab=monoidal-laws] {} { P|Q \equiv Q|P \and P|0 \equiv P \and P|(Q|R) \equiv (P|Q)|R }
\end{mathpar}

\begin{mathpar}
  \inferrule* [lab=alpha-equivalence] {} { (x)P \equiv (y)P\{y/x\} \and y \not\in \freenames{P} }
\end{mathpar}

\begin{definition}
Then two processes, $P,Q$, are alpha-equivalent if $P = Q\{\vec{y}/\vec{x}\}$ for
some $\vec{x} \in \boundnames{Q},\vec{y} \in \boundnames{P}$, where $Q\{\vec{y}/\vec{x}\}$
denotes the capture-avoiding substitution of $\vec{y}$ for $\vec{x}$ in $Q$.
\end{definition}

\begin{definition}
  The {\em structural congruence} \cite{SangiorgiWalker} , $\equiv$,
  between processes is the least congruence containing
  alpha-equivalence, satisfying the abelian monoid laws
  (associativity, commutativity and $\pzero$ as identity) for parallel
  composition $|$ and for summation $+$.
\end{definition}

\subsection{Name equivalence}

We take name equivalence, written $\nameeq$, to be the smallest
equivalence relation generated by the following rules.

\begin{mathpar}
\inferrule*[lab=Quote-drop]
{ }
{ \quotep{@{x}} \nameeq x }

\inferrule*[lab=Struct-equiv]
{ P \scong Q }
{ \quotep{P} \nameeq \quotep{Q} }
\end{mathpar}

The astute reader will have noticed that the mutual recursion of names
and processes imposes a mutual recursion on alpha-equivalence and
structural equivalence via name-equivalence. Fortunately, all of this
works out pleasantly and we may calculate in the natural way, free of
concern. The reader interested in the details is referred to the
appendix \ref{appendix:rho_details}.

\subsection{Substitution}

We use $\Proc$ for the set of processes, $\QProc$ for the set of
names, and $\id{\{}\vec{y} / \vec{x} \id{\}}$ to denote partial maps,
$s : \QProc \rightarrow \QProc$. A map, $s$ lifts, uniquely, to a map
on process terms, $\widehat{s} : \Proc \rightarrow \Proc$ by the
following equations.

\begin{mathpar}
  (0) \psubstp{Q}{P} := 0 \\
  (R \juxtap S) \psubstp{Q}{P}
  :=    
  (R)\psubstp{Q}{P} \juxtap (S) \psubstp{Q}{P} \\
  (x?(y).R) \psubstp{Q}{P}    
  :=    
  (x)\substp{Q}{P} (z)\concat( (R \psubstn{z}{y}) \psubstp{Q}{P} ) \\
  (\lift{x}{R}) \psubstp{Q}{P}  
  :=
  \lift{(x)\substp{Q}{P}}{ R \psubstp{Q}{P} } \\
%   (\dropn{x})  \psubstp{Q}{P}       
%   := 
%   \left\{ 
%     \begin{array}{ccc} 
%       \dropn{\quotep{Q}} & & x \nameeq \quotep{P} \\
%       \dropn{x} & & otherwise \\
%     \end{array}
%   \right. 
  (\dropn{x})  \psubstp{Q}{P}       
  := 
  \left\{ 
    \begin{array}{ccc} 
      Q & & x \nameeq \quotep{P} \\
      \dropn{x} & & otherwise \\
    \end{array}
  \right.
\end{mathpar}
 

where

\begin{eqnarray}
  (x)\id{\{} \lpquote Q \rpquote / \lpquote P \rpquote \id{\}}            = 
  \left\{ 
    \begin{array}{ccc}
      \lpquote Q \rpquote & & x \nameeq \lpquote P \rpquote \\
      x & & otherwise \\
    \end{array}
  \right. \nonumber
\end{eqnarray}

and $z$ is chosen distinct from $\quotep{P}$, $\quotep{Q}$, the free
names in $Q$, and all the names in $R$. Our $\alpha$-equivalence will
be built in the standard way from this substitution.

\begin{remark}\label{rem:no_self_referential_names}
  One consequence of these definitions is that $\forall P. \quotep{P}
  \not\in \freenames{P}$.
\end{remark}

\subsection{ Dynamic quote: an example }

Anticipating something of what's to come, consider applying the
substitution, $\widehat{\id{\{}u / z \id{\}}}$, to the following pair
of processes, $\lift{w}{y!(z)}$ and $w[ \lpquote y!(z) \rpquote ]$.

\begin{eqnarray}
	\lift{w}{y!(z)}\widehat{\id{\{}u / z \id{\}}}
		& = &
		\lift{w}{y!(u)} \nonumber\\
	w[ \lpquote y!(z) \rpquote ] \widehat{ \id{\{}u / z \id{\}} }
		& = &
		w[ \lpquote y!(z) \rpquote ] \nonumber
\end{eqnarray}

Because the body of the process between quotes is impervious to
substitution, we get radically different answers. In fact, by
examining the first process in an input context,
e.g. $x?(z).\lift{w}{y!(z)}$, we see that the process under the lift
operator may be shaped by prefixed inputs binding a name inside it. In
this sense, the lift operator will be seen as a way to dynamically
construct processes before reifying them as names.

Finally equipped with these standard features we can present the
dynamics of the calculus.

\subsubsection{Operational semantics} 

Finally, we introduce the computational dynamics. What marks these
algebras as distinct from other more traditionally studied algebraic
structures, e.g. vector spaces or polynomial rings, is the manner in
which dynamics is captured. In traditional structures, dynamics is typically
expressed through morphisms between such structures, as in linear maps
between vector spaces or morphisms between rings. In algebras
associated with the semantics of computation, the dynamics is
expressed as part of the algebraic structure itself, through a
reduction reduction relation typically denoted by $\red$. Below, we
give a recursive presentation of this relation for the calculus used
in the encoding.

$\red \subseteq \pi \times \pi$
$\red : \pi \to \mathcal{P}(\pi)$

\begin{mathpar}
  \inferrule* [lab=Comm] { \textsf{match}( x_{src}, x_{trgt} ) } { x_{trgt}?(y)P \; | \; x_{src}!\langle {Q} \rangle \red P\{\quotep{Q}/y}\} }
  \and \\
  \inferrule* [lab=Par] {{P} \red {P}'} {{{P} | {Q}} \red {{P}' | {Q}}}
  \and
  \inferrule* [lab=Equiv]{{{P} \scong {P}'} \andalso {{P}' \red {Q}'} \andalso {{Q}' \scong {Q}}}{{P} \red {Q}}
\end{mathpar}

\begin{eqnarray*}
  match_{\equiv} (\quotep{P},\quotep{Q}) & := & P \equiv Q \\
  match_{\dagger}(\quotep{P},\quotep{Q}) & := & \forall R. P|Q \red^{*} R => R \red^{*} 0 \\
  match_{K}(\quotep{P},\quotep{Q}) & := & K \mbox{ for some context } K
\end{eqnarray*}

$u?(x)P | u!\langle Q \rangle \red P\{\quotep{Q}/x\}$

%We write $\wred$ for $\red^*$, and $P\red$ if $\exists Q $ such that $ P \red Q$.
We write $P\red$ if $\exists Q $ such that $ P \red Q$ and $P\not\red$, otherwise.

\section{Replication}

As mentioned before, it is known that replication (and hence
recursion) can be implemented in a higher-order process algebra
\cite{SangiorgiWalker}. As our first example of calculation with the
machinery thus far presented we give the construction explicitly in
the {\rhoc}.

\begin{eqnarray}
	D_{x} & := & \prefix{x}{y}{(\binpar{\outputp{x}{y}}{@{y}})} \nonumber\\
	\bangp_{x}{P} & := & \binpar{{x}!\langle{\binpar{D_{x}}{P}}\rangle}{D_{x}} \nonumber
\end{eqnarray}

\begin{eqnarray}
	\bangp_{x}{P} & & \nonumber\\
	=
	& {x}!\langle{(\prefix{x}{y}{(\outputp{x}{y} | @{y})) | P}}\rangle 
	      | \prefix{x}{y}{(\outputp{x}{y} | @{y})} & \nonumber\\
	\red
	& (\outputp{x}{y} | @{y})\substn{\quotep{(\prefix{x}{y}{(@{y} | \outputp{x}{y})) | P}}}{y} & \nonumber\\
	=
	& \outputp{x}{\quotep{(\prefix{x}{y}{(\outputp{x}{y} | @{y})) | P}}}
	  | {(\prefix{x}{y}{(\outputp{x}{y} | @{y})) | P}} & \nonumber\\
	\red
	& \ldots & \nonumber\\
	\red^*
	& P | P | \ldots & \nonumber
\end{eqnarray}

Of course, this encoding, as an implementation, runs away, unfolding
$\bangp{P}$ eagerly. A lazier and more implementable replication
operator, restricted to input-guarded processes, may be obtained as follows.

\begin{eqnarray}
\bangp{\prefix{u}{v}{P}} 
	:= 
	\binpar{\lift{x}{\prefix{u}{v}{(\binpar{D(x)}{P})}}}{D(x)} \nonumber
\end{eqnarray}

\begin{remark}
  Note that the lazier definition still does not deal with summation
  or mixed summation (i.e. sums over input and output). The reader is
  invited to construct definitions of replication that deal with these
  features. 

  Further, the definitions are parameterized in a name, $x$. Can you,
  gentle reader, make a definition that eliminates this parameter and
  guarantees no accidental interaction between the replication
  machinery and the process being replicated -- i.e. no accidental
  sharing of names used by the process to get its work done and the
  name(s) used by the replication to effect copying. This latter
  revision of the definition of replication is crucial to obtaining
  the expected identity $!!P \sim !P$.
\end{remark}

\begin{remark}\label{rem:paradoxical_combinator}
  The reader familiar with the lambda calculus will have noticed the
  similarity between $D$ and the paradoxical combinator.

  [Ed. note: the existence of this seems to suggest we have to be more
  restrictive on the set of processes and names we admit if we are to
  support no-cloning.]
\end{remark}

\subsubsection{Bisimulation}

The computational dynamics gives rise to another kind of equivalence,
the equivalence of computational behavior. As previously mentioned
this is typically captured \emph{via} some form of bisimulation.

% The notion we use in this paper is weak barbed bisimulation
% \cite{milner91polyadicpi}.

The notion we use in this paper is derived from weak barbed
bisimulation \cite{milner91polyadicpi}. 

\begin{definition}
An \emph{observation relation}, $\downarrow_{\mathcal N}$, over a set
of names, $\mathcal N$, is the smallest relation satisfying the rules
below.

\infrule[Out-barb]{y \in {\mathcal N}, \; x \nameeq y}
		  {\outputp{x}{v} \downarrow_{\mathcal N} x}
\infrule[Par-barb]{\mbox{$P\downarrow_{\mathcal N} x$ or $Q\downarrow_{\mathcal N} x$}}
		  {\binpar{P}{Q} \downarrow_{\mathcal N} x}

We write $P \Downarrow_{\mathcal N} x$ if there is $Q$ such that 
$P \wred Q$ and $Q \downarrow_{\mathcal N} x$.
\end{definition}

\begin{definition}
%\label{def.bbisim}
An  ${\mathcal N}$-\emph{barbed bisimulation} over a set of names, ${\mathcal N}$, is a symmetric binary relation 
${\mathcal S}_{\mathcal N}$ between agents such that $P\rel{S}_{\mathcal N}Q$ implies:
\begin{enumerate}
\item If $P \red P'$ then $Q \wred Q'$ and $P'\rel{S}_{\mathcal N} Q'$.
\item If $P\downarrow_{\mathcal N} x$, then $Q\Downarrow_{\mathcal N} x$.
\end{enumerate}
$P$ is ${\mathcal N}$-barbed bisimilar to $Q$, written
$P \wbbisim_{\mathcal N} Q$, if $P \rel{S}_{\mathcal N} Q$ for some ${\mathcal N}$-barbed bisimulation ${\mathcal S}_{\mathcal N}$.
\end{definition}

$\mathcal{R} \subseteq \pi \times \pi$

$P \mathcal{R} Q => \forall P'. P \red P' \Rightarrow \exists Q'. Q \red Q', P' \mathcal{R} Q'$

$P \vdash x \Rightarrow Q \vdash x$

\begin{mathpar}
  \inferrule*[lab=Out-barb]{x \nameeq y}{{y}!\langle{Q}\rangle \vdash x}
  \and
  \inferrule*[lab=Par-barb]{\mbox{$P\vdash x$ or $Q\vdash x$}}{\binpar{P}{Q} \vdash x}
\end{mathpar}

\subsubsection{Contexts}

One of the principle advantages of computational calculi like the
$\pi$-calculus is a well-defined notion of context,
contextual-equivalence and a correlation between
contextual-equivalence and notions of bisimulation. The notion of
context allows the decomposition of a process into (sub-)process and
its syntactic environment, its context. Thus, a context may be
thought of as a process with a ``hole'' (written $\Box$) in it. The
application of a context $M$ to a process $P$, written $M[P]$, is
tantamount to filling the hole in $M$ with $P$. In this paper we do
not need the full weight of this theory, but do make use of the notion
of context in the proof the main theorem. 

\begin{mathpar}
  \inferrule* [lab=summation] {} {{M_{M},M_{N}} \bc \Box \;|\; x.M_{A} \;|\; M_{M}+M_{N}}
  \and
  \inferrule* [lab=agent] {} {{M_{A}} \bc (\vec{x})M_{P} \;| \; \clift{P_0,\ldots,M_{P},\ldots,P_N}}
  \and \\
  \inferrule* [lab=process] {} {{M_{P}} \bc M_{N} \;| \;P|M_{P} }
\end{mathpar} 

\begin{mathpar}
  \inferrule* [lab=sychronization] {} {M_{N} \bc \Box \;|\; x?M_{F} \;|\; x!M_{C}}
  \and
  \inferrule* [lab=abstraction] {} {{M_{F}} \bc (x)M_{P} }
  \and
  \inferrule* [lab=concretion] {} {{M_{C}} \bc \langle M_{P} \rangle }
  \and \\
  \inferrule* [lab=process] {} {{M_{P}} \bc M_{N} \;| \;P|M_{P} }
\end{mathpar}

\begin{definition}[contextual application] Given a context $M$, and
  process $P$, we define the \emph{contextual application}, $M[P] :=
  M\{P/\Box\}$. That is, the contextual application of M to P is the
  substitution of $P$ for $\Box$ in $M$.
\end{definition}

$\meaningof{-} : L \to \mathcal{P}(\pi)$

\begin{mathpar}
  \inferrule* [lab=collection] {} {\meaningof{true} = \pi, \and \meaningof{~E} = \pi \setminus \meaningof{E}, \and \meaningof{E_{1} \& E_{2}} = \meaningof{E_{1}} \cap \meaningof{E_{2}}}
\end{mathpar}

\begin{mathpar}
  \inferrule* [lab=structure] {} {\meaningof{0} = \{ P \in \pi | P \equiv 0 \}, \and \\ \meaningof{E_1 | E_2} = \{ P \in \pi | P \equiv P_{1} | P_{2}, P_{1} \in \meaningof{E_{1}}, P_{2} \in \meaningof{E_2}\} }
\end{mathpar}

\begin{mathpar}
 \inferrule* [lab=behavior] {} {\meaningof{\langle a?b \rangle E} = \{ P \in \pi | P \equiv Q | u?(y)P', \\ \and \\\\ \and \\ \;\;\; u \in \meaningof{a}, \forall z.P'\{z/y\} \in \meaningof{E\{z/b\}}\}, \and \\ \meaningof{a!E} = \{ P \in \pi | P \equiv Q | x!\langle P' \rangle, x \in \meaningof{a} P' \in \meaningof{E}\} }
\end{mathpar}

\begin{mathpar}
 \inferrule* [lab=nominal] {} {\meaningof{\quotep{E}} = \{ \quotep{P} \in \quotep{\pi} | P \in \meaningof{E} \}, \and \meaningof{\quotep{P}} = \{ \quotep{Q} \in \quotep{\pi} | P \equiv Q \} \and \\ \meaningof{@\quotep{E}} = \{ P \in \pi | P \equiv @x, x \in \meaningof{E} \}}
\end{mathpar}

\begin{eqnarray*}
  \\
  \meaningof{-} : TS \to ST
\end{eqnarray*}

\begin{eqnarray*}
  \\
  L : TS \to ST
\end{eqnarray*}

\begin{eqnarray*}
  \\
  P \models E \iff P \in \meaningof{E}
\end{eqnarray*}

\begin{eqnarray*}
  P \approx_{L} Q \iff \forall E \in L. P \models E \iff Q \models E
\end{eqnarray*}

\begin{eqnarray*}
  P \approx_{K} Q
\end{eqnarray*}

\begin{eqnarray*}
  P \approx Q
\end{eqnarray*}

$\approx_{K} = \approx = \approx_{L}$

\subsubsection{Contextual duality}

Note that contexts extend the quotation operation to a family of
operations from processes to names. Given a context, $M$, we can
define a \emph{nominal context}, $\quotep{M}$ by $\quotep{M}[P] :=
\quotep{M[P]}$. To foreshadow what is to come we observe that these
operations enjoy a duality with processes very much like the duality
between vectors and maps from vectors to scalars.

Further, because the calculus is essentially higher-order, we have a
correspondence between contexts and processes. More specifically,
given a name $x$ and a context $M$ we can construct $M^{*}_{x}$ such
that 

\begin{mathpar}
  M^{*}_{x} | \lift{x}{P} \red M[P]
\end{mathpar}

namely,

\begin{mathpar}
  M^{*}_{x} := x?(u).M[\dropn{u}]
\end{mathpar}

The dependence of $M^{*}_{x}$ on a name makes it an abstraction, 

\begin{mathpar}
  M^{*} := (x)x?(u).M[\dropn{u}]
\end{mathpar}

\subsection{Additional notation}

It will sometimes be convenient to denote the process a name
quotes. We already have the notation $x = \quotep{P}$, but it will be
convenient to introduce an alternate notation, $\procn{x}$, when we
want to emphasize the connection to the use of the name. Note that, by
virtue of name equivalence, $\quotep{\procn{x}} \nameeq x$; so, the
notation is consistent with previous definitions.

Further, because names have structure it is possible to effect
substitutions on the basis of that structure. This means we need to
upgrade our notation for substitutions, which we accomplish by
adapting comprehension notation. Thus,

\begin{mathpar}
  P\{ y / x : x \in S \}
\end{mathpar}

is interpreted to mean the process derived from P by replacing (in a
capture-avoiding manner) each occurrence of $x$ in $S$ by $y$. For example,

\begin{mathpar}
  P\{ \quotep{\procn{x}|\procn{x}} / x : x \in \freenames{P} \}
\end{mathpar}

will replace each (occurrence) of a free name $x$ in $P$ by
$\quotep{\procn{x}|\procn{x}}$.

Also, we will avail ourselves of the notation $x^{L}$ and $x^{R}$ to
denote injections of a name into disjoint copies of the name
space. There are numerous ways to accomplish this. One example can be
found in \cite{MeredithR05}. This notation overloads to vectors of
names: $\vec{x}^{\pi} := (x_{i}^{\pi} \; : \; 0 \leq i < |\vec{x}| )$ where $\pi \in \{L,R\}$.

We also use $P^{\Box} := P|\Box$.

In \cite{MeredithR05} an interpretation of the new operator is
given. It turns out that there are several possible interpretations
all enjoying the requisite algebraic properties of the operator (see
\cite{milner91polyadicpi}). We will therefore make liberal use of
$(\nu\; \vec{x})P$.

% subsection the_syntax_and_semantics_of_the_notation_system (end)   

\input{qm2pi.qmops} 

\input{qm2pi.sterngerlach} 

\input{qm2pi.metric} 

% section concurrent_process_calculi (end)

%\input{qm2pi.proofsketch}

% section proof sketch (end)

%\input{qm2pi.slviaknots} 

% section spatial logic via knots (end)

\input{qm2pi.conclusion}

% section conclusion (end)

%\input{qm2pi.dtcodes} 

% section wiring algorithm (end)

\input{qm2pi.ack} 

% section acknowledgments (end)

\newpage


\bibliographystyle{plain}   
\bibliography{../../biblios/main.bib}

\input{qm2pi.rhodetails}

\end{document}



% section proof sketch (end)

%\section{Unlikely characters: spatial logic for
  knots}\label{sub:characteristic_formulae} % (fold)

Associated to the mobile process calculi are a family of logics known
as the Hennessy-Milner logics. These logics typically enjoy a
semantics interpreting formulae as sets of processes that when
factored through the encoding outlined above allows an identification
of classes of knots with logical formulae. In the context of this
encoding the sub-family known as the spatial logics \cite{CairesC03}
\cite{CairesC04} \cite{Caires04} are of particular interest providing
several important features for expressing and reasoning about
properties (i.e. classes) of knots. We hint here at how this may be done.

%\begin{description}
%\item [structural connectives] 
\subsubsection{Structural connectives} The spatial logics enjoy
structural connectives corresponding, at the logical level, to the
parallel composition ($P | Q$) and new name ($(\nu \; x)P$)
connectives for processes. As illustrated in the examples below, these
connectives are extremely expressive given the shape of our encoding.
%\item [decideable satisfaction]

\subsubsection{Decideable satisfaction}
In \cite{Caires04} the satisfaction relation is shown to be decideable
for a rich class of processes. It further turns out that the image of
the our encoding is a proper subset of that class. This result
provides the basis for an algorithm by which to search for knots
enjoying a given property.
%\item [characteristic formulae]

\subsubsection{Characteristic formulae}
In the same paper \cite{Caires04} , Caires presents a means of calculating
characteristic formulae, selecting equivalence classes of processes
up to a pre--specified depth limit on the support set of names. Composed with our
encoding, this characteristic formula can be used to select
characteristic formulae for knots.
%\end{description}

\subsubsection{Spatial logic formulae}

The grammar below (segmented for comprehension) summarizes the syntax
of spatial logic formulae. We employ illustrative examples in the
sequel to provide an intuitive understanding of their meaning
referring the reader to \cite{Caires04} for a more detailed explication
of the semantics.

\begin{mathpar}
  \inferrule* [lab=boolean] {} {{A,B} \bc T \;|\; \neg A \;|\; A \wedge B \;|\; \eta = \eta'}
  \and
  \inferrule* [lab=spatial] {} {|\; \pzero \;|\; A | B \;|\; x \text{\textregistered} A \;|\; \forall x . A \;|\;  H x . A}
  \and
  \inferrule* [lab=behavioral] {} {|\; \alpha . A}
  \and 
  \inferrule* [lab=recursion] {} {|\; X(\vec{u}) \;|\; \mu X(\vec{u}) . A}
  \and
  \inferrule* [lab=action] {} {\alpha \bc \langle x?(\vec{y}) \rangle \;|\; \langle x!(\vec{y}) \rangle \;|\; \langle \tau \rangle}
  \and 
  \inferrule* [lab=name] {} {\eta \bc x \;|\; \tau}
\end{mathpar} 

% subsection characteristic_formulae (end)   	 

\subsection{Example formulae}\label{sub:example_formulae_} % (fold)

\subsubsection{Crossing as formula.}
% 
% \begin{align*}
%   \frac{d}{dx} \sin x &= \cos x 
%   & \frac{d}{dx} e^x &= e^x \\
%   \frac{d}{dx} \cos x &= - \sin x 
%   & \frac{d}{dx} \log x &= \frac{1}{x} \\
% \end{align*} 

\begin{align*}
 \mu C(x_{0},x_{1},y_{0},y_{1},u).&(\langle x_{0}?(z) \rangle(\langle u! \rangle\langle y_{1}!z \rangle C(x_{0},x_{1},y_{0},y_{1},u)) & \\
  & \wedge \langle y_{1}?(z) \rangle (\langle u! \rangle \langle x_{0}!z \rangle C(x_{0},x_{1},y_{0},y_{1},u)) & \\
  & \wedge \langle x_{1}?(z) \rangle (\langle u? \rangle \langle y_{0}!z \rangle C(x_{0},x_{1},y_{0},y_{1},u)) & \\
  & \wedge \langle y_{0}?(z) \rangle (\langle u? \rangle \langle x_{1}!z \rangle C(x_{0},x_{1},y_{0},y_{1},u))) &
\end{align*}

The lexicographical similarity between the shape of this formulae and
the shape of definition of the process representing a crossing reveals
the intuitive meaning of this formulae. It describes the capabilities
of a process that has the right to represent a crossing. For example
it picks out processes that may perform an input on the port $x_0$ in
its initial menu of capabilities. What differentiates the formula
from the process, however, is that the crossing process is the
smallest candidate to satisfy the formula. Infinitely many other
processes -- with internal behavior hidden behind this interface, so
to speak -- also satisfy this formula. Even this simple formula,
then, can be seen to open a new view onto knots, providing a
computational interpretation of \emph{virtual} knots.

Note that this formula is derived by hand. A similar formula can be
derived by employing Caires' calculation of characteristic formula
\cite{Caires04} to the process representing a crossing. In light of
this discussion, we let
$\meaningof{C}_{\phi}(x0,x1,y0,y1,u)$ denote a formula specifying the
dynamics we wish to capture of a crossing. To guarantee we preserve
the shape of the interface and minimal semantics we demand that
$\meaningof{C}_{\phi}(x0,x1,y0,y1,u) \Rightarrow
\textbf{C}(x0,x1,y0,y1,u)$ where $\textbf{C}(x0,x1,y0,y1,u)$ denotes
the formula above.
                            
\subsubsection{Crossing number constraints.}
The moral content of the context lemma (Lemma \ref{context}) is that the notion of
``locality'' in the Reidemeister moves is effectively captured by the
parallel composition operator of the process calculus. This intuition
extends through the logic. Given a formula,
$\meaningof{C}_{\phi}(x0,x1,y0,y1,u)$, we can use the structural
connectives to specify constraints on crossing numbers, such as at
least $n$ crossings, or exactly $n$ crossings.
\begin{mathpar}
  \inferrule* [lab=at-least-n] {} { K^{\geq n}_{\phi}(\vec{xs},\vec{ys}) := \Pi_{i=0}^{n-1} Hu . \meaningof{C}_{\phi}(xs_i,ys_i,u) | T }
  \and 
  \inferrule* [lab=exactly-n] {} { K^{= n}_{\phi}(\vec{xs},\vec{ys}) := \Pi_{i=0}^{n-1} Hu . \meaningof{C}_{\phi}(xs_i,ys_i,u) | \neg (\forall x_0,y_0,x_1,y_1,u . \meaningof{C}_{\phi}(x_0,y_0,x_1,y_1,u) | T) }
\end{mathpar}

To round out this section, recall that the encoding of an $n$-crossing
knot decomposes into a parallel composition of $n$ \emph{copies} of a
crossing process together with a wiring harness. To specify different
knot classes with the same crossing number amounts to specifying
logical constraints on the wiring harness. In the interest of space,
we defer examples to a forthcoming paper. Suffice it to say that both
the conditions ``alternating knot'' and ``contains the tangle
corresponding to 5/3'' are expressible. For example, it is possible to
calculate the characteristic formula of a process corresponding to the
tangle 5/3 and conjoin it into the classifying formula via the
composition connective of the logic.

Finally, we wish to observe that it is entirely within reason to
contemplate a more domain-specific version of spatial logic tailored
to the shape of processes in the image of the encoding. Such a
domain-specific logic would have a better claim to the title formal
language of knot properties.

% subsection example_formulae_ (end)

% section knots_as_processes (end) 

% section spatial logic via knots (end)

\section{Conclusions and future work}

\paragraph{Testing physical space}
You, gentle reader, may wonder why of all the theorems to be proved
given this set up we pick the one above. In some sense it's hardly
central to quantum mechanics. We see it as central in the sense that
it firmly establishes a notion of physical space arising from a notion
of the equivalence of behavior. Relating bisimulation to a metric is a
big step forward, but one is faced with interpreting the relationship
of that metric space to something more physical. Quantum mechanical
notions of ``physical'' space are still far from intuitive, but by
relating this idea of distance as testing to calculations that predict
physical circumstances we are making a not insignificant step forward
toward an understanding of the physical space we inhabit as
essentially dynamic.

\paragraph{Effectivity and simulation}
One of the observations we have yet to make is that the entire program
spelled out here is effective. We have built various interpreters for
the reflective calculus at work in this interpretation. In principle,
then, we can simulate quantum mechanics on a computer. The place where
the simulation may lose fidelity is the infinitely branching summation
for the annihilator.

In this connection i also want to point out that the evaluation style
calculation of the inner product puts the non-determinism of the
summation right at the heart of measurement. This suggests that
Milner's original reduction-based formulation of the dynamics of his
calculi in terms of sums was not just notationally suggestive of a
notion of measure-and-continue but captured some significant part of
the physics.

\paragraph{Quantum continuations}
In light of this last observation i want to point out that the
predominant account of quantum mechanics is missing a key aspect of a
truly compositional story of the physical situation. In a real lab,
when a measurement is made the observation can be made to feed into
another device that then makes another measurement conditioned on the
results of the first. This means that after the superposition was
collapsed the entire experimental set up remained in
superposition. While QM offers a means of writing this down it doesn't
quite line up well with the well-trodden formulation of computation
and continuation that we see so succinctly expressed in Milner's
calculi. This suggests that there might be advantages to this account
of dynamics waiting to be explored.

\paragraph{Quantum logic}
In this connection, we also note that by virtue of having the
Hennessy-Milner construction, we can pull the construction through the
interpretation of QM. This gives us a natural candidate for a quantum
logic that enjoys an extremely tight connection with it's domain of
interpretation, making the construction much less ad hoc (rather it is
the image of functor!).

\paragraph{Quantum probabiity}
i have questions about the basis of the interpretation of inner
product as probability amplitude. In particular, using which
axiomatization of probability theory does the notion of probability
amplitude earn the right to be so dubbed? In other words, where is the
proof that the operation for calculating a probability amplitude (and
then squaring) satisfies the axioms of what it means to calculate a
probability? Even if such a proof exists (i have yet to find it in the
literature), i wonder if it might not be possible to turn things on
their heads. Can we view the calculation of the probability amplitude
as an axiomatization of probability? If so, then the definition we
give for calculating probability amplitude may provide the basis for
an \emph{effective} theory of probability.

\paragraph{Quantum vs ``biological'' information}
Finally, i want to conclude with a more philosophical observation. At
a recent workshop in which QM was a predominant topic i noticed
something about quantum information. The speaker was giving a riveting
discussion of axiomatic QM and showing how properties of ``no
cloning'' and ``no deleting'' emerged as consequences of the
axiomatization. Theorems of this form are necessary to give us a sense
of confidence that our axioms characterize the physical theory. What
struck me, though, was that if quantum information is neither erasable
nor replicable it is markedly different from \emph{life}. Two of the
things we know about life is that

\begin{itemize}
  \item it ends;
  \item to gain some measure of persistence, to transcend it's
    finitude it is imminently copyable.
\end{itemize}

Both of these qualities are summarized succinctly in the aphorism: all
flesh is grass. For me these two kinds of ``information'' -- call them
quantum and biological -- are end points on a spectrum of strategies
for persistence. At one end, we have those curious entities that enjoy
uniqueness and permanence; at the other, we have those who in the face
of a certain end and an uncertain present make a go of passing
something on. To me one of the more remarkable aspects of the latter
strategy is that in the presence of noise (and certain features of
copying) we get a kind of dynamism, a chance for improvement against a
given persistent condition.

% subsection other_calculi_other_bisimulations_and_geometry_as_behavior (end)




% section conclusion (end)

%\documentclass[12pt]{llncs}
%\documentclass{jktr}

\usepackage[pdftex]{hyperref}                   
\usepackage {listings}
\usepackage {mathpartir}
\usepackage{bcprules}
%\usepackage{listings}
                       
\usepackage{graphicx} 
%\usepackage[margins=2.5cm,nohead,nofoot]{geometry}
%\usepackage{geometry}
\usepackage{amsfonts}
\usepackage{amstext}
\usepackage{latexsym}
\usepackage{amssymb}
\usepackage{color}


%\include{myPreamble}
\include{qm2pi.local} 

%\ifpdf
%\usepackage[pdftex]{graphicx}
%\else
%\usepackage{graphicx}
%\fi

 % \ifpdf
%  \usepackage{pdfsync}
%  \if


%\title{Brief Article}
%\author{David F. Snyder}
%\author{L.G. Meredith}

%\address{Dept. of Math., Texas State University--San Marcos, San Marcos, TX 78666}
       
\pagestyle{empty}


\begin{document}

\lstset{language=[Objective]Caml,frame=shadowbox}

\input{qm2pi.front}

% section front matter (end)

\input{qm2pi.intro} 
 
% section introduction (end)

% \input{qm2pi.knotations} 

% section notation (end)

\input{qm2pi.process.calculi} 

% section concurrent_process_calculi_and_spatial_logics_ (end)
    
%\input{qm2pi.knots2pi} 

%\input{qm2pi.trefoil} 

%\input{qm2pi.mainthm} 

% subsection basic_interpretation (end)

%\input{qm2pi.rho.presentation} 
\subsection{The syntax and semantics of the notation system}\label{sub:the_syntax_and_semantics_of_the_notation_system} % (fold)

We now summarize a technical presentation of the calculus that
embodies our theory of dynamics. The typical presentation of such a
calculus follows the style of giving generators and relations on
them. The grammar, below, describing term constructors, freely
generates the set of processes, $\Proc$. This set is then quotiented
by a relation known as structural congruence and it is over this set
that the notion of dynamics is expressed. This presentation is
essentially that of \cite{MeredithR05} with the addition of
polyadicity and summation. For readability we have relegated some of
the technical subtleties to an appendix.

\subsubsection{Process grammar}\label{subsub:process_grammar}

\begin{mathpar}
  \inferrule* [lab=synchronization] {} {{M} \bc \pzero \;|\; x?F \;|\; x!C }
  \and
  \inferrule* [lab=abstraction] {} {{F} \bc (x)P}
  \and
  \inferrule* [lab=concretion] {} {{C} \bc \langle Q \rangle}
  \and
  \inferrule* [lab=process] {} {{P,Q} \bc M \;| \;P|Q \;|\; @{x}}
  \and
  \inferrule* [lab=name] {} {{x} \bc \quotep{P}}
\end{mathpar} 

Note that $\vec{x}$ (resp. $\vec{P}$) denotes a vector of names
(resp. processes) of length $|\vec{x}|$ (resp. $|\vec{P}|$). We adopt
the following useful abbreviations.

\begin{mathpar}
   x?(\vec{y}).P := x.(\vec{y})P \and  x\clift{\vec{P}} := x.\clift{\vec{P}}
   \and x!(y) := \lift{x}{\dropn{y}}
   \and \Pi_{i=0}^{n-1}P_i := P_0 | \ldots | P_{n-1}
\end{mathpar}

\subsubsection{Structural congruence}

\paragraph{Free and bound names and alpha-equivalence.} At the
core of structural equivalence is alpha-equivalence which identifies
process that are the same up to a change of variable. Formally, we
recognize the distinction between free and bound names. The free names
of a process, $\freenames{P}$, may be calculated recursively as
follows:

\begin{mathpar}
\freenames{\pzero} := \emptyset
  \and \\
  \freenames{x?(y).P} := \{ x \} \cup (\freenames{P} \setminus \{ y \})
  \and 
  \freenames{x!\langle P \rangle} := \{ x \} \cup \{ P \} 
  \and \\
  \freenames{P|Q} := \freenames{P} \cup \freenames{Q}
  \and \\
  \freenames{@{x}} := \{ x \}
\end{mathpar}

$\pi$
$\quotep{\pi}$

$\freenames{-} : \pi \to \mathcal{P}(\quotep{\pi})$

\begin{eqnarray*}
  \freenames{\pzero} & := & \emptyset \\
  \freenames{x?(y).P} & := & \{ x \} \cup (\freenames{P} \setminus \{ y \}) \\
  \freenames{x!\langle P \rangle} & := & \{ x \} \cup \{ P \} \\
  \freenames{P|Q} & := & \freenames{P} \cup \freenames{Q} \\
  \freenames{\dropn{x}} & := & \{ x \}
\end{eqnarray*}

The bound names of a process, $\boundnames{P}$, are those names occurring in $P$
that are not free. For example, in $x?(y).0$, the name $x$ is free, while $y$ is bound.

\begin{mathpar}
  \inferrule* [lab=monoidal-laws] {} { P|Q \equiv Q|P \and P|0 \equiv P \and P|(Q|R) \equiv (P|Q)|R }
\end{mathpar}

\begin{mathpar}
  \inferrule* [lab=alpha-equivalence] {} { (x)P \equiv (y)P\{y/x\} \and y \not\in \freenames{P} }
\end{mathpar}

\begin{definition}
Then two processes, $P,Q$, are alpha-equivalent if $P = Q\{\vec{y}/\vec{x}\}$ for
some $\vec{x} \in \boundnames{Q},\vec{y} \in \boundnames{P}$, where $Q\{\vec{y}/\vec{x}\}$
denotes the capture-avoiding substitution of $\vec{y}$ for $\vec{x}$ in $Q$.
\end{definition}

\begin{definition}
  The {\em structural congruence} \cite{SangiorgiWalker} , $\equiv$,
  between processes is the least congruence containing
  alpha-equivalence, satisfying the abelian monoid laws
  (associativity, commutativity and $\pzero$ as identity) for parallel
  composition $|$ and for summation $+$.
\end{definition}

\subsection{Name equivalence}

We take name equivalence, written $\nameeq$, to be the smallest
equivalence relation generated by the following rules.

\begin{mathpar}
\inferrule*[lab=Quote-drop]
{ }
{ \quotep{@{x}} \nameeq x }

\inferrule*[lab=Struct-equiv]
{ P \scong Q }
{ \quotep{P} \nameeq \quotep{Q} }
\end{mathpar}

The astute reader will have noticed that the mutual recursion of names
and processes imposes a mutual recursion on alpha-equivalence and
structural equivalence via name-equivalence. Fortunately, all of this
works out pleasantly and we may calculate in the natural way, free of
concern. The reader interested in the details is referred to the
appendix \ref{appendix:rho_details}.

\subsection{Substitution}

We use $\Proc$ for the set of processes, $\QProc$ for the set of
names, and $\id{\{}\vec{y} / \vec{x} \id{\}}$ to denote partial maps,
$s : \QProc \rightarrow \QProc$. A map, $s$ lifts, uniquely, to a map
on process terms, $\widehat{s} : \Proc \rightarrow \Proc$ by the
following equations.

\begin{mathpar}
  (0) \psubstp{Q}{P} := 0 \\
  (R \juxtap S) \psubstp{Q}{P}
  :=    
  (R)\psubstp{Q}{P} \juxtap (S) \psubstp{Q}{P} \\
  (x?(y).R) \psubstp{Q}{P}    
  :=    
  (x)\substp{Q}{P} (z)\concat( (R \psubstn{z}{y}) \psubstp{Q}{P} ) \\
  (\lift{x}{R}) \psubstp{Q}{P}  
  :=
  \lift{(x)\substp{Q}{P}}{ R \psubstp{Q}{P} } \\
%   (\dropn{x})  \psubstp{Q}{P}       
%   := 
%   \left\{ 
%     \begin{array}{ccc} 
%       \dropn{\quotep{Q}} & & x \nameeq \quotep{P} \\
%       \dropn{x} & & otherwise \\
%     \end{array}
%   \right. 
  (\dropn{x})  \psubstp{Q}{P}       
  := 
  \left\{ 
    \begin{array}{ccc} 
      Q & & x \nameeq \quotep{P} \\
      \dropn{x} & & otherwise \\
    \end{array}
  \right.
\end{mathpar}
 

where

\begin{eqnarray}
  (x)\id{\{} \lpquote Q \rpquote / \lpquote P \rpquote \id{\}}            = 
  \left\{ 
    \begin{array}{ccc}
      \lpquote Q \rpquote & & x \nameeq \lpquote P \rpquote \\
      x & & otherwise \\
    \end{array}
  \right. \nonumber
\end{eqnarray}

and $z$ is chosen distinct from $\quotep{P}$, $\quotep{Q}$, the free
names in $Q$, and all the names in $R$. Our $\alpha$-equivalence will
be built in the standard way from this substitution.

\begin{remark}\label{rem:no_self_referential_names}
  One consequence of these definitions is that $\forall P. \quotep{P}
  \not\in \freenames{P}$.
\end{remark}

\subsection{ Dynamic quote: an example }

Anticipating something of what's to come, consider applying the
substitution, $\widehat{\id{\{}u / z \id{\}}}$, to the following pair
of processes, $\lift{w}{y!(z)}$ and $w[ \lpquote y!(z) \rpquote ]$.

\begin{eqnarray}
	\lift{w}{y!(z)}\widehat{\id{\{}u / z \id{\}}}
		& = &
		\lift{w}{y!(u)} \nonumber\\
	w[ \lpquote y!(z) \rpquote ] \widehat{ \id{\{}u / z \id{\}} }
		& = &
		w[ \lpquote y!(z) \rpquote ] \nonumber
\end{eqnarray}

Because the body of the process between quotes is impervious to
substitution, we get radically different answers. In fact, by
examining the first process in an input context,
e.g. $x?(z).\lift{w}{y!(z)}$, we see that the process under the lift
operator may be shaped by prefixed inputs binding a name inside it. In
this sense, the lift operator will be seen as a way to dynamically
construct processes before reifying them as names.

Finally equipped with these standard features we can present the
dynamics of the calculus.

\subsubsection{Operational semantics} 

Finally, we introduce the computational dynamics. What marks these
algebras as distinct from other more traditionally studied algebraic
structures, e.g. vector spaces or polynomial rings, is the manner in
which dynamics is captured. In traditional structures, dynamics is typically
expressed through morphisms between such structures, as in linear maps
between vector spaces or morphisms between rings. In algebras
associated with the semantics of computation, the dynamics is
expressed as part of the algebraic structure itself, through a
reduction reduction relation typically denoted by $\red$. Below, we
give a recursive presentation of this relation for the calculus used
in the encoding.

$\red \subseteq \pi \times \pi$
$\red : \pi \to \mathcal{P}(\pi)$

\begin{mathpar}
  \inferrule* [lab=Comm] { \textsf{match}( x_{src}, x_{trgt} ) } { x_{trgt}?(y)P \; | \; x_{src}!\langle {Q} \rangle \red P\{\quotep{Q}/y}\} }
  \and \\
  \inferrule* [lab=Par] {{P} \red {P}'} {{{P} | {Q}} \red {{P}' | {Q}}}
  \and
  \inferrule* [lab=Equiv]{{{P} \scong {P}'} \andalso {{P}' \red {Q}'} \andalso {{Q}' \scong {Q}}}{{P} \red {Q}}
\end{mathpar}

\begin{eqnarray*}
  match_{\equiv} (\quotep{P},\quotep{Q}) & := & P \equiv Q \\
  match_{\dagger}(\quotep{P},\quotep{Q}) & := & \forall R. P|Q \red^{*} R => R \red^{*} 0 \\
  match_{K}(\quotep{P},\quotep{Q}) & := & K \mbox{ for some context } K
\end{eqnarray*}

$u?(x)P | u!\langle Q \rangle \red P\{\quotep{Q}/x\}$

%We write $\wred$ for $\red^*$, and $P\red$ if $\exists Q $ such that $ P \red Q$.
We write $P\red$ if $\exists Q $ such that $ P \red Q$ and $P\not\red$, otherwise.

\section{Replication}

As mentioned before, it is known that replication (and hence
recursion) can be implemented in a higher-order process algebra
\cite{SangiorgiWalker}. As our first example of calculation with the
machinery thus far presented we give the construction explicitly in
the {\rhoc}.

\begin{eqnarray}
	D_{x} & := & \prefix{x}{y}{(\binpar{\outputp{x}{y}}{@{y}})} \nonumber\\
	\bangp_{x}{P} & := & \binpar{{x}!\langle{\binpar{D_{x}}{P}}\rangle}{D_{x}} \nonumber
\end{eqnarray}

\begin{eqnarray}
	\bangp_{x}{P} & & \nonumber\\
	=
	& {x}!\langle{(\prefix{x}{y}{(\outputp{x}{y} | @{y})) | P}}\rangle 
	      | \prefix{x}{y}{(\outputp{x}{y} | @{y})} & \nonumber\\
	\red
	& (\outputp{x}{y} | @{y})\substn{\quotep{(\prefix{x}{y}{(@{y} | \outputp{x}{y})) | P}}}{y} & \nonumber\\
	=
	& \outputp{x}{\quotep{(\prefix{x}{y}{(\outputp{x}{y} | @{y})) | P}}}
	  | {(\prefix{x}{y}{(\outputp{x}{y} | @{y})) | P}} & \nonumber\\
	\red
	& \ldots & \nonumber\\
	\red^*
	& P | P | \ldots & \nonumber
\end{eqnarray}

Of course, this encoding, as an implementation, runs away, unfolding
$\bangp{P}$ eagerly. A lazier and more implementable replication
operator, restricted to input-guarded processes, may be obtained as follows.

\begin{eqnarray}
\bangp{\prefix{u}{v}{P}} 
	:= 
	\binpar{\lift{x}{\prefix{u}{v}{(\binpar{D(x)}{P})}}}{D(x)} \nonumber
\end{eqnarray}

\begin{remark}
  Note that the lazier definition still does not deal with summation
  or mixed summation (i.e. sums over input and output). The reader is
  invited to construct definitions of replication that deal with these
  features. 

  Further, the definitions are parameterized in a name, $x$. Can you,
  gentle reader, make a definition that eliminates this parameter and
  guarantees no accidental interaction between the replication
  machinery and the process being replicated -- i.e. no accidental
  sharing of names used by the process to get its work done and the
  name(s) used by the replication to effect copying. This latter
  revision of the definition of replication is crucial to obtaining
  the expected identity $!!P \sim !P$.
\end{remark}

\begin{remark}\label{rem:paradoxical_combinator}
  The reader familiar with the lambda calculus will have noticed the
  similarity between $D$ and the paradoxical combinator.

  [Ed. note: the existence of this seems to suggest we have to be more
  restrictive on the set of processes and names we admit if we are to
  support no-cloning.]
\end{remark}

\subsubsection{Bisimulation}

The computational dynamics gives rise to another kind of equivalence,
the equivalence of computational behavior. As previously mentioned
this is typically captured \emph{via} some form of bisimulation.

% The notion we use in this paper is weak barbed bisimulation
% \cite{milner91polyadicpi}.

The notion we use in this paper is derived from weak barbed
bisimulation \cite{milner91polyadicpi}. 

\begin{definition}
An \emph{observation relation}, $\downarrow_{\mathcal N}$, over a set
of names, $\mathcal N$, is the smallest relation satisfying the rules
below.

\infrule[Out-barb]{y \in {\mathcal N}, \; x \nameeq y}
		  {\outputp{x}{v} \downarrow_{\mathcal N} x}
\infrule[Par-barb]{\mbox{$P\downarrow_{\mathcal N} x$ or $Q\downarrow_{\mathcal N} x$}}
		  {\binpar{P}{Q} \downarrow_{\mathcal N} x}

We write $P \Downarrow_{\mathcal N} x$ if there is $Q$ such that 
$P \wred Q$ and $Q \downarrow_{\mathcal N} x$.
\end{definition}

\begin{definition}
%\label{def.bbisim}
An  ${\mathcal N}$-\emph{barbed bisimulation} over a set of names, ${\mathcal N}$, is a symmetric binary relation 
${\mathcal S}_{\mathcal N}$ between agents such that $P\rel{S}_{\mathcal N}Q$ implies:
\begin{enumerate}
\item If $P \red P'$ then $Q \wred Q'$ and $P'\rel{S}_{\mathcal N} Q'$.
\item If $P\downarrow_{\mathcal N} x$, then $Q\Downarrow_{\mathcal N} x$.
\end{enumerate}
$P$ is ${\mathcal N}$-barbed bisimilar to $Q$, written
$P \wbbisim_{\mathcal N} Q$, if $P \rel{S}_{\mathcal N} Q$ for some ${\mathcal N}$-barbed bisimulation ${\mathcal S}_{\mathcal N}$.
\end{definition}

$\mathcal{R} \subseteq \pi \times \pi$

$P \mathcal{R} Q => \forall P'. P \red P' \Rightarrow \exists Q'. Q \red Q', P' \mathcal{R} Q'$

$P \vdash x \Rightarrow Q \vdash x$

\begin{mathpar}
  \inferrule*[lab=Out-barb]{x \nameeq y}{{y}!\langle{Q}\rangle \vdash x}
  \and
  \inferrule*[lab=Par-barb]{\mbox{$P\vdash x$ or $Q\vdash x$}}{\binpar{P}{Q} \vdash x}
\end{mathpar}

\subsubsection{Contexts}

One of the principle advantages of computational calculi like the
$\pi$-calculus is a well-defined notion of context,
contextual-equivalence and a correlation between
contextual-equivalence and notions of bisimulation. The notion of
context allows the decomposition of a process into (sub-)process and
its syntactic environment, its context. Thus, a context may be
thought of as a process with a ``hole'' (written $\Box$) in it. The
application of a context $M$ to a process $P$, written $M[P]$, is
tantamount to filling the hole in $M$ with $P$. In this paper we do
not need the full weight of this theory, but do make use of the notion
of context in the proof the main theorem. 

\begin{mathpar}
  \inferrule* [lab=summation] {} {{M_{M},M_{N}} \bc \Box \;|\; x.M_{A} \;|\; M_{M}+M_{N}}
  \and
  \inferrule* [lab=agent] {} {{M_{A}} \bc (\vec{x})M_{P} \;| \; \clift{P_0,\ldots,M_{P},\ldots,P_N}}
  \and \\
  \inferrule* [lab=process] {} {{M_{P}} \bc M_{N} \;| \;P|M_{P} }
\end{mathpar} 

\begin{mathpar}
  \inferrule* [lab=sychronization] {} {M_{N} \bc \Box \;|\; x?M_{F} \;|\; x!M_{C}}
  \and
  \inferrule* [lab=abstraction] {} {{M_{F}} \bc (x)M_{P} }
  \and
  \inferrule* [lab=concretion] {} {{M_{C}} \bc \langle M_{P} \rangle }
  \and \\
  \inferrule* [lab=process] {} {{M_{P}} \bc M_{N} \;| \;P|M_{P} }
\end{mathpar}

\begin{definition}[contextual application] Given a context $M$, and
  process $P$, we define the \emph{contextual application}, $M[P] :=
  M\{P/\Box\}$. That is, the contextual application of M to P is the
  substitution of $P$ for $\Box$ in $M$.
\end{definition}

$\meaningof{-} : L \to \mathcal{P}(\pi)$

\begin{mathpar}
  \inferrule* [lab=collection] {} {\meaningof{true} = \pi, \and \meaningof{~E} = \pi \setminus \meaningof{E}, \and \meaningof{E_{1} \& E_{2}} = \meaningof{E_{1}} \cap \meaningof{E_{2}}}
\end{mathpar}

\begin{mathpar}
  \inferrule* [lab=structure] {} {\meaningof{0} = \{ P \in \pi | P \equiv 0 \}, \and \\ \meaningof{E_1 | E_2} = \{ P \in \pi | P \equiv P_{1} | P_{2}, P_{1} \in \meaningof{E_{1}}, P_{2} \in \meaningof{E_2}\} }
\end{mathpar}

\begin{mathpar}
 \inferrule* [lab=behavior] {} {\meaningof{\langle a?b \rangle E} = \{ P \in \pi | P \equiv Q | u?(y)P', \\ \and \\\\ \and \\ \;\;\; u \in \meaningof{a}, \forall z.P'\{z/y\} \in \meaningof{E\{z/b\}}\}, \and \\ \meaningof{a!E} = \{ P \in \pi | P \equiv Q | x!\langle P' \rangle, x \in \meaningof{a} P' \in \meaningof{E}\} }
\end{mathpar}

\begin{mathpar}
 \inferrule* [lab=nominal] {} {\meaningof{\quotep{E}} = \{ \quotep{P} \in \quotep{\pi} | P \in \meaningof{E} \}, \and \meaningof{\quotep{P}} = \{ \quotep{Q} \in \quotep{\pi} | P \equiv Q \} \and \\ \meaningof{@\quotep{E}} = \{ P \in \pi | P \equiv @x, x \in \meaningof{E} \}}
\end{mathpar}

\begin{eqnarray*}
  \\
  \meaningof{-} : TS \to ST
\end{eqnarray*}

\begin{eqnarray*}
  \\
  L : TS \to ST
\end{eqnarray*}

\begin{eqnarray*}
  \\
  P \models E \iff P \in \meaningof{E}
\end{eqnarray*}

\begin{eqnarray*}
  P \approx_{L} Q \iff \forall E \in L. P \models E \iff Q \models E
\end{eqnarray*}

\begin{eqnarray*}
  P \approx_{K} Q
\end{eqnarray*}

\begin{eqnarray*}
  P \approx Q
\end{eqnarray*}

$\approx_{K} = \approx = \approx_{L}$

\subsubsection{Contextual duality}

Note that contexts extend the quotation operation to a family of
operations from processes to names. Given a context, $M$, we can
define a \emph{nominal context}, $\quotep{M}$ by $\quotep{M}[P] :=
\quotep{M[P]}$. To foreshadow what is to come we observe that these
operations enjoy a duality with processes very much like the duality
between vectors and maps from vectors to scalars.

Further, because the calculus is essentially higher-order, we have a
correspondence between contexts and processes. More specifically,
given a name $x$ and a context $M$ we can construct $M^{*}_{x}$ such
that 

\begin{mathpar}
  M^{*}_{x} | \lift{x}{P} \red M[P]
\end{mathpar}

namely,

\begin{mathpar}
  M^{*}_{x} := x?(u).M[\dropn{u}]
\end{mathpar}

The dependence of $M^{*}_{x}$ on a name makes it an abstraction, 

\begin{mathpar}
  M^{*} := (x)x?(u).M[\dropn{u}]
\end{mathpar}

\subsection{Additional notation}

It will sometimes be convenient to denote the process a name
quotes. We already have the notation $x = \quotep{P}$, but it will be
convenient to introduce an alternate notation, $\procn{x}$, when we
want to emphasize the connection to the use of the name. Note that, by
virtue of name equivalence, $\quotep{\procn{x}} \nameeq x$; so, the
notation is consistent with previous definitions.

Further, because names have structure it is possible to effect
substitutions on the basis of that structure. This means we need to
upgrade our notation for substitutions, which we accomplish by
adapting comprehension notation. Thus,

\begin{mathpar}
  P\{ y / x : x \in S \}
\end{mathpar}

is interpreted to mean the process derived from P by replacing (in a
capture-avoiding manner) each occurrence of $x$ in $S$ by $y$. For example,

\begin{mathpar}
  P\{ \quotep{\procn{x}|\procn{x}} / x : x \in \freenames{P} \}
\end{mathpar}

will replace each (occurrence) of a free name $x$ in $P$ by
$\quotep{\procn{x}|\procn{x}}$.

Also, we will avail ourselves of the notation $x^{L}$ and $x^{R}$ to
denote injections of a name into disjoint copies of the name
space. There are numerous ways to accomplish this. One example can be
found in \cite{MeredithR05}. This notation overloads to vectors of
names: $\vec{x}^{\pi} := (x_{i}^{\pi} \; : \; 0 \leq i < |\vec{x}| )$ where $\pi \in \{L,R\}$.

We also use $P^{\Box} := P|\Box$.

In \cite{MeredithR05} an interpretation of the new operator is
given. It turns out that there are several possible interpretations
all enjoying the requisite algebraic properties of the operator (see
\cite{milner91polyadicpi}). We will therefore make liberal use of
$(\nu\; \vec{x})P$.

% subsection the_syntax_and_semantics_of_the_notation_system (end)   

\input{qm2pi.qmops} 

\input{qm2pi.sterngerlach} 

\input{qm2pi.metric} 

% section concurrent_process_calculi (end)

%\input{qm2pi.proofsketch}

% section proof sketch (end)

%\input{qm2pi.slviaknots} 

% section spatial logic via knots (end)

\input{qm2pi.conclusion}

% section conclusion (end)

%\input{qm2pi.dtcodes} 

% section wiring algorithm (end)

\input{qm2pi.ack} 

% section acknowledgments (end)

\newpage


\bibliographystyle{plain}   
\bibliography{../../biblios/main.bib}

\input{qm2pi.rhodetails}

\end{document}

 

% section wiring algorithm (end)

\documentclass[12pt]{llncs}
%\documentclass{jktr}

\usepackage[pdftex]{hyperref}                   
\usepackage {listings}
\usepackage {mathpartir}
\usepackage{bcprules}
%\usepackage{listings}
                       
\usepackage{graphicx} 
%\usepackage[margins=2.5cm,nohead,nofoot]{geometry}
%\usepackage{geometry}
\usepackage{amsfonts}
\usepackage{amstext}
\usepackage{latexsym}
\usepackage{amssymb}
\usepackage{color}


%\include{myPreamble}
\include{qm2pi.local} 

%\ifpdf
%\usepackage[pdftex]{graphicx}
%\else
%\usepackage{graphicx}
%\fi

 % \ifpdf
%  \usepackage{pdfsync}
%  \if


%\title{Brief Article}
%\author{David F. Snyder}
%\author{L.G. Meredith}

%\address{Dept. of Math., Texas State University--San Marcos, San Marcos, TX 78666}
       
\pagestyle{empty}


\begin{document}

\lstset{language=[Objective]Caml,frame=shadowbox}

\input{qm2pi.front}

% section front matter (end)

\input{qm2pi.intro} 
 
% section introduction (end)

% \input{qm2pi.knotations} 

% section notation (end)

\input{qm2pi.process.calculi} 

% section concurrent_process_calculi_and_spatial_logics_ (end)
    
%\input{qm2pi.knots2pi} 

%\input{qm2pi.trefoil} 

%\input{qm2pi.mainthm} 

% subsection basic_interpretation (end)

%\input{qm2pi.rho.presentation} 
\subsection{The syntax and semantics of the notation system}\label{sub:the_syntax_and_semantics_of_the_notation_system} % (fold)

We now summarize a technical presentation of the calculus that
embodies our theory of dynamics. The typical presentation of such a
calculus follows the style of giving generators and relations on
them. The grammar, below, describing term constructors, freely
generates the set of processes, $\Proc$. This set is then quotiented
by a relation known as structural congruence and it is over this set
that the notion of dynamics is expressed. This presentation is
essentially that of \cite{MeredithR05} with the addition of
polyadicity and summation. For readability we have relegated some of
the technical subtleties to an appendix.

\subsubsection{Process grammar}\label{subsub:process_grammar}

\begin{mathpar}
  \inferrule* [lab=synchronization] {} {{M} \bc \pzero \;|\; x?F \;|\; x!C }
  \and
  \inferrule* [lab=abstraction] {} {{F} \bc (x)P}
  \and
  \inferrule* [lab=concretion] {} {{C} \bc \langle Q \rangle}
  \and
  \inferrule* [lab=process] {} {{P,Q} \bc M \;| \;P|Q \;|\; @{x}}
  \and
  \inferrule* [lab=name] {} {{x} \bc \quotep{P}}
\end{mathpar} 

Note that $\vec{x}$ (resp. $\vec{P}$) denotes a vector of names
(resp. processes) of length $|\vec{x}|$ (resp. $|\vec{P}|$). We adopt
the following useful abbreviations.

\begin{mathpar}
   x?(\vec{y}).P := x.(\vec{y})P \and  x\clift{\vec{P}} := x.\clift{\vec{P}}
   \and x!(y) := \lift{x}{\dropn{y}}
   \and \Pi_{i=0}^{n-1}P_i := P_0 | \ldots | P_{n-1}
\end{mathpar}

\subsubsection{Structural congruence}

\paragraph{Free and bound names and alpha-equivalence.} At the
core of structural equivalence is alpha-equivalence which identifies
process that are the same up to a change of variable. Formally, we
recognize the distinction between free and bound names. The free names
of a process, $\freenames{P}$, may be calculated recursively as
follows:

\begin{mathpar}
\freenames{\pzero} := \emptyset
  \and \\
  \freenames{x?(y).P} := \{ x \} \cup (\freenames{P} \setminus \{ y \})
  \and 
  \freenames{x!\langle P \rangle} := \{ x \} \cup \{ P \} 
  \and \\
  \freenames{P|Q} := \freenames{P} \cup \freenames{Q}
  \and \\
  \freenames{@{x}} := \{ x \}
\end{mathpar}

$\pi$
$\quotep{\pi}$

$\freenames{-} : \pi \to \mathcal{P}(\quotep{\pi})$

\begin{eqnarray*}
  \freenames{\pzero} & := & \emptyset \\
  \freenames{x?(y).P} & := & \{ x \} \cup (\freenames{P} \setminus \{ y \}) \\
  \freenames{x!\langle P \rangle} & := & \{ x \} \cup \{ P \} \\
  \freenames{P|Q} & := & \freenames{P} \cup \freenames{Q} \\
  \freenames{\dropn{x}} & := & \{ x \}
\end{eqnarray*}

The bound names of a process, $\boundnames{P}$, are those names occurring in $P$
that are not free. For example, in $x?(y).0$, the name $x$ is free, while $y$ is bound.

\begin{mathpar}
  \inferrule* [lab=monoidal-laws] {} { P|Q \equiv Q|P \and P|0 \equiv P \and P|(Q|R) \equiv (P|Q)|R }
\end{mathpar}

\begin{mathpar}
  \inferrule* [lab=alpha-equivalence] {} { (x)P \equiv (y)P\{y/x\} \and y \not\in \freenames{P} }
\end{mathpar}

\begin{definition}
Then two processes, $P,Q$, are alpha-equivalent if $P = Q\{\vec{y}/\vec{x}\}$ for
some $\vec{x} \in \boundnames{Q},\vec{y} \in \boundnames{P}$, where $Q\{\vec{y}/\vec{x}\}$
denotes the capture-avoiding substitution of $\vec{y}$ for $\vec{x}$ in $Q$.
\end{definition}

\begin{definition}
  The {\em structural congruence} \cite{SangiorgiWalker} , $\equiv$,
  between processes is the least congruence containing
  alpha-equivalence, satisfying the abelian monoid laws
  (associativity, commutativity and $\pzero$ as identity) for parallel
  composition $|$ and for summation $+$.
\end{definition}

\subsection{Name equivalence}

We take name equivalence, written $\nameeq$, to be the smallest
equivalence relation generated by the following rules.

\begin{mathpar}
\inferrule*[lab=Quote-drop]
{ }
{ \quotep{@{x}} \nameeq x }

\inferrule*[lab=Struct-equiv]
{ P \scong Q }
{ \quotep{P} \nameeq \quotep{Q} }
\end{mathpar}

The astute reader will have noticed that the mutual recursion of names
and processes imposes a mutual recursion on alpha-equivalence and
structural equivalence via name-equivalence. Fortunately, all of this
works out pleasantly and we may calculate in the natural way, free of
concern. The reader interested in the details is referred to the
appendix \ref{appendix:rho_details}.

\subsection{Substitution}

We use $\Proc$ for the set of processes, $\QProc$ for the set of
names, and $\id{\{}\vec{y} / \vec{x} \id{\}}$ to denote partial maps,
$s : \QProc \rightarrow \QProc$. A map, $s$ lifts, uniquely, to a map
on process terms, $\widehat{s} : \Proc \rightarrow \Proc$ by the
following equations.

\begin{mathpar}
  (0) \psubstp{Q}{P} := 0 \\
  (R \juxtap S) \psubstp{Q}{P}
  :=    
  (R)\psubstp{Q}{P} \juxtap (S) \psubstp{Q}{P} \\
  (x?(y).R) \psubstp{Q}{P}    
  :=    
  (x)\substp{Q}{P} (z)\concat( (R \psubstn{z}{y}) \psubstp{Q}{P} ) \\
  (\lift{x}{R}) \psubstp{Q}{P}  
  :=
  \lift{(x)\substp{Q}{P}}{ R \psubstp{Q}{P} } \\
%   (\dropn{x})  \psubstp{Q}{P}       
%   := 
%   \left\{ 
%     \begin{array}{ccc} 
%       \dropn{\quotep{Q}} & & x \nameeq \quotep{P} \\
%       \dropn{x} & & otherwise \\
%     \end{array}
%   \right. 
  (\dropn{x})  \psubstp{Q}{P}       
  := 
  \left\{ 
    \begin{array}{ccc} 
      Q & & x \nameeq \quotep{P} \\
      \dropn{x} & & otherwise \\
    \end{array}
  \right.
\end{mathpar}
 

where

\begin{eqnarray}
  (x)\id{\{} \lpquote Q \rpquote / \lpquote P \rpquote \id{\}}            = 
  \left\{ 
    \begin{array}{ccc}
      \lpquote Q \rpquote & & x \nameeq \lpquote P \rpquote \\
      x & & otherwise \\
    \end{array}
  \right. \nonumber
\end{eqnarray}

and $z$ is chosen distinct from $\quotep{P}$, $\quotep{Q}$, the free
names in $Q$, and all the names in $R$. Our $\alpha$-equivalence will
be built in the standard way from this substitution.

\begin{remark}\label{rem:no_self_referential_names}
  One consequence of these definitions is that $\forall P. \quotep{P}
  \not\in \freenames{P}$.
\end{remark}

\subsection{ Dynamic quote: an example }

Anticipating something of what's to come, consider applying the
substitution, $\widehat{\id{\{}u / z \id{\}}}$, to the following pair
of processes, $\lift{w}{y!(z)}$ and $w[ \lpquote y!(z) \rpquote ]$.

\begin{eqnarray}
	\lift{w}{y!(z)}\widehat{\id{\{}u / z \id{\}}}
		& = &
		\lift{w}{y!(u)} \nonumber\\
	w[ \lpquote y!(z) \rpquote ] \widehat{ \id{\{}u / z \id{\}} }
		& = &
		w[ \lpquote y!(z) \rpquote ] \nonumber
\end{eqnarray}

Because the body of the process between quotes is impervious to
substitution, we get radically different answers. In fact, by
examining the first process in an input context,
e.g. $x?(z).\lift{w}{y!(z)}$, we see that the process under the lift
operator may be shaped by prefixed inputs binding a name inside it. In
this sense, the lift operator will be seen as a way to dynamically
construct processes before reifying them as names.

Finally equipped with these standard features we can present the
dynamics of the calculus.

\subsubsection{Operational semantics} 

Finally, we introduce the computational dynamics. What marks these
algebras as distinct from other more traditionally studied algebraic
structures, e.g. vector spaces or polynomial rings, is the manner in
which dynamics is captured. In traditional structures, dynamics is typically
expressed through morphisms between such structures, as in linear maps
between vector spaces or morphisms between rings. In algebras
associated with the semantics of computation, the dynamics is
expressed as part of the algebraic structure itself, through a
reduction reduction relation typically denoted by $\red$. Below, we
give a recursive presentation of this relation for the calculus used
in the encoding.

$\red \subseteq \pi \times \pi$
$\red : \pi \to \mathcal{P}(\pi)$

\begin{mathpar}
  \inferrule* [lab=Comm] { \textsf{match}( x_{src}, x_{trgt} ) } { x_{trgt}?(y)P \; | \; x_{src}!\langle {Q} \rangle \red P\{\quotep{Q}/y}\} }
  \and \\
  \inferrule* [lab=Par] {{P} \red {P}'} {{{P} | {Q}} \red {{P}' | {Q}}}
  \and
  \inferrule* [lab=Equiv]{{{P} \scong {P}'} \andalso {{P}' \red {Q}'} \andalso {{Q}' \scong {Q}}}{{P} \red {Q}}
\end{mathpar}

\begin{eqnarray*}
  match_{\equiv} (\quotep{P},\quotep{Q}) & := & P \equiv Q \\
  match_{\dagger}(\quotep{P},\quotep{Q}) & := & \forall R. P|Q \red^{*} R => R \red^{*} 0 \\
  match_{K}(\quotep{P},\quotep{Q}) & := & K \mbox{ for some context } K
\end{eqnarray*}

$u?(x)P | u!\langle Q \rangle \red P\{\quotep{Q}/x\}$

%We write $\wred$ for $\red^*$, and $P\red$ if $\exists Q $ such that $ P \red Q$.
We write $P\red$ if $\exists Q $ such that $ P \red Q$ and $P\not\red$, otherwise.

\section{Replication}

As mentioned before, it is known that replication (and hence
recursion) can be implemented in a higher-order process algebra
\cite{SangiorgiWalker}. As our first example of calculation with the
machinery thus far presented we give the construction explicitly in
the {\rhoc}.

\begin{eqnarray}
	D_{x} & := & \prefix{x}{y}{(\binpar{\outputp{x}{y}}{@{y}})} \nonumber\\
	\bangp_{x}{P} & := & \binpar{{x}!\langle{\binpar{D_{x}}{P}}\rangle}{D_{x}} \nonumber
\end{eqnarray}

\begin{eqnarray}
	\bangp_{x}{P} & & \nonumber\\
	=
	& {x}!\langle{(\prefix{x}{y}{(\outputp{x}{y} | @{y})) | P}}\rangle 
	      | \prefix{x}{y}{(\outputp{x}{y} | @{y})} & \nonumber\\
	\red
	& (\outputp{x}{y} | @{y})\substn{\quotep{(\prefix{x}{y}{(@{y} | \outputp{x}{y})) | P}}}{y} & \nonumber\\
	=
	& \outputp{x}{\quotep{(\prefix{x}{y}{(\outputp{x}{y} | @{y})) | P}}}
	  | {(\prefix{x}{y}{(\outputp{x}{y} | @{y})) | P}} & \nonumber\\
	\red
	& \ldots & \nonumber\\
	\red^*
	& P | P | \ldots & \nonumber
\end{eqnarray}

Of course, this encoding, as an implementation, runs away, unfolding
$\bangp{P}$ eagerly. A lazier and more implementable replication
operator, restricted to input-guarded processes, may be obtained as follows.

\begin{eqnarray}
\bangp{\prefix{u}{v}{P}} 
	:= 
	\binpar{\lift{x}{\prefix{u}{v}{(\binpar{D(x)}{P})}}}{D(x)} \nonumber
\end{eqnarray}

\begin{remark}
  Note that the lazier definition still does not deal with summation
  or mixed summation (i.e. sums over input and output). The reader is
  invited to construct definitions of replication that deal with these
  features. 

  Further, the definitions are parameterized in a name, $x$. Can you,
  gentle reader, make a definition that eliminates this parameter and
  guarantees no accidental interaction between the replication
  machinery and the process being replicated -- i.e. no accidental
  sharing of names used by the process to get its work done and the
  name(s) used by the replication to effect copying. This latter
  revision of the definition of replication is crucial to obtaining
  the expected identity $!!P \sim !P$.
\end{remark}

\begin{remark}\label{rem:paradoxical_combinator}
  The reader familiar with the lambda calculus will have noticed the
  similarity between $D$ and the paradoxical combinator.

  [Ed. note: the existence of this seems to suggest we have to be more
  restrictive on the set of processes and names we admit if we are to
  support no-cloning.]
\end{remark}

\subsubsection{Bisimulation}

The computational dynamics gives rise to another kind of equivalence,
the equivalence of computational behavior. As previously mentioned
this is typically captured \emph{via} some form of bisimulation.

% The notion we use in this paper is weak barbed bisimulation
% \cite{milner91polyadicpi}.

The notion we use in this paper is derived from weak barbed
bisimulation \cite{milner91polyadicpi}. 

\begin{definition}
An \emph{observation relation}, $\downarrow_{\mathcal N}$, over a set
of names, $\mathcal N$, is the smallest relation satisfying the rules
below.

\infrule[Out-barb]{y \in {\mathcal N}, \; x \nameeq y}
		  {\outputp{x}{v} \downarrow_{\mathcal N} x}
\infrule[Par-barb]{\mbox{$P\downarrow_{\mathcal N} x$ or $Q\downarrow_{\mathcal N} x$}}
		  {\binpar{P}{Q} \downarrow_{\mathcal N} x}

We write $P \Downarrow_{\mathcal N} x$ if there is $Q$ such that 
$P \wred Q$ and $Q \downarrow_{\mathcal N} x$.
\end{definition}

\begin{definition}
%\label{def.bbisim}
An  ${\mathcal N}$-\emph{barbed bisimulation} over a set of names, ${\mathcal N}$, is a symmetric binary relation 
${\mathcal S}_{\mathcal N}$ between agents such that $P\rel{S}_{\mathcal N}Q$ implies:
\begin{enumerate}
\item If $P \red P'$ then $Q \wred Q'$ and $P'\rel{S}_{\mathcal N} Q'$.
\item If $P\downarrow_{\mathcal N} x$, then $Q\Downarrow_{\mathcal N} x$.
\end{enumerate}
$P$ is ${\mathcal N}$-barbed bisimilar to $Q$, written
$P \wbbisim_{\mathcal N} Q$, if $P \rel{S}_{\mathcal N} Q$ for some ${\mathcal N}$-barbed bisimulation ${\mathcal S}_{\mathcal N}$.
\end{definition}

$\mathcal{R} \subseteq \pi \times \pi$

$P \mathcal{R} Q => \forall P'. P \red P' \Rightarrow \exists Q'. Q \red Q', P' \mathcal{R} Q'$

$P \vdash x \Rightarrow Q \vdash x$

\begin{mathpar}
  \inferrule*[lab=Out-barb]{x \nameeq y}{{y}!\langle{Q}\rangle \vdash x}
  \and
  \inferrule*[lab=Par-barb]{\mbox{$P\vdash x$ or $Q\vdash x$}}{\binpar{P}{Q} \vdash x}
\end{mathpar}

\subsubsection{Contexts}

One of the principle advantages of computational calculi like the
$\pi$-calculus is a well-defined notion of context,
contextual-equivalence and a correlation between
contextual-equivalence and notions of bisimulation. The notion of
context allows the decomposition of a process into (sub-)process and
its syntactic environment, its context. Thus, a context may be
thought of as a process with a ``hole'' (written $\Box$) in it. The
application of a context $M$ to a process $P$, written $M[P]$, is
tantamount to filling the hole in $M$ with $P$. In this paper we do
not need the full weight of this theory, but do make use of the notion
of context in the proof the main theorem. 

\begin{mathpar}
  \inferrule* [lab=summation] {} {{M_{M},M_{N}} \bc \Box \;|\; x.M_{A} \;|\; M_{M}+M_{N}}
  \and
  \inferrule* [lab=agent] {} {{M_{A}} \bc (\vec{x})M_{P} \;| \; \clift{P_0,\ldots,M_{P},\ldots,P_N}}
  \and \\
  \inferrule* [lab=process] {} {{M_{P}} \bc M_{N} \;| \;P|M_{P} }
\end{mathpar} 

\begin{mathpar}
  \inferrule* [lab=sychronization] {} {M_{N} \bc \Box \;|\; x?M_{F} \;|\; x!M_{C}}
  \and
  \inferrule* [lab=abstraction] {} {{M_{F}} \bc (x)M_{P} }
  \and
  \inferrule* [lab=concretion] {} {{M_{C}} \bc \langle M_{P} \rangle }
  \and \\
  \inferrule* [lab=process] {} {{M_{P}} \bc M_{N} \;| \;P|M_{P} }
\end{mathpar}

\begin{definition}[contextual application] Given a context $M$, and
  process $P$, we define the \emph{contextual application}, $M[P] :=
  M\{P/\Box\}$. That is, the contextual application of M to P is the
  substitution of $P$ for $\Box$ in $M$.
\end{definition}

$\meaningof{-} : L \to \mathcal{P}(\pi)$

\begin{mathpar}
  \inferrule* [lab=collection] {} {\meaningof{true} = \pi, \and \meaningof{~E} = \pi \setminus \meaningof{E}, \and \meaningof{E_{1} \& E_{2}} = \meaningof{E_{1}} \cap \meaningof{E_{2}}}
\end{mathpar}

\begin{mathpar}
  \inferrule* [lab=structure] {} {\meaningof{0} = \{ P \in \pi | P \equiv 0 \}, \and \\ \meaningof{E_1 | E_2} = \{ P \in \pi | P \equiv P_{1} | P_{2}, P_{1} \in \meaningof{E_{1}}, P_{2} \in \meaningof{E_2}\} }
\end{mathpar}

\begin{mathpar}
 \inferrule* [lab=behavior] {} {\meaningof{\langle a?b \rangle E} = \{ P \in \pi | P \equiv Q | u?(y)P', \\ \and \\\\ \and \\ \;\;\; u \in \meaningof{a}, \forall z.P'\{z/y\} \in \meaningof{E\{z/b\}}\}, \and \\ \meaningof{a!E} = \{ P \in \pi | P \equiv Q | x!\langle P' \rangle, x \in \meaningof{a} P' \in \meaningof{E}\} }
\end{mathpar}

\begin{mathpar}
 \inferrule* [lab=nominal] {} {\meaningof{\quotep{E}} = \{ \quotep{P} \in \quotep{\pi} | P \in \meaningof{E} \}, \and \meaningof{\quotep{P}} = \{ \quotep{Q} \in \quotep{\pi} | P \equiv Q \} \and \\ \meaningof{@\quotep{E}} = \{ P \in \pi | P \equiv @x, x \in \meaningof{E} \}}
\end{mathpar}

\begin{eqnarray*}
  \\
  \meaningof{-} : TS \to ST
\end{eqnarray*}

\begin{eqnarray*}
  \\
  L : TS \to ST
\end{eqnarray*}

\begin{eqnarray*}
  \\
  P \models E \iff P \in \meaningof{E}
\end{eqnarray*}

\begin{eqnarray*}
  P \approx_{L} Q \iff \forall E \in L. P \models E \iff Q \models E
\end{eqnarray*}

\begin{eqnarray*}
  P \approx_{K} Q
\end{eqnarray*}

\begin{eqnarray*}
  P \approx Q
\end{eqnarray*}

$\approx_{K} = \approx = \approx_{L}$

\subsubsection{Contextual duality}

Note that contexts extend the quotation operation to a family of
operations from processes to names. Given a context, $M$, we can
define a \emph{nominal context}, $\quotep{M}$ by $\quotep{M}[P] :=
\quotep{M[P]}$. To foreshadow what is to come we observe that these
operations enjoy a duality with processes very much like the duality
between vectors and maps from vectors to scalars.

Further, because the calculus is essentially higher-order, we have a
correspondence between contexts and processes. More specifically,
given a name $x$ and a context $M$ we can construct $M^{*}_{x}$ such
that 

\begin{mathpar}
  M^{*}_{x} | \lift{x}{P} \red M[P]
\end{mathpar}

namely,

\begin{mathpar}
  M^{*}_{x} := x?(u).M[\dropn{u}]
\end{mathpar}

The dependence of $M^{*}_{x}$ on a name makes it an abstraction, 

\begin{mathpar}
  M^{*} := (x)x?(u).M[\dropn{u}]
\end{mathpar}

\subsection{Additional notation}

It will sometimes be convenient to denote the process a name
quotes. We already have the notation $x = \quotep{P}$, but it will be
convenient to introduce an alternate notation, $\procn{x}$, when we
want to emphasize the connection to the use of the name. Note that, by
virtue of name equivalence, $\quotep{\procn{x}} \nameeq x$; so, the
notation is consistent with previous definitions.

Further, because names have structure it is possible to effect
substitutions on the basis of that structure. This means we need to
upgrade our notation for substitutions, which we accomplish by
adapting comprehension notation. Thus,

\begin{mathpar}
  P\{ y / x : x \in S \}
\end{mathpar}

is interpreted to mean the process derived from P by replacing (in a
capture-avoiding manner) each occurrence of $x$ in $S$ by $y$. For example,

\begin{mathpar}
  P\{ \quotep{\procn{x}|\procn{x}} / x : x \in \freenames{P} \}
\end{mathpar}

will replace each (occurrence) of a free name $x$ in $P$ by
$\quotep{\procn{x}|\procn{x}}$.

Also, we will avail ourselves of the notation $x^{L}$ and $x^{R}$ to
denote injections of a name into disjoint copies of the name
space. There are numerous ways to accomplish this. One example can be
found in \cite{MeredithR05}. This notation overloads to vectors of
names: $\vec{x}^{\pi} := (x_{i}^{\pi} \; : \; 0 \leq i < |\vec{x}| )$ where $\pi \in \{L,R\}$.

We also use $P^{\Box} := P|\Box$.

In \cite{MeredithR05} an interpretation of the new operator is
given. It turns out that there are several possible interpretations
all enjoying the requisite algebraic properties of the operator (see
\cite{milner91polyadicpi}). We will therefore make liberal use of
$(\nu\; \vec{x})P$.

% subsection the_syntax_and_semantics_of_the_notation_system (end)   

\input{qm2pi.qmops} 

\input{qm2pi.sterngerlach} 

\input{qm2pi.metric} 

% section concurrent_process_calculi (end)

%\input{qm2pi.proofsketch}

% section proof sketch (end)

%\input{qm2pi.slviaknots} 

% section spatial logic via knots (end)

\input{qm2pi.conclusion}

% section conclusion (end)

%\input{qm2pi.dtcodes} 

% section wiring algorithm (end)

\input{qm2pi.ack} 

% section acknowledgments (end)

\newpage


\bibliographystyle{plain}   
\bibliography{../../biblios/main.bib}

\input{qm2pi.rhodetails}

\end{document}

 

% section acknowledgments (end)

\newpage


\bibliographystyle{plain}   
\bibliography{../../biblios/main.bib}

\documentclass[12pt]{llncs}
%\documentclass{jktr}

\usepackage[pdftex]{hyperref}                   
\usepackage {listings}
\usepackage {mathpartir}
\usepackage{bcprules}
%\usepackage{listings}
                       
\usepackage{graphicx} 
%\usepackage[margins=2.5cm,nohead,nofoot]{geometry}
%\usepackage{geometry}
\usepackage{amsfonts}
\usepackage{amstext}
\usepackage{latexsym}
\usepackage{amssymb}
\usepackage{color}


%\include{myPreamble}
\include{qm2pi.local} 

%\ifpdf
%\usepackage[pdftex]{graphicx}
%\else
%\usepackage{graphicx}
%\fi

 % \ifpdf
%  \usepackage{pdfsync}
%  \if


%\title{Brief Article}
%\author{David F. Snyder}
%\author{L.G. Meredith}

%\address{Dept. of Math., Texas State University--San Marcos, San Marcos, TX 78666}
       
\pagestyle{empty}


\begin{document}

\lstset{language=[Objective]Caml,frame=shadowbox}

\input{qm2pi.front}

% section front matter (end)

\input{qm2pi.intro} 
 
% section introduction (end)

% \input{qm2pi.knotations} 

% section notation (end)

\input{qm2pi.process.calculi} 

% section concurrent_process_calculi_and_spatial_logics_ (end)
    
%\input{qm2pi.knots2pi} 

%\input{qm2pi.trefoil} 

%\input{qm2pi.mainthm} 

% subsection basic_interpretation (end)

%\input{qm2pi.rho.presentation} 
\subsection{The syntax and semantics of the notation system}\label{sub:the_syntax_and_semantics_of_the_notation_system} % (fold)

We now summarize a technical presentation of the calculus that
embodies our theory of dynamics. The typical presentation of such a
calculus follows the style of giving generators and relations on
them. The grammar, below, describing term constructors, freely
generates the set of processes, $\Proc$. This set is then quotiented
by a relation known as structural congruence and it is over this set
that the notion of dynamics is expressed. This presentation is
essentially that of \cite{MeredithR05} with the addition of
polyadicity and summation. For readability we have relegated some of
the technical subtleties to an appendix.

\subsubsection{Process grammar}\label{subsub:process_grammar}

\begin{mathpar}
  \inferrule* [lab=synchronization] {} {{M} \bc \pzero \;|\; x?F \;|\; x!C }
  \and
  \inferrule* [lab=abstraction] {} {{F} \bc (x)P}
  \and
  \inferrule* [lab=concretion] {} {{C} \bc \langle Q \rangle}
  \and
  \inferrule* [lab=process] {} {{P,Q} \bc M \;| \;P|Q \;|\; @{x}}
  \and
  \inferrule* [lab=name] {} {{x} \bc \quotep{P}}
\end{mathpar} 

Note that $\vec{x}$ (resp. $\vec{P}$) denotes a vector of names
(resp. processes) of length $|\vec{x}|$ (resp. $|\vec{P}|$). We adopt
the following useful abbreviations.

\begin{mathpar}
   x?(\vec{y}).P := x.(\vec{y})P \and  x\clift{\vec{P}} := x.\clift{\vec{P}}
   \and x!(y) := \lift{x}{\dropn{y}}
   \and \Pi_{i=0}^{n-1}P_i := P_0 | \ldots | P_{n-1}
\end{mathpar}

\subsubsection{Structural congruence}

\paragraph{Free and bound names and alpha-equivalence.} At the
core of structural equivalence is alpha-equivalence which identifies
process that are the same up to a change of variable. Formally, we
recognize the distinction between free and bound names. The free names
of a process, $\freenames{P}$, may be calculated recursively as
follows:

\begin{mathpar}
\freenames{\pzero} := \emptyset
  \and \\
  \freenames{x?(y).P} := \{ x \} \cup (\freenames{P} \setminus \{ y \})
  \and 
  \freenames{x!\langle P \rangle} := \{ x \} \cup \{ P \} 
  \and \\
  \freenames{P|Q} := \freenames{P} \cup \freenames{Q}
  \and \\
  \freenames{@{x}} := \{ x \}
\end{mathpar}

$\pi$
$\quotep{\pi}$

$\freenames{-} : \pi \to \mathcal{P}(\quotep{\pi})$

\begin{eqnarray*}
  \freenames{\pzero} & := & \emptyset \\
  \freenames{x?(y).P} & := & \{ x \} \cup (\freenames{P} \setminus \{ y \}) \\
  \freenames{x!\langle P \rangle} & := & \{ x \} \cup \{ P \} \\
  \freenames{P|Q} & := & \freenames{P} \cup \freenames{Q} \\
  \freenames{\dropn{x}} & := & \{ x \}
\end{eqnarray*}

The bound names of a process, $\boundnames{P}$, are those names occurring in $P$
that are not free. For example, in $x?(y).0$, the name $x$ is free, while $y$ is bound.

\begin{mathpar}
  \inferrule* [lab=monoidal-laws] {} { P|Q \equiv Q|P \and P|0 \equiv P \and P|(Q|R) \equiv (P|Q)|R }
\end{mathpar}

\begin{mathpar}
  \inferrule* [lab=alpha-equivalence] {} { (x)P \equiv (y)P\{y/x\} \and y \not\in \freenames{P} }
\end{mathpar}

\begin{definition}
Then two processes, $P,Q$, are alpha-equivalent if $P = Q\{\vec{y}/\vec{x}\}$ for
some $\vec{x} \in \boundnames{Q},\vec{y} \in \boundnames{P}$, where $Q\{\vec{y}/\vec{x}\}$
denotes the capture-avoiding substitution of $\vec{y}$ for $\vec{x}$ in $Q$.
\end{definition}

\begin{definition}
  The {\em structural congruence} \cite{SangiorgiWalker} , $\equiv$,
  between processes is the least congruence containing
  alpha-equivalence, satisfying the abelian monoid laws
  (associativity, commutativity and $\pzero$ as identity) for parallel
  composition $|$ and for summation $+$.
\end{definition}

\subsection{Name equivalence}

We take name equivalence, written $\nameeq$, to be the smallest
equivalence relation generated by the following rules.

\begin{mathpar}
\inferrule*[lab=Quote-drop]
{ }
{ \quotep{@{x}} \nameeq x }

\inferrule*[lab=Struct-equiv]
{ P \scong Q }
{ \quotep{P} \nameeq \quotep{Q} }
\end{mathpar}

The astute reader will have noticed that the mutual recursion of names
and processes imposes a mutual recursion on alpha-equivalence and
structural equivalence via name-equivalence. Fortunately, all of this
works out pleasantly and we may calculate in the natural way, free of
concern. The reader interested in the details is referred to the
appendix \ref{appendix:rho_details}.

\subsection{Substitution}

We use $\Proc$ for the set of processes, $\QProc$ for the set of
names, and $\id{\{}\vec{y} / \vec{x} \id{\}}$ to denote partial maps,
$s : \QProc \rightarrow \QProc$. A map, $s$ lifts, uniquely, to a map
on process terms, $\widehat{s} : \Proc \rightarrow \Proc$ by the
following equations.

\begin{mathpar}
  (0) \psubstp{Q}{P} := 0 \\
  (R \juxtap S) \psubstp{Q}{P}
  :=    
  (R)\psubstp{Q}{P} \juxtap (S) \psubstp{Q}{P} \\
  (x?(y).R) \psubstp{Q}{P}    
  :=    
  (x)\substp{Q}{P} (z)\concat( (R \psubstn{z}{y}) \psubstp{Q}{P} ) \\
  (\lift{x}{R}) \psubstp{Q}{P}  
  :=
  \lift{(x)\substp{Q}{P}}{ R \psubstp{Q}{P} } \\
%   (\dropn{x})  \psubstp{Q}{P}       
%   := 
%   \left\{ 
%     \begin{array}{ccc} 
%       \dropn{\quotep{Q}} & & x \nameeq \quotep{P} \\
%       \dropn{x} & & otherwise \\
%     \end{array}
%   \right. 
  (\dropn{x})  \psubstp{Q}{P}       
  := 
  \left\{ 
    \begin{array}{ccc} 
      Q & & x \nameeq \quotep{P} \\
      \dropn{x} & & otherwise \\
    \end{array}
  \right.
\end{mathpar}
 

where

\begin{eqnarray}
  (x)\id{\{} \lpquote Q \rpquote / \lpquote P \rpquote \id{\}}            = 
  \left\{ 
    \begin{array}{ccc}
      \lpquote Q \rpquote & & x \nameeq \lpquote P \rpquote \\
      x & & otherwise \\
    \end{array}
  \right. \nonumber
\end{eqnarray}

and $z$ is chosen distinct from $\quotep{P}$, $\quotep{Q}$, the free
names in $Q$, and all the names in $R$. Our $\alpha$-equivalence will
be built in the standard way from this substitution.

\begin{remark}\label{rem:no_self_referential_names}
  One consequence of these definitions is that $\forall P. \quotep{P}
  \not\in \freenames{P}$.
\end{remark}

\subsection{ Dynamic quote: an example }

Anticipating something of what's to come, consider applying the
substitution, $\widehat{\id{\{}u / z \id{\}}}$, to the following pair
of processes, $\lift{w}{y!(z)}$ and $w[ \lpquote y!(z) \rpquote ]$.

\begin{eqnarray}
	\lift{w}{y!(z)}\widehat{\id{\{}u / z \id{\}}}
		& = &
		\lift{w}{y!(u)} \nonumber\\
	w[ \lpquote y!(z) \rpquote ] \widehat{ \id{\{}u / z \id{\}} }
		& = &
		w[ \lpquote y!(z) \rpquote ] \nonumber
\end{eqnarray}

Because the body of the process between quotes is impervious to
substitution, we get radically different answers. In fact, by
examining the first process in an input context,
e.g. $x?(z).\lift{w}{y!(z)}$, we see that the process under the lift
operator may be shaped by prefixed inputs binding a name inside it. In
this sense, the lift operator will be seen as a way to dynamically
construct processes before reifying them as names.

Finally equipped with these standard features we can present the
dynamics of the calculus.

\subsubsection{Operational semantics} 

Finally, we introduce the computational dynamics. What marks these
algebras as distinct from other more traditionally studied algebraic
structures, e.g. vector spaces or polynomial rings, is the manner in
which dynamics is captured. In traditional structures, dynamics is typically
expressed through morphisms between such structures, as in linear maps
between vector spaces or morphisms between rings. In algebras
associated with the semantics of computation, the dynamics is
expressed as part of the algebraic structure itself, through a
reduction reduction relation typically denoted by $\red$. Below, we
give a recursive presentation of this relation for the calculus used
in the encoding.

$\red \subseteq \pi \times \pi$
$\red : \pi \to \mathcal{P}(\pi)$

\begin{mathpar}
  \inferrule* [lab=Comm] { \textsf{match}( x_{src}, x_{trgt} ) } { x_{trgt}?(y)P \; | \; x_{src}!\langle {Q} \rangle \red P\{\quotep{Q}/y}\} }
  \and \\
  \inferrule* [lab=Par] {{P} \red {P}'} {{{P} | {Q}} \red {{P}' | {Q}}}
  \and
  \inferrule* [lab=Equiv]{{{P} \scong {P}'} \andalso {{P}' \red {Q}'} \andalso {{Q}' \scong {Q}}}{{P} \red {Q}}
\end{mathpar}

\begin{eqnarray*}
  match_{\equiv} (\quotep{P},\quotep{Q}) & := & P \equiv Q \\
  match_{\dagger}(\quotep{P},\quotep{Q}) & := & \forall R. P|Q \red^{*} R => R \red^{*} 0 \\
  match_{K}(\quotep{P},\quotep{Q}) & := & K \mbox{ for some context } K
\end{eqnarray*}

$u?(x)P | u!\langle Q \rangle \red P\{\quotep{Q}/x\}$

%We write $\wred$ for $\red^*$, and $P\red$ if $\exists Q $ such that $ P \red Q$.
We write $P\red$ if $\exists Q $ such that $ P \red Q$ and $P\not\red$, otherwise.

\section{Replication}

As mentioned before, it is known that replication (and hence
recursion) can be implemented in a higher-order process algebra
\cite{SangiorgiWalker}. As our first example of calculation with the
machinery thus far presented we give the construction explicitly in
the {\rhoc}.

\begin{eqnarray}
	D_{x} & := & \prefix{x}{y}{(\binpar{\outputp{x}{y}}{@{y}})} \nonumber\\
	\bangp_{x}{P} & := & \binpar{{x}!\langle{\binpar{D_{x}}{P}}\rangle}{D_{x}} \nonumber
\end{eqnarray}

\begin{eqnarray}
	\bangp_{x}{P} & & \nonumber\\
	=
	& {x}!\langle{(\prefix{x}{y}{(\outputp{x}{y} | @{y})) | P}}\rangle 
	      | \prefix{x}{y}{(\outputp{x}{y} | @{y})} & \nonumber\\
	\red
	& (\outputp{x}{y} | @{y})\substn{\quotep{(\prefix{x}{y}{(@{y} | \outputp{x}{y})) | P}}}{y} & \nonumber\\
	=
	& \outputp{x}{\quotep{(\prefix{x}{y}{(\outputp{x}{y} | @{y})) | P}}}
	  | {(\prefix{x}{y}{(\outputp{x}{y} | @{y})) | P}} & \nonumber\\
	\red
	& \ldots & \nonumber\\
	\red^*
	& P | P | \ldots & \nonumber
\end{eqnarray}

Of course, this encoding, as an implementation, runs away, unfolding
$\bangp{P}$ eagerly. A lazier and more implementable replication
operator, restricted to input-guarded processes, may be obtained as follows.

\begin{eqnarray}
\bangp{\prefix{u}{v}{P}} 
	:= 
	\binpar{\lift{x}{\prefix{u}{v}{(\binpar{D(x)}{P})}}}{D(x)} \nonumber
\end{eqnarray}

\begin{remark}
  Note that the lazier definition still does not deal with summation
  or mixed summation (i.e. sums over input and output). The reader is
  invited to construct definitions of replication that deal with these
  features. 

  Further, the definitions are parameterized in a name, $x$. Can you,
  gentle reader, make a definition that eliminates this parameter and
  guarantees no accidental interaction between the replication
  machinery and the process being replicated -- i.e. no accidental
  sharing of names used by the process to get its work done and the
  name(s) used by the replication to effect copying. This latter
  revision of the definition of replication is crucial to obtaining
  the expected identity $!!P \sim !P$.
\end{remark}

\begin{remark}\label{rem:paradoxical_combinator}
  The reader familiar with the lambda calculus will have noticed the
  similarity between $D$ and the paradoxical combinator.

  [Ed. note: the existence of this seems to suggest we have to be more
  restrictive on the set of processes and names we admit if we are to
  support no-cloning.]
\end{remark}

\subsubsection{Bisimulation}

The computational dynamics gives rise to another kind of equivalence,
the equivalence of computational behavior. As previously mentioned
this is typically captured \emph{via} some form of bisimulation.

% The notion we use in this paper is weak barbed bisimulation
% \cite{milner91polyadicpi}.

The notion we use in this paper is derived from weak barbed
bisimulation \cite{milner91polyadicpi}. 

\begin{definition}
An \emph{observation relation}, $\downarrow_{\mathcal N}$, over a set
of names, $\mathcal N$, is the smallest relation satisfying the rules
below.

\infrule[Out-barb]{y \in {\mathcal N}, \; x \nameeq y}
		  {\outputp{x}{v} \downarrow_{\mathcal N} x}
\infrule[Par-barb]{\mbox{$P\downarrow_{\mathcal N} x$ or $Q\downarrow_{\mathcal N} x$}}
		  {\binpar{P}{Q} \downarrow_{\mathcal N} x}

We write $P \Downarrow_{\mathcal N} x$ if there is $Q$ such that 
$P \wred Q$ and $Q \downarrow_{\mathcal N} x$.
\end{definition}

\begin{definition}
%\label{def.bbisim}
An  ${\mathcal N}$-\emph{barbed bisimulation} over a set of names, ${\mathcal N}$, is a symmetric binary relation 
${\mathcal S}_{\mathcal N}$ between agents such that $P\rel{S}_{\mathcal N}Q$ implies:
\begin{enumerate}
\item If $P \red P'$ then $Q \wred Q'$ and $P'\rel{S}_{\mathcal N} Q'$.
\item If $P\downarrow_{\mathcal N} x$, then $Q\Downarrow_{\mathcal N} x$.
\end{enumerate}
$P$ is ${\mathcal N}$-barbed bisimilar to $Q$, written
$P \wbbisim_{\mathcal N} Q$, if $P \rel{S}_{\mathcal N} Q$ for some ${\mathcal N}$-barbed bisimulation ${\mathcal S}_{\mathcal N}$.
\end{definition}

$\mathcal{R} \subseteq \pi \times \pi$

$P \mathcal{R} Q => \forall P'. P \red P' \Rightarrow \exists Q'. Q \red Q', P' \mathcal{R} Q'$

$P \vdash x \Rightarrow Q \vdash x$

\begin{mathpar}
  \inferrule*[lab=Out-barb]{x \nameeq y}{{y}!\langle{Q}\rangle \vdash x}
  \and
  \inferrule*[lab=Par-barb]{\mbox{$P\vdash x$ or $Q\vdash x$}}{\binpar{P}{Q} \vdash x}
\end{mathpar}

\subsubsection{Contexts}

One of the principle advantages of computational calculi like the
$\pi$-calculus is a well-defined notion of context,
contextual-equivalence and a correlation between
contextual-equivalence and notions of bisimulation. The notion of
context allows the decomposition of a process into (sub-)process and
its syntactic environment, its context. Thus, a context may be
thought of as a process with a ``hole'' (written $\Box$) in it. The
application of a context $M$ to a process $P$, written $M[P]$, is
tantamount to filling the hole in $M$ with $P$. In this paper we do
not need the full weight of this theory, but do make use of the notion
of context in the proof the main theorem. 

\begin{mathpar}
  \inferrule* [lab=summation] {} {{M_{M},M_{N}} \bc \Box \;|\; x.M_{A} \;|\; M_{M}+M_{N}}
  \and
  \inferrule* [lab=agent] {} {{M_{A}} \bc (\vec{x})M_{P} \;| \; \clift{P_0,\ldots,M_{P},\ldots,P_N}}
  \and \\
  \inferrule* [lab=process] {} {{M_{P}} \bc M_{N} \;| \;P|M_{P} }
\end{mathpar} 

\begin{mathpar}
  \inferrule* [lab=sychronization] {} {M_{N} \bc \Box \;|\; x?M_{F} \;|\; x!M_{C}}
  \and
  \inferrule* [lab=abstraction] {} {{M_{F}} \bc (x)M_{P} }
  \and
  \inferrule* [lab=concretion] {} {{M_{C}} \bc \langle M_{P} \rangle }
  \and \\
  \inferrule* [lab=process] {} {{M_{P}} \bc M_{N} \;| \;P|M_{P} }
\end{mathpar}

\begin{definition}[contextual application] Given a context $M$, and
  process $P$, we define the \emph{contextual application}, $M[P] :=
  M\{P/\Box\}$. That is, the contextual application of M to P is the
  substitution of $P$ for $\Box$ in $M$.
\end{definition}

$\meaningof{-} : L \to \mathcal{P}(\pi)$

\begin{mathpar}
  \inferrule* [lab=collection] {} {\meaningof{true} = \pi, \and \meaningof{~E} = \pi \setminus \meaningof{E}, \and \meaningof{E_{1} \& E_{2}} = \meaningof{E_{1}} \cap \meaningof{E_{2}}}
\end{mathpar}

\begin{mathpar}
  \inferrule* [lab=structure] {} {\meaningof{0} = \{ P \in \pi | P \equiv 0 \}, \and \\ \meaningof{E_1 | E_2} = \{ P \in \pi | P \equiv P_{1} | P_{2}, P_{1} \in \meaningof{E_{1}}, P_{2} \in \meaningof{E_2}\} }
\end{mathpar}

\begin{mathpar}
 \inferrule* [lab=behavior] {} {\meaningof{\langle a?b \rangle E} = \{ P \in \pi | P \equiv Q | u?(y)P', \\ \and \\\\ \and \\ \;\;\; u \in \meaningof{a}, \forall z.P'\{z/y\} \in \meaningof{E\{z/b\}}\}, \and \\ \meaningof{a!E} = \{ P \in \pi | P \equiv Q | x!\langle P' \rangle, x \in \meaningof{a} P' \in \meaningof{E}\} }
\end{mathpar}

\begin{mathpar}
 \inferrule* [lab=nominal] {} {\meaningof{\quotep{E}} = \{ \quotep{P} \in \quotep{\pi} | P \in \meaningof{E} \}, \and \meaningof{\quotep{P}} = \{ \quotep{Q} \in \quotep{\pi} | P \equiv Q \} \and \\ \meaningof{@\quotep{E}} = \{ P \in \pi | P \equiv @x, x \in \meaningof{E} \}}
\end{mathpar}

\begin{eqnarray*}
  \\
  \meaningof{-} : TS \to ST
\end{eqnarray*}

\begin{eqnarray*}
  \\
  L : TS \to ST
\end{eqnarray*}

\begin{eqnarray*}
  \\
  P \models E \iff P \in \meaningof{E}
\end{eqnarray*}

\begin{eqnarray*}
  P \approx_{L} Q \iff \forall E \in L. P \models E \iff Q \models E
\end{eqnarray*}

\begin{eqnarray*}
  P \approx_{K} Q
\end{eqnarray*}

\begin{eqnarray*}
  P \approx Q
\end{eqnarray*}

$\approx_{K} = \approx = \approx_{L}$

\subsubsection{Contextual duality}

Note that contexts extend the quotation operation to a family of
operations from processes to names. Given a context, $M$, we can
define a \emph{nominal context}, $\quotep{M}$ by $\quotep{M}[P] :=
\quotep{M[P]}$. To foreshadow what is to come we observe that these
operations enjoy a duality with processes very much like the duality
between vectors and maps from vectors to scalars.

Further, because the calculus is essentially higher-order, we have a
correspondence between contexts and processes. More specifically,
given a name $x$ and a context $M$ we can construct $M^{*}_{x}$ such
that 

\begin{mathpar}
  M^{*}_{x} | \lift{x}{P} \red M[P]
\end{mathpar}

namely,

\begin{mathpar}
  M^{*}_{x} := x?(u).M[\dropn{u}]
\end{mathpar}

The dependence of $M^{*}_{x}$ on a name makes it an abstraction, 

\begin{mathpar}
  M^{*} := (x)x?(u).M[\dropn{u}]
\end{mathpar}

\subsection{Additional notation}

It will sometimes be convenient to denote the process a name
quotes. We already have the notation $x = \quotep{P}$, but it will be
convenient to introduce an alternate notation, $\procn{x}$, when we
want to emphasize the connection to the use of the name. Note that, by
virtue of name equivalence, $\quotep{\procn{x}} \nameeq x$; so, the
notation is consistent with previous definitions.

Further, because names have structure it is possible to effect
substitutions on the basis of that structure. This means we need to
upgrade our notation for substitutions, which we accomplish by
adapting comprehension notation. Thus,

\begin{mathpar}
  P\{ y / x : x \in S \}
\end{mathpar}

is interpreted to mean the process derived from P by replacing (in a
capture-avoiding manner) each occurrence of $x$ in $S$ by $y$. For example,

\begin{mathpar}
  P\{ \quotep{\procn{x}|\procn{x}} / x : x \in \freenames{P} \}
\end{mathpar}

will replace each (occurrence) of a free name $x$ in $P$ by
$\quotep{\procn{x}|\procn{x}}$.

Also, we will avail ourselves of the notation $x^{L}$ and $x^{R}$ to
denote injections of a name into disjoint copies of the name
space. There are numerous ways to accomplish this. One example can be
found in \cite{MeredithR05}. This notation overloads to vectors of
names: $\vec{x}^{\pi} := (x_{i}^{\pi} \; : \; 0 \leq i < |\vec{x}| )$ where $\pi \in \{L,R\}$.

We also use $P^{\Box} := P|\Box$.

In \cite{MeredithR05} an interpretation of the new operator is
given. It turns out that there are several possible interpretations
all enjoying the requisite algebraic properties of the operator (see
\cite{milner91polyadicpi}). We will therefore make liberal use of
$(\nu\; \vec{x})P$.

% subsection the_syntax_and_semantics_of_the_notation_system (end)   

\input{qm2pi.qmops} 

\input{qm2pi.sterngerlach} 

\input{qm2pi.metric} 

% section concurrent_process_calculi (end)

%\input{qm2pi.proofsketch}

% section proof sketch (end)

%\input{qm2pi.slviaknots} 

% section spatial logic via knots (end)

\input{qm2pi.conclusion}

% section conclusion (end)

%\input{qm2pi.dtcodes} 

% section wiring algorithm (end)

\input{qm2pi.ack} 

% section acknowledgments (end)

\newpage


\bibliographystyle{plain}   
\bibliography{../../biblios/main.bib}

\input{qm2pi.rhodetails}

\end{document}



\end{document}

 

% subsection basic_interpretation (end)

%\input{qm2pi.rho.presentation} 
\subsection{The syntax and semantics of the notation system}\label{sub:the_syntax_and_semantics_of_the_notation_system} % (fold)

We now summarize a technical presentation of the calculus that
embodies our theory of dynamics. The typical presentation of such a
calculus follows the style of giving generators and relations on
them. The grammar, below, describing term constructors, freely
generates the set of processes, $\Proc$. This set is then quotiented
by a relation known as structural congruence and it is over this set
that the notion of dynamics is expressed. This presentation is
essentially that of \cite{MeredithR05} with the addition of
polyadicity and summation. For readability we have relegated some of
the technical subtleties to an appendix.

\subsubsection{Process grammar}\label{subsub:process_grammar}

\begin{mathpar}
  \inferrule* [lab=synchronization] {} {{M} \bc \pzero \;|\; x?F \;|\; x!C }
  \and
  \inferrule* [lab=abstraction] {} {{F} \bc (x)P}
  \and
  \inferrule* [lab=concretion] {} {{C} \bc \langle Q \rangle}
  \and
  \inferrule* [lab=process] {} {{P,Q} \bc M \;| \;P|Q \;|\; @{x}}
  \and
  \inferrule* [lab=name] {} {{x} \bc \quotep{P}}
\end{mathpar} 

Note that $\vec{x}$ (resp. $\vec{P}$) denotes a vector of names
(resp. processes) of length $|\vec{x}|$ (resp. $|\vec{P}|$). We adopt
the following useful abbreviations.

\begin{mathpar}
   x?(\vec{y}).P := x.(\vec{y})P \and  x\clift{\vec{P}} := x.\clift{\vec{P}}
   \and x!(y) := \lift{x}{\dropn{y}}
   \and \Pi_{i=0}^{n-1}P_i := P_0 | \ldots | P_{n-1}
\end{mathpar}

\subsubsection{Structural congruence}

\paragraph{Free and bound names and alpha-equivalence.} At the
core of structural equivalence is alpha-equivalence which identifies
process that are the same up to a change of variable. Formally, we
recognize the distinction between free and bound names. The free names
of a process, $\freenames{P}$, may be calculated recursively as
follows:

\begin{mathpar}
\freenames{\pzero} := \emptyset
  \and \\
  \freenames{x?(y).P} := \{ x \} \cup (\freenames{P} \setminus \{ y \})
  \and 
  \freenames{x!\langle P \rangle} := \{ x \} \cup \{ P \} 
  \and \\
  \freenames{P|Q} := \freenames{P} \cup \freenames{Q}
  \and \\
  \freenames{@{x}} := \{ x \}
\end{mathpar}

$\pi$
$\quotep{\pi}$

$\freenames{-} : \pi \to \mathcal{P}(\quotep{\pi})$

\begin{eqnarray*}
  \freenames{\pzero} & := & \emptyset \\
  \freenames{x?(y).P} & := & \{ x \} \cup (\freenames{P} \setminus \{ y \}) \\
  \freenames{x!\langle P \rangle} & := & \{ x \} \cup \{ P \} \\
  \freenames{P|Q} & := & \freenames{P} \cup \freenames{Q} \\
  \freenames{\dropn{x}} & := & \{ x \}
\end{eqnarray*}

The bound names of a process, $\boundnames{P}$, are those names occurring in $P$
that are not free. For example, in $x?(y).0$, the name $x$ is free, while $y$ is bound.

\begin{mathpar}
  \inferrule* [lab=monoidal-laws] {} { P|Q \equiv Q|P \and P|0 \equiv P \and P|(Q|R) \equiv (P|Q)|R }
\end{mathpar}

\begin{mathpar}
  \inferrule* [lab=alpha-equivalence] {} { (x)P \equiv (y)P\{y/x\} \and y \not\in \freenames{P} }
\end{mathpar}

\begin{definition}
Then two processes, $P,Q$, are alpha-equivalent if $P = Q\{\vec{y}/\vec{x}\}$ for
some $\vec{x} \in \boundnames{Q},\vec{y} \in \boundnames{P}$, where $Q\{\vec{y}/\vec{x}\}$
denotes the capture-avoiding substitution of $\vec{y}$ for $\vec{x}$ in $Q$.
\end{definition}

\begin{definition}
  The {\em structural congruence} \cite{SangiorgiWalker} , $\equiv$,
  between processes is the least congruence containing
  alpha-equivalence, satisfying the abelian monoid laws
  (associativity, commutativity and $\pzero$ as identity) for parallel
  composition $|$ and for summation $+$.
\end{definition}

\subsection{Name equivalence}

We take name equivalence, written $\nameeq$, to be the smallest
equivalence relation generated by the following rules.

\begin{mathpar}
\inferrule*[lab=Quote-drop]
{ }
{ \quotep{@{x}} \nameeq x }

\inferrule*[lab=Struct-equiv]
{ P \scong Q }
{ \quotep{P} \nameeq \quotep{Q} }
\end{mathpar}

The astute reader will have noticed that the mutual recursion of names
and processes imposes a mutual recursion on alpha-equivalence and
structural equivalence via name-equivalence. Fortunately, all of this
works out pleasantly and we may calculate in the natural way, free of
concern. The reader interested in the details is referred to the
appendix \ref{appendix:rho_details}.

\subsection{Substitution}

We use $\Proc$ for the set of processes, $\QProc$ for the set of
names, and $\id{\{}\vec{y} / \vec{x} \id{\}}$ to denote partial maps,
$s : \QProc \rightarrow \QProc$. A map, $s$ lifts, uniquely, to a map
on process terms, $\widehat{s} : \Proc \rightarrow \Proc$ by the
following equations.

\begin{mathpar}
  (0) \psubstp{Q}{P} := 0 \\
  (R \juxtap S) \psubstp{Q}{P}
  :=    
  (R)\psubstp{Q}{P} \juxtap (S) \psubstp{Q}{P} \\
  (x?(y).R) \psubstp{Q}{P}    
  :=    
  (x)\substp{Q}{P} (z)\concat( (R \psubstn{z}{y}) \psubstp{Q}{P} ) \\
  (\lift{x}{R}) \psubstp{Q}{P}  
  :=
  \lift{(x)\substp{Q}{P}}{ R \psubstp{Q}{P} } \\
%   (\dropn{x})  \psubstp{Q}{P}       
%   := 
%   \left\{ 
%     \begin{array}{ccc} 
%       \dropn{\quotep{Q}} & & x \nameeq \quotep{P} \\
%       \dropn{x} & & otherwise \\
%     \end{array}
%   \right. 
  (\dropn{x})  \psubstp{Q}{P}       
  := 
  \left\{ 
    \begin{array}{ccc} 
      Q & & x \nameeq \quotep{P} \\
      \dropn{x} & & otherwise \\
    \end{array}
  \right.
\end{mathpar}
 

where

\begin{eqnarray}
  (x)\id{\{} \lpquote Q \rpquote / \lpquote P \rpquote \id{\}}            = 
  \left\{ 
    \begin{array}{ccc}
      \lpquote Q \rpquote & & x \nameeq \lpquote P \rpquote \\
      x & & otherwise \\
    \end{array}
  \right. \nonumber
\end{eqnarray}

and $z$ is chosen distinct from $\quotep{P}$, $\quotep{Q}$, the free
names in $Q$, and all the names in $R$. Our $\alpha$-equivalence will
be built in the standard way from this substitution.

\begin{remark}\label{rem:no_self_referential_names}
  One consequence of these definitions is that $\forall P. \quotep{P}
  \not\in \freenames{P}$.
\end{remark}

\subsection{ Dynamic quote: an example }

Anticipating something of what's to come, consider applying the
substitution, $\widehat{\id{\{}u / z \id{\}}}$, to the following pair
of processes, $\lift{w}{y!(z)}$ and $w[ \lpquote y!(z) \rpquote ]$.

\begin{eqnarray}
	\lift{w}{y!(z)}\widehat{\id{\{}u / z \id{\}}}
		& = &
		\lift{w}{y!(u)} \nonumber\\
	w[ \lpquote y!(z) \rpquote ] \widehat{ \id{\{}u / z \id{\}} }
		& = &
		w[ \lpquote y!(z) \rpquote ] \nonumber
\end{eqnarray}

Because the body of the process between quotes is impervious to
substitution, we get radically different answers. In fact, by
examining the first process in an input context,
e.g. $x?(z).\lift{w}{y!(z)}$, we see that the process under the lift
operator may be shaped by prefixed inputs binding a name inside it. In
this sense, the lift operator will be seen as a way to dynamically
construct processes before reifying them as names.

Finally equipped with these standard features we can present the
dynamics of the calculus.

\subsubsection{Operational semantics} 

Finally, we introduce the computational dynamics. What marks these
algebras as distinct from other more traditionally studied algebraic
structures, e.g. vector spaces or polynomial rings, is the manner in
which dynamics is captured. In traditional structures, dynamics is typically
expressed through morphisms between such structures, as in linear maps
between vector spaces or morphisms between rings. In algebras
associated with the semantics of computation, the dynamics is
expressed as part of the algebraic structure itself, through a
reduction reduction relation typically denoted by $\red$. Below, we
give a recursive presentation of this relation for the calculus used
in the encoding.

$\red \subseteq \pi \times \pi$
$\red : \pi \to \mathcal{P}(\pi)$

\begin{mathpar}
  \inferrule* [lab=Comm] { \textsf{match}( x_{src}, x_{trgt} ) } { x_{trgt}?(y)P \; | \; x_{src}!\langle {Q} \rangle \red P\{\quotep{Q}/y}\} }
  \and \\
  \inferrule* [lab=Par] {{P} \red {P}'} {{{P} | {Q}} \red {{P}' | {Q}}}
  \and
  \inferrule* [lab=Equiv]{{{P} \scong {P}'} \andalso {{P}' \red {Q}'} \andalso {{Q}' \scong {Q}}}{{P} \red {Q}}
\end{mathpar}

\begin{eqnarray*}
  match_{\equiv} (\quotep{P},\quotep{Q}) & := & P \equiv Q \\
  match_{\dagger}(\quotep{P},\quotep{Q}) & := & \forall R. P|Q \red^{*} R => R \red^{*} 0 \\
  match_{K}(\quotep{P},\quotep{Q}) & := & K \mbox{ for some context } K
\end{eqnarray*}

$u?(x)P | u!\langle Q \rangle \red P\{\quotep{Q}/x\}$

%We write $\wred$ for $\red^*$, and $P\red$ if $\exists Q $ such that $ P \red Q$.
We write $P\red$ if $\exists Q $ such that $ P \red Q$ and $P\not\red$, otherwise.

\section{Replication}

As mentioned before, it is known that replication (and hence
recursion) can be implemented in a higher-order process algebra
\cite{SangiorgiWalker}. As our first example of calculation with the
machinery thus far presented we give the construction explicitly in
the {\rhoc}.

\begin{eqnarray}
	D_{x} & := & \prefix{x}{y}{(\binpar{\outputp{x}{y}}{@{y}})} \nonumber\\
	\bangp_{x}{P} & := & \binpar{{x}!\langle{\binpar{D_{x}}{P}}\rangle}{D_{x}} \nonumber
\end{eqnarray}

\begin{eqnarray}
	\bangp_{x}{P} & & \nonumber\\
	=
	& {x}!\langle{(\prefix{x}{y}{(\outputp{x}{y} | @{y})) | P}}\rangle 
	      | \prefix{x}{y}{(\outputp{x}{y} | @{y})} & \nonumber\\
	\red
	& (\outputp{x}{y} | @{y})\substn{\quotep{(\prefix{x}{y}{(@{y} | \outputp{x}{y})) | P}}}{y} & \nonumber\\
	=
	& \outputp{x}{\quotep{(\prefix{x}{y}{(\outputp{x}{y} | @{y})) | P}}}
	  | {(\prefix{x}{y}{(\outputp{x}{y} | @{y})) | P}} & \nonumber\\
	\red
	& \ldots & \nonumber\\
	\red^*
	& P | P | \ldots & \nonumber
\end{eqnarray}

Of course, this encoding, as an implementation, runs away, unfolding
$\bangp{P}$ eagerly. A lazier and more implementable replication
operator, restricted to input-guarded processes, may be obtained as follows.

\begin{eqnarray}
\bangp{\prefix{u}{v}{P}} 
	:= 
	\binpar{\lift{x}{\prefix{u}{v}{(\binpar{D(x)}{P})}}}{D(x)} \nonumber
\end{eqnarray}

\begin{remark}
  Note that the lazier definition still does not deal with summation
  or mixed summation (i.e. sums over input and output). The reader is
  invited to construct definitions of replication that deal with these
  features. 

  Further, the definitions are parameterized in a name, $x$. Can you,
  gentle reader, make a definition that eliminates this parameter and
  guarantees no accidental interaction between the replication
  machinery and the process being replicated -- i.e. no accidental
  sharing of names used by the process to get its work done and the
  name(s) used by the replication to effect copying. This latter
  revision of the definition of replication is crucial to obtaining
  the expected identity $!!P \sim !P$.
\end{remark}

\begin{remark}\label{rem:paradoxical_combinator}
  The reader familiar with the lambda calculus will have noticed the
  similarity between $D$ and the paradoxical combinator.

  [Ed. note: the existence of this seems to suggest we have to be more
  restrictive on the set of processes and names we admit if we are to
  support no-cloning.]
\end{remark}

\subsubsection{Bisimulation}

The computational dynamics gives rise to another kind of equivalence,
the equivalence of computational behavior. As previously mentioned
this is typically captured \emph{via} some form of bisimulation.

% The notion we use in this paper is weak barbed bisimulation
% \cite{milner91polyadicpi}.

The notion we use in this paper is derived from weak barbed
bisimulation \cite{milner91polyadicpi}. 

\begin{definition}
An \emph{observation relation}, $\downarrow_{\mathcal N}$, over a set
of names, $\mathcal N$, is the smallest relation satisfying the rules
below.

\infrule[Out-barb]{y \in {\mathcal N}, \; x \nameeq y}
		  {\outputp{x}{v} \downarrow_{\mathcal N} x}
\infrule[Par-barb]{\mbox{$P\downarrow_{\mathcal N} x$ or $Q\downarrow_{\mathcal N} x$}}
		  {\binpar{P}{Q} \downarrow_{\mathcal N} x}

We write $P \Downarrow_{\mathcal N} x$ if there is $Q$ such that 
$P \wred Q$ and $Q \downarrow_{\mathcal N} x$.
\end{definition}

\begin{definition}
%\label{def.bbisim}
An  ${\mathcal N}$-\emph{barbed bisimulation} over a set of names, ${\mathcal N}$, is a symmetric binary relation 
${\mathcal S}_{\mathcal N}$ between agents such that $P\rel{S}_{\mathcal N}Q$ implies:
\begin{enumerate}
\item If $P \red P'$ then $Q \wred Q'$ and $P'\rel{S}_{\mathcal N} Q'$.
\item If $P\downarrow_{\mathcal N} x$, then $Q\Downarrow_{\mathcal N} x$.
\end{enumerate}
$P$ is ${\mathcal N}$-barbed bisimilar to $Q$, written
$P \wbbisim_{\mathcal N} Q$, if $P \rel{S}_{\mathcal N} Q$ for some ${\mathcal N}$-barbed bisimulation ${\mathcal S}_{\mathcal N}$.
\end{definition}

$\mathcal{R} \subseteq \pi \times \pi$

$P \mathcal{R} Q => \forall P'. P \red P' \Rightarrow \exists Q'. Q \red Q', P' \mathcal{R} Q'$

$P \vdash x \Rightarrow Q \vdash x$

\begin{mathpar}
  \inferrule*[lab=Out-barb]{x \nameeq y}{{y}!\langle{Q}\rangle \vdash x}
  \and
  \inferrule*[lab=Par-barb]{\mbox{$P\vdash x$ or $Q\vdash x$}}{\binpar{P}{Q} \vdash x}
\end{mathpar}

\subsubsection{Contexts}

One of the principle advantages of computational calculi like the
$\pi$-calculus is a well-defined notion of context,
contextual-equivalence and a correlation between
contextual-equivalence and notions of bisimulation. The notion of
context allows the decomposition of a process into (sub-)process and
its syntactic environment, its context. Thus, a context may be
thought of as a process with a ``hole'' (written $\Box$) in it. The
application of a context $M$ to a process $P$, written $M[P]$, is
tantamount to filling the hole in $M$ with $P$. In this paper we do
not need the full weight of this theory, but do make use of the notion
of context in the proof the main theorem. 

\begin{mathpar}
  \inferrule* [lab=summation] {} {{M_{M},M_{N}} \bc \Box \;|\; x.M_{A} \;|\; M_{M}+M_{N}}
  \and
  \inferrule* [lab=agent] {} {{M_{A}} \bc (\vec{x})M_{P} \;| \; \clift{P_0,\ldots,M_{P},\ldots,P_N}}
  \and \\
  \inferrule* [lab=process] {} {{M_{P}} \bc M_{N} \;| \;P|M_{P} }
\end{mathpar} 

\begin{mathpar}
  \inferrule* [lab=sychronization] {} {M_{N} \bc \Box \;|\; x?M_{F} \;|\; x!M_{C}}
  \and
  \inferrule* [lab=abstraction] {} {{M_{F}} \bc (x)M_{P} }
  \and
  \inferrule* [lab=concretion] {} {{M_{C}} \bc \langle M_{P} \rangle }
  \and \\
  \inferrule* [lab=process] {} {{M_{P}} \bc M_{N} \;| \;P|M_{P} }
\end{mathpar}

\begin{definition}[contextual application] Given a context $M$, and
  process $P$, we define the \emph{contextual application}, $M[P] :=
  M\{P/\Box\}$. That is, the contextual application of M to P is the
  substitution of $P$ for $\Box$ in $M$.
\end{definition}

$\meaningof{-} : L \to \mathcal{P}(\pi)$

\begin{mathpar}
  \inferrule* [lab=collection] {} {\meaningof{true} = \pi, \and \meaningof{~E} = \pi \setminus \meaningof{E}, \and \meaningof{E_{1} \& E_{2}} = \meaningof{E_{1}} \cap \meaningof{E_{2}}}
\end{mathpar}

\begin{mathpar}
  \inferrule* [lab=structure] {} {\meaningof{0} = \{ P \in \pi | P \equiv 0 \}, \and \\ \meaningof{E_1 | E_2} = \{ P \in \pi | P \equiv P_{1} | P_{2}, P_{1} \in \meaningof{E_{1}}, P_{2} \in \meaningof{E_2}\} }
\end{mathpar}

\begin{mathpar}
 \inferrule* [lab=behavior] {} {\meaningof{\langle a?b \rangle E} = \{ P \in \pi | P \equiv Q | u?(y)P', \\ \and \\\\ \and \\ \;\;\; u \in \meaningof{a}, \forall z.P'\{z/y\} \in \meaningof{E\{z/b\}}\}, \and \\ \meaningof{a!E} = \{ P \in \pi | P \equiv Q | x!\langle P' \rangle, x \in \meaningof{a} P' \in \meaningof{E}\} }
\end{mathpar}

\begin{mathpar}
 \inferrule* [lab=nominal] {} {\meaningof{\quotep{E}} = \{ \quotep{P} \in \quotep{\pi} | P \in \meaningof{E} \}, \and \meaningof{\quotep{P}} = \{ \quotep{Q} \in \quotep{\pi} | P \equiv Q \} \and \\ \meaningof{@\quotep{E}} = \{ P \in \pi | P \equiv @x, x \in \meaningof{E} \}}
\end{mathpar}

\begin{eqnarray*}
  \\
  \meaningof{-} : TS \to ST
\end{eqnarray*}

\begin{eqnarray*}
  \\
  L : TS \to ST
\end{eqnarray*}

\begin{eqnarray*}
  \\
  P \models E \iff P \in \meaningof{E}
\end{eqnarray*}

\begin{eqnarray*}
  P \approx_{L} Q \iff \forall E \in L. P \models E \iff Q \models E
\end{eqnarray*}

\begin{eqnarray*}
  P \approx_{K} Q
\end{eqnarray*}

\begin{eqnarray*}
  P \approx Q
\end{eqnarray*}

$\approx_{K} = \approx = \approx_{L}$

\subsubsection{Contextual duality}

Note that contexts extend the quotation operation to a family of
operations from processes to names. Given a context, $M$, we can
define a \emph{nominal context}, $\quotep{M}$ by $\quotep{M}[P] :=
\quotep{M[P]}$. To foreshadow what is to come we observe that these
operations enjoy a duality with processes very much like the duality
between vectors and maps from vectors to scalars.

Further, because the calculus is essentially higher-order, we have a
correspondence between contexts and processes. More specifically,
given a name $x$ and a context $M$ we can construct $M^{*}_{x}$ such
that 

\begin{mathpar}
  M^{*}_{x} | \lift{x}{P} \red M[P]
\end{mathpar}

namely,

\begin{mathpar}
  M^{*}_{x} := x?(u).M[\dropn{u}]
\end{mathpar}

The dependence of $M^{*}_{x}$ on a name makes it an abstraction, 

\begin{mathpar}
  M^{*} := (x)x?(u).M[\dropn{u}]
\end{mathpar}

\subsection{Additional notation}

It will sometimes be convenient to denote the process a name
quotes. We already have the notation $x = \quotep{P}$, but it will be
convenient to introduce an alternate notation, $\procn{x}$, when we
want to emphasize the connection to the use of the name. Note that, by
virtue of name equivalence, $\quotep{\procn{x}} \nameeq x$; so, the
notation is consistent with previous definitions.

Further, because names have structure it is possible to effect
substitutions on the basis of that structure. This means we need to
upgrade our notation for substitutions, which we accomplish by
adapting comprehension notation. Thus,

\begin{mathpar}
  P\{ y / x : x \in S \}
\end{mathpar}

is interpreted to mean the process derived from P by replacing (in a
capture-avoiding manner) each occurrence of $x$ in $S$ by $y$. For example,

\begin{mathpar}
  P\{ \quotep{\procn{x}|\procn{x}} / x : x \in \freenames{P} \}
\end{mathpar}

will replace each (occurrence) of a free name $x$ in $P$ by
$\quotep{\procn{x}|\procn{x}}$.

Also, we will avail ourselves of the notation $x^{L}$ and $x^{R}$ to
denote injections of a name into disjoint copies of the name
space. There are numerous ways to accomplish this. One example can be
found in \cite{MeredithR05}. This notation overloads to vectors of
names: $\vec{x}^{\pi} := (x_{i}^{\pi} \; : \; 0 \leq i < |\vec{x}| )$ where $\pi \in \{L,R\}$.

We also use $P^{\Box} := P|\Box$.

In \cite{MeredithR05} an interpretation of the new operator is
given. It turns out that there are several possible interpretations
all enjoying the requisite algebraic properties of the operator (see
\cite{milner91polyadicpi}). We will therefore make liberal use of
$(\nu\; \vec{x})P$.

% subsection the_syntax_and_semantics_of_the_notation_system (end)   

\section{Interpretation of QM}
\subsection{Supporting definitions}
\subsubsection{Multiplication}
\begin{mathpar}
  \quotep{Q} \cdot \quotep{R} := \quotep{Q|R}
  \and \\
  \quotep{Q} \cdot P := P\{ \quotep{Q|R} / \quotep{R} : \quotep{R} \in \freenames{P} \}
\end{mathpar}

\paragraph{Discussion}
The first line needs little explanation. The second line says that
each free name of the process is replaced with the multiplication of
that name by the scalar. Multiplication of a scalar (name) by a state
(process) results in a process all the names of which have been `moved
over' by parallel composition with the process the scalar
quotes. There is a subtlety that the bound names have to be
manipulated so that multiplied names aren't accidentally
captured. There are many ways to achieve this.

\begin{remark}\label{rem:multiplication_identities}
  The reader is invited to verify that for all $x,y,z \in \QProc$ and $P \in \Proc$
  \begin{mathpar}
    x \cdot \quotep{0} \equiv x 
    \and
    x \cdot y \equiv y \cdot x
    \and
    x \cdot (y \cdot z) \equiv (x \cdot y) \cdot z
    \and \\
    \quotep{0} \cdot P \equiv P
    \and \\
    x \cdot (y \cdot P) \equiv (x \cdot y) \cdot P
    \and \\
    x \cdot (P|Q) \equiv (x \cdot P) | (x \cdot Q)
    \and \\    
  \end{mathpar}
\end{remark}

\subsubsection{Tensor product}

We define a tensor product on processes by structural induction.

\paragraph{Tensor of sums} First note that all summations, including
$\pzero$ and sequence, can be written $\Sigma_{i} x_{i}.A_{i} +
\Sigma_{j} x_{j}.C_{j}$, where we have grouped input-guarded processes
together and output-guarded processes together.

Thus, we can define the tensor product of two summations, $N_{1}\otimes N_{2}$, where

\begin{mathpar}
  N_{1} := \Sigma_{i} x_{i}.A_{i} + \Sigma_{j} x_{j}.C_{j}
  \and
  N_{2} := \Sigma_{i'} y_{i'}.B_{i'} + \Sigma_{j'} y_{j'}.D_{j'} 
\end{mathpar}

as follows.

\begin{mathpar}
  \Sigma_{i} x_{i}.A_{i} + \Sigma_{j} x_{j}.C_{j} \otimes \Sigma_{i'}
  y_{i'}.B_{i'} + \Sigma_{j'} y_{j'}.D_{j'} 
  \and \\
  := \; \Sigma_{i} \Sigma_{i'} \quotep{\stackrel{\vee}{x_{i}}| \stackrel{\vee}{y_{i'}}}.(A_{i}\otimes B_{i'}) \; | \; \Sigma_{i'} \Sigma_{i} \quotep{\stackrel{\vee}{y_{i'}}|\stackrel{\vee}{x_{i}}}.(B_{i'}\otimes A_{i})
  \and
  \;\; | \;\; \Sigma_{j} \Sigma_{j'} \quotep{\stackrel{\vee}{x_{j}}|\stackrel{\vee}{y_{j'}}}.(A_{j}\otimes B_{j'}) \; | \; \Sigma_{j'} \Sigma_{j} \quotep{\stackrel{\vee}{y_{j'}}|\stackrel{\vee}{x_{j}}}.(B_{j'}\otimes A_{j})
\end{mathpar}

\begin{remark}
  Do we need to $x^{L}$ and $y^{R}$ for this construction as well?
\end{remark}

\paragraph{Tensor of parallel compositions} Next, we distribute tensor
over par.

\begin{mathpar}
  P_{1}|P_{2} \otimes Q_{1}|Q_{2} := (P_{1} \otimes Q_{1}) | (P_{1}
  \otimes Q_{2}) | (P_{2} \otimes Q_{1}) | (P_{2} \otimes Q_{2})
\end{mathpar}

\paragraph{Tensor with dropped names} We treat tensor of a
process with a dropped name as parallel composition.

\begin{mathpar}
  P \otimes \dropn{x} := P | \dropn{x}
\end{mathpar}

\paragraph{Tensor of agents}

Finally, we need to define tensor on agents. Note that the definition
of tensor on normal products only tensors inputs with inputs and
outputs with outputs. Thus, we only have to define the operation on
``homogeneous'' pairings.

\begin{mathpar}
  (\vec{x})P \otimes (\vec{y})Q
  \and \\
  := (x_{0}^{L}|y_{0}^{R},\ldots,x_{0}^{L}|y_{n}^{R},\ldots,x_{m}^{L}|y_{0}^{R},\ldots,x_{m}^{L}|y_{n}^R)(P\{ \vec{x}^{L}/\vec{x}\} \otimes Q \{ \vec{y}^{R}/\vec{y}\})
  \and \\
  \clift{\vec{P}} \otimes \clift{\vec{Q}}
  \and \\
  := \clift{P_{0}\otimes Q_{0},\ldots,P_{0}\otimes Q_{n},\ldots,P_{m}\otimes Q_{0},\ldots,P_{m}\otimes Q_{n}}
\end{mathpar}

\begin{remark}
  Observe that arities of tensored abstractions matches arities of
  tensored concretions if the original arities matched. Note also that
  the length of the arities corresponds to the increase in dimension
  we see in ordinary vector space tensor product.
\end{remark}

\begin{remark}
  Operationally, this definition distributes the tensor down to
  components ``linked'' by summation. Tensor over summation is
  intriguing in that it mixes names. Moreover, as a consequence of the
  way it mixes names we have the identities for all $x \in \QProc$ and
  $P,Q \in \Proc$

  \begin{mathpar}
    (x \cdot P) \otimes Q \equiv x \cdot (P \otimes Q) \equiv P \otimes (x \cdot Q)
    \and
    P \otimes \pzero \equiv P
  \end{mathpar}

  that the reader is invited to verify.
\end{remark}

\subsubsection{Annihilation}
\begin{mathpar}
  P^{\perp} := \{ Q | \forall R. P|Q \red^{*} R \Rightarrow R \red^{*} \pzero \}
  \and \\
  P^{\underline{\perp}} := \Sigma_{Q \in P^{\perp}} \quotep{Q}?(y).(\dropn{y}|Q) | \Sigma_{Q \in P^{\perp}} \quotep{Q}\clift{\Box}
\end{mathpar}

\paragraph{Discussion} The reader will note that $P^{\perp}$ is a
\emph{set} of processes, while $P^{\underline{\perp}}$ is a
\emph{context}. We call the set $P^{\perp}$ the \emph{annihilators} of
$P$. The parallel composition of a process in the annihilators of $P$
with $P$ will result in a process, the state space of which has all
paths eventually leading to $\pzero$. Execution may endure loops; but
under reasonable conditions of fairness (naturally guaranteed under
most notions of bisimulation) such a composite process cannot get
stuck in such a loop and will, eventually pop out and terminate.

The context $P^{\underline{\perp}}$ is ready and willing to ``take the
$P$ out of'' the process to which it is applied. It will effectively
transmit the code of the process to which it is applied to one of the
annihilators and run the process against it.

\subsubsection{Evaluation}
We fix $M$ a domain of fully abstract interpretation with an equality
coincident with bisimulation. We take $\meaningof{\cdot} : \Proc \to
M$ to be the map interpreting processes and $\nmeaningof{\cdot} : \M
\to Proc$ to be the map running the other way. Then we define

\begin{mathpar}
  \int P := \nmeaningof{\meaningof{P}}
\end{mathpar}

\paragraph{Discussion}
There are many fully abstract interpretations of Milner's
$\pi$-calculus. Any of them can be used as a basis for interpreting
the reflective calculus here. Equipped with such a domain it is
largely a matter of grinding through to check that the Yoneda
construction for the normalization-by-evaluation program can be
extended to this setting.

\begin{remark}
  The reader is invited to verify that $\int (P^{\underline{\perp}}[P]) = 0$.
\end{remark}

\subsection{Quantum mechanics}

Table \ref{tbl:core_qm_op_defns} gives the core operational definitions

\begin{table}[htp]\label{tbl:core_qm_op_defns}
  \center{
    \fbox{
      \begin{tabular}{c|c}
        quantum mechanics & process calculus \\
        \hline
        scalar & $x := \quotep{P}$ \\
        state vector & $\state{P} := P$ \\
        dual & $\state{P}^{*} := \event{P^{\underline{\perp}}} := \quotep{P^{\underline{\perp}}}[-]$ \\
        matrix & $ \Sigma_{\alpha} \state{P_{\alpha}}x_{\alpha}\event{Q_{\alpha}}$ \\
        vector addition & $\state{P} + \state{Q} := \state{P | Q}$ \\
        tensor product & $\state{P} \otimes \state{Q} := \state{P \otimes Q}$ \\
        inner product & $\innerprod{P}{Q} := \quotep{\int P^{\underline{\perp}}[Q]}$ \\
      \end{tabular}
    }
  }
  \caption{QM - operational definitions}
\end{table}

where

\begin{mathpar}
  \prmatrix{P}{Q} := \fprmatrix{P}{\quotep{\pzero}}{Q}
  \and
  \fprmatrix{P}{x}{Q} := (\state{P},x,\event{Q})
  \and
  (\fprmatrix{P}{x}{Q})(\state{R}) := x \cdot \innerprod{Q}{R} \cdot \state{P}
  \and
  (\fprmatrix{P}{x}{Q})(\event{R}) := x \cdot \innerprod{R}{P} \cdot \event{Q}
\end{mathpar}

\paragraph{Discussion}
As promised: vectors (aka states) are represented as processes; duals
as contextual duals; inner product definition should be compared with
standard inner product definition for ....

\begin{remark}
  Assuming $\int (P^{\underline{\perp}}[P]) = 0$, the reader is
  invited to verify that $(\fprmatrix{P}{x}{P})(\state{P}) = x \cdot \state{P}$.
\end{remark}

\begin{remark}
  The reader is invited to verify that $\innerprod{P}{Q}$ could
  equally well have been written $\quotep{\int \stackrel{\vee}{x}}$
  where $x = \event{P^{\underline{\perp}}}(Q)$.

  One of the motivations for this remark is that there is another way
  to factor these operations. We could package up evaluation in the dual:

  \begin{mathpar}
    \state{P}^{*} := \event{\int P^{\underline{\perp}}} := \quotep{\int P^{\underline{\perp}}}[-]
  \end{mathpar}

  and then have inner product defined by
  
  \begin{mathpar}
    \innerprod{P}{Q} := \event{P}(Q)
  \end{mathpar}

  Hopefully, experience with the calculations will provide guidance on
  the best factoring.
\end{remark}

\begin{remark}
  Assuming $\int (P^{\underline{\perp}}[P]) = 0$, the reader is
  invited to verify that $\forall P,Q. (\prmatrix{0}{Q})(\state{0}) =
  \state{0}$ and dually $(\prmatrix{P}{0})(\event{0}) = \event{0}$.
\end{remark}

\begin{remark}
  i'm a little worried that i don't (yet) have proper support for
  complex conjugacy. But, the observation above may give us a
  clue. According to Abramsky, it must be the case that the scalars
  are iso to the homset of the identity for the tensor -- which the
  observation above characterizes. 

  For now, we will simply bookmark the notion with $\overline{x}$.
\end{remark}

\subsubsection{Adjointness}

We need to give a definition of $(\cdot)^{\dagger}$ for matrices. The
obvious candidate definition is
\begin{mathpar}
(\Sigma_{\alpha}\fprmatrix{P_{\alpha}}{x_{\alpha}}{Q_{\alpha}})^{\dagger}
= \Sigma_{\alpha}\fprmatrix{(Q_{\alpha}^{\underline{\perp}})^{*}}{\overline{x}_{\alpha}}{P_{\alpha}^{\underline{\perp}}} 
\end{mathpar}

But, $(Q_{\alpha}^{\underline{\perp}})^{*}$ requires a name along
which to communicate the process to achieve the context application.

\subsubsection{Basis for a basis}
If processes label states and ``addition'' of states (a.k.a. vector
addition) is interpreted as parallel composition, what corresponds to
notions of linear independence and basis? Here, we recall that Yoshida
has developed a set of \emph{combinators} for an asynchronous verison
of Milner's $\pi$-calculus. These are a finite set of processes such
any process can be expressed as parallel composition of these
combinators together with liberal uses of the new operator and
replication. We can simply give a translation of these into the
present calculus and have reasonable expectation that the property
carries over. That is, that the resultant set allows to express all
processes via parallel composition. Note, however, that there is no
new operator or replication in this calculus. As a result, we expect
that the corresponding set is actually infinite. That is, we expect
that the space is actually infinite dimensional.

\begin{remark}
  The attentive reader may be a bit concerned. Certainly, the
  collection $S$, $K$ and $I$ is a finite set of
  combinators. Shouldn't we expect to see a finite set of combinators
  for an effectively equivalent system? i am very sympathetic to this
  critique and feel it warrants full attention. On the other hand, i
  also have in mind the following analogy. The natural numbers, as a
  monoid under addition, has exactly $1$ generator, while the natural
  numbers, as a monoid under multiplication, has countably many
  generators (the primes). We observe that the application of the
  lambda calculus is much less resource sensitive than the parallel
  composition of the $\pi$-calculus. Could it be the case that we have
  an analogy of the form
  
  \begin{mathpar}
    m + n : MN :: m*n : M|N
  \end{mathpar}

  giving a similar blow up in the set of ``primes''?  This is such a
  wonderful thought that, even if it's not true, i think it's worth
  writing down.
\end{remark}
 

\documentclass[12pt]{llncs}
%\documentclass{jktr}

\usepackage[pdftex]{hyperref}                   
\usepackage {listings}
\usepackage {mathpartir}
\usepackage{bcprules}
%\usepackage{listings}
                       
\usepackage{graphicx} 
%\usepackage[margins=2.5cm,nohead,nofoot]{geometry}
%\usepackage{geometry}
\usepackage{amsfonts}
\usepackage{amstext}
\usepackage{latexsym}
\usepackage{amssymb}
\usepackage{color}


%\include{myPreamble}
\documentclass[12pt]{llncs}
%\documentclass{jktr}

\usepackage[pdftex]{hyperref}                   
\usepackage {listings}
\usepackage {mathpartir}
\usepackage{bcprules}
%\usepackage{listings}
                       
\usepackage{graphicx} 
%\usepackage[margins=2.5cm,nohead,nofoot]{geometry}
%\usepackage{geometry}
\usepackage{amsfonts}
\usepackage{amstext}
\usepackage{latexsym}
\usepackage{amssymb}
\usepackage{color}


%\include{myPreamble}
\include{qm2pi.local} 

%\ifpdf
%\usepackage[pdftex]{graphicx}
%\else
%\usepackage{graphicx}
%\fi

 % \ifpdf
%  \usepackage{pdfsync}
%  \if


%\title{Brief Article}
%\author{David F. Snyder}
%\author{L.G. Meredith}

%\address{Dept. of Math., Texas State University--San Marcos, San Marcos, TX 78666}
       
\pagestyle{empty}


\begin{document}

\lstset{language=[Objective]Caml,frame=shadowbox}

\input{qm2pi.front}

% section front matter (end)

\input{qm2pi.intro} 
 
% section introduction (end)

% \input{qm2pi.knotations} 

% section notation (end)

\input{qm2pi.process.calculi} 

% section concurrent_process_calculi_and_spatial_logics_ (end)
    
%\input{qm2pi.knots2pi} 

%\input{qm2pi.trefoil} 

%\input{qm2pi.mainthm} 

% subsection basic_interpretation (end)

%\input{qm2pi.rho.presentation} 
\subsection{The syntax and semantics of the notation system}\label{sub:the_syntax_and_semantics_of_the_notation_system} % (fold)

We now summarize a technical presentation of the calculus that
embodies our theory of dynamics. The typical presentation of such a
calculus follows the style of giving generators and relations on
them. The grammar, below, describing term constructors, freely
generates the set of processes, $\Proc$. This set is then quotiented
by a relation known as structural congruence and it is over this set
that the notion of dynamics is expressed. This presentation is
essentially that of \cite{MeredithR05} with the addition of
polyadicity and summation. For readability we have relegated some of
the technical subtleties to an appendix.

\subsubsection{Process grammar}\label{subsub:process_grammar}

\begin{mathpar}
  \inferrule* [lab=synchronization] {} {{M} \bc \pzero \;|\; x?F \;|\; x!C }
  \and
  \inferrule* [lab=abstraction] {} {{F} \bc (x)P}
  \and
  \inferrule* [lab=concretion] {} {{C} \bc \langle Q \rangle}
  \and
  \inferrule* [lab=process] {} {{P,Q} \bc M \;| \;P|Q \;|\; @{x}}
  \and
  \inferrule* [lab=name] {} {{x} \bc \quotep{P}}
\end{mathpar} 

Note that $\vec{x}$ (resp. $\vec{P}$) denotes a vector of names
(resp. processes) of length $|\vec{x}|$ (resp. $|\vec{P}|$). We adopt
the following useful abbreviations.

\begin{mathpar}
   x?(\vec{y}).P := x.(\vec{y})P \and  x\clift{\vec{P}} := x.\clift{\vec{P}}
   \and x!(y) := \lift{x}{\dropn{y}}
   \and \Pi_{i=0}^{n-1}P_i := P_0 | \ldots | P_{n-1}
\end{mathpar}

\subsubsection{Structural congruence}

\paragraph{Free and bound names and alpha-equivalence.} At the
core of structural equivalence is alpha-equivalence which identifies
process that are the same up to a change of variable. Formally, we
recognize the distinction between free and bound names. The free names
of a process, $\freenames{P}$, may be calculated recursively as
follows:

\begin{mathpar}
\freenames{\pzero} := \emptyset
  \and \\
  \freenames{x?(y).P} := \{ x \} \cup (\freenames{P} \setminus \{ y \})
  \and 
  \freenames{x!\langle P \rangle} := \{ x \} \cup \{ P \} 
  \and \\
  \freenames{P|Q} := \freenames{P} \cup \freenames{Q}
  \and \\
  \freenames{@{x}} := \{ x \}
\end{mathpar}

$\pi$
$\quotep{\pi}$

$\freenames{-} : \pi \to \mathcal{P}(\quotep{\pi})$

\begin{eqnarray*}
  \freenames{\pzero} & := & \emptyset \\
  \freenames{x?(y).P} & := & \{ x \} \cup (\freenames{P} \setminus \{ y \}) \\
  \freenames{x!\langle P \rangle} & := & \{ x \} \cup \{ P \} \\
  \freenames{P|Q} & := & \freenames{P} \cup \freenames{Q} \\
  \freenames{\dropn{x}} & := & \{ x \}
\end{eqnarray*}

The bound names of a process, $\boundnames{P}$, are those names occurring in $P$
that are not free. For example, in $x?(y).0$, the name $x$ is free, while $y$ is bound.

\begin{mathpar}
  \inferrule* [lab=monoidal-laws] {} { P|Q \equiv Q|P \and P|0 \equiv P \and P|(Q|R) \equiv (P|Q)|R }
\end{mathpar}

\begin{mathpar}
  \inferrule* [lab=alpha-equivalence] {} { (x)P \equiv (y)P\{y/x\} \and y \not\in \freenames{P} }
\end{mathpar}

\begin{definition}
Then two processes, $P,Q$, are alpha-equivalent if $P = Q\{\vec{y}/\vec{x}\}$ for
some $\vec{x} \in \boundnames{Q},\vec{y} \in \boundnames{P}$, where $Q\{\vec{y}/\vec{x}\}$
denotes the capture-avoiding substitution of $\vec{y}$ for $\vec{x}$ in $Q$.
\end{definition}

\begin{definition}
  The {\em structural congruence} \cite{SangiorgiWalker} , $\equiv$,
  between processes is the least congruence containing
  alpha-equivalence, satisfying the abelian monoid laws
  (associativity, commutativity and $\pzero$ as identity) for parallel
  composition $|$ and for summation $+$.
\end{definition}

\subsection{Name equivalence}

We take name equivalence, written $\nameeq$, to be the smallest
equivalence relation generated by the following rules.

\begin{mathpar}
\inferrule*[lab=Quote-drop]
{ }
{ \quotep{@{x}} \nameeq x }

\inferrule*[lab=Struct-equiv]
{ P \scong Q }
{ \quotep{P} \nameeq \quotep{Q} }
\end{mathpar}

The astute reader will have noticed that the mutual recursion of names
and processes imposes a mutual recursion on alpha-equivalence and
structural equivalence via name-equivalence. Fortunately, all of this
works out pleasantly and we may calculate in the natural way, free of
concern. The reader interested in the details is referred to the
appendix \ref{appendix:rho_details}.

\subsection{Substitution}

We use $\Proc$ for the set of processes, $\QProc$ for the set of
names, and $\id{\{}\vec{y} / \vec{x} \id{\}}$ to denote partial maps,
$s : \QProc \rightarrow \QProc$. A map, $s$ lifts, uniquely, to a map
on process terms, $\widehat{s} : \Proc \rightarrow \Proc$ by the
following equations.

\begin{mathpar}
  (0) \psubstp{Q}{P} := 0 \\
  (R \juxtap S) \psubstp{Q}{P}
  :=    
  (R)\psubstp{Q}{P} \juxtap (S) \psubstp{Q}{P} \\
  (x?(y).R) \psubstp{Q}{P}    
  :=    
  (x)\substp{Q}{P} (z)\concat( (R \psubstn{z}{y}) \psubstp{Q}{P} ) \\
  (\lift{x}{R}) \psubstp{Q}{P}  
  :=
  \lift{(x)\substp{Q}{P}}{ R \psubstp{Q}{P} } \\
%   (\dropn{x})  \psubstp{Q}{P}       
%   := 
%   \left\{ 
%     \begin{array}{ccc} 
%       \dropn{\quotep{Q}} & & x \nameeq \quotep{P} \\
%       \dropn{x} & & otherwise \\
%     \end{array}
%   \right. 
  (\dropn{x})  \psubstp{Q}{P}       
  := 
  \left\{ 
    \begin{array}{ccc} 
      Q & & x \nameeq \quotep{P} \\
      \dropn{x} & & otherwise \\
    \end{array}
  \right.
\end{mathpar}
 

where

\begin{eqnarray}
  (x)\id{\{} \lpquote Q \rpquote / \lpquote P \rpquote \id{\}}            = 
  \left\{ 
    \begin{array}{ccc}
      \lpquote Q \rpquote & & x \nameeq \lpquote P \rpquote \\
      x & & otherwise \\
    \end{array}
  \right. \nonumber
\end{eqnarray}

and $z$ is chosen distinct from $\quotep{P}$, $\quotep{Q}$, the free
names in $Q$, and all the names in $R$. Our $\alpha$-equivalence will
be built in the standard way from this substitution.

\begin{remark}\label{rem:no_self_referential_names}
  One consequence of these definitions is that $\forall P. \quotep{P}
  \not\in \freenames{P}$.
\end{remark}

\subsection{ Dynamic quote: an example }

Anticipating something of what's to come, consider applying the
substitution, $\widehat{\id{\{}u / z \id{\}}}$, to the following pair
of processes, $\lift{w}{y!(z)}$ and $w[ \lpquote y!(z) \rpquote ]$.

\begin{eqnarray}
	\lift{w}{y!(z)}\widehat{\id{\{}u / z \id{\}}}
		& = &
		\lift{w}{y!(u)} \nonumber\\
	w[ \lpquote y!(z) \rpquote ] \widehat{ \id{\{}u / z \id{\}} }
		& = &
		w[ \lpquote y!(z) \rpquote ] \nonumber
\end{eqnarray}

Because the body of the process between quotes is impervious to
substitution, we get radically different answers. In fact, by
examining the first process in an input context,
e.g. $x?(z).\lift{w}{y!(z)}$, we see that the process under the lift
operator may be shaped by prefixed inputs binding a name inside it. In
this sense, the lift operator will be seen as a way to dynamically
construct processes before reifying them as names.

Finally equipped with these standard features we can present the
dynamics of the calculus.

\subsubsection{Operational semantics} 

Finally, we introduce the computational dynamics. What marks these
algebras as distinct from other more traditionally studied algebraic
structures, e.g. vector spaces or polynomial rings, is the manner in
which dynamics is captured. In traditional structures, dynamics is typically
expressed through morphisms between such structures, as in linear maps
between vector spaces or morphisms between rings. In algebras
associated with the semantics of computation, the dynamics is
expressed as part of the algebraic structure itself, through a
reduction reduction relation typically denoted by $\red$. Below, we
give a recursive presentation of this relation for the calculus used
in the encoding.

$\red \subseteq \pi \times \pi$
$\red : \pi \to \mathcal{P}(\pi)$

\begin{mathpar}
  \inferrule* [lab=Comm] { \textsf{match}( x_{src}, x_{trgt} ) } { x_{trgt}?(y)P \; | \; x_{src}!\langle {Q} \rangle \red P\{\quotep{Q}/y}\} }
  \and \\
  \inferrule* [lab=Par] {{P} \red {P}'} {{{P} | {Q}} \red {{P}' | {Q}}}
  \and
  \inferrule* [lab=Equiv]{{{P} \scong {P}'} \andalso {{P}' \red {Q}'} \andalso {{Q}' \scong {Q}}}{{P} \red {Q}}
\end{mathpar}

\begin{eqnarray*}
  match_{\equiv} (\quotep{P},\quotep{Q}) & := & P \equiv Q \\
  match_{\dagger}(\quotep{P},\quotep{Q}) & := & \forall R. P|Q \red^{*} R => R \red^{*} 0 \\
  match_{K}(\quotep{P},\quotep{Q}) & := & K \mbox{ for some context } K
\end{eqnarray*}

$u?(x)P | u!\langle Q \rangle \red P\{\quotep{Q}/x\}$

%We write $\wred$ for $\red^*$, and $P\red$ if $\exists Q $ such that $ P \red Q$.
We write $P\red$ if $\exists Q $ such that $ P \red Q$ and $P\not\red$, otherwise.

\section{Replication}

As mentioned before, it is known that replication (and hence
recursion) can be implemented in a higher-order process algebra
\cite{SangiorgiWalker}. As our first example of calculation with the
machinery thus far presented we give the construction explicitly in
the {\rhoc}.

\begin{eqnarray}
	D_{x} & := & \prefix{x}{y}{(\binpar{\outputp{x}{y}}{@{y}})} \nonumber\\
	\bangp_{x}{P} & := & \binpar{{x}!\langle{\binpar{D_{x}}{P}}\rangle}{D_{x}} \nonumber
\end{eqnarray}

\begin{eqnarray}
	\bangp_{x}{P} & & \nonumber\\
	=
	& {x}!\langle{(\prefix{x}{y}{(\outputp{x}{y} | @{y})) | P}}\rangle 
	      | \prefix{x}{y}{(\outputp{x}{y} | @{y})} & \nonumber\\
	\red
	& (\outputp{x}{y} | @{y})\substn{\quotep{(\prefix{x}{y}{(@{y} | \outputp{x}{y})) | P}}}{y} & \nonumber\\
	=
	& \outputp{x}{\quotep{(\prefix{x}{y}{(\outputp{x}{y} | @{y})) | P}}}
	  | {(\prefix{x}{y}{(\outputp{x}{y} | @{y})) | P}} & \nonumber\\
	\red
	& \ldots & \nonumber\\
	\red^*
	& P | P | \ldots & \nonumber
\end{eqnarray}

Of course, this encoding, as an implementation, runs away, unfolding
$\bangp{P}$ eagerly. A lazier and more implementable replication
operator, restricted to input-guarded processes, may be obtained as follows.

\begin{eqnarray}
\bangp{\prefix{u}{v}{P}} 
	:= 
	\binpar{\lift{x}{\prefix{u}{v}{(\binpar{D(x)}{P})}}}{D(x)} \nonumber
\end{eqnarray}

\begin{remark}
  Note that the lazier definition still does not deal with summation
  or mixed summation (i.e. sums over input and output). The reader is
  invited to construct definitions of replication that deal with these
  features. 

  Further, the definitions are parameterized in a name, $x$. Can you,
  gentle reader, make a definition that eliminates this parameter and
  guarantees no accidental interaction between the replication
  machinery and the process being replicated -- i.e. no accidental
  sharing of names used by the process to get its work done and the
  name(s) used by the replication to effect copying. This latter
  revision of the definition of replication is crucial to obtaining
  the expected identity $!!P \sim !P$.
\end{remark}

\begin{remark}\label{rem:paradoxical_combinator}
  The reader familiar with the lambda calculus will have noticed the
  similarity between $D$ and the paradoxical combinator.

  [Ed. note: the existence of this seems to suggest we have to be more
  restrictive on the set of processes and names we admit if we are to
  support no-cloning.]
\end{remark}

\subsubsection{Bisimulation}

The computational dynamics gives rise to another kind of equivalence,
the equivalence of computational behavior. As previously mentioned
this is typically captured \emph{via} some form of bisimulation.

% The notion we use in this paper is weak barbed bisimulation
% \cite{milner91polyadicpi}.

The notion we use in this paper is derived from weak barbed
bisimulation \cite{milner91polyadicpi}. 

\begin{definition}
An \emph{observation relation}, $\downarrow_{\mathcal N}$, over a set
of names, $\mathcal N$, is the smallest relation satisfying the rules
below.

\infrule[Out-barb]{y \in {\mathcal N}, \; x \nameeq y}
		  {\outputp{x}{v} \downarrow_{\mathcal N} x}
\infrule[Par-barb]{\mbox{$P\downarrow_{\mathcal N} x$ or $Q\downarrow_{\mathcal N} x$}}
		  {\binpar{P}{Q} \downarrow_{\mathcal N} x}

We write $P \Downarrow_{\mathcal N} x$ if there is $Q$ such that 
$P \wred Q$ and $Q \downarrow_{\mathcal N} x$.
\end{definition}

\begin{definition}
%\label{def.bbisim}
An  ${\mathcal N}$-\emph{barbed bisimulation} over a set of names, ${\mathcal N}$, is a symmetric binary relation 
${\mathcal S}_{\mathcal N}$ between agents such that $P\rel{S}_{\mathcal N}Q$ implies:
\begin{enumerate}
\item If $P \red P'$ then $Q \wred Q'$ and $P'\rel{S}_{\mathcal N} Q'$.
\item If $P\downarrow_{\mathcal N} x$, then $Q\Downarrow_{\mathcal N} x$.
\end{enumerate}
$P$ is ${\mathcal N}$-barbed bisimilar to $Q$, written
$P \wbbisim_{\mathcal N} Q$, if $P \rel{S}_{\mathcal N} Q$ for some ${\mathcal N}$-barbed bisimulation ${\mathcal S}_{\mathcal N}$.
\end{definition}

$\mathcal{R} \subseteq \pi \times \pi$

$P \mathcal{R} Q => \forall P'. P \red P' \Rightarrow \exists Q'. Q \red Q', P' \mathcal{R} Q'$

$P \vdash x \Rightarrow Q \vdash x$

\begin{mathpar}
  \inferrule*[lab=Out-barb]{x \nameeq y}{{y}!\langle{Q}\rangle \vdash x}
  \and
  \inferrule*[lab=Par-barb]{\mbox{$P\vdash x$ or $Q\vdash x$}}{\binpar{P}{Q} \vdash x}
\end{mathpar}

\subsubsection{Contexts}

One of the principle advantages of computational calculi like the
$\pi$-calculus is a well-defined notion of context,
contextual-equivalence and a correlation between
contextual-equivalence and notions of bisimulation. The notion of
context allows the decomposition of a process into (sub-)process and
its syntactic environment, its context. Thus, a context may be
thought of as a process with a ``hole'' (written $\Box$) in it. The
application of a context $M$ to a process $P$, written $M[P]$, is
tantamount to filling the hole in $M$ with $P$. In this paper we do
not need the full weight of this theory, but do make use of the notion
of context in the proof the main theorem. 

\begin{mathpar}
  \inferrule* [lab=summation] {} {{M_{M},M_{N}} \bc \Box \;|\; x.M_{A} \;|\; M_{M}+M_{N}}
  \and
  \inferrule* [lab=agent] {} {{M_{A}} \bc (\vec{x})M_{P} \;| \; \clift{P_0,\ldots,M_{P},\ldots,P_N}}
  \and \\
  \inferrule* [lab=process] {} {{M_{P}} \bc M_{N} \;| \;P|M_{P} }
\end{mathpar} 

\begin{mathpar}
  \inferrule* [lab=sychronization] {} {M_{N} \bc \Box \;|\; x?M_{F} \;|\; x!M_{C}}
  \and
  \inferrule* [lab=abstraction] {} {{M_{F}} \bc (x)M_{P} }
  \and
  \inferrule* [lab=concretion] {} {{M_{C}} \bc \langle M_{P} \rangle }
  \and \\
  \inferrule* [lab=process] {} {{M_{P}} \bc M_{N} \;| \;P|M_{P} }
\end{mathpar}

\begin{definition}[contextual application] Given a context $M$, and
  process $P$, we define the \emph{contextual application}, $M[P] :=
  M\{P/\Box\}$. That is, the contextual application of M to P is the
  substitution of $P$ for $\Box$ in $M$.
\end{definition}

$\meaningof{-} : L \to \mathcal{P}(\pi)$

\begin{mathpar}
  \inferrule* [lab=collection] {} {\meaningof{true} = \pi, \and \meaningof{~E} = \pi \setminus \meaningof{E}, \and \meaningof{E_{1} \& E_{2}} = \meaningof{E_{1}} \cap \meaningof{E_{2}}}
\end{mathpar}

\begin{mathpar}
  \inferrule* [lab=structure] {} {\meaningof{0} = \{ P \in \pi | P \equiv 0 \}, \and \\ \meaningof{E_1 | E_2} = \{ P \in \pi | P \equiv P_{1} | P_{2}, P_{1} \in \meaningof{E_{1}}, P_{2} \in \meaningof{E_2}\} }
\end{mathpar}

\begin{mathpar}
 \inferrule* [lab=behavior] {} {\meaningof{\langle a?b \rangle E} = \{ P \in \pi | P \equiv Q | u?(y)P', \\ \and \\\\ \and \\ \;\;\; u \in \meaningof{a}, \forall z.P'\{z/y\} \in \meaningof{E\{z/b\}}\}, \and \\ \meaningof{a!E} = \{ P \in \pi | P \equiv Q | x!\langle P' \rangle, x \in \meaningof{a} P' \in \meaningof{E}\} }
\end{mathpar}

\begin{mathpar}
 \inferrule* [lab=nominal] {} {\meaningof{\quotep{E}} = \{ \quotep{P} \in \quotep{\pi} | P \in \meaningof{E} \}, \and \meaningof{\quotep{P}} = \{ \quotep{Q} \in \quotep{\pi} | P \equiv Q \} \and \\ \meaningof{@\quotep{E}} = \{ P \in \pi | P \equiv @x, x \in \meaningof{E} \}}
\end{mathpar}

\begin{eqnarray*}
  \\
  \meaningof{-} : TS \to ST
\end{eqnarray*}

\begin{eqnarray*}
  \\
  L : TS \to ST
\end{eqnarray*}

\begin{eqnarray*}
  \\
  P \models E \iff P \in \meaningof{E}
\end{eqnarray*}

\begin{eqnarray*}
  P \approx_{L} Q \iff \forall E \in L. P \models E \iff Q \models E
\end{eqnarray*}

\begin{eqnarray*}
  P \approx_{K} Q
\end{eqnarray*}

\begin{eqnarray*}
  P \approx Q
\end{eqnarray*}

$\approx_{K} = \approx = \approx_{L}$

\subsubsection{Contextual duality}

Note that contexts extend the quotation operation to a family of
operations from processes to names. Given a context, $M$, we can
define a \emph{nominal context}, $\quotep{M}$ by $\quotep{M}[P] :=
\quotep{M[P]}$. To foreshadow what is to come we observe that these
operations enjoy a duality with processes very much like the duality
between vectors and maps from vectors to scalars.

Further, because the calculus is essentially higher-order, we have a
correspondence between contexts and processes. More specifically,
given a name $x$ and a context $M$ we can construct $M^{*}_{x}$ such
that 

\begin{mathpar}
  M^{*}_{x} | \lift{x}{P} \red M[P]
\end{mathpar}

namely,

\begin{mathpar}
  M^{*}_{x} := x?(u).M[\dropn{u}]
\end{mathpar}

The dependence of $M^{*}_{x}$ on a name makes it an abstraction, 

\begin{mathpar}
  M^{*} := (x)x?(u).M[\dropn{u}]
\end{mathpar}

\subsection{Additional notation}

It will sometimes be convenient to denote the process a name
quotes. We already have the notation $x = \quotep{P}$, but it will be
convenient to introduce an alternate notation, $\procn{x}$, when we
want to emphasize the connection to the use of the name. Note that, by
virtue of name equivalence, $\quotep{\procn{x}} \nameeq x$; so, the
notation is consistent with previous definitions.

Further, because names have structure it is possible to effect
substitutions on the basis of that structure. This means we need to
upgrade our notation for substitutions, which we accomplish by
adapting comprehension notation. Thus,

\begin{mathpar}
  P\{ y / x : x \in S \}
\end{mathpar}

is interpreted to mean the process derived from P by replacing (in a
capture-avoiding manner) each occurrence of $x$ in $S$ by $y$. For example,

\begin{mathpar}
  P\{ \quotep{\procn{x}|\procn{x}} / x : x \in \freenames{P} \}
\end{mathpar}

will replace each (occurrence) of a free name $x$ in $P$ by
$\quotep{\procn{x}|\procn{x}}$.

Also, we will avail ourselves of the notation $x^{L}$ and $x^{R}$ to
denote injections of a name into disjoint copies of the name
space. There are numerous ways to accomplish this. One example can be
found in \cite{MeredithR05}. This notation overloads to vectors of
names: $\vec{x}^{\pi} := (x_{i}^{\pi} \; : \; 0 \leq i < |\vec{x}| )$ where $\pi \in \{L,R\}$.

We also use $P^{\Box} := P|\Box$.

In \cite{MeredithR05} an interpretation of the new operator is
given. It turns out that there are several possible interpretations
all enjoying the requisite algebraic properties of the operator (see
\cite{milner91polyadicpi}). We will therefore make liberal use of
$(\nu\; \vec{x})P$.

% subsection the_syntax_and_semantics_of_the_notation_system (end)   

\input{qm2pi.qmops} 

\input{qm2pi.sterngerlach} 

\input{qm2pi.metric} 

% section concurrent_process_calculi (end)

%\input{qm2pi.proofsketch}

% section proof sketch (end)

%\input{qm2pi.slviaknots} 

% section spatial logic via knots (end)

\input{qm2pi.conclusion}

% section conclusion (end)

%\input{qm2pi.dtcodes} 

% section wiring algorithm (end)

\input{qm2pi.ack} 

% section acknowledgments (end)

\newpage


\bibliographystyle{plain}   
\bibliography{../../biblios/main.bib}

\input{qm2pi.rhodetails}

\end{document}

 

%\ifpdf
%\usepackage[pdftex]{graphicx}
%\else
%\usepackage{graphicx}
%\fi

 % \ifpdf
%  \usepackage{pdfsync}
%  \if


%\title{Brief Article}
%\author{David F. Snyder}
%\author{L.G. Meredith}

%\address{Dept. of Math., Texas State University--San Marcos, San Marcos, TX 78666}
       
\pagestyle{empty}


\begin{document}

\lstset{language=[Objective]Caml,frame=shadowbox}

\documentclass[12pt]{llncs}
%\documentclass{jktr}

\usepackage[pdftex]{hyperref}                   
\usepackage {listings}
\usepackage {mathpartir}
\usepackage{bcprules}
%\usepackage{listings}
                       
\usepackage{graphicx} 
%\usepackage[margins=2.5cm,nohead,nofoot]{geometry}
%\usepackage{geometry}
\usepackage{amsfonts}
\usepackage{amstext}
\usepackage{latexsym}
\usepackage{amssymb}
\usepackage{color}


%\include{myPreamble}
\include{qm2pi.local} 

%\ifpdf
%\usepackage[pdftex]{graphicx}
%\else
%\usepackage{graphicx}
%\fi

 % \ifpdf
%  \usepackage{pdfsync}
%  \if


%\title{Brief Article}
%\author{David F. Snyder}
%\author{L.G. Meredith}

%\address{Dept. of Math., Texas State University--San Marcos, San Marcos, TX 78666}
       
\pagestyle{empty}


\begin{document}

\lstset{language=[Objective]Caml,frame=shadowbox}

\input{qm2pi.front}

% section front matter (end)

\input{qm2pi.intro} 
 
% section introduction (end)

% \input{qm2pi.knotations} 

% section notation (end)

\input{qm2pi.process.calculi} 

% section concurrent_process_calculi_and_spatial_logics_ (end)
    
%\input{qm2pi.knots2pi} 

%\input{qm2pi.trefoil} 

%\input{qm2pi.mainthm} 

% subsection basic_interpretation (end)

%\input{qm2pi.rho.presentation} 
\subsection{The syntax and semantics of the notation system}\label{sub:the_syntax_and_semantics_of_the_notation_system} % (fold)

We now summarize a technical presentation of the calculus that
embodies our theory of dynamics. The typical presentation of such a
calculus follows the style of giving generators and relations on
them. The grammar, below, describing term constructors, freely
generates the set of processes, $\Proc$. This set is then quotiented
by a relation known as structural congruence and it is over this set
that the notion of dynamics is expressed. This presentation is
essentially that of \cite{MeredithR05} with the addition of
polyadicity and summation. For readability we have relegated some of
the technical subtleties to an appendix.

\subsubsection{Process grammar}\label{subsub:process_grammar}

\begin{mathpar}
  \inferrule* [lab=synchronization] {} {{M} \bc \pzero \;|\; x?F \;|\; x!C }
  \and
  \inferrule* [lab=abstraction] {} {{F} \bc (x)P}
  \and
  \inferrule* [lab=concretion] {} {{C} \bc \langle Q \rangle}
  \and
  \inferrule* [lab=process] {} {{P,Q} \bc M \;| \;P|Q \;|\; @{x}}
  \and
  \inferrule* [lab=name] {} {{x} \bc \quotep{P}}
\end{mathpar} 

Note that $\vec{x}$ (resp. $\vec{P}$) denotes a vector of names
(resp. processes) of length $|\vec{x}|$ (resp. $|\vec{P}|$). We adopt
the following useful abbreviations.

\begin{mathpar}
   x?(\vec{y}).P := x.(\vec{y})P \and  x\clift{\vec{P}} := x.\clift{\vec{P}}
   \and x!(y) := \lift{x}{\dropn{y}}
   \and \Pi_{i=0}^{n-1}P_i := P_0 | \ldots | P_{n-1}
\end{mathpar}

\subsubsection{Structural congruence}

\paragraph{Free and bound names and alpha-equivalence.} At the
core of structural equivalence is alpha-equivalence which identifies
process that are the same up to a change of variable. Formally, we
recognize the distinction between free and bound names. The free names
of a process, $\freenames{P}$, may be calculated recursively as
follows:

\begin{mathpar}
\freenames{\pzero} := \emptyset
  \and \\
  \freenames{x?(y).P} := \{ x \} \cup (\freenames{P} \setminus \{ y \})
  \and 
  \freenames{x!\langle P \rangle} := \{ x \} \cup \{ P \} 
  \and \\
  \freenames{P|Q} := \freenames{P} \cup \freenames{Q}
  \and \\
  \freenames{@{x}} := \{ x \}
\end{mathpar}

$\pi$
$\quotep{\pi}$

$\freenames{-} : \pi \to \mathcal{P}(\quotep{\pi})$

\begin{eqnarray*}
  \freenames{\pzero} & := & \emptyset \\
  \freenames{x?(y).P} & := & \{ x \} \cup (\freenames{P} \setminus \{ y \}) \\
  \freenames{x!\langle P \rangle} & := & \{ x \} \cup \{ P \} \\
  \freenames{P|Q} & := & \freenames{P} \cup \freenames{Q} \\
  \freenames{\dropn{x}} & := & \{ x \}
\end{eqnarray*}

The bound names of a process, $\boundnames{P}$, are those names occurring in $P$
that are not free. For example, in $x?(y).0$, the name $x$ is free, while $y$ is bound.

\begin{mathpar}
  \inferrule* [lab=monoidal-laws] {} { P|Q \equiv Q|P \and P|0 \equiv P \and P|(Q|R) \equiv (P|Q)|R }
\end{mathpar}

\begin{mathpar}
  \inferrule* [lab=alpha-equivalence] {} { (x)P \equiv (y)P\{y/x\} \and y \not\in \freenames{P} }
\end{mathpar}

\begin{definition}
Then two processes, $P,Q$, are alpha-equivalent if $P = Q\{\vec{y}/\vec{x}\}$ for
some $\vec{x} \in \boundnames{Q},\vec{y} \in \boundnames{P}$, where $Q\{\vec{y}/\vec{x}\}$
denotes the capture-avoiding substitution of $\vec{y}$ for $\vec{x}$ in $Q$.
\end{definition}

\begin{definition}
  The {\em structural congruence} \cite{SangiorgiWalker} , $\equiv$,
  between processes is the least congruence containing
  alpha-equivalence, satisfying the abelian monoid laws
  (associativity, commutativity and $\pzero$ as identity) for parallel
  composition $|$ and for summation $+$.
\end{definition}

\subsection{Name equivalence}

We take name equivalence, written $\nameeq$, to be the smallest
equivalence relation generated by the following rules.

\begin{mathpar}
\inferrule*[lab=Quote-drop]
{ }
{ \quotep{@{x}} \nameeq x }

\inferrule*[lab=Struct-equiv]
{ P \scong Q }
{ \quotep{P} \nameeq \quotep{Q} }
\end{mathpar}

The astute reader will have noticed that the mutual recursion of names
and processes imposes a mutual recursion on alpha-equivalence and
structural equivalence via name-equivalence. Fortunately, all of this
works out pleasantly and we may calculate in the natural way, free of
concern. The reader interested in the details is referred to the
appendix \ref{appendix:rho_details}.

\subsection{Substitution}

We use $\Proc$ for the set of processes, $\QProc$ for the set of
names, and $\id{\{}\vec{y} / \vec{x} \id{\}}$ to denote partial maps,
$s : \QProc \rightarrow \QProc$. A map, $s$ lifts, uniquely, to a map
on process terms, $\widehat{s} : \Proc \rightarrow \Proc$ by the
following equations.

\begin{mathpar}
  (0) \psubstp{Q}{P} := 0 \\
  (R \juxtap S) \psubstp{Q}{P}
  :=    
  (R)\psubstp{Q}{P} \juxtap (S) \psubstp{Q}{P} \\
  (x?(y).R) \psubstp{Q}{P}    
  :=    
  (x)\substp{Q}{P} (z)\concat( (R \psubstn{z}{y}) \psubstp{Q}{P} ) \\
  (\lift{x}{R}) \psubstp{Q}{P}  
  :=
  \lift{(x)\substp{Q}{P}}{ R \psubstp{Q}{P} } \\
%   (\dropn{x})  \psubstp{Q}{P}       
%   := 
%   \left\{ 
%     \begin{array}{ccc} 
%       \dropn{\quotep{Q}} & & x \nameeq \quotep{P} \\
%       \dropn{x} & & otherwise \\
%     \end{array}
%   \right. 
  (\dropn{x})  \psubstp{Q}{P}       
  := 
  \left\{ 
    \begin{array}{ccc} 
      Q & & x \nameeq \quotep{P} \\
      \dropn{x} & & otherwise \\
    \end{array}
  \right.
\end{mathpar}
 

where

\begin{eqnarray}
  (x)\id{\{} \lpquote Q \rpquote / \lpquote P \rpquote \id{\}}            = 
  \left\{ 
    \begin{array}{ccc}
      \lpquote Q \rpquote & & x \nameeq \lpquote P \rpquote \\
      x & & otherwise \\
    \end{array}
  \right. \nonumber
\end{eqnarray}

and $z$ is chosen distinct from $\quotep{P}$, $\quotep{Q}$, the free
names in $Q$, and all the names in $R$. Our $\alpha$-equivalence will
be built in the standard way from this substitution.

\begin{remark}\label{rem:no_self_referential_names}
  One consequence of these definitions is that $\forall P. \quotep{P}
  \not\in \freenames{P}$.
\end{remark}

\subsection{ Dynamic quote: an example }

Anticipating something of what's to come, consider applying the
substitution, $\widehat{\id{\{}u / z \id{\}}}$, to the following pair
of processes, $\lift{w}{y!(z)}$ and $w[ \lpquote y!(z) \rpquote ]$.

\begin{eqnarray}
	\lift{w}{y!(z)}\widehat{\id{\{}u / z \id{\}}}
		& = &
		\lift{w}{y!(u)} \nonumber\\
	w[ \lpquote y!(z) \rpquote ] \widehat{ \id{\{}u / z \id{\}} }
		& = &
		w[ \lpquote y!(z) \rpquote ] \nonumber
\end{eqnarray}

Because the body of the process between quotes is impervious to
substitution, we get radically different answers. In fact, by
examining the first process in an input context,
e.g. $x?(z).\lift{w}{y!(z)}$, we see that the process under the lift
operator may be shaped by prefixed inputs binding a name inside it. In
this sense, the lift operator will be seen as a way to dynamically
construct processes before reifying them as names.

Finally equipped with these standard features we can present the
dynamics of the calculus.

\subsubsection{Operational semantics} 

Finally, we introduce the computational dynamics. What marks these
algebras as distinct from other more traditionally studied algebraic
structures, e.g. vector spaces or polynomial rings, is the manner in
which dynamics is captured. In traditional structures, dynamics is typically
expressed through morphisms between such structures, as in linear maps
between vector spaces or morphisms between rings. In algebras
associated with the semantics of computation, the dynamics is
expressed as part of the algebraic structure itself, through a
reduction reduction relation typically denoted by $\red$. Below, we
give a recursive presentation of this relation for the calculus used
in the encoding.

$\red \subseteq \pi \times \pi$
$\red : \pi \to \mathcal{P}(\pi)$

\begin{mathpar}
  \inferrule* [lab=Comm] { \textsf{match}( x_{src}, x_{trgt} ) } { x_{trgt}?(y)P \; | \; x_{src}!\langle {Q} \rangle \red P\{\quotep{Q}/y}\} }
  \and \\
  \inferrule* [lab=Par] {{P} \red {P}'} {{{P} | {Q}} \red {{P}' | {Q}}}
  \and
  \inferrule* [lab=Equiv]{{{P} \scong {P}'} \andalso {{P}' \red {Q}'} \andalso {{Q}' \scong {Q}}}{{P} \red {Q}}
\end{mathpar}

\begin{eqnarray*}
  match_{\equiv} (\quotep{P},\quotep{Q}) & := & P \equiv Q \\
  match_{\dagger}(\quotep{P},\quotep{Q}) & := & \forall R. P|Q \red^{*} R => R \red^{*} 0 \\
  match_{K}(\quotep{P},\quotep{Q}) & := & K \mbox{ for some context } K
\end{eqnarray*}

$u?(x)P | u!\langle Q \rangle \red P\{\quotep{Q}/x\}$

%We write $\wred$ for $\red^*$, and $P\red$ if $\exists Q $ such that $ P \red Q$.
We write $P\red$ if $\exists Q $ such that $ P \red Q$ and $P\not\red$, otherwise.

\section{Replication}

As mentioned before, it is known that replication (and hence
recursion) can be implemented in a higher-order process algebra
\cite{SangiorgiWalker}. As our first example of calculation with the
machinery thus far presented we give the construction explicitly in
the {\rhoc}.

\begin{eqnarray}
	D_{x} & := & \prefix{x}{y}{(\binpar{\outputp{x}{y}}{@{y}})} \nonumber\\
	\bangp_{x}{P} & := & \binpar{{x}!\langle{\binpar{D_{x}}{P}}\rangle}{D_{x}} \nonumber
\end{eqnarray}

\begin{eqnarray}
	\bangp_{x}{P} & & \nonumber\\
	=
	& {x}!\langle{(\prefix{x}{y}{(\outputp{x}{y} | @{y})) | P}}\rangle 
	      | \prefix{x}{y}{(\outputp{x}{y} | @{y})} & \nonumber\\
	\red
	& (\outputp{x}{y} | @{y})\substn{\quotep{(\prefix{x}{y}{(@{y} | \outputp{x}{y})) | P}}}{y} & \nonumber\\
	=
	& \outputp{x}{\quotep{(\prefix{x}{y}{(\outputp{x}{y} | @{y})) | P}}}
	  | {(\prefix{x}{y}{(\outputp{x}{y} | @{y})) | P}} & \nonumber\\
	\red
	& \ldots & \nonumber\\
	\red^*
	& P | P | \ldots & \nonumber
\end{eqnarray}

Of course, this encoding, as an implementation, runs away, unfolding
$\bangp{P}$ eagerly. A lazier and more implementable replication
operator, restricted to input-guarded processes, may be obtained as follows.

\begin{eqnarray}
\bangp{\prefix{u}{v}{P}} 
	:= 
	\binpar{\lift{x}{\prefix{u}{v}{(\binpar{D(x)}{P})}}}{D(x)} \nonumber
\end{eqnarray}

\begin{remark}
  Note that the lazier definition still does not deal with summation
  or mixed summation (i.e. sums over input and output). The reader is
  invited to construct definitions of replication that deal with these
  features. 

  Further, the definitions are parameterized in a name, $x$. Can you,
  gentle reader, make a definition that eliminates this parameter and
  guarantees no accidental interaction between the replication
  machinery and the process being replicated -- i.e. no accidental
  sharing of names used by the process to get its work done and the
  name(s) used by the replication to effect copying. This latter
  revision of the definition of replication is crucial to obtaining
  the expected identity $!!P \sim !P$.
\end{remark}

\begin{remark}\label{rem:paradoxical_combinator}
  The reader familiar with the lambda calculus will have noticed the
  similarity between $D$ and the paradoxical combinator.

  [Ed. note: the existence of this seems to suggest we have to be more
  restrictive on the set of processes and names we admit if we are to
  support no-cloning.]
\end{remark}

\subsubsection{Bisimulation}

The computational dynamics gives rise to another kind of equivalence,
the equivalence of computational behavior. As previously mentioned
this is typically captured \emph{via} some form of bisimulation.

% The notion we use in this paper is weak barbed bisimulation
% \cite{milner91polyadicpi}.

The notion we use in this paper is derived from weak barbed
bisimulation \cite{milner91polyadicpi}. 

\begin{definition}
An \emph{observation relation}, $\downarrow_{\mathcal N}$, over a set
of names, $\mathcal N$, is the smallest relation satisfying the rules
below.

\infrule[Out-barb]{y \in {\mathcal N}, \; x \nameeq y}
		  {\outputp{x}{v} \downarrow_{\mathcal N} x}
\infrule[Par-barb]{\mbox{$P\downarrow_{\mathcal N} x$ or $Q\downarrow_{\mathcal N} x$}}
		  {\binpar{P}{Q} \downarrow_{\mathcal N} x}

We write $P \Downarrow_{\mathcal N} x$ if there is $Q$ such that 
$P \wred Q$ and $Q \downarrow_{\mathcal N} x$.
\end{definition}

\begin{definition}
%\label{def.bbisim}
An  ${\mathcal N}$-\emph{barbed bisimulation} over a set of names, ${\mathcal N}$, is a symmetric binary relation 
${\mathcal S}_{\mathcal N}$ between agents such that $P\rel{S}_{\mathcal N}Q$ implies:
\begin{enumerate}
\item If $P \red P'$ then $Q \wred Q'$ and $P'\rel{S}_{\mathcal N} Q'$.
\item If $P\downarrow_{\mathcal N} x$, then $Q\Downarrow_{\mathcal N} x$.
\end{enumerate}
$P$ is ${\mathcal N}$-barbed bisimilar to $Q$, written
$P \wbbisim_{\mathcal N} Q$, if $P \rel{S}_{\mathcal N} Q$ for some ${\mathcal N}$-barbed bisimulation ${\mathcal S}_{\mathcal N}$.
\end{definition}

$\mathcal{R} \subseteq \pi \times \pi$

$P \mathcal{R} Q => \forall P'. P \red P' \Rightarrow \exists Q'. Q \red Q', P' \mathcal{R} Q'$

$P \vdash x \Rightarrow Q \vdash x$

\begin{mathpar}
  \inferrule*[lab=Out-barb]{x \nameeq y}{{y}!\langle{Q}\rangle \vdash x}
  \and
  \inferrule*[lab=Par-barb]{\mbox{$P\vdash x$ or $Q\vdash x$}}{\binpar{P}{Q} \vdash x}
\end{mathpar}

\subsubsection{Contexts}

One of the principle advantages of computational calculi like the
$\pi$-calculus is a well-defined notion of context,
contextual-equivalence and a correlation between
contextual-equivalence and notions of bisimulation. The notion of
context allows the decomposition of a process into (sub-)process and
its syntactic environment, its context. Thus, a context may be
thought of as a process with a ``hole'' (written $\Box$) in it. The
application of a context $M$ to a process $P$, written $M[P]$, is
tantamount to filling the hole in $M$ with $P$. In this paper we do
not need the full weight of this theory, but do make use of the notion
of context in the proof the main theorem. 

\begin{mathpar}
  \inferrule* [lab=summation] {} {{M_{M},M_{N}} \bc \Box \;|\; x.M_{A} \;|\; M_{M}+M_{N}}
  \and
  \inferrule* [lab=agent] {} {{M_{A}} \bc (\vec{x})M_{P} \;| \; \clift{P_0,\ldots,M_{P},\ldots,P_N}}
  \and \\
  \inferrule* [lab=process] {} {{M_{P}} \bc M_{N} \;| \;P|M_{P} }
\end{mathpar} 

\begin{mathpar}
  \inferrule* [lab=sychronization] {} {M_{N} \bc \Box \;|\; x?M_{F} \;|\; x!M_{C}}
  \and
  \inferrule* [lab=abstraction] {} {{M_{F}} \bc (x)M_{P} }
  \and
  \inferrule* [lab=concretion] {} {{M_{C}} \bc \langle M_{P} \rangle }
  \and \\
  \inferrule* [lab=process] {} {{M_{P}} \bc M_{N} \;| \;P|M_{P} }
\end{mathpar}

\begin{definition}[contextual application] Given a context $M$, and
  process $P$, we define the \emph{contextual application}, $M[P] :=
  M\{P/\Box\}$. That is, the contextual application of M to P is the
  substitution of $P$ for $\Box$ in $M$.
\end{definition}

$\meaningof{-} : L \to \mathcal{P}(\pi)$

\begin{mathpar}
  \inferrule* [lab=collection] {} {\meaningof{true} = \pi, \and \meaningof{~E} = \pi \setminus \meaningof{E}, \and \meaningof{E_{1} \& E_{2}} = \meaningof{E_{1}} \cap \meaningof{E_{2}}}
\end{mathpar}

\begin{mathpar}
  \inferrule* [lab=structure] {} {\meaningof{0} = \{ P \in \pi | P \equiv 0 \}, \and \\ \meaningof{E_1 | E_2} = \{ P \in \pi | P \equiv P_{1} | P_{2}, P_{1} \in \meaningof{E_{1}}, P_{2} \in \meaningof{E_2}\} }
\end{mathpar}

\begin{mathpar}
 \inferrule* [lab=behavior] {} {\meaningof{\langle a?b \rangle E} = \{ P \in \pi | P \equiv Q | u?(y)P', \\ \and \\\\ \and \\ \;\;\; u \in \meaningof{a}, \forall z.P'\{z/y\} \in \meaningof{E\{z/b\}}\}, \and \\ \meaningof{a!E} = \{ P \in \pi | P \equiv Q | x!\langle P' \rangle, x \in \meaningof{a} P' \in \meaningof{E}\} }
\end{mathpar}

\begin{mathpar}
 \inferrule* [lab=nominal] {} {\meaningof{\quotep{E}} = \{ \quotep{P} \in \quotep{\pi} | P \in \meaningof{E} \}, \and \meaningof{\quotep{P}} = \{ \quotep{Q} \in \quotep{\pi} | P \equiv Q \} \and \\ \meaningof{@\quotep{E}} = \{ P \in \pi | P \equiv @x, x \in \meaningof{E} \}}
\end{mathpar}

\begin{eqnarray*}
  \\
  \meaningof{-} : TS \to ST
\end{eqnarray*}

\begin{eqnarray*}
  \\
  L : TS \to ST
\end{eqnarray*}

\begin{eqnarray*}
  \\
  P \models E \iff P \in \meaningof{E}
\end{eqnarray*}

\begin{eqnarray*}
  P \approx_{L} Q \iff \forall E \in L. P \models E \iff Q \models E
\end{eqnarray*}

\begin{eqnarray*}
  P \approx_{K} Q
\end{eqnarray*}

\begin{eqnarray*}
  P \approx Q
\end{eqnarray*}

$\approx_{K} = \approx = \approx_{L}$

\subsubsection{Contextual duality}

Note that contexts extend the quotation operation to a family of
operations from processes to names. Given a context, $M$, we can
define a \emph{nominal context}, $\quotep{M}$ by $\quotep{M}[P] :=
\quotep{M[P]}$. To foreshadow what is to come we observe that these
operations enjoy a duality with processes very much like the duality
between vectors and maps from vectors to scalars.

Further, because the calculus is essentially higher-order, we have a
correspondence between contexts and processes. More specifically,
given a name $x$ and a context $M$ we can construct $M^{*}_{x}$ such
that 

\begin{mathpar}
  M^{*}_{x} | \lift{x}{P} \red M[P]
\end{mathpar}

namely,

\begin{mathpar}
  M^{*}_{x} := x?(u).M[\dropn{u}]
\end{mathpar}

The dependence of $M^{*}_{x}$ on a name makes it an abstraction, 

\begin{mathpar}
  M^{*} := (x)x?(u).M[\dropn{u}]
\end{mathpar}

\subsection{Additional notation}

It will sometimes be convenient to denote the process a name
quotes. We already have the notation $x = \quotep{P}$, but it will be
convenient to introduce an alternate notation, $\procn{x}$, when we
want to emphasize the connection to the use of the name. Note that, by
virtue of name equivalence, $\quotep{\procn{x}} \nameeq x$; so, the
notation is consistent with previous definitions.

Further, because names have structure it is possible to effect
substitutions on the basis of that structure. This means we need to
upgrade our notation for substitutions, which we accomplish by
adapting comprehension notation. Thus,

\begin{mathpar}
  P\{ y / x : x \in S \}
\end{mathpar}

is interpreted to mean the process derived from P by replacing (in a
capture-avoiding manner) each occurrence of $x$ in $S$ by $y$. For example,

\begin{mathpar}
  P\{ \quotep{\procn{x}|\procn{x}} / x : x \in \freenames{P} \}
\end{mathpar}

will replace each (occurrence) of a free name $x$ in $P$ by
$\quotep{\procn{x}|\procn{x}}$.

Also, we will avail ourselves of the notation $x^{L}$ and $x^{R}$ to
denote injections of a name into disjoint copies of the name
space. There are numerous ways to accomplish this. One example can be
found in \cite{MeredithR05}. This notation overloads to vectors of
names: $\vec{x}^{\pi} := (x_{i}^{\pi} \; : \; 0 \leq i < |\vec{x}| )$ where $\pi \in \{L,R\}$.

We also use $P^{\Box} := P|\Box$.

In \cite{MeredithR05} an interpretation of the new operator is
given. It turns out that there are several possible interpretations
all enjoying the requisite algebraic properties of the operator (see
\cite{milner91polyadicpi}). We will therefore make liberal use of
$(\nu\; \vec{x})P$.

% subsection the_syntax_and_semantics_of_the_notation_system (end)   

\input{qm2pi.qmops} 

\input{qm2pi.sterngerlach} 

\input{qm2pi.metric} 

% section concurrent_process_calculi (end)

%\input{qm2pi.proofsketch}

% section proof sketch (end)

%\input{qm2pi.slviaknots} 

% section spatial logic via knots (end)

\input{qm2pi.conclusion}

% section conclusion (end)

%\input{qm2pi.dtcodes} 

% section wiring algorithm (end)

\input{qm2pi.ack} 

% section acknowledgments (end)

\newpage


\bibliographystyle{plain}   
\bibliography{../../biblios/main.bib}

\input{qm2pi.rhodetails}

\end{document}



% section front matter (end)

\section{Introduction}\label{sec:introduction} % (fold)
In this draft of the material i am going to have to dispense with the
usual writing conventions adopted in papers on these topics. i'm going
to have adopt whatever tone i need at the time i'm writing up the
calculations. Sometimes this may be very conversational; others it may
be the barest mathematical grunts; others still it may be that i have
lifted text from one of my other papers because the exposition of some
point was better said there. i hope that my readers are not unduly put
out by this decision. i'm not doing this to flout convention or be
rebellious. i find these calculations very technically challenging. To
keep everything going technically, something has to give; i have to
let go of some cognitive burden. So, the academic writing style --
with all of its trade-offs in terms of facilitating technical
communication -- is what i'm letting go of. Perhaps subsequent drafts
can be tightened and polished, but for now, i'm going to speak as if
we were sitting together in a coffee shop with a laptop, wifi and a
pad of paper and a pencil.

So, here's what i have to say. We -- you and i, comfortably ensconced
in our coffee shop and well-equipped with our tools -- can realize and
carry out the calculations of quantum mechanics over a very different
formal theory of dynamics, a formal theory of dynamics that
corresponds to a theory of concurrent computation with
\emph{reflection}. It has the advantage that the underlying theory is
already `quantized', but supports analogues all of the continuuous
operations. Strikingly, this underlying theory has recently been
connected with a notion of metric that we can show, by calculating
together, coincides with the metric induced by the inner product.

There are a lot of reasons why you might be interested in seeing
calculations of this form. Here's why i'm interested. For the past
several centuries there has been no competitor to the ``Newtonian''
account of dynamics. As a result the predominant share of accounts of
dynamical systems and situations have had to be formulated in terms of
the Newtonian machinery. i view this as an intellectually dangerous
position to occupy. Everything, despite it's intrinsic shape, turns
into a nail to be hit with this hammer. Recently, however, the theory
of computation has matured to the point where we have candidates for
theories of dynamics that offer very different perspective on
reasoning about dynamical systems and situations. Testing these
candidates against very successful accounts of dynamical situations,
like quantum mechanics, is going to give us some sense of how mature
they are and some measure of the quality of these accounts of
dynamics.

\subsection{Summary of contributions and outline of paper}

So, we're going to develop an interpretation of the operations of
quantum mechanics normally interpreted by Hilbert spaces and
operators. We're going to do this over a theory of computation. Note
that this is very different than the usual quantum computation program
which develops notions of computation over quantum mechanics. Rather,
we are developing a story that aligns with Wheeler's slogan: It from
Bit. To do this we will first provide an account of the theory of
computation at play here. Then we will dive into a calculation-driven
interpretation of the operations of quantum mechanics.

The reason we take this approach is that -- until very recently --
there hasn't been an axiomatic account of quantum mechanics. As a
result there has been no sharp delineation of the mathematical theory
supporting interpretation of the physical theory and the physical
theory, itself. So, ambient features of the maths are free to be
exploited (or supressed) without a real accounting of their physical
relevance. There is no sharp statement ``here's the physical theory''
qua \emph{theory} and ``here's the mathematical interpretation''
enabling a judgment of how faithful the interpretation is -- apart
from experimental observation. When there is an axiomatic account we
can judge how well a given mathematical formalism supports an
interpretation of the axioms, independent of
experimentation. Likewise, we can judge how well we have captured our
physical evidence and experience with our axiomatics, independent of
any specific mathematical implementation, with accidental detail that
may or may not have physical significance. 

In lieu of a fully fleshed out and vetted axiomatic account of quantum
mechanics, interpreting the operational notions in service of modeling
physical systems will have to suffice. In other words, we are not in
the business of providing a model of Hilbert spaces and operators. We
are in the business of providing a model of quantum mechanics because
we are motivated by testing our notions of dynamics against physical
theory; and, the predictive calculations of the physical theory must
serve as the best formulation -- shy of a fully fleshed out axiomatic
account -- of the physical theory itself (as they have for scientific
theories since time immemorial). Put another way, despite a
whole-hearted commitment to an It-from-Bit ontology, we are firmly
aligned with the shut-up-and-calculate camp as the best way to obtain
results either from the physical perspective or as a quality assurance
measure of our fledgling theory of dynamics.

In detail, we present a reflective process calculus. Then we develop
intuitive correspondences between the notions available in this
calculus and the usual physical notions supporting quantum mechanical
calculations. Thus, 

\begin{table}[htp]
  \center{
    \fbox{
      \begin{tabular}{c|c}
        quantum mechanics & process calculus \\
        \hline
        scalar & name \\
        state vector & process \\
        dual & contextual duals \\
        matrix & formal sums of process-context-dual pairs \\
        orthogonality & process annihilation \\
        inner product & execution-formula + quoting
      \end{tabular}
    }
  }
  \caption{QM - process calculi correspondences}
\end{table}

Then we tighten up these intuitions to operational definitions. We
employ the Dirac notation as the best proxy we can find for an
abstract syntax of the quantum mechanical notions. The definitions we
develop put us in contact with equational constraints coming from the
theory that we demonstrate the definitions and calculations satisfy.

This puts us in a position to shut up and calculate for the
Stern-Gerlach experimental set up, showing how these predictive
calculations become calculations on processes in our theory of a
reflective process calculus.

Penultimately, we demonstrate that the notion of metric coming from
the inner product coincides with the notion of metric available from
the theory of bisimulation. This demonstration gives us the right to
think of space as arising from behavior. Finally, we consider where we
might go from the new vantage point we have obtained.

% section introduction (end) 
 
% section introduction (end)

% \documentclass[12pt]{llncs}
%\documentclass{jktr}

\usepackage[pdftex]{hyperref}                   
\usepackage {listings}
\usepackage {mathpartir}
\usepackage{bcprules}
%\usepackage{listings}
                       
\usepackage{graphicx} 
%\usepackage[margins=2.5cm,nohead,nofoot]{geometry}
%\usepackage{geometry}
\usepackage{amsfonts}
\usepackage{amstext}
\usepackage{latexsym}
\usepackage{amssymb}
\usepackage{color}


%\include{myPreamble}
\include{qm2pi.local} 

%\ifpdf
%\usepackage[pdftex]{graphicx}
%\else
%\usepackage{graphicx}
%\fi

 % \ifpdf
%  \usepackage{pdfsync}
%  \if


%\title{Brief Article}
%\author{David F. Snyder}
%\author{L.G. Meredith}

%\address{Dept. of Math., Texas State University--San Marcos, San Marcos, TX 78666}
       
\pagestyle{empty}


\begin{document}

\lstset{language=[Objective]Caml,frame=shadowbox}

\input{qm2pi.front}

% section front matter (end)

\input{qm2pi.intro} 
 
% section introduction (end)

% \input{qm2pi.knotations} 

% section notation (end)

\input{qm2pi.process.calculi} 

% section concurrent_process_calculi_and_spatial_logics_ (end)
    
%\input{qm2pi.knots2pi} 

%\input{qm2pi.trefoil} 

%\input{qm2pi.mainthm} 

% subsection basic_interpretation (end)

%\input{qm2pi.rho.presentation} 
\subsection{The syntax and semantics of the notation system}\label{sub:the_syntax_and_semantics_of_the_notation_system} % (fold)

We now summarize a technical presentation of the calculus that
embodies our theory of dynamics. The typical presentation of such a
calculus follows the style of giving generators and relations on
them. The grammar, below, describing term constructors, freely
generates the set of processes, $\Proc$. This set is then quotiented
by a relation known as structural congruence and it is over this set
that the notion of dynamics is expressed. This presentation is
essentially that of \cite{MeredithR05} with the addition of
polyadicity and summation. For readability we have relegated some of
the technical subtleties to an appendix.

\subsubsection{Process grammar}\label{subsub:process_grammar}

\begin{mathpar}
  \inferrule* [lab=synchronization] {} {{M} \bc \pzero \;|\; x?F \;|\; x!C }
  \and
  \inferrule* [lab=abstraction] {} {{F} \bc (x)P}
  \and
  \inferrule* [lab=concretion] {} {{C} \bc \langle Q \rangle}
  \and
  \inferrule* [lab=process] {} {{P,Q} \bc M \;| \;P|Q \;|\; @{x}}
  \and
  \inferrule* [lab=name] {} {{x} \bc \quotep{P}}
\end{mathpar} 

Note that $\vec{x}$ (resp. $\vec{P}$) denotes a vector of names
(resp. processes) of length $|\vec{x}|$ (resp. $|\vec{P}|$). We adopt
the following useful abbreviations.

\begin{mathpar}
   x?(\vec{y}).P := x.(\vec{y})P \and  x\clift{\vec{P}} := x.\clift{\vec{P}}
   \and x!(y) := \lift{x}{\dropn{y}}
   \and \Pi_{i=0}^{n-1}P_i := P_0 | \ldots | P_{n-1}
\end{mathpar}

\subsubsection{Structural congruence}

\paragraph{Free and bound names and alpha-equivalence.} At the
core of structural equivalence is alpha-equivalence which identifies
process that are the same up to a change of variable. Formally, we
recognize the distinction between free and bound names. The free names
of a process, $\freenames{P}$, may be calculated recursively as
follows:

\begin{mathpar}
\freenames{\pzero} := \emptyset
  \and \\
  \freenames{x?(y).P} := \{ x \} \cup (\freenames{P} \setminus \{ y \})
  \and 
  \freenames{x!\langle P \rangle} := \{ x \} \cup \{ P \} 
  \and \\
  \freenames{P|Q} := \freenames{P} \cup \freenames{Q}
  \and \\
  \freenames{@{x}} := \{ x \}
\end{mathpar}

$\pi$
$\quotep{\pi}$

$\freenames{-} : \pi \to \mathcal{P}(\quotep{\pi})$

\begin{eqnarray*}
  \freenames{\pzero} & := & \emptyset \\
  \freenames{x?(y).P} & := & \{ x \} \cup (\freenames{P} \setminus \{ y \}) \\
  \freenames{x!\langle P \rangle} & := & \{ x \} \cup \{ P \} \\
  \freenames{P|Q} & := & \freenames{P} \cup \freenames{Q} \\
  \freenames{\dropn{x}} & := & \{ x \}
\end{eqnarray*}

The bound names of a process, $\boundnames{P}$, are those names occurring in $P$
that are not free. For example, in $x?(y).0$, the name $x$ is free, while $y$ is bound.

\begin{mathpar}
  \inferrule* [lab=monoidal-laws] {} { P|Q \equiv Q|P \and P|0 \equiv P \and P|(Q|R) \equiv (P|Q)|R }
\end{mathpar}

\begin{mathpar}
  \inferrule* [lab=alpha-equivalence] {} { (x)P \equiv (y)P\{y/x\} \and y \not\in \freenames{P} }
\end{mathpar}

\begin{definition}
Then two processes, $P,Q$, are alpha-equivalent if $P = Q\{\vec{y}/\vec{x}\}$ for
some $\vec{x} \in \boundnames{Q},\vec{y} \in \boundnames{P}$, where $Q\{\vec{y}/\vec{x}\}$
denotes the capture-avoiding substitution of $\vec{y}$ for $\vec{x}$ in $Q$.
\end{definition}

\begin{definition}
  The {\em structural congruence} \cite{SangiorgiWalker} , $\equiv$,
  between processes is the least congruence containing
  alpha-equivalence, satisfying the abelian monoid laws
  (associativity, commutativity and $\pzero$ as identity) for parallel
  composition $|$ and for summation $+$.
\end{definition}

\subsection{Name equivalence}

We take name equivalence, written $\nameeq$, to be the smallest
equivalence relation generated by the following rules.

\begin{mathpar}
\inferrule*[lab=Quote-drop]
{ }
{ \quotep{@{x}} \nameeq x }

\inferrule*[lab=Struct-equiv]
{ P \scong Q }
{ \quotep{P} \nameeq \quotep{Q} }
\end{mathpar}

The astute reader will have noticed that the mutual recursion of names
and processes imposes a mutual recursion on alpha-equivalence and
structural equivalence via name-equivalence. Fortunately, all of this
works out pleasantly and we may calculate in the natural way, free of
concern. The reader interested in the details is referred to the
appendix \ref{appendix:rho_details}.

\subsection{Substitution}

We use $\Proc$ for the set of processes, $\QProc$ for the set of
names, and $\id{\{}\vec{y} / \vec{x} \id{\}}$ to denote partial maps,
$s : \QProc \rightarrow \QProc$. A map, $s$ lifts, uniquely, to a map
on process terms, $\widehat{s} : \Proc \rightarrow \Proc$ by the
following equations.

\begin{mathpar}
  (0) \psubstp{Q}{P} := 0 \\
  (R \juxtap S) \psubstp{Q}{P}
  :=    
  (R)\psubstp{Q}{P} \juxtap (S) \psubstp{Q}{P} \\
  (x?(y).R) \psubstp{Q}{P}    
  :=    
  (x)\substp{Q}{P} (z)\concat( (R \psubstn{z}{y}) \psubstp{Q}{P} ) \\
  (\lift{x}{R}) \psubstp{Q}{P}  
  :=
  \lift{(x)\substp{Q}{P}}{ R \psubstp{Q}{P} } \\
%   (\dropn{x})  \psubstp{Q}{P}       
%   := 
%   \left\{ 
%     \begin{array}{ccc} 
%       \dropn{\quotep{Q}} & & x \nameeq \quotep{P} \\
%       \dropn{x} & & otherwise \\
%     \end{array}
%   \right. 
  (\dropn{x})  \psubstp{Q}{P}       
  := 
  \left\{ 
    \begin{array}{ccc} 
      Q & & x \nameeq \quotep{P} \\
      \dropn{x} & & otherwise \\
    \end{array}
  \right.
\end{mathpar}
 

where

\begin{eqnarray}
  (x)\id{\{} \lpquote Q \rpquote / \lpquote P \rpquote \id{\}}            = 
  \left\{ 
    \begin{array}{ccc}
      \lpquote Q \rpquote & & x \nameeq \lpquote P \rpquote \\
      x & & otherwise \\
    \end{array}
  \right. \nonumber
\end{eqnarray}

and $z$ is chosen distinct from $\quotep{P}$, $\quotep{Q}$, the free
names in $Q$, and all the names in $R$. Our $\alpha$-equivalence will
be built in the standard way from this substitution.

\begin{remark}\label{rem:no_self_referential_names}
  One consequence of these definitions is that $\forall P. \quotep{P}
  \not\in \freenames{P}$.
\end{remark}

\subsection{ Dynamic quote: an example }

Anticipating something of what's to come, consider applying the
substitution, $\widehat{\id{\{}u / z \id{\}}}$, to the following pair
of processes, $\lift{w}{y!(z)}$ and $w[ \lpquote y!(z) \rpquote ]$.

\begin{eqnarray}
	\lift{w}{y!(z)}\widehat{\id{\{}u / z \id{\}}}
		& = &
		\lift{w}{y!(u)} \nonumber\\
	w[ \lpquote y!(z) \rpquote ] \widehat{ \id{\{}u / z \id{\}} }
		& = &
		w[ \lpquote y!(z) \rpquote ] \nonumber
\end{eqnarray}

Because the body of the process between quotes is impervious to
substitution, we get radically different answers. In fact, by
examining the first process in an input context,
e.g. $x?(z).\lift{w}{y!(z)}$, we see that the process under the lift
operator may be shaped by prefixed inputs binding a name inside it. In
this sense, the lift operator will be seen as a way to dynamically
construct processes before reifying them as names.

Finally equipped with these standard features we can present the
dynamics of the calculus.

\subsubsection{Operational semantics} 

Finally, we introduce the computational dynamics. What marks these
algebras as distinct from other more traditionally studied algebraic
structures, e.g. vector spaces or polynomial rings, is the manner in
which dynamics is captured. In traditional structures, dynamics is typically
expressed through morphisms between such structures, as in linear maps
between vector spaces or morphisms between rings. In algebras
associated with the semantics of computation, the dynamics is
expressed as part of the algebraic structure itself, through a
reduction reduction relation typically denoted by $\red$. Below, we
give a recursive presentation of this relation for the calculus used
in the encoding.

$\red \subseteq \pi \times \pi$
$\red : \pi \to \mathcal{P}(\pi)$

\begin{mathpar}
  \inferrule* [lab=Comm] { \textsf{match}( x_{src}, x_{trgt} ) } { x_{trgt}?(y)P \; | \; x_{src}!\langle {Q} \rangle \red P\{\quotep{Q}/y}\} }
  \and \\
  \inferrule* [lab=Par] {{P} \red {P}'} {{{P} | {Q}} \red {{P}' | {Q}}}
  \and
  \inferrule* [lab=Equiv]{{{P} \scong {P}'} \andalso {{P}' \red {Q}'} \andalso {{Q}' \scong {Q}}}{{P} \red {Q}}
\end{mathpar}

\begin{eqnarray*}
  match_{\equiv} (\quotep{P},\quotep{Q}) & := & P \equiv Q \\
  match_{\dagger}(\quotep{P},\quotep{Q}) & := & \forall R. P|Q \red^{*} R => R \red^{*} 0 \\
  match_{K}(\quotep{P},\quotep{Q}) & := & K \mbox{ for some context } K
\end{eqnarray*}

$u?(x)P | u!\langle Q \rangle \red P\{\quotep{Q}/x\}$

%We write $\wred$ for $\red^*$, and $P\red$ if $\exists Q $ such that $ P \red Q$.
We write $P\red$ if $\exists Q $ such that $ P \red Q$ and $P\not\red$, otherwise.

\section{Replication}

As mentioned before, it is known that replication (and hence
recursion) can be implemented in a higher-order process algebra
\cite{SangiorgiWalker}. As our first example of calculation with the
machinery thus far presented we give the construction explicitly in
the {\rhoc}.

\begin{eqnarray}
	D_{x} & := & \prefix{x}{y}{(\binpar{\outputp{x}{y}}{@{y}})} \nonumber\\
	\bangp_{x}{P} & := & \binpar{{x}!\langle{\binpar{D_{x}}{P}}\rangle}{D_{x}} \nonumber
\end{eqnarray}

\begin{eqnarray}
	\bangp_{x}{P} & & \nonumber\\
	=
	& {x}!\langle{(\prefix{x}{y}{(\outputp{x}{y} | @{y})) | P}}\rangle 
	      | \prefix{x}{y}{(\outputp{x}{y} | @{y})} & \nonumber\\
	\red
	& (\outputp{x}{y} | @{y})\substn{\quotep{(\prefix{x}{y}{(@{y} | \outputp{x}{y})) | P}}}{y} & \nonumber\\
	=
	& \outputp{x}{\quotep{(\prefix{x}{y}{(\outputp{x}{y} | @{y})) | P}}}
	  | {(\prefix{x}{y}{(\outputp{x}{y} | @{y})) | P}} & \nonumber\\
	\red
	& \ldots & \nonumber\\
	\red^*
	& P | P | \ldots & \nonumber
\end{eqnarray}

Of course, this encoding, as an implementation, runs away, unfolding
$\bangp{P}$ eagerly. A lazier and more implementable replication
operator, restricted to input-guarded processes, may be obtained as follows.

\begin{eqnarray}
\bangp{\prefix{u}{v}{P}} 
	:= 
	\binpar{\lift{x}{\prefix{u}{v}{(\binpar{D(x)}{P})}}}{D(x)} \nonumber
\end{eqnarray}

\begin{remark}
  Note that the lazier definition still does not deal with summation
  or mixed summation (i.e. sums over input and output). The reader is
  invited to construct definitions of replication that deal with these
  features. 

  Further, the definitions are parameterized in a name, $x$. Can you,
  gentle reader, make a definition that eliminates this parameter and
  guarantees no accidental interaction between the replication
  machinery and the process being replicated -- i.e. no accidental
  sharing of names used by the process to get its work done and the
  name(s) used by the replication to effect copying. This latter
  revision of the definition of replication is crucial to obtaining
  the expected identity $!!P \sim !P$.
\end{remark}

\begin{remark}\label{rem:paradoxical_combinator}
  The reader familiar with the lambda calculus will have noticed the
  similarity between $D$ and the paradoxical combinator.

  [Ed. note: the existence of this seems to suggest we have to be more
  restrictive on the set of processes and names we admit if we are to
  support no-cloning.]
\end{remark}

\subsubsection{Bisimulation}

The computational dynamics gives rise to another kind of equivalence,
the equivalence of computational behavior. As previously mentioned
this is typically captured \emph{via} some form of bisimulation.

% The notion we use in this paper is weak barbed bisimulation
% \cite{milner91polyadicpi}.

The notion we use in this paper is derived from weak barbed
bisimulation \cite{milner91polyadicpi}. 

\begin{definition}
An \emph{observation relation}, $\downarrow_{\mathcal N}$, over a set
of names, $\mathcal N$, is the smallest relation satisfying the rules
below.

\infrule[Out-barb]{y \in {\mathcal N}, \; x \nameeq y}
		  {\outputp{x}{v} \downarrow_{\mathcal N} x}
\infrule[Par-barb]{\mbox{$P\downarrow_{\mathcal N} x$ or $Q\downarrow_{\mathcal N} x$}}
		  {\binpar{P}{Q} \downarrow_{\mathcal N} x}

We write $P \Downarrow_{\mathcal N} x$ if there is $Q$ such that 
$P \wred Q$ and $Q \downarrow_{\mathcal N} x$.
\end{definition}

\begin{definition}
%\label{def.bbisim}
An  ${\mathcal N}$-\emph{barbed bisimulation} over a set of names, ${\mathcal N}$, is a symmetric binary relation 
${\mathcal S}_{\mathcal N}$ between agents such that $P\rel{S}_{\mathcal N}Q$ implies:
\begin{enumerate}
\item If $P \red P'$ then $Q \wred Q'$ and $P'\rel{S}_{\mathcal N} Q'$.
\item If $P\downarrow_{\mathcal N} x$, then $Q\Downarrow_{\mathcal N} x$.
\end{enumerate}
$P$ is ${\mathcal N}$-barbed bisimilar to $Q$, written
$P \wbbisim_{\mathcal N} Q$, if $P \rel{S}_{\mathcal N} Q$ for some ${\mathcal N}$-barbed bisimulation ${\mathcal S}_{\mathcal N}$.
\end{definition}

$\mathcal{R} \subseteq \pi \times \pi$

$P \mathcal{R} Q => \forall P'. P \red P' \Rightarrow \exists Q'. Q \red Q', P' \mathcal{R} Q'$

$P \vdash x \Rightarrow Q \vdash x$

\begin{mathpar}
  \inferrule*[lab=Out-barb]{x \nameeq y}{{y}!\langle{Q}\rangle \vdash x}
  \and
  \inferrule*[lab=Par-barb]{\mbox{$P\vdash x$ or $Q\vdash x$}}{\binpar{P}{Q} \vdash x}
\end{mathpar}

\subsubsection{Contexts}

One of the principle advantages of computational calculi like the
$\pi$-calculus is a well-defined notion of context,
contextual-equivalence and a correlation between
contextual-equivalence and notions of bisimulation. The notion of
context allows the decomposition of a process into (sub-)process and
its syntactic environment, its context. Thus, a context may be
thought of as a process with a ``hole'' (written $\Box$) in it. The
application of a context $M$ to a process $P$, written $M[P]$, is
tantamount to filling the hole in $M$ with $P$. In this paper we do
not need the full weight of this theory, but do make use of the notion
of context in the proof the main theorem. 

\begin{mathpar}
  \inferrule* [lab=summation] {} {{M_{M},M_{N}} \bc \Box \;|\; x.M_{A} \;|\; M_{M}+M_{N}}
  \and
  \inferrule* [lab=agent] {} {{M_{A}} \bc (\vec{x})M_{P} \;| \; \clift{P_0,\ldots,M_{P},\ldots,P_N}}
  \and \\
  \inferrule* [lab=process] {} {{M_{P}} \bc M_{N} \;| \;P|M_{P} }
\end{mathpar} 

\begin{mathpar}
  \inferrule* [lab=sychronization] {} {M_{N} \bc \Box \;|\; x?M_{F} \;|\; x!M_{C}}
  \and
  \inferrule* [lab=abstraction] {} {{M_{F}} \bc (x)M_{P} }
  \and
  \inferrule* [lab=concretion] {} {{M_{C}} \bc \langle M_{P} \rangle }
  \and \\
  \inferrule* [lab=process] {} {{M_{P}} \bc M_{N} \;| \;P|M_{P} }
\end{mathpar}

\begin{definition}[contextual application] Given a context $M$, and
  process $P$, we define the \emph{contextual application}, $M[P] :=
  M\{P/\Box\}$. That is, the contextual application of M to P is the
  substitution of $P$ for $\Box$ in $M$.
\end{definition}

$\meaningof{-} : L \to \mathcal{P}(\pi)$

\begin{mathpar}
  \inferrule* [lab=collection] {} {\meaningof{true} = \pi, \and \meaningof{~E} = \pi \setminus \meaningof{E}, \and \meaningof{E_{1} \& E_{2}} = \meaningof{E_{1}} \cap \meaningof{E_{2}}}
\end{mathpar}

\begin{mathpar}
  \inferrule* [lab=structure] {} {\meaningof{0} = \{ P \in \pi | P \equiv 0 \}, \and \\ \meaningof{E_1 | E_2} = \{ P \in \pi | P \equiv P_{1} | P_{2}, P_{1} \in \meaningof{E_{1}}, P_{2} \in \meaningof{E_2}\} }
\end{mathpar}

\begin{mathpar}
 \inferrule* [lab=behavior] {} {\meaningof{\langle a?b \rangle E} = \{ P \in \pi | P \equiv Q | u?(y)P', \\ \and \\\\ \and \\ \;\;\; u \in \meaningof{a}, \forall z.P'\{z/y\} \in \meaningof{E\{z/b\}}\}, \and \\ \meaningof{a!E} = \{ P \in \pi | P \equiv Q | x!\langle P' \rangle, x \in \meaningof{a} P' \in \meaningof{E}\} }
\end{mathpar}

\begin{mathpar}
 \inferrule* [lab=nominal] {} {\meaningof{\quotep{E}} = \{ \quotep{P} \in \quotep{\pi} | P \in \meaningof{E} \}, \and \meaningof{\quotep{P}} = \{ \quotep{Q} \in \quotep{\pi} | P \equiv Q \} \and \\ \meaningof{@\quotep{E}} = \{ P \in \pi | P \equiv @x, x \in \meaningof{E} \}}
\end{mathpar}

\begin{eqnarray*}
  \\
  \meaningof{-} : TS \to ST
\end{eqnarray*}

\begin{eqnarray*}
  \\
  L : TS \to ST
\end{eqnarray*}

\begin{eqnarray*}
  \\
  P \models E \iff P \in \meaningof{E}
\end{eqnarray*}

\begin{eqnarray*}
  P \approx_{L} Q \iff \forall E \in L. P \models E \iff Q \models E
\end{eqnarray*}

\begin{eqnarray*}
  P \approx_{K} Q
\end{eqnarray*}

\begin{eqnarray*}
  P \approx Q
\end{eqnarray*}

$\approx_{K} = \approx = \approx_{L}$

\subsubsection{Contextual duality}

Note that contexts extend the quotation operation to a family of
operations from processes to names. Given a context, $M$, we can
define a \emph{nominal context}, $\quotep{M}$ by $\quotep{M}[P] :=
\quotep{M[P]}$. To foreshadow what is to come we observe that these
operations enjoy a duality with processes very much like the duality
between vectors and maps from vectors to scalars.

Further, because the calculus is essentially higher-order, we have a
correspondence between contexts and processes. More specifically,
given a name $x$ and a context $M$ we can construct $M^{*}_{x}$ such
that 

\begin{mathpar}
  M^{*}_{x} | \lift{x}{P} \red M[P]
\end{mathpar}

namely,

\begin{mathpar}
  M^{*}_{x} := x?(u).M[\dropn{u}]
\end{mathpar}

The dependence of $M^{*}_{x}$ on a name makes it an abstraction, 

\begin{mathpar}
  M^{*} := (x)x?(u).M[\dropn{u}]
\end{mathpar}

\subsection{Additional notation}

It will sometimes be convenient to denote the process a name
quotes. We already have the notation $x = \quotep{P}$, but it will be
convenient to introduce an alternate notation, $\procn{x}$, when we
want to emphasize the connection to the use of the name. Note that, by
virtue of name equivalence, $\quotep{\procn{x}} \nameeq x$; so, the
notation is consistent with previous definitions.

Further, because names have structure it is possible to effect
substitutions on the basis of that structure. This means we need to
upgrade our notation for substitutions, which we accomplish by
adapting comprehension notation. Thus,

\begin{mathpar}
  P\{ y / x : x \in S \}
\end{mathpar}

is interpreted to mean the process derived from P by replacing (in a
capture-avoiding manner) each occurrence of $x$ in $S$ by $y$. For example,

\begin{mathpar}
  P\{ \quotep{\procn{x}|\procn{x}} / x : x \in \freenames{P} \}
\end{mathpar}

will replace each (occurrence) of a free name $x$ in $P$ by
$\quotep{\procn{x}|\procn{x}}$.

Also, we will avail ourselves of the notation $x^{L}$ and $x^{R}$ to
denote injections of a name into disjoint copies of the name
space. There are numerous ways to accomplish this. One example can be
found in \cite{MeredithR05}. This notation overloads to vectors of
names: $\vec{x}^{\pi} := (x_{i}^{\pi} \; : \; 0 \leq i < |\vec{x}| )$ where $\pi \in \{L,R\}$.

We also use $P^{\Box} := P|\Box$.

In \cite{MeredithR05} an interpretation of the new operator is
given. It turns out that there are several possible interpretations
all enjoying the requisite algebraic properties of the operator (see
\cite{milner91polyadicpi}). We will therefore make liberal use of
$(\nu\; \vec{x})P$.

% subsection the_syntax_and_semantics_of_the_notation_system (end)   

\input{qm2pi.qmops} 

\input{qm2pi.sterngerlach} 

\input{qm2pi.metric} 

% section concurrent_process_calculi (end)

%\input{qm2pi.proofsketch}

% section proof sketch (end)

%\input{qm2pi.slviaknots} 

% section spatial logic via knots (end)

\input{qm2pi.conclusion}

% section conclusion (end)

%\input{qm2pi.dtcodes} 

% section wiring algorithm (end)

\input{qm2pi.ack} 

% section acknowledgments (end)

\newpage


\bibliographystyle{plain}   
\bibliography{../../biblios/main.bib}

\input{qm2pi.rhodetails}

\end{document}

 

% section notation (end)

\input{qm2pi.process.calculi} 

% section concurrent_process_calculi_and_spatial_logics_ (end)
    
%\documentclass[12pt]{llncs}
%\documentclass{jktr}

\usepackage[pdftex]{hyperref}                   
\usepackage {listings}
\usepackage {mathpartir}
\usepackage{bcprules}
%\usepackage{listings}
                       
\usepackage{graphicx} 
%\usepackage[margins=2.5cm,nohead,nofoot]{geometry}
%\usepackage{geometry}
\usepackage{amsfonts}
\usepackage{amstext}
\usepackage{latexsym}
\usepackage{amssymb}
\usepackage{color}


%\include{myPreamble}
\include{qm2pi.local} 

%\ifpdf
%\usepackage[pdftex]{graphicx}
%\else
%\usepackage{graphicx}
%\fi

 % \ifpdf
%  \usepackage{pdfsync}
%  \if


%\title{Brief Article}
%\author{David F. Snyder}
%\author{L.G. Meredith}

%\address{Dept. of Math., Texas State University--San Marcos, San Marcos, TX 78666}
       
\pagestyle{empty}


\begin{document}

\lstset{language=[Objective]Caml,frame=shadowbox}

\input{qm2pi.front}

% section front matter (end)

\input{qm2pi.intro} 
 
% section introduction (end)

% \input{qm2pi.knotations} 

% section notation (end)

\input{qm2pi.process.calculi} 

% section concurrent_process_calculi_and_spatial_logics_ (end)
    
%\input{qm2pi.knots2pi} 

%\input{qm2pi.trefoil} 

%\input{qm2pi.mainthm} 

% subsection basic_interpretation (end)

%\input{qm2pi.rho.presentation} 
\subsection{The syntax and semantics of the notation system}\label{sub:the_syntax_and_semantics_of_the_notation_system} % (fold)

We now summarize a technical presentation of the calculus that
embodies our theory of dynamics. The typical presentation of such a
calculus follows the style of giving generators and relations on
them. The grammar, below, describing term constructors, freely
generates the set of processes, $\Proc$. This set is then quotiented
by a relation known as structural congruence and it is over this set
that the notion of dynamics is expressed. This presentation is
essentially that of \cite{MeredithR05} with the addition of
polyadicity and summation. For readability we have relegated some of
the technical subtleties to an appendix.

\subsubsection{Process grammar}\label{subsub:process_grammar}

\begin{mathpar}
  \inferrule* [lab=synchronization] {} {{M} \bc \pzero \;|\; x?F \;|\; x!C }
  \and
  \inferrule* [lab=abstraction] {} {{F} \bc (x)P}
  \and
  \inferrule* [lab=concretion] {} {{C} \bc \langle Q \rangle}
  \and
  \inferrule* [lab=process] {} {{P,Q} \bc M \;| \;P|Q \;|\; @{x}}
  \and
  \inferrule* [lab=name] {} {{x} \bc \quotep{P}}
\end{mathpar} 

Note that $\vec{x}$ (resp. $\vec{P}$) denotes a vector of names
(resp. processes) of length $|\vec{x}|$ (resp. $|\vec{P}|$). We adopt
the following useful abbreviations.

\begin{mathpar}
   x?(\vec{y}).P := x.(\vec{y})P \and  x\clift{\vec{P}} := x.\clift{\vec{P}}
   \and x!(y) := \lift{x}{\dropn{y}}
   \and \Pi_{i=0}^{n-1}P_i := P_0 | \ldots | P_{n-1}
\end{mathpar}

\subsubsection{Structural congruence}

\paragraph{Free and bound names and alpha-equivalence.} At the
core of structural equivalence is alpha-equivalence which identifies
process that are the same up to a change of variable. Formally, we
recognize the distinction between free and bound names. The free names
of a process, $\freenames{P}$, may be calculated recursively as
follows:

\begin{mathpar}
\freenames{\pzero} := \emptyset
  \and \\
  \freenames{x?(y).P} := \{ x \} \cup (\freenames{P} \setminus \{ y \})
  \and 
  \freenames{x!\langle P \rangle} := \{ x \} \cup \{ P \} 
  \and \\
  \freenames{P|Q} := \freenames{P} \cup \freenames{Q}
  \and \\
  \freenames{@{x}} := \{ x \}
\end{mathpar}

$\pi$
$\quotep{\pi}$

$\freenames{-} : \pi \to \mathcal{P}(\quotep{\pi})$

\begin{eqnarray*}
  \freenames{\pzero} & := & \emptyset \\
  \freenames{x?(y).P} & := & \{ x \} \cup (\freenames{P} \setminus \{ y \}) \\
  \freenames{x!\langle P \rangle} & := & \{ x \} \cup \{ P \} \\
  \freenames{P|Q} & := & \freenames{P} \cup \freenames{Q} \\
  \freenames{\dropn{x}} & := & \{ x \}
\end{eqnarray*}

The bound names of a process, $\boundnames{P}$, are those names occurring in $P$
that are not free. For example, in $x?(y).0$, the name $x$ is free, while $y$ is bound.

\begin{mathpar}
  \inferrule* [lab=monoidal-laws] {} { P|Q \equiv Q|P \and P|0 \equiv P \and P|(Q|R) \equiv (P|Q)|R }
\end{mathpar}

\begin{mathpar}
  \inferrule* [lab=alpha-equivalence] {} { (x)P \equiv (y)P\{y/x\} \and y \not\in \freenames{P} }
\end{mathpar}

\begin{definition}
Then two processes, $P,Q$, are alpha-equivalent if $P = Q\{\vec{y}/\vec{x}\}$ for
some $\vec{x} \in \boundnames{Q},\vec{y} \in \boundnames{P}$, where $Q\{\vec{y}/\vec{x}\}$
denotes the capture-avoiding substitution of $\vec{y}$ for $\vec{x}$ in $Q$.
\end{definition}

\begin{definition}
  The {\em structural congruence} \cite{SangiorgiWalker} , $\equiv$,
  between processes is the least congruence containing
  alpha-equivalence, satisfying the abelian monoid laws
  (associativity, commutativity and $\pzero$ as identity) for parallel
  composition $|$ and for summation $+$.
\end{definition}

\subsection{Name equivalence}

We take name equivalence, written $\nameeq$, to be the smallest
equivalence relation generated by the following rules.

\begin{mathpar}
\inferrule*[lab=Quote-drop]
{ }
{ \quotep{@{x}} \nameeq x }

\inferrule*[lab=Struct-equiv]
{ P \scong Q }
{ \quotep{P} \nameeq \quotep{Q} }
\end{mathpar}

The astute reader will have noticed that the mutual recursion of names
and processes imposes a mutual recursion on alpha-equivalence and
structural equivalence via name-equivalence. Fortunately, all of this
works out pleasantly and we may calculate in the natural way, free of
concern. The reader interested in the details is referred to the
appendix \ref{appendix:rho_details}.

\subsection{Substitution}

We use $\Proc$ for the set of processes, $\QProc$ for the set of
names, and $\id{\{}\vec{y} / \vec{x} \id{\}}$ to denote partial maps,
$s : \QProc \rightarrow \QProc$. A map, $s$ lifts, uniquely, to a map
on process terms, $\widehat{s} : \Proc \rightarrow \Proc$ by the
following equations.

\begin{mathpar}
  (0) \psubstp{Q}{P} := 0 \\
  (R \juxtap S) \psubstp{Q}{P}
  :=    
  (R)\psubstp{Q}{P} \juxtap (S) \psubstp{Q}{P} \\
  (x?(y).R) \psubstp{Q}{P}    
  :=    
  (x)\substp{Q}{P} (z)\concat( (R \psubstn{z}{y}) \psubstp{Q}{P} ) \\
  (\lift{x}{R}) \psubstp{Q}{P}  
  :=
  \lift{(x)\substp{Q}{P}}{ R \psubstp{Q}{P} } \\
%   (\dropn{x})  \psubstp{Q}{P}       
%   := 
%   \left\{ 
%     \begin{array}{ccc} 
%       \dropn{\quotep{Q}} & & x \nameeq \quotep{P} \\
%       \dropn{x} & & otherwise \\
%     \end{array}
%   \right. 
  (\dropn{x})  \psubstp{Q}{P}       
  := 
  \left\{ 
    \begin{array}{ccc} 
      Q & & x \nameeq \quotep{P} \\
      \dropn{x} & & otherwise \\
    \end{array}
  \right.
\end{mathpar}
 

where

\begin{eqnarray}
  (x)\id{\{} \lpquote Q \rpquote / \lpquote P \rpquote \id{\}}            = 
  \left\{ 
    \begin{array}{ccc}
      \lpquote Q \rpquote & & x \nameeq \lpquote P \rpquote \\
      x & & otherwise \\
    \end{array}
  \right. \nonumber
\end{eqnarray}

and $z$ is chosen distinct from $\quotep{P}$, $\quotep{Q}$, the free
names in $Q$, and all the names in $R$. Our $\alpha$-equivalence will
be built in the standard way from this substitution.

\begin{remark}\label{rem:no_self_referential_names}
  One consequence of these definitions is that $\forall P. \quotep{P}
  \not\in \freenames{P}$.
\end{remark}

\subsection{ Dynamic quote: an example }

Anticipating something of what's to come, consider applying the
substitution, $\widehat{\id{\{}u / z \id{\}}}$, to the following pair
of processes, $\lift{w}{y!(z)}$ and $w[ \lpquote y!(z) \rpquote ]$.

\begin{eqnarray}
	\lift{w}{y!(z)}\widehat{\id{\{}u / z \id{\}}}
		& = &
		\lift{w}{y!(u)} \nonumber\\
	w[ \lpquote y!(z) \rpquote ] \widehat{ \id{\{}u / z \id{\}} }
		& = &
		w[ \lpquote y!(z) \rpquote ] \nonumber
\end{eqnarray}

Because the body of the process between quotes is impervious to
substitution, we get radically different answers. In fact, by
examining the first process in an input context,
e.g. $x?(z).\lift{w}{y!(z)}$, we see that the process under the lift
operator may be shaped by prefixed inputs binding a name inside it. In
this sense, the lift operator will be seen as a way to dynamically
construct processes before reifying them as names.

Finally equipped with these standard features we can present the
dynamics of the calculus.

\subsubsection{Operational semantics} 

Finally, we introduce the computational dynamics. What marks these
algebras as distinct from other more traditionally studied algebraic
structures, e.g. vector spaces or polynomial rings, is the manner in
which dynamics is captured. In traditional structures, dynamics is typically
expressed through morphisms between such structures, as in linear maps
between vector spaces or morphisms between rings. In algebras
associated with the semantics of computation, the dynamics is
expressed as part of the algebraic structure itself, through a
reduction reduction relation typically denoted by $\red$. Below, we
give a recursive presentation of this relation for the calculus used
in the encoding.

$\red \subseteq \pi \times \pi$
$\red : \pi \to \mathcal{P}(\pi)$

\begin{mathpar}
  \inferrule* [lab=Comm] { \textsf{match}( x_{src}, x_{trgt} ) } { x_{trgt}?(y)P \; | \; x_{src}!\langle {Q} \rangle \red P\{\quotep{Q}/y}\} }
  \and \\
  \inferrule* [lab=Par] {{P} \red {P}'} {{{P} | {Q}} \red {{P}' | {Q}}}
  \and
  \inferrule* [lab=Equiv]{{{P} \scong {P}'} \andalso {{P}' \red {Q}'} \andalso {{Q}' \scong {Q}}}{{P} \red {Q}}
\end{mathpar}

\begin{eqnarray*}
  match_{\equiv} (\quotep{P},\quotep{Q}) & := & P \equiv Q \\
  match_{\dagger}(\quotep{P},\quotep{Q}) & := & \forall R. P|Q \red^{*} R => R \red^{*} 0 \\
  match_{K}(\quotep{P},\quotep{Q}) & := & K \mbox{ for some context } K
\end{eqnarray*}

$u?(x)P | u!\langle Q \rangle \red P\{\quotep{Q}/x\}$

%We write $\wred$ for $\red^*$, and $P\red$ if $\exists Q $ such that $ P \red Q$.
We write $P\red$ if $\exists Q $ such that $ P \red Q$ and $P\not\red$, otherwise.

\section{Replication}

As mentioned before, it is known that replication (and hence
recursion) can be implemented in a higher-order process algebra
\cite{SangiorgiWalker}. As our first example of calculation with the
machinery thus far presented we give the construction explicitly in
the {\rhoc}.

\begin{eqnarray}
	D_{x} & := & \prefix{x}{y}{(\binpar{\outputp{x}{y}}{@{y}})} \nonumber\\
	\bangp_{x}{P} & := & \binpar{{x}!\langle{\binpar{D_{x}}{P}}\rangle}{D_{x}} \nonumber
\end{eqnarray}

\begin{eqnarray}
	\bangp_{x}{P} & & \nonumber\\
	=
	& {x}!\langle{(\prefix{x}{y}{(\outputp{x}{y} | @{y})) | P}}\rangle 
	      | \prefix{x}{y}{(\outputp{x}{y} | @{y})} & \nonumber\\
	\red
	& (\outputp{x}{y} | @{y})\substn{\quotep{(\prefix{x}{y}{(@{y} | \outputp{x}{y})) | P}}}{y} & \nonumber\\
	=
	& \outputp{x}{\quotep{(\prefix{x}{y}{(\outputp{x}{y} | @{y})) | P}}}
	  | {(\prefix{x}{y}{(\outputp{x}{y} | @{y})) | P}} & \nonumber\\
	\red
	& \ldots & \nonumber\\
	\red^*
	& P | P | \ldots & \nonumber
\end{eqnarray}

Of course, this encoding, as an implementation, runs away, unfolding
$\bangp{P}$ eagerly. A lazier and more implementable replication
operator, restricted to input-guarded processes, may be obtained as follows.

\begin{eqnarray}
\bangp{\prefix{u}{v}{P}} 
	:= 
	\binpar{\lift{x}{\prefix{u}{v}{(\binpar{D(x)}{P})}}}{D(x)} \nonumber
\end{eqnarray}

\begin{remark}
  Note that the lazier definition still does not deal with summation
  or mixed summation (i.e. sums over input and output). The reader is
  invited to construct definitions of replication that deal with these
  features. 

  Further, the definitions are parameterized in a name, $x$. Can you,
  gentle reader, make a definition that eliminates this parameter and
  guarantees no accidental interaction between the replication
  machinery and the process being replicated -- i.e. no accidental
  sharing of names used by the process to get its work done and the
  name(s) used by the replication to effect copying. This latter
  revision of the definition of replication is crucial to obtaining
  the expected identity $!!P \sim !P$.
\end{remark}

\begin{remark}\label{rem:paradoxical_combinator}
  The reader familiar with the lambda calculus will have noticed the
  similarity between $D$ and the paradoxical combinator.

  [Ed. note: the existence of this seems to suggest we have to be more
  restrictive on the set of processes and names we admit if we are to
  support no-cloning.]
\end{remark}

\subsubsection{Bisimulation}

The computational dynamics gives rise to another kind of equivalence,
the equivalence of computational behavior. As previously mentioned
this is typically captured \emph{via} some form of bisimulation.

% The notion we use in this paper is weak barbed bisimulation
% \cite{milner91polyadicpi}.

The notion we use in this paper is derived from weak barbed
bisimulation \cite{milner91polyadicpi}. 

\begin{definition}
An \emph{observation relation}, $\downarrow_{\mathcal N}$, over a set
of names, $\mathcal N$, is the smallest relation satisfying the rules
below.

\infrule[Out-barb]{y \in {\mathcal N}, \; x \nameeq y}
		  {\outputp{x}{v} \downarrow_{\mathcal N} x}
\infrule[Par-barb]{\mbox{$P\downarrow_{\mathcal N} x$ or $Q\downarrow_{\mathcal N} x$}}
		  {\binpar{P}{Q} \downarrow_{\mathcal N} x}

We write $P \Downarrow_{\mathcal N} x$ if there is $Q$ such that 
$P \wred Q$ and $Q \downarrow_{\mathcal N} x$.
\end{definition}

\begin{definition}
%\label{def.bbisim}
An  ${\mathcal N}$-\emph{barbed bisimulation} over a set of names, ${\mathcal N}$, is a symmetric binary relation 
${\mathcal S}_{\mathcal N}$ between agents such that $P\rel{S}_{\mathcal N}Q$ implies:
\begin{enumerate}
\item If $P \red P'$ then $Q \wred Q'$ and $P'\rel{S}_{\mathcal N} Q'$.
\item If $P\downarrow_{\mathcal N} x$, then $Q\Downarrow_{\mathcal N} x$.
\end{enumerate}
$P$ is ${\mathcal N}$-barbed bisimilar to $Q$, written
$P \wbbisim_{\mathcal N} Q$, if $P \rel{S}_{\mathcal N} Q$ for some ${\mathcal N}$-barbed bisimulation ${\mathcal S}_{\mathcal N}$.
\end{definition}

$\mathcal{R} \subseteq \pi \times \pi$

$P \mathcal{R} Q => \forall P'. P \red P' \Rightarrow \exists Q'. Q \red Q', P' \mathcal{R} Q'$

$P \vdash x \Rightarrow Q \vdash x$

\begin{mathpar}
  \inferrule*[lab=Out-barb]{x \nameeq y}{{y}!\langle{Q}\rangle \vdash x}
  \and
  \inferrule*[lab=Par-barb]{\mbox{$P\vdash x$ or $Q\vdash x$}}{\binpar{P}{Q} \vdash x}
\end{mathpar}

\subsubsection{Contexts}

One of the principle advantages of computational calculi like the
$\pi$-calculus is a well-defined notion of context,
contextual-equivalence and a correlation between
contextual-equivalence and notions of bisimulation. The notion of
context allows the decomposition of a process into (sub-)process and
its syntactic environment, its context. Thus, a context may be
thought of as a process with a ``hole'' (written $\Box$) in it. The
application of a context $M$ to a process $P$, written $M[P]$, is
tantamount to filling the hole in $M$ with $P$. In this paper we do
not need the full weight of this theory, but do make use of the notion
of context in the proof the main theorem. 

\begin{mathpar}
  \inferrule* [lab=summation] {} {{M_{M},M_{N}} \bc \Box \;|\; x.M_{A} \;|\; M_{M}+M_{N}}
  \and
  \inferrule* [lab=agent] {} {{M_{A}} \bc (\vec{x})M_{P} \;| \; \clift{P_0,\ldots,M_{P},\ldots,P_N}}
  \and \\
  \inferrule* [lab=process] {} {{M_{P}} \bc M_{N} \;| \;P|M_{P} }
\end{mathpar} 

\begin{mathpar}
  \inferrule* [lab=sychronization] {} {M_{N} \bc \Box \;|\; x?M_{F} \;|\; x!M_{C}}
  \and
  \inferrule* [lab=abstraction] {} {{M_{F}} \bc (x)M_{P} }
  \and
  \inferrule* [lab=concretion] {} {{M_{C}} \bc \langle M_{P} \rangle }
  \and \\
  \inferrule* [lab=process] {} {{M_{P}} \bc M_{N} \;| \;P|M_{P} }
\end{mathpar}

\begin{definition}[contextual application] Given a context $M$, and
  process $P$, we define the \emph{contextual application}, $M[P] :=
  M\{P/\Box\}$. That is, the contextual application of M to P is the
  substitution of $P$ for $\Box$ in $M$.
\end{definition}

$\meaningof{-} : L \to \mathcal{P}(\pi)$

\begin{mathpar}
  \inferrule* [lab=collection] {} {\meaningof{true} = \pi, \and \meaningof{~E} = \pi \setminus \meaningof{E}, \and \meaningof{E_{1} \& E_{2}} = \meaningof{E_{1}} \cap \meaningof{E_{2}}}
\end{mathpar}

\begin{mathpar}
  \inferrule* [lab=structure] {} {\meaningof{0} = \{ P \in \pi | P \equiv 0 \}, \and \\ \meaningof{E_1 | E_2} = \{ P \in \pi | P \equiv P_{1} | P_{2}, P_{1} \in \meaningof{E_{1}}, P_{2} \in \meaningof{E_2}\} }
\end{mathpar}

\begin{mathpar}
 \inferrule* [lab=behavior] {} {\meaningof{\langle a?b \rangle E} = \{ P \in \pi | P \equiv Q | u?(y)P', \\ \and \\\\ \and \\ \;\;\; u \in \meaningof{a}, \forall z.P'\{z/y\} \in \meaningof{E\{z/b\}}\}, \and \\ \meaningof{a!E} = \{ P \in \pi | P \equiv Q | x!\langle P' \rangle, x \in \meaningof{a} P' \in \meaningof{E}\} }
\end{mathpar}

\begin{mathpar}
 \inferrule* [lab=nominal] {} {\meaningof{\quotep{E}} = \{ \quotep{P} \in \quotep{\pi} | P \in \meaningof{E} \}, \and \meaningof{\quotep{P}} = \{ \quotep{Q} \in \quotep{\pi} | P \equiv Q \} \and \\ \meaningof{@\quotep{E}} = \{ P \in \pi | P \equiv @x, x \in \meaningof{E} \}}
\end{mathpar}

\begin{eqnarray*}
  \\
  \meaningof{-} : TS \to ST
\end{eqnarray*}

\begin{eqnarray*}
  \\
  L : TS \to ST
\end{eqnarray*}

\begin{eqnarray*}
  \\
  P \models E \iff P \in \meaningof{E}
\end{eqnarray*}

\begin{eqnarray*}
  P \approx_{L} Q \iff \forall E \in L. P \models E \iff Q \models E
\end{eqnarray*}

\begin{eqnarray*}
  P \approx_{K} Q
\end{eqnarray*}

\begin{eqnarray*}
  P \approx Q
\end{eqnarray*}

$\approx_{K} = \approx = \approx_{L}$

\subsubsection{Contextual duality}

Note that contexts extend the quotation operation to a family of
operations from processes to names. Given a context, $M$, we can
define a \emph{nominal context}, $\quotep{M}$ by $\quotep{M}[P] :=
\quotep{M[P]}$. To foreshadow what is to come we observe that these
operations enjoy a duality with processes very much like the duality
between vectors and maps from vectors to scalars.

Further, because the calculus is essentially higher-order, we have a
correspondence between contexts and processes. More specifically,
given a name $x$ and a context $M$ we can construct $M^{*}_{x}$ such
that 

\begin{mathpar}
  M^{*}_{x} | \lift{x}{P} \red M[P]
\end{mathpar}

namely,

\begin{mathpar}
  M^{*}_{x} := x?(u).M[\dropn{u}]
\end{mathpar}

The dependence of $M^{*}_{x}$ on a name makes it an abstraction, 

\begin{mathpar}
  M^{*} := (x)x?(u).M[\dropn{u}]
\end{mathpar}

\subsection{Additional notation}

It will sometimes be convenient to denote the process a name
quotes. We already have the notation $x = \quotep{P}$, but it will be
convenient to introduce an alternate notation, $\procn{x}$, when we
want to emphasize the connection to the use of the name. Note that, by
virtue of name equivalence, $\quotep{\procn{x}} \nameeq x$; so, the
notation is consistent with previous definitions.

Further, because names have structure it is possible to effect
substitutions on the basis of that structure. This means we need to
upgrade our notation for substitutions, which we accomplish by
adapting comprehension notation. Thus,

\begin{mathpar}
  P\{ y / x : x \in S \}
\end{mathpar}

is interpreted to mean the process derived from P by replacing (in a
capture-avoiding manner) each occurrence of $x$ in $S$ by $y$. For example,

\begin{mathpar}
  P\{ \quotep{\procn{x}|\procn{x}} / x : x \in \freenames{P} \}
\end{mathpar}

will replace each (occurrence) of a free name $x$ in $P$ by
$\quotep{\procn{x}|\procn{x}}$.

Also, we will avail ourselves of the notation $x^{L}$ and $x^{R}$ to
denote injections of a name into disjoint copies of the name
space. There are numerous ways to accomplish this. One example can be
found in \cite{MeredithR05}. This notation overloads to vectors of
names: $\vec{x}^{\pi} := (x_{i}^{\pi} \; : \; 0 \leq i < |\vec{x}| )$ where $\pi \in \{L,R\}$.

We also use $P^{\Box} := P|\Box$.

In \cite{MeredithR05} an interpretation of the new operator is
given. It turns out that there are several possible interpretations
all enjoying the requisite algebraic properties of the operator (see
\cite{milner91polyadicpi}). We will therefore make liberal use of
$(\nu\; \vec{x})P$.

% subsection the_syntax_and_semantics_of_the_notation_system (end)   

\input{qm2pi.qmops} 

\input{qm2pi.sterngerlach} 

\input{qm2pi.metric} 

% section concurrent_process_calculi (end)

%\input{qm2pi.proofsketch}

% section proof sketch (end)

%\input{qm2pi.slviaknots} 

% section spatial logic via knots (end)

\input{qm2pi.conclusion}

% section conclusion (end)

%\input{qm2pi.dtcodes} 

% section wiring algorithm (end)

\input{qm2pi.ack} 

% section acknowledgments (end)

\newpage


\bibliographystyle{plain}   
\bibliography{../../biblios/main.bib}

\input{qm2pi.rhodetails}

\end{document}

 

%\documentclass[12pt]{llncs}
%\documentclass{jktr}

\usepackage[pdftex]{hyperref}                   
\usepackage {listings}
\usepackage {mathpartir}
\usepackage{bcprules}
%\usepackage{listings}
                       
\usepackage{graphicx} 
%\usepackage[margins=2.5cm,nohead,nofoot]{geometry}
%\usepackage{geometry}
\usepackage{amsfonts}
\usepackage{amstext}
\usepackage{latexsym}
\usepackage{amssymb}
\usepackage{color}


%\include{myPreamble}
\include{qm2pi.local} 

%\ifpdf
%\usepackage[pdftex]{graphicx}
%\else
%\usepackage{graphicx}
%\fi

 % \ifpdf
%  \usepackage{pdfsync}
%  \if


%\title{Brief Article}
%\author{David F. Snyder}
%\author{L.G. Meredith}

%\address{Dept. of Math., Texas State University--San Marcos, San Marcos, TX 78666}
       
\pagestyle{empty}


\begin{document}

\lstset{language=[Objective]Caml,frame=shadowbox}

\input{qm2pi.front}

% section front matter (end)

\input{qm2pi.intro} 
 
% section introduction (end)

% \input{qm2pi.knotations} 

% section notation (end)

\input{qm2pi.process.calculi} 

% section concurrent_process_calculi_and_spatial_logics_ (end)
    
%\input{qm2pi.knots2pi} 

%\input{qm2pi.trefoil} 

%\input{qm2pi.mainthm} 

% subsection basic_interpretation (end)

%\input{qm2pi.rho.presentation} 
\subsection{The syntax and semantics of the notation system}\label{sub:the_syntax_and_semantics_of_the_notation_system} % (fold)

We now summarize a technical presentation of the calculus that
embodies our theory of dynamics. The typical presentation of such a
calculus follows the style of giving generators and relations on
them. The grammar, below, describing term constructors, freely
generates the set of processes, $\Proc$. This set is then quotiented
by a relation known as structural congruence and it is over this set
that the notion of dynamics is expressed. This presentation is
essentially that of \cite{MeredithR05} with the addition of
polyadicity and summation. For readability we have relegated some of
the technical subtleties to an appendix.

\subsubsection{Process grammar}\label{subsub:process_grammar}

\begin{mathpar}
  \inferrule* [lab=synchronization] {} {{M} \bc \pzero \;|\; x?F \;|\; x!C }
  \and
  \inferrule* [lab=abstraction] {} {{F} \bc (x)P}
  \and
  \inferrule* [lab=concretion] {} {{C} \bc \langle Q \rangle}
  \and
  \inferrule* [lab=process] {} {{P,Q} \bc M \;| \;P|Q \;|\; @{x}}
  \and
  \inferrule* [lab=name] {} {{x} \bc \quotep{P}}
\end{mathpar} 

Note that $\vec{x}$ (resp. $\vec{P}$) denotes a vector of names
(resp. processes) of length $|\vec{x}|$ (resp. $|\vec{P}|$). We adopt
the following useful abbreviations.

\begin{mathpar}
   x?(\vec{y}).P := x.(\vec{y})P \and  x\clift{\vec{P}} := x.\clift{\vec{P}}
   \and x!(y) := \lift{x}{\dropn{y}}
   \and \Pi_{i=0}^{n-1}P_i := P_0 | \ldots | P_{n-1}
\end{mathpar}

\subsubsection{Structural congruence}

\paragraph{Free and bound names and alpha-equivalence.} At the
core of structural equivalence is alpha-equivalence which identifies
process that are the same up to a change of variable. Formally, we
recognize the distinction between free and bound names. The free names
of a process, $\freenames{P}$, may be calculated recursively as
follows:

\begin{mathpar}
\freenames{\pzero} := \emptyset
  \and \\
  \freenames{x?(y).P} := \{ x \} \cup (\freenames{P} \setminus \{ y \})
  \and 
  \freenames{x!\langle P \rangle} := \{ x \} \cup \{ P \} 
  \and \\
  \freenames{P|Q} := \freenames{P} \cup \freenames{Q}
  \and \\
  \freenames{@{x}} := \{ x \}
\end{mathpar}

$\pi$
$\quotep{\pi}$

$\freenames{-} : \pi \to \mathcal{P}(\quotep{\pi})$

\begin{eqnarray*}
  \freenames{\pzero} & := & \emptyset \\
  \freenames{x?(y).P} & := & \{ x \} \cup (\freenames{P} \setminus \{ y \}) \\
  \freenames{x!\langle P \rangle} & := & \{ x \} \cup \{ P \} \\
  \freenames{P|Q} & := & \freenames{P} \cup \freenames{Q} \\
  \freenames{\dropn{x}} & := & \{ x \}
\end{eqnarray*}

The bound names of a process, $\boundnames{P}$, are those names occurring in $P$
that are not free. For example, in $x?(y).0$, the name $x$ is free, while $y$ is bound.

\begin{mathpar}
  \inferrule* [lab=monoidal-laws] {} { P|Q \equiv Q|P \and P|0 \equiv P \and P|(Q|R) \equiv (P|Q)|R }
\end{mathpar}

\begin{mathpar}
  \inferrule* [lab=alpha-equivalence] {} { (x)P \equiv (y)P\{y/x\} \and y \not\in \freenames{P} }
\end{mathpar}

\begin{definition}
Then two processes, $P,Q$, are alpha-equivalent if $P = Q\{\vec{y}/\vec{x}\}$ for
some $\vec{x} \in \boundnames{Q},\vec{y} \in \boundnames{P}$, where $Q\{\vec{y}/\vec{x}\}$
denotes the capture-avoiding substitution of $\vec{y}$ for $\vec{x}$ in $Q$.
\end{definition}

\begin{definition}
  The {\em structural congruence} \cite{SangiorgiWalker} , $\equiv$,
  between processes is the least congruence containing
  alpha-equivalence, satisfying the abelian monoid laws
  (associativity, commutativity and $\pzero$ as identity) for parallel
  composition $|$ and for summation $+$.
\end{definition}

\subsection{Name equivalence}

We take name equivalence, written $\nameeq$, to be the smallest
equivalence relation generated by the following rules.

\begin{mathpar}
\inferrule*[lab=Quote-drop]
{ }
{ \quotep{@{x}} \nameeq x }

\inferrule*[lab=Struct-equiv]
{ P \scong Q }
{ \quotep{P} \nameeq \quotep{Q} }
\end{mathpar}

The astute reader will have noticed that the mutual recursion of names
and processes imposes a mutual recursion on alpha-equivalence and
structural equivalence via name-equivalence. Fortunately, all of this
works out pleasantly and we may calculate in the natural way, free of
concern. The reader interested in the details is referred to the
appendix \ref{appendix:rho_details}.

\subsection{Substitution}

We use $\Proc$ for the set of processes, $\QProc$ for the set of
names, and $\id{\{}\vec{y} / \vec{x} \id{\}}$ to denote partial maps,
$s : \QProc \rightarrow \QProc$. A map, $s$ lifts, uniquely, to a map
on process terms, $\widehat{s} : \Proc \rightarrow \Proc$ by the
following equations.

\begin{mathpar}
  (0) \psubstp{Q}{P} := 0 \\
  (R \juxtap S) \psubstp{Q}{P}
  :=    
  (R)\psubstp{Q}{P} \juxtap (S) \psubstp{Q}{P} \\
  (x?(y).R) \psubstp{Q}{P}    
  :=    
  (x)\substp{Q}{P} (z)\concat( (R \psubstn{z}{y}) \psubstp{Q}{P} ) \\
  (\lift{x}{R}) \psubstp{Q}{P}  
  :=
  \lift{(x)\substp{Q}{P}}{ R \psubstp{Q}{P} } \\
%   (\dropn{x})  \psubstp{Q}{P}       
%   := 
%   \left\{ 
%     \begin{array}{ccc} 
%       \dropn{\quotep{Q}} & & x \nameeq \quotep{P} \\
%       \dropn{x} & & otherwise \\
%     \end{array}
%   \right. 
  (\dropn{x})  \psubstp{Q}{P}       
  := 
  \left\{ 
    \begin{array}{ccc} 
      Q & & x \nameeq \quotep{P} \\
      \dropn{x} & & otherwise \\
    \end{array}
  \right.
\end{mathpar}
 

where

\begin{eqnarray}
  (x)\id{\{} \lpquote Q \rpquote / \lpquote P \rpquote \id{\}}            = 
  \left\{ 
    \begin{array}{ccc}
      \lpquote Q \rpquote & & x \nameeq \lpquote P \rpquote \\
      x & & otherwise \\
    \end{array}
  \right. \nonumber
\end{eqnarray}

and $z$ is chosen distinct from $\quotep{P}$, $\quotep{Q}$, the free
names in $Q$, and all the names in $R$. Our $\alpha$-equivalence will
be built in the standard way from this substitution.

\begin{remark}\label{rem:no_self_referential_names}
  One consequence of these definitions is that $\forall P. \quotep{P}
  \not\in \freenames{P}$.
\end{remark}

\subsection{ Dynamic quote: an example }

Anticipating something of what's to come, consider applying the
substitution, $\widehat{\id{\{}u / z \id{\}}}$, to the following pair
of processes, $\lift{w}{y!(z)}$ and $w[ \lpquote y!(z) \rpquote ]$.

\begin{eqnarray}
	\lift{w}{y!(z)}\widehat{\id{\{}u / z \id{\}}}
		& = &
		\lift{w}{y!(u)} \nonumber\\
	w[ \lpquote y!(z) \rpquote ] \widehat{ \id{\{}u / z \id{\}} }
		& = &
		w[ \lpquote y!(z) \rpquote ] \nonumber
\end{eqnarray}

Because the body of the process between quotes is impervious to
substitution, we get radically different answers. In fact, by
examining the first process in an input context,
e.g. $x?(z).\lift{w}{y!(z)}$, we see that the process under the lift
operator may be shaped by prefixed inputs binding a name inside it. In
this sense, the lift operator will be seen as a way to dynamically
construct processes before reifying them as names.

Finally equipped with these standard features we can present the
dynamics of the calculus.

\subsubsection{Operational semantics} 

Finally, we introduce the computational dynamics. What marks these
algebras as distinct from other more traditionally studied algebraic
structures, e.g. vector spaces or polynomial rings, is the manner in
which dynamics is captured. In traditional structures, dynamics is typically
expressed through morphisms between such structures, as in linear maps
between vector spaces or morphisms between rings. In algebras
associated with the semantics of computation, the dynamics is
expressed as part of the algebraic structure itself, through a
reduction reduction relation typically denoted by $\red$. Below, we
give a recursive presentation of this relation for the calculus used
in the encoding.

$\red \subseteq \pi \times \pi$
$\red : \pi \to \mathcal{P}(\pi)$

\begin{mathpar}
  \inferrule* [lab=Comm] { \textsf{match}( x_{src}, x_{trgt} ) } { x_{trgt}?(y)P \; | \; x_{src}!\langle {Q} \rangle \red P\{\quotep{Q}/y}\} }
  \and \\
  \inferrule* [lab=Par] {{P} \red {P}'} {{{P} | {Q}} \red {{P}' | {Q}}}
  \and
  \inferrule* [lab=Equiv]{{{P} \scong {P}'} \andalso {{P}' \red {Q}'} \andalso {{Q}' \scong {Q}}}{{P} \red {Q}}
\end{mathpar}

\begin{eqnarray*}
  match_{\equiv} (\quotep{P},\quotep{Q}) & := & P \equiv Q \\
  match_{\dagger}(\quotep{P},\quotep{Q}) & := & \forall R. P|Q \red^{*} R => R \red^{*} 0 \\
  match_{K}(\quotep{P},\quotep{Q}) & := & K \mbox{ for some context } K
\end{eqnarray*}

$u?(x)P | u!\langle Q \rangle \red P\{\quotep{Q}/x\}$

%We write $\wred$ for $\red^*$, and $P\red$ if $\exists Q $ such that $ P \red Q$.
We write $P\red$ if $\exists Q $ such that $ P \red Q$ and $P\not\red$, otherwise.

\section{Replication}

As mentioned before, it is known that replication (and hence
recursion) can be implemented in a higher-order process algebra
\cite{SangiorgiWalker}. As our first example of calculation with the
machinery thus far presented we give the construction explicitly in
the {\rhoc}.

\begin{eqnarray}
	D_{x} & := & \prefix{x}{y}{(\binpar{\outputp{x}{y}}{@{y}})} \nonumber\\
	\bangp_{x}{P} & := & \binpar{{x}!\langle{\binpar{D_{x}}{P}}\rangle}{D_{x}} \nonumber
\end{eqnarray}

\begin{eqnarray}
	\bangp_{x}{P} & & \nonumber\\
	=
	& {x}!\langle{(\prefix{x}{y}{(\outputp{x}{y} | @{y})) | P}}\rangle 
	      | \prefix{x}{y}{(\outputp{x}{y} | @{y})} & \nonumber\\
	\red
	& (\outputp{x}{y} | @{y})\substn{\quotep{(\prefix{x}{y}{(@{y} | \outputp{x}{y})) | P}}}{y} & \nonumber\\
	=
	& \outputp{x}{\quotep{(\prefix{x}{y}{(\outputp{x}{y} | @{y})) | P}}}
	  | {(\prefix{x}{y}{(\outputp{x}{y} | @{y})) | P}} & \nonumber\\
	\red
	& \ldots & \nonumber\\
	\red^*
	& P | P | \ldots & \nonumber
\end{eqnarray}

Of course, this encoding, as an implementation, runs away, unfolding
$\bangp{P}$ eagerly. A lazier and more implementable replication
operator, restricted to input-guarded processes, may be obtained as follows.

\begin{eqnarray}
\bangp{\prefix{u}{v}{P}} 
	:= 
	\binpar{\lift{x}{\prefix{u}{v}{(\binpar{D(x)}{P})}}}{D(x)} \nonumber
\end{eqnarray}

\begin{remark}
  Note that the lazier definition still does not deal with summation
  or mixed summation (i.e. sums over input and output). The reader is
  invited to construct definitions of replication that deal with these
  features. 

  Further, the definitions are parameterized in a name, $x$. Can you,
  gentle reader, make a definition that eliminates this parameter and
  guarantees no accidental interaction between the replication
  machinery and the process being replicated -- i.e. no accidental
  sharing of names used by the process to get its work done and the
  name(s) used by the replication to effect copying. This latter
  revision of the definition of replication is crucial to obtaining
  the expected identity $!!P \sim !P$.
\end{remark}

\begin{remark}\label{rem:paradoxical_combinator}
  The reader familiar with the lambda calculus will have noticed the
  similarity between $D$ and the paradoxical combinator.

  [Ed. note: the existence of this seems to suggest we have to be more
  restrictive on the set of processes and names we admit if we are to
  support no-cloning.]
\end{remark}

\subsubsection{Bisimulation}

The computational dynamics gives rise to another kind of equivalence,
the equivalence of computational behavior. As previously mentioned
this is typically captured \emph{via} some form of bisimulation.

% The notion we use in this paper is weak barbed bisimulation
% \cite{milner91polyadicpi}.

The notion we use in this paper is derived from weak barbed
bisimulation \cite{milner91polyadicpi}. 

\begin{definition}
An \emph{observation relation}, $\downarrow_{\mathcal N}$, over a set
of names, $\mathcal N$, is the smallest relation satisfying the rules
below.

\infrule[Out-barb]{y \in {\mathcal N}, \; x \nameeq y}
		  {\outputp{x}{v} \downarrow_{\mathcal N} x}
\infrule[Par-barb]{\mbox{$P\downarrow_{\mathcal N} x$ or $Q\downarrow_{\mathcal N} x$}}
		  {\binpar{P}{Q} \downarrow_{\mathcal N} x}

We write $P \Downarrow_{\mathcal N} x$ if there is $Q$ such that 
$P \wred Q$ and $Q \downarrow_{\mathcal N} x$.
\end{definition}

\begin{definition}
%\label{def.bbisim}
An  ${\mathcal N}$-\emph{barbed bisimulation} over a set of names, ${\mathcal N}$, is a symmetric binary relation 
${\mathcal S}_{\mathcal N}$ between agents such that $P\rel{S}_{\mathcal N}Q$ implies:
\begin{enumerate}
\item If $P \red P'$ then $Q \wred Q'$ and $P'\rel{S}_{\mathcal N} Q'$.
\item If $P\downarrow_{\mathcal N} x$, then $Q\Downarrow_{\mathcal N} x$.
\end{enumerate}
$P$ is ${\mathcal N}$-barbed bisimilar to $Q$, written
$P \wbbisim_{\mathcal N} Q$, if $P \rel{S}_{\mathcal N} Q$ for some ${\mathcal N}$-barbed bisimulation ${\mathcal S}_{\mathcal N}$.
\end{definition}

$\mathcal{R} \subseteq \pi \times \pi$

$P \mathcal{R} Q => \forall P'. P \red P' \Rightarrow \exists Q'. Q \red Q', P' \mathcal{R} Q'$

$P \vdash x \Rightarrow Q \vdash x$

\begin{mathpar}
  \inferrule*[lab=Out-barb]{x \nameeq y}{{y}!\langle{Q}\rangle \vdash x}
  \and
  \inferrule*[lab=Par-barb]{\mbox{$P\vdash x$ or $Q\vdash x$}}{\binpar{P}{Q} \vdash x}
\end{mathpar}

\subsubsection{Contexts}

One of the principle advantages of computational calculi like the
$\pi$-calculus is a well-defined notion of context,
contextual-equivalence and a correlation between
contextual-equivalence and notions of bisimulation. The notion of
context allows the decomposition of a process into (sub-)process and
its syntactic environment, its context. Thus, a context may be
thought of as a process with a ``hole'' (written $\Box$) in it. The
application of a context $M$ to a process $P$, written $M[P]$, is
tantamount to filling the hole in $M$ with $P$. In this paper we do
not need the full weight of this theory, but do make use of the notion
of context in the proof the main theorem. 

\begin{mathpar}
  \inferrule* [lab=summation] {} {{M_{M},M_{N}} \bc \Box \;|\; x.M_{A} \;|\; M_{M}+M_{N}}
  \and
  \inferrule* [lab=agent] {} {{M_{A}} \bc (\vec{x})M_{P} \;| \; \clift{P_0,\ldots,M_{P},\ldots,P_N}}
  \and \\
  \inferrule* [lab=process] {} {{M_{P}} \bc M_{N} \;| \;P|M_{P} }
\end{mathpar} 

\begin{mathpar}
  \inferrule* [lab=sychronization] {} {M_{N} \bc \Box \;|\; x?M_{F} \;|\; x!M_{C}}
  \and
  \inferrule* [lab=abstraction] {} {{M_{F}} \bc (x)M_{P} }
  \and
  \inferrule* [lab=concretion] {} {{M_{C}} \bc \langle M_{P} \rangle }
  \and \\
  \inferrule* [lab=process] {} {{M_{P}} \bc M_{N} \;| \;P|M_{P} }
\end{mathpar}

\begin{definition}[contextual application] Given a context $M$, and
  process $P$, we define the \emph{contextual application}, $M[P] :=
  M\{P/\Box\}$. That is, the contextual application of M to P is the
  substitution of $P$ for $\Box$ in $M$.
\end{definition}

$\meaningof{-} : L \to \mathcal{P}(\pi)$

\begin{mathpar}
  \inferrule* [lab=collection] {} {\meaningof{true} = \pi, \and \meaningof{~E} = \pi \setminus \meaningof{E}, \and \meaningof{E_{1} \& E_{2}} = \meaningof{E_{1}} \cap \meaningof{E_{2}}}
\end{mathpar}

\begin{mathpar}
  \inferrule* [lab=structure] {} {\meaningof{0} = \{ P \in \pi | P \equiv 0 \}, \and \\ \meaningof{E_1 | E_2} = \{ P \in \pi | P \equiv P_{1} | P_{2}, P_{1} \in \meaningof{E_{1}}, P_{2} \in \meaningof{E_2}\} }
\end{mathpar}

\begin{mathpar}
 \inferrule* [lab=behavior] {} {\meaningof{\langle a?b \rangle E} = \{ P \in \pi | P \equiv Q | u?(y)P', \\ \and \\\\ \and \\ \;\;\; u \in \meaningof{a}, \forall z.P'\{z/y\} \in \meaningof{E\{z/b\}}\}, \and \\ \meaningof{a!E} = \{ P \in \pi | P \equiv Q | x!\langle P' \rangle, x \in \meaningof{a} P' \in \meaningof{E}\} }
\end{mathpar}

\begin{mathpar}
 \inferrule* [lab=nominal] {} {\meaningof{\quotep{E}} = \{ \quotep{P} \in \quotep{\pi} | P \in \meaningof{E} \}, \and \meaningof{\quotep{P}} = \{ \quotep{Q} \in \quotep{\pi} | P \equiv Q \} \and \\ \meaningof{@\quotep{E}} = \{ P \in \pi | P \equiv @x, x \in \meaningof{E} \}}
\end{mathpar}

\begin{eqnarray*}
  \\
  \meaningof{-} : TS \to ST
\end{eqnarray*}

\begin{eqnarray*}
  \\
  L : TS \to ST
\end{eqnarray*}

\begin{eqnarray*}
  \\
  P \models E \iff P \in \meaningof{E}
\end{eqnarray*}

\begin{eqnarray*}
  P \approx_{L} Q \iff \forall E \in L. P \models E \iff Q \models E
\end{eqnarray*}

\begin{eqnarray*}
  P \approx_{K} Q
\end{eqnarray*}

\begin{eqnarray*}
  P \approx Q
\end{eqnarray*}

$\approx_{K} = \approx = \approx_{L}$

\subsubsection{Contextual duality}

Note that contexts extend the quotation operation to a family of
operations from processes to names. Given a context, $M$, we can
define a \emph{nominal context}, $\quotep{M}$ by $\quotep{M}[P] :=
\quotep{M[P]}$. To foreshadow what is to come we observe that these
operations enjoy a duality with processes very much like the duality
between vectors and maps from vectors to scalars.

Further, because the calculus is essentially higher-order, we have a
correspondence between contexts and processes. More specifically,
given a name $x$ and a context $M$ we can construct $M^{*}_{x}$ such
that 

\begin{mathpar}
  M^{*}_{x} | \lift{x}{P} \red M[P]
\end{mathpar}

namely,

\begin{mathpar}
  M^{*}_{x} := x?(u).M[\dropn{u}]
\end{mathpar}

The dependence of $M^{*}_{x}$ on a name makes it an abstraction, 

\begin{mathpar}
  M^{*} := (x)x?(u).M[\dropn{u}]
\end{mathpar}

\subsection{Additional notation}

It will sometimes be convenient to denote the process a name
quotes. We already have the notation $x = \quotep{P}$, but it will be
convenient to introduce an alternate notation, $\procn{x}$, when we
want to emphasize the connection to the use of the name. Note that, by
virtue of name equivalence, $\quotep{\procn{x}} \nameeq x$; so, the
notation is consistent with previous definitions.

Further, because names have structure it is possible to effect
substitutions on the basis of that structure. This means we need to
upgrade our notation for substitutions, which we accomplish by
adapting comprehension notation. Thus,

\begin{mathpar}
  P\{ y / x : x \in S \}
\end{mathpar}

is interpreted to mean the process derived from P by replacing (in a
capture-avoiding manner) each occurrence of $x$ in $S$ by $y$. For example,

\begin{mathpar}
  P\{ \quotep{\procn{x}|\procn{x}} / x : x \in \freenames{P} \}
\end{mathpar}

will replace each (occurrence) of a free name $x$ in $P$ by
$\quotep{\procn{x}|\procn{x}}$.

Also, we will avail ourselves of the notation $x^{L}$ and $x^{R}$ to
denote injections of a name into disjoint copies of the name
space. There are numerous ways to accomplish this. One example can be
found in \cite{MeredithR05}. This notation overloads to vectors of
names: $\vec{x}^{\pi} := (x_{i}^{\pi} \; : \; 0 \leq i < |\vec{x}| )$ where $\pi \in \{L,R\}$.

We also use $P^{\Box} := P|\Box$.

In \cite{MeredithR05} an interpretation of the new operator is
given. It turns out that there are several possible interpretations
all enjoying the requisite algebraic properties of the operator (see
\cite{milner91polyadicpi}). We will therefore make liberal use of
$(\nu\; \vec{x})P$.

% subsection the_syntax_and_semantics_of_the_notation_system (end)   

\input{qm2pi.qmops} 

\input{qm2pi.sterngerlach} 

\input{qm2pi.metric} 

% section concurrent_process_calculi (end)

%\input{qm2pi.proofsketch}

% section proof sketch (end)

%\input{qm2pi.slviaknots} 

% section spatial logic via knots (end)

\input{qm2pi.conclusion}

% section conclusion (end)

%\input{qm2pi.dtcodes} 

% section wiring algorithm (end)

\input{qm2pi.ack} 

% section acknowledgments (end)

\newpage


\bibliographystyle{plain}   
\bibliography{../../biblios/main.bib}

\input{qm2pi.rhodetails}

\end{document}

 

%\documentclass[12pt]{llncs}
%\documentclass{jktr}

\usepackage[pdftex]{hyperref}                   
\usepackage {listings}
\usepackage {mathpartir}
\usepackage{bcprules}
%\usepackage{listings}
                       
\usepackage{graphicx} 
%\usepackage[margins=2.5cm,nohead,nofoot]{geometry}
%\usepackage{geometry}
\usepackage{amsfonts}
\usepackage{amstext}
\usepackage{latexsym}
\usepackage{amssymb}
\usepackage{color}


%\include{myPreamble}
\include{qm2pi.local} 

%\ifpdf
%\usepackage[pdftex]{graphicx}
%\else
%\usepackage{graphicx}
%\fi

 % \ifpdf
%  \usepackage{pdfsync}
%  \if


%\title{Brief Article}
%\author{David F. Snyder}
%\author{L.G. Meredith}

%\address{Dept. of Math., Texas State University--San Marcos, San Marcos, TX 78666}
       
\pagestyle{empty}


\begin{document}

\lstset{language=[Objective]Caml,frame=shadowbox}

\input{qm2pi.front}

% section front matter (end)

\input{qm2pi.intro} 
 
% section introduction (end)

% \input{qm2pi.knotations} 

% section notation (end)

\input{qm2pi.process.calculi} 

% section concurrent_process_calculi_and_spatial_logics_ (end)
    
%\input{qm2pi.knots2pi} 

%\input{qm2pi.trefoil} 

%\input{qm2pi.mainthm} 

% subsection basic_interpretation (end)

%\input{qm2pi.rho.presentation} 
\subsection{The syntax and semantics of the notation system}\label{sub:the_syntax_and_semantics_of_the_notation_system} % (fold)

We now summarize a technical presentation of the calculus that
embodies our theory of dynamics. The typical presentation of such a
calculus follows the style of giving generators and relations on
them. The grammar, below, describing term constructors, freely
generates the set of processes, $\Proc$. This set is then quotiented
by a relation known as structural congruence and it is over this set
that the notion of dynamics is expressed. This presentation is
essentially that of \cite{MeredithR05} with the addition of
polyadicity and summation. For readability we have relegated some of
the technical subtleties to an appendix.

\subsubsection{Process grammar}\label{subsub:process_grammar}

\begin{mathpar}
  \inferrule* [lab=synchronization] {} {{M} \bc \pzero \;|\; x?F \;|\; x!C }
  \and
  \inferrule* [lab=abstraction] {} {{F} \bc (x)P}
  \and
  \inferrule* [lab=concretion] {} {{C} \bc \langle Q \rangle}
  \and
  \inferrule* [lab=process] {} {{P,Q} \bc M \;| \;P|Q \;|\; @{x}}
  \and
  \inferrule* [lab=name] {} {{x} \bc \quotep{P}}
\end{mathpar} 

Note that $\vec{x}$ (resp. $\vec{P}$) denotes a vector of names
(resp. processes) of length $|\vec{x}|$ (resp. $|\vec{P}|$). We adopt
the following useful abbreviations.

\begin{mathpar}
   x?(\vec{y}).P := x.(\vec{y})P \and  x\clift{\vec{P}} := x.\clift{\vec{P}}
   \and x!(y) := \lift{x}{\dropn{y}}
   \and \Pi_{i=0}^{n-1}P_i := P_0 | \ldots | P_{n-1}
\end{mathpar}

\subsubsection{Structural congruence}

\paragraph{Free and bound names and alpha-equivalence.} At the
core of structural equivalence is alpha-equivalence which identifies
process that are the same up to a change of variable. Formally, we
recognize the distinction between free and bound names. The free names
of a process, $\freenames{P}$, may be calculated recursively as
follows:

\begin{mathpar}
\freenames{\pzero} := \emptyset
  \and \\
  \freenames{x?(y).P} := \{ x \} \cup (\freenames{P} \setminus \{ y \})
  \and 
  \freenames{x!\langle P \rangle} := \{ x \} \cup \{ P \} 
  \and \\
  \freenames{P|Q} := \freenames{P} \cup \freenames{Q}
  \and \\
  \freenames{@{x}} := \{ x \}
\end{mathpar}

$\pi$
$\quotep{\pi}$

$\freenames{-} : \pi \to \mathcal{P}(\quotep{\pi})$

\begin{eqnarray*}
  \freenames{\pzero} & := & \emptyset \\
  \freenames{x?(y).P} & := & \{ x \} \cup (\freenames{P} \setminus \{ y \}) \\
  \freenames{x!\langle P \rangle} & := & \{ x \} \cup \{ P \} \\
  \freenames{P|Q} & := & \freenames{P} \cup \freenames{Q} \\
  \freenames{\dropn{x}} & := & \{ x \}
\end{eqnarray*}

The bound names of a process, $\boundnames{P}$, are those names occurring in $P$
that are not free. For example, in $x?(y).0$, the name $x$ is free, while $y$ is bound.

\begin{mathpar}
  \inferrule* [lab=monoidal-laws] {} { P|Q \equiv Q|P \and P|0 \equiv P \and P|(Q|R) \equiv (P|Q)|R }
\end{mathpar}

\begin{mathpar}
  \inferrule* [lab=alpha-equivalence] {} { (x)P \equiv (y)P\{y/x\} \and y \not\in \freenames{P} }
\end{mathpar}

\begin{definition}
Then two processes, $P,Q$, are alpha-equivalent if $P = Q\{\vec{y}/\vec{x}\}$ for
some $\vec{x} \in \boundnames{Q},\vec{y} \in \boundnames{P}$, where $Q\{\vec{y}/\vec{x}\}$
denotes the capture-avoiding substitution of $\vec{y}$ for $\vec{x}$ in $Q$.
\end{definition}

\begin{definition}
  The {\em structural congruence} \cite{SangiorgiWalker} , $\equiv$,
  between processes is the least congruence containing
  alpha-equivalence, satisfying the abelian monoid laws
  (associativity, commutativity and $\pzero$ as identity) for parallel
  composition $|$ and for summation $+$.
\end{definition}

\subsection{Name equivalence}

We take name equivalence, written $\nameeq$, to be the smallest
equivalence relation generated by the following rules.

\begin{mathpar}
\inferrule*[lab=Quote-drop]
{ }
{ \quotep{@{x}} \nameeq x }

\inferrule*[lab=Struct-equiv]
{ P \scong Q }
{ \quotep{P} \nameeq \quotep{Q} }
\end{mathpar}

The astute reader will have noticed that the mutual recursion of names
and processes imposes a mutual recursion on alpha-equivalence and
structural equivalence via name-equivalence. Fortunately, all of this
works out pleasantly and we may calculate in the natural way, free of
concern. The reader interested in the details is referred to the
appendix \ref{appendix:rho_details}.

\subsection{Substitution}

We use $\Proc$ for the set of processes, $\QProc$ for the set of
names, and $\id{\{}\vec{y} / \vec{x} \id{\}}$ to denote partial maps,
$s : \QProc \rightarrow \QProc$. A map, $s$ lifts, uniquely, to a map
on process terms, $\widehat{s} : \Proc \rightarrow \Proc$ by the
following equations.

\begin{mathpar}
  (0) \psubstp{Q}{P} := 0 \\
  (R \juxtap S) \psubstp{Q}{P}
  :=    
  (R)\psubstp{Q}{P} \juxtap (S) \psubstp{Q}{P} \\
  (x?(y).R) \psubstp{Q}{P}    
  :=    
  (x)\substp{Q}{P} (z)\concat( (R \psubstn{z}{y}) \psubstp{Q}{P} ) \\
  (\lift{x}{R}) \psubstp{Q}{P}  
  :=
  \lift{(x)\substp{Q}{P}}{ R \psubstp{Q}{P} } \\
%   (\dropn{x})  \psubstp{Q}{P}       
%   := 
%   \left\{ 
%     \begin{array}{ccc} 
%       \dropn{\quotep{Q}} & & x \nameeq \quotep{P} \\
%       \dropn{x} & & otherwise \\
%     \end{array}
%   \right. 
  (\dropn{x})  \psubstp{Q}{P}       
  := 
  \left\{ 
    \begin{array}{ccc} 
      Q & & x \nameeq \quotep{P} \\
      \dropn{x} & & otherwise \\
    \end{array}
  \right.
\end{mathpar}
 

where

\begin{eqnarray}
  (x)\id{\{} \lpquote Q \rpquote / \lpquote P \rpquote \id{\}}            = 
  \left\{ 
    \begin{array}{ccc}
      \lpquote Q \rpquote & & x \nameeq \lpquote P \rpquote \\
      x & & otherwise \\
    \end{array}
  \right. \nonumber
\end{eqnarray}

and $z$ is chosen distinct from $\quotep{P}$, $\quotep{Q}$, the free
names in $Q$, and all the names in $R$. Our $\alpha$-equivalence will
be built in the standard way from this substitution.

\begin{remark}\label{rem:no_self_referential_names}
  One consequence of these definitions is that $\forall P. \quotep{P}
  \not\in \freenames{P}$.
\end{remark}

\subsection{ Dynamic quote: an example }

Anticipating something of what's to come, consider applying the
substitution, $\widehat{\id{\{}u / z \id{\}}}$, to the following pair
of processes, $\lift{w}{y!(z)}$ and $w[ \lpquote y!(z) \rpquote ]$.

\begin{eqnarray}
	\lift{w}{y!(z)}\widehat{\id{\{}u / z \id{\}}}
		& = &
		\lift{w}{y!(u)} \nonumber\\
	w[ \lpquote y!(z) \rpquote ] \widehat{ \id{\{}u / z \id{\}} }
		& = &
		w[ \lpquote y!(z) \rpquote ] \nonumber
\end{eqnarray}

Because the body of the process between quotes is impervious to
substitution, we get radically different answers. In fact, by
examining the first process in an input context,
e.g. $x?(z).\lift{w}{y!(z)}$, we see that the process under the lift
operator may be shaped by prefixed inputs binding a name inside it. In
this sense, the lift operator will be seen as a way to dynamically
construct processes before reifying them as names.

Finally equipped with these standard features we can present the
dynamics of the calculus.

\subsubsection{Operational semantics} 

Finally, we introduce the computational dynamics. What marks these
algebras as distinct from other more traditionally studied algebraic
structures, e.g. vector spaces or polynomial rings, is the manner in
which dynamics is captured. In traditional structures, dynamics is typically
expressed through morphisms between such structures, as in linear maps
between vector spaces or morphisms between rings. In algebras
associated with the semantics of computation, the dynamics is
expressed as part of the algebraic structure itself, through a
reduction reduction relation typically denoted by $\red$. Below, we
give a recursive presentation of this relation for the calculus used
in the encoding.

$\red \subseteq \pi \times \pi$
$\red : \pi \to \mathcal{P}(\pi)$

\begin{mathpar}
  \inferrule* [lab=Comm] { \textsf{match}( x_{src}, x_{trgt} ) } { x_{trgt}?(y)P \; | \; x_{src}!\langle {Q} \rangle \red P\{\quotep{Q}/y}\} }
  \and \\
  \inferrule* [lab=Par] {{P} \red {P}'} {{{P} | {Q}} \red {{P}' | {Q}}}
  \and
  \inferrule* [lab=Equiv]{{{P} \scong {P}'} \andalso {{P}' \red {Q}'} \andalso {{Q}' \scong {Q}}}{{P} \red {Q}}
\end{mathpar}

\begin{eqnarray*}
  match_{\equiv} (\quotep{P},\quotep{Q}) & := & P \equiv Q \\
  match_{\dagger}(\quotep{P},\quotep{Q}) & := & \forall R. P|Q \red^{*} R => R \red^{*} 0 \\
  match_{K}(\quotep{P},\quotep{Q}) & := & K \mbox{ for some context } K
\end{eqnarray*}

$u?(x)P | u!\langle Q \rangle \red P\{\quotep{Q}/x\}$

%We write $\wred$ for $\red^*$, and $P\red$ if $\exists Q $ such that $ P \red Q$.
We write $P\red$ if $\exists Q $ such that $ P \red Q$ and $P\not\red$, otherwise.

\section{Replication}

As mentioned before, it is known that replication (and hence
recursion) can be implemented in a higher-order process algebra
\cite{SangiorgiWalker}. As our first example of calculation with the
machinery thus far presented we give the construction explicitly in
the {\rhoc}.

\begin{eqnarray}
	D_{x} & := & \prefix{x}{y}{(\binpar{\outputp{x}{y}}{@{y}})} \nonumber\\
	\bangp_{x}{P} & := & \binpar{{x}!\langle{\binpar{D_{x}}{P}}\rangle}{D_{x}} \nonumber
\end{eqnarray}

\begin{eqnarray}
	\bangp_{x}{P} & & \nonumber\\
	=
	& {x}!\langle{(\prefix{x}{y}{(\outputp{x}{y} | @{y})) | P}}\rangle 
	      | \prefix{x}{y}{(\outputp{x}{y} | @{y})} & \nonumber\\
	\red
	& (\outputp{x}{y} | @{y})\substn{\quotep{(\prefix{x}{y}{(@{y} | \outputp{x}{y})) | P}}}{y} & \nonumber\\
	=
	& \outputp{x}{\quotep{(\prefix{x}{y}{(\outputp{x}{y} | @{y})) | P}}}
	  | {(\prefix{x}{y}{(\outputp{x}{y} | @{y})) | P}} & \nonumber\\
	\red
	& \ldots & \nonumber\\
	\red^*
	& P | P | \ldots & \nonumber
\end{eqnarray}

Of course, this encoding, as an implementation, runs away, unfolding
$\bangp{P}$ eagerly. A lazier and more implementable replication
operator, restricted to input-guarded processes, may be obtained as follows.

\begin{eqnarray}
\bangp{\prefix{u}{v}{P}} 
	:= 
	\binpar{\lift{x}{\prefix{u}{v}{(\binpar{D(x)}{P})}}}{D(x)} \nonumber
\end{eqnarray}

\begin{remark}
  Note that the lazier definition still does not deal with summation
  or mixed summation (i.e. sums over input and output). The reader is
  invited to construct definitions of replication that deal with these
  features. 

  Further, the definitions are parameterized in a name, $x$. Can you,
  gentle reader, make a definition that eliminates this parameter and
  guarantees no accidental interaction between the replication
  machinery and the process being replicated -- i.e. no accidental
  sharing of names used by the process to get its work done and the
  name(s) used by the replication to effect copying. This latter
  revision of the definition of replication is crucial to obtaining
  the expected identity $!!P \sim !P$.
\end{remark}

\begin{remark}\label{rem:paradoxical_combinator}
  The reader familiar with the lambda calculus will have noticed the
  similarity between $D$ and the paradoxical combinator.

  [Ed. note: the existence of this seems to suggest we have to be more
  restrictive on the set of processes and names we admit if we are to
  support no-cloning.]
\end{remark}

\subsubsection{Bisimulation}

The computational dynamics gives rise to another kind of equivalence,
the equivalence of computational behavior. As previously mentioned
this is typically captured \emph{via} some form of bisimulation.

% The notion we use in this paper is weak barbed bisimulation
% \cite{milner91polyadicpi}.

The notion we use in this paper is derived from weak barbed
bisimulation \cite{milner91polyadicpi}. 

\begin{definition}
An \emph{observation relation}, $\downarrow_{\mathcal N}$, over a set
of names, $\mathcal N$, is the smallest relation satisfying the rules
below.

\infrule[Out-barb]{y \in {\mathcal N}, \; x \nameeq y}
		  {\outputp{x}{v} \downarrow_{\mathcal N} x}
\infrule[Par-barb]{\mbox{$P\downarrow_{\mathcal N} x$ or $Q\downarrow_{\mathcal N} x$}}
		  {\binpar{P}{Q} \downarrow_{\mathcal N} x}

We write $P \Downarrow_{\mathcal N} x$ if there is $Q$ such that 
$P \wred Q$ and $Q \downarrow_{\mathcal N} x$.
\end{definition}

\begin{definition}
%\label{def.bbisim}
An  ${\mathcal N}$-\emph{barbed bisimulation} over a set of names, ${\mathcal N}$, is a symmetric binary relation 
${\mathcal S}_{\mathcal N}$ between agents such that $P\rel{S}_{\mathcal N}Q$ implies:
\begin{enumerate}
\item If $P \red P'$ then $Q \wred Q'$ and $P'\rel{S}_{\mathcal N} Q'$.
\item If $P\downarrow_{\mathcal N} x$, then $Q\Downarrow_{\mathcal N} x$.
\end{enumerate}
$P$ is ${\mathcal N}$-barbed bisimilar to $Q$, written
$P \wbbisim_{\mathcal N} Q$, if $P \rel{S}_{\mathcal N} Q$ for some ${\mathcal N}$-barbed bisimulation ${\mathcal S}_{\mathcal N}$.
\end{definition}

$\mathcal{R} \subseteq \pi \times \pi$

$P \mathcal{R} Q => \forall P'. P \red P' \Rightarrow \exists Q'. Q \red Q', P' \mathcal{R} Q'$

$P \vdash x \Rightarrow Q \vdash x$

\begin{mathpar}
  \inferrule*[lab=Out-barb]{x \nameeq y}{{y}!\langle{Q}\rangle \vdash x}
  \and
  \inferrule*[lab=Par-barb]{\mbox{$P\vdash x$ or $Q\vdash x$}}{\binpar{P}{Q} \vdash x}
\end{mathpar}

\subsubsection{Contexts}

One of the principle advantages of computational calculi like the
$\pi$-calculus is a well-defined notion of context,
contextual-equivalence and a correlation between
contextual-equivalence and notions of bisimulation. The notion of
context allows the decomposition of a process into (sub-)process and
its syntactic environment, its context. Thus, a context may be
thought of as a process with a ``hole'' (written $\Box$) in it. The
application of a context $M$ to a process $P$, written $M[P]$, is
tantamount to filling the hole in $M$ with $P$. In this paper we do
not need the full weight of this theory, but do make use of the notion
of context in the proof the main theorem. 

\begin{mathpar}
  \inferrule* [lab=summation] {} {{M_{M},M_{N}} \bc \Box \;|\; x.M_{A} \;|\; M_{M}+M_{N}}
  \and
  \inferrule* [lab=agent] {} {{M_{A}} \bc (\vec{x})M_{P} \;| \; \clift{P_0,\ldots,M_{P},\ldots,P_N}}
  \and \\
  \inferrule* [lab=process] {} {{M_{P}} \bc M_{N} \;| \;P|M_{P} }
\end{mathpar} 

\begin{mathpar}
  \inferrule* [lab=sychronization] {} {M_{N} \bc \Box \;|\; x?M_{F} \;|\; x!M_{C}}
  \and
  \inferrule* [lab=abstraction] {} {{M_{F}} \bc (x)M_{P} }
  \and
  \inferrule* [lab=concretion] {} {{M_{C}} \bc \langle M_{P} \rangle }
  \and \\
  \inferrule* [lab=process] {} {{M_{P}} \bc M_{N} \;| \;P|M_{P} }
\end{mathpar}

\begin{definition}[contextual application] Given a context $M$, and
  process $P$, we define the \emph{contextual application}, $M[P] :=
  M\{P/\Box\}$. That is, the contextual application of M to P is the
  substitution of $P$ for $\Box$ in $M$.
\end{definition}

$\meaningof{-} : L \to \mathcal{P}(\pi)$

\begin{mathpar}
  \inferrule* [lab=collection] {} {\meaningof{true} = \pi, \and \meaningof{~E} = \pi \setminus \meaningof{E}, \and \meaningof{E_{1} \& E_{2}} = \meaningof{E_{1}} \cap \meaningof{E_{2}}}
\end{mathpar}

\begin{mathpar}
  \inferrule* [lab=structure] {} {\meaningof{0} = \{ P \in \pi | P \equiv 0 \}, \and \\ \meaningof{E_1 | E_2} = \{ P \in \pi | P \equiv P_{1} | P_{2}, P_{1} \in \meaningof{E_{1}}, P_{2} \in \meaningof{E_2}\} }
\end{mathpar}

\begin{mathpar}
 \inferrule* [lab=behavior] {} {\meaningof{\langle a?b \rangle E} = \{ P \in \pi | P \equiv Q | u?(y)P', \\ \and \\\\ \and \\ \;\;\; u \in \meaningof{a}, \forall z.P'\{z/y\} \in \meaningof{E\{z/b\}}\}, \and \\ \meaningof{a!E} = \{ P \in \pi | P \equiv Q | x!\langle P' \rangle, x \in \meaningof{a} P' \in \meaningof{E}\} }
\end{mathpar}

\begin{mathpar}
 \inferrule* [lab=nominal] {} {\meaningof{\quotep{E}} = \{ \quotep{P} \in \quotep{\pi} | P \in \meaningof{E} \}, \and \meaningof{\quotep{P}} = \{ \quotep{Q} \in \quotep{\pi} | P \equiv Q \} \and \\ \meaningof{@\quotep{E}} = \{ P \in \pi | P \equiv @x, x \in \meaningof{E} \}}
\end{mathpar}

\begin{eqnarray*}
  \\
  \meaningof{-} : TS \to ST
\end{eqnarray*}

\begin{eqnarray*}
  \\
  L : TS \to ST
\end{eqnarray*}

\begin{eqnarray*}
  \\
  P \models E \iff P \in \meaningof{E}
\end{eqnarray*}

\begin{eqnarray*}
  P \approx_{L} Q \iff \forall E \in L. P \models E \iff Q \models E
\end{eqnarray*}

\begin{eqnarray*}
  P \approx_{K} Q
\end{eqnarray*}

\begin{eqnarray*}
  P \approx Q
\end{eqnarray*}

$\approx_{K} = \approx = \approx_{L}$

\subsubsection{Contextual duality}

Note that contexts extend the quotation operation to a family of
operations from processes to names. Given a context, $M$, we can
define a \emph{nominal context}, $\quotep{M}$ by $\quotep{M}[P] :=
\quotep{M[P]}$. To foreshadow what is to come we observe that these
operations enjoy a duality with processes very much like the duality
between vectors and maps from vectors to scalars.

Further, because the calculus is essentially higher-order, we have a
correspondence between contexts and processes. More specifically,
given a name $x$ and a context $M$ we can construct $M^{*}_{x}$ such
that 

\begin{mathpar}
  M^{*}_{x} | \lift{x}{P} \red M[P]
\end{mathpar}

namely,

\begin{mathpar}
  M^{*}_{x} := x?(u).M[\dropn{u}]
\end{mathpar}

The dependence of $M^{*}_{x}$ on a name makes it an abstraction, 

\begin{mathpar}
  M^{*} := (x)x?(u).M[\dropn{u}]
\end{mathpar}

\subsection{Additional notation}

It will sometimes be convenient to denote the process a name
quotes. We already have the notation $x = \quotep{P}$, but it will be
convenient to introduce an alternate notation, $\procn{x}$, when we
want to emphasize the connection to the use of the name. Note that, by
virtue of name equivalence, $\quotep{\procn{x}} \nameeq x$; so, the
notation is consistent with previous definitions.

Further, because names have structure it is possible to effect
substitutions on the basis of that structure. This means we need to
upgrade our notation for substitutions, which we accomplish by
adapting comprehension notation. Thus,

\begin{mathpar}
  P\{ y / x : x \in S \}
\end{mathpar}

is interpreted to mean the process derived from P by replacing (in a
capture-avoiding manner) each occurrence of $x$ in $S$ by $y$. For example,

\begin{mathpar}
  P\{ \quotep{\procn{x}|\procn{x}} / x : x \in \freenames{P} \}
\end{mathpar}

will replace each (occurrence) of a free name $x$ in $P$ by
$\quotep{\procn{x}|\procn{x}}$.

Also, we will avail ourselves of the notation $x^{L}$ and $x^{R}$ to
denote injections of a name into disjoint copies of the name
space. There are numerous ways to accomplish this. One example can be
found in \cite{MeredithR05}. This notation overloads to vectors of
names: $\vec{x}^{\pi} := (x_{i}^{\pi} \; : \; 0 \leq i < |\vec{x}| )$ where $\pi \in \{L,R\}$.

We also use $P^{\Box} := P|\Box$.

In \cite{MeredithR05} an interpretation of the new operator is
given. It turns out that there are several possible interpretations
all enjoying the requisite algebraic properties of the operator (see
\cite{milner91polyadicpi}). We will therefore make liberal use of
$(\nu\; \vec{x})P$.

% subsection the_syntax_and_semantics_of_the_notation_system (end)   

\input{qm2pi.qmops} 

\input{qm2pi.sterngerlach} 

\input{qm2pi.metric} 

% section concurrent_process_calculi (end)

%\input{qm2pi.proofsketch}

% section proof sketch (end)

%\input{qm2pi.slviaknots} 

% section spatial logic via knots (end)

\input{qm2pi.conclusion}

% section conclusion (end)

%\input{qm2pi.dtcodes} 

% section wiring algorithm (end)

\input{qm2pi.ack} 

% section acknowledgments (end)

\newpage


\bibliographystyle{plain}   
\bibliography{../../biblios/main.bib}

\input{qm2pi.rhodetails}

\end{document}

 

% subsection basic_interpretation (end)

%\input{qm2pi.rho.presentation} 
\subsection{The syntax and semantics of the notation system}\label{sub:the_syntax_and_semantics_of_the_notation_system} % (fold)

We now summarize a technical presentation of the calculus that
embodies our theory of dynamics. The typical presentation of such a
calculus follows the style of giving generators and relations on
them. The grammar, below, describing term constructors, freely
generates the set of processes, $\Proc$. This set is then quotiented
by a relation known as structural congruence and it is over this set
that the notion of dynamics is expressed. This presentation is
essentially that of \cite{MeredithR05} with the addition of
polyadicity and summation. For readability we have relegated some of
the technical subtleties to an appendix.

\subsubsection{Process grammar}\label{subsub:process_grammar}

\begin{mathpar}
  \inferrule* [lab=synchronization] {} {{M} \bc \pzero \;|\; x?F \;|\; x!C }
  \and
  \inferrule* [lab=abstraction] {} {{F} \bc (x)P}
  \and
  \inferrule* [lab=concretion] {} {{C} \bc \langle Q \rangle}
  \and
  \inferrule* [lab=process] {} {{P,Q} \bc M \;| \;P|Q \;|\; @{x}}
  \and
  \inferrule* [lab=name] {} {{x} \bc \quotep{P}}
\end{mathpar} 

Note that $\vec{x}$ (resp. $\vec{P}$) denotes a vector of names
(resp. processes) of length $|\vec{x}|$ (resp. $|\vec{P}|$). We adopt
the following useful abbreviations.

\begin{mathpar}
   x?(\vec{y}).P := x.(\vec{y})P \and  x\clift{\vec{P}} := x.\clift{\vec{P}}
   \and x!(y) := \lift{x}{\dropn{y}}
   \and \Pi_{i=0}^{n-1}P_i := P_0 | \ldots | P_{n-1}
\end{mathpar}

\subsubsection{Structural congruence}

\paragraph{Free and bound names and alpha-equivalence.} At the
core of structural equivalence is alpha-equivalence which identifies
process that are the same up to a change of variable. Formally, we
recognize the distinction between free and bound names. The free names
of a process, $\freenames{P}$, may be calculated recursively as
follows:

\begin{mathpar}
\freenames{\pzero} := \emptyset
  \and \\
  \freenames{x?(y).P} := \{ x \} \cup (\freenames{P} \setminus \{ y \})
  \and 
  \freenames{x!\langle P \rangle} := \{ x \} \cup \{ P \} 
  \and \\
  \freenames{P|Q} := \freenames{P} \cup \freenames{Q}
  \and \\
  \freenames{@{x}} := \{ x \}
\end{mathpar}

$\pi$
$\quotep{\pi}$

$\freenames{-} : \pi \to \mathcal{P}(\quotep{\pi})$

\begin{eqnarray*}
  \freenames{\pzero} & := & \emptyset \\
  \freenames{x?(y).P} & := & \{ x \} \cup (\freenames{P} \setminus \{ y \}) \\
  \freenames{x!\langle P \rangle} & := & \{ x \} \cup \{ P \} \\
  \freenames{P|Q} & := & \freenames{P} \cup \freenames{Q} \\
  \freenames{\dropn{x}} & := & \{ x \}
\end{eqnarray*}

The bound names of a process, $\boundnames{P}$, are those names occurring in $P$
that are not free. For example, in $x?(y).0$, the name $x$ is free, while $y$ is bound.

\begin{mathpar}
  \inferrule* [lab=monoidal-laws] {} { P|Q \equiv Q|P \and P|0 \equiv P \and P|(Q|R) \equiv (P|Q)|R }
\end{mathpar}

\begin{mathpar}
  \inferrule* [lab=alpha-equivalence] {} { (x)P \equiv (y)P\{y/x\} \and y \not\in \freenames{P} }
\end{mathpar}

\begin{definition}
Then two processes, $P,Q$, are alpha-equivalent if $P = Q\{\vec{y}/\vec{x}\}$ for
some $\vec{x} \in \boundnames{Q},\vec{y} \in \boundnames{P}$, where $Q\{\vec{y}/\vec{x}\}$
denotes the capture-avoiding substitution of $\vec{y}$ for $\vec{x}$ in $Q$.
\end{definition}

\begin{definition}
  The {\em structural congruence} \cite{SangiorgiWalker} , $\equiv$,
  between processes is the least congruence containing
  alpha-equivalence, satisfying the abelian monoid laws
  (associativity, commutativity and $\pzero$ as identity) for parallel
  composition $|$ and for summation $+$.
\end{definition}

\subsection{Name equivalence}

We take name equivalence, written $\nameeq$, to be the smallest
equivalence relation generated by the following rules.

\begin{mathpar}
\inferrule*[lab=Quote-drop]
{ }
{ \quotep{@{x}} \nameeq x }

\inferrule*[lab=Struct-equiv]
{ P \scong Q }
{ \quotep{P} \nameeq \quotep{Q} }
\end{mathpar}

The astute reader will have noticed that the mutual recursion of names
and processes imposes a mutual recursion on alpha-equivalence and
structural equivalence via name-equivalence. Fortunately, all of this
works out pleasantly and we may calculate in the natural way, free of
concern. The reader interested in the details is referred to the
appendix \ref{appendix:rho_details}.

\subsection{Substitution}

We use $\Proc$ for the set of processes, $\QProc$ for the set of
names, and $\id{\{}\vec{y} / \vec{x} \id{\}}$ to denote partial maps,
$s : \QProc \rightarrow \QProc$. A map, $s$ lifts, uniquely, to a map
on process terms, $\widehat{s} : \Proc \rightarrow \Proc$ by the
following equations.

\begin{mathpar}
  (0) \psubstp{Q}{P} := 0 \\
  (R \juxtap S) \psubstp{Q}{P}
  :=    
  (R)\psubstp{Q}{P} \juxtap (S) \psubstp{Q}{P} \\
  (x?(y).R) \psubstp{Q}{P}    
  :=    
  (x)\substp{Q}{P} (z)\concat( (R \psubstn{z}{y}) \psubstp{Q}{P} ) \\
  (\lift{x}{R}) \psubstp{Q}{P}  
  :=
  \lift{(x)\substp{Q}{P}}{ R \psubstp{Q}{P} } \\
%   (\dropn{x})  \psubstp{Q}{P}       
%   := 
%   \left\{ 
%     \begin{array}{ccc} 
%       \dropn{\quotep{Q}} & & x \nameeq \quotep{P} \\
%       \dropn{x} & & otherwise \\
%     \end{array}
%   \right. 
  (\dropn{x})  \psubstp{Q}{P}       
  := 
  \left\{ 
    \begin{array}{ccc} 
      Q & & x \nameeq \quotep{P} \\
      \dropn{x} & & otherwise \\
    \end{array}
  \right.
\end{mathpar}
 

where

\begin{eqnarray}
  (x)\id{\{} \lpquote Q \rpquote / \lpquote P \rpquote \id{\}}            = 
  \left\{ 
    \begin{array}{ccc}
      \lpquote Q \rpquote & & x \nameeq \lpquote P \rpquote \\
      x & & otherwise \\
    \end{array}
  \right. \nonumber
\end{eqnarray}

and $z$ is chosen distinct from $\quotep{P}$, $\quotep{Q}$, the free
names in $Q$, and all the names in $R$. Our $\alpha$-equivalence will
be built in the standard way from this substitution.

\begin{remark}\label{rem:no_self_referential_names}
  One consequence of these definitions is that $\forall P. \quotep{P}
  \not\in \freenames{P}$.
\end{remark}

\subsection{ Dynamic quote: an example }

Anticipating something of what's to come, consider applying the
substitution, $\widehat{\id{\{}u / z \id{\}}}$, to the following pair
of processes, $\lift{w}{y!(z)}$ and $w[ \lpquote y!(z) \rpquote ]$.

\begin{eqnarray}
	\lift{w}{y!(z)}\widehat{\id{\{}u / z \id{\}}}
		& = &
		\lift{w}{y!(u)} \nonumber\\
	w[ \lpquote y!(z) \rpquote ] \widehat{ \id{\{}u / z \id{\}} }
		& = &
		w[ \lpquote y!(z) \rpquote ] \nonumber
\end{eqnarray}

Because the body of the process between quotes is impervious to
substitution, we get radically different answers. In fact, by
examining the first process in an input context,
e.g. $x?(z).\lift{w}{y!(z)}$, we see that the process under the lift
operator may be shaped by prefixed inputs binding a name inside it. In
this sense, the lift operator will be seen as a way to dynamically
construct processes before reifying them as names.

Finally equipped with these standard features we can present the
dynamics of the calculus.

\subsubsection{Operational semantics} 

Finally, we introduce the computational dynamics. What marks these
algebras as distinct from other more traditionally studied algebraic
structures, e.g. vector spaces or polynomial rings, is the manner in
which dynamics is captured. In traditional structures, dynamics is typically
expressed through morphisms between such structures, as in linear maps
between vector spaces or morphisms between rings. In algebras
associated with the semantics of computation, the dynamics is
expressed as part of the algebraic structure itself, through a
reduction reduction relation typically denoted by $\red$. Below, we
give a recursive presentation of this relation for the calculus used
in the encoding.

$\red \subseteq \pi \times \pi$
$\red : \pi \to \mathcal{P}(\pi)$

\begin{mathpar}
  \inferrule* [lab=Comm] { \textsf{match}( x_{src}, x_{trgt} ) } { x_{trgt}?(y)P \; | \; x_{src}!\langle {Q} \rangle \red P\{\quotep{Q}/y}\} }
  \and \\
  \inferrule* [lab=Par] {{P} \red {P}'} {{{P} | {Q}} \red {{P}' | {Q}}}
  \and
  \inferrule* [lab=Equiv]{{{P} \scong {P}'} \andalso {{P}' \red {Q}'} \andalso {{Q}' \scong {Q}}}{{P} \red {Q}}
\end{mathpar}

\begin{eqnarray*}
  match_{\equiv} (\quotep{P},\quotep{Q}) & := & P \equiv Q \\
  match_{\dagger}(\quotep{P},\quotep{Q}) & := & \forall R. P|Q \red^{*} R => R \red^{*} 0 \\
  match_{K}(\quotep{P},\quotep{Q}) & := & K \mbox{ for some context } K
\end{eqnarray*}

$u?(x)P | u!\langle Q \rangle \red P\{\quotep{Q}/x\}$

%We write $\wred$ for $\red^*$, and $P\red$ if $\exists Q $ such that $ P \red Q$.
We write $P\red$ if $\exists Q $ such that $ P \red Q$ and $P\not\red$, otherwise.

\section{Replication}

As mentioned before, it is known that replication (and hence
recursion) can be implemented in a higher-order process algebra
\cite{SangiorgiWalker}. As our first example of calculation with the
machinery thus far presented we give the construction explicitly in
the {\rhoc}.

\begin{eqnarray}
	D_{x} & := & \prefix{x}{y}{(\binpar{\outputp{x}{y}}{@{y}})} \nonumber\\
	\bangp_{x}{P} & := & \binpar{{x}!\langle{\binpar{D_{x}}{P}}\rangle}{D_{x}} \nonumber
\end{eqnarray}

\begin{eqnarray}
	\bangp_{x}{P} & & \nonumber\\
	=
	& {x}!\langle{(\prefix{x}{y}{(\outputp{x}{y} | @{y})) | P}}\rangle 
	      | \prefix{x}{y}{(\outputp{x}{y} | @{y})} & \nonumber\\
	\red
	& (\outputp{x}{y} | @{y})\substn{\quotep{(\prefix{x}{y}{(@{y} | \outputp{x}{y})) | P}}}{y} & \nonumber\\
	=
	& \outputp{x}{\quotep{(\prefix{x}{y}{(\outputp{x}{y} | @{y})) | P}}}
	  | {(\prefix{x}{y}{(\outputp{x}{y} | @{y})) | P}} & \nonumber\\
	\red
	& \ldots & \nonumber\\
	\red^*
	& P | P | \ldots & \nonumber
\end{eqnarray}

Of course, this encoding, as an implementation, runs away, unfolding
$\bangp{P}$ eagerly. A lazier and more implementable replication
operator, restricted to input-guarded processes, may be obtained as follows.

\begin{eqnarray}
\bangp{\prefix{u}{v}{P}} 
	:= 
	\binpar{\lift{x}{\prefix{u}{v}{(\binpar{D(x)}{P})}}}{D(x)} \nonumber
\end{eqnarray}

\begin{remark}
  Note that the lazier definition still does not deal with summation
  or mixed summation (i.e. sums over input and output). The reader is
  invited to construct definitions of replication that deal with these
  features. 

  Further, the definitions are parameterized in a name, $x$. Can you,
  gentle reader, make a definition that eliminates this parameter and
  guarantees no accidental interaction between the replication
  machinery and the process being replicated -- i.e. no accidental
  sharing of names used by the process to get its work done and the
  name(s) used by the replication to effect copying. This latter
  revision of the definition of replication is crucial to obtaining
  the expected identity $!!P \sim !P$.
\end{remark}

\begin{remark}\label{rem:paradoxical_combinator}
  The reader familiar with the lambda calculus will have noticed the
  similarity between $D$ and the paradoxical combinator.

  [Ed. note: the existence of this seems to suggest we have to be more
  restrictive on the set of processes and names we admit if we are to
  support no-cloning.]
\end{remark}

\subsubsection{Bisimulation}

The computational dynamics gives rise to another kind of equivalence,
the equivalence of computational behavior. As previously mentioned
this is typically captured \emph{via} some form of bisimulation.

% The notion we use in this paper is weak barbed bisimulation
% \cite{milner91polyadicpi}.

The notion we use in this paper is derived from weak barbed
bisimulation \cite{milner91polyadicpi}. 

\begin{definition}
An \emph{observation relation}, $\downarrow_{\mathcal N}$, over a set
of names, $\mathcal N$, is the smallest relation satisfying the rules
below.

\infrule[Out-barb]{y \in {\mathcal N}, \; x \nameeq y}
		  {\outputp{x}{v} \downarrow_{\mathcal N} x}
\infrule[Par-barb]{\mbox{$P\downarrow_{\mathcal N} x$ or $Q\downarrow_{\mathcal N} x$}}
		  {\binpar{P}{Q} \downarrow_{\mathcal N} x}

We write $P \Downarrow_{\mathcal N} x$ if there is $Q$ such that 
$P \wred Q$ and $Q \downarrow_{\mathcal N} x$.
\end{definition}

\begin{definition}
%\label{def.bbisim}
An  ${\mathcal N}$-\emph{barbed bisimulation} over a set of names, ${\mathcal N}$, is a symmetric binary relation 
${\mathcal S}_{\mathcal N}$ between agents such that $P\rel{S}_{\mathcal N}Q$ implies:
\begin{enumerate}
\item If $P \red P'$ then $Q \wred Q'$ and $P'\rel{S}_{\mathcal N} Q'$.
\item If $P\downarrow_{\mathcal N} x$, then $Q\Downarrow_{\mathcal N} x$.
\end{enumerate}
$P$ is ${\mathcal N}$-barbed bisimilar to $Q$, written
$P \wbbisim_{\mathcal N} Q$, if $P \rel{S}_{\mathcal N} Q$ for some ${\mathcal N}$-barbed bisimulation ${\mathcal S}_{\mathcal N}$.
\end{definition}

$\mathcal{R} \subseteq \pi \times \pi$

$P \mathcal{R} Q => \forall P'. P \red P' \Rightarrow \exists Q'. Q \red Q', P' \mathcal{R} Q'$

$P \vdash x \Rightarrow Q \vdash x$

\begin{mathpar}
  \inferrule*[lab=Out-barb]{x \nameeq y}{{y}!\langle{Q}\rangle \vdash x}
  \and
  \inferrule*[lab=Par-barb]{\mbox{$P\vdash x$ or $Q\vdash x$}}{\binpar{P}{Q} \vdash x}
\end{mathpar}

\subsubsection{Contexts}

One of the principle advantages of computational calculi like the
$\pi$-calculus is a well-defined notion of context,
contextual-equivalence and a correlation between
contextual-equivalence and notions of bisimulation. The notion of
context allows the decomposition of a process into (sub-)process and
its syntactic environment, its context. Thus, a context may be
thought of as a process with a ``hole'' (written $\Box$) in it. The
application of a context $M$ to a process $P$, written $M[P]$, is
tantamount to filling the hole in $M$ with $P$. In this paper we do
not need the full weight of this theory, but do make use of the notion
of context in the proof the main theorem. 

\begin{mathpar}
  \inferrule* [lab=summation] {} {{M_{M},M_{N}} \bc \Box \;|\; x.M_{A} \;|\; M_{M}+M_{N}}
  \and
  \inferrule* [lab=agent] {} {{M_{A}} \bc (\vec{x})M_{P} \;| \; \clift{P_0,\ldots,M_{P},\ldots,P_N}}
  \and \\
  \inferrule* [lab=process] {} {{M_{P}} \bc M_{N} \;| \;P|M_{P} }
\end{mathpar} 

\begin{mathpar}
  \inferrule* [lab=sychronization] {} {M_{N} \bc \Box \;|\; x?M_{F} \;|\; x!M_{C}}
  \and
  \inferrule* [lab=abstraction] {} {{M_{F}} \bc (x)M_{P} }
  \and
  \inferrule* [lab=concretion] {} {{M_{C}} \bc \langle M_{P} \rangle }
  \and \\
  \inferrule* [lab=process] {} {{M_{P}} \bc M_{N} \;| \;P|M_{P} }
\end{mathpar}

\begin{definition}[contextual application] Given a context $M$, and
  process $P$, we define the \emph{contextual application}, $M[P] :=
  M\{P/\Box\}$. That is, the contextual application of M to P is the
  substitution of $P$ for $\Box$ in $M$.
\end{definition}

$\meaningof{-} : L \to \mathcal{P}(\pi)$

\begin{mathpar}
  \inferrule* [lab=collection] {} {\meaningof{true} = \pi, \and \meaningof{~E} = \pi \setminus \meaningof{E}, \and \meaningof{E_{1} \& E_{2}} = \meaningof{E_{1}} \cap \meaningof{E_{2}}}
\end{mathpar}

\begin{mathpar}
  \inferrule* [lab=structure] {} {\meaningof{0} = \{ P \in \pi | P \equiv 0 \}, \and \\ \meaningof{E_1 | E_2} = \{ P \in \pi | P \equiv P_{1} | P_{2}, P_{1} \in \meaningof{E_{1}}, P_{2} \in \meaningof{E_2}\} }
\end{mathpar}

\begin{mathpar}
 \inferrule* [lab=behavior] {} {\meaningof{\langle a?b \rangle E} = \{ P \in \pi | P \equiv Q | u?(y)P', \\ \and \\\\ \and \\ \;\;\; u \in \meaningof{a}, \forall z.P'\{z/y\} \in \meaningof{E\{z/b\}}\}, \and \\ \meaningof{a!E} = \{ P \in \pi | P \equiv Q | x!\langle P' \rangle, x \in \meaningof{a} P' \in \meaningof{E}\} }
\end{mathpar}

\begin{mathpar}
 \inferrule* [lab=nominal] {} {\meaningof{\quotep{E}} = \{ \quotep{P} \in \quotep{\pi} | P \in \meaningof{E} \}, \and \meaningof{\quotep{P}} = \{ \quotep{Q} \in \quotep{\pi} | P \equiv Q \} \and \\ \meaningof{@\quotep{E}} = \{ P \in \pi | P \equiv @x, x \in \meaningof{E} \}}
\end{mathpar}

\begin{eqnarray*}
  \\
  \meaningof{-} : TS \to ST
\end{eqnarray*}

\begin{eqnarray*}
  \\
  L : TS \to ST
\end{eqnarray*}

\begin{eqnarray*}
  \\
  P \models E \iff P \in \meaningof{E}
\end{eqnarray*}

\begin{eqnarray*}
  P \approx_{L} Q \iff \forall E \in L. P \models E \iff Q \models E
\end{eqnarray*}

\begin{eqnarray*}
  P \approx_{K} Q
\end{eqnarray*}

\begin{eqnarray*}
  P \approx Q
\end{eqnarray*}

$\approx_{K} = \approx = \approx_{L}$

\subsubsection{Contextual duality}

Note that contexts extend the quotation operation to a family of
operations from processes to names. Given a context, $M$, we can
define a \emph{nominal context}, $\quotep{M}$ by $\quotep{M}[P] :=
\quotep{M[P]}$. To foreshadow what is to come we observe that these
operations enjoy a duality with processes very much like the duality
between vectors and maps from vectors to scalars.

Further, because the calculus is essentially higher-order, we have a
correspondence between contexts and processes. More specifically,
given a name $x$ and a context $M$ we can construct $M^{*}_{x}$ such
that 

\begin{mathpar}
  M^{*}_{x} | \lift{x}{P} \red M[P]
\end{mathpar}

namely,

\begin{mathpar}
  M^{*}_{x} := x?(u).M[\dropn{u}]
\end{mathpar}

The dependence of $M^{*}_{x}$ on a name makes it an abstraction, 

\begin{mathpar}
  M^{*} := (x)x?(u).M[\dropn{u}]
\end{mathpar}

\subsection{Additional notation}

It will sometimes be convenient to denote the process a name
quotes. We already have the notation $x = \quotep{P}$, but it will be
convenient to introduce an alternate notation, $\procn{x}$, when we
want to emphasize the connection to the use of the name. Note that, by
virtue of name equivalence, $\quotep{\procn{x}} \nameeq x$; so, the
notation is consistent with previous definitions.

Further, because names have structure it is possible to effect
substitutions on the basis of that structure. This means we need to
upgrade our notation for substitutions, which we accomplish by
adapting comprehension notation. Thus,

\begin{mathpar}
  P\{ y / x : x \in S \}
\end{mathpar}

is interpreted to mean the process derived from P by replacing (in a
capture-avoiding manner) each occurrence of $x$ in $S$ by $y$. For example,

\begin{mathpar}
  P\{ \quotep{\procn{x}|\procn{x}} / x : x \in \freenames{P} \}
\end{mathpar}

will replace each (occurrence) of a free name $x$ in $P$ by
$\quotep{\procn{x}|\procn{x}}$.

Also, we will avail ourselves of the notation $x^{L}$ and $x^{R}$ to
denote injections of a name into disjoint copies of the name
space. There are numerous ways to accomplish this. One example can be
found in \cite{MeredithR05}. This notation overloads to vectors of
names: $\vec{x}^{\pi} := (x_{i}^{\pi} \; : \; 0 \leq i < |\vec{x}| )$ where $\pi \in \{L,R\}$.

We also use $P^{\Box} := P|\Box$.

In \cite{MeredithR05} an interpretation of the new operator is
given. It turns out that there are several possible interpretations
all enjoying the requisite algebraic properties of the operator (see
\cite{milner91polyadicpi}). We will therefore make liberal use of
$(\nu\; \vec{x})P$.

% subsection the_syntax_and_semantics_of_the_notation_system (end)   

\section{Interpretation of QM}
\subsection{Supporting definitions}
\subsubsection{Multiplication}
\begin{mathpar}
  \quotep{Q} \cdot \quotep{R} := \quotep{Q|R}
  \and \\
  \quotep{Q} \cdot P := P\{ \quotep{Q|R} / \quotep{R} : \quotep{R} \in \freenames{P} \}
\end{mathpar}

\paragraph{Discussion}
The first line needs little explanation. The second line says that
each free name of the process is replaced with the multiplication of
that name by the scalar. Multiplication of a scalar (name) by a state
(process) results in a process all the names of which have been `moved
over' by parallel composition with the process the scalar
quotes. There is a subtlety that the bound names have to be
manipulated so that multiplied names aren't accidentally
captured. There are many ways to achieve this.

\begin{remark}\label{rem:multiplication_identities}
  The reader is invited to verify that for all $x,y,z \in \QProc$ and $P \in \Proc$
  \begin{mathpar}
    x \cdot \quotep{0} \equiv x 
    \and
    x \cdot y \equiv y \cdot x
    \and
    x \cdot (y \cdot z) \equiv (x \cdot y) \cdot z
    \and \\
    \quotep{0} \cdot P \equiv P
    \and \\
    x \cdot (y \cdot P) \equiv (x \cdot y) \cdot P
    \and \\
    x \cdot (P|Q) \equiv (x \cdot P) | (x \cdot Q)
    \and \\    
  \end{mathpar}
\end{remark}

\subsubsection{Tensor product}

We define a tensor product on processes by structural induction.

\paragraph{Tensor of sums} First note that all summations, including
$\pzero$ and sequence, can be written $\Sigma_{i} x_{i}.A_{i} +
\Sigma_{j} x_{j}.C_{j}$, where we have grouped input-guarded processes
together and output-guarded processes together.

Thus, we can define the tensor product of two summations, $N_{1}\otimes N_{2}$, where

\begin{mathpar}
  N_{1} := \Sigma_{i} x_{i}.A_{i} + \Sigma_{j} x_{j}.C_{j}
  \and
  N_{2} := \Sigma_{i'} y_{i'}.B_{i'} + \Sigma_{j'} y_{j'}.D_{j'} 
\end{mathpar}

as follows.

\begin{mathpar}
  \Sigma_{i} x_{i}.A_{i} + \Sigma_{j} x_{j}.C_{j} \otimes \Sigma_{i'}
  y_{i'}.B_{i'} + \Sigma_{j'} y_{j'}.D_{j'} 
  \and \\
  := \; \Sigma_{i} \Sigma_{i'} \quotep{\stackrel{\vee}{x_{i}}| \stackrel{\vee}{y_{i'}}}.(A_{i}\otimes B_{i'}) \; | \; \Sigma_{i'} \Sigma_{i} \quotep{\stackrel{\vee}{y_{i'}}|\stackrel{\vee}{x_{i}}}.(B_{i'}\otimes A_{i})
  \and
  \;\; | \;\; \Sigma_{j} \Sigma_{j'} \quotep{\stackrel{\vee}{x_{j}}|\stackrel{\vee}{y_{j'}}}.(A_{j}\otimes B_{j'}) \; | \; \Sigma_{j'} \Sigma_{j} \quotep{\stackrel{\vee}{y_{j'}}|\stackrel{\vee}{x_{j}}}.(B_{j'}\otimes A_{j})
\end{mathpar}

\begin{remark}
  Do we need to $x^{L}$ and $y^{R}$ for this construction as well?
\end{remark}

\paragraph{Tensor of parallel compositions} Next, we distribute tensor
over par.

\begin{mathpar}
  P_{1}|P_{2} \otimes Q_{1}|Q_{2} := (P_{1} \otimes Q_{1}) | (P_{1}
  \otimes Q_{2}) | (P_{2} \otimes Q_{1}) | (P_{2} \otimes Q_{2})
\end{mathpar}

\paragraph{Tensor with dropped names} We treat tensor of a
process with a dropped name as parallel composition.

\begin{mathpar}
  P \otimes \dropn{x} := P | \dropn{x}
\end{mathpar}

\paragraph{Tensor of agents}

Finally, we need to define tensor on agents. Note that the definition
of tensor on normal products only tensors inputs with inputs and
outputs with outputs. Thus, we only have to define the operation on
``homogeneous'' pairings.

\begin{mathpar}
  (\vec{x})P \otimes (\vec{y})Q
  \and \\
  := (x_{0}^{L}|y_{0}^{R},\ldots,x_{0}^{L}|y_{n}^{R},\ldots,x_{m}^{L}|y_{0}^{R},\ldots,x_{m}^{L}|y_{n}^R)(P\{ \vec{x}^{L}/\vec{x}\} \otimes Q \{ \vec{y}^{R}/\vec{y}\})
  \and \\
  \clift{\vec{P}} \otimes \clift{\vec{Q}}
  \and \\
  := \clift{P_{0}\otimes Q_{0},\ldots,P_{0}\otimes Q_{n},\ldots,P_{m}\otimes Q_{0},\ldots,P_{m}\otimes Q_{n}}
\end{mathpar}

\begin{remark}
  Observe that arities of tensored abstractions matches arities of
  tensored concretions if the original arities matched. Note also that
  the length of the arities corresponds to the increase in dimension
  we see in ordinary vector space tensor product.
\end{remark}

\begin{remark}
  Operationally, this definition distributes the tensor down to
  components ``linked'' by summation. Tensor over summation is
  intriguing in that it mixes names. Moreover, as a consequence of the
  way it mixes names we have the identities for all $x \in \QProc$ and
  $P,Q \in \Proc$

  \begin{mathpar}
    (x \cdot P) \otimes Q \equiv x \cdot (P \otimes Q) \equiv P \otimes (x \cdot Q)
    \and
    P \otimes \pzero \equiv P
  \end{mathpar}

  that the reader is invited to verify.
\end{remark}

\subsubsection{Annihilation}
\begin{mathpar}
  P^{\perp} := \{ Q | \forall R. P|Q \red^{*} R \Rightarrow R \red^{*} \pzero \}
  \and \\
  P^{\underline{\perp}} := \Sigma_{Q \in P^{\perp}} \quotep{Q}?(y).(\dropn{y}|Q) | \Sigma_{Q \in P^{\perp}} \quotep{Q}\clift{\Box}
\end{mathpar}

\paragraph{Discussion} The reader will note that $P^{\perp}$ is a
\emph{set} of processes, while $P^{\underline{\perp}}$ is a
\emph{context}. We call the set $P^{\perp}$ the \emph{annihilators} of
$P$. The parallel composition of a process in the annihilators of $P$
with $P$ will result in a process, the state space of which has all
paths eventually leading to $\pzero$. Execution may endure loops; but
under reasonable conditions of fairness (naturally guaranteed under
most notions of bisimulation) such a composite process cannot get
stuck in such a loop and will, eventually pop out and terminate.

The context $P^{\underline{\perp}}$ is ready and willing to ``take the
$P$ out of'' the process to which it is applied. It will effectively
transmit the code of the process to which it is applied to one of the
annihilators and run the process against it.

\subsubsection{Evaluation}
We fix $M$ a domain of fully abstract interpretation with an equality
coincident with bisimulation. We take $\meaningof{\cdot} : \Proc \to
M$ to be the map interpreting processes and $\nmeaningof{\cdot} : \M
\to Proc$ to be the map running the other way. Then we define

\begin{mathpar}
  \int P := \nmeaningof{\meaningof{P}}
\end{mathpar}

\paragraph{Discussion}
There are many fully abstract interpretations of Milner's
$\pi$-calculus. Any of them can be used as a basis for interpreting
the reflective calculus here. Equipped with such a domain it is
largely a matter of grinding through to check that the Yoneda
construction for the normalization-by-evaluation program can be
extended to this setting.

\begin{remark}
  The reader is invited to verify that $\int (P^{\underline{\perp}}[P]) = 0$.
\end{remark}

\subsection{Quantum mechanics}

Table \ref{tbl:core_qm_op_defns} gives the core operational definitions

\begin{table}[htp]\label{tbl:core_qm_op_defns}
  \center{
    \fbox{
      \begin{tabular}{c|c}
        quantum mechanics & process calculus \\
        \hline
        scalar & $x := \quotep{P}$ \\
        state vector & $\state{P} := P$ \\
        dual & $\state{P}^{*} := \event{P^{\underline{\perp}}} := \quotep{P^{\underline{\perp}}}[-]$ \\
        matrix & $ \Sigma_{\alpha} \state{P_{\alpha}}x_{\alpha}\event{Q_{\alpha}}$ \\
        vector addition & $\state{P} + \state{Q} := \state{P | Q}$ \\
        tensor product & $\state{P} \otimes \state{Q} := \state{P \otimes Q}$ \\
        inner product & $\innerprod{P}{Q} := \quotep{\int P^{\underline{\perp}}[Q]}$ \\
      \end{tabular}
    }
  }
  \caption{QM - operational definitions}
\end{table}

where

\begin{mathpar}
  \prmatrix{P}{Q} := \fprmatrix{P}{\quotep{\pzero}}{Q}
  \and
  \fprmatrix{P}{x}{Q} := (\state{P},x,\event{Q})
  \and
  (\fprmatrix{P}{x}{Q})(\state{R}) := x \cdot \innerprod{Q}{R} \cdot \state{P}
  \and
  (\fprmatrix{P}{x}{Q})(\event{R}) := x \cdot \innerprod{R}{P} \cdot \event{Q}
\end{mathpar}

\paragraph{Discussion}
As promised: vectors (aka states) are represented as processes; duals
as contextual duals; inner product definition should be compared with
standard inner product definition for ....

\begin{remark}
  Assuming $\int (P^{\underline{\perp}}[P]) = 0$, the reader is
  invited to verify that $(\fprmatrix{P}{x}{P})(\state{P}) = x \cdot \state{P}$.
\end{remark}

\begin{remark}
  The reader is invited to verify that $\innerprod{P}{Q}$ could
  equally well have been written $\quotep{\int \stackrel{\vee}{x}}$
  where $x = \event{P^{\underline{\perp}}}(Q)$.

  One of the motivations for this remark is that there is another way
  to factor these operations. We could package up evaluation in the dual:

  \begin{mathpar}
    \state{P}^{*} := \event{\int P^{\underline{\perp}}} := \quotep{\int P^{\underline{\perp}}}[-]
  \end{mathpar}

  and then have inner product defined by
  
  \begin{mathpar}
    \innerprod{P}{Q} := \event{P}(Q)
  \end{mathpar}

  Hopefully, experience with the calculations will provide guidance on
  the best factoring.
\end{remark}

\begin{remark}
  Assuming $\int (P^{\underline{\perp}}[P]) = 0$, the reader is
  invited to verify that $\forall P,Q. (\prmatrix{0}{Q})(\state{0}) =
  \state{0}$ and dually $(\prmatrix{P}{0})(\event{0}) = \event{0}$.
\end{remark}

\begin{remark}
  i'm a little worried that i don't (yet) have proper support for
  complex conjugacy. But, the observation above may give us a
  clue. According to Abramsky, it must be the case that the scalars
  are iso to the homset of the identity for the tensor -- which the
  observation above characterizes. 

  For now, we will simply bookmark the notion with $\overline{x}$.
\end{remark}

\subsubsection{Adjointness}

We need to give a definition of $(\cdot)^{\dagger}$ for matrices. The
obvious candidate definition is
\begin{mathpar}
(\Sigma_{\alpha}\fprmatrix{P_{\alpha}}{x_{\alpha}}{Q_{\alpha}})^{\dagger}
= \Sigma_{\alpha}\fprmatrix{(Q_{\alpha}^{\underline{\perp}})^{*}}{\overline{x}_{\alpha}}{P_{\alpha}^{\underline{\perp}}} 
\end{mathpar}

But, $(Q_{\alpha}^{\underline{\perp}})^{*}$ requires a name along
which to communicate the process to achieve the context application.

\subsubsection{Basis for a basis}
If processes label states and ``addition'' of states (a.k.a. vector
addition) is interpreted as parallel composition, what corresponds to
notions of linear independence and basis? Here, we recall that Yoshida
has developed a set of \emph{combinators} for an asynchronous verison
of Milner's $\pi$-calculus. These are a finite set of processes such
any process can be expressed as parallel composition of these
combinators together with liberal uses of the new operator and
replication. We can simply give a translation of these into the
present calculus and have reasonable expectation that the property
carries over. That is, that the resultant set allows to express all
processes via parallel composition. Note, however, that there is no
new operator or replication in this calculus. As a result, we expect
that the corresponding set is actually infinite. That is, we expect
that the space is actually infinite dimensional.

\begin{remark}
  The attentive reader may be a bit concerned. Certainly, the
  collection $S$, $K$ and $I$ is a finite set of
  combinators. Shouldn't we expect to see a finite set of combinators
  for an effectively equivalent system? i am very sympathetic to this
  critique and feel it warrants full attention. On the other hand, i
  also have in mind the following analogy. The natural numbers, as a
  monoid under addition, has exactly $1$ generator, while the natural
  numbers, as a monoid under multiplication, has countably many
  generators (the primes). We observe that the application of the
  lambda calculus is much less resource sensitive than the parallel
  composition of the $\pi$-calculus. Could it be the case that we have
  an analogy of the form
  
  \begin{mathpar}
    m + n : MN :: m*n : M|N
  \end{mathpar}

  giving a similar blow up in the set of ``primes''?  This is such a
  wonderful thought that, even if it's not true, i think it's worth
  writing down.
\end{remark}
 

\documentclass[12pt]{llncs}
%\documentclass{jktr}

\usepackage[pdftex]{hyperref}                   
\usepackage {listings}
\usepackage {mathpartir}
\usepackage{bcprules}
%\usepackage{listings}
                       
\usepackage{graphicx} 
%\usepackage[margins=2.5cm,nohead,nofoot]{geometry}
%\usepackage{geometry}
\usepackage{amsfonts}
\usepackage{amstext}
\usepackage{latexsym}
\usepackage{amssymb}
\usepackage{color}


%\include{myPreamble}
\include{qm2pi.local} 

%\ifpdf
%\usepackage[pdftex]{graphicx}
%\else
%\usepackage{graphicx}
%\fi

 % \ifpdf
%  \usepackage{pdfsync}
%  \if


%\title{Brief Article}
%\author{David F. Snyder}
%\author{L.G. Meredith}

%\address{Dept. of Math., Texas State University--San Marcos, San Marcos, TX 78666}
       
\pagestyle{empty}


\begin{document}

\lstset{language=[Objective]Caml,frame=shadowbox}

\input{qm2pi.front}

% section front matter (end)

\input{qm2pi.intro} 
 
% section introduction (end)

% \input{qm2pi.knotations} 

% section notation (end)

\input{qm2pi.process.calculi} 

% section concurrent_process_calculi_and_spatial_logics_ (end)
    
%\input{qm2pi.knots2pi} 

%\input{qm2pi.trefoil} 

%\input{qm2pi.mainthm} 

% subsection basic_interpretation (end)

%\input{qm2pi.rho.presentation} 
\subsection{The syntax and semantics of the notation system}\label{sub:the_syntax_and_semantics_of_the_notation_system} % (fold)

We now summarize a technical presentation of the calculus that
embodies our theory of dynamics. The typical presentation of such a
calculus follows the style of giving generators and relations on
them. The grammar, below, describing term constructors, freely
generates the set of processes, $\Proc$. This set is then quotiented
by a relation known as structural congruence and it is over this set
that the notion of dynamics is expressed. This presentation is
essentially that of \cite{MeredithR05} with the addition of
polyadicity and summation. For readability we have relegated some of
the technical subtleties to an appendix.

\subsubsection{Process grammar}\label{subsub:process_grammar}

\begin{mathpar}
  \inferrule* [lab=synchronization] {} {{M} \bc \pzero \;|\; x?F \;|\; x!C }
  \and
  \inferrule* [lab=abstraction] {} {{F} \bc (x)P}
  \and
  \inferrule* [lab=concretion] {} {{C} \bc \langle Q \rangle}
  \and
  \inferrule* [lab=process] {} {{P,Q} \bc M \;| \;P|Q \;|\; @{x}}
  \and
  \inferrule* [lab=name] {} {{x} \bc \quotep{P}}
\end{mathpar} 

Note that $\vec{x}$ (resp. $\vec{P}$) denotes a vector of names
(resp. processes) of length $|\vec{x}|$ (resp. $|\vec{P}|$). We adopt
the following useful abbreviations.

\begin{mathpar}
   x?(\vec{y}).P := x.(\vec{y})P \and  x\clift{\vec{P}} := x.\clift{\vec{P}}
   \and x!(y) := \lift{x}{\dropn{y}}
   \and \Pi_{i=0}^{n-1}P_i := P_0 | \ldots | P_{n-1}
\end{mathpar}

\subsubsection{Structural congruence}

\paragraph{Free and bound names and alpha-equivalence.} At the
core of structural equivalence is alpha-equivalence which identifies
process that are the same up to a change of variable. Formally, we
recognize the distinction between free and bound names. The free names
of a process, $\freenames{P}$, may be calculated recursively as
follows:

\begin{mathpar}
\freenames{\pzero} := \emptyset
  \and \\
  \freenames{x?(y).P} := \{ x \} \cup (\freenames{P} \setminus \{ y \})
  \and 
  \freenames{x!\langle P \rangle} := \{ x \} \cup \{ P \} 
  \and \\
  \freenames{P|Q} := \freenames{P} \cup \freenames{Q}
  \and \\
  \freenames{@{x}} := \{ x \}
\end{mathpar}

$\pi$
$\quotep{\pi}$

$\freenames{-} : \pi \to \mathcal{P}(\quotep{\pi})$

\begin{eqnarray*}
  \freenames{\pzero} & := & \emptyset \\
  \freenames{x?(y).P} & := & \{ x \} \cup (\freenames{P} \setminus \{ y \}) \\
  \freenames{x!\langle P \rangle} & := & \{ x \} \cup \{ P \} \\
  \freenames{P|Q} & := & \freenames{P} \cup \freenames{Q} \\
  \freenames{\dropn{x}} & := & \{ x \}
\end{eqnarray*}

The bound names of a process, $\boundnames{P}$, are those names occurring in $P$
that are not free. For example, in $x?(y).0$, the name $x$ is free, while $y$ is bound.

\begin{mathpar}
  \inferrule* [lab=monoidal-laws] {} { P|Q \equiv Q|P \and P|0 \equiv P \and P|(Q|R) \equiv (P|Q)|R }
\end{mathpar}

\begin{mathpar}
  \inferrule* [lab=alpha-equivalence] {} { (x)P \equiv (y)P\{y/x\} \and y \not\in \freenames{P} }
\end{mathpar}

\begin{definition}
Then two processes, $P,Q$, are alpha-equivalent if $P = Q\{\vec{y}/\vec{x}\}$ for
some $\vec{x} \in \boundnames{Q},\vec{y} \in \boundnames{P}$, where $Q\{\vec{y}/\vec{x}\}$
denotes the capture-avoiding substitution of $\vec{y}$ for $\vec{x}$ in $Q$.
\end{definition}

\begin{definition}
  The {\em structural congruence} \cite{SangiorgiWalker} , $\equiv$,
  between processes is the least congruence containing
  alpha-equivalence, satisfying the abelian monoid laws
  (associativity, commutativity and $\pzero$ as identity) for parallel
  composition $|$ and for summation $+$.
\end{definition}

\subsection{Name equivalence}

We take name equivalence, written $\nameeq$, to be the smallest
equivalence relation generated by the following rules.

\begin{mathpar}
\inferrule*[lab=Quote-drop]
{ }
{ \quotep{@{x}} \nameeq x }

\inferrule*[lab=Struct-equiv]
{ P \scong Q }
{ \quotep{P} \nameeq \quotep{Q} }
\end{mathpar}

The astute reader will have noticed that the mutual recursion of names
and processes imposes a mutual recursion on alpha-equivalence and
structural equivalence via name-equivalence. Fortunately, all of this
works out pleasantly and we may calculate in the natural way, free of
concern. The reader interested in the details is referred to the
appendix \ref{appendix:rho_details}.

\subsection{Substitution}

We use $\Proc$ for the set of processes, $\QProc$ for the set of
names, and $\id{\{}\vec{y} / \vec{x} \id{\}}$ to denote partial maps,
$s : \QProc \rightarrow \QProc$. A map, $s$ lifts, uniquely, to a map
on process terms, $\widehat{s} : \Proc \rightarrow \Proc$ by the
following equations.

\begin{mathpar}
  (0) \psubstp{Q}{P} := 0 \\
  (R \juxtap S) \psubstp{Q}{P}
  :=    
  (R)\psubstp{Q}{P} \juxtap (S) \psubstp{Q}{P} \\
  (x?(y).R) \psubstp{Q}{P}    
  :=    
  (x)\substp{Q}{P} (z)\concat( (R \psubstn{z}{y}) \psubstp{Q}{P} ) \\
  (\lift{x}{R}) \psubstp{Q}{P}  
  :=
  \lift{(x)\substp{Q}{P}}{ R \psubstp{Q}{P} } \\
%   (\dropn{x})  \psubstp{Q}{P}       
%   := 
%   \left\{ 
%     \begin{array}{ccc} 
%       \dropn{\quotep{Q}} & & x \nameeq \quotep{P} \\
%       \dropn{x} & & otherwise \\
%     \end{array}
%   \right. 
  (\dropn{x})  \psubstp{Q}{P}       
  := 
  \left\{ 
    \begin{array}{ccc} 
      Q & & x \nameeq \quotep{P} \\
      \dropn{x} & & otherwise \\
    \end{array}
  \right.
\end{mathpar}
 

where

\begin{eqnarray}
  (x)\id{\{} \lpquote Q \rpquote / \lpquote P \rpquote \id{\}}            = 
  \left\{ 
    \begin{array}{ccc}
      \lpquote Q \rpquote & & x \nameeq \lpquote P \rpquote \\
      x & & otherwise \\
    \end{array}
  \right. \nonumber
\end{eqnarray}

and $z$ is chosen distinct from $\quotep{P}$, $\quotep{Q}$, the free
names in $Q$, and all the names in $R$. Our $\alpha$-equivalence will
be built in the standard way from this substitution.

\begin{remark}\label{rem:no_self_referential_names}
  One consequence of these definitions is that $\forall P. \quotep{P}
  \not\in \freenames{P}$.
\end{remark}

\subsection{ Dynamic quote: an example }

Anticipating something of what's to come, consider applying the
substitution, $\widehat{\id{\{}u / z \id{\}}}$, to the following pair
of processes, $\lift{w}{y!(z)}$ and $w[ \lpquote y!(z) \rpquote ]$.

\begin{eqnarray}
	\lift{w}{y!(z)}\widehat{\id{\{}u / z \id{\}}}
		& = &
		\lift{w}{y!(u)} \nonumber\\
	w[ \lpquote y!(z) \rpquote ] \widehat{ \id{\{}u / z \id{\}} }
		& = &
		w[ \lpquote y!(z) \rpquote ] \nonumber
\end{eqnarray}

Because the body of the process between quotes is impervious to
substitution, we get radically different answers. In fact, by
examining the first process in an input context,
e.g. $x?(z).\lift{w}{y!(z)}$, we see that the process under the lift
operator may be shaped by prefixed inputs binding a name inside it. In
this sense, the lift operator will be seen as a way to dynamically
construct processes before reifying them as names.

Finally equipped with these standard features we can present the
dynamics of the calculus.

\subsubsection{Operational semantics} 

Finally, we introduce the computational dynamics. What marks these
algebras as distinct from other more traditionally studied algebraic
structures, e.g. vector spaces or polynomial rings, is the manner in
which dynamics is captured. In traditional structures, dynamics is typically
expressed through morphisms between such structures, as in linear maps
between vector spaces or morphisms between rings. In algebras
associated with the semantics of computation, the dynamics is
expressed as part of the algebraic structure itself, through a
reduction reduction relation typically denoted by $\red$. Below, we
give a recursive presentation of this relation for the calculus used
in the encoding.

$\red \subseteq \pi \times \pi$
$\red : \pi \to \mathcal{P}(\pi)$

\begin{mathpar}
  \inferrule* [lab=Comm] { \textsf{match}( x_{src}, x_{trgt} ) } { x_{trgt}?(y)P \; | \; x_{src}!\langle {Q} \rangle \red P\{\quotep{Q}/y}\} }
  \and \\
  \inferrule* [lab=Par] {{P} \red {P}'} {{{P} | {Q}} \red {{P}' | {Q}}}
  \and
  \inferrule* [lab=Equiv]{{{P} \scong {P}'} \andalso {{P}' \red {Q}'} \andalso {{Q}' \scong {Q}}}{{P} \red {Q}}
\end{mathpar}

\begin{eqnarray*}
  match_{\equiv} (\quotep{P},\quotep{Q}) & := & P \equiv Q \\
  match_{\dagger}(\quotep{P},\quotep{Q}) & := & \forall R. P|Q \red^{*} R => R \red^{*} 0 \\
  match_{K}(\quotep{P},\quotep{Q}) & := & K \mbox{ for some context } K
\end{eqnarray*}

$u?(x)P | u!\langle Q \rangle \red P\{\quotep{Q}/x\}$

%We write $\wred$ for $\red^*$, and $P\red$ if $\exists Q $ such that $ P \red Q$.
We write $P\red$ if $\exists Q $ such that $ P \red Q$ and $P\not\red$, otherwise.

\section{Replication}

As mentioned before, it is known that replication (and hence
recursion) can be implemented in a higher-order process algebra
\cite{SangiorgiWalker}. As our first example of calculation with the
machinery thus far presented we give the construction explicitly in
the {\rhoc}.

\begin{eqnarray}
	D_{x} & := & \prefix{x}{y}{(\binpar{\outputp{x}{y}}{@{y}})} \nonumber\\
	\bangp_{x}{P} & := & \binpar{{x}!\langle{\binpar{D_{x}}{P}}\rangle}{D_{x}} \nonumber
\end{eqnarray}

\begin{eqnarray}
	\bangp_{x}{P} & & \nonumber\\
	=
	& {x}!\langle{(\prefix{x}{y}{(\outputp{x}{y} | @{y})) | P}}\rangle 
	      | \prefix{x}{y}{(\outputp{x}{y} | @{y})} & \nonumber\\
	\red
	& (\outputp{x}{y} | @{y})\substn{\quotep{(\prefix{x}{y}{(@{y} | \outputp{x}{y})) | P}}}{y} & \nonumber\\
	=
	& \outputp{x}{\quotep{(\prefix{x}{y}{(\outputp{x}{y} | @{y})) | P}}}
	  | {(\prefix{x}{y}{(\outputp{x}{y} | @{y})) | P}} & \nonumber\\
	\red
	& \ldots & \nonumber\\
	\red^*
	& P | P | \ldots & \nonumber
\end{eqnarray}

Of course, this encoding, as an implementation, runs away, unfolding
$\bangp{P}$ eagerly. A lazier and more implementable replication
operator, restricted to input-guarded processes, may be obtained as follows.

\begin{eqnarray}
\bangp{\prefix{u}{v}{P}} 
	:= 
	\binpar{\lift{x}{\prefix{u}{v}{(\binpar{D(x)}{P})}}}{D(x)} \nonumber
\end{eqnarray}

\begin{remark}
  Note that the lazier definition still does not deal with summation
  or mixed summation (i.e. sums over input and output). The reader is
  invited to construct definitions of replication that deal with these
  features. 

  Further, the definitions are parameterized in a name, $x$. Can you,
  gentle reader, make a definition that eliminates this parameter and
  guarantees no accidental interaction between the replication
  machinery and the process being replicated -- i.e. no accidental
  sharing of names used by the process to get its work done and the
  name(s) used by the replication to effect copying. This latter
  revision of the definition of replication is crucial to obtaining
  the expected identity $!!P \sim !P$.
\end{remark}

\begin{remark}\label{rem:paradoxical_combinator}
  The reader familiar with the lambda calculus will have noticed the
  similarity between $D$ and the paradoxical combinator.

  [Ed. note: the existence of this seems to suggest we have to be more
  restrictive on the set of processes and names we admit if we are to
  support no-cloning.]
\end{remark}

\subsubsection{Bisimulation}

The computational dynamics gives rise to another kind of equivalence,
the equivalence of computational behavior. As previously mentioned
this is typically captured \emph{via} some form of bisimulation.

% The notion we use in this paper is weak barbed bisimulation
% \cite{milner91polyadicpi}.

The notion we use in this paper is derived from weak barbed
bisimulation \cite{milner91polyadicpi}. 

\begin{definition}
An \emph{observation relation}, $\downarrow_{\mathcal N}$, over a set
of names, $\mathcal N$, is the smallest relation satisfying the rules
below.

\infrule[Out-barb]{y \in {\mathcal N}, \; x \nameeq y}
		  {\outputp{x}{v} \downarrow_{\mathcal N} x}
\infrule[Par-barb]{\mbox{$P\downarrow_{\mathcal N} x$ or $Q\downarrow_{\mathcal N} x$}}
		  {\binpar{P}{Q} \downarrow_{\mathcal N} x}

We write $P \Downarrow_{\mathcal N} x$ if there is $Q$ such that 
$P \wred Q$ and $Q \downarrow_{\mathcal N} x$.
\end{definition}

\begin{definition}
%\label{def.bbisim}
An  ${\mathcal N}$-\emph{barbed bisimulation} over a set of names, ${\mathcal N}$, is a symmetric binary relation 
${\mathcal S}_{\mathcal N}$ between agents such that $P\rel{S}_{\mathcal N}Q$ implies:
\begin{enumerate}
\item If $P \red P'$ then $Q \wred Q'$ and $P'\rel{S}_{\mathcal N} Q'$.
\item If $P\downarrow_{\mathcal N} x$, then $Q\Downarrow_{\mathcal N} x$.
\end{enumerate}
$P$ is ${\mathcal N}$-barbed bisimilar to $Q$, written
$P \wbbisim_{\mathcal N} Q$, if $P \rel{S}_{\mathcal N} Q$ for some ${\mathcal N}$-barbed bisimulation ${\mathcal S}_{\mathcal N}$.
\end{definition}

$\mathcal{R} \subseteq \pi \times \pi$

$P \mathcal{R} Q => \forall P'. P \red P' \Rightarrow \exists Q'. Q \red Q', P' \mathcal{R} Q'$

$P \vdash x \Rightarrow Q \vdash x$

\begin{mathpar}
  \inferrule*[lab=Out-barb]{x \nameeq y}{{y}!\langle{Q}\rangle \vdash x}
  \and
  \inferrule*[lab=Par-barb]{\mbox{$P\vdash x$ or $Q\vdash x$}}{\binpar{P}{Q} \vdash x}
\end{mathpar}

\subsubsection{Contexts}

One of the principle advantages of computational calculi like the
$\pi$-calculus is a well-defined notion of context,
contextual-equivalence and a correlation between
contextual-equivalence and notions of bisimulation. The notion of
context allows the decomposition of a process into (sub-)process and
its syntactic environment, its context. Thus, a context may be
thought of as a process with a ``hole'' (written $\Box$) in it. The
application of a context $M$ to a process $P$, written $M[P]$, is
tantamount to filling the hole in $M$ with $P$. In this paper we do
not need the full weight of this theory, but do make use of the notion
of context in the proof the main theorem. 

\begin{mathpar}
  \inferrule* [lab=summation] {} {{M_{M},M_{N}} \bc \Box \;|\; x.M_{A} \;|\; M_{M}+M_{N}}
  \and
  \inferrule* [lab=agent] {} {{M_{A}} \bc (\vec{x})M_{P} \;| \; \clift{P_0,\ldots,M_{P},\ldots,P_N}}
  \and \\
  \inferrule* [lab=process] {} {{M_{P}} \bc M_{N} \;| \;P|M_{P} }
\end{mathpar} 

\begin{mathpar}
  \inferrule* [lab=sychronization] {} {M_{N} \bc \Box \;|\; x?M_{F} \;|\; x!M_{C}}
  \and
  \inferrule* [lab=abstraction] {} {{M_{F}} \bc (x)M_{P} }
  \and
  \inferrule* [lab=concretion] {} {{M_{C}} \bc \langle M_{P} \rangle }
  \and \\
  \inferrule* [lab=process] {} {{M_{P}} \bc M_{N} \;| \;P|M_{P} }
\end{mathpar}

\begin{definition}[contextual application] Given a context $M$, and
  process $P$, we define the \emph{contextual application}, $M[P] :=
  M\{P/\Box\}$. That is, the contextual application of M to P is the
  substitution of $P$ for $\Box$ in $M$.
\end{definition}

$\meaningof{-} : L \to \mathcal{P}(\pi)$

\begin{mathpar}
  \inferrule* [lab=collection] {} {\meaningof{true} = \pi, \and \meaningof{~E} = \pi \setminus \meaningof{E}, \and \meaningof{E_{1} \& E_{2}} = \meaningof{E_{1}} \cap \meaningof{E_{2}}}
\end{mathpar}

\begin{mathpar}
  \inferrule* [lab=structure] {} {\meaningof{0} = \{ P \in \pi | P \equiv 0 \}, \and \\ \meaningof{E_1 | E_2} = \{ P \in \pi | P \equiv P_{1} | P_{2}, P_{1} \in \meaningof{E_{1}}, P_{2} \in \meaningof{E_2}\} }
\end{mathpar}

\begin{mathpar}
 \inferrule* [lab=behavior] {} {\meaningof{\langle a?b \rangle E} = \{ P \in \pi | P \equiv Q | u?(y)P', \\ \and \\\\ \and \\ \;\;\; u \in \meaningof{a}, \forall z.P'\{z/y\} \in \meaningof{E\{z/b\}}\}, \and \\ \meaningof{a!E} = \{ P \in \pi | P \equiv Q | x!\langle P' \rangle, x \in \meaningof{a} P' \in \meaningof{E}\} }
\end{mathpar}

\begin{mathpar}
 \inferrule* [lab=nominal] {} {\meaningof{\quotep{E}} = \{ \quotep{P} \in \quotep{\pi} | P \in \meaningof{E} \}, \and \meaningof{\quotep{P}} = \{ \quotep{Q} \in \quotep{\pi} | P \equiv Q \} \and \\ \meaningof{@\quotep{E}} = \{ P \in \pi | P \equiv @x, x \in \meaningof{E} \}}
\end{mathpar}

\begin{eqnarray*}
  \\
  \meaningof{-} : TS \to ST
\end{eqnarray*}

\begin{eqnarray*}
  \\
  L : TS \to ST
\end{eqnarray*}

\begin{eqnarray*}
  \\
  P \models E \iff P \in \meaningof{E}
\end{eqnarray*}

\begin{eqnarray*}
  P \approx_{L} Q \iff \forall E \in L. P \models E \iff Q \models E
\end{eqnarray*}

\begin{eqnarray*}
  P \approx_{K} Q
\end{eqnarray*}

\begin{eqnarray*}
  P \approx Q
\end{eqnarray*}

$\approx_{K} = \approx = \approx_{L}$

\subsubsection{Contextual duality}

Note that contexts extend the quotation operation to a family of
operations from processes to names. Given a context, $M$, we can
define a \emph{nominal context}, $\quotep{M}$ by $\quotep{M}[P] :=
\quotep{M[P]}$. To foreshadow what is to come we observe that these
operations enjoy a duality with processes very much like the duality
between vectors and maps from vectors to scalars.

Further, because the calculus is essentially higher-order, we have a
correspondence between contexts and processes. More specifically,
given a name $x$ and a context $M$ we can construct $M^{*}_{x}$ such
that 

\begin{mathpar}
  M^{*}_{x} | \lift{x}{P} \red M[P]
\end{mathpar}

namely,

\begin{mathpar}
  M^{*}_{x} := x?(u).M[\dropn{u}]
\end{mathpar}

The dependence of $M^{*}_{x}$ on a name makes it an abstraction, 

\begin{mathpar}
  M^{*} := (x)x?(u).M[\dropn{u}]
\end{mathpar}

\subsection{Additional notation}

It will sometimes be convenient to denote the process a name
quotes. We already have the notation $x = \quotep{P}$, but it will be
convenient to introduce an alternate notation, $\procn{x}$, when we
want to emphasize the connection to the use of the name. Note that, by
virtue of name equivalence, $\quotep{\procn{x}} \nameeq x$; so, the
notation is consistent with previous definitions.

Further, because names have structure it is possible to effect
substitutions on the basis of that structure. This means we need to
upgrade our notation for substitutions, which we accomplish by
adapting comprehension notation. Thus,

\begin{mathpar}
  P\{ y / x : x \in S \}
\end{mathpar}

is interpreted to mean the process derived from P by replacing (in a
capture-avoiding manner) each occurrence of $x$ in $S$ by $y$. For example,

\begin{mathpar}
  P\{ \quotep{\procn{x}|\procn{x}} / x : x \in \freenames{P} \}
\end{mathpar}

will replace each (occurrence) of a free name $x$ in $P$ by
$\quotep{\procn{x}|\procn{x}}$.

Also, we will avail ourselves of the notation $x^{L}$ and $x^{R}$ to
denote injections of a name into disjoint copies of the name
space. There are numerous ways to accomplish this. One example can be
found in \cite{MeredithR05}. This notation overloads to vectors of
names: $\vec{x}^{\pi} := (x_{i}^{\pi} \; : \; 0 \leq i < |\vec{x}| )$ where $\pi \in \{L,R\}$.

We also use $P^{\Box} := P|\Box$.

In \cite{MeredithR05} an interpretation of the new operator is
given. It turns out that there are several possible interpretations
all enjoying the requisite algebraic properties of the operator (see
\cite{milner91polyadicpi}). We will therefore make liberal use of
$(\nu\; \vec{x})P$.

% subsection the_syntax_and_semantics_of_the_notation_system (end)   

\input{qm2pi.qmops} 

\input{qm2pi.sterngerlach} 

\input{qm2pi.metric} 

% section concurrent_process_calculi (end)

%\input{qm2pi.proofsketch}

% section proof sketch (end)

%\input{qm2pi.slviaknots} 

% section spatial logic via knots (end)

\input{qm2pi.conclusion}

% section conclusion (end)

%\input{qm2pi.dtcodes} 

% section wiring algorithm (end)

\input{qm2pi.ack} 

% section acknowledgments (end)

\newpage


\bibliographystyle{plain}   
\bibliography{../../biblios/main.bib}

\input{qm2pi.rhodetails}

\end{document}

 

\documentclass[12pt]{llncs}
%\documentclass{jktr}

\usepackage[pdftex]{hyperref}                   
\usepackage {listings}
\usepackage {mathpartir}
\usepackage{bcprules}
%\usepackage{listings}
                       
\usepackage{graphicx} 
%\usepackage[margins=2.5cm,nohead,nofoot]{geometry}
%\usepackage{geometry}
\usepackage{amsfonts}
\usepackage{amstext}
\usepackage{latexsym}
\usepackage{amssymb}
\usepackage{color}


%\include{myPreamble}
\include{qm2pi.local} 

%\ifpdf
%\usepackage[pdftex]{graphicx}
%\else
%\usepackage{graphicx}
%\fi

 % \ifpdf
%  \usepackage{pdfsync}
%  \if


%\title{Brief Article}
%\author{David F. Snyder}
%\author{L.G. Meredith}

%\address{Dept. of Math., Texas State University--San Marcos, San Marcos, TX 78666}
       
\pagestyle{empty}


\begin{document}

\lstset{language=[Objective]Caml,frame=shadowbox}

\input{qm2pi.front}

% section front matter (end)

\input{qm2pi.intro} 
 
% section introduction (end)

% \input{qm2pi.knotations} 

% section notation (end)

\input{qm2pi.process.calculi} 

% section concurrent_process_calculi_and_spatial_logics_ (end)
    
%\input{qm2pi.knots2pi} 

%\input{qm2pi.trefoil} 

%\input{qm2pi.mainthm} 

% subsection basic_interpretation (end)

%\input{qm2pi.rho.presentation} 
\subsection{The syntax and semantics of the notation system}\label{sub:the_syntax_and_semantics_of_the_notation_system} % (fold)

We now summarize a technical presentation of the calculus that
embodies our theory of dynamics. The typical presentation of such a
calculus follows the style of giving generators and relations on
them. The grammar, below, describing term constructors, freely
generates the set of processes, $\Proc$. This set is then quotiented
by a relation known as structural congruence and it is over this set
that the notion of dynamics is expressed. This presentation is
essentially that of \cite{MeredithR05} with the addition of
polyadicity and summation. For readability we have relegated some of
the technical subtleties to an appendix.

\subsubsection{Process grammar}\label{subsub:process_grammar}

\begin{mathpar}
  \inferrule* [lab=synchronization] {} {{M} \bc \pzero \;|\; x?F \;|\; x!C }
  \and
  \inferrule* [lab=abstraction] {} {{F} \bc (x)P}
  \and
  \inferrule* [lab=concretion] {} {{C} \bc \langle Q \rangle}
  \and
  \inferrule* [lab=process] {} {{P,Q} \bc M \;| \;P|Q \;|\; @{x}}
  \and
  \inferrule* [lab=name] {} {{x} \bc \quotep{P}}
\end{mathpar} 

Note that $\vec{x}$ (resp. $\vec{P}$) denotes a vector of names
(resp. processes) of length $|\vec{x}|$ (resp. $|\vec{P}|$). We adopt
the following useful abbreviations.

\begin{mathpar}
   x?(\vec{y}).P := x.(\vec{y})P \and  x\clift{\vec{P}} := x.\clift{\vec{P}}
   \and x!(y) := \lift{x}{\dropn{y}}
   \and \Pi_{i=0}^{n-1}P_i := P_0 | \ldots | P_{n-1}
\end{mathpar}

\subsubsection{Structural congruence}

\paragraph{Free and bound names and alpha-equivalence.} At the
core of structural equivalence is alpha-equivalence which identifies
process that are the same up to a change of variable. Formally, we
recognize the distinction between free and bound names. The free names
of a process, $\freenames{P}$, may be calculated recursively as
follows:

\begin{mathpar}
\freenames{\pzero} := \emptyset
  \and \\
  \freenames{x?(y).P} := \{ x \} \cup (\freenames{P} \setminus \{ y \})
  \and 
  \freenames{x!\langle P \rangle} := \{ x \} \cup \{ P \} 
  \and \\
  \freenames{P|Q} := \freenames{P} \cup \freenames{Q}
  \and \\
  \freenames{@{x}} := \{ x \}
\end{mathpar}

$\pi$
$\quotep{\pi}$

$\freenames{-} : \pi \to \mathcal{P}(\quotep{\pi})$

\begin{eqnarray*}
  \freenames{\pzero} & := & \emptyset \\
  \freenames{x?(y).P} & := & \{ x \} \cup (\freenames{P} \setminus \{ y \}) \\
  \freenames{x!\langle P \rangle} & := & \{ x \} \cup \{ P \} \\
  \freenames{P|Q} & := & \freenames{P} \cup \freenames{Q} \\
  \freenames{\dropn{x}} & := & \{ x \}
\end{eqnarray*}

The bound names of a process, $\boundnames{P}$, are those names occurring in $P$
that are not free. For example, in $x?(y).0$, the name $x$ is free, while $y$ is bound.

\begin{mathpar}
  \inferrule* [lab=monoidal-laws] {} { P|Q \equiv Q|P \and P|0 \equiv P \and P|(Q|R) \equiv (P|Q)|R }
\end{mathpar}

\begin{mathpar}
  \inferrule* [lab=alpha-equivalence] {} { (x)P \equiv (y)P\{y/x\} \and y \not\in \freenames{P} }
\end{mathpar}

\begin{definition}
Then two processes, $P,Q$, are alpha-equivalent if $P = Q\{\vec{y}/\vec{x}\}$ for
some $\vec{x} \in \boundnames{Q},\vec{y} \in \boundnames{P}$, where $Q\{\vec{y}/\vec{x}\}$
denotes the capture-avoiding substitution of $\vec{y}$ for $\vec{x}$ in $Q$.
\end{definition}

\begin{definition}
  The {\em structural congruence} \cite{SangiorgiWalker} , $\equiv$,
  between processes is the least congruence containing
  alpha-equivalence, satisfying the abelian monoid laws
  (associativity, commutativity and $\pzero$ as identity) for parallel
  composition $|$ and for summation $+$.
\end{definition}

\subsection{Name equivalence}

We take name equivalence, written $\nameeq$, to be the smallest
equivalence relation generated by the following rules.

\begin{mathpar}
\inferrule*[lab=Quote-drop]
{ }
{ \quotep{@{x}} \nameeq x }

\inferrule*[lab=Struct-equiv]
{ P \scong Q }
{ \quotep{P} \nameeq \quotep{Q} }
\end{mathpar}

The astute reader will have noticed that the mutual recursion of names
and processes imposes a mutual recursion on alpha-equivalence and
structural equivalence via name-equivalence. Fortunately, all of this
works out pleasantly and we may calculate in the natural way, free of
concern. The reader interested in the details is referred to the
appendix \ref{appendix:rho_details}.

\subsection{Substitution}

We use $\Proc$ for the set of processes, $\QProc$ for the set of
names, and $\id{\{}\vec{y} / \vec{x} \id{\}}$ to denote partial maps,
$s : \QProc \rightarrow \QProc$. A map, $s$ lifts, uniquely, to a map
on process terms, $\widehat{s} : \Proc \rightarrow \Proc$ by the
following equations.

\begin{mathpar}
  (0) \psubstp{Q}{P} := 0 \\
  (R \juxtap S) \psubstp{Q}{P}
  :=    
  (R)\psubstp{Q}{P} \juxtap (S) \psubstp{Q}{P} \\
  (x?(y).R) \psubstp{Q}{P}    
  :=    
  (x)\substp{Q}{P} (z)\concat( (R \psubstn{z}{y}) \psubstp{Q}{P} ) \\
  (\lift{x}{R}) \psubstp{Q}{P}  
  :=
  \lift{(x)\substp{Q}{P}}{ R \psubstp{Q}{P} } \\
%   (\dropn{x})  \psubstp{Q}{P}       
%   := 
%   \left\{ 
%     \begin{array}{ccc} 
%       \dropn{\quotep{Q}} & & x \nameeq \quotep{P} \\
%       \dropn{x} & & otherwise \\
%     \end{array}
%   \right. 
  (\dropn{x})  \psubstp{Q}{P}       
  := 
  \left\{ 
    \begin{array}{ccc} 
      Q & & x \nameeq \quotep{P} \\
      \dropn{x} & & otherwise \\
    \end{array}
  \right.
\end{mathpar}
 

where

\begin{eqnarray}
  (x)\id{\{} \lpquote Q \rpquote / \lpquote P \rpquote \id{\}}            = 
  \left\{ 
    \begin{array}{ccc}
      \lpquote Q \rpquote & & x \nameeq \lpquote P \rpquote \\
      x & & otherwise \\
    \end{array}
  \right. \nonumber
\end{eqnarray}

and $z$ is chosen distinct from $\quotep{P}$, $\quotep{Q}$, the free
names in $Q$, and all the names in $R$. Our $\alpha$-equivalence will
be built in the standard way from this substitution.

\begin{remark}\label{rem:no_self_referential_names}
  One consequence of these definitions is that $\forall P. \quotep{P}
  \not\in \freenames{P}$.
\end{remark}

\subsection{ Dynamic quote: an example }

Anticipating something of what's to come, consider applying the
substitution, $\widehat{\id{\{}u / z \id{\}}}$, to the following pair
of processes, $\lift{w}{y!(z)}$ and $w[ \lpquote y!(z) \rpquote ]$.

\begin{eqnarray}
	\lift{w}{y!(z)}\widehat{\id{\{}u / z \id{\}}}
		& = &
		\lift{w}{y!(u)} \nonumber\\
	w[ \lpquote y!(z) \rpquote ] \widehat{ \id{\{}u / z \id{\}} }
		& = &
		w[ \lpquote y!(z) \rpquote ] \nonumber
\end{eqnarray}

Because the body of the process between quotes is impervious to
substitution, we get radically different answers. In fact, by
examining the first process in an input context,
e.g. $x?(z).\lift{w}{y!(z)}$, we see that the process under the lift
operator may be shaped by prefixed inputs binding a name inside it. In
this sense, the lift operator will be seen as a way to dynamically
construct processes before reifying them as names.

Finally equipped with these standard features we can present the
dynamics of the calculus.

\subsubsection{Operational semantics} 

Finally, we introduce the computational dynamics. What marks these
algebras as distinct from other more traditionally studied algebraic
structures, e.g. vector spaces or polynomial rings, is the manner in
which dynamics is captured. In traditional structures, dynamics is typically
expressed through morphisms between such structures, as in linear maps
between vector spaces or morphisms between rings. In algebras
associated with the semantics of computation, the dynamics is
expressed as part of the algebraic structure itself, through a
reduction reduction relation typically denoted by $\red$. Below, we
give a recursive presentation of this relation for the calculus used
in the encoding.

$\red \subseteq \pi \times \pi$
$\red : \pi \to \mathcal{P}(\pi)$

\begin{mathpar}
  \inferrule* [lab=Comm] { \textsf{match}( x_{src}, x_{trgt} ) } { x_{trgt}?(y)P \; | \; x_{src}!\langle {Q} \rangle \red P\{\quotep{Q}/y}\} }
  \and \\
  \inferrule* [lab=Par] {{P} \red {P}'} {{{P} | {Q}} \red {{P}' | {Q}}}
  \and
  \inferrule* [lab=Equiv]{{{P} \scong {P}'} \andalso {{P}' \red {Q}'} \andalso {{Q}' \scong {Q}}}{{P} \red {Q}}
\end{mathpar}

\begin{eqnarray*}
  match_{\equiv} (\quotep{P},\quotep{Q}) & := & P \equiv Q \\
  match_{\dagger}(\quotep{P},\quotep{Q}) & := & \forall R. P|Q \red^{*} R => R \red^{*} 0 \\
  match_{K}(\quotep{P},\quotep{Q}) & := & K \mbox{ for some context } K
\end{eqnarray*}

$u?(x)P | u!\langle Q \rangle \red P\{\quotep{Q}/x\}$

%We write $\wred$ for $\red^*$, and $P\red$ if $\exists Q $ such that $ P \red Q$.
We write $P\red$ if $\exists Q $ such that $ P \red Q$ and $P\not\red$, otherwise.

\section{Replication}

As mentioned before, it is known that replication (and hence
recursion) can be implemented in a higher-order process algebra
\cite{SangiorgiWalker}. As our first example of calculation with the
machinery thus far presented we give the construction explicitly in
the {\rhoc}.

\begin{eqnarray}
	D_{x} & := & \prefix{x}{y}{(\binpar{\outputp{x}{y}}{@{y}})} \nonumber\\
	\bangp_{x}{P} & := & \binpar{{x}!\langle{\binpar{D_{x}}{P}}\rangle}{D_{x}} \nonumber
\end{eqnarray}

\begin{eqnarray}
	\bangp_{x}{P} & & \nonumber\\
	=
	& {x}!\langle{(\prefix{x}{y}{(\outputp{x}{y} | @{y})) | P}}\rangle 
	      | \prefix{x}{y}{(\outputp{x}{y} | @{y})} & \nonumber\\
	\red
	& (\outputp{x}{y} | @{y})\substn{\quotep{(\prefix{x}{y}{(@{y} | \outputp{x}{y})) | P}}}{y} & \nonumber\\
	=
	& \outputp{x}{\quotep{(\prefix{x}{y}{(\outputp{x}{y} | @{y})) | P}}}
	  | {(\prefix{x}{y}{(\outputp{x}{y} | @{y})) | P}} & \nonumber\\
	\red
	& \ldots & \nonumber\\
	\red^*
	& P | P | \ldots & \nonumber
\end{eqnarray}

Of course, this encoding, as an implementation, runs away, unfolding
$\bangp{P}$ eagerly. A lazier and more implementable replication
operator, restricted to input-guarded processes, may be obtained as follows.

\begin{eqnarray}
\bangp{\prefix{u}{v}{P}} 
	:= 
	\binpar{\lift{x}{\prefix{u}{v}{(\binpar{D(x)}{P})}}}{D(x)} \nonumber
\end{eqnarray}

\begin{remark}
  Note that the lazier definition still does not deal with summation
  or mixed summation (i.e. sums over input and output). The reader is
  invited to construct definitions of replication that deal with these
  features. 

  Further, the definitions are parameterized in a name, $x$. Can you,
  gentle reader, make a definition that eliminates this parameter and
  guarantees no accidental interaction between the replication
  machinery and the process being replicated -- i.e. no accidental
  sharing of names used by the process to get its work done and the
  name(s) used by the replication to effect copying. This latter
  revision of the definition of replication is crucial to obtaining
  the expected identity $!!P \sim !P$.
\end{remark}

\begin{remark}\label{rem:paradoxical_combinator}
  The reader familiar with the lambda calculus will have noticed the
  similarity between $D$ and the paradoxical combinator.

  [Ed. note: the existence of this seems to suggest we have to be more
  restrictive on the set of processes and names we admit if we are to
  support no-cloning.]
\end{remark}

\subsubsection{Bisimulation}

The computational dynamics gives rise to another kind of equivalence,
the equivalence of computational behavior. As previously mentioned
this is typically captured \emph{via} some form of bisimulation.

% The notion we use in this paper is weak barbed bisimulation
% \cite{milner91polyadicpi}.

The notion we use in this paper is derived from weak barbed
bisimulation \cite{milner91polyadicpi}. 

\begin{definition}
An \emph{observation relation}, $\downarrow_{\mathcal N}$, over a set
of names, $\mathcal N$, is the smallest relation satisfying the rules
below.

\infrule[Out-barb]{y \in {\mathcal N}, \; x \nameeq y}
		  {\outputp{x}{v} \downarrow_{\mathcal N} x}
\infrule[Par-barb]{\mbox{$P\downarrow_{\mathcal N} x$ or $Q\downarrow_{\mathcal N} x$}}
		  {\binpar{P}{Q} \downarrow_{\mathcal N} x}

We write $P \Downarrow_{\mathcal N} x$ if there is $Q$ such that 
$P \wred Q$ and $Q \downarrow_{\mathcal N} x$.
\end{definition}

\begin{definition}
%\label{def.bbisim}
An  ${\mathcal N}$-\emph{barbed bisimulation} over a set of names, ${\mathcal N}$, is a symmetric binary relation 
${\mathcal S}_{\mathcal N}$ between agents such that $P\rel{S}_{\mathcal N}Q$ implies:
\begin{enumerate}
\item If $P \red P'$ then $Q \wred Q'$ and $P'\rel{S}_{\mathcal N} Q'$.
\item If $P\downarrow_{\mathcal N} x$, then $Q\Downarrow_{\mathcal N} x$.
\end{enumerate}
$P$ is ${\mathcal N}$-barbed bisimilar to $Q$, written
$P \wbbisim_{\mathcal N} Q$, if $P \rel{S}_{\mathcal N} Q$ for some ${\mathcal N}$-barbed bisimulation ${\mathcal S}_{\mathcal N}$.
\end{definition}

$\mathcal{R} \subseteq \pi \times \pi$

$P \mathcal{R} Q => \forall P'. P \red P' \Rightarrow \exists Q'. Q \red Q', P' \mathcal{R} Q'$

$P \vdash x \Rightarrow Q \vdash x$

\begin{mathpar}
  \inferrule*[lab=Out-barb]{x \nameeq y}{{y}!\langle{Q}\rangle \vdash x}
  \and
  \inferrule*[lab=Par-barb]{\mbox{$P\vdash x$ or $Q\vdash x$}}{\binpar{P}{Q} \vdash x}
\end{mathpar}

\subsubsection{Contexts}

One of the principle advantages of computational calculi like the
$\pi$-calculus is a well-defined notion of context,
contextual-equivalence and a correlation between
contextual-equivalence and notions of bisimulation. The notion of
context allows the decomposition of a process into (sub-)process and
its syntactic environment, its context. Thus, a context may be
thought of as a process with a ``hole'' (written $\Box$) in it. The
application of a context $M$ to a process $P$, written $M[P]$, is
tantamount to filling the hole in $M$ with $P$. In this paper we do
not need the full weight of this theory, but do make use of the notion
of context in the proof the main theorem. 

\begin{mathpar}
  \inferrule* [lab=summation] {} {{M_{M},M_{N}} \bc \Box \;|\; x.M_{A} \;|\; M_{M}+M_{N}}
  \and
  \inferrule* [lab=agent] {} {{M_{A}} \bc (\vec{x})M_{P} \;| \; \clift{P_0,\ldots,M_{P},\ldots,P_N}}
  \and \\
  \inferrule* [lab=process] {} {{M_{P}} \bc M_{N} \;| \;P|M_{P} }
\end{mathpar} 

\begin{mathpar}
  \inferrule* [lab=sychronization] {} {M_{N} \bc \Box \;|\; x?M_{F} \;|\; x!M_{C}}
  \and
  \inferrule* [lab=abstraction] {} {{M_{F}} \bc (x)M_{P} }
  \and
  \inferrule* [lab=concretion] {} {{M_{C}} \bc \langle M_{P} \rangle }
  \and \\
  \inferrule* [lab=process] {} {{M_{P}} \bc M_{N} \;| \;P|M_{P} }
\end{mathpar}

\begin{definition}[contextual application] Given a context $M$, and
  process $P$, we define the \emph{contextual application}, $M[P] :=
  M\{P/\Box\}$. That is, the contextual application of M to P is the
  substitution of $P$ for $\Box$ in $M$.
\end{definition}

$\meaningof{-} : L \to \mathcal{P}(\pi)$

\begin{mathpar}
  \inferrule* [lab=collection] {} {\meaningof{true} = \pi, \and \meaningof{~E} = \pi \setminus \meaningof{E}, \and \meaningof{E_{1} \& E_{2}} = \meaningof{E_{1}} \cap \meaningof{E_{2}}}
\end{mathpar}

\begin{mathpar}
  \inferrule* [lab=structure] {} {\meaningof{0} = \{ P \in \pi | P \equiv 0 \}, \and \\ \meaningof{E_1 | E_2} = \{ P \in \pi | P \equiv P_{1} | P_{2}, P_{1} \in \meaningof{E_{1}}, P_{2} \in \meaningof{E_2}\} }
\end{mathpar}

\begin{mathpar}
 \inferrule* [lab=behavior] {} {\meaningof{\langle a?b \rangle E} = \{ P \in \pi | P \equiv Q | u?(y)P', \\ \and \\\\ \and \\ \;\;\; u \in \meaningof{a}, \forall z.P'\{z/y\} \in \meaningof{E\{z/b\}}\}, \and \\ \meaningof{a!E} = \{ P \in \pi | P \equiv Q | x!\langle P' \rangle, x \in \meaningof{a} P' \in \meaningof{E}\} }
\end{mathpar}

\begin{mathpar}
 \inferrule* [lab=nominal] {} {\meaningof{\quotep{E}} = \{ \quotep{P} \in \quotep{\pi} | P \in \meaningof{E} \}, \and \meaningof{\quotep{P}} = \{ \quotep{Q} \in \quotep{\pi} | P \equiv Q \} \and \\ \meaningof{@\quotep{E}} = \{ P \in \pi | P \equiv @x, x \in \meaningof{E} \}}
\end{mathpar}

\begin{eqnarray*}
  \\
  \meaningof{-} : TS \to ST
\end{eqnarray*}

\begin{eqnarray*}
  \\
  L : TS \to ST
\end{eqnarray*}

\begin{eqnarray*}
  \\
  P \models E \iff P \in \meaningof{E}
\end{eqnarray*}

\begin{eqnarray*}
  P \approx_{L} Q \iff \forall E \in L. P \models E \iff Q \models E
\end{eqnarray*}

\begin{eqnarray*}
  P \approx_{K} Q
\end{eqnarray*}

\begin{eqnarray*}
  P \approx Q
\end{eqnarray*}

$\approx_{K} = \approx = \approx_{L}$

\subsubsection{Contextual duality}

Note that contexts extend the quotation operation to a family of
operations from processes to names. Given a context, $M$, we can
define a \emph{nominal context}, $\quotep{M}$ by $\quotep{M}[P] :=
\quotep{M[P]}$. To foreshadow what is to come we observe that these
operations enjoy a duality with processes very much like the duality
between vectors and maps from vectors to scalars.

Further, because the calculus is essentially higher-order, we have a
correspondence between contexts and processes. More specifically,
given a name $x$ and a context $M$ we can construct $M^{*}_{x}$ such
that 

\begin{mathpar}
  M^{*}_{x} | \lift{x}{P} \red M[P]
\end{mathpar}

namely,

\begin{mathpar}
  M^{*}_{x} := x?(u).M[\dropn{u}]
\end{mathpar}

The dependence of $M^{*}_{x}$ on a name makes it an abstraction, 

\begin{mathpar}
  M^{*} := (x)x?(u).M[\dropn{u}]
\end{mathpar}

\subsection{Additional notation}

It will sometimes be convenient to denote the process a name
quotes. We already have the notation $x = \quotep{P}$, but it will be
convenient to introduce an alternate notation, $\procn{x}$, when we
want to emphasize the connection to the use of the name. Note that, by
virtue of name equivalence, $\quotep{\procn{x}} \nameeq x$; so, the
notation is consistent with previous definitions.

Further, because names have structure it is possible to effect
substitutions on the basis of that structure. This means we need to
upgrade our notation for substitutions, which we accomplish by
adapting comprehension notation. Thus,

\begin{mathpar}
  P\{ y / x : x \in S \}
\end{mathpar}

is interpreted to mean the process derived from P by replacing (in a
capture-avoiding manner) each occurrence of $x$ in $S$ by $y$. For example,

\begin{mathpar}
  P\{ \quotep{\procn{x}|\procn{x}} / x : x \in \freenames{P} \}
\end{mathpar}

will replace each (occurrence) of a free name $x$ in $P$ by
$\quotep{\procn{x}|\procn{x}}$.

Also, we will avail ourselves of the notation $x^{L}$ and $x^{R}$ to
denote injections of a name into disjoint copies of the name
space. There are numerous ways to accomplish this. One example can be
found in \cite{MeredithR05}. This notation overloads to vectors of
names: $\vec{x}^{\pi} := (x_{i}^{\pi} \; : \; 0 \leq i < |\vec{x}| )$ where $\pi \in \{L,R\}$.

We also use $P^{\Box} := P|\Box$.

In \cite{MeredithR05} an interpretation of the new operator is
given. It turns out that there are several possible interpretations
all enjoying the requisite algebraic properties of the operator (see
\cite{milner91polyadicpi}). We will therefore make liberal use of
$(\nu\; \vec{x})P$.

% subsection the_syntax_and_semantics_of_the_notation_system (end)   

\input{qm2pi.qmops} 

\input{qm2pi.sterngerlach} 

\input{qm2pi.metric} 

% section concurrent_process_calculi (end)

%\input{qm2pi.proofsketch}

% section proof sketch (end)

%\input{qm2pi.slviaknots} 

% section spatial logic via knots (end)

\input{qm2pi.conclusion}

% section conclusion (end)

%\input{qm2pi.dtcodes} 

% section wiring algorithm (end)

\input{qm2pi.ack} 

% section acknowledgments (end)

\newpage


\bibliographystyle{plain}   
\bibliography{../../biblios/main.bib}

\input{qm2pi.rhodetails}

\end{document}

 

% section concurrent_process_calculi (end)

%\documentclass[12pt]{llncs}
%\documentclass{jktr}

\usepackage[pdftex]{hyperref}                   
\usepackage {listings}
\usepackage {mathpartir}
\usepackage{bcprules}
%\usepackage{listings}
                       
\usepackage{graphicx} 
%\usepackage[margins=2.5cm,nohead,nofoot]{geometry}
%\usepackage{geometry}
\usepackage{amsfonts}
\usepackage{amstext}
\usepackage{latexsym}
\usepackage{amssymb}
\usepackage{color}


%\include{myPreamble}
\include{qm2pi.local} 

%\ifpdf
%\usepackage[pdftex]{graphicx}
%\else
%\usepackage{graphicx}
%\fi

 % \ifpdf
%  \usepackage{pdfsync}
%  \if


%\title{Brief Article}
%\author{David F. Snyder}
%\author{L.G. Meredith}

%\address{Dept. of Math., Texas State University--San Marcos, San Marcos, TX 78666}
       
\pagestyle{empty}


\begin{document}

\lstset{language=[Objective]Caml,frame=shadowbox}

\input{qm2pi.front}

% section front matter (end)

\input{qm2pi.intro} 
 
% section introduction (end)

% \input{qm2pi.knotations} 

% section notation (end)

\input{qm2pi.process.calculi} 

% section concurrent_process_calculi_and_spatial_logics_ (end)
    
%\input{qm2pi.knots2pi} 

%\input{qm2pi.trefoil} 

%\input{qm2pi.mainthm} 

% subsection basic_interpretation (end)

%\input{qm2pi.rho.presentation} 
\subsection{The syntax and semantics of the notation system}\label{sub:the_syntax_and_semantics_of_the_notation_system} % (fold)

We now summarize a technical presentation of the calculus that
embodies our theory of dynamics. The typical presentation of such a
calculus follows the style of giving generators and relations on
them. The grammar, below, describing term constructors, freely
generates the set of processes, $\Proc$. This set is then quotiented
by a relation known as structural congruence and it is over this set
that the notion of dynamics is expressed. This presentation is
essentially that of \cite{MeredithR05} with the addition of
polyadicity and summation. For readability we have relegated some of
the technical subtleties to an appendix.

\subsubsection{Process grammar}\label{subsub:process_grammar}

\begin{mathpar}
  \inferrule* [lab=synchronization] {} {{M} \bc \pzero \;|\; x?F \;|\; x!C }
  \and
  \inferrule* [lab=abstraction] {} {{F} \bc (x)P}
  \and
  \inferrule* [lab=concretion] {} {{C} \bc \langle Q \rangle}
  \and
  \inferrule* [lab=process] {} {{P,Q} \bc M \;| \;P|Q \;|\; @{x}}
  \and
  \inferrule* [lab=name] {} {{x} \bc \quotep{P}}
\end{mathpar} 

Note that $\vec{x}$ (resp. $\vec{P}$) denotes a vector of names
(resp. processes) of length $|\vec{x}|$ (resp. $|\vec{P}|$). We adopt
the following useful abbreviations.

\begin{mathpar}
   x?(\vec{y}).P := x.(\vec{y})P \and  x\clift{\vec{P}} := x.\clift{\vec{P}}
   \and x!(y) := \lift{x}{\dropn{y}}
   \and \Pi_{i=0}^{n-1}P_i := P_0 | \ldots | P_{n-1}
\end{mathpar}

\subsubsection{Structural congruence}

\paragraph{Free and bound names and alpha-equivalence.} At the
core of structural equivalence is alpha-equivalence which identifies
process that are the same up to a change of variable. Formally, we
recognize the distinction between free and bound names. The free names
of a process, $\freenames{P}$, may be calculated recursively as
follows:

\begin{mathpar}
\freenames{\pzero} := \emptyset
  \and \\
  \freenames{x?(y).P} := \{ x \} \cup (\freenames{P} \setminus \{ y \})
  \and 
  \freenames{x!\langle P \rangle} := \{ x \} \cup \{ P \} 
  \and \\
  \freenames{P|Q} := \freenames{P} \cup \freenames{Q}
  \and \\
  \freenames{@{x}} := \{ x \}
\end{mathpar}

$\pi$
$\quotep{\pi}$

$\freenames{-} : \pi \to \mathcal{P}(\quotep{\pi})$

\begin{eqnarray*}
  \freenames{\pzero} & := & \emptyset \\
  \freenames{x?(y).P} & := & \{ x \} \cup (\freenames{P} \setminus \{ y \}) \\
  \freenames{x!\langle P \rangle} & := & \{ x \} \cup \{ P \} \\
  \freenames{P|Q} & := & \freenames{P} \cup \freenames{Q} \\
  \freenames{\dropn{x}} & := & \{ x \}
\end{eqnarray*}

The bound names of a process, $\boundnames{P}$, are those names occurring in $P$
that are not free. For example, in $x?(y).0$, the name $x$ is free, while $y$ is bound.

\begin{mathpar}
  \inferrule* [lab=monoidal-laws] {} { P|Q \equiv Q|P \and P|0 \equiv P \and P|(Q|R) \equiv (P|Q)|R }
\end{mathpar}

\begin{mathpar}
  \inferrule* [lab=alpha-equivalence] {} { (x)P \equiv (y)P\{y/x\} \and y \not\in \freenames{P} }
\end{mathpar}

\begin{definition}
Then two processes, $P,Q$, are alpha-equivalent if $P = Q\{\vec{y}/\vec{x}\}$ for
some $\vec{x} \in \boundnames{Q},\vec{y} \in \boundnames{P}$, where $Q\{\vec{y}/\vec{x}\}$
denotes the capture-avoiding substitution of $\vec{y}$ for $\vec{x}$ in $Q$.
\end{definition}

\begin{definition}
  The {\em structural congruence} \cite{SangiorgiWalker} , $\equiv$,
  between processes is the least congruence containing
  alpha-equivalence, satisfying the abelian monoid laws
  (associativity, commutativity and $\pzero$ as identity) for parallel
  composition $|$ and for summation $+$.
\end{definition}

\subsection{Name equivalence}

We take name equivalence, written $\nameeq$, to be the smallest
equivalence relation generated by the following rules.

\begin{mathpar}
\inferrule*[lab=Quote-drop]
{ }
{ \quotep{@{x}} \nameeq x }

\inferrule*[lab=Struct-equiv]
{ P \scong Q }
{ \quotep{P} \nameeq \quotep{Q} }
\end{mathpar}

The astute reader will have noticed that the mutual recursion of names
and processes imposes a mutual recursion on alpha-equivalence and
structural equivalence via name-equivalence. Fortunately, all of this
works out pleasantly and we may calculate in the natural way, free of
concern. The reader interested in the details is referred to the
appendix \ref{appendix:rho_details}.

\subsection{Substitution}

We use $\Proc$ for the set of processes, $\QProc$ for the set of
names, and $\id{\{}\vec{y} / \vec{x} \id{\}}$ to denote partial maps,
$s : \QProc \rightarrow \QProc$. A map, $s$ lifts, uniquely, to a map
on process terms, $\widehat{s} : \Proc \rightarrow \Proc$ by the
following equations.

\begin{mathpar}
  (0) \psubstp{Q}{P} := 0 \\
  (R \juxtap S) \psubstp{Q}{P}
  :=    
  (R)\psubstp{Q}{P} \juxtap (S) \psubstp{Q}{P} \\
  (x?(y).R) \psubstp{Q}{P}    
  :=    
  (x)\substp{Q}{P} (z)\concat( (R \psubstn{z}{y}) \psubstp{Q}{P} ) \\
  (\lift{x}{R}) \psubstp{Q}{P}  
  :=
  \lift{(x)\substp{Q}{P}}{ R \psubstp{Q}{P} } \\
%   (\dropn{x})  \psubstp{Q}{P}       
%   := 
%   \left\{ 
%     \begin{array}{ccc} 
%       \dropn{\quotep{Q}} & & x \nameeq \quotep{P} \\
%       \dropn{x} & & otherwise \\
%     \end{array}
%   \right. 
  (\dropn{x})  \psubstp{Q}{P}       
  := 
  \left\{ 
    \begin{array}{ccc} 
      Q & & x \nameeq \quotep{P} \\
      \dropn{x} & & otherwise \\
    \end{array}
  \right.
\end{mathpar}
 

where

\begin{eqnarray}
  (x)\id{\{} \lpquote Q \rpquote / \lpquote P \rpquote \id{\}}            = 
  \left\{ 
    \begin{array}{ccc}
      \lpquote Q \rpquote & & x \nameeq \lpquote P \rpquote \\
      x & & otherwise \\
    \end{array}
  \right. \nonumber
\end{eqnarray}

and $z$ is chosen distinct from $\quotep{P}$, $\quotep{Q}$, the free
names in $Q$, and all the names in $R$. Our $\alpha$-equivalence will
be built in the standard way from this substitution.

\begin{remark}\label{rem:no_self_referential_names}
  One consequence of these definitions is that $\forall P. \quotep{P}
  \not\in \freenames{P}$.
\end{remark}

\subsection{ Dynamic quote: an example }

Anticipating something of what's to come, consider applying the
substitution, $\widehat{\id{\{}u / z \id{\}}}$, to the following pair
of processes, $\lift{w}{y!(z)}$ and $w[ \lpquote y!(z) \rpquote ]$.

\begin{eqnarray}
	\lift{w}{y!(z)}\widehat{\id{\{}u / z \id{\}}}
		& = &
		\lift{w}{y!(u)} \nonumber\\
	w[ \lpquote y!(z) \rpquote ] \widehat{ \id{\{}u / z \id{\}} }
		& = &
		w[ \lpquote y!(z) \rpquote ] \nonumber
\end{eqnarray}

Because the body of the process between quotes is impervious to
substitution, we get radically different answers. In fact, by
examining the first process in an input context,
e.g. $x?(z).\lift{w}{y!(z)}$, we see that the process under the lift
operator may be shaped by prefixed inputs binding a name inside it. In
this sense, the lift operator will be seen as a way to dynamically
construct processes before reifying them as names.

Finally equipped with these standard features we can present the
dynamics of the calculus.

\subsubsection{Operational semantics} 

Finally, we introduce the computational dynamics. What marks these
algebras as distinct from other more traditionally studied algebraic
structures, e.g. vector spaces or polynomial rings, is the manner in
which dynamics is captured. In traditional structures, dynamics is typically
expressed through morphisms between such structures, as in linear maps
between vector spaces or morphisms between rings. In algebras
associated with the semantics of computation, the dynamics is
expressed as part of the algebraic structure itself, through a
reduction reduction relation typically denoted by $\red$. Below, we
give a recursive presentation of this relation for the calculus used
in the encoding.

$\red \subseteq \pi \times \pi$
$\red : \pi \to \mathcal{P}(\pi)$

\begin{mathpar}
  \inferrule* [lab=Comm] { \textsf{match}( x_{src}, x_{trgt} ) } { x_{trgt}?(y)P \; | \; x_{src}!\langle {Q} \rangle \red P\{\quotep{Q}/y}\} }
  \and \\
  \inferrule* [lab=Par] {{P} \red {P}'} {{{P} | {Q}} \red {{P}' | {Q}}}
  \and
  \inferrule* [lab=Equiv]{{{P} \scong {P}'} \andalso {{P}' \red {Q}'} \andalso {{Q}' \scong {Q}}}{{P} \red {Q}}
\end{mathpar}

\begin{eqnarray*}
  match_{\equiv} (\quotep{P},\quotep{Q}) & := & P \equiv Q \\
  match_{\dagger}(\quotep{P},\quotep{Q}) & := & \forall R. P|Q \red^{*} R => R \red^{*} 0 \\
  match_{K}(\quotep{P},\quotep{Q}) & := & K \mbox{ for some context } K
\end{eqnarray*}

$u?(x)P | u!\langle Q \rangle \red P\{\quotep{Q}/x\}$

%We write $\wred$ for $\red^*$, and $P\red$ if $\exists Q $ such that $ P \red Q$.
We write $P\red$ if $\exists Q $ such that $ P \red Q$ and $P\not\red$, otherwise.

\section{Replication}

As mentioned before, it is known that replication (and hence
recursion) can be implemented in a higher-order process algebra
\cite{SangiorgiWalker}. As our first example of calculation with the
machinery thus far presented we give the construction explicitly in
the {\rhoc}.

\begin{eqnarray}
	D_{x} & := & \prefix{x}{y}{(\binpar{\outputp{x}{y}}{@{y}})} \nonumber\\
	\bangp_{x}{P} & := & \binpar{{x}!\langle{\binpar{D_{x}}{P}}\rangle}{D_{x}} \nonumber
\end{eqnarray}

\begin{eqnarray}
	\bangp_{x}{P} & & \nonumber\\
	=
	& {x}!\langle{(\prefix{x}{y}{(\outputp{x}{y} | @{y})) | P}}\rangle 
	      | \prefix{x}{y}{(\outputp{x}{y} | @{y})} & \nonumber\\
	\red
	& (\outputp{x}{y} | @{y})\substn{\quotep{(\prefix{x}{y}{(@{y} | \outputp{x}{y})) | P}}}{y} & \nonumber\\
	=
	& \outputp{x}{\quotep{(\prefix{x}{y}{(\outputp{x}{y} | @{y})) | P}}}
	  | {(\prefix{x}{y}{(\outputp{x}{y} | @{y})) | P}} & \nonumber\\
	\red
	& \ldots & \nonumber\\
	\red^*
	& P | P | \ldots & \nonumber
\end{eqnarray}

Of course, this encoding, as an implementation, runs away, unfolding
$\bangp{P}$ eagerly. A lazier and more implementable replication
operator, restricted to input-guarded processes, may be obtained as follows.

\begin{eqnarray}
\bangp{\prefix{u}{v}{P}} 
	:= 
	\binpar{\lift{x}{\prefix{u}{v}{(\binpar{D(x)}{P})}}}{D(x)} \nonumber
\end{eqnarray}

\begin{remark}
  Note that the lazier definition still does not deal with summation
  or mixed summation (i.e. sums over input and output). The reader is
  invited to construct definitions of replication that deal with these
  features. 

  Further, the definitions are parameterized in a name, $x$. Can you,
  gentle reader, make a definition that eliminates this parameter and
  guarantees no accidental interaction between the replication
  machinery and the process being replicated -- i.e. no accidental
  sharing of names used by the process to get its work done and the
  name(s) used by the replication to effect copying. This latter
  revision of the definition of replication is crucial to obtaining
  the expected identity $!!P \sim !P$.
\end{remark}

\begin{remark}\label{rem:paradoxical_combinator}
  The reader familiar with the lambda calculus will have noticed the
  similarity between $D$ and the paradoxical combinator.

  [Ed. note: the existence of this seems to suggest we have to be more
  restrictive on the set of processes and names we admit if we are to
  support no-cloning.]
\end{remark}

\subsubsection{Bisimulation}

The computational dynamics gives rise to another kind of equivalence,
the equivalence of computational behavior. As previously mentioned
this is typically captured \emph{via} some form of bisimulation.

% The notion we use in this paper is weak barbed bisimulation
% \cite{milner91polyadicpi}.

The notion we use in this paper is derived from weak barbed
bisimulation \cite{milner91polyadicpi}. 

\begin{definition}
An \emph{observation relation}, $\downarrow_{\mathcal N}$, over a set
of names, $\mathcal N$, is the smallest relation satisfying the rules
below.

\infrule[Out-barb]{y \in {\mathcal N}, \; x \nameeq y}
		  {\outputp{x}{v} \downarrow_{\mathcal N} x}
\infrule[Par-barb]{\mbox{$P\downarrow_{\mathcal N} x$ or $Q\downarrow_{\mathcal N} x$}}
		  {\binpar{P}{Q} \downarrow_{\mathcal N} x}

We write $P \Downarrow_{\mathcal N} x$ if there is $Q$ such that 
$P \wred Q$ and $Q \downarrow_{\mathcal N} x$.
\end{definition}

\begin{definition}
%\label{def.bbisim}
An  ${\mathcal N}$-\emph{barbed bisimulation} over a set of names, ${\mathcal N}$, is a symmetric binary relation 
${\mathcal S}_{\mathcal N}$ between agents such that $P\rel{S}_{\mathcal N}Q$ implies:
\begin{enumerate}
\item If $P \red P'$ then $Q \wred Q'$ and $P'\rel{S}_{\mathcal N} Q'$.
\item If $P\downarrow_{\mathcal N} x$, then $Q\Downarrow_{\mathcal N} x$.
\end{enumerate}
$P$ is ${\mathcal N}$-barbed bisimilar to $Q$, written
$P \wbbisim_{\mathcal N} Q$, if $P \rel{S}_{\mathcal N} Q$ for some ${\mathcal N}$-barbed bisimulation ${\mathcal S}_{\mathcal N}$.
\end{definition}

$\mathcal{R} \subseteq \pi \times \pi$

$P \mathcal{R} Q => \forall P'. P \red P' \Rightarrow \exists Q'. Q \red Q', P' \mathcal{R} Q'$

$P \vdash x \Rightarrow Q \vdash x$

\begin{mathpar}
  \inferrule*[lab=Out-barb]{x \nameeq y}{{y}!\langle{Q}\rangle \vdash x}
  \and
  \inferrule*[lab=Par-barb]{\mbox{$P\vdash x$ or $Q\vdash x$}}{\binpar{P}{Q} \vdash x}
\end{mathpar}

\subsubsection{Contexts}

One of the principle advantages of computational calculi like the
$\pi$-calculus is a well-defined notion of context,
contextual-equivalence and a correlation between
contextual-equivalence and notions of bisimulation. The notion of
context allows the decomposition of a process into (sub-)process and
its syntactic environment, its context. Thus, a context may be
thought of as a process with a ``hole'' (written $\Box$) in it. The
application of a context $M$ to a process $P$, written $M[P]$, is
tantamount to filling the hole in $M$ with $P$. In this paper we do
not need the full weight of this theory, but do make use of the notion
of context in the proof the main theorem. 

\begin{mathpar}
  \inferrule* [lab=summation] {} {{M_{M},M_{N}} \bc \Box \;|\; x.M_{A} \;|\; M_{M}+M_{N}}
  \and
  \inferrule* [lab=agent] {} {{M_{A}} \bc (\vec{x})M_{P} \;| \; \clift{P_0,\ldots,M_{P},\ldots,P_N}}
  \and \\
  \inferrule* [lab=process] {} {{M_{P}} \bc M_{N} \;| \;P|M_{P} }
\end{mathpar} 

\begin{mathpar}
  \inferrule* [lab=sychronization] {} {M_{N} \bc \Box \;|\; x?M_{F} \;|\; x!M_{C}}
  \and
  \inferrule* [lab=abstraction] {} {{M_{F}} \bc (x)M_{P} }
  \and
  \inferrule* [lab=concretion] {} {{M_{C}} \bc \langle M_{P} \rangle }
  \and \\
  \inferrule* [lab=process] {} {{M_{P}} \bc M_{N} \;| \;P|M_{P} }
\end{mathpar}

\begin{definition}[contextual application] Given a context $M$, and
  process $P$, we define the \emph{contextual application}, $M[P] :=
  M\{P/\Box\}$. That is, the contextual application of M to P is the
  substitution of $P$ for $\Box$ in $M$.
\end{definition}

$\meaningof{-} : L \to \mathcal{P}(\pi)$

\begin{mathpar}
  \inferrule* [lab=collection] {} {\meaningof{true} = \pi, \and \meaningof{~E} = \pi \setminus \meaningof{E}, \and \meaningof{E_{1} \& E_{2}} = \meaningof{E_{1}} \cap \meaningof{E_{2}}}
\end{mathpar}

\begin{mathpar}
  \inferrule* [lab=structure] {} {\meaningof{0} = \{ P \in \pi | P \equiv 0 \}, \and \\ \meaningof{E_1 | E_2} = \{ P \in \pi | P \equiv P_{1} | P_{2}, P_{1} \in \meaningof{E_{1}}, P_{2} \in \meaningof{E_2}\} }
\end{mathpar}

\begin{mathpar}
 \inferrule* [lab=behavior] {} {\meaningof{\langle a?b \rangle E} = \{ P \in \pi | P \equiv Q | u?(y)P', \\ \and \\\\ \and \\ \;\;\; u \in \meaningof{a}, \forall z.P'\{z/y\} \in \meaningof{E\{z/b\}}\}, \and \\ \meaningof{a!E} = \{ P \in \pi | P \equiv Q | x!\langle P' \rangle, x \in \meaningof{a} P' \in \meaningof{E}\} }
\end{mathpar}

\begin{mathpar}
 \inferrule* [lab=nominal] {} {\meaningof{\quotep{E}} = \{ \quotep{P} \in \quotep{\pi} | P \in \meaningof{E} \}, \and \meaningof{\quotep{P}} = \{ \quotep{Q} \in \quotep{\pi} | P \equiv Q \} \and \\ \meaningof{@\quotep{E}} = \{ P \in \pi | P \equiv @x, x \in \meaningof{E} \}}
\end{mathpar}

\begin{eqnarray*}
  \\
  \meaningof{-} : TS \to ST
\end{eqnarray*}

\begin{eqnarray*}
  \\
  L : TS \to ST
\end{eqnarray*}

\begin{eqnarray*}
  \\
  P \models E \iff P \in \meaningof{E}
\end{eqnarray*}

\begin{eqnarray*}
  P \approx_{L} Q \iff \forall E \in L. P \models E \iff Q \models E
\end{eqnarray*}

\begin{eqnarray*}
  P \approx_{K} Q
\end{eqnarray*}

\begin{eqnarray*}
  P \approx Q
\end{eqnarray*}

$\approx_{K} = \approx = \approx_{L}$

\subsubsection{Contextual duality}

Note that contexts extend the quotation operation to a family of
operations from processes to names. Given a context, $M$, we can
define a \emph{nominal context}, $\quotep{M}$ by $\quotep{M}[P] :=
\quotep{M[P]}$. To foreshadow what is to come we observe that these
operations enjoy a duality with processes very much like the duality
between vectors and maps from vectors to scalars.

Further, because the calculus is essentially higher-order, we have a
correspondence between contexts and processes. More specifically,
given a name $x$ and a context $M$ we can construct $M^{*}_{x}$ such
that 

\begin{mathpar}
  M^{*}_{x} | \lift{x}{P} \red M[P]
\end{mathpar}

namely,

\begin{mathpar}
  M^{*}_{x} := x?(u).M[\dropn{u}]
\end{mathpar}

The dependence of $M^{*}_{x}$ on a name makes it an abstraction, 

\begin{mathpar}
  M^{*} := (x)x?(u).M[\dropn{u}]
\end{mathpar}

\subsection{Additional notation}

It will sometimes be convenient to denote the process a name
quotes. We already have the notation $x = \quotep{P}$, but it will be
convenient to introduce an alternate notation, $\procn{x}$, when we
want to emphasize the connection to the use of the name. Note that, by
virtue of name equivalence, $\quotep{\procn{x}} \nameeq x$; so, the
notation is consistent with previous definitions.

Further, because names have structure it is possible to effect
substitutions on the basis of that structure. This means we need to
upgrade our notation for substitutions, which we accomplish by
adapting comprehension notation. Thus,

\begin{mathpar}
  P\{ y / x : x \in S \}
\end{mathpar}

is interpreted to mean the process derived from P by replacing (in a
capture-avoiding manner) each occurrence of $x$ in $S$ by $y$. For example,

\begin{mathpar}
  P\{ \quotep{\procn{x}|\procn{x}} / x : x \in \freenames{P} \}
\end{mathpar}

will replace each (occurrence) of a free name $x$ in $P$ by
$\quotep{\procn{x}|\procn{x}}$.

Also, we will avail ourselves of the notation $x^{L}$ and $x^{R}$ to
denote injections of a name into disjoint copies of the name
space. There are numerous ways to accomplish this. One example can be
found in \cite{MeredithR05}. This notation overloads to vectors of
names: $\vec{x}^{\pi} := (x_{i}^{\pi} \; : \; 0 \leq i < |\vec{x}| )$ where $\pi \in \{L,R\}$.

We also use $P^{\Box} := P|\Box$.

In \cite{MeredithR05} an interpretation of the new operator is
given. It turns out that there are several possible interpretations
all enjoying the requisite algebraic properties of the operator (see
\cite{milner91polyadicpi}). We will therefore make liberal use of
$(\nu\; \vec{x})P$.

% subsection the_syntax_and_semantics_of_the_notation_system (end)   

\input{qm2pi.qmops} 

\input{qm2pi.sterngerlach} 

\input{qm2pi.metric} 

% section concurrent_process_calculi (end)

%\input{qm2pi.proofsketch}

% section proof sketch (end)

%\input{qm2pi.slviaknots} 

% section spatial logic via knots (end)

\input{qm2pi.conclusion}

% section conclusion (end)

%\input{qm2pi.dtcodes} 

% section wiring algorithm (end)

\input{qm2pi.ack} 

% section acknowledgments (end)

\newpage


\bibliographystyle{plain}   
\bibliography{../../biblios/main.bib}

\input{qm2pi.rhodetails}

\end{document}



% section proof sketch (end)

%\section{Unlikely characters: spatial logic for
  knots}\label{sub:characteristic_formulae} % (fold)

Associated to the mobile process calculi are a family of logics known
as the Hennessy-Milner logics. These logics typically enjoy a
semantics interpreting formulae as sets of processes that when
factored through the encoding outlined above allows an identification
of classes of knots with logical formulae. In the context of this
encoding the sub-family known as the spatial logics \cite{CairesC03}
\cite{CairesC04} \cite{Caires04} are of particular interest providing
several important features for expressing and reasoning about
properties (i.e. classes) of knots. We hint here at how this may be done.

%\begin{description}
%\item [structural connectives] 
\subsubsection{Structural connectives} The spatial logics enjoy
structural connectives corresponding, at the logical level, to the
parallel composition ($P | Q$) and new name ($(\nu \; x)P$)
connectives for processes. As illustrated in the examples below, these
connectives are extremely expressive given the shape of our encoding.
%\item [decideable satisfaction]

\subsubsection{Decideable satisfaction}
In \cite{Caires04} the satisfaction relation is shown to be decideable
for a rich class of processes. It further turns out that the image of
the our encoding is a proper subset of that class. This result
provides the basis for an algorithm by which to search for knots
enjoying a given property.
%\item [characteristic formulae]

\subsubsection{Characteristic formulae}
In the same paper \cite{Caires04} , Caires presents a means of calculating
characteristic formulae, selecting equivalence classes of processes
up to a pre--specified depth limit on the support set of names. Composed with our
encoding, this characteristic formula can be used to select
characteristic formulae for knots.
%\end{description}

\subsubsection{Spatial logic formulae}

The grammar below (segmented for comprehension) summarizes the syntax
of spatial logic formulae. We employ illustrative examples in the
sequel to provide an intuitive understanding of their meaning
referring the reader to \cite{Caires04} for a more detailed explication
of the semantics.

\begin{mathpar}
  \inferrule* [lab=boolean] {} {{A,B} \bc T \;|\; \neg A \;|\; A \wedge B \;|\; \eta = \eta'}
  \and
  \inferrule* [lab=spatial] {} {|\; \pzero \;|\; A | B \;|\; x \text{\textregistered} A \;|\; \forall x . A \;|\;  H x . A}
  \and
  \inferrule* [lab=behavioral] {} {|\; \alpha . A}
  \and 
  \inferrule* [lab=recursion] {} {|\; X(\vec{u}) \;|\; \mu X(\vec{u}) . A}
  \and
  \inferrule* [lab=action] {} {\alpha \bc \langle x?(\vec{y}) \rangle \;|\; \langle x!(\vec{y}) \rangle \;|\; \langle \tau \rangle}
  \and 
  \inferrule* [lab=name] {} {\eta \bc x \;|\; \tau}
\end{mathpar} 

% subsection characteristic_formulae (end)   	 

\subsection{Example formulae}\label{sub:example_formulae_} % (fold)

\subsubsection{Crossing as formula.}
% 
% \begin{align*}
%   \frac{d}{dx} \sin x &= \cos x 
%   & \frac{d}{dx} e^x &= e^x \\
%   \frac{d}{dx} \cos x &= - \sin x 
%   & \frac{d}{dx} \log x &= \frac{1}{x} \\
% \end{align*} 

\begin{align*}
 \mu C(x_{0},x_{1},y_{0},y_{1},u).&(\langle x_{0}?(z) \rangle(\langle u! \rangle\langle y_{1}!z \rangle C(x_{0},x_{1},y_{0},y_{1},u)) & \\
  & \wedge \langle y_{1}?(z) \rangle (\langle u! \rangle \langle x_{0}!z \rangle C(x_{0},x_{1},y_{0},y_{1},u)) & \\
  & \wedge \langle x_{1}?(z) \rangle (\langle u? \rangle \langle y_{0}!z \rangle C(x_{0},x_{1},y_{0},y_{1},u)) & \\
  & \wedge \langle y_{0}?(z) \rangle (\langle u? \rangle \langle x_{1}!z \rangle C(x_{0},x_{1},y_{0},y_{1},u))) &
\end{align*}

The lexicographical similarity between the shape of this formulae and
the shape of definition of the process representing a crossing reveals
the intuitive meaning of this formulae. It describes the capabilities
of a process that has the right to represent a crossing. For example
it picks out processes that may perform an input on the port $x_0$ in
its initial menu of capabilities. What differentiates the formula
from the process, however, is that the crossing process is the
smallest candidate to satisfy the formula. Infinitely many other
processes -- with internal behavior hidden behind this interface, so
to speak -- also satisfy this formula. Even this simple formula,
then, can be seen to open a new view onto knots, providing a
computational interpretation of \emph{virtual} knots.

Note that this formula is derived by hand. A similar formula can be
derived by employing Caires' calculation of characteristic formula
\cite{Caires04} to the process representing a crossing. In light of
this discussion, we let
$\meaningof{C}_{\phi}(x0,x1,y0,y1,u)$ denote a formula specifying the
dynamics we wish to capture of a crossing. To guarantee we preserve
the shape of the interface and minimal semantics we demand that
$\meaningof{C}_{\phi}(x0,x1,y0,y1,u) \Rightarrow
\textbf{C}(x0,x1,y0,y1,u)$ where $\textbf{C}(x0,x1,y0,y1,u)$ denotes
the formula above.
                            
\subsubsection{Crossing number constraints.}
The moral content of the context lemma (Lemma \ref{context}) is that the notion of
``locality'' in the Reidemeister moves is effectively captured by the
parallel composition operator of the process calculus. This intuition
extends through the logic. Given a formula,
$\meaningof{C}_{\phi}(x0,x1,y0,y1,u)$, we can use the structural
connectives to specify constraints on crossing numbers, such as at
least $n$ crossings, or exactly $n$ crossings.
\begin{mathpar}
  \inferrule* [lab=at-least-n] {} { K^{\geq n}_{\phi}(\vec{xs},\vec{ys}) := \Pi_{i=0}^{n-1} Hu . \meaningof{C}_{\phi}(xs_i,ys_i,u) | T }
  \and 
  \inferrule* [lab=exactly-n] {} { K^{= n}_{\phi}(\vec{xs},\vec{ys}) := \Pi_{i=0}^{n-1} Hu . \meaningof{C}_{\phi}(xs_i,ys_i,u) | \neg (\forall x_0,y_0,x_1,y_1,u . \meaningof{C}_{\phi}(x_0,y_0,x_1,y_1,u) | T) }
\end{mathpar}

To round out this section, recall that the encoding of an $n$-crossing
knot decomposes into a parallel composition of $n$ \emph{copies} of a
crossing process together with a wiring harness. To specify different
knot classes with the same crossing number amounts to specifying
logical constraints on the wiring harness. In the interest of space,
we defer examples to a forthcoming paper. Suffice it to say that both
the conditions ``alternating knot'' and ``contains the tangle
corresponding to 5/3'' are expressible. For example, it is possible to
calculate the characteristic formula of a process corresponding to the
tangle 5/3 and conjoin it into the classifying formula via the
composition connective of the logic.

Finally, we wish to observe that it is entirely within reason to
contemplate a more domain-specific version of spatial logic tailored
to the shape of processes in the image of the encoding. Such a
domain-specific logic would have a better claim to the title formal
language of knot properties.

% subsection example_formulae_ (end)

% section knots_as_processes (end) 

% section spatial logic via knots (end)

\section{Conclusions and future work}

\paragraph{Testing physical space}
You, gentle reader, may wonder why of all the theorems to be proved
given this set up we pick the one above. In some sense it's hardly
central to quantum mechanics. We see it as central in the sense that
it firmly establishes a notion of physical space arising from a notion
of the equivalence of behavior. Relating bisimulation to a metric is a
big step forward, but one is faced with interpreting the relationship
of that metric space to something more physical. Quantum mechanical
notions of ``physical'' space are still far from intuitive, but by
relating this idea of distance as testing to calculations that predict
physical circumstances we are making a not insignificant step forward
toward an understanding of the physical space we inhabit as
essentially dynamic.

\paragraph{Effectivity and simulation}
One of the observations we have yet to make is that the entire program
spelled out here is effective. We have built various interpreters for
the reflective calculus at work in this interpretation. In principle,
then, we can simulate quantum mechanics on a computer. The place where
the simulation may lose fidelity is the infinitely branching summation
for the annihilator.

In this connection i also want to point out that the evaluation style
calculation of the inner product puts the non-determinism of the
summation right at the heart of measurement. This suggests that
Milner's original reduction-based formulation of the dynamics of his
calculi in terms of sums was not just notationally suggestive of a
notion of measure-and-continue but captured some significant part of
the physics.

\paragraph{Quantum continuations}
In light of this last observation i want to point out that the
predominant account of quantum mechanics is missing a key aspect of a
truly compositional story of the physical situation. In a real lab,
when a measurement is made the observation can be made to feed into
another device that then makes another measurement conditioned on the
results of the first. This means that after the superposition was
collapsed the entire experimental set up remained in
superposition. While QM offers a means of writing this down it doesn't
quite line up well with the well-trodden formulation of computation
and continuation that we see so succinctly expressed in Milner's
calculi. This suggests that there might be advantages to this account
of dynamics waiting to be explored.

\paragraph{Quantum logic}
In this connection, we also note that by virtue of having the
Hennessy-Milner construction, we can pull the construction through the
interpretation of QM. This gives us a natural candidate for a quantum
logic that enjoys an extremely tight connection with it's domain of
interpretation, making the construction much less ad hoc (rather it is
the image of functor!).

\paragraph{Quantum probabiity}
i have questions about the basis of the interpretation of inner
product as probability amplitude. In particular, using which
axiomatization of probability theory does the notion of probability
amplitude earn the right to be so dubbed? In other words, where is the
proof that the operation for calculating a probability amplitude (and
then squaring) satisfies the axioms of what it means to calculate a
probability? Even if such a proof exists (i have yet to find it in the
literature), i wonder if it might not be possible to turn things on
their heads. Can we view the calculation of the probability amplitude
as an axiomatization of probability? If so, then the definition we
give for calculating probability amplitude may provide the basis for
an \emph{effective} theory of probability.

\paragraph{Quantum vs ``biological'' information}
Finally, i want to conclude with a more philosophical observation. At
a recent workshop in which QM was a predominant topic i noticed
something about quantum information. The speaker was giving a riveting
discussion of axiomatic QM and showing how properties of ``no
cloning'' and ``no deleting'' emerged as consequences of the
axiomatization. Theorems of this form are necessary to give us a sense
of confidence that our axioms characterize the physical theory. What
struck me, though, was that if quantum information is neither erasable
nor replicable it is markedly different from \emph{life}. Two of the
things we know about life is that

\begin{itemize}
  \item it ends;
  \item to gain some measure of persistence, to transcend it's
    finitude it is imminently copyable.
\end{itemize}

Both of these qualities are summarized succinctly in the aphorism: all
flesh is grass. For me these two kinds of ``information'' -- call them
quantum and biological -- are end points on a spectrum of strategies
for persistence. At one end, we have those curious entities that enjoy
uniqueness and permanence; at the other, we have those who in the face
of a certain end and an uncertain present make a go of passing
something on. To me one of the more remarkable aspects of the latter
strategy is that in the presence of noise (and certain features of
copying) we get a kind of dynamism, a chance for improvement against a
given persistent condition.

% subsection other_calculi_other_bisimulations_and_geometry_as_behavior (end)




% section conclusion (end)

%\documentclass[12pt]{llncs}
%\documentclass{jktr}

\usepackage[pdftex]{hyperref}                   
\usepackage {listings}
\usepackage {mathpartir}
\usepackage{bcprules}
%\usepackage{listings}
                       
\usepackage{graphicx} 
%\usepackage[margins=2.5cm,nohead,nofoot]{geometry}
%\usepackage{geometry}
\usepackage{amsfonts}
\usepackage{amstext}
\usepackage{latexsym}
\usepackage{amssymb}
\usepackage{color}


%\include{myPreamble}
\include{qm2pi.local} 

%\ifpdf
%\usepackage[pdftex]{graphicx}
%\else
%\usepackage{graphicx}
%\fi

 % \ifpdf
%  \usepackage{pdfsync}
%  \if


%\title{Brief Article}
%\author{David F. Snyder}
%\author{L.G. Meredith}

%\address{Dept. of Math., Texas State University--San Marcos, San Marcos, TX 78666}
       
\pagestyle{empty}


\begin{document}

\lstset{language=[Objective]Caml,frame=shadowbox}

\input{qm2pi.front}

% section front matter (end)

\input{qm2pi.intro} 
 
% section introduction (end)

% \input{qm2pi.knotations} 

% section notation (end)

\input{qm2pi.process.calculi} 

% section concurrent_process_calculi_and_spatial_logics_ (end)
    
%\input{qm2pi.knots2pi} 

%\input{qm2pi.trefoil} 

%\input{qm2pi.mainthm} 

% subsection basic_interpretation (end)

%\input{qm2pi.rho.presentation} 
\subsection{The syntax and semantics of the notation system}\label{sub:the_syntax_and_semantics_of_the_notation_system} % (fold)

We now summarize a technical presentation of the calculus that
embodies our theory of dynamics. The typical presentation of such a
calculus follows the style of giving generators and relations on
them. The grammar, below, describing term constructors, freely
generates the set of processes, $\Proc$. This set is then quotiented
by a relation known as structural congruence and it is over this set
that the notion of dynamics is expressed. This presentation is
essentially that of \cite{MeredithR05} with the addition of
polyadicity and summation. For readability we have relegated some of
the technical subtleties to an appendix.

\subsubsection{Process grammar}\label{subsub:process_grammar}

\begin{mathpar}
  \inferrule* [lab=synchronization] {} {{M} \bc \pzero \;|\; x?F \;|\; x!C }
  \and
  \inferrule* [lab=abstraction] {} {{F} \bc (x)P}
  \and
  \inferrule* [lab=concretion] {} {{C} \bc \langle Q \rangle}
  \and
  \inferrule* [lab=process] {} {{P,Q} \bc M \;| \;P|Q \;|\; @{x}}
  \and
  \inferrule* [lab=name] {} {{x} \bc \quotep{P}}
\end{mathpar} 

Note that $\vec{x}$ (resp. $\vec{P}$) denotes a vector of names
(resp. processes) of length $|\vec{x}|$ (resp. $|\vec{P}|$). We adopt
the following useful abbreviations.

\begin{mathpar}
   x?(\vec{y}).P := x.(\vec{y})P \and  x\clift{\vec{P}} := x.\clift{\vec{P}}
   \and x!(y) := \lift{x}{\dropn{y}}
   \and \Pi_{i=0}^{n-1}P_i := P_0 | \ldots | P_{n-1}
\end{mathpar}

\subsubsection{Structural congruence}

\paragraph{Free and bound names and alpha-equivalence.} At the
core of structural equivalence is alpha-equivalence which identifies
process that are the same up to a change of variable. Formally, we
recognize the distinction between free and bound names. The free names
of a process, $\freenames{P}$, may be calculated recursively as
follows:

\begin{mathpar}
\freenames{\pzero} := \emptyset
  \and \\
  \freenames{x?(y).P} := \{ x \} \cup (\freenames{P} \setminus \{ y \})
  \and 
  \freenames{x!\langle P \rangle} := \{ x \} \cup \{ P \} 
  \and \\
  \freenames{P|Q} := \freenames{P} \cup \freenames{Q}
  \and \\
  \freenames{@{x}} := \{ x \}
\end{mathpar}

$\pi$
$\quotep{\pi}$

$\freenames{-} : \pi \to \mathcal{P}(\quotep{\pi})$

\begin{eqnarray*}
  \freenames{\pzero} & := & \emptyset \\
  \freenames{x?(y).P} & := & \{ x \} \cup (\freenames{P} \setminus \{ y \}) \\
  \freenames{x!\langle P \rangle} & := & \{ x \} \cup \{ P \} \\
  \freenames{P|Q} & := & \freenames{P} \cup \freenames{Q} \\
  \freenames{\dropn{x}} & := & \{ x \}
\end{eqnarray*}

The bound names of a process, $\boundnames{P}$, are those names occurring in $P$
that are not free. For example, in $x?(y).0$, the name $x$ is free, while $y$ is bound.

\begin{mathpar}
  \inferrule* [lab=monoidal-laws] {} { P|Q \equiv Q|P \and P|0 \equiv P \and P|(Q|R) \equiv (P|Q)|R }
\end{mathpar}

\begin{mathpar}
  \inferrule* [lab=alpha-equivalence] {} { (x)P \equiv (y)P\{y/x\} \and y \not\in \freenames{P} }
\end{mathpar}

\begin{definition}
Then two processes, $P,Q$, are alpha-equivalent if $P = Q\{\vec{y}/\vec{x}\}$ for
some $\vec{x} \in \boundnames{Q},\vec{y} \in \boundnames{P}$, where $Q\{\vec{y}/\vec{x}\}$
denotes the capture-avoiding substitution of $\vec{y}$ for $\vec{x}$ in $Q$.
\end{definition}

\begin{definition}
  The {\em structural congruence} \cite{SangiorgiWalker} , $\equiv$,
  between processes is the least congruence containing
  alpha-equivalence, satisfying the abelian monoid laws
  (associativity, commutativity and $\pzero$ as identity) for parallel
  composition $|$ and for summation $+$.
\end{definition}

\subsection{Name equivalence}

We take name equivalence, written $\nameeq$, to be the smallest
equivalence relation generated by the following rules.

\begin{mathpar}
\inferrule*[lab=Quote-drop]
{ }
{ \quotep{@{x}} \nameeq x }

\inferrule*[lab=Struct-equiv]
{ P \scong Q }
{ \quotep{P} \nameeq \quotep{Q} }
\end{mathpar}

The astute reader will have noticed that the mutual recursion of names
and processes imposes a mutual recursion on alpha-equivalence and
structural equivalence via name-equivalence. Fortunately, all of this
works out pleasantly and we may calculate in the natural way, free of
concern. The reader interested in the details is referred to the
appendix \ref{appendix:rho_details}.

\subsection{Substitution}

We use $\Proc$ for the set of processes, $\QProc$ for the set of
names, and $\id{\{}\vec{y} / \vec{x} \id{\}}$ to denote partial maps,
$s : \QProc \rightarrow \QProc$. A map, $s$ lifts, uniquely, to a map
on process terms, $\widehat{s} : \Proc \rightarrow \Proc$ by the
following equations.

\begin{mathpar}
  (0) \psubstp{Q}{P} := 0 \\
  (R \juxtap S) \psubstp{Q}{P}
  :=    
  (R)\psubstp{Q}{P} \juxtap (S) \psubstp{Q}{P} \\
  (x?(y).R) \psubstp{Q}{P}    
  :=    
  (x)\substp{Q}{P} (z)\concat( (R \psubstn{z}{y}) \psubstp{Q}{P} ) \\
  (\lift{x}{R}) \psubstp{Q}{P}  
  :=
  \lift{(x)\substp{Q}{P}}{ R \psubstp{Q}{P} } \\
%   (\dropn{x})  \psubstp{Q}{P}       
%   := 
%   \left\{ 
%     \begin{array}{ccc} 
%       \dropn{\quotep{Q}} & & x \nameeq \quotep{P} \\
%       \dropn{x} & & otherwise \\
%     \end{array}
%   \right. 
  (\dropn{x})  \psubstp{Q}{P}       
  := 
  \left\{ 
    \begin{array}{ccc} 
      Q & & x \nameeq \quotep{P} \\
      \dropn{x} & & otherwise \\
    \end{array}
  \right.
\end{mathpar}
 

where

\begin{eqnarray}
  (x)\id{\{} \lpquote Q \rpquote / \lpquote P \rpquote \id{\}}            = 
  \left\{ 
    \begin{array}{ccc}
      \lpquote Q \rpquote & & x \nameeq \lpquote P \rpquote \\
      x & & otherwise \\
    \end{array}
  \right. \nonumber
\end{eqnarray}

and $z$ is chosen distinct from $\quotep{P}$, $\quotep{Q}$, the free
names in $Q$, and all the names in $R$. Our $\alpha$-equivalence will
be built in the standard way from this substitution.

\begin{remark}\label{rem:no_self_referential_names}
  One consequence of these definitions is that $\forall P. \quotep{P}
  \not\in \freenames{P}$.
\end{remark}

\subsection{ Dynamic quote: an example }

Anticipating something of what's to come, consider applying the
substitution, $\widehat{\id{\{}u / z \id{\}}}$, to the following pair
of processes, $\lift{w}{y!(z)}$ and $w[ \lpquote y!(z) \rpquote ]$.

\begin{eqnarray}
	\lift{w}{y!(z)}\widehat{\id{\{}u / z \id{\}}}
		& = &
		\lift{w}{y!(u)} \nonumber\\
	w[ \lpquote y!(z) \rpquote ] \widehat{ \id{\{}u / z \id{\}} }
		& = &
		w[ \lpquote y!(z) \rpquote ] \nonumber
\end{eqnarray}

Because the body of the process between quotes is impervious to
substitution, we get radically different answers. In fact, by
examining the first process in an input context,
e.g. $x?(z).\lift{w}{y!(z)}$, we see that the process under the lift
operator may be shaped by prefixed inputs binding a name inside it. In
this sense, the lift operator will be seen as a way to dynamically
construct processes before reifying them as names.

Finally equipped with these standard features we can present the
dynamics of the calculus.

\subsubsection{Operational semantics} 

Finally, we introduce the computational dynamics. What marks these
algebras as distinct from other more traditionally studied algebraic
structures, e.g. vector spaces or polynomial rings, is the manner in
which dynamics is captured. In traditional structures, dynamics is typically
expressed through morphisms between such structures, as in linear maps
between vector spaces or morphisms between rings. In algebras
associated with the semantics of computation, the dynamics is
expressed as part of the algebraic structure itself, through a
reduction reduction relation typically denoted by $\red$. Below, we
give a recursive presentation of this relation for the calculus used
in the encoding.

$\red \subseteq \pi \times \pi$
$\red : \pi \to \mathcal{P}(\pi)$

\begin{mathpar}
  \inferrule* [lab=Comm] { \textsf{match}( x_{src}, x_{trgt} ) } { x_{trgt}?(y)P \; | \; x_{src}!\langle {Q} \rangle \red P\{\quotep{Q}/y}\} }
  \and \\
  \inferrule* [lab=Par] {{P} \red {P}'} {{{P} | {Q}} \red {{P}' | {Q}}}
  \and
  \inferrule* [lab=Equiv]{{{P} \scong {P}'} \andalso {{P}' \red {Q}'} \andalso {{Q}' \scong {Q}}}{{P} \red {Q}}
\end{mathpar}

\begin{eqnarray*}
  match_{\equiv} (\quotep{P},\quotep{Q}) & := & P \equiv Q \\
  match_{\dagger}(\quotep{P},\quotep{Q}) & := & \forall R. P|Q \red^{*} R => R \red^{*} 0 \\
  match_{K}(\quotep{P},\quotep{Q}) & := & K \mbox{ for some context } K
\end{eqnarray*}

$u?(x)P | u!\langle Q \rangle \red P\{\quotep{Q}/x\}$

%We write $\wred$ for $\red^*$, and $P\red$ if $\exists Q $ such that $ P \red Q$.
We write $P\red$ if $\exists Q $ such that $ P \red Q$ and $P\not\red$, otherwise.

\section{Replication}

As mentioned before, it is known that replication (and hence
recursion) can be implemented in a higher-order process algebra
\cite{SangiorgiWalker}. As our first example of calculation with the
machinery thus far presented we give the construction explicitly in
the {\rhoc}.

\begin{eqnarray}
	D_{x} & := & \prefix{x}{y}{(\binpar{\outputp{x}{y}}{@{y}})} \nonumber\\
	\bangp_{x}{P} & := & \binpar{{x}!\langle{\binpar{D_{x}}{P}}\rangle}{D_{x}} \nonumber
\end{eqnarray}

\begin{eqnarray}
	\bangp_{x}{P} & & \nonumber\\
	=
	& {x}!\langle{(\prefix{x}{y}{(\outputp{x}{y} | @{y})) | P}}\rangle 
	      | \prefix{x}{y}{(\outputp{x}{y} | @{y})} & \nonumber\\
	\red
	& (\outputp{x}{y} | @{y})\substn{\quotep{(\prefix{x}{y}{(@{y} | \outputp{x}{y})) | P}}}{y} & \nonumber\\
	=
	& \outputp{x}{\quotep{(\prefix{x}{y}{(\outputp{x}{y} | @{y})) | P}}}
	  | {(\prefix{x}{y}{(\outputp{x}{y} | @{y})) | P}} & \nonumber\\
	\red
	& \ldots & \nonumber\\
	\red^*
	& P | P | \ldots & \nonumber
\end{eqnarray}

Of course, this encoding, as an implementation, runs away, unfolding
$\bangp{P}$ eagerly. A lazier and more implementable replication
operator, restricted to input-guarded processes, may be obtained as follows.

\begin{eqnarray}
\bangp{\prefix{u}{v}{P}} 
	:= 
	\binpar{\lift{x}{\prefix{u}{v}{(\binpar{D(x)}{P})}}}{D(x)} \nonumber
\end{eqnarray}

\begin{remark}
  Note that the lazier definition still does not deal with summation
  or mixed summation (i.e. sums over input and output). The reader is
  invited to construct definitions of replication that deal with these
  features. 

  Further, the definitions are parameterized in a name, $x$. Can you,
  gentle reader, make a definition that eliminates this parameter and
  guarantees no accidental interaction between the replication
  machinery and the process being replicated -- i.e. no accidental
  sharing of names used by the process to get its work done and the
  name(s) used by the replication to effect copying. This latter
  revision of the definition of replication is crucial to obtaining
  the expected identity $!!P \sim !P$.
\end{remark}

\begin{remark}\label{rem:paradoxical_combinator}
  The reader familiar with the lambda calculus will have noticed the
  similarity between $D$ and the paradoxical combinator.

  [Ed. note: the existence of this seems to suggest we have to be more
  restrictive on the set of processes and names we admit if we are to
  support no-cloning.]
\end{remark}

\subsubsection{Bisimulation}

The computational dynamics gives rise to another kind of equivalence,
the equivalence of computational behavior. As previously mentioned
this is typically captured \emph{via} some form of bisimulation.

% The notion we use in this paper is weak barbed bisimulation
% \cite{milner91polyadicpi}.

The notion we use in this paper is derived from weak barbed
bisimulation \cite{milner91polyadicpi}. 

\begin{definition}
An \emph{observation relation}, $\downarrow_{\mathcal N}$, over a set
of names, $\mathcal N$, is the smallest relation satisfying the rules
below.

\infrule[Out-barb]{y \in {\mathcal N}, \; x \nameeq y}
		  {\outputp{x}{v} \downarrow_{\mathcal N} x}
\infrule[Par-barb]{\mbox{$P\downarrow_{\mathcal N} x$ or $Q\downarrow_{\mathcal N} x$}}
		  {\binpar{P}{Q} \downarrow_{\mathcal N} x}

We write $P \Downarrow_{\mathcal N} x$ if there is $Q$ such that 
$P \wred Q$ and $Q \downarrow_{\mathcal N} x$.
\end{definition}

\begin{definition}
%\label{def.bbisim}
An  ${\mathcal N}$-\emph{barbed bisimulation} over a set of names, ${\mathcal N}$, is a symmetric binary relation 
${\mathcal S}_{\mathcal N}$ between agents such that $P\rel{S}_{\mathcal N}Q$ implies:
\begin{enumerate}
\item If $P \red P'$ then $Q \wred Q'$ and $P'\rel{S}_{\mathcal N} Q'$.
\item If $P\downarrow_{\mathcal N} x$, then $Q\Downarrow_{\mathcal N} x$.
\end{enumerate}
$P$ is ${\mathcal N}$-barbed bisimilar to $Q$, written
$P \wbbisim_{\mathcal N} Q$, if $P \rel{S}_{\mathcal N} Q$ for some ${\mathcal N}$-barbed bisimulation ${\mathcal S}_{\mathcal N}$.
\end{definition}

$\mathcal{R} \subseteq \pi \times \pi$

$P \mathcal{R} Q => \forall P'. P \red P' \Rightarrow \exists Q'. Q \red Q', P' \mathcal{R} Q'$

$P \vdash x \Rightarrow Q \vdash x$

\begin{mathpar}
  \inferrule*[lab=Out-barb]{x \nameeq y}{{y}!\langle{Q}\rangle \vdash x}
  \and
  \inferrule*[lab=Par-barb]{\mbox{$P\vdash x$ or $Q\vdash x$}}{\binpar{P}{Q} \vdash x}
\end{mathpar}

\subsubsection{Contexts}

One of the principle advantages of computational calculi like the
$\pi$-calculus is a well-defined notion of context,
contextual-equivalence and a correlation between
contextual-equivalence and notions of bisimulation. The notion of
context allows the decomposition of a process into (sub-)process and
its syntactic environment, its context. Thus, a context may be
thought of as a process with a ``hole'' (written $\Box$) in it. The
application of a context $M$ to a process $P$, written $M[P]$, is
tantamount to filling the hole in $M$ with $P$. In this paper we do
not need the full weight of this theory, but do make use of the notion
of context in the proof the main theorem. 

\begin{mathpar}
  \inferrule* [lab=summation] {} {{M_{M},M_{N}} \bc \Box \;|\; x.M_{A} \;|\; M_{M}+M_{N}}
  \and
  \inferrule* [lab=agent] {} {{M_{A}} \bc (\vec{x})M_{P} \;| \; \clift{P_0,\ldots,M_{P},\ldots,P_N}}
  \and \\
  \inferrule* [lab=process] {} {{M_{P}} \bc M_{N} \;| \;P|M_{P} }
\end{mathpar} 

\begin{mathpar}
  \inferrule* [lab=sychronization] {} {M_{N} \bc \Box \;|\; x?M_{F} \;|\; x!M_{C}}
  \and
  \inferrule* [lab=abstraction] {} {{M_{F}} \bc (x)M_{P} }
  \and
  \inferrule* [lab=concretion] {} {{M_{C}} \bc \langle M_{P} \rangle }
  \and \\
  \inferrule* [lab=process] {} {{M_{P}} \bc M_{N} \;| \;P|M_{P} }
\end{mathpar}

\begin{definition}[contextual application] Given a context $M$, and
  process $P$, we define the \emph{contextual application}, $M[P] :=
  M\{P/\Box\}$. That is, the contextual application of M to P is the
  substitution of $P$ for $\Box$ in $M$.
\end{definition}

$\meaningof{-} : L \to \mathcal{P}(\pi)$

\begin{mathpar}
  \inferrule* [lab=collection] {} {\meaningof{true} = \pi, \and \meaningof{~E} = \pi \setminus \meaningof{E}, \and \meaningof{E_{1} \& E_{2}} = \meaningof{E_{1}} \cap \meaningof{E_{2}}}
\end{mathpar}

\begin{mathpar}
  \inferrule* [lab=structure] {} {\meaningof{0} = \{ P \in \pi | P \equiv 0 \}, \and \\ \meaningof{E_1 | E_2} = \{ P \in \pi | P \equiv P_{1} | P_{2}, P_{1} \in \meaningof{E_{1}}, P_{2} \in \meaningof{E_2}\} }
\end{mathpar}

\begin{mathpar}
 \inferrule* [lab=behavior] {} {\meaningof{\langle a?b \rangle E} = \{ P \in \pi | P \equiv Q | u?(y)P', \\ \and \\\\ \and \\ \;\;\; u \in \meaningof{a}, \forall z.P'\{z/y\} \in \meaningof{E\{z/b\}}\}, \and \\ \meaningof{a!E} = \{ P \in \pi | P \equiv Q | x!\langle P' \rangle, x \in \meaningof{a} P' \in \meaningof{E}\} }
\end{mathpar}

\begin{mathpar}
 \inferrule* [lab=nominal] {} {\meaningof{\quotep{E}} = \{ \quotep{P} \in \quotep{\pi} | P \in \meaningof{E} \}, \and \meaningof{\quotep{P}} = \{ \quotep{Q} \in \quotep{\pi} | P \equiv Q \} \and \\ \meaningof{@\quotep{E}} = \{ P \in \pi | P \equiv @x, x \in \meaningof{E} \}}
\end{mathpar}

\begin{eqnarray*}
  \\
  \meaningof{-} : TS \to ST
\end{eqnarray*}

\begin{eqnarray*}
  \\
  L : TS \to ST
\end{eqnarray*}

\begin{eqnarray*}
  \\
  P \models E \iff P \in \meaningof{E}
\end{eqnarray*}

\begin{eqnarray*}
  P \approx_{L} Q \iff \forall E \in L. P \models E \iff Q \models E
\end{eqnarray*}

\begin{eqnarray*}
  P \approx_{K} Q
\end{eqnarray*}

\begin{eqnarray*}
  P \approx Q
\end{eqnarray*}

$\approx_{K} = \approx = \approx_{L}$

\subsubsection{Contextual duality}

Note that contexts extend the quotation operation to a family of
operations from processes to names. Given a context, $M$, we can
define a \emph{nominal context}, $\quotep{M}$ by $\quotep{M}[P] :=
\quotep{M[P]}$. To foreshadow what is to come we observe that these
operations enjoy a duality with processes very much like the duality
between vectors and maps from vectors to scalars.

Further, because the calculus is essentially higher-order, we have a
correspondence between contexts and processes. More specifically,
given a name $x$ and a context $M$ we can construct $M^{*}_{x}$ such
that 

\begin{mathpar}
  M^{*}_{x} | \lift{x}{P} \red M[P]
\end{mathpar}

namely,

\begin{mathpar}
  M^{*}_{x} := x?(u).M[\dropn{u}]
\end{mathpar}

The dependence of $M^{*}_{x}$ on a name makes it an abstraction, 

\begin{mathpar}
  M^{*} := (x)x?(u).M[\dropn{u}]
\end{mathpar}

\subsection{Additional notation}

It will sometimes be convenient to denote the process a name
quotes. We already have the notation $x = \quotep{P}$, but it will be
convenient to introduce an alternate notation, $\procn{x}$, when we
want to emphasize the connection to the use of the name. Note that, by
virtue of name equivalence, $\quotep{\procn{x}} \nameeq x$; so, the
notation is consistent with previous definitions.

Further, because names have structure it is possible to effect
substitutions on the basis of that structure. This means we need to
upgrade our notation for substitutions, which we accomplish by
adapting comprehension notation. Thus,

\begin{mathpar}
  P\{ y / x : x \in S \}
\end{mathpar}

is interpreted to mean the process derived from P by replacing (in a
capture-avoiding manner) each occurrence of $x$ in $S$ by $y$. For example,

\begin{mathpar}
  P\{ \quotep{\procn{x}|\procn{x}} / x : x \in \freenames{P} \}
\end{mathpar}

will replace each (occurrence) of a free name $x$ in $P$ by
$\quotep{\procn{x}|\procn{x}}$.

Also, we will avail ourselves of the notation $x^{L}$ and $x^{R}$ to
denote injections of a name into disjoint copies of the name
space. There are numerous ways to accomplish this. One example can be
found in \cite{MeredithR05}. This notation overloads to vectors of
names: $\vec{x}^{\pi} := (x_{i}^{\pi} \; : \; 0 \leq i < |\vec{x}| )$ where $\pi \in \{L,R\}$.

We also use $P^{\Box} := P|\Box$.

In \cite{MeredithR05} an interpretation of the new operator is
given. It turns out that there are several possible interpretations
all enjoying the requisite algebraic properties of the operator (see
\cite{milner91polyadicpi}). We will therefore make liberal use of
$(\nu\; \vec{x})P$.

% subsection the_syntax_and_semantics_of_the_notation_system (end)   

\input{qm2pi.qmops} 

\input{qm2pi.sterngerlach} 

\input{qm2pi.metric} 

% section concurrent_process_calculi (end)

%\input{qm2pi.proofsketch}

% section proof sketch (end)

%\input{qm2pi.slviaknots} 

% section spatial logic via knots (end)

\input{qm2pi.conclusion}

% section conclusion (end)

%\input{qm2pi.dtcodes} 

% section wiring algorithm (end)

\input{qm2pi.ack} 

% section acknowledgments (end)

\newpage


\bibliographystyle{plain}   
\bibliography{../../biblios/main.bib}

\input{qm2pi.rhodetails}

\end{document}

 

% section wiring algorithm (end)

\documentclass[12pt]{llncs}
%\documentclass{jktr}

\usepackage[pdftex]{hyperref}                   
\usepackage {listings}
\usepackage {mathpartir}
\usepackage{bcprules}
%\usepackage{listings}
                       
\usepackage{graphicx} 
%\usepackage[margins=2.5cm,nohead,nofoot]{geometry}
%\usepackage{geometry}
\usepackage{amsfonts}
\usepackage{amstext}
\usepackage{latexsym}
\usepackage{amssymb}
\usepackage{color}


%\include{myPreamble}
\include{qm2pi.local} 

%\ifpdf
%\usepackage[pdftex]{graphicx}
%\else
%\usepackage{graphicx}
%\fi

 % \ifpdf
%  \usepackage{pdfsync}
%  \if


%\title{Brief Article}
%\author{David F. Snyder}
%\author{L.G. Meredith}

%\address{Dept. of Math., Texas State University--San Marcos, San Marcos, TX 78666}
       
\pagestyle{empty}


\begin{document}

\lstset{language=[Objective]Caml,frame=shadowbox}

\input{qm2pi.front}

% section front matter (end)

\input{qm2pi.intro} 
 
% section introduction (end)

% \input{qm2pi.knotations} 

% section notation (end)

\input{qm2pi.process.calculi} 

% section concurrent_process_calculi_and_spatial_logics_ (end)
    
%\input{qm2pi.knots2pi} 

%\input{qm2pi.trefoil} 

%\input{qm2pi.mainthm} 

% subsection basic_interpretation (end)

%\input{qm2pi.rho.presentation} 
\subsection{The syntax and semantics of the notation system}\label{sub:the_syntax_and_semantics_of_the_notation_system} % (fold)

We now summarize a technical presentation of the calculus that
embodies our theory of dynamics. The typical presentation of such a
calculus follows the style of giving generators and relations on
them. The grammar, below, describing term constructors, freely
generates the set of processes, $\Proc$. This set is then quotiented
by a relation known as structural congruence and it is over this set
that the notion of dynamics is expressed. This presentation is
essentially that of \cite{MeredithR05} with the addition of
polyadicity and summation. For readability we have relegated some of
the technical subtleties to an appendix.

\subsubsection{Process grammar}\label{subsub:process_grammar}

\begin{mathpar}
  \inferrule* [lab=synchronization] {} {{M} \bc \pzero \;|\; x?F \;|\; x!C }
  \and
  \inferrule* [lab=abstraction] {} {{F} \bc (x)P}
  \and
  \inferrule* [lab=concretion] {} {{C} \bc \langle Q \rangle}
  \and
  \inferrule* [lab=process] {} {{P,Q} \bc M \;| \;P|Q \;|\; @{x}}
  \and
  \inferrule* [lab=name] {} {{x} \bc \quotep{P}}
\end{mathpar} 

Note that $\vec{x}$ (resp. $\vec{P}$) denotes a vector of names
(resp. processes) of length $|\vec{x}|$ (resp. $|\vec{P}|$). We adopt
the following useful abbreviations.

\begin{mathpar}
   x?(\vec{y}).P := x.(\vec{y})P \and  x\clift{\vec{P}} := x.\clift{\vec{P}}
   \and x!(y) := \lift{x}{\dropn{y}}
   \and \Pi_{i=0}^{n-1}P_i := P_0 | \ldots | P_{n-1}
\end{mathpar}

\subsubsection{Structural congruence}

\paragraph{Free and bound names and alpha-equivalence.} At the
core of structural equivalence is alpha-equivalence which identifies
process that are the same up to a change of variable. Formally, we
recognize the distinction between free and bound names. The free names
of a process, $\freenames{P}$, may be calculated recursively as
follows:

\begin{mathpar}
\freenames{\pzero} := \emptyset
  \and \\
  \freenames{x?(y).P} := \{ x \} \cup (\freenames{P} \setminus \{ y \})
  \and 
  \freenames{x!\langle P \rangle} := \{ x \} \cup \{ P \} 
  \and \\
  \freenames{P|Q} := \freenames{P} \cup \freenames{Q}
  \and \\
  \freenames{@{x}} := \{ x \}
\end{mathpar}

$\pi$
$\quotep{\pi}$

$\freenames{-} : \pi \to \mathcal{P}(\quotep{\pi})$

\begin{eqnarray*}
  \freenames{\pzero} & := & \emptyset \\
  \freenames{x?(y).P} & := & \{ x \} \cup (\freenames{P} \setminus \{ y \}) \\
  \freenames{x!\langle P \rangle} & := & \{ x \} \cup \{ P \} \\
  \freenames{P|Q} & := & \freenames{P} \cup \freenames{Q} \\
  \freenames{\dropn{x}} & := & \{ x \}
\end{eqnarray*}

The bound names of a process, $\boundnames{P}$, are those names occurring in $P$
that are not free. For example, in $x?(y).0$, the name $x$ is free, while $y$ is bound.

\begin{mathpar}
  \inferrule* [lab=monoidal-laws] {} { P|Q \equiv Q|P \and P|0 \equiv P \and P|(Q|R) \equiv (P|Q)|R }
\end{mathpar}

\begin{mathpar}
  \inferrule* [lab=alpha-equivalence] {} { (x)P \equiv (y)P\{y/x\} \and y \not\in \freenames{P} }
\end{mathpar}

\begin{definition}
Then two processes, $P,Q$, are alpha-equivalent if $P = Q\{\vec{y}/\vec{x}\}$ for
some $\vec{x} \in \boundnames{Q},\vec{y} \in \boundnames{P}$, where $Q\{\vec{y}/\vec{x}\}$
denotes the capture-avoiding substitution of $\vec{y}$ for $\vec{x}$ in $Q$.
\end{definition}

\begin{definition}
  The {\em structural congruence} \cite{SangiorgiWalker} , $\equiv$,
  between processes is the least congruence containing
  alpha-equivalence, satisfying the abelian monoid laws
  (associativity, commutativity and $\pzero$ as identity) for parallel
  composition $|$ and for summation $+$.
\end{definition}

\subsection{Name equivalence}

We take name equivalence, written $\nameeq$, to be the smallest
equivalence relation generated by the following rules.

\begin{mathpar}
\inferrule*[lab=Quote-drop]
{ }
{ \quotep{@{x}} \nameeq x }

\inferrule*[lab=Struct-equiv]
{ P \scong Q }
{ \quotep{P} \nameeq \quotep{Q} }
\end{mathpar}

The astute reader will have noticed that the mutual recursion of names
and processes imposes a mutual recursion on alpha-equivalence and
structural equivalence via name-equivalence. Fortunately, all of this
works out pleasantly and we may calculate in the natural way, free of
concern. The reader interested in the details is referred to the
appendix \ref{appendix:rho_details}.

\subsection{Substitution}

We use $\Proc$ for the set of processes, $\QProc$ for the set of
names, and $\id{\{}\vec{y} / \vec{x} \id{\}}$ to denote partial maps,
$s : \QProc \rightarrow \QProc$. A map, $s$ lifts, uniquely, to a map
on process terms, $\widehat{s} : \Proc \rightarrow \Proc$ by the
following equations.

\begin{mathpar}
  (0) \psubstp{Q}{P} := 0 \\
  (R \juxtap S) \psubstp{Q}{P}
  :=    
  (R)\psubstp{Q}{P} \juxtap (S) \psubstp{Q}{P} \\
  (x?(y).R) \psubstp{Q}{P}    
  :=    
  (x)\substp{Q}{P} (z)\concat( (R \psubstn{z}{y}) \psubstp{Q}{P} ) \\
  (\lift{x}{R}) \psubstp{Q}{P}  
  :=
  \lift{(x)\substp{Q}{P}}{ R \psubstp{Q}{P} } \\
%   (\dropn{x})  \psubstp{Q}{P}       
%   := 
%   \left\{ 
%     \begin{array}{ccc} 
%       \dropn{\quotep{Q}} & & x \nameeq \quotep{P} \\
%       \dropn{x} & & otherwise \\
%     \end{array}
%   \right. 
  (\dropn{x})  \psubstp{Q}{P}       
  := 
  \left\{ 
    \begin{array}{ccc} 
      Q & & x \nameeq \quotep{P} \\
      \dropn{x} & & otherwise \\
    \end{array}
  \right.
\end{mathpar}
 

where

\begin{eqnarray}
  (x)\id{\{} \lpquote Q \rpquote / \lpquote P \rpquote \id{\}}            = 
  \left\{ 
    \begin{array}{ccc}
      \lpquote Q \rpquote & & x \nameeq \lpquote P \rpquote \\
      x & & otherwise \\
    \end{array}
  \right. \nonumber
\end{eqnarray}

and $z$ is chosen distinct from $\quotep{P}$, $\quotep{Q}$, the free
names in $Q$, and all the names in $R$. Our $\alpha$-equivalence will
be built in the standard way from this substitution.

\begin{remark}\label{rem:no_self_referential_names}
  One consequence of these definitions is that $\forall P. \quotep{P}
  \not\in \freenames{P}$.
\end{remark}

\subsection{ Dynamic quote: an example }

Anticipating something of what's to come, consider applying the
substitution, $\widehat{\id{\{}u / z \id{\}}}$, to the following pair
of processes, $\lift{w}{y!(z)}$ and $w[ \lpquote y!(z) \rpquote ]$.

\begin{eqnarray}
	\lift{w}{y!(z)}\widehat{\id{\{}u / z \id{\}}}
		& = &
		\lift{w}{y!(u)} \nonumber\\
	w[ \lpquote y!(z) \rpquote ] \widehat{ \id{\{}u / z \id{\}} }
		& = &
		w[ \lpquote y!(z) \rpquote ] \nonumber
\end{eqnarray}

Because the body of the process between quotes is impervious to
substitution, we get radically different answers. In fact, by
examining the first process in an input context,
e.g. $x?(z).\lift{w}{y!(z)}$, we see that the process under the lift
operator may be shaped by prefixed inputs binding a name inside it. In
this sense, the lift operator will be seen as a way to dynamically
construct processes before reifying them as names.

Finally equipped with these standard features we can present the
dynamics of the calculus.

\subsubsection{Operational semantics} 

Finally, we introduce the computational dynamics. What marks these
algebras as distinct from other more traditionally studied algebraic
structures, e.g. vector spaces or polynomial rings, is the manner in
which dynamics is captured. In traditional structures, dynamics is typically
expressed through morphisms between such structures, as in linear maps
between vector spaces or morphisms between rings. In algebras
associated with the semantics of computation, the dynamics is
expressed as part of the algebraic structure itself, through a
reduction reduction relation typically denoted by $\red$. Below, we
give a recursive presentation of this relation for the calculus used
in the encoding.

$\red \subseteq \pi \times \pi$
$\red : \pi \to \mathcal{P}(\pi)$

\begin{mathpar}
  \inferrule* [lab=Comm] { \textsf{match}( x_{src}, x_{trgt} ) } { x_{trgt}?(y)P \; | \; x_{src}!\langle {Q} \rangle \red P\{\quotep{Q}/y}\} }
  \and \\
  \inferrule* [lab=Par] {{P} \red {P}'} {{{P} | {Q}} \red {{P}' | {Q}}}
  \and
  \inferrule* [lab=Equiv]{{{P} \scong {P}'} \andalso {{P}' \red {Q}'} \andalso {{Q}' \scong {Q}}}{{P} \red {Q}}
\end{mathpar}

\begin{eqnarray*}
  match_{\equiv} (\quotep{P},\quotep{Q}) & := & P \equiv Q \\
  match_{\dagger}(\quotep{P},\quotep{Q}) & := & \forall R. P|Q \red^{*} R => R \red^{*} 0 \\
  match_{K}(\quotep{P},\quotep{Q}) & := & K \mbox{ for some context } K
\end{eqnarray*}

$u?(x)P | u!\langle Q \rangle \red P\{\quotep{Q}/x\}$

%We write $\wred$ for $\red^*$, and $P\red$ if $\exists Q $ such that $ P \red Q$.
We write $P\red$ if $\exists Q $ such that $ P \red Q$ and $P\not\red$, otherwise.

\section{Replication}

As mentioned before, it is known that replication (and hence
recursion) can be implemented in a higher-order process algebra
\cite{SangiorgiWalker}. As our first example of calculation with the
machinery thus far presented we give the construction explicitly in
the {\rhoc}.

\begin{eqnarray}
	D_{x} & := & \prefix{x}{y}{(\binpar{\outputp{x}{y}}{@{y}})} \nonumber\\
	\bangp_{x}{P} & := & \binpar{{x}!\langle{\binpar{D_{x}}{P}}\rangle}{D_{x}} \nonumber
\end{eqnarray}

\begin{eqnarray}
	\bangp_{x}{P} & & \nonumber\\
	=
	& {x}!\langle{(\prefix{x}{y}{(\outputp{x}{y} | @{y})) | P}}\rangle 
	      | \prefix{x}{y}{(\outputp{x}{y} | @{y})} & \nonumber\\
	\red
	& (\outputp{x}{y} | @{y})\substn{\quotep{(\prefix{x}{y}{(@{y} | \outputp{x}{y})) | P}}}{y} & \nonumber\\
	=
	& \outputp{x}{\quotep{(\prefix{x}{y}{(\outputp{x}{y} | @{y})) | P}}}
	  | {(\prefix{x}{y}{(\outputp{x}{y} | @{y})) | P}} & \nonumber\\
	\red
	& \ldots & \nonumber\\
	\red^*
	& P | P | \ldots & \nonumber
\end{eqnarray}

Of course, this encoding, as an implementation, runs away, unfolding
$\bangp{P}$ eagerly. A lazier and more implementable replication
operator, restricted to input-guarded processes, may be obtained as follows.

\begin{eqnarray}
\bangp{\prefix{u}{v}{P}} 
	:= 
	\binpar{\lift{x}{\prefix{u}{v}{(\binpar{D(x)}{P})}}}{D(x)} \nonumber
\end{eqnarray}

\begin{remark}
  Note that the lazier definition still does not deal with summation
  or mixed summation (i.e. sums over input and output). The reader is
  invited to construct definitions of replication that deal with these
  features. 

  Further, the definitions are parameterized in a name, $x$. Can you,
  gentle reader, make a definition that eliminates this parameter and
  guarantees no accidental interaction between the replication
  machinery and the process being replicated -- i.e. no accidental
  sharing of names used by the process to get its work done and the
  name(s) used by the replication to effect copying. This latter
  revision of the definition of replication is crucial to obtaining
  the expected identity $!!P \sim !P$.
\end{remark}

\begin{remark}\label{rem:paradoxical_combinator}
  The reader familiar with the lambda calculus will have noticed the
  similarity between $D$ and the paradoxical combinator.

  [Ed. note: the existence of this seems to suggest we have to be more
  restrictive on the set of processes and names we admit if we are to
  support no-cloning.]
\end{remark}

\subsubsection{Bisimulation}

The computational dynamics gives rise to another kind of equivalence,
the equivalence of computational behavior. As previously mentioned
this is typically captured \emph{via} some form of bisimulation.

% The notion we use in this paper is weak barbed bisimulation
% \cite{milner91polyadicpi}.

The notion we use in this paper is derived from weak barbed
bisimulation \cite{milner91polyadicpi}. 

\begin{definition}
An \emph{observation relation}, $\downarrow_{\mathcal N}$, over a set
of names, $\mathcal N$, is the smallest relation satisfying the rules
below.

\infrule[Out-barb]{y \in {\mathcal N}, \; x \nameeq y}
		  {\outputp{x}{v} \downarrow_{\mathcal N} x}
\infrule[Par-barb]{\mbox{$P\downarrow_{\mathcal N} x$ or $Q\downarrow_{\mathcal N} x$}}
		  {\binpar{P}{Q} \downarrow_{\mathcal N} x}

We write $P \Downarrow_{\mathcal N} x$ if there is $Q$ such that 
$P \wred Q$ and $Q \downarrow_{\mathcal N} x$.
\end{definition}

\begin{definition}
%\label{def.bbisim}
An  ${\mathcal N}$-\emph{barbed bisimulation} over a set of names, ${\mathcal N}$, is a symmetric binary relation 
${\mathcal S}_{\mathcal N}$ between agents such that $P\rel{S}_{\mathcal N}Q$ implies:
\begin{enumerate}
\item If $P \red P'$ then $Q \wred Q'$ and $P'\rel{S}_{\mathcal N} Q'$.
\item If $P\downarrow_{\mathcal N} x$, then $Q\Downarrow_{\mathcal N} x$.
\end{enumerate}
$P$ is ${\mathcal N}$-barbed bisimilar to $Q$, written
$P \wbbisim_{\mathcal N} Q$, if $P \rel{S}_{\mathcal N} Q$ for some ${\mathcal N}$-barbed bisimulation ${\mathcal S}_{\mathcal N}$.
\end{definition}

$\mathcal{R} \subseteq \pi \times \pi$

$P \mathcal{R} Q => \forall P'. P \red P' \Rightarrow \exists Q'. Q \red Q', P' \mathcal{R} Q'$

$P \vdash x \Rightarrow Q \vdash x$

\begin{mathpar}
  \inferrule*[lab=Out-barb]{x \nameeq y}{{y}!\langle{Q}\rangle \vdash x}
  \and
  \inferrule*[lab=Par-barb]{\mbox{$P\vdash x$ or $Q\vdash x$}}{\binpar{P}{Q} \vdash x}
\end{mathpar}

\subsubsection{Contexts}

One of the principle advantages of computational calculi like the
$\pi$-calculus is a well-defined notion of context,
contextual-equivalence and a correlation between
contextual-equivalence and notions of bisimulation. The notion of
context allows the decomposition of a process into (sub-)process and
its syntactic environment, its context. Thus, a context may be
thought of as a process with a ``hole'' (written $\Box$) in it. The
application of a context $M$ to a process $P$, written $M[P]$, is
tantamount to filling the hole in $M$ with $P$. In this paper we do
not need the full weight of this theory, but do make use of the notion
of context in the proof the main theorem. 

\begin{mathpar}
  \inferrule* [lab=summation] {} {{M_{M},M_{N}} \bc \Box \;|\; x.M_{A} \;|\; M_{M}+M_{N}}
  \and
  \inferrule* [lab=agent] {} {{M_{A}} \bc (\vec{x})M_{P} \;| \; \clift{P_0,\ldots,M_{P},\ldots,P_N}}
  \and \\
  \inferrule* [lab=process] {} {{M_{P}} \bc M_{N} \;| \;P|M_{P} }
\end{mathpar} 

\begin{mathpar}
  \inferrule* [lab=sychronization] {} {M_{N} \bc \Box \;|\; x?M_{F} \;|\; x!M_{C}}
  \and
  \inferrule* [lab=abstraction] {} {{M_{F}} \bc (x)M_{P} }
  \and
  \inferrule* [lab=concretion] {} {{M_{C}} \bc \langle M_{P} \rangle }
  \and \\
  \inferrule* [lab=process] {} {{M_{P}} \bc M_{N} \;| \;P|M_{P} }
\end{mathpar}

\begin{definition}[contextual application] Given a context $M$, and
  process $P$, we define the \emph{contextual application}, $M[P] :=
  M\{P/\Box\}$. That is, the contextual application of M to P is the
  substitution of $P$ for $\Box$ in $M$.
\end{definition}

$\meaningof{-} : L \to \mathcal{P}(\pi)$

\begin{mathpar}
  \inferrule* [lab=collection] {} {\meaningof{true} = \pi, \and \meaningof{~E} = \pi \setminus \meaningof{E}, \and \meaningof{E_{1} \& E_{2}} = \meaningof{E_{1}} \cap \meaningof{E_{2}}}
\end{mathpar}

\begin{mathpar}
  \inferrule* [lab=structure] {} {\meaningof{0} = \{ P \in \pi | P \equiv 0 \}, \and \\ \meaningof{E_1 | E_2} = \{ P \in \pi | P \equiv P_{1} | P_{2}, P_{1} \in \meaningof{E_{1}}, P_{2} \in \meaningof{E_2}\} }
\end{mathpar}

\begin{mathpar}
 \inferrule* [lab=behavior] {} {\meaningof{\langle a?b \rangle E} = \{ P \in \pi | P \equiv Q | u?(y)P', \\ \and \\\\ \and \\ \;\;\; u \in \meaningof{a}, \forall z.P'\{z/y\} \in \meaningof{E\{z/b\}}\}, \and \\ \meaningof{a!E} = \{ P \in \pi | P \equiv Q | x!\langle P' \rangle, x \in \meaningof{a} P' \in \meaningof{E}\} }
\end{mathpar}

\begin{mathpar}
 \inferrule* [lab=nominal] {} {\meaningof{\quotep{E}} = \{ \quotep{P} \in \quotep{\pi} | P \in \meaningof{E} \}, \and \meaningof{\quotep{P}} = \{ \quotep{Q} \in \quotep{\pi} | P \equiv Q \} \and \\ \meaningof{@\quotep{E}} = \{ P \in \pi | P \equiv @x, x \in \meaningof{E} \}}
\end{mathpar}

\begin{eqnarray*}
  \\
  \meaningof{-} : TS \to ST
\end{eqnarray*}

\begin{eqnarray*}
  \\
  L : TS \to ST
\end{eqnarray*}

\begin{eqnarray*}
  \\
  P \models E \iff P \in \meaningof{E}
\end{eqnarray*}

\begin{eqnarray*}
  P \approx_{L} Q \iff \forall E \in L. P \models E \iff Q \models E
\end{eqnarray*}

\begin{eqnarray*}
  P \approx_{K} Q
\end{eqnarray*}

\begin{eqnarray*}
  P \approx Q
\end{eqnarray*}

$\approx_{K} = \approx = \approx_{L}$

\subsubsection{Contextual duality}

Note that contexts extend the quotation operation to a family of
operations from processes to names. Given a context, $M$, we can
define a \emph{nominal context}, $\quotep{M}$ by $\quotep{M}[P] :=
\quotep{M[P]}$. To foreshadow what is to come we observe that these
operations enjoy a duality with processes very much like the duality
between vectors and maps from vectors to scalars.

Further, because the calculus is essentially higher-order, we have a
correspondence between contexts and processes. More specifically,
given a name $x$ and a context $M$ we can construct $M^{*}_{x}$ such
that 

\begin{mathpar}
  M^{*}_{x} | \lift{x}{P} \red M[P]
\end{mathpar}

namely,

\begin{mathpar}
  M^{*}_{x} := x?(u).M[\dropn{u}]
\end{mathpar}

The dependence of $M^{*}_{x}$ on a name makes it an abstraction, 

\begin{mathpar}
  M^{*} := (x)x?(u).M[\dropn{u}]
\end{mathpar}

\subsection{Additional notation}

It will sometimes be convenient to denote the process a name
quotes. We already have the notation $x = \quotep{P}$, but it will be
convenient to introduce an alternate notation, $\procn{x}$, when we
want to emphasize the connection to the use of the name. Note that, by
virtue of name equivalence, $\quotep{\procn{x}} \nameeq x$; so, the
notation is consistent with previous definitions.

Further, because names have structure it is possible to effect
substitutions on the basis of that structure. This means we need to
upgrade our notation for substitutions, which we accomplish by
adapting comprehension notation. Thus,

\begin{mathpar}
  P\{ y / x : x \in S \}
\end{mathpar}

is interpreted to mean the process derived from P by replacing (in a
capture-avoiding manner) each occurrence of $x$ in $S$ by $y$. For example,

\begin{mathpar}
  P\{ \quotep{\procn{x}|\procn{x}} / x : x \in \freenames{P} \}
\end{mathpar}

will replace each (occurrence) of a free name $x$ in $P$ by
$\quotep{\procn{x}|\procn{x}}$.

Also, we will avail ourselves of the notation $x^{L}$ and $x^{R}$ to
denote injections of a name into disjoint copies of the name
space. There are numerous ways to accomplish this. One example can be
found in \cite{MeredithR05}. This notation overloads to vectors of
names: $\vec{x}^{\pi} := (x_{i}^{\pi} \; : \; 0 \leq i < |\vec{x}| )$ where $\pi \in \{L,R\}$.

We also use $P^{\Box} := P|\Box$.

In \cite{MeredithR05} an interpretation of the new operator is
given. It turns out that there are several possible interpretations
all enjoying the requisite algebraic properties of the operator (see
\cite{milner91polyadicpi}). We will therefore make liberal use of
$(\nu\; \vec{x})P$.

% subsection the_syntax_and_semantics_of_the_notation_system (end)   

\input{qm2pi.qmops} 

\input{qm2pi.sterngerlach} 

\input{qm2pi.metric} 

% section concurrent_process_calculi (end)

%\input{qm2pi.proofsketch}

% section proof sketch (end)

%\input{qm2pi.slviaknots} 

% section spatial logic via knots (end)

\input{qm2pi.conclusion}

% section conclusion (end)

%\input{qm2pi.dtcodes} 

% section wiring algorithm (end)

\input{qm2pi.ack} 

% section acknowledgments (end)

\newpage


\bibliographystyle{plain}   
\bibliography{../../biblios/main.bib}

\input{qm2pi.rhodetails}

\end{document}

 

% section acknowledgments (end)

\newpage


\bibliographystyle{plain}   
\bibliography{../../biblios/main.bib}

\documentclass[12pt]{llncs}
%\documentclass{jktr}

\usepackage[pdftex]{hyperref}                   
\usepackage {listings}
\usepackage {mathpartir}
\usepackage{bcprules}
%\usepackage{listings}
                       
\usepackage{graphicx} 
%\usepackage[margins=2.5cm,nohead,nofoot]{geometry}
%\usepackage{geometry}
\usepackage{amsfonts}
\usepackage{amstext}
\usepackage{latexsym}
\usepackage{amssymb}
\usepackage{color}


%\include{myPreamble}
\include{qm2pi.local} 

%\ifpdf
%\usepackage[pdftex]{graphicx}
%\else
%\usepackage{graphicx}
%\fi

 % \ifpdf
%  \usepackage{pdfsync}
%  \if


%\title{Brief Article}
%\author{David F. Snyder}
%\author{L.G. Meredith}

%\address{Dept. of Math., Texas State University--San Marcos, San Marcos, TX 78666}
       
\pagestyle{empty}


\begin{document}

\lstset{language=[Objective]Caml,frame=shadowbox}

\input{qm2pi.front}

% section front matter (end)

\input{qm2pi.intro} 
 
% section introduction (end)

% \input{qm2pi.knotations} 

% section notation (end)

\input{qm2pi.process.calculi} 

% section concurrent_process_calculi_and_spatial_logics_ (end)
    
%\input{qm2pi.knots2pi} 

%\input{qm2pi.trefoil} 

%\input{qm2pi.mainthm} 

% subsection basic_interpretation (end)

%\input{qm2pi.rho.presentation} 
\subsection{The syntax and semantics of the notation system}\label{sub:the_syntax_and_semantics_of_the_notation_system} % (fold)

We now summarize a technical presentation of the calculus that
embodies our theory of dynamics. The typical presentation of such a
calculus follows the style of giving generators and relations on
them. The grammar, below, describing term constructors, freely
generates the set of processes, $\Proc$. This set is then quotiented
by a relation known as structural congruence and it is over this set
that the notion of dynamics is expressed. This presentation is
essentially that of \cite{MeredithR05} with the addition of
polyadicity and summation. For readability we have relegated some of
the technical subtleties to an appendix.

\subsubsection{Process grammar}\label{subsub:process_grammar}

\begin{mathpar}
  \inferrule* [lab=synchronization] {} {{M} \bc \pzero \;|\; x?F \;|\; x!C }
  \and
  \inferrule* [lab=abstraction] {} {{F} \bc (x)P}
  \and
  \inferrule* [lab=concretion] {} {{C} \bc \langle Q \rangle}
  \and
  \inferrule* [lab=process] {} {{P,Q} \bc M \;| \;P|Q \;|\; @{x}}
  \and
  \inferrule* [lab=name] {} {{x} \bc \quotep{P}}
\end{mathpar} 

Note that $\vec{x}$ (resp. $\vec{P}$) denotes a vector of names
(resp. processes) of length $|\vec{x}|$ (resp. $|\vec{P}|$). We adopt
the following useful abbreviations.

\begin{mathpar}
   x?(\vec{y}).P := x.(\vec{y})P \and  x\clift{\vec{P}} := x.\clift{\vec{P}}
   \and x!(y) := \lift{x}{\dropn{y}}
   \and \Pi_{i=0}^{n-1}P_i := P_0 | \ldots | P_{n-1}
\end{mathpar}

\subsubsection{Structural congruence}

\paragraph{Free and bound names and alpha-equivalence.} At the
core of structural equivalence is alpha-equivalence which identifies
process that are the same up to a change of variable. Formally, we
recognize the distinction between free and bound names. The free names
of a process, $\freenames{P}$, may be calculated recursively as
follows:

\begin{mathpar}
\freenames{\pzero} := \emptyset
  \and \\
  \freenames{x?(y).P} := \{ x \} \cup (\freenames{P} \setminus \{ y \})
  \and 
  \freenames{x!\langle P \rangle} := \{ x \} \cup \{ P \} 
  \and \\
  \freenames{P|Q} := \freenames{P} \cup \freenames{Q}
  \and \\
  \freenames{@{x}} := \{ x \}
\end{mathpar}

$\pi$
$\quotep{\pi}$

$\freenames{-} : \pi \to \mathcal{P}(\quotep{\pi})$

\begin{eqnarray*}
  \freenames{\pzero} & := & \emptyset \\
  \freenames{x?(y).P} & := & \{ x \} \cup (\freenames{P} \setminus \{ y \}) \\
  \freenames{x!\langle P \rangle} & := & \{ x \} \cup \{ P \} \\
  \freenames{P|Q} & := & \freenames{P} \cup \freenames{Q} \\
  \freenames{\dropn{x}} & := & \{ x \}
\end{eqnarray*}

The bound names of a process, $\boundnames{P}$, are those names occurring in $P$
that are not free. For example, in $x?(y).0$, the name $x$ is free, while $y$ is bound.

\begin{mathpar}
  \inferrule* [lab=monoidal-laws] {} { P|Q \equiv Q|P \and P|0 \equiv P \and P|(Q|R) \equiv (P|Q)|R }
\end{mathpar}

\begin{mathpar}
  \inferrule* [lab=alpha-equivalence] {} { (x)P \equiv (y)P\{y/x\} \and y \not\in \freenames{P} }
\end{mathpar}

\begin{definition}
Then two processes, $P,Q$, are alpha-equivalent if $P = Q\{\vec{y}/\vec{x}\}$ for
some $\vec{x} \in \boundnames{Q},\vec{y} \in \boundnames{P}$, where $Q\{\vec{y}/\vec{x}\}$
denotes the capture-avoiding substitution of $\vec{y}$ for $\vec{x}$ in $Q$.
\end{definition}

\begin{definition}
  The {\em structural congruence} \cite{SangiorgiWalker} , $\equiv$,
  between processes is the least congruence containing
  alpha-equivalence, satisfying the abelian monoid laws
  (associativity, commutativity and $\pzero$ as identity) for parallel
  composition $|$ and for summation $+$.
\end{definition}

\subsection{Name equivalence}

We take name equivalence, written $\nameeq$, to be the smallest
equivalence relation generated by the following rules.

\begin{mathpar}
\inferrule*[lab=Quote-drop]
{ }
{ \quotep{@{x}} \nameeq x }

\inferrule*[lab=Struct-equiv]
{ P \scong Q }
{ \quotep{P} \nameeq \quotep{Q} }
\end{mathpar}

The astute reader will have noticed that the mutual recursion of names
and processes imposes a mutual recursion on alpha-equivalence and
structural equivalence via name-equivalence. Fortunately, all of this
works out pleasantly and we may calculate in the natural way, free of
concern. The reader interested in the details is referred to the
appendix \ref{appendix:rho_details}.

\subsection{Substitution}

We use $\Proc$ for the set of processes, $\QProc$ for the set of
names, and $\id{\{}\vec{y} / \vec{x} \id{\}}$ to denote partial maps,
$s : \QProc \rightarrow \QProc$. A map, $s$ lifts, uniquely, to a map
on process terms, $\widehat{s} : \Proc \rightarrow \Proc$ by the
following equations.

\begin{mathpar}
  (0) \psubstp{Q}{P} := 0 \\
  (R \juxtap S) \psubstp{Q}{P}
  :=    
  (R)\psubstp{Q}{P} \juxtap (S) \psubstp{Q}{P} \\
  (x?(y).R) \psubstp{Q}{P}    
  :=    
  (x)\substp{Q}{P} (z)\concat( (R \psubstn{z}{y}) \psubstp{Q}{P} ) \\
  (\lift{x}{R}) \psubstp{Q}{P}  
  :=
  \lift{(x)\substp{Q}{P}}{ R \psubstp{Q}{P} } \\
%   (\dropn{x})  \psubstp{Q}{P}       
%   := 
%   \left\{ 
%     \begin{array}{ccc} 
%       \dropn{\quotep{Q}} & & x \nameeq \quotep{P} \\
%       \dropn{x} & & otherwise \\
%     \end{array}
%   \right. 
  (\dropn{x})  \psubstp{Q}{P}       
  := 
  \left\{ 
    \begin{array}{ccc} 
      Q & & x \nameeq \quotep{P} \\
      \dropn{x} & & otherwise \\
    \end{array}
  \right.
\end{mathpar}
 

where

\begin{eqnarray}
  (x)\id{\{} \lpquote Q \rpquote / \lpquote P \rpquote \id{\}}            = 
  \left\{ 
    \begin{array}{ccc}
      \lpquote Q \rpquote & & x \nameeq \lpquote P \rpquote \\
      x & & otherwise \\
    \end{array}
  \right. \nonumber
\end{eqnarray}

and $z$ is chosen distinct from $\quotep{P}$, $\quotep{Q}$, the free
names in $Q$, and all the names in $R$. Our $\alpha$-equivalence will
be built in the standard way from this substitution.

\begin{remark}\label{rem:no_self_referential_names}
  One consequence of these definitions is that $\forall P. \quotep{P}
  \not\in \freenames{P}$.
\end{remark}

\subsection{ Dynamic quote: an example }

Anticipating something of what's to come, consider applying the
substitution, $\widehat{\id{\{}u / z \id{\}}}$, to the following pair
of processes, $\lift{w}{y!(z)}$ and $w[ \lpquote y!(z) \rpquote ]$.

\begin{eqnarray}
	\lift{w}{y!(z)}\widehat{\id{\{}u / z \id{\}}}
		& = &
		\lift{w}{y!(u)} \nonumber\\
	w[ \lpquote y!(z) \rpquote ] \widehat{ \id{\{}u / z \id{\}} }
		& = &
		w[ \lpquote y!(z) \rpquote ] \nonumber
\end{eqnarray}

Because the body of the process between quotes is impervious to
substitution, we get radically different answers. In fact, by
examining the first process in an input context,
e.g. $x?(z).\lift{w}{y!(z)}$, we see that the process under the lift
operator may be shaped by prefixed inputs binding a name inside it. In
this sense, the lift operator will be seen as a way to dynamically
construct processes before reifying them as names.

Finally equipped with these standard features we can present the
dynamics of the calculus.

\subsubsection{Operational semantics} 

Finally, we introduce the computational dynamics. What marks these
algebras as distinct from other more traditionally studied algebraic
structures, e.g. vector spaces or polynomial rings, is the manner in
which dynamics is captured. In traditional structures, dynamics is typically
expressed through morphisms between such structures, as in linear maps
between vector spaces or morphisms between rings. In algebras
associated with the semantics of computation, the dynamics is
expressed as part of the algebraic structure itself, through a
reduction reduction relation typically denoted by $\red$. Below, we
give a recursive presentation of this relation for the calculus used
in the encoding.

$\red \subseteq \pi \times \pi$
$\red : \pi \to \mathcal{P}(\pi)$

\begin{mathpar}
  \inferrule* [lab=Comm] { \textsf{match}( x_{src}, x_{trgt} ) } { x_{trgt}?(y)P \; | \; x_{src}!\langle {Q} \rangle \red P\{\quotep{Q}/y}\} }
  \and \\
  \inferrule* [lab=Par] {{P} \red {P}'} {{{P} | {Q}} \red {{P}' | {Q}}}
  \and
  \inferrule* [lab=Equiv]{{{P} \scong {P}'} \andalso {{P}' \red {Q}'} \andalso {{Q}' \scong {Q}}}{{P} \red {Q}}
\end{mathpar}

\begin{eqnarray*}
  match_{\equiv} (\quotep{P},\quotep{Q}) & := & P \equiv Q \\
  match_{\dagger}(\quotep{P},\quotep{Q}) & := & \forall R. P|Q \red^{*} R => R \red^{*} 0 \\
  match_{K}(\quotep{P},\quotep{Q}) & := & K \mbox{ for some context } K
\end{eqnarray*}

$u?(x)P | u!\langle Q \rangle \red P\{\quotep{Q}/x\}$

%We write $\wred$ for $\red^*$, and $P\red$ if $\exists Q $ such that $ P \red Q$.
We write $P\red$ if $\exists Q $ such that $ P \red Q$ and $P\not\red$, otherwise.

\section{Replication}

As mentioned before, it is known that replication (and hence
recursion) can be implemented in a higher-order process algebra
\cite{SangiorgiWalker}. As our first example of calculation with the
machinery thus far presented we give the construction explicitly in
the {\rhoc}.

\begin{eqnarray}
	D_{x} & := & \prefix{x}{y}{(\binpar{\outputp{x}{y}}{@{y}})} \nonumber\\
	\bangp_{x}{P} & := & \binpar{{x}!\langle{\binpar{D_{x}}{P}}\rangle}{D_{x}} \nonumber
\end{eqnarray}

\begin{eqnarray}
	\bangp_{x}{P} & & \nonumber\\
	=
	& {x}!\langle{(\prefix{x}{y}{(\outputp{x}{y} | @{y})) | P}}\rangle 
	      | \prefix{x}{y}{(\outputp{x}{y} | @{y})} & \nonumber\\
	\red
	& (\outputp{x}{y} | @{y})\substn{\quotep{(\prefix{x}{y}{(@{y} | \outputp{x}{y})) | P}}}{y} & \nonumber\\
	=
	& \outputp{x}{\quotep{(\prefix{x}{y}{(\outputp{x}{y} | @{y})) | P}}}
	  | {(\prefix{x}{y}{(\outputp{x}{y} | @{y})) | P}} & \nonumber\\
	\red
	& \ldots & \nonumber\\
	\red^*
	& P | P | \ldots & \nonumber
\end{eqnarray}

Of course, this encoding, as an implementation, runs away, unfolding
$\bangp{P}$ eagerly. A lazier and more implementable replication
operator, restricted to input-guarded processes, may be obtained as follows.

\begin{eqnarray}
\bangp{\prefix{u}{v}{P}} 
	:= 
	\binpar{\lift{x}{\prefix{u}{v}{(\binpar{D(x)}{P})}}}{D(x)} \nonumber
\end{eqnarray}

\begin{remark}
  Note that the lazier definition still does not deal with summation
  or mixed summation (i.e. sums over input and output). The reader is
  invited to construct definitions of replication that deal with these
  features. 

  Further, the definitions are parameterized in a name, $x$. Can you,
  gentle reader, make a definition that eliminates this parameter and
  guarantees no accidental interaction between the replication
  machinery and the process being replicated -- i.e. no accidental
  sharing of names used by the process to get its work done and the
  name(s) used by the replication to effect copying. This latter
  revision of the definition of replication is crucial to obtaining
  the expected identity $!!P \sim !P$.
\end{remark}

\begin{remark}\label{rem:paradoxical_combinator}
  The reader familiar with the lambda calculus will have noticed the
  similarity between $D$ and the paradoxical combinator.

  [Ed. note: the existence of this seems to suggest we have to be more
  restrictive on the set of processes and names we admit if we are to
  support no-cloning.]
\end{remark}

\subsubsection{Bisimulation}

The computational dynamics gives rise to another kind of equivalence,
the equivalence of computational behavior. As previously mentioned
this is typically captured \emph{via} some form of bisimulation.

% The notion we use in this paper is weak barbed bisimulation
% \cite{milner91polyadicpi}.

The notion we use in this paper is derived from weak barbed
bisimulation \cite{milner91polyadicpi}. 

\begin{definition}
An \emph{observation relation}, $\downarrow_{\mathcal N}$, over a set
of names, $\mathcal N$, is the smallest relation satisfying the rules
below.

\infrule[Out-barb]{y \in {\mathcal N}, \; x \nameeq y}
		  {\outputp{x}{v} \downarrow_{\mathcal N} x}
\infrule[Par-barb]{\mbox{$P\downarrow_{\mathcal N} x$ or $Q\downarrow_{\mathcal N} x$}}
		  {\binpar{P}{Q} \downarrow_{\mathcal N} x}

We write $P \Downarrow_{\mathcal N} x$ if there is $Q$ such that 
$P \wred Q$ and $Q \downarrow_{\mathcal N} x$.
\end{definition}

\begin{definition}
%\label{def.bbisim}
An  ${\mathcal N}$-\emph{barbed bisimulation} over a set of names, ${\mathcal N}$, is a symmetric binary relation 
${\mathcal S}_{\mathcal N}$ between agents such that $P\rel{S}_{\mathcal N}Q$ implies:
\begin{enumerate}
\item If $P \red P'$ then $Q \wred Q'$ and $P'\rel{S}_{\mathcal N} Q'$.
\item If $P\downarrow_{\mathcal N} x$, then $Q\Downarrow_{\mathcal N} x$.
\end{enumerate}
$P$ is ${\mathcal N}$-barbed bisimilar to $Q$, written
$P \wbbisim_{\mathcal N} Q$, if $P \rel{S}_{\mathcal N} Q$ for some ${\mathcal N}$-barbed bisimulation ${\mathcal S}_{\mathcal N}$.
\end{definition}

$\mathcal{R} \subseteq \pi \times \pi$

$P \mathcal{R} Q => \forall P'. P \red P' \Rightarrow \exists Q'. Q \red Q', P' \mathcal{R} Q'$

$P \vdash x \Rightarrow Q \vdash x$

\begin{mathpar}
  \inferrule*[lab=Out-barb]{x \nameeq y}{{y}!\langle{Q}\rangle \vdash x}
  \and
  \inferrule*[lab=Par-barb]{\mbox{$P\vdash x$ or $Q\vdash x$}}{\binpar{P}{Q} \vdash x}
\end{mathpar}

\subsubsection{Contexts}

One of the principle advantages of computational calculi like the
$\pi$-calculus is a well-defined notion of context,
contextual-equivalence and a correlation between
contextual-equivalence and notions of bisimulation. The notion of
context allows the decomposition of a process into (sub-)process and
its syntactic environment, its context. Thus, a context may be
thought of as a process with a ``hole'' (written $\Box$) in it. The
application of a context $M$ to a process $P$, written $M[P]$, is
tantamount to filling the hole in $M$ with $P$. In this paper we do
not need the full weight of this theory, but do make use of the notion
of context in the proof the main theorem. 

\begin{mathpar}
  \inferrule* [lab=summation] {} {{M_{M},M_{N}} \bc \Box \;|\; x.M_{A} \;|\; M_{M}+M_{N}}
  \and
  \inferrule* [lab=agent] {} {{M_{A}} \bc (\vec{x})M_{P} \;| \; \clift{P_0,\ldots,M_{P},\ldots,P_N}}
  \and \\
  \inferrule* [lab=process] {} {{M_{P}} \bc M_{N} \;| \;P|M_{P} }
\end{mathpar} 

\begin{mathpar}
  \inferrule* [lab=sychronization] {} {M_{N} \bc \Box \;|\; x?M_{F} \;|\; x!M_{C}}
  \and
  \inferrule* [lab=abstraction] {} {{M_{F}} \bc (x)M_{P} }
  \and
  \inferrule* [lab=concretion] {} {{M_{C}} \bc \langle M_{P} \rangle }
  \and \\
  \inferrule* [lab=process] {} {{M_{P}} \bc M_{N} \;| \;P|M_{P} }
\end{mathpar}

\begin{definition}[contextual application] Given a context $M$, and
  process $P$, we define the \emph{contextual application}, $M[P] :=
  M\{P/\Box\}$. That is, the contextual application of M to P is the
  substitution of $P$ for $\Box$ in $M$.
\end{definition}

$\meaningof{-} : L \to \mathcal{P}(\pi)$

\begin{mathpar}
  \inferrule* [lab=collection] {} {\meaningof{true} = \pi, \and \meaningof{~E} = \pi \setminus \meaningof{E}, \and \meaningof{E_{1} \& E_{2}} = \meaningof{E_{1}} \cap \meaningof{E_{2}}}
\end{mathpar}

\begin{mathpar}
  \inferrule* [lab=structure] {} {\meaningof{0} = \{ P \in \pi | P \equiv 0 \}, \and \\ \meaningof{E_1 | E_2} = \{ P \in \pi | P \equiv P_{1} | P_{2}, P_{1} \in \meaningof{E_{1}}, P_{2} \in \meaningof{E_2}\} }
\end{mathpar}

\begin{mathpar}
 \inferrule* [lab=behavior] {} {\meaningof{\langle a?b \rangle E} = \{ P \in \pi | P \equiv Q | u?(y)P', \\ \and \\\\ \and \\ \;\;\; u \in \meaningof{a}, \forall z.P'\{z/y\} \in \meaningof{E\{z/b\}}\}, \and \\ \meaningof{a!E} = \{ P \in \pi | P \equiv Q | x!\langle P' \rangle, x \in \meaningof{a} P' \in \meaningof{E}\} }
\end{mathpar}

\begin{mathpar}
 \inferrule* [lab=nominal] {} {\meaningof{\quotep{E}} = \{ \quotep{P} \in \quotep{\pi} | P \in \meaningof{E} \}, \and \meaningof{\quotep{P}} = \{ \quotep{Q} \in \quotep{\pi} | P \equiv Q \} \and \\ \meaningof{@\quotep{E}} = \{ P \in \pi | P \equiv @x, x \in \meaningof{E} \}}
\end{mathpar}

\begin{eqnarray*}
  \\
  \meaningof{-} : TS \to ST
\end{eqnarray*}

\begin{eqnarray*}
  \\
  L : TS \to ST
\end{eqnarray*}

\begin{eqnarray*}
  \\
  P \models E \iff P \in \meaningof{E}
\end{eqnarray*}

\begin{eqnarray*}
  P \approx_{L} Q \iff \forall E \in L. P \models E \iff Q \models E
\end{eqnarray*}

\begin{eqnarray*}
  P \approx_{K} Q
\end{eqnarray*}

\begin{eqnarray*}
  P \approx Q
\end{eqnarray*}

$\approx_{K} = \approx = \approx_{L}$

\subsubsection{Contextual duality}

Note that contexts extend the quotation operation to a family of
operations from processes to names. Given a context, $M$, we can
define a \emph{nominal context}, $\quotep{M}$ by $\quotep{M}[P] :=
\quotep{M[P]}$. To foreshadow what is to come we observe that these
operations enjoy a duality with processes very much like the duality
between vectors and maps from vectors to scalars.

Further, because the calculus is essentially higher-order, we have a
correspondence between contexts and processes. More specifically,
given a name $x$ and a context $M$ we can construct $M^{*}_{x}$ such
that 

\begin{mathpar}
  M^{*}_{x} | \lift{x}{P} \red M[P]
\end{mathpar}

namely,

\begin{mathpar}
  M^{*}_{x} := x?(u).M[\dropn{u}]
\end{mathpar}

The dependence of $M^{*}_{x}$ on a name makes it an abstraction, 

\begin{mathpar}
  M^{*} := (x)x?(u).M[\dropn{u}]
\end{mathpar}

\subsection{Additional notation}

It will sometimes be convenient to denote the process a name
quotes. We already have the notation $x = \quotep{P}$, but it will be
convenient to introduce an alternate notation, $\procn{x}$, when we
want to emphasize the connection to the use of the name. Note that, by
virtue of name equivalence, $\quotep{\procn{x}} \nameeq x$; so, the
notation is consistent with previous definitions.

Further, because names have structure it is possible to effect
substitutions on the basis of that structure. This means we need to
upgrade our notation for substitutions, which we accomplish by
adapting comprehension notation. Thus,

\begin{mathpar}
  P\{ y / x : x \in S \}
\end{mathpar}

is interpreted to mean the process derived from P by replacing (in a
capture-avoiding manner) each occurrence of $x$ in $S$ by $y$. For example,

\begin{mathpar}
  P\{ \quotep{\procn{x}|\procn{x}} / x : x \in \freenames{P} \}
\end{mathpar}

will replace each (occurrence) of a free name $x$ in $P$ by
$\quotep{\procn{x}|\procn{x}}$.

Also, we will avail ourselves of the notation $x^{L}$ and $x^{R}$ to
denote injections of a name into disjoint copies of the name
space. There are numerous ways to accomplish this. One example can be
found in \cite{MeredithR05}. This notation overloads to vectors of
names: $\vec{x}^{\pi} := (x_{i}^{\pi} \; : \; 0 \leq i < |\vec{x}| )$ where $\pi \in \{L,R\}$.

We also use $P^{\Box} := P|\Box$.

In \cite{MeredithR05} an interpretation of the new operator is
given. It turns out that there are several possible interpretations
all enjoying the requisite algebraic properties of the operator (see
\cite{milner91polyadicpi}). We will therefore make liberal use of
$(\nu\; \vec{x})P$.

% subsection the_syntax_and_semantics_of_the_notation_system (end)   

\input{qm2pi.qmops} 

\input{qm2pi.sterngerlach} 

\input{qm2pi.metric} 

% section concurrent_process_calculi (end)

%\input{qm2pi.proofsketch}

% section proof sketch (end)

%\input{qm2pi.slviaknots} 

% section spatial logic via knots (end)

\input{qm2pi.conclusion}

% section conclusion (end)

%\input{qm2pi.dtcodes} 

% section wiring algorithm (end)

\input{qm2pi.ack} 

% section acknowledgments (end)

\newpage


\bibliographystyle{plain}   
\bibliography{../../biblios/main.bib}

\input{qm2pi.rhodetails}

\end{document}



\end{document}

 

\documentclass[12pt]{llncs}
%\documentclass{jktr}

\usepackage[pdftex]{hyperref}                   
\usepackage {listings}
\usepackage {mathpartir}
\usepackage{bcprules}
%\usepackage{listings}
                       
\usepackage{graphicx} 
%\usepackage[margins=2.5cm,nohead,nofoot]{geometry}
%\usepackage{geometry}
\usepackage{amsfonts}
\usepackage{amstext}
\usepackage{latexsym}
\usepackage{amssymb}
\usepackage{color}


%\include{myPreamble}
\documentclass[12pt]{llncs}
%\documentclass{jktr}

\usepackage[pdftex]{hyperref}                   
\usepackage {listings}
\usepackage {mathpartir}
\usepackage{bcprules}
%\usepackage{listings}
                       
\usepackage{graphicx} 
%\usepackage[margins=2.5cm,nohead,nofoot]{geometry}
%\usepackage{geometry}
\usepackage{amsfonts}
\usepackage{amstext}
\usepackage{latexsym}
\usepackage{amssymb}
\usepackage{color}


%\include{myPreamble}
\include{qm2pi.local} 

%\ifpdf
%\usepackage[pdftex]{graphicx}
%\else
%\usepackage{graphicx}
%\fi

 % \ifpdf
%  \usepackage{pdfsync}
%  \if


%\title{Brief Article}
%\author{David F. Snyder}
%\author{L.G. Meredith}

%\address{Dept. of Math., Texas State University--San Marcos, San Marcos, TX 78666}
       
\pagestyle{empty}


\begin{document}

\lstset{language=[Objective]Caml,frame=shadowbox}

\input{qm2pi.front}

% section front matter (end)

\input{qm2pi.intro} 
 
% section introduction (end)

% \input{qm2pi.knotations} 

% section notation (end)

\input{qm2pi.process.calculi} 

% section concurrent_process_calculi_and_spatial_logics_ (end)
    
%\input{qm2pi.knots2pi} 

%\input{qm2pi.trefoil} 

%\input{qm2pi.mainthm} 

% subsection basic_interpretation (end)

%\input{qm2pi.rho.presentation} 
\subsection{The syntax and semantics of the notation system}\label{sub:the_syntax_and_semantics_of_the_notation_system} % (fold)

We now summarize a technical presentation of the calculus that
embodies our theory of dynamics. The typical presentation of such a
calculus follows the style of giving generators and relations on
them. The grammar, below, describing term constructors, freely
generates the set of processes, $\Proc$. This set is then quotiented
by a relation known as structural congruence and it is over this set
that the notion of dynamics is expressed. This presentation is
essentially that of \cite{MeredithR05} with the addition of
polyadicity and summation. For readability we have relegated some of
the technical subtleties to an appendix.

\subsubsection{Process grammar}\label{subsub:process_grammar}

\begin{mathpar}
  \inferrule* [lab=synchronization] {} {{M} \bc \pzero \;|\; x?F \;|\; x!C }
  \and
  \inferrule* [lab=abstraction] {} {{F} \bc (x)P}
  \and
  \inferrule* [lab=concretion] {} {{C} \bc \langle Q \rangle}
  \and
  \inferrule* [lab=process] {} {{P,Q} \bc M \;| \;P|Q \;|\; @{x}}
  \and
  \inferrule* [lab=name] {} {{x} \bc \quotep{P}}
\end{mathpar} 

Note that $\vec{x}$ (resp. $\vec{P}$) denotes a vector of names
(resp. processes) of length $|\vec{x}|$ (resp. $|\vec{P}|$). We adopt
the following useful abbreviations.

\begin{mathpar}
   x?(\vec{y}).P := x.(\vec{y})P \and  x\clift{\vec{P}} := x.\clift{\vec{P}}
   \and x!(y) := \lift{x}{\dropn{y}}
   \and \Pi_{i=0}^{n-1}P_i := P_0 | \ldots | P_{n-1}
\end{mathpar}

\subsubsection{Structural congruence}

\paragraph{Free and bound names and alpha-equivalence.} At the
core of structural equivalence is alpha-equivalence which identifies
process that are the same up to a change of variable. Formally, we
recognize the distinction between free and bound names. The free names
of a process, $\freenames{P}$, may be calculated recursively as
follows:

\begin{mathpar}
\freenames{\pzero} := \emptyset
  \and \\
  \freenames{x?(y).P} := \{ x \} \cup (\freenames{P} \setminus \{ y \})
  \and 
  \freenames{x!\langle P \rangle} := \{ x \} \cup \{ P \} 
  \and \\
  \freenames{P|Q} := \freenames{P} \cup \freenames{Q}
  \and \\
  \freenames{@{x}} := \{ x \}
\end{mathpar}

$\pi$
$\quotep{\pi}$

$\freenames{-} : \pi \to \mathcal{P}(\quotep{\pi})$

\begin{eqnarray*}
  \freenames{\pzero} & := & \emptyset \\
  \freenames{x?(y).P} & := & \{ x \} \cup (\freenames{P} \setminus \{ y \}) \\
  \freenames{x!\langle P \rangle} & := & \{ x \} \cup \{ P \} \\
  \freenames{P|Q} & := & \freenames{P} \cup \freenames{Q} \\
  \freenames{\dropn{x}} & := & \{ x \}
\end{eqnarray*}

The bound names of a process, $\boundnames{P}$, are those names occurring in $P$
that are not free. For example, in $x?(y).0$, the name $x$ is free, while $y$ is bound.

\begin{mathpar}
  \inferrule* [lab=monoidal-laws] {} { P|Q \equiv Q|P \and P|0 \equiv P \and P|(Q|R) \equiv (P|Q)|R }
\end{mathpar}

\begin{mathpar}
  \inferrule* [lab=alpha-equivalence] {} { (x)P \equiv (y)P\{y/x\} \and y \not\in \freenames{P} }
\end{mathpar}

\begin{definition}
Then two processes, $P,Q$, are alpha-equivalent if $P = Q\{\vec{y}/\vec{x}\}$ for
some $\vec{x} \in \boundnames{Q},\vec{y} \in \boundnames{P}$, where $Q\{\vec{y}/\vec{x}\}$
denotes the capture-avoiding substitution of $\vec{y}$ for $\vec{x}$ in $Q$.
\end{definition}

\begin{definition}
  The {\em structural congruence} \cite{SangiorgiWalker} , $\equiv$,
  between processes is the least congruence containing
  alpha-equivalence, satisfying the abelian monoid laws
  (associativity, commutativity and $\pzero$ as identity) for parallel
  composition $|$ and for summation $+$.
\end{definition}

\subsection{Name equivalence}

We take name equivalence, written $\nameeq$, to be the smallest
equivalence relation generated by the following rules.

\begin{mathpar}
\inferrule*[lab=Quote-drop]
{ }
{ \quotep{@{x}} \nameeq x }

\inferrule*[lab=Struct-equiv]
{ P \scong Q }
{ \quotep{P} \nameeq \quotep{Q} }
\end{mathpar}

The astute reader will have noticed that the mutual recursion of names
and processes imposes a mutual recursion on alpha-equivalence and
structural equivalence via name-equivalence. Fortunately, all of this
works out pleasantly and we may calculate in the natural way, free of
concern. The reader interested in the details is referred to the
appendix \ref{appendix:rho_details}.

\subsection{Substitution}

We use $\Proc$ for the set of processes, $\QProc$ for the set of
names, and $\id{\{}\vec{y} / \vec{x} \id{\}}$ to denote partial maps,
$s : \QProc \rightarrow \QProc$. A map, $s$ lifts, uniquely, to a map
on process terms, $\widehat{s} : \Proc \rightarrow \Proc$ by the
following equations.

\begin{mathpar}
  (0) \psubstp{Q}{P} := 0 \\
  (R \juxtap S) \psubstp{Q}{P}
  :=    
  (R)\psubstp{Q}{P} \juxtap (S) \psubstp{Q}{P} \\
  (x?(y).R) \psubstp{Q}{P}    
  :=    
  (x)\substp{Q}{P} (z)\concat( (R \psubstn{z}{y}) \psubstp{Q}{P} ) \\
  (\lift{x}{R}) \psubstp{Q}{P}  
  :=
  \lift{(x)\substp{Q}{P}}{ R \psubstp{Q}{P} } \\
%   (\dropn{x})  \psubstp{Q}{P}       
%   := 
%   \left\{ 
%     \begin{array}{ccc} 
%       \dropn{\quotep{Q}} & & x \nameeq \quotep{P} \\
%       \dropn{x} & & otherwise \\
%     \end{array}
%   \right. 
  (\dropn{x})  \psubstp{Q}{P}       
  := 
  \left\{ 
    \begin{array}{ccc} 
      Q & & x \nameeq \quotep{P} \\
      \dropn{x} & & otherwise \\
    \end{array}
  \right.
\end{mathpar}
 

where

\begin{eqnarray}
  (x)\id{\{} \lpquote Q \rpquote / \lpquote P \rpquote \id{\}}            = 
  \left\{ 
    \begin{array}{ccc}
      \lpquote Q \rpquote & & x \nameeq \lpquote P \rpquote \\
      x & & otherwise \\
    \end{array}
  \right. \nonumber
\end{eqnarray}

and $z$ is chosen distinct from $\quotep{P}$, $\quotep{Q}$, the free
names in $Q$, and all the names in $R$. Our $\alpha$-equivalence will
be built in the standard way from this substitution.

\begin{remark}\label{rem:no_self_referential_names}
  One consequence of these definitions is that $\forall P. \quotep{P}
  \not\in \freenames{P}$.
\end{remark}

\subsection{ Dynamic quote: an example }

Anticipating something of what's to come, consider applying the
substitution, $\widehat{\id{\{}u / z \id{\}}}$, to the following pair
of processes, $\lift{w}{y!(z)}$ and $w[ \lpquote y!(z) \rpquote ]$.

\begin{eqnarray}
	\lift{w}{y!(z)}\widehat{\id{\{}u / z \id{\}}}
		& = &
		\lift{w}{y!(u)} \nonumber\\
	w[ \lpquote y!(z) \rpquote ] \widehat{ \id{\{}u / z \id{\}} }
		& = &
		w[ \lpquote y!(z) \rpquote ] \nonumber
\end{eqnarray}

Because the body of the process between quotes is impervious to
substitution, we get radically different answers. In fact, by
examining the first process in an input context,
e.g. $x?(z).\lift{w}{y!(z)}$, we see that the process under the lift
operator may be shaped by prefixed inputs binding a name inside it. In
this sense, the lift operator will be seen as a way to dynamically
construct processes before reifying them as names.

Finally equipped with these standard features we can present the
dynamics of the calculus.

\subsubsection{Operational semantics} 

Finally, we introduce the computational dynamics. What marks these
algebras as distinct from other more traditionally studied algebraic
structures, e.g. vector spaces or polynomial rings, is the manner in
which dynamics is captured. In traditional structures, dynamics is typically
expressed through morphisms between such structures, as in linear maps
between vector spaces or morphisms between rings. In algebras
associated with the semantics of computation, the dynamics is
expressed as part of the algebraic structure itself, through a
reduction reduction relation typically denoted by $\red$. Below, we
give a recursive presentation of this relation for the calculus used
in the encoding.

$\red \subseteq \pi \times \pi$
$\red : \pi \to \mathcal{P}(\pi)$

\begin{mathpar}
  \inferrule* [lab=Comm] { \textsf{match}( x_{src}, x_{trgt} ) } { x_{trgt}?(y)P \; | \; x_{src}!\langle {Q} \rangle \red P\{\quotep{Q}/y}\} }
  \and \\
  \inferrule* [lab=Par] {{P} \red {P}'} {{{P} | {Q}} \red {{P}' | {Q}}}
  \and
  \inferrule* [lab=Equiv]{{{P} \scong {P}'} \andalso {{P}' \red {Q}'} \andalso {{Q}' \scong {Q}}}{{P} \red {Q}}
\end{mathpar}

\begin{eqnarray*}
  match_{\equiv} (\quotep{P},\quotep{Q}) & := & P \equiv Q \\
  match_{\dagger}(\quotep{P},\quotep{Q}) & := & \forall R. P|Q \red^{*} R => R \red^{*} 0 \\
  match_{K}(\quotep{P},\quotep{Q}) & := & K \mbox{ for some context } K
\end{eqnarray*}

$u?(x)P | u!\langle Q \rangle \red P\{\quotep{Q}/x\}$

%We write $\wred$ for $\red^*$, and $P\red$ if $\exists Q $ such that $ P \red Q$.
We write $P\red$ if $\exists Q $ such that $ P \red Q$ and $P\not\red$, otherwise.

\section{Replication}

As mentioned before, it is known that replication (and hence
recursion) can be implemented in a higher-order process algebra
\cite{SangiorgiWalker}. As our first example of calculation with the
machinery thus far presented we give the construction explicitly in
the {\rhoc}.

\begin{eqnarray}
	D_{x} & := & \prefix{x}{y}{(\binpar{\outputp{x}{y}}{@{y}})} \nonumber\\
	\bangp_{x}{P} & := & \binpar{{x}!\langle{\binpar{D_{x}}{P}}\rangle}{D_{x}} \nonumber
\end{eqnarray}

\begin{eqnarray}
	\bangp_{x}{P} & & \nonumber\\
	=
	& {x}!\langle{(\prefix{x}{y}{(\outputp{x}{y} | @{y})) | P}}\rangle 
	      | \prefix{x}{y}{(\outputp{x}{y} | @{y})} & \nonumber\\
	\red
	& (\outputp{x}{y} | @{y})\substn{\quotep{(\prefix{x}{y}{(@{y} | \outputp{x}{y})) | P}}}{y} & \nonumber\\
	=
	& \outputp{x}{\quotep{(\prefix{x}{y}{(\outputp{x}{y} | @{y})) | P}}}
	  | {(\prefix{x}{y}{(\outputp{x}{y} | @{y})) | P}} & \nonumber\\
	\red
	& \ldots & \nonumber\\
	\red^*
	& P | P | \ldots & \nonumber
\end{eqnarray}

Of course, this encoding, as an implementation, runs away, unfolding
$\bangp{P}$ eagerly. A lazier and more implementable replication
operator, restricted to input-guarded processes, may be obtained as follows.

\begin{eqnarray}
\bangp{\prefix{u}{v}{P}} 
	:= 
	\binpar{\lift{x}{\prefix{u}{v}{(\binpar{D(x)}{P})}}}{D(x)} \nonumber
\end{eqnarray}

\begin{remark}
  Note that the lazier definition still does not deal with summation
  or mixed summation (i.e. sums over input and output). The reader is
  invited to construct definitions of replication that deal with these
  features. 

  Further, the definitions are parameterized in a name, $x$. Can you,
  gentle reader, make a definition that eliminates this parameter and
  guarantees no accidental interaction between the replication
  machinery and the process being replicated -- i.e. no accidental
  sharing of names used by the process to get its work done and the
  name(s) used by the replication to effect copying. This latter
  revision of the definition of replication is crucial to obtaining
  the expected identity $!!P \sim !P$.
\end{remark}

\begin{remark}\label{rem:paradoxical_combinator}
  The reader familiar with the lambda calculus will have noticed the
  similarity between $D$ and the paradoxical combinator.

  [Ed. note: the existence of this seems to suggest we have to be more
  restrictive on the set of processes and names we admit if we are to
  support no-cloning.]
\end{remark}

\subsubsection{Bisimulation}

The computational dynamics gives rise to another kind of equivalence,
the equivalence of computational behavior. As previously mentioned
this is typically captured \emph{via} some form of bisimulation.

% The notion we use in this paper is weak barbed bisimulation
% \cite{milner91polyadicpi}.

The notion we use in this paper is derived from weak barbed
bisimulation \cite{milner91polyadicpi}. 

\begin{definition}
An \emph{observation relation}, $\downarrow_{\mathcal N}$, over a set
of names, $\mathcal N$, is the smallest relation satisfying the rules
below.

\infrule[Out-barb]{y \in {\mathcal N}, \; x \nameeq y}
		  {\outputp{x}{v} \downarrow_{\mathcal N} x}
\infrule[Par-barb]{\mbox{$P\downarrow_{\mathcal N} x$ or $Q\downarrow_{\mathcal N} x$}}
		  {\binpar{P}{Q} \downarrow_{\mathcal N} x}

We write $P \Downarrow_{\mathcal N} x$ if there is $Q$ such that 
$P \wred Q$ and $Q \downarrow_{\mathcal N} x$.
\end{definition}

\begin{definition}
%\label{def.bbisim}
An  ${\mathcal N}$-\emph{barbed bisimulation} over a set of names, ${\mathcal N}$, is a symmetric binary relation 
${\mathcal S}_{\mathcal N}$ between agents such that $P\rel{S}_{\mathcal N}Q$ implies:
\begin{enumerate}
\item If $P \red P'$ then $Q \wred Q'$ and $P'\rel{S}_{\mathcal N} Q'$.
\item If $P\downarrow_{\mathcal N} x$, then $Q\Downarrow_{\mathcal N} x$.
\end{enumerate}
$P$ is ${\mathcal N}$-barbed bisimilar to $Q$, written
$P \wbbisim_{\mathcal N} Q$, if $P \rel{S}_{\mathcal N} Q$ for some ${\mathcal N}$-barbed bisimulation ${\mathcal S}_{\mathcal N}$.
\end{definition}

$\mathcal{R} \subseteq \pi \times \pi$

$P \mathcal{R} Q => \forall P'. P \red P' \Rightarrow \exists Q'. Q \red Q', P' \mathcal{R} Q'$

$P \vdash x \Rightarrow Q \vdash x$

\begin{mathpar}
  \inferrule*[lab=Out-barb]{x \nameeq y}{{y}!\langle{Q}\rangle \vdash x}
  \and
  \inferrule*[lab=Par-barb]{\mbox{$P\vdash x$ or $Q\vdash x$}}{\binpar{P}{Q} \vdash x}
\end{mathpar}

\subsubsection{Contexts}

One of the principle advantages of computational calculi like the
$\pi$-calculus is a well-defined notion of context,
contextual-equivalence and a correlation between
contextual-equivalence and notions of bisimulation. The notion of
context allows the decomposition of a process into (sub-)process and
its syntactic environment, its context. Thus, a context may be
thought of as a process with a ``hole'' (written $\Box$) in it. The
application of a context $M$ to a process $P$, written $M[P]$, is
tantamount to filling the hole in $M$ with $P$. In this paper we do
not need the full weight of this theory, but do make use of the notion
of context in the proof the main theorem. 

\begin{mathpar}
  \inferrule* [lab=summation] {} {{M_{M},M_{N}} \bc \Box \;|\; x.M_{A} \;|\; M_{M}+M_{N}}
  \and
  \inferrule* [lab=agent] {} {{M_{A}} \bc (\vec{x})M_{P} \;| \; \clift{P_0,\ldots,M_{P},\ldots,P_N}}
  \and \\
  \inferrule* [lab=process] {} {{M_{P}} \bc M_{N} \;| \;P|M_{P} }
\end{mathpar} 

\begin{mathpar}
  \inferrule* [lab=sychronization] {} {M_{N} \bc \Box \;|\; x?M_{F} \;|\; x!M_{C}}
  \and
  \inferrule* [lab=abstraction] {} {{M_{F}} \bc (x)M_{P} }
  \and
  \inferrule* [lab=concretion] {} {{M_{C}} \bc \langle M_{P} \rangle }
  \and \\
  \inferrule* [lab=process] {} {{M_{P}} \bc M_{N} \;| \;P|M_{P} }
\end{mathpar}

\begin{definition}[contextual application] Given a context $M$, and
  process $P$, we define the \emph{contextual application}, $M[P] :=
  M\{P/\Box\}$. That is, the contextual application of M to P is the
  substitution of $P$ for $\Box$ in $M$.
\end{definition}

$\meaningof{-} : L \to \mathcal{P}(\pi)$

\begin{mathpar}
  \inferrule* [lab=collection] {} {\meaningof{true} = \pi, \and \meaningof{~E} = \pi \setminus \meaningof{E}, \and \meaningof{E_{1} \& E_{2}} = \meaningof{E_{1}} \cap \meaningof{E_{2}}}
\end{mathpar}

\begin{mathpar}
  \inferrule* [lab=structure] {} {\meaningof{0} = \{ P \in \pi | P \equiv 0 \}, \and \\ \meaningof{E_1 | E_2} = \{ P \in \pi | P \equiv P_{1} | P_{2}, P_{1} \in \meaningof{E_{1}}, P_{2} \in \meaningof{E_2}\} }
\end{mathpar}

\begin{mathpar}
 \inferrule* [lab=behavior] {} {\meaningof{\langle a?b \rangle E} = \{ P \in \pi | P \equiv Q | u?(y)P', \\ \and \\\\ \and \\ \;\;\; u \in \meaningof{a}, \forall z.P'\{z/y\} \in \meaningof{E\{z/b\}}\}, \and \\ \meaningof{a!E} = \{ P \in \pi | P \equiv Q | x!\langle P' \rangle, x \in \meaningof{a} P' \in \meaningof{E}\} }
\end{mathpar}

\begin{mathpar}
 \inferrule* [lab=nominal] {} {\meaningof{\quotep{E}} = \{ \quotep{P} \in \quotep{\pi} | P \in \meaningof{E} \}, \and \meaningof{\quotep{P}} = \{ \quotep{Q} \in \quotep{\pi} | P \equiv Q \} \and \\ \meaningof{@\quotep{E}} = \{ P \in \pi | P \equiv @x, x \in \meaningof{E} \}}
\end{mathpar}

\begin{eqnarray*}
  \\
  \meaningof{-} : TS \to ST
\end{eqnarray*}

\begin{eqnarray*}
  \\
  L : TS \to ST
\end{eqnarray*}

\begin{eqnarray*}
  \\
  P \models E \iff P \in \meaningof{E}
\end{eqnarray*}

\begin{eqnarray*}
  P \approx_{L} Q \iff \forall E \in L. P \models E \iff Q \models E
\end{eqnarray*}

\begin{eqnarray*}
  P \approx_{K} Q
\end{eqnarray*}

\begin{eqnarray*}
  P \approx Q
\end{eqnarray*}

$\approx_{K} = \approx = \approx_{L}$

\subsubsection{Contextual duality}

Note that contexts extend the quotation operation to a family of
operations from processes to names. Given a context, $M$, we can
define a \emph{nominal context}, $\quotep{M}$ by $\quotep{M}[P] :=
\quotep{M[P]}$. To foreshadow what is to come we observe that these
operations enjoy a duality with processes very much like the duality
between vectors and maps from vectors to scalars.

Further, because the calculus is essentially higher-order, we have a
correspondence between contexts and processes. More specifically,
given a name $x$ and a context $M$ we can construct $M^{*}_{x}$ such
that 

\begin{mathpar}
  M^{*}_{x} | \lift{x}{P} \red M[P]
\end{mathpar}

namely,

\begin{mathpar}
  M^{*}_{x} := x?(u).M[\dropn{u}]
\end{mathpar}

The dependence of $M^{*}_{x}$ on a name makes it an abstraction, 

\begin{mathpar}
  M^{*} := (x)x?(u).M[\dropn{u}]
\end{mathpar}

\subsection{Additional notation}

It will sometimes be convenient to denote the process a name
quotes. We already have the notation $x = \quotep{P}$, but it will be
convenient to introduce an alternate notation, $\procn{x}$, when we
want to emphasize the connection to the use of the name. Note that, by
virtue of name equivalence, $\quotep{\procn{x}} \nameeq x$; so, the
notation is consistent with previous definitions.

Further, because names have structure it is possible to effect
substitutions on the basis of that structure. This means we need to
upgrade our notation for substitutions, which we accomplish by
adapting comprehension notation. Thus,

\begin{mathpar}
  P\{ y / x : x \in S \}
\end{mathpar}

is interpreted to mean the process derived from P by replacing (in a
capture-avoiding manner) each occurrence of $x$ in $S$ by $y$. For example,

\begin{mathpar}
  P\{ \quotep{\procn{x}|\procn{x}} / x : x \in \freenames{P} \}
\end{mathpar}

will replace each (occurrence) of a free name $x$ in $P$ by
$\quotep{\procn{x}|\procn{x}}$.

Also, we will avail ourselves of the notation $x^{L}$ and $x^{R}$ to
denote injections of a name into disjoint copies of the name
space. There are numerous ways to accomplish this. One example can be
found in \cite{MeredithR05}. This notation overloads to vectors of
names: $\vec{x}^{\pi} := (x_{i}^{\pi} \; : \; 0 \leq i < |\vec{x}| )$ where $\pi \in \{L,R\}$.

We also use $P^{\Box} := P|\Box$.

In \cite{MeredithR05} an interpretation of the new operator is
given. It turns out that there are several possible interpretations
all enjoying the requisite algebraic properties of the operator (see
\cite{milner91polyadicpi}). We will therefore make liberal use of
$(\nu\; \vec{x})P$.

% subsection the_syntax_and_semantics_of_the_notation_system (end)   

\input{qm2pi.qmops} 

\input{qm2pi.sterngerlach} 

\input{qm2pi.metric} 

% section concurrent_process_calculi (end)

%\input{qm2pi.proofsketch}

% section proof sketch (end)

%\input{qm2pi.slviaknots} 

% section spatial logic via knots (end)

\input{qm2pi.conclusion}

% section conclusion (end)

%\input{qm2pi.dtcodes} 

% section wiring algorithm (end)

\input{qm2pi.ack} 

% section acknowledgments (end)

\newpage


\bibliographystyle{plain}   
\bibliography{../../biblios/main.bib}

\input{qm2pi.rhodetails}

\end{document}

 

%\ifpdf
%\usepackage[pdftex]{graphicx}
%\else
%\usepackage{graphicx}
%\fi

 % \ifpdf
%  \usepackage{pdfsync}
%  \if


%\title{Brief Article}
%\author{David F. Snyder}
%\author{L.G. Meredith}

%\address{Dept. of Math., Texas State University--San Marcos, San Marcos, TX 78666}
       
\pagestyle{empty}


\begin{document}

\lstset{language=[Objective]Caml,frame=shadowbox}

\documentclass[12pt]{llncs}
%\documentclass{jktr}

\usepackage[pdftex]{hyperref}                   
\usepackage {listings}
\usepackage {mathpartir}
\usepackage{bcprules}
%\usepackage{listings}
                       
\usepackage{graphicx} 
%\usepackage[margins=2.5cm,nohead,nofoot]{geometry}
%\usepackage{geometry}
\usepackage{amsfonts}
\usepackage{amstext}
\usepackage{latexsym}
\usepackage{amssymb}
\usepackage{color}


%\include{myPreamble}
\include{qm2pi.local} 

%\ifpdf
%\usepackage[pdftex]{graphicx}
%\else
%\usepackage{graphicx}
%\fi

 % \ifpdf
%  \usepackage{pdfsync}
%  \if


%\title{Brief Article}
%\author{David F. Snyder}
%\author{L.G. Meredith}

%\address{Dept. of Math., Texas State University--San Marcos, San Marcos, TX 78666}
       
\pagestyle{empty}


\begin{document}

\lstset{language=[Objective]Caml,frame=shadowbox}

\input{qm2pi.front}

% section front matter (end)

\input{qm2pi.intro} 
 
% section introduction (end)

% \input{qm2pi.knotations} 

% section notation (end)

\input{qm2pi.process.calculi} 

% section concurrent_process_calculi_and_spatial_logics_ (end)
    
%\input{qm2pi.knots2pi} 

%\input{qm2pi.trefoil} 

%\input{qm2pi.mainthm} 

% subsection basic_interpretation (end)

%\input{qm2pi.rho.presentation} 
\subsection{The syntax and semantics of the notation system}\label{sub:the_syntax_and_semantics_of_the_notation_system} % (fold)

We now summarize a technical presentation of the calculus that
embodies our theory of dynamics. The typical presentation of such a
calculus follows the style of giving generators and relations on
them. The grammar, below, describing term constructors, freely
generates the set of processes, $\Proc$. This set is then quotiented
by a relation known as structural congruence and it is over this set
that the notion of dynamics is expressed. This presentation is
essentially that of \cite{MeredithR05} with the addition of
polyadicity and summation. For readability we have relegated some of
the technical subtleties to an appendix.

\subsubsection{Process grammar}\label{subsub:process_grammar}

\begin{mathpar}
  \inferrule* [lab=synchronization] {} {{M} \bc \pzero \;|\; x?F \;|\; x!C }
  \and
  \inferrule* [lab=abstraction] {} {{F} \bc (x)P}
  \and
  \inferrule* [lab=concretion] {} {{C} \bc \langle Q \rangle}
  \and
  \inferrule* [lab=process] {} {{P,Q} \bc M \;| \;P|Q \;|\; @{x}}
  \and
  \inferrule* [lab=name] {} {{x} \bc \quotep{P}}
\end{mathpar} 

Note that $\vec{x}$ (resp. $\vec{P}$) denotes a vector of names
(resp. processes) of length $|\vec{x}|$ (resp. $|\vec{P}|$). We adopt
the following useful abbreviations.

\begin{mathpar}
   x?(\vec{y}).P := x.(\vec{y})P \and  x\clift{\vec{P}} := x.\clift{\vec{P}}
   \and x!(y) := \lift{x}{\dropn{y}}
   \and \Pi_{i=0}^{n-1}P_i := P_0 | \ldots | P_{n-1}
\end{mathpar}

\subsubsection{Structural congruence}

\paragraph{Free and bound names and alpha-equivalence.} At the
core of structural equivalence is alpha-equivalence which identifies
process that are the same up to a change of variable. Formally, we
recognize the distinction between free and bound names. The free names
of a process, $\freenames{P}$, may be calculated recursively as
follows:

\begin{mathpar}
\freenames{\pzero} := \emptyset
  \and \\
  \freenames{x?(y).P} := \{ x \} \cup (\freenames{P} \setminus \{ y \})
  \and 
  \freenames{x!\langle P \rangle} := \{ x \} \cup \{ P \} 
  \and \\
  \freenames{P|Q} := \freenames{P} \cup \freenames{Q}
  \and \\
  \freenames{@{x}} := \{ x \}
\end{mathpar}

$\pi$
$\quotep{\pi}$

$\freenames{-} : \pi \to \mathcal{P}(\quotep{\pi})$

\begin{eqnarray*}
  \freenames{\pzero} & := & \emptyset \\
  \freenames{x?(y).P} & := & \{ x \} \cup (\freenames{P} \setminus \{ y \}) \\
  \freenames{x!\langle P \rangle} & := & \{ x \} \cup \{ P \} \\
  \freenames{P|Q} & := & \freenames{P} \cup \freenames{Q} \\
  \freenames{\dropn{x}} & := & \{ x \}
\end{eqnarray*}

The bound names of a process, $\boundnames{P}$, are those names occurring in $P$
that are not free. For example, in $x?(y).0$, the name $x$ is free, while $y$ is bound.

\begin{mathpar}
  \inferrule* [lab=monoidal-laws] {} { P|Q \equiv Q|P \and P|0 \equiv P \and P|(Q|R) \equiv (P|Q)|R }
\end{mathpar}

\begin{mathpar}
  \inferrule* [lab=alpha-equivalence] {} { (x)P \equiv (y)P\{y/x\} \and y \not\in \freenames{P} }
\end{mathpar}

\begin{definition}
Then two processes, $P,Q$, are alpha-equivalent if $P = Q\{\vec{y}/\vec{x}\}$ for
some $\vec{x} \in \boundnames{Q},\vec{y} \in \boundnames{P}$, where $Q\{\vec{y}/\vec{x}\}$
denotes the capture-avoiding substitution of $\vec{y}$ for $\vec{x}$ in $Q$.
\end{definition}

\begin{definition}
  The {\em structural congruence} \cite{SangiorgiWalker} , $\equiv$,
  between processes is the least congruence containing
  alpha-equivalence, satisfying the abelian monoid laws
  (associativity, commutativity and $\pzero$ as identity) for parallel
  composition $|$ and for summation $+$.
\end{definition}

\subsection{Name equivalence}

We take name equivalence, written $\nameeq$, to be the smallest
equivalence relation generated by the following rules.

\begin{mathpar}
\inferrule*[lab=Quote-drop]
{ }
{ \quotep{@{x}} \nameeq x }

\inferrule*[lab=Struct-equiv]
{ P \scong Q }
{ \quotep{P} \nameeq \quotep{Q} }
\end{mathpar}

The astute reader will have noticed that the mutual recursion of names
and processes imposes a mutual recursion on alpha-equivalence and
structural equivalence via name-equivalence. Fortunately, all of this
works out pleasantly and we may calculate in the natural way, free of
concern. The reader interested in the details is referred to the
appendix \ref{appendix:rho_details}.

\subsection{Substitution}

We use $\Proc$ for the set of processes, $\QProc$ for the set of
names, and $\id{\{}\vec{y} / \vec{x} \id{\}}$ to denote partial maps,
$s : \QProc \rightarrow \QProc$. A map, $s$ lifts, uniquely, to a map
on process terms, $\widehat{s} : \Proc \rightarrow \Proc$ by the
following equations.

\begin{mathpar}
  (0) \psubstp{Q}{P} := 0 \\
  (R \juxtap S) \psubstp{Q}{P}
  :=    
  (R)\psubstp{Q}{P} \juxtap (S) \psubstp{Q}{P} \\
  (x?(y).R) \psubstp{Q}{P}    
  :=    
  (x)\substp{Q}{P} (z)\concat( (R \psubstn{z}{y}) \psubstp{Q}{P} ) \\
  (\lift{x}{R}) \psubstp{Q}{P}  
  :=
  \lift{(x)\substp{Q}{P}}{ R \psubstp{Q}{P} } \\
%   (\dropn{x})  \psubstp{Q}{P}       
%   := 
%   \left\{ 
%     \begin{array}{ccc} 
%       \dropn{\quotep{Q}} & & x \nameeq \quotep{P} \\
%       \dropn{x} & & otherwise \\
%     \end{array}
%   \right. 
  (\dropn{x})  \psubstp{Q}{P}       
  := 
  \left\{ 
    \begin{array}{ccc} 
      Q & & x \nameeq \quotep{P} \\
      \dropn{x} & & otherwise \\
    \end{array}
  \right.
\end{mathpar}
 

where

\begin{eqnarray}
  (x)\id{\{} \lpquote Q \rpquote / \lpquote P \rpquote \id{\}}            = 
  \left\{ 
    \begin{array}{ccc}
      \lpquote Q \rpquote & & x \nameeq \lpquote P \rpquote \\
      x & & otherwise \\
    \end{array}
  \right. \nonumber
\end{eqnarray}

and $z$ is chosen distinct from $\quotep{P}$, $\quotep{Q}$, the free
names in $Q$, and all the names in $R$. Our $\alpha$-equivalence will
be built in the standard way from this substitution.

\begin{remark}\label{rem:no_self_referential_names}
  One consequence of these definitions is that $\forall P. \quotep{P}
  \not\in \freenames{P}$.
\end{remark}

\subsection{ Dynamic quote: an example }

Anticipating something of what's to come, consider applying the
substitution, $\widehat{\id{\{}u / z \id{\}}}$, to the following pair
of processes, $\lift{w}{y!(z)}$ and $w[ \lpquote y!(z) \rpquote ]$.

\begin{eqnarray}
	\lift{w}{y!(z)}\widehat{\id{\{}u / z \id{\}}}
		& = &
		\lift{w}{y!(u)} \nonumber\\
	w[ \lpquote y!(z) \rpquote ] \widehat{ \id{\{}u / z \id{\}} }
		& = &
		w[ \lpquote y!(z) \rpquote ] \nonumber
\end{eqnarray}

Because the body of the process between quotes is impervious to
substitution, we get radically different answers. In fact, by
examining the first process in an input context,
e.g. $x?(z).\lift{w}{y!(z)}$, we see that the process under the lift
operator may be shaped by prefixed inputs binding a name inside it. In
this sense, the lift operator will be seen as a way to dynamically
construct processes before reifying them as names.

Finally equipped with these standard features we can present the
dynamics of the calculus.

\subsubsection{Operational semantics} 

Finally, we introduce the computational dynamics. What marks these
algebras as distinct from other more traditionally studied algebraic
structures, e.g. vector spaces or polynomial rings, is the manner in
which dynamics is captured. In traditional structures, dynamics is typically
expressed through morphisms between such structures, as in linear maps
between vector spaces or morphisms between rings. In algebras
associated with the semantics of computation, the dynamics is
expressed as part of the algebraic structure itself, through a
reduction reduction relation typically denoted by $\red$. Below, we
give a recursive presentation of this relation for the calculus used
in the encoding.

$\red \subseteq \pi \times \pi$
$\red : \pi \to \mathcal{P}(\pi)$

\begin{mathpar}
  \inferrule* [lab=Comm] { \textsf{match}( x_{src}, x_{trgt} ) } { x_{trgt}?(y)P \; | \; x_{src}!\langle {Q} \rangle \red P\{\quotep{Q}/y}\} }
  \and \\
  \inferrule* [lab=Par] {{P} \red {P}'} {{{P} | {Q}} \red {{P}' | {Q}}}
  \and
  \inferrule* [lab=Equiv]{{{P} \scong {P}'} \andalso {{P}' \red {Q}'} \andalso {{Q}' \scong {Q}}}{{P} \red {Q}}
\end{mathpar}

\begin{eqnarray*}
  match_{\equiv} (\quotep{P},\quotep{Q}) & := & P \equiv Q \\
  match_{\dagger}(\quotep{P},\quotep{Q}) & := & \forall R. P|Q \red^{*} R => R \red^{*} 0 \\
  match_{K}(\quotep{P},\quotep{Q}) & := & K \mbox{ for some context } K
\end{eqnarray*}

$u?(x)P | u!\langle Q \rangle \red P\{\quotep{Q}/x\}$

%We write $\wred$ for $\red^*$, and $P\red$ if $\exists Q $ such that $ P \red Q$.
We write $P\red$ if $\exists Q $ such that $ P \red Q$ and $P\not\red$, otherwise.

\section{Replication}

As mentioned before, it is known that replication (and hence
recursion) can be implemented in a higher-order process algebra
\cite{SangiorgiWalker}. As our first example of calculation with the
machinery thus far presented we give the construction explicitly in
the {\rhoc}.

\begin{eqnarray}
	D_{x} & := & \prefix{x}{y}{(\binpar{\outputp{x}{y}}{@{y}})} \nonumber\\
	\bangp_{x}{P} & := & \binpar{{x}!\langle{\binpar{D_{x}}{P}}\rangle}{D_{x}} \nonumber
\end{eqnarray}

\begin{eqnarray}
	\bangp_{x}{P} & & \nonumber\\
	=
	& {x}!\langle{(\prefix{x}{y}{(\outputp{x}{y} | @{y})) | P}}\rangle 
	      | \prefix{x}{y}{(\outputp{x}{y} | @{y})} & \nonumber\\
	\red
	& (\outputp{x}{y} | @{y})\substn{\quotep{(\prefix{x}{y}{(@{y} | \outputp{x}{y})) | P}}}{y} & \nonumber\\
	=
	& \outputp{x}{\quotep{(\prefix{x}{y}{(\outputp{x}{y} | @{y})) | P}}}
	  | {(\prefix{x}{y}{(\outputp{x}{y} | @{y})) | P}} & \nonumber\\
	\red
	& \ldots & \nonumber\\
	\red^*
	& P | P | \ldots & \nonumber
\end{eqnarray}

Of course, this encoding, as an implementation, runs away, unfolding
$\bangp{P}$ eagerly. A lazier and more implementable replication
operator, restricted to input-guarded processes, may be obtained as follows.

\begin{eqnarray}
\bangp{\prefix{u}{v}{P}} 
	:= 
	\binpar{\lift{x}{\prefix{u}{v}{(\binpar{D(x)}{P})}}}{D(x)} \nonumber
\end{eqnarray}

\begin{remark}
  Note that the lazier definition still does not deal with summation
  or mixed summation (i.e. sums over input and output). The reader is
  invited to construct definitions of replication that deal with these
  features. 

  Further, the definitions are parameterized in a name, $x$. Can you,
  gentle reader, make a definition that eliminates this parameter and
  guarantees no accidental interaction between the replication
  machinery and the process being replicated -- i.e. no accidental
  sharing of names used by the process to get its work done and the
  name(s) used by the replication to effect copying. This latter
  revision of the definition of replication is crucial to obtaining
  the expected identity $!!P \sim !P$.
\end{remark}

\begin{remark}\label{rem:paradoxical_combinator}
  The reader familiar with the lambda calculus will have noticed the
  similarity between $D$ and the paradoxical combinator.

  [Ed. note: the existence of this seems to suggest we have to be more
  restrictive on the set of processes and names we admit if we are to
  support no-cloning.]
\end{remark}

\subsubsection{Bisimulation}

The computational dynamics gives rise to another kind of equivalence,
the equivalence of computational behavior. As previously mentioned
this is typically captured \emph{via} some form of bisimulation.

% The notion we use in this paper is weak barbed bisimulation
% \cite{milner91polyadicpi}.

The notion we use in this paper is derived from weak barbed
bisimulation \cite{milner91polyadicpi}. 

\begin{definition}
An \emph{observation relation}, $\downarrow_{\mathcal N}$, over a set
of names, $\mathcal N$, is the smallest relation satisfying the rules
below.

\infrule[Out-barb]{y \in {\mathcal N}, \; x \nameeq y}
		  {\outputp{x}{v} \downarrow_{\mathcal N} x}
\infrule[Par-barb]{\mbox{$P\downarrow_{\mathcal N} x$ or $Q\downarrow_{\mathcal N} x$}}
		  {\binpar{P}{Q} \downarrow_{\mathcal N} x}

We write $P \Downarrow_{\mathcal N} x$ if there is $Q$ such that 
$P \wred Q$ and $Q \downarrow_{\mathcal N} x$.
\end{definition}

\begin{definition}
%\label{def.bbisim}
An  ${\mathcal N}$-\emph{barbed bisimulation} over a set of names, ${\mathcal N}$, is a symmetric binary relation 
${\mathcal S}_{\mathcal N}$ between agents such that $P\rel{S}_{\mathcal N}Q$ implies:
\begin{enumerate}
\item If $P \red P'$ then $Q \wred Q'$ and $P'\rel{S}_{\mathcal N} Q'$.
\item If $P\downarrow_{\mathcal N} x$, then $Q\Downarrow_{\mathcal N} x$.
\end{enumerate}
$P$ is ${\mathcal N}$-barbed bisimilar to $Q$, written
$P \wbbisim_{\mathcal N} Q$, if $P \rel{S}_{\mathcal N} Q$ for some ${\mathcal N}$-barbed bisimulation ${\mathcal S}_{\mathcal N}$.
\end{definition}

$\mathcal{R} \subseteq \pi \times \pi$

$P \mathcal{R} Q => \forall P'. P \red P' \Rightarrow \exists Q'. Q \red Q', P' \mathcal{R} Q'$

$P \vdash x \Rightarrow Q \vdash x$

\begin{mathpar}
  \inferrule*[lab=Out-barb]{x \nameeq y}{{y}!\langle{Q}\rangle \vdash x}
  \and
  \inferrule*[lab=Par-barb]{\mbox{$P\vdash x$ or $Q\vdash x$}}{\binpar{P}{Q} \vdash x}
\end{mathpar}

\subsubsection{Contexts}

One of the principle advantages of computational calculi like the
$\pi$-calculus is a well-defined notion of context,
contextual-equivalence and a correlation between
contextual-equivalence and notions of bisimulation. The notion of
context allows the decomposition of a process into (sub-)process and
its syntactic environment, its context. Thus, a context may be
thought of as a process with a ``hole'' (written $\Box$) in it. The
application of a context $M$ to a process $P$, written $M[P]$, is
tantamount to filling the hole in $M$ with $P$. In this paper we do
not need the full weight of this theory, but do make use of the notion
of context in the proof the main theorem. 

\begin{mathpar}
  \inferrule* [lab=summation] {} {{M_{M},M_{N}} \bc \Box \;|\; x.M_{A} \;|\; M_{M}+M_{N}}
  \and
  \inferrule* [lab=agent] {} {{M_{A}} \bc (\vec{x})M_{P} \;| \; \clift{P_0,\ldots,M_{P},\ldots,P_N}}
  \and \\
  \inferrule* [lab=process] {} {{M_{P}} \bc M_{N} \;| \;P|M_{P} }
\end{mathpar} 

\begin{mathpar}
  \inferrule* [lab=sychronization] {} {M_{N} \bc \Box \;|\; x?M_{F} \;|\; x!M_{C}}
  \and
  \inferrule* [lab=abstraction] {} {{M_{F}} \bc (x)M_{P} }
  \and
  \inferrule* [lab=concretion] {} {{M_{C}} \bc \langle M_{P} \rangle }
  \and \\
  \inferrule* [lab=process] {} {{M_{P}} \bc M_{N} \;| \;P|M_{P} }
\end{mathpar}

\begin{definition}[contextual application] Given a context $M$, and
  process $P$, we define the \emph{contextual application}, $M[P] :=
  M\{P/\Box\}$. That is, the contextual application of M to P is the
  substitution of $P$ for $\Box$ in $M$.
\end{definition}

$\meaningof{-} : L \to \mathcal{P}(\pi)$

\begin{mathpar}
  \inferrule* [lab=collection] {} {\meaningof{true} = \pi, \and \meaningof{~E} = \pi \setminus \meaningof{E}, \and \meaningof{E_{1} \& E_{2}} = \meaningof{E_{1}} \cap \meaningof{E_{2}}}
\end{mathpar}

\begin{mathpar}
  \inferrule* [lab=structure] {} {\meaningof{0} = \{ P \in \pi | P \equiv 0 \}, \and \\ \meaningof{E_1 | E_2} = \{ P \in \pi | P \equiv P_{1} | P_{2}, P_{1} \in \meaningof{E_{1}}, P_{2} \in \meaningof{E_2}\} }
\end{mathpar}

\begin{mathpar}
 \inferrule* [lab=behavior] {} {\meaningof{\langle a?b \rangle E} = \{ P \in \pi | P \equiv Q | u?(y)P', \\ \and \\\\ \and \\ \;\;\; u \in \meaningof{a}, \forall z.P'\{z/y\} \in \meaningof{E\{z/b\}}\}, \and \\ \meaningof{a!E} = \{ P \in \pi | P \equiv Q | x!\langle P' \rangle, x \in \meaningof{a} P' \in \meaningof{E}\} }
\end{mathpar}

\begin{mathpar}
 \inferrule* [lab=nominal] {} {\meaningof{\quotep{E}} = \{ \quotep{P} \in \quotep{\pi} | P \in \meaningof{E} \}, \and \meaningof{\quotep{P}} = \{ \quotep{Q} \in \quotep{\pi} | P \equiv Q \} \and \\ \meaningof{@\quotep{E}} = \{ P \in \pi | P \equiv @x, x \in \meaningof{E} \}}
\end{mathpar}

\begin{eqnarray*}
  \\
  \meaningof{-} : TS \to ST
\end{eqnarray*}

\begin{eqnarray*}
  \\
  L : TS \to ST
\end{eqnarray*}

\begin{eqnarray*}
  \\
  P \models E \iff P \in \meaningof{E}
\end{eqnarray*}

\begin{eqnarray*}
  P \approx_{L} Q \iff \forall E \in L. P \models E \iff Q \models E
\end{eqnarray*}

\begin{eqnarray*}
  P \approx_{K} Q
\end{eqnarray*}

\begin{eqnarray*}
  P \approx Q
\end{eqnarray*}

$\approx_{K} = \approx = \approx_{L}$

\subsubsection{Contextual duality}

Note that contexts extend the quotation operation to a family of
operations from processes to names. Given a context, $M$, we can
define a \emph{nominal context}, $\quotep{M}$ by $\quotep{M}[P] :=
\quotep{M[P]}$. To foreshadow what is to come we observe that these
operations enjoy a duality with processes very much like the duality
between vectors and maps from vectors to scalars.

Further, because the calculus is essentially higher-order, we have a
correspondence between contexts and processes. More specifically,
given a name $x$ and a context $M$ we can construct $M^{*}_{x}$ such
that 

\begin{mathpar}
  M^{*}_{x} | \lift{x}{P} \red M[P]
\end{mathpar}

namely,

\begin{mathpar}
  M^{*}_{x} := x?(u).M[\dropn{u}]
\end{mathpar}

The dependence of $M^{*}_{x}$ on a name makes it an abstraction, 

\begin{mathpar}
  M^{*} := (x)x?(u).M[\dropn{u}]
\end{mathpar}

\subsection{Additional notation}

It will sometimes be convenient to denote the process a name
quotes. We already have the notation $x = \quotep{P}$, but it will be
convenient to introduce an alternate notation, $\procn{x}$, when we
want to emphasize the connection to the use of the name. Note that, by
virtue of name equivalence, $\quotep{\procn{x}} \nameeq x$; so, the
notation is consistent with previous definitions.

Further, because names have structure it is possible to effect
substitutions on the basis of that structure. This means we need to
upgrade our notation for substitutions, which we accomplish by
adapting comprehension notation. Thus,

\begin{mathpar}
  P\{ y / x : x \in S \}
\end{mathpar}

is interpreted to mean the process derived from P by replacing (in a
capture-avoiding manner) each occurrence of $x$ in $S$ by $y$. For example,

\begin{mathpar}
  P\{ \quotep{\procn{x}|\procn{x}} / x : x \in \freenames{P} \}
\end{mathpar}

will replace each (occurrence) of a free name $x$ in $P$ by
$\quotep{\procn{x}|\procn{x}}$.

Also, we will avail ourselves of the notation $x^{L}$ and $x^{R}$ to
denote injections of a name into disjoint copies of the name
space. There are numerous ways to accomplish this. One example can be
found in \cite{MeredithR05}. This notation overloads to vectors of
names: $\vec{x}^{\pi} := (x_{i}^{\pi} \; : \; 0 \leq i < |\vec{x}| )$ where $\pi \in \{L,R\}$.

We also use $P^{\Box} := P|\Box$.

In \cite{MeredithR05} an interpretation of the new operator is
given. It turns out that there are several possible interpretations
all enjoying the requisite algebraic properties of the operator (see
\cite{milner91polyadicpi}). We will therefore make liberal use of
$(\nu\; \vec{x})P$.

% subsection the_syntax_and_semantics_of_the_notation_system (end)   

\input{qm2pi.qmops} 

\input{qm2pi.sterngerlach} 

\input{qm2pi.metric} 

% section concurrent_process_calculi (end)

%\input{qm2pi.proofsketch}

% section proof sketch (end)

%\input{qm2pi.slviaknots} 

% section spatial logic via knots (end)

\input{qm2pi.conclusion}

% section conclusion (end)

%\input{qm2pi.dtcodes} 

% section wiring algorithm (end)

\input{qm2pi.ack} 

% section acknowledgments (end)

\newpage


\bibliographystyle{plain}   
\bibliography{../../biblios/main.bib}

\input{qm2pi.rhodetails}

\end{document}



% section front matter (end)

\section{Introduction}\label{sec:introduction} % (fold)
In this draft of the material i am going to have to dispense with the
usual writing conventions adopted in papers on these topics. i'm going
to have adopt whatever tone i need at the time i'm writing up the
calculations. Sometimes this may be very conversational; others it may
be the barest mathematical grunts; others still it may be that i have
lifted text from one of my other papers because the exposition of some
point was better said there. i hope that my readers are not unduly put
out by this decision. i'm not doing this to flout convention or be
rebellious. i find these calculations very technically challenging. To
keep everything going technically, something has to give; i have to
let go of some cognitive burden. So, the academic writing style --
with all of its trade-offs in terms of facilitating technical
communication -- is what i'm letting go of. Perhaps subsequent drafts
can be tightened and polished, but for now, i'm going to speak as if
we were sitting together in a coffee shop with a laptop, wifi and a
pad of paper and a pencil.

So, here's what i have to say. We -- you and i, comfortably ensconced
in our coffee shop and well-equipped with our tools -- can realize and
carry out the calculations of quantum mechanics over a very different
formal theory of dynamics, a formal theory of dynamics that
corresponds to a theory of concurrent computation with
\emph{reflection}. It has the advantage that the underlying theory is
already `quantized', but supports analogues all of the continuuous
operations. Strikingly, this underlying theory has recently been
connected with a notion of metric that we can show, by calculating
together, coincides with the metric induced by the inner product.

There are a lot of reasons why you might be interested in seeing
calculations of this form. Here's why i'm interested. For the past
several centuries there has been no competitor to the ``Newtonian''
account of dynamics. As a result the predominant share of accounts of
dynamical systems and situations have had to be formulated in terms of
the Newtonian machinery. i view this as an intellectually dangerous
position to occupy. Everything, despite it's intrinsic shape, turns
into a nail to be hit with this hammer. Recently, however, the theory
of computation has matured to the point where we have candidates for
theories of dynamics that offer very different perspective on
reasoning about dynamical systems and situations. Testing these
candidates against very successful accounts of dynamical situations,
like quantum mechanics, is going to give us some sense of how mature
they are and some measure of the quality of these accounts of
dynamics.

\subsection{Summary of contributions and outline of paper}

So, we're going to develop an interpretation of the operations of
quantum mechanics normally interpreted by Hilbert spaces and
operators. We're going to do this over a theory of computation. Note
that this is very different than the usual quantum computation program
which develops notions of computation over quantum mechanics. Rather,
we are developing a story that aligns with Wheeler's slogan: It from
Bit. To do this we will first provide an account of the theory of
computation at play here. Then we will dive into a calculation-driven
interpretation of the operations of quantum mechanics.

The reason we take this approach is that -- until very recently --
there hasn't been an axiomatic account of quantum mechanics. As a
result there has been no sharp delineation of the mathematical theory
supporting interpretation of the physical theory and the physical
theory, itself. So, ambient features of the maths are free to be
exploited (or supressed) without a real accounting of their physical
relevance. There is no sharp statement ``here's the physical theory''
qua \emph{theory} and ``here's the mathematical interpretation''
enabling a judgment of how faithful the interpretation is -- apart
from experimental observation. When there is an axiomatic account we
can judge how well a given mathematical formalism supports an
interpretation of the axioms, independent of
experimentation. Likewise, we can judge how well we have captured our
physical evidence and experience with our axiomatics, independent of
any specific mathematical implementation, with accidental detail that
may or may not have physical significance. 

In lieu of a fully fleshed out and vetted axiomatic account of quantum
mechanics, interpreting the operational notions in service of modeling
physical systems will have to suffice. In other words, we are not in
the business of providing a model of Hilbert spaces and operators. We
are in the business of providing a model of quantum mechanics because
we are motivated by testing our notions of dynamics against physical
theory; and, the predictive calculations of the physical theory must
serve as the best formulation -- shy of a fully fleshed out axiomatic
account -- of the physical theory itself (as they have for scientific
theories since time immemorial). Put another way, despite a
whole-hearted commitment to an It-from-Bit ontology, we are firmly
aligned with the shut-up-and-calculate camp as the best way to obtain
results either from the physical perspective or as a quality assurance
measure of our fledgling theory of dynamics.

In detail, we present a reflective process calculus. Then we develop
intuitive correspondences between the notions available in this
calculus and the usual physical notions supporting quantum mechanical
calculations. Thus, 

\begin{table}[htp]
  \center{
    \fbox{
      \begin{tabular}{c|c}
        quantum mechanics & process calculus \\
        \hline
        scalar & name \\
        state vector & process \\
        dual & contextual duals \\
        matrix & formal sums of process-context-dual pairs \\
        orthogonality & process annihilation \\
        inner product & execution-formula + quoting
      \end{tabular}
    }
  }
  \caption{QM - process calculi correspondences}
\end{table}

Then we tighten up these intuitions to operational definitions. We
employ the Dirac notation as the best proxy we can find for an
abstract syntax of the quantum mechanical notions. The definitions we
develop put us in contact with equational constraints coming from the
theory that we demonstrate the definitions and calculations satisfy.

This puts us in a position to shut up and calculate for the
Stern-Gerlach experimental set up, showing how these predictive
calculations become calculations on processes in our theory of a
reflective process calculus.

Penultimately, we demonstrate that the notion of metric coming from
the inner product coincides with the notion of metric available from
the theory of bisimulation. This demonstration gives us the right to
think of space as arising from behavior. Finally, we consider where we
might go from the new vantage point we have obtained.

% section introduction (end) 
 
% section introduction (end)

% \documentclass[12pt]{llncs}
%\documentclass{jktr}

\usepackage[pdftex]{hyperref}                   
\usepackage {listings}
\usepackage {mathpartir}
\usepackage{bcprules}
%\usepackage{listings}
                       
\usepackage{graphicx} 
%\usepackage[margins=2.5cm,nohead,nofoot]{geometry}
%\usepackage{geometry}
\usepackage{amsfonts}
\usepackage{amstext}
\usepackage{latexsym}
\usepackage{amssymb}
\usepackage{color}


%\include{myPreamble}
\include{qm2pi.local} 

%\ifpdf
%\usepackage[pdftex]{graphicx}
%\else
%\usepackage{graphicx}
%\fi

 % \ifpdf
%  \usepackage{pdfsync}
%  \if


%\title{Brief Article}
%\author{David F. Snyder}
%\author{L.G. Meredith}

%\address{Dept. of Math., Texas State University--San Marcos, San Marcos, TX 78666}
       
\pagestyle{empty}


\begin{document}

\lstset{language=[Objective]Caml,frame=shadowbox}

\input{qm2pi.front}

% section front matter (end)

\input{qm2pi.intro} 
 
% section introduction (end)

% \input{qm2pi.knotations} 

% section notation (end)

\input{qm2pi.process.calculi} 

% section concurrent_process_calculi_and_spatial_logics_ (end)
    
%\input{qm2pi.knots2pi} 

%\input{qm2pi.trefoil} 

%\input{qm2pi.mainthm} 

% subsection basic_interpretation (end)

%\input{qm2pi.rho.presentation} 
\subsection{The syntax and semantics of the notation system}\label{sub:the_syntax_and_semantics_of_the_notation_system} % (fold)

We now summarize a technical presentation of the calculus that
embodies our theory of dynamics. The typical presentation of such a
calculus follows the style of giving generators and relations on
them. The grammar, below, describing term constructors, freely
generates the set of processes, $\Proc$. This set is then quotiented
by a relation known as structural congruence and it is over this set
that the notion of dynamics is expressed. This presentation is
essentially that of \cite{MeredithR05} with the addition of
polyadicity and summation. For readability we have relegated some of
the technical subtleties to an appendix.

\subsubsection{Process grammar}\label{subsub:process_grammar}

\begin{mathpar}
  \inferrule* [lab=synchronization] {} {{M} \bc \pzero \;|\; x?F \;|\; x!C }
  \and
  \inferrule* [lab=abstraction] {} {{F} \bc (x)P}
  \and
  \inferrule* [lab=concretion] {} {{C} \bc \langle Q \rangle}
  \and
  \inferrule* [lab=process] {} {{P,Q} \bc M \;| \;P|Q \;|\; @{x}}
  \and
  \inferrule* [lab=name] {} {{x} \bc \quotep{P}}
\end{mathpar} 

Note that $\vec{x}$ (resp. $\vec{P}$) denotes a vector of names
(resp. processes) of length $|\vec{x}|$ (resp. $|\vec{P}|$). We adopt
the following useful abbreviations.

\begin{mathpar}
   x?(\vec{y}).P := x.(\vec{y})P \and  x\clift{\vec{P}} := x.\clift{\vec{P}}
   \and x!(y) := \lift{x}{\dropn{y}}
   \and \Pi_{i=0}^{n-1}P_i := P_0 | \ldots | P_{n-1}
\end{mathpar}

\subsubsection{Structural congruence}

\paragraph{Free and bound names and alpha-equivalence.} At the
core of structural equivalence is alpha-equivalence which identifies
process that are the same up to a change of variable. Formally, we
recognize the distinction between free and bound names. The free names
of a process, $\freenames{P}$, may be calculated recursively as
follows:

\begin{mathpar}
\freenames{\pzero} := \emptyset
  \and \\
  \freenames{x?(y).P} := \{ x \} \cup (\freenames{P} \setminus \{ y \})
  \and 
  \freenames{x!\langle P \rangle} := \{ x \} \cup \{ P \} 
  \and \\
  \freenames{P|Q} := \freenames{P} \cup \freenames{Q}
  \and \\
  \freenames{@{x}} := \{ x \}
\end{mathpar}

$\pi$
$\quotep{\pi}$

$\freenames{-} : \pi \to \mathcal{P}(\quotep{\pi})$

\begin{eqnarray*}
  \freenames{\pzero} & := & \emptyset \\
  \freenames{x?(y).P} & := & \{ x \} \cup (\freenames{P} \setminus \{ y \}) \\
  \freenames{x!\langle P \rangle} & := & \{ x \} \cup \{ P \} \\
  \freenames{P|Q} & := & \freenames{P} \cup \freenames{Q} \\
  \freenames{\dropn{x}} & := & \{ x \}
\end{eqnarray*}

The bound names of a process, $\boundnames{P}$, are those names occurring in $P$
that are not free. For example, in $x?(y).0$, the name $x$ is free, while $y$ is bound.

\begin{mathpar}
  \inferrule* [lab=monoidal-laws] {} { P|Q \equiv Q|P \and P|0 \equiv P \and P|(Q|R) \equiv (P|Q)|R }
\end{mathpar}

\begin{mathpar}
  \inferrule* [lab=alpha-equivalence] {} { (x)P \equiv (y)P\{y/x\} \and y \not\in \freenames{P} }
\end{mathpar}

\begin{definition}
Then two processes, $P,Q$, are alpha-equivalent if $P = Q\{\vec{y}/\vec{x}\}$ for
some $\vec{x} \in \boundnames{Q},\vec{y} \in \boundnames{P}$, where $Q\{\vec{y}/\vec{x}\}$
denotes the capture-avoiding substitution of $\vec{y}$ for $\vec{x}$ in $Q$.
\end{definition}

\begin{definition}
  The {\em structural congruence} \cite{SangiorgiWalker} , $\equiv$,
  between processes is the least congruence containing
  alpha-equivalence, satisfying the abelian monoid laws
  (associativity, commutativity and $\pzero$ as identity) for parallel
  composition $|$ and for summation $+$.
\end{definition}

\subsection{Name equivalence}

We take name equivalence, written $\nameeq$, to be the smallest
equivalence relation generated by the following rules.

\begin{mathpar}
\inferrule*[lab=Quote-drop]
{ }
{ \quotep{@{x}} \nameeq x }

\inferrule*[lab=Struct-equiv]
{ P \scong Q }
{ \quotep{P} \nameeq \quotep{Q} }
\end{mathpar}

The astute reader will have noticed that the mutual recursion of names
and processes imposes a mutual recursion on alpha-equivalence and
structural equivalence via name-equivalence. Fortunately, all of this
works out pleasantly and we may calculate in the natural way, free of
concern. The reader interested in the details is referred to the
appendix \ref{appendix:rho_details}.

\subsection{Substitution}

We use $\Proc$ for the set of processes, $\QProc$ for the set of
names, and $\id{\{}\vec{y} / \vec{x} \id{\}}$ to denote partial maps,
$s : \QProc \rightarrow \QProc$. A map, $s$ lifts, uniquely, to a map
on process terms, $\widehat{s} : \Proc \rightarrow \Proc$ by the
following equations.

\begin{mathpar}
  (0) \psubstp{Q}{P} := 0 \\
  (R \juxtap S) \psubstp{Q}{P}
  :=    
  (R)\psubstp{Q}{P} \juxtap (S) \psubstp{Q}{P} \\
  (x?(y).R) \psubstp{Q}{P}    
  :=    
  (x)\substp{Q}{P} (z)\concat( (R \psubstn{z}{y}) \psubstp{Q}{P} ) \\
  (\lift{x}{R}) \psubstp{Q}{P}  
  :=
  \lift{(x)\substp{Q}{P}}{ R \psubstp{Q}{P} } \\
%   (\dropn{x})  \psubstp{Q}{P}       
%   := 
%   \left\{ 
%     \begin{array}{ccc} 
%       \dropn{\quotep{Q}} & & x \nameeq \quotep{P} \\
%       \dropn{x} & & otherwise \\
%     \end{array}
%   \right. 
  (\dropn{x})  \psubstp{Q}{P}       
  := 
  \left\{ 
    \begin{array}{ccc} 
      Q & & x \nameeq \quotep{P} \\
      \dropn{x} & & otherwise \\
    \end{array}
  \right.
\end{mathpar}
 

where

\begin{eqnarray}
  (x)\id{\{} \lpquote Q \rpquote / \lpquote P \rpquote \id{\}}            = 
  \left\{ 
    \begin{array}{ccc}
      \lpquote Q \rpquote & & x \nameeq \lpquote P \rpquote \\
      x & & otherwise \\
    \end{array}
  \right. \nonumber
\end{eqnarray}

and $z$ is chosen distinct from $\quotep{P}$, $\quotep{Q}$, the free
names in $Q$, and all the names in $R$. Our $\alpha$-equivalence will
be built in the standard way from this substitution.

\begin{remark}\label{rem:no_self_referential_names}
  One consequence of these definitions is that $\forall P. \quotep{P}
  \not\in \freenames{P}$.
\end{remark}

\subsection{ Dynamic quote: an example }

Anticipating something of what's to come, consider applying the
substitution, $\widehat{\id{\{}u / z \id{\}}}$, to the following pair
of processes, $\lift{w}{y!(z)}$ and $w[ \lpquote y!(z) \rpquote ]$.

\begin{eqnarray}
	\lift{w}{y!(z)}\widehat{\id{\{}u / z \id{\}}}
		& = &
		\lift{w}{y!(u)} \nonumber\\
	w[ \lpquote y!(z) \rpquote ] \widehat{ \id{\{}u / z \id{\}} }
		& = &
		w[ \lpquote y!(z) \rpquote ] \nonumber
\end{eqnarray}

Because the body of the process between quotes is impervious to
substitution, we get radically different answers. In fact, by
examining the first process in an input context,
e.g. $x?(z).\lift{w}{y!(z)}$, we see that the process under the lift
operator may be shaped by prefixed inputs binding a name inside it. In
this sense, the lift operator will be seen as a way to dynamically
construct processes before reifying them as names.

Finally equipped with these standard features we can present the
dynamics of the calculus.

\subsubsection{Operational semantics} 

Finally, we introduce the computational dynamics. What marks these
algebras as distinct from other more traditionally studied algebraic
structures, e.g. vector spaces or polynomial rings, is the manner in
which dynamics is captured. In traditional structures, dynamics is typically
expressed through morphisms between such structures, as in linear maps
between vector spaces or morphisms between rings. In algebras
associated with the semantics of computation, the dynamics is
expressed as part of the algebraic structure itself, through a
reduction reduction relation typically denoted by $\red$. Below, we
give a recursive presentation of this relation for the calculus used
in the encoding.

$\red \subseteq \pi \times \pi$
$\red : \pi \to \mathcal{P}(\pi)$

\begin{mathpar}
  \inferrule* [lab=Comm] { \textsf{match}( x_{src}, x_{trgt} ) } { x_{trgt}?(y)P \; | \; x_{src}!\langle {Q} \rangle \red P\{\quotep{Q}/y}\} }
  \and \\
  \inferrule* [lab=Par] {{P} \red {P}'} {{{P} | {Q}} \red {{P}' | {Q}}}
  \and
  \inferrule* [lab=Equiv]{{{P} \scong {P}'} \andalso {{P}' \red {Q}'} \andalso {{Q}' \scong {Q}}}{{P} \red {Q}}
\end{mathpar}

\begin{eqnarray*}
  match_{\equiv} (\quotep{P},\quotep{Q}) & := & P \equiv Q \\
  match_{\dagger}(\quotep{P},\quotep{Q}) & := & \forall R. P|Q \red^{*} R => R \red^{*} 0 \\
  match_{K}(\quotep{P},\quotep{Q}) & := & K \mbox{ for some context } K
\end{eqnarray*}

$u?(x)P | u!\langle Q \rangle \red P\{\quotep{Q}/x\}$

%We write $\wred$ for $\red^*$, and $P\red$ if $\exists Q $ such that $ P \red Q$.
We write $P\red$ if $\exists Q $ such that $ P \red Q$ and $P\not\red$, otherwise.

\section{Replication}

As mentioned before, it is known that replication (and hence
recursion) can be implemented in a higher-order process algebra
\cite{SangiorgiWalker}. As our first example of calculation with the
machinery thus far presented we give the construction explicitly in
the {\rhoc}.

\begin{eqnarray}
	D_{x} & := & \prefix{x}{y}{(\binpar{\outputp{x}{y}}{@{y}})} \nonumber\\
	\bangp_{x}{P} & := & \binpar{{x}!\langle{\binpar{D_{x}}{P}}\rangle}{D_{x}} \nonumber
\end{eqnarray}

\begin{eqnarray}
	\bangp_{x}{P} & & \nonumber\\
	=
	& {x}!\langle{(\prefix{x}{y}{(\outputp{x}{y} | @{y})) | P}}\rangle 
	      | \prefix{x}{y}{(\outputp{x}{y} | @{y})} & \nonumber\\
	\red
	& (\outputp{x}{y} | @{y})\substn{\quotep{(\prefix{x}{y}{(@{y} | \outputp{x}{y})) | P}}}{y} & \nonumber\\
	=
	& \outputp{x}{\quotep{(\prefix{x}{y}{(\outputp{x}{y} | @{y})) | P}}}
	  | {(\prefix{x}{y}{(\outputp{x}{y} | @{y})) | P}} & \nonumber\\
	\red
	& \ldots & \nonumber\\
	\red^*
	& P | P | \ldots & \nonumber
\end{eqnarray}

Of course, this encoding, as an implementation, runs away, unfolding
$\bangp{P}$ eagerly. A lazier and more implementable replication
operator, restricted to input-guarded processes, may be obtained as follows.

\begin{eqnarray}
\bangp{\prefix{u}{v}{P}} 
	:= 
	\binpar{\lift{x}{\prefix{u}{v}{(\binpar{D(x)}{P})}}}{D(x)} \nonumber
\end{eqnarray}

\begin{remark}
  Note that the lazier definition still does not deal with summation
  or mixed summation (i.e. sums over input and output). The reader is
  invited to construct definitions of replication that deal with these
  features. 

  Further, the definitions are parameterized in a name, $x$. Can you,
  gentle reader, make a definition that eliminates this parameter and
  guarantees no accidental interaction between the replication
  machinery and the process being replicated -- i.e. no accidental
  sharing of names used by the process to get its work done and the
  name(s) used by the replication to effect copying. This latter
  revision of the definition of replication is crucial to obtaining
  the expected identity $!!P \sim !P$.
\end{remark}

\begin{remark}\label{rem:paradoxical_combinator}
  The reader familiar with the lambda calculus will have noticed the
  similarity between $D$ and the paradoxical combinator.

  [Ed. note: the existence of this seems to suggest we have to be more
  restrictive on the set of processes and names we admit if we are to
  support no-cloning.]
\end{remark}

\subsubsection{Bisimulation}

The computational dynamics gives rise to another kind of equivalence,
the equivalence of computational behavior. As previously mentioned
this is typically captured \emph{via} some form of bisimulation.

% The notion we use in this paper is weak barbed bisimulation
% \cite{milner91polyadicpi}.

The notion we use in this paper is derived from weak barbed
bisimulation \cite{milner91polyadicpi}. 

\begin{definition}
An \emph{observation relation}, $\downarrow_{\mathcal N}$, over a set
of names, $\mathcal N$, is the smallest relation satisfying the rules
below.

\infrule[Out-barb]{y \in {\mathcal N}, \; x \nameeq y}
		  {\outputp{x}{v} \downarrow_{\mathcal N} x}
\infrule[Par-barb]{\mbox{$P\downarrow_{\mathcal N} x$ or $Q\downarrow_{\mathcal N} x$}}
		  {\binpar{P}{Q} \downarrow_{\mathcal N} x}

We write $P \Downarrow_{\mathcal N} x$ if there is $Q$ such that 
$P \wred Q$ and $Q \downarrow_{\mathcal N} x$.
\end{definition}

\begin{definition}
%\label{def.bbisim}
An  ${\mathcal N}$-\emph{barbed bisimulation} over a set of names, ${\mathcal N}$, is a symmetric binary relation 
${\mathcal S}_{\mathcal N}$ between agents such that $P\rel{S}_{\mathcal N}Q$ implies:
\begin{enumerate}
\item If $P \red P'$ then $Q \wred Q'$ and $P'\rel{S}_{\mathcal N} Q'$.
\item If $P\downarrow_{\mathcal N} x$, then $Q\Downarrow_{\mathcal N} x$.
\end{enumerate}
$P$ is ${\mathcal N}$-barbed bisimilar to $Q$, written
$P \wbbisim_{\mathcal N} Q$, if $P \rel{S}_{\mathcal N} Q$ for some ${\mathcal N}$-barbed bisimulation ${\mathcal S}_{\mathcal N}$.
\end{definition}

$\mathcal{R} \subseteq \pi \times \pi$

$P \mathcal{R} Q => \forall P'. P \red P' \Rightarrow \exists Q'. Q \red Q', P' \mathcal{R} Q'$

$P \vdash x \Rightarrow Q \vdash x$

\begin{mathpar}
  \inferrule*[lab=Out-barb]{x \nameeq y}{{y}!\langle{Q}\rangle \vdash x}
  \and
  \inferrule*[lab=Par-barb]{\mbox{$P\vdash x$ or $Q\vdash x$}}{\binpar{P}{Q} \vdash x}
\end{mathpar}

\subsubsection{Contexts}

One of the principle advantages of computational calculi like the
$\pi$-calculus is a well-defined notion of context,
contextual-equivalence and a correlation between
contextual-equivalence and notions of bisimulation. The notion of
context allows the decomposition of a process into (sub-)process and
its syntactic environment, its context. Thus, a context may be
thought of as a process with a ``hole'' (written $\Box$) in it. The
application of a context $M$ to a process $P$, written $M[P]$, is
tantamount to filling the hole in $M$ with $P$. In this paper we do
not need the full weight of this theory, but do make use of the notion
of context in the proof the main theorem. 

\begin{mathpar}
  \inferrule* [lab=summation] {} {{M_{M},M_{N}} \bc \Box \;|\; x.M_{A} \;|\; M_{M}+M_{N}}
  \and
  \inferrule* [lab=agent] {} {{M_{A}} \bc (\vec{x})M_{P} \;| \; \clift{P_0,\ldots,M_{P},\ldots,P_N}}
  \and \\
  \inferrule* [lab=process] {} {{M_{P}} \bc M_{N} \;| \;P|M_{P} }
\end{mathpar} 

\begin{mathpar}
  \inferrule* [lab=sychronization] {} {M_{N} \bc \Box \;|\; x?M_{F} \;|\; x!M_{C}}
  \and
  \inferrule* [lab=abstraction] {} {{M_{F}} \bc (x)M_{P} }
  \and
  \inferrule* [lab=concretion] {} {{M_{C}} \bc \langle M_{P} \rangle }
  \and \\
  \inferrule* [lab=process] {} {{M_{P}} \bc M_{N} \;| \;P|M_{P} }
\end{mathpar}

\begin{definition}[contextual application] Given a context $M$, and
  process $P$, we define the \emph{contextual application}, $M[P] :=
  M\{P/\Box\}$. That is, the contextual application of M to P is the
  substitution of $P$ for $\Box$ in $M$.
\end{definition}

$\meaningof{-} : L \to \mathcal{P}(\pi)$

\begin{mathpar}
  \inferrule* [lab=collection] {} {\meaningof{true} = \pi, \and \meaningof{~E} = \pi \setminus \meaningof{E}, \and \meaningof{E_{1} \& E_{2}} = \meaningof{E_{1}} \cap \meaningof{E_{2}}}
\end{mathpar}

\begin{mathpar}
  \inferrule* [lab=structure] {} {\meaningof{0} = \{ P \in \pi | P \equiv 0 \}, \and \\ \meaningof{E_1 | E_2} = \{ P \in \pi | P \equiv P_{1} | P_{2}, P_{1} \in \meaningof{E_{1}}, P_{2} \in \meaningof{E_2}\} }
\end{mathpar}

\begin{mathpar}
 \inferrule* [lab=behavior] {} {\meaningof{\langle a?b \rangle E} = \{ P \in \pi | P \equiv Q | u?(y)P', \\ \and \\\\ \and \\ \;\;\; u \in \meaningof{a}, \forall z.P'\{z/y\} \in \meaningof{E\{z/b\}}\}, \and \\ \meaningof{a!E} = \{ P \in \pi | P \equiv Q | x!\langle P' \rangle, x \in \meaningof{a} P' \in \meaningof{E}\} }
\end{mathpar}

\begin{mathpar}
 \inferrule* [lab=nominal] {} {\meaningof{\quotep{E}} = \{ \quotep{P} \in \quotep{\pi} | P \in \meaningof{E} \}, \and \meaningof{\quotep{P}} = \{ \quotep{Q} \in \quotep{\pi} | P \equiv Q \} \and \\ \meaningof{@\quotep{E}} = \{ P \in \pi | P \equiv @x, x \in \meaningof{E} \}}
\end{mathpar}

\begin{eqnarray*}
  \\
  \meaningof{-} : TS \to ST
\end{eqnarray*}

\begin{eqnarray*}
  \\
  L : TS \to ST
\end{eqnarray*}

\begin{eqnarray*}
  \\
  P \models E \iff P \in \meaningof{E}
\end{eqnarray*}

\begin{eqnarray*}
  P \approx_{L} Q \iff \forall E \in L. P \models E \iff Q \models E
\end{eqnarray*}

\begin{eqnarray*}
  P \approx_{K} Q
\end{eqnarray*}

\begin{eqnarray*}
  P \approx Q
\end{eqnarray*}

$\approx_{K} = \approx = \approx_{L}$

\subsubsection{Contextual duality}

Note that contexts extend the quotation operation to a family of
operations from processes to names. Given a context, $M$, we can
define a \emph{nominal context}, $\quotep{M}$ by $\quotep{M}[P] :=
\quotep{M[P]}$. To foreshadow what is to come we observe that these
operations enjoy a duality with processes very much like the duality
between vectors and maps from vectors to scalars.

Further, because the calculus is essentially higher-order, we have a
correspondence between contexts and processes. More specifically,
given a name $x$ and a context $M$ we can construct $M^{*}_{x}$ such
that 

\begin{mathpar}
  M^{*}_{x} | \lift{x}{P} \red M[P]
\end{mathpar}

namely,

\begin{mathpar}
  M^{*}_{x} := x?(u).M[\dropn{u}]
\end{mathpar}

The dependence of $M^{*}_{x}$ on a name makes it an abstraction, 

\begin{mathpar}
  M^{*} := (x)x?(u).M[\dropn{u}]
\end{mathpar}

\subsection{Additional notation}

It will sometimes be convenient to denote the process a name
quotes. We already have the notation $x = \quotep{P}$, but it will be
convenient to introduce an alternate notation, $\procn{x}$, when we
want to emphasize the connection to the use of the name. Note that, by
virtue of name equivalence, $\quotep{\procn{x}} \nameeq x$; so, the
notation is consistent with previous definitions.

Further, because names have structure it is possible to effect
substitutions on the basis of that structure. This means we need to
upgrade our notation for substitutions, which we accomplish by
adapting comprehension notation. Thus,

\begin{mathpar}
  P\{ y / x : x \in S \}
\end{mathpar}

is interpreted to mean the process derived from P by replacing (in a
capture-avoiding manner) each occurrence of $x$ in $S$ by $y$. For example,

\begin{mathpar}
  P\{ \quotep{\procn{x}|\procn{x}} / x : x \in \freenames{P} \}
\end{mathpar}

will replace each (occurrence) of a free name $x$ in $P$ by
$\quotep{\procn{x}|\procn{x}}$.

Also, we will avail ourselves of the notation $x^{L}$ and $x^{R}$ to
denote injections of a name into disjoint copies of the name
space. There are numerous ways to accomplish this. One example can be
found in \cite{MeredithR05}. This notation overloads to vectors of
names: $\vec{x}^{\pi} := (x_{i}^{\pi} \; : \; 0 \leq i < |\vec{x}| )$ where $\pi \in \{L,R\}$.

We also use $P^{\Box} := P|\Box$.

In \cite{MeredithR05} an interpretation of the new operator is
given. It turns out that there are several possible interpretations
all enjoying the requisite algebraic properties of the operator (see
\cite{milner91polyadicpi}). We will therefore make liberal use of
$(\nu\; \vec{x})P$.

% subsection the_syntax_and_semantics_of_the_notation_system (end)   

\input{qm2pi.qmops} 

\input{qm2pi.sterngerlach} 

\input{qm2pi.metric} 

% section concurrent_process_calculi (end)

%\input{qm2pi.proofsketch}

% section proof sketch (end)

%\input{qm2pi.slviaknots} 

% section spatial logic via knots (end)

\input{qm2pi.conclusion}

% section conclusion (end)

%\input{qm2pi.dtcodes} 

% section wiring algorithm (end)

\input{qm2pi.ack} 

% section acknowledgments (end)

\newpage


\bibliographystyle{plain}   
\bibliography{../../biblios/main.bib}

\input{qm2pi.rhodetails}

\end{document}

 

% section notation (end)

\input{qm2pi.process.calculi} 

% section concurrent_process_calculi_and_spatial_logics_ (end)
    
%\documentclass[12pt]{llncs}
%\documentclass{jktr}

\usepackage[pdftex]{hyperref}                   
\usepackage {listings}
\usepackage {mathpartir}
\usepackage{bcprules}
%\usepackage{listings}
                       
\usepackage{graphicx} 
%\usepackage[margins=2.5cm,nohead,nofoot]{geometry}
%\usepackage{geometry}
\usepackage{amsfonts}
\usepackage{amstext}
\usepackage{latexsym}
\usepackage{amssymb}
\usepackage{color}


%\include{myPreamble}
\include{qm2pi.local} 

%\ifpdf
%\usepackage[pdftex]{graphicx}
%\else
%\usepackage{graphicx}
%\fi

 % \ifpdf
%  \usepackage{pdfsync}
%  \if


%\title{Brief Article}
%\author{David F. Snyder}
%\author{L.G. Meredith}

%\address{Dept. of Math., Texas State University--San Marcos, San Marcos, TX 78666}
       
\pagestyle{empty}


\begin{document}

\lstset{language=[Objective]Caml,frame=shadowbox}

\input{qm2pi.front}

% section front matter (end)

\input{qm2pi.intro} 
 
% section introduction (end)

% \input{qm2pi.knotations} 

% section notation (end)

\input{qm2pi.process.calculi} 

% section concurrent_process_calculi_and_spatial_logics_ (end)
    
%\input{qm2pi.knots2pi} 

%\input{qm2pi.trefoil} 

%\input{qm2pi.mainthm} 

% subsection basic_interpretation (end)

%\input{qm2pi.rho.presentation} 
\subsection{The syntax and semantics of the notation system}\label{sub:the_syntax_and_semantics_of_the_notation_system} % (fold)

We now summarize a technical presentation of the calculus that
embodies our theory of dynamics. The typical presentation of such a
calculus follows the style of giving generators and relations on
them. The grammar, below, describing term constructors, freely
generates the set of processes, $\Proc$. This set is then quotiented
by a relation known as structural congruence and it is over this set
that the notion of dynamics is expressed. This presentation is
essentially that of \cite{MeredithR05} with the addition of
polyadicity and summation. For readability we have relegated some of
the technical subtleties to an appendix.

\subsubsection{Process grammar}\label{subsub:process_grammar}

\begin{mathpar}
  \inferrule* [lab=synchronization] {} {{M} \bc \pzero \;|\; x?F \;|\; x!C }
  \and
  \inferrule* [lab=abstraction] {} {{F} \bc (x)P}
  \and
  \inferrule* [lab=concretion] {} {{C} \bc \langle Q \rangle}
  \and
  \inferrule* [lab=process] {} {{P,Q} \bc M \;| \;P|Q \;|\; @{x}}
  \and
  \inferrule* [lab=name] {} {{x} \bc \quotep{P}}
\end{mathpar} 

Note that $\vec{x}$ (resp. $\vec{P}$) denotes a vector of names
(resp. processes) of length $|\vec{x}|$ (resp. $|\vec{P}|$). We adopt
the following useful abbreviations.

\begin{mathpar}
   x?(\vec{y}).P := x.(\vec{y})P \and  x\clift{\vec{P}} := x.\clift{\vec{P}}
   \and x!(y) := \lift{x}{\dropn{y}}
   \and \Pi_{i=0}^{n-1}P_i := P_0 | \ldots | P_{n-1}
\end{mathpar}

\subsubsection{Structural congruence}

\paragraph{Free and bound names and alpha-equivalence.} At the
core of structural equivalence is alpha-equivalence which identifies
process that are the same up to a change of variable. Formally, we
recognize the distinction between free and bound names. The free names
of a process, $\freenames{P}$, may be calculated recursively as
follows:

\begin{mathpar}
\freenames{\pzero} := \emptyset
  \and \\
  \freenames{x?(y).P} := \{ x \} \cup (\freenames{P} \setminus \{ y \})
  \and 
  \freenames{x!\langle P \rangle} := \{ x \} \cup \{ P \} 
  \and \\
  \freenames{P|Q} := \freenames{P} \cup \freenames{Q}
  \and \\
  \freenames{@{x}} := \{ x \}
\end{mathpar}

$\pi$
$\quotep{\pi}$

$\freenames{-} : \pi \to \mathcal{P}(\quotep{\pi})$

\begin{eqnarray*}
  \freenames{\pzero} & := & \emptyset \\
  \freenames{x?(y).P} & := & \{ x \} \cup (\freenames{P} \setminus \{ y \}) \\
  \freenames{x!\langle P \rangle} & := & \{ x \} \cup \{ P \} \\
  \freenames{P|Q} & := & \freenames{P} \cup \freenames{Q} \\
  \freenames{\dropn{x}} & := & \{ x \}
\end{eqnarray*}

The bound names of a process, $\boundnames{P}$, are those names occurring in $P$
that are not free. For example, in $x?(y).0$, the name $x$ is free, while $y$ is bound.

\begin{mathpar}
  \inferrule* [lab=monoidal-laws] {} { P|Q \equiv Q|P \and P|0 \equiv P \and P|(Q|R) \equiv (P|Q)|R }
\end{mathpar}

\begin{mathpar}
  \inferrule* [lab=alpha-equivalence] {} { (x)P \equiv (y)P\{y/x\} \and y \not\in \freenames{P} }
\end{mathpar}

\begin{definition}
Then two processes, $P,Q$, are alpha-equivalent if $P = Q\{\vec{y}/\vec{x}\}$ for
some $\vec{x} \in \boundnames{Q},\vec{y} \in \boundnames{P}$, where $Q\{\vec{y}/\vec{x}\}$
denotes the capture-avoiding substitution of $\vec{y}$ for $\vec{x}$ in $Q$.
\end{definition}

\begin{definition}
  The {\em structural congruence} \cite{SangiorgiWalker} , $\equiv$,
  between processes is the least congruence containing
  alpha-equivalence, satisfying the abelian monoid laws
  (associativity, commutativity and $\pzero$ as identity) for parallel
  composition $|$ and for summation $+$.
\end{definition}

\subsection{Name equivalence}

We take name equivalence, written $\nameeq$, to be the smallest
equivalence relation generated by the following rules.

\begin{mathpar}
\inferrule*[lab=Quote-drop]
{ }
{ \quotep{@{x}} \nameeq x }

\inferrule*[lab=Struct-equiv]
{ P \scong Q }
{ \quotep{P} \nameeq \quotep{Q} }
\end{mathpar}

The astute reader will have noticed that the mutual recursion of names
and processes imposes a mutual recursion on alpha-equivalence and
structural equivalence via name-equivalence. Fortunately, all of this
works out pleasantly and we may calculate in the natural way, free of
concern. The reader interested in the details is referred to the
appendix \ref{appendix:rho_details}.

\subsection{Substitution}

We use $\Proc$ for the set of processes, $\QProc$ for the set of
names, and $\id{\{}\vec{y} / \vec{x} \id{\}}$ to denote partial maps,
$s : \QProc \rightarrow \QProc$. A map, $s$ lifts, uniquely, to a map
on process terms, $\widehat{s} : \Proc \rightarrow \Proc$ by the
following equations.

\begin{mathpar}
  (0) \psubstp{Q}{P} := 0 \\
  (R \juxtap S) \psubstp{Q}{P}
  :=    
  (R)\psubstp{Q}{P} \juxtap (S) \psubstp{Q}{P} \\
  (x?(y).R) \psubstp{Q}{P}    
  :=    
  (x)\substp{Q}{P} (z)\concat( (R \psubstn{z}{y}) \psubstp{Q}{P} ) \\
  (\lift{x}{R}) \psubstp{Q}{P}  
  :=
  \lift{(x)\substp{Q}{P}}{ R \psubstp{Q}{P} } \\
%   (\dropn{x})  \psubstp{Q}{P}       
%   := 
%   \left\{ 
%     \begin{array}{ccc} 
%       \dropn{\quotep{Q}} & & x \nameeq \quotep{P} \\
%       \dropn{x} & & otherwise \\
%     \end{array}
%   \right. 
  (\dropn{x})  \psubstp{Q}{P}       
  := 
  \left\{ 
    \begin{array}{ccc} 
      Q & & x \nameeq \quotep{P} \\
      \dropn{x} & & otherwise \\
    \end{array}
  \right.
\end{mathpar}
 

where

\begin{eqnarray}
  (x)\id{\{} \lpquote Q \rpquote / \lpquote P \rpquote \id{\}}            = 
  \left\{ 
    \begin{array}{ccc}
      \lpquote Q \rpquote & & x \nameeq \lpquote P \rpquote \\
      x & & otherwise \\
    \end{array}
  \right. \nonumber
\end{eqnarray}

and $z$ is chosen distinct from $\quotep{P}$, $\quotep{Q}$, the free
names in $Q$, and all the names in $R$. Our $\alpha$-equivalence will
be built in the standard way from this substitution.

\begin{remark}\label{rem:no_self_referential_names}
  One consequence of these definitions is that $\forall P. \quotep{P}
  \not\in \freenames{P}$.
\end{remark}

\subsection{ Dynamic quote: an example }

Anticipating something of what's to come, consider applying the
substitution, $\widehat{\id{\{}u / z \id{\}}}$, to the following pair
of processes, $\lift{w}{y!(z)}$ and $w[ \lpquote y!(z) \rpquote ]$.

\begin{eqnarray}
	\lift{w}{y!(z)}\widehat{\id{\{}u / z \id{\}}}
		& = &
		\lift{w}{y!(u)} \nonumber\\
	w[ \lpquote y!(z) \rpquote ] \widehat{ \id{\{}u / z \id{\}} }
		& = &
		w[ \lpquote y!(z) \rpquote ] \nonumber
\end{eqnarray}

Because the body of the process between quotes is impervious to
substitution, we get radically different answers. In fact, by
examining the first process in an input context,
e.g. $x?(z).\lift{w}{y!(z)}$, we see that the process under the lift
operator may be shaped by prefixed inputs binding a name inside it. In
this sense, the lift operator will be seen as a way to dynamically
construct processes before reifying them as names.

Finally equipped with these standard features we can present the
dynamics of the calculus.

\subsubsection{Operational semantics} 

Finally, we introduce the computational dynamics. What marks these
algebras as distinct from other more traditionally studied algebraic
structures, e.g. vector spaces or polynomial rings, is the manner in
which dynamics is captured. In traditional structures, dynamics is typically
expressed through morphisms between such structures, as in linear maps
between vector spaces or morphisms between rings. In algebras
associated with the semantics of computation, the dynamics is
expressed as part of the algebraic structure itself, through a
reduction reduction relation typically denoted by $\red$. Below, we
give a recursive presentation of this relation for the calculus used
in the encoding.

$\red \subseteq \pi \times \pi$
$\red : \pi \to \mathcal{P}(\pi)$

\begin{mathpar}
  \inferrule* [lab=Comm] { \textsf{match}( x_{src}, x_{trgt} ) } { x_{trgt}?(y)P \; | \; x_{src}!\langle {Q} \rangle \red P\{\quotep{Q}/y}\} }
  \and \\
  \inferrule* [lab=Par] {{P} \red {P}'} {{{P} | {Q}} \red {{P}' | {Q}}}
  \and
  \inferrule* [lab=Equiv]{{{P} \scong {P}'} \andalso {{P}' \red {Q}'} \andalso {{Q}' \scong {Q}}}{{P} \red {Q}}
\end{mathpar}

\begin{eqnarray*}
  match_{\equiv} (\quotep{P},\quotep{Q}) & := & P \equiv Q \\
  match_{\dagger}(\quotep{P},\quotep{Q}) & := & \forall R. P|Q \red^{*} R => R \red^{*} 0 \\
  match_{K}(\quotep{P},\quotep{Q}) & := & K \mbox{ for some context } K
\end{eqnarray*}

$u?(x)P | u!\langle Q \rangle \red P\{\quotep{Q}/x\}$

%We write $\wred$ for $\red^*$, and $P\red$ if $\exists Q $ such that $ P \red Q$.
We write $P\red$ if $\exists Q $ such that $ P \red Q$ and $P\not\red$, otherwise.

\section{Replication}

As mentioned before, it is known that replication (and hence
recursion) can be implemented in a higher-order process algebra
\cite{SangiorgiWalker}. As our first example of calculation with the
machinery thus far presented we give the construction explicitly in
the {\rhoc}.

\begin{eqnarray}
	D_{x} & := & \prefix{x}{y}{(\binpar{\outputp{x}{y}}{@{y}})} \nonumber\\
	\bangp_{x}{P} & := & \binpar{{x}!\langle{\binpar{D_{x}}{P}}\rangle}{D_{x}} \nonumber
\end{eqnarray}

\begin{eqnarray}
	\bangp_{x}{P} & & \nonumber\\
	=
	& {x}!\langle{(\prefix{x}{y}{(\outputp{x}{y} | @{y})) | P}}\rangle 
	      | \prefix{x}{y}{(\outputp{x}{y} | @{y})} & \nonumber\\
	\red
	& (\outputp{x}{y} | @{y})\substn{\quotep{(\prefix{x}{y}{(@{y} | \outputp{x}{y})) | P}}}{y} & \nonumber\\
	=
	& \outputp{x}{\quotep{(\prefix{x}{y}{(\outputp{x}{y} | @{y})) | P}}}
	  | {(\prefix{x}{y}{(\outputp{x}{y} | @{y})) | P}} & \nonumber\\
	\red
	& \ldots & \nonumber\\
	\red^*
	& P | P | \ldots & \nonumber
\end{eqnarray}

Of course, this encoding, as an implementation, runs away, unfolding
$\bangp{P}$ eagerly. A lazier and more implementable replication
operator, restricted to input-guarded processes, may be obtained as follows.

\begin{eqnarray}
\bangp{\prefix{u}{v}{P}} 
	:= 
	\binpar{\lift{x}{\prefix{u}{v}{(\binpar{D(x)}{P})}}}{D(x)} \nonumber
\end{eqnarray}

\begin{remark}
  Note that the lazier definition still does not deal with summation
  or mixed summation (i.e. sums over input and output). The reader is
  invited to construct definitions of replication that deal with these
  features. 

  Further, the definitions are parameterized in a name, $x$. Can you,
  gentle reader, make a definition that eliminates this parameter and
  guarantees no accidental interaction between the replication
  machinery and the process being replicated -- i.e. no accidental
  sharing of names used by the process to get its work done and the
  name(s) used by the replication to effect copying. This latter
  revision of the definition of replication is crucial to obtaining
  the expected identity $!!P \sim !P$.
\end{remark}

\begin{remark}\label{rem:paradoxical_combinator}
  The reader familiar with the lambda calculus will have noticed the
  similarity between $D$ and the paradoxical combinator.

  [Ed. note: the existence of this seems to suggest we have to be more
  restrictive on the set of processes and names we admit if we are to
  support no-cloning.]
\end{remark}

\subsubsection{Bisimulation}

The computational dynamics gives rise to another kind of equivalence,
the equivalence of computational behavior. As previously mentioned
this is typically captured \emph{via} some form of bisimulation.

% The notion we use in this paper is weak barbed bisimulation
% \cite{milner91polyadicpi}.

The notion we use in this paper is derived from weak barbed
bisimulation \cite{milner91polyadicpi}. 

\begin{definition}
An \emph{observation relation}, $\downarrow_{\mathcal N}$, over a set
of names, $\mathcal N$, is the smallest relation satisfying the rules
below.

\infrule[Out-barb]{y \in {\mathcal N}, \; x \nameeq y}
		  {\outputp{x}{v} \downarrow_{\mathcal N} x}
\infrule[Par-barb]{\mbox{$P\downarrow_{\mathcal N} x$ or $Q\downarrow_{\mathcal N} x$}}
		  {\binpar{P}{Q} \downarrow_{\mathcal N} x}

We write $P \Downarrow_{\mathcal N} x$ if there is $Q$ such that 
$P \wred Q$ and $Q \downarrow_{\mathcal N} x$.
\end{definition}

\begin{definition}
%\label{def.bbisim}
An  ${\mathcal N}$-\emph{barbed bisimulation} over a set of names, ${\mathcal N}$, is a symmetric binary relation 
${\mathcal S}_{\mathcal N}$ between agents such that $P\rel{S}_{\mathcal N}Q$ implies:
\begin{enumerate}
\item If $P \red P'$ then $Q \wred Q'$ and $P'\rel{S}_{\mathcal N} Q'$.
\item If $P\downarrow_{\mathcal N} x$, then $Q\Downarrow_{\mathcal N} x$.
\end{enumerate}
$P$ is ${\mathcal N}$-barbed bisimilar to $Q$, written
$P \wbbisim_{\mathcal N} Q$, if $P \rel{S}_{\mathcal N} Q$ for some ${\mathcal N}$-barbed bisimulation ${\mathcal S}_{\mathcal N}$.
\end{definition}

$\mathcal{R} \subseteq \pi \times \pi$

$P \mathcal{R} Q => \forall P'. P \red P' \Rightarrow \exists Q'. Q \red Q', P' \mathcal{R} Q'$

$P \vdash x \Rightarrow Q \vdash x$

\begin{mathpar}
  \inferrule*[lab=Out-barb]{x \nameeq y}{{y}!\langle{Q}\rangle \vdash x}
  \and
  \inferrule*[lab=Par-barb]{\mbox{$P\vdash x$ or $Q\vdash x$}}{\binpar{P}{Q} \vdash x}
\end{mathpar}

\subsubsection{Contexts}

One of the principle advantages of computational calculi like the
$\pi$-calculus is a well-defined notion of context,
contextual-equivalence and a correlation between
contextual-equivalence and notions of bisimulation. The notion of
context allows the decomposition of a process into (sub-)process and
its syntactic environment, its context. Thus, a context may be
thought of as a process with a ``hole'' (written $\Box$) in it. The
application of a context $M$ to a process $P$, written $M[P]$, is
tantamount to filling the hole in $M$ with $P$. In this paper we do
not need the full weight of this theory, but do make use of the notion
of context in the proof the main theorem. 

\begin{mathpar}
  \inferrule* [lab=summation] {} {{M_{M},M_{N}} \bc \Box \;|\; x.M_{A} \;|\; M_{M}+M_{N}}
  \and
  \inferrule* [lab=agent] {} {{M_{A}} \bc (\vec{x})M_{P} \;| \; \clift{P_0,\ldots,M_{P},\ldots,P_N}}
  \and \\
  \inferrule* [lab=process] {} {{M_{P}} \bc M_{N} \;| \;P|M_{P} }
\end{mathpar} 

\begin{mathpar}
  \inferrule* [lab=sychronization] {} {M_{N} \bc \Box \;|\; x?M_{F} \;|\; x!M_{C}}
  \and
  \inferrule* [lab=abstraction] {} {{M_{F}} \bc (x)M_{P} }
  \and
  \inferrule* [lab=concretion] {} {{M_{C}} \bc \langle M_{P} \rangle }
  \and \\
  \inferrule* [lab=process] {} {{M_{P}} \bc M_{N} \;| \;P|M_{P} }
\end{mathpar}

\begin{definition}[contextual application] Given a context $M$, and
  process $P$, we define the \emph{contextual application}, $M[P] :=
  M\{P/\Box\}$. That is, the contextual application of M to P is the
  substitution of $P$ for $\Box$ in $M$.
\end{definition}

$\meaningof{-} : L \to \mathcal{P}(\pi)$

\begin{mathpar}
  \inferrule* [lab=collection] {} {\meaningof{true} = \pi, \and \meaningof{~E} = \pi \setminus \meaningof{E}, \and \meaningof{E_{1} \& E_{2}} = \meaningof{E_{1}} \cap \meaningof{E_{2}}}
\end{mathpar}

\begin{mathpar}
  \inferrule* [lab=structure] {} {\meaningof{0} = \{ P \in \pi | P \equiv 0 \}, \and \\ \meaningof{E_1 | E_2} = \{ P \in \pi | P \equiv P_{1} | P_{2}, P_{1} \in \meaningof{E_{1}}, P_{2} \in \meaningof{E_2}\} }
\end{mathpar}

\begin{mathpar}
 \inferrule* [lab=behavior] {} {\meaningof{\langle a?b \rangle E} = \{ P \in \pi | P \equiv Q | u?(y)P', \\ \and \\\\ \and \\ \;\;\; u \in \meaningof{a}, \forall z.P'\{z/y\} \in \meaningof{E\{z/b\}}\}, \and \\ \meaningof{a!E} = \{ P \in \pi | P \equiv Q | x!\langle P' \rangle, x \in \meaningof{a} P' \in \meaningof{E}\} }
\end{mathpar}

\begin{mathpar}
 \inferrule* [lab=nominal] {} {\meaningof{\quotep{E}} = \{ \quotep{P} \in \quotep{\pi} | P \in \meaningof{E} \}, \and \meaningof{\quotep{P}} = \{ \quotep{Q} \in \quotep{\pi} | P \equiv Q \} \and \\ \meaningof{@\quotep{E}} = \{ P \in \pi | P \equiv @x, x \in \meaningof{E} \}}
\end{mathpar}

\begin{eqnarray*}
  \\
  \meaningof{-} : TS \to ST
\end{eqnarray*}

\begin{eqnarray*}
  \\
  L : TS \to ST
\end{eqnarray*}

\begin{eqnarray*}
  \\
  P \models E \iff P \in \meaningof{E}
\end{eqnarray*}

\begin{eqnarray*}
  P \approx_{L} Q \iff \forall E \in L. P \models E \iff Q \models E
\end{eqnarray*}

\begin{eqnarray*}
  P \approx_{K} Q
\end{eqnarray*}

\begin{eqnarray*}
  P \approx Q
\end{eqnarray*}

$\approx_{K} = \approx = \approx_{L}$

\subsubsection{Contextual duality}

Note that contexts extend the quotation operation to a family of
operations from processes to names. Given a context, $M$, we can
define a \emph{nominal context}, $\quotep{M}$ by $\quotep{M}[P] :=
\quotep{M[P]}$. To foreshadow what is to come we observe that these
operations enjoy a duality with processes very much like the duality
between vectors and maps from vectors to scalars.

Further, because the calculus is essentially higher-order, we have a
correspondence between contexts and processes. More specifically,
given a name $x$ and a context $M$ we can construct $M^{*}_{x}$ such
that 

\begin{mathpar}
  M^{*}_{x} | \lift{x}{P} \red M[P]
\end{mathpar}

namely,

\begin{mathpar}
  M^{*}_{x} := x?(u).M[\dropn{u}]
\end{mathpar}

The dependence of $M^{*}_{x}$ on a name makes it an abstraction, 

\begin{mathpar}
  M^{*} := (x)x?(u).M[\dropn{u}]
\end{mathpar}

\subsection{Additional notation}

It will sometimes be convenient to denote the process a name
quotes. We already have the notation $x = \quotep{P}$, but it will be
convenient to introduce an alternate notation, $\procn{x}$, when we
want to emphasize the connection to the use of the name. Note that, by
virtue of name equivalence, $\quotep{\procn{x}} \nameeq x$; so, the
notation is consistent with previous definitions.

Further, because names have structure it is possible to effect
substitutions on the basis of that structure. This means we need to
upgrade our notation for substitutions, which we accomplish by
adapting comprehension notation. Thus,

\begin{mathpar}
  P\{ y / x : x \in S \}
\end{mathpar}

is interpreted to mean the process derived from P by replacing (in a
capture-avoiding manner) each occurrence of $x$ in $S$ by $y$. For example,

\begin{mathpar}
  P\{ \quotep{\procn{x}|\procn{x}} / x : x \in \freenames{P} \}
\end{mathpar}

will replace each (occurrence) of a free name $x$ in $P$ by
$\quotep{\procn{x}|\procn{x}}$.

Also, we will avail ourselves of the notation $x^{L}$ and $x^{R}$ to
denote injections of a name into disjoint copies of the name
space. There are numerous ways to accomplish this. One example can be
found in \cite{MeredithR05}. This notation overloads to vectors of
names: $\vec{x}^{\pi} := (x_{i}^{\pi} \; : \; 0 \leq i < |\vec{x}| )$ where $\pi \in \{L,R\}$.

We also use $P^{\Box} := P|\Box$.

In \cite{MeredithR05} an interpretation of the new operator is
given. It turns out that there are several possible interpretations
all enjoying the requisite algebraic properties of the operator (see
\cite{milner91polyadicpi}). We will therefore make liberal use of
$(\nu\; \vec{x})P$.

% subsection the_syntax_and_semantics_of_the_notation_system (end)   

\input{qm2pi.qmops} 

\input{qm2pi.sterngerlach} 

\input{qm2pi.metric} 

% section concurrent_process_calculi (end)

%\input{qm2pi.proofsketch}

% section proof sketch (end)

%\input{qm2pi.slviaknots} 

% section spatial logic via knots (end)

\input{qm2pi.conclusion}

% section conclusion (end)

%\input{qm2pi.dtcodes} 

% section wiring algorithm (end)

\input{qm2pi.ack} 

% section acknowledgments (end)

\newpage


\bibliographystyle{plain}   
\bibliography{../../biblios/main.bib}

\input{qm2pi.rhodetails}

\end{document}

 

%\documentclass[12pt]{llncs}
%\documentclass{jktr}

\usepackage[pdftex]{hyperref}                   
\usepackage {listings}
\usepackage {mathpartir}
\usepackage{bcprules}
%\usepackage{listings}
                       
\usepackage{graphicx} 
%\usepackage[margins=2.5cm,nohead,nofoot]{geometry}
%\usepackage{geometry}
\usepackage{amsfonts}
\usepackage{amstext}
\usepackage{latexsym}
\usepackage{amssymb}
\usepackage{color}


%\include{myPreamble}
\include{qm2pi.local} 

%\ifpdf
%\usepackage[pdftex]{graphicx}
%\else
%\usepackage{graphicx}
%\fi

 % \ifpdf
%  \usepackage{pdfsync}
%  \if


%\title{Brief Article}
%\author{David F. Snyder}
%\author{L.G. Meredith}

%\address{Dept. of Math., Texas State University--San Marcos, San Marcos, TX 78666}
       
\pagestyle{empty}


\begin{document}

\lstset{language=[Objective]Caml,frame=shadowbox}

\input{qm2pi.front}

% section front matter (end)

\input{qm2pi.intro} 
 
% section introduction (end)

% \input{qm2pi.knotations} 

% section notation (end)

\input{qm2pi.process.calculi} 

% section concurrent_process_calculi_and_spatial_logics_ (end)
    
%\input{qm2pi.knots2pi} 

%\input{qm2pi.trefoil} 

%\input{qm2pi.mainthm} 

% subsection basic_interpretation (end)

%\input{qm2pi.rho.presentation} 
\subsection{The syntax and semantics of the notation system}\label{sub:the_syntax_and_semantics_of_the_notation_system} % (fold)

We now summarize a technical presentation of the calculus that
embodies our theory of dynamics. The typical presentation of such a
calculus follows the style of giving generators and relations on
them. The grammar, below, describing term constructors, freely
generates the set of processes, $\Proc$. This set is then quotiented
by a relation known as structural congruence and it is over this set
that the notion of dynamics is expressed. This presentation is
essentially that of \cite{MeredithR05} with the addition of
polyadicity and summation. For readability we have relegated some of
the technical subtleties to an appendix.

\subsubsection{Process grammar}\label{subsub:process_grammar}

\begin{mathpar}
  \inferrule* [lab=synchronization] {} {{M} \bc \pzero \;|\; x?F \;|\; x!C }
  \and
  \inferrule* [lab=abstraction] {} {{F} \bc (x)P}
  \and
  \inferrule* [lab=concretion] {} {{C} \bc \langle Q \rangle}
  \and
  \inferrule* [lab=process] {} {{P,Q} \bc M \;| \;P|Q \;|\; @{x}}
  \and
  \inferrule* [lab=name] {} {{x} \bc \quotep{P}}
\end{mathpar} 

Note that $\vec{x}$ (resp. $\vec{P}$) denotes a vector of names
(resp. processes) of length $|\vec{x}|$ (resp. $|\vec{P}|$). We adopt
the following useful abbreviations.

\begin{mathpar}
   x?(\vec{y}).P := x.(\vec{y})P \and  x\clift{\vec{P}} := x.\clift{\vec{P}}
   \and x!(y) := \lift{x}{\dropn{y}}
   \and \Pi_{i=0}^{n-1}P_i := P_0 | \ldots | P_{n-1}
\end{mathpar}

\subsubsection{Structural congruence}

\paragraph{Free and bound names and alpha-equivalence.} At the
core of structural equivalence is alpha-equivalence which identifies
process that are the same up to a change of variable. Formally, we
recognize the distinction between free and bound names. The free names
of a process, $\freenames{P}$, may be calculated recursively as
follows:

\begin{mathpar}
\freenames{\pzero} := \emptyset
  \and \\
  \freenames{x?(y).P} := \{ x \} \cup (\freenames{P} \setminus \{ y \})
  \and 
  \freenames{x!\langle P \rangle} := \{ x \} \cup \{ P \} 
  \and \\
  \freenames{P|Q} := \freenames{P} \cup \freenames{Q}
  \and \\
  \freenames{@{x}} := \{ x \}
\end{mathpar}

$\pi$
$\quotep{\pi}$

$\freenames{-} : \pi \to \mathcal{P}(\quotep{\pi})$

\begin{eqnarray*}
  \freenames{\pzero} & := & \emptyset \\
  \freenames{x?(y).P} & := & \{ x \} \cup (\freenames{P} \setminus \{ y \}) \\
  \freenames{x!\langle P \rangle} & := & \{ x \} \cup \{ P \} \\
  \freenames{P|Q} & := & \freenames{P} \cup \freenames{Q} \\
  \freenames{\dropn{x}} & := & \{ x \}
\end{eqnarray*}

The bound names of a process, $\boundnames{P}$, are those names occurring in $P$
that are not free. For example, in $x?(y).0$, the name $x$ is free, while $y$ is bound.

\begin{mathpar}
  \inferrule* [lab=monoidal-laws] {} { P|Q \equiv Q|P \and P|0 \equiv P \and P|(Q|R) \equiv (P|Q)|R }
\end{mathpar}

\begin{mathpar}
  \inferrule* [lab=alpha-equivalence] {} { (x)P \equiv (y)P\{y/x\} \and y \not\in \freenames{P} }
\end{mathpar}

\begin{definition}
Then two processes, $P,Q$, are alpha-equivalent if $P = Q\{\vec{y}/\vec{x}\}$ for
some $\vec{x} \in \boundnames{Q},\vec{y} \in \boundnames{P}$, where $Q\{\vec{y}/\vec{x}\}$
denotes the capture-avoiding substitution of $\vec{y}$ for $\vec{x}$ in $Q$.
\end{definition}

\begin{definition}
  The {\em structural congruence} \cite{SangiorgiWalker} , $\equiv$,
  between processes is the least congruence containing
  alpha-equivalence, satisfying the abelian monoid laws
  (associativity, commutativity and $\pzero$ as identity) for parallel
  composition $|$ and for summation $+$.
\end{definition}

\subsection{Name equivalence}

We take name equivalence, written $\nameeq$, to be the smallest
equivalence relation generated by the following rules.

\begin{mathpar}
\inferrule*[lab=Quote-drop]
{ }
{ \quotep{@{x}} \nameeq x }

\inferrule*[lab=Struct-equiv]
{ P \scong Q }
{ \quotep{P} \nameeq \quotep{Q} }
\end{mathpar}

The astute reader will have noticed that the mutual recursion of names
and processes imposes a mutual recursion on alpha-equivalence and
structural equivalence via name-equivalence. Fortunately, all of this
works out pleasantly and we may calculate in the natural way, free of
concern. The reader interested in the details is referred to the
appendix \ref{appendix:rho_details}.

\subsection{Substitution}

We use $\Proc$ for the set of processes, $\QProc$ for the set of
names, and $\id{\{}\vec{y} / \vec{x} \id{\}}$ to denote partial maps,
$s : \QProc \rightarrow \QProc$. A map, $s$ lifts, uniquely, to a map
on process terms, $\widehat{s} : \Proc \rightarrow \Proc$ by the
following equations.

\begin{mathpar}
  (0) \psubstp{Q}{P} := 0 \\
  (R \juxtap S) \psubstp{Q}{P}
  :=    
  (R)\psubstp{Q}{P} \juxtap (S) \psubstp{Q}{P} \\
  (x?(y).R) \psubstp{Q}{P}    
  :=    
  (x)\substp{Q}{P} (z)\concat( (R \psubstn{z}{y}) \psubstp{Q}{P} ) \\
  (\lift{x}{R}) \psubstp{Q}{P}  
  :=
  \lift{(x)\substp{Q}{P}}{ R \psubstp{Q}{P} } \\
%   (\dropn{x})  \psubstp{Q}{P}       
%   := 
%   \left\{ 
%     \begin{array}{ccc} 
%       \dropn{\quotep{Q}} & & x \nameeq \quotep{P} \\
%       \dropn{x} & & otherwise \\
%     \end{array}
%   \right. 
  (\dropn{x})  \psubstp{Q}{P}       
  := 
  \left\{ 
    \begin{array}{ccc} 
      Q & & x \nameeq \quotep{P} \\
      \dropn{x} & & otherwise \\
    \end{array}
  \right.
\end{mathpar}
 

where

\begin{eqnarray}
  (x)\id{\{} \lpquote Q \rpquote / \lpquote P \rpquote \id{\}}            = 
  \left\{ 
    \begin{array}{ccc}
      \lpquote Q \rpquote & & x \nameeq \lpquote P \rpquote \\
      x & & otherwise \\
    \end{array}
  \right. \nonumber
\end{eqnarray}

and $z$ is chosen distinct from $\quotep{P}$, $\quotep{Q}$, the free
names in $Q$, and all the names in $R$. Our $\alpha$-equivalence will
be built in the standard way from this substitution.

\begin{remark}\label{rem:no_self_referential_names}
  One consequence of these definitions is that $\forall P. \quotep{P}
  \not\in \freenames{P}$.
\end{remark}

\subsection{ Dynamic quote: an example }

Anticipating something of what's to come, consider applying the
substitution, $\widehat{\id{\{}u / z \id{\}}}$, to the following pair
of processes, $\lift{w}{y!(z)}$ and $w[ \lpquote y!(z) \rpquote ]$.

\begin{eqnarray}
	\lift{w}{y!(z)}\widehat{\id{\{}u / z \id{\}}}
		& = &
		\lift{w}{y!(u)} \nonumber\\
	w[ \lpquote y!(z) \rpquote ] \widehat{ \id{\{}u / z \id{\}} }
		& = &
		w[ \lpquote y!(z) \rpquote ] \nonumber
\end{eqnarray}

Because the body of the process between quotes is impervious to
substitution, we get radically different answers. In fact, by
examining the first process in an input context,
e.g. $x?(z).\lift{w}{y!(z)}$, we see that the process under the lift
operator may be shaped by prefixed inputs binding a name inside it. In
this sense, the lift operator will be seen as a way to dynamically
construct processes before reifying them as names.

Finally equipped with these standard features we can present the
dynamics of the calculus.

\subsubsection{Operational semantics} 

Finally, we introduce the computational dynamics. What marks these
algebras as distinct from other more traditionally studied algebraic
structures, e.g. vector spaces or polynomial rings, is the manner in
which dynamics is captured. In traditional structures, dynamics is typically
expressed through morphisms between such structures, as in linear maps
between vector spaces or morphisms between rings. In algebras
associated with the semantics of computation, the dynamics is
expressed as part of the algebraic structure itself, through a
reduction reduction relation typically denoted by $\red$. Below, we
give a recursive presentation of this relation for the calculus used
in the encoding.

$\red \subseteq \pi \times \pi$
$\red : \pi \to \mathcal{P}(\pi)$

\begin{mathpar}
  \inferrule* [lab=Comm] { \textsf{match}( x_{src}, x_{trgt} ) } { x_{trgt}?(y)P \; | \; x_{src}!\langle {Q} \rangle \red P\{\quotep{Q}/y}\} }
  \and \\
  \inferrule* [lab=Par] {{P} \red {P}'} {{{P} | {Q}} \red {{P}' | {Q}}}
  \and
  \inferrule* [lab=Equiv]{{{P} \scong {P}'} \andalso {{P}' \red {Q}'} \andalso {{Q}' \scong {Q}}}{{P} \red {Q}}
\end{mathpar}

\begin{eqnarray*}
  match_{\equiv} (\quotep{P},\quotep{Q}) & := & P \equiv Q \\
  match_{\dagger}(\quotep{P},\quotep{Q}) & := & \forall R. P|Q \red^{*} R => R \red^{*} 0 \\
  match_{K}(\quotep{P},\quotep{Q}) & := & K \mbox{ for some context } K
\end{eqnarray*}

$u?(x)P | u!\langle Q \rangle \red P\{\quotep{Q}/x\}$

%We write $\wred$ for $\red^*$, and $P\red$ if $\exists Q $ such that $ P \red Q$.
We write $P\red$ if $\exists Q $ such that $ P \red Q$ and $P\not\red$, otherwise.

\section{Replication}

As mentioned before, it is known that replication (and hence
recursion) can be implemented in a higher-order process algebra
\cite{SangiorgiWalker}. As our first example of calculation with the
machinery thus far presented we give the construction explicitly in
the {\rhoc}.

\begin{eqnarray}
	D_{x} & := & \prefix{x}{y}{(\binpar{\outputp{x}{y}}{@{y}})} \nonumber\\
	\bangp_{x}{P} & := & \binpar{{x}!\langle{\binpar{D_{x}}{P}}\rangle}{D_{x}} \nonumber
\end{eqnarray}

\begin{eqnarray}
	\bangp_{x}{P} & & \nonumber\\
	=
	& {x}!\langle{(\prefix{x}{y}{(\outputp{x}{y} | @{y})) | P}}\rangle 
	      | \prefix{x}{y}{(\outputp{x}{y} | @{y})} & \nonumber\\
	\red
	& (\outputp{x}{y} | @{y})\substn{\quotep{(\prefix{x}{y}{(@{y} | \outputp{x}{y})) | P}}}{y} & \nonumber\\
	=
	& \outputp{x}{\quotep{(\prefix{x}{y}{(\outputp{x}{y} | @{y})) | P}}}
	  | {(\prefix{x}{y}{(\outputp{x}{y} | @{y})) | P}} & \nonumber\\
	\red
	& \ldots & \nonumber\\
	\red^*
	& P | P | \ldots & \nonumber
\end{eqnarray}

Of course, this encoding, as an implementation, runs away, unfolding
$\bangp{P}$ eagerly. A lazier and more implementable replication
operator, restricted to input-guarded processes, may be obtained as follows.

\begin{eqnarray}
\bangp{\prefix{u}{v}{P}} 
	:= 
	\binpar{\lift{x}{\prefix{u}{v}{(\binpar{D(x)}{P})}}}{D(x)} \nonumber
\end{eqnarray}

\begin{remark}
  Note that the lazier definition still does not deal with summation
  or mixed summation (i.e. sums over input and output). The reader is
  invited to construct definitions of replication that deal with these
  features. 

  Further, the definitions are parameterized in a name, $x$. Can you,
  gentle reader, make a definition that eliminates this parameter and
  guarantees no accidental interaction between the replication
  machinery and the process being replicated -- i.e. no accidental
  sharing of names used by the process to get its work done and the
  name(s) used by the replication to effect copying. This latter
  revision of the definition of replication is crucial to obtaining
  the expected identity $!!P \sim !P$.
\end{remark}

\begin{remark}\label{rem:paradoxical_combinator}
  The reader familiar with the lambda calculus will have noticed the
  similarity between $D$ and the paradoxical combinator.

  [Ed. note: the existence of this seems to suggest we have to be more
  restrictive on the set of processes and names we admit if we are to
  support no-cloning.]
\end{remark}

\subsubsection{Bisimulation}

The computational dynamics gives rise to another kind of equivalence,
the equivalence of computational behavior. As previously mentioned
this is typically captured \emph{via} some form of bisimulation.

% The notion we use in this paper is weak barbed bisimulation
% \cite{milner91polyadicpi}.

The notion we use in this paper is derived from weak barbed
bisimulation \cite{milner91polyadicpi}. 

\begin{definition}
An \emph{observation relation}, $\downarrow_{\mathcal N}$, over a set
of names, $\mathcal N$, is the smallest relation satisfying the rules
below.

\infrule[Out-barb]{y \in {\mathcal N}, \; x \nameeq y}
		  {\outputp{x}{v} \downarrow_{\mathcal N} x}
\infrule[Par-barb]{\mbox{$P\downarrow_{\mathcal N} x$ or $Q\downarrow_{\mathcal N} x$}}
		  {\binpar{P}{Q} \downarrow_{\mathcal N} x}

We write $P \Downarrow_{\mathcal N} x$ if there is $Q$ such that 
$P \wred Q$ and $Q \downarrow_{\mathcal N} x$.
\end{definition}

\begin{definition}
%\label{def.bbisim}
An  ${\mathcal N}$-\emph{barbed bisimulation} over a set of names, ${\mathcal N}$, is a symmetric binary relation 
${\mathcal S}_{\mathcal N}$ between agents such that $P\rel{S}_{\mathcal N}Q$ implies:
\begin{enumerate}
\item If $P \red P'$ then $Q \wred Q'$ and $P'\rel{S}_{\mathcal N} Q'$.
\item If $P\downarrow_{\mathcal N} x$, then $Q\Downarrow_{\mathcal N} x$.
\end{enumerate}
$P$ is ${\mathcal N}$-barbed bisimilar to $Q$, written
$P \wbbisim_{\mathcal N} Q$, if $P \rel{S}_{\mathcal N} Q$ for some ${\mathcal N}$-barbed bisimulation ${\mathcal S}_{\mathcal N}$.
\end{definition}

$\mathcal{R} \subseteq \pi \times \pi$

$P \mathcal{R} Q => \forall P'. P \red P' \Rightarrow \exists Q'. Q \red Q', P' \mathcal{R} Q'$

$P \vdash x \Rightarrow Q \vdash x$

\begin{mathpar}
  \inferrule*[lab=Out-barb]{x \nameeq y}{{y}!\langle{Q}\rangle \vdash x}
  \and
  \inferrule*[lab=Par-barb]{\mbox{$P\vdash x$ or $Q\vdash x$}}{\binpar{P}{Q} \vdash x}
\end{mathpar}

\subsubsection{Contexts}

One of the principle advantages of computational calculi like the
$\pi$-calculus is a well-defined notion of context,
contextual-equivalence and a correlation between
contextual-equivalence and notions of bisimulation. The notion of
context allows the decomposition of a process into (sub-)process and
its syntactic environment, its context. Thus, a context may be
thought of as a process with a ``hole'' (written $\Box$) in it. The
application of a context $M$ to a process $P$, written $M[P]$, is
tantamount to filling the hole in $M$ with $P$. In this paper we do
not need the full weight of this theory, but do make use of the notion
of context in the proof the main theorem. 

\begin{mathpar}
  \inferrule* [lab=summation] {} {{M_{M},M_{N}} \bc \Box \;|\; x.M_{A} \;|\; M_{M}+M_{N}}
  \and
  \inferrule* [lab=agent] {} {{M_{A}} \bc (\vec{x})M_{P} \;| \; \clift{P_0,\ldots,M_{P},\ldots,P_N}}
  \and \\
  \inferrule* [lab=process] {} {{M_{P}} \bc M_{N} \;| \;P|M_{P} }
\end{mathpar} 

\begin{mathpar}
  \inferrule* [lab=sychronization] {} {M_{N} \bc \Box \;|\; x?M_{F} \;|\; x!M_{C}}
  \and
  \inferrule* [lab=abstraction] {} {{M_{F}} \bc (x)M_{P} }
  \and
  \inferrule* [lab=concretion] {} {{M_{C}} \bc \langle M_{P} \rangle }
  \and \\
  \inferrule* [lab=process] {} {{M_{P}} \bc M_{N} \;| \;P|M_{P} }
\end{mathpar}

\begin{definition}[contextual application] Given a context $M$, and
  process $P$, we define the \emph{contextual application}, $M[P] :=
  M\{P/\Box\}$. That is, the contextual application of M to P is the
  substitution of $P$ for $\Box$ in $M$.
\end{definition}

$\meaningof{-} : L \to \mathcal{P}(\pi)$

\begin{mathpar}
  \inferrule* [lab=collection] {} {\meaningof{true} = \pi, \and \meaningof{~E} = \pi \setminus \meaningof{E}, \and \meaningof{E_{1} \& E_{2}} = \meaningof{E_{1}} \cap \meaningof{E_{2}}}
\end{mathpar}

\begin{mathpar}
  \inferrule* [lab=structure] {} {\meaningof{0} = \{ P \in \pi | P \equiv 0 \}, \and \\ \meaningof{E_1 | E_2} = \{ P \in \pi | P \equiv P_{1} | P_{2}, P_{1} \in \meaningof{E_{1}}, P_{2} \in \meaningof{E_2}\} }
\end{mathpar}

\begin{mathpar}
 \inferrule* [lab=behavior] {} {\meaningof{\langle a?b \rangle E} = \{ P \in \pi | P \equiv Q | u?(y)P', \\ \and \\\\ \and \\ \;\;\; u \in \meaningof{a}, \forall z.P'\{z/y\} \in \meaningof{E\{z/b\}}\}, \and \\ \meaningof{a!E} = \{ P \in \pi | P \equiv Q | x!\langle P' \rangle, x \in \meaningof{a} P' \in \meaningof{E}\} }
\end{mathpar}

\begin{mathpar}
 \inferrule* [lab=nominal] {} {\meaningof{\quotep{E}} = \{ \quotep{P} \in \quotep{\pi} | P \in \meaningof{E} \}, \and \meaningof{\quotep{P}} = \{ \quotep{Q} \in \quotep{\pi} | P \equiv Q \} \and \\ \meaningof{@\quotep{E}} = \{ P \in \pi | P \equiv @x, x \in \meaningof{E} \}}
\end{mathpar}

\begin{eqnarray*}
  \\
  \meaningof{-} : TS \to ST
\end{eqnarray*}

\begin{eqnarray*}
  \\
  L : TS \to ST
\end{eqnarray*}

\begin{eqnarray*}
  \\
  P \models E \iff P \in \meaningof{E}
\end{eqnarray*}

\begin{eqnarray*}
  P \approx_{L} Q \iff \forall E \in L. P \models E \iff Q \models E
\end{eqnarray*}

\begin{eqnarray*}
  P \approx_{K} Q
\end{eqnarray*}

\begin{eqnarray*}
  P \approx Q
\end{eqnarray*}

$\approx_{K} = \approx = \approx_{L}$

\subsubsection{Contextual duality}

Note that contexts extend the quotation operation to a family of
operations from processes to names. Given a context, $M$, we can
define a \emph{nominal context}, $\quotep{M}$ by $\quotep{M}[P] :=
\quotep{M[P]}$. To foreshadow what is to come we observe that these
operations enjoy a duality with processes very much like the duality
between vectors and maps from vectors to scalars.

Further, because the calculus is essentially higher-order, we have a
correspondence between contexts and processes. More specifically,
given a name $x$ and a context $M$ we can construct $M^{*}_{x}$ such
that 

\begin{mathpar}
  M^{*}_{x} | \lift{x}{P} \red M[P]
\end{mathpar}

namely,

\begin{mathpar}
  M^{*}_{x} := x?(u).M[\dropn{u}]
\end{mathpar}

The dependence of $M^{*}_{x}$ on a name makes it an abstraction, 

\begin{mathpar}
  M^{*} := (x)x?(u).M[\dropn{u}]
\end{mathpar}

\subsection{Additional notation}

It will sometimes be convenient to denote the process a name
quotes. We already have the notation $x = \quotep{P}$, but it will be
convenient to introduce an alternate notation, $\procn{x}$, when we
want to emphasize the connection to the use of the name. Note that, by
virtue of name equivalence, $\quotep{\procn{x}} \nameeq x$; so, the
notation is consistent with previous definitions.

Further, because names have structure it is possible to effect
substitutions on the basis of that structure. This means we need to
upgrade our notation for substitutions, which we accomplish by
adapting comprehension notation. Thus,

\begin{mathpar}
  P\{ y / x : x \in S \}
\end{mathpar}

is interpreted to mean the process derived from P by replacing (in a
capture-avoiding manner) each occurrence of $x$ in $S$ by $y$. For example,

\begin{mathpar}
  P\{ \quotep{\procn{x}|\procn{x}} / x : x \in \freenames{P} \}
\end{mathpar}

will replace each (occurrence) of a free name $x$ in $P$ by
$\quotep{\procn{x}|\procn{x}}$.

Also, we will avail ourselves of the notation $x^{L}$ and $x^{R}$ to
denote injections of a name into disjoint copies of the name
space. There are numerous ways to accomplish this. One example can be
found in \cite{MeredithR05}. This notation overloads to vectors of
names: $\vec{x}^{\pi} := (x_{i}^{\pi} \; : \; 0 \leq i < |\vec{x}| )$ where $\pi \in \{L,R\}$.

We also use $P^{\Box} := P|\Box$.

In \cite{MeredithR05} an interpretation of the new operator is
given. It turns out that there are several possible interpretations
all enjoying the requisite algebraic properties of the operator (see
\cite{milner91polyadicpi}). We will therefore make liberal use of
$(\nu\; \vec{x})P$.

% subsection the_syntax_and_semantics_of_the_notation_system (end)   

\input{qm2pi.qmops} 

\input{qm2pi.sterngerlach} 

\input{qm2pi.metric} 

% section concurrent_process_calculi (end)

%\input{qm2pi.proofsketch}

% section proof sketch (end)

%\input{qm2pi.slviaknots} 

% section spatial logic via knots (end)

\input{qm2pi.conclusion}

% section conclusion (end)

%\input{qm2pi.dtcodes} 

% section wiring algorithm (end)

\input{qm2pi.ack} 

% section acknowledgments (end)

\newpage


\bibliographystyle{plain}   
\bibliography{../../biblios/main.bib}

\input{qm2pi.rhodetails}

\end{document}

 

%\documentclass[12pt]{llncs}
%\documentclass{jktr}

\usepackage[pdftex]{hyperref}                   
\usepackage {listings}
\usepackage {mathpartir}
\usepackage{bcprules}
%\usepackage{listings}
                       
\usepackage{graphicx} 
%\usepackage[margins=2.5cm,nohead,nofoot]{geometry}
%\usepackage{geometry}
\usepackage{amsfonts}
\usepackage{amstext}
\usepackage{latexsym}
\usepackage{amssymb}
\usepackage{color}


%\include{myPreamble}
\include{qm2pi.local} 

%\ifpdf
%\usepackage[pdftex]{graphicx}
%\else
%\usepackage{graphicx}
%\fi

 % \ifpdf
%  \usepackage{pdfsync}
%  \if


%\title{Brief Article}
%\author{David F. Snyder}
%\author{L.G. Meredith}

%\address{Dept. of Math., Texas State University--San Marcos, San Marcos, TX 78666}
       
\pagestyle{empty}


\begin{document}

\lstset{language=[Objective]Caml,frame=shadowbox}

\input{qm2pi.front}

% section front matter (end)

\input{qm2pi.intro} 
 
% section introduction (end)

% \input{qm2pi.knotations} 

% section notation (end)

\input{qm2pi.process.calculi} 

% section concurrent_process_calculi_and_spatial_logics_ (end)
    
%\input{qm2pi.knots2pi} 

%\input{qm2pi.trefoil} 

%\input{qm2pi.mainthm} 

% subsection basic_interpretation (end)

%\input{qm2pi.rho.presentation} 
\subsection{The syntax and semantics of the notation system}\label{sub:the_syntax_and_semantics_of_the_notation_system} % (fold)

We now summarize a technical presentation of the calculus that
embodies our theory of dynamics. The typical presentation of such a
calculus follows the style of giving generators and relations on
them. The grammar, below, describing term constructors, freely
generates the set of processes, $\Proc$. This set is then quotiented
by a relation known as structural congruence and it is over this set
that the notion of dynamics is expressed. This presentation is
essentially that of \cite{MeredithR05} with the addition of
polyadicity and summation. For readability we have relegated some of
the technical subtleties to an appendix.

\subsubsection{Process grammar}\label{subsub:process_grammar}

\begin{mathpar}
  \inferrule* [lab=synchronization] {} {{M} \bc \pzero \;|\; x?F \;|\; x!C }
  \and
  \inferrule* [lab=abstraction] {} {{F} \bc (x)P}
  \and
  \inferrule* [lab=concretion] {} {{C} \bc \langle Q \rangle}
  \and
  \inferrule* [lab=process] {} {{P,Q} \bc M \;| \;P|Q \;|\; @{x}}
  \and
  \inferrule* [lab=name] {} {{x} \bc \quotep{P}}
\end{mathpar} 

Note that $\vec{x}$ (resp. $\vec{P}$) denotes a vector of names
(resp. processes) of length $|\vec{x}|$ (resp. $|\vec{P}|$). We adopt
the following useful abbreviations.

\begin{mathpar}
   x?(\vec{y}).P := x.(\vec{y})P \and  x\clift{\vec{P}} := x.\clift{\vec{P}}
   \and x!(y) := \lift{x}{\dropn{y}}
   \and \Pi_{i=0}^{n-1}P_i := P_0 | \ldots | P_{n-1}
\end{mathpar}

\subsubsection{Structural congruence}

\paragraph{Free and bound names and alpha-equivalence.} At the
core of structural equivalence is alpha-equivalence which identifies
process that are the same up to a change of variable. Formally, we
recognize the distinction between free and bound names. The free names
of a process, $\freenames{P}$, may be calculated recursively as
follows:

\begin{mathpar}
\freenames{\pzero} := \emptyset
  \and \\
  \freenames{x?(y).P} := \{ x \} \cup (\freenames{P} \setminus \{ y \})
  \and 
  \freenames{x!\langle P \rangle} := \{ x \} \cup \{ P \} 
  \and \\
  \freenames{P|Q} := \freenames{P} \cup \freenames{Q}
  \and \\
  \freenames{@{x}} := \{ x \}
\end{mathpar}

$\pi$
$\quotep{\pi}$

$\freenames{-} : \pi \to \mathcal{P}(\quotep{\pi})$

\begin{eqnarray*}
  \freenames{\pzero} & := & \emptyset \\
  \freenames{x?(y).P} & := & \{ x \} \cup (\freenames{P} \setminus \{ y \}) \\
  \freenames{x!\langle P \rangle} & := & \{ x \} \cup \{ P \} \\
  \freenames{P|Q} & := & \freenames{P} \cup \freenames{Q} \\
  \freenames{\dropn{x}} & := & \{ x \}
\end{eqnarray*}

The bound names of a process, $\boundnames{P}$, are those names occurring in $P$
that are not free. For example, in $x?(y).0$, the name $x$ is free, while $y$ is bound.

\begin{mathpar}
  \inferrule* [lab=monoidal-laws] {} { P|Q \equiv Q|P \and P|0 \equiv P \and P|(Q|R) \equiv (P|Q)|R }
\end{mathpar}

\begin{mathpar}
  \inferrule* [lab=alpha-equivalence] {} { (x)P \equiv (y)P\{y/x\} \and y \not\in \freenames{P} }
\end{mathpar}

\begin{definition}
Then two processes, $P,Q$, are alpha-equivalent if $P = Q\{\vec{y}/\vec{x}\}$ for
some $\vec{x} \in \boundnames{Q},\vec{y} \in \boundnames{P}$, where $Q\{\vec{y}/\vec{x}\}$
denotes the capture-avoiding substitution of $\vec{y}$ for $\vec{x}$ in $Q$.
\end{definition}

\begin{definition}
  The {\em structural congruence} \cite{SangiorgiWalker} , $\equiv$,
  between processes is the least congruence containing
  alpha-equivalence, satisfying the abelian monoid laws
  (associativity, commutativity and $\pzero$ as identity) for parallel
  composition $|$ and for summation $+$.
\end{definition}

\subsection{Name equivalence}

We take name equivalence, written $\nameeq$, to be the smallest
equivalence relation generated by the following rules.

\begin{mathpar}
\inferrule*[lab=Quote-drop]
{ }
{ \quotep{@{x}} \nameeq x }

\inferrule*[lab=Struct-equiv]
{ P \scong Q }
{ \quotep{P} \nameeq \quotep{Q} }
\end{mathpar}

The astute reader will have noticed that the mutual recursion of names
and processes imposes a mutual recursion on alpha-equivalence and
structural equivalence via name-equivalence. Fortunately, all of this
works out pleasantly and we may calculate in the natural way, free of
concern. The reader interested in the details is referred to the
appendix \ref{appendix:rho_details}.

\subsection{Substitution}

We use $\Proc$ for the set of processes, $\QProc$ for the set of
names, and $\id{\{}\vec{y} / \vec{x} \id{\}}$ to denote partial maps,
$s : \QProc \rightarrow \QProc$. A map, $s$ lifts, uniquely, to a map
on process terms, $\widehat{s} : \Proc \rightarrow \Proc$ by the
following equations.

\begin{mathpar}
  (0) \psubstp{Q}{P} := 0 \\
  (R \juxtap S) \psubstp{Q}{P}
  :=    
  (R)\psubstp{Q}{P} \juxtap (S) \psubstp{Q}{P} \\
  (x?(y).R) \psubstp{Q}{P}    
  :=    
  (x)\substp{Q}{P} (z)\concat( (R \psubstn{z}{y}) \psubstp{Q}{P} ) \\
  (\lift{x}{R}) \psubstp{Q}{P}  
  :=
  \lift{(x)\substp{Q}{P}}{ R \psubstp{Q}{P} } \\
%   (\dropn{x})  \psubstp{Q}{P}       
%   := 
%   \left\{ 
%     \begin{array}{ccc} 
%       \dropn{\quotep{Q}} & & x \nameeq \quotep{P} \\
%       \dropn{x} & & otherwise \\
%     \end{array}
%   \right. 
  (\dropn{x})  \psubstp{Q}{P}       
  := 
  \left\{ 
    \begin{array}{ccc} 
      Q & & x \nameeq \quotep{P} \\
      \dropn{x} & & otherwise \\
    \end{array}
  \right.
\end{mathpar}
 

where

\begin{eqnarray}
  (x)\id{\{} \lpquote Q \rpquote / \lpquote P \rpquote \id{\}}            = 
  \left\{ 
    \begin{array}{ccc}
      \lpquote Q \rpquote & & x \nameeq \lpquote P \rpquote \\
      x & & otherwise \\
    \end{array}
  \right. \nonumber
\end{eqnarray}

and $z$ is chosen distinct from $\quotep{P}$, $\quotep{Q}$, the free
names in $Q$, and all the names in $R$. Our $\alpha$-equivalence will
be built in the standard way from this substitution.

\begin{remark}\label{rem:no_self_referential_names}
  One consequence of these definitions is that $\forall P. \quotep{P}
  \not\in \freenames{P}$.
\end{remark}

\subsection{ Dynamic quote: an example }

Anticipating something of what's to come, consider applying the
substitution, $\widehat{\id{\{}u / z \id{\}}}$, to the following pair
of processes, $\lift{w}{y!(z)}$ and $w[ \lpquote y!(z) \rpquote ]$.

\begin{eqnarray}
	\lift{w}{y!(z)}\widehat{\id{\{}u / z \id{\}}}
		& = &
		\lift{w}{y!(u)} \nonumber\\
	w[ \lpquote y!(z) \rpquote ] \widehat{ \id{\{}u / z \id{\}} }
		& = &
		w[ \lpquote y!(z) \rpquote ] \nonumber
\end{eqnarray}

Because the body of the process between quotes is impervious to
substitution, we get radically different answers. In fact, by
examining the first process in an input context,
e.g. $x?(z).\lift{w}{y!(z)}$, we see that the process under the lift
operator may be shaped by prefixed inputs binding a name inside it. In
this sense, the lift operator will be seen as a way to dynamically
construct processes before reifying them as names.

Finally equipped with these standard features we can present the
dynamics of the calculus.

\subsubsection{Operational semantics} 

Finally, we introduce the computational dynamics. What marks these
algebras as distinct from other more traditionally studied algebraic
structures, e.g. vector spaces or polynomial rings, is the manner in
which dynamics is captured. In traditional structures, dynamics is typically
expressed through morphisms between such structures, as in linear maps
between vector spaces or morphisms between rings. In algebras
associated with the semantics of computation, the dynamics is
expressed as part of the algebraic structure itself, through a
reduction reduction relation typically denoted by $\red$. Below, we
give a recursive presentation of this relation for the calculus used
in the encoding.

$\red \subseteq \pi \times \pi$
$\red : \pi \to \mathcal{P}(\pi)$

\begin{mathpar}
  \inferrule* [lab=Comm] { \textsf{match}( x_{src}, x_{trgt} ) } { x_{trgt}?(y)P \; | \; x_{src}!\langle {Q} \rangle \red P\{\quotep{Q}/y}\} }
  \and \\
  \inferrule* [lab=Par] {{P} \red {P}'} {{{P} | {Q}} \red {{P}' | {Q}}}
  \and
  \inferrule* [lab=Equiv]{{{P} \scong {P}'} \andalso {{P}' \red {Q}'} \andalso {{Q}' \scong {Q}}}{{P} \red {Q}}
\end{mathpar}

\begin{eqnarray*}
  match_{\equiv} (\quotep{P},\quotep{Q}) & := & P \equiv Q \\
  match_{\dagger}(\quotep{P},\quotep{Q}) & := & \forall R. P|Q \red^{*} R => R \red^{*} 0 \\
  match_{K}(\quotep{P},\quotep{Q}) & := & K \mbox{ for some context } K
\end{eqnarray*}

$u?(x)P | u!\langle Q \rangle \red P\{\quotep{Q}/x\}$

%We write $\wred$ for $\red^*$, and $P\red$ if $\exists Q $ such that $ P \red Q$.
We write $P\red$ if $\exists Q $ such that $ P \red Q$ and $P\not\red$, otherwise.

\section{Replication}

As mentioned before, it is known that replication (and hence
recursion) can be implemented in a higher-order process algebra
\cite{SangiorgiWalker}. As our first example of calculation with the
machinery thus far presented we give the construction explicitly in
the {\rhoc}.

\begin{eqnarray}
	D_{x} & := & \prefix{x}{y}{(\binpar{\outputp{x}{y}}{@{y}})} \nonumber\\
	\bangp_{x}{P} & := & \binpar{{x}!\langle{\binpar{D_{x}}{P}}\rangle}{D_{x}} \nonumber
\end{eqnarray}

\begin{eqnarray}
	\bangp_{x}{P} & & \nonumber\\
	=
	& {x}!\langle{(\prefix{x}{y}{(\outputp{x}{y} | @{y})) | P}}\rangle 
	      | \prefix{x}{y}{(\outputp{x}{y} | @{y})} & \nonumber\\
	\red
	& (\outputp{x}{y} | @{y})\substn{\quotep{(\prefix{x}{y}{(@{y} | \outputp{x}{y})) | P}}}{y} & \nonumber\\
	=
	& \outputp{x}{\quotep{(\prefix{x}{y}{(\outputp{x}{y} | @{y})) | P}}}
	  | {(\prefix{x}{y}{(\outputp{x}{y} | @{y})) | P}} & \nonumber\\
	\red
	& \ldots & \nonumber\\
	\red^*
	& P | P | \ldots & \nonumber
\end{eqnarray}

Of course, this encoding, as an implementation, runs away, unfolding
$\bangp{P}$ eagerly. A lazier and more implementable replication
operator, restricted to input-guarded processes, may be obtained as follows.

\begin{eqnarray}
\bangp{\prefix{u}{v}{P}} 
	:= 
	\binpar{\lift{x}{\prefix{u}{v}{(\binpar{D(x)}{P})}}}{D(x)} \nonumber
\end{eqnarray}

\begin{remark}
  Note that the lazier definition still does not deal with summation
  or mixed summation (i.e. sums over input and output). The reader is
  invited to construct definitions of replication that deal with these
  features. 

  Further, the definitions are parameterized in a name, $x$. Can you,
  gentle reader, make a definition that eliminates this parameter and
  guarantees no accidental interaction between the replication
  machinery and the process being replicated -- i.e. no accidental
  sharing of names used by the process to get its work done and the
  name(s) used by the replication to effect copying. This latter
  revision of the definition of replication is crucial to obtaining
  the expected identity $!!P \sim !P$.
\end{remark}

\begin{remark}\label{rem:paradoxical_combinator}
  The reader familiar with the lambda calculus will have noticed the
  similarity between $D$ and the paradoxical combinator.

  [Ed. note: the existence of this seems to suggest we have to be more
  restrictive on the set of processes and names we admit if we are to
  support no-cloning.]
\end{remark}

\subsubsection{Bisimulation}

The computational dynamics gives rise to another kind of equivalence,
the equivalence of computational behavior. As previously mentioned
this is typically captured \emph{via} some form of bisimulation.

% The notion we use in this paper is weak barbed bisimulation
% \cite{milner91polyadicpi}.

The notion we use in this paper is derived from weak barbed
bisimulation \cite{milner91polyadicpi}. 

\begin{definition}
An \emph{observation relation}, $\downarrow_{\mathcal N}$, over a set
of names, $\mathcal N$, is the smallest relation satisfying the rules
below.

\infrule[Out-barb]{y \in {\mathcal N}, \; x \nameeq y}
		  {\outputp{x}{v} \downarrow_{\mathcal N} x}
\infrule[Par-barb]{\mbox{$P\downarrow_{\mathcal N} x$ or $Q\downarrow_{\mathcal N} x$}}
		  {\binpar{P}{Q} \downarrow_{\mathcal N} x}

We write $P \Downarrow_{\mathcal N} x$ if there is $Q$ such that 
$P \wred Q$ and $Q \downarrow_{\mathcal N} x$.
\end{definition}

\begin{definition}
%\label{def.bbisim}
An  ${\mathcal N}$-\emph{barbed bisimulation} over a set of names, ${\mathcal N}$, is a symmetric binary relation 
${\mathcal S}_{\mathcal N}$ between agents such that $P\rel{S}_{\mathcal N}Q$ implies:
\begin{enumerate}
\item If $P \red P'$ then $Q \wred Q'$ and $P'\rel{S}_{\mathcal N} Q'$.
\item If $P\downarrow_{\mathcal N} x$, then $Q\Downarrow_{\mathcal N} x$.
\end{enumerate}
$P$ is ${\mathcal N}$-barbed bisimilar to $Q$, written
$P \wbbisim_{\mathcal N} Q$, if $P \rel{S}_{\mathcal N} Q$ for some ${\mathcal N}$-barbed bisimulation ${\mathcal S}_{\mathcal N}$.
\end{definition}

$\mathcal{R} \subseteq \pi \times \pi$

$P \mathcal{R} Q => \forall P'. P \red P' \Rightarrow \exists Q'. Q \red Q', P' \mathcal{R} Q'$

$P \vdash x \Rightarrow Q \vdash x$

\begin{mathpar}
  \inferrule*[lab=Out-barb]{x \nameeq y}{{y}!\langle{Q}\rangle \vdash x}
  \and
  \inferrule*[lab=Par-barb]{\mbox{$P\vdash x$ or $Q\vdash x$}}{\binpar{P}{Q} \vdash x}
\end{mathpar}

\subsubsection{Contexts}

One of the principle advantages of computational calculi like the
$\pi$-calculus is a well-defined notion of context,
contextual-equivalence and a correlation between
contextual-equivalence and notions of bisimulation. The notion of
context allows the decomposition of a process into (sub-)process and
its syntactic environment, its context. Thus, a context may be
thought of as a process with a ``hole'' (written $\Box$) in it. The
application of a context $M$ to a process $P$, written $M[P]$, is
tantamount to filling the hole in $M$ with $P$. In this paper we do
not need the full weight of this theory, but do make use of the notion
of context in the proof the main theorem. 

\begin{mathpar}
  \inferrule* [lab=summation] {} {{M_{M},M_{N}} \bc \Box \;|\; x.M_{A} \;|\; M_{M}+M_{N}}
  \and
  \inferrule* [lab=agent] {} {{M_{A}} \bc (\vec{x})M_{P} \;| \; \clift{P_0,\ldots,M_{P},\ldots,P_N}}
  \and \\
  \inferrule* [lab=process] {} {{M_{P}} \bc M_{N} \;| \;P|M_{P} }
\end{mathpar} 

\begin{mathpar}
  \inferrule* [lab=sychronization] {} {M_{N} \bc \Box \;|\; x?M_{F} \;|\; x!M_{C}}
  \and
  \inferrule* [lab=abstraction] {} {{M_{F}} \bc (x)M_{P} }
  \and
  \inferrule* [lab=concretion] {} {{M_{C}} \bc \langle M_{P} \rangle }
  \and \\
  \inferrule* [lab=process] {} {{M_{P}} \bc M_{N} \;| \;P|M_{P} }
\end{mathpar}

\begin{definition}[contextual application] Given a context $M$, and
  process $P$, we define the \emph{contextual application}, $M[P] :=
  M\{P/\Box\}$. That is, the contextual application of M to P is the
  substitution of $P$ for $\Box$ in $M$.
\end{definition}

$\meaningof{-} : L \to \mathcal{P}(\pi)$

\begin{mathpar}
  \inferrule* [lab=collection] {} {\meaningof{true} = \pi, \and \meaningof{~E} = \pi \setminus \meaningof{E}, \and \meaningof{E_{1} \& E_{2}} = \meaningof{E_{1}} \cap \meaningof{E_{2}}}
\end{mathpar}

\begin{mathpar}
  \inferrule* [lab=structure] {} {\meaningof{0} = \{ P \in \pi | P \equiv 0 \}, \and \\ \meaningof{E_1 | E_2} = \{ P \in \pi | P \equiv P_{1} | P_{2}, P_{1} \in \meaningof{E_{1}}, P_{2} \in \meaningof{E_2}\} }
\end{mathpar}

\begin{mathpar}
 \inferrule* [lab=behavior] {} {\meaningof{\langle a?b \rangle E} = \{ P \in \pi | P \equiv Q | u?(y)P', \\ \and \\\\ \and \\ \;\;\; u \in \meaningof{a}, \forall z.P'\{z/y\} \in \meaningof{E\{z/b\}}\}, \and \\ \meaningof{a!E} = \{ P \in \pi | P \equiv Q | x!\langle P' \rangle, x \in \meaningof{a} P' \in \meaningof{E}\} }
\end{mathpar}

\begin{mathpar}
 \inferrule* [lab=nominal] {} {\meaningof{\quotep{E}} = \{ \quotep{P} \in \quotep{\pi} | P \in \meaningof{E} \}, \and \meaningof{\quotep{P}} = \{ \quotep{Q} \in \quotep{\pi} | P \equiv Q \} \and \\ \meaningof{@\quotep{E}} = \{ P \in \pi | P \equiv @x, x \in \meaningof{E} \}}
\end{mathpar}

\begin{eqnarray*}
  \\
  \meaningof{-} : TS \to ST
\end{eqnarray*}

\begin{eqnarray*}
  \\
  L : TS \to ST
\end{eqnarray*}

\begin{eqnarray*}
  \\
  P \models E \iff P \in \meaningof{E}
\end{eqnarray*}

\begin{eqnarray*}
  P \approx_{L} Q \iff \forall E \in L. P \models E \iff Q \models E
\end{eqnarray*}

\begin{eqnarray*}
  P \approx_{K} Q
\end{eqnarray*}

\begin{eqnarray*}
  P \approx Q
\end{eqnarray*}

$\approx_{K} = \approx = \approx_{L}$

\subsubsection{Contextual duality}

Note that contexts extend the quotation operation to a family of
operations from processes to names. Given a context, $M$, we can
define a \emph{nominal context}, $\quotep{M}$ by $\quotep{M}[P] :=
\quotep{M[P]}$. To foreshadow what is to come we observe that these
operations enjoy a duality with processes very much like the duality
between vectors and maps from vectors to scalars.

Further, because the calculus is essentially higher-order, we have a
correspondence between contexts and processes. More specifically,
given a name $x$ and a context $M$ we can construct $M^{*}_{x}$ such
that 

\begin{mathpar}
  M^{*}_{x} | \lift{x}{P} \red M[P]
\end{mathpar}

namely,

\begin{mathpar}
  M^{*}_{x} := x?(u).M[\dropn{u}]
\end{mathpar}

The dependence of $M^{*}_{x}$ on a name makes it an abstraction, 

\begin{mathpar}
  M^{*} := (x)x?(u).M[\dropn{u}]
\end{mathpar}

\subsection{Additional notation}

It will sometimes be convenient to denote the process a name
quotes. We already have the notation $x = \quotep{P}$, but it will be
convenient to introduce an alternate notation, $\procn{x}$, when we
want to emphasize the connection to the use of the name. Note that, by
virtue of name equivalence, $\quotep{\procn{x}} \nameeq x$; so, the
notation is consistent with previous definitions.

Further, because names have structure it is possible to effect
substitutions on the basis of that structure. This means we need to
upgrade our notation for substitutions, which we accomplish by
adapting comprehension notation. Thus,

\begin{mathpar}
  P\{ y / x : x \in S \}
\end{mathpar}

is interpreted to mean the process derived from P by replacing (in a
capture-avoiding manner) each occurrence of $x$ in $S$ by $y$. For example,

\begin{mathpar}
  P\{ \quotep{\procn{x}|\procn{x}} / x : x \in \freenames{P} \}
\end{mathpar}

will replace each (occurrence) of a free name $x$ in $P$ by
$\quotep{\procn{x}|\procn{x}}$.

Also, we will avail ourselves of the notation $x^{L}$ and $x^{R}$ to
denote injections of a name into disjoint copies of the name
space. There are numerous ways to accomplish this. One example can be
found in \cite{MeredithR05}. This notation overloads to vectors of
names: $\vec{x}^{\pi} := (x_{i}^{\pi} \; : \; 0 \leq i < |\vec{x}| )$ where $\pi \in \{L,R\}$.

We also use $P^{\Box} := P|\Box$.

In \cite{MeredithR05} an interpretation of the new operator is
given. It turns out that there are several possible interpretations
all enjoying the requisite algebraic properties of the operator (see
\cite{milner91polyadicpi}). We will therefore make liberal use of
$(\nu\; \vec{x})P$.

% subsection the_syntax_and_semantics_of_the_notation_system (end)   

\input{qm2pi.qmops} 

\input{qm2pi.sterngerlach} 

\input{qm2pi.metric} 

% section concurrent_process_calculi (end)

%\input{qm2pi.proofsketch}

% section proof sketch (end)

%\input{qm2pi.slviaknots} 

% section spatial logic via knots (end)

\input{qm2pi.conclusion}

% section conclusion (end)

%\input{qm2pi.dtcodes} 

% section wiring algorithm (end)

\input{qm2pi.ack} 

% section acknowledgments (end)

\newpage


\bibliographystyle{plain}   
\bibliography{../../biblios/main.bib}

\input{qm2pi.rhodetails}

\end{document}

 

% subsection basic_interpretation (end)

%\input{qm2pi.rho.presentation} 
\subsection{The syntax and semantics of the notation system}\label{sub:the_syntax_and_semantics_of_the_notation_system} % (fold)

We now summarize a technical presentation of the calculus that
embodies our theory of dynamics. The typical presentation of such a
calculus follows the style of giving generators and relations on
them. The grammar, below, describing term constructors, freely
generates the set of processes, $\Proc$. This set is then quotiented
by a relation known as structural congruence and it is over this set
that the notion of dynamics is expressed. This presentation is
essentially that of \cite{MeredithR05} with the addition of
polyadicity and summation. For readability we have relegated some of
the technical subtleties to an appendix.

\subsubsection{Process grammar}\label{subsub:process_grammar}

\begin{mathpar}
  \inferrule* [lab=synchronization] {} {{M} \bc \pzero \;|\; x?F \;|\; x!C }
  \and
  \inferrule* [lab=abstraction] {} {{F} \bc (x)P}
  \and
  \inferrule* [lab=concretion] {} {{C} \bc \langle Q \rangle}
  \and
  \inferrule* [lab=process] {} {{P,Q} \bc M \;| \;P|Q \;|\; @{x}}
  \and
  \inferrule* [lab=name] {} {{x} \bc \quotep{P}}
\end{mathpar} 

Note that $\vec{x}$ (resp. $\vec{P}$) denotes a vector of names
(resp. processes) of length $|\vec{x}|$ (resp. $|\vec{P}|$). We adopt
the following useful abbreviations.

\begin{mathpar}
   x?(\vec{y}).P := x.(\vec{y})P \and  x\clift{\vec{P}} := x.\clift{\vec{P}}
   \and x!(y) := \lift{x}{\dropn{y}}
   \and \Pi_{i=0}^{n-1}P_i := P_0 | \ldots | P_{n-1}
\end{mathpar}

\subsubsection{Structural congruence}

\paragraph{Free and bound names and alpha-equivalence.} At the
core of structural equivalence is alpha-equivalence which identifies
process that are the same up to a change of variable. Formally, we
recognize the distinction between free and bound names. The free names
of a process, $\freenames{P}$, may be calculated recursively as
follows:

\begin{mathpar}
\freenames{\pzero} := \emptyset
  \and \\
  \freenames{x?(y).P} := \{ x \} \cup (\freenames{P} \setminus \{ y \})
  \and 
  \freenames{x!\langle P \rangle} := \{ x \} \cup \{ P \} 
  \and \\
  \freenames{P|Q} := \freenames{P} \cup \freenames{Q}
  \and \\
  \freenames{@{x}} := \{ x \}
\end{mathpar}

$\pi$
$\quotep{\pi}$

$\freenames{-} : \pi \to \mathcal{P}(\quotep{\pi})$

\begin{eqnarray*}
  \freenames{\pzero} & := & \emptyset \\
  \freenames{x?(y).P} & := & \{ x \} \cup (\freenames{P} \setminus \{ y \}) \\
  \freenames{x!\langle P \rangle} & := & \{ x \} \cup \{ P \} \\
  \freenames{P|Q} & := & \freenames{P} \cup \freenames{Q} \\
  \freenames{\dropn{x}} & := & \{ x \}
\end{eqnarray*}

The bound names of a process, $\boundnames{P}$, are those names occurring in $P$
that are not free. For example, in $x?(y).0$, the name $x$ is free, while $y$ is bound.

\begin{mathpar}
  \inferrule* [lab=monoidal-laws] {} { P|Q \equiv Q|P \and P|0 \equiv P \and P|(Q|R) \equiv (P|Q)|R }
\end{mathpar}

\begin{mathpar}
  \inferrule* [lab=alpha-equivalence] {} { (x)P \equiv (y)P\{y/x\} \and y \not\in \freenames{P} }
\end{mathpar}

\begin{definition}
Then two processes, $P,Q$, are alpha-equivalent if $P = Q\{\vec{y}/\vec{x}\}$ for
some $\vec{x} \in \boundnames{Q},\vec{y} \in \boundnames{P}$, where $Q\{\vec{y}/\vec{x}\}$
denotes the capture-avoiding substitution of $\vec{y}$ for $\vec{x}$ in $Q$.
\end{definition}

\begin{definition}
  The {\em structural congruence} \cite{SangiorgiWalker} , $\equiv$,
  between processes is the least congruence containing
  alpha-equivalence, satisfying the abelian monoid laws
  (associativity, commutativity and $\pzero$ as identity) for parallel
  composition $|$ and for summation $+$.
\end{definition}

\subsection{Name equivalence}

We take name equivalence, written $\nameeq$, to be the smallest
equivalence relation generated by the following rules.

\begin{mathpar}
\inferrule*[lab=Quote-drop]
{ }
{ \quotep{@{x}} \nameeq x }

\inferrule*[lab=Struct-equiv]
{ P \scong Q }
{ \quotep{P} \nameeq \quotep{Q} }
\end{mathpar}

The astute reader will have noticed that the mutual recursion of names
and processes imposes a mutual recursion on alpha-equivalence and
structural equivalence via name-equivalence. Fortunately, all of this
works out pleasantly and we may calculate in the natural way, free of
concern. The reader interested in the details is referred to the
appendix \ref{appendix:rho_details}.

\subsection{Substitution}

We use $\Proc$ for the set of processes, $\QProc$ for the set of
names, and $\id{\{}\vec{y} / \vec{x} \id{\}}$ to denote partial maps,
$s : \QProc \rightarrow \QProc$. A map, $s$ lifts, uniquely, to a map
on process terms, $\widehat{s} : \Proc \rightarrow \Proc$ by the
following equations.

\begin{mathpar}
  (0) \psubstp{Q}{P} := 0 \\
  (R \juxtap S) \psubstp{Q}{P}
  :=    
  (R)\psubstp{Q}{P} \juxtap (S) \psubstp{Q}{P} \\
  (x?(y).R) \psubstp{Q}{P}    
  :=    
  (x)\substp{Q}{P} (z)\concat( (R \psubstn{z}{y}) \psubstp{Q}{P} ) \\
  (\lift{x}{R}) \psubstp{Q}{P}  
  :=
  \lift{(x)\substp{Q}{P}}{ R \psubstp{Q}{P} } \\
%   (\dropn{x})  \psubstp{Q}{P}       
%   := 
%   \left\{ 
%     \begin{array}{ccc} 
%       \dropn{\quotep{Q}} & & x \nameeq \quotep{P} \\
%       \dropn{x} & & otherwise \\
%     \end{array}
%   \right. 
  (\dropn{x})  \psubstp{Q}{P}       
  := 
  \left\{ 
    \begin{array}{ccc} 
      Q & & x \nameeq \quotep{P} \\
      \dropn{x} & & otherwise \\
    \end{array}
  \right.
\end{mathpar}
 

where

\begin{eqnarray}
  (x)\id{\{} \lpquote Q \rpquote / \lpquote P \rpquote \id{\}}            = 
  \left\{ 
    \begin{array}{ccc}
      \lpquote Q \rpquote & & x \nameeq \lpquote P \rpquote \\
      x & & otherwise \\
    \end{array}
  \right. \nonumber
\end{eqnarray}

and $z$ is chosen distinct from $\quotep{P}$, $\quotep{Q}$, the free
names in $Q$, and all the names in $R$. Our $\alpha$-equivalence will
be built in the standard way from this substitution.

\begin{remark}\label{rem:no_self_referential_names}
  One consequence of these definitions is that $\forall P. \quotep{P}
  \not\in \freenames{P}$.
\end{remark}

\subsection{ Dynamic quote: an example }

Anticipating something of what's to come, consider applying the
substitution, $\widehat{\id{\{}u / z \id{\}}}$, to the following pair
of processes, $\lift{w}{y!(z)}$ and $w[ \lpquote y!(z) \rpquote ]$.

\begin{eqnarray}
	\lift{w}{y!(z)}\widehat{\id{\{}u / z \id{\}}}
		& = &
		\lift{w}{y!(u)} \nonumber\\
	w[ \lpquote y!(z) \rpquote ] \widehat{ \id{\{}u / z \id{\}} }
		& = &
		w[ \lpquote y!(z) \rpquote ] \nonumber
\end{eqnarray}

Because the body of the process between quotes is impervious to
substitution, we get radically different answers. In fact, by
examining the first process in an input context,
e.g. $x?(z).\lift{w}{y!(z)}$, we see that the process under the lift
operator may be shaped by prefixed inputs binding a name inside it. In
this sense, the lift operator will be seen as a way to dynamically
construct processes before reifying them as names.

Finally equipped with these standard features we can present the
dynamics of the calculus.

\subsubsection{Operational semantics} 

Finally, we introduce the computational dynamics. What marks these
algebras as distinct from other more traditionally studied algebraic
structures, e.g. vector spaces or polynomial rings, is the manner in
which dynamics is captured. In traditional structures, dynamics is typically
expressed through morphisms between such structures, as in linear maps
between vector spaces or morphisms between rings. In algebras
associated with the semantics of computation, the dynamics is
expressed as part of the algebraic structure itself, through a
reduction reduction relation typically denoted by $\red$. Below, we
give a recursive presentation of this relation for the calculus used
in the encoding.

$\red \subseteq \pi \times \pi$
$\red : \pi \to \mathcal{P}(\pi)$

\begin{mathpar}
  \inferrule* [lab=Comm] { \textsf{match}( x_{src}, x_{trgt} ) } { x_{trgt}?(y)P \; | \; x_{src}!\langle {Q} \rangle \red P\{\quotep{Q}/y}\} }
  \and \\
  \inferrule* [lab=Par] {{P} \red {P}'} {{{P} | {Q}} \red {{P}' | {Q}}}
  \and
  \inferrule* [lab=Equiv]{{{P} \scong {P}'} \andalso {{P}' \red {Q}'} \andalso {{Q}' \scong {Q}}}{{P} \red {Q}}
\end{mathpar}

\begin{eqnarray*}
  match_{\equiv} (\quotep{P},\quotep{Q}) & := & P \equiv Q \\
  match_{\dagger}(\quotep{P},\quotep{Q}) & := & \forall R. P|Q \red^{*} R => R \red^{*} 0 \\
  match_{K}(\quotep{P},\quotep{Q}) & := & K \mbox{ for some context } K
\end{eqnarray*}

$u?(x)P | u!\langle Q \rangle \red P\{\quotep{Q}/x\}$

%We write $\wred$ for $\red^*$, and $P\red$ if $\exists Q $ such that $ P \red Q$.
We write $P\red$ if $\exists Q $ such that $ P \red Q$ and $P\not\red$, otherwise.

\section{Replication}

As mentioned before, it is known that replication (and hence
recursion) can be implemented in a higher-order process algebra
\cite{SangiorgiWalker}. As our first example of calculation with the
machinery thus far presented we give the construction explicitly in
the {\rhoc}.

\begin{eqnarray}
	D_{x} & := & \prefix{x}{y}{(\binpar{\outputp{x}{y}}{@{y}})} \nonumber\\
	\bangp_{x}{P} & := & \binpar{{x}!\langle{\binpar{D_{x}}{P}}\rangle}{D_{x}} \nonumber
\end{eqnarray}

\begin{eqnarray}
	\bangp_{x}{P} & & \nonumber\\
	=
	& {x}!\langle{(\prefix{x}{y}{(\outputp{x}{y} | @{y})) | P}}\rangle 
	      | \prefix{x}{y}{(\outputp{x}{y} | @{y})} & \nonumber\\
	\red
	& (\outputp{x}{y} | @{y})\substn{\quotep{(\prefix{x}{y}{(@{y} | \outputp{x}{y})) | P}}}{y} & \nonumber\\
	=
	& \outputp{x}{\quotep{(\prefix{x}{y}{(\outputp{x}{y} | @{y})) | P}}}
	  | {(\prefix{x}{y}{(\outputp{x}{y} | @{y})) | P}} & \nonumber\\
	\red
	& \ldots & \nonumber\\
	\red^*
	& P | P | \ldots & \nonumber
\end{eqnarray}

Of course, this encoding, as an implementation, runs away, unfolding
$\bangp{P}$ eagerly. A lazier and more implementable replication
operator, restricted to input-guarded processes, may be obtained as follows.

\begin{eqnarray}
\bangp{\prefix{u}{v}{P}} 
	:= 
	\binpar{\lift{x}{\prefix{u}{v}{(\binpar{D(x)}{P})}}}{D(x)} \nonumber
\end{eqnarray}

\begin{remark}
  Note that the lazier definition still does not deal with summation
  or mixed summation (i.e. sums over input and output). The reader is
  invited to construct definitions of replication that deal with these
  features. 

  Further, the definitions are parameterized in a name, $x$. Can you,
  gentle reader, make a definition that eliminates this parameter and
  guarantees no accidental interaction between the replication
  machinery and the process being replicated -- i.e. no accidental
  sharing of names used by the process to get its work done and the
  name(s) used by the replication to effect copying. This latter
  revision of the definition of replication is crucial to obtaining
  the expected identity $!!P \sim !P$.
\end{remark}

\begin{remark}\label{rem:paradoxical_combinator}
  The reader familiar with the lambda calculus will have noticed the
  similarity between $D$ and the paradoxical combinator.

  [Ed. note: the existence of this seems to suggest we have to be more
  restrictive on the set of processes and names we admit if we are to
  support no-cloning.]
\end{remark}

\subsubsection{Bisimulation}

The computational dynamics gives rise to another kind of equivalence,
the equivalence of computational behavior. As previously mentioned
this is typically captured \emph{via} some form of bisimulation.

% The notion we use in this paper is weak barbed bisimulation
% \cite{milner91polyadicpi}.

The notion we use in this paper is derived from weak barbed
bisimulation \cite{milner91polyadicpi}. 

\begin{definition}
An \emph{observation relation}, $\downarrow_{\mathcal N}$, over a set
of names, $\mathcal N$, is the smallest relation satisfying the rules
below.

\infrule[Out-barb]{y \in {\mathcal N}, \; x \nameeq y}
		  {\outputp{x}{v} \downarrow_{\mathcal N} x}
\infrule[Par-barb]{\mbox{$P\downarrow_{\mathcal N} x$ or $Q\downarrow_{\mathcal N} x$}}
		  {\binpar{P}{Q} \downarrow_{\mathcal N} x}

We write $P \Downarrow_{\mathcal N} x$ if there is $Q$ such that 
$P \wred Q$ and $Q \downarrow_{\mathcal N} x$.
\end{definition}

\begin{definition}
%\label{def.bbisim}
An  ${\mathcal N}$-\emph{barbed bisimulation} over a set of names, ${\mathcal N}$, is a symmetric binary relation 
${\mathcal S}_{\mathcal N}$ between agents such that $P\rel{S}_{\mathcal N}Q$ implies:
\begin{enumerate}
\item If $P \red P'$ then $Q \wred Q'$ and $P'\rel{S}_{\mathcal N} Q'$.
\item If $P\downarrow_{\mathcal N} x$, then $Q\Downarrow_{\mathcal N} x$.
\end{enumerate}
$P$ is ${\mathcal N}$-barbed bisimilar to $Q$, written
$P \wbbisim_{\mathcal N} Q$, if $P \rel{S}_{\mathcal N} Q$ for some ${\mathcal N}$-barbed bisimulation ${\mathcal S}_{\mathcal N}$.
\end{definition}

$\mathcal{R} \subseteq \pi \times \pi$

$P \mathcal{R} Q => \forall P'. P \red P' \Rightarrow \exists Q'. Q \red Q', P' \mathcal{R} Q'$

$P \vdash x \Rightarrow Q \vdash x$

\begin{mathpar}
  \inferrule*[lab=Out-barb]{x \nameeq y}{{y}!\langle{Q}\rangle \vdash x}
  \and
  \inferrule*[lab=Par-barb]{\mbox{$P\vdash x$ or $Q\vdash x$}}{\binpar{P}{Q} \vdash x}
\end{mathpar}

\subsubsection{Contexts}

One of the principle advantages of computational calculi like the
$\pi$-calculus is a well-defined notion of context,
contextual-equivalence and a correlation between
contextual-equivalence and notions of bisimulation. The notion of
context allows the decomposition of a process into (sub-)process and
its syntactic environment, its context. Thus, a context may be
thought of as a process with a ``hole'' (written $\Box$) in it. The
application of a context $M$ to a process $P$, written $M[P]$, is
tantamount to filling the hole in $M$ with $P$. In this paper we do
not need the full weight of this theory, but do make use of the notion
of context in the proof the main theorem. 

\begin{mathpar}
  \inferrule* [lab=summation] {} {{M_{M},M_{N}} \bc \Box \;|\; x.M_{A} \;|\; M_{M}+M_{N}}
  \and
  \inferrule* [lab=agent] {} {{M_{A}} \bc (\vec{x})M_{P} \;| \; \clift{P_0,\ldots,M_{P},\ldots,P_N}}
  \and \\
  \inferrule* [lab=process] {} {{M_{P}} \bc M_{N} \;| \;P|M_{P} }
\end{mathpar} 

\begin{mathpar}
  \inferrule* [lab=sychronization] {} {M_{N} \bc \Box \;|\; x?M_{F} \;|\; x!M_{C}}
  \and
  \inferrule* [lab=abstraction] {} {{M_{F}} \bc (x)M_{P} }
  \and
  \inferrule* [lab=concretion] {} {{M_{C}} \bc \langle M_{P} \rangle }
  \and \\
  \inferrule* [lab=process] {} {{M_{P}} \bc M_{N} \;| \;P|M_{P} }
\end{mathpar}

\begin{definition}[contextual application] Given a context $M$, and
  process $P$, we define the \emph{contextual application}, $M[P] :=
  M\{P/\Box\}$. That is, the contextual application of M to P is the
  substitution of $P$ for $\Box$ in $M$.
\end{definition}

$\meaningof{-} : L \to \mathcal{P}(\pi)$

\begin{mathpar}
  \inferrule* [lab=collection] {} {\meaningof{true} = \pi, \and \meaningof{~E} = \pi \setminus \meaningof{E}, \and \meaningof{E_{1} \& E_{2}} = \meaningof{E_{1}} \cap \meaningof{E_{2}}}
\end{mathpar}

\begin{mathpar}
  \inferrule* [lab=structure] {} {\meaningof{0} = \{ P \in \pi | P \equiv 0 \}, \and \\ \meaningof{E_1 | E_2} = \{ P \in \pi | P \equiv P_{1} | P_{2}, P_{1} \in \meaningof{E_{1}}, P_{2} \in \meaningof{E_2}\} }
\end{mathpar}

\begin{mathpar}
 \inferrule* [lab=behavior] {} {\meaningof{\langle a?b \rangle E} = \{ P \in \pi | P \equiv Q | u?(y)P', \\ \and \\\\ \and \\ \;\;\; u \in \meaningof{a}, \forall z.P'\{z/y\} \in \meaningof{E\{z/b\}}\}, \and \\ \meaningof{a!E} = \{ P \in \pi | P \equiv Q | x!\langle P' \rangle, x \in \meaningof{a} P' \in \meaningof{E}\} }
\end{mathpar}

\begin{mathpar}
 \inferrule* [lab=nominal] {} {\meaningof{\quotep{E}} = \{ \quotep{P} \in \quotep{\pi} | P \in \meaningof{E} \}, \and \meaningof{\quotep{P}} = \{ \quotep{Q} \in \quotep{\pi} | P \equiv Q \} \and \\ \meaningof{@\quotep{E}} = \{ P \in \pi | P \equiv @x, x \in \meaningof{E} \}}
\end{mathpar}

\begin{eqnarray*}
  \\
  \meaningof{-} : TS \to ST
\end{eqnarray*}

\begin{eqnarray*}
  \\
  L : TS \to ST
\end{eqnarray*}

\begin{eqnarray*}
  \\
  P \models E \iff P \in \meaningof{E}
\end{eqnarray*}

\begin{eqnarray*}
  P \approx_{L} Q \iff \forall E \in L. P \models E \iff Q \models E
\end{eqnarray*}

\begin{eqnarray*}
  P \approx_{K} Q
\end{eqnarray*}

\begin{eqnarray*}
  P \approx Q
\end{eqnarray*}

$\approx_{K} = \approx = \approx_{L}$

\subsubsection{Contextual duality}

Note that contexts extend the quotation operation to a family of
operations from processes to names. Given a context, $M$, we can
define a \emph{nominal context}, $\quotep{M}$ by $\quotep{M}[P] :=
\quotep{M[P]}$. To foreshadow what is to come we observe that these
operations enjoy a duality with processes very much like the duality
between vectors and maps from vectors to scalars.

Further, because the calculus is essentially higher-order, we have a
correspondence between contexts and processes. More specifically,
given a name $x$ and a context $M$ we can construct $M^{*}_{x}$ such
that 

\begin{mathpar}
  M^{*}_{x} | \lift{x}{P} \red M[P]
\end{mathpar}

namely,

\begin{mathpar}
  M^{*}_{x} := x?(u).M[\dropn{u}]
\end{mathpar}

The dependence of $M^{*}_{x}$ on a name makes it an abstraction, 

\begin{mathpar}
  M^{*} := (x)x?(u).M[\dropn{u}]
\end{mathpar}

\subsection{Additional notation}

It will sometimes be convenient to denote the process a name
quotes. We already have the notation $x = \quotep{P}$, but it will be
convenient to introduce an alternate notation, $\procn{x}$, when we
want to emphasize the connection to the use of the name. Note that, by
virtue of name equivalence, $\quotep{\procn{x}} \nameeq x$; so, the
notation is consistent with previous definitions.

Further, because names have structure it is possible to effect
substitutions on the basis of that structure. This means we need to
upgrade our notation for substitutions, which we accomplish by
adapting comprehension notation. Thus,

\begin{mathpar}
  P\{ y / x : x \in S \}
\end{mathpar}

is interpreted to mean the process derived from P by replacing (in a
capture-avoiding manner) each occurrence of $x$ in $S$ by $y$. For example,

\begin{mathpar}
  P\{ \quotep{\procn{x}|\procn{x}} / x : x \in \freenames{P} \}
\end{mathpar}

will replace each (occurrence) of a free name $x$ in $P$ by
$\quotep{\procn{x}|\procn{x}}$.

Also, we will avail ourselves of the notation $x^{L}$ and $x^{R}$ to
denote injections of a name into disjoint copies of the name
space. There are numerous ways to accomplish this. One example can be
found in \cite{MeredithR05}. This notation overloads to vectors of
names: $\vec{x}^{\pi} := (x_{i}^{\pi} \; : \; 0 \leq i < |\vec{x}| )$ where $\pi \in \{L,R\}$.

We also use $P^{\Box} := P|\Box$.

In \cite{MeredithR05} an interpretation of the new operator is
given. It turns out that there are several possible interpretations
all enjoying the requisite algebraic properties of the operator (see
\cite{milner91polyadicpi}). We will therefore make liberal use of
$(\nu\; \vec{x})P$.

% subsection the_syntax_and_semantics_of_the_notation_system (end)   

\section{Interpretation of QM}
\subsection{Supporting definitions}
\subsubsection{Multiplication}
\begin{mathpar}
  \quotep{Q} \cdot \quotep{R} := \quotep{Q|R}
  \and \\
  \quotep{Q} \cdot P := P\{ \quotep{Q|R} / \quotep{R} : \quotep{R} \in \freenames{P} \}
\end{mathpar}

\paragraph{Discussion}
The first line needs little explanation. The second line says that
each free name of the process is replaced with the multiplication of
that name by the scalar. Multiplication of a scalar (name) by a state
(process) results in a process all the names of which have been `moved
over' by parallel composition with the process the scalar
quotes. There is a subtlety that the bound names have to be
manipulated so that multiplied names aren't accidentally
captured. There are many ways to achieve this.

\begin{remark}\label{rem:multiplication_identities}
  The reader is invited to verify that for all $x,y,z \in \QProc$ and $P \in \Proc$
  \begin{mathpar}
    x \cdot \quotep{0} \equiv x 
    \and
    x \cdot y \equiv y \cdot x
    \and
    x \cdot (y \cdot z) \equiv (x \cdot y) \cdot z
    \and \\
    \quotep{0} \cdot P \equiv P
    \and \\
    x \cdot (y \cdot P) \equiv (x \cdot y) \cdot P
    \and \\
    x \cdot (P|Q) \equiv (x \cdot P) | (x \cdot Q)
    \and \\    
  \end{mathpar}
\end{remark}

\subsubsection{Tensor product}

We define a tensor product on processes by structural induction.

\paragraph{Tensor of sums} First note that all summations, including
$\pzero$ and sequence, can be written $\Sigma_{i} x_{i}.A_{i} +
\Sigma_{j} x_{j}.C_{j}$, where we have grouped input-guarded processes
together and output-guarded processes together.

Thus, we can define the tensor product of two summations, $N_{1}\otimes N_{2}$, where

\begin{mathpar}
  N_{1} := \Sigma_{i} x_{i}.A_{i} + \Sigma_{j} x_{j}.C_{j}
  \and
  N_{2} := \Sigma_{i'} y_{i'}.B_{i'} + \Sigma_{j'} y_{j'}.D_{j'} 
\end{mathpar}

as follows.

\begin{mathpar}
  \Sigma_{i} x_{i}.A_{i} + \Sigma_{j} x_{j}.C_{j} \otimes \Sigma_{i'}
  y_{i'}.B_{i'} + \Sigma_{j'} y_{j'}.D_{j'} 
  \and \\
  := \; \Sigma_{i} \Sigma_{i'} \quotep{\stackrel{\vee}{x_{i}}| \stackrel{\vee}{y_{i'}}}.(A_{i}\otimes B_{i'}) \; | \; \Sigma_{i'} \Sigma_{i} \quotep{\stackrel{\vee}{y_{i'}}|\stackrel{\vee}{x_{i}}}.(B_{i'}\otimes A_{i})
  \and
  \;\; | \;\; \Sigma_{j} \Sigma_{j'} \quotep{\stackrel{\vee}{x_{j}}|\stackrel{\vee}{y_{j'}}}.(A_{j}\otimes B_{j'}) \; | \; \Sigma_{j'} \Sigma_{j} \quotep{\stackrel{\vee}{y_{j'}}|\stackrel{\vee}{x_{j}}}.(B_{j'}\otimes A_{j})
\end{mathpar}

\begin{remark}
  Do we need to $x^{L}$ and $y^{R}$ for this construction as well?
\end{remark}

\paragraph{Tensor of parallel compositions} Next, we distribute tensor
over par.

\begin{mathpar}
  P_{1}|P_{2} \otimes Q_{1}|Q_{2} := (P_{1} \otimes Q_{1}) | (P_{1}
  \otimes Q_{2}) | (P_{2} \otimes Q_{1}) | (P_{2} \otimes Q_{2})
\end{mathpar}

\paragraph{Tensor with dropped names} We treat tensor of a
process with a dropped name as parallel composition.

\begin{mathpar}
  P \otimes \dropn{x} := P | \dropn{x}
\end{mathpar}

\paragraph{Tensor of agents}

Finally, we need to define tensor on agents. Note that the definition
of tensor on normal products only tensors inputs with inputs and
outputs with outputs. Thus, we only have to define the operation on
``homogeneous'' pairings.

\begin{mathpar}
  (\vec{x})P \otimes (\vec{y})Q
  \and \\
  := (x_{0}^{L}|y_{0}^{R},\ldots,x_{0}^{L}|y_{n}^{R},\ldots,x_{m}^{L}|y_{0}^{R},\ldots,x_{m}^{L}|y_{n}^R)(P\{ \vec{x}^{L}/\vec{x}\} \otimes Q \{ \vec{y}^{R}/\vec{y}\})
  \and \\
  \clift{\vec{P}} \otimes \clift{\vec{Q}}
  \and \\
  := \clift{P_{0}\otimes Q_{0},\ldots,P_{0}\otimes Q_{n},\ldots,P_{m}\otimes Q_{0},\ldots,P_{m}\otimes Q_{n}}
\end{mathpar}

\begin{remark}
  Observe that arities of tensored abstractions matches arities of
  tensored concretions if the original arities matched. Note also that
  the length of the arities corresponds to the increase in dimension
  we see in ordinary vector space tensor product.
\end{remark}

\begin{remark}
  Operationally, this definition distributes the tensor down to
  components ``linked'' by summation. Tensor over summation is
  intriguing in that it mixes names. Moreover, as a consequence of the
  way it mixes names we have the identities for all $x \in \QProc$ and
  $P,Q \in \Proc$

  \begin{mathpar}
    (x \cdot P) \otimes Q \equiv x \cdot (P \otimes Q) \equiv P \otimes (x \cdot Q)
    \and
    P \otimes \pzero \equiv P
  \end{mathpar}

  that the reader is invited to verify.
\end{remark}

\subsubsection{Annihilation}
\begin{mathpar}
  P^{\perp} := \{ Q | \forall R. P|Q \red^{*} R \Rightarrow R \red^{*} \pzero \}
  \and \\
  P^{\underline{\perp}} := \Sigma_{Q \in P^{\perp}} \quotep{Q}?(y).(\dropn{y}|Q) | \Sigma_{Q \in P^{\perp}} \quotep{Q}\clift{\Box}
\end{mathpar}

\paragraph{Discussion} The reader will note that $P^{\perp}$ is a
\emph{set} of processes, while $P^{\underline{\perp}}$ is a
\emph{context}. We call the set $P^{\perp}$ the \emph{annihilators} of
$P$. The parallel composition of a process in the annihilators of $P$
with $P$ will result in a process, the state space of which has all
paths eventually leading to $\pzero$. Execution may endure loops; but
under reasonable conditions of fairness (naturally guaranteed under
most notions of bisimulation) such a composite process cannot get
stuck in such a loop and will, eventually pop out and terminate.

The context $P^{\underline{\perp}}$ is ready and willing to ``take the
$P$ out of'' the process to which it is applied. It will effectively
transmit the code of the process to which it is applied to one of the
annihilators and run the process against it.

\subsubsection{Evaluation}
We fix $M$ a domain of fully abstract interpretation with an equality
coincident with bisimulation. We take $\meaningof{\cdot} : \Proc \to
M$ to be the map interpreting processes and $\nmeaningof{\cdot} : \M
\to Proc$ to be the map running the other way. Then we define

\begin{mathpar}
  \int P := \nmeaningof{\meaningof{P}}
\end{mathpar}

\paragraph{Discussion}
There are many fully abstract interpretations of Milner's
$\pi$-calculus. Any of them can be used as a basis for interpreting
the reflective calculus here. Equipped with such a domain it is
largely a matter of grinding through to check that the Yoneda
construction for the normalization-by-evaluation program can be
extended to this setting.

\begin{remark}
  The reader is invited to verify that $\int (P^{\underline{\perp}}[P]) = 0$.
\end{remark}

\subsection{Quantum mechanics}

Table \ref{tbl:core_qm_op_defns} gives the core operational definitions

\begin{table}[htp]\label{tbl:core_qm_op_defns}
  \center{
    \fbox{
      \begin{tabular}{c|c}
        quantum mechanics & process calculus \\
        \hline
        scalar & $x := \quotep{P}$ \\
        state vector & $\state{P} := P$ \\
        dual & $\state{P}^{*} := \event{P^{\underline{\perp}}} := \quotep{P^{\underline{\perp}}}[-]$ \\
        matrix & $ \Sigma_{\alpha} \state{P_{\alpha}}x_{\alpha}\event{Q_{\alpha}}$ \\
        vector addition & $\state{P} + \state{Q} := \state{P | Q}$ \\
        tensor product & $\state{P} \otimes \state{Q} := \state{P \otimes Q}$ \\
        inner product & $\innerprod{P}{Q} := \quotep{\int P^{\underline{\perp}}[Q]}$ \\
      \end{tabular}
    }
  }
  \caption{QM - operational definitions}
\end{table}

where

\begin{mathpar}
  \prmatrix{P}{Q} := \fprmatrix{P}{\quotep{\pzero}}{Q}
  \and
  \fprmatrix{P}{x}{Q} := (\state{P},x,\event{Q})
  \and
  (\fprmatrix{P}{x}{Q})(\state{R}) := x \cdot \innerprod{Q}{R} \cdot \state{P}
  \and
  (\fprmatrix{P}{x}{Q})(\event{R}) := x \cdot \innerprod{R}{P} \cdot \event{Q}
\end{mathpar}

\paragraph{Discussion}
As promised: vectors (aka states) are represented as processes; duals
as contextual duals; inner product definition should be compared with
standard inner product definition for ....

\begin{remark}
  Assuming $\int (P^{\underline{\perp}}[P]) = 0$, the reader is
  invited to verify that $(\fprmatrix{P}{x}{P})(\state{P}) = x \cdot \state{P}$.
\end{remark}

\begin{remark}
  The reader is invited to verify that $\innerprod{P}{Q}$ could
  equally well have been written $\quotep{\int \stackrel{\vee}{x}}$
  where $x = \event{P^{\underline{\perp}}}(Q)$.

  One of the motivations for this remark is that there is another way
  to factor these operations. We could package up evaluation in the dual:

  \begin{mathpar}
    \state{P}^{*} := \event{\int P^{\underline{\perp}}} := \quotep{\int P^{\underline{\perp}}}[-]
  \end{mathpar}

  and then have inner product defined by
  
  \begin{mathpar}
    \innerprod{P}{Q} := \event{P}(Q)
  \end{mathpar}

  Hopefully, experience with the calculations will provide guidance on
  the best factoring.
\end{remark}

\begin{remark}
  Assuming $\int (P^{\underline{\perp}}[P]) = 0$, the reader is
  invited to verify that $\forall P,Q. (\prmatrix{0}{Q})(\state{0}) =
  \state{0}$ and dually $(\prmatrix{P}{0})(\event{0}) = \event{0}$.
\end{remark}

\begin{remark}
  i'm a little worried that i don't (yet) have proper support for
  complex conjugacy. But, the observation above may give us a
  clue. According to Abramsky, it must be the case that the scalars
  are iso to the homset of the identity for the tensor -- which the
  observation above characterizes. 

  For now, we will simply bookmark the notion with $\overline{x}$.
\end{remark}

\subsubsection{Adjointness}

We need to give a definition of $(\cdot)^{\dagger}$ for matrices. The
obvious candidate definition is
\begin{mathpar}
(\Sigma_{\alpha}\fprmatrix{P_{\alpha}}{x_{\alpha}}{Q_{\alpha}})^{\dagger}
= \Sigma_{\alpha}\fprmatrix{(Q_{\alpha}^{\underline{\perp}})^{*}}{\overline{x}_{\alpha}}{P_{\alpha}^{\underline{\perp}}} 
\end{mathpar}

But, $(Q_{\alpha}^{\underline{\perp}})^{*}$ requires a name along
which to communicate the process to achieve the context application.

\subsubsection{Basis for a basis}
If processes label states and ``addition'' of states (a.k.a. vector
addition) is interpreted as parallel composition, what corresponds to
notions of linear independence and basis? Here, we recall that Yoshida
has developed a set of \emph{combinators} for an asynchronous verison
of Milner's $\pi$-calculus. These are a finite set of processes such
any process can be expressed as parallel composition of these
combinators together with liberal uses of the new operator and
replication. We can simply give a translation of these into the
present calculus and have reasonable expectation that the property
carries over. That is, that the resultant set allows to express all
processes via parallel composition. Note, however, that there is no
new operator or replication in this calculus. As a result, we expect
that the corresponding set is actually infinite. That is, we expect
that the space is actually infinite dimensional.

\begin{remark}
  The attentive reader may be a bit concerned. Certainly, the
  collection $S$, $K$ and $I$ is a finite set of
  combinators. Shouldn't we expect to see a finite set of combinators
  for an effectively equivalent system? i am very sympathetic to this
  critique and feel it warrants full attention. On the other hand, i
  also have in mind the following analogy. The natural numbers, as a
  monoid under addition, has exactly $1$ generator, while the natural
  numbers, as a monoid under multiplication, has countably many
  generators (the primes). We observe that the application of the
  lambda calculus is much less resource sensitive than the parallel
  composition of the $\pi$-calculus. Could it be the case that we have
  an analogy of the form
  
  \begin{mathpar}
    m + n : MN :: m*n : M|N
  \end{mathpar}

  giving a similar blow up in the set of ``primes''?  This is such a
  wonderful thought that, even if it's not true, i think it's worth
  writing down.
\end{remark}
 

\documentclass[12pt]{llncs}
%\documentclass{jktr}

\usepackage[pdftex]{hyperref}                   
\usepackage {listings}
\usepackage {mathpartir}
\usepackage{bcprules}
%\usepackage{listings}
                       
\usepackage{graphicx} 
%\usepackage[margins=2.5cm,nohead,nofoot]{geometry}
%\usepackage{geometry}
\usepackage{amsfonts}
\usepackage{amstext}
\usepackage{latexsym}
\usepackage{amssymb}
\usepackage{color}


%\include{myPreamble}
\include{qm2pi.local} 

%\ifpdf
%\usepackage[pdftex]{graphicx}
%\else
%\usepackage{graphicx}
%\fi

 % \ifpdf
%  \usepackage{pdfsync}
%  \if


%\title{Brief Article}
%\author{David F. Snyder}
%\author{L.G. Meredith}

%\address{Dept. of Math., Texas State University--San Marcos, San Marcos, TX 78666}
       
\pagestyle{empty}


\begin{document}

\lstset{language=[Objective]Caml,frame=shadowbox}

\input{qm2pi.front}

% section front matter (end)

\input{qm2pi.intro} 
 
% section introduction (end)

% \input{qm2pi.knotations} 

% section notation (end)

\input{qm2pi.process.calculi} 

% section concurrent_process_calculi_and_spatial_logics_ (end)
    
%\input{qm2pi.knots2pi} 

%\input{qm2pi.trefoil} 

%\input{qm2pi.mainthm} 

% subsection basic_interpretation (end)

%\input{qm2pi.rho.presentation} 
\subsection{The syntax and semantics of the notation system}\label{sub:the_syntax_and_semantics_of_the_notation_system} % (fold)

We now summarize a technical presentation of the calculus that
embodies our theory of dynamics. The typical presentation of such a
calculus follows the style of giving generators and relations on
them. The grammar, below, describing term constructors, freely
generates the set of processes, $\Proc$. This set is then quotiented
by a relation known as structural congruence and it is over this set
that the notion of dynamics is expressed. This presentation is
essentially that of \cite{MeredithR05} with the addition of
polyadicity and summation. For readability we have relegated some of
the technical subtleties to an appendix.

\subsubsection{Process grammar}\label{subsub:process_grammar}

\begin{mathpar}
  \inferrule* [lab=synchronization] {} {{M} \bc \pzero \;|\; x?F \;|\; x!C }
  \and
  \inferrule* [lab=abstraction] {} {{F} \bc (x)P}
  \and
  \inferrule* [lab=concretion] {} {{C} \bc \langle Q \rangle}
  \and
  \inferrule* [lab=process] {} {{P,Q} \bc M \;| \;P|Q \;|\; @{x}}
  \and
  \inferrule* [lab=name] {} {{x} \bc \quotep{P}}
\end{mathpar} 

Note that $\vec{x}$ (resp. $\vec{P}$) denotes a vector of names
(resp. processes) of length $|\vec{x}|$ (resp. $|\vec{P}|$). We adopt
the following useful abbreviations.

\begin{mathpar}
   x?(\vec{y}).P := x.(\vec{y})P \and  x\clift{\vec{P}} := x.\clift{\vec{P}}
   \and x!(y) := \lift{x}{\dropn{y}}
   \and \Pi_{i=0}^{n-1}P_i := P_0 | \ldots | P_{n-1}
\end{mathpar}

\subsubsection{Structural congruence}

\paragraph{Free and bound names and alpha-equivalence.} At the
core of structural equivalence is alpha-equivalence which identifies
process that are the same up to a change of variable. Formally, we
recognize the distinction between free and bound names. The free names
of a process, $\freenames{P}$, may be calculated recursively as
follows:

\begin{mathpar}
\freenames{\pzero} := \emptyset
  \and \\
  \freenames{x?(y).P} := \{ x \} \cup (\freenames{P} \setminus \{ y \})
  \and 
  \freenames{x!\langle P \rangle} := \{ x \} \cup \{ P \} 
  \and \\
  \freenames{P|Q} := \freenames{P} \cup \freenames{Q}
  \and \\
  \freenames{@{x}} := \{ x \}
\end{mathpar}

$\pi$
$\quotep{\pi}$

$\freenames{-} : \pi \to \mathcal{P}(\quotep{\pi})$

\begin{eqnarray*}
  \freenames{\pzero} & := & \emptyset \\
  \freenames{x?(y).P} & := & \{ x \} \cup (\freenames{P} \setminus \{ y \}) \\
  \freenames{x!\langle P \rangle} & := & \{ x \} \cup \{ P \} \\
  \freenames{P|Q} & := & \freenames{P} \cup \freenames{Q} \\
  \freenames{\dropn{x}} & := & \{ x \}
\end{eqnarray*}

The bound names of a process, $\boundnames{P}$, are those names occurring in $P$
that are not free. For example, in $x?(y).0$, the name $x$ is free, while $y$ is bound.

\begin{mathpar}
  \inferrule* [lab=monoidal-laws] {} { P|Q \equiv Q|P \and P|0 \equiv P \and P|(Q|R) \equiv (P|Q)|R }
\end{mathpar}

\begin{mathpar}
  \inferrule* [lab=alpha-equivalence] {} { (x)P \equiv (y)P\{y/x\} \and y \not\in \freenames{P} }
\end{mathpar}

\begin{definition}
Then two processes, $P,Q$, are alpha-equivalent if $P = Q\{\vec{y}/\vec{x}\}$ for
some $\vec{x} \in \boundnames{Q},\vec{y} \in \boundnames{P}$, where $Q\{\vec{y}/\vec{x}\}$
denotes the capture-avoiding substitution of $\vec{y}$ for $\vec{x}$ in $Q$.
\end{definition}

\begin{definition}
  The {\em structural congruence} \cite{SangiorgiWalker} , $\equiv$,
  between processes is the least congruence containing
  alpha-equivalence, satisfying the abelian monoid laws
  (associativity, commutativity and $\pzero$ as identity) for parallel
  composition $|$ and for summation $+$.
\end{definition}

\subsection{Name equivalence}

We take name equivalence, written $\nameeq$, to be the smallest
equivalence relation generated by the following rules.

\begin{mathpar}
\inferrule*[lab=Quote-drop]
{ }
{ \quotep{@{x}} \nameeq x }

\inferrule*[lab=Struct-equiv]
{ P \scong Q }
{ \quotep{P} \nameeq \quotep{Q} }
\end{mathpar}

The astute reader will have noticed that the mutual recursion of names
and processes imposes a mutual recursion on alpha-equivalence and
structural equivalence via name-equivalence. Fortunately, all of this
works out pleasantly and we may calculate in the natural way, free of
concern. The reader interested in the details is referred to the
appendix \ref{appendix:rho_details}.

\subsection{Substitution}

We use $\Proc$ for the set of processes, $\QProc$ for the set of
names, and $\id{\{}\vec{y} / \vec{x} \id{\}}$ to denote partial maps,
$s : \QProc \rightarrow \QProc$. A map, $s$ lifts, uniquely, to a map
on process terms, $\widehat{s} : \Proc \rightarrow \Proc$ by the
following equations.

\begin{mathpar}
  (0) \psubstp{Q}{P} := 0 \\
  (R \juxtap S) \psubstp{Q}{P}
  :=    
  (R)\psubstp{Q}{P} \juxtap (S) \psubstp{Q}{P} \\
  (x?(y).R) \psubstp{Q}{P}    
  :=    
  (x)\substp{Q}{P} (z)\concat( (R \psubstn{z}{y}) \psubstp{Q}{P} ) \\
  (\lift{x}{R}) \psubstp{Q}{P}  
  :=
  \lift{(x)\substp{Q}{P}}{ R \psubstp{Q}{P} } \\
%   (\dropn{x})  \psubstp{Q}{P}       
%   := 
%   \left\{ 
%     \begin{array}{ccc} 
%       \dropn{\quotep{Q}} & & x \nameeq \quotep{P} \\
%       \dropn{x} & & otherwise \\
%     \end{array}
%   \right. 
  (\dropn{x})  \psubstp{Q}{P}       
  := 
  \left\{ 
    \begin{array}{ccc} 
      Q & & x \nameeq \quotep{P} \\
      \dropn{x} & & otherwise \\
    \end{array}
  \right.
\end{mathpar}
 

where

\begin{eqnarray}
  (x)\id{\{} \lpquote Q \rpquote / \lpquote P \rpquote \id{\}}            = 
  \left\{ 
    \begin{array}{ccc}
      \lpquote Q \rpquote & & x \nameeq \lpquote P \rpquote \\
      x & & otherwise \\
    \end{array}
  \right. \nonumber
\end{eqnarray}

and $z$ is chosen distinct from $\quotep{P}$, $\quotep{Q}$, the free
names in $Q$, and all the names in $R$. Our $\alpha$-equivalence will
be built in the standard way from this substitution.

\begin{remark}\label{rem:no_self_referential_names}
  One consequence of these definitions is that $\forall P. \quotep{P}
  \not\in \freenames{P}$.
\end{remark}

\subsection{ Dynamic quote: an example }

Anticipating something of what's to come, consider applying the
substitution, $\widehat{\id{\{}u / z \id{\}}}$, to the following pair
of processes, $\lift{w}{y!(z)}$ and $w[ \lpquote y!(z) \rpquote ]$.

\begin{eqnarray}
	\lift{w}{y!(z)}\widehat{\id{\{}u / z \id{\}}}
		& = &
		\lift{w}{y!(u)} \nonumber\\
	w[ \lpquote y!(z) \rpquote ] \widehat{ \id{\{}u / z \id{\}} }
		& = &
		w[ \lpquote y!(z) \rpquote ] \nonumber
\end{eqnarray}

Because the body of the process between quotes is impervious to
substitution, we get radically different answers. In fact, by
examining the first process in an input context,
e.g. $x?(z).\lift{w}{y!(z)}$, we see that the process under the lift
operator may be shaped by prefixed inputs binding a name inside it. In
this sense, the lift operator will be seen as a way to dynamically
construct processes before reifying them as names.

Finally equipped with these standard features we can present the
dynamics of the calculus.

\subsubsection{Operational semantics} 

Finally, we introduce the computational dynamics. What marks these
algebras as distinct from other more traditionally studied algebraic
structures, e.g. vector spaces or polynomial rings, is the manner in
which dynamics is captured. In traditional structures, dynamics is typically
expressed through morphisms between such structures, as in linear maps
between vector spaces or morphisms between rings. In algebras
associated with the semantics of computation, the dynamics is
expressed as part of the algebraic structure itself, through a
reduction reduction relation typically denoted by $\red$. Below, we
give a recursive presentation of this relation for the calculus used
in the encoding.

$\red \subseteq \pi \times \pi$
$\red : \pi \to \mathcal{P}(\pi)$

\begin{mathpar}
  \inferrule* [lab=Comm] { \textsf{match}( x_{src}, x_{trgt} ) } { x_{trgt}?(y)P \; | \; x_{src}!\langle {Q} \rangle \red P\{\quotep{Q}/y}\} }
  \and \\
  \inferrule* [lab=Par] {{P} \red {P}'} {{{P} | {Q}} \red {{P}' | {Q}}}
  \and
  \inferrule* [lab=Equiv]{{{P} \scong {P}'} \andalso {{P}' \red {Q}'} \andalso {{Q}' \scong {Q}}}{{P} \red {Q}}
\end{mathpar}

\begin{eqnarray*}
  match_{\equiv} (\quotep{P},\quotep{Q}) & := & P \equiv Q \\
  match_{\dagger}(\quotep{P},\quotep{Q}) & := & \forall R. P|Q \red^{*} R => R \red^{*} 0 \\
  match_{K}(\quotep{P},\quotep{Q}) & := & K \mbox{ for some context } K
\end{eqnarray*}

$u?(x)P | u!\langle Q \rangle \red P\{\quotep{Q}/x\}$

%We write $\wred$ for $\red^*$, and $P\red$ if $\exists Q $ such that $ P \red Q$.
We write $P\red$ if $\exists Q $ such that $ P \red Q$ and $P\not\red$, otherwise.

\section{Replication}

As mentioned before, it is known that replication (and hence
recursion) can be implemented in a higher-order process algebra
\cite{SangiorgiWalker}. As our first example of calculation with the
machinery thus far presented we give the construction explicitly in
the {\rhoc}.

\begin{eqnarray}
	D_{x} & := & \prefix{x}{y}{(\binpar{\outputp{x}{y}}{@{y}})} \nonumber\\
	\bangp_{x}{P} & := & \binpar{{x}!\langle{\binpar{D_{x}}{P}}\rangle}{D_{x}} \nonumber
\end{eqnarray}

\begin{eqnarray}
	\bangp_{x}{P} & & \nonumber\\
	=
	& {x}!\langle{(\prefix{x}{y}{(\outputp{x}{y} | @{y})) | P}}\rangle 
	      | \prefix{x}{y}{(\outputp{x}{y} | @{y})} & \nonumber\\
	\red
	& (\outputp{x}{y} | @{y})\substn{\quotep{(\prefix{x}{y}{(@{y} | \outputp{x}{y})) | P}}}{y} & \nonumber\\
	=
	& \outputp{x}{\quotep{(\prefix{x}{y}{(\outputp{x}{y} | @{y})) | P}}}
	  | {(\prefix{x}{y}{(\outputp{x}{y} | @{y})) | P}} & \nonumber\\
	\red
	& \ldots & \nonumber\\
	\red^*
	& P | P | \ldots & \nonumber
\end{eqnarray}

Of course, this encoding, as an implementation, runs away, unfolding
$\bangp{P}$ eagerly. A lazier and more implementable replication
operator, restricted to input-guarded processes, may be obtained as follows.

\begin{eqnarray}
\bangp{\prefix{u}{v}{P}} 
	:= 
	\binpar{\lift{x}{\prefix{u}{v}{(\binpar{D(x)}{P})}}}{D(x)} \nonumber
\end{eqnarray}

\begin{remark}
  Note that the lazier definition still does not deal with summation
  or mixed summation (i.e. sums over input and output). The reader is
  invited to construct definitions of replication that deal with these
  features. 

  Further, the definitions are parameterized in a name, $x$. Can you,
  gentle reader, make a definition that eliminates this parameter and
  guarantees no accidental interaction between the replication
  machinery and the process being replicated -- i.e. no accidental
  sharing of names used by the process to get its work done and the
  name(s) used by the replication to effect copying. This latter
  revision of the definition of replication is crucial to obtaining
  the expected identity $!!P \sim !P$.
\end{remark}

\begin{remark}\label{rem:paradoxical_combinator}
  The reader familiar with the lambda calculus will have noticed the
  similarity between $D$ and the paradoxical combinator.

  [Ed. note: the existence of this seems to suggest we have to be more
  restrictive on the set of processes and names we admit if we are to
  support no-cloning.]
\end{remark}

\subsubsection{Bisimulation}

The computational dynamics gives rise to another kind of equivalence,
the equivalence of computational behavior. As previously mentioned
this is typically captured \emph{via} some form of bisimulation.

% The notion we use in this paper is weak barbed bisimulation
% \cite{milner91polyadicpi}.

The notion we use in this paper is derived from weak barbed
bisimulation \cite{milner91polyadicpi}. 

\begin{definition}
An \emph{observation relation}, $\downarrow_{\mathcal N}$, over a set
of names, $\mathcal N$, is the smallest relation satisfying the rules
below.

\infrule[Out-barb]{y \in {\mathcal N}, \; x \nameeq y}
		  {\outputp{x}{v} \downarrow_{\mathcal N} x}
\infrule[Par-barb]{\mbox{$P\downarrow_{\mathcal N} x$ or $Q\downarrow_{\mathcal N} x$}}
		  {\binpar{P}{Q} \downarrow_{\mathcal N} x}

We write $P \Downarrow_{\mathcal N} x$ if there is $Q$ such that 
$P \wred Q$ and $Q \downarrow_{\mathcal N} x$.
\end{definition}

\begin{definition}
%\label{def.bbisim}
An  ${\mathcal N}$-\emph{barbed bisimulation} over a set of names, ${\mathcal N}$, is a symmetric binary relation 
${\mathcal S}_{\mathcal N}$ between agents such that $P\rel{S}_{\mathcal N}Q$ implies:
\begin{enumerate}
\item If $P \red P'$ then $Q \wred Q'$ and $P'\rel{S}_{\mathcal N} Q'$.
\item If $P\downarrow_{\mathcal N} x$, then $Q\Downarrow_{\mathcal N} x$.
\end{enumerate}
$P$ is ${\mathcal N}$-barbed bisimilar to $Q$, written
$P \wbbisim_{\mathcal N} Q$, if $P \rel{S}_{\mathcal N} Q$ for some ${\mathcal N}$-barbed bisimulation ${\mathcal S}_{\mathcal N}$.
\end{definition}

$\mathcal{R} \subseteq \pi \times \pi$

$P \mathcal{R} Q => \forall P'. P \red P' \Rightarrow \exists Q'. Q \red Q', P' \mathcal{R} Q'$

$P \vdash x \Rightarrow Q \vdash x$

\begin{mathpar}
  \inferrule*[lab=Out-barb]{x \nameeq y}{{y}!\langle{Q}\rangle \vdash x}
  \and
  \inferrule*[lab=Par-barb]{\mbox{$P\vdash x$ or $Q\vdash x$}}{\binpar{P}{Q} \vdash x}
\end{mathpar}

\subsubsection{Contexts}

One of the principle advantages of computational calculi like the
$\pi$-calculus is a well-defined notion of context,
contextual-equivalence and a correlation between
contextual-equivalence and notions of bisimulation. The notion of
context allows the decomposition of a process into (sub-)process and
its syntactic environment, its context. Thus, a context may be
thought of as a process with a ``hole'' (written $\Box$) in it. The
application of a context $M$ to a process $P$, written $M[P]$, is
tantamount to filling the hole in $M$ with $P$. In this paper we do
not need the full weight of this theory, but do make use of the notion
of context in the proof the main theorem. 

\begin{mathpar}
  \inferrule* [lab=summation] {} {{M_{M},M_{N}} \bc \Box \;|\; x.M_{A} \;|\; M_{M}+M_{N}}
  \and
  \inferrule* [lab=agent] {} {{M_{A}} \bc (\vec{x})M_{P} \;| \; \clift{P_0,\ldots,M_{P},\ldots,P_N}}
  \and \\
  \inferrule* [lab=process] {} {{M_{P}} \bc M_{N} \;| \;P|M_{P} }
\end{mathpar} 

\begin{mathpar}
  \inferrule* [lab=sychronization] {} {M_{N} \bc \Box \;|\; x?M_{F} \;|\; x!M_{C}}
  \and
  \inferrule* [lab=abstraction] {} {{M_{F}} \bc (x)M_{P} }
  \and
  \inferrule* [lab=concretion] {} {{M_{C}} \bc \langle M_{P} \rangle }
  \and \\
  \inferrule* [lab=process] {} {{M_{P}} \bc M_{N} \;| \;P|M_{P} }
\end{mathpar}

\begin{definition}[contextual application] Given a context $M$, and
  process $P$, we define the \emph{contextual application}, $M[P] :=
  M\{P/\Box\}$. That is, the contextual application of M to P is the
  substitution of $P$ for $\Box$ in $M$.
\end{definition}

$\meaningof{-} : L \to \mathcal{P}(\pi)$

\begin{mathpar}
  \inferrule* [lab=collection] {} {\meaningof{true} = \pi, \and \meaningof{~E} = \pi \setminus \meaningof{E}, \and \meaningof{E_{1} \& E_{2}} = \meaningof{E_{1}} \cap \meaningof{E_{2}}}
\end{mathpar}

\begin{mathpar}
  \inferrule* [lab=structure] {} {\meaningof{0} = \{ P \in \pi | P \equiv 0 \}, \and \\ \meaningof{E_1 | E_2} = \{ P \in \pi | P \equiv P_{1} | P_{2}, P_{1} \in \meaningof{E_{1}}, P_{2} \in \meaningof{E_2}\} }
\end{mathpar}

\begin{mathpar}
 \inferrule* [lab=behavior] {} {\meaningof{\langle a?b \rangle E} = \{ P \in \pi | P \equiv Q | u?(y)P', \\ \and \\\\ \and \\ \;\;\; u \in \meaningof{a}, \forall z.P'\{z/y\} \in \meaningof{E\{z/b\}}\}, \and \\ \meaningof{a!E} = \{ P \in \pi | P \equiv Q | x!\langle P' \rangle, x \in \meaningof{a} P' \in \meaningof{E}\} }
\end{mathpar}

\begin{mathpar}
 \inferrule* [lab=nominal] {} {\meaningof{\quotep{E}} = \{ \quotep{P} \in \quotep{\pi} | P \in \meaningof{E} \}, \and \meaningof{\quotep{P}} = \{ \quotep{Q} \in \quotep{\pi} | P \equiv Q \} \and \\ \meaningof{@\quotep{E}} = \{ P \in \pi | P \equiv @x, x \in \meaningof{E} \}}
\end{mathpar}

\begin{eqnarray*}
  \\
  \meaningof{-} : TS \to ST
\end{eqnarray*}

\begin{eqnarray*}
  \\
  L : TS \to ST
\end{eqnarray*}

\begin{eqnarray*}
  \\
  P \models E \iff P \in \meaningof{E}
\end{eqnarray*}

\begin{eqnarray*}
  P \approx_{L} Q \iff \forall E \in L. P \models E \iff Q \models E
\end{eqnarray*}

\begin{eqnarray*}
  P \approx_{K} Q
\end{eqnarray*}

\begin{eqnarray*}
  P \approx Q
\end{eqnarray*}

$\approx_{K} = \approx = \approx_{L}$

\subsubsection{Contextual duality}

Note that contexts extend the quotation operation to a family of
operations from processes to names. Given a context, $M$, we can
define a \emph{nominal context}, $\quotep{M}$ by $\quotep{M}[P] :=
\quotep{M[P]}$. To foreshadow what is to come we observe that these
operations enjoy a duality with processes very much like the duality
between vectors and maps from vectors to scalars.

Further, because the calculus is essentially higher-order, we have a
correspondence between contexts and processes. More specifically,
given a name $x$ and a context $M$ we can construct $M^{*}_{x}$ such
that 

\begin{mathpar}
  M^{*}_{x} | \lift{x}{P} \red M[P]
\end{mathpar}

namely,

\begin{mathpar}
  M^{*}_{x} := x?(u).M[\dropn{u}]
\end{mathpar}

The dependence of $M^{*}_{x}$ on a name makes it an abstraction, 

\begin{mathpar}
  M^{*} := (x)x?(u).M[\dropn{u}]
\end{mathpar}

\subsection{Additional notation}

It will sometimes be convenient to denote the process a name
quotes. We already have the notation $x = \quotep{P}$, but it will be
convenient to introduce an alternate notation, $\procn{x}$, when we
want to emphasize the connection to the use of the name. Note that, by
virtue of name equivalence, $\quotep{\procn{x}} \nameeq x$; so, the
notation is consistent with previous definitions.

Further, because names have structure it is possible to effect
substitutions on the basis of that structure. This means we need to
upgrade our notation for substitutions, which we accomplish by
adapting comprehension notation. Thus,

\begin{mathpar}
  P\{ y / x : x \in S \}
\end{mathpar}

is interpreted to mean the process derived from P by replacing (in a
capture-avoiding manner) each occurrence of $x$ in $S$ by $y$. For example,

\begin{mathpar}
  P\{ \quotep{\procn{x}|\procn{x}} / x : x \in \freenames{P} \}
\end{mathpar}

will replace each (occurrence) of a free name $x$ in $P$ by
$\quotep{\procn{x}|\procn{x}}$.

Also, we will avail ourselves of the notation $x^{L}$ and $x^{R}$ to
denote injections of a name into disjoint copies of the name
space. There are numerous ways to accomplish this. One example can be
found in \cite{MeredithR05}. This notation overloads to vectors of
names: $\vec{x}^{\pi} := (x_{i}^{\pi} \; : \; 0 \leq i < |\vec{x}| )$ where $\pi \in \{L,R\}$.

We also use $P^{\Box} := P|\Box$.

In \cite{MeredithR05} an interpretation of the new operator is
given. It turns out that there are several possible interpretations
all enjoying the requisite algebraic properties of the operator (see
\cite{milner91polyadicpi}). We will therefore make liberal use of
$(\nu\; \vec{x})P$.

% subsection the_syntax_and_semantics_of_the_notation_system (end)   

\input{qm2pi.qmops} 

\input{qm2pi.sterngerlach} 

\input{qm2pi.metric} 

% section concurrent_process_calculi (end)

%\input{qm2pi.proofsketch}

% section proof sketch (end)

%\input{qm2pi.slviaknots} 

% section spatial logic via knots (end)

\input{qm2pi.conclusion}

% section conclusion (end)

%\input{qm2pi.dtcodes} 

% section wiring algorithm (end)

\input{qm2pi.ack} 

% section acknowledgments (end)

\newpage


\bibliographystyle{plain}   
\bibliography{../../biblios/main.bib}

\input{qm2pi.rhodetails}

\end{document}

 

\documentclass[12pt]{llncs}
%\documentclass{jktr}

\usepackage[pdftex]{hyperref}                   
\usepackage {listings}
\usepackage {mathpartir}
\usepackage{bcprules}
%\usepackage{listings}
                       
\usepackage{graphicx} 
%\usepackage[margins=2.5cm,nohead,nofoot]{geometry}
%\usepackage{geometry}
\usepackage{amsfonts}
\usepackage{amstext}
\usepackage{latexsym}
\usepackage{amssymb}
\usepackage{color}


%\include{myPreamble}
\include{qm2pi.local} 

%\ifpdf
%\usepackage[pdftex]{graphicx}
%\else
%\usepackage{graphicx}
%\fi

 % \ifpdf
%  \usepackage{pdfsync}
%  \if


%\title{Brief Article}
%\author{David F. Snyder}
%\author{L.G. Meredith}

%\address{Dept. of Math., Texas State University--San Marcos, San Marcos, TX 78666}
       
\pagestyle{empty}


\begin{document}

\lstset{language=[Objective]Caml,frame=shadowbox}

\input{qm2pi.front}

% section front matter (end)

\input{qm2pi.intro} 
 
% section introduction (end)

% \input{qm2pi.knotations} 

% section notation (end)

\input{qm2pi.process.calculi} 

% section concurrent_process_calculi_and_spatial_logics_ (end)
    
%\input{qm2pi.knots2pi} 

%\input{qm2pi.trefoil} 

%\input{qm2pi.mainthm} 

% subsection basic_interpretation (end)

%\input{qm2pi.rho.presentation} 
\subsection{The syntax and semantics of the notation system}\label{sub:the_syntax_and_semantics_of_the_notation_system} % (fold)

We now summarize a technical presentation of the calculus that
embodies our theory of dynamics. The typical presentation of such a
calculus follows the style of giving generators and relations on
them. The grammar, below, describing term constructors, freely
generates the set of processes, $\Proc$. This set is then quotiented
by a relation known as structural congruence and it is over this set
that the notion of dynamics is expressed. This presentation is
essentially that of \cite{MeredithR05} with the addition of
polyadicity and summation. For readability we have relegated some of
the technical subtleties to an appendix.

\subsubsection{Process grammar}\label{subsub:process_grammar}

\begin{mathpar}
  \inferrule* [lab=synchronization] {} {{M} \bc \pzero \;|\; x?F \;|\; x!C }
  \and
  \inferrule* [lab=abstraction] {} {{F} \bc (x)P}
  \and
  \inferrule* [lab=concretion] {} {{C} \bc \langle Q \rangle}
  \and
  \inferrule* [lab=process] {} {{P,Q} \bc M \;| \;P|Q \;|\; @{x}}
  \and
  \inferrule* [lab=name] {} {{x} \bc \quotep{P}}
\end{mathpar} 

Note that $\vec{x}$ (resp. $\vec{P}$) denotes a vector of names
(resp. processes) of length $|\vec{x}|$ (resp. $|\vec{P}|$). We adopt
the following useful abbreviations.

\begin{mathpar}
   x?(\vec{y}).P := x.(\vec{y})P \and  x\clift{\vec{P}} := x.\clift{\vec{P}}
   \and x!(y) := \lift{x}{\dropn{y}}
   \and \Pi_{i=0}^{n-1}P_i := P_0 | \ldots | P_{n-1}
\end{mathpar}

\subsubsection{Structural congruence}

\paragraph{Free and bound names and alpha-equivalence.} At the
core of structural equivalence is alpha-equivalence which identifies
process that are the same up to a change of variable. Formally, we
recognize the distinction between free and bound names. The free names
of a process, $\freenames{P}$, may be calculated recursively as
follows:

\begin{mathpar}
\freenames{\pzero} := \emptyset
  \and \\
  \freenames{x?(y).P} := \{ x \} \cup (\freenames{P} \setminus \{ y \})
  \and 
  \freenames{x!\langle P \rangle} := \{ x \} \cup \{ P \} 
  \and \\
  \freenames{P|Q} := \freenames{P} \cup \freenames{Q}
  \and \\
  \freenames{@{x}} := \{ x \}
\end{mathpar}

$\pi$
$\quotep{\pi}$

$\freenames{-} : \pi \to \mathcal{P}(\quotep{\pi})$

\begin{eqnarray*}
  \freenames{\pzero} & := & \emptyset \\
  \freenames{x?(y).P} & := & \{ x \} \cup (\freenames{P} \setminus \{ y \}) \\
  \freenames{x!\langle P \rangle} & := & \{ x \} \cup \{ P \} \\
  \freenames{P|Q} & := & \freenames{P} \cup \freenames{Q} \\
  \freenames{\dropn{x}} & := & \{ x \}
\end{eqnarray*}

The bound names of a process, $\boundnames{P}$, are those names occurring in $P$
that are not free. For example, in $x?(y).0$, the name $x$ is free, while $y$ is bound.

\begin{mathpar}
  \inferrule* [lab=monoidal-laws] {} { P|Q \equiv Q|P \and P|0 \equiv P \and P|(Q|R) \equiv (P|Q)|R }
\end{mathpar}

\begin{mathpar}
  \inferrule* [lab=alpha-equivalence] {} { (x)P \equiv (y)P\{y/x\} \and y \not\in \freenames{P} }
\end{mathpar}

\begin{definition}
Then two processes, $P,Q$, are alpha-equivalent if $P = Q\{\vec{y}/\vec{x}\}$ for
some $\vec{x} \in \boundnames{Q},\vec{y} \in \boundnames{P}$, where $Q\{\vec{y}/\vec{x}\}$
denotes the capture-avoiding substitution of $\vec{y}$ for $\vec{x}$ in $Q$.
\end{definition}

\begin{definition}
  The {\em structural congruence} \cite{SangiorgiWalker} , $\equiv$,
  between processes is the least congruence containing
  alpha-equivalence, satisfying the abelian monoid laws
  (associativity, commutativity and $\pzero$ as identity) for parallel
  composition $|$ and for summation $+$.
\end{definition}

\subsection{Name equivalence}

We take name equivalence, written $\nameeq$, to be the smallest
equivalence relation generated by the following rules.

\begin{mathpar}
\inferrule*[lab=Quote-drop]
{ }
{ \quotep{@{x}} \nameeq x }

\inferrule*[lab=Struct-equiv]
{ P \scong Q }
{ \quotep{P} \nameeq \quotep{Q} }
\end{mathpar}

The astute reader will have noticed that the mutual recursion of names
and processes imposes a mutual recursion on alpha-equivalence and
structural equivalence via name-equivalence. Fortunately, all of this
works out pleasantly and we may calculate in the natural way, free of
concern. The reader interested in the details is referred to the
appendix \ref{appendix:rho_details}.

\subsection{Substitution}

We use $\Proc$ for the set of processes, $\QProc$ for the set of
names, and $\id{\{}\vec{y} / \vec{x} \id{\}}$ to denote partial maps,
$s : \QProc \rightarrow \QProc$. A map, $s$ lifts, uniquely, to a map
on process terms, $\widehat{s} : \Proc \rightarrow \Proc$ by the
following equations.

\begin{mathpar}
  (0) \psubstp{Q}{P} := 0 \\
  (R \juxtap S) \psubstp{Q}{P}
  :=    
  (R)\psubstp{Q}{P} \juxtap (S) \psubstp{Q}{P} \\
  (x?(y).R) \psubstp{Q}{P}    
  :=    
  (x)\substp{Q}{P} (z)\concat( (R \psubstn{z}{y}) \psubstp{Q}{P} ) \\
  (\lift{x}{R}) \psubstp{Q}{P}  
  :=
  \lift{(x)\substp{Q}{P}}{ R \psubstp{Q}{P} } \\
%   (\dropn{x})  \psubstp{Q}{P}       
%   := 
%   \left\{ 
%     \begin{array}{ccc} 
%       \dropn{\quotep{Q}} & & x \nameeq \quotep{P} \\
%       \dropn{x} & & otherwise \\
%     \end{array}
%   \right. 
  (\dropn{x})  \psubstp{Q}{P}       
  := 
  \left\{ 
    \begin{array}{ccc} 
      Q & & x \nameeq \quotep{P} \\
      \dropn{x} & & otherwise \\
    \end{array}
  \right.
\end{mathpar}
 

where

\begin{eqnarray}
  (x)\id{\{} \lpquote Q \rpquote / \lpquote P \rpquote \id{\}}            = 
  \left\{ 
    \begin{array}{ccc}
      \lpquote Q \rpquote & & x \nameeq \lpquote P \rpquote \\
      x & & otherwise \\
    \end{array}
  \right. \nonumber
\end{eqnarray}

and $z$ is chosen distinct from $\quotep{P}$, $\quotep{Q}$, the free
names in $Q$, and all the names in $R$. Our $\alpha$-equivalence will
be built in the standard way from this substitution.

\begin{remark}\label{rem:no_self_referential_names}
  One consequence of these definitions is that $\forall P. \quotep{P}
  \not\in \freenames{P}$.
\end{remark}

\subsection{ Dynamic quote: an example }

Anticipating something of what's to come, consider applying the
substitution, $\widehat{\id{\{}u / z \id{\}}}$, to the following pair
of processes, $\lift{w}{y!(z)}$ and $w[ \lpquote y!(z) \rpquote ]$.

\begin{eqnarray}
	\lift{w}{y!(z)}\widehat{\id{\{}u / z \id{\}}}
		& = &
		\lift{w}{y!(u)} \nonumber\\
	w[ \lpquote y!(z) \rpquote ] \widehat{ \id{\{}u / z \id{\}} }
		& = &
		w[ \lpquote y!(z) \rpquote ] \nonumber
\end{eqnarray}

Because the body of the process between quotes is impervious to
substitution, we get radically different answers. In fact, by
examining the first process in an input context,
e.g. $x?(z).\lift{w}{y!(z)}$, we see that the process under the lift
operator may be shaped by prefixed inputs binding a name inside it. In
this sense, the lift operator will be seen as a way to dynamically
construct processes before reifying them as names.

Finally equipped with these standard features we can present the
dynamics of the calculus.

\subsubsection{Operational semantics} 

Finally, we introduce the computational dynamics. What marks these
algebras as distinct from other more traditionally studied algebraic
structures, e.g. vector spaces or polynomial rings, is the manner in
which dynamics is captured. In traditional structures, dynamics is typically
expressed through morphisms between such structures, as in linear maps
between vector spaces or morphisms between rings. In algebras
associated with the semantics of computation, the dynamics is
expressed as part of the algebraic structure itself, through a
reduction reduction relation typically denoted by $\red$. Below, we
give a recursive presentation of this relation for the calculus used
in the encoding.

$\red \subseteq \pi \times \pi$
$\red : \pi \to \mathcal{P}(\pi)$

\begin{mathpar}
  \inferrule* [lab=Comm] { \textsf{match}( x_{src}, x_{trgt} ) } { x_{trgt}?(y)P \; | \; x_{src}!\langle {Q} \rangle \red P\{\quotep{Q}/y}\} }
  \and \\
  \inferrule* [lab=Par] {{P} \red {P}'} {{{P} | {Q}} \red {{P}' | {Q}}}
  \and
  \inferrule* [lab=Equiv]{{{P} \scong {P}'} \andalso {{P}' \red {Q}'} \andalso {{Q}' \scong {Q}}}{{P} \red {Q}}
\end{mathpar}

\begin{eqnarray*}
  match_{\equiv} (\quotep{P},\quotep{Q}) & := & P \equiv Q \\
  match_{\dagger}(\quotep{P},\quotep{Q}) & := & \forall R. P|Q \red^{*} R => R \red^{*} 0 \\
  match_{K}(\quotep{P},\quotep{Q}) & := & K \mbox{ for some context } K
\end{eqnarray*}

$u?(x)P | u!\langle Q \rangle \red P\{\quotep{Q}/x\}$

%We write $\wred$ for $\red^*$, and $P\red$ if $\exists Q $ such that $ P \red Q$.
We write $P\red$ if $\exists Q $ such that $ P \red Q$ and $P\not\red$, otherwise.

\section{Replication}

As mentioned before, it is known that replication (and hence
recursion) can be implemented in a higher-order process algebra
\cite{SangiorgiWalker}. As our first example of calculation with the
machinery thus far presented we give the construction explicitly in
the {\rhoc}.

\begin{eqnarray}
	D_{x} & := & \prefix{x}{y}{(\binpar{\outputp{x}{y}}{@{y}})} \nonumber\\
	\bangp_{x}{P} & := & \binpar{{x}!\langle{\binpar{D_{x}}{P}}\rangle}{D_{x}} \nonumber
\end{eqnarray}

\begin{eqnarray}
	\bangp_{x}{P} & & \nonumber\\
	=
	& {x}!\langle{(\prefix{x}{y}{(\outputp{x}{y} | @{y})) | P}}\rangle 
	      | \prefix{x}{y}{(\outputp{x}{y} | @{y})} & \nonumber\\
	\red
	& (\outputp{x}{y} | @{y})\substn{\quotep{(\prefix{x}{y}{(@{y} | \outputp{x}{y})) | P}}}{y} & \nonumber\\
	=
	& \outputp{x}{\quotep{(\prefix{x}{y}{(\outputp{x}{y} | @{y})) | P}}}
	  | {(\prefix{x}{y}{(\outputp{x}{y} | @{y})) | P}} & \nonumber\\
	\red
	& \ldots & \nonumber\\
	\red^*
	& P | P | \ldots & \nonumber
\end{eqnarray}

Of course, this encoding, as an implementation, runs away, unfolding
$\bangp{P}$ eagerly. A lazier and more implementable replication
operator, restricted to input-guarded processes, may be obtained as follows.

\begin{eqnarray}
\bangp{\prefix{u}{v}{P}} 
	:= 
	\binpar{\lift{x}{\prefix{u}{v}{(\binpar{D(x)}{P})}}}{D(x)} \nonumber
\end{eqnarray}

\begin{remark}
  Note that the lazier definition still does not deal with summation
  or mixed summation (i.e. sums over input and output). The reader is
  invited to construct definitions of replication that deal with these
  features. 

  Further, the definitions are parameterized in a name, $x$. Can you,
  gentle reader, make a definition that eliminates this parameter and
  guarantees no accidental interaction between the replication
  machinery and the process being replicated -- i.e. no accidental
  sharing of names used by the process to get its work done and the
  name(s) used by the replication to effect copying. This latter
  revision of the definition of replication is crucial to obtaining
  the expected identity $!!P \sim !P$.
\end{remark}

\begin{remark}\label{rem:paradoxical_combinator}
  The reader familiar with the lambda calculus will have noticed the
  similarity between $D$ and the paradoxical combinator.

  [Ed. note: the existence of this seems to suggest we have to be more
  restrictive on the set of processes and names we admit if we are to
  support no-cloning.]
\end{remark}

\subsubsection{Bisimulation}

The computational dynamics gives rise to another kind of equivalence,
the equivalence of computational behavior. As previously mentioned
this is typically captured \emph{via} some form of bisimulation.

% The notion we use in this paper is weak barbed bisimulation
% \cite{milner91polyadicpi}.

The notion we use in this paper is derived from weak barbed
bisimulation \cite{milner91polyadicpi}. 

\begin{definition}
An \emph{observation relation}, $\downarrow_{\mathcal N}$, over a set
of names, $\mathcal N$, is the smallest relation satisfying the rules
below.

\infrule[Out-barb]{y \in {\mathcal N}, \; x \nameeq y}
		  {\outputp{x}{v} \downarrow_{\mathcal N} x}
\infrule[Par-barb]{\mbox{$P\downarrow_{\mathcal N} x$ or $Q\downarrow_{\mathcal N} x$}}
		  {\binpar{P}{Q} \downarrow_{\mathcal N} x}

We write $P \Downarrow_{\mathcal N} x$ if there is $Q$ such that 
$P \wred Q$ and $Q \downarrow_{\mathcal N} x$.
\end{definition}

\begin{definition}
%\label{def.bbisim}
An  ${\mathcal N}$-\emph{barbed bisimulation} over a set of names, ${\mathcal N}$, is a symmetric binary relation 
${\mathcal S}_{\mathcal N}$ between agents such that $P\rel{S}_{\mathcal N}Q$ implies:
\begin{enumerate}
\item If $P \red P'$ then $Q \wred Q'$ and $P'\rel{S}_{\mathcal N} Q'$.
\item If $P\downarrow_{\mathcal N} x$, then $Q\Downarrow_{\mathcal N} x$.
\end{enumerate}
$P$ is ${\mathcal N}$-barbed bisimilar to $Q$, written
$P \wbbisim_{\mathcal N} Q$, if $P \rel{S}_{\mathcal N} Q$ for some ${\mathcal N}$-barbed bisimulation ${\mathcal S}_{\mathcal N}$.
\end{definition}

$\mathcal{R} \subseteq \pi \times \pi$

$P \mathcal{R} Q => \forall P'. P \red P' \Rightarrow \exists Q'. Q \red Q', P' \mathcal{R} Q'$

$P \vdash x \Rightarrow Q \vdash x$

\begin{mathpar}
  \inferrule*[lab=Out-barb]{x \nameeq y}{{y}!\langle{Q}\rangle \vdash x}
  \and
  \inferrule*[lab=Par-barb]{\mbox{$P\vdash x$ or $Q\vdash x$}}{\binpar{P}{Q} \vdash x}
\end{mathpar}

\subsubsection{Contexts}

One of the principle advantages of computational calculi like the
$\pi$-calculus is a well-defined notion of context,
contextual-equivalence and a correlation between
contextual-equivalence and notions of bisimulation. The notion of
context allows the decomposition of a process into (sub-)process and
its syntactic environment, its context. Thus, a context may be
thought of as a process with a ``hole'' (written $\Box$) in it. The
application of a context $M$ to a process $P$, written $M[P]$, is
tantamount to filling the hole in $M$ with $P$. In this paper we do
not need the full weight of this theory, but do make use of the notion
of context in the proof the main theorem. 

\begin{mathpar}
  \inferrule* [lab=summation] {} {{M_{M},M_{N}} \bc \Box \;|\; x.M_{A} \;|\; M_{M}+M_{N}}
  \and
  \inferrule* [lab=agent] {} {{M_{A}} \bc (\vec{x})M_{P} \;| \; \clift{P_0,\ldots,M_{P},\ldots,P_N}}
  \and \\
  \inferrule* [lab=process] {} {{M_{P}} \bc M_{N} \;| \;P|M_{P} }
\end{mathpar} 

\begin{mathpar}
  \inferrule* [lab=sychronization] {} {M_{N} \bc \Box \;|\; x?M_{F} \;|\; x!M_{C}}
  \and
  \inferrule* [lab=abstraction] {} {{M_{F}} \bc (x)M_{P} }
  \and
  \inferrule* [lab=concretion] {} {{M_{C}} \bc \langle M_{P} \rangle }
  \and \\
  \inferrule* [lab=process] {} {{M_{P}} \bc M_{N} \;| \;P|M_{P} }
\end{mathpar}

\begin{definition}[contextual application] Given a context $M$, and
  process $P$, we define the \emph{contextual application}, $M[P] :=
  M\{P/\Box\}$. That is, the contextual application of M to P is the
  substitution of $P$ for $\Box$ in $M$.
\end{definition}

$\meaningof{-} : L \to \mathcal{P}(\pi)$

\begin{mathpar}
  \inferrule* [lab=collection] {} {\meaningof{true} = \pi, \and \meaningof{~E} = \pi \setminus \meaningof{E}, \and \meaningof{E_{1} \& E_{2}} = \meaningof{E_{1}} \cap \meaningof{E_{2}}}
\end{mathpar}

\begin{mathpar}
  \inferrule* [lab=structure] {} {\meaningof{0} = \{ P \in \pi | P \equiv 0 \}, \and \\ \meaningof{E_1 | E_2} = \{ P \in \pi | P \equiv P_{1} | P_{2}, P_{1} \in \meaningof{E_{1}}, P_{2} \in \meaningof{E_2}\} }
\end{mathpar}

\begin{mathpar}
 \inferrule* [lab=behavior] {} {\meaningof{\langle a?b \rangle E} = \{ P \in \pi | P \equiv Q | u?(y)P', \\ \and \\\\ \and \\ \;\;\; u \in \meaningof{a}, \forall z.P'\{z/y\} \in \meaningof{E\{z/b\}}\}, \and \\ \meaningof{a!E} = \{ P \in \pi | P \equiv Q | x!\langle P' \rangle, x \in \meaningof{a} P' \in \meaningof{E}\} }
\end{mathpar}

\begin{mathpar}
 \inferrule* [lab=nominal] {} {\meaningof{\quotep{E}} = \{ \quotep{P} \in \quotep{\pi} | P \in \meaningof{E} \}, \and \meaningof{\quotep{P}} = \{ \quotep{Q} \in \quotep{\pi} | P \equiv Q \} \and \\ \meaningof{@\quotep{E}} = \{ P \in \pi | P \equiv @x, x \in \meaningof{E} \}}
\end{mathpar}

\begin{eqnarray*}
  \\
  \meaningof{-} : TS \to ST
\end{eqnarray*}

\begin{eqnarray*}
  \\
  L : TS \to ST
\end{eqnarray*}

\begin{eqnarray*}
  \\
  P \models E \iff P \in \meaningof{E}
\end{eqnarray*}

\begin{eqnarray*}
  P \approx_{L} Q \iff \forall E \in L. P \models E \iff Q \models E
\end{eqnarray*}

\begin{eqnarray*}
  P \approx_{K} Q
\end{eqnarray*}

\begin{eqnarray*}
  P \approx Q
\end{eqnarray*}

$\approx_{K} = \approx = \approx_{L}$

\subsubsection{Contextual duality}

Note that contexts extend the quotation operation to a family of
operations from processes to names. Given a context, $M$, we can
define a \emph{nominal context}, $\quotep{M}$ by $\quotep{M}[P] :=
\quotep{M[P]}$. To foreshadow what is to come we observe that these
operations enjoy a duality with processes very much like the duality
between vectors and maps from vectors to scalars.

Further, because the calculus is essentially higher-order, we have a
correspondence between contexts and processes. More specifically,
given a name $x$ and a context $M$ we can construct $M^{*}_{x}$ such
that 

\begin{mathpar}
  M^{*}_{x} | \lift{x}{P} \red M[P]
\end{mathpar}

namely,

\begin{mathpar}
  M^{*}_{x} := x?(u).M[\dropn{u}]
\end{mathpar}

The dependence of $M^{*}_{x}$ on a name makes it an abstraction, 

\begin{mathpar}
  M^{*} := (x)x?(u).M[\dropn{u}]
\end{mathpar}

\subsection{Additional notation}

It will sometimes be convenient to denote the process a name
quotes. We already have the notation $x = \quotep{P}$, but it will be
convenient to introduce an alternate notation, $\procn{x}$, when we
want to emphasize the connection to the use of the name. Note that, by
virtue of name equivalence, $\quotep{\procn{x}} \nameeq x$; so, the
notation is consistent with previous definitions.

Further, because names have structure it is possible to effect
substitutions on the basis of that structure. This means we need to
upgrade our notation for substitutions, which we accomplish by
adapting comprehension notation. Thus,

\begin{mathpar}
  P\{ y / x : x \in S \}
\end{mathpar}

is interpreted to mean the process derived from P by replacing (in a
capture-avoiding manner) each occurrence of $x$ in $S$ by $y$. For example,

\begin{mathpar}
  P\{ \quotep{\procn{x}|\procn{x}} / x : x \in \freenames{P} \}
\end{mathpar}

will replace each (occurrence) of a free name $x$ in $P$ by
$\quotep{\procn{x}|\procn{x}}$.

Also, we will avail ourselves of the notation $x^{L}$ and $x^{R}$ to
denote injections of a name into disjoint copies of the name
space. There are numerous ways to accomplish this. One example can be
found in \cite{MeredithR05}. This notation overloads to vectors of
names: $\vec{x}^{\pi} := (x_{i}^{\pi} \; : \; 0 \leq i < |\vec{x}| )$ where $\pi \in \{L,R\}$.

We also use $P^{\Box} := P|\Box$.

In \cite{MeredithR05} an interpretation of the new operator is
given. It turns out that there are several possible interpretations
all enjoying the requisite algebraic properties of the operator (see
\cite{milner91polyadicpi}). We will therefore make liberal use of
$(\nu\; \vec{x})P$.

% subsection the_syntax_and_semantics_of_the_notation_system (end)   

\input{qm2pi.qmops} 

\input{qm2pi.sterngerlach} 

\input{qm2pi.metric} 

% section concurrent_process_calculi (end)

%\input{qm2pi.proofsketch}

% section proof sketch (end)

%\input{qm2pi.slviaknots} 

% section spatial logic via knots (end)

\input{qm2pi.conclusion}

% section conclusion (end)

%\input{qm2pi.dtcodes} 

% section wiring algorithm (end)

\input{qm2pi.ack} 

% section acknowledgments (end)

\newpage


\bibliographystyle{plain}   
\bibliography{../../biblios/main.bib}

\input{qm2pi.rhodetails}

\end{document}

 

% section concurrent_process_calculi (end)

%\documentclass[12pt]{llncs}
%\documentclass{jktr}

\usepackage[pdftex]{hyperref}                   
\usepackage {listings}
\usepackage {mathpartir}
\usepackage{bcprules}
%\usepackage{listings}
                       
\usepackage{graphicx} 
%\usepackage[margins=2.5cm,nohead,nofoot]{geometry}
%\usepackage{geometry}
\usepackage{amsfonts}
\usepackage{amstext}
\usepackage{latexsym}
\usepackage{amssymb}
\usepackage{color}


%\include{myPreamble}
\include{qm2pi.local} 

%\ifpdf
%\usepackage[pdftex]{graphicx}
%\else
%\usepackage{graphicx}
%\fi

 % \ifpdf
%  \usepackage{pdfsync}
%  \if


%\title{Brief Article}
%\author{David F. Snyder}
%\author{L.G. Meredith}

%\address{Dept. of Math., Texas State University--San Marcos, San Marcos, TX 78666}
       
\pagestyle{empty}


\begin{document}

\lstset{language=[Objective]Caml,frame=shadowbox}

\input{qm2pi.front}

% section front matter (end)

\input{qm2pi.intro} 
 
% section introduction (end)

% \input{qm2pi.knotations} 

% section notation (end)

\input{qm2pi.process.calculi} 

% section concurrent_process_calculi_and_spatial_logics_ (end)
    
%\input{qm2pi.knots2pi} 

%\input{qm2pi.trefoil} 

%\input{qm2pi.mainthm} 

% subsection basic_interpretation (end)

%\input{qm2pi.rho.presentation} 
\subsection{The syntax and semantics of the notation system}\label{sub:the_syntax_and_semantics_of_the_notation_system} % (fold)

We now summarize a technical presentation of the calculus that
embodies our theory of dynamics. The typical presentation of such a
calculus follows the style of giving generators and relations on
them. The grammar, below, describing term constructors, freely
generates the set of processes, $\Proc$. This set is then quotiented
by a relation known as structural congruence and it is over this set
that the notion of dynamics is expressed. This presentation is
essentially that of \cite{MeredithR05} with the addition of
polyadicity and summation. For readability we have relegated some of
the technical subtleties to an appendix.

\subsubsection{Process grammar}\label{subsub:process_grammar}

\begin{mathpar}
  \inferrule* [lab=synchronization] {} {{M} \bc \pzero \;|\; x?F \;|\; x!C }
  \and
  \inferrule* [lab=abstraction] {} {{F} \bc (x)P}
  \and
  \inferrule* [lab=concretion] {} {{C} \bc \langle Q \rangle}
  \and
  \inferrule* [lab=process] {} {{P,Q} \bc M \;| \;P|Q \;|\; @{x}}
  \and
  \inferrule* [lab=name] {} {{x} \bc \quotep{P}}
\end{mathpar} 

Note that $\vec{x}$ (resp. $\vec{P}$) denotes a vector of names
(resp. processes) of length $|\vec{x}|$ (resp. $|\vec{P}|$). We adopt
the following useful abbreviations.

\begin{mathpar}
   x?(\vec{y}).P := x.(\vec{y})P \and  x\clift{\vec{P}} := x.\clift{\vec{P}}
   \and x!(y) := \lift{x}{\dropn{y}}
   \and \Pi_{i=0}^{n-1}P_i := P_0 | \ldots | P_{n-1}
\end{mathpar}

\subsubsection{Structural congruence}

\paragraph{Free and bound names and alpha-equivalence.} At the
core of structural equivalence is alpha-equivalence which identifies
process that are the same up to a change of variable. Formally, we
recognize the distinction between free and bound names. The free names
of a process, $\freenames{P}$, may be calculated recursively as
follows:

\begin{mathpar}
\freenames{\pzero} := \emptyset
  \and \\
  \freenames{x?(y).P} := \{ x \} \cup (\freenames{P} \setminus \{ y \})
  \and 
  \freenames{x!\langle P \rangle} := \{ x \} \cup \{ P \} 
  \and \\
  \freenames{P|Q} := \freenames{P} \cup \freenames{Q}
  \and \\
  \freenames{@{x}} := \{ x \}
\end{mathpar}

$\pi$
$\quotep{\pi}$

$\freenames{-} : \pi \to \mathcal{P}(\quotep{\pi})$

\begin{eqnarray*}
  \freenames{\pzero} & := & \emptyset \\
  \freenames{x?(y).P} & := & \{ x \} \cup (\freenames{P} \setminus \{ y \}) \\
  \freenames{x!\langle P \rangle} & := & \{ x \} \cup \{ P \} \\
  \freenames{P|Q} & := & \freenames{P} \cup \freenames{Q} \\
  \freenames{\dropn{x}} & := & \{ x \}
\end{eqnarray*}

The bound names of a process, $\boundnames{P}$, are those names occurring in $P$
that are not free. For example, in $x?(y).0$, the name $x$ is free, while $y$ is bound.

\begin{mathpar}
  \inferrule* [lab=monoidal-laws] {} { P|Q \equiv Q|P \and P|0 \equiv P \and P|(Q|R) \equiv (P|Q)|R }
\end{mathpar}

\begin{mathpar}
  \inferrule* [lab=alpha-equivalence] {} { (x)P \equiv (y)P\{y/x\} \and y \not\in \freenames{P} }
\end{mathpar}

\begin{definition}
Then two processes, $P,Q$, are alpha-equivalent if $P = Q\{\vec{y}/\vec{x}\}$ for
some $\vec{x} \in \boundnames{Q},\vec{y} \in \boundnames{P}$, where $Q\{\vec{y}/\vec{x}\}$
denotes the capture-avoiding substitution of $\vec{y}$ for $\vec{x}$ in $Q$.
\end{definition}

\begin{definition}
  The {\em structural congruence} \cite{SangiorgiWalker} , $\equiv$,
  between processes is the least congruence containing
  alpha-equivalence, satisfying the abelian monoid laws
  (associativity, commutativity and $\pzero$ as identity) for parallel
  composition $|$ and for summation $+$.
\end{definition}

\subsection{Name equivalence}

We take name equivalence, written $\nameeq$, to be the smallest
equivalence relation generated by the following rules.

\begin{mathpar}
\inferrule*[lab=Quote-drop]
{ }
{ \quotep{@{x}} \nameeq x }

\inferrule*[lab=Struct-equiv]
{ P \scong Q }
{ \quotep{P} \nameeq \quotep{Q} }
\end{mathpar}

The astute reader will have noticed that the mutual recursion of names
and processes imposes a mutual recursion on alpha-equivalence and
structural equivalence via name-equivalence. Fortunately, all of this
works out pleasantly and we may calculate in the natural way, free of
concern. The reader interested in the details is referred to the
appendix \ref{appendix:rho_details}.

\subsection{Substitution}

We use $\Proc$ for the set of processes, $\QProc$ for the set of
names, and $\id{\{}\vec{y} / \vec{x} \id{\}}$ to denote partial maps,
$s : \QProc \rightarrow \QProc$. A map, $s$ lifts, uniquely, to a map
on process terms, $\widehat{s} : \Proc \rightarrow \Proc$ by the
following equations.

\begin{mathpar}
  (0) \psubstp{Q}{P} := 0 \\
  (R \juxtap S) \psubstp{Q}{P}
  :=    
  (R)\psubstp{Q}{P} \juxtap (S) \psubstp{Q}{P} \\
  (x?(y).R) \psubstp{Q}{P}    
  :=    
  (x)\substp{Q}{P} (z)\concat( (R \psubstn{z}{y}) \psubstp{Q}{P} ) \\
  (\lift{x}{R}) \psubstp{Q}{P}  
  :=
  \lift{(x)\substp{Q}{P}}{ R \psubstp{Q}{P} } \\
%   (\dropn{x})  \psubstp{Q}{P}       
%   := 
%   \left\{ 
%     \begin{array}{ccc} 
%       \dropn{\quotep{Q}} & & x \nameeq \quotep{P} \\
%       \dropn{x} & & otherwise \\
%     \end{array}
%   \right. 
  (\dropn{x})  \psubstp{Q}{P}       
  := 
  \left\{ 
    \begin{array}{ccc} 
      Q & & x \nameeq \quotep{P} \\
      \dropn{x} & & otherwise \\
    \end{array}
  \right.
\end{mathpar}
 

where

\begin{eqnarray}
  (x)\id{\{} \lpquote Q \rpquote / \lpquote P \rpquote \id{\}}            = 
  \left\{ 
    \begin{array}{ccc}
      \lpquote Q \rpquote & & x \nameeq \lpquote P \rpquote \\
      x & & otherwise \\
    \end{array}
  \right. \nonumber
\end{eqnarray}

and $z$ is chosen distinct from $\quotep{P}$, $\quotep{Q}$, the free
names in $Q$, and all the names in $R$. Our $\alpha$-equivalence will
be built in the standard way from this substitution.

\begin{remark}\label{rem:no_self_referential_names}
  One consequence of these definitions is that $\forall P. \quotep{P}
  \not\in \freenames{P}$.
\end{remark}

\subsection{ Dynamic quote: an example }

Anticipating something of what's to come, consider applying the
substitution, $\widehat{\id{\{}u / z \id{\}}}$, to the following pair
of processes, $\lift{w}{y!(z)}$ and $w[ \lpquote y!(z) \rpquote ]$.

\begin{eqnarray}
	\lift{w}{y!(z)}\widehat{\id{\{}u / z \id{\}}}
		& = &
		\lift{w}{y!(u)} \nonumber\\
	w[ \lpquote y!(z) \rpquote ] \widehat{ \id{\{}u / z \id{\}} }
		& = &
		w[ \lpquote y!(z) \rpquote ] \nonumber
\end{eqnarray}

Because the body of the process between quotes is impervious to
substitution, we get radically different answers. In fact, by
examining the first process in an input context,
e.g. $x?(z).\lift{w}{y!(z)}$, we see that the process under the lift
operator may be shaped by prefixed inputs binding a name inside it. In
this sense, the lift operator will be seen as a way to dynamically
construct processes before reifying them as names.

Finally equipped with these standard features we can present the
dynamics of the calculus.

\subsubsection{Operational semantics} 

Finally, we introduce the computational dynamics. What marks these
algebras as distinct from other more traditionally studied algebraic
structures, e.g. vector spaces or polynomial rings, is the manner in
which dynamics is captured. In traditional structures, dynamics is typically
expressed through morphisms between such structures, as in linear maps
between vector spaces or morphisms between rings. In algebras
associated with the semantics of computation, the dynamics is
expressed as part of the algebraic structure itself, through a
reduction reduction relation typically denoted by $\red$. Below, we
give a recursive presentation of this relation for the calculus used
in the encoding.

$\red \subseteq \pi \times \pi$
$\red : \pi \to \mathcal{P}(\pi)$

\begin{mathpar}
  \inferrule* [lab=Comm] { \textsf{match}( x_{src}, x_{trgt} ) } { x_{trgt}?(y)P \; | \; x_{src}!\langle {Q} \rangle \red P\{\quotep{Q}/y}\} }
  \and \\
  \inferrule* [lab=Par] {{P} \red {P}'} {{{P} | {Q}} \red {{P}' | {Q}}}
  \and
  \inferrule* [lab=Equiv]{{{P} \scong {P}'} \andalso {{P}' \red {Q}'} \andalso {{Q}' \scong {Q}}}{{P} \red {Q}}
\end{mathpar}

\begin{eqnarray*}
  match_{\equiv} (\quotep{P},\quotep{Q}) & := & P \equiv Q \\
  match_{\dagger}(\quotep{P},\quotep{Q}) & := & \forall R. P|Q \red^{*} R => R \red^{*} 0 \\
  match_{K}(\quotep{P},\quotep{Q}) & := & K \mbox{ for some context } K
\end{eqnarray*}

$u?(x)P | u!\langle Q \rangle \red P\{\quotep{Q}/x\}$

%We write $\wred$ for $\red^*$, and $P\red$ if $\exists Q $ such that $ P \red Q$.
We write $P\red$ if $\exists Q $ such that $ P \red Q$ and $P\not\red$, otherwise.

\section{Replication}

As mentioned before, it is known that replication (and hence
recursion) can be implemented in a higher-order process algebra
\cite{SangiorgiWalker}. As our first example of calculation with the
machinery thus far presented we give the construction explicitly in
the {\rhoc}.

\begin{eqnarray}
	D_{x} & := & \prefix{x}{y}{(\binpar{\outputp{x}{y}}{@{y}})} \nonumber\\
	\bangp_{x}{P} & := & \binpar{{x}!\langle{\binpar{D_{x}}{P}}\rangle}{D_{x}} \nonumber
\end{eqnarray}

\begin{eqnarray}
	\bangp_{x}{P} & & \nonumber\\
	=
	& {x}!\langle{(\prefix{x}{y}{(\outputp{x}{y} | @{y})) | P}}\rangle 
	      | \prefix{x}{y}{(\outputp{x}{y} | @{y})} & \nonumber\\
	\red
	& (\outputp{x}{y} | @{y})\substn{\quotep{(\prefix{x}{y}{(@{y} | \outputp{x}{y})) | P}}}{y} & \nonumber\\
	=
	& \outputp{x}{\quotep{(\prefix{x}{y}{(\outputp{x}{y} | @{y})) | P}}}
	  | {(\prefix{x}{y}{(\outputp{x}{y} | @{y})) | P}} & \nonumber\\
	\red
	& \ldots & \nonumber\\
	\red^*
	& P | P | \ldots & \nonumber
\end{eqnarray}

Of course, this encoding, as an implementation, runs away, unfolding
$\bangp{P}$ eagerly. A lazier and more implementable replication
operator, restricted to input-guarded processes, may be obtained as follows.

\begin{eqnarray}
\bangp{\prefix{u}{v}{P}} 
	:= 
	\binpar{\lift{x}{\prefix{u}{v}{(\binpar{D(x)}{P})}}}{D(x)} \nonumber
\end{eqnarray}

\begin{remark}
  Note that the lazier definition still does not deal with summation
  or mixed summation (i.e. sums over input and output). The reader is
  invited to construct definitions of replication that deal with these
  features. 

  Further, the definitions are parameterized in a name, $x$. Can you,
  gentle reader, make a definition that eliminates this parameter and
  guarantees no accidental interaction between the replication
  machinery and the process being replicated -- i.e. no accidental
  sharing of names used by the process to get its work done and the
  name(s) used by the replication to effect copying. This latter
  revision of the definition of replication is crucial to obtaining
  the expected identity $!!P \sim !P$.
\end{remark}

\begin{remark}\label{rem:paradoxical_combinator}
  The reader familiar with the lambda calculus will have noticed the
  similarity between $D$ and the paradoxical combinator.

  [Ed. note: the existence of this seems to suggest we have to be more
  restrictive on the set of processes and names we admit if we are to
  support no-cloning.]
\end{remark}

\subsubsection{Bisimulation}

The computational dynamics gives rise to another kind of equivalence,
the equivalence of computational behavior. As previously mentioned
this is typically captured \emph{via} some form of bisimulation.

% The notion we use in this paper is weak barbed bisimulation
% \cite{milner91polyadicpi}.

The notion we use in this paper is derived from weak barbed
bisimulation \cite{milner91polyadicpi}. 

\begin{definition}
An \emph{observation relation}, $\downarrow_{\mathcal N}$, over a set
of names, $\mathcal N$, is the smallest relation satisfying the rules
below.

\infrule[Out-barb]{y \in {\mathcal N}, \; x \nameeq y}
		  {\outputp{x}{v} \downarrow_{\mathcal N} x}
\infrule[Par-barb]{\mbox{$P\downarrow_{\mathcal N} x$ or $Q\downarrow_{\mathcal N} x$}}
		  {\binpar{P}{Q} \downarrow_{\mathcal N} x}

We write $P \Downarrow_{\mathcal N} x$ if there is $Q$ such that 
$P \wred Q$ and $Q \downarrow_{\mathcal N} x$.
\end{definition}

\begin{definition}
%\label{def.bbisim}
An  ${\mathcal N}$-\emph{barbed bisimulation} over a set of names, ${\mathcal N}$, is a symmetric binary relation 
${\mathcal S}_{\mathcal N}$ between agents such that $P\rel{S}_{\mathcal N}Q$ implies:
\begin{enumerate}
\item If $P \red P'$ then $Q \wred Q'$ and $P'\rel{S}_{\mathcal N} Q'$.
\item If $P\downarrow_{\mathcal N} x$, then $Q\Downarrow_{\mathcal N} x$.
\end{enumerate}
$P$ is ${\mathcal N}$-barbed bisimilar to $Q$, written
$P \wbbisim_{\mathcal N} Q$, if $P \rel{S}_{\mathcal N} Q$ for some ${\mathcal N}$-barbed bisimulation ${\mathcal S}_{\mathcal N}$.
\end{definition}

$\mathcal{R} \subseteq \pi \times \pi$

$P \mathcal{R} Q => \forall P'. P \red P' \Rightarrow \exists Q'. Q \red Q', P' \mathcal{R} Q'$

$P \vdash x \Rightarrow Q \vdash x$

\begin{mathpar}
  \inferrule*[lab=Out-barb]{x \nameeq y}{{y}!\langle{Q}\rangle \vdash x}
  \and
  \inferrule*[lab=Par-barb]{\mbox{$P\vdash x$ or $Q\vdash x$}}{\binpar{P}{Q} \vdash x}
\end{mathpar}

\subsubsection{Contexts}

One of the principle advantages of computational calculi like the
$\pi$-calculus is a well-defined notion of context,
contextual-equivalence and a correlation between
contextual-equivalence and notions of bisimulation. The notion of
context allows the decomposition of a process into (sub-)process and
its syntactic environment, its context. Thus, a context may be
thought of as a process with a ``hole'' (written $\Box$) in it. The
application of a context $M$ to a process $P$, written $M[P]$, is
tantamount to filling the hole in $M$ with $P$. In this paper we do
not need the full weight of this theory, but do make use of the notion
of context in the proof the main theorem. 

\begin{mathpar}
  \inferrule* [lab=summation] {} {{M_{M},M_{N}} \bc \Box \;|\; x.M_{A} \;|\; M_{M}+M_{N}}
  \and
  \inferrule* [lab=agent] {} {{M_{A}} \bc (\vec{x})M_{P} \;| \; \clift{P_0,\ldots,M_{P},\ldots,P_N}}
  \and \\
  \inferrule* [lab=process] {} {{M_{P}} \bc M_{N} \;| \;P|M_{P} }
\end{mathpar} 

\begin{mathpar}
  \inferrule* [lab=sychronization] {} {M_{N} \bc \Box \;|\; x?M_{F} \;|\; x!M_{C}}
  \and
  \inferrule* [lab=abstraction] {} {{M_{F}} \bc (x)M_{P} }
  \and
  \inferrule* [lab=concretion] {} {{M_{C}} \bc \langle M_{P} \rangle }
  \and \\
  \inferrule* [lab=process] {} {{M_{P}} \bc M_{N} \;| \;P|M_{P} }
\end{mathpar}

\begin{definition}[contextual application] Given a context $M$, and
  process $P$, we define the \emph{contextual application}, $M[P] :=
  M\{P/\Box\}$. That is, the contextual application of M to P is the
  substitution of $P$ for $\Box$ in $M$.
\end{definition}

$\meaningof{-} : L \to \mathcal{P}(\pi)$

\begin{mathpar}
  \inferrule* [lab=collection] {} {\meaningof{true} = \pi, \and \meaningof{~E} = \pi \setminus \meaningof{E}, \and \meaningof{E_{1} \& E_{2}} = \meaningof{E_{1}} \cap \meaningof{E_{2}}}
\end{mathpar}

\begin{mathpar}
  \inferrule* [lab=structure] {} {\meaningof{0} = \{ P \in \pi | P \equiv 0 \}, \and \\ \meaningof{E_1 | E_2} = \{ P \in \pi | P \equiv P_{1} | P_{2}, P_{1} \in \meaningof{E_{1}}, P_{2} \in \meaningof{E_2}\} }
\end{mathpar}

\begin{mathpar}
 \inferrule* [lab=behavior] {} {\meaningof{\langle a?b \rangle E} = \{ P \in \pi | P \equiv Q | u?(y)P', \\ \and \\\\ \and \\ \;\;\; u \in \meaningof{a}, \forall z.P'\{z/y\} \in \meaningof{E\{z/b\}}\}, \and \\ \meaningof{a!E} = \{ P \in \pi | P \equiv Q | x!\langle P' \rangle, x \in \meaningof{a} P' \in \meaningof{E}\} }
\end{mathpar}

\begin{mathpar}
 \inferrule* [lab=nominal] {} {\meaningof{\quotep{E}} = \{ \quotep{P} \in \quotep{\pi} | P \in \meaningof{E} \}, \and \meaningof{\quotep{P}} = \{ \quotep{Q} \in \quotep{\pi} | P \equiv Q \} \and \\ \meaningof{@\quotep{E}} = \{ P \in \pi | P \equiv @x, x \in \meaningof{E} \}}
\end{mathpar}

\begin{eqnarray*}
  \\
  \meaningof{-} : TS \to ST
\end{eqnarray*}

\begin{eqnarray*}
  \\
  L : TS \to ST
\end{eqnarray*}

\begin{eqnarray*}
  \\
  P \models E \iff P \in \meaningof{E}
\end{eqnarray*}

\begin{eqnarray*}
  P \approx_{L} Q \iff \forall E \in L. P \models E \iff Q \models E
\end{eqnarray*}

\begin{eqnarray*}
  P \approx_{K} Q
\end{eqnarray*}

\begin{eqnarray*}
  P \approx Q
\end{eqnarray*}

$\approx_{K} = \approx = \approx_{L}$

\subsubsection{Contextual duality}

Note that contexts extend the quotation operation to a family of
operations from processes to names. Given a context, $M$, we can
define a \emph{nominal context}, $\quotep{M}$ by $\quotep{M}[P] :=
\quotep{M[P]}$. To foreshadow what is to come we observe that these
operations enjoy a duality with processes very much like the duality
between vectors and maps from vectors to scalars.

Further, because the calculus is essentially higher-order, we have a
correspondence between contexts and processes. More specifically,
given a name $x$ and a context $M$ we can construct $M^{*}_{x}$ such
that 

\begin{mathpar}
  M^{*}_{x} | \lift{x}{P} \red M[P]
\end{mathpar}

namely,

\begin{mathpar}
  M^{*}_{x} := x?(u).M[\dropn{u}]
\end{mathpar}

The dependence of $M^{*}_{x}$ on a name makes it an abstraction, 

\begin{mathpar}
  M^{*} := (x)x?(u).M[\dropn{u}]
\end{mathpar}

\subsection{Additional notation}

It will sometimes be convenient to denote the process a name
quotes. We already have the notation $x = \quotep{P}$, but it will be
convenient to introduce an alternate notation, $\procn{x}$, when we
want to emphasize the connection to the use of the name. Note that, by
virtue of name equivalence, $\quotep{\procn{x}} \nameeq x$; so, the
notation is consistent with previous definitions.

Further, because names have structure it is possible to effect
substitutions on the basis of that structure. This means we need to
upgrade our notation for substitutions, which we accomplish by
adapting comprehension notation. Thus,

\begin{mathpar}
  P\{ y / x : x \in S \}
\end{mathpar}

is interpreted to mean the process derived from P by replacing (in a
capture-avoiding manner) each occurrence of $x$ in $S$ by $y$. For example,

\begin{mathpar}
  P\{ \quotep{\procn{x}|\procn{x}} / x : x \in \freenames{P} \}
\end{mathpar}

will replace each (occurrence) of a free name $x$ in $P$ by
$\quotep{\procn{x}|\procn{x}}$.

Also, we will avail ourselves of the notation $x^{L}$ and $x^{R}$ to
denote injections of a name into disjoint copies of the name
space. There are numerous ways to accomplish this. One example can be
found in \cite{MeredithR05}. This notation overloads to vectors of
names: $\vec{x}^{\pi} := (x_{i}^{\pi} \; : \; 0 \leq i < |\vec{x}| )$ where $\pi \in \{L,R\}$.

We also use $P^{\Box} := P|\Box$.

In \cite{MeredithR05} an interpretation of the new operator is
given. It turns out that there are several possible interpretations
all enjoying the requisite algebraic properties of the operator (see
\cite{milner91polyadicpi}). We will therefore make liberal use of
$(\nu\; \vec{x})P$.

% subsection the_syntax_and_semantics_of_the_notation_system (end)   

\input{qm2pi.qmops} 

\input{qm2pi.sterngerlach} 

\input{qm2pi.metric} 

% section concurrent_process_calculi (end)

%\input{qm2pi.proofsketch}

% section proof sketch (end)

%\input{qm2pi.slviaknots} 

% section spatial logic via knots (end)

\input{qm2pi.conclusion}

% section conclusion (end)

%\input{qm2pi.dtcodes} 

% section wiring algorithm (end)

\input{qm2pi.ack} 

% section acknowledgments (end)

\newpage


\bibliographystyle{plain}   
\bibliography{../../biblios/main.bib}

\input{qm2pi.rhodetails}

\end{document}



% section proof sketch (end)

%\section{Unlikely characters: spatial logic for
  knots}\label{sub:characteristic_formulae} % (fold)

Associated to the mobile process calculi are a family of logics known
as the Hennessy-Milner logics. These logics typically enjoy a
semantics interpreting formulae as sets of processes that when
factored through the encoding outlined above allows an identification
of classes of knots with logical formulae. In the context of this
encoding the sub-family known as the spatial logics \cite{CairesC03}
\cite{CairesC04} \cite{Caires04} are of particular interest providing
several important features for expressing and reasoning about
properties (i.e. classes) of knots. We hint here at how this may be done.

%\begin{description}
%\item [structural connectives] 
\subsubsection{Structural connectives} The spatial logics enjoy
structural connectives corresponding, at the logical level, to the
parallel composition ($P | Q$) and new name ($(\nu \; x)P$)
connectives for processes. As illustrated in the examples below, these
connectives are extremely expressive given the shape of our encoding.
%\item [decideable satisfaction]

\subsubsection{Decideable satisfaction}
In \cite{Caires04} the satisfaction relation is shown to be decideable
for a rich class of processes. It further turns out that the image of
the our encoding is a proper subset of that class. This result
provides the basis for an algorithm by which to search for knots
enjoying a given property.
%\item [characteristic formulae]

\subsubsection{Characteristic formulae}
In the same paper \cite{Caires04} , Caires presents a means of calculating
characteristic formulae, selecting equivalence classes of processes
up to a pre--specified depth limit on the support set of names. Composed with our
encoding, this characteristic formula can be used to select
characteristic formulae for knots.
%\end{description}

\subsubsection{Spatial logic formulae}

The grammar below (segmented for comprehension) summarizes the syntax
of spatial logic formulae. We employ illustrative examples in the
sequel to provide an intuitive understanding of their meaning
referring the reader to \cite{Caires04} for a more detailed explication
of the semantics.

\begin{mathpar}
  \inferrule* [lab=boolean] {} {{A,B} \bc T \;|\; \neg A \;|\; A \wedge B \;|\; \eta = \eta'}
  \and
  \inferrule* [lab=spatial] {} {|\; \pzero \;|\; A | B \;|\; x \text{\textregistered} A \;|\; \forall x . A \;|\;  H x . A}
  \and
  \inferrule* [lab=behavioral] {} {|\; \alpha . A}
  \and 
  \inferrule* [lab=recursion] {} {|\; X(\vec{u}) \;|\; \mu X(\vec{u}) . A}
  \and
  \inferrule* [lab=action] {} {\alpha \bc \langle x?(\vec{y}) \rangle \;|\; \langle x!(\vec{y}) \rangle \;|\; \langle \tau \rangle}
  \and 
  \inferrule* [lab=name] {} {\eta \bc x \;|\; \tau}
\end{mathpar} 

% subsection characteristic_formulae (end)   	 

\subsection{Example formulae}\label{sub:example_formulae_} % (fold)

\subsubsection{Crossing as formula.}
% 
% \begin{align*}
%   \frac{d}{dx} \sin x &= \cos x 
%   & \frac{d}{dx} e^x &= e^x \\
%   \frac{d}{dx} \cos x &= - \sin x 
%   & \frac{d}{dx} \log x &= \frac{1}{x} \\
% \end{align*} 

\begin{align*}
 \mu C(x_{0},x_{1},y_{0},y_{1},u).&(\langle x_{0}?(z) \rangle(\langle u! \rangle\langle y_{1}!z \rangle C(x_{0},x_{1},y_{0},y_{1},u)) & \\
  & \wedge \langle y_{1}?(z) \rangle (\langle u! \rangle \langle x_{0}!z \rangle C(x_{0},x_{1},y_{0},y_{1},u)) & \\
  & \wedge \langle x_{1}?(z) \rangle (\langle u? \rangle \langle y_{0}!z \rangle C(x_{0},x_{1},y_{0},y_{1},u)) & \\
  & \wedge \langle y_{0}?(z) \rangle (\langle u? \rangle \langle x_{1}!z \rangle C(x_{0},x_{1},y_{0},y_{1},u))) &
\end{align*}

The lexicographical similarity between the shape of this formulae and
the shape of definition of the process representing a crossing reveals
the intuitive meaning of this formulae. It describes the capabilities
of a process that has the right to represent a crossing. For example
it picks out processes that may perform an input on the port $x_0$ in
its initial menu of capabilities. What differentiates the formula
from the process, however, is that the crossing process is the
smallest candidate to satisfy the formula. Infinitely many other
processes -- with internal behavior hidden behind this interface, so
to speak -- also satisfy this formula. Even this simple formula,
then, can be seen to open a new view onto knots, providing a
computational interpretation of \emph{virtual} knots.

Note that this formula is derived by hand. A similar formula can be
derived by employing Caires' calculation of characteristic formula
\cite{Caires04} to the process representing a crossing. In light of
this discussion, we let
$\meaningof{C}_{\phi}(x0,x1,y0,y1,u)$ denote a formula specifying the
dynamics we wish to capture of a crossing. To guarantee we preserve
the shape of the interface and minimal semantics we demand that
$\meaningof{C}_{\phi}(x0,x1,y0,y1,u) \Rightarrow
\textbf{C}(x0,x1,y0,y1,u)$ where $\textbf{C}(x0,x1,y0,y1,u)$ denotes
the formula above.
                            
\subsubsection{Crossing number constraints.}
The moral content of the context lemma (Lemma \ref{context}) is that the notion of
``locality'' in the Reidemeister moves is effectively captured by the
parallel composition operator of the process calculus. This intuition
extends through the logic. Given a formula,
$\meaningof{C}_{\phi}(x0,x1,y0,y1,u)$, we can use the structural
connectives to specify constraints on crossing numbers, such as at
least $n$ crossings, or exactly $n$ crossings.
\begin{mathpar}
  \inferrule* [lab=at-least-n] {} { K^{\geq n}_{\phi}(\vec{xs},\vec{ys}) := \Pi_{i=0}^{n-1} Hu . \meaningof{C}_{\phi}(xs_i,ys_i,u) | T }
  \and 
  \inferrule* [lab=exactly-n] {} { K^{= n}_{\phi}(\vec{xs},\vec{ys}) := \Pi_{i=0}^{n-1} Hu . \meaningof{C}_{\phi}(xs_i,ys_i,u) | \neg (\forall x_0,y_0,x_1,y_1,u . \meaningof{C}_{\phi}(x_0,y_0,x_1,y_1,u) | T) }
\end{mathpar}

To round out this section, recall that the encoding of an $n$-crossing
knot decomposes into a parallel composition of $n$ \emph{copies} of a
crossing process together with a wiring harness. To specify different
knot classes with the same crossing number amounts to specifying
logical constraints on the wiring harness. In the interest of space,
we defer examples to a forthcoming paper. Suffice it to say that both
the conditions ``alternating knot'' and ``contains the tangle
corresponding to 5/3'' are expressible. For example, it is possible to
calculate the characteristic formula of a process corresponding to the
tangle 5/3 and conjoin it into the classifying formula via the
composition connective of the logic.

Finally, we wish to observe that it is entirely within reason to
contemplate a more domain-specific version of spatial logic tailored
to the shape of processes in the image of the encoding. Such a
domain-specific logic would have a better claim to the title formal
language of knot properties.

% subsection example_formulae_ (end)

% section knots_as_processes (end) 

% section spatial logic via knots (end)

\section{Conclusions and future work}

\paragraph{Testing physical space}
You, gentle reader, may wonder why of all the theorems to be proved
given this set up we pick the one above. In some sense it's hardly
central to quantum mechanics. We see it as central in the sense that
it firmly establishes a notion of physical space arising from a notion
of the equivalence of behavior. Relating bisimulation to a metric is a
big step forward, but one is faced with interpreting the relationship
of that metric space to something more physical. Quantum mechanical
notions of ``physical'' space are still far from intuitive, but by
relating this idea of distance as testing to calculations that predict
physical circumstances we are making a not insignificant step forward
toward an understanding of the physical space we inhabit as
essentially dynamic.

\paragraph{Effectivity and simulation}
One of the observations we have yet to make is that the entire program
spelled out here is effective. We have built various interpreters for
the reflective calculus at work in this interpretation. In principle,
then, we can simulate quantum mechanics on a computer. The place where
the simulation may lose fidelity is the infinitely branching summation
for the annihilator.

In this connection i also want to point out that the evaluation style
calculation of the inner product puts the non-determinism of the
summation right at the heart of measurement. This suggests that
Milner's original reduction-based formulation of the dynamics of his
calculi in terms of sums was not just notationally suggestive of a
notion of measure-and-continue but captured some significant part of
the physics.

\paragraph{Quantum continuations}
In light of this last observation i want to point out that the
predominant account of quantum mechanics is missing a key aspect of a
truly compositional story of the physical situation. In a real lab,
when a measurement is made the observation can be made to feed into
another device that then makes another measurement conditioned on the
results of the first. This means that after the superposition was
collapsed the entire experimental set up remained in
superposition. While QM offers a means of writing this down it doesn't
quite line up well with the well-trodden formulation of computation
and continuation that we see so succinctly expressed in Milner's
calculi. This suggests that there might be advantages to this account
of dynamics waiting to be explored.

\paragraph{Quantum logic}
In this connection, we also note that by virtue of having the
Hennessy-Milner construction, we can pull the construction through the
interpretation of QM. This gives us a natural candidate for a quantum
logic that enjoys an extremely tight connection with it's domain of
interpretation, making the construction much less ad hoc (rather it is
the image of functor!).

\paragraph{Quantum probabiity}
i have questions about the basis of the interpretation of inner
product as probability amplitude. In particular, using which
axiomatization of probability theory does the notion of probability
amplitude earn the right to be so dubbed? In other words, where is the
proof that the operation for calculating a probability amplitude (and
then squaring) satisfies the axioms of what it means to calculate a
probability? Even if such a proof exists (i have yet to find it in the
literature), i wonder if it might not be possible to turn things on
their heads. Can we view the calculation of the probability amplitude
as an axiomatization of probability? If so, then the definition we
give for calculating probability amplitude may provide the basis for
an \emph{effective} theory of probability.

\paragraph{Quantum vs ``biological'' information}
Finally, i want to conclude with a more philosophical observation. At
a recent workshop in which QM was a predominant topic i noticed
something about quantum information. The speaker was giving a riveting
discussion of axiomatic QM and showing how properties of ``no
cloning'' and ``no deleting'' emerged as consequences of the
axiomatization. Theorems of this form are necessary to give us a sense
of confidence that our axioms characterize the physical theory. What
struck me, though, was that if quantum information is neither erasable
nor replicable it is markedly different from \emph{life}. Two of the
things we know about life is that

\begin{itemize}
  \item it ends;
  \item to gain some measure of persistence, to transcend it's
    finitude it is imminently copyable.
\end{itemize}

Both of these qualities are summarized succinctly in the aphorism: all
flesh is grass. For me these two kinds of ``information'' -- call them
quantum and biological -- are end points on a spectrum of strategies
for persistence. At one end, we have those curious entities that enjoy
uniqueness and permanence; at the other, we have those who in the face
of a certain end and an uncertain present make a go of passing
something on. To me one of the more remarkable aspects of the latter
strategy is that in the presence of noise (and certain features of
copying) we get a kind of dynamism, a chance for improvement against a
given persistent condition.

% subsection other_calculi_other_bisimulations_and_geometry_as_behavior (end)




% section conclusion (end)

%\documentclass[12pt]{llncs}
%\documentclass{jktr}

\usepackage[pdftex]{hyperref}                   
\usepackage {listings}
\usepackage {mathpartir}
\usepackage{bcprules}
%\usepackage{listings}
                       
\usepackage{graphicx} 
%\usepackage[margins=2.5cm,nohead,nofoot]{geometry}
%\usepackage{geometry}
\usepackage{amsfonts}
\usepackage{amstext}
\usepackage{latexsym}
\usepackage{amssymb}
\usepackage{color}


%\include{myPreamble}
\include{qm2pi.local} 

%\ifpdf
%\usepackage[pdftex]{graphicx}
%\else
%\usepackage{graphicx}
%\fi

 % \ifpdf
%  \usepackage{pdfsync}
%  \if


%\title{Brief Article}
%\author{David F. Snyder}
%\author{L.G. Meredith}

%\address{Dept. of Math., Texas State University--San Marcos, San Marcos, TX 78666}
       
\pagestyle{empty}


\begin{document}

\lstset{language=[Objective]Caml,frame=shadowbox}

\input{qm2pi.front}

% section front matter (end)

\input{qm2pi.intro} 
 
% section introduction (end)

% \input{qm2pi.knotations} 

% section notation (end)

\input{qm2pi.process.calculi} 

% section concurrent_process_calculi_and_spatial_logics_ (end)
    
%\input{qm2pi.knots2pi} 

%\input{qm2pi.trefoil} 

%\input{qm2pi.mainthm} 

% subsection basic_interpretation (end)

%\input{qm2pi.rho.presentation} 
\subsection{The syntax and semantics of the notation system}\label{sub:the_syntax_and_semantics_of_the_notation_system} % (fold)

We now summarize a technical presentation of the calculus that
embodies our theory of dynamics. The typical presentation of such a
calculus follows the style of giving generators and relations on
them. The grammar, below, describing term constructors, freely
generates the set of processes, $\Proc$. This set is then quotiented
by a relation known as structural congruence and it is over this set
that the notion of dynamics is expressed. This presentation is
essentially that of \cite{MeredithR05} with the addition of
polyadicity and summation. For readability we have relegated some of
the technical subtleties to an appendix.

\subsubsection{Process grammar}\label{subsub:process_grammar}

\begin{mathpar}
  \inferrule* [lab=synchronization] {} {{M} \bc \pzero \;|\; x?F \;|\; x!C }
  \and
  \inferrule* [lab=abstraction] {} {{F} \bc (x)P}
  \and
  \inferrule* [lab=concretion] {} {{C} \bc \langle Q \rangle}
  \and
  \inferrule* [lab=process] {} {{P,Q} \bc M \;| \;P|Q \;|\; @{x}}
  \and
  \inferrule* [lab=name] {} {{x} \bc \quotep{P}}
\end{mathpar} 

Note that $\vec{x}$ (resp. $\vec{P}$) denotes a vector of names
(resp. processes) of length $|\vec{x}|$ (resp. $|\vec{P}|$). We adopt
the following useful abbreviations.

\begin{mathpar}
   x?(\vec{y}).P := x.(\vec{y})P \and  x\clift{\vec{P}} := x.\clift{\vec{P}}
   \and x!(y) := \lift{x}{\dropn{y}}
   \and \Pi_{i=0}^{n-1}P_i := P_0 | \ldots | P_{n-1}
\end{mathpar}

\subsubsection{Structural congruence}

\paragraph{Free and bound names and alpha-equivalence.} At the
core of structural equivalence is alpha-equivalence which identifies
process that are the same up to a change of variable. Formally, we
recognize the distinction between free and bound names. The free names
of a process, $\freenames{P}$, may be calculated recursively as
follows:

\begin{mathpar}
\freenames{\pzero} := \emptyset
  \and \\
  \freenames{x?(y).P} := \{ x \} \cup (\freenames{P} \setminus \{ y \})
  \and 
  \freenames{x!\langle P \rangle} := \{ x \} \cup \{ P \} 
  \and \\
  \freenames{P|Q} := \freenames{P} \cup \freenames{Q}
  \and \\
  \freenames{@{x}} := \{ x \}
\end{mathpar}

$\pi$
$\quotep{\pi}$

$\freenames{-} : \pi \to \mathcal{P}(\quotep{\pi})$

\begin{eqnarray*}
  \freenames{\pzero} & := & \emptyset \\
  \freenames{x?(y).P} & := & \{ x \} \cup (\freenames{P} \setminus \{ y \}) \\
  \freenames{x!\langle P \rangle} & := & \{ x \} \cup \{ P \} \\
  \freenames{P|Q} & := & \freenames{P} \cup \freenames{Q} \\
  \freenames{\dropn{x}} & := & \{ x \}
\end{eqnarray*}

The bound names of a process, $\boundnames{P}$, are those names occurring in $P$
that are not free. For example, in $x?(y).0$, the name $x$ is free, while $y$ is bound.

\begin{mathpar}
  \inferrule* [lab=monoidal-laws] {} { P|Q \equiv Q|P \and P|0 \equiv P \and P|(Q|R) \equiv (P|Q)|R }
\end{mathpar}

\begin{mathpar}
  \inferrule* [lab=alpha-equivalence] {} { (x)P \equiv (y)P\{y/x\} \and y \not\in \freenames{P} }
\end{mathpar}

\begin{definition}
Then two processes, $P,Q$, are alpha-equivalent if $P = Q\{\vec{y}/\vec{x}\}$ for
some $\vec{x} \in \boundnames{Q},\vec{y} \in \boundnames{P}$, where $Q\{\vec{y}/\vec{x}\}$
denotes the capture-avoiding substitution of $\vec{y}$ for $\vec{x}$ in $Q$.
\end{definition}

\begin{definition}
  The {\em structural congruence} \cite{SangiorgiWalker} , $\equiv$,
  between processes is the least congruence containing
  alpha-equivalence, satisfying the abelian monoid laws
  (associativity, commutativity and $\pzero$ as identity) for parallel
  composition $|$ and for summation $+$.
\end{definition}

\subsection{Name equivalence}

We take name equivalence, written $\nameeq$, to be the smallest
equivalence relation generated by the following rules.

\begin{mathpar}
\inferrule*[lab=Quote-drop]
{ }
{ \quotep{@{x}} \nameeq x }

\inferrule*[lab=Struct-equiv]
{ P \scong Q }
{ \quotep{P} \nameeq \quotep{Q} }
\end{mathpar}

The astute reader will have noticed that the mutual recursion of names
and processes imposes a mutual recursion on alpha-equivalence and
structural equivalence via name-equivalence. Fortunately, all of this
works out pleasantly and we may calculate in the natural way, free of
concern. The reader interested in the details is referred to the
appendix \ref{appendix:rho_details}.

\subsection{Substitution}

We use $\Proc$ for the set of processes, $\QProc$ for the set of
names, and $\id{\{}\vec{y} / \vec{x} \id{\}}$ to denote partial maps,
$s : \QProc \rightarrow \QProc$. A map, $s$ lifts, uniquely, to a map
on process terms, $\widehat{s} : \Proc \rightarrow \Proc$ by the
following equations.

\begin{mathpar}
  (0) \psubstp{Q}{P} := 0 \\
  (R \juxtap S) \psubstp{Q}{P}
  :=    
  (R)\psubstp{Q}{P} \juxtap (S) \psubstp{Q}{P} \\
  (x?(y).R) \psubstp{Q}{P}    
  :=    
  (x)\substp{Q}{P} (z)\concat( (R \psubstn{z}{y}) \psubstp{Q}{P} ) \\
  (\lift{x}{R}) \psubstp{Q}{P}  
  :=
  \lift{(x)\substp{Q}{P}}{ R \psubstp{Q}{P} } \\
%   (\dropn{x})  \psubstp{Q}{P}       
%   := 
%   \left\{ 
%     \begin{array}{ccc} 
%       \dropn{\quotep{Q}} & & x \nameeq \quotep{P} \\
%       \dropn{x} & & otherwise \\
%     \end{array}
%   \right. 
  (\dropn{x})  \psubstp{Q}{P}       
  := 
  \left\{ 
    \begin{array}{ccc} 
      Q & & x \nameeq \quotep{P} \\
      \dropn{x} & & otherwise \\
    \end{array}
  \right.
\end{mathpar}
 

where

\begin{eqnarray}
  (x)\id{\{} \lpquote Q \rpquote / \lpquote P \rpquote \id{\}}            = 
  \left\{ 
    \begin{array}{ccc}
      \lpquote Q \rpquote & & x \nameeq \lpquote P \rpquote \\
      x & & otherwise \\
    \end{array}
  \right. \nonumber
\end{eqnarray}

and $z$ is chosen distinct from $\quotep{P}$, $\quotep{Q}$, the free
names in $Q$, and all the names in $R$. Our $\alpha$-equivalence will
be built in the standard way from this substitution.

\begin{remark}\label{rem:no_self_referential_names}
  One consequence of these definitions is that $\forall P. \quotep{P}
  \not\in \freenames{P}$.
\end{remark}

\subsection{ Dynamic quote: an example }

Anticipating something of what's to come, consider applying the
substitution, $\widehat{\id{\{}u / z \id{\}}}$, to the following pair
of processes, $\lift{w}{y!(z)}$ and $w[ \lpquote y!(z) \rpquote ]$.

\begin{eqnarray}
	\lift{w}{y!(z)}\widehat{\id{\{}u / z \id{\}}}
		& = &
		\lift{w}{y!(u)} \nonumber\\
	w[ \lpquote y!(z) \rpquote ] \widehat{ \id{\{}u / z \id{\}} }
		& = &
		w[ \lpquote y!(z) \rpquote ] \nonumber
\end{eqnarray}

Because the body of the process between quotes is impervious to
substitution, we get radically different answers. In fact, by
examining the first process in an input context,
e.g. $x?(z).\lift{w}{y!(z)}$, we see that the process under the lift
operator may be shaped by prefixed inputs binding a name inside it. In
this sense, the lift operator will be seen as a way to dynamically
construct processes before reifying them as names.

Finally equipped with these standard features we can present the
dynamics of the calculus.

\subsubsection{Operational semantics} 

Finally, we introduce the computational dynamics. What marks these
algebras as distinct from other more traditionally studied algebraic
structures, e.g. vector spaces or polynomial rings, is the manner in
which dynamics is captured. In traditional structures, dynamics is typically
expressed through morphisms between such structures, as in linear maps
between vector spaces or morphisms between rings. In algebras
associated with the semantics of computation, the dynamics is
expressed as part of the algebraic structure itself, through a
reduction reduction relation typically denoted by $\red$. Below, we
give a recursive presentation of this relation for the calculus used
in the encoding.

$\red \subseteq \pi \times \pi$
$\red : \pi \to \mathcal{P}(\pi)$

\begin{mathpar}
  \inferrule* [lab=Comm] { \textsf{match}( x_{src}, x_{trgt} ) } { x_{trgt}?(y)P \; | \; x_{src}!\langle {Q} \rangle \red P\{\quotep{Q}/y}\} }
  \and \\
  \inferrule* [lab=Par] {{P} \red {P}'} {{{P} | {Q}} \red {{P}' | {Q}}}
  \and
  \inferrule* [lab=Equiv]{{{P} \scong {P}'} \andalso {{P}' \red {Q}'} \andalso {{Q}' \scong {Q}}}{{P} \red {Q}}
\end{mathpar}

\begin{eqnarray*}
  match_{\equiv} (\quotep{P},\quotep{Q}) & := & P \equiv Q \\
  match_{\dagger}(\quotep{P},\quotep{Q}) & := & \forall R. P|Q \red^{*} R => R \red^{*} 0 \\
  match_{K}(\quotep{P},\quotep{Q}) & := & K \mbox{ for some context } K
\end{eqnarray*}

$u?(x)P | u!\langle Q \rangle \red P\{\quotep{Q}/x\}$

%We write $\wred$ for $\red^*$, and $P\red$ if $\exists Q $ such that $ P \red Q$.
We write $P\red$ if $\exists Q $ such that $ P \red Q$ and $P\not\red$, otherwise.

\section{Replication}

As mentioned before, it is known that replication (and hence
recursion) can be implemented in a higher-order process algebra
\cite{SangiorgiWalker}. As our first example of calculation with the
machinery thus far presented we give the construction explicitly in
the {\rhoc}.

\begin{eqnarray}
	D_{x} & := & \prefix{x}{y}{(\binpar{\outputp{x}{y}}{@{y}})} \nonumber\\
	\bangp_{x}{P} & := & \binpar{{x}!\langle{\binpar{D_{x}}{P}}\rangle}{D_{x}} \nonumber
\end{eqnarray}

\begin{eqnarray}
	\bangp_{x}{P} & & \nonumber\\
	=
	& {x}!\langle{(\prefix{x}{y}{(\outputp{x}{y} | @{y})) | P}}\rangle 
	      | \prefix{x}{y}{(\outputp{x}{y} | @{y})} & \nonumber\\
	\red
	& (\outputp{x}{y} | @{y})\substn{\quotep{(\prefix{x}{y}{(@{y} | \outputp{x}{y})) | P}}}{y} & \nonumber\\
	=
	& \outputp{x}{\quotep{(\prefix{x}{y}{(\outputp{x}{y} | @{y})) | P}}}
	  | {(\prefix{x}{y}{(\outputp{x}{y} | @{y})) | P}} & \nonumber\\
	\red
	& \ldots & \nonumber\\
	\red^*
	& P | P | \ldots & \nonumber
\end{eqnarray}

Of course, this encoding, as an implementation, runs away, unfolding
$\bangp{P}$ eagerly. A lazier and more implementable replication
operator, restricted to input-guarded processes, may be obtained as follows.

\begin{eqnarray}
\bangp{\prefix{u}{v}{P}} 
	:= 
	\binpar{\lift{x}{\prefix{u}{v}{(\binpar{D(x)}{P})}}}{D(x)} \nonumber
\end{eqnarray}

\begin{remark}
  Note that the lazier definition still does not deal with summation
  or mixed summation (i.e. sums over input and output). The reader is
  invited to construct definitions of replication that deal with these
  features. 

  Further, the definitions are parameterized in a name, $x$. Can you,
  gentle reader, make a definition that eliminates this parameter and
  guarantees no accidental interaction between the replication
  machinery and the process being replicated -- i.e. no accidental
  sharing of names used by the process to get its work done and the
  name(s) used by the replication to effect copying. This latter
  revision of the definition of replication is crucial to obtaining
  the expected identity $!!P \sim !P$.
\end{remark}

\begin{remark}\label{rem:paradoxical_combinator}
  The reader familiar with the lambda calculus will have noticed the
  similarity between $D$ and the paradoxical combinator.

  [Ed. note: the existence of this seems to suggest we have to be more
  restrictive on the set of processes and names we admit if we are to
  support no-cloning.]
\end{remark}

\subsubsection{Bisimulation}

The computational dynamics gives rise to another kind of equivalence,
the equivalence of computational behavior. As previously mentioned
this is typically captured \emph{via} some form of bisimulation.

% The notion we use in this paper is weak barbed bisimulation
% \cite{milner91polyadicpi}.

The notion we use in this paper is derived from weak barbed
bisimulation \cite{milner91polyadicpi}. 

\begin{definition}
An \emph{observation relation}, $\downarrow_{\mathcal N}$, over a set
of names, $\mathcal N$, is the smallest relation satisfying the rules
below.

\infrule[Out-barb]{y \in {\mathcal N}, \; x \nameeq y}
		  {\outputp{x}{v} \downarrow_{\mathcal N} x}
\infrule[Par-barb]{\mbox{$P\downarrow_{\mathcal N} x$ or $Q\downarrow_{\mathcal N} x$}}
		  {\binpar{P}{Q} \downarrow_{\mathcal N} x}

We write $P \Downarrow_{\mathcal N} x$ if there is $Q$ such that 
$P \wred Q$ and $Q \downarrow_{\mathcal N} x$.
\end{definition}

\begin{definition}
%\label{def.bbisim}
An  ${\mathcal N}$-\emph{barbed bisimulation} over a set of names, ${\mathcal N}$, is a symmetric binary relation 
${\mathcal S}_{\mathcal N}$ between agents such that $P\rel{S}_{\mathcal N}Q$ implies:
\begin{enumerate}
\item If $P \red P'$ then $Q \wred Q'$ and $P'\rel{S}_{\mathcal N} Q'$.
\item If $P\downarrow_{\mathcal N} x$, then $Q\Downarrow_{\mathcal N} x$.
\end{enumerate}
$P$ is ${\mathcal N}$-barbed bisimilar to $Q$, written
$P \wbbisim_{\mathcal N} Q$, if $P \rel{S}_{\mathcal N} Q$ for some ${\mathcal N}$-barbed bisimulation ${\mathcal S}_{\mathcal N}$.
\end{definition}

$\mathcal{R} \subseteq \pi \times \pi$

$P \mathcal{R} Q => \forall P'. P \red P' \Rightarrow \exists Q'. Q \red Q', P' \mathcal{R} Q'$

$P \vdash x \Rightarrow Q \vdash x$

\begin{mathpar}
  \inferrule*[lab=Out-barb]{x \nameeq y}{{y}!\langle{Q}\rangle \vdash x}
  \and
  \inferrule*[lab=Par-barb]{\mbox{$P\vdash x$ or $Q\vdash x$}}{\binpar{P}{Q} \vdash x}
\end{mathpar}

\subsubsection{Contexts}

One of the principle advantages of computational calculi like the
$\pi$-calculus is a well-defined notion of context,
contextual-equivalence and a correlation between
contextual-equivalence and notions of bisimulation. The notion of
context allows the decomposition of a process into (sub-)process and
its syntactic environment, its context. Thus, a context may be
thought of as a process with a ``hole'' (written $\Box$) in it. The
application of a context $M$ to a process $P$, written $M[P]$, is
tantamount to filling the hole in $M$ with $P$. In this paper we do
not need the full weight of this theory, but do make use of the notion
of context in the proof the main theorem. 

\begin{mathpar}
  \inferrule* [lab=summation] {} {{M_{M},M_{N}} \bc \Box \;|\; x.M_{A} \;|\; M_{M}+M_{N}}
  \and
  \inferrule* [lab=agent] {} {{M_{A}} \bc (\vec{x})M_{P} \;| \; \clift{P_0,\ldots,M_{P},\ldots,P_N}}
  \and \\
  \inferrule* [lab=process] {} {{M_{P}} \bc M_{N} \;| \;P|M_{P} }
\end{mathpar} 

\begin{mathpar}
  \inferrule* [lab=sychronization] {} {M_{N} \bc \Box \;|\; x?M_{F} \;|\; x!M_{C}}
  \and
  \inferrule* [lab=abstraction] {} {{M_{F}} \bc (x)M_{P} }
  \and
  \inferrule* [lab=concretion] {} {{M_{C}} \bc \langle M_{P} \rangle }
  \and \\
  \inferrule* [lab=process] {} {{M_{P}} \bc M_{N} \;| \;P|M_{P} }
\end{mathpar}

\begin{definition}[contextual application] Given a context $M$, and
  process $P$, we define the \emph{contextual application}, $M[P] :=
  M\{P/\Box\}$. That is, the contextual application of M to P is the
  substitution of $P$ for $\Box$ in $M$.
\end{definition}

$\meaningof{-} : L \to \mathcal{P}(\pi)$

\begin{mathpar}
  \inferrule* [lab=collection] {} {\meaningof{true} = \pi, \and \meaningof{~E} = \pi \setminus \meaningof{E}, \and \meaningof{E_{1} \& E_{2}} = \meaningof{E_{1}} \cap \meaningof{E_{2}}}
\end{mathpar}

\begin{mathpar}
  \inferrule* [lab=structure] {} {\meaningof{0} = \{ P \in \pi | P \equiv 0 \}, \and \\ \meaningof{E_1 | E_2} = \{ P \in \pi | P \equiv P_{1} | P_{2}, P_{1} \in \meaningof{E_{1}}, P_{2} \in \meaningof{E_2}\} }
\end{mathpar}

\begin{mathpar}
 \inferrule* [lab=behavior] {} {\meaningof{\langle a?b \rangle E} = \{ P \in \pi | P \equiv Q | u?(y)P', \\ \and \\\\ \and \\ \;\;\; u \in \meaningof{a}, \forall z.P'\{z/y\} \in \meaningof{E\{z/b\}}\}, \and \\ \meaningof{a!E} = \{ P \in \pi | P \equiv Q | x!\langle P' \rangle, x \in \meaningof{a} P' \in \meaningof{E}\} }
\end{mathpar}

\begin{mathpar}
 \inferrule* [lab=nominal] {} {\meaningof{\quotep{E}} = \{ \quotep{P} \in \quotep{\pi} | P \in \meaningof{E} \}, \and \meaningof{\quotep{P}} = \{ \quotep{Q} \in \quotep{\pi} | P \equiv Q \} \and \\ \meaningof{@\quotep{E}} = \{ P \in \pi | P \equiv @x, x \in \meaningof{E} \}}
\end{mathpar}

\begin{eqnarray*}
  \\
  \meaningof{-} : TS \to ST
\end{eqnarray*}

\begin{eqnarray*}
  \\
  L : TS \to ST
\end{eqnarray*}

\begin{eqnarray*}
  \\
  P \models E \iff P \in \meaningof{E}
\end{eqnarray*}

\begin{eqnarray*}
  P \approx_{L} Q \iff \forall E \in L. P \models E \iff Q \models E
\end{eqnarray*}

\begin{eqnarray*}
  P \approx_{K} Q
\end{eqnarray*}

\begin{eqnarray*}
  P \approx Q
\end{eqnarray*}

$\approx_{K} = \approx = \approx_{L}$

\subsubsection{Contextual duality}

Note that contexts extend the quotation operation to a family of
operations from processes to names. Given a context, $M$, we can
define a \emph{nominal context}, $\quotep{M}$ by $\quotep{M}[P] :=
\quotep{M[P]}$. To foreshadow what is to come we observe that these
operations enjoy a duality with processes very much like the duality
between vectors and maps from vectors to scalars.

Further, because the calculus is essentially higher-order, we have a
correspondence between contexts and processes. More specifically,
given a name $x$ and a context $M$ we can construct $M^{*}_{x}$ such
that 

\begin{mathpar}
  M^{*}_{x} | \lift{x}{P} \red M[P]
\end{mathpar}

namely,

\begin{mathpar}
  M^{*}_{x} := x?(u).M[\dropn{u}]
\end{mathpar}

The dependence of $M^{*}_{x}$ on a name makes it an abstraction, 

\begin{mathpar}
  M^{*} := (x)x?(u).M[\dropn{u}]
\end{mathpar}

\subsection{Additional notation}

It will sometimes be convenient to denote the process a name
quotes. We already have the notation $x = \quotep{P}$, but it will be
convenient to introduce an alternate notation, $\procn{x}$, when we
want to emphasize the connection to the use of the name. Note that, by
virtue of name equivalence, $\quotep{\procn{x}} \nameeq x$; so, the
notation is consistent with previous definitions.

Further, because names have structure it is possible to effect
substitutions on the basis of that structure. This means we need to
upgrade our notation for substitutions, which we accomplish by
adapting comprehension notation. Thus,

\begin{mathpar}
  P\{ y / x : x \in S \}
\end{mathpar}

is interpreted to mean the process derived from P by replacing (in a
capture-avoiding manner) each occurrence of $x$ in $S$ by $y$. For example,

\begin{mathpar}
  P\{ \quotep{\procn{x}|\procn{x}} / x : x \in \freenames{P} \}
\end{mathpar}

will replace each (occurrence) of a free name $x$ in $P$ by
$\quotep{\procn{x}|\procn{x}}$.

Also, we will avail ourselves of the notation $x^{L}$ and $x^{R}$ to
denote injections of a name into disjoint copies of the name
space. There are numerous ways to accomplish this. One example can be
found in \cite{MeredithR05}. This notation overloads to vectors of
names: $\vec{x}^{\pi} := (x_{i}^{\pi} \; : \; 0 \leq i < |\vec{x}| )$ where $\pi \in \{L,R\}$.

We also use $P^{\Box} := P|\Box$.

In \cite{MeredithR05} an interpretation of the new operator is
given. It turns out that there are several possible interpretations
all enjoying the requisite algebraic properties of the operator (see
\cite{milner91polyadicpi}). We will therefore make liberal use of
$(\nu\; \vec{x})P$.

% subsection the_syntax_and_semantics_of_the_notation_system (end)   

\input{qm2pi.qmops} 

\input{qm2pi.sterngerlach} 

\input{qm2pi.metric} 

% section concurrent_process_calculi (end)

%\input{qm2pi.proofsketch}

% section proof sketch (end)

%\input{qm2pi.slviaknots} 

% section spatial logic via knots (end)

\input{qm2pi.conclusion}

% section conclusion (end)

%\input{qm2pi.dtcodes} 

% section wiring algorithm (end)

\input{qm2pi.ack} 

% section acknowledgments (end)

\newpage


\bibliographystyle{plain}   
\bibliography{../../biblios/main.bib}

\input{qm2pi.rhodetails}

\end{document}

 

% section wiring algorithm (end)

\documentclass[12pt]{llncs}
%\documentclass{jktr}

\usepackage[pdftex]{hyperref}                   
\usepackage {listings}
\usepackage {mathpartir}
\usepackage{bcprules}
%\usepackage{listings}
                       
\usepackage{graphicx} 
%\usepackage[margins=2.5cm,nohead,nofoot]{geometry}
%\usepackage{geometry}
\usepackage{amsfonts}
\usepackage{amstext}
\usepackage{latexsym}
\usepackage{amssymb}
\usepackage{color}


%\include{myPreamble}
\include{qm2pi.local} 

%\ifpdf
%\usepackage[pdftex]{graphicx}
%\else
%\usepackage{graphicx}
%\fi

 % \ifpdf
%  \usepackage{pdfsync}
%  \if


%\title{Brief Article}
%\author{David F. Snyder}
%\author{L.G. Meredith}

%\address{Dept. of Math., Texas State University--San Marcos, San Marcos, TX 78666}
       
\pagestyle{empty}


\begin{document}

\lstset{language=[Objective]Caml,frame=shadowbox}

\input{qm2pi.front}

% section front matter (end)

\input{qm2pi.intro} 
 
% section introduction (end)

% \input{qm2pi.knotations} 

% section notation (end)

\input{qm2pi.process.calculi} 

% section concurrent_process_calculi_and_spatial_logics_ (end)
    
%\input{qm2pi.knots2pi} 

%\input{qm2pi.trefoil} 

%\input{qm2pi.mainthm} 

% subsection basic_interpretation (end)

%\input{qm2pi.rho.presentation} 
\subsection{The syntax and semantics of the notation system}\label{sub:the_syntax_and_semantics_of_the_notation_system} % (fold)

We now summarize a technical presentation of the calculus that
embodies our theory of dynamics. The typical presentation of such a
calculus follows the style of giving generators and relations on
them. The grammar, below, describing term constructors, freely
generates the set of processes, $\Proc$. This set is then quotiented
by a relation known as structural congruence and it is over this set
that the notion of dynamics is expressed. This presentation is
essentially that of \cite{MeredithR05} with the addition of
polyadicity and summation. For readability we have relegated some of
the technical subtleties to an appendix.

\subsubsection{Process grammar}\label{subsub:process_grammar}

\begin{mathpar}
  \inferrule* [lab=synchronization] {} {{M} \bc \pzero \;|\; x?F \;|\; x!C }
  \and
  \inferrule* [lab=abstraction] {} {{F} \bc (x)P}
  \and
  \inferrule* [lab=concretion] {} {{C} \bc \langle Q \rangle}
  \and
  \inferrule* [lab=process] {} {{P,Q} \bc M \;| \;P|Q \;|\; @{x}}
  \and
  \inferrule* [lab=name] {} {{x} \bc \quotep{P}}
\end{mathpar} 

Note that $\vec{x}$ (resp. $\vec{P}$) denotes a vector of names
(resp. processes) of length $|\vec{x}|$ (resp. $|\vec{P}|$). We adopt
the following useful abbreviations.

\begin{mathpar}
   x?(\vec{y}).P := x.(\vec{y})P \and  x\clift{\vec{P}} := x.\clift{\vec{P}}
   \and x!(y) := \lift{x}{\dropn{y}}
   \and \Pi_{i=0}^{n-1}P_i := P_0 | \ldots | P_{n-1}
\end{mathpar}

\subsubsection{Structural congruence}

\paragraph{Free and bound names and alpha-equivalence.} At the
core of structural equivalence is alpha-equivalence which identifies
process that are the same up to a change of variable. Formally, we
recognize the distinction between free and bound names. The free names
of a process, $\freenames{P}$, may be calculated recursively as
follows:

\begin{mathpar}
\freenames{\pzero} := \emptyset
  \and \\
  \freenames{x?(y).P} := \{ x \} \cup (\freenames{P} \setminus \{ y \})
  \and 
  \freenames{x!\langle P \rangle} := \{ x \} \cup \{ P \} 
  \and \\
  \freenames{P|Q} := \freenames{P} \cup \freenames{Q}
  \and \\
  \freenames{@{x}} := \{ x \}
\end{mathpar}

$\pi$
$\quotep{\pi}$

$\freenames{-} : \pi \to \mathcal{P}(\quotep{\pi})$

\begin{eqnarray*}
  \freenames{\pzero} & := & \emptyset \\
  \freenames{x?(y).P} & := & \{ x \} \cup (\freenames{P} \setminus \{ y \}) \\
  \freenames{x!\langle P \rangle} & := & \{ x \} \cup \{ P \} \\
  \freenames{P|Q} & := & \freenames{P} \cup \freenames{Q} \\
  \freenames{\dropn{x}} & := & \{ x \}
\end{eqnarray*}

The bound names of a process, $\boundnames{P}$, are those names occurring in $P$
that are not free. For example, in $x?(y).0$, the name $x$ is free, while $y$ is bound.

\begin{mathpar}
  \inferrule* [lab=monoidal-laws] {} { P|Q \equiv Q|P \and P|0 \equiv P \and P|(Q|R) \equiv (P|Q)|R }
\end{mathpar}

\begin{mathpar}
  \inferrule* [lab=alpha-equivalence] {} { (x)P \equiv (y)P\{y/x\} \and y \not\in \freenames{P} }
\end{mathpar}

\begin{definition}
Then two processes, $P,Q$, are alpha-equivalent if $P = Q\{\vec{y}/\vec{x}\}$ for
some $\vec{x} \in \boundnames{Q},\vec{y} \in \boundnames{P}$, where $Q\{\vec{y}/\vec{x}\}$
denotes the capture-avoiding substitution of $\vec{y}$ for $\vec{x}$ in $Q$.
\end{definition}

\begin{definition}
  The {\em structural congruence} \cite{SangiorgiWalker} , $\equiv$,
  between processes is the least congruence containing
  alpha-equivalence, satisfying the abelian monoid laws
  (associativity, commutativity and $\pzero$ as identity) for parallel
  composition $|$ and for summation $+$.
\end{definition}

\subsection{Name equivalence}

We take name equivalence, written $\nameeq$, to be the smallest
equivalence relation generated by the following rules.

\begin{mathpar}
\inferrule*[lab=Quote-drop]
{ }
{ \quotep{@{x}} \nameeq x }

\inferrule*[lab=Struct-equiv]
{ P \scong Q }
{ \quotep{P} \nameeq \quotep{Q} }
\end{mathpar}

The astute reader will have noticed that the mutual recursion of names
and processes imposes a mutual recursion on alpha-equivalence and
structural equivalence via name-equivalence. Fortunately, all of this
works out pleasantly and we may calculate in the natural way, free of
concern. The reader interested in the details is referred to the
appendix \ref{appendix:rho_details}.

\subsection{Substitution}

We use $\Proc$ for the set of processes, $\QProc$ for the set of
names, and $\id{\{}\vec{y} / \vec{x} \id{\}}$ to denote partial maps,
$s : \QProc \rightarrow \QProc$. A map, $s$ lifts, uniquely, to a map
on process terms, $\widehat{s} : \Proc \rightarrow \Proc$ by the
following equations.

\begin{mathpar}
  (0) \psubstp{Q}{P} := 0 \\
  (R \juxtap S) \psubstp{Q}{P}
  :=    
  (R)\psubstp{Q}{P} \juxtap (S) \psubstp{Q}{P} \\
  (x?(y).R) \psubstp{Q}{P}    
  :=    
  (x)\substp{Q}{P} (z)\concat( (R \psubstn{z}{y}) \psubstp{Q}{P} ) \\
  (\lift{x}{R}) \psubstp{Q}{P}  
  :=
  \lift{(x)\substp{Q}{P}}{ R \psubstp{Q}{P} } \\
%   (\dropn{x})  \psubstp{Q}{P}       
%   := 
%   \left\{ 
%     \begin{array}{ccc} 
%       \dropn{\quotep{Q}} & & x \nameeq \quotep{P} \\
%       \dropn{x} & & otherwise \\
%     \end{array}
%   \right. 
  (\dropn{x})  \psubstp{Q}{P}       
  := 
  \left\{ 
    \begin{array}{ccc} 
      Q & & x \nameeq \quotep{P} \\
      \dropn{x} & & otherwise \\
    \end{array}
  \right.
\end{mathpar}
 

where

\begin{eqnarray}
  (x)\id{\{} \lpquote Q \rpquote / \lpquote P \rpquote \id{\}}            = 
  \left\{ 
    \begin{array}{ccc}
      \lpquote Q \rpquote & & x \nameeq \lpquote P \rpquote \\
      x & & otherwise \\
    \end{array}
  \right. \nonumber
\end{eqnarray}

and $z$ is chosen distinct from $\quotep{P}$, $\quotep{Q}$, the free
names in $Q$, and all the names in $R$. Our $\alpha$-equivalence will
be built in the standard way from this substitution.

\begin{remark}\label{rem:no_self_referential_names}
  One consequence of these definitions is that $\forall P. \quotep{P}
  \not\in \freenames{P}$.
\end{remark}

\subsection{ Dynamic quote: an example }

Anticipating something of what's to come, consider applying the
substitution, $\widehat{\id{\{}u / z \id{\}}}$, to the following pair
of processes, $\lift{w}{y!(z)}$ and $w[ \lpquote y!(z) \rpquote ]$.

\begin{eqnarray}
	\lift{w}{y!(z)}\widehat{\id{\{}u / z \id{\}}}
		& = &
		\lift{w}{y!(u)} \nonumber\\
	w[ \lpquote y!(z) \rpquote ] \widehat{ \id{\{}u / z \id{\}} }
		& = &
		w[ \lpquote y!(z) \rpquote ] \nonumber
\end{eqnarray}

Because the body of the process between quotes is impervious to
substitution, we get radically different answers. In fact, by
examining the first process in an input context,
e.g. $x?(z).\lift{w}{y!(z)}$, we see that the process under the lift
operator may be shaped by prefixed inputs binding a name inside it. In
this sense, the lift operator will be seen as a way to dynamically
construct processes before reifying them as names.

Finally equipped with these standard features we can present the
dynamics of the calculus.

\subsubsection{Operational semantics} 

Finally, we introduce the computational dynamics. What marks these
algebras as distinct from other more traditionally studied algebraic
structures, e.g. vector spaces or polynomial rings, is the manner in
which dynamics is captured. In traditional structures, dynamics is typically
expressed through morphisms between such structures, as in linear maps
between vector spaces or morphisms between rings. In algebras
associated with the semantics of computation, the dynamics is
expressed as part of the algebraic structure itself, through a
reduction reduction relation typically denoted by $\red$. Below, we
give a recursive presentation of this relation for the calculus used
in the encoding.

$\red \subseteq \pi \times \pi$
$\red : \pi \to \mathcal{P}(\pi)$

\begin{mathpar}
  \inferrule* [lab=Comm] { \textsf{match}( x_{src}, x_{trgt} ) } { x_{trgt}?(y)P \; | \; x_{src}!\langle {Q} \rangle \red P\{\quotep{Q}/y}\} }
  \and \\
  \inferrule* [lab=Par] {{P} \red {P}'} {{{P} | {Q}} \red {{P}' | {Q}}}
  \and
  \inferrule* [lab=Equiv]{{{P} \scong {P}'} \andalso {{P}' \red {Q}'} \andalso {{Q}' \scong {Q}}}{{P} \red {Q}}
\end{mathpar}

\begin{eqnarray*}
  match_{\equiv} (\quotep{P},\quotep{Q}) & := & P \equiv Q \\
  match_{\dagger}(\quotep{P},\quotep{Q}) & := & \forall R. P|Q \red^{*} R => R \red^{*} 0 \\
  match_{K}(\quotep{P},\quotep{Q}) & := & K \mbox{ for some context } K
\end{eqnarray*}

$u?(x)P | u!\langle Q \rangle \red P\{\quotep{Q}/x\}$

%We write $\wred$ for $\red^*$, and $P\red$ if $\exists Q $ such that $ P \red Q$.
We write $P\red$ if $\exists Q $ such that $ P \red Q$ and $P\not\red$, otherwise.

\section{Replication}

As mentioned before, it is known that replication (and hence
recursion) can be implemented in a higher-order process algebra
\cite{SangiorgiWalker}. As our first example of calculation with the
machinery thus far presented we give the construction explicitly in
the {\rhoc}.

\begin{eqnarray}
	D_{x} & := & \prefix{x}{y}{(\binpar{\outputp{x}{y}}{@{y}})} \nonumber\\
	\bangp_{x}{P} & := & \binpar{{x}!\langle{\binpar{D_{x}}{P}}\rangle}{D_{x}} \nonumber
\end{eqnarray}

\begin{eqnarray}
	\bangp_{x}{P} & & \nonumber\\
	=
	& {x}!\langle{(\prefix{x}{y}{(\outputp{x}{y} | @{y})) | P}}\rangle 
	      | \prefix{x}{y}{(\outputp{x}{y} | @{y})} & \nonumber\\
	\red
	& (\outputp{x}{y} | @{y})\substn{\quotep{(\prefix{x}{y}{(@{y} | \outputp{x}{y})) | P}}}{y} & \nonumber\\
	=
	& \outputp{x}{\quotep{(\prefix{x}{y}{(\outputp{x}{y} | @{y})) | P}}}
	  | {(\prefix{x}{y}{(\outputp{x}{y} | @{y})) | P}} & \nonumber\\
	\red
	& \ldots & \nonumber\\
	\red^*
	& P | P | \ldots & \nonumber
\end{eqnarray}

Of course, this encoding, as an implementation, runs away, unfolding
$\bangp{P}$ eagerly. A lazier and more implementable replication
operator, restricted to input-guarded processes, may be obtained as follows.

\begin{eqnarray}
\bangp{\prefix{u}{v}{P}} 
	:= 
	\binpar{\lift{x}{\prefix{u}{v}{(\binpar{D(x)}{P})}}}{D(x)} \nonumber
\end{eqnarray}

\begin{remark}
  Note that the lazier definition still does not deal with summation
  or mixed summation (i.e. sums over input and output). The reader is
  invited to construct definitions of replication that deal with these
  features. 

  Further, the definitions are parameterized in a name, $x$. Can you,
  gentle reader, make a definition that eliminates this parameter and
  guarantees no accidental interaction between the replication
  machinery and the process being replicated -- i.e. no accidental
  sharing of names used by the process to get its work done and the
  name(s) used by the replication to effect copying. This latter
  revision of the definition of replication is crucial to obtaining
  the expected identity $!!P \sim !P$.
\end{remark}

\begin{remark}\label{rem:paradoxical_combinator}
  The reader familiar with the lambda calculus will have noticed the
  similarity between $D$ and the paradoxical combinator.

  [Ed. note: the existence of this seems to suggest we have to be more
  restrictive on the set of processes and names we admit if we are to
  support no-cloning.]
\end{remark}

\subsubsection{Bisimulation}

The computational dynamics gives rise to another kind of equivalence,
the equivalence of computational behavior. As previously mentioned
this is typically captured \emph{via} some form of bisimulation.

% The notion we use in this paper is weak barbed bisimulation
% \cite{milner91polyadicpi}.

The notion we use in this paper is derived from weak barbed
bisimulation \cite{milner91polyadicpi}. 

\begin{definition}
An \emph{observation relation}, $\downarrow_{\mathcal N}$, over a set
of names, $\mathcal N$, is the smallest relation satisfying the rules
below.

\infrule[Out-barb]{y \in {\mathcal N}, \; x \nameeq y}
		  {\outputp{x}{v} \downarrow_{\mathcal N} x}
\infrule[Par-barb]{\mbox{$P\downarrow_{\mathcal N} x$ or $Q\downarrow_{\mathcal N} x$}}
		  {\binpar{P}{Q} \downarrow_{\mathcal N} x}

We write $P \Downarrow_{\mathcal N} x$ if there is $Q$ such that 
$P \wred Q$ and $Q \downarrow_{\mathcal N} x$.
\end{definition}

\begin{definition}
%\label{def.bbisim}
An  ${\mathcal N}$-\emph{barbed bisimulation} over a set of names, ${\mathcal N}$, is a symmetric binary relation 
${\mathcal S}_{\mathcal N}$ between agents such that $P\rel{S}_{\mathcal N}Q$ implies:
\begin{enumerate}
\item If $P \red P'$ then $Q \wred Q'$ and $P'\rel{S}_{\mathcal N} Q'$.
\item If $P\downarrow_{\mathcal N} x$, then $Q\Downarrow_{\mathcal N} x$.
\end{enumerate}
$P$ is ${\mathcal N}$-barbed bisimilar to $Q$, written
$P \wbbisim_{\mathcal N} Q$, if $P \rel{S}_{\mathcal N} Q$ for some ${\mathcal N}$-barbed bisimulation ${\mathcal S}_{\mathcal N}$.
\end{definition}

$\mathcal{R} \subseteq \pi \times \pi$

$P \mathcal{R} Q => \forall P'. P \red P' \Rightarrow \exists Q'. Q \red Q', P' \mathcal{R} Q'$

$P \vdash x \Rightarrow Q \vdash x$

\begin{mathpar}
  \inferrule*[lab=Out-barb]{x \nameeq y}{{y}!\langle{Q}\rangle \vdash x}
  \and
  \inferrule*[lab=Par-barb]{\mbox{$P\vdash x$ or $Q\vdash x$}}{\binpar{P}{Q} \vdash x}
\end{mathpar}

\subsubsection{Contexts}

One of the principle advantages of computational calculi like the
$\pi$-calculus is a well-defined notion of context,
contextual-equivalence and a correlation between
contextual-equivalence and notions of bisimulation. The notion of
context allows the decomposition of a process into (sub-)process and
its syntactic environment, its context. Thus, a context may be
thought of as a process with a ``hole'' (written $\Box$) in it. The
application of a context $M$ to a process $P$, written $M[P]$, is
tantamount to filling the hole in $M$ with $P$. In this paper we do
not need the full weight of this theory, but do make use of the notion
of context in the proof the main theorem. 

\begin{mathpar}
  \inferrule* [lab=summation] {} {{M_{M},M_{N}} \bc \Box \;|\; x.M_{A} \;|\; M_{M}+M_{N}}
  \and
  \inferrule* [lab=agent] {} {{M_{A}} \bc (\vec{x})M_{P} \;| \; \clift{P_0,\ldots,M_{P},\ldots,P_N}}
  \and \\
  \inferrule* [lab=process] {} {{M_{P}} \bc M_{N} \;| \;P|M_{P} }
\end{mathpar} 

\begin{mathpar}
  \inferrule* [lab=sychronization] {} {M_{N} \bc \Box \;|\; x?M_{F} \;|\; x!M_{C}}
  \and
  \inferrule* [lab=abstraction] {} {{M_{F}} \bc (x)M_{P} }
  \and
  \inferrule* [lab=concretion] {} {{M_{C}} \bc \langle M_{P} \rangle }
  \and \\
  \inferrule* [lab=process] {} {{M_{P}} \bc M_{N} \;| \;P|M_{P} }
\end{mathpar}

\begin{definition}[contextual application] Given a context $M$, and
  process $P$, we define the \emph{contextual application}, $M[P] :=
  M\{P/\Box\}$. That is, the contextual application of M to P is the
  substitution of $P$ for $\Box$ in $M$.
\end{definition}

$\meaningof{-} : L \to \mathcal{P}(\pi)$

\begin{mathpar}
  \inferrule* [lab=collection] {} {\meaningof{true} = \pi, \and \meaningof{~E} = \pi \setminus \meaningof{E}, \and \meaningof{E_{1} \& E_{2}} = \meaningof{E_{1}} \cap \meaningof{E_{2}}}
\end{mathpar}

\begin{mathpar}
  \inferrule* [lab=structure] {} {\meaningof{0} = \{ P \in \pi | P \equiv 0 \}, \and \\ \meaningof{E_1 | E_2} = \{ P \in \pi | P \equiv P_{1} | P_{2}, P_{1} \in \meaningof{E_{1}}, P_{2} \in \meaningof{E_2}\} }
\end{mathpar}

\begin{mathpar}
 \inferrule* [lab=behavior] {} {\meaningof{\langle a?b \rangle E} = \{ P \in \pi | P \equiv Q | u?(y)P', \\ \and \\\\ \and \\ \;\;\; u \in \meaningof{a}, \forall z.P'\{z/y\} \in \meaningof{E\{z/b\}}\}, \and \\ \meaningof{a!E} = \{ P \in \pi | P \equiv Q | x!\langle P' \rangle, x \in \meaningof{a} P' \in \meaningof{E}\} }
\end{mathpar}

\begin{mathpar}
 \inferrule* [lab=nominal] {} {\meaningof{\quotep{E}} = \{ \quotep{P} \in \quotep{\pi} | P \in \meaningof{E} \}, \and \meaningof{\quotep{P}} = \{ \quotep{Q} \in \quotep{\pi} | P \equiv Q \} \and \\ \meaningof{@\quotep{E}} = \{ P \in \pi | P \equiv @x, x \in \meaningof{E} \}}
\end{mathpar}

\begin{eqnarray*}
  \\
  \meaningof{-} : TS \to ST
\end{eqnarray*}

\begin{eqnarray*}
  \\
  L : TS \to ST
\end{eqnarray*}

\begin{eqnarray*}
  \\
  P \models E \iff P \in \meaningof{E}
\end{eqnarray*}

\begin{eqnarray*}
  P \approx_{L} Q \iff \forall E \in L. P \models E \iff Q \models E
\end{eqnarray*}

\begin{eqnarray*}
  P \approx_{K} Q
\end{eqnarray*}

\begin{eqnarray*}
  P \approx Q
\end{eqnarray*}

$\approx_{K} = \approx = \approx_{L}$

\subsubsection{Contextual duality}

Note that contexts extend the quotation operation to a family of
operations from processes to names. Given a context, $M$, we can
define a \emph{nominal context}, $\quotep{M}$ by $\quotep{M}[P] :=
\quotep{M[P]}$. To foreshadow what is to come we observe that these
operations enjoy a duality with processes very much like the duality
between vectors and maps from vectors to scalars.

Further, because the calculus is essentially higher-order, we have a
correspondence between contexts and processes. More specifically,
given a name $x$ and a context $M$ we can construct $M^{*}_{x}$ such
that 

\begin{mathpar}
  M^{*}_{x} | \lift{x}{P} \red M[P]
\end{mathpar}

namely,

\begin{mathpar}
  M^{*}_{x} := x?(u).M[\dropn{u}]
\end{mathpar}

The dependence of $M^{*}_{x}$ on a name makes it an abstraction, 

\begin{mathpar}
  M^{*} := (x)x?(u).M[\dropn{u}]
\end{mathpar}

\subsection{Additional notation}

It will sometimes be convenient to denote the process a name
quotes. We already have the notation $x = \quotep{P}$, but it will be
convenient to introduce an alternate notation, $\procn{x}$, when we
want to emphasize the connection to the use of the name. Note that, by
virtue of name equivalence, $\quotep{\procn{x}} \nameeq x$; so, the
notation is consistent with previous definitions.

Further, because names have structure it is possible to effect
substitutions on the basis of that structure. This means we need to
upgrade our notation for substitutions, which we accomplish by
adapting comprehension notation. Thus,

\begin{mathpar}
  P\{ y / x : x \in S \}
\end{mathpar}

is interpreted to mean the process derived from P by replacing (in a
capture-avoiding manner) each occurrence of $x$ in $S$ by $y$. For example,

\begin{mathpar}
  P\{ \quotep{\procn{x}|\procn{x}} / x : x \in \freenames{P} \}
\end{mathpar}

will replace each (occurrence) of a free name $x$ in $P$ by
$\quotep{\procn{x}|\procn{x}}$.

Also, we will avail ourselves of the notation $x^{L}$ and $x^{R}$ to
denote injections of a name into disjoint copies of the name
space. There are numerous ways to accomplish this. One example can be
found in \cite{MeredithR05}. This notation overloads to vectors of
names: $\vec{x}^{\pi} := (x_{i}^{\pi} \; : \; 0 \leq i < |\vec{x}| )$ where $\pi \in \{L,R\}$.

We also use $P^{\Box} := P|\Box$.

In \cite{MeredithR05} an interpretation of the new operator is
given. It turns out that there are several possible interpretations
all enjoying the requisite algebraic properties of the operator (see
\cite{milner91polyadicpi}). We will therefore make liberal use of
$(\nu\; \vec{x})P$.

% subsection the_syntax_and_semantics_of_the_notation_system (end)   

\input{qm2pi.qmops} 

\input{qm2pi.sterngerlach} 

\input{qm2pi.metric} 

% section concurrent_process_calculi (end)

%\input{qm2pi.proofsketch}

% section proof sketch (end)

%\input{qm2pi.slviaknots} 

% section spatial logic via knots (end)

\input{qm2pi.conclusion}

% section conclusion (end)

%\input{qm2pi.dtcodes} 

% section wiring algorithm (end)

\input{qm2pi.ack} 

% section acknowledgments (end)

\newpage


\bibliographystyle{plain}   
\bibliography{../../biblios/main.bib}

\input{qm2pi.rhodetails}

\end{document}

 

% section acknowledgments (end)

\newpage


\bibliographystyle{plain}   
\bibliography{../../biblios/main.bib}

\documentclass[12pt]{llncs}
%\documentclass{jktr}

\usepackage[pdftex]{hyperref}                   
\usepackage {listings}
\usepackage {mathpartir}
\usepackage{bcprules}
%\usepackage{listings}
                       
\usepackage{graphicx} 
%\usepackage[margins=2.5cm,nohead,nofoot]{geometry}
%\usepackage{geometry}
\usepackage{amsfonts}
\usepackage{amstext}
\usepackage{latexsym}
\usepackage{amssymb}
\usepackage{color}


%\include{myPreamble}
\include{qm2pi.local} 

%\ifpdf
%\usepackage[pdftex]{graphicx}
%\else
%\usepackage{graphicx}
%\fi

 % \ifpdf
%  \usepackage{pdfsync}
%  \if


%\title{Brief Article}
%\author{David F. Snyder}
%\author{L.G. Meredith}

%\address{Dept. of Math., Texas State University--San Marcos, San Marcos, TX 78666}
       
\pagestyle{empty}


\begin{document}

\lstset{language=[Objective]Caml,frame=shadowbox}

\input{qm2pi.front}

% section front matter (end)

\input{qm2pi.intro} 
 
% section introduction (end)

% \input{qm2pi.knotations} 

% section notation (end)

\input{qm2pi.process.calculi} 

% section concurrent_process_calculi_and_spatial_logics_ (end)
    
%\input{qm2pi.knots2pi} 

%\input{qm2pi.trefoil} 

%\input{qm2pi.mainthm} 

% subsection basic_interpretation (end)

%\input{qm2pi.rho.presentation} 
\subsection{The syntax and semantics of the notation system}\label{sub:the_syntax_and_semantics_of_the_notation_system} % (fold)

We now summarize a technical presentation of the calculus that
embodies our theory of dynamics. The typical presentation of such a
calculus follows the style of giving generators and relations on
them. The grammar, below, describing term constructors, freely
generates the set of processes, $\Proc$. This set is then quotiented
by a relation known as structural congruence and it is over this set
that the notion of dynamics is expressed. This presentation is
essentially that of \cite{MeredithR05} with the addition of
polyadicity and summation. For readability we have relegated some of
the technical subtleties to an appendix.

\subsubsection{Process grammar}\label{subsub:process_grammar}

\begin{mathpar}
  \inferrule* [lab=synchronization] {} {{M} \bc \pzero \;|\; x?F \;|\; x!C }
  \and
  \inferrule* [lab=abstraction] {} {{F} \bc (x)P}
  \and
  \inferrule* [lab=concretion] {} {{C} \bc \langle Q \rangle}
  \and
  \inferrule* [lab=process] {} {{P,Q} \bc M \;| \;P|Q \;|\; @{x}}
  \and
  \inferrule* [lab=name] {} {{x} \bc \quotep{P}}
\end{mathpar} 

Note that $\vec{x}$ (resp. $\vec{P}$) denotes a vector of names
(resp. processes) of length $|\vec{x}|$ (resp. $|\vec{P}|$). We adopt
the following useful abbreviations.

\begin{mathpar}
   x?(\vec{y}).P := x.(\vec{y})P \and  x\clift{\vec{P}} := x.\clift{\vec{P}}
   \and x!(y) := \lift{x}{\dropn{y}}
   \and \Pi_{i=0}^{n-1}P_i := P_0 | \ldots | P_{n-1}
\end{mathpar}

\subsubsection{Structural congruence}

\paragraph{Free and bound names and alpha-equivalence.} At the
core of structural equivalence is alpha-equivalence which identifies
process that are the same up to a change of variable. Formally, we
recognize the distinction between free and bound names. The free names
of a process, $\freenames{P}$, may be calculated recursively as
follows:

\begin{mathpar}
\freenames{\pzero} := \emptyset
  \and \\
  \freenames{x?(y).P} := \{ x \} \cup (\freenames{P} \setminus \{ y \})
  \and 
  \freenames{x!\langle P \rangle} := \{ x \} \cup \{ P \} 
  \and \\
  \freenames{P|Q} := \freenames{P} \cup \freenames{Q}
  \and \\
  \freenames{@{x}} := \{ x \}
\end{mathpar}

$\pi$
$\quotep{\pi}$

$\freenames{-} : \pi \to \mathcal{P}(\quotep{\pi})$

\begin{eqnarray*}
  \freenames{\pzero} & := & \emptyset \\
  \freenames{x?(y).P} & := & \{ x \} \cup (\freenames{P} \setminus \{ y \}) \\
  \freenames{x!\langle P \rangle} & := & \{ x \} \cup \{ P \} \\
  \freenames{P|Q} & := & \freenames{P} \cup \freenames{Q} \\
  \freenames{\dropn{x}} & := & \{ x \}
\end{eqnarray*}

The bound names of a process, $\boundnames{P}$, are those names occurring in $P$
that are not free. For example, in $x?(y).0$, the name $x$ is free, while $y$ is bound.

\begin{mathpar}
  \inferrule* [lab=monoidal-laws] {} { P|Q \equiv Q|P \and P|0 \equiv P \and P|(Q|R) \equiv (P|Q)|R }
\end{mathpar}

\begin{mathpar}
  \inferrule* [lab=alpha-equivalence] {} { (x)P \equiv (y)P\{y/x\} \and y \not\in \freenames{P} }
\end{mathpar}

\begin{definition}
Then two processes, $P,Q$, are alpha-equivalent if $P = Q\{\vec{y}/\vec{x}\}$ for
some $\vec{x} \in \boundnames{Q},\vec{y} \in \boundnames{P}$, where $Q\{\vec{y}/\vec{x}\}$
denotes the capture-avoiding substitution of $\vec{y}$ for $\vec{x}$ in $Q$.
\end{definition}

\begin{definition}
  The {\em structural congruence} \cite{SangiorgiWalker} , $\equiv$,
  between processes is the least congruence containing
  alpha-equivalence, satisfying the abelian monoid laws
  (associativity, commutativity and $\pzero$ as identity) for parallel
  composition $|$ and for summation $+$.
\end{definition}

\subsection{Name equivalence}

We take name equivalence, written $\nameeq$, to be the smallest
equivalence relation generated by the following rules.

\begin{mathpar}
\inferrule*[lab=Quote-drop]
{ }
{ \quotep{@{x}} \nameeq x }

\inferrule*[lab=Struct-equiv]
{ P \scong Q }
{ \quotep{P} \nameeq \quotep{Q} }
\end{mathpar}

The astute reader will have noticed that the mutual recursion of names
and processes imposes a mutual recursion on alpha-equivalence and
structural equivalence via name-equivalence. Fortunately, all of this
works out pleasantly and we may calculate in the natural way, free of
concern. The reader interested in the details is referred to the
appendix \ref{appendix:rho_details}.

\subsection{Substitution}

We use $\Proc$ for the set of processes, $\QProc$ for the set of
names, and $\id{\{}\vec{y} / \vec{x} \id{\}}$ to denote partial maps,
$s : \QProc \rightarrow \QProc$. A map, $s$ lifts, uniquely, to a map
on process terms, $\widehat{s} : \Proc \rightarrow \Proc$ by the
following equations.

\begin{mathpar}
  (0) \psubstp{Q}{P} := 0 \\
  (R \juxtap S) \psubstp{Q}{P}
  :=    
  (R)\psubstp{Q}{P} \juxtap (S) \psubstp{Q}{P} \\
  (x?(y).R) \psubstp{Q}{P}    
  :=    
  (x)\substp{Q}{P} (z)\concat( (R \psubstn{z}{y}) \psubstp{Q}{P} ) \\
  (\lift{x}{R}) \psubstp{Q}{P}  
  :=
  \lift{(x)\substp{Q}{P}}{ R \psubstp{Q}{P} } \\
%   (\dropn{x})  \psubstp{Q}{P}       
%   := 
%   \left\{ 
%     \begin{array}{ccc} 
%       \dropn{\quotep{Q}} & & x \nameeq \quotep{P} \\
%       \dropn{x} & & otherwise \\
%     \end{array}
%   \right. 
  (\dropn{x})  \psubstp{Q}{P}       
  := 
  \left\{ 
    \begin{array}{ccc} 
      Q & & x \nameeq \quotep{P} \\
      \dropn{x} & & otherwise \\
    \end{array}
  \right.
\end{mathpar}
 

where

\begin{eqnarray}
  (x)\id{\{} \lpquote Q \rpquote / \lpquote P \rpquote \id{\}}            = 
  \left\{ 
    \begin{array}{ccc}
      \lpquote Q \rpquote & & x \nameeq \lpquote P \rpquote \\
      x & & otherwise \\
    \end{array}
  \right. \nonumber
\end{eqnarray}

and $z$ is chosen distinct from $\quotep{P}$, $\quotep{Q}$, the free
names in $Q$, and all the names in $R$. Our $\alpha$-equivalence will
be built in the standard way from this substitution.

\begin{remark}\label{rem:no_self_referential_names}
  One consequence of these definitions is that $\forall P. \quotep{P}
  \not\in \freenames{P}$.
\end{remark}

\subsection{ Dynamic quote: an example }

Anticipating something of what's to come, consider applying the
substitution, $\widehat{\id{\{}u / z \id{\}}}$, to the following pair
of processes, $\lift{w}{y!(z)}$ and $w[ \lpquote y!(z) \rpquote ]$.

\begin{eqnarray}
	\lift{w}{y!(z)}\widehat{\id{\{}u / z \id{\}}}
		& = &
		\lift{w}{y!(u)} \nonumber\\
	w[ \lpquote y!(z) \rpquote ] \widehat{ \id{\{}u / z \id{\}} }
		& = &
		w[ \lpquote y!(z) \rpquote ] \nonumber
\end{eqnarray}

Because the body of the process between quotes is impervious to
substitution, we get radically different answers. In fact, by
examining the first process in an input context,
e.g. $x?(z).\lift{w}{y!(z)}$, we see that the process under the lift
operator may be shaped by prefixed inputs binding a name inside it. In
this sense, the lift operator will be seen as a way to dynamically
construct processes before reifying them as names.

Finally equipped with these standard features we can present the
dynamics of the calculus.

\subsubsection{Operational semantics} 

Finally, we introduce the computational dynamics. What marks these
algebras as distinct from other more traditionally studied algebraic
structures, e.g. vector spaces or polynomial rings, is the manner in
which dynamics is captured. In traditional structures, dynamics is typically
expressed through morphisms between such structures, as in linear maps
between vector spaces or morphisms between rings. In algebras
associated with the semantics of computation, the dynamics is
expressed as part of the algebraic structure itself, through a
reduction reduction relation typically denoted by $\red$. Below, we
give a recursive presentation of this relation for the calculus used
in the encoding.

$\red \subseteq \pi \times \pi$
$\red : \pi \to \mathcal{P}(\pi)$

\begin{mathpar}
  \inferrule* [lab=Comm] { \textsf{match}( x_{src}, x_{trgt} ) } { x_{trgt}?(y)P \; | \; x_{src}!\langle {Q} \rangle \red P\{\quotep{Q}/y}\} }
  \and \\
  \inferrule* [lab=Par] {{P} \red {P}'} {{{P} | {Q}} \red {{P}' | {Q}}}
  \and
  \inferrule* [lab=Equiv]{{{P} \scong {P}'} \andalso {{P}' \red {Q}'} \andalso {{Q}' \scong {Q}}}{{P} \red {Q}}
\end{mathpar}

\begin{eqnarray*}
  match_{\equiv} (\quotep{P},\quotep{Q}) & := & P \equiv Q \\
  match_{\dagger}(\quotep{P},\quotep{Q}) & := & \forall R. P|Q \red^{*} R => R \red^{*} 0 \\
  match_{K}(\quotep{P},\quotep{Q}) & := & K \mbox{ for some context } K
\end{eqnarray*}

$u?(x)P | u!\langle Q \rangle \red P\{\quotep{Q}/x\}$

%We write $\wred$ for $\red^*$, and $P\red$ if $\exists Q $ such that $ P \red Q$.
We write $P\red$ if $\exists Q $ such that $ P \red Q$ and $P\not\red$, otherwise.

\section{Replication}

As mentioned before, it is known that replication (and hence
recursion) can be implemented in a higher-order process algebra
\cite{SangiorgiWalker}. As our first example of calculation with the
machinery thus far presented we give the construction explicitly in
the {\rhoc}.

\begin{eqnarray}
	D_{x} & := & \prefix{x}{y}{(\binpar{\outputp{x}{y}}{@{y}})} \nonumber\\
	\bangp_{x}{P} & := & \binpar{{x}!\langle{\binpar{D_{x}}{P}}\rangle}{D_{x}} \nonumber
\end{eqnarray}

\begin{eqnarray}
	\bangp_{x}{P} & & \nonumber\\
	=
	& {x}!\langle{(\prefix{x}{y}{(\outputp{x}{y} | @{y})) | P}}\rangle 
	      | \prefix{x}{y}{(\outputp{x}{y} | @{y})} & \nonumber\\
	\red
	& (\outputp{x}{y} | @{y})\substn{\quotep{(\prefix{x}{y}{(@{y} | \outputp{x}{y})) | P}}}{y} & \nonumber\\
	=
	& \outputp{x}{\quotep{(\prefix{x}{y}{(\outputp{x}{y} | @{y})) | P}}}
	  | {(\prefix{x}{y}{(\outputp{x}{y} | @{y})) | P}} & \nonumber\\
	\red
	& \ldots & \nonumber\\
	\red^*
	& P | P | \ldots & \nonumber
\end{eqnarray}

Of course, this encoding, as an implementation, runs away, unfolding
$\bangp{P}$ eagerly. A lazier and more implementable replication
operator, restricted to input-guarded processes, may be obtained as follows.

\begin{eqnarray}
\bangp{\prefix{u}{v}{P}} 
	:= 
	\binpar{\lift{x}{\prefix{u}{v}{(\binpar{D(x)}{P})}}}{D(x)} \nonumber
\end{eqnarray}

\begin{remark}
  Note that the lazier definition still does not deal with summation
  or mixed summation (i.e. sums over input and output). The reader is
  invited to construct definitions of replication that deal with these
  features. 

  Further, the definitions are parameterized in a name, $x$. Can you,
  gentle reader, make a definition that eliminates this parameter and
  guarantees no accidental interaction between the replication
  machinery and the process being replicated -- i.e. no accidental
  sharing of names used by the process to get its work done and the
  name(s) used by the replication to effect copying. This latter
  revision of the definition of replication is crucial to obtaining
  the expected identity $!!P \sim !P$.
\end{remark}

\begin{remark}\label{rem:paradoxical_combinator}
  The reader familiar with the lambda calculus will have noticed the
  similarity between $D$ and the paradoxical combinator.

  [Ed. note: the existence of this seems to suggest we have to be more
  restrictive on the set of processes and names we admit if we are to
  support no-cloning.]
\end{remark}

\subsubsection{Bisimulation}

The computational dynamics gives rise to another kind of equivalence,
the equivalence of computational behavior. As previously mentioned
this is typically captured \emph{via} some form of bisimulation.

% The notion we use in this paper is weak barbed bisimulation
% \cite{milner91polyadicpi}.

The notion we use in this paper is derived from weak barbed
bisimulation \cite{milner91polyadicpi}. 

\begin{definition}
An \emph{observation relation}, $\downarrow_{\mathcal N}$, over a set
of names, $\mathcal N$, is the smallest relation satisfying the rules
below.

\infrule[Out-barb]{y \in {\mathcal N}, \; x \nameeq y}
		  {\outputp{x}{v} \downarrow_{\mathcal N} x}
\infrule[Par-barb]{\mbox{$P\downarrow_{\mathcal N} x$ or $Q\downarrow_{\mathcal N} x$}}
		  {\binpar{P}{Q} \downarrow_{\mathcal N} x}

We write $P \Downarrow_{\mathcal N} x$ if there is $Q$ such that 
$P \wred Q$ and $Q \downarrow_{\mathcal N} x$.
\end{definition}

\begin{definition}
%\label{def.bbisim}
An  ${\mathcal N}$-\emph{barbed bisimulation} over a set of names, ${\mathcal N}$, is a symmetric binary relation 
${\mathcal S}_{\mathcal N}$ between agents such that $P\rel{S}_{\mathcal N}Q$ implies:
\begin{enumerate}
\item If $P \red P'$ then $Q \wred Q'$ and $P'\rel{S}_{\mathcal N} Q'$.
\item If $P\downarrow_{\mathcal N} x$, then $Q\Downarrow_{\mathcal N} x$.
\end{enumerate}
$P$ is ${\mathcal N}$-barbed bisimilar to $Q$, written
$P \wbbisim_{\mathcal N} Q$, if $P \rel{S}_{\mathcal N} Q$ for some ${\mathcal N}$-barbed bisimulation ${\mathcal S}_{\mathcal N}$.
\end{definition}

$\mathcal{R} \subseteq \pi \times \pi$

$P \mathcal{R} Q => \forall P'. P \red P' \Rightarrow \exists Q'. Q \red Q', P' \mathcal{R} Q'$

$P \vdash x \Rightarrow Q \vdash x$

\begin{mathpar}
  \inferrule*[lab=Out-barb]{x \nameeq y}{{y}!\langle{Q}\rangle \vdash x}
  \and
  \inferrule*[lab=Par-barb]{\mbox{$P\vdash x$ or $Q\vdash x$}}{\binpar{P}{Q} \vdash x}
\end{mathpar}

\subsubsection{Contexts}

One of the principle advantages of computational calculi like the
$\pi$-calculus is a well-defined notion of context,
contextual-equivalence and a correlation between
contextual-equivalence and notions of bisimulation. The notion of
context allows the decomposition of a process into (sub-)process and
its syntactic environment, its context. Thus, a context may be
thought of as a process with a ``hole'' (written $\Box$) in it. The
application of a context $M$ to a process $P$, written $M[P]$, is
tantamount to filling the hole in $M$ with $P$. In this paper we do
not need the full weight of this theory, but do make use of the notion
of context in the proof the main theorem. 

\begin{mathpar}
  \inferrule* [lab=summation] {} {{M_{M},M_{N}} \bc \Box \;|\; x.M_{A} \;|\; M_{M}+M_{N}}
  \and
  \inferrule* [lab=agent] {} {{M_{A}} \bc (\vec{x})M_{P} \;| \; \clift{P_0,\ldots,M_{P},\ldots,P_N}}
  \and \\
  \inferrule* [lab=process] {} {{M_{P}} \bc M_{N} \;| \;P|M_{P} }
\end{mathpar} 

\begin{mathpar}
  \inferrule* [lab=sychronization] {} {M_{N} \bc \Box \;|\; x?M_{F} \;|\; x!M_{C}}
  \and
  \inferrule* [lab=abstraction] {} {{M_{F}} \bc (x)M_{P} }
  \and
  \inferrule* [lab=concretion] {} {{M_{C}} \bc \langle M_{P} \rangle }
  \and \\
  \inferrule* [lab=process] {} {{M_{P}} \bc M_{N} \;| \;P|M_{P} }
\end{mathpar}

\begin{definition}[contextual application] Given a context $M$, and
  process $P$, we define the \emph{contextual application}, $M[P] :=
  M\{P/\Box\}$. That is, the contextual application of M to P is the
  substitution of $P$ for $\Box$ in $M$.
\end{definition}

$\meaningof{-} : L \to \mathcal{P}(\pi)$

\begin{mathpar}
  \inferrule* [lab=collection] {} {\meaningof{true} = \pi, \and \meaningof{~E} = \pi \setminus \meaningof{E}, \and \meaningof{E_{1} \& E_{2}} = \meaningof{E_{1}} \cap \meaningof{E_{2}}}
\end{mathpar}

\begin{mathpar}
  \inferrule* [lab=structure] {} {\meaningof{0} = \{ P \in \pi | P \equiv 0 \}, \and \\ \meaningof{E_1 | E_2} = \{ P \in \pi | P \equiv P_{1} | P_{2}, P_{1} \in \meaningof{E_{1}}, P_{2} \in \meaningof{E_2}\} }
\end{mathpar}

\begin{mathpar}
 \inferrule* [lab=behavior] {} {\meaningof{\langle a?b \rangle E} = \{ P \in \pi | P \equiv Q | u?(y)P', \\ \and \\\\ \and \\ \;\;\; u \in \meaningof{a}, \forall z.P'\{z/y\} \in \meaningof{E\{z/b\}}\}, \and \\ \meaningof{a!E} = \{ P \in \pi | P \equiv Q | x!\langle P' \rangle, x \in \meaningof{a} P' \in \meaningof{E}\} }
\end{mathpar}

\begin{mathpar}
 \inferrule* [lab=nominal] {} {\meaningof{\quotep{E}} = \{ \quotep{P} \in \quotep{\pi} | P \in \meaningof{E} \}, \and \meaningof{\quotep{P}} = \{ \quotep{Q} \in \quotep{\pi} | P \equiv Q \} \and \\ \meaningof{@\quotep{E}} = \{ P \in \pi | P \equiv @x, x \in \meaningof{E} \}}
\end{mathpar}

\begin{eqnarray*}
  \\
  \meaningof{-} : TS \to ST
\end{eqnarray*}

\begin{eqnarray*}
  \\
  L : TS \to ST
\end{eqnarray*}

\begin{eqnarray*}
  \\
  P \models E \iff P \in \meaningof{E}
\end{eqnarray*}

\begin{eqnarray*}
  P \approx_{L} Q \iff \forall E \in L. P \models E \iff Q \models E
\end{eqnarray*}

\begin{eqnarray*}
  P \approx_{K} Q
\end{eqnarray*}

\begin{eqnarray*}
  P \approx Q
\end{eqnarray*}

$\approx_{K} = \approx = \approx_{L}$

\subsubsection{Contextual duality}

Note that contexts extend the quotation operation to a family of
operations from processes to names. Given a context, $M$, we can
define a \emph{nominal context}, $\quotep{M}$ by $\quotep{M}[P] :=
\quotep{M[P]}$. To foreshadow what is to come we observe that these
operations enjoy a duality with processes very much like the duality
between vectors and maps from vectors to scalars.

Further, because the calculus is essentially higher-order, we have a
correspondence between contexts and processes. More specifically,
given a name $x$ and a context $M$ we can construct $M^{*}_{x}$ such
that 

\begin{mathpar}
  M^{*}_{x} | \lift{x}{P} \red M[P]
\end{mathpar}

namely,

\begin{mathpar}
  M^{*}_{x} := x?(u).M[\dropn{u}]
\end{mathpar}

The dependence of $M^{*}_{x}$ on a name makes it an abstraction, 

\begin{mathpar}
  M^{*} := (x)x?(u).M[\dropn{u}]
\end{mathpar}

\subsection{Additional notation}

It will sometimes be convenient to denote the process a name
quotes. We already have the notation $x = \quotep{P}$, but it will be
convenient to introduce an alternate notation, $\procn{x}$, when we
want to emphasize the connection to the use of the name. Note that, by
virtue of name equivalence, $\quotep{\procn{x}} \nameeq x$; so, the
notation is consistent with previous definitions.

Further, because names have structure it is possible to effect
substitutions on the basis of that structure. This means we need to
upgrade our notation for substitutions, which we accomplish by
adapting comprehension notation. Thus,

\begin{mathpar}
  P\{ y / x : x \in S \}
\end{mathpar}

is interpreted to mean the process derived from P by replacing (in a
capture-avoiding manner) each occurrence of $x$ in $S$ by $y$. For example,

\begin{mathpar}
  P\{ \quotep{\procn{x}|\procn{x}} / x : x \in \freenames{P} \}
\end{mathpar}

will replace each (occurrence) of a free name $x$ in $P$ by
$\quotep{\procn{x}|\procn{x}}$.

Also, we will avail ourselves of the notation $x^{L}$ and $x^{R}$ to
denote injections of a name into disjoint copies of the name
space. There are numerous ways to accomplish this. One example can be
found in \cite{MeredithR05}. This notation overloads to vectors of
names: $\vec{x}^{\pi} := (x_{i}^{\pi} \; : \; 0 \leq i < |\vec{x}| )$ where $\pi \in \{L,R\}$.

We also use $P^{\Box} := P|\Box$.

In \cite{MeredithR05} an interpretation of the new operator is
given. It turns out that there are several possible interpretations
all enjoying the requisite algebraic properties of the operator (see
\cite{milner91polyadicpi}). We will therefore make liberal use of
$(\nu\; \vec{x})P$.

% subsection the_syntax_and_semantics_of_the_notation_system (end)   

\input{qm2pi.qmops} 

\input{qm2pi.sterngerlach} 

\input{qm2pi.metric} 

% section concurrent_process_calculi (end)

%\input{qm2pi.proofsketch}

% section proof sketch (end)

%\input{qm2pi.slviaknots} 

% section spatial logic via knots (end)

\input{qm2pi.conclusion}

% section conclusion (end)

%\input{qm2pi.dtcodes} 

% section wiring algorithm (end)

\input{qm2pi.ack} 

% section acknowledgments (end)

\newpage


\bibliographystyle{plain}   
\bibliography{../../biblios/main.bib}

\input{qm2pi.rhodetails}

\end{document}



\end{document}

 

% section concurrent_process_calculi (end)

%\documentclass[12pt]{llncs}
%\documentclass{jktr}

\usepackage[pdftex]{hyperref}                   
\usepackage {listings}
\usepackage {mathpartir}
\usepackage{bcprules}
%\usepackage{listings}
                       
\usepackage{graphicx} 
%\usepackage[margins=2.5cm,nohead,nofoot]{geometry}
%\usepackage{geometry}
\usepackage{amsfonts}
\usepackage{amstext}
\usepackage{latexsym}
\usepackage{amssymb}
\usepackage{color}


%\include{myPreamble}
\documentclass[12pt]{llncs}
%\documentclass{jktr}

\usepackage[pdftex]{hyperref}                   
\usepackage {listings}
\usepackage {mathpartir}
\usepackage{bcprules}
%\usepackage{listings}
                       
\usepackage{graphicx} 
%\usepackage[margins=2.5cm,nohead,nofoot]{geometry}
%\usepackage{geometry}
\usepackage{amsfonts}
\usepackage{amstext}
\usepackage{latexsym}
\usepackage{amssymb}
\usepackage{color}


%\include{myPreamble}
\include{qm2pi.local} 

%\ifpdf
%\usepackage[pdftex]{graphicx}
%\else
%\usepackage{graphicx}
%\fi

 % \ifpdf
%  \usepackage{pdfsync}
%  \if


%\title{Brief Article}
%\author{David F. Snyder}
%\author{L.G. Meredith}

%\address{Dept. of Math., Texas State University--San Marcos, San Marcos, TX 78666}
       
\pagestyle{empty}


\begin{document}

\lstset{language=[Objective]Caml,frame=shadowbox}

\input{qm2pi.front}

% section front matter (end)

\input{qm2pi.intro} 
 
% section introduction (end)

% \input{qm2pi.knotations} 

% section notation (end)

\input{qm2pi.process.calculi} 

% section concurrent_process_calculi_and_spatial_logics_ (end)
    
%\input{qm2pi.knots2pi} 

%\input{qm2pi.trefoil} 

%\input{qm2pi.mainthm} 

% subsection basic_interpretation (end)

%\input{qm2pi.rho.presentation} 
\subsection{The syntax and semantics of the notation system}\label{sub:the_syntax_and_semantics_of_the_notation_system} % (fold)

We now summarize a technical presentation of the calculus that
embodies our theory of dynamics. The typical presentation of such a
calculus follows the style of giving generators and relations on
them. The grammar, below, describing term constructors, freely
generates the set of processes, $\Proc$. This set is then quotiented
by a relation known as structural congruence and it is over this set
that the notion of dynamics is expressed. This presentation is
essentially that of \cite{MeredithR05} with the addition of
polyadicity and summation. For readability we have relegated some of
the technical subtleties to an appendix.

\subsubsection{Process grammar}\label{subsub:process_grammar}

\begin{mathpar}
  \inferrule* [lab=synchronization] {} {{M} \bc \pzero \;|\; x?F \;|\; x!C }
  \and
  \inferrule* [lab=abstraction] {} {{F} \bc (x)P}
  \and
  \inferrule* [lab=concretion] {} {{C} \bc \langle Q \rangle}
  \and
  \inferrule* [lab=process] {} {{P,Q} \bc M \;| \;P|Q \;|\; @{x}}
  \and
  \inferrule* [lab=name] {} {{x} \bc \quotep{P}}
\end{mathpar} 

Note that $\vec{x}$ (resp. $\vec{P}$) denotes a vector of names
(resp. processes) of length $|\vec{x}|$ (resp. $|\vec{P}|$). We adopt
the following useful abbreviations.

\begin{mathpar}
   x?(\vec{y}).P := x.(\vec{y})P \and  x\clift{\vec{P}} := x.\clift{\vec{P}}
   \and x!(y) := \lift{x}{\dropn{y}}
   \and \Pi_{i=0}^{n-1}P_i := P_0 | \ldots | P_{n-1}
\end{mathpar}

\subsubsection{Structural congruence}

\paragraph{Free and bound names and alpha-equivalence.} At the
core of structural equivalence is alpha-equivalence which identifies
process that are the same up to a change of variable. Formally, we
recognize the distinction between free and bound names. The free names
of a process, $\freenames{P}$, may be calculated recursively as
follows:

\begin{mathpar}
\freenames{\pzero} := \emptyset
  \and \\
  \freenames{x?(y).P} := \{ x \} \cup (\freenames{P} \setminus \{ y \})
  \and 
  \freenames{x!\langle P \rangle} := \{ x \} \cup \{ P \} 
  \and \\
  \freenames{P|Q} := \freenames{P} \cup \freenames{Q}
  \and \\
  \freenames{@{x}} := \{ x \}
\end{mathpar}

$\pi$
$\quotep{\pi}$

$\freenames{-} : \pi \to \mathcal{P}(\quotep{\pi})$

\begin{eqnarray*}
  \freenames{\pzero} & := & \emptyset \\
  \freenames{x?(y).P} & := & \{ x \} \cup (\freenames{P} \setminus \{ y \}) \\
  \freenames{x!\langle P \rangle} & := & \{ x \} \cup \{ P \} \\
  \freenames{P|Q} & := & \freenames{P} \cup \freenames{Q} \\
  \freenames{\dropn{x}} & := & \{ x \}
\end{eqnarray*}

The bound names of a process, $\boundnames{P}$, are those names occurring in $P$
that are not free. For example, in $x?(y).0$, the name $x$ is free, while $y$ is bound.

\begin{mathpar}
  \inferrule* [lab=monoidal-laws] {} { P|Q \equiv Q|P \and P|0 \equiv P \and P|(Q|R) \equiv (P|Q)|R }
\end{mathpar}

\begin{mathpar}
  \inferrule* [lab=alpha-equivalence] {} { (x)P \equiv (y)P\{y/x\} \and y \not\in \freenames{P} }
\end{mathpar}

\begin{definition}
Then two processes, $P,Q$, are alpha-equivalent if $P = Q\{\vec{y}/\vec{x}\}$ for
some $\vec{x} \in \boundnames{Q},\vec{y} \in \boundnames{P}$, where $Q\{\vec{y}/\vec{x}\}$
denotes the capture-avoiding substitution of $\vec{y}$ for $\vec{x}$ in $Q$.
\end{definition}

\begin{definition}
  The {\em structural congruence} \cite{SangiorgiWalker} , $\equiv$,
  between processes is the least congruence containing
  alpha-equivalence, satisfying the abelian monoid laws
  (associativity, commutativity and $\pzero$ as identity) for parallel
  composition $|$ and for summation $+$.
\end{definition}

\subsection{Name equivalence}

We take name equivalence, written $\nameeq$, to be the smallest
equivalence relation generated by the following rules.

\begin{mathpar}
\inferrule*[lab=Quote-drop]
{ }
{ \quotep{@{x}} \nameeq x }

\inferrule*[lab=Struct-equiv]
{ P \scong Q }
{ \quotep{P} \nameeq \quotep{Q} }
\end{mathpar}

The astute reader will have noticed that the mutual recursion of names
and processes imposes a mutual recursion on alpha-equivalence and
structural equivalence via name-equivalence. Fortunately, all of this
works out pleasantly and we may calculate in the natural way, free of
concern. The reader interested in the details is referred to the
appendix \ref{appendix:rho_details}.

\subsection{Substitution}

We use $\Proc$ for the set of processes, $\QProc$ for the set of
names, and $\id{\{}\vec{y} / \vec{x} \id{\}}$ to denote partial maps,
$s : \QProc \rightarrow \QProc$. A map, $s$ lifts, uniquely, to a map
on process terms, $\widehat{s} : \Proc \rightarrow \Proc$ by the
following equations.

\begin{mathpar}
  (0) \psubstp{Q}{P} := 0 \\
  (R \juxtap S) \psubstp{Q}{P}
  :=    
  (R)\psubstp{Q}{P} \juxtap (S) \psubstp{Q}{P} \\
  (x?(y).R) \psubstp{Q}{P}    
  :=    
  (x)\substp{Q}{P} (z)\concat( (R \psubstn{z}{y}) \psubstp{Q}{P} ) \\
  (\lift{x}{R}) \psubstp{Q}{P}  
  :=
  \lift{(x)\substp{Q}{P}}{ R \psubstp{Q}{P} } \\
%   (\dropn{x})  \psubstp{Q}{P}       
%   := 
%   \left\{ 
%     \begin{array}{ccc} 
%       \dropn{\quotep{Q}} & & x \nameeq \quotep{P} \\
%       \dropn{x} & & otherwise \\
%     \end{array}
%   \right. 
  (\dropn{x})  \psubstp{Q}{P}       
  := 
  \left\{ 
    \begin{array}{ccc} 
      Q & & x \nameeq \quotep{P} \\
      \dropn{x} & & otherwise \\
    \end{array}
  \right.
\end{mathpar}
 

where

\begin{eqnarray}
  (x)\id{\{} \lpquote Q \rpquote / \lpquote P \rpquote \id{\}}            = 
  \left\{ 
    \begin{array}{ccc}
      \lpquote Q \rpquote & & x \nameeq \lpquote P \rpquote \\
      x & & otherwise \\
    \end{array}
  \right. \nonumber
\end{eqnarray}

and $z$ is chosen distinct from $\quotep{P}$, $\quotep{Q}$, the free
names in $Q$, and all the names in $R$. Our $\alpha$-equivalence will
be built in the standard way from this substitution.

\begin{remark}\label{rem:no_self_referential_names}
  One consequence of these definitions is that $\forall P. \quotep{P}
  \not\in \freenames{P}$.
\end{remark}

\subsection{ Dynamic quote: an example }

Anticipating something of what's to come, consider applying the
substitution, $\widehat{\id{\{}u / z \id{\}}}$, to the following pair
of processes, $\lift{w}{y!(z)}$ and $w[ \lpquote y!(z) \rpquote ]$.

\begin{eqnarray}
	\lift{w}{y!(z)}\widehat{\id{\{}u / z \id{\}}}
		& = &
		\lift{w}{y!(u)} \nonumber\\
	w[ \lpquote y!(z) \rpquote ] \widehat{ \id{\{}u / z \id{\}} }
		& = &
		w[ \lpquote y!(z) \rpquote ] \nonumber
\end{eqnarray}

Because the body of the process between quotes is impervious to
substitution, we get radically different answers. In fact, by
examining the first process in an input context,
e.g. $x?(z).\lift{w}{y!(z)}$, we see that the process under the lift
operator may be shaped by prefixed inputs binding a name inside it. In
this sense, the lift operator will be seen as a way to dynamically
construct processes before reifying them as names.

Finally equipped with these standard features we can present the
dynamics of the calculus.

\subsubsection{Operational semantics} 

Finally, we introduce the computational dynamics. What marks these
algebras as distinct from other more traditionally studied algebraic
structures, e.g. vector spaces or polynomial rings, is the manner in
which dynamics is captured. In traditional structures, dynamics is typically
expressed through morphisms between such structures, as in linear maps
between vector spaces or morphisms between rings. In algebras
associated with the semantics of computation, the dynamics is
expressed as part of the algebraic structure itself, through a
reduction reduction relation typically denoted by $\red$. Below, we
give a recursive presentation of this relation for the calculus used
in the encoding.

$\red \subseteq \pi \times \pi$
$\red : \pi \to \mathcal{P}(\pi)$

\begin{mathpar}
  \inferrule* [lab=Comm] { \textsf{match}( x_{src}, x_{trgt} ) } { x_{trgt}?(y)P \; | \; x_{src}!\langle {Q} \rangle \red P\{\quotep{Q}/y}\} }
  \and \\
  \inferrule* [lab=Par] {{P} \red {P}'} {{{P} | {Q}} \red {{P}' | {Q}}}
  \and
  \inferrule* [lab=Equiv]{{{P} \scong {P}'} \andalso {{P}' \red {Q}'} \andalso {{Q}' \scong {Q}}}{{P} \red {Q}}
\end{mathpar}

\begin{eqnarray*}
  match_{\equiv} (\quotep{P},\quotep{Q}) & := & P \equiv Q \\
  match_{\dagger}(\quotep{P},\quotep{Q}) & := & \forall R. P|Q \red^{*} R => R \red^{*} 0 \\
  match_{K}(\quotep{P},\quotep{Q}) & := & K \mbox{ for some context } K
\end{eqnarray*}

$u?(x)P | u!\langle Q \rangle \red P\{\quotep{Q}/x\}$

%We write $\wred$ for $\red^*$, and $P\red$ if $\exists Q $ such that $ P \red Q$.
We write $P\red$ if $\exists Q $ such that $ P \red Q$ and $P\not\red$, otherwise.

\section{Replication}

As mentioned before, it is known that replication (and hence
recursion) can be implemented in a higher-order process algebra
\cite{SangiorgiWalker}. As our first example of calculation with the
machinery thus far presented we give the construction explicitly in
the {\rhoc}.

\begin{eqnarray}
	D_{x} & := & \prefix{x}{y}{(\binpar{\outputp{x}{y}}{@{y}})} \nonumber\\
	\bangp_{x}{P} & := & \binpar{{x}!\langle{\binpar{D_{x}}{P}}\rangle}{D_{x}} \nonumber
\end{eqnarray}

\begin{eqnarray}
	\bangp_{x}{P} & & \nonumber\\
	=
	& {x}!\langle{(\prefix{x}{y}{(\outputp{x}{y} | @{y})) | P}}\rangle 
	      | \prefix{x}{y}{(\outputp{x}{y} | @{y})} & \nonumber\\
	\red
	& (\outputp{x}{y} | @{y})\substn{\quotep{(\prefix{x}{y}{(@{y} | \outputp{x}{y})) | P}}}{y} & \nonumber\\
	=
	& \outputp{x}{\quotep{(\prefix{x}{y}{(\outputp{x}{y} | @{y})) | P}}}
	  | {(\prefix{x}{y}{(\outputp{x}{y} | @{y})) | P}} & \nonumber\\
	\red
	& \ldots & \nonumber\\
	\red^*
	& P | P | \ldots & \nonumber
\end{eqnarray}

Of course, this encoding, as an implementation, runs away, unfolding
$\bangp{P}$ eagerly. A lazier and more implementable replication
operator, restricted to input-guarded processes, may be obtained as follows.

\begin{eqnarray}
\bangp{\prefix{u}{v}{P}} 
	:= 
	\binpar{\lift{x}{\prefix{u}{v}{(\binpar{D(x)}{P})}}}{D(x)} \nonumber
\end{eqnarray}

\begin{remark}
  Note that the lazier definition still does not deal with summation
  or mixed summation (i.e. sums over input and output). The reader is
  invited to construct definitions of replication that deal with these
  features. 

  Further, the definitions are parameterized in a name, $x$. Can you,
  gentle reader, make a definition that eliminates this parameter and
  guarantees no accidental interaction between the replication
  machinery and the process being replicated -- i.e. no accidental
  sharing of names used by the process to get its work done and the
  name(s) used by the replication to effect copying. This latter
  revision of the definition of replication is crucial to obtaining
  the expected identity $!!P \sim !P$.
\end{remark}

\begin{remark}\label{rem:paradoxical_combinator}
  The reader familiar with the lambda calculus will have noticed the
  similarity between $D$ and the paradoxical combinator.

  [Ed. note: the existence of this seems to suggest we have to be more
  restrictive on the set of processes and names we admit if we are to
  support no-cloning.]
\end{remark}

\subsubsection{Bisimulation}

The computational dynamics gives rise to another kind of equivalence,
the equivalence of computational behavior. As previously mentioned
this is typically captured \emph{via} some form of bisimulation.

% The notion we use in this paper is weak barbed bisimulation
% \cite{milner91polyadicpi}.

The notion we use in this paper is derived from weak barbed
bisimulation \cite{milner91polyadicpi}. 

\begin{definition}
An \emph{observation relation}, $\downarrow_{\mathcal N}$, over a set
of names, $\mathcal N$, is the smallest relation satisfying the rules
below.

\infrule[Out-barb]{y \in {\mathcal N}, \; x \nameeq y}
		  {\outputp{x}{v} \downarrow_{\mathcal N} x}
\infrule[Par-barb]{\mbox{$P\downarrow_{\mathcal N} x$ or $Q\downarrow_{\mathcal N} x$}}
		  {\binpar{P}{Q} \downarrow_{\mathcal N} x}

We write $P \Downarrow_{\mathcal N} x$ if there is $Q$ such that 
$P \wred Q$ and $Q \downarrow_{\mathcal N} x$.
\end{definition}

\begin{definition}
%\label{def.bbisim}
An  ${\mathcal N}$-\emph{barbed bisimulation} over a set of names, ${\mathcal N}$, is a symmetric binary relation 
${\mathcal S}_{\mathcal N}$ between agents such that $P\rel{S}_{\mathcal N}Q$ implies:
\begin{enumerate}
\item If $P \red P'$ then $Q \wred Q'$ and $P'\rel{S}_{\mathcal N} Q'$.
\item If $P\downarrow_{\mathcal N} x$, then $Q\Downarrow_{\mathcal N} x$.
\end{enumerate}
$P$ is ${\mathcal N}$-barbed bisimilar to $Q$, written
$P \wbbisim_{\mathcal N} Q$, if $P \rel{S}_{\mathcal N} Q$ for some ${\mathcal N}$-barbed bisimulation ${\mathcal S}_{\mathcal N}$.
\end{definition}

$\mathcal{R} \subseteq \pi \times \pi$

$P \mathcal{R} Q => \forall P'. P \red P' \Rightarrow \exists Q'. Q \red Q', P' \mathcal{R} Q'$

$P \vdash x \Rightarrow Q \vdash x$

\begin{mathpar}
  \inferrule*[lab=Out-barb]{x \nameeq y}{{y}!\langle{Q}\rangle \vdash x}
  \and
  \inferrule*[lab=Par-barb]{\mbox{$P\vdash x$ or $Q\vdash x$}}{\binpar{P}{Q} \vdash x}
\end{mathpar}

\subsubsection{Contexts}

One of the principle advantages of computational calculi like the
$\pi$-calculus is a well-defined notion of context,
contextual-equivalence and a correlation between
contextual-equivalence and notions of bisimulation. The notion of
context allows the decomposition of a process into (sub-)process and
its syntactic environment, its context. Thus, a context may be
thought of as a process with a ``hole'' (written $\Box$) in it. The
application of a context $M$ to a process $P$, written $M[P]$, is
tantamount to filling the hole in $M$ with $P$. In this paper we do
not need the full weight of this theory, but do make use of the notion
of context in the proof the main theorem. 

\begin{mathpar}
  \inferrule* [lab=summation] {} {{M_{M},M_{N}} \bc \Box \;|\; x.M_{A} \;|\; M_{M}+M_{N}}
  \and
  \inferrule* [lab=agent] {} {{M_{A}} \bc (\vec{x})M_{P} \;| \; \clift{P_0,\ldots,M_{P},\ldots,P_N}}
  \and \\
  \inferrule* [lab=process] {} {{M_{P}} \bc M_{N} \;| \;P|M_{P} }
\end{mathpar} 

\begin{mathpar}
  \inferrule* [lab=sychronization] {} {M_{N} \bc \Box \;|\; x?M_{F} \;|\; x!M_{C}}
  \and
  \inferrule* [lab=abstraction] {} {{M_{F}} \bc (x)M_{P} }
  \and
  \inferrule* [lab=concretion] {} {{M_{C}} \bc \langle M_{P} \rangle }
  \and \\
  \inferrule* [lab=process] {} {{M_{P}} \bc M_{N} \;| \;P|M_{P} }
\end{mathpar}

\begin{definition}[contextual application] Given a context $M$, and
  process $P$, we define the \emph{contextual application}, $M[P] :=
  M\{P/\Box\}$. That is, the contextual application of M to P is the
  substitution of $P$ for $\Box$ in $M$.
\end{definition}

$\meaningof{-} : L \to \mathcal{P}(\pi)$

\begin{mathpar}
  \inferrule* [lab=collection] {} {\meaningof{true} = \pi, \and \meaningof{~E} = \pi \setminus \meaningof{E}, \and \meaningof{E_{1} \& E_{2}} = \meaningof{E_{1}} \cap \meaningof{E_{2}}}
\end{mathpar}

\begin{mathpar}
  \inferrule* [lab=structure] {} {\meaningof{0} = \{ P \in \pi | P \equiv 0 \}, \and \\ \meaningof{E_1 | E_2} = \{ P \in \pi | P \equiv P_{1} | P_{2}, P_{1} \in \meaningof{E_{1}}, P_{2} \in \meaningof{E_2}\} }
\end{mathpar}

\begin{mathpar}
 \inferrule* [lab=behavior] {} {\meaningof{\langle a?b \rangle E} = \{ P \in \pi | P \equiv Q | u?(y)P', \\ \and \\\\ \and \\ \;\;\; u \in \meaningof{a}, \forall z.P'\{z/y\} \in \meaningof{E\{z/b\}}\}, \and \\ \meaningof{a!E} = \{ P \in \pi | P \equiv Q | x!\langle P' \rangle, x \in \meaningof{a} P' \in \meaningof{E}\} }
\end{mathpar}

\begin{mathpar}
 \inferrule* [lab=nominal] {} {\meaningof{\quotep{E}} = \{ \quotep{P} \in \quotep{\pi} | P \in \meaningof{E} \}, \and \meaningof{\quotep{P}} = \{ \quotep{Q} \in \quotep{\pi} | P \equiv Q \} \and \\ \meaningof{@\quotep{E}} = \{ P \in \pi | P \equiv @x, x \in \meaningof{E} \}}
\end{mathpar}

\begin{eqnarray*}
  \\
  \meaningof{-} : TS \to ST
\end{eqnarray*}

\begin{eqnarray*}
  \\
  L : TS \to ST
\end{eqnarray*}

\begin{eqnarray*}
  \\
  P \models E \iff P \in \meaningof{E}
\end{eqnarray*}

\begin{eqnarray*}
  P \approx_{L} Q \iff \forall E \in L. P \models E \iff Q \models E
\end{eqnarray*}

\begin{eqnarray*}
  P \approx_{K} Q
\end{eqnarray*}

\begin{eqnarray*}
  P \approx Q
\end{eqnarray*}

$\approx_{K} = \approx = \approx_{L}$

\subsubsection{Contextual duality}

Note that contexts extend the quotation operation to a family of
operations from processes to names. Given a context, $M$, we can
define a \emph{nominal context}, $\quotep{M}$ by $\quotep{M}[P] :=
\quotep{M[P]}$. To foreshadow what is to come we observe that these
operations enjoy a duality with processes very much like the duality
between vectors and maps from vectors to scalars.

Further, because the calculus is essentially higher-order, we have a
correspondence between contexts and processes. More specifically,
given a name $x$ and a context $M$ we can construct $M^{*}_{x}$ such
that 

\begin{mathpar}
  M^{*}_{x} | \lift{x}{P} \red M[P]
\end{mathpar}

namely,

\begin{mathpar}
  M^{*}_{x} := x?(u).M[\dropn{u}]
\end{mathpar}

The dependence of $M^{*}_{x}$ on a name makes it an abstraction, 

\begin{mathpar}
  M^{*} := (x)x?(u).M[\dropn{u}]
\end{mathpar}

\subsection{Additional notation}

It will sometimes be convenient to denote the process a name
quotes. We already have the notation $x = \quotep{P}$, but it will be
convenient to introduce an alternate notation, $\procn{x}$, when we
want to emphasize the connection to the use of the name. Note that, by
virtue of name equivalence, $\quotep{\procn{x}} \nameeq x$; so, the
notation is consistent with previous definitions.

Further, because names have structure it is possible to effect
substitutions on the basis of that structure. This means we need to
upgrade our notation for substitutions, which we accomplish by
adapting comprehension notation. Thus,

\begin{mathpar}
  P\{ y / x : x \in S \}
\end{mathpar}

is interpreted to mean the process derived from P by replacing (in a
capture-avoiding manner) each occurrence of $x$ in $S$ by $y$. For example,

\begin{mathpar}
  P\{ \quotep{\procn{x}|\procn{x}} / x : x \in \freenames{P} \}
\end{mathpar}

will replace each (occurrence) of a free name $x$ in $P$ by
$\quotep{\procn{x}|\procn{x}}$.

Also, we will avail ourselves of the notation $x^{L}$ and $x^{R}$ to
denote injections of a name into disjoint copies of the name
space. There are numerous ways to accomplish this. One example can be
found in \cite{MeredithR05}. This notation overloads to vectors of
names: $\vec{x}^{\pi} := (x_{i}^{\pi} \; : \; 0 \leq i < |\vec{x}| )$ where $\pi \in \{L,R\}$.

We also use $P^{\Box} := P|\Box$.

In \cite{MeredithR05} an interpretation of the new operator is
given. It turns out that there are several possible interpretations
all enjoying the requisite algebraic properties of the operator (see
\cite{milner91polyadicpi}). We will therefore make liberal use of
$(\nu\; \vec{x})P$.

% subsection the_syntax_and_semantics_of_the_notation_system (end)   

\input{qm2pi.qmops} 

\input{qm2pi.sterngerlach} 

\input{qm2pi.metric} 

% section concurrent_process_calculi (end)

%\input{qm2pi.proofsketch}

% section proof sketch (end)

%\input{qm2pi.slviaknots} 

% section spatial logic via knots (end)

\input{qm2pi.conclusion}

% section conclusion (end)

%\input{qm2pi.dtcodes} 

% section wiring algorithm (end)

\input{qm2pi.ack} 

% section acknowledgments (end)

\newpage


\bibliographystyle{plain}   
\bibliography{../../biblios/main.bib}

\input{qm2pi.rhodetails}

\end{document}

 

%\ifpdf
%\usepackage[pdftex]{graphicx}
%\else
%\usepackage{graphicx}
%\fi

 % \ifpdf
%  \usepackage{pdfsync}
%  \if


%\title{Brief Article}
%\author{David F. Snyder}
%\author{L.G. Meredith}

%\address{Dept. of Math., Texas State University--San Marcos, San Marcos, TX 78666}
       
\pagestyle{empty}


\begin{document}

\lstset{language=[Objective]Caml,frame=shadowbox}

\documentclass[12pt]{llncs}
%\documentclass{jktr}

\usepackage[pdftex]{hyperref}                   
\usepackage {listings}
\usepackage {mathpartir}
\usepackage{bcprules}
%\usepackage{listings}
                       
\usepackage{graphicx} 
%\usepackage[margins=2.5cm,nohead,nofoot]{geometry}
%\usepackage{geometry}
\usepackage{amsfonts}
\usepackage{amstext}
\usepackage{latexsym}
\usepackage{amssymb}
\usepackage{color}


%\include{myPreamble}
\include{qm2pi.local} 

%\ifpdf
%\usepackage[pdftex]{graphicx}
%\else
%\usepackage{graphicx}
%\fi

 % \ifpdf
%  \usepackage{pdfsync}
%  \if


%\title{Brief Article}
%\author{David F. Snyder}
%\author{L.G. Meredith}

%\address{Dept. of Math., Texas State University--San Marcos, San Marcos, TX 78666}
       
\pagestyle{empty}


\begin{document}

\lstset{language=[Objective]Caml,frame=shadowbox}

\input{qm2pi.front}

% section front matter (end)

\input{qm2pi.intro} 
 
% section introduction (end)

% \input{qm2pi.knotations} 

% section notation (end)

\input{qm2pi.process.calculi} 

% section concurrent_process_calculi_and_spatial_logics_ (end)
    
%\input{qm2pi.knots2pi} 

%\input{qm2pi.trefoil} 

%\input{qm2pi.mainthm} 

% subsection basic_interpretation (end)

%\input{qm2pi.rho.presentation} 
\subsection{The syntax and semantics of the notation system}\label{sub:the_syntax_and_semantics_of_the_notation_system} % (fold)

We now summarize a technical presentation of the calculus that
embodies our theory of dynamics. The typical presentation of such a
calculus follows the style of giving generators and relations on
them. The grammar, below, describing term constructors, freely
generates the set of processes, $\Proc$. This set is then quotiented
by a relation known as structural congruence and it is over this set
that the notion of dynamics is expressed. This presentation is
essentially that of \cite{MeredithR05} with the addition of
polyadicity and summation. For readability we have relegated some of
the technical subtleties to an appendix.

\subsubsection{Process grammar}\label{subsub:process_grammar}

\begin{mathpar}
  \inferrule* [lab=synchronization] {} {{M} \bc \pzero \;|\; x?F \;|\; x!C }
  \and
  \inferrule* [lab=abstraction] {} {{F} \bc (x)P}
  \and
  \inferrule* [lab=concretion] {} {{C} \bc \langle Q \rangle}
  \and
  \inferrule* [lab=process] {} {{P,Q} \bc M \;| \;P|Q \;|\; @{x}}
  \and
  \inferrule* [lab=name] {} {{x} \bc \quotep{P}}
\end{mathpar} 

Note that $\vec{x}$ (resp. $\vec{P}$) denotes a vector of names
(resp. processes) of length $|\vec{x}|$ (resp. $|\vec{P}|$). We adopt
the following useful abbreviations.

\begin{mathpar}
   x?(\vec{y}).P := x.(\vec{y})P \and  x\clift{\vec{P}} := x.\clift{\vec{P}}
   \and x!(y) := \lift{x}{\dropn{y}}
   \and \Pi_{i=0}^{n-1}P_i := P_0 | \ldots | P_{n-1}
\end{mathpar}

\subsubsection{Structural congruence}

\paragraph{Free and bound names and alpha-equivalence.} At the
core of structural equivalence is alpha-equivalence which identifies
process that are the same up to a change of variable. Formally, we
recognize the distinction between free and bound names. The free names
of a process, $\freenames{P}$, may be calculated recursively as
follows:

\begin{mathpar}
\freenames{\pzero} := \emptyset
  \and \\
  \freenames{x?(y).P} := \{ x \} \cup (\freenames{P} \setminus \{ y \})
  \and 
  \freenames{x!\langle P \rangle} := \{ x \} \cup \{ P \} 
  \and \\
  \freenames{P|Q} := \freenames{P} \cup \freenames{Q}
  \and \\
  \freenames{@{x}} := \{ x \}
\end{mathpar}

$\pi$
$\quotep{\pi}$

$\freenames{-} : \pi \to \mathcal{P}(\quotep{\pi})$

\begin{eqnarray*}
  \freenames{\pzero} & := & \emptyset \\
  \freenames{x?(y).P} & := & \{ x \} \cup (\freenames{P} \setminus \{ y \}) \\
  \freenames{x!\langle P \rangle} & := & \{ x \} \cup \{ P \} \\
  \freenames{P|Q} & := & \freenames{P} \cup \freenames{Q} \\
  \freenames{\dropn{x}} & := & \{ x \}
\end{eqnarray*}

The bound names of a process, $\boundnames{P}$, are those names occurring in $P$
that are not free. For example, in $x?(y).0$, the name $x$ is free, while $y$ is bound.

\begin{mathpar}
  \inferrule* [lab=monoidal-laws] {} { P|Q \equiv Q|P \and P|0 \equiv P \and P|(Q|R) \equiv (P|Q)|R }
\end{mathpar}

\begin{mathpar}
  \inferrule* [lab=alpha-equivalence] {} { (x)P \equiv (y)P\{y/x\} \and y \not\in \freenames{P} }
\end{mathpar}

\begin{definition}
Then two processes, $P,Q$, are alpha-equivalent if $P = Q\{\vec{y}/\vec{x}\}$ for
some $\vec{x} \in \boundnames{Q},\vec{y} \in \boundnames{P}$, where $Q\{\vec{y}/\vec{x}\}$
denotes the capture-avoiding substitution of $\vec{y}$ for $\vec{x}$ in $Q$.
\end{definition}

\begin{definition}
  The {\em structural congruence} \cite{SangiorgiWalker} , $\equiv$,
  between processes is the least congruence containing
  alpha-equivalence, satisfying the abelian monoid laws
  (associativity, commutativity and $\pzero$ as identity) for parallel
  composition $|$ and for summation $+$.
\end{definition}

\subsection{Name equivalence}

We take name equivalence, written $\nameeq$, to be the smallest
equivalence relation generated by the following rules.

\begin{mathpar}
\inferrule*[lab=Quote-drop]
{ }
{ \quotep{@{x}} \nameeq x }

\inferrule*[lab=Struct-equiv]
{ P \scong Q }
{ \quotep{P} \nameeq \quotep{Q} }
\end{mathpar}

The astute reader will have noticed that the mutual recursion of names
and processes imposes a mutual recursion on alpha-equivalence and
structural equivalence via name-equivalence. Fortunately, all of this
works out pleasantly and we may calculate in the natural way, free of
concern. The reader interested in the details is referred to the
appendix \ref{appendix:rho_details}.

\subsection{Substitution}

We use $\Proc$ for the set of processes, $\QProc$ for the set of
names, and $\id{\{}\vec{y} / \vec{x} \id{\}}$ to denote partial maps,
$s : \QProc \rightarrow \QProc$. A map, $s$ lifts, uniquely, to a map
on process terms, $\widehat{s} : \Proc \rightarrow \Proc$ by the
following equations.

\begin{mathpar}
  (0) \psubstp{Q}{P} := 0 \\
  (R \juxtap S) \psubstp{Q}{P}
  :=    
  (R)\psubstp{Q}{P} \juxtap (S) \psubstp{Q}{P} \\
  (x?(y).R) \psubstp{Q}{P}    
  :=    
  (x)\substp{Q}{P} (z)\concat( (R \psubstn{z}{y}) \psubstp{Q}{P} ) \\
  (\lift{x}{R}) \psubstp{Q}{P}  
  :=
  \lift{(x)\substp{Q}{P}}{ R \psubstp{Q}{P} } \\
%   (\dropn{x})  \psubstp{Q}{P}       
%   := 
%   \left\{ 
%     \begin{array}{ccc} 
%       \dropn{\quotep{Q}} & & x \nameeq \quotep{P} \\
%       \dropn{x} & & otherwise \\
%     \end{array}
%   \right. 
  (\dropn{x})  \psubstp{Q}{P}       
  := 
  \left\{ 
    \begin{array}{ccc} 
      Q & & x \nameeq \quotep{P} \\
      \dropn{x} & & otherwise \\
    \end{array}
  \right.
\end{mathpar}
 

where

\begin{eqnarray}
  (x)\id{\{} \lpquote Q \rpquote / \lpquote P \rpquote \id{\}}            = 
  \left\{ 
    \begin{array}{ccc}
      \lpquote Q \rpquote & & x \nameeq \lpquote P \rpquote \\
      x & & otherwise \\
    \end{array}
  \right. \nonumber
\end{eqnarray}

and $z$ is chosen distinct from $\quotep{P}$, $\quotep{Q}$, the free
names in $Q$, and all the names in $R$. Our $\alpha$-equivalence will
be built in the standard way from this substitution.

\begin{remark}\label{rem:no_self_referential_names}
  One consequence of these definitions is that $\forall P. \quotep{P}
  \not\in \freenames{P}$.
\end{remark}

\subsection{ Dynamic quote: an example }

Anticipating something of what's to come, consider applying the
substitution, $\widehat{\id{\{}u / z \id{\}}}$, to the following pair
of processes, $\lift{w}{y!(z)}$ and $w[ \lpquote y!(z) \rpquote ]$.

\begin{eqnarray}
	\lift{w}{y!(z)}\widehat{\id{\{}u / z \id{\}}}
		& = &
		\lift{w}{y!(u)} \nonumber\\
	w[ \lpquote y!(z) \rpquote ] \widehat{ \id{\{}u / z \id{\}} }
		& = &
		w[ \lpquote y!(z) \rpquote ] \nonumber
\end{eqnarray}

Because the body of the process between quotes is impervious to
substitution, we get radically different answers. In fact, by
examining the first process in an input context,
e.g. $x?(z).\lift{w}{y!(z)}$, we see that the process under the lift
operator may be shaped by prefixed inputs binding a name inside it. In
this sense, the lift operator will be seen as a way to dynamically
construct processes before reifying them as names.

Finally equipped with these standard features we can present the
dynamics of the calculus.

\subsubsection{Operational semantics} 

Finally, we introduce the computational dynamics. What marks these
algebras as distinct from other more traditionally studied algebraic
structures, e.g. vector spaces or polynomial rings, is the manner in
which dynamics is captured. In traditional structures, dynamics is typically
expressed through morphisms between such structures, as in linear maps
between vector spaces or morphisms between rings. In algebras
associated with the semantics of computation, the dynamics is
expressed as part of the algebraic structure itself, through a
reduction reduction relation typically denoted by $\red$. Below, we
give a recursive presentation of this relation for the calculus used
in the encoding.

$\red \subseteq \pi \times \pi$
$\red : \pi \to \mathcal{P}(\pi)$

\begin{mathpar}
  \inferrule* [lab=Comm] { \textsf{match}( x_{src}, x_{trgt} ) } { x_{trgt}?(y)P \; | \; x_{src}!\langle {Q} \rangle \red P\{\quotep{Q}/y}\} }
  \and \\
  \inferrule* [lab=Par] {{P} \red {P}'} {{{P} | {Q}} \red {{P}' | {Q}}}
  \and
  \inferrule* [lab=Equiv]{{{P} \scong {P}'} \andalso {{P}' \red {Q}'} \andalso {{Q}' \scong {Q}}}{{P} \red {Q}}
\end{mathpar}

\begin{eqnarray*}
  match_{\equiv} (\quotep{P},\quotep{Q}) & := & P \equiv Q \\
  match_{\dagger}(\quotep{P},\quotep{Q}) & := & \forall R. P|Q \red^{*} R => R \red^{*} 0 \\
  match_{K}(\quotep{P},\quotep{Q}) & := & K \mbox{ for some context } K
\end{eqnarray*}

$u?(x)P | u!\langle Q \rangle \red P\{\quotep{Q}/x\}$

%We write $\wred$ for $\red^*$, and $P\red$ if $\exists Q $ such that $ P \red Q$.
We write $P\red$ if $\exists Q $ such that $ P \red Q$ and $P\not\red$, otherwise.

\section{Replication}

As mentioned before, it is known that replication (and hence
recursion) can be implemented in a higher-order process algebra
\cite{SangiorgiWalker}. As our first example of calculation with the
machinery thus far presented we give the construction explicitly in
the {\rhoc}.

\begin{eqnarray}
	D_{x} & := & \prefix{x}{y}{(\binpar{\outputp{x}{y}}{@{y}})} \nonumber\\
	\bangp_{x}{P} & := & \binpar{{x}!\langle{\binpar{D_{x}}{P}}\rangle}{D_{x}} \nonumber
\end{eqnarray}

\begin{eqnarray}
	\bangp_{x}{P} & & \nonumber\\
	=
	& {x}!\langle{(\prefix{x}{y}{(\outputp{x}{y} | @{y})) | P}}\rangle 
	      | \prefix{x}{y}{(\outputp{x}{y} | @{y})} & \nonumber\\
	\red
	& (\outputp{x}{y} | @{y})\substn{\quotep{(\prefix{x}{y}{(@{y} | \outputp{x}{y})) | P}}}{y} & \nonumber\\
	=
	& \outputp{x}{\quotep{(\prefix{x}{y}{(\outputp{x}{y} | @{y})) | P}}}
	  | {(\prefix{x}{y}{(\outputp{x}{y} | @{y})) | P}} & \nonumber\\
	\red
	& \ldots & \nonumber\\
	\red^*
	& P | P | \ldots & \nonumber
\end{eqnarray}

Of course, this encoding, as an implementation, runs away, unfolding
$\bangp{P}$ eagerly. A lazier and more implementable replication
operator, restricted to input-guarded processes, may be obtained as follows.

\begin{eqnarray}
\bangp{\prefix{u}{v}{P}} 
	:= 
	\binpar{\lift{x}{\prefix{u}{v}{(\binpar{D(x)}{P})}}}{D(x)} \nonumber
\end{eqnarray}

\begin{remark}
  Note that the lazier definition still does not deal with summation
  or mixed summation (i.e. sums over input and output). The reader is
  invited to construct definitions of replication that deal with these
  features. 

  Further, the definitions are parameterized in a name, $x$. Can you,
  gentle reader, make a definition that eliminates this parameter and
  guarantees no accidental interaction between the replication
  machinery and the process being replicated -- i.e. no accidental
  sharing of names used by the process to get its work done and the
  name(s) used by the replication to effect copying. This latter
  revision of the definition of replication is crucial to obtaining
  the expected identity $!!P \sim !P$.
\end{remark}

\begin{remark}\label{rem:paradoxical_combinator}
  The reader familiar with the lambda calculus will have noticed the
  similarity between $D$ and the paradoxical combinator.

  [Ed. note: the existence of this seems to suggest we have to be more
  restrictive on the set of processes and names we admit if we are to
  support no-cloning.]
\end{remark}

\subsubsection{Bisimulation}

The computational dynamics gives rise to another kind of equivalence,
the equivalence of computational behavior. As previously mentioned
this is typically captured \emph{via} some form of bisimulation.

% The notion we use in this paper is weak barbed bisimulation
% \cite{milner91polyadicpi}.

The notion we use in this paper is derived from weak barbed
bisimulation \cite{milner91polyadicpi}. 

\begin{definition}
An \emph{observation relation}, $\downarrow_{\mathcal N}$, over a set
of names, $\mathcal N$, is the smallest relation satisfying the rules
below.

\infrule[Out-barb]{y \in {\mathcal N}, \; x \nameeq y}
		  {\outputp{x}{v} \downarrow_{\mathcal N} x}
\infrule[Par-barb]{\mbox{$P\downarrow_{\mathcal N} x$ or $Q\downarrow_{\mathcal N} x$}}
		  {\binpar{P}{Q} \downarrow_{\mathcal N} x}

We write $P \Downarrow_{\mathcal N} x$ if there is $Q$ such that 
$P \wred Q$ and $Q \downarrow_{\mathcal N} x$.
\end{definition}

\begin{definition}
%\label{def.bbisim}
An  ${\mathcal N}$-\emph{barbed bisimulation} over a set of names, ${\mathcal N}$, is a symmetric binary relation 
${\mathcal S}_{\mathcal N}$ between agents such that $P\rel{S}_{\mathcal N}Q$ implies:
\begin{enumerate}
\item If $P \red P'$ then $Q \wred Q'$ and $P'\rel{S}_{\mathcal N} Q'$.
\item If $P\downarrow_{\mathcal N} x$, then $Q\Downarrow_{\mathcal N} x$.
\end{enumerate}
$P$ is ${\mathcal N}$-barbed bisimilar to $Q$, written
$P \wbbisim_{\mathcal N} Q$, if $P \rel{S}_{\mathcal N} Q$ for some ${\mathcal N}$-barbed bisimulation ${\mathcal S}_{\mathcal N}$.
\end{definition}

$\mathcal{R} \subseteq \pi \times \pi$

$P \mathcal{R} Q => \forall P'. P \red P' \Rightarrow \exists Q'. Q \red Q', P' \mathcal{R} Q'$

$P \vdash x \Rightarrow Q \vdash x$

\begin{mathpar}
  \inferrule*[lab=Out-barb]{x \nameeq y}{{y}!\langle{Q}\rangle \vdash x}
  \and
  \inferrule*[lab=Par-barb]{\mbox{$P\vdash x$ or $Q\vdash x$}}{\binpar{P}{Q} \vdash x}
\end{mathpar}

\subsubsection{Contexts}

One of the principle advantages of computational calculi like the
$\pi$-calculus is a well-defined notion of context,
contextual-equivalence and a correlation between
contextual-equivalence and notions of bisimulation. The notion of
context allows the decomposition of a process into (sub-)process and
its syntactic environment, its context. Thus, a context may be
thought of as a process with a ``hole'' (written $\Box$) in it. The
application of a context $M$ to a process $P$, written $M[P]$, is
tantamount to filling the hole in $M$ with $P$. In this paper we do
not need the full weight of this theory, but do make use of the notion
of context in the proof the main theorem. 

\begin{mathpar}
  \inferrule* [lab=summation] {} {{M_{M},M_{N}} \bc \Box \;|\; x.M_{A} \;|\; M_{M}+M_{N}}
  \and
  \inferrule* [lab=agent] {} {{M_{A}} \bc (\vec{x})M_{P} \;| \; \clift{P_0,\ldots,M_{P},\ldots,P_N}}
  \and \\
  \inferrule* [lab=process] {} {{M_{P}} \bc M_{N} \;| \;P|M_{P} }
\end{mathpar} 

\begin{mathpar}
  \inferrule* [lab=sychronization] {} {M_{N} \bc \Box \;|\; x?M_{F} \;|\; x!M_{C}}
  \and
  \inferrule* [lab=abstraction] {} {{M_{F}} \bc (x)M_{P} }
  \and
  \inferrule* [lab=concretion] {} {{M_{C}} \bc \langle M_{P} \rangle }
  \and \\
  \inferrule* [lab=process] {} {{M_{P}} \bc M_{N} \;| \;P|M_{P} }
\end{mathpar}

\begin{definition}[contextual application] Given a context $M$, and
  process $P$, we define the \emph{contextual application}, $M[P] :=
  M\{P/\Box\}$. That is, the contextual application of M to P is the
  substitution of $P$ for $\Box$ in $M$.
\end{definition}

$\meaningof{-} : L \to \mathcal{P}(\pi)$

\begin{mathpar}
  \inferrule* [lab=collection] {} {\meaningof{true} = \pi, \and \meaningof{~E} = \pi \setminus \meaningof{E}, \and \meaningof{E_{1} \& E_{2}} = \meaningof{E_{1}} \cap \meaningof{E_{2}}}
\end{mathpar}

\begin{mathpar}
  \inferrule* [lab=structure] {} {\meaningof{0} = \{ P \in \pi | P \equiv 0 \}, \and \\ \meaningof{E_1 | E_2} = \{ P \in \pi | P \equiv P_{1} | P_{2}, P_{1} \in \meaningof{E_{1}}, P_{2} \in \meaningof{E_2}\} }
\end{mathpar}

\begin{mathpar}
 \inferrule* [lab=behavior] {} {\meaningof{\langle a?b \rangle E} = \{ P \in \pi | P \equiv Q | u?(y)P', \\ \and \\\\ \and \\ \;\;\; u \in \meaningof{a}, \forall z.P'\{z/y\} \in \meaningof{E\{z/b\}}\}, \and \\ \meaningof{a!E} = \{ P \in \pi | P \equiv Q | x!\langle P' \rangle, x \in \meaningof{a} P' \in \meaningof{E}\} }
\end{mathpar}

\begin{mathpar}
 \inferrule* [lab=nominal] {} {\meaningof{\quotep{E}} = \{ \quotep{P} \in \quotep{\pi} | P \in \meaningof{E} \}, \and \meaningof{\quotep{P}} = \{ \quotep{Q} \in \quotep{\pi} | P \equiv Q \} \and \\ \meaningof{@\quotep{E}} = \{ P \in \pi | P \equiv @x, x \in \meaningof{E} \}}
\end{mathpar}

\begin{eqnarray*}
  \\
  \meaningof{-} : TS \to ST
\end{eqnarray*}

\begin{eqnarray*}
  \\
  L : TS \to ST
\end{eqnarray*}

\begin{eqnarray*}
  \\
  P \models E \iff P \in \meaningof{E}
\end{eqnarray*}

\begin{eqnarray*}
  P \approx_{L} Q \iff \forall E \in L. P \models E \iff Q \models E
\end{eqnarray*}

\begin{eqnarray*}
  P \approx_{K} Q
\end{eqnarray*}

\begin{eqnarray*}
  P \approx Q
\end{eqnarray*}

$\approx_{K} = \approx = \approx_{L}$

\subsubsection{Contextual duality}

Note that contexts extend the quotation operation to a family of
operations from processes to names. Given a context, $M$, we can
define a \emph{nominal context}, $\quotep{M}$ by $\quotep{M}[P] :=
\quotep{M[P]}$. To foreshadow what is to come we observe that these
operations enjoy a duality with processes very much like the duality
between vectors and maps from vectors to scalars.

Further, because the calculus is essentially higher-order, we have a
correspondence between contexts and processes. More specifically,
given a name $x$ and a context $M$ we can construct $M^{*}_{x}$ such
that 

\begin{mathpar}
  M^{*}_{x} | \lift{x}{P} \red M[P]
\end{mathpar}

namely,

\begin{mathpar}
  M^{*}_{x} := x?(u).M[\dropn{u}]
\end{mathpar}

The dependence of $M^{*}_{x}$ on a name makes it an abstraction, 

\begin{mathpar}
  M^{*} := (x)x?(u).M[\dropn{u}]
\end{mathpar}

\subsection{Additional notation}

It will sometimes be convenient to denote the process a name
quotes. We already have the notation $x = \quotep{P}$, but it will be
convenient to introduce an alternate notation, $\procn{x}$, when we
want to emphasize the connection to the use of the name. Note that, by
virtue of name equivalence, $\quotep{\procn{x}} \nameeq x$; so, the
notation is consistent with previous definitions.

Further, because names have structure it is possible to effect
substitutions on the basis of that structure. This means we need to
upgrade our notation for substitutions, which we accomplish by
adapting comprehension notation. Thus,

\begin{mathpar}
  P\{ y / x : x \in S \}
\end{mathpar}

is interpreted to mean the process derived from P by replacing (in a
capture-avoiding manner) each occurrence of $x$ in $S$ by $y$. For example,

\begin{mathpar}
  P\{ \quotep{\procn{x}|\procn{x}} / x : x \in \freenames{P} \}
\end{mathpar}

will replace each (occurrence) of a free name $x$ in $P$ by
$\quotep{\procn{x}|\procn{x}}$.

Also, we will avail ourselves of the notation $x^{L}$ and $x^{R}$ to
denote injections of a name into disjoint copies of the name
space. There are numerous ways to accomplish this. One example can be
found in \cite{MeredithR05}. This notation overloads to vectors of
names: $\vec{x}^{\pi} := (x_{i}^{\pi} \; : \; 0 \leq i < |\vec{x}| )$ where $\pi \in \{L,R\}$.

We also use $P^{\Box} := P|\Box$.

In \cite{MeredithR05} an interpretation of the new operator is
given. It turns out that there are several possible interpretations
all enjoying the requisite algebraic properties of the operator (see
\cite{milner91polyadicpi}). We will therefore make liberal use of
$(\nu\; \vec{x})P$.

% subsection the_syntax_and_semantics_of_the_notation_system (end)   

\input{qm2pi.qmops} 

\input{qm2pi.sterngerlach} 

\input{qm2pi.metric} 

% section concurrent_process_calculi (end)

%\input{qm2pi.proofsketch}

% section proof sketch (end)

%\input{qm2pi.slviaknots} 

% section spatial logic via knots (end)

\input{qm2pi.conclusion}

% section conclusion (end)

%\input{qm2pi.dtcodes} 

% section wiring algorithm (end)

\input{qm2pi.ack} 

% section acknowledgments (end)

\newpage


\bibliographystyle{plain}   
\bibliography{../../biblios/main.bib}

\input{qm2pi.rhodetails}

\end{document}



% section front matter (end)

\section{Introduction}\label{sec:introduction} % (fold)
In this draft of the material i am going to have to dispense with the
usual writing conventions adopted in papers on these topics. i'm going
to have adopt whatever tone i need at the time i'm writing up the
calculations. Sometimes this may be very conversational; others it may
be the barest mathematical grunts; others still it may be that i have
lifted text from one of my other papers because the exposition of some
point was better said there. i hope that my readers are not unduly put
out by this decision. i'm not doing this to flout convention or be
rebellious. i find these calculations very technically challenging. To
keep everything going technically, something has to give; i have to
let go of some cognitive burden. So, the academic writing style --
with all of its trade-offs in terms of facilitating technical
communication -- is what i'm letting go of. Perhaps subsequent drafts
can be tightened and polished, but for now, i'm going to speak as if
we were sitting together in a coffee shop with a laptop, wifi and a
pad of paper and a pencil.

So, here's what i have to say. We -- you and i, comfortably ensconced
in our coffee shop and well-equipped with our tools -- can realize and
carry out the calculations of quantum mechanics over a very different
formal theory of dynamics, a formal theory of dynamics that
corresponds to a theory of concurrent computation with
\emph{reflection}. It has the advantage that the underlying theory is
already `quantized', but supports analogues all of the continuuous
operations. Strikingly, this underlying theory has recently been
connected with a notion of metric that we can show, by calculating
together, coincides with the metric induced by the inner product.

There are a lot of reasons why you might be interested in seeing
calculations of this form. Here's why i'm interested. For the past
several centuries there has been no competitor to the ``Newtonian''
account of dynamics. As a result the predominant share of accounts of
dynamical systems and situations have had to be formulated in terms of
the Newtonian machinery. i view this as an intellectually dangerous
position to occupy. Everything, despite it's intrinsic shape, turns
into a nail to be hit with this hammer. Recently, however, the theory
of computation has matured to the point where we have candidates for
theories of dynamics that offer very different perspective on
reasoning about dynamical systems and situations. Testing these
candidates against very successful accounts of dynamical situations,
like quantum mechanics, is going to give us some sense of how mature
they are and some measure of the quality of these accounts of
dynamics.

\subsection{Summary of contributions and outline of paper}

So, we're going to develop an interpretation of the operations of
quantum mechanics normally interpreted by Hilbert spaces and
operators. We're going to do this over a theory of computation. Note
that this is very different than the usual quantum computation program
which develops notions of computation over quantum mechanics. Rather,
we are developing a story that aligns with Wheeler's slogan: It from
Bit. To do this we will first provide an account of the theory of
computation at play here. Then we will dive into a calculation-driven
interpretation of the operations of quantum mechanics.

The reason we take this approach is that -- until very recently --
there hasn't been an axiomatic account of quantum mechanics. As a
result there has been no sharp delineation of the mathematical theory
supporting interpretation of the physical theory and the physical
theory, itself. So, ambient features of the maths are free to be
exploited (or supressed) without a real accounting of their physical
relevance. There is no sharp statement ``here's the physical theory''
qua \emph{theory} and ``here's the mathematical interpretation''
enabling a judgment of how faithful the interpretation is -- apart
from experimental observation. When there is an axiomatic account we
can judge how well a given mathematical formalism supports an
interpretation of the axioms, independent of
experimentation. Likewise, we can judge how well we have captured our
physical evidence and experience with our axiomatics, independent of
any specific mathematical implementation, with accidental detail that
may or may not have physical significance. 

In lieu of a fully fleshed out and vetted axiomatic account of quantum
mechanics, interpreting the operational notions in service of modeling
physical systems will have to suffice. In other words, we are not in
the business of providing a model of Hilbert spaces and operators. We
are in the business of providing a model of quantum mechanics because
we are motivated by testing our notions of dynamics against physical
theory; and, the predictive calculations of the physical theory must
serve as the best formulation -- shy of a fully fleshed out axiomatic
account -- of the physical theory itself (as they have for scientific
theories since time immemorial). Put another way, despite a
whole-hearted commitment to an It-from-Bit ontology, we are firmly
aligned with the shut-up-and-calculate camp as the best way to obtain
results either from the physical perspective or as a quality assurance
measure of our fledgling theory of dynamics.

In detail, we present a reflective process calculus. Then we develop
intuitive correspondences between the notions available in this
calculus and the usual physical notions supporting quantum mechanical
calculations. Thus, 

\begin{table}[htp]
  \center{
    \fbox{
      \begin{tabular}{c|c}
        quantum mechanics & process calculus \\
        \hline
        scalar & name \\
        state vector & process \\
        dual & contextual duals \\
        matrix & formal sums of process-context-dual pairs \\
        orthogonality & process annihilation \\
        inner product & execution-formula + quoting
      \end{tabular}
    }
  }
  \caption{QM - process calculi correspondences}
\end{table}

Then we tighten up these intuitions to operational definitions. We
employ the Dirac notation as the best proxy we can find for an
abstract syntax of the quantum mechanical notions. The definitions we
develop put us in contact with equational constraints coming from the
theory that we demonstrate the definitions and calculations satisfy.

This puts us in a position to shut up and calculate for the
Stern-Gerlach experimental set up, showing how these predictive
calculations become calculations on processes in our theory of a
reflective process calculus.

Penultimately, we demonstrate that the notion of metric coming from
the inner product coincides with the notion of metric available from
the theory of bisimulation. This demonstration gives us the right to
think of space as arising from behavior. Finally, we consider where we
might go from the new vantage point we have obtained.

% section introduction (end) 
 
% section introduction (end)

% \documentclass[12pt]{llncs}
%\documentclass{jktr}

\usepackage[pdftex]{hyperref}                   
\usepackage {listings}
\usepackage {mathpartir}
\usepackage{bcprules}
%\usepackage{listings}
                       
\usepackage{graphicx} 
%\usepackage[margins=2.5cm,nohead,nofoot]{geometry}
%\usepackage{geometry}
\usepackage{amsfonts}
\usepackage{amstext}
\usepackage{latexsym}
\usepackage{amssymb}
\usepackage{color}


%\include{myPreamble}
\include{qm2pi.local} 

%\ifpdf
%\usepackage[pdftex]{graphicx}
%\else
%\usepackage{graphicx}
%\fi

 % \ifpdf
%  \usepackage{pdfsync}
%  \if


%\title{Brief Article}
%\author{David F. Snyder}
%\author{L.G. Meredith}

%\address{Dept. of Math., Texas State University--San Marcos, San Marcos, TX 78666}
       
\pagestyle{empty}


\begin{document}

\lstset{language=[Objective]Caml,frame=shadowbox}

\input{qm2pi.front}

% section front matter (end)

\input{qm2pi.intro} 
 
% section introduction (end)

% \input{qm2pi.knotations} 

% section notation (end)

\input{qm2pi.process.calculi} 

% section concurrent_process_calculi_and_spatial_logics_ (end)
    
%\input{qm2pi.knots2pi} 

%\input{qm2pi.trefoil} 

%\input{qm2pi.mainthm} 

% subsection basic_interpretation (end)

%\input{qm2pi.rho.presentation} 
\subsection{The syntax and semantics of the notation system}\label{sub:the_syntax_and_semantics_of_the_notation_system} % (fold)

We now summarize a technical presentation of the calculus that
embodies our theory of dynamics. The typical presentation of such a
calculus follows the style of giving generators and relations on
them. The grammar, below, describing term constructors, freely
generates the set of processes, $\Proc$. This set is then quotiented
by a relation known as structural congruence and it is over this set
that the notion of dynamics is expressed. This presentation is
essentially that of \cite{MeredithR05} with the addition of
polyadicity and summation. For readability we have relegated some of
the technical subtleties to an appendix.

\subsubsection{Process grammar}\label{subsub:process_grammar}

\begin{mathpar}
  \inferrule* [lab=synchronization] {} {{M} \bc \pzero \;|\; x?F \;|\; x!C }
  \and
  \inferrule* [lab=abstraction] {} {{F} \bc (x)P}
  \and
  \inferrule* [lab=concretion] {} {{C} \bc \langle Q \rangle}
  \and
  \inferrule* [lab=process] {} {{P,Q} \bc M \;| \;P|Q \;|\; @{x}}
  \and
  \inferrule* [lab=name] {} {{x} \bc \quotep{P}}
\end{mathpar} 

Note that $\vec{x}$ (resp. $\vec{P}$) denotes a vector of names
(resp. processes) of length $|\vec{x}|$ (resp. $|\vec{P}|$). We adopt
the following useful abbreviations.

\begin{mathpar}
   x?(\vec{y}).P := x.(\vec{y})P \and  x\clift{\vec{P}} := x.\clift{\vec{P}}
   \and x!(y) := \lift{x}{\dropn{y}}
   \and \Pi_{i=0}^{n-1}P_i := P_0 | \ldots | P_{n-1}
\end{mathpar}

\subsubsection{Structural congruence}

\paragraph{Free and bound names and alpha-equivalence.} At the
core of structural equivalence is alpha-equivalence which identifies
process that are the same up to a change of variable. Formally, we
recognize the distinction between free and bound names. The free names
of a process, $\freenames{P}$, may be calculated recursively as
follows:

\begin{mathpar}
\freenames{\pzero} := \emptyset
  \and \\
  \freenames{x?(y).P} := \{ x \} \cup (\freenames{P} \setminus \{ y \})
  \and 
  \freenames{x!\langle P \rangle} := \{ x \} \cup \{ P \} 
  \and \\
  \freenames{P|Q} := \freenames{P} \cup \freenames{Q}
  \and \\
  \freenames{@{x}} := \{ x \}
\end{mathpar}

$\pi$
$\quotep{\pi}$

$\freenames{-} : \pi \to \mathcal{P}(\quotep{\pi})$

\begin{eqnarray*}
  \freenames{\pzero} & := & \emptyset \\
  \freenames{x?(y).P} & := & \{ x \} \cup (\freenames{P} \setminus \{ y \}) \\
  \freenames{x!\langle P \rangle} & := & \{ x \} \cup \{ P \} \\
  \freenames{P|Q} & := & \freenames{P} \cup \freenames{Q} \\
  \freenames{\dropn{x}} & := & \{ x \}
\end{eqnarray*}

The bound names of a process, $\boundnames{P}$, are those names occurring in $P$
that are not free. For example, in $x?(y).0$, the name $x$ is free, while $y$ is bound.

\begin{mathpar}
  \inferrule* [lab=monoidal-laws] {} { P|Q \equiv Q|P \and P|0 \equiv P \and P|(Q|R) \equiv (P|Q)|R }
\end{mathpar}

\begin{mathpar}
  \inferrule* [lab=alpha-equivalence] {} { (x)P \equiv (y)P\{y/x\} \and y \not\in \freenames{P} }
\end{mathpar}

\begin{definition}
Then two processes, $P,Q$, are alpha-equivalent if $P = Q\{\vec{y}/\vec{x}\}$ for
some $\vec{x} \in \boundnames{Q},\vec{y} \in \boundnames{P}$, where $Q\{\vec{y}/\vec{x}\}$
denotes the capture-avoiding substitution of $\vec{y}$ for $\vec{x}$ in $Q$.
\end{definition}

\begin{definition}
  The {\em structural congruence} \cite{SangiorgiWalker} , $\equiv$,
  between processes is the least congruence containing
  alpha-equivalence, satisfying the abelian monoid laws
  (associativity, commutativity and $\pzero$ as identity) for parallel
  composition $|$ and for summation $+$.
\end{definition}

\subsection{Name equivalence}

We take name equivalence, written $\nameeq$, to be the smallest
equivalence relation generated by the following rules.

\begin{mathpar}
\inferrule*[lab=Quote-drop]
{ }
{ \quotep{@{x}} \nameeq x }

\inferrule*[lab=Struct-equiv]
{ P \scong Q }
{ \quotep{P} \nameeq \quotep{Q} }
\end{mathpar}

The astute reader will have noticed that the mutual recursion of names
and processes imposes a mutual recursion on alpha-equivalence and
structural equivalence via name-equivalence. Fortunately, all of this
works out pleasantly and we may calculate in the natural way, free of
concern. The reader interested in the details is referred to the
appendix \ref{appendix:rho_details}.

\subsection{Substitution}

We use $\Proc$ for the set of processes, $\QProc$ for the set of
names, and $\id{\{}\vec{y} / \vec{x} \id{\}}$ to denote partial maps,
$s : \QProc \rightarrow \QProc$. A map, $s$ lifts, uniquely, to a map
on process terms, $\widehat{s} : \Proc \rightarrow \Proc$ by the
following equations.

\begin{mathpar}
  (0) \psubstp{Q}{P} := 0 \\
  (R \juxtap S) \psubstp{Q}{P}
  :=    
  (R)\psubstp{Q}{P} \juxtap (S) \psubstp{Q}{P} \\
  (x?(y).R) \psubstp{Q}{P}    
  :=    
  (x)\substp{Q}{P} (z)\concat( (R \psubstn{z}{y}) \psubstp{Q}{P} ) \\
  (\lift{x}{R}) \psubstp{Q}{P}  
  :=
  \lift{(x)\substp{Q}{P}}{ R \psubstp{Q}{P} } \\
%   (\dropn{x})  \psubstp{Q}{P}       
%   := 
%   \left\{ 
%     \begin{array}{ccc} 
%       \dropn{\quotep{Q}} & & x \nameeq \quotep{P} \\
%       \dropn{x} & & otherwise \\
%     \end{array}
%   \right. 
  (\dropn{x})  \psubstp{Q}{P}       
  := 
  \left\{ 
    \begin{array}{ccc} 
      Q & & x \nameeq \quotep{P} \\
      \dropn{x} & & otherwise \\
    \end{array}
  \right.
\end{mathpar}
 

where

\begin{eqnarray}
  (x)\id{\{} \lpquote Q \rpquote / \lpquote P \rpquote \id{\}}            = 
  \left\{ 
    \begin{array}{ccc}
      \lpquote Q \rpquote & & x \nameeq \lpquote P \rpquote \\
      x & & otherwise \\
    \end{array}
  \right. \nonumber
\end{eqnarray}

and $z$ is chosen distinct from $\quotep{P}$, $\quotep{Q}$, the free
names in $Q$, and all the names in $R$. Our $\alpha$-equivalence will
be built in the standard way from this substitution.

\begin{remark}\label{rem:no_self_referential_names}
  One consequence of these definitions is that $\forall P. \quotep{P}
  \not\in \freenames{P}$.
\end{remark}

\subsection{ Dynamic quote: an example }

Anticipating something of what's to come, consider applying the
substitution, $\widehat{\id{\{}u / z \id{\}}}$, to the following pair
of processes, $\lift{w}{y!(z)}$ and $w[ \lpquote y!(z) \rpquote ]$.

\begin{eqnarray}
	\lift{w}{y!(z)}\widehat{\id{\{}u / z \id{\}}}
		& = &
		\lift{w}{y!(u)} \nonumber\\
	w[ \lpquote y!(z) \rpquote ] \widehat{ \id{\{}u / z \id{\}} }
		& = &
		w[ \lpquote y!(z) \rpquote ] \nonumber
\end{eqnarray}

Because the body of the process between quotes is impervious to
substitution, we get radically different answers. In fact, by
examining the first process in an input context,
e.g. $x?(z).\lift{w}{y!(z)}$, we see that the process under the lift
operator may be shaped by prefixed inputs binding a name inside it. In
this sense, the lift operator will be seen as a way to dynamically
construct processes before reifying them as names.

Finally equipped with these standard features we can present the
dynamics of the calculus.

\subsubsection{Operational semantics} 

Finally, we introduce the computational dynamics. What marks these
algebras as distinct from other more traditionally studied algebraic
structures, e.g. vector spaces or polynomial rings, is the manner in
which dynamics is captured. In traditional structures, dynamics is typically
expressed through morphisms between such structures, as in linear maps
between vector spaces or morphisms between rings. In algebras
associated with the semantics of computation, the dynamics is
expressed as part of the algebraic structure itself, through a
reduction reduction relation typically denoted by $\red$. Below, we
give a recursive presentation of this relation for the calculus used
in the encoding.

$\red \subseteq \pi \times \pi$
$\red : \pi \to \mathcal{P}(\pi)$

\begin{mathpar}
  \inferrule* [lab=Comm] { \textsf{match}( x_{src}, x_{trgt} ) } { x_{trgt}?(y)P \; | \; x_{src}!\langle {Q} \rangle \red P\{\quotep{Q}/y}\} }
  \and \\
  \inferrule* [lab=Par] {{P} \red {P}'} {{{P} | {Q}} \red {{P}' | {Q}}}
  \and
  \inferrule* [lab=Equiv]{{{P} \scong {P}'} \andalso {{P}' \red {Q}'} \andalso {{Q}' \scong {Q}}}{{P} \red {Q}}
\end{mathpar}

\begin{eqnarray*}
  match_{\equiv} (\quotep{P},\quotep{Q}) & := & P \equiv Q \\
  match_{\dagger}(\quotep{P},\quotep{Q}) & := & \forall R. P|Q \red^{*} R => R \red^{*} 0 \\
  match_{K}(\quotep{P},\quotep{Q}) & := & K \mbox{ for some context } K
\end{eqnarray*}

$u?(x)P | u!\langle Q \rangle \red P\{\quotep{Q}/x\}$

%We write $\wred$ for $\red^*$, and $P\red$ if $\exists Q $ such that $ P \red Q$.
We write $P\red$ if $\exists Q $ such that $ P \red Q$ and $P\not\red$, otherwise.

\section{Replication}

As mentioned before, it is known that replication (and hence
recursion) can be implemented in a higher-order process algebra
\cite{SangiorgiWalker}. As our first example of calculation with the
machinery thus far presented we give the construction explicitly in
the {\rhoc}.

\begin{eqnarray}
	D_{x} & := & \prefix{x}{y}{(\binpar{\outputp{x}{y}}{@{y}})} \nonumber\\
	\bangp_{x}{P} & := & \binpar{{x}!\langle{\binpar{D_{x}}{P}}\rangle}{D_{x}} \nonumber
\end{eqnarray}

\begin{eqnarray}
	\bangp_{x}{P} & & \nonumber\\
	=
	& {x}!\langle{(\prefix{x}{y}{(\outputp{x}{y} | @{y})) | P}}\rangle 
	      | \prefix{x}{y}{(\outputp{x}{y} | @{y})} & \nonumber\\
	\red
	& (\outputp{x}{y} | @{y})\substn{\quotep{(\prefix{x}{y}{(@{y} | \outputp{x}{y})) | P}}}{y} & \nonumber\\
	=
	& \outputp{x}{\quotep{(\prefix{x}{y}{(\outputp{x}{y} | @{y})) | P}}}
	  | {(\prefix{x}{y}{(\outputp{x}{y} | @{y})) | P}} & \nonumber\\
	\red
	& \ldots & \nonumber\\
	\red^*
	& P | P | \ldots & \nonumber
\end{eqnarray}

Of course, this encoding, as an implementation, runs away, unfolding
$\bangp{P}$ eagerly. A lazier and more implementable replication
operator, restricted to input-guarded processes, may be obtained as follows.

\begin{eqnarray}
\bangp{\prefix{u}{v}{P}} 
	:= 
	\binpar{\lift{x}{\prefix{u}{v}{(\binpar{D(x)}{P})}}}{D(x)} \nonumber
\end{eqnarray}

\begin{remark}
  Note that the lazier definition still does not deal with summation
  or mixed summation (i.e. sums over input and output). The reader is
  invited to construct definitions of replication that deal with these
  features. 

  Further, the definitions are parameterized in a name, $x$. Can you,
  gentle reader, make a definition that eliminates this parameter and
  guarantees no accidental interaction between the replication
  machinery and the process being replicated -- i.e. no accidental
  sharing of names used by the process to get its work done and the
  name(s) used by the replication to effect copying. This latter
  revision of the definition of replication is crucial to obtaining
  the expected identity $!!P \sim !P$.
\end{remark}

\begin{remark}\label{rem:paradoxical_combinator}
  The reader familiar with the lambda calculus will have noticed the
  similarity between $D$ and the paradoxical combinator.

  [Ed. note: the existence of this seems to suggest we have to be more
  restrictive on the set of processes and names we admit if we are to
  support no-cloning.]
\end{remark}

\subsubsection{Bisimulation}

The computational dynamics gives rise to another kind of equivalence,
the equivalence of computational behavior. As previously mentioned
this is typically captured \emph{via} some form of bisimulation.

% The notion we use in this paper is weak barbed bisimulation
% \cite{milner91polyadicpi}.

The notion we use in this paper is derived from weak barbed
bisimulation \cite{milner91polyadicpi}. 

\begin{definition}
An \emph{observation relation}, $\downarrow_{\mathcal N}$, over a set
of names, $\mathcal N$, is the smallest relation satisfying the rules
below.

\infrule[Out-barb]{y \in {\mathcal N}, \; x \nameeq y}
		  {\outputp{x}{v} \downarrow_{\mathcal N} x}
\infrule[Par-barb]{\mbox{$P\downarrow_{\mathcal N} x$ or $Q\downarrow_{\mathcal N} x$}}
		  {\binpar{P}{Q} \downarrow_{\mathcal N} x}

We write $P \Downarrow_{\mathcal N} x$ if there is $Q$ such that 
$P \wred Q$ and $Q \downarrow_{\mathcal N} x$.
\end{definition}

\begin{definition}
%\label{def.bbisim}
An  ${\mathcal N}$-\emph{barbed bisimulation} over a set of names, ${\mathcal N}$, is a symmetric binary relation 
${\mathcal S}_{\mathcal N}$ between agents such that $P\rel{S}_{\mathcal N}Q$ implies:
\begin{enumerate}
\item If $P \red P'$ then $Q \wred Q'$ and $P'\rel{S}_{\mathcal N} Q'$.
\item If $P\downarrow_{\mathcal N} x$, then $Q\Downarrow_{\mathcal N} x$.
\end{enumerate}
$P$ is ${\mathcal N}$-barbed bisimilar to $Q$, written
$P \wbbisim_{\mathcal N} Q$, if $P \rel{S}_{\mathcal N} Q$ for some ${\mathcal N}$-barbed bisimulation ${\mathcal S}_{\mathcal N}$.
\end{definition}

$\mathcal{R} \subseteq \pi \times \pi$

$P \mathcal{R} Q => \forall P'. P \red P' \Rightarrow \exists Q'. Q \red Q', P' \mathcal{R} Q'$

$P \vdash x \Rightarrow Q \vdash x$

\begin{mathpar}
  \inferrule*[lab=Out-barb]{x \nameeq y}{{y}!\langle{Q}\rangle \vdash x}
  \and
  \inferrule*[lab=Par-barb]{\mbox{$P\vdash x$ or $Q\vdash x$}}{\binpar{P}{Q} \vdash x}
\end{mathpar}

\subsubsection{Contexts}

One of the principle advantages of computational calculi like the
$\pi$-calculus is a well-defined notion of context,
contextual-equivalence and a correlation between
contextual-equivalence and notions of bisimulation. The notion of
context allows the decomposition of a process into (sub-)process and
its syntactic environment, its context. Thus, a context may be
thought of as a process with a ``hole'' (written $\Box$) in it. The
application of a context $M$ to a process $P$, written $M[P]$, is
tantamount to filling the hole in $M$ with $P$. In this paper we do
not need the full weight of this theory, but do make use of the notion
of context in the proof the main theorem. 

\begin{mathpar}
  \inferrule* [lab=summation] {} {{M_{M},M_{N}} \bc \Box \;|\; x.M_{A} \;|\; M_{M}+M_{N}}
  \and
  \inferrule* [lab=agent] {} {{M_{A}} \bc (\vec{x})M_{P} \;| \; \clift{P_0,\ldots,M_{P},\ldots,P_N}}
  \and \\
  \inferrule* [lab=process] {} {{M_{P}} \bc M_{N} \;| \;P|M_{P} }
\end{mathpar} 

\begin{mathpar}
  \inferrule* [lab=sychronization] {} {M_{N} \bc \Box \;|\; x?M_{F} \;|\; x!M_{C}}
  \and
  \inferrule* [lab=abstraction] {} {{M_{F}} \bc (x)M_{P} }
  \and
  \inferrule* [lab=concretion] {} {{M_{C}} \bc \langle M_{P} \rangle }
  \and \\
  \inferrule* [lab=process] {} {{M_{P}} \bc M_{N} \;| \;P|M_{P} }
\end{mathpar}

\begin{definition}[contextual application] Given a context $M$, and
  process $P$, we define the \emph{contextual application}, $M[P] :=
  M\{P/\Box\}$. That is, the contextual application of M to P is the
  substitution of $P$ for $\Box$ in $M$.
\end{definition}

$\meaningof{-} : L \to \mathcal{P}(\pi)$

\begin{mathpar}
  \inferrule* [lab=collection] {} {\meaningof{true} = \pi, \and \meaningof{~E} = \pi \setminus \meaningof{E}, \and \meaningof{E_{1} \& E_{2}} = \meaningof{E_{1}} \cap \meaningof{E_{2}}}
\end{mathpar}

\begin{mathpar}
  \inferrule* [lab=structure] {} {\meaningof{0} = \{ P \in \pi | P \equiv 0 \}, \and \\ \meaningof{E_1 | E_2} = \{ P \in \pi | P \equiv P_{1} | P_{2}, P_{1} \in \meaningof{E_{1}}, P_{2} \in \meaningof{E_2}\} }
\end{mathpar}

\begin{mathpar}
 \inferrule* [lab=behavior] {} {\meaningof{\langle a?b \rangle E} = \{ P \in \pi | P \equiv Q | u?(y)P', \\ \and \\\\ \and \\ \;\;\; u \in \meaningof{a}, \forall z.P'\{z/y\} \in \meaningof{E\{z/b\}}\}, \and \\ \meaningof{a!E} = \{ P \in \pi | P \equiv Q | x!\langle P' \rangle, x \in \meaningof{a} P' \in \meaningof{E}\} }
\end{mathpar}

\begin{mathpar}
 \inferrule* [lab=nominal] {} {\meaningof{\quotep{E}} = \{ \quotep{P} \in \quotep{\pi} | P \in \meaningof{E} \}, \and \meaningof{\quotep{P}} = \{ \quotep{Q} \in \quotep{\pi} | P \equiv Q \} \and \\ \meaningof{@\quotep{E}} = \{ P \in \pi | P \equiv @x, x \in \meaningof{E} \}}
\end{mathpar}

\begin{eqnarray*}
  \\
  \meaningof{-} : TS \to ST
\end{eqnarray*}

\begin{eqnarray*}
  \\
  L : TS \to ST
\end{eqnarray*}

\begin{eqnarray*}
  \\
  P \models E \iff P \in \meaningof{E}
\end{eqnarray*}

\begin{eqnarray*}
  P \approx_{L} Q \iff \forall E \in L. P \models E \iff Q \models E
\end{eqnarray*}

\begin{eqnarray*}
  P \approx_{K} Q
\end{eqnarray*}

\begin{eqnarray*}
  P \approx Q
\end{eqnarray*}

$\approx_{K} = \approx = \approx_{L}$

\subsubsection{Contextual duality}

Note that contexts extend the quotation operation to a family of
operations from processes to names. Given a context, $M$, we can
define a \emph{nominal context}, $\quotep{M}$ by $\quotep{M}[P] :=
\quotep{M[P]}$. To foreshadow what is to come we observe that these
operations enjoy a duality with processes very much like the duality
between vectors and maps from vectors to scalars.

Further, because the calculus is essentially higher-order, we have a
correspondence between contexts and processes. More specifically,
given a name $x$ and a context $M$ we can construct $M^{*}_{x}$ such
that 

\begin{mathpar}
  M^{*}_{x} | \lift{x}{P} \red M[P]
\end{mathpar}

namely,

\begin{mathpar}
  M^{*}_{x} := x?(u).M[\dropn{u}]
\end{mathpar}

The dependence of $M^{*}_{x}$ on a name makes it an abstraction, 

\begin{mathpar}
  M^{*} := (x)x?(u).M[\dropn{u}]
\end{mathpar}

\subsection{Additional notation}

It will sometimes be convenient to denote the process a name
quotes. We already have the notation $x = \quotep{P}$, but it will be
convenient to introduce an alternate notation, $\procn{x}$, when we
want to emphasize the connection to the use of the name. Note that, by
virtue of name equivalence, $\quotep{\procn{x}} \nameeq x$; so, the
notation is consistent with previous definitions.

Further, because names have structure it is possible to effect
substitutions on the basis of that structure. This means we need to
upgrade our notation for substitutions, which we accomplish by
adapting comprehension notation. Thus,

\begin{mathpar}
  P\{ y / x : x \in S \}
\end{mathpar}

is interpreted to mean the process derived from P by replacing (in a
capture-avoiding manner) each occurrence of $x$ in $S$ by $y$. For example,

\begin{mathpar}
  P\{ \quotep{\procn{x}|\procn{x}} / x : x \in \freenames{P} \}
\end{mathpar}

will replace each (occurrence) of a free name $x$ in $P$ by
$\quotep{\procn{x}|\procn{x}}$.

Also, we will avail ourselves of the notation $x^{L}$ and $x^{R}$ to
denote injections of a name into disjoint copies of the name
space. There are numerous ways to accomplish this. One example can be
found in \cite{MeredithR05}. This notation overloads to vectors of
names: $\vec{x}^{\pi} := (x_{i}^{\pi} \; : \; 0 \leq i < |\vec{x}| )$ where $\pi \in \{L,R\}$.

We also use $P^{\Box} := P|\Box$.

In \cite{MeredithR05} an interpretation of the new operator is
given. It turns out that there are several possible interpretations
all enjoying the requisite algebraic properties of the operator (see
\cite{milner91polyadicpi}). We will therefore make liberal use of
$(\nu\; \vec{x})P$.

% subsection the_syntax_and_semantics_of_the_notation_system (end)   

\input{qm2pi.qmops} 

\input{qm2pi.sterngerlach} 

\input{qm2pi.metric} 

% section concurrent_process_calculi (end)

%\input{qm2pi.proofsketch}

% section proof sketch (end)

%\input{qm2pi.slviaknots} 

% section spatial logic via knots (end)

\input{qm2pi.conclusion}

% section conclusion (end)

%\input{qm2pi.dtcodes} 

% section wiring algorithm (end)

\input{qm2pi.ack} 

% section acknowledgments (end)

\newpage


\bibliographystyle{plain}   
\bibliography{../../biblios/main.bib}

\input{qm2pi.rhodetails}

\end{document}

 

% section notation (end)

\input{qm2pi.process.calculi} 

% section concurrent_process_calculi_and_spatial_logics_ (end)
    
%\documentclass[12pt]{llncs}
%\documentclass{jktr}

\usepackage[pdftex]{hyperref}                   
\usepackage {listings}
\usepackage {mathpartir}
\usepackage{bcprules}
%\usepackage{listings}
                       
\usepackage{graphicx} 
%\usepackage[margins=2.5cm,nohead,nofoot]{geometry}
%\usepackage{geometry}
\usepackage{amsfonts}
\usepackage{amstext}
\usepackage{latexsym}
\usepackage{amssymb}
\usepackage{color}


%\include{myPreamble}
\include{qm2pi.local} 

%\ifpdf
%\usepackage[pdftex]{graphicx}
%\else
%\usepackage{graphicx}
%\fi

 % \ifpdf
%  \usepackage{pdfsync}
%  \if


%\title{Brief Article}
%\author{David F. Snyder}
%\author{L.G. Meredith}

%\address{Dept. of Math., Texas State University--San Marcos, San Marcos, TX 78666}
       
\pagestyle{empty}


\begin{document}

\lstset{language=[Objective]Caml,frame=shadowbox}

\input{qm2pi.front}

% section front matter (end)

\input{qm2pi.intro} 
 
% section introduction (end)

% \input{qm2pi.knotations} 

% section notation (end)

\input{qm2pi.process.calculi} 

% section concurrent_process_calculi_and_spatial_logics_ (end)
    
%\input{qm2pi.knots2pi} 

%\input{qm2pi.trefoil} 

%\input{qm2pi.mainthm} 

% subsection basic_interpretation (end)

%\input{qm2pi.rho.presentation} 
\subsection{The syntax and semantics of the notation system}\label{sub:the_syntax_and_semantics_of_the_notation_system} % (fold)

We now summarize a technical presentation of the calculus that
embodies our theory of dynamics. The typical presentation of such a
calculus follows the style of giving generators and relations on
them. The grammar, below, describing term constructors, freely
generates the set of processes, $\Proc$. This set is then quotiented
by a relation known as structural congruence and it is over this set
that the notion of dynamics is expressed. This presentation is
essentially that of \cite{MeredithR05} with the addition of
polyadicity and summation. For readability we have relegated some of
the technical subtleties to an appendix.

\subsubsection{Process grammar}\label{subsub:process_grammar}

\begin{mathpar}
  \inferrule* [lab=synchronization] {} {{M} \bc \pzero \;|\; x?F \;|\; x!C }
  \and
  \inferrule* [lab=abstraction] {} {{F} \bc (x)P}
  \and
  \inferrule* [lab=concretion] {} {{C} \bc \langle Q \rangle}
  \and
  \inferrule* [lab=process] {} {{P,Q} \bc M \;| \;P|Q \;|\; @{x}}
  \and
  \inferrule* [lab=name] {} {{x} \bc \quotep{P}}
\end{mathpar} 

Note that $\vec{x}$ (resp. $\vec{P}$) denotes a vector of names
(resp. processes) of length $|\vec{x}|$ (resp. $|\vec{P}|$). We adopt
the following useful abbreviations.

\begin{mathpar}
   x?(\vec{y}).P := x.(\vec{y})P \and  x\clift{\vec{P}} := x.\clift{\vec{P}}
   \and x!(y) := \lift{x}{\dropn{y}}
   \and \Pi_{i=0}^{n-1}P_i := P_0 | \ldots | P_{n-1}
\end{mathpar}

\subsubsection{Structural congruence}

\paragraph{Free and bound names and alpha-equivalence.} At the
core of structural equivalence is alpha-equivalence which identifies
process that are the same up to a change of variable. Formally, we
recognize the distinction between free and bound names. The free names
of a process, $\freenames{P}$, may be calculated recursively as
follows:

\begin{mathpar}
\freenames{\pzero} := \emptyset
  \and \\
  \freenames{x?(y).P} := \{ x \} \cup (\freenames{P} \setminus \{ y \})
  \and 
  \freenames{x!\langle P \rangle} := \{ x \} \cup \{ P \} 
  \and \\
  \freenames{P|Q} := \freenames{P} \cup \freenames{Q}
  \and \\
  \freenames{@{x}} := \{ x \}
\end{mathpar}

$\pi$
$\quotep{\pi}$

$\freenames{-} : \pi \to \mathcal{P}(\quotep{\pi})$

\begin{eqnarray*}
  \freenames{\pzero} & := & \emptyset \\
  \freenames{x?(y).P} & := & \{ x \} \cup (\freenames{P} \setminus \{ y \}) \\
  \freenames{x!\langle P \rangle} & := & \{ x \} \cup \{ P \} \\
  \freenames{P|Q} & := & \freenames{P} \cup \freenames{Q} \\
  \freenames{\dropn{x}} & := & \{ x \}
\end{eqnarray*}

The bound names of a process, $\boundnames{P}$, are those names occurring in $P$
that are not free. For example, in $x?(y).0$, the name $x$ is free, while $y$ is bound.

\begin{mathpar}
  \inferrule* [lab=monoidal-laws] {} { P|Q \equiv Q|P \and P|0 \equiv P \and P|(Q|R) \equiv (P|Q)|R }
\end{mathpar}

\begin{mathpar}
  \inferrule* [lab=alpha-equivalence] {} { (x)P \equiv (y)P\{y/x\} \and y \not\in \freenames{P} }
\end{mathpar}

\begin{definition}
Then two processes, $P,Q$, are alpha-equivalent if $P = Q\{\vec{y}/\vec{x}\}$ for
some $\vec{x} \in \boundnames{Q},\vec{y} \in \boundnames{P}$, where $Q\{\vec{y}/\vec{x}\}$
denotes the capture-avoiding substitution of $\vec{y}$ for $\vec{x}$ in $Q$.
\end{definition}

\begin{definition}
  The {\em structural congruence} \cite{SangiorgiWalker} , $\equiv$,
  between processes is the least congruence containing
  alpha-equivalence, satisfying the abelian monoid laws
  (associativity, commutativity and $\pzero$ as identity) for parallel
  composition $|$ and for summation $+$.
\end{definition}

\subsection{Name equivalence}

We take name equivalence, written $\nameeq$, to be the smallest
equivalence relation generated by the following rules.

\begin{mathpar}
\inferrule*[lab=Quote-drop]
{ }
{ \quotep{@{x}} \nameeq x }

\inferrule*[lab=Struct-equiv]
{ P \scong Q }
{ \quotep{P} \nameeq \quotep{Q} }
\end{mathpar}

The astute reader will have noticed that the mutual recursion of names
and processes imposes a mutual recursion on alpha-equivalence and
structural equivalence via name-equivalence. Fortunately, all of this
works out pleasantly and we may calculate in the natural way, free of
concern. The reader interested in the details is referred to the
appendix \ref{appendix:rho_details}.

\subsection{Substitution}

We use $\Proc$ for the set of processes, $\QProc$ for the set of
names, and $\id{\{}\vec{y} / \vec{x} \id{\}}$ to denote partial maps,
$s : \QProc \rightarrow \QProc$. A map, $s$ lifts, uniquely, to a map
on process terms, $\widehat{s} : \Proc \rightarrow \Proc$ by the
following equations.

\begin{mathpar}
  (0) \psubstp{Q}{P} := 0 \\
  (R \juxtap S) \psubstp{Q}{P}
  :=    
  (R)\psubstp{Q}{P} \juxtap (S) \psubstp{Q}{P} \\
  (x?(y).R) \psubstp{Q}{P}    
  :=    
  (x)\substp{Q}{P} (z)\concat( (R \psubstn{z}{y}) \psubstp{Q}{P} ) \\
  (\lift{x}{R}) \psubstp{Q}{P}  
  :=
  \lift{(x)\substp{Q}{P}}{ R \psubstp{Q}{P} } \\
%   (\dropn{x})  \psubstp{Q}{P}       
%   := 
%   \left\{ 
%     \begin{array}{ccc} 
%       \dropn{\quotep{Q}} & & x \nameeq \quotep{P} \\
%       \dropn{x} & & otherwise \\
%     \end{array}
%   \right. 
  (\dropn{x})  \psubstp{Q}{P}       
  := 
  \left\{ 
    \begin{array}{ccc} 
      Q & & x \nameeq \quotep{P} \\
      \dropn{x} & & otherwise \\
    \end{array}
  \right.
\end{mathpar}
 

where

\begin{eqnarray}
  (x)\id{\{} \lpquote Q \rpquote / \lpquote P \rpquote \id{\}}            = 
  \left\{ 
    \begin{array}{ccc}
      \lpquote Q \rpquote & & x \nameeq \lpquote P \rpquote \\
      x & & otherwise \\
    \end{array}
  \right. \nonumber
\end{eqnarray}

and $z$ is chosen distinct from $\quotep{P}$, $\quotep{Q}$, the free
names in $Q$, and all the names in $R$. Our $\alpha$-equivalence will
be built in the standard way from this substitution.

\begin{remark}\label{rem:no_self_referential_names}
  One consequence of these definitions is that $\forall P. \quotep{P}
  \not\in \freenames{P}$.
\end{remark}

\subsection{ Dynamic quote: an example }

Anticipating something of what's to come, consider applying the
substitution, $\widehat{\id{\{}u / z \id{\}}}$, to the following pair
of processes, $\lift{w}{y!(z)}$ and $w[ \lpquote y!(z) \rpquote ]$.

\begin{eqnarray}
	\lift{w}{y!(z)}\widehat{\id{\{}u / z \id{\}}}
		& = &
		\lift{w}{y!(u)} \nonumber\\
	w[ \lpquote y!(z) \rpquote ] \widehat{ \id{\{}u / z \id{\}} }
		& = &
		w[ \lpquote y!(z) \rpquote ] \nonumber
\end{eqnarray}

Because the body of the process between quotes is impervious to
substitution, we get radically different answers. In fact, by
examining the first process in an input context,
e.g. $x?(z).\lift{w}{y!(z)}$, we see that the process under the lift
operator may be shaped by prefixed inputs binding a name inside it. In
this sense, the lift operator will be seen as a way to dynamically
construct processes before reifying them as names.

Finally equipped with these standard features we can present the
dynamics of the calculus.

\subsubsection{Operational semantics} 

Finally, we introduce the computational dynamics. What marks these
algebras as distinct from other more traditionally studied algebraic
structures, e.g. vector spaces or polynomial rings, is the manner in
which dynamics is captured. In traditional structures, dynamics is typically
expressed through morphisms between such structures, as in linear maps
between vector spaces or morphisms between rings. In algebras
associated with the semantics of computation, the dynamics is
expressed as part of the algebraic structure itself, through a
reduction reduction relation typically denoted by $\red$. Below, we
give a recursive presentation of this relation for the calculus used
in the encoding.

$\red \subseteq \pi \times \pi$
$\red : \pi \to \mathcal{P}(\pi)$

\begin{mathpar}
  \inferrule* [lab=Comm] { \textsf{match}( x_{src}, x_{trgt} ) } { x_{trgt}?(y)P \; | \; x_{src}!\langle {Q} \rangle \red P\{\quotep{Q}/y}\} }
  \and \\
  \inferrule* [lab=Par] {{P} \red {P}'} {{{P} | {Q}} \red {{P}' | {Q}}}
  \and
  \inferrule* [lab=Equiv]{{{P} \scong {P}'} \andalso {{P}' \red {Q}'} \andalso {{Q}' \scong {Q}}}{{P} \red {Q}}
\end{mathpar}

\begin{eqnarray*}
  match_{\equiv} (\quotep{P},\quotep{Q}) & := & P \equiv Q \\
  match_{\dagger}(\quotep{P},\quotep{Q}) & := & \forall R. P|Q \red^{*} R => R \red^{*} 0 \\
  match_{K}(\quotep{P},\quotep{Q}) & := & K \mbox{ for some context } K
\end{eqnarray*}

$u?(x)P | u!\langle Q \rangle \red P\{\quotep{Q}/x\}$

%We write $\wred$ for $\red^*$, and $P\red$ if $\exists Q $ such that $ P \red Q$.
We write $P\red$ if $\exists Q $ such that $ P \red Q$ and $P\not\red$, otherwise.

\section{Replication}

As mentioned before, it is known that replication (and hence
recursion) can be implemented in a higher-order process algebra
\cite{SangiorgiWalker}. As our first example of calculation with the
machinery thus far presented we give the construction explicitly in
the {\rhoc}.

\begin{eqnarray}
	D_{x} & := & \prefix{x}{y}{(\binpar{\outputp{x}{y}}{@{y}})} \nonumber\\
	\bangp_{x}{P} & := & \binpar{{x}!\langle{\binpar{D_{x}}{P}}\rangle}{D_{x}} \nonumber
\end{eqnarray}

\begin{eqnarray}
	\bangp_{x}{P} & & \nonumber\\
	=
	& {x}!\langle{(\prefix{x}{y}{(\outputp{x}{y} | @{y})) | P}}\rangle 
	      | \prefix{x}{y}{(\outputp{x}{y} | @{y})} & \nonumber\\
	\red
	& (\outputp{x}{y} | @{y})\substn{\quotep{(\prefix{x}{y}{(@{y} | \outputp{x}{y})) | P}}}{y} & \nonumber\\
	=
	& \outputp{x}{\quotep{(\prefix{x}{y}{(\outputp{x}{y} | @{y})) | P}}}
	  | {(\prefix{x}{y}{(\outputp{x}{y} | @{y})) | P}} & \nonumber\\
	\red
	& \ldots & \nonumber\\
	\red^*
	& P | P | \ldots & \nonumber
\end{eqnarray}

Of course, this encoding, as an implementation, runs away, unfolding
$\bangp{P}$ eagerly. A lazier and more implementable replication
operator, restricted to input-guarded processes, may be obtained as follows.

\begin{eqnarray}
\bangp{\prefix{u}{v}{P}} 
	:= 
	\binpar{\lift{x}{\prefix{u}{v}{(\binpar{D(x)}{P})}}}{D(x)} \nonumber
\end{eqnarray}

\begin{remark}
  Note that the lazier definition still does not deal with summation
  or mixed summation (i.e. sums over input and output). The reader is
  invited to construct definitions of replication that deal with these
  features. 

  Further, the definitions are parameterized in a name, $x$. Can you,
  gentle reader, make a definition that eliminates this parameter and
  guarantees no accidental interaction between the replication
  machinery and the process being replicated -- i.e. no accidental
  sharing of names used by the process to get its work done and the
  name(s) used by the replication to effect copying. This latter
  revision of the definition of replication is crucial to obtaining
  the expected identity $!!P \sim !P$.
\end{remark}

\begin{remark}\label{rem:paradoxical_combinator}
  The reader familiar with the lambda calculus will have noticed the
  similarity between $D$ and the paradoxical combinator.

  [Ed. note: the existence of this seems to suggest we have to be more
  restrictive on the set of processes and names we admit if we are to
  support no-cloning.]
\end{remark}

\subsubsection{Bisimulation}

The computational dynamics gives rise to another kind of equivalence,
the equivalence of computational behavior. As previously mentioned
this is typically captured \emph{via} some form of bisimulation.

% The notion we use in this paper is weak barbed bisimulation
% \cite{milner91polyadicpi}.

The notion we use in this paper is derived from weak barbed
bisimulation \cite{milner91polyadicpi}. 

\begin{definition}
An \emph{observation relation}, $\downarrow_{\mathcal N}$, over a set
of names, $\mathcal N$, is the smallest relation satisfying the rules
below.

\infrule[Out-barb]{y \in {\mathcal N}, \; x \nameeq y}
		  {\outputp{x}{v} \downarrow_{\mathcal N} x}
\infrule[Par-barb]{\mbox{$P\downarrow_{\mathcal N} x$ or $Q\downarrow_{\mathcal N} x$}}
		  {\binpar{P}{Q} \downarrow_{\mathcal N} x}

We write $P \Downarrow_{\mathcal N} x$ if there is $Q$ such that 
$P \wred Q$ and $Q \downarrow_{\mathcal N} x$.
\end{definition}

\begin{definition}
%\label{def.bbisim}
An  ${\mathcal N}$-\emph{barbed bisimulation} over a set of names, ${\mathcal N}$, is a symmetric binary relation 
${\mathcal S}_{\mathcal N}$ between agents such that $P\rel{S}_{\mathcal N}Q$ implies:
\begin{enumerate}
\item If $P \red P'$ then $Q \wred Q'$ and $P'\rel{S}_{\mathcal N} Q'$.
\item If $P\downarrow_{\mathcal N} x$, then $Q\Downarrow_{\mathcal N} x$.
\end{enumerate}
$P$ is ${\mathcal N}$-barbed bisimilar to $Q$, written
$P \wbbisim_{\mathcal N} Q$, if $P \rel{S}_{\mathcal N} Q$ for some ${\mathcal N}$-barbed bisimulation ${\mathcal S}_{\mathcal N}$.
\end{definition}

$\mathcal{R} \subseteq \pi \times \pi$

$P \mathcal{R} Q => \forall P'. P \red P' \Rightarrow \exists Q'. Q \red Q', P' \mathcal{R} Q'$

$P \vdash x \Rightarrow Q \vdash x$

\begin{mathpar}
  \inferrule*[lab=Out-barb]{x \nameeq y}{{y}!\langle{Q}\rangle \vdash x}
  \and
  \inferrule*[lab=Par-barb]{\mbox{$P\vdash x$ or $Q\vdash x$}}{\binpar{P}{Q} \vdash x}
\end{mathpar}

\subsubsection{Contexts}

One of the principle advantages of computational calculi like the
$\pi$-calculus is a well-defined notion of context,
contextual-equivalence and a correlation between
contextual-equivalence and notions of bisimulation. The notion of
context allows the decomposition of a process into (sub-)process and
its syntactic environment, its context. Thus, a context may be
thought of as a process with a ``hole'' (written $\Box$) in it. The
application of a context $M$ to a process $P$, written $M[P]$, is
tantamount to filling the hole in $M$ with $P$. In this paper we do
not need the full weight of this theory, but do make use of the notion
of context in the proof the main theorem. 

\begin{mathpar}
  \inferrule* [lab=summation] {} {{M_{M},M_{N}} \bc \Box \;|\; x.M_{A} \;|\; M_{M}+M_{N}}
  \and
  \inferrule* [lab=agent] {} {{M_{A}} \bc (\vec{x})M_{P} \;| \; \clift{P_0,\ldots,M_{P},\ldots,P_N}}
  \and \\
  \inferrule* [lab=process] {} {{M_{P}} \bc M_{N} \;| \;P|M_{P} }
\end{mathpar} 

\begin{mathpar}
  \inferrule* [lab=sychronization] {} {M_{N} \bc \Box \;|\; x?M_{F} \;|\; x!M_{C}}
  \and
  \inferrule* [lab=abstraction] {} {{M_{F}} \bc (x)M_{P} }
  \and
  \inferrule* [lab=concretion] {} {{M_{C}} \bc \langle M_{P} \rangle }
  \and \\
  \inferrule* [lab=process] {} {{M_{P}} \bc M_{N} \;| \;P|M_{P} }
\end{mathpar}

\begin{definition}[contextual application] Given a context $M$, and
  process $P$, we define the \emph{contextual application}, $M[P] :=
  M\{P/\Box\}$. That is, the contextual application of M to P is the
  substitution of $P$ for $\Box$ in $M$.
\end{definition}

$\meaningof{-} : L \to \mathcal{P}(\pi)$

\begin{mathpar}
  \inferrule* [lab=collection] {} {\meaningof{true} = \pi, \and \meaningof{~E} = \pi \setminus \meaningof{E}, \and \meaningof{E_{1} \& E_{2}} = \meaningof{E_{1}} \cap \meaningof{E_{2}}}
\end{mathpar}

\begin{mathpar}
  \inferrule* [lab=structure] {} {\meaningof{0} = \{ P \in \pi | P \equiv 0 \}, \and \\ \meaningof{E_1 | E_2} = \{ P \in \pi | P \equiv P_{1} | P_{2}, P_{1} \in \meaningof{E_{1}}, P_{2} \in \meaningof{E_2}\} }
\end{mathpar}

\begin{mathpar}
 \inferrule* [lab=behavior] {} {\meaningof{\langle a?b \rangle E} = \{ P \in \pi | P \equiv Q | u?(y)P', \\ \and \\\\ \and \\ \;\;\; u \in \meaningof{a}, \forall z.P'\{z/y\} \in \meaningof{E\{z/b\}}\}, \and \\ \meaningof{a!E} = \{ P \in \pi | P \equiv Q | x!\langle P' \rangle, x \in \meaningof{a} P' \in \meaningof{E}\} }
\end{mathpar}

\begin{mathpar}
 \inferrule* [lab=nominal] {} {\meaningof{\quotep{E}} = \{ \quotep{P} \in \quotep{\pi} | P \in \meaningof{E} \}, \and \meaningof{\quotep{P}} = \{ \quotep{Q} \in \quotep{\pi} | P \equiv Q \} \and \\ \meaningof{@\quotep{E}} = \{ P \in \pi | P \equiv @x, x \in \meaningof{E} \}}
\end{mathpar}

\begin{eqnarray*}
  \\
  \meaningof{-} : TS \to ST
\end{eqnarray*}

\begin{eqnarray*}
  \\
  L : TS \to ST
\end{eqnarray*}

\begin{eqnarray*}
  \\
  P \models E \iff P \in \meaningof{E}
\end{eqnarray*}

\begin{eqnarray*}
  P \approx_{L} Q \iff \forall E \in L. P \models E \iff Q \models E
\end{eqnarray*}

\begin{eqnarray*}
  P \approx_{K} Q
\end{eqnarray*}

\begin{eqnarray*}
  P \approx Q
\end{eqnarray*}

$\approx_{K} = \approx = \approx_{L}$

\subsubsection{Contextual duality}

Note that contexts extend the quotation operation to a family of
operations from processes to names. Given a context, $M$, we can
define a \emph{nominal context}, $\quotep{M}$ by $\quotep{M}[P] :=
\quotep{M[P]}$. To foreshadow what is to come we observe that these
operations enjoy a duality with processes very much like the duality
between vectors and maps from vectors to scalars.

Further, because the calculus is essentially higher-order, we have a
correspondence between contexts and processes. More specifically,
given a name $x$ and a context $M$ we can construct $M^{*}_{x}$ such
that 

\begin{mathpar}
  M^{*}_{x} | \lift{x}{P} \red M[P]
\end{mathpar}

namely,

\begin{mathpar}
  M^{*}_{x} := x?(u).M[\dropn{u}]
\end{mathpar}

The dependence of $M^{*}_{x}$ on a name makes it an abstraction, 

\begin{mathpar}
  M^{*} := (x)x?(u).M[\dropn{u}]
\end{mathpar}

\subsection{Additional notation}

It will sometimes be convenient to denote the process a name
quotes. We already have the notation $x = \quotep{P}$, but it will be
convenient to introduce an alternate notation, $\procn{x}$, when we
want to emphasize the connection to the use of the name. Note that, by
virtue of name equivalence, $\quotep{\procn{x}} \nameeq x$; so, the
notation is consistent with previous definitions.

Further, because names have structure it is possible to effect
substitutions on the basis of that structure. This means we need to
upgrade our notation for substitutions, which we accomplish by
adapting comprehension notation. Thus,

\begin{mathpar}
  P\{ y / x : x \in S \}
\end{mathpar}

is interpreted to mean the process derived from P by replacing (in a
capture-avoiding manner) each occurrence of $x$ in $S$ by $y$. For example,

\begin{mathpar}
  P\{ \quotep{\procn{x}|\procn{x}} / x : x \in \freenames{P} \}
\end{mathpar}

will replace each (occurrence) of a free name $x$ in $P$ by
$\quotep{\procn{x}|\procn{x}}$.

Also, we will avail ourselves of the notation $x^{L}$ and $x^{R}$ to
denote injections of a name into disjoint copies of the name
space. There are numerous ways to accomplish this. One example can be
found in \cite{MeredithR05}. This notation overloads to vectors of
names: $\vec{x}^{\pi} := (x_{i}^{\pi} \; : \; 0 \leq i < |\vec{x}| )$ where $\pi \in \{L,R\}$.

We also use $P^{\Box} := P|\Box$.

In \cite{MeredithR05} an interpretation of the new operator is
given. It turns out that there are several possible interpretations
all enjoying the requisite algebraic properties of the operator (see
\cite{milner91polyadicpi}). We will therefore make liberal use of
$(\nu\; \vec{x})P$.

% subsection the_syntax_and_semantics_of_the_notation_system (end)   

\input{qm2pi.qmops} 

\input{qm2pi.sterngerlach} 

\input{qm2pi.metric} 

% section concurrent_process_calculi (end)

%\input{qm2pi.proofsketch}

% section proof sketch (end)

%\input{qm2pi.slviaknots} 

% section spatial logic via knots (end)

\input{qm2pi.conclusion}

% section conclusion (end)

%\input{qm2pi.dtcodes} 

% section wiring algorithm (end)

\input{qm2pi.ack} 

% section acknowledgments (end)

\newpage


\bibliographystyle{plain}   
\bibliography{../../biblios/main.bib}

\input{qm2pi.rhodetails}

\end{document}

 

%\documentclass[12pt]{llncs}
%\documentclass{jktr}

\usepackage[pdftex]{hyperref}                   
\usepackage {listings}
\usepackage {mathpartir}
\usepackage{bcprules}
%\usepackage{listings}
                       
\usepackage{graphicx} 
%\usepackage[margins=2.5cm,nohead,nofoot]{geometry}
%\usepackage{geometry}
\usepackage{amsfonts}
\usepackage{amstext}
\usepackage{latexsym}
\usepackage{amssymb}
\usepackage{color}


%\include{myPreamble}
\include{qm2pi.local} 

%\ifpdf
%\usepackage[pdftex]{graphicx}
%\else
%\usepackage{graphicx}
%\fi

 % \ifpdf
%  \usepackage{pdfsync}
%  \if


%\title{Brief Article}
%\author{David F. Snyder}
%\author{L.G. Meredith}

%\address{Dept. of Math., Texas State University--San Marcos, San Marcos, TX 78666}
       
\pagestyle{empty}


\begin{document}

\lstset{language=[Objective]Caml,frame=shadowbox}

\input{qm2pi.front}

% section front matter (end)

\input{qm2pi.intro} 
 
% section introduction (end)

% \input{qm2pi.knotations} 

% section notation (end)

\input{qm2pi.process.calculi} 

% section concurrent_process_calculi_and_spatial_logics_ (end)
    
%\input{qm2pi.knots2pi} 

%\input{qm2pi.trefoil} 

%\input{qm2pi.mainthm} 

% subsection basic_interpretation (end)

%\input{qm2pi.rho.presentation} 
\subsection{The syntax and semantics of the notation system}\label{sub:the_syntax_and_semantics_of_the_notation_system} % (fold)

We now summarize a technical presentation of the calculus that
embodies our theory of dynamics. The typical presentation of such a
calculus follows the style of giving generators and relations on
them. The grammar, below, describing term constructors, freely
generates the set of processes, $\Proc$. This set is then quotiented
by a relation known as structural congruence and it is over this set
that the notion of dynamics is expressed. This presentation is
essentially that of \cite{MeredithR05} with the addition of
polyadicity and summation. For readability we have relegated some of
the technical subtleties to an appendix.

\subsubsection{Process grammar}\label{subsub:process_grammar}

\begin{mathpar}
  \inferrule* [lab=synchronization] {} {{M} \bc \pzero \;|\; x?F \;|\; x!C }
  \and
  \inferrule* [lab=abstraction] {} {{F} \bc (x)P}
  \and
  \inferrule* [lab=concretion] {} {{C} \bc \langle Q \rangle}
  \and
  \inferrule* [lab=process] {} {{P,Q} \bc M \;| \;P|Q \;|\; @{x}}
  \and
  \inferrule* [lab=name] {} {{x} \bc \quotep{P}}
\end{mathpar} 

Note that $\vec{x}$ (resp. $\vec{P}$) denotes a vector of names
(resp. processes) of length $|\vec{x}|$ (resp. $|\vec{P}|$). We adopt
the following useful abbreviations.

\begin{mathpar}
   x?(\vec{y}).P := x.(\vec{y})P \and  x\clift{\vec{P}} := x.\clift{\vec{P}}
   \and x!(y) := \lift{x}{\dropn{y}}
   \and \Pi_{i=0}^{n-1}P_i := P_0 | \ldots | P_{n-1}
\end{mathpar}

\subsubsection{Structural congruence}

\paragraph{Free and bound names and alpha-equivalence.} At the
core of structural equivalence is alpha-equivalence which identifies
process that are the same up to a change of variable. Formally, we
recognize the distinction between free and bound names. The free names
of a process, $\freenames{P}$, may be calculated recursively as
follows:

\begin{mathpar}
\freenames{\pzero} := \emptyset
  \and \\
  \freenames{x?(y).P} := \{ x \} \cup (\freenames{P} \setminus \{ y \})
  \and 
  \freenames{x!\langle P \rangle} := \{ x \} \cup \{ P \} 
  \and \\
  \freenames{P|Q} := \freenames{P} \cup \freenames{Q}
  \and \\
  \freenames{@{x}} := \{ x \}
\end{mathpar}

$\pi$
$\quotep{\pi}$

$\freenames{-} : \pi \to \mathcal{P}(\quotep{\pi})$

\begin{eqnarray*}
  \freenames{\pzero} & := & \emptyset \\
  \freenames{x?(y).P} & := & \{ x \} \cup (\freenames{P} \setminus \{ y \}) \\
  \freenames{x!\langle P \rangle} & := & \{ x \} \cup \{ P \} \\
  \freenames{P|Q} & := & \freenames{P} \cup \freenames{Q} \\
  \freenames{\dropn{x}} & := & \{ x \}
\end{eqnarray*}

The bound names of a process, $\boundnames{P}$, are those names occurring in $P$
that are not free. For example, in $x?(y).0$, the name $x$ is free, while $y$ is bound.

\begin{mathpar}
  \inferrule* [lab=monoidal-laws] {} { P|Q \equiv Q|P \and P|0 \equiv P \and P|(Q|R) \equiv (P|Q)|R }
\end{mathpar}

\begin{mathpar}
  \inferrule* [lab=alpha-equivalence] {} { (x)P \equiv (y)P\{y/x\} \and y \not\in \freenames{P} }
\end{mathpar}

\begin{definition}
Then two processes, $P,Q$, are alpha-equivalent if $P = Q\{\vec{y}/\vec{x}\}$ for
some $\vec{x} \in \boundnames{Q},\vec{y} \in \boundnames{P}$, where $Q\{\vec{y}/\vec{x}\}$
denotes the capture-avoiding substitution of $\vec{y}$ for $\vec{x}$ in $Q$.
\end{definition}

\begin{definition}
  The {\em structural congruence} \cite{SangiorgiWalker} , $\equiv$,
  between processes is the least congruence containing
  alpha-equivalence, satisfying the abelian monoid laws
  (associativity, commutativity and $\pzero$ as identity) for parallel
  composition $|$ and for summation $+$.
\end{definition}

\subsection{Name equivalence}

We take name equivalence, written $\nameeq$, to be the smallest
equivalence relation generated by the following rules.

\begin{mathpar}
\inferrule*[lab=Quote-drop]
{ }
{ \quotep{@{x}} \nameeq x }

\inferrule*[lab=Struct-equiv]
{ P \scong Q }
{ \quotep{P} \nameeq \quotep{Q} }
\end{mathpar}

The astute reader will have noticed that the mutual recursion of names
and processes imposes a mutual recursion on alpha-equivalence and
structural equivalence via name-equivalence. Fortunately, all of this
works out pleasantly and we may calculate in the natural way, free of
concern. The reader interested in the details is referred to the
appendix \ref{appendix:rho_details}.

\subsection{Substitution}

We use $\Proc$ for the set of processes, $\QProc$ for the set of
names, and $\id{\{}\vec{y} / \vec{x} \id{\}}$ to denote partial maps,
$s : \QProc \rightarrow \QProc$. A map, $s$ lifts, uniquely, to a map
on process terms, $\widehat{s} : \Proc \rightarrow \Proc$ by the
following equations.

\begin{mathpar}
  (0) \psubstp{Q}{P} := 0 \\
  (R \juxtap S) \psubstp{Q}{P}
  :=    
  (R)\psubstp{Q}{P} \juxtap (S) \psubstp{Q}{P} \\
  (x?(y).R) \psubstp{Q}{P}    
  :=    
  (x)\substp{Q}{P} (z)\concat( (R \psubstn{z}{y}) \psubstp{Q}{P} ) \\
  (\lift{x}{R}) \psubstp{Q}{P}  
  :=
  \lift{(x)\substp{Q}{P}}{ R \psubstp{Q}{P} } \\
%   (\dropn{x})  \psubstp{Q}{P}       
%   := 
%   \left\{ 
%     \begin{array}{ccc} 
%       \dropn{\quotep{Q}} & & x \nameeq \quotep{P} \\
%       \dropn{x} & & otherwise \\
%     \end{array}
%   \right. 
  (\dropn{x})  \psubstp{Q}{P}       
  := 
  \left\{ 
    \begin{array}{ccc} 
      Q & & x \nameeq \quotep{P} \\
      \dropn{x} & & otherwise \\
    \end{array}
  \right.
\end{mathpar}
 

where

\begin{eqnarray}
  (x)\id{\{} \lpquote Q \rpquote / \lpquote P \rpquote \id{\}}            = 
  \left\{ 
    \begin{array}{ccc}
      \lpquote Q \rpquote & & x \nameeq \lpquote P \rpquote \\
      x & & otherwise \\
    \end{array}
  \right. \nonumber
\end{eqnarray}

and $z$ is chosen distinct from $\quotep{P}$, $\quotep{Q}$, the free
names in $Q$, and all the names in $R$. Our $\alpha$-equivalence will
be built in the standard way from this substitution.

\begin{remark}\label{rem:no_self_referential_names}
  One consequence of these definitions is that $\forall P. \quotep{P}
  \not\in \freenames{P}$.
\end{remark}

\subsection{ Dynamic quote: an example }

Anticipating something of what's to come, consider applying the
substitution, $\widehat{\id{\{}u / z \id{\}}}$, to the following pair
of processes, $\lift{w}{y!(z)}$ and $w[ \lpquote y!(z) \rpquote ]$.

\begin{eqnarray}
	\lift{w}{y!(z)}\widehat{\id{\{}u / z \id{\}}}
		& = &
		\lift{w}{y!(u)} \nonumber\\
	w[ \lpquote y!(z) \rpquote ] \widehat{ \id{\{}u / z \id{\}} }
		& = &
		w[ \lpquote y!(z) \rpquote ] \nonumber
\end{eqnarray}

Because the body of the process between quotes is impervious to
substitution, we get radically different answers. In fact, by
examining the first process in an input context,
e.g. $x?(z).\lift{w}{y!(z)}$, we see that the process under the lift
operator may be shaped by prefixed inputs binding a name inside it. In
this sense, the lift operator will be seen as a way to dynamically
construct processes before reifying them as names.

Finally equipped with these standard features we can present the
dynamics of the calculus.

\subsubsection{Operational semantics} 

Finally, we introduce the computational dynamics. What marks these
algebras as distinct from other more traditionally studied algebraic
structures, e.g. vector spaces or polynomial rings, is the manner in
which dynamics is captured. In traditional structures, dynamics is typically
expressed through morphisms between such structures, as in linear maps
between vector spaces or morphisms between rings. In algebras
associated with the semantics of computation, the dynamics is
expressed as part of the algebraic structure itself, through a
reduction reduction relation typically denoted by $\red$. Below, we
give a recursive presentation of this relation for the calculus used
in the encoding.

$\red \subseteq \pi \times \pi$
$\red : \pi \to \mathcal{P}(\pi)$

\begin{mathpar}
  \inferrule* [lab=Comm] { \textsf{match}( x_{src}, x_{trgt} ) } { x_{trgt}?(y)P \; | \; x_{src}!\langle {Q} \rangle \red P\{\quotep{Q}/y}\} }
  \and \\
  \inferrule* [lab=Par] {{P} \red {P}'} {{{P} | {Q}} \red {{P}' | {Q}}}
  \and
  \inferrule* [lab=Equiv]{{{P} \scong {P}'} \andalso {{P}' \red {Q}'} \andalso {{Q}' \scong {Q}}}{{P} \red {Q}}
\end{mathpar}

\begin{eqnarray*}
  match_{\equiv} (\quotep{P},\quotep{Q}) & := & P \equiv Q \\
  match_{\dagger}(\quotep{P},\quotep{Q}) & := & \forall R. P|Q \red^{*} R => R \red^{*} 0 \\
  match_{K}(\quotep{P},\quotep{Q}) & := & K \mbox{ for some context } K
\end{eqnarray*}

$u?(x)P | u!\langle Q \rangle \red P\{\quotep{Q}/x\}$

%We write $\wred$ for $\red^*$, and $P\red$ if $\exists Q $ such that $ P \red Q$.
We write $P\red$ if $\exists Q $ such that $ P \red Q$ and $P\not\red$, otherwise.

\section{Replication}

As mentioned before, it is known that replication (and hence
recursion) can be implemented in a higher-order process algebra
\cite{SangiorgiWalker}. As our first example of calculation with the
machinery thus far presented we give the construction explicitly in
the {\rhoc}.

\begin{eqnarray}
	D_{x} & := & \prefix{x}{y}{(\binpar{\outputp{x}{y}}{@{y}})} \nonumber\\
	\bangp_{x}{P} & := & \binpar{{x}!\langle{\binpar{D_{x}}{P}}\rangle}{D_{x}} \nonumber
\end{eqnarray}

\begin{eqnarray}
	\bangp_{x}{P} & & \nonumber\\
	=
	& {x}!\langle{(\prefix{x}{y}{(\outputp{x}{y} | @{y})) | P}}\rangle 
	      | \prefix{x}{y}{(\outputp{x}{y} | @{y})} & \nonumber\\
	\red
	& (\outputp{x}{y} | @{y})\substn{\quotep{(\prefix{x}{y}{(@{y} | \outputp{x}{y})) | P}}}{y} & \nonumber\\
	=
	& \outputp{x}{\quotep{(\prefix{x}{y}{(\outputp{x}{y} | @{y})) | P}}}
	  | {(\prefix{x}{y}{(\outputp{x}{y} | @{y})) | P}} & \nonumber\\
	\red
	& \ldots & \nonumber\\
	\red^*
	& P | P | \ldots & \nonumber
\end{eqnarray}

Of course, this encoding, as an implementation, runs away, unfolding
$\bangp{P}$ eagerly. A lazier and more implementable replication
operator, restricted to input-guarded processes, may be obtained as follows.

\begin{eqnarray}
\bangp{\prefix{u}{v}{P}} 
	:= 
	\binpar{\lift{x}{\prefix{u}{v}{(\binpar{D(x)}{P})}}}{D(x)} \nonumber
\end{eqnarray}

\begin{remark}
  Note that the lazier definition still does not deal with summation
  or mixed summation (i.e. sums over input and output). The reader is
  invited to construct definitions of replication that deal with these
  features. 

  Further, the definitions are parameterized in a name, $x$. Can you,
  gentle reader, make a definition that eliminates this parameter and
  guarantees no accidental interaction between the replication
  machinery and the process being replicated -- i.e. no accidental
  sharing of names used by the process to get its work done and the
  name(s) used by the replication to effect copying. This latter
  revision of the definition of replication is crucial to obtaining
  the expected identity $!!P \sim !P$.
\end{remark}

\begin{remark}\label{rem:paradoxical_combinator}
  The reader familiar with the lambda calculus will have noticed the
  similarity between $D$ and the paradoxical combinator.

  [Ed. note: the existence of this seems to suggest we have to be more
  restrictive on the set of processes and names we admit if we are to
  support no-cloning.]
\end{remark}

\subsubsection{Bisimulation}

The computational dynamics gives rise to another kind of equivalence,
the equivalence of computational behavior. As previously mentioned
this is typically captured \emph{via} some form of bisimulation.

% The notion we use in this paper is weak barbed bisimulation
% \cite{milner91polyadicpi}.

The notion we use in this paper is derived from weak barbed
bisimulation \cite{milner91polyadicpi}. 

\begin{definition}
An \emph{observation relation}, $\downarrow_{\mathcal N}$, over a set
of names, $\mathcal N$, is the smallest relation satisfying the rules
below.

\infrule[Out-barb]{y \in {\mathcal N}, \; x \nameeq y}
		  {\outputp{x}{v} \downarrow_{\mathcal N} x}
\infrule[Par-barb]{\mbox{$P\downarrow_{\mathcal N} x$ or $Q\downarrow_{\mathcal N} x$}}
		  {\binpar{P}{Q} \downarrow_{\mathcal N} x}

We write $P \Downarrow_{\mathcal N} x$ if there is $Q$ such that 
$P \wred Q$ and $Q \downarrow_{\mathcal N} x$.
\end{definition}

\begin{definition}
%\label{def.bbisim}
An  ${\mathcal N}$-\emph{barbed bisimulation} over a set of names, ${\mathcal N}$, is a symmetric binary relation 
${\mathcal S}_{\mathcal N}$ between agents such that $P\rel{S}_{\mathcal N}Q$ implies:
\begin{enumerate}
\item If $P \red P'$ then $Q \wred Q'$ and $P'\rel{S}_{\mathcal N} Q'$.
\item If $P\downarrow_{\mathcal N} x$, then $Q\Downarrow_{\mathcal N} x$.
\end{enumerate}
$P$ is ${\mathcal N}$-barbed bisimilar to $Q$, written
$P \wbbisim_{\mathcal N} Q$, if $P \rel{S}_{\mathcal N} Q$ for some ${\mathcal N}$-barbed bisimulation ${\mathcal S}_{\mathcal N}$.
\end{definition}

$\mathcal{R} \subseteq \pi \times \pi$

$P \mathcal{R} Q => \forall P'. P \red P' \Rightarrow \exists Q'. Q \red Q', P' \mathcal{R} Q'$

$P \vdash x \Rightarrow Q \vdash x$

\begin{mathpar}
  \inferrule*[lab=Out-barb]{x \nameeq y}{{y}!\langle{Q}\rangle \vdash x}
  \and
  \inferrule*[lab=Par-barb]{\mbox{$P\vdash x$ or $Q\vdash x$}}{\binpar{P}{Q} \vdash x}
\end{mathpar}

\subsubsection{Contexts}

One of the principle advantages of computational calculi like the
$\pi$-calculus is a well-defined notion of context,
contextual-equivalence and a correlation between
contextual-equivalence and notions of bisimulation. The notion of
context allows the decomposition of a process into (sub-)process and
its syntactic environment, its context. Thus, a context may be
thought of as a process with a ``hole'' (written $\Box$) in it. The
application of a context $M$ to a process $P$, written $M[P]$, is
tantamount to filling the hole in $M$ with $P$. In this paper we do
not need the full weight of this theory, but do make use of the notion
of context in the proof the main theorem. 

\begin{mathpar}
  \inferrule* [lab=summation] {} {{M_{M},M_{N}} \bc \Box \;|\; x.M_{A} \;|\; M_{M}+M_{N}}
  \and
  \inferrule* [lab=agent] {} {{M_{A}} \bc (\vec{x})M_{P} \;| \; \clift{P_0,\ldots,M_{P},\ldots,P_N}}
  \and \\
  \inferrule* [lab=process] {} {{M_{P}} \bc M_{N} \;| \;P|M_{P} }
\end{mathpar} 

\begin{mathpar}
  \inferrule* [lab=sychronization] {} {M_{N} \bc \Box \;|\; x?M_{F} \;|\; x!M_{C}}
  \and
  \inferrule* [lab=abstraction] {} {{M_{F}} \bc (x)M_{P} }
  \and
  \inferrule* [lab=concretion] {} {{M_{C}} \bc \langle M_{P} \rangle }
  \and \\
  \inferrule* [lab=process] {} {{M_{P}} \bc M_{N} \;| \;P|M_{P} }
\end{mathpar}

\begin{definition}[contextual application] Given a context $M$, and
  process $P$, we define the \emph{contextual application}, $M[P] :=
  M\{P/\Box\}$. That is, the contextual application of M to P is the
  substitution of $P$ for $\Box$ in $M$.
\end{definition}

$\meaningof{-} : L \to \mathcal{P}(\pi)$

\begin{mathpar}
  \inferrule* [lab=collection] {} {\meaningof{true} = \pi, \and \meaningof{~E} = \pi \setminus \meaningof{E}, \and \meaningof{E_{1} \& E_{2}} = \meaningof{E_{1}} \cap \meaningof{E_{2}}}
\end{mathpar}

\begin{mathpar}
  \inferrule* [lab=structure] {} {\meaningof{0} = \{ P \in \pi | P \equiv 0 \}, \and \\ \meaningof{E_1 | E_2} = \{ P \in \pi | P \equiv P_{1} | P_{2}, P_{1} \in \meaningof{E_{1}}, P_{2} \in \meaningof{E_2}\} }
\end{mathpar}

\begin{mathpar}
 \inferrule* [lab=behavior] {} {\meaningof{\langle a?b \rangle E} = \{ P \in \pi | P \equiv Q | u?(y)P', \\ \and \\\\ \and \\ \;\;\; u \in \meaningof{a}, \forall z.P'\{z/y\} \in \meaningof{E\{z/b\}}\}, \and \\ \meaningof{a!E} = \{ P \in \pi | P \equiv Q | x!\langle P' \rangle, x \in \meaningof{a} P' \in \meaningof{E}\} }
\end{mathpar}

\begin{mathpar}
 \inferrule* [lab=nominal] {} {\meaningof{\quotep{E}} = \{ \quotep{P} \in \quotep{\pi} | P \in \meaningof{E} \}, \and \meaningof{\quotep{P}} = \{ \quotep{Q} \in \quotep{\pi} | P \equiv Q \} \and \\ \meaningof{@\quotep{E}} = \{ P \in \pi | P \equiv @x, x \in \meaningof{E} \}}
\end{mathpar}

\begin{eqnarray*}
  \\
  \meaningof{-} : TS \to ST
\end{eqnarray*}

\begin{eqnarray*}
  \\
  L : TS \to ST
\end{eqnarray*}

\begin{eqnarray*}
  \\
  P \models E \iff P \in \meaningof{E}
\end{eqnarray*}

\begin{eqnarray*}
  P \approx_{L} Q \iff \forall E \in L. P \models E \iff Q \models E
\end{eqnarray*}

\begin{eqnarray*}
  P \approx_{K} Q
\end{eqnarray*}

\begin{eqnarray*}
  P \approx Q
\end{eqnarray*}

$\approx_{K} = \approx = \approx_{L}$

\subsubsection{Contextual duality}

Note that contexts extend the quotation operation to a family of
operations from processes to names. Given a context, $M$, we can
define a \emph{nominal context}, $\quotep{M}$ by $\quotep{M}[P] :=
\quotep{M[P]}$. To foreshadow what is to come we observe that these
operations enjoy a duality with processes very much like the duality
between vectors and maps from vectors to scalars.

Further, because the calculus is essentially higher-order, we have a
correspondence between contexts and processes. More specifically,
given a name $x$ and a context $M$ we can construct $M^{*}_{x}$ such
that 

\begin{mathpar}
  M^{*}_{x} | \lift{x}{P} \red M[P]
\end{mathpar}

namely,

\begin{mathpar}
  M^{*}_{x} := x?(u).M[\dropn{u}]
\end{mathpar}

The dependence of $M^{*}_{x}$ on a name makes it an abstraction, 

\begin{mathpar}
  M^{*} := (x)x?(u).M[\dropn{u}]
\end{mathpar}

\subsection{Additional notation}

It will sometimes be convenient to denote the process a name
quotes. We already have the notation $x = \quotep{P}$, but it will be
convenient to introduce an alternate notation, $\procn{x}$, when we
want to emphasize the connection to the use of the name. Note that, by
virtue of name equivalence, $\quotep{\procn{x}} \nameeq x$; so, the
notation is consistent with previous definitions.

Further, because names have structure it is possible to effect
substitutions on the basis of that structure. This means we need to
upgrade our notation for substitutions, which we accomplish by
adapting comprehension notation. Thus,

\begin{mathpar}
  P\{ y / x : x \in S \}
\end{mathpar}

is interpreted to mean the process derived from P by replacing (in a
capture-avoiding manner) each occurrence of $x$ in $S$ by $y$. For example,

\begin{mathpar}
  P\{ \quotep{\procn{x}|\procn{x}} / x : x \in \freenames{P} \}
\end{mathpar}

will replace each (occurrence) of a free name $x$ in $P$ by
$\quotep{\procn{x}|\procn{x}}$.

Also, we will avail ourselves of the notation $x^{L}$ and $x^{R}$ to
denote injections of a name into disjoint copies of the name
space. There are numerous ways to accomplish this. One example can be
found in \cite{MeredithR05}. This notation overloads to vectors of
names: $\vec{x}^{\pi} := (x_{i}^{\pi} \; : \; 0 \leq i < |\vec{x}| )$ where $\pi \in \{L,R\}$.

We also use $P^{\Box} := P|\Box$.

In \cite{MeredithR05} an interpretation of the new operator is
given. It turns out that there are several possible interpretations
all enjoying the requisite algebraic properties of the operator (see
\cite{milner91polyadicpi}). We will therefore make liberal use of
$(\nu\; \vec{x})P$.

% subsection the_syntax_and_semantics_of_the_notation_system (end)   

\input{qm2pi.qmops} 

\input{qm2pi.sterngerlach} 

\input{qm2pi.metric} 

% section concurrent_process_calculi (end)

%\input{qm2pi.proofsketch}

% section proof sketch (end)

%\input{qm2pi.slviaknots} 

% section spatial logic via knots (end)

\input{qm2pi.conclusion}

% section conclusion (end)

%\input{qm2pi.dtcodes} 

% section wiring algorithm (end)

\input{qm2pi.ack} 

% section acknowledgments (end)

\newpage


\bibliographystyle{plain}   
\bibliography{../../biblios/main.bib}

\input{qm2pi.rhodetails}

\end{document}

 

%\documentclass[12pt]{llncs}
%\documentclass{jktr}

\usepackage[pdftex]{hyperref}                   
\usepackage {listings}
\usepackage {mathpartir}
\usepackage{bcprules}
%\usepackage{listings}
                       
\usepackage{graphicx} 
%\usepackage[margins=2.5cm,nohead,nofoot]{geometry}
%\usepackage{geometry}
\usepackage{amsfonts}
\usepackage{amstext}
\usepackage{latexsym}
\usepackage{amssymb}
\usepackage{color}


%\include{myPreamble}
\include{qm2pi.local} 

%\ifpdf
%\usepackage[pdftex]{graphicx}
%\else
%\usepackage{graphicx}
%\fi

 % \ifpdf
%  \usepackage{pdfsync}
%  \if


%\title{Brief Article}
%\author{David F. Snyder}
%\author{L.G. Meredith}

%\address{Dept. of Math., Texas State University--San Marcos, San Marcos, TX 78666}
       
\pagestyle{empty}


\begin{document}

\lstset{language=[Objective]Caml,frame=shadowbox}

\input{qm2pi.front}

% section front matter (end)

\input{qm2pi.intro} 
 
% section introduction (end)

% \input{qm2pi.knotations} 

% section notation (end)

\input{qm2pi.process.calculi} 

% section concurrent_process_calculi_and_spatial_logics_ (end)
    
%\input{qm2pi.knots2pi} 

%\input{qm2pi.trefoil} 

%\input{qm2pi.mainthm} 

% subsection basic_interpretation (end)

%\input{qm2pi.rho.presentation} 
\subsection{The syntax and semantics of the notation system}\label{sub:the_syntax_and_semantics_of_the_notation_system} % (fold)

We now summarize a technical presentation of the calculus that
embodies our theory of dynamics. The typical presentation of such a
calculus follows the style of giving generators and relations on
them. The grammar, below, describing term constructors, freely
generates the set of processes, $\Proc$. This set is then quotiented
by a relation known as structural congruence and it is over this set
that the notion of dynamics is expressed. This presentation is
essentially that of \cite{MeredithR05} with the addition of
polyadicity and summation. For readability we have relegated some of
the technical subtleties to an appendix.

\subsubsection{Process grammar}\label{subsub:process_grammar}

\begin{mathpar}
  \inferrule* [lab=synchronization] {} {{M} \bc \pzero \;|\; x?F \;|\; x!C }
  \and
  \inferrule* [lab=abstraction] {} {{F} \bc (x)P}
  \and
  \inferrule* [lab=concretion] {} {{C} \bc \langle Q \rangle}
  \and
  \inferrule* [lab=process] {} {{P,Q} \bc M \;| \;P|Q \;|\; @{x}}
  \and
  \inferrule* [lab=name] {} {{x} \bc \quotep{P}}
\end{mathpar} 

Note that $\vec{x}$ (resp. $\vec{P}$) denotes a vector of names
(resp. processes) of length $|\vec{x}|$ (resp. $|\vec{P}|$). We adopt
the following useful abbreviations.

\begin{mathpar}
   x?(\vec{y}).P := x.(\vec{y})P \and  x\clift{\vec{P}} := x.\clift{\vec{P}}
   \and x!(y) := \lift{x}{\dropn{y}}
   \and \Pi_{i=0}^{n-1}P_i := P_0 | \ldots | P_{n-1}
\end{mathpar}

\subsubsection{Structural congruence}

\paragraph{Free and bound names and alpha-equivalence.} At the
core of structural equivalence is alpha-equivalence which identifies
process that are the same up to a change of variable. Formally, we
recognize the distinction between free and bound names. The free names
of a process, $\freenames{P}$, may be calculated recursively as
follows:

\begin{mathpar}
\freenames{\pzero} := \emptyset
  \and \\
  \freenames{x?(y).P} := \{ x \} \cup (\freenames{P} \setminus \{ y \})
  \and 
  \freenames{x!\langle P \rangle} := \{ x \} \cup \{ P \} 
  \and \\
  \freenames{P|Q} := \freenames{P} \cup \freenames{Q}
  \and \\
  \freenames{@{x}} := \{ x \}
\end{mathpar}

$\pi$
$\quotep{\pi}$

$\freenames{-} : \pi \to \mathcal{P}(\quotep{\pi})$

\begin{eqnarray*}
  \freenames{\pzero} & := & \emptyset \\
  \freenames{x?(y).P} & := & \{ x \} \cup (\freenames{P} \setminus \{ y \}) \\
  \freenames{x!\langle P \rangle} & := & \{ x \} \cup \{ P \} \\
  \freenames{P|Q} & := & \freenames{P} \cup \freenames{Q} \\
  \freenames{\dropn{x}} & := & \{ x \}
\end{eqnarray*}

The bound names of a process, $\boundnames{P}$, are those names occurring in $P$
that are not free. For example, in $x?(y).0$, the name $x$ is free, while $y$ is bound.

\begin{mathpar}
  \inferrule* [lab=monoidal-laws] {} { P|Q \equiv Q|P \and P|0 \equiv P \and P|(Q|R) \equiv (P|Q)|R }
\end{mathpar}

\begin{mathpar}
  \inferrule* [lab=alpha-equivalence] {} { (x)P \equiv (y)P\{y/x\} \and y \not\in \freenames{P} }
\end{mathpar}

\begin{definition}
Then two processes, $P,Q$, are alpha-equivalent if $P = Q\{\vec{y}/\vec{x}\}$ for
some $\vec{x} \in \boundnames{Q},\vec{y} \in \boundnames{P}$, where $Q\{\vec{y}/\vec{x}\}$
denotes the capture-avoiding substitution of $\vec{y}$ for $\vec{x}$ in $Q$.
\end{definition}

\begin{definition}
  The {\em structural congruence} \cite{SangiorgiWalker} , $\equiv$,
  between processes is the least congruence containing
  alpha-equivalence, satisfying the abelian monoid laws
  (associativity, commutativity and $\pzero$ as identity) for parallel
  composition $|$ and for summation $+$.
\end{definition}

\subsection{Name equivalence}

We take name equivalence, written $\nameeq$, to be the smallest
equivalence relation generated by the following rules.

\begin{mathpar}
\inferrule*[lab=Quote-drop]
{ }
{ \quotep{@{x}} \nameeq x }

\inferrule*[lab=Struct-equiv]
{ P \scong Q }
{ \quotep{P} \nameeq \quotep{Q} }
\end{mathpar}

The astute reader will have noticed that the mutual recursion of names
and processes imposes a mutual recursion on alpha-equivalence and
structural equivalence via name-equivalence. Fortunately, all of this
works out pleasantly and we may calculate in the natural way, free of
concern. The reader interested in the details is referred to the
appendix \ref{appendix:rho_details}.

\subsection{Substitution}

We use $\Proc$ for the set of processes, $\QProc$ for the set of
names, and $\id{\{}\vec{y} / \vec{x} \id{\}}$ to denote partial maps,
$s : \QProc \rightarrow \QProc$. A map, $s$ lifts, uniquely, to a map
on process terms, $\widehat{s} : \Proc \rightarrow \Proc$ by the
following equations.

\begin{mathpar}
  (0) \psubstp{Q}{P} := 0 \\
  (R \juxtap S) \psubstp{Q}{P}
  :=    
  (R)\psubstp{Q}{P} \juxtap (S) \psubstp{Q}{P} \\
  (x?(y).R) \psubstp{Q}{P}    
  :=    
  (x)\substp{Q}{P} (z)\concat( (R \psubstn{z}{y}) \psubstp{Q}{P} ) \\
  (\lift{x}{R}) \psubstp{Q}{P}  
  :=
  \lift{(x)\substp{Q}{P}}{ R \psubstp{Q}{P} } \\
%   (\dropn{x})  \psubstp{Q}{P}       
%   := 
%   \left\{ 
%     \begin{array}{ccc} 
%       \dropn{\quotep{Q}} & & x \nameeq \quotep{P} \\
%       \dropn{x} & & otherwise \\
%     \end{array}
%   \right. 
  (\dropn{x})  \psubstp{Q}{P}       
  := 
  \left\{ 
    \begin{array}{ccc} 
      Q & & x \nameeq \quotep{P} \\
      \dropn{x} & & otherwise \\
    \end{array}
  \right.
\end{mathpar}
 

where

\begin{eqnarray}
  (x)\id{\{} \lpquote Q \rpquote / \lpquote P \rpquote \id{\}}            = 
  \left\{ 
    \begin{array}{ccc}
      \lpquote Q \rpquote & & x \nameeq \lpquote P \rpquote \\
      x & & otherwise \\
    \end{array}
  \right. \nonumber
\end{eqnarray}

and $z$ is chosen distinct from $\quotep{P}$, $\quotep{Q}$, the free
names in $Q$, and all the names in $R$. Our $\alpha$-equivalence will
be built in the standard way from this substitution.

\begin{remark}\label{rem:no_self_referential_names}
  One consequence of these definitions is that $\forall P. \quotep{P}
  \not\in \freenames{P}$.
\end{remark}

\subsection{ Dynamic quote: an example }

Anticipating something of what's to come, consider applying the
substitution, $\widehat{\id{\{}u / z \id{\}}}$, to the following pair
of processes, $\lift{w}{y!(z)}$ and $w[ \lpquote y!(z) \rpquote ]$.

\begin{eqnarray}
	\lift{w}{y!(z)}\widehat{\id{\{}u / z \id{\}}}
		& = &
		\lift{w}{y!(u)} \nonumber\\
	w[ \lpquote y!(z) \rpquote ] \widehat{ \id{\{}u / z \id{\}} }
		& = &
		w[ \lpquote y!(z) \rpquote ] \nonumber
\end{eqnarray}

Because the body of the process between quotes is impervious to
substitution, we get radically different answers. In fact, by
examining the first process in an input context,
e.g. $x?(z).\lift{w}{y!(z)}$, we see that the process under the lift
operator may be shaped by prefixed inputs binding a name inside it. In
this sense, the lift operator will be seen as a way to dynamically
construct processes before reifying them as names.

Finally equipped with these standard features we can present the
dynamics of the calculus.

\subsubsection{Operational semantics} 

Finally, we introduce the computational dynamics. What marks these
algebras as distinct from other more traditionally studied algebraic
structures, e.g. vector spaces or polynomial rings, is the manner in
which dynamics is captured. In traditional structures, dynamics is typically
expressed through morphisms between such structures, as in linear maps
between vector spaces or morphisms between rings. In algebras
associated with the semantics of computation, the dynamics is
expressed as part of the algebraic structure itself, through a
reduction reduction relation typically denoted by $\red$. Below, we
give a recursive presentation of this relation for the calculus used
in the encoding.

$\red \subseteq \pi \times \pi$
$\red : \pi \to \mathcal{P}(\pi)$

\begin{mathpar}
  \inferrule* [lab=Comm] { \textsf{match}( x_{src}, x_{trgt} ) } { x_{trgt}?(y)P \; | \; x_{src}!\langle {Q} \rangle \red P\{\quotep{Q}/y}\} }
  \and \\
  \inferrule* [lab=Par] {{P} \red {P}'} {{{P} | {Q}} \red {{P}' | {Q}}}
  \and
  \inferrule* [lab=Equiv]{{{P} \scong {P}'} \andalso {{P}' \red {Q}'} \andalso {{Q}' \scong {Q}}}{{P} \red {Q}}
\end{mathpar}

\begin{eqnarray*}
  match_{\equiv} (\quotep{P},\quotep{Q}) & := & P \equiv Q \\
  match_{\dagger}(\quotep{P},\quotep{Q}) & := & \forall R. P|Q \red^{*} R => R \red^{*} 0 \\
  match_{K}(\quotep{P},\quotep{Q}) & := & K \mbox{ for some context } K
\end{eqnarray*}

$u?(x)P | u!\langle Q \rangle \red P\{\quotep{Q}/x\}$

%We write $\wred$ for $\red^*$, and $P\red$ if $\exists Q $ such that $ P \red Q$.
We write $P\red$ if $\exists Q $ such that $ P \red Q$ and $P\not\red$, otherwise.

\section{Replication}

As mentioned before, it is known that replication (and hence
recursion) can be implemented in a higher-order process algebra
\cite{SangiorgiWalker}. As our first example of calculation with the
machinery thus far presented we give the construction explicitly in
the {\rhoc}.

\begin{eqnarray}
	D_{x} & := & \prefix{x}{y}{(\binpar{\outputp{x}{y}}{@{y}})} \nonumber\\
	\bangp_{x}{P} & := & \binpar{{x}!\langle{\binpar{D_{x}}{P}}\rangle}{D_{x}} \nonumber
\end{eqnarray}

\begin{eqnarray}
	\bangp_{x}{P} & & \nonumber\\
	=
	& {x}!\langle{(\prefix{x}{y}{(\outputp{x}{y} | @{y})) | P}}\rangle 
	      | \prefix{x}{y}{(\outputp{x}{y} | @{y})} & \nonumber\\
	\red
	& (\outputp{x}{y} | @{y})\substn{\quotep{(\prefix{x}{y}{(@{y} | \outputp{x}{y})) | P}}}{y} & \nonumber\\
	=
	& \outputp{x}{\quotep{(\prefix{x}{y}{(\outputp{x}{y} | @{y})) | P}}}
	  | {(\prefix{x}{y}{(\outputp{x}{y} | @{y})) | P}} & \nonumber\\
	\red
	& \ldots & \nonumber\\
	\red^*
	& P | P | \ldots & \nonumber
\end{eqnarray}

Of course, this encoding, as an implementation, runs away, unfolding
$\bangp{P}$ eagerly. A lazier and more implementable replication
operator, restricted to input-guarded processes, may be obtained as follows.

\begin{eqnarray}
\bangp{\prefix{u}{v}{P}} 
	:= 
	\binpar{\lift{x}{\prefix{u}{v}{(\binpar{D(x)}{P})}}}{D(x)} \nonumber
\end{eqnarray}

\begin{remark}
  Note that the lazier definition still does not deal with summation
  or mixed summation (i.e. sums over input and output). The reader is
  invited to construct definitions of replication that deal with these
  features. 

  Further, the definitions are parameterized in a name, $x$. Can you,
  gentle reader, make a definition that eliminates this parameter and
  guarantees no accidental interaction between the replication
  machinery and the process being replicated -- i.e. no accidental
  sharing of names used by the process to get its work done and the
  name(s) used by the replication to effect copying. This latter
  revision of the definition of replication is crucial to obtaining
  the expected identity $!!P \sim !P$.
\end{remark}

\begin{remark}\label{rem:paradoxical_combinator}
  The reader familiar with the lambda calculus will have noticed the
  similarity between $D$ and the paradoxical combinator.

  [Ed. note: the existence of this seems to suggest we have to be more
  restrictive on the set of processes and names we admit if we are to
  support no-cloning.]
\end{remark}

\subsubsection{Bisimulation}

The computational dynamics gives rise to another kind of equivalence,
the equivalence of computational behavior. As previously mentioned
this is typically captured \emph{via} some form of bisimulation.

% The notion we use in this paper is weak barbed bisimulation
% \cite{milner91polyadicpi}.

The notion we use in this paper is derived from weak barbed
bisimulation \cite{milner91polyadicpi}. 

\begin{definition}
An \emph{observation relation}, $\downarrow_{\mathcal N}$, over a set
of names, $\mathcal N$, is the smallest relation satisfying the rules
below.

\infrule[Out-barb]{y \in {\mathcal N}, \; x \nameeq y}
		  {\outputp{x}{v} \downarrow_{\mathcal N} x}
\infrule[Par-barb]{\mbox{$P\downarrow_{\mathcal N} x$ or $Q\downarrow_{\mathcal N} x$}}
		  {\binpar{P}{Q} \downarrow_{\mathcal N} x}

We write $P \Downarrow_{\mathcal N} x$ if there is $Q$ such that 
$P \wred Q$ and $Q \downarrow_{\mathcal N} x$.
\end{definition}

\begin{definition}
%\label{def.bbisim}
An  ${\mathcal N}$-\emph{barbed bisimulation} over a set of names, ${\mathcal N}$, is a symmetric binary relation 
${\mathcal S}_{\mathcal N}$ between agents such that $P\rel{S}_{\mathcal N}Q$ implies:
\begin{enumerate}
\item If $P \red P'$ then $Q \wred Q'$ and $P'\rel{S}_{\mathcal N} Q'$.
\item If $P\downarrow_{\mathcal N} x$, then $Q\Downarrow_{\mathcal N} x$.
\end{enumerate}
$P$ is ${\mathcal N}$-barbed bisimilar to $Q$, written
$P \wbbisim_{\mathcal N} Q$, if $P \rel{S}_{\mathcal N} Q$ for some ${\mathcal N}$-barbed bisimulation ${\mathcal S}_{\mathcal N}$.
\end{definition}

$\mathcal{R} \subseteq \pi \times \pi$

$P \mathcal{R} Q => \forall P'. P \red P' \Rightarrow \exists Q'. Q \red Q', P' \mathcal{R} Q'$

$P \vdash x \Rightarrow Q \vdash x$

\begin{mathpar}
  \inferrule*[lab=Out-barb]{x \nameeq y}{{y}!\langle{Q}\rangle \vdash x}
  \and
  \inferrule*[lab=Par-barb]{\mbox{$P\vdash x$ or $Q\vdash x$}}{\binpar{P}{Q} \vdash x}
\end{mathpar}

\subsubsection{Contexts}

One of the principle advantages of computational calculi like the
$\pi$-calculus is a well-defined notion of context,
contextual-equivalence and a correlation between
contextual-equivalence and notions of bisimulation. The notion of
context allows the decomposition of a process into (sub-)process and
its syntactic environment, its context. Thus, a context may be
thought of as a process with a ``hole'' (written $\Box$) in it. The
application of a context $M$ to a process $P$, written $M[P]$, is
tantamount to filling the hole in $M$ with $P$. In this paper we do
not need the full weight of this theory, but do make use of the notion
of context in the proof the main theorem. 

\begin{mathpar}
  \inferrule* [lab=summation] {} {{M_{M},M_{N}} \bc \Box \;|\; x.M_{A} \;|\; M_{M}+M_{N}}
  \and
  \inferrule* [lab=agent] {} {{M_{A}} \bc (\vec{x})M_{P} \;| \; \clift{P_0,\ldots,M_{P},\ldots,P_N}}
  \and \\
  \inferrule* [lab=process] {} {{M_{P}} \bc M_{N} \;| \;P|M_{P} }
\end{mathpar} 

\begin{mathpar}
  \inferrule* [lab=sychronization] {} {M_{N} \bc \Box \;|\; x?M_{F} \;|\; x!M_{C}}
  \and
  \inferrule* [lab=abstraction] {} {{M_{F}} \bc (x)M_{P} }
  \and
  \inferrule* [lab=concretion] {} {{M_{C}} \bc \langle M_{P} \rangle }
  \and \\
  \inferrule* [lab=process] {} {{M_{P}} \bc M_{N} \;| \;P|M_{P} }
\end{mathpar}

\begin{definition}[contextual application] Given a context $M$, and
  process $P$, we define the \emph{contextual application}, $M[P] :=
  M\{P/\Box\}$. That is, the contextual application of M to P is the
  substitution of $P$ for $\Box$ in $M$.
\end{definition}

$\meaningof{-} : L \to \mathcal{P}(\pi)$

\begin{mathpar}
  \inferrule* [lab=collection] {} {\meaningof{true} = \pi, \and \meaningof{~E} = \pi \setminus \meaningof{E}, \and \meaningof{E_{1} \& E_{2}} = \meaningof{E_{1}} \cap \meaningof{E_{2}}}
\end{mathpar}

\begin{mathpar}
  \inferrule* [lab=structure] {} {\meaningof{0} = \{ P \in \pi | P \equiv 0 \}, \and \\ \meaningof{E_1 | E_2} = \{ P \in \pi | P \equiv P_{1} | P_{2}, P_{1} \in \meaningof{E_{1}}, P_{2} \in \meaningof{E_2}\} }
\end{mathpar}

\begin{mathpar}
 \inferrule* [lab=behavior] {} {\meaningof{\langle a?b \rangle E} = \{ P \in \pi | P \equiv Q | u?(y)P', \\ \and \\\\ \and \\ \;\;\; u \in \meaningof{a}, \forall z.P'\{z/y\} \in \meaningof{E\{z/b\}}\}, \and \\ \meaningof{a!E} = \{ P \in \pi | P \equiv Q | x!\langle P' \rangle, x \in \meaningof{a} P' \in \meaningof{E}\} }
\end{mathpar}

\begin{mathpar}
 \inferrule* [lab=nominal] {} {\meaningof{\quotep{E}} = \{ \quotep{P} \in \quotep{\pi} | P \in \meaningof{E} \}, \and \meaningof{\quotep{P}} = \{ \quotep{Q} \in \quotep{\pi} | P \equiv Q \} \and \\ \meaningof{@\quotep{E}} = \{ P \in \pi | P \equiv @x, x \in \meaningof{E} \}}
\end{mathpar}

\begin{eqnarray*}
  \\
  \meaningof{-} : TS \to ST
\end{eqnarray*}

\begin{eqnarray*}
  \\
  L : TS \to ST
\end{eqnarray*}

\begin{eqnarray*}
  \\
  P \models E \iff P \in \meaningof{E}
\end{eqnarray*}

\begin{eqnarray*}
  P \approx_{L} Q \iff \forall E \in L. P \models E \iff Q \models E
\end{eqnarray*}

\begin{eqnarray*}
  P \approx_{K} Q
\end{eqnarray*}

\begin{eqnarray*}
  P \approx Q
\end{eqnarray*}

$\approx_{K} = \approx = \approx_{L}$

\subsubsection{Contextual duality}

Note that contexts extend the quotation operation to a family of
operations from processes to names. Given a context, $M$, we can
define a \emph{nominal context}, $\quotep{M}$ by $\quotep{M}[P] :=
\quotep{M[P]}$. To foreshadow what is to come we observe that these
operations enjoy a duality with processes very much like the duality
between vectors and maps from vectors to scalars.

Further, because the calculus is essentially higher-order, we have a
correspondence between contexts and processes. More specifically,
given a name $x$ and a context $M$ we can construct $M^{*}_{x}$ such
that 

\begin{mathpar}
  M^{*}_{x} | \lift{x}{P} \red M[P]
\end{mathpar}

namely,

\begin{mathpar}
  M^{*}_{x} := x?(u).M[\dropn{u}]
\end{mathpar}

The dependence of $M^{*}_{x}$ on a name makes it an abstraction, 

\begin{mathpar}
  M^{*} := (x)x?(u).M[\dropn{u}]
\end{mathpar}

\subsection{Additional notation}

It will sometimes be convenient to denote the process a name
quotes. We already have the notation $x = \quotep{P}$, but it will be
convenient to introduce an alternate notation, $\procn{x}$, when we
want to emphasize the connection to the use of the name. Note that, by
virtue of name equivalence, $\quotep{\procn{x}} \nameeq x$; so, the
notation is consistent with previous definitions.

Further, because names have structure it is possible to effect
substitutions on the basis of that structure. This means we need to
upgrade our notation for substitutions, which we accomplish by
adapting comprehension notation. Thus,

\begin{mathpar}
  P\{ y / x : x \in S \}
\end{mathpar}

is interpreted to mean the process derived from P by replacing (in a
capture-avoiding manner) each occurrence of $x$ in $S$ by $y$. For example,

\begin{mathpar}
  P\{ \quotep{\procn{x}|\procn{x}} / x : x \in \freenames{P} \}
\end{mathpar}

will replace each (occurrence) of a free name $x$ in $P$ by
$\quotep{\procn{x}|\procn{x}}$.

Also, we will avail ourselves of the notation $x^{L}$ and $x^{R}$ to
denote injections of a name into disjoint copies of the name
space. There are numerous ways to accomplish this. One example can be
found in \cite{MeredithR05}. This notation overloads to vectors of
names: $\vec{x}^{\pi} := (x_{i}^{\pi} \; : \; 0 \leq i < |\vec{x}| )$ where $\pi \in \{L,R\}$.

We also use $P^{\Box} := P|\Box$.

In \cite{MeredithR05} an interpretation of the new operator is
given. It turns out that there are several possible interpretations
all enjoying the requisite algebraic properties of the operator (see
\cite{milner91polyadicpi}). We will therefore make liberal use of
$(\nu\; \vec{x})P$.

% subsection the_syntax_and_semantics_of_the_notation_system (end)   

\input{qm2pi.qmops} 

\input{qm2pi.sterngerlach} 

\input{qm2pi.metric} 

% section concurrent_process_calculi (end)

%\input{qm2pi.proofsketch}

% section proof sketch (end)

%\input{qm2pi.slviaknots} 

% section spatial logic via knots (end)

\input{qm2pi.conclusion}

% section conclusion (end)

%\input{qm2pi.dtcodes} 

% section wiring algorithm (end)

\input{qm2pi.ack} 

% section acknowledgments (end)

\newpage


\bibliographystyle{plain}   
\bibliography{../../biblios/main.bib}

\input{qm2pi.rhodetails}

\end{document}

 

% subsection basic_interpretation (end)

%\input{qm2pi.rho.presentation} 
\subsection{The syntax and semantics of the notation system}\label{sub:the_syntax_and_semantics_of_the_notation_system} % (fold)

We now summarize a technical presentation of the calculus that
embodies our theory of dynamics. The typical presentation of such a
calculus follows the style of giving generators and relations on
them. The grammar, below, describing term constructors, freely
generates the set of processes, $\Proc$. This set is then quotiented
by a relation known as structural congruence and it is over this set
that the notion of dynamics is expressed. This presentation is
essentially that of \cite{MeredithR05} with the addition of
polyadicity and summation. For readability we have relegated some of
the technical subtleties to an appendix.

\subsubsection{Process grammar}\label{subsub:process_grammar}

\begin{mathpar}
  \inferrule* [lab=synchronization] {} {{M} \bc \pzero \;|\; x?F \;|\; x!C }
  \and
  \inferrule* [lab=abstraction] {} {{F} \bc (x)P}
  \and
  \inferrule* [lab=concretion] {} {{C} \bc \langle Q \rangle}
  \and
  \inferrule* [lab=process] {} {{P,Q} \bc M \;| \;P|Q \;|\; @{x}}
  \and
  \inferrule* [lab=name] {} {{x} \bc \quotep{P}}
\end{mathpar} 

Note that $\vec{x}$ (resp. $\vec{P}$) denotes a vector of names
(resp. processes) of length $|\vec{x}|$ (resp. $|\vec{P}|$). We adopt
the following useful abbreviations.

\begin{mathpar}
   x?(\vec{y}).P := x.(\vec{y})P \and  x\clift{\vec{P}} := x.\clift{\vec{P}}
   \and x!(y) := \lift{x}{\dropn{y}}
   \and \Pi_{i=0}^{n-1}P_i := P_0 | \ldots | P_{n-1}
\end{mathpar}

\subsubsection{Structural congruence}

\paragraph{Free and bound names and alpha-equivalence.} At the
core of structural equivalence is alpha-equivalence which identifies
process that are the same up to a change of variable. Formally, we
recognize the distinction between free and bound names. The free names
of a process, $\freenames{P}$, may be calculated recursively as
follows:

\begin{mathpar}
\freenames{\pzero} := \emptyset
  \and \\
  \freenames{x?(y).P} := \{ x \} \cup (\freenames{P} \setminus \{ y \})
  \and 
  \freenames{x!\langle P \rangle} := \{ x \} \cup \{ P \} 
  \and \\
  \freenames{P|Q} := \freenames{P} \cup \freenames{Q}
  \and \\
  \freenames{@{x}} := \{ x \}
\end{mathpar}

$\pi$
$\quotep{\pi}$

$\freenames{-} : \pi \to \mathcal{P}(\quotep{\pi})$

\begin{eqnarray*}
  \freenames{\pzero} & := & \emptyset \\
  \freenames{x?(y).P} & := & \{ x \} \cup (\freenames{P} \setminus \{ y \}) \\
  \freenames{x!\langle P \rangle} & := & \{ x \} \cup \{ P \} \\
  \freenames{P|Q} & := & \freenames{P} \cup \freenames{Q} \\
  \freenames{\dropn{x}} & := & \{ x \}
\end{eqnarray*}

The bound names of a process, $\boundnames{P}$, are those names occurring in $P$
that are not free. For example, in $x?(y).0$, the name $x$ is free, while $y$ is bound.

\begin{mathpar}
  \inferrule* [lab=monoidal-laws] {} { P|Q \equiv Q|P \and P|0 \equiv P \and P|(Q|R) \equiv (P|Q)|R }
\end{mathpar}

\begin{mathpar}
  \inferrule* [lab=alpha-equivalence] {} { (x)P \equiv (y)P\{y/x\} \and y \not\in \freenames{P} }
\end{mathpar}

\begin{definition}
Then two processes, $P,Q$, are alpha-equivalent if $P = Q\{\vec{y}/\vec{x}\}$ for
some $\vec{x} \in \boundnames{Q},\vec{y} \in \boundnames{P}$, where $Q\{\vec{y}/\vec{x}\}$
denotes the capture-avoiding substitution of $\vec{y}$ for $\vec{x}$ in $Q$.
\end{definition}

\begin{definition}
  The {\em structural congruence} \cite{SangiorgiWalker} , $\equiv$,
  between processes is the least congruence containing
  alpha-equivalence, satisfying the abelian monoid laws
  (associativity, commutativity and $\pzero$ as identity) for parallel
  composition $|$ and for summation $+$.
\end{definition}

\subsection{Name equivalence}

We take name equivalence, written $\nameeq$, to be the smallest
equivalence relation generated by the following rules.

\begin{mathpar}
\inferrule*[lab=Quote-drop]
{ }
{ \quotep{@{x}} \nameeq x }

\inferrule*[lab=Struct-equiv]
{ P \scong Q }
{ \quotep{P} \nameeq \quotep{Q} }
\end{mathpar}

The astute reader will have noticed that the mutual recursion of names
and processes imposes a mutual recursion on alpha-equivalence and
structural equivalence via name-equivalence. Fortunately, all of this
works out pleasantly and we may calculate in the natural way, free of
concern. The reader interested in the details is referred to the
appendix \ref{appendix:rho_details}.

\subsection{Substitution}

We use $\Proc$ for the set of processes, $\QProc$ for the set of
names, and $\id{\{}\vec{y} / \vec{x} \id{\}}$ to denote partial maps,
$s : \QProc \rightarrow \QProc$. A map, $s$ lifts, uniquely, to a map
on process terms, $\widehat{s} : \Proc \rightarrow \Proc$ by the
following equations.

\begin{mathpar}
  (0) \psubstp{Q}{P} := 0 \\
  (R \juxtap S) \psubstp{Q}{P}
  :=    
  (R)\psubstp{Q}{P} \juxtap (S) \psubstp{Q}{P} \\
  (x?(y).R) \psubstp{Q}{P}    
  :=    
  (x)\substp{Q}{P} (z)\concat( (R \psubstn{z}{y}) \psubstp{Q}{P} ) \\
  (\lift{x}{R}) \psubstp{Q}{P}  
  :=
  \lift{(x)\substp{Q}{P}}{ R \psubstp{Q}{P} } \\
%   (\dropn{x})  \psubstp{Q}{P}       
%   := 
%   \left\{ 
%     \begin{array}{ccc} 
%       \dropn{\quotep{Q}} & & x \nameeq \quotep{P} \\
%       \dropn{x} & & otherwise \\
%     \end{array}
%   \right. 
  (\dropn{x})  \psubstp{Q}{P}       
  := 
  \left\{ 
    \begin{array}{ccc} 
      Q & & x \nameeq \quotep{P} \\
      \dropn{x} & & otherwise \\
    \end{array}
  \right.
\end{mathpar}
 

where

\begin{eqnarray}
  (x)\id{\{} \lpquote Q \rpquote / \lpquote P \rpquote \id{\}}            = 
  \left\{ 
    \begin{array}{ccc}
      \lpquote Q \rpquote & & x \nameeq \lpquote P \rpquote \\
      x & & otherwise \\
    \end{array}
  \right. \nonumber
\end{eqnarray}

and $z$ is chosen distinct from $\quotep{P}$, $\quotep{Q}$, the free
names in $Q$, and all the names in $R$. Our $\alpha$-equivalence will
be built in the standard way from this substitution.

\begin{remark}\label{rem:no_self_referential_names}
  One consequence of these definitions is that $\forall P. \quotep{P}
  \not\in \freenames{P}$.
\end{remark}

\subsection{ Dynamic quote: an example }

Anticipating something of what's to come, consider applying the
substitution, $\widehat{\id{\{}u / z \id{\}}}$, to the following pair
of processes, $\lift{w}{y!(z)}$ and $w[ \lpquote y!(z) \rpquote ]$.

\begin{eqnarray}
	\lift{w}{y!(z)}\widehat{\id{\{}u / z \id{\}}}
		& = &
		\lift{w}{y!(u)} \nonumber\\
	w[ \lpquote y!(z) \rpquote ] \widehat{ \id{\{}u / z \id{\}} }
		& = &
		w[ \lpquote y!(z) \rpquote ] \nonumber
\end{eqnarray}

Because the body of the process between quotes is impervious to
substitution, we get radically different answers. In fact, by
examining the first process in an input context,
e.g. $x?(z).\lift{w}{y!(z)}$, we see that the process under the lift
operator may be shaped by prefixed inputs binding a name inside it. In
this sense, the lift operator will be seen as a way to dynamically
construct processes before reifying them as names.

Finally equipped with these standard features we can present the
dynamics of the calculus.

\subsubsection{Operational semantics} 

Finally, we introduce the computational dynamics. What marks these
algebras as distinct from other more traditionally studied algebraic
structures, e.g. vector spaces or polynomial rings, is the manner in
which dynamics is captured. In traditional structures, dynamics is typically
expressed through morphisms between such structures, as in linear maps
between vector spaces or morphisms between rings. In algebras
associated with the semantics of computation, the dynamics is
expressed as part of the algebraic structure itself, through a
reduction reduction relation typically denoted by $\red$. Below, we
give a recursive presentation of this relation for the calculus used
in the encoding.

$\red \subseteq \pi \times \pi$
$\red : \pi \to \mathcal{P}(\pi)$

\begin{mathpar}
  \inferrule* [lab=Comm] { \textsf{match}( x_{src}, x_{trgt} ) } { x_{trgt}?(y)P \; | \; x_{src}!\langle {Q} \rangle \red P\{\quotep{Q}/y}\} }
  \and \\
  \inferrule* [lab=Par] {{P} \red {P}'} {{{P} | {Q}} \red {{P}' | {Q}}}
  \and
  \inferrule* [lab=Equiv]{{{P} \scong {P}'} \andalso {{P}' \red {Q}'} \andalso {{Q}' \scong {Q}}}{{P} \red {Q}}
\end{mathpar}

\begin{eqnarray*}
  match_{\equiv} (\quotep{P},\quotep{Q}) & := & P \equiv Q \\
  match_{\dagger}(\quotep{P},\quotep{Q}) & := & \forall R. P|Q \red^{*} R => R \red^{*} 0 \\
  match_{K}(\quotep{P},\quotep{Q}) & := & K \mbox{ for some context } K
\end{eqnarray*}

$u?(x)P | u!\langle Q \rangle \red P\{\quotep{Q}/x\}$

%We write $\wred$ for $\red^*$, and $P\red$ if $\exists Q $ such that $ P \red Q$.
We write $P\red$ if $\exists Q $ such that $ P \red Q$ and $P\not\red$, otherwise.

\section{Replication}

As mentioned before, it is known that replication (and hence
recursion) can be implemented in a higher-order process algebra
\cite{SangiorgiWalker}. As our first example of calculation with the
machinery thus far presented we give the construction explicitly in
the {\rhoc}.

\begin{eqnarray}
	D_{x} & := & \prefix{x}{y}{(\binpar{\outputp{x}{y}}{@{y}})} \nonumber\\
	\bangp_{x}{P} & := & \binpar{{x}!\langle{\binpar{D_{x}}{P}}\rangle}{D_{x}} \nonumber
\end{eqnarray}

\begin{eqnarray}
	\bangp_{x}{P} & & \nonumber\\
	=
	& {x}!\langle{(\prefix{x}{y}{(\outputp{x}{y} | @{y})) | P}}\rangle 
	      | \prefix{x}{y}{(\outputp{x}{y} | @{y})} & \nonumber\\
	\red
	& (\outputp{x}{y} | @{y})\substn{\quotep{(\prefix{x}{y}{(@{y} | \outputp{x}{y})) | P}}}{y} & \nonumber\\
	=
	& \outputp{x}{\quotep{(\prefix{x}{y}{(\outputp{x}{y} | @{y})) | P}}}
	  | {(\prefix{x}{y}{(\outputp{x}{y} | @{y})) | P}} & \nonumber\\
	\red
	& \ldots & \nonumber\\
	\red^*
	& P | P | \ldots & \nonumber
\end{eqnarray}

Of course, this encoding, as an implementation, runs away, unfolding
$\bangp{P}$ eagerly. A lazier and more implementable replication
operator, restricted to input-guarded processes, may be obtained as follows.

\begin{eqnarray}
\bangp{\prefix{u}{v}{P}} 
	:= 
	\binpar{\lift{x}{\prefix{u}{v}{(\binpar{D(x)}{P})}}}{D(x)} \nonumber
\end{eqnarray}

\begin{remark}
  Note that the lazier definition still does not deal with summation
  or mixed summation (i.e. sums over input and output). The reader is
  invited to construct definitions of replication that deal with these
  features. 

  Further, the definitions are parameterized in a name, $x$. Can you,
  gentle reader, make a definition that eliminates this parameter and
  guarantees no accidental interaction between the replication
  machinery and the process being replicated -- i.e. no accidental
  sharing of names used by the process to get its work done and the
  name(s) used by the replication to effect copying. This latter
  revision of the definition of replication is crucial to obtaining
  the expected identity $!!P \sim !P$.
\end{remark}

\begin{remark}\label{rem:paradoxical_combinator}
  The reader familiar with the lambda calculus will have noticed the
  similarity between $D$ and the paradoxical combinator.

  [Ed. note: the existence of this seems to suggest we have to be more
  restrictive on the set of processes and names we admit if we are to
  support no-cloning.]
\end{remark}

\subsubsection{Bisimulation}

The computational dynamics gives rise to another kind of equivalence,
the equivalence of computational behavior. As previously mentioned
this is typically captured \emph{via} some form of bisimulation.

% The notion we use in this paper is weak barbed bisimulation
% \cite{milner91polyadicpi}.

The notion we use in this paper is derived from weak barbed
bisimulation \cite{milner91polyadicpi}. 

\begin{definition}
An \emph{observation relation}, $\downarrow_{\mathcal N}$, over a set
of names, $\mathcal N$, is the smallest relation satisfying the rules
below.

\infrule[Out-barb]{y \in {\mathcal N}, \; x \nameeq y}
		  {\outputp{x}{v} \downarrow_{\mathcal N} x}
\infrule[Par-barb]{\mbox{$P\downarrow_{\mathcal N} x$ or $Q\downarrow_{\mathcal N} x$}}
		  {\binpar{P}{Q} \downarrow_{\mathcal N} x}

We write $P \Downarrow_{\mathcal N} x$ if there is $Q$ such that 
$P \wred Q$ and $Q \downarrow_{\mathcal N} x$.
\end{definition}

\begin{definition}
%\label{def.bbisim}
An  ${\mathcal N}$-\emph{barbed bisimulation} over a set of names, ${\mathcal N}$, is a symmetric binary relation 
${\mathcal S}_{\mathcal N}$ between agents such that $P\rel{S}_{\mathcal N}Q$ implies:
\begin{enumerate}
\item If $P \red P'$ then $Q \wred Q'$ and $P'\rel{S}_{\mathcal N} Q'$.
\item If $P\downarrow_{\mathcal N} x$, then $Q\Downarrow_{\mathcal N} x$.
\end{enumerate}
$P$ is ${\mathcal N}$-barbed bisimilar to $Q$, written
$P \wbbisim_{\mathcal N} Q$, if $P \rel{S}_{\mathcal N} Q$ for some ${\mathcal N}$-barbed bisimulation ${\mathcal S}_{\mathcal N}$.
\end{definition}

$\mathcal{R} \subseteq \pi \times \pi$

$P \mathcal{R} Q => \forall P'. P \red P' \Rightarrow \exists Q'. Q \red Q', P' \mathcal{R} Q'$

$P \vdash x \Rightarrow Q \vdash x$

\begin{mathpar}
  \inferrule*[lab=Out-barb]{x \nameeq y}{{y}!\langle{Q}\rangle \vdash x}
  \and
  \inferrule*[lab=Par-barb]{\mbox{$P\vdash x$ or $Q\vdash x$}}{\binpar{P}{Q} \vdash x}
\end{mathpar}

\subsubsection{Contexts}

One of the principle advantages of computational calculi like the
$\pi$-calculus is a well-defined notion of context,
contextual-equivalence and a correlation between
contextual-equivalence and notions of bisimulation. The notion of
context allows the decomposition of a process into (sub-)process and
its syntactic environment, its context. Thus, a context may be
thought of as a process with a ``hole'' (written $\Box$) in it. The
application of a context $M$ to a process $P$, written $M[P]$, is
tantamount to filling the hole in $M$ with $P$. In this paper we do
not need the full weight of this theory, but do make use of the notion
of context in the proof the main theorem. 

\begin{mathpar}
  \inferrule* [lab=summation] {} {{M_{M},M_{N}} \bc \Box \;|\; x.M_{A} \;|\; M_{M}+M_{N}}
  \and
  \inferrule* [lab=agent] {} {{M_{A}} \bc (\vec{x})M_{P} \;| \; \clift{P_0,\ldots,M_{P},\ldots,P_N}}
  \and \\
  \inferrule* [lab=process] {} {{M_{P}} \bc M_{N} \;| \;P|M_{P} }
\end{mathpar} 

\begin{mathpar}
  \inferrule* [lab=sychronization] {} {M_{N} \bc \Box \;|\; x?M_{F} \;|\; x!M_{C}}
  \and
  \inferrule* [lab=abstraction] {} {{M_{F}} \bc (x)M_{P} }
  \and
  \inferrule* [lab=concretion] {} {{M_{C}} \bc \langle M_{P} \rangle }
  \and \\
  \inferrule* [lab=process] {} {{M_{P}} \bc M_{N} \;| \;P|M_{P} }
\end{mathpar}

\begin{definition}[contextual application] Given a context $M$, and
  process $P$, we define the \emph{contextual application}, $M[P] :=
  M\{P/\Box\}$. That is, the contextual application of M to P is the
  substitution of $P$ for $\Box$ in $M$.
\end{definition}

$\meaningof{-} : L \to \mathcal{P}(\pi)$

\begin{mathpar}
  \inferrule* [lab=collection] {} {\meaningof{true} = \pi, \and \meaningof{~E} = \pi \setminus \meaningof{E}, \and \meaningof{E_{1} \& E_{2}} = \meaningof{E_{1}} \cap \meaningof{E_{2}}}
\end{mathpar}

\begin{mathpar}
  \inferrule* [lab=structure] {} {\meaningof{0} = \{ P \in \pi | P \equiv 0 \}, \and \\ \meaningof{E_1 | E_2} = \{ P \in \pi | P \equiv P_{1} | P_{2}, P_{1} \in \meaningof{E_{1}}, P_{2} \in \meaningof{E_2}\} }
\end{mathpar}

\begin{mathpar}
 \inferrule* [lab=behavior] {} {\meaningof{\langle a?b \rangle E} = \{ P \in \pi | P \equiv Q | u?(y)P', \\ \and \\\\ \and \\ \;\;\; u \in \meaningof{a}, \forall z.P'\{z/y\} \in \meaningof{E\{z/b\}}\}, \and \\ \meaningof{a!E} = \{ P \in \pi | P \equiv Q | x!\langle P' \rangle, x \in \meaningof{a} P' \in \meaningof{E}\} }
\end{mathpar}

\begin{mathpar}
 \inferrule* [lab=nominal] {} {\meaningof{\quotep{E}} = \{ \quotep{P} \in \quotep{\pi} | P \in \meaningof{E} \}, \and \meaningof{\quotep{P}} = \{ \quotep{Q} \in \quotep{\pi} | P \equiv Q \} \and \\ \meaningof{@\quotep{E}} = \{ P \in \pi | P \equiv @x, x \in \meaningof{E} \}}
\end{mathpar}

\begin{eqnarray*}
  \\
  \meaningof{-} : TS \to ST
\end{eqnarray*}

\begin{eqnarray*}
  \\
  L : TS \to ST
\end{eqnarray*}

\begin{eqnarray*}
  \\
  P \models E \iff P \in \meaningof{E}
\end{eqnarray*}

\begin{eqnarray*}
  P \approx_{L} Q \iff \forall E \in L. P \models E \iff Q \models E
\end{eqnarray*}

\begin{eqnarray*}
  P \approx_{K} Q
\end{eqnarray*}

\begin{eqnarray*}
  P \approx Q
\end{eqnarray*}

$\approx_{K} = \approx = \approx_{L}$

\subsubsection{Contextual duality}

Note that contexts extend the quotation operation to a family of
operations from processes to names. Given a context, $M$, we can
define a \emph{nominal context}, $\quotep{M}$ by $\quotep{M}[P] :=
\quotep{M[P]}$. To foreshadow what is to come we observe that these
operations enjoy a duality with processes very much like the duality
between vectors and maps from vectors to scalars.

Further, because the calculus is essentially higher-order, we have a
correspondence between contexts and processes. More specifically,
given a name $x$ and a context $M$ we can construct $M^{*}_{x}$ such
that 

\begin{mathpar}
  M^{*}_{x} | \lift{x}{P} \red M[P]
\end{mathpar}

namely,

\begin{mathpar}
  M^{*}_{x} := x?(u).M[\dropn{u}]
\end{mathpar}

The dependence of $M^{*}_{x}$ on a name makes it an abstraction, 

\begin{mathpar}
  M^{*} := (x)x?(u).M[\dropn{u}]
\end{mathpar}

\subsection{Additional notation}

It will sometimes be convenient to denote the process a name
quotes. We already have the notation $x = \quotep{P}$, but it will be
convenient to introduce an alternate notation, $\procn{x}$, when we
want to emphasize the connection to the use of the name. Note that, by
virtue of name equivalence, $\quotep{\procn{x}} \nameeq x$; so, the
notation is consistent with previous definitions.

Further, because names have structure it is possible to effect
substitutions on the basis of that structure. This means we need to
upgrade our notation for substitutions, which we accomplish by
adapting comprehension notation. Thus,

\begin{mathpar}
  P\{ y / x : x \in S \}
\end{mathpar}

is interpreted to mean the process derived from P by replacing (in a
capture-avoiding manner) each occurrence of $x$ in $S$ by $y$. For example,

\begin{mathpar}
  P\{ \quotep{\procn{x}|\procn{x}} / x : x \in \freenames{P} \}
\end{mathpar}

will replace each (occurrence) of a free name $x$ in $P$ by
$\quotep{\procn{x}|\procn{x}}$.

Also, we will avail ourselves of the notation $x^{L}$ and $x^{R}$ to
denote injections of a name into disjoint copies of the name
space. There are numerous ways to accomplish this. One example can be
found in \cite{MeredithR05}. This notation overloads to vectors of
names: $\vec{x}^{\pi} := (x_{i}^{\pi} \; : \; 0 \leq i < |\vec{x}| )$ where $\pi \in \{L,R\}$.

We also use $P^{\Box} := P|\Box$.

In \cite{MeredithR05} an interpretation of the new operator is
given. It turns out that there are several possible interpretations
all enjoying the requisite algebraic properties of the operator (see
\cite{milner91polyadicpi}). We will therefore make liberal use of
$(\nu\; \vec{x})P$.

% subsection the_syntax_and_semantics_of_the_notation_system (end)   

\section{Interpretation of QM}
\subsection{Supporting definitions}
\subsubsection{Multiplication}
\begin{mathpar}
  \quotep{Q} \cdot \quotep{R} := \quotep{Q|R}
  \and \\
  \quotep{Q} \cdot P := P\{ \quotep{Q|R} / \quotep{R} : \quotep{R} \in \freenames{P} \}
\end{mathpar}

\paragraph{Discussion}
The first line needs little explanation. The second line says that
each free name of the process is replaced with the multiplication of
that name by the scalar. Multiplication of a scalar (name) by a state
(process) results in a process all the names of which have been `moved
over' by parallel composition with the process the scalar
quotes. There is a subtlety that the bound names have to be
manipulated so that multiplied names aren't accidentally
captured. There are many ways to achieve this.

\begin{remark}\label{rem:multiplication_identities}
  The reader is invited to verify that for all $x,y,z \in \QProc$ and $P \in \Proc$
  \begin{mathpar}
    x \cdot \quotep{0} \equiv x 
    \and
    x \cdot y \equiv y \cdot x
    \and
    x \cdot (y \cdot z) \equiv (x \cdot y) \cdot z
    \and \\
    \quotep{0} \cdot P \equiv P
    \and \\
    x \cdot (y \cdot P) \equiv (x \cdot y) \cdot P
    \and \\
    x \cdot (P|Q) \equiv (x \cdot P) | (x \cdot Q)
    \and \\    
  \end{mathpar}
\end{remark}

\subsubsection{Tensor product}

We define a tensor product on processes by structural induction.

\paragraph{Tensor of sums} First note that all summations, including
$\pzero$ and sequence, can be written $\Sigma_{i} x_{i}.A_{i} +
\Sigma_{j} x_{j}.C_{j}$, where we have grouped input-guarded processes
together and output-guarded processes together.

Thus, we can define the tensor product of two summations, $N_{1}\otimes N_{2}$, where

\begin{mathpar}
  N_{1} := \Sigma_{i} x_{i}.A_{i} + \Sigma_{j} x_{j}.C_{j}
  \and
  N_{2} := \Sigma_{i'} y_{i'}.B_{i'} + \Sigma_{j'} y_{j'}.D_{j'} 
\end{mathpar}

as follows.

\begin{mathpar}
  \Sigma_{i} x_{i}.A_{i} + \Sigma_{j} x_{j}.C_{j} \otimes \Sigma_{i'}
  y_{i'}.B_{i'} + \Sigma_{j'} y_{j'}.D_{j'} 
  \and \\
  := \; \Sigma_{i} \Sigma_{i'} \quotep{\stackrel{\vee}{x_{i}}| \stackrel{\vee}{y_{i'}}}.(A_{i}\otimes B_{i'}) \; | \; \Sigma_{i'} \Sigma_{i} \quotep{\stackrel{\vee}{y_{i'}}|\stackrel{\vee}{x_{i}}}.(B_{i'}\otimes A_{i})
  \and
  \;\; | \;\; \Sigma_{j} \Sigma_{j'} \quotep{\stackrel{\vee}{x_{j}}|\stackrel{\vee}{y_{j'}}}.(A_{j}\otimes B_{j'}) \; | \; \Sigma_{j'} \Sigma_{j} \quotep{\stackrel{\vee}{y_{j'}}|\stackrel{\vee}{x_{j}}}.(B_{j'}\otimes A_{j})
\end{mathpar}

\begin{remark}
  Do we need to $x^{L}$ and $y^{R}$ for this construction as well?
\end{remark}

\paragraph{Tensor of parallel compositions} Next, we distribute tensor
over par.

\begin{mathpar}
  P_{1}|P_{2} \otimes Q_{1}|Q_{2} := (P_{1} \otimes Q_{1}) | (P_{1}
  \otimes Q_{2}) | (P_{2} \otimes Q_{1}) | (P_{2} \otimes Q_{2})
\end{mathpar}

\paragraph{Tensor with dropped names} We treat tensor of a
process with a dropped name as parallel composition.

\begin{mathpar}
  P \otimes \dropn{x} := P | \dropn{x}
\end{mathpar}

\paragraph{Tensor of agents}

Finally, we need to define tensor on agents. Note that the definition
of tensor on normal products only tensors inputs with inputs and
outputs with outputs. Thus, we only have to define the operation on
``homogeneous'' pairings.

\begin{mathpar}
  (\vec{x})P \otimes (\vec{y})Q
  \and \\
  := (x_{0}^{L}|y_{0}^{R},\ldots,x_{0}^{L}|y_{n}^{R},\ldots,x_{m}^{L}|y_{0}^{R},\ldots,x_{m}^{L}|y_{n}^R)(P\{ \vec{x}^{L}/\vec{x}\} \otimes Q \{ \vec{y}^{R}/\vec{y}\})
  \and \\
  \clift{\vec{P}} \otimes \clift{\vec{Q}}
  \and \\
  := \clift{P_{0}\otimes Q_{0},\ldots,P_{0}\otimes Q_{n},\ldots,P_{m}\otimes Q_{0},\ldots,P_{m}\otimes Q_{n}}
\end{mathpar}

\begin{remark}
  Observe that arities of tensored abstractions matches arities of
  tensored concretions if the original arities matched. Note also that
  the length of the arities corresponds to the increase in dimension
  we see in ordinary vector space tensor product.
\end{remark}

\begin{remark}
  Operationally, this definition distributes the tensor down to
  components ``linked'' by summation. Tensor over summation is
  intriguing in that it mixes names. Moreover, as a consequence of the
  way it mixes names we have the identities for all $x \in \QProc$ and
  $P,Q \in \Proc$

  \begin{mathpar}
    (x \cdot P) \otimes Q \equiv x \cdot (P \otimes Q) \equiv P \otimes (x \cdot Q)
    \and
    P \otimes \pzero \equiv P
  \end{mathpar}

  that the reader is invited to verify.
\end{remark}

\subsubsection{Annihilation}
\begin{mathpar}
  P^{\perp} := \{ Q | \forall R. P|Q \red^{*} R \Rightarrow R \red^{*} \pzero \}
  \and \\
  P^{\underline{\perp}} := \Sigma_{Q \in P^{\perp}} \quotep{Q}?(y).(\dropn{y}|Q) | \Sigma_{Q \in P^{\perp}} \quotep{Q}\clift{\Box}
\end{mathpar}

\paragraph{Discussion} The reader will note that $P^{\perp}$ is a
\emph{set} of processes, while $P^{\underline{\perp}}$ is a
\emph{context}. We call the set $P^{\perp}$ the \emph{annihilators} of
$P$. The parallel composition of a process in the annihilators of $P$
with $P$ will result in a process, the state space of which has all
paths eventually leading to $\pzero$. Execution may endure loops; but
under reasonable conditions of fairness (naturally guaranteed under
most notions of bisimulation) such a composite process cannot get
stuck in such a loop and will, eventually pop out and terminate.

The context $P^{\underline{\perp}}$ is ready and willing to ``take the
$P$ out of'' the process to which it is applied. It will effectively
transmit the code of the process to which it is applied to one of the
annihilators and run the process against it.

\subsubsection{Evaluation}
We fix $M$ a domain of fully abstract interpretation with an equality
coincident with bisimulation. We take $\meaningof{\cdot} : \Proc \to
M$ to be the map interpreting processes and $\nmeaningof{\cdot} : \M
\to Proc$ to be the map running the other way. Then we define

\begin{mathpar}
  \int P := \nmeaningof{\meaningof{P}}
\end{mathpar}

\paragraph{Discussion}
There are many fully abstract interpretations of Milner's
$\pi$-calculus. Any of them can be used as a basis for interpreting
the reflective calculus here. Equipped with such a domain it is
largely a matter of grinding through to check that the Yoneda
construction for the normalization-by-evaluation program can be
extended to this setting.

\begin{remark}
  The reader is invited to verify that $\int (P^{\underline{\perp}}[P]) = 0$.
\end{remark}

\subsection{Quantum mechanics}

Table \ref{tbl:core_qm_op_defns} gives the core operational definitions

\begin{table}[htp]\label{tbl:core_qm_op_defns}
  \center{
    \fbox{
      \begin{tabular}{c|c}
        quantum mechanics & process calculus \\
        \hline
        scalar & $x := \quotep{P}$ \\
        state vector & $\state{P} := P$ \\
        dual & $\state{P}^{*} := \event{P^{\underline{\perp}}} := \quotep{P^{\underline{\perp}}}[-]$ \\
        matrix & $ \Sigma_{\alpha} \state{P_{\alpha}}x_{\alpha}\event{Q_{\alpha}}$ \\
        vector addition & $\state{P} + \state{Q} := \state{P | Q}$ \\
        tensor product & $\state{P} \otimes \state{Q} := \state{P \otimes Q}$ \\
        inner product & $\innerprod{P}{Q} := \quotep{\int P^{\underline{\perp}}[Q]}$ \\
      \end{tabular}
    }
  }
  \caption{QM - operational definitions}
\end{table}

where

\begin{mathpar}
  \prmatrix{P}{Q} := \fprmatrix{P}{\quotep{\pzero}}{Q}
  \and
  \fprmatrix{P}{x}{Q} := (\state{P},x,\event{Q})
  \and
  (\fprmatrix{P}{x}{Q})(\state{R}) := x \cdot \innerprod{Q}{R} \cdot \state{P}
  \and
  (\fprmatrix{P}{x}{Q})(\event{R}) := x \cdot \innerprod{R}{P} \cdot \event{Q}
\end{mathpar}

\paragraph{Discussion}
As promised: vectors (aka states) are represented as processes; duals
as contextual duals; inner product definition should be compared with
standard inner product definition for ....

\begin{remark}
  Assuming $\int (P^{\underline{\perp}}[P]) = 0$, the reader is
  invited to verify that $(\fprmatrix{P}{x}{P})(\state{P}) = x \cdot \state{P}$.
\end{remark}

\begin{remark}
  The reader is invited to verify that $\innerprod{P}{Q}$ could
  equally well have been written $\quotep{\int \stackrel{\vee}{x}}$
  where $x = \event{P^{\underline{\perp}}}(Q)$.

  One of the motivations for this remark is that there is another way
  to factor these operations. We could package up evaluation in the dual:

  \begin{mathpar}
    \state{P}^{*} := \event{\int P^{\underline{\perp}}} := \quotep{\int P^{\underline{\perp}}}[-]
  \end{mathpar}

  and then have inner product defined by
  
  \begin{mathpar}
    \innerprod{P}{Q} := \event{P}(Q)
  \end{mathpar}

  Hopefully, experience with the calculations will provide guidance on
  the best factoring.
\end{remark}

\begin{remark}
  Assuming $\int (P^{\underline{\perp}}[P]) = 0$, the reader is
  invited to verify that $\forall P,Q. (\prmatrix{0}{Q})(\state{0}) =
  \state{0}$ and dually $(\prmatrix{P}{0})(\event{0}) = \event{0}$.
\end{remark}

\begin{remark}
  i'm a little worried that i don't (yet) have proper support for
  complex conjugacy. But, the observation above may give us a
  clue. According to Abramsky, it must be the case that the scalars
  are iso to the homset of the identity for the tensor -- which the
  observation above characterizes. 

  For now, we will simply bookmark the notion with $\overline{x}$.
\end{remark}

\subsubsection{Adjointness}

We need to give a definition of $(\cdot)^{\dagger}$ for matrices. The
obvious candidate definition is
\begin{mathpar}
(\Sigma_{\alpha}\fprmatrix{P_{\alpha}}{x_{\alpha}}{Q_{\alpha}})^{\dagger}
= \Sigma_{\alpha}\fprmatrix{(Q_{\alpha}^{\underline{\perp}})^{*}}{\overline{x}_{\alpha}}{P_{\alpha}^{\underline{\perp}}} 
\end{mathpar}

But, $(Q_{\alpha}^{\underline{\perp}})^{*}$ requires a name along
which to communicate the process to achieve the context application.

\subsubsection{Basis for a basis}
If processes label states and ``addition'' of states (a.k.a. vector
addition) is interpreted as parallel composition, what corresponds to
notions of linear independence and basis? Here, we recall that Yoshida
has developed a set of \emph{combinators} for an asynchronous verison
of Milner's $\pi$-calculus. These are a finite set of processes such
any process can be expressed as parallel composition of these
combinators together with liberal uses of the new operator and
replication. We can simply give a translation of these into the
present calculus and have reasonable expectation that the property
carries over. That is, that the resultant set allows to express all
processes via parallel composition. Note, however, that there is no
new operator or replication in this calculus. As a result, we expect
that the corresponding set is actually infinite. That is, we expect
that the space is actually infinite dimensional.

\begin{remark}
  The attentive reader may be a bit concerned. Certainly, the
  collection $S$, $K$ and $I$ is a finite set of
  combinators. Shouldn't we expect to see a finite set of combinators
  for an effectively equivalent system? i am very sympathetic to this
  critique and feel it warrants full attention. On the other hand, i
  also have in mind the following analogy. The natural numbers, as a
  monoid under addition, has exactly $1$ generator, while the natural
  numbers, as a monoid under multiplication, has countably many
  generators (the primes). We observe that the application of the
  lambda calculus is much less resource sensitive than the parallel
  composition of the $\pi$-calculus. Could it be the case that we have
  an analogy of the form
  
  \begin{mathpar}
    m + n : MN :: m*n : M|N
  \end{mathpar}

  giving a similar blow up in the set of ``primes''?  This is such a
  wonderful thought that, even if it's not true, i think it's worth
  writing down.
\end{remark}
 

\documentclass[12pt]{llncs}
%\documentclass{jktr}

\usepackage[pdftex]{hyperref}                   
\usepackage {listings}
\usepackage {mathpartir}
\usepackage{bcprules}
%\usepackage{listings}
                       
\usepackage{graphicx} 
%\usepackage[margins=2.5cm,nohead,nofoot]{geometry}
%\usepackage{geometry}
\usepackage{amsfonts}
\usepackage{amstext}
\usepackage{latexsym}
\usepackage{amssymb}
\usepackage{color}


%\include{myPreamble}
\include{qm2pi.local} 

%\ifpdf
%\usepackage[pdftex]{graphicx}
%\else
%\usepackage{graphicx}
%\fi

 % \ifpdf
%  \usepackage{pdfsync}
%  \if


%\title{Brief Article}
%\author{David F. Snyder}
%\author{L.G. Meredith}

%\address{Dept. of Math., Texas State University--San Marcos, San Marcos, TX 78666}
       
\pagestyle{empty}


\begin{document}

\lstset{language=[Objective]Caml,frame=shadowbox}

\input{qm2pi.front}

% section front matter (end)

\input{qm2pi.intro} 
 
% section introduction (end)

% \input{qm2pi.knotations} 

% section notation (end)

\input{qm2pi.process.calculi} 

% section concurrent_process_calculi_and_spatial_logics_ (end)
    
%\input{qm2pi.knots2pi} 

%\input{qm2pi.trefoil} 

%\input{qm2pi.mainthm} 

% subsection basic_interpretation (end)

%\input{qm2pi.rho.presentation} 
\subsection{The syntax and semantics of the notation system}\label{sub:the_syntax_and_semantics_of_the_notation_system} % (fold)

We now summarize a technical presentation of the calculus that
embodies our theory of dynamics. The typical presentation of such a
calculus follows the style of giving generators and relations on
them. The grammar, below, describing term constructors, freely
generates the set of processes, $\Proc$. This set is then quotiented
by a relation known as structural congruence and it is over this set
that the notion of dynamics is expressed. This presentation is
essentially that of \cite{MeredithR05} with the addition of
polyadicity and summation. For readability we have relegated some of
the technical subtleties to an appendix.

\subsubsection{Process grammar}\label{subsub:process_grammar}

\begin{mathpar}
  \inferrule* [lab=synchronization] {} {{M} \bc \pzero \;|\; x?F \;|\; x!C }
  \and
  \inferrule* [lab=abstraction] {} {{F} \bc (x)P}
  \and
  \inferrule* [lab=concretion] {} {{C} \bc \langle Q \rangle}
  \and
  \inferrule* [lab=process] {} {{P,Q} \bc M \;| \;P|Q \;|\; @{x}}
  \and
  \inferrule* [lab=name] {} {{x} \bc \quotep{P}}
\end{mathpar} 

Note that $\vec{x}$ (resp. $\vec{P}$) denotes a vector of names
(resp. processes) of length $|\vec{x}|$ (resp. $|\vec{P}|$). We adopt
the following useful abbreviations.

\begin{mathpar}
   x?(\vec{y}).P := x.(\vec{y})P \and  x\clift{\vec{P}} := x.\clift{\vec{P}}
   \and x!(y) := \lift{x}{\dropn{y}}
   \and \Pi_{i=0}^{n-1}P_i := P_0 | \ldots | P_{n-1}
\end{mathpar}

\subsubsection{Structural congruence}

\paragraph{Free and bound names and alpha-equivalence.} At the
core of structural equivalence is alpha-equivalence which identifies
process that are the same up to a change of variable. Formally, we
recognize the distinction between free and bound names. The free names
of a process, $\freenames{P}$, may be calculated recursively as
follows:

\begin{mathpar}
\freenames{\pzero} := \emptyset
  \and \\
  \freenames{x?(y).P} := \{ x \} \cup (\freenames{P} \setminus \{ y \})
  \and 
  \freenames{x!\langle P \rangle} := \{ x \} \cup \{ P \} 
  \and \\
  \freenames{P|Q} := \freenames{P} \cup \freenames{Q}
  \and \\
  \freenames{@{x}} := \{ x \}
\end{mathpar}

$\pi$
$\quotep{\pi}$

$\freenames{-} : \pi \to \mathcal{P}(\quotep{\pi})$

\begin{eqnarray*}
  \freenames{\pzero} & := & \emptyset \\
  \freenames{x?(y).P} & := & \{ x \} \cup (\freenames{P} \setminus \{ y \}) \\
  \freenames{x!\langle P \rangle} & := & \{ x \} \cup \{ P \} \\
  \freenames{P|Q} & := & \freenames{P} \cup \freenames{Q} \\
  \freenames{\dropn{x}} & := & \{ x \}
\end{eqnarray*}

The bound names of a process, $\boundnames{P}$, are those names occurring in $P$
that are not free. For example, in $x?(y).0$, the name $x$ is free, while $y$ is bound.

\begin{mathpar}
  \inferrule* [lab=monoidal-laws] {} { P|Q \equiv Q|P \and P|0 \equiv P \and P|(Q|R) \equiv (P|Q)|R }
\end{mathpar}

\begin{mathpar}
  \inferrule* [lab=alpha-equivalence] {} { (x)P \equiv (y)P\{y/x\} \and y \not\in \freenames{P} }
\end{mathpar}

\begin{definition}
Then two processes, $P,Q$, are alpha-equivalent if $P = Q\{\vec{y}/\vec{x}\}$ for
some $\vec{x} \in \boundnames{Q},\vec{y} \in \boundnames{P}$, where $Q\{\vec{y}/\vec{x}\}$
denotes the capture-avoiding substitution of $\vec{y}$ for $\vec{x}$ in $Q$.
\end{definition}

\begin{definition}
  The {\em structural congruence} \cite{SangiorgiWalker} , $\equiv$,
  between processes is the least congruence containing
  alpha-equivalence, satisfying the abelian monoid laws
  (associativity, commutativity and $\pzero$ as identity) for parallel
  composition $|$ and for summation $+$.
\end{definition}

\subsection{Name equivalence}

We take name equivalence, written $\nameeq$, to be the smallest
equivalence relation generated by the following rules.

\begin{mathpar}
\inferrule*[lab=Quote-drop]
{ }
{ \quotep{@{x}} \nameeq x }

\inferrule*[lab=Struct-equiv]
{ P \scong Q }
{ \quotep{P} \nameeq \quotep{Q} }
\end{mathpar}

The astute reader will have noticed that the mutual recursion of names
and processes imposes a mutual recursion on alpha-equivalence and
structural equivalence via name-equivalence. Fortunately, all of this
works out pleasantly and we may calculate in the natural way, free of
concern. The reader interested in the details is referred to the
appendix \ref{appendix:rho_details}.

\subsection{Substitution}

We use $\Proc$ for the set of processes, $\QProc$ for the set of
names, and $\id{\{}\vec{y} / \vec{x} \id{\}}$ to denote partial maps,
$s : \QProc \rightarrow \QProc$. A map, $s$ lifts, uniquely, to a map
on process terms, $\widehat{s} : \Proc \rightarrow \Proc$ by the
following equations.

\begin{mathpar}
  (0) \psubstp{Q}{P} := 0 \\
  (R \juxtap S) \psubstp{Q}{P}
  :=    
  (R)\psubstp{Q}{P} \juxtap (S) \psubstp{Q}{P} \\
  (x?(y).R) \psubstp{Q}{P}    
  :=    
  (x)\substp{Q}{P} (z)\concat( (R \psubstn{z}{y}) \psubstp{Q}{P} ) \\
  (\lift{x}{R}) \psubstp{Q}{P}  
  :=
  \lift{(x)\substp{Q}{P}}{ R \psubstp{Q}{P} } \\
%   (\dropn{x})  \psubstp{Q}{P}       
%   := 
%   \left\{ 
%     \begin{array}{ccc} 
%       \dropn{\quotep{Q}} & & x \nameeq \quotep{P} \\
%       \dropn{x} & & otherwise \\
%     \end{array}
%   \right. 
  (\dropn{x})  \psubstp{Q}{P}       
  := 
  \left\{ 
    \begin{array}{ccc} 
      Q & & x \nameeq \quotep{P} \\
      \dropn{x} & & otherwise \\
    \end{array}
  \right.
\end{mathpar}
 

where

\begin{eqnarray}
  (x)\id{\{} \lpquote Q \rpquote / \lpquote P \rpquote \id{\}}            = 
  \left\{ 
    \begin{array}{ccc}
      \lpquote Q \rpquote & & x \nameeq \lpquote P \rpquote \\
      x & & otherwise \\
    \end{array}
  \right. \nonumber
\end{eqnarray}

and $z$ is chosen distinct from $\quotep{P}$, $\quotep{Q}$, the free
names in $Q$, and all the names in $R$. Our $\alpha$-equivalence will
be built in the standard way from this substitution.

\begin{remark}\label{rem:no_self_referential_names}
  One consequence of these definitions is that $\forall P. \quotep{P}
  \not\in \freenames{P}$.
\end{remark}

\subsection{ Dynamic quote: an example }

Anticipating something of what's to come, consider applying the
substitution, $\widehat{\id{\{}u / z \id{\}}}$, to the following pair
of processes, $\lift{w}{y!(z)}$ and $w[ \lpquote y!(z) \rpquote ]$.

\begin{eqnarray}
	\lift{w}{y!(z)}\widehat{\id{\{}u / z \id{\}}}
		& = &
		\lift{w}{y!(u)} \nonumber\\
	w[ \lpquote y!(z) \rpquote ] \widehat{ \id{\{}u / z \id{\}} }
		& = &
		w[ \lpquote y!(z) \rpquote ] \nonumber
\end{eqnarray}

Because the body of the process between quotes is impervious to
substitution, we get radically different answers. In fact, by
examining the first process in an input context,
e.g. $x?(z).\lift{w}{y!(z)}$, we see that the process under the lift
operator may be shaped by prefixed inputs binding a name inside it. In
this sense, the lift operator will be seen as a way to dynamically
construct processes before reifying them as names.

Finally equipped with these standard features we can present the
dynamics of the calculus.

\subsubsection{Operational semantics} 

Finally, we introduce the computational dynamics. What marks these
algebras as distinct from other more traditionally studied algebraic
structures, e.g. vector spaces or polynomial rings, is the manner in
which dynamics is captured. In traditional structures, dynamics is typically
expressed through morphisms between such structures, as in linear maps
between vector spaces or morphisms between rings. In algebras
associated with the semantics of computation, the dynamics is
expressed as part of the algebraic structure itself, through a
reduction reduction relation typically denoted by $\red$. Below, we
give a recursive presentation of this relation for the calculus used
in the encoding.

$\red \subseteq \pi \times \pi$
$\red : \pi \to \mathcal{P}(\pi)$

\begin{mathpar}
  \inferrule* [lab=Comm] { \textsf{match}( x_{src}, x_{trgt} ) } { x_{trgt}?(y)P \; | \; x_{src}!\langle {Q} \rangle \red P\{\quotep{Q}/y}\} }
  \and \\
  \inferrule* [lab=Par] {{P} \red {P}'} {{{P} | {Q}} \red {{P}' | {Q}}}
  \and
  \inferrule* [lab=Equiv]{{{P} \scong {P}'} \andalso {{P}' \red {Q}'} \andalso {{Q}' \scong {Q}}}{{P} \red {Q}}
\end{mathpar}

\begin{eqnarray*}
  match_{\equiv} (\quotep{P},\quotep{Q}) & := & P \equiv Q \\
  match_{\dagger}(\quotep{P},\quotep{Q}) & := & \forall R. P|Q \red^{*} R => R \red^{*} 0 \\
  match_{K}(\quotep{P},\quotep{Q}) & := & K \mbox{ for some context } K
\end{eqnarray*}

$u?(x)P | u!\langle Q \rangle \red P\{\quotep{Q}/x\}$

%We write $\wred$ for $\red^*$, and $P\red$ if $\exists Q $ such that $ P \red Q$.
We write $P\red$ if $\exists Q $ such that $ P \red Q$ and $P\not\red$, otherwise.

\section{Replication}

As mentioned before, it is known that replication (and hence
recursion) can be implemented in a higher-order process algebra
\cite{SangiorgiWalker}. As our first example of calculation with the
machinery thus far presented we give the construction explicitly in
the {\rhoc}.

\begin{eqnarray}
	D_{x} & := & \prefix{x}{y}{(\binpar{\outputp{x}{y}}{@{y}})} \nonumber\\
	\bangp_{x}{P} & := & \binpar{{x}!\langle{\binpar{D_{x}}{P}}\rangle}{D_{x}} \nonumber
\end{eqnarray}

\begin{eqnarray}
	\bangp_{x}{P} & & \nonumber\\
	=
	& {x}!\langle{(\prefix{x}{y}{(\outputp{x}{y} | @{y})) | P}}\rangle 
	      | \prefix{x}{y}{(\outputp{x}{y} | @{y})} & \nonumber\\
	\red
	& (\outputp{x}{y} | @{y})\substn{\quotep{(\prefix{x}{y}{(@{y} | \outputp{x}{y})) | P}}}{y} & \nonumber\\
	=
	& \outputp{x}{\quotep{(\prefix{x}{y}{(\outputp{x}{y} | @{y})) | P}}}
	  | {(\prefix{x}{y}{(\outputp{x}{y} | @{y})) | P}} & \nonumber\\
	\red
	& \ldots & \nonumber\\
	\red^*
	& P | P | \ldots & \nonumber
\end{eqnarray}

Of course, this encoding, as an implementation, runs away, unfolding
$\bangp{P}$ eagerly. A lazier and more implementable replication
operator, restricted to input-guarded processes, may be obtained as follows.

\begin{eqnarray}
\bangp{\prefix{u}{v}{P}} 
	:= 
	\binpar{\lift{x}{\prefix{u}{v}{(\binpar{D(x)}{P})}}}{D(x)} \nonumber
\end{eqnarray}

\begin{remark}
  Note that the lazier definition still does not deal with summation
  or mixed summation (i.e. sums over input and output). The reader is
  invited to construct definitions of replication that deal with these
  features. 

  Further, the definitions are parameterized in a name, $x$. Can you,
  gentle reader, make a definition that eliminates this parameter and
  guarantees no accidental interaction between the replication
  machinery and the process being replicated -- i.e. no accidental
  sharing of names used by the process to get its work done and the
  name(s) used by the replication to effect copying. This latter
  revision of the definition of replication is crucial to obtaining
  the expected identity $!!P \sim !P$.
\end{remark}

\begin{remark}\label{rem:paradoxical_combinator}
  The reader familiar with the lambda calculus will have noticed the
  similarity between $D$ and the paradoxical combinator.

  [Ed. note: the existence of this seems to suggest we have to be more
  restrictive on the set of processes and names we admit if we are to
  support no-cloning.]
\end{remark}

\subsubsection{Bisimulation}

The computational dynamics gives rise to another kind of equivalence,
the equivalence of computational behavior. As previously mentioned
this is typically captured \emph{via} some form of bisimulation.

% The notion we use in this paper is weak barbed bisimulation
% \cite{milner91polyadicpi}.

The notion we use in this paper is derived from weak barbed
bisimulation \cite{milner91polyadicpi}. 

\begin{definition}
An \emph{observation relation}, $\downarrow_{\mathcal N}$, over a set
of names, $\mathcal N$, is the smallest relation satisfying the rules
below.

\infrule[Out-barb]{y \in {\mathcal N}, \; x \nameeq y}
		  {\outputp{x}{v} \downarrow_{\mathcal N} x}
\infrule[Par-barb]{\mbox{$P\downarrow_{\mathcal N} x$ or $Q\downarrow_{\mathcal N} x$}}
		  {\binpar{P}{Q} \downarrow_{\mathcal N} x}

We write $P \Downarrow_{\mathcal N} x$ if there is $Q$ such that 
$P \wred Q$ and $Q \downarrow_{\mathcal N} x$.
\end{definition}

\begin{definition}
%\label{def.bbisim}
An  ${\mathcal N}$-\emph{barbed bisimulation} over a set of names, ${\mathcal N}$, is a symmetric binary relation 
${\mathcal S}_{\mathcal N}$ between agents such that $P\rel{S}_{\mathcal N}Q$ implies:
\begin{enumerate}
\item If $P \red P'$ then $Q \wred Q'$ and $P'\rel{S}_{\mathcal N} Q'$.
\item If $P\downarrow_{\mathcal N} x$, then $Q\Downarrow_{\mathcal N} x$.
\end{enumerate}
$P$ is ${\mathcal N}$-barbed bisimilar to $Q$, written
$P \wbbisim_{\mathcal N} Q$, if $P \rel{S}_{\mathcal N} Q$ for some ${\mathcal N}$-barbed bisimulation ${\mathcal S}_{\mathcal N}$.
\end{definition}

$\mathcal{R} \subseteq \pi \times \pi$

$P \mathcal{R} Q => \forall P'. P \red P' \Rightarrow \exists Q'. Q \red Q', P' \mathcal{R} Q'$

$P \vdash x \Rightarrow Q \vdash x$

\begin{mathpar}
  \inferrule*[lab=Out-barb]{x \nameeq y}{{y}!\langle{Q}\rangle \vdash x}
  \and
  \inferrule*[lab=Par-barb]{\mbox{$P\vdash x$ or $Q\vdash x$}}{\binpar{P}{Q} \vdash x}
\end{mathpar}

\subsubsection{Contexts}

One of the principle advantages of computational calculi like the
$\pi$-calculus is a well-defined notion of context,
contextual-equivalence and a correlation between
contextual-equivalence and notions of bisimulation. The notion of
context allows the decomposition of a process into (sub-)process and
its syntactic environment, its context. Thus, a context may be
thought of as a process with a ``hole'' (written $\Box$) in it. The
application of a context $M$ to a process $P$, written $M[P]$, is
tantamount to filling the hole in $M$ with $P$. In this paper we do
not need the full weight of this theory, but do make use of the notion
of context in the proof the main theorem. 

\begin{mathpar}
  \inferrule* [lab=summation] {} {{M_{M},M_{N}} \bc \Box \;|\; x.M_{A} \;|\; M_{M}+M_{N}}
  \and
  \inferrule* [lab=agent] {} {{M_{A}} \bc (\vec{x})M_{P} \;| \; \clift{P_0,\ldots,M_{P},\ldots,P_N}}
  \and \\
  \inferrule* [lab=process] {} {{M_{P}} \bc M_{N} \;| \;P|M_{P} }
\end{mathpar} 

\begin{mathpar}
  \inferrule* [lab=sychronization] {} {M_{N} \bc \Box \;|\; x?M_{F} \;|\; x!M_{C}}
  \and
  \inferrule* [lab=abstraction] {} {{M_{F}} \bc (x)M_{P} }
  \and
  \inferrule* [lab=concretion] {} {{M_{C}} \bc \langle M_{P} \rangle }
  \and \\
  \inferrule* [lab=process] {} {{M_{P}} \bc M_{N} \;| \;P|M_{P} }
\end{mathpar}

\begin{definition}[contextual application] Given a context $M$, and
  process $P$, we define the \emph{contextual application}, $M[P] :=
  M\{P/\Box\}$. That is, the contextual application of M to P is the
  substitution of $P$ for $\Box$ in $M$.
\end{definition}

$\meaningof{-} : L \to \mathcal{P}(\pi)$

\begin{mathpar}
  \inferrule* [lab=collection] {} {\meaningof{true} = \pi, \and \meaningof{~E} = \pi \setminus \meaningof{E}, \and \meaningof{E_{1} \& E_{2}} = \meaningof{E_{1}} \cap \meaningof{E_{2}}}
\end{mathpar}

\begin{mathpar}
  \inferrule* [lab=structure] {} {\meaningof{0} = \{ P \in \pi | P \equiv 0 \}, \and \\ \meaningof{E_1 | E_2} = \{ P \in \pi | P \equiv P_{1} | P_{2}, P_{1} \in \meaningof{E_{1}}, P_{2} \in \meaningof{E_2}\} }
\end{mathpar}

\begin{mathpar}
 \inferrule* [lab=behavior] {} {\meaningof{\langle a?b \rangle E} = \{ P \in \pi | P \equiv Q | u?(y)P', \\ \and \\\\ \and \\ \;\;\; u \in \meaningof{a}, \forall z.P'\{z/y\} \in \meaningof{E\{z/b\}}\}, \and \\ \meaningof{a!E} = \{ P \in \pi | P \equiv Q | x!\langle P' \rangle, x \in \meaningof{a} P' \in \meaningof{E}\} }
\end{mathpar}

\begin{mathpar}
 \inferrule* [lab=nominal] {} {\meaningof{\quotep{E}} = \{ \quotep{P} \in \quotep{\pi} | P \in \meaningof{E} \}, \and \meaningof{\quotep{P}} = \{ \quotep{Q} \in \quotep{\pi} | P \equiv Q \} \and \\ \meaningof{@\quotep{E}} = \{ P \in \pi | P \equiv @x, x \in \meaningof{E} \}}
\end{mathpar}

\begin{eqnarray*}
  \\
  \meaningof{-} : TS \to ST
\end{eqnarray*}

\begin{eqnarray*}
  \\
  L : TS \to ST
\end{eqnarray*}

\begin{eqnarray*}
  \\
  P \models E \iff P \in \meaningof{E}
\end{eqnarray*}

\begin{eqnarray*}
  P \approx_{L} Q \iff \forall E \in L. P \models E \iff Q \models E
\end{eqnarray*}

\begin{eqnarray*}
  P \approx_{K} Q
\end{eqnarray*}

\begin{eqnarray*}
  P \approx Q
\end{eqnarray*}

$\approx_{K} = \approx = \approx_{L}$

\subsubsection{Contextual duality}

Note that contexts extend the quotation operation to a family of
operations from processes to names. Given a context, $M$, we can
define a \emph{nominal context}, $\quotep{M}$ by $\quotep{M}[P] :=
\quotep{M[P]}$. To foreshadow what is to come we observe that these
operations enjoy a duality with processes very much like the duality
between vectors and maps from vectors to scalars.

Further, because the calculus is essentially higher-order, we have a
correspondence between contexts and processes. More specifically,
given a name $x$ and a context $M$ we can construct $M^{*}_{x}$ such
that 

\begin{mathpar}
  M^{*}_{x} | \lift{x}{P} \red M[P]
\end{mathpar}

namely,

\begin{mathpar}
  M^{*}_{x} := x?(u).M[\dropn{u}]
\end{mathpar}

The dependence of $M^{*}_{x}$ on a name makes it an abstraction, 

\begin{mathpar}
  M^{*} := (x)x?(u).M[\dropn{u}]
\end{mathpar}

\subsection{Additional notation}

It will sometimes be convenient to denote the process a name
quotes. We already have the notation $x = \quotep{P}$, but it will be
convenient to introduce an alternate notation, $\procn{x}$, when we
want to emphasize the connection to the use of the name. Note that, by
virtue of name equivalence, $\quotep{\procn{x}} \nameeq x$; so, the
notation is consistent with previous definitions.

Further, because names have structure it is possible to effect
substitutions on the basis of that structure. This means we need to
upgrade our notation for substitutions, which we accomplish by
adapting comprehension notation. Thus,

\begin{mathpar}
  P\{ y / x : x \in S \}
\end{mathpar}

is interpreted to mean the process derived from P by replacing (in a
capture-avoiding manner) each occurrence of $x$ in $S$ by $y$. For example,

\begin{mathpar}
  P\{ \quotep{\procn{x}|\procn{x}} / x : x \in \freenames{P} \}
\end{mathpar}

will replace each (occurrence) of a free name $x$ in $P$ by
$\quotep{\procn{x}|\procn{x}}$.

Also, we will avail ourselves of the notation $x^{L}$ and $x^{R}$ to
denote injections of a name into disjoint copies of the name
space. There are numerous ways to accomplish this. One example can be
found in \cite{MeredithR05}. This notation overloads to vectors of
names: $\vec{x}^{\pi} := (x_{i}^{\pi} \; : \; 0 \leq i < |\vec{x}| )$ where $\pi \in \{L,R\}$.

We also use $P^{\Box} := P|\Box$.

In \cite{MeredithR05} an interpretation of the new operator is
given. It turns out that there are several possible interpretations
all enjoying the requisite algebraic properties of the operator (see
\cite{milner91polyadicpi}). We will therefore make liberal use of
$(\nu\; \vec{x})P$.

% subsection the_syntax_and_semantics_of_the_notation_system (end)   

\input{qm2pi.qmops} 

\input{qm2pi.sterngerlach} 

\input{qm2pi.metric} 

% section concurrent_process_calculi (end)

%\input{qm2pi.proofsketch}

% section proof sketch (end)

%\input{qm2pi.slviaknots} 

% section spatial logic via knots (end)

\input{qm2pi.conclusion}

% section conclusion (end)

%\input{qm2pi.dtcodes} 

% section wiring algorithm (end)

\input{qm2pi.ack} 

% section acknowledgments (end)

\newpage


\bibliographystyle{plain}   
\bibliography{../../biblios/main.bib}

\input{qm2pi.rhodetails}

\end{document}

 

\documentclass[12pt]{llncs}
%\documentclass{jktr}

\usepackage[pdftex]{hyperref}                   
\usepackage {listings}
\usepackage {mathpartir}
\usepackage{bcprules}
%\usepackage{listings}
                       
\usepackage{graphicx} 
%\usepackage[margins=2.5cm,nohead,nofoot]{geometry}
%\usepackage{geometry}
\usepackage{amsfonts}
\usepackage{amstext}
\usepackage{latexsym}
\usepackage{amssymb}
\usepackage{color}


%\include{myPreamble}
\include{qm2pi.local} 

%\ifpdf
%\usepackage[pdftex]{graphicx}
%\else
%\usepackage{graphicx}
%\fi

 % \ifpdf
%  \usepackage{pdfsync}
%  \if


%\title{Brief Article}
%\author{David F. Snyder}
%\author{L.G. Meredith}

%\address{Dept. of Math., Texas State University--San Marcos, San Marcos, TX 78666}
       
\pagestyle{empty}


\begin{document}

\lstset{language=[Objective]Caml,frame=shadowbox}

\input{qm2pi.front}

% section front matter (end)

\input{qm2pi.intro} 
 
% section introduction (end)

% \input{qm2pi.knotations} 

% section notation (end)

\input{qm2pi.process.calculi} 

% section concurrent_process_calculi_and_spatial_logics_ (end)
    
%\input{qm2pi.knots2pi} 

%\input{qm2pi.trefoil} 

%\input{qm2pi.mainthm} 

% subsection basic_interpretation (end)

%\input{qm2pi.rho.presentation} 
\subsection{The syntax and semantics of the notation system}\label{sub:the_syntax_and_semantics_of_the_notation_system} % (fold)

We now summarize a technical presentation of the calculus that
embodies our theory of dynamics. The typical presentation of such a
calculus follows the style of giving generators and relations on
them. The grammar, below, describing term constructors, freely
generates the set of processes, $\Proc$. This set is then quotiented
by a relation known as structural congruence and it is over this set
that the notion of dynamics is expressed. This presentation is
essentially that of \cite{MeredithR05} with the addition of
polyadicity and summation. For readability we have relegated some of
the technical subtleties to an appendix.

\subsubsection{Process grammar}\label{subsub:process_grammar}

\begin{mathpar}
  \inferrule* [lab=synchronization] {} {{M} \bc \pzero \;|\; x?F \;|\; x!C }
  \and
  \inferrule* [lab=abstraction] {} {{F} \bc (x)P}
  \and
  \inferrule* [lab=concretion] {} {{C} \bc \langle Q \rangle}
  \and
  \inferrule* [lab=process] {} {{P,Q} \bc M \;| \;P|Q \;|\; @{x}}
  \and
  \inferrule* [lab=name] {} {{x} \bc \quotep{P}}
\end{mathpar} 

Note that $\vec{x}$ (resp. $\vec{P}$) denotes a vector of names
(resp. processes) of length $|\vec{x}|$ (resp. $|\vec{P}|$). We adopt
the following useful abbreviations.

\begin{mathpar}
   x?(\vec{y}).P := x.(\vec{y})P \and  x\clift{\vec{P}} := x.\clift{\vec{P}}
   \and x!(y) := \lift{x}{\dropn{y}}
   \and \Pi_{i=0}^{n-1}P_i := P_0 | \ldots | P_{n-1}
\end{mathpar}

\subsubsection{Structural congruence}

\paragraph{Free and bound names and alpha-equivalence.} At the
core of structural equivalence is alpha-equivalence which identifies
process that are the same up to a change of variable. Formally, we
recognize the distinction between free and bound names. The free names
of a process, $\freenames{P}$, may be calculated recursively as
follows:

\begin{mathpar}
\freenames{\pzero} := \emptyset
  \and \\
  \freenames{x?(y).P} := \{ x \} \cup (\freenames{P} \setminus \{ y \})
  \and 
  \freenames{x!\langle P \rangle} := \{ x \} \cup \{ P \} 
  \and \\
  \freenames{P|Q} := \freenames{P} \cup \freenames{Q}
  \and \\
  \freenames{@{x}} := \{ x \}
\end{mathpar}

$\pi$
$\quotep{\pi}$

$\freenames{-} : \pi \to \mathcal{P}(\quotep{\pi})$

\begin{eqnarray*}
  \freenames{\pzero} & := & \emptyset \\
  \freenames{x?(y).P} & := & \{ x \} \cup (\freenames{P} \setminus \{ y \}) \\
  \freenames{x!\langle P \rangle} & := & \{ x \} \cup \{ P \} \\
  \freenames{P|Q} & := & \freenames{P} \cup \freenames{Q} \\
  \freenames{\dropn{x}} & := & \{ x \}
\end{eqnarray*}

The bound names of a process, $\boundnames{P}$, are those names occurring in $P$
that are not free. For example, in $x?(y).0$, the name $x$ is free, while $y$ is bound.

\begin{mathpar}
  \inferrule* [lab=monoidal-laws] {} { P|Q \equiv Q|P \and P|0 \equiv P \and P|(Q|R) \equiv (P|Q)|R }
\end{mathpar}

\begin{mathpar}
  \inferrule* [lab=alpha-equivalence] {} { (x)P \equiv (y)P\{y/x\} \and y \not\in \freenames{P} }
\end{mathpar}

\begin{definition}
Then two processes, $P,Q$, are alpha-equivalent if $P = Q\{\vec{y}/\vec{x}\}$ for
some $\vec{x} \in \boundnames{Q},\vec{y} \in \boundnames{P}$, where $Q\{\vec{y}/\vec{x}\}$
denotes the capture-avoiding substitution of $\vec{y}$ for $\vec{x}$ in $Q$.
\end{definition}

\begin{definition}
  The {\em structural congruence} \cite{SangiorgiWalker} , $\equiv$,
  between processes is the least congruence containing
  alpha-equivalence, satisfying the abelian monoid laws
  (associativity, commutativity and $\pzero$ as identity) for parallel
  composition $|$ and for summation $+$.
\end{definition}

\subsection{Name equivalence}

We take name equivalence, written $\nameeq$, to be the smallest
equivalence relation generated by the following rules.

\begin{mathpar}
\inferrule*[lab=Quote-drop]
{ }
{ \quotep{@{x}} \nameeq x }

\inferrule*[lab=Struct-equiv]
{ P \scong Q }
{ \quotep{P} \nameeq \quotep{Q} }
\end{mathpar}

The astute reader will have noticed that the mutual recursion of names
and processes imposes a mutual recursion on alpha-equivalence and
structural equivalence via name-equivalence. Fortunately, all of this
works out pleasantly and we may calculate in the natural way, free of
concern. The reader interested in the details is referred to the
appendix \ref{appendix:rho_details}.

\subsection{Substitution}

We use $\Proc$ for the set of processes, $\QProc$ for the set of
names, and $\id{\{}\vec{y} / \vec{x} \id{\}}$ to denote partial maps,
$s : \QProc \rightarrow \QProc$. A map, $s$ lifts, uniquely, to a map
on process terms, $\widehat{s} : \Proc \rightarrow \Proc$ by the
following equations.

\begin{mathpar}
  (0) \psubstp{Q}{P} := 0 \\
  (R \juxtap S) \psubstp{Q}{P}
  :=    
  (R)\psubstp{Q}{P} \juxtap (S) \psubstp{Q}{P} \\
  (x?(y).R) \psubstp{Q}{P}    
  :=    
  (x)\substp{Q}{P} (z)\concat( (R \psubstn{z}{y}) \psubstp{Q}{P} ) \\
  (\lift{x}{R}) \psubstp{Q}{P}  
  :=
  \lift{(x)\substp{Q}{P}}{ R \psubstp{Q}{P} } \\
%   (\dropn{x})  \psubstp{Q}{P}       
%   := 
%   \left\{ 
%     \begin{array}{ccc} 
%       \dropn{\quotep{Q}} & & x \nameeq \quotep{P} \\
%       \dropn{x} & & otherwise \\
%     \end{array}
%   \right. 
  (\dropn{x})  \psubstp{Q}{P}       
  := 
  \left\{ 
    \begin{array}{ccc} 
      Q & & x \nameeq \quotep{P} \\
      \dropn{x} & & otherwise \\
    \end{array}
  \right.
\end{mathpar}
 

where

\begin{eqnarray}
  (x)\id{\{} \lpquote Q \rpquote / \lpquote P \rpquote \id{\}}            = 
  \left\{ 
    \begin{array}{ccc}
      \lpquote Q \rpquote & & x \nameeq \lpquote P \rpquote \\
      x & & otherwise \\
    \end{array}
  \right. \nonumber
\end{eqnarray}

and $z$ is chosen distinct from $\quotep{P}$, $\quotep{Q}$, the free
names in $Q$, and all the names in $R$. Our $\alpha$-equivalence will
be built in the standard way from this substitution.

\begin{remark}\label{rem:no_self_referential_names}
  One consequence of these definitions is that $\forall P. \quotep{P}
  \not\in \freenames{P}$.
\end{remark}

\subsection{ Dynamic quote: an example }

Anticipating something of what's to come, consider applying the
substitution, $\widehat{\id{\{}u / z \id{\}}}$, to the following pair
of processes, $\lift{w}{y!(z)}$ and $w[ \lpquote y!(z) \rpquote ]$.

\begin{eqnarray}
	\lift{w}{y!(z)}\widehat{\id{\{}u / z \id{\}}}
		& = &
		\lift{w}{y!(u)} \nonumber\\
	w[ \lpquote y!(z) \rpquote ] \widehat{ \id{\{}u / z \id{\}} }
		& = &
		w[ \lpquote y!(z) \rpquote ] \nonumber
\end{eqnarray}

Because the body of the process between quotes is impervious to
substitution, we get radically different answers. In fact, by
examining the first process in an input context,
e.g. $x?(z).\lift{w}{y!(z)}$, we see that the process under the lift
operator may be shaped by prefixed inputs binding a name inside it. In
this sense, the lift operator will be seen as a way to dynamically
construct processes before reifying them as names.

Finally equipped with these standard features we can present the
dynamics of the calculus.

\subsubsection{Operational semantics} 

Finally, we introduce the computational dynamics. What marks these
algebras as distinct from other more traditionally studied algebraic
structures, e.g. vector spaces or polynomial rings, is the manner in
which dynamics is captured. In traditional structures, dynamics is typically
expressed through morphisms between such structures, as in linear maps
between vector spaces or morphisms between rings. In algebras
associated with the semantics of computation, the dynamics is
expressed as part of the algebraic structure itself, through a
reduction reduction relation typically denoted by $\red$. Below, we
give a recursive presentation of this relation for the calculus used
in the encoding.

$\red \subseteq \pi \times \pi$
$\red : \pi \to \mathcal{P}(\pi)$

\begin{mathpar}
  \inferrule* [lab=Comm] { \textsf{match}( x_{src}, x_{trgt} ) } { x_{trgt}?(y)P \; | \; x_{src}!\langle {Q} \rangle \red P\{\quotep{Q}/y}\} }
  \and \\
  \inferrule* [lab=Par] {{P} \red {P}'} {{{P} | {Q}} \red {{P}' | {Q}}}
  \and
  \inferrule* [lab=Equiv]{{{P} \scong {P}'} \andalso {{P}' \red {Q}'} \andalso {{Q}' \scong {Q}}}{{P} \red {Q}}
\end{mathpar}

\begin{eqnarray*}
  match_{\equiv} (\quotep{P},\quotep{Q}) & := & P \equiv Q \\
  match_{\dagger}(\quotep{P},\quotep{Q}) & := & \forall R. P|Q \red^{*} R => R \red^{*} 0 \\
  match_{K}(\quotep{P},\quotep{Q}) & := & K \mbox{ for some context } K
\end{eqnarray*}

$u?(x)P | u!\langle Q \rangle \red P\{\quotep{Q}/x\}$

%We write $\wred$ for $\red^*$, and $P\red$ if $\exists Q $ such that $ P \red Q$.
We write $P\red$ if $\exists Q $ such that $ P \red Q$ and $P\not\red$, otherwise.

\section{Replication}

As mentioned before, it is known that replication (and hence
recursion) can be implemented in a higher-order process algebra
\cite{SangiorgiWalker}. As our first example of calculation with the
machinery thus far presented we give the construction explicitly in
the {\rhoc}.

\begin{eqnarray}
	D_{x} & := & \prefix{x}{y}{(\binpar{\outputp{x}{y}}{@{y}})} \nonumber\\
	\bangp_{x}{P} & := & \binpar{{x}!\langle{\binpar{D_{x}}{P}}\rangle}{D_{x}} \nonumber
\end{eqnarray}

\begin{eqnarray}
	\bangp_{x}{P} & & \nonumber\\
	=
	& {x}!\langle{(\prefix{x}{y}{(\outputp{x}{y} | @{y})) | P}}\rangle 
	      | \prefix{x}{y}{(\outputp{x}{y} | @{y})} & \nonumber\\
	\red
	& (\outputp{x}{y} | @{y})\substn{\quotep{(\prefix{x}{y}{(@{y} | \outputp{x}{y})) | P}}}{y} & \nonumber\\
	=
	& \outputp{x}{\quotep{(\prefix{x}{y}{(\outputp{x}{y} | @{y})) | P}}}
	  | {(\prefix{x}{y}{(\outputp{x}{y} | @{y})) | P}} & \nonumber\\
	\red
	& \ldots & \nonumber\\
	\red^*
	& P | P | \ldots & \nonumber
\end{eqnarray}

Of course, this encoding, as an implementation, runs away, unfolding
$\bangp{P}$ eagerly. A lazier and more implementable replication
operator, restricted to input-guarded processes, may be obtained as follows.

\begin{eqnarray}
\bangp{\prefix{u}{v}{P}} 
	:= 
	\binpar{\lift{x}{\prefix{u}{v}{(\binpar{D(x)}{P})}}}{D(x)} \nonumber
\end{eqnarray}

\begin{remark}
  Note that the lazier definition still does not deal with summation
  or mixed summation (i.e. sums over input and output). The reader is
  invited to construct definitions of replication that deal with these
  features. 

  Further, the definitions are parameterized in a name, $x$. Can you,
  gentle reader, make a definition that eliminates this parameter and
  guarantees no accidental interaction between the replication
  machinery and the process being replicated -- i.e. no accidental
  sharing of names used by the process to get its work done and the
  name(s) used by the replication to effect copying. This latter
  revision of the definition of replication is crucial to obtaining
  the expected identity $!!P \sim !P$.
\end{remark}

\begin{remark}\label{rem:paradoxical_combinator}
  The reader familiar with the lambda calculus will have noticed the
  similarity between $D$ and the paradoxical combinator.

  [Ed. note: the existence of this seems to suggest we have to be more
  restrictive on the set of processes and names we admit if we are to
  support no-cloning.]
\end{remark}

\subsubsection{Bisimulation}

The computational dynamics gives rise to another kind of equivalence,
the equivalence of computational behavior. As previously mentioned
this is typically captured \emph{via} some form of bisimulation.

% The notion we use in this paper is weak barbed bisimulation
% \cite{milner91polyadicpi}.

The notion we use in this paper is derived from weak barbed
bisimulation \cite{milner91polyadicpi}. 

\begin{definition}
An \emph{observation relation}, $\downarrow_{\mathcal N}$, over a set
of names, $\mathcal N$, is the smallest relation satisfying the rules
below.

\infrule[Out-barb]{y \in {\mathcal N}, \; x \nameeq y}
		  {\outputp{x}{v} \downarrow_{\mathcal N} x}
\infrule[Par-barb]{\mbox{$P\downarrow_{\mathcal N} x$ or $Q\downarrow_{\mathcal N} x$}}
		  {\binpar{P}{Q} \downarrow_{\mathcal N} x}

We write $P \Downarrow_{\mathcal N} x$ if there is $Q$ such that 
$P \wred Q$ and $Q \downarrow_{\mathcal N} x$.
\end{definition}

\begin{definition}
%\label{def.bbisim}
An  ${\mathcal N}$-\emph{barbed bisimulation} over a set of names, ${\mathcal N}$, is a symmetric binary relation 
${\mathcal S}_{\mathcal N}$ between agents such that $P\rel{S}_{\mathcal N}Q$ implies:
\begin{enumerate}
\item If $P \red P'$ then $Q \wred Q'$ and $P'\rel{S}_{\mathcal N} Q'$.
\item If $P\downarrow_{\mathcal N} x$, then $Q\Downarrow_{\mathcal N} x$.
\end{enumerate}
$P$ is ${\mathcal N}$-barbed bisimilar to $Q$, written
$P \wbbisim_{\mathcal N} Q$, if $P \rel{S}_{\mathcal N} Q$ for some ${\mathcal N}$-barbed bisimulation ${\mathcal S}_{\mathcal N}$.
\end{definition}

$\mathcal{R} \subseteq \pi \times \pi$

$P \mathcal{R} Q => \forall P'. P \red P' \Rightarrow \exists Q'. Q \red Q', P' \mathcal{R} Q'$

$P \vdash x \Rightarrow Q \vdash x$

\begin{mathpar}
  \inferrule*[lab=Out-barb]{x \nameeq y}{{y}!\langle{Q}\rangle \vdash x}
  \and
  \inferrule*[lab=Par-barb]{\mbox{$P\vdash x$ or $Q\vdash x$}}{\binpar{P}{Q} \vdash x}
\end{mathpar}

\subsubsection{Contexts}

One of the principle advantages of computational calculi like the
$\pi$-calculus is a well-defined notion of context,
contextual-equivalence and a correlation between
contextual-equivalence and notions of bisimulation. The notion of
context allows the decomposition of a process into (sub-)process and
its syntactic environment, its context. Thus, a context may be
thought of as a process with a ``hole'' (written $\Box$) in it. The
application of a context $M$ to a process $P$, written $M[P]$, is
tantamount to filling the hole in $M$ with $P$. In this paper we do
not need the full weight of this theory, but do make use of the notion
of context in the proof the main theorem. 

\begin{mathpar}
  \inferrule* [lab=summation] {} {{M_{M},M_{N}} \bc \Box \;|\; x.M_{A} \;|\; M_{M}+M_{N}}
  \and
  \inferrule* [lab=agent] {} {{M_{A}} \bc (\vec{x})M_{P} \;| \; \clift{P_0,\ldots,M_{P},\ldots,P_N}}
  \and \\
  \inferrule* [lab=process] {} {{M_{P}} \bc M_{N} \;| \;P|M_{P} }
\end{mathpar} 

\begin{mathpar}
  \inferrule* [lab=sychronization] {} {M_{N} \bc \Box \;|\; x?M_{F} \;|\; x!M_{C}}
  \and
  \inferrule* [lab=abstraction] {} {{M_{F}} \bc (x)M_{P} }
  \and
  \inferrule* [lab=concretion] {} {{M_{C}} \bc \langle M_{P} \rangle }
  \and \\
  \inferrule* [lab=process] {} {{M_{P}} \bc M_{N} \;| \;P|M_{P} }
\end{mathpar}

\begin{definition}[contextual application] Given a context $M$, and
  process $P$, we define the \emph{contextual application}, $M[P] :=
  M\{P/\Box\}$. That is, the contextual application of M to P is the
  substitution of $P$ for $\Box$ in $M$.
\end{definition}

$\meaningof{-} : L \to \mathcal{P}(\pi)$

\begin{mathpar}
  \inferrule* [lab=collection] {} {\meaningof{true} = \pi, \and \meaningof{~E} = \pi \setminus \meaningof{E}, \and \meaningof{E_{1} \& E_{2}} = \meaningof{E_{1}} \cap \meaningof{E_{2}}}
\end{mathpar}

\begin{mathpar}
  \inferrule* [lab=structure] {} {\meaningof{0} = \{ P \in \pi | P \equiv 0 \}, \and \\ \meaningof{E_1 | E_2} = \{ P \in \pi | P \equiv P_{1} | P_{2}, P_{1} \in \meaningof{E_{1}}, P_{2} \in \meaningof{E_2}\} }
\end{mathpar}

\begin{mathpar}
 \inferrule* [lab=behavior] {} {\meaningof{\langle a?b \rangle E} = \{ P \in \pi | P \equiv Q | u?(y)P', \\ \and \\\\ \and \\ \;\;\; u \in \meaningof{a}, \forall z.P'\{z/y\} \in \meaningof{E\{z/b\}}\}, \and \\ \meaningof{a!E} = \{ P \in \pi | P \equiv Q | x!\langle P' \rangle, x \in \meaningof{a} P' \in \meaningof{E}\} }
\end{mathpar}

\begin{mathpar}
 \inferrule* [lab=nominal] {} {\meaningof{\quotep{E}} = \{ \quotep{P} \in \quotep{\pi} | P \in \meaningof{E} \}, \and \meaningof{\quotep{P}} = \{ \quotep{Q} \in \quotep{\pi} | P \equiv Q \} \and \\ \meaningof{@\quotep{E}} = \{ P \in \pi | P \equiv @x, x \in \meaningof{E} \}}
\end{mathpar}

\begin{eqnarray*}
  \\
  \meaningof{-} : TS \to ST
\end{eqnarray*}

\begin{eqnarray*}
  \\
  L : TS \to ST
\end{eqnarray*}

\begin{eqnarray*}
  \\
  P \models E \iff P \in \meaningof{E}
\end{eqnarray*}

\begin{eqnarray*}
  P \approx_{L} Q \iff \forall E \in L. P \models E \iff Q \models E
\end{eqnarray*}

\begin{eqnarray*}
  P \approx_{K} Q
\end{eqnarray*}

\begin{eqnarray*}
  P \approx Q
\end{eqnarray*}

$\approx_{K} = \approx = \approx_{L}$

\subsubsection{Contextual duality}

Note that contexts extend the quotation operation to a family of
operations from processes to names. Given a context, $M$, we can
define a \emph{nominal context}, $\quotep{M}$ by $\quotep{M}[P] :=
\quotep{M[P]}$. To foreshadow what is to come we observe that these
operations enjoy a duality with processes very much like the duality
between vectors and maps from vectors to scalars.

Further, because the calculus is essentially higher-order, we have a
correspondence between contexts and processes. More specifically,
given a name $x$ and a context $M$ we can construct $M^{*}_{x}$ such
that 

\begin{mathpar}
  M^{*}_{x} | \lift{x}{P} \red M[P]
\end{mathpar}

namely,

\begin{mathpar}
  M^{*}_{x} := x?(u).M[\dropn{u}]
\end{mathpar}

The dependence of $M^{*}_{x}$ on a name makes it an abstraction, 

\begin{mathpar}
  M^{*} := (x)x?(u).M[\dropn{u}]
\end{mathpar}

\subsection{Additional notation}

It will sometimes be convenient to denote the process a name
quotes. We already have the notation $x = \quotep{P}$, but it will be
convenient to introduce an alternate notation, $\procn{x}$, when we
want to emphasize the connection to the use of the name. Note that, by
virtue of name equivalence, $\quotep{\procn{x}} \nameeq x$; so, the
notation is consistent with previous definitions.

Further, because names have structure it is possible to effect
substitutions on the basis of that structure. This means we need to
upgrade our notation for substitutions, which we accomplish by
adapting comprehension notation. Thus,

\begin{mathpar}
  P\{ y / x : x \in S \}
\end{mathpar}

is interpreted to mean the process derived from P by replacing (in a
capture-avoiding manner) each occurrence of $x$ in $S$ by $y$. For example,

\begin{mathpar}
  P\{ \quotep{\procn{x}|\procn{x}} / x : x \in \freenames{P} \}
\end{mathpar}

will replace each (occurrence) of a free name $x$ in $P$ by
$\quotep{\procn{x}|\procn{x}}$.

Also, we will avail ourselves of the notation $x^{L}$ and $x^{R}$ to
denote injections of a name into disjoint copies of the name
space. There are numerous ways to accomplish this. One example can be
found in \cite{MeredithR05}. This notation overloads to vectors of
names: $\vec{x}^{\pi} := (x_{i}^{\pi} \; : \; 0 \leq i < |\vec{x}| )$ where $\pi \in \{L,R\}$.

We also use $P^{\Box} := P|\Box$.

In \cite{MeredithR05} an interpretation of the new operator is
given. It turns out that there are several possible interpretations
all enjoying the requisite algebraic properties of the operator (see
\cite{milner91polyadicpi}). We will therefore make liberal use of
$(\nu\; \vec{x})P$.

% subsection the_syntax_and_semantics_of_the_notation_system (end)   

\input{qm2pi.qmops} 

\input{qm2pi.sterngerlach} 

\input{qm2pi.metric} 

% section concurrent_process_calculi (end)

%\input{qm2pi.proofsketch}

% section proof sketch (end)

%\input{qm2pi.slviaknots} 

% section spatial logic via knots (end)

\input{qm2pi.conclusion}

% section conclusion (end)

%\input{qm2pi.dtcodes} 

% section wiring algorithm (end)

\input{qm2pi.ack} 

% section acknowledgments (end)

\newpage


\bibliographystyle{plain}   
\bibliography{../../biblios/main.bib}

\input{qm2pi.rhodetails}

\end{document}

 

% section concurrent_process_calculi (end)

%\documentclass[12pt]{llncs}
%\documentclass{jktr}

\usepackage[pdftex]{hyperref}                   
\usepackage {listings}
\usepackage {mathpartir}
\usepackage{bcprules}
%\usepackage{listings}
                       
\usepackage{graphicx} 
%\usepackage[margins=2.5cm,nohead,nofoot]{geometry}
%\usepackage{geometry}
\usepackage{amsfonts}
\usepackage{amstext}
\usepackage{latexsym}
\usepackage{amssymb}
\usepackage{color}


%\include{myPreamble}
\include{qm2pi.local} 

%\ifpdf
%\usepackage[pdftex]{graphicx}
%\else
%\usepackage{graphicx}
%\fi

 % \ifpdf
%  \usepackage{pdfsync}
%  \if


%\title{Brief Article}
%\author{David F. Snyder}
%\author{L.G. Meredith}

%\address{Dept. of Math., Texas State University--San Marcos, San Marcos, TX 78666}
       
\pagestyle{empty}


\begin{document}

\lstset{language=[Objective]Caml,frame=shadowbox}

\input{qm2pi.front}

% section front matter (end)

\input{qm2pi.intro} 
 
% section introduction (end)

% \input{qm2pi.knotations} 

% section notation (end)

\input{qm2pi.process.calculi} 

% section concurrent_process_calculi_and_spatial_logics_ (end)
    
%\input{qm2pi.knots2pi} 

%\input{qm2pi.trefoil} 

%\input{qm2pi.mainthm} 

% subsection basic_interpretation (end)

%\input{qm2pi.rho.presentation} 
\subsection{The syntax and semantics of the notation system}\label{sub:the_syntax_and_semantics_of_the_notation_system} % (fold)

We now summarize a technical presentation of the calculus that
embodies our theory of dynamics. The typical presentation of such a
calculus follows the style of giving generators and relations on
them. The grammar, below, describing term constructors, freely
generates the set of processes, $\Proc$. This set is then quotiented
by a relation known as structural congruence and it is over this set
that the notion of dynamics is expressed. This presentation is
essentially that of \cite{MeredithR05} with the addition of
polyadicity and summation. For readability we have relegated some of
the technical subtleties to an appendix.

\subsubsection{Process grammar}\label{subsub:process_grammar}

\begin{mathpar}
  \inferrule* [lab=synchronization] {} {{M} \bc \pzero \;|\; x?F \;|\; x!C }
  \and
  \inferrule* [lab=abstraction] {} {{F} \bc (x)P}
  \and
  \inferrule* [lab=concretion] {} {{C} \bc \langle Q \rangle}
  \and
  \inferrule* [lab=process] {} {{P,Q} \bc M \;| \;P|Q \;|\; @{x}}
  \and
  \inferrule* [lab=name] {} {{x} \bc \quotep{P}}
\end{mathpar} 

Note that $\vec{x}$ (resp. $\vec{P}$) denotes a vector of names
(resp. processes) of length $|\vec{x}|$ (resp. $|\vec{P}|$). We adopt
the following useful abbreviations.

\begin{mathpar}
   x?(\vec{y}).P := x.(\vec{y})P \and  x\clift{\vec{P}} := x.\clift{\vec{P}}
   \and x!(y) := \lift{x}{\dropn{y}}
   \and \Pi_{i=0}^{n-1}P_i := P_0 | \ldots | P_{n-1}
\end{mathpar}

\subsubsection{Structural congruence}

\paragraph{Free and bound names and alpha-equivalence.} At the
core of structural equivalence is alpha-equivalence which identifies
process that are the same up to a change of variable. Formally, we
recognize the distinction between free and bound names. The free names
of a process, $\freenames{P}$, may be calculated recursively as
follows:

\begin{mathpar}
\freenames{\pzero} := \emptyset
  \and \\
  \freenames{x?(y).P} := \{ x \} \cup (\freenames{P} \setminus \{ y \})
  \and 
  \freenames{x!\langle P \rangle} := \{ x \} \cup \{ P \} 
  \and \\
  \freenames{P|Q} := \freenames{P} \cup \freenames{Q}
  \and \\
  \freenames{@{x}} := \{ x \}
\end{mathpar}

$\pi$
$\quotep{\pi}$

$\freenames{-} : \pi \to \mathcal{P}(\quotep{\pi})$

\begin{eqnarray*}
  \freenames{\pzero} & := & \emptyset \\
  \freenames{x?(y).P} & := & \{ x \} \cup (\freenames{P} \setminus \{ y \}) \\
  \freenames{x!\langle P \rangle} & := & \{ x \} \cup \{ P \} \\
  \freenames{P|Q} & := & \freenames{P} \cup \freenames{Q} \\
  \freenames{\dropn{x}} & := & \{ x \}
\end{eqnarray*}

The bound names of a process, $\boundnames{P}$, are those names occurring in $P$
that are not free. For example, in $x?(y).0$, the name $x$ is free, while $y$ is bound.

\begin{mathpar}
  \inferrule* [lab=monoidal-laws] {} { P|Q \equiv Q|P \and P|0 \equiv P \and P|(Q|R) \equiv (P|Q)|R }
\end{mathpar}

\begin{mathpar}
  \inferrule* [lab=alpha-equivalence] {} { (x)P \equiv (y)P\{y/x\} \and y \not\in \freenames{P} }
\end{mathpar}

\begin{definition}
Then two processes, $P,Q$, are alpha-equivalent if $P = Q\{\vec{y}/\vec{x}\}$ for
some $\vec{x} \in \boundnames{Q},\vec{y} \in \boundnames{P}$, where $Q\{\vec{y}/\vec{x}\}$
denotes the capture-avoiding substitution of $\vec{y}$ for $\vec{x}$ in $Q$.
\end{definition}

\begin{definition}
  The {\em structural congruence} \cite{SangiorgiWalker} , $\equiv$,
  between processes is the least congruence containing
  alpha-equivalence, satisfying the abelian monoid laws
  (associativity, commutativity and $\pzero$ as identity) for parallel
  composition $|$ and for summation $+$.
\end{definition}

\subsection{Name equivalence}

We take name equivalence, written $\nameeq$, to be the smallest
equivalence relation generated by the following rules.

\begin{mathpar}
\inferrule*[lab=Quote-drop]
{ }
{ \quotep{@{x}} \nameeq x }

\inferrule*[lab=Struct-equiv]
{ P \scong Q }
{ \quotep{P} \nameeq \quotep{Q} }
\end{mathpar}

The astute reader will have noticed that the mutual recursion of names
and processes imposes a mutual recursion on alpha-equivalence and
structural equivalence via name-equivalence. Fortunately, all of this
works out pleasantly and we may calculate in the natural way, free of
concern. The reader interested in the details is referred to the
appendix \ref{appendix:rho_details}.

\subsection{Substitution}

We use $\Proc$ for the set of processes, $\QProc$ for the set of
names, and $\id{\{}\vec{y} / \vec{x} \id{\}}$ to denote partial maps,
$s : \QProc \rightarrow \QProc$. A map, $s$ lifts, uniquely, to a map
on process terms, $\widehat{s} : \Proc \rightarrow \Proc$ by the
following equations.

\begin{mathpar}
  (0) \psubstp{Q}{P} := 0 \\
  (R \juxtap S) \psubstp{Q}{P}
  :=    
  (R)\psubstp{Q}{P} \juxtap (S) \psubstp{Q}{P} \\
  (x?(y).R) \psubstp{Q}{P}    
  :=    
  (x)\substp{Q}{P} (z)\concat( (R \psubstn{z}{y}) \psubstp{Q}{P} ) \\
  (\lift{x}{R}) \psubstp{Q}{P}  
  :=
  \lift{(x)\substp{Q}{P}}{ R \psubstp{Q}{P} } \\
%   (\dropn{x})  \psubstp{Q}{P}       
%   := 
%   \left\{ 
%     \begin{array}{ccc} 
%       \dropn{\quotep{Q}} & & x \nameeq \quotep{P} \\
%       \dropn{x} & & otherwise \\
%     \end{array}
%   \right. 
  (\dropn{x})  \psubstp{Q}{P}       
  := 
  \left\{ 
    \begin{array}{ccc} 
      Q & & x \nameeq \quotep{P} \\
      \dropn{x} & & otherwise \\
    \end{array}
  \right.
\end{mathpar}
 

where

\begin{eqnarray}
  (x)\id{\{} \lpquote Q \rpquote / \lpquote P \rpquote \id{\}}            = 
  \left\{ 
    \begin{array}{ccc}
      \lpquote Q \rpquote & & x \nameeq \lpquote P \rpquote \\
      x & & otherwise \\
    \end{array}
  \right. \nonumber
\end{eqnarray}

and $z$ is chosen distinct from $\quotep{P}$, $\quotep{Q}$, the free
names in $Q$, and all the names in $R$. Our $\alpha$-equivalence will
be built in the standard way from this substitution.

\begin{remark}\label{rem:no_self_referential_names}
  One consequence of these definitions is that $\forall P. \quotep{P}
  \not\in \freenames{P}$.
\end{remark}

\subsection{ Dynamic quote: an example }

Anticipating something of what's to come, consider applying the
substitution, $\widehat{\id{\{}u / z \id{\}}}$, to the following pair
of processes, $\lift{w}{y!(z)}$ and $w[ \lpquote y!(z) \rpquote ]$.

\begin{eqnarray}
	\lift{w}{y!(z)}\widehat{\id{\{}u / z \id{\}}}
		& = &
		\lift{w}{y!(u)} \nonumber\\
	w[ \lpquote y!(z) \rpquote ] \widehat{ \id{\{}u / z \id{\}} }
		& = &
		w[ \lpquote y!(z) \rpquote ] \nonumber
\end{eqnarray}

Because the body of the process between quotes is impervious to
substitution, we get radically different answers. In fact, by
examining the first process in an input context,
e.g. $x?(z).\lift{w}{y!(z)}$, we see that the process under the lift
operator may be shaped by prefixed inputs binding a name inside it. In
this sense, the lift operator will be seen as a way to dynamically
construct processes before reifying them as names.

Finally equipped with these standard features we can present the
dynamics of the calculus.

\subsubsection{Operational semantics} 

Finally, we introduce the computational dynamics. What marks these
algebras as distinct from other more traditionally studied algebraic
structures, e.g. vector spaces or polynomial rings, is the manner in
which dynamics is captured. In traditional structures, dynamics is typically
expressed through morphisms between such structures, as in linear maps
between vector spaces or morphisms between rings. In algebras
associated with the semantics of computation, the dynamics is
expressed as part of the algebraic structure itself, through a
reduction reduction relation typically denoted by $\red$. Below, we
give a recursive presentation of this relation for the calculus used
in the encoding.

$\red \subseteq \pi \times \pi$
$\red : \pi \to \mathcal{P}(\pi)$

\begin{mathpar}
  \inferrule* [lab=Comm] { \textsf{match}( x_{src}, x_{trgt} ) } { x_{trgt}?(y)P \; | \; x_{src}!\langle {Q} \rangle \red P\{\quotep{Q}/y}\} }
  \and \\
  \inferrule* [lab=Par] {{P} \red {P}'} {{{P} | {Q}} \red {{P}' | {Q}}}
  \and
  \inferrule* [lab=Equiv]{{{P} \scong {P}'} \andalso {{P}' \red {Q}'} \andalso {{Q}' \scong {Q}}}{{P} \red {Q}}
\end{mathpar}

\begin{eqnarray*}
  match_{\equiv} (\quotep{P},\quotep{Q}) & := & P \equiv Q \\
  match_{\dagger}(\quotep{P},\quotep{Q}) & := & \forall R. P|Q \red^{*} R => R \red^{*} 0 \\
  match_{K}(\quotep{P},\quotep{Q}) & := & K \mbox{ for some context } K
\end{eqnarray*}

$u?(x)P | u!\langle Q \rangle \red P\{\quotep{Q}/x\}$

%We write $\wred$ for $\red^*$, and $P\red$ if $\exists Q $ such that $ P \red Q$.
We write $P\red$ if $\exists Q $ such that $ P \red Q$ and $P\not\red$, otherwise.

\section{Replication}

As mentioned before, it is known that replication (and hence
recursion) can be implemented in a higher-order process algebra
\cite{SangiorgiWalker}. As our first example of calculation with the
machinery thus far presented we give the construction explicitly in
the {\rhoc}.

\begin{eqnarray}
	D_{x} & := & \prefix{x}{y}{(\binpar{\outputp{x}{y}}{@{y}})} \nonumber\\
	\bangp_{x}{P} & := & \binpar{{x}!\langle{\binpar{D_{x}}{P}}\rangle}{D_{x}} \nonumber
\end{eqnarray}

\begin{eqnarray}
	\bangp_{x}{P} & & \nonumber\\
	=
	& {x}!\langle{(\prefix{x}{y}{(\outputp{x}{y} | @{y})) | P}}\rangle 
	      | \prefix{x}{y}{(\outputp{x}{y} | @{y})} & \nonumber\\
	\red
	& (\outputp{x}{y} | @{y})\substn{\quotep{(\prefix{x}{y}{(@{y} | \outputp{x}{y})) | P}}}{y} & \nonumber\\
	=
	& \outputp{x}{\quotep{(\prefix{x}{y}{(\outputp{x}{y} | @{y})) | P}}}
	  | {(\prefix{x}{y}{(\outputp{x}{y} | @{y})) | P}} & \nonumber\\
	\red
	& \ldots & \nonumber\\
	\red^*
	& P | P | \ldots & \nonumber
\end{eqnarray}

Of course, this encoding, as an implementation, runs away, unfolding
$\bangp{P}$ eagerly. A lazier and more implementable replication
operator, restricted to input-guarded processes, may be obtained as follows.

\begin{eqnarray}
\bangp{\prefix{u}{v}{P}} 
	:= 
	\binpar{\lift{x}{\prefix{u}{v}{(\binpar{D(x)}{P})}}}{D(x)} \nonumber
\end{eqnarray}

\begin{remark}
  Note that the lazier definition still does not deal with summation
  or mixed summation (i.e. sums over input and output). The reader is
  invited to construct definitions of replication that deal with these
  features. 

  Further, the definitions are parameterized in a name, $x$. Can you,
  gentle reader, make a definition that eliminates this parameter and
  guarantees no accidental interaction between the replication
  machinery and the process being replicated -- i.e. no accidental
  sharing of names used by the process to get its work done and the
  name(s) used by the replication to effect copying. This latter
  revision of the definition of replication is crucial to obtaining
  the expected identity $!!P \sim !P$.
\end{remark}

\begin{remark}\label{rem:paradoxical_combinator}
  The reader familiar with the lambda calculus will have noticed the
  similarity between $D$ and the paradoxical combinator.

  [Ed. note: the existence of this seems to suggest we have to be more
  restrictive on the set of processes and names we admit if we are to
  support no-cloning.]
\end{remark}

\subsubsection{Bisimulation}

The computational dynamics gives rise to another kind of equivalence,
the equivalence of computational behavior. As previously mentioned
this is typically captured \emph{via} some form of bisimulation.

% The notion we use in this paper is weak barbed bisimulation
% \cite{milner91polyadicpi}.

The notion we use in this paper is derived from weak barbed
bisimulation \cite{milner91polyadicpi}. 

\begin{definition}
An \emph{observation relation}, $\downarrow_{\mathcal N}$, over a set
of names, $\mathcal N$, is the smallest relation satisfying the rules
below.

\infrule[Out-barb]{y \in {\mathcal N}, \; x \nameeq y}
		  {\outputp{x}{v} \downarrow_{\mathcal N} x}
\infrule[Par-barb]{\mbox{$P\downarrow_{\mathcal N} x$ or $Q\downarrow_{\mathcal N} x$}}
		  {\binpar{P}{Q} \downarrow_{\mathcal N} x}

We write $P \Downarrow_{\mathcal N} x$ if there is $Q$ such that 
$P \wred Q$ and $Q \downarrow_{\mathcal N} x$.
\end{definition}

\begin{definition}
%\label{def.bbisim}
An  ${\mathcal N}$-\emph{barbed bisimulation} over a set of names, ${\mathcal N}$, is a symmetric binary relation 
${\mathcal S}_{\mathcal N}$ between agents such that $P\rel{S}_{\mathcal N}Q$ implies:
\begin{enumerate}
\item If $P \red P'$ then $Q \wred Q'$ and $P'\rel{S}_{\mathcal N} Q'$.
\item If $P\downarrow_{\mathcal N} x$, then $Q\Downarrow_{\mathcal N} x$.
\end{enumerate}
$P$ is ${\mathcal N}$-barbed bisimilar to $Q$, written
$P \wbbisim_{\mathcal N} Q$, if $P \rel{S}_{\mathcal N} Q$ for some ${\mathcal N}$-barbed bisimulation ${\mathcal S}_{\mathcal N}$.
\end{definition}

$\mathcal{R} \subseteq \pi \times \pi$

$P \mathcal{R} Q => \forall P'. P \red P' \Rightarrow \exists Q'. Q \red Q', P' \mathcal{R} Q'$

$P \vdash x \Rightarrow Q \vdash x$

\begin{mathpar}
  \inferrule*[lab=Out-barb]{x \nameeq y}{{y}!\langle{Q}\rangle \vdash x}
  \and
  \inferrule*[lab=Par-barb]{\mbox{$P\vdash x$ or $Q\vdash x$}}{\binpar{P}{Q} \vdash x}
\end{mathpar}

\subsubsection{Contexts}

One of the principle advantages of computational calculi like the
$\pi$-calculus is a well-defined notion of context,
contextual-equivalence and a correlation between
contextual-equivalence and notions of bisimulation. The notion of
context allows the decomposition of a process into (sub-)process and
its syntactic environment, its context. Thus, a context may be
thought of as a process with a ``hole'' (written $\Box$) in it. The
application of a context $M$ to a process $P$, written $M[P]$, is
tantamount to filling the hole in $M$ with $P$. In this paper we do
not need the full weight of this theory, but do make use of the notion
of context in the proof the main theorem. 

\begin{mathpar}
  \inferrule* [lab=summation] {} {{M_{M},M_{N}} \bc \Box \;|\; x.M_{A} \;|\; M_{M}+M_{N}}
  \and
  \inferrule* [lab=agent] {} {{M_{A}} \bc (\vec{x})M_{P} \;| \; \clift{P_0,\ldots,M_{P},\ldots,P_N}}
  \and \\
  \inferrule* [lab=process] {} {{M_{P}} \bc M_{N} \;| \;P|M_{P} }
\end{mathpar} 

\begin{mathpar}
  \inferrule* [lab=sychronization] {} {M_{N} \bc \Box \;|\; x?M_{F} \;|\; x!M_{C}}
  \and
  \inferrule* [lab=abstraction] {} {{M_{F}} \bc (x)M_{P} }
  \and
  \inferrule* [lab=concretion] {} {{M_{C}} \bc \langle M_{P} \rangle }
  \and \\
  \inferrule* [lab=process] {} {{M_{P}} \bc M_{N} \;| \;P|M_{P} }
\end{mathpar}

\begin{definition}[contextual application] Given a context $M$, and
  process $P$, we define the \emph{contextual application}, $M[P] :=
  M\{P/\Box\}$. That is, the contextual application of M to P is the
  substitution of $P$ for $\Box$ in $M$.
\end{definition}

$\meaningof{-} : L \to \mathcal{P}(\pi)$

\begin{mathpar}
  \inferrule* [lab=collection] {} {\meaningof{true} = \pi, \and \meaningof{~E} = \pi \setminus \meaningof{E}, \and \meaningof{E_{1} \& E_{2}} = \meaningof{E_{1}} \cap \meaningof{E_{2}}}
\end{mathpar}

\begin{mathpar}
  \inferrule* [lab=structure] {} {\meaningof{0} = \{ P \in \pi | P \equiv 0 \}, \and \\ \meaningof{E_1 | E_2} = \{ P \in \pi | P \equiv P_{1} | P_{2}, P_{1} \in \meaningof{E_{1}}, P_{2} \in \meaningof{E_2}\} }
\end{mathpar}

\begin{mathpar}
 \inferrule* [lab=behavior] {} {\meaningof{\langle a?b \rangle E} = \{ P \in \pi | P \equiv Q | u?(y)P', \\ \and \\\\ \and \\ \;\;\; u \in \meaningof{a}, \forall z.P'\{z/y\} \in \meaningof{E\{z/b\}}\}, \and \\ \meaningof{a!E} = \{ P \in \pi | P \equiv Q | x!\langle P' \rangle, x \in \meaningof{a} P' \in \meaningof{E}\} }
\end{mathpar}

\begin{mathpar}
 \inferrule* [lab=nominal] {} {\meaningof{\quotep{E}} = \{ \quotep{P} \in \quotep{\pi} | P \in \meaningof{E} \}, \and \meaningof{\quotep{P}} = \{ \quotep{Q} \in \quotep{\pi} | P \equiv Q \} \and \\ \meaningof{@\quotep{E}} = \{ P \in \pi | P \equiv @x, x \in \meaningof{E} \}}
\end{mathpar}

\begin{eqnarray*}
  \\
  \meaningof{-} : TS \to ST
\end{eqnarray*}

\begin{eqnarray*}
  \\
  L : TS \to ST
\end{eqnarray*}

\begin{eqnarray*}
  \\
  P \models E \iff P \in \meaningof{E}
\end{eqnarray*}

\begin{eqnarray*}
  P \approx_{L} Q \iff \forall E \in L. P \models E \iff Q \models E
\end{eqnarray*}

\begin{eqnarray*}
  P \approx_{K} Q
\end{eqnarray*}

\begin{eqnarray*}
  P \approx Q
\end{eqnarray*}

$\approx_{K} = \approx = \approx_{L}$

\subsubsection{Contextual duality}

Note that contexts extend the quotation operation to a family of
operations from processes to names. Given a context, $M$, we can
define a \emph{nominal context}, $\quotep{M}$ by $\quotep{M}[P] :=
\quotep{M[P]}$. To foreshadow what is to come we observe that these
operations enjoy a duality with processes very much like the duality
between vectors and maps from vectors to scalars.

Further, because the calculus is essentially higher-order, we have a
correspondence between contexts and processes. More specifically,
given a name $x$ and a context $M$ we can construct $M^{*}_{x}$ such
that 

\begin{mathpar}
  M^{*}_{x} | \lift{x}{P} \red M[P]
\end{mathpar}

namely,

\begin{mathpar}
  M^{*}_{x} := x?(u).M[\dropn{u}]
\end{mathpar}

The dependence of $M^{*}_{x}$ on a name makes it an abstraction, 

\begin{mathpar}
  M^{*} := (x)x?(u).M[\dropn{u}]
\end{mathpar}

\subsection{Additional notation}

It will sometimes be convenient to denote the process a name
quotes. We already have the notation $x = \quotep{P}$, but it will be
convenient to introduce an alternate notation, $\procn{x}$, when we
want to emphasize the connection to the use of the name. Note that, by
virtue of name equivalence, $\quotep{\procn{x}} \nameeq x$; so, the
notation is consistent with previous definitions.

Further, because names have structure it is possible to effect
substitutions on the basis of that structure. This means we need to
upgrade our notation for substitutions, which we accomplish by
adapting comprehension notation. Thus,

\begin{mathpar}
  P\{ y / x : x \in S \}
\end{mathpar}

is interpreted to mean the process derived from P by replacing (in a
capture-avoiding manner) each occurrence of $x$ in $S$ by $y$. For example,

\begin{mathpar}
  P\{ \quotep{\procn{x}|\procn{x}} / x : x \in \freenames{P} \}
\end{mathpar}

will replace each (occurrence) of a free name $x$ in $P$ by
$\quotep{\procn{x}|\procn{x}}$.

Also, we will avail ourselves of the notation $x^{L}$ and $x^{R}$ to
denote injections of a name into disjoint copies of the name
space. There are numerous ways to accomplish this. One example can be
found in \cite{MeredithR05}. This notation overloads to vectors of
names: $\vec{x}^{\pi} := (x_{i}^{\pi} \; : \; 0 \leq i < |\vec{x}| )$ where $\pi \in \{L,R\}$.

We also use $P^{\Box} := P|\Box$.

In \cite{MeredithR05} an interpretation of the new operator is
given. It turns out that there are several possible interpretations
all enjoying the requisite algebraic properties of the operator (see
\cite{milner91polyadicpi}). We will therefore make liberal use of
$(\nu\; \vec{x})P$.

% subsection the_syntax_and_semantics_of_the_notation_system (end)   

\input{qm2pi.qmops} 

\input{qm2pi.sterngerlach} 

\input{qm2pi.metric} 

% section concurrent_process_calculi (end)

%\input{qm2pi.proofsketch}

% section proof sketch (end)

%\input{qm2pi.slviaknots} 

% section spatial logic via knots (end)

\input{qm2pi.conclusion}

% section conclusion (end)

%\input{qm2pi.dtcodes} 

% section wiring algorithm (end)

\input{qm2pi.ack} 

% section acknowledgments (end)

\newpage


\bibliographystyle{plain}   
\bibliography{../../biblios/main.bib}

\input{qm2pi.rhodetails}

\end{document}



% section proof sketch (end)

%\section{Unlikely characters: spatial logic for
  knots}\label{sub:characteristic_formulae} % (fold)

Associated to the mobile process calculi are a family of logics known
as the Hennessy-Milner logics. These logics typically enjoy a
semantics interpreting formulae as sets of processes that when
factored through the encoding outlined above allows an identification
of classes of knots with logical formulae. In the context of this
encoding the sub-family known as the spatial logics \cite{CairesC03}
\cite{CairesC04} \cite{Caires04} are of particular interest providing
several important features for expressing and reasoning about
properties (i.e. classes) of knots. We hint here at how this may be done.

%\begin{description}
%\item [structural connectives] 
\subsubsection{Structural connectives} The spatial logics enjoy
structural connectives corresponding, at the logical level, to the
parallel composition ($P | Q$) and new name ($(\nu \; x)P$)
connectives for processes. As illustrated in the examples below, these
connectives are extremely expressive given the shape of our encoding.
%\item [decideable satisfaction]

\subsubsection{Decideable satisfaction}
In \cite{Caires04} the satisfaction relation is shown to be decideable
for a rich class of processes. It further turns out that the image of
the our encoding is a proper subset of that class. This result
provides the basis for an algorithm by which to search for knots
enjoying a given property.
%\item [characteristic formulae]

\subsubsection{Characteristic formulae}
In the same paper \cite{Caires04} , Caires presents a means of calculating
characteristic formulae, selecting equivalence classes of processes
up to a pre--specified depth limit on the support set of names. Composed with our
encoding, this characteristic formula can be used to select
characteristic formulae for knots.
%\end{description}

\subsubsection{Spatial logic formulae}

The grammar below (segmented for comprehension) summarizes the syntax
of spatial logic formulae. We employ illustrative examples in the
sequel to provide an intuitive understanding of their meaning
referring the reader to \cite{Caires04} for a more detailed explication
of the semantics.

\begin{mathpar}
  \inferrule* [lab=boolean] {} {{A,B} \bc T \;|\; \neg A \;|\; A \wedge B \;|\; \eta = \eta'}
  \and
  \inferrule* [lab=spatial] {} {|\; \pzero \;|\; A | B \;|\; x \text{\textregistered} A \;|\; \forall x . A \;|\;  H x . A}
  \and
  \inferrule* [lab=behavioral] {} {|\; \alpha . A}
  \and 
  \inferrule* [lab=recursion] {} {|\; X(\vec{u}) \;|\; \mu X(\vec{u}) . A}
  \and
  \inferrule* [lab=action] {} {\alpha \bc \langle x?(\vec{y}) \rangle \;|\; \langle x!(\vec{y}) \rangle \;|\; \langle \tau \rangle}
  \and 
  \inferrule* [lab=name] {} {\eta \bc x \;|\; \tau}
\end{mathpar} 

% subsection characteristic_formulae (end)   	 

\subsection{Example formulae}\label{sub:example_formulae_} % (fold)

\subsubsection{Crossing as formula.}
% 
% \begin{align*}
%   \frac{d}{dx} \sin x &= \cos x 
%   & \frac{d}{dx} e^x &= e^x \\
%   \frac{d}{dx} \cos x &= - \sin x 
%   & \frac{d}{dx} \log x &= \frac{1}{x} \\
% \end{align*} 

\begin{align*}
 \mu C(x_{0},x_{1},y_{0},y_{1},u).&(\langle x_{0}?(z) \rangle(\langle u! \rangle\langle y_{1}!z \rangle C(x_{0},x_{1},y_{0},y_{1},u)) & \\
  & \wedge \langle y_{1}?(z) \rangle (\langle u! \rangle \langle x_{0}!z \rangle C(x_{0},x_{1},y_{0},y_{1},u)) & \\
  & \wedge \langle x_{1}?(z) \rangle (\langle u? \rangle \langle y_{0}!z \rangle C(x_{0},x_{1},y_{0},y_{1},u)) & \\
  & \wedge \langle y_{0}?(z) \rangle (\langle u? \rangle \langle x_{1}!z \rangle C(x_{0},x_{1},y_{0},y_{1},u))) &
\end{align*}

The lexicographical similarity between the shape of this formulae and
the shape of definition of the process representing a crossing reveals
the intuitive meaning of this formulae. It describes the capabilities
of a process that has the right to represent a crossing. For example
it picks out processes that may perform an input on the port $x_0$ in
its initial menu of capabilities. What differentiates the formula
from the process, however, is that the crossing process is the
smallest candidate to satisfy the formula. Infinitely many other
processes -- with internal behavior hidden behind this interface, so
to speak -- also satisfy this formula. Even this simple formula,
then, can be seen to open a new view onto knots, providing a
computational interpretation of \emph{virtual} knots.

Note that this formula is derived by hand. A similar formula can be
derived by employing Caires' calculation of characteristic formula
\cite{Caires04} to the process representing a crossing. In light of
this discussion, we let
$\meaningof{C}_{\phi}(x0,x1,y0,y1,u)$ denote a formula specifying the
dynamics we wish to capture of a crossing. To guarantee we preserve
the shape of the interface and minimal semantics we demand that
$\meaningof{C}_{\phi}(x0,x1,y0,y1,u) \Rightarrow
\textbf{C}(x0,x1,y0,y1,u)$ where $\textbf{C}(x0,x1,y0,y1,u)$ denotes
the formula above.
                            
\subsubsection{Crossing number constraints.}
The moral content of the context lemma (Lemma \ref{context}) is that the notion of
``locality'' in the Reidemeister moves is effectively captured by the
parallel composition operator of the process calculus. This intuition
extends through the logic. Given a formula,
$\meaningof{C}_{\phi}(x0,x1,y0,y1,u)$, we can use the structural
connectives to specify constraints on crossing numbers, such as at
least $n$ crossings, or exactly $n$ crossings.
\begin{mathpar}
  \inferrule* [lab=at-least-n] {} { K^{\geq n}_{\phi}(\vec{xs},\vec{ys}) := \Pi_{i=0}^{n-1} Hu . \meaningof{C}_{\phi}(xs_i,ys_i,u) | T }
  \and 
  \inferrule* [lab=exactly-n] {} { K^{= n}_{\phi}(\vec{xs},\vec{ys}) := \Pi_{i=0}^{n-1} Hu . \meaningof{C}_{\phi}(xs_i,ys_i,u) | \neg (\forall x_0,y_0,x_1,y_1,u . \meaningof{C}_{\phi}(x_0,y_0,x_1,y_1,u) | T) }
\end{mathpar}

To round out this section, recall that the encoding of an $n$-crossing
knot decomposes into a parallel composition of $n$ \emph{copies} of a
crossing process together with a wiring harness. To specify different
knot classes with the same crossing number amounts to specifying
logical constraints on the wiring harness. In the interest of space,
we defer examples to a forthcoming paper. Suffice it to say that both
the conditions ``alternating knot'' and ``contains the tangle
corresponding to 5/3'' are expressible. For example, it is possible to
calculate the characteristic formula of a process corresponding to the
tangle 5/3 and conjoin it into the classifying formula via the
composition connective of the logic.

Finally, we wish to observe that it is entirely within reason to
contemplate a more domain-specific version of spatial logic tailored
to the shape of processes in the image of the encoding. Such a
domain-specific logic would have a better claim to the title formal
language of knot properties.

% subsection example_formulae_ (end)

% section knots_as_processes (end) 

% section spatial logic via knots (end)

\section{Conclusions and future work}

\paragraph{Testing physical space}
You, gentle reader, may wonder why of all the theorems to be proved
given this set up we pick the one above. In some sense it's hardly
central to quantum mechanics. We see it as central in the sense that
it firmly establishes a notion of physical space arising from a notion
of the equivalence of behavior. Relating bisimulation to a metric is a
big step forward, but one is faced with interpreting the relationship
of that metric space to something more physical. Quantum mechanical
notions of ``physical'' space are still far from intuitive, but by
relating this idea of distance as testing to calculations that predict
physical circumstances we are making a not insignificant step forward
toward an understanding of the physical space we inhabit as
essentially dynamic.

\paragraph{Effectivity and simulation}
One of the observations we have yet to make is that the entire program
spelled out here is effective. We have built various interpreters for
the reflective calculus at work in this interpretation. In principle,
then, we can simulate quantum mechanics on a computer. The place where
the simulation may lose fidelity is the infinitely branching summation
for the annihilator.

In this connection i also want to point out that the evaluation style
calculation of the inner product puts the non-determinism of the
summation right at the heart of measurement. This suggests that
Milner's original reduction-based formulation of the dynamics of his
calculi in terms of sums was not just notationally suggestive of a
notion of measure-and-continue but captured some significant part of
the physics.

\paragraph{Quantum continuations}
In light of this last observation i want to point out that the
predominant account of quantum mechanics is missing a key aspect of a
truly compositional story of the physical situation. In a real lab,
when a measurement is made the observation can be made to feed into
another device that then makes another measurement conditioned on the
results of the first. This means that after the superposition was
collapsed the entire experimental set up remained in
superposition. While QM offers a means of writing this down it doesn't
quite line up well with the well-trodden formulation of computation
and continuation that we see so succinctly expressed in Milner's
calculi. This suggests that there might be advantages to this account
of dynamics waiting to be explored.

\paragraph{Quantum logic}
In this connection, we also note that by virtue of having the
Hennessy-Milner construction, we can pull the construction through the
interpretation of QM. This gives us a natural candidate for a quantum
logic that enjoys an extremely tight connection with it's domain of
interpretation, making the construction much less ad hoc (rather it is
the image of functor!).

\paragraph{Quantum probabiity}
i have questions about the basis of the interpretation of inner
product as probability amplitude. In particular, using which
axiomatization of probability theory does the notion of probability
amplitude earn the right to be so dubbed? In other words, where is the
proof that the operation for calculating a probability amplitude (and
then squaring) satisfies the axioms of what it means to calculate a
probability? Even if such a proof exists (i have yet to find it in the
literature), i wonder if it might not be possible to turn things on
their heads. Can we view the calculation of the probability amplitude
as an axiomatization of probability? If so, then the definition we
give for calculating probability amplitude may provide the basis for
an \emph{effective} theory of probability.

\paragraph{Quantum vs ``biological'' information}
Finally, i want to conclude with a more philosophical observation. At
a recent workshop in which QM was a predominant topic i noticed
something about quantum information. The speaker was giving a riveting
discussion of axiomatic QM and showing how properties of ``no
cloning'' and ``no deleting'' emerged as consequences of the
axiomatization. Theorems of this form are necessary to give us a sense
of confidence that our axioms characterize the physical theory. What
struck me, though, was that if quantum information is neither erasable
nor replicable it is markedly different from \emph{life}. Two of the
things we know about life is that

\begin{itemize}
  \item it ends;
  \item to gain some measure of persistence, to transcend it's
    finitude it is imminently copyable.
\end{itemize}

Both of these qualities are summarized succinctly in the aphorism: all
flesh is grass. For me these two kinds of ``information'' -- call them
quantum and biological -- are end points on a spectrum of strategies
for persistence. At one end, we have those curious entities that enjoy
uniqueness and permanence; at the other, we have those who in the face
of a certain end and an uncertain present make a go of passing
something on. To me one of the more remarkable aspects of the latter
strategy is that in the presence of noise (and certain features of
copying) we get a kind of dynamism, a chance for improvement against a
given persistent condition.

% subsection other_calculi_other_bisimulations_and_geometry_as_behavior (end)




% section conclusion (end)

%\documentclass[12pt]{llncs}
%\documentclass{jktr}

\usepackage[pdftex]{hyperref}                   
\usepackage {listings}
\usepackage {mathpartir}
\usepackage{bcprules}
%\usepackage{listings}
                       
\usepackage{graphicx} 
%\usepackage[margins=2.5cm,nohead,nofoot]{geometry}
%\usepackage{geometry}
\usepackage{amsfonts}
\usepackage{amstext}
\usepackage{latexsym}
\usepackage{amssymb}
\usepackage{color}


%\include{myPreamble}
\include{qm2pi.local} 

%\ifpdf
%\usepackage[pdftex]{graphicx}
%\else
%\usepackage{graphicx}
%\fi

 % \ifpdf
%  \usepackage{pdfsync}
%  \if


%\title{Brief Article}
%\author{David F. Snyder}
%\author{L.G. Meredith}

%\address{Dept. of Math., Texas State University--San Marcos, San Marcos, TX 78666}
       
\pagestyle{empty}


\begin{document}

\lstset{language=[Objective]Caml,frame=shadowbox}

\input{qm2pi.front}

% section front matter (end)

\input{qm2pi.intro} 
 
% section introduction (end)

% \input{qm2pi.knotations} 

% section notation (end)

\input{qm2pi.process.calculi} 

% section concurrent_process_calculi_and_spatial_logics_ (end)
    
%\input{qm2pi.knots2pi} 

%\input{qm2pi.trefoil} 

%\input{qm2pi.mainthm} 

% subsection basic_interpretation (end)

%\input{qm2pi.rho.presentation} 
\subsection{The syntax and semantics of the notation system}\label{sub:the_syntax_and_semantics_of_the_notation_system} % (fold)

We now summarize a technical presentation of the calculus that
embodies our theory of dynamics. The typical presentation of such a
calculus follows the style of giving generators and relations on
them. The grammar, below, describing term constructors, freely
generates the set of processes, $\Proc$. This set is then quotiented
by a relation known as structural congruence and it is over this set
that the notion of dynamics is expressed. This presentation is
essentially that of \cite{MeredithR05} with the addition of
polyadicity and summation. For readability we have relegated some of
the technical subtleties to an appendix.

\subsubsection{Process grammar}\label{subsub:process_grammar}

\begin{mathpar}
  \inferrule* [lab=synchronization] {} {{M} \bc \pzero \;|\; x?F \;|\; x!C }
  \and
  \inferrule* [lab=abstraction] {} {{F} \bc (x)P}
  \and
  \inferrule* [lab=concretion] {} {{C} \bc \langle Q \rangle}
  \and
  \inferrule* [lab=process] {} {{P,Q} \bc M \;| \;P|Q \;|\; @{x}}
  \and
  \inferrule* [lab=name] {} {{x} \bc \quotep{P}}
\end{mathpar} 

Note that $\vec{x}$ (resp. $\vec{P}$) denotes a vector of names
(resp. processes) of length $|\vec{x}|$ (resp. $|\vec{P}|$). We adopt
the following useful abbreviations.

\begin{mathpar}
   x?(\vec{y}).P := x.(\vec{y})P \and  x\clift{\vec{P}} := x.\clift{\vec{P}}
   \and x!(y) := \lift{x}{\dropn{y}}
   \and \Pi_{i=0}^{n-1}P_i := P_0 | \ldots | P_{n-1}
\end{mathpar}

\subsubsection{Structural congruence}

\paragraph{Free and bound names and alpha-equivalence.} At the
core of structural equivalence is alpha-equivalence which identifies
process that are the same up to a change of variable. Formally, we
recognize the distinction between free and bound names. The free names
of a process, $\freenames{P}$, may be calculated recursively as
follows:

\begin{mathpar}
\freenames{\pzero} := \emptyset
  \and \\
  \freenames{x?(y).P} := \{ x \} \cup (\freenames{P} \setminus \{ y \})
  \and 
  \freenames{x!\langle P \rangle} := \{ x \} \cup \{ P \} 
  \and \\
  \freenames{P|Q} := \freenames{P} \cup \freenames{Q}
  \and \\
  \freenames{@{x}} := \{ x \}
\end{mathpar}

$\pi$
$\quotep{\pi}$

$\freenames{-} : \pi \to \mathcal{P}(\quotep{\pi})$

\begin{eqnarray*}
  \freenames{\pzero} & := & \emptyset \\
  \freenames{x?(y).P} & := & \{ x \} \cup (\freenames{P} \setminus \{ y \}) \\
  \freenames{x!\langle P \rangle} & := & \{ x \} \cup \{ P \} \\
  \freenames{P|Q} & := & \freenames{P} \cup \freenames{Q} \\
  \freenames{\dropn{x}} & := & \{ x \}
\end{eqnarray*}

The bound names of a process, $\boundnames{P}$, are those names occurring in $P$
that are not free. For example, in $x?(y).0$, the name $x$ is free, while $y$ is bound.

\begin{mathpar}
  \inferrule* [lab=monoidal-laws] {} { P|Q \equiv Q|P \and P|0 \equiv P \and P|(Q|R) \equiv (P|Q)|R }
\end{mathpar}

\begin{mathpar}
  \inferrule* [lab=alpha-equivalence] {} { (x)P \equiv (y)P\{y/x\} \and y \not\in \freenames{P} }
\end{mathpar}

\begin{definition}
Then two processes, $P,Q$, are alpha-equivalent if $P = Q\{\vec{y}/\vec{x}\}$ for
some $\vec{x} \in \boundnames{Q},\vec{y} \in \boundnames{P}$, where $Q\{\vec{y}/\vec{x}\}$
denotes the capture-avoiding substitution of $\vec{y}$ for $\vec{x}$ in $Q$.
\end{definition}

\begin{definition}
  The {\em structural congruence} \cite{SangiorgiWalker} , $\equiv$,
  between processes is the least congruence containing
  alpha-equivalence, satisfying the abelian monoid laws
  (associativity, commutativity and $\pzero$ as identity) for parallel
  composition $|$ and for summation $+$.
\end{definition}

\subsection{Name equivalence}

We take name equivalence, written $\nameeq$, to be the smallest
equivalence relation generated by the following rules.

\begin{mathpar}
\inferrule*[lab=Quote-drop]
{ }
{ \quotep{@{x}} \nameeq x }

\inferrule*[lab=Struct-equiv]
{ P \scong Q }
{ \quotep{P} \nameeq \quotep{Q} }
\end{mathpar}

The astute reader will have noticed that the mutual recursion of names
and processes imposes a mutual recursion on alpha-equivalence and
structural equivalence via name-equivalence. Fortunately, all of this
works out pleasantly and we may calculate in the natural way, free of
concern. The reader interested in the details is referred to the
appendix \ref{appendix:rho_details}.

\subsection{Substitution}

We use $\Proc$ for the set of processes, $\QProc$ for the set of
names, and $\id{\{}\vec{y} / \vec{x} \id{\}}$ to denote partial maps,
$s : \QProc \rightarrow \QProc$. A map, $s$ lifts, uniquely, to a map
on process terms, $\widehat{s} : \Proc \rightarrow \Proc$ by the
following equations.

\begin{mathpar}
  (0) \psubstp{Q}{P} := 0 \\
  (R \juxtap S) \psubstp{Q}{P}
  :=    
  (R)\psubstp{Q}{P} \juxtap (S) \psubstp{Q}{P} \\
  (x?(y).R) \psubstp{Q}{P}    
  :=    
  (x)\substp{Q}{P} (z)\concat( (R \psubstn{z}{y}) \psubstp{Q}{P} ) \\
  (\lift{x}{R}) \psubstp{Q}{P}  
  :=
  \lift{(x)\substp{Q}{P}}{ R \psubstp{Q}{P} } \\
%   (\dropn{x})  \psubstp{Q}{P}       
%   := 
%   \left\{ 
%     \begin{array}{ccc} 
%       \dropn{\quotep{Q}} & & x \nameeq \quotep{P} \\
%       \dropn{x} & & otherwise \\
%     \end{array}
%   \right. 
  (\dropn{x})  \psubstp{Q}{P}       
  := 
  \left\{ 
    \begin{array}{ccc} 
      Q & & x \nameeq \quotep{P} \\
      \dropn{x} & & otherwise \\
    \end{array}
  \right.
\end{mathpar}
 

where

\begin{eqnarray}
  (x)\id{\{} \lpquote Q \rpquote / \lpquote P \rpquote \id{\}}            = 
  \left\{ 
    \begin{array}{ccc}
      \lpquote Q \rpquote & & x \nameeq \lpquote P \rpquote \\
      x & & otherwise \\
    \end{array}
  \right. \nonumber
\end{eqnarray}

and $z$ is chosen distinct from $\quotep{P}$, $\quotep{Q}$, the free
names in $Q$, and all the names in $R$. Our $\alpha$-equivalence will
be built in the standard way from this substitution.

\begin{remark}\label{rem:no_self_referential_names}
  One consequence of these definitions is that $\forall P. \quotep{P}
  \not\in \freenames{P}$.
\end{remark}

\subsection{ Dynamic quote: an example }

Anticipating something of what's to come, consider applying the
substitution, $\widehat{\id{\{}u / z \id{\}}}$, to the following pair
of processes, $\lift{w}{y!(z)}$ and $w[ \lpquote y!(z) \rpquote ]$.

\begin{eqnarray}
	\lift{w}{y!(z)}\widehat{\id{\{}u / z \id{\}}}
		& = &
		\lift{w}{y!(u)} \nonumber\\
	w[ \lpquote y!(z) \rpquote ] \widehat{ \id{\{}u / z \id{\}} }
		& = &
		w[ \lpquote y!(z) \rpquote ] \nonumber
\end{eqnarray}

Because the body of the process between quotes is impervious to
substitution, we get radically different answers. In fact, by
examining the first process in an input context,
e.g. $x?(z).\lift{w}{y!(z)}$, we see that the process under the lift
operator may be shaped by prefixed inputs binding a name inside it. In
this sense, the lift operator will be seen as a way to dynamically
construct processes before reifying them as names.

Finally equipped with these standard features we can present the
dynamics of the calculus.

\subsubsection{Operational semantics} 

Finally, we introduce the computational dynamics. What marks these
algebras as distinct from other more traditionally studied algebraic
structures, e.g. vector spaces or polynomial rings, is the manner in
which dynamics is captured. In traditional structures, dynamics is typically
expressed through morphisms between such structures, as in linear maps
between vector spaces or morphisms between rings. In algebras
associated with the semantics of computation, the dynamics is
expressed as part of the algebraic structure itself, through a
reduction reduction relation typically denoted by $\red$. Below, we
give a recursive presentation of this relation for the calculus used
in the encoding.

$\red \subseteq \pi \times \pi$
$\red : \pi \to \mathcal{P}(\pi)$

\begin{mathpar}
  \inferrule* [lab=Comm] { \textsf{match}( x_{src}, x_{trgt} ) } { x_{trgt}?(y)P \; | \; x_{src}!\langle {Q} \rangle \red P\{\quotep{Q}/y}\} }
  \and \\
  \inferrule* [lab=Par] {{P} \red {P}'} {{{P} | {Q}} \red {{P}' | {Q}}}
  \and
  \inferrule* [lab=Equiv]{{{P} \scong {P}'} \andalso {{P}' \red {Q}'} \andalso {{Q}' \scong {Q}}}{{P} \red {Q}}
\end{mathpar}

\begin{eqnarray*}
  match_{\equiv} (\quotep{P},\quotep{Q}) & := & P \equiv Q \\
  match_{\dagger}(\quotep{P},\quotep{Q}) & := & \forall R. P|Q \red^{*} R => R \red^{*} 0 \\
  match_{K}(\quotep{P},\quotep{Q}) & := & K \mbox{ for some context } K
\end{eqnarray*}

$u?(x)P | u!\langle Q \rangle \red P\{\quotep{Q}/x\}$

%We write $\wred$ for $\red^*$, and $P\red$ if $\exists Q $ such that $ P \red Q$.
We write $P\red$ if $\exists Q $ such that $ P \red Q$ and $P\not\red$, otherwise.

\section{Replication}

As mentioned before, it is known that replication (and hence
recursion) can be implemented in a higher-order process algebra
\cite{SangiorgiWalker}. As our first example of calculation with the
machinery thus far presented we give the construction explicitly in
the {\rhoc}.

\begin{eqnarray}
	D_{x} & := & \prefix{x}{y}{(\binpar{\outputp{x}{y}}{@{y}})} \nonumber\\
	\bangp_{x}{P} & := & \binpar{{x}!\langle{\binpar{D_{x}}{P}}\rangle}{D_{x}} \nonumber
\end{eqnarray}

\begin{eqnarray}
	\bangp_{x}{P} & & \nonumber\\
	=
	& {x}!\langle{(\prefix{x}{y}{(\outputp{x}{y} | @{y})) | P}}\rangle 
	      | \prefix{x}{y}{(\outputp{x}{y} | @{y})} & \nonumber\\
	\red
	& (\outputp{x}{y} | @{y})\substn{\quotep{(\prefix{x}{y}{(@{y} | \outputp{x}{y})) | P}}}{y} & \nonumber\\
	=
	& \outputp{x}{\quotep{(\prefix{x}{y}{(\outputp{x}{y} | @{y})) | P}}}
	  | {(\prefix{x}{y}{(\outputp{x}{y} | @{y})) | P}} & \nonumber\\
	\red
	& \ldots & \nonumber\\
	\red^*
	& P | P | \ldots & \nonumber
\end{eqnarray}

Of course, this encoding, as an implementation, runs away, unfolding
$\bangp{P}$ eagerly. A lazier and more implementable replication
operator, restricted to input-guarded processes, may be obtained as follows.

\begin{eqnarray}
\bangp{\prefix{u}{v}{P}} 
	:= 
	\binpar{\lift{x}{\prefix{u}{v}{(\binpar{D(x)}{P})}}}{D(x)} \nonumber
\end{eqnarray}

\begin{remark}
  Note that the lazier definition still does not deal with summation
  or mixed summation (i.e. sums over input and output). The reader is
  invited to construct definitions of replication that deal with these
  features. 

  Further, the definitions are parameterized in a name, $x$. Can you,
  gentle reader, make a definition that eliminates this parameter and
  guarantees no accidental interaction between the replication
  machinery and the process being replicated -- i.e. no accidental
  sharing of names used by the process to get its work done and the
  name(s) used by the replication to effect copying. This latter
  revision of the definition of replication is crucial to obtaining
  the expected identity $!!P \sim !P$.
\end{remark}

\begin{remark}\label{rem:paradoxical_combinator}
  The reader familiar with the lambda calculus will have noticed the
  similarity between $D$ and the paradoxical combinator.

  [Ed. note: the existence of this seems to suggest we have to be more
  restrictive on the set of processes and names we admit if we are to
  support no-cloning.]
\end{remark}

\subsubsection{Bisimulation}

The computational dynamics gives rise to another kind of equivalence,
the equivalence of computational behavior. As previously mentioned
this is typically captured \emph{via} some form of bisimulation.

% The notion we use in this paper is weak barbed bisimulation
% \cite{milner91polyadicpi}.

The notion we use in this paper is derived from weak barbed
bisimulation \cite{milner91polyadicpi}. 

\begin{definition}
An \emph{observation relation}, $\downarrow_{\mathcal N}$, over a set
of names, $\mathcal N$, is the smallest relation satisfying the rules
below.

\infrule[Out-barb]{y \in {\mathcal N}, \; x \nameeq y}
		  {\outputp{x}{v} \downarrow_{\mathcal N} x}
\infrule[Par-barb]{\mbox{$P\downarrow_{\mathcal N} x$ or $Q\downarrow_{\mathcal N} x$}}
		  {\binpar{P}{Q} \downarrow_{\mathcal N} x}

We write $P \Downarrow_{\mathcal N} x$ if there is $Q$ such that 
$P \wred Q$ and $Q \downarrow_{\mathcal N} x$.
\end{definition}

\begin{definition}
%\label{def.bbisim}
An  ${\mathcal N}$-\emph{barbed bisimulation} over a set of names, ${\mathcal N}$, is a symmetric binary relation 
${\mathcal S}_{\mathcal N}$ between agents such that $P\rel{S}_{\mathcal N}Q$ implies:
\begin{enumerate}
\item If $P \red P'$ then $Q \wred Q'$ and $P'\rel{S}_{\mathcal N} Q'$.
\item If $P\downarrow_{\mathcal N} x$, then $Q\Downarrow_{\mathcal N} x$.
\end{enumerate}
$P$ is ${\mathcal N}$-barbed bisimilar to $Q$, written
$P \wbbisim_{\mathcal N} Q$, if $P \rel{S}_{\mathcal N} Q$ for some ${\mathcal N}$-barbed bisimulation ${\mathcal S}_{\mathcal N}$.
\end{definition}

$\mathcal{R} \subseteq \pi \times \pi$

$P \mathcal{R} Q => \forall P'. P \red P' \Rightarrow \exists Q'. Q \red Q', P' \mathcal{R} Q'$

$P \vdash x \Rightarrow Q \vdash x$

\begin{mathpar}
  \inferrule*[lab=Out-barb]{x \nameeq y}{{y}!\langle{Q}\rangle \vdash x}
  \and
  \inferrule*[lab=Par-barb]{\mbox{$P\vdash x$ or $Q\vdash x$}}{\binpar{P}{Q} \vdash x}
\end{mathpar}

\subsubsection{Contexts}

One of the principle advantages of computational calculi like the
$\pi$-calculus is a well-defined notion of context,
contextual-equivalence and a correlation between
contextual-equivalence and notions of bisimulation. The notion of
context allows the decomposition of a process into (sub-)process and
its syntactic environment, its context. Thus, a context may be
thought of as a process with a ``hole'' (written $\Box$) in it. The
application of a context $M$ to a process $P$, written $M[P]$, is
tantamount to filling the hole in $M$ with $P$. In this paper we do
not need the full weight of this theory, but do make use of the notion
of context in the proof the main theorem. 

\begin{mathpar}
  \inferrule* [lab=summation] {} {{M_{M},M_{N}} \bc \Box \;|\; x.M_{A} \;|\; M_{M}+M_{N}}
  \and
  \inferrule* [lab=agent] {} {{M_{A}} \bc (\vec{x})M_{P} \;| \; \clift{P_0,\ldots,M_{P},\ldots,P_N}}
  \and \\
  \inferrule* [lab=process] {} {{M_{P}} \bc M_{N} \;| \;P|M_{P} }
\end{mathpar} 

\begin{mathpar}
  \inferrule* [lab=sychronization] {} {M_{N} \bc \Box \;|\; x?M_{F} \;|\; x!M_{C}}
  \and
  \inferrule* [lab=abstraction] {} {{M_{F}} \bc (x)M_{P} }
  \and
  \inferrule* [lab=concretion] {} {{M_{C}} \bc \langle M_{P} \rangle }
  \and \\
  \inferrule* [lab=process] {} {{M_{P}} \bc M_{N} \;| \;P|M_{P} }
\end{mathpar}

\begin{definition}[contextual application] Given a context $M$, and
  process $P$, we define the \emph{contextual application}, $M[P] :=
  M\{P/\Box\}$. That is, the contextual application of M to P is the
  substitution of $P$ for $\Box$ in $M$.
\end{definition}

$\meaningof{-} : L \to \mathcal{P}(\pi)$

\begin{mathpar}
  \inferrule* [lab=collection] {} {\meaningof{true} = \pi, \and \meaningof{~E} = \pi \setminus \meaningof{E}, \and \meaningof{E_{1} \& E_{2}} = \meaningof{E_{1}} \cap \meaningof{E_{2}}}
\end{mathpar}

\begin{mathpar}
  \inferrule* [lab=structure] {} {\meaningof{0} = \{ P \in \pi | P \equiv 0 \}, \and \\ \meaningof{E_1 | E_2} = \{ P \in \pi | P \equiv P_{1} | P_{2}, P_{1} \in \meaningof{E_{1}}, P_{2} \in \meaningof{E_2}\} }
\end{mathpar}

\begin{mathpar}
 \inferrule* [lab=behavior] {} {\meaningof{\langle a?b \rangle E} = \{ P \in \pi | P \equiv Q | u?(y)P', \\ \and \\\\ \and \\ \;\;\; u \in \meaningof{a}, \forall z.P'\{z/y\} \in \meaningof{E\{z/b\}}\}, \and \\ \meaningof{a!E} = \{ P \in \pi | P \equiv Q | x!\langle P' \rangle, x \in \meaningof{a} P' \in \meaningof{E}\} }
\end{mathpar}

\begin{mathpar}
 \inferrule* [lab=nominal] {} {\meaningof{\quotep{E}} = \{ \quotep{P} \in \quotep{\pi} | P \in \meaningof{E} \}, \and \meaningof{\quotep{P}} = \{ \quotep{Q} \in \quotep{\pi} | P \equiv Q \} \and \\ \meaningof{@\quotep{E}} = \{ P \in \pi | P \equiv @x, x \in \meaningof{E} \}}
\end{mathpar}

\begin{eqnarray*}
  \\
  \meaningof{-} : TS \to ST
\end{eqnarray*}

\begin{eqnarray*}
  \\
  L : TS \to ST
\end{eqnarray*}

\begin{eqnarray*}
  \\
  P \models E \iff P \in \meaningof{E}
\end{eqnarray*}

\begin{eqnarray*}
  P \approx_{L} Q \iff \forall E \in L. P \models E \iff Q \models E
\end{eqnarray*}

\begin{eqnarray*}
  P \approx_{K} Q
\end{eqnarray*}

\begin{eqnarray*}
  P \approx Q
\end{eqnarray*}

$\approx_{K} = \approx = \approx_{L}$

\subsubsection{Contextual duality}

Note that contexts extend the quotation operation to a family of
operations from processes to names. Given a context, $M$, we can
define a \emph{nominal context}, $\quotep{M}$ by $\quotep{M}[P] :=
\quotep{M[P]}$. To foreshadow what is to come we observe that these
operations enjoy a duality with processes very much like the duality
between vectors and maps from vectors to scalars.

Further, because the calculus is essentially higher-order, we have a
correspondence between contexts and processes. More specifically,
given a name $x$ and a context $M$ we can construct $M^{*}_{x}$ such
that 

\begin{mathpar}
  M^{*}_{x} | \lift{x}{P} \red M[P]
\end{mathpar}

namely,

\begin{mathpar}
  M^{*}_{x} := x?(u).M[\dropn{u}]
\end{mathpar}

The dependence of $M^{*}_{x}$ on a name makes it an abstraction, 

\begin{mathpar}
  M^{*} := (x)x?(u).M[\dropn{u}]
\end{mathpar}

\subsection{Additional notation}

It will sometimes be convenient to denote the process a name
quotes. We already have the notation $x = \quotep{P}$, but it will be
convenient to introduce an alternate notation, $\procn{x}$, when we
want to emphasize the connection to the use of the name. Note that, by
virtue of name equivalence, $\quotep{\procn{x}} \nameeq x$; so, the
notation is consistent with previous definitions.

Further, because names have structure it is possible to effect
substitutions on the basis of that structure. This means we need to
upgrade our notation for substitutions, which we accomplish by
adapting comprehension notation. Thus,

\begin{mathpar}
  P\{ y / x : x \in S \}
\end{mathpar}

is interpreted to mean the process derived from P by replacing (in a
capture-avoiding manner) each occurrence of $x$ in $S$ by $y$. For example,

\begin{mathpar}
  P\{ \quotep{\procn{x}|\procn{x}} / x : x \in \freenames{P} \}
\end{mathpar}

will replace each (occurrence) of a free name $x$ in $P$ by
$\quotep{\procn{x}|\procn{x}}$.

Also, we will avail ourselves of the notation $x^{L}$ and $x^{R}$ to
denote injections of a name into disjoint copies of the name
space. There are numerous ways to accomplish this. One example can be
found in \cite{MeredithR05}. This notation overloads to vectors of
names: $\vec{x}^{\pi} := (x_{i}^{\pi} \; : \; 0 \leq i < |\vec{x}| )$ where $\pi \in \{L,R\}$.

We also use $P^{\Box} := P|\Box$.

In \cite{MeredithR05} an interpretation of the new operator is
given. It turns out that there are several possible interpretations
all enjoying the requisite algebraic properties of the operator (see
\cite{milner91polyadicpi}). We will therefore make liberal use of
$(\nu\; \vec{x})P$.

% subsection the_syntax_and_semantics_of_the_notation_system (end)   

\input{qm2pi.qmops} 

\input{qm2pi.sterngerlach} 

\input{qm2pi.metric} 

% section concurrent_process_calculi (end)

%\input{qm2pi.proofsketch}

% section proof sketch (end)

%\input{qm2pi.slviaknots} 

% section spatial logic via knots (end)

\input{qm2pi.conclusion}

% section conclusion (end)

%\input{qm2pi.dtcodes} 

% section wiring algorithm (end)

\input{qm2pi.ack} 

% section acknowledgments (end)

\newpage


\bibliographystyle{plain}   
\bibliography{../../biblios/main.bib}

\input{qm2pi.rhodetails}

\end{document}

 

% section wiring algorithm (end)

\documentclass[12pt]{llncs}
%\documentclass{jktr}

\usepackage[pdftex]{hyperref}                   
\usepackage {listings}
\usepackage {mathpartir}
\usepackage{bcprules}
%\usepackage{listings}
                       
\usepackage{graphicx} 
%\usepackage[margins=2.5cm,nohead,nofoot]{geometry}
%\usepackage{geometry}
\usepackage{amsfonts}
\usepackage{amstext}
\usepackage{latexsym}
\usepackage{amssymb}
\usepackage{color}


%\include{myPreamble}
\include{qm2pi.local} 

%\ifpdf
%\usepackage[pdftex]{graphicx}
%\else
%\usepackage{graphicx}
%\fi

 % \ifpdf
%  \usepackage{pdfsync}
%  \if


%\title{Brief Article}
%\author{David F. Snyder}
%\author{L.G. Meredith}

%\address{Dept. of Math., Texas State University--San Marcos, San Marcos, TX 78666}
       
\pagestyle{empty}


\begin{document}

\lstset{language=[Objective]Caml,frame=shadowbox}

\input{qm2pi.front}

% section front matter (end)

\input{qm2pi.intro} 
 
% section introduction (end)

% \input{qm2pi.knotations} 

% section notation (end)

\input{qm2pi.process.calculi} 

% section concurrent_process_calculi_and_spatial_logics_ (end)
    
%\input{qm2pi.knots2pi} 

%\input{qm2pi.trefoil} 

%\input{qm2pi.mainthm} 

% subsection basic_interpretation (end)

%\input{qm2pi.rho.presentation} 
\subsection{The syntax and semantics of the notation system}\label{sub:the_syntax_and_semantics_of_the_notation_system} % (fold)

We now summarize a technical presentation of the calculus that
embodies our theory of dynamics. The typical presentation of such a
calculus follows the style of giving generators and relations on
them. The grammar, below, describing term constructors, freely
generates the set of processes, $\Proc$. This set is then quotiented
by a relation known as structural congruence and it is over this set
that the notion of dynamics is expressed. This presentation is
essentially that of \cite{MeredithR05} with the addition of
polyadicity and summation. For readability we have relegated some of
the technical subtleties to an appendix.

\subsubsection{Process grammar}\label{subsub:process_grammar}

\begin{mathpar}
  \inferrule* [lab=synchronization] {} {{M} \bc \pzero \;|\; x?F \;|\; x!C }
  \and
  \inferrule* [lab=abstraction] {} {{F} \bc (x)P}
  \and
  \inferrule* [lab=concretion] {} {{C} \bc \langle Q \rangle}
  \and
  \inferrule* [lab=process] {} {{P,Q} \bc M \;| \;P|Q \;|\; @{x}}
  \and
  \inferrule* [lab=name] {} {{x} \bc \quotep{P}}
\end{mathpar} 

Note that $\vec{x}$ (resp. $\vec{P}$) denotes a vector of names
(resp. processes) of length $|\vec{x}|$ (resp. $|\vec{P}|$). We adopt
the following useful abbreviations.

\begin{mathpar}
   x?(\vec{y}).P := x.(\vec{y})P \and  x\clift{\vec{P}} := x.\clift{\vec{P}}
   \and x!(y) := \lift{x}{\dropn{y}}
   \and \Pi_{i=0}^{n-1}P_i := P_0 | \ldots | P_{n-1}
\end{mathpar}

\subsubsection{Structural congruence}

\paragraph{Free and bound names and alpha-equivalence.} At the
core of structural equivalence is alpha-equivalence which identifies
process that are the same up to a change of variable. Formally, we
recognize the distinction between free and bound names. The free names
of a process, $\freenames{P}$, may be calculated recursively as
follows:

\begin{mathpar}
\freenames{\pzero} := \emptyset
  \and \\
  \freenames{x?(y).P} := \{ x \} \cup (\freenames{P} \setminus \{ y \})
  \and 
  \freenames{x!\langle P \rangle} := \{ x \} \cup \{ P \} 
  \and \\
  \freenames{P|Q} := \freenames{P} \cup \freenames{Q}
  \and \\
  \freenames{@{x}} := \{ x \}
\end{mathpar}

$\pi$
$\quotep{\pi}$

$\freenames{-} : \pi \to \mathcal{P}(\quotep{\pi})$

\begin{eqnarray*}
  \freenames{\pzero} & := & \emptyset \\
  \freenames{x?(y).P} & := & \{ x \} \cup (\freenames{P} \setminus \{ y \}) \\
  \freenames{x!\langle P \rangle} & := & \{ x \} \cup \{ P \} \\
  \freenames{P|Q} & := & \freenames{P} \cup \freenames{Q} \\
  \freenames{\dropn{x}} & := & \{ x \}
\end{eqnarray*}

The bound names of a process, $\boundnames{P}$, are those names occurring in $P$
that are not free. For example, in $x?(y).0$, the name $x$ is free, while $y$ is bound.

\begin{mathpar}
  \inferrule* [lab=monoidal-laws] {} { P|Q \equiv Q|P \and P|0 \equiv P \and P|(Q|R) \equiv (P|Q)|R }
\end{mathpar}

\begin{mathpar}
  \inferrule* [lab=alpha-equivalence] {} { (x)P \equiv (y)P\{y/x\} \and y \not\in \freenames{P} }
\end{mathpar}

\begin{definition}
Then two processes, $P,Q$, are alpha-equivalent if $P = Q\{\vec{y}/\vec{x}\}$ for
some $\vec{x} \in \boundnames{Q},\vec{y} \in \boundnames{P}$, where $Q\{\vec{y}/\vec{x}\}$
denotes the capture-avoiding substitution of $\vec{y}$ for $\vec{x}$ in $Q$.
\end{definition}

\begin{definition}
  The {\em structural congruence} \cite{SangiorgiWalker} , $\equiv$,
  between processes is the least congruence containing
  alpha-equivalence, satisfying the abelian monoid laws
  (associativity, commutativity and $\pzero$ as identity) for parallel
  composition $|$ and for summation $+$.
\end{definition}

\subsection{Name equivalence}

We take name equivalence, written $\nameeq$, to be the smallest
equivalence relation generated by the following rules.

\begin{mathpar}
\inferrule*[lab=Quote-drop]
{ }
{ \quotep{@{x}} \nameeq x }

\inferrule*[lab=Struct-equiv]
{ P \scong Q }
{ \quotep{P} \nameeq \quotep{Q} }
\end{mathpar}

The astute reader will have noticed that the mutual recursion of names
and processes imposes a mutual recursion on alpha-equivalence and
structural equivalence via name-equivalence. Fortunately, all of this
works out pleasantly and we may calculate in the natural way, free of
concern. The reader interested in the details is referred to the
appendix \ref{appendix:rho_details}.

\subsection{Substitution}

We use $\Proc$ for the set of processes, $\QProc$ for the set of
names, and $\id{\{}\vec{y} / \vec{x} \id{\}}$ to denote partial maps,
$s : \QProc \rightarrow \QProc$. A map, $s$ lifts, uniquely, to a map
on process terms, $\widehat{s} : \Proc \rightarrow \Proc$ by the
following equations.

\begin{mathpar}
  (0) \psubstp{Q}{P} := 0 \\
  (R \juxtap S) \psubstp{Q}{P}
  :=    
  (R)\psubstp{Q}{P} \juxtap (S) \psubstp{Q}{P} \\
  (x?(y).R) \psubstp{Q}{P}    
  :=    
  (x)\substp{Q}{P} (z)\concat( (R \psubstn{z}{y}) \psubstp{Q}{P} ) \\
  (\lift{x}{R}) \psubstp{Q}{P}  
  :=
  \lift{(x)\substp{Q}{P}}{ R \psubstp{Q}{P} } \\
%   (\dropn{x})  \psubstp{Q}{P}       
%   := 
%   \left\{ 
%     \begin{array}{ccc} 
%       \dropn{\quotep{Q}} & & x \nameeq \quotep{P} \\
%       \dropn{x} & & otherwise \\
%     \end{array}
%   \right. 
  (\dropn{x})  \psubstp{Q}{P}       
  := 
  \left\{ 
    \begin{array}{ccc} 
      Q & & x \nameeq \quotep{P} \\
      \dropn{x} & & otherwise \\
    \end{array}
  \right.
\end{mathpar}
 

where

\begin{eqnarray}
  (x)\id{\{} \lpquote Q \rpquote / \lpquote P \rpquote \id{\}}            = 
  \left\{ 
    \begin{array}{ccc}
      \lpquote Q \rpquote & & x \nameeq \lpquote P \rpquote \\
      x & & otherwise \\
    \end{array}
  \right. \nonumber
\end{eqnarray}

and $z$ is chosen distinct from $\quotep{P}$, $\quotep{Q}$, the free
names in $Q$, and all the names in $R$. Our $\alpha$-equivalence will
be built in the standard way from this substitution.

\begin{remark}\label{rem:no_self_referential_names}
  One consequence of these definitions is that $\forall P. \quotep{P}
  \not\in \freenames{P}$.
\end{remark}

\subsection{ Dynamic quote: an example }

Anticipating something of what's to come, consider applying the
substitution, $\widehat{\id{\{}u / z \id{\}}}$, to the following pair
of processes, $\lift{w}{y!(z)}$ and $w[ \lpquote y!(z) \rpquote ]$.

\begin{eqnarray}
	\lift{w}{y!(z)}\widehat{\id{\{}u / z \id{\}}}
		& = &
		\lift{w}{y!(u)} \nonumber\\
	w[ \lpquote y!(z) \rpquote ] \widehat{ \id{\{}u / z \id{\}} }
		& = &
		w[ \lpquote y!(z) \rpquote ] \nonumber
\end{eqnarray}

Because the body of the process between quotes is impervious to
substitution, we get radically different answers. In fact, by
examining the first process in an input context,
e.g. $x?(z).\lift{w}{y!(z)}$, we see that the process under the lift
operator may be shaped by prefixed inputs binding a name inside it. In
this sense, the lift operator will be seen as a way to dynamically
construct processes before reifying them as names.

Finally equipped with these standard features we can present the
dynamics of the calculus.

\subsubsection{Operational semantics} 

Finally, we introduce the computational dynamics. What marks these
algebras as distinct from other more traditionally studied algebraic
structures, e.g. vector spaces or polynomial rings, is the manner in
which dynamics is captured. In traditional structures, dynamics is typically
expressed through morphisms between such structures, as in linear maps
between vector spaces or morphisms between rings. In algebras
associated with the semantics of computation, the dynamics is
expressed as part of the algebraic structure itself, through a
reduction reduction relation typically denoted by $\red$. Below, we
give a recursive presentation of this relation for the calculus used
in the encoding.

$\red \subseteq \pi \times \pi$
$\red : \pi \to \mathcal{P}(\pi)$

\begin{mathpar}
  \inferrule* [lab=Comm] { \textsf{match}( x_{src}, x_{trgt} ) } { x_{trgt}?(y)P \; | \; x_{src}!\langle {Q} \rangle \red P\{\quotep{Q}/y}\} }
  \and \\
  \inferrule* [lab=Par] {{P} \red {P}'} {{{P} | {Q}} \red {{P}' | {Q}}}
  \and
  \inferrule* [lab=Equiv]{{{P} \scong {P}'} \andalso {{P}' \red {Q}'} \andalso {{Q}' \scong {Q}}}{{P} \red {Q}}
\end{mathpar}

\begin{eqnarray*}
  match_{\equiv} (\quotep{P},\quotep{Q}) & := & P \equiv Q \\
  match_{\dagger}(\quotep{P},\quotep{Q}) & := & \forall R. P|Q \red^{*} R => R \red^{*} 0 \\
  match_{K}(\quotep{P},\quotep{Q}) & := & K \mbox{ for some context } K
\end{eqnarray*}

$u?(x)P | u!\langle Q \rangle \red P\{\quotep{Q}/x\}$

%We write $\wred$ for $\red^*$, and $P\red$ if $\exists Q $ such that $ P \red Q$.
We write $P\red$ if $\exists Q $ such that $ P \red Q$ and $P\not\red$, otherwise.

\section{Replication}

As mentioned before, it is known that replication (and hence
recursion) can be implemented in a higher-order process algebra
\cite{SangiorgiWalker}. As our first example of calculation with the
machinery thus far presented we give the construction explicitly in
the {\rhoc}.

\begin{eqnarray}
	D_{x} & := & \prefix{x}{y}{(\binpar{\outputp{x}{y}}{@{y}})} \nonumber\\
	\bangp_{x}{P} & := & \binpar{{x}!\langle{\binpar{D_{x}}{P}}\rangle}{D_{x}} \nonumber
\end{eqnarray}

\begin{eqnarray}
	\bangp_{x}{P} & & \nonumber\\
	=
	& {x}!\langle{(\prefix{x}{y}{(\outputp{x}{y} | @{y})) | P}}\rangle 
	      | \prefix{x}{y}{(\outputp{x}{y} | @{y})} & \nonumber\\
	\red
	& (\outputp{x}{y} | @{y})\substn{\quotep{(\prefix{x}{y}{(@{y} | \outputp{x}{y})) | P}}}{y} & \nonumber\\
	=
	& \outputp{x}{\quotep{(\prefix{x}{y}{(\outputp{x}{y} | @{y})) | P}}}
	  | {(\prefix{x}{y}{(\outputp{x}{y} | @{y})) | P}} & \nonumber\\
	\red
	& \ldots & \nonumber\\
	\red^*
	& P | P | \ldots & \nonumber
\end{eqnarray}

Of course, this encoding, as an implementation, runs away, unfolding
$\bangp{P}$ eagerly. A lazier and more implementable replication
operator, restricted to input-guarded processes, may be obtained as follows.

\begin{eqnarray}
\bangp{\prefix{u}{v}{P}} 
	:= 
	\binpar{\lift{x}{\prefix{u}{v}{(\binpar{D(x)}{P})}}}{D(x)} \nonumber
\end{eqnarray}

\begin{remark}
  Note that the lazier definition still does not deal with summation
  or mixed summation (i.e. sums over input and output). The reader is
  invited to construct definitions of replication that deal with these
  features. 

  Further, the definitions are parameterized in a name, $x$. Can you,
  gentle reader, make a definition that eliminates this parameter and
  guarantees no accidental interaction between the replication
  machinery and the process being replicated -- i.e. no accidental
  sharing of names used by the process to get its work done and the
  name(s) used by the replication to effect copying. This latter
  revision of the definition of replication is crucial to obtaining
  the expected identity $!!P \sim !P$.
\end{remark}

\begin{remark}\label{rem:paradoxical_combinator}
  The reader familiar with the lambda calculus will have noticed the
  similarity between $D$ and the paradoxical combinator.

  [Ed. note: the existence of this seems to suggest we have to be more
  restrictive on the set of processes and names we admit if we are to
  support no-cloning.]
\end{remark}

\subsubsection{Bisimulation}

The computational dynamics gives rise to another kind of equivalence,
the equivalence of computational behavior. As previously mentioned
this is typically captured \emph{via} some form of bisimulation.

% The notion we use in this paper is weak barbed bisimulation
% \cite{milner91polyadicpi}.

The notion we use in this paper is derived from weak barbed
bisimulation \cite{milner91polyadicpi}. 

\begin{definition}
An \emph{observation relation}, $\downarrow_{\mathcal N}$, over a set
of names, $\mathcal N$, is the smallest relation satisfying the rules
below.

\infrule[Out-barb]{y \in {\mathcal N}, \; x \nameeq y}
		  {\outputp{x}{v} \downarrow_{\mathcal N} x}
\infrule[Par-barb]{\mbox{$P\downarrow_{\mathcal N} x$ or $Q\downarrow_{\mathcal N} x$}}
		  {\binpar{P}{Q} \downarrow_{\mathcal N} x}

We write $P \Downarrow_{\mathcal N} x$ if there is $Q$ such that 
$P \wred Q$ and $Q \downarrow_{\mathcal N} x$.
\end{definition}

\begin{definition}
%\label{def.bbisim}
An  ${\mathcal N}$-\emph{barbed bisimulation} over a set of names, ${\mathcal N}$, is a symmetric binary relation 
${\mathcal S}_{\mathcal N}$ between agents such that $P\rel{S}_{\mathcal N}Q$ implies:
\begin{enumerate}
\item If $P \red P'$ then $Q \wred Q'$ and $P'\rel{S}_{\mathcal N} Q'$.
\item If $P\downarrow_{\mathcal N} x$, then $Q\Downarrow_{\mathcal N} x$.
\end{enumerate}
$P$ is ${\mathcal N}$-barbed bisimilar to $Q$, written
$P \wbbisim_{\mathcal N} Q$, if $P \rel{S}_{\mathcal N} Q$ for some ${\mathcal N}$-barbed bisimulation ${\mathcal S}_{\mathcal N}$.
\end{definition}

$\mathcal{R} \subseteq \pi \times \pi$

$P \mathcal{R} Q => \forall P'. P \red P' \Rightarrow \exists Q'. Q \red Q', P' \mathcal{R} Q'$

$P \vdash x \Rightarrow Q \vdash x$

\begin{mathpar}
  \inferrule*[lab=Out-barb]{x \nameeq y}{{y}!\langle{Q}\rangle \vdash x}
  \and
  \inferrule*[lab=Par-barb]{\mbox{$P\vdash x$ or $Q\vdash x$}}{\binpar{P}{Q} \vdash x}
\end{mathpar}

\subsubsection{Contexts}

One of the principle advantages of computational calculi like the
$\pi$-calculus is a well-defined notion of context,
contextual-equivalence and a correlation between
contextual-equivalence and notions of bisimulation. The notion of
context allows the decomposition of a process into (sub-)process and
its syntactic environment, its context. Thus, a context may be
thought of as a process with a ``hole'' (written $\Box$) in it. The
application of a context $M$ to a process $P$, written $M[P]$, is
tantamount to filling the hole in $M$ with $P$. In this paper we do
not need the full weight of this theory, but do make use of the notion
of context in the proof the main theorem. 

\begin{mathpar}
  \inferrule* [lab=summation] {} {{M_{M},M_{N}} \bc \Box \;|\; x.M_{A} \;|\; M_{M}+M_{N}}
  \and
  \inferrule* [lab=agent] {} {{M_{A}} \bc (\vec{x})M_{P} \;| \; \clift{P_0,\ldots,M_{P},\ldots,P_N}}
  \and \\
  \inferrule* [lab=process] {} {{M_{P}} \bc M_{N} \;| \;P|M_{P} }
\end{mathpar} 

\begin{mathpar}
  \inferrule* [lab=sychronization] {} {M_{N} \bc \Box \;|\; x?M_{F} \;|\; x!M_{C}}
  \and
  \inferrule* [lab=abstraction] {} {{M_{F}} \bc (x)M_{P} }
  \and
  \inferrule* [lab=concretion] {} {{M_{C}} \bc \langle M_{P} \rangle }
  \and \\
  \inferrule* [lab=process] {} {{M_{P}} \bc M_{N} \;| \;P|M_{P} }
\end{mathpar}

\begin{definition}[contextual application] Given a context $M$, and
  process $P$, we define the \emph{contextual application}, $M[P] :=
  M\{P/\Box\}$. That is, the contextual application of M to P is the
  substitution of $P$ for $\Box$ in $M$.
\end{definition}

$\meaningof{-} : L \to \mathcal{P}(\pi)$

\begin{mathpar}
  \inferrule* [lab=collection] {} {\meaningof{true} = \pi, \and \meaningof{~E} = \pi \setminus \meaningof{E}, \and \meaningof{E_{1} \& E_{2}} = \meaningof{E_{1}} \cap \meaningof{E_{2}}}
\end{mathpar}

\begin{mathpar}
  \inferrule* [lab=structure] {} {\meaningof{0} = \{ P \in \pi | P \equiv 0 \}, \and \\ \meaningof{E_1 | E_2} = \{ P \in \pi | P \equiv P_{1} | P_{2}, P_{1} \in \meaningof{E_{1}}, P_{2} \in \meaningof{E_2}\} }
\end{mathpar}

\begin{mathpar}
 \inferrule* [lab=behavior] {} {\meaningof{\langle a?b \rangle E} = \{ P \in \pi | P \equiv Q | u?(y)P', \\ \and \\\\ \and \\ \;\;\; u \in \meaningof{a}, \forall z.P'\{z/y\} \in \meaningof{E\{z/b\}}\}, \and \\ \meaningof{a!E} = \{ P \in \pi | P \equiv Q | x!\langle P' \rangle, x \in \meaningof{a} P' \in \meaningof{E}\} }
\end{mathpar}

\begin{mathpar}
 \inferrule* [lab=nominal] {} {\meaningof{\quotep{E}} = \{ \quotep{P} \in \quotep{\pi} | P \in \meaningof{E} \}, \and \meaningof{\quotep{P}} = \{ \quotep{Q} \in \quotep{\pi} | P \equiv Q \} \and \\ \meaningof{@\quotep{E}} = \{ P \in \pi | P \equiv @x, x \in \meaningof{E} \}}
\end{mathpar}

\begin{eqnarray*}
  \\
  \meaningof{-} : TS \to ST
\end{eqnarray*}

\begin{eqnarray*}
  \\
  L : TS \to ST
\end{eqnarray*}

\begin{eqnarray*}
  \\
  P \models E \iff P \in \meaningof{E}
\end{eqnarray*}

\begin{eqnarray*}
  P \approx_{L} Q \iff \forall E \in L. P \models E \iff Q \models E
\end{eqnarray*}

\begin{eqnarray*}
  P \approx_{K} Q
\end{eqnarray*}

\begin{eqnarray*}
  P \approx Q
\end{eqnarray*}

$\approx_{K} = \approx = \approx_{L}$

\subsubsection{Contextual duality}

Note that contexts extend the quotation operation to a family of
operations from processes to names. Given a context, $M$, we can
define a \emph{nominal context}, $\quotep{M}$ by $\quotep{M}[P] :=
\quotep{M[P]}$. To foreshadow what is to come we observe that these
operations enjoy a duality with processes very much like the duality
between vectors and maps from vectors to scalars.

Further, because the calculus is essentially higher-order, we have a
correspondence between contexts and processes. More specifically,
given a name $x$ and a context $M$ we can construct $M^{*}_{x}$ such
that 

\begin{mathpar}
  M^{*}_{x} | \lift{x}{P} \red M[P]
\end{mathpar}

namely,

\begin{mathpar}
  M^{*}_{x} := x?(u).M[\dropn{u}]
\end{mathpar}

The dependence of $M^{*}_{x}$ on a name makes it an abstraction, 

\begin{mathpar}
  M^{*} := (x)x?(u).M[\dropn{u}]
\end{mathpar}

\subsection{Additional notation}

It will sometimes be convenient to denote the process a name
quotes. We already have the notation $x = \quotep{P}$, but it will be
convenient to introduce an alternate notation, $\procn{x}$, when we
want to emphasize the connection to the use of the name. Note that, by
virtue of name equivalence, $\quotep{\procn{x}} \nameeq x$; so, the
notation is consistent with previous definitions.

Further, because names have structure it is possible to effect
substitutions on the basis of that structure. This means we need to
upgrade our notation for substitutions, which we accomplish by
adapting comprehension notation. Thus,

\begin{mathpar}
  P\{ y / x : x \in S \}
\end{mathpar}

is interpreted to mean the process derived from P by replacing (in a
capture-avoiding manner) each occurrence of $x$ in $S$ by $y$. For example,

\begin{mathpar}
  P\{ \quotep{\procn{x}|\procn{x}} / x : x \in \freenames{P} \}
\end{mathpar}

will replace each (occurrence) of a free name $x$ in $P$ by
$\quotep{\procn{x}|\procn{x}}$.

Also, we will avail ourselves of the notation $x^{L}$ and $x^{R}$ to
denote injections of a name into disjoint copies of the name
space. There are numerous ways to accomplish this. One example can be
found in \cite{MeredithR05}. This notation overloads to vectors of
names: $\vec{x}^{\pi} := (x_{i}^{\pi} \; : \; 0 \leq i < |\vec{x}| )$ where $\pi \in \{L,R\}$.

We also use $P^{\Box} := P|\Box$.

In \cite{MeredithR05} an interpretation of the new operator is
given. It turns out that there are several possible interpretations
all enjoying the requisite algebraic properties of the operator (see
\cite{milner91polyadicpi}). We will therefore make liberal use of
$(\nu\; \vec{x})P$.

% subsection the_syntax_and_semantics_of_the_notation_system (end)   

\input{qm2pi.qmops} 

\input{qm2pi.sterngerlach} 

\input{qm2pi.metric} 

% section concurrent_process_calculi (end)

%\input{qm2pi.proofsketch}

% section proof sketch (end)

%\input{qm2pi.slviaknots} 

% section spatial logic via knots (end)

\input{qm2pi.conclusion}

% section conclusion (end)

%\input{qm2pi.dtcodes} 

% section wiring algorithm (end)

\input{qm2pi.ack} 

% section acknowledgments (end)

\newpage


\bibliographystyle{plain}   
\bibliography{../../biblios/main.bib}

\input{qm2pi.rhodetails}

\end{document}

 

% section acknowledgments (end)

\newpage


\bibliographystyle{plain}   
\bibliography{../../biblios/main.bib}

\documentclass[12pt]{llncs}
%\documentclass{jktr}

\usepackage[pdftex]{hyperref}                   
\usepackage {listings}
\usepackage {mathpartir}
\usepackage{bcprules}
%\usepackage{listings}
                       
\usepackage{graphicx} 
%\usepackage[margins=2.5cm,nohead,nofoot]{geometry}
%\usepackage{geometry}
\usepackage{amsfonts}
\usepackage{amstext}
\usepackage{latexsym}
\usepackage{amssymb}
\usepackage{color}


%\include{myPreamble}
\include{qm2pi.local} 

%\ifpdf
%\usepackage[pdftex]{graphicx}
%\else
%\usepackage{graphicx}
%\fi

 % \ifpdf
%  \usepackage{pdfsync}
%  \if


%\title{Brief Article}
%\author{David F. Snyder}
%\author{L.G. Meredith}

%\address{Dept. of Math., Texas State University--San Marcos, San Marcos, TX 78666}
       
\pagestyle{empty}


\begin{document}

\lstset{language=[Objective]Caml,frame=shadowbox}

\input{qm2pi.front}

% section front matter (end)

\input{qm2pi.intro} 
 
% section introduction (end)

% \input{qm2pi.knotations} 

% section notation (end)

\input{qm2pi.process.calculi} 

% section concurrent_process_calculi_and_spatial_logics_ (end)
    
%\input{qm2pi.knots2pi} 

%\input{qm2pi.trefoil} 

%\input{qm2pi.mainthm} 

% subsection basic_interpretation (end)

%\input{qm2pi.rho.presentation} 
\subsection{The syntax and semantics of the notation system}\label{sub:the_syntax_and_semantics_of_the_notation_system} % (fold)

We now summarize a technical presentation of the calculus that
embodies our theory of dynamics. The typical presentation of such a
calculus follows the style of giving generators and relations on
them. The grammar, below, describing term constructors, freely
generates the set of processes, $\Proc$. This set is then quotiented
by a relation known as structural congruence and it is over this set
that the notion of dynamics is expressed. This presentation is
essentially that of \cite{MeredithR05} with the addition of
polyadicity and summation. For readability we have relegated some of
the technical subtleties to an appendix.

\subsubsection{Process grammar}\label{subsub:process_grammar}

\begin{mathpar}
  \inferrule* [lab=synchronization] {} {{M} \bc \pzero \;|\; x?F \;|\; x!C }
  \and
  \inferrule* [lab=abstraction] {} {{F} \bc (x)P}
  \and
  \inferrule* [lab=concretion] {} {{C} \bc \langle Q \rangle}
  \and
  \inferrule* [lab=process] {} {{P,Q} \bc M \;| \;P|Q \;|\; @{x}}
  \and
  \inferrule* [lab=name] {} {{x} \bc \quotep{P}}
\end{mathpar} 

Note that $\vec{x}$ (resp. $\vec{P}$) denotes a vector of names
(resp. processes) of length $|\vec{x}|$ (resp. $|\vec{P}|$). We adopt
the following useful abbreviations.

\begin{mathpar}
   x?(\vec{y}).P := x.(\vec{y})P \and  x\clift{\vec{P}} := x.\clift{\vec{P}}
   \and x!(y) := \lift{x}{\dropn{y}}
   \and \Pi_{i=0}^{n-1}P_i := P_0 | \ldots | P_{n-1}
\end{mathpar}

\subsubsection{Structural congruence}

\paragraph{Free and bound names and alpha-equivalence.} At the
core of structural equivalence is alpha-equivalence which identifies
process that are the same up to a change of variable. Formally, we
recognize the distinction between free and bound names. The free names
of a process, $\freenames{P}$, may be calculated recursively as
follows:

\begin{mathpar}
\freenames{\pzero} := \emptyset
  \and \\
  \freenames{x?(y).P} := \{ x \} \cup (\freenames{P} \setminus \{ y \})
  \and 
  \freenames{x!\langle P \rangle} := \{ x \} \cup \{ P \} 
  \and \\
  \freenames{P|Q} := \freenames{P} \cup \freenames{Q}
  \and \\
  \freenames{@{x}} := \{ x \}
\end{mathpar}

$\pi$
$\quotep{\pi}$

$\freenames{-} : \pi \to \mathcal{P}(\quotep{\pi})$

\begin{eqnarray*}
  \freenames{\pzero} & := & \emptyset \\
  \freenames{x?(y).P} & := & \{ x \} \cup (\freenames{P} \setminus \{ y \}) \\
  \freenames{x!\langle P \rangle} & := & \{ x \} \cup \{ P \} \\
  \freenames{P|Q} & := & \freenames{P} \cup \freenames{Q} \\
  \freenames{\dropn{x}} & := & \{ x \}
\end{eqnarray*}

The bound names of a process, $\boundnames{P}$, are those names occurring in $P$
that are not free. For example, in $x?(y).0$, the name $x$ is free, while $y$ is bound.

\begin{mathpar}
  \inferrule* [lab=monoidal-laws] {} { P|Q \equiv Q|P \and P|0 \equiv P \and P|(Q|R) \equiv (P|Q)|R }
\end{mathpar}

\begin{mathpar}
  \inferrule* [lab=alpha-equivalence] {} { (x)P \equiv (y)P\{y/x\} \and y \not\in \freenames{P} }
\end{mathpar}

\begin{definition}
Then two processes, $P,Q$, are alpha-equivalent if $P = Q\{\vec{y}/\vec{x}\}$ for
some $\vec{x} \in \boundnames{Q},\vec{y} \in \boundnames{P}$, where $Q\{\vec{y}/\vec{x}\}$
denotes the capture-avoiding substitution of $\vec{y}$ for $\vec{x}$ in $Q$.
\end{definition}

\begin{definition}
  The {\em structural congruence} \cite{SangiorgiWalker} , $\equiv$,
  between processes is the least congruence containing
  alpha-equivalence, satisfying the abelian monoid laws
  (associativity, commutativity and $\pzero$ as identity) for parallel
  composition $|$ and for summation $+$.
\end{definition}

\subsection{Name equivalence}

We take name equivalence, written $\nameeq$, to be the smallest
equivalence relation generated by the following rules.

\begin{mathpar}
\inferrule*[lab=Quote-drop]
{ }
{ \quotep{@{x}} \nameeq x }

\inferrule*[lab=Struct-equiv]
{ P \scong Q }
{ \quotep{P} \nameeq \quotep{Q} }
\end{mathpar}

The astute reader will have noticed that the mutual recursion of names
and processes imposes a mutual recursion on alpha-equivalence and
structural equivalence via name-equivalence. Fortunately, all of this
works out pleasantly and we may calculate in the natural way, free of
concern. The reader interested in the details is referred to the
appendix \ref{appendix:rho_details}.

\subsection{Substitution}

We use $\Proc$ for the set of processes, $\QProc$ for the set of
names, and $\id{\{}\vec{y} / \vec{x} \id{\}}$ to denote partial maps,
$s : \QProc \rightarrow \QProc$. A map, $s$ lifts, uniquely, to a map
on process terms, $\widehat{s} : \Proc \rightarrow \Proc$ by the
following equations.

\begin{mathpar}
  (0) \psubstp{Q}{P} := 0 \\
  (R \juxtap S) \psubstp{Q}{P}
  :=    
  (R)\psubstp{Q}{P} \juxtap (S) \psubstp{Q}{P} \\
  (x?(y).R) \psubstp{Q}{P}    
  :=    
  (x)\substp{Q}{P} (z)\concat( (R \psubstn{z}{y}) \psubstp{Q}{P} ) \\
  (\lift{x}{R}) \psubstp{Q}{P}  
  :=
  \lift{(x)\substp{Q}{P}}{ R \psubstp{Q}{P} } \\
%   (\dropn{x})  \psubstp{Q}{P}       
%   := 
%   \left\{ 
%     \begin{array}{ccc} 
%       \dropn{\quotep{Q}} & & x \nameeq \quotep{P} \\
%       \dropn{x} & & otherwise \\
%     \end{array}
%   \right. 
  (\dropn{x})  \psubstp{Q}{P}       
  := 
  \left\{ 
    \begin{array}{ccc} 
      Q & & x \nameeq \quotep{P} \\
      \dropn{x} & & otherwise \\
    \end{array}
  \right.
\end{mathpar}
 

where

\begin{eqnarray}
  (x)\id{\{} \lpquote Q \rpquote / \lpquote P \rpquote \id{\}}            = 
  \left\{ 
    \begin{array}{ccc}
      \lpquote Q \rpquote & & x \nameeq \lpquote P \rpquote \\
      x & & otherwise \\
    \end{array}
  \right. \nonumber
\end{eqnarray}

and $z$ is chosen distinct from $\quotep{P}$, $\quotep{Q}$, the free
names in $Q$, and all the names in $R$. Our $\alpha$-equivalence will
be built in the standard way from this substitution.

\begin{remark}\label{rem:no_self_referential_names}
  One consequence of these definitions is that $\forall P. \quotep{P}
  \not\in \freenames{P}$.
\end{remark}

\subsection{ Dynamic quote: an example }

Anticipating something of what's to come, consider applying the
substitution, $\widehat{\id{\{}u / z \id{\}}}$, to the following pair
of processes, $\lift{w}{y!(z)}$ and $w[ \lpquote y!(z) \rpquote ]$.

\begin{eqnarray}
	\lift{w}{y!(z)}\widehat{\id{\{}u / z \id{\}}}
		& = &
		\lift{w}{y!(u)} \nonumber\\
	w[ \lpquote y!(z) \rpquote ] \widehat{ \id{\{}u / z \id{\}} }
		& = &
		w[ \lpquote y!(z) \rpquote ] \nonumber
\end{eqnarray}

Because the body of the process between quotes is impervious to
substitution, we get radically different answers. In fact, by
examining the first process in an input context,
e.g. $x?(z).\lift{w}{y!(z)}$, we see that the process under the lift
operator may be shaped by prefixed inputs binding a name inside it. In
this sense, the lift operator will be seen as a way to dynamically
construct processes before reifying them as names.

Finally equipped with these standard features we can present the
dynamics of the calculus.

\subsubsection{Operational semantics} 

Finally, we introduce the computational dynamics. What marks these
algebras as distinct from other more traditionally studied algebraic
structures, e.g. vector spaces or polynomial rings, is the manner in
which dynamics is captured. In traditional structures, dynamics is typically
expressed through morphisms between such structures, as in linear maps
between vector spaces or morphisms between rings. In algebras
associated with the semantics of computation, the dynamics is
expressed as part of the algebraic structure itself, through a
reduction reduction relation typically denoted by $\red$. Below, we
give a recursive presentation of this relation for the calculus used
in the encoding.

$\red \subseteq \pi \times \pi$
$\red : \pi \to \mathcal{P}(\pi)$

\begin{mathpar}
  \inferrule* [lab=Comm] { \textsf{match}( x_{src}, x_{trgt} ) } { x_{trgt}?(y)P \; | \; x_{src}!\langle {Q} \rangle \red P\{\quotep{Q}/y}\} }
  \and \\
  \inferrule* [lab=Par] {{P} \red {P}'} {{{P} | {Q}} \red {{P}' | {Q}}}
  \and
  \inferrule* [lab=Equiv]{{{P} \scong {P}'} \andalso {{P}' \red {Q}'} \andalso {{Q}' \scong {Q}}}{{P} \red {Q}}
\end{mathpar}

\begin{eqnarray*}
  match_{\equiv} (\quotep{P},\quotep{Q}) & := & P \equiv Q \\
  match_{\dagger}(\quotep{P},\quotep{Q}) & := & \forall R. P|Q \red^{*} R => R \red^{*} 0 \\
  match_{K}(\quotep{P},\quotep{Q}) & := & K \mbox{ for some context } K
\end{eqnarray*}

$u?(x)P | u!\langle Q \rangle \red P\{\quotep{Q}/x\}$

%We write $\wred$ for $\red^*$, and $P\red$ if $\exists Q $ such that $ P \red Q$.
We write $P\red$ if $\exists Q $ such that $ P \red Q$ and $P\not\red$, otherwise.

\section{Replication}

As mentioned before, it is known that replication (and hence
recursion) can be implemented in a higher-order process algebra
\cite{SangiorgiWalker}. As our first example of calculation with the
machinery thus far presented we give the construction explicitly in
the {\rhoc}.

\begin{eqnarray}
	D_{x} & := & \prefix{x}{y}{(\binpar{\outputp{x}{y}}{@{y}})} \nonumber\\
	\bangp_{x}{P} & := & \binpar{{x}!\langle{\binpar{D_{x}}{P}}\rangle}{D_{x}} \nonumber
\end{eqnarray}

\begin{eqnarray}
	\bangp_{x}{P} & & \nonumber\\
	=
	& {x}!\langle{(\prefix{x}{y}{(\outputp{x}{y} | @{y})) | P}}\rangle 
	      | \prefix{x}{y}{(\outputp{x}{y} | @{y})} & \nonumber\\
	\red
	& (\outputp{x}{y} | @{y})\substn{\quotep{(\prefix{x}{y}{(@{y} | \outputp{x}{y})) | P}}}{y} & \nonumber\\
	=
	& \outputp{x}{\quotep{(\prefix{x}{y}{(\outputp{x}{y} | @{y})) | P}}}
	  | {(\prefix{x}{y}{(\outputp{x}{y} | @{y})) | P}} & \nonumber\\
	\red
	& \ldots & \nonumber\\
	\red^*
	& P | P | \ldots & \nonumber
\end{eqnarray}

Of course, this encoding, as an implementation, runs away, unfolding
$\bangp{P}$ eagerly. A lazier and more implementable replication
operator, restricted to input-guarded processes, may be obtained as follows.

\begin{eqnarray}
\bangp{\prefix{u}{v}{P}} 
	:= 
	\binpar{\lift{x}{\prefix{u}{v}{(\binpar{D(x)}{P})}}}{D(x)} \nonumber
\end{eqnarray}

\begin{remark}
  Note that the lazier definition still does not deal with summation
  or mixed summation (i.e. sums over input and output). The reader is
  invited to construct definitions of replication that deal with these
  features. 

  Further, the definitions are parameterized in a name, $x$. Can you,
  gentle reader, make a definition that eliminates this parameter and
  guarantees no accidental interaction between the replication
  machinery and the process being replicated -- i.e. no accidental
  sharing of names used by the process to get its work done and the
  name(s) used by the replication to effect copying. This latter
  revision of the definition of replication is crucial to obtaining
  the expected identity $!!P \sim !P$.
\end{remark}

\begin{remark}\label{rem:paradoxical_combinator}
  The reader familiar with the lambda calculus will have noticed the
  similarity between $D$ and the paradoxical combinator.

  [Ed. note: the existence of this seems to suggest we have to be more
  restrictive on the set of processes and names we admit if we are to
  support no-cloning.]
\end{remark}

\subsubsection{Bisimulation}

The computational dynamics gives rise to another kind of equivalence,
the equivalence of computational behavior. As previously mentioned
this is typically captured \emph{via} some form of bisimulation.

% The notion we use in this paper is weak barbed bisimulation
% \cite{milner91polyadicpi}.

The notion we use in this paper is derived from weak barbed
bisimulation \cite{milner91polyadicpi}. 

\begin{definition}
An \emph{observation relation}, $\downarrow_{\mathcal N}$, over a set
of names, $\mathcal N$, is the smallest relation satisfying the rules
below.

\infrule[Out-barb]{y \in {\mathcal N}, \; x \nameeq y}
		  {\outputp{x}{v} \downarrow_{\mathcal N} x}
\infrule[Par-barb]{\mbox{$P\downarrow_{\mathcal N} x$ or $Q\downarrow_{\mathcal N} x$}}
		  {\binpar{P}{Q} \downarrow_{\mathcal N} x}

We write $P \Downarrow_{\mathcal N} x$ if there is $Q$ such that 
$P \wred Q$ and $Q \downarrow_{\mathcal N} x$.
\end{definition}

\begin{definition}
%\label{def.bbisim}
An  ${\mathcal N}$-\emph{barbed bisimulation} over a set of names, ${\mathcal N}$, is a symmetric binary relation 
${\mathcal S}_{\mathcal N}$ between agents such that $P\rel{S}_{\mathcal N}Q$ implies:
\begin{enumerate}
\item If $P \red P'$ then $Q \wred Q'$ and $P'\rel{S}_{\mathcal N} Q'$.
\item If $P\downarrow_{\mathcal N} x$, then $Q\Downarrow_{\mathcal N} x$.
\end{enumerate}
$P$ is ${\mathcal N}$-barbed bisimilar to $Q$, written
$P \wbbisim_{\mathcal N} Q$, if $P \rel{S}_{\mathcal N} Q$ for some ${\mathcal N}$-barbed bisimulation ${\mathcal S}_{\mathcal N}$.
\end{definition}

$\mathcal{R} \subseteq \pi \times \pi$

$P \mathcal{R} Q => \forall P'. P \red P' \Rightarrow \exists Q'. Q \red Q', P' \mathcal{R} Q'$

$P \vdash x \Rightarrow Q \vdash x$

\begin{mathpar}
  \inferrule*[lab=Out-barb]{x \nameeq y}{{y}!\langle{Q}\rangle \vdash x}
  \and
  \inferrule*[lab=Par-barb]{\mbox{$P\vdash x$ or $Q\vdash x$}}{\binpar{P}{Q} \vdash x}
\end{mathpar}

\subsubsection{Contexts}

One of the principle advantages of computational calculi like the
$\pi$-calculus is a well-defined notion of context,
contextual-equivalence and a correlation between
contextual-equivalence and notions of bisimulation. The notion of
context allows the decomposition of a process into (sub-)process and
its syntactic environment, its context. Thus, a context may be
thought of as a process with a ``hole'' (written $\Box$) in it. The
application of a context $M$ to a process $P$, written $M[P]$, is
tantamount to filling the hole in $M$ with $P$. In this paper we do
not need the full weight of this theory, but do make use of the notion
of context in the proof the main theorem. 

\begin{mathpar}
  \inferrule* [lab=summation] {} {{M_{M},M_{N}} \bc \Box \;|\; x.M_{A} \;|\; M_{M}+M_{N}}
  \and
  \inferrule* [lab=agent] {} {{M_{A}} \bc (\vec{x})M_{P} \;| \; \clift{P_0,\ldots,M_{P},\ldots,P_N}}
  \and \\
  \inferrule* [lab=process] {} {{M_{P}} \bc M_{N} \;| \;P|M_{P} }
\end{mathpar} 

\begin{mathpar}
  \inferrule* [lab=sychronization] {} {M_{N} \bc \Box \;|\; x?M_{F} \;|\; x!M_{C}}
  \and
  \inferrule* [lab=abstraction] {} {{M_{F}} \bc (x)M_{P} }
  \and
  \inferrule* [lab=concretion] {} {{M_{C}} \bc \langle M_{P} \rangle }
  \and \\
  \inferrule* [lab=process] {} {{M_{P}} \bc M_{N} \;| \;P|M_{P} }
\end{mathpar}

\begin{definition}[contextual application] Given a context $M$, and
  process $P$, we define the \emph{contextual application}, $M[P] :=
  M\{P/\Box\}$. That is, the contextual application of M to P is the
  substitution of $P$ for $\Box$ in $M$.
\end{definition}

$\meaningof{-} : L \to \mathcal{P}(\pi)$

\begin{mathpar}
  \inferrule* [lab=collection] {} {\meaningof{true} = \pi, \and \meaningof{~E} = \pi \setminus \meaningof{E}, \and \meaningof{E_{1} \& E_{2}} = \meaningof{E_{1}} \cap \meaningof{E_{2}}}
\end{mathpar}

\begin{mathpar}
  \inferrule* [lab=structure] {} {\meaningof{0} = \{ P \in \pi | P \equiv 0 \}, \and \\ \meaningof{E_1 | E_2} = \{ P \in \pi | P \equiv P_{1} | P_{2}, P_{1} \in \meaningof{E_{1}}, P_{2} \in \meaningof{E_2}\} }
\end{mathpar}

\begin{mathpar}
 \inferrule* [lab=behavior] {} {\meaningof{\langle a?b \rangle E} = \{ P \in \pi | P \equiv Q | u?(y)P', \\ \and \\\\ \and \\ \;\;\; u \in \meaningof{a}, \forall z.P'\{z/y\} \in \meaningof{E\{z/b\}}\}, \and \\ \meaningof{a!E} = \{ P \in \pi | P \equiv Q | x!\langle P' \rangle, x \in \meaningof{a} P' \in \meaningof{E}\} }
\end{mathpar}

\begin{mathpar}
 \inferrule* [lab=nominal] {} {\meaningof{\quotep{E}} = \{ \quotep{P} \in \quotep{\pi} | P \in \meaningof{E} \}, \and \meaningof{\quotep{P}} = \{ \quotep{Q} \in \quotep{\pi} | P \equiv Q \} \and \\ \meaningof{@\quotep{E}} = \{ P \in \pi | P \equiv @x, x \in \meaningof{E} \}}
\end{mathpar}

\begin{eqnarray*}
  \\
  \meaningof{-} : TS \to ST
\end{eqnarray*}

\begin{eqnarray*}
  \\
  L : TS \to ST
\end{eqnarray*}

\begin{eqnarray*}
  \\
  P \models E \iff P \in \meaningof{E}
\end{eqnarray*}

\begin{eqnarray*}
  P \approx_{L} Q \iff \forall E \in L. P \models E \iff Q \models E
\end{eqnarray*}

\begin{eqnarray*}
  P \approx_{K} Q
\end{eqnarray*}

\begin{eqnarray*}
  P \approx Q
\end{eqnarray*}

$\approx_{K} = \approx = \approx_{L}$

\subsubsection{Contextual duality}

Note that contexts extend the quotation operation to a family of
operations from processes to names. Given a context, $M$, we can
define a \emph{nominal context}, $\quotep{M}$ by $\quotep{M}[P] :=
\quotep{M[P]}$. To foreshadow what is to come we observe that these
operations enjoy a duality with processes very much like the duality
between vectors and maps from vectors to scalars.

Further, because the calculus is essentially higher-order, we have a
correspondence between contexts and processes. More specifically,
given a name $x$ and a context $M$ we can construct $M^{*}_{x}$ such
that 

\begin{mathpar}
  M^{*}_{x} | \lift{x}{P} \red M[P]
\end{mathpar}

namely,

\begin{mathpar}
  M^{*}_{x} := x?(u).M[\dropn{u}]
\end{mathpar}

The dependence of $M^{*}_{x}$ on a name makes it an abstraction, 

\begin{mathpar}
  M^{*} := (x)x?(u).M[\dropn{u}]
\end{mathpar}

\subsection{Additional notation}

It will sometimes be convenient to denote the process a name
quotes. We already have the notation $x = \quotep{P}$, but it will be
convenient to introduce an alternate notation, $\procn{x}$, when we
want to emphasize the connection to the use of the name. Note that, by
virtue of name equivalence, $\quotep{\procn{x}} \nameeq x$; so, the
notation is consistent with previous definitions.

Further, because names have structure it is possible to effect
substitutions on the basis of that structure. This means we need to
upgrade our notation for substitutions, which we accomplish by
adapting comprehension notation. Thus,

\begin{mathpar}
  P\{ y / x : x \in S \}
\end{mathpar}

is interpreted to mean the process derived from P by replacing (in a
capture-avoiding manner) each occurrence of $x$ in $S$ by $y$. For example,

\begin{mathpar}
  P\{ \quotep{\procn{x}|\procn{x}} / x : x \in \freenames{P} \}
\end{mathpar}

will replace each (occurrence) of a free name $x$ in $P$ by
$\quotep{\procn{x}|\procn{x}}$.

Also, we will avail ourselves of the notation $x^{L}$ and $x^{R}$ to
denote injections of a name into disjoint copies of the name
space. There are numerous ways to accomplish this. One example can be
found in \cite{MeredithR05}. This notation overloads to vectors of
names: $\vec{x}^{\pi} := (x_{i}^{\pi} \; : \; 0 \leq i < |\vec{x}| )$ where $\pi \in \{L,R\}$.

We also use $P^{\Box} := P|\Box$.

In \cite{MeredithR05} an interpretation of the new operator is
given. It turns out that there are several possible interpretations
all enjoying the requisite algebraic properties of the operator (see
\cite{milner91polyadicpi}). We will therefore make liberal use of
$(\nu\; \vec{x})P$.

% subsection the_syntax_and_semantics_of_the_notation_system (end)   

\input{qm2pi.qmops} 

\input{qm2pi.sterngerlach} 

\input{qm2pi.metric} 

% section concurrent_process_calculi (end)

%\input{qm2pi.proofsketch}

% section proof sketch (end)

%\input{qm2pi.slviaknots} 

% section spatial logic via knots (end)

\input{qm2pi.conclusion}

% section conclusion (end)

%\input{qm2pi.dtcodes} 

% section wiring algorithm (end)

\input{qm2pi.ack} 

% section acknowledgments (end)

\newpage


\bibliographystyle{plain}   
\bibliography{../../biblios/main.bib}

\input{qm2pi.rhodetails}

\end{document}



\end{document}



% section proof sketch (end)

%\section{Unlikely characters: spatial logic for
  knots}\label{sub:characteristic_formulae} % (fold)

Associated to the mobile process calculi are a family of logics known
as the Hennessy-Milner logics. These logics typically enjoy a
semantics interpreting formulae as sets of processes that when
factored through the encoding outlined above allows an identification
of classes of knots with logical formulae. In the context of this
encoding the sub-family known as the spatial logics \cite{CairesC03}
\cite{CairesC04} \cite{Caires04} are of particular interest providing
several important features for expressing and reasoning about
properties (i.e. classes) of knots. We hint here at how this may be done.

%\begin{description}
%\item [structural connectives] 
\subsubsection{Structural connectives} The spatial logics enjoy
structural connectives corresponding, at the logical level, to the
parallel composition ($P | Q$) and new name ($(\nu \; x)P$)
connectives for processes. As illustrated in the examples below, these
connectives are extremely expressive given the shape of our encoding.
%\item [decideable satisfaction]

\subsubsection{Decideable satisfaction}
In \cite{Caires04} the satisfaction relation is shown to be decideable
for a rich class of processes. It further turns out that the image of
the our encoding is a proper subset of that class. This result
provides the basis for an algorithm by which to search for knots
enjoying a given property.
%\item [characteristic formulae]

\subsubsection{Characteristic formulae}
In the same paper \cite{Caires04} , Caires presents a means of calculating
characteristic formulae, selecting equivalence classes of processes
up to a pre--specified depth limit on the support set of names. Composed with our
encoding, this characteristic formula can be used to select
characteristic formulae for knots.
%\end{description}

\subsubsection{Spatial logic formulae}

The grammar below (segmented for comprehension) summarizes the syntax
of spatial logic formulae. We employ illustrative examples in the
sequel to provide an intuitive understanding of their meaning
referring the reader to \cite{Caires04} for a more detailed explication
of the semantics.

\begin{mathpar}
  \inferrule* [lab=boolean] {} {{A,B} \bc T \;|\; \neg A \;|\; A \wedge B \;|\; \eta = \eta'}
  \and
  \inferrule* [lab=spatial] {} {|\; \pzero \;|\; A | B \;|\; x \text{\textregistered} A \;|\; \forall x . A \;|\;  H x . A}
  \and
  \inferrule* [lab=behavioral] {} {|\; \alpha . A}
  \and 
  \inferrule* [lab=recursion] {} {|\; X(\vec{u}) \;|\; \mu X(\vec{u}) . A}
  \and
  \inferrule* [lab=action] {} {\alpha \bc \langle x?(\vec{y}) \rangle \;|\; \langle x!(\vec{y}) \rangle \;|\; \langle \tau \rangle}
  \and 
  \inferrule* [lab=name] {} {\eta \bc x \;|\; \tau}
\end{mathpar} 

% subsection characteristic_formulae (end)   	 

\subsection{Example formulae}\label{sub:example_formulae_} % (fold)

\subsubsection{Crossing as formula.}
% 
% \begin{align*}
%   \frac{d}{dx} \sin x &= \cos x 
%   & \frac{d}{dx} e^x &= e^x \\
%   \frac{d}{dx} \cos x &= - \sin x 
%   & \frac{d}{dx} \log x &= \frac{1}{x} \\
% \end{align*} 

\begin{align*}
 \mu C(x_{0},x_{1},y_{0},y_{1},u).&(\langle x_{0}?(z) \rangle(\langle u! \rangle\langle y_{1}!z \rangle C(x_{0},x_{1},y_{0},y_{1},u)) & \\
  & \wedge \langle y_{1}?(z) \rangle (\langle u! \rangle \langle x_{0}!z \rangle C(x_{0},x_{1},y_{0},y_{1},u)) & \\
  & \wedge \langle x_{1}?(z) \rangle (\langle u? \rangle \langle y_{0}!z \rangle C(x_{0},x_{1},y_{0},y_{1},u)) & \\
  & \wedge \langle y_{0}?(z) \rangle (\langle u? \rangle \langle x_{1}!z \rangle C(x_{0},x_{1},y_{0},y_{1},u))) &
\end{align*}

The lexicographical similarity between the shape of this formulae and
the shape of definition of the process representing a crossing reveals
the intuitive meaning of this formulae. It describes the capabilities
of a process that has the right to represent a crossing. For example
it picks out processes that may perform an input on the port $x_0$ in
its initial menu of capabilities. What differentiates the formula
from the process, however, is that the crossing process is the
smallest candidate to satisfy the formula. Infinitely many other
processes -- with internal behavior hidden behind this interface, so
to speak -- also satisfy this formula. Even this simple formula,
then, can be seen to open a new view onto knots, providing a
computational interpretation of \emph{virtual} knots.

Note that this formula is derived by hand. A similar formula can be
derived by employing Caires' calculation of characteristic formula
\cite{Caires04} to the process representing a crossing. In light of
this discussion, we let
$\meaningof{C}_{\phi}(x0,x1,y0,y1,u)$ denote a formula specifying the
dynamics we wish to capture of a crossing. To guarantee we preserve
the shape of the interface and minimal semantics we demand that
$\meaningof{C}_{\phi}(x0,x1,y0,y1,u) \Rightarrow
\textbf{C}(x0,x1,y0,y1,u)$ where $\textbf{C}(x0,x1,y0,y1,u)$ denotes
the formula above.
                            
\subsubsection{Crossing number constraints.}
The moral content of the context lemma (Lemma \ref{context}) is that the notion of
``locality'' in the Reidemeister moves is effectively captured by the
parallel composition operator of the process calculus. This intuition
extends through the logic. Given a formula,
$\meaningof{C}_{\phi}(x0,x1,y0,y1,u)$, we can use the structural
connectives to specify constraints on crossing numbers, such as at
least $n$ crossings, or exactly $n$ crossings.
\begin{mathpar}
  \inferrule* [lab=at-least-n] {} { K^{\geq n}_{\phi}(\vec{xs},\vec{ys}) := \Pi_{i=0}^{n-1} Hu . \meaningof{C}_{\phi}(xs_i,ys_i,u) | T }
  \and 
  \inferrule* [lab=exactly-n] {} { K^{= n}_{\phi}(\vec{xs},\vec{ys}) := \Pi_{i=0}^{n-1} Hu . \meaningof{C}_{\phi}(xs_i,ys_i,u) | \neg (\forall x_0,y_0,x_1,y_1,u . \meaningof{C}_{\phi}(x_0,y_0,x_1,y_1,u) | T) }
\end{mathpar}

To round out this section, recall that the encoding of an $n$-crossing
knot decomposes into a parallel composition of $n$ \emph{copies} of a
crossing process together with a wiring harness. To specify different
knot classes with the same crossing number amounts to specifying
logical constraints on the wiring harness. In the interest of space,
we defer examples to a forthcoming paper. Suffice it to say that both
the conditions ``alternating knot'' and ``contains the tangle
corresponding to 5/3'' are expressible. For example, it is possible to
calculate the characteristic formula of a process corresponding to the
tangle 5/3 and conjoin it into the classifying formula via the
composition connective of the logic.

Finally, we wish to observe that it is entirely within reason to
contemplate a more domain-specific version of spatial logic tailored
to the shape of processes in the image of the encoding. Such a
domain-specific logic would have a better claim to the title formal
language of knot properties.

% subsection example_formulae_ (end)

% section knots_as_processes (end) 

% section spatial logic via knots (end)

\section{Conclusions and future work}

\paragraph{Testing physical space}
You, gentle reader, may wonder why of all the theorems to be proved
given this set up we pick the one above. In some sense it's hardly
central to quantum mechanics. We see it as central in the sense that
it firmly establishes a notion of physical space arising from a notion
of the equivalence of behavior. Relating bisimulation to a metric is a
big step forward, but one is faced with interpreting the relationship
of that metric space to something more physical. Quantum mechanical
notions of ``physical'' space are still far from intuitive, but by
relating this idea of distance as testing to calculations that predict
physical circumstances we are making a not insignificant step forward
toward an understanding of the physical space we inhabit as
essentially dynamic.

\paragraph{Effectivity and simulation}
One of the observations we have yet to make is that the entire program
spelled out here is effective. We have built various interpreters for
the reflective calculus at work in this interpretation. In principle,
then, we can simulate quantum mechanics on a computer. The place where
the simulation may lose fidelity is the infinitely branching summation
for the annihilator.

In this connection i also want to point out that the evaluation style
calculation of the inner product puts the non-determinism of the
summation right at the heart of measurement. This suggests that
Milner's original reduction-based formulation of the dynamics of his
calculi in terms of sums was not just notationally suggestive of a
notion of measure-and-continue but captured some significant part of
the physics.

\paragraph{Quantum continuations}
In light of this last observation i want to point out that the
predominant account of quantum mechanics is missing a key aspect of a
truly compositional story of the physical situation. In a real lab,
when a measurement is made the observation can be made to feed into
another device that then makes another measurement conditioned on the
results of the first. This means that after the superposition was
collapsed the entire experimental set up remained in
superposition. While QM offers a means of writing this down it doesn't
quite line up well with the well-trodden formulation of computation
and continuation that we see so succinctly expressed in Milner's
calculi. This suggests that there might be advantages to this account
of dynamics waiting to be explored.

\paragraph{Quantum logic}
In this connection, we also note that by virtue of having the
Hennessy-Milner construction, we can pull the construction through the
interpretation of QM. This gives us a natural candidate for a quantum
logic that enjoys an extremely tight connection with it's domain of
interpretation, making the construction much less ad hoc (rather it is
the image of functor!).

\paragraph{Quantum probabiity}
i have questions about the basis of the interpretation of inner
product as probability amplitude. In particular, using which
axiomatization of probability theory does the notion of probability
amplitude earn the right to be so dubbed? In other words, where is the
proof that the operation for calculating a probability amplitude (and
then squaring) satisfies the axioms of what it means to calculate a
probability? Even if such a proof exists (i have yet to find it in the
literature), i wonder if it might not be possible to turn things on
their heads. Can we view the calculation of the probability amplitude
as an axiomatization of probability? If so, then the definition we
give for calculating probability amplitude may provide the basis for
an \emph{effective} theory of probability.

\paragraph{Quantum vs ``biological'' information}
Finally, i want to conclude with a more philosophical observation. At
a recent workshop in which QM was a predominant topic i noticed
something about quantum information. The speaker was giving a riveting
discussion of axiomatic QM and showing how properties of ``no
cloning'' and ``no deleting'' emerged as consequences of the
axiomatization. Theorems of this form are necessary to give us a sense
of confidence that our axioms characterize the physical theory. What
struck me, though, was that if quantum information is neither erasable
nor replicable it is markedly different from \emph{life}. Two of the
things we know about life is that

\begin{itemize}
  \item it ends;
  \item to gain some measure of persistence, to transcend it's
    finitude it is imminently copyable.
\end{itemize}

Both of these qualities are summarized succinctly in the aphorism: all
flesh is grass. For me these two kinds of ``information'' -- call them
quantum and biological -- are end points on a spectrum of strategies
for persistence. At one end, we have those curious entities that enjoy
uniqueness and permanence; at the other, we have those who in the face
of a certain end and an uncertain present make a go of passing
something on. To me one of the more remarkable aspects of the latter
strategy is that in the presence of noise (and certain features of
copying) we get a kind of dynamism, a chance for improvement against a
given persistent condition.

% subsection other_calculi_other_bisimulations_and_geometry_as_behavior (end)




% section conclusion (end)

%\documentclass[12pt]{llncs}
%\documentclass{jktr}

\usepackage[pdftex]{hyperref}                   
\usepackage {listings}
\usepackage {mathpartir}
\usepackage{bcprules}
%\usepackage{listings}
                       
\usepackage{graphicx} 
%\usepackage[margins=2.5cm,nohead,nofoot]{geometry}
%\usepackage{geometry}
\usepackage{amsfonts}
\usepackage{amstext}
\usepackage{latexsym}
\usepackage{amssymb}
\usepackage{color}


%\include{myPreamble}
\documentclass[12pt]{llncs}
%\documentclass{jktr}

\usepackage[pdftex]{hyperref}                   
\usepackage {listings}
\usepackage {mathpartir}
\usepackage{bcprules}
%\usepackage{listings}
                       
\usepackage{graphicx} 
%\usepackage[margins=2.5cm,nohead,nofoot]{geometry}
%\usepackage{geometry}
\usepackage{amsfonts}
\usepackage{amstext}
\usepackage{latexsym}
\usepackage{amssymb}
\usepackage{color}


%\include{myPreamble}
\include{qm2pi.local} 

%\ifpdf
%\usepackage[pdftex]{graphicx}
%\else
%\usepackage{graphicx}
%\fi

 % \ifpdf
%  \usepackage{pdfsync}
%  \if


%\title{Brief Article}
%\author{David F. Snyder}
%\author{L.G. Meredith}

%\address{Dept. of Math., Texas State University--San Marcos, San Marcos, TX 78666}
       
\pagestyle{empty}


\begin{document}

\lstset{language=[Objective]Caml,frame=shadowbox}

\input{qm2pi.front}

% section front matter (end)

\input{qm2pi.intro} 
 
% section introduction (end)

% \input{qm2pi.knotations} 

% section notation (end)

\input{qm2pi.process.calculi} 

% section concurrent_process_calculi_and_spatial_logics_ (end)
    
%\input{qm2pi.knots2pi} 

%\input{qm2pi.trefoil} 

%\input{qm2pi.mainthm} 

% subsection basic_interpretation (end)

%\input{qm2pi.rho.presentation} 
\subsection{The syntax and semantics of the notation system}\label{sub:the_syntax_and_semantics_of_the_notation_system} % (fold)

We now summarize a technical presentation of the calculus that
embodies our theory of dynamics. The typical presentation of such a
calculus follows the style of giving generators and relations on
them. The grammar, below, describing term constructors, freely
generates the set of processes, $\Proc$. This set is then quotiented
by a relation known as structural congruence and it is over this set
that the notion of dynamics is expressed. This presentation is
essentially that of \cite{MeredithR05} with the addition of
polyadicity and summation. For readability we have relegated some of
the technical subtleties to an appendix.

\subsubsection{Process grammar}\label{subsub:process_grammar}

\begin{mathpar}
  \inferrule* [lab=synchronization] {} {{M} \bc \pzero \;|\; x?F \;|\; x!C }
  \and
  \inferrule* [lab=abstraction] {} {{F} \bc (x)P}
  \and
  \inferrule* [lab=concretion] {} {{C} \bc \langle Q \rangle}
  \and
  \inferrule* [lab=process] {} {{P,Q} \bc M \;| \;P|Q \;|\; @{x}}
  \and
  \inferrule* [lab=name] {} {{x} \bc \quotep{P}}
\end{mathpar} 

Note that $\vec{x}$ (resp. $\vec{P}$) denotes a vector of names
(resp. processes) of length $|\vec{x}|$ (resp. $|\vec{P}|$). We adopt
the following useful abbreviations.

\begin{mathpar}
   x?(\vec{y}).P := x.(\vec{y})P \and  x\clift{\vec{P}} := x.\clift{\vec{P}}
   \and x!(y) := \lift{x}{\dropn{y}}
   \and \Pi_{i=0}^{n-1}P_i := P_0 | \ldots | P_{n-1}
\end{mathpar}

\subsubsection{Structural congruence}

\paragraph{Free and bound names and alpha-equivalence.} At the
core of structural equivalence is alpha-equivalence which identifies
process that are the same up to a change of variable. Formally, we
recognize the distinction between free and bound names. The free names
of a process, $\freenames{P}$, may be calculated recursively as
follows:

\begin{mathpar}
\freenames{\pzero} := \emptyset
  \and \\
  \freenames{x?(y).P} := \{ x \} \cup (\freenames{P} \setminus \{ y \})
  \and 
  \freenames{x!\langle P \rangle} := \{ x \} \cup \{ P \} 
  \and \\
  \freenames{P|Q} := \freenames{P} \cup \freenames{Q}
  \and \\
  \freenames{@{x}} := \{ x \}
\end{mathpar}

$\pi$
$\quotep{\pi}$

$\freenames{-} : \pi \to \mathcal{P}(\quotep{\pi})$

\begin{eqnarray*}
  \freenames{\pzero} & := & \emptyset \\
  \freenames{x?(y).P} & := & \{ x \} \cup (\freenames{P} \setminus \{ y \}) \\
  \freenames{x!\langle P \rangle} & := & \{ x \} \cup \{ P \} \\
  \freenames{P|Q} & := & \freenames{P} \cup \freenames{Q} \\
  \freenames{\dropn{x}} & := & \{ x \}
\end{eqnarray*}

The bound names of a process, $\boundnames{P}$, are those names occurring in $P$
that are not free. For example, in $x?(y).0$, the name $x$ is free, while $y$ is bound.

\begin{mathpar}
  \inferrule* [lab=monoidal-laws] {} { P|Q \equiv Q|P \and P|0 \equiv P \and P|(Q|R) \equiv (P|Q)|R }
\end{mathpar}

\begin{mathpar}
  \inferrule* [lab=alpha-equivalence] {} { (x)P \equiv (y)P\{y/x\} \and y \not\in \freenames{P} }
\end{mathpar}

\begin{definition}
Then two processes, $P,Q$, are alpha-equivalent if $P = Q\{\vec{y}/\vec{x}\}$ for
some $\vec{x} \in \boundnames{Q},\vec{y} \in \boundnames{P}$, where $Q\{\vec{y}/\vec{x}\}$
denotes the capture-avoiding substitution of $\vec{y}$ for $\vec{x}$ in $Q$.
\end{definition}

\begin{definition}
  The {\em structural congruence} \cite{SangiorgiWalker} , $\equiv$,
  between processes is the least congruence containing
  alpha-equivalence, satisfying the abelian monoid laws
  (associativity, commutativity and $\pzero$ as identity) for parallel
  composition $|$ and for summation $+$.
\end{definition}

\subsection{Name equivalence}

We take name equivalence, written $\nameeq$, to be the smallest
equivalence relation generated by the following rules.

\begin{mathpar}
\inferrule*[lab=Quote-drop]
{ }
{ \quotep{@{x}} \nameeq x }

\inferrule*[lab=Struct-equiv]
{ P \scong Q }
{ \quotep{P} \nameeq \quotep{Q} }
\end{mathpar}

The astute reader will have noticed that the mutual recursion of names
and processes imposes a mutual recursion on alpha-equivalence and
structural equivalence via name-equivalence. Fortunately, all of this
works out pleasantly and we may calculate in the natural way, free of
concern. The reader interested in the details is referred to the
appendix \ref{appendix:rho_details}.

\subsection{Substitution}

We use $\Proc$ for the set of processes, $\QProc$ for the set of
names, and $\id{\{}\vec{y} / \vec{x} \id{\}}$ to denote partial maps,
$s : \QProc \rightarrow \QProc$. A map, $s$ lifts, uniquely, to a map
on process terms, $\widehat{s} : \Proc \rightarrow \Proc$ by the
following equations.

\begin{mathpar}
  (0) \psubstp{Q}{P} := 0 \\
  (R \juxtap S) \psubstp{Q}{P}
  :=    
  (R)\psubstp{Q}{P} \juxtap (S) \psubstp{Q}{P} \\
  (x?(y).R) \psubstp{Q}{P}    
  :=    
  (x)\substp{Q}{P} (z)\concat( (R \psubstn{z}{y}) \psubstp{Q}{P} ) \\
  (\lift{x}{R}) \psubstp{Q}{P}  
  :=
  \lift{(x)\substp{Q}{P}}{ R \psubstp{Q}{P} } \\
%   (\dropn{x})  \psubstp{Q}{P}       
%   := 
%   \left\{ 
%     \begin{array}{ccc} 
%       \dropn{\quotep{Q}} & & x \nameeq \quotep{P} \\
%       \dropn{x} & & otherwise \\
%     \end{array}
%   \right. 
  (\dropn{x})  \psubstp{Q}{P}       
  := 
  \left\{ 
    \begin{array}{ccc} 
      Q & & x \nameeq \quotep{P} \\
      \dropn{x} & & otherwise \\
    \end{array}
  \right.
\end{mathpar}
 

where

\begin{eqnarray}
  (x)\id{\{} \lpquote Q \rpquote / \lpquote P \rpquote \id{\}}            = 
  \left\{ 
    \begin{array}{ccc}
      \lpquote Q \rpquote & & x \nameeq \lpquote P \rpquote \\
      x & & otherwise \\
    \end{array}
  \right. \nonumber
\end{eqnarray}

and $z$ is chosen distinct from $\quotep{P}$, $\quotep{Q}$, the free
names in $Q$, and all the names in $R$. Our $\alpha$-equivalence will
be built in the standard way from this substitution.

\begin{remark}\label{rem:no_self_referential_names}
  One consequence of these definitions is that $\forall P. \quotep{P}
  \not\in \freenames{P}$.
\end{remark}

\subsection{ Dynamic quote: an example }

Anticipating something of what's to come, consider applying the
substitution, $\widehat{\id{\{}u / z \id{\}}}$, to the following pair
of processes, $\lift{w}{y!(z)}$ and $w[ \lpquote y!(z) \rpquote ]$.

\begin{eqnarray}
	\lift{w}{y!(z)}\widehat{\id{\{}u / z \id{\}}}
		& = &
		\lift{w}{y!(u)} \nonumber\\
	w[ \lpquote y!(z) \rpquote ] \widehat{ \id{\{}u / z \id{\}} }
		& = &
		w[ \lpquote y!(z) \rpquote ] \nonumber
\end{eqnarray}

Because the body of the process between quotes is impervious to
substitution, we get radically different answers. In fact, by
examining the first process in an input context,
e.g. $x?(z).\lift{w}{y!(z)}$, we see that the process under the lift
operator may be shaped by prefixed inputs binding a name inside it. In
this sense, the lift operator will be seen as a way to dynamically
construct processes before reifying them as names.

Finally equipped with these standard features we can present the
dynamics of the calculus.

\subsubsection{Operational semantics} 

Finally, we introduce the computational dynamics. What marks these
algebras as distinct from other more traditionally studied algebraic
structures, e.g. vector spaces or polynomial rings, is the manner in
which dynamics is captured. In traditional structures, dynamics is typically
expressed through morphisms between such structures, as in linear maps
between vector spaces or morphisms between rings. In algebras
associated with the semantics of computation, the dynamics is
expressed as part of the algebraic structure itself, through a
reduction reduction relation typically denoted by $\red$. Below, we
give a recursive presentation of this relation for the calculus used
in the encoding.

$\red \subseteq \pi \times \pi$
$\red : \pi \to \mathcal{P}(\pi)$

\begin{mathpar}
  \inferrule* [lab=Comm] { \textsf{match}( x_{src}, x_{trgt} ) } { x_{trgt}?(y)P \; | \; x_{src}!\langle {Q} \rangle \red P\{\quotep{Q}/y}\} }
  \and \\
  \inferrule* [lab=Par] {{P} \red {P}'} {{{P} | {Q}} \red {{P}' | {Q}}}
  \and
  \inferrule* [lab=Equiv]{{{P} \scong {P}'} \andalso {{P}' \red {Q}'} \andalso {{Q}' \scong {Q}}}{{P} \red {Q}}
\end{mathpar}

\begin{eqnarray*}
  match_{\equiv} (\quotep{P},\quotep{Q}) & := & P \equiv Q \\
  match_{\dagger}(\quotep{P},\quotep{Q}) & := & \forall R. P|Q \red^{*} R => R \red^{*} 0 \\
  match_{K}(\quotep{P},\quotep{Q}) & := & K \mbox{ for some context } K
\end{eqnarray*}

$u?(x)P | u!\langle Q \rangle \red P\{\quotep{Q}/x\}$

%We write $\wred$ for $\red^*$, and $P\red$ if $\exists Q $ such that $ P \red Q$.
We write $P\red$ if $\exists Q $ such that $ P \red Q$ and $P\not\red$, otherwise.

\section{Replication}

As mentioned before, it is known that replication (and hence
recursion) can be implemented in a higher-order process algebra
\cite{SangiorgiWalker}. As our first example of calculation with the
machinery thus far presented we give the construction explicitly in
the {\rhoc}.

\begin{eqnarray}
	D_{x} & := & \prefix{x}{y}{(\binpar{\outputp{x}{y}}{@{y}})} \nonumber\\
	\bangp_{x}{P} & := & \binpar{{x}!\langle{\binpar{D_{x}}{P}}\rangle}{D_{x}} \nonumber
\end{eqnarray}

\begin{eqnarray}
	\bangp_{x}{P} & & \nonumber\\
	=
	& {x}!\langle{(\prefix{x}{y}{(\outputp{x}{y} | @{y})) | P}}\rangle 
	      | \prefix{x}{y}{(\outputp{x}{y} | @{y})} & \nonumber\\
	\red
	& (\outputp{x}{y} | @{y})\substn{\quotep{(\prefix{x}{y}{(@{y} | \outputp{x}{y})) | P}}}{y} & \nonumber\\
	=
	& \outputp{x}{\quotep{(\prefix{x}{y}{(\outputp{x}{y} | @{y})) | P}}}
	  | {(\prefix{x}{y}{(\outputp{x}{y} | @{y})) | P}} & \nonumber\\
	\red
	& \ldots & \nonumber\\
	\red^*
	& P | P | \ldots & \nonumber
\end{eqnarray}

Of course, this encoding, as an implementation, runs away, unfolding
$\bangp{P}$ eagerly. A lazier and more implementable replication
operator, restricted to input-guarded processes, may be obtained as follows.

\begin{eqnarray}
\bangp{\prefix{u}{v}{P}} 
	:= 
	\binpar{\lift{x}{\prefix{u}{v}{(\binpar{D(x)}{P})}}}{D(x)} \nonumber
\end{eqnarray}

\begin{remark}
  Note that the lazier definition still does not deal with summation
  or mixed summation (i.e. sums over input and output). The reader is
  invited to construct definitions of replication that deal with these
  features. 

  Further, the definitions are parameterized in a name, $x$. Can you,
  gentle reader, make a definition that eliminates this parameter and
  guarantees no accidental interaction between the replication
  machinery and the process being replicated -- i.e. no accidental
  sharing of names used by the process to get its work done and the
  name(s) used by the replication to effect copying. This latter
  revision of the definition of replication is crucial to obtaining
  the expected identity $!!P \sim !P$.
\end{remark}

\begin{remark}\label{rem:paradoxical_combinator}
  The reader familiar with the lambda calculus will have noticed the
  similarity between $D$ and the paradoxical combinator.

  [Ed. note: the existence of this seems to suggest we have to be more
  restrictive on the set of processes and names we admit if we are to
  support no-cloning.]
\end{remark}

\subsubsection{Bisimulation}

The computational dynamics gives rise to another kind of equivalence,
the equivalence of computational behavior. As previously mentioned
this is typically captured \emph{via} some form of bisimulation.

% The notion we use in this paper is weak barbed bisimulation
% \cite{milner91polyadicpi}.

The notion we use in this paper is derived from weak barbed
bisimulation \cite{milner91polyadicpi}. 

\begin{definition}
An \emph{observation relation}, $\downarrow_{\mathcal N}$, over a set
of names, $\mathcal N$, is the smallest relation satisfying the rules
below.

\infrule[Out-barb]{y \in {\mathcal N}, \; x \nameeq y}
		  {\outputp{x}{v} \downarrow_{\mathcal N} x}
\infrule[Par-barb]{\mbox{$P\downarrow_{\mathcal N} x$ or $Q\downarrow_{\mathcal N} x$}}
		  {\binpar{P}{Q} \downarrow_{\mathcal N} x}

We write $P \Downarrow_{\mathcal N} x$ if there is $Q$ such that 
$P \wred Q$ and $Q \downarrow_{\mathcal N} x$.
\end{definition}

\begin{definition}
%\label{def.bbisim}
An  ${\mathcal N}$-\emph{barbed bisimulation} over a set of names, ${\mathcal N}$, is a symmetric binary relation 
${\mathcal S}_{\mathcal N}$ between agents such that $P\rel{S}_{\mathcal N}Q$ implies:
\begin{enumerate}
\item If $P \red P'$ then $Q \wred Q'$ and $P'\rel{S}_{\mathcal N} Q'$.
\item If $P\downarrow_{\mathcal N} x$, then $Q\Downarrow_{\mathcal N} x$.
\end{enumerate}
$P$ is ${\mathcal N}$-barbed bisimilar to $Q$, written
$P \wbbisim_{\mathcal N} Q$, if $P \rel{S}_{\mathcal N} Q$ for some ${\mathcal N}$-barbed bisimulation ${\mathcal S}_{\mathcal N}$.
\end{definition}

$\mathcal{R} \subseteq \pi \times \pi$

$P \mathcal{R} Q => \forall P'. P \red P' \Rightarrow \exists Q'. Q \red Q', P' \mathcal{R} Q'$

$P \vdash x \Rightarrow Q \vdash x$

\begin{mathpar}
  \inferrule*[lab=Out-barb]{x \nameeq y}{{y}!\langle{Q}\rangle \vdash x}
  \and
  \inferrule*[lab=Par-barb]{\mbox{$P\vdash x$ or $Q\vdash x$}}{\binpar{P}{Q} \vdash x}
\end{mathpar}

\subsubsection{Contexts}

One of the principle advantages of computational calculi like the
$\pi$-calculus is a well-defined notion of context,
contextual-equivalence and a correlation between
contextual-equivalence and notions of bisimulation. The notion of
context allows the decomposition of a process into (sub-)process and
its syntactic environment, its context. Thus, a context may be
thought of as a process with a ``hole'' (written $\Box$) in it. The
application of a context $M$ to a process $P$, written $M[P]$, is
tantamount to filling the hole in $M$ with $P$. In this paper we do
not need the full weight of this theory, but do make use of the notion
of context in the proof the main theorem. 

\begin{mathpar}
  \inferrule* [lab=summation] {} {{M_{M},M_{N}} \bc \Box \;|\; x.M_{A} \;|\; M_{M}+M_{N}}
  \and
  \inferrule* [lab=agent] {} {{M_{A}} \bc (\vec{x})M_{P} \;| \; \clift{P_0,\ldots,M_{P},\ldots,P_N}}
  \and \\
  \inferrule* [lab=process] {} {{M_{P}} \bc M_{N} \;| \;P|M_{P} }
\end{mathpar} 

\begin{mathpar}
  \inferrule* [lab=sychronization] {} {M_{N} \bc \Box \;|\; x?M_{F} \;|\; x!M_{C}}
  \and
  \inferrule* [lab=abstraction] {} {{M_{F}} \bc (x)M_{P} }
  \and
  \inferrule* [lab=concretion] {} {{M_{C}} \bc \langle M_{P} \rangle }
  \and \\
  \inferrule* [lab=process] {} {{M_{P}} \bc M_{N} \;| \;P|M_{P} }
\end{mathpar}

\begin{definition}[contextual application] Given a context $M$, and
  process $P$, we define the \emph{contextual application}, $M[P] :=
  M\{P/\Box\}$. That is, the contextual application of M to P is the
  substitution of $P$ for $\Box$ in $M$.
\end{definition}

$\meaningof{-} : L \to \mathcal{P}(\pi)$

\begin{mathpar}
  \inferrule* [lab=collection] {} {\meaningof{true} = \pi, \and \meaningof{~E} = \pi \setminus \meaningof{E}, \and \meaningof{E_{1} \& E_{2}} = \meaningof{E_{1}} \cap \meaningof{E_{2}}}
\end{mathpar}

\begin{mathpar}
  \inferrule* [lab=structure] {} {\meaningof{0} = \{ P \in \pi | P \equiv 0 \}, \and \\ \meaningof{E_1 | E_2} = \{ P \in \pi | P \equiv P_{1} | P_{2}, P_{1} \in \meaningof{E_{1}}, P_{2} \in \meaningof{E_2}\} }
\end{mathpar}

\begin{mathpar}
 \inferrule* [lab=behavior] {} {\meaningof{\langle a?b \rangle E} = \{ P \in \pi | P \equiv Q | u?(y)P', \\ \and \\\\ \and \\ \;\;\; u \in \meaningof{a}, \forall z.P'\{z/y\} \in \meaningof{E\{z/b\}}\}, \and \\ \meaningof{a!E} = \{ P \in \pi | P \equiv Q | x!\langle P' \rangle, x \in \meaningof{a} P' \in \meaningof{E}\} }
\end{mathpar}

\begin{mathpar}
 \inferrule* [lab=nominal] {} {\meaningof{\quotep{E}} = \{ \quotep{P} \in \quotep{\pi} | P \in \meaningof{E} \}, \and \meaningof{\quotep{P}} = \{ \quotep{Q} \in \quotep{\pi} | P \equiv Q \} \and \\ \meaningof{@\quotep{E}} = \{ P \in \pi | P \equiv @x, x \in \meaningof{E} \}}
\end{mathpar}

\begin{eqnarray*}
  \\
  \meaningof{-} : TS \to ST
\end{eqnarray*}

\begin{eqnarray*}
  \\
  L : TS \to ST
\end{eqnarray*}

\begin{eqnarray*}
  \\
  P \models E \iff P \in \meaningof{E}
\end{eqnarray*}

\begin{eqnarray*}
  P \approx_{L} Q \iff \forall E \in L. P \models E \iff Q \models E
\end{eqnarray*}

\begin{eqnarray*}
  P \approx_{K} Q
\end{eqnarray*}

\begin{eqnarray*}
  P \approx Q
\end{eqnarray*}

$\approx_{K} = \approx = \approx_{L}$

\subsubsection{Contextual duality}

Note that contexts extend the quotation operation to a family of
operations from processes to names. Given a context, $M$, we can
define a \emph{nominal context}, $\quotep{M}$ by $\quotep{M}[P] :=
\quotep{M[P]}$. To foreshadow what is to come we observe that these
operations enjoy a duality with processes very much like the duality
between vectors and maps from vectors to scalars.

Further, because the calculus is essentially higher-order, we have a
correspondence between contexts and processes. More specifically,
given a name $x$ and a context $M$ we can construct $M^{*}_{x}$ such
that 

\begin{mathpar}
  M^{*}_{x} | \lift{x}{P} \red M[P]
\end{mathpar}

namely,

\begin{mathpar}
  M^{*}_{x} := x?(u).M[\dropn{u}]
\end{mathpar}

The dependence of $M^{*}_{x}$ on a name makes it an abstraction, 

\begin{mathpar}
  M^{*} := (x)x?(u).M[\dropn{u}]
\end{mathpar}

\subsection{Additional notation}

It will sometimes be convenient to denote the process a name
quotes. We already have the notation $x = \quotep{P}$, but it will be
convenient to introduce an alternate notation, $\procn{x}$, when we
want to emphasize the connection to the use of the name. Note that, by
virtue of name equivalence, $\quotep{\procn{x}} \nameeq x$; so, the
notation is consistent with previous definitions.

Further, because names have structure it is possible to effect
substitutions on the basis of that structure. This means we need to
upgrade our notation for substitutions, which we accomplish by
adapting comprehension notation. Thus,

\begin{mathpar}
  P\{ y / x : x \in S \}
\end{mathpar}

is interpreted to mean the process derived from P by replacing (in a
capture-avoiding manner) each occurrence of $x$ in $S$ by $y$. For example,

\begin{mathpar}
  P\{ \quotep{\procn{x}|\procn{x}} / x : x \in \freenames{P} \}
\end{mathpar}

will replace each (occurrence) of a free name $x$ in $P$ by
$\quotep{\procn{x}|\procn{x}}$.

Also, we will avail ourselves of the notation $x^{L}$ and $x^{R}$ to
denote injections of a name into disjoint copies of the name
space. There are numerous ways to accomplish this. One example can be
found in \cite{MeredithR05}. This notation overloads to vectors of
names: $\vec{x}^{\pi} := (x_{i}^{\pi} \; : \; 0 \leq i < |\vec{x}| )$ where $\pi \in \{L,R\}$.

We also use $P^{\Box} := P|\Box$.

In \cite{MeredithR05} an interpretation of the new operator is
given. It turns out that there are several possible interpretations
all enjoying the requisite algebraic properties of the operator (see
\cite{milner91polyadicpi}). We will therefore make liberal use of
$(\nu\; \vec{x})P$.

% subsection the_syntax_and_semantics_of_the_notation_system (end)   

\input{qm2pi.qmops} 

\input{qm2pi.sterngerlach} 

\input{qm2pi.metric} 

% section concurrent_process_calculi (end)

%\input{qm2pi.proofsketch}

% section proof sketch (end)

%\input{qm2pi.slviaknots} 

% section spatial logic via knots (end)

\input{qm2pi.conclusion}

% section conclusion (end)

%\input{qm2pi.dtcodes} 

% section wiring algorithm (end)

\input{qm2pi.ack} 

% section acknowledgments (end)

\newpage


\bibliographystyle{plain}   
\bibliography{../../biblios/main.bib}

\input{qm2pi.rhodetails}

\end{document}

 

%\ifpdf
%\usepackage[pdftex]{graphicx}
%\else
%\usepackage{graphicx}
%\fi

 % \ifpdf
%  \usepackage{pdfsync}
%  \if


%\title{Brief Article}
%\author{David F. Snyder}
%\author{L.G. Meredith}

%\address{Dept. of Math., Texas State University--San Marcos, San Marcos, TX 78666}
       
\pagestyle{empty}


\begin{document}

\lstset{language=[Objective]Caml,frame=shadowbox}

\documentclass[12pt]{llncs}
%\documentclass{jktr}

\usepackage[pdftex]{hyperref}                   
\usepackage {listings}
\usepackage {mathpartir}
\usepackage{bcprules}
%\usepackage{listings}
                       
\usepackage{graphicx} 
%\usepackage[margins=2.5cm,nohead,nofoot]{geometry}
%\usepackage{geometry}
\usepackage{amsfonts}
\usepackage{amstext}
\usepackage{latexsym}
\usepackage{amssymb}
\usepackage{color}


%\include{myPreamble}
\include{qm2pi.local} 

%\ifpdf
%\usepackage[pdftex]{graphicx}
%\else
%\usepackage{graphicx}
%\fi

 % \ifpdf
%  \usepackage{pdfsync}
%  \if


%\title{Brief Article}
%\author{David F. Snyder}
%\author{L.G. Meredith}

%\address{Dept. of Math., Texas State University--San Marcos, San Marcos, TX 78666}
       
\pagestyle{empty}


\begin{document}

\lstset{language=[Objective]Caml,frame=shadowbox}

\input{qm2pi.front}

% section front matter (end)

\input{qm2pi.intro} 
 
% section introduction (end)

% \input{qm2pi.knotations} 

% section notation (end)

\input{qm2pi.process.calculi} 

% section concurrent_process_calculi_and_spatial_logics_ (end)
    
%\input{qm2pi.knots2pi} 

%\input{qm2pi.trefoil} 

%\input{qm2pi.mainthm} 

% subsection basic_interpretation (end)

%\input{qm2pi.rho.presentation} 
\subsection{The syntax and semantics of the notation system}\label{sub:the_syntax_and_semantics_of_the_notation_system} % (fold)

We now summarize a technical presentation of the calculus that
embodies our theory of dynamics. The typical presentation of such a
calculus follows the style of giving generators and relations on
them. The grammar, below, describing term constructors, freely
generates the set of processes, $\Proc$. This set is then quotiented
by a relation known as structural congruence and it is over this set
that the notion of dynamics is expressed. This presentation is
essentially that of \cite{MeredithR05} with the addition of
polyadicity and summation. For readability we have relegated some of
the technical subtleties to an appendix.

\subsubsection{Process grammar}\label{subsub:process_grammar}

\begin{mathpar}
  \inferrule* [lab=synchronization] {} {{M} \bc \pzero \;|\; x?F \;|\; x!C }
  \and
  \inferrule* [lab=abstraction] {} {{F} \bc (x)P}
  \and
  \inferrule* [lab=concretion] {} {{C} \bc \langle Q \rangle}
  \and
  \inferrule* [lab=process] {} {{P,Q} \bc M \;| \;P|Q \;|\; @{x}}
  \and
  \inferrule* [lab=name] {} {{x} \bc \quotep{P}}
\end{mathpar} 

Note that $\vec{x}$ (resp. $\vec{P}$) denotes a vector of names
(resp. processes) of length $|\vec{x}|$ (resp. $|\vec{P}|$). We adopt
the following useful abbreviations.

\begin{mathpar}
   x?(\vec{y}).P := x.(\vec{y})P \and  x\clift{\vec{P}} := x.\clift{\vec{P}}
   \and x!(y) := \lift{x}{\dropn{y}}
   \and \Pi_{i=0}^{n-1}P_i := P_0 | \ldots | P_{n-1}
\end{mathpar}

\subsubsection{Structural congruence}

\paragraph{Free and bound names and alpha-equivalence.} At the
core of structural equivalence is alpha-equivalence which identifies
process that are the same up to a change of variable. Formally, we
recognize the distinction between free and bound names. The free names
of a process, $\freenames{P}$, may be calculated recursively as
follows:

\begin{mathpar}
\freenames{\pzero} := \emptyset
  \and \\
  \freenames{x?(y).P} := \{ x \} \cup (\freenames{P} \setminus \{ y \})
  \and 
  \freenames{x!\langle P \rangle} := \{ x \} \cup \{ P \} 
  \and \\
  \freenames{P|Q} := \freenames{P} \cup \freenames{Q}
  \and \\
  \freenames{@{x}} := \{ x \}
\end{mathpar}

$\pi$
$\quotep{\pi}$

$\freenames{-} : \pi \to \mathcal{P}(\quotep{\pi})$

\begin{eqnarray*}
  \freenames{\pzero} & := & \emptyset \\
  \freenames{x?(y).P} & := & \{ x \} \cup (\freenames{P} \setminus \{ y \}) \\
  \freenames{x!\langle P \rangle} & := & \{ x \} \cup \{ P \} \\
  \freenames{P|Q} & := & \freenames{P} \cup \freenames{Q} \\
  \freenames{\dropn{x}} & := & \{ x \}
\end{eqnarray*}

The bound names of a process, $\boundnames{P}$, are those names occurring in $P$
that are not free. For example, in $x?(y).0$, the name $x$ is free, while $y$ is bound.

\begin{mathpar}
  \inferrule* [lab=monoidal-laws] {} { P|Q \equiv Q|P \and P|0 \equiv P \and P|(Q|R) \equiv (P|Q)|R }
\end{mathpar}

\begin{mathpar}
  \inferrule* [lab=alpha-equivalence] {} { (x)P \equiv (y)P\{y/x\} \and y \not\in \freenames{P} }
\end{mathpar}

\begin{definition}
Then two processes, $P,Q$, are alpha-equivalent if $P = Q\{\vec{y}/\vec{x}\}$ for
some $\vec{x} \in \boundnames{Q},\vec{y} \in \boundnames{P}$, where $Q\{\vec{y}/\vec{x}\}$
denotes the capture-avoiding substitution of $\vec{y}$ for $\vec{x}$ in $Q$.
\end{definition}

\begin{definition}
  The {\em structural congruence} \cite{SangiorgiWalker} , $\equiv$,
  between processes is the least congruence containing
  alpha-equivalence, satisfying the abelian monoid laws
  (associativity, commutativity and $\pzero$ as identity) for parallel
  composition $|$ and for summation $+$.
\end{definition}

\subsection{Name equivalence}

We take name equivalence, written $\nameeq$, to be the smallest
equivalence relation generated by the following rules.

\begin{mathpar}
\inferrule*[lab=Quote-drop]
{ }
{ \quotep{@{x}} \nameeq x }

\inferrule*[lab=Struct-equiv]
{ P \scong Q }
{ \quotep{P} \nameeq \quotep{Q} }
\end{mathpar}

The astute reader will have noticed that the mutual recursion of names
and processes imposes a mutual recursion on alpha-equivalence and
structural equivalence via name-equivalence. Fortunately, all of this
works out pleasantly and we may calculate in the natural way, free of
concern. The reader interested in the details is referred to the
appendix \ref{appendix:rho_details}.

\subsection{Substitution}

We use $\Proc$ for the set of processes, $\QProc$ for the set of
names, and $\id{\{}\vec{y} / \vec{x} \id{\}}$ to denote partial maps,
$s : \QProc \rightarrow \QProc$. A map, $s$ lifts, uniquely, to a map
on process terms, $\widehat{s} : \Proc \rightarrow \Proc$ by the
following equations.

\begin{mathpar}
  (0) \psubstp{Q}{P} := 0 \\
  (R \juxtap S) \psubstp{Q}{P}
  :=    
  (R)\psubstp{Q}{P} \juxtap (S) \psubstp{Q}{P} \\
  (x?(y).R) \psubstp{Q}{P}    
  :=    
  (x)\substp{Q}{P} (z)\concat( (R \psubstn{z}{y}) \psubstp{Q}{P} ) \\
  (\lift{x}{R}) \psubstp{Q}{P}  
  :=
  \lift{(x)\substp{Q}{P}}{ R \psubstp{Q}{P} } \\
%   (\dropn{x})  \psubstp{Q}{P}       
%   := 
%   \left\{ 
%     \begin{array}{ccc} 
%       \dropn{\quotep{Q}} & & x \nameeq \quotep{P} \\
%       \dropn{x} & & otherwise \\
%     \end{array}
%   \right. 
  (\dropn{x})  \psubstp{Q}{P}       
  := 
  \left\{ 
    \begin{array}{ccc} 
      Q & & x \nameeq \quotep{P} \\
      \dropn{x} & & otherwise \\
    \end{array}
  \right.
\end{mathpar}
 

where

\begin{eqnarray}
  (x)\id{\{} \lpquote Q \rpquote / \lpquote P \rpquote \id{\}}            = 
  \left\{ 
    \begin{array}{ccc}
      \lpquote Q \rpquote & & x \nameeq \lpquote P \rpquote \\
      x & & otherwise \\
    \end{array}
  \right. \nonumber
\end{eqnarray}

and $z$ is chosen distinct from $\quotep{P}$, $\quotep{Q}$, the free
names in $Q$, and all the names in $R$. Our $\alpha$-equivalence will
be built in the standard way from this substitution.

\begin{remark}\label{rem:no_self_referential_names}
  One consequence of these definitions is that $\forall P. \quotep{P}
  \not\in \freenames{P}$.
\end{remark}

\subsection{ Dynamic quote: an example }

Anticipating something of what's to come, consider applying the
substitution, $\widehat{\id{\{}u / z \id{\}}}$, to the following pair
of processes, $\lift{w}{y!(z)}$ and $w[ \lpquote y!(z) \rpquote ]$.

\begin{eqnarray}
	\lift{w}{y!(z)}\widehat{\id{\{}u / z \id{\}}}
		& = &
		\lift{w}{y!(u)} \nonumber\\
	w[ \lpquote y!(z) \rpquote ] \widehat{ \id{\{}u / z \id{\}} }
		& = &
		w[ \lpquote y!(z) \rpquote ] \nonumber
\end{eqnarray}

Because the body of the process between quotes is impervious to
substitution, we get radically different answers. In fact, by
examining the first process in an input context,
e.g. $x?(z).\lift{w}{y!(z)}$, we see that the process under the lift
operator may be shaped by prefixed inputs binding a name inside it. In
this sense, the lift operator will be seen as a way to dynamically
construct processes before reifying them as names.

Finally equipped with these standard features we can present the
dynamics of the calculus.

\subsubsection{Operational semantics} 

Finally, we introduce the computational dynamics. What marks these
algebras as distinct from other more traditionally studied algebraic
structures, e.g. vector spaces or polynomial rings, is the manner in
which dynamics is captured. In traditional structures, dynamics is typically
expressed through morphisms between such structures, as in linear maps
between vector spaces or morphisms between rings. In algebras
associated with the semantics of computation, the dynamics is
expressed as part of the algebraic structure itself, through a
reduction reduction relation typically denoted by $\red$. Below, we
give a recursive presentation of this relation for the calculus used
in the encoding.

$\red \subseteq \pi \times \pi$
$\red : \pi \to \mathcal{P}(\pi)$

\begin{mathpar}
  \inferrule* [lab=Comm] { \textsf{match}( x_{src}, x_{trgt} ) } { x_{trgt}?(y)P \; | \; x_{src}!\langle {Q} \rangle \red P\{\quotep{Q}/y}\} }
  \and \\
  \inferrule* [lab=Par] {{P} \red {P}'} {{{P} | {Q}} \red {{P}' | {Q}}}
  \and
  \inferrule* [lab=Equiv]{{{P} \scong {P}'} \andalso {{P}' \red {Q}'} \andalso {{Q}' \scong {Q}}}{{P} \red {Q}}
\end{mathpar}

\begin{eqnarray*}
  match_{\equiv} (\quotep{P},\quotep{Q}) & := & P \equiv Q \\
  match_{\dagger}(\quotep{P},\quotep{Q}) & := & \forall R. P|Q \red^{*} R => R \red^{*} 0 \\
  match_{K}(\quotep{P},\quotep{Q}) & := & K \mbox{ for some context } K
\end{eqnarray*}

$u?(x)P | u!\langle Q \rangle \red P\{\quotep{Q}/x\}$

%We write $\wred$ for $\red^*$, and $P\red$ if $\exists Q $ such that $ P \red Q$.
We write $P\red$ if $\exists Q $ such that $ P \red Q$ and $P\not\red$, otherwise.

\section{Replication}

As mentioned before, it is known that replication (and hence
recursion) can be implemented in a higher-order process algebra
\cite{SangiorgiWalker}. As our first example of calculation with the
machinery thus far presented we give the construction explicitly in
the {\rhoc}.

\begin{eqnarray}
	D_{x} & := & \prefix{x}{y}{(\binpar{\outputp{x}{y}}{@{y}})} \nonumber\\
	\bangp_{x}{P} & := & \binpar{{x}!\langle{\binpar{D_{x}}{P}}\rangle}{D_{x}} \nonumber
\end{eqnarray}

\begin{eqnarray}
	\bangp_{x}{P} & & \nonumber\\
	=
	& {x}!\langle{(\prefix{x}{y}{(\outputp{x}{y} | @{y})) | P}}\rangle 
	      | \prefix{x}{y}{(\outputp{x}{y} | @{y})} & \nonumber\\
	\red
	& (\outputp{x}{y} | @{y})\substn{\quotep{(\prefix{x}{y}{(@{y} | \outputp{x}{y})) | P}}}{y} & \nonumber\\
	=
	& \outputp{x}{\quotep{(\prefix{x}{y}{(\outputp{x}{y} | @{y})) | P}}}
	  | {(\prefix{x}{y}{(\outputp{x}{y} | @{y})) | P}} & \nonumber\\
	\red
	& \ldots & \nonumber\\
	\red^*
	& P | P | \ldots & \nonumber
\end{eqnarray}

Of course, this encoding, as an implementation, runs away, unfolding
$\bangp{P}$ eagerly. A lazier and more implementable replication
operator, restricted to input-guarded processes, may be obtained as follows.

\begin{eqnarray}
\bangp{\prefix{u}{v}{P}} 
	:= 
	\binpar{\lift{x}{\prefix{u}{v}{(\binpar{D(x)}{P})}}}{D(x)} \nonumber
\end{eqnarray}

\begin{remark}
  Note that the lazier definition still does not deal with summation
  or mixed summation (i.e. sums over input and output). The reader is
  invited to construct definitions of replication that deal with these
  features. 

  Further, the definitions are parameterized in a name, $x$. Can you,
  gentle reader, make a definition that eliminates this parameter and
  guarantees no accidental interaction between the replication
  machinery and the process being replicated -- i.e. no accidental
  sharing of names used by the process to get its work done and the
  name(s) used by the replication to effect copying. This latter
  revision of the definition of replication is crucial to obtaining
  the expected identity $!!P \sim !P$.
\end{remark}

\begin{remark}\label{rem:paradoxical_combinator}
  The reader familiar with the lambda calculus will have noticed the
  similarity between $D$ and the paradoxical combinator.

  [Ed. note: the existence of this seems to suggest we have to be more
  restrictive on the set of processes and names we admit if we are to
  support no-cloning.]
\end{remark}

\subsubsection{Bisimulation}

The computational dynamics gives rise to another kind of equivalence,
the equivalence of computational behavior. As previously mentioned
this is typically captured \emph{via} some form of bisimulation.

% The notion we use in this paper is weak barbed bisimulation
% \cite{milner91polyadicpi}.

The notion we use in this paper is derived from weak barbed
bisimulation \cite{milner91polyadicpi}. 

\begin{definition}
An \emph{observation relation}, $\downarrow_{\mathcal N}$, over a set
of names, $\mathcal N$, is the smallest relation satisfying the rules
below.

\infrule[Out-barb]{y \in {\mathcal N}, \; x \nameeq y}
		  {\outputp{x}{v} \downarrow_{\mathcal N} x}
\infrule[Par-barb]{\mbox{$P\downarrow_{\mathcal N} x$ or $Q\downarrow_{\mathcal N} x$}}
		  {\binpar{P}{Q} \downarrow_{\mathcal N} x}

We write $P \Downarrow_{\mathcal N} x$ if there is $Q$ such that 
$P \wred Q$ and $Q \downarrow_{\mathcal N} x$.
\end{definition}

\begin{definition}
%\label{def.bbisim}
An  ${\mathcal N}$-\emph{barbed bisimulation} over a set of names, ${\mathcal N}$, is a symmetric binary relation 
${\mathcal S}_{\mathcal N}$ between agents such that $P\rel{S}_{\mathcal N}Q$ implies:
\begin{enumerate}
\item If $P \red P'$ then $Q \wred Q'$ and $P'\rel{S}_{\mathcal N} Q'$.
\item If $P\downarrow_{\mathcal N} x$, then $Q\Downarrow_{\mathcal N} x$.
\end{enumerate}
$P$ is ${\mathcal N}$-barbed bisimilar to $Q$, written
$P \wbbisim_{\mathcal N} Q$, if $P \rel{S}_{\mathcal N} Q$ for some ${\mathcal N}$-barbed bisimulation ${\mathcal S}_{\mathcal N}$.
\end{definition}

$\mathcal{R} \subseteq \pi \times \pi$

$P \mathcal{R} Q => \forall P'. P \red P' \Rightarrow \exists Q'. Q \red Q', P' \mathcal{R} Q'$

$P \vdash x \Rightarrow Q \vdash x$

\begin{mathpar}
  \inferrule*[lab=Out-barb]{x \nameeq y}{{y}!\langle{Q}\rangle \vdash x}
  \and
  \inferrule*[lab=Par-barb]{\mbox{$P\vdash x$ or $Q\vdash x$}}{\binpar{P}{Q} \vdash x}
\end{mathpar}

\subsubsection{Contexts}

One of the principle advantages of computational calculi like the
$\pi$-calculus is a well-defined notion of context,
contextual-equivalence and a correlation between
contextual-equivalence and notions of bisimulation. The notion of
context allows the decomposition of a process into (sub-)process and
its syntactic environment, its context. Thus, a context may be
thought of as a process with a ``hole'' (written $\Box$) in it. The
application of a context $M$ to a process $P$, written $M[P]$, is
tantamount to filling the hole in $M$ with $P$. In this paper we do
not need the full weight of this theory, but do make use of the notion
of context in the proof the main theorem. 

\begin{mathpar}
  \inferrule* [lab=summation] {} {{M_{M},M_{N}} \bc \Box \;|\; x.M_{A} \;|\; M_{M}+M_{N}}
  \and
  \inferrule* [lab=agent] {} {{M_{A}} \bc (\vec{x})M_{P} \;| \; \clift{P_0,\ldots,M_{P},\ldots,P_N}}
  \and \\
  \inferrule* [lab=process] {} {{M_{P}} \bc M_{N} \;| \;P|M_{P} }
\end{mathpar} 

\begin{mathpar}
  \inferrule* [lab=sychronization] {} {M_{N} \bc \Box \;|\; x?M_{F} \;|\; x!M_{C}}
  \and
  \inferrule* [lab=abstraction] {} {{M_{F}} \bc (x)M_{P} }
  \and
  \inferrule* [lab=concretion] {} {{M_{C}} \bc \langle M_{P} \rangle }
  \and \\
  \inferrule* [lab=process] {} {{M_{P}} \bc M_{N} \;| \;P|M_{P} }
\end{mathpar}

\begin{definition}[contextual application] Given a context $M$, and
  process $P$, we define the \emph{contextual application}, $M[P] :=
  M\{P/\Box\}$. That is, the contextual application of M to P is the
  substitution of $P$ for $\Box$ in $M$.
\end{definition}

$\meaningof{-} : L \to \mathcal{P}(\pi)$

\begin{mathpar}
  \inferrule* [lab=collection] {} {\meaningof{true} = \pi, \and \meaningof{~E} = \pi \setminus \meaningof{E}, \and \meaningof{E_{1} \& E_{2}} = \meaningof{E_{1}} \cap \meaningof{E_{2}}}
\end{mathpar}

\begin{mathpar}
  \inferrule* [lab=structure] {} {\meaningof{0} = \{ P \in \pi | P \equiv 0 \}, \and \\ \meaningof{E_1 | E_2} = \{ P \in \pi | P \equiv P_{1} | P_{2}, P_{1} \in \meaningof{E_{1}}, P_{2} \in \meaningof{E_2}\} }
\end{mathpar}

\begin{mathpar}
 \inferrule* [lab=behavior] {} {\meaningof{\langle a?b \rangle E} = \{ P \in \pi | P \equiv Q | u?(y)P', \\ \and \\\\ \and \\ \;\;\; u \in \meaningof{a}, \forall z.P'\{z/y\} \in \meaningof{E\{z/b\}}\}, \and \\ \meaningof{a!E} = \{ P \in \pi | P \equiv Q | x!\langle P' \rangle, x \in \meaningof{a} P' \in \meaningof{E}\} }
\end{mathpar}

\begin{mathpar}
 \inferrule* [lab=nominal] {} {\meaningof{\quotep{E}} = \{ \quotep{P} \in \quotep{\pi} | P \in \meaningof{E} \}, \and \meaningof{\quotep{P}} = \{ \quotep{Q} \in \quotep{\pi} | P \equiv Q \} \and \\ \meaningof{@\quotep{E}} = \{ P \in \pi | P \equiv @x, x \in \meaningof{E} \}}
\end{mathpar}

\begin{eqnarray*}
  \\
  \meaningof{-} : TS \to ST
\end{eqnarray*}

\begin{eqnarray*}
  \\
  L : TS \to ST
\end{eqnarray*}

\begin{eqnarray*}
  \\
  P \models E \iff P \in \meaningof{E}
\end{eqnarray*}

\begin{eqnarray*}
  P \approx_{L} Q \iff \forall E \in L. P \models E \iff Q \models E
\end{eqnarray*}

\begin{eqnarray*}
  P \approx_{K} Q
\end{eqnarray*}

\begin{eqnarray*}
  P \approx Q
\end{eqnarray*}

$\approx_{K} = \approx = \approx_{L}$

\subsubsection{Contextual duality}

Note that contexts extend the quotation operation to a family of
operations from processes to names. Given a context, $M$, we can
define a \emph{nominal context}, $\quotep{M}$ by $\quotep{M}[P] :=
\quotep{M[P]}$. To foreshadow what is to come we observe that these
operations enjoy a duality with processes very much like the duality
between vectors and maps from vectors to scalars.

Further, because the calculus is essentially higher-order, we have a
correspondence between contexts and processes. More specifically,
given a name $x$ and a context $M$ we can construct $M^{*}_{x}$ such
that 

\begin{mathpar}
  M^{*}_{x} | \lift{x}{P} \red M[P]
\end{mathpar}

namely,

\begin{mathpar}
  M^{*}_{x} := x?(u).M[\dropn{u}]
\end{mathpar}

The dependence of $M^{*}_{x}$ on a name makes it an abstraction, 

\begin{mathpar}
  M^{*} := (x)x?(u).M[\dropn{u}]
\end{mathpar}

\subsection{Additional notation}

It will sometimes be convenient to denote the process a name
quotes. We already have the notation $x = \quotep{P}$, but it will be
convenient to introduce an alternate notation, $\procn{x}$, when we
want to emphasize the connection to the use of the name. Note that, by
virtue of name equivalence, $\quotep{\procn{x}} \nameeq x$; so, the
notation is consistent with previous definitions.

Further, because names have structure it is possible to effect
substitutions on the basis of that structure. This means we need to
upgrade our notation for substitutions, which we accomplish by
adapting comprehension notation. Thus,

\begin{mathpar}
  P\{ y / x : x \in S \}
\end{mathpar}

is interpreted to mean the process derived from P by replacing (in a
capture-avoiding manner) each occurrence of $x$ in $S$ by $y$. For example,

\begin{mathpar}
  P\{ \quotep{\procn{x}|\procn{x}} / x : x \in \freenames{P} \}
\end{mathpar}

will replace each (occurrence) of a free name $x$ in $P$ by
$\quotep{\procn{x}|\procn{x}}$.

Also, we will avail ourselves of the notation $x^{L}$ and $x^{R}$ to
denote injections of a name into disjoint copies of the name
space. There are numerous ways to accomplish this. One example can be
found in \cite{MeredithR05}. This notation overloads to vectors of
names: $\vec{x}^{\pi} := (x_{i}^{\pi} \; : \; 0 \leq i < |\vec{x}| )$ where $\pi \in \{L,R\}$.

We also use $P^{\Box} := P|\Box$.

In \cite{MeredithR05} an interpretation of the new operator is
given. It turns out that there are several possible interpretations
all enjoying the requisite algebraic properties of the operator (see
\cite{milner91polyadicpi}). We will therefore make liberal use of
$(\nu\; \vec{x})P$.

% subsection the_syntax_and_semantics_of_the_notation_system (end)   

\input{qm2pi.qmops} 

\input{qm2pi.sterngerlach} 

\input{qm2pi.metric} 

% section concurrent_process_calculi (end)

%\input{qm2pi.proofsketch}

% section proof sketch (end)

%\input{qm2pi.slviaknots} 

% section spatial logic via knots (end)

\input{qm2pi.conclusion}

% section conclusion (end)

%\input{qm2pi.dtcodes} 

% section wiring algorithm (end)

\input{qm2pi.ack} 

% section acknowledgments (end)

\newpage


\bibliographystyle{plain}   
\bibliography{../../biblios/main.bib}

\input{qm2pi.rhodetails}

\end{document}



% section front matter (end)

\section{Introduction}\label{sec:introduction} % (fold)
In this draft of the material i am going to have to dispense with the
usual writing conventions adopted in papers on these topics. i'm going
to have adopt whatever tone i need at the time i'm writing up the
calculations. Sometimes this may be very conversational; others it may
be the barest mathematical grunts; others still it may be that i have
lifted text from one of my other papers because the exposition of some
point was better said there. i hope that my readers are not unduly put
out by this decision. i'm not doing this to flout convention or be
rebellious. i find these calculations very technically challenging. To
keep everything going technically, something has to give; i have to
let go of some cognitive burden. So, the academic writing style --
with all of its trade-offs in terms of facilitating technical
communication -- is what i'm letting go of. Perhaps subsequent drafts
can be tightened and polished, but for now, i'm going to speak as if
we were sitting together in a coffee shop with a laptop, wifi and a
pad of paper and a pencil.

So, here's what i have to say. We -- you and i, comfortably ensconced
in our coffee shop and well-equipped with our tools -- can realize and
carry out the calculations of quantum mechanics over a very different
formal theory of dynamics, a formal theory of dynamics that
corresponds to a theory of concurrent computation with
\emph{reflection}. It has the advantage that the underlying theory is
already `quantized', but supports analogues all of the continuuous
operations. Strikingly, this underlying theory has recently been
connected with a notion of metric that we can show, by calculating
together, coincides with the metric induced by the inner product.

There are a lot of reasons why you might be interested in seeing
calculations of this form. Here's why i'm interested. For the past
several centuries there has been no competitor to the ``Newtonian''
account of dynamics. As a result the predominant share of accounts of
dynamical systems and situations have had to be formulated in terms of
the Newtonian machinery. i view this as an intellectually dangerous
position to occupy. Everything, despite it's intrinsic shape, turns
into a nail to be hit with this hammer. Recently, however, the theory
of computation has matured to the point where we have candidates for
theories of dynamics that offer very different perspective on
reasoning about dynamical systems and situations. Testing these
candidates against very successful accounts of dynamical situations,
like quantum mechanics, is going to give us some sense of how mature
they are and some measure of the quality of these accounts of
dynamics.

\subsection{Summary of contributions and outline of paper}

So, we're going to develop an interpretation of the operations of
quantum mechanics normally interpreted by Hilbert spaces and
operators. We're going to do this over a theory of computation. Note
that this is very different than the usual quantum computation program
which develops notions of computation over quantum mechanics. Rather,
we are developing a story that aligns with Wheeler's slogan: It from
Bit. To do this we will first provide an account of the theory of
computation at play here. Then we will dive into a calculation-driven
interpretation of the operations of quantum mechanics.

The reason we take this approach is that -- until very recently --
there hasn't been an axiomatic account of quantum mechanics. As a
result there has been no sharp delineation of the mathematical theory
supporting interpretation of the physical theory and the physical
theory, itself. So, ambient features of the maths are free to be
exploited (or supressed) without a real accounting of their physical
relevance. There is no sharp statement ``here's the physical theory''
qua \emph{theory} and ``here's the mathematical interpretation''
enabling a judgment of how faithful the interpretation is -- apart
from experimental observation. When there is an axiomatic account we
can judge how well a given mathematical formalism supports an
interpretation of the axioms, independent of
experimentation. Likewise, we can judge how well we have captured our
physical evidence and experience with our axiomatics, independent of
any specific mathematical implementation, with accidental detail that
may or may not have physical significance. 

In lieu of a fully fleshed out and vetted axiomatic account of quantum
mechanics, interpreting the operational notions in service of modeling
physical systems will have to suffice. In other words, we are not in
the business of providing a model of Hilbert spaces and operators. We
are in the business of providing a model of quantum mechanics because
we are motivated by testing our notions of dynamics against physical
theory; and, the predictive calculations of the physical theory must
serve as the best formulation -- shy of a fully fleshed out axiomatic
account -- of the physical theory itself (as they have for scientific
theories since time immemorial). Put another way, despite a
whole-hearted commitment to an It-from-Bit ontology, we are firmly
aligned with the shut-up-and-calculate camp as the best way to obtain
results either from the physical perspective or as a quality assurance
measure of our fledgling theory of dynamics.

In detail, we present a reflective process calculus. Then we develop
intuitive correspondences between the notions available in this
calculus and the usual physical notions supporting quantum mechanical
calculations. Thus, 

\begin{table}[htp]
  \center{
    \fbox{
      \begin{tabular}{c|c}
        quantum mechanics & process calculus \\
        \hline
        scalar & name \\
        state vector & process \\
        dual & contextual duals \\
        matrix & formal sums of process-context-dual pairs \\
        orthogonality & process annihilation \\
        inner product & execution-formula + quoting
      \end{tabular}
    }
  }
  \caption{QM - process calculi correspondences}
\end{table}

Then we tighten up these intuitions to operational definitions. We
employ the Dirac notation as the best proxy we can find for an
abstract syntax of the quantum mechanical notions. The definitions we
develop put us in contact with equational constraints coming from the
theory that we demonstrate the definitions and calculations satisfy.

This puts us in a position to shut up and calculate for the
Stern-Gerlach experimental set up, showing how these predictive
calculations become calculations on processes in our theory of a
reflective process calculus.

Penultimately, we demonstrate that the notion of metric coming from
the inner product coincides with the notion of metric available from
the theory of bisimulation. This demonstration gives us the right to
think of space as arising from behavior. Finally, we consider where we
might go from the new vantage point we have obtained.

% section introduction (end) 
 
% section introduction (end)

% \documentclass[12pt]{llncs}
%\documentclass{jktr}

\usepackage[pdftex]{hyperref}                   
\usepackage {listings}
\usepackage {mathpartir}
\usepackage{bcprules}
%\usepackage{listings}
                       
\usepackage{graphicx} 
%\usepackage[margins=2.5cm,nohead,nofoot]{geometry}
%\usepackage{geometry}
\usepackage{amsfonts}
\usepackage{amstext}
\usepackage{latexsym}
\usepackage{amssymb}
\usepackage{color}


%\include{myPreamble}
\include{qm2pi.local} 

%\ifpdf
%\usepackage[pdftex]{graphicx}
%\else
%\usepackage{graphicx}
%\fi

 % \ifpdf
%  \usepackage{pdfsync}
%  \if


%\title{Brief Article}
%\author{David F. Snyder}
%\author{L.G. Meredith}

%\address{Dept. of Math., Texas State University--San Marcos, San Marcos, TX 78666}
       
\pagestyle{empty}


\begin{document}

\lstset{language=[Objective]Caml,frame=shadowbox}

\input{qm2pi.front}

% section front matter (end)

\input{qm2pi.intro} 
 
% section introduction (end)

% \input{qm2pi.knotations} 

% section notation (end)

\input{qm2pi.process.calculi} 

% section concurrent_process_calculi_and_spatial_logics_ (end)
    
%\input{qm2pi.knots2pi} 

%\input{qm2pi.trefoil} 

%\input{qm2pi.mainthm} 

% subsection basic_interpretation (end)

%\input{qm2pi.rho.presentation} 
\subsection{The syntax and semantics of the notation system}\label{sub:the_syntax_and_semantics_of_the_notation_system} % (fold)

We now summarize a technical presentation of the calculus that
embodies our theory of dynamics. The typical presentation of such a
calculus follows the style of giving generators and relations on
them. The grammar, below, describing term constructors, freely
generates the set of processes, $\Proc$. This set is then quotiented
by a relation known as structural congruence and it is over this set
that the notion of dynamics is expressed. This presentation is
essentially that of \cite{MeredithR05} with the addition of
polyadicity and summation. For readability we have relegated some of
the technical subtleties to an appendix.

\subsubsection{Process grammar}\label{subsub:process_grammar}

\begin{mathpar}
  \inferrule* [lab=synchronization] {} {{M} \bc \pzero \;|\; x?F \;|\; x!C }
  \and
  \inferrule* [lab=abstraction] {} {{F} \bc (x)P}
  \and
  \inferrule* [lab=concretion] {} {{C} \bc \langle Q \rangle}
  \and
  \inferrule* [lab=process] {} {{P,Q} \bc M \;| \;P|Q \;|\; @{x}}
  \and
  \inferrule* [lab=name] {} {{x} \bc \quotep{P}}
\end{mathpar} 

Note that $\vec{x}$ (resp. $\vec{P}$) denotes a vector of names
(resp. processes) of length $|\vec{x}|$ (resp. $|\vec{P}|$). We adopt
the following useful abbreviations.

\begin{mathpar}
   x?(\vec{y}).P := x.(\vec{y})P \and  x\clift{\vec{P}} := x.\clift{\vec{P}}
   \and x!(y) := \lift{x}{\dropn{y}}
   \and \Pi_{i=0}^{n-1}P_i := P_0 | \ldots | P_{n-1}
\end{mathpar}

\subsubsection{Structural congruence}

\paragraph{Free and bound names and alpha-equivalence.} At the
core of structural equivalence is alpha-equivalence which identifies
process that are the same up to a change of variable. Formally, we
recognize the distinction between free and bound names. The free names
of a process, $\freenames{P}$, may be calculated recursively as
follows:

\begin{mathpar}
\freenames{\pzero} := \emptyset
  \and \\
  \freenames{x?(y).P} := \{ x \} \cup (\freenames{P} \setminus \{ y \})
  \and 
  \freenames{x!\langle P \rangle} := \{ x \} \cup \{ P \} 
  \and \\
  \freenames{P|Q} := \freenames{P} \cup \freenames{Q}
  \and \\
  \freenames{@{x}} := \{ x \}
\end{mathpar}

$\pi$
$\quotep{\pi}$

$\freenames{-} : \pi \to \mathcal{P}(\quotep{\pi})$

\begin{eqnarray*}
  \freenames{\pzero} & := & \emptyset \\
  \freenames{x?(y).P} & := & \{ x \} \cup (\freenames{P} \setminus \{ y \}) \\
  \freenames{x!\langle P \rangle} & := & \{ x \} \cup \{ P \} \\
  \freenames{P|Q} & := & \freenames{P} \cup \freenames{Q} \\
  \freenames{\dropn{x}} & := & \{ x \}
\end{eqnarray*}

The bound names of a process, $\boundnames{P}$, are those names occurring in $P$
that are not free. For example, in $x?(y).0$, the name $x$ is free, while $y$ is bound.

\begin{mathpar}
  \inferrule* [lab=monoidal-laws] {} { P|Q \equiv Q|P \and P|0 \equiv P \and P|(Q|R) \equiv (P|Q)|R }
\end{mathpar}

\begin{mathpar}
  \inferrule* [lab=alpha-equivalence] {} { (x)P \equiv (y)P\{y/x\} \and y \not\in \freenames{P} }
\end{mathpar}

\begin{definition}
Then two processes, $P,Q$, are alpha-equivalent if $P = Q\{\vec{y}/\vec{x}\}$ for
some $\vec{x} \in \boundnames{Q},\vec{y} \in \boundnames{P}$, where $Q\{\vec{y}/\vec{x}\}$
denotes the capture-avoiding substitution of $\vec{y}$ for $\vec{x}$ in $Q$.
\end{definition}

\begin{definition}
  The {\em structural congruence} \cite{SangiorgiWalker} , $\equiv$,
  between processes is the least congruence containing
  alpha-equivalence, satisfying the abelian monoid laws
  (associativity, commutativity and $\pzero$ as identity) for parallel
  composition $|$ and for summation $+$.
\end{definition}

\subsection{Name equivalence}

We take name equivalence, written $\nameeq$, to be the smallest
equivalence relation generated by the following rules.

\begin{mathpar}
\inferrule*[lab=Quote-drop]
{ }
{ \quotep{@{x}} \nameeq x }

\inferrule*[lab=Struct-equiv]
{ P \scong Q }
{ \quotep{P} \nameeq \quotep{Q} }
\end{mathpar}

The astute reader will have noticed that the mutual recursion of names
and processes imposes a mutual recursion on alpha-equivalence and
structural equivalence via name-equivalence. Fortunately, all of this
works out pleasantly and we may calculate in the natural way, free of
concern. The reader interested in the details is referred to the
appendix \ref{appendix:rho_details}.

\subsection{Substitution}

We use $\Proc$ for the set of processes, $\QProc$ for the set of
names, and $\id{\{}\vec{y} / \vec{x} \id{\}}$ to denote partial maps,
$s : \QProc \rightarrow \QProc$. A map, $s$ lifts, uniquely, to a map
on process terms, $\widehat{s} : \Proc \rightarrow \Proc$ by the
following equations.

\begin{mathpar}
  (0) \psubstp{Q}{P} := 0 \\
  (R \juxtap S) \psubstp{Q}{P}
  :=    
  (R)\psubstp{Q}{P} \juxtap (S) \psubstp{Q}{P} \\
  (x?(y).R) \psubstp{Q}{P}    
  :=    
  (x)\substp{Q}{P} (z)\concat( (R \psubstn{z}{y}) \psubstp{Q}{P} ) \\
  (\lift{x}{R}) \psubstp{Q}{P}  
  :=
  \lift{(x)\substp{Q}{P}}{ R \psubstp{Q}{P} } \\
%   (\dropn{x})  \psubstp{Q}{P}       
%   := 
%   \left\{ 
%     \begin{array}{ccc} 
%       \dropn{\quotep{Q}} & & x \nameeq \quotep{P} \\
%       \dropn{x} & & otherwise \\
%     \end{array}
%   \right. 
  (\dropn{x})  \psubstp{Q}{P}       
  := 
  \left\{ 
    \begin{array}{ccc} 
      Q & & x \nameeq \quotep{P} \\
      \dropn{x} & & otherwise \\
    \end{array}
  \right.
\end{mathpar}
 

where

\begin{eqnarray}
  (x)\id{\{} \lpquote Q \rpquote / \lpquote P \rpquote \id{\}}            = 
  \left\{ 
    \begin{array}{ccc}
      \lpquote Q \rpquote & & x \nameeq \lpquote P \rpquote \\
      x & & otherwise \\
    \end{array}
  \right. \nonumber
\end{eqnarray}

and $z$ is chosen distinct from $\quotep{P}$, $\quotep{Q}$, the free
names in $Q$, and all the names in $R$. Our $\alpha$-equivalence will
be built in the standard way from this substitution.

\begin{remark}\label{rem:no_self_referential_names}
  One consequence of these definitions is that $\forall P. \quotep{P}
  \not\in \freenames{P}$.
\end{remark}

\subsection{ Dynamic quote: an example }

Anticipating something of what's to come, consider applying the
substitution, $\widehat{\id{\{}u / z \id{\}}}$, to the following pair
of processes, $\lift{w}{y!(z)}$ and $w[ \lpquote y!(z) \rpquote ]$.

\begin{eqnarray}
	\lift{w}{y!(z)}\widehat{\id{\{}u / z \id{\}}}
		& = &
		\lift{w}{y!(u)} \nonumber\\
	w[ \lpquote y!(z) \rpquote ] \widehat{ \id{\{}u / z \id{\}} }
		& = &
		w[ \lpquote y!(z) \rpquote ] \nonumber
\end{eqnarray}

Because the body of the process between quotes is impervious to
substitution, we get radically different answers. In fact, by
examining the first process in an input context,
e.g. $x?(z).\lift{w}{y!(z)}$, we see that the process under the lift
operator may be shaped by prefixed inputs binding a name inside it. In
this sense, the lift operator will be seen as a way to dynamically
construct processes before reifying them as names.

Finally equipped with these standard features we can present the
dynamics of the calculus.

\subsubsection{Operational semantics} 

Finally, we introduce the computational dynamics. What marks these
algebras as distinct from other more traditionally studied algebraic
structures, e.g. vector spaces or polynomial rings, is the manner in
which dynamics is captured. In traditional structures, dynamics is typically
expressed through morphisms between such structures, as in linear maps
between vector spaces or morphisms between rings. In algebras
associated with the semantics of computation, the dynamics is
expressed as part of the algebraic structure itself, through a
reduction reduction relation typically denoted by $\red$. Below, we
give a recursive presentation of this relation for the calculus used
in the encoding.

$\red \subseteq \pi \times \pi$
$\red : \pi \to \mathcal{P}(\pi)$

\begin{mathpar}
  \inferrule* [lab=Comm] { \textsf{match}( x_{src}, x_{trgt} ) } { x_{trgt}?(y)P \; | \; x_{src}!\langle {Q} \rangle \red P\{\quotep{Q}/y}\} }
  \and \\
  \inferrule* [lab=Par] {{P} \red {P}'} {{{P} | {Q}} \red {{P}' | {Q}}}
  \and
  \inferrule* [lab=Equiv]{{{P} \scong {P}'} \andalso {{P}' \red {Q}'} \andalso {{Q}' \scong {Q}}}{{P} \red {Q}}
\end{mathpar}

\begin{eqnarray*}
  match_{\equiv} (\quotep{P},\quotep{Q}) & := & P \equiv Q \\
  match_{\dagger}(\quotep{P},\quotep{Q}) & := & \forall R. P|Q \red^{*} R => R \red^{*} 0 \\
  match_{K}(\quotep{P},\quotep{Q}) & := & K \mbox{ for some context } K
\end{eqnarray*}

$u?(x)P | u!\langle Q \rangle \red P\{\quotep{Q}/x\}$

%We write $\wred$ for $\red^*$, and $P\red$ if $\exists Q $ such that $ P \red Q$.
We write $P\red$ if $\exists Q $ such that $ P \red Q$ and $P\not\red$, otherwise.

\section{Replication}

As mentioned before, it is known that replication (and hence
recursion) can be implemented in a higher-order process algebra
\cite{SangiorgiWalker}. As our first example of calculation with the
machinery thus far presented we give the construction explicitly in
the {\rhoc}.

\begin{eqnarray}
	D_{x} & := & \prefix{x}{y}{(\binpar{\outputp{x}{y}}{@{y}})} \nonumber\\
	\bangp_{x}{P} & := & \binpar{{x}!\langle{\binpar{D_{x}}{P}}\rangle}{D_{x}} \nonumber
\end{eqnarray}

\begin{eqnarray}
	\bangp_{x}{P} & & \nonumber\\
	=
	& {x}!\langle{(\prefix{x}{y}{(\outputp{x}{y} | @{y})) | P}}\rangle 
	      | \prefix{x}{y}{(\outputp{x}{y} | @{y})} & \nonumber\\
	\red
	& (\outputp{x}{y} | @{y})\substn{\quotep{(\prefix{x}{y}{(@{y} | \outputp{x}{y})) | P}}}{y} & \nonumber\\
	=
	& \outputp{x}{\quotep{(\prefix{x}{y}{(\outputp{x}{y} | @{y})) | P}}}
	  | {(\prefix{x}{y}{(\outputp{x}{y} | @{y})) | P}} & \nonumber\\
	\red
	& \ldots & \nonumber\\
	\red^*
	& P | P | \ldots & \nonumber
\end{eqnarray}

Of course, this encoding, as an implementation, runs away, unfolding
$\bangp{P}$ eagerly. A lazier and more implementable replication
operator, restricted to input-guarded processes, may be obtained as follows.

\begin{eqnarray}
\bangp{\prefix{u}{v}{P}} 
	:= 
	\binpar{\lift{x}{\prefix{u}{v}{(\binpar{D(x)}{P})}}}{D(x)} \nonumber
\end{eqnarray}

\begin{remark}
  Note that the lazier definition still does not deal with summation
  or mixed summation (i.e. sums over input and output). The reader is
  invited to construct definitions of replication that deal with these
  features. 

  Further, the definitions are parameterized in a name, $x$. Can you,
  gentle reader, make a definition that eliminates this parameter and
  guarantees no accidental interaction between the replication
  machinery and the process being replicated -- i.e. no accidental
  sharing of names used by the process to get its work done and the
  name(s) used by the replication to effect copying. This latter
  revision of the definition of replication is crucial to obtaining
  the expected identity $!!P \sim !P$.
\end{remark}

\begin{remark}\label{rem:paradoxical_combinator}
  The reader familiar with the lambda calculus will have noticed the
  similarity between $D$ and the paradoxical combinator.

  [Ed. note: the existence of this seems to suggest we have to be more
  restrictive on the set of processes and names we admit if we are to
  support no-cloning.]
\end{remark}

\subsubsection{Bisimulation}

The computational dynamics gives rise to another kind of equivalence,
the equivalence of computational behavior. As previously mentioned
this is typically captured \emph{via} some form of bisimulation.

% The notion we use in this paper is weak barbed bisimulation
% \cite{milner91polyadicpi}.

The notion we use in this paper is derived from weak barbed
bisimulation \cite{milner91polyadicpi}. 

\begin{definition}
An \emph{observation relation}, $\downarrow_{\mathcal N}$, over a set
of names, $\mathcal N$, is the smallest relation satisfying the rules
below.

\infrule[Out-barb]{y \in {\mathcal N}, \; x \nameeq y}
		  {\outputp{x}{v} \downarrow_{\mathcal N} x}
\infrule[Par-barb]{\mbox{$P\downarrow_{\mathcal N} x$ or $Q\downarrow_{\mathcal N} x$}}
		  {\binpar{P}{Q} \downarrow_{\mathcal N} x}

We write $P \Downarrow_{\mathcal N} x$ if there is $Q$ such that 
$P \wred Q$ and $Q \downarrow_{\mathcal N} x$.
\end{definition}

\begin{definition}
%\label{def.bbisim}
An  ${\mathcal N}$-\emph{barbed bisimulation} over a set of names, ${\mathcal N}$, is a symmetric binary relation 
${\mathcal S}_{\mathcal N}$ between agents such that $P\rel{S}_{\mathcal N}Q$ implies:
\begin{enumerate}
\item If $P \red P'$ then $Q \wred Q'$ and $P'\rel{S}_{\mathcal N} Q'$.
\item If $P\downarrow_{\mathcal N} x$, then $Q\Downarrow_{\mathcal N} x$.
\end{enumerate}
$P$ is ${\mathcal N}$-barbed bisimilar to $Q$, written
$P \wbbisim_{\mathcal N} Q$, if $P \rel{S}_{\mathcal N} Q$ for some ${\mathcal N}$-barbed bisimulation ${\mathcal S}_{\mathcal N}$.
\end{definition}

$\mathcal{R} \subseteq \pi \times \pi$

$P \mathcal{R} Q => \forall P'. P \red P' \Rightarrow \exists Q'. Q \red Q', P' \mathcal{R} Q'$

$P \vdash x \Rightarrow Q \vdash x$

\begin{mathpar}
  \inferrule*[lab=Out-barb]{x \nameeq y}{{y}!\langle{Q}\rangle \vdash x}
  \and
  \inferrule*[lab=Par-barb]{\mbox{$P\vdash x$ or $Q\vdash x$}}{\binpar{P}{Q} \vdash x}
\end{mathpar}

\subsubsection{Contexts}

One of the principle advantages of computational calculi like the
$\pi$-calculus is a well-defined notion of context,
contextual-equivalence and a correlation between
contextual-equivalence and notions of bisimulation. The notion of
context allows the decomposition of a process into (sub-)process and
its syntactic environment, its context. Thus, a context may be
thought of as a process with a ``hole'' (written $\Box$) in it. The
application of a context $M$ to a process $P$, written $M[P]$, is
tantamount to filling the hole in $M$ with $P$. In this paper we do
not need the full weight of this theory, but do make use of the notion
of context in the proof the main theorem. 

\begin{mathpar}
  \inferrule* [lab=summation] {} {{M_{M},M_{N}} \bc \Box \;|\; x.M_{A} \;|\; M_{M}+M_{N}}
  \and
  \inferrule* [lab=agent] {} {{M_{A}} \bc (\vec{x})M_{P} \;| \; \clift{P_0,\ldots,M_{P},\ldots,P_N}}
  \and \\
  \inferrule* [lab=process] {} {{M_{P}} \bc M_{N} \;| \;P|M_{P} }
\end{mathpar} 

\begin{mathpar}
  \inferrule* [lab=sychronization] {} {M_{N} \bc \Box \;|\; x?M_{F} \;|\; x!M_{C}}
  \and
  \inferrule* [lab=abstraction] {} {{M_{F}} \bc (x)M_{P} }
  \and
  \inferrule* [lab=concretion] {} {{M_{C}} \bc \langle M_{P} \rangle }
  \and \\
  \inferrule* [lab=process] {} {{M_{P}} \bc M_{N} \;| \;P|M_{P} }
\end{mathpar}

\begin{definition}[contextual application] Given a context $M$, and
  process $P$, we define the \emph{contextual application}, $M[P] :=
  M\{P/\Box\}$. That is, the contextual application of M to P is the
  substitution of $P$ for $\Box$ in $M$.
\end{definition}

$\meaningof{-} : L \to \mathcal{P}(\pi)$

\begin{mathpar}
  \inferrule* [lab=collection] {} {\meaningof{true} = \pi, \and \meaningof{~E} = \pi \setminus \meaningof{E}, \and \meaningof{E_{1} \& E_{2}} = \meaningof{E_{1}} \cap \meaningof{E_{2}}}
\end{mathpar}

\begin{mathpar}
  \inferrule* [lab=structure] {} {\meaningof{0} = \{ P \in \pi | P \equiv 0 \}, \and \\ \meaningof{E_1 | E_2} = \{ P \in \pi | P \equiv P_{1} | P_{2}, P_{1} \in \meaningof{E_{1}}, P_{2} \in \meaningof{E_2}\} }
\end{mathpar}

\begin{mathpar}
 \inferrule* [lab=behavior] {} {\meaningof{\langle a?b \rangle E} = \{ P \in \pi | P \equiv Q | u?(y)P', \\ \and \\\\ \and \\ \;\;\; u \in \meaningof{a}, \forall z.P'\{z/y\} \in \meaningof{E\{z/b\}}\}, \and \\ \meaningof{a!E} = \{ P \in \pi | P \equiv Q | x!\langle P' \rangle, x \in \meaningof{a} P' \in \meaningof{E}\} }
\end{mathpar}

\begin{mathpar}
 \inferrule* [lab=nominal] {} {\meaningof{\quotep{E}} = \{ \quotep{P} \in \quotep{\pi} | P \in \meaningof{E} \}, \and \meaningof{\quotep{P}} = \{ \quotep{Q} \in \quotep{\pi} | P \equiv Q \} \and \\ \meaningof{@\quotep{E}} = \{ P \in \pi | P \equiv @x, x \in \meaningof{E} \}}
\end{mathpar}

\begin{eqnarray*}
  \\
  \meaningof{-} : TS \to ST
\end{eqnarray*}

\begin{eqnarray*}
  \\
  L : TS \to ST
\end{eqnarray*}

\begin{eqnarray*}
  \\
  P \models E \iff P \in \meaningof{E}
\end{eqnarray*}

\begin{eqnarray*}
  P \approx_{L} Q \iff \forall E \in L. P \models E \iff Q \models E
\end{eqnarray*}

\begin{eqnarray*}
  P \approx_{K} Q
\end{eqnarray*}

\begin{eqnarray*}
  P \approx Q
\end{eqnarray*}

$\approx_{K} = \approx = \approx_{L}$

\subsubsection{Contextual duality}

Note that contexts extend the quotation operation to a family of
operations from processes to names. Given a context, $M$, we can
define a \emph{nominal context}, $\quotep{M}$ by $\quotep{M}[P] :=
\quotep{M[P]}$. To foreshadow what is to come we observe that these
operations enjoy a duality with processes very much like the duality
between vectors and maps from vectors to scalars.

Further, because the calculus is essentially higher-order, we have a
correspondence between contexts and processes. More specifically,
given a name $x$ and a context $M$ we can construct $M^{*}_{x}$ such
that 

\begin{mathpar}
  M^{*}_{x} | \lift{x}{P} \red M[P]
\end{mathpar}

namely,

\begin{mathpar}
  M^{*}_{x} := x?(u).M[\dropn{u}]
\end{mathpar}

The dependence of $M^{*}_{x}$ on a name makes it an abstraction, 

\begin{mathpar}
  M^{*} := (x)x?(u).M[\dropn{u}]
\end{mathpar}

\subsection{Additional notation}

It will sometimes be convenient to denote the process a name
quotes. We already have the notation $x = \quotep{P}$, but it will be
convenient to introduce an alternate notation, $\procn{x}$, when we
want to emphasize the connection to the use of the name. Note that, by
virtue of name equivalence, $\quotep{\procn{x}} \nameeq x$; so, the
notation is consistent with previous definitions.

Further, because names have structure it is possible to effect
substitutions on the basis of that structure. This means we need to
upgrade our notation for substitutions, which we accomplish by
adapting comprehension notation. Thus,

\begin{mathpar}
  P\{ y / x : x \in S \}
\end{mathpar}

is interpreted to mean the process derived from P by replacing (in a
capture-avoiding manner) each occurrence of $x$ in $S$ by $y$. For example,

\begin{mathpar}
  P\{ \quotep{\procn{x}|\procn{x}} / x : x \in \freenames{P} \}
\end{mathpar}

will replace each (occurrence) of a free name $x$ in $P$ by
$\quotep{\procn{x}|\procn{x}}$.

Also, we will avail ourselves of the notation $x^{L}$ and $x^{R}$ to
denote injections of a name into disjoint copies of the name
space. There are numerous ways to accomplish this. One example can be
found in \cite{MeredithR05}. This notation overloads to vectors of
names: $\vec{x}^{\pi} := (x_{i}^{\pi} \; : \; 0 \leq i < |\vec{x}| )$ where $\pi \in \{L,R\}$.

We also use $P^{\Box} := P|\Box$.

In \cite{MeredithR05} an interpretation of the new operator is
given. It turns out that there are several possible interpretations
all enjoying the requisite algebraic properties of the operator (see
\cite{milner91polyadicpi}). We will therefore make liberal use of
$(\nu\; \vec{x})P$.

% subsection the_syntax_and_semantics_of_the_notation_system (end)   

\input{qm2pi.qmops} 

\input{qm2pi.sterngerlach} 

\input{qm2pi.metric} 

% section concurrent_process_calculi (end)

%\input{qm2pi.proofsketch}

% section proof sketch (end)

%\input{qm2pi.slviaknots} 

% section spatial logic via knots (end)

\input{qm2pi.conclusion}

% section conclusion (end)

%\input{qm2pi.dtcodes} 

% section wiring algorithm (end)

\input{qm2pi.ack} 

% section acknowledgments (end)

\newpage


\bibliographystyle{plain}   
\bibliography{../../biblios/main.bib}

\input{qm2pi.rhodetails}

\end{document}

 

% section notation (end)

\input{qm2pi.process.calculi} 

% section concurrent_process_calculi_and_spatial_logics_ (end)
    
%\documentclass[12pt]{llncs}
%\documentclass{jktr}

\usepackage[pdftex]{hyperref}                   
\usepackage {listings}
\usepackage {mathpartir}
\usepackage{bcprules}
%\usepackage{listings}
                       
\usepackage{graphicx} 
%\usepackage[margins=2.5cm,nohead,nofoot]{geometry}
%\usepackage{geometry}
\usepackage{amsfonts}
\usepackage{amstext}
\usepackage{latexsym}
\usepackage{amssymb}
\usepackage{color}


%\include{myPreamble}
\include{qm2pi.local} 

%\ifpdf
%\usepackage[pdftex]{graphicx}
%\else
%\usepackage{graphicx}
%\fi

 % \ifpdf
%  \usepackage{pdfsync}
%  \if


%\title{Brief Article}
%\author{David F. Snyder}
%\author{L.G. Meredith}

%\address{Dept. of Math., Texas State University--San Marcos, San Marcos, TX 78666}
       
\pagestyle{empty}


\begin{document}

\lstset{language=[Objective]Caml,frame=shadowbox}

\input{qm2pi.front}

% section front matter (end)

\input{qm2pi.intro} 
 
% section introduction (end)

% \input{qm2pi.knotations} 

% section notation (end)

\input{qm2pi.process.calculi} 

% section concurrent_process_calculi_and_spatial_logics_ (end)
    
%\input{qm2pi.knots2pi} 

%\input{qm2pi.trefoil} 

%\input{qm2pi.mainthm} 

% subsection basic_interpretation (end)

%\input{qm2pi.rho.presentation} 
\subsection{The syntax and semantics of the notation system}\label{sub:the_syntax_and_semantics_of_the_notation_system} % (fold)

We now summarize a technical presentation of the calculus that
embodies our theory of dynamics. The typical presentation of such a
calculus follows the style of giving generators and relations on
them. The grammar, below, describing term constructors, freely
generates the set of processes, $\Proc$. This set is then quotiented
by a relation known as structural congruence and it is over this set
that the notion of dynamics is expressed. This presentation is
essentially that of \cite{MeredithR05} with the addition of
polyadicity and summation. For readability we have relegated some of
the technical subtleties to an appendix.

\subsubsection{Process grammar}\label{subsub:process_grammar}

\begin{mathpar}
  \inferrule* [lab=synchronization] {} {{M} \bc \pzero \;|\; x?F \;|\; x!C }
  \and
  \inferrule* [lab=abstraction] {} {{F} \bc (x)P}
  \and
  \inferrule* [lab=concretion] {} {{C} \bc \langle Q \rangle}
  \and
  \inferrule* [lab=process] {} {{P,Q} \bc M \;| \;P|Q \;|\; @{x}}
  \and
  \inferrule* [lab=name] {} {{x} \bc \quotep{P}}
\end{mathpar} 

Note that $\vec{x}$ (resp. $\vec{P}$) denotes a vector of names
(resp. processes) of length $|\vec{x}|$ (resp. $|\vec{P}|$). We adopt
the following useful abbreviations.

\begin{mathpar}
   x?(\vec{y}).P := x.(\vec{y})P \and  x\clift{\vec{P}} := x.\clift{\vec{P}}
   \and x!(y) := \lift{x}{\dropn{y}}
   \and \Pi_{i=0}^{n-1}P_i := P_0 | \ldots | P_{n-1}
\end{mathpar}

\subsubsection{Structural congruence}

\paragraph{Free and bound names and alpha-equivalence.} At the
core of structural equivalence is alpha-equivalence which identifies
process that are the same up to a change of variable. Formally, we
recognize the distinction between free and bound names. The free names
of a process, $\freenames{P}$, may be calculated recursively as
follows:

\begin{mathpar}
\freenames{\pzero} := \emptyset
  \and \\
  \freenames{x?(y).P} := \{ x \} \cup (\freenames{P} \setminus \{ y \})
  \and 
  \freenames{x!\langle P \rangle} := \{ x \} \cup \{ P \} 
  \and \\
  \freenames{P|Q} := \freenames{P} \cup \freenames{Q}
  \and \\
  \freenames{@{x}} := \{ x \}
\end{mathpar}

$\pi$
$\quotep{\pi}$

$\freenames{-} : \pi \to \mathcal{P}(\quotep{\pi})$

\begin{eqnarray*}
  \freenames{\pzero} & := & \emptyset \\
  \freenames{x?(y).P} & := & \{ x \} \cup (\freenames{P} \setminus \{ y \}) \\
  \freenames{x!\langle P \rangle} & := & \{ x \} \cup \{ P \} \\
  \freenames{P|Q} & := & \freenames{P} \cup \freenames{Q} \\
  \freenames{\dropn{x}} & := & \{ x \}
\end{eqnarray*}

The bound names of a process, $\boundnames{P}$, are those names occurring in $P$
that are not free. For example, in $x?(y).0$, the name $x$ is free, while $y$ is bound.

\begin{mathpar}
  \inferrule* [lab=monoidal-laws] {} { P|Q \equiv Q|P \and P|0 \equiv P \and P|(Q|R) \equiv (P|Q)|R }
\end{mathpar}

\begin{mathpar}
  \inferrule* [lab=alpha-equivalence] {} { (x)P \equiv (y)P\{y/x\} \and y \not\in \freenames{P} }
\end{mathpar}

\begin{definition}
Then two processes, $P,Q$, are alpha-equivalent if $P = Q\{\vec{y}/\vec{x}\}$ for
some $\vec{x} \in \boundnames{Q},\vec{y} \in \boundnames{P}$, where $Q\{\vec{y}/\vec{x}\}$
denotes the capture-avoiding substitution of $\vec{y}$ for $\vec{x}$ in $Q$.
\end{definition}

\begin{definition}
  The {\em structural congruence} \cite{SangiorgiWalker} , $\equiv$,
  between processes is the least congruence containing
  alpha-equivalence, satisfying the abelian monoid laws
  (associativity, commutativity and $\pzero$ as identity) for parallel
  composition $|$ and for summation $+$.
\end{definition}

\subsection{Name equivalence}

We take name equivalence, written $\nameeq$, to be the smallest
equivalence relation generated by the following rules.

\begin{mathpar}
\inferrule*[lab=Quote-drop]
{ }
{ \quotep{@{x}} \nameeq x }

\inferrule*[lab=Struct-equiv]
{ P \scong Q }
{ \quotep{P} \nameeq \quotep{Q} }
\end{mathpar}

The astute reader will have noticed that the mutual recursion of names
and processes imposes a mutual recursion on alpha-equivalence and
structural equivalence via name-equivalence. Fortunately, all of this
works out pleasantly and we may calculate in the natural way, free of
concern. The reader interested in the details is referred to the
appendix \ref{appendix:rho_details}.

\subsection{Substitution}

We use $\Proc$ for the set of processes, $\QProc$ for the set of
names, and $\id{\{}\vec{y} / \vec{x} \id{\}}$ to denote partial maps,
$s : \QProc \rightarrow \QProc$. A map, $s$ lifts, uniquely, to a map
on process terms, $\widehat{s} : \Proc \rightarrow \Proc$ by the
following equations.

\begin{mathpar}
  (0) \psubstp{Q}{P} := 0 \\
  (R \juxtap S) \psubstp{Q}{P}
  :=    
  (R)\psubstp{Q}{P} \juxtap (S) \psubstp{Q}{P} \\
  (x?(y).R) \psubstp{Q}{P}    
  :=    
  (x)\substp{Q}{P} (z)\concat( (R \psubstn{z}{y}) \psubstp{Q}{P} ) \\
  (\lift{x}{R}) \psubstp{Q}{P}  
  :=
  \lift{(x)\substp{Q}{P}}{ R \psubstp{Q}{P} } \\
%   (\dropn{x})  \psubstp{Q}{P}       
%   := 
%   \left\{ 
%     \begin{array}{ccc} 
%       \dropn{\quotep{Q}} & & x \nameeq \quotep{P} \\
%       \dropn{x} & & otherwise \\
%     \end{array}
%   \right. 
  (\dropn{x})  \psubstp{Q}{P}       
  := 
  \left\{ 
    \begin{array}{ccc} 
      Q & & x \nameeq \quotep{P} \\
      \dropn{x} & & otherwise \\
    \end{array}
  \right.
\end{mathpar}
 

where

\begin{eqnarray}
  (x)\id{\{} \lpquote Q \rpquote / \lpquote P \rpquote \id{\}}            = 
  \left\{ 
    \begin{array}{ccc}
      \lpquote Q \rpquote & & x \nameeq \lpquote P \rpquote \\
      x & & otherwise \\
    \end{array}
  \right. \nonumber
\end{eqnarray}

and $z$ is chosen distinct from $\quotep{P}$, $\quotep{Q}$, the free
names in $Q$, and all the names in $R$. Our $\alpha$-equivalence will
be built in the standard way from this substitution.

\begin{remark}\label{rem:no_self_referential_names}
  One consequence of these definitions is that $\forall P. \quotep{P}
  \not\in \freenames{P}$.
\end{remark}

\subsection{ Dynamic quote: an example }

Anticipating something of what's to come, consider applying the
substitution, $\widehat{\id{\{}u / z \id{\}}}$, to the following pair
of processes, $\lift{w}{y!(z)}$ and $w[ \lpquote y!(z) \rpquote ]$.

\begin{eqnarray}
	\lift{w}{y!(z)}\widehat{\id{\{}u / z \id{\}}}
		& = &
		\lift{w}{y!(u)} \nonumber\\
	w[ \lpquote y!(z) \rpquote ] \widehat{ \id{\{}u / z \id{\}} }
		& = &
		w[ \lpquote y!(z) \rpquote ] \nonumber
\end{eqnarray}

Because the body of the process between quotes is impervious to
substitution, we get radically different answers. In fact, by
examining the first process in an input context,
e.g. $x?(z).\lift{w}{y!(z)}$, we see that the process under the lift
operator may be shaped by prefixed inputs binding a name inside it. In
this sense, the lift operator will be seen as a way to dynamically
construct processes before reifying them as names.

Finally equipped with these standard features we can present the
dynamics of the calculus.

\subsubsection{Operational semantics} 

Finally, we introduce the computational dynamics. What marks these
algebras as distinct from other more traditionally studied algebraic
structures, e.g. vector spaces or polynomial rings, is the manner in
which dynamics is captured. In traditional structures, dynamics is typically
expressed through morphisms between such structures, as in linear maps
between vector spaces or morphisms between rings. In algebras
associated with the semantics of computation, the dynamics is
expressed as part of the algebraic structure itself, through a
reduction reduction relation typically denoted by $\red$. Below, we
give a recursive presentation of this relation for the calculus used
in the encoding.

$\red \subseteq \pi \times \pi$
$\red : \pi \to \mathcal{P}(\pi)$

\begin{mathpar}
  \inferrule* [lab=Comm] { \textsf{match}( x_{src}, x_{trgt} ) } { x_{trgt}?(y)P \; | \; x_{src}!\langle {Q} \rangle \red P\{\quotep{Q}/y}\} }
  \and \\
  \inferrule* [lab=Par] {{P} \red {P}'} {{{P} | {Q}} \red {{P}' | {Q}}}
  \and
  \inferrule* [lab=Equiv]{{{P} \scong {P}'} \andalso {{P}' \red {Q}'} \andalso {{Q}' \scong {Q}}}{{P} \red {Q}}
\end{mathpar}

\begin{eqnarray*}
  match_{\equiv} (\quotep{P},\quotep{Q}) & := & P \equiv Q \\
  match_{\dagger}(\quotep{P},\quotep{Q}) & := & \forall R. P|Q \red^{*} R => R \red^{*} 0 \\
  match_{K}(\quotep{P},\quotep{Q}) & := & K \mbox{ for some context } K
\end{eqnarray*}

$u?(x)P | u!\langle Q \rangle \red P\{\quotep{Q}/x\}$

%We write $\wred$ for $\red^*$, and $P\red$ if $\exists Q $ such that $ P \red Q$.
We write $P\red$ if $\exists Q $ such that $ P \red Q$ and $P\not\red$, otherwise.

\section{Replication}

As mentioned before, it is known that replication (and hence
recursion) can be implemented in a higher-order process algebra
\cite{SangiorgiWalker}. As our first example of calculation with the
machinery thus far presented we give the construction explicitly in
the {\rhoc}.

\begin{eqnarray}
	D_{x} & := & \prefix{x}{y}{(\binpar{\outputp{x}{y}}{@{y}})} \nonumber\\
	\bangp_{x}{P} & := & \binpar{{x}!\langle{\binpar{D_{x}}{P}}\rangle}{D_{x}} \nonumber
\end{eqnarray}

\begin{eqnarray}
	\bangp_{x}{P} & & \nonumber\\
	=
	& {x}!\langle{(\prefix{x}{y}{(\outputp{x}{y} | @{y})) | P}}\rangle 
	      | \prefix{x}{y}{(\outputp{x}{y} | @{y})} & \nonumber\\
	\red
	& (\outputp{x}{y} | @{y})\substn{\quotep{(\prefix{x}{y}{(@{y} | \outputp{x}{y})) | P}}}{y} & \nonumber\\
	=
	& \outputp{x}{\quotep{(\prefix{x}{y}{(\outputp{x}{y} | @{y})) | P}}}
	  | {(\prefix{x}{y}{(\outputp{x}{y} | @{y})) | P}} & \nonumber\\
	\red
	& \ldots & \nonumber\\
	\red^*
	& P | P | \ldots & \nonumber
\end{eqnarray}

Of course, this encoding, as an implementation, runs away, unfolding
$\bangp{P}$ eagerly. A lazier and more implementable replication
operator, restricted to input-guarded processes, may be obtained as follows.

\begin{eqnarray}
\bangp{\prefix{u}{v}{P}} 
	:= 
	\binpar{\lift{x}{\prefix{u}{v}{(\binpar{D(x)}{P})}}}{D(x)} \nonumber
\end{eqnarray}

\begin{remark}
  Note that the lazier definition still does not deal with summation
  or mixed summation (i.e. sums over input and output). The reader is
  invited to construct definitions of replication that deal with these
  features. 

  Further, the definitions are parameterized in a name, $x$. Can you,
  gentle reader, make a definition that eliminates this parameter and
  guarantees no accidental interaction between the replication
  machinery and the process being replicated -- i.e. no accidental
  sharing of names used by the process to get its work done and the
  name(s) used by the replication to effect copying. This latter
  revision of the definition of replication is crucial to obtaining
  the expected identity $!!P \sim !P$.
\end{remark}

\begin{remark}\label{rem:paradoxical_combinator}
  The reader familiar with the lambda calculus will have noticed the
  similarity between $D$ and the paradoxical combinator.

  [Ed. note: the existence of this seems to suggest we have to be more
  restrictive on the set of processes and names we admit if we are to
  support no-cloning.]
\end{remark}

\subsubsection{Bisimulation}

The computational dynamics gives rise to another kind of equivalence,
the equivalence of computational behavior. As previously mentioned
this is typically captured \emph{via} some form of bisimulation.

% The notion we use in this paper is weak barbed bisimulation
% \cite{milner91polyadicpi}.

The notion we use in this paper is derived from weak barbed
bisimulation \cite{milner91polyadicpi}. 

\begin{definition}
An \emph{observation relation}, $\downarrow_{\mathcal N}$, over a set
of names, $\mathcal N$, is the smallest relation satisfying the rules
below.

\infrule[Out-barb]{y \in {\mathcal N}, \; x \nameeq y}
		  {\outputp{x}{v} \downarrow_{\mathcal N} x}
\infrule[Par-barb]{\mbox{$P\downarrow_{\mathcal N} x$ or $Q\downarrow_{\mathcal N} x$}}
		  {\binpar{P}{Q} \downarrow_{\mathcal N} x}

We write $P \Downarrow_{\mathcal N} x$ if there is $Q$ such that 
$P \wred Q$ and $Q \downarrow_{\mathcal N} x$.
\end{definition}

\begin{definition}
%\label{def.bbisim}
An  ${\mathcal N}$-\emph{barbed bisimulation} over a set of names, ${\mathcal N}$, is a symmetric binary relation 
${\mathcal S}_{\mathcal N}$ between agents such that $P\rel{S}_{\mathcal N}Q$ implies:
\begin{enumerate}
\item If $P \red P'$ then $Q \wred Q'$ and $P'\rel{S}_{\mathcal N} Q'$.
\item If $P\downarrow_{\mathcal N} x$, then $Q\Downarrow_{\mathcal N} x$.
\end{enumerate}
$P$ is ${\mathcal N}$-barbed bisimilar to $Q$, written
$P \wbbisim_{\mathcal N} Q$, if $P \rel{S}_{\mathcal N} Q$ for some ${\mathcal N}$-barbed bisimulation ${\mathcal S}_{\mathcal N}$.
\end{definition}

$\mathcal{R} \subseteq \pi \times \pi$

$P \mathcal{R} Q => \forall P'. P \red P' \Rightarrow \exists Q'. Q \red Q', P' \mathcal{R} Q'$

$P \vdash x \Rightarrow Q \vdash x$

\begin{mathpar}
  \inferrule*[lab=Out-barb]{x \nameeq y}{{y}!\langle{Q}\rangle \vdash x}
  \and
  \inferrule*[lab=Par-barb]{\mbox{$P\vdash x$ or $Q\vdash x$}}{\binpar{P}{Q} \vdash x}
\end{mathpar}

\subsubsection{Contexts}

One of the principle advantages of computational calculi like the
$\pi$-calculus is a well-defined notion of context,
contextual-equivalence and a correlation between
contextual-equivalence and notions of bisimulation. The notion of
context allows the decomposition of a process into (sub-)process and
its syntactic environment, its context. Thus, a context may be
thought of as a process with a ``hole'' (written $\Box$) in it. The
application of a context $M$ to a process $P$, written $M[P]$, is
tantamount to filling the hole in $M$ with $P$. In this paper we do
not need the full weight of this theory, but do make use of the notion
of context in the proof the main theorem. 

\begin{mathpar}
  \inferrule* [lab=summation] {} {{M_{M},M_{N}} \bc \Box \;|\; x.M_{A} \;|\; M_{M}+M_{N}}
  \and
  \inferrule* [lab=agent] {} {{M_{A}} \bc (\vec{x})M_{P} \;| \; \clift{P_0,\ldots,M_{P},\ldots,P_N}}
  \and \\
  \inferrule* [lab=process] {} {{M_{P}} \bc M_{N} \;| \;P|M_{P} }
\end{mathpar} 

\begin{mathpar}
  \inferrule* [lab=sychronization] {} {M_{N} \bc \Box \;|\; x?M_{F} \;|\; x!M_{C}}
  \and
  \inferrule* [lab=abstraction] {} {{M_{F}} \bc (x)M_{P} }
  \and
  \inferrule* [lab=concretion] {} {{M_{C}} \bc \langle M_{P} \rangle }
  \and \\
  \inferrule* [lab=process] {} {{M_{P}} \bc M_{N} \;| \;P|M_{P} }
\end{mathpar}

\begin{definition}[contextual application] Given a context $M$, and
  process $P$, we define the \emph{contextual application}, $M[P] :=
  M\{P/\Box\}$. That is, the contextual application of M to P is the
  substitution of $P$ for $\Box$ in $M$.
\end{definition}

$\meaningof{-} : L \to \mathcal{P}(\pi)$

\begin{mathpar}
  \inferrule* [lab=collection] {} {\meaningof{true} = \pi, \and \meaningof{~E} = \pi \setminus \meaningof{E}, \and \meaningof{E_{1} \& E_{2}} = \meaningof{E_{1}} \cap \meaningof{E_{2}}}
\end{mathpar}

\begin{mathpar}
  \inferrule* [lab=structure] {} {\meaningof{0} = \{ P \in \pi | P \equiv 0 \}, \and \\ \meaningof{E_1 | E_2} = \{ P \in \pi | P \equiv P_{1} | P_{2}, P_{1} \in \meaningof{E_{1}}, P_{2} \in \meaningof{E_2}\} }
\end{mathpar}

\begin{mathpar}
 \inferrule* [lab=behavior] {} {\meaningof{\langle a?b \rangle E} = \{ P \in \pi | P \equiv Q | u?(y)P', \\ \and \\\\ \and \\ \;\;\; u \in \meaningof{a}, \forall z.P'\{z/y\} \in \meaningof{E\{z/b\}}\}, \and \\ \meaningof{a!E} = \{ P \in \pi | P \equiv Q | x!\langle P' \rangle, x \in \meaningof{a} P' \in \meaningof{E}\} }
\end{mathpar}

\begin{mathpar}
 \inferrule* [lab=nominal] {} {\meaningof{\quotep{E}} = \{ \quotep{P} \in \quotep{\pi} | P \in \meaningof{E} \}, \and \meaningof{\quotep{P}} = \{ \quotep{Q} \in \quotep{\pi} | P \equiv Q \} \and \\ \meaningof{@\quotep{E}} = \{ P \in \pi | P \equiv @x, x \in \meaningof{E} \}}
\end{mathpar}

\begin{eqnarray*}
  \\
  \meaningof{-} : TS \to ST
\end{eqnarray*}

\begin{eqnarray*}
  \\
  L : TS \to ST
\end{eqnarray*}

\begin{eqnarray*}
  \\
  P \models E \iff P \in \meaningof{E}
\end{eqnarray*}

\begin{eqnarray*}
  P \approx_{L} Q \iff \forall E \in L. P \models E \iff Q \models E
\end{eqnarray*}

\begin{eqnarray*}
  P \approx_{K} Q
\end{eqnarray*}

\begin{eqnarray*}
  P \approx Q
\end{eqnarray*}

$\approx_{K} = \approx = \approx_{L}$

\subsubsection{Contextual duality}

Note that contexts extend the quotation operation to a family of
operations from processes to names. Given a context, $M$, we can
define a \emph{nominal context}, $\quotep{M}$ by $\quotep{M}[P] :=
\quotep{M[P]}$. To foreshadow what is to come we observe that these
operations enjoy a duality with processes very much like the duality
between vectors and maps from vectors to scalars.

Further, because the calculus is essentially higher-order, we have a
correspondence between contexts and processes. More specifically,
given a name $x$ and a context $M$ we can construct $M^{*}_{x}$ such
that 

\begin{mathpar}
  M^{*}_{x} | \lift{x}{P} \red M[P]
\end{mathpar}

namely,

\begin{mathpar}
  M^{*}_{x} := x?(u).M[\dropn{u}]
\end{mathpar}

The dependence of $M^{*}_{x}$ on a name makes it an abstraction, 

\begin{mathpar}
  M^{*} := (x)x?(u).M[\dropn{u}]
\end{mathpar}

\subsection{Additional notation}

It will sometimes be convenient to denote the process a name
quotes. We already have the notation $x = \quotep{P}$, but it will be
convenient to introduce an alternate notation, $\procn{x}$, when we
want to emphasize the connection to the use of the name. Note that, by
virtue of name equivalence, $\quotep{\procn{x}} \nameeq x$; so, the
notation is consistent with previous definitions.

Further, because names have structure it is possible to effect
substitutions on the basis of that structure. This means we need to
upgrade our notation for substitutions, which we accomplish by
adapting comprehension notation. Thus,

\begin{mathpar}
  P\{ y / x : x \in S \}
\end{mathpar}

is interpreted to mean the process derived from P by replacing (in a
capture-avoiding manner) each occurrence of $x$ in $S$ by $y$. For example,

\begin{mathpar}
  P\{ \quotep{\procn{x}|\procn{x}} / x : x \in \freenames{P} \}
\end{mathpar}

will replace each (occurrence) of a free name $x$ in $P$ by
$\quotep{\procn{x}|\procn{x}}$.

Also, we will avail ourselves of the notation $x^{L}$ and $x^{R}$ to
denote injections of a name into disjoint copies of the name
space. There are numerous ways to accomplish this. One example can be
found in \cite{MeredithR05}. This notation overloads to vectors of
names: $\vec{x}^{\pi} := (x_{i}^{\pi} \; : \; 0 \leq i < |\vec{x}| )$ where $\pi \in \{L,R\}$.

We also use $P^{\Box} := P|\Box$.

In \cite{MeredithR05} an interpretation of the new operator is
given. It turns out that there are several possible interpretations
all enjoying the requisite algebraic properties of the operator (see
\cite{milner91polyadicpi}). We will therefore make liberal use of
$(\nu\; \vec{x})P$.

% subsection the_syntax_and_semantics_of_the_notation_system (end)   

\input{qm2pi.qmops} 

\input{qm2pi.sterngerlach} 

\input{qm2pi.metric} 

% section concurrent_process_calculi (end)

%\input{qm2pi.proofsketch}

% section proof sketch (end)

%\input{qm2pi.slviaknots} 

% section spatial logic via knots (end)

\input{qm2pi.conclusion}

% section conclusion (end)

%\input{qm2pi.dtcodes} 

% section wiring algorithm (end)

\input{qm2pi.ack} 

% section acknowledgments (end)

\newpage


\bibliographystyle{plain}   
\bibliography{../../biblios/main.bib}

\input{qm2pi.rhodetails}

\end{document}

 

%\documentclass[12pt]{llncs}
%\documentclass{jktr}

\usepackage[pdftex]{hyperref}                   
\usepackage {listings}
\usepackage {mathpartir}
\usepackage{bcprules}
%\usepackage{listings}
                       
\usepackage{graphicx} 
%\usepackage[margins=2.5cm,nohead,nofoot]{geometry}
%\usepackage{geometry}
\usepackage{amsfonts}
\usepackage{amstext}
\usepackage{latexsym}
\usepackage{amssymb}
\usepackage{color}


%\include{myPreamble}
\include{qm2pi.local} 

%\ifpdf
%\usepackage[pdftex]{graphicx}
%\else
%\usepackage{graphicx}
%\fi

 % \ifpdf
%  \usepackage{pdfsync}
%  \if


%\title{Brief Article}
%\author{David F. Snyder}
%\author{L.G. Meredith}

%\address{Dept. of Math., Texas State University--San Marcos, San Marcos, TX 78666}
       
\pagestyle{empty}


\begin{document}

\lstset{language=[Objective]Caml,frame=shadowbox}

\input{qm2pi.front}

% section front matter (end)

\input{qm2pi.intro} 
 
% section introduction (end)

% \input{qm2pi.knotations} 

% section notation (end)

\input{qm2pi.process.calculi} 

% section concurrent_process_calculi_and_spatial_logics_ (end)
    
%\input{qm2pi.knots2pi} 

%\input{qm2pi.trefoil} 

%\input{qm2pi.mainthm} 

% subsection basic_interpretation (end)

%\input{qm2pi.rho.presentation} 
\subsection{The syntax and semantics of the notation system}\label{sub:the_syntax_and_semantics_of_the_notation_system} % (fold)

We now summarize a technical presentation of the calculus that
embodies our theory of dynamics. The typical presentation of such a
calculus follows the style of giving generators and relations on
them. The grammar, below, describing term constructors, freely
generates the set of processes, $\Proc$. This set is then quotiented
by a relation known as structural congruence and it is over this set
that the notion of dynamics is expressed. This presentation is
essentially that of \cite{MeredithR05} with the addition of
polyadicity and summation. For readability we have relegated some of
the technical subtleties to an appendix.

\subsubsection{Process grammar}\label{subsub:process_grammar}

\begin{mathpar}
  \inferrule* [lab=synchronization] {} {{M} \bc \pzero \;|\; x?F \;|\; x!C }
  \and
  \inferrule* [lab=abstraction] {} {{F} \bc (x)P}
  \and
  \inferrule* [lab=concretion] {} {{C} \bc \langle Q \rangle}
  \and
  \inferrule* [lab=process] {} {{P,Q} \bc M \;| \;P|Q \;|\; @{x}}
  \and
  \inferrule* [lab=name] {} {{x} \bc \quotep{P}}
\end{mathpar} 

Note that $\vec{x}$ (resp. $\vec{P}$) denotes a vector of names
(resp. processes) of length $|\vec{x}|$ (resp. $|\vec{P}|$). We adopt
the following useful abbreviations.

\begin{mathpar}
   x?(\vec{y}).P := x.(\vec{y})P \and  x\clift{\vec{P}} := x.\clift{\vec{P}}
   \and x!(y) := \lift{x}{\dropn{y}}
   \and \Pi_{i=0}^{n-1}P_i := P_0 | \ldots | P_{n-1}
\end{mathpar}

\subsubsection{Structural congruence}

\paragraph{Free and bound names and alpha-equivalence.} At the
core of structural equivalence is alpha-equivalence which identifies
process that are the same up to a change of variable. Formally, we
recognize the distinction between free and bound names. The free names
of a process, $\freenames{P}$, may be calculated recursively as
follows:

\begin{mathpar}
\freenames{\pzero} := \emptyset
  \and \\
  \freenames{x?(y).P} := \{ x \} \cup (\freenames{P} \setminus \{ y \})
  \and 
  \freenames{x!\langle P \rangle} := \{ x \} \cup \{ P \} 
  \and \\
  \freenames{P|Q} := \freenames{P} \cup \freenames{Q}
  \and \\
  \freenames{@{x}} := \{ x \}
\end{mathpar}

$\pi$
$\quotep{\pi}$

$\freenames{-} : \pi \to \mathcal{P}(\quotep{\pi})$

\begin{eqnarray*}
  \freenames{\pzero} & := & \emptyset \\
  \freenames{x?(y).P} & := & \{ x \} \cup (\freenames{P} \setminus \{ y \}) \\
  \freenames{x!\langle P \rangle} & := & \{ x \} \cup \{ P \} \\
  \freenames{P|Q} & := & \freenames{P} \cup \freenames{Q} \\
  \freenames{\dropn{x}} & := & \{ x \}
\end{eqnarray*}

The bound names of a process, $\boundnames{P}$, are those names occurring in $P$
that are not free. For example, in $x?(y).0$, the name $x$ is free, while $y$ is bound.

\begin{mathpar}
  \inferrule* [lab=monoidal-laws] {} { P|Q \equiv Q|P \and P|0 \equiv P \and P|(Q|R) \equiv (P|Q)|R }
\end{mathpar}

\begin{mathpar}
  \inferrule* [lab=alpha-equivalence] {} { (x)P \equiv (y)P\{y/x\} \and y \not\in \freenames{P} }
\end{mathpar}

\begin{definition}
Then two processes, $P,Q$, are alpha-equivalent if $P = Q\{\vec{y}/\vec{x}\}$ for
some $\vec{x} \in \boundnames{Q},\vec{y} \in \boundnames{P}$, where $Q\{\vec{y}/\vec{x}\}$
denotes the capture-avoiding substitution of $\vec{y}$ for $\vec{x}$ in $Q$.
\end{definition}

\begin{definition}
  The {\em structural congruence} \cite{SangiorgiWalker} , $\equiv$,
  between processes is the least congruence containing
  alpha-equivalence, satisfying the abelian monoid laws
  (associativity, commutativity and $\pzero$ as identity) for parallel
  composition $|$ and for summation $+$.
\end{definition}

\subsection{Name equivalence}

We take name equivalence, written $\nameeq$, to be the smallest
equivalence relation generated by the following rules.

\begin{mathpar}
\inferrule*[lab=Quote-drop]
{ }
{ \quotep{@{x}} \nameeq x }

\inferrule*[lab=Struct-equiv]
{ P \scong Q }
{ \quotep{P} \nameeq \quotep{Q} }
\end{mathpar}

The astute reader will have noticed that the mutual recursion of names
and processes imposes a mutual recursion on alpha-equivalence and
structural equivalence via name-equivalence. Fortunately, all of this
works out pleasantly and we may calculate in the natural way, free of
concern. The reader interested in the details is referred to the
appendix \ref{appendix:rho_details}.

\subsection{Substitution}

We use $\Proc$ for the set of processes, $\QProc$ for the set of
names, and $\id{\{}\vec{y} / \vec{x} \id{\}}$ to denote partial maps,
$s : \QProc \rightarrow \QProc$. A map, $s$ lifts, uniquely, to a map
on process terms, $\widehat{s} : \Proc \rightarrow \Proc$ by the
following equations.

\begin{mathpar}
  (0) \psubstp{Q}{P} := 0 \\
  (R \juxtap S) \psubstp{Q}{P}
  :=    
  (R)\psubstp{Q}{P} \juxtap (S) \psubstp{Q}{P} \\
  (x?(y).R) \psubstp{Q}{P}    
  :=    
  (x)\substp{Q}{P} (z)\concat( (R \psubstn{z}{y}) \psubstp{Q}{P} ) \\
  (\lift{x}{R}) \psubstp{Q}{P}  
  :=
  \lift{(x)\substp{Q}{P}}{ R \psubstp{Q}{P} } \\
%   (\dropn{x})  \psubstp{Q}{P}       
%   := 
%   \left\{ 
%     \begin{array}{ccc} 
%       \dropn{\quotep{Q}} & & x \nameeq \quotep{P} \\
%       \dropn{x} & & otherwise \\
%     \end{array}
%   \right. 
  (\dropn{x})  \psubstp{Q}{P}       
  := 
  \left\{ 
    \begin{array}{ccc} 
      Q & & x \nameeq \quotep{P} \\
      \dropn{x} & & otherwise \\
    \end{array}
  \right.
\end{mathpar}
 

where

\begin{eqnarray}
  (x)\id{\{} \lpquote Q \rpquote / \lpquote P \rpquote \id{\}}            = 
  \left\{ 
    \begin{array}{ccc}
      \lpquote Q \rpquote & & x \nameeq \lpquote P \rpquote \\
      x & & otherwise \\
    \end{array}
  \right. \nonumber
\end{eqnarray}

and $z$ is chosen distinct from $\quotep{P}$, $\quotep{Q}$, the free
names in $Q$, and all the names in $R$. Our $\alpha$-equivalence will
be built in the standard way from this substitution.

\begin{remark}\label{rem:no_self_referential_names}
  One consequence of these definitions is that $\forall P. \quotep{P}
  \not\in \freenames{P}$.
\end{remark}

\subsection{ Dynamic quote: an example }

Anticipating something of what's to come, consider applying the
substitution, $\widehat{\id{\{}u / z \id{\}}}$, to the following pair
of processes, $\lift{w}{y!(z)}$ and $w[ \lpquote y!(z) \rpquote ]$.

\begin{eqnarray}
	\lift{w}{y!(z)}\widehat{\id{\{}u / z \id{\}}}
		& = &
		\lift{w}{y!(u)} \nonumber\\
	w[ \lpquote y!(z) \rpquote ] \widehat{ \id{\{}u / z \id{\}} }
		& = &
		w[ \lpquote y!(z) \rpquote ] \nonumber
\end{eqnarray}

Because the body of the process between quotes is impervious to
substitution, we get radically different answers. In fact, by
examining the first process in an input context,
e.g. $x?(z).\lift{w}{y!(z)}$, we see that the process under the lift
operator may be shaped by prefixed inputs binding a name inside it. In
this sense, the lift operator will be seen as a way to dynamically
construct processes before reifying them as names.

Finally equipped with these standard features we can present the
dynamics of the calculus.

\subsubsection{Operational semantics} 

Finally, we introduce the computational dynamics. What marks these
algebras as distinct from other more traditionally studied algebraic
structures, e.g. vector spaces or polynomial rings, is the manner in
which dynamics is captured. In traditional structures, dynamics is typically
expressed through morphisms between such structures, as in linear maps
between vector spaces or morphisms between rings. In algebras
associated with the semantics of computation, the dynamics is
expressed as part of the algebraic structure itself, through a
reduction reduction relation typically denoted by $\red$. Below, we
give a recursive presentation of this relation for the calculus used
in the encoding.

$\red \subseteq \pi \times \pi$
$\red : \pi \to \mathcal{P}(\pi)$

\begin{mathpar}
  \inferrule* [lab=Comm] { \textsf{match}( x_{src}, x_{trgt} ) } { x_{trgt}?(y)P \; | \; x_{src}!\langle {Q} \rangle \red P\{\quotep{Q}/y}\} }
  \and \\
  \inferrule* [lab=Par] {{P} \red {P}'} {{{P} | {Q}} \red {{P}' | {Q}}}
  \and
  \inferrule* [lab=Equiv]{{{P} \scong {P}'} \andalso {{P}' \red {Q}'} \andalso {{Q}' \scong {Q}}}{{P} \red {Q}}
\end{mathpar}

\begin{eqnarray*}
  match_{\equiv} (\quotep{P},\quotep{Q}) & := & P \equiv Q \\
  match_{\dagger}(\quotep{P},\quotep{Q}) & := & \forall R. P|Q \red^{*} R => R \red^{*} 0 \\
  match_{K}(\quotep{P},\quotep{Q}) & := & K \mbox{ for some context } K
\end{eqnarray*}

$u?(x)P | u!\langle Q \rangle \red P\{\quotep{Q}/x\}$

%We write $\wred$ for $\red^*$, and $P\red$ if $\exists Q $ such that $ P \red Q$.
We write $P\red$ if $\exists Q $ such that $ P \red Q$ and $P\not\red$, otherwise.

\section{Replication}

As mentioned before, it is known that replication (and hence
recursion) can be implemented in a higher-order process algebra
\cite{SangiorgiWalker}. As our first example of calculation with the
machinery thus far presented we give the construction explicitly in
the {\rhoc}.

\begin{eqnarray}
	D_{x} & := & \prefix{x}{y}{(\binpar{\outputp{x}{y}}{@{y}})} \nonumber\\
	\bangp_{x}{P} & := & \binpar{{x}!\langle{\binpar{D_{x}}{P}}\rangle}{D_{x}} \nonumber
\end{eqnarray}

\begin{eqnarray}
	\bangp_{x}{P} & & \nonumber\\
	=
	& {x}!\langle{(\prefix{x}{y}{(\outputp{x}{y} | @{y})) | P}}\rangle 
	      | \prefix{x}{y}{(\outputp{x}{y} | @{y})} & \nonumber\\
	\red
	& (\outputp{x}{y} | @{y})\substn{\quotep{(\prefix{x}{y}{(@{y} | \outputp{x}{y})) | P}}}{y} & \nonumber\\
	=
	& \outputp{x}{\quotep{(\prefix{x}{y}{(\outputp{x}{y} | @{y})) | P}}}
	  | {(\prefix{x}{y}{(\outputp{x}{y} | @{y})) | P}} & \nonumber\\
	\red
	& \ldots & \nonumber\\
	\red^*
	& P | P | \ldots & \nonumber
\end{eqnarray}

Of course, this encoding, as an implementation, runs away, unfolding
$\bangp{P}$ eagerly. A lazier and more implementable replication
operator, restricted to input-guarded processes, may be obtained as follows.

\begin{eqnarray}
\bangp{\prefix{u}{v}{P}} 
	:= 
	\binpar{\lift{x}{\prefix{u}{v}{(\binpar{D(x)}{P})}}}{D(x)} \nonumber
\end{eqnarray}

\begin{remark}
  Note that the lazier definition still does not deal with summation
  or mixed summation (i.e. sums over input and output). The reader is
  invited to construct definitions of replication that deal with these
  features. 

  Further, the definitions are parameterized in a name, $x$. Can you,
  gentle reader, make a definition that eliminates this parameter and
  guarantees no accidental interaction between the replication
  machinery and the process being replicated -- i.e. no accidental
  sharing of names used by the process to get its work done and the
  name(s) used by the replication to effect copying. This latter
  revision of the definition of replication is crucial to obtaining
  the expected identity $!!P \sim !P$.
\end{remark}

\begin{remark}\label{rem:paradoxical_combinator}
  The reader familiar with the lambda calculus will have noticed the
  similarity between $D$ and the paradoxical combinator.

  [Ed. note: the existence of this seems to suggest we have to be more
  restrictive on the set of processes and names we admit if we are to
  support no-cloning.]
\end{remark}

\subsubsection{Bisimulation}

The computational dynamics gives rise to another kind of equivalence,
the equivalence of computational behavior. As previously mentioned
this is typically captured \emph{via} some form of bisimulation.

% The notion we use in this paper is weak barbed bisimulation
% \cite{milner91polyadicpi}.

The notion we use in this paper is derived from weak barbed
bisimulation \cite{milner91polyadicpi}. 

\begin{definition}
An \emph{observation relation}, $\downarrow_{\mathcal N}$, over a set
of names, $\mathcal N$, is the smallest relation satisfying the rules
below.

\infrule[Out-barb]{y \in {\mathcal N}, \; x \nameeq y}
		  {\outputp{x}{v} \downarrow_{\mathcal N} x}
\infrule[Par-barb]{\mbox{$P\downarrow_{\mathcal N} x$ or $Q\downarrow_{\mathcal N} x$}}
		  {\binpar{P}{Q} \downarrow_{\mathcal N} x}

We write $P \Downarrow_{\mathcal N} x$ if there is $Q$ such that 
$P \wred Q$ and $Q \downarrow_{\mathcal N} x$.
\end{definition}

\begin{definition}
%\label{def.bbisim}
An  ${\mathcal N}$-\emph{barbed bisimulation} over a set of names, ${\mathcal N}$, is a symmetric binary relation 
${\mathcal S}_{\mathcal N}$ between agents such that $P\rel{S}_{\mathcal N}Q$ implies:
\begin{enumerate}
\item If $P \red P'$ then $Q \wred Q'$ and $P'\rel{S}_{\mathcal N} Q'$.
\item If $P\downarrow_{\mathcal N} x$, then $Q\Downarrow_{\mathcal N} x$.
\end{enumerate}
$P$ is ${\mathcal N}$-barbed bisimilar to $Q$, written
$P \wbbisim_{\mathcal N} Q$, if $P \rel{S}_{\mathcal N} Q$ for some ${\mathcal N}$-barbed bisimulation ${\mathcal S}_{\mathcal N}$.
\end{definition}

$\mathcal{R} \subseteq \pi \times \pi$

$P \mathcal{R} Q => \forall P'. P \red P' \Rightarrow \exists Q'. Q \red Q', P' \mathcal{R} Q'$

$P \vdash x \Rightarrow Q \vdash x$

\begin{mathpar}
  \inferrule*[lab=Out-barb]{x \nameeq y}{{y}!\langle{Q}\rangle \vdash x}
  \and
  \inferrule*[lab=Par-barb]{\mbox{$P\vdash x$ or $Q\vdash x$}}{\binpar{P}{Q} \vdash x}
\end{mathpar}

\subsubsection{Contexts}

One of the principle advantages of computational calculi like the
$\pi$-calculus is a well-defined notion of context,
contextual-equivalence and a correlation between
contextual-equivalence and notions of bisimulation. The notion of
context allows the decomposition of a process into (sub-)process and
its syntactic environment, its context. Thus, a context may be
thought of as a process with a ``hole'' (written $\Box$) in it. The
application of a context $M$ to a process $P$, written $M[P]$, is
tantamount to filling the hole in $M$ with $P$. In this paper we do
not need the full weight of this theory, but do make use of the notion
of context in the proof the main theorem. 

\begin{mathpar}
  \inferrule* [lab=summation] {} {{M_{M},M_{N}} \bc \Box \;|\; x.M_{A} \;|\; M_{M}+M_{N}}
  \and
  \inferrule* [lab=agent] {} {{M_{A}} \bc (\vec{x})M_{P} \;| \; \clift{P_0,\ldots,M_{P},\ldots,P_N}}
  \and \\
  \inferrule* [lab=process] {} {{M_{P}} \bc M_{N} \;| \;P|M_{P} }
\end{mathpar} 

\begin{mathpar}
  \inferrule* [lab=sychronization] {} {M_{N} \bc \Box \;|\; x?M_{F} \;|\; x!M_{C}}
  \and
  \inferrule* [lab=abstraction] {} {{M_{F}} \bc (x)M_{P} }
  \and
  \inferrule* [lab=concretion] {} {{M_{C}} \bc \langle M_{P} \rangle }
  \and \\
  \inferrule* [lab=process] {} {{M_{P}} \bc M_{N} \;| \;P|M_{P} }
\end{mathpar}

\begin{definition}[contextual application] Given a context $M$, and
  process $P$, we define the \emph{contextual application}, $M[P] :=
  M\{P/\Box\}$. That is, the contextual application of M to P is the
  substitution of $P$ for $\Box$ in $M$.
\end{definition}

$\meaningof{-} : L \to \mathcal{P}(\pi)$

\begin{mathpar}
  \inferrule* [lab=collection] {} {\meaningof{true} = \pi, \and \meaningof{~E} = \pi \setminus \meaningof{E}, \and \meaningof{E_{1} \& E_{2}} = \meaningof{E_{1}} \cap \meaningof{E_{2}}}
\end{mathpar}

\begin{mathpar}
  \inferrule* [lab=structure] {} {\meaningof{0} = \{ P \in \pi | P \equiv 0 \}, \and \\ \meaningof{E_1 | E_2} = \{ P \in \pi | P \equiv P_{1} | P_{2}, P_{1} \in \meaningof{E_{1}}, P_{2} \in \meaningof{E_2}\} }
\end{mathpar}

\begin{mathpar}
 \inferrule* [lab=behavior] {} {\meaningof{\langle a?b \rangle E} = \{ P \in \pi | P \equiv Q | u?(y)P', \\ \and \\\\ \and \\ \;\;\; u \in \meaningof{a}, \forall z.P'\{z/y\} \in \meaningof{E\{z/b\}}\}, \and \\ \meaningof{a!E} = \{ P \in \pi | P \equiv Q | x!\langle P' \rangle, x \in \meaningof{a} P' \in \meaningof{E}\} }
\end{mathpar}

\begin{mathpar}
 \inferrule* [lab=nominal] {} {\meaningof{\quotep{E}} = \{ \quotep{P} \in \quotep{\pi} | P \in \meaningof{E} \}, \and \meaningof{\quotep{P}} = \{ \quotep{Q} \in \quotep{\pi} | P \equiv Q \} \and \\ \meaningof{@\quotep{E}} = \{ P \in \pi | P \equiv @x, x \in \meaningof{E} \}}
\end{mathpar}

\begin{eqnarray*}
  \\
  \meaningof{-} : TS \to ST
\end{eqnarray*}

\begin{eqnarray*}
  \\
  L : TS \to ST
\end{eqnarray*}

\begin{eqnarray*}
  \\
  P \models E \iff P \in \meaningof{E}
\end{eqnarray*}

\begin{eqnarray*}
  P \approx_{L} Q \iff \forall E \in L. P \models E \iff Q \models E
\end{eqnarray*}

\begin{eqnarray*}
  P \approx_{K} Q
\end{eqnarray*}

\begin{eqnarray*}
  P \approx Q
\end{eqnarray*}

$\approx_{K} = \approx = \approx_{L}$

\subsubsection{Contextual duality}

Note that contexts extend the quotation operation to a family of
operations from processes to names. Given a context, $M$, we can
define a \emph{nominal context}, $\quotep{M}$ by $\quotep{M}[P] :=
\quotep{M[P]}$. To foreshadow what is to come we observe that these
operations enjoy a duality with processes very much like the duality
between vectors and maps from vectors to scalars.

Further, because the calculus is essentially higher-order, we have a
correspondence between contexts and processes. More specifically,
given a name $x$ and a context $M$ we can construct $M^{*}_{x}$ such
that 

\begin{mathpar}
  M^{*}_{x} | \lift{x}{P} \red M[P]
\end{mathpar}

namely,

\begin{mathpar}
  M^{*}_{x} := x?(u).M[\dropn{u}]
\end{mathpar}

The dependence of $M^{*}_{x}$ on a name makes it an abstraction, 

\begin{mathpar}
  M^{*} := (x)x?(u).M[\dropn{u}]
\end{mathpar}

\subsection{Additional notation}

It will sometimes be convenient to denote the process a name
quotes. We already have the notation $x = \quotep{P}$, but it will be
convenient to introduce an alternate notation, $\procn{x}$, when we
want to emphasize the connection to the use of the name. Note that, by
virtue of name equivalence, $\quotep{\procn{x}} \nameeq x$; so, the
notation is consistent with previous definitions.

Further, because names have structure it is possible to effect
substitutions on the basis of that structure. This means we need to
upgrade our notation for substitutions, which we accomplish by
adapting comprehension notation. Thus,

\begin{mathpar}
  P\{ y / x : x \in S \}
\end{mathpar}

is interpreted to mean the process derived from P by replacing (in a
capture-avoiding manner) each occurrence of $x$ in $S$ by $y$. For example,

\begin{mathpar}
  P\{ \quotep{\procn{x}|\procn{x}} / x : x \in \freenames{P} \}
\end{mathpar}

will replace each (occurrence) of a free name $x$ in $P$ by
$\quotep{\procn{x}|\procn{x}}$.

Also, we will avail ourselves of the notation $x^{L}$ and $x^{R}$ to
denote injections of a name into disjoint copies of the name
space. There are numerous ways to accomplish this. One example can be
found in \cite{MeredithR05}. This notation overloads to vectors of
names: $\vec{x}^{\pi} := (x_{i}^{\pi} \; : \; 0 \leq i < |\vec{x}| )$ where $\pi \in \{L,R\}$.

We also use $P^{\Box} := P|\Box$.

In \cite{MeredithR05} an interpretation of the new operator is
given. It turns out that there are several possible interpretations
all enjoying the requisite algebraic properties of the operator (see
\cite{milner91polyadicpi}). We will therefore make liberal use of
$(\nu\; \vec{x})P$.

% subsection the_syntax_and_semantics_of_the_notation_system (end)   

\input{qm2pi.qmops} 

\input{qm2pi.sterngerlach} 

\input{qm2pi.metric} 

% section concurrent_process_calculi (end)

%\input{qm2pi.proofsketch}

% section proof sketch (end)

%\input{qm2pi.slviaknots} 

% section spatial logic via knots (end)

\input{qm2pi.conclusion}

% section conclusion (end)

%\input{qm2pi.dtcodes} 

% section wiring algorithm (end)

\input{qm2pi.ack} 

% section acknowledgments (end)

\newpage


\bibliographystyle{plain}   
\bibliography{../../biblios/main.bib}

\input{qm2pi.rhodetails}

\end{document}

 

%\documentclass[12pt]{llncs}
%\documentclass{jktr}

\usepackage[pdftex]{hyperref}                   
\usepackage {listings}
\usepackage {mathpartir}
\usepackage{bcprules}
%\usepackage{listings}
                       
\usepackage{graphicx} 
%\usepackage[margins=2.5cm,nohead,nofoot]{geometry}
%\usepackage{geometry}
\usepackage{amsfonts}
\usepackage{amstext}
\usepackage{latexsym}
\usepackage{amssymb}
\usepackage{color}


%\include{myPreamble}
\include{qm2pi.local} 

%\ifpdf
%\usepackage[pdftex]{graphicx}
%\else
%\usepackage{graphicx}
%\fi

 % \ifpdf
%  \usepackage{pdfsync}
%  \if


%\title{Brief Article}
%\author{David F. Snyder}
%\author{L.G. Meredith}

%\address{Dept. of Math., Texas State University--San Marcos, San Marcos, TX 78666}
       
\pagestyle{empty}


\begin{document}

\lstset{language=[Objective]Caml,frame=shadowbox}

\input{qm2pi.front}

% section front matter (end)

\input{qm2pi.intro} 
 
% section introduction (end)

% \input{qm2pi.knotations} 

% section notation (end)

\input{qm2pi.process.calculi} 

% section concurrent_process_calculi_and_spatial_logics_ (end)
    
%\input{qm2pi.knots2pi} 

%\input{qm2pi.trefoil} 

%\input{qm2pi.mainthm} 

% subsection basic_interpretation (end)

%\input{qm2pi.rho.presentation} 
\subsection{The syntax and semantics of the notation system}\label{sub:the_syntax_and_semantics_of_the_notation_system} % (fold)

We now summarize a technical presentation of the calculus that
embodies our theory of dynamics. The typical presentation of such a
calculus follows the style of giving generators and relations on
them. The grammar, below, describing term constructors, freely
generates the set of processes, $\Proc$. This set is then quotiented
by a relation known as structural congruence and it is over this set
that the notion of dynamics is expressed. This presentation is
essentially that of \cite{MeredithR05} with the addition of
polyadicity and summation. For readability we have relegated some of
the technical subtleties to an appendix.

\subsubsection{Process grammar}\label{subsub:process_grammar}

\begin{mathpar}
  \inferrule* [lab=synchronization] {} {{M} \bc \pzero \;|\; x?F \;|\; x!C }
  \and
  \inferrule* [lab=abstraction] {} {{F} \bc (x)P}
  \and
  \inferrule* [lab=concretion] {} {{C} \bc \langle Q \rangle}
  \and
  \inferrule* [lab=process] {} {{P,Q} \bc M \;| \;P|Q \;|\; @{x}}
  \and
  \inferrule* [lab=name] {} {{x} \bc \quotep{P}}
\end{mathpar} 

Note that $\vec{x}$ (resp. $\vec{P}$) denotes a vector of names
(resp. processes) of length $|\vec{x}|$ (resp. $|\vec{P}|$). We adopt
the following useful abbreviations.

\begin{mathpar}
   x?(\vec{y}).P := x.(\vec{y})P \and  x\clift{\vec{P}} := x.\clift{\vec{P}}
   \and x!(y) := \lift{x}{\dropn{y}}
   \and \Pi_{i=0}^{n-1}P_i := P_0 | \ldots | P_{n-1}
\end{mathpar}

\subsubsection{Structural congruence}

\paragraph{Free and bound names and alpha-equivalence.} At the
core of structural equivalence is alpha-equivalence which identifies
process that are the same up to a change of variable. Formally, we
recognize the distinction between free and bound names. The free names
of a process, $\freenames{P}$, may be calculated recursively as
follows:

\begin{mathpar}
\freenames{\pzero} := \emptyset
  \and \\
  \freenames{x?(y).P} := \{ x \} \cup (\freenames{P} \setminus \{ y \})
  \and 
  \freenames{x!\langle P \rangle} := \{ x \} \cup \{ P \} 
  \and \\
  \freenames{P|Q} := \freenames{P} \cup \freenames{Q}
  \and \\
  \freenames{@{x}} := \{ x \}
\end{mathpar}

$\pi$
$\quotep{\pi}$

$\freenames{-} : \pi \to \mathcal{P}(\quotep{\pi})$

\begin{eqnarray*}
  \freenames{\pzero} & := & \emptyset \\
  \freenames{x?(y).P} & := & \{ x \} \cup (\freenames{P} \setminus \{ y \}) \\
  \freenames{x!\langle P \rangle} & := & \{ x \} \cup \{ P \} \\
  \freenames{P|Q} & := & \freenames{P} \cup \freenames{Q} \\
  \freenames{\dropn{x}} & := & \{ x \}
\end{eqnarray*}

The bound names of a process, $\boundnames{P}$, are those names occurring in $P$
that are not free. For example, in $x?(y).0$, the name $x$ is free, while $y$ is bound.

\begin{mathpar}
  \inferrule* [lab=monoidal-laws] {} { P|Q \equiv Q|P \and P|0 \equiv P \and P|(Q|R) \equiv (P|Q)|R }
\end{mathpar}

\begin{mathpar}
  \inferrule* [lab=alpha-equivalence] {} { (x)P \equiv (y)P\{y/x\} \and y \not\in \freenames{P} }
\end{mathpar}

\begin{definition}
Then two processes, $P,Q$, are alpha-equivalent if $P = Q\{\vec{y}/\vec{x}\}$ for
some $\vec{x} \in \boundnames{Q},\vec{y} \in \boundnames{P}$, where $Q\{\vec{y}/\vec{x}\}$
denotes the capture-avoiding substitution of $\vec{y}$ for $\vec{x}$ in $Q$.
\end{definition}

\begin{definition}
  The {\em structural congruence} \cite{SangiorgiWalker} , $\equiv$,
  between processes is the least congruence containing
  alpha-equivalence, satisfying the abelian monoid laws
  (associativity, commutativity and $\pzero$ as identity) for parallel
  composition $|$ and for summation $+$.
\end{definition}

\subsection{Name equivalence}

We take name equivalence, written $\nameeq$, to be the smallest
equivalence relation generated by the following rules.

\begin{mathpar}
\inferrule*[lab=Quote-drop]
{ }
{ \quotep{@{x}} \nameeq x }

\inferrule*[lab=Struct-equiv]
{ P \scong Q }
{ \quotep{P} \nameeq \quotep{Q} }
\end{mathpar}

The astute reader will have noticed that the mutual recursion of names
and processes imposes a mutual recursion on alpha-equivalence and
structural equivalence via name-equivalence. Fortunately, all of this
works out pleasantly and we may calculate in the natural way, free of
concern. The reader interested in the details is referred to the
appendix \ref{appendix:rho_details}.

\subsection{Substitution}

We use $\Proc$ for the set of processes, $\QProc$ for the set of
names, and $\id{\{}\vec{y} / \vec{x} \id{\}}$ to denote partial maps,
$s : \QProc \rightarrow \QProc$. A map, $s$ lifts, uniquely, to a map
on process terms, $\widehat{s} : \Proc \rightarrow \Proc$ by the
following equations.

\begin{mathpar}
  (0) \psubstp{Q}{P} := 0 \\
  (R \juxtap S) \psubstp{Q}{P}
  :=    
  (R)\psubstp{Q}{P} \juxtap (S) \psubstp{Q}{P} \\
  (x?(y).R) \psubstp{Q}{P}    
  :=    
  (x)\substp{Q}{P} (z)\concat( (R \psubstn{z}{y}) \psubstp{Q}{P} ) \\
  (\lift{x}{R}) \psubstp{Q}{P}  
  :=
  \lift{(x)\substp{Q}{P}}{ R \psubstp{Q}{P} } \\
%   (\dropn{x})  \psubstp{Q}{P}       
%   := 
%   \left\{ 
%     \begin{array}{ccc} 
%       \dropn{\quotep{Q}} & & x \nameeq \quotep{P} \\
%       \dropn{x} & & otherwise \\
%     \end{array}
%   \right. 
  (\dropn{x})  \psubstp{Q}{P}       
  := 
  \left\{ 
    \begin{array}{ccc} 
      Q & & x \nameeq \quotep{P} \\
      \dropn{x} & & otherwise \\
    \end{array}
  \right.
\end{mathpar}
 

where

\begin{eqnarray}
  (x)\id{\{} \lpquote Q \rpquote / \lpquote P \rpquote \id{\}}            = 
  \left\{ 
    \begin{array}{ccc}
      \lpquote Q \rpquote & & x \nameeq \lpquote P \rpquote \\
      x & & otherwise \\
    \end{array}
  \right. \nonumber
\end{eqnarray}

and $z$ is chosen distinct from $\quotep{P}$, $\quotep{Q}$, the free
names in $Q$, and all the names in $R$. Our $\alpha$-equivalence will
be built in the standard way from this substitution.

\begin{remark}\label{rem:no_self_referential_names}
  One consequence of these definitions is that $\forall P. \quotep{P}
  \not\in \freenames{P}$.
\end{remark}

\subsection{ Dynamic quote: an example }

Anticipating something of what's to come, consider applying the
substitution, $\widehat{\id{\{}u / z \id{\}}}$, to the following pair
of processes, $\lift{w}{y!(z)}$ and $w[ \lpquote y!(z) \rpquote ]$.

\begin{eqnarray}
	\lift{w}{y!(z)}\widehat{\id{\{}u / z \id{\}}}
		& = &
		\lift{w}{y!(u)} \nonumber\\
	w[ \lpquote y!(z) \rpquote ] \widehat{ \id{\{}u / z \id{\}} }
		& = &
		w[ \lpquote y!(z) \rpquote ] \nonumber
\end{eqnarray}

Because the body of the process between quotes is impervious to
substitution, we get radically different answers. In fact, by
examining the first process in an input context,
e.g. $x?(z).\lift{w}{y!(z)}$, we see that the process under the lift
operator may be shaped by prefixed inputs binding a name inside it. In
this sense, the lift operator will be seen as a way to dynamically
construct processes before reifying them as names.

Finally equipped with these standard features we can present the
dynamics of the calculus.

\subsubsection{Operational semantics} 

Finally, we introduce the computational dynamics. What marks these
algebras as distinct from other more traditionally studied algebraic
structures, e.g. vector spaces or polynomial rings, is the manner in
which dynamics is captured. In traditional structures, dynamics is typically
expressed through morphisms between such structures, as in linear maps
between vector spaces or morphisms between rings. In algebras
associated with the semantics of computation, the dynamics is
expressed as part of the algebraic structure itself, through a
reduction reduction relation typically denoted by $\red$. Below, we
give a recursive presentation of this relation for the calculus used
in the encoding.

$\red \subseteq \pi \times \pi$
$\red : \pi \to \mathcal{P}(\pi)$

\begin{mathpar}
  \inferrule* [lab=Comm] { \textsf{match}( x_{src}, x_{trgt} ) } { x_{trgt}?(y)P \; | \; x_{src}!\langle {Q} \rangle \red P\{\quotep{Q}/y}\} }
  \and \\
  \inferrule* [lab=Par] {{P} \red {P}'} {{{P} | {Q}} \red {{P}' | {Q}}}
  \and
  \inferrule* [lab=Equiv]{{{P} \scong {P}'} \andalso {{P}' \red {Q}'} \andalso {{Q}' \scong {Q}}}{{P} \red {Q}}
\end{mathpar}

\begin{eqnarray*}
  match_{\equiv} (\quotep{P},\quotep{Q}) & := & P \equiv Q \\
  match_{\dagger}(\quotep{P},\quotep{Q}) & := & \forall R. P|Q \red^{*} R => R \red^{*} 0 \\
  match_{K}(\quotep{P},\quotep{Q}) & := & K \mbox{ for some context } K
\end{eqnarray*}

$u?(x)P | u!\langle Q \rangle \red P\{\quotep{Q}/x\}$

%We write $\wred$ for $\red^*$, and $P\red$ if $\exists Q $ such that $ P \red Q$.
We write $P\red$ if $\exists Q $ such that $ P \red Q$ and $P\not\red$, otherwise.

\section{Replication}

As mentioned before, it is known that replication (and hence
recursion) can be implemented in a higher-order process algebra
\cite{SangiorgiWalker}. As our first example of calculation with the
machinery thus far presented we give the construction explicitly in
the {\rhoc}.

\begin{eqnarray}
	D_{x} & := & \prefix{x}{y}{(\binpar{\outputp{x}{y}}{@{y}})} \nonumber\\
	\bangp_{x}{P} & := & \binpar{{x}!\langle{\binpar{D_{x}}{P}}\rangle}{D_{x}} \nonumber
\end{eqnarray}

\begin{eqnarray}
	\bangp_{x}{P} & & \nonumber\\
	=
	& {x}!\langle{(\prefix{x}{y}{(\outputp{x}{y} | @{y})) | P}}\rangle 
	      | \prefix{x}{y}{(\outputp{x}{y} | @{y})} & \nonumber\\
	\red
	& (\outputp{x}{y} | @{y})\substn{\quotep{(\prefix{x}{y}{(@{y} | \outputp{x}{y})) | P}}}{y} & \nonumber\\
	=
	& \outputp{x}{\quotep{(\prefix{x}{y}{(\outputp{x}{y} | @{y})) | P}}}
	  | {(\prefix{x}{y}{(\outputp{x}{y} | @{y})) | P}} & \nonumber\\
	\red
	& \ldots & \nonumber\\
	\red^*
	& P | P | \ldots & \nonumber
\end{eqnarray}

Of course, this encoding, as an implementation, runs away, unfolding
$\bangp{P}$ eagerly. A lazier and more implementable replication
operator, restricted to input-guarded processes, may be obtained as follows.

\begin{eqnarray}
\bangp{\prefix{u}{v}{P}} 
	:= 
	\binpar{\lift{x}{\prefix{u}{v}{(\binpar{D(x)}{P})}}}{D(x)} \nonumber
\end{eqnarray}

\begin{remark}
  Note that the lazier definition still does not deal with summation
  or mixed summation (i.e. sums over input and output). The reader is
  invited to construct definitions of replication that deal with these
  features. 

  Further, the definitions are parameterized in a name, $x$. Can you,
  gentle reader, make a definition that eliminates this parameter and
  guarantees no accidental interaction between the replication
  machinery and the process being replicated -- i.e. no accidental
  sharing of names used by the process to get its work done and the
  name(s) used by the replication to effect copying. This latter
  revision of the definition of replication is crucial to obtaining
  the expected identity $!!P \sim !P$.
\end{remark}

\begin{remark}\label{rem:paradoxical_combinator}
  The reader familiar with the lambda calculus will have noticed the
  similarity between $D$ and the paradoxical combinator.

  [Ed. note: the existence of this seems to suggest we have to be more
  restrictive on the set of processes and names we admit if we are to
  support no-cloning.]
\end{remark}

\subsubsection{Bisimulation}

The computational dynamics gives rise to another kind of equivalence,
the equivalence of computational behavior. As previously mentioned
this is typically captured \emph{via} some form of bisimulation.

% The notion we use in this paper is weak barbed bisimulation
% \cite{milner91polyadicpi}.

The notion we use in this paper is derived from weak barbed
bisimulation \cite{milner91polyadicpi}. 

\begin{definition}
An \emph{observation relation}, $\downarrow_{\mathcal N}$, over a set
of names, $\mathcal N$, is the smallest relation satisfying the rules
below.

\infrule[Out-barb]{y \in {\mathcal N}, \; x \nameeq y}
		  {\outputp{x}{v} \downarrow_{\mathcal N} x}
\infrule[Par-barb]{\mbox{$P\downarrow_{\mathcal N} x$ or $Q\downarrow_{\mathcal N} x$}}
		  {\binpar{P}{Q} \downarrow_{\mathcal N} x}

We write $P \Downarrow_{\mathcal N} x$ if there is $Q$ such that 
$P \wred Q$ and $Q \downarrow_{\mathcal N} x$.
\end{definition}

\begin{definition}
%\label{def.bbisim}
An  ${\mathcal N}$-\emph{barbed bisimulation} over a set of names, ${\mathcal N}$, is a symmetric binary relation 
${\mathcal S}_{\mathcal N}$ between agents such that $P\rel{S}_{\mathcal N}Q$ implies:
\begin{enumerate}
\item If $P \red P'$ then $Q \wred Q'$ and $P'\rel{S}_{\mathcal N} Q'$.
\item If $P\downarrow_{\mathcal N} x$, then $Q\Downarrow_{\mathcal N} x$.
\end{enumerate}
$P$ is ${\mathcal N}$-barbed bisimilar to $Q$, written
$P \wbbisim_{\mathcal N} Q$, if $P \rel{S}_{\mathcal N} Q$ for some ${\mathcal N}$-barbed bisimulation ${\mathcal S}_{\mathcal N}$.
\end{definition}

$\mathcal{R} \subseteq \pi \times \pi$

$P \mathcal{R} Q => \forall P'. P \red P' \Rightarrow \exists Q'. Q \red Q', P' \mathcal{R} Q'$

$P \vdash x \Rightarrow Q \vdash x$

\begin{mathpar}
  \inferrule*[lab=Out-barb]{x \nameeq y}{{y}!\langle{Q}\rangle \vdash x}
  \and
  \inferrule*[lab=Par-barb]{\mbox{$P\vdash x$ or $Q\vdash x$}}{\binpar{P}{Q} \vdash x}
\end{mathpar}

\subsubsection{Contexts}

One of the principle advantages of computational calculi like the
$\pi$-calculus is a well-defined notion of context,
contextual-equivalence and a correlation between
contextual-equivalence and notions of bisimulation. The notion of
context allows the decomposition of a process into (sub-)process and
its syntactic environment, its context. Thus, a context may be
thought of as a process with a ``hole'' (written $\Box$) in it. The
application of a context $M$ to a process $P$, written $M[P]$, is
tantamount to filling the hole in $M$ with $P$. In this paper we do
not need the full weight of this theory, but do make use of the notion
of context in the proof the main theorem. 

\begin{mathpar}
  \inferrule* [lab=summation] {} {{M_{M},M_{N}} \bc \Box \;|\; x.M_{A} \;|\; M_{M}+M_{N}}
  \and
  \inferrule* [lab=agent] {} {{M_{A}} \bc (\vec{x})M_{P} \;| \; \clift{P_0,\ldots,M_{P},\ldots,P_N}}
  \and \\
  \inferrule* [lab=process] {} {{M_{P}} \bc M_{N} \;| \;P|M_{P} }
\end{mathpar} 

\begin{mathpar}
  \inferrule* [lab=sychronization] {} {M_{N} \bc \Box \;|\; x?M_{F} \;|\; x!M_{C}}
  \and
  \inferrule* [lab=abstraction] {} {{M_{F}} \bc (x)M_{P} }
  \and
  \inferrule* [lab=concretion] {} {{M_{C}} \bc \langle M_{P} \rangle }
  \and \\
  \inferrule* [lab=process] {} {{M_{P}} \bc M_{N} \;| \;P|M_{P} }
\end{mathpar}

\begin{definition}[contextual application] Given a context $M$, and
  process $P$, we define the \emph{contextual application}, $M[P] :=
  M\{P/\Box\}$. That is, the contextual application of M to P is the
  substitution of $P$ for $\Box$ in $M$.
\end{definition}

$\meaningof{-} : L \to \mathcal{P}(\pi)$

\begin{mathpar}
  \inferrule* [lab=collection] {} {\meaningof{true} = \pi, \and \meaningof{~E} = \pi \setminus \meaningof{E}, \and \meaningof{E_{1} \& E_{2}} = \meaningof{E_{1}} \cap \meaningof{E_{2}}}
\end{mathpar}

\begin{mathpar}
  \inferrule* [lab=structure] {} {\meaningof{0} = \{ P \in \pi | P \equiv 0 \}, \and \\ \meaningof{E_1 | E_2} = \{ P \in \pi | P \equiv P_{1} | P_{2}, P_{1} \in \meaningof{E_{1}}, P_{2} \in \meaningof{E_2}\} }
\end{mathpar}

\begin{mathpar}
 \inferrule* [lab=behavior] {} {\meaningof{\langle a?b \rangle E} = \{ P \in \pi | P \equiv Q | u?(y)P', \\ \and \\\\ \and \\ \;\;\; u \in \meaningof{a}, \forall z.P'\{z/y\} \in \meaningof{E\{z/b\}}\}, \and \\ \meaningof{a!E} = \{ P \in \pi | P \equiv Q | x!\langle P' \rangle, x \in \meaningof{a} P' \in \meaningof{E}\} }
\end{mathpar}

\begin{mathpar}
 \inferrule* [lab=nominal] {} {\meaningof{\quotep{E}} = \{ \quotep{P} \in \quotep{\pi} | P \in \meaningof{E} \}, \and \meaningof{\quotep{P}} = \{ \quotep{Q} \in \quotep{\pi} | P \equiv Q \} \and \\ \meaningof{@\quotep{E}} = \{ P \in \pi | P \equiv @x, x \in \meaningof{E} \}}
\end{mathpar}

\begin{eqnarray*}
  \\
  \meaningof{-} : TS \to ST
\end{eqnarray*}

\begin{eqnarray*}
  \\
  L : TS \to ST
\end{eqnarray*}

\begin{eqnarray*}
  \\
  P \models E \iff P \in \meaningof{E}
\end{eqnarray*}

\begin{eqnarray*}
  P \approx_{L} Q \iff \forall E \in L. P \models E \iff Q \models E
\end{eqnarray*}

\begin{eqnarray*}
  P \approx_{K} Q
\end{eqnarray*}

\begin{eqnarray*}
  P \approx Q
\end{eqnarray*}

$\approx_{K} = \approx = \approx_{L}$

\subsubsection{Contextual duality}

Note that contexts extend the quotation operation to a family of
operations from processes to names. Given a context, $M$, we can
define a \emph{nominal context}, $\quotep{M}$ by $\quotep{M}[P] :=
\quotep{M[P]}$. To foreshadow what is to come we observe that these
operations enjoy a duality with processes very much like the duality
between vectors and maps from vectors to scalars.

Further, because the calculus is essentially higher-order, we have a
correspondence between contexts and processes. More specifically,
given a name $x$ and a context $M$ we can construct $M^{*}_{x}$ such
that 

\begin{mathpar}
  M^{*}_{x} | \lift{x}{P} \red M[P]
\end{mathpar}

namely,

\begin{mathpar}
  M^{*}_{x} := x?(u).M[\dropn{u}]
\end{mathpar}

The dependence of $M^{*}_{x}$ on a name makes it an abstraction, 

\begin{mathpar}
  M^{*} := (x)x?(u).M[\dropn{u}]
\end{mathpar}

\subsection{Additional notation}

It will sometimes be convenient to denote the process a name
quotes. We already have the notation $x = \quotep{P}$, but it will be
convenient to introduce an alternate notation, $\procn{x}$, when we
want to emphasize the connection to the use of the name. Note that, by
virtue of name equivalence, $\quotep{\procn{x}} \nameeq x$; so, the
notation is consistent with previous definitions.

Further, because names have structure it is possible to effect
substitutions on the basis of that structure. This means we need to
upgrade our notation for substitutions, which we accomplish by
adapting comprehension notation. Thus,

\begin{mathpar}
  P\{ y / x : x \in S \}
\end{mathpar}

is interpreted to mean the process derived from P by replacing (in a
capture-avoiding manner) each occurrence of $x$ in $S$ by $y$. For example,

\begin{mathpar}
  P\{ \quotep{\procn{x}|\procn{x}} / x : x \in \freenames{P} \}
\end{mathpar}

will replace each (occurrence) of a free name $x$ in $P$ by
$\quotep{\procn{x}|\procn{x}}$.

Also, we will avail ourselves of the notation $x^{L}$ and $x^{R}$ to
denote injections of a name into disjoint copies of the name
space. There are numerous ways to accomplish this. One example can be
found in \cite{MeredithR05}. This notation overloads to vectors of
names: $\vec{x}^{\pi} := (x_{i}^{\pi} \; : \; 0 \leq i < |\vec{x}| )$ where $\pi \in \{L,R\}$.

We also use $P^{\Box} := P|\Box$.

In \cite{MeredithR05} an interpretation of the new operator is
given. It turns out that there are several possible interpretations
all enjoying the requisite algebraic properties of the operator (see
\cite{milner91polyadicpi}). We will therefore make liberal use of
$(\nu\; \vec{x})P$.

% subsection the_syntax_and_semantics_of_the_notation_system (end)   

\input{qm2pi.qmops} 

\input{qm2pi.sterngerlach} 

\input{qm2pi.metric} 

% section concurrent_process_calculi (end)

%\input{qm2pi.proofsketch}

% section proof sketch (end)

%\input{qm2pi.slviaknots} 

% section spatial logic via knots (end)

\input{qm2pi.conclusion}

% section conclusion (end)

%\input{qm2pi.dtcodes} 

% section wiring algorithm (end)

\input{qm2pi.ack} 

% section acknowledgments (end)

\newpage


\bibliographystyle{plain}   
\bibliography{../../biblios/main.bib}

\input{qm2pi.rhodetails}

\end{document}

 

% subsection basic_interpretation (end)

%\input{qm2pi.rho.presentation} 
\subsection{The syntax and semantics of the notation system}\label{sub:the_syntax_and_semantics_of_the_notation_system} % (fold)

We now summarize a technical presentation of the calculus that
embodies our theory of dynamics. The typical presentation of such a
calculus follows the style of giving generators and relations on
them. The grammar, below, describing term constructors, freely
generates the set of processes, $\Proc$. This set is then quotiented
by a relation known as structural congruence and it is over this set
that the notion of dynamics is expressed. This presentation is
essentially that of \cite{MeredithR05} with the addition of
polyadicity and summation. For readability we have relegated some of
the technical subtleties to an appendix.

\subsubsection{Process grammar}\label{subsub:process_grammar}

\begin{mathpar}
  \inferrule* [lab=synchronization] {} {{M} \bc \pzero \;|\; x?F \;|\; x!C }
  \and
  \inferrule* [lab=abstraction] {} {{F} \bc (x)P}
  \and
  \inferrule* [lab=concretion] {} {{C} \bc \langle Q \rangle}
  \and
  \inferrule* [lab=process] {} {{P,Q} \bc M \;| \;P|Q \;|\; @{x}}
  \and
  \inferrule* [lab=name] {} {{x} \bc \quotep{P}}
\end{mathpar} 

Note that $\vec{x}$ (resp. $\vec{P}$) denotes a vector of names
(resp. processes) of length $|\vec{x}|$ (resp. $|\vec{P}|$). We adopt
the following useful abbreviations.

\begin{mathpar}
   x?(\vec{y}).P := x.(\vec{y})P \and  x\clift{\vec{P}} := x.\clift{\vec{P}}
   \and x!(y) := \lift{x}{\dropn{y}}
   \and \Pi_{i=0}^{n-1}P_i := P_0 | \ldots | P_{n-1}
\end{mathpar}

\subsubsection{Structural congruence}

\paragraph{Free and bound names and alpha-equivalence.} At the
core of structural equivalence is alpha-equivalence which identifies
process that are the same up to a change of variable. Formally, we
recognize the distinction between free and bound names. The free names
of a process, $\freenames{P}$, may be calculated recursively as
follows:

\begin{mathpar}
\freenames{\pzero} := \emptyset
  \and \\
  \freenames{x?(y).P} := \{ x \} \cup (\freenames{P} \setminus \{ y \})
  \and 
  \freenames{x!\langle P \rangle} := \{ x \} \cup \{ P \} 
  \and \\
  \freenames{P|Q} := \freenames{P} \cup \freenames{Q}
  \and \\
  \freenames{@{x}} := \{ x \}
\end{mathpar}

$\pi$
$\quotep{\pi}$

$\freenames{-} : \pi \to \mathcal{P}(\quotep{\pi})$

\begin{eqnarray*}
  \freenames{\pzero} & := & \emptyset \\
  \freenames{x?(y).P} & := & \{ x \} \cup (\freenames{P} \setminus \{ y \}) \\
  \freenames{x!\langle P \rangle} & := & \{ x \} \cup \{ P \} \\
  \freenames{P|Q} & := & \freenames{P} \cup \freenames{Q} \\
  \freenames{\dropn{x}} & := & \{ x \}
\end{eqnarray*}

The bound names of a process, $\boundnames{P}$, are those names occurring in $P$
that are not free. For example, in $x?(y).0$, the name $x$ is free, while $y$ is bound.

\begin{mathpar}
  \inferrule* [lab=monoidal-laws] {} { P|Q \equiv Q|P \and P|0 \equiv P \and P|(Q|R) \equiv (P|Q)|R }
\end{mathpar}

\begin{mathpar}
  \inferrule* [lab=alpha-equivalence] {} { (x)P \equiv (y)P\{y/x\} \and y \not\in \freenames{P} }
\end{mathpar}

\begin{definition}
Then two processes, $P,Q$, are alpha-equivalent if $P = Q\{\vec{y}/\vec{x}\}$ for
some $\vec{x} \in \boundnames{Q},\vec{y} \in \boundnames{P}$, where $Q\{\vec{y}/\vec{x}\}$
denotes the capture-avoiding substitution of $\vec{y}$ for $\vec{x}$ in $Q$.
\end{definition}

\begin{definition}
  The {\em structural congruence} \cite{SangiorgiWalker} , $\equiv$,
  between processes is the least congruence containing
  alpha-equivalence, satisfying the abelian monoid laws
  (associativity, commutativity and $\pzero$ as identity) for parallel
  composition $|$ and for summation $+$.
\end{definition}

\subsection{Name equivalence}

We take name equivalence, written $\nameeq$, to be the smallest
equivalence relation generated by the following rules.

\begin{mathpar}
\inferrule*[lab=Quote-drop]
{ }
{ \quotep{@{x}} \nameeq x }

\inferrule*[lab=Struct-equiv]
{ P \scong Q }
{ \quotep{P} \nameeq \quotep{Q} }
\end{mathpar}

The astute reader will have noticed that the mutual recursion of names
and processes imposes a mutual recursion on alpha-equivalence and
structural equivalence via name-equivalence. Fortunately, all of this
works out pleasantly and we may calculate in the natural way, free of
concern. The reader interested in the details is referred to the
appendix \ref{appendix:rho_details}.

\subsection{Substitution}

We use $\Proc$ for the set of processes, $\QProc$ for the set of
names, and $\id{\{}\vec{y} / \vec{x} \id{\}}$ to denote partial maps,
$s : \QProc \rightarrow \QProc$. A map, $s$ lifts, uniquely, to a map
on process terms, $\widehat{s} : \Proc \rightarrow \Proc$ by the
following equations.

\begin{mathpar}
  (0) \psubstp{Q}{P} := 0 \\
  (R \juxtap S) \psubstp{Q}{P}
  :=    
  (R)\psubstp{Q}{P} \juxtap (S) \psubstp{Q}{P} \\
  (x?(y).R) \psubstp{Q}{P}    
  :=    
  (x)\substp{Q}{P} (z)\concat( (R \psubstn{z}{y}) \psubstp{Q}{P} ) \\
  (\lift{x}{R}) \psubstp{Q}{P}  
  :=
  \lift{(x)\substp{Q}{P}}{ R \psubstp{Q}{P} } \\
%   (\dropn{x})  \psubstp{Q}{P}       
%   := 
%   \left\{ 
%     \begin{array}{ccc} 
%       \dropn{\quotep{Q}} & & x \nameeq \quotep{P} \\
%       \dropn{x} & & otherwise \\
%     \end{array}
%   \right. 
  (\dropn{x})  \psubstp{Q}{P}       
  := 
  \left\{ 
    \begin{array}{ccc} 
      Q & & x \nameeq \quotep{P} \\
      \dropn{x} & & otherwise \\
    \end{array}
  \right.
\end{mathpar}
 

where

\begin{eqnarray}
  (x)\id{\{} \lpquote Q \rpquote / \lpquote P \rpquote \id{\}}            = 
  \left\{ 
    \begin{array}{ccc}
      \lpquote Q \rpquote & & x \nameeq \lpquote P \rpquote \\
      x & & otherwise \\
    \end{array}
  \right. \nonumber
\end{eqnarray}

and $z$ is chosen distinct from $\quotep{P}$, $\quotep{Q}$, the free
names in $Q$, and all the names in $R$. Our $\alpha$-equivalence will
be built in the standard way from this substitution.

\begin{remark}\label{rem:no_self_referential_names}
  One consequence of these definitions is that $\forall P. \quotep{P}
  \not\in \freenames{P}$.
\end{remark}

\subsection{ Dynamic quote: an example }

Anticipating something of what's to come, consider applying the
substitution, $\widehat{\id{\{}u / z \id{\}}}$, to the following pair
of processes, $\lift{w}{y!(z)}$ and $w[ \lpquote y!(z) \rpquote ]$.

\begin{eqnarray}
	\lift{w}{y!(z)}\widehat{\id{\{}u / z \id{\}}}
		& = &
		\lift{w}{y!(u)} \nonumber\\
	w[ \lpquote y!(z) \rpquote ] \widehat{ \id{\{}u / z \id{\}} }
		& = &
		w[ \lpquote y!(z) \rpquote ] \nonumber
\end{eqnarray}

Because the body of the process between quotes is impervious to
substitution, we get radically different answers. In fact, by
examining the first process in an input context,
e.g. $x?(z).\lift{w}{y!(z)}$, we see that the process under the lift
operator may be shaped by prefixed inputs binding a name inside it. In
this sense, the lift operator will be seen as a way to dynamically
construct processes before reifying them as names.

Finally equipped with these standard features we can present the
dynamics of the calculus.

\subsubsection{Operational semantics} 

Finally, we introduce the computational dynamics. What marks these
algebras as distinct from other more traditionally studied algebraic
structures, e.g. vector spaces or polynomial rings, is the manner in
which dynamics is captured. In traditional structures, dynamics is typically
expressed through morphisms between such structures, as in linear maps
between vector spaces or morphisms between rings. In algebras
associated with the semantics of computation, the dynamics is
expressed as part of the algebraic structure itself, through a
reduction reduction relation typically denoted by $\red$. Below, we
give a recursive presentation of this relation for the calculus used
in the encoding.

$\red \subseteq \pi \times \pi$
$\red : \pi \to \mathcal{P}(\pi)$

\begin{mathpar}
  \inferrule* [lab=Comm] { \textsf{match}( x_{src}, x_{trgt} ) } { x_{trgt}?(y)P \; | \; x_{src}!\langle {Q} \rangle \red P\{\quotep{Q}/y}\} }
  \and \\
  \inferrule* [lab=Par] {{P} \red {P}'} {{{P} | {Q}} \red {{P}' | {Q}}}
  \and
  \inferrule* [lab=Equiv]{{{P} \scong {P}'} \andalso {{P}' \red {Q}'} \andalso {{Q}' \scong {Q}}}{{P} \red {Q}}
\end{mathpar}

\begin{eqnarray*}
  match_{\equiv} (\quotep{P},\quotep{Q}) & := & P \equiv Q \\
  match_{\dagger}(\quotep{P},\quotep{Q}) & := & \forall R. P|Q \red^{*} R => R \red^{*} 0 \\
  match_{K}(\quotep{P},\quotep{Q}) & := & K \mbox{ for some context } K
\end{eqnarray*}

$u?(x)P | u!\langle Q \rangle \red P\{\quotep{Q}/x\}$

%We write $\wred$ for $\red^*$, and $P\red$ if $\exists Q $ such that $ P \red Q$.
We write $P\red$ if $\exists Q $ such that $ P \red Q$ and $P\not\red$, otherwise.

\section{Replication}

As mentioned before, it is known that replication (and hence
recursion) can be implemented in a higher-order process algebra
\cite{SangiorgiWalker}. As our first example of calculation with the
machinery thus far presented we give the construction explicitly in
the {\rhoc}.

\begin{eqnarray}
	D_{x} & := & \prefix{x}{y}{(\binpar{\outputp{x}{y}}{@{y}})} \nonumber\\
	\bangp_{x}{P} & := & \binpar{{x}!\langle{\binpar{D_{x}}{P}}\rangle}{D_{x}} \nonumber
\end{eqnarray}

\begin{eqnarray}
	\bangp_{x}{P} & & \nonumber\\
	=
	& {x}!\langle{(\prefix{x}{y}{(\outputp{x}{y} | @{y})) | P}}\rangle 
	      | \prefix{x}{y}{(\outputp{x}{y} | @{y})} & \nonumber\\
	\red
	& (\outputp{x}{y} | @{y})\substn{\quotep{(\prefix{x}{y}{(@{y} | \outputp{x}{y})) | P}}}{y} & \nonumber\\
	=
	& \outputp{x}{\quotep{(\prefix{x}{y}{(\outputp{x}{y} | @{y})) | P}}}
	  | {(\prefix{x}{y}{(\outputp{x}{y} | @{y})) | P}} & \nonumber\\
	\red
	& \ldots & \nonumber\\
	\red^*
	& P | P | \ldots & \nonumber
\end{eqnarray}

Of course, this encoding, as an implementation, runs away, unfolding
$\bangp{P}$ eagerly. A lazier and more implementable replication
operator, restricted to input-guarded processes, may be obtained as follows.

\begin{eqnarray}
\bangp{\prefix{u}{v}{P}} 
	:= 
	\binpar{\lift{x}{\prefix{u}{v}{(\binpar{D(x)}{P})}}}{D(x)} \nonumber
\end{eqnarray}

\begin{remark}
  Note that the lazier definition still does not deal with summation
  or mixed summation (i.e. sums over input and output). The reader is
  invited to construct definitions of replication that deal with these
  features. 

  Further, the definitions are parameterized in a name, $x$. Can you,
  gentle reader, make a definition that eliminates this parameter and
  guarantees no accidental interaction between the replication
  machinery and the process being replicated -- i.e. no accidental
  sharing of names used by the process to get its work done and the
  name(s) used by the replication to effect copying. This latter
  revision of the definition of replication is crucial to obtaining
  the expected identity $!!P \sim !P$.
\end{remark}

\begin{remark}\label{rem:paradoxical_combinator}
  The reader familiar with the lambda calculus will have noticed the
  similarity between $D$ and the paradoxical combinator.

  [Ed. note: the existence of this seems to suggest we have to be more
  restrictive on the set of processes and names we admit if we are to
  support no-cloning.]
\end{remark}

\subsubsection{Bisimulation}

The computational dynamics gives rise to another kind of equivalence,
the equivalence of computational behavior. As previously mentioned
this is typically captured \emph{via} some form of bisimulation.

% The notion we use in this paper is weak barbed bisimulation
% \cite{milner91polyadicpi}.

The notion we use in this paper is derived from weak barbed
bisimulation \cite{milner91polyadicpi}. 

\begin{definition}
An \emph{observation relation}, $\downarrow_{\mathcal N}$, over a set
of names, $\mathcal N$, is the smallest relation satisfying the rules
below.

\infrule[Out-barb]{y \in {\mathcal N}, \; x \nameeq y}
		  {\outputp{x}{v} \downarrow_{\mathcal N} x}
\infrule[Par-barb]{\mbox{$P\downarrow_{\mathcal N} x$ or $Q\downarrow_{\mathcal N} x$}}
		  {\binpar{P}{Q} \downarrow_{\mathcal N} x}

We write $P \Downarrow_{\mathcal N} x$ if there is $Q$ such that 
$P \wred Q$ and $Q \downarrow_{\mathcal N} x$.
\end{definition}

\begin{definition}
%\label{def.bbisim}
An  ${\mathcal N}$-\emph{barbed bisimulation} over a set of names, ${\mathcal N}$, is a symmetric binary relation 
${\mathcal S}_{\mathcal N}$ between agents such that $P\rel{S}_{\mathcal N}Q$ implies:
\begin{enumerate}
\item If $P \red P'$ then $Q \wred Q'$ and $P'\rel{S}_{\mathcal N} Q'$.
\item If $P\downarrow_{\mathcal N} x$, then $Q\Downarrow_{\mathcal N} x$.
\end{enumerate}
$P$ is ${\mathcal N}$-barbed bisimilar to $Q$, written
$P \wbbisim_{\mathcal N} Q$, if $P \rel{S}_{\mathcal N} Q$ for some ${\mathcal N}$-barbed bisimulation ${\mathcal S}_{\mathcal N}$.
\end{definition}

$\mathcal{R} \subseteq \pi \times \pi$

$P \mathcal{R} Q => \forall P'. P \red P' \Rightarrow \exists Q'. Q \red Q', P' \mathcal{R} Q'$

$P \vdash x \Rightarrow Q \vdash x$

\begin{mathpar}
  \inferrule*[lab=Out-barb]{x \nameeq y}{{y}!\langle{Q}\rangle \vdash x}
  \and
  \inferrule*[lab=Par-barb]{\mbox{$P\vdash x$ or $Q\vdash x$}}{\binpar{P}{Q} \vdash x}
\end{mathpar}

\subsubsection{Contexts}

One of the principle advantages of computational calculi like the
$\pi$-calculus is a well-defined notion of context,
contextual-equivalence and a correlation between
contextual-equivalence and notions of bisimulation. The notion of
context allows the decomposition of a process into (sub-)process and
its syntactic environment, its context. Thus, a context may be
thought of as a process with a ``hole'' (written $\Box$) in it. The
application of a context $M$ to a process $P$, written $M[P]$, is
tantamount to filling the hole in $M$ with $P$. In this paper we do
not need the full weight of this theory, but do make use of the notion
of context in the proof the main theorem. 

\begin{mathpar}
  \inferrule* [lab=summation] {} {{M_{M},M_{N}} \bc \Box \;|\; x.M_{A} \;|\; M_{M}+M_{N}}
  \and
  \inferrule* [lab=agent] {} {{M_{A}} \bc (\vec{x})M_{P} \;| \; \clift{P_0,\ldots,M_{P},\ldots,P_N}}
  \and \\
  \inferrule* [lab=process] {} {{M_{P}} \bc M_{N} \;| \;P|M_{P} }
\end{mathpar} 

\begin{mathpar}
  \inferrule* [lab=sychronization] {} {M_{N} \bc \Box \;|\; x?M_{F} \;|\; x!M_{C}}
  \and
  \inferrule* [lab=abstraction] {} {{M_{F}} \bc (x)M_{P} }
  \and
  \inferrule* [lab=concretion] {} {{M_{C}} \bc \langle M_{P} \rangle }
  \and \\
  \inferrule* [lab=process] {} {{M_{P}} \bc M_{N} \;| \;P|M_{P} }
\end{mathpar}

\begin{definition}[contextual application] Given a context $M$, and
  process $P$, we define the \emph{contextual application}, $M[P] :=
  M\{P/\Box\}$. That is, the contextual application of M to P is the
  substitution of $P$ for $\Box$ in $M$.
\end{definition}

$\meaningof{-} : L \to \mathcal{P}(\pi)$

\begin{mathpar}
  \inferrule* [lab=collection] {} {\meaningof{true} = \pi, \and \meaningof{~E} = \pi \setminus \meaningof{E}, \and \meaningof{E_{1} \& E_{2}} = \meaningof{E_{1}} \cap \meaningof{E_{2}}}
\end{mathpar}

\begin{mathpar}
  \inferrule* [lab=structure] {} {\meaningof{0} = \{ P \in \pi | P \equiv 0 \}, \and \\ \meaningof{E_1 | E_2} = \{ P \in \pi | P \equiv P_{1} | P_{2}, P_{1} \in \meaningof{E_{1}}, P_{2} \in \meaningof{E_2}\} }
\end{mathpar}

\begin{mathpar}
 \inferrule* [lab=behavior] {} {\meaningof{\langle a?b \rangle E} = \{ P \in \pi | P \equiv Q | u?(y)P', \\ \and \\\\ \and \\ \;\;\; u \in \meaningof{a}, \forall z.P'\{z/y\} \in \meaningof{E\{z/b\}}\}, \and \\ \meaningof{a!E} = \{ P \in \pi | P \equiv Q | x!\langle P' \rangle, x \in \meaningof{a} P' \in \meaningof{E}\} }
\end{mathpar}

\begin{mathpar}
 \inferrule* [lab=nominal] {} {\meaningof{\quotep{E}} = \{ \quotep{P} \in \quotep{\pi} | P \in \meaningof{E} \}, \and \meaningof{\quotep{P}} = \{ \quotep{Q} \in \quotep{\pi} | P \equiv Q \} \and \\ \meaningof{@\quotep{E}} = \{ P \in \pi | P \equiv @x, x \in \meaningof{E} \}}
\end{mathpar}

\begin{eqnarray*}
  \\
  \meaningof{-} : TS \to ST
\end{eqnarray*}

\begin{eqnarray*}
  \\
  L : TS \to ST
\end{eqnarray*}

\begin{eqnarray*}
  \\
  P \models E \iff P \in \meaningof{E}
\end{eqnarray*}

\begin{eqnarray*}
  P \approx_{L} Q \iff \forall E \in L. P \models E \iff Q \models E
\end{eqnarray*}

\begin{eqnarray*}
  P \approx_{K} Q
\end{eqnarray*}

\begin{eqnarray*}
  P \approx Q
\end{eqnarray*}

$\approx_{K} = \approx = \approx_{L}$

\subsubsection{Contextual duality}

Note that contexts extend the quotation operation to a family of
operations from processes to names. Given a context, $M$, we can
define a \emph{nominal context}, $\quotep{M}$ by $\quotep{M}[P] :=
\quotep{M[P]}$. To foreshadow what is to come we observe that these
operations enjoy a duality with processes very much like the duality
between vectors and maps from vectors to scalars.

Further, because the calculus is essentially higher-order, we have a
correspondence between contexts and processes. More specifically,
given a name $x$ and a context $M$ we can construct $M^{*}_{x}$ such
that 

\begin{mathpar}
  M^{*}_{x} | \lift{x}{P} \red M[P]
\end{mathpar}

namely,

\begin{mathpar}
  M^{*}_{x} := x?(u).M[\dropn{u}]
\end{mathpar}

The dependence of $M^{*}_{x}$ on a name makes it an abstraction, 

\begin{mathpar}
  M^{*} := (x)x?(u).M[\dropn{u}]
\end{mathpar}

\subsection{Additional notation}

It will sometimes be convenient to denote the process a name
quotes. We already have the notation $x = \quotep{P}$, but it will be
convenient to introduce an alternate notation, $\procn{x}$, when we
want to emphasize the connection to the use of the name. Note that, by
virtue of name equivalence, $\quotep{\procn{x}} \nameeq x$; so, the
notation is consistent with previous definitions.

Further, because names have structure it is possible to effect
substitutions on the basis of that structure. This means we need to
upgrade our notation for substitutions, which we accomplish by
adapting comprehension notation. Thus,

\begin{mathpar}
  P\{ y / x : x \in S \}
\end{mathpar}

is interpreted to mean the process derived from P by replacing (in a
capture-avoiding manner) each occurrence of $x$ in $S$ by $y$. For example,

\begin{mathpar}
  P\{ \quotep{\procn{x}|\procn{x}} / x : x \in \freenames{P} \}
\end{mathpar}

will replace each (occurrence) of a free name $x$ in $P$ by
$\quotep{\procn{x}|\procn{x}}$.

Also, we will avail ourselves of the notation $x^{L}$ and $x^{R}$ to
denote injections of a name into disjoint copies of the name
space. There are numerous ways to accomplish this. One example can be
found in \cite{MeredithR05}. This notation overloads to vectors of
names: $\vec{x}^{\pi} := (x_{i}^{\pi} \; : \; 0 \leq i < |\vec{x}| )$ where $\pi \in \{L,R\}$.

We also use $P^{\Box} := P|\Box$.

In \cite{MeredithR05} an interpretation of the new operator is
given. It turns out that there are several possible interpretations
all enjoying the requisite algebraic properties of the operator (see
\cite{milner91polyadicpi}). We will therefore make liberal use of
$(\nu\; \vec{x})P$.

% subsection the_syntax_and_semantics_of_the_notation_system (end)   

\section{Interpretation of QM}
\subsection{Supporting definitions}
\subsubsection{Multiplication}
\begin{mathpar}
  \quotep{Q} \cdot \quotep{R} := \quotep{Q|R}
  \and \\
  \quotep{Q} \cdot P := P\{ \quotep{Q|R} / \quotep{R} : \quotep{R} \in \freenames{P} \}
\end{mathpar}

\paragraph{Discussion}
The first line needs little explanation. The second line says that
each free name of the process is replaced with the multiplication of
that name by the scalar. Multiplication of a scalar (name) by a state
(process) results in a process all the names of which have been `moved
over' by parallel composition with the process the scalar
quotes. There is a subtlety that the bound names have to be
manipulated so that multiplied names aren't accidentally
captured. There are many ways to achieve this.

\begin{remark}\label{rem:multiplication_identities}
  The reader is invited to verify that for all $x,y,z \in \QProc$ and $P \in \Proc$
  \begin{mathpar}
    x \cdot \quotep{0} \equiv x 
    \and
    x \cdot y \equiv y \cdot x
    \and
    x \cdot (y \cdot z) \equiv (x \cdot y) \cdot z
    \and \\
    \quotep{0} \cdot P \equiv P
    \and \\
    x \cdot (y \cdot P) \equiv (x \cdot y) \cdot P
    \and \\
    x \cdot (P|Q) \equiv (x \cdot P) | (x \cdot Q)
    \and \\    
  \end{mathpar}
\end{remark}

\subsubsection{Tensor product}

We define a tensor product on processes by structural induction.

\paragraph{Tensor of sums} First note that all summations, including
$\pzero$ and sequence, can be written $\Sigma_{i} x_{i}.A_{i} +
\Sigma_{j} x_{j}.C_{j}$, where we have grouped input-guarded processes
together and output-guarded processes together.

Thus, we can define the tensor product of two summations, $N_{1}\otimes N_{2}$, where

\begin{mathpar}
  N_{1} := \Sigma_{i} x_{i}.A_{i} + \Sigma_{j} x_{j}.C_{j}
  \and
  N_{2} := \Sigma_{i'} y_{i'}.B_{i'} + \Sigma_{j'} y_{j'}.D_{j'} 
\end{mathpar}

as follows.

\begin{mathpar}
  \Sigma_{i} x_{i}.A_{i} + \Sigma_{j} x_{j}.C_{j} \otimes \Sigma_{i'}
  y_{i'}.B_{i'} + \Sigma_{j'} y_{j'}.D_{j'} 
  \and \\
  := \; \Sigma_{i} \Sigma_{i'} \quotep{\stackrel{\vee}{x_{i}}| \stackrel{\vee}{y_{i'}}}.(A_{i}\otimes B_{i'}) \; | \; \Sigma_{i'} \Sigma_{i} \quotep{\stackrel{\vee}{y_{i'}}|\stackrel{\vee}{x_{i}}}.(B_{i'}\otimes A_{i})
  \and
  \;\; | \;\; \Sigma_{j} \Sigma_{j'} \quotep{\stackrel{\vee}{x_{j}}|\stackrel{\vee}{y_{j'}}}.(A_{j}\otimes B_{j'}) \; | \; \Sigma_{j'} \Sigma_{j} \quotep{\stackrel{\vee}{y_{j'}}|\stackrel{\vee}{x_{j}}}.(B_{j'}\otimes A_{j})
\end{mathpar}

\begin{remark}
  Do we need to $x^{L}$ and $y^{R}$ for this construction as well?
\end{remark}

\paragraph{Tensor of parallel compositions} Next, we distribute tensor
over par.

\begin{mathpar}
  P_{1}|P_{2} \otimes Q_{1}|Q_{2} := (P_{1} \otimes Q_{1}) | (P_{1}
  \otimes Q_{2}) | (P_{2} \otimes Q_{1}) | (P_{2} \otimes Q_{2})
\end{mathpar}

\paragraph{Tensor with dropped names} We treat tensor of a
process with a dropped name as parallel composition.

\begin{mathpar}
  P \otimes \dropn{x} := P | \dropn{x}
\end{mathpar}

\paragraph{Tensor of agents}

Finally, we need to define tensor on agents. Note that the definition
of tensor on normal products only tensors inputs with inputs and
outputs with outputs. Thus, we only have to define the operation on
``homogeneous'' pairings.

\begin{mathpar}
  (\vec{x})P \otimes (\vec{y})Q
  \and \\
  := (x_{0}^{L}|y_{0}^{R},\ldots,x_{0}^{L}|y_{n}^{R},\ldots,x_{m}^{L}|y_{0}^{R},\ldots,x_{m}^{L}|y_{n}^R)(P\{ \vec{x}^{L}/\vec{x}\} \otimes Q \{ \vec{y}^{R}/\vec{y}\})
  \and \\
  \clift{\vec{P}} \otimes \clift{\vec{Q}}
  \and \\
  := \clift{P_{0}\otimes Q_{0},\ldots,P_{0}\otimes Q_{n},\ldots,P_{m}\otimes Q_{0},\ldots,P_{m}\otimes Q_{n}}
\end{mathpar}

\begin{remark}
  Observe that arities of tensored abstractions matches arities of
  tensored concretions if the original arities matched. Note also that
  the length of the arities corresponds to the increase in dimension
  we see in ordinary vector space tensor product.
\end{remark}

\begin{remark}
  Operationally, this definition distributes the tensor down to
  components ``linked'' by summation. Tensor over summation is
  intriguing in that it mixes names. Moreover, as a consequence of the
  way it mixes names we have the identities for all $x \in \QProc$ and
  $P,Q \in \Proc$

  \begin{mathpar}
    (x \cdot P) \otimes Q \equiv x \cdot (P \otimes Q) \equiv P \otimes (x \cdot Q)
    \and
    P \otimes \pzero \equiv P
  \end{mathpar}

  that the reader is invited to verify.
\end{remark}

\subsubsection{Annihilation}
\begin{mathpar}
  P^{\perp} := \{ Q | \forall R. P|Q \red^{*} R \Rightarrow R \red^{*} \pzero \}
  \and \\
  P^{\underline{\perp}} := \Sigma_{Q \in P^{\perp}} \quotep{Q}?(y).(\dropn{y}|Q) | \Sigma_{Q \in P^{\perp}} \quotep{Q}\clift{\Box}
\end{mathpar}

\paragraph{Discussion} The reader will note that $P^{\perp}$ is a
\emph{set} of processes, while $P^{\underline{\perp}}$ is a
\emph{context}. We call the set $P^{\perp}$ the \emph{annihilators} of
$P$. The parallel composition of a process in the annihilators of $P$
with $P$ will result in a process, the state space of which has all
paths eventually leading to $\pzero$. Execution may endure loops; but
under reasonable conditions of fairness (naturally guaranteed under
most notions of bisimulation) such a composite process cannot get
stuck in such a loop and will, eventually pop out and terminate.

The context $P^{\underline{\perp}}$ is ready and willing to ``take the
$P$ out of'' the process to which it is applied. It will effectively
transmit the code of the process to which it is applied to one of the
annihilators and run the process against it.

\subsubsection{Evaluation}
We fix $M$ a domain of fully abstract interpretation with an equality
coincident with bisimulation. We take $\meaningof{\cdot} : \Proc \to
M$ to be the map interpreting processes and $\nmeaningof{\cdot} : \M
\to Proc$ to be the map running the other way. Then we define

\begin{mathpar}
  \int P := \nmeaningof{\meaningof{P}}
\end{mathpar}

\paragraph{Discussion}
There are many fully abstract interpretations of Milner's
$\pi$-calculus. Any of them can be used as a basis for interpreting
the reflective calculus here. Equipped with such a domain it is
largely a matter of grinding through to check that the Yoneda
construction for the normalization-by-evaluation program can be
extended to this setting.

\begin{remark}
  The reader is invited to verify that $\int (P^{\underline{\perp}}[P]) = 0$.
\end{remark}

\subsection{Quantum mechanics}

Table \ref{tbl:core_qm_op_defns} gives the core operational definitions

\begin{table}[htp]\label{tbl:core_qm_op_defns}
  \center{
    \fbox{
      \begin{tabular}{c|c}
        quantum mechanics & process calculus \\
        \hline
        scalar & $x := \quotep{P}$ \\
        state vector & $\state{P} := P$ \\
        dual & $\state{P}^{*} := \event{P^{\underline{\perp}}} := \quotep{P^{\underline{\perp}}}[-]$ \\
        matrix & $ \Sigma_{\alpha} \state{P_{\alpha}}x_{\alpha}\event{Q_{\alpha}}$ \\
        vector addition & $\state{P} + \state{Q} := \state{P | Q}$ \\
        tensor product & $\state{P} \otimes \state{Q} := \state{P \otimes Q}$ \\
        inner product & $\innerprod{P}{Q} := \quotep{\int P^{\underline{\perp}}[Q]}$ \\
      \end{tabular}
    }
  }
  \caption{QM - operational definitions}
\end{table}

where

\begin{mathpar}
  \prmatrix{P}{Q} := \fprmatrix{P}{\quotep{\pzero}}{Q}
  \and
  \fprmatrix{P}{x}{Q} := (\state{P},x,\event{Q})
  \and
  (\fprmatrix{P}{x}{Q})(\state{R}) := x \cdot \innerprod{Q}{R} \cdot \state{P}
  \and
  (\fprmatrix{P}{x}{Q})(\event{R}) := x \cdot \innerprod{R}{P} \cdot \event{Q}
\end{mathpar}

\paragraph{Discussion}
As promised: vectors (aka states) are represented as processes; duals
as contextual duals; inner product definition should be compared with
standard inner product definition for ....

\begin{remark}
  Assuming $\int (P^{\underline{\perp}}[P]) = 0$, the reader is
  invited to verify that $(\fprmatrix{P}{x}{P})(\state{P}) = x \cdot \state{P}$.
\end{remark}

\begin{remark}
  The reader is invited to verify that $\innerprod{P}{Q}$ could
  equally well have been written $\quotep{\int \stackrel{\vee}{x}}$
  where $x = \event{P^{\underline{\perp}}}(Q)$.

  One of the motivations for this remark is that there is another way
  to factor these operations. We could package up evaluation in the dual:

  \begin{mathpar}
    \state{P}^{*} := \event{\int P^{\underline{\perp}}} := \quotep{\int P^{\underline{\perp}}}[-]
  \end{mathpar}

  and then have inner product defined by
  
  \begin{mathpar}
    \innerprod{P}{Q} := \event{P}(Q)
  \end{mathpar}

  Hopefully, experience with the calculations will provide guidance on
  the best factoring.
\end{remark}

\begin{remark}
  Assuming $\int (P^{\underline{\perp}}[P]) = 0$, the reader is
  invited to verify that $\forall P,Q. (\prmatrix{0}{Q})(\state{0}) =
  \state{0}$ and dually $(\prmatrix{P}{0})(\event{0}) = \event{0}$.
\end{remark}

\begin{remark}
  i'm a little worried that i don't (yet) have proper support for
  complex conjugacy. But, the observation above may give us a
  clue. According to Abramsky, it must be the case that the scalars
  are iso to the homset of the identity for the tensor -- which the
  observation above characterizes. 

  For now, we will simply bookmark the notion with $\overline{x}$.
\end{remark}

\subsubsection{Adjointness}

We need to give a definition of $(\cdot)^{\dagger}$ for matrices. The
obvious candidate definition is
\begin{mathpar}
(\Sigma_{\alpha}\fprmatrix{P_{\alpha}}{x_{\alpha}}{Q_{\alpha}})^{\dagger}
= \Sigma_{\alpha}\fprmatrix{(Q_{\alpha}^{\underline{\perp}})^{*}}{\overline{x}_{\alpha}}{P_{\alpha}^{\underline{\perp}}} 
\end{mathpar}

But, $(Q_{\alpha}^{\underline{\perp}})^{*}$ requires a name along
which to communicate the process to achieve the context application.

\subsubsection{Basis for a basis}
If processes label states and ``addition'' of states (a.k.a. vector
addition) is interpreted as parallel composition, what corresponds to
notions of linear independence and basis? Here, we recall that Yoshida
has developed a set of \emph{combinators} for an asynchronous verison
of Milner's $\pi$-calculus. These are a finite set of processes such
any process can be expressed as parallel composition of these
combinators together with liberal uses of the new operator and
replication. We can simply give a translation of these into the
present calculus and have reasonable expectation that the property
carries over. That is, that the resultant set allows to express all
processes via parallel composition. Note, however, that there is no
new operator or replication in this calculus. As a result, we expect
that the corresponding set is actually infinite. That is, we expect
that the space is actually infinite dimensional.

\begin{remark}
  The attentive reader may be a bit concerned. Certainly, the
  collection $S$, $K$ and $I$ is a finite set of
  combinators. Shouldn't we expect to see a finite set of combinators
  for an effectively equivalent system? i am very sympathetic to this
  critique and feel it warrants full attention. On the other hand, i
  also have in mind the following analogy. The natural numbers, as a
  monoid under addition, has exactly $1$ generator, while the natural
  numbers, as a monoid under multiplication, has countably many
  generators (the primes). We observe that the application of the
  lambda calculus is much less resource sensitive than the parallel
  composition of the $\pi$-calculus. Could it be the case that we have
  an analogy of the form
  
  \begin{mathpar}
    m + n : MN :: m*n : M|N
  \end{mathpar}

  giving a similar blow up in the set of ``primes''?  This is such a
  wonderful thought that, even if it's not true, i think it's worth
  writing down.
\end{remark}
 

\documentclass[12pt]{llncs}
%\documentclass{jktr}

\usepackage[pdftex]{hyperref}                   
\usepackage {listings}
\usepackage {mathpartir}
\usepackage{bcprules}
%\usepackage{listings}
                       
\usepackage{graphicx} 
%\usepackage[margins=2.5cm,nohead,nofoot]{geometry}
%\usepackage{geometry}
\usepackage{amsfonts}
\usepackage{amstext}
\usepackage{latexsym}
\usepackage{amssymb}
\usepackage{color}


%\include{myPreamble}
\include{qm2pi.local} 

%\ifpdf
%\usepackage[pdftex]{graphicx}
%\else
%\usepackage{graphicx}
%\fi

 % \ifpdf
%  \usepackage{pdfsync}
%  \if


%\title{Brief Article}
%\author{David F. Snyder}
%\author{L.G. Meredith}

%\address{Dept. of Math., Texas State University--San Marcos, San Marcos, TX 78666}
       
\pagestyle{empty}


\begin{document}

\lstset{language=[Objective]Caml,frame=shadowbox}

\input{qm2pi.front}

% section front matter (end)

\input{qm2pi.intro} 
 
% section introduction (end)

% \input{qm2pi.knotations} 

% section notation (end)

\input{qm2pi.process.calculi} 

% section concurrent_process_calculi_and_spatial_logics_ (end)
    
%\input{qm2pi.knots2pi} 

%\input{qm2pi.trefoil} 

%\input{qm2pi.mainthm} 

% subsection basic_interpretation (end)

%\input{qm2pi.rho.presentation} 
\subsection{The syntax and semantics of the notation system}\label{sub:the_syntax_and_semantics_of_the_notation_system} % (fold)

We now summarize a technical presentation of the calculus that
embodies our theory of dynamics. The typical presentation of such a
calculus follows the style of giving generators and relations on
them. The grammar, below, describing term constructors, freely
generates the set of processes, $\Proc$. This set is then quotiented
by a relation known as structural congruence and it is over this set
that the notion of dynamics is expressed. This presentation is
essentially that of \cite{MeredithR05} with the addition of
polyadicity and summation. For readability we have relegated some of
the technical subtleties to an appendix.

\subsubsection{Process grammar}\label{subsub:process_grammar}

\begin{mathpar}
  \inferrule* [lab=synchronization] {} {{M} \bc \pzero \;|\; x?F \;|\; x!C }
  \and
  \inferrule* [lab=abstraction] {} {{F} \bc (x)P}
  \and
  \inferrule* [lab=concretion] {} {{C} \bc \langle Q \rangle}
  \and
  \inferrule* [lab=process] {} {{P,Q} \bc M \;| \;P|Q \;|\; @{x}}
  \and
  \inferrule* [lab=name] {} {{x} \bc \quotep{P}}
\end{mathpar} 

Note that $\vec{x}$ (resp. $\vec{P}$) denotes a vector of names
(resp. processes) of length $|\vec{x}|$ (resp. $|\vec{P}|$). We adopt
the following useful abbreviations.

\begin{mathpar}
   x?(\vec{y}).P := x.(\vec{y})P \and  x\clift{\vec{P}} := x.\clift{\vec{P}}
   \and x!(y) := \lift{x}{\dropn{y}}
   \and \Pi_{i=0}^{n-1}P_i := P_0 | \ldots | P_{n-1}
\end{mathpar}

\subsubsection{Structural congruence}

\paragraph{Free and bound names and alpha-equivalence.} At the
core of structural equivalence is alpha-equivalence which identifies
process that are the same up to a change of variable. Formally, we
recognize the distinction between free and bound names. The free names
of a process, $\freenames{P}$, may be calculated recursively as
follows:

\begin{mathpar}
\freenames{\pzero} := \emptyset
  \and \\
  \freenames{x?(y).P} := \{ x \} \cup (\freenames{P} \setminus \{ y \})
  \and 
  \freenames{x!\langle P \rangle} := \{ x \} \cup \{ P \} 
  \and \\
  \freenames{P|Q} := \freenames{P} \cup \freenames{Q}
  \and \\
  \freenames{@{x}} := \{ x \}
\end{mathpar}

$\pi$
$\quotep{\pi}$

$\freenames{-} : \pi \to \mathcal{P}(\quotep{\pi})$

\begin{eqnarray*}
  \freenames{\pzero} & := & \emptyset \\
  \freenames{x?(y).P} & := & \{ x \} \cup (\freenames{P} \setminus \{ y \}) \\
  \freenames{x!\langle P \rangle} & := & \{ x \} \cup \{ P \} \\
  \freenames{P|Q} & := & \freenames{P} \cup \freenames{Q} \\
  \freenames{\dropn{x}} & := & \{ x \}
\end{eqnarray*}

The bound names of a process, $\boundnames{P}$, are those names occurring in $P$
that are not free. For example, in $x?(y).0$, the name $x$ is free, while $y$ is bound.

\begin{mathpar}
  \inferrule* [lab=monoidal-laws] {} { P|Q \equiv Q|P \and P|0 \equiv P \and P|(Q|R) \equiv (P|Q)|R }
\end{mathpar}

\begin{mathpar}
  \inferrule* [lab=alpha-equivalence] {} { (x)P \equiv (y)P\{y/x\} \and y \not\in \freenames{P} }
\end{mathpar}

\begin{definition}
Then two processes, $P,Q$, are alpha-equivalent if $P = Q\{\vec{y}/\vec{x}\}$ for
some $\vec{x} \in \boundnames{Q},\vec{y} \in \boundnames{P}$, where $Q\{\vec{y}/\vec{x}\}$
denotes the capture-avoiding substitution of $\vec{y}$ for $\vec{x}$ in $Q$.
\end{definition}

\begin{definition}
  The {\em structural congruence} \cite{SangiorgiWalker} , $\equiv$,
  between processes is the least congruence containing
  alpha-equivalence, satisfying the abelian monoid laws
  (associativity, commutativity and $\pzero$ as identity) for parallel
  composition $|$ and for summation $+$.
\end{definition}

\subsection{Name equivalence}

We take name equivalence, written $\nameeq$, to be the smallest
equivalence relation generated by the following rules.

\begin{mathpar}
\inferrule*[lab=Quote-drop]
{ }
{ \quotep{@{x}} \nameeq x }

\inferrule*[lab=Struct-equiv]
{ P \scong Q }
{ \quotep{P} \nameeq \quotep{Q} }
\end{mathpar}

The astute reader will have noticed that the mutual recursion of names
and processes imposes a mutual recursion on alpha-equivalence and
structural equivalence via name-equivalence. Fortunately, all of this
works out pleasantly and we may calculate in the natural way, free of
concern. The reader interested in the details is referred to the
appendix \ref{appendix:rho_details}.

\subsection{Substitution}

We use $\Proc$ for the set of processes, $\QProc$ for the set of
names, and $\id{\{}\vec{y} / \vec{x} \id{\}}$ to denote partial maps,
$s : \QProc \rightarrow \QProc$. A map, $s$ lifts, uniquely, to a map
on process terms, $\widehat{s} : \Proc \rightarrow \Proc$ by the
following equations.

\begin{mathpar}
  (0) \psubstp{Q}{P} := 0 \\
  (R \juxtap S) \psubstp{Q}{P}
  :=    
  (R)\psubstp{Q}{P} \juxtap (S) \psubstp{Q}{P} \\
  (x?(y).R) \psubstp{Q}{P}    
  :=    
  (x)\substp{Q}{P} (z)\concat( (R \psubstn{z}{y}) \psubstp{Q}{P} ) \\
  (\lift{x}{R}) \psubstp{Q}{P}  
  :=
  \lift{(x)\substp{Q}{P}}{ R \psubstp{Q}{P} } \\
%   (\dropn{x})  \psubstp{Q}{P}       
%   := 
%   \left\{ 
%     \begin{array}{ccc} 
%       \dropn{\quotep{Q}} & & x \nameeq \quotep{P} \\
%       \dropn{x} & & otherwise \\
%     \end{array}
%   \right. 
  (\dropn{x})  \psubstp{Q}{P}       
  := 
  \left\{ 
    \begin{array}{ccc} 
      Q & & x \nameeq \quotep{P} \\
      \dropn{x} & & otherwise \\
    \end{array}
  \right.
\end{mathpar}
 

where

\begin{eqnarray}
  (x)\id{\{} \lpquote Q \rpquote / \lpquote P \rpquote \id{\}}            = 
  \left\{ 
    \begin{array}{ccc}
      \lpquote Q \rpquote & & x \nameeq \lpquote P \rpquote \\
      x & & otherwise \\
    \end{array}
  \right. \nonumber
\end{eqnarray}

and $z$ is chosen distinct from $\quotep{P}$, $\quotep{Q}$, the free
names in $Q$, and all the names in $R$. Our $\alpha$-equivalence will
be built in the standard way from this substitution.

\begin{remark}\label{rem:no_self_referential_names}
  One consequence of these definitions is that $\forall P. \quotep{P}
  \not\in \freenames{P}$.
\end{remark}

\subsection{ Dynamic quote: an example }

Anticipating something of what's to come, consider applying the
substitution, $\widehat{\id{\{}u / z \id{\}}}$, to the following pair
of processes, $\lift{w}{y!(z)}$ and $w[ \lpquote y!(z) \rpquote ]$.

\begin{eqnarray}
	\lift{w}{y!(z)}\widehat{\id{\{}u / z \id{\}}}
		& = &
		\lift{w}{y!(u)} \nonumber\\
	w[ \lpquote y!(z) \rpquote ] \widehat{ \id{\{}u / z \id{\}} }
		& = &
		w[ \lpquote y!(z) \rpquote ] \nonumber
\end{eqnarray}

Because the body of the process between quotes is impervious to
substitution, we get radically different answers. In fact, by
examining the first process in an input context,
e.g. $x?(z).\lift{w}{y!(z)}$, we see that the process under the lift
operator may be shaped by prefixed inputs binding a name inside it. In
this sense, the lift operator will be seen as a way to dynamically
construct processes before reifying them as names.

Finally equipped with these standard features we can present the
dynamics of the calculus.

\subsubsection{Operational semantics} 

Finally, we introduce the computational dynamics. What marks these
algebras as distinct from other more traditionally studied algebraic
structures, e.g. vector spaces or polynomial rings, is the manner in
which dynamics is captured. In traditional structures, dynamics is typically
expressed through morphisms between such structures, as in linear maps
between vector spaces or morphisms between rings. In algebras
associated with the semantics of computation, the dynamics is
expressed as part of the algebraic structure itself, through a
reduction reduction relation typically denoted by $\red$. Below, we
give a recursive presentation of this relation for the calculus used
in the encoding.

$\red \subseteq \pi \times \pi$
$\red : \pi \to \mathcal{P}(\pi)$

\begin{mathpar}
  \inferrule* [lab=Comm] { \textsf{match}( x_{src}, x_{trgt} ) } { x_{trgt}?(y)P \; | \; x_{src}!\langle {Q} \rangle \red P\{\quotep{Q}/y}\} }
  \and \\
  \inferrule* [lab=Par] {{P} \red {P}'} {{{P} | {Q}} \red {{P}' | {Q}}}
  \and
  \inferrule* [lab=Equiv]{{{P} \scong {P}'} \andalso {{P}' \red {Q}'} \andalso {{Q}' \scong {Q}}}{{P} \red {Q}}
\end{mathpar}

\begin{eqnarray*}
  match_{\equiv} (\quotep{P},\quotep{Q}) & := & P \equiv Q \\
  match_{\dagger}(\quotep{P},\quotep{Q}) & := & \forall R. P|Q \red^{*} R => R \red^{*} 0 \\
  match_{K}(\quotep{P},\quotep{Q}) & := & K \mbox{ for some context } K
\end{eqnarray*}

$u?(x)P | u!\langle Q \rangle \red P\{\quotep{Q}/x\}$

%We write $\wred$ for $\red^*$, and $P\red$ if $\exists Q $ such that $ P \red Q$.
We write $P\red$ if $\exists Q $ such that $ P \red Q$ and $P\not\red$, otherwise.

\section{Replication}

As mentioned before, it is known that replication (and hence
recursion) can be implemented in a higher-order process algebra
\cite{SangiorgiWalker}. As our first example of calculation with the
machinery thus far presented we give the construction explicitly in
the {\rhoc}.

\begin{eqnarray}
	D_{x} & := & \prefix{x}{y}{(\binpar{\outputp{x}{y}}{@{y}})} \nonumber\\
	\bangp_{x}{P} & := & \binpar{{x}!\langle{\binpar{D_{x}}{P}}\rangle}{D_{x}} \nonumber
\end{eqnarray}

\begin{eqnarray}
	\bangp_{x}{P} & & \nonumber\\
	=
	& {x}!\langle{(\prefix{x}{y}{(\outputp{x}{y} | @{y})) | P}}\rangle 
	      | \prefix{x}{y}{(\outputp{x}{y} | @{y})} & \nonumber\\
	\red
	& (\outputp{x}{y} | @{y})\substn{\quotep{(\prefix{x}{y}{(@{y} | \outputp{x}{y})) | P}}}{y} & \nonumber\\
	=
	& \outputp{x}{\quotep{(\prefix{x}{y}{(\outputp{x}{y} | @{y})) | P}}}
	  | {(\prefix{x}{y}{(\outputp{x}{y} | @{y})) | P}} & \nonumber\\
	\red
	& \ldots & \nonumber\\
	\red^*
	& P | P | \ldots & \nonumber
\end{eqnarray}

Of course, this encoding, as an implementation, runs away, unfolding
$\bangp{P}$ eagerly. A lazier and more implementable replication
operator, restricted to input-guarded processes, may be obtained as follows.

\begin{eqnarray}
\bangp{\prefix{u}{v}{P}} 
	:= 
	\binpar{\lift{x}{\prefix{u}{v}{(\binpar{D(x)}{P})}}}{D(x)} \nonumber
\end{eqnarray}

\begin{remark}
  Note that the lazier definition still does not deal with summation
  or mixed summation (i.e. sums over input and output). The reader is
  invited to construct definitions of replication that deal with these
  features. 

  Further, the definitions are parameterized in a name, $x$. Can you,
  gentle reader, make a definition that eliminates this parameter and
  guarantees no accidental interaction between the replication
  machinery and the process being replicated -- i.e. no accidental
  sharing of names used by the process to get its work done and the
  name(s) used by the replication to effect copying. This latter
  revision of the definition of replication is crucial to obtaining
  the expected identity $!!P \sim !P$.
\end{remark}

\begin{remark}\label{rem:paradoxical_combinator}
  The reader familiar with the lambda calculus will have noticed the
  similarity between $D$ and the paradoxical combinator.

  [Ed. note: the existence of this seems to suggest we have to be more
  restrictive on the set of processes and names we admit if we are to
  support no-cloning.]
\end{remark}

\subsubsection{Bisimulation}

The computational dynamics gives rise to another kind of equivalence,
the equivalence of computational behavior. As previously mentioned
this is typically captured \emph{via} some form of bisimulation.

% The notion we use in this paper is weak barbed bisimulation
% \cite{milner91polyadicpi}.

The notion we use in this paper is derived from weak barbed
bisimulation \cite{milner91polyadicpi}. 

\begin{definition}
An \emph{observation relation}, $\downarrow_{\mathcal N}$, over a set
of names, $\mathcal N$, is the smallest relation satisfying the rules
below.

\infrule[Out-barb]{y \in {\mathcal N}, \; x \nameeq y}
		  {\outputp{x}{v} \downarrow_{\mathcal N} x}
\infrule[Par-barb]{\mbox{$P\downarrow_{\mathcal N} x$ or $Q\downarrow_{\mathcal N} x$}}
		  {\binpar{P}{Q} \downarrow_{\mathcal N} x}

We write $P \Downarrow_{\mathcal N} x$ if there is $Q$ such that 
$P \wred Q$ and $Q \downarrow_{\mathcal N} x$.
\end{definition}

\begin{definition}
%\label{def.bbisim}
An  ${\mathcal N}$-\emph{barbed bisimulation} over a set of names, ${\mathcal N}$, is a symmetric binary relation 
${\mathcal S}_{\mathcal N}$ between agents such that $P\rel{S}_{\mathcal N}Q$ implies:
\begin{enumerate}
\item If $P \red P'$ then $Q \wred Q'$ and $P'\rel{S}_{\mathcal N} Q'$.
\item If $P\downarrow_{\mathcal N} x$, then $Q\Downarrow_{\mathcal N} x$.
\end{enumerate}
$P$ is ${\mathcal N}$-barbed bisimilar to $Q$, written
$P \wbbisim_{\mathcal N} Q$, if $P \rel{S}_{\mathcal N} Q$ for some ${\mathcal N}$-barbed bisimulation ${\mathcal S}_{\mathcal N}$.
\end{definition}

$\mathcal{R} \subseteq \pi \times \pi$

$P \mathcal{R} Q => \forall P'. P \red P' \Rightarrow \exists Q'. Q \red Q', P' \mathcal{R} Q'$

$P \vdash x \Rightarrow Q \vdash x$

\begin{mathpar}
  \inferrule*[lab=Out-barb]{x \nameeq y}{{y}!\langle{Q}\rangle \vdash x}
  \and
  \inferrule*[lab=Par-barb]{\mbox{$P\vdash x$ or $Q\vdash x$}}{\binpar{P}{Q} \vdash x}
\end{mathpar}

\subsubsection{Contexts}

One of the principle advantages of computational calculi like the
$\pi$-calculus is a well-defined notion of context,
contextual-equivalence and a correlation between
contextual-equivalence and notions of bisimulation. The notion of
context allows the decomposition of a process into (sub-)process and
its syntactic environment, its context. Thus, a context may be
thought of as a process with a ``hole'' (written $\Box$) in it. The
application of a context $M$ to a process $P$, written $M[P]$, is
tantamount to filling the hole in $M$ with $P$. In this paper we do
not need the full weight of this theory, but do make use of the notion
of context in the proof the main theorem. 

\begin{mathpar}
  \inferrule* [lab=summation] {} {{M_{M},M_{N}} \bc \Box \;|\; x.M_{A} \;|\; M_{M}+M_{N}}
  \and
  \inferrule* [lab=agent] {} {{M_{A}} \bc (\vec{x})M_{P} \;| \; \clift{P_0,\ldots,M_{P},\ldots,P_N}}
  \and \\
  \inferrule* [lab=process] {} {{M_{P}} \bc M_{N} \;| \;P|M_{P} }
\end{mathpar} 

\begin{mathpar}
  \inferrule* [lab=sychronization] {} {M_{N} \bc \Box \;|\; x?M_{F} \;|\; x!M_{C}}
  \and
  \inferrule* [lab=abstraction] {} {{M_{F}} \bc (x)M_{P} }
  \and
  \inferrule* [lab=concretion] {} {{M_{C}} \bc \langle M_{P} \rangle }
  \and \\
  \inferrule* [lab=process] {} {{M_{P}} \bc M_{N} \;| \;P|M_{P} }
\end{mathpar}

\begin{definition}[contextual application] Given a context $M$, and
  process $P$, we define the \emph{contextual application}, $M[P] :=
  M\{P/\Box\}$. That is, the contextual application of M to P is the
  substitution of $P$ for $\Box$ in $M$.
\end{definition}

$\meaningof{-} : L \to \mathcal{P}(\pi)$

\begin{mathpar}
  \inferrule* [lab=collection] {} {\meaningof{true} = \pi, \and \meaningof{~E} = \pi \setminus \meaningof{E}, \and \meaningof{E_{1} \& E_{2}} = \meaningof{E_{1}} \cap \meaningof{E_{2}}}
\end{mathpar}

\begin{mathpar}
  \inferrule* [lab=structure] {} {\meaningof{0} = \{ P \in \pi | P \equiv 0 \}, \and \\ \meaningof{E_1 | E_2} = \{ P \in \pi | P \equiv P_{1} | P_{2}, P_{1} \in \meaningof{E_{1}}, P_{2} \in \meaningof{E_2}\} }
\end{mathpar}

\begin{mathpar}
 \inferrule* [lab=behavior] {} {\meaningof{\langle a?b \rangle E} = \{ P \in \pi | P \equiv Q | u?(y)P', \\ \and \\\\ \and \\ \;\;\; u \in \meaningof{a}, \forall z.P'\{z/y\} \in \meaningof{E\{z/b\}}\}, \and \\ \meaningof{a!E} = \{ P \in \pi | P \equiv Q | x!\langle P' \rangle, x \in \meaningof{a} P' \in \meaningof{E}\} }
\end{mathpar}

\begin{mathpar}
 \inferrule* [lab=nominal] {} {\meaningof{\quotep{E}} = \{ \quotep{P} \in \quotep{\pi} | P \in \meaningof{E} \}, \and \meaningof{\quotep{P}} = \{ \quotep{Q} \in \quotep{\pi} | P \equiv Q \} \and \\ \meaningof{@\quotep{E}} = \{ P \in \pi | P \equiv @x, x \in \meaningof{E} \}}
\end{mathpar}

\begin{eqnarray*}
  \\
  \meaningof{-} : TS \to ST
\end{eqnarray*}

\begin{eqnarray*}
  \\
  L : TS \to ST
\end{eqnarray*}

\begin{eqnarray*}
  \\
  P \models E \iff P \in \meaningof{E}
\end{eqnarray*}

\begin{eqnarray*}
  P \approx_{L} Q \iff \forall E \in L. P \models E \iff Q \models E
\end{eqnarray*}

\begin{eqnarray*}
  P \approx_{K} Q
\end{eqnarray*}

\begin{eqnarray*}
  P \approx Q
\end{eqnarray*}

$\approx_{K} = \approx = \approx_{L}$

\subsubsection{Contextual duality}

Note that contexts extend the quotation operation to a family of
operations from processes to names. Given a context, $M$, we can
define a \emph{nominal context}, $\quotep{M}$ by $\quotep{M}[P] :=
\quotep{M[P]}$. To foreshadow what is to come we observe that these
operations enjoy a duality with processes very much like the duality
between vectors and maps from vectors to scalars.

Further, because the calculus is essentially higher-order, we have a
correspondence between contexts and processes. More specifically,
given a name $x$ and a context $M$ we can construct $M^{*}_{x}$ such
that 

\begin{mathpar}
  M^{*}_{x} | \lift{x}{P} \red M[P]
\end{mathpar}

namely,

\begin{mathpar}
  M^{*}_{x} := x?(u).M[\dropn{u}]
\end{mathpar}

The dependence of $M^{*}_{x}$ on a name makes it an abstraction, 

\begin{mathpar}
  M^{*} := (x)x?(u).M[\dropn{u}]
\end{mathpar}

\subsection{Additional notation}

It will sometimes be convenient to denote the process a name
quotes. We already have the notation $x = \quotep{P}$, but it will be
convenient to introduce an alternate notation, $\procn{x}$, when we
want to emphasize the connection to the use of the name. Note that, by
virtue of name equivalence, $\quotep{\procn{x}} \nameeq x$; so, the
notation is consistent with previous definitions.

Further, because names have structure it is possible to effect
substitutions on the basis of that structure. This means we need to
upgrade our notation for substitutions, which we accomplish by
adapting comprehension notation. Thus,

\begin{mathpar}
  P\{ y / x : x \in S \}
\end{mathpar}

is interpreted to mean the process derived from P by replacing (in a
capture-avoiding manner) each occurrence of $x$ in $S$ by $y$. For example,

\begin{mathpar}
  P\{ \quotep{\procn{x}|\procn{x}} / x : x \in \freenames{P} \}
\end{mathpar}

will replace each (occurrence) of a free name $x$ in $P$ by
$\quotep{\procn{x}|\procn{x}}$.

Also, we will avail ourselves of the notation $x^{L}$ and $x^{R}$ to
denote injections of a name into disjoint copies of the name
space. There are numerous ways to accomplish this. One example can be
found in \cite{MeredithR05}. This notation overloads to vectors of
names: $\vec{x}^{\pi} := (x_{i}^{\pi} \; : \; 0 \leq i < |\vec{x}| )$ where $\pi \in \{L,R\}$.

We also use $P^{\Box} := P|\Box$.

In \cite{MeredithR05} an interpretation of the new operator is
given. It turns out that there are several possible interpretations
all enjoying the requisite algebraic properties of the operator (see
\cite{milner91polyadicpi}). We will therefore make liberal use of
$(\nu\; \vec{x})P$.

% subsection the_syntax_and_semantics_of_the_notation_system (end)   

\input{qm2pi.qmops} 

\input{qm2pi.sterngerlach} 

\input{qm2pi.metric} 

% section concurrent_process_calculi (end)

%\input{qm2pi.proofsketch}

% section proof sketch (end)

%\input{qm2pi.slviaknots} 

% section spatial logic via knots (end)

\input{qm2pi.conclusion}

% section conclusion (end)

%\input{qm2pi.dtcodes} 

% section wiring algorithm (end)

\input{qm2pi.ack} 

% section acknowledgments (end)

\newpage


\bibliographystyle{plain}   
\bibliography{../../biblios/main.bib}

\input{qm2pi.rhodetails}

\end{document}

 

\documentclass[12pt]{llncs}
%\documentclass{jktr}

\usepackage[pdftex]{hyperref}                   
\usepackage {listings}
\usepackage {mathpartir}
\usepackage{bcprules}
%\usepackage{listings}
                       
\usepackage{graphicx} 
%\usepackage[margins=2.5cm,nohead,nofoot]{geometry}
%\usepackage{geometry}
\usepackage{amsfonts}
\usepackage{amstext}
\usepackage{latexsym}
\usepackage{amssymb}
\usepackage{color}


%\include{myPreamble}
\include{qm2pi.local} 

%\ifpdf
%\usepackage[pdftex]{graphicx}
%\else
%\usepackage{graphicx}
%\fi

 % \ifpdf
%  \usepackage{pdfsync}
%  \if


%\title{Brief Article}
%\author{David F. Snyder}
%\author{L.G. Meredith}

%\address{Dept. of Math., Texas State University--San Marcos, San Marcos, TX 78666}
       
\pagestyle{empty}


\begin{document}

\lstset{language=[Objective]Caml,frame=shadowbox}

\input{qm2pi.front}

% section front matter (end)

\input{qm2pi.intro} 
 
% section introduction (end)

% \input{qm2pi.knotations} 

% section notation (end)

\input{qm2pi.process.calculi} 

% section concurrent_process_calculi_and_spatial_logics_ (end)
    
%\input{qm2pi.knots2pi} 

%\input{qm2pi.trefoil} 

%\input{qm2pi.mainthm} 

% subsection basic_interpretation (end)

%\input{qm2pi.rho.presentation} 
\subsection{The syntax and semantics of the notation system}\label{sub:the_syntax_and_semantics_of_the_notation_system} % (fold)

We now summarize a technical presentation of the calculus that
embodies our theory of dynamics. The typical presentation of such a
calculus follows the style of giving generators and relations on
them. The grammar, below, describing term constructors, freely
generates the set of processes, $\Proc$. This set is then quotiented
by a relation known as structural congruence and it is over this set
that the notion of dynamics is expressed. This presentation is
essentially that of \cite{MeredithR05} with the addition of
polyadicity and summation. For readability we have relegated some of
the technical subtleties to an appendix.

\subsubsection{Process grammar}\label{subsub:process_grammar}

\begin{mathpar}
  \inferrule* [lab=synchronization] {} {{M} \bc \pzero \;|\; x?F \;|\; x!C }
  \and
  \inferrule* [lab=abstraction] {} {{F} \bc (x)P}
  \and
  \inferrule* [lab=concretion] {} {{C} \bc \langle Q \rangle}
  \and
  \inferrule* [lab=process] {} {{P,Q} \bc M \;| \;P|Q \;|\; @{x}}
  \and
  \inferrule* [lab=name] {} {{x} \bc \quotep{P}}
\end{mathpar} 

Note that $\vec{x}$ (resp. $\vec{P}$) denotes a vector of names
(resp. processes) of length $|\vec{x}|$ (resp. $|\vec{P}|$). We adopt
the following useful abbreviations.

\begin{mathpar}
   x?(\vec{y}).P := x.(\vec{y})P \and  x\clift{\vec{P}} := x.\clift{\vec{P}}
   \and x!(y) := \lift{x}{\dropn{y}}
   \and \Pi_{i=0}^{n-1}P_i := P_0 | \ldots | P_{n-1}
\end{mathpar}

\subsubsection{Structural congruence}

\paragraph{Free and bound names and alpha-equivalence.} At the
core of structural equivalence is alpha-equivalence which identifies
process that are the same up to a change of variable. Formally, we
recognize the distinction between free and bound names. The free names
of a process, $\freenames{P}$, may be calculated recursively as
follows:

\begin{mathpar}
\freenames{\pzero} := \emptyset
  \and \\
  \freenames{x?(y).P} := \{ x \} \cup (\freenames{P} \setminus \{ y \})
  \and 
  \freenames{x!\langle P \rangle} := \{ x \} \cup \{ P \} 
  \and \\
  \freenames{P|Q} := \freenames{P} \cup \freenames{Q}
  \and \\
  \freenames{@{x}} := \{ x \}
\end{mathpar}

$\pi$
$\quotep{\pi}$

$\freenames{-} : \pi \to \mathcal{P}(\quotep{\pi})$

\begin{eqnarray*}
  \freenames{\pzero} & := & \emptyset \\
  \freenames{x?(y).P} & := & \{ x \} \cup (\freenames{P} \setminus \{ y \}) \\
  \freenames{x!\langle P \rangle} & := & \{ x \} \cup \{ P \} \\
  \freenames{P|Q} & := & \freenames{P} \cup \freenames{Q} \\
  \freenames{\dropn{x}} & := & \{ x \}
\end{eqnarray*}

The bound names of a process, $\boundnames{P}$, are those names occurring in $P$
that are not free. For example, in $x?(y).0$, the name $x$ is free, while $y$ is bound.

\begin{mathpar}
  \inferrule* [lab=monoidal-laws] {} { P|Q \equiv Q|P \and P|0 \equiv P \and P|(Q|R) \equiv (P|Q)|R }
\end{mathpar}

\begin{mathpar}
  \inferrule* [lab=alpha-equivalence] {} { (x)P \equiv (y)P\{y/x\} \and y \not\in \freenames{P} }
\end{mathpar}

\begin{definition}
Then two processes, $P,Q$, are alpha-equivalent if $P = Q\{\vec{y}/\vec{x}\}$ for
some $\vec{x} \in \boundnames{Q},\vec{y} \in \boundnames{P}$, where $Q\{\vec{y}/\vec{x}\}$
denotes the capture-avoiding substitution of $\vec{y}$ for $\vec{x}$ in $Q$.
\end{definition}

\begin{definition}
  The {\em structural congruence} \cite{SangiorgiWalker} , $\equiv$,
  between processes is the least congruence containing
  alpha-equivalence, satisfying the abelian monoid laws
  (associativity, commutativity and $\pzero$ as identity) for parallel
  composition $|$ and for summation $+$.
\end{definition}

\subsection{Name equivalence}

We take name equivalence, written $\nameeq$, to be the smallest
equivalence relation generated by the following rules.

\begin{mathpar}
\inferrule*[lab=Quote-drop]
{ }
{ \quotep{@{x}} \nameeq x }

\inferrule*[lab=Struct-equiv]
{ P \scong Q }
{ \quotep{P} \nameeq \quotep{Q} }
\end{mathpar}

The astute reader will have noticed that the mutual recursion of names
and processes imposes a mutual recursion on alpha-equivalence and
structural equivalence via name-equivalence. Fortunately, all of this
works out pleasantly and we may calculate in the natural way, free of
concern. The reader interested in the details is referred to the
appendix \ref{appendix:rho_details}.

\subsection{Substitution}

We use $\Proc$ for the set of processes, $\QProc$ for the set of
names, and $\id{\{}\vec{y} / \vec{x} \id{\}}$ to denote partial maps,
$s : \QProc \rightarrow \QProc$. A map, $s$ lifts, uniquely, to a map
on process terms, $\widehat{s} : \Proc \rightarrow \Proc$ by the
following equations.

\begin{mathpar}
  (0) \psubstp{Q}{P} := 0 \\
  (R \juxtap S) \psubstp{Q}{P}
  :=    
  (R)\psubstp{Q}{P} \juxtap (S) \psubstp{Q}{P} \\
  (x?(y).R) \psubstp{Q}{P}    
  :=    
  (x)\substp{Q}{P} (z)\concat( (R \psubstn{z}{y}) \psubstp{Q}{P} ) \\
  (\lift{x}{R}) \psubstp{Q}{P}  
  :=
  \lift{(x)\substp{Q}{P}}{ R \psubstp{Q}{P} } \\
%   (\dropn{x})  \psubstp{Q}{P}       
%   := 
%   \left\{ 
%     \begin{array}{ccc} 
%       \dropn{\quotep{Q}} & & x \nameeq \quotep{P} \\
%       \dropn{x} & & otherwise \\
%     \end{array}
%   \right. 
  (\dropn{x})  \psubstp{Q}{P}       
  := 
  \left\{ 
    \begin{array}{ccc} 
      Q & & x \nameeq \quotep{P} \\
      \dropn{x} & & otherwise \\
    \end{array}
  \right.
\end{mathpar}
 

where

\begin{eqnarray}
  (x)\id{\{} \lpquote Q \rpquote / \lpquote P \rpquote \id{\}}            = 
  \left\{ 
    \begin{array}{ccc}
      \lpquote Q \rpquote & & x \nameeq \lpquote P \rpquote \\
      x & & otherwise \\
    \end{array}
  \right. \nonumber
\end{eqnarray}

and $z$ is chosen distinct from $\quotep{P}$, $\quotep{Q}$, the free
names in $Q$, and all the names in $R$. Our $\alpha$-equivalence will
be built in the standard way from this substitution.

\begin{remark}\label{rem:no_self_referential_names}
  One consequence of these definitions is that $\forall P. \quotep{P}
  \not\in \freenames{P}$.
\end{remark}

\subsection{ Dynamic quote: an example }

Anticipating something of what's to come, consider applying the
substitution, $\widehat{\id{\{}u / z \id{\}}}$, to the following pair
of processes, $\lift{w}{y!(z)}$ and $w[ \lpquote y!(z) \rpquote ]$.

\begin{eqnarray}
	\lift{w}{y!(z)}\widehat{\id{\{}u / z \id{\}}}
		& = &
		\lift{w}{y!(u)} \nonumber\\
	w[ \lpquote y!(z) \rpquote ] \widehat{ \id{\{}u / z \id{\}} }
		& = &
		w[ \lpquote y!(z) \rpquote ] \nonumber
\end{eqnarray}

Because the body of the process between quotes is impervious to
substitution, we get radically different answers. In fact, by
examining the first process in an input context,
e.g. $x?(z).\lift{w}{y!(z)}$, we see that the process under the lift
operator may be shaped by prefixed inputs binding a name inside it. In
this sense, the lift operator will be seen as a way to dynamically
construct processes before reifying them as names.

Finally equipped with these standard features we can present the
dynamics of the calculus.

\subsubsection{Operational semantics} 

Finally, we introduce the computational dynamics. What marks these
algebras as distinct from other more traditionally studied algebraic
structures, e.g. vector spaces or polynomial rings, is the manner in
which dynamics is captured. In traditional structures, dynamics is typically
expressed through morphisms between such structures, as in linear maps
between vector spaces or morphisms between rings. In algebras
associated with the semantics of computation, the dynamics is
expressed as part of the algebraic structure itself, through a
reduction reduction relation typically denoted by $\red$. Below, we
give a recursive presentation of this relation for the calculus used
in the encoding.

$\red \subseteq \pi \times \pi$
$\red : \pi \to \mathcal{P}(\pi)$

\begin{mathpar}
  \inferrule* [lab=Comm] { \textsf{match}( x_{src}, x_{trgt} ) } { x_{trgt}?(y)P \; | \; x_{src}!\langle {Q} \rangle \red P\{\quotep{Q}/y}\} }
  \and \\
  \inferrule* [lab=Par] {{P} \red {P}'} {{{P} | {Q}} \red {{P}' | {Q}}}
  \and
  \inferrule* [lab=Equiv]{{{P} \scong {P}'} \andalso {{P}' \red {Q}'} \andalso {{Q}' \scong {Q}}}{{P} \red {Q}}
\end{mathpar}

\begin{eqnarray*}
  match_{\equiv} (\quotep{P},\quotep{Q}) & := & P \equiv Q \\
  match_{\dagger}(\quotep{P},\quotep{Q}) & := & \forall R. P|Q \red^{*} R => R \red^{*} 0 \\
  match_{K}(\quotep{P},\quotep{Q}) & := & K \mbox{ for some context } K
\end{eqnarray*}

$u?(x)P | u!\langle Q \rangle \red P\{\quotep{Q}/x\}$

%We write $\wred$ for $\red^*$, and $P\red$ if $\exists Q $ such that $ P \red Q$.
We write $P\red$ if $\exists Q $ such that $ P \red Q$ and $P\not\red$, otherwise.

\section{Replication}

As mentioned before, it is known that replication (and hence
recursion) can be implemented in a higher-order process algebra
\cite{SangiorgiWalker}. As our first example of calculation with the
machinery thus far presented we give the construction explicitly in
the {\rhoc}.

\begin{eqnarray}
	D_{x} & := & \prefix{x}{y}{(\binpar{\outputp{x}{y}}{@{y}})} \nonumber\\
	\bangp_{x}{P} & := & \binpar{{x}!\langle{\binpar{D_{x}}{P}}\rangle}{D_{x}} \nonumber
\end{eqnarray}

\begin{eqnarray}
	\bangp_{x}{P} & & \nonumber\\
	=
	& {x}!\langle{(\prefix{x}{y}{(\outputp{x}{y} | @{y})) | P}}\rangle 
	      | \prefix{x}{y}{(\outputp{x}{y} | @{y})} & \nonumber\\
	\red
	& (\outputp{x}{y} | @{y})\substn{\quotep{(\prefix{x}{y}{(@{y} | \outputp{x}{y})) | P}}}{y} & \nonumber\\
	=
	& \outputp{x}{\quotep{(\prefix{x}{y}{(\outputp{x}{y} | @{y})) | P}}}
	  | {(\prefix{x}{y}{(\outputp{x}{y} | @{y})) | P}} & \nonumber\\
	\red
	& \ldots & \nonumber\\
	\red^*
	& P | P | \ldots & \nonumber
\end{eqnarray}

Of course, this encoding, as an implementation, runs away, unfolding
$\bangp{P}$ eagerly. A lazier and more implementable replication
operator, restricted to input-guarded processes, may be obtained as follows.

\begin{eqnarray}
\bangp{\prefix{u}{v}{P}} 
	:= 
	\binpar{\lift{x}{\prefix{u}{v}{(\binpar{D(x)}{P})}}}{D(x)} \nonumber
\end{eqnarray}

\begin{remark}
  Note that the lazier definition still does not deal with summation
  or mixed summation (i.e. sums over input and output). The reader is
  invited to construct definitions of replication that deal with these
  features. 

  Further, the definitions are parameterized in a name, $x$. Can you,
  gentle reader, make a definition that eliminates this parameter and
  guarantees no accidental interaction between the replication
  machinery and the process being replicated -- i.e. no accidental
  sharing of names used by the process to get its work done and the
  name(s) used by the replication to effect copying. This latter
  revision of the definition of replication is crucial to obtaining
  the expected identity $!!P \sim !P$.
\end{remark}

\begin{remark}\label{rem:paradoxical_combinator}
  The reader familiar with the lambda calculus will have noticed the
  similarity between $D$ and the paradoxical combinator.

  [Ed. note: the existence of this seems to suggest we have to be more
  restrictive on the set of processes and names we admit if we are to
  support no-cloning.]
\end{remark}

\subsubsection{Bisimulation}

The computational dynamics gives rise to another kind of equivalence,
the equivalence of computational behavior. As previously mentioned
this is typically captured \emph{via} some form of bisimulation.

% The notion we use in this paper is weak barbed bisimulation
% \cite{milner91polyadicpi}.

The notion we use in this paper is derived from weak barbed
bisimulation \cite{milner91polyadicpi}. 

\begin{definition}
An \emph{observation relation}, $\downarrow_{\mathcal N}$, over a set
of names, $\mathcal N$, is the smallest relation satisfying the rules
below.

\infrule[Out-barb]{y \in {\mathcal N}, \; x \nameeq y}
		  {\outputp{x}{v} \downarrow_{\mathcal N} x}
\infrule[Par-barb]{\mbox{$P\downarrow_{\mathcal N} x$ or $Q\downarrow_{\mathcal N} x$}}
		  {\binpar{P}{Q} \downarrow_{\mathcal N} x}

We write $P \Downarrow_{\mathcal N} x$ if there is $Q$ such that 
$P \wred Q$ and $Q \downarrow_{\mathcal N} x$.
\end{definition}

\begin{definition}
%\label{def.bbisim}
An  ${\mathcal N}$-\emph{barbed bisimulation} over a set of names, ${\mathcal N}$, is a symmetric binary relation 
${\mathcal S}_{\mathcal N}$ between agents such that $P\rel{S}_{\mathcal N}Q$ implies:
\begin{enumerate}
\item If $P \red P'$ then $Q \wred Q'$ and $P'\rel{S}_{\mathcal N} Q'$.
\item If $P\downarrow_{\mathcal N} x$, then $Q\Downarrow_{\mathcal N} x$.
\end{enumerate}
$P$ is ${\mathcal N}$-barbed bisimilar to $Q$, written
$P \wbbisim_{\mathcal N} Q$, if $P \rel{S}_{\mathcal N} Q$ for some ${\mathcal N}$-barbed bisimulation ${\mathcal S}_{\mathcal N}$.
\end{definition}

$\mathcal{R} \subseteq \pi \times \pi$

$P \mathcal{R} Q => \forall P'. P \red P' \Rightarrow \exists Q'. Q \red Q', P' \mathcal{R} Q'$

$P \vdash x \Rightarrow Q \vdash x$

\begin{mathpar}
  \inferrule*[lab=Out-barb]{x \nameeq y}{{y}!\langle{Q}\rangle \vdash x}
  \and
  \inferrule*[lab=Par-barb]{\mbox{$P\vdash x$ or $Q\vdash x$}}{\binpar{P}{Q} \vdash x}
\end{mathpar}

\subsubsection{Contexts}

One of the principle advantages of computational calculi like the
$\pi$-calculus is a well-defined notion of context,
contextual-equivalence and a correlation between
contextual-equivalence and notions of bisimulation. The notion of
context allows the decomposition of a process into (sub-)process and
its syntactic environment, its context. Thus, a context may be
thought of as a process with a ``hole'' (written $\Box$) in it. The
application of a context $M$ to a process $P$, written $M[P]$, is
tantamount to filling the hole in $M$ with $P$. In this paper we do
not need the full weight of this theory, but do make use of the notion
of context in the proof the main theorem. 

\begin{mathpar}
  \inferrule* [lab=summation] {} {{M_{M},M_{N}} \bc \Box \;|\; x.M_{A} \;|\; M_{M}+M_{N}}
  \and
  \inferrule* [lab=agent] {} {{M_{A}} \bc (\vec{x})M_{P} \;| \; \clift{P_0,\ldots,M_{P},\ldots,P_N}}
  \and \\
  \inferrule* [lab=process] {} {{M_{P}} \bc M_{N} \;| \;P|M_{P} }
\end{mathpar} 

\begin{mathpar}
  \inferrule* [lab=sychronization] {} {M_{N} \bc \Box \;|\; x?M_{F} \;|\; x!M_{C}}
  \and
  \inferrule* [lab=abstraction] {} {{M_{F}} \bc (x)M_{P} }
  \and
  \inferrule* [lab=concretion] {} {{M_{C}} \bc \langle M_{P} \rangle }
  \and \\
  \inferrule* [lab=process] {} {{M_{P}} \bc M_{N} \;| \;P|M_{P} }
\end{mathpar}

\begin{definition}[contextual application] Given a context $M$, and
  process $P$, we define the \emph{contextual application}, $M[P] :=
  M\{P/\Box\}$. That is, the contextual application of M to P is the
  substitution of $P$ for $\Box$ in $M$.
\end{definition}

$\meaningof{-} : L \to \mathcal{P}(\pi)$

\begin{mathpar}
  \inferrule* [lab=collection] {} {\meaningof{true} = \pi, \and \meaningof{~E} = \pi \setminus \meaningof{E}, \and \meaningof{E_{1} \& E_{2}} = \meaningof{E_{1}} \cap \meaningof{E_{2}}}
\end{mathpar}

\begin{mathpar}
  \inferrule* [lab=structure] {} {\meaningof{0} = \{ P \in \pi | P \equiv 0 \}, \and \\ \meaningof{E_1 | E_2} = \{ P \in \pi | P \equiv P_{1} | P_{2}, P_{1} \in \meaningof{E_{1}}, P_{2} \in \meaningof{E_2}\} }
\end{mathpar}

\begin{mathpar}
 \inferrule* [lab=behavior] {} {\meaningof{\langle a?b \rangle E} = \{ P \in \pi | P \equiv Q | u?(y)P', \\ \and \\\\ \and \\ \;\;\; u \in \meaningof{a}, \forall z.P'\{z/y\} \in \meaningof{E\{z/b\}}\}, \and \\ \meaningof{a!E} = \{ P \in \pi | P \equiv Q | x!\langle P' \rangle, x \in \meaningof{a} P' \in \meaningof{E}\} }
\end{mathpar}

\begin{mathpar}
 \inferrule* [lab=nominal] {} {\meaningof{\quotep{E}} = \{ \quotep{P} \in \quotep{\pi} | P \in \meaningof{E} \}, \and \meaningof{\quotep{P}} = \{ \quotep{Q} \in \quotep{\pi} | P \equiv Q \} \and \\ \meaningof{@\quotep{E}} = \{ P \in \pi | P \equiv @x, x \in \meaningof{E} \}}
\end{mathpar}

\begin{eqnarray*}
  \\
  \meaningof{-} : TS \to ST
\end{eqnarray*}

\begin{eqnarray*}
  \\
  L : TS \to ST
\end{eqnarray*}

\begin{eqnarray*}
  \\
  P \models E \iff P \in \meaningof{E}
\end{eqnarray*}

\begin{eqnarray*}
  P \approx_{L} Q \iff \forall E \in L. P \models E \iff Q \models E
\end{eqnarray*}

\begin{eqnarray*}
  P \approx_{K} Q
\end{eqnarray*}

\begin{eqnarray*}
  P \approx Q
\end{eqnarray*}

$\approx_{K} = \approx = \approx_{L}$

\subsubsection{Contextual duality}

Note that contexts extend the quotation operation to a family of
operations from processes to names. Given a context, $M$, we can
define a \emph{nominal context}, $\quotep{M}$ by $\quotep{M}[P] :=
\quotep{M[P]}$. To foreshadow what is to come we observe that these
operations enjoy a duality with processes very much like the duality
between vectors and maps from vectors to scalars.

Further, because the calculus is essentially higher-order, we have a
correspondence between contexts and processes. More specifically,
given a name $x$ and a context $M$ we can construct $M^{*}_{x}$ such
that 

\begin{mathpar}
  M^{*}_{x} | \lift{x}{P} \red M[P]
\end{mathpar}

namely,

\begin{mathpar}
  M^{*}_{x} := x?(u).M[\dropn{u}]
\end{mathpar}

The dependence of $M^{*}_{x}$ on a name makes it an abstraction, 

\begin{mathpar}
  M^{*} := (x)x?(u).M[\dropn{u}]
\end{mathpar}

\subsection{Additional notation}

It will sometimes be convenient to denote the process a name
quotes. We already have the notation $x = \quotep{P}$, but it will be
convenient to introduce an alternate notation, $\procn{x}$, when we
want to emphasize the connection to the use of the name. Note that, by
virtue of name equivalence, $\quotep{\procn{x}} \nameeq x$; so, the
notation is consistent with previous definitions.

Further, because names have structure it is possible to effect
substitutions on the basis of that structure. This means we need to
upgrade our notation for substitutions, which we accomplish by
adapting comprehension notation. Thus,

\begin{mathpar}
  P\{ y / x : x \in S \}
\end{mathpar}

is interpreted to mean the process derived from P by replacing (in a
capture-avoiding manner) each occurrence of $x$ in $S$ by $y$. For example,

\begin{mathpar}
  P\{ \quotep{\procn{x}|\procn{x}} / x : x \in \freenames{P} \}
\end{mathpar}

will replace each (occurrence) of a free name $x$ in $P$ by
$\quotep{\procn{x}|\procn{x}}$.

Also, we will avail ourselves of the notation $x^{L}$ and $x^{R}$ to
denote injections of a name into disjoint copies of the name
space. There are numerous ways to accomplish this. One example can be
found in \cite{MeredithR05}. This notation overloads to vectors of
names: $\vec{x}^{\pi} := (x_{i}^{\pi} \; : \; 0 \leq i < |\vec{x}| )$ where $\pi \in \{L,R\}$.

We also use $P^{\Box} := P|\Box$.

In \cite{MeredithR05} an interpretation of the new operator is
given. It turns out that there are several possible interpretations
all enjoying the requisite algebraic properties of the operator (see
\cite{milner91polyadicpi}). We will therefore make liberal use of
$(\nu\; \vec{x})P$.

% subsection the_syntax_and_semantics_of_the_notation_system (end)   

\input{qm2pi.qmops} 

\input{qm2pi.sterngerlach} 

\input{qm2pi.metric} 

% section concurrent_process_calculi (end)

%\input{qm2pi.proofsketch}

% section proof sketch (end)

%\input{qm2pi.slviaknots} 

% section spatial logic via knots (end)

\input{qm2pi.conclusion}

% section conclusion (end)

%\input{qm2pi.dtcodes} 

% section wiring algorithm (end)

\input{qm2pi.ack} 

% section acknowledgments (end)

\newpage


\bibliographystyle{plain}   
\bibliography{../../biblios/main.bib}

\input{qm2pi.rhodetails}

\end{document}

 

% section concurrent_process_calculi (end)

%\documentclass[12pt]{llncs}
%\documentclass{jktr}

\usepackage[pdftex]{hyperref}                   
\usepackage {listings}
\usepackage {mathpartir}
\usepackage{bcprules}
%\usepackage{listings}
                       
\usepackage{graphicx} 
%\usepackage[margins=2.5cm,nohead,nofoot]{geometry}
%\usepackage{geometry}
\usepackage{amsfonts}
\usepackage{amstext}
\usepackage{latexsym}
\usepackage{amssymb}
\usepackage{color}


%\include{myPreamble}
\include{qm2pi.local} 

%\ifpdf
%\usepackage[pdftex]{graphicx}
%\else
%\usepackage{graphicx}
%\fi

 % \ifpdf
%  \usepackage{pdfsync}
%  \if


%\title{Brief Article}
%\author{David F. Snyder}
%\author{L.G. Meredith}

%\address{Dept. of Math., Texas State University--San Marcos, San Marcos, TX 78666}
       
\pagestyle{empty}


\begin{document}

\lstset{language=[Objective]Caml,frame=shadowbox}

\input{qm2pi.front}

% section front matter (end)

\input{qm2pi.intro} 
 
% section introduction (end)

% \input{qm2pi.knotations} 

% section notation (end)

\input{qm2pi.process.calculi} 

% section concurrent_process_calculi_and_spatial_logics_ (end)
    
%\input{qm2pi.knots2pi} 

%\input{qm2pi.trefoil} 

%\input{qm2pi.mainthm} 

% subsection basic_interpretation (end)

%\input{qm2pi.rho.presentation} 
\subsection{The syntax and semantics of the notation system}\label{sub:the_syntax_and_semantics_of_the_notation_system} % (fold)

We now summarize a technical presentation of the calculus that
embodies our theory of dynamics. The typical presentation of such a
calculus follows the style of giving generators and relations on
them. The grammar, below, describing term constructors, freely
generates the set of processes, $\Proc$. This set is then quotiented
by a relation known as structural congruence and it is over this set
that the notion of dynamics is expressed. This presentation is
essentially that of \cite{MeredithR05} with the addition of
polyadicity and summation. For readability we have relegated some of
the technical subtleties to an appendix.

\subsubsection{Process grammar}\label{subsub:process_grammar}

\begin{mathpar}
  \inferrule* [lab=synchronization] {} {{M} \bc \pzero \;|\; x?F \;|\; x!C }
  \and
  \inferrule* [lab=abstraction] {} {{F} \bc (x)P}
  \and
  \inferrule* [lab=concretion] {} {{C} \bc \langle Q \rangle}
  \and
  \inferrule* [lab=process] {} {{P,Q} \bc M \;| \;P|Q \;|\; @{x}}
  \and
  \inferrule* [lab=name] {} {{x} \bc \quotep{P}}
\end{mathpar} 

Note that $\vec{x}$ (resp. $\vec{P}$) denotes a vector of names
(resp. processes) of length $|\vec{x}|$ (resp. $|\vec{P}|$). We adopt
the following useful abbreviations.

\begin{mathpar}
   x?(\vec{y}).P := x.(\vec{y})P \and  x\clift{\vec{P}} := x.\clift{\vec{P}}
   \and x!(y) := \lift{x}{\dropn{y}}
   \and \Pi_{i=0}^{n-1}P_i := P_0 | \ldots | P_{n-1}
\end{mathpar}

\subsubsection{Structural congruence}

\paragraph{Free and bound names and alpha-equivalence.} At the
core of structural equivalence is alpha-equivalence which identifies
process that are the same up to a change of variable. Formally, we
recognize the distinction between free and bound names. The free names
of a process, $\freenames{P}$, may be calculated recursively as
follows:

\begin{mathpar}
\freenames{\pzero} := \emptyset
  \and \\
  \freenames{x?(y).P} := \{ x \} \cup (\freenames{P} \setminus \{ y \})
  \and 
  \freenames{x!\langle P \rangle} := \{ x \} \cup \{ P \} 
  \and \\
  \freenames{P|Q} := \freenames{P} \cup \freenames{Q}
  \and \\
  \freenames{@{x}} := \{ x \}
\end{mathpar}

$\pi$
$\quotep{\pi}$

$\freenames{-} : \pi \to \mathcal{P}(\quotep{\pi})$

\begin{eqnarray*}
  \freenames{\pzero} & := & \emptyset \\
  \freenames{x?(y).P} & := & \{ x \} \cup (\freenames{P} \setminus \{ y \}) \\
  \freenames{x!\langle P \rangle} & := & \{ x \} \cup \{ P \} \\
  \freenames{P|Q} & := & \freenames{P} \cup \freenames{Q} \\
  \freenames{\dropn{x}} & := & \{ x \}
\end{eqnarray*}

The bound names of a process, $\boundnames{P}$, are those names occurring in $P$
that are not free. For example, in $x?(y).0$, the name $x$ is free, while $y$ is bound.

\begin{mathpar}
  \inferrule* [lab=monoidal-laws] {} { P|Q \equiv Q|P \and P|0 \equiv P \and P|(Q|R) \equiv (P|Q)|R }
\end{mathpar}

\begin{mathpar}
  \inferrule* [lab=alpha-equivalence] {} { (x)P \equiv (y)P\{y/x\} \and y \not\in \freenames{P} }
\end{mathpar}

\begin{definition}
Then two processes, $P,Q$, are alpha-equivalent if $P = Q\{\vec{y}/\vec{x}\}$ for
some $\vec{x} \in \boundnames{Q},\vec{y} \in \boundnames{P}$, where $Q\{\vec{y}/\vec{x}\}$
denotes the capture-avoiding substitution of $\vec{y}$ for $\vec{x}$ in $Q$.
\end{definition}

\begin{definition}
  The {\em structural congruence} \cite{SangiorgiWalker} , $\equiv$,
  between processes is the least congruence containing
  alpha-equivalence, satisfying the abelian monoid laws
  (associativity, commutativity and $\pzero$ as identity) for parallel
  composition $|$ and for summation $+$.
\end{definition}

\subsection{Name equivalence}

We take name equivalence, written $\nameeq$, to be the smallest
equivalence relation generated by the following rules.

\begin{mathpar}
\inferrule*[lab=Quote-drop]
{ }
{ \quotep{@{x}} \nameeq x }

\inferrule*[lab=Struct-equiv]
{ P \scong Q }
{ \quotep{P} \nameeq \quotep{Q} }
\end{mathpar}

The astute reader will have noticed that the mutual recursion of names
and processes imposes a mutual recursion on alpha-equivalence and
structural equivalence via name-equivalence. Fortunately, all of this
works out pleasantly and we may calculate in the natural way, free of
concern. The reader interested in the details is referred to the
appendix \ref{appendix:rho_details}.

\subsection{Substitution}

We use $\Proc$ for the set of processes, $\QProc$ for the set of
names, and $\id{\{}\vec{y} / \vec{x} \id{\}}$ to denote partial maps,
$s : \QProc \rightarrow \QProc$. A map, $s$ lifts, uniquely, to a map
on process terms, $\widehat{s} : \Proc \rightarrow \Proc$ by the
following equations.

\begin{mathpar}
  (0) \psubstp{Q}{P} := 0 \\
  (R \juxtap S) \psubstp{Q}{P}
  :=    
  (R)\psubstp{Q}{P} \juxtap (S) \psubstp{Q}{P} \\
  (x?(y).R) \psubstp{Q}{P}    
  :=    
  (x)\substp{Q}{P} (z)\concat( (R \psubstn{z}{y}) \psubstp{Q}{P} ) \\
  (\lift{x}{R}) \psubstp{Q}{P}  
  :=
  \lift{(x)\substp{Q}{P}}{ R \psubstp{Q}{P} } \\
%   (\dropn{x})  \psubstp{Q}{P}       
%   := 
%   \left\{ 
%     \begin{array}{ccc} 
%       \dropn{\quotep{Q}} & & x \nameeq \quotep{P} \\
%       \dropn{x} & & otherwise \\
%     \end{array}
%   \right. 
  (\dropn{x})  \psubstp{Q}{P}       
  := 
  \left\{ 
    \begin{array}{ccc} 
      Q & & x \nameeq \quotep{P} \\
      \dropn{x} & & otherwise \\
    \end{array}
  \right.
\end{mathpar}
 

where

\begin{eqnarray}
  (x)\id{\{} \lpquote Q \rpquote / \lpquote P \rpquote \id{\}}            = 
  \left\{ 
    \begin{array}{ccc}
      \lpquote Q \rpquote & & x \nameeq \lpquote P \rpquote \\
      x & & otherwise \\
    \end{array}
  \right. \nonumber
\end{eqnarray}

and $z$ is chosen distinct from $\quotep{P}$, $\quotep{Q}$, the free
names in $Q$, and all the names in $R$. Our $\alpha$-equivalence will
be built in the standard way from this substitution.

\begin{remark}\label{rem:no_self_referential_names}
  One consequence of these definitions is that $\forall P. \quotep{P}
  \not\in \freenames{P}$.
\end{remark}

\subsection{ Dynamic quote: an example }

Anticipating something of what's to come, consider applying the
substitution, $\widehat{\id{\{}u / z \id{\}}}$, to the following pair
of processes, $\lift{w}{y!(z)}$ and $w[ \lpquote y!(z) \rpquote ]$.

\begin{eqnarray}
	\lift{w}{y!(z)}\widehat{\id{\{}u / z \id{\}}}
		& = &
		\lift{w}{y!(u)} \nonumber\\
	w[ \lpquote y!(z) \rpquote ] \widehat{ \id{\{}u / z \id{\}} }
		& = &
		w[ \lpquote y!(z) \rpquote ] \nonumber
\end{eqnarray}

Because the body of the process between quotes is impervious to
substitution, we get radically different answers. In fact, by
examining the first process in an input context,
e.g. $x?(z).\lift{w}{y!(z)}$, we see that the process under the lift
operator may be shaped by prefixed inputs binding a name inside it. In
this sense, the lift operator will be seen as a way to dynamically
construct processes before reifying them as names.

Finally equipped with these standard features we can present the
dynamics of the calculus.

\subsubsection{Operational semantics} 

Finally, we introduce the computational dynamics. What marks these
algebras as distinct from other more traditionally studied algebraic
structures, e.g. vector spaces or polynomial rings, is the manner in
which dynamics is captured. In traditional structures, dynamics is typically
expressed through morphisms between such structures, as in linear maps
between vector spaces or morphisms between rings. In algebras
associated with the semantics of computation, the dynamics is
expressed as part of the algebraic structure itself, through a
reduction reduction relation typically denoted by $\red$. Below, we
give a recursive presentation of this relation for the calculus used
in the encoding.

$\red \subseteq \pi \times \pi$
$\red : \pi \to \mathcal{P}(\pi)$

\begin{mathpar}
  \inferrule* [lab=Comm] { \textsf{match}( x_{src}, x_{trgt} ) } { x_{trgt}?(y)P \; | \; x_{src}!\langle {Q} \rangle \red P\{\quotep{Q}/y}\} }
  \and \\
  \inferrule* [lab=Par] {{P} \red {P}'} {{{P} | {Q}} \red {{P}' | {Q}}}
  \and
  \inferrule* [lab=Equiv]{{{P} \scong {P}'} \andalso {{P}' \red {Q}'} \andalso {{Q}' \scong {Q}}}{{P} \red {Q}}
\end{mathpar}

\begin{eqnarray*}
  match_{\equiv} (\quotep{P},\quotep{Q}) & := & P \equiv Q \\
  match_{\dagger}(\quotep{P},\quotep{Q}) & := & \forall R. P|Q \red^{*} R => R \red^{*} 0 \\
  match_{K}(\quotep{P},\quotep{Q}) & := & K \mbox{ for some context } K
\end{eqnarray*}

$u?(x)P | u!\langle Q \rangle \red P\{\quotep{Q}/x\}$

%We write $\wred$ for $\red^*$, and $P\red$ if $\exists Q $ such that $ P \red Q$.
We write $P\red$ if $\exists Q $ such that $ P \red Q$ and $P\not\red$, otherwise.

\section{Replication}

As mentioned before, it is known that replication (and hence
recursion) can be implemented in a higher-order process algebra
\cite{SangiorgiWalker}. As our first example of calculation with the
machinery thus far presented we give the construction explicitly in
the {\rhoc}.

\begin{eqnarray}
	D_{x} & := & \prefix{x}{y}{(\binpar{\outputp{x}{y}}{@{y}})} \nonumber\\
	\bangp_{x}{P} & := & \binpar{{x}!\langle{\binpar{D_{x}}{P}}\rangle}{D_{x}} \nonumber
\end{eqnarray}

\begin{eqnarray}
	\bangp_{x}{P} & & \nonumber\\
	=
	& {x}!\langle{(\prefix{x}{y}{(\outputp{x}{y} | @{y})) | P}}\rangle 
	      | \prefix{x}{y}{(\outputp{x}{y} | @{y})} & \nonumber\\
	\red
	& (\outputp{x}{y} | @{y})\substn{\quotep{(\prefix{x}{y}{(@{y} | \outputp{x}{y})) | P}}}{y} & \nonumber\\
	=
	& \outputp{x}{\quotep{(\prefix{x}{y}{(\outputp{x}{y} | @{y})) | P}}}
	  | {(\prefix{x}{y}{(\outputp{x}{y} | @{y})) | P}} & \nonumber\\
	\red
	& \ldots & \nonumber\\
	\red^*
	& P | P | \ldots & \nonumber
\end{eqnarray}

Of course, this encoding, as an implementation, runs away, unfolding
$\bangp{P}$ eagerly. A lazier and more implementable replication
operator, restricted to input-guarded processes, may be obtained as follows.

\begin{eqnarray}
\bangp{\prefix{u}{v}{P}} 
	:= 
	\binpar{\lift{x}{\prefix{u}{v}{(\binpar{D(x)}{P})}}}{D(x)} \nonumber
\end{eqnarray}

\begin{remark}
  Note that the lazier definition still does not deal with summation
  or mixed summation (i.e. sums over input and output). The reader is
  invited to construct definitions of replication that deal with these
  features. 

  Further, the definitions are parameterized in a name, $x$. Can you,
  gentle reader, make a definition that eliminates this parameter and
  guarantees no accidental interaction between the replication
  machinery and the process being replicated -- i.e. no accidental
  sharing of names used by the process to get its work done and the
  name(s) used by the replication to effect copying. This latter
  revision of the definition of replication is crucial to obtaining
  the expected identity $!!P \sim !P$.
\end{remark}

\begin{remark}\label{rem:paradoxical_combinator}
  The reader familiar with the lambda calculus will have noticed the
  similarity between $D$ and the paradoxical combinator.

  [Ed. note: the existence of this seems to suggest we have to be more
  restrictive on the set of processes and names we admit if we are to
  support no-cloning.]
\end{remark}

\subsubsection{Bisimulation}

The computational dynamics gives rise to another kind of equivalence,
the equivalence of computational behavior. As previously mentioned
this is typically captured \emph{via} some form of bisimulation.

% The notion we use in this paper is weak barbed bisimulation
% \cite{milner91polyadicpi}.

The notion we use in this paper is derived from weak barbed
bisimulation \cite{milner91polyadicpi}. 

\begin{definition}
An \emph{observation relation}, $\downarrow_{\mathcal N}$, over a set
of names, $\mathcal N$, is the smallest relation satisfying the rules
below.

\infrule[Out-barb]{y \in {\mathcal N}, \; x \nameeq y}
		  {\outputp{x}{v} \downarrow_{\mathcal N} x}
\infrule[Par-barb]{\mbox{$P\downarrow_{\mathcal N} x$ or $Q\downarrow_{\mathcal N} x$}}
		  {\binpar{P}{Q} \downarrow_{\mathcal N} x}

We write $P \Downarrow_{\mathcal N} x$ if there is $Q$ such that 
$P \wred Q$ and $Q \downarrow_{\mathcal N} x$.
\end{definition}

\begin{definition}
%\label{def.bbisim}
An  ${\mathcal N}$-\emph{barbed bisimulation} over a set of names, ${\mathcal N}$, is a symmetric binary relation 
${\mathcal S}_{\mathcal N}$ between agents such that $P\rel{S}_{\mathcal N}Q$ implies:
\begin{enumerate}
\item If $P \red P'$ then $Q \wred Q'$ and $P'\rel{S}_{\mathcal N} Q'$.
\item If $P\downarrow_{\mathcal N} x$, then $Q\Downarrow_{\mathcal N} x$.
\end{enumerate}
$P$ is ${\mathcal N}$-barbed bisimilar to $Q$, written
$P \wbbisim_{\mathcal N} Q$, if $P \rel{S}_{\mathcal N} Q$ for some ${\mathcal N}$-barbed bisimulation ${\mathcal S}_{\mathcal N}$.
\end{definition}

$\mathcal{R} \subseteq \pi \times \pi$

$P \mathcal{R} Q => \forall P'. P \red P' \Rightarrow \exists Q'. Q \red Q', P' \mathcal{R} Q'$

$P \vdash x \Rightarrow Q \vdash x$

\begin{mathpar}
  \inferrule*[lab=Out-barb]{x \nameeq y}{{y}!\langle{Q}\rangle \vdash x}
  \and
  \inferrule*[lab=Par-barb]{\mbox{$P\vdash x$ or $Q\vdash x$}}{\binpar{P}{Q} \vdash x}
\end{mathpar}

\subsubsection{Contexts}

One of the principle advantages of computational calculi like the
$\pi$-calculus is a well-defined notion of context,
contextual-equivalence and a correlation between
contextual-equivalence and notions of bisimulation. The notion of
context allows the decomposition of a process into (sub-)process and
its syntactic environment, its context. Thus, a context may be
thought of as a process with a ``hole'' (written $\Box$) in it. The
application of a context $M$ to a process $P$, written $M[P]$, is
tantamount to filling the hole in $M$ with $P$. In this paper we do
not need the full weight of this theory, but do make use of the notion
of context in the proof the main theorem. 

\begin{mathpar}
  \inferrule* [lab=summation] {} {{M_{M},M_{N}} \bc \Box \;|\; x.M_{A} \;|\; M_{M}+M_{N}}
  \and
  \inferrule* [lab=agent] {} {{M_{A}} \bc (\vec{x})M_{P} \;| \; \clift{P_0,\ldots,M_{P},\ldots,P_N}}
  \and \\
  \inferrule* [lab=process] {} {{M_{P}} \bc M_{N} \;| \;P|M_{P} }
\end{mathpar} 

\begin{mathpar}
  \inferrule* [lab=sychronization] {} {M_{N} \bc \Box \;|\; x?M_{F} \;|\; x!M_{C}}
  \and
  \inferrule* [lab=abstraction] {} {{M_{F}} \bc (x)M_{P} }
  \and
  \inferrule* [lab=concretion] {} {{M_{C}} \bc \langle M_{P} \rangle }
  \and \\
  \inferrule* [lab=process] {} {{M_{P}} \bc M_{N} \;| \;P|M_{P} }
\end{mathpar}

\begin{definition}[contextual application] Given a context $M$, and
  process $P$, we define the \emph{contextual application}, $M[P] :=
  M\{P/\Box\}$. That is, the contextual application of M to P is the
  substitution of $P$ for $\Box$ in $M$.
\end{definition}

$\meaningof{-} : L \to \mathcal{P}(\pi)$

\begin{mathpar}
  \inferrule* [lab=collection] {} {\meaningof{true} = \pi, \and \meaningof{~E} = \pi \setminus \meaningof{E}, \and \meaningof{E_{1} \& E_{2}} = \meaningof{E_{1}} \cap \meaningof{E_{2}}}
\end{mathpar}

\begin{mathpar}
  \inferrule* [lab=structure] {} {\meaningof{0} = \{ P \in \pi | P \equiv 0 \}, \and \\ \meaningof{E_1 | E_2} = \{ P \in \pi | P \equiv P_{1} | P_{2}, P_{1} \in \meaningof{E_{1}}, P_{2} \in \meaningof{E_2}\} }
\end{mathpar}

\begin{mathpar}
 \inferrule* [lab=behavior] {} {\meaningof{\langle a?b \rangle E} = \{ P \in \pi | P \equiv Q | u?(y)P', \\ \and \\\\ \and \\ \;\;\; u \in \meaningof{a}, \forall z.P'\{z/y\} \in \meaningof{E\{z/b\}}\}, \and \\ \meaningof{a!E} = \{ P \in \pi | P \equiv Q | x!\langle P' \rangle, x \in \meaningof{a} P' \in \meaningof{E}\} }
\end{mathpar}

\begin{mathpar}
 \inferrule* [lab=nominal] {} {\meaningof{\quotep{E}} = \{ \quotep{P} \in \quotep{\pi} | P \in \meaningof{E} \}, \and \meaningof{\quotep{P}} = \{ \quotep{Q} \in \quotep{\pi} | P \equiv Q \} \and \\ \meaningof{@\quotep{E}} = \{ P \in \pi | P \equiv @x, x \in \meaningof{E} \}}
\end{mathpar}

\begin{eqnarray*}
  \\
  \meaningof{-} : TS \to ST
\end{eqnarray*}

\begin{eqnarray*}
  \\
  L : TS \to ST
\end{eqnarray*}

\begin{eqnarray*}
  \\
  P \models E \iff P \in \meaningof{E}
\end{eqnarray*}

\begin{eqnarray*}
  P \approx_{L} Q \iff \forall E \in L. P \models E \iff Q \models E
\end{eqnarray*}

\begin{eqnarray*}
  P \approx_{K} Q
\end{eqnarray*}

\begin{eqnarray*}
  P \approx Q
\end{eqnarray*}

$\approx_{K} = \approx = \approx_{L}$

\subsubsection{Contextual duality}

Note that contexts extend the quotation operation to a family of
operations from processes to names. Given a context, $M$, we can
define a \emph{nominal context}, $\quotep{M}$ by $\quotep{M}[P] :=
\quotep{M[P]}$. To foreshadow what is to come we observe that these
operations enjoy a duality with processes very much like the duality
between vectors and maps from vectors to scalars.

Further, because the calculus is essentially higher-order, we have a
correspondence between contexts and processes. More specifically,
given a name $x$ and a context $M$ we can construct $M^{*}_{x}$ such
that 

\begin{mathpar}
  M^{*}_{x} | \lift{x}{P} \red M[P]
\end{mathpar}

namely,

\begin{mathpar}
  M^{*}_{x} := x?(u).M[\dropn{u}]
\end{mathpar}

The dependence of $M^{*}_{x}$ on a name makes it an abstraction, 

\begin{mathpar}
  M^{*} := (x)x?(u).M[\dropn{u}]
\end{mathpar}

\subsection{Additional notation}

It will sometimes be convenient to denote the process a name
quotes. We already have the notation $x = \quotep{P}$, but it will be
convenient to introduce an alternate notation, $\procn{x}$, when we
want to emphasize the connection to the use of the name. Note that, by
virtue of name equivalence, $\quotep{\procn{x}} \nameeq x$; so, the
notation is consistent with previous definitions.

Further, because names have structure it is possible to effect
substitutions on the basis of that structure. This means we need to
upgrade our notation for substitutions, which we accomplish by
adapting comprehension notation. Thus,

\begin{mathpar}
  P\{ y / x : x \in S \}
\end{mathpar}

is interpreted to mean the process derived from P by replacing (in a
capture-avoiding manner) each occurrence of $x$ in $S$ by $y$. For example,

\begin{mathpar}
  P\{ \quotep{\procn{x}|\procn{x}} / x : x \in \freenames{P} \}
\end{mathpar}

will replace each (occurrence) of a free name $x$ in $P$ by
$\quotep{\procn{x}|\procn{x}}$.

Also, we will avail ourselves of the notation $x^{L}$ and $x^{R}$ to
denote injections of a name into disjoint copies of the name
space. There are numerous ways to accomplish this. One example can be
found in \cite{MeredithR05}. This notation overloads to vectors of
names: $\vec{x}^{\pi} := (x_{i}^{\pi} \; : \; 0 \leq i < |\vec{x}| )$ where $\pi \in \{L,R\}$.

We also use $P^{\Box} := P|\Box$.

In \cite{MeredithR05} an interpretation of the new operator is
given. It turns out that there are several possible interpretations
all enjoying the requisite algebraic properties of the operator (see
\cite{milner91polyadicpi}). We will therefore make liberal use of
$(\nu\; \vec{x})P$.

% subsection the_syntax_and_semantics_of_the_notation_system (end)   

\input{qm2pi.qmops} 

\input{qm2pi.sterngerlach} 

\input{qm2pi.metric} 

% section concurrent_process_calculi (end)

%\input{qm2pi.proofsketch}

% section proof sketch (end)

%\input{qm2pi.slviaknots} 

% section spatial logic via knots (end)

\input{qm2pi.conclusion}

% section conclusion (end)

%\input{qm2pi.dtcodes} 

% section wiring algorithm (end)

\input{qm2pi.ack} 

% section acknowledgments (end)

\newpage


\bibliographystyle{plain}   
\bibliography{../../biblios/main.bib}

\input{qm2pi.rhodetails}

\end{document}



% section proof sketch (end)

%\section{Unlikely characters: spatial logic for
  knots}\label{sub:characteristic_formulae} % (fold)

Associated to the mobile process calculi are a family of logics known
as the Hennessy-Milner logics. These logics typically enjoy a
semantics interpreting formulae as sets of processes that when
factored through the encoding outlined above allows an identification
of classes of knots with logical formulae. In the context of this
encoding the sub-family known as the spatial logics \cite{CairesC03}
\cite{CairesC04} \cite{Caires04} are of particular interest providing
several important features for expressing and reasoning about
properties (i.e. classes) of knots. We hint here at how this may be done.

%\begin{description}
%\item [structural connectives] 
\subsubsection{Structural connectives} The spatial logics enjoy
structural connectives corresponding, at the logical level, to the
parallel composition ($P | Q$) and new name ($(\nu \; x)P$)
connectives for processes. As illustrated in the examples below, these
connectives are extremely expressive given the shape of our encoding.
%\item [decideable satisfaction]

\subsubsection{Decideable satisfaction}
In \cite{Caires04} the satisfaction relation is shown to be decideable
for a rich class of processes. It further turns out that the image of
the our encoding is a proper subset of that class. This result
provides the basis for an algorithm by which to search for knots
enjoying a given property.
%\item [characteristic formulae]

\subsubsection{Characteristic formulae}
In the same paper \cite{Caires04} , Caires presents a means of calculating
characteristic formulae, selecting equivalence classes of processes
up to a pre--specified depth limit on the support set of names. Composed with our
encoding, this characteristic formula can be used to select
characteristic formulae for knots.
%\end{description}

\subsubsection{Spatial logic formulae}

The grammar below (segmented for comprehension) summarizes the syntax
of spatial logic formulae. We employ illustrative examples in the
sequel to provide an intuitive understanding of their meaning
referring the reader to \cite{Caires04} for a more detailed explication
of the semantics.

\begin{mathpar}
  \inferrule* [lab=boolean] {} {{A,B} \bc T \;|\; \neg A \;|\; A \wedge B \;|\; \eta = \eta'}
  \and
  \inferrule* [lab=spatial] {} {|\; \pzero \;|\; A | B \;|\; x \text{\textregistered} A \;|\; \forall x . A \;|\;  H x . A}
  \and
  \inferrule* [lab=behavioral] {} {|\; \alpha . A}
  \and 
  \inferrule* [lab=recursion] {} {|\; X(\vec{u}) \;|\; \mu X(\vec{u}) . A}
  \and
  \inferrule* [lab=action] {} {\alpha \bc \langle x?(\vec{y}) \rangle \;|\; \langle x!(\vec{y}) \rangle \;|\; \langle \tau \rangle}
  \and 
  \inferrule* [lab=name] {} {\eta \bc x \;|\; \tau}
\end{mathpar} 

% subsection characteristic_formulae (end)   	 

\subsection{Example formulae}\label{sub:example_formulae_} % (fold)

\subsubsection{Crossing as formula.}
% 
% \begin{align*}
%   \frac{d}{dx} \sin x &= \cos x 
%   & \frac{d}{dx} e^x &= e^x \\
%   \frac{d}{dx} \cos x &= - \sin x 
%   & \frac{d}{dx} \log x &= \frac{1}{x} \\
% \end{align*} 

\begin{align*}
 \mu C(x_{0},x_{1},y_{0},y_{1},u).&(\langle x_{0}?(z) \rangle(\langle u! \rangle\langle y_{1}!z \rangle C(x_{0},x_{1},y_{0},y_{1},u)) & \\
  & \wedge \langle y_{1}?(z) \rangle (\langle u! \rangle \langle x_{0}!z \rangle C(x_{0},x_{1},y_{0},y_{1},u)) & \\
  & \wedge \langle x_{1}?(z) \rangle (\langle u? \rangle \langle y_{0}!z \rangle C(x_{0},x_{1},y_{0},y_{1},u)) & \\
  & \wedge \langle y_{0}?(z) \rangle (\langle u? \rangle \langle x_{1}!z \rangle C(x_{0},x_{1},y_{0},y_{1},u))) &
\end{align*}

The lexicographical similarity between the shape of this formulae and
the shape of definition of the process representing a crossing reveals
the intuitive meaning of this formulae. It describes the capabilities
of a process that has the right to represent a crossing. For example
it picks out processes that may perform an input on the port $x_0$ in
its initial menu of capabilities. What differentiates the formula
from the process, however, is that the crossing process is the
smallest candidate to satisfy the formula. Infinitely many other
processes -- with internal behavior hidden behind this interface, so
to speak -- also satisfy this formula. Even this simple formula,
then, can be seen to open a new view onto knots, providing a
computational interpretation of \emph{virtual} knots.

Note that this formula is derived by hand. A similar formula can be
derived by employing Caires' calculation of characteristic formula
\cite{Caires04} to the process representing a crossing. In light of
this discussion, we let
$\meaningof{C}_{\phi}(x0,x1,y0,y1,u)$ denote a formula specifying the
dynamics we wish to capture of a crossing. To guarantee we preserve
the shape of the interface and minimal semantics we demand that
$\meaningof{C}_{\phi}(x0,x1,y0,y1,u) \Rightarrow
\textbf{C}(x0,x1,y0,y1,u)$ where $\textbf{C}(x0,x1,y0,y1,u)$ denotes
the formula above.
                            
\subsubsection{Crossing number constraints.}
The moral content of the context lemma (Lemma \ref{context}) is that the notion of
``locality'' in the Reidemeister moves is effectively captured by the
parallel composition operator of the process calculus. This intuition
extends through the logic. Given a formula,
$\meaningof{C}_{\phi}(x0,x1,y0,y1,u)$, we can use the structural
connectives to specify constraints on crossing numbers, such as at
least $n$ crossings, or exactly $n$ crossings.
\begin{mathpar}
  \inferrule* [lab=at-least-n] {} { K^{\geq n}_{\phi}(\vec{xs},\vec{ys}) := \Pi_{i=0}^{n-1} Hu . \meaningof{C}_{\phi}(xs_i,ys_i,u) | T }
  \and 
  \inferrule* [lab=exactly-n] {} { K^{= n}_{\phi}(\vec{xs},\vec{ys}) := \Pi_{i=0}^{n-1} Hu . \meaningof{C}_{\phi}(xs_i,ys_i,u) | \neg (\forall x_0,y_0,x_1,y_1,u . \meaningof{C}_{\phi}(x_0,y_0,x_1,y_1,u) | T) }
\end{mathpar}

To round out this section, recall that the encoding of an $n$-crossing
knot decomposes into a parallel composition of $n$ \emph{copies} of a
crossing process together with a wiring harness. To specify different
knot classes with the same crossing number amounts to specifying
logical constraints on the wiring harness. In the interest of space,
we defer examples to a forthcoming paper. Suffice it to say that both
the conditions ``alternating knot'' and ``contains the tangle
corresponding to 5/3'' are expressible. For example, it is possible to
calculate the characteristic formula of a process corresponding to the
tangle 5/3 and conjoin it into the classifying formula via the
composition connective of the logic.

Finally, we wish to observe that it is entirely within reason to
contemplate a more domain-specific version of spatial logic tailored
to the shape of processes in the image of the encoding. Such a
domain-specific logic would have a better claim to the title formal
language of knot properties.

% subsection example_formulae_ (end)

% section knots_as_processes (end) 

% section spatial logic via knots (end)

\section{Conclusions and future work}

\paragraph{Testing physical space}
You, gentle reader, may wonder why of all the theorems to be proved
given this set up we pick the one above. In some sense it's hardly
central to quantum mechanics. We see it as central in the sense that
it firmly establishes a notion of physical space arising from a notion
of the equivalence of behavior. Relating bisimulation to a metric is a
big step forward, but one is faced with interpreting the relationship
of that metric space to something more physical. Quantum mechanical
notions of ``physical'' space are still far from intuitive, but by
relating this idea of distance as testing to calculations that predict
physical circumstances we are making a not insignificant step forward
toward an understanding of the physical space we inhabit as
essentially dynamic.

\paragraph{Effectivity and simulation}
One of the observations we have yet to make is that the entire program
spelled out here is effective. We have built various interpreters for
the reflective calculus at work in this interpretation. In principle,
then, we can simulate quantum mechanics on a computer. The place where
the simulation may lose fidelity is the infinitely branching summation
for the annihilator.

In this connection i also want to point out that the evaluation style
calculation of the inner product puts the non-determinism of the
summation right at the heart of measurement. This suggests that
Milner's original reduction-based formulation of the dynamics of his
calculi in terms of sums was not just notationally suggestive of a
notion of measure-and-continue but captured some significant part of
the physics.

\paragraph{Quantum continuations}
In light of this last observation i want to point out that the
predominant account of quantum mechanics is missing a key aspect of a
truly compositional story of the physical situation. In a real lab,
when a measurement is made the observation can be made to feed into
another device that then makes another measurement conditioned on the
results of the first. This means that after the superposition was
collapsed the entire experimental set up remained in
superposition. While QM offers a means of writing this down it doesn't
quite line up well with the well-trodden formulation of computation
and continuation that we see so succinctly expressed in Milner's
calculi. This suggests that there might be advantages to this account
of dynamics waiting to be explored.

\paragraph{Quantum logic}
In this connection, we also note that by virtue of having the
Hennessy-Milner construction, we can pull the construction through the
interpretation of QM. This gives us a natural candidate for a quantum
logic that enjoys an extremely tight connection with it's domain of
interpretation, making the construction much less ad hoc (rather it is
the image of functor!).

\paragraph{Quantum probabiity}
i have questions about the basis of the interpretation of inner
product as probability amplitude. In particular, using which
axiomatization of probability theory does the notion of probability
amplitude earn the right to be so dubbed? In other words, where is the
proof that the operation for calculating a probability amplitude (and
then squaring) satisfies the axioms of what it means to calculate a
probability? Even if such a proof exists (i have yet to find it in the
literature), i wonder if it might not be possible to turn things on
their heads. Can we view the calculation of the probability amplitude
as an axiomatization of probability? If so, then the definition we
give for calculating probability amplitude may provide the basis for
an \emph{effective} theory of probability.

\paragraph{Quantum vs ``biological'' information}
Finally, i want to conclude with a more philosophical observation. At
a recent workshop in which QM was a predominant topic i noticed
something about quantum information. The speaker was giving a riveting
discussion of axiomatic QM and showing how properties of ``no
cloning'' and ``no deleting'' emerged as consequences of the
axiomatization. Theorems of this form are necessary to give us a sense
of confidence that our axioms characterize the physical theory. What
struck me, though, was that if quantum information is neither erasable
nor replicable it is markedly different from \emph{life}. Two of the
things we know about life is that

\begin{itemize}
  \item it ends;
  \item to gain some measure of persistence, to transcend it's
    finitude it is imminently copyable.
\end{itemize}

Both of these qualities are summarized succinctly in the aphorism: all
flesh is grass. For me these two kinds of ``information'' -- call them
quantum and biological -- are end points on a spectrum of strategies
for persistence. At one end, we have those curious entities that enjoy
uniqueness and permanence; at the other, we have those who in the face
of a certain end and an uncertain present make a go of passing
something on. To me one of the more remarkable aspects of the latter
strategy is that in the presence of noise (and certain features of
copying) we get a kind of dynamism, a chance for improvement against a
given persistent condition.

% subsection other_calculi_other_bisimulations_and_geometry_as_behavior (end)




% section conclusion (end)

%\documentclass[12pt]{llncs}
%\documentclass{jktr}

\usepackage[pdftex]{hyperref}                   
\usepackage {listings}
\usepackage {mathpartir}
\usepackage{bcprules}
%\usepackage{listings}
                       
\usepackage{graphicx} 
%\usepackage[margins=2.5cm,nohead,nofoot]{geometry}
%\usepackage{geometry}
\usepackage{amsfonts}
\usepackage{amstext}
\usepackage{latexsym}
\usepackage{amssymb}
\usepackage{color}


%\include{myPreamble}
\include{qm2pi.local} 

%\ifpdf
%\usepackage[pdftex]{graphicx}
%\else
%\usepackage{graphicx}
%\fi

 % \ifpdf
%  \usepackage{pdfsync}
%  \if


%\title{Brief Article}
%\author{David F. Snyder}
%\author{L.G. Meredith}

%\address{Dept. of Math., Texas State University--San Marcos, San Marcos, TX 78666}
       
\pagestyle{empty}


\begin{document}

\lstset{language=[Objective]Caml,frame=shadowbox}

\input{qm2pi.front}

% section front matter (end)

\input{qm2pi.intro} 
 
% section introduction (end)

% \input{qm2pi.knotations} 

% section notation (end)

\input{qm2pi.process.calculi} 

% section concurrent_process_calculi_and_spatial_logics_ (end)
    
%\input{qm2pi.knots2pi} 

%\input{qm2pi.trefoil} 

%\input{qm2pi.mainthm} 

% subsection basic_interpretation (end)

%\input{qm2pi.rho.presentation} 
\subsection{The syntax and semantics of the notation system}\label{sub:the_syntax_and_semantics_of_the_notation_system} % (fold)

We now summarize a technical presentation of the calculus that
embodies our theory of dynamics. The typical presentation of such a
calculus follows the style of giving generators and relations on
them. The grammar, below, describing term constructors, freely
generates the set of processes, $\Proc$. This set is then quotiented
by a relation known as structural congruence and it is over this set
that the notion of dynamics is expressed. This presentation is
essentially that of \cite{MeredithR05} with the addition of
polyadicity and summation. For readability we have relegated some of
the technical subtleties to an appendix.

\subsubsection{Process grammar}\label{subsub:process_grammar}

\begin{mathpar}
  \inferrule* [lab=synchronization] {} {{M} \bc \pzero \;|\; x?F \;|\; x!C }
  \and
  \inferrule* [lab=abstraction] {} {{F} \bc (x)P}
  \and
  \inferrule* [lab=concretion] {} {{C} \bc \langle Q \rangle}
  \and
  \inferrule* [lab=process] {} {{P,Q} \bc M \;| \;P|Q \;|\; @{x}}
  \and
  \inferrule* [lab=name] {} {{x} \bc \quotep{P}}
\end{mathpar} 

Note that $\vec{x}$ (resp. $\vec{P}$) denotes a vector of names
(resp. processes) of length $|\vec{x}|$ (resp. $|\vec{P}|$). We adopt
the following useful abbreviations.

\begin{mathpar}
   x?(\vec{y}).P := x.(\vec{y})P \and  x\clift{\vec{P}} := x.\clift{\vec{P}}
   \and x!(y) := \lift{x}{\dropn{y}}
   \and \Pi_{i=0}^{n-1}P_i := P_0 | \ldots | P_{n-1}
\end{mathpar}

\subsubsection{Structural congruence}

\paragraph{Free and bound names and alpha-equivalence.} At the
core of structural equivalence is alpha-equivalence which identifies
process that are the same up to a change of variable. Formally, we
recognize the distinction between free and bound names. The free names
of a process, $\freenames{P}$, may be calculated recursively as
follows:

\begin{mathpar}
\freenames{\pzero} := \emptyset
  \and \\
  \freenames{x?(y).P} := \{ x \} \cup (\freenames{P} \setminus \{ y \})
  \and 
  \freenames{x!\langle P \rangle} := \{ x \} \cup \{ P \} 
  \and \\
  \freenames{P|Q} := \freenames{P} \cup \freenames{Q}
  \and \\
  \freenames{@{x}} := \{ x \}
\end{mathpar}

$\pi$
$\quotep{\pi}$

$\freenames{-} : \pi \to \mathcal{P}(\quotep{\pi})$

\begin{eqnarray*}
  \freenames{\pzero} & := & \emptyset \\
  \freenames{x?(y).P} & := & \{ x \} \cup (\freenames{P} \setminus \{ y \}) \\
  \freenames{x!\langle P \rangle} & := & \{ x \} \cup \{ P \} \\
  \freenames{P|Q} & := & \freenames{P} \cup \freenames{Q} \\
  \freenames{\dropn{x}} & := & \{ x \}
\end{eqnarray*}

The bound names of a process, $\boundnames{P}$, are those names occurring in $P$
that are not free. For example, in $x?(y).0$, the name $x$ is free, while $y$ is bound.

\begin{mathpar}
  \inferrule* [lab=monoidal-laws] {} { P|Q \equiv Q|P \and P|0 \equiv P \and P|(Q|R) \equiv (P|Q)|R }
\end{mathpar}

\begin{mathpar}
  \inferrule* [lab=alpha-equivalence] {} { (x)P \equiv (y)P\{y/x\} \and y \not\in \freenames{P} }
\end{mathpar}

\begin{definition}
Then two processes, $P,Q$, are alpha-equivalent if $P = Q\{\vec{y}/\vec{x}\}$ for
some $\vec{x} \in \boundnames{Q},\vec{y} \in \boundnames{P}$, where $Q\{\vec{y}/\vec{x}\}$
denotes the capture-avoiding substitution of $\vec{y}$ for $\vec{x}$ in $Q$.
\end{definition}

\begin{definition}
  The {\em structural congruence} \cite{SangiorgiWalker} , $\equiv$,
  between processes is the least congruence containing
  alpha-equivalence, satisfying the abelian monoid laws
  (associativity, commutativity and $\pzero$ as identity) for parallel
  composition $|$ and for summation $+$.
\end{definition}

\subsection{Name equivalence}

We take name equivalence, written $\nameeq$, to be the smallest
equivalence relation generated by the following rules.

\begin{mathpar}
\inferrule*[lab=Quote-drop]
{ }
{ \quotep{@{x}} \nameeq x }

\inferrule*[lab=Struct-equiv]
{ P \scong Q }
{ \quotep{P} \nameeq \quotep{Q} }
\end{mathpar}

The astute reader will have noticed that the mutual recursion of names
and processes imposes a mutual recursion on alpha-equivalence and
structural equivalence via name-equivalence. Fortunately, all of this
works out pleasantly and we may calculate in the natural way, free of
concern. The reader interested in the details is referred to the
appendix \ref{appendix:rho_details}.

\subsection{Substitution}

We use $\Proc$ for the set of processes, $\QProc$ for the set of
names, and $\id{\{}\vec{y} / \vec{x} \id{\}}$ to denote partial maps,
$s : \QProc \rightarrow \QProc$. A map, $s$ lifts, uniquely, to a map
on process terms, $\widehat{s} : \Proc \rightarrow \Proc$ by the
following equations.

\begin{mathpar}
  (0) \psubstp{Q}{P} := 0 \\
  (R \juxtap S) \psubstp{Q}{P}
  :=    
  (R)\psubstp{Q}{P} \juxtap (S) \psubstp{Q}{P} \\
  (x?(y).R) \psubstp{Q}{P}    
  :=    
  (x)\substp{Q}{P} (z)\concat( (R \psubstn{z}{y}) \psubstp{Q}{P} ) \\
  (\lift{x}{R}) \psubstp{Q}{P}  
  :=
  \lift{(x)\substp{Q}{P}}{ R \psubstp{Q}{P} } \\
%   (\dropn{x})  \psubstp{Q}{P}       
%   := 
%   \left\{ 
%     \begin{array}{ccc} 
%       \dropn{\quotep{Q}} & & x \nameeq \quotep{P} \\
%       \dropn{x} & & otherwise \\
%     \end{array}
%   \right. 
  (\dropn{x})  \psubstp{Q}{P}       
  := 
  \left\{ 
    \begin{array}{ccc} 
      Q & & x \nameeq \quotep{P} \\
      \dropn{x} & & otherwise \\
    \end{array}
  \right.
\end{mathpar}
 

where

\begin{eqnarray}
  (x)\id{\{} \lpquote Q \rpquote / \lpquote P \rpquote \id{\}}            = 
  \left\{ 
    \begin{array}{ccc}
      \lpquote Q \rpquote & & x \nameeq \lpquote P \rpquote \\
      x & & otherwise \\
    \end{array}
  \right. \nonumber
\end{eqnarray}

and $z$ is chosen distinct from $\quotep{P}$, $\quotep{Q}$, the free
names in $Q$, and all the names in $R$. Our $\alpha$-equivalence will
be built in the standard way from this substitution.

\begin{remark}\label{rem:no_self_referential_names}
  One consequence of these definitions is that $\forall P. \quotep{P}
  \not\in \freenames{P}$.
\end{remark}

\subsection{ Dynamic quote: an example }

Anticipating something of what's to come, consider applying the
substitution, $\widehat{\id{\{}u / z \id{\}}}$, to the following pair
of processes, $\lift{w}{y!(z)}$ and $w[ \lpquote y!(z) \rpquote ]$.

\begin{eqnarray}
	\lift{w}{y!(z)}\widehat{\id{\{}u / z \id{\}}}
		& = &
		\lift{w}{y!(u)} \nonumber\\
	w[ \lpquote y!(z) \rpquote ] \widehat{ \id{\{}u / z \id{\}} }
		& = &
		w[ \lpquote y!(z) \rpquote ] \nonumber
\end{eqnarray}

Because the body of the process between quotes is impervious to
substitution, we get radically different answers. In fact, by
examining the first process in an input context,
e.g. $x?(z).\lift{w}{y!(z)}$, we see that the process under the lift
operator may be shaped by prefixed inputs binding a name inside it. In
this sense, the lift operator will be seen as a way to dynamically
construct processes before reifying them as names.

Finally equipped with these standard features we can present the
dynamics of the calculus.

\subsubsection{Operational semantics} 

Finally, we introduce the computational dynamics. What marks these
algebras as distinct from other more traditionally studied algebraic
structures, e.g. vector spaces or polynomial rings, is the manner in
which dynamics is captured. In traditional structures, dynamics is typically
expressed through morphisms between such structures, as in linear maps
between vector spaces or morphisms between rings. In algebras
associated with the semantics of computation, the dynamics is
expressed as part of the algebraic structure itself, through a
reduction reduction relation typically denoted by $\red$. Below, we
give a recursive presentation of this relation for the calculus used
in the encoding.

$\red \subseteq \pi \times \pi$
$\red : \pi \to \mathcal{P}(\pi)$

\begin{mathpar}
  \inferrule* [lab=Comm] { \textsf{match}( x_{src}, x_{trgt} ) } { x_{trgt}?(y)P \; | \; x_{src}!\langle {Q} \rangle \red P\{\quotep{Q}/y}\} }
  \and \\
  \inferrule* [lab=Par] {{P} \red {P}'} {{{P} | {Q}} \red {{P}' | {Q}}}
  \and
  \inferrule* [lab=Equiv]{{{P} \scong {P}'} \andalso {{P}' \red {Q}'} \andalso {{Q}' \scong {Q}}}{{P} \red {Q}}
\end{mathpar}

\begin{eqnarray*}
  match_{\equiv} (\quotep{P},\quotep{Q}) & := & P \equiv Q \\
  match_{\dagger}(\quotep{P},\quotep{Q}) & := & \forall R. P|Q \red^{*} R => R \red^{*} 0 \\
  match_{K}(\quotep{P},\quotep{Q}) & := & K \mbox{ for some context } K
\end{eqnarray*}

$u?(x)P | u!\langle Q \rangle \red P\{\quotep{Q}/x\}$

%We write $\wred$ for $\red^*$, and $P\red$ if $\exists Q $ such that $ P \red Q$.
We write $P\red$ if $\exists Q $ such that $ P \red Q$ and $P\not\red$, otherwise.

\section{Replication}

As mentioned before, it is known that replication (and hence
recursion) can be implemented in a higher-order process algebra
\cite{SangiorgiWalker}. As our first example of calculation with the
machinery thus far presented we give the construction explicitly in
the {\rhoc}.

\begin{eqnarray}
	D_{x} & := & \prefix{x}{y}{(\binpar{\outputp{x}{y}}{@{y}})} \nonumber\\
	\bangp_{x}{P} & := & \binpar{{x}!\langle{\binpar{D_{x}}{P}}\rangle}{D_{x}} \nonumber
\end{eqnarray}

\begin{eqnarray}
	\bangp_{x}{P} & & \nonumber\\
	=
	& {x}!\langle{(\prefix{x}{y}{(\outputp{x}{y} | @{y})) | P}}\rangle 
	      | \prefix{x}{y}{(\outputp{x}{y} | @{y})} & \nonumber\\
	\red
	& (\outputp{x}{y} | @{y})\substn{\quotep{(\prefix{x}{y}{(@{y} | \outputp{x}{y})) | P}}}{y} & \nonumber\\
	=
	& \outputp{x}{\quotep{(\prefix{x}{y}{(\outputp{x}{y} | @{y})) | P}}}
	  | {(\prefix{x}{y}{(\outputp{x}{y} | @{y})) | P}} & \nonumber\\
	\red
	& \ldots & \nonumber\\
	\red^*
	& P | P | \ldots & \nonumber
\end{eqnarray}

Of course, this encoding, as an implementation, runs away, unfolding
$\bangp{P}$ eagerly. A lazier and more implementable replication
operator, restricted to input-guarded processes, may be obtained as follows.

\begin{eqnarray}
\bangp{\prefix{u}{v}{P}} 
	:= 
	\binpar{\lift{x}{\prefix{u}{v}{(\binpar{D(x)}{P})}}}{D(x)} \nonumber
\end{eqnarray}

\begin{remark}
  Note that the lazier definition still does not deal with summation
  or mixed summation (i.e. sums over input and output). The reader is
  invited to construct definitions of replication that deal with these
  features. 

  Further, the definitions are parameterized in a name, $x$. Can you,
  gentle reader, make a definition that eliminates this parameter and
  guarantees no accidental interaction between the replication
  machinery and the process being replicated -- i.e. no accidental
  sharing of names used by the process to get its work done and the
  name(s) used by the replication to effect copying. This latter
  revision of the definition of replication is crucial to obtaining
  the expected identity $!!P \sim !P$.
\end{remark}

\begin{remark}\label{rem:paradoxical_combinator}
  The reader familiar with the lambda calculus will have noticed the
  similarity between $D$ and the paradoxical combinator.

  [Ed. note: the existence of this seems to suggest we have to be more
  restrictive on the set of processes and names we admit if we are to
  support no-cloning.]
\end{remark}

\subsubsection{Bisimulation}

The computational dynamics gives rise to another kind of equivalence,
the equivalence of computational behavior. As previously mentioned
this is typically captured \emph{via} some form of bisimulation.

% The notion we use in this paper is weak barbed bisimulation
% \cite{milner91polyadicpi}.

The notion we use in this paper is derived from weak barbed
bisimulation \cite{milner91polyadicpi}. 

\begin{definition}
An \emph{observation relation}, $\downarrow_{\mathcal N}$, over a set
of names, $\mathcal N$, is the smallest relation satisfying the rules
below.

\infrule[Out-barb]{y \in {\mathcal N}, \; x \nameeq y}
		  {\outputp{x}{v} \downarrow_{\mathcal N} x}
\infrule[Par-barb]{\mbox{$P\downarrow_{\mathcal N} x$ or $Q\downarrow_{\mathcal N} x$}}
		  {\binpar{P}{Q} \downarrow_{\mathcal N} x}

We write $P \Downarrow_{\mathcal N} x$ if there is $Q$ such that 
$P \wred Q$ and $Q \downarrow_{\mathcal N} x$.
\end{definition}

\begin{definition}
%\label{def.bbisim}
An  ${\mathcal N}$-\emph{barbed bisimulation} over a set of names, ${\mathcal N}$, is a symmetric binary relation 
${\mathcal S}_{\mathcal N}$ between agents such that $P\rel{S}_{\mathcal N}Q$ implies:
\begin{enumerate}
\item If $P \red P'$ then $Q \wred Q'$ and $P'\rel{S}_{\mathcal N} Q'$.
\item If $P\downarrow_{\mathcal N} x$, then $Q\Downarrow_{\mathcal N} x$.
\end{enumerate}
$P$ is ${\mathcal N}$-barbed bisimilar to $Q$, written
$P \wbbisim_{\mathcal N} Q$, if $P \rel{S}_{\mathcal N} Q$ for some ${\mathcal N}$-barbed bisimulation ${\mathcal S}_{\mathcal N}$.
\end{definition}

$\mathcal{R} \subseteq \pi \times \pi$

$P \mathcal{R} Q => \forall P'. P \red P' \Rightarrow \exists Q'. Q \red Q', P' \mathcal{R} Q'$

$P \vdash x \Rightarrow Q \vdash x$

\begin{mathpar}
  \inferrule*[lab=Out-barb]{x \nameeq y}{{y}!\langle{Q}\rangle \vdash x}
  \and
  \inferrule*[lab=Par-barb]{\mbox{$P\vdash x$ or $Q\vdash x$}}{\binpar{P}{Q} \vdash x}
\end{mathpar}

\subsubsection{Contexts}

One of the principle advantages of computational calculi like the
$\pi$-calculus is a well-defined notion of context,
contextual-equivalence and a correlation between
contextual-equivalence and notions of bisimulation. The notion of
context allows the decomposition of a process into (sub-)process and
its syntactic environment, its context. Thus, a context may be
thought of as a process with a ``hole'' (written $\Box$) in it. The
application of a context $M$ to a process $P$, written $M[P]$, is
tantamount to filling the hole in $M$ with $P$. In this paper we do
not need the full weight of this theory, but do make use of the notion
of context in the proof the main theorem. 

\begin{mathpar}
  \inferrule* [lab=summation] {} {{M_{M},M_{N}} \bc \Box \;|\; x.M_{A} \;|\; M_{M}+M_{N}}
  \and
  \inferrule* [lab=agent] {} {{M_{A}} \bc (\vec{x})M_{P} \;| \; \clift{P_0,\ldots,M_{P},\ldots,P_N}}
  \and \\
  \inferrule* [lab=process] {} {{M_{P}} \bc M_{N} \;| \;P|M_{P} }
\end{mathpar} 

\begin{mathpar}
  \inferrule* [lab=sychronization] {} {M_{N} \bc \Box \;|\; x?M_{F} \;|\; x!M_{C}}
  \and
  \inferrule* [lab=abstraction] {} {{M_{F}} \bc (x)M_{P} }
  \and
  \inferrule* [lab=concretion] {} {{M_{C}} \bc \langle M_{P} \rangle }
  \and \\
  \inferrule* [lab=process] {} {{M_{P}} \bc M_{N} \;| \;P|M_{P} }
\end{mathpar}

\begin{definition}[contextual application] Given a context $M$, and
  process $P$, we define the \emph{contextual application}, $M[P] :=
  M\{P/\Box\}$. That is, the contextual application of M to P is the
  substitution of $P$ for $\Box$ in $M$.
\end{definition}

$\meaningof{-} : L \to \mathcal{P}(\pi)$

\begin{mathpar}
  \inferrule* [lab=collection] {} {\meaningof{true} = \pi, \and \meaningof{~E} = \pi \setminus \meaningof{E}, \and \meaningof{E_{1} \& E_{2}} = \meaningof{E_{1}} \cap \meaningof{E_{2}}}
\end{mathpar}

\begin{mathpar}
  \inferrule* [lab=structure] {} {\meaningof{0} = \{ P \in \pi | P \equiv 0 \}, \and \\ \meaningof{E_1 | E_2} = \{ P \in \pi | P \equiv P_{1} | P_{2}, P_{1} \in \meaningof{E_{1}}, P_{2} \in \meaningof{E_2}\} }
\end{mathpar}

\begin{mathpar}
 \inferrule* [lab=behavior] {} {\meaningof{\langle a?b \rangle E} = \{ P \in \pi | P \equiv Q | u?(y)P', \\ \and \\\\ \and \\ \;\;\; u \in \meaningof{a}, \forall z.P'\{z/y\} \in \meaningof{E\{z/b\}}\}, \and \\ \meaningof{a!E} = \{ P \in \pi | P \equiv Q | x!\langle P' \rangle, x \in \meaningof{a} P' \in \meaningof{E}\} }
\end{mathpar}

\begin{mathpar}
 \inferrule* [lab=nominal] {} {\meaningof{\quotep{E}} = \{ \quotep{P} \in \quotep{\pi} | P \in \meaningof{E} \}, \and \meaningof{\quotep{P}} = \{ \quotep{Q} \in \quotep{\pi} | P \equiv Q \} \and \\ \meaningof{@\quotep{E}} = \{ P \in \pi | P \equiv @x, x \in \meaningof{E} \}}
\end{mathpar}

\begin{eqnarray*}
  \\
  \meaningof{-} : TS \to ST
\end{eqnarray*}

\begin{eqnarray*}
  \\
  L : TS \to ST
\end{eqnarray*}

\begin{eqnarray*}
  \\
  P \models E \iff P \in \meaningof{E}
\end{eqnarray*}

\begin{eqnarray*}
  P \approx_{L} Q \iff \forall E \in L. P \models E \iff Q \models E
\end{eqnarray*}

\begin{eqnarray*}
  P \approx_{K} Q
\end{eqnarray*}

\begin{eqnarray*}
  P \approx Q
\end{eqnarray*}

$\approx_{K} = \approx = \approx_{L}$

\subsubsection{Contextual duality}

Note that contexts extend the quotation operation to a family of
operations from processes to names. Given a context, $M$, we can
define a \emph{nominal context}, $\quotep{M}$ by $\quotep{M}[P] :=
\quotep{M[P]}$. To foreshadow what is to come we observe that these
operations enjoy a duality with processes very much like the duality
between vectors and maps from vectors to scalars.

Further, because the calculus is essentially higher-order, we have a
correspondence between contexts and processes. More specifically,
given a name $x$ and a context $M$ we can construct $M^{*}_{x}$ such
that 

\begin{mathpar}
  M^{*}_{x} | \lift{x}{P} \red M[P]
\end{mathpar}

namely,

\begin{mathpar}
  M^{*}_{x} := x?(u).M[\dropn{u}]
\end{mathpar}

The dependence of $M^{*}_{x}$ on a name makes it an abstraction, 

\begin{mathpar}
  M^{*} := (x)x?(u).M[\dropn{u}]
\end{mathpar}

\subsection{Additional notation}

It will sometimes be convenient to denote the process a name
quotes. We already have the notation $x = \quotep{P}$, but it will be
convenient to introduce an alternate notation, $\procn{x}$, when we
want to emphasize the connection to the use of the name. Note that, by
virtue of name equivalence, $\quotep{\procn{x}} \nameeq x$; so, the
notation is consistent with previous definitions.

Further, because names have structure it is possible to effect
substitutions on the basis of that structure. This means we need to
upgrade our notation for substitutions, which we accomplish by
adapting comprehension notation. Thus,

\begin{mathpar}
  P\{ y / x : x \in S \}
\end{mathpar}

is interpreted to mean the process derived from P by replacing (in a
capture-avoiding manner) each occurrence of $x$ in $S$ by $y$. For example,

\begin{mathpar}
  P\{ \quotep{\procn{x}|\procn{x}} / x : x \in \freenames{P} \}
\end{mathpar}

will replace each (occurrence) of a free name $x$ in $P$ by
$\quotep{\procn{x}|\procn{x}}$.

Also, we will avail ourselves of the notation $x^{L}$ and $x^{R}$ to
denote injections of a name into disjoint copies of the name
space. There are numerous ways to accomplish this. One example can be
found in \cite{MeredithR05}. This notation overloads to vectors of
names: $\vec{x}^{\pi} := (x_{i}^{\pi} \; : \; 0 \leq i < |\vec{x}| )$ where $\pi \in \{L,R\}$.

We also use $P^{\Box} := P|\Box$.

In \cite{MeredithR05} an interpretation of the new operator is
given. It turns out that there are several possible interpretations
all enjoying the requisite algebraic properties of the operator (see
\cite{milner91polyadicpi}). We will therefore make liberal use of
$(\nu\; \vec{x})P$.

% subsection the_syntax_and_semantics_of_the_notation_system (end)   

\input{qm2pi.qmops} 

\input{qm2pi.sterngerlach} 

\input{qm2pi.metric} 

% section concurrent_process_calculi (end)

%\input{qm2pi.proofsketch}

% section proof sketch (end)

%\input{qm2pi.slviaknots} 

% section spatial logic via knots (end)

\input{qm2pi.conclusion}

% section conclusion (end)

%\input{qm2pi.dtcodes} 

% section wiring algorithm (end)

\input{qm2pi.ack} 

% section acknowledgments (end)

\newpage


\bibliographystyle{plain}   
\bibliography{../../biblios/main.bib}

\input{qm2pi.rhodetails}

\end{document}

 

% section wiring algorithm (end)

\documentclass[12pt]{llncs}
%\documentclass{jktr}

\usepackage[pdftex]{hyperref}                   
\usepackage {listings}
\usepackage {mathpartir}
\usepackage{bcprules}
%\usepackage{listings}
                       
\usepackage{graphicx} 
%\usepackage[margins=2.5cm,nohead,nofoot]{geometry}
%\usepackage{geometry}
\usepackage{amsfonts}
\usepackage{amstext}
\usepackage{latexsym}
\usepackage{amssymb}
\usepackage{color}


%\include{myPreamble}
\include{qm2pi.local} 

%\ifpdf
%\usepackage[pdftex]{graphicx}
%\else
%\usepackage{graphicx}
%\fi

 % \ifpdf
%  \usepackage{pdfsync}
%  \if


%\title{Brief Article}
%\author{David F. Snyder}
%\author{L.G. Meredith}

%\address{Dept. of Math., Texas State University--San Marcos, San Marcos, TX 78666}
       
\pagestyle{empty}


\begin{document}

\lstset{language=[Objective]Caml,frame=shadowbox}

\input{qm2pi.front}

% section front matter (end)

\input{qm2pi.intro} 
 
% section introduction (end)

% \input{qm2pi.knotations} 

% section notation (end)

\input{qm2pi.process.calculi} 

% section concurrent_process_calculi_and_spatial_logics_ (end)
    
%\input{qm2pi.knots2pi} 

%\input{qm2pi.trefoil} 

%\input{qm2pi.mainthm} 

% subsection basic_interpretation (end)

%\input{qm2pi.rho.presentation} 
\subsection{The syntax and semantics of the notation system}\label{sub:the_syntax_and_semantics_of_the_notation_system} % (fold)

We now summarize a technical presentation of the calculus that
embodies our theory of dynamics. The typical presentation of such a
calculus follows the style of giving generators and relations on
them. The grammar, below, describing term constructors, freely
generates the set of processes, $\Proc$. This set is then quotiented
by a relation known as structural congruence and it is over this set
that the notion of dynamics is expressed. This presentation is
essentially that of \cite{MeredithR05} with the addition of
polyadicity and summation. For readability we have relegated some of
the technical subtleties to an appendix.

\subsubsection{Process grammar}\label{subsub:process_grammar}

\begin{mathpar}
  \inferrule* [lab=synchronization] {} {{M} \bc \pzero \;|\; x?F \;|\; x!C }
  \and
  \inferrule* [lab=abstraction] {} {{F} \bc (x)P}
  \and
  \inferrule* [lab=concretion] {} {{C} \bc \langle Q \rangle}
  \and
  \inferrule* [lab=process] {} {{P,Q} \bc M \;| \;P|Q \;|\; @{x}}
  \and
  \inferrule* [lab=name] {} {{x} \bc \quotep{P}}
\end{mathpar} 

Note that $\vec{x}$ (resp. $\vec{P}$) denotes a vector of names
(resp. processes) of length $|\vec{x}|$ (resp. $|\vec{P}|$). We adopt
the following useful abbreviations.

\begin{mathpar}
   x?(\vec{y}).P := x.(\vec{y})P \and  x\clift{\vec{P}} := x.\clift{\vec{P}}
   \and x!(y) := \lift{x}{\dropn{y}}
   \and \Pi_{i=0}^{n-1}P_i := P_0 | \ldots | P_{n-1}
\end{mathpar}

\subsubsection{Structural congruence}

\paragraph{Free and bound names and alpha-equivalence.} At the
core of structural equivalence is alpha-equivalence which identifies
process that are the same up to a change of variable. Formally, we
recognize the distinction between free and bound names. The free names
of a process, $\freenames{P}$, may be calculated recursively as
follows:

\begin{mathpar}
\freenames{\pzero} := \emptyset
  \and \\
  \freenames{x?(y).P} := \{ x \} \cup (\freenames{P} \setminus \{ y \})
  \and 
  \freenames{x!\langle P \rangle} := \{ x \} \cup \{ P \} 
  \and \\
  \freenames{P|Q} := \freenames{P} \cup \freenames{Q}
  \and \\
  \freenames{@{x}} := \{ x \}
\end{mathpar}

$\pi$
$\quotep{\pi}$

$\freenames{-} : \pi \to \mathcal{P}(\quotep{\pi})$

\begin{eqnarray*}
  \freenames{\pzero} & := & \emptyset \\
  \freenames{x?(y).P} & := & \{ x \} \cup (\freenames{P} \setminus \{ y \}) \\
  \freenames{x!\langle P \rangle} & := & \{ x \} \cup \{ P \} \\
  \freenames{P|Q} & := & \freenames{P} \cup \freenames{Q} \\
  \freenames{\dropn{x}} & := & \{ x \}
\end{eqnarray*}

The bound names of a process, $\boundnames{P}$, are those names occurring in $P$
that are not free. For example, in $x?(y).0$, the name $x$ is free, while $y$ is bound.

\begin{mathpar}
  \inferrule* [lab=monoidal-laws] {} { P|Q \equiv Q|P \and P|0 \equiv P \and P|(Q|R) \equiv (P|Q)|R }
\end{mathpar}

\begin{mathpar}
  \inferrule* [lab=alpha-equivalence] {} { (x)P \equiv (y)P\{y/x\} \and y \not\in \freenames{P} }
\end{mathpar}

\begin{definition}
Then two processes, $P,Q$, are alpha-equivalent if $P = Q\{\vec{y}/\vec{x}\}$ for
some $\vec{x} \in \boundnames{Q},\vec{y} \in \boundnames{P}$, where $Q\{\vec{y}/\vec{x}\}$
denotes the capture-avoiding substitution of $\vec{y}$ for $\vec{x}$ in $Q$.
\end{definition}

\begin{definition}
  The {\em structural congruence} \cite{SangiorgiWalker} , $\equiv$,
  between processes is the least congruence containing
  alpha-equivalence, satisfying the abelian monoid laws
  (associativity, commutativity and $\pzero$ as identity) for parallel
  composition $|$ and for summation $+$.
\end{definition}

\subsection{Name equivalence}

We take name equivalence, written $\nameeq$, to be the smallest
equivalence relation generated by the following rules.

\begin{mathpar}
\inferrule*[lab=Quote-drop]
{ }
{ \quotep{@{x}} \nameeq x }

\inferrule*[lab=Struct-equiv]
{ P \scong Q }
{ \quotep{P} \nameeq \quotep{Q} }
\end{mathpar}

The astute reader will have noticed that the mutual recursion of names
and processes imposes a mutual recursion on alpha-equivalence and
structural equivalence via name-equivalence. Fortunately, all of this
works out pleasantly and we may calculate in the natural way, free of
concern. The reader interested in the details is referred to the
appendix \ref{appendix:rho_details}.

\subsection{Substitution}

We use $\Proc$ for the set of processes, $\QProc$ for the set of
names, and $\id{\{}\vec{y} / \vec{x} \id{\}}$ to denote partial maps,
$s : \QProc \rightarrow \QProc$. A map, $s$ lifts, uniquely, to a map
on process terms, $\widehat{s} : \Proc \rightarrow \Proc$ by the
following equations.

\begin{mathpar}
  (0) \psubstp{Q}{P} := 0 \\
  (R \juxtap S) \psubstp{Q}{P}
  :=    
  (R)\psubstp{Q}{P} \juxtap (S) \psubstp{Q}{P} \\
  (x?(y).R) \psubstp{Q}{P}    
  :=    
  (x)\substp{Q}{P} (z)\concat( (R \psubstn{z}{y}) \psubstp{Q}{P} ) \\
  (\lift{x}{R}) \psubstp{Q}{P}  
  :=
  \lift{(x)\substp{Q}{P}}{ R \psubstp{Q}{P} } \\
%   (\dropn{x})  \psubstp{Q}{P}       
%   := 
%   \left\{ 
%     \begin{array}{ccc} 
%       \dropn{\quotep{Q}} & & x \nameeq \quotep{P} \\
%       \dropn{x} & & otherwise \\
%     \end{array}
%   \right. 
  (\dropn{x})  \psubstp{Q}{P}       
  := 
  \left\{ 
    \begin{array}{ccc} 
      Q & & x \nameeq \quotep{P} \\
      \dropn{x} & & otherwise \\
    \end{array}
  \right.
\end{mathpar}
 

where

\begin{eqnarray}
  (x)\id{\{} \lpquote Q \rpquote / \lpquote P \rpquote \id{\}}            = 
  \left\{ 
    \begin{array}{ccc}
      \lpquote Q \rpquote & & x \nameeq \lpquote P \rpquote \\
      x & & otherwise \\
    \end{array}
  \right. \nonumber
\end{eqnarray}

and $z$ is chosen distinct from $\quotep{P}$, $\quotep{Q}$, the free
names in $Q$, and all the names in $R$. Our $\alpha$-equivalence will
be built in the standard way from this substitution.

\begin{remark}\label{rem:no_self_referential_names}
  One consequence of these definitions is that $\forall P. \quotep{P}
  \not\in \freenames{P}$.
\end{remark}

\subsection{ Dynamic quote: an example }

Anticipating something of what's to come, consider applying the
substitution, $\widehat{\id{\{}u / z \id{\}}}$, to the following pair
of processes, $\lift{w}{y!(z)}$ and $w[ \lpquote y!(z) \rpquote ]$.

\begin{eqnarray}
	\lift{w}{y!(z)}\widehat{\id{\{}u / z \id{\}}}
		& = &
		\lift{w}{y!(u)} \nonumber\\
	w[ \lpquote y!(z) \rpquote ] \widehat{ \id{\{}u / z \id{\}} }
		& = &
		w[ \lpquote y!(z) \rpquote ] \nonumber
\end{eqnarray}

Because the body of the process between quotes is impervious to
substitution, we get radically different answers. In fact, by
examining the first process in an input context,
e.g. $x?(z).\lift{w}{y!(z)}$, we see that the process under the lift
operator may be shaped by prefixed inputs binding a name inside it. In
this sense, the lift operator will be seen as a way to dynamically
construct processes before reifying them as names.

Finally equipped with these standard features we can present the
dynamics of the calculus.

\subsubsection{Operational semantics} 

Finally, we introduce the computational dynamics. What marks these
algebras as distinct from other more traditionally studied algebraic
structures, e.g. vector spaces or polynomial rings, is the manner in
which dynamics is captured. In traditional structures, dynamics is typically
expressed through morphisms between such structures, as in linear maps
between vector spaces or morphisms between rings. In algebras
associated with the semantics of computation, the dynamics is
expressed as part of the algebraic structure itself, through a
reduction reduction relation typically denoted by $\red$. Below, we
give a recursive presentation of this relation for the calculus used
in the encoding.

$\red \subseteq \pi \times \pi$
$\red : \pi \to \mathcal{P}(\pi)$

\begin{mathpar}
  \inferrule* [lab=Comm] { \textsf{match}( x_{src}, x_{trgt} ) } { x_{trgt}?(y)P \; | \; x_{src}!\langle {Q} \rangle \red P\{\quotep{Q}/y}\} }
  \and \\
  \inferrule* [lab=Par] {{P} \red {P}'} {{{P} | {Q}} \red {{P}' | {Q}}}
  \and
  \inferrule* [lab=Equiv]{{{P} \scong {P}'} \andalso {{P}' \red {Q}'} \andalso {{Q}' \scong {Q}}}{{P} \red {Q}}
\end{mathpar}

\begin{eqnarray*}
  match_{\equiv} (\quotep{P},\quotep{Q}) & := & P \equiv Q \\
  match_{\dagger}(\quotep{P},\quotep{Q}) & := & \forall R. P|Q \red^{*} R => R \red^{*} 0 \\
  match_{K}(\quotep{P},\quotep{Q}) & := & K \mbox{ for some context } K
\end{eqnarray*}

$u?(x)P | u!\langle Q \rangle \red P\{\quotep{Q}/x\}$

%We write $\wred$ for $\red^*$, and $P\red$ if $\exists Q $ such that $ P \red Q$.
We write $P\red$ if $\exists Q $ such that $ P \red Q$ and $P\not\red$, otherwise.

\section{Replication}

As mentioned before, it is known that replication (and hence
recursion) can be implemented in a higher-order process algebra
\cite{SangiorgiWalker}. As our first example of calculation with the
machinery thus far presented we give the construction explicitly in
the {\rhoc}.

\begin{eqnarray}
	D_{x} & := & \prefix{x}{y}{(\binpar{\outputp{x}{y}}{@{y}})} \nonumber\\
	\bangp_{x}{P} & := & \binpar{{x}!\langle{\binpar{D_{x}}{P}}\rangle}{D_{x}} \nonumber
\end{eqnarray}

\begin{eqnarray}
	\bangp_{x}{P} & & \nonumber\\
	=
	& {x}!\langle{(\prefix{x}{y}{(\outputp{x}{y} | @{y})) | P}}\rangle 
	      | \prefix{x}{y}{(\outputp{x}{y} | @{y})} & \nonumber\\
	\red
	& (\outputp{x}{y} | @{y})\substn{\quotep{(\prefix{x}{y}{(@{y} | \outputp{x}{y})) | P}}}{y} & \nonumber\\
	=
	& \outputp{x}{\quotep{(\prefix{x}{y}{(\outputp{x}{y} | @{y})) | P}}}
	  | {(\prefix{x}{y}{(\outputp{x}{y} | @{y})) | P}} & \nonumber\\
	\red
	& \ldots & \nonumber\\
	\red^*
	& P | P | \ldots & \nonumber
\end{eqnarray}

Of course, this encoding, as an implementation, runs away, unfolding
$\bangp{P}$ eagerly. A lazier and more implementable replication
operator, restricted to input-guarded processes, may be obtained as follows.

\begin{eqnarray}
\bangp{\prefix{u}{v}{P}} 
	:= 
	\binpar{\lift{x}{\prefix{u}{v}{(\binpar{D(x)}{P})}}}{D(x)} \nonumber
\end{eqnarray}

\begin{remark}
  Note that the lazier definition still does not deal with summation
  or mixed summation (i.e. sums over input and output). The reader is
  invited to construct definitions of replication that deal with these
  features. 

  Further, the definitions are parameterized in a name, $x$. Can you,
  gentle reader, make a definition that eliminates this parameter and
  guarantees no accidental interaction between the replication
  machinery and the process being replicated -- i.e. no accidental
  sharing of names used by the process to get its work done and the
  name(s) used by the replication to effect copying. This latter
  revision of the definition of replication is crucial to obtaining
  the expected identity $!!P \sim !P$.
\end{remark}

\begin{remark}\label{rem:paradoxical_combinator}
  The reader familiar with the lambda calculus will have noticed the
  similarity between $D$ and the paradoxical combinator.

  [Ed. note: the existence of this seems to suggest we have to be more
  restrictive on the set of processes and names we admit if we are to
  support no-cloning.]
\end{remark}

\subsubsection{Bisimulation}

The computational dynamics gives rise to another kind of equivalence,
the equivalence of computational behavior. As previously mentioned
this is typically captured \emph{via} some form of bisimulation.

% The notion we use in this paper is weak barbed bisimulation
% \cite{milner91polyadicpi}.

The notion we use in this paper is derived from weak barbed
bisimulation \cite{milner91polyadicpi}. 

\begin{definition}
An \emph{observation relation}, $\downarrow_{\mathcal N}$, over a set
of names, $\mathcal N$, is the smallest relation satisfying the rules
below.

\infrule[Out-barb]{y \in {\mathcal N}, \; x \nameeq y}
		  {\outputp{x}{v} \downarrow_{\mathcal N} x}
\infrule[Par-barb]{\mbox{$P\downarrow_{\mathcal N} x$ or $Q\downarrow_{\mathcal N} x$}}
		  {\binpar{P}{Q} \downarrow_{\mathcal N} x}

We write $P \Downarrow_{\mathcal N} x$ if there is $Q$ such that 
$P \wred Q$ and $Q \downarrow_{\mathcal N} x$.
\end{definition}

\begin{definition}
%\label{def.bbisim}
An  ${\mathcal N}$-\emph{barbed bisimulation} over a set of names, ${\mathcal N}$, is a symmetric binary relation 
${\mathcal S}_{\mathcal N}$ between agents such that $P\rel{S}_{\mathcal N}Q$ implies:
\begin{enumerate}
\item If $P \red P'$ then $Q \wred Q'$ and $P'\rel{S}_{\mathcal N} Q'$.
\item If $P\downarrow_{\mathcal N} x$, then $Q\Downarrow_{\mathcal N} x$.
\end{enumerate}
$P$ is ${\mathcal N}$-barbed bisimilar to $Q$, written
$P \wbbisim_{\mathcal N} Q$, if $P \rel{S}_{\mathcal N} Q$ for some ${\mathcal N}$-barbed bisimulation ${\mathcal S}_{\mathcal N}$.
\end{definition}

$\mathcal{R} \subseteq \pi \times \pi$

$P \mathcal{R} Q => \forall P'. P \red P' \Rightarrow \exists Q'. Q \red Q', P' \mathcal{R} Q'$

$P \vdash x \Rightarrow Q \vdash x$

\begin{mathpar}
  \inferrule*[lab=Out-barb]{x \nameeq y}{{y}!\langle{Q}\rangle \vdash x}
  \and
  \inferrule*[lab=Par-barb]{\mbox{$P\vdash x$ or $Q\vdash x$}}{\binpar{P}{Q} \vdash x}
\end{mathpar}

\subsubsection{Contexts}

One of the principle advantages of computational calculi like the
$\pi$-calculus is a well-defined notion of context,
contextual-equivalence and a correlation between
contextual-equivalence and notions of bisimulation. The notion of
context allows the decomposition of a process into (sub-)process and
its syntactic environment, its context. Thus, a context may be
thought of as a process with a ``hole'' (written $\Box$) in it. The
application of a context $M$ to a process $P$, written $M[P]$, is
tantamount to filling the hole in $M$ with $P$. In this paper we do
not need the full weight of this theory, but do make use of the notion
of context in the proof the main theorem. 

\begin{mathpar}
  \inferrule* [lab=summation] {} {{M_{M},M_{N}} \bc \Box \;|\; x.M_{A} \;|\; M_{M}+M_{N}}
  \and
  \inferrule* [lab=agent] {} {{M_{A}} \bc (\vec{x})M_{P} \;| \; \clift{P_0,\ldots,M_{P},\ldots,P_N}}
  \and \\
  \inferrule* [lab=process] {} {{M_{P}} \bc M_{N} \;| \;P|M_{P} }
\end{mathpar} 

\begin{mathpar}
  \inferrule* [lab=sychronization] {} {M_{N} \bc \Box \;|\; x?M_{F} \;|\; x!M_{C}}
  \and
  \inferrule* [lab=abstraction] {} {{M_{F}} \bc (x)M_{P} }
  \and
  \inferrule* [lab=concretion] {} {{M_{C}} \bc \langle M_{P} \rangle }
  \and \\
  \inferrule* [lab=process] {} {{M_{P}} \bc M_{N} \;| \;P|M_{P} }
\end{mathpar}

\begin{definition}[contextual application] Given a context $M$, and
  process $P$, we define the \emph{contextual application}, $M[P] :=
  M\{P/\Box\}$. That is, the contextual application of M to P is the
  substitution of $P$ for $\Box$ in $M$.
\end{definition}

$\meaningof{-} : L \to \mathcal{P}(\pi)$

\begin{mathpar}
  \inferrule* [lab=collection] {} {\meaningof{true} = \pi, \and \meaningof{~E} = \pi \setminus \meaningof{E}, \and \meaningof{E_{1} \& E_{2}} = \meaningof{E_{1}} \cap \meaningof{E_{2}}}
\end{mathpar}

\begin{mathpar}
  \inferrule* [lab=structure] {} {\meaningof{0} = \{ P \in \pi | P \equiv 0 \}, \and \\ \meaningof{E_1 | E_2} = \{ P \in \pi | P \equiv P_{1} | P_{2}, P_{1} \in \meaningof{E_{1}}, P_{2} \in \meaningof{E_2}\} }
\end{mathpar}

\begin{mathpar}
 \inferrule* [lab=behavior] {} {\meaningof{\langle a?b \rangle E} = \{ P \in \pi | P \equiv Q | u?(y)P', \\ \and \\\\ \and \\ \;\;\; u \in \meaningof{a}, \forall z.P'\{z/y\} \in \meaningof{E\{z/b\}}\}, \and \\ \meaningof{a!E} = \{ P \in \pi | P \equiv Q | x!\langle P' \rangle, x \in \meaningof{a} P' \in \meaningof{E}\} }
\end{mathpar}

\begin{mathpar}
 \inferrule* [lab=nominal] {} {\meaningof{\quotep{E}} = \{ \quotep{P} \in \quotep{\pi} | P \in \meaningof{E} \}, \and \meaningof{\quotep{P}} = \{ \quotep{Q} \in \quotep{\pi} | P \equiv Q \} \and \\ \meaningof{@\quotep{E}} = \{ P \in \pi | P \equiv @x, x \in \meaningof{E} \}}
\end{mathpar}

\begin{eqnarray*}
  \\
  \meaningof{-} : TS \to ST
\end{eqnarray*}

\begin{eqnarray*}
  \\
  L : TS \to ST
\end{eqnarray*}

\begin{eqnarray*}
  \\
  P \models E \iff P \in \meaningof{E}
\end{eqnarray*}

\begin{eqnarray*}
  P \approx_{L} Q \iff \forall E \in L. P \models E \iff Q \models E
\end{eqnarray*}

\begin{eqnarray*}
  P \approx_{K} Q
\end{eqnarray*}

\begin{eqnarray*}
  P \approx Q
\end{eqnarray*}

$\approx_{K} = \approx = \approx_{L}$

\subsubsection{Contextual duality}

Note that contexts extend the quotation operation to a family of
operations from processes to names. Given a context, $M$, we can
define a \emph{nominal context}, $\quotep{M}$ by $\quotep{M}[P] :=
\quotep{M[P]}$. To foreshadow what is to come we observe that these
operations enjoy a duality with processes very much like the duality
between vectors and maps from vectors to scalars.

Further, because the calculus is essentially higher-order, we have a
correspondence between contexts and processes. More specifically,
given a name $x$ and a context $M$ we can construct $M^{*}_{x}$ such
that 

\begin{mathpar}
  M^{*}_{x} | \lift{x}{P} \red M[P]
\end{mathpar}

namely,

\begin{mathpar}
  M^{*}_{x} := x?(u).M[\dropn{u}]
\end{mathpar}

The dependence of $M^{*}_{x}$ on a name makes it an abstraction, 

\begin{mathpar}
  M^{*} := (x)x?(u).M[\dropn{u}]
\end{mathpar}

\subsection{Additional notation}

It will sometimes be convenient to denote the process a name
quotes. We already have the notation $x = \quotep{P}$, but it will be
convenient to introduce an alternate notation, $\procn{x}$, when we
want to emphasize the connection to the use of the name. Note that, by
virtue of name equivalence, $\quotep{\procn{x}} \nameeq x$; so, the
notation is consistent with previous definitions.

Further, because names have structure it is possible to effect
substitutions on the basis of that structure. This means we need to
upgrade our notation for substitutions, which we accomplish by
adapting comprehension notation. Thus,

\begin{mathpar}
  P\{ y / x : x \in S \}
\end{mathpar}

is interpreted to mean the process derived from P by replacing (in a
capture-avoiding manner) each occurrence of $x$ in $S$ by $y$. For example,

\begin{mathpar}
  P\{ \quotep{\procn{x}|\procn{x}} / x : x \in \freenames{P} \}
\end{mathpar}

will replace each (occurrence) of a free name $x$ in $P$ by
$\quotep{\procn{x}|\procn{x}}$.

Also, we will avail ourselves of the notation $x^{L}$ and $x^{R}$ to
denote injections of a name into disjoint copies of the name
space. There are numerous ways to accomplish this. One example can be
found in \cite{MeredithR05}. This notation overloads to vectors of
names: $\vec{x}^{\pi} := (x_{i}^{\pi} \; : \; 0 \leq i < |\vec{x}| )$ where $\pi \in \{L,R\}$.

We also use $P^{\Box} := P|\Box$.

In \cite{MeredithR05} an interpretation of the new operator is
given. It turns out that there are several possible interpretations
all enjoying the requisite algebraic properties of the operator (see
\cite{milner91polyadicpi}). We will therefore make liberal use of
$(\nu\; \vec{x})P$.

% subsection the_syntax_and_semantics_of_the_notation_system (end)   

\input{qm2pi.qmops} 

\input{qm2pi.sterngerlach} 

\input{qm2pi.metric} 

% section concurrent_process_calculi (end)

%\input{qm2pi.proofsketch}

% section proof sketch (end)

%\input{qm2pi.slviaknots} 

% section spatial logic via knots (end)

\input{qm2pi.conclusion}

% section conclusion (end)

%\input{qm2pi.dtcodes} 

% section wiring algorithm (end)

\input{qm2pi.ack} 

% section acknowledgments (end)

\newpage


\bibliographystyle{plain}   
\bibliography{../../biblios/main.bib}

\input{qm2pi.rhodetails}

\end{document}

 

% section acknowledgments (end)

\newpage


\bibliographystyle{plain}   
\bibliography{../../biblios/main.bib}

\documentclass[12pt]{llncs}
%\documentclass{jktr}

\usepackage[pdftex]{hyperref}                   
\usepackage {listings}
\usepackage {mathpartir}
\usepackage{bcprules}
%\usepackage{listings}
                       
\usepackage{graphicx} 
%\usepackage[margins=2.5cm,nohead,nofoot]{geometry}
%\usepackage{geometry}
\usepackage{amsfonts}
\usepackage{amstext}
\usepackage{latexsym}
\usepackage{amssymb}
\usepackage{color}


%\include{myPreamble}
\include{qm2pi.local} 

%\ifpdf
%\usepackage[pdftex]{graphicx}
%\else
%\usepackage{graphicx}
%\fi

 % \ifpdf
%  \usepackage{pdfsync}
%  \if


%\title{Brief Article}
%\author{David F. Snyder}
%\author{L.G. Meredith}

%\address{Dept. of Math., Texas State University--San Marcos, San Marcos, TX 78666}
       
\pagestyle{empty}


\begin{document}

\lstset{language=[Objective]Caml,frame=shadowbox}

\input{qm2pi.front}

% section front matter (end)

\input{qm2pi.intro} 
 
% section introduction (end)

% \input{qm2pi.knotations} 

% section notation (end)

\input{qm2pi.process.calculi} 

% section concurrent_process_calculi_and_spatial_logics_ (end)
    
%\input{qm2pi.knots2pi} 

%\input{qm2pi.trefoil} 

%\input{qm2pi.mainthm} 

% subsection basic_interpretation (end)

%\input{qm2pi.rho.presentation} 
\subsection{The syntax and semantics of the notation system}\label{sub:the_syntax_and_semantics_of_the_notation_system} % (fold)

We now summarize a technical presentation of the calculus that
embodies our theory of dynamics. The typical presentation of such a
calculus follows the style of giving generators and relations on
them. The grammar, below, describing term constructors, freely
generates the set of processes, $\Proc$. This set is then quotiented
by a relation known as structural congruence and it is over this set
that the notion of dynamics is expressed. This presentation is
essentially that of \cite{MeredithR05} with the addition of
polyadicity and summation. For readability we have relegated some of
the technical subtleties to an appendix.

\subsubsection{Process grammar}\label{subsub:process_grammar}

\begin{mathpar}
  \inferrule* [lab=synchronization] {} {{M} \bc \pzero \;|\; x?F \;|\; x!C }
  \and
  \inferrule* [lab=abstraction] {} {{F} \bc (x)P}
  \and
  \inferrule* [lab=concretion] {} {{C} \bc \langle Q \rangle}
  \and
  \inferrule* [lab=process] {} {{P,Q} \bc M \;| \;P|Q \;|\; @{x}}
  \and
  \inferrule* [lab=name] {} {{x} \bc \quotep{P}}
\end{mathpar} 

Note that $\vec{x}$ (resp. $\vec{P}$) denotes a vector of names
(resp. processes) of length $|\vec{x}|$ (resp. $|\vec{P}|$). We adopt
the following useful abbreviations.

\begin{mathpar}
   x?(\vec{y}).P := x.(\vec{y})P \and  x\clift{\vec{P}} := x.\clift{\vec{P}}
   \and x!(y) := \lift{x}{\dropn{y}}
   \and \Pi_{i=0}^{n-1}P_i := P_0 | \ldots | P_{n-1}
\end{mathpar}

\subsubsection{Structural congruence}

\paragraph{Free and bound names and alpha-equivalence.} At the
core of structural equivalence is alpha-equivalence which identifies
process that are the same up to a change of variable. Formally, we
recognize the distinction between free and bound names. The free names
of a process, $\freenames{P}$, may be calculated recursively as
follows:

\begin{mathpar}
\freenames{\pzero} := \emptyset
  \and \\
  \freenames{x?(y).P} := \{ x \} \cup (\freenames{P} \setminus \{ y \})
  \and 
  \freenames{x!\langle P \rangle} := \{ x \} \cup \{ P \} 
  \and \\
  \freenames{P|Q} := \freenames{P} \cup \freenames{Q}
  \and \\
  \freenames{@{x}} := \{ x \}
\end{mathpar}

$\pi$
$\quotep{\pi}$

$\freenames{-} : \pi \to \mathcal{P}(\quotep{\pi})$

\begin{eqnarray*}
  \freenames{\pzero} & := & \emptyset \\
  \freenames{x?(y).P} & := & \{ x \} \cup (\freenames{P} \setminus \{ y \}) \\
  \freenames{x!\langle P \rangle} & := & \{ x \} \cup \{ P \} \\
  \freenames{P|Q} & := & \freenames{P} \cup \freenames{Q} \\
  \freenames{\dropn{x}} & := & \{ x \}
\end{eqnarray*}

The bound names of a process, $\boundnames{P}$, are those names occurring in $P$
that are not free. For example, in $x?(y).0$, the name $x$ is free, while $y$ is bound.

\begin{mathpar}
  \inferrule* [lab=monoidal-laws] {} { P|Q \equiv Q|P \and P|0 \equiv P \and P|(Q|R) \equiv (P|Q)|R }
\end{mathpar}

\begin{mathpar}
  \inferrule* [lab=alpha-equivalence] {} { (x)P \equiv (y)P\{y/x\} \and y \not\in \freenames{P} }
\end{mathpar}

\begin{definition}
Then two processes, $P,Q$, are alpha-equivalent if $P = Q\{\vec{y}/\vec{x}\}$ for
some $\vec{x} \in \boundnames{Q},\vec{y} \in \boundnames{P}$, where $Q\{\vec{y}/\vec{x}\}$
denotes the capture-avoiding substitution of $\vec{y}$ for $\vec{x}$ in $Q$.
\end{definition}

\begin{definition}
  The {\em structural congruence} \cite{SangiorgiWalker} , $\equiv$,
  between processes is the least congruence containing
  alpha-equivalence, satisfying the abelian monoid laws
  (associativity, commutativity and $\pzero$ as identity) for parallel
  composition $|$ and for summation $+$.
\end{definition}

\subsection{Name equivalence}

We take name equivalence, written $\nameeq$, to be the smallest
equivalence relation generated by the following rules.

\begin{mathpar}
\inferrule*[lab=Quote-drop]
{ }
{ \quotep{@{x}} \nameeq x }

\inferrule*[lab=Struct-equiv]
{ P \scong Q }
{ \quotep{P} \nameeq \quotep{Q} }
\end{mathpar}

The astute reader will have noticed that the mutual recursion of names
and processes imposes a mutual recursion on alpha-equivalence and
structural equivalence via name-equivalence. Fortunately, all of this
works out pleasantly and we may calculate in the natural way, free of
concern. The reader interested in the details is referred to the
appendix \ref{appendix:rho_details}.

\subsection{Substitution}

We use $\Proc$ for the set of processes, $\QProc$ for the set of
names, and $\id{\{}\vec{y} / \vec{x} \id{\}}$ to denote partial maps,
$s : \QProc \rightarrow \QProc$. A map, $s$ lifts, uniquely, to a map
on process terms, $\widehat{s} : \Proc \rightarrow \Proc$ by the
following equations.

\begin{mathpar}
  (0) \psubstp{Q}{P} := 0 \\
  (R \juxtap S) \psubstp{Q}{P}
  :=    
  (R)\psubstp{Q}{P} \juxtap (S) \psubstp{Q}{P} \\
  (x?(y).R) \psubstp{Q}{P}    
  :=    
  (x)\substp{Q}{P} (z)\concat( (R \psubstn{z}{y}) \psubstp{Q}{P} ) \\
  (\lift{x}{R}) \psubstp{Q}{P}  
  :=
  \lift{(x)\substp{Q}{P}}{ R \psubstp{Q}{P} } \\
%   (\dropn{x})  \psubstp{Q}{P}       
%   := 
%   \left\{ 
%     \begin{array}{ccc} 
%       \dropn{\quotep{Q}} & & x \nameeq \quotep{P} \\
%       \dropn{x} & & otherwise \\
%     \end{array}
%   \right. 
  (\dropn{x})  \psubstp{Q}{P}       
  := 
  \left\{ 
    \begin{array}{ccc} 
      Q & & x \nameeq \quotep{P} \\
      \dropn{x} & & otherwise \\
    \end{array}
  \right.
\end{mathpar}
 

where

\begin{eqnarray}
  (x)\id{\{} \lpquote Q \rpquote / \lpquote P \rpquote \id{\}}            = 
  \left\{ 
    \begin{array}{ccc}
      \lpquote Q \rpquote & & x \nameeq \lpquote P \rpquote \\
      x & & otherwise \\
    \end{array}
  \right. \nonumber
\end{eqnarray}

and $z$ is chosen distinct from $\quotep{P}$, $\quotep{Q}$, the free
names in $Q$, and all the names in $R$. Our $\alpha$-equivalence will
be built in the standard way from this substitution.

\begin{remark}\label{rem:no_self_referential_names}
  One consequence of these definitions is that $\forall P. \quotep{P}
  \not\in \freenames{P}$.
\end{remark}

\subsection{ Dynamic quote: an example }

Anticipating something of what's to come, consider applying the
substitution, $\widehat{\id{\{}u / z \id{\}}}$, to the following pair
of processes, $\lift{w}{y!(z)}$ and $w[ \lpquote y!(z) \rpquote ]$.

\begin{eqnarray}
	\lift{w}{y!(z)}\widehat{\id{\{}u / z \id{\}}}
		& = &
		\lift{w}{y!(u)} \nonumber\\
	w[ \lpquote y!(z) \rpquote ] \widehat{ \id{\{}u / z \id{\}} }
		& = &
		w[ \lpquote y!(z) \rpquote ] \nonumber
\end{eqnarray}

Because the body of the process between quotes is impervious to
substitution, we get radically different answers. In fact, by
examining the first process in an input context,
e.g. $x?(z).\lift{w}{y!(z)}$, we see that the process under the lift
operator may be shaped by prefixed inputs binding a name inside it. In
this sense, the lift operator will be seen as a way to dynamically
construct processes before reifying them as names.

Finally equipped with these standard features we can present the
dynamics of the calculus.

\subsubsection{Operational semantics} 

Finally, we introduce the computational dynamics. What marks these
algebras as distinct from other more traditionally studied algebraic
structures, e.g. vector spaces or polynomial rings, is the manner in
which dynamics is captured. In traditional structures, dynamics is typically
expressed through morphisms between such structures, as in linear maps
between vector spaces or morphisms between rings. In algebras
associated with the semantics of computation, the dynamics is
expressed as part of the algebraic structure itself, through a
reduction reduction relation typically denoted by $\red$. Below, we
give a recursive presentation of this relation for the calculus used
in the encoding.

$\red \subseteq \pi \times \pi$
$\red : \pi \to \mathcal{P}(\pi)$

\begin{mathpar}
  \inferrule* [lab=Comm] { \textsf{match}( x_{src}, x_{trgt} ) } { x_{trgt}?(y)P \; | \; x_{src}!\langle {Q} \rangle \red P\{\quotep{Q}/y}\} }
  \and \\
  \inferrule* [lab=Par] {{P} \red {P}'} {{{P} | {Q}} \red {{P}' | {Q}}}
  \and
  \inferrule* [lab=Equiv]{{{P} \scong {P}'} \andalso {{P}' \red {Q}'} \andalso {{Q}' \scong {Q}}}{{P} \red {Q}}
\end{mathpar}

\begin{eqnarray*}
  match_{\equiv} (\quotep{P},\quotep{Q}) & := & P \equiv Q \\
  match_{\dagger}(\quotep{P},\quotep{Q}) & := & \forall R. P|Q \red^{*} R => R \red^{*} 0 \\
  match_{K}(\quotep{P},\quotep{Q}) & := & K \mbox{ for some context } K
\end{eqnarray*}

$u?(x)P | u!\langle Q \rangle \red P\{\quotep{Q}/x\}$

%We write $\wred$ for $\red^*$, and $P\red$ if $\exists Q $ such that $ P \red Q$.
We write $P\red$ if $\exists Q $ such that $ P \red Q$ and $P\not\red$, otherwise.

\section{Replication}

As mentioned before, it is known that replication (and hence
recursion) can be implemented in a higher-order process algebra
\cite{SangiorgiWalker}. As our first example of calculation with the
machinery thus far presented we give the construction explicitly in
the {\rhoc}.

\begin{eqnarray}
	D_{x} & := & \prefix{x}{y}{(\binpar{\outputp{x}{y}}{@{y}})} \nonumber\\
	\bangp_{x}{P} & := & \binpar{{x}!\langle{\binpar{D_{x}}{P}}\rangle}{D_{x}} \nonumber
\end{eqnarray}

\begin{eqnarray}
	\bangp_{x}{P} & & \nonumber\\
	=
	& {x}!\langle{(\prefix{x}{y}{(\outputp{x}{y} | @{y})) | P}}\rangle 
	      | \prefix{x}{y}{(\outputp{x}{y} | @{y})} & \nonumber\\
	\red
	& (\outputp{x}{y} | @{y})\substn{\quotep{(\prefix{x}{y}{(@{y} | \outputp{x}{y})) | P}}}{y} & \nonumber\\
	=
	& \outputp{x}{\quotep{(\prefix{x}{y}{(\outputp{x}{y} | @{y})) | P}}}
	  | {(\prefix{x}{y}{(\outputp{x}{y} | @{y})) | P}} & \nonumber\\
	\red
	& \ldots & \nonumber\\
	\red^*
	& P | P | \ldots & \nonumber
\end{eqnarray}

Of course, this encoding, as an implementation, runs away, unfolding
$\bangp{P}$ eagerly. A lazier and more implementable replication
operator, restricted to input-guarded processes, may be obtained as follows.

\begin{eqnarray}
\bangp{\prefix{u}{v}{P}} 
	:= 
	\binpar{\lift{x}{\prefix{u}{v}{(\binpar{D(x)}{P})}}}{D(x)} \nonumber
\end{eqnarray}

\begin{remark}
  Note that the lazier definition still does not deal with summation
  or mixed summation (i.e. sums over input and output). The reader is
  invited to construct definitions of replication that deal with these
  features. 

  Further, the definitions are parameterized in a name, $x$. Can you,
  gentle reader, make a definition that eliminates this parameter and
  guarantees no accidental interaction between the replication
  machinery and the process being replicated -- i.e. no accidental
  sharing of names used by the process to get its work done and the
  name(s) used by the replication to effect copying. This latter
  revision of the definition of replication is crucial to obtaining
  the expected identity $!!P \sim !P$.
\end{remark}

\begin{remark}\label{rem:paradoxical_combinator}
  The reader familiar with the lambda calculus will have noticed the
  similarity between $D$ and the paradoxical combinator.

  [Ed. note: the existence of this seems to suggest we have to be more
  restrictive on the set of processes and names we admit if we are to
  support no-cloning.]
\end{remark}

\subsubsection{Bisimulation}

The computational dynamics gives rise to another kind of equivalence,
the equivalence of computational behavior. As previously mentioned
this is typically captured \emph{via} some form of bisimulation.

% The notion we use in this paper is weak barbed bisimulation
% \cite{milner91polyadicpi}.

The notion we use in this paper is derived from weak barbed
bisimulation \cite{milner91polyadicpi}. 

\begin{definition}
An \emph{observation relation}, $\downarrow_{\mathcal N}$, over a set
of names, $\mathcal N$, is the smallest relation satisfying the rules
below.

\infrule[Out-barb]{y \in {\mathcal N}, \; x \nameeq y}
		  {\outputp{x}{v} \downarrow_{\mathcal N} x}
\infrule[Par-barb]{\mbox{$P\downarrow_{\mathcal N} x$ or $Q\downarrow_{\mathcal N} x$}}
		  {\binpar{P}{Q} \downarrow_{\mathcal N} x}

We write $P \Downarrow_{\mathcal N} x$ if there is $Q$ such that 
$P \wred Q$ and $Q \downarrow_{\mathcal N} x$.
\end{definition}

\begin{definition}
%\label{def.bbisim}
An  ${\mathcal N}$-\emph{barbed bisimulation} over a set of names, ${\mathcal N}$, is a symmetric binary relation 
${\mathcal S}_{\mathcal N}$ between agents such that $P\rel{S}_{\mathcal N}Q$ implies:
\begin{enumerate}
\item If $P \red P'$ then $Q \wred Q'$ and $P'\rel{S}_{\mathcal N} Q'$.
\item If $P\downarrow_{\mathcal N} x$, then $Q\Downarrow_{\mathcal N} x$.
\end{enumerate}
$P$ is ${\mathcal N}$-barbed bisimilar to $Q$, written
$P \wbbisim_{\mathcal N} Q$, if $P \rel{S}_{\mathcal N} Q$ for some ${\mathcal N}$-barbed bisimulation ${\mathcal S}_{\mathcal N}$.
\end{definition}

$\mathcal{R} \subseteq \pi \times \pi$

$P \mathcal{R} Q => \forall P'. P \red P' \Rightarrow \exists Q'. Q \red Q', P' \mathcal{R} Q'$

$P \vdash x \Rightarrow Q \vdash x$

\begin{mathpar}
  \inferrule*[lab=Out-barb]{x \nameeq y}{{y}!\langle{Q}\rangle \vdash x}
  \and
  \inferrule*[lab=Par-barb]{\mbox{$P\vdash x$ or $Q\vdash x$}}{\binpar{P}{Q} \vdash x}
\end{mathpar}

\subsubsection{Contexts}

One of the principle advantages of computational calculi like the
$\pi$-calculus is a well-defined notion of context,
contextual-equivalence and a correlation between
contextual-equivalence and notions of bisimulation. The notion of
context allows the decomposition of a process into (sub-)process and
its syntactic environment, its context. Thus, a context may be
thought of as a process with a ``hole'' (written $\Box$) in it. The
application of a context $M$ to a process $P$, written $M[P]$, is
tantamount to filling the hole in $M$ with $P$. In this paper we do
not need the full weight of this theory, but do make use of the notion
of context in the proof the main theorem. 

\begin{mathpar}
  \inferrule* [lab=summation] {} {{M_{M},M_{N}} \bc \Box \;|\; x.M_{A} \;|\; M_{M}+M_{N}}
  \and
  \inferrule* [lab=agent] {} {{M_{A}} \bc (\vec{x})M_{P} \;| \; \clift{P_0,\ldots,M_{P},\ldots,P_N}}
  \and \\
  \inferrule* [lab=process] {} {{M_{P}} \bc M_{N} \;| \;P|M_{P} }
\end{mathpar} 

\begin{mathpar}
  \inferrule* [lab=sychronization] {} {M_{N} \bc \Box \;|\; x?M_{F} \;|\; x!M_{C}}
  \and
  \inferrule* [lab=abstraction] {} {{M_{F}} \bc (x)M_{P} }
  \and
  \inferrule* [lab=concretion] {} {{M_{C}} \bc \langle M_{P} \rangle }
  \and \\
  \inferrule* [lab=process] {} {{M_{P}} \bc M_{N} \;| \;P|M_{P} }
\end{mathpar}

\begin{definition}[contextual application] Given a context $M$, and
  process $P$, we define the \emph{contextual application}, $M[P] :=
  M\{P/\Box\}$. That is, the contextual application of M to P is the
  substitution of $P$ for $\Box$ in $M$.
\end{definition}

$\meaningof{-} : L \to \mathcal{P}(\pi)$

\begin{mathpar}
  \inferrule* [lab=collection] {} {\meaningof{true} = \pi, \and \meaningof{~E} = \pi \setminus \meaningof{E}, \and \meaningof{E_{1} \& E_{2}} = \meaningof{E_{1}} \cap \meaningof{E_{2}}}
\end{mathpar}

\begin{mathpar}
  \inferrule* [lab=structure] {} {\meaningof{0} = \{ P \in \pi | P \equiv 0 \}, \and \\ \meaningof{E_1 | E_2} = \{ P \in \pi | P \equiv P_{1} | P_{2}, P_{1} \in \meaningof{E_{1}}, P_{2} \in \meaningof{E_2}\} }
\end{mathpar}

\begin{mathpar}
 \inferrule* [lab=behavior] {} {\meaningof{\langle a?b \rangle E} = \{ P \in \pi | P \equiv Q | u?(y)P', \\ \and \\\\ \and \\ \;\;\; u \in \meaningof{a}, \forall z.P'\{z/y\} \in \meaningof{E\{z/b\}}\}, \and \\ \meaningof{a!E} = \{ P \in \pi | P \equiv Q | x!\langle P' \rangle, x \in \meaningof{a} P' \in \meaningof{E}\} }
\end{mathpar}

\begin{mathpar}
 \inferrule* [lab=nominal] {} {\meaningof{\quotep{E}} = \{ \quotep{P} \in \quotep{\pi} | P \in \meaningof{E} \}, \and \meaningof{\quotep{P}} = \{ \quotep{Q} \in \quotep{\pi} | P \equiv Q \} \and \\ \meaningof{@\quotep{E}} = \{ P \in \pi | P \equiv @x, x \in \meaningof{E} \}}
\end{mathpar}

\begin{eqnarray*}
  \\
  \meaningof{-} : TS \to ST
\end{eqnarray*}

\begin{eqnarray*}
  \\
  L : TS \to ST
\end{eqnarray*}

\begin{eqnarray*}
  \\
  P \models E \iff P \in \meaningof{E}
\end{eqnarray*}

\begin{eqnarray*}
  P \approx_{L} Q \iff \forall E \in L. P \models E \iff Q \models E
\end{eqnarray*}

\begin{eqnarray*}
  P \approx_{K} Q
\end{eqnarray*}

\begin{eqnarray*}
  P \approx Q
\end{eqnarray*}

$\approx_{K} = \approx = \approx_{L}$

\subsubsection{Contextual duality}

Note that contexts extend the quotation operation to a family of
operations from processes to names. Given a context, $M$, we can
define a \emph{nominal context}, $\quotep{M}$ by $\quotep{M}[P] :=
\quotep{M[P]}$. To foreshadow what is to come we observe that these
operations enjoy a duality with processes very much like the duality
between vectors and maps from vectors to scalars.

Further, because the calculus is essentially higher-order, we have a
correspondence between contexts and processes. More specifically,
given a name $x$ and a context $M$ we can construct $M^{*}_{x}$ such
that 

\begin{mathpar}
  M^{*}_{x} | \lift{x}{P} \red M[P]
\end{mathpar}

namely,

\begin{mathpar}
  M^{*}_{x} := x?(u).M[\dropn{u}]
\end{mathpar}

The dependence of $M^{*}_{x}$ on a name makes it an abstraction, 

\begin{mathpar}
  M^{*} := (x)x?(u).M[\dropn{u}]
\end{mathpar}

\subsection{Additional notation}

It will sometimes be convenient to denote the process a name
quotes. We already have the notation $x = \quotep{P}$, but it will be
convenient to introduce an alternate notation, $\procn{x}$, when we
want to emphasize the connection to the use of the name. Note that, by
virtue of name equivalence, $\quotep{\procn{x}} \nameeq x$; so, the
notation is consistent with previous definitions.

Further, because names have structure it is possible to effect
substitutions on the basis of that structure. This means we need to
upgrade our notation for substitutions, which we accomplish by
adapting comprehension notation. Thus,

\begin{mathpar}
  P\{ y / x : x \in S \}
\end{mathpar}

is interpreted to mean the process derived from P by replacing (in a
capture-avoiding manner) each occurrence of $x$ in $S$ by $y$. For example,

\begin{mathpar}
  P\{ \quotep{\procn{x}|\procn{x}} / x : x \in \freenames{P} \}
\end{mathpar}

will replace each (occurrence) of a free name $x$ in $P$ by
$\quotep{\procn{x}|\procn{x}}$.

Also, we will avail ourselves of the notation $x^{L}$ and $x^{R}$ to
denote injections of a name into disjoint copies of the name
space. There are numerous ways to accomplish this. One example can be
found in \cite{MeredithR05}. This notation overloads to vectors of
names: $\vec{x}^{\pi} := (x_{i}^{\pi} \; : \; 0 \leq i < |\vec{x}| )$ where $\pi \in \{L,R\}$.

We also use $P^{\Box} := P|\Box$.

In \cite{MeredithR05} an interpretation of the new operator is
given. It turns out that there are several possible interpretations
all enjoying the requisite algebraic properties of the operator (see
\cite{milner91polyadicpi}). We will therefore make liberal use of
$(\nu\; \vec{x})P$.

% subsection the_syntax_and_semantics_of_the_notation_system (end)   

\input{qm2pi.qmops} 

\input{qm2pi.sterngerlach} 

\input{qm2pi.metric} 

% section concurrent_process_calculi (end)

%\input{qm2pi.proofsketch}

% section proof sketch (end)

%\input{qm2pi.slviaknots} 

% section spatial logic via knots (end)

\input{qm2pi.conclusion}

% section conclusion (end)

%\input{qm2pi.dtcodes} 

% section wiring algorithm (end)

\input{qm2pi.ack} 

% section acknowledgments (end)

\newpage


\bibliographystyle{plain}   
\bibliography{../../biblios/main.bib}

\input{qm2pi.rhodetails}

\end{document}



\end{document}

 

% section wiring algorithm (end)

\documentclass[12pt]{llncs}
%\documentclass{jktr}

\usepackage[pdftex]{hyperref}                   
\usepackage {listings}
\usepackage {mathpartir}
\usepackage{bcprules}
%\usepackage{listings}
                       
\usepackage{graphicx} 
%\usepackage[margins=2.5cm,nohead,nofoot]{geometry}
%\usepackage{geometry}
\usepackage{amsfonts}
\usepackage{amstext}
\usepackage{latexsym}
\usepackage{amssymb}
\usepackage{color}


%\include{myPreamble}
\documentclass[12pt]{llncs}
%\documentclass{jktr}

\usepackage[pdftex]{hyperref}                   
\usepackage {listings}
\usepackage {mathpartir}
\usepackage{bcprules}
%\usepackage{listings}
                       
\usepackage{graphicx} 
%\usepackage[margins=2.5cm,nohead,nofoot]{geometry}
%\usepackage{geometry}
\usepackage{amsfonts}
\usepackage{amstext}
\usepackage{latexsym}
\usepackage{amssymb}
\usepackage{color}


%\include{myPreamble}
\include{qm2pi.local} 

%\ifpdf
%\usepackage[pdftex]{graphicx}
%\else
%\usepackage{graphicx}
%\fi

 % \ifpdf
%  \usepackage{pdfsync}
%  \if


%\title{Brief Article}
%\author{David F. Snyder}
%\author{L.G. Meredith}

%\address{Dept. of Math., Texas State University--San Marcos, San Marcos, TX 78666}
       
\pagestyle{empty}


\begin{document}

\lstset{language=[Objective]Caml,frame=shadowbox}

\input{qm2pi.front}

% section front matter (end)

\input{qm2pi.intro} 
 
% section introduction (end)

% \input{qm2pi.knotations} 

% section notation (end)

\input{qm2pi.process.calculi} 

% section concurrent_process_calculi_and_spatial_logics_ (end)
    
%\input{qm2pi.knots2pi} 

%\input{qm2pi.trefoil} 

%\input{qm2pi.mainthm} 

% subsection basic_interpretation (end)

%\input{qm2pi.rho.presentation} 
\subsection{The syntax and semantics of the notation system}\label{sub:the_syntax_and_semantics_of_the_notation_system} % (fold)

We now summarize a technical presentation of the calculus that
embodies our theory of dynamics. The typical presentation of such a
calculus follows the style of giving generators and relations on
them. The grammar, below, describing term constructors, freely
generates the set of processes, $\Proc$. This set is then quotiented
by a relation known as structural congruence and it is over this set
that the notion of dynamics is expressed. This presentation is
essentially that of \cite{MeredithR05} with the addition of
polyadicity and summation. For readability we have relegated some of
the technical subtleties to an appendix.

\subsubsection{Process grammar}\label{subsub:process_grammar}

\begin{mathpar}
  \inferrule* [lab=synchronization] {} {{M} \bc \pzero \;|\; x?F \;|\; x!C }
  \and
  \inferrule* [lab=abstraction] {} {{F} \bc (x)P}
  \and
  \inferrule* [lab=concretion] {} {{C} \bc \langle Q \rangle}
  \and
  \inferrule* [lab=process] {} {{P,Q} \bc M \;| \;P|Q \;|\; @{x}}
  \and
  \inferrule* [lab=name] {} {{x} \bc \quotep{P}}
\end{mathpar} 

Note that $\vec{x}$ (resp. $\vec{P}$) denotes a vector of names
(resp. processes) of length $|\vec{x}|$ (resp. $|\vec{P}|$). We adopt
the following useful abbreviations.

\begin{mathpar}
   x?(\vec{y}).P := x.(\vec{y})P \and  x\clift{\vec{P}} := x.\clift{\vec{P}}
   \and x!(y) := \lift{x}{\dropn{y}}
   \and \Pi_{i=0}^{n-1}P_i := P_0 | \ldots | P_{n-1}
\end{mathpar}

\subsubsection{Structural congruence}

\paragraph{Free and bound names and alpha-equivalence.} At the
core of structural equivalence is alpha-equivalence which identifies
process that are the same up to a change of variable. Formally, we
recognize the distinction between free and bound names. The free names
of a process, $\freenames{P}$, may be calculated recursively as
follows:

\begin{mathpar}
\freenames{\pzero} := \emptyset
  \and \\
  \freenames{x?(y).P} := \{ x \} \cup (\freenames{P} \setminus \{ y \})
  \and 
  \freenames{x!\langle P \rangle} := \{ x \} \cup \{ P \} 
  \and \\
  \freenames{P|Q} := \freenames{P} \cup \freenames{Q}
  \and \\
  \freenames{@{x}} := \{ x \}
\end{mathpar}

$\pi$
$\quotep{\pi}$

$\freenames{-} : \pi \to \mathcal{P}(\quotep{\pi})$

\begin{eqnarray*}
  \freenames{\pzero} & := & \emptyset \\
  \freenames{x?(y).P} & := & \{ x \} \cup (\freenames{P} \setminus \{ y \}) \\
  \freenames{x!\langle P \rangle} & := & \{ x \} \cup \{ P \} \\
  \freenames{P|Q} & := & \freenames{P} \cup \freenames{Q} \\
  \freenames{\dropn{x}} & := & \{ x \}
\end{eqnarray*}

The bound names of a process, $\boundnames{P}$, are those names occurring in $P$
that are not free. For example, in $x?(y).0$, the name $x$ is free, while $y$ is bound.

\begin{mathpar}
  \inferrule* [lab=monoidal-laws] {} { P|Q \equiv Q|P \and P|0 \equiv P \and P|(Q|R) \equiv (P|Q)|R }
\end{mathpar}

\begin{mathpar}
  \inferrule* [lab=alpha-equivalence] {} { (x)P \equiv (y)P\{y/x\} \and y \not\in \freenames{P} }
\end{mathpar}

\begin{definition}
Then two processes, $P,Q$, are alpha-equivalent if $P = Q\{\vec{y}/\vec{x}\}$ for
some $\vec{x} \in \boundnames{Q},\vec{y} \in \boundnames{P}$, where $Q\{\vec{y}/\vec{x}\}$
denotes the capture-avoiding substitution of $\vec{y}$ for $\vec{x}$ in $Q$.
\end{definition}

\begin{definition}
  The {\em structural congruence} \cite{SangiorgiWalker} , $\equiv$,
  between processes is the least congruence containing
  alpha-equivalence, satisfying the abelian monoid laws
  (associativity, commutativity and $\pzero$ as identity) for parallel
  composition $|$ and for summation $+$.
\end{definition}

\subsection{Name equivalence}

We take name equivalence, written $\nameeq$, to be the smallest
equivalence relation generated by the following rules.

\begin{mathpar}
\inferrule*[lab=Quote-drop]
{ }
{ \quotep{@{x}} \nameeq x }

\inferrule*[lab=Struct-equiv]
{ P \scong Q }
{ \quotep{P} \nameeq \quotep{Q} }
\end{mathpar}

The astute reader will have noticed that the mutual recursion of names
and processes imposes a mutual recursion on alpha-equivalence and
structural equivalence via name-equivalence. Fortunately, all of this
works out pleasantly and we may calculate in the natural way, free of
concern. The reader interested in the details is referred to the
appendix \ref{appendix:rho_details}.

\subsection{Substitution}

We use $\Proc$ for the set of processes, $\QProc$ for the set of
names, and $\id{\{}\vec{y} / \vec{x} \id{\}}$ to denote partial maps,
$s : \QProc \rightarrow \QProc$. A map, $s$ lifts, uniquely, to a map
on process terms, $\widehat{s} : \Proc \rightarrow \Proc$ by the
following equations.

\begin{mathpar}
  (0) \psubstp{Q}{P} := 0 \\
  (R \juxtap S) \psubstp{Q}{P}
  :=    
  (R)\psubstp{Q}{P} \juxtap (S) \psubstp{Q}{P} \\
  (x?(y).R) \psubstp{Q}{P}    
  :=    
  (x)\substp{Q}{P} (z)\concat( (R \psubstn{z}{y}) \psubstp{Q}{P} ) \\
  (\lift{x}{R}) \psubstp{Q}{P}  
  :=
  \lift{(x)\substp{Q}{P}}{ R \psubstp{Q}{P} } \\
%   (\dropn{x})  \psubstp{Q}{P}       
%   := 
%   \left\{ 
%     \begin{array}{ccc} 
%       \dropn{\quotep{Q}} & & x \nameeq \quotep{P} \\
%       \dropn{x} & & otherwise \\
%     \end{array}
%   \right. 
  (\dropn{x})  \psubstp{Q}{P}       
  := 
  \left\{ 
    \begin{array}{ccc} 
      Q & & x \nameeq \quotep{P} \\
      \dropn{x} & & otherwise \\
    \end{array}
  \right.
\end{mathpar}
 

where

\begin{eqnarray}
  (x)\id{\{} \lpquote Q \rpquote / \lpquote P \rpquote \id{\}}            = 
  \left\{ 
    \begin{array}{ccc}
      \lpquote Q \rpquote & & x \nameeq \lpquote P \rpquote \\
      x & & otherwise \\
    \end{array}
  \right. \nonumber
\end{eqnarray}

and $z$ is chosen distinct from $\quotep{P}$, $\quotep{Q}$, the free
names in $Q$, and all the names in $R$. Our $\alpha$-equivalence will
be built in the standard way from this substitution.

\begin{remark}\label{rem:no_self_referential_names}
  One consequence of these definitions is that $\forall P. \quotep{P}
  \not\in \freenames{P}$.
\end{remark}

\subsection{ Dynamic quote: an example }

Anticipating something of what's to come, consider applying the
substitution, $\widehat{\id{\{}u / z \id{\}}}$, to the following pair
of processes, $\lift{w}{y!(z)}$ and $w[ \lpquote y!(z) \rpquote ]$.

\begin{eqnarray}
	\lift{w}{y!(z)}\widehat{\id{\{}u / z \id{\}}}
		& = &
		\lift{w}{y!(u)} \nonumber\\
	w[ \lpquote y!(z) \rpquote ] \widehat{ \id{\{}u / z \id{\}} }
		& = &
		w[ \lpquote y!(z) \rpquote ] \nonumber
\end{eqnarray}

Because the body of the process between quotes is impervious to
substitution, we get radically different answers. In fact, by
examining the first process in an input context,
e.g. $x?(z).\lift{w}{y!(z)}$, we see that the process under the lift
operator may be shaped by prefixed inputs binding a name inside it. In
this sense, the lift operator will be seen as a way to dynamically
construct processes before reifying them as names.

Finally equipped with these standard features we can present the
dynamics of the calculus.

\subsubsection{Operational semantics} 

Finally, we introduce the computational dynamics. What marks these
algebras as distinct from other more traditionally studied algebraic
structures, e.g. vector spaces or polynomial rings, is the manner in
which dynamics is captured. In traditional structures, dynamics is typically
expressed through morphisms between such structures, as in linear maps
between vector spaces or morphisms between rings. In algebras
associated with the semantics of computation, the dynamics is
expressed as part of the algebraic structure itself, through a
reduction reduction relation typically denoted by $\red$. Below, we
give a recursive presentation of this relation for the calculus used
in the encoding.

$\red \subseteq \pi \times \pi$
$\red : \pi \to \mathcal{P}(\pi)$

\begin{mathpar}
  \inferrule* [lab=Comm] { \textsf{match}( x_{src}, x_{trgt} ) } { x_{trgt}?(y)P \; | \; x_{src}!\langle {Q} \rangle \red P\{\quotep{Q}/y}\} }
  \and \\
  \inferrule* [lab=Par] {{P} \red {P}'} {{{P} | {Q}} \red {{P}' | {Q}}}
  \and
  \inferrule* [lab=Equiv]{{{P} \scong {P}'} \andalso {{P}' \red {Q}'} \andalso {{Q}' \scong {Q}}}{{P} \red {Q}}
\end{mathpar}

\begin{eqnarray*}
  match_{\equiv} (\quotep{P},\quotep{Q}) & := & P \equiv Q \\
  match_{\dagger}(\quotep{P},\quotep{Q}) & := & \forall R. P|Q \red^{*} R => R \red^{*} 0 \\
  match_{K}(\quotep{P},\quotep{Q}) & := & K \mbox{ for some context } K
\end{eqnarray*}

$u?(x)P | u!\langle Q \rangle \red P\{\quotep{Q}/x\}$

%We write $\wred$ for $\red^*$, and $P\red$ if $\exists Q $ such that $ P \red Q$.
We write $P\red$ if $\exists Q $ such that $ P \red Q$ and $P\not\red$, otherwise.

\section{Replication}

As mentioned before, it is known that replication (and hence
recursion) can be implemented in a higher-order process algebra
\cite{SangiorgiWalker}. As our first example of calculation with the
machinery thus far presented we give the construction explicitly in
the {\rhoc}.

\begin{eqnarray}
	D_{x} & := & \prefix{x}{y}{(\binpar{\outputp{x}{y}}{@{y}})} \nonumber\\
	\bangp_{x}{P} & := & \binpar{{x}!\langle{\binpar{D_{x}}{P}}\rangle}{D_{x}} \nonumber
\end{eqnarray}

\begin{eqnarray}
	\bangp_{x}{P} & & \nonumber\\
	=
	& {x}!\langle{(\prefix{x}{y}{(\outputp{x}{y} | @{y})) | P}}\rangle 
	      | \prefix{x}{y}{(\outputp{x}{y} | @{y})} & \nonumber\\
	\red
	& (\outputp{x}{y} | @{y})\substn{\quotep{(\prefix{x}{y}{(@{y} | \outputp{x}{y})) | P}}}{y} & \nonumber\\
	=
	& \outputp{x}{\quotep{(\prefix{x}{y}{(\outputp{x}{y} | @{y})) | P}}}
	  | {(\prefix{x}{y}{(\outputp{x}{y} | @{y})) | P}} & \nonumber\\
	\red
	& \ldots & \nonumber\\
	\red^*
	& P | P | \ldots & \nonumber
\end{eqnarray}

Of course, this encoding, as an implementation, runs away, unfolding
$\bangp{P}$ eagerly. A lazier and more implementable replication
operator, restricted to input-guarded processes, may be obtained as follows.

\begin{eqnarray}
\bangp{\prefix{u}{v}{P}} 
	:= 
	\binpar{\lift{x}{\prefix{u}{v}{(\binpar{D(x)}{P})}}}{D(x)} \nonumber
\end{eqnarray}

\begin{remark}
  Note that the lazier definition still does not deal with summation
  or mixed summation (i.e. sums over input and output). The reader is
  invited to construct definitions of replication that deal with these
  features. 

  Further, the definitions are parameterized in a name, $x$. Can you,
  gentle reader, make a definition that eliminates this parameter and
  guarantees no accidental interaction between the replication
  machinery and the process being replicated -- i.e. no accidental
  sharing of names used by the process to get its work done and the
  name(s) used by the replication to effect copying. This latter
  revision of the definition of replication is crucial to obtaining
  the expected identity $!!P \sim !P$.
\end{remark}

\begin{remark}\label{rem:paradoxical_combinator}
  The reader familiar with the lambda calculus will have noticed the
  similarity between $D$ and the paradoxical combinator.

  [Ed. note: the existence of this seems to suggest we have to be more
  restrictive on the set of processes and names we admit if we are to
  support no-cloning.]
\end{remark}

\subsubsection{Bisimulation}

The computational dynamics gives rise to another kind of equivalence,
the equivalence of computational behavior. As previously mentioned
this is typically captured \emph{via} some form of bisimulation.

% The notion we use in this paper is weak barbed bisimulation
% \cite{milner91polyadicpi}.

The notion we use in this paper is derived from weak barbed
bisimulation \cite{milner91polyadicpi}. 

\begin{definition}
An \emph{observation relation}, $\downarrow_{\mathcal N}$, over a set
of names, $\mathcal N$, is the smallest relation satisfying the rules
below.

\infrule[Out-barb]{y \in {\mathcal N}, \; x \nameeq y}
		  {\outputp{x}{v} \downarrow_{\mathcal N} x}
\infrule[Par-barb]{\mbox{$P\downarrow_{\mathcal N} x$ or $Q\downarrow_{\mathcal N} x$}}
		  {\binpar{P}{Q} \downarrow_{\mathcal N} x}

We write $P \Downarrow_{\mathcal N} x$ if there is $Q$ such that 
$P \wred Q$ and $Q \downarrow_{\mathcal N} x$.
\end{definition}

\begin{definition}
%\label{def.bbisim}
An  ${\mathcal N}$-\emph{barbed bisimulation} over a set of names, ${\mathcal N}$, is a symmetric binary relation 
${\mathcal S}_{\mathcal N}$ between agents such that $P\rel{S}_{\mathcal N}Q$ implies:
\begin{enumerate}
\item If $P \red P'$ then $Q \wred Q'$ and $P'\rel{S}_{\mathcal N} Q'$.
\item If $P\downarrow_{\mathcal N} x$, then $Q\Downarrow_{\mathcal N} x$.
\end{enumerate}
$P$ is ${\mathcal N}$-barbed bisimilar to $Q$, written
$P \wbbisim_{\mathcal N} Q$, if $P \rel{S}_{\mathcal N} Q$ for some ${\mathcal N}$-barbed bisimulation ${\mathcal S}_{\mathcal N}$.
\end{definition}

$\mathcal{R} \subseteq \pi \times \pi$

$P \mathcal{R} Q => \forall P'. P \red P' \Rightarrow \exists Q'. Q \red Q', P' \mathcal{R} Q'$

$P \vdash x \Rightarrow Q \vdash x$

\begin{mathpar}
  \inferrule*[lab=Out-barb]{x \nameeq y}{{y}!\langle{Q}\rangle \vdash x}
  \and
  \inferrule*[lab=Par-barb]{\mbox{$P\vdash x$ or $Q\vdash x$}}{\binpar{P}{Q} \vdash x}
\end{mathpar}

\subsubsection{Contexts}

One of the principle advantages of computational calculi like the
$\pi$-calculus is a well-defined notion of context,
contextual-equivalence and a correlation between
contextual-equivalence and notions of bisimulation. The notion of
context allows the decomposition of a process into (sub-)process and
its syntactic environment, its context. Thus, a context may be
thought of as a process with a ``hole'' (written $\Box$) in it. The
application of a context $M$ to a process $P$, written $M[P]$, is
tantamount to filling the hole in $M$ with $P$. In this paper we do
not need the full weight of this theory, but do make use of the notion
of context in the proof the main theorem. 

\begin{mathpar}
  \inferrule* [lab=summation] {} {{M_{M},M_{N}} \bc \Box \;|\; x.M_{A} \;|\; M_{M}+M_{N}}
  \and
  \inferrule* [lab=agent] {} {{M_{A}} \bc (\vec{x})M_{P} \;| \; \clift{P_0,\ldots,M_{P},\ldots,P_N}}
  \and \\
  \inferrule* [lab=process] {} {{M_{P}} \bc M_{N} \;| \;P|M_{P} }
\end{mathpar} 

\begin{mathpar}
  \inferrule* [lab=sychronization] {} {M_{N} \bc \Box \;|\; x?M_{F} \;|\; x!M_{C}}
  \and
  \inferrule* [lab=abstraction] {} {{M_{F}} \bc (x)M_{P} }
  \and
  \inferrule* [lab=concretion] {} {{M_{C}} \bc \langle M_{P} \rangle }
  \and \\
  \inferrule* [lab=process] {} {{M_{P}} \bc M_{N} \;| \;P|M_{P} }
\end{mathpar}

\begin{definition}[contextual application] Given a context $M$, and
  process $P$, we define the \emph{contextual application}, $M[P] :=
  M\{P/\Box\}$. That is, the contextual application of M to P is the
  substitution of $P$ for $\Box$ in $M$.
\end{definition}

$\meaningof{-} : L \to \mathcal{P}(\pi)$

\begin{mathpar}
  \inferrule* [lab=collection] {} {\meaningof{true} = \pi, \and \meaningof{~E} = \pi \setminus \meaningof{E}, \and \meaningof{E_{1} \& E_{2}} = \meaningof{E_{1}} \cap \meaningof{E_{2}}}
\end{mathpar}

\begin{mathpar}
  \inferrule* [lab=structure] {} {\meaningof{0} = \{ P \in \pi | P \equiv 0 \}, \and \\ \meaningof{E_1 | E_2} = \{ P \in \pi | P \equiv P_{1} | P_{2}, P_{1} \in \meaningof{E_{1}}, P_{2} \in \meaningof{E_2}\} }
\end{mathpar}

\begin{mathpar}
 \inferrule* [lab=behavior] {} {\meaningof{\langle a?b \rangle E} = \{ P \in \pi | P \equiv Q | u?(y)P', \\ \and \\\\ \and \\ \;\;\; u \in \meaningof{a}, \forall z.P'\{z/y\} \in \meaningof{E\{z/b\}}\}, \and \\ \meaningof{a!E} = \{ P \in \pi | P \equiv Q | x!\langle P' \rangle, x \in \meaningof{a} P' \in \meaningof{E}\} }
\end{mathpar}

\begin{mathpar}
 \inferrule* [lab=nominal] {} {\meaningof{\quotep{E}} = \{ \quotep{P} \in \quotep{\pi} | P \in \meaningof{E} \}, \and \meaningof{\quotep{P}} = \{ \quotep{Q} \in \quotep{\pi} | P \equiv Q \} \and \\ \meaningof{@\quotep{E}} = \{ P \in \pi | P \equiv @x, x \in \meaningof{E} \}}
\end{mathpar}

\begin{eqnarray*}
  \\
  \meaningof{-} : TS \to ST
\end{eqnarray*}

\begin{eqnarray*}
  \\
  L : TS \to ST
\end{eqnarray*}

\begin{eqnarray*}
  \\
  P \models E \iff P \in \meaningof{E}
\end{eqnarray*}

\begin{eqnarray*}
  P \approx_{L} Q \iff \forall E \in L. P \models E \iff Q \models E
\end{eqnarray*}

\begin{eqnarray*}
  P \approx_{K} Q
\end{eqnarray*}

\begin{eqnarray*}
  P \approx Q
\end{eqnarray*}

$\approx_{K} = \approx = \approx_{L}$

\subsubsection{Contextual duality}

Note that contexts extend the quotation operation to a family of
operations from processes to names. Given a context, $M$, we can
define a \emph{nominal context}, $\quotep{M}$ by $\quotep{M}[P] :=
\quotep{M[P]}$. To foreshadow what is to come we observe that these
operations enjoy a duality with processes very much like the duality
between vectors and maps from vectors to scalars.

Further, because the calculus is essentially higher-order, we have a
correspondence between contexts and processes. More specifically,
given a name $x$ and a context $M$ we can construct $M^{*}_{x}$ such
that 

\begin{mathpar}
  M^{*}_{x} | \lift{x}{P} \red M[P]
\end{mathpar}

namely,

\begin{mathpar}
  M^{*}_{x} := x?(u).M[\dropn{u}]
\end{mathpar}

The dependence of $M^{*}_{x}$ on a name makes it an abstraction, 

\begin{mathpar}
  M^{*} := (x)x?(u).M[\dropn{u}]
\end{mathpar}

\subsection{Additional notation}

It will sometimes be convenient to denote the process a name
quotes. We already have the notation $x = \quotep{P}$, but it will be
convenient to introduce an alternate notation, $\procn{x}$, when we
want to emphasize the connection to the use of the name. Note that, by
virtue of name equivalence, $\quotep{\procn{x}} \nameeq x$; so, the
notation is consistent with previous definitions.

Further, because names have structure it is possible to effect
substitutions on the basis of that structure. This means we need to
upgrade our notation for substitutions, which we accomplish by
adapting comprehension notation. Thus,

\begin{mathpar}
  P\{ y / x : x \in S \}
\end{mathpar}

is interpreted to mean the process derived from P by replacing (in a
capture-avoiding manner) each occurrence of $x$ in $S$ by $y$. For example,

\begin{mathpar}
  P\{ \quotep{\procn{x}|\procn{x}} / x : x \in \freenames{P} \}
\end{mathpar}

will replace each (occurrence) of a free name $x$ in $P$ by
$\quotep{\procn{x}|\procn{x}}$.

Also, we will avail ourselves of the notation $x^{L}$ and $x^{R}$ to
denote injections of a name into disjoint copies of the name
space. There are numerous ways to accomplish this. One example can be
found in \cite{MeredithR05}. This notation overloads to vectors of
names: $\vec{x}^{\pi} := (x_{i}^{\pi} \; : \; 0 \leq i < |\vec{x}| )$ where $\pi \in \{L,R\}$.

We also use $P^{\Box} := P|\Box$.

In \cite{MeredithR05} an interpretation of the new operator is
given. It turns out that there are several possible interpretations
all enjoying the requisite algebraic properties of the operator (see
\cite{milner91polyadicpi}). We will therefore make liberal use of
$(\nu\; \vec{x})P$.

% subsection the_syntax_and_semantics_of_the_notation_system (end)   

\input{qm2pi.qmops} 

\input{qm2pi.sterngerlach} 

\input{qm2pi.metric} 

% section concurrent_process_calculi (end)

%\input{qm2pi.proofsketch}

% section proof sketch (end)

%\input{qm2pi.slviaknots} 

% section spatial logic via knots (end)

\input{qm2pi.conclusion}

% section conclusion (end)

%\input{qm2pi.dtcodes} 

% section wiring algorithm (end)

\input{qm2pi.ack} 

% section acknowledgments (end)

\newpage


\bibliographystyle{plain}   
\bibliography{../../biblios/main.bib}

\input{qm2pi.rhodetails}

\end{document}

 

%\ifpdf
%\usepackage[pdftex]{graphicx}
%\else
%\usepackage{graphicx}
%\fi

 % \ifpdf
%  \usepackage{pdfsync}
%  \if


%\title{Brief Article}
%\author{David F. Snyder}
%\author{L.G. Meredith}

%\address{Dept. of Math., Texas State University--San Marcos, San Marcos, TX 78666}
       
\pagestyle{empty}


\begin{document}

\lstset{language=[Objective]Caml,frame=shadowbox}

\documentclass[12pt]{llncs}
%\documentclass{jktr}

\usepackage[pdftex]{hyperref}                   
\usepackage {listings}
\usepackage {mathpartir}
\usepackage{bcprules}
%\usepackage{listings}
                       
\usepackage{graphicx} 
%\usepackage[margins=2.5cm,nohead,nofoot]{geometry}
%\usepackage{geometry}
\usepackage{amsfonts}
\usepackage{amstext}
\usepackage{latexsym}
\usepackage{amssymb}
\usepackage{color}


%\include{myPreamble}
\include{qm2pi.local} 

%\ifpdf
%\usepackage[pdftex]{graphicx}
%\else
%\usepackage{graphicx}
%\fi

 % \ifpdf
%  \usepackage{pdfsync}
%  \if


%\title{Brief Article}
%\author{David F. Snyder}
%\author{L.G. Meredith}

%\address{Dept. of Math., Texas State University--San Marcos, San Marcos, TX 78666}
       
\pagestyle{empty}


\begin{document}

\lstset{language=[Objective]Caml,frame=shadowbox}

\input{qm2pi.front}

% section front matter (end)

\input{qm2pi.intro} 
 
% section introduction (end)

% \input{qm2pi.knotations} 

% section notation (end)

\input{qm2pi.process.calculi} 

% section concurrent_process_calculi_and_spatial_logics_ (end)
    
%\input{qm2pi.knots2pi} 

%\input{qm2pi.trefoil} 

%\input{qm2pi.mainthm} 

% subsection basic_interpretation (end)

%\input{qm2pi.rho.presentation} 
\subsection{The syntax and semantics of the notation system}\label{sub:the_syntax_and_semantics_of_the_notation_system} % (fold)

We now summarize a technical presentation of the calculus that
embodies our theory of dynamics. The typical presentation of such a
calculus follows the style of giving generators and relations on
them. The grammar, below, describing term constructors, freely
generates the set of processes, $\Proc$. This set is then quotiented
by a relation known as structural congruence and it is over this set
that the notion of dynamics is expressed. This presentation is
essentially that of \cite{MeredithR05} with the addition of
polyadicity and summation. For readability we have relegated some of
the technical subtleties to an appendix.

\subsubsection{Process grammar}\label{subsub:process_grammar}

\begin{mathpar}
  \inferrule* [lab=synchronization] {} {{M} \bc \pzero \;|\; x?F \;|\; x!C }
  \and
  \inferrule* [lab=abstraction] {} {{F} \bc (x)P}
  \and
  \inferrule* [lab=concretion] {} {{C} \bc \langle Q \rangle}
  \and
  \inferrule* [lab=process] {} {{P,Q} \bc M \;| \;P|Q \;|\; @{x}}
  \and
  \inferrule* [lab=name] {} {{x} \bc \quotep{P}}
\end{mathpar} 

Note that $\vec{x}$ (resp. $\vec{P}$) denotes a vector of names
(resp. processes) of length $|\vec{x}|$ (resp. $|\vec{P}|$). We adopt
the following useful abbreviations.

\begin{mathpar}
   x?(\vec{y}).P := x.(\vec{y})P \and  x\clift{\vec{P}} := x.\clift{\vec{P}}
   \and x!(y) := \lift{x}{\dropn{y}}
   \and \Pi_{i=0}^{n-1}P_i := P_0 | \ldots | P_{n-1}
\end{mathpar}

\subsubsection{Structural congruence}

\paragraph{Free and bound names and alpha-equivalence.} At the
core of structural equivalence is alpha-equivalence which identifies
process that are the same up to a change of variable. Formally, we
recognize the distinction between free and bound names. The free names
of a process, $\freenames{P}$, may be calculated recursively as
follows:

\begin{mathpar}
\freenames{\pzero} := \emptyset
  \and \\
  \freenames{x?(y).P} := \{ x \} \cup (\freenames{P} \setminus \{ y \})
  \and 
  \freenames{x!\langle P \rangle} := \{ x \} \cup \{ P \} 
  \and \\
  \freenames{P|Q} := \freenames{P} \cup \freenames{Q}
  \and \\
  \freenames{@{x}} := \{ x \}
\end{mathpar}

$\pi$
$\quotep{\pi}$

$\freenames{-} : \pi \to \mathcal{P}(\quotep{\pi})$

\begin{eqnarray*}
  \freenames{\pzero} & := & \emptyset \\
  \freenames{x?(y).P} & := & \{ x \} \cup (\freenames{P} \setminus \{ y \}) \\
  \freenames{x!\langle P \rangle} & := & \{ x \} \cup \{ P \} \\
  \freenames{P|Q} & := & \freenames{P} \cup \freenames{Q} \\
  \freenames{\dropn{x}} & := & \{ x \}
\end{eqnarray*}

The bound names of a process, $\boundnames{P}$, are those names occurring in $P$
that are not free. For example, in $x?(y).0$, the name $x$ is free, while $y$ is bound.

\begin{mathpar}
  \inferrule* [lab=monoidal-laws] {} { P|Q \equiv Q|P \and P|0 \equiv P \and P|(Q|R) \equiv (P|Q)|R }
\end{mathpar}

\begin{mathpar}
  \inferrule* [lab=alpha-equivalence] {} { (x)P \equiv (y)P\{y/x\} \and y \not\in \freenames{P} }
\end{mathpar}

\begin{definition}
Then two processes, $P,Q$, are alpha-equivalent if $P = Q\{\vec{y}/\vec{x}\}$ for
some $\vec{x} \in \boundnames{Q},\vec{y} \in \boundnames{P}$, where $Q\{\vec{y}/\vec{x}\}$
denotes the capture-avoiding substitution of $\vec{y}$ for $\vec{x}$ in $Q$.
\end{definition}

\begin{definition}
  The {\em structural congruence} \cite{SangiorgiWalker} , $\equiv$,
  between processes is the least congruence containing
  alpha-equivalence, satisfying the abelian monoid laws
  (associativity, commutativity and $\pzero$ as identity) for parallel
  composition $|$ and for summation $+$.
\end{definition}

\subsection{Name equivalence}

We take name equivalence, written $\nameeq$, to be the smallest
equivalence relation generated by the following rules.

\begin{mathpar}
\inferrule*[lab=Quote-drop]
{ }
{ \quotep{@{x}} \nameeq x }

\inferrule*[lab=Struct-equiv]
{ P \scong Q }
{ \quotep{P} \nameeq \quotep{Q} }
\end{mathpar}

The astute reader will have noticed that the mutual recursion of names
and processes imposes a mutual recursion on alpha-equivalence and
structural equivalence via name-equivalence. Fortunately, all of this
works out pleasantly and we may calculate in the natural way, free of
concern. The reader interested in the details is referred to the
appendix \ref{appendix:rho_details}.

\subsection{Substitution}

We use $\Proc$ for the set of processes, $\QProc$ for the set of
names, and $\id{\{}\vec{y} / \vec{x} \id{\}}$ to denote partial maps,
$s : \QProc \rightarrow \QProc$. A map, $s$ lifts, uniquely, to a map
on process terms, $\widehat{s} : \Proc \rightarrow \Proc$ by the
following equations.

\begin{mathpar}
  (0) \psubstp{Q}{P} := 0 \\
  (R \juxtap S) \psubstp{Q}{P}
  :=    
  (R)\psubstp{Q}{P} \juxtap (S) \psubstp{Q}{P} \\
  (x?(y).R) \psubstp{Q}{P}    
  :=    
  (x)\substp{Q}{P} (z)\concat( (R \psubstn{z}{y}) \psubstp{Q}{P} ) \\
  (\lift{x}{R}) \psubstp{Q}{P}  
  :=
  \lift{(x)\substp{Q}{P}}{ R \psubstp{Q}{P} } \\
%   (\dropn{x})  \psubstp{Q}{P}       
%   := 
%   \left\{ 
%     \begin{array}{ccc} 
%       \dropn{\quotep{Q}} & & x \nameeq \quotep{P} \\
%       \dropn{x} & & otherwise \\
%     \end{array}
%   \right. 
  (\dropn{x})  \psubstp{Q}{P}       
  := 
  \left\{ 
    \begin{array}{ccc} 
      Q & & x \nameeq \quotep{P} \\
      \dropn{x} & & otherwise \\
    \end{array}
  \right.
\end{mathpar}
 

where

\begin{eqnarray}
  (x)\id{\{} \lpquote Q \rpquote / \lpquote P \rpquote \id{\}}            = 
  \left\{ 
    \begin{array}{ccc}
      \lpquote Q \rpquote & & x \nameeq \lpquote P \rpquote \\
      x & & otherwise \\
    \end{array}
  \right. \nonumber
\end{eqnarray}

and $z$ is chosen distinct from $\quotep{P}$, $\quotep{Q}$, the free
names in $Q$, and all the names in $R$. Our $\alpha$-equivalence will
be built in the standard way from this substitution.

\begin{remark}\label{rem:no_self_referential_names}
  One consequence of these definitions is that $\forall P. \quotep{P}
  \not\in \freenames{P}$.
\end{remark}

\subsection{ Dynamic quote: an example }

Anticipating something of what's to come, consider applying the
substitution, $\widehat{\id{\{}u / z \id{\}}}$, to the following pair
of processes, $\lift{w}{y!(z)}$ and $w[ \lpquote y!(z) \rpquote ]$.

\begin{eqnarray}
	\lift{w}{y!(z)}\widehat{\id{\{}u / z \id{\}}}
		& = &
		\lift{w}{y!(u)} \nonumber\\
	w[ \lpquote y!(z) \rpquote ] \widehat{ \id{\{}u / z \id{\}} }
		& = &
		w[ \lpquote y!(z) \rpquote ] \nonumber
\end{eqnarray}

Because the body of the process between quotes is impervious to
substitution, we get radically different answers. In fact, by
examining the first process in an input context,
e.g. $x?(z).\lift{w}{y!(z)}$, we see that the process under the lift
operator may be shaped by prefixed inputs binding a name inside it. In
this sense, the lift operator will be seen as a way to dynamically
construct processes before reifying them as names.

Finally equipped with these standard features we can present the
dynamics of the calculus.

\subsubsection{Operational semantics} 

Finally, we introduce the computational dynamics. What marks these
algebras as distinct from other more traditionally studied algebraic
structures, e.g. vector spaces or polynomial rings, is the manner in
which dynamics is captured. In traditional structures, dynamics is typically
expressed through morphisms between such structures, as in linear maps
between vector spaces or morphisms between rings. In algebras
associated with the semantics of computation, the dynamics is
expressed as part of the algebraic structure itself, through a
reduction reduction relation typically denoted by $\red$. Below, we
give a recursive presentation of this relation for the calculus used
in the encoding.

$\red \subseteq \pi \times \pi$
$\red : \pi \to \mathcal{P}(\pi)$

\begin{mathpar}
  \inferrule* [lab=Comm] { \textsf{match}( x_{src}, x_{trgt} ) } { x_{trgt}?(y)P \; | \; x_{src}!\langle {Q} \rangle \red P\{\quotep{Q}/y}\} }
  \and \\
  \inferrule* [lab=Par] {{P} \red {P}'} {{{P} | {Q}} \red {{P}' | {Q}}}
  \and
  \inferrule* [lab=Equiv]{{{P} \scong {P}'} \andalso {{P}' \red {Q}'} \andalso {{Q}' \scong {Q}}}{{P} \red {Q}}
\end{mathpar}

\begin{eqnarray*}
  match_{\equiv} (\quotep{P},\quotep{Q}) & := & P \equiv Q \\
  match_{\dagger}(\quotep{P},\quotep{Q}) & := & \forall R. P|Q \red^{*} R => R \red^{*} 0 \\
  match_{K}(\quotep{P},\quotep{Q}) & := & K \mbox{ for some context } K
\end{eqnarray*}

$u?(x)P | u!\langle Q \rangle \red P\{\quotep{Q}/x\}$

%We write $\wred$ for $\red^*$, and $P\red$ if $\exists Q $ such that $ P \red Q$.
We write $P\red$ if $\exists Q $ such that $ P \red Q$ and $P\not\red$, otherwise.

\section{Replication}

As mentioned before, it is known that replication (and hence
recursion) can be implemented in a higher-order process algebra
\cite{SangiorgiWalker}. As our first example of calculation with the
machinery thus far presented we give the construction explicitly in
the {\rhoc}.

\begin{eqnarray}
	D_{x} & := & \prefix{x}{y}{(\binpar{\outputp{x}{y}}{@{y}})} \nonumber\\
	\bangp_{x}{P} & := & \binpar{{x}!\langle{\binpar{D_{x}}{P}}\rangle}{D_{x}} \nonumber
\end{eqnarray}

\begin{eqnarray}
	\bangp_{x}{P} & & \nonumber\\
	=
	& {x}!\langle{(\prefix{x}{y}{(\outputp{x}{y} | @{y})) | P}}\rangle 
	      | \prefix{x}{y}{(\outputp{x}{y} | @{y})} & \nonumber\\
	\red
	& (\outputp{x}{y} | @{y})\substn{\quotep{(\prefix{x}{y}{(@{y} | \outputp{x}{y})) | P}}}{y} & \nonumber\\
	=
	& \outputp{x}{\quotep{(\prefix{x}{y}{(\outputp{x}{y} | @{y})) | P}}}
	  | {(\prefix{x}{y}{(\outputp{x}{y} | @{y})) | P}} & \nonumber\\
	\red
	& \ldots & \nonumber\\
	\red^*
	& P | P | \ldots & \nonumber
\end{eqnarray}

Of course, this encoding, as an implementation, runs away, unfolding
$\bangp{P}$ eagerly. A lazier and more implementable replication
operator, restricted to input-guarded processes, may be obtained as follows.

\begin{eqnarray}
\bangp{\prefix{u}{v}{P}} 
	:= 
	\binpar{\lift{x}{\prefix{u}{v}{(\binpar{D(x)}{P})}}}{D(x)} \nonumber
\end{eqnarray}

\begin{remark}
  Note that the lazier definition still does not deal with summation
  or mixed summation (i.e. sums over input and output). The reader is
  invited to construct definitions of replication that deal with these
  features. 

  Further, the definitions are parameterized in a name, $x$. Can you,
  gentle reader, make a definition that eliminates this parameter and
  guarantees no accidental interaction between the replication
  machinery and the process being replicated -- i.e. no accidental
  sharing of names used by the process to get its work done and the
  name(s) used by the replication to effect copying. This latter
  revision of the definition of replication is crucial to obtaining
  the expected identity $!!P \sim !P$.
\end{remark}

\begin{remark}\label{rem:paradoxical_combinator}
  The reader familiar with the lambda calculus will have noticed the
  similarity between $D$ and the paradoxical combinator.

  [Ed. note: the existence of this seems to suggest we have to be more
  restrictive on the set of processes and names we admit if we are to
  support no-cloning.]
\end{remark}

\subsubsection{Bisimulation}

The computational dynamics gives rise to another kind of equivalence,
the equivalence of computational behavior. As previously mentioned
this is typically captured \emph{via} some form of bisimulation.

% The notion we use in this paper is weak barbed bisimulation
% \cite{milner91polyadicpi}.

The notion we use in this paper is derived from weak barbed
bisimulation \cite{milner91polyadicpi}. 

\begin{definition}
An \emph{observation relation}, $\downarrow_{\mathcal N}$, over a set
of names, $\mathcal N$, is the smallest relation satisfying the rules
below.

\infrule[Out-barb]{y \in {\mathcal N}, \; x \nameeq y}
		  {\outputp{x}{v} \downarrow_{\mathcal N} x}
\infrule[Par-barb]{\mbox{$P\downarrow_{\mathcal N} x$ or $Q\downarrow_{\mathcal N} x$}}
		  {\binpar{P}{Q} \downarrow_{\mathcal N} x}

We write $P \Downarrow_{\mathcal N} x$ if there is $Q$ such that 
$P \wred Q$ and $Q \downarrow_{\mathcal N} x$.
\end{definition}

\begin{definition}
%\label{def.bbisim}
An  ${\mathcal N}$-\emph{barbed bisimulation} over a set of names, ${\mathcal N}$, is a symmetric binary relation 
${\mathcal S}_{\mathcal N}$ between agents such that $P\rel{S}_{\mathcal N}Q$ implies:
\begin{enumerate}
\item If $P \red P'$ then $Q \wred Q'$ and $P'\rel{S}_{\mathcal N} Q'$.
\item If $P\downarrow_{\mathcal N} x$, then $Q\Downarrow_{\mathcal N} x$.
\end{enumerate}
$P$ is ${\mathcal N}$-barbed bisimilar to $Q$, written
$P \wbbisim_{\mathcal N} Q$, if $P \rel{S}_{\mathcal N} Q$ for some ${\mathcal N}$-barbed bisimulation ${\mathcal S}_{\mathcal N}$.
\end{definition}

$\mathcal{R} \subseteq \pi \times \pi$

$P \mathcal{R} Q => \forall P'. P \red P' \Rightarrow \exists Q'. Q \red Q', P' \mathcal{R} Q'$

$P \vdash x \Rightarrow Q \vdash x$

\begin{mathpar}
  \inferrule*[lab=Out-barb]{x \nameeq y}{{y}!\langle{Q}\rangle \vdash x}
  \and
  \inferrule*[lab=Par-barb]{\mbox{$P\vdash x$ or $Q\vdash x$}}{\binpar{P}{Q} \vdash x}
\end{mathpar}

\subsubsection{Contexts}

One of the principle advantages of computational calculi like the
$\pi$-calculus is a well-defined notion of context,
contextual-equivalence and a correlation between
contextual-equivalence and notions of bisimulation. The notion of
context allows the decomposition of a process into (sub-)process and
its syntactic environment, its context. Thus, a context may be
thought of as a process with a ``hole'' (written $\Box$) in it. The
application of a context $M$ to a process $P$, written $M[P]$, is
tantamount to filling the hole in $M$ with $P$. In this paper we do
not need the full weight of this theory, but do make use of the notion
of context in the proof the main theorem. 

\begin{mathpar}
  \inferrule* [lab=summation] {} {{M_{M},M_{N}} \bc \Box \;|\; x.M_{A} \;|\; M_{M}+M_{N}}
  \and
  \inferrule* [lab=agent] {} {{M_{A}} \bc (\vec{x})M_{P} \;| \; \clift{P_0,\ldots,M_{P},\ldots,P_N}}
  \and \\
  \inferrule* [lab=process] {} {{M_{P}} \bc M_{N} \;| \;P|M_{P} }
\end{mathpar} 

\begin{mathpar}
  \inferrule* [lab=sychronization] {} {M_{N} \bc \Box \;|\; x?M_{F} \;|\; x!M_{C}}
  \and
  \inferrule* [lab=abstraction] {} {{M_{F}} \bc (x)M_{P} }
  \and
  \inferrule* [lab=concretion] {} {{M_{C}} \bc \langle M_{P} \rangle }
  \and \\
  \inferrule* [lab=process] {} {{M_{P}} \bc M_{N} \;| \;P|M_{P} }
\end{mathpar}

\begin{definition}[contextual application] Given a context $M$, and
  process $P$, we define the \emph{contextual application}, $M[P] :=
  M\{P/\Box\}$. That is, the contextual application of M to P is the
  substitution of $P$ for $\Box$ in $M$.
\end{definition}

$\meaningof{-} : L \to \mathcal{P}(\pi)$

\begin{mathpar}
  \inferrule* [lab=collection] {} {\meaningof{true} = \pi, \and \meaningof{~E} = \pi \setminus \meaningof{E}, \and \meaningof{E_{1} \& E_{2}} = \meaningof{E_{1}} \cap \meaningof{E_{2}}}
\end{mathpar}

\begin{mathpar}
  \inferrule* [lab=structure] {} {\meaningof{0} = \{ P \in \pi | P \equiv 0 \}, \and \\ \meaningof{E_1 | E_2} = \{ P \in \pi | P \equiv P_{1} | P_{2}, P_{1} \in \meaningof{E_{1}}, P_{2} \in \meaningof{E_2}\} }
\end{mathpar}

\begin{mathpar}
 \inferrule* [lab=behavior] {} {\meaningof{\langle a?b \rangle E} = \{ P \in \pi | P \equiv Q | u?(y)P', \\ \and \\\\ \and \\ \;\;\; u \in \meaningof{a}, \forall z.P'\{z/y\} \in \meaningof{E\{z/b\}}\}, \and \\ \meaningof{a!E} = \{ P \in \pi | P \equiv Q | x!\langle P' \rangle, x \in \meaningof{a} P' \in \meaningof{E}\} }
\end{mathpar}

\begin{mathpar}
 \inferrule* [lab=nominal] {} {\meaningof{\quotep{E}} = \{ \quotep{P} \in \quotep{\pi} | P \in \meaningof{E} \}, \and \meaningof{\quotep{P}} = \{ \quotep{Q} \in \quotep{\pi} | P \equiv Q \} \and \\ \meaningof{@\quotep{E}} = \{ P \in \pi | P \equiv @x, x \in \meaningof{E} \}}
\end{mathpar}

\begin{eqnarray*}
  \\
  \meaningof{-} : TS \to ST
\end{eqnarray*}

\begin{eqnarray*}
  \\
  L : TS \to ST
\end{eqnarray*}

\begin{eqnarray*}
  \\
  P \models E \iff P \in \meaningof{E}
\end{eqnarray*}

\begin{eqnarray*}
  P \approx_{L} Q \iff \forall E \in L. P \models E \iff Q \models E
\end{eqnarray*}

\begin{eqnarray*}
  P \approx_{K} Q
\end{eqnarray*}

\begin{eqnarray*}
  P \approx Q
\end{eqnarray*}

$\approx_{K} = \approx = \approx_{L}$

\subsubsection{Contextual duality}

Note that contexts extend the quotation operation to a family of
operations from processes to names. Given a context, $M$, we can
define a \emph{nominal context}, $\quotep{M}$ by $\quotep{M}[P] :=
\quotep{M[P]}$. To foreshadow what is to come we observe that these
operations enjoy a duality with processes very much like the duality
between vectors and maps from vectors to scalars.

Further, because the calculus is essentially higher-order, we have a
correspondence between contexts and processes. More specifically,
given a name $x$ and a context $M$ we can construct $M^{*}_{x}$ such
that 

\begin{mathpar}
  M^{*}_{x} | \lift{x}{P} \red M[P]
\end{mathpar}

namely,

\begin{mathpar}
  M^{*}_{x} := x?(u).M[\dropn{u}]
\end{mathpar}

The dependence of $M^{*}_{x}$ on a name makes it an abstraction, 

\begin{mathpar}
  M^{*} := (x)x?(u).M[\dropn{u}]
\end{mathpar}

\subsection{Additional notation}

It will sometimes be convenient to denote the process a name
quotes. We already have the notation $x = \quotep{P}$, but it will be
convenient to introduce an alternate notation, $\procn{x}$, when we
want to emphasize the connection to the use of the name. Note that, by
virtue of name equivalence, $\quotep{\procn{x}} \nameeq x$; so, the
notation is consistent with previous definitions.

Further, because names have structure it is possible to effect
substitutions on the basis of that structure. This means we need to
upgrade our notation for substitutions, which we accomplish by
adapting comprehension notation. Thus,

\begin{mathpar}
  P\{ y / x : x \in S \}
\end{mathpar}

is interpreted to mean the process derived from P by replacing (in a
capture-avoiding manner) each occurrence of $x$ in $S$ by $y$. For example,

\begin{mathpar}
  P\{ \quotep{\procn{x}|\procn{x}} / x : x \in \freenames{P} \}
\end{mathpar}

will replace each (occurrence) of a free name $x$ in $P$ by
$\quotep{\procn{x}|\procn{x}}$.

Also, we will avail ourselves of the notation $x^{L}$ and $x^{R}$ to
denote injections of a name into disjoint copies of the name
space. There are numerous ways to accomplish this. One example can be
found in \cite{MeredithR05}. This notation overloads to vectors of
names: $\vec{x}^{\pi} := (x_{i}^{\pi} \; : \; 0 \leq i < |\vec{x}| )$ where $\pi \in \{L,R\}$.

We also use $P^{\Box} := P|\Box$.

In \cite{MeredithR05} an interpretation of the new operator is
given. It turns out that there are several possible interpretations
all enjoying the requisite algebraic properties of the operator (see
\cite{milner91polyadicpi}). We will therefore make liberal use of
$(\nu\; \vec{x})P$.

% subsection the_syntax_and_semantics_of_the_notation_system (end)   

\input{qm2pi.qmops} 

\input{qm2pi.sterngerlach} 

\input{qm2pi.metric} 

% section concurrent_process_calculi (end)

%\input{qm2pi.proofsketch}

% section proof sketch (end)

%\input{qm2pi.slviaknots} 

% section spatial logic via knots (end)

\input{qm2pi.conclusion}

% section conclusion (end)

%\input{qm2pi.dtcodes} 

% section wiring algorithm (end)

\input{qm2pi.ack} 

% section acknowledgments (end)

\newpage


\bibliographystyle{plain}   
\bibliography{../../biblios/main.bib}

\input{qm2pi.rhodetails}

\end{document}



% section front matter (end)

\section{Introduction}\label{sec:introduction} % (fold)
In this draft of the material i am going to have to dispense with the
usual writing conventions adopted in papers on these topics. i'm going
to have adopt whatever tone i need at the time i'm writing up the
calculations. Sometimes this may be very conversational; others it may
be the barest mathematical grunts; others still it may be that i have
lifted text from one of my other papers because the exposition of some
point was better said there. i hope that my readers are not unduly put
out by this decision. i'm not doing this to flout convention or be
rebellious. i find these calculations very technically challenging. To
keep everything going technically, something has to give; i have to
let go of some cognitive burden. So, the academic writing style --
with all of its trade-offs in terms of facilitating technical
communication -- is what i'm letting go of. Perhaps subsequent drafts
can be tightened and polished, but for now, i'm going to speak as if
we were sitting together in a coffee shop with a laptop, wifi and a
pad of paper and a pencil.

So, here's what i have to say. We -- you and i, comfortably ensconced
in our coffee shop and well-equipped with our tools -- can realize and
carry out the calculations of quantum mechanics over a very different
formal theory of dynamics, a formal theory of dynamics that
corresponds to a theory of concurrent computation with
\emph{reflection}. It has the advantage that the underlying theory is
already `quantized', but supports analogues all of the continuuous
operations. Strikingly, this underlying theory has recently been
connected with a notion of metric that we can show, by calculating
together, coincides with the metric induced by the inner product.

There are a lot of reasons why you might be interested in seeing
calculations of this form. Here's why i'm interested. For the past
several centuries there has been no competitor to the ``Newtonian''
account of dynamics. As a result the predominant share of accounts of
dynamical systems and situations have had to be formulated in terms of
the Newtonian machinery. i view this as an intellectually dangerous
position to occupy. Everything, despite it's intrinsic shape, turns
into a nail to be hit with this hammer. Recently, however, the theory
of computation has matured to the point where we have candidates for
theories of dynamics that offer very different perspective on
reasoning about dynamical systems and situations. Testing these
candidates against very successful accounts of dynamical situations,
like quantum mechanics, is going to give us some sense of how mature
they are and some measure of the quality of these accounts of
dynamics.

\subsection{Summary of contributions and outline of paper}

So, we're going to develop an interpretation of the operations of
quantum mechanics normally interpreted by Hilbert spaces and
operators. We're going to do this over a theory of computation. Note
that this is very different than the usual quantum computation program
which develops notions of computation over quantum mechanics. Rather,
we are developing a story that aligns with Wheeler's slogan: It from
Bit. To do this we will first provide an account of the theory of
computation at play here. Then we will dive into a calculation-driven
interpretation of the operations of quantum mechanics.

The reason we take this approach is that -- until very recently --
there hasn't been an axiomatic account of quantum mechanics. As a
result there has been no sharp delineation of the mathematical theory
supporting interpretation of the physical theory and the physical
theory, itself. So, ambient features of the maths are free to be
exploited (or supressed) without a real accounting of their physical
relevance. There is no sharp statement ``here's the physical theory''
qua \emph{theory} and ``here's the mathematical interpretation''
enabling a judgment of how faithful the interpretation is -- apart
from experimental observation. When there is an axiomatic account we
can judge how well a given mathematical formalism supports an
interpretation of the axioms, independent of
experimentation. Likewise, we can judge how well we have captured our
physical evidence and experience with our axiomatics, independent of
any specific mathematical implementation, with accidental detail that
may or may not have physical significance. 

In lieu of a fully fleshed out and vetted axiomatic account of quantum
mechanics, interpreting the operational notions in service of modeling
physical systems will have to suffice. In other words, we are not in
the business of providing a model of Hilbert spaces and operators. We
are in the business of providing a model of quantum mechanics because
we are motivated by testing our notions of dynamics against physical
theory; and, the predictive calculations of the physical theory must
serve as the best formulation -- shy of a fully fleshed out axiomatic
account -- of the physical theory itself (as they have for scientific
theories since time immemorial). Put another way, despite a
whole-hearted commitment to an It-from-Bit ontology, we are firmly
aligned with the shut-up-and-calculate camp as the best way to obtain
results either from the physical perspective or as a quality assurance
measure of our fledgling theory of dynamics.

In detail, we present a reflective process calculus. Then we develop
intuitive correspondences between the notions available in this
calculus and the usual physical notions supporting quantum mechanical
calculations. Thus, 

\begin{table}[htp]
  \center{
    \fbox{
      \begin{tabular}{c|c}
        quantum mechanics & process calculus \\
        \hline
        scalar & name \\
        state vector & process \\
        dual & contextual duals \\
        matrix & formal sums of process-context-dual pairs \\
        orthogonality & process annihilation \\
        inner product & execution-formula + quoting
      \end{tabular}
    }
  }
  \caption{QM - process calculi correspondences}
\end{table}

Then we tighten up these intuitions to operational definitions. We
employ the Dirac notation as the best proxy we can find for an
abstract syntax of the quantum mechanical notions. The definitions we
develop put us in contact with equational constraints coming from the
theory that we demonstrate the definitions and calculations satisfy.

This puts us in a position to shut up and calculate for the
Stern-Gerlach experimental set up, showing how these predictive
calculations become calculations on processes in our theory of a
reflective process calculus.

Penultimately, we demonstrate that the notion of metric coming from
the inner product coincides with the notion of metric available from
the theory of bisimulation. This demonstration gives us the right to
think of space as arising from behavior. Finally, we consider where we
might go from the new vantage point we have obtained.

% section introduction (end) 
 
% section introduction (end)

% \documentclass[12pt]{llncs}
%\documentclass{jktr}

\usepackage[pdftex]{hyperref}                   
\usepackage {listings}
\usepackage {mathpartir}
\usepackage{bcprules}
%\usepackage{listings}
                       
\usepackage{graphicx} 
%\usepackage[margins=2.5cm,nohead,nofoot]{geometry}
%\usepackage{geometry}
\usepackage{amsfonts}
\usepackage{amstext}
\usepackage{latexsym}
\usepackage{amssymb}
\usepackage{color}


%\include{myPreamble}
\include{qm2pi.local} 

%\ifpdf
%\usepackage[pdftex]{graphicx}
%\else
%\usepackage{graphicx}
%\fi

 % \ifpdf
%  \usepackage{pdfsync}
%  \if


%\title{Brief Article}
%\author{David F. Snyder}
%\author{L.G. Meredith}

%\address{Dept. of Math., Texas State University--San Marcos, San Marcos, TX 78666}
       
\pagestyle{empty}


\begin{document}

\lstset{language=[Objective]Caml,frame=shadowbox}

\input{qm2pi.front}

% section front matter (end)

\input{qm2pi.intro} 
 
% section introduction (end)

% \input{qm2pi.knotations} 

% section notation (end)

\input{qm2pi.process.calculi} 

% section concurrent_process_calculi_and_spatial_logics_ (end)
    
%\input{qm2pi.knots2pi} 

%\input{qm2pi.trefoil} 

%\input{qm2pi.mainthm} 

% subsection basic_interpretation (end)

%\input{qm2pi.rho.presentation} 
\subsection{The syntax and semantics of the notation system}\label{sub:the_syntax_and_semantics_of_the_notation_system} % (fold)

We now summarize a technical presentation of the calculus that
embodies our theory of dynamics. The typical presentation of such a
calculus follows the style of giving generators and relations on
them. The grammar, below, describing term constructors, freely
generates the set of processes, $\Proc$. This set is then quotiented
by a relation known as structural congruence and it is over this set
that the notion of dynamics is expressed. This presentation is
essentially that of \cite{MeredithR05} with the addition of
polyadicity and summation. For readability we have relegated some of
the technical subtleties to an appendix.

\subsubsection{Process grammar}\label{subsub:process_grammar}

\begin{mathpar}
  \inferrule* [lab=synchronization] {} {{M} \bc \pzero \;|\; x?F \;|\; x!C }
  \and
  \inferrule* [lab=abstraction] {} {{F} \bc (x)P}
  \and
  \inferrule* [lab=concretion] {} {{C} \bc \langle Q \rangle}
  \and
  \inferrule* [lab=process] {} {{P,Q} \bc M \;| \;P|Q \;|\; @{x}}
  \and
  \inferrule* [lab=name] {} {{x} \bc \quotep{P}}
\end{mathpar} 

Note that $\vec{x}$ (resp. $\vec{P}$) denotes a vector of names
(resp. processes) of length $|\vec{x}|$ (resp. $|\vec{P}|$). We adopt
the following useful abbreviations.

\begin{mathpar}
   x?(\vec{y}).P := x.(\vec{y})P \and  x\clift{\vec{P}} := x.\clift{\vec{P}}
   \and x!(y) := \lift{x}{\dropn{y}}
   \and \Pi_{i=0}^{n-1}P_i := P_0 | \ldots | P_{n-1}
\end{mathpar}

\subsubsection{Structural congruence}

\paragraph{Free and bound names and alpha-equivalence.} At the
core of structural equivalence is alpha-equivalence which identifies
process that are the same up to a change of variable. Formally, we
recognize the distinction between free and bound names. The free names
of a process, $\freenames{P}$, may be calculated recursively as
follows:

\begin{mathpar}
\freenames{\pzero} := \emptyset
  \and \\
  \freenames{x?(y).P} := \{ x \} \cup (\freenames{P} \setminus \{ y \})
  \and 
  \freenames{x!\langle P \rangle} := \{ x \} \cup \{ P \} 
  \and \\
  \freenames{P|Q} := \freenames{P} \cup \freenames{Q}
  \and \\
  \freenames{@{x}} := \{ x \}
\end{mathpar}

$\pi$
$\quotep{\pi}$

$\freenames{-} : \pi \to \mathcal{P}(\quotep{\pi})$

\begin{eqnarray*}
  \freenames{\pzero} & := & \emptyset \\
  \freenames{x?(y).P} & := & \{ x \} \cup (\freenames{P} \setminus \{ y \}) \\
  \freenames{x!\langle P \rangle} & := & \{ x \} \cup \{ P \} \\
  \freenames{P|Q} & := & \freenames{P} \cup \freenames{Q} \\
  \freenames{\dropn{x}} & := & \{ x \}
\end{eqnarray*}

The bound names of a process, $\boundnames{P}$, are those names occurring in $P$
that are not free. For example, in $x?(y).0$, the name $x$ is free, while $y$ is bound.

\begin{mathpar}
  \inferrule* [lab=monoidal-laws] {} { P|Q \equiv Q|P \and P|0 \equiv P \and P|(Q|R) \equiv (P|Q)|R }
\end{mathpar}

\begin{mathpar}
  \inferrule* [lab=alpha-equivalence] {} { (x)P \equiv (y)P\{y/x\} \and y \not\in \freenames{P} }
\end{mathpar}

\begin{definition}
Then two processes, $P,Q$, are alpha-equivalent if $P = Q\{\vec{y}/\vec{x}\}$ for
some $\vec{x} \in \boundnames{Q},\vec{y} \in \boundnames{P}$, where $Q\{\vec{y}/\vec{x}\}$
denotes the capture-avoiding substitution of $\vec{y}$ for $\vec{x}$ in $Q$.
\end{definition}

\begin{definition}
  The {\em structural congruence} \cite{SangiorgiWalker} , $\equiv$,
  between processes is the least congruence containing
  alpha-equivalence, satisfying the abelian monoid laws
  (associativity, commutativity and $\pzero$ as identity) for parallel
  composition $|$ and for summation $+$.
\end{definition}

\subsection{Name equivalence}

We take name equivalence, written $\nameeq$, to be the smallest
equivalence relation generated by the following rules.

\begin{mathpar}
\inferrule*[lab=Quote-drop]
{ }
{ \quotep{@{x}} \nameeq x }

\inferrule*[lab=Struct-equiv]
{ P \scong Q }
{ \quotep{P} \nameeq \quotep{Q} }
\end{mathpar}

The astute reader will have noticed that the mutual recursion of names
and processes imposes a mutual recursion on alpha-equivalence and
structural equivalence via name-equivalence. Fortunately, all of this
works out pleasantly and we may calculate in the natural way, free of
concern. The reader interested in the details is referred to the
appendix \ref{appendix:rho_details}.

\subsection{Substitution}

We use $\Proc$ for the set of processes, $\QProc$ for the set of
names, and $\id{\{}\vec{y} / \vec{x} \id{\}}$ to denote partial maps,
$s : \QProc \rightarrow \QProc$. A map, $s$ lifts, uniquely, to a map
on process terms, $\widehat{s} : \Proc \rightarrow \Proc$ by the
following equations.

\begin{mathpar}
  (0) \psubstp{Q}{P} := 0 \\
  (R \juxtap S) \psubstp{Q}{P}
  :=    
  (R)\psubstp{Q}{P} \juxtap (S) \psubstp{Q}{P} \\
  (x?(y).R) \psubstp{Q}{P}    
  :=    
  (x)\substp{Q}{P} (z)\concat( (R \psubstn{z}{y}) \psubstp{Q}{P} ) \\
  (\lift{x}{R}) \psubstp{Q}{P}  
  :=
  \lift{(x)\substp{Q}{P}}{ R \psubstp{Q}{P} } \\
%   (\dropn{x})  \psubstp{Q}{P}       
%   := 
%   \left\{ 
%     \begin{array}{ccc} 
%       \dropn{\quotep{Q}} & & x \nameeq \quotep{P} \\
%       \dropn{x} & & otherwise \\
%     \end{array}
%   \right. 
  (\dropn{x})  \psubstp{Q}{P}       
  := 
  \left\{ 
    \begin{array}{ccc} 
      Q & & x \nameeq \quotep{P} \\
      \dropn{x} & & otherwise \\
    \end{array}
  \right.
\end{mathpar}
 

where

\begin{eqnarray}
  (x)\id{\{} \lpquote Q \rpquote / \lpquote P \rpquote \id{\}}            = 
  \left\{ 
    \begin{array}{ccc}
      \lpquote Q \rpquote & & x \nameeq \lpquote P \rpquote \\
      x & & otherwise \\
    \end{array}
  \right. \nonumber
\end{eqnarray}

and $z$ is chosen distinct from $\quotep{P}$, $\quotep{Q}$, the free
names in $Q$, and all the names in $R$. Our $\alpha$-equivalence will
be built in the standard way from this substitution.

\begin{remark}\label{rem:no_self_referential_names}
  One consequence of these definitions is that $\forall P. \quotep{P}
  \not\in \freenames{P}$.
\end{remark}

\subsection{ Dynamic quote: an example }

Anticipating something of what's to come, consider applying the
substitution, $\widehat{\id{\{}u / z \id{\}}}$, to the following pair
of processes, $\lift{w}{y!(z)}$ and $w[ \lpquote y!(z) \rpquote ]$.

\begin{eqnarray}
	\lift{w}{y!(z)}\widehat{\id{\{}u / z \id{\}}}
		& = &
		\lift{w}{y!(u)} \nonumber\\
	w[ \lpquote y!(z) \rpquote ] \widehat{ \id{\{}u / z \id{\}} }
		& = &
		w[ \lpquote y!(z) \rpquote ] \nonumber
\end{eqnarray}

Because the body of the process between quotes is impervious to
substitution, we get radically different answers. In fact, by
examining the first process in an input context,
e.g. $x?(z).\lift{w}{y!(z)}$, we see that the process under the lift
operator may be shaped by prefixed inputs binding a name inside it. In
this sense, the lift operator will be seen as a way to dynamically
construct processes before reifying them as names.

Finally equipped with these standard features we can present the
dynamics of the calculus.

\subsubsection{Operational semantics} 

Finally, we introduce the computational dynamics. What marks these
algebras as distinct from other more traditionally studied algebraic
structures, e.g. vector spaces or polynomial rings, is the manner in
which dynamics is captured. In traditional structures, dynamics is typically
expressed through morphisms between such structures, as in linear maps
between vector spaces or morphisms between rings. In algebras
associated with the semantics of computation, the dynamics is
expressed as part of the algebraic structure itself, through a
reduction reduction relation typically denoted by $\red$. Below, we
give a recursive presentation of this relation for the calculus used
in the encoding.

$\red \subseteq \pi \times \pi$
$\red : \pi \to \mathcal{P}(\pi)$

\begin{mathpar}
  \inferrule* [lab=Comm] { \textsf{match}( x_{src}, x_{trgt} ) } { x_{trgt}?(y)P \; | \; x_{src}!\langle {Q} \rangle \red P\{\quotep{Q}/y}\} }
  \and \\
  \inferrule* [lab=Par] {{P} \red {P}'} {{{P} | {Q}} \red {{P}' | {Q}}}
  \and
  \inferrule* [lab=Equiv]{{{P} \scong {P}'} \andalso {{P}' \red {Q}'} \andalso {{Q}' \scong {Q}}}{{P} \red {Q}}
\end{mathpar}

\begin{eqnarray*}
  match_{\equiv} (\quotep{P},\quotep{Q}) & := & P \equiv Q \\
  match_{\dagger}(\quotep{P},\quotep{Q}) & := & \forall R. P|Q \red^{*} R => R \red^{*} 0 \\
  match_{K}(\quotep{P},\quotep{Q}) & := & K \mbox{ for some context } K
\end{eqnarray*}

$u?(x)P | u!\langle Q \rangle \red P\{\quotep{Q}/x\}$

%We write $\wred$ for $\red^*$, and $P\red$ if $\exists Q $ such that $ P \red Q$.
We write $P\red$ if $\exists Q $ such that $ P \red Q$ and $P\not\red$, otherwise.

\section{Replication}

As mentioned before, it is known that replication (and hence
recursion) can be implemented in a higher-order process algebra
\cite{SangiorgiWalker}. As our first example of calculation with the
machinery thus far presented we give the construction explicitly in
the {\rhoc}.

\begin{eqnarray}
	D_{x} & := & \prefix{x}{y}{(\binpar{\outputp{x}{y}}{@{y}})} \nonumber\\
	\bangp_{x}{P} & := & \binpar{{x}!\langle{\binpar{D_{x}}{P}}\rangle}{D_{x}} \nonumber
\end{eqnarray}

\begin{eqnarray}
	\bangp_{x}{P} & & \nonumber\\
	=
	& {x}!\langle{(\prefix{x}{y}{(\outputp{x}{y} | @{y})) | P}}\rangle 
	      | \prefix{x}{y}{(\outputp{x}{y} | @{y})} & \nonumber\\
	\red
	& (\outputp{x}{y} | @{y})\substn{\quotep{(\prefix{x}{y}{(@{y} | \outputp{x}{y})) | P}}}{y} & \nonumber\\
	=
	& \outputp{x}{\quotep{(\prefix{x}{y}{(\outputp{x}{y} | @{y})) | P}}}
	  | {(\prefix{x}{y}{(\outputp{x}{y} | @{y})) | P}} & \nonumber\\
	\red
	& \ldots & \nonumber\\
	\red^*
	& P | P | \ldots & \nonumber
\end{eqnarray}

Of course, this encoding, as an implementation, runs away, unfolding
$\bangp{P}$ eagerly. A lazier and more implementable replication
operator, restricted to input-guarded processes, may be obtained as follows.

\begin{eqnarray}
\bangp{\prefix{u}{v}{P}} 
	:= 
	\binpar{\lift{x}{\prefix{u}{v}{(\binpar{D(x)}{P})}}}{D(x)} \nonumber
\end{eqnarray}

\begin{remark}
  Note that the lazier definition still does not deal with summation
  or mixed summation (i.e. sums over input and output). The reader is
  invited to construct definitions of replication that deal with these
  features. 

  Further, the definitions are parameterized in a name, $x$. Can you,
  gentle reader, make a definition that eliminates this parameter and
  guarantees no accidental interaction between the replication
  machinery and the process being replicated -- i.e. no accidental
  sharing of names used by the process to get its work done and the
  name(s) used by the replication to effect copying. This latter
  revision of the definition of replication is crucial to obtaining
  the expected identity $!!P \sim !P$.
\end{remark}

\begin{remark}\label{rem:paradoxical_combinator}
  The reader familiar with the lambda calculus will have noticed the
  similarity between $D$ and the paradoxical combinator.

  [Ed. note: the existence of this seems to suggest we have to be more
  restrictive on the set of processes and names we admit if we are to
  support no-cloning.]
\end{remark}

\subsubsection{Bisimulation}

The computational dynamics gives rise to another kind of equivalence,
the equivalence of computational behavior. As previously mentioned
this is typically captured \emph{via} some form of bisimulation.

% The notion we use in this paper is weak barbed bisimulation
% \cite{milner91polyadicpi}.

The notion we use in this paper is derived from weak barbed
bisimulation \cite{milner91polyadicpi}. 

\begin{definition}
An \emph{observation relation}, $\downarrow_{\mathcal N}$, over a set
of names, $\mathcal N$, is the smallest relation satisfying the rules
below.

\infrule[Out-barb]{y \in {\mathcal N}, \; x \nameeq y}
		  {\outputp{x}{v} \downarrow_{\mathcal N} x}
\infrule[Par-barb]{\mbox{$P\downarrow_{\mathcal N} x$ or $Q\downarrow_{\mathcal N} x$}}
		  {\binpar{P}{Q} \downarrow_{\mathcal N} x}

We write $P \Downarrow_{\mathcal N} x$ if there is $Q$ such that 
$P \wred Q$ and $Q \downarrow_{\mathcal N} x$.
\end{definition}

\begin{definition}
%\label{def.bbisim}
An  ${\mathcal N}$-\emph{barbed bisimulation} over a set of names, ${\mathcal N}$, is a symmetric binary relation 
${\mathcal S}_{\mathcal N}$ between agents such that $P\rel{S}_{\mathcal N}Q$ implies:
\begin{enumerate}
\item If $P \red P'$ then $Q \wred Q'$ and $P'\rel{S}_{\mathcal N} Q'$.
\item If $P\downarrow_{\mathcal N} x$, then $Q\Downarrow_{\mathcal N} x$.
\end{enumerate}
$P$ is ${\mathcal N}$-barbed bisimilar to $Q$, written
$P \wbbisim_{\mathcal N} Q$, if $P \rel{S}_{\mathcal N} Q$ for some ${\mathcal N}$-barbed bisimulation ${\mathcal S}_{\mathcal N}$.
\end{definition}

$\mathcal{R} \subseteq \pi \times \pi$

$P \mathcal{R} Q => \forall P'. P \red P' \Rightarrow \exists Q'. Q \red Q', P' \mathcal{R} Q'$

$P \vdash x \Rightarrow Q \vdash x$

\begin{mathpar}
  \inferrule*[lab=Out-barb]{x \nameeq y}{{y}!\langle{Q}\rangle \vdash x}
  \and
  \inferrule*[lab=Par-barb]{\mbox{$P\vdash x$ or $Q\vdash x$}}{\binpar{P}{Q} \vdash x}
\end{mathpar}

\subsubsection{Contexts}

One of the principle advantages of computational calculi like the
$\pi$-calculus is a well-defined notion of context,
contextual-equivalence and a correlation between
contextual-equivalence and notions of bisimulation. The notion of
context allows the decomposition of a process into (sub-)process and
its syntactic environment, its context. Thus, a context may be
thought of as a process with a ``hole'' (written $\Box$) in it. The
application of a context $M$ to a process $P$, written $M[P]$, is
tantamount to filling the hole in $M$ with $P$. In this paper we do
not need the full weight of this theory, but do make use of the notion
of context in the proof the main theorem. 

\begin{mathpar}
  \inferrule* [lab=summation] {} {{M_{M},M_{N}} \bc \Box \;|\; x.M_{A} \;|\; M_{M}+M_{N}}
  \and
  \inferrule* [lab=agent] {} {{M_{A}} \bc (\vec{x})M_{P} \;| \; \clift{P_0,\ldots,M_{P},\ldots,P_N}}
  \and \\
  \inferrule* [lab=process] {} {{M_{P}} \bc M_{N} \;| \;P|M_{P} }
\end{mathpar} 

\begin{mathpar}
  \inferrule* [lab=sychronization] {} {M_{N} \bc \Box \;|\; x?M_{F} \;|\; x!M_{C}}
  \and
  \inferrule* [lab=abstraction] {} {{M_{F}} \bc (x)M_{P} }
  \and
  \inferrule* [lab=concretion] {} {{M_{C}} \bc \langle M_{P} \rangle }
  \and \\
  \inferrule* [lab=process] {} {{M_{P}} \bc M_{N} \;| \;P|M_{P} }
\end{mathpar}

\begin{definition}[contextual application] Given a context $M$, and
  process $P$, we define the \emph{contextual application}, $M[P] :=
  M\{P/\Box\}$. That is, the contextual application of M to P is the
  substitution of $P$ for $\Box$ in $M$.
\end{definition}

$\meaningof{-} : L \to \mathcal{P}(\pi)$

\begin{mathpar}
  \inferrule* [lab=collection] {} {\meaningof{true} = \pi, \and \meaningof{~E} = \pi \setminus \meaningof{E}, \and \meaningof{E_{1} \& E_{2}} = \meaningof{E_{1}} \cap \meaningof{E_{2}}}
\end{mathpar}

\begin{mathpar}
  \inferrule* [lab=structure] {} {\meaningof{0} = \{ P \in \pi | P \equiv 0 \}, \and \\ \meaningof{E_1 | E_2} = \{ P \in \pi | P \equiv P_{1} | P_{2}, P_{1} \in \meaningof{E_{1}}, P_{2} \in \meaningof{E_2}\} }
\end{mathpar}

\begin{mathpar}
 \inferrule* [lab=behavior] {} {\meaningof{\langle a?b \rangle E} = \{ P \in \pi | P \equiv Q | u?(y)P', \\ \and \\\\ \and \\ \;\;\; u \in \meaningof{a}, \forall z.P'\{z/y\} \in \meaningof{E\{z/b\}}\}, \and \\ \meaningof{a!E} = \{ P \in \pi | P \equiv Q | x!\langle P' \rangle, x \in \meaningof{a} P' \in \meaningof{E}\} }
\end{mathpar}

\begin{mathpar}
 \inferrule* [lab=nominal] {} {\meaningof{\quotep{E}} = \{ \quotep{P} \in \quotep{\pi} | P \in \meaningof{E} \}, \and \meaningof{\quotep{P}} = \{ \quotep{Q} \in \quotep{\pi} | P \equiv Q \} \and \\ \meaningof{@\quotep{E}} = \{ P \in \pi | P \equiv @x, x \in \meaningof{E} \}}
\end{mathpar}

\begin{eqnarray*}
  \\
  \meaningof{-} : TS \to ST
\end{eqnarray*}

\begin{eqnarray*}
  \\
  L : TS \to ST
\end{eqnarray*}

\begin{eqnarray*}
  \\
  P \models E \iff P \in \meaningof{E}
\end{eqnarray*}

\begin{eqnarray*}
  P \approx_{L} Q \iff \forall E \in L. P \models E \iff Q \models E
\end{eqnarray*}

\begin{eqnarray*}
  P \approx_{K} Q
\end{eqnarray*}

\begin{eqnarray*}
  P \approx Q
\end{eqnarray*}

$\approx_{K} = \approx = \approx_{L}$

\subsubsection{Contextual duality}

Note that contexts extend the quotation operation to a family of
operations from processes to names. Given a context, $M$, we can
define a \emph{nominal context}, $\quotep{M}$ by $\quotep{M}[P] :=
\quotep{M[P]}$. To foreshadow what is to come we observe that these
operations enjoy a duality with processes very much like the duality
between vectors and maps from vectors to scalars.

Further, because the calculus is essentially higher-order, we have a
correspondence between contexts and processes. More specifically,
given a name $x$ and a context $M$ we can construct $M^{*}_{x}$ such
that 

\begin{mathpar}
  M^{*}_{x} | \lift{x}{P} \red M[P]
\end{mathpar}

namely,

\begin{mathpar}
  M^{*}_{x} := x?(u).M[\dropn{u}]
\end{mathpar}

The dependence of $M^{*}_{x}$ on a name makes it an abstraction, 

\begin{mathpar}
  M^{*} := (x)x?(u).M[\dropn{u}]
\end{mathpar}

\subsection{Additional notation}

It will sometimes be convenient to denote the process a name
quotes. We already have the notation $x = \quotep{P}$, but it will be
convenient to introduce an alternate notation, $\procn{x}$, when we
want to emphasize the connection to the use of the name. Note that, by
virtue of name equivalence, $\quotep{\procn{x}} \nameeq x$; so, the
notation is consistent with previous definitions.

Further, because names have structure it is possible to effect
substitutions on the basis of that structure. This means we need to
upgrade our notation for substitutions, which we accomplish by
adapting comprehension notation. Thus,

\begin{mathpar}
  P\{ y / x : x \in S \}
\end{mathpar}

is interpreted to mean the process derived from P by replacing (in a
capture-avoiding manner) each occurrence of $x$ in $S$ by $y$. For example,

\begin{mathpar}
  P\{ \quotep{\procn{x}|\procn{x}} / x : x \in \freenames{P} \}
\end{mathpar}

will replace each (occurrence) of a free name $x$ in $P$ by
$\quotep{\procn{x}|\procn{x}}$.

Also, we will avail ourselves of the notation $x^{L}$ and $x^{R}$ to
denote injections of a name into disjoint copies of the name
space. There are numerous ways to accomplish this. One example can be
found in \cite{MeredithR05}. This notation overloads to vectors of
names: $\vec{x}^{\pi} := (x_{i}^{\pi} \; : \; 0 \leq i < |\vec{x}| )$ where $\pi \in \{L,R\}$.

We also use $P^{\Box} := P|\Box$.

In \cite{MeredithR05} an interpretation of the new operator is
given. It turns out that there are several possible interpretations
all enjoying the requisite algebraic properties of the operator (see
\cite{milner91polyadicpi}). We will therefore make liberal use of
$(\nu\; \vec{x})P$.

% subsection the_syntax_and_semantics_of_the_notation_system (end)   

\input{qm2pi.qmops} 

\input{qm2pi.sterngerlach} 

\input{qm2pi.metric} 

% section concurrent_process_calculi (end)

%\input{qm2pi.proofsketch}

% section proof sketch (end)

%\input{qm2pi.slviaknots} 

% section spatial logic via knots (end)

\input{qm2pi.conclusion}

% section conclusion (end)

%\input{qm2pi.dtcodes} 

% section wiring algorithm (end)

\input{qm2pi.ack} 

% section acknowledgments (end)

\newpage


\bibliographystyle{plain}   
\bibliography{../../biblios/main.bib}

\input{qm2pi.rhodetails}

\end{document}

 

% section notation (end)

\input{qm2pi.process.calculi} 

% section concurrent_process_calculi_and_spatial_logics_ (end)
    
%\documentclass[12pt]{llncs}
%\documentclass{jktr}

\usepackage[pdftex]{hyperref}                   
\usepackage {listings}
\usepackage {mathpartir}
\usepackage{bcprules}
%\usepackage{listings}
                       
\usepackage{graphicx} 
%\usepackage[margins=2.5cm,nohead,nofoot]{geometry}
%\usepackage{geometry}
\usepackage{amsfonts}
\usepackage{amstext}
\usepackage{latexsym}
\usepackage{amssymb}
\usepackage{color}


%\include{myPreamble}
\include{qm2pi.local} 

%\ifpdf
%\usepackage[pdftex]{graphicx}
%\else
%\usepackage{graphicx}
%\fi

 % \ifpdf
%  \usepackage{pdfsync}
%  \if


%\title{Brief Article}
%\author{David F. Snyder}
%\author{L.G. Meredith}

%\address{Dept. of Math., Texas State University--San Marcos, San Marcos, TX 78666}
       
\pagestyle{empty}


\begin{document}

\lstset{language=[Objective]Caml,frame=shadowbox}

\input{qm2pi.front}

% section front matter (end)

\input{qm2pi.intro} 
 
% section introduction (end)

% \input{qm2pi.knotations} 

% section notation (end)

\input{qm2pi.process.calculi} 

% section concurrent_process_calculi_and_spatial_logics_ (end)
    
%\input{qm2pi.knots2pi} 

%\input{qm2pi.trefoil} 

%\input{qm2pi.mainthm} 

% subsection basic_interpretation (end)

%\input{qm2pi.rho.presentation} 
\subsection{The syntax and semantics of the notation system}\label{sub:the_syntax_and_semantics_of_the_notation_system} % (fold)

We now summarize a technical presentation of the calculus that
embodies our theory of dynamics. The typical presentation of such a
calculus follows the style of giving generators and relations on
them. The grammar, below, describing term constructors, freely
generates the set of processes, $\Proc$. This set is then quotiented
by a relation known as structural congruence and it is over this set
that the notion of dynamics is expressed. This presentation is
essentially that of \cite{MeredithR05} with the addition of
polyadicity and summation. For readability we have relegated some of
the technical subtleties to an appendix.

\subsubsection{Process grammar}\label{subsub:process_grammar}

\begin{mathpar}
  \inferrule* [lab=synchronization] {} {{M} \bc \pzero \;|\; x?F \;|\; x!C }
  \and
  \inferrule* [lab=abstraction] {} {{F} \bc (x)P}
  \and
  \inferrule* [lab=concretion] {} {{C} \bc \langle Q \rangle}
  \and
  \inferrule* [lab=process] {} {{P,Q} \bc M \;| \;P|Q \;|\; @{x}}
  \and
  \inferrule* [lab=name] {} {{x} \bc \quotep{P}}
\end{mathpar} 

Note that $\vec{x}$ (resp. $\vec{P}$) denotes a vector of names
(resp. processes) of length $|\vec{x}|$ (resp. $|\vec{P}|$). We adopt
the following useful abbreviations.

\begin{mathpar}
   x?(\vec{y}).P := x.(\vec{y})P \and  x\clift{\vec{P}} := x.\clift{\vec{P}}
   \and x!(y) := \lift{x}{\dropn{y}}
   \and \Pi_{i=0}^{n-1}P_i := P_0 | \ldots | P_{n-1}
\end{mathpar}

\subsubsection{Structural congruence}

\paragraph{Free and bound names and alpha-equivalence.} At the
core of structural equivalence is alpha-equivalence which identifies
process that are the same up to a change of variable. Formally, we
recognize the distinction between free and bound names. The free names
of a process, $\freenames{P}$, may be calculated recursively as
follows:

\begin{mathpar}
\freenames{\pzero} := \emptyset
  \and \\
  \freenames{x?(y).P} := \{ x \} \cup (\freenames{P} \setminus \{ y \})
  \and 
  \freenames{x!\langle P \rangle} := \{ x \} \cup \{ P \} 
  \and \\
  \freenames{P|Q} := \freenames{P} \cup \freenames{Q}
  \and \\
  \freenames{@{x}} := \{ x \}
\end{mathpar}

$\pi$
$\quotep{\pi}$

$\freenames{-} : \pi \to \mathcal{P}(\quotep{\pi})$

\begin{eqnarray*}
  \freenames{\pzero} & := & \emptyset \\
  \freenames{x?(y).P} & := & \{ x \} \cup (\freenames{P} \setminus \{ y \}) \\
  \freenames{x!\langle P \rangle} & := & \{ x \} \cup \{ P \} \\
  \freenames{P|Q} & := & \freenames{P} \cup \freenames{Q} \\
  \freenames{\dropn{x}} & := & \{ x \}
\end{eqnarray*}

The bound names of a process, $\boundnames{P}$, are those names occurring in $P$
that are not free. For example, in $x?(y).0$, the name $x$ is free, while $y$ is bound.

\begin{mathpar}
  \inferrule* [lab=monoidal-laws] {} { P|Q \equiv Q|P \and P|0 \equiv P \and P|(Q|R) \equiv (P|Q)|R }
\end{mathpar}

\begin{mathpar}
  \inferrule* [lab=alpha-equivalence] {} { (x)P \equiv (y)P\{y/x\} \and y \not\in \freenames{P} }
\end{mathpar}

\begin{definition}
Then two processes, $P,Q$, are alpha-equivalent if $P = Q\{\vec{y}/\vec{x}\}$ for
some $\vec{x} \in \boundnames{Q},\vec{y} \in \boundnames{P}$, where $Q\{\vec{y}/\vec{x}\}$
denotes the capture-avoiding substitution of $\vec{y}$ for $\vec{x}$ in $Q$.
\end{definition}

\begin{definition}
  The {\em structural congruence} \cite{SangiorgiWalker} , $\equiv$,
  between processes is the least congruence containing
  alpha-equivalence, satisfying the abelian monoid laws
  (associativity, commutativity and $\pzero$ as identity) for parallel
  composition $|$ and for summation $+$.
\end{definition}

\subsection{Name equivalence}

We take name equivalence, written $\nameeq$, to be the smallest
equivalence relation generated by the following rules.

\begin{mathpar}
\inferrule*[lab=Quote-drop]
{ }
{ \quotep{@{x}} \nameeq x }

\inferrule*[lab=Struct-equiv]
{ P \scong Q }
{ \quotep{P} \nameeq \quotep{Q} }
\end{mathpar}

The astute reader will have noticed that the mutual recursion of names
and processes imposes a mutual recursion on alpha-equivalence and
structural equivalence via name-equivalence. Fortunately, all of this
works out pleasantly and we may calculate in the natural way, free of
concern. The reader interested in the details is referred to the
appendix \ref{appendix:rho_details}.

\subsection{Substitution}

We use $\Proc$ for the set of processes, $\QProc$ for the set of
names, and $\id{\{}\vec{y} / \vec{x} \id{\}}$ to denote partial maps,
$s : \QProc \rightarrow \QProc$. A map, $s$ lifts, uniquely, to a map
on process terms, $\widehat{s} : \Proc \rightarrow \Proc$ by the
following equations.

\begin{mathpar}
  (0) \psubstp{Q}{P} := 0 \\
  (R \juxtap S) \psubstp{Q}{P}
  :=    
  (R)\psubstp{Q}{P} \juxtap (S) \psubstp{Q}{P} \\
  (x?(y).R) \psubstp{Q}{P}    
  :=    
  (x)\substp{Q}{P} (z)\concat( (R \psubstn{z}{y}) \psubstp{Q}{P} ) \\
  (\lift{x}{R}) \psubstp{Q}{P}  
  :=
  \lift{(x)\substp{Q}{P}}{ R \psubstp{Q}{P} } \\
%   (\dropn{x})  \psubstp{Q}{P}       
%   := 
%   \left\{ 
%     \begin{array}{ccc} 
%       \dropn{\quotep{Q}} & & x \nameeq \quotep{P} \\
%       \dropn{x} & & otherwise \\
%     \end{array}
%   \right. 
  (\dropn{x})  \psubstp{Q}{P}       
  := 
  \left\{ 
    \begin{array}{ccc} 
      Q & & x \nameeq \quotep{P} \\
      \dropn{x} & & otherwise \\
    \end{array}
  \right.
\end{mathpar}
 

where

\begin{eqnarray}
  (x)\id{\{} \lpquote Q \rpquote / \lpquote P \rpquote \id{\}}            = 
  \left\{ 
    \begin{array}{ccc}
      \lpquote Q \rpquote & & x \nameeq \lpquote P \rpquote \\
      x & & otherwise \\
    \end{array}
  \right. \nonumber
\end{eqnarray}

and $z$ is chosen distinct from $\quotep{P}$, $\quotep{Q}$, the free
names in $Q$, and all the names in $R$. Our $\alpha$-equivalence will
be built in the standard way from this substitution.

\begin{remark}\label{rem:no_self_referential_names}
  One consequence of these definitions is that $\forall P. \quotep{P}
  \not\in \freenames{P}$.
\end{remark}

\subsection{ Dynamic quote: an example }

Anticipating something of what's to come, consider applying the
substitution, $\widehat{\id{\{}u / z \id{\}}}$, to the following pair
of processes, $\lift{w}{y!(z)}$ and $w[ \lpquote y!(z) \rpquote ]$.

\begin{eqnarray}
	\lift{w}{y!(z)}\widehat{\id{\{}u / z \id{\}}}
		& = &
		\lift{w}{y!(u)} \nonumber\\
	w[ \lpquote y!(z) \rpquote ] \widehat{ \id{\{}u / z \id{\}} }
		& = &
		w[ \lpquote y!(z) \rpquote ] \nonumber
\end{eqnarray}

Because the body of the process between quotes is impervious to
substitution, we get radically different answers. In fact, by
examining the first process in an input context,
e.g. $x?(z).\lift{w}{y!(z)}$, we see that the process under the lift
operator may be shaped by prefixed inputs binding a name inside it. In
this sense, the lift operator will be seen as a way to dynamically
construct processes before reifying them as names.

Finally equipped with these standard features we can present the
dynamics of the calculus.

\subsubsection{Operational semantics} 

Finally, we introduce the computational dynamics. What marks these
algebras as distinct from other more traditionally studied algebraic
structures, e.g. vector spaces or polynomial rings, is the manner in
which dynamics is captured. In traditional structures, dynamics is typically
expressed through morphisms between such structures, as in linear maps
between vector spaces or morphisms between rings. In algebras
associated with the semantics of computation, the dynamics is
expressed as part of the algebraic structure itself, through a
reduction reduction relation typically denoted by $\red$. Below, we
give a recursive presentation of this relation for the calculus used
in the encoding.

$\red \subseteq \pi \times \pi$
$\red : \pi \to \mathcal{P}(\pi)$

\begin{mathpar}
  \inferrule* [lab=Comm] { \textsf{match}( x_{src}, x_{trgt} ) } { x_{trgt}?(y)P \; | \; x_{src}!\langle {Q} \rangle \red P\{\quotep{Q}/y}\} }
  \and \\
  \inferrule* [lab=Par] {{P} \red {P}'} {{{P} | {Q}} \red {{P}' | {Q}}}
  \and
  \inferrule* [lab=Equiv]{{{P} \scong {P}'} \andalso {{P}' \red {Q}'} \andalso {{Q}' \scong {Q}}}{{P} \red {Q}}
\end{mathpar}

\begin{eqnarray*}
  match_{\equiv} (\quotep{P},\quotep{Q}) & := & P \equiv Q \\
  match_{\dagger}(\quotep{P},\quotep{Q}) & := & \forall R. P|Q \red^{*} R => R \red^{*} 0 \\
  match_{K}(\quotep{P},\quotep{Q}) & := & K \mbox{ for some context } K
\end{eqnarray*}

$u?(x)P | u!\langle Q \rangle \red P\{\quotep{Q}/x\}$

%We write $\wred$ for $\red^*$, and $P\red$ if $\exists Q $ such that $ P \red Q$.
We write $P\red$ if $\exists Q $ such that $ P \red Q$ and $P\not\red$, otherwise.

\section{Replication}

As mentioned before, it is known that replication (and hence
recursion) can be implemented in a higher-order process algebra
\cite{SangiorgiWalker}. As our first example of calculation with the
machinery thus far presented we give the construction explicitly in
the {\rhoc}.

\begin{eqnarray}
	D_{x} & := & \prefix{x}{y}{(\binpar{\outputp{x}{y}}{@{y}})} \nonumber\\
	\bangp_{x}{P} & := & \binpar{{x}!\langle{\binpar{D_{x}}{P}}\rangle}{D_{x}} \nonumber
\end{eqnarray}

\begin{eqnarray}
	\bangp_{x}{P} & & \nonumber\\
	=
	& {x}!\langle{(\prefix{x}{y}{(\outputp{x}{y} | @{y})) | P}}\rangle 
	      | \prefix{x}{y}{(\outputp{x}{y} | @{y})} & \nonumber\\
	\red
	& (\outputp{x}{y} | @{y})\substn{\quotep{(\prefix{x}{y}{(@{y} | \outputp{x}{y})) | P}}}{y} & \nonumber\\
	=
	& \outputp{x}{\quotep{(\prefix{x}{y}{(\outputp{x}{y} | @{y})) | P}}}
	  | {(\prefix{x}{y}{(\outputp{x}{y} | @{y})) | P}} & \nonumber\\
	\red
	& \ldots & \nonumber\\
	\red^*
	& P | P | \ldots & \nonumber
\end{eqnarray}

Of course, this encoding, as an implementation, runs away, unfolding
$\bangp{P}$ eagerly. A lazier and more implementable replication
operator, restricted to input-guarded processes, may be obtained as follows.

\begin{eqnarray}
\bangp{\prefix{u}{v}{P}} 
	:= 
	\binpar{\lift{x}{\prefix{u}{v}{(\binpar{D(x)}{P})}}}{D(x)} \nonumber
\end{eqnarray}

\begin{remark}
  Note that the lazier definition still does not deal with summation
  or mixed summation (i.e. sums over input and output). The reader is
  invited to construct definitions of replication that deal with these
  features. 

  Further, the definitions are parameterized in a name, $x$. Can you,
  gentle reader, make a definition that eliminates this parameter and
  guarantees no accidental interaction between the replication
  machinery and the process being replicated -- i.e. no accidental
  sharing of names used by the process to get its work done and the
  name(s) used by the replication to effect copying. This latter
  revision of the definition of replication is crucial to obtaining
  the expected identity $!!P \sim !P$.
\end{remark}

\begin{remark}\label{rem:paradoxical_combinator}
  The reader familiar with the lambda calculus will have noticed the
  similarity between $D$ and the paradoxical combinator.

  [Ed. note: the existence of this seems to suggest we have to be more
  restrictive on the set of processes and names we admit if we are to
  support no-cloning.]
\end{remark}

\subsubsection{Bisimulation}

The computational dynamics gives rise to another kind of equivalence,
the equivalence of computational behavior. As previously mentioned
this is typically captured \emph{via} some form of bisimulation.

% The notion we use in this paper is weak barbed bisimulation
% \cite{milner91polyadicpi}.

The notion we use in this paper is derived from weak barbed
bisimulation \cite{milner91polyadicpi}. 

\begin{definition}
An \emph{observation relation}, $\downarrow_{\mathcal N}$, over a set
of names, $\mathcal N$, is the smallest relation satisfying the rules
below.

\infrule[Out-barb]{y \in {\mathcal N}, \; x \nameeq y}
		  {\outputp{x}{v} \downarrow_{\mathcal N} x}
\infrule[Par-barb]{\mbox{$P\downarrow_{\mathcal N} x$ or $Q\downarrow_{\mathcal N} x$}}
		  {\binpar{P}{Q} \downarrow_{\mathcal N} x}

We write $P \Downarrow_{\mathcal N} x$ if there is $Q$ such that 
$P \wred Q$ and $Q \downarrow_{\mathcal N} x$.
\end{definition}

\begin{definition}
%\label{def.bbisim}
An  ${\mathcal N}$-\emph{barbed bisimulation} over a set of names, ${\mathcal N}$, is a symmetric binary relation 
${\mathcal S}_{\mathcal N}$ between agents such that $P\rel{S}_{\mathcal N}Q$ implies:
\begin{enumerate}
\item If $P \red P'$ then $Q \wred Q'$ and $P'\rel{S}_{\mathcal N} Q'$.
\item If $P\downarrow_{\mathcal N} x$, then $Q\Downarrow_{\mathcal N} x$.
\end{enumerate}
$P$ is ${\mathcal N}$-barbed bisimilar to $Q$, written
$P \wbbisim_{\mathcal N} Q$, if $P \rel{S}_{\mathcal N} Q$ for some ${\mathcal N}$-barbed bisimulation ${\mathcal S}_{\mathcal N}$.
\end{definition}

$\mathcal{R} \subseteq \pi \times \pi$

$P \mathcal{R} Q => \forall P'. P \red P' \Rightarrow \exists Q'. Q \red Q', P' \mathcal{R} Q'$

$P \vdash x \Rightarrow Q \vdash x$

\begin{mathpar}
  \inferrule*[lab=Out-barb]{x \nameeq y}{{y}!\langle{Q}\rangle \vdash x}
  \and
  \inferrule*[lab=Par-barb]{\mbox{$P\vdash x$ or $Q\vdash x$}}{\binpar{P}{Q} \vdash x}
\end{mathpar}

\subsubsection{Contexts}

One of the principle advantages of computational calculi like the
$\pi$-calculus is a well-defined notion of context,
contextual-equivalence and a correlation between
contextual-equivalence and notions of bisimulation. The notion of
context allows the decomposition of a process into (sub-)process and
its syntactic environment, its context. Thus, a context may be
thought of as a process with a ``hole'' (written $\Box$) in it. The
application of a context $M$ to a process $P$, written $M[P]$, is
tantamount to filling the hole in $M$ with $P$. In this paper we do
not need the full weight of this theory, but do make use of the notion
of context in the proof the main theorem. 

\begin{mathpar}
  \inferrule* [lab=summation] {} {{M_{M},M_{N}} \bc \Box \;|\; x.M_{A} \;|\; M_{M}+M_{N}}
  \and
  \inferrule* [lab=agent] {} {{M_{A}} \bc (\vec{x})M_{P} \;| \; \clift{P_0,\ldots,M_{P},\ldots,P_N}}
  \and \\
  \inferrule* [lab=process] {} {{M_{P}} \bc M_{N} \;| \;P|M_{P} }
\end{mathpar} 

\begin{mathpar}
  \inferrule* [lab=sychronization] {} {M_{N} \bc \Box \;|\; x?M_{F} \;|\; x!M_{C}}
  \and
  \inferrule* [lab=abstraction] {} {{M_{F}} \bc (x)M_{P} }
  \and
  \inferrule* [lab=concretion] {} {{M_{C}} \bc \langle M_{P} \rangle }
  \and \\
  \inferrule* [lab=process] {} {{M_{P}} \bc M_{N} \;| \;P|M_{P} }
\end{mathpar}

\begin{definition}[contextual application] Given a context $M$, and
  process $P$, we define the \emph{contextual application}, $M[P] :=
  M\{P/\Box\}$. That is, the contextual application of M to P is the
  substitution of $P$ for $\Box$ in $M$.
\end{definition}

$\meaningof{-} : L \to \mathcal{P}(\pi)$

\begin{mathpar}
  \inferrule* [lab=collection] {} {\meaningof{true} = \pi, \and \meaningof{~E} = \pi \setminus \meaningof{E}, \and \meaningof{E_{1} \& E_{2}} = \meaningof{E_{1}} \cap \meaningof{E_{2}}}
\end{mathpar}

\begin{mathpar}
  \inferrule* [lab=structure] {} {\meaningof{0} = \{ P \in \pi | P \equiv 0 \}, \and \\ \meaningof{E_1 | E_2} = \{ P \in \pi | P \equiv P_{1} | P_{2}, P_{1} \in \meaningof{E_{1}}, P_{2} \in \meaningof{E_2}\} }
\end{mathpar}

\begin{mathpar}
 \inferrule* [lab=behavior] {} {\meaningof{\langle a?b \rangle E} = \{ P \in \pi | P \equiv Q | u?(y)P', \\ \and \\\\ \and \\ \;\;\; u \in \meaningof{a}, \forall z.P'\{z/y\} \in \meaningof{E\{z/b\}}\}, \and \\ \meaningof{a!E} = \{ P \in \pi | P \equiv Q | x!\langle P' \rangle, x \in \meaningof{a} P' \in \meaningof{E}\} }
\end{mathpar}

\begin{mathpar}
 \inferrule* [lab=nominal] {} {\meaningof{\quotep{E}} = \{ \quotep{P} \in \quotep{\pi} | P \in \meaningof{E} \}, \and \meaningof{\quotep{P}} = \{ \quotep{Q} \in \quotep{\pi} | P \equiv Q \} \and \\ \meaningof{@\quotep{E}} = \{ P \in \pi | P \equiv @x, x \in \meaningof{E} \}}
\end{mathpar}

\begin{eqnarray*}
  \\
  \meaningof{-} : TS \to ST
\end{eqnarray*}

\begin{eqnarray*}
  \\
  L : TS \to ST
\end{eqnarray*}

\begin{eqnarray*}
  \\
  P \models E \iff P \in \meaningof{E}
\end{eqnarray*}

\begin{eqnarray*}
  P \approx_{L} Q \iff \forall E \in L. P \models E \iff Q \models E
\end{eqnarray*}

\begin{eqnarray*}
  P \approx_{K} Q
\end{eqnarray*}

\begin{eqnarray*}
  P \approx Q
\end{eqnarray*}

$\approx_{K} = \approx = \approx_{L}$

\subsubsection{Contextual duality}

Note that contexts extend the quotation operation to a family of
operations from processes to names. Given a context, $M$, we can
define a \emph{nominal context}, $\quotep{M}$ by $\quotep{M}[P] :=
\quotep{M[P]}$. To foreshadow what is to come we observe that these
operations enjoy a duality with processes very much like the duality
between vectors and maps from vectors to scalars.

Further, because the calculus is essentially higher-order, we have a
correspondence between contexts and processes. More specifically,
given a name $x$ and a context $M$ we can construct $M^{*}_{x}$ such
that 

\begin{mathpar}
  M^{*}_{x} | \lift{x}{P} \red M[P]
\end{mathpar}

namely,

\begin{mathpar}
  M^{*}_{x} := x?(u).M[\dropn{u}]
\end{mathpar}

The dependence of $M^{*}_{x}$ on a name makes it an abstraction, 

\begin{mathpar}
  M^{*} := (x)x?(u).M[\dropn{u}]
\end{mathpar}

\subsection{Additional notation}

It will sometimes be convenient to denote the process a name
quotes. We already have the notation $x = \quotep{P}$, but it will be
convenient to introduce an alternate notation, $\procn{x}$, when we
want to emphasize the connection to the use of the name. Note that, by
virtue of name equivalence, $\quotep{\procn{x}} \nameeq x$; so, the
notation is consistent with previous definitions.

Further, because names have structure it is possible to effect
substitutions on the basis of that structure. This means we need to
upgrade our notation for substitutions, which we accomplish by
adapting comprehension notation. Thus,

\begin{mathpar}
  P\{ y / x : x \in S \}
\end{mathpar}

is interpreted to mean the process derived from P by replacing (in a
capture-avoiding manner) each occurrence of $x$ in $S$ by $y$. For example,

\begin{mathpar}
  P\{ \quotep{\procn{x}|\procn{x}} / x : x \in \freenames{P} \}
\end{mathpar}

will replace each (occurrence) of a free name $x$ in $P$ by
$\quotep{\procn{x}|\procn{x}}$.

Also, we will avail ourselves of the notation $x^{L}$ and $x^{R}$ to
denote injections of a name into disjoint copies of the name
space. There are numerous ways to accomplish this. One example can be
found in \cite{MeredithR05}. This notation overloads to vectors of
names: $\vec{x}^{\pi} := (x_{i}^{\pi} \; : \; 0 \leq i < |\vec{x}| )$ where $\pi \in \{L,R\}$.

We also use $P^{\Box} := P|\Box$.

In \cite{MeredithR05} an interpretation of the new operator is
given. It turns out that there are several possible interpretations
all enjoying the requisite algebraic properties of the operator (see
\cite{milner91polyadicpi}). We will therefore make liberal use of
$(\nu\; \vec{x})P$.

% subsection the_syntax_and_semantics_of_the_notation_system (end)   

\input{qm2pi.qmops} 

\input{qm2pi.sterngerlach} 

\input{qm2pi.metric} 

% section concurrent_process_calculi (end)

%\input{qm2pi.proofsketch}

% section proof sketch (end)

%\input{qm2pi.slviaknots} 

% section spatial logic via knots (end)

\input{qm2pi.conclusion}

% section conclusion (end)

%\input{qm2pi.dtcodes} 

% section wiring algorithm (end)

\input{qm2pi.ack} 

% section acknowledgments (end)

\newpage


\bibliographystyle{plain}   
\bibliography{../../biblios/main.bib}

\input{qm2pi.rhodetails}

\end{document}

 

%\documentclass[12pt]{llncs}
%\documentclass{jktr}

\usepackage[pdftex]{hyperref}                   
\usepackage {listings}
\usepackage {mathpartir}
\usepackage{bcprules}
%\usepackage{listings}
                       
\usepackage{graphicx} 
%\usepackage[margins=2.5cm,nohead,nofoot]{geometry}
%\usepackage{geometry}
\usepackage{amsfonts}
\usepackage{amstext}
\usepackage{latexsym}
\usepackage{amssymb}
\usepackage{color}


%\include{myPreamble}
\include{qm2pi.local} 

%\ifpdf
%\usepackage[pdftex]{graphicx}
%\else
%\usepackage{graphicx}
%\fi

 % \ifpdf
%  \usepackage{pdfsync}
%  \if


%\title{Brief Article}
%\author{David F. Snyder}
%\author{L.G. Meredith}

%\address{Dept. of Math., Texas State University--San Marcos, San Marcos, TX 78666}
       
\pagestyle{empty}


\begin{document}

\lstset{language=[Objective]Caml,frame=shadowbox}

\input{qm2pi.front}

% section front matter (end)

\input{qm2pi.intro} 
 
% section introduction (end)

% \input{qm2pi.knotations} 

% section notation (end)

\input{qm2pi.process.calculi} 

% section concurrent_process_calculi_and_spatial_logics_ (end)
    
%\input{qm2pi.knots2pi} 

%\input{qm2pi.trefoil} 

%\input{qm2pi.mainthm} 

% subsection basic_interpretation (end)

%\input{qm2pi.rho.presentation} 
\subsection{The syntax and semantics of the notation system}\label{sub:the_syntax_and_semantics_of_the_notation_system} % (fold)

We now summarize a technical presentation of the calculus that
embodies our theory of dynamics. The typical presentation of such a
calculus follows the style of giving generators and relations on
them. The grammar, below, describing term constructors, freely
generates the set of processes, $\Proc$. This set is then quotiented
by a relation known as structural congruence and it is over this set
that the notion of dynamics is expressed. This presentation is
essentially that of \cite{MeredithR05} with the addition of
polyadicity and summation. For readability we have relegated some of
the technical subtleties to an appendix.

\subsubsection{Process grammar}\label{subsub:process_grammar}

\begin{mathpar}
  \inferrule* [lab=synchronization] {} {{M} \bc \pzero \;|\; x?F \;|\; x!C }
  \and
  \inferrule* [lab=abstraction] {} {{F} \bc (x)P}
  \and
  \inferrule* [lab=concretion] {} {{C} \bc \langle Q \rangle}
  \and
  \inferrule* [lab=process] {} {{P,Q} \bc M \;| \;P|Q \;|\; @{x}}
  \and
  \inferrule* [lab=name] {} {{x} \bc \quotep{P}}
\end{mathpar} 

Note that $\vec{x}$ (resp. $\vec{P}$) denotes a vector of names
(resp. processes) of length $|\vec{x}|$ (resp. $|\vec{P}|$). We adopt
the following useful abbreviations.

\begin{mathpar}
   x?(\vec{y}).P := x.(\vec{y})P \and  x\clift{\vec{P}} := x.\clift{\vec{P}}
   \and x!(y) := \lift{x}{\dropn{y}}
   \and \Pi_{i=0}^{n-1}P_i := P_0 | \ldots | P_{n-1}
\end{mathpar}

\subsubsection{Structural congruence}

\paragraph{Free and bound names and alpha-equivalence.} At the
core of structural equivalence is alpha-equivalence which identifies
process that are the same up to a change of variable. Formally, we
recognize the distinction between free and bound names. The free names
of a process, $\freenames{P}$, may be calculated recursively as
follows:

\begin{mathpar}
\freenames{\pzero} := \emptyset
  \and \\
  \freenames{x?(y).P} := \{ x \} \cup (\freenames{P} \setminus \{ y \})
  \and 
  \freenames{x!\langle P \rangle} := \{ x \} \cup \{ P \} 
  \and \\
  \freenames{P|Q} := \freenames{P} \cup \freenames{Q}
  \and \\
  \freenames{@{x}} := \{ x \}
\end{mathpar}

$\pi$
$\quotep{\pi}$

$\freenames{-} : \pi \to \mathcal{P}(\quotep{\pi})$

\begin{eqnarray*}
  \freenames{\pzero} & := & \emptyset \\
  \freenames{x?(y).P} & := & \{ x \} \cup (\freenames{P} \setminus \{ y \}) \\
  \freenames{x!\langle P \rangle} & := & \{ x \} \cup \{ P \} \\
  \freenames{P|Q} & := & \freenames{P} \cup \freenames{Q} \\
  \freenames{\dropn{x}} & := & \{ x \}
\end{eqnarray*}

The bound names of a process, $\boundnames{P}$, are those names occurring in $P$
that are not free. For example, in $x?(y).0$, the name $x$ is free, while $y$ is bound.

\begin{mathpar}
  \inferrule* [lab=monoidal-laws] {} { P|Q \equiv Q|P \and P|0 \equiv P \and P|(Q|R) \equiv (P|Q)|R }
\end{mathpar}

\begin{mathpar}
  \inferrule* [lab=alpha-equivalence] {} { (x)P \equiv (y)P\{y/x\} \and y \not\in \freenames{P} }
\end{mathpar}

\begin{definition}
Then two processes, $P,Q$, are alpha-equivalent if $P = Q\{\vec{y}/\vec{x}\}$ for
some $\vec{x} \in \boundnames{Q},\vec{y} \in \boundnames{P}$, where $Q\{\vec{y}/\vec{x}\}$
denotes the capture-avoiding substitution of $\vec{y}$ for $\vec{x}$ in $Q$.
\end{definition}

\begin{definition}
  The {\em structural congruence} \cite{SangiorgiWalker} , $\equiv$,
  between processes is the least congruence containing
  alpha-equivalence, satisfying the abelian monoid laws
  (associativity, commutativity and $\pzero$ as identity) for parallel
  composition $|$ and for summation $+$.
\end{definition}

\subsection{Name equivalence}

We take name equivalence, written $\nameeq$, to be the smallest
equivalence relation generated by the following rules.

\begin{mathpar}
\inferrule*[lab=Quote-drop]
{ }
{ \quotep{@{x}} \nameeq x }

\inferrule*[lab=Struct-equiv]
{ P \scong Q }
{ \quotep{P} \nameeq \quotep{Q} }
\end{mathpar}

The astute reader will have noticed that the mutual recursion of names
and processes imposes a mutual recursion on alpha-equivalence and
structural equivalence via name-equivalence. Fortunately, all of this
works out pleasantly and we may calculate in the natural way, free of
concern. The reader interested in the details is referred to the
appendix \ref{appendix:rho_details}.

\subsection{Substitution}

We use $\Proc$ for the set of processes, $\QProc$ for the set of
names, and $\id{\{}\vec{y} / \vec{x} \id{\}}$ to denote partial maps,
$s : \QProc \rightarrow \QProc$. A map, $s$ lifts, uniquely, to a map
on process terms, $\widehat{s} : \Proc \rightarrow \Proc$ by the
following equations.

\begin{mathpar}
  (0) \psubstp{Q}{P} := 0 \\
  (R \juxtap S) \psubstp{Q}{P}
  :=    
  (R)\psubstp{Q}{P} \juxtap (S) \psubstp{Q}{P} \\
  (x?(y).R) \psubstp{Q}{P}    
  :=    
  (x)\substp{Q}{P} (z)\concat( (R \psubstn{z}{y}) \psubstp{Q}{P} ) \\
  (\lift{x}{R}) \psubstp{Q}{P}  
  :=
  \lift{(x)\substp{Q}{P}}{ R \psubstp{Q}{P} } \\
%   (\dropn{x})  \psubstp{Q}{P}       
%   := 
%   \left\{ 
%     \begin{array}{ccc} 
%       \dropn{\quotep{Q}} & & x \nameeq \quotep{P} \\
%       \dropn{x} & & otherwise \\
%     \end{array}
%   \right. 
  (\dropn{x})  \psubstp{Q}{P}       
  := 
  \left\{ 
    \begin{array}{ccc} 
      Q & & x \nameeq \quotep{P} \\
      \dropn{x} & & otherwise \\
    \end{array}
  \right.
\end{mathpar}
 

where

\begin{eqnarray}
  (x)\id{\{} \lpquote Q \rpquote / \lpquote P \rpquote \id{\}}            = 
  \left\{ 
    \begin{array}{ccc}
      \lpquote Q \rpquote & & x \nameeq \lpquote P \rpquote \\
      x & & otherwise \\
    \end{array}
  \right. \nonumber
\end{eqnarray}

and $z$ is chosen distinct from $\quotep{P}$, $\quotep{Q}$, the free
names in $Q$, and all the names in $R$. Our $\alpha$-equivalence will
be built in the standard way from this substitution.

\begin{remark}\label{rem:no_self_referential_names}
  One consequence of these definitions is that $\forall P. \quotep{P}
  \not\in \freenames{P}$.
\end{remark}

\subsection{ Dynamic quote: an example }

Anticipating something of what's to come, consider applying the
substitution, $\widehat{\id{\{}u / z \id{\}}}$, to the following pair
of processes, $\lift{w}{y!(z)}$ and $w[ \lpquote y!(z) \rpquote ]$.

\begin{eqnarray}
	\lift{w}{y!(z)}\widehat{\id{\{}u / z \id{\}}}
		& = &
		\lift{w}{y!(u)} \nonumber\\
	w[ \lpquote y!(z) \rpquote ] \widehat{ \id{\{}u / z \id{\}} }
		& = &
		w[ \lpquote y!(z) \rpquote ] \nonumber
\end{eqnarray}

Because the body of the process between quotes is impervious to
substitution, we get radically different answers. In fact, by
examining the first process in an input context,
e.g. $x?(z).\lift{w}{y!(z)}$, we see that the process under the lift
operator may be shaped by prefixed inputs binding a name inside it. In
this sense, the lift operator will be seen as a way to dynamically
construct processes before reifying them as names.

Finally equipped with these standard features we can present the
dynamics of the calculus.

\subsubsection{Operational semantics} 

Finally, we introduce the computational dynamics. What marks these
algebras as distinct from other more traditionally studied algebraic
structures, e.g. vector spaces or polynomial rings, is the manner in
which dynamics is captured. In traditional structures, dynamics is typically
expressed through morphisms between such structures, as in linear maps
between vector spaces or morphisms between rings. In algebras
associated with the semantics of computation, the dynamics is
expressed as part of the algebraic structure itself, through a
reduction reduction relation typically denoted by $\red$. Below, we
give a recursive presentation of this relation for the calculus used
in the encoding.

$\red \subseteq \pi \times \pi$
$\red : \pi \to \mathcal{P}(\pi)$

\begin{mathpar}
  \inferrule* [lab=Comm] { \textsf{match}( x_{src}, x_{trgt} ) } { x_{trgt}?(y)P \; | \; x_{src}!\langle {Q} \rangle \red P\{\quotep{Q}/y}\} }
  \and \\
  \inferrule* [lab=Par] {{P} \red {P}'} {{{P} | {Q}} \red {{P}' | {Q}}}
  \and
  \inferrule* [lab=Equiv]{{{P} \scong {P}'} \andalso {{P}' \red {Q}'} \andalso {{Q}' \scong {Q}}}{{P} \red {Q}}
\end{mathpar}

\begin{eqnarray*}
  match_{\equiv} (\quotep{P},\quotep{Q}) & := & P \equiv Q \\
  match_{\dagger}(\quotep{P},\quotep{Q}) & := & \forall R. P|Q \red^{*} R => R \red^{*} 0 \\
  match_{K}(\quotep{P},\quotep{Q}) & := & K \mbox{ for some context } K
\end{eqnarray*}

$u?(x)P | u!\langle Q \rangle \red P\{\quotep{Q}/x\}$

%We write $\wred$ for $\red^*$, and $P\red$ if $\exists Q $ such that $ P \red Q$.
We write $P\red$ if $\exists Q $ such that $ P \red Q$ and $P\not\red$, otherwise.

\section{Replication}

As mentioned before, it is known that replication (and hence
recursion) can be implemented in a higher-order process algebra
\cite{SangiorgiWalker}. As our first example of calculation with the
machinery thus far presented we give the construction explicitly in
the {\rhoc}.

\begin{eqnarray}
	D_{x} & := & \prefix{x}{y}{(\binpar{\outputp{x}{y}}{@{y}})} \nonumber\\
	\bangp_{x}{P} & := & \binpar{{x}!\langle{\binpar{D_{x}}{P}}\rangle}{D_{x}} \nonumber
\end{eqnarray}

\begin{eqnarray}
	\bangp_{x}{P} & & \nonumber\\
	=
	& {x}!\langle{(\prefix{x}{y}{(\outputp{x}{y} | @{y})) | P}}\rangle 
	      | \prefix{x}{y}{(\outputp{x}{y} | @{y})} & \nonumber\\
	\red
	& (\outputp{x}{y} | @{y})\substn{\quotep{(\prefix{x}{y}{(@{y} | \outputp{x}{y})) | P}}}{y} & \nonumber\\
	=
	& \outputp{x}{\quotep{(\prefix{x}{y}{(\outputp{x}{y} | @{y})) | P}}}
	  | {(\prefix{x}{y}{(\outputp{x}{y} | @{y})) | P}} & \nonumber\\
	\red
	& \ldots & \nonumber\\
	\red^*
	& P | P | \ldots & \nonumber
\end{eqnarray}

Of course, this encoding, as an implementation, runs away, unfolding
$\bangp{P}$ eagerly. A lazier and more implementable replication
operator, restricted to input-guarded processes, may be obtained as follows.

\begin{eqnarray}
\bangp{\prefix{u}{v}{P}} 
	:= 
	\binpar{\lift{x}{\prefix{u}{v}{(\binpar{D(x)}{P})}}}{D(x)} \nonumber
\end{eqnarray}

\begin{remark}
  Note that the lazier definition still does not deal with summation
  or mixed summation (i.e. sums over input and output). The reader is
  invited to construct definitions of replication that deal with these
  features. 

  Further, the definitions are parameterized in a name, $x$. Can you,
  gentle reader, make a definition that eliminates this parameter and
  guarantees no accidental interaction between the replication
  machinery and the process being replicated -- i.e. no accidental
  sharing of names used by the process to get its work done and the
  name(s) used by the replication to effect copying. This latter
  revision of the definition of replication is crucial to obtaining
  the expected identity $!!P \sim !P$.
\end{remark}

\begin{remark}\label{rem:paradoxical_combinator}
  The reader familiar with the lambda calculus will have noticed the
  similarity between $D$ and the paradoxical combinator.

  [Ed. note: the existence of this seems to suggest we have to be more
  restrictive on the set of processes and names we admit if we are to
  support no-cloning.]
\end{remark}

\subsubsection{Bisimulation}

The computational dynamics gives rise to another kind of equivalence,
the equivalence of computational behavior. As previously mentioned
this is typically captured \emph{via} some form of bisimulation.

% The notion we use in this paper is weak barbed bisimulation
% \cite{milner91polyadicpi}.

The notion we use in this paper is derived from weak barbed
bisimulation \cite{milner91polyadicpi}. 

\begin{definition}
An \emph{observation relation}, $\downarrow_{\mathcal N}$, over a set
of names, $\mathcal N$, is the smallest relation satisfying the rules
below.

\infrule[Out-barb]{y \in {\mathcal N}, \; x \nameeq y}
		  {\outputp{x}{v} \downarrow_{\mathcal N} x}
\infrule[Par-barb]{\mbox{$P\downarrow_{\mathcal N} x$ or $Q\downarrow_{\mathcal N} x$}}
		  {\binpar{P}{Q} \downarrow_{\mathcal N} x}

We write $P \Downarrow_{\mathcal N} x$ if there is $Q$ such that 
$P \wred Q$ and $Q \downarrow_{\mathcal N} x$.
\end{definition}

\begin{definition}
%\label{def.bbisim}
An  ${\mathcal N}$-\emph{barbed bisimulation} over a set of names, ${\mathcal N}$, is a symmetric binary relation 
${\mathcal S}_{\mathcal N}$ between agents such that $P\rel{S}_{\mathcal N}Q$ implies:
\begin{enumerate}
\item If $P \red P'$ then $Q \wred Q'$ and $P'\rel{S}_{\mathcal N} Q'$.
\item If $P\downarrow_{\mathcal N} x$, then $Q\Downarrow_{\mathcal N} x$.
\end{enumerate}
$P$ is ${\mathcal N}$-barbed bisimilar to $Q$, written
$P \wbbisim_{\mathcal N} Q$, if $P \rel{S}_{\mathcal N} Q$ for some ${\mathcal N}$-barbed bisimulation ${\mathcal S}_{\mathcal N}$.
\end{definition}

$\mathcal{R} \subseteq \pi \times \pi$

$P \mathcal{R} Q => \forall P'. P \red P' \Rightarrow \exists Q'. Q \red Q', P' \mathcal{R} Q'$

$P \vdash x \Rightarrow Q \vdash x$

\begin{mathpar}
  \inferrule*[lab=Out-barb]{x \nameeq y}{{y}!\langle{Q}\rangle \vdash x}
  \and
  \inferrule*[lab=Par-barb]{\mbox{$P\vdash x$ or $Q\vdash x$}}{\binpar{P}{Q} \vdash x}
\end{mathpar}

\subsubsection{Contexts}

One of the principle advantages of computational calculi like the
$\pi$-calculus is a well-defined notion of context,
contextual-equivalence and a correlation between
contextual-equivalence and notions of bisimulation. The notion of
context allows the decomposition of a process into (sub-)process and
its syntactic environment, its context. Thus, a context may be
thought of as a process with a ``hole'' (written $\Box$) in it. The
application of a context $M$ to a process $P$, written $M[P]$, is
tantamount to filling the hole in $M$ with $P$. In this paper we do
not need the full weight of this theory, but do make use of the notion
of context in the proof the main theorem. 

\begin{mathpar}
  \inferrule* [lab=summation] {} {{M_{M},M_{N}} \bc \Box \;|\; x.M_{A} \;|\; M_{M}+M_{N}}
  \and
  \inferrule* [lab=agent] {} {{M_{A}} \bc (\vec{x})M_{P} \;| \; \clift{P_0,\ldots,M_{P},\ldots,P_N}}
  \and \\
  \inferrule* [lab=process] {} {{M_{P}} \bc M_{N} \;| \;P|M_{P} }
\end{mathpar} 

\begin{mathpar}
  \inferrule* [lab=sychronization] {} {M_{N} \bc \Box \;|\; x?M_{F} \;|\; x!M_{C}}
  \and
  \inferrule* [lab=abstraction] {} {{M_{F}} \bc (x)M_{P} }
  \and
  \inferrule* [lab=concretion] {} {{M_{C}} \bc \langle M_{P} \rangle }
  \and \\
  \inferrule* [lab=process] {} {{M_{P}} \bc M_{N} \;| \;P|M_{P} }
\end{mathpar}

\begin{definition}[contextual application] Given a context $M$, and
  process $P$, we define the \emph{contextual application}, $M[P] :=
  M\{P/\Box\}$. That is, the contextual application of M to P is the
  substitution of $P$ for $\Box$ in $M$.
\end{definition}

$\meaningof{-} : L \to \mathcal{P}(\pi)$

\begin{mathpar}
  \inferrule* [lab=collection] {} {\meaningof{true} = \pi, \and \meaningof{~E} = \pi \setminus \meaningof{E}, \and \meaningof{E_{1} \& E_{2}} = \meaningof{E_{1}} \cap \meaningof{E_{2}}}
\end{mathpar}

\begin{mathpar}
  \inferrule* [lab=structure] {} {\meaningof{0} = \{ P \in \pi | P \equiv 0 \}, \and \\ \meaningof{E_1 | E_2} = \{ P \in \pi | P \equiv P_{1} | P_{2}, P_{1} \in \meaningof{E_{1}}, P_{2} \in \meaningof{E_2}\} }
\end{mathpar}

\begin{mathpar}
 \inferrule* [lab=behavior] {} {\meaningof{\langle a?b \rangle E} = \{ P \in \pi | P \equiv Q | u?(y)P', \\ \and \\\\ \and \\ \;\;\; u \in \meaningof{a}, \forall z.P'\{z/y\} \in \meaningof{E\{z/b\}}\}, \and \\ \meaningof{a!E} = \{ P \in \pi | P \equiv Q | x!\langle P' \rangle, x \in \meaningof{a} P' \in \meaningof{E}\} }
\end{mathpar}

\begin{mathpar}
 \inferrule* [lab=nominal] {} {\meaningof{\quotep{E}} = \{ \quotep{P} \in \quotep{\pi} | P \in \meaningof{E} \}, \and \meaningof{\quotep{P}} = \{ \quotep{Q} \in \quotep{\pi} | P \equiv Q \} \and \\ \meaningof{@\quotep{E}} = \{ P \in \pi | P \equiv @x, x \in \meaningof{E} \}}
\end{mathpar}

\begin{eqnarray*}
  \\
  \meaningof{-} : TS \to ST
\end{eqnarray*}

\begin{eqnarray*}
  \\
  L : TS \to ST
\end{eqnarray*}

\begin{eqnarray*}
  \\
  P \models E \iff P \in \meaningof{E}
\end{eqnarray*}

\begin{eqnarray*}
  P \approx_{L} Q \iff \forall E \in L. P \models E \iff Q \models E
\end{eqnarray*}

\begin{eqnarray*}
  P \approx_{K} Q
\end{eqnarray*}

\begin{eqnarray*}
  P \approx Q
\end{eqnarray*}

$\approx_{K} = \approx = \approx_{L}$

\subsubsection{Contextual duality}

Note that contexts extend the quotation operation to a family of
operations from processes to names. Given a context, $M$, we can
define a \emph{nominal context}, $\quotep{M}$ by $\quotep{M}[P] :=
\quotep{M[P]}$. To foreshadow what is to come we observe that these
operations enjoy a duality with processes very much like the duality
between vectors and maps from vectors to scalars.

Further, because the calculus is essentially higher-order, we have a
correspondence between contexts and processes. More specifically,
given a name $x$ and a context $M$ we can construct $M^{*}_{x}$ such
that 

\begin{mathpar}
  M^{*}_{x} | \lift{x}{P} \red M[P]
\end{mathpar}

namely,

\begin{mathpar}
  M^{*}_{x} := x?(u).M[\dropn{u}]
\end{mathpar}

The dependence of $M^{*}_{x}$ on a name makes it an abstraction, 

\begin{mathpar}
  M^{*} := (x)x?(u).M[\dropn{u}]
\end{mathpar}

\subsection{Additional notation}

It will sometimes be convenient to denote the process a name
quotes. We already have the notation $x = \quotep{P}$, but it will be
convenient to introduce an alternate notation, $\procn{x}$, when we
want to emphasize the connection to the use of the name. Note that, by
virtue of name equivalence, $\quotep{\procn{x}} \nameeq x$; so, the
notation is consistent with previous definitions.

Further, because names have structure it is possible to effect
substitutions on the basis of that structure. This means we need to
upgrade our notation for substitutions, which we accomplish by
adapting comprehension notation. Thus,

\begin{mathpar}
  P\{ y / x : x \in S \}
\end{mathpar}

is interpreted to mean the process derived from P by replacing (in a
capture-avoiding manner) each occurrence of $x$ in $S$ by $y$. For example,

\begin{mathpar}
  P\{ \quotep{\procn{x}|\procn{x}} / x : x \in \freenames{P} \}
\end{mathpar}

will replace each (occurrence) of a free name $x$ in $P$ by
$\quotep{\procn{x}|\procn{x}}$.

Also, we will avail ourselves of the notation $x^{L}$ and $x^{R}$ to
denote injections of a name into disjoint copies of the name
space. There are numerous ways to accomplish this. One example can be
found in \cite{MeredithR05}. This notation overloads to vectors of
names: $\vec{x}^{\pi} := (x_{i}^{\pi} \; : \; 0 \leq i < |\vec{x}| )$ where $\pi \in \{L,R\}$.

We also use $P^{\Box} := P|\Box$.

In \cite{MeredithR05} an interpretation of the new operator is
given. It turns out that there are several possible interpretations
all enjoying the requisite algebraic properties of the operator (see
\cite{milner91polyadicpi}). We will therefore make liberal use of
$(\nu\; \vec{x})P$.

% subsection the_syntax_and_semantics_of_the_notation_system (end)   

\input{qm2pi.qmops} 

\input{qm2pi.sterngerlach} 

\input{qm2pi.metric} 

% section concurrent_process_calculi (end)

%\input{qm2pi.proofsketch}

% section proof sketch (end)

%\input{qm2pi.slviaknots} 

% section spatial logic via knots (end)

\input{qm2pi.conclusion}

% section conclusion (end)

%\input{qm2pi.dtcodes} 

% section wiring algorithm (end)

\input{qm2pi.ack} 

% section acknowledgments (end)

\newpage


\bibliographystyle{plain}   
\bibliography{../../biblios/main.bib}

\input{qm2pi.rhodetails}

\end{document}

 

%\documentclass[12pt]{llncs}
%\documentclass{jktr}

\usepackage[pdftex]{hyperref}                   
\usepackage {listings}
\usepackage {mathpartir}
\usepackage{bcprules}
%\usepackage{listings}
                       
\usepackage{graphicx} 
%\usepackage[margins=2.5cm,nohead,nofoot]{geometry}
%\usepackage{geometry}
\usepackage{amsfonts}
\usepackage{amstext}
\usepackage{latexsym}
\usepackage{amssymb}
\usepackage{color}


%\include{myPreamble}
\include{qm2pi.local} 

%\ifpdf
%\usepackage[pdftex]{graphicx}
%\else
%\usepackage{graphicx}
%\fi

 % \ifpdf
%  \usepackage{pdfsync}
%  \if


%\title{Brief Article}
%\author{David F. Snyder}
%\author{L.G. Meredith}

%\address{Dept. of Math., Texas State University--San Marcos, San Marcos, TX 78666}
       
\pagestyle{empty}


\begin{document}

\lstset{language=[Objective]Caml,frame=shadowbox}

\input{qm2pi.front}

% section front matter (end)

\input{qm2pi.intro} 
 
% section introduction (end)

% \input{qm2pi.knotations} 

% section notation (end)

\input{qm2pi.process.calculi} 

% section concurrent_process_calculi_and_spatial_logics_ (end)
    
%\input{qm2pi.knots2pi} 

%\input{qm2pi.trefoil} 

%\input{qm2pi.mainthm} 

% subsection basic_interpretation (end)

%\input{qm2pi.rho.presentation} 
\subsection{The syntax and semantics of the notation system}\label{sub:the_syntax_and_semantics_of_the_notation_system} % (fold)

We now summarize a technical presentation of the calculus that
embodies our theory of dynamics. The typical presentation of such a
calculus follows the style of giving generators and relations on
them. The grammar, below, describing term constructors, freely
generates the set of processes, $\Proc$. This set is then quotiented
by a relation known as structural congruence and it is over this set
that the notion of dynamics is expressed. This presentation is
essentially that of \cite{MeredithR05} with the addition of
polyadicity and summation. For readability we have relegated some of
the technical subtleties to an appendix.

\subsubsection{Process grammar}\label{subsub:process_grammar}

\begin{mathpar}
  \inferrule* [lab=synchronization] {} {{M} \bc \pzero \;|\; x?F \;|\; x!C }
  \and
  \inferrule* [lab=abstraction] {} {{F} \bc (x)P}
  \and
  \inferrule* [lab=concretion] {} {{C} \bc \langle Q \rangle}
  \and
  \inferrule* [lab=process] {} {{P,Q} \bc M \;| \;P|Q \;|\; @{x}}
  \and
  \inferrule* [lab=name] {} {{x} \bc \quotep{P}}
\end{mathpar} 

Note that $\vec{x}$ (resp. $\vec{P}$) denotes a vector of names
(resp. processes) of length $|\vec{x}|$ (resp. $|\vec{P}|$). We adopt
the following useful abbreviations.

\begin{mathpar}
   x?(\vec{y}).P := x.(\vec{y})P \and  x\clift{\vec{P}} := x.\clift{\vec{P}}
   \and x!(y) := \lift{x}{\dropn{y}}
   \and \Pi_{i=0}^{n-1}P_i := P_0 | \ldots | P_{n-1}
\end{mathpar}

\subsubsection{Structural congruence}

\paragraph{Free and bound names and alpha-equivalence.} At the
core of structural equivalence is alpha-equivalence which identifies
process that are the same up to a change of variable. Formally, we
recognize the distinction between free and bound names. The free names
of a process, $\freenames{P}$, may be calculated recursively as
follows:

\begin{mathpar}
\freenames{\pzero} := \emptyset
  \and \\
  \freenames{x?(y).P} := \{ x \} \cup (\freenames{P} \setminus \{ y \})
  \and 
  \freenames{x!\langle P \rangle} := \{ x \} \cup \{ P \} 
  \and \\
  \freenames{P|Q} := \freenames{P} \cup \freenames{Q}
  \and \\
  \freenames{@{x}} := \{ x \}
\end{mathpar}

$\pi$
$\quotep{\pi}$

$\freenames{-} : \pi \to \mathcal{P}(\quotep{\pi})$

\begin{eqnarray*}
  \freenames{\pzero} & := & \emptyset \\
  \freenames{x?(y).P} & := & \{ x \} \cup (\freenames{P} \setminus \{ y \}) \\
  \freenames{x!\langle P \rangle} & := & \{ x \} \cup \{ P \} \\
  \freenames{P|Q} & := & \freenames{P} \cup \freenames{Q} \\
  \freenames{\dropn{x}} & := & \{ x \}
\end{eqnarray*}

The bound names of a process, $\boundnames{P}$, are those names occurring in $P$
that are not free. For example, in $x?(y).0$, the name $x$ is free, while $y$ is bound.

\begin{mathpar}
  \inferrule* [lab=monoidal-laws] {} { P|Q \equiv Q|P \and P|0 \equiv P \and P|(Q|R) \equiv (P|Q)|R }
\end{mathpar}

\begin{mathpar}
  \inferrule* [lab=alpha-equivalence] {} { (x)P \equiv (y)P\{y/x\} \and y \not\in \freenames{P} }
\end{mathpar}

\begin{definition}
Then two processes, $P,Q$, are alpha-equivalent if $P = Q\{\vec{y}/\vec{x}\}$ for
some $\vec{x} \in \boundnames{Q},\vec{y} \in \boundnames{P}$, where $Q\{\vec{y}/\vec{x}\}$
denotes the capture-avoiding substitution of $\vec{y}$ for $\vec{x}$ in $Q$.
\end{definition}

\begin{definition}
  The {\em structural congruence} \cite{SangiorgiWalker} , $\equiv$,
  between processes is the least congruence containing
  alpha-equivalence, satisfying the abelian monoid laws
  (associativity, commutativity and $\pzero$ as identity) for parallel
  composition $|$ and for summation $+$.
\end{definition}

\subsection{Name equivalence}

We take name equivalence, written $\nameeq$, to be the smallest
equivalence relation generated by the following rules.

\begin{mathpar}
\inferrule*[lab=Quote-drop]
{ }
{ \quotep{@{x}} \nameeq x }

\inferrule*[lab=Struct-equiv]
{ P \scong Q }
{ \quotep{P} \nameeq \quotep{Q} }
\end{mathpar}

The astute reader will have noticed that the mutual recursion of names
and processes imposes a mutual recursion on alpha-equivalence and
structural equivalence via name-equivalence. Fortunately, all of this
works out pleasantly and we may calculate in the natural way, free of
concern. The reader interested in the details is referred to the
appendix \ref{appendix:rho_details}.

\subsection{Substitution}

We use $\Proc$ for the set of processes, $\QProc$ for the set of
names, and $\id{\{}\vec{y} / \vec{x} \id{\}}$ to denote partial maps,
$s : \QProc \rightarrow \QProc$. A map, $s$ lifts, uniquely, to a map
on process terms, $\widehat{s} : \Proc \rightarrow \Proc$ by the
following equations.

\begin{mathpar}
  (0) \psubstp{Q}{P} := 0 \\
  (R \juxtap S) \psubstp{Q}{P}
  :=    
  (R)\psubstp{Q}{P} \juxtap (S) \psubstp{Q}{P} \\
  (x?(y).R) \psubstp{Q}{P}    
  :=    
  (x)\substp{Q}{P} (z)\concat( (R \psubstn{z}{y}) \psubstp{Q}{P} ) \\
  (\lift{x}{R}) \psubstp{Q}{P}  
  :=
  \lift{(x)\substp{Q}{P}}{ R \psubstp{Q}{P} } \\
%   (\dropn{x})  \psubstp{Q}{P}       
%   := 
%   \left\{ 
%     \begin{array}{ccc} 
%       \dropn{\quotep{Q}} & & x \nameeq \quotep{P} \\
%       \dropn{x} & & otherwise \\
%     \end{array}
%   \right. 
  (\dropn{x})  \psubstp{Q}{P}       
  := 
  \left\{ 
    \begin{array}{ccc} 
      Q & & x \nameeq \quotep{P} \\
      \dropn{x} & & otherwise \\
    \end{array}
  \right.
\end{mathpar}
 

where

\begin{eqnarray}
  (x)\id{\{} \lpquote Q \rpquote / \lpquote P \rpquote \id{\}}            = 
  \left\{ 
    \begin{array}{ccc}
      \lpquote Q \rpquote & & x \nameeq \lpquote P \rpquote \\
      x & & otherwise \\
    \end{array}
  \right. \nonumber
\end{eqnarray}

and $z$ is chosen distinct from $\quotep{P}$, $\quotep{Q}$, the free
names in $Q$, and all the names in $R$. Our $\alpha$-equivalence will
be built in the standard way from this substitution.

\begin{remark}\label{rem:no_self_referential_names}
  One consequence of these definitions is that $\forall P. \quotep{P}
  \not\in \freenames{P}$.
\end{remark}

\subsection{ Dynamic quote: an example }

Anticipating something of what's to come, consider applying the
substitution, $\widehat{\id{\{}u / z \id{\}}}$, to the following pair
of processes, $\lift{w}{y!(z)}$ and $w[ \lpquote y!(z) \rpquote ]$.

\begin{eqnarray}
	\lift{w}{y!(z)}\widehat{\id{\{}u / z \id{\}}}
		& = &
		\lift{w}{y!(u)} \nonumber\\
	w[ \lpquote y!(z) \rpquote ] \widehat{ \id{\{}u / z \id{\}} }
		& = &
		w[ \lpquote y!(z) \rpquote ] \nonumber
\end{eqnarray}

Because the body of the process between quotes is impervious to
substitution, we get radically different answers. In fact, by
examining the first process in an input context,
e.g. $x?(z).\lift{w}{y!(z)}$, we see that the process under the lift
operator may be shaped by prefixed inputs binding a name inside it. In
this sense, the lift operator will be seen as a way to dynamically
construct processes before reifying them as names.

Finally equipped with these standard features we can present the
dynamics of the calculus.

\subsubsection{Operational semantics} 

Finally, we introduce the computational dynamics. What marks these
algebras as distinct from other more traditionally studied algebraic
structures, e.g. vector spaces or polynomial rings, is the manner in
which dynamics is captured. In traditional structures, dynamics is typically
expressed through morphisms between such structures, as in linear maps
between vector spaces or morphisms between rings. In algebras
associated with the semantics of computation, the dynamics is
expressed as part of the algebraic structure itself, through a
reduction reduction relation typically denoted by $\red$. Below, we
give a recursive presentation of this relation for the calculus used
in the encoding.

$\red \subseteq \pi \times \pi$
$\red : \pi \to \mathcal{P}(\pi)$

\begin{mathpar}
  \inferrule* [lab=Comm] { \textsf{match}( x_{src}, x_{trgt} ) } { x_{trgt}?(y)P \; | \; x_{src}!\langle {Q} \rangle \red P\{\quotep{Q}/y}\} }
  \and \\
  \inferrule* [lab=Par] {{P} \red {P}'} {{{P} | {Q}} \red {{P}' | {Q}}}
  \and
  \inferrule* [lab=Equiv]{{{P} \scong {P}'} \andalso {{P}' \red {Q}'} \andalso {{Q}' \scong {Q}}}{{P} \red {Q}}
\end{mathpar}

\begin{eqnarray*}
  match_{\equiv} (\quotep{P},\quotep{Q}) & := & P \equiv Q \\
  match_{\dagger}(\quotep{P},\quotep{Q}) & := & \forall R. P|Q \red^{*} R => R \red^{*} 0 \\
  match_{K}(\quotep{P},\quotep{Q}) & := & K \mbox{ for some context } K
\end{eqnarray*}

$u?(x)P | u!\langle Q \rangle \red P\{\quotep{Q}/x\}$

%We write $\wred$ for $\red^*$, and $P\red$ if $\exists Q $ such that $ P \red Q$.
We write $P\red$ if $\exists Q $ such that $ P \red Q$ and $P\not\red$, otherwise.

\section{Replication}

As mentioned before, it is known that replication (and hence
recursion) can be implemented in a higher-order process algebra
\cite{SangiorgiWalker}. As our first example of calculation with the
machinery thus far presented we give the construction explicitly in
the {\rhoc}.

\begin{eqnarray}
	D_{x} & := & \prefix{x}{y}{(\binpar{\outputp{x}{y}}{@{y}})} \nonumber\\
	\bangp_{x}{P} & := & \binpar{{x}!\langle{\binpar{D_{x}}{P}}\rangle}{D_{x}} \nonumber
\end{eqnarray}

\begin{eqnarray}
	\bangp_{x}{P} & & \nonumber\\
	=
	& {x}!\langle{(\prefix{x}{y}{(\outputp{x}{y} | @{y})) | P}}\rangle 
	      | \prefix{x}{y}{(\outputp{x}{y} | @{y})} & \nonumber\\
	\red
	& (\outputp{x}{y} | @{y})\substn{\quotep{(\prefix{x}{y}{(@{y} | \outputp{x}{y})) | P}}}{y} & \nonumber\\
	=
	& \outputp{x}{\quotep{(\prefix{x}{y}{(\outputp{x}{y} | @{y})) | P}}}
	  | {(\prefix{x}{y}{(\outputp{x}{y} | @{y})) | P}} & \nonumber\\
	\red
	& \ldots & \nonumber\\
	\red^*
	& P | P | \ldots & \nonumber
\end{eqnarray}

Of course, this encoding, as an implementation, runs away, unfolding
$\bangp{P}$ eagerly. A lazier and more implementable replication
operator, restricted to input-guarded processes, may be obtained as follows.

\begin{eqnarray}
\bangp{\prefix{u}{v}{P}} 
	:= 
	\binpar{\lift{x}{\prefix{u}{v}{(\binpar{D(x)}{P})}}}{D(x)} \nonumber
\end{eqnarray}

\begin{remark}
  Note that the lazier definition still does not deal with summation
  or mixed summation (i.e. sums over input and output). The reader is
  invited to construct definitions of replication that deal with these
  features. 

  Further, the definitions are parameterized in a name, $x$. Can you,
  gentle reader, make a definition that eliminates this parameter and
  guarantees no accidental interaction between the replication
  machinery and the process being replicated -- i.e. no accidental
  sharing of names used by the process to get its work done and the
  name(s) used by the replication to effect copying. This latter
  revision of the definition of replication is crucial to obtaining
  the expected identity $!!P \sim !P$.
\end{remark}

\begin{remark}\label{rem:paradoxical_combinator}
  The reader familiar with the lambda calculus will have noticed the
  similarity between $D$ and the paradoxical combinator.

  [Ed. note: the existence of this seems to suggest we have to be more
  restrictive on the set of processes and names we admit if we are to
  support no-cloning.]
\end{remark}

\subsubsection{Bisimulation}

The computational dynamics gives rise to another kind of equivalence,
the equivalence of computational behavior. As previously mentioned
this is typically captured \emph{via} some form of bisimulation.

% The notion we use in this paper is weak barbed bisimulation
% \cite{milner91polyadicpi}.

The notion we use in this paper is derived from weak barbed
bisimulation \cite{milner91polyadicpi}. 

\begin{definition}
An \emph{observation relation}, $\downarrow_{\mathcal N}$, over a set
of names, $\mathcal N$, is the smallest relation satisfying the rules
below.

\infrule[Out-barb]{y \in {\mathcal N}, \; x \nameeq y}
		  {\outputp{x}{v} \downarrow_{\mathcal N} x}
\infrule[Par-barb]{\mbox{$P\downarrow_{\mathcal N} x$ or $Q\downarrow_{\mathcal N} x$}}
		  {\binpar{P}{Q} \downarrow_{\mathcal N} x}

We write $P \Downarrow_{\mathcal N} x$ if there is $Q$ such that 
$P \wred Q$ and $Q \downarrow_{\mathcal N} x$.
\end{definition}

\begin{definition}
%\label{def.bbisim}
An  ${\mathcal N}$-\emph{barbed bisimulation} over a set of names, ${\mathcal N}$, is a symmetric binary relation 
${\mathcal S}_{\mathcal N}$ between agents such that $P\rel{S}_{\mathcal N}Q$ implies:
\begin{enumerate}
\item If $P \red P'$ then $Q \wred Q'$ and $P'\rel{S}_{\mathcal N} Q'$.
\item If $P\downarrow_{\mathcal N} x$, then $Q\Downarrow_{\mathcal N} x$.
\end{enumerate}
$P$ is ${\mathcal N}$-barbed bisimilar to $Q$, written
$P \wbbisim_{\mathcal N} Q$, if $P \rel{S}_{\mathcal N} Q$ for some ${\mathcal N}$-barbed bisimulation ${\mathcal S}_{\mathcal N}$.
\end{definition}

$\mathcal{R} \subseteq \pi \times \pi$

$P \mathcal{R} Q => \forall P'. P \red P' \Rightarrow \exists Q'. Q \red Q', P' \mathcal{R} Q'$

$P \vdash x \Rightarrow Q \vdash x$

\begin{mathpar}
  \inferrule*[lab=Out-barb]{x \nameeq y}{{y}!\langle{Q}\rangle \vdash x}
  \and
  \inferrule*[lab=Par-barb]{\mbox{$P\vdash x$ or $Q\vdash x$}}{\binpar{P}{Q} \vdash x}
\end{mathpar}

\subsubsection{Contexts}

One of the principle advantages of computational calculi like the
$\pi$-calculus is a well-defined notion of context,
contextual-equivalence and a correlation between
contextual-equivalence and notions of bisimulation. The notion of
context allows the decomposition of a process into (sub-)process and
its syntactic environment, its context. Thus, a context may be
thought of as a process with a ``hole'' (written $\Box$) in it. The
application of a context $M$ to a process $P$, written $M[P]$, is
tantamount to filling the hole in $M$ with $P$. In this paper we do
not need the full weight of this theory, but do make use of the notion
of context in the proof the main theorem. 

\begin{mathpar}
  \inferrule* [lab=summation] {} {{M_{M},M_{N}} \bc \Box \;|\; x.M_{A} \;|\; M_{M}+M_{N}}
  \and
  \inferrule* [lab=agent] {} {{M_{A}} \bc (\vec{x})M_{P} \;| \; \clift{P_0,\ldots,M_{P},\ldots,P_N}}
  \and \\
  \inferrule* [lab=process] {} {{M_{P}} \bc M_{N} \;| \;P|M_{P} }
\end{mathpar} 

\begin{mathpar}
  \inferrule* [lab=sychronization] {} {M_{N} \bc \Box \;|\; x?M_{F} \;|\; x!M_{C}}
  \and
  \inferrule* [lab=abstraction] {} {{M_{F}} \bc (x)M_{P} }
  \and
  \inferrule* [lab=concretion] {} {{M_{C}} \bc \langle M_{P} \rangle }
  \and \\
  \inferrule* [lab=process] {} {{M_{P}} \bc M_{N} \;| \;P|M_{P} }
\end{mathpar}

\begin{definition}[contextual application] Given a context $M$, and
  process $P$, we define the \emph{contextual application}, $M[P] :=
  M\{P/\Box\}$. That is, the contextual application of M to P is the
  substitution of $P$ for $\Box$ in $M$.
\end{definition}

$\meaningof{-} : L \to \mathcal{P}(\pi)$

\begin{mathpar}
  \inferrule* [lab=collection] {} {\meaningof{true} = \pi, \and \meaningof{~E} = \pi \setminus \meaningof{E}, \and \meaningof{E_{1} \& E_{2}} = \meaningof{E_{1}} \cap \meaningof{E_{2}}}
\end{mathpar}

\begin{mathpar}
  \inferrule* [lab=structure] {} {\meaningof{0} = \{ P \in \pi | P \equiv 0 \}, \and \\ \meaningof{E_1 | E_2} = \{ P \in \pi | P \equiv P_{1} | P_{2}, P_{1} \in \meaningof{E_{1}}, P_{2} \in \meaningof{E_2}\} }
\end{mathpar}

\begin{mathpar}
 \inferrule* [lab=behavior] {} {\meaningof{\langle a?b \rangle E} = \{ P \in \pi | P \equiv Q | u?(y)P', \\ \and \\\\ \and \\ \;\;\; u \in \meaningof{a}, \forall z.P'\{z/y\} \in \meaningof{E\{z/b\}}\}, \and \\ \meaningof{a!E} = \{ P \in \pi | P \equiv Q | x!\langle P' \rangle, x \in \meaningof{a} P' \in \meaningof{E}\} }
\end{mathpar}

\begin{mathpar}
 \inferrule* [lab=nominal] {} {\meaningof{\quotep{E}} = \{ \quotep{P} \in \quotep{\pi} | P \in \meaningof{E} \}, \and \meaningof{\quotep{P}} = \{ \quotep{Q} \in \quotep{\pi} | P \equiv Q \} \and \\ \meaningof{@\quotep{E}} = \{ P \in \pi | P \equiv @x, x \in \meaningof{E} \}}
\end{mathpar}

\begin{eqnarray*}
  \\
  \meaningof{-} : TS \to ST
\end{eqnarray*}

\begin{eqnarray*}
  \\
  L : TS \to ST
\end{eqnarray*}

\begin{eqnarray*}
  \\
  P \models E \iff P \in \meaningof{E}
\end{eqnarray*}

\begin{eqnarray*}
  P \approx_{L} Q \iff \forall E \in L. P \models E \iff Q \models E
\end{eqnarray*}

\begin{eqnarray*}
  P \approx_{K} Q
\end{eqnarray*}

\begin{eqnarray*}
  P \approx Q
\end{eqnarray*}

$\approx_{K} = \approx = \approx_{L}$

\subsubsection{Contextual duality}

Note that contexts extend the quotation operation to a family of
operations from processes to names. Given a context, $M$, we can
define a \emph{nominal context}, $\quotep{M}$ by $\quotep{M}[P] :=
\quotep{M[P]}$. To foreshadow what is to come we observe that these
operations enjoy a duality with processes very much like the duality
between vectors and maps from vectors to scalars.

Further, because the calculus is essentially higher-order, we have a
correspondence between contexts and processes. More specifically,
given a name $x$ and a context $M$ we can construct $M^{*}_{x}$ such
that 

\begin{mathpar}
  M^{*}_{x} | \lift{x}{P} \red M[P]
\end{mathpar}

namely,

\begin{mathpar}
  M^{*}_{x} := x?(u).M[\dropn{u}]
\end{mathpar}

The dependence of $M^{*}_{x}$ on a name makes it an abstraction, 

\begin{mathpar}
  M^{*} := (x)x?(u).M[\dropn{u}]
\end{mathpar}

\subsection{Additional notation}

It will sometimes be convenient to denote the process a name
quotes. We already have the notation $x = \quotep{P}$, but it will be
convenient to introduce an alternate notation, $\procn{x}$, when we
want to emphasize the connection to the use of the name. Note that, by
virtue of name equivalence, $\quotep{\procn{x}} \nameeq x$; so, the
notation is consistent with previous definitions.

Further, because names have structure it is possible to effect
substitutions on the basis of that structure. This means we need to
upgrade our notation for substitutions, which we accomplish by
adapting comprehension notation. Thus,

\begin{mathpar}
  P\{ y / x : x \in S \}
\end{mathpar}

is interpreted to mean the process derived from P by replacing (in a
capture-avoiding manner) each occurrence of $x$ in $S$ by $y$. For example,

\begin{mathpar}
  P\{ \quotep{\procn{x}|\procn{x}} / x : x \in \freenames{P} \}
\end{mathpar}

will replace each (occurrence) of a free name $x$ in $P$ by
$\quotep{\procn{x}|\procn{x}}$.

Also, we will avail ourselves of the notation $x^{L}$ and $x^{R}$ to
denote injections of a name into disjoint copies of the name
space. There are numerous ways to accomplish this. One example can be
found in \cite{MeredithR05}. This notation overloads to vectors of
names: $\vec{x}^{\pi} := (x_{i}^{\pi} \; : \; 0 \leq i < |\vec{x}| )$ where $\pi \in \{L,R\}$.

We also use $P^{\Box} := P|\Box$.

In \cite{MeredithR05} an interpretation of the new operator is
given. It turns out that there are several possible interpretations
all enjoying the requisite algebraic properties of the operator (see
\cite{milner91polyadicpi}). We will therefore make liberal use of
$(\nu\; \vec{x})P$.

% subsection the_syntax_and_semantics_of_the_notation_system (end)   

\input{qm2pi.qmops} 

\input{qm2pi.sterngerlach} 

\input{qm2pi.metric} 

% section concurrent_process_calculi (end)

%\input{qm2pi.proofsketch}

% section proof sketch (end)

%\input{qm2pi.slviaknots} 

% section spatial logic via knots (end)

\input{qm2pi.conclusion}

% section conclusion (end)

%\input{qm2pi.dtcodes} 

% section wiring algorithm (end)

\input{qm2pi.ack} 

% section acknowledgments (end)

\newpage


\bibliographystyle{plain}   
\bibliography{../../biblios/main.bib}

\input{qm2pi.rhodetails}

\end{document}

 

% subsection basic_interpretation (end)

%\input{qm2pi.rho.presentation} 
\subsection{The syntax and semantics of the notation system}\label{sub:the_syntax_and_semantics_of_the_notation_system} % (fold)

We now summarize a technical presentation of the calculus that
embodies our theory of dynamics. The typical presentation of such a
calculus follows the style of giving generators and relations on
them. The grammar, below, describing term constructors, freely
generates the set of processes, $\Proc$. This set is then quotiented
by a relation known as structural congruence and it is over this set
that the notion of dynamics is expressed. This presentation is
essentially that of \cite{MeredithR05} with the addition of
polyadicity and summation. For readability we have relegated some of
the technical subtleties to an appendix.

\subsubsection{Process grammar}\label{subsub:process_grammar}

\begin{mathpar}
  \inferrule* [lab=synchronization] {} {{M} \bc \pzero \;|\; x?F \;|\; x!C }
  \and
  \inferrule* [lab=abstraction] {} {{F} \bc (x)P}
  \and
  \inferrule* [lab=concretion] {} {{C} \bc \langle Q \rangle}
  \and
  \inferrule* [lab=process] {} {{P,Q} \bc M \;| \;P|Q \;|\; @{x}}
  \and
  \inferrule* [lab=name] {} {{x} \bc \quotep{P}}
\end{mathpar} 

Note that $\vec{x}$ (resp. $\vec{P}$) denotes a vector of names
(resp. processes) of length $|\vec{x}|$ (resp. $|\vec{P}|$). We adopt
the following useful abbreviations.

\begin{mathpar}
   x?(\vec{y}).P := x.(\vec{y})P \and  x\clift{\vec{P}} := x.\clift{\vec{P}}
   \and x!(y) := \lift{x}{\dropn{y}}
   \and \Pi_{i=0}^{n-1}P_i := P_0 | \ldots | P_{n-1}
\end{mathpar}

\subsubsection{Structural congruence}

\paragraph{Free and bound names and alpha-equivalence.} At the
core of structural equivalence is alpha-equivalence which identifies
process that are the same up to a change of variable. Formally, we
recognize the distinction between free and bound names. The free names
of a process, $\freenames{P}$, may be calculated recursively as
follows:

\begin{mathpar}
\freenames{\pzero} := \emptyset
  \and \\
  \freenames{x?(y).P} := \{ x \} \cup (\freenames{P} \setminus \{ y \})
  \and 
  \freenames{x!\langle P \rangle} := \{ x \} \cup \{ P \} 
  \and \\
  \freenames{P|Q} := \freenames{P} \cup \freenames{Q}
  \and \\
  \freenames{@{x}} := \{ x \}
\end{mathpar}

$\pi$
$\quotep{\pi}$

$\freenames{-} : \pi \to \mathcal{P}(\quotep{\pi})$

\begin{eqnarray*}
  \freenames{\pzero} & := & \emptyset \\
  \freenames{x?(y).P} & := & \{ x \} \cup (\freenames{P} \setminus \{ y \}) \\
  \freenames{x!\langle P \rangle} & := & \{ x \} \cup \{ P \} \\
  \freenames{P|Q} & := & \freenames{P} \cup \freenames{Q} \\
  \freenames{\dropn{x}} & := & \{ x \}
\end{eqnarray*}

The bound names of a process, $\boundnames{P}$, are those names occurring in $P$
that are not free. For example, in $x?(y).0$, the name $x$ is free, while $y$ is bound.

\begin{mathpar}
  \inferrule* [lab=monoidal-laws] {} { P|Q \equiv Q|P \and P|0 \equiv P \and P|(Q|R) \equiv (P|Q)|R }
\end{mathpar}

\begin{mathpar}
  \inferrule* [lab=alpha-equivalence] {} { (x)P \equiv (y)P\{y/x\} \and y \not\in \freenames{P} }
\end{mathpar}

\begin{definition}
Then two processes, $P,Q$, are alpha-equivalent if $P = Q\{\vec{y}/\vec{x}\}$ for
some $\vec{x} \in \boundnames{Q},\vec{y} \in \boundnames{P}$, where $Q\{\vec{y}/\vec{x}\}$
denotes the capture-avoiding substitution of $\vec{y}$ for $\vec{x}$ in $Q$.
\end{definition}

\begin{definition}
  The {\em structural congruence} \cite{SangiorgiWalker} , $\equiv$,
  between processes is the least congruence containing
  alpha-equivalence, satisfying the abelian monoid laws
  (associativity, commutativity and $\pzero$ as identity) for parallel
  composition $|$ and for summation $+$.
\end{definition}

\subsection{Name equivalence}

We take name equivalence, written $\nameeq$, to be the smallest
equivalence relation generated by the following rules.

\begin{mathpar}
\inferrule*[lab=Quote-drop]
{ }
{ \quotep{@{x}} \nameeq x }

\inferrule*[lab=Struct-equiv]
{ P \scong Q }
{ \quotep{P} \nameeq \quotep{Q} }
\end{mathpar}

The astute reader will have noticed that the mutual recursion of names
and processes imposes a mutual recursion on alpha-equivalence and
structural equivalence via name-equivalence. Fortunately, all of this
works out pleasantly and we may calculate in the natural way, free of
concern. The reader interested in the details is referred to the
appendix \ref{appendix:rho_details}.

\subsection{Substitution}

We use $\Proc$ for the set of processes, $\QProc$ for the set of
names, and $\id{\{}\vec{y} / \vec{x} \id{\}}$ to denote partial maps,
$s : \QProc \rightarrow \QProc$. A map, $s$ lifts, uniquely, to a map
on process terms, $\widehat{s} : \Proc \rightarrow \Proc$ by the
following equations.

\begin{mathpar}
  (0) \psubstp{Q}{P} := 0 \\
  (R \juxtap S) \psubstp{Q}{P}
  :=    
  (R)\psubstp{Q}{P} \juxtap (S) \psubstp{Q}{P} \\
  (x?(y).R) \psubstp{Q}{P}    
  :=    
  (x)\substp{Q}{P} (z)\concat( (R \psubstn{z}{y}) \psubstp{Q}{P} ) \\
  (\lift{x}{R}) \psubstp{Q}{P}  
  :=
  \lift{(x)\substp{Q}{P}}{ R \psubstp{Q}{P} } \\
%   (\dropn{x})  \psubstp{Q}{P}       
%   := 
%   \left\{ 
%     \begin{array}{ccc} 
%       \dropn{\quotep{Q}} & & x \nameeq \quotep{P} \\
%       \dropn{x} & & otherwise \\
%     \end{array}
%   \right. 
  (\dropn{x})  \psubstp{Q}{P}       
  := 
  \left\{ 
    \begin{array}{ccc} 
      Q & & x \nameeq \quotep{P} \\
      \dropn{x} & & otherwise \\
    \end{array}
  \right.
\end{mathpar}
 

where

\begin{eqnarray}
  (x)\id{\{} \lpquote Q \rpquote / \lpquote P \rpquote \id{\}}            = 
  \left\{ 
    \begin{array}{ccc}
      \lpquote Q \rpquote & & x \nameeq \lpquote P \rpquote \\
      x & & otherwise \\
    \end{array}
  \right. \nonumber
\end{eqnarray}

and $z$ is chosen distinct from $\quotep{P}$, $\quotep{Q}$, the free
names in $Q$, and all the names in $R$. Our $\alpha$-equivalence will
be built in the standard way from this substitution.

\begin{remark}\label{rem:no_self_referential_names}
  One consequence of these definitions is that $\forall P. \quotep{P}
  \not\in \freenames{P}$.
\end{remark}

\subsection{ Dynamic quote: an example }

Anticipating something of what's to come, consider applying the
substitution, $\widehat{\id{\{}u / z \id{\}}}$, to the following pair
of processes, $\lift{w}{y!(z)}$ and $w[ \lpquote y!(z) \rpquote ]$.

\begin{eqnarray}
	\lift{w}{y!(z)}\widehat{\id{\{}u / z \id{\}}}
		& = &
		\lift{w}{y!(u)} \nonumber\\
	w[ \lpquote y!(z) \rpquote ] \widehat{ \id{\{}u / z \id{\}} }
		& = &
		w[ \lpquote y!(z) \rpquote ] \nonumber
\end{eqnarray}

Because the body of the process between quotes is impervious to
substitution, we get radically different answers. In fact, by
examining the first process in an input context,
e.g. $x?(z).\lift{w}{y!(z)}$, we see that the process under the lift
operator may be shaped by prefixed inputs binding a name inside it. In
this sense, the lift operator will be seen as a way to dynamically
construct processes before reifying them as names.

Finally equipped with these standard features we can present the
dynamics of the calculus.

\subsubsection{Operational semantics} 

Finally, we introduce the computational dynamics. What marks these
algebras as distinct from other more traditionally studied algebraic
structures, e.g. vector spaces or polynomial rings, is the manner in
which dynamics is captured. In traditional structures, dynamics is typically
expressed through morphisms between such structures, as in linear maps
between vector spaces or morphisms between rings. In algebras
associated with the semantics of computation, the dynamics is
expressed as part of the algebraic structure itself, through a
reduction reduction relation typically denoted by $\red$. Below, we
give a recursive presentation of this relation for the calculus used
in the encoding.

$\red \subseteq \pi \times \pi$
$\red : \pi \to \mathcal{P}(\pi)$

\begin{mathpar}
  \inferrule* [lab=Comm] { \textsf{match}( x_{src}, x_{trgt} ) } { x_{trgt}?(y)P \; | \; x_{src}!\langle {Q} \rangle \red P\{\quotep{Q}/y}\} }
  \and \\
  \inferrule* [lab=Par] {{P} \red {P}'} {{{P} | {Q}} \red {{P}' | {Q}}}
  \and
  \inferrule* [lab=Equiv]{{{P} \scong {P}'} \andalso {{P}' \red {Q}'} \andalso {{Q}' \scong {Q}}}{{P} \red {Q}}
\end{mathpar}

\begin{eqnarray*}
  match_{\equiv} (\quotep{P},\quotep{Q}) & := & P \equiv Q \\
  match_{\dagger}(\quotep{P},\quotep{Q}) & := & \forall R. P|Q \red^{*} R => R \red^{*} 0 \\
  match_{K}(\quotep{P},\quotep{Q}) & := & K \mbox{ for some context } K
\end{eqnarray*}

$u?(x)P | u!\langle Q \rangle \red P\{\quotep{Q}/x\}$

%We write $\wred$ for $\red^*$, and $P\red$ if $\exists Q $ such that $ P \red Q$.
We write $P\red$ if $\exists Q $ such that $ P \red Q$ and $P\not\red$, otherwise.

\section{Replication}

As mentioned before, it is known that replication (and hence
recursion) can be implemented in a higher-order process algebra
\cite{SangiorgiWalker}. As our first example of calculation with the
machinery thus far presented we give the construction explicitly in
the {\rhoc}.

\begin{eqnarray}
	D_{x} & := & \prefix{x}{y}{(\binpar{\outputp{x}{y}}{@{y}})} \nonumber\\
	\bangp_{x}{P} & := & \binpar{{x}!\langle{\binpar{D_{x}}{P}}\rangle}{D_{x}} \nonumber
\end{eqnarray}

\begin{eqnarray}
	\bangp_{x}{P} & & \nonumber\\
	=
	& {x}!\langle{(\prefix{x}{y}{(\outputp{x}{y} | @{y})) | P}}\rangle 
	      | \prefix{x}{y}{(\outputp{x}{y} | @{y})} & \nonumber\\
	\red
	& (\outputp{x}{y} | @{y})\substn{\quotep{(\prefix{x}{y}{(@{y} | \outputp{x}{y})) | P}}}{y} & \nonumber\\
	=
	& \outputp{x}{\quotep{(\prefix{x}{y}{(\outputp{x}{y} | @{y})) | P}}}
	  | {(\prefix{x}{y}{(\outputp{x}{y} | @{y})) | P}} & \nonumber\\
	\red
	& \ldots & \nonumber\\
	\red^*
	& P | P | \ldots & \nonumber
\end{eqnarray}

Of course, this encoding, as an implementation, runs away, unfolding
$\bangp{P}$ eagerly. A lazier and more implementable replication
operator, restricted to input-guarded processes, may be obtained as follows.

\begin{eqnarray}
\bangp{\prefix{u}{v}{P}} 
	:= 
	\binpar{\lift{x}{\prefix{u}{v}{(\binpar{D(x)}{P})}}}{D(x)} \nonumber
\end{eqnarray}

\begin{remark}
  Note that the lazier definition still does not deal with summation
  or mixed summation (i.e. sums over input and output). The reader is
  invited to construct definitions of replication that deal with these
  features. 

  Further, the definitions are parameterized in a name, $x$. Can you,
  gentle reader, make a definition that eliminates this parameter and
  guarantees no accidental interaction between the replication
  machinery and the process being replicated -- i.e. no accidental
  sharing of names used by the process to get its work done and the
  name(s) used by the replication to effect copying. This latter
  revision of the definition of replication is crucial to obtaining
  the expected identity $!!P \sim !P$.
\end{remark}

\begin{remark}\label{rem:paradoxical_combinator}
  The reader familiar with the lambda calculus will have noticed the
  similarity between $D$ and the paradoxical combinator.

  [Ed. note: the existence of this seems to suggest we have to be more
  restrictive on the set of processes and names we admit if we are to
  support no-cloning.]
\end{remark}

\subsubsection{Bisimulation}

The computational dynamics gives rise to another kind of equivalence,
the equivalence of computational behavior. As previously mentioned
this is typically captured \emph{via} some form of bisimulation.

% The notion we use in this paper is weak barbed bisimulation
% \cite{milner91polyadicpi}.

The notion we use in this paper is derived from weak barbed
bisimulation \cite{milner91polyadicpi}. 

\begin{definition}
An \emph{observation relation}, $\downarrow_{\mathcal N}$, over a set
of names, $\mathcal N$, is the smallest relation satisfying the rules
below.

\infrule[Out-barb]{y \in {\mathcal N}, \; x \nameeq y}
		  {\outputp{x}{v} \downarrow_{\mathcal N} x}
\infrule[Par-barb]{\mbox{$P\downarrow_{\mathcal N} x$ or $Q\downarrow_{\mathcal N} x$}}
		  {\binpar{P}{Q} \downarrow_{\mathcal N} x}

We write $P \Downarrow_{\mathcal N} x$ if there is $Q$ such that 
$P \wred Q$ and $Q \downarrow_{\mathcal N} x$.
\end{definition}

\begin{definition}
%\label{def.bbisim}
An  ${\mathcal N}$-\emph{barbed bisimulation} over a set of names, ${\mathcal N}$, is a symmetric binary relation 
${\mathcal S}_{\mathcal N}$ between agents such that $P\rel{S}_{\mathcal N}Q$ implies:
\begin{enumerate}
\item If $P \red P'$ then $Q \wred Q'$ and $P'\rel{S}_{\mathcal N} Q'$.
\item If $P\downarrow_{\mathcal N} x$, then $Q\Downarrow_{\mathcal N} x$.
\end{enumerate}
$P$ is ${\mathcal N}$-barbed bisimilar to $Q$, written
$P \wbbisim_{\mathcal N} Q$, if $P \rel{S}_{\mathcal N} Q$ for some ${\mathcal N}$-barbed bisimulation ${\mathcal S}_{\mathcal N}$.
\end{definition}

$\mathcal{R} \subseteq \pi \times \pi$

$P \mathcal{R} Q => \forall P'. P \red P' \Rightarrow \exists Q'. Q \red Q', P' \mathcal{R} Q'$

$P \vdash x \Rightarrow Q \vdash x$

\begin{mathpar}
  \inferrule*[lab=Out-barb]{x \nameeq y}{{y}!\langle{Q}\rangle \vdash x}
  \and
  \inferrule*[lab=Par-barb]{\mbox{$P\vdash x$ or $Q\vdash x$}}{\binpar{P}{Q} \vdash x}
\end{mathpar}

\subsubsection{Contexts}

One of the principle advantages of computational calculi like the
$\pi$-calculus is a well-defined notion of context,
contextual-equivalence and a correlation between
contextual-equivalence and notions of bisimulation. The notion of
context allows the decomposition of a process into (sub-)process and
its syntactic environment, its context. Thus, a context may be
thought of as a process with a ``hole'' (written $\Box$) in it. The
application of a context $M$ to a process $P$, written $M[P]$, is
tantamount to filling the hole in $M$ with $P$. In this paper we do
not need the full weight of this theory, but do make use of the notion
of context in the proof the main theorem. 

\begin{mathpar}
  \inferrule* [lab=summation] {} {{M_{M},M_{N}} \bc \Box \;|\; x.M_{A} \;|\; M_{M}+M_{N}}
  \and
  \inferrule* [lab=agent] {} {{M_{A}} \bc (\vec{x})M_{P} \;| \; \clift{P_0,\ldots,M_{P},\ldots,P_N}}
  \and \\
  \inferrule* [lab=process] {} {{M_{P}} \bc M_{N} \;| \;P|M_{P} }
\end{mathpar} 

\begin{mathpar}
  \inferrule* [lab=sychronization] {} {M_{N} \bc \Box \;|\; x?M_{F} \;|\; x!M_{C}}
  \and
  \inferrule* [lab=abstraction] {} {{M_{F}} \bc (x)M_{P} }
  \and
  \inferrule* [lab=concretion] {} {{M_{C}} \bc \langle M_{P} \rangle }
  \and \\
  \inferrule* [lab=process] {} {{M_{P}} \bc M_{N} \;| \;P|M_{P} }
\end{mathpar}

\begin{definition}[contextual application] Given a context $M$, and
  process $P$, we define the \emph{contextual application}, $M[P] :=
  M\{P/\Box\}$. That is, the contextual application of M to P is the
  substitution of $P$ for $\Box$ in $M$.
\end{definition}

$\meaningof{-} : L \to \mathcal{P}(\pi)$

\begin{mathpar}
  \inferrule* [lab=collection] {} {\meaningof{true} = \pi, \and \meaningof{~E} = \pi \setminus \meaningof{E}, \and \meaningof{E_{1} \& E_{2}} = \meaningof{E_{1}} \cap \meaningof{E_{2}}}
\end{mathpar}

\begin{mathpar}
  \inferrule* [lab=structure] {} {\meaningof{0} = \{ P \in \pi | P \equiv 0 \}, \and \\ \meaningof{E_1 | E_2} = \{ P \in \pi | P \equiv P_{1} | P_{2}, P_{1} \in \meaningof{E_{1}}, P_{2} \in \meaningof{E_2}\} }
\end{mathpar}

\begin{mathpar}
 \inferrule* [lab=behavior] {} {\meaningof{\langle a?b \rangle E} = \{ P \in \pi | P \equiv Q | u?(y)P', \\ \and \\\\ \and \\ \;\;\; u \in \meaningof{a}, \forall z.P'\{z/y\} \in \meaningof{E\{z/b\}}\}, \and \\ \meaningof{a!E} = \{ P \in \pi | P \equiv Q | x!\langle P' \rangle, x \in \meaningof{a} P' \in \meaningof{E}\} }
\end{mathpar}

\begin{mathpar}
 \inferrule* [lab=nominal] {} {\meaningof{\quotep{E}} = \{ \quotep{P} \in \quotep{\pi} | P \in \meaningof{E} \}, \and \meaningof{\quotep{P}} = \{ \quotep{Q} \in \quotep{\pi} | P \equiv Q \} \and \\ \meaningof{@\quotep{E}} = \{ P \in \pi | P \equiv @x, x \in \meaningof{E} \}}
\end{mathpar}

\begin{eqnarray*}
  \\
  \meaningof{-} : TS \to ST
\end{eqnarray*}

\begin{eqnarray*}
  \\
  L : TS \to ST
\end{eqnarray*}

\begin{eqnarray*}
  \\
  P \models E \iff P \in \meaningof{E}
\end{eqnarray*}

\begin{eqnarray*}
  P \approx_{L} Q \iff \forall E \in L. P \models E \iff Q \models E
\end{eqnarray*}

\begin{eqnarray*}
  P \approx_{K} Q
\end{eqnarray*}

\begin{eqnarray*}
  P \approx Q
\end{eqnarray*}

$\approx_{K} = \approx = \approx_{L}$

\subsubsection{Contextual duality}

Note that contexts extend the quotation operation to a family of
operations from processes to names. Given a context, $M$, we can
define a \emph{nominal context}, $\quotep{M}$ by $\quotep{M}[P] :=
\quotep{M[P]}$. To foreshadow what is to come we observe that these
operations enjoy a duality with processes very much like the duality
between vectors and maps from vectors to scalars.

Further, because the calculus is essentially higher-order, we have a
correspondence between contexts and processes. More specifically,
given a name $x$ and a context $M$ we can construct $M^{*}_{x}$ such
that 

\begin{mathpar}
  M^{*}_{x} | \lift{x}{P} \red M[P]
\end{mathpar}

namely,

\begin{mathpar}
  M^{*}_{x} := x?(u).M[\dropn{u}]
\end{mathpar}

The dependence of $M^{*}_{x}$ on a name makes it an abstraction, 

\begin{mathpar}
  M^{*} := (x)x?(u).M[\dropn{u}]
\end{mathpar}

\subsection{Additional notation}

It will sometimes be convenient to denote the process a name
quotes. We already have the notation $x = \quotep{P}$, but it will be
convenient to introduce an alternate notation, $\procn{x}$, when we
want to emphasize the connection to the use of the name. Note that, by
virtue of name equivalence, $\quotep{\procn{x}} \nameeq x$; so, the
notation is consistent with previous definitions.

Further, because names have structure it is possible to effect
substitutions on the basis of that structure. This means we need to
upgrade our notation for substitutions, which we accomplish by
adapting comprehension notation. Thus,

\begin{mathpar}
  P\{ y / x : x \in S \}
\end{mathpar}

is interpreted to mean the process derived from P by replacing (in a
capture-avoiding manner) each occurrence of $x$ in $S$ by $y$. For example,

\begin{mathpar}
  P\{ \quotep{\procn{x}|\procn{x}} / x : x \in \freenames{P} \}
\end{mathpar}

will replace each (occurrence) of a free name $x$ in $P$ by
$\quotep{\procn{x}|\procn{x}}$.

Also, we will avail ourselves of the notation $x^{L}$ and $x^{R}$ to
denote injections of a name into disjoint copies of the name
space. There are numerous ways to accomplish this. One example can be
found in \cite{MeredithR05}. This notation overloads to vectors of
names: $\vec{x}^{\pi} := (x_{i}^{\pi} \; : \; 0 \leq i < |\vec{x}| )$ where $\pi \in \{L,R\}$.

We also use $P^{\Box} := P|\Box$.

In \cite{MeredithR05} an interpretation of the new operator is
given. It turns out that there are several possible interpretations
all enjoying the requisite algebraic properties of the operator (see
\cite{milner91polyadicpi}). We will therefore make liberal use of
$(\nu\; \vec{x})P$.

% subsection the_syntax_and_semantics_of_the_notation_system (end)   

\section{Interpretation of QM}
\subsection{Supporting definitions}
\subsubsection{Multiplication}
\begin{mathpar}
  \quotep{Q} \cdot \quotep{R} := \quotep{Q|R}
  \and \\
  \quotep{Q} \cdot P := P\{ \quotep{Q|R} / \quotep{R} : \quotep{R} \in \freenames{P} \}
\end{mathpar}

\paragraph{Discussion}
The first line needs little explanation. The second line says that
each free name of the process is replaced with the multiplication of
that name by the scalar. Multiplication of a scalar (name) by a state
(process) results in a process all the names of which have been `moved
over' by parallel composition with the process the scalar
quotes. There is a subtlety that the bound names have to be
manipulated so that multiplied names aren't accidentally
captured. There are many ways to achieve this.

\begin{remark}\label{rem:multiplication_identities}
  The reader is invited to verify that for all $x,y,z \in \QProc$ and $P \in \Proc$
  \begin{mathpar}
    x \cdot \quotep{0} \equiv x 
    \and
    x \cdot y \equiv y \cdot x
    \and
    x \cdot (y \cdot z) \equiv (x \cdot y) \cdot z
    \and \\
    \quotep{0} \cdot P \equiv P
    \and \\
    x \cdot (y \cdot P) \equiv (x \cdot y) \cdot P
    \and \\
    x \cdot (P|Q) \equiv (x \cdot P) | (x \cdot Q)
    \and \\    
  \end{mathpar}
\end{remark}

\subsubsection{Tensor product}

We define a tensor product on processes by structural induction.

\paragraph{Tensor of sums} First note that all summations, including
$\pzero$ and sequence, can be written $\Sigma_{i} x_{i}.A_{i} +
\Sigma_{j} x_{j}.C_{j}$, where we have grouped input-guarded processes
together and output-guarded processes together.

Thus, we can define the tensor product of two summations, $N_{1}\otimes N_{2}$, where

\begin{mathpar}
  N_{1} := \Sigma_{i} x_{i}.A_{i} + \Sigma_{j} x_{j}.C_{j}
  \and
  N_{2} := \Sigma_{i'} y_{i'}.B_{i'} + \Sigma_{j'} y_{j'}.D_{j'} 
\end{mathpar}

as follows.

\begin{mathpar}
  \Sigma_{i} x_{i}.A_{i} + \Sigma_{j} x_{j}.C_{j} \otimes \Sigma_{i'}
  y_{i'}.B_{i'} + \Sigma_{j'} y_{j'}.D_{j'} 
  \and \\
  := \; \Sigma_{i} \Sigma_{i'} \quotep{\stackrel{\vee}{x_{i}}| \stackrel{\vee}{y_{i'}}}.(A_{i}\otimes B_{i'}) \; | \; \Sigma_{i'} \Sigma_{i} \quotep{\stackrel{\vee}{y_{i'}}|\stackrel{\vee}{x_{i}}}.(B_{i'}\otimes A_{i})
  \and
  \;\; | \;\; \Sigma_{j} \Sigma_{j'} \quotep{\stackrel{\vee}{x_{j}}|\stackrel{\vee}{y_{j'}}}.(A_{j}\otimes B_{j'}) \; | \; \Sigma_{j'} \Sigma_{j} \quotep{\stackrel{\vee}{y_{j'}}|\stackrel{\vee}{x_{j}}}.(B_{j'}\otimes A_{j})
\end{mathpar}

\begin{remark}
  Do we need to $x^{L}$ and $y^{R}$ for this construction as well?
\end{remark}

\paragraph{Tensor of parallel compositions} Next, we distribute tensor
over par.

\begin{mathpar}
  P_{1}|P_{2} \otimes Q_{1}|Q_{2} := (P_{1} \otimes Q_{1}) | (P_{1}
  \otimes Q_{2}) | (P_{2} \otimes Q_{1}) | (P_{2} \otimes Q_{2})
\end{mathpar}

\paragraph{Tensor with dropped names} We treat tensor of a
process with a dropped name as parallel composition.

\begin{mathpar}
  P \otimes \dropn{x} := P | \dropn{x}
\end{mathpar}

\paragraph{Tensor of agents}

Finally, we need to define tensor on agents. Note that the definition
of tensor on normal products only tensors inputs with inputs and
outputs with outputs. Thus, we only have to define the operation on
``homogeneous'' pairings.

\begin{mathpar}
  (\vec{x})P \otimes (\vec{y})Q
  \and \\
  := (x_{0}^{L}|y_{0}^{R},\ldots,x_{0}^{L}|y_{n}^{R},\ldots,x_{m}^{L}|y_{0}^{R},\ldots,x_{m}^{L}|y_{n}^R)(P\{ \vec{x}^{L}/\vec{x}\} \otimes Q \{ \vec{y}^{R}/\vec{y}\})
  \and \\
  \clift{\vec{P}} \otimes \clift{\vec{Q}}
  \and \\
  := \clift{P_{0}\otimes Q_{0},\ldots,P_{0}\otimes Q_{n},\ldots,P_{m}\otimes Q_{0},\ldots,P_{m}\otimes Q_{n}}
\end{mathpar}

\begin{remark}
  Observe that arities of tensored abstractions matches arities of
  tensored concretions if the original arities matched. Note also that
  the length of the arities corresponds to the increase in dimension
  we see in ordinary vector space tensor product.
\end{remark}

\begin{remark}
  Operationally, this definition distributes the tensor down to
  components ``linked'' by summation. Tensor over summation is
  intriguing in that it mixes names. Moreover, as a consequence of the
  way it mixes names we have the identities for all $x \in \QProc$ and
  $P,Q \in \Proc$

  \begin{mathpar}
    (x \cdot P) \otimes Q \equiv x \cdot (P \otimes Q) \equiv P \otimes (x \cdot Q)
    \and
    P \otimes \pzero \equiv P
  \end{mathpar}

  that the reader is invited to verify.
\end{remark}

\subsubsection{Annihilation}
\begin{mathpar}
  P^{\perp} := \{ Q | \forall R. P|Q \red^{*} R \Rightarrow R \red^{*} \pzero \}
  \and \\
  P^{\underline{\perp}} := \Sigma_{Q \in P^{\perp}} \quotep{Q}?(y).(\dropn{y}|Q) | \Sigma_{Q \in P^{\perp}} \quotep{Q}\clift{\Box}
\end{mathpar}

\paragraph{Discussion} The reader will note that $P^{\perp}$ is a
\emph{set} of processes, while $P^{\underline{\perp}}$ is a
\emph{context}. We call the set $P^{\perp}$ the \emph{annihilators} of
$P$. The parallel composition of a process in the annihilators of $P$
with $P$ will result in a process, the state space of which has all
paths eventually leading to $\pzero$. Execution may endure loops; but
under reasonable conditions of fairness (naturally guaranteed under
most notions of bisimulation) such a composite process cannot get
stuck in such a loop and will, eventually pop out and terminate.

The context $P^{\underline{\perp}}$ is ready and willing to ``take the
$P$ out of'' the process to which it is applied. It will effectively
transmit the code of the process to which it is applied to one of the
annihilators and run the process against it.

\subsubsection{Evaluation}
We fix $M$ a domain of fully abstract interpretation with an equality
coincident with bisimulation. We take $\meaningof{\cdot} : \Proc \to
M$ to be the map interpreting processes and $\nmeaningof{\cdot} : \M
\to Proc$ to be the map running the other way. Then we define

\begin{mathpar}
  \int P := \nmeaningof{\meaningof{P}}
\end{mathpar}

\paragraph{Discussion}
There are many fully abstract interpretations of Milner's
$\pi$-calculus. Any of them can be used as a basis for interpreting
the reflective calculus here. Equipped with such a domain it is
largely a matter of grinding through to check that the Yoneda
construction for the normalization-by-evaluation program can be
extended to this setting.

\begin{remark}
  The reader is invited to verify that $\int (P^{\underline{\perp}}[P]) = 0$.
\end{remark}

\subsection{Quantum mechanics}

Table \ref{tbl:core_qm_op_defns} gives the core operational definitions

\begin{table}[htp]\label{tbl:core_qm_op_defns}
  \center{
    \fbox{
      \begin{tabular}{c|c}
        quantum mechanics & process calculus \\
        \hline
        scalar & $x := \quotep{P}$ \\
        state vector & $\state{P} := P$ \\
        dual & $\state{P}^{*} := \event{P^{\underline{\perp}}} := \quotep{P^{\underline{\perp}}}[-]$ \\
        matrix & $ \Sigma_{\alpha} \state{P_{\alpha}}x_{\alpha}\event{Q_{\alpha}}$ \\
        vector addition & $\state{P} + \state{Q} := \state{P | Q}$ \\
        tensor product & $\state{P} \otimes \state{Q} := \state{P \otimes Q}$ \\
        inner product & $\innerprod{P}{Q} := \quotep{\int P^{\underline{\perp}}[Q]}$ \\
      \end{tabular}
    }
  }
  \caption{QM - operational definitions}
\end{table}

where

\begin{mathpar}
  \prmatrix{P}{Q} := \fprmatrix{P}{\quotep{\pzero}}{Q}
  \and
  \fprmatrix{P}{x}{Q} := (\state{P},x,\event{Q})
  \and
  (\fprmatrix{P}{x}{Q})(\state{R}) := x \cdot \innerprod{Q}{R} \cdot \state{P}
  \and
  (\fprmatrix{P}{x}{Q})(\event{R}) := x \cdot \innerprod{R}{P} \cdot \event{Q}
\end{mathpar}

\paragraph{Discussion}
As promised: vectors (aka states) are represented as processes; duals
as contextual duals; inner product definition should be compared with
standard inner product definition for ....

\begin{remark}
  Assuming $\int (P^{\underline{\perp}}[P]) = 0$, the reader is
  invited to verify that $(\fprmatrix{P}{x}{P})(\state{P}) = x \cdot \state{P}$.
\end{remark}

\begin{remark}
  The reader is invited to verify that $\innerprod{P}{Q}$ could
  equally well have been written $\quotep{\int \stackrel{\vee}{x}}$
  where $x = \event{P^{\underline{\perp}}}(Q)$.

  One of the motivations for this remark is that there is another way
  to factor these operations. We could package up evaluation in the dual:

  \begin{mathpar}
    \state{P}^{*} := \event{\int P^{\underline{\perp}}} := \quotep{\int P^{\underline{\perp}}}[-]
  \end{mathpar}

  and then have inner product defined by
  
  \begin{mathpar}
    \innerprod{P}{Q} := \event{P}(Q)
  \end{mathpar}

  Hopefully, experience with the calculations will provide guidance on
  the best factoring.
\end{remark}

\begin{remark}
  Assuming $\int (P^{\underline{\perp}}[P]) = 0$, the reader is
  invited to verify that $\forall P,Q. (\prmatrix{0}{Q})(\state{0}) =
  \state{0}$ and dually $(\prmatrix{P}{0})(\event{0}) = \event{0}$.
\end{remark}

\begin{remark}
  i'm a little worried that i don't (yet) have proper support for
  complex conjugacy. But, the observation above may give us a
  clue. According to Abramsky, it must be the case that the scalars
  are iso to the homset of the identity for the tensor -- which the
  observation above characterizes. 

  For now, we will simply bookmark the notion with $\overline{x}$.
\end{remark}

\subsubsection{Adjointness}

We need to give a definition of $(\cdot)^{\dagger}$ for matrices. The
obvious candidate definition is
\begin{mathpar}
(\Sigma_{\alpha}\fprmatrix{P_{\alpha}}{x_{\alpha}}{Q_{\alpha}})^{\dagger}
= \Sigma_{\alpha}\fprmatrix{(Q_{\alpha}^{\underline{\perp}})^{*}}{\overline{x}_{\alpha}}{P_{\alpha}^{\underline{\perp}}} 
\end{mathpar}

But, $(Q_{\alpha}^{\underline{\perp}})^{*}$ requires a name along
which to communicate the process to achieve the context application.

\subsubsection{Basis for a basis}
If processes label states and ``addition'' of states (a.k.a. vector
addition) is interpreted as parallel composition, what corresponds to
notions of linear independence and basis? Here, we recall that Yoshida
has developed a set of \emph{combinators} for an asynchronous verison
of Milner's $\pi$-calculus. These are a finite set of processes such
any process can be expressed as parallel composition of these
combinators together with liberal uses of the new operator and
replication. We can simply give a translation of these into the
present calculus and have reasonable expectation that the property
carries over. That is, that the resultant set allows to express all
processes via parallel composition. Note, however, that there is no
new operator or replication in this calculus. As a result, we expect
that the corresponding set is actually infinite. That is, we expect
that the space is actually infinite dimensional.

\begin{remark}
  The attentive reader may be a bit concerned. Certainly, the
  collection $S$, $K$ and $I$ is a finite set of
  combinators. Shouldn't we expect to see a finite set of combinators
  for an effectively equivalent system? i am very sympathetic to this
  critique and feel it warrants full attention. On the other hand, i
  also have in mind the following analogy. The natural numbers, as a
  monoid under addition, has exactly $1$ generator, while the natural
  numbers, as a monoid under multiplication, has countably many
  generators (the primes). We observe that the application of the
  lambda calculus is much less resource sensitive than the parallel
  composition of the $\pi$-calculus. Could it be the case that we have
  an analogy of the form
  
  \begin{mathpar}
    m + n : MN :: m*n : M|N
  \end{mathpar}

  giving a similar blow up in the set of ``primes''?  This is such a
  wonderful thought that, even if it's not true, i think it's worth
  writing down.
\end{remark}
 

\documentclass[12pt]{llncs}
%\documentclass{jktr}

\usepackage[pdftex]{hyperref}                   
\usepackage {listings}
\usepackage {mathpartir}
\usepackage{bcprules}
%\usepackage{listings}
                       
\usepackage{graphicx} 
%\usepackage[margins=2.5cm,nohead,nofoot]{geometry}
%\usepackage{geometry}
\usepackage{amsfonts}
\usepackage{amstext}
\usepackage{latexsym}
\usepackage{amssymb}
\usepackage{color}


%\include{myPreamble}
\include{qm2pi.local} 

%\ifpdf
%\usepackage[pdftex]{graphicx}
%\else
%\usepackage{graphicx}
%\fi

 % \ifpdf
%  \usepackage{pdfsync}
%  \if


%\title{Brief Article}
%\author{David F. Snyder}
%\author{L.G. Meredith}

%\address{Dept. of Math., Texas State University--San Marcos, San Marcos, TX 78666}
       
\pagestyle{empty}


\begin{document}

\lstset{language=[Objective]Caml,frame=shadowbox}

\input{qm2pi.front}

% section front matter (end)

\input{qm2pi.intro} 
 
% section introduction (end)

% \input{qm2pi.knotations} 

% section notation (end)

\input{qm2pi.process.calculi} 

% section concurrent_process_calculi_and_spatial_logics_ (end)
    
%\input{qm2pi.knots2pi} 

%\input{qm2pi.trefoil} 

%\input{qm2pi.mainthm} 

% subsection basic_interpretation (end)

%\input{qm2pi.rho.presentation} 
\subsection{The syntax and semantics of the notation system}\label{sub:the_syntax_and_semantics_of_the_notation_system} % (fold)

We now summarize a technical presentation of the calculus that
embodies our theory of dynamics. The typical presentation of such a
calculus follows the style of giving generators and relations on
them. The grammar, below, describing term constructors, freely
generates the set of processes, $\Proc$. This set is then quotiented
by a relation known as structural congruence and it is over this set
that the notion of dynamics is expressed. This presentation is
essentially that of \cite{MeredithR05} with the addition of
polyadicity and summation. For readability we have relegated some of
the technical subtleties to an appendix.

\subsubsection{Process grammar}\label{subsub:process_grammar}

\begin{mathpar}
  \inferrule* [lab=synchronization] {} {{M} \bc \pzero \;|\; x?F \;|\; x!C }
  \and
  \inferrule* [lab=abstraction] {} {{F} \bc (x)P}
  \and
  \inferrule* [lab=concretion] {} {{C} \bc \langle Q \rangle}
  \and
  \inferrule* [lab=process] {} {{P,Q} \bc M \;| \;P|Q \;|\; @{x}}
  \and
  \inferrule* [lab=name] {} {{x} \bc \quotep{P}}
\end{mathpar} 

Note that $\vec{x}$ (resp. $\vec{P}$) denotes a vector of names
(resp. processes) of length $|\vec{x}|$ (resp. $|\vec{P}|$). We adopt
the following useful abbreviations.

\begin{mathpar}
   x?(\vec{y}).P := x.(\vec{y})P \and  x\clift{\vec{P}} := x.\clift{\vec{P}}
   \and x!(y) := \lift{x}{\dropn{y}}
   \and \Pi_{i=0}^{n-1}P_i := P_0 | \ldots | P_{n-1}
\end{mathpar}

\subsubsection{Structural congruence}

\paragraph{Free and bound names and alpha-equivalence.} At the
core of structural equivalence is alpha-equivalence which identifies
process that are the same up to a change of variable. Formally, we
recognize the distinction between free and bound names. The free names
of a process, $\freenames{P}$, may be calculated recursively as
follows:

\begin{mathpar}
\freenames{\pzero} := \emptyset
  \and \\
  \freenames{x?(y).P} := \{ x \} \cup (\freenames{P} \setminus \{ y \})
  \and 
  \freenames{x!\langle P \rangle} := \{ x \} \cup \{ P \} 
  \and \\
  \freenames{P|Q} := \freenames{P} \cup \freenames{Q}
  \and \\
  \freenames{@{x}} := \{ x \}
\end{mathpar}

$\pi$
$\quotep{\pi}$

$\freenames{-} : \pi \to \mathcal{P}(\quotep{\pi})$

\begin{eqnarray*}
  \freenames{\pzero} & := & \emptyset \\
  \freenames{x?(y).P} & := & \{ x \} \cup (\freenames{P} \setminus \{ y \}) \\
  \freenames{x!\langle P \rangle} & := & \{ x \} \cup \{ P \} \\
  \freenames{P|Q} & := & \freenames{P} \cup \freenames{Q} \\
  \freenames{\dropn{x}} & := & \{ x \}
\end{eqnarray*}

The bound names of a process, $\boundnames{P}$, are those names occurring in $P$
that are not free. For example, in $x?(y).0$, the name $x$ is free, while $y$ is bound.

\begin{mathpar}
  \inferrule* [lab=monoidal-laws] {} { P|Q \equiv Q|P \and P|0 \equiv P \and P|(Q|R) \equiv (P|Q)|R }
\end{mathpar}

\begin{mathpar}
  \inferrule* [lab=alpha-equivalence] {} { (x)P \equiv (y)P\{y/x\} \and y \not\in \freenames{P} }
\end{mathpar}

\begin{definition}
Then two processes, $P,Q$, are alpha-equivalent if $P = Q\{\vec{y}/\vec{x}\}$ for
some $\vec{x} \in \boundnames{Q},\vec{y} \in \boundnames{P}$, where $Q\{\vec{y}/\vec{x}\}$
denotes the capture-avoiding substitution of $\vec{y}$ for $\vec{x}$ in $Q$.
\end{definition}

\begin{definition}
  The {\em structural congruence} \cite{SangiorgiWalker} , $\equiv$,
  between processes is the least congruence containing
  alpha-equivalence, satisfying the abelian monoid laws
  (associativity, commutativity and $\pzero$ as identity) for parallel
  composition $|$ and for summation $+$.
\end{definition}

\subsection{Name equivalence}

We take name equivalence, written $\nameeq$, to be the smallest
equivalence relation generated by the following rules.

\begin{mathpar}
\inferrule*[lab=Quote-drop]
{ }
{ \quotep{@{x}} \nameeq x }

\inferrule*[lab=Struct-equiv]
{ P \scong Q }
{ \quotep{P} \nameeq \quotep{Q} }
\end{mathpar}

The astute reader will have noticed that the mutual recursion of names
and processes imposes a mutual recursion on alpha-equivalence and
structural equivalence via name-equivalence. Fortunately, all of this
works out pleasantly and we may calculate in the natural way, free of
concern. The reader interested in the details is referred to the
appendix \ref{appendix:rho_details}.

\subsection{Substitution}

We use $\Proc$ for the set of processes, $\QProc$ for the set of
names, and $\id{\{}\vec{y} / \vec{x} \id{\}}$ to denote partial maps,
$s : \QProc \rightarrow \QProc$. A map, $s$ lifts, uniquely, to a map
on process terms, $\widehat{s} : \Proc \rightarrow \Proc$ by the
following equations.

\begin{mathpar}
  (0) \psubstp{Q}{P} := 0 \\
  (R \juxtap S) \psubstp{Q}{P}
  :=    
  (R)\psubstp{Q}{P} \juxtap (S) \psubstp{Q}{P} \\
  (x?(y).R) \psubstp{Q}{P}    
  :=    
  (x)\substp{Q}{P} (z)\concat( (R \psubstn{z}{y}) \psubstp{Q}{P} ) \\
  (\lift{x}{R}) \psubstp{Q}{P}  
  :=
  \lift{(x)\substp{Q}{P}}{ R \psubstp{Q}{P} } \\
%   (\dropn{x})  \psubstp{Q}{P}       
%   := 
%   \left\{ 
%     \begin{array}{ccc} 
%       \dropn{\quotep{Q}} & & x \nameeq \quotep{P} \\
%       \dropn{x} & & otherwise \\
%     \end{array}
%   \right. 
  (\dropn{x})  \psubstp{Q}{P}       
  := 
  \left\{ 
    \begin{array}{ccc} 
      Q & & x \nameeq \quotep{P} \\
      \dropn{x} & & otherwise \\
    \end{array}
  \right.
\end{mathpar}
 

where

\begin{eqnarray}
  (x)\id{\{} \lpquote Q \rpquote / \lpquote P \rpquote \id{\}}            = 
  \left\{ 
    \begin{array}{ccc}
      \lpquote Q \rpquote & & x \nameeq \lpquote P \rpquote \\
      x & & otherwise \\
    \end{array}
  \right. \nonumber
\end{eqnarray}

and $z$ is chosen distinct from $\quotep{P}$, $\quotep{Q}$, the free
names in $Q$, and all the names in $R$. Our $\alpha$-equivalence will
be built in the standard way from this substitution.

\begin{remark}\label{rem:no_self_referential_names}
  One consequence of these definitions is that $\forall P. \quotep{P}
  \not\in \freenames{P}$.
\end{remark}

\subsection{ Dynamic quote: an example }

Anticipating something of what's to come, consider applying the
substitution, $\widehat{\id{\{}u / z \id{\}}}$, to the following pair
of processes, $\lift{w}{y!(z)}$ and $w[ \lpquote y!(z) \rpquote ]$.

\begin{eqnarray}
	\lift{w}{y!(z)}\widehat{\id{\{}u / z \id{\}}}
		& = &
		\lift{w}{y!(u)} \nonumber\\
	w[ \lpquote y!(z) \rpquote ] \widehat{ \id{\{}u / z \id{\}} }
		& = &
		w[ \lpquote y!(z) \rpquote ] \nonumber
\end{eqnarray}

Because the body of the process between quotes is impervious to
substitution, we get radically different answers. In fact, by
examining the first process in an input context,
e.g. $x?(z).\lift{w}{y!(z)}$, we see that the process under the lift
operator may be shaped by prefixed inputs binding a name inside it. In
this sense, the lift operator will be seen as a way to dynamically
construct processes before reifying them as names.

Finally equipped with these standard features we can present the
dynamics of the calculus.

\subsubsection{Operational semantics} 

Finally, we introduce the computational dynamics. What marks these
algebras as distinct from other more traditionally studied algebraic
structures, e.g. vector spaces or polynomial rings, is the manner in
which dynamics is captured. In traditional structures, dynamics is typically
expressed through morphisms between such structures, as in linear maps
between vector spaces or morphisms between rings. In algebras
associated with the semantics of computation, the dynamics is
expressed as part of the algebraic structure itself, through a
reduction reduction relation typically denoted by $\red$. Below, we
give a recursive presentation of this relation for the calculus used
in the encoding.

$\red \subseteq \pi \times \pi$
$\red : \pi \to \mathcal{P}(\pi)$

\begin{mathpar}
  \inferrule* [lab=Comm] { \textsf{match}( x_{src}, x_{trgt} ) } { x_{trgt}?(y)P \; | \; x_{src}!\langle {Q} \rangle \red P\{\quotep{Q}/y}\} }
  \and \\
  \inferrule* [lab=Par] {{P} \red {P}'} {{{P} | {Q}} \red {{P}' | {Q}}}
  \and
  \inferrule* [lab=Equiv]{{{P} \scong {P}'} \andalso {{P}' \red {Q}'} \andalso {{Q}' \scong {Q}}}{{P} \red {Q}}
\end{mathpar}

\begin{eqnarray*}
  match_{\equiv} (\quotep{P},\quotep{Q}) & := & P \equiv Q \\
  match_{\dagger}(\quotep{P},\quotep{Q}) & := & \forall R. P|Q \red^{*} R => R \red^{*} 0 \\
  match_{K}(\quotep{P},\quotep{Q}) & := & K \mbox{ for some context } K
\end{eqnarray*}

$u?(x)P | u!\langle Q \rangle \red P\{\quotep{Q}/x\}$

%We write $\wred$ for $\red^*$, and $P\red$ if $\exists Q $ such that $ P \red Q$.
We write $P\red$ if $\exists Q $ such that $ P \red Q$ and $P\not\red$, otherwise.

\section{Replication}

As mentioned before, it is known that replication (and hence
recursion) can be implemented in a higher-order process algebra
\cite{SangiorgiWalker}. As our first example of calculation with the
machinery thus far presented we give the construction explicitly in
the {\rhoc}.

\begin{eqnarray}
	D_{x} & := & \prefix{x}{y}{(\binpar{\outputp{x}{y}}{@{y}})} \nonumber\\
	\bangp_{x}{P} & := & \binpar{{x}!\langle{\binpar{D_{x}}{P}}\rangle}{D_{x}} \nonumber
\end{eqnarray}

\begin{eqnarray}
	\bangp_{x}{P} & & \nonumber\\
	=
	& {x}!\langle{(\prefix{x}{y}{(\outputp{x}{y} | @{y})) | P}}\rangle 
	      | \prefix{x}{y}{(\outputp{x}{y} | @{y})} & \nonumber\\
	\red
	& (\outputp{x}{y} | @{y})\substn{\quotep{(\prefix{x}{y}{(@{y} | \outputp{x}{y})) | P}}}{y} & \nonumber\\
	=
	& \outputp{x}{\quotep{(\prefix{x}{y}{(\outputp{x}{y} | @{y})) | P}}}
	  | {(\prefix{x}{y}{(\outputp{x}{y} | @{y})) | P}} & \nonumber\\
	\red
	& \ldots & \nonumber\\
	\red^*
	& P | P | \ldots & \nonumber
\end{eqnarray}

Of course, this encoding, as an implementation, runs away, unfolding
$\bangp{P}$ eagerly. A lazier and more implementable replication
operator, restricted to input-guarded processes, may be obtained as follows.

\begin{eqnarray}
\bangp{\prefix{u}{v}{P}} 
	:= 
	\binpar{\lift{x}{\prefix{u}{v}{(\binpar{D(x)}{P})}}}{D(x)} \nonumber
\end{eqnarray}

\begin{remark}
  Note that the lazier definition still does not deal with summation
  or mixed summation (i.e. sums over input and output). The reader is
  invited to construct definitions of replication that deal with these
  features. 

  Further, the definitions are parameterized in a name, $x$. Can you,
  gentle reader, make a definition that eliminates this parameter and
  guarantees no accidental interaction between the replication
  machinery and the process being replicated -- i.e. no accidental
  sharing of names used by the process to get its work done and the
  name(s) used by the replication to effect copying. This latter
  revision of the definition of replication is crucial to obtaining
  the expected identity $!!P \sim !P$.
\end{remark}

\begin{remark}\label{rem:paradoxical_combinator}
  The reader familiar with the lambda calculus will have noticed the
  similarity between $D$ and the paradoxical combinator.

  [Ed. note: the existence of this seems to suggest we have to be more
  restrictive on the set of processes and names we admit if we are to
  support no-cloning.]
\end{remark}

\subsubsection{Bisimulation}

The computational dynamics gives rise to another kind of equivalence,
the equivalence of computational behavior. As previously mentioned
this is typically captured \emph{via} some form of bisimulation.

% The notion we use in this paper is weak barbed bisimulation
% \cite{milner91polyadicpi}.

The notion we use in this paper is derived from weak barbed
bisimulation \cite{milner91polyadicpi}. 

\begin{definition}
An \emph{observation relation}, $\downarrow_{\mathcal N}$, over a set
of names, $\mathcal N$, is the smallest relation satisfying the rules
below.

\infrule[Out-barb]{y \in {\mathcal N}, \; x \nameeq y}
		  {\outputp{x}{v} \downarrow_{\mathcal N} x}
\infrule[Par-barb]{\mbox{$P\downarrow_{\mathcal N} x$ or $Q\downarrow_{\mathcal N} x$}}
		  {\binpar{P}{Q} \downarrow_{\mathcal N} x}

We write $P \Downarrow_{\mathcal N} x$ if there is $Q$ such that 
$P \wred Q$ and $Q \downarrow_{\mathcal N} x$.
\end{definition}

\begin{definition}
%\label{def.bbisim}
An  ${\mathcal N}$-\emph{barbed bisimulation} over a set of names, ${\mathcal N}$, is a symmetric binary relation 
${\mathcal S}_{\mathcal N}$ between agents such that $P\rel{S}_{\mathcal N}Q$ implies:
\begin{enumerate}
\item If $P \red P'$ then $Q \wred Q'$ and $P'\rel{S}_{\mathcal N} Q'$.
\item If $P\downarrow_{\mathcal N} x$, then $Q\Downarrow_{\mathcal N} x$.
\end{enumerate}
$P$ is ${\mathcal N}$-barbed bisimilar to $Q$, written
$P \wbbisim_{\mathcal N} Q$, if $P \rel{S}_{\mathcal N} Q$ for some ${\mathcal N}$-barbed bisimulation ${\mathcal S}_{\mathcal N}$.
\end{definition}

$\mathcal{R} \subseteq \pi \times \pi$

$P \mathcal{R} Q => \forall P'. P \red P' \Rightarrow \exists Q'. Q \red Q', P' \mathcal{R} Q'$

$P \vdash x \Rightarrow Q \vdash x$

\begin{mathpar}
  \inferrule*[lab=Out-barb]{x \nameeq y}{{y}!\langle{Q}\rangle \vdash x}
  \and
  \inferrule*[lab=Par-barb]{\mbox{$P\vdash x$ or $Q\vdash x$}}{\binpar{P}{Q} \vdash x}
\end{mathpar}

\subsubsection{Contexts}

One of the principle advantages of computational calculi like the
$\pi$-calculus is a well-defined notion of context,
contextual-equivalence and a correlation between
contextual-equivalence and notions of bisimulation. The notion of
context allows the decomposition of a process into (sub-)process and
its syntactic environment, its context. Thus, a context may be
thought of as a process with a ``hole'' (written $\Box$) in it. The
application of a context $M$ to a process $P$, written $M[P]$, is
tantamount to filling the hole in $M$ with $P$. In this paper we do
not need the full weight of this theory, but do make use of the notion
of context in the proof the main theorem. 

\begin{mathpar}
  \inferrule* [lab=summation] {} {{M_{M},M_{N}} \bc \Box \;|\; x.M_{A} \;|\; M_{M}+M_{N}}
  \and
  \inferrule* [lab=agent] {} {{M_{A}} \bc (\vec{x})M_{P} \;| \; \clift{P_0,\ldots,M_{P},\ldots,P_N}}
  \and \\
  \inferrule* [lab=process] {} {{M_{P}} \bc M_{N} \;| \;P|M_{P} }
\end{mathpar} 

\begin{mathpar}
  \inferrule* [lab=sychronization] {} {M_{N} \bc \Box \;|\; x?M_{F} \;|\; x!M_{C}}
  \and
  \inferrule* [lab=abstraction] {} {{M_{F}} \bc (x)M_{P} }
  \and
  \inferrule* [lab=concretion] {} {{M_{C}} \bc \langle M_{P} \rangle }
  \and \\
  \inferrule* [lab=process] {} {{M_{P}} \bc M_{N} \;| \;P|M_{P} }
\end{mathpar}

\begin{definition}[contextual application] Given a context $M$, and
  process $P$, we define the \emph{contextual application}, $M[P] :=
  M\{P/\Box\}$. That is, the contextual application of M to P is the
  substitution of $P$ for $\Box$ in $M$.
\end{definition}

$\meaningof{-} : L \to \mathcal{P}(\pi)$

\begin{mathpar}
  \inferrule* [lab=collection] {} {\meaningof{true} = \pi, \and \meaningof{~E} = \pi \setminus \meaningof{E}, \and \meaningof{E_{1} \& E_{2}} = \meaningof{E_{1}} \cap \meaningof{E_{2}}}
\end{mathpar}

\begin{mathpar}
  \inferrule* [lab=structure] {} {\meaningof{0} = \{ P \in \pi | P \equiv 0 \}, \and \\ \meaningof{E_1 | E_2} = \{ P \in \pi | P \equiv P_{1} | P_{2}, P_{1} \in \meaningof{E_{1}}, P_{2} \in \meaningof{E_2}\} }
\end{mathpar}

\begin{mathpar}
 \inferrule* [lab=behavior] {} {\meaningof{\langle a?b \rangle E} = \{ P \in \pi | P \equiv Q | u?(y)P', \\ \and \\\\ \and \\ \;\;\; u \in \meaningof{a}, \forall z.P'\{z/y\} \in \meaningof{E\{z/b\}}\}, \and \\ \meaningof{a!E} = \{ P \in \pi | P \equiv Q | x!\langle P' \rangle, x \in \meaningof{a} P' \in \meaningof{E}\} }
\end{mathpar}

\begin{mathpar}
 \inferrule* [lab=nominal] {} {\meaningof{\quotep{E}} = \{ \quotep{P} \in \quotep{\pi} | P \in \meaningof{E} \}, \and \meaningof{\quotep{P}} = \{ \quotep{Q} \in \quotep{\pi} | P \equiv Q \} \and \\ \meaningof{@\quotep{E}} = \{ P \in \pi | P \equiv @x, x \in \meaningof{E} \}}
\end{mathpar}

\begin{eqnarray*}
  \\
  \meaningof{-} : TS \to ST
\end{eqnarray*}

\begin{eqnarray*}
  \\
  L : TS \to ST
\end{eqnarray*}

\begin{eqnarray*}
  \\
  P \models E \iff P \in \meaningof{E}
\end{eqnarray*}

\begin{eqnarray*}
  P \approx_{L} Q \iff \forall E \in L. P \models E \iff Q \models E
\end{eqnarray*}

\begin{eqnarray*}
  P \approx_{K} Q
\end{eqnarray*}

\begin{eqnarray*}
  P \approx Q
\end{eqnarray*}

$\approx_{K} = \approx = \approx_{L}$

\subsubsection{Contextual duality}

Note that contexts extend the quotation operation to a family of
operations from processes to names. Given a context, $M$, we can
define a \emph{nominal context}, $\quotep{M}$ by $\quotep{M}[P] :=
\quotep{M[P]}$. To foreshadow what is to come we observe that these
operations enjoy a duality with processes very much like the duality
between vectors and maps from vectors to scalars.

Further, because the calculus is essentially higher-order, we have a
correspondence between contexts and processes. More specifically,
given a name $x$ and a context $M$ we can construct $M^{*}_{x}$ such
that 

\begin{mathpar}
  M^{*}_{x} | \lift{x}{P} \red M[P]
\end{mathpar}

namely,

\begin{mathpar}
  M^{*}_{x} := x?(u).M[\dropn{u}]
\end{mathpar}

The dependence of $M^{*}_{x}$ on a name makes it an abstraction, 

\begin{mathpar}
  M^{*} := (x)x?(u).M[\dropn{u}]
\end{mathpar}

\subsection{Additional notation}

It will sometimes be convenient to denote the process a name
quotes. We already have the notation $x = \quotep{P}$, but it will be
convenient to introduce an alternate notation, $\procn{x}$, when we
want to emphasize the connection to the use of the name. Note that, by
virtue of name equivalence, $\quotep{\procn{x}} \nameeq x$; so, the
notation is consistent with previous definitions.

Further, because names have structure it is possible to effect
substitutions on the basis of that structure. This means we need to
upgrade our notation for substitutions, which we accomplish by
adapting comprehension notation. Thus,

\begin{mathpar}
  P\{ y / x : x \in S \}
\end{mathpar}

is interpreted to mean the process derived from P by replacing (in a
capture-avoiding manner) each occurrence of $x$ in $S$ by $y$. For example,

\begin{mathpar}
  P\{ \quotep{\procn{x}|\procn{x}} / x : x \in \freenames{P} \}
\end{mathpar}

will replace each (occurrence) of a free name $x$ in $P$ by
$\quotep{\procn{x}|\procn{x}}$.

Also, we will avail ourselves of the notation $x^{L}$ and $x^{R}$ to
denote injections of a name into disjoint copies of the name
space. There are numerous ways to accomplish this. One example can be
found in \cite{MeredithR05}. This notation overloads to vectors of
names: $\vec{x}^{\pi} := (x_{i}^{\pi} \; : \; 0 \leq i < |\vec{x}| )$ where $\pi \in \{L,R\}$.

We also use $P^{\Box} := P|\Box$.

In \cite{MeredithR05} an interpretation of the new operator is
given. It turns out that there are several possible interpretations
all enjoying the requisite algebraic properties of the operator (see
\cite{milner91polyadicpi}). We will therefore make liberal use of
$(\nu\; \vec{x})P$.

% subsection the_syntax_and_semantics_of_the_notation_system (end)   

\input{qm2pi.qmops} 

\input{qm2pi.sterngerlach} 

\input{qm2pi.metric} 

% section concurrent_process_calculi (end)

%\input{qm2pi.proofsketch}

% section proof sketch (end)

%\input{qm2pi.slviaknots} 

% section spatial logic via knots (end)

\input{qm2pi.conclusion}

% section conclusion (end)

%\input{qm2pi.dtcodes} 

% section wiring algorithm (end)

\input{qm2pi.ack} 

% section acknowledgments (end)

\newpage


\bibliographystyle{plain}   
\bibliography{../../biblios/main.bib}

\input{qm2pi.rhodetails}

\end{document}

 

\documentclass[12pt]{llncs}
%\documentclass{jktr}

\usepackage[pdftex]{hyperref}                   
\usepackage {listings}
\usepackage {mathpartir}
\usepackage{bcprules}
%\usepackage{listings}
                       
\usepackage{graphicx} 
%\usepackage[margins=2.5cm,nohead,nofoot]{geometry}
%\usepackage{geometry}
\usepackage{amsfonts}
\usepackage{amstext}
\usepackage{latexsym}
\usepackage{amssymb}
\usepackage{color}


%\include{myPreamble}
\include{qm2pi.local} 

%\ifpdf
%\usepackage[pdftex]{graphicx}
%\else
%\usepackage{graphicx}
%\fi

 % \ifpdf
%  \usepackage{pdfsync}
%  \if


%\title{Brief Article}
%\author{David F. Snyder}
%\author{L.G. Meredith}

%\address{Dept. of Math., Texas State University--San Marcos, San Marcos, TX 78666}
       
\pagestyle{empty}


\begin{document}

\lstset{language=[Objective]Caml,frame=shadowbox}

\input{qm2pi.front}

% section front matter (end)

\input{qm2pi.intro} 
 
% section introduction (end)

% \input{qm2pi.knotations} 

% section notation (end)

\input{qm2pi.process.calculi} 

% section concurrent_process_calculi_and_spatial_logics_ (end)
    
%\input{qm2pi.knots2pi} 

%\input{qm2pi.trefoil} 

%\input{qm2pi.mainthm} 

% subsection basic_interpretation (end)

%\input{qm2pi.rho.presentation} 
\subsection{The syntax and semantics of the notation system}\label{sub:the_syntax_and_semantics_of_the_notation_system} % (fold)

We now summarize a technical presentation of the calculus that
embodies our theory of dynamics. The typical presentation of such a
calculus follows the style of giving generators and relations on
them. The grammar, below, describing term constructors, freely
generates the set of processes, $\Proc$. This set is then quotiented
by a relation known as structural congruence and it is over this set
that the notion of dynamics is expressed. This presentation is
essentially that of \cite{MeredithR05} with the addition of
polyadicity and summation. For readability we have relegated some of
the technical subtleties to an appendix.

\subsubsection{Process grammar}\label{subsub:process_grammar}

\begin{mathpar}
  \inferrule* [lab=synchronization] {} {{M} \bc \pzero \;|\; x?F \;|\; x!C }
  \and
  \inferrule* [lab=abstraction] {} {{F} \bc (x)P}
  \and
  \inferrule* [lab=concretion] {} {{C} \bc \langle Q \rangle}
  \and
  \inferrule* [lab=process] {} {{P,Q} \bc M \;| \;P|Q \;|\; @{x}}
  \and
  \inferrule* [lab=name] {} {{x} \bc \quotep{P}}
\end{mathpar} 

Note that $\vec{x}$ (resp. $\vec{P}$) denotes a vector of names
(resp. processes) of length $|\vec{x}|$ (resp. $|\vec{P}|$). We adopt
the following useful abbreviations.

\begin{mathpar}
   x?(\vec{y}).P := x.(\vec{y})P \and  x\clift{\vec{P}} := x.\clift{\vec{P}}
   \and x!(y) := \lift{x}{\dropn{y}}
   \and \Pi_{i=0}^{n-1}P_i := P_0 | \ldots | P_{n-1}
\end{mathpar}

\subsubsection{Structural congruence}

\paragraph{Free and bound names and alpha-equivalence.} At the
core of structural equivalence is alpha-equivalence which identifies
process that are the same up to a change of variable. Formally, we
recognize the distinction between free and bound names. The free names
of a process, $\freenames{P}$, may be calculated recursively as
follows:

\begin{mathpar}
\freenames{\pzero} := \emptyset
  \and \\
  \freenames{x?(y).P} := \{ x \} \cup (\freenames{P} \setminus \{ y \})
  \and 
  \freenames{x!\langle P \rangle} := \{ x \} \cup \{ P \} 
  \and \\
  \freenames{P|Q} := \freenames{P} \cup \freenames{Q}
  \and \\
  \freenames{@{x}} := \{ x \}
\end{mathpar}

$\pi$
$\quotep{\pi}$

$\freenames{-} : \pi \to \mathcal{P}(\quotep{\pi})$

\begin{eqnarray*}
  \freenames{\pzero} & := & \emptyset \\
  \freenames{x?(y).P} & := & \{ x \} \cup (\freenames{P} \setminus \{ y \}) \\
  \freenames{x!\langle P \rangle} & := & \{ x \} \cup \{ P \} \\
  \freenames{P|Q} & := & \freenames{P} \cup \freenames{Q} \\
  \freenames{\dropn{x}} & := & \{ x \}
\end{eqnarray*}

The bound names of a process, $\boundnames{P}$, are those names occurring in $P$
that are not free. For example, in $x?(y).0$, the name $x$ is free, while $y$ is bound.

\begin{mathpar}
  \inferrule* [lab=monoidal-laws] {} { P|Q \equiv Q|P \and P|0 \equiv P \and P|(Q|R) \equiv (P|Q)|R }
\end{mathpar}

\begin{mathpar}
  \inferrule* [lab=alpha-equivalence] {} { (x)P \equiv (y)P\{y/x\} \and y \not\in \freenames{P} }
\end{mathpar}

\begin{definition}
Then two processes, $P,Q$, are alpha-equivalent if $P = Q\{\vec{y}/\vec{x}\}$ for
some $\vec{x} \in \boundnames{Q},\vec{y} \in \boundnames{P}$, where $Q\{\vec{y}/\vec{x}\}$
denotes the capture-avoiding substitution of $\vec{y}$ for $\vec{x}$ in $Q$.
\end{definition}

\begin{definition}
  The {\em structural congruence} \cite{SangiorgiWalker} , $\equiv$,
  between processes is the least congruence containing
  alpha-equivalence, satisfying the abelian monoid laws
  (associativity, commutativity and $\pzero$ as identity) for parallel
  composition $|$ and for summation $+$.
\end{definition}

\subsection{Name equivalence}

We take name equivalence, written $\nameeq$, to be the smallest
equivalence relation generated by the following rules.

\begin{mathpar}
\inferrule*[lab=Quote-drop]
{ }
{ \quotep{@{x}} \nameeq x }

\inferrule*[lab=Struct-equiv]
{ P \scong Q }
{ \quotep{P} \nameeq \quotep{Q} }
\end{mathpar}

The astute reader will have noticed that the mutual recursion of names
and processes imposes a mutual recursion on alpha-equivalence and
structural equivalence via name-equivalence. Fortunately, all of this
works out pleasantly and we may calculate in the natural way, free of
concern. The reader interested in the details is referred to the
appendix \ref{appendix:rho_details}.

\subsection{Substitution}

We use $\Proc$ for the set of processes, $\QProc$ for the set of
names, and $\id{\{}\vec{y} / \vec{x} \id{\}}$ to denote partial maps,
$s : \QProc \rightarrow \QProc$. A map, $s$ lifts, uniquely, to a map
on process terms, $\widehat{s} : \Proc \rightarrow \Proc$ by the
following equations.

\begin{mathpar}
  (0) \psubstp{Q}{P} := 0 \\
  (R \juxtap S) \psubstp{Q}{P}
  :=    
  (R)\psubstp{Q}{P} \juxtap (S) \psubstp{Q}{P} \\
  (x?(y).R) \psubstp{Q}{P}    
  :=    
  (x)\substp{Q}{P} (z)\concat( (R \psubstn{z}{y}) \psubstp{Q}{P} ) \\
  (\lift{x}{R}) \psubstp{Q}{P}  
  :=
  \lift{(x)\substp{Q}{P}}{ R \psubstp{Q}{P} } \\
%   (\dropn{x})  \psubstp{Q}{P}       
%   := 
%   \left\{ 
%     \begin{array}{ccc} 
%       \dropn{\quotep{Q}} & & x \nameeq \quotep{P} \\
%       \dropn{x} & & otherwise \\
%     \end{array}
%   \right. 
  (\dropn{x})  \psubstp{Q}{P}       
  := 
  \left\{ 
    \begin{array}{ccc} 
      Q & & x \nameeq \quotep{P} \\
      \dropn{x} & & otherwise \\
    \end{array}
  \right.
\end{mathpar}
 

where

\begin{eqnarray}
  (x)\id{\{} \lpquote Q \rpquote / \lpquote P \rpquote \id{\}}            = 
  \left\{ 
    \begin{array}{ccc}
      \lpquote Q \rpquote & & x \nameeq \lpquote P \rpquote \\
      x & & otherwise \\
    \end{array}
  \right. \nonumber
\end{eqnarray}

and $z$ is chosen distinct from $\quotep{P}$, $\quotep{Q}$, the free
names in $Q$, and all the names in $R$. Our $\alpha$-equivalence will
be built in the standard way from this substitution.

\begin{remark}\label{rem:no_self_referential_names}
  One consequence of these definitions is that $\forall P. \quotep{P}
  \not\in \freenames{P}$.
\end{remark}

\subsection{ Dynamic quote: an example }

Anticipating something of what's to come, consider applying the
substitution, $\widehat{\id{\{}u / z \id{\}}}$, to the following pair
of processes, $\lift{w}{y!(z)}$ and $w[ \lpquote y!(z) \rpquote ]$.

\begin{eqnarray}
	\lift{w}{y!(z)}\widehat{\id{\{}u / z \id{\}}}
		& = &
		\lift{w}{y!(u)} \nonumber\\
	w[ \lpquote y!(z) \rpquote ] \widehat{ \id{\{}u / z \id{\}} }
		& = &
		w[ \lpquote y!(z) \rpquote ] \nonumber
\end{eqnarray}

Because the body of the process between quotes is impervious to
substitution, we get radically different answers. In fact, by
examining the first process in an input context,
e.g. $x?(z).\lift{w}{y!(z)}$, we see that the process under the lift
operator may be shaped by prefixed inputs binding a name inside it. In
this sense, the lift operator will be seen as a way to dynamically
construct processes before reifying them as names.

Finally equipped with these standard features we can present the
dynamics of the calculus.

\subsubsection{Operational semantics} 

Finally, we introduce the computational dynamics. What marks these
algebras as distinct from other more traditionally studied algebraic
structures, e.g. vector spaces or polynomial rings, is the manner in
which dynamics is captured. In traditional structures, dynamics is typically
expressed through morphisms between such structures, as in linear maps
between vector spaces or morphisms between rings. In algebras
associated with the semantics of computation, the dynamics is
expressed as part of the algebraic structure itself, through a
reduction reduction relation typically denoted by $\red$. Below, we
give a recursive presentation of this relation for the calculus used
in the encoding.

$\red \subseteq \pi \times \pi$
$\red : \pi \to \mathcal{P}(\pi)$

\begin{mathpar}
  \inferrule* [lab=Comm] { \textsf{match}( x_{src}, x_{trgt} ) } { x_{trgt}?(y)P \; | \; x_{src}!\langle {Q} \rangle \red P\{\quotep{Q}/y}\} }
  \and \\
  \inferrule* [lab=Par] {{P} \red {P}'} {{{P} | {Q}} \red {{P}' | {Q}}}
  \and
  \inferrule* [lab=Equiv]{{{P} \scong {P}'} \andalso {{P}' \red {Q}'} \andalso {{Q}' \scong {Q}}}{{P} \red {Q}}
\end{mathpar}

\begin{eqnarray*}
  match_{\equiv} (\quotep{P},\quotep{Q}) & := & P \equiv Q \\
  match_{\dagger}(\quotep{P},\quotep{Q}) & := & \forall R. P|Q \red^{*} R => R \red^{*} 0 \\
  match_{K}(\quotep{P},\quotep{Q}) & := & K \mbox{ for some context } K
\end{eqnarray*}

$u?(x)P | u!\langle Q \rangle \red P\{\quotep{Q}/x\}$

%We write $\wred$ for $\red^*$, and $P\red$ if $\exists Q $ such that $ P \red Q$.
We write $P\red$ if $\exists Q $ such that $ P \red Q$ and $P\not\red$, otherwise.

\section{Replication}

As mentioned before, it is known that replication (and hence
recursion) can be implemented in a higher-order process algebra
\cite{SangiorgiWalker}. As our first example of calculation with the
machinery thus far presented we give the construction explicitly in
the {\rhoc}.

\begin{eqnarray}
	D_{x} & := & \prefix{x}{y}{(\binpar{\outputp{x}{y}}{@{y}})} \nonumber\\
	\bangp_{x}{P} & := & \binpar{{x}!\langle{\binpar{D_{x}}{P}}\rangle}{D_{x}} \nonumber
\end{eqnarray}

\begin{eqnarray}
	\bangp_{x}{P} & & \nonumber\\
	=
	& {x}!\langle{(\prefix{x}{y}{(\outputp{x}{y} | @{y})) | P}}\rangle 
	      | \prefix{x}{y}{(\outputp{x}{y} | @{y})} & \nonumber\\
	\red
	& (\outputp{x}{y} | @{y})\substn{\quotep{(\prefix{x}{y}{(@{y} | \outputp{x}{y})) | P}}}{y} & \nonumber\\
	=
	& \outputp{x}{\quotep{(\prefix{x}{y}{(\outputp{x}{y} | @{y})) | P}}}
	  | {(\prefix{x}{y}{(\outputp{x}{y} | @{y})) | P}} & \nonumber\\
	\red
	& \ldots & \nonumber\\
	\red^*
	& P | P | \ldots & \nonumber
\end{eqnarray}

Of course, this encoding, as an implementation, runs away, unfolding
$\bangp{P}$ eagerly. A lazier and more implementable replication
operator, restricted to input-guarded processes, may be obtained as follows.

\begin{eqnarray}
\bangp{\prefix{u}{v}{P}} 
	:= 
	\binpar{\lift{x}{\prefix{u}{v}{(\binpar{D(x)}{P})}}}{D(x)} \nonumber
\end{eqnarray}

\begin{remark}
  Note that the lazier definition still does not deal with summation
  or mixed summation (i.e. sums over input and output). The reader is
  invited to construct definitions of replication that deal with these
  features. 

  Further, the definitions are parameterized in a name, $x$. Can you,
  gentle reader, make a definition that eliminates this parameter and
  guarantees no accidental interaction between the replication
  machinery and the process being replicated -- i.e. no accidental
  sharing of names used by the process to get its work done and the
  name(s) used by the replication to effect copying. This latter
  revision of the definition of replication is crucial to obtaining
  the expected identity $!!P \sim !P$.
\end{remark}

\begin{remark}\label{rem:paradoxical_combinator}
  The reader familiar with the lambda calculus will have noticed the
  similarity between $D$ and the paradoxical combinator.

  [Ed. note: the existence of this seems to suggest we have to be more
  restrictive on the set of processes and names we admit if we are to
  support no-cloning.]
\end{remark}

\subsubsection{Bisimulation}

The computational dynamics gives rise to another kind of equivalence,
the equivalence of computational behavior. As previously mentioned
this is typically captured \emph{via} some form of bisimulation.

% The notion we use in this paper is weak barbed bisimulation
% \cite{milner91polyadicpi}.

The notion we use in this paper is derived from weak barbed
bisimulation \cite{milner91polyadicpi}. 

\begin{definition}
An \emph{observation relation}, $\downarrow_{\mathcal N}$, over a set
of names, $\mathcal N$, is the smallest relation satisfying the rules
below.

\infrule[Out-barb]{y \in {\mathcal N}, \; x \nameeq y}
		  {\outputp{x}{v} \downarrow_{\mathcal N} x}
\infrule[Par-barb]{\mbox{$P\downarrow_{\mathcal N} x$ or $Q\downarrow_{\mathcal N} x$}}
		  {\binpar{P}{Q} \downarrow_{\mathcal N} x}

We write $P \Downarrow_{\mathcal N} x$ if there is $Q$ such that 
$P \wred Q$ and $Q \downarrow_{\mathcal N} x$.
\end{definition}

\begin{definition}
%\label{def.bbisim}
An  ${\mathcal N}$-\emph{barbed bisimulation} over a set of names, ${\mathcal N}$, is a symmetric binary relation 
${\mathcal S}_{\mathcal N}$ between agents such that $P\rel{S}_{\mathcal N}Q$ implies:
\begin{enumerate}
\item If $P \red P'$ then $Q \wred Q'$ and $P'\rel{S}_{\mathcal N} Q'$.
\item If $P\downarrow_{\mathcal N} x$, then $Q\Downarrow_{\mathcal N} x$.
\end{enumerate}
$P$ is ${\mathcal N}$-barbed bisimilar to $Q$, written
$P \wbbisim_{\mathcal N} Q$, if $P \rel{S}_{\mathcal N} Q$ for some ${\mathcal N}$-barbed bisimulation ${\mathcal S}_{\mathcal N}$.
\end{definition}

$\mathcal{R} \subseteq \pi \times \pi$

$P \mathcal{R} Q => \forall P'. P \red P' \Rightarrow \exists Q'. Q \red Q', P' \mathcal{R} Q'$

$P \vdash x \Rightarrow Q \vdash x$

\begin{mathpar}
  \inferrule*[lab=Out-barb]{x \nameeq y}{{y}!\langle{Q}\rangle \vdash x}
  \and
  \inferrule*[lab=Par-barb]{\mbox{$P\vdash x$ or $Q\vdash x$}}{\binpar{P}{Q} \vdash x}
\end{mathpar}

\subsubsection{Contexts}

One of the principle advantages of computational calculi like the
$\pi$-calculus is a well-defined notion of context,
contextual-equivalence and a correlation between
contextual-equivalence and notions of bisimulation. The notion of
context allows the decomposition of a process into (sub-)process and
its syntactic environment, its context. Thus, a context may be
thought of as a process with a ``hole'' (written $\Box$) in it. The
application of a context $M$ to a process $P$, written $M[P]$, is
tantamount to filling the hole in $M$ with $P$. In this paper we do
not need the full weight of this theory, but do make use of the notion
of context in the proof the main theorem. 

\begin{mathpar}
  \inferrule* [lab=summation] {} {{M_{M},M_{N}} \bc \Box \;|\; x.M_{A} \;|\; M_{M}+M_{N}}
  \and
  \inferrule* [lab=agent] {} {{M_{A}} \bc (\vec{x})M_{P} \;| \; \clift{P_0,\ldots,M_{P},\ldots,P_N}}
  \and \\
  \inferrule* [lab=process] {} {{M_{P}} \bc M_{N} \;| \;P|M_{P} }
\end{mathpar} 

\begin{mathpar}
  \inferrule* [lab=sychronization] {} {M_{N} \bc \Box \;|\; x?M_{F} \;|\; x!M_{C}}
  \and
  \inferrule* [lab=abstraction] {} {{M_{F}} \bc (x)M_{P} }
  \and
  \inferrule* [lab=concretion] {} {{M_{C}} \bc \langle M_{P} \rangle }
  \and \\
  \inferrule* [lab=process] {} {{M_{P}} \bc M_{N} \;| \;P|M_{P} }
\end{mathpar}

\begin{definition}[contextual application] Given a context $M$, and
  process $P$, we define the \emph{contextual application}, $M[P] :=
  M\{P/\Box\}$. That is, the contextual application of M to P is the
  substitution of $P$ for $\Box$ in $M$.
\end{definition}

$\meaningof{-} : L \to \mathcal{P}(\pi)$

\begin{mathpar}
  \inferrule* [lab=collection] {} {\meaningof{true} = \pi, \and \meaningof{~E} = \pi \setminus \meaningof{E}, \and \meaningof{E_{1} \& E_{2}} = \meaningof{E_{1}} \cap \meaningof{E_{2}}}
\end{mathpar}

\begin{mathpar}
  \inferrule* [lab=structure] {} {\meaningof{0} = \{ P \in \pi | P \equiv 0 \}, \and \\ \meaningof{E_1 | E_2} = \{ P \in \pi | P \equiv P_{1} | P_{2}, P_{1} \in \meaningof{E_{1}}, P_{2} \in \meaningof{E_2}\} }
\end{mathpar}

\begin{mathpar}
 \inferrule* [lab=behavior] {} {\meaningof{\langle a?b \rangle E} = \{ P \in \pi | P \equiv Q | u?(y)P', \\ \and \\\\ \and \\ \;\;\; u \in \meaningof{a}, \forall z.P'\{z/y\} \in \meaningof{E\{z/b\}}\}, \and \\ \meaningof{a!E} = \{ P \in \pi | P \equiv Q | x!\langle P' \rangle, x \in \meaningof{a} P' \in \meaningof{E}\} }
\end{mathpar}

\begin{mathpar}
 \inferrule* [lab=nominal] {} {\meaningof{\quotep{E}} = \{ \quotep{P} \in \quotep{\pi} | P \in \meaningof{E} \}, \and \meaningof{\quotep{P}} = \{ \quotep{Q} \in \quotep{\pi} | P \equiv Q \} \and \\ \meaningof{@\quotep{E}} = \{ P \in \pi | P \equiv @x, x \in \meaningof{E} \}}
\end{mathpar}

\begin{eqnarray*}
  \\
  \meaningof{-} : TS \to ST
\end{eqnarray*}

\begin{eqnarray*}
  \\
  L : TS \to ST
\end{eqnarray*}

\begin{eqnarray*}
  \\
  P \models E \iff P \in \meaningof{E}
\end{eqnarray*}

\begin{eqnarray*}
  P \approx_{L} Q \iff \forall E \in L. P \models E \iff Q \models E
\end{eqnarray*}

\begin{eqnarray*}
  P \approx_{K} Q
\end{eqnarray*}

\begin{eqnarray*}
  P \approx Q
\end{eqnarray*}

$\approx_{K} = \approx = \approx_{L}$

\subsubsection{Contextual duality}

Note that contexts extend the quotation operation to a family of
operations from processes to names. Given a context, $M$, we can
define a \emph{nominal context}, $\quotep{M}$ by $\quotep{M}[P] :=
\quotep{M[P]}$. To foreshadow what is to come we observe that these
operations enjoy a duality with processes very much like the duality
between vectors and maps from vectors to scalars.

Further, because the calculus is essentially higher-order, we have a
correspondence between contexts and processes. More specifically,
given a name $x$ and a context $M$ we can construct $M^{*}_{x}$ such
that 

\begin{mathpar}
  M^{*}_{x} | \lift{x}{P} \red M[P]
\end{mathpar}

namely,

\begin{mathpar}
  M^{*}_{x} := x?(u).M[\dropn{u}]
\end{mathpar}

The dependence of $M^{*}_{x}$ on a name makes it an abstraction, 

\begin{mathpar}
  M^{*} := (x)x?(u).M[\dropn{u}]
\end{mathpar}

\subsection{Additional notation}

It will sometimes be convenient to denote the process a name
quotes. We already have the notation $x = \quotep{P}$, but it will be
convenient to introduce an alternate notation, $\procn{x}$, when we
want to emphasize the connection to the use of the name. Note that, by
virtue of name equivalence, $\quotep{\procn{x}} \nameeq x$; so, the
notation is consistent with previous definitions.

Further, because names have structure it is possible to effect
substitutions on the basis of that structure. This means we need to
upgrade our notation for substitutions, which we accomplish by
adapting comprehension notation. Thus,

\begin{mathpar}
  P\{ y / x : x \in S \}
\end{mathpar}

is interpreted to mean the process derived from P by replacing (in a
capture-avoiding manner) each occurrence of $x$ in $S$ by $y$. For example,

\begin{mathpar}
  P\{ \quotep{\procn{x}|\procn{x}} / x : x \in \freenames{P} \}
\end{mathpar}

will replace each (occurrence) of a free name $x$ in $P$ by
$\quotep{\procn{x}|\procn{x}}$.

Also, we will avail ourselves of the notation $x^{L}$ and $x^{R}$ to
denote injections of a name into disjoint copies of the name
space. There are numerous ways to accomplish this. One example can be
found in \cite{MeredithR05}. This notation overloads to vectors of
names: $\vec{x}^{\pi} := (x_{i}^{\pi} \; : \; 0 \leq i < |\vec{x}| )$ where $\pi \in \{L,R\}$.

We also use $P^{\Box} := P|\Box$.

In \cite{MeredithR05} an interpretation of the new operator is
given. It turns out that there are several possible interpretations
all enjoying the requisite algebraic properties of the operator (see
\cite{milner91polyadicpi}). We will therefore make liberal use of
$(\nu\; \vec{x})P$.

% subsection the_syntax_and_semantics_of_the_notation_system (end)   

\input{qm2pi.qmops} 

\input{qm2pi.sterngerlach} 

\input{qm2pi.metric} 

% section concurrent_process_calculi (end)

%\input{qm2pi.proofsketch}

% section proof sketch (end)

%\input{qm2pi.slviaknots} 

% section spatial logic via knots (end)

\input{qm2pi.conclusion}

% section conclusion (end)

%\input{qm2pi.dtcodes} 

% section wiring algorithm (end)

\input{qm2pi.ack} 

% section acknowledgments (end)

\newpage


\bibliographystyle{plain}   
\bibliography{../../biblios/main.bib}

\input{qm2pi.rhodetails}

\end{document}

 

% section concurrent_process_calculi (end)

%\documentclass[12pt]{llncs}
%\documentclass{jktr}

\usepackage[pdftex]{hyperref}                   
\usepackage {listings}
\usepackage {mathpartir}
\usepackage{bcprules}
%\usepackage{listings}
                       
\usepackage{graphicx} 
%\usepackage[margins=2.5cm,nohead,nofoot]{geometry}
%\usepackage{geometry}
\usepackage{amsfonts}
\usepackage{amstext}
\usepackage{latexsym}
\usepackage{amssymb}
\usepackage{color}


%\include{myPreamble}
\include{qm2pi.local} 

%\ifpdf
%\usepackage[pdftex]{graphicx}
%\else
%\usepackage{graphicx}
%\fi

 % \ifpdf
%  \usepackage{pdfsync}
%  \if


%\title{Brief Article}
%\author{David F. Snyder}
%\author{L.G. Meredith}

%\address{Dept. of Math., Texas State University--San Marcos, San Marcos, TX 78666}
       
\pagestyle{empty}


\begin{document}

\lstset{language=[Objective]Caml,frame=shadowbox}

\input{qm2pi.front}

% section front matter (end)

\input{qm2pi.intro} 
 
% section introduction (end)

% \input{qm2pi.knotations} 

% section notation (end)

\input{qm2pi.process.calculi} 

% section concurrent_process_calculi_and_spatial_logics_ (end)
    
%\input{qm2pi.knots2pi} 

%\input{qm2pi.trefoil} 

%\input{qm2pi.mainthm} 

% subsection basic_interpretation (end)

%\input{qm2pi.rho.presentation} 
\subsection{The syntax and semantics of the notation system}\label{sub:the_syntax_and_semantics_of_the_notation_system} % (fold)

We now summarize a technical presentation of the calculus that
embodies our theory of dynamics. The typical presentation of such a
calculus follows the style of giving generators and relations on
them. The grammar, below, describing term constructors, freely
generates the set of processes, $\Proc$. This set is then quotiented
by a relation known as structural congruence and it is over this set
that the notion of dynamics is expressed. This presentation is
essentially that of \cite{MeredithR05} with the addition of
polyadicity and summation. For readability we have relegated some of
the technical subtleties to an appendix.

\subsubsection{Process grammar}\label{subsub:process_grammar}

\begin{mathpar}
  \inferrule* [lab=synchronization] {} {{M} \bc \pzero \;|\; x?F \;|\; x!C }
  \and
  \inferrule* [lab=abstraction] {} {{F} \bc (x)P}
  \and
  \inferrule* [lab=concretion] {} {{C} \bc \langle Q \rangle}
  \and
  \inferrule* [lab=process] {} {{P,Q} \bc M \;| \;P|Q \;|\; @{x}}
  \and
  \inferrule* [lab=name] {} {{x} \bc \quotep{P}}
\end{mathpar} 

Note that $\vec{x}$ (resp. $\vec{P}$) denotes a vector of names
(resp. processes) of length $|\vec{x}|$ (resp. $|\vec{P}|$). We adopt
the following useful abbreviations.

\begin{mathpar}
   x?(\vec{y}).P := x.(\vec{y})P \and  x\clift{\vec{P}} := x.\clift{\vec{P}}
   \and x!(y) := \lift{x}{\dropn{y}}
   \and \Pi_{i=0}^{n-1}P_i := P_0 | \ldots | P_{n-1}
\end{mathpar}

\subsubsection{Structural congruence}

\paragraph{Free and bound names and alpha-equivalence.} At the
core of structural equivalence is alpha-equivalence which identifies
process that are the same up to a change of variable. Formally, we
recognize the distinction between free and bound names. The free names
of a process, $\freenames{P}$, may be calculated recursively as
follows:

\begin{mathpar}
\freenames{\pzero} := \emptyset
  \and \\
  \freenames{x?(y).P} := \{ x \} \cup (\freenames{P} \setminus \{ y \})
  \and 
  \freenames{x!\langle P \rangle} := \{ x \} \cup \{ P \} 
  \and \\
  \freenames{P|Q} := \freenames{P} \cup \freenames{Q}
  \and \\
  \freenames{@{x}} := \{ x \}
\end{mathpar}

$\pi$
$\quotep{\pi}$

$\freenames{-} : \pi \to \mathcal{P}(\quotep{\pi})$

\begin{eqnarray*}
  \freenames{\pzero} & := & \emptyset \\
  \freenames{x?(y).P} & := & \{ x \} \cup (\freenames{P} \setminus \{ y \}) \\
  \freenames{x!\langle P \rangle} & := & \{ x \} \cup \{ P \} \\
  \freenames{P|Q} & := & \freenames{P} \cup \freenames{Q} \\
  \freenames{\dropn{x}} & := & \{ x \}
\end{eqnarray*}

The bound names of a process, $\boundnames{P}$, are those names occurring in $P$
that are not free. For example, in $x?(y).0$, the name $x$ is free, while $y$ is bound.

\begin{mathpar}
  \inferrule* [lab=monoidal-laws] {} { P|Q \equiv Q|P \and P|0 \equiv P \and P|(Q|R) \equiv (P|Q)|R }
\end{mathpar}

\begin{mathpar}
  \inferrule* [lab=alpha-equivalence] {} { (x)P \equiv (y)P\{y/x\} \and y \not\in \freenames{P} }
\end{mathpar}

\begin{definition}
Then two processes, $P,Q$, are alpha-equivalent if $P = Q\{\vec{y}/\vec{x}\}$ for
some $\vec{x} \in \boundnames{Q},\vec{y} \in \boundnames{P}$, where $Q\{\vec{y}/\vec{x}\}$
denotes the capture-avoiding substitution of $\vec{y}$ for $\vec{x}$ in $Q$.
\end{definition}

\begin{definition}
  The {\em structural congruence} \cite{SangiorgiWalker} , $\equiv$,
  between processes is the least congruence containing
  alpha-equivalence, satisfying the abelian monoid laws
  (associativity, commutativity and $\pzero$ as identity) for parallel
  composition $|$ and for summation $+$.
\end{definition}

\subsection{Name equivalence}

We take name equivalence, written $\nameeq$, to be the smallest
equivalence relation generated by the following rules.

\begin{mathpar}
\inferrule*[lab=Quote-drop]
{ }
{ \quotep{@{x}} \nameeq x }

\inferrule*[lab=Struct-equiv]
{ P \scong Q }
{ \quotep{P} \nameeq \quotep{Q} }
\end{mathpar}

The astute reader will have noticed that the mutual recursion of names
and processes imposes a mutual recursion on alpha-equivalence and
structural equivalence via name-equivalence. Fortunately, all of this
works out pleasantly and we may calculate in the natural way, free of
concern. The reader interested in the details is referred to the
appendix \ref{appendix:rho_details}.

\subsection{Substitution}

We use $\Proc$ for the set of processes, $\QProc$ for the set of
names, and $\id{\{}\vec{y} / \vec{x} \id{\}}$ to denote partial maps,
$s : \QProc \rightarrow \QProc$. A map, $s$ lifts, uniquely, to a map
on process terms, $\widehat{s} : \Proc \rightarrow \Proc$ by the
following equations.

\begin{mathpar}
  (0) \psubstp{Q}{P} := 0 \\
  (R \juxtap S) \psubstp{Q}{P}
  :=    
  (R)\psubstp{Q}{P} \juxtap (S) \psubstp{Q}{P} \\
  (x?(y).R) \psubstp{Q}{P}    
  :=    
  (x)\substp{Q}{P} (z)\concat( (R \psubstn{z}{y}) \psubstp{Q}{P} ) \\
  (\lift{x}{R}) \psubstp{Q}{P}  
  :=
  \lift{(x)\substp{Q}{P}}{ R \psubstp{Q}{P} } \\
%   (\dropn{x})  \psubstp{Q}{P}       
%   := 
%   \left\{ 
%     \begin{array}{ccc} 
%       \dropn{\quotep{Q}} & & x \nameeq \quotep{P} \\
%       \dropn{x} & & otherwise \\
%     \end{array}
%   \right. 
  (\dropn{x})  \psubstp{Q}{P}       
  := 
  \left\{ 
    \begin{array}{ccc} 
      Q & & x \nameeq \quotep{P} \\
      \dropn{x} & & otherwise \\
    \end{array}
  \right.
\end{mathpar}
 

where

\begin{eqnarray}
  (x)\id{\{} \lpquote Q \rpquote / \lpquote P \rpquote \id{\}}            = 
  \left\{ 
    \begin{array}{ccc}
      \lpquote Q \rpquote & & x \nameeq \lpquote P \rpquote \\
      x & & otherwise \\
    \end{array}
  \right. \nonumber
\end{eqnarray}

and $z$ is chosen distinct from $\quotep{P}$, $\quotep{Q}$, the free
names in $Q$, and all the names in $R$. Our $\alpha$-equivalence will
be built in the standard way from this substitution.

\begin{remark}\label{rem:no_self_referential_names}
  One consequence of these definitions is that $\forall P. \quotep{P}
  \not\in \freenames{P}$.
\end{remark}

\subsection{ Dynamic quote: an example }

Anticipating something of what's to come, consider applying the
substitution, $\widehat{\id{\{}u / z \id{\}}}$, to the following pair
of processes, $\lift{w}{y!(z)}$ and $w[ \lpquote y!(z) \rpquote ]$.

\begin{eqnarray}
	\lift{w}{y!(z)}\widehat{\id{\{}u / z \id{\}}}
		& = &
		\lift{w}{y!(u)} \nonumber\\
	w[ \lpquote y!(z) \rpquote ] \widehat{ \id{\{}u / z \id{\}} }
		& = &
		w[ \lpquote y!(z) \rpquote ] \nonumber
\end{eqnarray}

Because the body of the process between quotes is impervious to
substitution, we get radically different answers. In fact, by
examining the first process in an input context,
e.g. $x?(z).\lift{w}{y!(z)}$, we see that the process under the lift
operator may be shaped by prefixed inputs binding a name inside it. In
this sense, the lift operator will be seen as a way to dynamically
construct processes before reifying them as names.

Finally equipped with these standard features we can present the
dynamics of the calculus.

\subsubsection{Operational semantics} 

Finally, we introduce the computational dynamics. What marks these
algebras as distinct from other more traditionally studied algebraic
structures, e.g. vector spaces or polynomial rings, is the manner in
which dynamics is captured. In traditional structures, dynamics is typically
expressed through morphisms between such structures, as in linear maps
between vector spaces or morphisms between rings. In algebras
associated with the semantics of computation, the dynamics is
expressed as part of the algebraic structure itself, through a
reduction reduction relation typically denoted by $\red$. Below, we
give a recursive presentation of this relation for the calculus used
in the encoding.

$\red \subseteq \pi \times \pi$
$\red : \pi \to \mathcal{P}(\pi)$

\begin{mathpar}
  \inferrule* [lab=Comm] { \textsf{match}( x_{src}, x_{trgt} ) } { x_{trgt}?(y)P \; | \; x_{src}!\langle {Q} \rangle \red P\{\quotep{Q}/y}\} }
  \and \\
  \inferrule* [lab=Par] {{P} \red {P}'} {{{P} | {Q}} \red {{P}' | {Q}}}
  \and
  \inferrule* [lab=Equiv]{{{P} \scong {P}'} \andalso {{P}' \red {Q}'} \andalso {{Q}' \scong {Q}}}{{P} \red {Q}}
\end{mathpar}

\begin{eqnarray*}
  match_{\equiv} (\quotep{P},\quotep{Q}) & := & P \equiv Q \\
  match_{\dagger}(\quotep{P},\quotep{Q}) & := & \forall R. P|Q \red^{*} R => R \red^{*} 0 \\
  match_{K}(\quotep{P},\quotep{Q}) & := & K \mbox{ for some context } K
\end{eqnarray*}

$u?(x)P | u!\langle Q \rangle \red P\{\quotep{Q}/x\}$

%We write $\wred$ for $\red^*$, and $P\red$ if $\exists Q $ such that $ P \red Q$.
We write $P\red$ if $\exists Q $ such that $ P \red Q$ and $P\not\red$, otherwise.

\section{Replication}

As mentioned before, it is known that replication (and hence
recursion) can be implemented in a higher-order process algebra
\cite{SangiorgiWalker}. As our first example of calculation with the
machinery thus far presented we give the construction explicitly in
the {\rhoc}.

\begin{eqnarray}
	D_{x} & := & \prefix{x}{y}{(\binpar{\outputp{x}{y}}{@{y}})} \nonumber\\
	\bangp_{x}{P} & := & \binpar{{x}!\langle{\binpar{D_{x}}{P}}\rangle}{D_{x}} \nonumber
\end{eqnarray}

\begin{eqnarray}
	\bangp_{x}{P} & & \nonumber\\
	=
	& {x}!\langle{(\prefix{x}{y}{(\outputp{x}{y} | @{y})) | P}}\rangle 
	      | \prefix{x}{y}{(\outputp{x}{y} | @{y})} & \nonumber\\
	\red
	& (\outputp{x}{y} | @{y})\substn{\quotep{(\prefix{x}{y}{(@{y} | \outputp{x}{y})) | P}}}{y} & \nonumber\\
	=
	& \outputp{x}{\quotep{(\prefix{x}{y}{(\outputp{x}{y} | @{y})) | P}}}
	  | {(\prefix{x}{y}{(\outputp{x}{y} | @{y})) | P}} & \nonumber\\
	\red
	& \ldots & \nonumber\\
	\red^*
	& P | P | \ldots & \nonumber
\end{eqnarray}

Of course, this encoding, as an implementation, runs away, unfolding
$\bangp{P}$ eagerly. A lazier and more implementable replication
operator, restricted to input-guarded processes, may be obtained as follows.

\begin{eqnarray}
\bangp{\prefix{u}{v}{P}} 
	:= 
	\binpar{\lift{x}{\prefix{u}{v}{(\binpar{D(x)}{P})}}}{D(x)} \nonumber
\end{eqnarray}

\begin{remark}
  Note that the lazier definition still does not deal with summation
  or mixed summation (i.e. sums over input and output). The reader is
  invited to construct definitions of replication that deal with these
  features. 

  Further, the definitions are parameterized in a name, $x$. Can you,
  gentle reader, make a definition that eliminates this parameter and
  guarantees no accidental interaction between the replication
  machinery and the process being replicated -- i.e. no accidental
  sharing of names used by the process to get its work done and the
  name(s) used by the replication to effect copying. This latter
  revision of the definition of replication is crucial to obtaining
  the expected identity $!!P \sim !P$.
\end{remark}

\begin{remark}\label{rem:paradoxical_combinator}
  The reader familiar with the lambda calculus will have noticed the
  similarity between $D$ and the paradoxical combinator.

  [Ed. note: the existence of this seems to suggest we have to be more
  restrictive on the set of processes and names we admit if we are to
  support no-cloning.]
\end{remark}

\subsubsection{Bisimulation}

The computational dynamics gives rise to another kind of equivalence,
the equivalence of computational behavior. As previously mentioned
this is typically captured \emph{via} some form of bisimulation.

% The notion we use in this paper is weak barbed bisimulation
% \cite{milner91polyadicpi}.

The notion we use in this paper is derived from weak barbed
bisimulation \cite{milner91polyadicpi}. 

\begin{definition}
An \emph{observation relation}, $\downarrow_{\mathcal N}$, over a set
of names, $\mathcal N$, is the smallest relation satisfying the rules
below.

\infrule[Out-barb]{y \in {\mathcal N}, \; x \nameeq y}
		  {\outputp{x}{v} \downarrow_{\mathcal N} x}
\infrule[Par-barb]{\mbox{$P\downarrow_{\mathcal N} x$ or $Q\downarrow_{\mathcal N} x$}}
		  {\binpar{P}{Q} \downarrow_{\mathcal N} x}

We write $P \Downarrow_{\mathcal N} x$ if there is $Q$ such that 
$P \wred Q$ and $Q \downarrow_{\mathcal N} x$.
\end{definition}

\begin{definition}
%\label{def.bbisim}
An  ${\mathcal N}$-\emph{barbed bisimulation} over a set of names, ${\mathcal N}$, is a symmetric binary relation 
${\mathcal S}_{\mathcal N}$ between agents such that $P\rel{S}_{\mathcal N}Q$ implies:
\begin{enumerate}
\item If $P \red P'$ then $Q \wred Q'$ and $P'\rel{S}_{\mathcal N} Q'$.
\item If $P\downarrow_{\mathcal N} x$, then $Q\Downarrow_{\mathcal N} x$.
\end{enumerate}
$P$ is ${\mathcal N}$-barbed bisimilar to $Q$, written
$P \wbbisim_{\mathcal N} Q$, if $P \rel{S}_{\mathcal N} Q$ for some ${\mathcal N}$-barbed bisimulation ${\mathcal S}_{\mathcal N}$.
\end{definition}

$\mathcal{R} \subseteq \pi \times \pi$

$P \mathcal{R} Q => \forall P'. P \red P' \Rightarrow \exists Q'. Q \red Q', P' \mathcal{R} Q'$

$P \vdash x \Rightarrow Q \vdash x$

\begin{mathpar}
  \inferrule*[lab=Out-barb]{x \nameeq y}{{y}!\langle{Q}\rangle \vdash x}
  \and
  \inferrule*[lab=Par-barb]{\mbox{$P\vdash x$ or $Q\vdash x$}}{\binpar{P}{Q} \vdash x}
\end{mathpar}

\subsubsection{Contexts}

One of the principle advantages of computational calculi like the
$\pi$-calculus is a well-defined notion of context,
contextual-equivalence and a correlation between
contextual-equivalence and notions of bisimulation. The notion of
context allows the decomposition of a process into (sub-)process and
its syntactic environment, its context. Thus, a context may be
thought of as a process with a ``hole'' (written $\Box$) in it. The
application of a context $M$ to a process $P$, written $M[P]$, is
tantamount to filling the hole in $M$ with $P$. In this paper we do
not need the full weight of this theory, but do make use of the notion
of context in the proof the main theorem. 

\begin{mathpar}
  \inferrule* [lab=summation] {} {{M_{M},M_{N}} \bc \Box \;|\; x.M_{A} \;|\; M_{M}+M_{N}}
  \and
  \inferrule* [lab=agent] {} {{M_{A}} \bc (\vec{x})M_{P} \;| \; \clift{P_0,\ldots,M_{P},\ldots,P_N}}
  \and \\
  \inferrule* [lab=process] {} {{M_{P}} \bc M_{N} \;| \;P|M_{P} }
\end{mathpar} 

\begin{mathpar}
  \inferrule* [lab=sychronization] {} {M_{N} \bc \Box \;|\; x?M_{F} \;|\; x!M_{C}}
  \and
  \inferrule* [lab=abstraction] {} {{M_{F}} \bc (x)M_{P} }
  \and
  \inferrule* [lab=concretion] {} {{M_{C}} \bc \langle M_{P} \rangle }
  \and \\
  \inferrule* [lab=process] {} {{M_{P}} \bc M_{N} \;| \;P|M_{P} }
\end{mathpar}

\begin{definition}[contextual application] Given a context $M$, and
  process $P$, we define the \emph{contextual application}, $M[P] :=
  M\{P/\Box\}$. That is, the contextual application of M to P is the
  substitution of $P$ for $\Box$ in $M$.
\end{definition}

$\meaningof{-} : L \to \mathcal{P}(\pi)$

\begin{mathpar}
  \inferrule* [lab=collection] {} {\meaningof{true} = \pi, \and \meaningof{~E} = \pi \setminus \meaningof{E}, \and \meaningof{E_{1} \& E_{2}} = \meaningof{E_{1}} \cap \meaningof{E_{2}}}
\end{mathpar}

\begin{mathpar}
  \inferrule* [lab=structure] {} {\meaningof{0} = \{ P \in \pi | P \equiv 0 \}, \and \\ \meaningof{E_1 | E_2} = \{ P \in \pi | P \equiv P_{1} | P_{2}, P_{1} \in \meaningof{E_{1}}, P_{2} \in \meaningof{E_2}\} }
\end{mathpar}

\begin{mathpar}
 \inferrule* [lab=behavior] {} {\meaningof{\langle a?b \rangle E} = \{ P \in \pi | P \equiv Q | u?(y)P', \\ \and \\\\ \and \\ \;\;\; u \in \meaningof{a}, \forall z.P'\{z/y\} \in \meaningof{E\{z/b\}}\}, \and \\ \meaningof{a!E} = \{ P \in \pi | P \equiv Q | x!\langle P' \rangle, x \in \meaningof{a} P' \in \meaningof{E}\} }
\end{mathpar}

\begin{mathpar}
 \inferrule* [lab=nominal] {} {\meaningof{\quotep{E}} = \{ \quotep{P} \in \quotep{\pi} | P \in \meaningof{E} \}, \and \meaningof{\quotep{P}} = \{ \quotep{Q} \in \quotep{\pi} | P \equiv Q \} \and \\ \meaningof{@\quotep{E}} = \{ P \in \pi | P \equiv @x, x \in \meaningof{E} \}}
\end{mathpar}

\begin{eqnarray*}
  \\
  \meaningof{-} : TS \to ST
\end{eqnarray*}

\begin{eqnarray*}
  \\
  L : TS \to ST
\end{eqnarray*}

\begin{eqnarray*}
  \\
  P \models E \iff P \in \meaningof{E}
\end{eqnarray*}

\begin{eqnarray*}
  P \approx_{L} Q \iff \forall E \in L. P \models E \iff Q \models E
\end{eqnarray*}

\begin{eqnarray*}
  P \approx_{K} Q
\end{eqnarray*}

\begin{eqnarray*}
  P \approx Q
\end{eqnarray*}

$\approx_{K} = \approx = \approx_{L}$

\subsubsection{Contextual duality}

Note that contexts extend the quotation operation to a family of
operations from processes to names. Given a context, $M$, we can
define a \emph{nominal context}, $\quotep{M}$ by $\quotep{M}[P] :=
\quotep{M[P]}$. To foreshadow what is to come we observe that these
operations enjoy a duality with processes very much like the duality
between vectors and maps from vectors to scalars.

Further, because the calculus is essentially higher-order, we have a
correspondence between contexts and processes. More specifically,
given a name $x$ and a context $M$ we can construct $M^{*}_{x}$ such
that 

\begin{mathpar}
  M^{*}_{x} | \lift{x}{P} \red M[P]
\end{mathpar}

namely,

\begin{mathpar}
  M^{*}_{x} := x?(u).M[\dropn{u}]
\end{mathpar}

The dependence of $M^{*}_{x}$ on a name makes it an abstraction, 

\begin{mathpar}
  M^{*} := (x)x?(u).M[\dropn{u}]
\end{mathpar}

\subsection{Additional notation}

It will sometimes be convenient to denote the process a name
quotes. We already have the notation $x = \quotep{P}$, but it will be
convenient to introduce an alternate notation, $\procn{x}$, when we
want to emphasize the connection to the use of the name. Note that, by
virtue of name equivalence, $\quotep{\procn{x}} \nameeq x$; so, the
notation is consistent with previous definitions.

Further, because names have structure it is possible to effect
substitutions on the basis of that structure. This means we need to
upgrade our notation for substitutions, which we accomplish by
adapting comprehension notation. Thus,

\begin{mathpar}
  P\{ y / x : x \in S \}
\end{mathpar}

is interpreted to mean the process derived from P by replacing (in a
capture-avoiding manner) each occurrence of $x$ in $S$ by $y$. For example,

\begin{mathpar}
  P\{ \quotep{\procn{x}|\procn{x}} / x : x \in \freenames{P} \}
\end{mathpar}

will replace each (occurrence) of a free name $x$ in $P$ by
$\quotep{\procn{x}|\procn{x}}$.

Also, we will avail ourselves of the notation $x^{L}$ and $x^{R}$ to
denote injections of a name into disjoint copies of the name
space. There are numerous ways to accomplish this. One example can be
found in \cite{MeredithR05}. This notation overloads to vectors of
names: $\vec{x}^{\pi} := (x_{i}^{\pi} \; : \; 0 \leq i < |\vec{x}| )$ where $\pi \in \{L,R\}$.

We also use $P^{\Box} := P|\Box$.

In \cite{MeredithR05} an interpretation of the new operator is
given. It turns out that there are several possible interpretations
all enjoying the requisite algebraic properties of the operator (see
\cite{milner91polyadicpi}). We will therefore make liberal use of
$(\nu\; \vec{x})P$.

% subsection the_syntax_and_semantics_of_the_notation_system (end)   

\input{qm2pi.qmops} 

\input{qm2pi.sterngerlach} 

\input{qm2pi.metric} 

% section concurrent_process_calculi (end)

%\input{qm2pi.proofsketch}

% section proof sketch (end)

%\input{qm2pi.slviaknots} 

% section spatial logic via knots (end)

\input{qm2pi.conclusion}

% section conclusion (end)

%\input{qm2pi.dtcodes} 

% section wiring algorithm (end)

\input{qm2pi.ack} 

% section acknowledgments (end)

\newpage


\bibliographystyle{plain}   
\bibliography{../../biblios/main.bib}

\input{qm2pi.rhodetails}

\end{document}



% section proof sketch (end)

%\section{Unlikely characters: spatial logic for
  knots}\label{sub:characteristic_formulae} % (fold)

Associated to the mobile process calculi are a family of logics known
as the Hennessy-Milner logics. These logics typically enjoy a
semantics interpreting formulae as sets of processes that when
factored through the encoding outlined above allows an identification
of classes of knots with logical formulae. In the context of this
encoding the sub-family known as the spatial logics \cite{CairesC03}
\cite{CairesC04} \cite{Caires04} are of particular interest providing
several important features for expressing and reasoning about
properties (i.e. classes) of knots. We hint here at how this may be done.

%\begin{description}
%\item [structural connectives] 
\subsubsection{Structural connectives} The spatial logics enjoy
structural connectives corresponding, at the logical level, to the
parallel composition ($P | Q$) and new name ($(\nu \; x)P$)
connectives for processes. As illustrated in the examples below, these
connectives are extremely expressive given the shape of our encoding.
%\item [decideable satisfaction]

\subsubsection{Decideable satisfaction}
In \cite{Caires04} the satisfaction relation is shown to be decideable
for a rich class of processes. It further turns out that the image of
the our encoding is a proper subset of that class. This result
provides the basis for an algorithm by which to search for knots
enjoying a given property.
%\item [characteristic formulae]

\subsubsection{Characteristic formulae}
In the same paper \cite{Caires04} , Caires presents a means of calculating
characteristic formulae, selecting equivalence classes of processes
up to a pre--specified depth limit on the support set of names. Composed with our
encoding, this characteristic formula can be used to select
characteristic formulae for knots.
%\end{description}

\subsubsection{Spatial logic formulae}

The grammar below (segmented for comprehension) summarizes the syntax
of spatial logic formulae. We employ illustrative examples in the
sequel to provide an intuitive understanding of their meaning
referring the reader to \cite{Caires04} for a more detailed explication
of the semantics.

\begin{mathpar}
  \inferrule* [lab=boolean] {} {{A,B} \bc T \;|\; \neg A \;|\; A \wedge B \;|\; \eta = \eta'}
  \and
  \inferrule* [lab=spatial] {} {|\; \pzero \;|\; A | B \;|\; x \text{\textregistered} A \;|\; \forall x . A \;|\;  H x . A}
  \and
  \inferrule* [lab=behavioral] {} {|\; \alpha . A}
  \and 
  \inferrule* [lab=recursion] {} {|\; X(\vec{u}) \;|\; \mu X(\vec{u}) . A}
  \and
  \inferrule* [lab=action] {} {\alpha \bc \langle x?(\vec{y}) \rangle \;|\; \langle x!(\vec{y}) \rangle \;|\; \langle \tau \rangle}
  \and 
  \inferrule* [lab=name] {} {\eta \bc x \;|\; \tau}
\end{mathpar} 

% subsection characteristic_formulae (end)   	 

\subsection{Example formulae}\label{sub:example_formulae_} % (fold)

\subsubsection{Crossing as formula.}
% 
% \begin{align*}
%   \frac{d}{dx} \sin x &= \cos x 
%   & \frac{d}{dx} e^x &= e^x \\
%   \frac{d}{dx} \cos x &= - \sin x 
%   & \frac{d}{dx} \log x &= \frac{1}{x} \\
% \end{align*} 

\begin{align*}
 \mu C(x_{0},x_{1},y_{0},y_{1},u).&(\langle x_{0}?(z) \rangle(\langle u! \rangle\langle y_{1}!z \rangle C(x_{0},x_{1},y_{0},y_{1},u)) & \\
  & \wedge \langle y_{1}?(z) \rangle (\langle u! \rangle \langle x_{0}!z \rangle C(x_{0},x_{1},y_{0},y_{1},u)) & \\
  & \wedge \langle x_{1}?(z) \rangle (\langle u? \rangle \langle y_{0}!z \rangle C(x_{0},x_{1},y_{0},y_{1},u)) & \\
  & \wedge \langle y_{0}?(z) \rangle (\langle u? \rangle \langle x_{1}!z \rangle C(x_{0},x_{1},y_{0},y_{1},u))) &
\end{align*}

The lexicographical similarity between the shape of this formulae and
the shape of definition of the process representing a crossing reveals
the intuitive meaning of this formulae. It describes the capabilities
of a process that has the right to represent a crossing. For example
it picks out processes that may perform an input on the port $x_0$ in
its initial menu of capabilities. What differentiates the formula
from the process, however, is that the crossing process is the
smallest candidate to satisfy the formula. Infinitely many other
processes -- with internal behavior hidden behind this interface, so
to speak -- also satisfy this formula. Even this simple formula,
then, can be seen to open a new view onto knots, providing a
computational interpretation of \emph{virtual} knots.

Note that this formula is derived by hand. A similar formula can be
derived by employing Caires' calculation of characteristic formula
\cite{Caires04} to the process representing a crossing. In light of
this discussion, we let
$\meaningof{C}_{\phi}(x0,x1,y0,y1,u)$ denote a formula specifying the
dynamics we wish to capture of a crossing. To guarantee we preserve
the shape of the interface and minimal semantics we demand that
$\meaningof{C}_{\phi}(x0,x1,y0,y1,u) \Rightarrow
\textbf{C}(x0,x1,y0,y1,u)$ where $\textbf{C}(x0,x1,y0,y1,u)$ denotes
the formula above.
                            
\subsubsection{Crossing number constraints.}
The moral content of the context lemma (Lemma \ref{context}) is that the notion of
``locality'' in the Reidemeister moves is effectively captured by the
parallel composition operator of the process calculus. This intuition
extends through the logic. Given a formula,
$\meaningof{C}_{\phi}(x0,x1,y0,y1,u)$, we can use the structural
connectives to specify constraints on crossing numbers, such as at
least $n$ crossings, or exactly $n$ crossings.
\begin{mathpar}
  \inferrule* [lab=at-least-n] {} { K^{\geq n}_{\phi}(\vec{xs},\vec{ys}) := \Pi_{i=0}^{n-1} Hu . \meaningof{C}_{\phi}(xs_i,ys_i,u) | T }
  \and 
  \inferrule* [lab=exactly-n] {} { K^{= n}_{\phi}(\vec{xs},\vec{ys}) := \Pi_{i=0}^{n-1} Hu . \meaningof{C}_{\phi}(xs_i,ys_i,u) | \neg (\forall x_0,y_0,x_1,y_1,u . \meaningof{C}_{\phi}(x_0,y_0,x_1,y_1,u) | T) }
\end{mathpar}

To round out this section, recall that the encoding of an $n$-crossing
knot decomposes into a parallel composition of $n$ \emph{copies} of a
crossing process together with a wiring harness. To specify different
knot classes with the same crossing number amounts to specifying
logical constraints on the wiring harness. In the interest of space,
we defer examples to a forthcoming paper. Suffice it to say that both
the conditions ``alternating knot'' and ``contains the tangle
corresponding to 5/3'' are expressible. For example, it is possible to
calculate the characteristic formula of a process corresponding to the
tangle 5/3 and conjoin it into the classifying formula via the
composition connective of the logic.

Finally, we wish to observe that it is entirely within reason to
contemplate a more domain-specific version of spatial logic tailored
to the shape of processes in the image of the encoding. Such a
domain-specific logic would have a better claim to the title formal
language of knot properties.

% subsection example_formulae_ (end)

% section knots_as_processes (end) 

% section spatial logic via knots (end)

\section{Conclusions and future work}

\paragraph{Testing physical space}
You, gentle reader, may wonder why of all the theorems to be proved
given this set up we pick the one above. In some sense it's hardly
central to quantum mechanics. We see it as central in the sense that
it firmly establishes a notion of physical space arising from a notion
of the equivalence of behavior. Relating bisimulation to a metric is a
big step forward, but one is faced with interpreting the relationship
of that metric space to something more physical. Quantum mechanical
notions of ``physical'' space are still far from intuitive, but by
relating this idea of distance as testing to calculations that predict
physical circumstances we are making a not insignificant step forward
toward an understanding of the physical space we inhabit as
essentially dynamic.

\paragraph{Effectivity and simulation}
One of the observations we have yet to make is that the entire program
spelled out here is effective. We have built various interpreters for
the reflective calculus at work in this interpretation. In principle,
then, we can simulate quantum mechanics on a computer. The place where
the simulation may lose fidelity is the infinitely branching summation
for the annihilator.

In this connection i also want to point out that the evaluation style
calculation of the inner product puts the non-determinism of the
summation right at the heart of measurement. This suggests that
Milner's original reduction-based formulation of the dynamics of his
calculi in terms of sums was not just notationally suggestive of a
notion of measure-and-continue but captured some significant part of
the physics.

\paragraph{Quantum continuations}
In light of this last observation i want to point out that the
predominant account of quantum mechanics is missing a key aspect of a
truly compositional story of the physical situation. In a real lab,
when a measurement is made the observation can be made to feed into
another device that then makes another measurement conditioned on the
results of the first. This means that after the superposition was
collapsed the entire experimental set up remained in
superposition. While QM offers a means of writing this down it doesn't
quite line up well with the well-trodden formulation of computation
and continuation that we see so succinctly expressed in Milner's
calculi. This suggests that there might be advantages to this account
of dynamics waiting to be explored.

\paragraph{Quantum logic}
In this connection, we also note that by virtue of having the
Hennessy-Milner construction, we can pull the construction through the
interpretation of QM. This gives us a natural candidate for a quantum
logic that enjoys an extremely tight connection with it's domain of
interpretation, making the construction much less ad hoc (rather it is
the image of functor!).

\paragraph{Quantum probabiity}
i have questions about the basis of the interpretation of inner
product as probability amplitude. In particular, using which
axiomatization of probability theory does the notion of probability
amplitude earn the right to be so dubbed? In other words, where is the
proof that the operation for calculating a probability amplitude (and
then squaring) satisfies the axioms of what it means to calculate a
probability? Even if such a proof exists (i have yet to find it in the
literature), i wonder if it might not be possible to turn things on
their heads. Can we view the calculation of the probability amplitude
as an axiomatization of probability? If so, then the definition we
give for calculating probability amplitude may provide the basis for
an \emph{effective} theory of probability.

\paragraph{Quantum vs ``biological'' information}
Finally, i want to conclude with a more philosophical observation. At
a recent workshop in which QM was a predominant topic i noticed
something about quantum information. The speaker was giving a riveting
discussion of axiomatic QM and showing how properties of ``no
cloning'' and ``no deleting'' emerged as consequences of the
axiomatization. Theorems of this form are necessary to give us a sense
of confidence that our axioms characterize the physical theory. What
struck me, though, was that if quantum information is neither erasable
nor replicable it is markedly different from \emph{life}. Two of the
things we know about life is that

\begin{itemize}
  \item it ends;
  \item to gain some measure of persistence, to transcend it's
    finitude it is imminently copyable.
\end{itemize}

Both of these qualities are summarized succinctly in the aphorism: all
flesh is grass. For me these two kinds of ``information'' -- call them
quantum and biological -- are end points on a spectrum of strategies
for persistence. At one end, we have those curious entities that enjoy
uniqueness and permanence; at the other, we have those who in the face
of a certain end and an uncertain present make a go of passing
something on. To me one of the more remarkable aspects of the latter
strategy is that in the presence of noise (and certain features of
copying) we get a kind of dynamism, a chance for improvement against a
given persistent condition.

% subsection other_calculi_other_bisimulations_and_geometry_as_behavior (end)




% section conclusion (end)

%\documentclass[12pt]{llncs}
%\documentclass{jktr}

\usepackage[pdftex]{hyperref}                   
\usepackage {listings}
\usepackage {mathpartir}
\usepackage{bcprules}
%\usepackage{listings}
                       
\usepackage{graphicx} 
%\usepackage[margins=2.5cm,nohead,nofoot]{geometry}
%\usepackage{geometry}
\usepackage{amsfonts}
\usepackage{amstext}
\usepackage{latexsym}
\usepackage{amssymb}
\usepackage{color}


%\include{myPreamble}
\include{qm2pi.local} 

%\ifpdf
%\usepackage[pdftex]{graphicx}
%\else
%\usepackage{graphicx}
%\fi

 % \ifpdf
%  \usepackage{pdfsync}
%  \if


%\title{Brief Article}
%\author{David F. Snyder}
%\author{L.G. Meredith}

%\address{Dept. of Math., Texas State University--San Marcos, San Marcos, TX 78666}
       
\pagestyle{empty}


\begin{document}

\lstset{language=[Objective]Caml,frame=shadowbox}

\input{qm2pi.front}

% section front matter (end)

\input{qm2pi.intro} 
 
% section introduction (end)

% \input{qm2pi.knotations} 

% section notation (end)

\input{qm2pi.process.calculi} 

% section concurrent_process_calculi_and_spatial_logics_ (end)
    
%\input{qm2pi.knots2pi} 

%\input{qm2pi.trefoil} 

%\input{qm2pi.mainthm} 

% subsection basic_interpretation (end)

%\input{qm2pi.rho.presentation} 
\subsection{The syntax and semantics of the notation system}\label{sub:the_syntax_and_semantics_of_the_notation_system} % (fold)

We now summarize a technical presentation of the calculus that
embodies our theory of dynamics. The typical presentation of such a
calculus follows the style of giving generators and relations on
them. The grammar, below, describing term constructors, freely
generates the set of processes, $\Proc$. This set is then quotiented
by a relation known as structural congruence and it is over this set
that the notion of dynamics is expressed. This presentation is
essentially that of \cite{MeredithR05} with the addition of
polyadicity and summation. For readability we have relegated some of
the technical subtleties to an appendix.

\subsubsection{Process grammar}\label{subsub:process_grammar}

\begin{mathpar}
  \inferrule* [lab=synchronization] {} {{M} \bc \pzero \;|\; x?F \;|\; x!C }
  \and
  \inferrule* [lab=abstraction] {} {{F} \bc (x)P}
  \and
  \inferrule* [lab=concretion] {} {{C} \bc \langle Q \rangle}
  \and
  \inferrule* [lab=process] {} {{P,Q} \bc M \;| \;P|Q \;|\; @{x}}
  \and
  \inferrule* [lab=name] {} {{x} \bc \quotep{P}}
\end{mathpar} 

Note that $\vec{x}$ (resp. $\vec{P}$) denotes a vector of names
(resp. processes) of length $|\vec{x}|$ (resp. $|\vec{P}|$). We adopt
the following useful abbreviations.

\begin{mathpar}
   x?(\vec{y}).P := x.(\vec{y})P \and  x\clift{\vec{P}} := x.\clift{\vec{P}}
   \and x!(y) := \lift{x}{\dropn{y}}
   \and \Pi_{i=0}^{n-1}P_i := P_0 | \ldots | P_{n-1}
\end{mathpar}

\subsubsection{Structural congruence}

\paragraph{Free and bound names and alpha-equivalence.} At the
core of structural equivalence is alpha-equivalence which identifies
process that are the same up to a change of variable. Formally, we
recognize the distinction between free and bound names. The free names
of a process, $\freenames{P}$, may be calculated recursively as
follows:

\begin{mathpar}
\freenames{\pzero} := \emptyset
  \and \\
  \freenames{x?(y).P} := \{ x \} \cup (\freenames{P} \setminus \{ y \})
  \and 
  \freenames{x!\langle P \rangle} := \{ x \} \cup \{ P \} 
  \and \\
  \freenames{P|Q} := \freenames{P} \cup \freenames{Q}
  \and \\
  \freenames{@{x}} := \{ x \}
\end{mathpar}

$\pi$
$\quotep{\pi}$

$\freenames{-} : \pi \to \mathcal{P}(\quotep{\pi})$

\begin{eqnarray*}
  \freenames{\pzero} & := & \emptyset \\
  \freenames{x?(y).P} & := & \{ x \} \cup (\freenames{P} \setminus \{ y \}) \\
  \freenames{x!\langle P \rangle} & := & \{ x \} \cup \{ P \} \\
  \freenames{P|Q} & := & \freenames{P} \cup \freenames{Q} \\
  \freenames{\dropn{x}} & := & \{ x \}
\end{eqnarray*}

The bound names of a process, $\boundnames{P}$, are those names occurring in $P$
that are not free. For example, in $x?(y).0$, the name $x$ is free, while $y$ is bound.

\begin{mathpar}
  \inferrule* [lab=monoidal-laws] {} { P|Q \equiv Q|P \and P|0 \equiv P \and P|(Q|R) \equiv (P|Q)|R }
\end{mathpar}

\begin{mathpar}
  \inferrule* [lab=alpha-equivalence] {} { (x)P \equiv (y)P\{y/x\} \and y \not\in \freenames{P} }
\end{mathpar}

\begin{definition}
Then two processes, $P,Q$, are alpha-equivalent if $P = Q\{\vec{y}/\vec{x}\}$ for
some $\vec{x} \in \boundnames{Q},\vec{y} \in \boundnames{P}$, where $Q\{\vec{y}/\vec{x}\}$
denotes the capture-avoiding substitution of $\vec{y}$ for $\vec{x}$ in $Q$.
\end{definition}

\begin{definition}
  The {\em structural congruence} \cite{SangiorgiWalker} , $\equiv$,
  between processes is the least congruence containing
  alpha-equivalence, satisfying the abelian monoid laws
  (associativity, commutativity and $\pzero$ as identity) for parallel
  composition $|$ and for summation $+$.
\end{definition}

\subsection{Name equivalence}

We take name equivalence, written $\nameeq$, to be the smallest
equivalence relation generated by the following rules.

\begin{mathpar}
\inferrule*[lab=Quote-drop]
{ }
{ \quotep{@{x}} \nameeq x }

\inferrule*[lab=Struct-equiv]
{ P \scong Q }
{ \quotep{P} \nameeq \quotep{Q} }
\end{mathpar}

The astute reader will have noticed that the mutual recursion of names
and processes imposes a mutual recursion on alpha-equivalence and
structural equivalence via name-equivalence. Fortunately, all of this
works out pleasantly and we may calculate in the natural way, free of
concern. The reader interested in the details is referred to the
appendix \ref{appendix:rho_details}.

\subsection{Substitution}

We use $\Proc$ for the set of processes, $\QProc$ for the set of
names, and $\id{\{}\vec{y} / \vec{x} \id{\}}$ to denote partial maps,
$s : \QProc \rightarrow \QProc$. A map, $s$ lifts, uniquely, to a map
on process terms, $\widehat{s} : \Proc \rightarrow \Proc$ by the
following equations.

\begin{mathpar}
  (0) \psubstp{Q}{P} := 0 \\
  (R \juxtap S) \psubstp{Q}{P}
  :=    
  (R)\psubstp{Q}{P} \juxtap (S) \psubstp{Q}{P} \\
  (x?(y).R) \psubstp{Q}{P}    
  :=    
  (x)\substp{Q}{P} (z)\concat( (R \psubstn{z}{y}) \psubstp{Q}{P} ) \\
  (\lift{x}{R}) \psubstp{Q}{P}  
  :=
  \lift{(x)\substp{Q}{P}}{ R \psubstp{Q}{P} } \\
%   (\dropn{x})  \psubstp{Q}{P}       
%   := 
%   \left\{ 
%     \begin{array}{ccc} 
%       \dropn{\quotep{Q}} & & x \nameeq \quotep{P} \\
%       \dropn{x} & & otherwise \\
%     \end{array}
%   \right. 
  (\dropn{x})  \psubstp{Q}{P}       
  := 
  \left\{ 
    \begin{array}{ccc} 
      Q & & x \nameeq \quotep{P} \\
      \dropn{x} & & otherwise \\
    \end{array}
  \right.
\end{mathpar}
 

where

\begin{eqnarray}
  (x)\id{\{} \lpquote Q \rpquote / \lpquote P \rpquote \id{\}}            = 
  \left\{ 
    \begin{array}{ccc}
      \lpquote Q \rpquote & & x \nameeq \lpquote P \rpquote \\
      x & & otherwise \\
    \end{array}
  \right. \nonumber
\end{eqnarray}

and $z$ is chosen distinct from $\quotep{P}$, $\quotep{Q}$, the free
names in $Q$, and all the names in $R$. Our $\alpha$-equivalence will
be built in the standard way from this substitution.

\begin{remark}\label{rem:no_self_referential_names}
  One consequence of these definitions is that $\forall P. \quotep{P}
  \not\in \freenames{P}$.
\end{remark}

\subsection{ Dynamic quote: an example }

Anticipating something of what's to come, consider applying the
substitution, $\widehat{\id{\{}u / z \id{\}}}$, to the following pair
of processes, $\lift{w}{y!(z)}$ and $w[ \lpquote y!(z) \rpquote ]$.

\begin{eqnarray}
	\lift{w}{y!(z)}\widehat{\id{\{}u / z \id{\}}}
		& = &
		\lift{w}{y!(u)} \nonumber\\
	w[ \lpquote y!(z) \rpquote ] \widehat{ \id{\{}u / z \id{\}} }
		& = &
		w[ \lpquote y!(z) \rpquote ] \nonumber
\end{eqnarray}

Because the body of the process between quotes is impervious to
substitution, we get radically different answers. In fact, by
examining the first process in an input context,
e.g. $x?(z).\lift{w}{y!(z)}$, we see that the process under the lift
operator may be shaped by prefixed inputs binding a name inside it. In
this sense, the lift operator will be seen as a way to dynamically
construct processes before reifying them as names.

Finally equipped with these standard features we can present the
dynamics of the calculus.

\subsubsection{Operational semantics} 

Finally, we introduce the computational dynamics. What marks these
algebras as distinct from other more traditionally studied algebraic
structures, e.g. vector spaces or polynomial rings, is the manner in
which dynamics is captured. In traditional structures, dynamics is typically
expressed through morphisms between such structures, as in linear maps
between vector spaces or morphisms between rings. In algebras
associated with the semantics of computation, the dynamics is
expressed as part of the algebraic structure itself, through a
reduction reduction relation typically denoted by $\red$. Below, we
give a recursive presentation of this relation for the calculus used
in the encoding.

$\red \subseteq \pi \times \pi$
$\red : \pi \to \mathcal{P}(\pi)$

\begin{mathpar}
  \inferrule* [lab=Comm] { \textsf{match}( x_{src}, x_{trgt} ) } { x_{trgt}?(y)P \; | \; x_{src}!\langle {Q} \rangle \red P\{\quotep{Q}/y}\} }
  \and \\
  \inferrule* [lab=Par] {{P} \red {P}'} {{{P} | {Q}} \red {{P}' | {Q}}}
  \and
  \inferrule* [lab=Equiv]{{{P} \scong {P}'} \andalso {{P}' \red {Q}'} \andalso {{Q}' \scong {Q}}}{{P} \red {Q}}
\end{mathpar}

\begin{eqnarray*}
  match_{\equiv} (\quotep{P},\quotep{Q}) & := & P \equiv Q \\
  match_{\dagger}(\quotep{P},\quotep{Q}) & := & \forall R. P|Q \red^{*} R => R \red^{*} 0 \\
  match_{K}(\quotep{P},\quotep{Q}) & := & K \mbox{ for some context } K
\end{eqnarray*}

$u?(x)P | u!\langle Q \rangle \red P\{\quotep{Q}/x\}$

%We write $\wred$ for $\red^*$, and $P\red$ if $\exists Q $ such that $ P \red Q$.
We write $P\red$ if $\exists Q $ such that $ P \red Q$ and $P\not\red$, otherwise.

\section{Replication}

As mentioned before, it is known that replication (and hence
recursion) can be implemented in a higher-order process algebra
\cite{SangiorgiWalker}. As our first example of calculation with the
machinery thus far presented we give the construction explicitly in
the {\rhoc}.

\begin{eqnarray}
	D_{x} & := & \prefix{x}{y}{(\binpar{\outputp{x}{y}}{@{y}})} \nonumber\\
	\bangp_{x}{P} & := & \binpar{{x}!\langle{\binpar{D_{x}}{P}}\rangle}{D_{x}} \nonumber
\end{eqnarray}

\begin{eqnarray}
	\bangp_{x}{P} & & \nonumber\\
	=
	& {x}!\langle{(\prefix{x}{y}{(\outputp{x}{y} | @{y})) | P}}\rangle 
	      | \prefix{x}{y}{(\outputp{x}{y} | @{y})} & \nonumber\\
	\red
	& (\outputp{x}{y} | @{y})\substn{\quotep{(\prefix{x}{y}{(@{y} | \outputp{x}{y})) | P}}}{y} & \nonumber\\
	=
	& \outputp{x}{\quotep{(\prefix{x}{y}{(\outputp{x}{y} | @{y})) | P}}}
	  | {(\prefix{x}{y}{(\outputp{x}{y} | @{y})) | P}} & \nonumber\\
	\red
	& \ldots & \nonumber\\
	\red^*
	& P | P | \ldots & \nonumber
\end{eqnarray}

Of course, this encoding, as an implementation, runs away, unfolding
$\bangp{P}$ eagerly. A lazier and more implementable replication
operator, restricted to input-guarded processes, may be obtained as follows.

\begin{eqnarray}
\bangp{\prefix{u}{v}{P}} 
	:= 
	\binpar{\lift{x}{\prefix{u}{v}{(\binpar{D(x)}{P})}}}{D(x)} \nonumber
\end{eqnarray}

\begin{remark}
  Note that the lazier definition still does not deal with summation
  or mixed summation (i.e. sums over input and output). The reader is
  invited to construct definitions of replication that deal with these
  features. 

  Further, the definitions are parameterized in a name, $x$. Can you,
  gentle reader, make a definition that eliminates this parameter and
  guarantees no accidental interaction between the replication
  machinery and the process being replicated -- i.e. no accidental
  sharing of names used by the process to get its work done and the
  name(s) used by the replication to effect copying. This latter
  revision of the definition of replication is crucial to obtaining
  the expected identity $!!P \sim !P$.
\end{remark}

\begin{remark}\label{rem:paradoxical_combinator}
  The reader familiar with the lambda calculus will have noticed the
  similarity between $D$ and the paradoxical combinator.

  [Ed. note: the existence of this seems to suggest we have to be more
  restrictive on the set of processes and names we admit if we are to
  support no-cloning.]
\end{remark}

\subsubsection{Bisimulation}

The computational dynamics gives rise to another kind of equivalence,
the equivalence of computational behavior. As previously mentioned
this is typically captured \emph{via} some form of bisimulation.

% The notion we use in this paper is weak barbed bisimulation
% \cite{milner91polyadicpi}.

The notion we use in this paper is derived from weak barbed
bisimulation \cite{milner91polyadicpi}. 

\begin{definition}
An \emph{observation relation}, $\downarrow_{\mathcal N}$, over a set
of names, $\mathcal N$, is the smallest relation satisfying the rules
below.

\infrule[Out-barb]{y \in {\mathcal N}, \; x \nameeq y}
		  {\outputp{x}{v} \downarrow_{\mathcal N} x}
\infrule[Par-barb]{\mbox{$P\downarrow_{\mathcal N} x$ or $Q\downarrow_{\mathcal N} x$}}
		  {\binpar{P}{Q} \downarrow_{\mathcal N} x}

We write $P \Downarrow_{\mathcal N} x$ if there is $Q$ such that 
$P \wred Q$ and $Q \downarrow_{\mathcal N} x$.
\end{definition}

\begin{definition}
%\label{def.bbisim}
An  ${\mathcal N}$-\emph{barbed bisimulation} over a set of names, ${\mathcal N}$, is a symmetric binary relation 
${\mathcal S}_{\mathcal N}$ between agents such that $P\rel{S}_{\mathcal N}Q$ implies:
\begin{enumerate}
\item If $P \red P'$ then $Q \wred Q'$ and $P'\rel{S}_{\mathcal N} Q'$.
\item If $P\downarrow_{\mathcal N} x$, then $Q\Downarrow_{\mathcal N} x$.
\end{enumerate}
$P$ is ${\mathcal N}$-barbed bisimilar to $Q$, written
$P \wbbisim_{\mathcal N} Q$, if $P \rel{S}_{\mathcal N} Q$ for some ${\mathcal N}$-barbed bisimulation ${\mathcal S}_{\mathcal N}$.
\end{definition}

$\mathcal{R} \subseteq \pi \times \pi$

$P \mathcal{R} Q => \forall P'. P \red P' \Rightarrow \exists Q'. Q \red Q', P' \mathcal{R} Q'$

$P \vdash x \Rightarrow Q \vdash x$

\begin{mathpar}
  \inferrule*[lab=Out-barb]{x \nameeq y}{{y}!\langle{Q}\rangle \vdash x}
  \and
  \inferrule*[lab=Par-barb]{\mbox{$P\vdash x$ or $Q\vdash x$}}{\binpar{P}{Q} \vdash x}
\end{mathpar}

\subsubsection{Contexts}

One of the principle advantages of computational calculi like the
$\pi$-calculus is a well-defined notion of context,
contextual-equivalence and a correlation between
contextual-equivalence and notions of bisimulation. The notion of
context allows the decomposition of a process into (sub-)process and
its syntactic environment, its context. Thus, a context may be
thought of as a process with a ``hole'' (written $\Box$) in it. The
application of a context $M$ to a process $P$, written $M[P]$, is
tantamount to filling the hole in $M$ with $P$. In this paper we do
not need the full weight of this theory, but do make use of the notion
of context in the proof the main theorem. 

\begin{mathpar}
  \inferrule* [lab=summation] {} {{M_{M},M_{N}} \bc \Box \;|\; x.M_{A} \;|\; M_{M}+M_{N}}
  \and
  \inferrule* [lab=agent] {} {{M_{A}} \bc (\vec{x})M_{P} \;| \; \clift{P_0,\ldots,M_{P},\ldots,P_N}}
  \and \\
  \inferrule* [lab=process] {} {{M_{P}} \bc M_{N} \;| \;P|M_{P} }
\end{mathpar} 

\begin{mathpar}
  \inferrule* [lab=sychronization] {} {M_{N} \bc \Box \;|\; x?M_{F} \;|\; x!M_{C}}
  \and
  \inferrule* [lab=abstraction] {} {{M_{F}} \bc (x)M_{P} }
  \and
  \inferrule* [lab=concretion] {} {{M_{C}} \bc \langle M_{P} \rangle }
  \and \\
  \inferrule* [lab=process] {} {{M_{P}} \bc M_{N} \;| \;P|M_{P} }
\end{mathpar}

\begin{definition}[contextual application] Given a context $M$, and
  process $P$, we define the \emph{contextual application}, $M[P] :=
  M\{P/\Box\}$. That is, the contextual application of M to P is the
  substitution of $P$ for $\Box$ in $M$.
\end{definition}

$\meaningof{-} : L \to \mathcal{P}(\pi)$

\begin{mathpar}
  \inferrule* [lab=collection] {} {\meaningof{true} = \pi, \and \meaningof{~E} = \pi \setminus \meaningof{E}, \and \meaningof{E_{1} \& E_{2}} = \meaningof{E_{1}} \cap \meaningof{E_{2}}}
\end{mathpar}

\begin{mathpar}
  \inferrule* [lab=structure] {} {\meaningof{0} = \{ P \in \pi | P \equiv 0 \}, \and \\ \meaningof{E_1 | E_2} = \{ P \in \pi | P \equiv P_{1} | P_{2}, P_{1} \in \meaningof{E_{1}}, P_{2} \in \meaningof{E_2}\} }
\end{mathpar}

\begin{mathpar}
 \inferrule* [lab=behavior] {} {\meaningof{\langle a?b \rangle E} = \{ P \in \pi | P \equiv Q | u?(y)P', \\ \and \\\\ \and \\ \;\;\; u \in \meaningof{a}, \forall z.P'\{z/y\} \in \meaningof{E\{z/b\}}\}, \and \\ \meaningof{a!E} = \{ P \in \pi | P \equiv Q | x!\langle P' \rangle, x \in \meaningof{a} P' \in \meaningof{E}\} }
\end{mathpar}

\begin{mathpar}
 \inferrule* [lab=nominal] {} {\meaningof{\quotep{E}} = \{ \quotep{P} \in \quotep{\pi} | P \in \meaningof{E} \}, \and \meaningof{\quotep{P}} = \{ \quotep{Q} \in \quotep{\pi} | P \equiv Q \} \and \\ \meaningof{@\quotep{E}} = \{ P \in \pi | P \equiv @x, x \in \meaningof{E} \}}
\end{mathpar}

\begin{eqnarray*}
  \\
  \meaningof{-} : TS \to ST
\end{eqnarray*}

\begin{eqnarray*}
  \\
  L : TS \to ST
\end{eqnarray*}

\begin{eqnarray*}
  \\
  P \models E \iff P \in \meaningof{E}
\end{eqnarray*}

\begin{eqnarray*}
  P \approx_{L} Q \iff \forall E \in L. P \models E \iff Q \models E
\end{eqnarray*}

\begin{eqnarray*}
  P \approx_{K} Q
\end{eqnarray*}

\begin{eqnarray*}
  P \approx Q
\end{eqnarray*}

$\approx_{K} = \approx = \approx_{L}$

\subsubsection{Contextual duality}

Note that contexts extend the quotation operation to a family of
operations from processes to names. Given a context, $M$, we can
define a \emph{nominal context}, $\quotep{M}$ by $\quotep{M}[P] :=
\quotep{M[P]}$. To foreshadow what is to come we observe that these
operations enjoy a duality with processes very much like the duality
between vectors and maps from vectors to scalars.

Further, because the calculus is essentially higher-order, we have a
correspondence between contexts and processes. More specifically,
given a name $x$ and a context $M$ we can construct $M^{*}_{x}$ such
that 

\begin{mathpar}
  M^{*}_{x} | \lift{x}{P} \red M[P]
\end{mathpar}

namely,

\begin{mathpar}
  M^{*}_{x} := x?(u).M[\dropn{u}]
\end{mathpar}

The dependence of $M^{*}_{x}$ on a name makes it an abstraction, 

\begin{mathpar}
  M^{*} := (x)x?(u).M[\dropn{u}]
\end{mathpar}

\subsection{Additional notation}

It will sometimes be convenient to denote the process a name
quotes. We already have the notation $x = \quotep{P}$, but it will be
convenient to introduce an alternate notation, $\procn{x}$, when we
want to emphasize the connection to the use of the name. Note that, by
virtue of name equivalence, $\quotep{\procn{x}} \nameeq x$; so, the
notation is consistent with previous definitions.

Further, because names have structure it is possible to effect
substitutions on the basis of that structure. This means we need to
upgrade our notation for substitutions, which we accomplish by
adapting comprehension notation. Thus,

\begin{mathpar}
  P\{ y / x : x \in S \}
\end{mathpar}

is interpreted to mean the process derived from P by replacing (in a
capture-avoiding manner) each occurrence of $x$ in $S$ by $y$. For example,

\begin{mathpar}
  P\{ \quotep{\procn{x}|\procn{x}} / x : x \in \freenames{P} \}
\end{mathpar}

will replace each (occurrence) of a free name $x$ in $P$ by
$\quotep{\procn{x}|\procn{x}}$.

Also, we will avail ourselves of the notation $x^{L}$ and $x^{R}$ to
denote injections of a name into disjoint copies of the name
space. There are numerous ways to accomplish this. One example can be
found in \cite{MeredithR05}. This notation overloads to vectors of
names: $\vec{x}^{\pi} := (x_{i}^{\pi} \; : \; 0 \leq i < |\vec{x}| )$ where $\pi \in \{L,R\}$.

We also use $P^{\Box} := P|\Box$.

In \cite{MeredithR05} an interpretation of the new operator is
given. It turns out that there are several possible interpretations
all enjoying the requisite algebraic properties of the operator (see
\cite{milner91polyadicpi}). We will therefore make liberal use of
$(\nu\; \vec{x})P$.

% subsection the_syntax_and_semantics_of_the_notation_system (end)   

\input{qm2pi.qmops} 

\input{qm2pi.sterngerlach} 

\input{qm2pi.metric} 

% section concurrent_process_calculi (end)

%\input{qm2pi.proofsketch}

% section proof sketch (end)

%\input{qm2pi.slviaknots} 

% section spatial logic via knots (end)

\input{qm2pi.conclusion}

% section conclusion (end)

%\input{qm2pi.dtcodes} 

% section wiring algorithm (end)

\input{qm2pi.ack} 

% section acknowledgments (end)

\newpage


\bibliographystyle{plain}   
\bibliography{../../biblios/main.bib}

\input{qm2pi.rhodetails}

\end{document}

 

% section wiring algorithm (end)

\documentclass[12pt]{llncs}
%\documentclass{jktr}

\usepackage[pdftex]{hyperref}                   
\usepackage {listings}
\usepackage {mathpartir}
\usepackage{bcprules}
%\usepackage{listings}
                       
\usepackage{graphicx} 
%\usepackage[margins=2.5cm,nohead,nofoot]{geometry}
%\usepackage{geometry}
\usepackage{amsfonts}
\usepackage{amstext}
\usepackage{latexsym}
\usepackage{amssymb}
\usepackage{color}


%\include{myPreamble}
\include{qm2pi.local} 

%\ifpdf
%\usepackage[pdftex]{graphicx}
%\else
%\usepackage{graphicx}
%\fi

 % \ifpdf
%  \usepackage{pdfsync}
%  \if


%\title{Brief Article}
%\author{David F. Snyder}
%\author{L.G. Meredith}

%\address{Dept. of Math., Texas State University--San Marcos, San Marcos, TX 78666}
       
\pagestyle{empty}


\begin{document}

\lstset{language=[Objective]Caml,frame=shadowbox}

\input{qm2pi.front}

% section front matter (end)

\input{qm2pi.intro} 
 
% section introduction (end)

% \input{qm2pi.knotations} 

% section notation (end)

\input{qm2pi.process.calculi} 

% section concurrent_process_calculi_and_spatial_logics_ (end)
    
%\input{qm2pi.knots2pi} 

%\input{qm2pi.trefoil} 

%\input{qm2pi.mainthm} 

% subsection basic_interpretation (end)

%\input{qm2pi.rho.presentation} 
\subsection{The syntax and semantics of the notation system}\label{sub:the_syntax_and_semantics_of_the_notation_system} % (fold)

We now summarize a technical presentation of the calculus that
embodies our theory of dynamics. The typical presentation of such a
calculus follows the style of giving generators and relations on
them. The grammar, below, describing term constructors, freely
generates the set of processes, $\Proc$. This set is then quotiented
by a relation known as structural congruence and it is over this set
that the notion of dynamics is expressed. This presentation is
essentially that of \cite{MeredithR05} with the addition of
polyadicity and summation. For readability we have relegated some of
the technical subtleties to an appendix.

\subsubsection{Process grammar}\label{subsub:process_grammar}

\begin{mathpar}
  \inferrule* [lab=synchronization] {} {{M} \bc \pzero \;|\; x?F \;|\; x!C }
  \and
  \inferrule* [lab=abstraction] {} {{F} \bc (x)P}
  \and
  \inferrule* [lab=concretion] {} {{C} \bc \langle Q \rangle}
  \and
  \inferrule* [lab=process] {} {{P,Q} \bc M \;| \;P|Q \;|\; @{x}}
  \and
  \inferrule* [lab=name] {} {{x} \bc \quotep{P}}
\end{mathpar} 

Note that $\vec{x}$ (resp. $\vec{P}$) denotes a vector of names
(resp. processes) of length $|\vec{x}|$ (resp. $|\vec{P}|$). We adopt
the following useful abbreviations.

\begin{mathpar}
   x?(\vec{y}).P := x.(\vec{y})P \and  x\clift{\vec{P}} := x.\clift{\vec{P}}
   \and x!(y) := \lift{x}{\dropn{y}}
   \and \Pi_{i=0}^{n-1}P_i := P_0 | \ldots | P_{n-1}
\end{mathpar}

\subsubsection{Structural congruence}

\paragraph{Free and bound names and alpha-equivalence.} At the
core of structural equivalence is alpha-equivalence which identifies
process that are the same up to a change of variable. Formally, we
recognize the distinction between free and bound names. The free names
of a process, $\freenames{P}$, may be calculated recursively as
follows:

\begin{mathpar}
\freenames{\pzero} := \emptyset
  \and \\
  \freenames{x?(y).P} := \{ x \} \cup (\freenames{P} \setminus \{ y \})
  \and 
  \freenames{x!\langle P \rangle} := \{ x \} \cup \{ P \} 
  \and \\
  \freenames{P|Q} := \freenames{P} \cup \freenames{Q}
  \and \\
  \freenames{@{x}} := \{ x \}
\end{mathpar}

$\pi$
$\quotep{\pi}$

$\freenames{-} : \pi \to \mathcal{P}(\quotep{\pi})$

\begin{eqnarray*}
  \freenames{\pzero} & := & \emptyset \\
  \freenames{x?(y).P} & := & \{ x \} \cup (\freenames{P} \setminus \{ y \}) \\
  \freenames{x!\langle P \rangle} & := & \{ x \} \cup \{ P \} \\
  \freenames{P|Q} & := & \freenames{P} \cup \freenames{Q} \\
  \freenames{\dropn{x}} & := & \{ x \}
\end{eqnarray*}

The bound names of a process, $\boundnames{P}$, are those names occurring in $P$
that are not free. For example, in $x?(y).0$, the name $x$ is free, while $y$ is bound.

\begin{mathpar}
  \inferrule* [lab=monoidal-laws] {} { P|Q \equiv Q|P \and P|0 \equiv P \and P|(Q|R) \equiv (P|Q)|R }
\end{mathpar}

\begin{mathpar}
  \inferrule* [lab=alpha-equivalence] {} { (x)P \equiv (y)P\{y/x\} \and y \not\in \freenames{P} }
\end{mathpar}

\begin{definition}
Then two processes, $P,Q$, are alpha-equivalent if $P = Q\{\vec{y}/\vec{x}\}$ for
some $\vec{x} \in \boundnames{Q},\vec{y} \in \boundnames{P}$, where $Q\{\vec{y}/\vec{x}\}$
denotes the capture-avoiding substitution of $\vec{y}$ for $\vec{x}$ in $Q$.
\end{definition}

\begin{definition}
  The {\em structural congruence} \cite{SangiorgiWalker} , $\equiv$,
  between processes is the least congruence containing
  alpha-equivalence, satisfying the abelian monoid laws
  (associativity, commutativity and $\pzero$ as identity) for parallel
  composition $|$ and for summation $+$.
\end{definition}

\subsection{Name equivalence}

We take name equivalence, written $\nameeq$, to be the smallest
equivalence relation generated by the following rules.

\begin{mathpar}
\inferrule*[lab=Quote-drop]
{ }
{ \quotep{@{x}} \nameeq x }

\inferrule*[lab=Struct-equiv]
{ P \scong Q }
{ \quotep{P} \nameeq \quotep{Q} }
\end{mathpar}

The astute reader will have noticed that the mutual recursion of names
and processes imposes a mutual recursion on alpha-equivalence and
structural equivalence via name-equivalence. Fortunately, all of this
works out pleasantly and we may calculate in the natural way, free of
concern. The reader interested in the details is referred to the
appendix \ref{appendix:rho_details}.

\subsection{Substitution}

We use $\Proc$ for the set of processes, $\QProc$ for the set of
names, and $\id{\{}\vec{y} / \vec{x} \id{\}}$ to denote partial maps,
$s : \QProc \rightarrow \QProc$. A map, $s$ lifts, uniquely, to a map
on process terms, $\widehat{s} : \Proc \rightarrow \Proc$ by the
following equations.

\begin{mathpar}
  (0) \psubstp{Q}{P} := 0 \\
  (R \juxtap S) \psubstp{Q}{P}
  :=    
  (R)\psubstp{Q}{P} \juxtap (S) \psubstp{Q}{P} \\
  (x?(y).R) \psubstp{Q}{P}    
  :=    
  (x)\substp{Q}{P} (z)\concat( (R \psubstn{z}{y}) \psubstp{Q}{P} ) \\
  (\lift{x}{R}) \psubstp{Q}{P}  
  :=
  \lift{(x)\substp{Q}{P}}{ R \psubstp{Q}{P} } \\
%   (\dropn{x})  \psubstp{Q}{P}       
%   := 
%   \left\{ 
%     \begin{array}{ccc} 
%       \dropn{\quotep{Q}} & & x \nameeq \quotep{P} \\
%       \dropn{x} & & otherwise \\
%     \end{array}
%   \right. 
  (\dropn{x})  \psubstp{Q}{P}       
  := 
  \left\{ 
    \begin{array}{ccc} 
      Q & & x \nameeq \quotep{P} \\
      \dropn{x} & & otherwise \\
    \end{array}
  \right.
\end{mathpar}
 

where

\begin{eqnarray}
  (x)\id{\{} \lpquote Q \rpquote / \lpquote P \rpquote \id{\}}            = 
  \left\{ 
    \begin{array}{ccc}
      \lpquote Q \rpquote & & x \nameeq \lpquote P \rpquote \\
      x & & otherwise \\
    \end{array}
  \right. \nonumber
\end{eqnarray}

and $z$ is chosen distinct from $\quotep{P}$, $\quotep{Q}$, the free
names in $Q$, and all the names in $R$. Our $\alpha$-equivalence will
be built in the standard way from this substitution.

\begin{remark}\label{rem:no_self_referential_names}
  One consequence of these definitions is that $\forall P. \quotep{P}
  \not\in \freenames{P}$.
\end{remark}

\subsection{ Dynamic quote: an example }

Anticipating something of what's to come, consider applying the
substitution, $\widehat{\id{\{}u / z \id{\}}}$, to the following pair
of processes, $\lift{w}{y!(z)}$ and $w[ \lpquote y!(z) \rpquote ]$.

\begin{eqnarray}
	\lift{w}{y!(z)}\widehat{\id{\{}u / z \id{\}}}
		& = &
		\lift{w}{y!(u)} \nonumber\\
	w[ \lpquote y!(z) \rpquote ] \widehat{ \id{\{}u / z \id{\}} }
		& = &
		w[ \lpquote y!(z) \rpquote ] \nonumber
\end{eqnarray}

Because the body of the process between quotes is impervious to
substitution, we get radically different answers. In fact, by
examining the first process in an input context,
e.g. $x?(z).\lift{w}{y!(z)}$, we see that the process under the lift
operator may be shaped by prefixed inputs binding a name inside it. In
this sense, the lift operator will be seen as a way to dynamically
construct processes before reifying them as names.

Finally equipped with these standard features we can present the
dynamics of the calculus.

\subsubsection{Operational semantics} 

Finally, we introduce the computational dynamics. What marks these
algebras as distinct from other more traditionally studied algebraic
structures, e.g. vector spaces or polynomial rings, is the manner in
which dynamics is captured. In traditional structures, dynamics is typically
expressed through morphisms between such structures, as in linear maps
between vector spaces or morphisms between rings. In algebras
associated with the semantics of computation, the dynamics is
expressed as part of the algebraic structure itself, through a
reduction reduction relation typically denoted by $\red$. Below, we
give a recursive presentation of this relation for the calculus used
in the encoding.

$\red \subseteq \pi \times \pi$
$\red : \pi \to \mathcal{P}(\pi)$

\begin{mathpar}
  \inferrule* [lab=Comm] { \textsf{match}( x_{src}, x_{trgt} ) } { x_{trgt}?(y)P \; | \; x_{src}!\langle {Q} \rangle \red P\{\quotep{Q}/y}\} }
  \and \\
  \inferrule* [lab=Par] {{P} \red {P}'} {{{P} | {Q}} \red {{P}' | {Q}}}
  \and
  \inferrule* [lab=Equiv]{{{P} \scong {P}'} \andalso {{P}' \red {Q}'} \andalso {{Q}' \scong {Q}}}{{P} \red {Q}}
\end{mathpar}

\begin{eqnarray*}
  match_{\equiv} (\quotep{P},\quotep{Q}) & := & P \equiv Q \\
  match_{\dagger}(\quotep{P},\quotep{Q}) & := & \forall R. P|Q \red^{*} R => R \red^{*} 0 \\
  match_{K}(\quotep{P},\quotep{Q}) & := & K \mbox{ for some context } K
\end{eqnarray*}

$u?(x)P | u!\langle Q \rangle \red P\{\quotep{Q}/x\}$

%We write $\wred$ for $\red^*$, and $P\red$ if $\exists Q $ such that $ P \red Q$.
We write $P\red$ if $\exists Q $ such that $ P \red Q$ and $P\not\red$, otherwise.

\section{Replication}

As mentioned before, it is known that replication (and hence
recursion) can be implemented in a higher-order process algebra
\cite{SangiorgiWalker}. As our first example of calculation with the
machinery thus far presented we give the construction explicitly in
the {\rhoc}.

\begin{eqnarray}
	D_{x} & := & \prefix{x}{y}{(\binpar{\outputp{x}{y}}{@{y}})} \nonumber\\
	\bangp_{x}{P} & := & \binpar{{x}!\langle{\binpar{D_{x}}{P}}\rangle}{D_{x}} \nonumber
\end{eqnarray}

\begin{eqnarray}
	\bangp_{x}{P} & & \nonumber\\
	=
	& {x}!\langle{(\prefix{x}{y}{(\outputp{x}{y} | @{y})) | P}}\rangle 
	      | \prefix{x}{y}{(\outputp{x}{y} | @{y})} & \nonumber\\
	\red
	& (\outputp{x}{y} | @{y})\substn{\quotep{(\prefix{x}{y}{(@{y} | \outputp{x}{y})) | P}}}{y} & \nonumber\\
	=
	& \outputp{x}{\quotep{(\prefix{x}{y}{(\outputp{x}{y} | @{y})) | P}}}
	  | {(\prefix{x}{y}{(\outputp{x}{y} | @{y})) | P}} & \nonumber\\
	\red
	& \ldots & \nonumber\\
	\red^*
	& P | P | \ldots & \nonumber
\end{eqnarray}

Of course, this encoding, as an implementation, runs away, unfolding
$\bangp{P}$ eagerly. A lazier and more implementable replication
operator, restricted to input-guarded processes, may be obtained as follows.

\begin{eqnarray}
\bangp{\prefix{u}{v}{P}} 
	:= 
	\binpar{\lift{x}{\prefix{u}{v}{(\binpar{D(x)}{P})}}}{D(x)} \nonumber
\end{eqnarray}

\begin{remark}
  Note that the lazier definition still does not deal with summation
  or mixed summation (i.e. sums over input and output). The reader is
  invited to construct definitions of replication that deal with these
  features. 

  Further, the definitions are parameterized in a name, $x$. Can you,
  gentle reader, make a definition that eliminates this parameter and
  guarantees no accidental interaction between the replication
  machinery and the process being replicated -- i.e. no accidental
  sharing of names used by the process to get its work done and the
  name(s) used by the replication to effect copying. This latter
  revision of the definition of replication is crucial to obtaining
  the expected identity $!!P \sim !P$.
\end{remark}

\begin{remark}\label{rem:paradoxical_combinator}
  The reader familiar with the lambda calculus will have noticed the
  similarity between $D$ and the paradoxical combinator.

  [Ed. note: the existence of this seems to suggest we have to be more
  restrictive on the set of processes and names we admit if we are to
  support no-cloning.]
\end{remark}

\subsubsection{Bisimulation}

The computational dynamics gives rise to another kind of equivalence,
the equivalence of computational behavior. As previously mentioned
this is typically captured \emph{via} some form of bisimulation.

% The notion we use in this paper is weak barbed bisimulation
% \cite{milner91polyadicpi}.

The notion we use in this paper is derived from weak barbed
bisimulation \cite{milner91polyadicpi}. 

\begin{definition}
An \emph{observation relation}, $\downarrow_{\mathcal N}$, over a set
of names, $\mathcal N$, is the smallest relation satisfying the rules
below.

\infrule[Out-barb]{y \in {\mathcal N}, \; x \nameeq y}
		  {\outputp{x}{v} \downarrow_{\mathcal N} x}
\infrule[Par-barb]{\mbox{$P\downarrow_{\mathcal N} x$ or $Q\downarrow_{\mathcal N} x$}}
		  {\binpar{P}{Q} \downarrow_{\mathcal N} x}

We write $P \Downarrow_{\mathcal N} x$ if there is $Q$ such that 
$P \wred Q$ and $Q \downarrow_{\mathcal N} x$.
\end{definition}

\begin{definition}
%\label{def.bbisim}
An  ${\mathcal N}$-\emph{barbed bisimulation} over a set of names, ${\mathcal N}$, is a symmetric binary relation 
${\mathcal S}_{\mathcal N}$ between agents such that $P\rel{S}_{\mathcal N}Q$ implies:
\begin{enumerate}
\item If $P \red P'$ then $Q \wred Q'$ and $P'\rel{S}_{\mathcal N} Q'$.
\item If $P\downarrow_{\mathcal N} x$, then $Q\Downarrow_{\mathcal N} x$.
\end{enumerate}
$P$ is ${\mathcal N}$-barbed bisimilar to $Q$, written
$P \wbbisim_{\mathcal N} Q$, if $P \rel{S}_{\mathcal N} Q$ for some ${\mathcal N}$-barbed bisimulation ${\mathcal S}_{\mathcal N}$.
\end{definition}

$\mathcal{R} \subseteq \pi \times \pi$

$P \mathcal{R} Q => \forall P'. P \red P' \Rightarrow \exists Q'. Q \red Q', P' \mathcal{R} Q'$

$P \vdash x \Rightarrow Q \vdash x$

\begin{mathpar}
  \inferrule*[lab=Out-barb]{x \nameeq y}{{y}!\langle{Q}\rangle \vdash x}
  \and
  \inferrule*[lab=Par-barb]{\mbox{$P\vdash x$ or $Q\vdash x$}}{\binpar{P}{Q} \vdash x}
\end{mathpar}

\subsubsection{Contexts}

One of the principle advantages of computational calculi like the
$\pi$-calculus is a well-defined notion of context,
contextual-equivalence and a correlation between
contextual-equivalence and notions of bisimulation. The notion of
context allows the decomposition of a process into (sub-)process and
its syntactic environment, its context. Thus, a context may be
thought of as a process with a ``hole'' (written $\Box$) in it. The
application of a context $M$ to a process $P$, written $M[P]$, is
tantamount to filling the hole in $M$ with $P$. In this paper we do
not need the full weight of this theory, but do make use of the notion
of context in the proof the main theorem. 

\begin{mathpar}
  \inferrule* [lab=summation] {} {{M_{M},M_{N}} \bc \Box \;|\; x.M_{A} \;|\; M_{M}+M_{N}}
  \and
  \inferrule* [lab=agent] {} {{M_{A}} \bc (\vec{x})M_{P} \;| \; \clift{P_0,\ldots,M_{P},\ldots,P_N}}
  \and \\
  \inferrule* [lab=process] {} {{M_{P}} \bc M_{N} \;| \;P|M_{P} }
\end{mathpar} 

\begin{mathpar}
  \inferrule* [lab=sychronization] {} {M_{N} \bc \Box \;|\; x?M_{F} \;|\; x!M_{C}}
  \and
  \inferrule* [lab=abstraction] {} {{M_{F}} \bc (x)M_{P} }
  \and
  \inferrule* [lab=concretion] {} {{M_{C}} \bc \langle M_{P} \rangle }
  \and \\
  \inferrule* [lab=process] {} {{M_{P}} \bc M_{N} \;| \;P|M_{P} }
\end{mathpar}

\begin{definition}[contextual application] Given a context $M$, and
  process $P$, we define the \emph{contextual application}, $M[P] :=
  M\{P/\Box\}$. That is, the contextual application of M to P is the
  substitution of $P$ for $\Box$ in $M$.
\end{definition}

$\meaningof{-} : L \to \mathcal{P}(\pi)$

\begin{mathpar}
  \inferrule* [lab=collection] {} {\meaningof{true} = \pi, \and \meaningof{~E} = \pi \setminus \meaningof{E}, \and \meaningof{E_{1} \& E_{2}} = \meaningof{E_{1}} \cap \meaningof{E_{2}}}
\end{mathpar}

\begin{mathpar}
  \inferrule* [lab=structure] {} {\meaningof{0} = \{ P \in \pi | P \equiv 0 \}, \and \\ \meaningof{E_1 | E_2} = \{ P \in \pi | P \equiv P_{1} | P_{2}, P_{1} \in \meaningof{E_{1}}, P_{2} \in \meaningof{E_2}\} }
\end{mathpar}

\begin{mathpar}
 \inferrule* [lab=behavior] {} {\meaningof{\langle a?b \rangle E} = \{ P \in \pi | P \equiv Q | u?(y)P', \\ \and \\\\ \and \\ \;\;\; u \in \meaningof{a}, \forall z.P'\{z/y\} \in \meaningof{E\{z/b\}}\}, \and \\ \meaningof{a!E} = \{ P \in \pi | P \equiv Q | x!\langle P' \rangle, x \in \meaningof{a} P' \in \meaningof{E}\} }
\end{mathpar}

\begin{mathpar}
 \inferrule* [lab=nominal] {} {\meaningof{\quotep{E}} = \{ \quotep{P} \in \quotep{\pi} | P \in \meaningof{E} \}, \and \meaningof{\quotep{P}} = \{ \quotep{Q} \in \quotep{\pi} | P \equiv Q \} \and \\ \meaningof{@\quotep{E}} = \{ P \in \pi | P \equiv @x, x \in \meaningof{E} \}}
\end{mathpar}

\begin{eqnarray*}
  \\
  \meaningof{-} : TS \to ST
\end{eqnarray*}

\begin{eqnarray*}
  \\
  L : TS \to ST
\end{eqnarray*}

\begin{eqnarray*}
  \\
  P \models E \iff P \in \meaningof{E}
\end{eqnarray*}

\begin{eqnarray*}
  P \approx_{L} Q \iff \forall E \in L. P \models E \iff Q \models E
\end{eqnarray*}

\begin{eqnarray*}
  P \approx_{K} Q
\end{eqnarray*}

\begin{eqnarray*}
  P \approx Q
\end{eqnarray*}

$\approx_{K} = \approx = \approx_{L}$

\subsubsection{Contextual duality}

Note that contexts extend the quotation operation to a family of
operations from processes to names. Given a context, $M$, we can
define a \emph{nominal context}, $\quotep{M}$ by $\quotep{M}[P] :=
\quotep{M[P]}$. To foreshadow what is to come we observe that these
operations enjoy a duality with processes very much like the duality
between vectors and maps from vectors to scalars.

Further, because the calculus is essentially higher-order, we have a
correspondence between contexts and processes. More specifically,
given a name $x$ and a context $M$ we can construct $M^{*}_{x}$ such
that 

\begin{mathpar}
  M^{*}_{x} | \lift{x}{P} \red M[P]
\end{mathpar}

namely,

\begin{mathpar}
  M^{*}_{x} := x?(u).M[\dropn{u}]
\end{mathpar}

The dependence of $M^{*}_{x}$ on a name makes it an abstraction, 

\begin{mathpar}
  M^{*} := (x)x?(u).M[\dropn{u}]
\end{mathpar}

\subsection{Additional notation}

It will sometimes be convenient to denote the process a name
quotes. We already have the notation $x = \quotep{P}$, but it will be
convenient to introduce an alternate notation, $\procn{x}$, when we
want to emphasize the connection to the use of the name. Note that, by
virtue of name equivalence, $\quotep{\procn{x}} \nameeq x$; so, the
notation is consistent with previous definitions.

Further, because names have structure it is possible to effect
substitutions on the basis of that structure. This means we need to
upgrade our notation for substitutions, which we accomplish by
adapting comprehension notation. Thus,

\begin{mathpar}
  P\{ y / x : x \in S \}
\end{mathpar}

is interpreted to mean the process derived from P by replacing (in a
capture-avoiding manner) each occurrence of $x$ in $S$ by $y$. For example,

\begin{mathpar}
  P\{ \quotep{\procn{x}|\procn{x}} / x : x \in \freenames{P} \}
\end{mathpar}

will replace each (occurrence) of a free name $x$ in $P$ by
$\quotep{\procn{x}|\procn{x}}$.

Also, we will avail ourselves of the notation $x^{L}$ and $x^{R}$ to
denote injections of a name into disjoint copies of the name
space. There are numerous ways to accomplish this. One example can be
found in \cite{MeredithR05}. This notation overloads to vectors of
names: $\vec{x}^{\pi} := (x_{i}^{\pi} \; : \; 0 \leq i < |\vec{x}| )$ where $\pi \in \{L,R\}$.

We also use $P^{\Box} := P|\Box$.

In \cite{MeredithR05} an interpretation of the new operator is
given. It turns out that there are several possible interpretations
all enjoying the requisite algebraic properties of the operator (see
\cite{milner91polyadicpi}). We will therefore make liberal use of
$(\nu\; \vec{x})P$.

% subsection the_syntax_and_semantics_of_the_notation_system (end)   

\input{qm2pi.qmops} 

\input{qm2pi.sterngerlach} 

\input{qm2pi.metric} 

% section concurrent_process_calculi (end)

%\input{qm2pi.proofsketch}

% section proof sketch (end)

%\input{qm2pi.slviaknots} 

% section spatial logic via knots (end)

\input{qm2pi.conclusion}

% section conclusion (end)

%\input{qm2pi.dtcodes} 

% section wiring algorithm (end)

\input{qm2pi.ack} 

% section acknowledgments (end)

\newpage


\bibliographystyle{plain}   
\bibliography{../../biblios/main.bib}

\input{qm2pi.rhodetails}

\end{document}

 

% section acknowledgments (end)

\newpage


\bibliographystyle{plain}   
\bibliography{../../biblios/main.bib}

\documentclass[12pt]{llncs}
%\documentclass{jktr}

\usepackage[pdftex]{hyperref}                   
\usepackage {listings}
\usepackage {mathpartir}
\usepackage{bcprules}
%\usepackage{listings}
                       
\usepackage{graphicx} 
%\usepackage[margins=2.5cm,nohead,nofoot]{geometry}
%\usepackage{geometry}
\usepackage{amsfonts}
\usepackage{amstext}
\usepackage{latexsym}
\usepackage{amssymb}
\usepackage{color}


%\include{myPreamble}
\include{qm2pi.local} 

%\ifpdf
%\usepackage[pdftex]{graphicx}
%\else
%\usepackage{graphicx}
%\fi

 % \ifpdf
%  \usepackage{pdfsync}
%  \if


%\title{Brief Article}
%\author{David F. Snyder}
%\author{L.G. Meredith}

%\address{Dept. of Math., Texas State University--San Marcos, San Marcos, TX 78666}
       
\pagestyle{empty}


\begin{document}

\lstset{language=[Objective]Caml,frame=shadowbox}

\input{qm2pi.front}

% section front matter (end)

\input{qm2pi.intro} 
 
% section introduction (end)

% \input{qm2pi.knotations} 

% section notation (end)

\input{qm2pi.process.calculi} 

% section concurrent_process_calculi_and_spatial_logics_ (end)
    
%\input{qm2pi.knots2pi} 

%\input{qm2pi.trefoil} 

%\input{qm2pi.mainthm} 

% subsection basic_interpretation (end)

%\input{qm2pi.rho.presentation} 
\subsection{The syntax and semantics of the notation system}\label{sub:the_syntax_and_semantics_of_the_notation_system} % (fold)

We now summarize a technical presentation of the calculus that
embodies our theory of dynamics. The typical presentation of such a
calculus follows the style of giving generators and relations on
them. The grammar, below, describing term constructors, freely
generates the set of processes, $\Proc$. This set is then quotiented
by a relation known as structural congruence and it is over this set
that the notion of dynamics is expressed. This presentation is
essentially that of \cite{MeredithR05} with the addition of
polyadicity and summation. For readability we have relegated some of
the technical subtleties to an appendix.

\subsubsection{Process grammar}\label{subsub:process_grammar}

\begin{mathpar}
  \inferrule* [lab=synchronization] {} {{M} \bc \pzero \;|\; x?F \;|\; x!C }
  \and
  \inferrule* [lab=abstraction] {} {{F} \bc (x)P}
  \and
  \inferrule* [lab=concretion] {} {{C} \bc \langle Q \rangle}
  \and
  \inferrule* [lab=process] {} {{P,Q} \bc M \;| \;P|Q \;|\; @{x}}
  \and
  \inferrule* [lab=name] {} {{x} \bc \quotep{P}}
\end{mathpar} 

Note that $\vec{x}$ (resp. $\vec{P}$) denotes a vector of names
(resp. processes) of length $|\vec{x}|$ (resp. $|\vec{P}|$). We adopt
the following useful abbreviations.

\begin{mathpar}
   x?(\vec{y}).P := x.(\vec{y})P \and  x\clift{\vec{P}} := x.\clift{\vec{P}}
   \and x!(y) := \lift{x}{\dropn{y}}
   \and \Pi_{i=0}^{n-1}P_i := P_0 | \ldots | P_{n-1}
\end{mathpar}

\subsubsection{Structural congruence}

\paragraph{Free and bound names and alpha-equivalence.} At the
core of structural equivalence is alpha-equivalence which identifies
process that are the same up to a change of variable. Formally, we
recognize the distinction between free and bound names. The free names
of a process, $\freenames{P}$, may be calculated recursively as
follows:

\begin{mathpar}
\freenames{\pzero} := \emptyset
  \and \\
  \freenames{x?(y).P} := \{ x \} \cup (\freenames{P} \setminus \{ y \})
  \and 
  \freenames{x!\langle P \rangle} := \{ x \} \cup \{ P \} 
  \and \\
  \freenames{P|Q} := \freenames{P} \cup \freenames{Q}
  \and \\
  \freenames{@{x}} := \{ x \}
\end{mathpar}

$\pi$
$\quotep{\pi}$

$\freenames{-} : \pi \to \mathcal{P}(\quotep{\pi})$

\begin{eqnarray*}
  \freenames{\pzero} & := & \emptyset \\
  \freenames{x?(y).P} & := & \{ x \} \cup (\freenames{P} \setminus \{ y \}) \\
  \freenames{x!\langle P \rangle} & := & \{ x \} \cup \{ P \} \\
  \freenames{P|Q} & := & \freenames{P} \cup \freenames{Q} \\
  \freenames{\dropn{x}} & := & \{ x \}
\end{eqnarray*}

The bound names of a process, $\boundnames{P}$, are those names occurring in $P$
that are not free. For example, in $x?(y).0$, the name $x$ is free, while $y$ is bound.

\begin{mathpar}
  \inferrule* [lab=monoidal-laws] {} { P|Q \equiv Q|P \and P|0 \equiv P \and P|(Q|R) \equiv (P|Q)|R }
\end{mathpar}

\begin{mathpar}
  \inferrule* [lab=alpha-equivalence] {} { (x)P \equiv (y)P\{y/x\} \and y \not\in \freenames{P} }
\end{mathpar}

\begin{definition}
Then two processes, $P,Q$, are alpha-equivalent if $P = Q\{\vec{y}/\vec{x}\}$ for
some $\vec{x} \in \boundnames{Q},\vec{y} \in \boundnames{P}$, where $Q\{\vec{y}/\vec{x}\}$
denotes the capture-avoiding substitution of $\vec{y}$ for $\vec{x}$ in $Q$.
\end{definition}

\begin{definition}
  The {\em structural congruence} \cite{SangiorgiWalker} , $\equiv$,
  between processes is the least congruence containing
  alpha-equivalence, satisfying the abelian monoid laws
  (associativity, commutativity and $\pzero$ as identity) for parallel
  composition $|$ and for summation $+$.
\end{definition}

\subsection{Name equivalence}

We take name equivalence, written $\nameeq$, to be the smallest
equivalence relation generated by the following rules.

\begin{mathpar}
\inferrule*[lab=Quote-drop]
{ }
{ \quotep{@{x}} \nameeq x }

\inferrule*[lab=Struct-equiv]
{ P \scong Q }
{ \quotep{P} \nameeq \quotep{Q} }
\end{mathpar}

The astute reader will have noticed that the mutual recursion of names
and processes imposes a mutual recursion on alpha-equivalence and
structural equivalence via name-equivalence. Fortunately, all of this
works out pleasantly and we may calculate in the natural way, free of
concern. The reader interested in the details is referred to the
appendix \ref{appendix:rho_details}.

\subsection{Substitution}

We use $\Proc$ for the set of processes, $\QProc$ for the set of
names, and $\id{\{}\vec{y} / \vec{x} \id{\}}$ to denote partial maps,
$s : \QProc \rightarrow \QProc$. A map, $s$ lifts, uniquely, to a map
on process terms, $\widehat{s} : \Proc \rightarrow \Proc$ by the
following equations.

\begin{mathpar}
  (0) \psubstp{Q}{P} := 0 \\
  (R \juxtap S) \psubstp{Q}{P}
  :=    
  (R)\psubstp{Q}{P} \juxtap (S) \psubstp{Q}{P} \\
  (x?(y).R) \psubstp{Q}{P}    
  :=    
  (x)\substp{Q}{P} (z)\concat( (R \psubstn{z}{y}) \psubstp{Q}{P} ) \\
  (\lift{x}{R}) \psubstp{Q}{P}  
  :=
  \lift{(x)\substp{Q}{P}}{ R \psubstp{Q}{P} } \\
%   (\dropn{x})  \psubstp{Q}{P}       
%   := 
%   \left\{ 
%     \begin{array}{ccc} 
%       \dropn{\quotep{Q}} & & x \nameeq \quotep{P} \\
%       \dropn{x} & & otherwise \\
%     \end{array}
%   \right. 
  (\dropn{x})  \psubstp{Q}{P}       
  := 
  \left\{ 
    \begin{array}{ccc} 
      Q & & x \nameeq \quotep{P} \\
      \dropn{x} & & otherwise \\
    \end{array}
  \right.
\end{mathpar}
 

where

\begin{eqnarray}
  (x)\id{\{} \lpquote Q \rpquote / \lpquote P \rpquote \id{\}}            = 
  \left\{ 
    \begin{array}{ccc}
      \lpquote Q \rpquote & & x \nameeq \lpquote P \rpquote \\
      x & & otherwise \\
    \end{array}
  \right. \nonumber
\end{eqnarray}

and $z$ is chosen distinct from $\quotep{P}$, $\quotep{Q}$, the free
names in $Q$, and all the names in $R$. Our $\alpha$-equivalence will
be built in the standard way from this substitution.

\begin{remark}\label{rem:no_self_referential_names}
  One consequence of these definitions is that $\forall P. \quotep{P}
  \not\in \freenames{P}$.
\end{remark}

\subsection{ Dynamic quote: an example }

Anticipating something of what's to come, consider applying the
substitution, $\widehat{\id{\{}u / z \id{\}}}$, to the following pair
of processes, $\lift{w}{y!(z)}$ and $w[ \lpquote y!(z) \rpquote ]$.

\begin{eqnarray}
	\lift{w}{y!(z)}\widehat{\id{\{}u / z \id{\}}}
		& = &
		\lift{w}{y!(u)} \nonumber\\
	w[ \lpquote y!(z) \rpquote ] \widehat{ \id{\{}u / z \id{\}} }
		& = &
		w[ \lpquote y!(z) \rpquote ] \nonumber
\end{eqnarray}

Because the body of the process between quotes is impervious to
substitution, we get radically different answers. In fact, by
examining the first process in an input context,
e.g. $x?(z).\lift{w}{y!(z)}$, we see that the process under the lift
operator may be shaped by prefixed inputs binding a name inside it. In
this sense, the lift operator will be seen as a way to dynamically
construct processes before reifying them as names.

Finally equipped with these standard features we can present the
dynamics of the calculus.

\subsubsection{Operational semantics} 

Finally, we introduce the computational dynamics. What marks these
algebras as distinct from other more traditionally studied algebraic
structures, e.g. vector spaces or polynomial rings, is the manner in
which dynamics is captured. In traditional structures, dynamics is typically
expressed through morphisms between such structures, as in linear maps
between vector spaces or morphisms between rings. In algebras
associated with the semantics of computation, the dynamics is
expressed as part of the algebraic structure itself, through a
reduction reduction relation typically denoted by $\red$. Below, we
give a recursive presentation of this relation for the calculus used
in the encoding.

$\red \subseteq \pi \times \pi$
$\red : \pi \to \mathcal{P}(\pi)$

\begin{mathpar}
  \inferrule* [lab=Comm] { \textsf{match}( x_{src}, x_{trgt} ) } { x_{trgt}?(y)P \; | \; x_{src}!\langle {Q} \rangle \red P\{\quotep{Q}/y}\} }
  \and \\
  \inferrule* [lab=Par] {{P} \red {P}'} {{{P} | {Q}} \red {{P}' | {Q}}}
  \and
  \inferrule* [lab=Equiv]{{{P} \scong {P}'} \andalso {{P}' \red {Q}'} \andalso {{Q}' \scong {Q}}}{{P} \red {Q}}
\end{mathpar}

\begin{eqnarray*}
  match_{\equiv} (\quotep{P},\quotep{Q}) & := & P \equiv Q \\
  match_{\dagger}(\quotep{P},\quotep{Q}) & := & \forall R. P|Q \red^{*} R => R \red^{*} 0 \\
  match_{K}(\quotep{P},\quotep{Q}) & := & K \mbox{ for some context } K
\end{eqnarray*}

$u?(x)P | u!\langle Q \rangle \red P\{\quotep{Q}/x\}$

%We write $\wred$ for $\red^*$, and $P\red$ if $\exists Q $ such that $ P \red Q$.
We write $P\red$ if $\exists Q $ such that $ P \red Q$ and $P\not\red$, otherwise.

\section{Replication}

As mentioned before, it is known that replication (and hence
recursion) can be implemented in a higher-order process algebra
\cite{SangiorgiWalker}. As our first example of calculation with the
machinery thus far presented we give the construction explicitly in
the {\rhoc}.

\begin{eqnarray}
	D_{x} & := & \prefix{x}{y}{(\binpar{\outputp{x}{y}}{@{y}})} \nonumber\\
	\bangp_{x}{P} & := & \binpar{{x}!\langle{\binpar{D_{x}}{P}}\rangle}{D_{x}} \nonumber
\end{eqnarray}

\begin{eqnarray}
	\bangp_{x}{P} & & \nonumber\\
	=
	& {x}!\langle{(\prefix{x}{y}{(\outputp{x}{y} | @{y})) | P}}\rangle 
	      | \prefix{x}{y}{(\outputp{x}{y} | @{y})} & \nonumber\\
	\red
	& (\outputp{x}{y} | @{y})\substn{\quotep{(\prefix{x}{y}{(@{y} | \outputp{x}{y})) | P}}}{y} & \nonumber\\
	=
	& \outputp{x}{\quotep{(\prefix{x}{y}{(\outputp{x}{y} | @{y})) | P}}}
	  | {(\prefix{x}{y}{(\outputp{x}{y} | @{y})) | P}} & \nonumber\\
	\red
	& \ldots & \nonumber\\
	\red^*
	& P | P | \ldots & \nonumber
\end{eqnarray}

Of course, this encoding, as an implementation, runs away, unfolding
$\bangp{P}$ eagerly. A lazier and more implementable replication
operator, restricted to input-guarded processes, may be obtained as follows.

\begin{eqnarray}
\bangp{\prefix{u}{v}{P}} 
	:= 
	\binpar{\lift{x}{\prefix{u}{v}{(\binpar{D(x)}{P})}}}{D(x)} \nonumber
\end{eqnarray}

\begin{remark}
  Note that the lazier definition still does not deal with summation
  or mixed summation (i.e. sums over input and output). The reader is
  invited to construct definitions of replication that deal with these
  features. 

  Further, the definitions are parameterized in a name, $x$. Can you,
  gentle reader, make a definition that eliminates this parameter and
  guarantees no accidental interaction between the replication
  machinery and the process being replicated -- i.e. no accidental
  sharing of names used by the process to get its work done and the
  name(s) used by the replication to effect copying. This latter
  revision of the definition of replication is crucial to obtaining
  the expected identity $!!P \sim !P$.
\end{remark}

\begin{remark}\label{rem:paradoxical_combinator}
  The reader familiar with the lambda calculus will have noticed the
  similarity between $D$ and the paradoxical combinator.

  [Ed. note: the existence of this seems to suggest we have to be more
  restrictive on the set of processes and names we admit if we are to
  support no-cloning.]
\end{remark}

\subsubsection{Bisimulation}

The computational dynamics gives rise to another kind of equivalence,
the equivalence of computational behavior. As previously mentioned
this is typically captured \emph{via} some form of bisimulation.

% The notion we use in this paper is weak barbed bisimulation
% \cite{milner91polyadicpi}.

The notion we use in this paper is derived from weak barbed
bisimulation \cite{milner91polyadicpi}. 

\begin{definition}
An \emph{observation relation}, $\downarrow_{\mathcal N}$, over a set
of names, $\mathcal N$, is the smallest relation satisfying the rules
below.

\infrule[Out-barb]{y \in {\mathcal N}, \; x \nameeq y}
		  {\outputp{x}{v} \downarrow_{\mathcal N} x}
\infrule[Par-barb]{\mbox{$P\downarrow_{\mathcal N} x$ or $Q\downarrow_{\mathcal N} x$}}
		  {\binpar{P}{Q} \downarrow_{\mathcal N} x}

We write $P \Downarrow_{\mathcal N} x$ if there is $Q$ such that 
$P \wred Q$ and $Q \downarrow_{\mathcal N} x$.
\end{definition}

\begin{definition}
%\label{def.bbisim}
An  ${\mathcal N}$-\emph{barbed bisimulation} over a set of names, ${\mathcal N}$, is a symmetric binary relation 
${\mathcal S}_{\mathcal N}$ between agents such that $P\rel{S}_{\mathcal N}Q$ implies:
\begin{enumerate}
\item If $P \red P'$ then $Q \wred Q'$ and $P'\rel{S}_{\mathcal N} Q'$.
\item If $P\downarrow_{\mathcal N} x$, then $Q\Downarrow_{\mathcal N} x$.
\end{enumerate}
$P$ is ${\mathcal N}$-barbed bisimilar to $Q$, written
$P \wbbisim_{\mathcal N} Q$, if $P \rel{S}_{\mathcal N} Q$ for some ${\mathcal N}$-barbed bisimulation ${\mathcal S}_{\mathcal N}$.
\end{definition}

$\mathcal{R} \subseteq \pi \times \pi$

$P \mathcal{R} Q => \forall P'. P \red P' \Rightarrow \exists Q'. Q \red Q', P' \mathcal{R} Q'$

$P \vdash x \Rightarrow Q \vdash x$

\begin{mathpar}
  \inferrule*[lab=Out-barb]{x \nameeq y}{{y}!\langle{Q}\rangle \vdash x}
  \and
  \inferrule*[lab=Par-barb]{\mbox{$P\vdash x$ or $Q\vdash x$}}{\binpar{P}{Q} \vdash x}
\end{mathpar}

\subsubsection{Contexts}

One of the principle advantages of computational calculi like the
$\pi$-calculus is a well-defined notion of context,
contextual-equivalence and a correlation between
contextual-equivalence and notions of bisimulation. The notion of
context allows the decomposition of a process into (sub-)process and
its syntactic environment, its context. Thus, a context may be
thought of as a process with a ``hole'' (written $\Box$) in it. The
application of a context $M$ to a process $P$, written $M[P]$, is
tantamount to filling the hole in $M$ with $P$. In this paper we do
not need the full weight of this theory, but do make use of the notion
of context in the proof the main theorem. 

\begin{mathpar}
  \inferrule* [lab=summation] {} {{M_{M},M_{N}} \bc \Box \;|\; x.M_{A} \;|\; M_{M}+M_{N}}
  \and
  \inferrule* [lab=agent] {} {{M_{A}} \bc (\vec{x})M_{P} \;| \; \clift{P_0,\ldots,M_{P},\ldots,P_N}}
  \and \\
  \inferrule* [lab=process] {} {{M_{P}} \bc M_{N} \;| \;P|M_{P} }
\end{mathpar} 

\begin{mathpar}
  \inferrule* [lab=sychronization] {} {M_{N} \bc \Box \;|\; x?M_{F} \;|\; x!M_{C}}
  \and
  \inferrule* [lab=abstraction] {} {{M_{F}} \bc (x)M_{P} }
  \and
  \inferrule* [lab=concretion] {} {{M_{C}} \bc \langle M_{P} \rangle }
  \and \\
  \inferrule* [lab=process] {} {{M_{P}} \bc M_{N} \;| \;P|M_{P} }
\end{mathpar}

\begin{definition}[contextual application] Given a context $M$, and
  process $P$, we define the \emph{contextual application}, $M[P] :=
  M\{P/\Box\}$. That is, the contextual application of M to P is the
  substitution of $P$ for $\Box$ in $M$.
\end{definition}

$\meaningof{-} : L \to \mathcal{P}(\pi)$

\begin{mathpar}
  \inferrule* [lab=collection] {} {\meaningof{true} = \pi, \and \meaningof{~E} = \pi \setminus \meaningof{E}, \and \meaningof{E_{1} \& E_{2}} = \meaningof{E_{1}} \cap \meaningof{E_{2}}}
\end{mathpar}

\begin{mathpar}
  \inferrule* [lab=structure] {} {\meaningof{0} = \{ P \in \pi | P \equiv 0 \}, \and \\ \meaningof{E_1 | E_2} = \{ P \in \pi | P \equiv P_{1} | P_{2}, P_{1} \in \meaningof{E_{1}}, P_{2} \in \meaningof{E_2}\} }
\end{mathpar}

\begin{mathpar}
 \inferrule* [lab=behavior] {} {\meaningof{\langle a?b \rangle E} = \{ P \in \pi | P \equiv Q | u?(y)P', \\ \and \\\\ \and \\ \;\;\; u \in \meaningof{a}, \forall z.P'\{z/y\} \in \meaningof{E\{z/b\}}\}, \and \\ \meaningof{a!E} = \{ P \in \pi | P \equiv Q | x!\langle P' \rangle, x \in \meaningof{a} P' \in \meaningof{E}\} }
\end{mathpar}

\begin{mathpar}
 \inferrule* [lab=nominal] {} {\meaningof{\quotep{E}} = \{ \quotep{P} \in \quotep{\pi} | P \in \meaningof{E} \}, \and \meaningof{\quotep{P}} = \{ \quotep{Q} \in \quotep{\pi} | P \equiv Q \} \and \\ \meaningof{@\quotep{E}} = \{ P \in \pi | P \equiv @x, x \in \meaningof{E} \}}
\end{mathpar}

\begin{eqnarray*}
  \\
  \meaningof{-} : TS \to ST
\end{eqnarray*}

\begin{eqnarray*}
  \\
  L : TS \to ST
\end{eqnarray*}

\begin{eqnarray*}
  \\
  P \models E \iff P \in \meaningof{E}
\end{eqnarray*}

\begin{eqnarray*}
  P \approx_{L} Q \iff \forall E \in L. P \models E \iff Q \models E
\end{eqnarray*}

\begin{eqnarray*}
  P \approx_{K} Q
\end{eqnarray*}

\begin{eqnarray*}
  P \approx Q
\end{eqnarray*}

$\approx_{K} = \approx = \approx_{L}$

\subsubsection{Contextual duality}

Note that contexts extend the quotation operation to a family of
operations from processes to names. Given a context, $M$, we can
define a \emph{nominal context}, $\quotep{M}$ by $\quotep{M}[P] :=
\quotep{M[P]}$. To foreshadow what is to come we observe that these
operations enjoy a duality with processes very much like the duality
between vectors and maps from vectors to scalars.

Further, because the calculus is essentially higher-order, we have a
correspondence between contexts and processes. More specifically,
given a name $x$ and a context $M$ we can construct $M^{*}_{x}$ such
that 

\begin{mathpar}
  M^{*}_{x} | \lift{x}{P} \red M[P]
\end{mathpar}

namely,

\begin{mathpar}
  M^{*}_{x} := x?(u).M[\dropn{u}]
\end{mathpar}

The dependence of $M^{*}_{x}$ on a name makes it an abstraction, 

\begin{mathpar}
  M^{*} := (x)x?(u).M[\dropn{u}]
\end{mathpar}

\subsection{Additional notation}

It will sometimes be convenient to denote the process a name
quotes. We already have the notation $x = \quotep{P}$, but it will be
convenient to introduce an alternate notation, $\procn{x}$, when we
want to emphasize the connection to the use of the name. Note that, by
virtue of name equivalence, $\quotep{\procn{x}} \nameeq x$; so, the
notation is consistent with previous definitions.

Further, because names have structure it is possible to effect
substitutions on the basis of that structure. This means we need to
upgrade our notation for substitutions, which we accomplish by
adapting comprehension notation. Thus,

\begin{mathpar}
  P\{ y / x : x \in S \}
\end{mathpar}

is interpreted to mean the process derived from P by replacing (in a
capture-avoiding manner) each occurrence of $x$ in $S$ by $y$. For example,

\begin{mathpar}
  P\{ \quotep{\procn{x}|\procn{x}} / x : x \in \freenames{P} \}
\end{mathpar}

will replace each (occurrence) of a free name $x$ in $P$ by
$\quotep{\procn{x}|\procn{x}}$.

Also, we will avail ourselves of the notation $x^{L}$ and $x^{R}$ to
denote injections of a name into disjoint copies of the name
space. There are numerous ways to accomplish this. One example can be
found in \cite{MeredithR05}. This notation overloads to vectors of
names: $\vec{x}^{\pi} := (x_{i}^{\pi} \; : \; 0 \leq i < |\vec{x}| )$ where $\pi \in \{L,R\}$.

We also use $P^{\Box} := P|\Box$.

In \cite{MeredithR05} an interpretation of the new operator is
given. It turns out that there are several possible interpretations
all enjoying the requisite algebraic properties of the operator (see
\cite{milner91polyadicpi}). We will therefore make liberal use of
$(\nu\; \vec{x})P$.

% subsection the_syntax_and_semantics_of_the_notation_system (end)   

\input{qm2pi.qmops} 

\input{qm2pi.sterngerlach} 

\input{qm2pi.metric} 

% section concurrent_process_calculi (end)

%\input{qm2pi.proofsketch}

% section proof sketch (end)

%\input{qm2pi.slviaknots} 

% section spatial logic via knots (end)

\input{qm2pi.conclusion}

% section conclusion (end)

%\input{qm2pi.dtcodes} 

% section wiring algorithm (end)

\input{qm2pi.ack} 

% section acknowledgments (end)

\newpage


\bibliographystyle{plain}   
\bibliography{../../biblios/main.bib}

\input{qm2pi.rhodetails}

\end{document}



\end{document}

 

% section acknowledgments (end)

\newpage


\bibliographystyle{plain}   
\bibliography{../../biblios/main.bib}

\documentclass[12pt]{llncs}
%\documentclass{jktr}

\usepackage[pdftex]{hyperref}                   
\usepackage {listings}
\usepackage {mathpartir}
\usepackage{bcprules}
%\usepackage{listings}
                       
\usepackage{graphicx} 
%\usepackage[margins=2.5cm,nohead,nofoot]{geometry}
%\usepackage{geometry}
\usepackage{amsfonts}
\usepackage{amstext}
\usepackage{latexsym}
\usepackage{amssymb}
\usepackage{color}


%\include{myPreamble}
\documentclass[12pt]{llncs}
%\documentclass{jktr}

\usepackage[pdftex]{hyperref}                   
\usepackage {listings}
\usepackage {mathpartir}
\usepackage{bcprules}
%\usepackage{listings}
                       
\usepackage{graphicx} 
%\usepackage[margins=2.5cm,nohead,nofoot]{geometry}
%\usepackage{geometry}
\usepackage{amsfonts}
\usepackage{amstext}
\usepackage{latexsym}
\usepackage{amssymb}
\usepackage{color}


%\include{myPreamble}
\include{qm2pi.local} 

%\ifpdf
%\usepackage[pdftex]{graphicx}
%\else
%\usepackage{graphicx}
%\fi

 % \ifpdf
%  \usepackage{pdfsync}
%  \if


%\title{Brief Article}
%\author{David F. Snyder}
%\author{L.G. Meredith}

%\address{Dept. of Math., Texas State University--San Marcos, San Marcos, TX 78666}
       
\pagestyle{empty}


\begin{document}

\lstset{language=[Objective]Caml,frame=shadowbox}

\input{qm2pi.front}

% section front matter (end)

\input{qm2pi.intro} 
 
% section introduction (end)

% \input{qm2pi.knotations} 

% section notation (end)

\input{qm2pi.process.calculi} 

% section concurrent_process_calculi_and_spatial_logics_ (end)
    
%\input{qm2pi.knots2pi} 

%\input{qm2pi.trefoil} 

%\input{qm2pi.mainthm} 

% subsection basic_interpretation (end)

%\input{qm2pi.rho.presentation} 
\subsection{The syntax and semantics of the notation system}\label{sub:the_syntax_and_semantics_of_the_notation_system} % (fold)

We now summarize a technical presentation of the calculus that
embodies our theory of dynamics. The typical presentation of such a
calculus follows the style of giving generators and relations on
them. The grammar, below, describing term constructors, freely
generates the set of processes, $\Proc$. This set is then quotiented
by a relation known as structural congruence and it is over this set
that the notion of dynamics is expressed. This presentation is
essentially that of \cite{MeredithR05} with the addition of
polyadicity and summation. For readability we have relegated some of
the technical subtleties to an appendix.

\subsubsection{Process grammar}\label{subsub:process_grammar}

\begin{mathpar}
  \inferrule* [lab=synchronization] {} {{M} \bc \pzero \;|\; x?F \;|\; x!C }
  \and
  \inferrule* [lab=abstraction] {} {{F} \bc (x)P}
  \and
  \inferrule* [lab=concretion] {} {{C} \bc \langle Q \rangle}
  \and
  \inferrule* [lab=process] {} {{P,Q} \bc M \;| \;P|Q \;|\; @{x}}
  \and
  \inferrule* [lab=name] {} {{x} \bc \quotep{P}}
\end{mathpar} 

Note that $\vec{x}$ (resp. $\vec{P}$) denotes a vector of names
(resp. processes) of length $|\vec{x}|$ (resp. $|\vec{P}|$). We adopt
the following useful abbreviations.

\begin{mathpar}
   x?(\vec{y}).P := x.(\vec{y})P \and  x\clift{\vec{P}} := x.\clift{\vec{P}}
   \and x!(y) := \lift{x}{\dropn{y}}
   \and \Pi_{i=0}^{n-1}P_i := P_0 | \ldots | P_{n-1}
\end{mathpar}

\subsubsection{Structural congruence}

\paragraph{Free and bound names and alpha-equivalence.} At the
core of structural equivalence is alpha-equivalence which identifies
process that are the same up to a change of variable. Formally, we
recognize the distinction between free and bound names. The free names
of a process, $\freenames{P}$, may be calculated recursively as
follows:

\begin{mathpar}
\freenames{\pzero} := \emptyset
  \and \\
  \freenames{x?(y).P} := \{ x \} \cup (\freenames{P} \setminus \{ y \})
  \and 
  \freenames{x!\langle P \rangle} := \{ x \} \cup \{ P \} 
  \and \\
  \freenames{P|Q} := \freenames{P} \cup \freenames{Q}
  \and \\
  \freenames{@{x}} := \{ x \}
\end{mathpar}

$\pi$
$\quotep{\pi}$

$\freenames{-} : \pi \to \mathcal{P}(\quotep{\pi})$

\begin{eqnarray*}
  \freenames{\pzero} & := & \emptyset \\
  \freenames{x?(y).P} & := & \{ x \} \cup (\freenames{P} \setminus \{ y \}) \\
  \freenames{x!\langle P \rangle} & := & \{ x \} \cup \{ P \} \\
  \freenames{P|Q} & := & \freenames{P} \cup \freenames{Q} \\
  \freenames{\dropn{x}} & := & \{ x \}
\end{eqnarray*}

The bound names of a process, $\boundnames{P}$, are those names occurring in $P$
that are not free. For example, in $x?(y).0$, the name $x$ is free, while $y$ is bound.

\begin{mathpar}
  \inferrule* [lab=monoidal-laws] {} { P|Q \equiv Q|P \and P|0 \equiv P \and P|(Q|R) \equiv (P|Q)|R }
\end{mathpar}

\begin{mathpar}
  \inferrule* [lab=alpha-equivalence] {} { (x)P \equiv (y)P\{y/x\} \and y \not\in \freenames{P} }
\end{mathpar}

\begin{definition}
Then two processes, $P,Q$, are alpha-equivalent if $P = Q\{\vec{y}/\vec{x}\}$ for
some $\vec{x} \in \boundnames{Q},\vec{y} \in \boundnames{P}$, where $Q\{\vec{y}/\vec{x}\}$
denotes the capture-avoiding substitution of $\vec{y}$ for $\vec{x}$ in $Q$.
\end{definition}

\begin{definition}
  The {\em structural congruence} \cite{SangiorgiWalker} , $\equiv$,
  between processes is the least congruence containing
  alpha-equivalence, satisfying the abelian monoid laws
  (associativity, commutativity and $\pzero$ as identity) for parallel
  composition $|$ and for summation $+$.
\end{definition}

\subsection{Name equivalence}

We take name equivalence, written $\nameeq$, to be the smallest
equivalence relation generated by the following rules.

\begin{mathpar}
\inferrule*[lab=Quote-drop]
{ }
{ \quotep{@{x}} \nameeq x }

\inferrule*[lab=Struct-equiv]
{ P \scong Q }
{ \quotep{P} \nameeq \quotep{Q} }
\end{mathpar}

The astute reader will have noticed that the mutual recursion of names
and processes imposes a mutual recursion on alpha-equivalence and
structural equivalence via name-equivalence. Fortunately, all of this
works out pleasantly and we may calculate in the natural way, free of
concern. The reader interested in the details is referred to the
appendix \ref{appendix:rho_details}.

\subsection{Substitution}

We use $\Proc$ for the set of processes, $\QProc$ for the set of
names, and $\id{\{}\vec{y} / \vec{x} \id{\}}$ to denote partial maps,
$s : \QProc \rightarrow \QProc$. A map, $s$ lifts, uniquely, to a map
on process terms, $\widehat{s} : \Proc \rightarrow \Proc$ by the
following equations.

\begin{mathpar}
  (0) \psubstp{Q}{P} := 0 \\
  (R \juxtap S) \psubstp{Q}{P}
  :=    
  (R)\psubstp{Q}{P} \juxtap (S) \psubstp{Q}{P} \\
  (x?(y).R) \psubstp{Q}{P}    
  :=    
  (x)\substp{Q}{P} (z)\concat( (R \psubstn{z}{y}) \psubstp{Q}{P} ) \\
  (\lift{x}{R}) \psubstp{Q}{P}  
  :=
  \lift{(x)\substp{Q}{P}}{ R \psubstp{Q}{P} } \\
%   (\dropn{x})  \psubstp{Q}{P}       
%   := 
%   \left\{ 
%     \begin{array}{ccc} 
%       \dropn{\quotep{Q}} & & x \nameeq \quotep{P} \\
%       \dropn{x} & & otherwise \\
%     \end{array}
%   \right. 
  (\dropn{x})  \psubstp{Q}{P}       
  := 
  \left\{ 
    \begin{array}{ccc} 
      Q & & x \nameeq \quotep{P} \\
      \dropn{x} & & otherwise \\
    \end{array}
  \right.
\end{mathpar}
 

where

\begin{eqnarray}
  (x)\id{\{} \lpquote Q \rpquote / \lpquote P \rpquote \id{\}}            = 
  \left\{ 
    \begin{array}{ccc}
      \lpquote Q \rpquote & & x \nameeq \lpquote P \rpquote \\
      x & & otherwise \\
    \end{array}
  \right. \nonumber
\end{eqnarray}

and $z$ is chosen distinct from $\quotep{P}$, $\quotep{Q}$, the free
names in $Q$, and all the names in $R$. Our $\alpha$-equivalence will
be built in the standard way from this substitution.

\begin{remark}\label{rem:no_self_referential_names}
  One consequence of these definitions is that $\forall P. \quotep{P}
  \not\in \freenames{P}$.
\end{remark}

\subsection{ Dynamic quote: an example }

Anticipating something of what's to come, consider applying the
substitution, $\widehat{\id{\{}u / z \id{\}}}$, to the following pair
of processes, $\lift{w}{y!(z)}$ and $w[ \lpquote y!(z) \rpquote ]$.

\begin{eqnarray}
	\lift{w}{y!(z)}\widehat{\id{\{}u / z \id{\}}}
		& = &
		\lift{w}{y!(u)} \nonumber\\
	w[ \lpquote y!(z) \rpquote ] \widehat{ \id{\{}u / z \id{\}} }
		& = &
		w[ \lpquote y!(z) \rpquote ] \nonumber
\end{eqnarray}

Because the body of the process between quotes is impervious to
substitution, we get radically different answers. In fact, by
examining the first process in an input context,
e.g. $x?(z).\lift{w}{y!(z)}$, we see that the process under the lift
operator may be shaped by prefixed inputs binding a name inside it. In
this sense, the lift operator will be seen as a way to dynamically
construct processes before reifying them as names.

Finally equipped with these standard features we can present the
dynamics of the calculus.

\subsubsection{Operational semantics} 

Finally, we introduce the computational dynamics. What marks these
algebras as distinct from other more traditionally studied algebraic
structures, e.g. vector spaces or polynomial rings, is the manner in
which dynamics is captured. In traditional structures, dynamics is typically
expressed through morphisms between such structures, as in linear maps
between vector spaces or morphisms between rings. In algebras
associated with the semantics of computation, the dynamics is
expressed as part of the algebraic structure itself, through a
reduction reduction relation typically denoted by $\red$. Below, we
give a recursive presentation of this relation for the calculus used
in the encoding.

$\red \subseteq \pi \times \pi$
$\red : \pi \to \mathcal{P}(\pi)$

\begin{mathpar}
  \inferrule* [lab=Comm] { \textsf{match}( x_{src}, x_{trgt} ) } { x_{trgt}?(y)P \; | \; x_{src}!\langle {Q} \rangle \red P\{\quotep{Q}/y}\} }
  \and \\
  \inferrule* [lab=Par] {{P} \red {P}'} {{{P} | {Q}} \red {{P}' | {Q}}}
  \and
  \inferrule* [lab=Equiv]{{{P} \scong {P}'} \andalso {{P}' \red {Q}'} \andalso {{Q}' \scong {Q}}}{{P} \red {Q}}
\end{mathpar}

\begin{eqnarray*}
  match_{\equiv} (\quotep{P},\quotep{Q}) & := & P \equiv Q \\
  match_{\dagger}(\quotep{P},\quotep{Q}) & := & \forall R. P|Q \red^{*} R => R \red^{*} 0 \\
  match_{K}(\quotep{P},\quotep{Q}) & := & K \mbox{ for some context } K
\end{eqnarray*}

$u?(x)P | u!\langle Q \rangle \red P\{\quotep{Q}/x\}$

%We write $\wred$ for $\red^*$, and $P\red$ if $\exists Q $ such that $ P \red Q$.
We write $P\red$ if $\exists Q $ such that $ P \red Q$ and $P\not\red$, otherwise.

\section{Replication}

As mentioned before, it is known that replication (and hence
recursion) can be implemented in a higher-order process algebra
\cite{SangiorgiWalker}. As our first example of calculation with the
machinery thus far presented we give the construction explicitly in
the {\rhoc}.

\begin{eqnarray}
	D_{x} & := & \prefix{x}{y}{(\binpar{\outputp{x}{y}}{@{y}})} \nonumber\\
	\bangp_{x}{P} & := & \binpar{{x}!\langle{\binpar{D_{x}}{P}}\rangle}{D_{x}} \nonumber
\end{eqnarray}

\begin{eqnarray}
	\bangp_{x}{P} & & \nonumber\\
	=
	& {x}!\langle{(\prefix{x}{y}{(\outputp{x}{y} | @{y})) | P}}\rangle 
	      | \prefix{x}{y}{(\outputp{x}{y} | @{y})} & \nonumber\\
	\red
	& (\outputp{x}{y} | @{y})\substn{\quotep{(\prefix{x}{y}{(@{y} | \outputp{x}{y})) | P}}}{y} & \nonumber\\
	=
	& \outputp{x}{\quotep{(\prefix{x}{y}{(\outputp{x}{y} | @{y})) | P}}}
	  | {(\prefix{x}{y}{(\outputp{x}{y} | @{y})) | P}} & \nonumber\\
	\red
	& \ldots & \nonumber\\
	\red^*
	& P | P | \ldots & \nonumber
\end{eqnarray}

Of course, this encoding, as an implementation, runs away, unfolding
$\bangp{P}$ eagerly. A lazier and more implementable replication
operator, restricted to input-guarded processes, may be obtained as follows.

\begin{eqnarray}
\bangp{\prefix{u}{v}{P}} 
	:= 
	\binpar{\lift{x}{\prefix{u}{v}{(\binpar{D(x)}{P})}}}{D(x)} \nonumber
\end{eqnarray}

\begin{remark}
  Note that the lazier definition still does not deal with summation
  or mixed summation (i.e. sums over input and output). The reader is
  invited to construct definitions of replication that deal with these
  features. 

  Further, the definitions are parameterized in a name, $x$. Can you,
  gentle reader, make a definition that eliminates this parameter and
  guarantees no accidental interaction between the replication
  machinery and the process being replicated -- i.e. no accidental
  sharing of names used by the process to get its work done and the
  name(s) used by the replication to effect copying. This latter
  revision of the definition of replication is crucial to obtaining
  the expected identity $!!P \sim !P$.
\end{remark}

\begin{remark}\label{rem:paradoxical_combinator}
  The reader familiar with the lambda calculus will have noticed the
  similarity between $D$ and the paradoxical combinator.

  [Ed. note: the existence of this seems to suggest we have to be more
  restrictive on the set of processes and names we admit if we are to
  support no-cloning.]
\end{remark}

\subsubsection{Bisimulation}

The computational dynamics gives rise to another kind of equivalence,
the equivalence of computational behavior. As previously mentioned
this is typically captured \emph{via} some form of bisimulation.

% The notion we use in this paper is weak barbed bisimulation
% \cite{milner91polyadicpi}.

The notion we use in this paper is derived from weak barbed
bisimulation \cite{milner91polyadicpi}. 

\begin{definition}
An \emph{observation relation}, $\downarrow_{\mathcal N}$, over a set
of names, $\mathcal N$, is the smallest relation satisfying the rules
below.

\infrule[Out-barb]{y \in {\mathcal N}, \; x \nameeq y}
		  {\outputp{x}{v} \downarrow_{\mathcal N} x}
\infrule[Par-barb]{\mbox{$P\downarrow_{\mathcal N} x$ or $Q\downarrow_{\mathcal N} x$}}
		  {\binpar{P}{Q} \downarrow_{\mathcal N} x}

We write $P \Downarrow_{\mathcal N} x$ if there is $Q$ such that 
$P \wred Q$ and $Q \downarrow_{\mathcal N} x$.
\end{definition}

\begin{definition}
%\label{def.bbisim}
An  ${\mathcal N}$-\emph{barbed bisimulation} over a set of names, ${\mathcal N}$, is a symmetric binary relation 
${\mathcal S}_{\mathcal N}$ between agents such that $P\rel{S}_{\mathcal N}Q$ implies:
\begin{enumerate}
\item If $P \red P'$ then $Q \wred Q'$ and $P'\rel{S}_{\mathcal N} Q'$.
\item If $P\downarrow_{\mathcal N} x$, then $Q\Downarrow_{\mathcal N} x$.
\end{enumerate}
$P$ is ${\mathcal N}$-barbed bisimilar to $Q$, written
$P \wbbisim_{\mathcal N} Q$, if $P \rel{S}_{\mathcal N} Q$ for some ${\mathcal N}$-barbed bisimulation ${\mathcal S}_{\mathcal N}$.
\end{definition}

$\mathcal{R} \subseteq \pi \times \pi$

$P \mathcal{R} Q => \forall P'. P \red P' \Rightarrow \exists Q'. Q \red Q', P' \mathcal{R} Q'$

$P \vdash x \Rightarrow Q \vdash x$

\begin{mathpar}
  \inferrule*[lab=Out-barb]{x \nameeq y}{{y}!\langle{Q}\rangle \vdash x}
  \and
  \inferrule*[lab=Par-barb]{\mbox{$P\vdash x$ or $Q\vdash x$}}{\binpar{P}{Q} \vdash x}
\end{mathpar}

\subsubsection{Contexts}

One of the principle advantages of computational calculi like the
$\pi$-calculus is a well-defined notion of context,
contextual-equivalence and a correlation between
contextual-equivalence and notions of bisimulation. The notion of
context allows the decomposition of a process into (sub-)process and
its syntactic environment, its context. Thus, a context may be
thought of as a process with a ``hole'' (written $\Box$) in it. The
application of a context $M$ to a process $P$, written $M[P]$, is
tantamount to filling the hole in $M$ with $P$. In this paper we do
not need the full weight of this theory, but do make use of the notion
of context in the proof the main theorem. 

\begin{mathpar}
  \inferrule* [lab=summation] {} {{M_{M},M_{N}} \bc \Box \;|\; x.M_{A} \;|\; M_{M}+M_{N}}
  \and
  \inferrule* [lab=agent] {} {{M_{A}} \bc (\vec{x})M_{P} \;| \; \clift{P_0,\ldots,M_{P},\ldots,P_N}}
  \and \\
  \inferrule* [lab=process] {} {{M_{P}} \bc M_{N} \;| \;P|M_{P} }
\end{mathpar} 

\begin{mathpar}
  \inferrule* [lab=sychronization] {} {M_{N} \bc \Box \;|\; x?M_{F} \;|\; x!M_{C}}
  \and
  \inferrule* [lab=abstraction] {} {{M_{F}} \bc (x)M_{P} }
  \and
  \inferrule* [lab=concretion] {} {{M_{C}} \bc \langle M_{P} \rangle }
  \and \\
  \inferrule* [lab=process] {} {{M_{P}} \bc M_{N} \;| \;P|M_{P} }
\end{mathpar}

\begin{definition}[contextual application] Given a context $M$, and
  process $P$, we define the \emph{contextual application}, $M[P] :=
  M\{P/\Box\}$. That is, the contextual application of M to P is the
  substitution of $P$ for $\Box$ in $M$.
\end{definition}

$\meaningof{-} : L \to \mathcal{P}(\pi)$

\begin{mathpar}
  \inferrule* [lab=collection] {} {\meaningof{true} = \pi, \and \meaningof{~E} = \pi \setminus \meaningof{E}, \and \meaningof{E_{1} \& E_{2}} = \meaningof{E_{1}} \cap \meaningof{E_{2}}}
\end{mathpar}

\begin{mathpar}
  \inferrule* [lab=structure] {} {\meaningof{0} = \{ P \in \pi | P \equiv 0 \}, \and \\ \meaningof{E_1 | E_2} = \{ P \in \pi | P \equiv P_{1} | P_{2}, P_{1} \in \meaningof{E_{1}}, P_{2} \in \meaningof{E_2}\} }
\end{mathpar}

\begin{mathpar}
 \inferrule* [lab=behavior] {} {\meaningof{\langle a?b \rangle E} = \{ P \in \pi | P \equiv Q | u?(y)P', \\ \and \\\\ \and \\ \;\;\; u \in \meaningof{a}, \forall z.P'\{z/y\} \in \meaningof{E\{z/b\}}\}, \and \\ \meaningof{a!E} = \{ P \in \pi | P \equiv Q | x!\langle P' \rangle, x \in \meaningof{a} P' \in \meaningof{E}\} }
\end{mathpar}

\begin{mathpar}
 \inferrule* [lab=nominal] {} {\meaningof{\quotep{E}} = \{ \quotep{P} \in \quotep{\pi} | P \in \meaningof{E} \}, \and \meaningof{\quotep{P}} = \{ \quotep{Q} \in \quotep{\pi} | P \equiv Q \} \and \\ \meaningof{@\quotep{E}} = \{ P \in \pi | P \equiv @x, x \in \meaningof{E} \}}
\end{mathpar}

\begin{eqnarray*}
  \\
  \meaningof{-} : TS \to ST
\end{eqnarray*}

\begin{eqnarray*}
  \\
  L : TS \to ST
\end{eqnarray*}

\begin{eqnarray*}
  \\
  P \models E \iff P \in \meaningof{E}
\end{eqnarray*}

\begin{eqnarray*}
  P \approx_{L} Q \iff \forall E \in L. P \models E \iff Q \models E
\end{eqnarray*}

\begin{eqnarray*}
  P \approx_{K} Q
\end{eqnarray*}

\begin{eqnarray*}
  P \approx Q
\end{eqnarray*}

$\approx_{K} = \approx = \approx_{L}$

\subsubsection{Contextual duality}

Note that contexts extend the quotation operation to a family of
operations from processes to names. Given a context, $M$, we can
define a \emph{nominal context}, $\quotep{M}$ by $\quotep{M}[P] :=
\quotep{M[P]}$. To foreshadow what is to come we observe that these
operations enjoy a duality with processes very much like the duality
between vectors and maps from vectors to scalars.

Further, because the calculus is essentially higher-order, we have a
correspondence between contexts and processes. More specifically,
given a name $x$ and a context $M$ we can construct $M^{*}_{x}$ such
that 

\begin{mathpar}
  M^{*}_{x} | \lift{x}{P} \red M[P]
\end{mathpar}

namely,

\begin{mathpar}
  M^{*}_{x} := x?(u).M[\dropn{u}]
\end{mathpar}

The dependence of $M^{*}_{x}$ on a name makes it an abstraction, 

\begin{mathpar}
  M^{*} := (x)x?(u).M[\dropn{u}]
\end{mathpar}

\subsection{Additional notation}

It will sometimes be convenient to denote the process a name
quotes. We already have the notation $x = \quotep{P}$, but it will be
convenient to introduce an alternate notation, $\procn{x}$, when we
want to emphasize the connection to the use of the name. Note that, by
virtue of name equivalence, $\quotep{\procn{x}} \nameeq x$; so, the
notation is consistent with previous definitions.

Further, because names have structure it is possible to effect
substitutions on the basis of that structure. This means we need to
upgrade our notation for substitutions, which we accomplish by
adapting comprehension notation. Thus,

\begin{mathpar}
  P\{ y / x : x \in S \}
\end{mathpar}

is interpreted to mean the process derived from P by replacing (in a
capture-avoiding manner) each occurrence of $x$ in $S$ by $y$. For example,

\begin{mathpar}
  P\{ \quotep{\procn{x}|\procn{x}} / x : x \in \freenames{P} \}
\end{mathpar}

will replace each (occurrence) of a free name $x$ in $P$ by
$\quotep{\procn{x}|\procn{x}}$.

Also, we will avail ourselves of the notation $x^{L}$ and $x^{R}$ to
denote injections of a name into disjoint copies of the name
space. There are numerous ways to accomplish this. One example can be
found in \cite{MeredithR05}. This notation overloads to vectors of
names: $\vec{x}^{\pi} := (x_{i}^{\pi} \; : \; 0 \leq i < |\vec{x}| )$ where $\pi \in \{L,R\}$.

We also use $P^{\Box} := P|\Box$.

In \cite{MeredithR05} an interpretation of the new operator is
given. It turns out that there are several possible interpretations
all enjoying the requisite algebraic properties of the operator (see
\cite{milner91polyadicpi}). We will therefore make liberal use of
$(\nu\; \vec{x})P$.

% subsection the_syntax_and_semantics_of_the_notation_system (end)   

\input{qm2pi.qmops} 

\input{qm2pi.sterngerlach} 

\input{qm2pi.metric} 

% section concurrent_process_calculi (end)

%\input{qm2pi.proofsketch}

% section proof sketch (end)

%\input{qm2pi.slviaknots} 

% section spatial logic via knots (end)

\input{qm2pi.conclusion}

% section conclusion (end)

%\input{qm2pi.dtcodes} 

% section wiring algorithm (end)

\input{qm2pi.ack} 

% section acknowledgments (end)

\newpage


\bibliographystyle{plain}   
\bibliography{../../biblios/main.bib}

\input{qm2pi.rhodetails}

\end{document}

 

%\ifpdf
%\usepackage[pdftex]{graphicx}
%\else
%\usepackage{graphicx}
%\fi

 % \ifpdf
%  \usepackage{pdfsync}
%  \if


%\title{Brief Article}
%\author{David F. Snyder}
%\author{L.G. Meredith}

%\address{Dept. of Math., Texas State University--San Marcos, San Marcos, TX 78666}
       
\pagestyle{empty}


\begin{document}

\lstset{language=[Objective]Caml,frame=shadowbox}

\documentclass[12pt]{llncs}
%\documentclass{jktr}

\usepackage[pdftex]{hyperref}                   
\usepackage {listings}
\usepackage {mathpartir}
\usepackage{bcprules}
%\usepackage{listings}
                       
\usepackage{graphicx} 
%\usepackage[margins=2.5cm,nohead,nofoot]{geometry}
%\usepackage{geometry}
\usepackage{amsfonts}
\usepackage{amstext}
\usepackage{latexsym}
\usepackage{amssymb}
\usepackage{color}


%\include{myPreamble}
\include{qm2pi.local} 

%\ifpdf
%\usepackage[pdftex]{graphicx}
%\else
%\usepackage{graphicx}
%\fi

 % \ifpdf
%  \usepackage{pdfsync}
%  \if


%\title{Brief Article}
%\author{David F. Snyder}
%\author{L.G. Meredith}

%\address{Dept. of Math., Texas State University--San Marcos, San Marcos, TX 78666}
       
\pagestyle{empty}


\begin{document}

\lstset{language=[Objective]Caml,frame=shadowbox}

\input{qm2pi.front}

% section front matter (end)

\input{qm2pi.intro} 
 
% section introduction (end)

% \input{qm2pi.knotations} 

% section notation (end)

\input{qm2pi.process.calculi} 

% section concurrent_process_calculi_and_spatial_logics_ (end)
    
%\input{qm2pi.knots2pi} 

%\input{qm2pi.trefoil} 

%\input{qm2pi.mainthm} 

% subsection basic_interpretation (end)

%\input{qm2pi.rho.presentation} 
\subsection{The syntax and semantics of the notation system}\label{sub:the_syntax_and_semantics_of_the_notation_system} % (fold)

We now summarize a technical presentation of the calculus that
embodies our theory of dynamics. The typical presentation of such a
calculus follows the style of giving generators and relations on
them. The grammar, below, describing term constructors, freely
generates the set of processes, $\Proc$. This set is then quotiented
by a relation known as structural congruence and it is over this set
that the notion of dynamics is expressed. This presentation is
essentially that of \cite{MeredithR05} with the addition of
polyadicity and summation. For readability we have relegated some of
the technical subtleties to an appendix.

\subsubsection{Process grammar}\label{subsub:process_grammar}

\begin{mathpar}
  \inferrule* [lab=synchronization] {} {{M} \bc \pzero \;|\; x?F \;|\; x!C }
  \and
  \inferrule* [lab=abstraction] {} {{F} \bc (x)P}
  \and
  \inferrule* [lab=concretion] {} {{C} \bc \langle Q \rangle}
  \and
  \inferrule* [lab=process] {} {{P,Q} \bc M \;| \;P|Q \;|\; @{x}}
  \and
  \inferrule* [lab=name] {} {{x} \bc \quotep{P}}
\end{mathpar} 

Note that $\vec{x}$ (resp. $\vec{P}$) denotes a vector of names
(resp. processes) of length $|\vec{x}|$ (resp. $|\vec{P}|$). We adopt
the following useful abbreviations.

\begin{mathpar}
   x?(\vec{y}).P := x.(\vec{y})P \and  x\clift{\vec{P}} := x.\clift{\vec{P}}
   \and x!(y) := \lift{x}{\dropn{y}}
   \and \Pi_{i=0}^{n-1}P_i := P_0 | \ldots | P_{n-1}
\end{mathpar}

\subsubsection{Structural congruence}

\paragraph{Free and bound names and alpha-equivalence.} At the
core of structural equivalence is alpha-equivalence which identifies
process that are the same up to a change of variable. Formally, we
recognize the distinction between free and bound names. The free names
of a process, $\freenames{P}$, may be calculated recursively as
follows:

\begin{mathpar}
\freenames{\pzero} := \emptyset
  \and \\
  \freenames{x?(y).P} := \{ x \} \cup (\freenames{P} \setminus \{ y \})
  \and 
  \freenames{x!\langle P \rangle} := \{ x \} \cup \{ P \} 
  \and \\
  \freenames{P|Q} := \freenames{P} \cup \freenames{Q}
  \and \\
  \freenames{@{x}} := \{ x \}
\end{mathpar}

$\pi$
$\quotep{\pi}$

$\freenames{-} : \pi \to \mathcal{P}(\quotep{\pi})$

\begin{eqnarray*}
  \freenames{\pzero} & := & \emptyset \\
  \freenames{x?(y).P} & := & \{ x \} \cup (\freenames{P} \setminus \{ y \}) \\
  \freenames{x!\langle P \rangle} & := & \{ x \} \cup \{ P \} \\
  \freenames{P|Q} & := & \freenames{P} \cup \freenames{Q} \\
  \freenames{\dropn{x}} & := & \{ x \}
\end{eqnarray*}

The bound names of a process, $\boundnames{P}$, are those names occurring in $P$
that are not free. For example, in $x?(y).0$, the name $x$ is free, while $y$ is bound.

\begin{mathpar}
  \inferrule* [lab=monoidal-laws] {} { P|Q \equiv Q|P \and P|0 \equiv P \and P|(Q|R) \equiv (P|Q)|R }
\end{mathpar}

\begin{mathpar}
  \inferrule* [lab=alpha-equivalence] {} { (x)P \equiv (y)P\{y/x\} \and y \not\in \freenames{P} }
\end{mathpar}

\begin{definition}
Then two processes, $P,Q$, are alpha-equivalent if $P = Q\{\vec{y}/\vec{x}\}$ for
some $\vec{x} \in \boundnames{Q},\vec{y} \in \boundnames{P}$, where $Q\{\vec{y}/\vec{x}\}$
denotes the capture-avoiding substitution of $\vec{y}$ for $\vec{x}$ in $Q$.
\end{definition}

\begin{definition}
  The {\em structural congruence} \cite{SangiorgiWalker} , $\equiv$,
  between processes is the least congruence containing
  alpha-equivalence, satisfying the abelian monoid laws
  (associativity, commutativity and $\pzero$ as identity) for parallel
  composition $|$ and for summation $+$.
\end{definition}

\subsection{Name equivalence}

We take name equivalence, written $\nameeq$, to be the smallest
equivalence relation generated by the following rules.

\begin{mathpar}
\inferrule*[lab=Quote-drop]
{ }
{ \quotep{@{x}} \nameeq x }

\inferrule*[lab=Struct-equiv]
{ P \scong Q }
{ \quotep{P} \nameeq \quotep{Q} }
\end{mathpar}

The astute reader will have noticed that the mutual recursion of names
and processes imposes a mutual recursion on alpha-equivalence and
structural equivalence via name-equivalence. Fortunately, all of this
works out pleasantly and we may calculate in the natural way, free of
concern. The reader interested in the details is referred to the
appendix \ref{appendix:rho_details}.

\subsection{Substitution}

We use $\Proc$ for the set of processes, $\QProc$ for the set of
names, and $\id{\{}\vec{y} / \vec{x} \id{\}}$ to denote partial maps,
$s : \QProc \rightarrow \QProc$. A map, $s$ lifts, uniquely, to a map
on process terms, $\widehat{s} : \Proc \rightarrow \Proc$ by the
following equations.

\begin{mathpar}
  (0) \psubstp{Q}{P} := 0 \\
  (R \juxtap S) \psubstp{Q}{P}
  :=    
  (R)\psubstp{Q}{P} \juxtap (S) \psubstp{Q}{P} \\
  (x?(y).R) \psubstp{Q}{P}    
  :=    
  (x)\substp{Q}{P} (z)\concat( (R \psubstn{z}{y}) \psubstp{Q}{P} ) \\
  (\lift{x}{R}) \psubstp{Q}{P}  
  :=
  \lift{(x)\substp{Q}{P}}{ R \psubstp{Q}{P} } \\
%   (\dropn{x})  \psubstp{Q}{P}       
%   := 
%   \left\{ 
%     \begin{array}{ccc} 
%       \dropn{\quotep{Q}} & & x \nameeq \quotep{P} \\
%       \dropn{x} & & otherwise \\
%     \end{array}
%   \right. 
  (\dropn{x})  \psubstp{Q}{P}       
  := 
  \left\{ 
    \begin{array}{ccc} 
      Q & & x \nameeq \quotep{P} \\
      \dropn{x} & & otherwise \\
    \end{array}
  \right.
\end{mathpar}
 

where

\begin{eqnarray}
  (x)\id{\{} \lpquote Q \rpquote / \lpquote P \rpquote \id{\}}            = 
  \left\{ 
    \begin{array}{ccc}
      \lpquote Q \rpquote & & x \nameeq \lpquote P \rpquote \\
      x & & otherwise \\
    \end{array}
  \right. \nonumber
\end{eqnarray}

and $z$ is chosen distinct from $\quotep{P}$, $\quotep{Q}$, the free
names in $Q$, and all the names in $R$. Our $\alpha$-equivalence will
be built in the standard way from this substitution.

\begin{remark}\label{rem:no_self_referential_names}
  One consequence of these definitions is that $\forall P. \quotep{P}
  \not\in \freenames{P}$.
\end{remark}

\subsection{ Dynamic quote: an example }

Anticipating something of what's to come, consider applying the
substitution, $\widehat{\id{\{}u / z \id{\}}}$, to the following pair
of processes, $\lift{w}{y!(z)}$ and $w[ \lpquote y!(z) \rpquote ]$.

\begin{eqnarray}
	\lift{w}{y!(z)}\widehat{\id{\{}u / z \id{\}}}
		& = &
		\lift{w}{y!(u)} \nonumber\\
	w[ \lpquote y!(z) \rpquote ] \widehat{ \id{\{}u / z \id{\}} }
		& = &
		w[ \lpquote y!(z) \rpquote ] \nonumber
\end{eqnarray}

Because the body of the process between quotes is impervious to
substitution, we get radically different answers. In fact, by
examining the first process in an input context,
e.g. $x?(z).\lift{w}{y!(z)}$, we see that the process under the lift
operator may be shaped by prefixed inputs binding a name inside it. In
this sense, the lift operator will be seen as a way to dynamically
construct processes before reifying them as names.

Finally equipped with these standard features we can present the
dynamics of the calculus.

\subsubsection{Operational semantics} 

Finally, we introduce the computational dynamics. What marks these
algebras as distinct from other more traditionally studied algebraic
structures, e.g. vector spaces or polynomial rings, is the manner in
which dynamics is captured. In traditional structures, dynamics is typically
expressed through morphisms between such structures, as in linear maps
between vector spaces or morphisms between rings. In algebras
associated with the semantics of computation, the dynamics is
expressed as part of the algebraic structure itself, through a
reduction reduction relation typically denoted by $\red$. Below, we
give a recursive presentation of this relation for the calculus used
in the encoding.

$\red \subseteq \pi \times \pi$
$\red : \pi \to \mathcal{P}(\pi)$

\begin{mathpar}
  \inferrule* [lab=Comm] { \textsf{match}( x_{src}, x_{trgt} ) } { x_{trgt}?(y)P \; | \; x_{src}!\langle {Q} \rangle \red P\{\quotep{Q}/y}\} }
  \and \\
  \inferrule* [lab=Par] {{P} \red {P}'} {{{P} | {Q}} \red {{P}' | {Q}}}
  \and
  \inferrule* [lab=Equiv]{{{P} \scong {P}'} \andalso {{P}' \red {Q}'} \andalso {{Q}' \scong {Q}}}{{P} \red {Q}}
\end{mathpar}

\begin{eqnarray*}
  match_{\equiv} (\quotep{P},\quotep{Q}) & := & P \equiv Q \\
  match_{\dagger}(\quotep{P},\quotep{Q}) & := & \forall R. P|Q \red^{*} R => R \red^{*} 0 \\
  match_{K}(\quotep{P},\quotep{Q}) & := & K \mbox{ for some context } K
\end{eqnarray*}

$u?(x)P | u!\langle Q \rangle \red P\{\quotep{Q}/x\}$

%We write $\wred$ for $\red^*$, and $P\red$ if $\exists Q $ such that $ P \red Q$.
We write $P\red$ if $\exists Q $ such that $ P \red Q$ and $P\not\red$, otherwise.

\section{Replication}

As mentioned before, it is known that replication (and hence
recursion) can be implemented in a higher-order process algebra
\cite{SangiorgiWalker}. As our first example of calculation with the
machinery thus far presented we give the construction explicitly in
the {\rhoc}.

\begin{eqnarray}
	D_{x} & := & \prefix{x}{y}{(\binpar{\outputp{x}{y}}{@{y}})} \nonumber\\
	\bangp_{x}{P} & := & \binpar{{x}!\langle{\binpar{D_{x}}{P}}\rangle}{D_{x}} \nonumber
\end{eqnarray}

\begin{eqnarray}
	\bangp_{x}{P} & & \nonumber\\
	=
	& {x}!\langle{(\prefix{x}{y}{(\outputp{x}{y} | @{y})) | P}}\rangle 
	      | \prefix{x}{y}{(\outputp{x}{y} | @{y})} & \nonumber\\
	\red
	& (\outputp{x}{y} | @{y})\substn{\quotep{(\prefix{x}{y}{(@{y} | \outputp{x}{y})) | P}}}{y} & \nonumber\\
	=
	& \outputp{x}{\quotep{(\prefix{x}{y}{(\outputp{x}{y} | @{y})) | P}}}
	  | {(\prefix{x}{y}{(\outputp{x}{y} | @{y})) | P}} & \nonumber\\
	\red
	& \ldots & \nonumber\\
	\red^*
	& P | P | \ldots & \nonumber
\end{eqnarray}

Of course, this encoding, as an implementation, runs away, unfolding
$\bangp{P}$ eagerly. A lazier and more implementable replication
operator, restricted to input-guarded processes, may be obtained as follows.

\begin{eqnarray}
\bangp{\prefix{u}{v}{P}} 
	:= 
	\binpar{\lift{x}{\prefix{u}{v}{(\binpar{D(x)}{P})}}}{D(x)} \nonumber
\end{eqnarray}

\begin{remark}
  Note that the lazier definition still does not deal with summation
  or mixed summation (i.e. sums over input and output). The reader is
  invited to construct definitions of replication that deal with these
  features. 

  Further, the definitions are parameterized in a name, $x$. Can you,
  gentle reader, make a definition that eliminates this parameter and
  guarantees no accidental interaction between the replication
  machinery and the process being replicated -- i.e. no accidental
  sharing of names used by the process to get its work done and the
  name(s) used by the replication to effect copying. This latter
  revision of the definition of replication is crucial to obtaining
  the expected identity $!!P \sim !P$.
\end{remark}

\begin{remark}\label{rem:paradoxical_combinator}
  The reader familiar with the lambda calculus will have noticed the
  similarity between $D$ and the paradoxical combinator.

  [Ed. note: the existence of this seems to suggest we have to be more
  restrictive on the set of processes and names we admit if we are to
  support no-cloning.]
\end{remark}

\subsubsection{Bisimulation}

The computational dynamics gives rise to another kind of equivalence,
the equivalence of computational behavior. As previously mentioned
this is typically captured \emph{via} some form of bisimulation.

% The notion we use in this paper is weak barbed bisimulation
% \cite{milner91polyadicpi}.

The notion we use in this paper is derived from weak barbed
bisimulation \cite{milner91polyadicpi}. 

\begin{definition}
An \emph{observation relation}, $\downarrow_{\mathcal N}$, over a set
of names, $\mathcal N$, is the smallest relation satisfying the rules
below.

\infrule[Out-barb]{y \in {\mathcal N}, \; x \nameeq y}
		  {\outputp{x}{v} \downarrow_{\mathcal N} x}
\infrule[Par-barb]{\mbox{$P\downarrow_{\mathcal N} x$ or $Q\downarrow_{\mathcal N} x$}}
		  {\binpar{P}{Q} \downarrow_{\mathcal N} x}

We write $P \Downarrow_{\mathcal N} x$ if there is $Q$ such that 
$P \wred Q$ and $Q \downarrow_{\mathcal N} x$.
\end{definition}

\begin{definition}
%\label{def.bbisim}
An  ${\mathcal N}$-\emph{barbed bisimulation} over a set of names, ${\mathcal N}$, is a symmetric binary relation 
${\mathcal S}_{\mathcal N}$ between agents such that $P\rel{S}_{\mathcal N}Q$ implies:
\begin{enumerate}
\item If $P \red P'$ then $Q \wred Q'$ and $P'\rel{S}_{\mathcal N} Q'$.
\item If $P\downarrow_{\mathcal N} x$, then $Q\Downarrow_{\mathcal N} x$.
\end{enumerate}
$P$ is ${\mathcal N}$-barbed bisimilar to $Q$, written
$P \wbbisim_{\mathcal N} Q$, if $P \rel{S}_{\mathcal N} Q$ for some ${\mathcal N}$-barbed bisimulation ${\mathcal S}_{\mathcal N}$.
\end{definition}

$\mathcal{R} \subseteq \pi \times \pi$

$P \mathcal{R} Q => \forall P'. P \red P' \Rightarrow \exists Q'. Q \red Q', P' \mathcal{R} Q'$

$P \vdash x \Rightarrow Q \vdash x$

\begin{mathpar}
  \inferrule*[lab=Out-barb]{x \nameeq y}{{y}!\langle{Q}\rangle \vdash x}
  \and
  \inferrule*[lab=Par-barb]{\mbox{$P\vdash x$ or $Q\vdash x$}}{\binpar{P}{Q} \vdash x}
\end{mathpar}

\subsubsection{Contexts}

One of the principle advantages of computational calculi like the
$\pi$-calculus is a well-defined notion of context,
contextual-equivalence and a correlation between
contextual-equivalence and notions of bisimulation. The notion of
context allows the decomposition of a process into (sub-)process and
its syntactic environment, its context. Thus, a context may be
thought of as a process with a ``hole'' (written $\Box$) in it. The
application of a context $M$ to a process $P$, written $M[P]$, is
tantamount to filling the hole in $M$ with $P$. In this paper we do
not need the full weight of this theory, but do make use of the notion
of context in the proof the main theorem. 

\begin{mathpar}
  \inferrule* [lab=summation] {} {{M_{M},M_{N}} \bc \Box \;|\; x.M_{A} \;|\; M_{M}+M_{N}}
  \and
  \inferrule* [lab=agent] {} {{M_{A}} \bc (\vec{x})M_{P} \;| \; \clift{P_0,\ldots,M_{P},\ldots,P_N}}
  \and \\
  \inferrule* [lab=process] {} {{M_{P}} \bc M_{N} \;| \;P|M_{P} }
\end{mathpar} 

\begin{mathpar}
  \inferrule* [lab=sychronization] {} {M_{N} \bc \Box \;|\; x?M_{F} \;|\; x!M_{C}}
  \and
  \inferrule* [lab=abstraction] {} {{M_{F}} \bc (x)M_{P} }
  \and
  \inferrule* [lab=concretion] {} {{M_{C}} \bc \langle M_{P} \rangle }
  \and \\
  \inferrule* [lab=process] {} {{M_{P}} \bc M_{N} \;| \;P|M_{P} }
\end{mathpar}

\begin{definition}[contextual application] Given a context $M$, and
  process $P$, we define the \emph{contextual application}, $M[P] :=
  M\{P/\Box\}$. That is, the contextual application of M to P is the
  substitution of $P$ for $\Box$ in $M$.
\end{definition}

$\meaningof{-} : L \to \mathcal{P}(\pi)$

\begin{mathpar}
  \inferrule* [lab=collection] {} {\meaningof{true} = \pi, \and \meaningof{~E} = \pi \setminus \meaningof{E}, \and \meaningof{E_{1} \& E_{2}} = \meaningof{E_{1}} \cap \meaningof{E_{2}}}
\end{mathpar}

\begin{mathpar}
  \inferrule* [lab=structure] {} {\meaningof{0} = \{ P \in \pi | P \equiv 0 \}, \and \\ \meaningof{E_1 | E_2} = \{ P \in \pi | P \equiv P_{1} | P_{2}, P_{1} \in \meaningof{E_{1}}, P_{2} \in \meaningof{E_2}\} }
\end{mathpar}

\begin{mathpar}
 \inferrule* [lab=behavior] {} {\meaningof{\langle a?b \rangle E} = \{ P \in \pi | P \equiv Q | u?(y)P', \\ \and \\\\ \and \\ \;\;\; u \in \meaningof{a}, \forall z.P'\{z/y\} \in \meaningof{E\{z/b\}}\}, \and \\ \meaningof{a!E} = \{ P \in \pi | P \equiv Q | x!\langle P' \rangle, x \in \meaningof{a} P' \in \meaningof{E}\} }
\end{mathpar}

\begin{mathpar}
 \inferrule* [lab=nominal] {} {\meaningof{\quotep{E}} = \{ \quotep{P} \in \quotep{\pi} | P \in \meaningof{E} \}, \and \meaningof{\quotep{P}} = \{ \quotep{Q} \in \quotep{\pi} | P \equiv Q \} \and \\ \meaningof{@\quotep{E}} = \{ P \in \pi | P \equiv @x, x \in \meaningof{E} \}}
\end{mathpar}

\begin{eqnarray*}
  \\
  \meaningof{-} : TS \to ST
\end{eqnarray*}

\begin{eqnarray*}
  \\
  L : TS \to ST
\end{eqnarray*}

\begin{eqnarray*}
  \\
  P \models E \iff P \in \meaningof{E}
\end{eqnarray*}

\begin{eqnarray*}
  P \approx_{L} Q \iff \forall E \in L. P \models E \iff Q \models E
\end{eqnarray*}

\begin{eqnarray*}
  P \approx_{K} Q
\end{eqnarray*}

\begin{eqnarray*}
  P \approx Q
\end{eqnarray*}

$\approx_{K} = \approx = \approx_{L}$

\subsubsection{Contextual duality}

Note that contexts extend the quotation operation to a family of
operations from processes to names. Given a context, $M$, we can
define a \emph{nominal context}, $\quotep{M}$ by $\quotep{M}[P] :=
\quotep{M[P]}$. To foreshadow what is to come we observe that these
operations enjoy a duality with processes very much like the duality
between vectors and maps from vectors to scalars.

Further, because the calculus is essentially higher-order, we have a
correspondence between contexts and processes. More specifically,
given a name $x$ and a context $M$ we can construct $M^{*}_{x}$ such
that 

\begin{mathpar}
  M^{*}_{x} | \lift{x}{P} \red M[P]
\end{mathpar}

namely,

\begin{mathpar}
  M^{*}_{x} := x?(u).M[\dropn{u}]
\end{mathpar}

The dependence of $M^{*}_{x}$ on a name makes it an abstraction, 

\begin{mathpar}
  M^{*} := (x)x?(u).M[\dropn{u}]
\end{mathpar}

\subsection{Additional notation}

It will sometimes be convenient to denote the process a name
quotes. We already have the notation $x = \quotep{P}$, but it will be
convenient to introduce an alternate notation, $\procn{x}$, when we
want to emphasize the connection to the use of the name. Note that, by
virtue of name equivalence, $\quotep{\procn{x}} \nameeq x$; so, the
notation is consistent with previous definitions.

Further, because names have structure it is possible to effect
substitutions on the basis of that structure. This means we need to
upgrade our notation for substitutions, which we accomplish by
adapting comprehension notation. Thus,

\begin{mathpar}
  P\{ y / x : x \in S \}
\end{mathpar}

is interpreted to mean the process derived from P by replacing (in a
capture-avoiding manner) each occurrence of $x$ in $S$ by $y$. For example,

\begin{mathpar}
  P\{ \quotep{\procn{x}|\procn{x}} / x : x \in \freenames{P} \}
\end{mathpar}

will replace each (occurrence) of a free name $x$ in $P$ by
$\quotep{\procn{x}|\procn{x}}$.

Also, we will avail ourselves of the notation $x^{L}$ and $x^{R}$ to
denote injections of a name into disjoint copies of the name
space. There are numerous ways to accomplish this. One example can be
found in \cite{MeredithR05}. This notation overloads to vectors of
names: $\vec{x}^{\pi} := (x_{i}^{\pi} \; : \; 0 \leq i < |\vec{x}| )$ where $\pi \in \{L,R\}$.

We also use $P^{\Box} := P|\Box$.

In \cite{MeredithR05} an interpretation of the new operator is
given. It turns out that there are several possible interpretations
all enjoying the requisite algebraic properties of the operator (see
\cite{milner91polyadicpi}). We will therefore make liberal use of
$(\nu\; \vec{x})P$.

% subsection the_syntax_and_semantics_of_the_notation_system (end)   

\input{qm2pi.qmops} 

\input{qm2pi.sterngerlach} 

\input{qm2pi.metric} 

% section concurrent_process_calculi (end)

%\input{qm2pi.proofsketch}

% section proof sketch (end)

%\input{qm2pi.slviaknots} 

% section spatial logic via knots (end)

\input{qm2pi.conclusion}

% section conclusion (end)

%\input{qm2pi.dtcodes} 

% section wiring algorithm (end)

\input{qm2pi.ack} 

% section acknowledgments (end)

\newpage


\bibliographystyle{plain}   
\bibliography{../../biblios/main.bib}

\input{qm2pi.rhodetails}

\end{document}



% section front matter (end)

\section{Introduction}\label{sec:introduction} % (fold)
In this draft of the material i am going to have to dispense with the
usual writing conventions adopted in papers on these topics. i'm going
to have adopt whatever tone i need at the time i'm writing up the
calculations. Sometimes this may be very conversational; others it may
be the barest mathematical grunts; others still it may be that i have
lifted text from one of my other papers because the exposition of some
point was better said there. i hope that my readers are not unduly put
out by this decision. i'm not doing this to flout convention or be
rebellious. i find these calculations very technically challenging. To
keep everything going technically, something has to give; i have to
let go of some cognitive burden. So, the academic writing style --
with all of its trade-offs in terms of facilitating technical
communication -- is what i'm letting go of. Perhaps subsequent drafts
can be tightened and polished, but for now, i'm going to speak as if
we were sitting together in a coffee shop with a laptop, wifi and a
pad of paper and a pencil.

So, here's what i have to say. We -- you and i, comfortably ensconced
in our coffee shop and well-equipped with our tools -- can realize and
carry out the calculations of quantum mechanics over a very different
formal theory of dynamics, a formal theory of dynamics that
corresponds to a theory of concurrent computation with
\emph{reflection}. It has the advantage that the underlying theory is
already `quantized', but supports analogues all of the continuuous
operations. Strikingly, this underlying theory has recently been
connected with a notion of metric that we can show, by calculating
together, coincides with the metric induced by the inner product.

There are a lot of reasons why you might be interested in seeing
calculations of this form. Here's why i'm interested. For the past
several centuries there has been no competitor to the ``Newtonian''
account of dynamics. As a result the predominant share of accounts of
dynamical systems and situations have had to be formulated in terms of
the Newtonian machinery. i view this as an intellectually dangerous
position to occupy. Everything, despite it's intrinsic shape, turns
into a nail to be hit with this hammer. Recently, however, the theory
of computation has matured to the point where we have candidates for
theories of dynamics that offer very different perspective on
reasoning about dynamical systems and situations. Testing these
candidates against very successful accounts of dynamical situations,
like quantum mechanics, is going to give us some sense of how mature
they are and some measure of the quality of these accounts of
dynamics.

\subsection{Summary of contributions and outline of paper}

So, we're going to develop an interpretation of the operations of
quantum mechanics normally interpreted by Hilbert spaces and
operators. We're going to do this over a theory of computation. Note
that this is very different than the usual quantum computation program
which develops notions of computation over quantum mechanics. Rather,
we are developing a story that aligns with Wheeler's slogan: It from
Bit. To do this we will first provide an account of the theory of
computation at play here. Then we will dive into a calculation-driven
interpretation of the operations of quantum mechanics.

The reason we take this approach is that -- until very recently --
there hasn't been an axiomatic account of quantum mechanics. As a
result there has been no sharp delineation of the mathematical theory
supporting interpretation of the physical theory and the physical
theory, itself. So, ambient features of the maths are free to be
exploited (or supressed) without a real accounting of their physical
relevance. There is no sharp statement ``here's the physical theory''
qua \emph{theory} and ``here's the mathematical interpretation''
enabling a judgment of how faithful the interpretation is -- apart
from experimental observation. When there is an axiomatic account we
can judge how well a given mathematical formalism supports an
interpretation of the axioms, independent of
experimentation. Likewise, we can judge how well we have captured our
physical evidence and experience with our axiomatics, independent of
any specific mathematical implementation, with accidental detail that
may or may not have physical significance. 

In lieu of a fully fleshed out and vetted axiomatic account of quantum
mechanics, interpreting the operational notions in service of modeling
physical systems will have to suffice. In other words, we are not in
the business of providing a model of Hilbert spaces and operators. We
are in the business of providing a model of quantum mechanics because
we are motivated by testing our notions of dynamics against physical
theory; and, the predictive calculations of the physical theory must
serve as the best formulation -- shy of a fully fleshed out axiomatic
account -- of the physical theory itself (as they have for scientific
theories since time immemorial). Put another way, despite a
whole-hearted commitment to an It-from-Bit ontology, we are firmly
aligned with the shut-up-and-calculate camp as the best way to obtain
results either from the physical perspective or as a quality assurance
measure of our fledgling theory of dynamics.

In detail, we present a reflective process calculus. Then we develop
intuitive correspondences between the notions available in this
calculus and the usual physical notions supporting quantum mechanical
calculations. Thus, 

\begin{table}[htp]
  \center{
    \fbox{
      \begin{tabular}{c|c}
        quantum mechanics & process calculus \\
        \hline
        scalar & name \\
        state vector & process \\
        dual & contextual duals \\
        matrix & formal sums of process-context-dual pairs \\
        orthogonality & process annihilation \\
        inner product & execution-formula + quoting
      \end{tabular}
    }
  }
  \caption{QM - process calculi correspondences}
\end{table}

Then we tighten up these intuitions to operational definitions. We
employ the Dirac notation as the best proxy we can find for an
abstract syntax of the quantum mechanical notions. The definitions we
develop put us in contact with equational constraints coming from the
theory that we demonstrate the definitions and calculations satisfy.

This puts us in a position to shut up and calculate for the
Stern-Gerlach experimental set up, showing how these predictive
calculations become calculations on processes in our theory of a
reflective process calculus.

Penultimately, we demonstrate that the notion of metric coming from
the inner product coincides with the notion of metric available from
the theory of bisimulation. This demonstration gives us the right to
think of space as arising from behavior. Finally, we consider where we
might go from the new vantage point we have obtained.

% section introduction (end) 
 
% section introduction (end)

% \documentclass[12pt]{llncs}
%\documentclass{jktr}

\usepackage[pdftex]{hyperref}                   
\usepackage {listings}
\usepackage {mathpartir}
\usepackage{bcprules}
%\usepackage{listings}
                       
\usepackage{graphicx} 
%\usepackage[margins=2.5cm,nohead,nofoot]{geometry}
%\usepackage{geometry}
\usepackage{amsfonts}
\usepackage{amstext}
\usepackage{latexsym}
\usepackage{amssymb}
\usepackage{color}


%\include{myPreamble}
\include{qm2pi.local} 

%\ifpdf
%\usepackage[pdftex]{graphicx}
%\else
%\usepackage{graphicx}
%\fi

 % \ifpdf
%  \usepackage{pdfsync}
%  \if


%\title{Brief Article}
%\author{David F. Snyder}
%\author{L.G. Meredith}

%\address{Dept. of Math., Texas State University--San Marcos, San Marcos, TX 78666}
       
\pagestyle{empty}


\begin{document}

\lstset{language=[Objective]Caml,frame=shadowbox}

\input{qm2pi.front}

% section front matter (end)

\input{qm2pi.intro} 
 
% section introduction (end)

% \input{qm2pi.knotations} 

% section notation (end)

\input{qm2pi.process.calculi} 

% section concurrent_process_calculi_and_spatial_logics_ (end)
    
%\input{qm2pi.knots2pi} 

%\input{qm2pi.trefoil} 

%\input{qm2pi.mainthm} 

% subsection basic_interpretation (end)

%\input{qm2pi.rho.presentation} 
\subsection{The syntax and semantics of the notation system}\label{sub:the_syntax_and_semantics_of_the_notation_system} % (fold)

We now summarize a technical presentation of the calculus that
embodies our theory of dynamics. The typical presentation of such a
calculus follows the style of giving generators and relations on
them. The grammar, below, describing term constructors, freely
generates the set of processes, $\Proc$. This set is then quotiented
by a relation known as structural congruence and it is over this set
that the notion of dynamics is expressed. This presentation is
essentially that of \cite{MeredithR05} with the addition of
polyadicity and summation. For readability we have relegated some of
the technical subtleties to an appendix.

\subsubsection{Process grammar}\label{subsub:process_grammar}

\begin{mathpar}
  \inferrule* [lab=synchronization] {} {{M} \bc \pzero \;|\; x?F \;|\; x!C }
  \and
  \inferrule* [lab=abstraction] {} {{F} \bc (x)P}
  \and
  \inferrule* [lab=concretion] {} {{C} \bc \langle Q \rangle}
  \and
  \inferrule* [lab=process] {} {{P,Q} \bc M \;| \;P|Q \;|\; @{x}}
  \and
  \inferrule* [lab=name] {} {{x} \bc \quotep{P}}
\end{mathpar} 

Note that $\vec{x}$ (resp. $\vec{P}$) denotes a vector of names
(resp. processes) of length $|\vec{x}|$ (resp. $|\vec{P}|$). We adopt
the following useful abbreviations.

\begin{mathpar}
   x?(\vec{y}).P := x.(\vec{y})P \and  x\clift{\vec{P}} := x.\clift{\vec{P}}
   \and x!(y) := \lift{x}{\dropn{y}}
   \and \Pi_{i=0}^{n-1}P_i := P_0 | \ldots | P_{n-1}
\end{mathpar}

\subsubsection{Structural congruence}

\paragraph{Free and bound names and alpha-equivalence.} At the
core of structural equivalence is alpha-equivalence which identifies
process that are the same up to a change of variable. Formally, we
recognize the distinction between free and bound names. The free names
of a process, $\freenames{P}$, may be calculated recursively as
follows:

\begin{mathpar}
\freenames{\pzero} := \emptyset
  \and \\
  \freenames{x?(y).P} := \{ x \} \cup (\freenames{P} \setminus \{ y \})
  \and 
  \freenames{x!\langle P \rangle} := \{ x \} \cup \{ P \} 
  \and \\
  \freenames{P|Q} := \freenames{P} \cup \freenames{Q}
  \and \\
  \freenames{@{x}} := \{ x \}
\end{mathpar}

$\pi$
$\quotep{\pi}$

$\freenames{-} : \pi \to \mathcal{P}(\quotep{\pi})$

\begin{eqnarray*}
  \freenames{\pzero} & := & \emptyset \\
  \freenames{x?(y).P} & := & \{ x \} \cup (\freenames{P} \setminus \{ y \}) \\
  \freenames{x!\langle P \rangle} & := & \{ x \} \cup \{ P \} \\
  \freenames{P|Q} & := & \freenames{P} \cup \freenames{Q} \\
  \freenames{\dropn{x}} & := & \{ x \}
\end{eqnarray*}

The bound names of a process, $\boundnames{P}$, are those names occurring in $P$
that are not free. For example, in $x?(y).0$, the name $x$ is free, while $y$ is bound.

\begin{mathpar}
  \inferrule* [lab=monoidal-laws] {} { P|Q \equiv Q|P \and P|0 \equiv P \and P|(Q|R) \equiv (P|Q)|R }
\end{mathpar}

\begin{mathpar}
  \inferrule* [lab=alpha-equivalence] {} { (x)P \equiv (y)P\{y/x\} \and y \not\in \freenames{P} }
\end{mathpar}

\begin{definition}
Then two processes, $P,Q$, are alpha-equivalent if $P = Q\{\vec{y}/\vec{x}\}$ for
some $\vec{x} \in \boundnames{Q},\vec{y} \in \boundnames{P}$, where $Q\{\vec{y}/\vec{x}\}$
denotes the capture-avoiding substitution of $\vec{y}$ for $\vec{x}$ in $Q$.
\end{definition}

\begin{definition}
  The {\em structural congruence} \cite{SangiorgiWalker} , $\equiv$,
  between processes is the least congruence containing
  alpha-equivalence, satisfying the abelian monoid laws
  (associativity, commutativity and $\pzero$ as identity) for parallel
  composition $|$ and for summation $+$.
\end{definition}

\subsection{Name equivalence}

We take name equivalence, written $\nameeq$, to be the smallest
equivalence relation generated by the following rules.

\begin{mathpar}
\inferrule*[lab=Quote-drop]
{ }
{ \quotep{@{x}} \nameeq x }

\inferrule*[lab=Struct-equiv]
{ P \scong Q }
{ \quotep{P} \nameeq \quotep{Q} }
\end{mathpar}

The astute reader will have noticed that the mutual recursion of names
and processes imposes a mutual recursion on alpha-equivalence and
structural equivalence via name-equivalence. Fortunately, all of this
works out pleasantly and we may calculate in the natural way, free of
concern. The reader interested in the details is referred to the
appendix \ref{appendix:rho_details}.

\subsection{Substitution}

We use $\Proc$ for the set of processes, $\QProc$ for the set of
names, and $\id{\{}\vec{y} / \vec{x} \id{\}}$ to denote partial maps,
$s : \QProc \rightarrow \QProc$. A map, $s$ lifts, uniquely, to a map
on process terms, $\widehat{s} : \Proc \rightarrow \Proc$ by the
following equations.

\begin{mathpar}
  (0) \psubstp{Q}{P} := 0 \\
  (R \juxtap S) \psubstp{Q}{P}
  :=    
  (R)\psubstp{Q}{P} \juxtap (S) \psubstp{Q}{P} \\
  (x?(y).R) \psubstp{Q}{P}    
  :=    
  (x)\substp{Q}{P} (z)\concat( (R \psubstn{z}{y}) \psubstp{Q}{P} ) \\
  (\lift{x}{R}) \psubstp{Q}{P}  
  :=
  \lift{(x)\substp{Q}{P}}{ R \psubstp{Q}{P} } \\
%   (\dropn{x})  \psubstp{Q}{P}       
%   := 
%   \left\{ 
%     \begin{array}{ccc} 
%       \dropn{\quotep{Q}} & & x \nameeq \quotep{P} \\
%       \dropn{x} & & otherwise \\
%     \end{array}
%   \right. 
  (\dropn{x})  \psubstp{Q}{P}       
  := 
  \left\{ 
    \begin{array}{ccc} 
      Q & & x \nameeq \quotep{P} \\
      \dropn{x} & & otherwise \\
    \end{array}
  \right.
\end{mathpar}
 

where

\begin{eqnarray}
  (x)\id{\{} \lpquote Q \rpquote / \lpquote P \rpquote \id{\}}            = 
  \left\{ 
    \begin{array}{ccc}
      \lpquote Q \rpquote & & x \nameeq \lpquote P \rpquote \\
      x & & otherwise \\
    \end{array}
  \right. \nonumber
\end{eqnarray}

and $z$ is chosen distinct from $\quotep{P}$, $\quotep{Q}$, the free
names in $Q$, and all the names in $R$. Our $\alpha$-equivalence will
be built in the standard way from this substitution.

\begin{remark}\label{rem:no_self_referential_names}
  One consequence of these definitions is that $\forall P. \quotep{P}
  \not\in \freenames{P}$.
\end{remark}

\subsection{ Dynamic quote: an example }

Anticipating something of what's to come, consider applying the
substitution, $\widehat{\id{\{}u / z \id{\}}}$, to the following pair
of processes, $\lift{w}{y!(z)}$ and $w[ \lpquote y!(z) \rpquote ]$.

\begin{eqnarray}
	\lift{w}{y!(z)}\widehat{\id{\{}u / z \id{\}}}
		& = &
		\lift{w}{y!(u)} \nonumber\\
	w[ \lpquote y!(z) \rpquote ] \widehat{ \id{\{}u / z \id{\}} }
		& = &
		w[ \lpquote y!(z) \rpquote ] \nonumber
\end{eqnarray}

Because the body of the process between quotes is impervious to
substitution, we get radically different answers. In fact, by
examining the first process in an input context,
e.g. $x?(z).\lift{w}{y!(z)}$, we see that the process under the lift
operator may be shaped by prefixed inputs binding a name inside it. In
this sense, the lift operator will be seen as a way to dynamically
construct processes before reifying them as names.

Finally equipped with these standard features we can present the
dynamics of the calculus.

\subsubsection{Operational semantics} 

Finally, we introduce the computational dynamics. What marks these
algebras as distinct from other more traditionally studied algebraic
structures, e.g. vector spaces or polynomial rings, is the manner in
which dynamics is captured. In traditional structures, dynamics is typically
expressed through morphisms between such structures, as in linear maps
between vector spaces or morphisms between rings. In algebras
associated with the semantics of computation, the dynamics is
expressed as part of the algebraic structure itself, through a
reduction reduction relation typically denoted by $\red$. Below, we
give a recursive presentation of this relation for the calculus used
in the encoding.

$\red \subseteq \pi \times \pi$
$\red : \pi \to \mathcal{P}(\pi)$

\begin{mathpar}
  \inferrule* [lab=Comm] { \textsf{match}( x_{src}, x_{trgt} ) } { x_{trgt}?(y)P \; | \; x_{src}!\langle {Q} \rangle \red P\{\quotep{Q}/y}\} }
  \and \\
  \inferrule* [lab=Par] {{P} \red {P}'} {{{P} | {Q}} \red {{P}' | {Q}}}
  \and
  \inferrule* [lab=Equiv]{{{P} \scong {P}'} \andalso {{P}' \red {Q}'} \andalso {{Q}' \scong {Q}}}{{P} \red {Q}}
\end{mathpar}

\begin{eqnarray*}
  match_{\equiv} (\quotep{P},\quotep{Q}) & := & P \equiv Q \\
  match_{\dagger}(\quotep{P},\quotep{Q}) & := & \forall R. P|Q \red^{*} R => R \red^{*} 0 \\
  match_{K}(\quotep{P},\quotep{Q}) & := & K \mbox{ for some context } K
\end{eqnarray*}

$u?(x)P | u!\langle Q \rangle \red P\{\quotep{Q}/x\}$

%We write $\wred$ for $\red^*$, and $P\red$ if $\exists Q $ such that $ P \red Q$.
We write $P\red$ if $\exists Q $ such that $ P \red Q$ and $P\not\red$, otherwise.

\section{Replication}

As mentioned before, it is known that replication (and hence
recursion) can be implemented in a higher-order process algebra
\cite{SangiorgiWalker}. As our first example of calculation with the
machinery thus far presented we give the construction explicitly in
the {\rhoc}.

\begin{eqnarray}
	D_{x} & := & \prefix{x}{y}{(\binpar{\outputp{x}{y}}{@{y}})} \nonumber\\
	\bangp_{x}{P} & := & \binpar{{x}!\langle{\binpar{D_{x}}{P}}\rangle}{D_{x}} \nonumber
\end{eqnarray}

\begin{eqnarray}
	\bangp_{x}{P} & & \nonumber\\
	=
	& {x}!\langle{(\prefix{x}{y}{(\outputp{x}{y} | @{y})) | P}}\rangle 
	      | \prefix{x}{y}{(\outputp{x}{y} | @{y})} & \nonumber\\
	\red
	& (\outputp{x}{y} | @{y})\substn{\quotep{(\prefix{x}{y}{(@{y} | \outputp{x}{y})) | P}}}{y} & \nonumber\\
	=
	& \outputp{x}{\quotep{(\prefix{x}{y}{(\outputp{x}{y} | @{y})) | P}}}
	  | {(\prefix{x}{y}{(\outputp{x}{y} | @{y})) | P}} & \nonumber\\
	\red
	& \ldots & \nonumber\\
	\red^*
	& P | P | \ldots & \nonumber
\end{eqnarray}

Of course, this encoding, as an implementation, runs away, unfolding
$\bangp{P}$ eagerly. A lazier and more implementable replication
operator, restricted to input-guarded processes, may be obtained as follows.

\begin{eqnarray}
\bangp{\prefix{u}{v}{P}} 
	:= 
	\binpar{\lift{x}{\prefix{u}{v}{(\binpar{D(x)}{P})}}}{D(x)} \nonumber
\end{eqnarray}

\begin{remark}
  Note that the lazier definition still does not deal with summation
  or mixed summation (i.e. sums over input and output). The reader is
  invited to construct definitions of replication that deal with these
  features. 

  Further, the definitions are parameterized in a name, $x$. Can you,
  gentle reader, make a definition that eliminates this parameter and
  guarantees no accidental interaction between the replication
  machinery and the process being replicated -- i.e. no accidental
  sharing of names used by the process to get its work done and the
  name(s) used by the replication to effect copying. This latter
  revision of the definition of replication is crucial to obtaining
  the expected identity $!!P \sim !P$.
\end{remark}

\begin{remark}\label{rem:paradoxical_combinator}
  The reader familiar with the lambda calculus will have noticed the
  similarity between $D$ and the paradoxical combinator.

  [Ed. note: the existence of this seems to suggest we have to be more
  restrictive on the set of processes and names we admit if we are to
  support no-cloning.]
\end{remark}

\subsubsection{Bisimulation}

The computational dynamics gives rise to another kind of equivalence,
the equivalence of computational behavior. As previously mentioned
this is typically captured \emph{via} some form of bisimulation.

% The notion we use in this paper is weak barbed bisimulation
% \cite{milner91polyadicpi}.

The notion we use in this paper is derived from weak barbed
bisimulation \cite{milner91polyadicpi}. 

\begin{definition}
An \emph{observation relation}, $\downarrow_{\mathcal N}$, over a set
of names, $\mathcal N$, is the smallest relation satisfying the rules
below.

\infrule[Out-barb]{y \in {\mathcal N}, \; x \nameeq y}
		  {\outputp{x}{v} \downarrow_{\mathcal N} x}
\infrule[Par-barb]{\mbox{$P\downarrow_{\mathcal N} x$ or $Q\downarrow_{\mathcal N} x$}}
		  {\binpar{P}{Q} \downarrow_{\mathcal N} x}

We write $P \Downarrow_{\mathcal N} x$ if there is $Q$ such that 
$P \wred Q$ and $Q \downarrow_{\mathcal N} x$.
\end{definition}

\begin{definition}
%\label{def.bbisim}
An  ${\mathcal N}$-\emph{barbed bisimulation} over a set of names, ${\mathcal N}$, is a symmetric binary relation 
${\mathcal S}_{\mathcal N}$ between agents such that $P\rel{S}_{\mathcal N}Q$ implies:
\begin{enumerate}
\item If $P \red P'$ then $Q \wred Q'$ and $P'\rel{S}_{\mathcal N} Q'$.
\item If $P\downarrow_{\mathcal N} x$, then $Q\Downarrow_{\mathcal N} x$.
\end{enumerate}
$P$ is ${\mathcal N}$-barbed bisimilar to $Q$, written
$P \wbbisim_{\mathcal N} Q$, if $P \rel{S}_{\mathcal N} Q$ for some ${\mathcal N}$-barbed bisimulation ${\mathcal S}_{\mathcal N}$.
\end{definition}

$\mathcal{R} \subseteq \pi \times \pi$

$P \mathcal{R} Q => \forall P'. P \red P' \Rightarrow \exists Q'. Q \red Q', P' \mathcal{R} Q'$

$P \vdash x \Rightarrow Q \vdash x$

\begin{mathpar}
  \inferrule*[lab=Out-barb]{x \nameeq y}{{y}!\langle{Q}\rangle \vdash x}
  \and
  \inferrule*[lab=Par-barb]{\mbox{$P\vdash x$ or $Q\vdash x$}}{\binpar{P}{Q} \vdash x}
\end{mathpar}

\subsubsection{Contexts}

One of the principle advantages of computational calculi like the
$\pi$-calculus is a well-defined notion of context,
contextual-equivalence and a correlation between
contextual-equivalence and notions of bisimulation. The notion of
context allows the decomposition of a process into (sub-)process and
its syntactic environment, its context. Thus, a context may be
thought of as a process with a ``hole'' (written $\Box$) in it. The
application of a context $M$ to a process $P$, written $M[P]$, is
tantamount to filling the hole in $M$ with $P$. In this paper we do
not need the full weight of this theory, but do make use of the notion
of context in the proof the main theorem. 

\begin{mathpar}
  \inferrule* [lab=summation] {} {{M_{M},M_{N}} \bc \Box \;|\; x.M_{A} \;|\; M_{M}+M_{N}}
  \and
  \inferrule* [lab=agent] {} {{M_{A}} \bc (\vec{x})M_{P} \;| \; \clift{P_0,\ldots,M_{P},\ldots,P_N}}
  \and \\
  \inferrule* [lab=process] {} {{M_{P}} \bc M_{N} \;| \;P|M_{P} }
\end{mathpar} 

\begin{mathpar}
  \inferrule* [lab=sychronization] {} {M_{N} \bc \Box \;|\; x?M_{F} \;|\; x!M_{C}}
  \and
  \inferrule* [lab=abstraction] {} {{M_{F}} \bc (x)M_{P} }
  \and
  \inferrule* [lab=concretion] {} {{M_{C}} \bc \langle M_{P} \rangle }
  \and \\
  \inferrule* [lab=process] {} {{M_{P}} \bc M_{N} \;| \;P|M_{P} }
\end{mathpar}

\begin{definition}[contextual application] Given a context $M$, and
  process $P$, we define the \emph{contextual application}, $M[P] :=
  M\{P/\Box\}$. That is, the contextual application of M to P is the
  substitution of $P$ for $\Box$ in $M$.
\end{definition}

$\meaningof{-} : L \to \mathcal{P}(\pi)$

\begin{mathpar}
  \inferrule* [lab=collection] {} {\meaningof{true} = \pi, \and \meaningof{~E} = \pi \setminus \meaningof{E}, \and \meaningof{E_{1} \& E_{2}} = \meaningof{E_{1}} \cap \meaningof{E_{2}}}
\end{mathpar}

\begin{mathpar}
  \inferrule* [lab=structure] {} {\meaningof{0} = \{ P \in \pi | P \equiv 0 \}, \and \\ \meaningof{E_1 | E_2} = \{ P \in \pi | P \equiv P_{1} | P_{2}, P_{1} \in \meaningof{E_{1}}, P_{2} \in \meaningof{E_2}\} }
\end{mathpar}

\begin{mathpar}
 \inferrule* [lab=behavior] {} {\meaningof{\langle a?b \rangle E} = \{ P \in \pi | P \equiv Q | u?(y)P', \\ \and \\\\ \and \\ \;\;\; u \in \meaningof{a}, \forall z.P'\{z/y\} \in \meaningof{E\{z/b\}}\}, \and \\ \meaningof{a!E} = \{ P \in \pi | P \equiv Q | x!\langle P' \rangle, x \in \meaningof{a} P' \in \meaningof{E}\} }
\end{mathpar}

\begin{mathpar}
 \inferrule* [lab=nominal] {} {\meaningof{\quotep{E}} = \{ \quotep{P} \in \quotep{\pi} | P \in \meaningof{E} \}, \and \meaningof{\quotep{P}} = \{ \quotep{Q} \in \quotep{\pi} | P \equiv Q \} \and \\ \meaningof{@\quotep{E}} = \{ P \in \pi | P \equiv @x, x \in \meaningof{E} \}}
\end{mathpar}

\begin{eqnarray*}
  \\
  \meaningof{-} : TS \to ST
\end{eqnarray*}

\begin{eqnarray*}
  \\
  L : TS \to ST
\end{eqnarray*}

\begin{eqnarray*}
  \\
  P \models E \iff P \in \meaningof{E}
\end{eqnarray*}

\begin{eqnarray*}
  P \approx_{L} Q \iff \forall E \in L. P \models E \iff Q \models E
\end{eqnarray*}

\begin{eqnarray*}
  P \approx_{K} Q
\end{eqnarray*}

\begin{eqnarray*}
  P \approx Q
\end{eqnarray*}

$\approx_{K} = \approx = \approx_{L}$

\subsubsection{Contextual duality}

Note that contexts extend the quotation operation to a family of
operations from processes to names. Given a context, $M$, we can
define a \emph{nominal context}, $\quotep{M}$ by $\quotep{M}[P] :=
\quotep{M[P]}$. To foreshadow what is to come we observe that these
operations enjoy a duality with processes very much like the duality
between vectors and maps from vectors to scalars.

Further, because the calculus is essentially higher-order, we have a
correspondence between contexts and processes. More specifically,
given a name $x$ and a context $M$ we can construct $M^{*}_{x}$ such
that 

\begin{mathpar}
  M^{*}_{x} | \lift{x}{P} \red M[P]
\end{mathpar}

namely,

\begin{mathpar}
  M^{*}_{x} := x?(u).M[\dropn{u}]
\end{mathpar}

The dependence of $M^{*}_{x}$ on a name makes it an abstraction, 

\begin{mathpar}
  M^{*} := (x)x?(u).M[\dropn{u}]
\end{mathpar}

\subsection{Additional notation}

It will sometimes be convenient to denote the process a name
quotes. We already have the notation $x = \quotep{P}$, but it will be
convenient to introduce an alternate notation, $\procn{x}$, when we
want to emphasize the connection to the use of the name. Note that, by
virtue of name equivalence, $\quotep{\procn{x}} \nameeq x$; so, the
notation is consistent with previous definitions.

Further, because names have structure it is possible to effect
substitutions on the basis of that structure. This means we need to
upgrade our notation for substitutions, which we accomplish by
adapting comprehension notation. Thus,

\begin{mathpar}
  P\{ y / x : x \in S \}
\end{mathpar}

is interpreted to mean the process derived from P by replacing (in a
capture-avoiding manner) each occurrence of $x$ in $S$ by $y$. For example,

\begin{mathpar}
  P\{ \quotep{\procn{x}|\procn{x}} / x : x \in \freenames{P} \}
\end{mathpar}

will replace each (occurrence) of a free name $x$ in $P$ by
$\quotep{\procn{x}|\procn{x}}$.

Also, we will avail ourselves of the notation $x^{L}$ and $x^{R}$ to
denote injections of a name into disjoint copies of the name
space. There are numerous ways to accomplish this. One example can be
found in \cite{MeredithR05}. This notation overloads to vectors of
names: $\vec{x}^{\pi} := (x_{i}^{\pi} \; : \; 0 \leq i < |\vec{x}| )$ where $\pi \in \{L,R\}$.

We also use $P^{\Box} := P|\Box$.

In \cite{MeredithR05} an interpretation of the new operator is
given. It turns out that there are several possible interpretations
all enjoying the requisite algebraic properties of the operator (see
\cite{milner91polyadicpi}). We will therefore make liberal use of
$(\nu\; \vec{x})P$.

% subsection the_syntax_and_semantics_of_the_notation_system (end)   

\input{qm2pi.qmops} 

\input{qm2pi.sterngerlach} 

\input{qm2pi.metric} 

% section concurrent_process_calculi (end)

%\input{qm2pi.proofsketch}

% section proof sketch (end)

%\input{qm2pi.slviaknots} 

% section spatial logic via knots (end)

\input{qm2pi.conclusion}

% section conclusion (end)

%\input{qm2pi.dtcodes} 

% section wiring algorithm (end)

\input{qm2pi.ack} 

% section acknowledgments (end)

\newpage


\bibliographystyle{plain}   
\bibliography{../../biblios/main.bib}

\input{qm2pi.rhodetails}

\end{document}

 

% section notation (end)

\input{qm2pi.process.calculi} 

% section concurrent_process_calculi_and_spatial_logics_ (end)
    
%\documentclass[12pt]{llncs}
%\documentclass{jktr}

\usepackage[pdftex]{hyperref}                   
\usepackage {listings}
\usepackage {mathpartir}
\usepackage{bcprules}
%\usepackage{listings}
                       
\usepackage{graphicx} 
%\usepackage[margins=2.5cm,nohead,nofoot]{geometry}
%\usepackage{geometry}
\usepackage{amsfonts}
\usepackage{amstext}
\usepackage{latexsym}
\usepackage{amssymb}
\usepackage{color}


%\include{myPreamble}
\include{qm2pi.local} 

%\ifpdf
%\usepackage[pdftex]{graphicx}
%\else
%\usepackage{graphicx}
%\fi

 % \ifpdf
%  \usepackage{pdfsync}
%  \if


%\title{Brief Article}
%\author{David F. Snyder}
%\author{L.G. Meredith}

%\address{Dept. of Math., Texas State University--San Marcos, San Marcos, TX 78666}
       
\pagestyle{empty}


\begin{document}

\lstset{language=[Objective]Caml,frame=shadowbox}

\input{qm2pi.front}

% section front matter (end)

\input{qm2pi.intro} 
 
% section introduction (end)

% \input{qm2pi.knotations} 

% section notation (end)

\input{qm2pi.process.calculi} 

% section concurrent_process_calculi_and_spatial_logics_ (end)
    
%\input{qm2pi.knots2pi} 

%\input{qm2pi.trefoil} 

%\input{qm2pi.mainthm} 

% subsection basic_interpretation (end)

%\input{qm2pi.rho.presentation} 
\subsection{The syntax and semantics of the notation system}\label{sub:the_syntax_and_semantics_of_the_notation_system} % (fold)

We now summarize a technical presentation of the calculus that
embodies our theory of dynamics. The typical presentation of such a
calculus follows the style of giving generators and relations on
them. The grammar, below, describing term constructors, freely
generates the set of processes, $\Proc$. This set is then quotiented
by a relation known as structural congruence and it is over this set
that the notion of dynamics is expressed. This presentation is
essentially that of \cite{MeredithR05} with the addition of
polyadicity and summation. For readability we have relegated some of
the technical subtleties to an appendix.

\subsubsection{Process grammar}\label{subsub:process_grammar}

\begin{mathpar}
  \inferrule* [lab=synchronization] {} {{M} \bc \pzero \;|\; x?F \;|\; x!C }
  \and
  \inferrule* [lab=abstraction] {} {{F} \bc (x)P}
  \and
  \inferrule* [lab=concretion] {} {{C} \bc \langle Q \rangle}
  \and
  \inferrule* [lab=process] {} {{P,Q} \bc M \;| \;P|Q \;|\; @{x}}
  \and
  \inferrule* [lab=name] {} {{x} \bc \quotep{P}}
\end{mathpar} 

Note that $\vec{x}$ (resp. $\vec{P}$) denotes a vector of names
(resp. processes) of length $|\vec{x}|$ (resp. $|\vec{P}|$). We adopt
the following useful abbreviations.

\begin{mathpar}
   x?(\vec{y}).P := x.(\vec{y})P \and  x\clift{\vec{P}} := x.\clift{\vec{P}}
   \and x!(y) := \lift{x}{\dropn{y}}
   \and \Pi_{i=0}^{n-1}P_i := P_0 | \ldots | P_{n-1}
\end{mathpar}

\subsubsection{Structural congruence}

\paragraph{Free and bound names and alpha-equivalence.} At the
core of structural equivalence is alpha-equivalence which identifies
process that are the same up to a change of variable. Formally, we
recognize the distinction between free and bound names. The free names
of a process, $\freenames{P}$, may be calculated recursively as
follows:

\begin{mathpar}
\freenames{\pzero} := \emptyset
  \and \\
  \freenames{x?(y).P} := \{ x \} \cup (\freenames{P} \setminus \{ y \})
  \and 
  \freenames{x!\langle P \rangle} := \{ x \} \cup \{ P \} 
  \and \\
  \freenames{P|Q} := \freenames{P} \cup \freenames{Q}
  \and \\
  \freenames{@{x}} := \{ x \}
\end{mathpar}

$\pi$
$\quotep{\pi}$

$\freenames{-} : \pi \to \mathcal{P}(\quotep{\pi})$

\begin{eqnarray*}
  \freenames{\pzero} & := & \emptyset \\
  \freenames{x?(y).P} & := & \{ x \} \cup (\freenames{P} \setminus \{ y \}) \\
  \freenames{x!\langle P \rangle} & := & \{ x \} \cup \{ P \} \\
  \freenames{P|Q} & := & \freenames{P} \cup \freenames{Q} \\
  \freenames{\dropn{x}} & := & \{ x \}
\end{eqnarray*}

The bound names of a process, $\boundnames{P}$, are those names occurring in $P$
that are not free. For example, in $x?(y).0$, the name $x$ is free, while $y$ is bound.

\begin{mathpar}
  \inferrule* [lab=monoidal-laws] {} { P|Q \equiv Q|P \and P|0 \equiv P \and P|(Q|R) \equiv (P|Q)|R }
\end{mathpar}

\begin{mathpar}
  \inferrule* [lab=alpha-equivalence] {} { (x)P \equiv (y)P\{y/x\} \and y \not\in \freenames{P} }
\end{mathpar}

\begin{definition}
Then two processes, $P,Q$, are alpha-equivalent if $P = Q\{\vec{y}/\vec{x}\}$ for
some $\vec{x} \in \boundnames{Q},\vec{y} \in \boundnames{P}$, where $Q\{\vec{y}/\vec{x}\}$
denotes the capture-avoiding substitution of $\vec{y}$ for $\vec{x}$ in $Q$.
\end{definition}

\begin{definition}
  The {\em structural congruence} \cite{SangiorgiWalker} , $\equiv$,
  between processes is the least congruence containing
  alpha-equivalence, satisfying the abelian monoid laws
  (associativity, commutativity and $\pzero$ as identity) for parallel
  composition $|$ and for summation $+$.
\end{definition}

\subsection{Name equivalence}

We take name equivalence, written $\nameeq$, to be the smallest
equivalence relation generated by the following rules.

\begin{mathpar}
\inferrule*[lab=Quote-drop]
{ }
{ \quotep{@{x}} \nameeq x }

\inferrule*[lab=Struct-equiv]
{ P \scong Q }
{ \quotep{P} \nameeq \quotep{Q} }
\end{mathpar}

The astute reader will have noticed that the mutual recursion of names
and processes imposes a mutual recursion on alpha-equivalence and
structural equivalence via name-equivalence. Fortunately, all of this
works out pleasantly and we may calculate in the natural way, free of
concern. The reader interested in the details is referred to the
appendix \ref{appendix:rho_details}.

\subsection{Substitution}

We use $\Proc$ for the set of processes, $\QProc$ for the set of
names, and $\id{\{}\vec{y} / \vec{x} \id{\}}$ to denote partial maps,
$s : \QProc \rightarrow \QProc$. A map, $s$ lifts, uniquely, to a map
on process terms, $\widehat{s} : \Proc \rightarrow \Proc$ by the
following equations.

\begin{mathpar}
  (0) \psubstp{Q}{P} := 0 \\
  (R \juxtap S) \psubstp{Q}{P}
  :=    
  (R)\psubstp{Q}{P} \juxtap (S) \psubstp{Q}{P} \\
  (x?(y).R) \psubstp{Q}{P}    
  :=    
  (x)\substp{Q}{P} (z)\concat( (R \psubstn{z}{y}) \psubstp{Q}{P} ) \\
  (\lift{x}{R}) \psubstp{Q}{P}  
  :=
  \lift{(x)\substp{Q}{P}}{ R \psubstp{Q}{P} } \\
%   (\dropn{x})  \psubstp{Q}{P}       
%   := 
%   \left\{ 
%     \begin{array}{ccc} 
%       \dropn{\quotep{Q}} & & x \nameeq \quotep{P} \\
%       \dropn{x} & & otherwise \\
%     \end{array}
%   \right. 
  (\dropn{x})  \psubstp{Q}{P}       
  := 
  \left\{ 
    \begin{array}{ccc} 
      Q & & x \nameeq \quotep{P} \\
      \dropn{x} & & otherwise \\
    \end{array}
  \right.
\end{mathpar}
 

where

\begin{eqnarray}
  (x)\id{\{} \lpquote Q \rpquote / \lpquote P \rpquote \id{\}}            = 
  \left\{ 
    \begin{array}{ccc}
      \lpquote Q \rpquote & & x \nameeq \lpquote P \rpquote \\
      x & & otherwise \\
    \end{array}
  \right. \nonumber
\end{eqnarray}

and $z$ is chosen distinct from $\quotep{P}$, $\quotep{Q}$, the free
names in $Q$, and all the names in $R$. Our $\alpha$-equivalence will
be built in the standard way from this substitution.

\begin{remark}\label{rem:no_self_referential_names}
  One consequence of these definitions is that $\forall P. \quotep{P}
  \not\in \freenames{P}$.
\end{remark}

\subsection{ Dynamic quote: an example }

Anticipating something of what's to come, consider applying the
substitution, $\widehat{\id{\{}u / z \id{\}}}$, to the following pair
of processes, $\lift{w}{y!(z)}$ and $w[ \lpquote y!(z) \rpquote ]$.

\begin{eqnarray}
	\lift{w}{y!(z)}\widehat{\id{\{}u / z \id{\}}}
		& = &
		\lift{w}{y!(u)} \nonumber\\
	w[ \lpquote y!(z) \rpquote ] \widehat{ \id{\{}u / z \id{\}} }
		& = &
		w[ \lpquote y!(z) \rpquote ] \nonumber
\end{eqnarray}

Because the body of the process between quotes is impervious to
substitution, we get radically different answers. In fact, by
examining the first process in an input context,
e.g. $x?(z).\lift{w}{y!(z)}$, we see that the process under the lift
operator may be shaped by prefixed inputs binding a name inside it. In
this sense, the lift operator will be seen as a way to dynamically
construct processes before reifying them as names.

Finally equipped with these standard features we can present the
dynamics of the calculus.

\subsubsection{Operational semantics} 

Finally, we introduce the computational dynamics. What marks these
algebras as distinct from other more traditionally studied algebraic
structures, e.g. vector spaces or polynomial rings, is the manner in
which dynamics is captured. In traditional structures, dynamics is typically
expressed through morphisms between such structures, as in linear maps
between vector spaces or morphisms between rings. In algebras
associated with the semantics of computation, the dynamics is
expressed as part of the algebraic structure itself, through a
reduction reduction relation typically denoted by $\red$. Below, we
give a recursive presentation of this relation for the calculus used
in the encoding.

$\red \subseteq \pi \times \pi$
$\red : \pi \to \mathcal{P}(\pi)$

\begin{mathpar}
  \inferrule* [lab=Comm] { \textsf{match}( x_{src}, x_{trgt} ) } { x_{trgt}?(y)P \; | \; x_{src}!\langle {Q} \rangle \red P\{\quotep{Q}/y}\} }
  \and \\
  \inferrule* [lab=Par] {{P} \red {P}'} {{{P} | {Q}} \red {{P}' | {Q}}}
  \and
  \inferrule* [lab=Equiv]{{{P} \scong {P}'} \andalso {{P}' \red {Q}'} \andalso {{Q}' \scong {Q}}}{{P} \red {Q}}
\end{mathpar}

\begin{eqnarray*}
  match_{\equiv} (\quotep{P},\quotep{Q}) & := & P \equiv Q \\
  match_{\dagger}(\quotep{P},\quotep{Q}) & := & \forall R. P|Q \red^{*} R => R \red^{*} 0 \\
  match_{K}(\quotep{P},\quotep{Q}) & := & K \mbox{ for some context } K
\end{eqnarray*}

$u?(x)P | u!\langle Q \rangle \red P\{\quotep{Q}/x\}$

%We write $\wred$ for $\red^*$, and $P\red$ if $\exists Q $ such that $ P \red Q$.
We write $P\red$ if $\exists Q $ such that $ P \red Q$ and $P\not\red$, otherwise.

\section{Replication}

As mentioned before, it is known that replication (and hence
recursion) can be implemented in a higher-order process algebra
\cite{SangiorgiWalker}. As our first example of calculation with the
machinery thus far presented we give the construction explicitly in
the {\rhoc}.

\begin{eqnarray}
	D_{x} & := & \prefix{x}{y}{(\binpar{\outputp{x}{y}}{@{y}})} \nonumber\\
	\bangp_{x}{P} & := & \binpar{{x}!\langle{\binpar{D_{x}}{P}}\rangle}{D_{x}} \nonumber
\end{eqnarray}

\begin{eqnarray}
	\bangp_{x}{P} & & \nonumber\\
	=
	& {x}!\langle{(\prefix{x}{y}{(\outputp{x}{y} | @{y})) | P}}\rangle 
	      | \prefix{x}{y}{(\outputp{x}{y} | @{y})} & \nonumber\\
	\red
	& (\outputp{x}{y} | @{y})\substn{\quotep{(\prefix{x}{y}{(@{y} | \outputp{x}{y})) | P}}}{y} & \nonumber\\
	=
	& \outputp{x}{\quotep{(\prefix{x}{y}{(\outputp{x}{y} | @{y})) | P}}}
	  | {(\prefix{x}{y}{(\outputp{x}{y} | @{y})) | P}} & \nonumber\\
	\red
	& \ldots & \nonumber\\
	\red^*
	& P | P | \ldots & \nonumber
\end{eqnarray}

Of course, this encoding, as an implementation, runs away, unfolding
$\bangp{P}$ eagerly. A lazier and more implementable replication
operator, restricted to input-guarded processes, may be obtained as follows.

\begin{eqnarray}
\bangp{\prefix{u}{v}{P}} 
	:= 
	\binpar{\lift{x}{\prefix{u}{v}{(\binpar{D(x)}{P})}}}{D(x)} \nonumber
\end{eqnarray}

\begin{remark}
  Note that the lazier definition still does not deal with summation
  or mixed summation (i.e. sums over input and output). The reader is
  invited to construct definitions of replication that deal with these
  features. 

  Further, the definitions are parameterized in a name, $x$. Can you,
  gentle reader, make a definition that eliminates this parameter and
  guarantees no accidental interaction between the replication
  machinery and the process being replicated -- i.e. no accidental
  sharing of names used by the process to get its work done and the
  name(s) used by the replication to effect copying. This latter
  revision of the definition of replication is crucial to obtaining
  the expected identity $!!P \sim !P$.
\end{remark}

\begin{remark}\label{rem:paradoxical_combinator}
  The reader familiar with the lambda calculus will have noticed the
  similarity between $D$ and the paradoxical combinator.

  [Ed. note: the existence of this seems to suggest we have to be more
  restrictive on the set of processes and names we admit if we are to
  support no-cloning.]
\end{remark}

\subsubsection{Bisimulation}

The computational dynamics gives rise to another kind of equivalence,
the equivalence of computational behavior. As previously mentioned
this is typically captured \emph{via} some form of bisimulation.

% The notion we use in this paper is weak barbed bisimulation
% \cite{milner91polyadicpi}.

The notion we use in this paper is derived from weak barbed
bisimulation \cite{milner91polyadicpi}. 

\begin{definition}
An \emph{observation relation}, $\downarrow_{\mathcal N}$, over a set
of names, $\mathcal N$, is the smallest relation satisfying the rules
below.

\infrule[Out-barb]{y \in {\mathcal N}, \; x \nameeq y}
		  {\outputp{x}{v} \downarrow_{\mathcal N} x}
\infrule[Par-barb]{\mbox{$P\downarrow_{\mathcal N} x$ or $Q\downarrow_{\mathcal N} x$}}
		  {\binpar{P}{Q} \downarrow_{\mathcal N} x}

We write $P \Downarrow_{\mathcal N} x$ if there is $Q$ such that 
$P \wred Q$ and $Q \downarrow_{\mathcal N} x$.
\end{definition}

\begin{definition}
%\label{def.bbisim}
An  ${\mathcal N}$-\emph{barbed bisimulation} over a set of names, ${\mathcal N}$, is a symmetric binary relation 
${\mathcal S}_{\mathcal N}$ between agents such that $P\rel{S}_{\mathcal N}Q$ implies:
\begin{enumerate}
\item If $P \red P'$ then $Q \wred Q'$ and $P'\rel{S}_{\mathcal N} Q'$.
\item If $P\downarrow_{\mathcal N} x$, then $Q\Downarrow_{\mathcal N} x$.
\end{enumerate}
$P$ is ${\mathcal N}$-barbed bisimilar to $Q$, written
$P \wbbisim_{\mathcal N} Q$, if $P \rel{S}_{\mathcal N} Q$ for some ${\mathcal N}$-barbed bisimulation ${\mathcal S}_{\mathcal N}$.
\end{definition}

$\mathcal{R} \subseteq \pi \times \pi$

$P \mathcal{R} Q => \forall P'. P \red P' \Rightarrow \exists Q'. Q \red Q', P' \mathcal{R} Q'$

$P \vdash x \Rightarrow Q \vdash x$

\begin{mathpar}
  \inferrule*[lab=Out-barb]{x \nameeq y}{{y}!\langle{Q}\rangle \vdash x}
  \and
  \inferrule*[lab=Par-barb]{\mbox{$P\vdash x$ or $Q\vdash x$}}{\binpar{P}{Q} \vdash x}
\end{mathpar}

\subsubsection{Contexts}

One of the principle advantages of computational calculi like the
$\pi$-calculus is a well-defined notion of context,
contextual-equivalence and a correlation between
contextual-equivalence and notions of bisimulation. The notion of
context allows the decomposition of a process into (sub-)process and
its syntactic environment, its context. Thus, a context may be
thought of as a process with a ``hole'' (written $\Box$) in it. The
application of a context $M$ to a process $P$, written $M[P]$, is
tantamount to filling the hole in $M$ with $P$. In this paper we do
not need the full weight of this theory, but do make use of the notion
of context in the proof the main theorem. 

\begin{mathpar}
  \inferrule* [lab=summation] {} {{M_{M},M_{N}} \bc \Box \;|\; x.M_{A} \;|\; M_{M}+M_{N}}
  \and
  \inferrule* [lab=agent] {} {{M_{A}} \bc (\vec{x})M_{P} \;| \; \clift{P_0,\ldots,M_{P},\ldots,P_N}}
  \and \\
  \inferrule* [lab=process] {} {{M_{P}} \bc M_{N} \;| \;P|M_{P} }
\end{mathpar} 

\begin{mathpar}
  \inferrule* [lab=sychronization] {} {M_{N} \bc \Box \;|\; x?M_{F} \;|\; x!M_{C}}
  \and
  \inferrule* [lab=abstraction] {} {{M_{F}} \bc (x)M_{P} }
  \and
  \inferrule* [lab=concretion] {} {{M_{C}} \bc \langle M_{P} \rangle }
  \and \\
  \inferrule* [lab=process] {} {{M_{P}} \bc M_{N} \;| \;P|M_{P} }
\end{mathpar}

\begin{definition}[contextual application] Given a context $M$, and
  process $P$, we define the \emph{contextual application}, $M[P] :=
  M\{P/\Box\}$. That is, the contextual application of M to P is the
  substitution of $P$ for $\Box$ in $M$.
\end{definition}

$\meaningof{-} : L \to \mathcal{P}(\pi)$

\begin{mathpar}
  \inferrule* [lab=collection] {} {\meaningof{true} = \pi, \and \meaningof{~E} = \pi \setminus \meaningof{E}, \and \meaningof{E_{1} \& E_{2}} = \meaningof{E_{1}} \cap \meaningof{E_{2}}}
\end{mathpar}

\begin{mathpar}
  \inferrule* [lab=structure] {} {\meaningof{0} = \{ P \in \pi | P \equiv 0 \}, \and \\ \meaningof{E_1 | E_2} = \{ P \in \pi | P \equiv P_{1} | P_{2}, P_{1} \in \meaningof{E_{1}}, P_{2} \in \meaningof{E_2}\} }
\end{mathpar}

\begin{mathpar}
 \inferrule* [lab=behavior] {} {\meaningof{\langle a?b \rangle E} = \{ P \in \pi | P \equiv Q | u?(y)P', \\ \and \\\\ \and \\ \;\;\; u \in \meaningof{a}, \forall z.P'\{z/y\} \in \meaningof{E\{z/b\}}\}, \and \\ \meaningof{a!E} = \{ P \in \pi | P \equiv Q | x!\langle P' \rangle, x \in \meaningof{a} P' \in \meaningof{E}\} }
\end{mathpar}

\begin{mathpar}
 \inferrule* [lab=nominal] {} {\meaningof{\quotep{E}} = \{ \quotep{P} \in \quotep{\pi} | P \in \meaningof{E} \}, \and \meaningof{\quotep{P}} = \{ \quotep{Q} \in \quotep{\pi} | P \equiv Q \} \and \\ \meaningof{@\quotep{E}} = \{ P \in \pi | P \equiv @x, x \in \meaningof{E} \}}
\end{mathpar}

\begin{eqnarray*}
  \\
  \meaningof{-} : TS \to ST
\end{eqnarray*}

\begin{eqnarray*}
  \\
  L : TS \to ST
\end{eqnarray*}

\begin{eqnarray*}
  \\
  P \models E \iff P \in \meaningof{E}
\end{eqnarray*}

\begin{eqnarray*}
  P \approx_{L} Q \iff \forall E \in L. P \models E \iff Q \models E
\end{eqnarray*}

\begin{eqnarray*}
  P \approx_{K} Q
\end{eqnarray*}

\begin{eqnarray*}
  P \approx Q
\end{eqnarray*}

$\approx_{K} = \approx = \approx_{L}$

\subsubsection{Contextual duality}

Note that contexts extend the quotation operation to a family of
operations from processes to names. Given a context, $M$, we can
define a \emph{nominal context}, $\quotep{M}$ by $\quotep{M}[P] :=
\quotep{M[P]}$. To foreshadow what is to come we observe that these
operations enjoy a duality with processes very much like the duality
between vectors and maps from vectors to scalars.

Further, because the calculus is essentially higher-order, we have a
correspondence between contexts and processes. More specifically,
given a name $x$ and a context $M$ we can construct $M^{*}_{x}$ such
that 

\begin{mathpar}
  M^{*}_{x} | \lift{x}{P} \red M[P]
\end{mathpar}

namely,

\begin{mathpar}
  M^{*}_{x} := x?(u).M[\dropn{u}]
\end{mathpar}

The dependence of $M^{*}_{x}$ on a name makes it an abstraction, 

\begin{mathpar}
  M^{*} := (x)x?(u).M[\dropn{u}]
\end{mathpar}

\subsection{Additional notation}

It will sometimes be convenient to denote the process a name
quotes. We already have the notation $x = \quotep{P}$, but it will be
convenient to introduce an alternate notation, $\procn{x}$, when we
want to emphasize the connection to the use of the name. Note that, by
virtue of name equivalence, $\quotep{\procn{x}} \nameeq x$; so, the
notation is consistent with previous definitions.

Further, because names have structure it is possible to effect
substitutions on the basis of that structure. This means we need to
upgrade our notation for substitutions, which we accomplish by
adapting comprehension notation. Thus,

\begin{mathpar}
  P\{ y / x : x \in S \}
\end{mathpar}

is interpreted to mean the process derived from P by replacing (in a
capture-avoiding manner) each occurrence of $x$ in $S$ by $y$. For example,

\begin{mathpar}
  P\{ \quotep{\procn{x}|\procn{x}} / x : x \in \freenames{P} \}
\end{mathpar}

will replace each (occurrence) of a free name $x$ in $P$ by
$\quotep{\procn{x}|\procn{x}}$.

Also, we will avail ourselves of the notation $x^{L}$ and $x^{R}$ to
denote injections of a name into disjoint copies of the name
space. There are numerous ways to accomplish this. One example can be
found in \cite{MeredithR05}. This notation overloads to vectors of
names: $\vec{x}^{\pi} := (x_{i}^{\pi} \; : \; 0 \leq i < |\vec{x}| )$ where $\pi \in \{L,R\}$.

We also use $P^{\Box} := P|\Box$.

In \cite{MeredithR05} an interpretation of the new operator is
given. It turns out that there are several possible interpretations
all enjoying the requisite algebraic properties of the operator (see
\cite{milner91polyadicpi}). We will therefore make liberal use of
$(\nu\; \vec{x})P$.

% subsection the_syntax_and_semantics_of_the_notation_system (end)   

\input{qm2pi.qmops} 

\input{qm2pi.sterngerlach} 

\input{qm2pi.metric} 

% section concurrent_process_calculi (end)

%\input{qm2pi.proofsketch}

% section proof sketch (end)

%\input{qm2pi.slviaknots} 

% section spatial logic via knots (end)

\input{qm2pi.conclusion}

% section conclusion (end)

%\input{qm2pi.dtcodes} 

% section wiring algorithm (end)

\input{qm2pi.ack} 

% section acknowledgments (end)

\newpage


\bibliographystyle{plain}   
\bibliography{../../biblios/main.bib}

\input{qm2pi.rhodetails}

\end{document}

 

%\documentclass[12pt]{llncs}
%\documentclass{jktr}

\usepackage[pdftex]{hyperref}                   
\usepackage {listings}
\usepackage {mathpartir}
\usepackage{bcprules}
%\usepackage{listings}
                       
\usepackage{graphicx} 
%\usepackage[margins=2.5cm,nohead,nofoot]{geometry}
%\usepackage{geometry}
\usepackage{amsfonts}
\usepackage{amstext}
\usepackage{latexsym}
\usepackage{amssymb}
\usepackage{color}


%\include{myPreamble}
\include{qm2pi.local} 

%\ifpdf
%\usepackage[pdftex]{graphicx}
%\else
%\usepackage{graphicx}
%\fi

 % \ifpdf
%  \usepackage{pdfsync}
%  \if


%\title{Brief Article}
%\author{David F. Snyder}
%\author{L.G. Meredith}

%\address{Dept. of Math., Texas State University--San Marcos, San Marcos, TX 78666}
       
\pagestyle{empty}


\begin{document}

\lstset{language=[Objective]Caml,frame=shadowbox}

\input{qm2pi.front}

% section front matter (end)

\input{qm2pi.intro} 
 
% section introduction (end)

% \input{qm2pi.knotations} 

% section notation (end)

\input{qm2pi.process.calculi} 

% section concurrent_process_calculi_and_spatial_logics_ (end)
    
%\input{qm2pi.knots2pi} 

%\input{qm2pi.trefoil} 

%\input{qm2pi.mainthm} 

% subsection basic_interpretation (end)

%\input{qm2pi.rho.presentation} 
\subsection{The syntax and semantics of the notation system}\label{sub:the_syntax_and_semantics_of_the_notation_system} % (fold)

We now summarize a technical presentation of the calculus that
embodies our theory of dynamics. The typical presentation of such a
calculus follows the style of giving generators and relations on
them. The grammar, below, describing term constructors, freely
generates the set of processes, $\Proc$. This set is then quotiented
by a relation known as structural congruence and it is over this set
that the notion of dynamics is expressed. This presentation is
essentially that of \cite{MeredithR05} with the addition of
polyadicity and summation. For readability we have relegated some of
the technical subtleties to an appendix.

\subsubsection{Process grammar}\label{subsub:process_grammar}

\begin{mathpar}
  \inferrule* [lab=synchronization] {} {{M} \bc \pzero \;|\; x?F \;|\; x!C }
  \and
  \inferrule* [lab=abstraction] {} {{F} \bc (x)P}
  \and
  \inferrule* [lab=concretion] {} {{C} \bc \langle Q \rangle}
  \and
  \inferrule* [lab=process] {} {{P,Q} \bc M \;| \;P|Q \;|\; @{x}}
  \and
  \inferrule* [lab=name] {} {{x} \bc \quotep{P}}
\end{mathpar} 

Note that $\vec{x}$ (resp. $\vec{P}$) denotes a vector of names
(resp. processes) of length $|\vec{x}|$ (resp. $|\vec{P}|$). We adopt
the following useful abbreviations.

\begin{mathpar}
   x?(\vec{y}).P := x.(\vec{y})P \and  x\clift{\vec{P}} := x.\clift{\vec{P}}
   \and x!(y) := \lift{x}{\dropn{y}}
   \and \Pi_{i=0}^{n-1}P_i := P_0 | \ldots | P_{n-1}
\end{mathpar}

\subsubsection{Structural congruence}

\paragraph{Free and bound names and alpha-equivalence.} At the
core of structural equivalence is alpha-equivalence which identifies
process that are the same up to a change of variable. Formally, we
recognize the distinction between free and bound names. The free names
of a process, $\freenames{P}$, may be calculated recursively as
follows:

\begin{mathpar}
\freenames{\pzero} := \emptyset
  \and \\
  \freenames{x?(y).P} := \{ x \} \cup (\freenames{P} \setminus \{ y \})
  \and 
  \freenames{x!\langle P \rangle} := \{ x \} \cup \{ P \} 
  \and \\
  \freenames{P|Q} := \freenames{P} \cup \freenames{Q}
  \and \\
  \freenames{@{x}} := \{ x \}
\end{mathpar}

$\pi$
$\quotep{\pi}$

$\freenames{-} : \pi \to \mathcal{P}(\quotep{\pi})$

\begin{eqnarray*}
  \freenames{\pzero} & := & \emptyset \\
  \freenames{x?(y).P} & := & \{ x \} \cup (\freenames{P} \setminus \{ y \}) \\
  \freenames{x!\langle P \rangle} & := & \{ x \} \cup \{ P \} \\
  \freenames{P|Q} & := & \freenames{P} \cup \freenames{Q} \\
  \freenames{\dropn{x}} & := & \{ x \}
\end{eqnarray*}

The bound names of a process, $\boundnames{P}$, are those names occurring in $P$
that are not free. For example, in $x?(y).0$, the name $x$ is free, while $y$ is bound.

\begin{mathpar}
  \inferrule* [lab=monoidal-laws] {} { P|Q \equiv Q|P \and P|0 \equiv P \and P|(Q|R) \equiv (P|Q)|R }
\end{mathpar}

\begin{mathpar}
  \inferrule* [lab=alpha-equivalence] {} { (x)P \equiv (y)P\{y/x\} \and y \not\in \freenames{P} }
\end{mathpar}

\begin{definition}
Then two processes, $P,Q$, are alpha-equivalent if $P = Q\{\vec{y}/\vec{x}\}$ for
some $\vec{x} \in \boundnames{Q},\vec{y} \in \boundnames{P}$, where $Q\{\vec{y}/\vec{x}\}$
denotes the capture-avoiding substitution of $\vec{y}$ for $\vec{x}$ in $Q$.
\end{definition}

\begin{definition}
  The {\em structural congruence} \cite{SangiorgiWalker} , $\equiv$,
  between processes is the least congruence containing
  alpha-equivalence, satisfying the abelian monoid laws
  (associativity, commutativity and $\pzero$ as identity) for parallel
  composition $|$ and for summation $+$.
\end{definition}

\subsection{Name equivalence}

We take name equivalence, written $\nameeq$, to be the smallest
equivalence relation generated by the following rules.

\begin{mathpar}
\inferrule*[lab=Quote-drop]
{ }
{ \quotep{@{x}} \nameeq x }

\inferrule*[lab=Struct-equiv]
{ P \scong Q }
{ \quotep{P} \nameeq \quotep{Q} }
\end{mathpar}

The astute reader will have noticed that the mutual recursion of names
and processes imposes a mutual recursion on alpha-equivalence and
structural equivalence via name-equivalence. Fortunately, all of this
works out pleasantly and we may calculate in the natural way, free of
concern. The reader interested in the details is referred to the
appendix \ref{appendix:rho_details}.

\subsection{Substitution}

We use $\Proc$ for the set of processes, $\QProc$ for the set of
names, and $\id{\{}\vec{y} / \vec{x} \id{\}}$ to denote partial maps,
$s : \QProc \rightarrow \QProc$. A map, $s$ lifts, uniquely, to a map
on process terms, $\widehat{s} : \Proc \rightarrow \Proc$ by the
following equations.

\begin{mathpar}
  (0) \psubstp{Q}{P} := 0 \\
  (R \juxtap S) \psubstp{Q}{P}
  :=    
  (R)\psubstp{Q}{P} \juxtap (S) \psubstp{Q}{P} \\
  (x?(y).R) \psubstp{Q}{P}    
  :=    
  (x)\substp{Q}{P} (z)\concat( (R \psubstn{z}{y}) \psubstp{Q}{P} ) \\
  (\lift{x}{R}) \psubstp{Q}{P}  
  :=
  \lift{(x)\substp{Q}{P}}{ R \psubstp{Q}{P} } \\
%   (\dropn{x})  \psubstp{Q}{P}       
%   := 
%   \left\{ 
%     \begin{array}{ccc} 
%       \dropn{\quotep{Q}} & & x \nameeq \quotep{P} \\
%       \dropn{x} & & otherwise \\
%     \end{array}
%   \right. 
  (\dropn{x})  \psubstp{Q}{P}       
  := 
  \left\{ 
    \begin{array}{ccc} 
      Q & & x \nameeq \quotep{P} \\
      \dropn{x} & & otherwise \\
    \end{array}
  \right.
\end{mathpar}
 

where

\begin{eqnarray}
  (x)\id{\{} \lpquote Q \rpquote / \lpquote P \rpquote \id{\}}            = 
  \left\{ 
    \begin{array}{ccc}
      \lpquote Q \rpquote & & x \nameeq \lpquote P \rpquote \\
      x & & otherwise \\
    \end{array}
  \right. \nonumber
\end{eqnarray}

and $z$ is chosen distinct from $\quotep{P}$, $\quotep{Q}$, the free
names in $Q$, and all the names in $R$. Our $\alpha$-equivalence will
be built in the standard way from this substitution.

\begin{remark}\label{rem:no_self_referential_names}
  One consequence of these definitions is that $\forall P. \quotep{P}
  \not\in \freenames{P}$.
\end{remark}

\subsection{ Dynamic quote: an example }

Anticipating something of what's to come, consider applying the
substitution, $\widehat{\id{\{}u / z \id{\}}}$, to the following pair
of processes, $\lift{w}{y!(z)}$ and $w[ \lpquote y!(z) \rpquote ]$.

\begin{eqnarray}
	\lift{w}{y!(z)}\widehat{\id{\{}u / z \id{\}}}
		& = &
		\lift{w}{y!(u)} \nonumber\\
	w[ \lpquote y!(z) \rpquote ] \widehat{ \id{\{}u / z \id{\}} }
		& = &
		w[ \lpquote y!(z) \rpquote ] \nonumber
\end{eqnarray}

Because the body of the process between quotes is impervious to
substitution, we get radically different answers. In fact, by
examining the first process in an input context,
e.g. $x?(z).\lift{w}{y!(z)}$, we see that the process under the lift
operator may be shaped by prefixed inputs binding a name inside it. In
this sense, the lift operator will be seen as a way to dynamically
construct processes before reifying them as names.

Finally equipped with these standard features we can present the
dynamics of the calculus.

\subsubsection{Operational semantics} 

Finally, we introduce the computational dynamics. What marks these
algebras as distinct from other more traditionally studied algebraic
structures, e.g. vector spaces or polynomial rings, is the manner in
which dynamics is captured. In traditional structures, dynamics is typically
expressed through morphisms between such structures, as in linear maps
between vector spaces or morphisms between rings. In algebras
associated with the semantics of computation, the dynamics is
expressed as part of the algebraic structure itself, through a
reduction reduction relation typically denoted by $\red$. Below, we
give a recursive presentation of this relation for the calculus used
in the encoding.

$\red \subseteq \pi \times \pi$
$\red : \pi \to \mathcal{P}(\pi)$

\begin{mathpar}
  \inferrule* [lab=Comm] { \textsf{match}( x_{src}, x_{trgt} ) } { x_{trgt}?(y)P \; | \; x_{src}!\langle {Q} \rangle \red P\{\quotep{Q}/y}\} }
  \and \\
  \inferrule* [lab=Par] {{P} \red {P}'} {{{P} | {Q}} \red {{P}' | {Q}}}
  \and
  \inferrule* [lab=Equiv]{{{P} \scong {P}'} \andalso {{P}' \red {Q}'} \andalso {{Q}' \scong {Q}}}{{P} \red {Q}}
\end{mathpar}

\begin{eqnarray*}
  match_{\equiv} (\quotep{P},\quotep{Q}) & := & P \equiv Q \\
  match_{\dagger}(\quotep{P},\quotep{Q}) & := & \forall R. P|Q \red^{*} R => R \red^{*} 0 \\
  match_{K}(\quotep{P},\quotep{Q}) & := & K \mbox{ for some context } K
\end{eqnarray*}

$u?(x)P | u!\langle Q \rangle \red P\{\quotep{Q}/x\}$

%We write $\wred$ for $\red^*$, and $P\red$ if $\exists Q $ such that $ P \red Q$.
We write $P\red$ if $\exists Q $ such that $ P \red Q$ and $P\not\red$, otherwise.

\section{Replication}

As mentioned before, it is known that replication (and hence
recursion) can be implemented in a higher-order process algebra
\cite{SangiorgiWalker}. As our first example of calculation with the
machinery thus far presented we give the construction explicitly in
the {\rhoc}.

\begin{eqnarray}
	D_{x} & := & \prefix{x}{y}{(\binpar{\outputp{x}{y}}{@{y}})} \nonumber\\
	\bangp_{x}{P} & := & \binpar{{x}!\langle{\binpar{D_{x}}{P}}\rangle}{D_{x}} \nonumber
\end{eqnarray}

\begin{eqnarray}
	\bangp_{x}{P} & & \nonumber\\
	=
	& {x}!\langle{(\prefix{x}{y}{(\outputp{x}{y} | @{y})) | P}}\rangle 
	      | \prefix{x}{y}{(\outputp{x}{y} | @{y})} & \nonumber\\
	\red
	& (\outputp{x}{y} | @{y})\substn{\quotep{(\prefix{x}{y}{(@{y} | \outputp{x}{y})) | P}}}{y} & \nonumber\\
	=
	& \outputp{x}{\quotep{(\prefix{x}{y}{(\outputp{x}{y} | @{y})) | P}}}
	  | {(\prefix{x}{y}{(\outputp{x}{y} | @{y})) | P}} & \nonumber\\
	\red
	& \ldots & \nonumber\\
	\red^*
	& P | P | \ldots & \nonumber
\end{eqnarray}

Of course, this encoding, as an implementation, runs away, unfolding
$\bangp{P}$ eagerly. A lazier and more implementable replication
operator, restricted to input-guarded processes, may be obtained as follows.

\begin{eqnarray}
\bangp{\prefix{u}{v}{P}} 
	:= 
	\binpar{\lift{x}{\prefix{u}{v}{(\binpar{D(x)}{P})}}}{D(x)} \nonumber
\end{eqnarray}

\begin{remark}
  Note that the lazier definition still does not deal with summation
  or mixed summation (i.e. sums over input and output). The reader is
  invited to construct definitions of replication that deal with these
  features. 

  Further, the definitions are parameterized in a name, $x$. Can you,
  gentle reader, make a definition that eliminates this parameter and
  guarantees no accidental interaction between the replication
  machinery and the process being replicated -- i.e. no accidental
  sharing of names used by the process to get its work done and the
  name(s) used by the replication to effect copying. This latter
  revision of the definition of replication is crucial to obtaining
  the expected identity $!!P \sim !P$.
\end{remark}

\begin{remark}\label{rem:paradoxical_combinator}
  The reader familiar with the lambda calculus will have noticed the
  similarity between $D$ and the paradoxical combinator.

  [Ed. note: the existence of this seems to suggest we have to be more
  restrictive on the set of processes and names we admit if we are to
  support no-cloning.]
\end{remark}

\subsubsection{Bisimulation}

The computational dynamics gives rise to another kind of equivalence,
the equivalence of computational behavior. As previously mentioned
this is typically captured \emph{via} some form of bisimulation.

% The notion we use in this paper is weak barbed bisimulation
% \cite{milner91polyadicpi}.

The notion we use in this paper is derived from weak barbed
bisimulation \cite{milner91polyadicpi}. 

\begin{definition}
An \emph{observation relation}, $\downarrow_{\mathcal N}$, over a set
of names, $\mathcal N$, is the smallest relation satisfying the rules
below.

\infrule[Out-barb]{y \in {\mathcal N}, \; x \nameeq y}
		  {\outputp{x}{v} \downarrow_{\mathcal N} x}
\infrule[Par-barb]{\mbox{$P\downarrow_{\mathcal N} x$ or $Q\downarrow_{\mathcal N} x$}}
		  {\binpar{P}{Q} \downarrow_{\mathcal N} x}

We write $P \Downarrow_{\mathcal N} x$ if there is $Q$ such that 
$P \wred Q$ and $Q \downarrow_{\mathcal N} x$.
\end{definition}

\begin{definition}
%\label{def.bbisim}
An  ${\mathcal N}$-\emph{barbed bisimulation} over a set of names, ${\mathcal N}$, is a symmetric binary relation 
${\mathcal S}_{\mathcal N}$ between agents such that $P\rel{S}_{\mathcal N}Q$ implies:
\begin{enumerate}
\item If $P \red P'$ then $Q \wred Q'$ and $P'\rel{S}_{\mathcal N} Q'$.
\item If $P\downarrow_{\mathcal N} x$, then $Q\Downarrow_{\mathcal N} x$.
\end{enumerate}
$P$ is ${\mathcal N}$-barbed bisimilar to $Q$, written
$P \wbbisim_{\mathcal N} Q$, if $P \rel{S}_{\mathcal N} Q$ for some ${\mathcal N}$-barbed bisimulation ${\mathcal S}_{\mathcal N}$.
\end{definition}

$\mathcal{R} \subseteq \pi \times \pi$

$P \mathcal{R} Q => \forall P'. P \red P' \Rightarrow \exists Q'. Q \red Q', P' \mathcal{R} Q'$

$P \vdash x \Rightarrow Q \vdash x$

\begin{mathpar}
  \inferrule*[lab=Out-barb]{x \nameeq y}{{y}!\langle{Q}\rangle \vdash x}
  \and
  \inferrule*[lab=Par-barb]{\mbox{$P\vdash x$ or $Q\vdash x$}}{\binpar{P}{Q} \vdash x}
\end{mathpar}

\subsubsection{Contexts}

One of the principle advantages of computational calculi like the
$\pi$-calculus is a well-defined notion of context,
contextual-equivalence and a correlation between
contextual-equivalence and notions of bisimulation. The notion of
context allows the decomposition of a process into (sub-)process and
its syntactic environment, its context. Thus, a context may be
thought of as a process with a ``hole'' (written $\Box$) in it. The
application of a context $M$ to a process $P$, written $M[P]$, is
tantamount to filling the hole in $M$ with $P$. In this paper we do
not need the full weight of this theory, but do make use of the notion
of context in the proof the main theorem. 

\begin{mathpar}
  \inferrule* [lab=summation] {} {{M_{M},M_{N}} \bc \Box \;|\; x.M_{A} \;|\; M_{M}+M_{N}}
  \and
  \inferrule* [lab=agent] {} {{M_{A}} \bc (\vec{x})M_{P} \;| \; \clift{P_0,\ldots,M_{P},\ldots,P_N}}
  \and \\
  \inferrule* [lab=process] {} {{M_{P}} \bc M_{N} \;| \;P|M_{P} }
\end{mathpar} 

\begin{mathpar}
  \inferrule* [lab=sychronization] {} {M_{N} \bc \Box \;|\; x?M_{F} \;|\; x!M_{C}}
  \and
  \inferrule* [lab=abstraction] {} {{M_{F}} \bc (x)M_{P} }
  \and
  \inferrule* [lab=concretion] {} {{M_{C}} \bc \langle M_{P} \rangle }
  \and \\
  \inferrule* [lab=process] {} {{M_{P}} \bc M_{N} \;| \;P|M_{P} }
\end{mathpar}

\begin{definition}[contextual application] Given a context $M$, and
  process $P$, we define the \emph{contextual application}, $M[P] :=
  M\{P/\Box\}$. That is, the contextual application of M to P is the
  substitution of $P$ for $\Box$ in $M$.
\end{definition}

$\meaningof{-} : L \to \mathcal{P}(\pi)$

\begin{mathpar}
  \inferrule* [lab=collection] {} {\meaningof{true} = \pi, \and \meaningof{~E} = \pi \setminus \meaningof{E}, \and \meaningof{E_{1} \& E_{2}} = \meaningof{E_{1}} \cap \meaningof{E_{2}}}
\end{mathpar}

\begin{mathpar}
  \inferrule* [lab=structure] {} {\meaningof{0} = \{ P \in \pi | P \equiv 0 \}, \and \\ \meaningof{E_1 | E_2} = \{ P \in \pi | P \equiv P_{1} | P_{2}, P_{1} \in \meaningof{E_{1}}, P_{2} \in \meaningof{E_2}\} }
\end{mathpar}

\begin{mathpar}
 \inferrule* [lab=behavior] {} {\meaningof{\langle a?b \rangle E} = \{ P \in \pi | P \equiv Q | u?(y)P', \\ \and \\\\ \and \\ \;\;\; u \in \meaningof{a}, \forall z.P'\{z/y\} \in \meaningof{E\{z/b\}}\}, \and \\ \meaningof{a!E} = \{ P \in \pi | P \equiv Q | x!\langle P' \rangle, x \in \meaningof{a} P' \in \meaningof{E}\} }
\end{mathpar}

\begin{mathpar}
 \inferrule* [lab=nominal] {} {\meaningof{\quotep{E}} = \{ \quotep{P} \in \quotep{\pi} | P \in \meaningof{E} \}, \and \meaningof{\quotep{P}} = \{ \quotep{Q} \in \quotep{\pi} | P \equiv Q \} \and \\ \meaningof{@\quotep{E}} = \{ P \in \pi | P \equiv @x, x \in \meaningof{E} \}}
\end{mathpar}

\begin{eqnarray*}
  \\
  \meaningof{-} : TS \to ST
\end{eqnarray*}

\begin{eqnarray*}
  \\
  L : TS \to ST
\end{eqnarray*}

\begin{eqnarray*}
  \\
  P \models E \iff P \in \meaningof{E}
\end{eqnarray*}

\begin{eqnarray*}
  P \approx_{L} Q \iff \forall E \in L. P \models E \iff Q \models E
\end{eqnarray*}

\begin{eqnarray*}
  P \approx_{K} Q
\end{eqnarray*}

\begin{eqnarray*}
  P \approx Q
\end{eqnarray*}

$\approx_{K} = \approx = \approx_{L}$

\subsubsection{Contextual duality}

Note that contexts extend the quotation operation to a family of
operations from processes to names. Given a context, $M$, we can
define a \emph{nominal context}, $\quotep{M}$ by $\quotep{M}[P] :=
\quotep{M[P]}$. To foreshadow what is to come we observe that these
operations enjoy a duality with processes very much like the duality
between vectors and maps from vectors to scalars.

Further, because the calculus is essentially higher-order, we have a
correspondence between contexts and processes. More specifically,
given a name $x$ and a context $M$ we can construct $M^{*}_{x}$ such
that 

\begin{mathpar}
  M^{*}_{x} | \lift{x}{P} \red M[P]
\end{mathpar}

namely,

\begin{mathpar}
  M^{*}_{x} := x?(u).M[\dropn{u}]
\end{mathpar}

The dependence of $M^{*}_{x}$ on a name makes it an abstraction, 

\begin{mathpar}
  M^{*} := (x)x?(u).M[\dropn{u}]
\end{mathpar}

\subsection{Additional notation}

It will sometimes be convenient to denote the process a name
quotes. We already have the notation $x = \quotep{P}$, but it will be
convenient to introduce an alternate notation, $\procn{x}$, when we
want to emphasize the connection to the use of the name. Note that, by
virtue of name equivalence, $\quotep{\procn{x}} \nameeq x$; so, the
notation is consistent with previous definitions.

Further, because names have structure it is possible to effect
substitutions on the basis of that structure. This means we need to
upgrade our notation for substitutions, which we accomplish by
adapting comprehension notation. Thus,

\begin{mathpar}
  P\{ y / x : x \in S \}
\end{mathpar}

is interpreted to mean the process derived from P by replacing (in a
capture-avoiding manner) each occurrence of $x$ in $S$ by $y$. For example,

\begin{mathpar}
  P\{ \quotep{\procn{x}|\procn{x}} / x : x \in \freenames{P} \}
\end{mathpar}

will replace each (occurrence) of a free name $x$ in $P$ by
$\quotep{\procn{x}|\procn{x}}$.

Also, we will avail ourselves of the notation $x^{L}$ and $x^{R}$ to
denote injections of a name into disjoint copies of the name
space. There are numerous ways to accomplish this. One example can be
found in \cite{MeredithR05}. This notation overloads to vectors of
names: $\vec{x}^{\pi} := (x_{i}^{\pi} \; : \; 0 \leq i < |\vec{x}| )$ where $\pi \in \{L,R\}$.

We also use $P^{\Box} := P|\Box$.

In \cite{MeredithR05} an interpretation of the new operator is
given. It turns out that there are several possible interpretations
all enjoying the requisite algebraic properties of the operator (see
\cite{milner91polyadicpi}). We will therefore make liberal use of
$(\nu\; \vec{x})P$.

% subsection the_syntax_and_semantics_of_the_notation_system (end)   

\input{qm2pi.qmops} 

\input{qm2pi.sterngerlach} 

\input{qm2pi.metric} 

% section concurrent_process_calculi (end)

%\input{qm2pi.proofsketch}

% section proof sketch (end)

%\input{qm2pi.slviaknots} 

% section spatial logic via knots (end)

\input{qm2pi.conclusion}

% section conclusion (end)

%\input{qm2pi.dtcodes} 

% section wiring algorithm (end)

\input{qm2pi.ack} 

% section acknowledgments (end)

\newpage


\bibliographystyle{plain}   
\bibliography{../../biblios/main.bib}

\input{qm2pi.rhodetails}

\end{document}

 

%\documentclass[12pt]{llncs}
%\documentclass{jktr}

\usepackage[pdftex]{hyperref}                   
\usepackage {listings}
\usepackage {mathpartir}
\usepackage{bcprules}
%\usepackage{listings}
                       
\usepackage{graphicx} 
%\usepackage[margins=2.5cm,nohead,nofoot]{geometry}
%\usepackage{geometry}
\usepackage{amsfonts}
\usepackage{amstext}
\usepackage{latexsym}
\usepackage{amssymb}
\usepackage{color}


%\include{myPreamble}
\include{qm2pi.local} 

%\ifpdf
%\usepackage[pdftex]{graphicx}
%\else
%\usepackage{graphicx}
%\fi

 % \ifpdf
%  \usepackage{pdfsync}
%  \if


%\title{Brief Article}
%\author{David F. Snyder}
%\author{L.G. Meredith}

%\address{Dept. of Math., Texas State University--San Marcos, San Marcos, TX 78666}
       
\pagestyle{empty}


\begin{document}

\lstset{language=[Objective]Caml,frame=shadowbox}

\input{qm2pi.front}

% section front matter (end)

\input{qm2pi.intro} 
 
% section introduction (end)

% \input{qm2pi.knotations} 

% section notation (end)

\input{qm2pi.process.calculi} 

% section concurrent_process_calculi_and_spatial_logics_ (end)
    
%\input{qm2pi.knots2pi} 

%\input{qm2pi.trefoil} 

%\input{qm2pi.mainthm} 

% subsection basic_interpretation (end)

%\input{qm2pi.rho.presentation} 
\subsection{The syntax and semantics of the notation system}\label{sub:the_syntax_and_semantics_of_the_notation_system} % (fold)

We now summarize a technical presentation of the calculus that
embodies our theory of dynamics. The typical presentation of such a
calculus follows the style of giving generators and relations on
them. The grammar, below, describing term constructors, freely
generates the set of processes, $\Proc$. This set is then quotiented
by a relation known as structural congruence and it is over this set
that the notion of dynamics is expressed. This presentation is
essentially that of \cite{MeredithR05} with the addition of
polyadicity and summation. For readability we have relegated some of
the technical subtleties to an appendix.

\subsubsection{Process grammar}\label{subsub:process_grammar}

\begin{mathpar}
  \inferrule* [lab=synchronization] {} {{M} \bc \pzero \;|\; x?F \;|\; x!C }
  \and
  \inferrule* [lab=abstraction] {} {{F} \bc (x)P}
  \and
  \inferrule* [lab=concretion] {} {{C} \bc \langle Q \rangle}
  \and
  \inferrule* [lab=process] {} {{P,Q} \bc M \;| \;P|Q \;|\; @{x}}
  \and
  \inferrule* [lab=name] {} {{x} \bc \quotep{P}}
\end{mathpar} 

Note that $\vec{x}$ (resp. $\vec{P}$) denotes a vector of names
(resp. processes) of length $|\vec{x}|$ (resp. $|\vec{P}|$). We adopt
the following useful abbreviations.

\begin{mathpar}
   x?(\vec{y}).P := x.(\vec{y})P \and  x\clift{\vec{P}} := x.\clift{\vec{P}}
   \and x!(y) := \lift{x}{\dropn{y}}
   \and \Pi_{i=0}^{n-1}P_i := P_0 | \ldots | P_{n-1}
\end{mathpar}

\subsubsection{Structural congruence}

\paragraph{Free and bound names and alpha-equivalence.} At the
core of structural equivalence is alpha-equivalence which identifies
process that are the same up to a change of variable. Formally, we
recognize the distinction between free and bound names. The free names
of a process, $\freenames{P}$, may be calculated recursively as
follows:

\begin{mathpar}
\freenames{\pzero} := \emptyset
  \and \\
  \freenames{x?(y).P} := \{ x \} \cup (\freenames{P} \setminus \{ y \})
  \and 
  \freenames{x!\langle P \rangle} := \{ x \} \cup \{ P \} 
  \and \\
  \freenames{P|Q} := \freenames{P} \cup \freenames{Q}
  \and \\
  \freenames{@{x}} := \{ x \}
\end{mathpar}

$\pi$
$\quotep{\pi}$

$\freenames{-} : \pi \to \mathcal{P}(\quotep{\pi})$

\begin{eqnarray*}
  \freenames{\pzero} & := & \emptyset \\
  \freenames{x?(y).P} & := & \{ x \} \cup (\freenames{P} \setminus \{ y \}) \\
  \freenames{x!\langle P \rangle} & := & \{ x \} \cup \{ P \} \\
  \freenames{P|Q} & := & \freenames{P} \cup \freenames{Q} \\
  \freenames{\dropn{x}} & := & \{ x \}
\end{eqnarray*}

The bound names of a process, $\boundnames{P}$, are those names occurring in $P$
that are not free. For example, in $x?(y).0$, the name $x$ is free, while $y$ is bound.

\begin{mathpar}
  \inferrule* [lab=monoidal-laws] {} { P|Q \equiv Q|P \and P|0 \equiv P \and P|(Q|R) \equiv (P|Q)|R }
\end{mathpar}

\begin{mathpar}
  \inferrule* [lab=alpha-equivalence] {} { (x)P \equiv (y)P\{y/x\} \and y \not\in \freenames{P} }
\end{mathpar}

\begin{definition}
Then two processes, $P,Q$, are alpha-equivalent if $P = Q\{\vec{y}/\vec{x}\}$ for
some $\vec{x} \in \boundnames{Q},\vec{y} \in \boundnames{P}$, where $Q\{\vec{y}/\vec{x}\}$
denotes the capture-avoiding substitution of $\vec{y}$ for $\vec{x}$ in $Q$.
\end{definition}

\begin{definition}
  The {\em structural congruence} \cite{SangiorgiWalker} , $\equiv$,
  between processes is the least congruence containing
  alpha-equivalence, satisfying the abelian monoid laws
  (associativity, commutativity and $\pzero$ as identity) for parallel
  composition $|$ and for summation $+$.
\end{definition}

\subsection{Name equivalence}

We take name equivalence, written $\nameeq$, to be the smallest
equivalence relation generated by the following rules.

\begin{mathpar}
\inferrule*[lab=Quote-drop]
{ }
{ \quotep{@{x}} \nameeq x }

\inferrule*[lab=Struct-equiv]
{ P \scong Q }
{ \quotep{P} \nameeq \quotep{Q} }
\end{mathpar}

The astute reader will have noticed that the mutual recursion of names
and processes imposes a mutual recursion on alpha-equivalence and
structural equivalence via name-equivalence. Fortunately, all of this
works out pleasantly and we may calculate in the natural way, free of
concern. The reader interested in the details is referred to the
appendix \ref{appendix:rho_details}.

\subsection{Substitution}

We use $\Proc$ for the set of processes, $\QProc$ for the set of
names, and $\id{\{}\vec{y} / \vec{x} \id{\}}$ to denote partial maps,
$s : \QProc \rightarrow \QProc$. A map, $s$ lifts, uniquely, to a map
on process terms, $\widehat{s} : \Proc \rightarrow \Proc$ by the
following equations.

\begin{mathpar}
  (0) \psubstp{Q}{P} := 0 \\
  (R \juxtap S) \psubstp{Q}{P}
  :=    
  (R)\psubstp{Q}{P} \juxtap (S) \psubstp{Q}{P} \\
  (x?(y).R) \psubstp{Q}{P}    
  :=    
  (x)\substp{Q}{P} (z)\concat( (R \psubstn{z}{y}) \psubstp{Q}{P} ) \\
  (\lift{x}{R}) \psubstp{Q}{P}  
  :=
  \lift{(x)\substp{Q}{P}}{ R \psubstp{Q}{P} } \\
%   (\dropn{x})  \psubstp{Q}{P}       
%   := 
%   \left\{ 
%     \begin{array}{ccc} 
%       \dropn{\quotep{Q}} & & x \nameeq \quotep{P} \\
%       \dropn{x} & & otherwise \\
%     \end{array}
%   \right. 
  (\dropn{x})  \psubstp{Q}{P}       
  := 
  \left\{ 
    \begin{array}{ccc} 
      Q & & x \nameeq \quotep{P} \\
      \dropn{x} & & otherwise \\
    \end{array}
  \right.
\end{mathpar}
 

where

\begin{eqnarray}
  (x)\id{\{} \lpquote Q \rpquote / \lpquote P \rpquote \id{\}}            = 
  \left\{ 
    \begin{array}{ccc}
      \lpquote Q \rpquote & & x \nameeq \lpquote P \rpquote \\
      x & & otherwise \\
    \end{array}
  \right. \nonumber
\end{eqnarray}

and $z$ is chosen distinct from $\quotep{P}$, $\quotep{Q}$, the free
names in $Q$, and all the names in $R$. Our $\alpha$-equivalence will
be built in the standard way from this substitution.

\begin{remark}\label{rem:no_self_referential_names}
  One consequence of these definitions is that $\forall P. \quotep{P}
  \not\in \freenames{P}$.
\end{remark}

\subsection{ Dynamic quote: an example }

Anticipating something of what's to come, consider applying the
substitution, $\widehat{\id{\{}u / z \id{\}}}$, to the following pair
of processes, $\lift{w}{y!(z)}$ and $w[ \lpquote y!(z) \rpquote ]$.

\begin{eqnarray}
	\lift{w}{y!(z)}\widehat{\id{\{}u / z \id{\}}}
		& = &
		\lift{w}{y!(u)} \nonumber\\
	w[ \lpquote y!(z) \rpquote ] \widehat{ \id{\{}u / z \id{\}} }
		& = &
		w[ \lpquote y!(z) \rpquote ] \nonumber
\end{eqnarray}

Because the body of the process between quotes is impervious to
substitution, we get radically different answers. In fact, by
examining the first process in an input context,
e.g. $x?(z).\lift{w}{y!(z)}$, we see that the process under the lift
operator may be shaped by prefixed inputs binding a name inside it. In
this sense, the lift operator will be seen as a way to dynamically
construct processes before reifying them as names.

Finally equipped with these standard features we can present the
dynamics of the calculus.

\subsubsection{Operational semantics} 

Finally, we introduce the computational dynamics. What marks these
algebras as distinct from other more traditionally studied algebraic
structures, e.g. vector spaces or polynomial rings, is the manner in
which dynamics is captured. In traditional structures, dynamics is typically
expressed through morphisms between such structures, as in linear maps
between vector spaces or morphisms between rings. In algebras
associated with the semantics of computation, the dynamics is
expressed as part of the algebraic structure itself, through a
reduction reduction relation typically denoted by $\red$. Below, we
give a recursive presentation of this relation for the calculus used
in the encoding.

$\red \subseteq \pi \times \pi$
$\red : \pi \to \mathcal{P}(\pi)$

\begin{mathpar}
  \inferrule* [lab=Comm] { \textsf{match}( x_{src}, x_{trgt} ) } { x_{trgt}?(y)P \; | \; x_{src}!\langle {Q} \rangle \red P\{\quotep{Q}/y}\} }
  \and \\
  \inferrule* [lab=Par] {{P} \red {P}'} {{{P} | {Q}} \red {{P}' | {Q}}}
  \and
  \inferrule* [lab=Equiv]{{{P} \scong {P}'} \andalso {{P}' \red {Q}'} \andalso {{Q}' \scong {Q}}}{{P} \red {Q}}
\end{mathpar}

\begin{eqnarray*}
  match_{\equiv} (\quotep{P},\quotep{Q}) & := & P \equiv Q \\
  match_{\dagger}(\quotep{P},\quotep{Q}) & := & \forall R. P|Q \red^{*} R => R \red^{*} 0 \\
  match_{K}(\quotep{P},\quotep{Q}) & := & K \mbox{ for some context } K
\end{eqnarray*}

$u?(x)P | u!\langle Q \rangle \red P\{\quotep{Q}/x\}$

%We write $\wred$ for $\red^*$, and $P\red$ if $\exists Q $ such that $ P \red Q$.
We write $P\red$ if $\exists Q $ such that $ P \red Q$ and $P\not\red$, otherwise.

\section{Replication}

As mentioned before, it is known that replication (and hence
recursion) can be implemented in a higher-order process algebra
\cite{SangiorgiWalker}. As our first example of calculation with the
machinery thus far presented we give the construction explicitly in
the {\rhoc}.

\begin{eqnarray}
	D_{x} & := & \prefix{x}{y}{(\binpar{\outputp{x}{y}}{@{y}})} \nonumber\\
	\bangp_{x}{P} & := & \binpar{{x}!\langle{\binpar{D_{x}}{P}}\rangle}{D_{x}} \nonumber
\end{eqnarray}

\begin{eqnarray}
	\bangp_{x}{P} & & \nonumber\\
	=
	& {x}!\langle{(\prefix{x}{y}{(\outputp{x}{y} | @{y})) | P}}\rangle 
	      | \prefix{x}{y}{(\outputp{x}{y} | @{y})} & \nonumber\\
	\red
	& (\outputp{x}{y} | @{y})\substn{\quotep{(\prefix{x}{y}{(@{y} | \outputp{x}{y})) | P}}}{y} & \nonumber\\
	=
	& \outputp{x}{\quotep{(\prefix{x}{y}{(\outputp{x}{y} | @{y})) | P}}}
	  | {(\prefix{x}{y}{(\outputp{x}{y} | @{y})) | P}} & \nonumber\\
	\red
	& \ldots & \nonumber\\
	\red^*
	& P | P | \ldots & \nonumber
\end{eqnarray}

Of course, this encoding, as an implementation, runs away, unfolding
$\bangp{P}$ eagerly. A lazier and more implementable replication
operator, restricted to input-guarded processes, may be obtained as follows.

\begin{eqnarray}
\bangp{\prefix{u}{v}{P}} 
	:= 
	\binpar{\lift{x}{\prefix{u}{v}{(\binpar{D(x)}{P})}}}{D(x)} \nonumber
\end{eqnarray}

\begin{remark}
  Note that the lazier definition still does not deal with summation
  or mixed summation (i.e. sums over input and output). The reader is
  invited to construct definitions of replication that deal with these
  features. 

  Further, the definitions are parameterized in a name, $x$. Can you,
  gentle reader, make a definition that eliminates this parameter and
  guarantees no accidental interaction between the replication
  machinery and the process being replicated -- i.e. no accidental
  sharing of names used by the process to get its work done and the
  name(s) used by the replication to effect copying. This latter
  revision of the definition of replication is crucial to obtaining
  the expected identity $!!P \sim !P$.
\end{remark}

\begin{remark}\label{rem:paradoxical_combinator}
  The reader familiar with the lambda calculus will have noticed the
  similarity between $D$ and the paradoxical combinator.

  [Ed. note: the existence of this seems to suggest we have to be more
  restrictive on the set of processes and names we admit if we are to
  support no-cloning.]
\end{remark}

\subsubsection{Bisimulation}

The computational dynamics gives rise to another kind of equivalence,
the equivalence of computational behavior. As previously mentioned
this is typically captured \emph{via} some form of bisimulation.

% The notion we use in this paper is weak barbed bisimulation
% \cite{milner91polyadicpi}.

The notion we use in this paper is derived from weak barbed
bisimulation \cite{milner91polyadicpi}. 

\begin{definition}
An \emph{observation relation}, $\downarrow_{\mathcal N}$, over a set
of names, $\mathcal N$, is the smallest relation satisfying the rules
below.

\infrule[Out-barb]{y \in {\mathcal N}, \; x \nameeq y}
		  {\outputp{x}{v} \downarrow_{\mathcal N} x}
\infrule[Par-barb]{\mbox{$P\downarrow_{\mathcal N} x$ or $Q\downarrow_{\mathcal N} x$}}
		  {\binpar{P}{Q} \downarrow_{\mathcal N} x}

We write $P \Downarrow_{\mathcal N} x$ if there is $Q$ such that 
$P \wred Q$ and $Q \downarrow_{\mathcal N} x$.
\end{definition}

\begin{definition}
%\label{def.bbisim}
An  ${\mathcal N}$-\emph{barbed bisimulation} over a set of names, ${\mathcal N}$, is a symmetric binary relation 
${\mathcal S}_{\mathcal N}$ between agents such that $P\rel{S}_{\mathcal N}Q$ implies:
\begin{enumerate}
\item If $P \red P'$ then $Q \wred Q'$ and $P'\rel{S}_{\mathcal N} Q'$.
\item If $P\downarrow_{\mathcal N} x$, then $Q\Downarrow_{\mathcal N} x$.
\end{enumerate}
$P$ is ${\mathcal N}$-barbed bisimilar to $Q$, written
$P \wbbisim_{\mathcal N} Q$, if $P \rel{S}_{\mathcal N} Q$ for some ${\mathcal N}$-barbed bisimulation ${\mathcal S}_{\mathcal N}$.
\end{definition}

$\mathcal{R} \subseteq \pi \times \pi$

$P \mathcal{R} Q => \forall P'. P \red P' \Rightarrow \exists Q'. Q \red Q', P' \mathcal{R} Q'$

$P \vdash x \Rightarrow Q \vdash x$

\begin{mathpar}
  \inferrule*[lab=Out-barb]{x \nameeq y}{{y}!\langle{Q}\rangle \vdash x}
  \and
  \inferrule*[lab=Par-barb]{\mbox{$P\vdash x$ or $Q\vdash x$}}{\binpar{P}{Q} \vdash x}
\end{mathpar}

\subsubsection{Contexts}

One of the principle advantages of computational calculi like the
$\pi$-calculus is a well-defined notion of context,
contextual-equivalence and a correlation between
contextual-equivalence and notions of bisimulation. The notion of
context allows the decomposition of a process into (sub-)process and
its syntactic environment, its context. Thus, a context may be
thought of as a process with a ``hole'' (written $\Box$) in it. The
application of a context $M$ to a process $P$, written $M[P]$, is
tantamount to filling the hole in $M$ with $P$. In this paper we do
not need the full weight of this theory, but do make use of the notion
of context in the proof the main theorem. 

\begin{mathpar}
  \inferrule* [lab=summation] {} {{M_{M},M_{N}} \bc \Box \;|\; x.M_{A} \;|\; M_{M}+M_{N}}
  \and
  \inferrule* [lab=agent] {} {{M_{A}} \bc (\vec{x})M_{P} \;| \; \clift{P_0,\ldots,M_{P},\ldots,P_N}}
  \and \\
  \inferrule* [lab=process] {} {{M_{P}} \bc M_{N} \;| \;P|M_{P} }
\end{mathpar} 

\begin{mathpar}
  \inferrule* [lab=sychronization] {} {M_{N} \bc \Box \;|\; x?M_{F} \;|\; x!M_{C}}
  \and
  \inferrule* [lab=abstraction] {} {{M_{F}} \bc (x)M_{P} }
  \and
  \inferrule* [lab=concretion] {} {{M_{C}} \bc \langle M_{P} \rangle }
  \and \\
  \inferrule* [lab=process] {} {{M_{P}} \bc M_{N} \;| \;P|M_{P} }
\end{mathpar}

\begin{definition}[contextual application] Given a context $M$, and
  process $P$, we define the \emph{contextual application}, $M[P] :=
  M\{P/\Box\}$. That is, the contextual application of M to P is the
  substitution of $P$ for $\Box$ in $M$.
\end{definition}

$\meaningof{-} : L \to \mathcal{P}(\pi)$

\begin{mathpar}
  \inferrule* [lab=collection] {} {\meaningof{true} = \pi, \and \meaningof{~E} = \pi \setminus \meaningof{E}, \and \meaningof{E_{1} \& E_{2}} = \meaningof{E_{1}} \cap \meaningof{E_{2}}}
\end{mathpar}

\begin{mathpar}
  \inferrule* [lab=structure] {} {\meaningof{0} = \{ P \in \pi | P \equiv 0 \}, \and \\ \meaningof{E_1 | E_2} = \{ P \in \pi | P \equiv P_{1} | P_{2}, P_{1} \in \meaningof{E_{1}}, P_{2} \in \meaningof{E_2}\} }
\end{mathpar}

\begin{mathpar}
 \inferrule* [lab=behavior] {} {\meaningof{\langle a?b \rangle E} = \{ P \in \pi | P \equiv Q | u?(y)P', \\ \and \\\\ \and \\ \;\;\; u \in \meaningof{a}, \forall z.P'\{z/y\} \in \meaningof{E\{z/b\}}\}, \and \\ \meaningof{a!E} = \{ P \in \pi | P \equiv Q | x!\langle P' \rangle, x \in \meaningof{a} P' \in \meaningof{E}\} }
\end{mathpar}

\begin{mathpar}
 \inferrule* [lab=nominal] {} {\meaningof{\quotep{E}} = \{ \quotep{P} \in \quotep{\pi} | P \in \meaningof{E} \}, \and \meaningof{\quotep{P}} = \{ \quotep{Q} \in \quotep{\pi} | P \equiv Q \} \and \\ \meaningof{@\quotep{E}} = \{ P \in \pi | P \equiv @x, x \in \meaningof{E} \}}
\end{mathpar}

\begin{eqnarray*}
  \\
  \meaningof{-} : TS \to ST
\end{eqnarray*}

\begin{eqnarray*}
  \\
  L : TS \to ST
\end{eqnarray*}

\begin{eqnarray*}
  \\
  P \models E \iff P \in \meaningof{E}
\end{eqnarray*}

\begin{eqnarray*}
  P \approx_{L} Q \iff \forall E \in L. P \models E \iff Q \models E
\end{eqnarray*}

\begin{eqnarray*}
  P \approx_{K} Q
\end{eqnarray*}

\begin{eqnarray*}
  P \approx Q
\end{eqnarray*}

$\approx_{K} = \approx = \approx_{L}$

\subsubsection{Contextual duality}

Note that contexts extend the quotation operation to a family of
operations from processes to names. Given a context, $M$, we can
define a \emph{nominal context}, $\quotep{M}$ by $\quotep{M}[P] :=
\quotep{M[P]}$. To foreshadow what is to come we observe that these
operations enjoy a duality with processes very much like the duality
between vectors and maps from vectors to scalars.

Further, because the calculus is essentially higher-order, we have a
correspondence between contexts and processes. More specifically,
given a name $x$ and a context $M$ we can construct $M^{*}_{x}$ such
that 

\begin{mathpar}
  M^{*}_{x} | \lift{x}{P} \red M[P]
\end{mathpar}

namely,

\begin{mathpar}
  M^{*}_{x} := x?(u).M[\dropn{u}]
\end{mathpar}

The dependence of $M^{*}_{x}$ on a name makes it an abstraction, 

\begin{mathpar}
  M^{*} := (x)x?(u).M[\dropn{u}]
\end{mathpar}

\subsection{Additional notation}

It will sometimes be convenient to denote the process a name
quotes. We already have the notation $x = \quotep{P}$, but it will be
convenient to introduce an alternate notation, $\procn{x}$, when we
want to emphasize the connection to the use of the name. Note that, by
virtue of name equivalence, $\quotep{\procn{x}} \nameeq x$; so, the
notation is consistent with previous definitions.

Further, because names have structure it is possible to effect
substitutions on the basis of that structure. This means we need to
upgrade our notation for substitutions, which we accomplish by
adapting comprehension notation. Thus,

\begin{mathpar}
  P\{ y / x : x \in S \}
\end{mathpar}

is interpreted to mean the process derived from P by replacing (in a
capture-avoiding manner) each occurrence of $x$ in $S$ by $y$. For example,

\begin{mathpar}
  P\{ \quotep{\procn{x}|\procn{x}} / x : x \in \freenames{P} \}
\end{mathpar}

will replace each (occurrence) of a free name $x$ in $P$ by
$\quotep{\procn{x}|\procn{x}}$.

Also, we will avail ourselves of the notation $x^{L}$ and $x^{R}$ to
denote injections of a name into disjoint copies of the name
space. There are numerous ways to accomplish this. One example can be
found in \cite{MeredithR05}. This notation overloads to vectors of
names: $\vec{x}^{\pi} := (x_{i}^{\pi} \; : \; 0 \leq i < |\vec{x}| )$ where $\pi \in \{L,R\}$.

We also use $P^{\Box} := P|\Box$.

In \cite{MeredithR05} an interpretation of the new operator is
given. It turns out that there are several possible interpretations
all enjoying the requisite algebraic properties of the operator (see
\cite{milner91polyadicpi}). We will therefore make liberal use of
$(\nu\; \vec{x})P$.

% subsection the_syntax_and_semantics_of_the_notation_system (end)   

\input{qm2pi.qmops} 

\input{qm2pi.sterngerlach} 

\input{qm2pi.metric} 

% section concurrent_process_calculi (end)

%\input{qm2pi.proofsketch}

% section proof sketch (end)

%\input{qm2pi.slviaknots} 

% section spatial logic via knots (end)

\input{qm2pi.conclusion}

% section conclusion (end)

%\input{qm2pi.dtcodes} 

% section wiring algorithm (end)

\input{qm2pi.ack} 

% section acknowledgments (end)

\newpage


\bibliographystyle{plain}   
\bibliography{../../biblios/main.bib}

\input{qm2pi.rhodetails}

\end{document}

 

% subsection basic_interpretation (end)

%\input{qm2pi.rho.presentation} 
\subsection{The syntax and semantics of the notation system}\label{sub:the_syntax_and_semantics_of_the_notation_system} % (fold)

We now summarize a technical presentation of the calculus that
embodies our theory of dynamics. The typical presentation of such a
calculus follows the style of giving generators and relations on
them. The grammar, below, describing term constructors, freely
generates the set of processes, $\Proc$. This set is then quotiented
by a relation known as structural congruence and it is over this set
that the notion of dynamics is expressed. This presentation is
essentially that of \cite{MeredithR05} with the addition of
polyadicity and summation. For readability we have relegated some of
the technical subtleties to an appendix.

\subsubsection{Process grammar}\label{subsub:process_grammar}

\begin{mathpar}
  \inferrule* [lab=synchronization] {} {{M} \bc \pzero \;|\; x?F \;|\; x!C }
  \and
  \inferrule* [lab=abstraction] {} {{F} \bc (x)P}
  \and
  \inferrule* [lab=concretion] {} {{C} \bc \langle Q \rangle}
  \and
  \inferrule* [lab=process] {} {{P,Q} \bc M \;| \;P|Q \;|\; @{x}}
  \and
  \inferrule* [lab=name] {} {{x} \bc \quotep{P}}
\end{mathpar} 

Note that $\vec{x}$ (resp. $\vec{P}$) denotes a vector of names
(resp. processes) of length $|\vec{x}|$ (resp. $|\vec{P}|$). We adopt
the following useful abbreviations.

\begin{mathpar}
   x?(\vec{y}).P := x.(\vec{y})P \and  x\clift{\vec{P}} := x.\clift{\vec{P}}
   \and x!(y) := \lift{x}{\dropn{y}}
   \and \Pi_{i=0}^{n-1}P_i := P_0 | \ldots | P_{n-1}
\end{mathpar}

\subsubsection{Structural congruence}

\paragraph{Free and bound names and alpha-equivalence.} At the
core of structural equivalence is alpha-equivalence which identifies
process that are the same up to a change of variable. Formally, we
recognize the distinction between free and bound names. The free names
of a process, $\freenames{P}$, may be calculated recursively as
follows:

\begin{mathpar}
\freenames{\pzero} := \emptyset
  \and \\
  \freenames{x?(y).P} := \{ x \} \cup (\freenames{P} \setminus \{ y \})
  \and 
  \freenames{x!\langle P \rangle} := \{ x \} \cup \{ P \} 
  \and \\
  \freenames{P|Q} := \freenames{P} \cup \freenames{Q}
  \and \\
  \freenames{@{x}} := \{ x \}
\end{mathpar}

$\pi$
$\quotep{\pi}$

$\freenames{-} : \pi \to \mathcal{P}(\quotep{\pi})$

\begin{eqnarray*}
  \freenames{\pzero} & := & \emptyset \\
  \freenames{x?(y).P} & := & \{ x \} \cup (\freenames{P} \setminus \{ y \}) \\
  \freenames{x!\langle P \rangle} & := & \{ x \} \cup \{ P \} \\
  \freenames{P|Q} & := & \freenames{P} \cup \freenames{Q} \\
  \freenames{\dropn{x}} & := & \{ x \}
\end{eqnarray*}

The bound names of a process, $\boundnames{P}$, are those names occurring in $P$
that are not free. For example, in $x?(y).0$, the name $x$ is free, while $y$ is bound.

\begin{mathpar}
  \inferrule* [lab=monoidal-laws] {} { P|Q \equiv Q|P \and P|0 \equiv P \and P|(Q|R) \equiv (P|Q)|R }
\end{mathpar}

\begin{mathpar}
  \inferrule* [lab=alpha-equivalence] {} { (x)P \equiv (y)P\{y/x\} \and y \not\in \freenames{P} }
\end{mathpar}

\begin{definition}
Then two processes, $P,Q$, are alpha-equivalent if $P = Q\{\vec{y}/\vec{x}\}$ for
some $\vec{x} \in \boundnames{Q},\vec{y} \in \boundnames{P}$, where $Q\{\vec{y}/\vec{x}\}$
denotes the capture-avoiding substitution of $\vec{y}$ for $\vec{x}$ in $Q$.
\end{definition}

\begin{definition}
  The {\em structural congruence} \cite{SangiorgiWalker} , $\equiv$,
  between processes is the least congruence containing
  alpha-equivalence, satisfying the abelian monoid laws
  (associativity, commutativity and $\pzero$ as identity) for parallel
  composition $|$ and for summation $+$.
\end{definition}

\subsection{Name equivalence}

We take name equivalence, written $\nameeq$, to be the smallest
equivalence relation generated by the following rules.

\begin{mathpar}
\inferrule*[lab=Quote-drop]
{ }
{ \quotep{@{x}} \nameeq x }

\inferrule*[lab=Struct-equiv]
{ P \scong Q }
{ \quotep{P} \nameeq \quotep{Q} }
\end{mathpar}

The astute reader will have noticed that the mutual recursion of names
and processes imposes a mutual recursion on alpha-equivalence and
structural equivalence via name-equivalence. Fortunately, all of this
works out pleasantly and we may calculate in the natural way, free of
concern. The reader interested in the details is referred to the
appendix \ref{appendix:rho_details}.

\subsection{Substitution}

We use $\Proc$ for the set of processes, $\QProc$ for the set of
names, and $\id{\{}\vec{y} / \vec{x} \id{\}}$ to denote partial maps,
$s : \QProc \rightarrow \QProc$. A map, $s$ lifts, uniquely, to a map
on process terms, $\widehat{s} : \Proc \rightarrow \Proc$ by the
following equations.

\begin{mathpar}
  (0) \psubstp{Q}{P} := 0 \\
  (R \juxtap S) \psubstp{Q}{P}
  :=    
  (R)\psubstp{Q}{P} \juxtap (S) \psubstp{Q}{P} \\
  (x?(y).R) \psubstp{Q}{P}    
  :=    
  (x)\substp{Q}{P} (z)\concat( (R \psubstn{z}{y}) \psubstp{Q}{P} ) \\
  (\lift{x}{R}) \psubstp{Q}{P}  
  :=
  \lift{(x)\substp{Q}{P}}{ R \psubstp{Q}{P} } \\
%   (\dropn{x})  \psubstp{Q}{P}       
%   := 
%   \left\{ 
%     \begin{array}{ccc} 
%       \dropn{\quotep{Q}} & & x \nameeq \quotep{P} \\
%       \dropn{x} & & otherwise \\
%     \end{array}
%   \right. 
  (\dropn{x})  \psubstp{Q}{P}       
  := 
  \left\{ 
    \begin{array}{ccc} 
      Q & & x \nameeq \quotep{P} \\
      \dropn{x} & & otherwise \\
    \end{array}
  \right.
\end{mathpar}
 

where

\begin{eqnarray}
  (x)\id{\{} \lpquote Q \rpquote / \lpquote P \rpquote \id{\}}            = 
  \left\{ 
    \begin{array}{ccc}
      \lpquote Q \rpquote & & x \nameeq \lpquote P \rpquote \\
      x & & otherwise \\
    \end{array}
  \right. \nonumber
\end{eqnarray}

and $z$ is chosen distinct from $\quotep{P}$, $\quotep{Q}$, the free
names in $Q$, and all the names in $R$. Our $\alpha$-equivalence will
be built in the standard way from this substitution.

\begin{remark}\label{rem:no_self_referential_names}
  One consequence of these definitions is that $\forall P. \quotep{P}
  \not\in \freenames{P}$.
\end{remark}

\subsection{ Dynamic quote: an example }

Anticipating something of what's to come, consider applying the
substitution, $\widehat{\id{\{}u / z \id{\}}}$, to the following pair
of processes, $\lift{w}{y!(z)}$ and $w[ \lpquote y!(z) \rpquote ]$.

\begin{eqnarray}
	\lift{w}{y!(z)}\widehat{\id{\{}u / z \id{\}}}
		& = &
		\lift{w}{y!(u)} \nonumber\\
	w[ \lpquote y!(z) \rpquote ] \widehat{ \id{\{}u / z \id{\}} }
		& = &
		w[ \lpquote y!(z) \rpquote ] \nonumber
\end{eqnarray}

Because the body of the process between quotes is impervious to
substitution, we get radically different answers. In fact, by
examining the first process in an input context,
e.g. $x?(z).\lift{w}{y!(z)}$, we see that the process under the lift
operator may be shaped by prefixed inputs binding a name inside it. In
this sense, the lift operator will be seen as a way to dynamically
construct processes before reifying them as names.

Finally equipped with these standard features we can present the
dynamics of the calculus.

\subsubsection{Operational semantics} 

Finally, we introduce the computational dynamics. What marks these
algebras as distinct from other more traditionally studied algebraic
structures, e.g. vector spaces or polynomial rings, is the manner in
which dynamics is captured. In traditional structures, dynamics is typically
expressed through morphisms between such structures, as in linear maps
between vector spaces or morphisms between rings. In algebras
associated with the semantics of computation, the dynamics is
expressed as part of the algebraic structure itself, through a
reduction reduction relation typically denoted by $\red$. Below, we
give a recursive presentation of this relation for the calculus used
in the encoding.

$\red \subseteq \pi \times \pi$
$\red : \pi \to \mathcal{P}(\pi)$

\begin{mathpar}
  \inferrule* [lab=Comm] { \textsf{match}( x_{src}, x_{trgt} ) } { x_{trgt}?(y)P \; | \; x_{src}!\langle {Q} \rangle \red P\{\quotep{Q}/y}\} }
  \and \\
  \inferrule* [lab=Par] {{P} \red {P}'} {{{P} | {Q}} \red {{P}' | {Q}}}
  \and
  \inferrule* [lab=Equiv]{{{P} \scong {P}'} \andalso {{P}' \red {Q}'} \andalso {{Q}' \scong {Q}}}{{P} \red {Q}}
\end{mathpar}

\begin{eqnarray*}
  match_{\equiv} (\quotep{P},\quotep{Q}) & := & P \equiv Q \\
  match_{\dagger}(\quotep{P},\quotep{Q}) & := & \forall R. P|Q \red^{*} R => R \red^{*} 0 \\
  match_{K}(\quotep{P},\quotep{Q}) & := & K \mbox{ for some context } K
\end{eqnarray*}

$u?(x)P | u!\langle Q \rangle \red P\{\quotep{Q}/x\}$

%We write $\wred$ for $\red^*$, and $P\red$ if $\exists Q $ such that $ P \red Q$.
We write $P\red$ if $\exists Q $ such that $ P \red Q$ and $P\not\red$, otherwise.

\section{Replication}

As mentioned before, it is known that replication (and hence
recursion) can be implemented in a higher-order process algebra
\cite{SangiorgiWalker}. As our first example of calculation with the
machinery thus far presented we give the construction explicitly in
the {\rhoc}.

\begin{eqnarray}
	D_{x} & := & \prefix{x}{y}{(\binpar{\outputp{x}{y}}{@{y}})} \nonumber\\
	\bangp_{x}{P} & := & \binpar{{x}!\langle{\binpar{D_{x}}{P}}\rangle}{D_{x}} \nonumber
\end{eqnarray}

\begin{eqnarray}
	\bangp_{x}{P} & & \nonumber\\
	=
	& {x}!\langle{(\prefix{x}{y}{(\outputp{x}{y} | @{y})) | P}}\rangle 
	      | \prefix{x}{y}{(\outputp{x}{y} | @{y})} & \nonumber\\
	\red
	& (\outputp{x}{y} | @{y})\substn{\quotep{(\prefix{x}{y}{(@{y} | \outputp{x}{y})) | P}}}{y} & \nonumber\\
	=
	& \outputp{x}{\quotep{(\prefix{x}{y}{(\outputp{x}{y} | @{y})) | P}}}
	  | {(\prefix{x}{y}{(\outputp{x}{y} | @{y})) | P}} & \nonumber\\
	\red
	& \ldots & \nonumber\\
	\red^*
	& P | P | \ldots & \nonumber
\end{eqnarray}

Of course, this encoding, as an implementation, runs away, unfolding
$\bangp{P}$ eagerly. A lazier and more implementable replication
operator, restricted to input-guarded processes, may be obtained as follows.

\begin{eqnarray}
\bangp{\prefix{u}{v}{P}} 
	:= 
	\binpar{\lift{x}{\prefix{u}{v}{(\binpar{D(x)}{P})}}}{D(x)} \nonumber
\end{eqnarray}

\begin{remark}
  Note that the lazier definition still does not deal with summation
  or mixed summation (i.e. sums over input and output). The reader is
  invited to construct definitions of replication that deal with these
  features. 

  Further, the definitions are parameterized in a name, $x$. Can you,
  gentle reader, make a definition that eliminates this parameter and
  guarantees no accidental interaction between the replication
  machinery and the process being replicated -- i.e. no accidental
  sharing of names used by the process to get its work done and the
  name(s) used by the replication to effect copying. This latter
  revision of the definition of replication is crucial to obtaining
  the expected identity $!!P \sim !P$.
\end{remark}

\begin{remark}\label{rem:paradoxical_combinator}
  The reader familiar with the lambda calculus will have noticed the
  similarity between $D$ and the paradoxical combinator.

  [Ed. note: the existence of this seems to suggest we have to be more
  restrictive on the set of processes and names we admit if we are to
  support no-cloning.]
\end{remark}

\subsubsection{Bisimulation}

The computational dynamics gives rise to another kind of equivalence,
the equivalence of computational behavior. As previously mentioned
this is typically captured \emph{via} some form of bisimulation.

% The notion we use in this paper is weak barbed bisimulation
% \cite{milner91polyadicpi}.

The notion we use in this paper is derived from weak barbed
bisimulation \cite{milner91polyadicpi}. 

\begin{definition}
An \emph{observation relation}, $\downarrow_{\mathcal N}$, over a set
of names, $\mathcal N$, is the smallest relation satisfying the rules
below.

\infrule[Out-barb]{y \in {\mathcal N}, \; x \nameeq y}
		  {\outputp{x}{v} \downarrow_{\mathcal N} x}
\infrule[Par-barb]{\mbox{$P\downarrow_{\mathcal N} x$ or $Q\downarrow_{\mathcal N} x$}}
		  {\binpar{P}{Q} \downarrow_{\mathcal N} x}

We write $P \Downarrow_{\mathcal N} x$ if there is $Q$ such that 
$P \wred Q$ and $Q \downarrow_{\mathcal N} x$.
\end{definition}

\begin{definition}
%\label{def.bbisim}
An  ${\mathcal N}$-\emph{barbed bisimulation} over a set of names, ${\mathcal N}$, is a symmetric binary relation 
${\mathcal S}_{\mathcal N}$ between agents such that $P\rel{S}_{\mathcal N}Q$ implies:
\begin{enumerate}
\item If $P \red P'$ then $Q \wred Q'$ and $P'\rel{S}_{\mathcal N} Q'$.
\item If $P\downarrow_{\mathcal N} x$, then $Q\Downarrow_{\mathcal N} x$.
\end{enumerate}
$P$ is ${\mathcal N}$-barbed bisimilar to $Q$, written
$P \wbbisim_{\mathcal N} Q$, if $P \rel{S}_{\mathcal N} Q$ for some ${\mathcal N}$-barbed bisimulation ${\mathcal S}_{\mathcal N}$.
\end{definition}

$\mathcal{R} \subseteq \pi \times \pi$

$P \mathcal{R} Q => \forall P'. P \red P' \Rightarrow \exists Q'. Q \red Q', P' \mathcal{R} Q'$

$P \vdash x \Rightarrow Q \vdash x$

\begin{mathpar}
  \inferrule*[lab=Out-barb]{x \nameeq y}{{y}!\langle{Q}\rangle \vdash x}
  \and
  \inferrule*[lab=Par-barb]{\mbox{$P\vdash x$ or $Q\vdash x$}}{\binpar{P}{Q} \vdash x}
\end{mathpar}

\subsubsection{Contexts}

One of the principle advantages of computational calculi like the
$\pi$-calculus is a well-defined notion of context,
contextual-equivalence and a correlation between
contextual-equivalence and notions of bisimulation. The notion of
context allows the decomposition of a process into (sub-)process and
its syntactic environment, its context. Thus, a context may be
thought of as a process with a ``hole'' (written $\Box$) in it. The
application of a context $M$ to a process $P$, written $M[P]$, is
tantamount to filling the hole in $M$ with $P$. In this paper we do
not need the full weight of this theory, but do make use of the notion
of context in the proof the main theorem. 

\begin{mathpar}
  \inferrule* [lab=summation] {} {{M_{M},M_{N}} \bc \Box \;|\; x.M_{A} \;|\; M_{M}+M_{N}}
  \and
  \inferrule* [lab=agent] {} {{M_{A}} \bc (\vec{x})M_{P} \;| \; \clift{P_0,\ldots,M_{P},\ldots,P_N}}
  \and \\
  \inferrule* [lab=process] {} {{M_{P}} \bc M_{N} \;| \;P|M_{P} }
\end{mathpar} 

\begin{mathpar}
  \inferrule* [lab=sychronization] {} {M_{N} \bc \Box \;|\; x?M_{F} \;|\; x!M_{C}}
  \and
  \inferrule* [lab=abstraction] {} {{M_{F}} \bc (x)M_{P} }
  \and
  \inferrule* [lab=concretion] {} {{M_{C}} \bc \langle M_{P} \rangle }
  \and \\
  \inferrule* [lab=process] {} {{M_{P}} \bc M_{N} \;| \;P|M_{P} }
\end{mathpar}

\begin{definition}[contextual application] Given a context $M$, and
  process $P$, we define the \emph{contextual application}, $M[P] :=
  M\{P/\Box\}$. That is, the contextual application of M to P is the
  substitution of $P$ for $\Box$ in $M$.
\end{definition}

$\meaningof{-} : L \to \mathcal{P}(\pi)$

\begin{mathpar}
  \inferrule* [lab=collection] {} {\meaningof{true} = \pi, \and \meaningof{~E} = \pi \setminus \meaningof{E}, \and \meaningof{E_{1} \& E_{2}} = \meaningof{E_{1}} \cap \meaningof{E_{2}}}
\end{mathpar}

\begin{mathpar}
  \inferrule* [lab=structure] {} {\meaningof{0} = \{ P \in \pi | P \equiv 0 \}, \and \\ \meaningof{E_1 | E_2} = \{ P \in \pi | P \equiv P_{1} | P_{2}, P_{1} \in \meaningof{E_{1}}, P_{2} \in \meaningof{E_2}\} }
\end{mathpar}

\begin{mathpar}
 \inferrule* [lab=behavior] {} {\meaningof{\langle a?b \rangle E} = \{ P \in \pi | P \equiv Q | u?(y)P', \\ \and \\\\ \and \\ \;\;\; u \in \meaningof{a}, \forall z.P'\{z/y\} \in \meaningof{E\{z/b\}}\}, \and \\ \meaningof{a!E} = \{ P \in \pi | P \equiv Q | x!\langle P' \rangle, x \in \meaningof{a} P' \in \meaningof{E}\} }
\end{mathpar}

\begin{mathpar}
 \inferrule* [lab=nominal] {} {\meaningof{\quotep{E}} = \{ \quotep{P} \in \quotep{\pi} | P \in \meaningof{E} \}, \and \meaningof{\quotep{P}} = \{ \quotep{Q} \in \quotep{\pi} | P \equiv Q \} \and \\ \meaningof{@\quotep{E}} = \{ P \in \pi | P \equiv @x, x \in \meaningof{E} \}}
\end{mathpar}

\begin{eqnarray*}
  \\
  \meaningof{-} : TS \to ST
\end{eqnarray*}

\begin{eqnarray*}
  \\
  L : TS \to ST
\end{eqnarray*}

\begin{eqnarray*}
  \\
  P \models E \iff P \in \meaningof{E}
\end{eqnarray*}

\begin{eqnarray*}
  P \approx_{L} Q \iff \forall E \in L. P \models E \iff Q \models E
\end{eqnarray*}

\begin{eqnarray*}
  P \approx_{K} Q
\end{eqnarray*}

\begin{eqnarray*}
  P \approx Q
\end{eqnarray*}

$\approx_{K} = \approx = \approx_{L}$

\subsubsection{Contextual duality}

Note that contexts extend the quotation operation to a family of
operations from processes to names. Given a context, $M$, we can
define a \emph{nominal context}, $\quotep{M}$ by $\quotep{M}[P] :=
\quotep{M[P]}$. To foreshadow what is to come we observe that these
operations enjoy a duality with processes very much like the duality
between vectors and maps from vectors to scalars.

Further, because the calculus is essentially higher-order, we have a
correspondence between contexts and processes. More specifically,
given a name $x$ and a context $M$ we can construct $M^{*}_{x}$ such
that 

\begin{mathpar}
  M^{*}_{x} | \lift{x}{P} \red M[P]
\end{mathpar}

namely,

\begin{mathpar}
  M^{*}_{x} := x?(u).M[\dropn{u}]
\end{mathpar}

The dependence of $M^{*}_{x}$ on a name makes it an abstraction, 

\begin{mathpar}
  M^{*} := (x)x?(u).M[\dropn{u}]
\end{mathpar}

\subsection{Additional notation}

It will sometimes be convenient to denote the process a name
quotes. We already have the notation $x = \quotep{P}$, but it will be
convenient to introduce an alternate notation, $\procn{x}$, when we
want to emphasize the connection to the use of the name. Note that, by
virtue of name equivalence, $\quotep{\procn{x}} \nameeq x$; so, the
notation is consistent with previous definitions.

Further, because names have structure it is possible to effect
substitutions on the basis of that structure. This means we need to
upgrade our notation for substitutions, which we accomplish by
adapting comprehension notation. Thus,

\begin{mathpar}
  P\{ y / x : x \in S \}
\end{mathpar}

is interpreted to mean the process derived from P by replacing (in a
capture-avoiding manner) each occurrence of $x$ in $S$ by $y$. For example,

\begin{mathpar}
  P\{ \quotep{\procn{x}|\procn{x}} / x : x \in \freenames{P} \}
\end{mathpar}

will replace each (occurrence) of a free name $x$ in $P$ by
$\quotep{\procn{x}|\procn{x}}$.

Also, we will avail ourselves of the notation $x^{L}$ and $x^{R}$ to
denote injections of a name into disjoint copies of the name
space. There are numerous ways to accomplish this. One example can be
found in \cite{MeredithR05}. This notation overloads to vectors of
names: $\vec{x}^{\pi} := (x_{i}^{\pi} \; : \; 0 \leq i < |\vec{x}| )$ where $\pi \in \{L,R\}$.

We also use $P^{\Box} := P|\Box$.

In \cite{MeredithR05} an interpretation of the new operator is
given. It turns out that there are several possible interpretations
all enjoying the requisite algebraic properties of the operator (see
\cite{milner91polyadicpi}). We will therefore make liberal use of
$(\nu\; \vec{x})P$.

% subsection the_syntax_and_semantics_of_the_notation_system (end)   

\section{Interpretation of QM}
\subsection{Supporting definitions}
\subsubsection{Multiplication}
\begin{mathpar}
  \quotep{Q} \cdot \quotep{R} := \quotep{Q|R}
  \and \\
  \quotep{Q} \cdot P := P\{ \quotep{Q|R} / \quotep{R} : \quotep{R} \in \freenames{P} \}
\end{mathpar}

\paragraph{Discussion}
The first line needs little explanation. The second line says that
each free name of the process is replaced with the multiplication of
that name by the scalar. Multiplication of a scalar (name) by a state
(process) results in a process all the names of which have been `moved
over' by parallel composition with the process the scalar
quotes. There is a subtlety that the bound names have to be
manipulated so that multiplied names aren't accidentally
captured. There are many ways to achieve this.

\begin{remark}\label{rem:multiplication_identities}
  The reader is invited to verify that for all $x,y,z \in \QProc$ and $P \in \Proc$
  \begin{mathpar}
    x \cdot \quotep{0} \equiv x 
    \and
    x \cdot y \equiv y \cdot x
    \and
    x \cdot (y \cdot z) \equiv (x \cdot y) \cdot z
    \and \\
    \quotep{0} \cdot P \equiv P
    \and \\
    x \cdot (y \cdot P) \equiv (x \cdot y) \cdot P
    \and \\
    x \cdot (P|Q) \equiv (x \cdot P) | (x \cdot Q)
    \and \\    
  \end{mathpar}
\end{remark}

\subsubsection{Tensor product}

We define a tensor product on processes by structural induction.

\paragraph{Tensor of sums} First note that all summations, including
$\pzero$ and sequence, can be written $\Sigma_{i} x_{i}.A_{i} +
\Sigma_{j} x_{j}.C_{j}$, where we have grouped input-guarded processes
together and output-guarded processes together.

Thus, we can define the tensor product of two summations, $N_{1}\otimes N_{2}$, where

\begin{mathpar}
  N_{1} := \Sigma_{i} x_{i}.A_{i} + \Sigma_{j} x_{j}.C_{j}
  \and
  N_{2} := \Sigma_{i'} y_{i'}.B_{i'} + \Sigma_{j'} y_{j'}.D_{j'} 
\end{mathpar}

as follows.

\begin{mathpar}
  \Sigma_{i} x_{i}.A_{i} + \Sigma_{j} x_{j}.C_{j} \otimes \Sigma_{i'}
  y_{i'}.B_{i'} + \Sigma_{j'} y_{j'}.D_{j'} 
  \and \\
  := \; \Sigma_{i} \Sigma_{i'} \quotep{\stackrel{\vee}{x_{i}}| \stackrel{\vee}{y_{i'}}}.(A_{i}\otimes B_{i'}) \; | \; \Sigma_{i'} \Sigma_{i} \quotep{\stackrel{\vee}{y_{i'}}|\stackrel{\vee}{x_{i}}}.(B_{i'}\otimes A_{i})
  \and
  \;\; | \;\; \Sigma_{j} \Sigma_{j'} \quotep{\stackrel{\vee}{x_{j}}|\stackrel{\vee}{y_{j'}}}.(A_{j}\otimes B_{j'}) \; | \; \Sigma_{j'} \Sigma_{j} \quotep{\stackrel{\vee}{y_{j'}}|\stackrel{\vee}{x_{j}}}.(B_{j'}\otimes A_{j})
\end{mathpar}

\begin{remark}
  Do we need to $x^{L}$ and $y^{R}$ for this construction as well?
\end{remark}

\paragraph{Tensor of parallel compositions} Next, we distribute tensor
over par.

\begin{mathpar}
  P_{1}|P_{2} \otimes Q_{1}|Q_{2} := (P_{1} \otimes Q_{1}) | (P_{1}
  \otimes Q_{2}) | (P_{2} \otimes Q_{1}) | (P_{2} \otimes Q_{2})
\end{mathpar}

\paragraph{Tensor with dropped names} We treat tensor of a
process with a dropped name as parallel composition.

\begin{mathpar}
  P \otimes \dropn{x} := P | \dropn{x}
\end{mathpar}

\paragraph{Tensor of agents}

Finally, we need to define tensor on agents. Note that the definition
of tensor on normal products only tensors inputs with inputs and
outputs with outputs. Thus, we only have to define the operation on
``homogeneous'' pairings.

\begin{mathpar}
  (\vec{x})P \otimes (\vec{y})Q
  \and \\
  := (x_{0}^{L}|y_{0}^{R},\ldots,x_{0}^{L}|y_{n}^{R},\ldots,x_{m}^{L}|y_{0}^{R},\ldots,x_{m}^{L}|y_{n}^R)(P\{ \vec{x}^{L}/\vec{x}\} \otimes Q \{ \vec{y}^{R}/\vec{y}\})
  \and \\
  \clift{\vec{P}} \otimes \clift{\vec{Q}}
  \and \\
  := \clift{P_{0}\otimes Q_{0},\ldots,P_{0}\otimes Q_{n},\ldots,P_{m}\otimes Q_{0},\ldots,P_{m}\otimes Q_{n}}
\end{mathpar}

\begin{remark}
  Observe that arities of tensored abstractions matches arities of
  tensored concretions if the original arities matched. Note also that
  the length of the arities corresponds to the increase in dimension
  we see in ordinary vector space tensor product.
\end{remark}

\begin{remark}
  Operationally, this definition distributes the tensor down to
  components ``linked'' by summation. Tensor over summation is
  intriguing in that it mixes names. Moreover, as a consequence of the
  way it mixes names we have the identities for all $x \in \QProc$ and
  $P,Q \in \Proc$

  \begin{mathpar}
    (x \cdot P) \otimes Q \equiv x \cdot (P \otimes Q) \equiv P \otimes (x \cdot Q)
    \and
    P \otimes \pzero \equiv P
  \end{mathpar}

  that the reader is invited to verify.
\end{remark}

\subsubsection{Annihilation}
\begin{mathpar}
  P^{\perp} := \{ Q | \forall R. P|Q \red^{*} R \Rightarrow R \red^{*} \pzero \}
  \and \\
  P^{\underline{\perp}} := \Sigma_{Q \in P^{\perp}} \quotep{Q}?(y).(\dropn{y}|Q) | \Sigma_{Q \in P^{\perp}} \quotep{Q}\clift{\Box}
\end{mathpar}

\paragraph{Discussion} The reader will note that $P^{\perp}$ is a
\emph{set} of processes, while $P^{\underline{\perp}}$ is a
\emph{context}. We call the set $P^{\perp}$ the \emph{annihilators} of
$P$. The parallel composition of a process in the annihilators of $P$
with $P$ will result in a process, the state space of which has all
paths eventually leading to $\pzero$. Execution may endure loops; but
under reasonable conditions of fairness (naturally guaranteed under
most notions of bisimulation) such a composite process cannot get
stuck in such a loop and will, eventually pop out and terminate.

The context $P^{\underline{\perp}}$ is ready and willing to ``take the
$P$ out of'' the process to which it is applied. It will effectively
transmit the code of the process to which it is applied to one of the
annihilators and run the process against it.

\subsubsection{Evaluation}
We fix $M$ a domain of fully abstract interpretation with an equality
coincident with bisimulation. We take $\meaningof{\cdot} : \Proc \to
M$ to be the map interpreting processes and $\nmeaningof{\cdot} : \M
\to Proc$ to be the map running the other way. Then we define

\begin{mathpar}
  \int P := \nmeaningof{\meaningof{P}}
\end{mathpar}

\paragraph{Discussion}
There are many fully abstract interpretations of Milner's
$\pi$-calculus. Any of them can be used as a basis for interpreting
the reflective calculus here. Equipped with such a domain it is
largely a matter of grinding through to check that the Yoneda
construction for the normalization-by-evaluation program can be
extended to this setting.

\begin{remark}
  The reader is invited to verify that $\int (P^{\underline{\perp}}[P]) = 0$.
\end{remark}

\subsection{Quantum mechanics}

Table \ref{tbl:core_qm_op_defns} gives the core operational definitions

\begin{table}[htp]\label{tbl:core_qm_op_defns}
  \center{
    \fbox{
      \begin{tabular}{c|c}
        quantum mechanics & process calculus \\
        \hline
        scalar & $x := \quotep{P}$ \\
        state vector & $\state{P} := P$ \\
        dual & $\state{P}^{*} := \event{P^{\underline{\perp}}} := \quotep{P^{\underline{\perp}}}[-]$ \\
        matrix & $ \Sigma_{\alpha} \state{P_{\alpha}}x_{\alpha}\event{Q_{\alpha}}$ \\
        vector addition & $\state{P} + \state{Q} := \state{P | Q}$ \\
        tensor product & $\state{P} \otimes \state{Q} := \state{P \otimes Q}$ \\
        inner product & $\innerprod{P}{Q} := \quotep{\int P^{\underline{\perp}}[Q]}$ \\
      \end{tabular}
    }
  }
  \caption{QM - operational definitions}
\end{table}

where

\begin{mathpar}
  \prmatrix{P}{Q} := \fprmatrix{P}{\quotep{\pzero}}{Q}
  \and
  \fprmatrix{P}{x}{Q} := (\state{P},x,\event{Q})
  \and
  (\fprmatrix{P}{x}{Q})(\state{R}) := x \cdot \innerprod{Q}{R} \cdot \state{P}
  \and
  (\fprmatrix{P}{x}{Q})(\event{R}) := x \cdot \innerprod{R}{P} \cdot \event{Q}
\end{mathpar}

\paragraph{Discussion}
As promised: vectors (aka states) are represented as processes; duals
as contextual duals; inner product definition should be compared with
standard inner product definition for ....

\begin{remark}
  Assuming $\int (P^{\underline{\perp}}[P]) = 0$, the reader is
  invited to verify that $(\fprmatrix{P}{x}{P})(\state{P}) = x \cdot \state{P}$.
\end{remark}

\begin{remark}
  The reader is invited to verify that $\innerprod{P}{Q}$ could
  equally well have been written $\quotep{\int \stackrel{\vee}{x}}$
  where $x = \event{P^{\underline{\perp}}}(Q)$.

  One of the motivations for this remark is that there is another way
  to factor these operations. We could package up evaluation in the dual:

  \begin{mathpar}
    \state{P}^{*} := \event{\int P^{\underline{\perp}}} := \quotep{\int P^{\underline{\perp}}}[-]
  \end{mathpar}

  and then have inner product defined by
  
  \begin{mathpar}
    \innerprod{P}{Q} := \event{P}(Q)
  \end{mathpar}

  Hopefully, experience with the calculations will provide guidance on
  the best factoring.
\end{remark}

\begin{remark}
  Assuming $\int (P^{\underline{\perp}}[P]) = 0$, the reader is
  invited to verify that $\forall P,Q. (\prmatrix{0}{Q})(\state{0}) =
  \state{0}$ and dually $(\prmatrix{P}{0})(\event{0}) = \event{0}$.
\end{remark}

\begin{remark}
  i'm a little worried that i don't (yet) have proper support for
  complex conjugacy. But, the observation above may give us a
  clue. According to Abramsky, it must be the case that the scalars
  are iso to the homset of the identity for the tensor -- which the
  observation above characterizes. 

  For now, we will simply bookmark the notion with $\overline{x}$.
\end{remark}

\subsubsection{Adjointness}

We need to give a definition of $(\cdot)^{\dagger}$ for matrices. The
obvious candidate definition is
\begin{mathpar}
(\Sigma_{\alpha}\fprmatrix{P_{\alpha}}{x_{\alpha}}{Q_{\alpha}})^{\dagger}
= \Sigma_{\alpha}\fprmatrix{(Q_{\alpha}^{\underline{\perp}})^{*}}{\overline{x}_{\alpha}}{P_{\alpha}^{\underline{\perp}}} 
\end{mathpar}

But, $(Q_{\alpha}^{\underline{\perp}})^{*}$ requires a name along
which to communicate the process to achieve the context application.

\subsubsection{Basis for a basis}
If processes label states and ``addition'' of states (a.k.a. vector
addition) is interpreted as parallel composition, what corresponds to
notions of linear independence and basis? Here, we recall that Yoshida
has developed a set of \emph{combinators} for an asynchronous verison
of Milner's $\pi$-calculus. These are a finite set of processes such
any process can be expressed as parallel composition of these
combinators together with liberal uses of the new operator and
replication. We can simply give a translation of these into the
present calculus and have reasonable expectation that the property
carries over. That is, that the resultant set allows to express all
processes via parallel composition. Note, however, that there is no
new operator or replication in this calculus. As a result, we expect
that the corresponding set is actually infinite. That is, we expect
that the space is actually infinite dimensional.

\begin{remark}
  The attentive reader may be a bit concerned. Certainly, the
  collection $S$, $K$ and $I$ is a finite set of
  combinators. Shouldn't we expect to see a finite set of combinators
  for an effectively equivalent system? i am very sympathetic to this
  critique and feel it warrants full attention. On the other hand, i
  also have in mind the following analogy. The natural numbers, as a
  monoid under addition, has exactly $1$ generator, while the natural
  numbers, as a monoid under multiplication, has countably many
  generators (the primes). We observe that the application of the
  lambda calculus is much less resource sensitive than the parallel
  composition of the $\pi$-calculus. Could it be the case that we have
  an analogy of the form
  
  \begin{mathpar}
    m + n : MN :: m*n : M|N
  \end{mathpar}

  giving a similar blow up in the set of ``primes''?  This is such a
  wonderful thought that, even if it's not true, i think it's worth
  writing down.
\end{remark}
 

\documentclass[12pt]{llncs}
%\documentclass{jktr}

\usepackage[pdftex]{hyperref}                   
\usepackage {listings}
\usepackage {mathpartir}
\usepackage{bcprules}
%\usepackage{listings}
                       
\usepackage{graphicx} 
%\usepackage[margins=2.5cm,nohead,nofoot]{geometry}
%\usepackage{geometry}
\usepackage{amsfonts}
\usepackage{amstext}
\usepackage{latexsym}
\usepackage{amssymb}
\usepackage{color}


%\include{myPreamble}
\include{qm2pi.local} 

%\ifpdf
%\usepackage[pdftex]{graphicx}
%\else
%\usepackage{graphicx}
%\fi

 % \ifpdf
%  \usepackage{pdfsync}
%  \if


%\title{Brief Article}
%\author{David F. Snyder}
%\author{L.G. Meredith}

%\address{Dept. of Math., Texas State University--San Marcos, San Marcos, TX 78666}
       
\pagestyle{empty}


\begin{document}

\lstset{language=[Objective]Caml,frame=shadowbox}

\input{qm2pi.front}

% section front matter (end)

\input{qm2pi.intro} 
 
% section introduction (end)

% \input{qm2pi.knotations} 

% section notation (end)

\input{qm2pi.process.calculi} 

% section concurrent_process_calculi_and_spatial_logics_ (end)
    
%\input{qm2pi.knots2pi} 

%\input{qm2pi.trefoil} 

%\input{qm2pi.mainthm} 

% subsection basic_interpretation (end)

%\input{qm2pi.rho.presentation} 
\subsection{The syntax and semantics of the notation system}\label{sub:the_syntax_and_semantics_of_the_notation_system} % (fold)

We now summarize a technical presentation of the calculus that
embodies our theory of dynamics. The typical presentation of such a
calculus follows the style of giving generators and relations on
them. The grammar, below, describing term constructors, freely
generates the set of processes, $\Proc$. This set is then quotiented
by a relation known as structural congruence and it is over this set
that the notion of dynamics is expressed. This presentation is
essentially that of \cite{MeredithR05} with the addition of
polyadicity and summation. For readability we have relegated some of
the technical subtleties to an appendix.

\subsubsection{Process grammar}\label{subsub:process_grammar}

\begin{mathpar}
  \inferrule* [lab=synchronization] {} {{M} \bc \pzero \;|\; x?F \;|\; x!C }
  \and
  \inferrule* [lab=abstraction] {} {{F} \bc (x)P}
  \and
  \inferrule* [lab=concretion] {} {{C} \bc \langle Q \rangle}
  \and
  \inferrule* [lab=process] {} {{P,Q} \bc M \;| \;P|Q \;|\; @{x}}
  \and
  \inferrule* [lab=name] {} {{x} \bc \quotep{P}}
\end{mathpar} 

Note that $\vec{x}$ (resp. $\vec{P}$) denotes a vector of names
(resp. processes) of length $|\vec{x}|$ (resp. $|\vec{P}|$). We adopt
the following useful abbreviations.

\begin{mathpar}
   x?(\vec{y}).P := x.(\vec{y})P \and  x\clift{\vec{P}} := x.\clift{\vec{P}}
   \and x!(y) := \lift{x}{\dropn{y}}
   \and \Pi_{i=0}^{n-1}P_i := P_0 | \ldots | P_{n-1}
\end{mathpar}

\subsubsection{Structural congruence}

\paragraph{Free and bound names and alpha-equivalence.} At the
core of structural equivalence is alpha-equivalence which identifies
process that are the same up to a change of variable. Formally, we
recognize the distinction between free and bound names. The free names
of a process, $\freenames{P}$, may be calculated recursively as
follows:

\begin{mathpar}
\freenames{\pzero} := \emptyset
  \and \\
  \freenames{x?(y).P} := \{ x \} \cup (\freenames{P} \setminus \{ y \})
  \and 
  \freenames{x!\langle P \rangle} := \{ x \} \cup \{ P \} 
  \and \\
  \freenames{P|Q} := \freenames{P} \cup \freenames{Q}
  \and \\
  \freenames{@{x}} := \{ x \}
\end{mathpar}

$\pi$
$\quotep{\pi}$

$\freenames{-} : \pi \to \mathcal{P}(\quotep{\pi})$

\begin{eqnarray*}
  \freenames{\pzero} & := & \emptyset \\
  \freenames{x?(y).P} & := & \{ x \} \cup (\freenames{P} \setminus \{ y \}) \\
  \freenames{x!\langle P \rangle} & := & \{ x \} \cup \{ P \} \\
  \freenames{P|Q} & := & \freenames{P} \cup \freenames{Q} \\
  \freenames{\dropn{x}} & := & \{ x \}
\end{eqnarray*}

The bound names of a process, $\boundnames{P}$, are those names occurring in $P$
that are not free. For example, in $x?(y).0$, the name $x$ is free, while $y$ is bound.

\begin{mathpar}
  \inferrule* [lab=monoidal-laws] {} { P|Q \equiv Q|P \and P|0 \equiv P \and P|(Q|R) \equiv (P|Q)|R }
\end{mathpar}

\begin{mathpar}
  \inferrule* [lab=alpha-equivalence] {} { (x)P \equiv (y)P\{y/x\} \and y \not\in \freenames{P} }
\end{mathpar}

\begin{definition}
Then two processes, $P,Q$, are alpha-equivalent if $P = Q\{\vec{y}/\vec{x}\}$ for
some $\vec{x} \in \boundnames{Q},\vec{y} \in \boundnames{P}$, where $Q\{\vec{y}/\vec{x}\}$
denotes the capture-avoiding substitution of $\vec{y}$ for $\vec{x}$ in $Q$.
\end{definition}

\begin{definition}
  The {\em structural congruence} \cite{SangiorgiWalker} , $\equiv$,
  between processes is the least congruence containing
  alpha-equivalence, satisfying the abelian monoid laws
  (associativity, commutativity and $\pzero$ as identity) for parallel
  composition $|$ and for summation $+$.
\end{definition}

\subsection{Name equivalence}

We take name equivalence, written $\nameeq$, to be the smallest
equivalence relation generated by the following rules.

\begin{mathpar}
\inferrule*[lab=Quote-drop]
{ }
{ \quotep{@{x}} \nameeq x }

\inferrule*[lab=Struct-equiv]
{ P \scong Q }
{ \quotep{P} \nameeq \quotep{Q} }
\end{mathpar}

The astute reader will have noticed that the mutual recursion of names
and processes imposes a mutual recursion on alpha-equivalence and
structural equivalence via name-equivalence. Fortunately, all of this
works out pleasantly and we may calculate in the natural way, free of
concern. The reader interested in the details is referred to the
appendix \ref{appendix:rho_details}.

\subsection{Substitution}

We use $\Proc$ for the set of processes, $\QProc$ for the set of
names, and $\id{\{}\vec{y} / \vec{x} \id{\}}$ to denote partial maps,
$s : \QProc \rightarrow \QProc$. A map, $s$ lifts, uniquely, to a map
on process terms, $\widehat{s} : \Proc \rightarrow \Proc$ by the
following equations.

\begin{mathpar}
  (0) \psubstp{Q}{P} := 0 \\
  (R \juxtap S) \psubstp{Q}{P}
  :=    
  (R)\psubstp{Q}{P} \juxtap (S) \psubstp{Q}{P} \\
  (x?(y).R) \psubstp{Q}{P}    
  :=    
  (x)\substp{Q}{P} (z)\concat( (R \psubstn{z}{y}) \psubstp{Q}{P} ) \\
  (\lift{x}{R}) \psubstp{Q}{P}  
  :=
  \lift{(x)\substp{Q}{P}}{ R \psubstp{Q}{P} } \\
%   (\dropn{x})  \psubstp{Q}{P}       
%   := 
%   \left\{ 
%     \begin{array}{ccc} 
%       \dropn{\quotep{Q}} & & x \nameeq \quotep{P} \\
%       \dropn{x} & & otherwise \\
%     \end{array}
%   \right. 
  (\dropn{x})  \psubstp{Q}{P}       
  := 
  \left\{ 
    \begin{array}{ccc} 
      Q & & x \nameeq \quotep{P} \\
      \dropn{x} & & otherwise \\
    \end{array}
  \right.
\end{mathpar}
 

where

\begin{eqnarray}
  (x)\id{\{} \lpquote Q \rpquote / \lpquote P \rpquote \id{\}}            = 
  \left\{ 
    \begin{array}{ccc}
      \lpquote Q \rpquote & & x \nameeq \lpquote P \rpquote \\
      x & & otherwise \\
    \end{array}
  \right. \nonumber
\end{eqnarray}

and $z$ is chosen distinct from $\quotep{P}$, $\quotep{Q}$, the free
names in $Q$, and all the names in $R$. Our $\alpha$-equivalence will
be built in the standard way from this substitution.

\begin{remark}\label{rem:no_self_referential_names}
  One consequence of these definitions is that $\forall P. \quotep{P}
  \not\in \freenames{P}$.
\end{remark}

\subsection{ Dynamic quote: an example }

Anticipating something of what's to come, consider applying the
substitution, $\widehat{\id{\{}u / z \id{\}}}$, to the following pair
of processes, $\lift{w}{y!(z)}$ and $w[ \lpquote y!(z) \rpquote ]$.

\begin{eqnarray}
	\lift{w}{y!(z)}\widehat{\id{\{}u / z \id{\}}}
		& = &
		\lift{w}{y!(u)} \nonumber\\
	w[ \lpquote y!(z) \rpquote ] \widehat{ \id{\{}u / z \id{\}} }
		& = &
		w[ \lpquote y!(z) \rpquote ] \nonumber
\end{eqnarray}

Because the body of the process between quotes is impervious to
substitution, we get radically different answers. In fact, by
examining the first process in an input context,
e.g. $x?(z).\lift{w}{y!(z)}$, we see that the process under the lift
operator may be shaped by prefixed inputs binding a name inside it. In
this sense, the lift operator will be seen as a way to dynamically
construct processes before reifying them as names.

Finally equipped with these standard features we can present the
dynamics of the calculus.

\subsubsection{Operational semantics} 

Finally, we introduce the computational dynamics. What marks these
algebras as distinct from other more traditionally studied algebraic
structures, e.g. vector spaces or polynomial rings, is the manner in
which dynamics is captured. In traditional structures, dynamics is typically
expressed through morphisms between such structures, as in linear maps
between vector spaces or morphisms between rings. In algebras
associated with the semantics of computation, the dynamics is
expressed as part of the algebraic structure itself, through a
reduction reduction relation typically denoted by $\red$. Below, we
give a recursive presentation of this relation for the calculus used
in the encoding.

$\red \subseteq \pi \times \pi$
$\red : \pi \to \mathcal{P}(\pi)$

\begin{mathpar}
  \inferrule* [lab=Comm] { \textsf{match}( x_{src}, x_{trgt} ) } { x_{trgt}?(y)P \; | \; x_{src}!\langle {Q} \rangle \red P\{\quotep{Q}/y}\} }
  \and \\
  \inferrule* [lab=Par] {{P} \red {P}'} {{{P} | {Q}} \red {{P}' | {Q}}}
  \and
  \inferrule* [lab=Equiv]{{{P} \scong {P}'} \andalso {{P}' \red {Q}'} \andalso {{Q}' \scong {Q}}}{{P} \red {Q}}
\end{mathpar}

\begin{eqnarray*}
  match_{\equiv} (\quotep{P},\quotep{Q}) & := & P \equiv Q \\
  match_{\dagger}(\quotep{P},\quotep{Q}) & := & \forall R. P|Q \red^{*} R => R \red^{*} 0 \\
  match_{K}(\quotep{P},\quotep{Q}) & := & K \mbox{ for some context } K
\end{eqnarray*}

$u?(x)P | u!\langle Q \rangle \red P\{\quotep{Q}/x\}$

%We write $\wred$ for $\red^*$, and $P\red$ if $\exists Q $ such that $ P \red Q$.
We write $P\red$ if $\exists Q $ such that $ P \red Q$ and $P\not\red$, otherwise.

\section{Replication}

As mentioned before, it is known that replication (and hence
recursion) can be implemented in a higher-order process algebra
\cite{SangiorgiWalker}. As our first example of calculation with the
machinery thus far presented we give the construction explicitly in
the {\rhoc}.

\begin{eqnarray}
	D_{x} & := & \prefix{x}{y}{(\binpar{\outputp{x}{y}}{@{y}})} \nonumber\\
	\bangp_{x}{P} & := & \binpar{{x}!\langle{\binpar{D_{x}}{P}}\rangle}{D_{x}} \nonumber
\end{eqnarray}

\begin{eqnarray}
	\bangp_{x}{P} & & \nonumber\\
	=
	& {x}!\langle{(\prefix{x}{y}{(\outputp{x}{y} | @{y})) | P}}\rangle 
	      | \prefix{x}{y}{(\outputp{x}{y} | @{y})} & \nonumber\\
	\red
	& (\outputp{x}{y} | @{y})\substn{\quotep{(\prefix{x}{y}{(@{y} | \outputp{x}{y})) | P}}}{y} & \nonumber\\
	=
	& \outputp{x}{\quotep{(\prefix{x}{y}{(\outputp{x}{y} | @{y})) | P}}}
	  | {(\prefix{x}{y}{(\outputp{x}{y} | @{y})) | P}} & \nonumber\\
	\red
	& \ldots & \nonumber\\
	\red^*
	& P | P | \ldots & \nonumber
\end{eqnarray}

Of course, this encoding, as an implementation, runs away, unfolding
$\bangp{P}$ eagerly. A lazier and more implementable replication
operator, restricted to input-guarded processes, may be obtained as follows.

\begin{eqnarray}
\bangp{\prefix{u}{v}{P}} 
	:= 
	\binpar{\lift{x}{\prefix{u}{v}{(\binpar{D(x)}{P})}}}{D(x)} \nonumber
\end{eqnarray}

\begin{remark}
  Note that the lazier definition still does not deal with summation
  or mixed summation (i.e. sums over input and output). The reader is
  invited to construct definitions of replication that deal with these
  features. 

  Further, the definitions are parameterized in a name, $x$. Can you,
  gentle reader, make a definition that eliminates this parameter and
  guarantees no accidental interaction between the replication
  machinery and the process being replicated -- i.e. no accidental
  sharing of names used by the process to get its work done and the
  name(s) used by the replication to effect copying. This latter
  revision of the definition of replication is crucial to obtaining
  the expected identity $!!P \sim !P$.
\end{remark}

\begin{remark}\label{rem:paradoxical_combinator}
  The reader familiar with the lambda calculus will have noticed the
  similarity between $D$ and the paradoxical combinator.

  [Ed. note: the existence of this seems to suggest we have to be more
  restrictive on the set of processes and names we admit if we are to
  support no-cloning.]
\end{remark}

\subsubsection{Bisimulation}

The computational dynamics gives rise to another kind of equivalence,
the equivalence of computational behavior. As previously mentioned
this is typically captured \emph{via} some form of bisimulation.

% The notion we use in this paper is weak barbed bisimulation
% \cite{milner91polyadicpi}.

The notion we use in this paper is derived from weak barbed
bisimulation \cite{milner91polyadicpi}. 

\begin{definition}
An \emph{observation relation}, $\downarrow_{\mathcal N}$, over a set
of names, $\mathcal N$, is the smallest relation satisfying the rules
below.

\infrule[Out-barb]{y \in {\mathcal N}, \; x \nameeq y}
		  {\outputp{x}{v} \downarrow_{\mathcal N} x}
\infrule[Par-barb]{\mbox{$P\downarrow_{\mathcal N} x$ or $Q\downarrow_{\mathcal N} x$}}
		  {\binpar{P}{Q} \downarrow_{\mathcal N} x}

We write $P \Downarrow_{\mathcal N} x$ if there is $Q$ such that 
$P \wred Q$ and $Q \downarrow_{\mathcal N} x$.
\end{definition}

\begin{definition}
%\label{def.bbisim}
An  ${\mathcal N}$-\emph{barbed bisimulation} over a set of names, ${\mathcal N}$, is a symmetric binary relation 
${\mathcal S}_{\mathcal N}$ between agents such that $P\rel{S}_{\mathcal N}Q$ implies:
\begin{enumerate}
\item If $P \red P'$ then $Q \wred Q'$ and $P'\rel{S}_{\mathcal N} Q'$.
\item If $P\downarrow_{\mathcal N} x$, then $Q\Downarrow_{\mathcal N} x$.
\end{enumerate}
$P$ is ${\mathcal N}$-barbed bisimilar to $Q$, written
$P \wbbisim_{\mathcal N} Q$, if $P \rel{S}_{\mathcal N} Q$ for some ${\mathcal N}$-barbed bisimulation ${\mathcal S}_{\mathcal N}$.
\end{definition}

$\mathcal{R} \subseteq \pi \times \pi$

$P \mathcal{R} Q => \forall P'. P \red P' \Rightarrow \exists Q'. Q \red Q', P' \mathcal{R} Q'$

$P \vdash x \Rightarrow Q \vdash x$

\begin{mathpar}
  \inferrule*[lab=Out-barb]{x \nameeq y}{{y}!\langle{Q}\rangle \vdash x}
  \and
  \inferrule*[lab=Par-barb]{\mbox{$P\vdash x$ or $Q\vdash x$}}{\binpar{P}{Q} \vdash x}
\end{mathpar}

\subsubsection{Contexts}

One of the principle advantages of computational calculi like the
$\pi$-calculus is a well-defined notion of context,
contextual-equivalence and a correlation between
contextual-equivalence and notions of bisimulation. The notion of
context allows the decomposition of a process into (sub-)process and
its syntactic environment, its context. Thus, a context may be
thought of as a process with a ``hole'' (written $\Box$) in it. The
application of a context $M$ to a process $P$, written $M[P]$, is
tantamount to filling the hole in $M$ with $P$. In this paper we do
not need the full weight of this theory, but do make use of the notion
of context in the proof the main theorem. 

\begin{mathpar}
  \inferrule* [lab=summation] {} {{M_{M},M_{N}} \bc \Box \;|\; x.M_{A} \;|\; M_{M}+M_{N}}
  \and
  \inferrule* [lab=agent] {} {{M_{A}} \bc (\vec{x})M_{P} \;| \; \clift{P_0,\ldots,M_{P},\ldots,P_N}}
  \and \\
  \inferrule* [lab=process] {} {{M_{P}} \bc M_{N} \;| \;P|M_{P} }
\end{mathpar} 

\begin{mathpar}
  \inferrule* [lab=sychronization] {} {M_{N} \bc \Box \;|\; x?M_{F} \;|\; x!M_{C}}
  \and
  \inferrule* [lab=abstraction] {} {{M_{F}} \bc (x)M_{P} }
  \and
  \inferrule* [lab=concretion] {} {{M_{C}} \bc \langle M_{P} \rangle }
  \and \\
  \inferrule* [lab=process] {} {{M_{P}} \bc M_{N} \;| \;P|M_{P} }
\end{mathpar}

\begin{definition}[contextual application] Given a context $M$, and
  process $P$, we define the \emph{contextual application}, $M[P] :=
  M\{P/\Box\}$. That is, the contextual application of M to P is the
  substitution of $P$ for $\Box$ in $M$.
\end{definition}

$\meaningof{-} : L \to \mathcal{P}(\pi)$

\begin{mathpar}
  \inferrule* [lab=collection] {} {\meaningof{true} = \pi, \and \meaningof{~E} = \pi \setminus \meaningof{E}, \and \meaningof{E_{1} \& E_{2}} = \meaningof{E_{1}} \cap \meaningof{E_{2}}}
\end{mathpar}

\begin{mathpar}
  \inferrule* [lab=structure] {} {\meaningof{0} = \{ P \in \pi | P \equiv 0 \}, \and \\ \meaningof{E_1 | E_2} = \{ P \in \pi | P \equiv P_{1} | P_{2}, P_{1} \in \meaningof{E_{1}}, P_{2} \in \meaningof{E_2}\} }
\end{mathpar}

\begin{mathpar}
 \inferrule* [lab=behavior] {} {\meaningof{\langle a?b \rangle E} = \{ P \in \pi | P \equiv Q | u?(y)P', \\ \and \\\\ \and \\ \;\;\; u \in \meaningof{a}, \forall z.P'\{z/y\} \in \meaningof{E\{z/b\}}\}, \and \\ \meaningof{a!E} = \{ P \in \pi | P \equiv Q | x!\langle P' \rangle, x \in \meaningof{a} P' \in \meaningof{E}\} }
\end{mathpar}

\begin{mathpar}
 \inferrule* [lab=nominal] {} {\meaningof{\quotep{E}} = \{ \quotep{P} \in \quotep{\pi} | P \in \meaningof{E} \}, \and \meaningof{\quotep{P}} = \{ \quotep{Q} \in \quotep{\pi} | P \equiv Q \} \and \\ \meaningof{@\quotep{E}} = \{ P \in \pi | P \equiv @x, x \in \meaningof{E} \}}
\end{mathpar}

\begin{eqnarray*}
  \\
  \meaningof{-} : TS \to ST
\end{eqnarray*}

\begin{eqnarray*}
  \\
  L : TS \to ST
\end{eqnarray*}

\begin{eqnarray*}
  \\
  P \models E \iff P \in \meaningof{E}
\end{eqnarray*}

\begin{eqnarray*}
  P \approx_{L} Q \iff \forall E \in L. P \models E \iff Q \models E
\end{eqnarray*}

\begin{eqnarray*}
  P \approx_{K} Q
\end{eqnarray*}

\begin{eqnarray*}
  P \approx Q
\end{eqnarray*}

$\approx_{K} = \approx = \approx_{L}$

\subsubsection{Contextual duality}

Note that contexts extend the quotation operation to a family of
operations from processes to names. Given a context, $M$, we can
define a \emph{nominal context}, $\quotep{M}$ by $\quotep{M}[P] :=
\quotep{M[P]}$. To foreshadow what is to come we observe that these
operations enjoy a duality with processes very much like the duality
between vectors and maps from vectors to scalars.

Further, because the calculus is essentially higher-order, we have a
correspondence between contexts and processes. More specifically,
given a name $x$ and a context $M$ we can construct $M^{*}_{x}$ such
that 

\begin{mathpar}
  M^{*}_{x} | \lift{x}{P} \red M[P]
\end{mathpar}

namely,

\begin{mathpar}
  M^{*}_{x} := x?(u).M[\dropn{u}]
\end{mathpar}

The dependence of $M^{*}_{x}$ on a name makes it an abstraction, 

\begin{mathpar}
  M^{*} := (x)x?(u).M[\dropn{u}]
\end{mathpar}

\subsection{Additional notation}

It will sometimes be convenient to denote the process a name
quotes. We already have the notation $x = \quotep{P}$, but it will be
convenient to introduce an alternate notation, $\procn{x}$, when we
want to emphasize the connection to the use of the name. Note that, by
virtue of name equivalence, $\quotep{\procn{x}} \nameeq x$; so, the
notation is consistent with previous definitions.

Further, because names have structure it is possible to effect
substitutions on the basis of that structure. This means we need to
upgrade our notation for substitutions, which we accomplish by
adapting comprehension notation. Thus,

\begin{mathpar}
  P\{ y / x : x \in S \}
\end{mathpar}

is interpreted to mean the process derived from P by replacing (in a
capture-avoiding manner) each occurrence of $x$ in $S$ by $y$. For example,

\begin{mathpar}
  P\{ \quotep{\procn{x}|\procn{x}} / x : x \in \freenames{P} \}
\end{mathpar}

will replace each (occurrence) of a free name $x$ in $P$ by
$\quotep{\procn{x}|\procn{x}}$.

Also, we will avail ourselves of the notation $x^{L}$ and $x^{R}$ to
denote injections of a name into disjoint copies of the name
space. There are numerous ways to accomplish this. One example can be
found in \cite{MeredithR05}. This notation overloads to vectors of
names: $\vec{x}^{\pi} := (x_{i}^{\pi} \; : \; 0 \leq i < |\vec{x}| )$ where $\pi \in \{L,R\}$.

We also use $P^{\Box} := P|\Box$.

In \cite{MeredithR05} an interpretation of the new operator is
given. It turns out that there are several possible interpretations
all enjoying the requisite algebraic properties of the operator (see
\cite{milner91polyadicpi}). We will therefore make liberal use of
$(\nu\; \vec{x})P$.

% subsection the_syntax_and_semantics_of_the_notation_system (end)   

\input{qm2pi.qmops} 

\input{qm2pi.sterngerlach} 

\input{qm2pi.metric} 

% section concurrent_process_calculi (end)

%\input{qm2pi.proofsketch}

% section proof sketch (end)

%\input{qm2pi.slviaknots} 

% section spatial logic via knots (end)

\input{qm2pi.conclusion}

% section conclusion (end)

%\input{qm2pi.dtcodes} 

% section wiring algorithm (end)

\input{qm2pi.ack} 

% section acknowledgments (end)

\newpage


\bibliographystyle{plain}   
\bibliography{../../biblios/main.bib}

\input{qm2pi.rhodetails}

\end{document}

 

\documentclass[12pt]{llncs}
%\documentclass{jktr}

\usepackage[pdftex]{hyperref}                   
\usepackage {listings}
\usepackage {mathpartir}
\usepackage{bcprules}
%\usepackage{listings}
                       
\usepackage{graphicx} 
%\usepackage[margins=2.5cm,nohead,nofoot]{geometry}
%\usepackage{geometry}
\usepackage{amsfonts}
\usepackage{amstext}
\usepackage{latexsym}
\usepackage{amssymb}
\usepackage{color}


%\include{myPreamble}
\include{qm2pi.local} 

%\ifpdf
%\usepackage[pdftex]{graphicx}
%\else
%\usepackage{graphicx}
%\fi

 % \ifpdf
%  \usepackage{pdfsync}
%  \if


%\title{Brief Article}
%\author{David F. Snyder}
%\author{L.G. Meredith}

%\address{Dept. of Math., Texas State University--San Marcos, San Marcos, TX 78666}
       
\pagestyle{empty}


\begin{document}

\lstset{language=[Objective]Caml,frame=shadowbox}

\input{qm2pi.front}

% section front matter (end)

\input{qm2pi.intro} 
 
% section introduction (end)

% \input{qm2pi.knotations} 

% section notation (end)

\input{qm2pi.process.calculi} 

% section concurrent_process_calculi_and_spatial_logics_ (end)
    
%\input{qm2pi.knots2pi} 

%\input{qm2pi.trefoil} 

%\input{qm2pi.mainthm} 

% subsection basic_interpretation (end)

%\input{qm2pi.rho.presentation} 
\subsection{The syntax and semantics of the notation system}\label{sub:the_syntax_and_semantics_of_the_notation_system} % (fold)

We now summarize a technical presentation of the calculus that
embodies our theory of dynamics. The typical presentation of such a
calculus follows the style of giving generators and relations on
them. The grammar, below, describing term constructors, freely
generates the set of processes, $\Proc$. This set is then quotiented
by a relation known as structural congruence and it is over this set
that the notion of dynamics is expressed. This presentation is
essentially that of \cite{MeredithR05} with the addition of
polyadicity and summation. For readability we have relegated some of
the technical subtleties to an appendix.

\subsubsection{Process grammar}\label{subsub:process_grammar}

\begin{mathpar}
  \inferrule* [lab=synchronization] {} {{M} \bc \pzero \;|\; x?F \;|\; x!C }
  \and
  \inferrule* [lab=abstraction] {} {{F} \bc (x)P}
  \and
  \inferrule* [lab=concretion] {} {{C} \bc \langle Q \rangle}
  \and
  \inferrule* [lab=process] {} {{P,Q} \bc M \;| \;P|Q \;|\; @{x}}
  \and
  \inferrule* [lab=name] {} {{x} \bc \quotep{P}}
\end{mathpar} 

Note that $\vec{x}$ (resp. $\vec{P}$) denotes a vector of names
(resp. processes) of length $|\vec{x}|$ (resp. $|\vec{P}|$). We adopt
the following useful abbreviations.

\begin{mathpar}
   x?(\vec{y}).P := x.(\vec{y})P \and  x\clift{\vec{P}} := x.\clift{\vec{P}}
   \and x!(y) := \lift{x}{\dropn{y}}
   \and \Pi_{i=0}^{n-1}P_i := P_0 | \ldots | P_{n-1}
\end{mathpar}

\subsubsection{Structural congruence}

\paragraph{Free and bound names and alpha-equivalence.} At the
core of structural equivalence is alpha-equivalence which identifies
process that are the same up to a change of variable. Formally, we
recognize the distinction between free and bound names. The free names
of a process, $\freenames{P}$, may be calculated recursively as
follows:

\begin{mathpar}
\freenames{\pzero} := \emptyset
  \and \\
  \freenames{x?(y).P} := \{ x \} \cup (\freenames{P} \setminus \{ y \})
  \and 
  \freenames{x!\langle P \rangle} := \{ x \} \cup \{ P \} 
  \and \\
  \freenames{P|Q} := \freenames{P} \cup \freenames{Q}
  \and \\
  \freenames{@{x}} := \{ x \}
\end{mathpar}

$\pi$
$\quotep{\pi}$

$\freenames{-} : \pi \to \mathcal{P}(\quotep{\pi})$

\begin{eqnarray*}
  \freenames{\pzero} & := & \emptyset \\
  \freenames{x?(y).P} & := & \{ x \} \cup (\freenames{P} \setminus \{ y \}) \\
  \freenames{x!\langle P \rangle} & := & \{ x \} \cup \{ P \} \\
  \freenames{P|Q} & := & \freenames{P} \cup \freenames{Q} \\
  \freenames{\dropn{x}} & := & \{ x \}
\end{eqnarray*}

The bound names of a process, $\boundnames{P}$, are those names occurring in $P$
that are not free. For example, in $x?(y).0$, the name $x$ is free, while $y$ is bound.

\begin{mathpar}
  \inferrule* [lab=monoidal-laws] {} { P|Q \equiv Q|P \and P|0 \equiv P \and P|(Q|R) \equiv (P|Q)|R }
\end{mathpar}

\begin{mathpar}
  \inferrule* [lab=alpha-equivalence] {} { (x)P \equiv (y)P\{y/x\} \and y \not\in \freenames{P} }
\end{mathpar}

\begin{definition}
Then two processes, $P,Q$, are alpha-equivalent if $P = Q\{\vec{y}/\vec{x}\}$ for
some $\vec{x} \in \boundnames{Q},\vec{y} \in \boundnames{P}$, where $Q\{\vec{y}/\vec{x}\}$
denotes the capture-avoiding substitution of $\vec{y}$ for $\vec{x}$ in $Q$.
\end{definition}

\begin{definition}
  The {\em structural congruence} \cite{SangiorgiWalker} , $\equiv$,
  between processes is the least congruence containing
  alpha-equivalence, satisfying the abelian monoid laws
  (associativity, commutativity and $\pzero$ as identity) for parallel
  composition $|$ and for summation $+$.
\end{definition}

\subsection{Name equivalence}

We take name equivalence, written $\nameeq$, to be the smallest
equivalence relation generated by the following rules.

\begin{mathpar}
\inferrule*[lab=Quote-drop]
{ }
{ \quotep{@{x}} \nameeq x }

\inferrule*[lab=Struct-equiv]
{ P \scong Q }
{ \quotep{P} \nameeq \quotep{Q} }
\end{mathpar}

The astute reader will have noticed that the mutual recursion of names
and processes imposes a mutual recursion on alpha-equivalence and
structural equivalence via name-equivalence. Fortunately, all of this
works out pleasantly and we may calculate in the natural way, free of
concern. The reader interested in the details is referred to the
appendix \ref{appendix:rho_details}.

\subsection{Substitution}

We use $\Proc$ for the set of processes, $\QProc$ for the set of
names, and $\id{\{}\vec{y} / \vec{x} \id{\}}$ to denote partial maps,
$s : \QProc \rightarrow \QProc$. A map, $s$ lifts, uniquely, to a map
on process terms, $\widehat{s} : \Proc \rightarrow \Proc$ by the
following equations.

\begin{mathpar}
  (0) \psubstp{Q}{P} := 0 \\
  (R \juxtap S) \psubstp{Q}{P}
  :=    
  (R)\psubstp{Q}{P} \juxtap (S) \psubstp{Q}{P} \\
  (x?(y).R) \psubstp{Q}{P}    
  :=    
  (x)\substp{Q}{P} (z)\concat( (R \psubstn{z}{y}) \psubstp{Q}{P} ) \\
  (\lift{x}{R}) \psubstp{Q}{P}  
  :=
  \lift{(x)\substp{Q}{P}}{ R \psubstp{Q}{P} } \\
%   (\dropn{x})  \psubstp{Q}{P}       
%   := 
%   \left\{ 
%     \begin{array}{ccc} 
%       \dropn{\quotep{Q}} & & x \nameeq \quotep{P} \\
%       \dropn{x} & & otherwise \\
%     \end{array}
%   \right. 
  (\dropn{x})  \psubstp{Q}{P}       
  := 
  \left\{ 
    \begin{array}{ccc} 
      Q & & x \nameeq \quotep{P} \\
      \dropn{x} & & otherwise \\
    \end{array}
  \right.
\end{mathpar}
 

where

\begin{eqnarray}
  (x)\id{\{} \lpquote Q \rpquote / \lpquote P \rpquote \id{\}}            = 
  \left\{ 
    \begin{array}{ccc}
      \lpquote Q \rpquote & & x \nameeq \lpquote P \rpquote \\
      x & & otherwise \\
    \end{array}
  \right. \nonumber
\end{eqnarray}

and $z$ is chosen distinct from $\quotep{P}$, $\quotep{Q}$, the free
names in $Q$, and all the names in $R$. Our $\alpha$-equivalence will
be built in the standard way from this substitution.

\begin{remark}\label{rem:no_self_referential_names}
  One consequence of these definitions is that $\forall P. \quotep{P}
  \not\in \freenames{P}$.
\end{remark}

\subsection{ Dynamic quote: an example }

Anticipating something of what's to come, consider applying the
substitution, $\widehat{\id{\{}u / z \id{\}}}$, to the following pair
of processes, $\lift{w}{y!(z)}$ and $w[ \lpquote y!(z) \rpquote ]$.

\begin{eqnarray}
	\lift{w}{y!(z)}\widehat{\id{\{}u / z \id{\}}}
		& = &
		\lift{w}{y!(u)} \nonumber\\
	w[ \lpquote y!(z) \rpquote ] \widehat{ \id{\{}u / z \id{\}} }
		& = &
		w[ \lpquote y!(z) \rpquote ] \nonumber
\end{eqnarray}

Because the body of the process between quotes is impervious to
substitution, we get radically different answers. In fact, by
examining the first process in an input context,
e.g. $x?(z).\lift{w}{y!(z)}$, we see that the process under the lift
operator may be shaped by prefixed inputs binding a name inside it. In
this sense, the lift operator will be seen as a way to dynamically
construct processes before reifying them as names.

Finally equipped with these standard features we can present the
dynamics of the calculus.

\subsubsection{Operational semantics} 

Finally, we introduce the computational dynamics. What marks these
algebras as distinct from other more traditionally studied algebraic
structures, e.g. vector spaces or polynomial rings, is the manner in
which dynamics is captured. In traditional structures, dynamics is typically
expressed through morphisms between such structures, as in linear maps
between vector spaces or morphisms between rings. In algebras
associated with the semantics of computation, the dynamics is
expressed as part of the algebraic structure itself, through a
reduction reduction relation typically denoted by $\red$. Below, we
give a recursive presentation of this relation for the calculus used
in the encoding.

$\red \subseteq \pi \times \pi$
$\red : \pi \to \mathcal{P}(\pi)$

\begin{mathpar}
  \inferrule* [lab=Comm] { \textsf{match}( x_{src}, x_{trgt} ) } { x_{trgt}?(y)P \; | \; x_{src}!\langle {Q} \rangle \red P\{\quotep{Q}/y}\} }
  \and \\
  \inferrule* [lab=Par] {{P} \red {P}'} {{{P} | {Q}} \red {{P}' | {Q}}}
  \and
  \inferrule* [lab=Equiv]{{{P} \scong {P}'} \andalso {{P}' \red {Q}'} \andalso {{Q}' \scong {Q}}}{{P} \red {Q}}
\end{mathpar}

\begin{eqnarray*}
  match_{\equiv} (\quotep{P},\quotep{Q}) & := & P \equiv Q \\
  match_{\dagger}(\quotep{P},\quotep{Q}) & := & \forall R. P|Q \red^{*} R => R \red^{*} 0 \\
  match_{K}(\quotep{P},\quotep{Q}) & := & K \mbox{ for some context } K
\end{eqnarray*}

$u?(x)P | u!\langle Q \rangle \red P\{\quotep{Q}/x\}$

%We write $\wred$ for $\red^*$, and $P\red$ if $\exists Q $ such that $ P \red Q$.
We write $P\red$ if $\exists Q $ such that $ P \red Q$ and $P\not\red$, otherwise.

\section{Replication}

As mentioned before, it is known that replication (and hence
recursion) can be implemented in a higher-order process algebra
\cite{SangiorgiWalker}. As our first example of calculation with the
machinery thus far presented we give the construction explicitly in
the {\rhoc}.

\begin{eqnarray}
	D_{x} & := & \prefix{x}{y}{(\binpar{\outputp{x}{y}}{@{y}})} \nonumber\\
	\bangp_{x}{P} & := & \binpar{{x}!\langle{\binpar{D_{x}}{P}}\rangle}{D_{x}} \nonumber
\end{eqnarray}

\begin{eqnarray}
	\bangp_{x}{P} & & \nonumber\\
	=
	& {x}!\langle{(\prefix{x}{y}{(\outputp{x}{y} | @{y})) | P}}\rangle 
	      | \prefix{x}{y}{(\outputp{x}{y} | @{y})} & \nonumber\\
	\red
	& (\outputp{x}{y} | @{y})\substn{\quotep{(\prefix{x}{y}{(@{y} | \outputp{x}{y})) | P}}}{y} & \nonumber\\
	=
	& \outputp{x}{\quotep{(\prefix{x}{y}{(\outputp{x}{y} | @{y})) | P}}}
	  | {(\prefix{x}{y}{(\outputp{x}{y} | @{y})) | P}} & \nonumber\\
	\red
	& \ldots & \nonumber\\
	\red^*
	& P | P | \ldots & \nonumber
\end{eqnarray}

Of course, this encoding, as an implementation, runs away, unfolding
$\bangp{P}$ eagerly. A lazier and more implementable replication
operator, restricted to input-guarded processes, may be obtained as follows.

\begin{eqnarray}
\bangp{\prefix{u}{v}{P}} 
	:= 
	\binpar{\lift{x}{\prefix{u}{v}{(\binpar{D(x)}{P})}}}{D(x)} \nonumber
\end{eqnarray}

\begin{remark}
  Note that the lazier definition still does not deal with summation
  or mixed summation (i.e. sums over input and output). The reader is
  invited to construct definitions of replication that deal with these
  features. 

  Further, the definitions are parameterized in a name, $x$. Can you,
  gentle reader, make a definition that eliminates this parameter and
  guarantees no accidental interaction between the replication
  machinery and the process being replicated -- i.e. no accidental
  sharing of names used by the process to get its work done and the
  name(s) used by the replication to effect copying. This latter
  revision of the definition of replication is crucial to obtaining
  the expected identity $!!P \sim !P$.
\end{remark}

\begin{remark}\label{rem:paradoxical_combinator}
  The reader familiar with the lambda calculus will have noticed the
  similarity between $D$ and the paradoxical combinator.

  [Ed. note: the existence of this seems to suggest we have to be more
  restrictive on the set of processes and names we admit if we are to
  support no-cloning.]
\end{remark}

\subsubsection{Bisimulation}

The computational dynamics gives rise to another kind of equivalence,
the equivalence of computational behavior. As previously mentioned
this is typically captured \emph{via} some form of bisimulation.

% The notion we use in this paper is weak barbed bisimulation
% \cite{milner91polyadicpi}.

The notion we use in this paper is derived from weak barbed
bisimulation \cite{milner91polyadicpi}. 

\begin{definition}
An \emph{observation relation}, $\downarrow_{\mathcal N}$, over a set
of names, $\mathcal N$, is the smallest relation satisfying the rules
below.

\infrule[Out-barb]{y \in {\mathcal N}, \; x \nameeq y}
		  {\outputp{x}{v} \downarrow_{\mathcal N} x}
\infrule[Par-barb]{\mbox{$P\downarrow_{\mathcal N} x$ or $Q\downarrow_{\mathcal N} x$}}
		  {\binpar{P}{Q} \downarrow_{\mathcal N} x}

We write $P \Downarrow_{\mathcal N} x$ if there is $Q$ such that 
$P \wred Q$ and $Q \downarrow_{\mathcal N} x$.
\end{definition}

\begin{definition}
%\label{def.bbisim}
An  ${\mathcal N}$-\emph{barbed bisimulation} over a set of names, ${\mathcal N}$, is a symmetric binary relation 
${\mathcal S}_{\mathcal N}$ between agents such that $P\rel{S}_{\mathcal N}Q$ implies:
\begin{enumerate}
\item If $P \red P'$ then $Q \wred Q'$ and $P'\rel{S}_{\mathcal N} Q'$.
\item If $P\downarrow_{\mathcal N} x$, then $Q\Downarrow_{\mathcal N} x$.
\end{enumerate}
$P$ is ${\mathcal N}$-barbed bisimilar to $Q$, written
$P \wbbisim_{\mathcal N} Q$, if $P \rel{S}_{\mathcal N} Q$ for some ${\mathcal N}$-barbed bisimulation ${\mathcal S}_{\mathcal N}$.
\end{definition}

$\mathcal{R} \subseteq \pi \times \pi$

$P \mathcal{R} Q => \forall P'. P \red P' \Rightarrow \exists Q'. Q \red Q', P' \mathcal{R} Q'$

$P \vdash x \Rightarrow Q \vdash x$

\begin{mathpar}
  \inferrule*[lab=Out-barb]{x \nameeq y}{{y}!\langle{Q}\rangle \vdash x}
  \and
  \inferrule*[lab=Par-barb]{\mbox{$P\vdash x$ or $Q\vdash x$}}{\binpar{P}{Q} \vdash x}
\end{mathpar}

\subsubsection{Contexts}

One of the principle advantages of computational calculi like the
$\pi$-calculus is a well-defined notion of context,
contextual-equivalence and a correlation between
contextual-equivalence and notions of bisimulation. The notion of
context allows the decomposition of a process into (sub-)process and
its syntactic environment, its context. Thus, a context may be
thought of as a process with a ``hole'' (written $\Box$) in it. The
application of a context $M$ to a process $P$, written $M[P]$, is
tantamount to filling the hole in $M$ with $P$. In this paper we do
not need the full weight of this theory, but do make use of the notion
of context in the proof the main theorem. 

\begin{mathpar}
  \inferrule* [lab=summation] {} {{M_{M},M_{N}} \bc \Box \;|\; x.M_{A} \;|\; M_{M}+M_{N}}
  \and
  \inferrule* [lab=agent] {} {{M_{A}} \bc (\vec{x})M_{P} \;| \; \clift{P_0,\ldots,M_{P},\ldots,P_N}}
  \and \\
  \inferrule* [lab=process] {} {{M_{P}} \bc M_{N} \;| \;P|M_{P} }
\end{mathpar} 

\begin{mathpar}
  \inferrule* [lab=sychronization] {} {M_{N} \bc \Box \;|\; x?M_{F} \;|\; x!M_{C}}
  \and
  \inferrule* [lab=abstraction] {} {{M_{F}} \bc (x)M_{P} }
  \and
  \inferrule* [lab=concretion] {} {{M_{C}} \bc \langle M_{P} \rangle }
  \and \\
  \inferrule* [lab=process] {} {{M_{P}} \bc M_{N} \;| \;P|M_{P} }
\end{mathpar}

\begin{definition}[contextual application] Given a context $M$, and
  process $P$, we define the \emph{contextual application}, $M[P] :=
  M\{P/\Box\}$. That is, the contextual application of M to P is the
  substitution of $P$ for $\Box$ in $M$.
\end{definition}

$\meaningof{-} : L \to \mathcal{P}(\pi)$

\begin{mathpar}
  \inferrule* [lab=collection] {} {\meaningof{true} = \pi, \and \meaningof{~E} = \pi \setminus \meaningof{E}, \and \meaningof{E_{1} \& E_{2}} = \meaningof{E_{1}} \cap \meaningof{E_{2}}}
\end{mathpar}

\begin{mathpar}
  \inferrule* [lab=structure] {} {\meaningof{0} = \{ P \in \pi | P \equiv 0 \}, \and \\ \meaningof{E_1 | E_2} = \{ P \in \pi | P \equiv P_{1} | P_{2}, P_{1} \in \meaningof{E_{1}}, P_{2} \in \meaningof{E_2}\} }
\end{mathpar}

\begin{mathpar}
 \inferrule* [lab=behavior] {} {\meaningof{\langle a?b \rangle E} = \{ P \in \pi | P \equiv Q | u?(y)P', \\ \and \\\\ \and \\ \;\;\; u \in \meaningof{a}, \forall z.P'\{z/y\} \in \meaningof{E\{z/b\}}\}, \and \\ \meaningof{a!E} = \{ P \in \pi | P \equiv Q | x!\langle P' \rangle, x \in \meaningof{a} P' \in \meaningof{E}\} }
\end{mathpar}

\begin{mathpar}
 \inferrule* [lab=nominal] {} {\meaningof{\quotep{E}} = \{ \quotep{P} \in \quotep{\pi} | P \in \meaningof{E} \}, \and \meaningof{\quotep{P}} = \{ \quotep{Q} \in \quotep{\pi} | P \equiv Q \} \and \\ \meaningof{@\quotep{E}} = \{ P \in \pi | P \equiv @x, x \in \meaningof{E} \}}
\end{mathpar}

\begin{eqnarray*}
  \\
  \meaningof{-} : TS \to ST
\end{eqnarray*}

\begin{eqnarray*}
  \\
  L : TS \to ST
\end{eqnarray*}

\begin{eqnarray*}
  \\
  P \models E \iff P \in \meaningof{E}
\end{eqnarray*}

\begin{eqnarray*}
  P \approx_{L} Q \iff \forall E \in L. P \models E \iff Q \models E
\end{eqnarray*}

\begin{eqnarray*}
  P \approx_{K} Q
\end{eqnarray*}

\begin{eqnarray*}
  P \approx Q
\end{eqnarray*}

$\approx_{K} = \approx = \approx_{L}$

\subsubsection{Contextual duality}

Note that contexts extend the quotation operation to a family of
operations from processes to names. Given a context, $M$, we can
define a \emph{nominal context}, $\quotep{M}$ by $\quotep{M}[P] :=
\quotep{M[P]}$. To foreshadow what is to come we observe that these
operations enjoy a duality with processes very much like the duality
between vectors and maps from vectors to scalars.

Further, because the calculus is essentially higher-order, we have a
correspondence between contexts and processes. More specifically,
given a name $x$ and a context $M$ we can construct $M^{*}_{x}$ such
that 

\begin{mathpar}
  M^{*}_{x} | \lift{x}{P} \red M[P]
\end{mathpar}

namely,

\begin{mathpar}
  M^{*}_{x} := x?(u).M[\dropn{u}]
\end{mathpar}

The dependence of $M^{*}_{x}$ on a name makes it an abstraction, 

\begin{mathpar}
  M^{*} := (x)x?(u).M[\dropn{u}]
\end{mathpar}

\subsection{Additional notation}

It will sometimes be convenient to denote the process a name
quotes. We already have the notation $x = \quotep{P}$, but it will be
convenient to introduce an alternate notation, $\procn{x}$, when we
want to emphasize the connection to the use of the name. Note that, by
virtue of name equivalence, $\quotep{\procn{x}} \nameeq x$; so, the
notation is consistent with previous definitions.

Further, because names have structure it is possible to effect
substitutions on the basis of that structure. This means we need to
upgrade our notation for substitutions, which we accomplish by
adapting comprehension notation. Thus,

\begin{mathpar}
  P\{ y / x : x \in S \}
\end{mathpar}

is interpreted to mean the process derived from P by replacing (in a
capture-avoiding manner) each occurrence of $x$ in $S$ by $y$. For example,

\begin{mathpar}
  P\{ \quotep{\procn{x}|\procn{x}} / x : x \in \freenames{P} \}
\end{mathpar}

will replace each (occurrence) of a free name $x$ in $P$ by
$\quotep{\procn{x}|\procn{x}}$.

Also, we will avail ourselves of the notation $x^{L}$ and $x^{R}$ to
denote injections of a name into disjoint copies of the name
space. There are numerous ways to accomplish this. One example can be
found in \cite{MeredithR05}. This notation overloads to vectors of
names: $\vec{x}^{\pi} := (x_{i}^{\pi} \; : \; 0 \leq i < |\vec{x}| )$ where $\pi \in \{L,R\}$.

We also use $P^{\Box} := P|\Box$.

In \cite{MeredithR05} an interpretation of the new operator is
given. It turns out that there are several possible interpretations
all enjoying the requisite algebraic properties of the operator (see
\cite{milner91polyadicpi}). We will therefore make liberal use of
$(\nu\; \vec{x})P$.

% subsection the_syntax_and_semantics_of_the_notation_system (end)   

\input{qm2pi.qmops} 

\input{qm2pi.sterngerlach} 

\input{qm2pi.metric} 

% section concurrent_process_calculi (end)

%\input{qm2pi.proofsketch}

% section proof sketch (end)

%\input{qm2pi.slviaknots} 

% section spatial logic via knots (end)

\input{qm2pi.conclusion}

% section conclusion (end)

%\input{qm2pi.dtcodes} 

% section wiring algorithm (end)

\input{qm2pi.ack} 

% section acknowledgments (end)

\newpage


\bibliographystyle{plain}   
\bibliography{../../biblios/main.bib}

\input{qm2pi.rhodetails}

\end{document}

 

% section concurrent_process_calculi (end)

%\documentclass[12pt]{llncs}
%\documentclass{jktr}

\usepackage[pdftex]{hyperref}                   
\usepackage {listings}
\usepackage {mathpartir}
\usepackage{bcprules}
%\usepackage{listings}
                       
\usepackage{graphicx} 
%\usepackage[margins=2.5cm,nohead,nofoot]{geometry}
%\usepackage{geometry}
\usepackage{amsfonts}
\usepackage{amstext}
\usepackage{latexsym}
\usepackage{amssymb}
\usepackage{color}


%\include{myPreamble}
\include{qm2pi.local} 

%\ifpdf
%\usepackage[pdftex]{graphicx}
%\else
%\usepackage{graphicx}
%\fi

 % \ifpdf
%  \usepackage{pdfsync}
%  \if


%\title{Brief Article}
%\author{David F. Snyder}
%\author{L.G. Meredith}

%\address{Dept. of Math., Texas State University--San Marcos, San Marcos, TX 78666}
       
\pagestyle{empty}


\begin{document}

\lstset{language=[Objective]Caml,frame=shadowbox}

\input{qm2pi.front}

% section front matter (end)

\input{qm2pi.intro} 
 
% section introduction (end)

% \input{qm2pi.knotations} 

% section notation (end)

\input{qm2pi.process.calculi} 

% section concurrent_process_calculi_and_spatial_logics_ (end)
    
%\input{qm2pi.knots2pi} 

%\input{qm2pi.trefoil} 

%\input{qm2pi.mainthm} 

% subsection basic_interpretation (end)

%\input{qm2pi.rho.presentation} 
\subsection{The syntax and semantics of the notation system}\label{sub:the_syntax_and_semantics_of_the_notation_system} % (fold)

We now summarize a technical presentation of the calculus that
embodies our theory of dynamics. The typical presentation of such a
calculus follows the style of giving generators and relations on
them. The grammar, below, describing term constructors, freely
generates the set of processes, $\Proc$. This set is then quotiented
by a relation known as structural congruence and it is over this set
that the notion of dynamics is expressed. This presentation is
essentially that of \cite{MeredithR05} with the addition of
polyadicity and summation. For readability we have relegated some of
the technical subtleties to an appendix.

\subsubsection{Process grammar}\label{subsub:process_grammar}

\begin{mathpar}
  \inferrule* [lab=synchronization] {} {{M} \bc \pzero \;|\; x?F \;|\; x!C }
  \and
  \inferrule* [lab=abstraction] {} {{F} \bc (x)P}
  \and
  \inferrule* [lab=concretion] {} {{C} \bc \langle Q \rangle}
  \and
  \inferrule* [lab=process] {} {{P,Q} \bc M \;| \;P|Q \;|\; @{x}}
  \and
  \inferrule* [lab=name] {} {{x} \bc \quotep{P}}
\end{mathpar} 

Note that $\vec{x}$ (resp. $\vec{P}$) denotes a vector of names
(resp. processes) of length $|\vec{x}|$ (resp. $|\vec{P}|$). We adopt
the following useful abbreviations.

\begin{mathpar}
   x?(\vec{y}).P := x.(\vec{y})P \and  x\clift{\vec{P}} := x.\clift{\vec{P}}
   \and x!(y) := \lift{x}{\dropn{y}}
   \and \Pi_{i=0}^{n-1}P_i := P_0 | \ldots | P_{n-1}
\end{mathpar}

\subsubsection{Structural congruence}

\paragraph{Free and bound names and alpha-equivalence.} At the
core of structural equivalence is alpha-equivalence which identifies
process that are the same up to a change of variable. Formally, we
recognize the distinction between free and bound names. The free names
of a process, $\freenames{P}$, may be calculated recursively as
follows:

\begin{mathpar}
\freenames{\pzero} := \emptyset
  \and \\
  \freenames{x?(y).P} := \{ x \} \cup (\freenames{P} \setminus \{ y \})
  \and 
  \freenames{x!\langle P \rangle} := \{ x \} \cup \{ P \} 
  \and \\
  \freenames{P|Q} := \freenames{P} \cup \freenames{Q}
  \and \\
  \freenames{@{x}} := \{ x \}
\end{mathpar}

$\pi$
$\quotep{\pi}$

$\freenames{-} : \pi \to \mathcal{P}(\quotep{\pi})$

\begin{eqnarray*}
  \freenames{\pzero} & := & \emptyset \\
  \freenames{x?(y).P} & := & \{ x \} \cup (\freenames{P} \setminus \{ y \}) \\
  \freenames{x!\langle P \rangle} & := & \{ x \} \cup \{ P \} \\
  \freenames{P|Q} & := & \freenames{P} \cup \freenames{Q} \\
  \freenames{\dropn{x}} & := & \{ x \}
\end{eqnarray*}

The bound names of a process, $\boundnames{P}$, are those names occurring in $P$
that are not free. For example, in $x?(y).0$, the name $x$ is free, while $y$ is bound.

\begin{mathpar}
  \inferrule* [lab=monoidal-laws] {} { P|Q \equiv Q|P \and P|0 \equiv P \and P|(Q|R) \equiv (P|Q)|R }
\end{mathpar}

\begin{mathpar}
  \inferrule* [lab=alpha-equivalence] {} { (x)P \equiv (y)P\{y/x\} \and y \not\in \freenames{P} }
\end{mathpar}

\begin{definition}
Then two processes, $P,Q$, are alpha-equivalent if $P = Q\{\vec{y}/\vec{x}\}$ for
some $\vec{x} \in \boundnames{Q},\vec{y} \in \boundnames{P}$, where $Q\{\vec{y}/\vec{x}\}$
denotes the capture-avoiding substitution of $\vec{y}$ for $\vec{x}$ in $Q$.
\end{definition}

\begin{definition}
  The {\em structural congruence} \cite{SangiorgiWalker} , $\equiv$,
  between processes is the least congruence containing
  alpha-equivalence, satisfying the abelian monoid laws
  (associativity, commutativity and $\pzero$ as identity) for parallel
  composition $|$ and for summation $+$.
\end{definition}

\subsection{Name equivalence}

We take name equivalence, written $\nameeq$, to be the smallest
equivalence relation generated by the following rules.

\begin{mathpar}
\inferrule*[lab=Quote-drop]
{ }
{ \quotep{@{x}} \nameeq x }

\inferrule*[lab=Struct-equiv]
{ P \scong Q }
{ \quotep{P} \nameeq \quotep{Q} }
\end{mathpar}

The astute reader will have noticed that the mutual recursion of names
and processes imposes a mutual recursion on alpha-equivalence and
structural equivalence via name-equivalence. Fortunately, all of this
works out pleasantly and we may calculate in the natural way, free of
concern. The reader interested in the details is referred to the
appendix \ref{appendix:rho_details}.

\subsection{Substitution}

We use $\Proc$ for the set of processes, $\QProc$ for the set of
names, and $\id{\{}\vec{y} / \vec{x} \id{\}}$ to denote partial maps,
$s : \QProc \rightarrow \QProc$. A map, $s$ lifts, uniquely, to a map
on process terms, $\widehat{s} : \Proc \rightarrow \Proc$ by the
following equations.

\begin{mathpar}
  (0) \psubstp{Q}{P} := 0 \\
  (R \juxtap S) \psubstp{Q}{P}
  :=    
  (R)\psubstp{Q}{P} \juxtap (S) \psubstp{Q}{P} \\
  (x?(y).R) \psubstp{Q}{P}    
  :=    
  (x)\substp{Q}{P} (z)\concat( (R \psubstn{z}{y}) \psubstp{Q}{P} ) \\
  (\lift{x}{R}) \psubstp{Q}{P}  
  :=
  \lift{(x)\substp{Q}{P}}{ R \psubstp{Q}{P} } \\
%   (\dropn{x})  \psubstp{Q}{P}       
%   := 
%   \left\{ 
%     \begin{array}{ccc} 
%       \dropn{\quotep{Q}} & & x \nameeq \quotep{P} \\
%       \dropn{x} & & otherwise \\
%     \end{array}
%   \right. 
  (\dropn{x})  \psubstp{Q}{P}       
  := 
  \left\{ 
    \begin{array}{ccc} 
      Q & & x \nameeq \quotep{P} \\
      \dropn{x} & & otherwise \\
    \end{array}
  \right.
\end{mathpar}
 

where

\begin{eqnarray}
  (x)\id{\{} \lpquote Q \rpquote / \lpquote P \rpquote \id{\}}            = 
  \left\{ 
    \begin{array}{ccc}
      \lpquote Q \rpquote & & x \nameeq \lpquote P \rpquote \\
      x & & otherwise \\
    \end{array}
  \right. \nonumber
\end{eqnarray}

and $z$ is chosen distinct from $\quotep{P}$, $\quotep{Q}$, the free
names in $Q$, and all the names in $R$. Our $\alpha$-equivalence will
be built in the standard way from this substitution.

\begin{remark}\label{rem:no_self_referential_names}
  One consequence of these definitions is that $\forall P. \quotep{P}
  \not\in \freenames{P}$.
\end{remark}

\subsection{ Dynamic quote: an example }

Anticipating something of what's to come, consider applying the
substitution, $\widehat{\id{\{}u / z \id{\}}}$, to the following pair
of processes, $\lift{w}{y!(z)}$ and $w[ \lpquote y!(z) \rpquote ]$.

\begin{eqnarray}
	\lift{w}{y!(z)}\widehat{\id{\{}u / z \id{\}}}
		& = &
		\lift{w}{y!(u)} \nonumber\\
	w[ \lpquote y!(z) \rpquote ] \widehat{ \id{\{}u / z \id{\}} }
		& = &
		w[ \lpquote y!(z) \rpquote ] \nonumber
\end{eqnarray}

Because the body of the process between quotes is impervious to
substitution, we get radically different answers. In fact, by
examining the first process in an input context,
e.g. $x?(z).\lift{w}{y!(z)}$, we see that the process under the lift
operator may be shaped by prefixed inputs binding a name inside it. In
this sense, the lift operator will be seen as a way to dynamically
construct processes before reifying them as names.

Finally equipped with these standard features we can present the
dynamics of the calculus.

\subsubsection{Operational semantics} 

Finally, we introduce the computational dynamics. What marks these
algebras as distinct from other more traditionally studied algebraic
structures, e.g. vector spaces or polynomial rings, is the manner in
which dynamics is captured. In traditional structures, dynamics is typically
expressed through morphisms between such structures, as in linear maps
between vector spaces or morphisms between rings. In algebras
associated with the semantics of computation, the dynamics is
expressed as part of the algebraic structure itself, through a
reduction reduction relation typically denoted by $\red$. Below, we
give a recursive presentation of this relation for the calculus used
in the encoding.

$\red \subseteq \pi \times \pi$
$\red : \pi \to \mathcal{P}(\pi)$

\begin{mathpar}
  \inferrule* [lab=Comm] { \textsf{match}( x_{src}, x_{trgt} ) } { x_{trgt}?(y)P \; | \; x_{src}!\langle {Q} \rangle \red P\{\quotep{Q}/y}\} }
  \and \\
  \inferrule* [lab=Par] {{P} \red {P}'} {{{P} | {Q}} \red {{P}' | {Q}}}
  \and
  \inferrule* [lab=Equiv]{{{P} \scong {P}'} \andalso {{P}' \red {Q}'} \andalso {{Q}' \scong {Q}}}{{P} \red {Q}}
\end{mathpar}

\begin{eqnarray*}
  match_{\equiv} (\quotep{P},\quotep{Q}) & := & P \equiv Q \\
  match_{\dagger}(\quotep{P},\quotep{Q}) & := & \forall R. P|Q \red^{*} R => R \red^{*} 0 \\
  match_{K}(\quotep{P},\quotep{Q}) & := & K \mbox{ for some context } K
\end{eqnarray*}

$u?(x)P | u!\langle Q \rangle \red P\{\quotep{Q}/x\}$

%We write $\wred$ for $\red^*$, and $P\red$ if $\exists Q $ such that $ P \red Q$.
We write $P\red$ if $\exists Q $ such that $ P \red Q$ and $P\not\red$, otherwise.

\section{Replication}

As mentioned before, it is known that replication (and hence
recursion) can be implemented in a higher-order process algebra
\cite{SangiorgiWalker}. As our first example of calculation with the
machinery thus far presented we give the construction explicitly in
the {\rhoc}.

\begin{eqnarray}
	D_{x} & := & \prefix{x}{y}{(\binpar{\outputp{x}{y}}{@{y}})} \nonumber\\
	\bangp_{x}{P} & := & \binpar{{x}!\langle{\binpar{D_{x}}{P}}\rangle}{D_{x}} \nonumber
\end{eqnarray}

\begin{eqnarray}
	\bangp_{x}{P} & & \nonumber\\
	=
	& {x}!\langle{(\prefix{x}{y}{(\outputp{x}{y} | @{y})) | P}}\rangle 
	      | \prefix{x}{y}{(\outputp{x}{y} | @{y})} & \nonumber\\
	\red
	& (\outputp{x}{y} | @{y})\substn{\quotep{(\prefix{x}{y}{(@{y} | \outputp{x}{y})) | P}}}{y} & \nonumber\\
	=
	& \outputp{x}{\quotep{(\prefix{x}{y}{(\outputp{x}{y} | @{y})) | P}}}
	  | {(\prefix{x}{y}{(\outputp{x}{y} | @{y})) | P}} & \nonumber\\
	\red
	& \ldots & \nonumber\\
	\red^*
	& P | P | \ldots & \nonumber
\end{eqnarray}

Of course, this encoding, as an implementation, runs away, unfolding
$\bangp{P}$ eagerly. A lazier and more implementable replication
operator, restricted to input-guarded processes, may be obtained as follows.

\begin{eqnarray}
\bangp{\prefix{u}{v}{P}} 
	:= 
	\binpar{\lift{x}{\prefix{u}{v}{(\binpar{D(x)}{P})}}}{D(x)} \nonumber
\end{eqnarray}

\begin{remark}
  Note that the lazier definition still does not deal with summation
  or mixed summation (i.e. sums over input and output). The reader is
  invited to construct definitions of replication that deal with these
  features. 

  Further, the definitions are parameterized in a name, $x$. Can you,
  gentle reader, make a definition that eliminates this parameter and
  guarantees no accidental interaction between the replication
  machinery and the process being replicated -- i.e. no accidental
  sharing of names used by the process to get its work done and the
  name(s) used by the replication to effect copying. This latter
  revision of the definition of replication is crucial to obtaining
  the expected identity $!!P \sim !P$.
\end{remark}

\begin{remark}\label{rem:paradoxical_combinator}
  The reader familiar with the lambda calculus will have noticed the
  similarity between $D$ and the paradoxical combinator.

  [Ed. note: the existence of this seems to suggest we have to be more
  restrictive on the set of processes and names we admit if we are to
  support no-cloning.]
\end{remark}

\subsubsection{Bisimulation}

The computational dynamics gives rise to another kind of equivalence,
the equivalence of computational behavior. As previously mentioned
this is typically captured \emph{via} some form of bisimulation.

% The notion we use in this paper is weak barbed bisimulation
% \cite{milner91polyadicpi}.

The notion we use in this paper is derived from weak barbed
bisimulation \cite{milner91polyadicpi}. 

\begin{definition}
An \emph{observation relation}, $\downarrow_{\mathcal N}$, over a set
of names, $\mathcal N$, is the smallest relation satisfying the rules
below.

\infrule[Out-barb]{y \in {\mathcal N}, \; x \nameeq y}
		  {\outputp{x}{v} \downarrow_{\mathcal N} x}
\infrule[Par-barb]{\mbox{$P\downarrow_{\mathcal N} x$ or $Q\downarrow_{\mathcal N} x$}}
		  {\binpar{P}{Q} \downarrow_{\mathcal N} x}

We write $P \Downarrow_{\mathcal N} x$ if there is $Q$ such that 
$P \wred Q$ and $Q \downarrow_{\mathcal N} x$.
\end{definition}

\begin{definition}
%\label{def.bbisim}
An  ${\mathcal N}$-\emph{barbed bisimulation} over a set of names, ${\mathcal N}$, is a symmetric binary relation 
${\mathcal S}_{\mathcal N}$ between agents such that $P\rel{S}_{\mathcal N}Q$ implies:
\begin{enumerate}
\item If $P \red P'$ then $Q \wred Q'$ and $P'\rel{S}_{\mathcal N} Q'$.
\item If $P\downarrow_{\mathcal N} x$, then $Q\Downarrow_{\mathcal N} x$.
\end{enumerate}
$P$ is ${\mathcal N}$-barbed bisimilar to $Q$, written
$P \wbbisim_{\mathcal N} Q$, if $P \rel{S}_{\mathcal N} Q$ for some ${\mathcal N}$-barbed bisimulation ${\mathcal S}_{\mathcal N}$.
\end{definition}

$\mathcal{R} \subseteq \pi \times \pi$

$P \mathcal{R} Q => \forall P'. P \red P' \Rightarrow \exists Q'. Q \red Q', P' \mathcal{R} Q'$

$P \vdash x \Rightarrow Q \vdash x$

\begin{mathpar}
  \inferrule*[lab=Out-barb]{x \nameeq y}{{y}!\langle{Q}\rangle \vdash x}
  \and
  \inferrule*[lab=Par-barb]{\mbox{$P\vdash x$ or $Q\vdash x$}}{\binpar{P}{Q} \vdash x}
\end{mathpar}

\subsubsection{Contexts}

One of the principle advantages of computational calculi like the
$\pi$-calculus is a well-defined notion of context,
contextual-equivalence and a correlation between
contextual-equivalence and notions of bisimulation. The notion of
context allows the decomposition of a process into (sub-)process and
its syntactic environment, its context. Thus, a context may be
thought of as a process with a ``hole'' (written $\Box$) in it. The
application of a context $M$ to a process $P$, written $M[P]$, is
tantamount to filling the hole in $M$ with $P$. In this paper we do
not need the full weight of this theory, but do make use of the notion
of context in the proof the main theorem. 

\begin{mathpar}
  \inferrule* [lab=summation] {} {{M_{M},M_{N}} \bc \Box \;|\; x.M_{A} \;|\; M_{M}+M_{N}}
  \and
  \inferrule* [lab=agent] {} {{M_{A}} \bc (\vec{x})M_{P} \;| \; \clift{P_0,\ldots,M_{P},\ldots,P_N}}
  \and \\
  \inferrule* [lab=process] {} {{M_{P}} \bc M_{N} \;| \;P|M_{P} }
\end{mathpar} 

\begin{mathpar}
  \inferrule* [lab=sychronization] {} {M_{N} \bc \Box \;|\; x?M_{F} \;|\; x!M_{C}}
  \and
  \inferrule* [lab=abstraction] {} {{M_{F}} \bc (x)M_{P} }
  \and
  \inferrule* [lab=concretion] {} {{M_{C}} \bc \langle M_{P} \rangle }
  \and \\
  \inferrule* [lab=process] {} {{M_{P}} \bc M_{N} \;| \;P|M_{P} }
\end{mathpar}

\begin{definition}[contextual application] Given a context $M$, and
  process $P$, we define the \emph{contextual application}, $M[P] :=
  M\{P/\Box\}$. That is, the contextual application of M to P is the
  substitution of $P$ for $\Box$ in $M$.
\end{definition}

$\meaningof{-} : L \to \mathcal{P}(\pi)$

\begin{mathpar}
  \inferrule* [lab=collection] {} {\meaningof{true} = \pi, \and \meaningof{~E} = \pi \setminus \meaningof{E}, \and \meaningof{E_{1} \& E_{2}} = \meaningof{E_{1}} \cap \meaningof{E_{2}}}
\end{mathpar}

\begin{mathpar}
  \inferrule* [lab=structure] {} {\meaningof{0} = \{ P \in \pi | P \equiv 0 \}, \and \\ \meaningof{E_1 | E_2} = \{ P \in \pi | P \equiv P_{1} | P_{2}, P_{1} \in \meaningof{E_{1}}, P_{2} \in \meaningof{E_2}\} }
\end{mathpar}

\begin{mathpar}
 \inferrule* [lab=behavior] {} {\meaningof{\langle a?b \rangle E} = \{ P \in \pi | P \equiv Q | u?(y)P', \\ \and \\\\ \and \\ \;\;\; u \in \meaningof{a}, \forall z.P'\{z/y\} \in \meaningof{E\{z/b\}}\}, \and \\ \meaningof{a!E} = \{ P \in \pi | P \equiv Q | x!\langle P' \rangle, x \in \meaningof{a} P' \in \meaningof{E}\} }
\end{mathpar}

\begin{mathpar}
 \inferrule* [lab=nominal] {} {\meaningof{\quotep{E}} = \{ \quotep{P} \in \quotep{\pi} | P \in \meaningof{E} \}, \and \meaningof{\quotep{P}} = \{ \quotep{Q} \in \quotep{\pi} | P \equiv Q \} \and \\ \meaningof{@\quotep{E}} = \{ P \in \pi | P \equiv @x, x \in \meaningof{E} \}}
\end{mathpar}

\begin{eqnarray*}
  \\
  \meaningof{-} : TS \to ST
\end{eqnarray*}

\begin{eqnarray*}
  \\
  L : TS \to ST
\end{eqnarray*}

\begin{eqnarray*}
  \\
  P \models E \iff P \in \meaningof{E}
\end{eqnarray*}

\begin{eqnarray*}
  P \approx_{L} Q \iff \forall E \in L. P \models E \iff Q \models E
\end{eqnarray*}

\begin{eqnarray*}
  P \approx_{K} Q
\end{eqnarray*}

\begin{eqnarray*}
  P \approx Q
\end{eqnarray*}

$\approx_{K} = \approx = \approx_{L}$

\subsubsection{Contextual duality}

Note that contexts extend the quotation operation to a family of
operations from processes to names. Given a context, $M$, we can
define a \emph{nominal context}, $\quotep{M}$ by $\quotep{M}[P] :=
\quotep{M[P]}$. To foreshadow what is to come we observe that these
operations enjoy a duality with processes very much like the duality
between vectors and maps from vectors to scalars.

Further, because the calculus is essentially higher-order, we have a
correspondence between contexts and processes. More specifically,
given a name $x$ and a context $M$ we can construct $M^{*}_{x}$ such
that 

\begin{mathpar}
  M^{*}_{x} | \lift{x}{P} \red M[P]
\end{mathpar}

namely,

\begin{mathpar}
  M^{*}_{x} := x?(u).M[\dropn{u}]
\end{mathpar}

The dependence of $M^{*}_{x}$ on a name makes it an abstraction, 

\begin{mathpar}
  M^{*} := (x)x?(u).M[\dropn{u}]
\end{mathpar}

\subsection{Additional notation}

It will sometimes be convenient to denote the process a name
quotes. We already have the notation $x = \quotep{P}$, but it will be
convenient to introduce an alternate notation, $\procn{x}$, when we
want to emphasize the connection to the use of the name. Note that, by
virtue of name equivalence, $\quotep{\procn{x}} \nameeq x$; so, the
notation is consistent with previous definitions.

Further, because names have structure it is possible to effect
substitutions on the basis of that structure. This means we need to
upgrade our notation for substitutions, which we accomplish by
adapting comprehension notation. Thus,

\begin{mathpar}
  P\{ y / x : x \in S \}
\end{mathpar}

is interpreted to mean the process derived from P by replacing (in a
capture-avoiding manner) each occurrence of $x$ in $S$ by $y$. For example,

\begin{mathpar}
  P\{ \quotep{\procn{x}|\procn{x}} / x : x \in \freenames{P} \}
\end{mathpar}

will replace each (occurrence) of a free name $x$ in $P$ by
$\quotep{\procn{x}|\procn{x}}$.

Also, we will avail ourselves of the notation $x^{L}$ and $x^{R}$ to
denote injections of a name into disjoint copies of the name
space. There are numerous ways to accomplish this. One example can be
found in \cite{MeredithR05}. This notation overloads to vectors of
names: $\vec{x}^{\pi} := (x_{i}^{\pi} \; : \; 0 \leq i < |\vec{x}| )$ where $\pi \in \{L,R\}$.

We also use $P^{\Box} := P|\Box$.

In \cite{MeredithR05} an interpretation of the new operator is
given. It turns out that there are several possible interpretations
all enjoying the requisite algebraic properties of the operator (see
\cite{milner91polyadicpi}). We will therefore make liberal use of
$(\nu\; \vec{x})P$.

% subsection the_syntax_and_semantics_of_the_notation_system (end)   

\input{qm2pi.qmops} 

\input{qm2pi.sterngerlach} 

\input{qm2pi.metric} 

% section concurrent_process_calculi (end)

%\input{qm2pi.proofsketch}

% section proof sketch (end)

%\input{qm2pi.slviaknots} 

% section spatial logic via knots (end)

\input{qm2pi.conclusion}

% section conclusion (end)

%\input{qm2pi.dtcodes} 

% section wiring algorithm (end)

\input{qm2pi.ack} 

% section acknowledgments (end)

\newpage


\bibliographystyle{plain}   
\bibliography{../../biblios/main.bib}

\input{qm2pi.rhodetails}

\end{document}



% section proof sketch (end)

%\section{Unlikely characters: spatial logic for
  knots}\label{sub:characteristic_formulae} % (fold)

Associated to the mobile process calculi are a family of logics known
as the Hennessy-Milner logics. These logics typically enjoy a
semantics interpreting formulae as sets of processes that when
factored through the encoding outlined above allows an identification
of classes of knots with logical formulae. In the context of this
encoding the sub-family known as the spatial logics \cite{CairesC03}
\cite{CairesC04} \cite{Caires04} are of particular interest providing
several important features for expressing and reasoning about
properties (i.e. classes) of knots. We hint here at how this may be done.

%\begin{description}
%\item [structural connectives] 
\subsubsection{Structural connectives} The spatial logics enjoy
structural connectives corresponding, at the logical level, to the
parallel composition ($P | Q$) and new name ($(\nu \; x)P$)
connectives for processes. As illustrated in the examples below, these
connectives are extremely expressive given the shape of our encoding.
%\item [decideable satisfaction]

\subsubsection{Decideable satisfaction}
In \cite{Caires04} the satisfaction relation is shown to be decideable
for a rich class of processes. It further turns out that the image of
the our encoding is a proper subset of that class. This result
provides the basis for an algorithm by which to search for knots
enjoying a given property.
%\item [characteristic formulae]

\subsubsection{Characteristic formulae}
In the same paper \cite{Caires04} , Caires presents a means of calculating
characteristic formulae, selecting equivalence classes of processes
up to a pre--specified depth limit on the support set of names. Composed with our
encoding, this characteristic formula can be used to select
characteristic formulae for knots.
%\end{description}

\subsubsection{Spatial logic formulae}

The grammar below (segmented for comprehension) summarizes the syntax
of spatial logic formulae. We employ illustrative examples in the
sequel to provide an intuitive understanding of their meaning
referring the reader to \cite{Caires04} for a more detailed explication
of the semantics.

\begin{mathpar}
  \inferrule* [lab=boolean] {} {{A,B} \bc T \;|\; \neg A \;|\; A \wedge B \;|\; \eta = \eta'}
  \and
  \inferrule* [lab=spatial] {} {|\; \pzero \;|\; A | B \;|\; x \text{\textregistered} A \;|\; \forall x . A \;|\;  H x . A}
  \and
  \inferrule* [lab=behavioral] {} {|\; \alpha . A}
  \and 
  \inferrule* [lab=recursion] {} {|\; X(\vec{u}) \;|\; \mu X(\vec{u}) . A}
  \and
  \inferrule* [lab=action] {} {\alpha \bc \langle x?(\vec{y}) \rangle \;|\; \langle x!(\vec{y}) \rangle \;|\; \langle \tau \rangle}
  \and 
  \inferrule* [lab=name] {} {\eta \bc x \;|\; \tau}
\end{mathpar} 

% subsection characteristic_formulae (end)   	 

\subsection{Example formulae}\label{sub:example_formulae_} % (fold)

\subsubsection{Crossing as formula.}
% 
% \begin{align*}
%   \frac{d}{dx} \sin x &= \cos x 
%   & \frac{d}{dx} e^x &= e^x \\
%   \frac{d}{dx} \cos x &= - \sin x 
%   & \frac{d}{dx} \log x &= \frac{1}{x} \\
% \end{align*} 

\begin{align*}
 \mu C(x_{0},x_{1},y_{0},y_{1},u).&(\langle x_{0}?(z) \rangle(\langle u! \rangle\langle y_{1}!z \rangle C(x_{0},x_{1},y_{0},y_{1},u)) & \\
  & \wedge \langle y_{1}?(z) \rangle (\langle u! \rangle \langle x_{0}!z \rangle C(x_{0},x_{1},y_{0},y_{1},u)) & \\
  & \wedge \langle x_{1}?(z) \rangle (\langle u? \rangle \langle y_{0}!z \rangle C(x_{0},x_{1},y_{0},y_{1},u)) & \\
  & \wedge \langle y_{0}?(z) \rangle (\langle u? \rangle \langle x_{1}!z \rangle C(x_{0},x_{1},y_{0},y_{1},u))) &
\end{align*}

The lexicographical similarity between the shape of this formulae and
the shape of definition of the process representing a crossing reveals
the intuitive meaning of this formulae. It describes the capabilities
of a process that has the right to represent a crossing. For example
it picks out processes that may perform an input on the port $x_0$ in
its initial menu of capabilities. What differentiates the formula
from the process, however, is that the crossing process is the
smallest candidate to satisfy the formula. Infinitely many other
processes -- with internal behavior hidden behind this interface, so
to speak -- also satisfy this formula. Even this simple formula,
then, can be seen to open a new view onto knots, providing a
computational interpretation of \emph{virtual} knots.

Note that this formula is derived by hand. A similar formula can be
derived by employing Caires' calculation of characteristic formula
\cite{Caires04} to the process representing a crossing. In light of
this discussion, we let
$\meaningof{C}_{\phi}(x0,x1,y0,y1,u)$ denote a formula specifying the
dynamics we wish to capture of a crossing. To guarantee we preserve
the shape of the interface and minimal semantics we demand that
$\meaningof{C}_{\phi}(x0,x1,y0,y1,u) \Rightarrow
\textbf{C}(x0,x1,y0,y1,u)$ where $\textbf{C}(x0,x1,y0,y1,u)$ denotes
the formula above.
                            
\subsubsection{Crossing number constraints.}
The moral content of the context lemma (Lemma \ref{context}) is that the notion of
``locality'' in the Reidemeister moves is effectively captured by the
parallel composition operator of the process calculus. This intuition
extends through the logic. Given a formula,
$\meaningof{C}_{\phi}(x0,x1,y0,y1,u)$, we can use the structural
connectives to specify constraints on crossing numbers, such as at
least $n$ crossings, or exactly $n$ crossings.
\begin{mathpar}
  \inferrule* [lab=at-least-n] {} { K^{\geq n}_{\phi}(\vec{xs},\vec{ys}) := \Pi_{i=0}^{n-1} Hu . \meaningof{C}_{\phi}(xs_i,ys_i,u) | T }
  \and 
  \inferrule* [lab=exactly-n] {} { K^{= n}_{\phi}(\vec{xs},\vec{ys}) := \Pi_{i=0}^{n-1} Hu . \meaningof{C}_{\phi}(xs_i,ys_i,u) | \neg (\forall x_0,y_0,x_1,y_1,u . \meaningof{C}_{\phi}(x_0,y_0,x_1,y_1,u) | T) }
\end{mathpar}

To round out this section, recall that the encoding of an $n$-crossing
knot decomposes into a parallel composition of $n$ \emph{copies} of a
crossing process together with a wiring harness. To specify different
knot classes with the same crossing number amounts to specifying
logical constraints on the wiring harness. In the interest of space,
we defer examples to a forthcoming paper. Suffice it to say that both
the conditions ``alternating knot'' and ``contains the tangle
corresponding to 5/3'' are expressible. For example, it is possible to
calculate the characteristic formula of a process corresponding to the
tangle 5/3 and conjoin it into the classifying formula via the
composition connective of the logic.

Finally, we wish to observe that it is entirely within reason to
contemplate a more domain-specific version of spatial logic tailored
to the shape of processes in the image of the encoding. Such a
domain-specific logic would have a better claim to the title formal
language of knot properties.

% subsection example_formulae_ (end)

% section knots_as_processes (end) 

% section spatial logic via knots (end)

\section{Conclusions and future work}

\paragraph{Testing physical space}
You, gentle reader, may wonder why of all the theorems to be proved
given this set up we pick the one above. In some sense it's hardly
central to quantum mechanics. We see it as central in the sense that
it firmly establishes a notion of physical space arising from a notion
of the equivalence of behavior. Relating bisimulation to a metric is a
big step forward, but one is faced with interpreting the relationship
of that metric space to something more physical. Quantum mechanical
notions of ``physical'' space are still far from intuitive, but by
relating this idea of distance as testing to calculations that predict
physical circumstances we are making a not insignificant step forward
toward an understanding of the physical space we inhabit as
essentially dynamic.

\paragraph{Effectivity and simulation}
One of the observations we have yet to make is that the entire program
spelled out here is effective. We have built various interpreters for
the reflective calculus at work in this interpretation. In principle,
then, we can simulate quantum mechanics on a computer. The place where
the simulation may lose fidelity is the infinitely branching summation
for the annihilator.

In this connection i also want to point out that the evaluation style
calculation of the inner product puts the non-determinism of the
summation right at the heart of measurement. This suggests that
Milner's original reduction-based formulation of the dynamics of his
calculi in terms of sums was not just notationally suggestive of a
notion of measure-and-continue but captured some significant part of
the physics.

\paragraph{Quantum continuations}
In light of this last observation i want to point out that the
predominant account of quantum mechanics is missing a key aspect of a
truly compositional story of the physical situation. In a real lab,
when a measurement is made the observation can be made to feed into
another device that then makes another measurement conditioned on the
results of the first. This means that after the superposition was
collapsed the entire experimental set up remained in
superposition. While QM offers a means of writing this down it doesn't
quite line up well with the well-trodden formulation of computation
and continuation that we see so succinctly expressed in Milner's
calculi. This suggests that there might be advantages to this account
of dynamics waiting to be explored.

\paragraph{Quantum logic}
In this connection, we also note that by virtue of having the
Hennessy-Milner construction, we can pull the construction through the
interpretation of QM. This gives us a natural candidate for a quantum
logic that enjoys an extremely tight connection with it's domain of
interpretation, making the construction much less ad hoc (rather it is
the image of functor!).

\paragraph{Quantum probabiity}
i have questions about the basis of the interpretation of inner
product as probability amplitude. In particular, using which
axiomatization of probability theory does the notion of probability
amplitude earn the right to be so dubbed? In other words, where is the
proof that the operation for calculating a probability amplitude (and
then squaring) satisfies the axioms of what it means to calculate a
probability? Even if such a proof exists (i have yet to find it in the
literature), i wonder if it might not be possible to turn things on
their heads. Can we view the calculation of the probability amplitude
as an axiomatization of probability? If so, then the definition we
give for calculating probability amplitude may provide the basis for
an \emph{effective} theory of probability.

\paragraph{Quantum vs ``biological'' information}
Finally, i want to conclude with a more philosophical observation. At
a recent workshop in which QM was a predominant topic i noticed
something about quantum information. The speaker was giving a riveting
discussion of axiomatic QM and showing how properties of ``no
cloning'' and ``no deleting'' emerged as consequences of the
axiomatization. Theorems of this form are necessary to give us a sense
of confidence that our axioms characterize the physical theory. What
struck me, though, was that if quantum information is neither erasable
nor replicable it is markedly different from \emph{life}. Two of the
things we know about life is that

\begin{itemize}
  \item it ends;
  \item to gain some measure of persistence, to transcend it's
    finitude it is imminently copyable.
\end{itemize}

Both of these qualities are summarized succinctly in the aphorism: all
flesh is grass. For me these two kinds of ``information'' -- call them
quantum and biological -- are end points on a spectrum of strategies
for persistence. At one end, we have those curious entities that enjoy
uniqueness and permanence; at the other, we have those who in the face
of a certain end and an uncertain present make a go of passing
something on. To me one of the more remarkable aspects of the latter
strategy is that in the presence of noise (and certain features of
copying) we get a kind of dynamism, a chance for improvement against a
given persistent condition.

% subsection other_calculi_other_bisimulations_and_geometry_as_behavior (end)




% section conclusion (end)

%\documentclass[12pt]{llncs}
%\documentclass{jktr}

\usepackage[pdftex]{hyperref}                   
\usepackage {listings}
\usepackage {mathpartir}
\usepackage{bcprules}
%\usepackage{listings}
                       
\usepackage{graphicx} 
%\usepackage[margins=2.5cm,nohead,nofoot]{geometry}
%\usepackage{geometry}
\usepackage{amsfonts}
\usepackage{amstext}
\usepackage{latexsym}
\usepackage{amssymb}
\usepackage{color}


%\include{myPreamble}
\include{qm2pi.local} 

%\ifpdf
%\usepackage[pdftex]{graphicx}
%\else
%\usepackage{graphicx}
%\fi

 % \ifpdf
%  \usepackage{pdfsync}
%  \if


%\title{Brief Article}
%\author{David F. Snyder}
%\author{L.G. Meredith}

%\address{Dept. of Math., Texas State University--San Marcos, San Marcos, TX 78666}
       
\pagestyle{empty}


\begin{document}

\lstset{language=[Objective]Caml,frame=shadowbox}

\input{qm2pi.front}

% section front matter (end)

\input{qm2pi.intro} 
 
% section introduction (end)

% \input{qm2pi.knotations} 

% section notation (end)

\input{qm2pi.process.calculi} 

% section concurrent_process_calculi_and_spatial_logics_ (end)
    
%\input{qm2pi.knots2pi} 

%\input{qm2pi.trefoil} 

%\input{qm2pi.mainthm} 

% subsection basic_interpretation (end)

%\input{qm2pi.rho.presentation} 
\subsection{The syntax and semantics of the notation system}\label{sub:the_syntax_and_semantics_of_the_notation_system} % (fold)

We now summarize a technical presentation of the calculus that
embodies our theory of dynamics. The typical presentation of such a
calculus follows the style of giving generators and relations on
them. The grammar, below, describing term constructors, freely
generates the set of processes, $\Proc$. This set is then quotiented
by a relation known as structural congruence and it is over this set
that the notion of dynamics is expressed. This presentation is
essentially that of \cite{MeredithR05} with the addition of
polyadicity and summation. For readability we have relegated some of
the technical subtleties to an appendix.

\subsubsection{Process grammar}\label{subsub:process_grammar}

\begin{mathpar}
  \inferrule* [lab=synchronization] {} {{M} \bc \pzero \;|\; x?F \;|\; x!C }
  \and
  \inferrule* [lab=abstraction] {} {{F} \bc (x)P}
  \and
  \inferrule* [lab=concretion] {} {{C} \bc \langle Q \rangle}
  \and
  \inferrule* [lab=process] {} {{P,Q} \bc M \;| \;P|Q \;|\; @{x}}
  \and
  \inferrule* [lab=name] {} {{x} \bc \quotep{P}}
\end{mathpar} 

Note that $\vec{x}$ (resp. $\vec{P}$) denotes a vector of names
(resp. processes) of length $|\vec{x}|$ (resp. $|\vec{P}|$). We adopt
the following useful abbreviations.

\begin{mathpar}
   x?(\vec{y}).P := x.(\vec{y})P \and  x\clift{\vec{P}} := x.\clift{\vec{P}}
   \and x!(y) := \lift{x}{\dropn{y}}
   \and \Pi_{i=0}^{n-1}P_i := P_0 | \ldots | P_{n-1}
\end{mathpar}

\subsubsection{Structural congruence}

\paragraph{Free and bound names and alpha-equivalence.} At the
core of structural equivalence is alpha-equivalence which identifies
process that are the same up to a change of variable. Formally, we
recognize the distinction between free and bound names. The free names
of a process, $\freenames{P}$, may be calculated recursively as
follows:

\begin{mathpar}
\freenames{\pzero} := \emptyset
  \and \\
  \freenames{x?(y).P} := \{ x \} \cup (\freenames{P} \setminus \{ y \})
  \and 
  \freenames{x!\langle P \rangle} := \{ x \} \cup \{ P \} 
  \and \\
  \freenames{P|Q} := \freenames{P} \cup \freenames{Q}
  \and \\
  \freenames{@{x}} := \{ x \}
\end{mathpar}

$\pi$
$\quotep{\pi}$

$\freenames{-} : \pi \to \mathcal{P}(\quotep{\pi})$

\begin{eqnarray*}
  \freenames{\pzero} & := & \emptyset \\
  \freenames{x?(y).P} & := & \{ x \} \cup (\freenames{P} \setminus \{ y \}) \\
  \freenames{x!\langle P \rangle} & := & \{ x \} \cup \{ P \} \\
  \freenames{P|Q} & := & \freenames{P} \cup \freenames{Q} \\
  \freenames{\dropn{x}} & := & \{ x \}
\end{eqnarray*}

The bound names of a process, $\boundnames{P}$, are those names occurring in $P$
that are not free. For example, in $x?(y).0$, the name $x$ is free, while $y$ is bound.

\begin{mathpar}
  \inferrule* [lab=monoidal-laws] {} { P|Q \equiv Q|P \and P|0 \equiv P \and P|(Q|R) \equiv (P|Q)|R }
\end{mathpar}

\begin{mathpar}
  \inferrule* [lab=alpha-equivalence] {} { (x)P \equiv (y)P\{y/x\} \and y \not\in \freenames{P} }
\end{mathpar}

\begin{definition}
Then two processes, $P,Q$, are alpha-equivalent if $P = Q\{\vec{y}/\vec{x}\}$ for
some $\vec{x} \in \boundnames{Q},\vec{y} \in \boundnames{P}$, where $Q\{\vec{y}/\vec{x}\}$
denotes the capture-avoiding substitution of $\vec{y}$ for $\vec{x}$ in $Q$.
\end{definition}

\begin{definition}
  The {\em structural congruence} \cite{SangiorgiWalker} , $\equiv$,
  between processes is the least congruence containing
  alpha-equivalence, satisfying the abelian monoid laws
  (associativity, commutativity and $\pzero$ as identity) for parallel
  composition $|$ and for summation $+$.
\end{definition}

\subsection{Name equivalence}

We take name equivalence, written $\nameeq$, to be the smallest
equivalence relation generated by the following rules.

\begin{mathpar}
\inferrule*[lab=Quote-drop]
{ }
{ \quotep{@{x}} \nameeq x }

\inferrule*[lab=Struct-equiv]
{ P \scong Q }
{ \quotep{P} \nameeq \quotep{Q} }
\end{mathpar}

The astute reader will have noticed that the mutual recursion of names
and processes imposes a mutual recursion on alpha-equivalence and
structural equivalence via name-equivalence. Fortunately, all of this
works out pleasantly and we may calculate in the natural way, free of
concern. The reader interested in the details is referred to the
appendix \ref{appendix:rho_details}.

\subsection{Substitution}

We use $\Proc$ for the set of processes, $\QProc$ for the set of
names, and $\id{\{}\vec{y} / \vec{x} \id{\}}$ to denote partial maps,
$s : \QProc \rightarrow \QProc$. A map, $s$ lifts, uniquely, to a map
on process terms, $\widehat{s} : \Proc \rightarrow \Proc$ by the
following equations.

\begin{mathpar}
  (0) \psubstp{Q}{P} := 0 \\
  (R \juxtap S) \psubstp{Q}{P}
  :=    
  (R)\psubstp{Q}{P} \juxtap (S) \psubstp{Q}{P} \\
  (x?(y).R) \psubstp{Q}{P}    
  :=    
  (x)\substp{Q}{P} (z)\concat( (R \psubstn{z}{y}) \psubstp{Q}{P} ) \\
  (\lift{x}{R}) \psubstp{Q}{P}  
  :=
  \lift{(x)\substp{Q}{P}}{ R \psubstp{Q}{P} } \\
%   (\dropn{x})  \psubstp{Q}{P}       
%   := 
%   \left\{ 
%     \begin{array}{ccc} 
%       \dropn{\quotep{Q}} & & x \nameeq \quotep{P} \\
%       \dropn{x} & & otherwise \\
%     \end{array}
%   \right. 
  (\dropn{x})  \psubstp{Q}{P}       
  := 
  \left\{ 
    \begin{array}{ccc} 
      Q & & x \nameeq \quotep{P} \\
      \dropn{x} & & otherwise \\
    \end{array}
  \right.
\end{mathpar}
 

where

\begin{eqnarray}
  (x)\id{\{} \lpquote Q \rpquote / \lpquote P \rpquote \id{\}}            = 
  \left\{ 
    \begin{array}{ccc}
      \lpquote Q \rpquote & & x \nameeq \lpquote P \rpquote \\
      x & & otherwise \\
    \end{array}
  \right. \nonumber
\end{eqnarray}

and $z$ is chosen distinct from $\quotep{P}$, $\quotep{Q}$, the free
names in $Q$, and all the names in $R$. Our $\alpha$-equivalence will
be built in the standard way from this substitution.

\begin{remark}\label{rem:no_self_referential_names}
  One consequence of these definitions is that $\forall P. \quotep{P}
  \not\in \freenames{P}$.
\end{remark}

\subsection{ Dynamic quote: an example }

Anticipating something of what's to come, consider applying the
substitution, $\widehat{\id{\{}u / z \id{\}}}$, to the following pair
of processes, $\lift{w}{y!(z)}$ and $w[ \lpquote y!(z) \rpquote ]$.

\begin{eqnarray}
	\lift{w}{y!(z)}\widehat{\id{\{}u / z \id{\}}}
		& = &
		\lift{w}{y!(u)} \nonumber\\
	w[ \lpquote y!(z) \rpquote ] \widehat{ \id{\{}u / z \id{\}} }
		& = &
		w[ \lpquote y!(z) \rpquote ] \nonumber
\end{eqnarray}

Because the body of the process between quotes is impervious to
substitution, we get radically different answers. In fact, by
examining the first process in an input context,
e.g. $x?(z).\lift{w}{y!(z)}$, we see that the process under the lift
operator may be shaped by prefixed inputs binding a name inside it. In
this sense, the lift operator will be seen as a way to dynamically
construct processes before reifying them as names.

Finally equipped with these standard features we can present the
dynamics of the calculus.

\subsubsection{Operational semantics} 

Finally, we introduce the computational dynamics. What marks these
algebras as distinct from other more traditionally studied algebraic
structures, e.g. vector spaces or polynomial rings, is the manner in
which dynamics is captured. In traditional structures, dynamics is typically
expressed through morphisms between such structures, as in linear maps
between vector spaces or morphisms between rings. In algebras
associated with the semantics of computation, the dynamics is
expressed as part of the algebraic structure itself, through a
reduction reduction relation typically denoted by $\red$. Below, we
give a recursive presentation of this relation for the calculus used
in the encoding.

$\red \subseteq \pi \times \pi$
$\red : \pi \to \mathcal{P}(\pi)$

\begin{mathpar}
  \inferrule* [lab=Comm] { \textsf{match}( x_{src}, x_{trgt} ) } { x_{trgt}?(y)P \; | \; x_{src}!\langle {Q} \rangle \red P\{\quotep{Q}/y}\} }
  \and \\
  \inferrule* [lab=Par] {{P} \red {P}'} {{{P} | {Q}} \red {{P}' | {Q}}}
  \and
  \inferrule* [lab=Equiv]{{{P} \scong {P}'} \andalso {{P}' \red {Q}'} \andalso {{Q}' \scong {Q}}}{{P} \red {Q}}
\end{mathpar}

\begin{eqnarray*}
  match_{\equiv} (\quotep{P},\quotep{Q}) & := & P \equiv Q \\
  match_{\dagger}(\quotep{P},\quotep{Q}) & := & \forall R. P|Q \red^{*} R => R \red^{*} 0 \\
  match_{K}(\quotep{P},\quotep{Q}) & := & K \mbox{ for some context } K
\end{eqnarray*}

$u?(x)P | u!\langle Q \rangle \red P\{\quotep{Q}/x\}$

%We write $\wred$ for $\red^*$, and $P\red$ if $\exists Q $ such that $ P \red Q$.
We write $P\red$ if $\exists Q $ such that $ P \red Q$ and $P\not\red$, otherwise.

\section{Replication}

As mentioned before, it is known that replication (and hence
recursion) can be implemented in a higher-order process algebra
\cite{SangiorgiWalker}. As our first example of calculation with the
machinery thus far presented we give the construction explicitly in
the {\rhoc}.

\begin{eqnarray}
	D_{x} & := & \prefix{x}{y}{(\binpar{\outputp{x}{y}}{@{y}})} \nonumber\\
	\bangp_{x}{P} & := & \binpar{{x}!\langle{\binpar{D_{x}}{P}}\rangle}{D_{x}} \nonumber
\end{eqnarray}

\begin{eqnarray}
	\bangp_{x}{P} & & \nonumber\\
	=
	& {x}!\langle{(\prefix{x}{y}{(\outputp{x}{y} | @{y})) | P}}\rangle 
	      | \prefix{x}{y}{(\outputp{x}{y} | @{y})} & \nonumber\\
	\red
	& (\outputp{x}{y} | @{y})\substn{\quotep{(\prefix{x}{y}{(@{y} | \outputp{x}{y})) | P}}}{y} & \nonumber\\
	=
	& \outputp{x}{\quotep{(\prefix{x}{y}{(\outputp{x}{y} | @{y})) | P}}}
	  | {(\prefix{x}{y}{(\outputp{x}{y} | @{y})) | P}} & \nonumber\\
	\red
	& \ldots & \nonumber\\
	\red^*
	& P | P | \ldots & \nonumber
\end{eqnarray}

Of course, this encoding, as an implementation, runs away, unfolding
$\bangp{P}$ eagerly. A lazier and more implementable replication
operator, restricted to input-guarded processes, may be obtained as follows.

\begin{eqnarray}
\bangp{\prefix{u}{v}{P}} 
	:= 
	\binpar{\lift{x}{\prefix{u}{v}{(\binpar{D(x)}{P})}}}{D(x)} \nonumber
\end{eqnarray}

\begin{remark}
  Note that the lazier definition still does not deal with summation
  or mixed summation (i.e. sums over input and output). The reader is
  invited to construct definitions of replication that deal with these
  features. 

  Further, the definitions are parameterized in a name, $x$. Can you,
  gentle reader, make a definition that eliminates this parameter and
  guarantees no accidental interaction between the replication
  machinery and the process being replicated -- i.e. no accidental
  sharing of names used by the process to get its work done and the
  name(s) used by the replication to effect copying. This latter
  revision of the definition of replication is crucial to obtaining
  the expected identity $!!P \sim !P$.
\end{remark}

\begin{remark}\label{rem:paradoxical_combinator}
  The reader familiar with the lambda calculus will have noticed the
  similarity between $D$ and the paradoxical combinator.

  [Ed. note: the existence of this seems to suggest we have to be more
  restrictive on the set of processes and names we admit if we are to
  support no-cloning.]
\end{remark}

\subsubsection{Bisimulation}

The computational dynamics gives rise to another kind of equivalence,
the equivalence of computational behavior. As previously mentioned
this is typically captured \emph{via} some form of bisimulation.

% The notion we use in this paper is weak barbed bisimulation
% \cite{milner91polyadicpi}.

The notion we use in this paper is derived from weak barbed
bisimulation \cite{milner91polyadicpi}. 

\begin{definition}
An \emph{observation relation}, $\downarrow_{\mathcal N}$, over a set
of names, $\mathcal N$, is the smallest relation satisfying the rules
below.

\infrule[Out-barb]{y \in {\mathcal N}, \; x \nameeq y}
		  {\outputp{x}{v} \downarrow_{\mathcal N} x}
\infrule[Par-barb]{\mbox{$P\downarrow_{\mathcal N} x$ or $Q\downarrow_{\mathcal N} x$}}
		  {\binpar{P}{Q} \downarrow_{\mathcal N} x}

We write $P \Downarrow_{\mathcal N} x$ if there is $Q$ such that 
$P \wred Q$ and $Q \downarrow_{\mathcal N} x$.
\end{definition}

\begin{definition}
%\label{def.bbisim}
An  ${\mathcal N}$-\emph{barbed bisimulation} over a set of names, ${\mathcal N}$, is a symmetric binary relation 
${\mathcal S}_{\mathcal N}$ between agents such that $P\rel{S}_{\mathcal N}Q$ implies:
\begin{enumerate}
\item If $P \red P'$ then $Q \wred Q'$ and $P'\rel{S}_{\mathcal N} Q'$.
\item If $P\downarrow_{\mathcal N} x$, then $Q\Downarrow_{\mathcal N} x$.
\end{enumerate}
$P$ is ${\mathcal N}$-barbed bisimilar to $Q$, written
$P \wbbisim_{\mathcal N} Q$, if $P \rel{S}_{\mathcal N} Q$ for some ${\mathcal N}$-barbed bisimulation ${\mathcal S}_{\mathcal N}$.
\end{definition}

$\mathcal{R} \subseteq \pi \times \pi$

$P \mathcal{R} Q => \forall P'. P \red P' \Rightarrow \exists Q'. Q \red Q', P' \mathcal{R} Q'$

$P \vdash x \Rightarrow Q \vdash x$

\begin{mathpar}
  \inferrule*[lab=Out-barb]{x \nameeq y}{{y}!\langle{Q}\rangle \vdash x}
  \and
  \inferrule*[lab=Par-barb]{\mbox{$P\vdash x$ or $Q\vdash x$}}{\binpar{P}{Q} \vdash x}
\end{mathpar}

\subsubsection{Contexts}

One of the principle advantages of computational calculi like the
$\pi$-calculus is a well-defined notion of context,
contextual-equivalence and a correlation between
contextual-equivalence and notions of bisimulation. The notion of
context allows the decomposition of a process into (sub-)process and
its syntactic environment, its context. Thus, a context may be
thought of as a process with a ``hole'' (written $\Box$) in it. The
application of a context $M$ to a process $P$, written $M[P]$, is
tantamount to filling the hole in $M$ with $P$. In this paper we do
not need the full weight of this theory, but do make use of the notion
of context in the proof the main theorem. 

\begin{mathpar}
  \inferrule* [lab=summation] {} {{M_{M},M_{N}} \bc \Box \;|\; x.M_{A} \;|\; M_{M}+M_{N}}
  \and
  \inferrule* [lab=agent] {} {{M_{A}} \bc (\vec{x})M_{P} \;| \; \clift{P_0,\ldots,M_{P},\ldots,P_N}}
  \and \\
  \inferrule* [lab=process] {} {{M_{P}} \bc M_{N} \;| \;P|M_{P} }
\end{mathpar} 

\begin{mathpar}
  \inferrule* [lab=sychronization] {} {M_{N} \bc \Box \;|\; x?M_{F} \;|\; x!M_{C}}
  \and
  \inferrule* [lab=abstraction] {} {{M_{F}} \bc (x)M_{P} }
  \and
  \inferrule* [lab=concretion] {} {{M_{C}} \bc \langle M_{P} \rangle }
  \and \\
  \inferrule* [lab=process] {} {{M_{P}} \bc M_{N} \;| \;P|M_{P} }
\end{mathpar}

\begin{definition}[contextual application] Given a context $M$, and
  process $P$, we define the \emph{contextual application}, $M[P] :=
  M\{P/\Box\}$. That is, the contextual application of M to P is the
  substitution of $P$ for $\Box$ in $M$.
\end{definition}

$\meaningof{-} : L \to \mathcal{P}(\pi)$

\begin{mathpar}
  \inferrule* [lab=collection] {} {\meaningof{true} = \pi, \and \meaningof{~E} = \pi \setminus \meaningof{E}, \and \meaningof{E_{1} \& E_{2}} = \meaningof{E_{1}} \cap \meaningof{E_{2}}}
\end{mathpar}

\begin{mathpar}
  \inferrule* [lab=structure] {} {\meaningof{0} = \{ P \in \pi | P \equiv 0 \}, \and \\ \meaningof{E_1 | E_2} = \{ P \in \pi | P \equiv P_{1} | P_{2}, P_{1} \in \meaningof{E_{1}}, P_{2} \in \meaningof{E_2}\} }
\end{mathpar}

\begin{mathpar}
 \inferrule* [lab=behavior] {} {\meaningof{\langle a?b \rangle E} = \{ P \in \pi | P \equiv Q | u?(y)P', \\ \and \\\\ \and \\ \;\;\; u \in \meaningof{a}, \forall z.P'\{z/y\} \in \meaningof{E\{z/b\}}\}, \and \\ \meaningof{a!E} = \{ P \in \pi | P \equiv Q | x!\langle P' \rangle, x \in \meaningof{a} P' \in \meaningof{E}\} }
\end{mathpar}

\begin{mathpar}
 \inferrule* [lab=nominal] {} {\meaningof{\quotep{E}} = \{ \quotep{P} \in \quotep{\pi} | P \in \meaningof{E} \}, \and \meaningof{\quotep{P}} = \{ \quotep{Q} \in \quotep{\pi} | P \equiv Q \} \and \\ \meaningof{@\quotep{E}} = \{ P \in \pi | P \equiv @x, x \in \meaningof{E} \}}
\end{mathpar}

\begin{eqnarray*}
  \\
  \meaningof{-} : TS \to ST
\end{eqnarray*}

\begin{eqnarray*}
  \\
  L : TS \to ST
\end{eqnarray*}

\begin{eqnarray*}
  \\
  P \models E \iff P \in \meaningof{E}
\end{eqnarray*}

\begin{eqnarray*}
  P \approx_{L} Q \iff \forall E \in L. P \models E \iff Q \models E
\end{eqnarray*}

\begin{eqnarray*}
  P \approx_{K} Q
\end{eqnarray*}

\begin{eqnarray*}
  P \approx Q
\end{eqnarray*}

$\approx_{K} = \approx = \approx_{L}$

\subsubsection{Contextual duality}

Note that contexts extend the quotation operation to a family of
operations from processes to names. Given a context, $M$, we can
define a \emph{nominal context}, $\quotep{M}$ by $\quotep{M}[P] :=
\quotep{M[P]}$. To foreshadow what is to come we observe that these
operations enjoy a duality with processes very much like the duality
between vectors and maps from vectors to scalars.

Further, because the calculus is essentially higher-order, we have a
correspondence between contexts and processes. More specifically,
given a name $x$ and a context $M$ we can construct $M^{*}_{x}$ such
that 

\begin{mathpar}
  M^{*}_{x} | \lift{x}{P} \red M[P]
\end{mathpar}

namely,

\begin{mathpar}
  M^{*}_{x} := x?(u).M[\dropn{u}]
\end{mathpar}

The dependence of $M^{*}_{x}$ on a name makes it an abstraction, 

\begin{mathpar}
  M^{*} := (x)x?(u).M[\dropn{u}]
\end{mathpar}

\subsection{Additional notation}

It will sometimes be convenient to denote the process a name
quotes. We already have the notation $x = \quotep{P}$, but it will be
convenient to introduce an alternate notation, $\procn{x}$, when we
want to emphasize the connection to the use of the name. Note that, by
virtue of name equivalence, $\quotep{\procn{x}} \nameeq x$; so, the
notation is consistent with previous definitions.

Further, because names have structure it is possible to effect
substitutions on the basis of that structure. This means we need to
upgrade our notation for substitutions, which we accomplish by
adapting comprehension notation. Thus,

\begin{mathpar}
  P\{ y / x : x \in S \}
\end{mathpar}

is interpreted to mean the process derived from P by replacing (in a
capture-avoiding manner) each occurrence of $x$ in $S$ by $y$. For example,

\begin{mathpar}
  P\{ \quotep{\procn{x}|\procn{x}} / x : x \in \freenames{P} \}
\end{mathpar}

will replace each (occurrence) of a free name $x$ in $P$ by
$\quotep{\procn{x}|\procn{x}}$.

Also, we will avail ourselves of the notation $x^{L}$ and $x^{R}$ to
denote injections of a name into disjoint copies of the name
space. There are numerous ways to accomplish this. One example can be
found in \cite{MeredithR05}. This notation overloads to vectors of
names: $\vec{x}^{\pi} := (x_{i}^{\pi} \; : \; 0 \leq i < |\vec{x}| )$ where $\pi \in \{L,R\}$.

We also use $P^{\Box} := P|\Box$.

In \cite{MeredithR05} an interpretation of the new operator is
given. It turns out that there are several possible interpretations
all enjoying the requisite algebraic properties of the operator (see
\cite{milner91polyadicpi}). We will therefore make liberal use of
$(\nu\; \vec{x})P$.

% subsection the_syntax_and_semantics_of_the_notation_system (end)   

\input{qm2pi.qmops} 

\input{qm2pi.sterngerlach} 

\input{qm2pi.metric} 

% section concurrent_process_calculi (end)

%\input{qm2pi.proofsketch}

% section proof sketch (end)

%\input{qm2pi.slviaknots} 

% section spatial logic via knots (end)

\input{qm2pi.conclusion}

% section conclusion (end)

%\input{qm2pi.dtcodes} 

% section wiring algorithm (end)

\input{qm2pi.ack} 

% section acknowledgments (end)

\newpage


\bibliographystyle{plain}   
\bibliography{../../biblios/main.bib}

\input{qm2pi.rhodetails}

\end{document}

 

% section wiring algorithm (end)

\documentclass[12pt]{llncs}
%\documentclass{jktr}

\usepackage[pdftex]{hyperref}                   
\usepackage {listings}
\usepackage {mathpartir}
\usepackage{bcprules}
%\usepackage{listings}
                       
\usepackage{graphicx} 
%\usepackage[margins=2.5cm,nohead,nofoot]{geometry}
%\usepackage{geometry}
\usepackage{amsfonts}
\usepackage{amstext}
\usepackage{latexsym}
\usepackage{amssymb}
\usepackage{color}


%\include{myPreamble}
\include{qm2pi.local} 

%\ifpdf
%\usepackage[pdftex]{graphicx}
%\else
%\usepackage{graphicx}
%\fi

 % \ifpdf
%  \usepackage{pdfsync}
%  \if


%\title{Brief Article}
%\author{David F. Snyder}
%\author{L.G. Meredith}

%\address{Dept. of Math., Texas State University--San Marcos, San Marcos, TX 78666}
       
\pagestyle{empty}


\begin{document}

\lstset{language=[Objective]Caml,frame=shadowbox}

\input{qm2pi.front}

% section front matter (end)

\input{qm2pi.intro} 
 
% section introduction (end)

% \input{qm2pi.knotations} 

% section notation (end)

\input{qm2pi.process.calculi} 

% section concurrent_process_calculi_and_spatial_logics_ (end)
    
%\input{qm2pi.knots2pi} 

%\input{qm2pi.trefoil} 

%\input{qm2pi.mainthm} 

% subsection basic_interpretation (end)

%\input{qm2pi.rho.presentation} 
\subsection{The syntax and semantics of the notation system}\label{sub:the_syntax_and_semantics_of_the_notation_system} % (fold)

We now summarize a technical presentation of the calculus that
embodies our theory of dynamics. The typical presentation of such a
calculus follows the style of giving generators and relations on
them. The grammar, below, describing term constructors, freely
generates the set of processes, $\Proc$. This set is then quotiented
by a relation known as structural congruence and it is over this set
that the notion of dynamics is expressed. This presentation is
essentially that of \cite{MeredithR05} with the addition of
polyadicity and summation. For readability we have relegated some of
the technical subtleties to an appendix.

\subsubsection{Process grammar}\label{subsub:process_grammar}

\begin{mathpar}
  \inferrule* [lab=synchronization] {} {{M} \bc \pzero \;|\; x?F \;|\; x!C }
  \and
  \inferrule* [lab=abstraction] {} {{F} \bc (x)P}
  \and
  \inferrule* [lab=concretion] {} {{C} \bc \langle Q \rangle}
  \and
  \inferrule* [lab=process] {} {{P,Q} \bc M \;| \;P|Q \;|\; @{x}}
  \and
  \inferrule* [lab=name] {} {{x} \bc \quotep{P}}
\end{mathpar} 

Note that $\vec{x}$ (resp. $\vec{P}$) denotes a vector of names
(resp. processes) of length $|\vec{x}|$ (resp. $|\vec{P}|$). We adopt
the following useful abbreviations.

\begin{mathpar}
   x?(\vec{y}).P := x.(\vec{y})P \and  x\clift{\vec{P}} := x.\clift{\vec{P}}
   \and x!(y) := \lift{x}{\dropn{y}}
   \and \Pi_{i=0}^{n-1}P_i := P_0 | \ldots | P_{n-1}
\end{mathpar}

\subsubsection{Structural congruence}

\paragraph{Free and bound names and alpha-equivalence.} At the
core of structural equivalence is alpha-equivalence which identifies
process that are the same up to a change of variable. Formally, we
recognize the distinction between free and bound names. The free names
of a process, $\freenames{P}$, may be calculated recursively as
follows:

\begin{mathpar}
\freenames{\pzero} := \emptyset
  \and \\
  \freenames{x?(y).P} := \{ x \} \cup (\freenames{P} \setminus \{ y \})
  \and 
  \freenames{x!\langle P \rangle} := \{ x \} \cup \{ P \} 
  \and \\
  \freenames{P|Q} := \freenames{P} \cup \freenames{Q}
  \and \\
  \freenames{@{x}} := \{ x \}
\end{mathpar}

$\pi$
$\quotep{\pi}$

$\freenames{-} : \pi \to \mathcal{P}(\quotep{\pi})$

\begin{eqnarray*}
  \freenames{\pzero} & := & \emptyset \\
  \freenames{x?(y).P} & := & \{ x \} \cup (\freenames{P} \setminus \{ y \}) \\
  \freenames{x!\langle P \rangle} & := & \{ x \} \cup \{ P \} \\
  \freenames{P|Q} & := & \freenames{P} \cup \freenames{Q} \\
  \freenames{\dropn{x}} & := & \{ x \}
\end{eqnarray*}

The bound names of a process, $\boundnames{P}$, are those names occurring in $P$
that are not free. For example, in $x?(y).0$, the name $x$ is free, while $y$ is bound.

\begin{mathpar}
  \inferrule* [lab=monoidal-laws] {} { P|Q \equiv Q|P \and P|0 \equiv P \and P|(Q|R) \equiv (P|Q)|R }
\end{mathpar}

\begin{mathpar}
  \inferrule* [lab=alpha-equivalence] {} { (x)P \equiv (y)P\{y/x\} \and y \not\in \freenames{P} }
\end{mathpar}

\begin{definition}
Then two processes, $P,Q$, are alpha-equivalent if $P = Q\{\vec{y}/\vec{x}\}$ for
some $\vec{x} \in \boundnames{Q},\vec{y} \in \boundnames{P}$, where $Q\{\vec{y}/\vec{x}\}$
denotes the capture-avoiding substitution of $\vec{y}$ for $\vec{x}$ in $Q$.
\end{definition}

\begin{definition}
  The {\em structural congruence} \cite{SangiorgiWalker} , $\equiv$,
  between processes is the least congruence containing
  alpha-equivalence, satisfying the abelian monoid laws
  (associativity, commutativity and $\pzero$ as identity) for parallel
  composition $|$ and for summation $+$.
\end{definition}

\subsection{Name equivalence}

We take name equivalence, written $\nameeq$, to be the smallest
equivalence relation generated by the following rules.

\begin{mathpar}
\inferrule*[lab=Quote-drop]
{ }
{ \quotep{@{x}} \nameeq x }

\inferrule*[lab=Struct-equiv]
{ P \scong Q }
{ \quotep{P} \nameeq \quotep{Q} }
\end{mathpar}

The astute reader will have noticed that the mutual recursion of names
and processes imposes a mutual recursion on alpha-equivalence and
structural equivalence via name-equivalence. Fortunately, all of this
works out pleasantly and we may calculate in the natural way, free of
concern. The reader interested in the details is referred to the
appendix \ref{appendix:rho_details}.

\subsection{Substitution}

We use $\Proc$ for the set of processes, $\QProc$ for the set of
names, and $\id{\{}\vec{y} / \vec{x} \id{\}}$ to denote partial maps,
$s : \QProc \rightarrow \QProc$. A map, $s$ lifts, uniquely, to a map
on process terms, $\widehat{s} : \Proc \rightarrow \Proc$ by the
following equations.

\begin{mathpar}
  (0) \psubstp{Q}{P} := 0 \\
  (R \juxtap S) \psubstp{Q}{P}
  :=    
  (R)\psubstp{Q}{P} \juxtap (S) \psubstp{Q}{P} \\
  (x?(y).R) \psubstp{Q}{P}    
  :=    
  (x)\substp{Q}{P} (z)\concat( (R \psubstn{z}{y}) \psubstp{Q}{P} ) \\
  (\lift{x}{R}) \psubstp{Q}{P}  
  :=
  \lift{(x)\substp{Q}{P}}{ R \psubstp{Q}{P} } \\
%   (\dropn{x})  \psubstp{Q}{P}       
%   := 
%   \left\{ 
%     \begin{array}{ccc} 
%       \dropn{\quotep{Q}} & & x \nameeq \quotep{P} \\
%       \dropn{x} & & otherwise \\
%     \end{array}
%   \right. 
  (\dropn{x})  \psubstp{Q}{P}       
  := 
  \left\{ 
    \begin{array}{ccc} 
      Q & & x \nameeq \quotep{P} \\
      \dropn{x} & & otherwise \\
    \end{array}
  \right.
\end{mathpar}
 

where

\begin{eqnarray}
  (x)\id{\{} \lpquote Q \rpquote / \lpquote P \rpquote \id{\}}            = 
  \left\{ 
    \begin{array}{ccc}
      \lpquote Q \rpquote & & x \nameeq \lpquote P \rpquote \\
      x & & otherwise \\
    \end{array}
  \right. \nonumber
\end{eqnarray}

and $z$ is chosen distinct from $\quotep{P}$, $\quotep{Q}$, the free
names in $Q$, and all the names in $R$. Our $\alpha$-equivalence will
be built in the standard way from this substitution.

\begin{remark}\label{rem:no_self_referential_names}
  One consequence of these definitions is that $\forall P. \quotep{P}
  \not\in \freenames{P}$.
\end{remark}

\subsection{ Dynamic quote: an example }

Anticipating something of what's to come, consider applying the
substitution, $\widehat{\id{\{}u / z \id{\}}}$, to the following pair
of processes, $\lift{w}{y!(z)}$ and $w[ \lpquote y!(z) \rpquote ]$.

\begin{eqnarray}
	\lift{w}{y!(z)}\widehat{\id{\{}u / z \id{\}}}
		& = &
		\lift{w}{y!(u)} \nonumber\\
	w[ \lpquote y!(z) \rpquote ] \widehat{ \id{\{}u / z \id{\}} }
		& = &
		w[ \lpquote y!(z) \rpquote ] \nonumber
\end{eqnarray}

Because the body of the process between quotes is impervious to
substitution, we get radically different answers. In fact, by
examining the first process in an input context,
e.g. $x?(z).\lift{w}{y!(z)}$, we see that the process under the lift
operator may be shaped by prefixed inputs binding a name inside it. In
this sense, the lift operator will be seen as a way to dynamically
construct processes before reifying them as names.

Finally equipped with these standard features we can present the
dynamics of the calculus.

\subsubsection{Operational semantics} 

Finally, we introduce the computational dynamics. What marks these
algebras as distinct from other more traditionally studied algebraic
structures, e.g. vector spaces or polynomial rings, is the manner in
which dynamics is captured. In traditional structures, dynamics is typically
expressed through morphisms between such structures, as in linear maps
between vector spaces or morphisms between rings. In algebras
associated with the semantics of computation, the dynamics is
expressed as part of the algebraic structure itself, through a
reduction reduction relation typically denoted by $\red$. Below, we
give a recursive presentation of this relation for the calculus used
in the encoding.

$\red \subseteq \pi \times \pi$
$\red : \pi \to \mathcal{P}(\pi)$

\begin{mathpar}
  \inferrule* [lab=Comm] { \textsf{match}( x_{src}, x_{trgt} ) } { x_{trgt}?(y)P \; | \; x_{src}!\langle {Q} \rangle \red P\{\quotep{Q}/y}\} }
  \and \\
  \inferrule* [lab=Par] {{P} \red {P}'} {{{P} | {Q}} \red {{P}' | {Q}}}
  \and
  \inferrule* [lab=Equiv]{{{P} \scong {P}'} \andalso {{P}' \red {Q}'} \andalso {{Q}' \scong {Q}}}{{P} \red {Q}}
\end{mathpar}

\begin{eqnarray*}
  match_{\equiv} (\quotep{P},\quotep{Q}) & := & P \equiv Q \\
  match_{\dagger}(\quotep{P},\quotep{Q}) & := & \forall R. P|Q \red^{*} R => R \red^{*} 0 \\
  match_{K}(\quotep{P},\quotep{Q}) & := & K \mbox{ for some context } K
\end{eqnarray*}

$u?(x)P | u!\langle Q \rangle \red P\{\quotep{Q}/x\}$

%We write $\wred$ for $\red^*$, and $P\red$ if $\exists Q $ such that $ P \red Q$.
We write $P\red$ if $\exists Q $ such that $ P \red Q$ and $P\not\red$, otherwise.

\section{Replication}

As mentioned before, it is known that replication (and hence
recursion) can be implemented in a higher-order process algebra
\cite{SangiorgiWalker}. As our first example of calculation with the
machinery thus far presented we give the construction explicitly in
the {\rhoc}.

\begin{eqnarray}
	D_{x} & := & \prefix{x}{y}{(\binpar{\outputp{x}{y}}{@{y}})} \nonumber\\
	\bangp_{x}{P} & := & \binpar{{x}!\langle{\binpar{D_{x}}{P}}\rangle}{D_{x}} \nonumber
\end{eqnarray}

\begin{eqnarray}
	\bangp_{x}{P} & & \nonumber\\
	=
	& {x}!\langle{(\prefix{x}{y}{(\outputp{x}{y} | @{y})) | P}}\rangle 
	      | \prefix{x}{y}{(\outputp{x}{y} | @{y})} & \nonumber\\
	\red
	& (\outputp{x}{y} | @{y})\substn{\quotep{(\prefix{x}{y}{(@{y} | \outputp{x}{y})) | P}}}{y} & \nonumber\\
	=
	& \outputp{x}{\quotep{(\prefix{x}{y}{(\outputp{x}{y} | @{y})) | P}}}
	  | {(\prefix{x}{y}{(\outputp{x}{y} | @{y})) | P}} & \nonumber\\
	\red
	& \ldots & \nonumber\\
	\red^*
	& P | P | \ldots & \nonumber
\end{eqnarray}

Of course, this encoding, as an implementation, runs away, unfolding
$\bangp{P}$ eagerly. A lazier and more implementable replication
operator, restricted to input-guarded processes, may be obtained as follows.

\begin{eqnarray}
\bangp{\prefix{u}{v}{P}} 
	:= 
	\binpar{\lift{x}{\prefix{u}{v}{(\binpar{D(x)}{P})}}}{D(x)} \nonumber
\end{eqnarray}

\begin{remark}
  Note that the lazier definition still does not deal with summation
  or mixed summation (i.e. sums over input and output). The reader is
  invited to construct definitions of replication that deal with these
  features. 

  Further, the definitions are parameterized in a name, $x$. Can you,
  gentle reader, make a definition that eliminates this parameter and
  guarantees no accidental interaction between the replication
  machinery and the process being replicated -- i.e. no accidental
  sharing of names used by the process to get its work done and the
  name(s) used by the replication to effect copying. This latter
  revision of the definition of replication is crucial to obtaining
  the expected identity $!!P \sim !P$.
\end{remark}

\begin{remark}\label{rem:paradoxical_combinator}
  The reader familiar with the lambda calculus will have noticed the
  similarity between $D$ and the paradoxical combinator.

  [Ed. note: the existence of this seems to suggest we have to be more
  restrictive on the set of processes and names we admit if we are to
  support no-cloning.]
\end{remark}

\subsubsection{Bisimulation}

The computational dynamics gives rise to another kind of equivalence,
the equivalence of computational behavior. As previously mentioned
this is typically captured \emph{via} some form of bisimulation.

% The notion we use in this paper is weak barbed bisimulation
% \cite{milner91polyadicpi}.

The notion we use in this paper is derived from weak barbed
bisimulation \cite{milner91polyadicpi}. 

\begin{definition}
An \emph{observation relation}, $\downarrow_{\mathcal N}$, over a set
of names, $\mathcal N$, is the smallest relation satisfying the rules
below.

\infrule[Out-barb]{y \in {\mathcal N}, \; x \nameeq y}
		  {\outputp{x}{v} \downarrow_{\mathcal N} x}
\infrule[Par-barb]{\mbox{$P\downarrow_{\mathcal N} x$ or $Q\downarrow_{\mathcal N} x$}}
		  {\binpar{P}{Q} \downarrow_{\mathcal N} x}

We write $P \Downarrow_{\mathcal N} x$ if there is $Q$ such that 
$P \wred Q$ and $Q \downarrow_{\mathcal N} x$.
\end{definition}

\begin{definition}
%\label{def.bbisim}
An  ${\mathcal N}$-\emph{barbed bisimulation} over a set of names, ${\mathcal N}$, is a symmetric binary relation 
${\mathcal S}_{\mathcal N}$ between agents such that $P\rel{S}_{\mathcal N}Q$ implies:
\begin{enumerate}
\item If $P \red P'$ then $Q \wred Q'$ and $P'\rel{S}_{\mathcal N} Q'$.
\item If $P\downarrow_{\mathcal N} x$, then $Q\Downarrow_{\mathcal N} x$.
\end{enumerate}
$P$ is ${\mathcal N}$-barbed bisimilar to $Q$, written
$P \wbbisim_{\mathcal N} Q$, if $P \rel{S}_{\mathcal N} Q$ for some ${\mathcal N}$-barbed bisimulation ${\mathcal S}_{\mathcal N}$.
\end{definition}

$\mathcal{R} \subseteq \pi \times \pi$

$P \mathcal{R} Q => \forall P'. P \red P' \Rightarrow \exists Q'. Q \red Q', P' \mathcal{R} Q'$

$P \vdash x \Rightarrow Q \vdash x$

\begin{mathpar}
  \inferrule*[lab=Out-barb]{x \nameeq y}{{y}!\langle{Q}\rangle \vdash x}
  \and
  \inferrule*[lab=Par-barb]{\mbox{$P\vdash x$ or $Q\vdash x$}}{\binpar{P}{Q} \vdash x}
\end{mathpar}

\subsubsection{Contexts}

One of the principle advantages of computational calculi like the
$\pi$-calculus is a well-defined notion of context,
contextual-equivalence and a correlation between
contextual-equivalence and notions of bisimulation. The notion of
context allows the decomposition of a process into (sub-)process and
its syntactic environment, its context. Thus, a context may be
thought of as a process with a ``hole'' (written $\Box$) in it. The
application of a context $M$ to a process $P$, written $M[P]$, is
tantamount to filling the hole in $M$ with $P$. In this paper we do
not need the full weight of this theory, but do make use of the notion
of context in the proof the main theorem. 

\begin{mathpar}
  \inferrule* [lab=summation] {} {{M_{M},M_{N}} \bc \Box \;|\; x.M_{A} \;|\; M_{M}+M_{N}}
  \and
  \inferrule* [lab=agent] {} {{M_{A}} \bc (\vec{x})M_{P} \;| \; \clift{P_0,\ldots,M_{P},\ldots,P_N}}
  \and \\
  \inferrule* [lab=process] {} {{M_{P}} \bc M_{N} \;| \;P|M_{P} }
\end{mathpar} 

\begin{mathpar}
  \inferrule* [lab=sychronization] {} {M_{N} \bc \Box \;|\; x?M_{F} \;|\; x!M_{C}}
  \and
  \inferrule* [lab=abstraction] {} {{M_{F}} \bc (x)M_{P} }
  \and
  \inferrule* [lab=concretion] {} {{M_{C}} \bc \langle M_{P} \rangle }
  \and \\
  \inferrule* [lab=process] {} {{M_{P}} \bc M_{N} \;| \;P|M_{P} }
\end{mathpar}

\begin{definition}[contextual application] Given a context $M$, and
  process $P$, we define the \emph{contextual application}, $M[P] :=
  M\{P/\Box\}$. That is, the contextual application of M to P is the
  substitution of $P$ for $\Box$ in $M$.
\end{definition}

$\meaningof{-} : L \to \mathcal{P}(\pi)$

\begin{mathpar}
  \inferrule* [lab=collection] {} {\meaningof{true} = \pi, \and \meaningof{~E} = \pi \setminus \meaningof{E}, \and \meaningof{E_{1} \& E_{2}} = \meaningof{E_{1}} \cap \meaningof{E_{2}}}
\end{mathpar}

\begin{mathpar}
  \inferrule* [lab=structure] {} {\meaningof{0} = \{ P \in \pi | P \equiv 0 \}, \and \\ \meaningof{E_1 | E_2} = \{ P \in \pi | P \equiv P_{1} | P_{2}, P_{1} \in \meaningof{E_{1}}, P_{2} \in \meaningof{E_2}\} }
\end{mathpar}

\begin{mathpar}
 \inferrule* [lab=behavior] {} {\meaningof{\langle a?b \rangle E} = \{ P \in \pi | P \equiv Q | u?(y)P', \\ \and \\\\ \and \\ \;\;\; u \in \meaningof{a}, \forall z.P'\{z/y\} \in \meaningof{E\{z/b\}}\}, \and \\ \meaningof{a!E} = \{ P \in \pi | P \equiv Q | x!\langle P' \rangle, x \in \meaningof{a} P' \in \meaningof{E}\} }
\end{mathpar}

\begin{mathpar}
 \inferrule* [lab=nominal] {} {\meaningof{\quotep{E}} = \{ \quotep{P} \in \quotep{\pi} | P \in \meaningof{E} \}, \and \meaningof{\quotep{P}} = \{ \quotep{Q} \in \quotep{\pi} | P \equiv Q \} \and \\ \meaningof{@\quotep{E}} = \{ P \in \pi | P \equiv @x, x \in \meaningof{E} \}}
\end{mathpar}

\begin{eqnarray*}
  \\
  \meaningof{-} : TS \to ST
\end{eqnarray*}

\begin{eqnarray*}
  \\
  L : TS \to ST
\end{eqnarray*}

\begin{eqnarray*}
  \\
  P \models E \iff P \in \meaningof{E}
\end{eqnarray*}

\begin{eqnarray*}
  P \approx_{L} Q \iff \forall E \in L. P \models E \iff Q \models E
\end{eqnarray*}

\begin{eqnarray*}
  P \approx_{K} Q
\end{eqnarray*}

\begin{eqnarray*}
  P \approx Q
\end{eqnarray*}

$\approx_{K} = \approx = \approx_{L}$

\subsubsection{Contextual duality}

Note that contexts extend the quotation operation to a family of
operations from processes to names. Given a context, $M$, we can
define a \emph{nominal context}, $\quotep{M}$ by $\quotep{M}[P] :=
\quotep{M[P]}$. To foreshadow what is to come we observe that these
operations enjoy a duality with processes very much like the duality
between vectors and maps from vectors to scalars.

Further, because the calculus is essentially higher-order, we have a
correspondence between contexts and processes. More specifically,
given a name $x$ and a context $M$ we can construct $M^{*}_{x}$ such
that 

\begin{mathpar}
  M^{*}_{x} | \lift{x}{P} \red M[P]
\end{mathpar}

namely,

\begin{mathpar}
  M^{*}_{x} := x?(u).M[\dropn{u}]
\end{mathpar}

The dependence of $M^{*}_{x}$ on a name makes it an abstraction, 

\begin{mathpar}
  M^{*} := (x)x?(u).M[\dropn{u}]
\end{mathpar}

\subsection{Additional notation}

It will sometimes be convenient to denote the process a name
quotes. We already have the notation $x = \quotep{P}$, but it will be
convenient to introduce an alternate notation, $\procn{x}$, when we
want to emphasize the connection to the use of the name. Note that, by
virtue of name equivalence, $\quotep{\procn{x}} \nameeq x$; so, the
notation is consistent with previous definitions.

Further, because names have structure it is possible to effect
substitutions on the basis of that structure. This means we need to
upgrade our notation for substitutions, which we accomplish by
adapting comprehension notation. Thus,

\begin{mathpar}
  P\{ y / x : x \in S \}
\end{mathpar}

is interpreted to mean the process derived from P by replacing (in a
capture-avoiding manner) each occurrence of $x$ in $S$ by $y$. For example,

\begin{mathpar}
  P\{ \quotep{\procn{x}|\procn{x}} / x : x \in \freenames{P} \}
\end{mathpar}

will replace each (occurrence) of a free name $x$ in $P$ by
$\quotep{\procn{x}|\procn{x}}$.

Also, we will avail ourselves of the notation $x^{L}$ and $x^{R}$ to
denote injections of a name into disjoint copies of the name
space. There are numerous ways to accomplish this. One example can be
found in \cite{MeredithR05}. This notation overloads to vectors of
names: $\vec{x}^{\pi} := (x_{i}^{\pi} \; : \; 0 \leq i < |\vec{x}| )$ where $\pi \in \{L,R\}$.

We also use $P^{\Box} := P|\Box$.

In \cite{MeredithR05} an interpretation of the new operator is
given. It turns out that there are several possible interpretations
all enjoying the requisite algebraic properties of the operator (see
\cite{milner91polyadicpi}). We will therefore make liberal use of
$(\nu\; \vec{x})P$.

% subsection the_syntax_and_semantics_of_the_notation_system (end)   

\input{qm2pi.qmops} 

\input{qm2pi.sterngerlach} 

\input{qm2pi.metric} 

% section concurrent_process_calculi (end)

%\input{qm2pi.proofsketch}

% section proof sketch (end)

%\input{qm2pi.slviaknots} 

% section spatial logic via knots (end)

\input{qm2pi.conclusion}

% section conclusion (end)

%\input{qm2pi.dtcodes} 

% section wiring algorithm (end)

\input{qm2pi.ack} 

% section acknowledgments (end)

\newpage


\bibliographystyle{plain}   
\bibliography{../../biblios/main.bib}

\input{qm2pi.rhodetails}

\end{document}

 

% section acknowledgments (end)

\newpage


\bibliographystyle{plain}   
\bibliography{../../biblios/main.bib}

\documentclass[12pt]{llncs}
%\documentclass{jktr}

\usepackage[pdftex]{hyperref}                   
\usepackage {listings}
\usepackage {mathpartir}
\usepackage{bcprules}
%\usepackage{listings}
                       
\usepackage{graphicx} 
%\usepackage[margins=2.5cm,nohead,nofoot]{geometry}
%\usepackage{geometry}
\usepackage{amsfonts}
\usepackage{amstext}
\usepackage{latexsym}
\usepackage{amssymb}
\usepackage{color}


%\include{myPreamble}
\include{qm2pi.local} 

%\ifpdf
%\usepackage[pdftex]{graphicx}
%\else
%\usepackage{graphicx}
%\fi

 % \ifpdf
%  \usepackage{pdfsync}
%  \if


%\title{Brief Article}
%\author{David F. Snyder}
%\author{L.G. Meredith}

%\address{Dept. of Math., Texas State University--San Marcos, San Marcos, TX 78666}
       
\pagestyle{empty}


\begin{document}

\lstset{language=[Objective]Caml,frame=shadowbox}

\input{qm2pi.front}

% section front matter (end)

\input{qm2pi.intro} 
 
% section introduction (end)

% \input{qm2pi.knotations} 

% section notation (end)

\input{qm2pi.process.calculi} 

% section concurrent_process_calculi_and_spatial_logics_ (end)
    
%\input{qm2pi.knots2pi} 

%\input{qm2pi.trefoil} 

%\input{qm2pi.mainthm} 

% subsection basic_interpretation (end)

%\input{qm2pi.rho.presentation} 
\subsection{The syntax and semantics of the notation system}\label{sub:the_syntax_and_semantics_of_the_notation_system} % (fold)

We now summarize a technical presentation of the calculus that
embodies our theory of dynamics. The typical presentation of such a
calculus follows the style of giving generators and relations on
them. The grammar, below, describing term constructors, freely
generates the set of processes, $\Proc$. This set is then quotiented
by a relation known as structural congruence and it is over this set
that the notion of dynamics is expressed. This presentation is
essentially that of \cite{MeredithR05} with the addition of
polyadicity and summation. For readability we have relegated some of
the technical subtleties to an appendix.

\subsubsection{Process grammar}\label{subsub:process_grammar}

\begin{mathpar}
  \inferrule* [lab=synchronization] {} {{M} \bc \pzero \;|\; x?F \;|\; x!C }
  \and
  \inferrule* [lab=abstraction] {} {{F} \bc (x)P}
  \and
  \inferrule* [lab=concretion] {} {{C} \bc \langle Q \rangle}
  \and
  \inferrule* [lab=process] {} {{P,Q} \bc M \;| \;P|Q \;|\; @{x}}
  \and
  \inferrule* [lab=name] {} {{x} \bc \quotep{P}}
\end{mathpar} 

Note that $\vec{x}$ (resp. $\vec{P}$) denotes a vector of names
(resp. processes) of length $|\vec{x}|$ (resp. $|\vec{P}|$). We adopt
the following useful abbreviations.

\begin{mathpar}
   x?(\vec{y}).P := x.(\vec{y})P \and  x\clift{\vec{P}} := x.\clift{\vec{P}}
   \and x!(y) := \lift{x}{\dropn{y}}
   \and \Pi_{i=0}^{n-1}P_i := P_0 | \ldots | P_{n-1}
\end{mathpar}

\subsubsection{Structural congruence}

\paragraph{Free and bound names and alpha-equivalence.} At the
core of structural equivalence is alpha-equivalence which identifies
process that are the same up to a change of variable. Formally, we
recognize the distinction between free and bound names. The free names
of a process, $\freenames{P}$, may be calculated recursively as
follows:

\begin{mathpar}
\freenames{\pzero} := \emptyset
  \and \\
  \freenames{x?(y).P} := \{ x \} \cup (\freenames{P} \setminus \{ y \})
  \and 
  \freenames{x!\langle P \rangle} := \{ x \} \cup \{ P \} 
  \and \\
  \freenames{P|Q} := \freenames{P} \cup \freenames{Q}
  \and \\
  \freenames{@{x}} := \{ x \}
\end{mathpar}

$\pi$
$\quotep{\pi}$

$\freenames{-} : \pi \to \mathcal{P}(\quotep{\pi})$

\begin{eqnarray*}
  \freenames{\pzero} & := & \emptyset \\
  \freenames{x?(y).P} & := & \{ x \} \cup (\freenames{P} \setminus \{ y \}) \\
  \freenames{x!\langle P \rangle} & := & \{ x \} \cup \{ P \} \\
  \freenames{P|Q} & := & \freenames{P} \cup \freenames{Q} \\
  \freenames{\dropn{x}} & := & \{ x \}
\end{eqnarray*}

The bound names of a process, $\boundnames{P}$, are those names occurring in $P$
that are not free. For example, in $x?(y).0$, the name $x$ is free, while $y$ is bound.

\begin{mathpar}
  \inferrule* [lab=monoidal-laws] {} { P|Q \equiv Q|P \and P|0 \equiv P \and P|(Q|R) \equiv (P|Q)|R }
\end{mathpar}

\begin{mathpar}
  \inferrule* [lab=alpha-equivalence] {} { (x)P \equiv (y)P\{y/x\} \and y \not\in \freenames{P} }
\end{mathpar}

\begin{definition}
Then two processes, $P,Q$, are alpha-equivalent if $P = Q\{\vec{y}/\vec{x}\}$ for
some $\vec{x} \in \boundnames{Q},\vec{y} \in \boundnames{P}$, where $Q\{\vec{y}/\vec{x}\}$
denotes the capture-avoiding substitution of $\vec{y}$ for $\vec{x}$ in $Q$.
\end{definition}

\begin{definition}
  The {\em structural congruence} \cite{SangiorgiWalker} , $\equiv$,
  between processes is the least congruence containing
  alpha-equivalence, satisfying the abelian monoid laws
  (associativity, commutativity and $\pzero$ as identity) for parallel
  composition $|$ and for summation $+$.
\end{definition}

\subsection{Name equivalence}

We take name equivalence, written $\nameeq$, to be the smallest
equivalence relation generated by the following rules.

\begin{mathpar}
\inferrule*[lab=Quote-drop]
{ }
{ \quotep{@{x}} \nameeq x }

\inferrule*[lab=Struct-equiv]
{ P \scong Q }
{ \quotep{P} \nameeq \quotep{Q} }
\end{mathpar}

The astute reader will have noticed that the mutual recursion of names
and processes imposes a mutual recursion on alpha-equivalence and
structural equivalence via name-equivalence. Fortunately, all of this
works out pleasantly and we may calculate in the natural way, free of
concern. The reader interested in the details is referred to the
appendix \ref{appendix:rho_details}.

\subsection{Substitution}

We use $\Proc$ for the set of processes, $\QProc$ for the set of
names, and $\id{\{}\vec{y} / \vec{x} \id{\}}$ to denote partial maps,
$s : \QProc \rightarrow \QProc$. A map, $s$ lifts, uniquely, to a map
on process terms, $\widehat{s} : \Proc \rightarrow \Proc$ by the
following equations.

\begin{mathpar}
  (0) \psubstp{Q}{P} := 0 \\
  (R \juxtap S) \psubstp{Q}{P}
  :=    
  (R)\psubstp{Q}{P} \juxtap (S) \psubstp{Q}{P} \\
  (x?(y).R) \psubstp{Q}{P}    
  :=    
  (x)\substp{Q}{P} (z)\concat( (R \psubstn{z}{y}) \psubstp{Q}{P} ) \\
  (\lift{x}{R}) \psubstp{Q}{P}  
  :=
  \lift{(x)\substp{Q}{P}}{ R \psubstp{Q}{P} } \\
%   (\dropn{x})  \psubstp{Q}{P}       
%   := 
%   \left\{ 
%     \begin{array}{ccc} 
%       \dropn{\quotep{Q}} & & x \nameeq \quotep{P} \\
%       \dropn{x} & & otherwise \\
%     \end{array}
%   \right. 
  (\dropn{x})  \psubstp{Q}{P}       
  := 
  \left\{ 
    \begin{array}{ccc} 
      Q & & x \nameeq \quotep{P} \\
      \dropn{x} & & otherwise \\
    \end{array}
  \right.
\end{mathpar}
 

where

\begin{eqnarray}
  (x)\id{\{} \lpquote Q \rpquote / \lpquote P \rpquote \id{\}}            = 
  \left\{ 
    \begin{array}{ccc}
      \lpquote Q \rpquote & & x \nameeq \lpquote P \rpquote \\
      x & & otherwise \\
    \end{array}
  \right. \nonumber
\end{eqnarray}

and $z$ is chosen distinct from $\quotep{P}$, $\quotep{Q}$, the free
names in $Q$, and all the names in $R$. Our $\alpha$-equivalence will
be built in the standard way from this substitution.

\begin{remark}\label{rem:no_self_referential_names}
  One consequence of these definitions is that $\forall P. \quotep{P}
  \not\in \freenames{P}$.
\end{remark}

\subsection{ Dynamic quote: an example }

Anticipating something of what's to come, consider applying the
substitution, $\widehat{\id{\{}u / z \id{\}}}$, to the following pair
of processes, $\lift{w}{y!(z)}$ and $w[ \lpquote y!(z) \rpquote ]$.

\begin{eqnarray}
	\lift{w}{y!(z)}\widehat{\id{\{}u / z \id{\}}}
		& = &
		\lift{w}{y!(u)} \nonumber\\
	w[ \lpquote y!(z) \rpquote ] \widehat{ \id{\{}u / z \id{\}} }
		& = &
		w[ \lpquote y!(z) \rpquote ] \nonumber
\end{eqnarray}

Because the body of the process between quotes is impervious to
substitution, we get radically different answers. In fact, by
examining the first process in an input context,
e.g. $x?(z).\lift{w}{y!(z)}$, we see that the process under the lift
operator may be shaped by prefixed inputs binding a name inside it. In
this sense, the lift operator will be seen as a way to dynamically
construct processes before reifying them as names.

Finally equipped with these standard features we can present the
dynamics of the calculus.

\subsubsection{Operational semantics} 

Finally, we introduce the computational dynamics. What marks these
algebras as distinct from other more traditionally studied algebraic
structures, e.g. vector spaces or polynomial rings, is the manner in
which dynamics is captured. In traditional structures, dynamics is typically
expressed through morphisms between such structures, as in linear maps
between vector spaces or morphisms between rings. In algebras
associated with the semantics of computation, the dynamics is
expressed as part of the algebraic structure itself, through a
reduction reduction relation typically denoted by $\red$. Below, we
give a recursive presentation of this relation for the calculus used
in the encoding.

$\red \subseteq \pi \times \pi$
$\red : \pi \to \mathcal{P}(\pi)$

\begin{mathpar}
  \inferrule* [lab=Comm] { \textsf{match}( x_{src}, x_{trgt} ) } { x_{trgt}?(y)P \; | \; x_{src}!\langle {Q} \rangle \red P\{\quotep{Q}/y}\} }
  \and \\
  \inferrule* [lab=Par] {{P} \red {P}'} {{{P} | {Q}} \red {{P}' | {Q}}}
  \and
  \inferrule* [lab=Equiv]{{{P} \scong {P}'} \andalso {{P}' \red {Q}'} \andalso {{Q}' \scong {Q}}}{{P} \red {Q}}
\end{mathpar}

\begin{eqnarray*}
  match_{\equiv} (\quotep{P},\quotep{Q}) & := & P \equiv Q \\
  match_{\dagger}(\quotep{P},\quotep{Q}) & := & \forall R. P|Q \red^{*} R => R \red^{*} 0 \\
  match_{K}(\quotep{P},\quotep{Q}) & := & K \mbox{ for some context } K
\end{eqnarray*}

$u?(x)P | u!\langle Q \rangle \red P\{\quotep{Q}/x\}$

%We write $\wred$ for $\red^*$, and $P\red$ if $\exists Q $ such that $ P \red Q$.
We write $P\red$ if $\exists Q $ such that $ P \red Q$ and $P\not\red$, otherwise.

\section{Replication}

As mentioned before, it is known that replication (and hence
recursion) can be implemented in a higher-order process algebra
\cite{SangiorgiWalker}. As our first example of calculation with the
machinery thus far presented we give the construction explicitly in
the {\rhoc}.

\begin{eqnarray}
	D_{x} & := & \prefix{x}{y}{(\binpar{\outputp{x}{y}}{@{y}})} \nonumber\\
	\bangp_{x}{P} & := & \binpar{{x}!\langle{\binpar{D_{x}}{P}}\rangle}{D_{x}} \nonumber
\end{eqnarray}

\begin{eqnarray}
	\bangp_{x}{P} & & \nonumber\\
	=
	& {x}!\langle{(\prefix{x}{y}{(\outputp{x}{y} | @{y})) | P}}\rangle 
	      | \prefix{x}{y}{(\outputp{x}{y} | @{y})} & \nonumber\\
	\red
	& (\outputp{x}{y} | @{y})\substn{\quotep{(\prefix{x}{y}{(@{y} | \outputp{x}{y})) | P}}}{y} & \nonumber\\
	=
	& \outputp{x}{\quotep{(\prefix{x}{y}{(\outputp{x}{y} | @{y})) | P}}}
	  | {(\prefix{x}{y}{(\outputp{x}{y} | @{y})) | P}} & \nonumber\\
	\red
	& \ldots & \nonumber\\
	\red^*
	& P | P | \ldots & \nonumber
\end{eqnarray}

Of course, this encoding, as an implementation, runs away, unfolding
$\bangp{P}$ eagerly. A lazier and more implementable replication
operator, restricted to input-guarded processes, may be obtained as follows.

\begin{eqnarray}
\bangp{\prefix{u}{v}{P}} 
	:= 
	\binpar{\lift{x}{\prefix{u}{v}{(\binpar{D(x)}{P})}}}{D(x)} \nonumber
\end{eqnarray}

\begin{remark}
  Note that the lazier definition still does not deal with summation
  or mixed summation (i.e. sums over input and output). The reader is
  invited to construct definitions of replication that deal with these
  features. 

  Further, the definitions are parameterized in a name, $x$. Can you,
  gentle reader, make a definition that eliminates this parameter and
  guarantees no accidental interaction between the replication
  machinery and the process being replicated -- i.e. no accidental
  sharing of names used by the process to get its work done and the
  name(s) used by the replication to effect copying. This latter
  revision of the definition of replication is crucial to obtaining
  the expected identity $!!P \sim !P$.
\end{remark}

\begin{remark}\label{rem:paradoxical_combinator}
  The reader familiar with the lambda calculus will have noticed the
  similarity between $D$ and the paradoxical combinator.

  [Ed. note: the existence of this seems to suggest we have to be more
  restrictive on the set of processes and names we admit if we are to
  support no-cloning.]
\end{remark}

\subsubsection{Bisimulation}

The computational dynamics gives rise to another kind of equivalence,
the equivalence of computational behavior. As previously mentioned
this is typically captured \emph{via} some form of bisimulation.

% The notion we use in this paper is weak barbed bisimulation
% \cite{milner91polyadicpi}.

The notion we use in this paper is derived from weak barbed
bisimulation \cite{milner91polyadicpi}. 

\begin{definition}
An \emph{observation relation}, $\downarrow_{\mathcal N}$, over a set
of names, $\mathcal N$, is the smallest relation satisfying the rules
below.

\infrule[Out-barb]{y \in {\mathcal N}, \; x \nameeq y}
		  {\outputp{x}{v} \downarrow_{\mathcal N} x}
\infrule[Par-barb]{\mbox{$P\downarrow_{\mathcal N} x$ or $Q\downarrow_{\mathcal N} x$}}
		  {\binpar{P}{Q} \downarrow_{\mathcal N} x}

We write $P \Downarrow_{\mathcal N} x$ if there is $Q$ such that 
$P \wred Q$ and $Q \downarrow_{\mathcal N} x$.
\end{definition}

\begin{definition}
%\label{def.bbisim}
An  ${\mathcal N}$-\emph{barbed bisimulation} over a set of names, ${\mathcal N}$, is a symmetric binary relation 
${\mathcal S}_{\mathcal N}$ between agents such that $P\rel{S}_{\mathcal N}Q$ implies:
\begin{enumerate}
\item If $P \red P'$ then $Q \wred Q'$ and $P'\rel{S}_{\mathcal N} Q'$.
\item If $P\downarrow_{\mathcal N} x$, then $Q\Downarrow_{\mathcal N} x$.
\end{enumerate}
$P$ is ${\mathcal N}$-barbed bisimilar to $Q$, written
$P \wbbisim_{\mathcal N} Q$, if $P \rel{S}_{\mathcal N} Q$ for some ${\mathcal N}$-barbed bisimulation ${\mathcal S}_{\mathcal N}$.
\end{definition}

$\mathcal{R} \subseteq \pi \times \pi$

$P \mathcal{R} Q => \forall P'. P \red P' \Rightarrow \exists Q'. Q \red Q', P' \mathcal{R} Q'$

$P \vdash x \Rightarrow Q \vdash x$

\begin{mathpar}
  \inferrule*[lab=Out-barb]{x \nameeq y}{{y}!\langle{Q}\rangle \vdash x}
  \and
  \inferrule*[lab=Par-barb]{\mbox{$P\vdash x$ or $Q\vdash x$}}{\binpar{P}{Q} \vdash x}
\end{mathpar}

\subsubsection{Contexts}

One of the principle advantages of computational calculi like the
$\pi$-calculus is a well-defined notion of context,
contextual-equivalence and a correlation between
contextual-equivalence and notions of bisimulation. The notion of
context allows the decomposition of a process into (sub-)process and
its syntactic environment, its context. Thus, a context may be
thought of as a process with a ``hole'' (written $\Box$) in it. The
application of a context $M$ to a process $P$, written $M[P]$, is
tantamount to filling the hole in $M$ with $P$. In this paper we do
not need the full weight of this theory, but do make use of the notion
of context in the proof the main theorem. 

\begin{mathpar}
  \inferrule* [lab=summation] {} {{M_{M},M_{N}} \bc \Box \;|\; x.M_{A} \;|\; M_{M}+M_{N}}
  \and
  \inferrule* [lab=agent] {} {{M_{A}} \bc (\vec{x})M_{P} \;| \; \clift{P_0,\ldots,M_{P},\ldots,P_N}}
  \and \\
  \inferrule* [lab=process] {} {{M_{P}} \bc M_{N} \;| \;P|M_{P} }
\end{mathpar} 

\begin{mathpar}
  \inferrule* [lab=sychronization] {} {M_{N} \bc \Box \;|\; x?M_{F} \;|\; x!M_{C}}
  \and
  \inferrule* [lab=abstraction] {} {{M_{F}} \bc (x)M_{P} }
  \and
  \inferrule* [lab=concretion] {} {{M_{C}} \bc \langle M_{P} \rangle }
  \and \\
  \inferrule* [lab=process] {} {{M_{P}} \bc M_{N} \;| \;P|M_{P} }
\end{mathpar}

\begin{definition}[contextual application] Given a context $M$, and
  process $P$, we define the \emph{contextual application}, $M[P] :=
  M\{P/\Box\}$. That is, the contextual application of M to P is the
  substitution of $P$ for $\Box$ in $M$.
\end{definition}

$\meaningof{-} : L \to \mathcal{P}(\pi)$

\begin{mathpar}
  \inferrule* [lab=collection] {} {\meaningof{true} = \pi, \and \meaningof{~E} = \pi \setminus \meaningof{E}, \and \meaningof{E_{1} \& E_{2}} = \meaningof{E_{1}} \cap \meaningof{E_{2}}}
\end{mathpar}

\begin{mathpar}
  \inferrule* [lab=structure] {} {\meaningof{0} = \{ P \in \pi | P \equiv 0 \}, \and \\ \meaningof{E_1 | E_2} = \{ P \in \pi | P \equiv P_{1} | P_{2}, P_{1} \in \meaningof{E_{1}}, P_{2} \in \meaningof{E_2}\} }
\end{mathpar}

\begin{mathpar}
 \inferrule* [lab=behavior] {} {\meaningof{\langle a?b \rangle E} = \{ P \in \pi | P \equiv Q | u?(y)P', \\ \and \\\\ \and \\ \;\;\; u \in \meaningof{a}, \forall z.P'\{z/y\} \in \meaningof{E\{z/b\}}\}, \and \\ \meaningof{a!E} = \{ P \in \pi | P \equiv Q | x!\langle P' \rangle, x \in \meaningof{a} P' \in \meaningof{E}\} }
\end{mathpar}

\begin{mathpar}
 \inferrule* [lab=nominal] {} {\meaningof{\quotep{E}} = \{ \quotep{P} \in \quotep{\pi} | P \in \meaningof{E} \}, \and \meaningof{\quotep{P}} = \{ \quotep{Q} \in \quotep{\pi} | P \equiv Q \} \and \\ \meaningof{@\quotep{E}} = \{ P \in \pi | P \equiv @x, x \in \meaningof{E} \}}
\end{mathpar}

\begin{eqnarray*}
  \\
  \meaningof{-} : TS \to ST
\end{eqnarray*}

\begin{eqnarray*}
  \\
  L : TS \to ST
\end{eqnarray*}

\begin{eqnarray*}
  \\
  P \models E \iff P \in \meaningof{E}
\end{eqnarray*}

\begin{eqnarray*}
  P \approx_{L} Q \iff \forall E \in L. P \models E \iff Q \models E
\end{eqnarray*}

\begin{eqnarray*}
  P \approx_{K} Q
\end{eqnarray*}

\begin{eqnarray*}
  P \approx Q
\end{eqnarray*}

$\approx_{K} = \approx = \approx_{L}$

\subsubsection{Contextual duality}

Note that contexts extend the quotation operation to a family of
operations from processes to names. Given a context, $M$, we can
define a \emph{nominal context}, $\quotep{M}$ by $\quotep{M}[P] :=
\quotep{M[P]}$. To foreshadow what is to come we observe that these
operations enjoy a duality with processes very much like the duality
between vectors and maps from vectors to scalars.

Further, because the calculus is essentially higher-order, we have a
correspondence between contexts and processes. More specifically,
given a name $x$ and a context $M$ we can construct $M^{*}_{x}$ such
that 

\begin{mathpar}
  M^{*}_{x} | \lift{x}{P} \red M[P]
\end{mathpar}

namely,

\begin{mathpar}
  M^{*}_{x} := x?(u).M[\dropn{u}]
\end{mathpar}

The dependence of $M^{*}_{x}$ on a name makes it an abstraction, 

\begin{mathpar}
  M^{*} := (x)x?(u).M[\dropn{u}]
\end{mathpar}

\subsection{Additional notation}

It will sometimes be convenient to denote the process a name
quotes. We already have the notation $x = \quotep{P}$, but it will be
convenient to introduce an alternate notation, $\procn{x}$, when we
want to emphasize the connection to the use of the name. Note that, by
virtue of name equivalence, $\quotep{\procn{x}} \nameeq x$; so, the
notation is consistent with previous definitions.

Further, because names have structure it is possible to effect
substitutions on the basis of that structure. This means we need to
upgrade our notation for substitutions, which we accomplish by
adapting comprehension notation. Thus,

\begin{mathpar}
  P\{ y / x : x \in S \}
\end{mathpar}

is interpreted to mean the process derived from P by replacing (in a
capture-avoiding manner) each occurrence of $x$ in $S$ by $y$. For example,

\begin{mathpar}
  P\{ \quotep{\procn{x}|\procn{x}} / x : x \in \freenames{P} \}
\end{mathpar}

will replace each (occurrence) of a free name $x$ in $P$ by
$\quotep{\procn{x}|\procn{x}}$.

Also, we will avail ourselves of the notation $x^{L}$ and $x^{R}$ to
denote injections of a name into disjoint copies of the name
space. There are numerous ways to accomplish this. One example can be
found in \cite{MeredithR05}. This notation overloads to vectors of
names: $\vec{x}^{\pi} := (x_{i}^{\pi} \; : \; 0 \leq i < |\vec{x}| )$ where $\pi \in \{L,R\}$.

We also use $P^{\Box} := P|\Box$.

In \cite{MeredithR05} an interpretation of the new operator is
given. It turns out that there are several possible interpretations
all enjoying the requisite algebraic properties of the operator (see
\cite{milner91polyadicpi}). We will therefore make liberal use of
$(\nu\; \vec{x})P$.

% subsection the_syntax_and_semantics_of_the_notation_system (end)   

\input{qm2pi.qmops} 

\input{qm2pi.sterngerlach} 

\input{qm2pi.metric} 

% section concurrent_process_calculi (end)

%\input{qm2pi.proofsketch}

% section proof sketch (end)

%\input{qm2pi.slviaknots} 

% section spatial logic via knots (end)

\input{qm2pi.conclusion}

% section conclusion (end)

%\input{qm2pi.dtcodes} 

% section wiring algorithm (end)

\input{qm2pi.ack} 

% section acknowledgments (end)

\newpage


\bibliographystyle{plain}   
\bibliography{../../biblios/main.bib}

\input{qm2pi.rhodetails}

\end{document}



\end{document}



\end{document}

 

% section acknowledgments (end)

\newpage


\bibliographystyle{plain}   
\bibliography{../../biblios/main.bib}

\documentclass[12pt]{llncs}
%\documentclass{jktr}

\usepackage[pdftex]{hyperref}                   
\usepackage {listings}
\usepackage {mathpartir}
\usepackage{bcprules}
%\usepackage{listings}
                       
\usepackage{graphicx} 
%\usepackage[margins=2.5cm,nohead,nofoot]{geometry}
%\usepackage{geometry}
\usepackage{amsfonts}
\usepackage{amstext}
\usepackage{latexsym}
\usepackage{amssymb}
\usepackage{color}


%\include{myPreamble}
\documentclass[12pt]{llncs}
%\documentclass{jktr}

\usepackage[pdftex]{hyperref}                   
\usepackage {listings}
\usepackage {mathpartir}
\usepackage{bcprules}
%\usepackage{listings}
                       
\usepackage{graphicx} 
%\usepackage[margins=2.5cm,nohead,nofoot]{geometry}
%\usepackage{geometry}
\usepackage{amsfonts}
\usepackage{amstext}
\usepackage{latexsym}
\usepackage{amssymb}
\usepackage{color}


%\include{myPreamble}
\documentclass[12pt]{llncs}
%\documentclass{jktr}

\usepackage[pdftex]{hyperref}                   
\usepackage {listings}
\usepackage {mathpartir}
\usepackage{bcprules}
%\usepackage{listings}
                       
\usepackage{graphicx} 
%\usepackage[margins=2.5cm,nohead,nofoot]{geometry}
%\usepackage{geometry}
\usepackage{amsfonts}
\usepackage{amstext}
\usepackage{latexsym}
\usepackage{amssymb}
\usepackage{color}


%\include{myPreamble}
\include{qm2pi.local} 

%\ifpdf
%\usepackage[pdftex]{graphicx}
%\else
%\usepackage{graphicx}
%\fi

 % \ifpdf
%  \usepackage{pdfsync}
%  \if


%\title{Brief Article}
%\author{David F. Snyder}
%\author{L.G. Meredith}

%\address{Dept. of Math., Texas State University--San Marcos, San Marcos, TX 78666}
       
\pagestyle{empty}


\begin{document}

\lstset{language=[Objective]Caml,frame=shadowbox}

\input{qm2pi.front}

% section front matter (end)

\input{qm2pi.intro} 
 
% section introduction (end)

% \input{qm2pi.knotations} 

% section notation (end)

\input{qm2pi.process.calculi} 

% section concurrent_process_calculi_and_spatial_logics_ (end)
    
%\input{qm2pi.knots2pi} 

%\input{qm2pi.trefoil} 

%\input{qm2pi.mainthm} 

% subsection basic_interpretation (end)

%\input{qm2pi.rho.presentation} 
\subsection{The syntax and semantics of the notation system}\label{sub:the_syntax_and_semantics_of_the_notation_system} % (fold)

We now summarize a technical presentation of the calculus that
embodies our theory of dynamics. The typical presentation of such a
calculus follows the style of giving generators and relations on
them. The grammar, below, describing term constructors, freely
generates the set of processes, $\Proc$. This set is then quotiented
by a relation known as structural congruence and it is over this set
that the notion of dynamics is expressed. This presentation is
essentially that of \cite{MeredithR05} with the addition of
polyadicity and summation. For readability we have relegated some of
the technical subtleties to an appendix.

\subsubsection{Process grammar}\label{subsub:process_grammar}

\begin{mathpar}
  \inferrule* [lab=synchronization] {} {{M} \bc \pzero \;|\; x?F \;|\; x!C }
  \and
  \inferrule* [lab=abstraction] {} {{F} \bc (x)P}
  \and
  \inferrule* [lab=concretion] {} {{C} \bc \langle Q \rangle}
  \and
  \inferrule* [lab=process] {} {{P,Q} \bc M \;| \;P|Q \;|\; @{x}}
  \and
  \inferrule* [lab=name] {} {{x} \bc \quotep{P}}
\end{mathpar} 

Note that $\vec{x}$ (resp. $\vec{P}$) denotes a vector of names
(resp. processes) of length $|\vec{x}|$ (resp. $|\vec{P}|$). We adopt
the following useful abbreviations.

\begin{mathpar}
   x?(\vec{y}).P := x.(\vec{y})P \and  x\clift{\vec{P}} := x.\clift{\vec{P}}
   \and x!(y) := \lift{x}{\dropn{y}}
   \and \Pi_{i=0}^{n-1}P_i := P_0 | \ldots | P_{n-1}
\end{mathpar}

\subsubsection{Structural congruence}

\paragraph{Free and bound names and alpha-equivalence.} At the
core of structural equivalence is alpha-equivalence which identifies
process that are the same up to a change of variable. Formally, we
recognize the distinction between free and bound names. The free names
of a process, $\freenames{P}$, may be calculated recursively as
follows:

\begin{mathpar}
\freenames{\pzero} := \emptyset
  \and \\
  \freenames{x?(y).P} := \{ x \} \cup (\freenames{P} \setminus \{ y \})
  \and 
  \freenames{x!\langle P \rangle} := \{ x \} \cup \{ P \} 
  \and \\
  \freenames{P|Q} := \freenames{P} \cup \freenames{Q}
  \and \\
  \freenames{@{x}} := \{ x \}
\end{mathpar}

$\pi$
$\quotep{\pi}$

$\freenames{-} : \pi \to \mathcal{P}(\quotep{\pi})$

\begin{eqnarray*}
  \freenames{\pzero} & := & \emptyset \\
  \freenames{x?(y).P} & := & \{ x \} \cup (\freenames{P} \setminus \{ y \}) \\
  \freenames{x!\langle P \rangle} & := & \{ x \} \cup \{ P \} \\
  \freenames{P|Q} & := & \freenames{P} \cup \freenames{Q} \\
  \freenames{\dropn{x}} & := & \{ x \}
\end{eqnarray*}

The bound names of a process, $\boundnames{P}$, are those names occurring in $P$
that are not free. For example, in $x?(y).0$, the name $x$ is free, while $y$ is bound.

\begin{mathpar}
  \inferrule* [lab=monoidal-laws] {} { P|Q \equiv Q|P \and P|0 \equiv P \and P|(Q|R) \equiv (P|Q)|R }
\end{mathpar}

\begin{mathpar}
  \inferrule* [lab=alpha-equivalence] {} { (x)P \equiv (y)P\{y/x\} \and y \not\in \freenames{P} }
\end{mathpar}

\begin{definition}
Then two processes, $P,Q$, are alpha-equivalent if $P = Q\{\vec{y}/\vec{x}\}$ for
some $\vec{x} \in \boundnames{Q},\vec{y} \in \boundnames{P}$, where $Q\{\vec{y}/\vec{x}\}$
denotes the capture-avoiding substitution of $\vec{y}$ for $\vec{x}$ in $Q$.
\end{definition}

\begin{definition}
  The {\em structural congruence} \cite{SangiorgiWalker} , $\equiv$,
  between processes is the least congruence containing
  alpha-equivalence, satisfying the abelian monoid laws
  (associativity, commutativity and $\pzero$ as identity) for parallel
  composition $|$ and for summation $+$.
\end{definition}

\subsection{Name equivalence}

We take name equivalence, written $\nameeq$, to be the smallest
equivalence relation generated by the following rules.

\begin{mathpar}
\inferrule*[lab=Quote-drop]
{ }
{ \quotep{@{x}} \nameeq x }

\inferrule*[lab=Struct-equiv]
{ P \scong Q }
{ \quotep{P} \nameeq \quotep{Q} }
\end{mathpar}

The astute reader will have noticed that the mutual recursion of names
and processes imposes a mutual recursion on alpha-equivalence and
structural equivalence via name-equivalence. Fortunately, all of this
works out pleasantly and we may calculate in the natural way, free of
concern. The reader interested in the details is referred to the
appendix \ref{appendix:rho_details}.

\subsection{Substitution}

We use $\Proc$ for the set of processes, $\QProc$ for the set of
names, and $\id{\{}\vec{y} / \vec{x} \id{\}}$ to denote partial maps,
$s : \QProc \rightarrow \QProc$. A map, $s$ lifts, uniquely, to a map
on process terms, $\widehat{s} : \Proc \rightarrow \Proc$ by the
following equations.

\begin{mathpar}
  (0) \psubstp{Q}{P} := 0 \\
  (R \juxtap S) \psubstp{Q}{P}
  :=    
  (R)\psubstp{Q}{P} \juxtap (S) \psubstp{Q}{P} \\
  (x?(y).R) \psubstp{Q}{P}    
  :=    
  (x)\substp{Q}{P} (z)\concat( (R \psubstn{z}{y}) \psubstp{Q}{P} ) \\
  (\lift{x}{R}) \psubstp{Q}{P}  
  :=
  \lift{(x)\substp{Q}{P}}{ R \psubstp{Q}{P} } \\
%   (\dropn{x})  \psubstp{Q}{P}       
%   := 
%   \left\{ 
%     \begin{array}{ccc} 
%       \dropn{\quotep{Q}} & & x \nameeq \quotep{P} \\
%       \dropn{x} & & otherwise \\
%     \end{array}
%   \right. 
  (\dropn{x})  \psubstp{Q}{P}       
  := 
  \left\{ 
    \begin{array}{ccc} 
      Q & & x \nameeq \quotep{P} \\
      \dropn{x} & & otherwise \\
    \end{array}
  \right.
\end{mathpar}
 

where

\begin{eqnarray}
  (x)\id{\{} \lpquote Q \rpquote / \lpquote P \rpquote \id{\}}            = 
  \left\{ 
    \begin{array}{ccc}
      \lpquote Q \rpquote & & x \nameeq \lpquote P \rpquote \\
      x & & otherwise \\
    \end{array}
  \right. \nonumber
\end{eqnarray}

and $z$ is chosen distinct from $\quotep{P}$, $\quotep{Q}$, the free
names in $Q$, and all the names in $R$. Our $\alpha$-equivalence will
be built in the standard way from this substitution.

\begin{remark}\label{rem:no_self_referential_names}
  One consequence of these definitions is that $\forall P. \quotep{P}
  \not\in \freenames{P}$.
\end{remark}

\subsection{ Dynamic quote: an example }

Anticipating something of what's to come, consider applying the
substitution, $\widehat{\id{\{}u / z \id{\}}}$, to the following pair
of processes, $\lift{w}{y!(z)}$ and $w[ \lpquote y!(z) \rpquote ]$.

\begin{eqnarray}
	\lift{w}{y!(z)}\widehat{\id{\{}u / z \id{\}}}
		& = &
		\lift{w}{y!(u)} \nonumber\\
	w[ \lpquote y!(z) \rpquote ] \widehat{ \id{\{}u / z \id{\}} }
		& = &
		w[ \lpquote y!(z) \rpquote ] \nonumber
\end{eqnarray}

Because the body of the process between quotes is impervious to
substitution, we get radically different answers. In fact, by
examining the first process in an input context,
e.g. $x?(z).\lift{w}{y!(z)}$, we see that the process under the lift
operator may be shaped by prefixed inputs binding a name inside it. In
this sense, the lift operator will be seen as a way to dynamically
construct processes before reifying them as names.

Finally equipped with these standard features we can present the
dynamics of the calculus.

\subsubsection{Operational semantics} 

Finally, we introduce the computational dynamics. What marks these
algebras as distinct from other more traditionally studied algebraic
structures, e.g. vector spaces or polynomial rings, is the manner in
which dynamics is captured. In traditional structures, dynamics is typically
expressed through morphisms between such structures, as in linear maps
between vector spaces or morphisms between rings. In algebras
associated with the semantics of computation, the dynamics is
expressed as part of the algebraic structure itself, through a
reduction reduction relation typically denoted by $\red$. Below, we
give a recursive presentation of this relation for the calculus used
in the encoding.

$\red \subseteq \pi \times \pi$
$\red : \pi \to \mathcal{P}(\pi)$

\begin{mathpar}
  \inferrule* [lab=Comm] { \textsf{match}( x_{src}, x_{trgt} ) } { x_{trgt}?(y)P \; | \; x_{src}!\langle {Q} \rangle \red P\{\quotep{Q}/y}\} }
  \and \\
  \inferrule* [lab=Par] {{P} \red {P}'} {{{P} | {Q}} \red {{P}' | {Q}}}
  \and
  \inferrule* [lab=Equiv]{{{P} \scong {P}'} \andalso {{P}' \red {Q}'} \andalso {{Q}' \scong {Q}}}{{P} \red {Q}}
\end{mathpar}

\begin{eqnarray*}
  match_{\equiv} (\quotep{P},\quotep{Q}) & := & P \equiv Q \\
  match_{\dagger}(\quotep{P},\quotep{Q}) & := & \forall R. P|Q \red^{*} R => R \red^{*} 0 \\
  match_{K}(\quotep{P},\quotep{Q}) & := & K \mbox{ for some context } K
\end{eqnarray*}

$u?(x)P | u!\langle Q \rangle \red P\{\quotep{Q}/x\}$

%We write $\wred$ for $\red^*$, and $P\red$ if $\exists Q $ such that $ P \red Q$.
We write $P\red$ if $\exists Q $ such that $ P \red Q$ and $P\not\red$, otherwise.

\section{Replication}

As mentioned before, it is known that replication (and hence
recursion) can be implemented in a higher-order process algebra
\cite{SangiorgiWalker}. As our first example of calculation with the
machinery thus far presented we give the construction explicitly in
the {\rhoc}.

\begin{eqnarray}
	D_{x} & := & \prefix{x}{y}{(\binpar{\outputp{x}{y}}{@{y}})} \nonumber\\
	\bangp_{x}{P} & := & \binpar{{x}!\langle{\binpar{D_{x}}{P}}\rangle}{D_{x}} \nonumber
\end{eqnarray}

\begin{eqnarray}
	\bangp_{x}{P} & & \nonumber\\
	=
	& {x}!\langle{(\prefix{x}{y}{(\outputp{x}{y} | @{y})) | P}}\rangle 
	      | \prefix{x}{y}{(\outputp{x}{y} | @{y})} & \nonumber\\
	\red
	& (\outputp{x}{y} | @{y})\substn{\quotep{(\prefix{x}{y}{(@{y} | \outputp{x}{y})) | P}}}{y} & \nonumber\\
	=
	& \outputp{x}{\quotep{(\prefix{x}{y}{(\outputp{x}{y} | @{y})) | P}}}
	  | {(\prefix{x}{y}{(\outputp{x}{y} | @{y})) | P}} & \nonumber\\
	\red
	& \ldots & \nonumber\\
	\red^*
	& P | P | \ldots & \nonumber
\end{eqnarray}

Of course, this encoding, as an implementation, runs away, unfolding
$\bangp{P}$ eagerly. A lazier and more implementable replication
operator, restricted to input-guarded processes, may be obtained as follows.

\begin{eqnarray}
\bangp{\prefix{u}{v}{P}} 
	:= 
	\binpar{\lift{x}{\prefix{u}{v}{(\binpar{D(x)}{P})}}}{D(x)} \nonumber
\end{eqnarray}

\begin{remark}
  Note that the lazier definition still does not deal with summation
  or mixed summation (i.e. sums over input and output). The reader is
  invited to construct definitions of replication that deal with these
  features. 

  Further, the definitions are parameterized in a name, $x$. Can you,
  gentle reader, make a definition that eliminates this parameter and
  guarantees no accidental interaction between the replication
  machinery and the process being replicated -- i.e. no accidental
  sharing of names used by the process to get its work done and the
  name(s) used by the replication to effect copying. This latter
  revision of the definition of replication is crucial to obtaining
  the expected identity $!!P \sim !P$.
\end{remark}

\begin{remark}\label{rem:paradoxical_combinator}
  The reader familiar with the lambda calculus will have noticed the
  similarity between $D$ and the paradoxical combinator.

  [Ed. note: the existence of this seems to suggest we have to be more
  restrictive on the set of processes and names we admit if we are to
  support no-cloning.]
\end{remark}

\subsubsection{Bisimulation}

The computational dynamics gives rise to another kind of equivalence,
the equivalence of computational behavior. As previously mentioned
this is typically captured \emph{via} some form of bisimulation.

% The notion we use in this paper is weak barbed bisimulation
% \cite{milner91polyadicpi}.

The notion we use in this paper is derived from weak barbed
bisimulation \cite{milner91polyadicpi}. 

\begin{definition}
An \emph{observation relation}, $\downarrow_{\mathcal N}$, over a set
of names, $\mathcal N$, is the smallest relation satisfying the rules
below.

\infrule[Out-barb]{y \in {\mathcal N}, \; x \nameeq y}
		  {\outputp{x}{v} \downarrow_{\mathcal N} x}
\infrule[Par-barb]{\mbox{$P\downarrow_{\mathcal N} x$ or $Q\downarrow_{\mathcal N} x$}}
		  {\binpar{P}{Q} \downarrow_{\mathcal N} x}

We write $P \Downarrow_{\mathcal N} x$ if there is $Q$ such that 
$P \wred Q$ and $Q \downarrow_{\mathcal N} x$.
\end{definition}

\begin{definition}
%\label{def.bbisim}
An  ${\mathcal N}$-\emph{barbed bisimulation} over a set of names, ${\mathcal N}$, is a symmetric binary relation 
${\mathcal S}_{\mathcal N}$ between agents such that $P\rel{S}_{\mathcal N}Q$ implies:
\begin{enumerate}
\item If $P \red P'$ then $Q \wred Q'$ and $P'\rel{S}_{\mathcal N} Q'$.
\item If $P\downarrow_{\mathcal N} x$, then $Q\Downarrow_{\mathcal N} x$.
\end{enumerate}
$P$ is ${\mathcal N}$-barbed bisimilar to $Q$, written
$P \wbbisim_{\mathcal N} Q$, if $P \rel{S}_{\mathcal N} Q$ for some ${\mathcal N}$-barbed bisimulation ${\mathcal S}_{\mathcal N}$.
\end{definition}

$\mathcal{R} \subseteq \pi \times \pi$

$P \mathcal{R} Q => \forall P'. P \red P' \Rightarrow \exists Q'. Q \red Q', P' \mathcal{R} Q'$

$P \vdash x \Rightarrow Q \vdash x$

\begin{mathpar}
  \inferrule*[lab=Out-barb]{x \nameeq y}{{y}!\langle{Q}\rangle \vdash x}
  \and
  \inferrule*[lab=Par-barb]{\mbox{$P\vdash x$ or $Q\vdash x$}}{\binpar{P}{Q} \vdash x}
\end{mathpar}

\subsubsection{Contexts}

One of the principle advantages of computational calculi like the
$\pi$-calculus is a well-defined notion of context,
contextual-equivalence and a correlation between
contextual-equivalence and notions of bisimulation. The notion of
context allows the decomposition of a process into (sub-)process and
its syntactic environment, its context. Thus, a context may be
thought of as a process with a ``hole'' (written $\Box$) in it. The
application of a context $M$ to a process $P$, written $M[P]$, is
tantamount to filling the hole in $M$ with $P$. In this paper we do
not need the full weight of this theory, but do make use of the notion
of context in the proof the main theorem. 

\begin{mathpar}
  \inferrule* [lab=summation] {} {{M_{M},M_{N}} \bc \Box \;|\; x.M_{A} \;|\; M_{M}+M_{N}}
  \and
  \inferrule* [lab=agent] {} {{M_{A}} \bc (\vec{x})M_{P} \;| \; \clift{P_0,\ldots,M_{P},\ldots,P_N}}
  \and \\
  \inferrule* [lab=process] {} {{M_{P}} \bc M_{N} \;| \;P|M_{P} }
\end{mathpar} 

\begin{mathpar}
  \inferrule* [lab=sychronization] {} {M_{N} \bc \Box \;|\; x?M_{F} \;|\; x!M_{C}}
  \and
  \inferrule* [lab=abstraction] {} {{M_{F}} \bc (x)M_{P} }
  \and
  \inferrule* [lab=concretion] {} {{M_{C}} \bc \langle M_{P} \rangle }
  \and \\
  \inferrule* [lab=process] {} {{M_{P}} \bc M_{N} \;| \;P|M_{P} }
\end{mathpar}

\begin{definition}[contextual application] Given a context $M$, and
  process $P$, we define the \emph{contextual application}, $M[P] :=
  M\{P/\Box\}$. That is, the contextual application of M to P is the
  substitution of $P$ for $\Box$ in $M$.
\end{definition}

$\meaningof{-} : L \to \mathcal{P}(\pi)$

\begin{mathpar}
  \inferrule* [lab=collection] {} {\meaningof{true} = \pi, \and \meaningof{~E} = \pi \setminus \meaningof{E}, \and \meaningof{E_{1} \& E_{2}} = \meaningof{E_{1}} \cap \meaningof{E_{2}}}
\end{mathpar}

\begin{mathpar}
  \inferrule* [lab=structure] {} {\meaningof{0} = \{ P \in \pi | P \equiv 0 \}, \and \\ \meaningof{E_1 | E_2} = \{ P \in \pi | P \equiv P_{1} | P_{2}, P_{1} \in \meaningof{E_{1}}, P_{2} \in \meaningof{E_2}\} }
\end{mathpar}

\begin{mathpar}
 \inferrule* [lab=behavior] {} {\meaningof{\langle a?b \rangle E} = \{ P \in \pi | P \equiv Q | u?(y)P', \\ \and \\\\ \and \\ \;\;\; u \in \meaningof{a}, \forall z.P'\{z/y\} \in \meaningof{E\{z/b\}}\}, \and \\ \meaningof{a!E} = \{ P \in \pi | P \equiv Q | x!\langle P' \rangle, x \in \meaningof{a} P' \in \meaningof{E}\} }
\end{mathpar}

\begin{mathpar}
 \inferrule* [lab=nominal] {} {\meaningof{\quotep{E}} = \{ \quotep{P} \in \quotep{\pi} | P \in \meaningof{E} \}, \and \meaningof{\quotep{P}} = \{ \quotep{Q} \in \quotep{\pi} | P \equiv Q \} \and \\ \meaningof{@\quotep{E}} = \{ P \in \pi | P \equiv @x, x \in \meaningof{E} \}}
\end{mathpar}

\begin{eqnarray*}
  \\
  \meaningof{-} : TS \to ST
\end{eqnarray*}

\begin{eqnarray*}
  \\
  L : TS \to ST
\end{eqnarray*}

\begin{eqnarray*}
  \\
  P \models E \iff P \in \meaningof{E}
\end{eqnarray*}

\begin{eqnarray*}
  P \approx_{L} Q \iff \forall E \in L. P \models E \iff Q \models E
\end{eqnarray*}

\begin{eqnarray*}
  P \approx_{K} Q
\end{eqnarray*}

\begin{eqnarray*}
  P \approx Q
\end{eqnarray*}

$\approx_{K} = \approx = \approx_{L}$

\subsubsection{Contextual duality}

Note that contexts extend the quotation operation to a family of
operations from processes to names. Given a context, $M$, we can
define a \emph{nominal context}, $\quotep{M}$ by $\quotep{M}[P] :=
\quotep{M[P]}$. To foreshadow what is to come we observe that these
operations enjoy a duality with processes very much like the duality
between vectors and maps from vectors to scalars.

Further, because the calculus is essentially higher-order, we have a
correspondence between contexts and processes. More specifically,
given a name $x$ and a context $M$ we can construct $M^{*}_{x}$ such
that 

\begin{mathpar}
  M^{*}_{x} | \lift{x}{P} \red M[P]
\end{mathpar}

namely,

\begin{mathpar}
  M^{*}_{x} := x?(u).M[\dropn{u}]
\end{mathpar}

The dependence of $M^{*}_{x}$ on a name makes it an abstraction, 

\begin{mathpar}
  M^{*} := (x)x?(u).M[\dropn{u}]
\end{mathpar}

\subsection{Additional notation}

It will sometimes be convenient to denote the process a name
quotes. We already have the notation $x = \quotep{P}$, but it will be
convenient to introduce an alternate notation, $\procn{x}$, when we
want to emphasize the connection to the use of the name. Note that, by
virtue of name equivalence, $\quotep{\procn{x}} \nameeq x$; so, the
notation is consistent with previous definitions.

Further, because names have structure it is possible to effect
substitutions on the basis of that structure. This means we need to
upgrade our notation for substitutions, which we accomplish by
adapting comprehension notation. Thus,

\begin{mathpar}
  P\{ y / x : x \in S \}
\end{mathpar}

is interpreted to mean the process derived from P by replacing (in a
capture-avoiding manner) each occurrence of $x$ in $S$ by $y$. For example,

\begin{mathpar}
  P\{ \quotep{\procn{x}|\procn{x}} / x : x \in \freenames{P} \}
\end{mathpar}

will replace each (occurrence) of a free name $x$ in $P$ by
$\quotep{\procn{x}|\procn{x}}$.

Also, we will avail ourselves of the notation $x^{L}$ and $x^{R}$ to
denote injections of a name into disjoint copies of the name
space. There are numerous ways to accomplish this. One example can be
found in \cite{MeredithR05}. This notation overloads to vectors of
names: $\vec{x}^{\pi} := (x_{i}^{\pi} \; : \; 0 \leq i < |\vec{x}| )$ where $\pi \in \{L,R\}$.

We also use $P^{\Box} := P|\Box$.

In \cite{MeredithR05} an interpretation of the new operator is
given. It turns out that there are several possible interpretations
all enjoying the requisite algebraic properties of the operator (see
\cite{milner91polyadicpi}). We will therefore make liberal use of
$(\nu\; \vec{x})P$.

% subsection the_syntax_and_semantics_of_the_notation_system (end)   

\input{qm2pi.qmops} 

\input{qm2pi.sterngerlach} 

\input{qm2pi.metric} 

% section concurrent_process_calculi (end)

%\input{qm2pi.proofsketch}

% section proof sketch (end)

%\input{qm2pi.slviaknots} 

% section spatial logic via knots (end)

\input{qm2pi.conclusion}

% section conclusion (end)

%\input{qm2pi.dtcodes} 

% section wiring algorithm (end)

\input{qm2pi.ack} 

% section acknowledgments (end)

\newpage


\bibliographystyle{plain}   
\bibliography{../../biblios/main.bib}

\input{qm2pi.rhodetails}

\end{document}

 

%\ifpdf
%\usepackage[pdftex]{graphicx}
%\else
%\usepackage{graphicx}
%\fi

 % \ifpdf
%  \usepackage{pdfsync}
%  \if


%\title{Brief Article}
%\author{David F. Snyder}
%\author{L.G. Meredith}

%\address{Dept. of Math., Texas State University--San Marcos, San Marcos, TX 78666}
       
\pagestyle{empty}


\begin{document}

\lstset{language=[Objective]Caml,frame=shadowbox}

\documentclass[12pt]{llncs}
%\documentclass{jktr}

\usepackage[pdftex]{hyperref}                   
\usepackage {listings}
\usepackage {mathpartir}
\usepackage{bcprules}
%\usepackage{listings}
                       
\usepackage{graphicx} 
%\usepackage[margins=2.5cm,nohead,nofoot]{geometry}
%\usepackage{geometry}
\usepackage{amsfonts}
\usepackage{amstext}
\usepackage{latexsym}
\usepackage{amssymb}
\usepackage{color}


%\include{myPreamble}
\include{qm2pi.local} 

%\ifpdf
%\usepackage[pdftex]{graphicx}
%\else
%\usepackage{graphicx}
%\fi

 % \ifpdf
%  \usepackage{pdfsync}
%  \if


%\title{Brief Article}
%\author{David F. Snyder}
%\author{L.G. Meredith}

%\address{Dept. of Math., Texas State University--San Marcos, San Marcos, TX 78666}
       
\pagestyle{empty}


\begin{document}

\lstset{language=[Objective]Caml,frame=shadowbox}

\input{qm2pi.front}

% section front matter (end)

\input{qm2pi.intro} 
 
% section introduction (end)

% \input{qm2pi.knotations} 

% section notation (end)

\input{qm2pi.process.calculi} 

% section concurrent_process_calculi_and_spatial_logics_ (end)
    
%\input{qm2pi.knots2pi} 

%\input{qm2pi.trefoil} 

%\input{qm2pi.mainthm} 

% subsection basic_interpretation (end)

%\input{qm2pi.rho.presentation} 
\subsection{The syntax and semantics of the notation system}\label{sub:the_syntax_and_semantics_of_the_notation_system} % (fold)

We now summarize a technical presentation of the calculus that
embodies our theory of dynamics. The typical presentation of such a
calculus follows the style of giving generators and relations on
them. The grammar, below, describing term constructors, freely
generates the set of processes, $\Proc$. This set is then quotiented
by a relation known as structural congruence and it is over this set
that the notion of dynamics is expressed. This presentation is
essentially that of \cite{MeredithR05} with the addition of
polyadicity and summation. For readability we have relegated some of
the technical subtleties to an appendix.

\subsubsection{Process grammar}\label{subsub:process_grammar}

\begin{mathpar}
  \inferrule* [lab=synchronization] {} {{M} \bc \pzero \;|\; x?F \;|\; x!C }
  \and
  \inferrule* [lab=abstraction] {} {{F} \bc (x)P}
  \and
  \inferrule* [lab=concretion] {} {{C} \bc \langle Q \rangle}
  \and
  \inferrule* [lab=process] {} {{P,Q} \bc M \;| \;P|Q \;|\; @{x}}
  \and
  \inferrule* [lab=name] {} {{x} \bc \quotep{P}}
\end{mathpar} 

Note that $\vec{x}$ (resp. $\vec{P}$) denotes a vector of names
(resp. processes) of length $|\vec{x}|$ (resp. $|\vec{P}|$). We adopt
the following useful abbreviations.

\begin{mathpar}
   x?(\vec{y}).P := x.(\vec{y})P \and  x\clift{\vec{P}} := x.\clift{\vec{P}}
   \and x!(y) := \lift{x}{\dropn{y}}
   \and \Pi_{i=0}^{n-1}P_i := P_0 | \ldots | P_{n-1}
\end{mathpar}

\subsubsection{Structural congruence}

\paragraph{Free and bound names and alpha-equivalence.} At the
core of structural equivalence is alpha-equivalence which identifies
process that are the same up to a change of variable. Formally, we
recognize the distinction between free and bound names. The free names
of a process, $\freenames{P}$, may be calculated recursively as
follows:

\begin{mathpar}
\freenames{\pzero} := \emptyset
  \and \\
  \freenames{x?(y).P} := \{ x \} \cup (\freenames{P} \setminus \{ y \})
  \and 
  \freenames{x!\langle P \rangle} := \{ x \} \cup \{ P \} 
  \and \\
  \freenames{P|Q} := \freenames{P} \cup \freenames{Q}
  \and \\
  \freenames{@{x}} := \{ x \}
\end{mathpar}

$\pi$
$\quotep{\pi}$

$\freenames{-} : \pi \to \mathcal{P}(\quotep{\pi})$

\begin{eqnarray*}
  \freenames{\pzero} & := & \emptyset \\
  \freenames{x?(y).P} & := & \{ x \} \cup (\freenames{P} \setminus \{ y \}) \\
  \freenames{x!\langle P \rangle} & := & \{ x \} \cup \{ P \} \\
  \freenames{P|Q} & := & \freenames{P} \cup \freenames{Q} \\
  \freenames{\dropn{x}} & := & \{ x \}
\end{eqnarray*}

The bound names of a process, $\boundnames{P}$, are those names occurring in $P$
that are not free. For example, in $x?(y).0$, the name $x$ is free, while $y$ is bound.

\begin{mathpar}
  \inferrule* [lab=monoidal-laws] {} { P|Q \equiv Q|P \and P|0 \equiv P \and P|(Q|R) \equiv (P|Q)|R }
\end{mathpar}

\begin{mathpar}
  \inferrule* [lab=alpha-equivalence] {} { (x)P \equiv (y)P\{y/x\} \and y \not\in \freenames{P} }
\end{mathpar}

\begin{definition}
Then two processes, $P,Q$, are alpha-equivalent if $P = Q\{\vec{y}/\vec{x}\}$ for
some $\vec{x} \in \boundnames{Q},\vec{y} \in \boundnames{P}$, where $Q\{\vec{y}/\vec{x}\}$
denotes the capture-avoiding substitution of $\vec{y}$ for $\vec{x}$ in $Q$.
\end{definition}

\begin{definition}
  The {\em structural congruence} \cite{SangiorgiWalker} , $\equiv$,
  between processes is the least congruence containing
  alpha-equivalence, satisfying the abelian monoid laws
  (associativity, commutativity and $\pzero$ as identity) for parallel
  composition $|$ and for summation $+$.
\end{definition}

\subsection{Name equivalence}

We take name equivalence, written $\nameeq$, to be the smallest
equivalence relation generated by the following rules.

\begin{mathpar}
\inferrule*[lab=Quote-drop]
{ }
{ \quotep{@{x}} \nameeq x }

\inferrule*[lab=Struct-equiv]
{ P \scong Q }
{ \quotep{P} \nameeq \quotep{Q} }
\end{mathpar}

The astute reader will have noticed that the mutual recursion of names
and processes imposes a mutual recursion on alpha-equivalence and
structural equivalence via name-equivalence. Fortunately, all of this
works out pleasantly and we may calculate in the natural way, free of
concern. The reader interested in the details is referred to the
appendix \ref{appendix:rho_details}.

\subsection{Substitution}

We use $\Proc$ for the set of processes, $\QProc$ for the set of
names, and $\id{\{}\vec{y} / \vec{x} \id{\}}$ to denote partial maps,
$s : \QProc \rightarrow \QProc$. A map, $s$ lifts, uniquely, to a map
on process terms, $\widehat{s} : \Proc \rightarrow \Proc$ by the
following equations.

\begin{mathpar}
  (0) \psubstp{Q}{P} := 0 \\
  (R \juxtap S) \psubstp{Q}{P}
  :=    
  (R)\psubstp{Q}{P} \juxtap (S) \psubstp{Q}{P} \\
  (x?(y).R) \psubstp{Q}{P}    
  :=    
  (x)\substp{Q}{P} (z)\concat( (R \psubstn{z}{y}) \psubstp{Q}{P} ) \\
  (\lift{x}{R}) \psubstp{Q}{P}  
  :=
  \lift{(x)\substp{Q}{P}}{ R \psubstp{Q}{P} } \\
%   (\dropn{x})  \psubstp{Q}{P}       
%   := 
%   \left\{ 
%     \begin{array}{ccc} 
%       \dropn{\quotep{Q}} & & x \nameeq \quotep{P} \\
%       \dropn{x} & & otherwise \\
%     \end{array}
%   \right. 
  (\dropn{x})  \psubstp{Q}{P}       
  := 
  \left\{ 
    \begin{array}{ccc} 
      Q & & x \nameeq \quotep{P} \\
      \dropn{x} & & otherwise \\
    \end{array}
  \right.
\end{mathpar}
 

where

\begin{eqnarray}
  (x)\id{\{} \lpquote Q \rpquote / \lpquote P \rpquote \id{\}}            = 
  \left\{ 
    \begin{array}{ccc}
      \lpquote Q \rpquote & & x \nameeq \lpquote P \rpquote \\
      x & & otherwise \\
    \end{array}
  \right. \nonumber
\end{eqnarray}

and $z$ is chosen distinct from $\quotep{P}$, $\quotep{Q}$, the free
names in $Q$, and all the names in $R$. Our $\alpha$-equivalence will
be built in the standard way from this substitution.

\begin{remark}\label{rem:no_self_referential_names}
  One consequence of these definitions is that $\forall P. \quotep{P}
  \not\in \freenames{P}$.
\end{remark}

\subsection{ Dynamic quote: an example }

Anticipating something of what's to come, consider applying the
substitution, $\widehat{\id{\{}u / z \id{\}}}$, to the following pair
of processes, $\lift{w}{y!(z)}$ and $w[ \lpquote y!(z) \rpquote ]$.

\begin{eqnarray}
	\lift{w}{y!(z)}\widehat{\id{\{}u / z \id{\}}}
		& = &
		\lift{w}{y!(u)} \nonumber\\
	w[ \lpquote y!(z) \rpquote ] \widehat{ \id{\{}u / z \id{\}} }
		& = &
		w[ \lpquote y!(z) \rpquote ] \nonumber
\end{eqnarray}

Because the body of the process between quotes is impervious to
substitution, we get radically different answers. In fact, by
examining the first process in an input context,
e.g. $x?(z).\lift{w}{y!(z)}$, we see that the process under the lift
operator may be shaped by prefixed inputs binding a name inside it. In
this sense, the lift operator will be seen as a way to dynamically
construct processes before reifying them as names.

Finally equipped with these standard features we can present the
dynamics of the calculus.

\subsubsection{Operational semantics} 

Finally, we introduce the computational dynamics. What marks these
algebras as distinct from other more traditionally studied algebraic
structures, e.g. vector spaces or polynomial rings, is the manner in
which dynamics is captured. In traditional structures, dynamics is typically
expressed through morphisms between such structures, as in linear maps
between vector spaces or morphisms between rings. In algebras
associated with the semantics of computation, the dynamics is
expressed as part of the algebraic structure itself, through a
reduction reduction relation typically denoted by $\red$. Below, we
give a recursive presentation of this relation for the calculus used
in the encoding.

$\red \subseteq \pi \times \pi$
$\red : \pi \to \mathcal{P}(\pi)$

\begin{mathpar}
  \inferrule* [lab=Comm] { \textsf{match}( x_{src}, x_{trgt} ) } { x_{trgt}?(y)P \; | \; x_{src}!\langle {Q} \rangle \red P\{\quotep{Q}/y}\} }
  \and \\
  \inferrule* [lab=Par] {{P} \red {P}'} {{{P} | {Q}} \red {{P}' | {Q}}}
  \and
  \inferrule* [lab=Equiv]{{{P} \scong {P}'} \andalso {{P}' \red {Q}'} \andalso {{Q}' \scong {Q}}}{{P} \red {Q}}
\end{mathpar}

\begin{eqnarray*}
  match_{\equiv} (\quotep{P},\quotep{Q}) & := & P \equiv Q \\
  match_{\dagger}(\quotep{P},\quotep{Q}) & := & \forall R. P|Q \red^{*} R => R \red^{*} 0 \\
  match_{K}(\quotep{P},\quotep{Q}) & := & K \mbox{ for some context } K
\end{eqnarray*}

$u?(x)P | u!\langle Q \rangle \red P\{\quotep{Q}/x\}$

%We write $\wred$ for $\red^*$, and $P\red$ if $\exists Q $ such that $ P \red Q$.
We write $P\red$ if $\exists Q $ such that $ P \red Q$ and $P\not\red$, otherwise.

\section{Replication}

As mentioned before, it is known that replication (and hence
recursion) can be implemented in a higher-order process algebra
\cite{SangiorgiWalker}. As our first example of calculation with the
machinery thus far presented we give the construction explicitly in
the {\rhoc}.

\begin{eqnarray}
	D_{x} & := & \prefix{x}{y}{(\binpar{\outputp{x}{y}}{@{y}})} \nonumber\\
	\bangp_{x}{P} & := & \binpar{{x}!\langle{\binpar{D_{x}}{P}}\rangle}{D_{x}} \nonumber
\end{eqnarray}

\begin{eqnarray}
	\bangp_{x}{P} & & \nonumber\\
	=
	& {x}!\langle{(\prefix{x}{y}{(\outputp{x}{y} | @{y})) | P}}\rangle 
	      | \prefix{x}{y}{(\outputp{x}{y} | @{y})} & \nonumber\\
	\red
	& (\outputp{x}{y} | @{y})\substn{\quotep{(\prefix{x}{y}{(@{y} | \outputp{x}{y})) | P}}}{y} & \nonumber\\
	=
	& \outputp{x}{\quotep{(\prefix{x}{y}{(\outputp{x}{y} | @{y})) | P}}}
	  | {(\prefix{x}{y}{(\outputp{x}{y} | @{y})) | P}} & \nonumber\\
	\red
	& \ldots & \nonumber\\
	\red^*
	& P | P | \ldots & \nonumber
\end{eqnarray}

Of course, this encoding, as an implementation, runs away, unfolding
$\bangp{P}$ eagerly. A lazier and more implementable replication
operator, restricted to input-guarded processes, may be obtained as follows.

\begin{eqnarray}
\bangp{\prefix{u}{v}{P}} 
	:= 
	\binpar{\lift{x}{\prefix{u}{v}{(\binpar{D(x)}{P})}}}{D(x)} \nonumber
\end{eqnarray}

\begin{remark}
  Note that the lazier definition still does not deal with summation
  or mixed summation (i.e. sums over input and output). The reader is
  invited to construct definitions of replication that deal with these
  features. 

  Further, the definitions are parameterized in a name, $x$. Can you,
  gentle reader, make a definition that eliminates this parameter and
  guarantees no accidental interaction between the replication
  machinery and the process being replicated -- i.e. no accidental
  sharing of names used by the process to get its work done and the
  name(s) used by the replication to effect copying. This latter
  revision of the definition of replication is crucial to obtaining
  the expected identity $!!P \sim !P$.
\end{remark}

\begin{remark}\label{rem:paradoxical_combinator}
  The reader familiar with the lambda calculus will have noticed the
  similarity between $D$ and the paradoxical combinator.

  [Ed. note: the existence of this seems to suggest we have to be more
  restrictive on the set of processes and names we admit if we are to
  support no-cloning.]
\end{remark}

\subsubsection{Bisimulation}

The computational dynamics gives rise to another kind of equivalence,
the equivalence of computational behavior. As previously mentioned
this is typically captured \emph{via} some form of bisimulation.

% The notion we use in this paper is weak barbed bisimulation
% \cite{milner91polyadicpi}.

The notion we use in this paper is derived from weak barbed
bisimulation \cite{milner91polyadicpi}. 

\begin{definition}
An \emph{observation relation}, $\downarrow_{\mathcal N}$, over a set
of names, $\mathcal N$, is the smallest relation satisfying the rules
below.

\infrule[Out-barb]{y \in {\mathcal N}, \; x \nameeq y}
		  {\outputp{x}{v} \downarrow_{\mathcal N} x}
\infrule[Par-barb]{\mbox{$P\downarrow_{\mathcal N} x$ or $Q\downarrow_{\mathcal N} x$}}
		  {\binpar{P}{Q} \downarrow_{\mathcal N} x}

We write $P \Downarrow_{\mathcal N} x$ if there is $Q$ such that 
$P \wred Q$ and $Q \downarrow_{\mathcal N} x$.
\end{definition}

\begin{definition}
%\label{def.bbisim}
An  ${\mathcal N}$-\emph{barbed bisimulation} over a set of names, ${\mathcal N}$, is a symmetric binary relation 
${\mathcal S}_{\mathcal N}$ between agents such that $P\rel{S}_{\mathcal N}Q$ implies:
\begin{enumerate}
\item If $P \red P'$ then $Q \wred Q'$ and $P'\rel{S}_{\mathcal N} Q'$.
\item If $P\downarrow_{\mathcal N} x$, then $Q\Downarrow_{\mathcal N} x$.
\end{enumerate}
$P$ is ${\mathcal N}$-barbed bisimilar to $Q$, written
$P \wbbisim_{\mathcal N} Q$, if $P \rel{S}_{\mathcal N} Q$ for some ${\mathcal N}$-barbed bisimulation ${\mathcal S}_{\mathcal N}$.
\end{definition}

$\mathcal{R} \subseteq \pi \times \pi$

$P \mathcal{R} Q => \forall P'. P \red P' \Rightarrow \exists Q'. Q \red Q', P' \mathcal{R} Q'$

$P \vdash x \Rightarrow Q \vdash x$

\begin{mathpar}
  \inferrule*[lab=Out-barb]{x \nameeq y}{{y}!\langle{Q}\rangle \vdash x}
  \and
  \inferrule*[lab=Par-barb]{\mbox{$P\vdash x$ or $Q\vdash x$}}{\binpar{P}{Q} \vdash x}
\end{mathpar}

\subsubsection{Contexts}

One of the principle advantages of computational calculi like the
$\pi$-calculus is a well-defined notion of context,
contextual-equivalence and a correlation between
contextual-equivalence and notions of bisimulation. The notion of
context allows the decomposition of a process into (sub-)process and
its syntactic environment, its context. Thus, a context may be
thought of as a process with a ``hole'' (written $\Box$) in it. The
application of a context $M$ to a process $P$, written $M[P]$, is
tantamount to filling the hole in $M$ with $P$. In this paper we do
not need the full weight of this theory, but do make use of the notion
of context in the proof the main theorem. 

\begin{mathpar}
  \inferrule* [lab=summation] {} {{M_{M},M_{N}} \bc \Box \;|\; x.M_{A} \;|\; M_{M}+M_{N}}
  \and
  \inferrule* [lab=agent] {} {{M_{A}} \bc (\vec{x})M_{P} \;| \; \clift{P_0,\ldots,M_{P},\ldots,P_N}}
  \and \\
  \inferrule* [lab=process] {} {{M_{P}} \bc M_{N} \;| \;P|M_{P} }
\end{mathpar} 

\begin{mathpar}
  \inferrule* [lab=sychronization] {} {M_{N} \bc \Box \;|\; x?M_{F} \;|\; x!M_{C}}
  \and
  \inferrule* [lab=abstraction] {} {{M_{F}} \bc (x)M_{P} }
  \and
  \inferrule* [lab=concretion] {} {{M_{C}} \bc \langle M_{P} \rangle }
  \and \\
  \inferrule* [lab=process] {} {{M_{P}} \bc M_{N} \;| \;P|M_{P} }
\end{mathpar}

\begin{definition}[contextual application] Given a context $M$, and
  process $P$, we define the \emph{contextual application}, $M[P] :=
  M\{P/\Box\}$. That is, the contextual application of M to P is the
  substitution of $P$ for $\Box$ in $M$.
\end{definition}

$\meaningof{-} : L \to \mathcal{P}(\pi)$

\begin{mathpar}
  \inferrule* [lab=collection] {} {\meaningof{true} = \pi, \and \meaningof{~E} = \pi \setminus \meaningof{E}, \and \meaningof{E_{1} \& E_{2}} = \meaningof{E_{1}} \cap \meaningof{E_{2}}}
\end{mathpar}

\begin{mathpar}
  \inferrule* [lab=structure] {} {\meaningof{0} = \{ P \in \pi | P \equiv 0 \}, \and \\ \meaningof{E_1 | E_2} = \{ P \in \pi | P \equiv P_{1} | P_{2}, P_{1} \in \meaningof{E_{1}}, P_{2} \in \meaningof{E_2}\} }
\end{mathpar}

\begin{mathpar}
 \inferrule* [lab=behavior] {} {\meaningof{\langle a?b \rangle E} = \{ P \in \pi | P \equiv Q | u?(y)P', \\ \and \\\\ \and \\ \;\;\; u \in \meaningof{a}, \forall z.P'\{z/y\} \in \meaningof{E\{z/b\}}\}, \and \\ \meaningof{a!E} = \{ P \in \pi | P \equiv Q | x!\langle P' \rangle, x \in \meaningof{a} P' \in \meaningof{E}\} }
\end{mathpar}

\begin{mathpar}
 \inferrule* [lab=nominal] {} {\meaningof{\quotep{E}} = \{ \quotep{P} \in \quotep{\pi} | P \in \meaningof{E} \}, \and \meaningof{\quotep{P}} = \{ \quotep{Q} \in \quotep{\pi} | P \equiv Q \} \and \\ \meaningof{@\quotep{E}} = \{ P \in \pi | P \equiv @x, x \in \meaningof{E} \}}
\end{mathpar}

\begin{eqnarray*}
  \\
  \meaningof{-} : TS \to ST
\end{eqnarray*}

\begin{eqnarray*}
  \\
  L : TS \to ST
\end{eqnarray*}

\begin{eqnarray*}
  \\
  P \models E \iff P \in \meaningof{E}
\end{eqnarray*}

\begin{eqnarray*}
  P \approx_{L} Q \iff \forall E \in L. P \models E \iff Q \models E
\end{eqnarray*}

\begin{eqnarray*}
  P \approx_{K} Q
\end{eqnarray*}

\begin{eqnarray*}
  P \approx Q
\end{eqnarray*}

$\approx_{K} = \approx = \approx_{L}$

\subsubsection{Contextual duality}

Note that contexts extend the quotation operation to a family of
operations from processes to names. Given a context, $M$, we can
define a \emph{nominal context}, $\quotep{M}$ by $\quotep{M}[P] :=
\quotep{M[P]}$. To foreshadow what is to come we observe that these
operations enjoy a duality with processes very much like the duality
between vectors and maps from vectors to scalars.

Further, because the calculus is essentially higher-order, we have a
correspondence between contexts and processes. More specifically,
given a name $x$ and a context $M$ we can construct $M^{*}_{x}$ such
that 

\begin{mathpar}
  M^{*}_{x} | \lift{x}{P} \red M[P]
\end{mathpar}

namely,

\begin{mathpar}
  M^{*}_{x} := x?(u).M[\dropn{u}]
\end{mathpar}

The dependence of $M^{*}_{x}$ on a name makes it an abstraction, 

\begin{mathpar}
  M^{*} := (x)x?(u).M[\dropn{u}]
\end{mathpar}

\subsection{Additional notation}

It will sometimes be convenient to denote the process a name
quotes. We already have the notation $x = \quotep{P}$, but it will be
convenient to introduce an alternate notation, $\procn{x}$, when we
want to emphasize the connection to the use of the name. Note that, by
virtue of name equivalence, $\quotep{\procn{x}} \nameeq x$; so, the
notation is consistent with previous definitions.

Further, because names have structure it is possible to effect
substitutions on the basis of that structure. This means we need to
upgrade our notation for substitutions, which we accomplish by
adapting comprehension notation. Thus,

\begin{mathpar}
  P\{ y / x : x \in S \}
\end{mathpar}

is interpreted to mean the process derived from P by replacing (in a
capture-avoiding manner) each occurrence of $x$ in $S$ by $y$. For example,

\begin{mathpar}
  P\{ \quotep{\procn{x}|\procn{x}} / x : x \in \freenames{P} \}
\end{mathpar}

will replace each (occurrence) of a free name $x$ in $P$ by
$\quotep{\procn{x}|\procn{x}}$.

Also, we will avail ourselves of the notation $x^{L}$ and $x^{R}$ to
denote injections of a name into disjoint copies of the name
space. There are numerous ways to accomplish this. One example can be
found in \cite{MeredithR05}. This notation overloads to vectors of
names: $\vec{x}^{\pi} := (x_{i}^{\pi} \; : \; 0 \leq i < |\vec{x}| )$ where $\pi \in \{L,R\}$.

We also use $P^{\Box} := P|\Box$.

In \cite{MeredithR05} an interpretation of the new operator is
given. It turns out that there are several possible interpretations
all enjoying the requisite algebraic properties of the operator (see
\cite{milner91polyadicpi}). We will therefore make liberal use of
$(\nu\; \vec{x})P$.

% subsection the_syntax_and_semantics_of_the_notation_system (end)   

\input{qm2pi.qmops} 

\input{qm2pi.sterngerlach} 

\input{qm2pi.metric} 

% section concurrent_process_calculi (end)

%\input{qm2pi.proofsketch}

% section proof sketch (end)

%\input{qm2pi.slviaknots} 

% section spatial logic via knots (end)

\input{qm2pi.conclusion}

% section conclusion (end)

%\input{qm2pi.dtcodes} 

% section wiring algorithm (end)

\input{qm2pi.ack} 

% section acknowledgments (end)

\newpage


\bibliographystyle{plain}   
\bibliography{../../biblios/main.bib}

\input{qm2pi.rhodetails}

\end{document}



% section front matter (end)

\section{Introduction}\label{sec:introduction} % (fold)
In this draft of the material i am going to have to dispense with the
usual writing conventions adopted in papers on these topics. i'm going
to have adopt whatever tone i need at the time i'm writing up the
calculations. Sometimes this may be very conversational; others it may
be the barest mathematical grunts; others still it may be that i have
lifted text from one of my other papers because the exposition of some
point was better said there. i hope that my readers are not unduly put
out by this decision. i'm not doing this to flout convention or be
rebellious. i find these calculations very technically challenging. To
keep everything going technically, something has to give; i have to
let go of some cognitive burden. So, the academic writing style --
with all of its trade-offs in terms of facilitating technical
communication -- is what i'm letting go of. Perhaps subsequent drafts
can be tightened and polished, but for now, i'm going to speak as if
we were sitting together in a coffee shop with a laptop, wifi and a
pad of paper and a pencil.

So, here's what i have to say. We -- you and i, comfortably ensconced
in our coffee shop and well-equipped with our tools -- can realize and
carry out the calculations of quantum mechanics over a very different
formal theory of dynamics, a formal theory of dynamics that
corresponds to a theory of concurrent computation with
\emph{reflection}. It has the advantage that the underlying theory is
already `quantized', but supports analogues all of the continuuous
operations. Strikingly, this underlying theory has recently been
connected with a notion of metric that we can show, by calculating
together, coincides with the metric induced by the inner product.

There are a lot of reasons why you might be interested in seeing
calculations of this form. Here's why i'm interested. For the past
several centuries there has been no competitor to the ``Newtonian''
account of dynamics. As a result the predominant share of accounts of
dynamical systems and situations have had to be formulated in terms of
the Newtonian machinery. i view this as an intellectually dangerous
position to occupy. Everything, despite it's intrinsic shape, turns
into a nail to be hit with this hammer. Recently, however, the theory
of computation has matured to the point where we have candidates for
theories of dynamics that offer very different perspective on
reasoning about dynamical systems and situations. Testing these
candidates against very successful accounts of dynamical situations,
like quantum mechanics, is going to give us some sense of how mature
they are and some measure of the quality of these accounts of
dynamics.

\subsection{Summary of contributions and outline of paper}

So, we're going to develop an interpretation of the operations of
quantum mechanics normally interpreted by Hilbert spaces and
operators. We're going to do this over a theory of computation. Note
that this is very different than the usual quantum computation program
which develops notions of computation over quantum mechanics. Rather,
we are developing a story that aligns with Wheeler's slogan: It from
Bit. To do this we will first provide an account of the theory of
computation at play here. Then we will dive into a calculation-driven
interpretation of the operations of quantum mechanics.

The reason we take this approach is that -- until very recently --
there hasn't been an axiomatic account of quantum mechanics. As a
result there has been no sharp delineation of the mathematical theory
supporting interpretation of the physical theory and the physical
theory, itself. So, ambient features of the maths are free to be
exploited (or supressed) without a real accounting of their physical
relevance. There is no sharp statement ``here's the physical theory''
qua \emph{theory} and ``here's the mathematical interpretation''
enabling a judgment of how faithful the interpretation is -- apart
from experimental observation. When there is an axiomatic account we
can judge how well a given mathematical formalism supports an
interpretation of the axioms, independent of
experimentation. Likewise, we can judge how well we have captured our
physical evidence and experience with our axiomatics, independent of
any specific mathematical implementation, with accidental detail that
may or may not have physical significance. 

In lieu of a fully fleshed out and vetted axiomatic account of quantum
mechanics, interpreting the operational notions in service of modeling
physical systems will have to suffice. In other words, we are not in
the business of providing a model of Hilbert spaces and operators. We
are in the business of providing a model of quantum mechanics because
we are motivated by testing our notions of dynamics against physical
theory; and, the predictive calculations of the physical theory must
serve as the best formulation -- shy of a fully fleshed out axiomatic
account -- of the physical theory itself (as they have for scientific
theories since time immemorial). Put another way, despite a
whole-hearted commitment to an It-from-Bit ontology, we are firmly
aligned with the shut-up-and-calculate camp as the best way to obtain
results either from the physical perspective or as a quality assurance
measure of our fledgling theory of dynamics.

In detail, we present a reflective process calculus. Then we develop
intuitive correspondences between the notions available in this
calculus and the usual physical notions supporting quantum mechanical
calculations. Thus, 

\begin{table}[htp]
  \center{
    \fbox{
      \begin{tabular}{c|c}
        quantum mechanics & process calculus \\
        \hline
        scalar & name \\
        state vector & process \\
        dual & contextual duals \\
        matrix & formal sums of process-context-dual pairs \\
        orthogonality & process annihilation \\
        inner product & execution-formula + quoting
      \end{tabular}
    }
  }
  \caption{QM - process calculi correspondences}
\end{table}

Then we tighten up these intuitions to operational definitions. We
employ the Dirac notation as the best proxy we can find for an
abstract syntax of the quantum mechanical notions. The definitions we
develop put us in contact with equational constraints coming from the
theory that we demonstrate the definitions and calculations satisfy.

This puts us in a position to shut up and calculate for the
Stern-Gerlach experimental set up, showing how these predictive
calculations become calculations on processes in our theory of a
reflective process calculus.

Penultimately, we demonstrate that the notion of metric coming from
the inner product coincides with the notion of metric available from
the theory of bisimulation. This demonstration gives us the right to
think of space as arising from behavior. Finally, we consider where we
might go from the new vantage point we have obtained.

% section introduction (end) 
 
% section introduction (end)

% \documentclass[12pt]{llncs}
%\documentclass{jktr}

\usepackage[pdftex]{hyperref}                   
\usepackage {listings}
\usepackage {mathpartir}
\usepackage{bcprules}
%\usepackage{listings}
                       
\usepackage{graphicx} 
%\usepackage[margins=2.5cm,nohead,nofoot]{geometry}
%\usepackage{geometry}
\usepackage{amsfonts}
\usepackage{amstext}
\usepackage{latexsym}
\usepackage{amssymb}
\usepackage{color}


%\include{myPreamble}
\include{qm2pi.local} 

%\ifpdf
%\usepackage[pdftex]{graphicx}
%\else
%\usepackage{graphicx}
%\fi

 % \ifpdf
%  \usepackage{pdfsync}
%  \if


%\title{Brief Article}
%\author{David F. Snyder}
%\author{L.G. Meredith}

%\address{Dept. of Math., Texas State University--San Marcos, San Marcos, TX 78666}
       
\pagestyle{empty}


\begin{document}

\lstset{language=[Objective]Caml,frame=shadowbox}

\input{qm2pi.front}

% section front matter (end)

\input{qm2pi.intro} 
 
% section introduction (end)

% \input{qm2pi.knotations} 

% section notation (end)

\input{qm2pi.process.calculi} 

% section concurrent_process_calculi_and_spatial_logics_ (end)
    
%\input{qm2pi.knots2pi} 

%\input{qm2pi.trefoil} 

%\input{qm2pi.mainthm} 

% subsection basic_interpretation (end)

%\input{qm2pi.rho.presentation} 
\subsection{The syntax and semantics of the notation system}\label{sub:the_syntax_and_semantics_of_the_notation_system} % (fold)

We now summarize a technical presentation of the calculus that
embodies our theory of dynamics. The typical presentation of such a
calculus follows the style of giving generators and relations on
them. The grammar, below, describing term constructors, freely
generates the set of processes, $\Proc$. This set is then quotiented
by a relation known as structural congruence and it is over this set
that the notion of dynamics is expressed. This presentation is
essentially that of \cite{MeredithR05} with the addition of
polyadicity and summation. For readability we have relegated some of
the technical subtleties to an appendix.

\subsubsection{Process grammar}\label{subsub:process_grammar}

\begin{mathpar}
  \inferrule* [lab=synchronization] {} {{M} \bc \pzero \;|\; x?F \;|\; x!C }
  \and
  \inferrule* [lab=abstraction] {} {{F} \bc (x)P}
  \and
  \inferrule* [lab=concretion] {} {{C} \bc \langle Q \rangle}
  \and
  \inferrule* [lab=process] {} {{P,Q} \bc M \;| \;P|Q \;|\; @{x}}
  \and
  \inferrule* [lab=name] {} {{x} \bc \quotep{P}}
\end{mathpar} 

Note that $\vec{x}$ (resp. $\vec{P}$) denotes a vector of names
(resp. processes) of length $|\vec{x}|$ (resp. $|\vec{P}|$). We adopt
the following useful abbreviations.

\begin{mathpar}
   x?(\vec{y}).P := x.(\vec{y})P \and  x\clift{\vec{P}} := x.\clift{\vec{P}}
   \and x!(y) := \lift{x}{\dropn{y}}
   \and \Pi_{i=0}^{n-1}P_i := P_0 | \ldots | P_{n-1}
\end{mathpar}

\subsubsection{Structural congruence}

\paragraph{Free and bound names and alpha-equivalence.} At the
core of structural equivalence is alpha-equivalence which identifies
process that are the same up to a change of variable. Formally, we
recognize the distinction between free and bound names. The free names
of a process, $\freenames{P}$, may be calculated recursively as
follows:

\begin{mathpar}
\freenames{\pzero} := \emptyset
  \and \\
  \freenames{x?(y).P} := \{ x \} \cup (\freenames{P} \setminus \{ y \})
  \and 
  \freenames{x!\langle P \rangle} := \{ x \} \cup \{ P \} 
  \and \\
  \freenames{P|Q} := \freenames{P} \cup \freenames{Q}
  \and \\
  \freenames{@{x}} := \{ x \}
\end{mathpar}

$\pi$
$\quotep{\pi}$

$\freenames{-} : \pi \to \mathcal{P}(\quotep{\pi})$

\begin{eqnarray*}
  \freenames{\pzero} & := & \emptyset \\
  \freenames{x?(y).P} & := & \{ x \} \cup (\freenames{P} \setminus \{ y \}) \\
  \freenames{x!\langle P \rangle} & := & \{ x \} \cup \{ P \} \\
  \freenames{P|Q} & := & \freenames{P} \cup \freenames{Q} \\
  \freenames{\dropn{x}} & := & \{ x \}
\end{eqnarray*}

The bound names of a process, $\boundnames{P}$, are those names occurring in $P$
that are not free. For example, in $x?(y).0$, the name $x$ is free, while $y$ is bound.

\begin{mathpar}
  \inferrule* [lab=monoidal-laws] {} { P|Q \equiv Q|P \and P|0 \equiv P \and P|(Q|R) \equiv (P|Q)|R }
\end{mathpar}

\begin{mathpar}
  \inferrule* [lab=alpha-equivalence] {} { (x)P \equiv (y)P\{y/x\} \and y \not\in \freenames{P} }
\end{mathpar}

\begin{definition}
Then two processes, $P,Q$, are alpha-equivalent if $P = Q\{\vec{y}/\vec{x}\}$ for
some $\vec{x} \in \boundnames{Q},\vec{y} \in \boundnames{P}$, where $Q\{\vec{y}/\vec{x}\}$
denotes the capture-avoiding substitution of $\vec{y}$ for $\vec{x}$ in $Q$.
\end{definition}

\begin{definition}
  The {\em structural congruence} \cite{SangiorgiWalker} , $\equiv$,
  between processes is the least congruence containing
  alpha-equivalence, satisfying the abelian monoid laws
  (associativity, commutativity and $\pzero$ as identity) for parallel
  composition $|$ and for summation $+$.
\end{definition}

\subsection{Name equivalence}

We take name equivalence, written $\nameeq$, to be the smallest
equivalence relation generated by the following rules.

\begin{mathpar}
\inferrule*[lab=Quote-drop]
{ }
{ \quotep{@{x}} \nameeq x }

\inferrule*[lab=Struct-equiv]
{ P \scong Q }
{ \quotep{P} \nameeq \quotep{Q} }
\end{mathpar}

The astute reader will have noticed that the mutual recursion of names
and processes imposes a mutual recursion on alpha-equivalence and
structural equivalence via name-equivalence. Fortunately, all of this
works out pleasantly and we may calculate in the natural way, free of
concern. The reader interested in the details is referred to the
appendix \ref{appendix:rho_details}.

\subsection{Substitution}

We use $\Proc$ for the set of processes, $\QProc$ for the set of
names, and $\id{\{}\vec{y} / \vec{x} \id{\}}$ to denote partial maps,
$s : \QProc \rightarrow \QProc$. A map, $s$ lifts, uniquely, to a map
on process terms, $\widehat{s} : \Proc \rightarrow \Proc$ by the
following equations.

\begin{mathpar}
  (0) \psubstp{Q}{P} := 0 \\
  (R \juxtap S) \psubstp{Q}{P}
  :=    
  (R)\psubstp{Q}{P} \juxtap (S) \psubstp{Q}{P} \\
  (x?(y).R) \psubstp{Q}{P}    
  :=    
  (x)\substp{Q}{P} (z)\concat( (R \psubstn{z}{y}) \psubstp{Q}{P} ) \\
  (\lift{x}{R}) \psubstp{Q}{P}  
  :=
  \lift{(x)\substp{Q}{P}}{ R \psubstp{Q}{P} } \\
%   (\dropn{x})  \psubstp{Q}{P}       
%   := 
%   \left\{ 
%     \begin{array}{ccc} 
%       \dropn{\quotep{Q}} & & x \nameeq \quotep{P} \\
%       \dropn{x} & & otherwise \\
%     \end{array}
%   \right. 
  (\dropn{x})  \psubstp{Q}{P}       
  := 
  \left\{ 
    \begin{array}{ccc} 
      Q & & x \nameeq \quotep{P} \\
      \dropn{x} & & otherwise \\
    \end{array}
  \right.
\end{mathpar}
 

where

\begin{eqnarray}
  (x)\id{\{} \lpquote Q \rpquote / \lpquote P \rpquote \id{\}}            = 
  \left\{ 
    \begin{array}{ccc}
      \lpquote Q \rpquote & & x \nameeq \lpquote P \rpquote \\
      x & & otherwise \\
    \end{array}
  \right. \nonumber
\end{eqnarray}

and $z$ is chosen distinct from $\quotep{P}$, $\quotep{Q}$, the free
names in $Q$, and all the names in $R$. Our $\alpha$-equivalence will
be built in the standard way from this substitution.

\begin{remark}\label{rem:no_self_referential_names}
  One consequence of these definitions is that $\forall P. \quotep{P}
  \not\in \freenames{P}$.
\end{remark}

\subsection{ Dynamic quote: an example }

Anticipating something of what's to come, consider applying the
substitution, $\widehat{\id{\{}u / z \id{\}}}$, to the following pair
of processes, $\lift{w}{y!(z)}$ and $w[ \lpquote y!(z) \rpquote ]$.

\begin{eqnarray}
	\lift{w}{y!(z)}\widehat{\id{\{}u / z \id{\}}}
		& = &
		\lift{w}{y!(u)} \nonumber\\
	w[ \lpquote y!(z) \rpquote ] \widehat{ \id{\{}u / z \id{\}} }
		& = &
		w[ \lpquote y!(z) \rpquote ] \nonumber
\end{eqnarray}

Because the body of the process between quotes is impervious to
substitution, we get radically different answers. In fact, by
examining the first process in an input context,
e.g. $x?(z).\lift{w}{y!(z)}$, we see that the process under the lift
operator may be shaped by prefixed inputs binding a name inside it. In
this sense, the lift operator will be seen as a way to dynamically
construct processes before reifying them as names.

Finally equipped with these standard features we can present the
dynamics of the calculus.

\subsubsection{Operational semantics} 

Finally, we introduce the computational dynamics. What marks these
algebras as distinct from other more traditionally studied algebraic
structures, e.g. vector spaces or polynomial rings, is the manner in
which dynamics is captured. In traditional structures, dynamics is typically
expressed through morphisms between such structures, as in linear maps
between vector spaces or morphisms between rings. In algebras
associated with the semantics of computation, the dynamics is
expressed as part of the algebraic structure itself, through a
reduction reduction relation typically denoted by $\red$. Below, we
give a recursive presentation of this relation for the calculus used
in the encoding.

$\red \subseteq \pi \times \pi$
$\red : \pi \to \mathcal{P}(\pi)$

\begin{mathpar}
  \inferrule* [lab=Comm] { \textsf{match}( x_{src}, x_{trgt} ) } { x_{trgt}?(y)P \; | \; x_{src}!\langle {Q} \rangle \red P\{\quotep{Q}/y}\} }
  \and \\
  \inferrule* [lab=Par] {{P} \red {P}'} {{{P} | {Q}} \red {{P}' | {Q}}}
  \and
  \inferrule* [lab=Equiv]{{{P} \scong {P}'} \andalso {{P}' \red {Q}'} \andalso {{Q}' \scong {Q}}}{{P} \red {Q}}
\end{mathpar}

\begin{eqnarray*}
  match_{\equiv} (\quotep{P},\quotep{Q}) & := & P \equiv Q \\
  match_{\dagger}(\quotep{P},\quotep{Q}) & := & \forall R. P|Q \red^{*} R => R \red^{*} 0 \\
  match_{K}(\quotep{P},\quotep{Q}) & := & K \mbox{ for some context } K
\end{eqnarray*}

$u?(x)P | u!\langle Q \rangle \red P\{\quotep{Q}/x\}$

%We write $\wred$ for $\red^*$, and $P\red$ if $\exists Q $ such that $ P \red Q$.
We write $P\red$ if $\exists Q $ such that $ P \red Q$ and $P\not\red$, otherwise.

\section{Replication}

As mentioned before, it is known that replication (and hence
recursion) can be implemented in a higher-order process algebra
\cite{SangiorgiWalker}. As our first example of calculation with the
machinery thus far presented we give the construction explicitly in
the {\rhoc}.

\begin{eqnarray}
	D_{x} & := & \prefix{x}{y}{(\binpar{\outputp{x}{y}}{@{y}})} \nonumber\\
	\bangp_{x}{P} & := & \binpar{{x}!\langle{\binpar{D_{x}}{P}}\rangle}{D_{x}} \nonumber
\end{eqnarray}

\begin{eqnarray}
	\bangp_{x}{P} & & \nonumber\\
	=
	& {x}!\langle{(\prefix{x}{y}{(\outputp{x}{y} | @{y})) | P}}\rangle 
	      | \prefix{x}{y}{(\outputp{x}{y} | @{y})} & \nonumber\\
	\red
	& (\outputp{x}{y} | @{y})\substn{\quotep{(\prefix{x}{y}{(@{y} | \outputp{x}{y})) | P}}}{y} & \nonumber\\
	=
	& \outputp{x}{\quotep{(\prefix{x}{y}{(\outputp{x}{y} | @{y})) | P}}}
	  | {(\prefix{x}{y}{(\outputp{x}{y} | @{y})) | P}} & \nonumber\\
	\red
	& \ldots & \nonumber\\
	\red^*
	& P | P | \ldots & \nonumber
\end{eqnarray}

Of course, this encoding, as an implementation, runs away, unfolding
$\bangp{P}$ eagerly. A lazier and more implementable replication
operator, restricted to input-guarded processes, may be obtained as follows.

\begin{eqnarray}
\bangp{\prefix{u}{v}{P}} 
	:= 
	\binpar{\lift{x}{\prefix{u}{v}{(\binpar{D(x)}{P})}}}{D(x)} \nonumber
\end{eqnarray}

\begin{remark}
  Note that the lazier definition still does not deal with summation
  or mixed summation (i.e. sums over input and output). The reader is
  invited to construct definitions of replication that deal with these
  features. 

  Further, the definitions are parameterized in a name, $x$. Can you,
  gentle reader, make a definition that eliminates this parameter and
  guarantees no accidental interaction between the replication
  machinery and the process being replicated -- i.e. no accidental
  sharing of names used by the process to get its work done and the
  name(s) used by the replication to effect copying. This latter
  revision of the definition of replication is crucial to obtaining
  the expected identity $!!P \sim !P$.
\end{remark}

\begin{remark}\label{rem:paradoxical_combinator}
  The reader familiar with the lambda calculus will have noticed the
  similarity between $D$ and the paradoxical combinator.

  [Ed. note: the existence of this seems to suggest we have to be more
  restrictive on the set of processes and names we admit if we are to
  support no-cloning.]
\end{remark}

\subsubsection{Bisimulation}

The computational dynamics gives rise to another kind of equivalence,
the equivalence of computational behavior. As previously mentioned
this is typically captured \emph{via} some form of bisimulation.

% The notion we use in this paper is weak barbed bisimulation
% \cite{milner91polyadicpi}.

The notion we use in this paper is derived from weak barbed
bisimulation \cite{milner91polyadicpi}. 

\begin{definition}
An \emph{observation relation}, $\downarrow_{\mathcal N}$, over a set
of names, $\mathcal N$, is the smallest relation satisfying the rules
below.

\infrule[Out-barb]{y \in {\mathcal N}, \; x \nameeq y}
		  {\outputp{x}{v} \downarrow_{\mathcal N} x}
\infrule[Par-barb]{\mbox{$P\downarrow_{\mathcal N} x$ or $Q\downarrow_{\mathcal N} x$}}
		  {\binpar{P}{Q} \downarrow_{\mathcal N} x}

We write $P \Downarrow_{\mathcal N} x$ if there is $Q$ such that 
$P \wred Q$ and $Q \downarrow_{\mathcal N} x$.
\end{definition}

\begin{definition}
%\label{def.bbisim}
An  ${\mathcal N}$-\emph{barbed bisimulation} over a set of names, ${\mathcal N}$, is a symmetric binary relation 
${\mathcal S}_{\mathcal N}$ between agents such that $P\rel{S}_{\mathcal N}Q$ implies:
\begin{enumerate}
\item If $P \red P'$ then $Q \wred Q'$ and $P'\rel{S}_{\mathcal N} Q'$.
\item If $P\downarrow_{\mathcal N} x$, then $Q\Downarrow_{\mathcal N} x$.
\end{enumerate}
$P$ is ${\mathcal N}$-barbed bisimilar to $Q$, written
$P \wbbisim_{\mathcal N} Q$, if $P \rel{S}_{\mathcal N} Q$ for some ${\mathcal N}$-barbed bisimulation ${\mathcal S}_{\mathcal N}$.
\end{definition}

$\mathcal{R} \subseteq \pi \times \pi$

$P \mathcal{R} Q => \forall P'. P \red P' \Rightarrow \exists Q'. Q \red Q', P' \mathcal{R} Q'$

$P \vdash x \Rightarrow Q \vdash x$

\begin{mathpar}
  \inferrule*[lab=Out-barb]{x \nameeq y}{{y}!\langle{Q}\rangle \vdash x}
  \and
  \inferrule*[lab=Par-barb]{\mbox{$P\vdash x$ or $Q\vdash x$}}{\binpar{P}{Q} \vdash x}
\end{mathpar}

\subsubsection{Contexts}

One of the principle advantages of computational calculi like the
$\pi$-calculus is a well-defined notion of context,
contextual-equivalence and a correlation between
contextual-equivalence and notions of bisimulation. The notion of
context allows the decomposition of a process into (sub-)process and
its syntactic environment, its context. Thus, a context may be
thought of as a process with a ``hole'' (written $\Box$) in it. The
application of a context $M$ to a process $P$, written $M[P]$, is
tantamount to filling the hole in $M$ with $P$. In this paper we do
not need the full weight of this theory, but do make use of the notion
of context in the proof the main theorem. 

\begin{mathpar}
  \inferrule* [lab=summation] {} {{M_{M},M_{N}} \bc \Box \;|\; x.M_{A} \;|\; M_{M}+M_{N}}
  \and
  \inferrule* [lab=agent] {} {{M_{A}} \bc (\vec{x})M_{P} \;| \; \clift{P_0,\ldots,M_{P},\ldots,P_N}}
  \and \\
  \inferrule* [lab=process] {} {{M_{P}} \bc M_{N} \;| \;P|M_{P} }
\end{mathpar} 

\begin{mathpar}
  \inferrule* [lab=sychronization] {} {M_{N} \bc \Box \;|\; x?M_{F} \;|\; x!M_{C}}
  \and
  \inferrule* [lab=abstraction] {} {{M_{F}} \bc (x)M_{P} }
  \and
  \inferrule* [lab=concretion] {} {{M_{C}} \bc \langle M_{P} \rangle }
  \and \\
  \inferrule* [lab=process] {} {{M_{P}} \bc M_{N} \;| \;P|M_{P} }
\end{mathpar}

\begin{definition}[contextual application] Given a context $M$, and
  process $P$, we define the \emph{contextual application}, $M[P] :=
  M\{P/\Box\}$. That is, the contextual application of M to P is the
  substitution of $P$ for $\Box$ in $M$.
\end{definition}

$\meaningof{-} : L \to \mathcal{P}(\pi)$

\begin{mathpar}
  \inferrule* [lab=collection] {} {\meaningof{true} = \pi, \and \meaningof{~E} = \pi \setminus \meaningof{E}, \and \meaningof{E_{1} \& E_{2}} = \meaningof{E_{1}} \cap \meaningof{E_{2}}}
\end{mathpar}

\begin{mathpar}
  \inferrule* [lab=structure] {} {\meaningof{0} = \{ P \in \pi | P \equiv 0 \}, \and \\ \meaningof{E_1 | E_2} = \{ P \in \pi | P \equiv P_{1} | P_{2}, P_{1} \in \meaningof{E_{1}}, P_{2} \in \meaningof{E_2}\} }
\end{mathpar}

\begin{mathpar}
 \inferrule* [lab=behavior] {} {\meaningof{\langle a?b \rangle E} = \{ P \in \pi | P \equiv Q | u?(y)P', \\ \and \\\\ \and \\ \;\;\; u \in \meaningof{a}, \forall z.P'\{z/y\} \in \meaningof{E\{z/b\}}\}, \and \\ \meaningof{a!E} = \{ P \in \pi | P \equiv Q | x!\langle P' \rangle, x \in \meaningof{a} P' \in \meaningof{E}\} }
\end{mathpar}

\begin{mathpar}
 \inferrule* [lab=nominal] {} {\meaningof{\quotep{E}} = \{ \quotep{P} \in \quotep{\pi} | P \in \meaningof{E} \}, \and \meaningof{\quotep{P}} = \{ \quotep{Q} \in \quotep{\pi} | P \equiv Q \} \and \\ \meaningof{@\quotep{E}} = \{ P \in \pi | P \equiv @x, x \in \meaningof{E} \}}
\end{mathpar}

\begin{eqnarray*}
  \\
  \meaningof{-} : TS \to ST
\end{eqnarray*}

\begin{eqnarray*}
  \\
  L : TS \to ST
\end{eqnarray*}

\begin{eqnarray*}
  \\
  P \models E \iff P \in \meaningof{E}
\end{eqnarray*}

\begin{eqnarray*}
  P \approx_{L} Q \iff \forall E \in L. P \models E \iff Q \models E
\end{eqnarray*}

\begin{eqnarray*}
  P \approx_{K} Q
\end{eqnarray*}

\begin{eqnarray*}
  P \approx Q
\end{eqnarray*}

$\approx_{K} = \approx = \approx_{L}$

\subsubsection{Contextual duality}

Note that contexts extend the quotation operation to a family of
operations from processes to names. Given a context, $M$, we can
define a \emph{nominal context}, $\quotep{M}$ by $\quotep{M}[P] :=
\quotep{M[P]}$. To foreshadow what is to come we observe that these
operations enjoy a duality with processes very much like the duality
between vectors and maps from vectors to scalars.

Further, because the calculus is essentially higher-order, we have a
correspondence between contexts and processes. More specifically,
given a name $x$ and a context $M$ we can construct $M^{*}_{x}$ such
that 

\begin{mathpar}
  M^{*}_{x} | \lift{x}{P} \red M[P]
\end{mathpar}

namely,

\begin{mathpar}
  M^{*}_{x} := x?(u).M[\dropn{u}]
\end{mathpar}

The dependence of $M^{*}_{x}$ on a name makes it an abstraction, 

\begin{mathpar}
  M^{*} := (x)x?(u).M[\dropn{u}]
\end{mathpar}

\subsection{Additional notation}

It will sometimes be convenient to denote the process a name
quotes. We already have the notation $x = \quotep{P}$, but it will be
convenient to introduce an alternate notation, $\procn{x}$, when we
want to emphasize the connection to the use of the name. Note that, by
virtue of name equivalence, $\quotep{\procn{x}} \nameeq x$; so, the
notation is consistent with previous definitions.

Further, because names have structure it is possible to effect
substitutions on the basis of that structure. This means we need to
upgrade our notation for substitutions, which we accomplish by
adapting comprehension notation. Thus,

\begin{mathpar}
  P\{ y / x : x \in S \}
\end{mathpar}

is interpreted to mean the process derived from P by replacing (in a
capture-avoiding manner) each occurrence of $x$ in $S$ by $y$. For example,

\begin{mathpar}
  P\{ \quotep{\procn{x}|\procn{x}} / x : x \in \freenames{P} \}
\end{mathpar}

will replace each (occurrence) of a free name $x$ in $P$ by
$\quotep{\procn{x}|\procn{x}}$.

Also, we will avail ourselves of the notation $x^{L}$ and $x^{R}$ to
denote injections of a name into disjoint copies of the name
space. There are numerous ways to accomplish this. One example can be
found in \cite{MeredithR05}. This notation overloads to vectors of
names: $\vec{x}^{\pi} := (x_{i}^{\pi} \; : \; 0 \leq i < |\vec{x}| )$ where $\pi \in \{L,R\}$.

We also use $P^{\Box} := P|\Box$.

In \cite{MeredithR05} an interpretation of the new operator is
given. It turns out that there are several possible interpretations
all enjoying the requisite algebraic properties of the operator (see
\cite{milner91polyadicpi}). We will therefore make liberal use of
$(\nu\; \vec{x})P$.

% subsection the_syntax_and_semantics_of_the_notation_system (end)   

\input{qm2pi.qmops} 

\input{qm2pi.sterngerlach} 

\input{qm2pi.metric} 

% section concurrent_process_calculi (end)

%\input{qm2pi.proofsketch}

% section proof sketch (end)

%\input{qm2pi.slviaknots} 

% section spatial logic via knots (end)

\input{qm2pi.conclusion}

% section conclusion (end)

%\input{qm2pi.dtcodes} 

% section wiring algorithm (end)

\input{qm2pi.ack} 

% section acknowledgments (end)

\newpage


\bibliographystyle{plain}   
\bibliography{../../biblios/main.bib}

\input{qm2pi.rhodetails}

\end{document}

 

% section notation (end)

\input{qm2pi.process.calculi} 

% section concurrent_process_calculi_and_spatial_logics_ (end)
    
%\documentclass[12pt]{llncs}
%\documentclass{jktr}

\usepackage[pdftex]{hyperref}                   
\usepackage {listings}
\usepackage {mathpartir}
\usepackage{bcprules}
%\usepackage{listings}
                       
\usepackage{graphicx} 
%\usepackage[margins=2.5cm,nohead,nofoot]{geometry}
%\usepackage{geometry}
\usepackage{amsfonts}
\usepackage{amstext}
\usepackage{latexsym}
\usepackage{amssymb}
\usepackage{color}


%\include{myPreamble}
\include{qm2pi.local} 

%\ifpdf
%\usepackage[pdftex]{graphicx}
%\else
%\usepackage{graphicx}
%\fi

 % \ifpdf
%  \usepackage{pdfsync}
%  \if


%\title{Brief Article}
%\author{David F. Snyder}
%\author{L.G. Meredith}

%\address{Dept. of Math., Texas State University--San Marcos, San Marcos, TX 78666}
       
\pagestyle{empty}


\begin{document}

\lstset{language=[Objective]Caml,frame=shadowbox}

\input{qm2pi.front}

% section front matter (end)

\input{qm2pi.intro} 
 
% section introduction (end)

% \input{qm2pi.knotations} 

% section notation (end)

\input{qm2pi.process.calculi} 

% section concurrent_process_calculi_and_spatial_logics_ (end)
    
%\input{qm2pi.knots2pi} 

%\input{qm2pi.trefoil} 

%\input{qm2pi.mainthm} 

% subsection basic_interpretation (end)

%\input{qm2pi.rho.presentation} 
\subsection{The syntax and semantics of the notation system}\label{sub:the_syntax_and_semantics_of_the_notation_system} % (fold)

We now summarize a technical presentation of the calculus that
embodies our theory of dynamics. The typical presentation of such a
calculus follows the style of giving generators and relations on
them. The grammar, below, describing term constructors, freely
generates the set of processes, $\Proc$. This set is then quotiented
by a relation known as structural congruence and it is over this set
that the notion of dynamics is expressed. This presentation is
essentially that of \cite{MeredithR05} with the addition of
polyadicity and summation. For readability we have relegated some of
the technical subtleties to an appendix.

\subsubsection{Process grammar}\label{subsub:process_grammar}

\begin{mathpar}
  \inferrule* [lab=synchronization] {} {{M} \bc \pzero \;|\; x?F \;|\; x!C }
  \and
  \inferrule* [lab=abstraction] {} {{F} \bc (x)P}
  \and
  \inferrule* [lab=concretion] {} {{C} \bc \langle Q \rangle}
  \and
  \inferrule* [lab=process] {} {{P,Q} \bc M \;| \;P|Q \;|\; @{x}}
  \and
  \inferrule* [lab=name] {} {{x} \bc \quotep{P}}
\end{mathpar} 

Note that $\vec{x}$ (resp. $\vec{P}$) denotes a vector of names
(resp. processes) of length $|\vec{x}|$ (resp. $|\vec{P}|$). We adopt
the following useful abbreviations.

\begin{mathpar}
   x?(\vec{y}).P := x.(\vec{y})P \and  x\clift{\vec{P}} := x.\clift{\vec{P}}
   \and x!(y) := \lift{x}{\dropn{y}}
   \and \Pi_{i=0}^{n-1}P_i := P_0 | \ldots | P_{n-1}
\end{mathpar}

\subsubsection{Structural congruence}

\paragraph{Free and bound names and alpha-equivalence.} At the
core of structural equivalence is alpha-equivalence which identifies
process that are the same up to a change of variable. Formally, we
recognize the distinction between free and bound names. The free names
of a process, $\freenames{P}$, may be calculated recursively as
follows:

\begin{mathpar}
\freenames{\pzero} := \emptyset
  \and \\
  \freenames{x?(y).P} := \{ x \} \cup (\freenames{P} \setminus \{ y \})
  \and 
  \freenames{x!\langle P \rangle} := \{ x \} \cup \{ P \} 
  \and \\
  \freenames{P|Q} := \freenames{P} \cup \freenames{Q}
  \and \\
  \freenames{@{x}} := \{ x \}
\end{mathpar}

$\pi$
$\quotep{\pi}$

$\freenames{-} : \pi \to \mathcal{P}(\quotep{\pi})$

\begin{eqnarray*}
  \freenames{\pzero} & := & \emptyset \\
  \freenames{x?(y).P} & := & \{ x \} \cup (\freenames{P} \setminus \{ y \}) \\
  \freenames{x!\langle P \rangle} & := & \{ x \} \cup \{ P \} \\
  \freenames{P|Q} & := & \freenames{P} \cup \freenames{Q} \\
  \freenames{\dropn{x}} & := & \{ x \}
\end{eqnarray*}

The bound names of a process, $\boundnames{P}$, are those names occurring in $P$
that are not free. For example, in $x?(y).0$, the name $x$ is free, while $y$ is bound.

\begin{mathpar}
  \inferrule* [lab=monoidal-laws] {} { P|Q \equiv Q|P \and P|0 \equiv P \and P|(Q|R) \equiv (P|Q)|R }
\end{mathpar}

\begin{mathpar}
  \inferrule* [lab=alpha-equivalence] {} { (x)P \equiv (y)P\{y/x\} \and y \not\in \freenames{P} }
\end{mathpar}

\begin{definition}
Then two processes, $P,Q$, are alpha-equivalent if $P = Q\{\vec{y}/\vec{x}\}$ for
some $\vec{x} \in \boundnames{Q},\vec{y} \in \boundnames{P}$, where $Q\{\vec{y}/\vec{x}\}$
denotes the capture-avoiding substitution of $\vec{y}$ for $\vec{x}$ in $Q$.
\end{definition}

\begin{definition}
  The {\em structural congruence} \cite{SangiorgiWalker} , $\equiv$,
  between processes is the least congruence containing
  alpha-equivalence, satisfying the abelian monoid laws
  (associativity, commutativity and $\pzero$ as identity) for parallel
  composition $|$ and for summation $+$.
\end{definition}

\subsection{Name equivalence}

We take name equivalence, written $\nameeq$, to be the smallest
equivalence relation generated by the following rules.

\begin{mathpar}
\inferrule*[lab=Quote-drop]
{ }
{ \quotep{@{x}} \nameeq x }

\inferrule*[lab=Struct-equiv]
{ P \scong Q }
{ \quotep{P} \nameeq \quotep{Q} }
\end{mathpar}

The astute reader will have noticed that the mutual recursion of names
and processes imposes a mutual recursion on alpha-equivalence and
structural equivalence via name-equivalence. Fortunately, all of this
works out pleasantly and we may calculate in the natural way, free of
concern. The reader interested in the details is referred to the
appendix \ref{appendix:rho_details}.

\subsection{Substitution}

We use $\Proc$ for the set of processes, $\QProc$ for the set of
names, and $\id{\{}\vec{y} / \vec{x} \id{\}}$ to denote partial maps,
$s : \QProc \rightarrow \QProc$. A map, $s$ lifts, uniquely, to a map
on process terms, $\widehat{s} : \Proc \rightarrow \Proc$ by the
following equations.

\begin{mathpar}
  (0) \psubstp{Q}{P} := 0 \\
  (R \juxtap S) \psubstp{Q}{P}
  :=    
  (R)\psubstp{Q}{P} \juxtap (S) \psubstp{Q}{P} \\
  (x?(y).R) \psubstp{Q}{P}    
  :=    
  (x)\substp{Q}{P} (z)\concat( (R \psubstn{z}{y}) \psubstp{Q}{P} ) \\
  (\lift{x}{R}) \psubstp{Q}{P}  
  :=
  \lift{(x)\substp{Q}{P}}{ R \psubstp{Q}{P} } \\
%   (\dropn{x})  \psubstp{Q}{P}       
%   := 
%   \left\{ 
%     \begin{array}{ccc} 
%       \dropn{\quotep{Q}} & & x \nameeq \quotep{P} \\
%       \dropn{x} & & otherwise \\
%     \end{array}
%   \right. 
  (\dropn{x})  \psubstp{Q}{P}       
  := 
  \left\{ 
    \begin{array}{ccc} 
      Q & & x \nameeq \quotep{P} \\
      \dropn{x} & & otherwise \\
    \end{array}
  \right.
\end{mathpar}
 

where

\begin{eqnarray}
  (x)\id{\{} \lpquote Q \rpquote / \lpquote P \rpquote \id{\}}            = 
  \left\{ 
    \begin{array}{ccc}
      \lpquote Q \rpquote & & x \nameeq \lpquote P \rpquote \\
      x & & otherwise \\
    \end{array}
  \right. \nonumber
\end{eqnarray}

and $z$ is chosen distinct from $\quotep{P}$, $\quotep{Q}$, the free
names in $Q$, and all the names in $R$. Our $\alpha$-equivalence will
be built in the standard way from this substitution.

\begin{remark}\label{rem:no_self_referential_names}
  One consequence of these definitions is that $\forall P. \quotep{P}
  \not\in \freenames{P}$.
\end{remark}

\subsection{ Dynamic quote: an example }

Anticipating something of what's to come, consider applying the
substitution, $\widehat{\id{\{}u / z \id{\}}}$, to the following pair
of processes, $\lift{w}{y!(z)}$ and $w[ \lpquote y!(z) \rpquote ]$.

\begin{eqnarray}
	\lift{w}{y!(z)}\widehat{\id{\{}u / z \id{\}}}
		& = &
		\lift{w}{y!(u)} \nonumber\\
	w[ \lpquote y!(z) \rpquote ] \widehat{ \id{\{}u / z \id{\}} }
		& = &
		w[ \lpquote y!(z) \rpquote ] \nonumber
\end{eqnarray}

Because the body of the process between quotes is impervious to
substitution, we get radically different answers. In fact, by
examining the first process in an input context,
e.g. $x?(z).\lift{w}{y!(z)}$, we see that the process under the lift
operator may be shaped by prefixed inputs binding a name inside it. In
this sense, the lift operator will be seen as a way to dynamically
construct processes before reifying them as names.

Finally equipped with these standard features we can present the
dynamics of the calculus.

\subsubsection{Operational semantics} 

Finally, we introduce the computational dynamics. What marks these
algebras as distinct from other more traditionally studied algebraic
structures, e.g. vector spaces or polynomial rings, is the manner in
which dynamics is captured. In traditional structures, dynamics is typically
expressed through morphisms between such structures, as in linear maps
between vector spaces or morphisms between rings. In algebras
associated with the semantics of computation, the dynamics is
expressed as part of the algebraic structure itself, through a
reduction reduction relation typically denoted by $\red$. Below, we
give a recursive presentation of this relation for the calculus used
in the encoding.

$\red \subseteq \pi \times \pi$
$\red : \pi \to \mathcal{P}(\pi)$

\begin{mathpar}
  \inferrule* [lab=Comm] { \textsf{match}( x_{src}, x_{trgt} ) } { x_{trgt}?(y)P \; | \; x_{src}!\langle {Q} \rangle \red P\{\quotep{Q}/y}\} }
  \and \\
  \inferrule* [lab=Par] {{P} \red {P}'} {{{P} | {Q}} \red {{P}' | {Q}}}
  \and
  \inferrule* [lab=Equiv]{{{P} \scong {P}'} \andalso {{P}' \red {Q}'} \andalso {{Q}' \scong {Q}}}{{P} \red {Q}}
\end{mathpar}

\begin{eqnarray*}
  match_{\equiv} (\quotep{P},\quotep{Q}) & := & P \equiv Q \\
  match_{\dagger}(\quotep{P},\quotep{Q}) & := & \forall R. P|Q \red^{*} R => R \red^{*} 0 \\
  match_{K}(\quotep{P},\quotep{Q}) & := & K \mbox{ for some context } K
\end{eqnarray*}

$u?(x)P | u!\langle Q \rangle \red P\{\quotep{Q}/x\}$

%We write $\wred$ for $\red^*$, and $P\red$ if $\exists Q $ such that $ P \red Q$.
We write $P\red$ if $\exists Q $ such that $ P \red Q$ and $P\not\red$, otherwise.

\section{Replication}

As mentioned before, it is known that replication (and hence
recursion) can be implemented in a higher-order process algebra
\cite{SangiorgiWalker}. As our first example of calculation with the
machinery thus far presented we give the construction explicitly in
the {\rhoc}.

\begin{eqnarray}
	D_{x} & := & \prefix{x}{y}{(\binpar{\outputp{x}{y}}{@{y}})} \nonumber\\
	\bangp_{x}{P} & := & \binpar{{x}!\langle{\binpar{D_{x}}{P}}\rangle}{D_{x}} \nonumber
\end{eqnarray}

\begin{eqnarray}
	\bangp_{x}{P} & & \nonumber\\
	=
	& {x}!\langle{(\prefix{x}{y}{(\outputp{x}{y} | @{y})) | P}}\rangle 
	      | \prefix{x}{y}{(\outputp{x}{y} | @{y})} & \nonumber\\
	\red
	& (\outputp{x}{y} | @{y})\substn{\quotep{(\prefix{x}{y}{(@{y} | \outputp{x}{y})) | P}}}{y} & \nonumber\\
	=
	& \outputp{x}{\quotep{(\prefix{x}{y}{(\outputp{x}{y} | @{y})) | P}}}
	  | {(\prefix{x}{y}{(\outputp{x}{y} | @{y})) | P}} & \nonumber\\
	\red
	& \ldots & \nonumber\\
	\red^*
	& P | P | \ldots & \nonumber
\end{eqnarray}

Of course, this encoding, as an implementation, runs away, unfolding
$\bangp{P}$ eagerly. A lazier and more implementable replication
operator, restricted to input-guarded processes, may be obtained as follows.

\begin{eqnarray}
\bangp{\prefix{u}{v}{P}} 
	:= 
	\binpar{\lift{x}{\prefix{u}{v}{(\binpar{D(x)}{P})}}}{D(x)} \nonumber
\end{eqnarray}

\begin{remark}
  Note that the lazier definition still does not deal with summation
  or mixed summation (i.e. sums over input and output). The reader is
  invited to construct definitions of replication that deal with these
  features. 

  Further, the definitions are parameterized in a name, $x$. Can you,
  gentle reader, make a definition that eliminates this parameter and
  guarantees no accidental interaction between the replication
  machinery and the process being replicated -- i.e. no accidental
  sharing of names used by the process to get its work done and the
  name(s) used by the replication to effect copying. This latter
  revision of the definition of replication is crucial to obtaining
  the expected identity $!!P \sim !P$.
\end{remark}

\begin{remark}\label{rem:paradoxical_combinator}
  The reader familiar with the lambda calculus will have noticed the
  similarity between $D$ and the paradoxical combinator.

  [Ed. note: the existence of this seems to suggest we have to be more
  restrictive on the set of processes and names we admit if we are to
  support no-cloning.]
\end{remark}

\subsubsection{Bisimulation}

The computational dynamics gives rise to another kind of equivalence,
the equivalence of computational behavior. As previously mentioned
this is typically captured \emph{via} some form of bisimulation.

% The notion we use in this paper is weak barbed bisimulation
% \cite{milner91polyadicpi}.

The notion we use in this paper is derived from weak barbed
bisimulation \cite{milner91polyadicpi}. 

\begin{definition}
An \emph{observation relation}, $\downarrow_{\mathcal N}$, over a set
of names, $\mathcal N$, is the smallest relation satisfying the rules
below.

\infrule[Out-barb]{y \in {\mathcal N}, \; x \nameeq y}
		  {\outputp{x}{v} \downarrow_{\mathcal N} x}
\infrule[Par-barb]{\mbox{$P\downarrow_{\mathcal N} x$ or $Q\downarrow_{\mathcal N} x$}}
		  {\binpar{P}{Q} \downarrow_{\mathcal N} x}

We write $P \Downarrow_{\mathcal N} x$ if there is $Q$ such that 
$P \wred Q$ and $Q \downarrow_{\mathcal N} x$.
\end{definition}

\begin{definition}
%\label{def.bbisim}
An  ${\mathcal N}$-\emph{barbed bisimulation} over a set of names, ${\mathcal N}$, is a symmetric binary relation 
${\mathcal S}_{\mathcal N}$ between agents such that $P\rel{S}_{\mathcal N}Q$ implies:
\begin{enumerate}
\item If $P \red P'$ then $Q \wred Q'$ and $P'\rel{S}_{\mathcal N} Q'$.
\item If $P\downarrow_{\mathcal N} x$, then $Q\Downarrow_{\mathcal N} x$.
\end{enumerate}
$P$ is ${\mathcal N}$-barbed bisimilar to $Q$, written
$P \wbbisim_{\mathcal N} Q$, if $P \rel{S}_{\mathcal N} Q$ for some ${\mathcal N}$-barbed bisimulation ${\mathcal S}_{\mathcal N}$.
\end{definition}

$\mathcal{R} \subseteq \pi \times \pi$

$P \mathcal{R} Q => \forall P'. P \red P' \Rightarrow \exists Q'. Q \red Q', P' \mathcal{R} Q'$

$P \vdash x \Rightarrow Q \vdash x$

\begin{mathpar}
  \inferrule*[lab=Out-barb]{x \nameeq y}{{y}!\langle{Q}\rangle \vdash x}
  \and
  \inferrule*[lab=Par-barb]{\mbox{$P\vdash x$ or $Q\vdash x$}}{\binpar{P}{Q} \vdash x}
\end{mathpar}

\subsubsection{Contexts}

One of the principle advantages of computational calculi like the
$\pi$-calculus is a well-defined notion of context,
contextual-equivalence and a correlation between
contextual-equivalence and notions of bisimulation. The notion of
context allows the decomposition of a process into (sub-)process and
its syntactic environment, its context. Thus, a context may be
thought of as a process with a ``hole'' (written $\Box$) in it. The
application of a context $M$ to a process $P$, written $M[P]$, is
tantamount to filling the hole in $M$ with $P$. In this paper we do
not need the full weight of this theory, but do make use of the notion
of context in the proof the main theorem. 

\begin{mathpar}
  \inferrule* [lab=summation] {} {{M_{M},M_{N}} \bc \Box \;|\; x.M_{A} \;|\; M_{M}+M_{N}}
  \and
  \inferrule* [lab=agent] {} {{M_{A}} \bc (\vec{x})M_{P} \;| \; \clift{P_0,\ldots,M_{P},\ldots,P_N}}
  \and \\
  \inferrule* [lab=process] {} {{M_{P}} \bc M_{N} \;| \;P|M_{P} }
\end{mathpar} 

\begin{mathpar}
  \inferrule* [lab=sychronization] {} {M_{N} \bc \Box \;|\; x?M_{F} \;|\; x!M_{C}}
  \and
  \inferrule* [lab=abstraction] {} {{M_{F}} \bc (x)M_{P} }
  \and
  \inferrule* [lab=concretion] {} {{M_{C}} \bc \langle M_{P} \rangle }
  \and \\
  \inferrule* [lab=process] {} {{M_{P}} \bc M_{N} \;| \;P|M_{P} }
\end{mathpar}

\begin{definition}[contextual application] Given a context $M$, and
  process $P$, we define the \emph{contextual application}, $M[P] :=
  M\{P/\Box\}$. That is, the contextual application of M to P is the
  substitution of $P$ for $\Box$ in $M$.
\end{definition}

$\meaningof{-} : L \to \mathcal{P}(\pi)$

\begin{mathpar}
  \inferrule* [lab=collection] {} {\meaningof{true} = \pi, \and \meaningof{~E} = \pi \setminus \meaningof{E}, \and \meaningof{E_{1} \& E_{2}} = \meaningof{E_{1}} \cap \meaningof{E_{2}}}
\end{mathpar}

\begin{mathpar}
  \inferrule* [lab=structure] {} {\meaningof{0} = \{ P \in \pi | P \equiv 0 \}, \and \\ \meaningof{E_1 | E_2} = \{ P \in \pi | P \equiv P_{1} | P_{2}, P_{1} \in \meaningof{E_{1}}, P_{2} \in \meaningof{E_2}\} }
\end{mathpar}

\begin{mathpar}
 \inferrule* [lab=behavior] {} {\meaningof{\langle a?b \rangle E} = \{ P \in \pi | P \equiv Q | u?(y)P', \\ \and \\\\ \and \\ \;\;\; u \in \meaningof{a}, \forall z.P'\{z/y\} \in \meaningof{E\{z/b\}}\}, \and \\ \meaningof{a!E} = \{ P \in \pi | P \equiv Q | x!\langle P' \rangle, x \in \meaningof{a} P' \in \meaningof{E}\} }
\end{mathpar}

\begin{mathpar}
 \inferrule* [lab=nominal] {} {\meaningof{\quotep{E}} = \{ \quotep{P} \in \quotep{\pi} | P \in \meaningof{E} \}, \and \meaningof{\quotep{P}} = \{ \quotep{Q} \in \quotep{\pi} | P \equiv Q \} \and \\ \meaningof{@\quotep{E}} = \{ P \in \pi | P \equiv @x, x \in \meaningof{E} \}}
\end{mathpar}

\begin{eqnarray*}
  \\
  \meaningof{-} : TS \to ST
\end{eqnarray*}

\begin{eqnarray*}
  \\
  L : TS \to ST
\end{eqnarray*}

\begin{eqnarray*}
  \\
  P \models E \iff P \in \meaningof{E}
\end{eqnarray*}

\begin{eqnarray*}
  P \approx_{L} Q \iff \forall E \in L. P \models E \iff Q \models E
\end{eqnarray*}

\begin{eqnarray*}
  P \approx_{K} Q
\end{eqnarray*}

\begin{eqnarray*}
  P \approx Q
\end{eqnarray*}

$\approx_{K} = \approx = \approx_{L}$

\subsubsection{Contextual duality}

Note that contexts extend the quotation operation to a family of
operations from processes to names. Given a context, $M$, we can
define a \emph{nominal context}, $\quotep{M}$ by $\quotep{M}[P] :=
\quotep{M[P]}$. To foreshadow what is to come we observe that these
operations enjoy a duality with processes very much like the duality
between vectors and maps from vectors to scalars.

Further, because the calculus is essentially higher-order, we have a
correspondence between contexts and processes. More specifically,
given a name $x$ and a context $M$ we can construct $M^{*}_{x}$ such
that 

\begin{mathpar}
  M^{*}_{x} | \lift{x}{P} \red M[P]
\end{mathpar}

namely,

\begin{mathpar}
  M^{*}_{x} := x?(u).M[\dropn{u}]
\end{mathpar}

The dependence of $M^{*}_{x}$ on a name makes it an abstraction, 

\begin{mathpar}
  M^{*} := (x)x?(u).M[\dropn{u}]
\end{mathpar}

\subsection{Additional notation}

It will sometimes be convenient to denote the process a name
quotes. We already have the notation $x = \quotep{P}$, but it will be
convenient to introduce an alternate notation, $\procn{x}$, when we
want to emphasize the connection to the use of the name. Note that, by
virtue of name equivalence, $\quotep{\procn{x}} \nameeq x$; so, the
notation is consistent with previous definitions.

Further, because names have structure it is possible to effect
substitutions on the basis of that structure. This means we need to
upgrade our notation for substitutions, which we accomplish by
adapting comprehension notation. Thus,

\begin{mathpar}
  P\{ y / x : x \in S \}
\end{mathpar}

is interpreted to mean the process derived from P by replacing (in a
capture-avoiding manner) each occurrence of $x$ in $S$ by $y$. For example,

\begin{mathpar}
  P\{ \quotep{\procn{x}|\procn{x}} / x : x \in \freenames{P} \}
\end{mathpar}

will replace each (occurrence) of a free name $x$ in $P$ by
$\quotep{\procn{x}|\procn{x}}$.

Also, we will avail ourselves of the notation $x^{L}$ and $x^{R}$ to
denote injections of a name into disjoint copies of the name
space. There are numerous ways to accomplish this. One example can be
found in \cite{MeredithR05}. This notation overloads to vectors of
names: $\vec{x}^{\pi} := (x_{i}^{\pi} \; : \; 0 \leq i < |\vec{x}| )$ where $\pi \in \{L,R\}$.

We also use $P^{\Box} := P|\Box$.

In \cite{MeredithR05} an interpretation of the new operator is
given. It turns out that there are several possible interpretations
all enjoying the requisite algebraic properties of the operator (see
\cite{milner91polyadicpi}). We will therefore make liberal use of
$(\nu\; \vec{x})P$.

% subsection the_syntax_and_semantics_of_the_notation_system (end)   

\input{qm2pi.qmops} 

\input{qm2pi.sterngerlach} 

\input{qm2pi.metric} 

% section concurrent_process_calculi (end)

%\input{qm2pi.proofsketch}

% section proof sketch (end)

%\input{qm2pi.slviaknots} 

% section spatial logic via knots (end)

\input{qm2pi.conclusion}

% section conclusion (end)

%\input{qm2pi.dtcodes} 

% section wiring algorithm (end)

\input{qm2pi.ack} 

% section acknowledgments (end)

\newpage


\bibliographystyle{plain}   
\bibliography{../../biblios/main.bib}

\input{qm2pi.rhodetails}

\end{document}

 

%\documentclass[12pt]{llncs}
%\documentclass{jktr}

\usepackage[pdftex]{hyperref}                   
\usepackage {listings}
\usepackage {mathpartir}
\usepackage{bcprules}
%\usepackage{listings}
                       
\usepackage{graphicx} 
%\usepackage[margins=2.5cm,nohead,nofoot]{geometry}
%\usepackage{geometry}
\usepackage{amsfonts}
\usepackage{amstext}
\usepackage{latexsym}
\usepackage{amssymb}
\usepackage{color}


%\include{myPreamble}
\include{qm2pi.local} 

%\ifpdf
%\usepackage[pdftex]{graphicx}
%\else
%\usepackage{graphicx}
%\fi

 % \ifpdf
%  \usepackage{pdfsync}
%  \if


%\title{Brief Article}
%\author{David F. Snyder}
%\author{L.G. Meredith}

%\address{Dept. of Math., Texas State University--San Marcos, San Marcos, TX 78666}
       
\pagestyle{empty}


\begin{document}

\lstset{language=[Objective]Caml,frame=shadowbox}

\input{qm2pi.front}

% section front matter (end)

\input{qm2pi.intro} 
 
% section introduction (end)

% \input{qm2pi.knotations} 

% section notation (end)

\input{qm2pi.process.calculi} 

% section concurrent_process_calculi_and_spatial_logics_ (end)
    
%\input{qm2pi.knots2pi} 

%\input{qm2pi.trefoil} 

%\input{qm2pi.mainthm} 

% subsection basic_interpretation (end)

%\input{qm2pi.rho.presentation} 
\subsection{The syntax and semantics of the notation system}\label{sub:the_syntax_and_semantics_of_the_notation_system} % (fold)

We now summarize a technical presentation of the calculus that
embodies our theory of dynamics. The typical presentation of such a
calculus follows the style of giving generators and relations on
them. The grammar, below, describing term constructors, freely
generates the set of processes, $\Proc$. This set is then quotiented
by a relation known as structural congruence and it is over this set
that the notion of dynamics is expressed. This presentation is
essentially that of \cite{MeredithR05} with the addition of
polyadicity and summation. For readability we have relegated some of
the technical subtleties to an appendix.

\subsubsection{Process grammar}\label{subsub:process_grammar}

\begin{mathpar}
  \inferrule* [lab=synchronization] {} {{M} \bc \pzero \;|\; x?F \;|\; x!C }
  \and
  \inferrule* [lab=abstraction] {} {{F} \bc (x)P}
  \and
  \inferrule* [lab=concretion] {} {{C} \bc \langle Q \rangle}
  \and
  \inferrule* [lab=process] {} {{P,Q} \bc M \;| \;P|Q \;|\; @{x}}
  \and
  \inferrule* [lab=name] {} {{x} \bc \quotep{P}}
\end{mathpar} 

Note that $\vec{x}$ (resp. $\vec{P}$) denotes a vector of names
(resp. processes) of length $|\vec{x}|$ (resp. $|\vec{P}|$). We adopt
the following useful abbreviations.

\begin{mathpar}
   x?(\vec{y}).P := x.(\vec{y})P \and  x\clift{\vec{P}} := x.\clift{\vec{P}}
   \and x!(y) := \lift{x}{\dropn{y}}
   \and \Pi_{i=0}^{n-1}P_i := P_0 | \ldots | P_{n-1}
\end{mathpar}

\subsubsection{Structural congruence}

\paragraph{Free and bound names and alpha-equivalence.} At the
core of structural equivalence is alpha-equivalence which identifies
process that are the same up to a change of variable. Formally, we
recognize the distinction between free and bound names. The free names
of a process, $\freenames{P}$, may be calculated recursively as
follows:

\begin{mathpar}
\freenames{\pzero} := \emptyset
  \and \\
  \freenames{x?(y).P} := \{ x \} \cup (\freenames{P} \setminus \{ y \})
  \and 
  \freenames{x!\langle P \rangle} := \{ x \} \cup \{ P \} 
  \and \\
  \freenames{P|Q} := \freenames{P} \cup \freenames{Q}
  \and \\
  \freenames{@{x}} := \{ x \}
\end{mathpar}

$\pi$
$\quotep{\pi}$

$\freenames{-} : \pi \to \mathcal{P}(\quotep{\pi})$

\begin{eqnarray*}
  \freenames{\pzero} & := & \emptyset \\
  \freenames{x?(y).P} & := & \{ x \} \cup (\freenames{P} \setminus \{ y \}) \\
  \freenames{x!\langle P \rangle} & := & \{ x \} \cup \{ P \} \\
  \freenames{P|Q} & := & \freenames{P} \cup \freenames{Q} \\
  \freenames{\dropn{x}} & := & \{ x \}
\end{eqnarray*}

The bound names of a process, $\boundnames{P}$, are those names occurring in $P$
that are not free. For example, in $x?(y).0$, the name $x$ is free, while $y$ is bound.

\begin{mathpar}
  \inferrule* [lab=monoidal-laws] {} { P|Q \equiv Q|P \and P|0 \equiv P \and P|(Q|R) \equiv (P|Q)|R }
\end{mathpar}

\begin{mathpar}
  \inferrule* [lab=alpha-equivalence] {} { (x)P \equiv (y)P\{y/x\} \and y \not\in \freenames{P} }
\end{mathpar}

\begin{definition}
Then two processes, $P,Q$, are alpha-equivalent if $P = Q\{\vec{y}/\vec{x}\}$ for
some $\vec{x} \in \boundnames{Q},\vec{y} \in \boundnames{P}$, where $Q\{\vec{y}/\vec{x}\}$
denotes the capture-avoiding substitution of $\vec{y}$ for $\vec{x}$ in $Q$.
\end{definition}

\begin{definition}
  The {\em structural congruence} \cite{SangiorgiWalker} , $\equiv$,
  between processes is the least congruence containing
  alpha-equivalence, satisfying the abelian monoid laws
  (associativity, commutativity and $\pzero$ as identity) for parallel
  composition $|$ and for summation $+$.
\end{definition}

\subsection{Name equivalence}

We take name equivalence, written $\nameeq$, to be the smallest
equivalence relation generated by the following rules.

\begin{mathpar}
\inferrule*[lab=Quote-drop]
{ }
{ \quotep{@{x}} \nameeq x }

\inferrule*[lab=Struct-equiv]
{ P \scong Q }
{ \quotep{P} \nameeq \quotep{Q} }
\end{mathpar}

The astute reader will have noticed that the mutual recursion of names
and processes imposes a mutual recursion on alpha-equivalence and
structural equivalence via name-equivalence. Fortunately, all of this
works out pleasantly and we may calculate in the natural way, free of
concern. The reader interested in the details is referred to the
appendix \ref{appendix:rho_details}.

\subsection{Substitution}

We use $\Proc$ for the set of processes, $\QProc$ for the set of
names, and $\id{\{}\vec{y} / \vec{x} \id{\}}$ to denote partial maps,
$s : \QProc \rightarrow \QProc$. A map, $s$ lifts, uniquely, to a map
on process terms, $\widehat{s} : \Proc \rightarrow \Proc$ by the
following equations.

\begin{mathpar}
  (0) \psubstp{Q}{P} := 0 \\
  (R \juxtap S) \psubstp{Q}{P}
  :=    
  (R)\psubstp{Q}{P} \juxtap (S) \psubstp{Q}{P} \\
  (x?(y).R) \psubstp{Q}{P}    
  :=    
  (x)\substp{Q}{P} (z)\concat( (R \psubstn{z}{y}) \psubstp{Q}{P} ) \\
  (\lift{x}{R}) \psubstp{Q}{P}  
  :=
  \lift{(x)\substp{Q}{P}}{ R \psubstp{Q}{P} } \\
%   (\dropn{x})  \psubstp{Q}{P}       
%   := 
%   \left\{ 
%     \begin{array}{ccc} 
%       \dropn{\quotep{Q}} & & x \nameeq \quotep{P} \\
%       \dropn{x} & & otherwise \\
%     \end{array}
%   \right. 
  (\dropn{x})  \psubstp{Q}{P}       
  := 
  \left\{ 
    \begin{array}{ccc} 
      Q & & x \nameeq \quotep{P} \\
      \dropn{x} & & otherwise \\
    \end{array}
  \right.
\end{mathpar}
 

where

\begin{eqnarray}
  (x)\id{\{} \lpquote Q \rpquote / \lpquote P \rpquote \id{\}}            = 
  \left\{ 
    \begin{array}{ccc}
      \lpquote Q \rpquote & & x \nameeq \lpquote P \rpquote \\
      x & & otherwise \\
    \end{array}
  \right. \nonumber
\end{eqnarray}

and $z$ is chosen distinct from $\quotep{P}$, $\quotep{Q}$, the free
names in $Q$, and all the names in $R$. Our $\alpha$-equivalence will
be built in the standard way from this substitution.

\begin{remark}\label{rem:no_self_referential_names}
  One consequence of these definitions is that $\forall P. \quotep{P}
  \not\in \freenames{P}$.
\end{remark}

\subsection{ Dynamic quote: an example }

Anticipating something of what's to come, consider applying the
substitution, $\widehat{\id{\{}u / z \id{\}}}$, to the following pair
of processes, $\lift{w}{y!(z)}$ and $w[ \lpquote y!(z) \rpquote ]$.

\begin{eqnarray}
	\lift{w}{y!(z)}\widehat{\id{\{}u / z \id{\}}}
		& = &
		\lift{w}{y!(u)} \nonumber\\
	w[ \lpquote y!(z) \rpquote ] \widehat{ \id{\{}u / z \id{\}} }
		& = &
		w[ \lpquote y!(z) \rpquote ] \nonumber
\end{eqnarray}

Because the body of the process between quotes is impervious to
substitution, we get radically different answers. In fact, by
examining the first process in an input context,
e.g. $x?(z).\lift{w}{y!(z)}$, we see that the process under the lift
operator may be shaped by prefixed inputs binding a name inside it. In
this sense, the lift operator will be seen as a way to dynamically
construct processes before reifying them as names.

Finally equipped with these standard features we can present the
dynamics of the calculus.

\subsubsection{Operational semantics} 

Finally, we introduce the computational dynamics. What marks these
algebras as distinct from other more traditionally studied algebraic
structures, e.g. vector spaces or polynomial rings, is the manner in
which dynamics is captured. In traditional structures, dynamics is typically
expressed through morphisms between such structures, as in linear maps
between vector spaces or morphisms between rings. In algebras
associated with the semantics of computation, the dynamics is
expressed as part of the algebraic structure itself, through a
reduction reduction relation typically denoted by $\red$. Below, we
give a recursive presentation of this relation for the calculus used
in the encoding.

$\red \subseteq \pi \times \pi$
$\red : \pi \to \mathcal{P}(\pi)$

\begin{mathpar}
  \inferrule* [lab=Comm] { \textsf{match}( x_{src}, x_{trgt} ) } { x_{trgt}?(y)P \; | \; x_{src}!\langle {Q} \rangle \red P\{\quotep{Q}/y}\} }
  \and \\
  \inferrule* [lab=Par] {{P} \red {P}'} {{{P} | {Q}} \red {{P}' | {Q}}}
  \and
  \inferrule* [lab=Equiv]{{{P} \scong {P}'} \andalso {{P}' \red {Q}'} \andalso {{Q}' \scong {Q}}}{{P} \red {Q}}
\end{mathpar}

\begin{eqnarray*}
  match_{\equiv} (\quotep{P},\quotep{Q}) & := & P \equiv Q \\
  match_{\dagger}(\quotep{P},\quotep{Q}) & := & \forall R. P|Q \red^{*} R => R \red^{*} 0 \\
  match_{K}(\quotep{P},\quotep{Q}) & := & K \mbox{ for some context } K
\end{eqnarray*}

$u?(x)P | u!\langle Q \rangle \red P\{\quotep{Q}/x\}$

%We write $\wred$ for $\red^*$, and $P\red$ if $\exists Q $ such that $ P \red Q$.
We write $P\red$ if $\exists Q $ such that $ P \red Q$ and $P\not\red$, otherwise.

\section{Replication}

As mentioned before, it is known that replication (and hence
recursion) can be implemented in a higher-order process algebra
\cite{SangiorgiWalker}. As our first example of calculation with the
machinery thus far presented we give the construction explicitly in
the {\rhoc}.

\begin{eqnarray}
	D_{x} & := & \prefix{x}{y}{(\binpar{\outputp{x}{y}}{@{y}})} \nonumber\\
	\bangp_{x}{P} & := & \binpar{{x}!\langle{\binpar{D_{x}}{P}}\rangle}{D_{x}} \nonumber
\end{eqnarray}

\begin{eqnarray}
	\bangp_{x}{P} & & \nonumber\\
	=
	& {x}!\langle{(\prefix{x}{y}{(\outputp{x}{y} | @{y})) | P}}\rangle 
	      | \prefix{x}{y}{(\outputp{x}{y} | @{y})} & \nonumber\\
	\red
	& (\outputp{x}{y} | @{y})\substn{\quotep{(\prefix{x}{y}{(@{y} | \outputp{x}{y})) | P}}}{y} & \nonumber\\
	=
	& \outputp{x}{\quotep{(\prefix{x}{y}{(\outputp{x}{y} | @{y})) | P}}}
	  | {(\prefix{x}{y}{(\outputp{x}{y} | @{y})) | P}} & \nonumber\\
	\red
	& \ldots & \nonumber\\
	\red^*
	& P | P | \ldots & \nonumber
\end{eqnarray}

Of course, this encoding, as an implementation, runs away, unfolding
$\bangp{P}$ eagerly. A lazier and more implementable replication
operator, restricted to input-guarded processes, may be obtained as follows.

\begin{eqnarray}
\bangp{\prefix{u}{v}{P}} 
	:= 
	\binpar{\lift{x}{\prefix{u}{v}{(\binpar{D(x)}{P})}}}{D(x)} \nonumber
\end{eqnarray}

\begin{remark}
  Note that the lazier definition still does not deal with summation
  or mixed summation (i.e. sums over input and output). The reader is
  invited to construct definitions of replication that deal with these
  features. 

  Further, the definitions are parameterized in a name, $x$. Can you,
  gentle reader, make a definition that eliminates this parameter and
  guarantees no accidental interaction between the replication
  machinery and the process being replicated -- i.e. no accidental
  sharing of names used by the process to get its work done and the
  name(s) used by the replication to effect copying. This latter
  revision of the definition of replication is crucial to obtaining
  the expected identity $!!P \sim !P$.
\end{remark}

\begin{remark}\label{rem:paradoxical_combinator}
  The reader familiar with the lambda calculus will have noticed the
  similarity between $D$ and the paradoxical combinator.

  [Ed. note: the existence of this seems to suggest we have to be more
  restrictive on the set of processes and names we admit if we are to
  support no-cloning.]
\end{remark}

\subsubsection{Bisimulation}

The computational dynamics gives rise to another kind of equivalence,
the equivalence of computational behavior. As previously mentioned
this is typically captured \emph{via} some form of bisimulation.

% The notion we use in this paper is weak barbed bisimulation
% \cite{milner91polyadicpi}.

The notion we use in this paper is derived from weak barbed
bisimulation \cite{milner91polyadicpi}. 

\begin{definition}
An \emph{observation relation}, $\downarrow_{\mathcal N}$, over a set
of names, $\mathcal N$, is the smallest relation satisfying the rules
below.

\infrule[Out-barb]{y \in {\mathcal N}, \; x \nameeq y}
		  {\outputp{x}{v} \downarrow_{\mathcal N} x}
\infrule[Par-barb]{\mbox{$P\downarrow_{\mathcal N} x$ or $Q\downarrow_{\mathcal N} x$}}
		  {\binpar{P}{Q} \downarrow_{\mathcal N} x}

We write $P \Downarrow_{\mathcal N} x$ if there is $Q$ such that 
$P \wred Q$ and $Q \downarrow_{\mathcal N} x$.
\end{definition}

\begin{definition}
%\label{def.bbisim}
An  ${\mathcal N}$-\emph{barbed bisimulation} over a set of names, ${\mathcal N}$, is a symmetric binary relation 
${\mathcal S}_{\mathcal N}$ between agents such that $P\rel{S}_{\mathcal N}Q$ implies:
\begin{enumerate}
\item If $P \red P'$ then $Q \wred Q'$ and $P'\rel{S}_{\mathcal N} Q'$.
\item If $P\downarrow_{\mathcal N} x$, then $Q\Downarrow_{\mathcal N} x$.
\end{enumerate}
$P$ is ${\mathcal N}$-barbed bisimilar to $Q$, written
$P \wbbisim_{\mathcal N} Q$, if $P \rel{S}_{\mathcal N} Q$ for some ${\mathcal N}$-barbed bisimulation ${\mathcal S}_{\mathcal N}$.
\end{definition}

$\mathcal{R} \subseteq \pi \times \pi$

$P \mathcal{R} Q => \forall P'. P \red P' \Rightarrow \exists Q'. Q \red Q', P' \mathcal{R} Q'$

$P \vdash x \Rightarrow Q \vdash x$

\begin{mathpar}
  \inferrule*[lab=Out-barb]{x \nameeq y}{{y}!\langle{Q}\rangle \vdash x}
  \and
  \inferrule*[lab=Par-barb]{\mbox{$P\vdash x$ or $Q\vdash x$}}{\binpar{P}{Q} \vdash x}
\end{mathpar}

\subsubsection{Contexts}

One of the principle advantages of computational calculi like the
$\pi$-calculus is a well-defined notion of context,
contextual-equivalence and a correlation between
contextual-equivalence and notions of bisimulation. The notion of
context allows the decomposition of a process into (sub-)process and
its syntactic environment, its context. Thus, a context may be
thought of as a process with a ``hole'' (written $\Box$) in it. The
application of a context $M$ to a process $P$, written $M[P]$, is
tantamount to filling the hole in $M$ with $P$. In this paper we do
not need the full weight of this theory, but do make use of the notion
of context in the proof the main theorem. 

\begin{mathpar}
  \inferrule* [lab=summation] {} {{M_{M},M_{N}} \bc \Box \;|\; x.M_{A} \;|\; M_{M}+M_{N}}
  \and
  \inferrule* [lab=agent] {} {{M_{A}} \bc (\vec{x})M_{P} \;| \; \clift{P_0,\ldots,M_{P},\ldots,P_N}}
  \and \\
  \inferrule* [lab=process] {} {{M_{P}} \bc M_{N} \;| \;P|M_{P} }
\end{mathpar} 

\begin{mathpar}
  \inferrule* [lab=sychronization] {} {M_{N} \bc \Box \;|\; x?M_{F} \;|\; x!M_{C}}
  \and
  \inferrule* [lab=abstraction] {} {{M_{F}} \bc (x)M_{P} }
  \and
  \inferrule* [lab=concretion] {} {{M_{C}} \bc \langle M_{P} \rangle }
  \and \\
  \inferrule* [lab=process] {} {{M_{P}} \bc M_{N} \;| \;P|M_{P} }
\end{mathpar}

\begin{definition}[contextual application] Given a context $M$, and
  process $P$, we define the \emph{contextual application}, $M[P] :=
  M\{P/\Box\}$. That is, the contextual application of M to P is the
  substitution of $P$ for $\Box$ in $M$.
\end{definition}

$\meaningof{-} : L \to \mathcal{P}(\pi)$

\begin{mathpar}
  \inferrule* [lab=collection] {} {\meaningof{true} = \pi, \and \meaningof{~E} = \pi \setminus \meaningof{E}, \and \meaningof{E_{1} \& E_{2}} = \meaningof{E_{1}} \cap \meaningof{E_{2}}}
\end{mathpar}

\begin{mathpar}
  \inferrule* [lab=structure] {} {\meaningof{0} = \{ P \in \pi | P \equiv 0 \}, \and \\ \meaningof{E_1 | E_2} = \{ P \in \pi | P \equiv P_{1} | P_{2}, P_{1} \in \meaningof{E_{1}}, P_{2} \in \meaningof{E_2}\} }
\end{mathpar}

\begin{mathpar}
 \inferrule* [lab=behavior] {} {\meaningof{\langle a?b \rangle E} = \{ P \in \pi | P \equiv Q | u?(y)P', \\ \and \\\\ \and \\ \;\;\; u \in \meaningof{a}, \forall z.P'\{z/y\} \in \meaningof{E\{z/b\}}\}, \and \\ \meaningof{a!E} = \{ P \in \pi | P \equiv Q | x!\langle P' \rangle, x \in \meaningof{a} P' \in \meaningof{E}\} }
\end{mathpar}

\begin{mathpar}
 \inferrule* [lab=nominal] {} {\meaningof{\quotep{E}} = \{ \quotep{P} \in \quotep{\pi} | P \in \meaningof{E} \}, \and \meaningof{\quotep{P}} = \{ \quotep{Q} \in \quotep{\pi} | P \equiv Q \} \and \\ \meaningof{@\quotep{E}} = \{ P \in \pi | P \equiv @x, x \in \meaningof{E} \}}
\end{mathpar}

\begin{eqnarray*}
  \\
  \meaningof{-} : TS \to ST
\end{eqnarray*}

\begin{eqnarray*}
  \\
  L : TS \to ST
\end{eqnarray*}

\begin{eqnarray*}
  \\
  P \models E \iff P \in \meaningof{E}
\end{eqnarray*}

\begin{eqnarray*}
  P \approx_{L} Q \iff \forall E \in L. P \models E \iff Q \models E
\end{eqnarray*}

\begin{eqnarray*}
  P \approx_{K} Q
\end{eqnarray*}

\begin{eqnarray*}
  P \approx Q
\end{eqnarray*}

$\approx_{K} = \approx = \approx_{L}$

\subsubsection{Contextual duality}

Note that contexts extend the quotation operation to a family of
operations from processes to names. Given a context, $M$, we can
define a \emph{nominal context}, $\quotep{M}$ by $\quotep{M}[P] :=
\quotep{M[P]}$. To foreshadow what is to come we observe that these
operations enjoy a duality with processes very much like the duality
between vectors and maps from vectors to scalars.

Further, because the calculus is essentially higher-order, we have a
correspondence between contexts and processes. More specifically,
given a name $x$ and a context $M$ we can construct $M^{*}_{x}$ such
that 

\begin{mathpar}
  M^{*}_{x} | \lift{x}{P} \red M[P]
\end{mathpar}

namely,

\begin{mathpar}
  M^{*}_{x} := x?(u).M[\dropn{u}]
\end{mathpar}

The dependence of $M^{*}_{x}$ on a name makes it an abstraction, 

\begin{mathpar}
  M^{*} := (x)x?(u).M[\dropn{u}]
\end{mathpar}

\subsection{Additional notation}

It will sometimes be convenient to denote the process a name
quotes. We already have the notation $x = \quotep{P}$, but it will be
convenient to introduce an alternate notation, $\procn{x}$, when we
want to emphasize the connection to the use of the name. Note that, by
virtue of name equivalence, $\quotep{\procn{x}} \nameeq x$; so, the
notation is consistent with previous definitions.

Further, because names have structure it is possible to effect
substitutions on the basis of that structure. This means we need to
upgrade our notation for substitutions, which we accomplish by
adapting comprehension notation. Thus,

\begin{mathpar}
  P\{ y / x : x \in S \}
\end{mathpar}

is interpreted to mean the process derived from P by replacing (in a
capture-avoiding manner) each occurrence of $x$ in $S$ by $y$. For example,

\begin{mathpar}
  P\{ \quotep{\procn{x}|\procn{x}} / x : x \in \freenames{P} \}
\end{mathpar}

will replace each (occurrence) of a free name $x$ in $P$ by
$\quotep{\procn{x}|\procn{x}}$.

Also, we will avail ourselves of the notation $x^{L}$ and $x^{R}$ to
denote injections of a name into disjoint copies of the name
space. There are numerous ways to accomplish this. One example can be
found in \cite{MeredithR05}. This notation overloads to vectors of
names: $\vec{x}^{\pi} := (x_{i}^{\pi} \; : \; 0 \leq i < |\vec{x}| )$ where $\pi \in \{L,R\}$.

We also use $P^{\Box} := P|\Box$.

In \cite{MeredithR05} an interpretation of the new operator is
given. It turns out that there are several possible interpretations
all enjoying the requisite algebraic properties of the operator (see
\cite{milner91polyadicpi}). We will therefore make liberal use of
$(\nu\; \vec{x})P$.

% subsection the_syntax_and_semantics_of_the_notation_system (end)   

\input{qm2pi.qmops} 

\input{qm2pi.sterngerlach} 

\input{qm2pi.metric} 

% section concurrent_process_calculi (end)

%\input{qm2pi.proofsketch}

% section proof sketch (end)

%\input{qm2pi.slviaknots} 

% section spatial logic via knots (end)

\input{qm2pi.conclusion}

% section conclusion (end)

%\input{qm2pi.dtcodes} 

% section wiring algorithm (end)

\input{qm2pi.ack} 

% section acknowledgments (end)

\newpage


\bibliographystyle{plain}   
\bibliography{../../biblios/main.bib}

\input{qm2pi.rhodetails}

\end{document}

 

%\documentclass[12pt]{llncs}
%\documentclass{jktr}

\usepackage[pdftex]{hyperref}                   
\usepackage {listings}
\usepackage {mathpartir}
\usepackage{bcprules}
%\usepackage{listings}
                       
\usepackage{graphicx} 
%\usepackage[margins=2.5cm,nohead,nofoot]{geometry}
%\usepackage{geometry}
\usepackage{amsfonts}
\usepackage{amstext}
\usepackage{latexsym}
\usepackage{amssymb}
\usepackage{color}


%\include{myPreamble}
\include{qm2pi.local} 

%\ifpdf
%\usepackage[pdftex]{graphicx}
%\else
%\usepackage{graphicx}
%\fi

 % \ifpdf
%  \usepackage{pdfsync}
%  \if


%\title{Brief Article}
%\author{David F. Snyder}
%\author{L.G. Meredith}

%\address{Dept. of Math., Texas State University--San Marcos, San Marcos, TX 78666}
       
\pagestyle{empty}


\begin{document}

\lstset{language=[Objective]Caml,frame=shadowbox}

\input{qm2pi.front}

% section front matter (end)

\input{qm2pi.intro} 
 
% section introduction (end)

% \input{qm2pi.knotations} 

% section notation (end)

\input{qm2pi.process.calculi} 

% section concurrent_process_calculi_and_spatial_logics_ (end)
    
%\input{qm2pi.knots2pi} 

%\input{qm2pi.trefoil} 

%\input{qm2pi.mainthm} 

% subsection basic_interpretation (end)

%\input{qm2pi.rho.presentation} 
\subsection{The syntax and semantics of the notation system}\label{sub:the_syntax_and_semantics_of_the_notation_system} % (fold)

We now summarize a technical presentation of the calculus that
embodies our theory of dynamics. The typical presentation of such a
calculus follows the style of giving generators and relations on
them. The grammar, below, describing term constructors, freely
generates the set of processes, $\Proc$. This set is then quotiented
by a relation known as structural congruence and it is over this set
that the notion of dynamics is expressed. This presentation is
essentially that of \cite{MeredithR05} with the addition of
polyadicity and summation. For readability we have relegated some of
the technical subtleties to an appendix.

\subsubsection{Process grammar}\label{subsub:process_grammar}

\begin{mathpar}
  \inferrule* [lab=synchronization] {} {{M} \bc \pzero \;|\; x?F \;|\; x!C }
  \and
  \inferrule* [lab=abstraction] {} {{F} \bc (x)P}
  \and
  \inferrule* [lab=concretion] {} {{C} \bc \langle Q \rangle}
  \and
  \inferrule* [lab=process] {} {{P,Q} \bc M \;| \;P|Q \;|\; @{x}}
  \and
  \inferrule* [lab=name] {} {{x} \bc \quotep{P}}
\end{mathpar} 

Note that $\vec{x}$ (resp. $\vec{P}$) denotes a vector of names
(resp. processes) of length $|\vec{x}|$ (resp. $|\vec{P}|$). We adopt
the following useful abbreviations.

\begin{mathpar}
   x?(\vec{y}).P := x.(\vec{y})P \and  x\clift{\vec{P}} := x.\clift{\vec{P}}
   \and x!(y) := \lift{x}{\dropn{y}}
   \and \Pi_{i=0}^{n-1}P_i := P_0 | \ldots | P_{n-1}
\end{mathpar}

\subsubsection{Structural congruence}

\paragraph{Free and bound names and alpha-equivalence.} At the
core of structural equivalence is alpha-equivalence which identifies
process that are the same up to a change of variable. Formally, we
recognize the distinction between free and bound names. The free names
of a process, $\freenames{P}$, may be calculated recursively as
follows:

\begin{mathpar}
\freenames{\pzero} := \emptyset
  \and \\
  \freenames{x?(y).P} := \{ x \} \cup (\freenames{P} \setminus \{ y \})
  \and 
  \freenames{x!\langle P \rangle} := \{ x \} \cup \{ P \} 
  \and \\
  \freenames{P|Q} := \freenames{P} \cup \freenames{Q}
  \and \\
  \freenames{@{x}} := \{ x \}
\end{mathpar}

$\pi$
$\quotep{\pi}$

$\freenames{-} : \pi \to \mathcal{P}(\quotep{\pi})$

\begin{eqnarray*}
  \freenames{\pzero} & := & \emptyset \\
  \freenames{x?(y).P} & := & \{ x \} \cup (\freenames{P} \setminus \{ y \}) \\
  \freenames{x!\langle P \rangle} & := & \{ x \} \cup \{ P \} \\
  \freenames{P|Q} & := & \freenames{P} \cup \freenames{Q} \\
  \freenames{\dropn{x}} & := & \{ x \}
\end{eqnarray*}

The bound names of a process, $\boundnames{P}$, are those names occurring in $P$
that are not free. For example, in $x?(y).0$, the name $x$ is free, while $y$ is bound.

\begin{mathpar}
  \inferrule* [lab=monoidal-laws] {} { P|Q \equiv Q|P \and P|0 \equiv P \and P|(Q|R) \equiv (P|Q)|R }
\end{mathpar}

\begin{mathpar}
  \inferrule* [lab=alpha-equivalence] {} { (x)P \equiv (y)P\{y/x\} \and y \not\in \freenames{P} }
\end{mathpar}

\begin{definition}
Then two processes, $P,Q$, are alpha-equivalent if $P = Q\{\vec{y}/\vec{x}\}$ for
some $\vec{x} \in \boundnames{Q},\vec{y} \in \boundnames{P}$, where $Q\{\vec{y}/\vec{x}\}$
denotes the capture-avoiding substitution of $\vec{y}$ for $\vec{x}$ in $Q$.
\end{definition}

\begin{definition}
  The {\em structural congruence} \cite{SangiorgiWalker} , $\equiv$,
  between processes is the least congruence containing
  alpha-equivalence, satisfying the abelian monoid laws
  (associativity, commutativity and $\pzero$ as identity) for parallel
  composition $|$ and for summation $+$.
\end{definition}

\subsection{Name equivalence}

We take name equivalence, written $\nameeq$, to be the smallest
equivalence relation generated by the following rules.

\begin{mathpar}
\inferrule*[lab=Quote-drop]
{ }
{ \quotep{@{x}} \nameeq x }

\inferrule*[lab=Struct-equiv]
{ P \scong Q }
{ \quotep{P} \nameeq \quotep{Q} }
\end{mathpar}

The astute reader will have noticed that the mutual recursion of names
and processes imposes a mutual recursion on alpha-equivalence and
structural equivalence via name-equivalence. Fortunately, all of this
works out pleasantly and we may calculate in the natural way, free of
concern. The reader interested in the details is referred to the
appendix \ref{appendix:rho_details}.

\subsection{Substitution}

We use $\Proc$ for the set of processes, $\QProc$ for the set of
names, and $\id{\{}\vec{y} / \vec{x} \id{\}}$ to denote partial maps,
$s : \QProc \rightarrow \QProc$. A map, $s$ lifts, uniquely, to a map
on process terms, $\widehat{s} : \Proc \rightarrow \Proc$ by the
following equations.

\begin{mathpar}
  (0) \psubstp{Q}{P} := 0 \\
  (R \juxtap S) \psubstp{Q}{P}
  :=    
  (R)\psubstp{Q}{P} \juxtap (S) \psubstp{Q}{P} \\
  (x?(y).R) \psubstp{Q}{P}    
  :=    
  (x)\substp{Q}{P} (z)\concat( (R \psubstn{z}{y}) \psubstp{Q}{P} ) \\
  (\lift{x}{R}) \psubstp{Q}{P}  
  :=
  \lift{(x)\substp{Q}{P}}{ R \psubstp{Q}{P} } \\
%   (\dropn{x})  \psubstp{Q}{P}       
%   := 
%   \left\{ 
%     \begin{array}{ccc} 
%       \dropn{\quotep{Q}} & & x \nameeq \quotep{P} \\
%       \dropn{x} & & otherwise \\
%     \end{array}
%   \right. 
  (\dropn{x})  \psubstp{Q}{P}       
  := 
  \left\{ 
    \begin{array}{ccc} 
      Q & & x \nameeq \quotep{P} \\
      \dropn{x} & & otherwise \\
    \end{array}
  \right.
\end{mathpar}
 

where

\begin{eqnarray}
  (x)\id{\{} \lpquote Q \rpquote / \lpquote P \rpquote \id{\}}            = 
  \left\{ 
    \begin{array}{ccc}
      \lpquote Q \rpquote & & x \nameeq \lpquote P \rpquote \\
      x & & otherwise \\
    \end{array}
  \right. \nonumber
\end{eqnarray}

and $z$ is chosen distinct from $\quotep{P}$, $\quotep{Q}$, the free
names in $Q$, and all the names in $R$. Our $\alpha$-equivalence will
be built in the standard way from this substitution.

\begin{remark}\label{rem:no_self_referential_names}
  One consequence of these definitions is that $\forall P. \quotep{P}
  \not\in \freenames{P}$.
\end{remark}

\subsection{ Dynamic quote: an example }

Anticipating something of what's to come, consider applying the
substitution, $\widehat{\id{\{}u / z \id{\}}}$, to the following pair
of processes, $\lift{w}{y!(z)}$ and $w[ \lpquote y!(z) \rpquote ]$.

\begin{eqnarray}
	\lift{w}{y!(z)}\widehat{\id{\{}u / z \id{\}}}
		& = &
		\lift{w}{y!(u)} \nonumber\\
	w[ \lpquote y!(z) \rpquote ] \widehat{ \id{\{}u / z \id{\}} }
		& = &
		w[ \lpquote y!(z) \rpquote ] \nonumber
\end{eqnarray}

Because the body of the process between quotes is impervious to
substitution, we get radically different answers. In fact, by
examining the first process in an input context,
e.g. $x?(z).\lift{w}{y!(z)}$, we see that the process under the lift
operator may be shaped by prefixed inputs binding a name inside it. In
this sense, the lift operator will be seen as a way to dynamically
construct processes before reifying them as names.

Finally equipped with these standard features we can present the
dynamics of the calculus.

\subsubsection{Operational semantics} 

Finally, we introduce the computational dynamics. What marks these
algebras as distinct from other more traditionally studied algebraic
structures, e.g. vector spaces or polynomial rings, is the manner in
which dynamics is captured. In traditional structures, dynamics is typically
expressed through morphisms between such structures, as in linear maps
between vector spaces or morphisms between rings. In algebras
associated with the semantics of computation, the dynamics is
expressed as part of the algebraic structure itself, through a
reduction reduction relation typically denoted by $\red$. Below, we
give a recursive presentation of this relation for the calculus used
in the encoding.

$\red \subseteq \pi \times \pi$
$\red : \pi \to \mathcal{P}(\pi)$

\begin{mathpar}
  \inferrule* [lab=Comm] { \textsf{match}( x_{src}, x_{trgt} ) } { x_{trgt}?(y)P \; | \; x_{src}!\langle {Q} \rangle \red P\{\quotep{Q}/y}\} }
  \and \\
  \inferrule* [lab=Par] {{P} \red {P}'} {{{P} | {Q}} \red {{P}' | {Q}}}
  \and
  \inferrule* [lab=Equiv]{{{P} \scong {P}'} \andalso {{P}' \red {Q}'} \andalso {{Q}' \scong {Q}}}{{P} \red {Q}}
\end{mathpar}

\begin{eqnarray*}
  match_{\equiv} (\quotep{P},\quotep{Q}) & := & P \equiv Q \\
  match_{\dagger}(\quotep{P},\quotep{Q}) & := & \forall R. P|Q \red^{*} R => R \red^{*} 0 \\
  match_{K}(\quotep{P},\quotep{Q}) & := & K \mbox{ for some context } K
\end{eqnarray*}

$u?(x)P | u!\langle Q \rangle \red P\{\quotep{Q}/x\}$

%We write $\wred$ for $\red^*$, and $P\red$ if $\exists Q $ such that $ P \red Q$.
We write $P\red$ if $\exists Q $ such that $ P \red Q$ and $P\not\red$, otherwise.

\section{Replication}

As mentioned before, it is known that replication (and hence
recursion) can be implemented in a higher-order process algebra
\cite{SangiorgiWalker}. As our first example of calculation with the
machinery thus far presented we give the construction explicitly in
the {\rhoc}.

\begin{eqnarray}
	D_{x} & := & \prefix{x}{y}{(\binpar{\outputp{x}{y}}{@{y}})} \nonumber\\
	\bangp_{x}{P} & := & \binpar{{x}!\langle{\binpar{D_{x}}{P}}\rangle}{D_{x}} \nonumber
\end{eqnarray}

\begin{eqnarray}
	\bangp_{x}{P} & & \nonumber\\
	=
	& {x}!\langle{(\prefix{x}{y}{(\outputp{x}{y} | @{y})) | P}}\rangle 
	      | \prefix{x}{y}{(\outputp{x}{y} | @{y})} & \nonumber\\
	\red
	& (\outputp{x}{y} | @{y})\substn{\quotep{(\prefix{x}{y}{(@{y} | \outputp{x}{y})) | P}}}{y} & \nonumber\\
	=
	& \outputp{x}{\quotep{(\prefix{x}{y}{(\outputp{x}{y} | @{y})) | P}}}
	  | {(\prefix{x}{y}{(\outputp{x}{y} | @{y})) | P}} & \nonumber\\
	\red
	& \ldots & \nonumber\\
	\red^*
	& P | P | \ldots & \nonumber
\end{eqnarray}

Of course, this encoding, as an implementation, runs away, unfolding
$\bangp{P}$ eagerly. A lazier and more implementable replication
operator, restricted to input-guarded processes, may be obtained as follows.

\begin{eqnarray}
\bangp{\prefix{u}{v}{P}} 
	:= 
	\binpar{\lift{x}{\prefix{u}{v}{(\binpar{D(x)}{P})}}}{D(x)} \nonumber
\end{eqnarray}

\begin{remark}
  Note that the lazier definition still does not deal with summation
  or mixed summation (i.e. sums over input and output). The reader is
  invited to construct definitions of replication that deal with these
  features. 

  Further, the definitions are parameterized in a name, $x$. Can you,
  gentle reader, make a definition that eliminates this parameter and
  guarantees no accidental interaction between the replication
  machinery and the process being replicated -- i.e. no accidental
  sharing of names used by the process to get its work done and the
  name(s) used by the replication to effect copying. This latter
  revision of the definition of replication is crucial to obtaining
  the expected identity $!!P \sim !P$.
\end{remark}

\begin{remark}\label{rem:paradoxical_combinator}
  The reader familiar with the lambda calculus will have noticed the
  similarity between $D$ and the paradoxical combinator.

  [Ed. note: the existence of this seems to suggest we have to be more
  restrictive on the set of processes and names we admit if we are to
  support no-cloning.]
\end{remark}

\subsubsection{Bisimulation}

The computational dynamics gives rise to another kind of equivalence,
the equivalence of computational behavior. As previously mentioned
this is typically captured \emph{via} some form of bisimulation.

% The notion we use in this paper is weak barbed bisimulation
% \cite{milner91polyadicpi}.

The notion we use in this paper is derived from weak barbed
bisimulation \cite{milner91polyadicpi}. 

\begin{definition}
An \emph{observation relation}, $\downarrow_{\mathcal N}$, over a set
of names, $\mathcal N$, is the smallest relation satisfying the rules
below.

\infrule[Out-barb]{y \in {\mathcal N}, \; x \nameeq y}
		  {\outputp{x}{v} \downarrow_{\mathcal N} x}
\infrule[Par-barb]{\mbox{$P\downarrow_{\mathcal N} x$ or $Q\downarrow_{\mathcal N} x$}}
		  {\binpar{P}{Q} \downarrow_{\mathcal N} x}

We write $P \Downarrow_{\mathcal N} x$ if there is $Q$ such that 
$P \wred Q$ and $Q \downarrow_{\mathcal N} x$.
\end{definition}

\begin{definition}
%\label{def.bbisim}
An  ${\mathcal N}$-\emph{barbed bisimulation} over a set of names, ${\mathcal N}$, is a symmetric binary relation 
${\mathcal S}_{\mathcal N}$ between agents such that $P\rel{S}_{\mathcal N}Q$ implies:
\begin{enumerate}
\item If $P \red P'$ then $Q \wred Q'$ and $P'\rel{S}_{\mathcal N} Q'$.
\item If $P\downarrow_{\mathcal N} x$, then $Q\Downarrow_{\mathcal N} x$.
\end{enumerate}
$P$ is ${\mathcal N}$-barbed bisimilar to $Q$, written
$P \wbbisim_{\mathcal N} Q$, if $P \rel{S}_{\mathcal N} Q$ for some ${\mathcal N}$-barbed bisimulation ${\mathcal S}_{\mathcal N}$.
\end{definition}

$\mathcal{R} \subseteq \pi \times \pi$

$P \mathcal{R} Q => \forall P'. P \red P' \Rightarrow \exists Q'. Q \red Q', P' \mathcal{R} Q'$

$P \vdash x \Rightarrow Q \vdash x$

\begin{mathpar}
  \inferrule*[lab=Out-barb]{x \nameeq y}{{y}!\langle{Q}\rangle \vdash x}
  \and
  \inferrule*[lab=Par-barb]{\mbox{$P\vdash x$ or $Q\vdash x$}}{\binpar{P}{Q} \vdash x}
\end{mathpar}

\subsubsection{Contexts}

One of the principle advantages of computational calculi like the
$\pi$-calculus is a well-defined notion of context,
contextual-equivalence and a correlation between
contextual-equivalence and notions of bisimulation. The notion of
context allows the decomposition of a process into (sub-)process and
its syntactic environment, its context. Thus, a context may be
thought of as a process with a ``hole'' (written $\Box$) in it. The
application of a context $M$ to a process $P$, written $M[P]$, is
tantamount to filling the hole in $M$ with $P$. In this paper we do
not need the full weight of this theory, but do make use of the notion
of context in the proof the main theorem. 

\begin{mathpar}
  \inferrule* [lab=summation] {} {{M_{M},M_{N}} \bc \Box \;|\; x.M_{A} \;|\; M_{M}+M_{N}}
  \and
  \inferrule* [lab=agent] {} {{M_{A}} \bc (\vec{x})M_{P} \;| \; \clift{P_0,\ldots,M_{P},\ldots,P_N}}
  \and \\
  \inferrule* [lab=process] {} {{M_{P}} \bc M_{N} \;| \;P|M_{P} }
\end{mathpar} 

\begin{mathpar}
  \inferrule* [lab=sychronization] {} {M_{N} \bc \Box \;|\; x?M_{F} \;|\; x!M_{C}}
  \and
  \inferrule* [lab=abstraction] {} {{M_{F}} \bc (x)M_{P} }
  \and
  \inferrule* [lab=concretion] {} {{M_{C}} \bc \langle M_{P} \rangle }
  \and \\
  \inferrule* [lab=process] {} {{M_{P}} \bc M_{N} \;| \;P|M_{P} }
\end{mathpar}

\begin{definition}[contextual application] Given a context $M$, and
  process $P$, we define the \emph{contextual application}, $M[P] :=
  M\{P/\Box\}$. That is, the contextual application of M to P is the
  substitution of $P$ for $\Box$ in $M$.
\end{definition}

$\meaningof{-} : L \to \mathcal{P}(\pi)$

\begin{mathpar}
  \inferrule* [lab=collection] {} {\meaningof{true} = \pi, \and \meaningof{~E} = \pi \setminus \meaningof{E}, \and \meaningof{E_{1} \& E_{2}} = \meaningof{E_{1}} \cap \meaningof{E_{2}}}
\end{mathpar}

\begin{mathpar}
  \inferrule* [lab=structure] {} {\meaningof{0} = \{ P \in \pi | P \equiv 0 \}, \and \\ \meaningof{E_1 | E_2} = \{ P \in \pi | P \equiv P_{1} | P_{2}, P_{1} \in \meaningof{E_{1}}, P_{2} \in \meaningof{E_2}\} }
\end{mathpar}

\begin{mathpar}
 \inferrule* [lab=behavior] {} {\meaningof{\langle a?b \rangle E} = \{ P \in \pi | P \equiv Q | u?(y)P', \\ \and \\\\ \and \\ \;\;\; u \in \meaningof{a}, \forall z.P'\{z/y\} \in \meaningof{E\{z/b\}}\}, \and \\ \meaningof{a!E} = \{ P \in \pi | P \equiv Q | x!\langle P' \rangle, x \in \meaningof{a} P' \in \meaningof{E}\} }
\end{mathpar}

\begin{mathpar}
 \inferrule* [lab=nominal] {} {\meaningof{\quotep{E}} = \{ \quotep{P} \in \quotep{\pi} | P \in \meaningof{E} \}, \and \meaningof{\quotep{P}} = \{ \quotep{Q} \in \quotep{\pi} | P \equiv Q \} \and \\ \meaningof{@\quotep{E}} = \{ P \in \pi | P \equiv @x, x \in \meaningof{E} \}}
\end{mathpar}

\begin{eqnarray*}
  \\
  \meaningof{-} : TS \to ST
\end{eqnarray*}

\begin{eqnarray*}
  \\
  L : TS \to ST
\end{eqnarray*}

\begin{eqnarray*}
  \\
  P \models E \iff P \in \meaningof{E}
\end{eqnarray*}

\begin{eqnarray*}
  P \approx_{L} Q \iff \forall E \in L. P \models E \iff Q \models E
\end{eqnarray*}

\begin{eqnarray*}
  P \approx_{K} Q
\end{eqnarray*}

\begin{eqnarray*}
  P \approx Q
\end{eqnarray*}

$\approx_{K} = \approx = \approx_{L}$

\subsubsection{Contextual duality}

Note that contexts extend the quotation operation to a family of
operations from processes to names. Given a context, $M$, we can
define a \emph{nominal context}, $\quotep{M}$ by $\quotep{M}[P] :=
\quotep{M[P]}$. To foreshadow what is to come we observe that these
operations enjoy a duality with processes very much like the duality
between vectors and maps from vectors to scalars.

Further, because the calculus is essentially higher-order, we have a
correspondence between contexts and processes. More specifically,
given a name $x$ and a context $M$ we can construct $M^{*}_{x}$ such
that 

\begin{mathpar}
  M^{*}_{x} | \lift{x}{P} \red M[P]
\end{mathpar}

namely,

\begin{mathpar}
  M^{*}_{x} := x?(u).M[\dropn{u}]
\end{mathpar}

The dependence of $M^{*}_{x}$ on a name makes it an abstraction, 

\begin{mathpar}
  M^{*} := (x)x?(u).M[\dropn{u}]
\end{mathpar}

\subsection{Additional notation}

It will sometimes be convenient to denote the process a name
quotes. We already have the notation $x = \quotep{P}$, but it will be
convenient to introduce an alternate notation, $\procn{x}$, when we
want to emphasize the connection to the use of the name. Note that, by
virtue of name equivalence, $\quotep{\procn{x}} \nameeq x$; so, the
notation is consistent with previous definitions.

Further, because names have structure it is possible to effect
substitutions on the basis of that structure. This means we need to
upgrade our notation for substitutions, which we accomplish by
adapting comprehension notation. Thus,

\begin{mathpar}
  P\{ y / x : x \in S \}
\end{mathpar}

is interpreted to mean the process derived from P by replacing (in a
capture-avoiding manner) each occurrence of $x$ in $S$ by $y$. For example,

\begin{mathpar}
  P\{ \quotep{\procn{x}|\procn{x}} / x : x \in \freenames{P} \}
\end{mathpar}

will replace each (occurrence) of a free name $x$ in $P$ by
$\quotep{\procn{x}|\procn{x}}$.

Also, we will avail ourselves of the notation $x^{L}$ and $x^{R}$ to
denote injections of a name into disjoint copies of the name
space. There are numerous ways to accomplish this. One example can be
found in \cite{MeredithR05}. This notation overloads to vectors of
names: $\vec{x}^{\pi} := (x_{i}^{\pi} \; : \; 0 \leq i < |\vec{x}| )$ where $\pi \in \{L,R\}$.

We also use $P^{\Box} := P|\Box$.

In \cite{MeredithR05} an interpretation of the new operator is
given. It turns out that there are several possible interpretations
all enjoying the requisite algebraic properties of the operator (see
\cite{milner91polyadicpi}). We will therefore make liberal use of
$(\nu\; \vec{x})P$.

% subsection the_syntax_and_semantics_of_the_notation_system (end)   

\input{qm2pi.qmops} 

\input{qm2pi.sterngerlach} 

\input{qm2pi.metric} 

% section concurrent_process_calculi (end)

%\input{qm2pi.proofsketch}

% section proof sketch (end)

%\input{qm2pi.slviaknots} 

% section spatial logic via knots (end)

\input{qm2pi.conclusion}

% section conclusion (end)

%\input{qm2pi.dtcodes} 

% section wiring algorithm (end)

\input{qm2pi.ack} 

% section acknowledgments (end)

\newpage


\bibliographystyle{plain}   
\bibliography{../../biblios/main.bib}

\input{qm2pi.rhodetails}

\end{document}

 

% subsection basic_interpretation (end)

%\input{qm2pi.rho.presentation} 
\subsection{The syntax and semantics of the notation system}\label{sub:the_syntax_and_semantics_of_the_notation_system} % (fold)

We now summarize a technical presentation of the calculus that
embodies our theory of dynamics. The typical presentation of such a
calculus follows the style of giving generators and relations on
them. The grammar, below, describing term constructors, freely
generates the set of processes, $\Proc$. This set is then quotiented
by a relation known as structural congruence and it is over this set
that the notion of dynamics is expressed. This presentation is
essentially that of \cite{MeredithR05} with the addition of
polyadicity and summation. For readability we have relegated some of
the technical subtleties to an appendix.

\subsubsection{Process grammar}\label{subsub:process_grammar}

\begin{mathpar}
  \inferrule* [lab=synchronization] {} {{M} \bc \pzero \;|\; x?F \;|\; x!C }
  \and
  \inferrule* [lab=abstraction] {} {{F} \bc (x)P}
  \and
  \inferrule* [lab=concretion] {} {{C} \bc \langle Q \rangle}
  \and
  \inferrule* [lab=process] {} {{P,Q} \bc M \;| \;P|Q \;|\; @{x}}
  \and
  \inferrule* [lab=name] {} {{x} \bc \quotep{P}}
\end{mathpar} 

Note that $\vec{x}$ (resp. $\vec{P}$) denotes a vector of names
(resp. processes) of length $|\vec{x}|$ (resp. $|\vec{P}|$). We adopt
the following useful abbreviations.

\begin{mathpar}
   x?(\vec{y}).P := x.(\vec{y})P \and  x\clift{\vec{P}} := x.\clift{\vec{P}}
   \and x!(y) := \lift{x}{\dropn{y}}
   \and \Pi_{i=0}^{n-1}P_i := P_0 | \ldots | P_{n-1}
\end{mathpar}

\subsubsection{Structural congruence}

\paragraph{Free and bound names and alpha-equivalence.} At the
core of structural equivalence is alpha-equivalence which identifies
process that are the same up to a change of variable. Formally, we
recognize the distinction between free and bound names. The free names
of a process, $\freenames{P}$, may be calculated recursively as
follows:

\begin{mathpar}
\freenames{\pzero} := \emptyset
  \and \\
  \freenames{x?(y).P} := \{ x \} \cup (\freenames{P} \setminus \{ y \})
  \and 
  \freenames{x!\langle P \rangle} := \{ x \} \cup \{ P \} 
  \and \\
  \freenames{P|Q} := \freenames{P} \cup \freenames{Q}
  \and \\
  \freenames{@{x}} := \{ x \}
\end{mathpar}

$\pi$
$\quotep{\pi}$

$\freenames{-} : \pi \to \mathcal{P}(\quotep{\pi})$

\begin{eqnarray*}
  \freenames{\pzero} & := & \emptyset \\
  \freenames{x?(y).P} & := & \{ x \} \cup (\freenames{P} \setminus \{ y \}) \\
  \freenames{x!\langle P \rangle} & := & \{ x \} \cup \{ P \} \\
  \freenames{P|Q} & := & \freenames{P} \cup \freenames{Q} \\
  \freenames{\dropn{x}} & := & \{ x \}
\end{eqnarray*}

The bound names of a process, $\boundnames{P}$, are those names occurring in $P$
that are not free. For example, in $x?(y).0$, the name $x$ is free, while $y$ is bound.

\begin{mathpar}
  \inferrule* [lab=monoidal-laws] {} { P|Q \equiv Q|P \and P|0 \equiv P \and P|(Q|R) \equiv (P|Q)|R }
\end{mathpar}

\begin{mathpar}
  \inferrule* [lab=alpha-equivalence] {} { (x)P \equiv (y)P\{y/x\} \and y \not\in \freenames{P} }
\end{mathpar}

\begin{definition}
Then two processes, $P,Q$, are alpha-equivalent if $P = Q\{\vec{y}/\vec{x}\}$ for
some $\vec{x} \in \boundnames{Q},\vec{y} \in \boundnames{P}$, where $Q\{\vec{y}/\vec{x}\}$
denotes the capture-avoiding substitution of $\vec{y}$ for $\vec{x}$ in $Q$.
\end{definition}

\begin{definition}
  The {\em structural congruence} \cite{SangiorgiWalker} , $\equiv$,
  between processes is the least congruence containing
  alpha-equivalence, satisfying the abelian monoid laws
  (associativity, commutativity and $\pzero$ as identity) for parallel
  composition $|$ and for summation $+$.
\end{definition}

\subsection{Name equivalence}

We take name equivalence, written $\nameeq$, to be the smallest
equivalence relation generated by the following rules.

\begin{mathpar}
\inferrule*[lab=Quote-drop]
{ }
{ \quotep{@{x}} \nameeq x }

\inferrule*[lab=Struct-equiv]
{ P \scong Q }
{ \quotep{P} \nameeq \quotep{Q} }
\end{mathpar}

The astute reader will have noticed that the mutual recursion of names
and processes imposes a mutual recursion on alpha-equivalence and
structural equivalence via name-equivalence. Fortunately, all of this
works out pleasantly and we may calculate in the natural way, free of
concern. The reader interested in the details is referred to the
appendix \ref{appendix:rho_details}.

\subsection{Substitution}

We use $\Proc$ for the set of processes, $\QProc$ for the set of
names, and $\id{\{}\vec{y} / \vec{x} \id{\}}$ to denote partial maps,
$s : \QProc \rightarrow \QProc$. A map, $s$ lifts, uniquely, to a map
on process terms, $\widehat{s} : \Proc \rightarrow \Proc$ by the
following equations.

\begin{mathpar}
  (0) \psubstp{Q}{P} := 0 \\
  (R \juxtap S) \psubstp{Q}{P}
  :=    
  (R)\psubstp{Q}{P} \juxtap (S) \psubstp{Q}{P} \\
  (x?(y).R) \psubstp{Q}{P}    
  :=    
  (x)\substp{Q}{P} (z)\concat( (R \psubstn{z}{y}) \psubstp{Q}{P} ) \\
  (\lift{x}{R}) \psubstp{Q}{P}  
  :=
  \lift{(x)\substp{Q}{P}}{ R \psubstp{Q}{P} } \\
%   (\dropn{x})  \psubstp{Q}{P}       
%   := 
%   \left\{ 
%     \begin{array}{ccc} 
%       \dropn{\quotep{Q}} & & x \nameeq \quotep{P} \\
%       \dropn{x} & & otherwise \\
%     \end{array}
%   \right. 
  (\dropn{x})  \psubstp{Q}{P}       
  := 
  \left\{ 
    \begin{array}{ccc} 
      Q & & x \nameeq \quotep{P} \\
      \dropn{x} & & otherwise \\
    \end{array}
  \right.
\end{mathpar}
 

where

\begin{eqnarray}
  (x)\id{\{} \lpquote Q \rpquote / \lpquote P \rpquote \id{\}}            = 
  \left\{ 
    \begin{array}{ccc}
      \lpquote Q \rpquote & & x \nameeq \lpquote P \rpquote \\
      x & & otherwise \\
    \end{array}
  \right. \nonumber
\end{eqnarray}

and $z$ is chosen distinct from $\quotep{P}$, $\quotep{Q}$, the free
names in $Q$, and all the names in $R$. Our $\alpha$-equivalence will
be built in the standard way from this substitution.

\begin{remark}\label{rem:no_self_referential_names}
  One consequence of these definitions is that $\forall P. \quotep{P}
  \not\in \freenames{P}$.
\end{remark}

\subsection{ Dynamic quote: an example }

Anticipating something of what's to come, consider applying the
substitution, $\widehat{\id{\{}u / z \id{\}}}$, to the following pair
of processes, $\lift{w}{y!(z)}$ and $w[ \lpquote y!(z) \rpquote ]$.

\begin{eqnarray}
	\lift{w}{y!(z)}\widehat{\id{\{}u / z \id{\}}}
		& = &
		\lift{w}{y!(u)} \nonumber\\
	w[ \lpquote y!(z) \rpquote ] \widehat{ \id{\{}u / z \id{\}} }
		& = &
		w[ \lpquote y!(z) \rpquote ] \nonumber
\end{eqnarray}

Because the body of the process between quotes is impervious to
substitution, we get radically different answers. In fact, by
examining the first process in an input context,
e.g. $x?(z).\lift{w}{y!(z)}$, we see that the process under the lift
operator may be shaped by prefixed inputs binding a name inside it. In
this sense, the lift operator will be seen as a way to dynamically
construct processes before reifying them as names.

Finally equipped with these standard features we can present the
dynamics of the calculus.

\subsubsection{Operational semantics} 

Finally, we introduce the computational dynamics. What marks these
algebras as distinct from other more traditionally studied algebraic
structures, e.g. vector spaces or polynomial rings, is the manner in
which dynamics is captured. In traditional structures, dynamics is typically
expressed through morphisms between such structures, as in linear maps
between vector spaces or morphisms between rings. In algebras
associated with the semantics of computation, the dynamics is
expressed as part of the algebraic structure itself, through a
reduction reduction relation typically denoted by $\red$. Below, we
give a recursive presentation of this relation for the calculus used
in the encoding.

$\red \subseteq \pi \times \pi$
$\red : \pi \to \mathcal{P}(\pi)$

\begin{mathpar}
  \inferrule* [lab=Comm] { \textsf{match}( x_{src}, x_{trgt} ) } { x_{trgt}?(y)P \; | \; x_{src}!\langle {Q} \rangle \red P\{\quotep{Q}/y}\} }
  \and \\
  \inferrule* [lab=Par] {{P} \red {P}'} {{{P} | {Q}} \red {{P}' | {Q}}}
  \and
  \inferrule* [lab=Equiv]{{{P} \scong {P}'} \andalso {{P}' \red {Q}'} \andalso {{Q}' \scong {Q}}}{{P} \red {Q}}
\end{mathpar}

\begin{eqnarray*}
  match_{\equiv} (\quotep{P},\quotep{Q}) & := & P \equiv Q \\
  match_{\dagger}(\quotep{P},\quotep{Q}) & := & \forall R. P|Q \red^{*} R => R \red^{*} 0 \\
  match_{K}(\quotep{P},\quotep{Q}) & := & K \mbox{ for some context } K
\end{eqnarray*}

$u?(x)P | u!\langle Q \rangle \red P\{\quotep{Q}/x\}$

%We write $\wred$ for $\red^*$, and $P\red$ if $\exists Q $ such that $ P \red Q$.
We write $P\red$ if $\exists Q $ such that $ P \red Q$ and $P\not\red$, otherwise.

\section{Replication}

As mentioned before, it is known that replication (and hence
recursion) can be implemented in a higher-order process algebra
\cite{SangiorgiWalker}. As our first example of calculation with the
machinery thus far presented we give the construction explicitly in
the {\rhoc}.

\begin{eqnarray}
	D_{x} & := & \prefix{x}{y}{(\binpar{\outputp{x}{y}}{@{y}})} \nonumber\\
	\bangp_{x}{P} & := & \binpar{{x}!\langle{\binpar{D_{x}}{P}}\rangle}{D_{x}} \nonumber
\end{eqnarray}

\begin{eqnarray}
	\bangp_{x}{P} & & \nonumber\\
	=
	& {x}!\langle{(\prefix{x}{y}{(\outputp{x}{y} | @{y})) | P}}\rangle 
	      | \prefix{x}{y}{(\outputp{x}{y} | @{y})} & \nonumber\\
	\red
	& (\outputp{x}{y} | @{y})\substn{\quotep{(\prefix{x}{y}{(@{y} | \outputp{x}{y})) | P}}}{y} & \nonumber\\
	=
	& \outputp{x}{\quotep{(\prefix{x}{y}{(\outputp{x}{y} | @{y})) | P}}}
	  | {(\prefix{x}{y}{(\outputp{x}{y} | @{y})) | P}} & \nonumber\\
	\red
	& \ldots & \nonumber\\
	\red^*
	& P | P | \ldots & \nonumber
\end{eqnarray}

Of course, this encoding, as an implementation, runs away, unfolding
$\bangp{P}$ eagerly. A lazier and more implementable replication
operator, restricted to input-guarded processes, may be obtained as follows.

\begin{eqnarray}
\bangp{\prefix{u}{v}{P}} 
	:= 
	\binpar{\lift{x}{\prefix{u}{v}{(\binpar{D(x)}{P})}}}{D(x)} \nonumber
\end{eqnarray}

\begin{remark}
  Note that the lazier definition still does not deal with summation
  or mixed summation (i.e. sums over input and output). The reader is
  invited to construct definitions of replication that deal with these
  features. 

  Further, the definitions are parameterized in a name, $x$. Can you,
  gentle reader, make a definition that eliminates this parameter and
  guarantees no accidental interaction between the replication
  machinery and the process being replicated -- i.e. no accidental
  sharing of names used by the process to get its work done and the
  name(s) used by the replication to effect copying. This latter
  revision of the definition of replication is crucial to obtaining
  the expected identity $!!P \sim !P$.
\end{remark}

\begin{remark}\label{rem:paradoxical_combinator}
  The reader familiar with the lambda calculus will have noticed the
  similarity between $D$ and the paradoxical combinator.

  [Ed. note: the existence of this seems to suggest we have to be more
  restrictive on the set of processes and names we admit if we are to
  support no-cloning.]
\end{remark}

\subsubsection{Bisimulation}

The computational dynamics gives rise to another kind of equivalence,
the equivalence of computational behavior. As previously mentioned
this is typically captured \emph{via} some form of bisimulation.

% The notion we use in this paper is weak barbed bisimulation
% \cite{milner91polyadicpi}.

The notion we use in this paper is derived from weak barbed
bisimulation \cite{milner91polyadicpi}. 

\begin{definition}
An \emph{observation relation}, $\downarrow_{\mathcal N}$, over a set
of names, $\mathcal N$, is the smallest relation satisfying the rules
below.

\infrule[Out-barb]{y \in {\mathcal N}, \; x \nameeq y}
		  {\outputp{x}{v} \downarrow_{\mathcal N} x}
\infrule[Par-barb]{\mbox{$P\downarrow_{\mathcal N} x$ or $Q\downarrow_{\mathcal N} x$}}
		  {\binpar{P}{Q} \downarrow_{\mathcal N} x}

We write $P \Downarrow_{\mathcal N} x$ if there is $Q$ such that 
$P \wred Q$ and $Q \downarrow_{\mathcal N} x$.
\end{definition}

\begin{definition}
%\label{def.bbisim}
An  ${\mathcal N}$-\emph{barbed bisimulation} over a set of names, ${\mathcal N}$, is a symmetric binary relation 
${\mathcal S}_{\mathcal N}$ between agents such that $P\rel{S}_{\mathcal N}Q$ implies:
\begin{enumerate}
\item If $P \red P'$ then $Q \wred Q'$ and $P'\rel{S}_{\mathcal N} Q'$.
\item If $P\downarrow_{\mathcal N} x$, then $Q\Downarrow_{\mathcal N} x$.
\end{enumerate}
$P$ is ${\mathcal N}$-barbed bisimilar to $Q$, written
$P \wbbisim_{\mathcal N} Q$, if $P \rel{S}_{\mathcal N} Q$ for some ${\mathcal N}$-barbed bisimulation ${\mathcal S}_{\mathcal N}$.
\end{definition}

$\mathcal{R} \subseteq \pi \times \pi$

$P \mathcal{R} Q => \forall P'. P \red P' \Rightarrow \exists Q'. Q \red Q', P' \mathcal{R} Q'$

$P \vdash x \Rightarrow Q \vdash x$

\begin{mathpar}
  \inferrule*[lab=Out-barb]{x \nameeq y}{{y}!\langle{Q}\rangle \vdash x}
  \and
  \inferrule*[lab=Par-barb]{\mbox{$P\vdash x$ or $Q\vdash x$}}{\binpar{P}{Q} \vdash x}
\end{mathpar}

\subsubsection{Contexts}

One of the principle advantages of computational calculi like the
$\pi$-calculus is a well-defined notion of context,
contextual-equivalence and a correlation between
contextual-equivalence and notions of bisimulation. The notion of
context allows the decomposition of a process into (sub-)process and
its syntactic environment, its context. Thus, a context may be
thought of as a process with a ``hole'' (written $\Box$) in it. The
application of a context $M$ to a process $P$, written $M[P]$, is
tantamount to filling the hole in $M$ with $P$. In this paper we do
not need the full weight of this theory, but do make use of the notion
of context in the proof the main theorem. 

\begin{mathpar}
  \inferrule* [lab=summation] {} {{M_{M},M_{N}} \bc \Box \;|\; x.M_{A} \;|\; M_{M}+M_{N}}
  \and
  \inferrule* [lab=agent] {} {{M_{A}} \bc (\vec{x})M_{P} \;| \; \clift{P_0,\ldots,M_{P},\ldots,P_N}}
  \and \\
  \inferrule* [lab=process] {} {{M_{P}} \bc M_{N} \;| \;P|M_{P} }
\end{mathpar} 

\begin{mathpar}
  \inferrule* [lab=sychronization] {} {M_{N} \bc \Box \;|\; x?M_{F} \;|\; x!M_{C}}
  \and
  \inferrule* [lab=abstraction] {} {{M_{F}} \bc (x)M_{P} }
  \and
  \inferrule* [lab=concretion] {} {{M_{C}} \bc \langle M_{P} \rangle }
  \and \\
  \inferrule* [lab=process] {} {{M_{P}} \bc M_{N} \;| \;P|M_{P} }
\end{mathpar}

\begin{definition}[contextual application] Given a context $M$, and
  process $P$, we define the \emph{contextual application}, $M[P] :=
  M\{P/\Box\}$. That is, the contextual application of M to P is the
  substitution of $P$ for $\Box$ in $M$.
\end{definition}

$\meaningof{-} : L \to \mathcal{P}(\pi)$

\begin{mathpar}
  \inferrule* [lab=collection] {} {\meaningof{true} = \pi, \and \meaningof{~E} = \pi \setminus \meaningof{E}, \and \meaningof{E_{1} \& E_{2}} = \meaningof{E_{1}} \cap \meaningof{E_{2}}}
\end{mathpar}

\begin{mathpar}
  \inferrule* [lab=structure] {} {\meaningof{0} = \{ P \in \pi | P \equiv 0 \}, \and \\ \meaningof{E_1 | E_2} = \{ P \in \pi | P \equiv P_{1} | P_{2}, P_{1} \in \meaningof{E_{1}}, P_{2} \in \meaningof{E_2}\} }
\end{mathpar}

\begin{mathpar}
 \inferrule* [lab=behavior] {} {\meaningof{\langle a?b \rangle E} = \{ P \in \pi | P \equiv Q | u?(y)P', \\ \and \\\\ \and \\ \;\;\; u \in \meaningof{a}, \forall z.P'\{z/y\} \in \meaningof{E\{z/b\}}\}, \and \\ \meaningof{a!E} = \{ P \in \pi | P \equiv Q | x!\langle P' \rangle, x \in \meaningof{a} P' \in \meaningof{E}\} }
\end{mathpar}

\begin{mathpar}
 \inferrule* [lab=nominal] {} {\meaningof{\quotep{E}} = \{ \quotep{P} \in \quotep{\pi} | P \in \meaningof{E} \}, \and \meaningof{\quotep{P}} = \{ \quotep{Q} \in \quotep{\pi} | P \equiv Q \} \and \\ \meaningof{@\quotep{E}} = \{ P \in \pi | P \equiv @x, x \in \meaningof{E} \}}
\end{mathpar}

\begin{eqnarray*}
  \\
  \meaningof{-} : TS \to ST
\end{eqnarray*}

\begin{eqnarray*}
  \\
  L : TS \to ST
\end{eqnarray*}

\begin{eqnarray*}
  \\
  P \models E \iff P \in \meaningof{E}
\end{eqnarray*}

\begin{eqnarray*}
  P \approx_{L} Q \iff \forall E \in L. P \models E \iff Q \models E
\end{eqnarray*}

\begin{eqnarray*}
  P \approx_{K} Q
\end{eqnarray*}

\begin{eqnarray*}
  P \approx Q
\end{eqnarray*}

$\approx_{K} = \approx = \approx_{L}$

\subsubsection{Contextual duality}

Note that contexts extend the quotation operation to a family of
operations from processes to names. Given a context, $M$, we can
define a \emph{nominal context}, $\quotep{M}$ by $\quotep{M}[P] :=
\quotep{M[P]}$. To foreshadow what is to come we observe that these
operations enjoy a duality with processes very much like the duality
between vectors and maps from vectors to scalars.

Further, because the calculus is essentially higher-order, we have a
correspondence between contexts and processes. More specifically,
given a name $x$ and a context $M$ we can construct $M^{*}_{x}$ such
that 

\begin{mathpar}
  M^{*}_{x} | \lift{x}{P} \red M[P]
\end{mathpar}

namely,

\begin{mathpar}
  M^{*}_{x} := x?(u).M[\dropn{u}]
\end{mathpar}

The dependence of $M^{*}_{x}$ on a name makes it an abstraction, 

\begin{mathpar}
  M^{*} := (x)x?(u).M[\dropn{u}]
\end{mathpar}

\subsection{Additional notation}

It will sometimes be convenient to denote the process a name
quotes. We already have the notation $x = \quotep{P}$, but it will be
convenient to introduce an alternate notation, $\procn{x}$, when we
want to emphasize the connection to the use of the name. Note that, by
virtue of name equivalence, $\quotep{\procn{x}} \nameeq x$; so, the
notation is consistent with previous definitions.

Further, because names have structure it is possible to effect
substitutions on the basis of that structure. This means we need to
upgrade our notation for substitutions, which we accomplish by
adapting comprehension notation. Thus,

\begin{mathpar}
  P\{ y / x : x \in S \}
\end{mathpar}

is interpreted to mean the process derived from P by replacing (in a
capture-avoiding manner) each occurrence of $x$ in $S$ by $y$. For example,

\begin{mathpar}
  P\{ \quotep{\procn{x}|\procn{x}} / x : x \in \freenames{P} \}
\end{mathpar}

will replace each (occurrence) of a free name $x$ in $P$ by
$\quotep{\procn{x}|\procn{x}}$.

Also, we will avail ourselves of the notation $x^{L}$ and $x^{R}$ to
denote injections of a name into disjoint copies of the name
space. There are numerous ways to accomplish this. One example can be
found in \cite{MeredithR05}. This notation overloads to vectors of
names: $\vec{x}^{\pi} := (x_{i}^{\pi} \; : \; 0 \leq i < |\vec{x}| )$ where $\pi \in \{L,R\}$.

We also use $P^{\Box} := P|\Box$.

In \cite{MeredithR05} an interpretation of the new operator is
given. It turns out that there are several possible interpretations
all enjoying the requisite algebraic properties of the operator (see
\cite{milner91polyadicpi}). We will therefore make liberal use of
$(\nu\; \vec{x})P$.

% subsection the_syntax_and_semantics_of_the_notation_system (end)   

\section{Interpretation of QM}
\subsection{Supporting definitions}
\subsubsection{Multiplication}
\begin{mathpar}
  \quotep{Q} \cdot \quotep{R} := \quotep{Q|R}
  \and \\
  \quotep{Q} \cdot P := P\{ \quotep{Q|R} / \quotep{R} : \quotep{R} \in \freenames{P} \}
\end{mathpar}

\paragraph{Discussion}
The first line needs little explanation. The second line says that
each free name of the process is replaced with the multiplication of
that name by the scalar. Multiplication of a scalar (name) by a state
(process) results in a process all the names of which have been `moved
over' by parallel composition with the process the scalar
quotes. There is a subtlety that the bound names have to be
manipulated so that multiplied names aren't accidentally
captured. There are many ways to achieve this.

\begin{remark}\label{rem:multiplication_identities}
  The reader is invited to verify that for all $x,y,z \in \QProc$ and $P \in \Proc$
  \begin{mathpar}
    x \cdot \quotep{0} \equiv x 
    \and
    x \cdot y \equiv y \cdot x
    \and
    x \cdot (y \cdot z) \equiv (x \cdot y) \cdot z
    \and \\
    \quotep{0} \cdot P \equiv P
    \and \\
    x \cdot (y \cdot P) \equiv (x \cdot y) \cdot P
    \and \\
    x \cdot (P|Q) \equiv (x \cdot P) | (x \cdot Q)
    \and \\    
  \end{mathpar}
\end{remark}

\subsubsection{Tensor product}

We define a tensor product on processes by structural induction.

\paragraph{Tensor of sums} First note that all summations, including
$\pzero$ and sequence, can be written $\Sigma_{i} x_{i}.A_{i} +
\Sigma_{j} x_{j}.C_{j}$, where we have grouped input-guarded processes
together and output-guarded processes together.

Thus, we can define the tensor product of two summations, $N_{1}\otimes N_{2}$, where

\begin{mathpar}
  N_{1} := \Sigma_{i} x_{i}.A_{i} + \Sigma_{j} x_{j}.C_{j}
  \and
  N_{2} := \Sigma_{i'} y_{i'}.B_{i'} + \Sigma_{j'} y_{j'}.D_{j'} 
\end{mathpar}

as follows.

\begin{mathpar}
  \Sigma_{i} x_{i}.A_{i} + \Sigma_{j} x_{j}.C_{j} \otimes \Sigma_{i'}
  y_{i'}.B_{i'} + \Sigma_{j'} y_{j'}.D_{j'} 
  \and \\
  := \; \Sigma_{i} \Sigma_{i'} \quotep{\stackrel{\vee}{x_{i}}| \stackrel{\vee}{y_{i'}}}.(A_{i}\otimes B_{i'}) \; | \; \Sigma_{i'} \Sigma_{i} \quotep{\stackrel{\vee}{y_{i'}}|\stackrel{\vee}{x_{i}}}.(B_{i'}\otimes A_{i})
  \and
  \;\; | \;\; \Sigma_{j} \Sigma_{j'} \quotep{\stackrel{\vee}{x_{j}}|\stackrel{\vee}{y_{j'}}}.(A_{j}\otimes B_{j'}) \; | \; \Sigma_{j'} \Sigma_{j} \quotep{\stackrel{\vee}{y_{j'}}|\stackrel{\vee}{x_{j}}}.(B_{j'}\otimes A_{j})
\end{mathpar}

\begin{remark}
  Do we need to $x^{L}$ and $y^{R}$ for this construction as well?
\end{remark}

\paragraph{Tensor of parallel compositions} Next, we distribute tensor
over par.

\begin{mathpar}
  P_{1}|P_{2} \otimes Q_{1}|Q_{2} := (P_{1} \otimes Q_{1}) | (P_{1}
  \otimes Q_{2}) | (P_{2} \otimes Q_{1}) | (P_{2} \otimes Q_{2})
\end{mathpar}

\paragraph{Tensor with dropped names} We treat tensor of a
process with a dropped name as parallel composition.

\begin{mathpar}
  P \otimes \dropn{x} := P | \dropn{x}
\end{mathpar}

\paragraph{Tensor of agents}

Finally, we need to define tensor on agents. Note that the definition
of tensor on normal products only tensors inputs with inputs and
outputs with outputs. Thus, we only have to define the operation on
``homogeneous'' pairings.

\begin{mathpar}
  (\vec{x})P \otimes (\vec{y})Q
  \and \\
  := (x_{0}^{L}|y_{0}^{R},\ldots,x_{0}^{L}|y_{n}^{R},\ldots,x_{m}^{L}|y_{0}^{R},\ldots,x_{m}^{L}|y_{n}^R)(P\{ \vec{x}^{L}/\vec{x}\} \otimes Q \{ \vec{y}^{R}/\vec{y}\})
  \and \\
  \clift{\vec{P}} \otimes \clift{\vec{Q}}
  \and \\
  := \clift{P_{0}\otimes Q_{0},\ldots,P_{0}\otimes Q_{n},\ldots,P_{m}\otimes Q_{0},\ldots,P_{m}\otimes Q_{n}}
\end{mathpar}

\begin{remark}
  Observe that arities of tensored abstractions matches arities of
  tensored concretions if the original arities matched. Note also that
  the length of the arities corresponds to the increase in dimension
  we see in ordinary vector space tensor product.
\end{remark}

\begin{remark}
  Operationally, this definition distributes the tensor down to
  components ``linked'' by summation. Tensor over summation is
  intriguing in that it mixes names. Moreover, as a consequence of the
  way it mixes names we have the identities for all $x \in \QProc$ and
  $P,Q \in \Proc$

  \begin{mathpar}
    (x \cdot P) \otimes Q \equiv x \cdot (P \otimes Q) \equiv P \otimes (x \cdot Q)
    \and
    P \otimes \pzero \equiv P
  \end{mathpar}

  that the reader is invited to verify.
\end{remark}

\subsubsection{Annihilation}
\begin{mathpar}
  P^{\perp} := \{ Q | \forall R. P|Q \red^{*} R \Rightarrow R \red^{*} \pzero \}
  \and \\
  P^{\underline{\perp}} := \Sigma_{Q \in P^{\perp}} \quotep{Q}?(y).(\dropn{y}|Q) | \Sigma_{Q \in P^{\perp}} \quotep{Q}\clift{\Box}
\end{mathpar}

\paragraph{Discussion} The reader will note that $P^{\perp}$ is a
\emph{set} of processes, while $P^{\underline{\perp}}$ is a
\emph{context}. We call the set $P^{\perp}$ the \emph{annihilators} of
$P$. The parallel composition of a process in the annihilators of $P$
with $P$ will result in a process, the state space of which has all
paths eventually leading to $\pzero$. Execution may endure loops; but
under reasonable conditions of fairness (naturally guaranteed under
most notions of bisimulation) such a composite process cannot get
stuck in such a loop and will, eventually pop out and terminate.

The context $P^{\underline{\perp}}$ is ready and willing to ``take the
$P$ out of'' the process to which it is applied. It will effectively
transmit the code of the process to which it is applied to one of the
annihilators and run the process against it.

\subsubsection{Evaluation}
We fix $M$ a domain of fully abstract interpretation with an equality
coincident with bisimulation. We take $\meaningof{\cdot} : \Proc \to
M$ to be the map interpreting processes and $\nmeaningof{\cdot} : \M
\to Proc$ to be the map running the other way. Then we define

\begin{mathpar}
  \int P := \nmeaningof{\meaningof{P}}
\end{mathpar}

\paragraph{Discussion}
There are many fully abstract interpretations of Milner's
$\pi$-calculus. Any of them can be used as a basis for interpreting
the reflective calculus here. Equipped with such a domain it is
largely a matter of grinding through to check that the Yoneda
construction for the normalization-by-evaluation program can be
extended to this setting.

\begin{remark}
  The reader is invited to verify that $\int (P^{\underline{\perp}}[P]) = 0$.
\end{remark}

\subsection{Quantum mechanics}

Table \ref{tbl:core_qm_op_defns} gives the core operational definitions

\begin{table}[htp]\label{tbl:core_qm_op_defns}
  \center{
    \fbox{
      \begin{tabular}{c|c}
        quantum mechanics & process calculus \\
        \hline
        scalar & $x := \quotep{P}$ \\
        state vector & $\state{P} := P$ \\
        dual & $\state{P}^{*} := \event{P^{\underline{\perp}}} := \quotep{P^{\underline{\perp}}}[-]$ \\
        matrix & $ \Sigma_{\alpha} \state{P_{\alpha}}x_{\alpha}\event{Q_{\alpha}}$ \\
        vector addition & $\state{P} + \state{Q} := \state{P | Q}$ \\
        tensor product & $\state{P} \otimes \state{Q} := \state{P \otimes Q}$ \\
        inner product & $\innerprod{P}{Q} := \quotep{\int P^{\underline{\perp}}[Q]}$ \\
      \end{tabular}
    }
  }
  \caption{QM - operational definitions}
\end{table}

where

\begin{mathpar}
  \prmatrix{P}{Q} := \fprmatrix{P}{\quotep{\pzero}}{Q}
  \and
  \fprmatrix{P}{x}{Q} := (\state{P},x,\event{Q})
  \and
  (\fprmatrix{P}{x}{Q})(\state{R}) := x \cdot \innerprod{Q}{R} \cdot \state{P}
  \and
  (\fprmatrix{P}{x}{Q})(\event{R}) := x \cdot \innerprod{R}{P} \cdot \event{Q}
\end{mathpar}

\paragraph{Discussion}
As promised: vectors (aka states) are represented as processes; duals
as contextual duals; inner product definition should be compared with
standard inner product definition for ....

\begin{remark}
  Assuming $\int (P^{\underline{\perp}}[P]) = 0$, the reader is
  invited to verify that $(\fprmatrix{P}{x}{P})(\state{P}) = x \cdot \state{P}$.
\end{remark}

\begin{remark}
  The reader is invited to verify that $\innerprod{P}{Q}$ could
  equally well have been written $\quotep{\int \stackrel{\vee}{x}}$
  where $x = \event{P^{\underline{\perp}}}(Q)$.

  One of the motivations for this remark is that there is another way
  to factor these operations. We could package up evaluation in the dual:

  \begin{mathpar}
    \state{P}^{*} := \event{\int P^{\underline{\perp}}} := \quotep{\int P^{\underline{\perp}}}[-]
  \end{mathpar}

  and then have inner product defined by
  
  \begin{mathpar}
    \innerprod{P}{Q} := \event{P}(Q)
  \end{mathpar}

  Hopefully, experience with the calculations will provide guidance on
  the best factoring.
\end{remark}

\begin{remark}
  Assuming $\int (P^{\underline{\perp}}[P]) = 0$, the reader is
  invited to verify that $\forall P,Q. (\prmatrix{0}{Q})(\state{0}) =
  \state{0}$ and dually $(\prmatrix{P}{0})(\event{0}) = \event{0}$.
\end{remark}

\begin{remark}
  i'm a little worried that i don't (yet) have proper support for
  complex conjugacy. But, the observation above may give us a
  clue. According to Abramsky, it must be the case that the scalars
  are iso to the homset of the identity for the tensor -- which the
  observation above characterizes. 

  For now, we will simply bookmark the notion with $\overline{x}$.
\end{remark}

\subsubsection{Adjointness}

We need to give a definition of $(\cdot)^{\dagger}$ for matrices. The
obvious candidate definition is
\begin{mathpar}
(\Sigma_{\alpha}\fprmatrix{P_{\alpha}}{x_{\alpha}}{Q_{\alpha}})^{\dagger}
= \Sigma_{\alpha}\fprmatrix{(Q_{\alpha}^{\underline{\perp}})^{*}}{\overline{x}_{\alpha}}{P_{\alpha}^{\underline{\perp}}} 
\end{mathpar}

But, $(Q_{\alpha}^{\underline{\perp}})^{*}$ requires a name along
which to communicate the process to achieve the context application.

\subsubsection{Basis for a basis}
If processes label states and ``addition'' of states (a.k.a. vector
addition) is interpreted as parallel composition, what corresponds to
notions of linear independence and basis? Here, we recall that Yoshida
has developed a set of \emph{combinators} for an asynchronous verison
of Milner's $\pi$-calculus. These are a finite set of processes such
any process can be expressed as parallel composition of these
combinators together with liberal uses of the new operator and
replication. We can simply give a translation of these into the
present calculus and have reasonable expectation that the property
carries over. That is, that the resultant set allows to express all
processes via parallel composition. Note, however, that there is no
new operator or replication in this calculus. As a result, we expect
that the corresponding set is actually infinite. That is, we expect
that the space is actually infinite dimensional.

\begin{remark}
  The attentive reader may be a bit concerned. Certainly, the
  collection $S$, $K$ and $I$ is a finite set of
  combinators. Shouldn't we expect to see a finite set of combinators
  for an effectively equivalent system? i am very sympathetic to this
  critique and feel it warrants full attention. On the other hand, i
  also have in mind the following analogy. The natural numbers, as a
  monoid under addition, has exactly $1$ generator, while the natural
  numbers, as a monoid under multiplication, has countably many
  generators (the primes). We observe that the application of the
  lambda calculus is much less resource sensitive than the parallel
  composition of the $\pi$-calculus. Could it be the case that we have
  an analogy of the form
  
  \begin{mathpar}
    m + n : MN :: m*n : M|N
  \end{mathpar}

  giving a similar blow up in the set of ``primes''?  This is such a
  wonderful thought that, even if it's not true, i think it's worth
  writing down.
\end{remark}
 

\documentclass[12pt]{llncs}
%\documentclass{jktr}

\usepackage[pdftex]{hyperref}                   
\usepackage {listings}
\usepackage {mathpartir}
\usepackage{bcprules}
%\usepackage{listings}
                       
\usepackage{graphicx} 
%\usepackage[margins=2.5cm,nohead,nofoot]{geometry}
%\usepackage{geometry}
\usepackage{amsfonts}
\usepackage{amstext}
\usepackage{latexsym}
\usepackage{amssymb}
\usepackage{color}


%\include{myPreamble}
\include{qm2pi.local} 

%\ifpdf
%\usepackage[pdftex]{graphicx}
%\else
%\usepackage{graphicx}
%\fi

 % \ifpdf
%  \usepackage{pdfsync}
%  \if


%\title{Brief Article}
%\author{David F. Snyder}
%\author{L.G. Meredith}

%\address{Dept. of Math., Texas State University--San Marcos, San Marcos, TX 78666}
       
\pagestyle{empty}


\begin{document}

\lstset{language=[Objective]Caml,frame=shadowbox}

\input{qm2pi.front}

% section front matter (end)

\input{qm2pi.intro} 
 
% section introduction (end)

% \input{qm2pi.knotations} 

% section notation (end)

\input{qm2pi.process.calculi} 

% section concurrent_process_calculi_and_spatial_logics_ (end)
    
%\input{qm2pi.knots2pi} 

%\input{qm2pi.trefoil} 

%\input{qm2pi.mainthm} 

% subsection basic_interpretation (end)

%\input{qm2pi.rho.presentation} 
\subsection{The syntax and semantics of the notation system}\label{sub:the_syntax_and_semantics_of_the_notation_system} % (fold)

We now summarize a technical presentation of the calculus that
embodies our theory of dynamics. The typical presentation of such a
calculus follows the style of giving generators and relations on
them. The grammar, below, describing term constructors, freely
generates the set of processes, $\Proc$. This set is then quotiented
by a relation known as structural congruence and it is over this set
that the notion of dynamics is expressed. This presentation is
essentially that of \cite{MeredithR05} with the addition of
polyadicity and summation. For readability we have relegated some of
the technical subtleties to an appendix.

\subsubsection{Process grammar}\label{subsub:process_grammar}

\begin{mathpar}
  \inferrule* [lab=synchronization] {} {{M} \bc \pzero \;|\; x?F \;|\; x!C }
  \and
  \inferrule* [lab=abstraction] {} {{F} \bc (x)P}
  \and
  \inferrule* [lab=concretion] {} {{C} \bc \langle Q \rangle}
  \and
  \inferrule* [lab=process] {} {{P,Q} \bc M \;| \;P|Q \;|\; @{x}}
  \and
  \inferrule* [lab=name] {} {{x} \bc \quotep{P}}
\end{mathpar} 

Note that $\vec{x}$ (resp. $\vec{P}$) denotes a vector of names
(resp. processes) of length $|\vec{x}|$ (resp. $|\vec{P}|$). We adopt
the following useful abbreviations.

\begin{mathpar}
   x?(\vec{y}).P := x.(\vec{y})P \and  x\clift{\vec{P}} := x.\clift{\vec{P}}
   \and x!(y) := \lift{x}{\dropn{y}}
   \and \Pi_{i=0}^{n-1}P_i := P_0 | \ldots | P_{n-1}
\end{mathpar}

\subsubsection{Structural congruence}

\paragraph{Free and bound names and alpha-equivalence.} At the
core of structural equivalence is alpha-equivalence which identifies
process that are the same up to a change of variable. Formally, we
recognize the distinction between free and bound names. The free names
of a process, $\freenames{P}$, may be calculated recursively as
follows:

\begin{mathpar}
\freenames{\pzero} := \emptyset
  \and \\
  \freenames{x?(y).P} := \{ x \} \cup (\freenames{P} \setminus \{ y \})
  \and 
  \freenames{x!\langle P \rangle} := \{ x \} \cup \{ P \} 
  \and \\
  \freenames{P|Q} := \freenames{P} \cup \freenames{Q}
  \and \\
  \freenames{@{x}} := \{ x \}
\end{mathpar}

$\pi$
$\quotep{\pi}$

$\freenames{-} : \pi \to \mathcal{P}(\quotep{\pi})$

\begin{eqnarray*}
  \freenames{\pzero} & := & \emptyset \\
  \freenames{x?(y).P} & := & \{ x \} \cup (\freenames{P} \setminus \{ y \}) \\
  \freenames{x!\langle P \rangle} & := & \{ x \} \cup \{ P \} \\
  \freenames{P|Q} & := & \freenames{P} \cup \freenames{Q} \\
  \freenames{\dropn{x}} & := & \{ x \}
\end{eqnarray*}

The bound names of a process, $\boundnames{P}$, are those names occurring in $P$
that are not free. For example, in $x?(y).0$, the name $x$ is free, while $y$ is bound.

\begin{mathpar}
  \inferrule* [lab=monoidal-laws] {} { P|Q \equiv Q|P \and P|0 \equiv P \and P|(Q|R) \equiv (P|Q)|R }
\end{mathpar}

\begin{mathpar}
  \inferrule* [lab=alpha-equivalence] {} { (x)P \equiv (y)P\{y/x\} \and y \not\in \freenames{P} }
\end{mathpar}

\begin{definition}
Then two processes, $P,Q$, are alpha-equivalent if $P = Q\{\vec{y}/\vec{x}\}$ for
some $\vec{x} \in \boundnames{Q},\vec{y} \in \boundnames{P}$, where $Q\{\vec{y}/\vec{x}\}$
denotes the capture-avoiding substitution of $\vec{y}$ for $\vec{x}$ in $Q$.
\end{definition}

\begin{definition}
  The {\em structural congruence} \cite{SangiorgiWalker} , $\equiv$,
  between processes is the least congruence containing
  alpha-equivalence, satisfying the abelian monoid laws
  (associativity, commutativity and $\pzero$ as identity) for parallel
  composition $|$ and for summation $+$.
\end{definition}

\subsection{Name equivalence}

We take name equivalence, written $\nameeq$, to be the smallest
equivalence relation generated by the following rules.

\begin{mathpar}
\inferrule*[lab=Quote-drop]
{ }
{ \quotep{@{x}} \nameeq x }

\inferrule*[lab=Struct-equiv]
{ P \scong Q }
{ \quotep{P} \nameeq \quotep{Q} }
\end{mathpar}

The astute reader will have noticed that the mutual recursion of names
and processes imposes a mutual recursion on alpha-equivalence and
structural equivalence via name-equivalence. Fortunately, all of this
works out pleasantly and we may calculate in the natural way, free of
concern. The reader interested in the details is referred to the
appendix \ref{appendix:rho_details}.

\subsection{Substitution}

We use $\Proc$ for the set of processes, $\QProc$ for the set of
names, and $\id{\{}\vec{y} / \vec{x} \id{\}}$ to denote partial maps,
$s : \QProc \rightarrow \QProc$. A map, $s$ lifts, uniquely, to a map
on process terms, $\widehat{s} : \Proc \rightarrow \Proc$ by the
following equations.

\begin{mathpar}
  (0) \psubstp{Q}{P} := 0 \\
  (R \juxtap S) \psubstp{Q}{P}
  :=    
  (R)\psubstp{Q}{P} \juxtap (S) \psubstp{Q}{P} \\
  (x?(y).R) \psubstp{Q}{P}    
  :=    
  (x)\substp{Q}{P} (z)\concat( (R \psubstn{z}{y}) \psubstp{Q}{P} ) \\
  (\lift{x}{R}) \psubstp{Q}{P}  
  :=
  \lift{(x)\substp{Q}{P}}{ R \psubstp{Q}{P} } \\
%   (\dropn{x})  \psubstp{Q}{P}       
%   := 
%   \left\{ 
%     \begin{array}{ccc} 
%       \dropn{\quotep{Q}} & & x \nameeq \quotep{P} \\
%       \dropn{x} & & otherwise \\
%     \end{array}
%   \right. 
  (\dropn{x})  \psubstp{Q}{P}       
  := 
  \left\{ 
    \begin{array}{ccc} 
      Q & & x \nameeq \quotep{P} \\
      \dropn{x} & & otherwise \\
    \end{array}
  \right.
\end{mathpar}
 

where

\begin{eqnarray}
  (x)\id{\{} \lpquote Q \rpquote / \lpquote P \rpquote \id{\}}            = 
  \left\{ 
    \begin{array}{ccc}
      \lpquote Q \rpquote & & x \nameeq \lpquote P \rpquote \\
      x & & otherwise \\
    \end{array}
  \right. \nonumber
\end{eqnarray}

and $z$ is chosen distinct from $\quotep{P}$, $\quotep{Q}$, the free
names in $Q$, and all the names in $R$. Our $\alpha$-equivalence will
be built in the standard way from this substitution.

\begin{remark}\label{rem:no_self_referential_names}
  One consequence of these definitions is that $\forall P. \quotep{P}
  \not\in \freenames{P}$.
\end{remark}

\subsection{ Dynamic quote: an example }

Anticipating something of what's to come, consider applying the
substitution, $\widehat{\id{\{}u / z \id{\}}}$, to the following pair
of processes, $\lift{w}{y!(z)}$ and $w[ \lpquote y!(z) \rpquote ]$.

\begin{eqnarray}
	\lift{w}{y!(z)}\widehat{\id{\{}u / z \id{\}}}
		& = &
		\lift{w}{y!(u)} \nonumber\\
	w[ \lpquote y!(z) \rpquote ] \widehat{ \id{\{}u / z \id{\}} }
		& = &
		w[ \lpquote y!(z) \rpquote ] \nonumber
\end{eqnarray}

Because the body of the process between quotes is impervious to
substitution, we get radically different answers. In fact, by
examining the first process in an input context,
e.g. $x?(z).\lift{w}{y!(z)}$, we see that the process under the lift
operator may be shaped by prefixed inputs binding a name inside it. In
this sense, the lift operator will be seen as a way to dynamically
construct processes before reifying them as names.

Finally equipped with these standard features we can present the
dynamics of the calculus.

\subsubsection{Operational semantics} 

Finally, we introduce the computational dynamics. What marks these
algebras as distinct from other more traditionally studied algebraic
structures, e.g. vector spaces or polynomial rings, is the manner in
which dynamics is captured. In traditional structures, dynamics is typically
expressed through morphisms between such structures, as in linear maps
between vector spaces or morphisms between rings. In algebras
associated with the semantics of computation, the dynamics is
expressed as part of the algebraic structure itself, through a
reduction reduction relation typically denoted by $\red$. Below, we
give a recursive presentation of this relation for the calculus used
in the encoding.

$\red \subseteq \pi \times \pi$
$\red : \pi \to \mathcal{P}(\pi)$

\begin{mathpar}
  \inferrule* [lab=Comm] { \textsf{match}( x_{src}, x_{trgt} ) } { x_{trgt}?(y)P \; | \; x_{src}!\langle {Q} \rangle \red P\{\quotep{Q}/y}\} }
  \and \\
  \inferrule* [lab=Par] {{P} \red {P}'} {{{P} | {Q}} \red {{P}' | {Q}}}
  \and
  \inferrule* [lab=Equiv]{{{P} \scong {P}'} \andalso {{P}' \red {Q}'} \andalso {{Q}' \scong {Q}}}{{P} \red {Q}}
\end{mathpar}

\begin{eqnarray*}
  match_{\equiv} (\quotep{P},\quotep{Q}) & := & P \equiv Q \\
  match_{\dagger}(\quotep{P},\quotep{Q}) & := & \forall R. P|Q \red^{*} R => R \red^{*} 0 \\
  match_{K}(\quotep{P},\quotep{Q}) & := & K \mbox{ for some context } K
\end{eqnarray*}

$u?(x)P | u!\langle Q \rangle \red P\{\quotep{Q}/x\}$

%We write $\wred$ for $\red^*$, and $P\red$ if $\exists Q $ such that $ P \red Q$.
We write $P\red$ if $\exists Q $ such that $ P \red Q$ and $P\not\red$, otherwise.

\section{Replication}

As mentioned before, it is known that replication (and hence
recursion) can be implemented in a higher-order process algebra
\cite{SangiorgiWalker}. As our first example of calculation with the
machinery thus far presented we give the construction explicitly in
the {\rhoc}.

\begin{eqnarray}
	D_{x} & := & \prefix{x}{y}{(\binpar{\outputp{x}{y}}{@{y}})} \nonumber\\
	\bangp_{x}{P} & := & \binpar{{x}!\langle{\binpar{D_{x}}{P}}\rangle}{D_{x}} \nonumber
\end{eqnarray}

\begin{eqnarray}
	\bangp_{x}{P} & & \nonumber\\
	=
	& {x}!\langle{(\prefix{x}{y}{(\outputp{x}{y} | @{y})) | P}}\rangle 
	      | \prefix{x}{y}{(\outputp{x}{y} | @{y})} & \nonumber\\
	\red
	& (\outputp{x}{y} | @{y})\substn{\quotep{(\prefix{x}{y}{(@{y} | \outputp{x}{y})) | P}}}{y} & \nonumber\\
	=
	& \outputp{x}{\quotep{(\prefix{x}{y}{(\outputp{x}{y} | @{y})) | P}}}
	  | {(\prefix{x}{y}{(\outputp{x}{y} | @{y})) | P}} & \nonumber\\
	\red
	& \ldots & \nonumber\\
	\red^*
	& P | P | \ldots & \nonumber
\end{eqnarray}

Of course, this encoding, as an implementation, runs away, unfolding
$\bangp{P}$ eagerly. A lazier and more implementable replication
operator, restricted to input-guarded processes, may be obtained as follows.

\begin{eqnarray}
\bangp{\prefix{u}{v}{P}} 
	:= 
	\binpar{\lift{x}{\prefix{u}{v}{(\binpar{D(x)}{P})}}}{D(x)} \nonumber
\end{eqnarray}

\begin{remark}
  Note that the lazier definition still does not deal with summation
  or mixed summation (i.e. sums over input and output). The reader is
  invited to construct definitions of replication that deal with these
  features. 

  Further, the definitions are parameterized in a name, $x$. Can you,
  gentle reader, make a definition that eliminates this parameter and
  guarantees no accidental interaction between the replication
  machinery and the process being replicated -- i.e. no accidental
  sharing of names used by the process to get its work done and the
  name(s) used by the replication to effect copying. This latter
  revision of the definition of replication is crucial to obtaining
  the expected identity $!!P \sim !P$.
\end{remark}

\begin{remark}\label{rem:paradoxical_combinator}
  The reader familiar with the lambda calculus will have noticed the
  similarity between $D$ and the paradoxical combinator.

  [Ed. note: the existence of this seems to suggest we have to be more
  restrictive on the set of processes and names we admit if we are to
  support no-cloning.]
\end{remark}

\subsubsection{Bisimulation}

The computational dynamics gives rise to another kind of equivalence,
the equivalence of computational behavior. As previously mentioned
this is typically captured \emph{via} some form of bisimulation.

% The notion we use in this paper is weak barbed bisimulation
% \cite{milner91polyadicpi}.

The notion we use in this paper is derived from weak barbed
bisimulation \cite{milner91polyadicpi}. 

\begin{definition}
An \emph{observation relation}, $\downarrow_{\mathcal N}$, over a set
of names, $\mathcal N$, is the smallest relation satisfying the rules
below.

\infrule[Out-barb]{y \in {\mathcal N}, \; x \nameeq y}
		  {\outputp{x}{v} \downarrow_{\mathcal N} x}
\infrule[Par-barb]{\mbox{$P\downarrow_{\mathcal N} x$ or $Q\downarrow_{\mathcal N} x$}}
		  {\binpar{P}{Q} \downarrow_{\mathcal N} x}

We write $P \Downarrow_{\mathcal N} x$ if there is $Q$ such that 
$P \wred Q$ and $Q \downarrow_{\mathcal N} x$.
\end{definition}

\begin{definition}
%\label{def.bbisim}
An  ${\mathcal N}$-\emph{barbed bisimulation} over a set of names, ${\mathcal N}$, is a symmetric binary relation 
${\mathcal S}_{\mathcal N}$ between agents such that $P\rel{S}_{\mathcal N}Q$ implies:
\begin{enumerate}
\item If $P \red P'$ then $Q \wred Q'$ and $P'\rel{S}_{\mathcal N} Q'$.
\item If $P\downarrow_{\mathcal N} x$, then $Q\Downarrow_{\mathcal N} x$.
\end{enumerate}
$P$ is ${\mathcal N}$-barbed bisimilar to $Q$, written
$P \wbbisim_{\mathcal N} Q$, if $P \rel{S}_{\mathcal N} Q$ for some ${\mathcal N}$-barbed bisimulation ${\mathcal S}_{\mathcal N}$.
\end{definition}

$\mathcal{R} \subseteq \pi \times \pi$

$P \mathcal{R} Q => \forall P'. P \red P' \Rightarrow \exists Q'. Q \red Q', P' \mathcal{R} Q'$

$P \vdash x \Rightarrow Q \vdash x$

\begin{mathpar}
  \inferrule*[lab=Out-barb]{x \nameeq y}{{y}!\langle{Q}\rangle \vdash x}
  \and
  \inferrule*[lab=Par-barb]{\mbox{$P\vdash x$ or $Q\vdash x$}}{\binpar{P}{Q} \vdash x}
\end{mathpar}

\subsubsection{Contexts}

One of the principle advantages of computational calculi like the
$\pi$-calculus is a well-defined notion of context,
contextual-equivalence and a correlation between
contextual-equivalence and notions of bisimulation. The notion of
context allows the decomposition of a process into (sub-)process and
its syntactic environment, its context. Thus, a context may be
thought of as a process with a ``hole'' (written $\Box$) in it. The
application of a context $M$ to a process $P$, written $M[P]$, is
tantamount to filling the hole in $M$ with $P$. In this paper we do
not need the full weight of this theory, but do make use of the notion
of context in the proof the main theorem. 

\begin{mathpar}
  \inferrule* [lab=summation] {} {{M_{M},M_{N}} \bc \Box \;|\; x.M_{A} \;|\; M_{M}+M_{N}}
  \and
  \inferrule* [lab=agent] {} {{M_{A}} \bc (\vec{x})M_{P} \;| \; \clift{P_0,\ldots,M_{P},\ldots,P_N}}
  \and \\
  \inferrule* [lab=process] {} {{M_{P}} \bc M_{N} \;| \;P|M_{P} }
\end{mathpar} 

\begin{mathpar}
  \inferrule* [lab=sychronization] {} {M_{N} \bc \Box \;|\; x?M_{F} \;|\; x!M_{C}}
  \and
  \inferrule* [lab=abstraction] {} {{M_{F}} \bc (x)M_{P} }
  \and
  \inferrule* [lab=concretion] {} {{M_{C}} \bc \langle M_{P} \rangle }
  \and \\
  \inferrule* [lab=process] {} {{M_{P}} \bc M_{N} \;| \;P|M_{P} }
\end{mathpar}

\begin{definition}[contextual application] Given a context $M$, and
  process $P$, we define the \emph{contextual application}, $M[P] :=
  M\{P/\Box\}$. That is, the contextual application of M to P is the
  substitution of $P$ for $\Box$ in $M$.
\end{definition}

$\meaningof{-} : L \to \mathcal{P}(\pi)$

\begin{mathpar}
  \inferrule* [lab=collection] {} {\meaningof{true} = \pi, \and \meaningof{~E} = \pi \setminus \meaningof{E}, \and \meaningof{E_{1} \& E_{2}} = \meaningof{E_{1}} \cap \meaningof{E_{2}}}
\end{mathpar}

\begin{mathpar}
  \inferrule* [lab=structure] {} {\meaningof{0} = \{ P \in \pi | P \equiv 0 \}, \and \\ \meaningof{E_1 | E_2} = \{ P \in \pi | P \equiv P_{1} | P_{2}, P_{1} \in \meaningof{E_{1}}, P_{2} \in \meaningof{E_2}\} }
\end{mathpar}

\begin{mathpar}
 \inferrule* [lab=behavior] {} {\meaningof{\langle a?b \rangle E} = \{ P \in \pi | P \equiv Q | u?(y)P', \\ \and \\\\ \and \\ \;\;\; u \in \meaningof{a}, \forall z.P'\{z/y\} \in \meaningof{E\{z/b\}}\}, \and \\ \meaningof{a!E} = \{ P \in \pi | P \equiv Q | x!\langle P' \rangle, x \in \meaningof{a} P' \in \meaningof{E}\} }
\end{mathpar}

\begin{mathpar}
 \inferrule* [lab=nominal] {} {\meaningof{\quotep{E}} = \{ \quotep{P} \in \quotep{\pi} | P \in \meaningof{E} \}, \and \meaningof{\quotep{P}} = \{ \quotep{Q} \in \quotep{\pi} | P \equiv Q \} \and \\ \meaningof{@\quotep{E}} = \{ P \in \pi | P \equiv @x, x \in \meaningof{E} \}}
\end{mathpar}

\begin{eqnarray*}
  \\
  \meaningof{-} : TS \to ST
\end{eqnarray*}

\begin{eqnarray*}
  \\
  L : TS \to ST
\end{eqnarray*}

\begin{eqnarray*}
  \\
  P \models E \iff P \in \meaningof{E}
\end{eqnarray*}

\begin{eqnarray*}
  P \approx_{L} Q \iff \forall E \in L. P \models E \iff Q \models E
\end{eqnarray*}

\begin{eqnarray*}
  P \approx_{K} Q
\end{eqnarray*}

\begin{eqnarray*}
  P \approx Q
\end{eqnarray*}

$\approx_{K} = \approx = \approx_{L}$

\subsubsection{Contextual duality}

Note that contexts extend the quotation operation to a family of
operations from processes to names. Given a context, $M$, we can
define a \emph{nominal context}, $\quotep{M}$ by $\quotep{M}[P] :=
\quotep{M[P]}$. To foreshadow what is to come we observe that these
operations enjoy a duality with processes very much like the duality
between vectors and maps from vectors to scalars.

Further, because the calculus is essentially higher-order, we have a
correspondence between contexts and processes. More specifically,
given a name $x$ and a context $M$ we can construct $M^{*}_{x}$ such
that 

\begin{mathpar}
  M^{*}_{x} | \lift{x}{P} \red M[P]
\end{mathpar}

namely,

\begin{mathpar}
  M^{*}_{x} := x?(u).M[\dropn{u}]
\end{mathpar}

The dependence of $M^{*}_{x}$ on a name makes it an abstraction, 

\begin{mathpar}
  M^{*} := (x)x?(u).M[\dropn{u}]
\end{mathpar}

\subsection{Additional notation}

It will sometimes be convenient to denote the process a name
quotes. We already have the notation $x = \quotep{P}$, but it will be
convenient to introduce an alternate notation, $\procn{x}$, when we
want to emphasize the connection to the use of the name. Note that, by
virtue of name equivalence, $\quotep{\procn{x}} \nameeq x$; so, the
notation is consistent with previous definitions.

Further, because names have structure it is possible to effect
substitutions on the basis of that structure. This means we need to
upgrade our notation for substitutions, which we accomplish by
adapting comprehension notation. Thus,

\begin{mathpar}
  P\{ y / x : x \in S \}
\end{mathpar}

is interpreted to mean the process derived from P by replacing (in a
capture-avoiding manner) each occurrence of $x$ in $S$ by $y$. For example,

\begin{mathpar}
  P\{ \quotep{\procn{x}|\procn{x}} / x : x \in \freenames{P} \}
\end{mathpar}

will replace each (occurrence) of a free name $x$ in $P$ by
$\quotep{\procn{x}|\procn{x}}$.

Also, we will avail ourselves of the notation $x^{L}$ and $x^{R}$ to
denote injections of a name into disjoint copies of the name
space. There are numerous ways to accomplish this. One example can be
found in \cite{MeredithR05}. This notation overloads to vectors of
names: $\vec{x}^{\pi} := (x_{i}^{\pi} \; : \; 0 \leq i < |\vec{x}| )$ where $\pi \in \{L,R\}$.

We also use $P^{\Box} := P|\Box$.

In \cite{MeredithR05} an interpretation of the new operator is
given. It turns out that there are several possible interpretations
all enjoying the requisite algebraic properties of the operator (see
\cite{milner91polyadicpi}). We will therefore make liberal use of
$(\nu\; \vec{x})P$.

% subsection the_syntax_and_semantics_of_the_notation_system (end)   

\input{qm2pi.qmops} 

\input{qm2pi.sterngerlach} 

\input{qm2pi.metric} 

% section concurrent_process_calculi (end)

%\input{qm2pi.proofsketch}

% section proof sketch (end)

%\input{qm2pi.slviaknots} 

% section spatial logic via knots (end)

\input{qm2pi.conclusion}

% section conclusion (end)

%\input{qm2pi.dtcodes} 

% section wiring algorithm (end)

\input{qm2pi.ack} 

% section acknowledgments (end)

\newpage


\bibliographystyle{plain}   
\bibliography{../../biblios/main.bib}

\input{qm2pi.rhodetails}

\end{document}

 

\documentclass[12pt]{llncs}
%\documentclass{jktr}

\usepackage[pdftex]{hyperref}                   
\usepackage {listings}
\usepackage {mathpartir}
\usepackage{bcprules}
%\usepackage{listings}
                       
\usepackage{graphicx} 
%\usepackage[margins=2.5cm,nohead,nofoot]{geometry}
%\usepackage{geometry}
\usepackage{amsfonts}
\usepackage{amstext}
\usepackage{latexsym}
\usepackage{amssymb}
\usepackage{color}


%\include{myPreamble}
\include{qm2pi.local} 

%\ifpdf
%\usepackage[pdftex]{graphicx}
%\else
%\usepackage{graphicx}
%\fi

 % \ifpdf
%  \usepackage{pdfsync}
%  \if


%\title{Brief Article}
%\author{David F. Snyder}
%\author{L.G. Meredith}

%\address{Dept. of Math., Texas State University--San Marcos, San Marcos, TX 78666}
       
\pagestyle{empty}


\begin{document}

\lstset{language=[Objective]Caml,frame=shadowbox}

\input{qm2pi.front}

% section front matter (end)

\input{qm2pi.intro} 
 
% section introduction (end)

% \input{qm2pi.knotations} 

% section notation (end)

\input{qm2pi.process.calculi} 

% section concurrent_process_calculi_and_spatial_logics_ (end)
    
%\input{qm2pi.knots2pi} 

%\input{qm2pi.trefoil} 

%\input{qm2pi.mainthm} 

% subsection basic_interpretation (end)

%\input{qm2pi.rho.presentation} 
\subsection{The syntax and semantics of the notation system}\label{sub:the_syntax_and_semantics_of_the_notation_system} % (fold)

We now summarize a technical presentation of the calculus that
embodies our theory of dynamics. The typical presentation of such a
calculus follows the style of giving generators and relations on
them. The grammar, below, describing term constructors, freely
generates the set of processes, $\Proc$. This set is then quotiented
by a relation known as structural congruence and it is over this set
that the notion of dynamics is expressed. This presentation is
essentially that of \cite{MeredithR05} with the addition of
polyadicity and summation. For readability we have relegated some of
the technical subtleties to an appendix.

\subsubsection{Process grammar}\label{subsub:process_grammar}

\begin{mathpar}
  \inferrule* [lab=synchronization] {} {{M} \bc \pzero \;|\; x?F \;|\; x!C }
  \and
  \inferrule* [lab=abstraction] {} {{F} \bc (x)P}
  \and
  \inferrule* [lab=concretion] {} {{C} \bc \langle Q \rangle}
  \and
  \inferrule* [lab=process] {} {{P,Q} \bc M \;| \;P|Q \;|\; @{x}}
  \and
  \inferrule* [lab=name] {} {{x} \bc \quotep{P}}
\end{mathpar} 

Note that $\vec{x}$ (resp. $\vec{P}$) denotes a vector of names
(resp. processes) of length $|\vec{x}|$ (resp. $|\vec{P}|$). We adopt
the following useful abbreviations.

\begin{mathpar}
   x?(\vec{y}).P := x.(\vec{y})P \and  x\clift{\vec{P}} := x.\clift{\vec{P}}
   \and x!(y) := \lift{x}{\dropn{y}}
   \and \Pi_{i=0}^{n-1}P_i := P_0 | \ldots | P_{n-1}
\end{mathpar}

\subsubsection{Structural congruence}

\paragraph{Free and bound names and alpha-equivalence.} At the
core of structural equivalence is alpha-equivalence which identifies
process that are the same up to a change of variable. Formally, we
recognize the distinction between free and bound names. The free names
of a process, $\freenames{P}$, may be calculated recursively as
follows:

\begin{mathpar}
\freenames{\pzero} := \emptyset
  \and \\
  \freenames{x?(y).P} := \{ x \} \cup (\freenames{P} \setminus \{ y \})
  \and 
  \freenames{x!\langle P \rangle} := \{ x \} \cup \{ P \} 
  \and \\
  \freenames{P|Q} := \freenames{P} \cup \freenames{Q}
  \and \\
  \freenames{@{x}} := \{ x \}
\end{mathpar}

$\pi$
$\quotep{\pi}$

$\freenames{-} : \pi \to \mathcal{P}(\quotep{\pi})$

\begin{eqnarray*}
  \freenames{\pzero} & := & \emptyset \\
  \freenames{x?(y).P} & := & \{ x \} \cup (\freenames{P} \setminus \{ y \}) \\
  \freenames{x!\langle P \rangle} & := & \{ x \} \cup \{ P \} \\
  \freenames{P|Q} & := & \freenames{P} \cup \freenames{Q} \\
  \freenames{\dropn{x}} & := & \{ x \}
\end{eqnarray*}

The bound names of a process, $\boundnames{P}$, are those names occurring in $P$
that are not free. For example, in $x?(y).0$, the name $x$ is free, while $y$ is bound.

\begin{mathpar}
  \inferrule* [lab=monoidal-laws] {} { P|Q \equiv Q|P \and P|0 \equiv P \and P|(Q|R) \equiv (P|Q)|R }
\end{mathpar}

\begin{mathpar}
  \inferrule* [lab=alpha-equivalence] {} { (x)P \equiv (y)P\{y/x\} \and y \not\in \freenames{P} }
\end{mathpar}

\begin{definition}
Then two processes, $P,Q$, are alpha-equivalent if $P = Q\{\vec{y}/\vec{x}\}$ for
some $\vec{x} \in \boundnames{Q},\vec{y} \in \boundnames{P}$, where $Q\{\vec{y}/\vec{x}\}$
denotes the capture-avoiding substitution of $\vec{y}$ for $\vec{x}$ in $Q$.
\end{definition}

\begin{definition}
  The {\em structural congruence} \cite{SangiorgiWalker} , $\equiv$,
  between processes is the least congruence containing
  alpha-equivalence, satisfying the abelian monoid laws
  (associativity, commutativity and $\pzero$ as identity) for parallel
  composition $|$ and for summation $+$.
\end{definition}

\subsection{Name equivalence}

We take name equivalence, written $\nameeq$, to be the smallest
equivalence relation generated by the following rules.

\begin{mathpar}
\inferrule*[lab=Quote-drop]
{ }
{ \quotep{@{x}} \nameeq x }

\inferrule*[lab=Struct-equiv]
{ P \scong Q }
{ \quotep{P} \nameeq \quotep{Q} }
\end{mathpar}

The astute reader will have noticed that the mutual recursion of names
and processes imposes a mutual recursion on alpha-equivalence and
structural equivalence via name-equivalence. Fortunately, all of this
works out pleasantly and we may calculate in the natural way, free of
concern. The reader interested in the details is referred to the
appendix \ref{appendix:rho_details}.

\subsection{Substitution}

We use $\Proc$ for the set of processes, $\QProc$ for the set of
names, and $\id{\{}\vec{y} / \vec{x} \id{\}}$ to denote partial maps,
$s : \QProc \rightarrow \QProc$. A map, $s$ lifts, uniquely, to a map
on process terms, $\widehat{s} : \Proc \rightarrow \Proc$ by the
following equations.

\begin{mathpar}
  (0) \psubstp{Q}{P} := 0 \\
  (R \juxtap S) \psubstp{Q}{P}
  :=    
  (R)\psubstp{Q}{P} \juxtap (S) \psubstp{Q}{P} \\
  (x?(y).R) \psubstp{Q}{P}    
  :=    
  (x)\substp{Q}{P} (z)\concat( (R \psubstn{z}{y}) \psubstp{Q}{P} ) \\
  (\lift{x}{R}) \psubstp{Q}{P}  
  :=
  \lift{(x)\substp{Q}{P}}{ R \psubstp{Q}{P} } \\
%   (\dropn{x})  \psubstp{Q}{P}       
%   := 
%   \left\{ 
%     \begin{array}{ccc} 
%       \dropn{\quotep{Q}} & & x \nameeq \quotep{P} \\
%       \dropn{x} & & otherwise \\
%     \end{array}
%   \right. 
  (\dropn{x})  \psubstp{Q}{P}       
  := 
  \left\{ 
    \begin{array}{ccc} 
      Q & & x \nameeq \quotep{P} \\
      \dropn{x} & & otherwise \\
    \end{array}
  \right.
\end{mathpar}
 

where

\begin{eqnarray}
  (x)\id{\{} \lpquote Q \rpquote / \lpquote P \rpquote \id{\}}            = 
  \left\{ 
    \begin{array}{ccc}
      \lpquote Q \rpquote & & x \nameeq \lpquote P \rpquote \\
      x & & otherwise \\
    \end{array}
  \right. \nonumber
\end{eqnarray}

and $z$ is chosen distinct from $\quotep{P}$, $\quotep{Q}$, the free
names in $Q$, and all the names in $R$. Our $\alpha$-equivalence will
be built in the standard way from this substitution.

\begin{remark}\label{rem:no_self_referential_names}
  One consequence of these definitions is that $\forall P. \quotep{P}
  \not\in \freenames{P}$.
\end{remark}

\subsection{ Dynamic quote: an example }

Anticipating something of what's to come, consider applying the
substitution, $\widehat{\id{\{}u / z \id{\}}}$, to the following pair
of processes, $\lift{w}{y!(z)}$ and $w[ \lpquote y!(z) \rpquote ]$.

\begin{eqnarray}
	\lift{w}{y!(z)}\widehat{\id{\{}u / z \id{\}}}
		& = &
		\lift{w}{y!(u)} \nonumber\\
	w[ \lpquote y!(z) \rpquote ] \widehat{ \id{\{}u / z \id{\}} }
		& = &
		w[ \lpquote y!(z) \rpquote ] \nonumber
\end{eqnarray}

Because the body of the process between quotes is impervious to
substitution, we get radically different answers. In fact, by
examining the first process in an input context,
e.g. $x?(z).\lift{w}{y!(z)}$, we see that the process under the lift
operator may be shaped by prefixed inputs binding a name inside it. In
this sense, the lift operator will be seen as a way to dynamically
construct processes before reifying them as names.

Finally equipped with these standard features we can present the
dynamics of the calculus.

\subsubsection{Operational semantics} 

Finally, we introduce the computational dynamics. What marks these
algebras as distinct from other more traditionally studied algebraic
structures, e.g. vector spaces or polynomial rings, is the manner in
which dynamics is captured. In traditional structures, dynamics is typically
expressed through morphisms between such structures, as in linear maps
between vector spaces or morphisms between rings. In algebras
associated with the semantics of computation, the dynamics is
expressed as part of the algebraic structure itself, through a
reduction reduction relation typically denoted by $\red$. Below, we
give a recursive presentation of this relation for the calculus used
in the encoding.

$\red \subseteq \pi \times \pi$
$\red : \pi \to \mathcal{P}(\pi)$

\begin{mathpar}
  \inferrule* [lab=Comm] { \textsf{match}( x_{src}, x_{trgt} ) } { x_{trgt}?(y)P \; | \; x_{src}!\langle {Q} \rangle \red P\{\quotep{Q}/y}\} }
  \and \\
  \inferrule* [lab=Par] {{P} \red {P}'} {{{P} | {Q}} \red {{P}' | {Q}}}
  \and
  \inferrule* [lab=Equiv]{{{P} \scong {P}'} \andalso {{P}' \red {Q}'} \andalso {{Q}' \scong {Q}}}{{P} \red {Q}}
\end{mathpar}

\begin{eqnarray*}
  match_{\equiv} (\quotep{P},\quotep{Q}) & := & P \equiv Q \\
  match_{\dagger}(\quotep{P},\quotep{Q}) & := & \forall R. P|Q \red^{*} R => R \red^{*} 0 \\
  match_{K}(\quotep{P},\quotep{Q}) & := & K \mbox{ for some context } K
\end{eqnarray*}

$u?(x)P | u!\langle Q \rangle \red P\{\quotep{Q}/x\}$

%We write $\wred$ for $\red^*$, and $P\red$ if $\exists Q $ such that $ P \red Q$.
We write $P\red$ if $\exists Q $ such that $ P \red Q$ and $P\not\red$, otherwise.

\section{Replication}

As mentioned before, it is known that replication (and hence
recursion) can be implemented in a higher-order process algebra
\cite{SangiorgiWalker}. As our first example of calculation with the
machinery thus far presented we give the construction explicitly in
the {\rhoc}.

\begin{eqnarray}
	D_{x} & := & \prefix{x}{y}{(\binpar{\outputp{x}{y}}{@{y}})} \nonumber\\
	\bangp_{x}{P} & := & \binpar{{x}!\langle{\binpar{D_{x}}{P}}\rangle}{D_{x}} \nonumber
\end{eqnarray}

\begin{eqnarray}
	\bangp_{x}{P} & & \nonumber\\
	=
	& {x}!\langle{(\prefix{x}{y}{(\outputp{x}{y} | @{y})) | P}}\rangle 
	      | \prefix{x}{y}{(\outputp{x}{y} | @{y})} & \nonumber\\
	\red
	& (\outputp{x}{y} | @{y})\substn{\quotep{(\prefix{x}{y}{(@{y} | \outputp{x}{y})) | P}}}{y} & \nonumber\\
	=
	& \outputp{x}{\quotep{(\prefix{x}{y}{(\outputp{x}{y} | @{y})) | P}}}
	  | {(\prefix{x}{y}{(\outputp{x}{y} | @{y})) | P}} & \nonumber\\
	\red
	& \ldots & \nonumber\\
	\red^*
	& P | P | \ldots & \nonumber
\end{eqnarray}

Of course, this encoding, as an implementation, runs away, unfolding
$\bangp{P}$ eagerly. A lazier and more implementable replication
operator, restricted to input-guarded processes, may be obtained as follows.

\begin{eqnarray}
\bangp{\prefix{u}{v}{P}} 
	:= 
	\binpar{\lift{x}{\prefix{u}{v}{(\binpar{D(x)}{P})}}}{D(x)} \nonumber
\end{eqnarray}

\begin{remark}
  Note that the lazier definition still does not deal with summation
  or mixed summation (i.e. sums over input and output). The reader is
  invited to construct definitions of replication that deal with these
  features. 

  Further, the definitions are parameterized in a name, $x$. Can you,
  gentle reader, make a definition that eliminates this parameter and
  guarantees no accidental interaction between the replication
  machinery and the process being replicated -- i.e. no accidental
  sharing of names used by the process to get its work done and the
  name(s) used by the replication to effect copying. This latter
  revision of the definition of replication is crucial to obtaining
  the expected identity $!!P \sim !P$.
\end{remark}

\begin{remark}\label{rem:paradoxical_combinator}
  The reader familiar with the lambda calculus will have noticed the
  similarity between $D$ and the paradoxical combinator.

  [Ed. note: the existence of this seems to suggest we have to be more
  restrictive on the set of processes and names we admit if we are to
  support no-cloning.]
\end{remark}

\subsubsection{Bisimulation}

The computational dynamics gives rise to another kind of equivalence,
the equivalence of computational behavior. As previously mentioned
this is typically captured \emph{via} some form of bisimulation.

% The notion we use in this paper is weak barbed bisimulation
% \cite{milner91polyadicpi}.

The notion we use in this paper is derived from weak barbed
bisimulation \cite{milner91polyadicpi}. 

\begin{definition}
An \emph{observation relation}, $\downarrow_{\mathcal N}$, over a set
of names, $\mathcal N$, is the smallest relation satisfying the rules
below.

\infrule[Out-barb]{y \in {\mathcal N}, \; x \nameeq y}
		  {\outputp{x}{v} \downarrow_{\mathcal N} x}
\infrule[Par-barb]{\mbox{$P\downarrow_{\mathcal N} x$ or $Q\downarrow_{\mathcal N} x$}}
		  {\binpar{P}{Q} \downarrow_{\mathcal N} x}

We write $P \Downarrow_{\mathcal N} x$ if there is $Q$ such that 
$P \wred Q$ and $Q \downarrow_{\mathcal N} x$.
\end{definition}

\begin{definition}
%\label{def.bbisim}
An  ${\mathcal N}$-\emph{barbed bisimulation} over a set of names, ${\mathcal N}$, is a symmetric binary relation 
${\mathcal S}_{\mathcal N}$ between agents such that $P\rel{S}_{\mathcal N}Q$ implies:
\begin{enumerate}
\item If $P \red P'$ then $Q \wred Q'$ and $P'\rel{S}_{\mathcal N} Q'$.
\item If $P\downarrow_{\mathcal N} x$, then $Q\Downarrow_{\mathcal N} x$.
\end{enumerate}
$P$ is ${\mathcal N}$-barbed bisimilar to $Q$, written
$P \wbbisim_{\mathcal N} Q$, if $P \rel{S}_{\mathcal N} Q$ for some ${\mathcal N}$-barbed bisimulation ${\mathcal S}_{\mathcal N}$.
\end{definition}

$\mathcal{R} \subseteq \pi \times \pi$

$P \mathcal{R} Q => \forall P'. P \red P' \Rightarrow \exists Q'. Q \red Q', P' \mathcal{R} Q'$

$P \vdash x \Rightarrow Q \vdash x$

\begin{mathpar}
  \inferrule*[lab=Out-barb]{x \nameeq y}{{y}!\langle{Q}\rangle \vdash x}
  \and
  \inferrule*[lab=Par-barb]{\mbox{$P\vdash x$ or $Q\vdash x$}}{\binpar{P}{Q} \vdash x}
\end{mathpar}

\subsubsection{Contexts}

One of the principle advantages of computational calculi like the
$\pi$-calculus is a well-defined notion of context,
contextual-equivalence and a correlation between
contextual-equivalence and notions of bisimulation. The notion of
context allows the decomposition of a process into (sub-)process and
its syntactic environment, its context. Thus, a context may be
thought of as a process with a ``hole'' (written $\Box$) in it. The
application of a context $M$ to a process $P$, written $M[P]$, is
tantamount to filling the hole in $M$ with $P$. In this paper we do
not need the full weight of this theory, but do make use of the notion
of context in the proof the main theorem. 

\begin{mathpar}
  \inferrule* [lab=summation] {} {{M_{M},M_{N}} \bc \Box \;|\; x.M_{A} \;|\; M_{M}+M_{N}}
  \and
  \inferrule* [lab=agent] {} {{M_{A}} \bc (\vec{x})M_{P} \;| \; \clift{P_0,\ldots,M_{P},\ldots,P_N}}
  \and \\
  \inferrule* [lab=process] {} {{M_{P}} \bc M_{N} \;| \;P|M_{P} }
\end{mathpar} 

\begin{mathpar}
  \inferrule* [lab=sychronization] {} {M_{N} \bc \Box \;|\; x?M_{F} \;|\; x!M_{C}}
  \and
  \inferrule* [lab=abstraction] {} {{M_{F}} \bc (x)M_{P} }
  \and
  \inferrule* [lab=concretion] {} {{M_{C}} \bc \langle M_{P} \rangle }
  \and \\
  \inferrule* [lab=process] {} {{M_{P}} \bc M_{N} \;| \;P|M_{P} }
\end{mathpar}

\begin{definition}[contextual application] Given a context $M$, and
  process $P$, we define the \emph{contextual application}, $M[P] :=
  M\{P/\Box\}$. That is, the contextual application of M to P is the
  substitution of $P$ for $\Box$ in $M$.
\end{definition}

$\meaningof{-} : L \to \mathcal{P}(\pi)$

\begin{mathpar}
  \inferrule* [lab=collection] {} {\meaningof{true} = \pi, \and \meaningof{~E} = \pi \setminus \meaningof{E}, \and \meaningof{E_{1} \& E_{2}} = \meaningof{E_{1}} \cap \meaningof{E_{2}}}
\end{mathpar}

\begin{mathpar}
  \inferrule* [lab=structure] {} {\meaningof{0} = \{ P \in \pi | P \equiv 0 \}, \and \\ \meaningof{E_1 | E_2} = \{ P \in \pi | P \equiv P_{1} | P_{2}, P_{1} \in \meaningof{E_{1}}, P_{2} \in \meaningof{E_2}\} }
\end{mathpar}

\begin{mathpar}
 \inferrule* [lab=behavior] {} {\meaningof{\langle a?b \rangle E} = \{ P \in \pi | P \equiv Q | u?(y)P', \\ \and \\\\ \and \\ \;\;\; u \in \meaningof{a}, \forall z.P'\{z/y\} \in \meaningof{E\{z/b\}}\}, \and \\ \meaningof{a!E} = \{ P \in \pi | P \equiv Q | x!\langle P' \rangle, x \in \meaningof{a} P' \in \meaningof{E}\} }
\end{mathpar}

\begin{mathpar}
 \inferrule* [lab=nominal] {} {\meaningof{\quotep{E}} = \{ \quotep{P} \in \quotep{\pi} | P \in \meaningof{E} \}, \and \meaningof{\quotep{P}} = \{ \quotep{Q} \in \quotep{\pi} | P \equiv Q \} \and \\ \meaningof{@\quotep{E}} = \{ P \in \pi | P \equiv @x, x \in \meaningof{E} \}}
\end{mathpar}

\begin{eqnarray*}
  \\
  \meaningof{-} : TS \to ST
\end{eqnarray*}

\begin{eqnarray*}
  \\
  L : TS \to ST
\end{eqnarray*}

\begin{eqnarray*}
  \\
  P \models E \iff P \in \meaningof{E}
\end{eqnarray*}

\begin{eqnarray*}
  P \approx_{L} Q \iff \forall E \in L. P \models E \iff Q \models E
\end{eqnarray*}

\begin{eqnarray*}
  P \approx_{K} Q
\end{eqnarray*}

\begin{eqnarray*}
  P \approx Q
\end{eqnarray*}

$\approx_{K} = \approx = \approx_{L}$

\subsubsection{Contextual duality}

Note that contexts extend the quotation operation to a family of
operations from processes to names. Given a context, $M$, we can
define a \emph{nominal context}, $\quotep{M}$ by $\quotep{M}[P] :=
\quotep{M[P]}$. To foreshadow what is to come we observe that these
operations enjoy a duality with processes very much like the duality
between vectors and maps from vectors to scalars.

Further, because the calculus is essentially higher-order, we have a
correspondence between contexts and processes. More specifically,
given a name $x$ and a context $M$ we can construct $M^{*}_{x}$ such
that 

\begin{mathpar}
  M^{*}_{x} | \lift{x}{P} \red M[P]
\end{mathpar}

namely,

\begin{mathpar}
  M^{*}_{x} := x?(u).M[\dropn{u}]
\end{mathpar}

The dependence of $M^{*}_{x}$ on a name makes it an abstraction, 

\begin{mathpar}
  M^{*} := (x)x?(u).M[\dropn{u}]
\end{mathpar}

\subsection{Additional notation}

It will sometimes be convenient to denote the process a name
quotes. We already have the notation $x = \quotep{P}$, but it will be
convenient to introduce an alternate notation, $\procn{x}$, when we
want to emphasize the connection to the use of the name. Note that, by
virtue of name equivalence, $\quotep{\procn{x}} \nameeq x$; so, the
notation is consistent with previous definitions.

Further, because names have structure it is possible to effect
substitutions on the basis of that structure. This means we need to
upgrade our notation for substitutions, which we accomplish by
adapting comprehension notation. Thus,

\begin{mathpar}
  P\{ y / x : x \in S \}
\end{mathpar}

is interpreted to mean the process derived from P by replacing (in a
capture-avoiding manner) each occurrence of $x$ in $S$ by $y$. For example,

\begin{mathpar}
  P\{ \quotep{\procn{x}|\procn{x}} / x : x \in \freenames{P} \}
\end{mathpar}

will replace each (occurrence) of a free name $x$ in $P$ by
$\quotep{\procn{x}|\procn{x}}$.

Also, we will avail ourselves of the notation $x^{L}$ and $x^{R}$ to
denote injections of a name into disjoint copies of the name
space. There are numerous ways to accomplish this. One example can be
found in \cite{MeredithR05}. This notation overloads to vectors of
names: $\vec{x}^{\pi} := (x_{i}^{\pi} \; : \; 0 \leq i < |\vec{x}| )$ where $\pi \in \{L,R\}$.

We also use $P^{\Box} := P|\Box$.

In \cite{MeredithR05} an interpretation of the new operator is
given. It turns out that there are several possible interpretations
all enjoying the requisite algebraic properties of the operator (see
\cite{milner91polyadicpi}). We will therefore make liberal use of
$(\nu\; \vec{x})P$.

% subsection the_syntax_and_semantics_of_the_notation_system (end)   

\input{qm2pi.qmops} 

\input{qm2pi.sterngerlach} 

\input{qm2pi.metric} 

% section concurrent_process_calculi (end)

%\input{qm2pi.proofsketch}

% section proof sketch (end)

%\input{qm2pi.slviaknots} 

% section spatial logic via knots (end)

\input{qm2pi.conclusion}

% section conclusion (end)

%\input{qm2pi.dtcodes} 

% section wiring algorithm (end)

\input{qm2pi.ack} 

% section acknowledgments (end)

\newpage


\bibliographystyle{plain}   
\bibliography{../../biblios/main.bib}

\input{qm2pi.rhodetails}

\end{document}

 

% section concurrent_process_calculi (end)

%\documentclass[12pt]{llncs}
%\documentclass{jktr}

\usepackage[pdftex]{hyperref}                   
\usepackage {listings}
\usepackage {mathpartir}
\usepackage{bcprules}
%\usepackage{listings}
                       
\usepackage{graphicx} 
%\usepackage[margins=2.5cm,nohead,nofoot]{geometry}
%\usepackage{geometry}
\usepackage{amsfonts}
\usepackage{amstext}
\usepackage{latexsym}
\usepackage{amssymb}
\usepackage{color}


%\include{myPreamble}
\include{qm2pi.local} 

%\ifpdf
%\usepackage[pdftex]{graphicx}
%\else
%\usepackage{graphicx}
%\fi

 % \ifpdf
%  \usepackage{pdfsync}
%  \if


%\title{Brief Article}
%\author{David F. Snyder}
%\author{L.G. Meredith}

%\address{Dept. of Math., Texas State University--San Marcos, San Marcos, TX 78666}
       
\pagestyle{empty}


\begin{document}

\lstset{language=[Objective]Caml,frame=shadowbox}

\input{qm2pi.front}

% section front matter (end)

\input{qm2pi.intro} 
 
% section introduction (end)

% \input{qm2pi.knotations} 

% section notation (end)

\input{qm2pi.process.calculi} 

% section concurrent_process_calculi_and_spatial_logics_ (end)
    
%\input{qm2pi.knots2pi} 

%\input{qm2pi.trefoil} 

%\input{qm2pi.mainthm} 

% subsection basic_interpretation (end)

%\input{qm2pi.rho.presentation} 
\subsection{The syntax and semantics of the notation system}\label{sub:the_syntax_and_semantics_of_the_notation_system} % (fold)

We now summarize a technical presentation of the calculus that
embodies our theory of dynamics. The typical presentation of such a
calculus follows the style of giving generators and relations on
them. The grammar, below, describing term constructors, freely
generates the set of processes, $\Proc$. This set is then quotiented
by a relation known as structural congruence and it is over this set
that the notion of dynamics is expressed. This presentation is
essentially that of \cite{MeredithR05} with the addition of
polyadicity and summation. For readability we have relegated some of
the technical subtleties to an appendix.

\subsubsection{Process grammar}\label{subsub:process_grammar}

\begin{mathpar}
  \inferrule* [lab=synchronization] {} {{M} \bc \pzero \;|\; x?F \;|\; x!C }
  \and
  \inferrule* [lab=abstraction] {} {{F} \bc (x)P}
  \and
  \inferrule* [lab=concretion] {} {{C} \bc \langle Q \rangle}
  \and
  \inferrule* [lab=process] {} {{P,Q} \bc M \;| \;P|Q \;|\; @{x}}
  \and
  \inferrule* [lab=name] {} {{x} \bc \quotep{P}}
\end{mathpar} 

Note that $\vec{x}$ (resp. $\vec{P}$) denotes a vector of names
(resp. processes) of length $|\vec{x}|$ (resp. $|\vec{P}|$). We adopt
the following useful abbreviations.

\begin{mathpar}
   x?(\vec{y}).P := x.(\vec{y})P \and  x\clift{\vec{P}} := x.\clift{\vec{P}}
   \and x!(y) := \lift{x}{\dropn{y}}
   \and \Pi_{i=0}^{n-1}P_i := P_0 | \ldots | P_{n-1}
\end{mathpar}

\subsubsection{Structural congruence}

\paragraph{Free and bound names and alpha-equivalence.} At the
core of structural equivalence is alpha-equivalence which identifies
process that are the same up to a change of variable. Formally, we
recognize the distinction between free and bound names. The free names
of a process, $\freenames{P}$, may be calculated recursively as
follows:

\begin{mathpar}
\freenames{\pzero} := \emptyset
  \and \\
  \freenames{x?(y).P} := \{ x \} \cup (\freenames{P} \setminus \{ y \})
  \and 
  \freenames{x!\langle P \rangle} := \{ x \} \cup \{ P \} 
  \and \\
  \freenames{P|Q} := \freenames{P} \cup \freenames{Q}
  \and \\
  \freenames{@{x}} := \{ x \}
\end{mathpar}

$\pi$
$\quotep{\pi}$

$\freenames{-} : \pi \to \mathcal{P}(\quotep{\pi})$

\begin{eqnarray*}
  \freenames{\pzero} & := & \emptyset \\
  \freenames{x?(y).P} & := & \{ x \} \cup (\freenames{P} \setminus \{ y \}) \\
  \freenames{x!\langle P \rangle} & := & \{ x \} \cup \{ P \} \\
  \freenames{P|Q} & := & \freenames{P} \cup \freenames{Q} \\
  \freenames{\dropn{x}} & := & \{ x \}
\end{eqnarray*}

The bound names of a process, $\boundnames{P}$, are those names occurring in $P$
that are not free. For example, in $x?(y).0$, the name $x$ is free, while $y$ is bound.

\begin{mathpar}
  \inferrule* [lab=monoidal-laws] {} { P|Q \equiv Q|P \and P|0 \equiv P \and P|(Q|R) \equiv (P|Q)|R }
\end{mathpar}

\begin{mathpar}
  \inferrule* [lab=alpha-equivalence] {} { (x)P \equiv (y)P\{y/x\} \and y \not\in \freenames{P} }
\end{mathpar}

\begin{definition}
Then two processes, $P,Q$, are alpha-equivalent if $P = Q\{\vec{y}/\vec{x}\}$ for
some $\vec{x} \in \boundnames{Q},\vec{y} \in \boundnames{P}$, where $Q\{\vec{y}/\vec{x}\}$
denotes the capture-avoiding substitution of $\vec{y}$ for $\vec{x}$ in $Q$.
\end{definition}

\begin{definition}
  The {\em structural congruence} \cite{SangiorgiWalker} , $\equiv$,
  between processes is the least congruence containing
  alpha-equivalence, satisfying the abelian monoid laws
  (associativity, commutativity and $\pzero$ as identity) for parallel
  composition $|$ and for summation $+$.
\end{definition}

\subsection{Name equivalence}

We take name equivalence, written $\nameeq$, to be the smallest
equivalence relation generated by the following rules.

\begin{mathpar}
\inferrule*[lab=Quote-drop]
{ }
{ \quotep{@{x}} \nameeq x }

\inferrule*[lab=Struct-equiv]
{ P \scong Q }
{ \quotep{P} \nameeq \quotep{Q} }
\end{mathpar}

The astute reader will have noticed that the mutual recursion of names
and processes imposes a mutual recursion on alpha-equivalence and
structural equivalence via name-equivalence. Fortunately, all of this
works out pleasantly and we may calculate in the natural way, free of
concern. The reader interested in the details is referred to the
appendix \ref{appendix:rho_details}.

\subsection{Substitution}

We use $\Proc$ for the set of processes, $\QProc$ for the set of
names, and $\id{\{}\vec{y} / \vec{x} \id{\}}$ to denote partial maps,
$s : \QProc \rightarrow \QProc$. A map, $s$ lifts, uniquely, to a map
on process terms, $\widehat{s} : \Proc \rightarrow \Proc$ by the
following equations.

\begin{mathpar}
  (0) \psubstp{Q}{P} := 0 \\
  (R \juxtap S) \psubstp{Q}{P}
  :=    
  (R)\psubstp{Q}{P} \juxtap (S) \psubstp{Q}{P} \\
  (x?(y).R) \psubstp{Q}{P}    
  :=    
  (x)\substp{Q}{P} (z)\concat( (R \psubstn{z}{y}) \psubstp{Q}{P} ) \\
  (\lift{x}{R}) \psubstp{Q}{P}  
  :=
  \lift{(x)\substp{Q}{P}}{ R \psubstp{Q}{P} } \\
%   (\dropn{x})  \psubstp{Q}{P}       
%   := 
%   \left\{ 
%     \begin{array}{ccc} 
%       \dropn{\quotep{Q}} & & x \nameeq \quotep{P} \\
%       \dropn{x} & & otherwise \\
%     \end{array}
%   \right. 
  (\dropn{x})  \psubstp{Q}{P}       
  := 
  \left\{ 
    \begin{array}{ccc} 
      Q & & x \nameeq \quotep{P} \\
      \dropn{x} & & otherwise \\
    \end{array}
  \right.
\end{mathpar}
 

where

\begin{eqnarray}
  (x)\id{\{} \lpquote Q \rpquote / \lpquote P \rpquote \id{\}}            = 
  \left\{ 
    \begin{array}{ccc}
      \lpquote Q \rpquote & & x \nameeq \lpquote P \rpquote \\
      x & & otherwise \\
    \end{array}
  \right. \nonumber
\end{eqnarray}

and $z$ is chosen distinct from $\quotep{P}$, $\quotep{Q}$, the free
names in $Q$, and all the names in $R$. Our $\alpha$-equivalence will
be built in the standard way from this substitution.

\begin{remark}\label{rem:no_self_referential_names}
  One consequence of these definitions is that $\forall P. \quotep{P}
  \not\in \freenames{P}$.
\end{remark}

\subsection{ Dynamic quote: an example }

Anticipating something of what's to come, consider applying the
substitution, $\widehat{\id{\{}u / z \id{\}}}$, to the following pair
of processes, $\lift{w}{y!(z)}$ and $w[ \lpquote y!(z) \rpquote ]$.

\begin{eqnarray}
	\lift{w}{y!(z)}\widehat{\id{\{}u / z \id{\}}}
		& = &
		\lift{w}{y!(u)} \nonumber\\
	w[ \lpquote y!(z) \rpquote ] \widehat{ \id{\{}u / z \id{\}} }
		& = &
		w[ \lpquote y!(z) \rpquote ] \nonumber
\end{eqnarray}

Because the body of the process between quotes is impervious to
substitution, we get radically different answers. In fact, by
examining the first process in an input context,
e.g. $x?(z).\lift{w}{y!(z)}$, we see that the process under the lift
operator may be shaped by prefixed inputs binding a name inside it. In
this sense, the lift operator will be seen as a way to dynamically
construct processes before reifying them as names.

Finally equipped with these standard features we can present the
dynamics of the calculus.

\subsubsection{Operational semantics} 

Finally, we introduce the computational dynamics. What marks these
algebras as distinct from other more traditionally studied algebraic
structures, e.g. vector spaces or polynomial rings, is the manner in
which dynamics is captured. In traditional structures, dynamics is typically
expressed through morphisms between such structures, as in linear maps
between vector spaces or morphisms between rings. In algebras
associated with the semantics of computation, the dynamics is
expressed as part of the algebraic structure itself, through a
reduction reduction relation typically denoted by $\red$. Below, we
give a recursive presentation of this relation for the calculus used
in the encoding.

$\red \subseteq \pi \times \pi$
$\red : \pi \to \mathcal{P}(\pi)$

\begin{mathpar}
  \inferrule* [lab=Comm] { \textsf{match}( x_{src}, x_{trgt} ) } { x_{trgt}?(y)P \; | \; x_{src}!\langle {Q} \rangle \red P\{\quotep{Q}/y}\} }
  \and \\
  \inferrule* [lab=Par] {{P} \red {P}'} {{{P} | {Q}} \red {{P}' | {Q}}}
  \and
  \inferrule* [lab=Equiv]{{{P} \scong {P}'} \andalso {{P}' \red {Q}'} \andalso {{Q}' \scong {Q}}}{{P} \red {Q}}
\end{mathpar}

\begin{eqnarray*}
  match_{\equiv} (\quotep{P},\quotep{Q}) & := & P \equiv Q \\
  match_{\dagger}(\quotep{P},\quotep{Q}) & := & \forall R. P|Q \red^{*} R => R \red^{*} 0 \\
  match_{K}(\quotep{P},\quotep{Q}) & := & K \mbox{ for some context } K
\end{eqnarray*}

$u?(x)P | u!\langle Q \rangle \red P\{\quotep{Q}/x\}$

%We write $\wred$ for $\red^*$, and $P\red$ if $\exists Q $ such that $ P \red Q$.
We write $P\red$ if $\exists Q $ such that $ P \red Q$ and $P\not\red$, otherwise.

\section{Replication}

As mentioned before, it is known that replication (and hence
recursion) can be implemented in a higher-order process algebra
\cite{SangiorgiWalker}. As our first example of calculation with the
machinery thus far presented we give the construction explicitly in
the {\rhoc}.

\begin{eqnarray}
	D_{x} & := & \prefix{x}{y}{(\binpar{\outputp{x}{y}}{@{y}})} \nonumber\\
	\bangp_{x}{P} & := & \binpar{{x}!\langle{\binpar{D_{x}}{P}}\rangle}{D_{x}} \nonumber
\end{eqnarray}

\begin{eqnarray}
	\bangp_{x}{P} & & \nonumber\\
	=
	& {x}!\langle{(\prefix{x}{y}{(\outputp{x}{y} | @{y})) | P}}\rangle 
	      | \prefix{x}{y}{(\outputp{x}{y} | @{y})} & \nonumber\\
	\red
	& (\outputp{x}{y} | @{y})\substn{\quotep{(\prefix{x}{y}{(@{y} | \outputp{x}{y})) | P}}}{y} & \nonumber\\
	=
	& \outputp{x}{\quotep{(\prefix{x}{y}{(\outputp{x}{y} | @{y})) | P}}}
	  | {(\prefix{x}{y}{(\outputp{x}{y} | @{y})) | P}} & \nonumber\\
	\red
	& \ldots & \nonumber\\
	\red^*
	& P | P | \ldots & \nonumber
\end{eqnarray}

Of course, this encoding, as an implementation, runs away, unfolding
$\bangp{P}$ eagerly. A lazier and more implementable replication
operator, restricted to input-guarded processes, may be obtained as follows.

\begin{eqnarray}
\bangp{\prefix{u}{v}{P}} 
	:= 
	\binpar{\lift{x}{\prefix{u}{v}{(\binpar{D(x)}{P})}}}{D(x)} \nonumber
\end{eqnarray}

\begin{remark}
  Note that the lazier definition still does not deal with summation
  or mixed summation (i.e. sums over input and output). The reader is
  invited to construct definitions of replication that deal with these
  features. 

  Further, the definitions are parameterized in a name, $x$. Can you,
  gentle reader, make a definition that eliminates this parameter and
  guarantees no accidental interaction between the replication
  machinery and the process being replicated -- i.e. no accidental
  sharing of names used by the process to get its work done and the
  name(s) used by the replication to effect copying. This latter
  revision of the definition of replication is crucial to obtaining
  the expected identity $!!P \sim !P$.
\end{remark}

\begin{remark}\label{rem:paradoxical_combinator}
  The reader familiar with the lambda calculus will have noticed the
  similarity between $D$ and the paradoxical combinator.

  [Ed. note: the existence of this seems to suggest we have to be more
  restrictive on the set of processes and names we admit if we are to
  support no-cloning.]
\end{remark}

\subsubsection{Bisimulation}

The computational dynamics gives rise to another kind of equivalence,
the equivalence of computational behavior. As previously mentioned
this is typically captured \emph{via} some form of bisimulation.

% The notion we use in this paper is weak barbed bisimulation
% \cite{milner91polyadicpi}.

The notion we use in this paper is derived from weak barbed
bisimulation \cite{milner91polyadicpi}. 

\begin{definition}
An \emph{observation relation}, $\downarrow_{\mathcal N}$, over a set
of names, $\mathcal N$, is the smallest relation satisfying the rules
below.

\infrule[Out-barb]{y \in {\mathcal N}, \; x \nameeq y}
		  {\outputp{x}{v} \downarrow_{\mathcal N} x}
\infrule[Par-barb]{\mbox{$P\downarrow_{\mathcal N} x$ or $Q\downarrow_{\mathcal N} x$}}
		  {\binpar{P}{Q} \downarrow_{\mathcal N} x}

We write $P \Downarrow_{\mathcal N} x$ if there is $Q$ such that 
$P \wred Q$ and $Q \downarrow_{\mathcal N} x$.
\end{definition}

\begin{definition}
%\label{def.bbisim}
An  ${\mathcal N}$-\emph{barbed bisimulation} over a set of names, ${\mathcal N}$, is a symmetric binary relation 
${\mathcal S}_{\mathcal N}$ between agents such that $P\rel{S}_{\mathcal N}Q$ implies:
\begin{enumerate}
\item If $P \red P'$ then $Q \wred Q'$ and $P'\rel{S}_{\mathcal N} Q'$.
\item If $P\downarrow_{\mathcal N} x$, then $Q\Downarrow_{\mathcal N} x$.
\end{enumerate}
$P$ is ${\mathcal N}$-barbed bisimilar to $Q$, written
$P \wbbisim_{\mathcal N} Q$, if $P \rel{S}_{\mathcal N} Q$ for some ${\mathcal N}$-barbed bisimulation ${\mathcal S}_{\mathcal N}$.
\end{definition}

$\mathcal{R} \subseteq \pi \times \pi$

$P \mathcal{R} Q => \forall P'. P \red P' \Rightarrow \exists Q'. Q \red Q', P' \mathcal{R} Q'$

$P \vdash x \Rightarrow Q \vdash x$

\begin{mathpar}
  \inferrule*[lab=Out-barb]{x \nameeq y}{{y}!\langle{Q}\rangle \vdash x}
  \and
  \inferrule*[lab=Par-barb]{\mbox{$P\vdash x$ or $Q\vdash x$}}{\binpar{P}{Q} \vdash x}
\end{mathpar}

\subsubsection{Contexts}

One of the principle advantages of computational calculi like the
$\pi$-calculus is a well-defined notion of context,
contextual-equivalence and a correlation between
contextual-equivalence and notions of bisimulation. The notion of
context allows the decomposition of a process into (sub-)process and
its syntactic environment, its context. Thus, a context may be
thought of as a process with a ``hole'' (written $\Box$) in it. The
application of a context $M$ to a process $P$, written $M[P]$, is
tantamount to filling the hole in $M$ with $P$. In this paper we do
not need the full weight of this theory, but do make use of the notion
of context in the proof the main theorem. 

\begin{mathpar}
  \inferrule* [lab=summation] {} {{M_{M},M_{N}} \bc \Box \;|\; x.M_{A} \;|\; M_{M}+M_{N}}
  \and
  \inferrule* [lab=agent] {} {{M_{A}} \bc (\vec{x})M_{P} \;| \; \clift{P_0,\ldots,M_{P},\ldots,P_N}}
  \and \\
  \inferrule* [lab=process] {} {{M_{P}} \bc M_{N} \;| \;P|M_{P} }
\end{mathpar} 

\begin{mathpar}
  \inferrule* [lab=sychronization] {} {M_{N} \bc \Box \;|\; x?M_{F} \;|\; x!M_{C}}
  \and
  \inferrule* [lab=abstraction] {} {{M_{F}} \bc (x)M_{P} }
  \and
  \inferrule* [lab=concretion] {} {{M_{C}} \bc \langle M_{P} \rangle }
  \and \\
  \inferrule* [lab=process] {} {{M_{P}} \bc M_{N} \;| \;P|M_{P} }
\end{mathpar}

\begin{definition}[contextual application] Given a context $M$, and
  process $P$, we define the \emph{contextual application}, $M[P] :=
  M\{P/\Box\}$. That is, the contextual application of M to P is the
  substitution of $P$ for $\Box$ in $M$.
\end{definition}

$\meaningof{-} : L \to \mathcal{P}(\pi)$

\begin{mathpar}
  \inferrule* [lab=collection] {} {\meaningof{true} = \pi, \and \meaningof{~E} = \pi \setminus \meaningof{E}, \and \meaningof{E_{1} \& E_{2}} = \meaningof{E_{1}} \cap \meaningof{E_{2}}}
\end{mathpar}

\begin{mathpar}
  \inferrule* [lab=structure] {} {\meaningof{0} = \{ P \in \pi | P \equiv 0 \}, \and \\ \meaningof{E_1 | E_2} = \{ P \in \pi | P \equiv P_{1} | P_{2}, P_{1} \in \meaningof{E_{1}}, P_{2} \in \meaningof{E_2}\} }
\end{mathpar}

\begin{mathpar}
 \inferrule* [lab=behavior] {} {\meaningof{\langle a?b \rangle E} = \{ P \in \pi | P \equiv Q | u?(y)P', \\ \and \\\\ \and \\ \;\;\; u \in \meaningof{a}, \forall z.P'\{z/y\} \in \meaningof{E\{z/b\}}\}, \and \\ \meaningof{a!E} = \{ P \in \pi | P \equiv Q | x!\langle P' \rangle, x \in \meaningof{a} P' \in \meaningof{E}\} }
\end{mathpar}

\begin{mathpar}
 \inferrule* [lab=nominal] {} {\meaningof{\quotep{E}} = \{ \quotep{P} \in \quotep{\pi} | P \in \meaningof{E} \}, \and \meaningof{\quotep{P}} = \{ \quotep{Q} \in \quotep{\pi} | P \equiv Q \} \and \\ \meaningof{@\quotep{E}} = \{ P \in \pi | P \equiv @x, x \in \meaningof{E} \}}
\end{mathpar}

\begin{eqnarray*}
  \\
  \meaningof{-} : TS \to ST
\end{eqnarray*}

\begin{eqnarray*}
  \\
  L : TS \to ST
\end{eqnarray*}

\begin{eqnarray*}
  \\
  P \models E \iff P \in \meaningof{E}
\end{eqnarray*}

\begin{eqnarray*}
  P \approx_{L} Q \iff \forall E \in L. P \models E \iff Q \models E
\end{eqnarray*}

\begin{eqnarray*}
  P \approx_{K} Q
\end{eqnarray*}

\begin{eqnarray*}
  P \approx Q
\end{eqnarray*}

$\approx_{K} = \approx = \approx_{L}$

\subsubsection{Contextual duality}

Note that contexts extend the quotation operation to a family of
operations from processes to names. Given a context, $M$, we can
define a \emph{nominal context}, $\quotep{M}$ by $\quotep{M}[P] :=
\quotep{M[P]}$. To foreshadow what is to come we observe that these
operations enjoy a duality with processes very much like the duality
between vectors and maps from vectors to scalars.

Further, because the calculus is essentially higher-order, we have a
correspondence between contexts and processes. More specifically,
given a name $x$ and a context $M$ we can construct $M^{*}_{x}$ such
that 

\begin{mathpar}
  M^{*}_{x} | \lift{x}{P} \red M[P]
\end{mathpar}

namely,

\begin{mathpar}
  M^{*}_{x} := x?(u).M[\dropn{u}]
\end{mathpar}

The dependence of $M^{*}_{x}$ on a name makes it an abstraction, 

\begin{mathpar}
  M^{*} := (x)x?(u).M[\dropn{u}]
\end{mathpar}

\subsection{Additional notation}

It will sometimes be convenient to denote the process a name
quotes. We already have the notation $x = \quotep{P}$, but it will be
convenient to introduce an alternate notation, $\procn{x}$, when we
want to emphasize the connection to the use of the name. Note that, by
virtue of name equivalence, $\quotep{\procn{x}} \nameeq x$; so, the
notation is consistent with previous definitions.

Further, because names have structure it is possible to effect
substitutions on the basis of that structure. This means we need to
upgrade our notation for substitutions, which we accomplish by
adapting comprehension notation. Thus,

\begin{mathpar}
  P\{ y / x : x \in S \}
\end{mathpar}

is interpreted to mean the process derived from P by replacing (in a
capture-avoiding manner) each occurrence of $x$ in $S$ by $y$. For example,

\begin{mathpar}
  P\{ \quotep{\procn{x}|\procn{x}} / x : x \in \freenames{P} \}
\end{mathpar}

will replace each (occurrence) of a free name $x$ in $P$ by
$\quotep{\procn{x}|\procn{x}}$.

Also, we will avail ourselves of the notation $x^{L}$ and $x^{R}$ to
denote injections of a name into disjoint copies of the name
space. There are numerous ways to accomplish this. One example can be
found in \cite{MeredithR05}. This notation overloads to vectors of
names: $\vec{x}^{\pi} := (x_{i}^{\pi} \; : \; 0 \leq i < |\vec{x}| )$ where $\pi \in \{L,R\}$.

We also use $P^{\Box} := P|\Box$.

In \cite{MeredithR05} an interpretation of the new operator is
given. It turns out that there are several possible interpretations
all enjoying the requisite algebraic properties of the operator (see
\cite{milner91polyadicpi}). We will therefore make liberal use of
$(\nu\; \vec{x})P$.

% subsection the_syntax_and_semantics_of_the_notation_system (end)   

\input{qm2pi.qmops} 

\input{qm2pi.sterngerlach} 

\input{qm2pi.metric} 

% section concurrent_process_calculi (end)

%\input{qm2pi.proofsketch}

% section proof sketch (end)

%\input{qm2pi.slviaknots} 

% section spatial logic via knots (end)

\input{qm2pi.conclusion}

% section conclusion (end)

%\input{qm2pi.dtcodes} 

% section wiring algorithm (end)

\input{qm2pi.ack} 

% section acknowledgments (end)

\newpage


\bibliographystyle{plain}   
\bibliography{../../biblios/main.bib}

\input{qm2pi.rhodetails}

\end{document}



% section proof sketch (end)

%\section{Unlikely characters: spatial logic for
  knots}\label{sub:characteristic_formulae} % (fold)

Associated to the mobile process calculi are a family of logics known
as the Hennessy-Milner logics. These logics typically enjoy a
semantics interpreting formulae as sets of processes that when
factored through the encoding outlined above allows an identification
of classes of knots with logical formulae. In the context of this
encoding the sub-family known as the spatial logics \cite{CairesC03}
\cite{CairesC04} \cite{Caires04} are of particular interest providing
several important features for expressing and reasoning about
properties (i.e. classes) of knots. We hint here at how this may be done.

%\begin{description}
%\item [structural connectives] 
\subsubsection{Structural connectives} The spatial logics enjoy
structural connectives corresponding, at the logical level, to the
parallel composition ($P | Q$) and new name ($(\nu \; x)P$)
connectives for processes. As illustrated in the examples below, these
connectives are extremely expressive given the shape of our encoding.
%\item [decideable satisfaction]

\subsubsection{Decideable satisfaction}
In \cite{Caires04} the satisfaction relation is shown to be decideable
for a rich class of processes. It further turns out that the image of
the our encoding is a proper subset of that class. This result
provides the basis for an algorithm by which to search for knots
enjoying a given property.
%\item [characteristic formulae]

\subsubsection{Characteristic formulae}
In the same paper \cite{Caires04} , Caires presents a means of calculating
characteristic formulae, selecting equivalence classes of processes
up to a pre--specified depth limit on the support set of names. Composed with our
encoding, this characteristic formula can be used to select
characteristic formulae for knots.
%\end{description}

\subsubsection{Spatial logic formulae}

The grammar below (segmented for comprehension) summarizes the syntax
of spatial logic formulae. We employ illustrative examples in the
sequel to provide an intuitive understanding of their meaning
referring the reader to \cite{Caires04} for a more detailed explication
of the semantics.

\begin{mathpar}
  \inferrule* [lab=boolean] {} {{A,B} \bc T \;|\; \neg A \;|\; A \wedge B \;|\; \eta = \eta'}
  \and
  \inferrule* [lab=spatial] {} {|\; \pzero \;|\; A | B \;|\; x \text{\textregistered} A \;|\; \forall x . A \;|\;  H x . A}
  \and
  \inferrule* [lab=behavioral] {} {|\; \alpha . A}
  \and 
  \inferrule* [lab=recursion] {} {|\; X(\vec{u}) \;|\; \mu X(\vec{u}) . A}
  \and
  \inferrule* [lab=action] {} {\alpha \bc \langle x?(\vec{y}) \rangle \;|\; \langle x!(\vec{y}) \rangle \;|\; \langle \tau \rangle}
  \and 
  \inferrule* [lab=name] {} {\eta \bc x \;|\; \tau}
\end{mathpar} 

% subsection characteristic_formulae (end)   	 

\subsection{Example formulae}\label{sub:example_formulae_} % (fold)

\subsubsection{Crossing as formula.}
% 
% \begin{align*}
%   \frac{d}{dx} \sin x &= \cos x 
%   & \frac{d}{dx} e^x &= e^x \\
%   \frac{d}{dx} \cos x &= - \sin x 
%   & \frac{d}{dx} \log x &= \frac{1}{x} \\
% \end{align*} 

\begin{align*}
 \mu C(x_{0},x_{1},y_{0},y_{1},u).&(\langle x_{0}?(z) \rangle(\langle u! \rangle\langle y_{1}!z \rangle C(x_{0},x_{1},y_{0},y_{1},u)) & \\
  & \wedge \langle y_{1}?(z) \rangle (\langle u! \rangle \langle x_{0}!z \rangle C(x_{0},x_{1},y_{0},y_{1},u)) & \\
  & \wedge \langle x_{1}?(z) \rangle (\langle u? \rangle \langle y_{0}!z \rangle C(x_{0},x_{1},y_{0},y_{1},u)) & \\
  & \wedge \langle y_{0}?(z) \rangle (\langle u? \rangle \langle x_{1}!z \rangle C(x_{0},x_{1},y_{0},y_{1},u))) &
\end{align*}

The lexicographical similarity between the shape of this formulae and
the shape of definition of the process representing a crossing reveals
the intuitive meaning of this formulae. It describes the capabilities
of a process that has the right to represent a crossing. For example
it picks out processes that may perform an input on the port $x_0$ in
its initial menu of capabilities. What differentiates the formula
from the process, however, is that the crossing process is the
smallest candidate to satisfy the formula. Infinitely many other
processes -- with internal behavior hidden behind this interface, so
to speak -- also satisfy this formula. Even this simple formula,
then, can be seen to open a new view onto knots, providing a
computational interpretation of \emph{virtual} knots.

Note that this formula is derived by hand. A similar formula can be
derived by employing Caires' calculation of characteristic formula
\cite{Caires04} to the process representing a crossing. In light of
this discussion, we let
$\meaningof{C}_{\phi}(x0,x1,y0,y1,u)$ denote a formula specifying the
dynamics we wish to capture of a crossing. To guarantee we preserve
the shape of the interface and minimal semantics we demand that
$\meaningof{C}_{\phi}(x0,x1,y0,y1,u) \Rightarrow
\textbf{C}(x0,x1,y0,y1,u)$ where $\textbf{C}(x0,x1,y0,y1,u)$ denotes
the formula above.
                            
\subsubsection{Crossing number constraints.}
The moral content of the context lemma (Lemma \ref{context}) is that the notion of
``locality'' in the Reidemeister moves is effectively captured by the
parallel composition operator of the process calculus. This intuition
extends through the logic. Given a formula,
$\meaningof{C}_{\phi}(x0,x1,y0,y1,u)$, we can use the structural
connectives to specify constraints on crossing numbers, such as at
least $n$ crossings, or exactly $n$ crossings.
\begin{mathpar}
  \inferrule* [lab=at-least-n] {} { K^{\geq n}_{\phi}(\vec{xs},\vec{ys}) := \Pi_{i=0}^{n-1} Hu . \meaningof{C}_{\phi}(xs_i,ys_i,u) | T }
  \and 
  \inferrule* [lab=exactly-n] {} { K^{= n}_{\phi}(\vec{xs},\vec{ys}) := \Pi_{i=0}^{n-1} Hu . \meaningof{C}_{\phi}(xs_i,ys_i,u) | \neg (\forall x_0,y_0,x_1,y_1,u . \meaningof{C}_{\phi}(x_0,y_0,x_1,y_1,u) | T) }
\end{mathpar}

To round out this section, recall that the encoding of an $n$-crossing
knot decomposes into a parallel composition of $n$ \emph{copies} of a
crossing process together with a wiring harness. To specify different
knot classes with the same crossing number amounts to specifying
logical constraints on the wiring harness. In the interest of space,
we defer examples to a forthcoming paper. Suffice it to say that both
the conditions ``alternating knot'' and ``contains the tangle
corresponding to 5/3'' are expressible. For example, it is possible to
calculate the characteristic formula of a process corresponding to the
tangle 5/3 and conjoin it into the classifying formula via the
composition connective of the logic.

Finally, we wish to observe that it is entirely within reason to
contemplate a more domain-specific version of spatial logic tailored
to the shape of processes in the image of the encoding. Such a
domain-specific logic would have a better claim to the title formal
language of knot properties.

% subsection example_formulae_ (end)

% section knots_as_processes (end) 

% section spatial logic via knots (end)

\section{Conclusions and future work}

\paragraph{Testing physical space}
You, gentle reader, may wonder why of all the theorems to be proved
given this set up we pick the one above. In some sense it's hardly
central to quantum mechanics. We see it as central in the sense that
it firmly establishes a notion of physical space arising from a notion
of the equivalence of behavior. Relating bisimulation to a metric is a
big step forward, but one is faced with interpreting the relationship
of that metric space to something more physical. Quantum mechanical
notions of ``physical'' space are still far from intuitive, but by
relating this idea of distance as testing to calculations that predict
physical circumstances we are making a not insignificant step forward
toward an understanding of the physical space we inhabit as
essentially dynamic.

\paragraph{Effectivity and simulation}
One of the observations we have yet to make is that the entire program
spelled out here is effective. We have built various interpreters for
the reflective calculus at work in this interpretation. In principle,
then, we can simulate quantum mechanics on a computer. The place where
the simulation may lose fidelity is the infinitely branching summation
for the annihilator.

In this connection i also want to point out that the evaluation style
calculation of the inner product puts the non-determinism of the
summation right at the heart of measurement. This suggests that
Milner's original reduction-based formulation of the dynamics of his
calculi in terms of sums was not just notationally suggestive of a
notion of measure-and-continue but captured some significant part of
the physics.

\paragraph{Quantum continuations}
In light of this last observation i want to point out that the
predominant account of quantum mechanics is missing a key aspect of a
truly compositional story of the physical situation. In a real lab,
when a measurement is made the observation can be made to feed into
another device that then makes another measurement conditioned on the
results of the first. This means that after the superposition was
collapsed the entire experimental set up remained in
superposition. While QM offers a means of writing this down it doesn't
quite line up well with the well-trodden formulation of computation
and continuation that we see so succinctly expressed in Milner's
calculi. This suggests that there might be advantages to this account
of dynamics waiting to be explored.

\paragraph{Quantum logic}
In this connection, we also note that by virtue of having the
Hennessy-Milner construction, we can pull the construction through the
interpretation of QM. This gives us a natural candidate for a quantum
logic that enjoys an extremely tight connection with it's domain of
interpretation, making the construction much less ad hoc (rather it is
the image of functor!).

\paragraph{Quantum probabiity}
i have questions about the basis of the interpretation of inner
product as probability amplitude. In particular, using which
axiomatization of probability theory does the notion of probability
amplitude earn the right to be so dubbed? In other words, where is the
proof that the operation for calculating a probability amplitude (and
then squaring) satisfies the axioms of what it means to calculate a
probability? Even if such a proof exists (i have yet to find it in the
literature), i wonder if it might not be possible to turn things on
their heads. Can we view the calculation of the probability amplitude
as an axiomatization of probability? If so, then the definition we
give for calculating probability amplitude may provide the basis for
an \emph{effective} theory of probability.

\paragraph{Quantum vs ``biological'' information}
Finally, i want to conclude with a more philosophical observation. At
a recent workshop in which QM was a predominant topic i noticed
something about quantum information. The speaker was giving a riveting
discussion of axiomatic QM and showing how properties of ``no
cloning'' and ``no deleting'' emerged as consequences of the
axiomatization. Theorems of this form are necessary to give us a sense
of confidence that our axioms characterize the physical theory. What
struck me, though, was that if quantum information is neither erasable
nor replicable it is markedly different from \emph{life}. Two of the
things we know about life is that

\begin{itemize}
  \item it ends;
  \item to gain some measure of persistence, to transcend it's
    finitude it is imminently copyable.
\end{itemize}

Both of these qualities are summarized succinctly in the aphorism: all
flesh is grass. For me these two kinds of ``information'' -- call them
quantum and biological -- are end points on a spectrum of strategies
for persistence. At one end, we have those curious entities that enjoy
uniqueness and permanence; at the other, we have those who in the face
of a certain end and an uncertain present make a go of passing
something on. To me one of the more remarkable aspects of the latter
strategy is that in the presence of noise (and certain features of
copying) we get a kind of dynamism, a chance for improvement against a
given persistent condition.

% subsection other_calculi_other_bisimulations_and_geometry_as_behavior (end)




% section conclusion (end)

%\documentclass[12pt]{llncs}
%\documentclass{jktr}

\usepackage[pdftex]{hyperref}                   
\usepackage {listings}
\usepackage {mathpartir}
\usepackage{bcprules}
%\usepackage{listings}
                       
\usepackage{graphicx} 
%\usepackage[margins=2.5cm,nohead,nofoot]{geometry}
%\usepackage{geometry}
\usepackage{amsfonts}
\usepackage{amstext}
\usepackage{latexsym}
\usepackage{amssymb}
\usepackage{color}


%\include{myPreamble}
\include{qm2pi.local} 

%\ifpdf
%\usepackage[pdftex]{graphicx}
%\else
%\usepackage{graphicx}
%\fi

 % \ifpdf
%  \usepackage{pdfsync}
%  \if


%\title{Brief Article}
%\author{David F. Snyder}
%\author{L.G. Meredith}

%\address{Dept. of Math., Texas State University--San Marcos, San Marcos, TX 78666}
       
\pagestyle{empty}


\begin{document}

\lstset{language=[Objective]Caml,frame=shadowbox}

\input{qm2pi.front}

% section front matter (end)

\input{qm2pi.intro} 
 
% section introduction (end)

% \input{qm2pi.knotations} 

% section notation (end)

\input{qm2pi.process.calculi} 

% section concurrent_process_calculi_and_spatial_logics_ (end)
    
%\input{qm2pi.knots2pi} 

%\input{qm2pi.trefoil} 

%\input{qm2pi.mainthm} 

% subsection basic_interpretation (end)

%\input{qm2pi.rho.presentation} 
\subsection{The syntax and semantics of the notation system}\label{sub:the_syntax_and_semantics_of_the_notation_system} % (fold)

We now summarize a technical presentation of the calculus that
embodies our theory of dynamics. The typical presentation of such a
calculus follows the style of giving generators and relations on
them. The grammar, below, describing term constructors, freely
generates the set of processes, $\Proc$. This set is then quotiented
by a relation known as structural congruence and it is over this set
that the notion of dynamics is expressed. This presentation is
essentially that of \cite{MeredithR05} with the addition of
polyadicity and summation. For readability we have relegated some of
the technical subtleties to an appendix.

\subsubsection{Process grammar}\label{subsub:process_grammar}

\begin{mathpar}
  \inferrule* [lab=synchronization] {} {{M} \bc \pzero \;|\; x?F \;|\; x!C }
  \and
  \inferrule* [lab=abstraction] {} {{F} \bc (x)P}
  \and
  \inferrule* [lab=concretion] {} {{C} \bc \langle Q \rangle}
  \and
  \inferrule* [lab=process] {} {{P,Q} \bc M \;| \;P|Q \;|\; @{x}}
  \and
  \inferrule* [lab=name] {} {{x} \bc \quotep{P}}
\end{mathpar} 

Note that $\vec{x}$ (resp. $\vec{P}$) denotes a vector of names
(resp. processes) of length $|\vec{x}|$ (resp. $|\vec{P}|$). We adopt
the following useful abbreviations.

\begin{mathpar}
   x?(\vec{y}).P := x.(\vec{y})P \and  x\clift{\vec{P}} := x.\clift{\vec{P}}
   \and x!(y) := \lift{x}{\dropn{y}}
   \and \Pi_{i=0}^{n-1}P_i := P_0 | \ldots | P_{n-1}
\end{mathpar}

\subsubsection{Structural congruence}

\paragraph{Free and bound names and alpha-equivalence.} At the
core of structural equivalence is alpha-equivalence which identifies
process that are the same up to a change of variable. Formally, we
recognize the distinction between free and bound names. The free names
of a process, $\freenames{P}$, may be calculated recursively as
follows:

\begin{mathpar}
\freenames{\pzero} := \emptyset
  \and \\
  \freenames{x?(y).P} := \{ x \} \cup (\freenames{P} \setminus \{ y \})
  \and 
  \freenames{x!\langle P \rangle} := \{ x \} \cup \{ P \} 
  \and \\
  \freenames{P|Q} := \freenames{P} \cup \freenames{Q}
  \and \\
  \freenames{@{x}} := \{ x \}
\end{mathpar}

$\pi$
$\quotep{\pi}$

$\freenames{-} : \pi \to \mathcal{P}(\quotep{\pi})$

\begin{eqnarray*}
  \freenames{\pzero} & := & \emptyset \\
  \freenames{x?(y).P} & := & \{ x \} \cup (\freenames{P} \setminus \{ y \}) \\
  \freenames{x!\langle P \rangle} & := & \{ x \} \cup \{ P \} \\
  \freenames{P|Q} & := & \freenames{P} \cup \freenames{Q} \\
  \freenames{\dropn{x}} & := & \{ x \}
\end{eqnarray*}

The bound names of a process, $\boundnames{P}$, are those names occurring in $P$
that are not free. For example, in $x?(y).0$, the name $x$ is free, while $y$ is bound.

\begin{mathpar}
  \inferrule* [lab=monoidal-laws] {} { P|Q \equiv Q|P \and P|0 \equiv P \and P|(Q|R) \equiv (P|Q)|R }
\end{mathpar}

\begin{mathpar}
  \inferrule* [lab=alpha-equivalence] {} { (x)P \equiv (y)P\{y/x\} \and y \not\in \freenames{P} }
\end{mathpar}

\begin{definition}
Then two processes, $P,Q$, are alpha-equivalent if $P = Q\{\vec{y}/\vec{x}\}$ for
some $\vec{x} \in \boundnames{Q},\vec{y} \in \boundnames{P}$, where $Q\{\vec{y}/\vec{x}\}$
denotes the capture-avoiding substitution of $\vec{y}$ for $\vec{x}$ in $Q$.
\end{definition}

\begin{definition}
  The {\em structural congruence} \cite{SangiorgiWalker} , $\equiv$,
  between processes is the least congruence containing
  alpha-equivalence, satisfying the abelian monoid laws
  (associativity, commutativity and $\pzero$ as identity) for parallel
  composition $|$ and for summation $+$.
\end{definition}

\subsection{Name equivalence}

We take name equivalence, written $\nameeq$, to be the smallest
equivalence relation generated by the following rules.

\begin{mathpar}
\inferrule*[lab=Quote-drop]
{ }
{ \quotep{@{x}} \nameeq x }

\inferrule*[lab=Struct-equiv]
{ P \scong Q }
{ \quotep{P} \nameeq \quotep{Q} }
\end{mathpar}

The astute reader will have noticed that the mutual recursion of names
and processes imposes a mutual recursion on alpha-equivalence and
structural equivalence via name-equivalence. Fortunately, all of this
works out pleasantly and we may calculate in the natural way, free of
concern. The reader interested in the details is referred to the
appendix \ref{appendix:rho_details}.

\subsection{Substitution}

We use $\Proc$ for the set of processes, $\QProc$ for the set of
names, and $\id{\{}\vec{y} / \vec{x} \id{\}}$ to denote partial maps,
$s : \QProc \rightarrow \QProc$. A map, $s$ lifts, uniquely, to a map
on process terms, $\widehat{s} : \Proc \rightarrow \Proc$ by the
following equations.

\begin{mathpar}
  (0) \psubstp{Q}{P} := 0 \\
  (R \juxtap S) \psubstp{Q}{P}
  :=    
  (R)\psubstp{Q}{P} \juxtap (S) \psubstp{Q}{P} \\
  (x?(y).R) \psubstp{Q}{P}    
  :=    
  (x)\substp{Q}{P} (z)\concat( (R \psubstn{z}{y}) \psubstp{Q}{P} ) \\
  (\lift{x}{R}) \psubstp{Q}{P}  
  :=
  \lift{(x)\substp{Q}{P}}{ R \psubstp{Q}{P} } \\
%   (\dropn{x})  \psubstp{Q}{P}       
%   := 
%   \left\{ 
%     \begin{array}{ccc} 
%       \dropn{\quotep{Q}} & & x \nameeq \quotep{P} \\
%       \dropn{x} & & otherwise \\
%     \end{array}
%   \right. 
  (\dropn{x})  \psubstp{Q}{P}       
  := 
  \left\{ 
    \begin{array}{ccc} 
      Q & & x \nameeq \quotep{P} \\
      \dropn{x} & & otherwise \\
    \end{array}
  \right.
\end{mathpar}
 

where

\begin{eqnarray}
  (x)\id{\{} \lpquote Q \rpquote / \lpquote P \rpquote \id{\}}            = 
  \left\{ 
    \begin{array}{ccc}
      \lpquote Q \rpquote & & x \nameeq \lpquote P \rpquote \\
      x & & otherwise \\
    \end{array}
  \right. \nonumber
\end{eqnarray}

and $z$ is chosen distinct from $\quotep{P}$, $\quotep{Q}$, the free
names in $Q$, and all the names in $R$. Our $\alpha$-equivalence will
be built in the standard way from this substitution.

\begin{remark}\label{rem:no_self_referential_names}
  One consequence of these definitions is that $\forall P. \quotep{P}
  \not\in \freenames{P}$.
\end{remark}

\subsection{ Dynamic quote: an example }

Anticipating something of what's to come, consider applying the
substitution, $\widehat{\id{\{}u / z \id{\}}}$, to the following pair
of processes, $\lift{w}{y!(z)}$ and $w[ \lpquote y!(z) \rpquote ]$.

\begin{eqnarray}
	\lift{w}{y!(z)}\widehat{\id{\{}u / z \id{\}}}
		& = &
		\lift{w}{y!(u)} \nonumber\\
	w[ \lpquote y!(z) \rpquote ] \widehat{ \id{\{}u / z \id{\}} }
		& = &
		w[ \lpquote y!(z) \rpquote ] \nonumber
\end{eqnarray}

Because the body of the process between quotes is impervious to
substitution, we get radically different answers. In fact, by
examining the first process in an input context,
e.g. $x?(z).\lift{w}{y!(z)}$, we see that the process under the lift
operator may be shaped by prefixed inputs binding a name inside it. In
this sense, the lift operator will be seen as a way to dynamically
construct processes before reifying them as names.

Finally equipped with these standard features we can present the
dynamics of the calculus.

\subsubsection{Operational semantics} 

Finally, we introduce the computational dynamics. What marks these
algebras as distinct from other more traditionally studied algebraic
structures, e.g. vector spaces or polynomial rings, is the manner in
which dynamics is captured. In traditional structures, dynamics is typically
expressed through morphisms between such structures, as in linear maps
between vector spaces or morphisms between rings. In algebras
associated with the semantics of computation, the dynamics is
expressed as part of the algebraic structure itself, through a
reduction reduction relation typically denoted by $\red$. Below, we
give a recursive presentation of this relation for the calculus used
in the encoding.

$\red \subseteq \pi \times \pi$
$\red : \pi \to \mathcal{P}(\pi)$

\begin{mathpar}
  \inferrule* [lab=Comm] { \textsf{match}( x_{src}, x_{trgt} ) } { x_{trgt}?(y)P \; | \; x_{src}!\langle {Q} \rangle \red P\{\quotep{Q}/y}\} }
  \and \\
  \inferrule* [lab=Par] {{P} \red {P}'} {{{P} | {Q}} \red {{P}' | {Q}}}
  \and
  \inferrule* [lab=Equiv]{{{P} \scong {P}'} \andalso {{P}' \red {Q}'} \andalso {{Q}' \scong {Q}}}{{P} \red {Q}}
\end{mathpar}

\begin{eqnarray*}
  match_{\equiv} (\quotep{P},\quotep{Q}) & := & P \equiv Q \\
  match_{\dagger}(\quotep{P},\quotep{Q}) & := & \forall R. P|Q \red^{*} R => R \red^{*} 0 \\
  match_{K}(\quotep{P},\quotep{Q}) & := & K \mbox{ for some context } K
\end{eqnarray*}

$u?(x)P | u!\langle Q \rangle \red P\{\quotep{Q}/x\}$

%We write $\wred$ for $\red^*$, and $P\red$ if $\exists Q $ such that $ P \red Q$.
We write $P\red$ if $\exists Q $ such that $ P \red Q$ and $P\not\red$, otherwise.

\section{Replication}

As mentioned before, it is known that replication (and hence
recursion) can be implemented in a higher-order process algebra
\cite{SangiorgiWalker}. As our first example of calculation with the
machinery thus far presented we give the construction explicitly in
the {\rhoc}.

\begin{eqnarray}
	D_{x} & := & \prefix{x}{y}{(\binpar{\outputp{x}{y}}{@{y}})} \nonumber\\
	\bangp_{x}{P} & := & \binpar{{x}!\langle{\binpar{D_{x}}{P}}\rangle}{D_{x}} \nonumber
\end{eqnarray}

\begin{eqnarray}
	\bangp_{x}{P} & & \nonumber\\
	=
	& {x}!\langle{(\prefix{x}{y}{(\outputp{x}{y} | @{y})) | P}}\rangle 
	      | \prefix{x}{y}{(\outputp{x}{y} | @{y})} & \nonumber\\
	\red
	& (\outputp{x}{y} | @{y})\substn{\quotep{(\prefix{x}{y}{(@{y} | \outputp{x}{y})) | P}}}{y} & \nonumber\\
	=
	& \outputp{x}{\quotep{(\prefix{x}{y}{(\outputp{x}{y} | @{y})) | P}}}
	  | {(\prefix{x}{y}{(\outputp{x}{y} | @{y})) | P}} & \nonumber\\
	\red
	& \ldots & \nonumber\\
	\red^*
	& P | P | \ldots & \nonumber
\end{eqnarray}

Of course, this encoding, as an implementation, runs away, unfolding
$\bangp{P}$ eagerly. A lazier and more implementable replication
operator, restricted to input-guarded processes, may be obtained as follows.

\begin{eqnarray}
\bangp{\prefix{u}{v}{P}} 
	:= 
	\binpar{\lift{x}{\prefix{u}{v}{(\binpar{D(x)}{P})}}}{D(x)} \nonumber
\end{eqnarray}

\begin{remark}
  Note that the lazier definition still does not deal with summation
  or mixed summation (i.e. sums over input and output). The reader is
  invited to construct definitions of replication that deal with these
  features. 

  Further, the definitions are parameterized in a name, $x$. Can you,
  gentle reader, make a definition that eliminates this parameter and
  guarantees no accidental interaction between the replication
  machinery and the process being replicated -- i.e. no accidental
  sharing of names used by the process to get its work done and the
  name(s) used by the replication to effect copying. This latter
  revision of the definition of replication is crucial to obtaining
  the expected identity $!!P \sim !P$.
\end{remark}

\begin{remark}\label{rem:paradoxical_combinator}
  The reader familiar with the lambda calculus will have noticed the
  similarity between $D$ and the paradoxical combinator.

  [Ed. note: the existence of this seems to suggest we have to be more
  restrictive on the set of processes and names we admit if we are to
  support no-cloning.]
\end{remark}

\subsubsection{Bisimulation}

The computational dynamics gives rise to another kind of equivalence,
the equivalence of computational behavior. As previously mentioned
this is typically captured \emph{via} some form of bisimulation.

% The notion we use in this paper is weak barbed bisimulation
% \cite{milner91polyadicpi}.

The notion we use in this paper is derived from weak barbed
bisimulation \cite{milner91polyadicpi}. 

\begin{definition}
An \emph{observation relation}, $\downarrow_{\mathcal N}$, over a set
of names, $\mathcal N$, is the smallest relation satisfying the rules
below.

\infrule[Out-barb]{y \in {\mathcal N}, \; x \nameeq y}
		  {\outputp{x}{v} \downarrow_{\mathcal N} x}
\infrule[Par-barb]{\mbox{$P\downarrow_{\mathcal N} x$ or $Q\downarrow_{\mathcal N} x$}}
		  {\binpar{P}{Q} \downarrow_{\mathcal N} x}

We write $P \Downarrow_{\mathcal N} x$ if there is $Q$ such that 
$P \wred Q$ and $Q \downarrow_{\mathcal N} x$.
\end{definition}

\begin{definition}
%\label{def.bbisim}
An  ${\mathcal N}$-\emph{barbed bisimulation} over a set of names, ${\mathcal N}$, is a symmetric binary relation 
${\mathcal S}_{\mathcal N}$ between agents such that $P\rel{S}_{\mathcal N}Q$ implies:
\begin{enumerate}
\item If $P \red P'$ then $Q \wred Q'$ and $P'\rel{S}_{\mathcal N} Q'$.
\item If $P\downarrow_{\mathcal N} x$, then $Q\Downarrow_{\mathcal N} x$.
\end{enumerate}
$P$ is ${\mathcal N}$-barbed bisimilar to $Q$, written
$P \wbbisim_{\mathcal N} Q$, if $P \rel{S}_{\mathcal N} Q$ for some ${\mathcal N}$-barbed bisimulation ${\mathcal S}_{\mathcal N}$.
\end{definition}

$\mathcal{R} \subseteq \pi \times \pi$

$P \mathcal{R} Q => \forall P'. P \red P' \Rightarrow \exists Q'. Q \red Q', P' \mathcal{R} Q'$

$P \vdash x \Rightarrow Q \vdash x$

\begin{mathpar}
  \inferrule*[lab=Out-barb]{x \nameeq y}{{y}!\langle{Q}\rangle \vdash x}
  \and
  \inferrule*[lab=Par-barb]{\mbox{$P\vdash x$ or $Q\vdash x$}}{\binpar{P}{Q} \vdash x}
\end{mathpar}

\subsubsection{Contexts}

One of the principle advantages of computational calculi like the
$\pi$-calculus is a well-defined notion of context,
contextual-equivalence and a correlation between
contextual-equivalence and notions of bisimulation. The notion of
context allows the decomposition of a process into (sub-)process and
its syntactic environment, its context. Thus, a context may be
thought of as a process with a ``hole'' (written $\Box$) in it. The
application of a context $M$ to a process $P$, written $M[P]$, is
tantamount to filling the hole in $M$ with $P$. In this paper we do
not need the full weight of this theory, but do make use of the notion
of context in the proof the main theorem. 

\begin{mathpar}
  \inferrule* [lab=summation] {} {{M_{M},M_{N}} \bc \Box \;|\; x.M_{A} \;|\; M_{M}+M_{N}}
  \and
  \inferrule* [lab=agent] {} {{M_{A}} \bc (\vec{x})M_{P} \;| \; \clift{P_0,\ldots,M_{P},\ldots,P_N}}
  \and \\
  \inferrule* [lab=process] {} {{M_{P}} \bc M_{N} \;| \;P|M_{P} }
\end{mathpar} 

\begin{mathpar}
  \inferrule* [lab=sychronization] {} {M_{N} \bc \Box \;|\; x?M_{F} \;|\; x!M_{C}}
  \and
  \inferrule* [lab=abstraction] {} {{M_{F}} \bc (x)M_{P} }
  \and
  \inferrule* [lab=concretion] {} {{M_{C}} \bc \langle M_{P} \rangle }
  \and \\
  \inferrule* [lab=process] {} {{M_{P}} \bc M_{N} \;| \;P|M_{P} }
\end{mathpar}

\begin{definition}[contextual application] Given a context $M$, and
  process $P$, we define the \emph{contextual application}, $M[P] :=
  M\{P/\Box\}$. That is, the contextual application of M to P is the
  substitution of $P$ for $\Box$ in $M$.
\end{definition}

$\meaningof{-} : L \to \mathcal{P}(\pi)$

\begin{mathpar}
  \inferrule* [lab=collection] {} {\meaningof{true} = \pi, \and \meaningof{~E} = \pi \setminus \meaningof{E}, \and \meaningof{E_{1} \& E_{2}} = \meaningof{E_{1}} \cap \meaningof{E_{2}}}
\end{mathpar}

\begin{mathpar}
  \inferrule* [lab=structure] {} {\meaningof{0} = \{ P \in \pi | P \equiv 0 \}, \and \\ \meaningof{E_1 | E_2} = \{ P \in \pi | P \equiv P_{1} | P_{2}, P_{1} \in \meaningof{E_{1}}, P_{2} \in \meaningof{E_2}\} }
\end{mathpar}

\begin{mathpar}
 \inferrule* [lab=behavior] {} {\meaningof{\langle a?b \rangle E} = \{ P \in \pi | P \equiv Q | u?(y)P', \\ \and \\\\ \and \\ \;\;\; u \in \meaningof{a}, \forall z.P'\{z/y\} \in \meaningof{E\{z/b\}}\}, \and \\ \meaningof{a!E} = \{ P \in \pi | P \equiv Q | x!\langle P' \rangle, x \in \meaningof{a} P' \in \meaningof{E}\} }
\end{mathpar}

\begin{mathpar}
 \inferrule* [lab=nominal] {} {\meaningof{\quotep{E}} = \{ \quotep{P} \in \quotep{\pi} | P \in \meaningof{E} \}, \and \meaningof{\quotep{P}} = \{ \quotep{Q} \in \quotep{\pi} | P \equiv Q \} \and \\ \meaningof{@\quotep{E}} = \{ P \in \pi | P \equiv @x, x \in \meaningof{E} \}}
\end{mathpar}

\begin{eqnarray*}
  \\
  \meaningof{-} : TS \to ST
\end{eqnarray*}

\begin{eqnarray*}
  \\
  L : TS \to ST
\end{eqnarray*}

\begin{eqnarray*}
  \\
  P \models E \iff P \in \meaningof{E}
\end{eqnarray*}

\begin{eqnarray*}
  P \approx_{L} Q \iff \forall E \in L. P \models E \iff Q \models E
\end{eqnarray*}

\begin{eqnarray*}
  P \approx_{K} Q
\end{eqnarray*}

\begin{eqnarray*}
  P \approx Q
\end{eqnarray*}

$\approx_{K} = \approx = \approx_{L}$

\subsubsection{Contextual duality}

Note that contexts extend the quotation operation to a family of
operations from processes to names. Given a context, $M$, we can
define a \emph{nominal context}, $\quotep{M}$ by $\quotep{M}[P] :=
\quotep{M[P]}$. To foreshadow what is to come we observe that these
operations enjoy a duality with processes very much like the duality
between vectors and maps from vectors to scalars.

Further, because the calculus is essentially higher-order, we have a
correspondence between contexts and processes. More specifically,
given a name $x$ and a context $M$ we can construct $M^{*}_{x}$ such
that 

\begin{mathpar}
  M^{*}_{x} | \lift{x}{P} \red M[P]
\end{mathpar}

namely,

\begin{mathpar}
  M^{*}_{x} := x?(u).M[\dropn{u}]
\end{mathpar}

The dependence of $M^{*}_{x}$ on a name makes it an abstraction, 

\begin{mathpar}
  M^{*} := (x)x?(u).M[\dropn{u}]
\end{mathpar}

\subsection{Additional notation}

It will sometimes be convenient to denote the process a name
quotes. We already have the notation $x = \quotep{P}$, but it will be
convenient to introduce an alternate notation, $\procn{x}$, when we
want to emphasize the connection to the use of the name. Note that, by
virtue of name equivalence, $\quotep{\procn{x}} \nameeq x$; so, the
notation is consistent with previous definitions.

Further, because names have structure it is possible to effect
substitutions on the basis of that structure. This means we need to
upgrade our notation for substitutions, which we accomplish by
adapting comprehension notation. Thus,

\begin{mathpar}
  P\{ y / x : x \in S \}
\end{mathpar}

is interpreted to mean the process derived from P by replacing (in a
capture-avoiding manner) each occurrence of $x$ in $S$ by $y$. For example,

\begin{mathpar}
  P\{ \quotep{\procn{x}|\procn{x}} / x : x \in \freenames{P} \}
\end{mathpar}

will replace each (occurrence) of a free name $x$ in $P$ by
$\quotep{\procn{x}|\procn{x}}$.

Also, we will avail ourselves of the notation $x^{L}$ and $x^{R}$ to
denote injections of a name into disjoint copies of the name
space. There are numerous ways to accomplish this. One example can be
found in \cite{MeredithR05}. This notation overloads to vectors of
names: $\vec{x}^{\pi} := (x_{i}^{\pi} \; : \; 0 \leq i < |\vec{x}| )$ where $\pi \in \{L,R\}$.

We also use $P^{\Box} := P|\Box$.

In \cite{MeredithR05} an interpretation of the new operator is
given. It turns out that there are several possible interpretations
all enjoying the requisite algebraic properties of the operator (see
\cite{milner91polyadicpi}). We will therefore make liberal use of
$(\nu\; \vec{x})P$.

% subsection the_syntax_and_semantics_of_the_notation_system (end)   

\input{qm2pi.qmops} 

\input{qm2pi.sterngerlach} 

\input{qm2pi.metric} 

% section concurrent_process_calculi (end)

%\input{qm2pi.proofsketch}

% section proof sketch (end)

%\input{qm2pi.slviaknots} 

% section spatial logic via knots (end)

\input{qm2pi.conclusion}

% section conclusion (end)

%\input{qm2pi.dtcodes} 

% section wiring algorithm (end)

\input{qm2pi.ack} 

% section acknowledgments (end)

\newpage


\bibliographystyle{plain}   
\bibliography{../../biblios/main.bib}

\input{qm2pi.rhodetails}

\end{document}

 

% section wiring algorithm (end)

\documentclass[12pt]{llncs}
%\documentclass{jktr}

\usepackage[pdftex]{hyperref}                   
\usepackage {listings}
\usepackage {mathpartir}
\usepackage{bcprules}
%\usepackage{listings}
                       
\usepackage{graphicx} 
%\usepackage[margins=2.5cm,nohead,nofoot]{geometry}
%\usepackage{geometry}
\usepackage{amsfonts}
\usepackage{amstext}
\usepackage{latexsym}
\usepackage{amssymb}
\usepackage{color}


%\include{myPreamble}
\include{qm2pi.local} 

%\ifpdf
%\usepackage[pdftex]{graphicx}
%\else
%\usepackage{graphicx}
%\fi

 % \ifpdf
%  \usepackage{pdfsync}
%  \if


%\title{Brief Article}
%\author{David F. Snyder}
%\author{L.G. Meredith}

%\address{Dept. of Math., Texas State University--San Marcos, San Marcos, TX 78666}
       
\pagestyle{empty}


\begin{document}

\lstset{language=[Objective]Caml,frame=shadowbox}

\input{qm2pi.front}

% section front matter (end)

\input{qm2pi.intro} 
 
% section introduction (end)

% \input{qm2pi.knotations} 

% section notation (end)

\input{qm2pi.process.calculi} 

% section concurrent_process_calculi_and_spatial_logics_ (end)
    
%\input{qm2pi.knots2pi} 

%\input{qm2pi.trefoil} 

%\input{qm2pi.mainthm} 

% subsection basic_interpretation (end)

%\input{qm2pi.rho.presentation} 
\subsection{The syntax and semantics of the notation system}\label{sub:the_syntax_and_semantics_of_the_notation_system} % (fold)

We now summarize a technical presentation of the calculus that
embodies our theory of dynamics. The typical presentation of such a
calculus follows the style of giving generators and relations on
them. The grammar, below, describing term constructors, freely
generates the set of processes, $\Proc$. This set is then quotiented
by a relation known as structural congruence and it is over this set
that the notion of dynamics is expressed. This presentation is
essentially that of \cite{MeredithR05} with the addition of
polyadicity and summation. For readability we have relegated some of
the technical subtleties to an appendix.

\subsubsection{Process grammar}\label{subsub:process_grammar}

\begin{mathpar}
  \inferrule* [lab=synchronization] {} {{M} \bc \pzero \;|\; x?F \;|\; x!C }
  \and
  \inferrule* [lab=abstraction] {} {{F} \bc (x)P}
  \and
  \inferrule* [lab=concretion] {} {{C} \bc \langle Q \rangle}
  \and
  \inferrule* [lab=process] {} {{P,Q} \bc M \;| \;P|Q \;|\; @{x}}
  \and
  \inferrule* [lab=name] {} {{x} \bc \quotep{P}}
\end{mathpar} 

Note that $\vec{x}$ (resp. $\vec{P}$) denotes a vector of names
(resp. processes) of length $|\vec{x}|$ (resp. $|\vec{P}|$). We adopt
the following useful abbreviations.

\begin{mathpar}
   x?(\vec{y}).P := x.(\vec{y})P \and  x\clift{\vec{P}} := x.\clift{\vec{P}}
   \and x!(y) := \lift{x}{\dropn{y}}
   \and \Pi_{i=0}^{n-1}P_i := P_0 | \ldots | P_{n-1}
\end{mathpar}

\subsubsection{Structural congruence}

\paragraph{Free and bound names and alpha-equivalence.} At the
core of structural equivalence is alpha-equivalence which identifies
process that are the same up to a change of variable. Formally, we
recognize the distinction between free and bound names. The free names
of a process, $\freenames{P}$, may be calculated recursively as
follows:

\begin{mathpar}
\freenames{\pzero} := \emptyset
  \and \\
  \freenames{x?(y).P} := \{ x \} \cup (\freenames{P} \setminus \{ y \})
  \and 
  \freenames{x!\langle P \rangle} := \{ x \} \cup \{ P \} 
  \and \\
  \freenames{P|Q} := \freenames{P} \cup \freenames{Q}
  \and \\
  \freenames{@{x}} := \{ x \}
\end{mathpar}

$\pi$
$\quotep{\pi}$

$\freenames{-} : \pi \to \mathcal{P}(\quotep{\pi})$

\begin{eqnarray*}
  \freenames{\pzero} & := & \emptyset \\
  \freenames{x?(y).P} & := & \{ x \} \cup (\freenames{P} \setminus \{ y \}) \\
  \freenames{x!\langle P \rangle} & := & \{ x \} \cup \{ P \} \\
  \freenames{P|Q} & := & \freenames{P} \cup \freenames{Q} \\
  \freenames{\dropn{x}} & := & \{ x \}
\end{eqnarray*}

The bound names of a process, $\boundnames{P}$, are those names occurring in $P$
that are not free. For example, in $x?(y).0$, the name $x$ is free, while $y$ is bound.

\begin{mathpar}
  \inferrule* [lab=monoidal-laws] {} { P|Q \equiv Q|P \and P|0 \equiv P \and P|(Q|R) \equiv (P|Q)|R }
\end{mathpar}

\begin{mathpar}
  \inferrule* [lab=alpha-equivalence] {} { (x)P \equiv (y)P\{y/x\} \and y \not\in \freenames{P} }
\end{mathpar}

\begin{definition}
Then two processes, $P,Q$, are alpha-equivalent if $P = Q\{\vec{y}/\vec{x}\}$ for
some $\vec{x} \in \boundnames{Q},\vec{y} \in \boundnames{P}$, where $Q\{\vec{y}/\vec{x}\}$
denotes the capture-avoiding substitution of $\vec{y}$ for $\vec{x}$ in $Q$.
\end{definition}

\begin{definition}
  The {\em structural congruence} \cite{SangiorgiWalker} , $\equiv$,
  between processes is the least congruence containing
  alpha-equivalence, satisfying the abelian monoid laws
  (associativity, commutativity and $\pzero$ as identity) for parallel
  composition $|$ and for summation $+$.
\end{definition}

\subsection{Name equivalence}

We take name equivalence, written $\nameeq$, to be the smallest
equivalence relation generated by the following rules.

\begin{mathpar}
\inferrule*[lab=Quote-drop]
{ }
{ \quotep{@{x}} \nameeq x }

\inferrule*[lab=Struct-equiv]
{ P \scong Q }
{ \quotep{P} \nameeq \quotep{Q} }
\end{mathpar}

The astute reader will have noticed that the mutual recursion of names
and processes imposes a mutual recursion on alpha-equivalence and
structural equivalence via name-equivalence. Fortunately, all of this
works out pleasantly and we may calculate in the natural way, free of
concern. The reader interested in the details is referred to the
appendix \ref{appendix:rho_details}.

\subsection{Substitution}

We use $\Proc$ for the set of processes, $\QProc$ for the set of
names, and $\id{\{}\vec{y} / \vec{x} \id{\}}$ to denote partial maps,
$s : \QProc \rightarrow \QProc$. A map, $s$ lifts, uniquely, to a map
on process terms, $\widehat{s} : \Proc \rightarrow \Proc$ by the
following equations.

\begin{mathpar}
  (0) \psubstp{Q}{P} := 0 \\
  (R \juxtap S) \psubstp{Q}{P}
  :=    
  (R)\psubstp{Q}{P} \juxtap (S) \psubstp{Q}{P} \\
  (x?(y).R) \psubstp{Q}{P}    
  :=    
  (x)\substp{Q}{P} (z)\concat( (R \psubstn{z}{y}) \psubstp{Q}{P} ) \\
  (\lift{x}{R}) \psubstp{Q}{P}  
  :=
  \lift{(x)\substp{Q}{P}}{ R \psubstp{Q}{P} } \\
%   (\dropn{x})  \psubstp{Q}{P}       
%   := 
%   \left\{ 
%     \begin{array}{ccc} 
%       \dropn{\quotep{Q}} & & x \nameeq \quotep{P} \\
%       \dropn{x} & & otherwise \\
%     \end{array}
%   \right. 
  (\dropn{x})  \psubstp{Q}{P}       
  := 
  \left\{ 
    \begin{array}{ccc} 
      Q & & x \nameeq \quotep{P} \\
      \dropn{x} & & otherwise \\
    \end{array}
  \right.
\end{mathpar}
 

where

\begin{eqnarray}
  (x)\id{\{} \lpquote Q \rpquote / \lpquote P \rpquote \id{\}}            = 
  \left\{ 
    \begin{array}{ccc}
      \lpquote Q \rpquote & & x \nameeq \lpquote P \rpquote \\
      x & & otherwise \\
    \end{array}
  \right. \nonumber
\end{eqnarray}

and $z$ is chosen distinct from $\quotep{P}$, $\quotep{Q}$, the free
names in $Q$, and all the names in $R$. Our $\alpha$-equivalence will
be built in the standard way from this substitution.

\begin{remark}\label{rem:no_self_referential_names}
  One consequence of these definitions is that $\forall P. \quotep{P}
  \not\in \freenames{P}$.
\end{remark}

\subsection{ Dynamic quote: an example }

Anticipating something of what's to come, consider applying the
substitution, $\widehat{\id{\{}u / z \id{\}}}$, to the following pair
of processes, $\lift{w}{y!(z)}$ and $w[ \lpquote y!(z) \rpquote ]$.

\begin{eqnarray}
	\lift{w}{y!(z)}\widehat{\id{\{}u / z \id{\}}}
		& = &
		\lift{w}{y!(u)} \nonumber\\
	w[ \lpquote y!(z) \rpquote ] \widehat{ \id{\{}u / z \id{\}} }
		& = &
		w[ \lpquote y!(z) \rpquote ] \nonumber
\end{eqnarray}

Because the body of the process between quotes is impervious to
substitution, we get radically different answers. In fact, by
examining the first process in an input context,
e.g. $x?(z).\lift{w}{y!(z)}$, we see that the process under the lift
operator may be shaped by prefixed inputs binding a name inside it. In
this sense, the lift operator will be seen as a way to dynamically
construct processes before reifying them as names.

Finally equipped with these standard features we can present the
dynamics of the calculus.

\subsubsection{Operational semantics} 

Finally, we introduce the computational dynamics. What marks these
algebras as distinct from other more traditionally studied algebraic
structures, e.g. vector spaces or polynomial rings, is the manner in
which dynamics is captured. In traditional structures, dynamics is typically
expressed through morphisms between such structures, as in linear maps
between vector spaces or morphisms between rings. In algebras
associated with the semantics of computation, the dynamics is
expressed as part of the algebraic structure itself, through a
reduction reduction relation typically denoted by $\red$. Below, we
give a recursive presentation of this relation for the calculus used
in the encoding.

$\red \subseteq \pi \times \pi$
$\red : \pi \to \mathcal{P}(\pi)$

\begin{mathpar}
  \inferrule* [lab=Comm] { \textsf{match}( x_{src}, x_{trgt} ) } { x_{trgt}?(y)P \; | \; x_{src}!\langle {Q} \rangle \red P\{\quotep{Q}/y}\} }
  \and \\
  \inferrule* [lab=Par] {{P} \red {P}'} {{{P} | {Q}} \red {{P}' | {Q}}}
  \and
  \inferrule* [lab=Equiv]{{{P} \scong {P}'} \andalso {{P}' \red {Q}'} \andalso {{Q}' \scong {Q}}}{{P} \red {Q}}
\end{mathpar}

\begin{eqnarray*}
  match_{\equiv} (\quotep{P},\quotep{Q}) & := & P \equiv Q \\
  match_{\dagger}(\quotep{P},\quotep{Q}) & := & \forall R. P|Q \red^{*} R => R \red^{*} 0 \\
  match_{K}(\quotep{P},\quotep{Q}) & := & K \mbox{ for some context } K
\end{eqnarray*}

$u?(x)P | u!\langle Q \rangle \red P\{\quotep{Q}/x\}$

%We write $\wred$ for $\red^*$, and $P\red$ if $\exists Q $ such that $ P \red Q$.
We write $P\red$ if $\exists Q $ such that $ P \red Q$ and $P\not\red$, otherwise.

\section{Replication}

As mentioned before, it is known that replication (and hence
recursion) can be implemented in a higher-order process algebra
\cite{SangiorgiWalker}. As our first example of calculation with the
machinery thus far presented we give the construction explicitly in
the {\rhoc}.

\begin{eqnarray}
	D_{x} & := & \prefix{x}{y}{(\binpar{\outputp{x}{y}}{@{y}})} \nonumber\\
	\bangp_{x}{P} & := & \binpar{{x}!\langle{\binpar{D_{x}}{P}}\rangle}{D_{x}} \nonumber
\end{eqnarray}

\begin{eqnarray}
	\bangp_{x}{P} & & \nonumber\\
	=
	& {x}!\langle{(\prefix{x}{y}{(\outputp{x}{y} | @{y})) | P}}\rangle 
	      | \prefix{x}{y}{(\outputp{x}{y} | @{y})} & \nonumber\\
	\red
	& (\outputp{x}{y} | @{y})\substn{\quotep{(\prefix{x}{y}{(@{y} | \outputp{x}{y})) | P}}}{y} & \nonumber\\
	=
	& \outputp{x}{\quotep{(\prefix{x}{y}{(\outputp{x}{y} | @{y})) | P}}}
	  | {(\prefix{x}{y}{(\outputp{x}{y} | @{y})) | P}} & \nonumber\\
	\red
	& \ldots & \nonumber\\
	\red^*
	& P | P | \ldots & \nonumber
\end{eqnarray}

Of course, this encoding, as an implementation, runs away, unfolding
$\bangp{P}$ eagerly. A lazier and more implementable replication
operator, restricted to input-guarded processes, may be obtained as follows.

\begin{eqnarray}
\bangp{\prefix{u}{v}{P}} 
	:= 
	\binpar{\lift{x}{\prefix{u}{v}{(\binpar{D(x)}{P})}}}{D(x)} \nonumber
\end{eqnarray}

\begin{remark}
  Note that the lazier definition still does not deal with summation
  or mixed summation (i.e. sums over input and output). The reader is
  invited to construct definitions of replication that deal with these
  features. 

  Further, the definitions are parameterized in a name, $x$. Can you,
  gentle reader, make a definition that eliminates this parameter and
  guarantees no accidental interaction between the replication
  machinery and the process being replicated -- i.e. no accidental
  sharing of names used by the process to get its work done and the
  name(s) used by the replication to effect copying. This latter
  revision of the definition of replication is crucial to obtaining
  the expected identity $!!P \sim !P$.
\end{remark}

\begin{remark}\label{rem:paradoxical_combinator}
  The reader familiar with the lambda calculus will have noticed the
  similarity between $D$ and the paradoxical combinator.

  [Ed. note: the existence of this seems to suggest we have to be more
  restrictive on the set of processes and names we admit if we are to
  support no-cloning.]
\end{remark}

\subsubsection{Bisimulation}

The computational dynamics gives rise to another kind of equivalence,
the equivalence of computational behavior. As previously mentioned
this is typically captured \emph{via} some form of bisimulation.

% The notion we use in this paper is weak barbed bisimulation
% \cite{milner91polyadicpi}.

The notion we use in this paper is derived from weak barbed
bisimulation \cite{milner91polyadicpi}. 

\begin{definition}
An \emph{observation relation}, $\downarrow_{\mathcal N}$, over a set
of names, $\mathcal N$, is the smallest relation satisfying the rules
below.

\infrule[Out-barb]{y \in {\mathcal N}, \; x \nameeq y}
		  {\outputp{x}{v} \downarrow_{\mathcal N} x}
\infrule[Par-barb]{\mbox{$P\downarrow_{\mathcal N} x$ or $Q\downarrow_{\mathcal N} x$}}
		  {\binpar{P}{Q} \downarrow_{\mathcal N} x}

We write $P \Downarrow_{\mathcal N} x$ if there is $Q$ such that 
$P \wred Q$ and $Q \downarrow_{\mathcal N} x$.
\end{definition}

\begin{definition}
%\label{def.bbisim}
An  ${\mathcal N}$-\emph{barbed bisimulation} over a set of names, ${\mathcal N}$, is a symmetric binary relation 
${\mathcal S}_{\mathcal N}$ between agents such that $P\rel{S}_{\mathcal N}Q$ implies:
\begin{enumerate}
\item If $P \red P'$ then $Q \wred Q'$ and $P'\rel{S}_{\mathcal N} Q'$.
\item If $P\downarrow_{\mathcal N} x$, then $Q\Downarrow_{\mathcal N} x$.
\end{enumerate}
$P$ is ${\mathcal N}$-barbed bisimilar to $Q$, written
$P \wbbisim_{\mathcal N} Q$, if $P \rel{S}_{\mathcal N} Q$ for some ${\mathcal N}$-barbed bisimulation ${\mathcal S}_{\mathcal N}$.
\end{definition}

$\mathcal{R} \subseteq \pi \times \pi$

$P \mathcal{R} Q => \forall P'. P \red P' \Rightarrow \exists Q'. Q \red Q', P' \mathcal{R} Q'$

$P \vdash x \Rightarrow Q \vdash x$

\begin{mathpar}
  \inferrule*[lab=Out-barb]{x \nameeq y}{{y}!\langle{Q}\rangle \vdash x}
  \and
  \inferrule*[lab=Par-barb]{\mbox{$P\vdash x$ or $Q\vdash x$}}{\binpar{P}{Q} \vdash x}
\end{mathpar}

\subsubsection{Contexts}

One of the principle advantages of computational calculi like the
$\pi$-calculus is a well-defined notion of context,
contextual-equivalence and a correlation between
contextual-equivalence and notions of bisimulation. The notion of
context allows the decomposition of a process into (sub-)process and
its syntactic environment, its context. Thus, a context may be
thought of as a process with a ``hole'' (written $\Box$) in it. The
application of a context $M$ to a process $P$, written $M[P]$, is
tantamount to filling the hole in $M$ with $P$. In this paper we do
not need the full weight of this theory, but do make use of the notion
of context in the proof the main theorem. 

\begin{mathpar}
  \inferrule* [lab=summation] {} {{M_{M},M_{N}} \bc \Box \;|\; x.M_{A} \;|\; M_{M}+M_{N}}
  \and
  \inferrule* [lab=agent] {} {{M_{A}} \bc (\vec{x})M_{P} \;| \; \clift{P_0,\ldots,M_{P},\ldots,P_N}}
  \and \\
  \inferrule* [lab=process] {} {{M_{P}} \bc M_{N} \;| \;P|M_{P} }
\end{mathpar} 

\begin{mathpar}
  \inferrule* [lab=sychronization] {} {M_{N} \bc \Box \;|\; x?M_{F} \;|\; x!M_{C}}
  \and
  \inferrule* [lab=abstraction] {} {{M_{F}} \bc (x)M_{P} }
  \and
  \inferrule* [lab=concretion] {} {{M_{C}} \bc \langle M_{P} \rangle }
  \and \\
  \inferrule* [lab=process] {} {{M_{P}} \bc M_{N} \;| \;P|M_{P} }
\end{mathpar}

\begin{definition}[contextual application] Given a context $M$, and
  process $P$, we define the \emph{contextual application}, $M[P] :=
  M\{P/\Box\}$. That is, the contextual application of M to P is the
  substitution of $P$ for $\Box$ in $M$.
\end{definition}

$\meaningof{-} : L \to \mathcal{P}(\pi)$

\begin{mathpar}
  \inferrule* [lab=collection] {} {\meaningof{true} = \pi, \and \meaningof{~E} = \pi \setminus \meaningof{E}, \and \meaningof{E_{1} \& E_{2}} = \meaningof{E_{1}} \cap \meaningof{E_{2}}}
\end{mathpar}

\begin{mathpar}
  \inferrule* [lab=structure] {} {\meaningof{0} = \{ P \in \pi | P \equiv 0 \}, \and \\ \meaningof{E_1 | E_2} = \{ P \in \pi | P \equiv P_{1} | P_{2}, P_{1} \in \meaningof{E_{1}}, P_{2} \in \meaningof{E_2}\} }
\end{mathpar}

\begin{mathpar}
 \inferrule* [lab=behavior] {} {\meaningof{\langle a?b \rangle E} = \{ P \in \pi | P \equiv Q | u?(y)P', \\ \and \\\\ \and \\ \;\;\; u \in \meaningof{a}, \forall z.P'\{z/y\} \in \meaningof{E\{z/b\}}\}, \and \\ \meaningof{a!E} = \{ P \in \pi | P \equiv Q | x!\langle P' \rangle, x \in \meaningof{a} P' \in \meaningof{E}\} }
\end{mathpar}

\begin{mathpar}
 \inferrule* [lab=nominal] {} {\meaningof{\quotep{E}} = \{ \quotep{P} \in \quotep{\pi} | P \in \meaningof{E} \}, \and \meaningof{\quotep{P}} = \{ \quotep{Q} \in \quotep{\pi} | P \equiv Q \} \and \\ \meaningof{@\quotep{E}} = \{ P \in \pi | P \equiv @x, x \in \meaningof{E} \}}
\end{mathpar}

\begin{eqnarray*}
  \\
  \meaningof{-} : TS \to ST
\end{eqnarray*}

\begin{eqnarray*}
  \\
  L : TS \to ST
\end{eqnarray*}

\begin{eqnarray*}
  \\
  P \models E \iff P \in \meaningof{E}
\end{eqnarray*}

\begin{eqnarray*}
  P \approx_{L} Q \iff \forall E \in L. P \models E \iff Q \models E
\end{eqnarray*}

\begin{eqnarray*}
  P \approx_{K} Q
\end{eqnarray*}

\begin{eqnarray*}
  P \approx Q
\end{eqnarray*}

$\approx_{K} = \approx = \approx_{L}$

\subsubsection{Contextual duality}

Note that contexts extend the quotation operation to a family of
operations from processes to names. Given a context, $M$, we can
define a \emph{nominal context}, $\quotep{M}$ by $\quotep{M}[P] :=
\quotep{M[P]}$. To foreshadow what is to come we observe that these
operations enjoy a duality with processes very much like the duality
between vectors and maps from vectors to scalars.

Further, because the calculus is essentially higher-order, we have a
correspondence between contexts and processes. More specifically,
given a name $x$ and a context $M$ we can construct $M^{*}_{x}$ such
that 

\begin{mathpar}
  M^{*}_{x} | \lift{x}{P} \red M[P]
\end{mathpar}

namely,

\begin{mathpar}
  M^{*}_{x} := x?(u).M[\dropn{u}]
\end{mathpar}

The dependence of $M^{*}_{x}$ on a name makes it an abstraction, 

\begin{mathpar}
  M^{*} := (x)x?(u).M[\dropn{u}]
\end{mathpar}

\subsection{Additional notation}

It will sometimes be convenient to denote the process a name
quotes. We already have the notation $x = \quotep{P}$, but it will be
convenient to introduce an alternate notation, $\procn{x}$, when we
want to emphasize the connection to the use of the name. Note that, by
virtue of name equivalence, $\quotep{\procn{x}} \nameeq x$; so, the
notation is consistent with previous definitions.

Further, because names have structure it is possible to effect
substitutions on the basis of that structure. This means we need to
upgrade our notation for substitutions, which we accomplish by
adapting comprehension notation. Thus,

\begin{mathpar}
  P\{ y / x : x \in S \}
\end{mathpar}

is interpreted to mean the process derived from P by replacing (in a
capture-avoiding manner) each occurrence of $x$ in $S$ by $y$. For example,

\begin{mathpar}
  P\{ \quotep{\procn{x}|\procn{x}} / x : x \in \freenames{P} \}
\end{mathpar}

will replace each (occurrence) of a free name $x$ in $P$ by
$\quotep{\procn{x}|\procn{x}}$.

Also, we will avail ourselves of the notation $x^{L}$ and $x^{R}$ to
denote injections of a name into disjoint copies of the name
space. There are numerous ways to accomplish this. One example can be
found in \cite{MeredithR05}. This notation overloads to vectors of
names: $\vec{x}^{\pi} := (x_{i}^{\pi} \; : \; 0 \leq i < |\vec{x}| )$ where $\pi \in \{L,R\}$.

We also use $P^{\Box} := P|\Box$.

In \cite{MeredithR05} an interpretation of the new operator is
given. It turns out that there are several possible interpretations
all enjoying the requisite algebraic properties of the operator (see
\cite{milner91polyadicpi}). We will therefore make liberal use of
$(\nu\; \vec{x})P$.

% subsection the_syntax_and_semantics_of_the_notation_system (end)   

\input{qm2pi.qmops} 

\input{qm2pi.sterngerlach} 

\input{qm2pi.metric} 

% section concurrent_process_calculi (end)

%\input{qm2pi.proofsketch}

% section proof sketch (end)

%\input{qm2pi.slviaknots} 

% section spatial logic via knots (end)

\input{qm2pi.conclusion}

% section conclusion (end)

%\input{qm2pi.dtcodes} 

% section wiring algorithm (end)

\input{qm2pi.ack} 

% section acknowledgments (end)

\newpage


\bibliographystyle{plain}   
\bibliography{../../biblios/main.bib}

\input{qm2pi.rhodetails}

\end{document}

 

% section acknowledgments (end)

\newpage


\bibliographystyle{plain}   
\bibliography{../../biblios/main.bib}

\documentclass[12pt]{llncs}
%\documentclass{jktr}

\usepackage[pdftex]{hyperref}                   
\usepackage {listings}
\usepackage {mathpartir}
\usepackage{bcprules}
%\usepackage{listings}
                       
\usepackage{graphicx} 
%\usepackage[margins=2.5cm,nohead,nofoot]{geometry}
%\usepackage{geometry}
\usepackage{amsfonts}
\usepackage{amstext}
\usepackage{latexsym}
\usepackage{amssymb}
\usepackage{color}


%\include{myPreamble}
\include{qm2pi.local} 

%\ifpdf
%\usepackage[pdftex]{graphicx}
%\else
%\usepackage{graphicx}
%\fi

 % \ifpdf
%  \usepackage{pdfsync}
%  \if


%\title{Brief Article}
%\author{David F. Snyder}
%\author{L.G. Meredith}

%\address{Dept. of Math., Texas State University--San Marcos, San Marcos, TX 78666}
       
\pagestyle{empty}


\begin{document}

\lstset{language=[Objective]Caml,frame=shadowbox}

\input{qm2pi.front}

% section front matter (end)

\input{qm2pi.intro} 
 
% section introduction (end)

% \input{qm2pi.knotations} 

% section notation (end)

\input{qm2pi.process.calculi} 

% section concurrent_process_calculi_and_spatial_logics_ (end)
    
%\input{qm2pi.knots2pi} 

%\input{qm2pi.trefoil} 

%\input{qm2pi.mainthm} 

% subsection basic_interpretation (end)

%\input{qm2pi.rho.presentation} 
\subsection{The syntax and semantics of the notation system}\label{sub:the_syntax_and_semantics_of_the_notation_system} % (fold)

We now summarize a technical presentation of the calculus that
embodies our theory of dynamics. The typical presentation of such a
calculus follows the style of giving generators and relations on
them. The grammar, below, describing term constructors, freely
generates the set of processes, $\Proc$. This set is then quotiented
by a relation known as structural congruence and it is over this set
that the notion of dynamics is expressed. This presentation is
essentially that of \cite{MeredithR05} with the addition of
polyadicity and summation. For readability we have relegated some of
the technical subtleties to an appendix.

\subsubsection{Process grammar}\label{subsub:process_grammar}

\begin{mathpar}
  \inferrule* [lab=synchronization] {} {{M} \bc \pzero \;|\; x?F \;|\; x!C }
  \and
  \inferrule* [lab=abstraction] {} {{F} \bc (x)P}
  \and
  \inferrule* [lab=concretion] {} {{C} \bc \langle Q \rangle}
  \and
  \inferrule* [lab=process] {} {{P,Q} \bc M \;| \;P|Q \;|\; @{x}}
  \and
  \inferrule* [lab=name] {} {{x} \bc \quotep{P}}
\end{mathpar} 

Note that $\vec{x}$ (resp. $\vec{P}$) denotes a vector of names
(resp. processes) of length $|\vec{x}|$ (resp. $|\vec{P}|$). We adopt
the following useful abbreviations.

\begin{mathpar}
   x?(\vec{y}).P := x.(\vec{y})P \and  x\clift{\vec{P}} := x.\clift{\vec{P}}
   \and x!(y) := \lift{x}{\dropn{y}}
   \and \Pi_{i=0}^{n-1}P_i := P_0 | \ldots | P_{n-1}
\end{mathpar}

\subsubsection{Structural congruence}

\paragraph{Free and bound names and alpha-equivalence.} At the
core of structural equivalence is alpha-equivalence which identifies
process that are the same up to a change of variable. Formally, we
recognize the distinction between free and bound names. The free names
of a process, $\freenames{P}$, may be calculated recursively as
follows:

\begin{mathpar}
\freenames{\pzero} := \emptyset
  \and \\
  \freenames{x?(y).P} := \{ x \} \cup (\freenames{P} \setminus \{ y \})
  \and 
  \freenames{x!\langle P \rangle} := \{ x \} \cup \{ P \} 
  \and \\
  \freenames{P|Q} := \freenames{P} \cup \freenames{Q}
  \and \\
  \freenames{@{x}} := \{ x \}
\end{mathpar}

$\pi$
$\quotep{\pi}$

$\freenames{-} : \pi \to \mathcal{P}(\quotep{\pi})$

\begin{eqnarray*}
  \freenames{\pzero} & := & \emptyset \\
  \freenames{x?(y).P} & := & \{ x \} \cup (\freenames{P} \setminus \{ y \}) \\
  \freenames{x!\langle P \rangle} & := & \{ x \} \cup \{ P \} \\
  \freenames{P|Q} & := & \freenames{P} \cup \freenames{Q} \\
  \freenames{\dropn{x}} & := & \{ x \}
\end{eqnarray*}

The bound names of a process, $\boundnames{P}$, are those names occurring in $P$
that are not free. For example, in $x?(y).0$, the name $x$ is free, while $y$ is bound.

\begin{mathpar}
  \inferrule* [lab=monoidal-laws] {} { P|Q \equiv Q|P \and P|0 \equiv P \and P|(Q|R) \equiv (P|Q)|R }
\end{mathpar}

\begin{mathpar}
  \inferrule* [lab=alpha-equivalence] {} { (x)P \equiv (y)P\{y/x\} \and y \not\in \freenames{P} }
\end{mathpar}

\begin{definition}
Then two processes, $P,Q$, are alpha-equivalent if $P = Q\{\vec{y}/\vec{x}\}$ for
some $\vec{x} \in \boundnames{Q},\vec{y} \in \boundnames{P}$, where $Q\{\vec{y}/\vec{x}\}$
denotes the capture-avoiding substitution of $\vec{y}$ for $\vec{x}$ in $Q$.
\end{definition}

\begin{definition}
  The {\em structural congruence} \cite{SangiorgiWalker} , $\equiv$,
  between processes is the least congruence containing
  alpha-equivalence, satisfying the abelian monoid laws
  (associativity, commutativity and $\pzero$ as identity) for parallel
  composition $|$ and for summation $+$.
\end{definition}

\subsection{Name equivalence}

We take name equivalence, written $\nameeq$, to be the smallest
equivalence relation generated by the following rules.

\begin{mathpar}
\inferrule*[lab=Quote-drop]
{ }
{ \quotep{@{x}} \nameeq x }

\inferrule*[lab=Struct-equiv]
{ P \scong Q }
{ \quotep{P} \nameeq \quotep{Q} }
\end{mathpar}

The astute reader will have noticed that the mutual recursion of names
and processes imposes a mutual recursion on alpha-equivalence and
structural equivalence via name-equivalence. Fortunately, all of this
works out pleasantly and we may calculate in the natural way, free of
concern. The reader interested in the details is referred to the
appendix \ref{appendix:rho_details}.

\subsection{Substitution}

We use $\Proc$ for the set of processes, $\QProc$ for the set of
names, and $\id{\{}\vec{y} / \vec{x} \id{\}}$ to denote partial maps,
$s : \QProc \rightarrow \QProc$. A map, $s$ lifts, uniquely, to a map
on process terms, $\widehat{s} : \Proc \rightarrow \Proc$ by the
following equations.

\begin{mathpar}
  (0) \psubstp{Q}{P} := 0 \\
  (R \juxtap S) \psubstp{Q}{P}
  :=    
  (R)\psubstp{Q}{P} \juxtap (S) \psubstp{Q}{P} \\
  (x?(y).R) \psubstp{Q}{P}    
  :=    
  (x)\substp{Q}{P} (z)\concat( (R \psubstn{z}{y}) \psubstp{Q}{P} ) \\
  (\lift{x}{R}) \psubstp{Q}{P}  
  :=
  \lift{(x)\substp{Q}{P}}{ R \psubstp{Q}{P} } \\
%   (\dropn{x})  \psubstp{Q}{P}       
%   := 
%   \left\{ 
%     \begin{array}{ccc} 
%       \dropn{\quotep{Q}} & & x \nameeq \quotep{P} \\
%       \dropn{x} & & otherwise \\
%     \end{array}
%   \right. 
  (\dropn{x})  \psubstp{Q}{P}       
  := 
  \left\{ 
    \begin{array}{ccc} 
      Q & & x \nameeq \quotep{P} \\
      \dropn{x} & & otherwise \\
    \end{array}
  \right.
\end{mathpar}
 

where

\begin{eqnarray}
  (x)\id{\{} \lpquote Q \rpquote / \lpquote P \rpquote \id{\}}            = 
  \left\{ 
    \begin{array}{ccc}
      \lpquote Q \rpquote & & x \nameeq \lpquote P \rpquote \\
      x & & otherwise \\
    \end{array}
  \right. \nonumber
\end{eqnarray}

and $z$ is chosen distinct from $\quotep{P}$, $\quotep{Q}$, the free
names in $Q$, and all the names in $R$. Our $\alpha$-equivalence will
be built in the standard way from this substitution.

\begin{remark}\label{rem:no_self_referential_names}
  One consequence of these definitions is that $\forall P. \quotep{P}
  \not\in \freenames{P}$.
\end{remark}

\subsection{ Dynamic quote: an example }

Anticipating something of what's to come, consider applying the
substitution, $\widehat{\id{\{}u / z \id{\}}}$, to the following pair
of processes, $\lift{w}{y!(z)}$ and $w[ \lpquote y!(z) \rpquote ]$.

\begin{eqnarray}
	\lift{w}{y!(z)}\widehat{\id{\{}u / z \id{\}}}
		& = &
		\lift{w}{y!(u)} \nonumber\\
	w[ \lpquote y!(z) \rpquote ] \widehat{ \id{\{}u / z \id{\}} }
		& = &
		w[ \lpquote y!(z) \rpquote ] \nonumber
\end{eqnarray}

Because the body of the process between quotes is impervious to
substitution, we get radically different answers. In fact, by
examining the first process in an input context,
e.g. $x?(z).\lift{w}{y!(z)}$, we see that the process under the lift
operator may be shaped by prefixed inputs binding a name inside it. In
this sense, the lift operator will be seen as a way to dynamically
construct processes before reifying them as names.

Finally equipped with these standard features we can present the
dynamics of the calculus.

\subsubsection{Operational semantics} 

Finally, we introduce the computational dynamics. What marks these
algebras as distinct from other more traditionally studied algebraic
structures, e.g. vector spaces or polynomial rings, is the manner in
which dynamics is captured. In traditional structures, dynamics is typically
expressed through morphisms between such structures, as in linear maps
between vector spaces or morphisms between rings. In algebras
associated with the semantics of computation, the dynamics is
expressed as part of the algebraic structure itself, through a
reduction reduction relation typically denoted by $\red$. Below, we
give a recursive presentation of this relation for the calculus used
in the encoding.

$\red \subseteq \pi \times \pi$
$\red : \pi \to \mathcal{P}(\pi)$

\begin{mathpar}
  \inferrule* [lab=Comm] { \textsf{match}( x_{src}, x_{trgt} ) } { x_{trgt}?(y)P \; | \; x_{src}!\langle {Q} \rangle \red P\{\quotep{Q}/y}\} }
  \and \\
  \inferrule* [lab=Par] {{P} \red {P}'} {{{P} | {Q}} \red {{P}' | {Q}}}
  \and
  \inferrule* [lab=Equiv]{{{P} \scong {P}'} \andalso {{P}' \red {Q}'} \andalso {{Q}' \scong {Q}}}{{P} \red {Q}}
\end{mathpar}

\begin{eqnarray*}
  match_{\equiv} (\quotep{P},\quotep{Q}) & := & P \equiv Q \\
  match_{\dagger}(\quotep{P},\quotep{Q}) & := & \forall R. P|Q \red^{*} R => R \red^{*} 0 \\
  match_{K}(\quotep{P},\quotep{Q}) & := & K \mbox{ for some context } K
\end{eqnarray*}

$u?(x)P | u!\langle Q \rangle \red P\{\quotep{Q}/x\}$

%We write $\wred$ for $\red^*$, and $P\red$ if $\exists Q $ such that $ P \red Q$.
We write $P\red$ if $\exists Q $ such that $ P \red Q$ and $P\not\red$, otherwise.

\section{Replication}

As mentioned before, it is known that replication (and hence
recursion) can be implemented in a higher-order process algebra
\cite{SangiorgiWalker}. As our first example of calculation with the
machinery thus far presented we give the construction explicitly in
the {\rhoc}.

\begin{eqnarray}
	D_{x} & := & \prefix{x}{y}{(\binpar{\outputp{x}{y}}{@{y}})} \nonumber\\
	\bangp_{x}{P} & := & \binpar{{x}!\langle{\binpar{D_{x}}{P}}\rangle}{D_{x}} \nonumber
\end{eqnarray}

\begin{eqnarray}
	\bangp_{x}{P} & & \nonumber\\
	=
	& {x}!\langle{(\prefix{x}{y}{(\outputp{x}{y} | @{y})) | P}}\rangle 
	      | \prefix{x}{y}{(\outputp{x}{y} | @{y})} & \nonumber\\
	\red
	& (\outputp{x}{y} | @{y})\substn{\quotep{(\prefix{x}{y}{(@{y} | \outputp{x}{y})) | P}}}{y} & \nonumber\\
	=
	& \outputp{x}{\quotep{(\prefix{x}{y}{(\outputp{x}{y} | @{y})) | P}}}
	  | {(\prefix{x}{y}{(\outputp{x}{y} | @{y})) | P}} & \nonumber\\
	\red
	& \ldots & \nonumber\\
	\red^*
	& P | P | \ldots & \nonumber
\end{eqnarray}

Of course, this encoding, as an implementation, runs away, unfolding
$\bangp{P}$ eagerly. A lazier and more implementable replication
operator, restricted to input-guarded processes, may be obtained as follows.

\begin{eqnarray}
\bangp{\prefix{u}{v}{P}} 
	:= 
	\binpar{\lift{x}{\prefix{u}{v}{(\binpar{D(x)}{P})}}}{D(x)} \nonumber
\end{eqnarray}

\begin{remark}
  Note that the lazier definition still does not deal with summation
  or mixed summation (i.e. sums over input and output). The reader is
  invited to construct definitions of replication that deal with these
  features. 

  Further, the definitions are parameterized in a name, $x$. Can you,
  gentle reader, make a definition that eliminates this parameter and
  guarantees no accidental interaction between the replication
  machinery and the process being replicated -- i.e. no accidental
  sharing of names used by the process to get its work done and the
  name(s) used by the replication to effect copying. This latter
  revision of the definition of replication is crucial to obtaining
  the expected identity $!!P \sim !P$.
\end{remark}

\begin{remark}\label{rem:paradoxical_combinator}
  The reader familiar with the lambda calculus will have noticed the
  similarity between $D$ and the paradoxical combinator.

  [Ed. note: the existence of this seems to suggest we have to be more
  restrictive on the set of processes and names we admit if we are to
  support no-cloning.]
\end{remark}

\subsubsection{Bisimulation}

The computational dynamics gives rise to another kind of equivalence,
the equivalence of computational behavior. As previously mentioned
this is typically captured \emph{via} some form of bisimulation.

% The notion we use in this paper is weak barbed bisimulation
% \cite{milner91polyadicpi}.

The notion we use in this paper is derived from weak barbed
bisimulation \cite{milner91polyadicpi}. 

\begin{definition}
An \emph{observation relation}, $\downarrow_{\mathcal N}$, over a set
of names, $\mathcal N$, is the smallest relation satisfying the rules
below.

\infrule[Out-barb]{y \in {\mathcal N}, \; x \nameeq y}
		  {\outputp{x}{v} \downarrow_{\mathcal N} x}
\infrule[Par-barb]{\mbox{$P\downarrow_{\mathcal N} x$ or $Q\downarrow_{\mathcal N} x$}}
		  {\binpar{P}{Q} \downarrow_{\mathcal N} x}

We write $P \Downarrow_{\mathcal N} x$ if there is $Q$ such that 
$P \wred Q$ and $Q \downarrow_{\mathcal N} x$.
\end{definition}

\begin{definition}
%\label{def.bbisim}
An  ${\mathcal N}$-\emph{barbed bisimulation} over a set of names, ${\mathcal N}$, is a symmetric binary relation 
${\mathcal S}_{\mathcal N}$ between agents such that $P\rel{S}_{\mathcal N}Q$ implies:
\begin{enumerate}
\item If $P \red P'$ then $Q \wred Q'$ and $P'\rel{S}_{\mathcal N} Q'$.
\item If $P\downarrow_{\mathcal N} x$, then $Q\Downarrow_{\mathcal N} x$.
\end{enumerate}
$P$ is ${\mathcal N}$-barbed bisimilar to $Q$, written
$P \wbbisim_{\mathcal N} Q$, if $P \rel{S}_{\mathcal N} Q$ for some ${\mathcal N}$-barbed bisimulation ${\mathcal S}_{\mathcal N}$.
\end{definition}

$\mathcal{R} \subseteq \pi \times \pi$

$P \mathcal{R} Q => \forall P'. P \red P' \Rightarrow \exists Q'. Q \red Q', P' \mathcal{R} Q'$

$P \vdash x \Rightarrow Q \vdash x$

\begin{mathpar}
  \inferrule*[lab=Out-barb]{x \nameeq y}{{y}!\langle{Q}\rangle \vdash x}
  \and
  \inferrule*[lab=Par-barb]{\mbox{$P\vdash x$ or $Q\vdash x$}}{\binpar{P}{Q} \vdash x}
\end{mathpar}

\subsubsection{Contexts}

One of the principle advantages of computational calculi like the
$\pi$-calculus is a well-defined notion of context,
contextual-equivalence and a correlation between
contextual-equivalence and notions of bisimulation. The notion of
context allows the decomposition of a process into (sub-)process and
its syntactic environment, its context. Thus, a context may be
thought of as a process with a ``hole'' (written $\Box$) in it. The
application of a context $M$ to a process $P$, written $M[P]$, is
tantamount to filling the hole in $M$ with $P$. In this paper we do
not need the full weight of this theory, but do make use of the notion
of context in the proof the main theorem. 

\begin{mathpar}
  \inferrule* [lab=summation] {} {{M_{M},M_{N}} \bc \Box \;|\; x.M_{A} \;|\; M_{M}+M_{N}}
  \and
  \inferrule* [lab=agent] {} {{M_{A}} \bc (\vec{x})M_{P} \;| \; \clift{P_0,\ldots,M_{P},\ldots,P_N}}
  \and \\
  \inferrule* [lab=process] {} {{M_{P}} \bc M_{N} \;| \;P|M_{P} }
\end{mathpar} 

\begin{mathpar}
  \inferrule* [lab=sychronization] {} {M_{N} \bc \Box \;|\; x?M_{F} \;|\; x!M_{C}}
  \and
  \inferrule* [lab=abstraction] {} {{M_{F}} \bc (x)M_{P} }
  \and
  \inferrule* [lab=concretion] {} {{M_{C}} \bc \langle M_{P} \rangle }
  \and \\
  \inferrule* [lab=process] {} {{M_{P}} \bc M_{N} \;| \;P|M_{P} }
\end{mathpar}

\begin{definition}[contextual application] Given a context $M$, and
  process $P$, we define the \emph{contextual application}, $M[P] :=
  M\{P/\Box\}$. That is, the contextual application of M to P is the
  substitution of $P$ for $\Box$ in $M$.
\end{definition}

$\meaningof{-} : L \to \mathcal{P}(\pi)$

\begin{mathpar}
  \inferrule* [lab=collection] {} {\meaningof{true} = \pi, \and \meaningof{~E} = \pi \setminus \meaningof{E}, \and \meaningof{E_{1} \& E_{2}} = \meaningof{E_{1}} \cap \meaningof{E_{2}}}
\end{mathpar}

\begin{mathpar}
  \inferrule* [lab=structure] {} {\meaningof{0} = \{ P \in \pi | P \equiv 0 \}, \and \\ \meaningof{E_1 | E_2} = \{ P \in \pi | P \equiv P_{1} | P_{2}, P_{1} \in \meaningof{E_{1}}, P_{2} \in \meaningof{E_2}\} }
\end{mathpar}

\begin{mathpar}
 \inferrule* [lab=behavior] {} {\meaningof{\langle a?b \rangle E} = \{ P \in \pi | P \equiv Q | u?(y)P', \\ \and \\\\ \and \\ \;\;\; u \in \meaningof{a}, \forall z.P'\{z/y\} \in \meaningof{E\{z/b\}}\}, \and \\ \meaningof{a!E} = \{ P \in \pi | P \equiv Q | x!\langle P' \rangle, x \in \meaningof{a} P' \in \meaningof{E}\} }
\end{mathpar}

\begin{mathpar}
 \inferrule* [lab=nominal] {} {\meaningof{\quotep{E}} = \{ \quotep{P} \in \quotep{\pi} | P \in \meaningof{E} \}, \and \meaningof{\quotep{P}} = \{ \quotep{Q} \in \quotep{\pi} | P \equiv Q \} \and \\ \meaningof{@\quotep{E}} = \{ P \in \pi | P \equiv @x, x \in \meaningof{E} \}}
\end{mathpar}

\begin{eqnarray*}
  \\
  \meaningof{-} : TS \to ST
\end{eqnarray*}

\begin{eqnarray*}
  \\
  L : TS \to ST
\end{eqnarray*}

\begin{eqnarray*}
  \\
  P \models E \iff P \in \meaningof{E}
\end{eqnarray*}

\begin{eqnarray*}
  P \approx_{L} Q \iff \forall E \in L. P \models E \iff Q \models E
\end{eqnarray*}

\begin{eqnarray*}
  P \approx_{K} Q
\end{eqnarray*}

\begin{eqnarray*}
  P \approx Q
\end{eqnarray*}

$\approx_{K} = \approx = \approx_{L}$

\subsubsection{Contextual duality}

Note that contexts extend the quotation operation to a family of
operations from processes to names. Given a context, $M$, we can
define a \emph{nominal context}, $\quotep{M}$ by $\quotep{M}[P] :=
\quotep{M[P]}$. To foreshadow what is to come we observe that these
operations enjoy a duality with processes very much like the duality
between vectors and maps from vectors to scalars.

Further, because the calculus is essentially higher-order, we have a
correspondence between contexts and processes. More specifically,
given a name $x$ and a context $M$ we can construct $M^{*}_{x}$ such
that 

\begin{mathpar}
  M^{*}_{x} | \lift{x}{P} \red M[P]
\end{mathpar}

namely,

\begin{mathpar}
  M^{*}_{x} := x?(u).M[\dropn{u}]
\end{mathpar}

The dependence of $M^{*}_{x}$ on a name makes it an abstraction, 

\begin{mathpar}
  M^{*} := (x)x?(u).M[\dropn{u}]
\end{mathpar}

\subsection{Additional notation}

It will sometimes be convenient to denote the process a name
quotes. We already have the notation $x = \quotep{P}$, but it will be
convenient to introduce an alternate notation, $\procn{x}$, when we
want to emphasize the connection to the use of the name. Note that, by
virtue of name equivalence, $\quotep{\procn{x}} \nameeq x$; so, the
notation is consistent with previous definitions.

Further, because names have structure it is possible to effect
substitutions on the basis of that structure. This means we need to
upgrade our notation for substitutions, which we accomplish by
adapting comprehension notation. Thus,

\begin{mathpar}
  P\{ y / x : x \in S \}
\end{mathpar}

is interpreted to mean the process derived from P by replacing (in a
capture-avoiding manner) each occurrence of $x$ in $S$ by $y$. For example,

\begin{mathpar}
  P\{ \quotep{\procn{x}|\procn{x}} / x : x \in \freenames{P} \}
\end{mathpar}

will replace each (occurrence) of a free name $x$ in $P$ by
$\quotep{\procn{x}|\procn{x}}$.

Also, we will avail ourselves of the notation $x^{L}$ and $x^{R}$ to
denote injections of a name into disjoint copies of the name
space. There are numerous ways to accomplish this. One example can be
found in \cite{MeredithR05}. This notation overloads to vectors of
names: $\vec{x}^{\pi} := (x_{i}^{\pi} \; : \; 0 \leq i < |\vec{x}| )$ where $\pi \in \{L,R\}$.

We also use $P^{\Box} := P|\Box$.

In \cite{MeredithR05} an interpretation of the new operator is
given. It turns out that there are several possible interpretations
all enjoying the requisite algebraic properties of the operator (see
\cite{milner91polyadicpi}). We will therefore make liberal use of
$(\nu\; \vec{x})P$.

% subsection the_syntax_and_semantics_of_the_notation_system (end)   

\input{qm2pi.qmops} 

\input{qm2pi.sterngerlach} 

\input{qm2pi.metric} 

% section concurrent_process_calculi (end)

%\input{qm2pi.proofsketch}

% section proof sketch (end)

%\input{qm2pi.slviaknots} 

% section spatial logic via knots (end)

\input{qm2pi.conclusion}

% section conclusion (end)

%\input{qm2pi.dtcodes} 

% section wiring algorithm (end)

\input{qm2pi.ack} 

% section acknowledgments (end)

\newpage


\bibliographystyle{plain}   
\bibliography{../../biblios/main.bib}

\input{qm2pi.rhodetails}

\end{document}



\end{document}

 

%\ifpdf
%\usepackage[pdftex]{graphicx}
%\else
%\usepackage{graphicx}
%\fi

 % \ifpdf
%  \usepackage{pdfsync}
%  \if


%\title{Brief Article}
%\author{David F. Snyder}
%\author{L.G. Meredith}

%\address{Dept. of Math., Texas State University--San Marcos, San Marcos, TX 78666}
       
\pagestyle{empty}


\begin{document}

\lstset{language=[Objective]Caml,frame=shadowbox}

\documentclass[12pt]{llncs}
%\documentclass{jktr}

\usepackage[pdftex]{hyperref}                   
\usepackage {listings}
\usepackage {mathpartir}
\usepackage{bcprules}
%\usepackage{listings}
                       
\usepackage{graphicx} 
%\usepackage[margins=2.5cm,nohead,nofoot]{geometry}
%\usepackage{geometry}
\usepackage{amsfonts}
\usepackage{amstext}
\usepackage{latexsym}
\usepackage{amssymb}
\usepackage{color}


%\include{myPreamble}
\documentclass[12pt]{llncs}
%\documentclass{jktr}

\usepackage[pdftex]{hyperref}                   
\usepackage {listings}
\usepackage {mathpartir}
\usepackage{bcprules}
%\usepackage{listings}
                       
\usepackage{graphicx} 
%\usepackage[margins=2.5cm,nohead,nofoot]{geometry}
%\usepackage{geometry}
\usepackage{amsfonts}
\usepackage{amstext}
\usepackage{latexsym}
\usepackage{amssymb}
\usepackage{color}


%\include{myPreamble}
\include{qm2pi.local} 

%\ifpdf
%\usepackage[pdftex]{graphicx}
%\else
%\usepackage{graphicx}
%\fi

 % \ifpdf
%  \usepackage{pdfsync}
%  \if


%\title{Brief Article}
%\author{David F. Snyder}
%\author{L.G. Meredith}

%\address{Dept. of Math., Texas State University--San Marcos, San Marcos, TX 78666}
       
\pagestyle{empty}


\begin{document}

\lstset{language=[Objective]Caml,frame=shadowbox}

\input{qm2pi.front}

% section front matter (end)

\input{qm2pi.intro} 
 
% section introduction (end)

% \input{qm2pi.knotations} 

% section notation (end)

\input{qm2pi.process.calculi} 

% section concurrent_process_calculi_and_spatial_logics_ (end)
    
%\input{qm2pi.knots2pi} 

%\input{qm2pi.trefoil} 

%\input{qm2pi.mainthm} 

% subsection basic_interpretation (end)

%\input{qm2pi.rho.presentation} 
\subsection{The syntax and semantics of the notation system}\label{sub:the_syntax_and_semantics_of_the_notation_system} % (fold)

We now summarize a technical presentation of the calculus that
embodies our theory of dynamics. The typical presentation of such a
calculus follows the style of giving generators and relations on
them. The grammar, below, describing term constructors, freely
generates the set of processes, $\Proc$. This set is then quotiented
by a relation known as structural congruence and it is over this set
that the notion of dynamics is expressed. This presentation is
essentially that of \cite{MeredithR05} with the addition of
polyadicity and summation. For readability we have relegated some of
the technical subtleties to an appendix.

\subsubsection{Process grammar}\label{subsub:process_grammar}

\begin{mathpar}
  \inferrule* [lab=synchronization] {} {{M} \bc \pzero \;|\; x?F \;|\; x!C }
  \and
  \inferrule* [lab=abstraction] {} {{F} \bc (x)P}
  \and
  \inferrule* [lab=concretion] {} {{C} \bc \langle Q \rangle}
  \and
  \inferrule* [lab=process] {} {{P,Q} \bc M \;| \;P|Q \;|\; @{x}}
  \and
  \inferrule* [lab=name] {} {{x} \bc \quotep{P}}
\end{mathpar} 

Note that $\vec{x}$ (resp. $\vec{P}$) denotes a vector of names
(resp. processes) of length $|\vec{x}|$ (resp. $|\vec{P}|$). We adopt
the following useful abbreviations.

\begin{mathpar}
   x?(\vec{y}).P := x.(\vec{y})P \and  x\clift{\vec{P}} := x.\clift{\vec{P}}
   \and x!(y) := \lift{x}{\dropn{y}}
   \and \Pi_{i=0}^{n-1}P_i := P_0 | \ldots | P_{n-1}
\end{mathpar}

\subsubsection{Structural congruence}

\paragraph{Free and bound names and alpha-equivalence.} At the
core of structural equivalence is alpha-equivalence which identifies
process that are the same up to a change of variable. Formally, we
recognize the distinction between free and bound names. The free names
of a process, $\freenames{P}$, may be calculated recursively as
follows:

\begin{mathpar}
\freenames{\pzero} := \emptyset
  \and \\
  \freenames{x?(y).P} := \{ x \} \cup (\freenames{P} \setminus \{ y \})
  \and 
  \freenames{x!\langle P \rangle} := \{ x \} \cup \{ P \} 
  \and \\
  \freenames{P|Q} := \freenames{P} \cup \freenames{Q}
  \and \\
  \freenames{@{x}} := \{ x \}
\end{mathpar}

$\pi$
$\quotep{\pi}$

$\freenames{-} : \pi \to \mathcal{P}(\quotep{\pi})$

\begin{eqnarray*}
  \freenames{\pzero} & := & \emptyset \\
  \freenames{x?(y).P} & := & \{ x \} \cup (\freenames{P} \setminus \{ y \}) \\
  \freenames{x!\langle P \rangle} & := & \{ x \} \cup \{ P \} \\
  \freenames{P|Q} & := & \freenames{P} \cup \freenames{Q} \\
  \freenames{\dropn{x}} & := & \{ x \}
\end{eqnarray*}

The bound names of a process, $\boundnames{P}$, are those names occurring in $P$
that are not free. For example, in $x?(y).0$, the name $x$ is free, while $y$ is bound.

\begin{mathpar}
  \inferrule* [lab=monoidal-laws] {} { P|Q \equiv Q|P \and P|0 \equiv P \and P|(Q|R) \equiv (P|Q)|R }
\end{mathpar}

\begin{mathpar}
  \inferrule* [lab=alpha-equivalence] {} { (x)P \equiv (y)P\{y/x\} \and y \not\in \freenames{P} }
\end{mathpar}

\begin{definition}
Then two processes, $P,Q$, are alpha-equivalent if $P = Q\{\vec{y}/\vec{x}\}$ for
some $\vec{x} \in \boundnames{Q},\vec{y} \in \boundnames{P}$, where $Q\{\vec{y}/\vec{x}\}$
denotes the capture-avoiding substitution of $\vec{y}$ for $\vec{x}$ in $Q$.
\end{definition}

\begin{definition}
  The {\em structural congruence} \cite{SangiorgiWalker} , $\equiv$,
  between processes is the least congruence containing
  alpha-equivalence, satisfying the abelian monoid laws
  (associativity, commutativity and $\pzero$ as identity) for parallel
  composition $|$ and for summation $+$.
\end{definition}

\subsection{Name equivalence}

We take name equivalence, written $\nameeq$, to be the smallest
equivalence relation generated by the following rules.

\begin{mathpar}
\inferrule*[lab=Quote-drop]
{ }
{ \quotep{@{x}} \nameeq x }

\inferrule*[lab=Struct-equiv]
{ P \scong Q }
{ \quotep{P} \nameeq \quotep{Q} }
\end{mathpar}

The astute reader will have noticed that the mutual recursion of names
and processes imposes a mutual recursion on alpha-equivalence and
structural equivalence via name-equivalence. Fortunately, all of this
works out pleasantly and we may calculate in the natural way, free of
concern. The reader interested in the details is referred to the
appendix \ref{appendix:rho_details}.

\subsection{Substitution}

We use $\Proc$ for the set of processes, $\QProc$ for the set of
names, and $\id{\{}\vec{y} / \vec{x} \id{\}}$ to denote partial maps,
$s : \QProc \rightarrow \QProc$. A map, $s$ lifts, uniquely, to a map
on process terms, $\widehat{s} : \Proc \rightarrow \Proc$ by the
following equations.

\begin{mathpar}
  (0) \psubstp{Q}{P} := 0 \\
  (R \juxtap S) \psubstp{Q}{P}
  :=    
  (R)\psubstp{Q}{P} \juxtap (S) \psubstp{Q}{P} \\
  (x?(y).R) \psubstp{Q}{P}    
  :=    
  (x)\substp{Q}{P} (z)\concat( (R \psubstn{z}{y}) \psubstp{Q}{P} ) \\
  (\lift{x}{R}) \psubstp{Q}{P}  
  :=
  \lift{(x)\substp{Q}{P}}{ R \psubstp{Q}{P} } \\
%   (\dropn{x})  \psubstp{Q}{P}       
%   := 
%   \left\{ 
%     \begin{array}{ccc} 
%       \dropn{\quotep{Q}} & & x \nameeq \quotep{P} \\
%       \dropn{x} & & otherwise \\
%     \end{array}
%   \right. 
  (\dropn{x})  \psubstp{Q}{P}       
  := 
  \left\{ 
    \begin{array}{ccc} 
      Q & & x \nameeq \quotep{P} \\
      \dropn{x} & & otherwise \\
    \end{array}
  \right.
\end{mathpar}
 

where

\begin{eqnarray}
  (x)\id{\{} \lpquote Q \rpquote / \lpquote P \rpquote \id{\}}            = 
  \left\{ 
    \begin{array}{ccc}
      \lpquote Q \rpquote & & x \nameeq \lpquote P \rpquote \\
      x & & otherwise \\
    \end{array}
  \right. \nonumber
\end{eqnarray}

and $z$ is chosen distinct from $\quotep{P}$, $\quotep{Q}$, the free
names in $Q$, and all the names in $R$. Our $\alpha$-equivalence will
be built in the standard way from this substitution.

\begin{remark}\label{rem:no_self_referential_names}
  One consequence of these definitions is that $\forall P. \quotep{P}
  \not\in \freenames{P}$.
\end{remark}

\subsection{ Dynamic quote: an example }

Anticipating something of what's to come, consider applying the
substitution, $\widehat{\id{\{}u / z \id{\}}}$, to the following pair
of processes, $\lift{w}{y!(z)}$ and $w[ \lpquote y!(z) \rpquote ]$.

\begin{eqnarray}
	\lift{w}{y!(z)}\widehat{\id{\{}u / z \id{\}}}
		& = &
		\lift{w}{y!(u)} \nonumber\\
	w[ \lpquote y!(z) \rpquote ] \widehat{ \id{\{}u / z \id{\}} }
		& = &
		w[ \lpquote y!(z) \rpquote ] \nonumber
\end{eqnarray}

Because the body of the process between quotes is impervious to
substitution, we get radically different answers. In fact, by
examining the first process in an input context,
e.g. $x?(z).\lift{w}{y!(z)}$, we see that the process under the lift
operator may be shaped by prefixed inputs binding a name inside it. In
this sense, the lift operator will be seen as a way to dynamically
construct processes before reifying them as names.

Finally equipped with these standard features we can present the
dynamics of the calculus.

\subsubsection{Operational semantics} 

Finally, we introduce the computational dynamics. What marks these
algebras as distinct from other more traditionally studied algebraic
structures, e.g. vector spaces or polynomial rings, is the manner in
which dynamics is captured. In traditional structures, dynamics is typically
expressed through morphisms between such structures, as in linear maps
between vector spaces or morphisms between rings. In algebras
associated with the semantics of computation, the dynamics is
expressed as part of the algebraic structure itself, through a
reduction reduction relation typically denoted by $\red$. Below, we
give a recursive presentation of this relation for the calculus used
in the encoding.

$\red \subseteq \pi \times \pi$
$\red : \pi \to \mathcal{P}(\pi)$

\begin{mathpar}
  \inferrule* [lab=Comm] { \textsf{match}( x_{src}, x_{trgt} ) } { x_{trgt}?(y)P \; | \; x_{src}!\langle {Q} \rangle \red P\{\quotep{Q}/y}\} }
  \and \\
  \inferrule* [lab=Par] {{P} \red {P}'} {{{P} | {Q}} \red {{P}' | {Q}}}
  \and
  \inferrule* [lab=Equiv]{{{P} \scong {P}'} \andalso {{P}' \red {Q}'} \andalso {{Q}' \scong {Q}}}{{P} \red {Q}}
\end{mathpar}

\begin{eqnarray*}
  match_{\equiv} (\quotep{P},\quotep{Q}) & := & P \equiv Q \\
  match_{\dagger}(\quotep{P},\quotep{Q}) & := & \forall R. P|Q \red^{*} R => R \red^{*} 0 \\
  match_{K}(\quotep{P},\quotep{Q}) & := & K \mbox{ for some context } K
\end{eqnarray*}

$u?(x)P | u!\langle Q \rangle \red P\{\quotep{Q}/x\}$

%We write $\wred$ for $\red^*$, and $P\red$ if $\exists Q $ such that $ P \red Q$.
We write $P\red$ if $\exists Q $ such that $ P \red Q$ and $P\not\red$, otherwise.

\section{Replication}

As mentioned before, it is known that replication (and hence
recursion) can be implemented in a higher-order process algebra
\cite{SangiorgiWalker}. As our first example of calculation with the
machinery thus far presented we give the construction explicitly in
the {\rhoc}.

\begin{eqnarray}
	D_{x} & := & \prefix{x}{y}{(\binpar{\outputp{x}{y}}{@{y}})} \nonumber\\
	\bangp_{x}{P} & := & \binpar{{x}!\langle{\binpar{D_{x}}{P}}\rangle}{D_{x}} \nonumber
\end{eqnarray}

\begin{eqnarray}
	\bangp_{x}{P} & & \nonumber\\
	=
	& {x}!\langle{(\prefix{x}{y}{(\outputp{x}{y} | @{y})) | P}}\rangle 
	      | \prefix{x}{y}{(\outputp{x}{y} | @{y})} & \nonumber\\
	\red
	& (\outputp{x}{y} | @{y})\substn{\quotep{(\prefix{x}{y}{(@{y} | \outputp{x}{y})) | P}}}{y} & \nonumber\\
	=
	& \outputp{x}{\quotep{(\prefix{x}{y}{(\outputp{x}{y} | @{y})) | P}}}
	  | {(\prefix{x}{y}{(\outputp{x}{y} | @{y})) | P}} & \nonumber\\
	\red
	& \ldots & \nonumber\\
	\red^*
	& P | P | \ldots & \nonumber
\end{eqnarray}

Of course, this encoding, as an implementation, runs away, unfolding
$\bangp{P}$ eagerly. A lazier and more implementable replication
operator, restricted to input-guarded processes, may be obtained as follows.

\begin{eqnarray}
\bangp{\prefix{u}{v}{P}} 
	:= 
	\binpar{\lift{x}{\prefix{u}{v}{(\binpar{D(x)}{P})}}}{D(x)} \nonumber
\end{eqnarray}

\begin{remark}
  Note that the lazier definition still does not deal with summation
  or mixed summation (i.e. sums over input and output). The reader is
  invited to construct definitions of replication that deal with these
  features. 

  Further, the definitions are parameterized in a name, $x$. Can you,
  gentle reader, make a definition that eliminates this parameter and
  guarantees no accidental interaction between the replication
  machinery and the process being replicated -- i.e. no accidental
  sharing of names used by the process to get its work done and the
  name(s) used by the replication to effect copying. This latter
  revision of the definition of replication is crucial to obtaining
  the expected identity $!!P \sim !P$.
\end{remark}

\begin{remark}\label{rem:paradoxical_combinator}
  The reader familiar with the lambda calculus will have noticed the
  similarity between $D$ and the paradoxical combinator.

  [Ed. note: the existence of this seems to suggest we have to be more
  restrictive on the set of processes and names we admit if we are to
  support no-cloning.]
\end{remark}

\subsubsection{Bisimulation}

The computational dynamics gives rise to another kind of equivalence,
the equivalence of computational behavior. As previously mentioned
this is typically captured \emph{via} some form of bisimulation.

% The notion we use in this paper is weak barbed bisimulation
% \cite{milner91polyadicpi}.

The notion we use in this paper is derived from weak barbed
bisimulation \cite{milner91polyadicpi}. 

\begin{definition}
An \emph{observation relation}, $\downarrow_{\mathcal N}$, over a set
of names, $\mathcal N$, is the smallest relation satisfying the rules
below.

\infrule[Out-barb]{y \in {\mathcal N}, \; x \nameeq y}
		  {\outputp{x}{v} \downarrow_{\mathcal N} x}
\infrule[Par-barb]{\mbox{$P\downarrow_{\mathcal N} x$ or $Q\downarrow_{\mathcal N} x$}}
		  {\binpar{P}{Q} \downarrow_{\mathcal N} x}

We write $P \Downarrow_{\mathcal N} x$ if there is $Q$ such that 
$P \wred Q$ and $Q \downarrow_{\mathcal N} x$.
\end{definition}

\begin{definition}
%\label{def.bbisim}
An  ${\mathcal N}$-\emph{barbed bisimulation} over a set of names, ${\mathcal N}$, is a symmetric binary relation 
${\mathcal S}_{\mathcal N}$ between agents such that $P\rel{S}_{\mathcal N}Q$ implies:
\begin{enumerate}
\item If $P \red P'$ then $Q \wred Q'$ and $P'\rel{S}_{\mathcal N} Q'$.
\item If $P\downarrow_{\mathcal N} x$, then $Q\Downarrow_{\mathcal N} x$.
\end{enumerate}
$P$ is ${\mathcal N}$-barbed bisimilar to $Q$, written
$P \wbbisim_{\mathcal N} Q$, if $P \rel{S}_{\mathcal N} Q$ for some ${\mathcal N}$-barbed bisimulation ${\mathcal S}_{\mathcal N}$.
\end{definition}

$\mathcal{R} \subseteq \pi \times \pi$

$P \mathcal{R} Q => \forall P'. P \red P' \Rightarrow \exists Q'. Q \red Q', P' \mathcal{R} Q'$

$P \vdash x \Rightarrow Q \vdash x$

\begin{mathpar}
  \inferrule*[lab=Out-barb]{x \nameeq y}{{y}!\langle{Q}\rangle \vdash x}
  \and
  \inferrule*[lab=Par-barb]{\mbox{$P\vdash x$ or $Q\vdash x$}}{\binpar{P}{Q} \vdash x}
\end{mathpar}

\subsubsection{Contexts}

One of the principle advantages of computational calculi like the
$\pi$-calculus is a well-defined notion of context,
contextual-equivalence and a correlation between
contextual-equivalence and notions of bisimulation. The notion of
context allows the decomposition of a process into (sub-)process and
its syntactic environment, its context. Thus, a context may be
thought of as a process with a ``hole'' (written $\Box$) in it. The
application of a context $M$ to a process $P$, written $M[P]$, is
tantamount to filling the hole in $M$ with $P$. In this paper we do
not need the full weight of this theory, but do make use of the notion
of context in the proof the main theorem. 

\begin{mathpar}
  \inferrule* [lab=summation] {} {{M_{M},M_{N}} \bc \Box \;|\; x.M_{A} \;|\; M_{M}+M_{N}}
  \and
  \inferrule* [lab=agent] {} {{M_{A}} \bc (\vec{x})M_{P} \;| \; \clift{P_0,\ldots,M_{P},\ldots,P_N}}
  \and \\
  \inferrule* [lab=process] {} {{M_{P}} \bc M_{N} \;| \;P|M_{P} }
\end{mathpar} 

\begin{mathpar}
  \inferrule* [lab=sychronization] {} {M_{N} \bc \Box \;|\; x?M_{F} \;|\; x!M_{C}}
  \and
  \inferrule* [lab=abstraction] {} {{M_{F}} \bc (x)M_{P} }
  \and
  \inferrule* [lab=concretion] {} {{M_{C}} \bc \langle M_{P} \rangle }
  \and \\
  \inferrule* [lab=process] {} {{M_{P}} \bc M_{N} \;| \;P|M_{P} }
\end{mathpar}

\begin{definition}[contextual application] Given a context $M$, and
  process $P$, we define the \emph{contextual application}, $M[P] :=
  M\{P/\Box\}$. That is, the contextual application of M to P is the
  substitution of $P$ for $\Box$ in $M$.
\end{definition}

$\meaningof{-} : L \to \mathcal{P}(\pi)$

\begin{mathpar}
  \inferrule* [lab=collection] {} {\meaningof{true} = \pi, \and \meaningof{~E} = \pi \setminus \meaningof{E}, \and \meaningof{E_{1} \& E_{2}} = \meaningof{E_{1}} \cap \meaningof{E_{2}}}
\end{mathpar}

\begin{mathpar}
  \inferrule* [lab=structure] {} {\meaningof{0} = \{ P \in \pi | P \equiv 0 \}, \and \\ \meaningof{E_1 | E_2} = \{ P \in \pi | P \equiv P_{1} | P_{2}, P_{1} \in \meaningof{E_{1}}, P_{2} \in \meaningof{E_2}\} }
\end{mathpar}

\begin{mathpar}
 \inferrule* [lab=behavior] {} {\meaningof{\langle a?b \rangle E} = \{ P \in \pi | P \equiv Q | u?(y)P', \\ \and \\\\ \and \\ \;\;\; u \in \meaningof{a}, \forall z.P'\{z/y\} \in \meaningof{E\{z/b\}}\}, \and \\ \meaningof{a!E} = \{ P \in \pi | P \equiv Q | x!\langle P' \rangle, x \in \meaningof{a} P' \in \meaningof{E}\} }
\end{mathpar}

\begin{mathpar}
 \inferrule* [lab=nominal] {} {\meaningof{\quotep{E}} = \{ \quotep{P} \in \quotep{\pi} | P \in \meaningof{E} \}, \and \meaningof{\quotep{P}} = \{ \quotep{Q} \in \quotep{\pi} | P \equiv Q \} \and \\ \meaningof{@\quotep{E}} = \{ P \in \pi | P \equiv @x, x \in \meaningof{E} \}}
\end{mathpar}

\begin{eqnarray*}
  \\
  \meaningof{-} : TS \to ST
\end{eqnarray*}

\begin{eqnarray*}
  \\
  L : TS \to ST
\end{eqnarray*}

\begin{eqnarray*}
  \\
  P \models E \iff P \in \meaningof{E}
\end{eqnarray*}

\begin{eqnarray*}
  P \approx_{L} Q \iff \forall E \in L. P \models E \iff Q \models E
\end{eqnarray*}

\begin{eqnarray*}
  P \approx_{K} Q
\end{eqnarray*}

\begin{eqnarray*}
  P \approx Q
\end{eqnarray*}

$\approx_{K} = \approx = \approx_{L}$

\subsubsection{Contextual duality}

Note that contexts extend the quotation operation to a family of
operations from processes to names. Given a context, $M$, we can
define a \emph{nominal context}, $\quotep{M}$ by $\quotep{M}[P] :=
\quotep{M[P]}$. To foreshadow what is to come we observe that these
operations enjoy a duality with processes very much like the duality
between vectors and maps from vectors to scalars.

Further, because the calculus is essentially higher-order, we have a
correspondence between contexts and processes. More specifically,
given a name $x$ and a context $M$ we can construct $M^{*}_{x}$ such
that 

\begin{mathpar}
  M^{*}_{x} | \lift{x}{P} \red M[P]
\end{mathpar}

namely,

\begin{mathpar}
  M^{*}_{x} := x?(u).M[\dropn{u}]
\end{mathpar}

The dependence of $M^{*}_{x}$ on a name makes it an abstraction, 

\begin{mathpar}
  M^{*} := (x)x?(u).M[\dropn{u}]
\end{mathpar}

\subsection{Additional notation}

It will sometimes be convenient to denote the process a name
quotes. We already have the notation $x = \quotep{P}$, but it will be
convenient to introduce an alternate notation, $\procn{x}$, when we
want to emphasize the connection to the use of the name. Note that, by
virtue of name equivalence, $\quotep{\procn{x}} \nameeq x$; so, the
notation is consistent with previous definitions.

Further, because names have structure it is possible to effect
substitutions on the basis of that structure. This means we need to
upgrade our notation for substitutions, which we accomplish by
adapting comprehension notation. Thus,

\begin{mathpar}
  P\{ y / x : x \in S \}
\end{mathpar}

is interpreted to mean the process derived from P by replacing (in a
capture-avoiding manner) each occurrence of $x$ in $S$ by $y$. For example,

\begin{mathpar}
  P\{ \quotep{\procn{x}|\procn{x}} / x : x \in \freenames{P} \}
\end{mathpar}

will replace each (occurrence) of a free name $x$ in $P$ by
$\quotep{\procn{x}|\procn{x}}$.

Also, we will avail ourselves of the notation $x^{L}$ and $x^{R}$ to
denote injections of a name into disjoint copies of the name
space. There are numerous ways to accomplish this. One example can be
found in \cite{MeredithR05}. This notation overloads to vectors of
names: $\vec{x}^{\pi} := (x_{i}^{\pi} \; : \; 0 \leq i < |\vec{x}| )$ where $\pi \in \{L,R\}$.

We also use $P^{\Box} := P|\Box$.

In \cite{MeredithR05} an interpretation of the new operator is
given. It turns out that there are several possible interpretations
all enjoying the requisite algebraic properties of the operator (see
\cite{milner91polyadicpi}). We will therefore make liberal use of
$(\nu\; \vec{x})P$.

% subsection the_syntax_and_semantics_of_the_notation_system (end)   

\input{qm2pi.qmops} 

\input{qm2pi.sterngerlach} 

\input{qm2pi.metric} 

% section concurrent_process_calculi (end)

%\input{qm2pi.proofsketch}

% section proof sketch (end)

%\input{qm2pi.slviaknots} 

% section spatial logic via knots (end)

\input{qm2pi.conclusion}

% section conclusion (end)

%\input{qm2pi.dtcodes} 

% section wiring algorithm (end)

\input{qm2pi.ack} 

% section acknowledgments (end)

\newpage


\bibliographystyle{plain}   
\bibliography{../../biblios/main.bib}

\input{qm2pi.rhodetails}

\end{document}

 

%\ifpdf
%\usepackage[pdftex]{graphicx}
%\else
%\usepackage{graphicx}
%\fi

 % \ifpdf
%  \usepackage{pdfsync}
%  \if


%\title{Brief Article}
%\author{David F. Snyder}
%\author{L.G. Meredith}

%\address{Dept. of Math., Texas State University--San Marcos, San Marcos, TX 78666}
       
\pagestyle{empty}


\begin{document}

\lstset{language=[Objective]Caml,frame=shadowbox}

\documentclass[12pt]{llncs}
%\documentclass{jktr}

\usepackage[pdftex]{hyperref}                   
\usepackage {listings}
\usepackage {mathpartir}
\usepackage{bcprules}
%\usepackage{listings}
                       
\usepackage{graphicx} 
%\usepackage[margins=2.5cm,nohead,nofoot]{geometry}
%\usepackage{geometry}
\usepackage{amsfonts}
\usepackage{amstext}
\usepackage{latexsym}
\usepackage{amssymb}
\usepackage{color}


%\include{myPreamble}
\include{qm2pi.local} 

%\ifpdf
%\usepackage[pdftex]{graphicx}
%\else
%\usepackage{graphicx}
%\fi

 % \ifpdf
%  \usepackage{pdfsync}
%  \if


%\title{Brief Article}
%\author{David F. Snyder}
%\author{L.G. Meredith}

%\address{Dept. of Math., Texas State University--San Marcos, San Marcos, TX 78666}
       
\pagestyle{empty}


\begin{document}

\lstset{language=[Objective]Caml,frame=shadowbox}

\input{qm2pi.front}

% section front matter (end)

\input{qm2pi.intro} 
 
% section introduction (end)

% \input{qm2pi.knotations} 

% section notation (end)

\input{qm2pi.process.calculi} 

% section concurrent_process_calculi_and_spatial_logics_ (end)
    
%\input{qm2pi.knots2pi} 

%\input{qm2pi.trefoil} 

%\input{qm2pi.mainthm} 

% subsection basic_interpretation (end)

%\input{qm2pi.rho.presentation} 
\subsection{The syntax and semantics of the notation system}\label{sub:the_syntax_and_semantics_of_the_notation_system} % (fold)

We now summarize a technical presentation of the calculus that
embodies our theory of dynamics. The typical presentation of such a
calculus follows the style of giving generators and relations on
them. The grammar, below, describing term constructors, freely
generates the set of processes, $\Proc$. This set is then quotiented
by a relation known as structural congruence and it is over this set
that the notion of dynamics is expressed. This presentation is
essentially that of \cite{MeredithR05} with the addition of
polyadicity and summation. For readability we have relegated some of
the technical subtleties to an appendix.

\subsubsection{Process grammar}\label{subsub:process_grammar}

\begin{mathpar}
  \inferrule* [lab=synchronization] {} {{M} \bc \pzero \;|\; x?F \;|\; x!C }
  \and
  \inferrule* [lab=abstraction] {} {{F} \bc (x)P}
  \and
  \inferrule* [lab=concretion] {} {{C} \bc \langle Q \rangle}
  \and
  \inferrule* [lab=process] {} {{P,Q} \bc M \;| \;P|Q \;|\; @{x}}
  \and
  \inferrule* [lab=name] {} {{x} \bc \quotep{P}}
\end{mathpar} 

Note that $\vec{x}$ (resp. $\vec{P}$) denotes a vector of names
(resp. processes) of length $|\vec{x}|$ (resp. $|\vec{P}|$). We adopt
the following useful abbreviations.

\begin{mathpar}
   x?(\vec{y}).P := x.(\vec{y})P \and  x\clift{\vec{P}} := x.\clift{\vec{P}}
   \and x!(y) := \lift{x}{\dropn{y}}
   \and \Pi_{i=0}^{n-1}P_i := P_0 | \ldots | P_{n-1}
\end{mathpar}

\subsubsection{Structural congruence}

\paragraph{Free and bound names and alpha-equivalence.} At the
core of structural equivalence is alpha-equivalence which identifies
process that are the same up to a change of variable. Formally, we
recognize the distinction between free and bound names. The free names
of a process, $\freenames{P}$, may be calculated recursively as
follows:

\begin{mathpar}
\freenames{\pzero} := \emptyset
  \and \\
  \freenames{x?(y).P} := \{ x \} \cup (\freenames{P} \setminus \{ y \})
  \and 
  \freenames{x!\langle P \rangle} := \{ x \} \cup \{ P \} 
  \and \\
  \freenames{P|Q} := \freenames{P} \cup \freenames{Q}
  \and \\
  \freenames{@{x}} := \{ x \}
\end{mathpar}

$\pi$
$\quotep{\pi}$

$\freenames{-} : \pi \to \mathcal{P}(\quotep{\pi})$

\begin{eqnarray*}
  \freenames{\pzero} & := & \emptyset \\
  \freenames{x?(y).P} & := & \{ x \} \cup (\freenames{P} \setminus \{ y \}) \\
  \freenames{x!\langle P \rangle} & := & \{ x \} \cup \{ P \} \\
  \freenames{P|Q} & := & \freenames{P} \cup \freenames{Q} \\
  \freenames{\dropn{x}} & := & \{ x \}
\end{eqnarray*}

The bound names of a process, $\boundnames{P}$, are those names occurring in $P$
that are not free. For example, in $x?(y).0$, the name $x$ is free, while $y$ is bound.

\begin{mathpar}
  \inferrule* [lab=monoidal-laws] {} { P|Q \equiv Q|P \and P|0 \equiv P \and P|(Q|R) \equiv (P|Q)|R }
\end{mathpar}

\begin{mathpar}
  \inferrule* [lab=alpha-equivalence] {} { (x)P \equiv (y)P\{y/x\} \and y \not\in \freenames{P} }
\end{mathpar}

\begin{definition}
Then two processes, $P,Q$, are alpha-equivalent if $P = Q\{\vec{y}/\vec{x}\}$ for
some $\vec{x} \in \boundnames{Q},\vec{y} \in \boundnames{P}$, where $Q\{\vec{y}/\vec{x}\}$
denotes the capture-avoiding substitution of $\vec{y}$ for $\vec{x}$ in $Q$.
\end{definition}

\begin{definition}
  The {\em structural congruence} \cite{SangiorgiWalker} , $\equiv$,
  between processes is the least congruence containing
  alpha-equivalence, satisfying the abelian monoid laws
  (associativity, commutativity and $\pzero$ as identity) for parallel
  composition $|$ and for summation $+$.
\end{definition}

\subsection{Name equivalence}

We take name equivalence, written $\nameeq$, to be the smallest
equivalence relation generated by the following rules.

\begin{mathpar}
\inferrule*[lab=Quote-drop]
{ }
{ \quotep{@{x}} \nameeq x }

\inferrule*[lab=Struct-equiv]
{ P \scong Q }
{ \quotep{P} \nameeq \quotep{Q} }
\end{mathpar}

The astute reader will have noticed that the mutual recursion of names
and processes imposes a mutual recursion on alpha-equivalence and
structural equivalence via name-equivalence. Fortunately, all of this
works out pleasantly and we may calculate in the natural way, free of
concern. The reader interested in the details is referred to the
appendix \ref{appendix:rho_details}.

\subsection{Substitution}

We use $\Proc$ for the set of processes, $\QProc$ for the set of
names, and $\id{\{}\vec{y} / \vec{x} \id{\}}$ to denote partial maps,
$s : \QProc \rightarrow \QProc$. A map, $s$ lifts, uniquely, to a map
on process terms, $\widehat{s} : \Proc \rightarrow \Proc$ by the
following equations.

\begin{mathpar}
  (0) \psubstp{Q}{P} := 0 \\
  (R \juxtap S) \psubstp{Q}{P}
  :=    
  (R)\psubstp{Q}{P} \juxtap (S) \psubstp{Q}{P} \\
  (x?(y).R) \psubstp{Q}{P}    
  :=    
  (x)\substp{Q}{P} (z)\concat( (R \psubstn{z}{y}) \psubstp{Q}{P} ) \\
  (\lift{x}{R}) \psubstp{Q}{P}  
  :=
  \lift{(x)\substp{Q}{P}}{ R \psubstp{Q}{P} } \\
%   (\dropn{x})  \psubstp{Q}{P}       
%   := 
%   \left\{ 
%     \begin{array}{ccc} 
%       \dropn{\quotep{Q}} & & x \nameeq \quotep{P} \\
%       \dropn{x} & & otherwise \\
%     \end{array}
%   \right. 
  (\dropn{x})  \psubstp{Q}{P}       
  := 
  \left\{ 
    \begin{array}{ccc} 
      Q & & x \nameeq \quotep{P} \\
      \dropn{x} & & otherwise \\
    \end{array}
  \right.
\end{mathpar}
 

where

\begin{eqnarray}
  (x)\id{\{} \lpquote Q \rpquote / \lpquote P \rpquote \id{\}}            = 
  \left\{ 
    \begin{array}{ccc}
      \lpquote Q \rpquote & & x \nameeq \lpquote P \rpquote \\
      x & & otherwise \\
    \end{array}
  \right. \nonumber
\end{eqnarray}

and $z$ is chosen distinct from $\quotep{P}$, $\quotep{Q}$, the free
names in $Q$, and all the names in $R$. Our $\alpha$-equivalence will
be built in the standard way from this substitution.

\begin{remark}\label{rem:no_self_referential_names}
  One consequence of these definitions is that $\forall P. \quotep{P}
  \not\in \freenames{P}$.
\end{remark}

\subsection{ Dynamic quote: an example }

Anticipating something of what's to come, consider applying the
substitution, $\widehat{\id{\{}u / z \id{\}}}$, to the following pair
of processes, $\lift{w}{y!(z)}$ and $w[ \lpquote y!(z) \rpquote ]$.

\begin{eqnarray}
	\lift{w}{y!(z)}\widehat{\id{\{}u / z \id{\}}}
		& = &
		\lift{w}{y!(u)} \nonumber\\
	w[ \lpquote y!(z) \rpquote ] \widehat{ \id{\{}u / z \id{\}} }
		& = &
		w[ \lpquote y!(z) \rpquote ] \nonumber
\end{eqnarray}

Because the body of the process between quotes is impervious to
substitution, we get radically different answers. In fact, by
examining the first process in an input context,
e.g. $x?(z).\lift{w}{y!(z)}$, we see that the process under the lift
operator may be shaped by prefixed inputs binding a name inside it. In
this sense, the lift operator will be seen as a way to dynamically
construct processes before reifying them as names.

Finally equipped with these standard features we can present the
dynamics of the calculus.

\subsubsection{Operational semantics} 

Finally, we introduce the computational dynamics. What marks these
algebras as distinct from other more traditionally studied algebraic
structures, e.g. vector spaces or polynomial rings, is the manner in
which dynamics is captured. In traditional structures, dynamics is typically
expressed through morphisms between such structures, as in linear maps
between vector spaces or morphisms between rings. In algebras
associated with the semantics of computation, the dynamics is
expressed as part of the algebraic structure itself, through a
reduction reduction relation typically denoted by $\red$. Below, we
give a recursive presentation of this relation for the calculus used
in the encoding.

$\red \subseteq \pi \times \pi$
$\red : \pi \to \mathcal{P}(\pi)$

\begin{mathpar}
  \inferrule* [lab=Comm] { \textsf{match}( x_{src}, x_{trgt} ) } { x_{trgt}?(y)P \; | \; x_{src}!\langle {Q} \rangle \red P\{\quotep{Q}/y}\} }
  \and \\
  \inferrule* [lab=Par] {{P} \red {P}'} {{{P} | {Q}} \red {{P}' | {Q}}}
  \and
  \inferrule* [lab=Equiv]{{{P} \scong {P}'} \andalso {{P}' \red {Q}'} \andalso {{Q}' \scong {Q}}}{{P} \red {Q}}
\end{mathpar}

\begin{eqnarray*}
  match_{\equiv} (\quotep{P},\quotep{Q}) & := & P \equiv Q \\
  match_{\dagger}(\quotep{P},\quotep{Q}) & := & \forall R. P|Q \red^{*} R => R \red^{*} 0 \\
  match_{K}(\quotep{P},\quotep{Q}) & := & K \mbox{ for some context } K
\end{eqnarray*}

$u?(x)P | u!\langle Q \rangle \red P\{\quotep{Q}/x\}$

%We write $\wred$ for $\red^*$, and $P\red$ if $\exists Q $ such that $ P \red Q$.
We write $P\red$ if $\exists Q $ such that $ P \red Q$ and $P\not\red$, otherwise.

\section{Replication}

As mentioned before, it is known that replication (and hence
recursion) can be implemented in a higher-order process algebra
\cite{SangiorgiWalker}. As our first example of calculation with the
machinery thus far presented we give the construction explicitly in
the {\rhoc}.

\begin{eqnarray}
	D_{x} & := & \prefix{x}{y}{(\binpar{\outputp{x}{y}}{@{y}})} \nonumber\\
	\bangp_{x}{P} & := & \binpar{{x}!\langle{\binpar{D_{x}}{P}}\rangle}{D_{x}} \nonumber
\end{eqnarray}

\begin{eqnarray}
	\bangp_{x}{P} & & \nonumber\\
	=
	& {x}!\langle{(\prefix{x}{y}{(\outputp{x}{y} | @{y})) | P}}\rangle 
	      | \prefix{x}{y}{(\outputp{x}{y} | @{y})} & \nonumber\\
	\red
	& (\outputp{x}{y} | @{y})\substn{\quotep{(\prefix{x}{y}{(@{y} | \outputp{x}{y})) | P}}}{y} & \nonumber\\
	=
	& \outputp{x}{\quotep{(\prefix{x}{y}{(\outputp{x}{y} | @{y})) | P}}}
	  | {(\prefix{x}{y}{(\outputp{x}{y} | @{y})) | P}} & \nonumber\\
	\red
	& \ldots & \nonumber\\
	\red^*
	& P | P | \ldots & \nonumber
\end{eqnarray}

Of course, this encoding, as an implementation, runs away, unfolding
$\bangp{P}$ eagerly. A lazier and more implementable replication
operator, restricted to input-guarded processes, may be obtained as follows.

\begin{eqnarray}
\bangp{\prefix{u}{v}{P}} 
	:= 
	\binpar{\lift{x}{\prefix{u}{v}{(\binpar{D(x)}{P})}}}{D(x)} \nonumber
\end{eqnarray}

\begin{remark}
  Note that the lazier definition still does not deal with summation
  or mixed summation (i.e. sums over input and output). The reader is
  invited to construct definitions of replication that deal with these
  features. 

  Further, the definitions are parameterized in a name, $x$. Can you,
  gentle reader, make a definition that eliminates this parameter and
  guarantees no accidental interaction between the replication
  machinery and the process being replicated -- i.e. no accidental
  sharing of names used by the process to get its work done and the
  name(s) used by the replication to effect copying. This latter
  revision of the definition of replication is crucial to obtaining
  the expected identity $!!P \sim !P$.
\end{remark}

\begin{remark}\label{rem:paradoxical_combinator}
  The reader familiar with the lambda calculus will have noticed the
  similarity between $D$ and the paradoxical combinator.

  [Ed. note: the existence of this seems to suggest we have to be more
  restrictive on the set of processes and names we admit if we are to
  support no-cloning.]
\end{remark}

\subsubsection{Bisimulation}

The computational dynamics gives rise to another kind of equivalence,
the equivalence of computational behavior. As previously mentioned
this is typically captured \emph{via} some form of bisimulation.

% The notion we use in this paper is weak barbed bisimulation
% \cite{milner91polyadicpi}.

The notion we use in this paper is derived from weak barbed
bisimulation \cite{milner91polyadicpi}. 

\begin{definition}
An \emph{observation relation}, $\downarrow_{\mathcal N}$, over a set
of names, $\mathcal N$, is the smallest relation satisfying the rules
below.

\infrule[Out-barb]{y \in {\mathcal N}, \; x \nameeq y}
		  {\outputp{x}{v} \downarrow_{\mathcal N} x}
\infrule[Par-barb]{\mbox{$P\downarrow_{\mathcal N} x$ or $Q\downarrow_{\mathcal N} x$}}
		  {\binpar{P}{Q} \downarrow_{\mathcal N} x}

We write $P \Downarrow_{\mathcal N} x$ if there is $Q$ such that 
$P \wred Q$ and $Q \downarrow_{\mathcal N} x$.
\end{definition}

\begin{definition}
%\label{def.bbisim}
An  ${\mathcal N}$-\emph{barbed bisimulation} over a set of names, ${\mathcal N}$, is a symmetric binary relation 
${\mathcal S}_{\mathcal N}$ between agents such that $P\rel{S}_{\mathcal N}Q$ implies:
\begin{enumerate}
\item If $P \red P'$ then $Q \wred Q'$ and $P'\rel{S}_{\mathcal N} Q'$.
\item If $P\downarrow_{\mathcal N} x$, then $Q\Downarrow_{\mathcal N} x$.
\end{enumerate}
$P$ is ${\mathcal N}$-barbed bisimilar to $Q$, written
$P \wbbisim_{\mathcal N} Q$, if $P \rel{S}_{\mathcal N} Q$ for some ${\mathcal N}$-barbed bisimulation ${\mathcal S}_{\mathcal N}$.
\end{definition}

$\mathcal{R} \subseteq \pi \times \pi$

$P \mathcal{R} Q => \forall P'. P \red P' \Rightarrow \exists Q'. Q \red Q', P' \mathcal{R} Q'$

$P \vdash x \Rightarrow Q \vdash x$

\begin{mathpar}
  \inferrule*[lab=Out-barb]{x \nameeq y}{{y}!\langle{Q}\rangle \vdash x}
  \and
  \inferrule*[lab=Par-barb]{\mbox{$P\vdash x$ or $Q\vdash x$}}{\binpar{P}{Q} \vdash x}
\end{mathpar}

\subsubsection{Contexts}

One of the principle advantages of computational calculi like the
$\pi$-calculus is a well-defined notion of context,
contextual-equivalence and a correlation between
contextual-equivalence and notions of bisimulation. The notion of
context allows the decomposition of a process into (sub-)process and
its syntactic environment, its context. Thus, a context may be
thought of as a process with a ``hole'' (written $\Box$) in it. The
application of a context $M$ to a process $P$, written $M[P]$, is
tantamount to filling the hole in $M$ with $P$. In this paper we do
not need the full weight of this theory, but do make use of the notion
of context in the proof the main theorem. 

\begin{mathpar}
  \inferrule* [lab=summation] {} {{M_{M},M_{N}} \bc \Box \;|\; x.M_{A} \;|\; M_{M}+M_{N}}
  \and
  \inferrule* [lab=agent] {} {{M_{A}} \bc (\vec{x})M_{P} \;| \; \clift{P_0,\ldots,M_{P},\ldots,P_N}}
  \and \\
  \inferrule* [lab=process] {} {{M_{P}} \bc M_{N} \;| \;P|M_{P} }
\end{mathpar} 

\begin{mathpar}
  \inferrule* [lab=sychronization] {} {M_{N} \bc \Box \;|\; x?M_{F} \;|\; x!M_{C}}
  \and
  \inferrule* [lab=abstraction] {} {{M_{F}} \bc (x)M_{P} }
  \and
  \inferrule* [lab=concretion] {} {{M_{C}} \bc \langle M_{P} \rangle }
  \and \\
  \inferrule* [lab=process] {} {{M_{P}} \bc M_{N} \;| \;P|M_{P} }
\end{mathpar}

\begin{definition}[contextual application] Given a context $M$, and
  process $P$, we define the \emph{contextual application}, $M[P] :=
  M\{P/\Box\}$. That is, the contextual application of M to P is the
  substitution of $P$ for $\Box$ in $M$.
\end{definition}

$\meaningof{-} : L \to \mathcal{P}(\pi)$

\begin{mathpar}
  \inferrule* [lab=collection] {} {\meaningof{true} = \pi, \and \meaningof{~E} = \pi \setminus \meaningof{E}, \and \meaningof{E_{1} \& E_{2}} = \meaningof{E_{1}} \cap \meaningof{E_{2}}}
\end{mathpar}

\begin{mathpar}
  \inferrule* [lab=structure] {} {\meaningof{0} = \{ P \in \pi | P \equiv 0 \}, \and \\ \meaningof{E_1 | E_2} = \{ P \in \pi | P \equiv P_{1} | P_{2}, P_{1} \in \meaningof{E_{1}}, P_{2} \in \meaningof{E_2}\} }
\end{mathpar}

\begin{mathpar}
 \inferrule* [lab=behavior] {} {\meaningof{\langle a?b \rangle E} = \{ P \in \pi | P \equiv Q | u?(y)P', \\ \and \\\\ \and \\ \;\;\; u \in \meaningof{a}, \forall z.P'\{z/y\} \in \meaningof{E\{z/b\}}\}, \and \\ \meaningof{a!E} = \{ P \in \pi | P \equiv Q | x!\langle P' \rangle, x \in \meaningof{a} P' \in \meaningof{E}\} }
\end{mathpar}

\begin{mathpar}
 \inferrule* [lab=nominal] {} {\meaningof{\quotep{E}} = \{ \quotep{P} \in \quotep{\pi} | P \in \meaningof{E} \}, \and \meaningof{\quotep{P}} = \{ \quotep{Q} \in \quotep{\pi} | P \equiv Q \} \and \\ \meaningof{@\quotep{E}} = \{ P \in \pi | P \equiv @x, x \in \meaningof{E} \}}
\end{mathpar}

\begin{eqnarray*}
  \\
  \meaningof{-} : TS \to ST
\end{eqnarray*}

\begin{eqnarray*}
  \\
  L : TS \to ST
\end{eqnarray*}

\begin{eqnarray*}
  \\
  P \models E \iff P \in \meaningof{E}
\end{eqnarray*}

\begin{eqnarray*}
  P \approx_{L} Q \iff \forall E \in L. P \models E \iff Q \models E
\end{eqnarray*}

\begin{eqnarray*}
  P \approx_{K} Q
\end{eqnarray*}

\begin{eqnarray*}
  P \approx Q
\end{eqnarray*}

$\approx_{K} = \approx = \approx_{L}$

\subsubsection{Contextual duality}

Note that contexts extend the quotation operation to a family of
operations from processes to names. Given a context, $M$, we can
define a \emph{nominal context}, $\quotep{M}$ by $\quotep{M}[P] :=
\quotep{M[P]}$. To foreshadow what is to come we observe that these
operations enjoy a duality with processes very much like the duality
between vectors and maps from vectors to scalars.

Further, because the calculus is essentially higher-order, we have a
correspondence between contexts and processes. More specifically,
given a name $x$ and a context $M$ we can construct $M^{*}_{x}$ such
that 

\begin{mathpar}
  M^{*}_{x} | \lift{x}{P} \red M[P]
\end{mathpar}

namely,

\begin{mathpar}
  M^{*}_{x} := x?(u).M[\dropn{u}]
\end{mathpar}

The dependence of $M^{*}_{x}$ on a name makes it an abstraction, 

\begin{mathpar}
  M^{*} := (x)x?(u).M[\dropn{u}]
\end{mathpar}

\subsection{Additional notation}

It will sometimes be convenient to denote the process a name
quotes. We already have the notation $x = \quotep{P}$, but it will be
convenient to introduce an alternate notation, $\procn{x}$, when we
want to emphasize the connection to the use of the name. Note that, by
virtue of name equivalence, $\quotep{\procn{x}} \nameeq x$; so, the
notation is consistent with previous definitions.

Further, because names have structure it is possible to effect
substitutions on the basis of that structure. This means we need to
upgrade our notation for substitutions, which we accomplish by
adapting comprehension notation. Thus,

\begin{mathpar}
  P\{ y / x : x \in S \}
\end{mathpar}

is interpreted to mean the process derived from P by replacing (in a
capture-avoiding manner) each occurrence of $x$ in $S$ by $y$. For example,

\begin{mathpar}
  P\{ \quotep{\procn{x}|\procn{x}} / x : x \in \freenames{P} \}
\end{mathpar}

will replace each (occurrence) of a free name $x$ in $P$ by
$\quotep{\procn{x}|\procn{x}}$.

Also, we will avail ourselves of the notation $x^{L}$ and $x^{R}$ to
denote injections of a name into disjoint copies of the name
space. There are numerous ways to accomplish this. One example can be
found in \cite{MeredithR05}. This notation overloads to vectors of
names: $\vec{x}^{\pi} := (x_{i}^{\pi} \; : \; 0 \leq i < |\vec{x}| )$ where $\pi \in \{L,R\}$.

We also use $P^{\Box} := P|\Box$.

In \cite{MeredithR05} an interpretation of the new operator is
given. It turns out that there are several possible interpretations
all enjoying the requisite algebraic properties of the operator (see
\cite{milner91polyadicpi}). We will therefore make liberal use of
$(\nu\; \vec{x})P$.

% subsection the_syntax_and_semantics_of_the_notation_system (end)   

\input{qm2pi.qmops} 

\input{qm2pi.sterngerlach} 

\input{qm2pi.metric} 

% section concurrent_process_calculi (end)

%\input{qm2pi.proofsketch}

% section proof sketch (end)

%\input{qm2pi.slviaknots} 

% section spatial logic via knots (end)

\input{qm2pi.conclusion}

% section conclusion (end)

%\input{qm2pi.dtcodes} 

% section wiring algorithm (end)

\input{qm2pi.ack} 

% section acknowledgments (end)

\newpage


\bibliographystyle{plain}   
\bibliography{../../biblios/main.bib}

\input{qm2pi.rhodetails}

\end{document}



% section front matter (end)

\section{Introduction}\label{sec:introduction} % (fold)
In this draft of the material i am going to have to dispense with the
usual writing conventions adopted in papers on these topics. i'm going
to have adopt whatever tone i need at the time i'm writing up the
calculations. Sometimes this may be very conversational; others it may
be the barest mathematical grunts; others still it may be that i have
lifted text from one of my other papers because the exposition of some
point was better said there. i hope that my readers are not unduly put
out by this decision. i'm not doing this to flout convention or be
rebellious. i find these calculations very technically challenging. To
keep everything going technically, something has to give; i have to
let go of some cognitive burden. So, the academic writing style --
with all of its trade-offs in terms of facilitating technical
communication -- is what i'm letting go of. Perhaps subsequent drafts
can be tightened and polished, but for now, i'm going to speak as if
we were sitting together in a coffee shop with a laptop, wifi and a
pad of paper and a pencil.

So, here's what i have to say. We -- you and i, comfortably ensconced
in our coffee shop and well-equipped with our tools -- can realize and
carry out the calculations of quantum mechanics over a very different
formal theory of dynamics, a formal theory of dynamics that
corresponds to a theory of concurrent computation with
\emph{reflection}. It has the advantage that the underlying theory is
already `quantized', but supports analogues all of the continuuous
operations. Strikingly, this underlying theory has recently been
connected with a notion of metric that we can show, by calculating
together, coincides with the metric induced by the inner product.

There are a lot of reasons why you might be interested in seeing
calculations of this form. Here's why i'm interested. For the past
several centuries there has been no competitor to the ``Newtonian''
account of dynamics. As a result the predominant share of accounts of
dynamical systems and situations have had to be formulated in terms of
the Newtonian machinery. i view this as an intellectually dangerous
position to occupy. Everything, despite it's intrinsic shape, turns
into a nail to be hit with this hammer. Recently, however, the theory
of computation has matured to the point where we have candidates for
theories of dynamics that offer very different perspective on
reasoning about dynamical systems and situations. Testing these
candidates against very successful accounts of dynamical situations,
like quantum mechanics, is going to give us some sense of how mature
they are and some measure of the quality of these accounts of
dynamics.

\subsection{Summary of contributions and outline of paper}

So, we're going to develop an interpretation of the operations of
quantum mechanics normally interpreted by Hilbert spaces and
operators. We're going to do this over a theory of computation. Note
that this is very different than the usual quantum computation program
which develops notions of computation over quantum mechanics. Rather,
we are developing a story that aligns with Wheeler's slogan: It from
Bit. To do this we will first provide an account of the theory of
computation at play here. Then we will dive into a calculation-driven
interpretation of the operations of quantum mechanics.

The reason we take this approach is that -- until very recently --
there hasn't been an axiomatic account of quantum mechanics. As a
result there has been no sharp delineation of the mathematical theory
supporting interpretation of the physical theory and the physical
theory, itself. So, ambient features of the maths are free to be
exploited (or supressed) without a real accounting of their physical
relevance. There is no sharp statement ``here's the physical theory''
qua \emph{theory} and ``here's the mathematical interpretation''
enabling a judgment of how faithful the interpretation is -- apart
from experimental observation. When there is an axiomatic account we
can judge how well a given mathematical formalism supports an
interpretation of the axioms, independent of
experimentation. Likewise, we can judge how well we have captured our
physical evidence and experience with our axiomatics, independent of
any specific mathematical implementation, with accidental detail that
may or may not have physical significance. 

In lieu of a fully fleshed out and vetted axiomatic account of quantum
mechanics, interpreting the operational notions in service of modeling
physical systems will have to suffice. In other words, we are not in
the business of providing a model of Hilbert spaces and operators. We
are in the business of providing a model of quantum mechanics because
we are motivated by testing our notions of dynamics against physical
theory; and, the predictive calculations of the physical theory must
serve as the best formulation -- shy of a fully fleshed out axiomatic
account -- of the physical theory itself (as they have for scientific
theories since time immemorial). Put another way, despite a
whole-hearted commitment to an It-from-Bit ontology, we are firmly
aligned with the shut-up-and-calculate camp as the best way to obtain
results either from the physical perspective or as a quality assurance
measure of our fledgling theory of dynamics.

In detail, we present a reflective process calculus. Then we develop
intuitive correspondences between the notions available in this
calculus and the usual physical notions supporting quantum mechanical
calculations. Thus, 

\begin{table}[htp]
  \center{
    \fbox{
      \begin{tabular}{c|c}
        quantum mechanics & process calculus \\
        \hline
        scalar & name \\
        state vector & process \\
        dual & contextual duals \\
        matrix & formal sums of process-context-dual pairs \\
        orthogonality & process annihilation \\
        inner product & execution-formula + quoting
      \end{tabular}
    }
  }
  \caption{QM - process calculi correspondences}
\end{table}

Then we tighten up these intuitions to operational definitions. We
employ the Dirac notation as the best proxy we can find for an
abstract syntax of the quantum mechanical notions. The definitions we
develop put us in contact with equational constraints coming from the
theory that we demonstrate the definitions and calculations satisfy.

This puts us in a position to shut up and calculate for the
Stern-Gerlach experimental set up, showing how these predictive
calculations become calculations on processes in our theory of a
reflective process calculus.

Penultimately, we demonstrate that the notion of metric coming from
the inner product coincides with the notion of metric available from
the theory of bisimulation. This demonstration gives us the right to
think of space as arising from behavior. Finally, we consider where we
might go from the new vantage point we have obtained.

% section introduction (end) 
 
% section introduction (end)

% \documentclass[12pt]{llncs}
%\documentclass{jktr}

\usepackage[pdftex]{hyperref}                   
\usepackage {listings}
\usepackage {mathpartir}
\usepackage{bcprules}
%\usepackage{listings}
                       
\usepackage{graphicx} 
%\usepackage[margins=2.5cm,nohead,nofoot]{geometry}
%\usepackage{geometry}
\usepackage{amsfonts}
\usepackage{amstext}
\usepackage{latexsym}
\usepackage{amssymb}
\usepackage{color}


%\include{myPreamble}
\include{qm2pi.local} 

%\ifpdf
%\usepackage[pdftex]{graphicx}
%\else
%\usepackage{graphicx}
%\fi

 % \ifpdf
%  \usepackage{pdfsync}
%  \if


%\title{Brief Article}
%\author{David F. Snyder}
%\author{L.G. Meredith}

%\address{Dept. of Math., Texas State University--San Marcos, San Marcos, TX 78666}
       
\pagestyle{empty}


\begin{document}

\lstset{language=[Objective]Caml,frame=shadowbox}

\input{qm2pi.front}

% section front matter (end)

\input{qm2pi.intro} 
 
% section introduction (end)

% \input{qm2pi.knotations} 

% section notation (end)

\input{qm2pi.process.calculi} 

% section concurrent_process_calculi_and_spatial_logics_ (end)
    
%\input{qm2pi.knots2pi} 

%\input{qm2pi.trefoil} 

%\input{qm2pi.mainthm} 

% subsection basic_interpretation (end)

%\input{qm2pi.rho.presentation} 
\subsection{The syntax and semantics of the notation system}\label{sub:the_syntax_and_semantics_of_the_notation_system} % (fold)

We now summarize a technical presentation of the calculus that
embodies our theory of dynamics. The typical presentation of such a
calculus follows the style of giving generators and relations on
them. The grammar, below, describing term constructors, freely
generates the set of processes, $\Proc$. This set is then quotiented
by a relation known as structural congruence and it is over this set
that the notion of dynamics is expressed. This presentation is
essentially that of \cite{MeredithR05} with the addition of
polyadicity and summation. For readability we have relegated some of
the technical subtleties to an appendix.

\subsubsection{Process grammar}\label{subsub:process_grammar}

\begin{mathpar}
  \inferrule* [lab=synchronization] {} {{M} \bc \pzero \;|\; x?F \;|\; x!C }
  \and
  \inferrule* [lab=abstraction] {} {{F} \bc (x)P}
  \and
  \inferrule* [lab=concretion] {} {{C} \bc \langle Q \rangle}
  \and
  \inferrule* [lab=process] {} {{P,Q} \bc M \;| \;P|Q \;|\; @{x}}
  \and
  \inferrule* [lab=name] {} {{x} \bc \quotep{P}}
\end{mathpar} 

Note that $\vec{x}$ (resp. $\vec{P}$) denotes a vector of names
(resp. processes) of length $|\vec{x}|$ (resp. $|\vec{P}|$). We adopt
the following useful abbreviations.

\begin{mathpar}
   x?(\vec{y}).P := x.(\vec{y})P \and  x\clift{\vec{P}} := x.\clift{\vec{P}}
   \and x!(y) := \lift{x}{\dropn{y}}
   \and \Pi_{i=0}^{n-1}P_i := P_0 | \ldots | P_{n-1}
\end{mathpar}

\subsubsection{Structural congruence}

\paragraph{Free and bound names and alpha-equivalence.} At the
core of structural equivalence is alpha-equivalence which identifies
process that are the same up to a change of variable. Formally, we
recognize the distinction between free and bound names. The free names
of a process, $\freenames{P}$, may be calculated recursively as
follows:

\begin{mathpar}
\freenames{\pzero} := \emptyset
  \and \\
  \freenames{x?(y).P} := \{ x \} \cup (\freenames{P} \setminus \{ y \})
  \and 
  \freenames{x!\langle P \rangle} := \{ x \} \cup \{ P \} 
  \and \\
  \freenames{P|Q} := \freenames{P} \cup \freenames{Q}
  \and \\
  \freenames{@{x}} := \{ x \}
\end{mathpar}

$\pi$
$\quotep{\pi}$

$\freenames{-} : \pi \to \mathcal{P}(\quotep{\pi})$

\begin{eqnarray*}
  \freenames{\pzero} & := & \emptyset \\
  \freenames{x?(y).P} & := & \{ x \} \cup (\freenames{P} \setminus \{ y \}) \\
  \freenames{x!\langle P \rangle} & := & \{ x \} \cup \{ P \} \\
  \freenames{P|Q} & := & \freenames{P} \cup \freenames{Q} \\
  \freenames{\dropn{x}} & := & \{ x \}
\end{eqnarray*}

The bound names of a process, $\boundnames{P}$, are those names occurring in $P$
that are not free. For example, in $x?(y).0$, the name $x$ is free, while $y$ is bound.

\begin{mathpar}
  \inferrule* [lab=monoidal-laws] {} { P|Q \equiv Q|P \and P|0 \equiv P \and P|(Q|R) \equiv (P|Q)|R }
\end{mathpar}

\begin{mathpar}
  \inferrule* [lab=alpha-equivalence] {} { (x)P \equiv (y)P\{y/x\} \and y \not\in \freenames{P} }
\end{mathpar}

\begin{definition}
Then two processes, $P,Q$, are alpha-equivalent if $P = Q\{\vec{y}/\vec{x}\}$ for
some $\vec{x} \in \boundnames{Q},\vec{y} \in \boundnames{P}$, where $Q\{\vec{y}/\vec{x}\}$
denotes the capture-avoiding substitution of $\vec{y}$ for $\vec{x}$ in $Q$.
\end{definition}

\begin{definition}
  The {\em structural congruence} \cite{SangiorgiWalker} , $\equiv$,
  between processes is the least congruence containing
  alpha-equivalence, satisfying the abelian monoid laws
  (associativity, commutativity and $\pzero$ as identity) for parallel
  composition $|$ and for summation $+$.
\end{definition}

\subsection{Name equivalence}

We take name equivalence, written $\nameeq$, to be the smallest
equivalence relation generated by the following rules.

\begin{mathpar}
\inferrule*[lab=Quote-drop]
{ }
{ \quotep{@{x}} \nameeq x }

\inferrule*[lab=Struct-equiv]
{ P \scong Q }
{ \quotep{P} \nameeq \quotep{Q} }
\end{mathpar}

The astute reader will have noticed that the mutual recursion of names
and processes imposes a mutual recursion on alpha-equivalence and
structural equivalence via name-equivalence. Fortunately, all of this
works out pleasantly and we may calculate in the natural way, free of
concern. The reader interested in the details is referred to the
appendix \ref{appendix:rho_details}.

\subsection{Substitution}

We use $\Proc$ for the set of processes, $\QProc$ for the set of
names, and $\id{\{}\vec{y} / \vec{x} \id{\}}$ to denote partial maps,
$s : \QProc \rightarrow \QProc$. A map, $s$ lifts, uniquely, to a map
on process terms, $\widehat{s} : \Proc \rightarrow \Proc$ by the
following equations.

\begin{mathpar}
  (0) \psubstp{Q}{P} := 0 \\
  (R \juxtap S) \psubstp{Q}{P}
  :=    
  (R)\psubstp{Q}{P} \juxtap (S) \psubstp{Q}{P} \\
  (x?(y).R) \psubstp{Q}{P}    
  :=    
  (x)\substp{Q}{P} (z)\concat( (R \psubstn{z}{y}) \psubstp{Q}{P} ) \\
  (\lift{x}{R}) \psubstp{Q}{P}  
  :=
  \lift{(x)\substp{Q}{P}}{ R \psubstp{Q}{P} } \\
%   (\dropn{x})  \psubstp{Q}{P}       
%   := 
%   \left\{ 
%     \begin{array}{ccc} 
%       \dropn{\quotep{Q}} & & x \nameeq \quotep{P} \\
%       \dropn{x} & & otherwise \\
%     \end{array}
%   \right. 
  (\dropn{x})  \psubstp{Q}{P}       
  := 
  \left\{ 
    \begin{array}{ccc} 
      Q & & x \nameeq \quotep{P} \\
      \dropn{x} & & otherwise \\
    \end{array}
  \right.
\end{mathpar}
 

where

\begin{eqnarray}
  (x)\id{\{} \lpquote Q \rpquote / \lpquote P \rpquote \id{\}}            = 
  \left\{ 
    \begin{array}{ccc}
      \lpquote Q \rpquote & & x \nameeq \lpquote P \rpquote \\
      x & & otherwise \\
    \end{array}
  \right. \nonumber
\end{eqnarray}

and $z$ is chosen distinct from $\quotep{P}$, $\quotep{Q}$, the free
names in $Q$, and all the names in $R$. Our $\alpha$-equivalence will
be built in the standard way from this substitution.

\begin{remark}\label{rem:no_self_referential_names}
  One consequence of these definitions is that $\forall P. \quotep{P}
  \not\in \freenames{P}$.
\end{remark}

\subsection{ Dynamic quote: an example }

Anticipating something of what's to come, consider applying the
substitution, $\widehat{\id{\{}u / z \id{\}}}$, to the following pair
of processes, $\lift{w}{y!(z)}$ and $w[ \lpquote y!(z) \rpquote ]$.

\begin{eqnarray}
	\lift{w}{y!(z)}\widehat{\id{\{}u / z \id{\}}}
		& = &
		\lift{w}{y!(u)} \nonumber\\
	w[ \lpquote y!(z) \rpquote ] \widehat{ \id{\{}u / z \id{\}} }
		& = &
		w[ \lpquote y!(z) \rpquote ] \nonumber
\end{eqnarray}

Because the body of the process between quotes is impervious to
substitution, we get radically different answers. In fact, by
examining the first process in an input context,
e.g. $x?(z).\lift{w}{y!(z)}$, we see that the process under the lift
operator may be shaped by prefixed inputs binding a name inside it. In
this sense, the lift operator will be seen as a way to dynamically
construct processes before reifying them as names.

Finally equipped with these standard features we can present the
dynamics of the calculus.

\subsubsection{Operational semantics} 

Finally, we introduce the computational dynamics. What marks these
algebras as distinct from other more traditionally studied algebraic
structures, e.g. vector spaces or polynomial rings, is the manner in
which dynamics is captured. In traditional structures, dynamics is typically
expressed through morphisms between such structures, as in linear maps
between vector spaces or morphisms between rings. In algebras
associated with the semantics of computation, the dynamics is
expressed as part of the algebraic structure itself, through a
reduction reduction relation typically denoted by $\red$. Below, we
give a recursive presentation of this relation for the calculus used
in the encoding.

$\red \subseteq \pi \times \pi$
$\red : \pi \to \mathcal{P}(\pi)$

\begin{mathpar}
  \inferrule* [lab=Comm] { \textsf{match}( x_{src}, x_{trgt} ) } { x_{trgt}?(y)P \; | \; x_{src}!\langle {Q} \rangle \red P\{\quotep{Q}/y}\} }
  \and \\
  \inferrule* [lab=Par] {{P} \red {P}'} {{{P} | {Q}} \red {{P}' | {Q}}}
  \and
  \inferrule* [lab=Equiv]{{{P} \scong {P}'} \andalso {{P}' \red {Q}'} \andalso {{Q}' \scong {Q}}}{{P} \red {Q}}
\end{mathpar}

\begin{eqnarray*}
  match_{\equiv} (\quotep{P},\quotep{Q}) & := & P \equiv Q \\
  match_{\dagger}(\quotep{P},\quotep{Q}) & := & \forall R. P|Q \red^{*} R => R \red^{*} 0 \\
  match_{K}(\quotep{P},\quotep{Q}) & := & K \mbox{ for some context } K
\end{eqnarray*}

$u?(x)P | u!\langle Q \rangle \red P\{\quotep{Q}/x\}$

%We write $\wred$ for $\red^*$, and $P\red$ if $\exists Q $ such that $ P \red Q$.
We write $P\red$ if $\exists Q $ such that $ P \red Q$ and $P\not\red$, otherwise.

\section{Replication}

As mentioned before, it is known that replication (and hence
recursion) can be implemented in a higher-order process algebra
\cite{SangiorgiWalker}. As our first example of calculation with the
machinery thus far presented we give the construction explicitly in
the {\rhoc}.

\begin{eqnarray}
	D_{x} & := & \prefix{x}{y}{(\binpar{\outputp{x}{y}}{@{y}})} \nonumber\\
	\bangp_{x}{P} & := & \binpar{{x}!\langle{\binpar{D_{x}}{P}}\rangle}{D_{x}} \nonumber
\end{eqnarray}

\begin{eqnarray}
	\bangp_{x}{P} & & \nonumber\\
	=
	& {x}!\langle{(\prefix{x}{y}{(\outputp{x}{y} | @{y})) | P}}\rangle 
	      | \prefix{x}{y}{(\outputp{x}{y} | @{y})} & \nonumber\\
	\red
	& (\outputp{x}{y} | @{y})\substn{\quotep{(\prefix{x}{y}{(@{y} | \outputp{x}{y})) | P}}}{y} & \nonumber\\
	=
	& \outputp{x}{\quotep{(\prefix{x}{y}{(\outputp{x}{y} | @{y})) | P}}}
	  | {(\prefix{x}{y}{(\outputp{x}{y} | @{y})) | P}} & \nonumber\\
	\red
	& \ldots & \nonumber\\
	\red^*
	& P | P | \ldots & \nonumber
\end{eqnarray}

Of course, this encoding, as an implementation, runs away, unfolding
$\bangp{P}$ eagerly. A lazier and more implementable replication
operator, restricted to input-guarded processes, may be obtained as follows.

\begin{eqnarray}
\bangp{\prefix{u}{v}{P}} 
	:= 
	\binpar{\lift{x}{\prefix{u}{v}{(\binpar{D(x)}{P})}}}{D(x)} \nonumber
\end{eqnarray}

\begin{remark}
  Note that the lazier definition still does not deal with summation
  or mixed summation (i.e. sums over input and output). The reader is
  invited to construct definitions of replication that deal with these
  features. 

  Further, the definitions are parameterized in a name, $x$. Can you,
  gentle reader, make a definition that eliminates this parameter and
  guarantees no accidental interaction between the replication
  machinery and the process being replicated -- i.e. no accidental
  sharing of names used by the process to get its work done and the
  name(s) used by the replication to effect copying. This latter
  revision of the definition of replication is crucial to obtaining
  the expected identity $!!P \sim !P$.
\end{remark}

\begin{remark}\label{rem:paradoxical_combinator}
  The reader familiar with the lambda calculus will have noticed the
  similarity between $D$ and the paradoxical combinator.

  [Ed. note: the existence of this seems to suggest we have to be more
  restrictive on the set of processes and names we admit if we are to
  support no-cloning.]
\end{remark}

\subsubsection{Bisimulation}

The computational dynamics gives rise to another kind of equivalence,
the equivalence of computational behavior. As previously mentioned
this is typically captured \emph{via} some form of bisimulation.

% The notion we use in this paper is weak barbed bisimulation
% \cite{milner91polyadicpi}.

The notion we use in this paper is derived from weak barbed
bisimulation \cite{milner91polyadicpi}. 

\begin{definition}
An \emph{observation relation}, $\downarrow_{\mathcal N}$, over a set
of names, $\mathcal N$, is the smallest relation satisfying the rules
below.

\infrule[Out-barb]{y \in {\mathcal N}, \; x \nameeq y}
		  {\outputp{x}{v} \downarrow_{\mathcal N} x}
\infrule[Par-barb]{\mbox{$P\downarrow_{\mathcal N} x$ or $Q\downarrow_{\mathcal N} x$}}
		  {\binpar{P}{Q} \downarrow_{\mathcal N} x}

We write $P \Downarrow_{\mathcal N} x$ if there is $Q$ such that 
$P \wred Q$ and $Q \downarrow_{\mathcal N} x$.
\end{definition}

\begin{definition}
%\label{def.bbisim}
An  ${\mathcal N}$-\emph{barbed bisimulation} over a set of names, ${\mathcal N}$, is a symmetric binary relation 
${\mathcal S}_{\mathcal N}$ between agents such that $P\rel{S}_{\mathcal N}Q$ implies:
\begin{enumerate}
\item If $P \red P'$ then $Q \wred Q'$ and $P'\rel{S}_{\mathcal N} Q'$.
\item If $P\downarrow_{\mathcal N} x$, then $Q\Downarrow_{\mathcal N} x$.
\end{enumerate}
$P$ is ${\mathcal N}$-barbed bisimilar to $Q$, written
$P \wbbisim_{\mathcal N} Q$, if $P \rel{S}_{\mathcal N} Q$ for some ${\mathcal N}$-barbed bisimulation ${\mathcal S}_{\mathcal N}$.
\end{definition}

$\mathcal{R} \subseteq \pi \times \pi$

$P \mathcal{R} Q => \forall P'. P \red P' \Rightarrow \exists Q'. Q \red Q', P' \mathcal{R} Q'$

$P \vdash x \Rightarrow Q \vdash x$

\begin{mathpar}
  \inferrule*[lab=Out-barb]{x \nameeq y}{{y}!\langle{Q}\rangle \vdash x}
  \and
  \inferrule*[lab=Par-barb]{\mbox{$P\vdash x$ or $Q\vdash x$}}{\binpar{P}{Q} \vdash x}
\end{mathpar}

\subsubsection{Contexts}

One of the principle advantages of computational calculi like the
$\pi$-calculus is a well-defined notion of context,
contextual-equivalence and a correlation between
contextual-equivalence and notions of bisimulation. The notion of
context allows the decomposition of a process into (sub-)process and
its syntactic environment, its context. Thus, a context may be
thought of as a process with a ``hole'' (written $\Box$) in it. The
application of a context $M$ to a process $P$, written $M[P]$, is
tantamount to filling the hole in $M$ with $P$. In this paper we do
not need the full weight of this theory, but do make use of the notion
of context in the proof the main theorem. 

\begin{mathpar}
  \inferrule* [lab=summation] {} {{M_{M},M_{N}} \bc \Box \;|\; x.M_{A} \;|\; M_{M}+M_{N}}
  \and
  \inferrule* [lab=agent] {} {{M_{A}} \bc (\vec{x})M_{P} \;| \; \clift{P_0,\ldots,M_{P},\ldots,P_N}}
  \and \\
  \inferrule* [lab=process] {} {{M_{P}} \bc M_{N} \;| \;P|M_{P} }
\end{mathpar} 

\begin{mathpar}
  \inferrule* [lab=sychronization] {} {M_{N} \bc \Box \;|\; x?M_{F} \;|\; x!M_{C}}
  \and
  \inferrule* [lab=abstraction] {} {{M_{F}} \bc (x)M_{P} }
  \and
  \inferrule* [lab=concretion] {} {{M_{C}} \bc \langle M_{P} \rangle }
  \and \\
  \inferrule* [lab=process] {} {{M_{P}} \bc M_{N} \;| \;P|M_{P} }
\end{mathpar}

\begin{definition}[contextual application] Given a context $M$, and
  process $P$, we define the \emph{contextual application}, $M[P] :=
  M\{P/\Box\}$. That is, the contextual application of M to P is the
  substitution of $P$ for $\Box$ in $M$.
\end{definition}

$\meaningof{-} : L \to \mathcal{P}(\pi)$

\begin{mathpar}
  \inferrule* [lab=collection] {} {\meaningof{true} = \pi, \and \meaningof{~E} = \pi \setminus \meaningof{E}, \and \meaningof{E_{1} \& E_{2}} = \meaningof{E_{1}} \cap \meaningof{E_{2}}}
\end{mathpar}

\begin{mathpar}
  \inferrule* [lab=structure] {} {\meaningof{0} = \{ P \in \pi | P \equiv 0 \}, \and \\ \meaningof{E_1 | E_2} = \{ P \in \pi | P \equiv P_{1} | P_{2}, P_{1} \in \meaningof{E_{1}}, P_{2} \in \meaningof{E_2}\} }
\end{mathpar}

\begin{mathpar}
 \inferrule* [lab=behavior] {} {\meaningof{\langle a?b \rangle E} = \{ P \in \pi | P \equiv Q | u?(y)P', \\ \and \\\\ \and \\ \;\;\; u \in \meaningof{a}, \forall z.P'\{z/y\} \in \meaningof{E\{z/b\}}\}, \and \\ \meaningof{a!E} = \{ P \in \pi | P \equiv Q | x!\langle P' \rangle, x \in \meaningof{a} P' \in \meaningof{E}\} }
\end{mathpar}

\begin{mathpar}
 \inferrule* [lab=nominal] {} {\meaningof{\quotep{E}} = \{ \quotep{P} \in \quotep{\pi} | P \in \meaningof{E} \}, \and \meaningof{\quotep{P}} = \{ \quotep{Q} \in \quotep{\pi} | P \equiv Q \} \and \\ \meaningof{@\quotep{E}} = \{ P \in \pi | P \equiv @x, x \in \meaningof{E} \}}
\end{mathpar}

\begin{eqnarray*}
  \\
  \meaningof{-} : TS \to ST
\end{eqnarray*}

\begin{eqnarray*}
  \\
  L : TS \to ST
\end{eqnarray*}

\begin{eqnarray*}
  \\
  P \models E \iff P \in \meaningof{E}
\end{eqnarray*}

\begin{eqnarray*}
  P \approx_{L} Q \iff \forall E \in L. P \models E \iff Q \models E
\end{eqnarray*}

\begin{eqnarray*}
  P \approx_{K} Q
\end{eqnarray*}

\begin{eqnarray*}
  P \approx Q
\end{eqnarray*}

$\approx_{K} = \approx = \approx_{L}$

\subsubsection{Contextual duality}

Note that contexts extend the quotation operation to a family of
operations from processes to names. Given a context, $M$, we can
define a \emph{nominal context}, $\quotep{M}$ by $\quotep{M}[P] :=
\quotep{M[P]}$. To foreshadow what is to come we observe that these
operations enjoy a duality with processes very much like the duality
between vectors and maps from vectors to scalars.

Further, because the calculus is essentially higher-order, we have a
correspondence between contexts and processes. More specifically,
given a name $x$ and a context $M$ we can construct $M^{*}_{x}$ such
that 

\begin{mathpar}
  M^{*}_{x} | \lift{x}{P} \red M[P]
\end{mathpar}

namely,

\begin{mathpar}
  M^{*}_{x} := x?(u).M[\dropn{u}]
\end{mathpar}

The dependence of $M^{*}_{x}$ on a name makes it an abstraction, 

\begin{mathpar}
  M^{*} := (x)x?(u).M[\dropn{u}]
\end{mathpar}

\subsection{Additional notation}

It will sometimes be convenient to denote the process a name
quotes. We already have the notation $x = \quotep{P}$, but it will be
convenient to introduce an alternate notation, $\procn{x}$, when we
want to emphasize the connection to the use of the name. Note that, by
virtue of name equivalence, $\quotep{\procn{x}} \nameeq x$; so, the
notation is consistent with previous definitions.

Further, because names have structure it is possible to effect
substitutions on the basis of that structure. This means we need to
upgrade our notation for substitutions, which we accomplish by
adapting comprehension notation. Thus,

\begin{mathpar}
  P\{ y / x : x \in S \}
\end{mathpar}

is interpreted to mean the process derived from P by replacing (in a
capture-avoiding manner) each occurrence of $x$ in $S$ by $y$. For example,

\begin{mathpar}
  P\{ \quotep{\procn{x}|\procn{x}} / x : x \in \freenames{P} \}
\end{mathpar}

will replace each (occurrence) of a free name $x$ in $P$ by
$\quotep{\procn{x}|\procn{x}}$.

Also, we will avail ourselves of the notation $x^{L}$ and $x^{R}$ to
denote injections of a name into disjoint copies of the name
space. There are numerous ways to accomplish this. One example can be
found in \cite{MeredithR05}. This notation overloads to vectors of
names: $\vec{x}^{\pi} := (x_{i}^{\pi} \; : \; 0 \leq i < |\vec{x}| )$ where $\pi \in \{L,R\}$.

We also use $P^{\Box} := P|\Box$.

In \cite{MeredithR05} an interpretation of the new operator is
given. It turns out that there are several possible interpretations
all enjoying the requisite algebraic properties of the operator (see
\cite{milner91polyadicpi}). We will therefore make liberal use of
$(\nu\; \vec{x})P$.

% subsection the_syntax_and_semantics_of_the_notation_system (end)   

\input{qm2pi.qmops} 

\input{qm2pi.sterngerlach} 

\input{qm2pi.metric} 

% section concurrent_process_calculi (end)

%\input{qm2pi.proofsketch}

% section proof sketch (end)

%\input{qm2pi.slviaknots} 

% section spatial logic via knots (end)

\input{qm2pi.conclusion}

% section conclusion (end)

%\input{qm2pi.dtcodes} 

% section wiring algorithm (end)

\input{qm2pi.ack} 

% section acknowledgments (end)

\newpage


\bibliographystyle{plain}   
\bibliography{../../biblios/main.bib}

\input{qm2pi.rhodetails}

\end{document}

 

% section notation (end)

\input{qm2pi.process.calculi} 

% section concurrent_process_calculi_and_spatial_logics_ (end)
    
%\documentclass[12pt]{llncs}
%\documentclass{jktr}

\usepackage[pdftex]{hyperref}                   
\usepackage {listings}
\usepackage {mathpartir}
\usepackage{bcprules}
%\usepackage{listings}
                       
\usepackage{graphicx} 
%\usepackage[margins=2.5cm,nohead,nofoot]{geometry}
%\usepackage{geometry}
\usepackage{amsfonts}
\usepackage{amstext}
\usepackage{latexsym}
\usepackage{amssymb}
\usepackage{color}


%\include{myPreamble}
\include{qm2pi.local} 

%\ifpdf
%\usepackage[pdftex]{graphicx}
%\else
%\usepackage{graphicx}
%\fi

 % \ifpdf
%  \usepackage{pdfsync}
%  \if


%\title{Brief Article}
%\author{David F. Snyder}
%\author{L.G. Meredith}

%\address{Dept. of Math., Texas State University--San Marcos, San Marcos, TX 78666}
       
\pagestyle{empty}


\begin{document}

\lstset{language=[Objective]Caml,frame=shadowbox}

\input{qm2pi.front}

% section front matter (end)

\input{qm2pi.intro} 
 
% section introduction (end)

% \input{qm2pi.knotations} 

% section notation (end)

\input{qm2pi.process.calculi} 

% section concurrent_process_calculi_and_spatial_logics_ (end)
    
%\input{qm2pi.knots2pi} 

%\input{qm2pi.trefoil} 

%\input{qm2pi.mainthm} 

% subsection basic_interpretation (end)

%\input{qm2pi.rho.presentation} 
\subsection{The syntax and semantics of the notation system}\label{sub:the_syntax_and_semantics_of_the_notation_system} % (fold)

We now summarize a technical presentation of the calculus that
embodies our theory of dynamics. The typical presentation of such a
calculus follows the style of giving generators and relations on
them. The grammar, below, describing term constructors, freely
generates the set of processes, $\Proc$. This set is then quotiented
by a relation known as structural congruence and it is over this set
that the notion of dynamics is expressed. This presentation is
essentially that of \cite{MeredithR05} with the addition of
polyadicity and summation. For readability we have relegated some of
the technical subtleties to an appendix.

\subsubsection{Process grammar}\label{subsub:process_grammar}

\begin{mathpar}
  \inferrule* [lab=synchronization] {} {{M} \bc \pzero \;|\; x?F \;|\; x!C }
  \and
  \inferrule* [lab=abstraction] {} {{F} \bc (x)P}
  \and
  \inferrule* [lab=concretion] {} {{C} \bc \langle Q \rangle}
  \and
  \inferrule* [lab=process] {} {{P,Q} \bc M \;| \;P|Q \;|\; @{x}}
  \and
  \inferrule* [lab=name] {} {{x} \bc \quotep{P}}
\end{mathpar} 

Note that $\vec{x}$ (resp. $\vec{P}$) denotes a vector of names
(resp. processes) of length $|\vec{x}|$ (resp. $|\vec{P}|$). We adopt
the following useful abbreviations.

\begin{mathpar}
   x?(\vec{y}).P := x.(\vec{y})P \and  x\clift{\vec{P}} := x.\clift{\vec{P}}
   \and x!(y) := \lift{x}{\dropn{y}}
   \and \Pi_{i=0}^{n-1}P_i := P_0 | \ldots | P_{n-1}
\end{mathpar}

\subsubsection{Structural congruence}

\paragraph{Free and bound names and alpha-equivalence.} At the
core of structural equivalence is alpha-equivalence which identifies
process that are the same up to a change of variable. Formally, we
recognize the distinction between free and bound names. The free names
of a process, $\freenames{P}$, may be calculated recursively as
follows:

\begin{mathpar}
\freenames{\pzero} := \emptyset
  \and \\
  \freenames{x?(y).P} := \{ x \} \cup (\freenames{P} \setminus \{ y \})
  \and 
  \freenames{x!\langle P \rangle} := \{ x \} \cup \{ P \} 
  \and \\
  \freenames{P|Q} := \freenames{P} \cup \freenames{Q}
  \and \\
  \freenames{@{x}} := \{ x \}
\end{mathpar}

$\pi$
$\quotep{\pi}$

$\freenames{-} : \pi \to \mathcal{P}(\quotep{\pi})$

\begin{eqnarray*}
  \freenames{\pzero} & := & \emptyset \\
  \freenames{x?(y).P} & := & \{ x \} \cup (\freenames{P} \setminus \{ y \}) \\
  \freenames{x!\langle P \rangle} & := & \{ x \} \cup \{ P \} \\
  \freenames{P|Q} & := & \freenames{P} \cup \freenames{Q} \\
  \freenames{\dropn{x}} & := & \{ x \}
\end{eqnarray*}

The bound names of a process, $\boundnames{P}$, are those names occurring in $P$
that are not free. For example, in $x?(y).0$, the name $x$ is free, while $y$ is bound.

\begin{mathpar}
  \inferrule* [lab=monoidal-laws] {} { P|Q \equiv Q|P \and P|0 \equiv P \and P|(Q|R) \equiv (P|Q)|R }
\end{mathpar}

\begin{mathpar}
  \inferrule* [lab=alpha-equivalence] {} { (x)P \equiv (y)P\{y/x\} \and y \not\in \freenames{P} }
\end{mathpar}

\begin{definition}
Then two processes, $P,Q$, are alpha-equivalent if $P = Q\{\vec{y}/\vec{x}\}$ for
some $\vec{x} \in \boundnames{Q},\vec{y} \in \boundnames{P}$, where $Q\{\vec{y}/\vec{x}\}$
denotes the capture-avoiding substitution of $\vec{y}$ for $\vec{x}$ in $Q$.
\end{definition}

\begin{definition}
  The {\em structural congruence} \cite{SangiorgiWalker} , $\equiv$,
  between processes is the least congruence containing
  alpha-equivalence, satisfying the abelian monoid laws
  (associativity, commutativity and $\pzero$ as identity) for parallel
  composition $|$ and for summation $+$.
\end{definition}

\subsection{Name equivalence}

We take name equivalence, written $\nameeq$, to be the smallest
equivalence relation generated by the following rules.

\begin{mathpar}
\inferrule*[lab=Quote-drop]
{ }
{ \quotep{@{x}} \nameeq x }

\inferrule*[lab=Struct-equiv]
{ P \scong Q }
{ \quotep{P} \nameeq \quotep{Q} }
\end{mathpar}

The astute reader will have noticed that the mutual recursion of names
and processes imposes a mutual recursion on alpha-equivalence and
structural equivalence via name-equivalence. Fortunately, all of this
works out pleasantly and we may calculate in the natural way, free of
concern. The reader interested in the details is referred to the
appendix \ref{appendix:rho_details}.

\subsection{Substitution}

We use $\Proc$ for the set of processes, $\QProc$ for the set of
names, and $\id{\{}\vec{y} / \vec{x} \id{\}}$ to denote partial maps,
$s : \QProc \rightarrow \QProc$. A map, $s$ lifts, uniquely, to a map
on process terms, $\widehat{s} : \Proc \rightarrow \Proc$ by the
following equations.

\begin{mathpar}
  (0) \psubstp{Q}{P} := 0 \\
  (R \juxtap S) \psubstp{Q}{P}
  :=    
  (R)\psubstp{Q}{P} \juxtap (S) \psubstp{Q}{P} \\
  (x?(y).R) \psubstp{Q}{P}    
  :=    
  (x)\substp{Q}{P} (z)\concat( (R \psubstn{z}{y}) \psubstp{Q}{P} ) \\
  (\lift{x}{R}) \psubstp{Q}{P}  
  :=
  \lift{(x)\substp{Q}{P}}{ R \psubstp{Q}{P} } \\
%   (\dropn{x})  \psubstp{Q}{P}       
%   := 
%   \left\{ 
%     \begin{array}{ccc} 
%       \dropn{\quotep{Q}} & & x \nameeq \quotep{P} \\
%       \dropn{x} & & otherwise \\
%     \end{array}
%   \right. 
  (\dropn{x})  \psubstp{Q}{P}       
  := 
  \left\{ 
    \begin{array}{ccc} 
      Q & & x \nameeq \quotep{P} \\
      \dropn{x} & & otherwise \\
    \end{array}
  \right.
\end{mathpar}
 

where

\begin{eqnarray}
  (x)\id{\{} \lpquote Q \rpquote / \lpquote P \rpquote \id{\}}            = 
  \left\{ 
    \begin{array}{ccc}
      \lpquote Q \rpquote & & x \nameeq \lpquote P \rpquote \\
      x & & otherwise \\
    \end{array}
  \right. \nonumber
\end{eqnarray}

and $z$ is chosen distinct from $\quotep{P}$, $\quotep{Q}$, the free
names in $Q$, and all the names in $R$. Our $\alpha$-equivalence will
be built in the standard way from this substitution.

\begin{remark}\label{rem:no_self_referential_names}
  One consequence of these definitions is that $\forall P. \quotep{P}
  \not\in \freenames{P}$.
\end{remark}

\subsection{ Dynamic quote: an example }

Anticipating something of what's to come, consider applying the
substitution, $\widehat{\id{\{}u / z \id{\}}}$, to the following pair
of processes, $\lift{w}{y!(z)}$ and $w[ \lpquote y!(z) \rpquote ]$.

\begin{eqnarray}
	\lift{w}{y!(z)}\widehat{\id{\{}u / z \id{\}}}
		& = &
		\lift{w}{y!(u)} \nonumber\\
	w[ \lpquote y!(z) \rpquote ] \widehat{ \id{\{}u / z \id{\}} }
		& = &
		w[ \lpquote y!(z) \rpquote ] \nonumber
\end{eqnarray}

Because the body of the process between quotes is impervious to
substitution, we get radically different answers. In fact, by
examining the first process in an input context,
e.g. $x?(z).\lift{w}{y!(z)}$, we see that the process under the lift
operator may be shaped by prefixed inputs binding a name inside it. In
this sense, the lift operator will be seen as a way to dynamically
construct processes before reifying them as names.

Finally equipped with these standard features we can present the
dynamics of the calculus.

\subsubsection{Operational semantics} 

Finally, we introduce the computational dynamics. What marks these
algebras as distinct from other more traditionally studied algebraic
structures, e.g. vector spaces or polynomial rings, is the manner in
which dynamics is captured. In traditional structures, dynamics is typically
expressed through morphisms between such structures, as in linear maps
between vector spaces or morphisms between rings. In algebras
associated with the semantics of computation, the dynamics is
expressed as part of the algebraic structure itself, through a
reduction reduction relation typically denoted by $\red$. Below, we
give a recursive presentation of this relation for the calculus used
in the encoding.

$\red \subseteq \pi \times \pi$
$\red : \pi \to \mathcal{P}(\pi)$

\begin{mathpar}
  \inferrule* [lab=Comm] { \textsf{match}( x_{src}, x_{trgt} ) } { x_{trgt}?(y)P \; | \; x_{src}!\langle {Q} \rangle \red P\{\quotep{Q}/y}\} }
  \and \\
  \inferrule* [lab=Par] {{P} \red {P}'} {{{P} | {Q}} \red {{P}' | {Q}}}
  \and
  \inferrule* [lab=Equiv]{{{P} \scong {P}'} \andalso {{P}' \red {Q}'} \andalso {{Q}' \scong {Q}}}{{P} \red {Q}}
\end{mathpar}

\begin{eqnarray*}
  match_{\equiv} (\quotep{P},\quotep{Q}) & := & P \equiv Q \\
  match_{\dagger}(\quotep{P},\quotep{Q}) & := & \forall R. P|Q \red^{*} R => R \red^{*} 0 \\
  match_{K}(\quotep{P},\quotep{Q}) & := & K \mbox{ for some context } K
\end{eqnarray*}

$u?(x)P | u!\langle Q \rangle \red P\{\quotep{Q}/x\}$

%We write $\wred$ for $\red^*$, and $P\red$ if $\exists Q $ such that $ P \red Q$.
We write $P\red$ if $\exists Q $ such that $ P \red Q$ and $P\not\red$, otherwise.

\section{Replication}

As mentioned before, it is known that replication (and hence
recursion) can be implemented in a higher-order process algebra
\cite{SangiorgiWalker}. As our first example of calculation with the
machinery thus far presented we give the construction explicitly in
the {\rhoc}.

\begin{eqnarray}
	D_{x} & := & \prefix{x}{y}{(\binpar{\outputp{x}{y}}{@{y}})} \nonumber\\
	\bangp_{x}{P} & := & \binpar{{x}!\langle{\binpar{D_{x}}{P}}\rangle}{D_{x}} \nonumber
\end{eqnarray}

\begin{eqnarray}
	\bangp_{x}{P} & & \nonumber\\
	=
	& {x}!\langle{(\prefix{x}{y}{(\outputp{x}{y} | @{y})) | P}}\rangle 
	      | \prefix{x}{y}{(\outputp{x}{y} | @{y})} & \nonumber\\
	\red
	& (\outputp{x}{y} | @{y})\substn{\quotep{(\prefix{x}{y}{(@{y} | \outputp{x}{y})) | P}}}{y} & \nonumber\\
	=
	& \outputp{x}{\quotep{(\prefix{x}{y}{(\outputp{x}{y} | @{y})) | P}}}
	  | {(\prefix{x}{y}{(\outputp{x}{y} | @{y})) | P}} & \nonumber\\
	\red
	& \ldots & \nonumber\\
	\red^*
	& P | P | \ldots & \nonumber
\end{eqnarray}

Of course, this encoding, as an implementation, runs away, unfolding
$\bangp{P}$ eagerly. A lazier and more implementable replication
operator, restricted to input-guarded processes, may be obtained as follows.

\begin{eqnarray}
\bangp{\prefix{u}{v}{P}} 
	:= 
	\binpar{\lift{x}{\prefix{u}{v}{(\binpar{D(x)}{P})}}}{D(x)} \nonumber
\end{eqnarray}

\begin{remark}
  Note that the lazier definition still does not deal with summation
  or mixed summation (i.e. sums over input and output). The reader is
  invited to construct definitions of replication that deal with these
  features. 

  Further, the definitions are parameterized in a name, $x$. Can you,
  gentle reader, make a definition that eliminates this parameter and
  guarantees no accidental interaction between the replication
  machinery and the process being replicated -- i.e. no accidental
  sharing of names used by the process to get its work done and the
  name(s) used by the replication to effect copying. This latter
  revision of the definition of replication is crucial to obtaining
  the expected identity $!!P \sim !P$.
\end{remark}

\begin{remark}\label{rem:paradoxical_combinator}
  The reader familiar with the lambda calculus will have noticed the
  similarity between $D$ and the paradoxical combinator.

  [Ed. note: the existence of this seems to suggest we have to be more
  restrictive on the set of processes and names we admit if we are to
  support no-cloning.]
\end{remark}

\subsubsection{Bisimulation}

The computational dynamics gives rise to another kind of equivalence,
the equivalence of computational behavior. As previously mentioned
this is typically captured \emph{via} some form of bisimulation.

% The notion we use in this paper is weak barbed bisimulation
% \cite{milner91polyadicpi}.

The notion we use in this paper is derived from weak barbed
bisimulation \cite{milner91polyadicpi}. 

\begin{definition}
An \emph{observation relation}, $\downarrow_{\mathcal N}$, over a set
of names, $\mathcal N$, is the smallest relation satisfying the rules
below.

\infrule[Out-barb]{y \in {\mathcal N}, \; x \nameeq y}
		  {\outputp{x}{v} \downarrow_{\mathcal N} x}
\infrule[Par-barb]{\mbox{$P\downarrow_{\mathcal N} x$ or $Q\downarrow_{\mathcal N} x$}}
		  {\binpar{P}{Q} \downarrow_{\mathcal N} x}

We write $P \Downarrow_{\mathcal N} x$ if there is $Q$ such that 
$P \wred Q$ and $Q \downarrow_{\mathcal N} x$.
\end{definition}

\begin{definition}
%\label{def.bbisim}
An  ${\mathcal N}$-\emph{barbed bisimulation} over a set of names, ${\mathcal N}$, is a symmetric binary relation 
${\mathcal S}_{\mathcal N}$ between agents such that $P\rel{S}_{\mathcal N}Q$ implies:
\begin{enumerate}
\item If $P \red P'$ then $Q \wred Q'$ and $P'\rel{S}_{\mathcal N} Q'$.
\item If $P\downarrow_{\mathcal N} x$, then $Q\Downarrow_{\mathcal N} x$.
\end{enumerate}
$P$ is ${\mathcal N}$-barbed bisimilar to $Q$, written
$P \wbbisim_{\mathcal N} Q$, if $P \rel{S}_{\mathcal N} Q$ for some ${\mathcal N}$-barbed bisimulation ${\mathcal S}_{\mathcal N}$.
\end{definition}

$\mathcal{R} \subseteq \pi \times \pi$

$P \mathcal{R} Q => \forall P'. P \red P' \Rightarrow \exists Q'. Q \red Q', P' \mathcal{R} Q'$

$P \vdash x \Rightarrow Q \vdash x$

\begin{mathpar}
  \inferrule*[lab=Out-barb]{x \nameeq y}{{y}!\langle{Q}\rangle \vdash x}
  \and
  \inferrule*[lab=Par-barb]{\mbox{$P\vdash x$ or $Q\vdash x$}}{\binpar{P}{Q} \vdash x}
\end{mathpar}

\subsubsection{Contexts}

One of the principle advantages of computational calculi like the
$\pi$-calculus is a well-defined notion of context,
contextual-equivalence and a correlation between
contextual-equivalence and notions of bisimulation. The notion of
context allows the decomposition of a process into (sub-)process and
its syntactic environment, its context. Thus, a context may be
thought of as a process with a ``hole'' (written $\Box$) in it. The
application of a context $M$ to a process $P$, written $M[P]$, is
tantamount to filling the hole in $M$ with $P$. In this paper we do
not need the full weight of this theory, but do make use of the notion
of context in the proof the main theorem. 

\begin{mathpar}
  \inferrule* [lab=summation] {} {{M_{M},M_{N}} \bc \Box \;|\; x.M_{A} \;|\; M_{M}+M_{N}}
  \and
  \inferrule* [lab=agent] {} {{M_{A}} \bc (\vec{x})M_{P} \;| \; \clift{P_0,\ldots,M_{P},\ldots,P_N}}
  \and \\
  \inferrule* [lab=process] {} {{M_{P}} \bc M_{N} \;| \;P|M_{P} }
\end{mathpar} 

\begin{mathpar}
  \inferrule* [lab=sychronization] {} {M_{N} \bc \Box \;|\; x?M_{F} \;|\; x!M_{C}}
  \and
  \inferrule* [lab=abstraction] {} {{M_{F}} \bc (x)M_{P} }
  \and
  \inferrule* [lab=concretion] {} {{M_{C}} \bc \langle M_{P} \rangle }
  \and \\
  \inferrule* [lab=process] {} {{M_{P}} \bc M_{N} \;| \;P|M_{P} }
\end{mathpar}

\begin{definition}[contextual application] Given a context $M$, and
  process $P$, we define the \emph{contextual application}, $M[P] :=
  M\{P/\Box\}$. That is, the contextual application of M to P is the
  substitution of $P$ for $\Box$ in $M$.
\end{definition}

$\meaningof{-} : L \to \mathcal{P}(\pi)$

\begin{mathpar}
  \inferrule* [lab=collection] {} {\meaningof{true} = \pi, \and \meaningof{~E} = \pi \setminus \meaningof{E}, \and \meaningof{E_{1} \& E_{2}} = \meaningof{E_{1}} \cap \meaningof{E_{2}}}
\end{mathpar}

\begin{mathpar}
  \inferrule* [lab=structure] {} {\meaningof{0} = \{ P \in \pi | P \equiv 0 \}, \and \\ \meaningof{E_1 | E_2} = \{ P \in \pi | P \equiv P_{1} | P_{2}, P_{1} \in \meaningof{E_{1}}, P_{2} \in \meaningof{E_2}\} }
\end{mathpar}

\begin{mathpar}
 \inferrule* [lab=behavior] {} {\meaningof{\langle a?b \rangle E} = \{ P \in \pi | P \equiv Q | u?(y)P', \\ \and \\\\ \and \\ \;\;\; u \in \meaningof{a}, \forall z.P'\{z/y\} \in \meaningof{E\{z/b\}}\}, \and \\ \meaningof{a!E} = \{ P \in \pi | P \equiv Q | x!\langle P' \rangle, x \in \meaningof{a} P' \in \meaningof{E}\} }
\end{mathpar}

\begin{mathpar}
 \inferrule* [lab=nominal] {} {\meaningof{\quotep{E}} = \{ \quotep{P} \in \quotep{\pi} | P \in \meaningof{E} \}, \and \meaningof{\quotep{P}} = \{ \quotep{Q} \in \quotep{\pi} | P \equiv Q \} \and \\ \meaningof{@\quotep{E}} = \{ P \in \pi | P \equiv @x, x \in \meaningof{E} \}}
\end{mathpar}

\begin{eqnarray*}
  \\
  \meaningof{-} : TS \to ST
\end{eqnarray*}

\begin{eqnarray*}
  \\
  L : TS \to ST
\end{eqnarray*}

\begin{eqnarray*}
  \\
  P \models E \iff P \in \meaningof{E}
\end{eqnarray*}

\begin{eqnarray*}
  P \approx_{L} Q \iff \forall E \in L. P \models E \iff Q \models E
\end{eqnarray*}

\begin{eqnarray*}
  P \approx_{K} Q
\end{eqnarray*}

\begin{eqnarray*}
  P \approx Q
\end{eqnarray*}

$\approx_{K} = \approx = \approx_{L}$

\subsubsection{Contextual duality}

Note that contexts extend the quotation operation to a family of
operations from processes to names. Given a context, $M$, we can
define a \emph{nominal context}, $\quotep{M}$ by $\quotep{M}[P] :=
\quotep{M[P]}$. To foreshadow what is to come we observe that these
operations enjoy a duality with processes very much like the duality
between vectors and maps from vectors to scalars.

Further, because the calculus is essentially higher-order, we have a
correspondence between contexts and processes. More specifically,
given a name $x$ and a context $M$ we can construct $M^{*}_{x}$ such
that 

\begin{mathpar}
  M^{*}_{x} | \lift{x}{P} \red M[P]
\end{mathpar}

namely,

\begin{mathpar}
  M^{*}_{x} := x?(u).M[\dropn{u}]
\end{mathpar}

The dependence of $M^{*}_{x}$ on a name makes it an abstraction, 

\begin{mathpar}
  M^{*} := (x)x?(u).M[\dropn{u}]
\end{mathpar}

\subsection{Additional notation}

It will sometimes be convenient to denote the process a name
quotes. We already have the notation $x = \quotep{P}$, but it will be
convenient to introduce an alternate notation, $\procn{x}$, when we
want to emphasize the connection to the use of the name. Note that, by
virtue of name equivalence, $\quotep{\procn{x}} \nameeq x$; so, the
notation is consistent with previous definitions.

Further, because names have structure it is possible to effect
substitutions on the basis of that structure. This means we need to
upgrade our notation for substitutions, which we accomplish by
adapting comprehension notation. Thus,

\begin{mathpar}
  P\{ y / x : x \in S \}
\end{mathpar}

is interpreted to mean the process derived from P by replacing (in a
capture-avoiding manner) each occurrence of $x$ in $S$ by $y$. For example,

\begin{mathpar}
  P\{ \quotep{\procn{x}|\procn{x}} / x : x \in \freenames{P} \}
\end{mathpar}

will replace each (occurrence) of a free name $x$ in $P$ by
$\quotep{\procn{x}|\procn{x}}$.

Also, we will avail ourselves of the notation $x^{L}$ and $x^{R}$ to
denote injections of a name into disjoint copies of the name
space. There are numerous ways to accomplish this. One example can be
found in \cite{MeredithR05}. This notation overloads to vectors of
names: $\vec{x}^{\pi} := (x_{i}^{\pi} \; : \; 0 \leq i < |\vec{x}| )$ where $\pi \in \{L,R\}$.

We also use $P^{\Box} := P|\Box$.

In \cite{MeredithR05} an interpretation of the new operator is
given. It turns out that there are several possible interpretations
all enjoying the requisite algebraic properties of the operator (see
\cite{milner91polyadicpi}). We will therefore make liberal use of
$(\nu\; \vec{x})P$.

% subsection the_syntax_and_semantics_of_the_notation_system (end)   

\input{qm2pi.qmops} 

\input{qm2pi.sterngerlach} 

\input{qm2pi.metric} 

% section concurrent_process_calculi (end)

%\input{qm2pi.proofsketch}

% section proof sketch (end)

%\input{qm2pi.slviaknots} 

% section spatial logic via knots (end)

\input{qm2pi.conclusion}

% section conclusion (end)

%\input{qm2pi.dtcodes} 

% section wiring algorithm (end)

\input{qm2pi.ack} 

% section acknowledgments (end)

\newpage


\bibliographystyle{plain}   
\bibliography{../../biblios/main.bib}

\input{qm2pi.rhodetails}

\end{document}

 

%\documentclass[12pt]{llncs}
%\documentclass{jktr}

\usepackage[pdftex]{hyperref}                   
\usepackage {listings}
\usepackage {mathpartir}
\usepackage{bcprules}
%\usepackage{listings}
                       
\usepackage{graphicx} 
%\usepackage[margins=2.5cm,nohead,nofoot]{geometry}
%\usepackage{geometry}
\usepackage{amsfonts}
\usepackage{amstext}
\usepackage{latexsym}
\usepackage{amssymb}
\usepackage{color}


%\include{myPreamble}
\include{qm2pi.local} 

%\ifpdf
%\usepackage[pdftex]{graphicx}
%\else
%\usepackage{graphicx}
%\fi

 % \ifpdf
%  \usepackage{pdfsync}
%  \if


%\title{Brief Article}
%\author{David F. Snyder}
%\author{L.G. Meredith}

%\address{Dept. of Math., Texas State University--San Marcos, San Marcos, TX 78666}
       
\pagestyle{empty}


\begin{document}

\lstset{language=[Objective]Caml,frame=shadowbox}

\input{qm2pi.front}

% section front matter (end)

\input{qm2pi.intro} 
 
% section introduction (end)

% \input{qm2pi.knotations} 

% section notation (end)

\input{qm2pi.process.calculi} 

% section concurrent_process_calculi_and_spatial_logics_ (end)
    
%\input{qm2pi.knots2pi} 

%\input{qm2pi.trefoil} 

%\input{qm2pi.mainthm} 

% subsection basic_interpretation (end)

%\input{qm2pi.rho.presentation} 
\subsection{The syntax and semantics of the notation system}\label{sub:the_syntax_and_semantics_of_the_notation_system} % (fold)

We now summarize a technical presentation of the calculus that
embodies our theory of dynamics. The typical presentation of such a
calculus follows the style of giving generators and relations on
them. The grammar, below, describing term constructors, freely
generates the set of processes, $\Proc$. This set is then quotiented
by a relation known as structural congruence and it is over this set
that the notion of dynamics is expressed. This presentation is
essentially that of \cite{MeredithR05} with the addition of
polyadicity and summation. For readability we have relegated some of
the technical subtleties to an appendix.

\subsubsection{Process grammar}\label{subsub:process_grammar}

\begin{mathpar}
  \inferrule* [lab=synchronization] {} {{M} \bc \pzero \;|\; x?F \;|\; x!C }
  \and
  \inferrule* [lab=abstraction] {} {{F} \bc (x)P}
  \and
  \inferrule* [lab=concretion] {} {{C} \bc \langle Q \rangle}
  \and
  \inferrule* [lab=process] {} {{P,Q} \bc M \;| \;P|Q \;|\; @{x}}
  \and
  \inferrule* [lab=name] {} {{x} \bc \quotep{P}}
\end{mathpar} 

Note that $\vec{x}$ (resp. $\vec{P}$) denotes a vector of names
(resp. processes) of length $|\vec{x}|$ (resp. $|\vec{P}|$). We adopt
the following useful abbreviations.

\begin{mathpar}
   x?(\vec{y}).P := x.(\vec{y})P \and  x\clift{\vec{P}} := x.\clift{\vec{P}}
   \and x!(y) := \lift{x}{\dropn{y}}
   \and \Pi_{i=0}^{n-1}P_i := P_0 | \ldots | P_{n-1}
\end{mathpar}

\subsubsection{Structural congruence}

\paragraph{Free and bound names and alpha-equivalence.} At the
core of structural equivalence is alpha-equivalence which identifies
process that are the same up to a change of variable. Formally, we
recognize the distinction between free and bound names. The free names
of a process, $\freenames{P}$, may be calculated recursively as
follows:

\begin{mathpar}
\freenames{\pzero} := \emptyset
  \and \\
  \freenames{x?(y).P} := \{ x \} \cup (\freenames{P} \setminus \{ y \})
  \and 
  \freenames{x!\langle P \rangle} := \{ x \} \cup \{ P \} 
  \and \\
  \freenames{P|Q} := \freenames{P} \cup \freenames{Q}
  \and \\
  \freenames{@{x}} := \{ x \}
\end{mathpar}

$\pi$
$\quotep{\pi}$

$\freenames{-} : \pi \to \mathcal{P}(\quotep{\pi})$

\begin{eqnarray*}
  \freenames{\pzero} & := & \emptyset \\
  \freenames{x?(y).P} & := & \{ x \} \cup (\freenames{P} \setminus \{ y \}) \\
  \freenames{x!\langle P \rangle} & := & \{ x \} \cup \{ P \} \\
  \freenames{P|Q} & := & \freenames{P} \cup \freenames{Q} \\
  \freenames{\dropn{x}} & := & \{ x \}
\end{eqnarray*}

The bound names of a process, $\boundnames{P}$, are those names occurring in $P$
that are not free. For example, in $x?(y).0$, the name $x$ is free, while $y$ is bound.

\begin{mathpar}
  \inferrule* [lab=monoidal-laws] {} { P|Q \equiv Q|P \and P|0 \equiv P \and P|(Q|R) \equiv (P|Q)|R }
\end{mathpar}

\begin{mathpar}
  \inferrule* [lab=alpha-equivalence] {} { (x)P \equiv (y)P\{y/x\} \and y \not\in \freenames{P} }
\end{mathpar}

\begin{definition}
Then two processes, $P,Q$, are alpha-equivalent if $P = Q\{\vec{y}/\vec{x}\}$ for
some $\vec{x} \in \boundnames{Q},\vec{y} \in \boundnames{P}$, where $Q\{\vec{y}/\vec{x}\}$
denotes the capture-avoiding substitution of $\vec{y}$ for $\vec{x}$ in $Q$.
\end{definition}

\begin{definition}
  The {\em structural congruence} \cite{SangiorgiWalker} , $\equiv$,
  between processes is the least congruence containing
  alpha-equivalence, satisfying the abelian monoid laws
  (associativity, commutativity and $\pzero$ as identity) for parallel
  composition $|$ and for summation $+$.
\end{definition}

\subsection{Name equivalence}

We take name equivalence, written $\nameeq$, to be the smallest
equivalence relation generated by the following rules.

\begin{mathpar}
\inferrule*[lab=Quote-drop]
{ }
{ \quotep{@{x}} \nameeq x }

\inferrule*[lab=Struct-equiv]
{ P \scong Q }
{ \quotep{P} \nameeq \quotep{Q} }
\end{mathpar}

The astute reader will have noticed that the mutual recursion of names
and processes imposes a mutual recursion on alpha-equivalence and
structural equivalence via name-equivalence. Fortunately, all of this
works out pleasantly and we may calculate in the natural way, free of
concern. The reader interested in the details is referred to the
appendix \ref{appendix:rho_details}.

\subsection{Substitution}

We use $\Proc$ for the set of processes, $\QProc$ for the set of
names, and $\id{\{}\vec{y} / \vec{x} \id{\}}$ to denote partial maps,
$s : \QProc \rightarrow \QProc$. A map, $s$ lifts, uniquely, to a map
on process terms, $\widehat{s} : \Proc \rightarrow \Proc$ by the
following equations.

\begin{mathpar}
  (0) \psubstp{Q}{P} := 0 \\
  (R \juxtap S) \psubstp{Q}{P}
  :=    
  (R)\psubstp{Q}{P} \juxtap (S) \psubstp{Q}{P} \\
  (x?(y).R) \psubstp{Q}{P}    
  :=    
  (x)\substp{Q}{P} (z)\concat( (R \psubstn{z}{y}) \psubstp{Q}{P} ) \\
  (\lift{x}{R}) \psubstp{Q}{P}  
  :=
  \lift{(x)\substp{Q}{P}}{ R \psubstp{Q}{P} } \\
%   (\dropn{x})  \psubstp{Q}{P}       
%   := 
%   \left\{ 
%     \begin{array}{ccc} 
%       \dropn{\quotep{Q}} & & x \nameeq \quotep{P} \\
%       \dropn{x} & & otherwise \\
%     \end{array}
%   \right. 
  (\dropn{x})  \psubstp{Q}{P}       
  := 
  \left\{ 
    \begin{array}{ccc} 
      Q & & x \nameeq \quotep{P} \\
      \dropn{x} & & otherwise \\
    \end{array}
  \right.
\end{mathpar}
 

where

\begin{eqnarray}
  (x)\id{\{} \lpquote Q \rpquote / \lpquote P \rpquote \id{\}}            = 
  \left\{ 
    \begin{array}{ccc}
      \lpquote Q \rpquote & & x \nameeq \lpquote P \rpquote \\
      x & & otherwise \\
    \end{array}
  \right. \nonumber
\end{eqnarray}

and $z$ is chosen distinct from $\quotep{P}$, $\quotep{Q}$, the free
names in $Q$, and all the names in $R$. Our $\alpha$-equivalence will
be built in the standard way from this substitution.

\begin{remark}\label{rem:no_self_referential_names}
  One consequence of these definitions is that $\forall P. \quotep{P}
  \not\in \freenames{P}$.
\end{remark}

\subsection{ Dynamic quote: an example }

Anticipating something of what's to come, consider applying the
substitution, $\widehat{\id{\{}u / z \id{\}}}$, to the following pair
of processes, $\lift{w}{y!(z)}$ and $w[ \lpquote y!(z) \rpquote ]$.

\begin{eqnarray}
	\lift{w}{y!(z)}\widehat{\id{\{}u / z \id{\}}}
		& = &
		\lift{w}{y!(u)} \nonumber\\
	w[ \lpquote y!(z) \rpquote ] \widehat{ \id{\{}u / z \id{\}} }
		& = &
		w[ \lpquote y!(z) \rpquote ] \nonumber
\end{eqnarray}

Because the body of the process between quotes is impervious to
substitution, we get radically different answers. In fact, by
examining the first process in an input context,
e.g. $x?(z).\lift{w}{y!(z)}$, we see that the process under the lift
operator may be shaped by prefixed inputs binding a name inside it. In
this sense, the lift operator will be seen as a way to dynamically
construct processes before reifying them as names.

Finally equipped with these standard features we can present the
dynamics of the calculus.

\subsubsection{Operational semantics} 

Finally, we introduce the computational dynamics. What marks these
algebras as distinct from other more traditionally studied algebraic
structures, e.g. vector spaces or polynomial rings, is the manner in
which dynamics is captured. In traditional structures, dynamics is typically
expressed through morphisms between such structures, as in linear maps
between vector spaces or morphisms between rings. In algebras
associated with the semantics of computation, the dynamics is
expressed as part of the algebraic structure itself, through a
reduction reduction relation typically denoted by $\red$. Below, we
give a recursive presentation of this relation for the calculus used
in the encoding.

$\red \subseteq \pi \times \pi$
$\red : \pi \to \mathcal{P}(\pi)$

\begin{mathpar}
  \inferrule* [lab=Comm] { \textsf{match}( x_{src}, x_{trgt} ) } { x_{trgt}?(y)P \; | \; x_{src}!\langle {Q} \rangle \red P\{\quotep{Q}/y}\} }
  \and \\
  \inferrule* [lab=Par] {{P} \red {P}'} {{{P} | {Q}} \red {{P}' | {Q}}}
  \and
  \inferrule* [lab=Equiv]{{{P} \scong {P}'} \andalso {{P}' \red {Q}'} \andalso {{Q}' \scong {Q}}}{{P} \red {Q}}
\end{mathpar}

\begin{eqnarray*}
  match_{\equiv} (\quotep{P},\quotep{Q}) & := & P \equiv Q \\
  match_{\dagger}(\quotep{P},\quotep{Q}) & := & \forall R. P|Q \red^{*} R => R \red^{*} 0 \\
  match_{K}(\quotep{P},\quotep{Q}) & := & K \mbox{ for some context } K
\end{eqnarray*}

$u?(x)P | u!\langle Q \rangle \red P\{\quotep{Q}/x\}$

%We write $\wred$ for $\red^*$, and $P\red$ if $\exists Q $ such that $ P \red Q$.
We write $P\red$ if $\exists Q $ such that $ P \red Q$ and $P\not\red$, otherwise.

\section{Replication}

As mentioned before, it is known that replication (and hence
recursion) can be implemented in a higher-order process algebra
\cite{SangiorgiWalker}. As our first example of calculation with the
machinery thus far presented we give the construction explicitly in
the {\rhoc}.

\begin{eqnarray}
	D_{x} & := & \prefix{x}{y}{(\binpar{\outputp{x}{y}}{@{y}})} \nonumber\\
	\bangp_{x}{P} & := & \binpar{{x}!\langle{\binpar{D_{x}}{P}}\rangle}{D_{x}} \nonumber
\end{eqnarray}

\begin{eqnarray}
	\bangp_{x}{P} & & \nonumber\\
	=
	& {x}!\langle{(\prefix{x}{y}{(\outputp{x}{y} | @{y})) | P}}\rangle 
	      | \prefix{x}{y}{(\outputp{x}{y} | @{y})} & \nonumber\\
	\red
	& (\outputp{x}{y} | @{y})\substn{\quotep{(\prefix{x}{y}{(@{y} | \outputp{x}{y})) | P}}}{y} & \nonumber\\
	=
	& \outputp{x}{\quotep{(\prefix{x}{y}{(\outputp{x}{y} | @{y})) | P}}}
	  | {(\prefix{x}{y}{(\outputp{x}{y} | @{y})) | P}} & \nonumber\\
	\red
	& \ldots & \nonumber\\
	\red^*
	& P | P | \ldots & \nonumber
\end{eqnarray}

Of course, this encoding, as an implementation, runs away, unfolding
$\bangp{P}$ eagerly. A lazier and more implementable replication
operator, restricted to input-guarded processes, may be obtained as follows.

\begin{eqnarray}
\bangp{\prefix{u}{v}{P}} 
	:= 
	\binpar{\lift{x}{\prefix{u}{v}{(\binpar{D(x)}{P})}}}{D(x)} \nonumber
\end{eqnarray}

\begin{remark}
  Note that the lazier definition still does not deal with summation
  or mixed summation (i.e. sums over input and output). The reader is
  invited to construct definitions of replication that deal with these
  features. 

  Further, the definitions are parameterized in a name, $x$. Can you,
  gentle reader, make a definition that eliminates this parameter and
  guarantees no accidental interaction between the replication
  machinery and the process being replicated -- i.e. no accidental
  sharing of names used by the process to get its work done and the
  name(s) used by the replication to effect copying. This latter
  revision of the definition of replication is crucial to obtaining
  the expected identity $!!P \sim !P$.
\end{remark}

\begin{remark}\label{rem:paradoxical_combinator}
  The reader familiar with the lambda calculus will have noticed the
  similarity between $D$ and the paradoxical combinator.

  [Ed. note: the existence of this seems to suggest we have to be more
  restrictive on the set of processes and names we admit if we are to
  support no-cloning.]
\end{remark}

\subsubsection{Bisimulation}

The computational dynamics gives rise to another kind of equivalence,
the equivalence of computational behavior. As previously mentioned
this is typically captured \emph{via} some form of bisimulation.

% The notion we use in this paper is weak barbed bisimulation
% \cite{milner91polyadicpi}.

The notion we use in this paper is derived from weak barbed
bisimulation \cite{milner91polyadicpi}. 

\begin{definition}
An \emph{observation relation}, $\downarrow_{\mathcal N}$, over a set
of names, $\mathcal N$, is the smallest relation satisfying the rules
below.

\infrule[Out-barb]{y \in {\mathcal N}, \; x \nameeq y}
		  {\outputp{x}{v} \downarrow_{\mathcal N} x}
\infrule[Par-barb]{\mbox{$P\downarrow_{\mathcal N} x$ or $Q\downarrow_{\mathcal N} x$}}
		  {\binpar{P}{Q} \downarrow_{\mathcal N} x}

We write $P \Downarrow_{\mathcal N} x$ if there is $Q$ such that 
$P \wred Q$ and $Q \downarrow_{\mathcal N} x$.
\end{definition}

\begin{definition}
%\label{def.bbisim}
An  ${\mathcal N}$-\emph{barbed bisimulation} over a set of names, ${\mathcal N}$, is a symmetric binary relation 
${\mathcal S}_{\mathcal N}$ between agents such that $P\rel{S}_{\mathcal N}Q$ implies:
\begin{enumerate}
\item If $P \red P'$ then $Q \wred Q'$ and $P'\rel{S}_{\mathcal N} Q'$.
\item If $P\downarrow_{\mathcal N} x$, then $Q\Downarrow_{\mathcal N} x$.
\end{enumerate}
$P$ is ${\mathcal N}$-barbed bisimilar to $Q$, written
$P \wbbisim_{\mathcal N} Q$, if $P \rel{S}_{\mathcal N} Q$ for some ${\mathcal N}$-barbed bisimulation ${\mathcal S}_{\mathcal N}$.
\end{definition}

$\mathcal{R} \subseteq \pi \times \pi$

$P \mathcal{R} Q => \forall P'. P \red P' \Rightarrow \exists Q'. Q \red Q', P' \mathcal{R} Q'$

$P \vdash x \Rightarrow Q \vdash x$

\begin{mathpar}
  \inferrule*[lab=Out-barb]{x \nameeq y}{{y}!\langle{Q}\rangle \vdash x}
  \and
  \inferrule*[lab=Par-barb]{\mbox{$P\vdash x$ or $Q\vdash x$}}{\binpar{P}{Q} \vdash x}
\end{mathpar}

\subsubsection{Contexts}

One of the principle advantages of computational calculi like the
$\pi$-calculus is a well-defined notion of context,
contextual-equivalence and a correlation between
contextual-equivalence and notions of bisimulation. The notion of
context allows the decomposition of a process into (sub-)process and
its syntactic environment, its context. Thus, a context may be
thought of as a process with a ``hole'' (written $\Box$) in it. The
application of a context $M$ to a process $P$, written $M[P]$, is
tantamount to filling the hole in $M$ with $P$. In this paper we do
not need the full weight of this theory, but do make use of the notion
of context in the proof the main theorem. 

\begin{mathpar}
  \inferrule* [lab=summation] {} {{M_{M},M_{N}} \bc \Box \;|\; x.M_{A} \;|\; M_{M}+M_{N}}
  \and
  \inferrule* [lab=agent] {} {{M_{A}} \bc (\vec{x})M_{P} \;| \; \clift{P_0,\ldots,M_{P},\ldots,P_N}}
  \and \\
  \inferrule* [lab=process] {} {{M_{P}} \bc M_{N} \;| \;P|M_{P} }
\end{mathpar} 

\begin{mathpar}
  \inferrule* [lab=sychronization] {} {M_{N} \bc \Box \;|\; x?M_{F} \;|\; x!M_{C}}
  \and
  \inferrule* [lab=abstraction] {} {{M_{F}} \bc (x)M_{P} }
  \and
  \inferrule* [lab=concretion] {} {{M_{C}} \bc \langle M_{P} \rangle }
  \and \\
  \inferrule* [lab=process] {} {{M_{P}} \bc M_{N} \;| \;P|M_{P} }
\end{mathpar}

\begin{definition}[contextual application] Given a context $M$, and
  process $P$, we define the \emph{contextual application}, $M[P] :=
  M\{P/\Box\}$. That is, the contextual application of M to P is the
  substitution of $P$ for $\Box$ in $M$.
\end{definition}

$\meaningof{-} : L \to \mathcal{P}(\pi)$

\begin{mathpar}
  \inferrule* [lab=collection] {} {\meaningof{true} = \pi, \and \meaningof{~E} = \pi \setminus \meaningof{E}, \and \meaningof{E_{1} \& E_{2}} = \meaningof{E_{1}} \cap \meaningof{E_{2}}}
\end{mathpar}

\begin{mathpar}
  \inferrule* [lab=structure] {} {\meaningof{0} = \{ P \in \pi | P \equiv 0 \}, \and \\ \meaningof{E_1 | E_2} = \{ P \in \pi | P \equiv P_{1} | P_{2}, P_{1} \in \meaningof{E_{1}}, P_{2} \in \meaningof{E_2}\} }
\end{mathpar}

\begin{mathpar}
 \inferrule* [lab=behavior] {} {\meaningof{\langle a?b \rangle E} = \{ P \in \pi | P \equiv Q | u?(y)P', \\ \and \\\\ \and \\ \;\;\; u \in \meaningof{a}, \forall z.P'\{z/y\} \in \meaningof{E\{z/b\}}\}, \and \\ \meaningof{a!E} = \{ P \in \pi | P \equiv Q | x!\langle P' \rangle, x \in \meaningof{a} P' \in \meaningof{E}\} }
\end{mathpar}

\begin{mathpar}
 \inferrule* [lab=nominal] {} {\meaningof{\quotep{E}} = \{ \quotep{P} \in \quotep{\pi} | P \in \meaningof{E} \}, \and \meaningof{\quotep{P}} = \{ \quotep{Q} \in \quotep{\pi} | P \equiv Q \} \and \\ \meaningof{@\quotep{E}} = \{ P \in \pi | P \equiv @x, x \in \meaningof{E} \}}
\end{mathpar}

\begin{eqnarray*}
  \\
  \meaningof{-} : TS \to ST
\end{eqnarray*}

\begin{eqnarray*}
  \\
  L : TS \to ST
\end{eqnarray*}

\begin{eqnarray*}
  \\
  P \models E \iff P \in \meaningof{E}
\end{eqnarray*}

\begin{eqnarray*}
  P \approx_{L} Q \iff \forall E \in L. P \models E \iff Q \models E
\end{eqnarray*}

\begin{eqnarray*}
  P \approx_{K} Q
\end{eqnarray*}

\begin{eqnarray*}
  P \approx Q
\end{eqnarray*}

$\approx_{K} = \approx = \approx_{L}$

\subsubsection{Contextual duality}

Note that contexts extend the quotation operation to a family of
operations from processes to names. Given a context, $M$, we can
define a \emph{nominal context}, $\quotep{M}$ by $\quotep{M}[P] :=
\quotep{M[P]}$. To foreshadow what is to come we observe that these
operations enjoy a duality with processes very much like the duality
between vectors and maps from vectors to scalars.

Further, because the calculus is essentially higher-order, we have a
correspondence between contexts and processes. More specifically,
given a name $x$ and a context $M$ we can construct $M^{*}_{x}$ such
that 

\begin{mathpar}
  M^{*}_{x} | \lift{x}{P} \red M[P]
\end{mathpar}

namely,

\begin{mathpar}
  M^{*}_{x} := x?(u).M[\dropn{u}]
\end{mathpar}

The dependence of $M^{*}_{x}$ on a name makes it an abstraction, 

\begin{mathpar}
  M^{*} := (x)x?(u).M[\dropn{u}]
\end{mathpar}

\subsection{Additional notation}

It will sometimes be convenient to denote the process a name
quotes. We already have the notation $x = \quotep{P}$, but it will be
convenient to introduce an alternate notation, $\procn{x}$, when we
want to emphasize the connection to the use of the name. Note that, by
virtue of name equivalence, $\quotep{\procn{x}} \nameeq x$; so, the
notation is consistent with previous definitions.

Further, because names have structure it is possible to effect
substitutions on the basis of that structure. This means we need to
upgrade our notation for substitutions, which we accomplish by
adapting comprehension notation. Thus,

\begin{mathpar}
  P\{ y / x : x \in S \}
\end{mathpar}

is interpreted to mean the process derived from P by replacing (in a
capture-avoiding manner) each occurrence of $x$ in $S$ by $y$. For example,

\begin{mathpar}
  P\{ \quotep{\procn{x}|\procn{x}} / x : x \in \freenames{P} \}
\end{mathpar}

will replace each (occurrence) of a free name $x$ in $P$ by
$\quotep{\procn{x}|\procn{x}}$.

Also, we will avail ourselves of the notation $x^{L}$ and $x^{R}$ to
denote injections of a name into disjoint copies of the name
space. There are numerous ways to accomplish this. One example can be
found in \cite{MeredithR05}. This notation overloads to vectors of
names: $\vec{x}^{\pi} := (x_{i}^{\pi} \; : \; 0 \leq i < |\vec{x}| )$ where $\pi \in \{L,R\}$.

We also use $P^{\Box} := P|\Box$.

In \cite{MeredithR05} an interpretation of the new operator is
given. It turns out that there are several possible interpretations
all enjoying the requisite algebraic properties of the operator (see
\cite{milner91polyadicpi}). We will therefore make liberal use of
$(\nu\; \vec{x})P$.

% subsection the_syntax_and_semantics_of_the_notation_system (end)   

\input{qm2pi.qmops} 

\input{qm2pi.sterngerlach} 

\input{qm2pi.metric} 

% section concurrent_process_calculi (end)

%\input{qm2pi.proofsketch}

% section proof sketch (end)

%\input{qm2pi.slviaknots} 

% section spatial logic via knots (end)

\input{qm2pi.conclusion}

% section conclusion (end)

%\input{qm2pi.dtcodes} 

% section wiring algorithm (end)

\input{qm2pi.ack} 

% section acknowledgments (end)

\newpage


\bibliographystyle{plain}   
\bibliography{../../biblios/main.bib}

\input{qm2pi.rhodetails}

\end{document}

 

%\documentclass[12pt]{llncs}
%\documentclass{jktr}

\usepackage[pdftex]{hyperref}                   
\usepackage {listings}
\usepackage {mathpartir}
\usepackage{bcprules}
%\usepackage{listings}
                       
\usepackage{graphicx} 
%\usepackage[margins=2.5cm,nohead,nofoot]{geometry}
%\usepackage{geometry}
\usepackage{amsfonts}
\usepackage{amstext}
\usepackage{latexsym}
\usepackage{amssymb}
\usepackage{color}


%\include{myPreamble}
\include{qm2pi.local} 

%\ifpdf
%\usepackage[pdftex]{graphicx}
%\else
%\usepackage{graphicx}
%\fi

 % \ifpdf
%  \usepackage{pdfsync}
%  \if


%\title{Brief Article}
%\author{David F. Snyder}
%\author{L.G. Meredith}

%\address{Dept. of Math., Texas State University--San Marcos, San Marcos, TX 78666}
       
\pagestyle{empty}


\begin{document}

\lstset{language=[Objective]Caml,frame=shadowbox}

\input{qm2pi.front}

% section front matter (end)

\input{qm2pi.intro} 
 
% section introduction (end)

% \input{qm2pi.knotations} 

% section notation (end)

\input{qm2pi.process.calculi} 

% section concurrent_process_calculi_and_spatial_logics_ (end)
    
%\input{qm2pi.knots2pi} 

%\input{qm2pi.trefoil} 

%\input{qm2pi.mainthm} 

% subsection basic_interpretation (end)

%\input{qm2pi.rho.presentation} 
\subsection{The syntax and semantics of the notation system}\label{sub:the_syntax_and_semantics_of_the_notation_system} % (fold)

We now summarize a technical presentation of the calculus that
embodies our theory of dynamics. The typical presentation of such a
calculus follows the style of giving generators and relations on
them. The grammar, below, describing term constructors, freely
generates the set of processes, $\Proc$. This set is then quotiented
by a relation known as structural congruence and it is over this set
that the notion of dynamics is expressed. This presentation is
essentially that of \cite{MeredithR05} with the addition of
polyadicity and summation. For readability we have relegated some of
the technical subtleties to an appendix.

\subsubsection{Process grammar}\label{subsub:process_grammar}

\begin{mathpar}
  \inferrule* [lab=synchronization] {} {{M} \bc \pzero \;|\; x?F \;|\; x!C }
  \and
  \inferrule* [lab=abstraction] {} {{F} \bc (x)P}
  \and
  \inferrule* [lab=concretion] {} {{C} \bc \langle Q \rangle}
  \and
  \inferrule* [lab=process] {} {{P,Q} \bc M \;| \;P|Q \;|\; @{x}}
  \and
  \inferrule* [lab=name] {} {{x} \bc \quotep{P}}
\end{mathpar} 

Note that $\vec{x}$ (resp. $\vec{P}$) denotes a vector of names
(resp. processes) of length $|\vec{x}|$ (resp. $|\vec{P}|$). We adopt
the following useful abbreviations.

\begin{mathpar}
   x?(\vec{y}).P := x.(\vec{y})P \and  x\clift{\vec{P}} := x.\clift{\vec{P}}
   \and x!(y) := \lift{x}{\dropn{y}}
   \and \Pi_{i=0}^{n-1}P_i := P_0 | \ldots | P_{n-1}
\end{mathpar}

\subsubsection{Structural congruence}

\paragraph{Free and bound names and alpha-equivalence.} At the
core of structural equivalence is alpha-equivalence which identifies
process that are the same up to a change of variable. Formally, we
recognize the distinction between free and bound names. The free names
of a process, $\freenames{P}$, may be calculated recursively as
follows:

\begin{mathpar}
\freenames{\pzero} := \emptyset
  \and \\
  \freenames{x?(y).P} := \{ x \} \cup (\freenames{P} \setminus \{ y \})
  \and 
  \freenames{x!\langle P \rangle} := \{ x \} \cup \{ P \} 
  \and \\
  \freenames{P|Q} := \freenames{P} \cup \freenames{Q}
  \and \\
  \freenames{@{x}} := \{ x \}
\end{mathpar}

$\pi$
$\quotep{\pi}$

$\freenames{-} : \pi \to \mathcal{P}(\quotep{\pi})$

\begin{eqnarray*}
  \freenames{\pzero} & := & \emptyset \\
  \freenames{x?(y).P} & := & \{ x \} \cup (\freenames{P} \setminus \{ y \}) \\
  \freenames{x!\langle P \rangle} & := & \{ x \} \cup \{ P \} \\
  \freenames{P|Q} & := & \freenames{P} \cup \freenames{Q} \\
  \freenames{\dropn{x}} & := & \{ x \}
\end{eqnarray*}

The bound names of a process, $\boundnames{P}$, are those names occurring in $P$
that are not free. For example, in $x?(y).0$, the name $x$ is free, while $y$ is bound.

\begin{mathpar}
  \inferrule* [lab=monoidal-laws] {} { P|Q \equiv Q|P \and P|0 \equiv P \and P|(Q|R) \equiv (P|Q)|R }
\end{mathpar}

\begin{mathpar}
  \inferrule* [lab=alpha-equivalence] {} { (x)P \equiv (y)P\{y/x\} \and y \not\in \freenames{P} }
\end{mathpar}

\begin{definition}
Then two processes, $P,Q$, are alpha-equivalent if $P = Q\{\vec{y}/\vec{x}\}$ for
some $\vec{x} \in \boundnames{Q},\vec{y} \in \boundnames{P}$, where $Q\{\vec{y}/\vec{x}\}$
denotes the capture-avoiding substitution of $\vec{y}$ for $\vec{x}$ in $Q$.
\end{definition}

\begin{definition}
  The {\em structural congruence} \cite{SangiorgiWalker} , $\equiv$,
  between processes is the least congruence containing
  alpha-equivalence, satisfying the abelian monoid laws
  (associativity, commutativity and $\pzero$ as identity) for parallel
  composition $|$ and for summation $+$.
\end{definition}

\subsection{Name equivalence}

We take name equivalence, written $\nameeq$, to be the smallest
equivalence relation generated by the following rules.

\begin{mathpar}
\inferrule*[lab=Quote-drop]
{ }
{ \quotep{@{x}} \nameeq x }

\inferrule*[lab=Struct-equiv]
{ P \scong Q }
{ \quotep{P} \nameeq \quotep{Q} }
\end{mathpar}

The astute reader will have noticed that the mutual recursion of names
and processes imposes a mutual recursion on alpha-equivalence and
structural equivalence via name-equivalence. Fortunately, all of this
works out pleasantly and we may calculate in the natural way, free of
concern. The reader interested in the details is referred to the
appendix \ref{appendix:rho_details}.

\subsection{Substitution}

We use $\Proc$ for the set of processes, $\QProc$ for the set of
names, and $\id{\{}\vec{y} / \vec{x} \id{\}}$ to denote partial maps,
$s : \QProc \rightarrow \QProc$. A map, $s$ lifts, uniquely, to a map
on process terms, $\widehat{s} : \Proc \rightarrow \Proc$ by the
following equations.

\begin{mathpar}
  (0) \psubstp{Q}{P} := 0 \\
  (R \juxtap S) \psubstp{Q}{P}
  :=    
  (R)\psubstp{Q}{P} \juxtap (S) \psubstp{Q}{P} \\
  (x?(y).R) \psubstp{Q}{P}    
  :=    
  (x)\substp{Q}{P} (z)\concat( (R \psubstn{z}{y}) \psubstp{Q}{P} ) \\
  (\lift{x}{R}) \psubstp{Q}{P}  
  :=
  \lift{(x)\substp{Q}{P}}{ R \psubstp{Q}{P} } \\
%   (\dropn{x})  \psubstp{Q}{P}       
%   := 
%   \left\{ 
%     \begin{array}{ccc} 
%       \dropn{\quotep{Q}} & & x \nameeq \quotep{P} \\
%       \dropn{x} & & otherwise \\
%     \end{array}
%   \right. 
  (\dropn{x})  \psubstp{Q}{P}       
  := 
  \left\{ 
    \begin{array}{ccc} 
      Q & & x \nameeq \quotep{P} \\
      \dropn{x} & & otherwise \\
    \end{array}
  \right.
\end{mathpar}
 

where

\begin{eqnarray}
  (x)\id{\{} \lpquote Q \rpquote / \lpquote P \rpquote \id{\}}            = 
  \left\{ 
    \begin{array}{ccc}
      \lpquote Q \rpquote & & x \nameeq \lpquote P \rpquote \\
      x & & otherwise \\
    \end{array}
  \right. \nonumber
\end{eqnarray}

and $z$ is chosen distinct from $\quotep{P}$, $\quotep{Q}$, the free
names in $Q$, and all the names in $R$. Our $\alpha$-equivalence will
be built in the standard way from this substitution.

\begin{remark}\label{rem:no_self_referential_names}
  One consequence of these definitions is that $\forall P. \quotep{P}
  \not\in \freenames{P}$.
\end{remark}

\subsection{ Dynamic quote: an example }

Anticipating something of what's to come, consider applying the
substitution, $\widehat{\id{\{}u / z \id{\}}}$, to the following pair
of processes, $\lift{w}{y!(z)}$ and $w[ \lpquote y!(z) \rpquote ]$.

\begin{eqnarray}
	\lift{w}{y!(z)}\widehat{\id{\{}u / z \id{\}}}
		& = &
		\lift{w}{y!(u)} \nonumber\\
	w[ \lpquote y!(z) \rpquote ] \widehat{ \id{\{}u / z \id{\}} }
		& = &
		w[ \lpquote y!(z) \rpquote ] \nonumber
\end{eqnarray}

Because the body of the process between quotes is impervious to
substitution, we get radically different answers. In fact, by
examining the first process in an input context,
e.g. $x?(z).\lift{w}{y!(z)}$, we see that the process under the lift
operator may be shaped by prefixed inputs binding a name inside it. In
this sense, the lift operator will be seen as a way to dynamically
construct processes before reifying them as names.

Finally equipped with these standard features we can present the
dynamics of the calculus.

\subsubsection{Operational semantics} 

Finally, we introduce the computational dynamics. What marks these
algebras as distinct from other more traditionally studied algebraic
structures, e.g. vector spaces or polynomial rings, is the manner in
which dynamics is captured. In traditional structures, dynamics is typically
expressed through morphisms between such structures, as in linear maps
between vector spaces or morphisms between rings. In algebras
associated with the semantics of computation, the dynamics is
expressed as part of the algebraic structure itself, through a
reduction reduction relation typically denoted by $\red$. Below, we
give a recursive presentation of this relation for the calculus used
in the encoding.

$\red \subseteq \pi \times \pi$
$\red : \pi \to \mathcal{P}(\pi)$

\begin{mathpar}
  \inferrule* [lab=Comm] { \textsf{match}( x_{src}, x_{trgt} ) } { x_{trgt}?(y)P \; | \; x_{src}!\langle {Q} \rangle \red P\{\quotep{Q}/y}\} }
  \and \\
  \inferrule* [lab=Par] {{P} \red {P}'} {{{P} | {Q}} \red {{P}' | {Q}}}
  \and
  \inferrule* [lab=Equiv]{{{P} \scong {P}'} \andalso {{P}' \red {Q}'} \andalso {{Q}' \scong {Q}}}{{P} \red {Q}}
\end{mathpar}

\begin{eqnarray*}
  match_{\equiv} (\quotep{P},\quotep{Q}) & := & P \equiv Q \\
  match_{\dagger}(\quotep{P},\quotep{Q}) & := & \forall R. P|Q \red^{*} R => R \red^{*} 0 \\
  match_{K}(\quotep{P},\quotep{Q}) & := & K \mbox{ for some context } K
\end{eqnarray*}

$u?(x)P | u!\langle Q \rangle \red P\{\quotep{Q}/x\}$

%We write $\wred$ for $\red^*$, and $P\red$ if $\exists Q $ such that $ P \red Q$.
We write $P\red$ if $\exists Q $ such that $ P \red Q$ and $P\not\red$, otherwise.

\section{Replication}

As mentioned before, it is known that replication (and hence
recursion) can be implemented in a higher-order process algebra
\cite{SangiorgiWalker}. As our first example of calculation with the
machinery thus far presented we give the construction explicitly in
the {\rhoc}.

\begin{eqnarray}
	D_{x} & := & \prefix{x}{y}{(\binpar{\outputp{x}{y}}{@{y}})} \nonumber\\
	\bangp_{x}{P} & := & \binpar{{x}!\langle{\binpar{D_{x}}{P}}\rangle}{D_{x}} \nonumber
\end{eqnarray}

\begin{eqnarray}
	\bangp_{x}{P} & & \nonumber\\
	=
	& {x}!\langle{(\prefix{x}{y}{(\outputp{x}{y} | @{y})) | P}}\rangle 
	      | \prefix{x}{y}{(\outputp{x}{y} | @{y})} & \nonumber\\
	\red
	& (\outputp{x}{y} | @{y})\substn{\quotep{(\prefix{x}{y}{(@{y} | \outputp{x}{y})) | P}}}{y} & \nonumber\\
	=
	& \outputp{x}{\quotep{(\prefix{x}{y}{(\outputp{x}{y} | @{y})) | P}}}
	  | {(\prefix{x}{y}{(\outputp{x}{y} | @{y})) | P}} & \nonumber\\
	\red
	& \ldots & \nonumber\\
	\red^*
	& P | P | \ldots & \nonumber
\end{eqnarray}

Of course, this encoding, as an implementation, runs away, unfolding
$\bangp{P}$ eagerly. A lazier and more implementable replication
operator, restricted to input-guarded processes, may be obtained as follows.

\begin{eqnarray}
\bangp{\prefix{u}{v}{P}} 
	:= 
	\binpar{\lift{x}{\prefix{u}{v}{(\binpar{D(x)}{P})}}}{D(x)} \nonumber
\end{eqnarray}

\begin{remark}
  Note that the lazier definition still does not deal with summation
  or mixed summation (i.e. sums over input and output). The reader is
  invited to construct definitions of replication that deal with these
  features. 

  Further, the definitions are parameterized in a name, $x$. Can you,
  gentle reader, make a definition that eliminates this parameter and
  guarantees no accidental interaction between the replication
  machinery and the process being replicated -- i.e. no accidental
  sharing of names used by the process to get its work done and the
  name(s) used by the replication to effect copying. This latter
  revision of the definition of replication is crucial to obtaining
  the expected identity $!!P \sim !P$.
\end{remark}

\begin{remark}\label{rem:paradoxical_combinator}
  The reader familiar with the lambda calculus will have noticed the
  similarity between $D$ and the paradoxical combinator.

  [Ed. note: the existence of this seems to suggest we have to be more
  restrictive on the set of processes and names we admit if we are to
  support no-cloning.]
\end{remark}

\subsubsection{Bisimulation}

The computational dynamics gives rise to another kind of equivalence,
the equivalence of computational behavior. As previously mentioned
this is typically captured \emph{via} some form of bisimulation.

% The notion we use in this paper is weak barbed bisimulation
% \cite{milner91polyadicpi}.

The notion we use in this paper is derived from weak barbed
bisimulation \cite{milner91polyadicpi}. 

\begin{definition}
An \emph{observation relation}, $\downarrow_{\mathcal N}$, over a set
of names, $\mathcal N$, is the smallest relation satisfying the rules
below.

\infrule[Out-barb]{y \in {\mathcal N}, \; x \nameeq y}
		  {\outputp{x}{v} \downarrow_{\mathcal N} x}
\infrule[Par-barb]{\mbox{$P\downarrow_{\mathcal N} x$ or $Q\downarrow_{\mathcal N} x$}}
		  {\binpar{P}{Q} \downarrow_{\mathcal N} x}

We write $P \Downarrow_{\mathcal N} x$ if there is $Q$ such that 
$P \wred Q$ and $Q \downarrow_{\mathcal N} x$.
\end{definition}

\begin{definition}
%\label{def.bbisim}
An  ${\mathcal N}$-\emph{barbed bisimulation} over a set of names, ${\mathcal N}$, is a symmetric binary relation 
${\mathcal S}_{\mathcal N}$ between agents such that $P\rel{S}_{\mathcal N}Q$ implies:
\begin{enumerate}
\item If $P \red P'$ then $Q \wred Q'$ and $P'\rel{S}_{\mathcal N} Q'$.
\item If $P\downarrow_{\mathcal N} x$, then $Q\Downarrow_{\mathcal N} x$.
\end{enumerate}
$P$ is ${\mathcal N}$-barbed bisimilar to $Q$, written
$P \wbbisim_{\mathcal N} Q$, if $P \rel{S}_{\mathcal N} Q$ for some ${\mathcal N}$-barbed bisimulation ${\mathcal S}_{\mathcal N}$.
\end{definition}

$\mathcal{R} \subseteq \pi \times \pi$

$P \mathcal{R} Q => \forall P'. P \red P' \Rightarrow \exists Q'. Q \red Q', P' \mathcal{R} Q'$

$P \vdash x \Rightarrow Q \vdash x$

\begin{mathpar}
  \inferrule*[lab=Out-barb]{x \nameeq y}{{y}!\langle{Q}\rangle \vdash x}
  \and
  \inferrule*[lab=Par-barb]{\mbox{$P\vdash x$ or $Q\vdash x$}}{\binpar{P}{Q} \vdash x}
\end{mathpar}

\subsubsection{Contexts}

One of the principle advantages of computational calculi like the
$\pi$-calculus is a well-defined notion of context,
contextual-equivalence and a correlation between
contextual-equivalence and notions of bisimulation. The notion of
context allows the decomposition of a process into (sub-)process and
its syntactic environment, its context. Thus, a context may be
thought of as a process with a ``hole'' (written $\Box$) in it. The
application of a context $M$ to a process $P$, written $M[P]$, is
tantamount to filling the hole in $M$ with $P$. In this paper we do
not need the full weight of this theory, but do make use of the notion
of context in the proof the main theorem. 

\begin{mathpar}
  \inferrule* [lab=summation] {} {{M_{M},M_{N}} \bc \Box \;|\; x.M_{A} \;|\; M_{M}+M_{N}}
  \and
  \inferrule* [lab=agent] {} {{M_{A}} \bc (\vec{x})M_{P} \;| \; \clift{P_0,\ldots,M_{P},\ldots,P_N}}
  \and \\
  \inferrule* [lab=process] {} {{M_{P}} \bc M_{N} \;| \;P|M_{P} }
\end{mathpar} 

\begin{mathpar}
  \inferrule* [lab=sychronization] {} {M_{N} \bc \Box \;|\; x?M_{F} \;|\; x!M_{C}}
  \and
  \inferrule* [lab=abstraction] {} {{M_{F}} \bc (x)M_{P} }
  \and
  \inferrule* [lab=concretion] {} {{M_{C}} \bc \langle M_{P} \rangle }
  \and \\
  \inferrule* [lab=process] {} {{M_{P}} \bc M_{N} \;| \;P|M_{P} }
\end{mathpar}

\begin{definition}[contextual application] Given a context $M$, and
  process $P$, we define the \emph{contextual application}, $M[P] :=
  M\{P/\Box\}$. That is, the contextual application of M to P is the
  substitution of $P$ for $\Box$ in $M$.
\end{definition}

$\meaningof{-} : L \to \mathcal{P}(\pi)$

\begin{mathpar}
  \inferrule* [lab=collection] {} {\meaningof{true} = \pi, \and \meaningof{~E} = \pi \setminus \meaningof{E}, \and \meaningof{E_{1} \& E_{2}} = \meaningof{E_{1}} \cap \meaningof{E_{2}}}
\end{mathpar}

\begin{mathpar}
  \inferrule* [lab=structure] {} {\meaningof{0} = \{ P \in \pi | P \equiv 0 \}, \and \\ \meaningof{E_1 | E_2} = \{ P \in \pi | P \equiv P_{1} | P_{2}, P_{1} \in \meaningof{E_{1}}, P_{2} \in \meaningof{E_2}\} }
\end{mathpar}

\begin{mathpar}
 \inferrule* [lab=behavior] {} {\meaningof{\langle a?b \rangle E} = \{ P \in \pi | P \equiv Q | u?(y)P', \\ \and \\\\ \and \\ \;\;\; u \in \meaningof{a}, \forall z.P'\{z/y\} \in \meaningof{E\{z/b\}}\}, \and \\ \meaningof{a!E} = \{ P \in \pi | P \equiv Q | x!\langle P' \rangle, x \in \meaningof{a} P' \in \meaningof{E}\} }
\end{mathpar}

\begin{mathpar}
 \inferrule* [lab=nominal] {} {\meaningof{\quotep{E}} = \{ \quotep{P} \in \quotep{\pi} | P \in \meaningof{E} \}, \and \meaningof{\quotep{P}} = \{ \quotep{Q} \in \quotep{\pi} | P \equiv Q \} \and \\ \meaningof{@\quotep{E}} = \{ P \in \pi | P \equiv @x, x \in \meaningof{E} \}}
\end{mathpar}

\begin{eqnarray*}
  \\
  \meaningof{-} : TS \to ST
\end{eqnarray*}

\begin{eqnarray*}
  \\
  L : TS \to ST
\end{eqnarray*}

\begin{eqnarray*}
  \\
  P \models E \iff P \in \meaningof{E}
\end{eqnarray*}

\begin{eqnarray*}
  P \approx_{L} Q \iff \forall E \in L. P \models E \iff Q \models E
\end{eqnarray*}

\begin{eqnarray*}
  P \approx_{K} Q
\end{eqnarray*}

\begin{eqnarray*}
  P \approx Q
\end{eqnarray*}

$\approx_{K} = \approx = \approx_{L}$

\subsubsection{Contextual duality}

Note that contexts extend the quotation operation to a family of
operations from processes to names. Given a context, $M$, we can
define a \emph{nominal context}, $\quotep{M}$ by $\quotep{M}[P] :=
\quotep{M[P]}$. To foreshadow what is to come we observe that these
operations enjoy a duality with processes very much like the duality
between vectors and maps from vectors to scalars.

Further, because the calculus is essentially higher-order, we have a
correspondence between contexts and processes. More specifically,
given a name $x$ and a context $M$ we can construct $M^{*}_{x}$ such
that 

\begin{mathpar}
  M^{*}_{x} | \lift{x}{P} \red M[P]
\end{mathpar}

namely,

\begin{mathpar}
  M^{*}_{x} := x?(u).M[\dropn{u}]
\end{mathpar}

The dependence of $M^{*}_{x}$ on a name makes it an abstraction, 

\begin{mathpar}
  M^{*} := (x)x?(u).M[\dropn{u}]
\end{mathpar}

\subsection{Additional notation}

It will sometimes be convenient to denote the process a name
quotes. We already have the notation $x = \quotep{P}$, but it will be
convenient to introduce an alternate notation, $\procn{x}$, when we
want to emphasize the connection to the use of the name. Note that, by
virtue of name equivalence, $\quotep{\procn{x}} \nameeq x$; so, the
notation is consistent with previous definitions.

Further, because names have structure it is possible to effect
substitutions on the basis of that structure. This means we need to
upgrade our notation for substitutions, which we accomplish by
adapting comprehension notation. Thus,

\begin{mathpar}
  P\{ y / x : x \in S \}
\end{mathpar}

is interpreted to mean the process derived from P by replacing (in a
capture-avoiding manner) each occurrence of $x$ in $S$ by $y$. For example,

\begin{mathpar}
  P\{ \quotep{\procn{x}|\procn{x}} / x : x \in \freenames{P} \}
\end{mathpar}

will replace each (occurrence) of a free name $x$ in $P$ by
$\quotep{\procn{x}|\procn{x}}$.

Also, we will avail ourselves of the notation $x^{L}$ and $x^{R}$ to
denote injections of a name into disjoint copies of the name
space. There are numerous ways to accomplish this. One example can be
found in \cite{MeredithR05}. This notation overloads to vectors of
names: $\vec{x}^{\pi} := (x_{i}^{\pi} \; : \; 0 \leq i < |\vec{x}| )$ where $\pi \in \{L,R\}$.

We also use $P^{\Box} := P|\Box$.

In \cite{MeredithR05} an interpretation of the new operator is
given. It turns out that there are several possible interpretations
all enjoying the requisite algebraic properties of the operator (see
\cite{milner91polyadicpi}). We will therefore make liberal use of
$(\nu\; \vec{x})P$.

% subsection the_syntax_and_semantics_of_the_notation_system (end)   

\input{qm2pi.qmops} 

\input{qm2pi.sterngerlach} 

\input{qm2pi.metric} 

% section concurrent_process_calculi (end)

%\input{qm2pi.proofsketch}

% section proof sketch (end)

%\input{qm2pi.slviaknots} 

% section spatial logic via knots (end)

\input{qm2pi.conclusion}

% section conclusion (end)

%\input{qm2pi.dtcodes} 

% section wiring algorithm (end)

\input{qm2pi.ack} 

% section acknowledgments (end)

\newpage


\bibliographystyle{plain}   
\bibliography{../../biblios/main.bib}

\input{qm2pi.rhodetails}

\end{document}

 

% subsection basic_interpretation (end)

%\input{qm2pi.rho.presentation} 
\subsection{The syntax and semantics of the notation system}\label{sub:the_syntax_and_semantics_of_the_notation_system} % (fold)

We now summarize a technical presentation of the calculus that
embodies our theory of dynamics. The typical presentation of such a
calculus follows the style of giving generators and relations on
them. The grammar, below, describing term constructors, freely
generates the set of processes, $\Proc$. This set is then quotiented
by a relation known as structural congruence and it is over this set
that the notion of dynamics is expressed. This presentation is
essentially that of \cite{MeredithR05} with the addition of
polyadicity and summation. For readability we have relegated some of
the technical subtleties to an appendix.

\subsubsection{Process grammar}\label{subsub:process_grammar}

\begin{mathpar}
  \inferrule* [lab=synchronization] {} {{M} \bc \pzero \;|\; x?F \;|\; x!C }
  \and
  \inferrule* [lab=abstraction] {} {{F} \bc (x)P}
  \and
  \inferrule* [lab=concretion] {} {{C} \bc \langle Q \rangle}
  \and
  \inferrule* [lab=process] {} {{P,Q} \bc M \;| \;P|Q \;|\; @{x}}
  \and
  \inferrule* [lab=name] {} {{x} \bc \quotep{P}}
\end{mathpar} 

Note that $\vec{x}$ (resp. $\vec{P}$) denotes a vector of names
(resp. processes) of length $|\vec{x}|$ (resp. $|\vec{P}|$). We adopt
the following useful abbreviations.

\begin{mathpar}
   x?(\vec{y}).P := x.(\vec{y})P \and  x\clift{\vec{P}} := x.\clift{\vec{P}}
   \and x!(y) := \lift{x}{\dropn{y}}
   \and \Pi_{i=0}^{n-1}P_i := P_0 | \ldots | P_{n-1}
\end{mathpar}

\subsubsection{Structural congruence}

\paragraph{Free and bound names and alpha-equivalence.} At the
core of structural equivalence is alpha-equivalence which identifies
process that are the same up to a change of variable. Formally, we
recognize the distinction between free and bound names. The free names
of a process, $\freenames{P}$, may be calculated recursively as
follows:

\begin{mathpar}
\freenames{\pzero} := \emptyset
  \and \\
  \freenames{x?(y).P} := \{ x \} \cup (\freenames{P} \setminus \{ y \})
  \and 
  \freenames{x!\langle P \rangle} := \{ x \} \cup \{ P \} 
  \and \\
  \freenames{P|Q} := \freenames{P} \cup \freenames{Q}
  \and \\
  \freenames{@{x}} := \{ x \}
\end{mathpar}

$\pi$
$\quotep{\pi}$

$\freenames{-} : \pi \to \mathcal{P}(\quotep{\pi})$

\begin{eqnarray*}
  \freenames{\pzero} & := & \emptyset \\
  \freenames{x?(y).P} & := & \{ x \} \cup (\freenames{P} \setminus \{ y \}) \\
  \freenames{x!\langle P \rangle} & := & \{ x \} \cup \{ P \} \\
  \freenames{P|Q} & := & \freenames{P} \cup \freenames{Q} \\
  \freenames{\dropn{x}} & := & \{ x \}
\end{eqnarray*}

The bound names of a process, $\boundnames{P}$, are those names occurring in $P$
that are not free. For example, in $x?(y).0$, the name $x$ is free, while $y$ is bound.

\begin{mathpar}
  \inferrule* [lab=monoidal-laws] {} { P|Q \equiv Q|P \and P|0 \equiv P \and P|(Q|R) \equiv (P|Q)|R }
\end{mathpar}

\begin{mathpar}
  \inferrule* [lab=alpha-equivalence] {} { (x)P \equiv (y)P\{y/x\} \and y \not\in \freenames{P} }
\end{mathpar}

\begin{definition}
Then two processes, $P,Q$, are alpha-equivalent if $P = Q\{\vec{y}/\vec{x}\}$ for
some $\vec{x} \in \boundnames{Q},\vec{y} \in \boundnames{P}$, where $Q\{\vec{y}/\vec{x}\}$
denotes the capture-avoiding substitution of $\vec{y}$ for $\vec{x}$ in $Q$.
\end{definition}

\begin{definition}
  The {\em structural congruence} \cite{SangiorgiWalker} , $\equiv$,
  between processes is the least congruence containing
  alpha-equivalence, satisfying the abelian monoid laws
  (associativity, commutativity and $\pzero$ as identity) for parallel
  composition $|$ and for summation $+$.
\end{definition}

\subsection{Name equivalence}

We take name equivalence, written $\nameeq$, to be the smallest
equivalence relation generated by the following rules.

\begin{mathpar}
\inferrule*[lab=Quote-drop]
{ }
{ \quotep{@{x}} \nameeq x }

\inferrule*[lab=Struct-equiv]
{ P \scong Q }
{ \quotep{P} \nameeq \quotep{Q} }
\end{mathpar}

The astute reader will have noticed that the mutual recursion of names
and processes imposes a mutual recursion on alpha-equivalence and
structural equivalence via name-equivalence. Fortunately, all of this
works out pleasantly and we may calculate in the natural way, free of
concern. The reader interested in the details is referred to the
appendix \ref{appendix:rho_details}.

\subsection{Substitution}

We use $\Proc$ for the set of processes, $\QProc$ for the set of
names, and $\id{\{}\vec{y} / \vec{x} \id{\}}$ to denote partial maps,
$s : \QProc \rightarrow \QProc$. A map, $s$ lifts, uniquely, to a map
on process terms, $\widehat{s} : \Proc \rightarrow \Proc$ by the
following equations.

\begin{mathpar}
  (0) \psubstp{Q}{P} := 0 \\
  (R \juxtap S) \psubstp{Q}{P}
  :=    
  (R)\psubstp{Q}{P} \juxtap (S) \psubstp{Q}{P} \\
  (x?(y).R) \psubstp{Q}{P}    
  :=    
  (x)\substp{Q}{P} (z)\concat( (R \psubstn{z}{y}) \psubstp{Q}{P} ) \\
  (\lift{x}{R}) \psubstp{Q}{P}  
  :=
  \lift{(x)\substp{Q}{P}}{ R \psubstp{Q}{P} } \\
%   (\dropn{x})  \psubstp{Q}{P}       
%   := 
%   \left\{ 
%     \begin{array}{ccc} 
%       \dropn{\quotep{Q}} & & x \nameeq \quotep{P} \\
%       \dropn{x} & & otherwise \\
%     \end{array}
%   \right. 
  (\dropn{x})  \psubstp{Q}{P}       
  := 
  \left\{ 
    \begin{array}{ccc} 
      Q & & x \nameeq \quotep{P} \\
      \dropn{x} & & otherwise \\
    \end{array}
  \right.
\end{mathpar}
 

where

\begin{eqnarray}
  (x)\id{\{} \lpquote Q \rpquote / \lpquote P \rpquote \id{\}}            = 
  \left\{ 
    \begin{array}{ccc}
      \lpquote Q \rpquote & & x \nameeq \lpquote P \rpquote \\
      x & & otherwise \\
    \end{array}
  \right. \nonumber
\end{eqnarray}

and $z$ is chosen distinct from $\quotep{P}$, $\quotep{Q}$, the free
names in $Q$, and all the names in $R$. Our $\alpha$-equivalence will
be built in the standard way from this substitution.

\begin{remark}\label{rem:no_self_referential_names}
  One consequence of these definitions is that $\forall P. \quotep{P}
  \not\in \freenames{P}$.
\end{remark}

\subsection{ Dynamic quote: an example }

Anticipating something of what's to come, consider applying the
substitution, $\widehat{\id{\{}u / z \id{\}}}$, to the following pair
of processes, $\lift{w}{y!(z)}$ and $w[ \lpquote y!(z) \rpquote ]$.

\begin{eqnarray}
	\lift{w}{y!(z)}\widehat{\id{\{}u / z \id{\}}}
		& = &
		\lift{w}{y!(u)} \nonumber\\
	w[ \lpquote y!(z) \rpquote ] \widehat{ \id{\{}u / z \id{\}} }
		& = &
		w[ \lpquote y!(z) \rpquote ] \nonumber
\end{eqnarray}

Because the body of the process between quotes is impervious to
substitution, we get radically different answers. In fact, by
examining the first process in an input context,
e.g. $x?(z).\lift{w}{y!(z)}$, we see that the process under the lift
operator may be shaped by prefixed inputs binding a name inside it. In
this sense, the lift operator will be seen as a way to dynamically
construct processes before reifying them as names.

Finally equipped with these standard features we can present the
dynamics of the calculus.

\subsubsection{Operational semantics} 

Finally, we introduce the computational dynamics. What marks these
algebras as distinct from other more traditionally studied algebraic
structures, e.g. vector spaces or polynomial rings, is the manner in
which dynamics is captured. In traditional structures, dynamics is typically
expressed through morphisms between such structures, as in linear maps
between vector spaces or morphisms between rings. In algebras
associated with the semantics of computation, the dynamics is
expressed as part of the algebraic structure itself, through a
reduction reduction relation typically denoted by $\red$. Below, we
give a recursive presentation of this relation for the calculus used
in the encoding.

$\red \subseteq \pi \times \pi$
$\red : \pi \to \mathcal{P}(\pi)$

\begin{mathpar}
  \inferrule* [lab=Comm] { \textsf{match}( x_{src}, x_{trgt} ) } { x_{trgt}?(y)P \; | \; x_{src}!\langle {Q} \rangle \red P\{\quotep{Q}/y}\} }
  \and \\
  \inferrule* [lab=Par] {{P} \red {P}'} {{{P} | {Q}} \red {{P}' | {Q}}}
  \and
  \inferrule* [lab=Equiv]{{{P} \scong {P}'} \andalso {{P}' \red {Q}'} \andalso {{Q}' \scong {Q}}}{{P} \red {Q}}
\end{mathpar}

\begin{eqnarray*}
  match_{\equiv} (\quotep{P},\quotep{Q}) & := & P \equiv Q \\
  match_{\dagger}(\quotep{P},\quotep{Q}) & := & \forall R. P|Q \red^{*} R => R \red^{*} 0 \\
  match_{K}(\quotep{P},\quotep{Q}) & := & K \mbox{ for some context } K
\end{eqnarray*}

$u?(x)P | u!\langle Q \rangle \red P\{\quotep{Q}/x\}$

%We write $\wred$ for $\red^*$, and $P\red$ if $\exists Q $ such that $ P \red Q$.
We write $P\red$ if $\exists Q $ such that $ P \red Q$ and $P\not\red$, otherwise.

\section{Replication}

As mentioned before, it is known that replication (and hence
recursion) can be implemented in a higher-order process algebra
\cite{SangiorgiWalker}. As our first example of calculation with the
machinery thus far presented we give the construction explicitly in
the {\rhoc}.

\begin{eqnarray}
	D_{x} & := & \prefix{x}{y}{(\binpar{\outputp{x}{y}}{@{y}})} \nonumber\\
	\bangp_{x}{P} & := & \binpar{{x}!\langle{\binpar{D_{x}}{P}}\rangle}{D_{x}} \nonumber
\end{eqnarray}

\begin{eqnarray}
	\bangp_{x}{P} & & \nonumber\\
	=
	& {x}!\langle{(\prefix{x}{y}{(\outputp{x}{y} | @{y})) | P}}\rangle 
	      | \prefix{x}{y}{(\outputp{x}{y} | @{y})} & \nonumber\\
	\red
	& (\outputp{x}{y} | @{y})\substn{\quotep{(\prefix{x}{y}{(@{y} | \outputp{x}{y})) | P}}}{y} & \nonumber\\
	=
	& \outputp{x}{\quotep{(\prefix{x}{y}{(\outputp{x}{y} | @{y})) | P}}}
	  | {(\prefix{x}{y}{(\outputp{x}{y} | @{y})) | P}} & \nonumber\\
	\red
	& \ldots & \nonumber\\
	\red^*
	& P | P | \ldots & \nonumber
\end{eqnarray}

Of course, this encoding, as an implementation, runs away, unfolding
$\bangp{P}$ eagerly. A lazier and more implementable replication
operator, restricted to input-guarded processes, may be obtained as follows.

\begin{eqnarray}
\bangp{\prefix{u}{v}{P}} 
	:= 
	\binpar{\lift{x}{\prefix{u}{v}{(\binpar{D(x)}{P})}}}{D(x)} \nonumber
\end{eqnarray}

\begin{remark}
  Note that the lazier definition still does not deal with summation
  or mixed summation (i.e. sums over input and output). The reader is
  invited to construct definitions of replication that deal with these
  features. 

  Further, the definitions are parameterized in a name, $x$. Can you,
  gentle reader, make a definition that eliminates this parameter and
  guarantees no accidental interaction between the replication
  machinery and the process being replicated -- i.e. no accidental
  sharing of names used by the process to get its work done and the
  name(s) used by the replication to effect copying. This latter
  revision of the definition of replication is crucial to obtaining
  the expected identity $!!P \sim !P$.
\end{remark}

\begin{remark}\label{rem:paradoxical_combinator}
  The reader familiar with the lambda calculus will have noticed the
  similarity between $D$ and the paradoxical combinator.

  [Ed. note: the existence of this seems to suggest we have to be more
  restrictive on the set of processes and names we admit if we are to
  support no-cloning.]
\end{remark}

\subsubsection{Bisimulation}

The computational dynamics gives rise to another kind of equivalence,
the equivalence of computational behavior. As previously mentioned
this is typically captured \emph{via} some form of bisimulation.

% The notion we use in this paper is weak barbed bisimulation
% \cite{milner91polyadicpi}.

The notion we use in this paper is derived from weak barbed
bisimulation \cite{milner91polyadicpi}. 

\begin{definition}
An \emph{observation relation}, $\downarrow_{\mathcal N}$, over a set
of names, $\mathcal N$, is the smallest relation satisfying the rules
below.

\infrule[Out-barb]{y \in {\mathcal N}, \; x \nameeq y}
		  {\outputp{x}{v} \downarrow_{\mathcal N} x}
\infrule[Par-barb]{\mbox{$P\downarrow_{\mathcal N} x$ or $Q\downarrow_{\mathcal N} x$}}
		  {\binpar{P}{Q} \downarrow_{\mathcal N} x}

We write $P \Downarrow_{\mathcal N} x$ if there is $Q$ such that 
$P \wred Q$ and $Q \downarrow_{\mathcal N} x$.
\end{definition}

\begin{definition}
%\label{def.bbisim}
An  ${\mathcal N}$-\emph{barbed bisimulation} over a set of names, ${\mathcal N}$, is a symmetric binary relation 
${\mathcal S}_{\mathcal N}$ between agents such that $P\rel{S}_{\mathcal N}Q$ implies:
\begin{enumerate}
\item If $P \red P'$ then $Q \wred Q'$ and $P'\rel{S}_{\mathcal N} Q'$.
\item If $P\downarrow_{\mathcal N} x$, then $Q\Downarrow_{\mathcal N} x$.
\end{enumerate}
$P$ is ${\mathcal N}$-barbed bisimilar to $Q$, written
$P \wbbisim_{\mathcal N} Q$, if $P \rel{S}_{\mathcal N} Q$ for some ${\mathcal N}$-barbed bisimulation ${\mathcal S}_{\mathcal N}$.
\end{definition}

$\mathcal{R} \subseteq \pi \times \pi$

$P \mathcal{R} Q => \forall P'. P \red P' \Rightarrow \exists Q'. Q \red Q', P' \mathcal{R} Q'$

$P \vdash x \Rightarrow Q \vdash x$

\begin{mathpar}
  \inferrule*[lab=Out-barb]{x \nameeq y}{{y}!\langle{Q}\rangle \vdash x}
  \and
  \inferrule*[lab=Par-barb]{\mbox{$P\vdash x$ or $Q\vdash x$}}{\binpar{P}{Q} \vdash x}
\end{mathpar}

\subsubsection{Contexts}

One of the principle advantages of computational calculi like the
$\pi$-calculus is a well-defined notion of context,
contextual-equivalence and a correlation between
contextual-equivalence and notions of bisimulation. The notion of
context allows the decomposition of a process into (sub-)process and
its syntactic environment, its context. Thus, a context may be
thought of as a process with a ``hole'' (written $\Box$) in it. The
application of a context $M$ to a process $P$, written $M[P]$, is
tantamount to filling the hole in $M$ with $P$. In this paper we do
not need the full weight of this theory, but do make use of the notion
of context in the proof the main theorem. 

\begin{mathpar}
  \inferrule* [lab=summation] {} {{M_{M},M_{N}} \bc \Box \;|\; x.M_{A} \;|\; M_{M}+M_{N}}
  \and
  \inferrule* [lab=agent] {} {{M_{A}} \bc (\vec{x})M_{P} \;| \; \clift{P_0,\ldots,M_{P},\ldots,P_N}}
  \and \\
  \inferrule* [lab=process] {} {{M_{P}} \bc M_{N} \;| \;P|M_{P} }
\end{mathpar} 

\begin{mathpar}
  \inferrule* [lab=sychronization] {} {M_{N} \bc \Box \;|\; x?M_{F} \;|\; x!M_{C}}
  \and
  \inferrule* [lab=abstraction] {} {{M_{F}} \bc (x)M_{P} }
  \and
  \inferrule* [lab=concretion] {} {{M_{C}} \bc \langle M_{P} \rangle }
  \and \\
  \inferrule* [lab=process] {} {{M_{P}} \bc M_{N} \;| \;P|M_{P} }
\end{mathpar}

\begin{definition}[contextual application] Given a context $M$, and
  process $P$, we define the \emph{contextual application}, $M[P] :=
  M\{P/\Box\}$. That is, the contextual application of M to P is the
  substitution of $P$ for $\Box$ in $M$.
\end{definition}

$\meaningof{-} : L \to \mathcal{P}(\pi)$

\begin{mathpar}
  \inferrule* [lab=collection] {} {\meaningof{true} = \pi, \and \meaningof{~E} = \pi \setminus \meaningof{E}, \and \meaningof{E_{1} \& E_{2}} = \meaningof{E_{1}} \cap \meaningof{E_{2}}}
\end{mathpar}

\begin{mathpar}
  \inferrule* [lab=structure] {} {\meaningof{0} = \{ P \in \pi | P \equiv 0 \}, \and \\ \meaningof{E_1 | E_2} = \{ P \in \pi | P \equiv P_{1} | P_{2}, P_{1} \in \meaningof{E_{1}}, P_{2} \in \meaningof{E_2}\} }
\end{mathpar}

\begin{mathpar}
 \inferrule* [lab=behavior] {} {\meaningof{\langle a?b \rangle E} = \{ P \in \pi | P \equiv Q | u?(y)P', \\ \and \\\\ \and \\ \;\;\; u \in \meaningof{a}, \forall z.P'\{z/y\} \in \meaningof{E\{z/b\}}\}, \and \\ \meaningof{a!E} = \{ P \in \pi | P \equiv Q | x!\langle P' \rangle, x \in \meaningof{a} P' \in \meaningof{E}\} }
\end{mathpar}

\begin{mathpar}
 \inferrule* [lab=nominal] {} {\meaningof{\quotep{E}} = \{ \quotep{P} \in \quotep{\pi} | P \in \meaningof{E} \}, \and \meaningof{\quotep{P}} = \{ \quotep{Q} \in \quotep{\pi} | P \equiv Q \} \and \\ \meaningof{@\quotep{E}} = \{ P \in \pi | P \equiv @x, x \in \meaningof{E} \}}
\end{mathpar}

\begin{eqnarray*}
  \\
  \meaningof{-} : TS \to ST
\end{eqnarray*}

\begin{eqnarray*}
  \\
  L : TS \to ST
\end{eqnarray*}

\begin{eqnarray*}
  \\
  P \models E \iff P \in \meaningof{E}
\end{eqnarray*}

\begin{eqnarray*}
  P \approx_{L} Q \iff \forall E \in L. P \models E \iff Q \models E
\end{eqnarray*}

\begin{eqnarray*}
  P \approx_{K} Q
\end{eqnarray*}

\begin{eqnarray*}
  P \approx Q
\end{eqnarray*}

$\approx_{K} = \approx = \approx_{L}$

\subsubsection{Contextual duality}

Note that contexts extend the quotation operation to a family of
operations from processes to names. Given a context, $M$, we can
define a \emph{nominal context}, $\quotep{M}$ by $\quotep{M}[P] :=
\quotep{M[P]}$. To foreshadow what is to come we observe that these
operations enjoy a duality with processes very much like the duality
between vectors and maps from vectors to scalars.

Further, because the calculus is essentially higher-order, we have a
correspondence between contexts and processes. More specifically,
given a name $x$ and a context $M$ we can construct $M^{*}_{x}$ such
that 

\begin{mathpar}
  M^{*}_{x} | \lift{x}{P} \red M[P]
\end{mathpar}

namely,

\begin{mathpar}
  M^{*}_{x} := x?(u).M[\dropn{u}]
\end{mathpar}

The dependence of $M^{*}_{x}$ on a name makes it an abstraction, 

\begin{mathpar}
  M^{*} := (x)x?(u).M[\dropn{u}]
\end{mathpar}

\subsection{Additional notation}

It will sometimes be convenient to denote the process a name
quotes. We already have the notation $x = \quotep{P}$, but it will be
convenient to introduce an alternate notation, $\procn{x}$, when we
want to emphasize the connection to the use of the name. Note that, by
virtue of name equivalence, $\quotep{\procn{x}} \nameeq x$; so, the
notation is consistent with previous definitions.

Further, because names have structure it is possible to effect
substitutions on the basis of that structure. This means we need to
upgrade our notation for substitutions, which we accomplish by
adapting comprehension notation. Thus,

\begin{mathpar}
  P\{ y / x : x \in S \}
\end{mathpar}

is interpreted to mean the process derived from P by replacing (in a
capture-avoiding manner) each occurrence of $x$ in $S$ by $y$. For example,

\begin{mathpar}
  P\{ \quotep{\procn{x}|\procn{x}} / x : x \in \freenames{P} \}
\end{mathpar}

will replace each (occurrence) of a free name $x$ in $P$ by
$\quotep{\procn{x}|\procn{x}}$.

Also, we will avail ourselves of the notation $x^{L}$ and $x^{R}$ to
denote injections of a name into disjoint copies of the name
space. There are numerous ways to accomplish this. One example can be
found in \cite{MeredithR05}. This notation overloads to vectors of
names: $\vec{x}^{\pi} := (x_{i}^{\pi} \; : \; 0 \leq i < |\vec{x}| )$ where $\pi \in \{L,R\}$.

We also use $P^{\Box} := P|\Box$.

In \cite{MeredithR05} an interpretation of the new operator is
given. It turns out that there are several possible interpretations
all enjoying the requisite algebraic properties of the operator (see
\cite{milner91polyadicpi}). We will therefore make liberal use of
$(\nu\; \vec{x})P$.

% subsection the_syntax_and_semantics_of_the_notation_system (end)   

\section{Interpretation of QM}
\subsection{Supporting definitions}
\subsubsection{Multiplication}
\begin{mathpar}
  \quotep{Q} \cdot \quotep{R} := \quotep{Q|R}
  \and \\
  \quotep{Q} \cdot P := P\{ \quotep{Q|R} / \quotep{R} : \quotep{R} \in \freenames{P} \}
\end{mathpar}

\paragraph{Discussion}
The first line needs little explanation. The second line says that
each free name of the process is replaced with the multiplication of
that name by the scalar. Multiplication of a scalar (name) by a state
(process) results in a process all the names of which have been `moved
over' by parallel composition with the process the scalar
quotes. There is a subtlety that the bound names have to be
manipulated so that multiplied names aren't accidentally
captured. There are many ways to achieve this.

\begin{remark}\label{rem:multiplication_identities}
  The reader is invited to verify that for all $x,y,z \in \QProc$ and $P \in \Proc$
  \begin{mathpar}
    x \cdot \quotep{0} \equiv x 
    \and
    x \cdot y \equiv y \cdot x
    \and
    x \cdot (y \cdot z) \equiv (x \cdot y) \cdot z
    \and \\
    \quotep{0} \cdot P \equiv P
    \and \\
    x \cdot (y \cdot P) \equiv (x \cdot y) \cdot P
    \and \\
    x \cdot (P|Q) \equiv (x \cdot P) | (x \cdot Q)
    \and \\    
  \end{mathpar}
\end{remark}

\subsubsection{Tensor product}

We define a tensor product on processes by structural induction.

\paragraph{Tensor of sums} First note that all summations, including
$\pzero$ and sequence, can be written $\Sigma_{i} x_{i}.A_{i} +
\Sigma_{j} x_{j}.C_{j}$, where we have grouped input-guarded processes
together and output-guarded processes together.

Thus, we can define the tensor product of two summations, $N_{1}\otimes N_{2}$, where

\begin{mathpar}
  N_{1} := \Sigma_{i} x_{i}.A_{i} + \Sigma_{j} x_{j}.C_{j}
  \and
  N_{2} := \Sigma_{i'} y_{i'}.B_{i'} + \Sigma_{j'} y_{j'}.D_{j'} 
\end{mathpar}

as follows.

\begin{mathpar}
  \Sigma_{i} x_{i}.A_{i} + \Sigma_{j} x_{j}.C_{j} \otimes \Sigma_{i'}
  y_{i'}.B_{i'} + \Sigma_{j'} y_{j'}.D_{j'} 
  \and \\
  := \; \Sigma_{i} \Sigma_{i'} \quotep{\stackrel{\vee}{x_{i}}| \stackrel{\vee}{y_{i'}}}.(A_{i}\otimes B_{i'}) \; | \; \Sigma_{i'} \Sigma_{i} \quotep{\stackrel{\vee}{y_{i'}}|\stackrel{\vee}{x_{i}}}.(B_{i'}\otimes A_{i})
  \and
  \;\; | \;\; \Sigma_{j} \Sigma_{j'} \quotep{\stackrel{\vee}{x_{j}}|\stackrel{\vee}{y_{j'}}}.(A_{j}\otimes B_{j'}) \; | \; \Sigma_{j'} \Sigma_{j} \quotep{\stackrel{\vee}{y_{j'}}|\stackrel{\vee}{x_{j}}}.(B_{j'}\otimes A_{j})
\end{mathpar}

\begin{remark}
  Do we need to $x^{L}$ and $y^{R}$ for this construction as well?
\end{remark}

\paragraph{Tensor of parallel compositions} Next, we distribute tensor
over par.

\begin{mathpar}
  P_{1}|P_{2} \otimes Q_{1}|Q_{2} := (P_{1} \otimes Q_{1}) | (P_{1}
  \otimes Q_{2}) | (P_{2} \otimes Q_{1}) | (P_{2} \otimes Q_{2})
\end{mathpar}

\paragraph{Tensor with dropped names} We treat tensor of a
process with a dropped name as parallel composition.

\begin{mathpar}
  P \otimes \dropn{x} := P | \dropn{x}
\end{mathpar}

\paragraph{Tensor of agents}

Finally, we need to define tensor on agents. Note that the definition
of tensor on normal products only tensors inputs with inputs and
outputs with outputs. Thus, we only have to define the operation on
``homogeneous'' pairings.

\begin{mathpar}
  (\vec{x})P \otimes (\vec{y})Q
  \and \\
  := (x_{0}^{L}|y_{0}^{R},\ldots,x_{0}^{L}|y_{n}^{R},\ldots,x_{m}^{L}|y_{0}^{R},\ldots,x_{m}^{L}|y_{n}^R)(P\{ \vec{x}^{L}/\vec{x}\} \otimes Q \{ \vec{y}^{R}/\vec{y}\})
  \and \\
  \clift{\vec{P}} \otimes \clift{\vec{Q}}
  \and \\
  := \clift{P_{0}\otimes Q_{0},\ldots,P_{0}\otimes Q_{n},\ldots,P_{m}\otimes Q_{0},\ldots,P_{m}\otimes Q_{n}}
\end{mathpar}

\begin{remark}
  Observe that arities of tensored abstractions matches arities of
  tensored concretions if the original arities matched. Note also that
  the length of the arities corresponds to the increase in dimension
  we see in ordinary vector space tensor product.
\end{remark}

\begin{remark}
  Operationally, this definition distributes the tensor down to
  components ``linked'' by summation. Tensor over summation is
  intriguing in that it mixes names. Moreover, as a consequence of the
  way it mixes names we have the identities for all $x \in \QProc$ and
  $P,Q \in \Proc$

  \begin{mathpar}
    (x \cdot P) \otimes Q \equiv x \cdot (P \otimes Q) \equiv P \otimes (x \cdot Q)
    \and
    P \otimes \pzero \equiv P
  \end{mathpar}

  that the reader is invited to verify.
\end{remark}

\subsubsection{Annihilation}
\begin{mathpar}
  P^{\perp} := \{ Q | \forall R. P|Q \red^{*} R \Rightarrow R \red^{*} \pzero \}
  \and \\
  P^{\underline{\perp}} := \Sigma_{Q \in P^{\perp}} \quotep{Q}?(y).(\dropn{y}|Q) | \Sigma_{Q \in P^{\perp}} \quotep{Q}\clift{\Box}
\end{mathpar}

\paragraph{Discussion} The reader will note that $P^{\perp}$ is a
\emph{set} of processes, while $P^{\underline{\perp}}$ is a
\emph{context}. We call the set $P^{\perp}$ the \emph{annihilators} of
$P$. The parallel composition of a process in the annihilators of $P$
with $P$ will result in a process, the state space of which has all
paths eventually leading to $\pzero$. Execution may endure loops; but
under reasonable conditions of fairness (naturally guaranteed under
most notions of bisimulation) such a composite process cannot get
stuck in such a loop and will, eventually pop out and terminate.

The context $P^{\underline{\perp}}$ is ready and willing to ``take the
$P$ out of'' the process to which it is applied. It will effectively
transmit the code of the process to which it is applied to one of the
annihilators and run the process against it.

\subsubsection{Evaluation}
We fix $M$ a domain of fully abstract interpretation with an equality
coincident with bisimulation. We take $\meaningof{\cdot} : \Proc \to
M$ to be the map interpreting processes and $\nmeaningof{\cdot} : \M
\to Proc$ to be the map running the other way. Then we define

\begin{mathpar}
  \int P := \nmeaningof{\meaningof{P}}
\end{mathpar}

\paragraph{Discussion}
There are many fully abstract interpretations of Milner's
$\pi$-calculus. Any of them can be used as a basis for interpreting
the reflective calculus here. Equipped with such a domain it is
largely a matter of grinding through to check that the Yoneda
construction for the normalization-by-evaluation program can be
extended to this setting.

\begin{remark}
  The reader is invited to verify that $\int (P^{\underline{\perp}}[P]) = 0$.
\end{remark}

\subsection{Quantum mechanics}

Table \ref{tbl:core_qm_op_defns} gives the core operational definitions

\begin{table}[htp]\label{tbl:core_qm_op_defns}
  \center{
    \fbox{
      \begin{tabular}{c|c}
        quantum mechanics & process calculus \\
        \hline
        scalar & $x := \quotep{P}$ \\
        state vector & $\state{P} := P$ \\
        dual & $\state{P}^{*} := \event{P^{\underline{\perp}}} := \quotep{P^{\underline{\perp}}}[-]$ \\
        matrix & $ \Sigma_{\alpha} \state{P_{\alpha}}x_{\alpha}\event{Q_{\alpha}}$ \\
        vector addition & $\state{P} + \state{Q} := \state{P | Q}$ \\
        tensor product & $\state{P} \otimes \state{Q} := \state{P \otimes Q}$ \\
        inner product & $\innerprod{P}{Q} := \quotep{\int P^{\underline{\perp}}[Q]}$ \\
      \end{tabular}
    }
  }
  \caption{QM - operational definitions}
\end{table}

where

\begin{mathpar}
  \prmatrix{P}{Q} := \fprmatrix{P}{\quotep{\pzero}}{Q}
  \and
  \fprmatrix{P}{x}{Q} := (\state{P},x,\event{Q})
  \and
  (\fprmatrix{P}{x}{Q})(\state{R}) := x \cdot \innerprod{Q}{R} \cdot \state{P}
  \and
  (\fprmatrix{P}{x}{Q})(\event{R}) := x \cdot \innerprod{R}{P} \cdot \event{Q}
\end{mathpar}

\paragraph{Discussion}
As promised: vectors (aka states) are represented as processes; duals
as contextual duals; inner product definition should be compared with
standard inner product definition for ....

\begin{remark}
  Assuming $\int (P^{\underline{\perp}}[P]) = 0$, the reader is
  invited to verify that $(\fprmatrix{P}{x}{P})(\state{P}) = x \cdot \state{P}$.
\end{remark}

\begin{remark}
  The reader is invited to verify that $\innerprod{P}{Q}$ could
  equally well have been written $\quotep{\int \stackrel{\vee}{x}}$
  where $x = \event{P^{\underline{\perp}}}(Q)$.

  One of the motivations for this remark is that there is another way
  to factor these operations. We could package up evaluation in the dual:

  \begin{mathpar}
    \state{P}^{*} := \event{\int P^{\underline{\perp}}} := \quotep{\int P^{\underline{\perp}}}[-]
  \end{mathpar}

  and then have inner product defined by
  
  \begin{mathpar}
    \innerprod{P}{Q} := \event{P}(Q)
  \end{mathpar}

  Hopefully, experience with the calculations will provide guidance on
  the best factoring.
\end{remark}

\begin{remark}
  Assuming $\int (P^{\underline{\perp}}[P]) = 0$, the reader is
  invited to verify that $\forall P,Q. (\prmatrix{0}{Q})(\state{0}) =
  \state{0}$ and dually $(\prmatrix{P}{0})(\event{0}) = \event{0}$.
\end{remark}

\begin{remark}
  i'm a little worried that i don't (yet) have proper support for
  complex conjugacy. But, the observation above may give us a
  clue. According to Abramsky, it must be the case that the scalars
  are iso to the homset of the identity for the tensor -- which the
  observation above characterizes. 

  For now, we will simply bookmark the notion with $\overline{x}$.
\end{remark}

\subsubsection{Adjointness}

We need to give a definition of $(\cdot)^{\dagger}$ for matrices. The
obvious candidate definition is
\begin{mathpar}
(\Sigma_{\alpha}\fprmatrix{P_{\alpha}}{x_{\alpha}}{Q_{\alpha}})^{\dagger}
= \Sigma_{\alpha}\fprmatrix{(Q_{\alpha}^{\underline{\perp}})^{*}}{\overline{x}_{\alpha}}{P_{\alpha}^{\underline{\perp}}} 
\end{mathpar}

But, $(Q_{\alpha}^{\underline{\perp}})^{*}$ requires a name along
which to communicate the process to achieve the context application.

\subsubsection{Basis for a basis}
If processes label states and ``addition'' of states (a.k.a. vector
addition) is interpreted as parallel composition, what corresponds to
notions of linear independence and basis? Here, we recall that Yoshida
has developed a set of \emph{combinators} for an asynchronous verison
of Milner's $\pi$-calculus. These are a finite set of processes such
any process can be expressed as parallel composition of these
combinators together with liberal uses of the new operator and
replication. We can simply give a translation of these into the
present calculus and have reasonable expectation that the property
carries over. That is, that the resultant set allows to express all
processes via parallel composition. Note, however, that there is no
new operator or replication in this calculus. As a result, we expect
that the corresponding set is actually infinite. That is, we expect
that the space is actually infinite dimensional.

\begin{remark}
  The attentive reader may be a bit concerned. Certainly, the
  collection $S$, $K$ and $I$ is a finite set of
  combinators. Shouldn't we expect to see a finite set of combinators
  for an effectively equivalent system? i am very sympathetic to this
  critique and feel it warrants full attention. On the other hand, i
  also have in mind the following analogy. The natural numbers, as a
  monoid under addition, has exactly $1$ generator, while the natural
  numbers, as a monoid under multiplication, has countably many
  generators (the primes). We observe that the application of the
  lambda calculus is much less resource sensitive than the parallel
  composition of the $\pi$-calculus. Could it be the case that we have
  an analogy of the form
  
  \begin{mathpar}
    m + n : MN :: m*n : M|N
  \end{mathpar}

  giving a similar blow up in the set of ``primes''?  This is such a
  wonderful thought that, even if it's not true, i think it's worth
  writing down.
\end{remark}
 

\documentclass[12pt]{llncs}
%\documentclass{jktr}

\usepackage[pdftex]{hyperref}                   
\usepackage {listings}
\usepackage {mathpartir}
\usepackage{bcprules}
%\usepackage{listings}
                       
\usepackage{graphicx} 
%\usepackage[margins=2.5cm,nohead,nofoot]{geometry}
%\usepackage{geometry}
\usepackage{amsfonts}
\usepackage{amstext}
\usepackage{latexsym}
\usepackage{amssymb}
\usepackage{color}


%\include{myPreamble}
\include{qm2pi.local} 

%\ifpdf
%\usepackage[pdftex]{graphicx}
%\else
%\usepackage{graphicx}
%\fi

 % \ifpdf
%  \usepackage{pdfsync}
%  \if


%\title{Brief Article}
%\author{David F. Snyder}
%\author{L.G. Meredith}

%\address{Dept. of Math., Texas State University--San Marcos, San Marcos, TX 78666}
       
\pagestyle{empty}


\begin{document}

\lstset{language=[Objective]Caml,frame=shadowbox}

\input{qm2pi.front}

% section front matter (end)

\input{qm2pi.intro} 
 
% section introduction (end)

% \input{qm2pi.knotations} 

% section notation (end)

\input{qm2pi.process.calculi} 

% section concurrent_process_calculi_and_spatial_logics_ (end)
    
%\input{qm2pi.knots2pi} 

%\input{qm2pi.trefoil} 

%\input{qm2pi.mainthm} 

% subsection basic_interpretation (end)

%\input{qm2pi.rho.presentation} 
\subsection{The syntax and semantics of the notation system}\label{sub:the_syntax_and_semantics_of_the_notation_system} % (fold)

We now summarize a technical presentation of the calculus that
embodies our theory of dynamics. The typical presentation of such a
calculus follows the style of giving generators and relations on
them. The grammar, below, describing term constructors, freely
generates the set of processes, $\Proc$. This set is then quotiented
by a relation known as structural congruence and it is over this set
that the notion of dynamics is expressed. This presentation is
essentially that of \cite{MeredithR05} with the addition of
polyadicity and summation. For readability we have relegated some of
the technical subtleties to an appendix.

\subsubsection{Process grammar}\label{subsub:process_grammar}

\begin{mathpar}
  \inferrule* [lab=synchronization] {} {{M} \bc \pzero \;|\; x?F \;|\; x!C }
  \and
  \inferrule* [lab=abstraction] {} {{F} \bc (x)P}
  \and
  \inferrule* [lab=concretion] {} {{C} \bc \langle Q \rangle}
  \and
  \inferrule* [lab=process] {} {{P,Q} \bc M \;| \;P|Q \;|\; @{x}}
  \and
  \inferrule* [lab=name] {} {{x} \bc \quotep{P}}
\end{mathpar} 

Note that $\vec{x}$ (resp. $\vec{P}$) denotes a vector of names
(resp. processes) of length $|\vec{x}|$ (resp. $|\vec{P}|$). We adopt
the following useful abbreviations.

\begin{mathpar}
   x?(\vec{y}).P := x.(\vec{y})P \and  x\clift{\vec{P}} := x.\clift{\vec{P}}
   \and x!(y) := \lift{x}{\dropn{y}}
   \and \Pi_{i=0}^{n-1}P_i := P_0 | \ldots | P_{n-1}
\end{mathpar}

\subsubsection{Structural congruence}

\paragraph{Free and bound names and alpha-equivalence.} At the
core of structural equivalence is alpha-equivalence which identifies
process that are the same up to a change of variable. Formally, we
recognize the distinction between free and bound names. The free names
of a process, $\freenames{P}$, may be calculated recursively as
follows:

\begin{mathpar}
\freenames{\pzero} := \emptyset
  \and \\
  \freenames{x?(y).P} := \{ x \} \cup (\freenames{P} \setminus \{ y \})
  \and 
  \freenames{x!\langle P \rangle} := \{ x \} \cup \{ P \} 
  \and \\
  \freenames{P|Q} := \freenames{P} \cup \freenames{Q}
  \and \\
  \freenames{@{x}} := \{ x \}
\end{mathpar}

$\pi$
$\quotep{\pi}$

$\freenames{-} : \pi \to \mathcal{P}(\quotep{\pi})$

\begin{eqnarray*}
  \freenames{\pzero} & := & \emptyset \\
  \freenames{x?(y).P} & := & \{ x \} \cup (\freenames{P} \setminus \{ y \}) \\
  \freenames{x!\langle P \rangle} & := & \{ x \} \cup \{ P \} \\
  \freenames{P|Q} & := & \freenames{P} \cup \freenames{Q} \\
  \freenames{\dropn{x}} & := & \{ x \}
\end{eqnarray*}

The bound names of a process, $\boundnames{P}$, are those names occurring in $P$
that are not free. For example, in $x?(y).0$, the name $x$ is free, while $y$ is bound.

\begin{mathpar}
  \inferrule* [lab=monoidal-laws] {} { P|Q \equiv Q|P \and P|0 \equiv P \and P|(Q|R) \equiv (P|Q)|R }
\end{mathpar}

\begin{mathpar}
  \inferrule* [lab=alpha-equivalence] {} { (x)P \equiv (y)P\{y/x\} \and y \not\in \freenames{P} }
\end{mathpar}

\begin{definition}
Then two processes, $P,Q$, are alpha-equivalent if $P = Q\{\vec{y}/\vec{x}\}$ for
some $\vec{x} \in \boundnames{Q},\vec{y} \in \boundnames{P}$, where $Q\{\vec{y}/\vec{x}\}$
denotes the capture-avoiding substitution of $\vec{y}$ for $\vec{x}$ in $Q$.
\end{definition}

\begin{definition}
  The {\em structural congruence} \cite{SangiorgiWalker} , $\equiv$,
  between processes is the least congruence containing
  alpha-equivalence, satisfying the abelian monoid laws
  (associativity, commutativity and $\pzero$ as identity) for parallel
  composition $|$ and for summation $+$.
\end{definition}

\subsection{Name equivalence}

We take name equivalence, written $\nameeq$, to be the smallest
equivalence relation generated by the following rules.

\begin{mathpar}
\inferrule*[lab=Quote-drop]
{ }
{ \quotep{@{x}} \nameeq x }

\inferrule*[lab=Struct-equiv]
{ P \scong Q }
{ \quotep{P} \nameeq \quotep{Q} }
\end{mathpar}

The astute reader will have noticed that the mutual recursion of names
and processes imposes a mutual recursion on alpha-equivalence and
structural equivalence via name-equivalence. Fortunately, all of this
works out pleasantly and we may calculate in the natural way, free of
concern. The reader interested in the details is referred to the
appendix \ref{appendix:rho_details}.

\subsection{Substitution}

We use $\Proc$ for the set of processes, $\QProc$ for the set of
names, and $\id{\{}\vec{y} / \vec{x} \id{\}}$ to denote partial maps,
$s : \QProc \rightarrow \QProc$. A map, $s$ lifts, uniquely, to a map
on process terms, $\widehat{s} : \Proc \rightarrow \Proc$ by the
following equations.

\begin{mathpar}
  (0) \psubstp{Q}{P} := 0 \\
  (R \juxtap S) \psubstp{Q}{P}
  :=    
  (R)\psubstp{Q}{P} \juxtap (S) \psubstp{Q}{P} \\
  (x?(y).R) \psubstp{Q}{P}    
  :=    
  (x)\substp{Q}{P} (z)\concat( (R \psubstn{z}{y}) \psubstp{Q}{P} ) \\
  (\lift{x}{R}) \psubstp{Q}{P}  
  :=
  \lift{(x)\substp{Q}{P}}{ R \psubstp{Q}{P} } \\
%   (\dropn{x})  \psubstp{Q}{P}       
%   := 
%   \left\{ 
%     \begin{array}{ccc} 
%       \dropn{\quotep{Q}} & & x \nameeq \quotep{P} \\
%       \dropn{x} & & otherwise \\
%     \end{array}
%   \right. 
  (\dropn{x})  \psubstp{Q}{P}       
  := 
  \left\{ 
    \begin{array}{ccc} 
      Q & & x \nameeq \quotep{P} \\
      \dropn{x} & & otherwise \\
    \end{array}
  \right.
\end{mathpar}
 

where

\begin{eqnarray}
  (x)\id{\{} \lpquote Q \rpquote / \lpquote P \rpquote \id{\}}            = 
  \left\{ 
    \begin{array}{ccc}
      \lpquote Q \rpquote & & x \nameeq \lpquote P \rpquote \\
      x & & otherwise \\
    \end{array}
  \right. \nonumber
\end{eqnarray}

and $z$ is chosen distinct from $\quotep{P}$, $\quotep{Q}$, the free
names in $Q$, and all the names in $R$. Our $\alpha$-equivalence will
be built in the standard way from this substitution.

\begin{remark}\label{rem:no_self_referential_names}
  One consequence of these definitions is that $\forall P. \quotep{P}
  \not\in \freenames{P}$.
\end{remark}

\subsection{ Dynamic quote: an example }

Anticipating something of what's to come, consider applying the
substitution, $\widehat{\id{\{}u / z \id{\}}}$, to the following pair
of processes, $\lift{w}{y!(z)}$ and $w[ \lpquote y!(z) \rpquote ]$.

\begin{eqnarray}
	\lift{w}{y!(z)}\widehat{\id{\{}u / z \id{\}}}
		& = &
		\lift{w}{y!(u)} \nonumber\\
	w[ \lpquote y!(z) \rpquote ] \widehat{ \id{\{}u / z \id{\}} }
		& = &
		w[ \lpquote y!(z) \rpquote ] \nonumber
\end{eqnarray}

Because the body of the process between quotes is impervious to
substitution, we get radically different answers. In fact, by
examining the first process in an input context,
e.g. $x?(z).\lift{w}{y!(z)}$, we see that the process under the lift
operator may be shaped by prefixed inputs binding a name inside it. In
this sense, the lift operator will be seen as a way to dynamically
construct processes before reifying them as names.

Finally equipped with these standard features we can present the
dynamics of the calculus.

\subsubsection{Operational semantics} 

Finally, we introduce the computational dynamics. What marks these
algebras as distinct from other more traditionally studied algebraic
structures, e.g. vector spaces or polynomial rings, is the manner in
which dynamics is captured. In traditional structures, dynamics is typically
expressed through morphisms between such structures, as in linear maps
between vector spaces or morphisms between rings. In algebras
associated with the semantics of computation, the dynamics is
expressed as part of the algebraic structure itself, through a
reduction reduction relation typically denoted by $\red$. Below, we
give a recursive presentation of this relation for the calculus used
in the encoding.

$\red \subseteq \pi \times \pi$
$\red : \pi \to \mathcal{P}(\pi)$

\begin{mathpar}
  \inferrule* [lab=Comm] { \textsf{match}( x_{src}, x_{trgt} ) } { x_{trgt}?(y)P \; | \; x_{src}!\langle {Q} \rangle \red P\{\quotep{Q}/y}\} }
  \and \\
  \inferrule* [lab=Par] {{P} \red {P}'} {{{P} | {Q}} \red {{P}' | {Q}}}
  \and
  \inferrule* [lab=Equiv]{{{P} \scong {P}'} \andalso {{P}' \red {Q}'} \andalso {{Q}' \scong {Q}}}{{P} \red {Q}}
\end{mathpar}

\begin{eqnarray*}
  match_{\equiv} (\quotep{P},\quotep{Q}) & := & P \equiv Q \\
  match_{\dagger}(\quotep{P},\quotep{Q}) & := & \forall R. P|Q \red^{*} R => R \red^{*} 0 \\
  match_{K}(\quotep{P},\quotep{Q}) & := & K \mbox{ for some context } K
\end{eqnarray*}

$u?(x)P | u!\langle Q \rangle \red P\{\quotep{Q}/x\}$

%We write $\wred$ for $\red^*$, and $P\red$ if $\exists Q $ such that $ P \red Q$.
We write $P\red$ if $\exists Q $ such that $ P \red Q$ and $P\not\red$, otherwise.

\section{Replication}

As mentioned before, it is known that replication (and hence
recursion) can be implemented in a higher-order process algebra
\cite{SangiorgiWalker}. As our first example of calculation with the
machinery thus far presented we give the construction explicitly in
the {\rhoc}.

\begin{eqnarray}
	D_{x} & := & \prefix{x}{y}{(\binpar{\outputp{x}{y}}{@{y}})} \nonumber\\
	\bangp_{x}{P} & := & \binpar{{x}!\langle{\binpar{D_{x}}{P}}\rangle}{D_{x}} \nonumber
\end{eqnarray}

\begin{eqnarray}
	\bangp_{x}{P} & & \nonumber\\
	=
	& {x}!\langle{(\prefix{x}{y}{(\outputp{x}{y} | @{y})) | P}}\rangle 
	      | \prefix{x}{y}{(\outputp{x}{y} | @{y})} & \nonumber\\
	\red
	& (\outputp{x}{y} | @{y})\substn{\quotep{(\prefix{x}{y}{(@{y} | \outputp{x}{y})) | P}}}{y} & \nonumber\\
	=
	& \outputp{x}{\quotep{(\prefix{x}{y}{(\outputp{x}{y} | @{y})) | P}}}
	  | {(\prefix{x}{y}{(\outputp{x}{y} | @{y})) | P}} & \nonumber\\
	\red
	& \ldots & \nonumber\\
	\red^*
	& P | P | \ldots & \nonumber
\end{eqnarray}

Of course, this encoding, as an implementation, runs away, unfolding
$\bangp{P}$ eagerly. A lazier and more implementable replication
operator, restricted to input-guarded processes, may be obtained as follows.

\begin{eqnarray}
\bangp{\prefix{u}{v}{P}} 
	:= 
	\binpar{\lift{x}{\prefix{u}{v}{(\binpar{D(x)}{P})}}}{D(x)} \nonumber
\end{eqnarray}

\begin{remark}
  Note that the lazier definition still does not deal with summation
  or mixed summation (i.e. sums over input and output). The reader is
  invited to construct definitions of replication that deal with these
  features. 

  Further, the definitions are parameterized in a name, $x$. Can you,
  gentle reader, make a definition that eliminates this parameter and
  guarantees no accidental interaction between the replication
  machinery and the process being replicated -- i.e. no accidental
  sharing of names used by the process to get its work done and the
  name(s) used by the replication to effect copying. This latter
  revision of the definition of replication is crucial to obtaining
  the expected identity $!!P \sim !P$.
\end{remark}

\begin{remark}\label{rem:paradoxical_combinator}
  The reader familiar with the lambda calculus will have noticed the
  similarity between $D$ and the paradoxical combinator.

  [Ed. note: the existence of this seems to suggest we have to be more
  restrictive on the set of processes and names we admit if we are to
  support no-cloning.]
\end{remark}

\subsubsection{Bisimulation}

The computational dynamics gives rise to another kind of equivalence,
the equivalence of computational behavior. As previously mentioned
this is typically captured \emph{via} some form of bisimulation.

% The notion we use in this paper is weak barbed bisimulation
% \cite{milner91polyadicpi}.

The notion we use in this paper is derived from weak barbed
bisimulation \cite{milner91polyadicpi}. 

\begin{definition}
An \emph{observation relation}, $\downarrow_{\mathcal N}$, over a set
of names, $\mathcal N$, is the smallest relation satisfying the rules
below.

\infrule[Out-barb]{y \in {\mathcal N}, \; x \nameeq y}
		  {\outputp{x}{v} \downarrow_{\mathcal N} x}
\infrule[Par-barb]{\mbox{$P\downarrow_{\mathcal N} x$ or $Q\downarrow_{\mathcal N} x$}}
		  {\binpar{P}{Q} \downarrow_{\mathcal N} x}

We write $P \Downarrow_{\mathcal N} x$ if there is $Q$ such that 
$P \wred Q$ and $Q \downarrow_{\mathcal N} x$.
\end{definition}

\begin{definition}
%\label{def.bbisim}
An  ${\mathcal N}$-\emph{barbed bisimulation} over a set of names, ${\mathcal N}$, is a symmetric binary relation 
${\mathcal S}_{\mathcal N}$ between agents such that $P\rel{S}_{\mathcal N}Q$ implies:
\begin{enumerate}
\item If $P \red P'$ then $Q \wred Q'$ and $P'\rel{S}_{\mathcal N} Q'$.
\item If $P\downarrow_{\mathcal N} x$, then $Q\Downarrow_{\mathcal N} x$.
\end{enumerate}
$P$ is ${\mathcal N}$-barbed bisimilar to $Q$, written
$P \wbbisim_{\mathcal N} Q$, if $P \rel{S}_{\mathcal N} Q$ for some ${\mathcal N}$-barbed bisimulation ${\mathcal S}_{\mathcal N}$.
\end{definition}

$\mathcal{R} \subseteq \pi \times \pi$

$P \mathcal{R} Q => \forall P'. P \red P' \Rightarrow \exists Q'. Q \red Q', P' \mathcal{R} Q'$

$P \vdash x \Rightarrow Q \vdash x$

\begin{mathpar}
  \inferrule*[lab=Out-barb]{x \nameeq y}{{y}!\langle{Q}\rangle \vdash x}
  \and
  \inferrule*[lab=Par-barb]{\mbox{$P\vdash x$ or $Q\vdash x$}}{\binpar{P}{Q} \vdash x}
\end{mathpar}

\subsubsection{Contexts}

One of the principle advantages of computational calculi like the
$\pi$-calculus is a well-defined notion of context,
contextual-equivalence and a correlation between
contextual-equivalence and notions of bisimulation. The notion of
context allows the decomposition of a process into (sub-)process and
its syntactic environment, its context. Thus, a context may be
thought of as a process with a ``hole'' (written $\Box$) in it. The
application of a context $M$ to a process $P$, written $M[P]$, is
tantamount to filling the hole in $M$ with $P$. In this paper we do
not need the full weight of this theory, but do make use of the notion
of context in the proof the main theorem. 

\begin{mathpar}
  \inferrule* [lab=summation] {} {{M_{M},M_{N}} \bc \Box \;|\; x.M_{A} \;|\; M_{M}+M_{N}}
  \and
  \inferrule* [lab=agent] {} {{M_{A}} \bc (\vec{x})M_{P} \;| \; \clift{P_0,\ldots,M_{P},\ldots,P_N}}
  \and \\
  \inferrule* [lab=process] {} {{M_{P}} \bc M_{N} \;| \;P|M_{P} }
\end{mathpar} 

\begin{mathpar}
  \inferrule* [lab=sychronization] {} {M_{N} \bc \Box \;|\; x?M_{F} \;|\; x!M_{C}}
  \and
  \inferrule* [lab=abstraction] {} {{M_{F}} \bc (x)M_{P} }
  \and
  \inferrule* [lab=concretion] {} {{M_{C}} \bc \langle M_{P} \rangle }
  \and \\
  \inferrule* [lab=process] {} {{M_{P}} \bc M_{N} \;| \;P|M_{P} }
\end{mathpar}

\begin{definition}[contextual application] Given a context $M$, and
  process $P$, we define the \emph{contextual application}, $M[P] :=
  M\{P/\Box\}$. That is, the contextual application of M to P is the
  substitution of $P$ for $\Box$ in $M$.
\end{definition}

$\meaningof{-} : L \to \mathcal{P}(\pi)$

\begin{mathpar}
  \inferrule* [lab=collection] {} {\meaningof{true} = \pi, \and \meaningof{~E} = \pi \setminus \meaningof{E}, \and \meaningof{E_{1} \& E_{2}} = \meaningof{E_{1}} \cap \meaningof{E_{2}}}
\end{mathpar}

\begin{mathpar}
  \inferrule* [lab=structure] {} {\meaningof{0} = \{ P \in \pi | P \equiv 0 \}, \and \\ \meaningof{E_1 | E_2} = \{ P \in \pi | P \equiv P_{1} | P_{2}, P_{1} \in \meaningof{E_{1}}, P_{2} \in \meaningof{E_2}\} }
\end{mathpar}

\begin{mathpar}
 \inferrule* [lab=behavior] {} {\meaningof{\langle a?b \rangle E} = \{ P \in \pi | P \equiv Q | u?(y)P', \\ \and \\\\ \and \\ \;\;\; u \in \meaningof{a}, \forall z.P'\{z/y\} \in \meaningof{E\{z/b\}}\}, \and \\ \meaningof{a!E} = \{ P \in \pi | P \equiv Q | x!\langle P' \rangle, x \in \meaningof{a} P' \in \meaningof{E}\} }
\end{mathpar}

\begin{mathpar}
 \inferrule* [lab=nominal] {} {\meaningof{\quotep{E}} = \{ \quotep{P} \in \quotep{\pi} | P \in \meaningof{E} \}, \and \meaningof{\quotep{P}} = \{ \quotep{Q} \in \quotep{\pi} | P \equiv Q \} \and \\ \meaningof{@\quotep{E}} = \{ P \in \pi | P \equiv @x, x \in \meaningof{E} \}}
\end{mathpar}

\begin{eqnarray*}
  \\
  \meaningof{-} : TS \to ST
\end{eqnarray*}

\begin{eqnarray*}
  \\
  L : TS \to ST
\end{eqnarray*}

\begin{eqnarray*}
  \\
  P \models E \iff P \in \meaningof{E}
\end{eqnarray*}

\begin{eqnarray*}
  P \approx_{L} Q \iff \forall E \in L. P \models E \iff Q \models E
\end{eqnarray*}

\begin{eqnarray*}
  P \approx_{K} Q
\end{eqnarray*}

\begin{eqnarray*}
  P \approx Q
\end{eqnarray*}

$\approx_{K} = \approx = \approx_{L}$

\subsubsection{Contextual duality}

Note that contexts extend the quotation operation to a family of
operations from processes to names. Given a context, $M$, we can
define a \emph{nominal context}, $\quotep{M}$ by $\quotep{M}[P] :=
\quotep{M[P]}$. To foreshadow what is to come we observe that these
operations enjoy a duality with processes very much like the duality
between vectors and maps from vectors to scalars.

Further, because the calculus is essentially higher-order, we have a
correspondence between contexts and processes. More specifically,
given a name $x$ and a context $M$ we can construct $M^{*}_{x}$ such
that 

\begin{mathpar}
  M^{*}_{x} | \lift{x}{P} \red M[P]
\end{mathpar}

namely,

\begin{mathpar}
  M^{*}_{x} := x?(u).M[\dropn{u}]
\end{mathpar}

The dependence of $M^{*}_{x}$ on a name makes it an abstraction, 

\begin{mathpar}
  M^{*} := (x)x?(u).M[\dropn{u}]
\end{mathpar}

\subsection{Additional notation}

It will sometimes be convenient to denote the process a name
quotes. We already have the notation $x = \quotep{P}$, but it will be
convenient to introduce an alternate notation, $\procn{x}$, when we
want to emphasize the connection to the use of the name. Note that, by
virtue of name equivalence, $\quotep{\procn{x}} \nameeq x$; so, the
notation is consistent with previous definitions.

Further, because names have structure it is possible to effect
substitutions on the basis of that structure. This means we need to
upgrade our notation for substitutions, which we accomplish by
adapting comprehension notation. Thus,

\begin{mathpar}
  P\{ y / x : x \in S \}
\end{mathpar}

is interpreted to mean the process derived from P by replacing (in a
capture-avoiding manner) each occurrence of $x$ in $S$ by $y$. For example,

\begin{mathpar}
  P\{ \quotep{\procn{x}|\procn{x}} / x : x \in \freenames{P} \}
\end{mathpar}

will replace each (occurrence) of a free name $x$ in $P$ by
$\quotep{\procn{x}|\procn{x}}$.

Also, we will avail ourselves of the notation $x^{L}$ and $x^{R}$ to
denote injections of a name into disjoint copies of the name
space. There are numerous ways to accomplish this. One example can be
found in \cite{MeredithR05}. This notation overloads to vectors of
names: $\vec{x}^{\pi} := (x_{i}^{\pi} \; : \; 0 \leq i < |\vec{x}| )$ where $\pi \in \{L,R\}$.

We also use $P^{\Box} := P|\Box$.

In \cite{MeredithR05} an interpretation of the new operator is
given. It turns out that there are several possible interpretations
all enjoying the requisite algebraic properties of the operator (see
\cite{milner91polyadicpi}). We will therefore make liberal use of
$(\nu\; \vec{x})P$.

% subsection the_syntax_and_semantics_of_the_notation_system (end)   

\input{qm2pi.qmops} 

\input{qm2pi.sterngerlach} 

\input{qm2pi.metric} 

% section concurrent_process_calculi (end)

%\input{qm2pi.proofsketch}

% section proof sketch (end)

%\input{qm2pi.slviaknots} 

% section spatial logic via knots (end)

\input{qm2pi.conclusion}

% section conclusion (end)

%\input{qm2pi.dtcodes} 

% section wiring algorithm (end)

\input{qm2pi.ack} 

% section acknowledgments (end)

\newpage


\bibliographystyle{plain}   
\bibliography{../../biblios/main.bib}

\input{qm2pi.rhodetails}

\end{document}

 

\documentclass[12pt]{llncs}
%\documentclass{jktr}

\usepackage[pdftex]{hyperref}                   
\usepackage {listings}
\usepackage {mathpartir}
\usepackage{bcprules}
%\usepackage{listings}
                       
\usepackage{graphicx} 
%\usepackage[margins=2.5cm,nohead,nofoot]{geometry}
%\usepackage{geometry}
\usepackage{amsfonts}
\usepackage{amstext}
\usepackage{latexsym}
\usepackage{amssymb}
\usepackage{color}


%\include{myPreamble}
\include{qm2pi.local} 

%\ifpdf
%\usepackage[pdftex]{graphicx}
%\else
%\usepackage{graphicx}
%\fi

 % \ifpdf
%  \usepackage{pdfsync}
%  \if


%\title{Brief Article}
%\author{David F. Snyder}
%\author{L.G. Meredith}

%\address{Dept. of Math., Texas State University--San Marcos, San Marcos, TX 78666}
       
\pagestyle{empty}


\begin{document}

\lstset{language=[Objective]Caml,frame=shadowbox}

\input{qm2pi.front}

% section front matter (end)

\input{qm2pi.intro} 
 
% section introduction (end)

% \input{qm2pi.knotations} 

% section notation (end)

\input{qm2pi.process.calculi} 

% section concurrent_process_calculi_and_spatial_logics_ (end)
    
%\input{qm2pi.knots2pi} 

%\input{qm2pi.trefoil} 

%\input{qm2pi.mainthm} 

% subsection basic_interpretation (end)

%\input{qm2pi.rho.presentation} 
\subsection{The syntax and semantics of the notation system}\label{sub:the_syntax_and_semantics_of_the_notation_system} % (fold)

We now summarize a technical presentation of the calculus that
embodies our theory of dynamics. The typical presentation of such a
calculus follows the style of giving generators and relations on
them. The grammar, below, describing term constructors, freely
generates the set of processes, $\Proc$. This set is then quotiented
by a relation known as structural congruence and it is over this set
that the notion of dynamics is expressed. This presentation is
essentially that of \cite{MeredithR05} with the addition of
polyadicity and summation. For readability we have relegated some of
the technical subtleties to an appendix.

\subsubsection{Process grammar}\label{subsub:process_grammar}

\begin{mathpar}
  \inferrule* [lab=synchronization] {} {{M} \bc \pzero \;|\; x?F \;|\; x!C }
  \and
  \inferrule* [lab=abstraction] {} {{F} \bc (x)P}
  \and
  \inferrule* [lab=concretion] {} {{C} \bc \langle Q \rangle}
  \and
  \inferrule* [lab=process] {} {{P,Q} \bc M \;| \;P|Q \;|\; @{x}}
  \and
  \inferrule* [lab=name] {} {{x} \bc \quotep{P}}
\end{mathpar} 

Note that $\vec{x}$ (resp. $\vec{P}$) denotes a vector of names
(resp. processes) of length $|\vec{x}|$ (resp. $|\vec{P}|$). We adopt
the following useful abbreviations.

\begin{mathpar}
   x?(\vec{y}).P := x.(\vec{y})P \and  x\clift{\vec{P}} := x.\clift{\vec{P}}
   \and x!(y) := \lift{x}{\dropn{y}}
   \and \Pi_{i=0}^{n-1}P_i := P_0 | \ldots | P_{n-1}
\end{mathpar}

\subsubsection{Structural congruence}

\paragraph{Free and bound names and alpha-equivalence.} At the
core of structural equivalence is alpha-equivalence which identifies
process that are the same up to a change of variable. Formally, we
recognize the distinction between free and bound names. The free names
of a process, $\freenames{P}$, may be calculated recursively as
follows:

\begin{mathpar}
\freenames{\pzero} := \emptyset
  \and \\
  \freenames{x?(y).P} := \{ x \} \cup (\freenames{P} \setminus \{ y \})
  \and 
  \freenames{x!\langle P \rangle} := \{ x \} \cup \{ P \} 
  \and \\
  \freenames{P|Q} := \freenames{P} \cup \freenames{Q}
  \and \\
  \freenames{@{x}} := \{ x \}
\end{mathpar}

$\pi$
$\quotep{\pi}$

$\freenames{-} : \pi \to \mathcal{P}(\quotep{\pi})$

\begin{eqnarray*}
  \freenames{\pzero} & := & \emptyset \\
  \freenames{x?(y).P} & := & \{ x \} \cup (\freenames{P} \setminus \{ y \}) \\
  \freenames{x!\langle P \rangle} & := & \{ x \} \cup \{ P \} \\
  \freenames{P|Q} & := & \freenames{P} \cup \freenames{Q} \\
  \freenames{\dropn{x}} & := & \{ x \}
\end{eqnarray*}

The bound names of a process, $\boundnames{P}$, are those names occurring in $P$
that are not free. For example, in $x?(y).0$, the name $x$ is free, while $y$ is bound.

\begin{mathpar}
  \inferrule* [lab=monoidal-laws] {} { P|Q \equiv Q|P \and P|0 \equiv P \and P|(Q|R) \equiv (P|Q)|R }
\end{mathpar}

\begin{mathpar}
  \inferrule* [lab=alpha-equivalence] {} { (x)P \equiv (y)P\{y/x\} \and y \not\in \freenames{P} }
\end{mathpar}

\begin{definition}
Then two processes, $P,Q$, are alpha-equivalent if $P = Q\{\vec{y}/\vec{x}\}$ for
some $\vec{x} \in \boundnames{Q},\vec{y} \in \boundnames{P}$, where $Q\{\vec{y}/\vec{x}\}$
denotes the capture-avoiding substitution of $\vec{y}$ for $\vec{x}$ in $Q$.
\end{definition}

\begin{definition}
  The {\em structural congruence} \cite{SangiorgiWalker} , $\equiv$,
  between processes is the least congruence containing
  alpha-equivalence, satisfying the abelian monoid laws
  (associativity, commutativity and $\pzero$ as identity) for parallel
  composition $|$ and for summation $+$.
\end{definition}

\subsection{Name equivalence}

We take name equivalence, written $\nameeq$, to be the smallest
equivalence relation generated by the following rules.

\begin{mathpar}
\inferrule*[lab=Quote-drop]
{ }
{ \quotep{@{x}} \nameeq x }

\inferrule*[lab=Struct-equiv]
{ P \scong Q }
{ \quotep{P} \nameeq \quotep{Q} }
\end{mathpar}

The astute reader will have noticed that the mutual recursion of names
and processes imposes a mutual recursion on alpha-equivalence and
structural equivalence via name-equivalence. Fortunately, all of this
works out pleasantly and we may calculate in the natural way, free of
concern. The reader interested in the details is referred to the
appendix \ref{appendix:rho_details}.

\subsection{Substitution}

We use $\Proc$ for the set of processes, $\QProc$ for the set of
names, and $\id{\{}\vec{y} / \vec{x} \id{\}}$ to denote partial maps,
$s : \QProc \rightarrow \QProc$. A map, $s$ lifts, uniquely, to a map
on process terms, $\widehat{s} : \Proc \rightarrow \Proc$ by the
following equations.

\begin{mathpar}
  (0) \psubstp{Q}{P} := 0 \\
  (R \juxtap S) \psubstp{Q}{P}
  :=    
  (R)\psubstp{Q}{P} \juxtap (S) \psubstp{Q}{P} \\
  (x?(y).R) \psubstp{Q}{P}    
  :=    
  (x)\substp{Q}{P} (z)\concat( (R \psubstn{z}{y}) \psubstp{Q}{P} ) \\
  (\lift{x}{R}) \psubstp{Q}{P}  
  :=
  \lift{(x)\substp{Q}{P}}{ R \psubstp{Q}{P} } \\
%   (\dropn{x})  \psubstp{Q}{P}       
%   := 
%   \left\{ 
%     \begin{array}{ccc} 
%       \dropn{\quotep{Q}} & & x \nameeq \quotep{P} \\
%       \dropn{x} & & otherwise \\
%     \end{array}
%   \right. 
  (\dropn{x})  \psubstp{Q}{P}       
  := 
  \left\{ 
    \begin{array}{ccc} 
      Q & & x \nameeq \quotep{P} \\
      \dropn{x} & & otherwise \\
    \end{array}
  \right.
\end{mathpar}
 

where

\begin{eqnarray}
  (x)\id{\{} \lpquote Q \rpquote / \lpquote P \rpquote \id{\}}            = 
  \left\{ 
    \begin{array}{ccc}
      \lpquote Q \rpquote & & x \nameeq \lpquote P \rpquote \\
      x & & otherwise \\
    \end{array}
  \right. \nonumber
\end{eqnarray}

and $z$ is chosen distinct from $\quotep{P}$, $\quotep{Q}$, the free
names in $Q$, and all the names in $R$. Our $\alpha$-equivalence will
be built in the standard way from this substitution.

\begin{remark}\label{rem:no_self_referential_names}
  One consequence of these definitions is that $\forall P. \quotep{P}
  \not\in \freenames{P}$.
\end{remark}

\subsection{ Dynamic quote: an example }

Anticipating something of what's to come, consider applying the
substitution, $\widehat{\id{\{}u / z \id{\}}}$, to the following pair
of processes, $\lift{w}{y!(z)}$ and $w[ \lpquote y!(z) \rpquote ]$.

\begin{eqnarray}
	\lift{w}{y!(z)}\widehat{\id{\{}u / z \id{\}}}
		& = &
		\lift{w}{y!(u)} \nonumber\\
	w[ \lpquote y!(z) \rpquote ] \widehat{ \id{\{}u / z \id{\}} }
		& = &
		w[ \lpquote y!(z) \rpquote ] \nonumber
\end{eqnarray}

Because the body of the process between quotes is impervious to
substitution, we get radically different answers. In fact, by
examining the first process in an input context,
e.g. $x?(z).\lift{w}{y!(z)}$, we see that the process under the lift
operator may be shaped by prefixed inputs binding a name inside it. In
this sense, the lift operator will be seen as a way to dynamically
construct processes before reifying them as names.

Finally equipped with these standard features we can present the
dynamics of the calculus.

\subsubsection{Operational semantics} 

Finally, we introduce the computational dynamics. What marks these
algebras as distinct from other more traditionally studied algebraic
structures, e.g. vector spaces or polynomial rings, is the manner in
which dynamics is captured. In traditional structures, dynamics is typically
expressed through morphisms between such structures, as in linear maps
between vector spaces or morphisms between rings. In algebras
associated with the semantics of computation, the dynamics is
expressed as part of the algebraic structure itself, through a
reduction reduction relation typically denoted by $\red$. Below, we
give a recursive presentation of this relation for the calculus used
in the encoding.

$\red \subseteq \pi \times \pi$
$\red : \pi \to \mathcal{P}(\pi)$

\begin{mathpar}
  \inferrule* [lab=Comm] { \textsf{match}( x_{src}, x_{trgt} ) } { x_{trgt}?(y)P \; | \; x_{src}!\langle {Q} \rangle \red P\{\quotep{Q}/y}\} }
  \and \\
  \inferrule* [lab=Par] {{P} \red {P}'} {{{P} | {Q}} \red {{P}' | {Q}}}
  \and
  \inferrule* [lab=Equiv]{{{P} \scong {P}'} \andalso {{P}' \red {Q}'} \andalso {{Q}' \scong {Q}}}{{P} \red {Q}}
\end{mathpar}

\begin{eqnarray*}
  match_{\equiv} (\quotep{P},\quotep{Q}) & := & P \equiv Q \\
  match_{\dagger}(\quotep{P},\quotep{Q}) & := & \forall R. P|Q \red^{*} R => R \red^{*} 0 \\
  match_{K}(\quotep{P},\quotep{Q}) & := & K \mbox{ for some context } K
\end{eqnarray*}

$u?(x)P | u!\langle Q \rangle \red P\{\quotep{Q}/x\}$

%We write $\wred$ for $\red^*$, and $P\red$ if $\exists Q $ such that $ P \red Q$.
We write $P\red$ if $\exists Q $ such that $ P \red Q$ and $P\not\red$, otherwise.

\section{Replication}

As mentioned before, it is known that replication (and hence
recursion) can be implemented in a higher-order process algebra
\cite{SangiorgiWalker}. As our first example of calculation with the
machinery thus far presented we give the construction explicitly in
the {\rhoc}.

\begin{eqnarray}
	D_{x} & := & \prefix{x}{y}{(\binpar{\outputp{x}{y}}{@{y}})} \nonumber\\
	\bangp_{x}{P} & := & \binpar{{x}!\langle{\binpar{D_{x}}{P}}\rangle}{D_{x}} \nonumber
\end{eqnarray}

\begin{eqnarray}
	\bangp_{x}{P} & & \nonumber\\
	=
	& {x}!\langle{(\prefix{x}{y}{(\outputp{x}{y} | @{y})) | P}}\rangle 
	      | \prefix{x}{y}{(\outputp{x}{y} | @{y})} & \nonumber\\
	\red
	& (\outputp{x}{y} | @{y})\substn{\quotep{(\prefix{x}{y}{(@{y} | \outputp{x}{y})) | P}}}{y} & \nonumber\\
	=
	& \outputp{x}{\quotep{(\prefix{x}{y}{(\outputp{x}{y} | @{y})) | P}}}
	  | {(\prefix{x}{y}{(\outputp{x}{y} | @{y})) | P}} & \nonumber\\
	\red
	& \ldots & \nonumber\\
	\red^*
	& P | P | \ldots & \nonumber
\end{eqnarray}

Of course, this encoding, as an implementation, runs away, unfolding
$\bangp{P}$ eagerly. A lazier and more implementable replication
operator, restricted to input-guarded processes, may be obtained as follows.

\begin{eqnarray}
\bangp{\prefix{u}{v}{P}} 
	:= 
	\binpar{\lift{x}{\prefix{u}{v}{(\binpar{D(x)}{P})}}}{D(x)} \nonumber
\end{eqnarray}

\begin{remark}
  Note that the lazier definition still does not deal with summation
  or mixed summation (i.e. sums over input and output). The reader is
  invited to construct definitions of replication that deal with these
  features. 

  Further, the definitions are parameterized in a name, $x$. Can you,
  gentle reader, make a definition that eliminates this parameter and
  guarantees no accidental interaction between the replication
  machinery and the process being replicated -- i.e. no accidental
  sharing of names used by the process to get its work done and the
  name(s) used by the replication to effect copying. This latter
  revision of the definition of replication is crucial to obtaining
  the expected identity $!!P \sim !P$.
\end{remark}

\begin{remark}\label{rem:paradoxical_combinator}
  The reader familiar with the lambda calculus will have noticed the
  similarity between $D$ and the paradoxical combinator.

  [Ed. note: the existence of this seems to suggest we have to be more
  restrictive on the set of processes and names we admit if we are to
  support no-cloning.]
\end{remark}

\subsubsection{Bisimulation}

The computational dynamics gives rise to another kind of equivalence,
the equivalence of computational behavior. As previously mentioned
this is typically captured \emph{via} some form of bisimulation.

% The notion we use in this paper is weak barbed bisimulation
% \cite{milner91polyadicpi}.

The notion we use in this paper is derived from weak barbed
bisimulation \cite{milner91polyadicpi}. 

\begin{definition}
An \emph{observation relation}, $\downarrow_{\mathcal N}$, over a set
of names, $\mathcal N$, is the smallest relation satisfying the rules
below.

\infrule[Out-barb]{y \in {\mathcal N}, \; x \nameeq y}
		  {\outputp{x}{v} \downarrow_{\mathcal N} x}
\infrule[Par-barb]{\mbox{$P\downarrow_{\mathcal N} x$ or $Q\downarrow_{\mathcal N} x$}}
		  {\binpar{P}{Q} \downarrow_{\mathcal N} x}

We write $P \Downarrow_{\mathcal N} x$ if there is $Q$ such that 
$P \wred Q$ and $Q \downarrow_{\mathcal N} x$.
\end{definition}

\begin{definition}
%\label{def.bbisim}
An  ${\mathcal N}$-\emph{barbed bisimulation} over a set of names, ${\mathcal N}$, is a symmetric binary relation 
${\mathcal S}_{\mathcal N}$ between agents such that $P\rel{S}_{\mathcal N}Q$ implies:
\begin{enumerate}
\item If $P \red P'$ then $Q \wred Q'$ and $P'\rel{S}_{\mathcal N} Q'$.
\item If $P\downarrow_{\mathcal N} x$, then $Q\Downarrow_{\mathcal N} x$.
\end{enumerate}
$P$ is ${\mathcal N}$-barbed bisimilar to $Q$, written
$P \wbbisim_{\mathcal N} Q$, if $P \rel{S}_{\mathcal N} Q$ for some ${\mathcal N}$-barbed bisimulation ${\mathcal S}_{\mathcal N}$.
\end{definition}

$\mathcal{R} \subseteq \pi \times \pi$

$P \mathcal{R} Q => \forall P'. P \red P' \Rightarrow \exists Q'. Q \red Q', P' \mathcal{R} Q'$

$P \vdash x \Rightarrow Q \vdash x$

\begin{mathpar}
  \inferrule*[lab=Out-barb]{x \nameeq y}{{y}!\langle{Q}\rangle \vdash x}
  \and
  \inferrule*[lab=Par-barb]{\mbox{$P\vdash x$ or $Q\vdash x$}}{\binpar{P}{Q} \vdash x}
\end{mathpar}

\subsubsection{Contexts}

One of the principle advantages of computational calculi like the
$\pi$-calculus is a well-defined notion of context,
contextual-equivalence and a correlation between
contextual-equivalence and notions of bisimulation. The notion of
context allows the decomposition of a process into (sub-)process and
its syntactic environment, its context. Thus, a context may be
thought of as a process with a ``hole'' (written $\Box$) in it. The
application of a context $M$ to a process $P$, written $M[P]$, is
tantamount to filling the hole in $M$ with $P$. In this paper we do
not need the full weight of this theory, but do make use of the notion
of context in the proof the main theorem. 

\begin{mathpar}
  \inferrule* [lab=summation] {} {{M_{M},M_{N}} \bc \Box \;|\; x.M_{A} \;|\; M_{M}+M_{N}}
  \and
  \inferrule* [lab=agent] {} {{M_{A}} \bc (\vec{x})M_{P} \;| \; \clift{P_0,\ldots,M_{P},\ldots,P_N}}
  \and \\
  \inferrule* [lab=process] {} {{M_{P}} \bc M_{N} \;| \;P|M_{P} }
\end{mathpar} 

\begin{mathpar}
  \inferrule* [lab=sychronization] {} {M_{N} \bc \Box \;|\; x?M_{F} \;|\; x!M_{C}}
  \and
  \inferrule* [lab=abstraction] {} {{M_{F}} \bc (x)M_{P} }
  \and
  \inferrule* [lab=concretion] {} {{M_{C}} \bc \langle M_{P} \rangle }
  \and \\
  \inferrule* [lab=process] {} {{M_{P}} \bc M_{N} \;| \;P|M_{P} }
\end{mathpar}

\begin{definition}[contextual application] Given a context $M$, and
  process $P$, we define the \emph{contextual application}, $M[P] :=
  M\{P/\Box\}$. That is, the contextual application of M to P is the
  substitution of $P$ for $\Box$ in $M$.
\end{definition}

$\meaningof{-} : L \to \mathcal{P}(\pi)$

\begin{mathpar}
  \inferrule* [lab=collection] {} {\meaningof{true} = \pi, \and \meaningof{~E} = \pi \setminus \meaningof{E}, \and \meaningof{E_{1} \& E_{2}} = \meaningof{E_{1}} \cap \meaningof{E_{2}}}
\end{mathpar}

\begin{mathpar}
  \inferrule* [lab=structure] {} {\meaningof{0} = \{ P \in \pi | P \equiv 0 \}, \and \\ \meaningof{E_1 | E_2} = \{ P \in \pi | P \equiv P_{1} | P_{2}, P_{1} \in \meaningof{E_{1}}, P_{2} \in \meaningof{E_2}\} }
\end{mathpar}

\begin{mathpar}
 \inferrule* [lab=behavior] {} {\meaningof{\langle a?b \rangle E} = \{ P \in \pi | P \equiv Q | u?(y)P', \\ \and \\\\ \and \\ \;\;\; u \in \meaningof{a}, \forall z.P'\{z/y\} \in \meaningof{E\{z/b\}}\}, \and \\ \meaningof{a!E} = \{ P \in \pi | P \equiv Q | x!\langle P' \rangle, x \in \meaningof{a} P' \in \meaningof{E}\} }
\end{mathpar}

\begin{mathpar}
 \inferrule* [lab=nominal] {} {\meaningof{\quotep{E}} = \{ \quotep{P} \in \quotep{\pi} | P \in \meaningof{E} \}, \and \meaningof{\quotep{P}} = \{ \quotep{Q} \in \quotep{\pi} | P \equiv Q \} \and \\ \meaningof{@\quotep{E}} = \{ P \in \pi | P \equiv @x, x \in \meaningof{E} \}}
\end{mathpar}

\begin{eqnarray*}
  \\
  \meaningof{-} : TS \to ST
\end{eqnarray*}

\begin{eqnarray*}
  \\
  L : TS \to ST
\end{eqnarray*}

\begin{eqnarray*}
  \\
  P \models E \iff P \in \meaningof{E}
\end{eqnarray*}

\begin{eqnarray*}
  P \approx_{L} Q \iff \forall E \in L. P \models E \iff Q \models E
\end{eqnarray*}

\begin{eqnarray*}
  P \approx_{K} Q
\end{eqnarray*}

\begin{eqnarray*}
  P \approx Q
\end{eqnarray*}

$\approx_{K} = \approx = \approx_{L}$

\subsubsection{Contextual duality}

Note that contexts extend the quotation operation to a family of
operations from processes to names. Given a context, $M$, we can
define a \emph{nominal context}, $\quotep{M}$ by $\quotep{M}[P] :=
\quotep{M[P]}$. To foreshadow what is to come we observe that these
operations enjoy a duality with processes very much like the duality
between vectors and maps from vectors to scalars.

Further, because the calculus is essentially higher-order, we have a
correspondence between contexts and processes. More specifically,
given a name $x$ and a context $M$ we can construct $M^{*}_{x}$ such
that 

\begin{mathpar}
  M^{*}_{x} | \lift{x}{P} \red M[P]
\end{mathpar}

namely,

\begin{mathpar}
  M^{*}_{x} := x?(u).M[\dropn{u}]
\end{mathpar}

The dependence of $M^{*}_{x}$ on a name makes it an abstraction, 

\begin{mathpar}
  M^{*} := (x)x?(u).M[\dropn{u}]
\end{mathpar}

\subsection{Additional notation}

It will sometimes be convenient to denote the process a name
quotes. We already have the notation $x = \quotep{P}$, but it will be
convenient to introduce an alternate notation, $\procn{x}$, when we
want to emphasize the connection to the use of the name. Note that, by
virtue of name equivalence, $\quotep{\procn{x}} \nameeq x$; so, the
notation is consistent with previous definitions.

Further, because names have structure it is possible to effect
substitutions on the basis of that structure. This means we need to
upgrade our notation for substitutions, which we accomplish by
adapting comprehension notation. Thus,

\begin{mathpar}
  P\{ y / x : x \in S \}
\end{mathpar}

is interpreted to mean the process derived from P by replacing (in a
capture-avoiding manner) each occurrence of $x$ in $S$ by $y$. For example,

\begin{mathpar}
  P\{ \quotep{\procn{x}|\procn{x}} / x : x \in \freenames{P} \}
\end{mathpar}

will replace each (occurrence) of a free name $x$ in $P$ by
$\quotep{\procn{x}|\procn{x}}$.

Also, we will avail ourselves of the notation $x^{L}$ and $x^{R}$ to
denote injections of a name into disjoint copies of the name
space. There are numerous ways to accomplish this. One example can be
found in \cite{MeredithR05}. This notation overloads to vectors of
names: $\vec{x}^{\pi} := (x_{i}^{\pi} \; : \; 0 \leq i < |\vec{x}| )$ where $\pi \in \{L,R\}$.

We also use $P^{\Box} := P|\Box$.

In \cite{MeredithR05} an interpretation of the new operator is
given. It turns out that there are several possible interpretations
all enjoying the requisite algebraic properties of the operator (see
\cite{milner91polyadicpi}). We will therefore make liberal use of
$(\nu\; \vec{x})P$.

% subsection the_syntax_and_semantics_of_the_notation_system (end)   

\input{qm2pi.qmops} 

\input{qm2pi.sterngerlach} 

\input{qm2pi.metric} 

% section concurrent_process_calculi (end)

%\input{qm2pi.proofsketch}

% section proof sketch (end)

%\input{qm2pi.slviaknots} 

% section spatial logic via knots (end)

\input{qm2pi.conclusion}

% section conclusion (end)

%\input{qm2pi.dtcodes} 

% section wiring algorithm (end)

\input{qm2pi.ack} 

% section acknowledgments (end)

\newpage


\bibliographystyle{plain}   
\bibliography{../../biblios/main.bib}

\input{qm2pi.rhodetails}

\end{document}

 

% section concurrent_process_calculi (end)

%\documentclass[12pt]{llncs}
%\documentclass{jktr}

\usepackage[pdftex]{hyperref}                   
\usepackage {listings}
\usepackage {mathpartir}
\usepackage{bcprules}
%\usepackage{listings}
                       
\usepackage{graphicx} 
%\usepackage[margins=2.5cm,nohead,nofoot]{geometry}
%\usepackage{geometry}
\usepackage{amsfonts}
\usepackage{amstext}
\usepackage{latexsym}
\usepackage{amssymb}
\usepackage{color}


%\include{myPreamble}
\include{qm2pi.local} 

%\ifpdf
%\usepackage[pdftex]{graphicx}
%\else
%\usepackage{graphicx}
%\fi

 % \ifpdf
%  \usepackage{pdfsync}
%  \if


%\title{Brief Article}
%\author{David F. Snyder}
%\author{L.G. Meredith}

%\address{Dept. of Math., Texas State University--San Marcos, San Marcos, TX 78666}
       
\pagestyle{empty}


\begin{document}

\lstset{language=[Objective]Caml,frame=shadowbox}

\input{qm2pi.front}

% section front matter (end)

\input{qm2pi.intro} 
 
% section introduction (end)

% \input{qm2pi.knotations} 

% section notation (end)

\input{qm2pi.process.calculi} 

% section concurrent_process_calculi_and_spatial_logics_ (end)
    
%\input{qm2pi.knots2pi} 

%\input{qm2pi.trefoil} 

%\input{qm2pi.mainthm} 

% subsection basic_interpretation (end)

%\input{qm2pi.rho.presentation} 
\subsection{The syntax and semantics of the notation system}\label{sub:the_syntax_and_semantics_of_the_notation_system} % (fold)

We now summarize a technical presentation of the calculus that
embodies our theory of dynamics. The typical presentation of such a
calculus follows the style of giving generators and relations on
them. The grammar, below, describing term constructors, freely
generates the set of processes, $\Proc$. This set is then quotiented
by a relation known as structural congruence and it is over this set
that the notion of dynamics is expressed. This presentation is
essentially that of \cite{MeredithR05} with the addition of
polyadicity and summation. For readability we have relegated some of
the technical subtleties to an appendix.

\subsubsection{Process grammar}\label{subsub:process_grammar}

\begin{mathpar}
  \inferrule* [lab=synchronization] {} {{M} \bc \pzero \;|\; x?F \;|\; x!C }
  \and
  \inferrule* [lab=abstraction] {} {{F} \bc (x)P}
  \and
  \inferrule* [lab=concretion] {} {{C} \bc \langle Q \rangle}
  \and
  \inferrule* [lab=process] {} {{P,Q} \bc M \;| \;P|Q \;|\; @{x}}
  \and
  \inferrule* [lab=name] {} {{x} \bc \quotep{P}}
\end{mathpar} 

Note that $\vec{x}$ (resp. $\vec{P}$) denotes a vector of names
(resp. processes) of length $|\vec{x}|$ (resp. $|\vec{P}|$). We adopt
the following useful abbreviations.

\begin{mathpar}
   x?(\vec{y}).P := x.(\vec{y})P \and  x\clift{\vec{P}} := x.\clift{\vec{P}}
   \and x!(y) := \lift{x}{\dropn{y}}
   \and \Pi_{i=0}^{n-1}P_i := P_0 | \ldots | P_{n-1}
\end{mathpar}

\subsubsection{Structural congruence}

\paragraph{Free and bound names and alpha-equivalence.} At the
core of structural equivalence is alpha-equivalence which identifies
process that are the same up to a change of variable. Formally, we
recognize the distinction between free and bound names. The free names
of a process, $\freenames{P}$, may be calculated recursively as
follows:

\begin{mathpar}
\freenames{\pzero} := \emptyset
  \and \\
  \freenames{x?(y).P} := \{ x \} \cup (\freenames{P} \setminus \{ y \})
  \and 
  \freenames{x!\langle P \rangle} := \{ x \} \cup \{ P \} 
  \and \\
  \freenames{P|Q} := \freenames{P} \cup \freenames{Q}
  \and \\
  \freenames{@{x}} := \{ x \}
\end{mathpar}

$\pi$
$\quotep{\pi}$

$\freenames{-} : \pi \to \mathcal{P}(\quotep{\pi})$

\begin{eqnarray*}
  \freenames{\pzero} & := & \emptyset \\
  \freenames{x?(y).P} & := & \{ x \} \cup (\freenames{P} \setminus \{ y \}) \\
  \freenames{x!\langle P \rangle} & := & \{ x \} \cup \{ P \} \\
  \freenames{P|Q} & := & \freenames{P} \cup \freenames{Q} \\
  \freenames{\dropn{x}} & := & \{ x \}
\end{eqnarray*}

The bound names of a process, $\boundnames{P}$, are those names occurring in $P$
that are not free. For example, in $x?(y).0$, the name $x$ is free, while $y$ is bound.

\begin{mathpar}
  \inferrule* [lab=monoidal-laws] {} { P|Q \equiv Q|P \and P|0 \equiv P \and P|(Q|R) \equiv (P|Q)|R }
\end{mathpar}

\begin{mathpar}
  \inferrule* [lab=alpha-equivalence] {} { (x)P \equiv (y)P\{y/x\} \and y \not\in \freenames{P} }
\end{mathpar}

\begin{definition}
Then two processes, $P,Q$, are alpha-equivalent if $P = Q\{\vec{y}/\vec{x}\}$ for
some $\vec{x} \in \boundnames{Q},\vec{y} \in \boundnames{P}$, where $Q\{\vec{y}/\vec{x}\}$
denotes the capture-avoiding substitution of $\vec{y}$ for $\vec{x}$ in $Q$.
\end{definition}

\begin{definition}
  The {\em structural congruence} \cite{SangiorgiWalker} , $\equiv$,
  between processes is the least congruence containing
  alpha-equivalence, satisfying the abelian monoid laws
  (associativity, commutativity and $\pzero$ as identity) for parallel
  composition $|$ and for summation $+$.
\end{definition}

\subsection{Name equivalence}

We take name equivalence, written $\nameeq$, to be the smallest
equivalence relation generated by the following rules.

\begin{mathpar}
\inferrule*[lab=Quote-drop]
{ }
{ \quotep{@{x}} \nameeq x }

\inferrule*[lab=Struct-equiv]
{ P \scong Q }
{ \quotep{P} \nameeq \quotep{Q} }
\end{mathpar}

The astute reader will have noticed that the mutual recursion of names
and processes imposes a mutual recursion on alpha-equivalence and
structural equivalence via name-equivalence. Fortunately, all of this
works out pleasantly and we may calculate in the natural way, free of
concern. The reader interested in the details is referred to the
appendix \ref{appendix:rho_details}.

\subsection{Substitution}

We use $\Proc$ for the set of processes, $\QProc$ for the set of
names, and $\id{\{}\vec{y} / \vec{x} \id{\}}$ to denote partial maps,
$s : \QProc \rightarrow \QProc$. A map, $s$ lifts, uniquely, to a map
on process terms, $\widehat{s} : \Proc \rightarrow \Proc$ by the
following equations.

\begin{mathpar}
  (0) \psubstp{Q}{P} := 0 \\
  (R \juxtap S) \psubstp{Q}{P}
  :=    
  (R)\psubstp{Q}{P} \juxtap (S) \psubstp{Q}{P} \\
  (x?(y).R) \psubstp{Q}{P}    
  :=    
  (x)\substp{Q}{P} (z)\concat( (R \psubstn{z}{y}) \psubstp{Q}{P} ) \\
  (\lift{x}{R}) \psubstp{Q}{P}  
  :=
  \lift{(x)\substp{Q}{P}}{ R \psubstp{Q}{P} } \\
%   (\dropn{x})  \psubstp{Q}{P}       
%   := 
%   \left\{ 
%     \begin{array}{ccc} 
%       \dropn{\quotep{Q}} & & x \nameeq \quotep{P} \\
%       \dropn{x} & & otherwise \\
%     \end{array}
%   \right. 
  (\dropn{x})  \psubstp{Q}{P}       
  := 
  \left\{ 
    \begin{array}{ccc} 
      Q & & x \nameeq \quotep{P} \\
      \dropn{x} & & otherwise \\
    \end{array}
  \right.
\end{mathpar}
 

where

\begin{eqnarray}
  (x)\id{\{} \lpquote Q \rpquote / \lpquote P \rpquote \id{\}}            = 
  \left\{ 
    \begin{array}{ccc}
      \lpquote Q \rpquote & & x \nameeq \lpquote P \rpquote \\
      x & & otherwise \\
    \end{array}
  \right. \nonumber
\end{eqnarray}

and $z$ is chosen distinct from $\quotep{P}$, $\quotep{Q}$, the free
names in $Q$, and all the names in $R$. Our $\alpha$-equivalence will
be built in the standard way from this substitution.

\begin{remark}\label{rem:no_self_referential_names}
  One consequence of these definitions is that $\forall P. \quotep{P}
  \not\in \freenames{P}$.
\end{remark}

\subsection{ Dynamic quote: an example }

Anticipating something of what's to come, consider applying the
substitution, $\widehat{\id{\{}u / z \id{\}}}$, to the following pair
of processes, $\lift{w}{y!(z)}$ and $w[ \lpquote y!(z) \rpquote ]$.

\begin{eqnarray}
	\lift{w}{y!(z)}\widehat{\id{\{}u / z \id{\}}}
		& = &
		\lift{w}{y!(u)} \nonumber\\
	w[ \lpquote y!(z) \rpquote ] \widehat{ \id{\{}u / z \id{\}} }
		& = &
		w[ \lpquote y!(z) \rpquote ] \nonumber
\end{eqnarray}

Because the body of the process between quotes is impervious to
substitution, we get radically different answers. In fact, by
examining the first process in an input context,
e.g. $x?(z).\lift{w}{y!(z)}$, we see that the process under the lift
operator may be shaped by prefixed inputs binding a name inside it. In
this sense, the lift operator will be seen as a way to dynamically
construct processes before reifying them as names.

Finally equipped with these standard features we can present the
dynamics of the calculus.

\subsubsection{Operational semantics} 

Finally, we introduce the computational dynamics. What marks these
algebras as distinct from other more traditionally studied algebraic
structures, e.g. vector spaces or polynomial rings, is the manner in
which dynamics is captured. In traditional structures, dynamics is typically
expressed through morphisms between such structures, as in linear maps
between vector spaces or morphisms between rings. In algebras
associated with the semantics of computation, the dynamics is
expressed as part of the algebraic structure itself, through a
reduction reduction relation typically denoted by $\red$. Below, we
give a recursive presentation of this relation for the calculus used
in the encoding.

$\red \subseteq \pi \times \pi$
$\red : \pi \to \mathcal{P}(\pi)$

\begin{mathpar}
  \inferrule* [lab=Comm] { \textsf{match}( x_{src}, x_{trgt} ) } { x_{trgt}?(y)P \; | \; x_{src}!\langle {Q} \rangle \red P\{\quotep{Q}/y}\} }
  \and \\
  \inferrule* [lab=Par] {{P} \red {P}'} {{{P} | {Q}} \red {{P}' | {Q}}}
  \and
  \inferrule* [lab=Equiv]{{{P} \scong {P}'} \andalso {{P}' \red {Q}'} \andalso {{Q}' \scong {Q}}}{{P} \red {Q}}
\end{mathpar}

\begin{eqnarray*}
  match_{\equiv} (\quotep{P},\quotep{Q}) & := & P \equiv Q \\
  match_{\dagger}(\quotep{P},\quotep{Q}) & := & \forall R. P|Q \red^{*} R => R \red^{*} 0 \\
  match_{K}(\quotep{P},\quotep{Q}) & := & K \mbox{ for some context } K
\end{eqnarray*}

$u?(x)P | u!\langle Q \rangle \red P\{\quotep{Q}/x\}$

%We write $\wred$ for $\red^*$, and $P\red$ if $\exists Q $ such that $ P \red Q$.
We write $P\red$ if $\exists Q $ such that $ P \red Q$ and $P\not\red$, otherwise.

\section{Replication}

As mentioned before, it is known that replication (and hence
recursion) can be implemented in a higher-order process algebra
\cite{SangiorgiWalker}. As our first example of calculation with the
machinery thus far presented we give the construction explicitly in
the {\rhoc}.

\begin{eqnarray}
	D_{x} & := & \prefix{x}{y}{(\binpar{\outputp{x}{y}}{@{y}})} \nonumber\\
	\bangp_{x}{P} & := & \binpar{{x}!\langle{\binpar{D_{x}}{P}}\rangle}{D_{x}} \nonumber
\end{eqnarray}

\begin{eqnarray}
	\bangp_{x}{P} & & \nonumber\\
	=
	& {x}!\langle{(\prefix{x}{y}{(\outputp{x}{y} | @{y})) | P}}\rangle 
	      | \prefix{x}{y}{(\outputp{x}{y} | @{y})} & \nonumber\\
	\red
	& (\outputp{x}{y} | @{y})\substn{\quotep{(\prefix{x}{y}{(@{y} | \outputp{x}{y})) | P}}}{y} & \nonumber\\
	=
	& \outputp{x}{\quotep{(\prefix{x}{y}{(\outputp{x}{y} | @{y})) | P}}}
	  | {(\prefix{x}{y}{(\outputp{x}{y} | @{y})) | P}} & \nonumber\\
	\red
	& \ldots & \nonumber\\
	\red^*
	& P | P | \ldots & \nonumber
\end{eqnarray}

Of course, this encoding, as an implementation, runs away, unfolding
$\bangp{P}$ eagerly. A lazier and more implementable replication
operator, restricted to input-guarded processes, may be obtained as follows.

\begin{eqnarray}
\bangp{\prefix{u}{v}{P}} 
	:= 
	\binpar{\lift{x}{\prefix{u}{v}{(\binpar{D(x)}{P})}}}{D(x)} \nonumber
\end{eqnarray}

\begin{remark}
  Note that the lazier definition still does not deal with summation
  or mixed summation (i.e. sums over input and output). The reader is
  invited to construct definitions of replication that deal with these
  features. 

  Further, the definitions are parameterized in a name, $x$. Can you,
  gentle reader, make a definition that eliminates this parameter and
  guarantees no accidental interaction between the replication
  machinery and the process being replicated -- i.e. no accidental
  sharing of names used by the process to get its work done and the
  name(s) used by the replication to effect copying. This latter
  revision of the definition of replication is crucial to obtaining
  the expected identity $!!P \sim !P$.
\end{remark}

\begin{remark}\label{rem:paradoxical_combinator}
  The reader familiar with the lambda calculus will have noticed the
  similarity between $D$ and the paradoxical combinator.

  [Ed. note: the existence of this seems to suggest we have to be more
  restrictive on the set of processes and names we admit if we are to
  support no-cloning.]
\end{remark}

\subsubsection{Bisimulation}

The computational dynamics gives rise to another kind of equivalence,
the equivalence of computational behavior. As previously mentioned
this is typically captured \emph{via} some form of bisimulation.

% The notion we use in this paper is weak barbed bisimulation
% \cite{milner91polyadicpi}.

The notion we use in this paper is derived from weak barbed
bisimulation \cite{milner91polyadicpi}. 

\begin{definition}
An \emph{observation relation}, $\downarrow_{\mathcal N}$, over a set
of names, $\mathcal N$, is the smallest relation satisfying the rules
below.

\infrule[Out-barb]{y \in {\mathcal N}, \; x \nameeq y}
		  {\outputp{x}{v} \downarrow_{\mathcal N} x}
\infrule[Par-barb]{\mbox{$P\downarrow_{\mathcal N} x$ or $Q\downarrow_{\mathcal N} x$}}
		  {\binpar{P}{Q} \downarrow_{\mathcal N} x}

We write $P \Downarrow_{\mathcal N} x$ if there is $Q$ such that 
$P \wred Q$ and $Q \downarrow_{\mathcal N} x$.
\end{definition}

\begin{definition}
%\label{def.bbisim}
An  ${\mathcal N}$-\emph{barbed bisimulation} over a set of names, ${\mathcal N}$, is a symmetric binary relation 
${\mathcal S}_{\mathcal N}$ between agents such that $P\rel{S}_{\mathcal N}Q$ implies:
\begin{enumerate}
\item If $P \red P'$ then $Q \wred Q'$ and $P'\rel{S}_{\mathcal N} Q'$.
\item If $P\downarrow_{\mathcal N} x$, then $Q\Downarrow_{\mathcal N} x$.
\end{enumerate}
$P$ is ${\mathcal N}$-barbed bisimilar to $Q$, written
$P \wbbisim_{\mathcal N} Q$, if $P \rel{S}_{\mathcal N} Q$ for some ${\mathcal N}$-barbed bisimulation ${\mathcal S}_{\mathcal N}$.
\end{definition}

$\mathcal{R} \subseteq \pi \times \pi$

$P \mathcal{R} Q => \forall P'. P \red P' \Rightarrow \exists Q'. Q \red Q', P' \mathcal{R} Q'$

$P \vdash x \Rightarrow Q \vdash x$

\begin{mathpar}
  \inferrule*[lab=Out-barb]{x \nameeq y}{{y}!\langle{Q}\rangle \vdash x}
  \and
  \inferrule*[lab=Par-barb]{\mbox{$P\vdash x$ or $Q\vdash x$}}{\binpar{P}{Q} \vdash x}
\end{mathpar}

\subsubsection{Contexts}

One of the principle advantages of computational calculi like the
$\pi$-calculus is a well-defined notion of context,
contextual-equivalence and a correlation between
contextual-equivalence and notions of bisimulation. The notion of
context allows the decomposition of a process into (sub-)process and
its syntactic environment, its context. Thus, a context may be
thought of as a process with a ``hole'' (written $\Box$) in it. The
application of a context $M$ to a process $P$, written $M[P]$, is
tantamount to filling the hole in $M$ with $P$. In this paper we do
not need the full weight of this theory, but do make use of the notion
of context in the proof the main theorem. 

\begin{mathpar}
  \inferrule* [lab=summation] {} {{M_{M},M_{N}} \bc \Box \;|\; x.M_{A} \;|\; M_{M}+M_{N}}
  \and
  \inferrule* [lab=agent] {} {{M_{A}} \bc (\vec{x})M_{P} \;| \; \clift{P_0,\ldots,M_{P},\ldots,P_N}}
  \and \\
  \inferrule* [lab=process] {} {{M_{P}} \bc M_{N} \;| \;P|M_{P} }
\end{mathpar} 

\begin{mathpar}
  \inferrule* [lab=sychronization] {} {M_{N} \bc \Box \;|\; x?M_{F} \;|\; x!M_{C}}
  \and
  \inferrule* [lab=abstraction] {} {{M_{F}} \bc (x)M_{P} }
  \and
  \inferrule* [lab=concretion] {} {{M_{C}} \bc \langle M_{P} \rangle }
  \and \\
  \inferrule* [lab=process] {} {{M_{P}} \bc M_{N} \;| \;P|M_{P} }
\end{mathpar}

\begin{definition}[contextual application] Given a context $M$, and
  process $P$, we define the \emph{contextual application}, $M[P] :=
  M\{P/\Box\}$. That is, the contextual application of M to P is the
  substitution of $P$ for $\Box$ in $M$.
\end{definition}

$\meaningof{-} : L \to \mathcal{P}(\pi)$

\begin{mathpar}
  \inferrule* [lab=collection] {} {\meaningof{true} = \pi, \and \meaningof{~E} = \pi \setminus \meaningof{E}, \and \meaningof{E_{1} \& E_{2}} = \meaningof{E_{1}} \cap \meaningof{E_{2}}}
\end{mathpar}

\begin{mathpar}
  \inferrule* [lab=structure] {} {\meaningof{0} = \{ P \in \pi | P \equiv 0 \}, \and \\ \meaningof{E_1 | E_2} = \{ P \in \pi | P \equiv P_{1} | P_{2}, P_{1} \in \meaningof{E_{1}}, P_{2} \in \meaningof{E_2}\} }
\end{mathpar}

\begin{mathpar}
 \inferrule* [lab=behavior] {} {\meaningof{\langle a?b \rangle E} = \{ P \in \pi | P \equiv Q | u?(y)P', \\ \and \\\\ \and \\ \;\;\; u \in \meaningof{a}, \forall z.P'\{z/y\} \in \meaningof{E\{z/b\}}\}, \and \\ \meaningof{a!E} = \{ P \in \pi | P \equiv Q | x!\langle P' \rangle, x \in \meaningof{a} P' \in \meaningof{E}\} }
\end{mathpar}

\begin{mathpar}
 \inferrule* [lab=nominal] {} {\meaningof{\quotep{E}} = \{ \quotep{P} \in \quotep{\pi} | P \in \meaningof{E} \}, \and \meaningof{\quotep{P}} = \{ \quotep{Q} \in \quotep{\pi} | P \equiv Q \} \and \\ \meaningof{@\quotep{E}} = \{ P \in \pi | P \equiv @x, x \in \meaningof{E} \}}
\end{mathpar}

\begin{eqnarray*}
  \\
  \meaningof{-} : TS \to ST
\end{eqnarray*}

\begin{eqnarray*}
  \\
  L : TS \to ST
\end{eqnarray*}

\begin{eqnarray*}
  \\
  P \models E \iff P \in \meaningof{E}
\end{eqnarray*}

\begin{eqnarray*}
  P \approx_{L} Q \iff \forall E \in L. P \models E \iff Q \models E
\end{eqnarray*}

\begin{eqnarray*}
  P \approx_{K} Q
\end{eqnarray*}

\begin{eqnarray*}
  P \approx Q
\end{eqnarray*}

$\approx_{K} = \approx = \approx_{L}$

\subsubsection{Contextual duality}

Note that contexts extend the quotation operation to a family of
operations from processes to names. Given a context, $M$, we can
define a \emph{nominal context}, $\quotep{M}$ by $\quotep{M}[P] :=
\quotep{M[P]}$. To foreshadow what is to come we observe that these
operations enjoy a duality with processes very much like the duality
between vectors and maps from vectors to scalars.

Further, because the calculus is essentially higher-order, we have a
correspondence between contexts and processes. More specifically,
given a name $x$ and a context $M$ we can construct $M^{*}_{x}$ such
that 

\begin{mathpar}
  M^{*}_{x} | \lift{x}{P} \red M[P]
\end{mathpar}

namely,

\begin{mathpar}
  M^{*}_{x} := x?(u).M[\dropn{u}]
\end{mathpar}

The dependence of $M^{*}_{x}$ on a name makes it an abstraction, 

\begin{mathpar}
  M^{*} := (x)x?(u).M[\dropn{u}]
\end{mathpar}

\subsection{Additional notation}

It will sometimes be convenient to denote the process a name
quotes. We already have the notation $x = \quotep{P}$, but it will be
convenient to introduce an alternate notation, $\procn{x}$, when we
want to emphasize the connection to the use of the name. Note that, by
virtue of name equivalence, $\quotep{\procn{x}} \nameeq x$; so, the
notation is consistent with previous definitions.

Further, because names have structure it is possible to effect
substitutions on the basis of that structure. This means we need to
upgrade our notation for substitutions, which we accomplish by
adapting comprehension notation. Thus,

\begin{mathpar}
  P\{ y / x : x \in S \}
\end{mathpar}

is interpreted to mean the process derived from P by replacing (in a
capture-avoiding manner) each occurrence of $x$ in $S$ by $y$. For example,

\begin{mathpar}
  P\{ \quotep{\procn{x}|\procn{x}} / x : x \in \freenames{P} \}
\end{mathpar}

will replace each (occurrence) of a free name $x$ in $P$ by
$\quotep{\procn{x}|\procn{x}}$.

Also, we will avail ourselves of the notation $x^{L}$ and $x^{R}$ to
denote injections of a name into disjoint copies of the name
space. There are numerous ways to accomplish this. One example can be
found in \cite{MeredithR05}. This notation overloads to vectors of
names: $\vec{x}^{\pi} := (x_{i}^{\pi} \; : \; 0 \leq i < |\vec{x}| )$ where $\pi \in \{L,R\}$.

We also use $P^{\Box} := P|\Box$.

In \cite{MeredithR05} an interpretation of the new operator is
given. It turns out that there are several possible interpretations
all enjoying the requisite algebraic properties of the operator (see
\cite{milner91polyadicpi}). We will therefore make liberal use of
$(\nu\; \vec{x})P$.

% subsection the_syntax_and_semantics_of_the_notation_system (end)   

\input{qm2pi.qmops} 

\input{qm2pi.sterngerlach} 

\input{qm2pi.metric} 

% section concurrent_process_calculi (end)

%\input{qm2pi.proofsketch}

% section proof sketch (end)

%\input{qm2pi.slviaknots} 

% section spatial logic via knots (end)

\input{qm2pi.conclusion}

% section conclusion (end)

%\input{qm2pi.dtcodes} 

% section wiring algorithm (end)

\input{qm2pi.ack} 

% section acknowledgments (end)

\newpage


\bibliographystyle{plain}   
\bibliography{../../biblios/main.bib}

\input{qm2pi.rhodetails}

\end{document}



% section proof sketch (end)

%\section{Unlikely characters: spatial logic for
  knots}\label{sub:characteristic_formulae} % (fold)

Associated to the mobile process calculi are a family of logics known
as the Hennessy-Milner logics. These logics typically enjoy a
semantics interpreting formulae as sets of processes that when
factored through the encoding outlined above allows an identification
of classes of knots with logical formulae. In the context of this
encoding the sub-family known as the spatial logics \cite{CairesC03}
\cite{CairesC04} \cite{Caires04} are of particular interest providing
several important features for expressing and reasoning about
properties (i.e. classes) of knots. We hint here at how this may be done.

%\begin{description}
%\item [structural connectives] 
\subsubsection{Structural connectives} The spatial logics enjoy
structural connectives corresponding, at the logical level, to the
parallel composition ($P | Q$) and new name ($(\nu \; x)P$)
connectives for processes. As illustrated in the examples below, these
connectives are extremely expressive given the shape of our encoding.
%\item [decideable satisfaction]

\subsubsection{Decideable satisfaction}
In \cite{Caires04} the satisfaction relation is shown to be decideable
for a rich class of processes. It further turns out that the image of
the our encoding is a proper subset of that class. This result
provides the basis for an algorithm by which to search for knots
enjoying a given property.
%\item [characteristic formulae]

\subsubsection{Characteristic formulae}
In the same paper \cite{Caires04} , Caires presents a means of calculating
characteristic formulae, selecting equivalence classes of processes
up to a pre--specified depth limit on the support set of names. Composed with our
encoding, this characteristic formula can be used to select
characteristic formulae for knots.
%\end{description}

\subsubsection{Spatial logic formulae}

The grammar below (segmented for comprehension) summarizes the syntax
of spatial logic formulae. We employ illustrative examples in the
sequel to provide an intuitive understanding of their meaning
referring the reader to \cite{Caires04} for a more detailed explication
of the semantics.

\begin{mathpar}
  \inferrule* [lab=boolean] {} {{A,B} \bc T \;|\; \neg A \;|\; A \wedge B \;|\; \eta = \eta'}
  \and
  \inferrule* [lab=spatial] {} {|\; \pzero \;|\; A | B \;|\; x \text{\textregistered} A \;|\; \forall x . A \;|\;  H x . A}
  \and
  \inferrule* [lab=behavioral] {} {|\; \alpha . A}
  \and 
  \inferrule* [lab=recursion] {} {|\; X(\vec{u}) \;|\; \mu X(\vec{u}) . A}
  \and
  \inferrule* [lab=action] {} {\alpha \bc \langle x?(\vec{y}) \rangle \;|\; \langle x!(\vec{y}) \rangle \;|\; \langle \tau \rangle}
  \and 
  \inferrule* [lab=name] {} {\eta \bc x \;|\; \tau}
\end{mathpar} 

% subsection characteristic_formulae (end)   	 

\subsection{Example formulae}\label{sub:example_formulae_} % (fold)

\subsubsection{Crossing as formula.}
% 
% \begin{align*}
%   \frac{d}{dx} \sin x &= \cos x 
%   & \frac{d}{dx} e^x &= e^x \\
%   \frac{d}{dx} \cos x &= - \sin x 
%   & \frac{d}{dx} \log x &= \frac{1}{x} \\
% \end{align*} 

\begin{align*}
 \mu C(x_{0},x_{1},y_{0},y_{1},u).&(\langle x_{0}?(z) \rangle(\langle u! \rangle\langle y_{1}!z \rangle C(x_{0},x_{1},y_{0},y_{1},u)) & \\
  & \wedge \langle y_{1}?(z) \rangle (\langle u! \rangle \langle x_{0}!z \rangle C(x_{0},x_{1},y_{0},y_{1},u)) & \\
  & \wedge \langle x_{1}?(z) \rangle (\langle u? \rangle \langle y_{0}!z \rangle C(x_{0},x_{1},y_{0},y_{1},u)) & \\
  & \wedge \langle y_{0}?(z) \rangle (\langle u? \rangle \langle x_{1}!z \rangle C(x_{0},x_{1},y_{0},y_{1},u))) &
\end{align*}

The lexicographical similarity between the shape of this formulae and
the shape of definition of the process representing a crossing reveals
the intuitive meaning of this formulae. It describes the capabilities
of a process that has the right to represent a crossing. For example
it picks out processes that may perform an input on the port $x_0$ in
its initial menu of capabilities. What differentiates the formula
from the process, however, is that the crossing process is the
smallest candidate to satisfy the formula. Infinitely many other
processes -- with internal behavior hidden behind this interface, so
to speak -- also satisfy this formula. Even this simple formula,
then, can be seen to open a new view onto knots, providing a
computational interpretation of \emph{virtual} knots.

Note that this formula is derived by hand. A similar formula can be
derived by employing Caires' calculation of characteristic formula
\cite{Caires04} to the process representing a crossing. In light of
this discussion, we let
$\meaningof{C}_{\phi}(x0,x1,y0,y1,u)$ denote a formula specifying the
dynamics we wish to capture of a crossing. To guarantee we preserve
the shape of the interface and minimal semantics we demand that
$\meaningof{C}_{\phi}(x0,x1,y0,y1,u) \Rightarrow
\textbf{C}(x0,x1,y0,y1,u)$ where $\textbf{C}(x0,x1,y0,y1,u)$ denotes
the formula above.
                            
\subsubsection{Crossing number constraints.}
The moral content of the context lemma (Lemma \ref{context}) is that the notion of
``locality'' in the Reidemeister moves is effectively captured by the
parallel composition operator of the process calculus. This intuition
extends through the logic. Given a formula,
$\meaningof{C}_{\phi}(x0,x1,y0,y1,u)$, we can use the structural
connectives to specify constraints on crossing numbers, such as at
least $n$ crossings, or exactly $n$ crossings.
\begin{mathpar}
  \inferrule* [lab=at-least-n] {} { K^{\geq n}_{\phi}(\vec{xs},\vec{ys}) := \Pi_{i=0}^{n-1} Hu . \meaningof{C}_{\phi}(xs_i,ys_i,u) | T }
  \and 
  \inferrule* [lab=exactly-n] {} { K^{= n}_{\phi}(\vec{xs},\vec{ys}) := \Pi_{i=0}^{n-1} Hu . \meaningof{C}_{\phi}(xs_i,ys_i,u) | \neg (\forall x_0,y_0,x_1,y_1,u . \meaningof{C}_{\phi}(x_0,y_0,x_1,y_1,u) | T) }
\end{mathpar}

To round out this section, recall that the encoding of an $n$-crossing
knot decomposes into a parallel composition of $n$ \emph{copies} of a
crossing process together with a wiring harness. To specify different
knot classes with the same crossing number amounts to specifying
logical constraints on the wiring harness. In the interest of space,
we defer examples to a forthcoming paper. Suffice it to say that both
the conditions ``alternating knot'' and ``contains the tangle
corresponding to 5/3'' are expressible. For example, it is possible to
calculate the characteristic formula of a process corresponding to the
tangle 5/3 and conjoin it into the classifying formula via the
composition connective of the logic.

Finally, we wish to observe that it is entirely within reason to
contemplate a more domain-specific version of spatial logic tailored
to the shape of processes in the image of the encoding. Such a
domain-specific logic would have a better claim to the title formal
language of knot properties.

% subsection example_formulae_ (end)

% section knots_as_processes (end) 

% section spatial logic via knots (end)

\section{Conclusions and future work}

\paragraph{Testing physical space}
You, gentle reader, may wonder why of all the theorems to be proved
given this set up we pick the one above. In some sense it's hardly
central to quantum mechanics. We see it as central in the sense that
it firmly establishes a notion of physical space arising from a notion
of the equivalence of behavior. Relating bisimulation to a metric is a
big step forward, but one is faced with interpreting the relationship
of that metric space to something more physical. Quantum mechanical
notions of ``physical'' space are still far from intuitive, but by
relating this idea of distance as testing to calculations that predict
physical circumstances we are making a not insignificant step forward
toward an understanding of the physical space we inhabit as
essentially dynamic.

\paragraph{Effectivity and simulation}
One of the observations we have yet to make is that the entire program
spelled out here is effective. We have built various interpreters for
the reflective calculus at work in this interpretation. In principle,
then, we can simulate quantum mechanics on a computer. The place where
the simulation may lose fidelity is the infinitely branching summation
for the annihilator.

In this connection i also want to point out that the evaluation style
calculation of the inner product puts the non-determinism of the
summation right at the heart of measurement. This suggests that
Milner's original reduction-based formulation of the dynamics of his
calculi in terms of sums was not just notationally suggestive of a
notion of measure-and-continue but captured some significant part of
the physics.

\paragraph{Quantum continuations}
In light of this last observation i want to point out that the
predominant account of quantum mechanics is missing a key aspect of a
truly compositional story of the physical situation. In a real lab,
when a measurement is made the observation can be made to feed into
another device that then makes another measurement conditioned on the
results of the first. This means that after the superposition was
collapsed the entire experimental set up remained in
superposition. While QM offers a means of writing this down it doesn't
quite line up well with the well-trodden formulation of computation
and continuation that we see so succinctly expressed in Milner's
calculi. This suggests that there might be advantages to this account
of dynamics waiting to be explored.

\paragraph{Quantum logic}
In this connection, we also note that by virtue of having the
Hennessy-Milner construction, we can pull the construction through the
interpretation of QM. This gives us a natural candidate for a quantum
logic that enjoys an extremely tight connection with it's domain of
interpretation, making the construction much less ad hoc (rather it is
the image of functor!).

\paragraph{Quantum probabiity}
i have questions about the basis of the interpretation of inner
product as probability amplitude. In particular, using which
axiomatization of probability theory does the notion of probability
amplitude earn the right to be so dubbed? In other words, where is the
proof that the operation for calculating a probability amplitude (and
then squaring) satisfies the axioms of what it means to calculate a
probability? Even if such a proof exists (i have yet to find it in the
literature), i wonder if it might not be possible to turn things on
their heads. Can we view the calculation of the probability amplitude
as an axiomatization of probability? If so, then the definition we
give for calculating probability amplitude may provide the basis for
an \emph{effective} theory of probability.

\paragraph{Quantum vs ``biological'' information}
Finally, i want to conclude with a more philosophical observation. At
a recent workshop in which QM was a predominant topic i noticed
something about quantum information. The speaker was giving a riveting
discussion of axiomatic QM and showing how properties of ``no
cloning'' and ``no deleting'' emerged as consequences of the
axiomatization. Theorems of this form are necessary to give us a sense
of confidence that our axioms characterize the physical theory. What
struck me, though, was that if quantum information is neither erasable
nor replicable it is markedly different from \emph{life}. Two of the
things we know about life is that

\begin{itemize}
  \item it ends;
  \item to gain some measure of persistence, to transcend it's
    finitude it is imminently copyable.
\end{itemize}

Both of these qualities are summarized succinctly in the aphorism: all
flesh is grass. For me these two kinds of ``information'' -- call them
quantum and biological -- are end points on a spectrum of strategies
for persistence. At one end, we have those curious entities that enjoy
uniqueness and permanence; at the other, we have those who in the face
of a certain end and an uncertain present make a go of passing
something on. To me one of the more remarkable aspects of the latter
strategy is that in the presence of noise (and certain features of
copying) we get a kind of dynamism, a chance for improvement against a
given persistent condition.

% subsection other_calculi_other_bisimulations_and_geometry_as_behavior (end)




% section conclusion (end)

%\documentclass[12pt]{llncs}
%\documentclass{jktr}

\usepackage[pdftex]{hyperref}                   
\usepackage {listings}
\usepackage {mathpartir}
\usepackage{bcprules}
%\usepackage{listings}
                       
\usepackage{graphicx} 
%\usepackage[margins=2.5cm,nohead,nofoot]{geometry}
%\usepackage{geometry}
\usepackage{amsfonts}
\usepackage{amstext}
\usepackage{latexsym}
\usepackage{amssymb}
\usepackage{color}


%\include{myPreamble}
\include{qm2pi.local} 

%\ifpdf
%\usepackage[pdftex]{graphicx}
%\else
%\usepackage{graphicx}
%\fi

 % \ifpdf
%  \usepackage{pdfsync}
%  \if


%\title{Brief Article}
%\author{David F. Snyder}
%\author{L.G. Meredith}

%\address{Dept. of Math., Texas State University--San Marcos, San Marcos, TX 78666}
       
\pagestyle{empty}


\begin{document}

\lstset{language=[Objective]Caml,frame=shadowbox}

\input{qm2pi.front}

% section front matter (end)

\input{qm2pi.intro} 
 
% section introduction (end)

% \input{qm2pi.knotations} 

% section notation (end)

\input{qm2pi.process.calculi} 

% section concurrent_process_calculi_and_spatial_logics_ (end)
    
%\input{qm2pi.knots2pi} 

%\input{qm2pi.trefoil} 

%\input{qm2pi.mainthm} 

% subsection basic_interpretation (end)

%\input{qm2pi.rho.presentation} 
\subsection{The syntax and semantics of the notation system}\label{sub:the_syntax_and_semantics_of_the_notation_system} % (fold)

We now summarize a technical presentation of the calculus that
embodies our theory of dynamics. The typical presentation of such a
calculus follows the style of giving generators and relations on
them. The grammar, below, describing term constructors, freely
generates the set of processes, $\Proc$. This set is then quotiented
by a relation known as structural congruence and it is over this set
that the notion of dynamics is expressed. This presentation is
essentially that of \cite{MeredithR05} with the addition of
polyadicity and summation. For readability we have relegated some of
the technical subtleties to an appendix.

\subsubsection{Process grammar}\label{subsub:process_grammar}

\begin{mathpar}
  \inferrule* [lab=synchronization] {} {{M} \bc \pzero \;|\; x?F \;|\; x!C }
  \and
  \inferrule* [lab=abstraction] {} {{F} \bc (x)P}
  \and
  \inferrule* [lab=concretion] {} {{C} \bc \langle Q \rangle}
  \and
  \inferrule* [lab=process] {} {{P,Q} \bc M \;| \;P|Q \;|\; @{x}}
  \and
  \inferrule* [lab=name] {} {{x} \bc \quotep{P}}
\end{mathpar} 

Note that $\vec{x}$ (resp. $\vec{P}$) denotes a vector of names
(resp. processes) of length $|\vec{x}|$ (resp. $|\vec{P}|$). We adopt
the following useful abbreviations.

\begin{mathpar}
   x?(\vec{y}).P := x.(\vec{y})P \and  x\clift{\vec{P}} := x.\clift{\vec{P}}
   \and x!(y) := \lift{x}{\dropn{y}}
   \and \Pi_{i=0}^{n-1}P_i := P_0 | \ldots | P_{n-1}
\end{mathpar}

\subsubsection{Structural congruence}

\paragraph{Free and bound names and alpha-equivalence.} At the
core of structural equivalence is alpha-equivalence which identifies
process that are the same up to a change of variable. Formally, we
recognize the distinction between free and bound names. The free names
of a process, $\freenames{P}$, may be calculated recursively as
follows:

\begin{mathpar}
\freenames{\pzero} := \emptyset
  \and \\
  \freenames{x?(y).P} := \{ x \} \cup (\freenames{P} \setminus \{ y \})
  \and 
  \freenames{x!\langle P \rangle} := \{ x \} \cup \{ P \} 
  \and \\
  \freenames{P|Q} := \freenames{P} \cup \freenames{Q}
  \and \\
  \freenames{@{x}} := \{ x \}
\end{mathpar}

$\pi$
$\quotep{\pi}$

$\freenames{-} : \pi \to \mathcal{P}(\quotep{\pi})$

\begin{eqnarray*}
  \freenames{\pzero} & := & \emptyset \\
  \freenames{x?(y).P} & := & \{ x \} \cup (\freenames{P} \setminus \{ y \}) \\
  \freenames{x!\langle P \rangle} & := & \{ x \} \cup \{ P \} \\
  \freenames{P|Q} & := & \freenames{P} \cup \freenames{Q} \\
  \freenames{\dropn{x}} & := & \{ x \}
\end{eqnarray*}

The bound names of a process, $\boundnames{P}$, are those names occurring in $P$
that are not free. For example, in $x?(y).0$, the name $x$ is free, while $y$ is bound.

\begin{mathpar}
  \inferrule* [lab=monoidal-laws] {} { P|Q \equiv Q|P \and P|0 \equiv P \and P|(Q|R) \equiv (P|Q)|R }
\end{mathpar}

\begin{mathpar}
  \inferrule* [lab=alpha-equivalence] {} { (x)P \equiv (y)P\{y/x\} \and y \not\in \freenames{P} }
\end{mathpar}

\begin{definition}
Then two processes, $P,Q$, are alpha-equivalent if $P = Q\{\vec{y}/\vec{x}\}$ for
some $\vec{x} \in \boundnames{Q},\vec{y} \in \boundnames{P}$, where $Q\{\vec{y}/\vec{x}\}$
denotes the capture-avoiding substitution of $\vec{y}$ for $\vec{x}$ in $Q$.
\end{definition}

\begin{definition}
  The {\em structural congruence} \cite{SangiorgiWalker} , $\equiv$,
  between processes is the least congruence containing
  alpha-equivalence, satisfying the abelian monoid laws
  (associativity, commutativity and $\pzero$ as identity) for parallel
  composition $|$ and for summation $+$.
\end{definition}

\subsection{Name equivalence}

We take name equivalence, written $\nameeq$, to be the smallest
equivalence relation generated by the following rules.

\begin{mathpar}
\inferrule*[lab=Quote-drop]
{ }
{ \quotep{@{x}} \nameeq x }

\inferrule*[lab=Struct-equiv]
{ P \scong Q }
{ \quotep{P} \nameeq \quotep{Q} }
\end{mathpar}

The astute reader will have noticed that the mutual recursion of names
and processes imposes a mutual recursion on alpha-equivalence and
structural equivalence via name-equivalence. Fortunately, all of this
works out pleasantly and we may calculate in the natural way, free of
concern. The reader interested in the details is referred to the
appendix \ref{appendix:rho_details}.

\subsection{Substitution}

We use $\Proc$ for the set of processes, $\QProc$ for the set of
names, and $\id{\{}\vec{y} / \vec{x} \id{\}}$ to denote partial maps,
$s : \QProc \rightarrow \QProc$. A map, $s$ lifts, uniquely, to a map
on process terms, $\widehat{s} : \Proc \rightarrow \Proc$ by the
following equations.

\begin{mathpar}
  (0) \psubstp{Q}{P} := 0 \\
  (R \juxtap S) \psubstp{Q}{P}
  :=    
  (R)\psubstp{Q}{P} \juxtap (S) \psubstp{Q}{P} \\
  (x?(y).R) \psubstp{Q}{P}    
  :=    
  (x)\substp{Q}{P} (z)\concat( (R \psubstn{z}{y}) \psubstp{Q}{P} ) \\
  (\lift{x}{R}) \psubstp{Q}{P}  
  :=
  \lift{(x)\substp{Q}{P}}{ R \psubstp{Q}{P} } \\
%   (\dropn{x})  \psubstp{Q}{P}       
%   := 
%   \left\{ 
%     \begin{array}{ccc} 
%       \dropn{\quotep{Q}} & & x \nameeq \quotep{P} \\
%       \dropn{x} & & otherwise \\
%     \end{array}
%   \right. 
  (\dropn{x})  \psubstp{Q}{P}       
  := 
  \left\{ 
    \begin{array}{ccc} 
      Q & & x \nameeq \quotep{P} \\
      \dropn{x} & & otherwise \\
    \end{array}
  \right.
\end{mathpar}
 

where

\begin{eqnarray}
  (x)\id{\{} \lpquote Q \rpquote / \lpquote P \rpquote \id{\}}            = 
  \left\{ 
    \begin{array}{ccc}
      \lpquote Q \rpquote & & x \nameeq \lpquote P \rpquote \\
      x & & otherwise \\
    \end{array}
  \right. \nonumber
\end{eqnarray}

and $z$ is chosen distinct from $\quotep{P}$, $\quotep{Q}$, the free
names in $Q$, and all the names in $R$. Our $\alpha$-equivalence will
be built in the standard way from this substitution.

\begin{remark}\label{rem:no_self_referential_names}
  One consequence of these definitions is that $\forall P. \quotep{P}
  \not\in \freenames{P}$.
\end{remark}

\subsection{ Dynamic quote: an example }

Anticipating something of what's to come, consider applying the
substitution, $\widehat{\id{\{}u / z \id{\}}}$, to the following pair
of processes, $\lift{w}{y!(z)}$ and $w[ \lpquote y!(z) \rpquote ]$.

\begin{eqnarray}
	\lift{w}{y!(z)}\widehat{\id{\{}u / z \id{\}}}
		& = &
		\lift{w}{y!(u)} \nonumber\\
	w[ \lpquote y!(z) \rpquote ] \widehat{ \id{\{}u / z \id{\}} }
		& = &
		w[ \lpquote y!(z) \rpquote ] \nonumber
\end{eqnarray}

Because the body of the process between quotes is impervious to
substitution, we get radically different answers. In fact, by
examining the first process in an input context,
e.g. $x?(z).\lift{w}{y!(z)}$, we see that the process under the lift
operator may be shaped by prefixed inputs binding a name inside it. In
this sense, the lift operator will be seen as a way to dynamically
construct processes before reifying them as names.

Finally equipped with these standard features we can present the
dynamics of the calculus.

\subsubsection{Operational semantics} 

Finally, we introduce the computational dynamics. What marks these
algebras as distinct from other more traditionally studied algebraic
structures, e.g. vector spaces or polynomial rings, is the manner in
which dynamics is captured. In traditional structures, dynamics is typically
expressed through morphisms between such structures, as in linear maps
between vector spaces or morphisms between rings. In algebras
associated with the semantics of computation, the dynamics is
expressed as part of the algebraic structure itself, through a
reduction reduction relation typically denoted by $\red$. Below, we
give a recursive presentation of this relation for the calculus used
in the encoding.

$\red \subseteq \pi \times \pi$
$\red : \pi \to \mathcal{P}(\pi)$

\begin{mathpar}
  \inferrule* [lab=Comm] { \textsf{match}( x_{src}, x_{trgt} ) } { x_{trgt}?(y)P \; | \; x_{src}!\langle {Q} \rangle \red P\{\quotep{Q}/y}\} }
  \and \\
  \inferrule* [lab=Par] {{P} \red {P}'} {{{P} | {Q}} \red {{P}' | {Q}}}
  \and
  \inferrule* [lab=Equiv]{{{P} \scong {P}'} \andalso {{P}' \red {Q}'} \andalso {{Q}' \scong {Q}}}{{P} \red {Q}}
\end{mathpar}

\begin{eqnarray*}
  match_{\equiv} (\quotep{P},\quotep{Q}) & := & P \equiv Q \\
  match_{\dagger}(\quotep{P},\quotep{Q}) & := & \forall R. P|Q \red^{*} R => R \red^{*} 0 \\
  match_{K}(\quotep{P},\quotep{Q}) & := & K \mbox{ for some context } K
\end{eqnarray*}

$u?(x)P | u!\langle Q \rangle \red P\{\quotep{Q}/x\}$

%We write $\wred$ for $\red^*$, and $P\red$ if $\exists Q $ such that $ P \red Q$.
We write $P\red$ if $\exists Q $ such that $ P \red Q$ and $P\not\red$, otherwise.

\section{Replication}

As mentioned before, it is known that replication (and hence
recursion) can be implemented in a higher-order process algebra
\cite{SangiorgiWalker}. As our first example of calculation with the
machinery thus far presented we give the construction explicitly in
the {\rhoc}.

\begin{eqnarray}
	D_{x} & := & \prefix{x}{y}{(\binpar{\outputp{x}{y}}{@{y}})} \nonumber\\
	\bangp_{x}{P} & := & \binpar{{x}!\langle{\binpar{D_{x}}{P}}\rangle}{D_{x}} \nonumber
\end{eqnarray}

\begin{eqnarray}
	\bangp_{x}{P} & & \nonumber\\
	=
	& {x}!\langle{(\prefix{x}{y}{(\outputp{x}{y} | @{y})) | P}}\rangle 
	      | \prefix{x}{y}{(\outputp{x}{y} | @{y})} & \nonumber\\
	\red
	& (\outputp{x}{y} | @{y})\substn{\quotep{(\prefix{x}{y}{(@{y} | \outputp{x}{y})) | P}}}{y} & \nonumber\\
	=
	& \outputp{x}{\quotep{(\prefix{x}{y}{(\outputp{x}{y} | @{y})) | P}}}
	  | {(\prefix{x}{y}{(\outputp{x}{y} | @{y})) | P}} & \nonumber\\
	\red
	& \ldots & \nonumber\\
	\red^*
	& P | P | \ldots & \nonumber
\end{eqnarray}

Of course, this encoding, as an implementation, runs away, unfolding
$\bangp{P}$ eagerly. A lazier and more implementable replication
operator, restricted to input-guarded processes, may be obtained as follows.

\begin{eqnarray}
\bangp{\prefix{u}{v}{P}} 
	:= 
	\binpar{\lift{x}{\prefix{u}{v}{(\binpar{D(x)}{P})}}}{D(x)} \nonumber
\end{eqnarray}

\begin{remark}
  Note that the lazier definition still does not deal with summation
  or mixed summation (i.e. sums over input and output). The reader is
  invited to construct definitions of replication that deal with these
  features. 

  Further, the definitions are parameterized in a name, $x$. Can you,
  gentle reader, make a definition that eliminates this parameter and
  guarantees no accidental interaction between the replication
  machinery and the process being replicated -- i.e. no accidental
  sharing of names used by the process to get its work done and the
  name(s) used by the replication to effect copying. This latter
  revision of the definition of replication is crucial to obtaining
  the expected identity $!!P \sim !P$.
\end{remark}

\begin{remark}\label{rem:paradoxical_combinator}
  The reader familiar with the lambda calculus will have noticed the
  similarity between $D$ and the paradoxical combinator.

  [Ed. note: the existence of this seems to suggest we have to be more
  restrictive on the set of processes and names we admit if we are to
  support no-cloning.]
\end{remark}

\subsubsection{Bisimulation}

The computational dynamics gives rise to another kind of equivalence,
the equivalence of computational behavior. As previously mentioned
this is typically captured \emph{via} some form of bisimulation.

% The notion we use in this paper is weak barbed bisimulation
% \cite{milner91polyadicpi}.

The notion we use in this paper is derived from weak barbed
bisimulation \cite{milner91polyadicpi}. 

\begin{definition}
An \emph{observation relation}, $\downarrow_{\mathcal N}$, over a set
of names, $\mathcal N$, is the smallest relation satisfying the rules
below.

\infrule[Out-barb]{y \in {\mathcal N}, \; x \nameeq y}
		  {\outputp{x}{v} \downarrow_{\mathcal N} x}
\infrule[Par-barb]{\mbox{$P\downarrow_{\mathcal N} x$ or $Q\downarrow_{\mathcal N} x$}}
		  {\binpar{P}{Q} \downarrow_{\mathcal N} x}

We write $P \Downarrow_{\mathcal N} x$ if there is $Q$ such that 
$P \wred Q$ and $Q \downarrow_{\mathcal N} x$.
\end{definition}

\begin{definition}
%\label{def.bbisim}
An  ${\mathcal N}$-\emph{barbed bisimulation} over a set of names, ${\mathcal N}$, is a symmetric binary relation 
${\mathcal S}_{\mathcal N}$ between agents such that $P\rel{S}_{\mathcal N}Q$ implies:
\begin{enumerate}
\item If $P \red P'$ then $Q \wred Q'$ and $P'\rel{S}_{\mathcal N} Q'$.
\item If $P\downarrow_{\mathcal N} x$, then $Q\Downarrow_{\mathcal N} x$.
\end{enumerate}
$P$ is ${\mathcal N}$-barbed bisimilar to $Q$, written
$P \wbbisim_{\mathcal N} Q$, if $P \rel{S}_{\mathcal N} Q$ for some ${\mathcal N}$-barbed bisimulation ${\mathcal S}_{\mathcal N}$.
\end{definition}

$\mathcal{R} \subseteq \pi \times \pi$

$P \mathcal{R} Q => \forall P'. P \red P' \Rightarrow \exists Q'. Q \red Q', P' \mathcal{R} Q'$

$P \vdash x \Rightarrow Q \vdash x$

\begin{mathpar}
  \inferrule*[lab=Out-barb]{x \nameeq y}{{y}!\langle{Q}\rangle \vdash x}
  \and
  \inferrule*[lab=Par-barb]{\mbox{$P\vdash x$ or $Q\vdash x$}}{\binpar{P}{Q} \vdash x}
\end{mathpar}

\subsubsection{Contexts}

One of the principle advantages of computational calculi like the
$\pi$-calculus is a well-defined notion of context,
contextual-equivalence and a correlation between
contextual-equivalence and notions of bisimulation. The notion of
context allows the decomposition of a process into (sub-)process and
its syntactic environment, its context. Thus, a context may be
thought of as a process with a ``hole'' (written $\Box$) in it. The
application of a context $M$ to a process $P$, written $M[P]$, is
tantamount to filling the hole in $M$ with $P$. In this paper we do
not need the full weight of this theory, but do make use of the notion
of context in the proof the main theorem. 

\begin{mathpar}
  \inferrule* [lab=summation] {} {{M_{M},M_{N}} \bc \Box \;|\; x.M_{A} \;|\; M_{M}+M_{N}}
  \and
  \inferrule* [lab=agent] {} {{M_{A}} \bc (\vec{x})M_{P} \;| \; \clift{P_0,\ldots,M_{P},\ldots,P_N}}
  \and \\
  \inferrule* [lab=process] {} {{M_{P}} \bc M_{N} \;| \;P|M_{P} }
\end{mathpar} 

\begin{mathpar}
  \inferrule* [lab=sychronization] {} {M_{N} \bc \Box \;|\; x?M_{F} \;|\; x!M_{C}}
  \and
  \inferrule* [lab=abstraction] {} {{M_{F}} \bc (x)M_{P} }
  \and
  \inferrule* [lab=concretion] {} {{M_{C}} \bc \langle M_{P} \rangle }
  \and \\
  \inferrule* [lab=process] {} {{M_{P}} \bc M_{N} \;| \;P|M_{P} }
\end{mathpar}

\begin{definition}[contextual application] Given a context $M$, and
  process $P$, we define the \emph{contextual application}, $M[P] :=
  M\{P/\Box\}$. That is, the contextual application of M to P is the
  substitution of $P$ for $\Box$ in $M$.
\end{definition}

$\meaningof{-} : L \to \mathcal{P}(\pi)$

\begin{mathpar}
  \inferrule* [lab=collection] {} {\meaningof{true} = \pi, \and \meaningof{~E} = \pi \setminus \meaningof{E}, \and \meaningof{E_{1} \& E_{2}} = \meaningof{E_{1}} \cap \meaningof{E_{2}}}
\end{mathpar}

\begin{mathpar}
  \inferrule* [lab=structure] {} {\meaningof{0} = \{ P \in \pi | P \equiv 0 \}, \and \\ \meaningof{E_1 | E_2} = \{ P \in \pi | P \equiv P_{1} | P_{2}, P_{1} \in \meaningof{E_{1}}, P_{2} \in \meaningof{E_2}\} }
\end{mathpar}

\begin{mathpar}
 \inferrule* [lab=behavior] {} {\meaningof{\langle a?b \rangle E} = \{ P \in \pi | P \equiv Q | u?(y)P', \\ \and \\\\ \and \\ \;\;\; u \in \meaningof{a}, \forall z.P'\{z/y\} \in \meaningof{E\{z/b\}}\}, \and \\ \meaningof{a!E} = \{ P \in \pi | P \equiv Q | x!\langle P' \rangle, x \in \meaningof{a} P' \in \meaningof{E}\} }
\end{mathpar}

\begin{mathpar}
 \inferrule* [lab=nominal] {} {\meaningof{\quotep{E}} = \{ \quotep{P} \in \quotep{\pi} | P \in \meaningof{E} \}, \and \meaningof{\quotep{P}} = \{ \quotep{Q} \in \quotep{\pi} | P \equiv Q \} \and \\ \meaningof{@\quotep{E}} = \{ P \in \pi | P \equiv @x, x \in \meaningof{E} \}}
\end{mathpar}

\begin{eqnarray*}
  \\
  \meaningof{-} : TS \to ST
\end{eqnarray*}

\begin{eqnarray*}
  \\
  L : TS \to ST
\end{eqnarray*}

\begin{eqnarray*}
  \\
  P \models E \iff P \in \meaningof{E}
\end{eqnarray*}

\begin{eqnarray*}
  P \approx_{L} Q \iff \forall E \in L. P \models E \iff Q \models E
\end{eqnarray*}

\begin{eqnarray*}
  P \approx_{K} Q
\end{eqnarray*}

\begin{eqnarray*}
  P \approx Q
\end{eqnarray*}

$\approx_{K} = \approx = \approx_{L}$

\subsubsection{Contextual duality}

Note that contexts extend the quotation operation to a family of
operations from processes to names. Given a context, $M$, we can
define a \emph{nominal context}, $\quotep{M}$ by $\quotep{M}[P] :=
\quotep{M[P]}$. To foreshadow what is to come we observe that these
operations enjoy a duality with processes very much like the duality
between vectors and maps from vectors to scalars.

Further, because the calculus is essentially higher-order, we have a
correspondence between contexts and processes. More specifically,
given a name $x$ and a context $M$ we can construct $M^{*}_{x}$ such
that 

\begin{mathpar}
  M^{*}_{x} | \lift{x}{P} \red M[P]
\end{mathpar}

namely,

\begin{mathpar}
  M^{*}_{x} := x?(u).M[\dropn{u}]
\end{mathpar}

The dependence of $M^{*}_{x}$ on a name makes it an abstraction, 

\begin{mathpar}
  M^{*} := (x)x?(u).M[\dropn{u}]
\end{mathpar}

\subsection{Additional notation}

It will sometimes be convenient to denote the process a name
quotes. We already have the notation $x = \quotep{P}$, but it will be
convenient to introduce an alternate notation, $\procn{x}$, when we
want to emphasize the connection to the use of the name. Note that, by
virtue of name equivalence, $\quotep{\procn{x}} \nameeq x$; so, the
notation is consistent with previous definitions.

Further, because names have structure it is possible to effect
substitutions on the basis of that structure. This means we need to
upgrade our notation for substitutions, which we accomplish by
adapting comprehension notation. Thus,

\begin{mathpar}
  P\{ y / x : x \in S \}
\end{mathpar}

is interpreted to mean the process derived from P by replacing (in a
capture-avoiding manner) each occurrence of $x$ in $S$ by $y$. For example,

\begin{mathpar}
  P\{ \quotep{\procn{x}|\procn{x}} / x : x \in \freenames{P} \}
\end{mathpar}

will replace each (occurrence) of a free name $x$ in $P$ by
$\quotep{\procn{x}|\procn{x}}$.

Also, we will avail ourselves of the notation $x^{L}$ and $x^{R}$ to
denote injections of a name into disjoint copies of the name
space. There are numerous ways to accomplish this. One example can be
found in \cite{MeredithR05}. This notation overloads to vectors of
names: $\vec{x}^{\pi} := (x_{i}^{\pi} \; : \; 0 \leq i < |\vec{x}| )$ where $\pi \in \{L,R\}$.

We also use $P^{\Box} := P|\Box$.

In \cite{MeredithR05} an interpretation of the new operator is
given. It turns out that there are several possible interpretations
all enjoying the requisite algebraic properties of the operator (see
\cite{milner91polyadicpi}). We will therefore make liberal use of
$(\nu\; \vec{x})P$.

% subsection the_syntax_and_semantics_of_the_notation_system (end)   

\input{qm2pi.qmops} 

\input{qm2pi.sterngerlach} 

\input{qm2pi.metric} 

% section concurrent_process_calculi (end)

%\input{qm2pi.proofsketch}

% section proof sketch (end)

%\input{qm2pi.slviaknots} 

% section spatial logic via knots (end)

\input{qm2pi.conclusion}

% section conclusion (end)

%\input{qm2pi.dtcodes} 

% section wiring algorithm (end)

\input{qm2pi.ack} 

% section acknowledgments (end)

\newpage


\bibliographystyle{plain}   
\bibliography{../../biblios/main.bib}

\input{qm2pi.rhodetails}

\end{document}

 

% section wiring algorithm (end)

\documentclass[12pt]{llncs}
%\documentclass{jktr}

\usepackage[pdftex]{hyperref}                   
\usepackage {listings}
\usepackage {mathpartir}
\usepackage{bcprules}
%\usepackage{listings}
                       
\usepackage{graphicx} 
%\usepackage[margins=2.5cm,nohead,nofoot]{geometry}
%\usepackage{geometry}
\usepackage{amsfonts}
\usepackage{amstext}
\usepackage{latexsym}
\usepackage{amssymb}
\usepackage{color}


%\include{myPreamble}
\include{qm2pi.local} 

%\ifpdf
%\usepackage[pdftex]{graphicx}
%\else
%\usepackage{graphicx}
%\fi

 % \ifpdf
%  \usepackage{pdfsync}
%  \if


%\title{Brief Article}
%\author{David F. Snyder}
%\author{L.G. Meredith}

%\address{Dept. of Math., Texas State University--San Marcos, San Marcos, TX 78666}
       
\pagestyle{empty}


\begin{document}

\lstset{language=[Objective]Caml,frame=shadowbox}

\input{qm2pi.front}

% section front matter (end)

\input{qm2pi.intro} 
 
% section introduction (end)

% \input{qm2pi.knotations} 

% section notation (end)

\input{qm2pi.process.calculi} 

% section concurrent_process_calculi_and_spatial_logics_ (end)
    
%\input{qm2pi.knots2pi} 

%\input{qm2pi.trefoil} 

%\input{qm2pi.mainthm} 

% subsection basic_interpretation (end)

%\input{qm2pi.rho.presentation} 
\subsection{The syntax and semantics of the notation system}\label{sub:the_syntax_and_semantics_of_the_notation_system} % (fold)

We now summarize a technical presentation of the calculus that
embodies our theory of dynamics. The typical presentation of such a
calculus follows the style of giving generators and relations on
them. The grammar, below, describing term constructors, freely
generates the set of processes, $\Proc$. This set is then quotiented
by a relation known as structural congruence and it is over this set
that the notion of dynamics is expressed. This presentation is
essentially that of \cite{MeredithR05} with the addition of
polyadicity and summation. For readability we have relegated some of
the technical subtleties to an appendix.

\subsubsection{Process grammar}\label{subsub:process_grammar}

\begin{mathpar}
  \inferrule* [lab=synchronization] {} {{M} \bc \pzero \;|\; x?F \;|\; x!C }
  \and
  \inferrule* [lab=abstraction] {} {{F} \bc (x)P}
  \and
  \inferrule* [lab=concretion] {} {{C} \bc \langle Q \rangle}
  \and
  \inferrule* [lab=process] {} {{P,Q} \bc M \;| \;P|Q \;|\; @{x}}
  \and
  \inferrule* [lab=name] {} {{x} \bc \quotep{P}}
\end{mathpar} 

Note that $\vec{x}$ (resp. $\vec{P}$) denotes a vector of names
(resp. processes) of length $|\vec{x}|$ (resp. $|\vec{P}|$). We adopt
the following useful abbreviations.

\begin{mathpar}
   x?(\vec{y}).P := x.(\vec{y})P \and  x\clift{\vec{P}} := x.\clift{\vec{P}}
   \and x!(y) := \lift{x}{\dropn{y}}
   \and \Pi_{i=0}^{n-1}P_i := P_0 | \ldots | P_{n-1}
\end{mathpar}

\subsubsection{Structural congruence}

\paragraph{Free and bound names and alpha-equivalence.} At the
core of structural equivalence is alpha-equivalence which identifies
process that are the same up to a change of variable. Formally, we
recognize the distinction between free and bound names. The free names
of a process, $\freenames{P}$, may be calculated recursively as
follows:

\begin{mathpar}
\freenames{\pzero} := \emptyset
  \and \\
  \freenames{x?(y).P} := \{ x \} \cup (\freenames{P} \setminus \{ y \})
  \and 
  \freenames{x!\langle P \rangle} := \{ x \} \cup \{ P \} 
  \and \\
  \freenames{P|Q} := \freenames{P} \cup \freenames{Q}
  \and \\
  \freenames{@{x}} := \{ x \}
\end{mathpar}

$\pi$
$\quotep{\pi}$

$\freenames{-} : \pi \to \mathcal{P}(\quotep{\pi})$

\begin{eqnarray*}
  \freenames{\pzero} & := & \emptyset \\
  \freenames{x?(y).P} & := & \{ x \} \cup (\freenames{P} \setminus \{ y \}) \\
  \freenames{x!\langle P \rangle} & := & \{ x \} \cup \{ P \} \\
  \freenames{P|Q} & := & \freenames{P} \cup \freenames{Q} \\
  \freenames{\dropn{x}} & := & \{ x \}
\end{eqnarray*}

The bound names of a process, $\boundnames{P}$, are those names occurring in $P$
that are not free. For example, in $x?(y).0$, the name $x$ is free, while $y$ is bound.

\begin{mathpar}
  \inferrule* [lab=monoidal-laws] {} { P|Q \equiv Q|P \and P|0 \equiv P \and P|(Q|R) \equiv (P|Q)|R }
\end{mathpar}

\begin{mathpar}
  \inferrule* [lab=alpha-equivalence] {} { (x)P \equiv (y)P\{y/x\} \and y \not\in \freenames{P} }
\end{mathpar}

\begin{definition}
Then two processes, $P,Q$, are alpha-equivalent if $P = Q\{\vec{y}/\vec{x}\}$ for
some $\vec{x} \in \boundnames{Q},\vec{y} \in \boundnames{P}$, where $Q\{\vec{y}/\vec{x}\}$
denotes the capture-avoiding substitution of $\vec{y}$ for $\vec{x}$ in $Q$.
\end{definition}

\begin{definition}
  The {\em structural congruence} \cite{SangiorgiWalker} , $\equiv$,
  between processes is the least congruence containing
  alpha-equivalence, satisfying the abelian monoid laws
  (associativity, commutativity and $\pzero$ as identity) for parallel
  composition $|$ and for summation $+$.
\end{definition}

\subsection{Name equivalence}

We take name equivalence, written $\nameeq$, to be the smallest
equivalence relation generated by the following rules.

\begin{mathpar}
\inferrule*[lab=Quote-drop]
{ }
{ \quotep{@{x}} \nameeq x }

\inferrule*[lab=Struct-equiv]
{ P \scong Q }
{ \quotep{P} \nameeq \quotep{Q} }
\end{mathpar}

The astute reader will have noticed that the mutual recursion of names
and processes imposes a mutual recursion on alpha-equivalence and
structural equivalence via name-equivalence. Fortunately, all of this
works out pleasantly and we may calculate in the natural way, free of
concern. The reader interested in the details is referred to the
appendix \ref{appendix:rho_details}.

\subsection{Substitution}

We use $\Proc$ for the set of processes, $\QProc$ for the set of
names, and $\id{\{}\vec{y} / \vec{x} \id{\}}$ to denote partial maps,
$s : \QProc \rightarrow \QProc$. A map, $s$ lifts, uniquely, to a map
on process terms, $\widehat{s} : \Proc \rightarrow \Proc$ by the
following equations.

\begin{mathpar}
  (0) \psubstp{Q}{P} := 0 \\
  (R \juxtap S) \psubstp{Q}{P}
  :=    
  (R)\psubstp{Q}{P} \juxtap (S) \psubstp{Q}{P} \\
  (x?(y).R) \psubstp{Q}{P}    
  :=    
  (x)\substp{Q}{P} (z)\concat( (R \psubstn{z}{y}) \psubstp{Q}{P} ) \\
  (\lift{x}{R}) \psubstp{Q}{P}  
  :=
  \lift{(x)\substp{Q}{P}}{ R \psubstp{Q}{P} } \\
%   (\dropn{x})  \psubstp{Q}{P}       
%   := 
%   \left\{ 
%     \begin{array}{ccc} 
%       \dropn{\quotep{Q}} & & x \nameeq \quotep{P} \\
%       \dropn{x} & & otherwise \\
%     \end{array}
%   \right. 
  (\dropn{x})  \psubstp{Q}{P}       
  := 
  \left\{ 
    \begin{array}{ccc} 
      Q & & x \nameeq \quotep{P} \\
      \dropn{x} & & otherwise \\
    \end{array}
  \right.
\end{mathpar}
 

where

\begin{eqnarray}
  (x)\id{\{} \lpquote Q \rpquote / \lpquote P \rpquote \id{\}}            = 
  \left\{ 
    \begin{array}{ccc}
      \lpquote Q \rpquote & & x \nameeq \lpquote P \rpquote \\
      x & & otherwise \\
    \end{array}
  \right. \nonumber
\end{eqnarray}

and $z$ is chosen distinct from $\quotep{P}$, $\quotep{Q}$, the free
names in $Q$, and all the names in $R$. Our $\alpha$-equivalence will
be built in the standard way from this substitution.

\begin{remark}\label{rem:no_self_referential_names}
  One consequence of these definitions is that $\forall P. \quotep{P}
  \not\in \freenames{P}$.
\end{remark}

\subsection{ Dynamic quote: an example }

Anticipating something of what's to come, consider applying the
substitution, $\widehat{\id{\{}u / z \id{\}}}$, to the following pair
of processes, $\lift{w}{y!(z)}$ and $w[ \lpquote y!(z) \rpquote ]$.

\begin{eqnarray}
	\lift{w}{y!(z)}\widehat{\id{\{}u / z \id{\}}}
		& = &
		\lift{w}{y!(u)} \nonumber\\
	w[ \lpquote y!(z) \rpquote ] \widehat{ \id{\{}u / z \id{\}} }
		& = &
		w[ \lpquote y!(z) \rpquote ] \nonumber
\end{eqnarray}

Because the body of the process between quotes is impervious to
substitution, we get radically different answers. In fact, by
examining the first process in an input context,
e.g. $x?(z).\lift{w}{y!(z)}$, we see that the process under the lift
operator may be shaped by prefixed inputs binding a name inside it. In
this sense, the lift operator will be seen as a way to dynamically
construct processes before reifying them as names.

Finally equipped with these standard features we can present the
dynamics of the calculus.

\subsubsection{Operational semantics} 

Finally, we introduce the computational dynamics. What marks these
algebras as distinct from other more traditionally studied algebraic
structures, e.g. vector spaces or polynomial rings, is the manner in
which dynamics is captured. In traditional structures, dynamics is typically
expressed through morphisms between such structures, as in linear maps
between vector spaces or morphisms between rings. In algebras
associated with the semantics of computation, the dynamics is
expressed as part of the algebraic structure itself, through a
reduction reduction relation typically denoted by $\red$. Below, we
give a recursive presentation of this relation for the calculus used
in the encoding.

$\red \subseteq \pi \times \pi$
$\red : \pi \to \mathcal{P}(\pi)$

\begin{mathpar}
  \inferrule* [lab=Comm] { \textsf{match}( x_{src}, x_{trgt} ) } { x_{trgt}?(y)P \; | \; x_{src}!\langle {Q} \rangle \red P\{\quotep{Q}/y}\} }
  \and \\
  \inferrule* [lab=Par] {{P} \red {P}'} {{{P} | {Q}} \red {{P}' | {Q}}}
  \and
  \inferrule* [lab=Equiv]{{{P} \scong {P}'} \andalso {{P}' \red {Q}'} \andalso {{Q}' \scong {Q}}}{{P} \red {Q}}
\end{mathpar}

\begin{eqnarray*}
  match_{\equiv} (\quotep{P},\quotep{Q}) & := & P \equiv Q \\
  match_{\dagger}(\quotep{P},\quotep{Q}) & := & \forall R. P|Q \red^{*} R => R \red^{*} 0 \\
  match_{K}(\quotep{P},\quotep{Q}) & := & K \mbox{ for some context } K
\end{eqnarray*}

$u?(x)P | u!\langle Q \rangle \red P\{\quotep{Q}/x\}$

%We write $\wred$ for $\red^*$, and $P\red$ if $\exists Q $ such that $ P \red Q$.
We write $P\red$ if $\exists Q $ such that $ P \red Q$ and $P\not\red$, otherwise.

\section{Replication}

As mentioned before, it is known that replication (and hence
recursion) can be implemented in a higher-order process algebra
\cite{SangiorgiWalker}. As our first example of calculation with the
machinery thus far presented we give the construction explicitly in
the {\rhoc}.

\begin{eqnarray}
	D_{x} & := & \prefix{x}{y}{(\binpar{\outputp{x}{y}}{@{y}})} \nonumber\\
	\bangp_{x}{P} & := & \binpar{{x}!\langle{\binpar{D_{x}}{P}}\rangle}{D_{x}} \nonumber
\end{eqnarray}

\begin{eqnarray}
	\bangp_{x}{P} & & \nonumber\\
	=
	& {x}!\langle{(\prefix{x}{y}{(\outputp{x}{y} | @{y})) | P}}\rangle 
	      | \prefix{x}{y}{(\outputp{x}{y} | @{y})} & \nonumber\\
	\red
	& (\outputp{x}{y} | @{y})\substn{\quotep{(\prefix{x}{y}{(@{y} | \outputp{x}{y})) | P}}}{y} & \nonumber\\
	=
	& \outputp{x}{\quotep{(\prefix{x}{y}{(\outputp{x}{y} | @{y})) | P}}}
	  | {(\prefix{x}{y}{(\outputp{x}{y} | @{y})) | P}} & \nonumber\\
	\red
	& \ldots & \nonumber\\
	\red^*
	& P | P | \ldots & \nonumber
\end{eqnarray}

Of course, this encoding, as an implementation, runs away, unfolding
$\bangp{P}$ eagerly. A lazier and more implementable replication
operator, restricted to input-guarded processes, may be obtained as follows.

\begin{eqnarray}
\bangp{\prefix{u}{v}{P}} 
	:= 
	\binpar{\lift{x}{\prefix{u}{v}{(\binpar{D(x)}{P})}}}{D(x)} \nonumber
\end{eqnarray}

\begin{remark}
  Note that the lazier definition still does not deal with summation
  or mixed summation (i.e. sums over input and output). The reader is
  invited to construct definitions of replication that deal with these
  features. 

  Further, the definitions are parameterized in a name, $x$. Can you,
  gentle reader, make a definition that eliminates this parameter and
  guarantees no accidental interaction between the replication
  machinery and the process being replicated -- i.e. no accidental
  sharing of names used by the process to get its work done and the
  name(s) used by the replication to effect copying. This latter
  revision of the definition of replication is crucial to obtaining
  the expected identity $!!P \sim !P$.
\end{remark}

\begin{remark}\label{rem:paradoxical_combinator}
  The reader familiar with the lambda calculus will have noticed the
  similarity between $D$ and the paradoxical combinator.

  [Ed. note: the existence of this seems to suggest we have to be more
  restrictive on the set of processes and names we admit if we are to
  support no-cloning.]
\end{remark}

\subsubsection{Bisimulation}

The computational dynamics gives rise to another kind of equivalence,
the equivalence of computational behavior. As previously mentioned
this is typically captured \emph{via} some form of bisimulation.

% The notion we use in this paper is weak barbed bisimulation
% \cite{milner91polyadicpi}.

The notion we use in this paper is derived from weak barbed
bisimulation \cite{milner91polyadicpi}. 

\begin{definition}
An \emph{observation relation}, $\downarrow_{\mathcal N}$, over a set
of names, $\mathcal N$, is the smallest relation satisfying the rules
below.

\infrule[Out-barb]{y \in {\mathcal N}, \; x \nameeq y}
		  {\outputp{x}{v} \downarrow_{\mathcal N} x}
\infrule[Par-barb]{\mbox{$P\downarrow_{\mathcal N} x$ or $Q\downarrow_{\mathcal N} x$}}
		  {\binpar{P}{Q} \downarrow_{\mathcal N} x}

We write $P \Downarrow_{\mathcal N} x$ if there is $Q$ such that 
$P \wred Q$ and $Q \downarrow_{\mathcal N} x$.
\end{definition}

\begin{definition}
%\label{def.bbisim}
An  ${\mathcal N}$-\emph{barbed bisimulation} over a set of names, ${\mathcal N}$, is a symmetric binary relation 
${\mathcal S}_{\mathcal N}$ between agents such that $P\rel{S}_{\mathcal N}Q$ implies:
\begin{enumerate}
\item If $P \red P'$ then $Q \wred Q'$ and $P'\rel{S}_{\mathcal N} Q'$.
\item If $P\downarrow_{\mathcal N} x$, then $Q\Downarrow_{\mathcal N} x$.
\end{enumerate}
$P$ is ${\mathcal N}$-barbed bisimilar to $Q$, written
$P \wbbisim_{\mathcal N} Q$, if $P \rel{S}_{\mathcal N} Q$ for some ${\mathcal N}$-barbed bisimulation ${\mathcal S}_{\mathcal N}$.
\end{definition}

$\mathcal{R} \subseteq \pi \times \pi$

$P \mathcal{R} Q => \forall P'. P \red P' \Rightarrow \exists Q'. Q \red Q', P' \mathcal{R} Q'$

$P \vdash x \Rightarrow Q \vdash x$

\begin{mathpar}
  \inferrule*[lab=Out-barb]{x \nameeq y}{{y}!\langle{Q}\rangle \vdash x}
  \and
  \inferrule*[lab=Par-barb]{\mbox{$P\vdash x$ or $Q\vdash x$}}{\binpar{P}{Q} \vdash x}
\end{mathpar}

\subsubsection{Contexts}

One of the principle advantages of computational calculi like the
$\pi$-calculus is a well-defined notion of context,
contextual-equivalence and a correlation between
contextual-equivalence and notions of bisimulation. The notion of
context allows the decomposition of a process into (sub-)process and
its syntactic environment, its context. Thus, a context may be
thought of as a process with a ``hole'' (written $\Box$) in it. The
application of a context $M$ to a process $P$, written $M[P]$, is
tantamount to filling the hole in $M$ with $P$. In this paper we do
not need the full weight of this theory, but do make use of the notion
of context in the proof the main theorem. 

\begin{mathpar}
  \inferrule* [lab=summation] {} {{M_{M},M_{N}} \bc \Box \;|\; x.M_{A} \;|\; M_{M}+M_{N}}
  \and
  \inferrule* [lab=agent] {} {{M_{A}} \bc (\vec{x})M_{P} \;| \; \clift{P_0,\ldots,M_{P},\ldots,P_N}}
  \and \\
  \inferrule* [lab=process] {} {{M_{P}} \bc M_{N} \;| \;P|M_{P} }
\end{mathpar} 

\begin{mathpar}
  \inferrule* [lab=sychronization] {} {M_{N} \bc \Box \;|\; x?M_{F} \;|\; x!M_{C}}
  \and
  \inferrule* [lab=abstraction] {} {{M_{F}} \bc (x)M_{P} }
  \and
  \inferrule* [lab=concretion] {} {{M_{C}} \bc \langle M_{P} \rangle }
  \and \\
  \inferrule* [lab=process] {} {{M_{P}} \bc M_{N} \;| \;P|M_{P} }
\end{mathpar}

\begin{definition}[contextual application] Given a context $M$, and
  process $P$, we define the \emph{contextual application}, $M[P] :=
  M\{P/\Box\}$. That is, the contextual application of M to P is the
  substitution of $P$ for $\Box$ in $M$.
\end{definition}

$\meaningof{-} : L \to \mathcal{P}(\pi)$

\begin{mathpar}
  \inferrule* [lab=collection] {} {\meaningof{true} = \pi, \and \meaningof{~E} = \pi \setminus \meaningof{E}, \and \meaningof{E_{1} \& E_{2}} = \meaningof{E_{1}} \cap \meaningof{E_{2}}}
\end{mathpar}

\begin{mathpar}
  \inferrule* [lab=structure] {} {\meaningof{0} = \{ P \in \pi | P \equiv 0 \}, \and \\ \meaningof{E_1 | E_2} = \{ P \in \pi | P \equiv P_{1} | P_{2}, P_{1} \in \meaningof{E_{1}}, P_{2} \in \meaningof{E_2}\} }
\end{mathpar}

\begin{mathpar}
 \inferrule* [lab=behavior] {} {\meaningof{\langle a?b \rangle E} = \{ P \in \pi | P \equiv Q | u?(y)P', \\ \and \\\\ \and \\ \;\;\; u \in \meaningof{a}, \forall z.P'\{z/y\} \in \meaningof{E\{z/b\}}\}, \and \\ \meaningof{a!E} = \{ P \in \pi | P \equiv Q | x!\langle P' \rangle, x \in \meaningof{a} P' \in \meaningof{E}\} }
\end{mathpar}

\begin{mathpar}
 \inferrule* [lab=nominal] {} {\meaningof{\quotep{E}} = \{ \quotep{P} \in \quotep{\pi} | P \in \meaningof{E} \}, \and \meaningof{\quotep{P}} = \{ \quotep{Q} \in \quotep{\pi} | P \equiv Q \} \and \\ \meaningof{@\quotep{E}} = \{ P \in \pi | P \equiv @x, x \in \meaningof{E} \}}
\end{mathpar}

\begin{eqnarray*}
  \\
  \meaningof{-} : TS \to ST
\end{eqnarray*}

\begin{eqnarray*}
  \\
  L : TS \to ST
\end{eqnarray*}

\begin{eqnarray*}
  \\
  P \models E \iff P \in \meaningof{E}
\end{eqnarray*}

\begin{eqnarray*}
  P \approx_{L} Q \iff \forall E \in L. P \models E \iff Q \models E
\end{eqnarray*}

\begin{eqnarray*}
  P \approx_{K} Q
\end{eqnarray*}

\begin{eqnarray*}
  P \approx Q
\end{eqnarray*}

$\approx_{K} = \approx = \approx_{L}$

\subsubsection{Contextual duality}

Note that contexts extend the quotation operation to a family of
operations from processes to names. Given a context, $M$, we can
define a \emph{nominal context}, $\quotep{M}$ by $\quotep{M}[P] :=
\quotep{M[P]}$. To foreshadow what is to come we observe that these
operations enjoy a duality with processes very much like the duality
between vectors and maps from vectors to scalars.

Further, because the calculus is essentially higher-order, we have a
correspondence between contexts and processes. More specifically,
given a name $x$ and a context $M$ we can construct $M^{*}_{x}$ such
that 

\begin{mathpar}
  M^{*}_{x} | \lift{x}{P} \red M[P]
\end{mathpar}

namely,

\begin{mathpar}
  M^{*}_{x} := x?(u).M[\dropn{u}]
\end{mathpar}

The dependence of $M^{*}_{x}$ on a name makes it an abstraction, 

\begin{mathpar}
  M^{*} := (x)x?(u).M[\dropn{u}]
\end{mathpar}

\subsection{Additional notation}

It will sometimes be convenient to denote the process a name
quotes. We already have the notation $x = \quotep{P}$, but it will be
convenient to introduce an alternate notation, $\procn{x}$, when we
want to emphasize the connection to the use of the name. Note that, by
virtue of name equivalence, $\quotep{\procn{x}} \nameeq x$; so, the
notation is consistent with previous definitions.

Further, because names have structure it is possible to effect
substitutions on the basis of that structure. This means we need to
upgrade our notation for substitutions, which we accomplish by
adapting comprehension notation. Thus,

\begin{mathpar}
  P\{ y / x : x \in S \}
\end{mathpar}

is interpreted to mean the process derived from P by replacing (in a
capture-avoiding manner) each occurrence of $x$ in $S$ by $y$. For example,

\begin{mathpar}
  P\{ \quotep{\procn{x}|\procn{x}} / x : x \in \freenames{P} \}
\end{mathpar}

will replace each (occurrence) of a free name $x$ in $P$ by
$\quotep{\procn{x}|\procn{x}}$.

Also, we will avail ourselves of the notation $x^{L}$ and $x^{R}$ to
denote injections of a name into disjoint copies of the name
space. There are numerous ways to accomplish this. One example can be
found in \cite{MeredithR05}. This notation overloads to vectors of
names: $\vec{x}^{\pi} := (x_{i}^{\pi} \; : \; 0 \leq i < |\vec{x}| )$ where $\pi \in \{L,R\}$.

We also use $P^{\Box} := P|\Box$.

In \cite{MeredithR05} an interpretation of the new operator is
given. It turns out that there are several possible interpretations
all enjoying the requisite algebraic properties of the operator (see
\cite{milner91polyadicpi}). We will therefore make liberal use of
$(\nu\; \vec{x})P$.

% subsection the_syntax_and_semantics_of_the_notation_system (end)   

\input{qm2pi.qmops} 

\input{qm2pi.sterngerlach} 

\input{qm2pi.metric} 

% section concurrent_process_calculi (end)

%\input{qm2pi.proofsketch}

% section proof sketch (end)

%\input{qm2pi.slviaknots} 

% section spatial logic via knots (end)

\input{qm2pi.conclusion}

% section conclusion (end)

%\input{qm2pi.dtcodes} 

% section wiring algorithm (end)

\input{qm2pi.ack} 

% section acknowledgments (end)

\newpage


\bibliographystyle{plain}   
\bibliography{../../biblios/main.bib}

\input{qm2pi.rhodetails}

\end{document}

 

% section acknowledgments (end)

\newpage


\bibliographystyle{plain}   
\bibliography{../../biblios/main.bib}

\documentclass[12pt]{llncs}
%\documentclass{jktr}

\usepackage[pdftex]{hyperref}                   
\usepackage {listings}
\usepackage {mathpartir}
\usepackage{bcprules}
%\usepackage{listings}
                       
\usepackage{graphicx} 
%\usepackage[margins=2.5cm,nohead,nofoot]{geometry}
%\usepackage{geometry}
\usepackage{amsfonts}
\usepackage{amstext}
\usepackage{latexsym}
\usepackage{amssymb}
\usepackage{color}


%\include{myPreamble}
\include{qm2pi.local} 

%\ifpdf
%\usepackage[pdftex]{graphicx}
%\else
%\usepackage{graphicx}
%\fi

 % \ifpdf
%  \usepackage{pdfsync}
%  \if


%\title{Brief Article}
%\author{David F. Snyder}
%\author{L.G. Meredith}

%\address{Dept. of Math., Texas State University--San Marcos, San Marcos, TX 78666}
       
\pagestyle{empty}


\begin{document}

\lstset{language=[Objective]Caml,frame=shadowbox}

\input{qm2pi.front}

% section front matter (end)

\input{qm2pi.intro} 
 
% section introduction (end)

% \input{qm2pi.knotations} 

% section notation (end)

\input{qm2pi.process.calculi} 

% section concurrent_process_calculi_and_spatial_logics_ (end)
    
%\input{qm2pi.knots2pi} 

%\input{qm2pi.trefoil} 

%\input{qm2pi.mainthm} 

% subsection basic_interpretation (end)

%\input{qm2pi.rho.presentation} 
\subsection{The syntax and semantics of the notation system}\label{sub:the_syntax_and_semantics_of_the_notation_system} % (fold)

We now summarize a technical presentation of the calculus that
embodies our theory of dynamics. The typical presentation of such a
calculus follows the style of giving generators and relations on
them. The grammar, below, describing term constructors, freely
generates the set of processes, $\Proc$. This set is then quotiented
by a relation known as structural congruence and it is over this set
that the notion of dynamics is expressed. This presentation is
essentially that of \cite{MeredithR05} with the addition of
polyadicity and summation. For readability we have relegated some of
the technical subtleties to an appendix.

\subsubsection{Process grammar}\label{subsub:process_grammar}

\begin{mathpar}
  \inferrule* [lab=synchronization] {} {{M} \bc \pzero \;|\; x?F \;|\; x!C }
  \and
  \inferrule* [lab=abstraction] {} {{F} \bc (x)P}
  \and
  \inferrule* [lab=concretion] {} {{C} \bc \langle Q \rangle}
  \and
  \inferrule* [lab=process] {} {{P,Q} \bc M \;| \;P|Q \;|\; @{x}}
  \and
  \inferrule* [lab=name] {} {{x} \bc \quotep{P}}
\end{mathpar} 

Note that $\vec{x}$ (resp. $\vec{P}$) denotes a vector of names
(resp. processes) of length $|\vec{x}|$ (resp. $|\vec{P}|$). We adopt
the following useful abbreviations.

\begin{mathpar}
   x?(\vec{y}).P := x.(\vec{y})P \and  x\clift{\vec{P}} := x.\clift{\vec{P}}
   \and x!(y) := \lift{x}{\dropn{y}}
   \and \Pi_{i=0}^{n-1}P_i := P_0 | \ldots | P_{n-1}
\end{mathpar}

\subsubsection{Structural congruence}

\paragraph{Free and bound names and alpha-equivalence.} At the
core of structural equivalence is alpha-equivalence which identifies
process that are the same up to a change of variable. Formally, we
recognize the distinction between free and bound names. The free names
of a process, $\freenames{P}$, may be calculated recursively as
follows:

\begin{mathpar}
\freenames{\pzero} := \emptyset
  \and \\
  \freenames{x?(y).P} := \{ x \} \cup (\freenames{P} \setminus \{ y \})
  \and 
  \freenames{x!\langle P \rangle} := \{ x \} \cup \{ P \} 
  \and \\
  \freenames{P|Q} := \freenames{P} \cup \freenames{Q}
  \and \\
  \freenames{@{x}} := \{ x \}
\end{mathpar}

$\pi$
$\quotep{\pi}$

$\freenames{-} : \pi \to \mathcal{P}(\quotep{\pi})$

\begin{eqnarray*}
  \freenames{\pzero} & := & \emptyset \\
  \freenames{x?(y).P} & := & \{ x \} \cup (\freenames{P} \setminus \{ y \}) \\
  \freenames{x!\langle P \rangle} & := & \{ x \} \cup \{ P \} \\
  \freenames{P|Q} & := & \freenames{P} \cup \freenames{Q} \\
  \freenames{\dropn{x}} & := & \{ x \}
\end{eqnarray*}

The bound names of a process, $\boundnames{P}$, are those names occurring in $P$
that are not free. For example, in $x?(y).0$, the name $x$ is free, while $y$ is bound.

\begin{mathpar}
  \inferrule* [lab=monoidal-laws] {} { P|Q \equiv Q|P \and P|0 \equiv P \and P|(Q|R) \equiv (P|Q)|R }
\end{mathpar}

\begin{mathpar}
  \inferrule* [lab=alpha-equivalence] {} { (x)P \equiv (y)P\{y/x\} \and y \not\in \freenames{P} }
\end{mathpar}

\begin{definition}
Then two processes, $P,Q$, are alpha-equivalent if $P = Q\{\vec{y}/\vec{x}\}$ for
some $\vec{x} \in \boundnames{Q},\vec{y} \in \boundnames{P}$, where $Q\{\vec{y}/\vec{x}\}$
denotes the capture-avoiding substitution of $\vec{y}$ for $\vec{x}$ in $Q$.
\end{definition}

\begin{definition}
  The {\em structural congruence} \cite{SangiorgiWalker} , $\equiv$,
  between processes is the least congruence containing
  alpha-equivalence, satisfying the abelian monoid laws
  (associativity, commutativity and $\pzero$ as identity) for parallel
  composition $|$ and for summation $+$.
\end{definition}

\subsection{Name equivalence}

We take name equivalence, written $\nameeq$, to be the smallest
equivalence relation generated by the following rules.

\begin{mathpar}
\inferrule*[lab=Quote-drop]
{ }
{ \quotep{@{x}} \nameeq x }

\inferrule*[lab=Struct-equiv]
{ P \scong Q }
{ \quotep{P} \nameeq \quotep{Q} }
\end{mathpar}

The astute reader will have noticed that the mutual recursion of names
and processes imposes a mutual recursion on alpha-equivalence and
structural equivalence via name-equivalence. Fortunately, all of this
works out pleasantly and we may calculate in the natural way, free of
concern. The reader interested in the details is referred to the
appendix \ref{appendix:rho_details}.

\subsection{Substitution}

We use $\Proc$ for the set of processes, $\QProc$ for the set of
names, and $\id{\{}\vec{y} / \vec{x} \id{\}}$ to denote partial maps,
$s : \QProc \rightarrow \QProc$. A map, $s$ lifts, uniquely, to a map
on process terms, $\widehat{s} : \Proc \rightarrow \Proc$ by the
following equations.

\begin{mathpar}
  (0) \psubstp{Q}{P} := 0 \\
  (R \juxtap S) \psubstp{Q}{P}
  :=    
  (R)\psubstp{Q}{P} \juxtap (S) \psubstp{Q}{P} \\
  (x?(y).R) \psubstp{Q}{P}    
  :=    
  (x)\substp{Q}{P} (z)\concat( (R \psubstn{z}{y}) \psubstp{Q}{P} ) \\
  (\lift{x}{R}) \psubstp{Q}{P}  
  :=
  \lift{(x)\substp{Q}{P}}{ R \psubstp{Q}{P} } \\
%   (\dropn{x})  \psubstp{Q}{P}       
%   := 
%   \left\{ 
%     \begin{array}{ccc} 
%       \dropn{\quotep{Q}} & & x \nameeq \quotep{P} \\
%       \dropn{x} & & otherwise \\
%     \end{array}
%   \right. 
  (\dropn{x})  \psubstp{Q}{P}       
  := 
  \left\{ 
    \begin{array}{ccc} 
      Q & & x \nameeq \quotep{P} \\
      \dropn{x} & & otherwise \\
    \end{array}
  \right.
\end{mathpar}
 

where

\begin{eqnarray}
  (x)\id{\{} \lpquote Q \rpquote / \lpquote P \rpquote \id{\}}            = 
  \left\{ 
    \begin{array}{ccc}
      \lpquote Q \rpquote & & x \nameeq \lpquote P \rpquote \\
      x & & otherwise \\
    \end{array}
  \right. \nonumber
\end{eqnarray}

and $z$ is chosen distinct from $\quotep{P}$, $\quotep{Q}$, the free
names in $Q$, and all the names in $R$. Our $\alpha$-equivalence will
be built in the standard way from this substitution.

\begin{remark}\label{rem:no_self_referential_names}
  One consequence of these definitions is that $\forall P. \quotep{P}
  \not\in \freenames{P}$.
\end{remark}

\subsection{ Dynamic quote: an example }

Anticipating something of what's to come, consider applying the
substitution, $\widehat{\id{\{}u / z \id{\}}}$, to the following pair
of processes, $\lift{w}{y!(z)}$ and $w[ \lpquote y!(z) \rpquote ]$.

\begin{eqnarray}
	\lift{w}{y!(z)}\widehat{\id{\{}u / z \id{\}}}
		& = &
		\lift{w}{y!(u)} \nonumber\\
	w[ \lpquote y!(z) \rpquote ] \widehat{ \id{\{}u / z \id{\}} }
		& = &
		w[ \lpquote y!(z) \rpquote ] \nonumber
\end{eqnarray}

Because the body of the process between quotes is impervious to
substitution, we get radically different answers. In fact, by
examining the first process in an input context,
e.g. $x?(z).\lift{w}{y!(z)}$, we see that the process under the lift
operator may be shaped by prefixed inputs binding a name inside it. In
this sense, the lift operator will be seen as a way to dynamically
construct processes before reifying them as names.

Finally equipped with these standard features we can present the
dynamics of the calculus.

\subsubsection{Operational semantics} 

Finally, we introduce the computational dynamics. What marks these
algebras as distinct from other more traditionally studied algebraic
structures, e.g. vector spaces or polynomial rings, is the manner in
which dynamics is captured. In traditional structures, dynamics is typically
expressed through morphisms between such structures, as in linear maps
between vector spaces or morphisms between rings. In algebras
associated with the semantics of computation, the dynamics is
expressed as part of the algebraic structure itself, through a
reduction reduction relation typically denoted by $\red$. Below, we
give a recursive presentation of this relation for the calculus used
in the encoding.

$\red \subseteq \pi \times \pi$
$\red : \pi \to \mathcal{P}(\pi)$

\begin{mathpar}
  \inferrule* [lab=Comm] { \textsf{match}( x_{src}, x_{trgt} ) } { x_{trgt}?(y)P \; | \; x_{src}!\langle {Q} \rangle \red P\{\quotep{Q}/y}\} }
  \and \\
  \inferrule* [lab=Par] {{P} \red {P}'} {{{P} | {Q}} \red {{P}' | {Q}}}
  \and
  \inferrule* [lab=Equiv]{{{P} \scong {P}'} \andalso {{P}' \red {Q}'} \andalso {{Q}' \scong {Q}}}{{P} \red {Q}}
\end{mathpar}

\begin{eqnarray*}
  match_{\equiv} (\quotep{P},\quotep{Q}) & := & P \equiv Q \\
  match_{\dagger}(\quotep{P},\quotep{Q}) & := & \forall R. P|Q \red^{*} R => R \red^{*} 0 \\
  match_{K}(\quotep{P},\quotep{Q}) & := & K \mbox{ for some context } K
\end{eqnarray*}

$u?(x)P | u!\langle Q \rangle \red P\{\quotep{Q}/x\}$

%We write $\wred$ for $\red^*$, and $P\red$ if $\exists Q $ such that $ P \red Q$.
We write $P\red$ if $\exists Q $ such that $ P \red Q$ and $P\not\red$, otherwise.

\section{Replication}

As mentioned before, it is known that replication (and hence
recursion) can be implemented in a higher-order process algebra
\cite{SangiorgiWalker}. As our first example of calculation with the
machinery thus far presented we give the construction explicitly in
the {\rhoc}.

\begin{eqnarray}
	D_{x} & := & \prefix{x}{y}{(\binpar{\outputp{x}{y}}{@{y}})} \nonumber\\
	\bangp_{x}{P} & := & \binpar{{x}!\langle{\binpar{D_{x}}{P}}\rangle}{D_{x}} \nonumber
\end{eqnarray}

\begin{eqnarray}
	\bangp_{x}{P} & & \nonumber\\
	=
	& {x}!\langle{(\prefix{x}{y}{(\outputp{x}{y} | @{y})) | P}}\rangle 
	      | \prefix{x}{y}{(\outputp{x}{y} | @{y})} & \nonumber\\
	\red
	& (\outputp{x}{y} | @{y})\substn{\quotep{(\prefix{x}{y}{(@{y} | \outputp{x}{y})) | P}}}{y} & \nonumber\\
	=
	& \outputp{x}{\quotep{(\prefix{x}{y}{(\outputp{x}{y} | @{y})) | P}}}
	  | {(\prefix{x}{y}{(\outputp{x}{y} | @{y})) | P}} & \nonumber\\
	\red
	& \ldots & \nonumber\\
	\red^*
	& P | P | \ldots & \nonumber
\end{eqnarray}

Of course, this encoding, as an implementation, runs away, unfolding
$\bangp{P}$ eagerly. A lazier and more implementable replication
operator, restricted to input-guarded processes, may be obtained as follows.

\begin{eqnarray}
\bangp{\prefix{u}{v}{P}} 
	:= 
	\binpar{\lift{x}{\prefix{u}{v}{(\binpar{D(x)}{P})}}}{D(x)} \nonumber
\end{eqnarray}

\begin{remark}
  Note that the lazier definition still does not deal with summation
  or mixed summation (i.e. sums over input and output). The reader is
  invited to construct definitions of replication that deal with these
  features. 

  Further, the definitions are parameterized in a name, $x$. Can you,
  gentle reader, make a definition that eliminates this parameter and
  guarantees no accidental interaction between the replication
  machinery and the process being replicated -- i.e. no accidental
  sharing of names used by the process to get its work done and the
  name(s) used by the replication to effect copying. This latter
  revision of the definition of replication is crucial to obtaining
  the expected identity $!!P \sim !P$.
\end{remark}

\begin{remark}\label{rem:paradoxical_combinator}
  The reader familiar with the lambda calculus will have noticed the
  similarity between $D$ and the paradoxical combinator.

  [Ed. note: the existence of this seems to suggest we have to be more
  restrictive on the set of processes and names we admit if we are to
  support no-cloning.]
\end{remark}

\subsubsection{Bisimulation}

The computational dynamics gives rise to another kind of equivalence,
the equivalence of computational behavior. As previously mentioned
this is typically captured \emph{via} some form of bisimulation.

% The notion we use in this paper is weak barbed bisimulation
% \cite{milner91polyadicpi}.

The notion we use in this paper is derived from weak barbed
bisimulation \cite{milner91polyadicpi}. 

\begin{definition}
An \emph{observation relation}, $\downarrow_{\mathcal N}$, over a set
of names, $\mathcal N$, is the smallest relation satisfying the rules
below.

\infrule[Out-barb]{y \in {\mathcal N}, \; x \nameeq y}
		  {\outputp{x}{v} \downarrow_{\mathcal N} x}
\infrule[Par-barb]{\mbox{$P\downarrow_{\mathcal N} x$ or $Q\downarrow_{\mathcal N} x$}}
		  {\binpar{P}{Q} \downarrow_{\mathcal N} x}

We write $P \Downarrow_{\mathcal N} x$ if there is $Q$ such that 
$P \wred Q$ and $Q \downarrow_{\mathcal N} x$.
\end{definition}

\begin{definition}
%\label{def.bbisim}
An  ${\mathcal N}$-\emph{barbed bisimulation} over a set of names, ${\mathcal N}$, is a symmetric binary relation 
${\mathcal S}_{\mathcal N}$ between agents such that $P\rel{S}_{\mathcal N}Q$ implies:
\begin{enumerate}
\item If $P \red P'$ then $Q \wred Q'$ and $P'\rel{S}_{\mathcal N} Q'$.
\item If $P\downarrow_{\mathcal N} x$, then $Q\Downarrow_{\mathcal N} x$.
\end{enumerate}
$P$ is ${\mathcal N}$-barbed bisimilar to $Q$, written
$P \wbbisim_{\mathcal N} Q$, if $P \rel{S}_{\mathcal N} Q$ for some ${\mathcal N}$-barbed bisimulation ${\mathcal S}_{\mathcal N}$.
\end{definition}

$\mathcal{R} \subseteq \pi \times \pi$

$P \mathcal{R} Q => \forall P'. P \red P' \Rightarrow \exists Q'. Q \red Q', P' \mathcal{R} Q'$

$P \vdash x \Rightarrow Q \vdash x$

\begin{mathpar}
  \inferrule*[lab=Out-barb]{x \nameeq y}{{y}!\langle{Q}\rangle \vdash x}
  \and
  \inferrule*[lab=Par-barb]{\mbox{$P\vdash x$ or $Q\vdash x$}}{\binpar{P}{Q} \vdash x}
\end{mathpar}

\subsubsection{Contexts}

One of the principle advantages of computational calculi like the
$\pi$-calculus is a well-defined notion of context,
contextual-equivalence and a correlation between
contextual-equivalence and notions of bisimulation. The notion of
context allows the decomposition of a process into (sub-)process and
its syntactic environment, its context. Thus, a context may be
thought of as a process with a ``hole'' (written $\Box$) in it. The
application of a context $M$ to a process $P$, written $M[P]$, is
tantamount to filling the hole in $M$ with $P$. In this paper we do
not need the full weight of this theory, but do make use of the notion
of context in the proof the main theorem. 

\begin{mathpar}
  \inferrule* [lab=summation] {} {{M_{M},M_{N}} \bc \Box \;|\; x.M_{A} \;|\; M_{M}+M_{N}}
  \and
  \inferrule* [lab=agent] {} {{M_{A}} \bc (\vec{x})M_{P} \;| \; \clift{P_0,\ldots,M_{P},\ldots,P_N}}
  \and \\
  \inferrule* [lab=process] {} {{M_{P}} \bc M_{N} \;| \;P|M_{P} }
\end{mathpar} 

\begin{mathpar}
  \inferrule* [lab=sychronization] {} {M_{N} \bc \Box \;|\; x?M_{F} \;|\; x!M_{C}}
  \and
  \inferrule* [lab=abstraction] {} {{M_{F}} \bc (x)M_{P} }
  \and
  \inferrule* [lab=concretion] {} {{M_{C}} \bc \langle M_{P} \rangle }
  \and \\
  \inferrule* [lab=process] {} {{M_{P}} \bc M_{N} \;| \;P|M_{P} }
\end{mathpar}

\begin{definition}[contextual application] Given a context $M$, and
  process $P$, we define the \emph{contextual application}, $M[P] :=
  M\{P/\Box\}$. That is, the contextual application of M to P is the
  substitution of $P$ for $\Box$ in $M$.
\end{definition}

$\meaningof{-} : L \to \mathcal{P}(\pi)$

\begin{mathpar}
  \inferrule* [lab=collection] {} {\meaningof{true} = \pi, \and \meaningof{~E} = \pi \setminus \meaningof{E}, \and \meaningof{E_{1} \& E_{2}} = \meaningof{E_{1}} \cap \meaningof{E_{2}}}
\end{mathpar}

\begin{mathpar}
  \inferrule* [lab=structure] {} {\meaningof{0} = \{ P \in \pi | P \equiv 0 \}, \and \\ \meaningof{E_1 | E_2} = \{ P \in \pi | P \equiv P_{1} | P_{2}, P_{1} \in \meaningof{E_{1}}, P_{2} \in \meaningof{E_2}\} }
\end{mathpar}

\begin{mathpar}
 \inferrule* [lab=behavior] {} {\meaningof{\langle a?b \rangle E} = \{ P \in \pi | P \equiv Q | u?(y)P', \\ \and \\\\ \and \\ \;\;\; u \in \meaningof{a}, \forall z.P'\{z/y\} \in \meaningof{E\{z/b\}}\}, \and \\ \meaningof{a!E} = \{ P \in \pi | P \equiv Q | x!\langle P' \rangle, x \in \meaningof{a} P' \in \meaningof{E}\} }
\end{mathpar}

\begin{mathpar}
 \inferrule* [lab=nominal] {} {\meaningof{\quotep{E}} = \{ \quotep{P} \in \quotep{\pi} | P \in \meaningof{E} \}, \and \meaningof{\quotep{P}} = \{ \quotep{Q} \in \quotep{\pi} | P \equiv Q \} \and \\ \meaningof{@\quotep{E}} = \{ P \in \pi | P \equiv @x, x \in \meaningof{E} \}}
\end{mathpar}

\begin{eqnarray*}
  \\
  \meaningof{-} : TS \to ST
\end{eqnarray*}

\begin{eqnarray*}
  \\
  L : TS \to ST
\end{eqnarray*}

\begin{eqnarray*}
  \\
  P \models E \iff P \in \meaningof{E}
\end{eqnarray*}

\begin{eqnarray*}
  P \approx_{L} Q \iff \forall E \in L. P \models E \iff Q \models E
\end{eqnarray*}

\begin{eqnarray*}
  P \approx_{K} Q
\end{eqnarray*}

\begin{eqnarray*}
  P \approx Q
\end{eqnarray*}

$\approx_{K} = \approx = \approx_{L}$

\subsubsection{Contextual duality}

Note that contexts extend the quotation operation to a family of
operations from processes to names. Given a context, $M$, we can
define a \emph{nominal context}, $\quotep{M}$ by $\quotep{M}[P] :=
\quotep{M[P]}$. To foreshadow what is to come we observe that these
operations enjoy a duality with processes very much like the duality
between vectors and maps from vectors to scalars.

Further, because the calculus is essentially higher-order, we have a
correspondence between contexts and processes. More specifically,
given a name $x$ and a context $M$ we can construct $M^{*}_{x}$ such
that 

\begin{mathpar}
  M^{*}_{x} | \lift{x}{P} \red M[P]
\end{mathpar}

namely,

\begin{mathpar}
  M^{*}_{x} := x?(u).M[\dropn{u}]
\end{mathpar}

The dependence of $M^{*}_{x}$ on a name makes it an abstraction, 

\begin{mathpar}
  M^{*} := (x)x?(u).M[\dropn{u}]
\end{mathpar}

\subsection{Additional notation}

It will sometimes be convenient to denote the process a name
quotes. We already have the notation $x = \quotep{P}$, but it will be
convenient to introduce an alternate notation, $\procn{x}$, when we
want to emphasize the connection to the use of the name. Note that, by
virtue of name equivalence, $\quotep{\procn{x}} \nameeq x$; so, the
notation is consistent with previous definitions.

Further, because names have structure it is possible to effect
substitutions on the basis of that structure. This means we need to
upgrade our notation for substitutions, which we accomplish by
adapting comprehension notation. Thus,

\begin{mathpar}
  P\{ y / x : x \in S \}
\end{mathpar}

is interpreted to mean the process derived from P by replacing (in a
capture-avoiding manner) each occurrence of $x$ in $S$ by $y$. For example,

\begin{mathpar}
  P\{ \quotep{\procn{x}|\procn{x}} / x : x \in \freenames{P} \}
\end{mathpar}

will replace each (occurrence) of a free name $x$ in $P$ by
$\quotep{\procn{x}|\procn{x}}$.

Also, we will avail ourselves of the notation $x^{L}$ and $x^{R}$ to
denote injections of a name into disjoint copies of the name
space. There are numerous ways to accomplish this. One example can be
found in \cite{MeredithR05}. This notation overloads to vectors of
names: $\vec{x}^{\pi} := (x_{i}^{\pi} \; : \; 0 \leq i < |\vec{x}| )$ where $\pi \in \{L,R\}$.

We also use $P^{\Box} := P|\Box$.

In \cite{MeredithR05} an interpretation of the new operator is
given. It turns out that there are several possible interpretations
all enjoying the requisite algebraic properties of the operator (see
\cite{milner91polyadicpi}). We will therefore make liberal use of
$(\nu\; \vec{x})P$.

% subsection the_syntax_and_semantics_of_the_notation_system (end)   

\input{qm2pi.qmops} 

\input{qm2pi.sterngerlach} 

\input{qm2pi.metric} 

% section concurrent_process_calculi (end)

%\input{qm2pi.proofsketch}

% section proof sketch (end)

%\input{qm2pi.slviaknots} 

% section spatial logic via knots (end)

\input{qm2pi.conclusion}

% section conclusion (end)

%\input{qm2pi.dtcodes} 

% section wiring algorithm (end)

\input{qm2pi.ack} 

% section acknowledgments (end)

\newpage


\bibliographystyle{plain}   
\bibliography{../../biblios/main.bib}

\input{qm2pi.rhodetails}

\end{document}



\end{document}



% section front matter (end)

\section{Introduction}\label{sec:introduction} % (fold)
In this draft of the material i am going to have to dispense with the
usual writing conventions adopted in papers on these topics. i'm going
to have adopt whatever tone i need at the time i'm writing up the
calculations. Sometimes this may be very conversational; others it may
be the barest mathematical grunts; others still it may be that i have
lifted text from one of my other papers because the exposition of some
point was better said there. i hope that my readers are not unduly put
out by this decision. i'm not doing this to flout convention or be
rebellious. i find these calculations very technically challenging. To
keep everything going technically, something has to give; i have to
let go of some cognitive burden. So, the academic writing style --
with all of its trade-offs in terms of facilitating technical
communication -- is what i'm letting go of. Perhaps subsequent drafts
can be tightened and polished, but for now, i'm going to speak as if
we were sitting together in a coffee shop with a laptop, wifi and a
pad of paper and a pencil.

So, here's what i have to say. We -- you and i, comfortably ensconced
in our coffee shop and well-equipped with our tools -- can realize and
carry out the calculations of quantum mechanics over a very different
formal theory of dynamics, a formal theory of dynamics that
corresponds to a theory of concurrent computation with
\emph{reflection}. It has the advantage that the underlying theory is
already `quantized', but supports analogues all of the continuuous
operations. Strikingly, this underlying theory has recently been
connected with a notion of metric that we can show, by calculating
together, coincides with the metric induced by the inner product.

There are a lot of reasons why you might be interested in seeing
calculations of this form. Here's why i'm interested. For the past
several centuries there has been no competitor to the ``Newtonian''
account of dynamics. As a result the predominant share of accounts of
dynamical systems and situations have had to be formulated in terms of
the Newtonian machinery. i view this as an intellectually dangerous
position to occupy. Everything, despite it's intrinsic shape, turns
into a nail to be hit with this hammer. Recently, however, the theory
of computation has matured to the point where we have candidates for
theories of dynamics that offer very different perspective on
reasoning about dynamical systems and situations. Testing these
candidates against very successful accounts of dynamical situations,
like quantum mechanics, is going to give us some sense of how mature
they are and some measure of the quality of these accounts of
dynamics.

\subsection{Summary of contributions and outline of paper}

So, we're going to develop an interpretation of the operations of
quantum mechanics normally interpreted by Hilbert spaces and
operators. We're going to do this over a theory of computation. Note
that this is very different than the usual quantum computation program
which develops notions of computation over quantum mechanics. Rather,
we are developing a story that aligns with Wheeler's slogan: It from
Bit. To do this we will first provide an account of the theory of
computation at play here. Then we will dive into a calculation-driven
interpretation of the operations of quantum mechanics.

The reason we take this approach is that -- until very recently --
there hasn't been an axiomatic account of quantum mechanics. As a
result there has been no sharp delineation of the mathematical theory
supporting interpretation of the physical theory and the physical
theory, itself. So, ambient features of the maths are free to be
exploited (or supressed) without a real accounting of their physical
relevance. There is no sharp statement ``here's the physical theory''
qua \emph{theory} and ``here's the mathematical interpretation''
enabling a judgment of how faithful the interpretation is -- apart
from experimental observation. When there is an axiomatic account we
can judge how well a given mathematical formalism supports an
interpretation of the axioms, independent of
experimentation. Likewise, we can judge how well we have captured our
physical evidence and experience with our axiomatics, independent of
any specific mathematical implementation, with accidental detail that
may or may not have physical significance. 

In lieu of a fully fleshed out and vetted axiomatic account of quantum
mechanics, interpreting the operational notions in service of modeling
physical systems will have to suffice. In other words, we are not in
the business of providing a model of Hilbert spaces and operators. We
are in the business of providing a model of quantum mechanics because
we are motivated by testing our notions of dynamics against physical
theory; and, the predictive calculations of the physical theory must
serve as the best formulation -- shy of a fully fleshed out axiomatic
account -- of the physical theory itself (as they have for scientific
theories since time immemorial). Put another way, despite a
whole-hearted commitment to an It-from-Bit ontology, we are firmly
aligned with the shut-up-and-calculate camp as the best way to obtain
results either from the physical perspective or as a quality assurance
measure of our fledgling theory of dynamics.

In detail, we present a reflective process calculus. Then we develop
intuitive correspondences between the notions available in this
calculus and the usual physical notions supporting quantum mechanical
calculations. Thus, 

\begin{table}[htp]
  \center{
    \fbox{
      \begin{tabular}{c|c}
        quantum mechanics & process calculus \\
        \hline
        scalar & name \\
        state vector & process \\
        dual & contextual duals \\
        matrix & formal sums of process-context-dual pairs \\
        orthogonality & process annihilation \\
        inner product & execution-formula + quoting
      \end{tabular}
    }
  }
  \caption{QM - process calculi correspondences}
\end{table}

Then we tighten up these intuitions to operational definitions. We
employ the Dirac notation as the best proxy we can find for an
abstract syntax of the quantum mechanical notions. The definitions we
develop put us in contact with equational constraints coming from the
theory that we demonstrate the definitions and calculations satisfy.

This puts us in a position to shut up and calculate for the
Stern-Gerlach experimental set up, showing how these predictive
calculations become calculations on processes in our theory of a
reflective process calculus.

Penultimately, we demonstrate that the notion of metric coming from
the inner product coincides with the notion of metric available from
the theory of bisimulation. This demonstration gives us the right to
think of space as arising from behavior. Finally, we consider where we
might go from the new vantage point we have obtained.

% section introduction (end) 
 
% section introduction (end)

% \documentclass[12pt]{llncs}
%\documentclass{jktr}

\usepackage[pdftex]{hyperref}                   
\usepackage {listings}
\usepackage {mathpartir}
\usepackage{bcprules}
%\usepackage{listings}
                       
\usepackage{graphicx} 
%\usepackage[margins=2.5cm,nohead,nofoot]{geometry}
%\usepackage{geometry}
\usepackage{amsfonts}
\usepackage{amstext}
\usepackage{latexsym}
\usepackage{amssymb}
\usepackage{color}


%\include{myPreamble}
\documentclass[12pt]{llncs}
%\documentclass{jktr}

\usepackage[pdftex]{hyperref}                   
\usepackage {listings}
\usepackage {mathpartir}
\usepackage{bcprules}
%\usepackage{listings}
                       
\usepackage{graphicx} 
%\usepackage[margins=2.5cm,nohead,nofoot]{geometry}
%\usepackage{geometry}
\usepackage{amsfonts}
\usepackage{amstext}
\usepackage{latexsym}
\usepackage{amssymb}
\usepackage{color}


%\include{myPreamble}
\include{qm2pi.local} 

%\ifpdf
%\usepackage[pdftex]{graphicx}
%\else
%\usepackage{graphicx}
%\fi

 % \ifpdf
%  \usepackage{pdfsync}
%  \if


%\title{Brief Article}
%\author{David F. Snyder}
%\author{L.G. Meredith}

%\address{Dept. of Math., Texas State University--San Marcos, San Marcos, TX 78666}
       
\pagestyle{empty}


\begin{document}

\lstset{language=[Objective]Caml,frame=shadowbox}

\input{qm2pi.front}

% section front matter (end)

\input{qm2pi.intro} 
 
% section introduction (end)

% \input{qm2pi.knotations} 

% section notation (end)

\input{qm2pi.process.calculi} 

% section concurrent_process_calculi_and_spatial_logics_ (end)
    
%\input{qm2pi.knots2pi} 

%\input{qm2pi.trefoil} 

%\input{qm2pi.mainthm} 

% subsection basic_interpretation (end)

%\input{qm2pi.rho.presentation} 
\subsection{The syntax and semantics of the notation system}\label{sub:the_syntax_and_semantics_of_the_notation_system} % (fold)

We now summarize a technical presentation of the calculus that
embodies our theory of dynamics. The typical presentation of such a
calculus follows the style of giving generators and relations on
them. The grammar, below, describing term constructors, freely
generates the set of processes, $\Proc$. This set is then quotiented
by a relation known as structural congruence and it is over this set
that the notion of dynamics is expressed. This presentation is
essentially that of \cite{MeredithR05} with the addition of
polyadicity and summation. For readability we have relegated some of
the technical subtleties to an appendix.

\subsubsection{Process grammar}\label{subsub:process_grammar}

\begin{mathpar}
  \inferrule* [lab=synchronization] {} {{M} \bc \pzero \;|\; x?F \;|\; x!C }
  \and
  \inferrule* [lab=abstraction] {} {{F} \bc (x)P}
  \and
  \inferrule* [lab=concretion] {} {{C} \bc \langle Q \rangle}
  \and
  \inferrule* [lab=process] {} {{P,Q} \bc M \;| \;P|Q \;|\; @{x}}
  \and
  \inferrule* [lab=name] {} {{x} \bc \quotep{P}}
\end{mathpar} 

Note that $\vec{x}$ (resp. $\vec{P}$) denotes a vector of names
(resp. processes) of length $|\vec{x}|$ (resp. $|\vec{P}|$). We adopt
the following useful abbreviations.

\begin{mathpar}
   x?(\vec{y}).P := x.(\vec{y})P \and  x\clift{\vec{P}} := x.\clift{\vec{P}}
   \and x!(y) := \lift{x}{\dropn{y}}
   \and \Pi_{i=0}^{n-1}P_i := P_0 | \ldots | P_{n-1}
\end{mathpar}

\subsubsection{Structural congruence}

\paragraph{Free and bound names and alpha-equivalence.} At the
core of structural equivalence is alpha-equivalence which identifies
process that are the same up to a change of variable. Formally, we
recognize the distinction between free and bound names. The free names
of a process, $\freenames{P}$, may be calculated recursively as
follows:

\begin{mathpar}
\freenames{\pzero} := \emptyset
  \and \\
  \freenames{x?(y).P} := \{ x \} \cup (\freenames{P} \setminus \{ y \})
  \and 
  \freenames{x!\langle P \rangle} := \{ x \} \cup \{ P \} 
  \and \\
  \freenames{P|Q} := \freenames{P} \cup \freenames{Q}
  \and \\
  \freenames{@{x}} := \{ x \}
\end{mathpar}

$\pi$
$\quotep{\pi}$

$\freenames{-} : \pi \to \mathcal{P}(\quotep{\pi})$

\begin{eqnarray*}
  \freenames{\pzero} & := & \emptyset \\
  \freenames{x?(y).P} & := & \{ x \} \cup (\freenames{P} \setminus \{ y \}) \\
  \freenames{x!\langle P \rangle} & := & \{ x \} \cup \{ P \} \\
  \freenames{P|Q} & := & \freenames{P} \cup \freenames{Q} \\
  \freenames{\dropn{x}} & := & \{ x \}
\end{eqnarray*}

The bound names of a process, $\boundnames{P}$, are those names occurring in $P$
that are not free. For example, in $x?(y).0$, the name $x$ is free, while $y$ is bound.

\begin{mathpar}
  \inferrule* [lab=monoidal-laws] {} { P|Q \equiv Q|P \and P|0 \equiv P \and P|(Q|R) \equiv (P|Q)|R }
\end{mathpar}

\begin{mathpar}
  \inferrule* [lab=alpha-equivalence] {} { (x)P \equiv (y)P\{y/x\} \and y \not\in \freenames{P} }
\end{mathpar}

\begin{definition}
Then two processes, $P,Q$, are alpha-equivalent if $P = Q\{\vec{y}/\vec{x}\}$ for
some $\vec{x} \in \boundnames{Q},\vec{y} \in \boundnames{P}$, where $Q\{\vec{y}/\vec{x}\}$
denotes the capture-avoiding substitution of $\vec{y}$ for $\vec{x}$ in $Q$.
\end{definition}

\begin{definition}
  The {\em structural congruence} \cite{SangiorgiWalker} , $\equiv$,
  between processes is the least congruence containing
  alpha-equivalence, satisfying the abelian monoid laws
  (associativity, commutativity and $\pzero$ as identity) for parallel
  composition $|$ and for summation $+$.
\end{definition}

\subsection{Name equivalence}

We take name equivalence, written $\nameeq$, to be the smallest
equivalence relation generated by the following rules.

\begin{mathpar}
\inferrule*[lab=Quote-drop]
{ }
{ \quotep{@{x}} \nameeq x }

\inferrule*[lab=Struct-equiv]
{ P \scong Q }
{ \quotep{P} \nameeq \quotep{Q} }
\end{mathpar}

The astute reader will have noticed that the mutual recursion of names
and processes imposes a mutual recursion on alpha-equivalence and
structural equivalence via name-equivalence. Fortunately, all of this
works out pleasantly and we may calculate in the natural way, free of
concern. The reader interested in the details is referred to the
appendix \ref{appendix:rho_details}.

\subsection{Substitution}

We use $\Proc$ for the set of processes, $\QProc$ for the set of
names, and $\id{\{}\vec{y} / \vec{x} \id{\}}$ to denote partial maps,
$s : \QProc \rightarrow \QProc$. A map, $s$ lifts, uniquely, to a map
on process terms, $\widehat{s} : \Proc \rightarrow \Proc$ by the
following equations.

\begin{mathpar}
  (0) \psubstp{Q}{P} := 0 \\
  (R \juxtap S) \psubstp{Q}{P}
  :=    
  (R)\psubstp{Q}{P} \juxtap (S) \psubstp{Q}{P} \\
  (x?(y).R) \psubstp{Q}{P}    
  :=    
  (x)\substp{Q}{P} (z)\concat( (R \psubstn{z}{y}) \psubstp{Q}{P} ) \\
  (\lift{x}{R}) \psubstp{Q}{P}  
  :=
  \lift{(x)\substp{Q}{P}}{ R \psubstp{Q}{P} } \\
%   (\dropn{x})  \psubstp{Q}{P}       
%   := 
%   \left\{ 
%     \begin{array}{ccc} 
%       \dropn{\quotep{Q}} & & x \nameeq \quotep{P} \\
%       \dropn{x} & & otherwise \\
%     \end{array}
%   \right. 
  (\dropn{x})  \psubstp{Q}{P}       
  := 
  \left\{ 
    \begin{array}{ccc} 
      Q & & x \nameeq \quotep{P} \\
      \dropn{x} & & otherwise \\
    \end{array}
  \right.
\end{mathpar}
 

where

\begin{eqnarray}
  (x)\id{\{} \lpquote Q \rpquote / \lpquote P \rpquote \id{\}}            = 
  \left\{ 
    \begin{array}{ccc}
      \lpquote Q \rpquote & & x \nameeq \lpquote P \rpquote \\
      x & & otherwise \\
    \end{array}
  \right. \nonumber
\end{eqnarray}

and $z$ is chosen distinct from $\quotep{P}$, $\quotep{Q}$, the free
names in $Q$, and all the names in $R$. Our $\alpha$-equivalence will
be built in the standard way from this substitution.

\begin{remark}\label{rem:no_self_referential_names}
  One consequence of these definitions is that $\forall P. \quotep{P}
  \not\in \freenames{P}$.
\end{remark}

\subsection{ Dynamic quote: an example }

Anticipating something of what's to come, consider applying the
substitution, $\widehat{\id{\{}u / z \id{\}}}$, to the following pair
of processes, $\lift{w}{y!(z)}$ and $w[ \lpquote y!(z) \rpquote ]$.

\begin{eqnarray}
	\lift{w}{y!(z)}\widehat{\id{\{}u / z \id{\}}}
		& = &
		\lift{w}{y!(u)} \nonumber\\
	w[ \lpquote y!(z) \rpquote ] \widehat{ \id{\{}u / z \id{\}} }
		& = &
		w[ \lpquote y!(z) \rpquote ] \nonumber
\end{eqnarray}

Because the body of the process between quotes is impervious to
substitution, we get radically different answers. In fact, by
examining the first process in an input context,
e.g. $x?(z).\lift{w}{y!(z)}$, we see that the process under the lift
operator may be shaped by prefixed inputs binding a name inside it. In
this sense, the lift operator will be seen as a way to dynamically
construct processes before reifying them as names.

Finally equipped with these standard features we can present the
dynamics of the calculus.

\subsubsection{Operational semantics} 

Finally, we introduce the computational dynamics. What marks these
algebras as distinct from other more traditionally studied algebraic
structures, e.g. vector spaces or polynomial rings, is the manner in
which dynamics is captured. In traditional structures, dynamics is typically
expressed through morphisms between such structures, as in linear maps
between vector spaces or morphisms between rings. In algebras
associated with the semantics of computation, the dynamics is
expressed as part of the algebraic structure itself, through a
reduction reduction relation typically denoted by $\red$. Below, we
give a recursive presentation of this relation for the calculus used
in the encoding.

$\red \subseteq \pi \times \pi$
$\red : \pi \to \mathcal{P}(\pi)$

\begin{mathpar}
  \inferrule* [lab=Comm] { \textsf{match}( x_{src}, x_{trgt} ) } { x_{trgt}?(y)P \; | \; x_{src}!\langle {Q} \rangle \red P\{\quotep{Q}/y}\} }
  \and \\
  \inferrule* [lab=Par] {{P} \red {P}'} {{{P} | {Q}} \red {{P}' | {Q}}}
  \and
  \inferrule* [lab=Equiv]{{{P} \scong {P}'} \andalso {{P}' \red {Q}'} \andalso {{Q}' \scong {Q}}}{{P} \red {Q}}
\end{mathpar}

\begin{eqnarray*}
  match_{\equiv} (\quotep{P},\quotep{Q}) & := & P \equiv Q \\
  match_{\dagger}(\quotep{P},\quotep{Q}) & := & \forall R. P|Q \red^{*} R => R \red^{*} 0 \\
  match_{K}(\quotep{P},\quotep{Q}) & := & K \mbox{ for some context } K
\end{eqnarray*}

$u?(x)P | u!\langle Q \rangle \red P\{\quotep{Q}/x\}$

%We write $\wred$ for $\red^*$, and $P\red$ if $\exists Q $ such that $ P \red Q$.
We write $P\red$ if $\exists Q $ such that $ P \red Q$ and $P\not\red$, otherwise.

\section{Replication}

As mentioned before, it is known that replication (and hence
recursion) can be implemented in a higher-order process algebra
\cite{SangiorgiWalker}. As our first example of calculation with the
machinery thus far presented we give the construction explicitly in
the {\rhoc}.

\begin{eqnarray}
	D_{x} & := & \prefix{x}{y}{(\binpar{\outputp{x}{y}}{@{y}})} \nonumber\\
	\bangp_{x}{P} & := & \binpar{{x}!\langle{\binpar{D_{x}}{P}}\rangle}{D_{x}} \nonumber
\end{eqnarray}

\begin{eqnarray}
	\bangp_{x}{P} & & \nonumber\\
	=
	& {x}!\langle{(\prefix{x}{y}{(\outputp{x}{y} | @{y})) | P}}\rangle 
	      | \prefix{x}{y}{(\outputp{x}{y} | @{y})} & \nonumber\\
	\red
	& (\outputp{x}{y} | @{y})\substn{\quotep{(\prefix{x}{y}{(@{y} | \outputp{x}{y})) | P}}}{y} & \nonumber\\
	=
	& \outputp{x}{\quotep{(\prefix{x}{y}{(\outputp{x}{y} | @{y})) | P}}}
	  | {(\prefix{x}{y}{(\outputp{x}{y} | @{y})) | P}} & \nonumber\\
	\red
	& \ldots & \nonumber\\
	\red^*
	& P | P | \ldots & \nonumber
\end{eqnarray}

Of course, this encoding, as an implementation, runs away, unfolding
$\bangp{P}$ eagerly. A lazier and more implementable replication
operator, restricted to input-guarded processes, may be obtained as follows.

\begin{eqnarray}
\bangp{\prefix{u}{v}{P}} 
	:= 
	\binpar{\lift{x}{\prefix{u}{v}{(\binpar{D(x)}{P})}}}{D(x)} \nonumber
\end{eqnarray}

\begin{remark}
  Note that the lazier definition still does not deal with summation
  or mixed summation (i.e. sums over input and output). The reader is
  invited to construct definitions of replication that deal with these
  features. 

  Further, the definitions are parameterized in a name, $x$. Can you,
  gentle reader, make a definition that eliminates this parameter and
  guarantees no accidental interaction between the replication
  machinery and the process being replicated -- i.e. no accidental
  sharing of names used by the process to get its work done and the
  name(s) used by the replication to effect copying. This latter
  revision of the definition of replication is crucial to obtaining
  the expected identity $!!P \sim !P$.
\end{remark}

\begin{remark}\label{rem:paradoxical_combinator}
  The reader familiar with the lambda calculus will have noticed the
  similarity between $D$ and the paradoxical combinator.

  [Ed. note: the existence of this seems to suggest we have to be more
  restrictive on the set of processes and names we admit if we are to
  support no-cloning.]
\end{remark}

\subsubsection{Bisimulation}

The computational dynamics gives rise to another kind of equivalence,
the equivalence of computational behavior. As previously mentioned
this is typically captured \emph{via} some form of bisimulation.

% The notion we use in this paper is weak barbed bisimulation
% \cite{milner91polyadicpi}.

The notion we use in this paper is derived from weak barbed
bisimulation \cite{milner91polyadicpi}. 

\begin{definition}
An \emph{observation relation}, $\downarrow_{\mathcal N}$, over a set
of names, $\mathcal N$, is the smallest relation satisfying the rules
below.

\infrule[Out-barb]{y \in {\mathcal N}, \; x \nameeq y}
		  {\outputp{x}{v} \downarrow_{\mathcal N} x}
\infrule[Par-barb]{\mbox{$P\downarrow_{\mathcal N} x$ or $Q\downarrow_{\mathcal N} x$}}
		  {\binpar{P}{Q} \downarrow_{\mathcal N} x}

We write $P \Downarrow_{\mathcal N} x$ if there is $Q$ such that 
$P \wred Q$ and $Q \downarrow_{\mathcal N} x$.
\end{definition}

\begin{definition}
%\label{def.bbisim}
An  ${\mathcal N}$-\emph{barbed bisimulation} over a set of names, ${\mathcal N}$, is a symmetric binary relation 
${\mathcal S}_{\mathcal N}$ between agents such that $P\rel{S}_{\mathcal N}Q$ implies:
\begin{enumerate}
\item If $P \red P'$ then $Q \wred Q'$ and $P'\rel{S}_{\mathcal N} Q'$.
\item If $P\downarrow_{\mathcal N} x$, then $Q\Downarrow_{\mathcal N} x$.
\end{enumerate}
$P$ is ${\mathcal N}$-barbed bisimilar to $Q$, written
$P \wbbisim_{\mathcal N} Q$, if $P \rel{S}_{\mathcal N} Q$ for some ${\mathcal N}$-barbed bisimulation ${\mathcal S}_{\mathcal N}$.
\end{definition}

$\mathcal{R} \subseteq \pi \times \pi$

$P \mathcal{R} Q => \forall P'. P \red P' \Rightarrow \exists Q'. Q \red Q', P' \mathcal{R} Q'$

$P \vdash x \Rightarrow Q \vdash x$

\begin{mathpar}
  \inferrule*[lab=Out-barb]{x \nameeq y}{{y}!\langle{Q}\rangle \vdash x}
  \and
  \inferrule*[lab=Par-barb]{\mbox{$P\vdash x$ or $Q\vdash x$}}{\binpar{P}{Q} \vdash x}
\end{mathpar}

\subsubsection{Contexts}

One of the principle advantages of computational calculi like the
$\pi$-calculus is a well-defined notion of context,
contextual-equivalence and a correlation between
contextual-equivalence and notions of bisimulation. The notion of
context allows the decomposition of a process into (sub-)process and
its syntactic environment, its context. Thus, a context may be
thought of as a process with a ``hole'' (written $\Box$) in it. The
application of a context $M$ to a process $P$, written $M[P]$, is
tantamount to filling the hole in $M$ with $P$. In this paper we do
not need the full weight of this theory, but do make use of the notion
of context in the proof the main theorem. 

\begin{mathpar}
  \inferrule* [lab=summation] {} {{M_{M},M_{N}} \bc \Box \;|\; x.M_{A} \;|\; M_{M}+M_{N}}
  \and
  \inferrule* [lab=agent] {} {{M_{A}} \bc (\vec{x})M_{P} \;| \; \clift{P_0,\ldots,M_{P},\ldots,P_N}}
  \and \\
  \inferrule* [lab=process] {} {{M_{P}} \bc M_{N} \;| \;P|M_{P} }
\end{mathpar} 

\begin{mathpar}
  \inferrule* [lab=sychronization] {} {M_{N} \bc \Box \;|\; x?M_{F} \;|\; x!M_{C}}
  \and
  \inferrule* [lab=abstraction] {} {{M_{F}} \bc (x)M_{P} }
  \and
  \inferrule* [lab=concretion] {} {{M_{C}} \bc \langle M_{P} \rangle }
  \and \\
  \inferrule* [lab=process] {} {{M_{P}} \bc M_{N} \;| \;P|M_{P} }
\end{mathpar}

\begin{definition}[contextual application] Given a context $M$, and
  process $P$, we define the \emph{contextual application}, $M[P] :=
  M\{P/\Box\}$. That is, the contextual application of M to P is the
  substitution of $P$ for $\Box$ in $M$.
\end{definition}

$\meaningof{-} : L \to \mathcal{P}(\pi)$

\begin{mathpar}
  \inferrule* [lab=collection] {} {\meaningof{true} = \pi, \and \meaningof{~E} = \pi \setminus \meaningof{E}, \and \meaningof{E_{1} \& E_{2}} = \meaningof{E_{1}} \cap \meaningof{E_{2}}}
\end{mathpar}

\begin{mathpar}
  \inferrule* [lab=structure] {} {\meaningof{0} = \{ P \in \pi | P \equiv 0 \}, \and \\ \meaningof{E_1 | E_2} = \{ P \in \pi | P \equiv P_{1} | P_{2}, P_{1} \in \meaningof{E_{1}}, P_{2} \in \meaningof{E_2}\} }
\end{mathpar}

\begin{mathpar}
 \inferrule* [lab=behavior] {} {\meaningof{\langle a?b \rangle E} = \{ P \in \pi | P \equiv Q | u?(y)P', \\ \and \\\\ \and \\ \;\;\; u \in \meaningof{a}, \forall z.P'\{z/y\} \in \meaningof{E\{z/b\}}\}, \and \\ \meaningof{a!E} = \{ P \in \pi | P \equiv Q | x!\langle P' \rangle, x \in \meaningof{a} P' \in \meaningof{E}\} }
\end{mathpar}

\begin{mathpar}
 \inferrule* [lab=nominal] {} {\meaningof{\quotep{E}} = \{ \quotep{P} \in \quotep{\pi} | P \in \meaningof{E} \}, \and \meaningof{\quotep{P}} = \{ \quotep{Q} \in \quotep{\pi} | P \equiv Q \} \and \\ \meaningof{@\quotep{E}} = \{ P \in \pi | P \equiv @x, x \in \meaningof{E} \}}
\end{mathpar}

\begin{eqnarray*}
  \\
  \meaningof{-} : TS \to ST
\end{eqnarray*}

\begin{eqnarray*}
  \\
  L : TS \to ST
\end{eqnarray*}

\begin{eqnarray*}
  \\
  P \models E \iff P \in \meaningof{E}
\end{eqnarray*}

\begin{eqnarray*}
  P \approx_{L} Q \iff \forall E \in L. P \models E \iff Q \models E
\end{eqnarray*}

\begin{eqnarray*}
  P \approx_{K} Q
\end{eqnarray*}

\begin{eqnarray*}
  P \approx Q
\end{eqnarray*}

$\approx_{K} = \approx = \approx_{L}$

\subsubsection{Contextual duality}

Note that contexts extend the quotation operation to a family of
operations from processes to names. Given a context, $M$, we can
define a \emph{nominal context}, $\quotep{M}$ by $\quotep{M}[P] :=
\quotep{M[P]}$. To foreshadow what is to come we observe that these
operations enjoy a duality with processes very much like the duality
between vectors and maps from vectors to scalars.

Further, because the calculus is essentially higher-order, we have a
correspondence between contexts and processes. More specifically,
given a name $x$ and a context $M$ we can construct $M^{*}_{x}$ such
that 

\begin{mathpar}
  M^{*}_{x} | \lift{x}{P} \red M[P]
\end{mathpar}

namely,

\begin{mathpar}
  M^{*}_{x} := x?(u).M[\dropn{u}]
\end{mathpar}

The dependence of $M^{*}_{x}$ on a name makes it an abstraction, 

\begin{mathpar}
  M^{*} := (x)x?(u).M[\dropn{u}]
\end{mathpar}

\subsection{Additional notation}

It will sometimes be convenient to denote the process a name
quotes. We already have the notation $x = \quotep{P}$, but it will be
convenient to introduce an alternate notation, $\procn{x}$, when we
want to emphasize the connection to the use of the name. Note that, by
virtue of name equivalence, $\quotep{\procn{x}} \nameeq x$; so, the
notation is consistent with previous definitions.

Further, because names have structure it is possible to effect
substitutions on the basis of that structure. This means we need to
upgrade our notation for substitutions, which we accomplish by
adapting comprehension notation. Thus,

\begin{mathpar}
  P\{ y / x : x \in S \}
\end{mathpar}

is interpreted to mean the process derived from P by replacing (in a
capture-avoiding manner) each occurrence of $x$ in $S$ by $y$. For example,

\begin{mathpar}
  P\{ \quotep{\procn{x}|\procn{x}} / x : x \in \freenames{P} \}
\end{mathpar}

will replace each (occurrence) of a free name $x$ in $P$ by
$\quotep{\procn{x}|\procn{x}}$.

Also, we will avail ourselves of the notation $x^{L}$ and $x^{R}$ to
denote injections of a name into disjoint copies of the name
space. There are numerous ways to accomplish this. One example can be
found in \cite{MeredithR05}. This notation overloads to vectors of
names: $\vec{x}^{\pi} := (x_{i}^{\pi} \; : \; 0 \leq i < |\vec{x}| )$ where $\pi \in \{L,R\}$.

We also use $P^{\Box} := P|\Box$.

In \cite{MeredithR05} an interpretation of the new operator is
given. It turns out that there are several possible interpretations
all enjoying the requisite algebraic properties of the operator (see
\cite{milner91polyadicpi}). We will therefore make liberal use of
$(\nu\; \vec{x})P$.

% subsection the_syntax_and_semantics_of_the_notation_system (end)   

\input{qm2pi.qmops} 

\input{qm2pi.sterngerlach} 

\input{qm2pi.metric} 

% section concurrent_process_calculi (end)

%\input{qm2pi.proofsketch}

% section proof sketch (end)

%\input{qm2pi.slviaknots} 

% section spatial logic via knots (end)

\input{qm2pi.conclusion}

% section conclusion (end)

%\input{qm2pi.dtcodes} 

% section wiring algorithm (end)

\input{qm2pi.ack} 

% section acknowledgments (end)

\newpage


\bibliographystyle{plain}   
\bibliography{../../biblios/main.bib}

\input{qm2pi.rhodetails}

\end{document}

 

%\ifpdf
%\usepackage[pdftex]{graphicx}
%\else
%\usepackage{graphicx}
%\fi

 % \ifpdf
%  \usepackage{pdfsync}
%  \if


%\title{Brief Article}
%\author{David F. Snyder}
%\author{L.G. Meredith}

%\address{Dept. of Math., Texas State University--San Marcos, San Marcos, TX 78666}
       
\pagestyle{empty}


\begin{document}

\lstset{language=[Objective]Caml,frame=shadowbox}

\documentclass[12pt]{llncs}
%\documentclass{jktr}

\usepackage[pdftex]{hyperref}                   
\usepackage {listings}
\usepackage {mathpartir}
\usepackage{bcprules}
%\usepackage{listings}
                       
\usepackage{graphicx} 
%\usepackage[margins=2.5cm,nohead,nofoot]{geometry}
%\usepackage{geometry}
\usepackage{amsfonts}
\usepackage{amstext}
\usepackage{latexsym}
\usepackage{amssymb}
\usepackage{color}


%\include{myPreamble}
\include{qm2pi.local} 

%\ifpdf
%\usepackage[pdftex]{graphicx}
%\else
%\usepackage{graphicx}
%\fi

 % \ifpdf
%  \usepackage{pdfsync}
%  \if


%\title{Brief Article}
%\author{David F. Snyder}
%\author{L.G. Meredith}

%\address{Dept. of Math., Texas State University--San Marcos, San Marcos, TX 78666}
       
\pagestyle{empty}


\begin{document}

\lstset{language=[Objective]Caml,frame=shadowbox}

\input{qm2pi.front}

% section front matter (end)

\input{qm2pi.intro} 
 
% section introduction (end)

% \input{qm2pi.knotations} 

% section notation (end)

\input{qm2pi.process.calculi} 

% section concurrent_process_calculi_and_spatial_logics_ (end)
    
%\input{qm2pi.knots2pi} 

%\input{qm2pi.trefoil} 

%\input{qm2pi.mainthm} 

% subsection basic_interpretation (end)

%\input{qm2pi.rho.presentation} 
\subsection{The syntax and semantics of the notation system}\label{sub:the_syntax_and_semantics_of_the_notation_system} % (fold)

We now summarize a technical presentation of the calculus that
embodies our theory of dynamics. The typical presentation of such a
calculus follows the style of giving generators and relations on
them. The grammar, below, describing term constructors, freely
generates the set of processes, $\Proc$. This set is then quotiented
by a relation known as structural congruence and it is over this set
that the notion of dynamics is expressed. This presentation is
essentially that of \cite{MeredithR05} with the addition of
polyadicity and summation. For readability we have relegated some of
the technical subtleties to an appendix.

\subsubsection{Process grammar}\label{subsub:process_grammar}

\begin{mathpar}
  \inferrule* [lab=synchronization] {} {{M} \bc \pzero \;|\; x?F \;|\; x!C }
  \and
  \inferrule* [lab=abstraction] {} {{F} \bc (x)P}
  \and
  \inferrule* [lab=concretion] {} {{C} \bc \langle Q \rangle}
  \and
  \inferrule* [lab=process] {} {{P,Q} \bc M \;| \;P|Q \;|\; @{x}}
  \and
  \inferrule* [lab=name] {} {{x} \bc \quotep{P}}
\end{mathpar} 

Note that $\vec{x}$ (resp. $\vec{P}$) denotes a vector of names
(resp. processes) of length $|\vec{x}|$ (resp. $|\vec{P}|$). We adopt
the following useful abbreviations.

\begin{mathpar}
   x?(\vec{y}).P := x.(\vec{y})P \and  x\clift{\vec{P}} := x.\clift{\vec{P}}
   \and x!(y) := \lift{x}{\dropn{y}}
   \and \Pi_{i=0}^{n-1}P_i := P_0 | \ldots | P_{n-1}
\end{mathpar}

\subsubsection{Structural congruence}

\paragraph{Free and bound names and alpha-equivalence.} At the
core of structural equivalence is alpha-equivalence which identifies
process that are the same up to a change of variable. Formally, we
recognize the distinction between free and bound names. The free names
of a process, $\freenames{P}$, may be calculated recursively as
follows:

\begin{mathpar}
\freenames{\pzero} := \emptyset
  \and \\
  \freenames{x?(y).P} := \{ x \} \cup (\freenames{P} \setminus \{ y \})
  \and 
  \freenames{x!\langle P \rangle} := \{ x \} \cup \{ P \} 
  \and \\
  \freenames{P|Q} := \freenames{P} \cup \freenames{Q}
  \and \\
  \freenames{@{x}} := \{ x \}
\end{mathpar}

$\pi$
$\quotep{\pi}$

$\freenames{-} : \pi \to \mathcal{P}(\quotep{\pi})$

\begin{eqnarray*}
  \freenames{\pzero} & := & \emptyset \\
  \freenames{x?(y).P} & := & \{ x \} \cup (\freenames{P} \setminus \{ y \}) \\
  \freenames{x!\langle P \rangle} & := & \{ x \} \cup \{ P \} \\
  \freenames{P|Q} & := & \freenames{P} \cup \freenames{Q} \\
  \freenames{\dropn{x}} & := & \{ x \}
\end{eqnarray*}

The bound names of a process, $\boundnames{P}$, are those names occurring in $P$
that are not free. For example, in $x?(y).0$, the name $x$ is free, while $y$ is bound.

\begin{mathpar}
  \inferrule* [lab=monoidal-laws] {} { P|Q \equiv Q|P \and P|0 \equiv P \and P|(Q|R) \equiv (P|Q)|R }
\end{mathpar}

\begin{mathpar}
  \inferrule* [lab=alpha-equivalence] {} { (x)P \equiv (y)P\{y/x\} \and y \not\in \freenames{P} }
\end{mathpar}

\begin{definition}
Then two processes, $P,Q$, are alpha-equivalent if $P = Q\{\vec{y}/\vec{x}\}$ for
some $\vec{x} \in \boundnames{Q},\vec{y} \in \boundnames{P}$, where $Q\{\vec{y}/\vec{x}\}$
denotes the capture-avoiding substitution of $\vec{y}$ for $\vec{x}$ in $Q$.
\end{definition}

\begin{definition}
  The {\em structural congruence} \cite{SangiorgiWalker} , $\equiv$,
  between processes is the least congruence containing
  alpha-equivalence, satisfying the abelian monoid laws
  (associativity, commutativity and $\pzero$ as identity) for parallel
  composition $|$ and for summation $+$.
\end{definition}

\subsection{Name equivalence}

We take name equivalence, written $\nameeq$, to be the smallest
equivalence relation generated by the following rules.

\begin{mathpar}
\inferrule*[lab=Quote-drop]
{ }
{ \quotep{@{x}} \nameeq x }

\inferrule*[lab=Struct-equiv]
{ P \scong Q }
{ \quotep{P} \nameeq \quotep{Q} }
\end{mathpar}

The astute reader will have noticed that the mutual recursion of names
and processes imposes a mutual recursion on alpha-equivalence and
structural equivalence via name-equivalence. Fortunately, all of this
works out pleasantly and we may calculate in the natural way, free of
concern. The reader interested in the details is referred to the
appendix \ref{appendix:rho_details}.

\subsection{Substitution}

We use $\Proc$ for the set of processes, $\QProc$ for the set of
names, and $\id{\{}\vec{y} / \vec{x} \id{\}}$ to denote partial maps,
$s : \QProc \rightarrow \QProc$. A map, $s$ lifts, uniquely, to a map
on process terms, $\widehat{s} : \Proc \rightarrow \Proc$ by the
following equations.

\begin{mathpar}
  (0) \psubstp{Q}{P} := 0 \\
  (R \juxtap S) \psubstp{Q}{P}
  :=    
  (R)\psubstp{Q}{P} \juxtap (S) \psubstp{Q}{P} \\
  (x?(y).R) \psubstp{Q}{P}    
  :=    
  (x)\substp{Q}{P} (z)\concat( (R \psubstn{z}{y}) \psubstp{Q}{P} ) \\
  (\lift{x}{R}) \psubstp{Q}{P}  
  :=
  \lift{(x)\substp{Q}{P}}{ R \psubstp{Q}{P} } \\
%   (\dropn{x})  \psubstp{Q}{P}       
%   := 
%   \left\{ 
%     \begin{array}{ccc} 
%       \dropn{\quotep{Q}} & & x \nameeq \quotep{P} \\
%       \dropn{x} & & otherwise \\
%     \end{array}
%   \right. 
  (\dropn{x})  \psubstp{Q}{P}       
  := 
  \left\{ 
    \begin{array}{ccc} 
      Q & & x \nameeq \quotep{P} \\
      \dropn{x} & & otherwise \\
    \end{array}
  \right.
\end{mathpar}
 

where

\begin{eqnarray}
  (x)\id{\{} \lpquote Q \rpquote / \lpquote P \rpquote \id{\}}            = 
  \left\{ 
    \begin{array}{ccc}
      \lpquote Q \rpquote & & x \nameeq \lpquote P \rpquote \\
      x & & otherwise \\
    \end{array}
  \right. \nonumber
\end{eqnarray}

and $z$ is chosen distinct from $\quotep{P}$, $\quotep{Q}$, the free
names in $Q$, and all the names in $R$. Our $\alpha$-equivalence will
be built in the standard way from this substitution.

\begin{remark}\label{rem:no_self_referential_names}
  One consequence of these definitions is that $\forall P. \quotep{P}
  \not\in \freenames{P}$.
\end{remark}

\subsection{ Dynamic quote: an example }

Anticipating something of what's to come, consider applying the
substitution, $\widehat{\id{\{}u / z \id{\}}}$, to the following pair
of processes, $\lift{w}{y!(z)}$ and $w[ \lpquote y!(z) \rpquote ]$.

\begin{eqnarray}
	\lift{w}{y!(z)}\widehat{\id{\{}u / z \id{\}}}
		& = &
		\lift{w}{y!(u)} \nonumber\\
	w[ \lpquote y!(z) \rpquote ] \widehat{ \id{\{}u / z \id{\}} }
		& = &
		w[ \lpquote y!(z) \rpquote ] \nonumber
\end{eqnarray}

Because the body of the process between quotes is impervious to
substitution, we get radically different answers. In fact, by
examining the first process in an input context,
e.g. $x?(z).\lift{w}{y!(z)}$, we see that the process under the lift
operator may be shaped by prefixed inputs binding a name inside it. In
this sense, the lift operator will be seen as a way to dynamically
construct processes before reifying them as names.

Finally equipped with these standard features we can present the
dynamics of the calculus.

\subsubsection{Operational semantics} 

Finally, we introduce the computational dynamics. What marks these
algebras as distinct from other more traditionally studied algebraic
structures, e.g. vector spaces or polynomial rings, is the manner in
which dynamics is captured. In traditional structures, dynamics is typically
expressed through morphisms between such structures, as in linear maps
between vector spaces or morphisms between rings. In algebras
associated with the semantics of computation, the dynamics is
expressed as part of the algebraic structure itself, through a
reduction reduction relation typically denoted by $\red$. Below, we
give a recursive presentation of this relation for the calculus used
in the encoding.

$\red \subseteq \pi \times \pi$
$\red : \pi \to \mathcal{P}(\pi)$

\begin{mathpar}
  \inferrule* [lab=Comm] { \textsf{match}( x_{src}, x_{trgt} ) } { x_{trgt}?(y)P \; | \; x_{src}!\langle {Q} \rangle \red P\{\quotep{Q}/y}\} }
  \and \\
  \inferrule* [lab=Par] {{P} \red {P}'} {{{P} | {Q}} \red {{P}' | {Q}}}
  \and
  \inferrule* [lab=Equiv]{{{P} \scong {P}'} \andalso {{P}' \red {Q}'} \andalso {{Q}' \scong {Q}}}{{P} \red {Q}}
\end{mathpar}

\begin{eqnarray*}
  match_{\equiv} (\quotep{P},\quotep{Q}) & := & P \equiv Q \\
  match_{\dagger}(\quotep{P},\quotep{Q}) & := & \forall R. P|Q \red^{*} R => R \red^{*} 0 \\
  match_{K}(\quotep{P},\quotep{Q}) & := & K \mbox{ for some context } K
\end{eqnarray*}

$u?(x)P | u!\langle Q \rangle \red P\{\quotep{Q}/x\}$

%We write $\wred$ for $\red^*$, and $P\red$ if $\exists Q $ such that $ P \red Q$.
We write $P\red$ if $\exists Q $ such that $ P \red Q$ and $P\not\red$, otherwise.

\section{Replication}

As mentioned before, it is known that replication (and hence
recursion) can be implemented in a higher-order process algebra
\cite{SangiorgiWalker}. As our first example of calculation with the
machinery thus far presented we give the construction explicitly in
the {\rhoc}.

\begin{eqnarray}
	D_{x} & := & \prefix{x}{y}{(\binpar{\outputp{x}{y}}{@{y}})} \nonumber\\
	\bangp_{x}{P} & := & \binpar{{x}!\langle{\binpar{D_{x}}{P}}\rangle}{D_{x}} \nonumber
\end{eqnarray}

\begin{eqnarray}
	\bangp_{x}{P} & & \nonumber\\
	=
	& {x}!\langle{(\prefix{x}{y}{(\outputp{x}{y} | @{y})) | P}}\rangle 
	      | \prefix{x}{y}{(\outputp{x}{y} | @{y})} & \nonumber\\
	\red
	& (\outputp{x}{y} | @{y})\substn{\quotep{(\prefix{x}{y}{(@{y} | \outputp{x}{y})) | P}}}{y} & \nonumber\\
	=
	& \outputp{x}{\quotep{(\prefix{x}{y}{(\outputp{x}{y} | @{y})) | P}}}
	  | {(\prefix{x}{y}{(\outputp{x}{y} | @{y})) | P}} & \nonumber\\
	\red
	& \ldots & \nonumber\\
	\red^*
	& P | P | \ldots & \nonumber
\end{eqnarray}

Of course, this encoding, as an implementation, runs away, unfolding
$\bangp{P}$ eagerly. A lazier and more implementable replication
operator, restricted to input-guarded processes, may be obtained as follows.

\begin{eqnarray}
\bangp{\prefix{u}{v}{P}} 
	:= 
	\binpar{\lift{x}{\prefix{u}{v}{(\binpar{D(x)}{P})}}}{D(x)} \nonumber
\end{eqnarray}

\begin{remark}
  Note that the lazier definition still does not deal with summation
  or mixed summation (i.e. sums over input and output). The reader is
  invited to construct definitions of replication that deal with these
  features. 

  Further, the definitions are parameterized in a name, $x$. Can you,
  gentle reader, make a definition that eliminates this parameter and
  guarantees no accidental interaction between the replication
  machinery and the process being replicated -- i.e. no accidental
  sharing of names used by the process to get its work done and the
  name(s) used by the replication to effect copying. This latter
  revision of the definition of replication is crucial to obtaining
  the expected identity $!!P \sim !P$.
\end{remark}

\begin{remark}\label{rem:paradoxical_combinator}
  The reader familiar with the lambda calculus will have noticed the
  similarity between $D$ and the paradoxical combinator.

  [Ed. note: the existence of this seems to suggest we have to be more
  restrictive on the set of processes and names we admit if we are to
  support no-cloning.]
\end{remark}

\subsubsection{Bisimulation}

The computational dynamics gives rise to another kind of equivalence,
the equivalence of computational behavior. As previously mentioned
this is typically captured \emph{via} some form of bisimulation.

% The notion we use in this paper is weak barbed bisimulation
% \cite{milner91polyadicpi}.

The notion we use in this paper is derived from weak barbed
bisimulation \cite{milner91polyadicpi}. 

\begin{definition}
An \emph{observation relation}, $\downarrow_{\mathcal N}$, over a set
of names, $\mathcal N$, is the smallest relation satisfying the rules
below.

\infrule[Out-barb]{y \in {\mathcal N}, \; x \nameeq y}
		  {\outputp{x}{v} \downarrow_{\mathcal N} x}
\infrule[Par-barb]{\mbox{$P\downarrow_{\mathcal N} x$ or $Q\downarrow_{\mathcal N} x$}}
		  {\binpar{P}{Q} \downarrow_{\mathcal N} x}

We write $P \Downarrow_{\mathcal N} x$ if there is $Q$ such that 
$P \wred Q$ and $Q \downarrow_{\mathcal N} x$.
\end{definition}

\begin{definition}
%\label{def.bbisim}
An  ${\mathcal N}$-\emph{barbed bisimulation} over a set of names, ${\mathcal N}$, is a symmetric binary relation 
${\mathcal S}_{\mathcal N}$ between agents such that $P\rel{S}_{\mathcal N}Q$ implies:
\begin{enumerate}
\item If $P \red P'$ then $Q \wred Q'$ and $P'\rel{S}_{\mathcal N} Q'$.
\item If $P\downarrow_{\mathcal N} x$, then $Q\Downarrow_{\mathcal N} x$.
\end{enumerate}
$P$ is ${\mathcal N}$-barbed bisimilar to $Q$, written
$P \wbbisim_{\mathcal N} Q$, if $P \rel{S}_{\mathcal N} Q$ for some ${\mathcal N}$-barbed bisimulation ${\mathcal S}_{\mathcal N}$.
\end{definition}

$\mathcal{R} \subseteq \pi \times \pi$

$P \mathcal{R} Q => \forall P'. P \red P' \Rightarrow \exists Q'. Q \red Q', P' \mathcal{R} Q'$

$P \vdash x \Rightarrow Q \vdash x$

\begin{mathpar}
  \inferrule*[lab=Out-barb]{x \nameeq y}{{y}!\langle{Q}\rangle \vdash x}
  \and
  \inferrule*[lab=Par-barb]{\mbox{$P\vdash x$ or $Q\vdash x$}}{\binpar{P}{Q} \vdash x}
\end{mathpar}

\subsubsection{Contexts}

One of the principle advantages of computational calculi like the
$\pi$-calculus is a well-defined notion of context,
contextual-equivalence and a correlation between
contextual-equivalence and notions of bisimulation. The notion of
context allows the decomposition of a process into (sub-)process and
its syntactic environment, its context. Thus, a context may be
thought of as a process with a ``hole'' (written $\Box$) in it. The
application of a context $M$ to a process $P$, written $M[P]$, is
tantamount to filling the hole in $M$ with $P$. In this paper we do
not need the full weight of this theory, but do make use of the notion
of context in the proof the main theorem. 

\begin{mathpar}
  \inferrule* [lab=summation] {} {{M_{M},M_{N}} \bc \Box \;|\; x.M_{A} \;|\; M_{M}+M_{N}}
  \and
  \inferrule* [lab=agent] {} {{M_{A}} \bc (\vec{x})M_{P} \;| \; \clift{P_0,\ldots,M_{P},\ldots,P_N}}
  \and \\
  \inferrule* [lab=process] {} {{M_{P}} \bc M_{N} \;| \;P|M_{P} }
\end{mathpar} 

\begin{mathpar}
  \inferrule* [lab=sychronization] {} {M_{N} \bc \Box \;|\; x?M_{F} \;|\; x!M_{C}}
  \and
  \inferrule* [lab=abstraction] {} {{M_{F}} \bc (x)M_{P} }
  \and
  \inferrule* [lab=concretion] {} {{M_{C}} \bc \langle M_{P} \rangle }
  \and \\
  \inferrule* [lab=process] {} {{M_{P}} \bc M_{N} \;| \;P|M_{P} }
\end{mathpar}

\begin{definition}[contextual application] Given a context $M$, and
  process $P$, we define the \emph{contextual application}, $M[P] :=
  M\{P/\Box\}$. That is, the contextual application of M to P is the
  substitution of $P$ for $\Box$ in $M$.
\end{definition}

$\meaningof{-} : L \to \mathcal{P}(\pi)$

\begin{mathpar}
  \inferrule* [lab=collection] {} {\meaningof{true} = \pi, \and \meaningof{~E} = \pi \setminus \meaningof{E}, \and \meaningof{E_{1} \& E_{2}} = \meaningof{E_{1}} \cap \meaningof{E_{2}}}
\end{mathpar}

\begin{mathpar}
  \inferrule* [lab=structure] {} {\meaningof{0} = \{ P \in \pi | P \equiv 0 \}, \and \\ \meaningof{E_1 | E_2} = \{ P \in \pi | P \equiv P_{1} | P_{2}, P_{1} \in \meaningof{E_{1}}, P_{2} \in \meaningof{E_2}\} }
\end{mathpar}

\begin{mathpar}
 \inferrule* [lab=behavior] {} {\meaningof{\langle a?b \rangle E} = \{ P \in \pi | P \equiv Q | u?(y)P', \\ \and \\\\ \and \\ \;\;\; u \in \meaningof{a}, \forall z.P'\{z/y\} \in \meaningof{E\{z/b\}}\}, \and \\ \meaningof{a!E} = \{ P \in \pi | P \equiv Q | x!\langle P' \rangle, x \in \meaningof{a} P' \in \meaningof{E}\} }
\end{mathpar}

\begin{mathpar}
 \inferrule* [lab=nominal] {} {\meaningof{\quotep{E}} = \{ \quotep{P} \in \quotep{\pi} | P \in \meaningof{E} \}, \and \meaningof{\quotep{P}} = \{ \quotep{Q} \in \quotep{\pi} | P \equiv Q \} \and \\ \meaningof{@\quotep{E}} = \{ P \in \pi | P \equiv @x, x \in \meaningof{E} \}}
\end{mathpar}

\begin{eqnarray*}
  \\
  \meaningof{-} : TS \to ST
\end{eqnarray*}

\begin{eqnarray*}
  \\
  L : TS \to ST
\end{eqnarray*}

\begin{eqnarray*}
  \\
  P \models E \iff P \in \meaningof{E}
\end{eqnarray*}

\begin{eqnarray*}
  P \approx_{L} Q \iff \forall E \in L. P \models E \iff Q \models E
\end{eqnarray*}

\begin{eqnarray*}
  P \approx_{K} Q
\end{eqnarray*}

\begin{eqnarray*}
  P \approx Q
\end{eqnarray*}

$\approx_{K} = \approx = \approx_{L}$

\subsubsection{Contextual duality}

Note that contexts extend the quotation operation to a family of
operations from processes to names. Given a context, $M$, we can
define a \emph{nominal context}, $\quotep{M}$ by $\quotep{M}[P] :=
\quotep{M[P]}$. To foreshadow what is to come we observe that these
operations enjoy a duality with processes very much like the duality
between vectors and maps from vectors to scalars.

Further, because the calculus is essentially higher-order, we have a
correspondence between contexts and processes. More specifically,
given a name $x$ and a context $M$ we can construct $M^{*}_{x}$ such
that 

\begin{mathpar}
  M^{*}_{x} | \lift{x}{P} \red M[P]
\end{mathpar}

namely,

\begin{mathpar}
  M^{*}_{x} := x?(u).M[\dropn{u}]
\end{mathpar}

The dependence of $M^{*}_{x}$ on a name makes it an abstraction, 

\begin{mathpar}
  M^{*} := (x)x?(u).M[\dropn{u}]
\end{mathpar}

\subsection{Additional notation}

It will sometimes be convenient to denote the process a name
quotes. We already have the notation $x = \quotep{P}$, but it will be
convenient to introduce an alternate notation, $\procn{x}$, when we
want to emphasize the connection to the use of the name. Note that, by
virtue of name equivalence, $\quotep{\procn{x}} \nameeq x$; so, the
notation is consistent with previous definitions.

Further, because names have structure it is possible to effect
substitutions on the basis of that structure. This means we need to
upgrade our notation for substitutions, which we accomplish by
adapting comprehension notation. Thus,

\begin{mathpar}
  P\{ y / x : x \in S \}
\end{mathpar}

is interpreted to mean the process derived from P by replacing (in a
capture-avoiding manner) each occurrence of $x$ in $S$ by $y$. For example,

\begin{mathpar}
  P\{ \quotep{\procn{x}|\procn{x}} / x : x \in \freenames{P} \}
\end{mathpar}

will replace each (occurrence) of a free name $x$ in $P$ by
$\quotep{\procn{x}|\procn{x}}$.

Also, we will avail ourselves of the notation $x^{L}$ and $x^{R}$ to
denote injections of a name into disjoint copies of the name
space. There are numerous ways to accomplish this. One example can be
found in \cite{MeredithR05}. This notation overloads to vectors of
names: $\vec{x}^{\pi} := (x_{i}^{\pi} \; : \; 0 \leq i < |\vec{x}| )$ where $\pi \in \{L,R\}$.

We also use $P^{\Box} := P|\Box$.

In \cite{MeredithR05} an interpretation of the new operator is
given. It turns out that there are several possible interpretations
all enjoying the requisite algebraic properties of the operator (see
\cite{milner91polyadicpi}). We will therefore make liberal use of
$(\nu\; \vec{x})P$.

% subsection the_syntax_and_semantics_of_the_notation_system (end)   

\input{qm2pi.qmops} 

\input{qm2pi.sterngerlach} 

\input{qm2pi.metric} 

% section concurrent_process_calculi (end)

%\input{qm2pi.proofsketch}

% section proof sketch (end)

%\input{qm2pi.slviaknots} 

% section spatial logic via knots (end)

\input{qm2pi.conclusion}

% section conclusion (end)

%\input{qm2pi.dtcodes} 

% section wiring algorithm (end)

\input{qm2pi.ack} 

% section acknowledgments (end)

\newpage


\bibliographystyle{plain}   
\bibliography{../../biblios/main.bib}

\input{qm2pi.rhodetails}

\end{document}



% section front matter (end)

\section{Introduction}\label{sec:introduction} % (fold)
In this draft of the material i am going to have to dispense with the
usual writing conventions adopted in papers on these topics. i'm going
to have adopt whatever tone i need at the time i'm writing up the
calculations. Sometimes this may be very conversational; others it may
be the barest mathematical grunts; others still it may be that i have
lifted text from one of my other papers because the exposition of some
point was better said there. i hope that my readers are not unduly put
out by this decision. i'm not doing this to flout convention or be
rebellious. i find these calculations very technically challenging. To
keep everything going technically, something has to give; i have to
let go of some cognitive burden. So, the academic writing style --
with all of its trade-offs in terms of facilitating technical
communication -- is what i'm letting go of. Perhaps subsequent drafts
can be tightened and polished, but for now, i'm going to speak as if
we were sitting together in a coffee shop with a laptop, wifi and a
pad of paper and a pencil.

So, here's what i have to say. We -- you and i, comfortably ensconced
in our coffee shop and well-equipped with our tools -- can realize and
carry out the calculations of quantum mechanics over a very different
formal theory of dynamics, a formal theory of dynamics that
corresponds to a theory of concurrent computation with
\emph{reflection}. It has the advantage that the underlying theory is
already `quantized', but supports analogues all of the continuuous
operations. Strikingly, this underlying theory has recently been
connected with a notion of metric that we can show, by calculating
together, coincides with the metric induced by the inner product.

There are a lot of reasons why you might be interested in seeing
calculations of this form. Here's why i'm interested. For the past
several centuries there has been no competitor to the ``Newtonian''
account of dynamics. As a result the predominant share of accounts of
dynamical systems and situations have had to be formulated in terms of
the Newtonian machinery. i view this as an intellectually dangerous
position to occupy. Everything, despite it's intrinsic shape, turns
into a nail to be hit with this hammer. Recently, however, the theory
of computation has matured to the point where we have candidates for
theories of dynamics that offer very different perspective on
reasoning about dynamical systems and situations. Testing these
candidates against very successful accounts of dynamical situations,
like quantum mechanics, is going to give us some sense of how mature
they are and some measure of the quality of these accounts of
dynamics.

\subsection{Summary of contributions and outline of paper}

So, we're going to develop an interpretation of the operations of
quantum mechanics normally interpreted by Hilbert spaces and
operators. We're going to do this over a theory of computation. Note
that this is very different than the usual quantum computation program
which develops notions of computation over quantum mechanics. Rather,
we are developing a story that aligns with Wheeler's slogan: It from
Bit. To do this we will first provide an account of the theory of
computation at play here. Then we will dive into a calculation-driven
interpretation of the operations of quantum mechanics.

The reason we take this approach is that -- until very recently --
there hasn't been an axiomatic account of quantum mechanics. As a
result there has been no sharp delineation of the mathematical theory
supporting interpretation of the physical theory and the physical
theory, itself. So, ambient features of the maths are free to be
exploited (or supressed) without a real accounting of their physical
relevance. There is no sharp statement ``here's the physical theory''
qua \emph{theory} and ``here's the mathematical interpretation''
enabling a judgment of how faithful the interpretation is -- apart
from experimental observation. When there is an axiomatic account we
can judge how well a given mathematical formalism supports an
interpretation of the axioms, independent of
experimentation. Likewise, we can judge how well we have captured our
physical evidence and experience with our axiomatics, independent of
any specific mathematical implementation, with accidental detail that
may or may not have physical significance. 

In lieu of a fully fleshed out and vetted axiomatic account of quantum
mechanics, interpreting the operational notions in service of modeling
physical systems will have to suffice. In other words, we are not in
the business of providing a model of Hilbert spaces and operators. We
are in the business of providing a model of quantum mechanics because
we are motivated by testing our notions of dynamics against physical
theory; and, the predictive calculations of the physical theory must
serve as the best formulation -- shy of a fully fleshed out axiomatic
account -- of the physical theory itself (as they have for scientific
theories since time immemorial). Put another way, despite a
whole-hearted commitment to an It-from-Bit ontology, we are firmly
aligned with the shut-up-and-calculate camp as the best way to obtain
results either from the physical perspective or as a quality assurance
measure of our fledgling theory of dynamics.

In detail, we present a reflective process calculus. Then we develop
intuitive correspondences between the notions available in this
calculus and the usual physical notions supporting quantum mechanical
calculations. Thus, 

\begin{table}[htp]
  \center{
    \fbox{
      \begin{tabular}{c|c}
        quantum mechanics & process calculus \\
        \hline
        scalar & name \\
        state vector & process \\
        dual & contextual duals \\
        matrix & formal sums of process-context-dual pairs \\
        orthogonality & process annihilation \\
        inner product & execution-formula + quoting
      \end{tabular}
    }
  }
  \caption{QM - process calculi correspondences}
\end{table}

Then we tighten up these intuitions to operational definitions. We
employ the Dirac notation as the best proxy we can find for an
abstract syntax of the quantum mechanical notions. The definitions we
develop put us in contact with equational constraints coming from the
theory that we demonstrate the definitions and calculations satisfy.

This puts us in a position to shut up and calculate for the
Stern-Gerlach experimental set up, showing how these predictive
calculations become calculations on processes in our theory of a
reflective process calculus.

Penultimately, we demonstrate that the notion of metric coming from
the inner product coincides with the notion of metric available from
the theory of bisimulation. This demonstration gives us the right to
think of space as arising from behavior. Finally, we consider where we
might go from the new vantage point we have obtained.

% section introduction (end) 
 
% section introduction (end)

% \documentclass[12pt]{llncs}
%\documentclass{jktr}

\usepackage[pdftex]{hyperref}                   
\usepackage {listings}
\usepackage {mathpartir}
\usepackage{bcprules}
%\usepackage{listings}
                       
\usepackage{graphicx} 
%\usepackage[margins=2.5cm,nohead,nofoot]{geometry}
%\usepackage{geometry}
\usepackage{amsfonts}
\usepackage{amstext}
\usepackage{latexsym}
\usepackage{amssymb}
\usepackage{color}


%\include{myPreamble}
\include{qm2pi.local} 

%\ifpdf
%\usepackage[pdftex]{graphicx}
%\else
%\usepackage{graphicx}
%\fi

 % \ifpdf
%  \usepackage{pdfsync}
%  \if


%\title{Brief Article}
%\author{David F. Snyder}
%\author{L.G. Meredith}

%\address{Dept. of Math., Texas State University--San Marcos, San Marcos, TX 78666}
       
\pagestyle{empty}


\begin{document}

\lstset{language=[Objective]Caml,frame=shadowbox}

\input{qm2pi.front}

% section front matter (end)

\input{qm2pi.intro} 
 
% section introduction (end)

% \input{qm2pi.knotations} 

% section notation (end)

\input{qm2pi.process.calculi} 

% section concurrent_process_calculi_and_spatial_logics_ (end)
    
%\input{qm2pi.knots2pi} 

%\input{qm2pi.trefoil} 

%\input{qm2pi.mainthm} 

% subsection basic_interpretation (end)

%\input{qm2pi.rho.presentation} 
\subsection{The syntax and semantics of the notation system}\label{sub:the_syntax_and_semantics_of_the_notation_system} % (fold)

We now summarize a technical presentation of the calculus that
embodies our theory of dynamics. The typical presentation of such a
calculus follows the style of giving generators and relations on
them. The grammar, below, describing term constructors, freely
generates the set of processes, $\Proc$. This set is then quotiented
by a relation known as structural congruence and it is over this set
that the notion of dynamics is expressed. This presentation is
essentially that of \cite{MeredithR05} with the addition of
polyadicity and summation. For readability we have relegated some of
the technical subtleties to an appendix.

\subsubsection{Process grammar}\label{subsub:process_grammar}

\begin{mathpar}
  \inferrule* [lab=synchronization] {} {{M} \bc \pzero \;|\; x?F \;|\; x!C }
  \and
  \inferrule* [lab=abstraction] {} {{F} \bc (x)P}
  \and
  \inferrule* [lab=concretion] {} {{C} \bc \langle Q \rangle}
  \and
  \inferrule* [lab=process] {} {{P,Q} \bc M \;| \;P|Q \;|\; @{x}}
  \and
  \inferrule* [lab=name] {} {{x} \bc \quotep{P}}
\end{mathpar} 

Note that $\vec{x}$ (resp. $\vec{P}$) denotes a vector of names
(resp. processes) of length $|\vec{x}|$ (resp. $|\vec{P}|$). We adopt
the following useful abbreviations.

\begin{mathpar}
   x?(\vec{y}).P := x.(\vec{y})P \and  x\clift{\vec{P}} := x.\clift{\vec{P}}
   \and x!(y) := \lift{x}{\dropn{y}}
   \and \Pi_{i=0}^{n-1}P_i := P_0 | \ldots | P_{n-1}
\end{mathpar}

\subsubsection{Structural congruence}

\paragraph{Free and bound names and alpha-equivalence.} At the
core of structural equivalence is alpha-equivalence which identifies
process that are the same up to a change of variable. Formally, we
recognize the distinction between free and bound names. The free names
of a process, $\freenames{P}$, may be calculated recursively as
follows:

\begin{mathpar}
\freenames{\pzero} := \emptyset
  \and \\
  \freenames{x?(y).P} := \{ x \} \cup (\freenames{P} \setminus \{ y \})
  \and 
  \freenames{x!\langle P \rangle} := \{ x \} \cup \{ P \} 
  \and \\
  \freenames{P|Q} := \freenames{P} \cup \freenames{Q}
  \and \\
  \freenames{@{x}} := \{ x \}
\end{mathpar}

$\pi$
$\quotep{\pi}$

$\freenames{-} : \pi \to \mathcal{P}(\quotep{\pi})$

\begin{eqnarray*}
  \freenames{\pzero} & := & \emptyset \\
  \freenames{x?(y).P} & := & \{ x \} \cup (\freenames{P} \setminus \{ y \}) \\
  \freenames{x!\langle P \rangle} & := & \{ x \} \cup \{ P \} \\
  \freenames{P|Q} & := & \freenames{P} \cup \freenames{Q} \\
  \freenames{\dropn{x}} & := & \{ x \}
\end{eqnarray*}

The bound names of a process, $\boundnames{P}$, are those names occurring in $P$
that are not free. For example, in $x?(y).0$, the name $x$ is free, while $y$ is bound.

\begin{mathpar}
  \inferrule* [lab=monoidal-laws] {} { P|Q \equiv Q|P \and P|0 \equiv P \and P|(Q|R) \equiv (P|Q)|R }
\end{mathpar}

\begin{mathpar}
  \inferrule* [lab=alpha-equivalence] {} { (x)P \equiv (y)P\{y/x\} \and y \not\in \freenames{P} }
\end{mathpar}

\begin{definition}
Then two processes, $P,Q$, are alpha-equivalent if $P = Q\{\vec{y}/\vec{x}\}$ for
some $\vec{x} \in \boundnames{Q},\vec{y} \in \boundnames{P}$, where $Q\{\vec{y}/\vec{x}\}$
denotes the capture-avoiding substitution of $\vec{y}$ for $\vec{x}$ in $Q$.
\end{definition}

\begin{definition}
  The {\em structural congruence} \cite{SangiorgiWalker} , $\equiv$,
  between processes is the least congruence containing
  alpha-equivalence, satisfying the abelian monoid laws
  (associativity, commutativity and $\pzero$ as identity) for parallel
  composition $|$ and for summation $+$.
\end{definition}

\subsection{Name equivalence}

We take name equivalence, written $\nameeq$, to be the smallest
equivalence relation generated by the following rules.

\begin{mathpar}
\inferrule*[lab=Quote-drop]
{ }
{ \quotep{@{x}} \nameeq x }

\inferrule*[lab=Struct-equiv]
{ P \scong Q }
{ \quotep{P} \nameeq \quotep{Q} }
\end{mathpar}

The astute reader will have noticed that the mutual recursion of names
and processes imposes a mutual recursion on alpha-equivalence and
structural equivalence via name-equivalence. Fortunately, all of this
works out pleasantly and we may calculate in the natural way, free of
concern. The reader interested in the details is referred to the
appendix \ref{appendix:rho_details}.

\subsection{Substitution}

We use $\Proc$ for the set of processes, $\QProc$ for the set of
names, and $\id{\{}\vec{y} / \vec{x} \id{\}}$ to denote partial maps,
$s : \QProc \rightarrow \QProc$. A map, $s$ lifts, uniquely, to a map
on process terms, $\widehat{s} : \Proc \rightarrow \Proc$ by the
following equations.

\begin{mathpar}
  (0) \psubstp{Q}{P} := 0 \\
  (R \juxtap S) \psubstp{Q}{P}
  :=    
  (R)\psubstp{Q}{P} \juxtap (S) \psubstp{Q}{P} \\
  (x?(y).R) \psubstp{Q}{P}    
  :=    
  (x)\substp{Q}{P} (z)\concat( (R \psubstn{z}{y}) \psubstp{Q}{P} ) \\
  (\lift{x}{R}) \psubstp{Q}{P}  
  :=
  \lift{(x)\substp{Q}{P}}{ R \psubstp{Q}{P} } \\
%   (\dropn{x})  \psubstp{Q}{P}       
%   := 
%   \left\{ 
%     \begin{array}{ccc} 
%       \dropn{\quotep{Q}} & & x \nameeq \quotep{P} \\
%       \dropn{x} & & otherwise \\
%     \end{array}
%   \right. 
  (\dropn{x})  \psubstp{Q}{P}       
  := 
  \left\{ 
    \begin{array}{ccc} 
      Q & & x \nameeq \quotep{P} \\
      \dropn{x} & & otherwise \\
    \end{array}
  \right.
\end{mathpar}
 

where

\begin{eqnarray}
  (x)\id{\{} \lpquote Q \rpquote / \lpquote P \rpquote \id{\}}            = 
  \left\{ 
    \begin{array}{ccc}
      \lpquote Q \rpquote & & x \nameeq \lpquote P \rpquote \\
      x & & otherwise \\
    \end{array}
  \right. \nonumber
\end{eqnarray}

and $z$ is chosen distinct from $\quotep{P}$, $\quotep{Q}$, the free
names in $Q$, and all the names in $R$. Our $\alpha$-equivalence will
be built in the standard way from this substitution.

\begin{remark}\label{rem:no_self_referential_names}
  One consequence of these definitions is that $\forall P. \quotep{P}
  \not\in \freenames{P}$.
\end{remark}

\subsection{ Dynamic quote: an example }

Anticipating something of what's to come, consider applying the
substitution, $\widehat{\id{\{}u / z \id{\}}}$, to the following pair
of processes, $\lift{w}{y!(z)}$ and $w[ \lpquote y!(z) \rpquote ]$.

\begin{eqnarray}
	\lift{w}{y!(z)}\widehat{\id{\{}u / z \id{\}}}
		& = &
		\lift{w}{y!(u)} \nonumber\\
	w[ \lpquote y!(z) \rpquote ] \widehat{ \id{\{}u / z \id{\}} }
		& = &
		w[ \lpquote y!(z) \rpquote ] \nonumber
\end{eqnarray}

Because the body of the process between quotes is impervious to
substitution, we get radically different answers. In fact, by
examining the first process in an input context,
e.g. $x?(z).\lift{w}{y!(z)}$, we see that the process under the lift
operator may be shaped by prefixed inputs binding a name inside it. In
this sense, the lift operator will be seen as a way to dynamically
construct processes before reifying them as names.

Finally equipped with these standard features we can present the
dynamics of the calculus.

\subsubsection{Operational semantics} 

Finally, we introduce the computational dynamics. What marks these
algebras as distinct from other more traditionally studied algebraic
structures, e.g. vector spaces or polynomial rings, is the manner in
which dynamics is captured. In traditional structures, dynamics is typically
expressed through morphisms between such structures, as in linear maps
between vector spaces or morphisms between rings. In algebras
associated with the semantics of computation, the dynamics is
expressed as part of the algebraic structure itself, through a
reduction reduction relation typically denoted by $\red$. Below, we
give a recursive presentation of this relation for the calculus used
in the encoding.

$\red \subseteq \pi \times \pi$
$\red : \pi \to \mathcal{P}(\pi)$

\begin{mathpar}
  \inferrule* [lab=Comm] { \textsf{match}( x_{src}, x_{trgt} ) } { x_{trgt}?(y)P \; | \; x_{src}!\langle {Q} \rangle \red P\{\quotep{Q}/y}\} }
  \and \\
  \inferrule* [lab=Par] {{P} \red {P}'} {{{P} | {Q}} \red {{P}' | {Q}}}
  \and
  \inferrule* [lab=Equiv]{{{P} \scong {P}'} \andalso {{P}' \red {Q}'} \andalso {{Q}' \scong {Q}}}{{P} \red {Q}}
\end{mathpar}

\begin{eqnarray*}
  match_{\equiv} (\quotep{P},\quotep{Q}) & := & P \equiv Q \\
  match_{\dagger}(\quotep{P},\quotep{Q}) & := & \forall R. P|Q \red^{*} R => R \red^{*} 0 \\
  match_{K}(\quotep{P},\quotep{Q}) & := & K \mbox{ for some context } K
\end{eqnarray*}

$u?(x)P | u!\langle Q \rangle \red P\{\quotep{Q}/x\}$

%We write $\wred$ for $\red^*$, and $P\red$ if $\exists Q $ such that $ P \red Q$.
We write $P\red$ if $\exists Q $ such that $ P \red Q$ and $P\not\red$, otherwise.

\section{Replication}

As mentioned before, it is known that replication (and hence
recursion) can be implemented in a higher-order process algebra
\cite{SangiorgiWalker}. As our first example of calculation with the
machinery thus far presented we give the construction explicitly in
the {\rhoc}.

\begin{eqnarray}
	D_{x} & := & \prefix{x}{y}{(\binpar{\outputp{x}{y}}{@{y}})} \nonumber\\
	\bangp_{x}{P} & := & \binpar{{x}!\langle{\binpar{D_{x}}{P}}\rangle}{D_{x}} \nonumber
\end{eqnarray}

\begin{eqnarray}
	\bangp_{x}{P} & & \nonumber\\
	=
	& {x}!\langle{(\prefix{x}{y}{(\outputp{x}{y} | @{y})) | P}}\rangle 
	      | \prefix{x}{y}{(\outputp{x}{y} | @{y})} & \nonumber\\
	\red
	& (\outputp{x}{y} | @{y})\substn{\quotep{(\prefix{x}{y}{(@{y} | \outputp{x}{y})) | P}}}{y} & \nonumber\\
	=
	& \outputp{x}{\quotep{(\prefix{x}{y}{(\outputp{x}{y} | @{y})) | P}}}
	  | {(\prefix{x}{y}{(\outputp{x}{y} | @{y})) | P}} & \nonumber\\
	\red
	& \ldots & \nonumber\\
	\red^*
	& P | P | \ldots & \nonumber
\end{eqnarray}

Of course, this encoding, as an implementation, runs away, unfolding
$\bangp{P}$ eagerly. A lazier and more implementable replication
operator, restricted to input-guarded processes, may be obtained as follows.

\begin{eqnarray}
\bangp{\prefix{u}{v}{P}} 
	:= 
	\binpar{\lift{x}{\prefix{u}{v}{(\binpar{D(x)}{P})}}}{D(x)} \nonumber
\end{eqnarray}

\begin{remark}
  Note that the lazier definition still does not deal with summation
  or mixed summation (i.e. sums over input and output). The reader is
  invited to construct definitions of replication that deal with these
  features. 

  Further, the definitions are parameterized in a name, $x$. Can you,
  gentle reader, make a definition that eliminates this parameter and
  guarantees no accidental interaction between the replication
  machinery and the process being replicated -- i.e. no accidental
  sharing of names used by the process to get its work done and the
  name(s) used by the replication to effect copying. This latter
  revision of the definition of replication is crucial to obtaining
  the expected identity $!!P \sim !P$.
\end{remark}

\begin{remark}\label{rem:paradoxical_combinator}
  The reader familiar with the lambda calculus will have noticed the
  similarity between $D$ and the paradoxical combinator.

  [Ed. note: the existence of this seems to suggest we have to be more
  restrictive on the set of processes and names we admit if we are to
  support no-cloning.]
\end{remark}

\subsubsection{Bisimulation}

The computational dynamics gives rise to another kind of equivalence,
the equivalence of computational behavior. As previously mentioned
this is typically captured \emph{via} some form of bisimulation.

% The notion we use in this paper is weak barbed bisimulation
% \cite{milner91polyadicpi}.

The notion we use in this paper is derived from weak barbed
bisimulation \cite{milner91polyadicpi}. 

\begin{definition}
An \emph{observation relation}, $\downarrow_{\mathcal N}$, over a set
of names, $\mathcal N$, is the smallest relation satisfying the rules
below.

\infrule[Out-barb]{y \in {\mathcal N}, \; x \nameeq y}
		  {\outputp{x}{v} \downarrow_{\mathcal N} x}
\infrule[Par-barb]{\mbox{$P\downarrow_{\mathcal N} x$ or $Q\downarrow_{\mathcal N} x$}}
		  {\binpar{P}{Q} \downarrow_{\mathcal N} x}

We write $P \Downarrow_{\mathcal N} x$ if there is $Q$ such that 
$P \wred Q$ and $Q \downarrow_{\mathcal N} x$.
\end{definition}

\begin{definition}
%\label{def.bbisim}
An  ${\mathcal N}$-\emph{barbed bisimulation} over a set of names, ${\mathcal N}$, is a symmetric binary relation 
${\mathcal S}_{\mathcal N}$ between agents such that $P\rel{S}_{\mathcal N}Q$ implies:
\begin{enumerate}
\item If $P \red P'$ then $Q \wred Q'$ and $P'\rel{S}_{\mathcal N} Q'$.
\item If $P\downarrow_{\mathcal N} x$, then $Q\Downarrow_{\mathcal N} x$.
\end{enumerate}
$P$ is ${\mathcal N}$-barbed bisimilar to $Q$, written
$P \wbbisim_{\mathcal N} Q$, if $P \rel{S}_{\mathcal N} Q$ for some ${\mathcal N}$-barbed bisimulation ${\mathcal S}_{\mathcal N}$.
\end{definition}

$\mathcal{R} \subseteq \pi \times \pi$

$P \mathcal{R} Q => \forall P'. P \red P' \Rightarrow \exists Q'. Q \red Q', P' \mathcal{R} Q'$

$P \vdash x \Rightarrow Q \vdash x$

\begin{mathpar}
  \inferrule*[lab=Out-barb]{x \nameeq y}{{y}!\langle{Q}\rangle \vdash x}
  \and
  \inferrule*[lab=Par-barb]{\mbox{$P\vdash x$ or $Q\vdash x$}}{\binpar{P}{Q} \vdash x}
\end{mathpar}

\subsubsection{Contexts}

One of the principle advantages of computational calculi like the
$\pi$-calculus is a well-defined notion of context,
contextual-equivalence and a correlation between
contextual-equivalence and notions of bisimulation. The notion of
context allows the decomposition of a process into (sub-)process and
its syntactic environment, its context. Thus, a context may be
thought of as a process with a ``hole'' (written $\Box$) in it. The
application of a context $M$ to a process $P$, written $M[P]$, is
tantamount to filling the hole in $M$ with $P$. In this paper we do
not need the full weight of this theory, but do make use of the notion
of context in the proof the main theorem. 

\begin{mathpar}
  \inferrule* [lab=summation] {} {{M_{M},M_{N}} \bc \Box \;|\; x.M_{A} \;|\; M_{M}+M_{N}}
  \and
  \inferrule* [lab=agent] {} {{M_{A}} \bc (\vec{x})M_{P} \;| \; \clift{P_0,\ldots,M_{P},\ldots,P_N}}
  \and \\
  \inferrule* [lab=process] {} {{M_{P}} \bc M_{N} \;| \;P|M_{P} }
\end{mathpar} 

\begin{mathpar}
  \inferrule* [lab=sychronization] {} {M_{N} \bc \Box \;|\; x?M_{F} \;|\; x!M_{C}}
  \and
  \inferrule* [lab=abstraction] {} {{M_{F}} \bc (x)M_{P} }
  \and
  \inferrule* [lab=concretion] {} {{M_{C}} \bc \langle M_{P} \rangle }
  \and \\
  \inferrule* [lab=process] {} {{M_{P}} \bc M_{N} \;| \;P|M_{P} }
\end{mathpar}

\begin{definition}[contextual application] Given a context $M$, and
  process $P$, we define the \emph{contextual application}, $M[P] :=
  M\{P/\Box\}$. That is, the contextual application of M to P is the
  substitution of $P$ for $\Box$ in $M$.
\end{definition}

$\meaningof{-} : L \to \mathcal{P}(\pi)$

\begin{mathpar}
  \inferrule* [lab=collection] {} {\meaningof{true} = \pi, \and \meaningof{~E} = \pi \setminus \meaningof{E}, \and \meaningof{E_{1} \& E_{2}} = \meaningof{E_{1}} \cap \meaningof{E_{2}}}
\end{mathpar}

\begin{mathpar}
  \inferrule* [lab=structure] {} {\meaningof{0} = \{ P \in \pi | P \equiv 0 \}, \and \\ \meaningof{E_1 | E_2} = \{ P \in \pi | P \equiv P_{1} | P_{2}, P_{1} \in \meaningof{E_{1}}, P_{2} \in \meaningof{E_2}\} }
\end{mathpar}

\begin{mathpar}
 \inferrule* [lab=behavior] {} {\meaningof{\langle a?b \rangle E} = \{ P \in \pi | P \equiv Q | u?(y)P', \\ \and \\\\ \and \\ \;\;\; u \in \meaningof{a}, \forall z.P'\{z/y\} \in \meaningof{E\{z/b\}}\}, \and \\ \meaningof{a!E} = \{ P \in \pi | P \equiv Q | x!\langle P' \rangle, x \in \meaningof{a} P' \in \meaningof{E}\} }
\end{mathpar}

\begin{mathpar}
 \inferrule* [lab=nominal] {} {\meaningof{\quotep{E}} = \{ \quotep{P} \in \quotep{\pi} | P \in \meaningof{E} \}, \and \meaningof{\quotep{P}} = \{ \quotep{Q} \in \quotep{\pi} | P \equiv Q \} \and \\ \meaningof{@\quotep{E}} = \{ P \in \pi | P \equiv @x, x \in \meaningof{E} \}}
\end{mathpar}

\begin{eqnarray*}
  \\
  \meaningof{-} : TS \to ST
\end{eqnarray*}

\begin{eqnarray*}
  \\
  L : TS \to ST
\end{eqnarray*}

\begin{eqnarray*}
  \\
  P \models E \iff P \in \meaningof{E}
\end{eqnarray*}

\begin{eqnarray*}
  P \approx_{L} Q \iff \forall E \in L. P \models E \iff Q \models E
\end{eqnarray*}

\begin{eqnarray*}
  P \approx_{K} Q
\end{eqnarray*}

\begin{eqnarray*}
  P \approx Q
\end{eqnarray*}

$\approx_{K} = \approx = \approx_{L}$

\subsubsection{Contextual duality}

Note that contexts extend the quotation operation to a family of
operations from processes to names. Given a context, $M$, we can
define a \emph{nominal context}, $\quotep{M}$ by $\quotep{M}[P] :=
\quotep{M[P]}$. To foreshadow what is to come we observe that these
operations enjoy a duality with processes very much like the duality
between vectors and maps from vectors to scalars.

Further, because the calculus is essentially higher-order, we have a
correspondence between contexts and processes. More specifically,
given a name $x$ and a context $M$ we can construct $M^{*}_{x}$ such
that 

\begin{mathpar}
  M^{*}_{x} | \lift{x}{P} \red M[P]
\end{mathpar}

namely,

\begin{mathpar}
  M^{*}_{x} := x?(u).M[\dropn{u}]
\end{mathpar}

The dependence of $M^{*}_{x}$ on a name makes it an abstraction, 

\begin{mathpar}
  M^{*} := (x)x?(u).M[\dropn{u}]
\end{mathpar}

\subsection{Additional notation}

It will sometimes be convenient to denote the process a name
quotes. We already have the notation $x = \quotep{P}$, but it will be
convenient to introduce an alternate notation, $\procn{x}$, when we
want to emphasize the connection to the use of the name. Note that, by
virtue of name equivalence, $\quotep{\procn{x}} \nameeq x$; so, the
notation is consistent with previous definitions.

Further, because names have structure it is possible to effect
substitutions on the basis of that structure. This means we need to
upgrade our notation for substitutions, which we accomplish by
adapting comprehension notation. Thus,

\begin{mathpar}
  P\{ y / x : x \in S \}
\end{mathpar}

is interpreted to mean the process derived from P by replacing (in a
capture-avoiding manner) each occurrence of $x$ in $S$ by $y$. For example,

\begin{mathpar}
  P\{ \quotep{\procn{x}|\procn{x}} / x : x \in \freenames{P} \}
\end{mathpar}

will replace each (occurrence) of a free name $x$ in $P$ by
$\quotep{\procn{x}|\procn{x}}$.

Also, we will avail ourselves of the notation $x^{L}$ and $x^{R}$ to
denote injections of a name into disjoint copies of the name
space. There are numerous ways to accomplish this. One example can be
found in \cite{MeredithR05}. This notation overloads to vectors of
names: $\vec{x}^{\pi} := (x_{i}^{\pi} \; : \; 0 \leq i < |\vec{x}| )$ where $\pi \in \{L,R\}$.

We also use $P^{\Box} := P|\Box$.

In \cite{MeredithR05} an interpretation of the new operator is
given. It turns out that there are several possible interpretations
all enjoying the requisite algebraic properties of the operator (see
\cite{milner91polyadicpi}). We will therefore make liberal use of
$(\nu\; \vec{x})P$.

% subsection the_syntax_and_semantics_of_the_notation_system (end)   

\input{qm2pi.qmops} 

\input{qm2pi.sterngerlach} 

\input{qm2pi.metric} 

% section concurrent_process_calculi (end)

%\input{qm2pi.proofsketch}

% section proof sketch (end)

%\input{qm2pi.slviaknots} 

% section spatial logic via knots (end)

\input{qm2pi.conclusion}

% section conclusion (end)

%\input{qm2pi.dtcodes} 

% section wiring algorithm (end)

\input{qm2pi.ack} 

% section acknowledgments (end)

\newpage


\bibliographystyle{plain}   
\bibliography{../../biblios/main.bib}

\input{qm2pi.rhodetails}

\end{document}

 

% section notation (end)

\input{qm2pi.process.calculi} 

% section concurrent_process_calculi_and_spatial_logics_ (end)
    
%\documentclass[12pt]{llncs}
%\documentclass{jktr}

\usepackage[pdftex]{hyperref}                   
\usepackage {listings}
\usepackage {mathpartir}
\usepackage{bcprules}
%\usepackage{listings}
                       
\usepackage{graphicx} 
%\usepackage[margins=2.5cm,nohead,nofoot]{geometry}
%\usepackage{geometry}
\usepackage{amsfonts}
\usepackage{amstext}
\usepackage{latexsym}
\usepackage{amssymb}
\usepackage{color}


%\include{myPreamble}
\include{qm2pi.local} 

%\ifpdf
%\usepackage[pdftex]{graphicx}
%\else
%\usepackage{graphicx}
%\fi

 % \ifpdf
%  \usepackage{pdfsync}
%  \if


%\title{Brief Article}
%\author{David F. Snyder}
%\author{L.G. Meredith}

%\address{Dept. of Math., Texas State University--San Marcos, San Marcos, TX 78666}
       
\pagestyle{empty}


\begin{document}

\lstset{language=[Objective]Caml,frame=shadowbox}

\input{qm2pi.front}

% section front matter (end)

\input{qm2pi.intro} 
 
% section introduction (end)

% \input{qm2pi.knotations} 

% section notation (end)

\input{qm2pi.process.calculi} 

% section concurrent_process_calculi_and_spatial_logics_ (end)
    
%\input{qm2pi.knots2pi} 

%\input{qm2pi.trefoil} 

%\input{qm2pi.mainthm} 

% subsection basic_interpretation (end)

%\input{qm2pi.rho.presentation} 
\subsection{The syntax and semantics of the notation system}\label{sub:the_syntax_and_semantics_of_the_notation_system} % (fold)

We now summarize a technical presentation of the calculus that
embodies our theory of dynamics. The typical presentation of such a
calculus follows the style of giving generators and relations on
them. The grammar, below, describing term constructors, freely
generates the set of processes, $\Proc$. This set is then quotiented
by a relation known as structural congruence and it is over this set
that the notion of dynamics is expressed. This presentation is
essentially that of \cite{MeredithR05} with the addition of
polyadicity and summation. For readability we have relegated some of
the technical subtleties to an appendix.

\subsubsection{Process grammar}\label{subsub:process_grammar}

\begin{mathpar}
  \inferrule* [lab=synchronization] {} {{M} \bc \pzero \;|\; x?F \;|\; x!C }
  \and
  \inferrule* [lab=abstraction] {} {{F} \bc (x)P}
  \and
  \inferrule* [lab=concretion] {} {{C} \bc \langle Q \rangle}
  \and
  \inferrule* [lab=process] {} {{P,Q} \bc M \;| \;P|Q \;|\; @{x}}
  \and
  \inferrule* [lab=name] {} {{x} \bc \quotep{P}}
\end{mathpar} 

Note that $\vec{x}$ (resp. $\vec{P}$) denotes a vector of names
(resp. processes) of length $|\vec{x}|$ (resp. $|\vec{P}|$). We adopt
the following useful abbreviations.

\begin{mathpar}
   x?(\vec{y}).P := x.(\vec{y})P \and  x\clift{\vec{P}} := x.\clift{\vec{P}}
   \and x!(y) := \lift{x}{\dropn{y}}
   \and \Pi_{i=0}^{n-1}P_i := P_0 | \ldots | P_{n-1}
\end{mathpar}

\subsubsection{Structural congruence}

\paragraph{Free and bound names and alpha-equivalence.} At the
core of structural equivalence is alpha-equivalence which identifies
process that are the same up to a change of variable. Formally, we
recognize the distinction between free and bound names. The free names
of a process, $\freenames{P}$, may be calculated recursively as
follows:

\begin{mathpar}
\freenames{\pzero} := \emptyset
  \and \\
  \freenames{x?(y).P} := \{ x \} \cup (\freenames{P} \setminus \{ y \})
  \and 
  \freenames{x!\langle P \rangle} := \{ x \} \cup \{ P \} 
  \and \\
  \freenames{P|Q} := \freenames{P} \cup \freenames{Q}
  \and \\
  \freenames{@{x}} := \{ x \}
\end{mathpar}

$\pi$
$\quotep{\pi}$

$\freenames{-} : \pi \to \mathcal{P}(\quotep{\pi})$

\begin{eqnarray*}
  \freenames{\pzero} & := & \emptyset \\
  \freenames{x?(y).P} & := & \{ x \} \cup (\freenames{P} \setminus \{ y \}) \\
  \freenames{x!\langle P \rangle} & := & \{ x \} \cup \{ P \} \\
  \freenames{P|Q} & := & \freenames{P} \cup \freenames{Q} \\
  \freenames{\dropn{x}} & := & \{ x \}
\end{eqnarray*}

The bound names of a process, $\boundnames{P}$, are those names occurring in $P$
that are not free. For example, in $x?(y).0$, the name $x$ is free, while $y$ is bound.

\begin{mathpar}
  \inferrule* [lab=monoidal-laws] {} { P|Q \equiv Q|P \and P|0 \equiv P \and P|(Q|R) \equiv (P|Q)|R }
\end{mathpar}

\begin{mathpar}
  \inferrule* [lab=alpha-equivalence] {} { (x)P \equiv (y)P\{y/x\} \and y \not\in \freenames{P} }
\end{mathpar}

\begin{definition}
Then two processes, $P,Q$, are alpha-equivalent if $P = Q\{\vec{y}/\vec{x}\}$ for
some $\vec{x} \in \boundnames{Q},\vec{y} \in \boundnames{P}$, where $Q\{\vec{y}/\vec{x}\}$
denotes the capture-avoiding substitution of $\vec{y}$ for $\vec{x}$ in $Q$.
\end{definition}

\begin{definition}
  The {\em structural congruence} \cite{SangiorgiWalker} , $\equiv$,
  between processes is the least congruence containing
  alpha-equivalence, satisfying the abelian monoid laws
  (associativity, commutativity and $\pzero$ as identity) for parallel
  composition $|$ and for summation $+$.
\end{definition}

\subsection{Name equivalence}

We take name equivalence, written $\nameeq$, to be the smallest
equivalence relation generated by the following rules.

\begin{mathpar}
\inferrule*[lab=Quote-drop]
{ }
{ \quotep{@{x}} \nameeq x }

\inferrule*[lab=Struct-equiv]
{ P \scong Q }
{ \quotep{P} \nameeq \quotep{Q} }
\end{mathpar}

The astute reader will have noticed that the mutual recursion of names
and processes imposes a mutual recursion on alpha-equivalence and
structural equivalence via name-equivalence. Fortunately, all of this
works out pleasantly and we may calculate in the natural way, free of
concern. The reader interested in the details is referred to the
appendix \ref{appendix:rho_details}.

\subsection{Substitution}

We use $\Proc$ for the set of processes, $\QProc$ for the set of
names, and $\id{\{}\vec{y} / \vec{x} \id{\}}$ to denote partial maps,
$s : \QProc \rightarrow \QProc$. A map, $s$ lifts, uniquely, to a map
on process terms, $\widehat{s} : \Proc \rightarrow \Proc$ by the
following equations.

\begin{mathpar}
  (0) \psubstp{Q}{P} := 0 \\
  (R \juxtap S) \psubstp{Q}{P}
  :=    
  (R)\psubstp{Q}{P} \juxtap (S) \psubstp{Q}{P} \\
  (x?(y).R) \psubstp{Q}{P}    
  :=    
  (x)\substp{Q}{P} (z)\concat( (R \psubstn{z}{y}) \psubstp{Q}{P} ) \\
  (\lift{x}{R}) \psubstp{Q}{P}  
  :=
  \lift{(x)\substp{Q}{P}}{ R \psubstp{Q}{P} } \\
%   (\dropn{x})  \psubstp{Q}{P}       
%   := 
%   \left\{ 
%     \begin{array}{ccc} 
%       \dropn{\quotep{Q}} & & x \nameeq \quotep{P} \\
%       \dropn{x} & & otherwise \\
%     \end{array}
%   \right. 
  (\dropn{x})  \psubstp{Q}{P}       
  := 
  \left\{ 
    \begin{array}{ccc} 
      Q & & x \nameeq \quotep{P} \\
      \dropn{x} & & otherwise \\
    \end{array}
  \right.
\end{mathpar}
 

where

\begin{eqnarray}
  (x)\id{\{} \lpquote Q \rpquote / \lpquote P \rpquote \id{\}}            = 
  \left\{ 
    \begin{array}{ccc}
      \lpquote Q \rpquote & & x \nameeq \lpquote P \rpquote \\
      x & & otherwise \\
    \end{array}
  \right. \nonumber
\end{eqnarray}

and $z$ is chosen distinct from $\quotep{P}$, $\quotep{Q}$, the free
names in $Q$, and all the names in $R$. Our $\alpha$-equivalence will
be built in the standard way from this substitution.

\begin{remark}\label{rem:no_self_referential_names}
  One consequence of these definitions is that $\forall P. \quotep{P}
  \not\in \freenames{P}$.
\end{remark}

\subsection{ Dynamic quote: an example }

Anticipating something of what's to come, consider applying the
substitution, $\widehat{\id{\{}u / z \id{\}}}$, to the following pair
of processes, $\lift{w}{y!(z)}$ and $w[ \lpquote y!(z) \rpquote ]$.

\begin{eqnarray}
	\lift{w}{y!(z)}\widehat{\id{\{}u / z \id{\}}}
		& = &
		\lift{w}{y!(u)} \nonumber\\
	w[ \lpquote y!(z) \rpquote ] \widehat{ \id{\{}u / z \id{\}} }
		& = &
		w[ \lpquote y!(z) \rpquote ] \nonumber
\end{eqnarray}

Because the body of the process between quotes is impervious to
substitution, we get radically different answers. In fact, by
examining the first process in an input context,
e.g. $x?(z).\lift{w}{y!(z)}$, we see that the process under the lift
operator may be shaped by prefixed inputs binding a name inside it. In
this sense, the lift operator will be seen as a way to dynamically
construct processes before reifying them as names.

Finally equipped with these standard features we can present the
dynamics of the calculus.

\subsubsection{Operational semantics} 

Finally, we introduce the computational dynamics. What marks these
algebras as distinct from other more traditionally studied algebraic
structures, e.g. vector spaces or polynomial rings, is the manner in
which dynamics is captured. In traditional structures, dynamics is typically
expressed through morphisms between such structures, as in linear maps
between vector spaces or morphisms between rings. In algebras
associated with the semantics of computation, the dynamics is
expressed as part of the algebraic structure itself, through a
reduction reduction relation typically denoted by $\red$. Below, we
give a recursive presentation of this relation for the calculus used
in the encoding.

$\red \subseteq \pi \times \pi$
$\red : \pi \to \mathcal{P}(\pi)$

\begin{mathpar}
  \inferrule* [lab=Comm] { \textsf{match}( x_{src}, x_{trgt} ) } { x_{trgt}?(y)P \; | \; x_{src}!\langle {Q} \rangle \red P\{\quotep{Q}/y}\} }
  \and \\
  \inferrule* [lab=Par] {{P} \red {P}'} {{{P} | {Q}} \red {{P}' | {Q}}}
  \and
  \inferrule* [lab=Equiv]{{{P} \scong {P}'} \andalso {{P}' \red {Q}'} \andalso {{Q}' \scong {Q}}}{{P} \red {Q}}
\end{mathpar}

\begin{eqnarray*}
  match_{\equiv} (\quotep{P},\quotep{Q}) & := & P \equiv Q \\
  match_{\dagger}(\quotep{P},\quotep{Q}) & := & \forall R. P|Q \red^{*} R => R \red^{*} 0 \\
  match_{K}(\quotep{P},\quotep{Q}) & := & K \mbox{ for some context } K
\end{eqnarray*}

$u?(x)P | u!\langle Q \rangle \red P\{\quotep{Q}/x\}$

%We write $\wred$ for $\red^*$, and $P\red$ if $\exists Q $ such that $ P \red Q$.
We write $P\red$ if $\exists Q $ such that $ P \red Q$ and $P\not\red$, otherwise.

\section{Replication}

As mentioned before, it is known that replication (and hence
recursion) can be implemented in a higher-order process algebra
\cite{SangiorgiWalker}. As our first example of calculation with the
machinery thus far presented we give the construction explicitly in
the {\rhoc}.

\begin{eqnarray}
	D_{x} & := & \prefix{x}{y}{(\binpar{\outputp{x}{y}}{@{y}})} \nonumber\\
	\bangp_{x}{P} & := & \binpar{{x}!\langle{\binpar{D_{x}}{P}}\rangle}{D_{x}} \nonumber
\end{eqnarray}

\begin{eqnarray}
	\bangp_{x}{P} & & \nonumber\\
	=
	& {x}!\langle{(\prefix{x}{y}{(\outputp{x}{y} | @{y})) | P}}\rangle 
	      | \prefix{x}{y}{(\outputp{x}{y} | @{y})} & \nonumber\\
	\red
	& (\outputp{x}{y} | @{y})\substn{\quotep{(\prefix{x}{y}{(@{y} | \outputp{x}{y})) | P}}}{y} & \nonumber\\
	=
	& \outputp{x}{\quotep{(\prefix{x}{y}{(\outputp{x}{y} | @{y})) | P}}}
	  | {(\prefix{x}{y}{(\outputp{x}{y} | @{y})) | P}} & \nonumber\\
	\red
	& \ldots & \nonumber\\
	\red^*
	& P | P | \ldots & \nonumber
\end{eqnarray}

Of course, this encoding, as an implementation, runs away, unfolding
$\bangp{P}$ eagerly. A lazier and more implementable replication
operator, restricted to input-guarded processes, may be obtained as follows.

\begin{eqnarray}
\bangp{\prefix{u}{v}{P}} 
	:= 
	\binpar{\lift{x}{\prefix{u}{v}{(\binpar{D(x)}{P})}}}{D(x)} \nonumber
\end{eqnarray}

\begin{remark}
  Note that the lazier definition still does not deal with summation
  or mixed summation (i.e. sums over input and output). The reader is
  invited to construct definitions of replication that deal with these
  features. 

  Further, the definitions are parameterized in a name, $x$. Can you,
  gentle reader, make a definition that eliminates this parameter and
  guarantees no accidental interaction between the replication
  machinery and the process being replicated -- i.e. no accidental
  sharing of names used by the process to get its work done and the
  name(s) used by the replication to effect copying. This latter
  revision of the definition of replication is crucial to obtaining
  the expected identity $!!P \sim !P$.
\end{remark}

\begin{remark}\label{rem:paradoxical_combinator}
  The reader familiar with the lambda calculus will have noticed the
  similarity between $D$ and the paradoxical combinator.

  [Ed. note: the existence of this seems to suggest we have to be more
  restrictive on the set of processes and names we admit if we are to
  support no-cloning.]
\end{remark}

\subsubsection{Bisimulation}

The computational dynamics gives rise to another kind of equivalence,
the equivalence of computational behavior. As previously mentioned
this is typically captured \emph{via} some form of bisimulation.

% The notion we use in this paper is weak barbed bisimulation
% \cite{milner91polyadicpi}.

The notion we use in this paper is derived from weak barbed
bisimulation \cite{milner91polyadicpi}. 

\begin{definition}
An \emph{observation relation}, $\downarrow_{\mathcal N}$, over a set
of names, $\mathcal N$, is the smallest relation satisfying the rules
below.

\infrule[Out-barb]{y \in {\mathcal N}, \; x \nameeq y}
		  {\outputp{x}{v} \downarrow_{\mathcal N} x}
\infrule[Par-barb]{\mbox{$P\downarrow_{\mathcal N} x$ or $Q\downarrow_{\mathcal N} x$}}
		  {\binpar{P}{Q} \downarrow_{\mathcal N} x}

We write $P \Downarrow_{\mathcal N} x$ if there is $Q$ such that 
$P \wred Q$ and $Q \downarrow_{\mathcal N} x$.
\end{definition}

\begin{definition}
%\label{def.bbisim}
An  ${\mathcal N}$-\emph{barbed bisimulation} over a set of names, ${\mathcal N}$, is a symmetric binary relation 
${\mathcal S}_{\mathcal N}$ between agents such that $P\rel{S}_{\mathcal N}Q$ implies:
\begin{enumerate}
\item If $P \red P'$ then $Q \wred Q'$ and $P'\rel{S}_{\mathcal N} Q'$.
\item If $P\downarrow_{\mathcal N} x$, then $Q\Downarrow_{\mathcal N} x$.
\end{enumerate}
$P$ is ${\mathcal N}$-barbed bisimilar to $Q$, written
$P \wbbisim_{\mathcal N} Q$, if $P \rel{S}_{\mathcal N} Q$ for some ${\mathcal N}$-barbed bisimulation ${\mathcal S}_{\mathcal N}$.
\end{definition}

$\mathcal{R} \subseteq \pi \times \pi$

$P \mathcal{R} Q => \forall P'. P \red P' \Rightarrow \exists Q'. Q \red Q', P' \mathcal{R} Q'$

$P \vdash x \Rightarrow Q \vdash x$

\begin{mathpar}
  \inferrule*[lab=Out-barb]{x \nameeq y}{{y}!\langle{Q}\rangle \vdash x}
  \and
  \inferrule*[lab=Par-barb]{\mbox{$P\vdash x$ or $Q\vdash x$}}{\binpar{P}{Q} \vdash x}
\end{mathpar}

\subsubsection{Contexts}

One of the principle advantages of computational calculi like the
$\pi$-calculus is a well-defined notion of context,
contextual-equivalence and a correlation between
contextual-equivalence and notions of bisimulation. The notion of
context allows the decomposition of a process into (sub-)process and
its syntactic environment, its context. Thus, a context may be
thought of as a process with a ``hole'' (written $\Box$) in it. The
application of a context $M$ to a process $P$, written $M[P]$, is
tantamount to filling the hole in $M$ with $P$. In this paper we do
not need the full weight of this theory, but do make use of the notion
of context in the proof the main theorem. 

\begin{mathpar}
  \inferrule* [lab=summation] {} {{M_{M},M_{N}} \bc \Box \;|\; x.M_{A} \;|\; M_{M}+M_{N}}
  \and
  \inferrule* [lab=agent] {} {{M_{A}} \bc (\vec{x})M_{P} \;| \; \clift{P_0,\ldots,M_{P},\ldots,P_N}}
  \and \\
  \inferrule* [lab=process] {} {{M_{P}} \bc M_{N} \;| \;P|M_{P} }
\end{mathpar} 

\begin{mathpar}
  \inferrule* [lab=sychronization] {} {M_{N} \bc \Box \;|\; x?M_{F} \;|\; x!M_{C}}
  \and
  \inferrule* [lab=abstraction] {} {{M_{F}} \bc (x)M_{P} }
  \and
  \inferrule* [lab=concretion] {} {{M_{C}} \bc \langle M_{P} \rangle }
  \and \\
  \inferrule* [lab=process] {} {{M_{P}} \bc M_{N} \;| \;P|M_{P} }
\end{mathpar}

\begin{definition}[contextual application] Given a context $M$, and
  process $P$, we define the \emph{contextual application}, $M[P] :=
  M\{P/\Box\}$. That is, the contextual application of M to P is the
  substitution of $P$ for $\Box$ in $M$.
\end{definition}

$\meaningof{-} : L \to \mathcal{P}(\pi)$

\begin{mathpar}
  \inferrule* [lab=collection] {} {\meaningof{true} = \pi, \and \meaningof{~E} = \pi \setminus \meaningof{E}, \and \meaningof{E_{1} \& E_{2}} = \meaningof{E_{1}} \cap \meaningof{E_{2}}}
\end{mathpar}

\begin{mathpar}
  \inferrule* [lab=structure] {} {\meaningof{0} = \{ P \in \pi | P \equiv 0 \}, \and \\ \meaningof{E_1 | E_2} = \{ P \in \pi | P \equiv P_{1} | P_{2}, P_{1} \in \meaningof{E_{1}}, P_{2} \in \meaningof{E_2}\} }
\end{mathpar}

\begin{mathpar}
 \inferrule* [lab=behavior] {} {\meaningof{\langle a?b \rangle E} = \{ P \in \pi | P \equiv Q | u?(y)P', \\ \and \\\\ \and \\ \;\;\; u \in \meaningof{a}, \forall z.P'\{z/y\} \in \meaningof{E\{z/b\}}\}, \and \\ \meaningof{a!E} = \{ P \in \pi | P \equiv Q | x!\langle P' \rangle, x \in \meaningof{a} P' \in \meaningof{E}\} }
\end{mathpar}

\begin{mathpar}
 \inferrule* [lab=nominal] {} {\meaningof{\quotep{E}} = \{ \quotep{P} \in \quotep{\pi} | P \in \meaningof{E} \}, \and \meaningof{\quotep{P}} = \{ \quotep{Q} \in \quotep{\pi} | P \equiv Q \} \and \\ \meaningof{@\quotep{E}} = \{ P \in \pi | P \equiv @x, x \in \meaningof{E} \}}
\end{mathpar}

\begin{eqnarray*}
  \\
  \meaningof{-} : TS \to ST
\end{eqnarray*}

\begin{eqnarray*}
  \\
  L : TS \to ST
\end{eqnarray*}

\begin{eqnarray*}
  \\
  P \models E \iff P \in \meaningof{E}
\end{eqnarray*}

\begin{eqnarray*}
  P \approx_{L} Q \iff \forall E \in L. P \models E \iff Q \models E
\end{eqnarray*}

\begin{eqnarray*}
  P \approx_{K} Q
\end{eqnarray*}

\begin{eqnarray*}
  P \approx Q
\end{eqnarray*}

$\approx_{K} = \approx = \approx_{L}$

\subsubsection{Contextual duality}

Note that contexts extend the quotation operation to a family of
operations from processes to names. Given a context, $M$, we can
define a \emph{nominal context}, $\quotep{M}$ by $\quotep{M}[P] :=
\quotep{M[P]}$. To foreshadow what is to come we observe that these
operations enjoy a duality with processes very much like the duality
between vectors and maps from vectors to scalars.

Further, because the calculus is essentially higher-order, we have a
correspondence between contexts and processes. More specifically,
given a name $x$ and a context $M$ we can construct $M^{*}_{x}$ such
that 

\begin{mathpar}
  M^{*}_{x} | \lift{x}{P} \red M[P]
\end{mathpar}

namely,

\begin{mathpar}
  M^{*}_{x} := x?(u).M[\dropn{u}]
\end{mathpar}

The dependence of $M^{*}_{x}$ on a name makes it an abstraction, 

\begin{mathpar}
  M^{*} := (x)x?(u).M[\dropn{u}]
\end{mathpar}

\subsection{Additional notation}

It will sometimes be convenient to denote the process a name
quotes. We already have the notation $x = \quotep{P}$, but it will be
convenient to introduce an alternate notation, $\procn{x}$, when we
want to emphasize the connection to the use of the name. Note that, by
virtue of name equivalence, $\quotep{\procn{x}} \nameeq x$; so, the
notation is consistent with previous definitions.

Further, because names have structure it is possible to effect
substitutions on the basis of that structure. This means we need to
upgrade our notation for substitutions, which we accomplish by
adapting comprehension notation. Thus,

\begin{mathpar}
  P\{ y / x : x \in S \}
\end{mathpar}

is interpreted to mean the process derived from P by replacing (in a
capture-avoiding manner) each occurrence of $x$ in $S$ by $y$. For example,

\begin{mathpar}
  P\{ \quotep{\procn{x}|\procn{x}} / x : x \in \freenames{P} \}
\end{mathpar}

will replace each (occurrence) of a free name $x$ in $P$ by
$\quotep{\procn{x}|\procn{x}}$.

Also, we will avail ourselves of the notation $x^{L}$ and $x^{R}$ to
denote injections of a name into disjoint copies of the name
space. There are numerous ways to accomplish this. One example can be
found in \cite{MeredithR05}. This notation overloads to vectors of
names: $\vec{x}^{\pi} := (x_{i}^{\pi} \; : \; 0 \leq i < |\vec{x}| )$ where $\pi \in \{L,R\}$.

We also use $P^{\Box} := P|\Box$.

In \cite{MeredithR05} an interpretation of the new operator is
given. It turns out that there are several possible interpretations
all enjoying the requisite algebraic properties of the operator (see
\cite{milner91polyadicpi}). We will therefore make liberal use of
$(\nu\; \vec{x})P$.

% subsection the_syntax_and_semantics_of_the_notation_system (end)   

\input{qm2pi.qmops} 

\input{qm2pi.sterngerlach} 

\input{qm2pi.metric} 

% section concurrent_process_calculi (end)

%\input{qm2pi.proofsketch}

% section proof sketch (end)

%\input{qm2pi.slviaknots} 

% section spatial logic via knots (end)

\input{qm2pi.conclusion}

% section conclusion (end)

%\input{qm2pi.dtcodes} 

% section wiring algorithm (end)

\input{qm2pi.ack} 

% section acknowledgments (end)

\newpage


\bibliographystyle{plain}   
\bibliography{../../biblios/main.bib}

\input{qm2pi.rhodetails}

\end{document}

 

%\documentclass[12pt]{llncs}
%\documentclass{jktr}

\usepackage[pdftex]{hyperref}                   
\usepackage {listings}
\usepackage {mathpartir}
\usepackage{bcprules}
%\usepackage{listings}
                       
\usepackage{graphicx} 
%\usepackage[margins=2.5cm,nohead,nofoot]{geometry}
%\usepackage{geometry}
\usepackage{amsfonts}
\usepackage{amstext}
\usepackage{latexsym}
\usepackage{amssymb}
\usepackage{color}


%\include{myPreamble}
\include{qm2pi.local} 

%\ifpdf
%\usepackage[pdftex]{graphicx}
%\else
%\usepackage{graphicx}
%\fi

 % \ifpdf
%  \usepackage{pdfsync}
%  \if


%\title{Brief Article}
%\author{David F. Snyder}
%\author{L.G. Meredith}

%\address{Dept. of Math., Texas State University--San Marcos, San Marcos, TX 78666}
       
\pagestyle{empty}


\begin{document}

\lstset{language=[Objective]Caml,frame=shadowbox}

\input{qm2pi.front}

% section front matter (end)

\input{qm2pi.intro} 
 
% section introduction (end)

% \input{qm2pi.knotations} 

% section notation (end)

\input{qm2pi.process.calculi} 

% section concurrent_process_calculi_and_spatial_logics_ (end)
    
%\input{qm2pi.knots2pi} 

%\input{qm2pi.trefoil} 

%\input{qm2pi.mainthm} 

% subsection basic_interpretation (end)

%\input{qm2pi.rho.presentation} 
\subsection{The syntax and semantics of the notation system}\label{sub:the_syntax_and_semantics_of_the_notation_system} % (fold)

We now summarize a technical presentation of the calculus that
embodies our theory of dynamics. The typical presentation of such a
calculus follows the style of giving generators and relations on
them. The grammar, below, describing term constructors, freely
generates the set of processes, $\Proc$. This set is then quotiented
by a relation known as structural congruence and it is over this set
that the notion of dynamics is expressed. This presentation is
essentially that of \cite{MeredithR05} with the addition of
polyadicity and summation. For readability we have relegated some of
the technical subtleties to an appendix.

\subsubsection{Process grammar}\label{subsub:process_grammar}

\begin{mathpar}
  \inferrule* [lab=synchronization] {} {{M} \bc \pzero \;|\; x?F \;|\; x!C }
  \and
  \inferrule* [lab=abstraction] {} {{F} \bc (x)P}
  \and
  \inferrule* [lab=concretion] {} {{C} \bc \langle Q \rangle}
  \and
  \inferrule* [lab=process] {} {{P,Q} \bc M \;| \;P|Q \;|\; @{x}}
  \and
  \inferrule* [lab=name] {} {{x} \bc \quotep{P}}
\end{mathpar} 

Note that $\vec{x}$ (resp. $\vec{P}$) denotes a vector of names
(resp. processes) of length $|\vec{x}|$ (resp. $|\vec{P}|$). We adopt
the following useful abbreviations.

\begin{mathpar}
   x?(\vec{y}).P := x.(\vec{y})P \and  x\clift{\vec{P}} := x.\clift{\vec{P}}
   \and x!(y) := \lift{x}{\dropn{y}}
   \and \Pi_{i=0}^{n-1}P_i := P_0 | \ldots | P_{n-1}
\end{mathpar}

\subsubsection{Structural congruence}

\paragraph{Free and bound names and alpha-equivalence.} At the
core of structural equivalence is alpha-equivalence which identifies
process that are the same up to a change of variable. Formally, we
recognize the distinction between free and bound names. The free names
of a process, $\freenames{P}$, may be calculated recursively as
follows:

\begin{mathpar}
\freenames{\pzero} := \emptyset
  \and \\
  \freenames{x?(y).P} := \{ x \} \cup (\freenames{P} \setminus \{ y \})
  \and 
  \freenames{x!\langle P \rangle} := \{ x \} \cup \{ P \} 
  \and \\
  \freenames{P|Q} := \freenames{P} \cup \freenames{Q}
  \and \\
  \freenames{@{x}} := \{ x \}
\end{mathpar}

$\pi$
$\quotep{\pi}$

$\freenames{-} : \pi \to \mathcal{P}(\quotep{\pi})$

\begin{eqnarray*}
  \freenames{\pzero} & := & \emptyset \\
  \freenames{x?(y).P} & := & \{ x \} \cup (\freenames{P} \setminus \{ y \}) \\
  \freenames{x!\langle P \rangle} & := & \{ x \} \cup \{ P \} \\
  \freenames{P|Q} & := & \freenames{P} \cup \freenames{Q} \\
  \freenames{\dropn{x}} & := & \{ x \}
\end{eqnarray*}

The bound names of a process, $\boundnames{P}$, are those names occurring in $P$
that are not free. For example, in $x?(y).0$, the name $x$ is free, while $y$ is bound.

\begin{mathpar}
  \inferrule* [lab=monoidal-laws] {} { P|Q \equiv Q|P \and P|0 \equiv P \and P|(Q|R) \equiv (P|Q)|R }
\end{mathpar}

\begin{mathpar}
  \inferrule* [lab=alpha-equivalence] {} { (x)P \equiv (y)P\{y/x\} \and y \not\in \freenames{P} }
\end{mathpar}

\begin{definition}
Then two processes, $P,Q$, are alpha-equivalent if $P = Q\{\vec{y}/\vec{x}\}$ for
some $\vec{x} \in \boundnames{Q},\vec{y} \in \boundnames{P}$, where $Q\{\vec{y}/\vec{x}\}$
denotes the capture-avoiding substitution of $\vec{y}$ for $\vec{x}$ in $Q$.
\end{definition}

\begin{definition}
  The {\em structural congruence} \cite{SangiorgiWalker} , $\equiv$,
  between processes is the least congruence containing
  alpha-equivalence, satisfying the abelian monoid laws
  (associativity, commutativity and $\pzero$ as identity) for parallel
  composition $|$ and for summation $+$.
\end{definition}

\subsection{Name equivalence}

We take name equivalence, written $\nameeq$, to be the smallest
equivalence relation generated by the following rules.

\begin{mathpar}
\inferrule*[lab=Quote-drop]
{ }
{ \quotep{@{x}} \nameeq x }

\inferrule*[lab=Struct-equiv]
{ P \scong Q }
{ \quotep{P} \nameeq \quotep{Q} }
\end{mathpar}

The astute reader will have noticed that the mutual recursion of names
and processes imposes a mutual recursion on alpha-equivalence and
structural equivalence via name-equivalence. Fortunately, all of this
works out pleasantly and we may calculate in the natural way, free of
concern. The reader interested in the details is referred to the
appendix \ref{appendix:rho_details}.

\subsection{Substitution}

We use $\Proc$ for the set of processes, $\QProc$ for the set of
names, and $\id{\{}\vec{y} / \vec{x} \id{\}}$ to denote partial maps,
$s : \QProc \rightarrow \QProc$. A map, $s$ lifts, uniquely, to a map
on process terms, $\widehat{s} : \Proc \rightarrow \Proc$ by the
following equations.

\begin{mathpar}
  (0) \psubstp{Q}{P} := 0 \\
  (R \juxtap S) \psubstp{Q}{P}
  :=    
  (R)\psubstp{Q}{P} \juxtap (S) \psubstp{Q}{P} \\
  (x?(y).R) \psubstp{Q}{P}    
  :=    
  (x)\substp{Q}{P} (z)\concat( (R \psubstn{z}{y}) \psubstp{Q}{P} ) \\
  (\lift{x}{R}) \psubstp{Q}{P}  
  :=
  \lift{(x)\substp{Q}{P}}{ R \psubstp{Q}{P} } \\
%   (\dropn{x})  \psubstp{Q}{P}       
%   := 
%   \left\{ 
%     \begin{array}{ccc} 
%       \dropn{\quotep{Q}} & & x \nameeq \quotep{P} \\
%       \dropn{x} & & otherwise \\
%     \end{array}
%   \right. 
  (\dropn{x})  \psubstp{Q}{P}       
  := 
  \left\{ 
    \begin{array}{ccc} 
      Q & & x \nameeq \quotep{P} \\
      \dropn{x} & & otherwise \\
    \end{array}
  \right.
\end{mathpar}
 

where

\begin{eqnarray}
  (x)\id{\{} \lpquote Q \rpquote / \lpquote P \rpquote \id{\}}            = 
  \left\{ 
    \begin{array}{ccc}
      \lpquote Q \rpquote & & x \nameeq \lpquote P \rpquote \\
      x & & otherwise \\
    \end{array}
  \right. \nonumber
\end{eqnarray}

and $z$ is chosen distinct from $\quotep{P}$, $\quotep{Q}$, the free
names in $Q$, and all the names in $R$. Our $\alpha$-equivalence will
be built in the standard way from this substitution.

\begin{remark}\label{rem:no_self_referential_names}
  One consequence of these definitions is that $\forall P. \quotep{P}
  \not\in \freenames{P}$.
\end{remark}

\subsection{ Dynamic quote: an example }

Anticipating something of what's to come, consider applying the
substitution, $\widehat{\id{\{}u / z \id{\}}}$, to the following pair
of processes, $\lift{w}{y!(z)}$ and $w[ \lpquote y!(z) \rpquote ]$.

\begin{eqnarray}
	\lift{w}{y!(z)}\widehat{\id{\{}u / z \id{\}}}
		& = &
		\lift{w}{y!(u)} \nonumber\\
	w[ \lpquote y!(z) \rpquote ] \widehat{ \id{\{}u / z \id{\}} }
		& = &
		w[ \lpquote y!(z) \rpquote ] \nonumber
\end{eqnarray}

Because the body of the process between quotes is impervious to
substitution, we get radically different answers. In fact, by
examining the first process in an input context,
e.g. $x?(z).\lift{w}{y!(z)}$, we see that the process under the lift
operator may be shaped by prefixed inputs binding a name inside it. In
this sense, the lift operator will be seen as a way to dynamically
construct processes before reifying them as names.

Finally equipped with these standard features we can present the
dynamics of the calculus.

\subsubsection{Operational semantics} 

Finally, we introduce the computational dynamics. What marks these
algebras as distinct from other more traditionally studied algebraic
structures, e.g. vector spaces or polynomial rings, is the manner in
which dynamics is captured. In traditional structures, dynamics is typically
expressed through morphisms between such structures, as in linear maps
between vector spaces or morphisms between rings. In algebras
associated with the semantics of computation, the dynamics is
expressed as part of the algebraic structure itself, through a
reduction reduction relation typically denoted by $\red$. Below, we
give a recursive presentation of this relation for the calculus used
in the encoding.

$\red \subseteq \pi \times \pi$
$\red : \pi \to \mathcal{P}(\pi)$

\begin{mathpar}
  \inferrule* [lab=Comm] { \textsf{match}( x_{src}, x_{trgt} ) } { x_{trgt}?(y)P \; | \; x_{src}!\langle {Q} \rangle \red P\{\quotep{Q}/y}\} }
  \and \\
  \inferrule* [lab=Par] {{P} \red {P}'} {{{P} | {Q}} \red {{P}' | {Q}}}
  \and
  \inferrule* [lab=Equiv]{{{P} \scong {P}'} \andalso {{P}' \red {Q}'} \andalso {{Q}' \scong {Q}}}{{P} \red {Q}}
\end{mathpar}

\begin{eqnarray*}
  match_{\equiv} (\quotep{P},\quotep{Q}) & := & P \equiv Q \\
  match_{\dagger}(\quotep{P},\quotep{Q}) & := & \forall R. P|Q \red^{*} R => R \red^{*} 0 \\
  match_{K}(\quotep{P},\quotep{Q}) & := & K \mbox{ for some context } K
\end{eqnarray*}

$u?(x)P | u!\langle Q \rangle \red P\{\quotep{Q}/x\}$

%We write $\wred$ for $\red^*$, and $P\red$ if $\exists Q $ such that $ P \red Q$.
We write $P\red$ if $\exists Q $ such that $ P \red Q$ and $P\not\red$, otherwise.

\section{Replication}

As mentioned before, it is known that replication (and hence
recursion) can be implemented in a higher-order process algebra
\cite{SangiorgiWalker}. As our first example of calculation with the
machinery thus far presented we give the construction explicitly in
the {\rhoc}.

\begin{eqnarray}
	D_{x} & := & \prefix{x}{y}{(\binpar{\outputp{x}{y}}{@{y}})} \nonumber\\
	\bangp_{x}{P} & := & \binpar{{x}!\langle{\binpar{D_{x}}{P}}\rangle}{D_{x}} \nonumber
\end{eqnarray}

\begin{eqnarray}
	\bangp_{x}{P} & & \nonumber\\
	=
	& {x}!\langle{(\prefix{x}{y}{(\outputp{x}{y} | @{y})) | P}}\rangle 
	      | \prefix{x}{y}{(\outputp{x}{y} | @{y})} & \nonumber\\
	\red
	& (\outputp{x}{y} | @{y})\substn{\quotep{(\prefix{x}{y}{(@{y} | \outputp{x}{y})) | P}}}{y} & \nonumber\\
	=
	& \outputp{x}{\quotep{(\prefix{x}{y}{(\outputp{x}{y} | @{y})) | P}}}
	  | {(\prefix{x}{y}{(\outputp{x}{y} | @{y})) | P}} & \nonumber\\
	\red
	& \ldots & \nonumber\\
	\red^*
	& P | P | \ldots & \nonumber
\end{eqnarray}

Of course, this encoding, as an implementation, runs away, unfolding
$\bangp{P}$ eagerly. A lazier and more implementable replication
operator, restricted to input-guarded processes, may be obtained as follows.

\begin{eqnarray}
\bangp{\prefix{u}{v}{P}} 
	:= 
	\binpar{\lift{x}{\prefix{u}{v}{(\binpar{D(x)}{P})}}}{D(x)} \nonumber
\end{eqnarray}

\begin{remark}
  Note that the lazier definition still does not deal with summation
  or mixed summation (i.e. sums over input and output). The reader is
  invited to construct definitions of replication that deal with these
  features. 

  Further, the definitions are parameterized in a name, $x$. Can you,
  gentle reader, make a definition that eliminates this parameter and
  guarantees no accidental interaction between the replication
  machinery and the process being replicated -- i.e. no accidental
  sharing of names used by the process to get its work done and the
  name(s) used by the replication to effect copying. This latter
  revision of the definition of replication is crucial to obtaining
  the expected identity $!!P \sim !P$.
\end{remark}

\begin{remark}\label{rem:paradoxical_combinator}
  The reader familiar with the lambda calculus will have noticed the
  similarity between $D$ and the paradoxical combinator.

  [Ed. note: the existence of this seems to suggest we have to be more
  restrictive on the set of processes and names we admit if we are to
  support no-cloning.]
\end{remark}

\subsubsection{Bisimulation}

The computational dynamics gives rise to another kind of equivalence,
the equivalence of computational behavior. As previously mentioned
this is typically captured \emph{via} some form of bisimulation.

% The notion we use in this paper is weak barbed bisimulation
% \cite{milner91polyadicpi}.

The notion we use in this paper is derived from weak barbed
bisimulation \cite{milner91polyadicpi}. 

\begin{definition}
An \emph{observation relation}, $\downarrow_{\mathcal N}$, over a set
of names, $\mathcal N$, is the smallest relation satisfying the rules
below.

\infrule[Out-barb]{y \in {\mathcal N}, \; x \nameeq y}
		  {\outputp{x}{v} \downarrow_{\mathcal N} x}
\infrule[Par-barb]{\mbox{$P\downarrow_{\mathcal N} x$ or $Q\downarrow_{\mathcal N} x$}}
		  {\binpar{P}{Q} \downarrow_{\mathcal N} x}

We write $P \Downarrow_{\mathcal N} x$ if there is $Q$ such that 
$P \wred Q$ and $Q \downarrow_{\mathcal N} x$.
\end{definition}

\begin{definition}
%\label{def.bbisim}
An  ${\mathcal N}$-\emph{barbed bisimulation} over a set of names, ${\mathcal N}$, is a symmetric binary relation 
${\mathcal S}_{\mathcal N}$ between agents such that $P\rel{S}_{\mathcal N}Q$ implies:
\begin{enumerate}
\item If $P \red P'$ then $Q \wred Q'$ and $P'\rel{S}_{\mathcal N} Q'$.
\item If $P\downarrow_{\mathcal N} x$, then $Q\Downarrow_{\mathcal N} x$.
\end{enumerate}
$P$ is ${\mathcal N}$-barbed bisimilar to $Q$, written
$P \wbbisim_{\mathcal N} Q$, if $P \rel{S}_{\mathcal N} Q$ for some ${\mathcal N}$-barbed bisimulation ${\mathcal S}_{\mathcal N}$.
\end{definition}

$\mathcal{R} \subseteq \pi \times \pi$

$P \mathcal{R} Q => \forall P'. P \red P' \Rightarrow \exists Q'. Q \red Q', P' \mathcal{R} Q'$

$P \vdash x \Rightarrow Q \vdash x$

\begin{mathpar}
  \inferrule*[lab=Out-barb]{x \nameeq y}{{y}!\langle{Q}\rangle \vdash x}
  \and
  \inferrule*[lab=Par-barb]{\mbox{$P\vdash x$ or $Q\vdash x$}}{\binpar{P}{Q} \vdash x}
\end{mathpar}

\subsubsection{Contexts}

One of the principle advantages of computational calculi like the
$\pi$-calculus is a well-defined notion of context,
contextual-equivalence and a correlation between
contextual-equivalence and notions of bisimulation. The notion of
context allows the decomposition of a process into (sub-)process and
its syntactic environment, its context. Thus, a context may be
thought of as a process with a ``hole'' (written $\Box$) in it. The
application of a context $M$ to a process $P$, written $M[P]$, is
tantamount to filling the hole in $M$ with $P$. In this paper we do
not need the full weight of this theory, but do make use of the notion
of context in the proof the main theorem. 

\begin{mathpar}
  \inferrule* [lab=summation] {} {{M_{M},M_{N}} \bc \Box \;|\; x.M_{A} \;|\; M_{M}+M_{N}}
  \and
  \inferrule* [lab=agent] {} {{M_{A}} \bc (\vec{x})M_{P} \;| \; \clift{P_0,\ldots,M_{P},\ldots,P_N}}
  \and \\
  \inferrule* [lab=process] {} {{M_{P}} \bc M_{N} \;| \;P|M_{P} }
\end{mathpar} 

\begin{mathpar}
  \inferrule* [lab=sychronization] {} {M_{N} \bc \Box \;|\; x?M_{F} \;|\; x!M_{C}}
  \and
  \inferrule* [lab=abstraction] {} {{M_{F}} \bc (x)M_{P} }
  \and
  \inferrule* [lab=concretion] {} {{M_{C}} \bc \langle M_{P} \rangle }
  \and \\
  \inferrule* [lab=process] {} {{M_{P}} \bc M_{N} \;| \;P|M_{P} }
\end{mathpar}

\begin{definition}[contextual application] Given a context $M$, and
  process $P$, we define the \emph{contextual application}, $M[P] :=
  M\{P/\Box\}$. That is, the contextual application of M to P is the
  substitution of $P$ for $\Box$ in $M$.
\end{definition}

$\meaningof{-} : L \to \mathcal{P}(\pi)$

\begin{mathpar}
  \inferrule* [lab=collection] {} {\meaningof{true} = \pi, \and \meaningof{~E} = \pi \setminus \meaningof{E}, \and \meaningof{E_{1} \& E_{2}} = \meaningof{E_{1}} \cap \meaningof{E_{2}}}
\end{mathpar}

\begin{mathpar}
  \inferrule* [lab=structure] {} {\meaningof{0} = \{ P \in \pi | P \equiv 0 \}, \and \\ \meaningof{E_1 | E_2} = \{ P \in \pi | P \equiv P_{1} | P_{2}, P_{1} \in \meaningof{E_{1}}, P_{2} \in \meaningof{E_2}\} }
\end{mathpar}

\begin{mathpar}
 \inferrule* [lab=behavior] {} {\meaningof{\langle a?b \rangle E} = \{ P \in \pi | P \equiv Q | u?(y)P', \\ \and \\\\ \and \\ \;\;\; u \in \meaningof{a}, \forall z.P'\{z/y\} \in \meaningof{E\{z/b\}}\}, \and \\ \meaningof{a!E} = \{ P \in \pi | P \equiv Q | x!\langle P' \rangle, x \in \meaningof{a} P' \in \meaningof{E}\} }
\end{mathpar}

\begin{mathpar}
 \inferrule* [lab=nominal] {} {\meaningof{\quotep{E}} = \{ \quotep{P} \in \quotep{\pi} | P \in \meaningof{E} \}, \and \meaningof{\quotep{P}} = \{ \quotep{Q} \in \quotep{\pi} | P \equiv Q \} \and \\ \meaningof{@\quotep{E}} = \{ P \in \pi | P \equiv @x, x \in \meaningof{E} \}}
\end{mathpar}

\begin{eqnarray*}
  \\
  \meaningof{-} : TS \to ST
\end{eqnarray*}

\begin{eqnarray*}
  \\
  L : TS \to ST
\end{eqnarray*}

\begin{eqnarray*}
  \\
  P \models E \iff P \in \meaningof{E}
\end{eqnarray*}

\begin{eqnarray*}
  P \approx_{L} Q \iff \forall E \in L. P \models E \iff Q \models E
\end{eqnarray*}

\begin{eqnarray*}
  P \approx_{K} Q
\end{eqnarray*}

\begin{eqnarray*}
  P \approx Q
\end{eqnarray*}

$\approx_{K} = \approx = \approx_{L}$

\subsubsection{Contextual duality}

Note that contexts extend the quotation operation to a family of
operations from processes to names. Given a context, $M$, we can
define a \emph{nominal context}, $\quotep{M}$ by $\quotep{M}[P] :=
\quotep{M[P]}$. To foreshadow what is to come we observe that these
operations enjoy a duality with processes very much like the duality
between vectors and maps from vectors to scalars.

Further, because the calculus is essentially higher-order, we have a
correspondence between contexts and processes. More specifically,
given a name $x$ and a context $M$ we can construct $M^{*}_{x}$ such
that 

\begin{mathpar}
  M^{*}_{x} | \lift{x}{P} \red M[P]
\end{mathpar}

namely,

\begin{mathpar}
  M^{*}_{x} := x?(u).M[\dropn{u}]
\end{mathpar}

The dependence of $M^{*}_{x}$ on a name makes it an abstraction, 

\begin{mathpar}
  M^{*} := (x)x?(u).M[\dropn{u}]
\end{mathpar}

\subsection{Additional notation}

It will sometimes be convenient to denote the process a name
quotes. We already have the notation $x = \quotep{P}$, but it will be
convenient to introduce an alternate notation, $\procn{x}$, when we
want to emphasize the connection to the use of the name. Note that, by
virtue of name equivalence, $\quotep{\procn{x}} \nameeq x$; so, the
notation is consistent with previous definitions.

Further, because names have structure it is possible to effect
substitutions on the basis of that structure. This means we need to
upgrade our notation for substitutions, which we accomplish by
adapting comprehension notation. Thus,

\begin{mathpar}
  P\{ y / x : x \in S \}
\end{mathpar}

is interpreted to mean the process derived from P by replacing (in a
capture-avoiding manner) each occurrence of $x$ in $S$ by $y$. For example,

\begin{mathpar}
  P\{ \quotep{\procn{x}|\procn{x}} / x : x \in \freenames{P} \}
\end{mathpar}

will replace each (occurrence) of a free name $x$ in $P$ by
$\quotep{\procn{x}|\procn{x}}$.

Also, we will avail ourselves of the notation $x^{L}$ and $x^{R}$ to
denote injections of a name into disjoint copies of the name
space. There are numerous ways to accomplish this. One example can be
found in \cite{MeredithR05}. This notation overloads to vectors of
names: $\vec{x}^{\pi} := (x_{i}^{\pi} \; : \; 0 \leq i < |\vec{x}| )$ where $\pi \in \{L,R\}$.

We also use $P^{\Box} := P|\Box$.

In \cite{MeredithR05} an interpretation of the new operator is
given. It turns out that there are several possible interpretations
all enjoying the requisite algebraic properties of the operator (see
\cite{milner91polyadicpi}). We will therefore make liberal use of
$(\nu\; \vec{x})P$.

% subsection the_syntax_and_semantics_of_the_notation_system (end)   

\input{qm2pi.qmops} 

\input{qm2pi.sterngerlach} 

\input{qm2pi.metric} 

% section concurrent_process_calculi (end)

%\input{qm2pi.proofsketch}

% section proof sketch (end)

%\input{qm2pi.slviaknots} 

% section spatial logic via knots (end)

\input{qm2pi.conclusion}

% section conclusion (end)

%\input{qm2pi.dtcodes} 

% section wiring algorithm (end)

\input{qm2pi.ack} 

% section acknowledgments (end)

\newpage


\bibliographystyle{plain}   
\bibliography{../../biblios/main.bib}

\input{qm2pi.rhodetails}

\end{document}

 

%\documentclass[12pt]{llncs}
%\documentclass{jktr}

\usepackage[pdftex]{hyperref}                   
\usepackage {listings}
\usepackage {mathpartir}
\usepackage{bcprules}
%\usepackage{listings}
                       
\usepackage{graphicx} 
%\usepackage[margins=2.5cm,nohead,nofoot]{geometry}
%\usepackage{geometry}
\usepackage{amsfonts}
\usepackage{amstext}
\usepackage{latexsym}
\usepackage{amssymb}
\usepackage{color}


%\include{myPreamble}
\include{qm2pi.local} 

%\ifpdf
%\usepackage[pdftex]{graphicx}
%\else
%\usepackage{graphicx}
%\fi

 % \ifpdf
%  \usepackage{pdfsync}
%  \if


%\title{Brief Article}
%\author{David F. Snyder}
%\author{L.G. Meredith}

%\address{Dept. of Math., Texas State University--San Marcos, San Marcos, TX 78666}
       
\pagestyle{empty}


\begin{document}

\lstset{language=[Objective]Caml,frame=shadowbox}

\input{qm2pi.front}

% section front matter (end)

\input{qm2pi.intro} 
 
% section introduction (end)

% \input{qm2pi.knotations} 

% section notation (end)

\input{qm2pi.process.calculi} 

% section concurrent_process_calculi_and_spatial_logics_ (end)
    
%\input{qm2pi.knots2pi} 

%\input{qm2pi.trefoil} 

%\input{qm2pi.mainthm} 

% subsection basic_interpretation (end)

%\input{qm2pi.rho.presentation} 
\subsection{The syntax and semantics of the notation system}\label{sub:the_syntax_and_semantics_of_the_notation_system} % (fold)

We now summarize a technical presentation of the calculus that
embodies our theory of dynamics. The typical presentation of such a
calculus follows the style of giving generators and relations on
them. The grammar, below, describing term constructors, freely
generates the set of processes, $\Proc$. This set is then quotiented
by a relation known as structural congruence and it is over this set
that the notion of dynamics is expressed. This presentation is
essentially that of \cite{MeredithR05} with the addition of
polyadicity and summation. For readability we have relegated some of
the technical subtleties to an appendix.

\subsubsection{Process grammar}\label{subsub:process_grammar}

\begin{mathpar}
  \inferrule* [lab=synchronization] {} {{M} \bc \pzero \;|\; x?F \;|\; x!C }
  \and
  \inferrule* [lab=abstraction] {} {{F} \bc (x)P}
  \and
  \inferrule* [lab=concretion] {} {{C} \bc \langle Q \rangle}
  \and
  \inferrule* [lab=process] {} {{P,Q} \bc M \;| \;P|Q \;|\; @{x}}
  \and
  \inferrule* [lab=name] {} {{x} \bc \quotep{P}}
\end{mathpar} 

Note that $\vec{x}$ (resp. $\vec{P}$) denotes a vector of names
(resp. processes) of length $|\vec{x}|$ (resp. $|\vec{P}|$). We adopt
the following useful abbreviations.

\begin{mathpar}
   x?(\vec{y}).P := x.(\vec{y})P \and  x\clift{\vec{P}} := x.\clift{\vec{P}}
   \and x!(y) := \lift{x}{\dropn{y}}
   \and \Pi_{i=0}^{n-1}P_i := P_0 | \ldots | P_{n-1}
\end{mathpar}

\subsubsection{Structural congruence}

\paragraph{Free and bound names and alpha-equivalence.} At the
core of structural equivalence is alpha-equivalence which identifies
process that are the same up to a change of variable. Formally, we
recognize the distinction between free and bound names. The free names
of a process, $\freenames{P}$, may be calculated recursively as
follows:

\begin{mathpar}
\freenames{\pzero} := \emptyset
  \and \\
  \freenames{x?(y).P} := \{ x \} \cup (\freenames{P} \setminus \{ y \})
  \and 
  \freenames{x!\langle P \rangle} := \{ x \} \cup \{ P \} 
  \and \\
  \freenames{P|Q} := \freenames{P} \cup \freenames{Q}
  \and \\
  \freenames{@{x}} := \{ x \}
\end{mathpar}

$\pi$
$\quotep{\pi}$

$\freenames{-} : \pi \to \mathcal{P}(\quotep{\pi})$

\begin{eqnarray*}
  \freenames{\pzero} & := & \emptyset \\
  \freenames{x?(y).P} & := & \{ x \} \cup (\freenames{P} \setminus \{ y \}) \\
  \freenames{x!\langle P \rangle} & := & \{ x \} \cup \{ P \} \\
  \freenames{P|Q} & := & \freenames{P} \cup \freenames{Q} \\
  \freenames{\dropn{x}} & := & \{ x \}
\end{eqnarray*}

The bound names of a process, $\boundnames{P}$, are those names occurring in $P$
that are not free. For example, in $x?(y).0$, the name $x$ is free, while $y$ is bound.

\begin{mathpar}
  \inferrule* [lab=monoidal-laws] {} { P|Q \equiv Q|P \and P|0 \equiv P \and P|(Q|R) \equiv (P|Q)|R }
\end{mathpar}

\begin{mathpar}
  \inferrule* [lab=alpha-equivalence] {} { (x)P \equiv (y)P\{y/x\} \and y \not\in \freenames{P} }
\end{mathpar}

\begin{definition}
Then two processes, $P,Q$, are alpha-equivalent if $P = Q\{\vec{y}/\vec{x}\}$ for
some $\vec{x} \in \boundnames{Q},\vec{y} \in \boundnames{P}$, where $Q\{\vec{y}/\vec{x}\}$
denotes the capture-avoiding substitution of $\vec{y}$ for $\vec{x}$ in $Q$.
\end{definition}

\begin{definition}
  The {\em structural congruence} \cite{SangiorgiWalker} , $\equiv$,
  between processes is the least congruence containing
  alpha-equivalence, satisfying the abelian monoid laws
  (associativity, commutativity and $\pzero$ as identity) for parallel
  composition $|$ and for summation $+$.
\end{definition}

\subsection{Name equivalence}

We take name equivalence, written $\nameeq$, to be the smallest
equivalence relation generated by the following rules.

\begin{mathpar}
\inferrule*[lab=Quote-drop]
{ }
{ \quotep{@{x}} \nameeq x }

\inferrule*[lab=Struct-equiv]
{ P \scong Q }
{ \quotep{P} \nameeq \quotep{Q} }
\end{mathpar}

The astute reader will have noticed that the mutual recursion of names
and processes imposes a mutual recursion on alpha-equivalence and
structural equivalence via name-equivalence. Fortunately, all of this
works out pleasantly and we may calculate in the natural way, free of
concern. The reader interested in the details is referred to the
appendix \ref{appendix:rho_details}.

\subsection{Substitution}

We use $\Proc$ for the set of processes, $\QProc$ for the set of
names, and $\id{\{}\vec{y} / \vec{x} \id{\}}$ to denote partial maps,
$s : \QProc \rightarrow \QProc$. A map, $s$ lifts, uniquely, to a map
on process terms, $\widehat{s} : \Proc \rightarrow \Proc$ by the
following equations.

\begin{mathpar}
  (0) \psubstp{Q}{P} := 0 \\
  (R \juxtap S) \psubstp{Q}{P}
  :=    
  (R)\psubstp{Q}{P} \juxtap (S) \psubstp{Q}{P} \\
  (x?(y).R) \psubstp{Q}{P}    
  :=    
  (x)\substp{Q}{P} (z)\concat( (R \psubstn{z}{y}) \psubstp{Q}{P} ) \\
  (\lift{x}{R}) \psubstp{Q}{P}  
  :=
  \lift{(x)\substp{Q}{P}}{ R \psubstp{Q}{P} } \\
%   (\dropn{x})  \psubstp{Q}{P}       
%   := 
%   \left\{ 
%     \begin{array}{ccc} 
%       \dropn{\quotep{Q}} & & x \nameeq \quotep{P} \\
%       \dropn{x} & & otherwise \\
%     \end{array}
%   \right. 
  (\dropn{x})  \psubstp{Q}{P}       
  := 
  \left\{ 
    \begin{array}{ccc} 
      Q & & x \nameeq \quotep{P} \\
      \dropn{x} & & otherwise \\
    \end{array}
  \right.
\end{mathpar}
 

where

\begin{eqnarray}
  (x)\id{\{} \lpquote Q \rpquote / \lpquote P \rpquote \id{\}}            = 
  \left\{ 
    \begin{array}{ccc}
      \lpquote Q \rpquote & & x \nameeq \lpquote P \rpquote \\
      x & & otherwise \\
    \end{array}
  \right. \nonumber
\end{eqnarray}

and $z$ is chosen distinct from $\quotep{P}$, $\quotep{Q}$, the free
names in $Q$, and all the names in $R$. Our $\alpha$-equivalence will
be built in the standard way from this substitution.

\begin{remark}\label{rem:no_self_referential_names}
  One consequence of these definitions is that $\forall P. \quotep{P}
  \not\in \freenames{P}$.
\end{remark}

\subsection{ Dynamic quote: an example }

Anticipating something of what's to come, consider applying the
substitution, $\widehat{\id{\{}u / z \id{\}}}$, to the following pair
of processes, $\lift{w}{y!(z)}$ and $w[ \lpquote y!(z) \rpquote ]$.

\begin{eqnarray}
	\lift{w}{y!(z)}\widehat{\id{\{}u / z \id{\}}}
		& = &
		\lift{w}{y!(u)} \nonumber\\
	w[ \lpquote y!(z) \rpquote ] \widehat{ \id{\{}u / z \id{\}} }
		& = &
		w[ \lpquote y!(z) \rpquote ] \nonumber
\end{eqnarray}

Because the body of the process between quotes is impervious to
substitution, we get radically different answers. In fact, by
examining the first process in an input context,
e.g. $x?(z).\lift{w}{y!(z)}$, we see that the process under the lift
operator may be shaped by prefixed inputs binding a name inside it. In
this sense, the lift operator will be seen as a way to dynamically
construct processes before reifying them as names.

Finally equipped with these standard features we can present the
dynamics of the calculus.

\subsubsection{Operational semantics} 

Finally, we introduce the computational dynamics. What marks these
algebras as distinct from other more traditionally studied algebraic
structures, e.g. vector spaces or polynomial rings, is the manner in
which dynamics is captured. In traditional structures, dynamics is typically
expressed through morphisms between such structures, as in linear maps
between vector spaces or morphisms between rings. In algebras
associated with the semantics of computation, the dynamics is
expressed as part of the algebraic structure itself, through a
reduction reduction relation typically denoted by $\red$. Below, we
give a recursive presentation of this relation for the calculus used
in the encoding.

$\red \subseteq \pi \times \pi$
$\red : \pi \to \mathcal{P}(\pi)$

\begin{mathpar}
  \inferrule* [lab=Comm] { \textsf{match}( x_{src}, x_{trgt} ) } { x_{trgt}?(y)P \; | \; x_{src}!\langle {Q} \rangle \red P\{\quotep{Q}/y}\} }
  \and \\
  \inferrule* [lab=Par] {{P} \red {P}'} {{{P} | {Q}} \red {{P}' | {Q}}}
  \and
  \inferrule* [lab=Equiv]{{{P} \scong {P}'} \andalso {{P}' \red {Q}'} \andalso {{Q}' \scong {Q}}}{{P} \red {Q}}
\end{mathpar}

\begin{eqnarray*}
  match_{\equiv} (\quotep{P},\quotep{Q}) & := & P \equiv Q \\
  match_{\dagger}(\quotep{P},\quotep{Q}) & := & \forall R. P|Q \red^{*} R => R \red^{*} 0 \\
  match_{K}(\quotep{P},\quotep{Q}) & := & K \mbox{ for some context } K
\end{eqnarray*}

$u?(x)P | u!\langle Q \rangle \red P\{\quotep{Q}/x\}$

%We write $\wred$ for $\red^*$, and $P\red$ if $\exists Q $ such that $ P \red Q$.
We write $P\red$ if $\exists Q $ such that $ P \red Q$ and $P\not\red$, otherwise.

\section{Replication}

As mentioned before, it is known that replication (and hence
recursion) can be implemented in a higher-order process algebra
\cite{SangiorgiWalker}. As our first example of calculation with the
machinery thus far presented we give the construction explicitly in
the {\rhoc}.

\begin{eqnarray}
	D_{x} & := & \prefix{x}{y}{(\binpar{\outputp{x}{y}}{@{y}})} \nonumber\\
	\bangp_{x}{P} & := & \binpar{{x}!\langle{\binpar{D_{x}}{P}}\rangle}{D_{x}} \nonumber
\end{eqnarray}

\begin{eqnarray}
	\bangp_{x}{P} & & \nonumber\\
	=
	& {x}!\langle{(\prefix{x}{y}{(\outputp{x}{y} | @{y})) | P}}\rangle 
	      | \prefix{x}{y}{(\outputp{x}{y} | @{y})} & \nonumber\\
	\red
	& (\outputp{x}{y} | @{y})\substn{\quotep{(\prefix{x}{y}{(@{y} | \outputp{x}{y})) | P}}}{y} & \nonumber\\
	=
	& \outputp{x}{\quotep{(\prefix{x}{y}{(\outputp{x}{y} | @{y})) | P}}}
	  | {(\prefix{x}{y}{(\outputp{x}{y} | @{y})) | P}} & \nonumber\\
	\red
	& \ldots & \nonumber\\
	\red^*
	& P | P | \ldots & \nonumber
\end{eqnarray}

Of course, this encoding, as an implementation, runs away, unfolding
$\bangp{P}$ eagerly. A lazier and more implementable replication
operator, restricted to input-guarded processes, may be obtained as follows.

\begin{eqnarray}
\bangp{\prefix{u}{v}{P}} 
	:= 
	\binpar{\lift{x}{\prefix{u}{v}{(\binpar{D(x)}{P})}}}{D(x)} \nonumber
\end{eqnarray}

\begin{remark}
  Note that the lazier definition still does not deal with summation
  or mixed summation (i.e. sums over input and output). The reader is
  invited to construct definitions of replication that deal with these
  features. 

  Further, the definitions are parameterized in a name, $x$. Can you,
  gentle reader, make a definition that eliminates this parameter and
  guarantees no accidental interaction between the replication
  machinery and the process being replicated -- i.e. no accidental
  sharing of names used by the process to get its work done and the
  name(s) used by the replication to effect copying. This latter
  revision of the definition of replication is crucial to obtaining
  the expected identity $!!P \sim !P$.
\end{remark}

\begin{remark}\label{rem:paradoxical_combinator}
  The reader familiar with the lambda calculus will have noticed the
  similarity between $D$ and the paradoxical combinator.

  [Ed. note: the existence of this seems to suggest we have to be more
  restrictive on the set of processes and names we admit if we are to
  support no-cloning.]
\end{remark}

\subsubsection{Bisimulation}

The computational dynamics gives rise to another kind of equivalence,
the equivalence of computational behavior. As previously mentioned
this is typically captured \emph{via} some form of bisimulation.

% The notion we use in this paper is weak barbed bisimulation
% \cite{milner91polyadicpi}.

The notion we use in this paper is derived from weak barbed
bisimulation \cite{milner91polyadicpi}. 

\begin{definition}
An \emph{observation relation}, $\downarrow_{\mathcal N}$, over a set
of names, $\mathcal N$, is the smallest relation satisfying the rules
below.

\infrule[Out-barb]{y \in {\mathcal N}, \; x \nameeq y}
		  {\outputp{x}{v} \downarrow_{\mathcal N} x}
\infrule[Par-barb]{\mbox{$P\downarrow_{\mathcal N} x$ or $Q\downarrow_{\mathcal N} x$}}
		  {\binpar{P}{Q} \downarrow_{\mathcal N} x}

We write $P \Downarrow_{\mathcal N} x$ if there is $Q$ such that 
$P \wred Q$ and $Q \downarrow_{\mathcal N} x$.
\end{definition}

\begin{definition}
%\label{def.bbisim}
An  ${\mathcal N}$-\emph{barbed bisimulation} over a set of names, ${\mathcal N}$, is a symmetric binary relation 
${\mathcal S}_{\mathcal N}$ between agents such that $P\rel{S}_{\mathcal N}Q$ implies:
\begin{enumerate}
\item If $P \red P'$ then $Q \wred Q'$ and $P'\rel{S}_{\mathcal N} Q'$.
\item If $P\downarrow_{\mathcal N} x$, then $Q\Downarrow_{\mathcal N} x$.
\end{enumerate}
$P$ is ${\mathcal N}$-barbed bisimilar to $Q$, written
$P \wbbisim_{\mathcal N} Q$, if $P \rel{S}_{\mathcal N} Q$ for some ${\mathcal N}$-barbed bisimulation ${\mathcal S}_{\mathcal N}$.
\end{definition}

$\mathcal{R} \subseteq \pi \times \pi$

$P \mathcal{R} Q => \forall P'. P \red P' \Rightarrow \exists Q'. Q \red Q', P' \mathcal{R} Q'$

$P \vdash x \Rightarrow Q \vdash x$

\begin{mathpar}
  \inferrule*[lab=Out-barb]{x \nameeq y}{{y}!\langle{Q}\rangle \vdash x}
  \and
  \inferrule*[lab=Par-barb]{\mbox{$P\vdash x$ or $Q\vdash x$}}{\binpar{P}{Q} \vdash x}
\end{mathpar}

\subsubsection{Contexts}

One of the principle advantages of computational calculi like the
$\pi$-calculus is a well-defined notion of context,
contextual-equivalence and a correlation between
contextual-equivalence and notions of bisimulation. The notion of
context allows the decomposition of a process into (sub-)process and
its syntactic environment, its context. Thus, a context may be
thought of as a process with a ``hole'' (written $\Box$) in it. The
application of a context $M$ to a process $P$, written $M[P]$, is
tantamount to filling the hole in $M$ with $P$. In this paper we do
not need the full weight of this theory, but do make use of the notion
of context in the proof the main theorem. 

\begin{mathpar}
  \inferrule* [lab=summation] {} {{M_{M},M_{N}} \bc \Box \;|\; x.M_{A} \;|\; M_{M}+M_{N}}
  \and
  \inferrule* [lab=agent] {} {{M_{A}} \bc (\vec{x})M_{P} \;| \; \clift{P_0,\ldots,M_{P},\ldots,P_N}}
  \and \\
  \inferrule* [lab=process] {} {{M_{P}} \bc M_{N} \;| \;P|M_{P} }
\end{mathpar} 

\begin{mathpar}
  \inferrule* [lab=sychronization] {} {M_{N} \bc \Box \;|\; x?M_{F} \;|\; x!M_{C}}
  \and
  \inferrule* [lab=abstraction] {} {{M_{F}} \bc (x)M_{P} }
  \and
  \inferrule* [lab=concretion] {} {{M_{C}} \bc \langle M_{P} \rangle }
  \and \\
  \inferrule* [lab=process] {} {{M_{P}} \bc M_{N} \;| \;P|M_{P} }
\end{mathpar}

\begin{definition}[contextual application] Given a context $M$, and
  process $P$, we define the \emph{contextual application}, $M[P] :=
  M\{P/\Box\}$. That is, the contextual application of M to P is the
  substitution of $P$ for $\Box$ in $M$.
\end{definition}

$\meaningof{-} : L \to \mathcal{P}(\pi)$

\begin{mathpar}
  \inferrule* [lab=collection] {} {\meaningof{true} = \pi, \and \meaningof{~E} = \pi \setminus \meaningof{E}, \and \meaningof{E_{1} \& E_{2}} = \meaningof{E_{1}} \cap \meaningof{E_{2}}}
\end{mathpar}

\begin{mathpar}
  \inferrule* [lab=structure] {} {\meaningof{0} = \{ P \in \pi | P \equiv 0 \}, \and \\ \meaningof{E_1 | E_2} = \{ P \in \pi | P \equiv P_{1} | P_{2}, P_{1} \in \meaningof{E_{1}}, P_{2} \in \meaningof{E_2}\} }
\end{mathpar}

\begin{mathpar}
 \inferrule* [lab=behavior] {} {\meaningof{\langle a?b \rangle E} = \{ P \in \pi | P \equiv Q | u?(y)P', \\ \and \\\\ \and \\ \;\;\; u \in \meaningof{a}, \forall z.P'\{z/y\} \in \meaningof{E\{z/b\}}\}, \and \\ \meaningof{a!E} = \{ P \in \pi | P \equiv Q | x!\langle P' \rangle, x \in \meaningof{a} P' \in \meaningof{E}\} }
\end{mathpar}

\begin{mathpar}
 \inferrule* [lab=nominal] {} {\meaningof{\quotep{E}} = \{ \quotep{P} \in \quotep{\pi} | P \in \meaningof{E} \}, \and \meaningof{\quotep{P}} = \{ \quotep{Q} \in \quotep{\pi} | P \equiv Q \} \and \\ \meaningof{@\quotep{E}} = \{ P \in \pi | P \equiv @x, x \in \meaningof{E} \}}
\end{mathpar}

\begin{eqnarray*}
  \\
  \meaningof{-} : TS \to ST
\end{eqnarray*}

\begin{eqnarray*}
  \\
  L : TS \to ST
\end{eqnarray*}

\begin{eqnarray*}
  \\
  P \models E \iff P \in \meaningof{E}
\end{eqnarray*}

\begin{eqnarray*}
  P \approx_{L} Q \iff \forall E \in L. P \models E \iff Q \models E
\end{eqnarray*}

\begin{eqnarray*}
  P \approx_{K} Q
\end{eqnarray*}

\begin{eqnarray*}
  P \approx Q
\end{eqnarray*}

$\approx_{K} = \approx = \approx_{L}$

\subsubsection{Contextual duality}

Note that contexts extend the quotation operation to a family of
operations from processes to names. Given a context, $M$, we can
define a \emph{nominal context}, $\quotep{M}$ by $\quotep{M}[P] :=
\quotep{M[P]}$. To foreshadow what is to come we observe that these
operations enjoy a duality with processes very much like the duality
between vectors and maps from vectors to scalars.

Further, because the calculus is essentially higher-order, we have a
correspondence between contexts and processes. More specifically,
given a name $x$ and a context $M$ we can construct $M^{*}_{x}$ such
that 

\begin{mathpar}
  M^{*}_{x} | \lift{x}{P} \red M[P]
\end{mathpar}

namely,

\begin{mathpar}
  M^{*}_{x} := x?(u).M[\dropn{u}]
\end{mathpar}

The dependence of $M^{*}_{x}$ on a name makes it an abstraction, 

\begin{mathpar}
  M^{*} := (x)x?(u).M[\dropn{u}]
\end{mathpar}

\subsection{Additional notation}

It will sometimes be convenient to denote the process a name
quotes. We already have the notation $x = \quotep{P}$, but it will be
convenient to introduce an alternate notation, $\procn{x}$, when we
want to emphasize the connection to the use of the name. Note that, by
virtue of name equivalence, $\quotep{\procn{x}} \nameeq x$; so, the
notation is consistent with previous definitions.

Further, because names have structure it is possible to effect
substitutions on the basis of that structure. This means we need to
upgrade our notation for substitutions, which we accomplish by
adapting comprehension notation. Thus,

\begin{mathpar}
  P\{ y / x : x \in S \}
\end{mathpar}

is interpreted to mean the process derived from P by replacing (in a
capture-avoiding manner) each occurrence of $x$ in $S$ by $y$. For example,

\begin{mathpar}
  P\{ \quotep{\procn{x}|\procn{x}} / x : x \in \freenames{P} \}
\end{mathpar}

will replace each (occurrence) of a free name $x$ in $P$ by
$\quotep{\procn{x}|\procn{x}}$.

Also, we will avail ourselves of the notation $x^{L}$ and $x^{R}$ to
denote injections of a name into disjoint copies of the name
space. There are numerous ways to accomplish this. One example can be
found in \cite{MeredithR05}. This notation overloads to vectors of
names: $\vec{x}^{\pi} := (x_{i}^{\pi} \; : \; 0 \leq i < |\vec{x}| )$ where $\pi \in \{L,R\}$.

We also use $P^{\Box} := P|\Box$.

In \cite{MeredithR05} an interpretation of the new operator is
given. It turns out that there are several possible interpretations
all enjoying the requisite algebraic properties of the operator (see
\cite{milner91polyadicpi}). We will therefore make liberal use of
$(\nu\; \vec{x})P$.

% subsection the_syntax_and_semantics_of_the_notation_system (end)   

\input{qm2pi.qmops} 

\input{qm2pi.sterngerlach} 

\input{qm2pi.metric} 

% section concurrent_process_calculi (end)

%\input{qm2pi.proofsketch}

% section proof sketch (end)

%\input{qm2pi.slviaknots} 

% section spatial logic via knots (end)

\input{qm2pi.conclusion}

% section conclusion (end)

%\input{qm2pi.dtcodes} 

% section wiring algorithm (end)

\input{qm2pi.ack} 

% section acknowledgments (end)

\newpage


\bibliographystyle{plain}   
\bibliography{../../biblios/main.bib}

\input{qm2pi.rhodetails}

\end{document}

 

% subsection basic_interpretation (end)

%\input{qm2pi.rho.presentation} 
\subsection{The syntax and semantics of the notation system}\label{sub:the_syntax_and_semantics_of_the_notation_system} % (fold)

We now summarize a technical presentation of the calculus that
embodies our theory of dynamics. The typical presentation of such a
calculus follows the style of giving generators and relations on
them. The grammar, below, describing term constructors, freely
generates the set of processes, $\Proc$. This set is then quotiented
by a relation known as structural congruence and it is over this set
that the notion of dynamics is expressed. This presentation is
essentially that of \cite{MeredithR05} with the addition of
polyadicity and summation. For readability we have relegated some of
the technical subtleties to an appendix.

\subsubsection{Process grammar}\label{subsub:process_grammar}

\begin{mathpar}
  \inferrule* [lab=synchronization] {} {{M} \bc \pzero \;|\; x?F \;|\; x!C }
  \and
  \inferrule* [lab=abstraction] {} {{F} \bc (x)P}
  \and
  \inferrule* [lab=concretion] {} {{C} \bc \langle Q \rangle}
  \and
  \inferrule* [lab=process] {} {{P,Q} \bc M \;| \;P|Q \;|\; @{x}}
  \and
  \inferrule* [lab=name] {} {{x} \bc \quotep{P}}
\end{mathpar} 

Note that $\vec{x}$ (resp. $\vec{P}$) denotes a vector of names
(resp. processes) of length $|\vec{x}|$ (resp. $|\vec{P}|$). We adopt
the following useful abbreviations.

\begin{mathpar}
   x?(\vec{y}).P := x.(\vec{y})P \and  x\clift{\vec{P}} := x.\clift{\vec{P}}
   \and x!(y) := \lift{x}{\dropn{y}}
   \and \Pi_{i=0}^{n-1}P_i := P_0 | \ldots | P_{n-1}
\end{mathpar}

\subsubsection{Structural congruence}

\paragraph{Free and bound names and alpha-equivalence.} At the
core of structural equivalence is alpha-equivalence which identifies
process that are the same up to a change of variable. Formally, we
recognize the distinction between free and bound names. The free names
of a process, $\freenames{P}$, may be calculated recursively as
follows:

\begin{mathpar}
\freenames{\pzero} := \emptyset
  \and \\
  \freenames{x?(y).P} := \{ x \} \cup (\freenames{P} \setminus \{ y \})
  \and 
  \freenames{x!\langle P \rangle} := \{ x \} \cup \{ P \} 
  \and \\
  \freenames{P|Q} := \freenames{P} \cup \freenames{Q}
  \and \\
  \freenames{@{x}} := \{ x \}
\end{mathpar}

$\pi$
$\quotep{\pi}$

$\freenames{-} : \pi \to \mathcal{P}(\quotep{\pi})$

\begin{eqnarray*}
  \freenames{\pzero} & := & \emptyset \\
  \freenames{x?(y).P} & := & \{ x \} \cup (\freenames{P} \setminus \{ y \}) \\
  \freenames{x!\langle P \rangle} & := & \{ x \} \cup \{ P \} \\
  \freenames{P|Q} & := & \freenames{P} \cup \freenames{Q} \\
  \freenames{\dropn{x}} & := & \{ x \}
\end{eqnarray*}

The bound names of a process, $\boundnames{P}$, are those names occurring in $P$
that are not free. For example, in $x?(y).0$, the name $x$ is free, while $y$ is bound.

\begin{mathpar}
  \inferrule* [lab=monoidal-laws] {} { P|Q \equiv Q|P \and P|0 \equiv P \and P|(Q|R) \equiv (P|Q)|R }
\end{mathpar}

\begin{mathpar}
  \inferrule* [lab=alpha-equivalence] {} { (x)P \equiv (y)P\{y/x\} \and y \not\in \freenames{P} }
\end{mathpar}

\begin{definition}
Then two processes, $P,Q$, are alpha-equivalent if $P = Q\{\vec{y}/\vec{x}\}$ for
some $\vec{x} \in \boundnames{Q},\vec{y} \in \boundnames{P}$, where $Q\{\vec{y}/\vec{x}\}$
denotes the capture-avoiding substitution of $\vec{y}$ for $\vec{x}$ in $Q$.
\end{definition}

\begin{definition}
  The {\em structural congruence} \cite{SangiorgiWalker} , $\equiv$,
  between processes is the least congruence containing
  alpha-equivalence, satisfying the abelian monoid laws
  (associativity, commutativity and $\pzero$ as identity) for parallel
  composition $|$ and for summation $+$.
\end{definition}

\subsection{Name equivalence}

We take name equivalence, written $\nameeq$, to be the smallest
equivalence relation generated by the following rules.

\begin{mathpar}
\inferrule*[lab=Quote-drop]
{ }
{ \quotep{@{x}} \nameeq x }

\inferrule*[lab=Struct-equiv]
{ P \scong Q }
{ \quotep{P} \nameeq \quotep{Q} }
\end{mathpar}

The astute reader will have noticed that the mutual recursion of names
and processes imposes a mutual recursion on alpha-equivalence and
structural equivalence via name-equivalence. Fortunately, all of this
works out pleasantly and we may calculate in the natural way, free of
concern. The reader interested in the details is referred to the
appendix \ref{appendix:rho_details}.

\subsection{Substitution}

We use $\Proc$ for the set of processes, $\QProc$ for the set of
names, and $\id{\{}\vec{y} / \vec{x} \id{\}}$ to denote partial maps,
$s : \QProc \rightarrow \QProc$. A map, $s$ lifts, uniquely, to a map
on process terms, $\widehat{s} : \Proc \rightarrow \Proc$ by the
following equations.

\begin{mathpar}
  (0) \psubstp{Q}{P} := 0 \\
  (R \juxtap S) \psubstp{Q}{P}
  :=    
  (R)\psubstp{Q}{P} \juxtap (S) \psubstp{Q}{P} \\
  (x?(y).R) \psubstp{Q}{P}    
  :=    
  (x)\substp{Q}{P} (z)\concat( (R \psubstn{z}{y}) \psubstp{Q}{P} ) \\
  (\lift{x}{R}) \psubstp{Q}{P}  
  :=
  \lift{(x)\substp{Q}{P}}{ R \psubstp{Q}{P} } \\
%   (\dropn{x})  \psubstp{Q}{P}       
%   := 
%   \left\{ 
%     \begin{array}{ccc} 
%       \dropn{\quotep{Q}} & & x \nameeq \quotep{P} \\
%       \dropn{x} & & otherwise \\
%     \end{array}
%   \right. 
  (\dropn{x})  \psubstp{Q}{P}       
  := 
  \left\{ 
    \begin{array}{ccc} 
      Q & & x \nameeq \quotep{P} \\
      \dropn{x} & & otherwise \\
    \end{array}
  \right.
\end{mathpar}
 

where

\begin{eqnarray}
  (x)\id{\{} \lpquote Q \rpquote / \lpquote P \rpquote \id{\}}            = 
  \left\{ 
    \begin{array}{ccc}
      \lpquote Q \rpquote & & x \nameeq \lpquote P \rpquote \\
      x & & otherwise \\
    \end{array}
  \right. \nonumber
\end{eqnarray}

and $z$ is chosen distinct from $\quotep{P}$, $\quotep{Q}$, the free
names in $Q$, and all the names in $R$. Our $\alpha$-equivalence will
be built in the standard way from this substitution.

\begin{remark}\label{rem:no_self_referential_names}
  One consequence of these definitions is that $\forall P. \quotep{P}
  \not\in \freenames{P}$.
\end{remark}

\subsection{ Dynamic quote: an example }

Anticipating something of what's to come, consider applying the
substitution, $\widehat{\id{\{}u / z \id{\}}}$, to the following pair
of processes, $\lift{w}{y!(z)}$ and $w[ \lpquote y!(z) \rpquote ]$.

\begin{eqnarray}
	\lift{w}{y!(z)}\widehat{\id{\{}u / z \id{\}}}
		& = &
		\lift{w}{y!(u)} \nonumber\\
	w[ \lpquote y!(z) \rpquote ] \widehat{ \id{\{}u / z \id{\}} }
		& = &
		w[ \lpquote y!(z) \rpquote ] \nonumber
\end{eqnarray}

Because the body of the process between quotes is impervious to
substitution, we get radically different answers. In fact, by
examining the first process in an input context,
e.g. $x?(z).\lift{w}{y!(z)}$, we see that the process under the lift
operator may be shaped by prefixed inputs binding a name inside it. In
this sense, the lift operator will be seen as a way to dynamically
construct processes before reifying them as names.

Finally equipped with these standard features we can present the
dynamics of the calculus.

\subsubsection{Operational semantics} 

Finally, we introduce the computational dynamics. What marks these
algebras as distinct from other more traditionally studied algebraic
structures, e.g. vector spaces or polynomial rings, is the manner in
which dynamics is captured. In traditional structures, dynamics is typically
expressed through morphisms between such structures, as in linear maps
between vector spaces or morphisms between rings. In algebras
associated with the semantics of computation, the dynamics is
expressed as part of the algebraic structure itself, through a
reduction reduction relation typically denoted by $\red$. Below, we
give a recursive presentation of this relation for the calculus used
in the encoding.

$\red \subseteq \pi \times \pi$
$\red : \pi \to \mathcal{P}(\pi)$

\begin{mathpar}
  \inferrule* [lab=Comm] { \textsf{match}( x_{src}, x_{trgt} ) } { x_{trgt}?(y)P \; | \; x_{src}!\langle {Q} \rangle \red P\{\quotep{Q}/y}\} }
  \and \\
  \inferrule* [lab=Par] {{P} \red {P}'} {{{P} | {Q}} \red {{P}' | {Q}}}
  \and
  \inferrule* [lab=Equiv]{{{P} \scong {P}'} \andalso {{P}' \red {Q}'} \andalso {{Q}' \scong {Q}}}{{P} \red {Q}}
\end{mathpar}

\begin{eqnarray*}
  match_{\equiv} (\quotep{P},\quotep{Q}) & := & P \equiv Q \\
  match_{\dagger}(\quotep{P},\quotep{Q}) & := & \forall R. P|Q \red^{*} R => R \red^{*} 0 \\
  match_{K}(\quotep{P},\quotep{Q}) & := & K \mbox{ for some context } K
\end{eqnarray*}

$u?(x)P | u!\langle Q \rangle \red P\{\quotep{Q}/x\}$

%We write $\wred$ for $\red^*$, and $P\red$ if $\exists Q $ such that $ P \red Q$.
We write $P\red$ if $\exists Q $ such that $ P \red Q$ and $P\not\red$, otherwise.

\section{Replication}

As mentioned before, it is known that replication (and hence
recursion) can be implemented in a higher-order process algebra
\cite{SangiorgiWalker}. As our first example of calculation with the
machinery thus far presented we give the construction explicitly in
the {\rhoc}.

\begin{eqnarray}
	D_{x} & := & \prefix{x}{y}{(\binpar{\outputp{x}{y}}{@{y}})} \nonumber\\
	\bangp_{x}{P} & := & \binpar{{x}!\langle{\binpar{D_{x}}{P}}\rangle}{D_{x}} \nonumber
\end{eqnarray}

\begin{eqnarray}
	\bangp_{x}{P} & & \nonumber\\
	=
	& {x}!\langle{(\prefix{x}{y}{(\outputp{x}{y} | @{y})) | P}}\rangle 
	      | \prefix{x}{y}{(\outputp{x}{y} | @{y})} & \nonumber\\
	\red
	& (\outputp{x}{y} | @{y})\substn{\quotep{(\prefix{x}{y}{(@{y} | \outputp{x}{y})) | P}}}{y} & \nonumber\\
	=
	& \outputp{x}{\quotep{(\prefix{x}{y}{(\outputp{x}{y} | @{y})) | P}}}
	  | {(\prefix{x}{y}{(\outputp{x}{y} | @{y})) | P}} & \nonumber\\
	\red
	& \ldots & \nonumber\\
	\red^*
	& P | P | \ldots & \nonumber
\end{eqnarray}

Of course, this encoding, as an implementation, runs away, unfolding
$\bangp{P}$ eagerly. A lazier and more implementable replication
operator, restricted to input-guarded processes, may be obtained as follows.

\begin{eqnarray}
\bangp{\prefix{u}{v}{P}} 
	:= 
	\binpar{\lift{x}{\prefix{u}{v}{(\binpar{D(x)}{P})}}}{D(x)} \nonumber
\end{eqnarray}

\begin{remark}
  Note that the lazier definition still does not deal with summation
  or mixed summation (i.e. sums over input and output). The reader is
  invited to construct definitions of replication that deal with these
  features. 

  Further, the definitions are parameterized in a name, $x$. Can you,
  gentle reader, make a definition that eliminates this parameter and
  guarantees no accidental interaction between the replication
  machinery and the process being replicated -- i.e. no accidental
  sharing of names used by the process to get its work done and the
  name(s) used by the replication to effect copying. This latter
  revision of the definition of replication is crucial to obtaining
  the expected identity $!!P \sim !P$.
\end{remark}

\begin{remark}\label{rem:paradoxical_combinator}
  The reader familiar with the lambda calculus will have noticed the
  similarity between $D$ and the paradoxical combinator.

  [Ed. note: the existence of this seems to suggest we have to be more
  restrictive on the set of processes and names we admit if we are to
  support no-cloning.]
\end{remark}

\subsubsection{Bisimulation}

The computational dynamics gives rise to another kind of equivalence,
the equivalence of computational behavior. As previously mentioned
this is typically captured \emph{via} some form of bisimulation.

% The notion we use in this paper is weak barbed bisimulation
% \cite{milner91polyadicpi}.

The notion we use in this paper is derived from weak barbed
bisimulation \cite{milner91polyadicpi}. 

\begin{definition}
An \emph{observation relation}, $\downarrow_{\mathcal N}$, over a set
of names, $\mathcal N$, is the smallest relation satisfying the rules
below.

\infrule[Out-barb]{y \in {\mathcal N}, \; x \nameeq y}
		  {\outputp{x}{v} \downarrow_{\mathcal N} x}
\infrule[Par-barb]{\mbox{$P\downarrow_{\mathcal N} x$ or $Q\downarrow_{\mathcal N} x$}}
		  {\binpar{P}{Q} \downarrow_{\mathcal N} x}

We write $P \Downarrow_{\mathcal N} x$ if there is $Q$ such that 
$P \wred Q$ and $Q \downarrow_{\mathcal N} x$.
\end{definition}

\begin{definition}
%\label{def.bbisim}
An  ${\mathcal N}$-\emph{barbed bisimulation} over a set of names, ${\mathcal N}$, is a symmetric binary relation 
${\mathcal S}_{\mathcal N}$ between agents such that $P\rel{S}_{\mathcal N}Q$ implies:
\begin{enumerate}
\item If $P \red P'$ then $Q \wred Q'$ and $P'\rel{S}_{\mathcal N} Q'$.
\item If $P\downarrow_{\mathcal N} x$, then $Q\Downarrow_{\mathcal N} x$.
\end{enumerate}
$P$ is ${\mathcal N}$-barbed bisimilar to $Q$, written
$P \wbbisim_{\mathcal N} Q$, if $P \rel{S}_{\mathcal N} Q$ for some ${\mathcal N}$-barbed bisimulation ${\mathcal S}_{\mathcal N}$.
\end{definition}

$\mathcal{R} \subseteq \pi \times \pi$

$P \mathcal{R} Q => \forall P'. P \red P' \Rightarrow \exists Q'. Q \red Q', P' \mathcal{R} Q'$

$P \vdash x \Rightarrow Q \vdash x$

\begin{mathpar}
  \inferrule*[lab=Out-barb]{x \nameeq y}{{y}!\langle{Q}\rangle \vdash x}
  \and
  \inferrule*[lab=Par-barb]{\mbox{$P\vdash x$ or $Q\vdash x$}}{\binpar{P}{Q} \vdash x}
\end{mathpar}

\subsubsection{Contexts}

One of the principle advantages of computational calculi like the
$\pi$-calculus is a well-defined notion of context,
contextual-equivalence and a correlation between
contextual-equivalence and notions of bisimulation. The notion of
context allows the decomposition of a process into (sub-)process and
its syntactic environment, its context. Thus, a context may be
thought of as a process with a ``hole'' (written $\Box$) in it. The
application of a context $M$ to a process $P$, written $M[P]$, is
tantamount to filling the hole in $M$ with $P$. In this paper we do
not need the full weight of this theory, but do make use of the notion
of context in the proof the main theorem. 

\begin{mathpar}
  \inferrule* [lab=summation] {} {{M_{M},M_{N}} \bc \Box \;|\; x.M_{A} \;|\; M_{M}+M_{N}}
  \and
  \inferrule* [lab=agent] {} {{M_{A}} \bc (\vec{x})M_{P} \;| \; \clift{P_0,\ldots,M_{P},\ldots,P_N}}
  \and \\
  \inferrule* [lab=process] {} {{M_{P}} \bc M_{N} \;| \;P|M_{P} }
\end{mathpar} 

\begin{mathpar}
  \inferrule* [lab=sychronization] {} {M_{N} \bc \Box \;|\; x?M_{F} \;|\; x!M_{C}}
  \and
  \inferrule* [lab=abstraction] {} {{M_{F}} \bc (x)M_{P} }
  \and
  \inferrule* [lab=concretion] {} {{M_{C}} \bc \langle M_{P} \rangle }
  \and \\
  \inferrule* [lab=process] {} {{M_{P}} \bc M_{N} \;| \;P|M_{P} }
\end{mathpar}

\begin{definition}[contextual application] Given a context $M$, and
  process $P$, we define the \emph{contextual application}, $M[P] :=
  M\{P/\Box\}$. That is, the contextual application of M to P is the
  substitution of $P$ for $\Box$ in $M$.
\end{definition}

$\meaningof{-} : L \to \mathcal{P}(\pi)$

\begin{mathpar}
  \inferrule* [lab=collection] {} {\meaningof{true} = \pi, \and \meaningof{~E} = \pi \setminus \meaningof{E}, \and \meaningof{E_{1} \& E_{2}} = \meaningof{E_{1}} \cap \meaningof{E_{2}}}
\end{mathpar}

\begin{mathpar}
  \inferrule* [lab=structure] {} {\meaningof{0} = \{ P \in \pi | P \equiv 0 \}, \and \\ \meaningof{E_1 | E_2} = \{ P \in \pi | P \equiv P_{1} | P_{2}, P_{1} \in \meaningof{E_{1}}, P_{2} \in \meaningof{E_2}\} }
\end{mathpar}

\begin{mathpar}
 \inferrule* [lab=behavior] {} {\meaningof{\langle a?b \rangle E} = \{ P \in \pi | P \equiv Q | u?(y)P', \\ \and \\\\ \and \\ \;\;\; u \in \meaningof{a}, \forall z.P'\{z/y\} \in \meaningof{E\{z/b\}}\}, \and \\ \meaningof{a!E} = \{ P \in \pi | P \equiv Q | x!\langle P' \rangle, x \in \meaningof{a} P' \in \meaningof{E}\} }
\end{mathpar}

\begin{mathpar}
 \inferrule* [lab=nominal] {} {\meaningof{\quotep{E}} = \{ \quotep{P} \in \quotep{\pi} | P \in \meaningof{E} \}, \and \meaningof{\quotep{P}} = \{ \quotep{Q} \in \quotep{\pi} | P \equiv Q \} \and \\ \meaningof{@\quotep{E}} = \{ P \in \pi | P \equiv @x, x \in \meaningof{E} \}}
\end{mathpar}

\begin{eqnarray*}
  \\
  \meaningof{-} : TS \to ST
\end{eqnarray*}

\begin{eqnarray*}
  \\
  L : TS \to ST
\end{eqnarray*}

\begin{eqnarray*}
  \\
  P \models E \iff P \in \meaningof{E}
\end{eqnarray*}

\begin{eqnarray*}
  P \approx_{L} Q \iff \forall E \in L. P \models E \iff Q \models E
\end{eqnarray*}

\begin{eqnarray*}
  P \approx_{K} Q
\end{eqnarray*}

\begin{eqnarray*}
  P \approx Q
\end{eqnarray*}

$\approx_{K} = \approx = \approx_{L}$

\subsubsection{Contextual duality}

Note that contexts extend the quotation operation to a family of
operations from processes to names. Given a context, $M$, we can
define a \emph{nominal context}, $\quotep{M}$ by $\quotep{M}[P] :=
\quotep{M[P]}$. To foreshadow what is to come we observe that these
operations enjoy a duality with processes very much like the duality
between vectors and maps from vectors to scalars.

Further, because the calculus is essentially higher-order, we have a
correspondence between contexts and processes. More specifically,
given a name $x$ and a context $M$ we can construct $M^{*}_{x}$ such
that 

\begin{mathpar}
  M^{*}_{x} | \lift{x}{P} \red M[P]
\end{mathpar}

namely,

\begin{mathpar}
  M^{*}_{x} := x?(u).M[\dropn{u}]
\end{mathpar}

The dependence of $M^{*}_{x}$ on a name makes it an abstraction, 

\begin{mathpar}
  M^{*} := (x)x?(u).M[\dropn{u}]
\end{mathpar}

\subsection{Additional notation}

It will sometimes be convenient to denote the process a name
quotes. We already have the notation $x = \quotep{P}$, but it will be
convenient to introduce an alternate notation, $\procn{x}$, when we
want to emphasize the connection to the use of the name. Note that, by
virtue of name equivalence, $\quotep{\procn{x}} \nameeq x$; so, the
notation is consistent with previous definitions.

Further, because names have structure it is possible to effect
substitutions on the basis of that structure. This means we need to
upgrade our notation for substitutions, which we accomplish by
adapting comprehension notation. Thus,

\begin{mathpar}
  P\{ y / x : x \in S \}
\end{mathpar}

is interpreted to mean the process derived from P by replacing (in a
capture-avoiding manner) each occurrence of $x$ in $S$ by $y$. For example,

\begin{mathpar}
  P\{ \quotep{\procn{x}|\procn{x}} / x : x \in \freenames{P} \}
\end{mathpar}

will replace each (occurrence) of a free name $x$ in $P$ by
$\quotep{\procn{x}|\procn{x}}$.

Also, we will avail ourselves of the notation $x^{L}$ and $x^{R}$ to
denote injections of a name into disjoint copies of the name
space. There are numerous ways to accomplish this. One example can be
found in \cite{MeredithR05}. This notation overloads to vectors of
names: $\vec{x}^{\pi} := (x_{i}^{\pi} \; : \; 0 \leq i < |\vec{x}| )$ where $\pi \in \{L,R\}$.

We also use $P^{\Box} := P|\Box$.

In \cite{MeredithR05} an interpretation of the new operator is
given. It turns out that there are several possible interpretations
all enjoying the requisite algebraic properties of the operator (see
\cite{milner91polyadicpi}). We will therefore make liberal use of
$(\nu\; \vec{x})P$.

% subsection the_syntax_and_semantics_of_the_notation_system (end)   

\section{Interpretation of QM}
\subsection{Supporting definitions}
\subsubsection{Multiplication}
\begin{mathpar}
  \quotep{Q} \cdot \quotep{R} := \quotep{Q|R}
  \and \\
  \quotep{Q} \cdot P := P\{ \quotep{Q|R} / \quotep{R} : \quotep{R} \in \freenames{P} \}
\end{mathpar}

\paragraph{Discussion}
The first line needs little explanation. The second line says that
each free name of the process is replaced with the multiplication of
that name by the scalar. Multiplication of a scalar (name) by a state
(process) results in a process all the names of which have been `moved
over' by parallel composition with the process the scalar
quotes. There is a subtlety that the bound names have to be
manipulated so that multiplied names aren't accidentally
captured. There are many ways to achieve this.

\begin{remark}\label{rem:multiplication_identities}
  The reader is invited to verify that for all $x,y,z \in \QProc$ and $P \in \Proc$
  \begin{mathpar}
    x \cdot \quotep{0} \equiv x 
    \and
    x \cdot y \equiv y \cdot x
    \and
    x \cdot (y \cdot z) \equiv (x \cdot y) \cdot z
    \and \\
    \quotep{0} \cdot P \equiv P
    \and \\
    x \cdot (y \cdot P) \equiv (x \cdot y) \cdot P
    \and \\
    x \cdot (P|Q) \equiv (x \cdot P) | (x \cdot Q)
    \and \\    
  \end{mathpar}
\end{remark}

\subsubsection{Tensor product}

We define a tensor product on processes by structural induction.

\paragraph{Tensor of sums} First note that all summations, including
$\pzero$ and sequence, can be written $\Sigma_{i} x_{i}.A_{i} +
\Sigma_{j} x_{j}.C_{j}$, where we have grouped input-guarded processes
together and output-guarded processes together.

Thus, we can define the tensor product of two summations, $N_{1}\otimes N_{2}$, where

\begin{mathpar}
  N_{1} := \Sigma_{i} x_{i}.A_{i} + \Sigma_{j} x_{j}.C_{j}
  \and
  N_{2} := \Sigma_{i'} y_{i'}.B_{i'} + \Sigma_{j'} y_{j'}.D_{j'} 
\end{mathpar}

as follows.

\begin{mathpar}
  \Sigma_{i} x_{i}.A_{i} + \Sigma_{j} x_{j}.C_{j} \otimes \Sigma_{i'}
  y_{i'}.B_{i'} + \Sigma_{j'} y_{j'}.D_{j'} 
  \and \\
  := \; \Sigma_{i} \Sigma_{i'} \quotep{\stackrel{\vee}{x_{i}}| \stackrel{\vee}{y_{i'}}}.(A_{i}\otimes B_{i'}) \; | \; \Sigma_{i'} \Sigma_{i} \quotep{\stackrel{\vee}{y_{i'}}|\stackrel{\vee}{x_{i}}}.(B_{i'}\otimes A_{i})
  \and
  \;\; | \;\; \Sigma_{j} \Sigma_{j'} \quotep{\stackrel{\vee}{x_{j}}|\stackrel{\vee}{y_{j'}}}.(A_{j}\otimes B_{j'}) \; | \; \Sigma_{j'} \Sigma_{j} \quotep{\stackrel{\vee}{y_{j'}}|\stackrel{\vee}{x_{j}}}.(B_{j'}\otimes A_{j})
\end{mathpar}

\begin{remark}
  Do we need to $x^{L}$ and $y^{R}$ for this construction as well?
\end{remark}

\paragraph{Tensor of parallel compositions} Next, we distribute tensor
over par.

\begin{mathpar}
  P_{1}|P_{2} \otimes Q_{1}|Q_{2} := (P_{1} \otimes Q_{1}) | (P_{1}
  \otimes Q_{2}) | (P_{2} \otimes Q_{1}) | (P_{2} \otimes Q_{2})
\end{mathpar}

\paragraph{Tensor with dropped names} We treat tensor of a
process with a dropped name as parallel composition.

\begin{mathpar}
  P \otimes \dropn{x} := P | \dropn{x}
\end{mathpar}

\paragraph{Tensor of agents}

Finally, we need to define tensor on agents. Note that the definition
of tensor on normal products only tensors inputs with inputs and
outputs with outputs. Thus, we only have to define the operation on
``homogeneous'' pairings.

\begin{mathpar}
  (\vec{x})P \otimes (\vec{y})Q
  \and \\
  := (x_{0}^{L}|y_{0}^{R},\ldots,x_{0}^{L}|y_{n}^{R},\ldots,x_{m}^{L}|y_{0}^{R},\ldots,x_{m}^{L}|y_{n}^R)(P\{ \vec{x}^{L}/\vec{x}\} \otimes Q \{ \vec{y}^{R}/\vec{y}\})
  \and \\
  \clift{\vec{P}} \otimes \clift{\vec{Q}}
  \and \\
  := \clift{P_{0}\otimes Q_{0},\ldots,P_{0}\otimes Q_{n},\ldots,P_{m}\otimes Q_{0},\ldots,P_{m}\otimes Q_{n}}
\end{mathpar}

\begin{remark}
  Observe that arities of tensored abstractions matches arities of
  tensored concretions if the original arities matched. Note also that
  the length of the arities corresponds to the increase in dimension
  we see in ordinary vector space tensor product.
\end{remark}

\begin{remark}
  Operationally, this definition distributes the tensor down to
  components ``linked'' by summation. Tensor over summation is
  intriguing in that it mixes names. Moreover, as a consequence of the
  way it mixes names we have the identities for all $x \in \QProc$ and
  $P,Q \in \Proc$

  \begin{mathpar}
    (x \cdot P) \otimes Q \equiv x \cdot (P \otimes Q) \equiv P \otimes (x \cdot Q)
    \and
    P \otimes \pzero \equiv P
  \end{mathpar}

  that the reader is invited to verify.
\end{remark}

\subsubsection{Annihilation}
\begin{mathpar}
  P^{\perp} := \{ Q | \forall R. P|Q \red^{*} R \Rightarrow R \red^{*} \pzero \}
  \and \\
  P^{\underline{\perp}} := \Sigma_{Q \in P^{\perp}} \quotep{Q}?(y).(\dropn{y}|Q) | \Sigma_{Q \in P^{\perp}} \quotep{Q}\clift{\Box}
\end{mathpar}

\paragraph{Discussion} The reader will note that $P^{\perp}$ is a
\emph{set} of processes, while $P^{\underline{\perp}}$ is a
\emph{context}. We call the set $P^{\perp}$ the \emph{annihilators} of
$P$. The parallel composition of a process in the annihilators of $P$
with $P$ will result in a process, the state space of which has all
paths eventually leading to $\pzero$. Execution may endure loops; but
under reasonable conditions of fairness (naturally guaranteed under
most notions of bisimulation) such a composite process cannot get
stuck in such a loop and will, eventually pop out and terminate.

The context $P^{\underline{\perp}}$ is ready and willing to ``take the
$P$ out of'' the process to which it is applied. It will effectively
transmit the code of the process to which it is applied to one of the
annihilators and run the process against it.

\subsubsection{Evaluation}
We fix $M$ a domain of fully abstract interpretation with an equality
coincident with bisimulation. We take $\meaningof{\cdot} : \Proc \to
M$ to be the map interpreting processes and $\nmeaningof{\cdot} : \M
\to Proc$ to be the map running the other way. Then we define

\begin{mathpar}
  \int P := \nmeaningof{\meaningof{P}}
\end{mathpar}

\paragraph{Discussion}
There are many fully abstract interpretations of Milner's
$\pi$-calculus. Any of them can be used as a basis for interpreting
the reflective calculus here. Equipped with such a domain it is
largely a matter of grinding through to check that the Yoneda
construction for the normalization-by-evaluation program can be
extended to this setting.

\begin{remark}
  The reader is invited to verify that $\int (P^{\underline{\perp}}[P]) = 0$.
\end{remark}

\subsection{Quantum mechanics}

Table \ref{tbl:core_qm_op_defns} gives the core operational definitions

\begin{table}[htp]\label{tbl:core_qm_op_defns}
  \center{
    \fbox{
      \begin{tabular}{c|c}
        quantum mechanics & process calculus \\
        \hline
        scalar & $x := \quotep{P}$ \\
        state vector & $\state{P} := P$ \\
        dual & $\state{P}^{*} := \event{P^{\underline{\perp}}} := \quotep{P^{\underline{\perp}}}[-]$ \\
        matrix & $ \Sigma_{\alpha} \state{P_{\alpha}}x_{\alpha}\event{Q_{\alpha}}$ \\
        vector addition & $\state{P} + \state{Q} := \state{P | Q}$ \\
        tensor product & $\state{P} \otimes \state{Q} := \state{P \otimes Q}$ \\
        inner product & $\innerprod{P}{Q} := \quotep{\int P^{\underline{\perp}}[Q]}$ \\
      \end{tabular}
    }
  }
  \caption{QM - operational definitions}
\end{table}

where

\begin{mathpar}
  \prmatrix{P}{Q} := \fprmatrix{P}{\quotep{\pzero}}{Q}
  \and
  \fprmatrix{P}{x}{Q} := (\state{P},x,\event{Q})
  \and
  (\fprmatrix{P}{x}{Q})(\state{R}) := x \cdot \innerprod{Q}{R} \cdot \state{P}
  \and
  (\fprmatrix{P}{x}{Q})(\event{R}) := x \cdot \innerprod{R}{P} \cdot \event{Q}
\end{mathpar}

\paragraph{Discussion}
As promised: vectors (aka states) are represented as processes; duals
as contextual duals; inner product definition should be compared with
standard inner product definition for ....

\begin{remark}
  Assuming $\int (P^{\underline{\perp}}[P]) = 0$, the reader is
  invited to verify that $(\fprmatrix{P}{x}{P})(\state{P}) = x \cdot \state{P}$.
\end{remark}

\begin{remark}
  The reader is invited to verify that $\innerprod{P}{Q}$ could
  equally well have been written $\quotep{\int \stackrel{\vee}{x}}$
  where $x = \event{P^{\underline{\perp}}}(Q)$.

  One of the motivations for this remark is that there is another way
  to factor these operations. We could package up evaluation in the dual:

  \begin{mathpar}
    \state{P}^{*} := \event{\int P^{\underline{\perp}}} := \quotep{\int P^{\underline{\perp}}}[-]
  \end{mathpar}

  and then have inner product defined by
  
  \begin{mathpar}
    \innerprod{P}{Q} := \event{P}(Q)
  \end{mathpar}

  Hopefully, experience with the calculations will provide guidance on
  the best factoring.
\end{remark}

\begin{remark}
  Assuming $\int (P^{\underline{\perp}}[P]) = 0$, the reader is
  invited to verify that $\forall P,Q. (\prmatrix{0}{Q})(\state{0}) =
  \state{0}$ and dually $(\prmatrix{P}{0})(\event{0}) = \event{0}$.
\end{remark}

\begin{remark}
  i'm a little worried that i don't (yet) have proper support for
  complex conjugacy. But, the observation above may give us a
  clue. According to Abramsky, it must be the case that the scalars
  are iso to the homset of the identity for the tensor -- which the
  observation above characterizes. 

  For now, we will simply bookmark the notion with $\overline{x}$.
\end{remark}

\subsubsection{Adjointness}

We need to give a definition of $(\cdot)^{\dagger}$ for matrices. The
obvious candidate definition is
\begin{mathpar}
(\Sigma_{\alpha}\fprmatrix{P_{\alpha}}{x_{\alpha}}{Q_{\alpha}})^{\dagger}
= \Sigma_{\alpha}\fprmatrix{(Q_{\alpha}^{\underline{\perp}})^{*}}{\overline{x}_{\alpha}}{P_{\alpha}^{\underline{\perp}}} 
\end{mathpar}

But, $(Q_{\alpha}^{\underline{\perp}})^{*}$ requires a name along
which to communicate the process to achieve the context application.

\subsubsection{Basis for a basis}
If processes label states and ``addition'' of states (a.k.a. vector
addition) is interpreted as parallel composition, what corresponds to
notions of linear independence and basis? Here, we recall that Yoshida
has developed a set of \emph{combinators} for an asynchronous verison
of Milner's $\pi$-calculus. These are a finite set of processes such
any process can be expressed as parallel composition of these
combinators together with liberal uses of the new operator and
replication. We can simply give a translation of these into the
present calculus and have reasonable expectation that the property
carries over. That is, that the resultant set allows to express all
processes via parallel composition. Note, however, that there is no
new operator or replication in this calculus. As a result, we expect
that the corresponding set is actually infinite. That is, we expect
that the space is actually infinite dimensional.

\begin{remark}
  The attentive reader may be a bit concerned. Certainly, the
  collection $S$, $K$ and $I$ is a finite set of
  combinators. Shouldn't we expect to see a finite set of combinators
  for an effectively equivalent system? i am very sympathetic to this
  critique and feel it warrants full attention. On the other hand, i
  also have in mind the following analogy. The natural numbers, as a
  monoid under addition, has exactly $1$ generator, while the natural
  numbers, as a monoid under multiplication, has countably many
  generators (the primes). We observe that the application of the
  lambda calculus is much less resource sensitive than the parallel
  composition of the $\pi$-calculus. Could it be the case that we have
  an analogy of the form
  
  \begin{mathpar}
    m + n : MN :: m*n : M|N
  \end{mathpar}

  giving a similar blow up in the set of ``primes''?  This is such a
  wonderful thought that, even if it's not true, i think it's worth
  writing down.
\end{remark}
 

\documentclass[12pt]{llncs}
%\documentclass{jktr}

\usepackage[pdftex]{hyperref}                   
\usepackage {listings}
\usepackage {mathpartir}
\usepackage{bcprules}
%\usepackage{listings}
                       
\usepackage{graphicx} 
%\usepackage[margins=2.5cm,nohead,nofoot]{geometry}
%\usepackage{geometry}
\usepackage{amsfonts}
\usepackage{amstext}
\usepackage{latexsym}
\usepackage{amssymb}
\usepackage{color}


%\include{myPreamble}
\include{qm2pi.local} 

%\ifpdf
%\usepackage[pdftex]{graphicx}
%\else
%\usepackage{graphicx}
%\fi

 % \ifpdf
%  \usepackage{pdfsync}
%  \if


%\title{Brief Article}
%\author{David F. Snyder}
%\author{L.G. Meredith}

%\address{Dept. of Math., Texas State University--San Marcos, San Marcos, TX 78666}
       
\pagestyle{empty}


\begin{document}

\lstset{language=[Objective]Caml,frame=shadowbox}

\input{qm2pi.front}

% section front matter (end)

\input{qm2pi.intro} 
 
% section introduction (end)

% \input{qm2pi.knotations} 

% section notation (end)

\input{qm2pi.process.calculi} 

% section concurrent_process_calculi_and_spatial_logics_ (end)
    
%\input{qm2pi.knots2pi} 

%\input{qm2pi.trefoil} 

%\input{qm2pi.mainthm} 

% subsection basic_interpretation (end)

%\input{qm2pi.rho.presentation} 
\subsection{The syntax and semantics of the notation system}\label{sub:the_syntax_and_semantics_of_the_notation_system} % (fold)

We now summarize a technical presentation of the calculus that
embodies our theory of dynamics. The typical presentation of such a
calculus follows the style of giving generators and relations on
them. The grammar, below, describing term constructors, freely
generates the set of processes, $\Proc$. This set is then quotiented
by a relation known as structural congruence and it is over this set
that the notion of dynamics is expressed. This presentation is
essentially that of \cite{MeredithR05} with the addition of
polyadicity and summation. For readability we have relegated some of
the technical subtleties to an appendix.

\subsubsection{Process grammar}\label{subsub:process_grammar}

\begin{mathpar}
  \inferrule* [lab=synchronization] {} {{M} \bc \pzero \;|\; x?F \;|\; x!C }
  \and
  \inferrule* [lab=abstraction] {} {{F} \bc (x)P}
  \and
  \inferrule* [lab=concretion] {} {{C} \bc \langle Q \rangle}
  \and
  \inferrule* [lab=process] {} {{P,Q} \bc M \;| \;P|Q \;|\; @{x}}
  \and
  \inferrule* [lab=name] {} {{x} \bc \quotep{P}}
\end{mathpar} 

Note that $\vec{x}$ (resp. $\vec{P}$) denotes a vector of names
(resp. processes) of length $|\vec{x}|$ (resp. $|\vec{P}|$). We adopt
the following useful abbreviations.

\begin{mathpar}
   x?(\vec{y}).P := x.(\vec{y})P \and  x\clift{\vec{P}} := x.\clift{\vec{P}}
   \and x!(y) := \lift{x}{\dropn{y}}
   \and \Pi_{i=0}^{n-1}P_i := P_0 | \ldots | P_{n-1}
\end{mathpar}

\subsubsection{Structural congruence}

\paragraph{Free and bound names and alpha-equivalence.} At the
core of structural equivalence is alpha-equivalence which identifies
process that are the same up to a change of variable. Formally, we
recognize the distinction between free and bound names. The free names
of a process, $\freenames{P}$, may be calculated recursively as
follows:

\begin{mathpar}
\freenames{\pzero} := \emptyset
  \and \\
  \freenames{x?(y).P} := \{ x \} \cup (\freenames{P} \setminus \{ y \})
  \and 
  \freenames{x!\langle P \rangle} := \{ x \} \cup \{ P \} 
  \and \\
  \freenames{P|Q} := \freenames{P} \cup \freenames{Q}
  \and \\
  \freenames{@{x}} := \{ x \}
\end{mathpar}

$\pi$
$\quotep{\pi}$

$\freenames{-} : \pi \to \mathcal{P}(\quotep{\pi})$

\begin{eqnarray*}
  \freenames{\pzero} & := & \emptyset \\
  \freenames{x?(y).P} & := & \{ x \} \cup (\freenames{P} \setminus \{ y \}) \\
  \freenames{x!\langle P \rangle} & := & \{ x \} \cup \{ P \} \\
  \freenames{P|Q} & := & \freenames{P} \cup \freenames{Q} \\
  \freenames{\dropn{x}} & := & \{ x \}
\end{eqnarray*}

The bound names of a process, $\boundnames{P}$, are those names occurring in $P$
that are not free. For example, in $x?(y).0$, the name $x$ is free, while $y$ is bound.

\begin{mathpar}
  \inferrule* [lab=monoidal-laws] {} { P|Q \equiv Q|P \and P|0 \equiv P \and P|(Q|R) \equiv (P|Q)|R }
\end{mathpar}

\begin{mathpar}
  \inferrule* [lab=alpha-equivalence] {} { (x)P \equiv (y)P\{y/x\} \and y \not\in \freenames{P} }
\end{mathpar}

\begin{definition}
Then two processes, $P,Q$, are alpha-equivalent if $P = Q\{\vec{y}/\vec{x}\}$ for
some $\vec{x} \in \boundnames{Q},\vec{y} \in \boundnames{P}$, where $Q\{\vec{y}/\vec{x}\}$
denotes the capture-avoiding substitution of $\vec{y}$ for $\vec{x}$ in $Q$.
\end{definition}

\begin{definition}
  The {\em structural congruence} \cite{SangiorgiWalker} , $\equiv$,
  between processes is the least congruence containing
  alpha-equivalence, satisfying the abelian monoid laws
  (associativity, commutativity and $\pzero$ as identity) for parallel
  composition $|$ and for summation $+$.
\end{definition}

\subsection{Name equivalence}

We take name equivalence, written $\nameeq$, to be the smallest
equivalence relation generated by the following rules.

\begin{mathpar}
\inferrule*[lab=Quote-drop]
{ }
{ \quotep{@{x}} \nameeq x }

\inferrule*[lab=Struct-equiv]
{ P \scong Q }
{ \quotep{P} \nameeq \quotep{Q} }
\end{mathpar}

The astute reader will have noticed that the mutual recursion of names
and processes imposes a mutual recursion on alpha-equivalence and
structural equivalence via name-equivalence. Fortunately, all of this
works out pleasantly and we may calculate in the natural way, free of
concern. The reader interested in the details is referred to the
appendix \ref{appendix:rho_details}.

\subsection{Substitution}

We use $\Proc$ for the set of processes, $\QProc$ for the set of
names, and $\id{\{}\vec{y} / \vec{x} \id{\}}$ to denote partial maps,
$s : \QProc \rightarrow \QProc$. A map, $s$ lifts, uniquely, to a map
on process terms, $\widehat{s} : \Proc \rightarrow \Proc$ by the
following equations.

\begin{mathpar}
  (0) \psubstp{Q}{P} := 0 \\
  (R \juxtap S) \psubstp{Q}{P}
  :=    
  (R)\psubstp{Q}{P} \juxtap (S) \psubstp{Q}{P} \\
  (x?(y).R) \psubstp{Q}{P}    
  :=    
  (x)\substp{Q}{P} (z)\concat( (R \psubstn{z}{y}) \psubstp{Q}{P} ) \\
  (\lift{x}{R}) \psubstp{Q}{P}  
  :=
  \lift{(x)\substp{Q}{P}}{ R \psubstp{Q}{P} } \\
%   (\dropn{x})  \psubstp{Q}{P}       
%   := 
%   \left\{ 
%     \begin{array}{ccc} 
%       \dropn{\quotep{Q}} & & x \nameeq \quotep{P} \\
%       \dropn{x} & & otherwise \\
%     \end{array}
%   \right. 
  (\dropn{x})  \psubstp{Q}{P}       
  := 
  \left\{ 
    \begin{array}{ccc} 
      Q & & x \nameeq \quotep{P} \\
      \dropn{x} & & otherwise \\
    \end{array}
  \right.
\end{mathpar}
 

where

\begin{eqnarray}
  (x)\id{\{} \lpquote Q \rpquote / \lpquote P \rpquote \id{\}}            = 
  \left\{ 
    \begin{array}{ccc}
      \lpquote Q \rpquote & & x \nameeq \lpquote P \rpquote \\
      x & & otherwise \\
    \end{array}
  \right. \nonumber
\end{eqnarray}

and $z$ is chosen distinct from $\quotep{P}$, $\quotep{Q}$, the free
names in $Q$, and all the names in $R$. Our $\alpha$-equivalence will
be built in the standard way from this substitution.

\begin{remark}\label{rem:no_self_referential_names}
  One consequence of these definitions is that $\forall P. \quotep{P}
  \not\in \freenames{P}$.
\end{remark}

\subsection{ Dynamic quote: an example }

Anticipating something of what's to come, consider applying the
substitution, $\widehat{\id{\{}u / z \id{\}}}$, to the following pair
of processes, $\lift{w}{y!(z)}$ and $w[ \lpquote y!(z) \rpquote ]$.

\begin{eqnarray}
	\lift{w}{y!(z)}\widehat{\id{\{}u / z \id{\}}}
		& = &
		\lift{w}{y!(u)} \nonumber\\
	w[ \lpquote y!(z) \rpquote ] \widehat{ \id{\{}u / z \id{\}} }
		& = &
		w[ \lpquote y!(z) \rpquote ] \nonumber
\end{eqnarray}

Because the body of the process between quotes is impervious to
substitution, we get radically different answers. In fact, by
examining the first process in an input context,
e.g. $x?(z).\lift{w}{y!(z)}$, we see that the process under the lift
operator may be shaped by prefixed inputs binding a name inside it. In
this sense, the lift operator will be seen as a way to dynamically
construct processes before reifying them as names.

Finally equipped with these standard features we can present the
dynamics of the calculus.

\subsubsection{Operational semantics} 

Finally, we introduce the computational dynamics. What marks these
algebras as distinct from other more traditionally studied algebraic
structures, e.g. vector spaces or polynomial rings, is the manner in
which dynamics is captured. In traditional structures, dynamics is typically
expressed through morphisms between such structures, as in linear maps
between vector spaces or morphisms between rings. In algebras
associated with the semantics of computation, the dynamics is
expressed as part of the algebraic structure itself, through a
reduction reduction relation typically denoted by $\red$. Below, we
give a recursive presentation of this relation for the calculus used
in the encoding.

$\red \subseteq \pi \times \pi$
$\red : \pi \to \mathcal{P}(\pi)$

\begin{mathpar}
  \inferrule* [lab=Comm] { \textsf{match}( x_{src}, x_{trgt} ) } { x_{trgt}?(y)P \; | \; x_{src}!\langle {Q} \rangle \red P\{\quotep{Q}/y}\} }
  \and \\
  \inferrule* [lab=Par] {{P} \red {P}'} {{{P} | {Q}} \red {{P}' | {Q}}}
  \and
  \inferrule* [lab=Equiv]{{{P} \scong {P}'} \andalso {{P}' \red {Q}'} \andalso {{Q}' \scong {Q}}}{{P} \red {Q}}
\end{mathpar}

\begin{eqnarray*}
  match_{\equiv} (\quotep{P},\quotep{Q}) & := & P \equiv Q \\
  match_{\dagger}(\quotep{P},\quotep{Q}) & := & \forall R. P|Q \red^{*} R => R \red^{*} 0 \\
  match_{K}(\quotep{P},\quotep{Q}) & := & K \mbox{ for some context } K
\end{eqnarray*}

$u?(x)P | u!\langle Q \rangle \red P\{\quotep{Q}/x\}$

%We write $\wred$ for $\red^*$, and $P\red$ if $\exists Q $ such that $ P \red Q$.
We write $P\red$ if $\exists Q $ such that $ P \red Q$ and $P\not\red$, otherwise.

\section{Replication}

As mentioned before, it is known that replication (and hence
recursion) can be implemented in a higher-order process algebra
\cite{SangiorgiWalker}. As our first example of calculation with the
machinery thus far presented we give the construction explicitly in
the {\rhoc}.

\begin{eqnarray}
	D_{x} & := & \prefix{x}{y}{(\binpar{\outputp{x}{y}}{@{y}})} \nonumber\\
	\bangp_{x}{P} & := & \binpar{{x}!\langle{\binpar{D_{x}}{P}}\rangle}{D_{x}} \nonumber
\end{eqnarray}

\begin{eqnarray}
	\bangp_{x}{P} & & \nonumber\\
	=
	& {x}!\langle{(\prefix{x}{y}{(\outputp{x}{y} | @{y})) | P}}\rangle 
	      | \prefix{x}{y}{(\outputp{x}{y} | @{y})} & \nonumber\\
	\red
	& (\outputp{x}{y} | @{y})\substn{\quotep{(\prefix{x}{y}{(@{y} | \outputp{x}{y})) | P}}}{y} & \nonumber\\
	=
	& \outputp{x}{\quotep{(\prefix{x}{y}{(\outputp{x}{y} | @{y})) | P}}}
	  | {(\prefix{x}{y}{(\outputp{x}{y} | @{y})) | P}} & \nonumber\\
	\red
	& \ldots & \nonumber\\
	\red^*
	& P | P | \ldots & \nonumber
\end{eqnarray}

Of course, this encoding, as an implementation, runs away, unfolding
$\bangp{P}$ eagerly. A lazier and more implementable replication
operator, restricted to input-guarded processes, may be obtained as follows.

\begin{eqnarray}
\bangp{\prefix{u}{v}{P}} 
	:= 
	\binpar{\lift{x}{\prefix{u}{v}{(\binpar{D(x)}{P})}}}{D(x)} \nonumber
\end{eqnarray}

\begin{remark}
  Note that the lazier definition still does not deal with summation
  or mixed summation (i.e. sums over input and output). The reader is
  invited to construct definitions of replication that deal with these
  features. 

  Further, the definitions are parameterized in a name, $x$. Can you,
  gentle reader, make a definition that eliminates this parameter and
  guarantees no accidental interaction between the replication
  machinery and the process being replicated -- i.e. no accidental
  sharing of names used by the process to get its work done and the
  name(s) used by the replication to effect copying. This latter
  revision of the definition of replication is crucial to obtaining
  the expected identity $!!P \sim !P$.
\end{remark}

\begin{remark}\label{rem:paradoxical_combinator}
  The reader familiar with the lambda calculus will have noticed the
  similarity between $D$ and the paradoxical combinator.

  [Ed. note: the existence of this seems to suggest we have to be more
  restrictive on the set of processes and names we admit if we are to
  support no-cloning.]
\end{remark}

\subsubsection{Bisimulation}

The computational dynamics gives rise to another kind of equivalence,
the equivalence of computational behavior. As previously mentioned
this is typically captured \emph{via} some form of bisimulation.

% The notion we use in this paper is weak barbed bisimulation
% \cite{milner91polyadicpi}.

The notion we use in this paper is derived from weak barbed
bisimulation \cite{milner91polyadicpi}. 

\begin{definition}
An \emph{observation relation}, $\downarrow_{\mathcal N}$, over a set
of names, $\mathcal N$, is the smallest relation satisfying the rules
below.

\infrule[Out-barb]{y \in {\mathcal N}, \; x \nameeq y}
		  {\outputp{x}{v} \downarrow_{\mathcal N} x}
\infrule[Par-barb]{\mbox{$P\downarrow_{\mathcal N} x$ or $Q\downarrow_{\mathcal N} x$}}
		  {\binpar{P}{Q} \downarrow_{\mathcal N} x}

We write $P \Downarrow_{\mathcal N} x$ if there is $Q$ such that 
$P \wred Q$ and $Q \downarrow_{\mathcal N} x$.
\end{definition}

\begin{definition}
%\label{def.bbisim}
An  ${\mathcal N}$-\emph{barbed bisimulation} over a set of names, ${\mathcal N}$, is a symmetric binary relation 
${\mathcal S}_{\mathcal N}$ between agents such that $P\rel{S}_{\mathcal N}Q$ implies:
\begin{enumerate}
\item If $P \red P'$ then $Q \wred Q'$ and $P'\rel{S}_{\mathcal N} Q'$.
\item If $P\downarrow_{\mathcal N} x$, then $Q\Downarrow_{\mathcal N} x$.
\end{enumerate}
$P$ is ${\mathcal N}$-barbed bisimilar to $Q$, written
$P \wbbisim_{\mathcal N} Q$, if $P \rel{S}_{\mathcal N} Q$ for some ${\mathcal N}$-barbed bisimulation ${\mathcal S}_{\mathcal N}$.
\end{definition}

$\mathcal{R} \subseteq \pi \times \pi$

$P \mathcal{R} Q => \forall P'. P \red P' \Rightarrow \exists Q'. Q \red Q', P' \mathcal{R} Q'$

$P \vdash x \Rightarrow Q \vdash x$

\begin{mathpar}
  \inferrule*[lab=Out-barb]{x \nameeq y}{{y}!\langle{Q}\rangle \vdash x}
  \and
  \inferrule*[lab=Par-barb]{\mbox{$P\vdash x$ or $Q\vdash x$}}{\binpar{P}{Q} \vdash x}
\end{mathpar}

\subsubsection{Contexts}

One of the principle advantages of computational calculi like the
$\pi$-calculus is a well-defined notion of context,
contextual-equivalence and a correlation between
contextual-equivalence and notions of bisimulation. The notion of
context allows the decomposition of a process into (sub-)process and
its syntactic environment, its context. Thus, a context may be
thought of as a process with a ``hole'' (written $\Box$) in it. The
application of a context $M$ to a process $P$, written $M[P]$, is
tantamount to filling the hole in $M$ with $P$. In this paper we do
not need the full weight of this theory, but do make use of the notion
of context in the proof the main theorem. 

\begin{mathpar}
  \inferrule* [lab=summation] {} {{M_{M},M_{N}} \bc \Box \;|\; x.M_{A} \;|\; M_{M}+M_{N}}
  \and
  \inferrule* [lab=agent] {} {{M_{A}} \bc (\vec{x})M_{P} \;| \; \clift{P_0,\ldots,M_{P},\ldots,P_N}}
  \and \\
  \inferrule* [lab=process] {} {{M_{P}} \bc M_{N} \;| \;P|M_{P} }
\end{mathpar} 

\begin{mathpar}
  \inferrule* [lab=sychronization] {} {M_{N} \bc \Box \;|\; x?M_{F} \;|\; x!M_{C}}
  \and
  \inferrule* [lab=abstraction] {} {{M_{F}} \bc (x)M_{P} }
  \and
  \inferrule* [lab=concretion] {} {{M_{C}} \bc \langle M_{P} \rangle }
  \and \\
  \inferrule* [lab=process] {} {{M_{P}} \bc M_{N} \;| \;P|M_{P} }
\end{mathpar}

\begin{definition}[contextual application] Given a context $M$, and
  process $P$, we define the \emph{contextual application}, $M[P] :=
  M\{P/\Box\}$. That is, the contextual application of M to P is the
  substitution of $P$ for $\Box$ in $M$.
\end{definition}

$\meaningof{-} : L \to \mathcal{P}(\pi)$

\begin{mathpar}
  \inferrule* [lab=collection] {} {\meaningof{true} = \pi, \and \meaningof{~E} = \pi \setminus \meaningof{E}, \and \meaningof{E_{1} \& E_{2}} = \meaningof{E_{1}} \cap \meaningof{E_{2}}}
\end{mathpar}

\begin{mathpar}
  \inferrule* [lab=structure] {} {\meaningof{0} = \{ P \in \pi | P \equiv 0 \}, \and \\ \meaningof{E_1 | E_2} = \{ P \in \pi | P \equiv P_{1} | P_{2}, P_{1} \in \meaningof{E_{1}}, P_{2} \in \meaningof{E_2}\} }
\end{mathpar}

\begin{mathpar}
 \inferrule* [lab=behavior] {} {\meaningof{\langle a?b \rangle E} = \{ P \in \pi | P \equiv Q | u?(y)P', \\ \and \\\\ \and \\ \;\;\; u \in \meaningof{a}, \forall z.P'\{z/y\} \in \meaningof{E\{z/b\}}\}, \and \\ \meaningof{a!E} = \{ P \in \pi | P \equiv Q | x!\langle P' \rangle, x \in \meaningof{a} P' \in \meaningof{E}\} }
\end{mathpar}

\begin{mathpar}
 \inferrule* [lab=nominal] {} {\meaningof{\quotep{E}} = \{ \quotep{P} \in \quotep{\pi} | P \in \meaningof{E} \}, \and \meaningof{\quotep{P}} = \{ \quotep{Q} \in \quotep{\pi} | P \equiv Q \} \and \\ \meaningof{@\quotep{E}} = \{ P \in \pi | P \equiv @x, x \in \meaningof{E} \}}
\end{mathpar}

\begin{eqnarray*}
  \\
  \meaningof{-} : TS \to ST
\end{eqnarray*}

\begin{eqnarray*}
  \\
  L : TS \to ST
\end{eqnarray*}

\begin{eqnarray*}
  \\
  P \models E \iff P \in \meaningof{E}
\end{eqnarray*}

\begin{eqnarray*}
  P \approx_{L} Q \iff \forall E \in L. P \models E \iff Q \models E
\end{eqnarray*}

\begin{eqnarray*}
  P \approx_{K} Q
\end{eqnarray*}

\begin{eqnarray*}
  P \approx Q
\end{eqnarray*}

$\approx_{K} = \approx = \approx_{L}$

\subsubsection{Contextual duality}

Note that contexts extend the quotation operation to a family of
operations from processes to names. Given a context, $M$, we can
define a \emph{nominal context}, $\quotep{M}$ by $\quotep{M}[P] :=
\quotep{M[P]}$. To foreshadow what is to come we observe that these
operations enjoy a duality with processes very much like the duality
between vectors and maps from vectors to scalars.

Further, because the calculus is essentially higher-order, we have a
correspondence between contexts and processes. More specifically,
given a name $x$ and a context $M$ we can construct $M^{*}_{x}$ such
that 

\begin{mathpar}
  M^{*}_{x} | \lift{x}{P} \red M[P]
\end{mathpar}

namely,

\begin{mathpar}
  M^{*}_{x} := x?(u).M[\dropn{u}]
\end{mathpar}

The dependence of $M^{*}_{x}$ on a name makes it an abstraction, 

\begin{mathpar}
  M^{*} := (x)x?(u).M[\dropn{u}]
\end{mathpar}

\subsection{Additional notation}

It will sometimes be convenient to denote the process a name
quotes. We already have the notation $x = \quotep{P}$, but it will be
convenient to introduce an alternate notation, $\procn{x}$, when we
want to emphasize the connection to the use of the name. Note that, by
virtue of name equivalence, $\quotep{\procn{x}} \nameeq x$; so, the
notation is consistent with previous definitions.

Further, because names have structure it is possible to effect
substitutions on the basis of that structure. This means we need to
upgrade our notation for substitutions, which we accomplish by
adapting comprehension notation. Thus,

\begin{mathpar}
  P\{ y / x : x \in S \}
\end{mathpar}

is interpreted to mean the process derived from P by replacing (in a
capture-avoiding manner) each occurrence of $x$ in $S$ by $y$. For example,

\begin{mathpar}
  P\{ \quotep{\procn{x}|\procn{x}} / x : x \in \freenames{P} \}
\end{mathpar}

will replace each (occurrence) of a free name $x$ in $P$ by
$\quotep{\procn{x}|\procn{x}}$.

Also, we will avail ourselves of the notation $x^{L}$ and $x^{R}$ to
denote injections of a name into disjoint copies of the name
space. There are numerous ways to accomplish this. One example can be
found in \cite{MeredithR05}. This notation overloads to vectors of
names: $\vec{x}^{\pi} := (x_{i}^{\pi} \; : \; 0 \leq i < |\vec{x}| )$ where $\pi \in \{L,R\}$.

We also use $P^{\Box} := P|\Box$.

In \cite{MeredithR05} an interpretation of the new operator is
given. It turns out that there are several possible interpretations
all enjoying the requisite algebraic properties of the operator (see
\cite{milner91polyadicpi}). We will therefore make liberal use of
$(\nu\; \vec{x})P$.

% subsection the_syntax_and_semantics_of_the_notation_system (end)   

\input{qm2pi.qmops} 

\input{qm2pi.sterngerlach} 

\input{qm2pi.metric} 

% section concurrent_process_calculi (end)

%\input{qm2pi.proofsketch}

% section proof sketch (end)

%\input{qm2pi.slviaknots} 

% section spatial logic via knots (end)

\input{qm2pi.conclusion}

% section conclusion (end)

%\input{qm2pi.dtcodes} 

% section wiring algorithm (end)

\input{qm2pi.ack} 

% section acknowledgments (end)

\newpage


\bibliographystyle{plain}   
\bibliography{../../biblios/main.bib}

\input{qm2pi.rhodetails}

\end{document}

 

\documentclass[12pt]{llncs}
%\documentclass{jktr}

\usepackage[pdftex]{hyperref}                   
\usepackage {listings}
\usepackage {mathpartir}
\usepackage{bcprules}
%\usepackage{listings}
                       
\usepackage{graphicx} 
%\usepackage[margins=2.5cm,nohead,nofoot]{geometry}
%\usepackage{geometry}
\usepackage{amsfonts}
\usepackage{amstext}
\usepackage{latexsym}
\usepackage{amssymb}
\usepackage{color}


%\include{myPreamble}
\include{qm2pi.local} 

%\ifpdf
%\usepackage[pdftex]{graphicx}
%\else
%\usepackage{graphicx}
%\fi

 % \ifpdf
%  \usepackage{pdfsync}
%  \if


%\title{Brief Article}
%\author{David F. Snyder}
%\author{L.G. Meredith}

%\address{Dept. of Math., Texas State University--San Marcos, San Marcos, TX 78666}
       
\pagestyle{empty}


\begin{document}

\lstset{language=[Objective]Caml,frame=shadowbox}

\input{qm2pi.front}

% section front matter (end)

\input{qm2pi.intro} 
 
% section introduction (end)

% \input{qm2pi.knotations} 

% section notation (end)

\input{qm2pi.process.calculi} 

% section concurrent_process_calculi_and_spatial_logics_ (end)
    
%\input{qm2pi.knots2pi} 

%\input{qm2pi.trefoil} 

%\input{qm2pi.mainthm} 

% subsection basic_interpretation (end)

%\input{qm2pi.rho.presentation} 
\subsection{The syntax and semantics of the notation system}\label{sub:the_syntax_and_semantics_of_the_notation_system} % (fold)

We now summarize a technical presentation of the calculus that
embodies our theory of dynamics. The typical presentation of such a
calculus follows the style of giving generators and relations on
them. The grammar, below, describing term constructors, freely
generates the set of processes, $\Proc$. This set is then quotiented
by a relation known as structural congruence and it is over this set
that the notion of dynamics is expressed. This presentation is
essentially that of \cite{MeredithR05} with the addition of
polyadicity and summation. For readability we have relegated some of
the technical subtleties to an appendix.

\subsubsection{Process grammar}\label{subsub:process_grammar}

\begin{mathpar}
  \inferrule* [lab=synchronization] {} {{M} \bc \pzero \;|\; x?F \;|\; x!C }
  \and
  \inferrule* [lab=abstraction] {} {{F} \bc (x)P}
  \and
  \inferrule* [lab=concretion] {} {{C} \bc \langle Q \rangle}
  \and
  \inferrule* [lab=process] {} {{P,Q} \bc M \;| \;P|Q \;|\; @{x}}
  \and
  \inferrule* [lab=name] {} {{x} \bc \quotep{P}}
\end{mathpar} 

Note that $\vec{x}$ (resp. $\vec{P}$) denotes a vector of names
(resp. processes) of length $|\vec{x}|$ (resp. $|\vec{P}|$). We adopt
the following useful abbreviations.

\begin{mathpar}
   x?(\vec{y}).P := x.(\vec{y})P \and  x\clift{\vec{P}} := x.\clift{\vec{P}}
   \and x!(y) := \lift{x}{\dropn{y}}
   \and \Pi_{i=0}^{n-1}P_i := P_0 | \ldots | P_{n-1}
\end{mathpar}

\subsubsection{Structural congruence}

\paragraph{Free and bound names and alpha-equivalence.} At the
core of structural equivalence is alpha-equivalence which identifies
process that are the same up to a change of variable. Formally, we
recognize the distinction between free and bound names. The free names
of a process, $\freenames{P}$, may be calculated recursively as
follows:

\begin{mathpar}
\freenames{\pzero} := \emptyset
  \and \\
  \freenames{x?(y).P} := \{ x \} \cup (\freenames{P} \setminus \{ y \})
  \and 
  \freenames{x!\langle P \rangle} := \{ x \} \cup \{ P \} 
  \and \\
  \freenames{P|Q} := \freenames{P} \cup \freenames{Q}
  \and \\
  \freenames{@{x}} := \{ x \}
\end{mathpar}

$\pi$
$\quotep{\pi}$

$\freenames{-} : \pi \to \mathcal{P}(\quotep{\pi})$

\begin{eqnarray*}
  \freenames{\pzero} & := & \emptyset \\
  \freenames{x?(y).P} & := & \{ x \} \cup (\freenames{P} \setminus \{ y \}) \\
  \freenames{x!\langle P \rangle} & := & \{ x \} \cup \{ P \} \\
  \freenames{P|Q} & := & \freenames{P} \cup \freenames{Q} \\
  \freenames{\dropn{x}} & := & \{ x \}
\end{eqnarray*}

The bound names of a process, $\boundnames{P}$, are those names occurring in $P$
that are not free. For example, in $x?(y).0$, the name $x$ is free, while $y$ is bound.

\begin{mathpar}
  \inferrule* [lab=monoidal-laws] {} { P|Q \equiv Q|P \and P|0 \equiv P \and P|(Q|R) \equiv (P|Q)|R }
\end{mathpar}

\begin{mathpar}
  \inferrule* [lab=alpha-equivalence] {} { (x)P \equiv (y)P\{y/x\} \and y \not\in \freenames{P} }
\end{mathpar}

\begin{definition}
Then two processes, $P,Q$, are alpha-equivalent if $P = Q\{\vec{y}/\vec{x}\}$ for
some $\vec{x} \in \boundnames{Q},\vec{y} \in \boundnames{P}$, where $Q\{\vec{y}/\vec{x}\}$
denotes the capture-avoiding substitution of $\vec{y}$ for $\vec{x}$ in $Q$.
\end{definition}

\begin{definition}
  The {\em structural congruence} \cite{SangiorgiWalker} , $\equiv$,
  between processes is the least congruence containing
  alpha-equivalence, satisfying the abelian monoid laws
  (associativity, commutativity and $\pzero$ as identity) for parallel
  composition $|$ and for summation $+$.
\end{definition}

\subsection{Name equivalence}

We take name equivalence, written $\nameeq$, to be the smallest
equivalence relation generated by the following rules.

\begin{mathpar}
\inferrule*[lab=Quote-drop]
{ }
{ \quotep{@{x}} \nameeq x }

\inferrule*[lab=Struct-equiv]
{ P \scong Q }
{ \quotep{P} \nameeq \quotep{Q} }
\end{mathpar}

The astute reader will have noticed that the mutual recursion of names
and processes imposes a mutual recursion on alpha-equivalence and
structural equivalence via name-equivalence. Fortunately, all of this
works out pleasantly and we may calculate in the natural way, free of
concern. The reader interested in the details is referred to the
appendix \ref{appendix:rho_details}.

\subsection{Substitution}

We use $\Proc$ for the set of processes, $\QProc$ for the set of
names, and $\id{\{}\vec{y} / \vec{x} \id{\}}$ to denote partial maps,
$s : \QProc \rightarrow \QProc$. A map, $s$ lifts, uniquely, to a map
on process terms, $\widehat{s} : \Proc \rightarrow \Proc$ by the
following equations.

\begin{mathpar}
  (0) \psubstp{Q}{P} := 0 \\
  (R \juxtap S) \psubstp{Q}{P}
  :=    
  (R)\psubstp{Q}{P} \juxtap (S) \psubstp{Q}{P} \\
  (x?(y).R) \psubstp{Q}{P}    
  :=    
  (x)\substp{Q}{P} (z)\concat( (R \psubstn{z}{y}) \psubstp{Q}{P} ) \\
  (\lift{x}{R}) \psubstp{Q}{P}  
  :=
  \lift{(x)\substp{Q}{P}}{ R \psubstp{Q}{P} } \\
%   (\dropn{x})  \psubstp{Q}{P}       
%   := 
%   \left\{ 
%     \begin{array}{ccc} 
%       \dropn{\quotep{Q}} & & x \nameeq \quotep{P} \\
%       \dropn{x} & & otherwise \\
%     \end{array}
%   \right. 
  (\dropn{x})  \psubstp{Q}{P}       
  := 
  \left\{ 
    \begin{array}{ccc} 
      Q & & x \nameeq \quotep{P} \\
      \dropn{x} & & otherwise \\
    \end{array}
  \right.
\end{mathpar}
 

where

\begin{eqnarray}
  (x)\id{\{} \lpquote Q \rpquote / \lpquote P \rpquote \id{\}}            = 
  \left\{ 
    \begin{array}{ccc}
      \lpquote Q \rpquote & & x \nameeq \lpquote P \rpquote \\
      x & & otherwise \\
    \end{array}
  \right. \nonumber
\end{eqnarray}

and $z$ is chosen distinct from $\quotep{P}$, $\quotep{Q}$, the free
names in $Q$, and all the names in $R$. Our $\alpha$-equivalence will
be built in the standard way from this substitution.

\begin{remark}\label{rem:no_self_referential_names}
  One consequence of these definitions is that $\forall P. \quotep{P}
  \not\in \freenames{P}$.
\end{remark}

\subsection{ Dynamic quote: an example }

Anticipating something of what's to come, consider applying the
substitution, $\widehat{\id{\{}u / z \id{\}}}$, to the following pair
of processes, $\lift{w}{y!(z)}$ and $w[ \lpquote y!(z) \rpquote ]$.

\begin{eqnarray}
	\lift{w}{y!(z)}\widehat{\id{\{}u / z \id{\}}}
		& = &
		\lift{w}{y!(u)} \nonumber\\
	w[ \lpquote y!(z) \rpquote ] \widehat{ \id{\{}u / z \id{\}} }
		& = &
		w[ \lpquote y!(z) \rpquote ] \nonumber
\end{eqnarray}

Because the body of the process between quotes is impervious to
substitution, we get radically different answers. In fact, by
examining the first process in an input context,
e.g. $x?(z).\lift{w}{y!(z)}$, we see that the process under the lift
operator may be shaped by prefixed inputs binding a name inside it. In
this sense, the lift operator will be seen as a way to dynamically
construct processes before reifying them as names.

Finally equipped with these standard features we can present the
dynamics of the calculus.

\subsubsection{Operational semantics} 

Finally, we introduce the computational dynamics. What marks these
algebras as distinct from other more traditionally studied algebraic
structures, e.g. vector spaces or polynomial rings, is the manner in
which dynamics is captured. In traditional structures, dynamics is typically
expressed through morphisms between such structures, as in linear maps
between vector spaces or morphisms between rings. In algebras
associated with the semantics of computation, the dynamics is
expressed as part of the algebraic structure itself, through a
reduction reduction relation typically denoted by $\red$. Below, we
give a recursive presentation of this relation for the calculus used
in the encoding.

$\red \subseteq \pi \times \pi$
$\red : \pi \to \mathcal{P}(\pi)$

\begin{mathpar}
  \inferrule* [lab=Comm] { \textsf{match}( x_{src}, x_{trgt} ) } { x_{trgt}?(y)P \; | \; x_{src}!\langle {Q} \rangle \red P\{\quotep{Q}/y}\} }
  \and \\
  \inferrule* [lab=Par] {{P} \red {P}'} {{{P} | {Q}} \red {{P}' | {Q}}}
  \and
  \inferrule* [lab=Equiv]{{{P} \scong {P}'} \andalso {{P}' \red {Q}'} \andalso {{Q}' \scong {Q}}}{{P} \red {Q}}
\end{mathpar}

\begin{eqnarray*}
  match_{\equiv} (\quotep{P},\quotep{Q}) & := & P \equiv Q \\
  match_{\dagger}(\quotep{P},\quotep{Q}) & := & \forall R. P|Q \red^{*} R => R \red^{*} 0 \\
  match_{K}(\quotep{P},\quotep{Q}) & := & K \mbox{ for some context } K
\end{eqnarray*}

$u?(x)P | u!\langle Q \rangle \red P\{\quotep{Q}/x\}$

%We write $\wred$ for $\red^*$, and $P\red$ if $\exists Q $ such that $ P \red Q$.
We write $P\red$ if $\exists Q $ such that $ P \red Q$ and $P\not\red$, otherwise.

\section{Replication}

As mentioned before, it is known that replication (and hence
recursion) can be implemented in a higher-order process algebra
\cite{SangiorgiWalker}. As our first example of calculation with the
machinery thus far presented we give the construction explicitly in
the {\rhoc}.

\begin{eqnarray}
	D_{x} & := & \prefix{x}{y}{(\binpar{\outputp{x}{y}}{@{y}})} \nonumber\\
	\bangp_{x}{P} & := & \binpar{{x}!\langle{\binpar{D_{x}}{P}}\rangle}{D_{x}} \nonumber
\end{eqnarray}

\begin{eqnarray}
	\bangp_{x}{P} & & \nonumber\\
	=
	& {x}!\langle{(\prefix{x}{y}{(\outputp{x}{y} | @{y})) | P}}\rangle 
	      | \prefix{x}{y}{(\outputp{x}{y} | @{y})} & \nonumber\\
	\red
	& (\outputp{x}{y} | @{y})\substn{\quotep{(\prefix{x}{y}{(@{y} | \outputp{x}{y})) | P}}}{y} & \nonumber\\
	=
	& \outputp{x}{\quotep{(\prefix{x}{y}{(\outputp{x}{y} | @{y})) | P}}}
	  | {(\prefix{x}{y}{(\outputp{x}{y} | @{y})) | P}} & \nonumber\\
	\red
	& \ldots & \nonumber\\
	\red^*
	& P | P | \ldots & \nonumber
\end{eqnarray}

Of course, this encoding, as an implementation, runs away, unfolding
$\bangp{P}$ eagerly. A lazier and more implementable replication
operator, restricted to input-guarded processes, may be obtained as follows.

\begin{eqnarray}
\bangp{\prefix{u}{v}{P}} 
	:= 
	\binpar{\lift{x}{\prefix{u}{v}{(\binpar{D(x)}{P})}}}{D(x)} \nonumber
\end{eqnarray}

\begin{remark}
  Note that the lazier definition still does not deal with summation
  or mixed summation (i.e. sums over input and output). The reader is
  invited to construct definitions of replication that deal with these
  features. 

  Further, the definitions are parameterized in a name, $x$. Can you,
  gentle reader, make a definition that eliminates this parameter and
  guarantees no accidental interaction between the replication
  machinery and the process being replicated -- i.e. no accidental
  sharing of names used by the process to get its work done and the
  name(s) used by the replication to effect copying. This latter
  revision of the definition of replication is crucial to obtaining
  the expected identity $!!P \sim !P$.
\end{remark}

\begin{remark}\label{rem:paradoxical_combinator}
  The reader familiar with the lambda calculus will have noticed the
  similarity between $D$ and the paradoxical combinator.

  [Ed. note: the existence of this seems to suggest we have to be more
  restrictive on the set of processes and names we admit if we are to
  support no-cloning.]
\end{remark}

\subsubsection{Bisimulation}

The computational dynamics gives rise to another kind of equivalence,
the equivalence of computational behavior. As previously mentioned
this is typically captured \emph{via} some form of bisimulation.

% The notion we use in this paper is weak barbed bisimulation
% \cite{milner91polyadicpi}.

The notion we use in this paper is derived from weak barbed
bisimulation \cite{milner91polyadicpi}. 

\begin{definition}
An \emph{observation relation}, $\downarrow_{\mathcal N}$, over a set
of names, $\mathcal N$, is the smallest relation satisfying the rules
below.

\infrule[Out-barb]{y \in {\mathcal N}, \; x \nameeq y}
		  {\outputp{x}{v} \downarrow_{\mathcal N} x}
\infrule[Par-barb]{\mbox{$P\downarrow_{\mathcal N} x$ or $Q\downarrow_{\mathcal N} x$}}
		  {\binpar{P}{Q} \downarrow_{\mathcal N} x}

We write $P \Downarrow_{\mathcal N} x$ if there is $Q$ such that 
$P \wred Q$ and $Q \downarrow_{\mathcal N} x$.
\end{definition}

\begin{definition}
%\label{def.bbisim}
An  ${\mathcal N}$-\emph{barbed bisimulation} over a set of names, ${\mathcal N}$, is a symmetric binary relation 
${\mathcal S}_{\mathcal N}$ between agents such that $P\rel{S}_{\mathcal N}Q$ implies:
\begin{enumerate}
\item If $P \red P'$ then $Q \wred Q'$ and $P'\rel{S}_{\mathcal N} Q'$.
\item If $P\downarrow_{\mathcal N} x$, then $Q\Downarrow_{\mathcal N} x$.
\end{enumerate}
$P$ is ${\mathcal N}$-barbed bisimilar to $Q$, written
$P \wbbisim_{\mathcal N} Q$, if $P \rel{S}_{\mathcal N} Q$ for some ${\mathcal N}$-barbed bisimulation ${\mathcal S}_{\mathcal N}$.
\end{definition}

$\mathcal{R} \subseteq \pi \times \pi$

$P \mathcal{R} Q => \forall P'. P \red P' \Rightarrow \exists Q'. Q \red Q', P' \mathcal{R} Q'$

$P \vdash x \Rightarrow Q \vdash x$

\begin{mathpar}
  \inferrule*[lab=Out-barb]{x \nameeq y}{{y}!\langle{Q}\rangle \vdash x}
  \and
  \inferrule*[lab=Par-barb]{\mbox{$P\vdash x$ or $Q\vdash x$}}{\binpar{P}{Q} \vdash x}
\end{mathpar}

\subsubsection{Contexts}

One of the principle advantages of computational calculi like the
$\pi$-calculus is a well-defined notion of context,
contextual-equivalence and a correlation between
contextual-equivalence and notions of bisimulation. The notion of
context allows the decomposition of a process into (sub-)process and
its syntactic environment, its context. Thus, a context may be
thought of as a process with a ``hole'' (written $\Box$) in it. The
application of a context $M$ to a process $P$, written $M[P]$, is
tantamount to filling the hole in $M$ with $P$. In this paper we do
not need the full weight of this theory, but do make use of the notion
of context in the proof the main theorem. 

\begin{mathpar}
  \inferrule* [lab=summation] {} {{M_{M},M_{N}} \bc \Box \;|\; x.M_{A} \;|\; M_{M}+M_{N}}
  \and
  \inferrule* [lab=agent] {} {{M_{A}} \bc (\vec{x})M_{P} \;| \; \clift{P_0,\ldots,M_{P},\ldots,P_N}}
  \and \\
  \inferrule* [lab=process] {} {{M_{P}} \bc M_{N} \;| \;P|M_{P} }
\end{mathpar} 

\begin{mathpar}
  \inferrule* [lab=sychronization] {} {M_{N} \bc \Box \;|\; x?M_{F} \;|\; x!M_{C}}
  \and
  \inferrule* [lab=abstraction] {} {{M_{F}} \bc (x)M_{P} }
  \and
  \inferrule* [lab=concretion] {} {{M_{C}} \bc \langle M_{P} \rangle }
  \and \\
  \inferrule* [lab=process] {} {{M_{P}} \bc M_{N} \;| \;P|M_{P} }
\end{mathpar}

\begin{definition}[contextual application] Given a context $M$, and
  process $P$, we define the \emph{contextual application}, $M[P] :=
  M\{P/\Box\}$. That is, the contextual application of M to P is the
  substitution of $P$ for $\Box$ in $M$.
\end{definition}

$\meaningof{-} : L \to \mathcal{P}(\pi)$

\begin{mathpar}
  \inferrule* [lab=collection] {} {\meaningof{true} = \pi, \and \meaningof{~E} = \pi \setminus \meaningof{E}, \and \meaningof{E_{1} \& E_{2}} = \meaningof{E_{1}} \cap \meaningof{E_{2}}}
\end{mathpar}

\begin{mathpar}
  \inferrule* [lab=structure] {} {\meaningof{0} = \{ P \in \pi | P \equiv 0 \}, \and \\ \meaningof{E_1 | E_2} = \{ P \in \pi | P \equiv P_{1} | P_{2}, P_{1} \in \meaningof{E_{1}}, P_{2} \in \meaningof{E_2}\} }
\end{mathpar}

\begin{mathpar}
 \inferrule* [lab=behavior] {} {\meaningof{\langle a?b \rangle E} = \{ P \in \pi | P \equiv Q | u?(y)P', \\ \and \\\\ \and \\ \;\;\; u \in \meaningof{a}, \forall z.P'\{z/y\} \in \meaningof{E\{z/b\}}\}, \and \\ \meaningof{a!E} = \{ P \in \pi | P \equiv Q | x!\langle P' \rangle, x \in \meaningof{a} P' \in \meaningof{E}\} }
\end{mathpar}

\begin{mathpar}
 \inferrule* [lab=nominal] {} {\meaningof{\quotep{E}} = \{ \quotep{P} \in \quotep{\pi} | P \in \meaningof{E} \}, \and \meaningof{\quotep{P}} = \{ \quotep{Q} \in \quotep{\pi} | P \equiv Q \} \and \\ \meaningof{@\quotep{E}} = \{ P \in \pi | P \equiv @x, x \in \meaningof{E} \}}
\end{mathpar}

\begin{eqnarray*}
  \\
  \meaningof{-} : TS \to ST
\end{eqnarray*}

\begin{eqnarray*}
  \\
  L : TS \to ST
\end{eqnarray*}

\begin{eqnarray*}
  \\
  P \models E \iff P \in \meaningof{E}
\end{eqnarray*}

\begin{eqnarray*}
  P \approx_{L} Q \iff \forall E \in L. P \models E \iff Q \models E
\end{eqnarray*}

\begin{eqnarray*}
  P \approx_{K} Q
\end{eqnarray*}

\begin{eqnarray*}
  P \approx Q
\end{eqnarray*}

$\approx_{K} = \approx = \approx_{L}$

\subsubsection{Contextual duality}

Note that contexts extend the quotation operation to a family of
operations from processes to names. Given a context, $M$, we can
define a \emph{nominal context}, $\quotep{M}$ by $\quotep{M}[P] :=
\quotep{M[P]}$. To foreshadow what is to come we observe that these
operations enjoy a duality with processes very much like the duality
between vectors and maps from vectors to scalars.

Further, because the calculus is essentially higher-order, we have a
correspondence between contexts and processes. More specifically,
given a name $x$ and a context $M$ we can construct $M^{*}_{x}$ such
that 

\begin{mathpar}
  M^{*}_{x} | \lift{x}{P} \red M[P]
\end{mathpar}

namely,

\begin{mathpar}
  M^{*}_{x} := x?(u).M[\dropn{u}]
\end{mathpar}

The dependence of $M^{*}_{x}$ on a name makes it an abstraction, 

\begin{mathpar}
  M^{*} := (x)x?(u).M[\dropn{u}]
\end{mathpar}

\subsection{Additional notation}

It will sometimes be convenient to denote the process a name
quotes. We already have the notation $x = \quotep{P}$, but it will be
convenient to introduce an alternate notation, $\procn{x}$, when we
want to emphasize the connection to the use of the name. Note that, by
virtue of name equivalence, $\quotep{\procn{x}} \nameeq x$; so, the
notation is consistent with previous definitions.

Further, because names have structure it is possible to effect
substitutions on the basis of that structure. This means we need to
upgrade our notation for substitutions, which we accomplish by
adapting comprehension notation. Thus,

\begin{mathpar}
  P\{ y / x : x \in S \}
\end{mathpar}

is interpreted to mean the process derived from P by replacing (in a
capture-avoiding manner) each occurrence of $x$ in $S$ by $y$. For example,

\begin{mathpar}
  P\{ \quotep{\procn{x}|\procn{x}} / x : x \in \freenames{P} \}
\end{mathpar}

will replace each (occurrence) of a free name $x$ in $P$ by
$\quotep{\procn{x}|\procn{x}}$.

Also, we will avail ourselves of the notation $x^{L}$ and $x^{R}$ to
denote injections of a name into disjoint copies of the name
space. There are numerous ways to accomplish this. One example can be
found in \cite{MeredithR05}. This notation overloads to vectors of
names: $\vec{x}^{\pi} := (x_{i}^{\pi} \; : \; 0 \leq i < |\vec{x}| )$ where $\pi \in \{L,R\}$.

We also use $P^{\Box} := P|\Box$.

In \cite{MeredithR05} an interpretation of the new operator is
given. It turns out that there are several possible interpretations
all enjoying the requisite algebraic properties of the operator (see
\cite{milner91polyadicpi}). We will therefore make liberal use of
$(\nu\; \vec{x})P$.

% subsection the_syntax_and_semantics_of_the_notation_system (end)   

\input{qm2pi.qmops} 

\input{qm2pi.sterngerlach} 

\input{qm2pi.metric} 

% section concurrent_process_calculi (end)

%\input{qm2pi.proofsketch}

% section proof sketch (end)

%\input{qm2pi.slviaknots} 

% section spatial logic via knots (end)

\input{qm2pi.conclusion}

% section conclusion (end)

%\input{qm2pi.dtcodes} 

% section wiring algorithm (end)

\input{qm2pi.ack} 

% section acknowledgments (end)

\newpage


\bibliographystyle{plain}   
\bibliography{../../biblios/main.bib}

\input{qm2pi.rhodetails}

\end{document}

 

% section concurrent_process_calculi (end)

%\documentclass[12pt]{llncs}
%\documentclass{jktr}

\usepackage[pdftex]{hyperref}                   
\usepackage {listings}
\usepackage {mathpartir}
\usepackage{bcprules}
%\usepackage{listings}
                       
\usepackage{graphicx} 
%\usepackage[margins=2.5cm,nohead,nofoot]{geometry}
%\usepackage{geometry}
\usepackage{amsfonts}
\usepackage{amstext}
\usepackage{latexsym}
\usepackage{amssymb}
\usepackage{color}


%\include{myPreamble}
\include{qm2pi.local} 

%\ifpdf
%\usepackage[pdftex]{graphicx}
%\else
%\usepackage{graphicx}
%\fi

 % \ifpdf
%  \usepackage{pdfsync}
%  \if


%\title{Brief Article}
%\author{David F. Snyder}
%\author{L.G. Meredith}

%\address{Dept. of Math., Texas State University--San Marcos, San Marcos, TX 78666}
       
\pagestyle{empty}


\begin{document}

\lstset{language=[Objective]Caml,frame=shadowbox}

\input{qm2pi.front}

% section front matter (end)

\input{qm2pi.intro} 
 
% section introduction (end)

% \input{qm2pi.knotations} 

% section notation (end)

\input{qm2pi.process.calculi} 

% section concurrent_process_calculi_and_spatial_logics_ (end)
    
%\input{qm2pi.knots2pi} 

%\input{qm2pi.trefoil} 

%\input{qm2pi.mainthm} 

% subsection basic_interpretation (end)

%\input{qm2pi.rho.presentation} 
\subsection{The syntax and semantics of the notation system}\label{sub:the_syntax_and_semantics_of_the_notation_system} % (fold)

We now summarize a technical presentation of the calculus that
embodies our theory of dynamics. The typical presentation of such a
calculus follows the style of giving generators and relations on
them. The grammar, below, describing term constructors, freely
generates the set of processes, $\Proc$. This set is then quotiented
by a relation known as structural congruence and it is over this set
that the notion of dynamics is expressed. This presentation is
essentially that of \cite{MeredithR05} with the addition of
polyadicity and summation. For readability we have relegated some of
the technical subtleties to an appendix.

\subsubsection{Process grammar}\label{subsub:process_grammar}

\begin{mathpar}
  \inferrule* [lab=synchronization] {} {{M} \bc \pzero \;|\; x?F \;|\; x!C }
  \and
  \inferrule* [lab=abstraction] {} {{F} \bc (x)P}
  \and
  \inferrule* [lab=concretion] {} {{C} \bc \langle Q \rangle}
  \and
  \inferrule* [lab=process] {} {{P,Q} \bc M \;| \;P|Q \;|\; @{x}}
  \and
  \inferrule* [lab=name] {} {{x} \bc \quotep{P}}
\end{mathpar} 

Note that $\vec{x}$ (resp. $\vec{P}$) denotes a vector of names
(resp. processes) of length $|\vec{x}|$ (resp. $|\vec{P}|$). We adopt
the following useful abbreviations.

\begin{mathpar}
   x?(\vec{y}).P := x.(\vec{y})P \and  x\clift{\vec{P}} := x.\clift{\vec{P}}
   \and x!(y) := \lift{x}{\dropn{y}}
   \and \Pi_{i=0}^{n-1}P_i := P_0 | \ldots | P_{n-1}
\end{mathpar}

\subsubsection{Structural congruence}

\paragraph{Free and bound names and alpha-equivalence.} At the
core of structural equivalence is alpha-equivalence which identifies
process that are the same up to a change of variable. Formally, we
recognize the distinction between free and bound names. The free names
of a process, $\freenames{P}$, may be calculated recursively as
follows:

\begin{mathpar}
\freenames{\pzero} := \emptyset
  \and \\
  \freenames{x?(y).P} := \{ x \} \cup (\freenames{P} \setminus \{ y \})
  \and 
  \freenames{x!\langle P \rangle} := \{ x \} \cup \{ P \} 
  \and \\
  \freenames{P|Q} := \freenames{P} \cup \freenames{Q}
  \and \\
  \freenames{@{x}} := \{ x \}
\end{mathpar}

$\pi$
$\quotep{\pi}$

$\freenames{-} : \pi \to \mathcal{P}(\quotep{\pi})$

\begin{eqnarray*}
  \freenames{\pzero} & := & \emptyset \\
  \freenames{x?(y).P} & := & \{ x \} \cup (\freenames{P} \setminus \{ y \}) \\
  \freenames{x!\langle P \rangle} & := & \{ x \} \cup \{ P \} \\
  \freenames{P|Q} & := & \freenames{P} \cup \freenames{Q} \\
  \freenames{\dropn{x}} & := & \{ x \}
\end{eqnarray*}

The bound names of a process, $\boundnames{P}$, are those names occurring in $P$
that are not free. For example, in $x?(y).0$, the name $x$ is free, while $y$ is bound.

\begin{mathpar}
  \inferrule* [lab=monoidal-laws] {} { P|Q \equiv Q|P \and P|0 \equiv P \and P|(Q|R) \equiv (P|Q)|R }
\end{mathpar}

\begin{mathpar}
  \inferrule* [lab=alpha-equivalence] {} { (x)P \equiv (y)P\{y/x\} \and y \not\in \freenames{P} }
\end{mathpar}

\begin{definition}
Then two processes, $P,Q$, are alpha-equivalent if $P = Q\{\vec{y}/\vec{x}\}$ for
some $\vec{x} \in \boundnames{Q},\vec{y} \in \boundnames{P}$, where $Q\{\vec{y}/\vec{x}\}$
denotes the capture-avoiding substitution of $\vec{y}$ for $\vec{x}$ in $Q$.
\end{definition}

\begin{definition}
  The {\em structural congruence} \cite{SangiorgiWalker} , $\equiv$,
  between processes is the least congruence containing
  alpha-equivalence, satisfying the abelian monoid laws
  (associativity, commutativity and $\pzero$ as identity) for parallel
  composition $|$ and for summation $+$.
\end{definition}

\subsection{Name equivalence}

We take name equivalence, written $\nameeq$, to be the smallest
equivalence relation generated by the following rules.

\begin{mathpar}
\inferrule*[lab=Quote-drop]
{ }
{ \quotep{@{x}} \nameeq x }

\inferrule*[lab=Struct-equiv]
{ P \scong Q }
{ \quotep{P} \nameeq \quotep{Q} }
\end{mathpar}

The astute reader will have noticed that the mutual recursion of names
and processes imposes a mutual recursion on alpha-equivalence and
structural equivalence via name-equivalence. Fortunately, all of this
works out pleasantly and we may calculate in the natural way, free of
concern. The reader interested in the details is referred to the
appendix \ref{appendix:rho_details}.

\subsection{Substitution}

We use $\Proc$ for the set of processes, $\QProc$ for the set of
names, and $\id{\{}\vec{y} / \vec{x} \id{\}}$ to denote partial maps,
$s : \QProc \rightarrow \QProc$. A map, $s$ lifts, uniquely, to a map
on process terms, $\widehat{s} : \Proc \rightarrow \Proc$ by the
following equations.

\begin{mathpar}
  (0) \psubstp{Q}{P} := 0 \\
  (R \juxtap S) \psubstp{Q}{P}
  :=    
  (R)\psubstp{Q}{P} \juxtap (S) \psubstp{Q}{P} \\
  (x?(y).R) \psubstp{Q}{P}    
  :=    
  (x)\substp{Q}{P} (z)\concat( (R \psubstn{z}{y}) \psubstp{Q}{P} ) \\
  (\lift{x}{R}) \psubstp{Q}{P}  
  :=
  \lift{(x)\substp{Q}{P}}{ R \psubstp{Q}{P} } \\
%   (\dropn{x})  \psubstp{Q}{P}       
%   := 
%   \left\{ 
%     \begin{array}{ccc} 
%       \dropn{\quotep{Q}} & & x \nameeq \quotep{P} \\
%       \dropn{x} & & otherwise \\
%     \end{array}
%   \right. 
  (\dropn{x})  \psubstp{Q}{P}       
  := 
  \left\{ 
    \begin{array}{ccc} 
      Q & & x \nameeq \quotep{P} \\
      \dropn{x} & & otherwise \\
    \end{array}
  \right.
\end{mathpar}
 

where

\begin{eqnarray}
  (x)\id{\{} \lpquote Q \rpquote / \lpquote P \rpquote \id{\}}            = 
  \left\{ 
    \begin{array}{ccc}
      \lpquote Q \rpquote & & x \nameeq \lpquote P \rpquote \\
      x & & otherwise \\
    \end{array}
  \right. \nonumber
\end{eqnarray}

and $z$ is chosen distinct from $\quotep{P}$, $\quotep{Q}$, the free
names in $Q$, and all the names in $R$. Our $\alpha$-equivalence will
be built in the standard way from this substitution.

\begin{remark}\label{rem:no_self_referential_names}
  One consequence of these definitions is that $\forall P. \quotep{P}
  \not\in \freenames{P}$.
\end{remark}

\subsection{ Dynamic quote: an example }

Anticipating something of what's to come, consider applying the
substitution, $\widehat{\id{\{}u / z \id{\}}}$, to the following pair
of processes, $\lift{w}{y!(z)}$ and $w[ \lpquote y!(z) \rpquote ]$.

\begin{eqnarray}
	\lift{w}{y!(z)}\widehat{\id{\{}u / z \id{\}}}
		& = &
		\lift{w}{y!(u)} \nonumber\\
	w[ \lpquote y!(z) \rpquote ] \widehat{ \id{\{}u / z \id{\}} }
		& = &
		w[ \lpquote y!(z) \rpquote ] \nonumber
\end{eqnarray}

Because the body of the process between quotes is impervious to
substitution, we get radically different answers. In fact, by
examining the first process in an input context,
e.g. $x?(z).\lift{w}{y!(z)}$, we see that the process under the lift
operator may be shaped by prefixed inputs binding a name inside it. In
this sense, the lift operator will be seen as a way to dynamically
construct processes before reifying them as names.

Finally equipped with these standard features we can present the
dynamics of the calculus.

\subsubsection{Operational semantics} 

Finally, we introduce the computational dynamics. What marks these
algebras as distinct from other more traditionally studied algebraic
structures, e.g. vector spaces or polynomial rings, is the manner in
which dynamics is captured. In traditional structures, dynamics is typically
expressed through morphisms between such structures, as in linear maps
between vector spaces or morphisms between rings. In algebras
associated with the semantics of computation, the dynamics is
expressed as part of the algebraic structure itself, through a
reduction reduction relation typically denoted by $\red$. Below, we
give a recursive presentation of this relation for the calculus used
in the encoding.

$\red \subseteq \pi \times \pi$
$\red : \pi \to \mathcal{P}(\pi)$

\begin{mathpar}
  \inferrule* [lab=Comm] { \textsf{match}( x_{src}, x_{trgt} ) } { x_{trgt}?(y)P \; | \; x_{src}!\langle {Q} \rangle \red P\{\quotep{Q}/y}\} }
  \and \\
  \inferrule* [lab=Par] {{P} \red {P}'} {{{P} | {Q}} \red {{P}' | {Q}}}
  \and
  \inferrule* [lab=Equiv]{{{P} \scong {P}'} \andalso {{P}' \red {Q}'} \andalso {{Q}' \scong {Q}}}{{P} \red {Q}}
\end{mathpar}

\begin{eqnarray*}
  match_{\equiv} (\quotep{P},\quotep{Q}) & := & P \equiv Q \\
  match_{\dagger}(\quotep{P},\quotep{Q}) & := & \forall R. P|Q \red^{*} R => R \red^{*} 0 \\
  match_{K}(\quotep{P},\quotep{Q}) & := & K \mbox{ for some context } K
\end{eqnarray*}

$u?(x)P | u!\langle Q \rangle \red P\{\quotep{Q}/x\}$

%We write $\wred$ for $\red^*$, and $P\red$ if $\exists Q $ such that $ P \red Q$.
We write $P\red$ if $\exists Q $ such that $ P \red Q$ and $P\not\red$, otherwise.

\section{Replication}

As mentioned before, it is known that replication (and hence
recursion) can be implemented in a higher-order process algebra
\cite{SangiorgiWalker}. As our first example of calculation with the
machinery thus far presented we give the construction explicitly in
the {\rhoc}.

\begin{eqnarray}
	D_{x} & := & \prefix{x}{y}{(\binpar{\outputp{x}{y}}{@{y}})} \nonumber\\
	\bangp_{x}{P} & := & \binpar{{x}!\langle{\binpar{D_{x}}{P}}\rangle}{D_{x}} \nonumber
\end{eqnarray}

\begin{eqnarray}
	\bangp_{x}{P} & & \nonumber\\
	=
	& {x}!\langle{(\prefix{x}{y}{(\outputp{x}{y} | @{y})) | P}}\rangle 
	      | \prefix{x}{y}{(\outputp{x}{y} | @{y})} & \nonumber\\
	\red
	& (\outputp{x}{y} | @{y})\substn{\quotep{(\prefix{x}{y}{(@{y} | \outputp{x}{y})) | P}}}{y} & \nonumber\\
	=
	& \outputp{x}{\quotep{(\prefix{x}{y}{(\outputp{x}{y} | @{y})) | P}}}
	  | {(\prefix{x}{y}{(\outputp{x}{y} | @{y})) | P}} & \nonumber\\
	\red
	& \ldots & \nonumber\\
	\red^*
	& P | P | \ldots & \nonumber
\end{eqnarray}

Of course, this encoding, as an implementation, runs away, unfolding
$\bangp{P}$ eagerly. A lazier and more implementable replication
operator, restricted to input-guarded processes, may be obtained as follows.

\begin{eqnarray}
\bangp{\prefix{u}{v}{P}} 
	:= 
	\binpar{\lift{x}{\prefix{u}{v}{(\binpar{D(x)}{P})}}}{D(x)} \nonumber
\end{eqnarray}

\begin{remark}
  Note that the lazier definition still does not deal with summation
  or mixed summation (i.e. sums over input and output). The reader is
  invited to construct definitions of replication that deal with these
  features. 

  Further, the definitions are parameterized in a name, $x$. Can you,
  gentle reader, make a definition that eliminates this parameter and
  guarantees no accidental interaction between the replication
  machinery and the process being replicated -- i.e. no accidental
  sharing of names used by the process to get its work done and the
  name(s) used by the replication to effect copying. This latter
  revision of the definition of replication is crucial to obtaining
  the expected identity $!!P \sim !P$.
\end{remark}

\begin{remark}\label{rem:paradoxical_combinator}
  The reader familiar with the lambda calculus will have noticed the
  similarity between $D$ and the paradoxical combinator.

  [Ed. note: the existence of this seems to suggest we have to be more
  restrictive on the set of processes and names we admit if we are to
  support no-cloning.]
\end{remark}

\subsubsection{Bisimulation}

The computational dynamics gives rise to another kind of equivalence,
the equivalence of computational behavior. As previously mentioned
this is typically captured \emph{via} some form of bisimulation.

% The notion we use in this paper is weak barbed bisimulation
% \cite{milner91polyadicpi}.

The notion we use in this paper is derived from weak barbed
bisimulation \cite{milner91polyadicpi}. 

\begin{definition}
An \emph{observation relation}, $\downarrow_{\mathcal N}$, over a set
of names, $\mathcal N$, is the smallest relation satisfying the rules
below.

\infrule[Out-barb]{y \in {\mathcal N}, \; x \nameeq y}
		  {\outputp{x}{v} \downarrow_{\mathcal N} x}
\infrule[Par-barb]{\mbox{$P\downarrow_{\mathcal N} x$ or $Q\downarrow_{\mathcal N} x$}}
		  {\binpar{P}{Q} \downarrow_{\mathcal N} x}

We write $P \Downarrow_{\mathcal N} x$ if there is $Q$ such that 
$P \wred Q$ and $Q \downarrow_{\mathcal N} x$.
\end{definition}

\begin{definition}
%\label{def.bbisim}
An  ${\mathcal N}$-\emph{barbed bisimulation} over a set of names, ${\mathcal N}$, is a symmetric binary relation 
${\mathcal S}_{\mathcal N}$ between agents such that $P\rel{S}_{\mathcal N}Q$ implies:
\begin{enumerate}
\item If $P \red P'$ then $Q \wred Q'$ and $P'\rel{S}_{\mathcal N} Q'$.
\item If $P\downarrow_{\mathcal N} x$, then $Q\Downarrow_{\mathcal N} x$.
\end{enumerate}
$P$ is ${\mathcal N}$-barbed bisimilar to $Q$, written
$P \wbbisim_{\mathcal N} Q$, if $P \rel{S}_{\mathcal N} Q$ for some ${\mathcal N}$-barbed bisimulation ${\mathcal S}_{\mathcal N}$.
\end{definition}

$\mathcal{R} \subseteq \pi \times \pi$

$P \mathcal{R} Q => \forall P'. P \red P' \Rightarrow \exists Q'. Q \red Q', P' \mathcal{R} Q'$

$P \vdash x \Rightarrow Q \vdash x$

\begin{mathpar}
  \inferrule*[lab=Out-barb]{x \nameeq y}{{y}!\langle{Q}\rangle \vdash x}
  \and
  \inferrule*[lab=Par-barb]{\mbox{$P\vdash x$ or $Q\vdash x$}}{\binpar{P}{Q} \vdash x}
\end{mathpar}

\subsubsection{Contexts}

One of the principle advantages of computational calculi like the
$\pi$-calculus is a well-defined notion of context,
contextual-equivalence and a correlation between
contextual-equivalence and notions of bisimulation. The notion of
context allows the decomposition of a process into (sub-)process and
its syntactic environment, its context. Thus, a context may be
thought of as a process with a ``hole'' (written $\Box$) in it. The
application of a context $M$ to a process $P$, written $M[P]$, is
tantamount to filling the hole in $M$ with $P$. In this paper we do
not need the full weight of this theory, but do make use of the notion
of context in the proof the main theorem. 

\begin{mathpar}
  \inferrule* [lab=summation] {} {{M_{M},M_{N}} \bc \Box \;|\; x.M_{A} \;|\; M_{M}+M_{N}}
  \and
  \inferrule* [lab=agent] {} {{M_{A}} \bc (\vec{x})M_{P} \;| \; \clift{P_0,\ldots,M_{P},\ldots,P_N}}
  \and \\
  \inferrule* [lab=process] {} {{M_{P}} \bc M_{N} \;| \;P|M_{P} }
\end{mathpar} 

\begin{mathpar}
  \inferrule* [lab=sychronization] {} {M_{N} \bc \Box \;|\; x?M_{F} \;|\; x!M_{C}}
  \and
  \inferrule* [lab=abstraction] {} {{M_{F}} \bc (x)M_{P} }
  \and
  \inferrule* [lab=concretion] {} {{M_{C}} \bc \langle M_{P} \rangle }
  \and \\
  \inferrule* [lab=process] {} {{M_{P}} \bc M_{N} \;| \;P|M_{P} }
\end{mathpar}

\begin{definition}[contextual application] Given a context $M$, and
  process $P$, we define the \emph{contextual application}, $M[P] :=
  M\{P/\Box\}$. That is, the contextual application of M to P is the
  substitution of $P$ for $\Box$ in $M$.
\end{definition}

$\meaningof{-} : L \to \mathcal{P}(\pi)$

\begin{mathpar}
  \inferrule* [lab=collection] {} {\meaningof{true} = \pi, \and \meaningof{~E} = \pi \setminus \meaningof{E}, \and \meaningof{E_{1} \& E_{2}} = \meaningof{E_{1}} \cap \meaningof{E_{2}}}
\end{mathpar}

\begin{mathpar}
  \inferrule* [lab=structure] {} {\meaningof{0} = \{ P \in \pi | P \equiv 0 \}, \and \\ \meaningof{E_1 | E_2} = \{ P \in \pi | P \equiv P_{1} | P_{2}, P_{1} \in \meaningof{E_{1}}, P_{2} \in \meaningof{E_2}\} }
\end{mathpar}

\begin{mathpar}
 \inferrule* [lab=behavior] {} {\meaningof{\langle a?b \rangle E} = \{ P \in \pi | P \equiv Q | u?(y)P', \\ \and \\\\ \and \\ \;\;\; u \in \meaningof{a}, \forall z.P'\{z/y\} \in \meaningof{E\{z/b\}}\}, \and \\ \meaningof{a!E} = \{ P \in \pi | P \equiv Q | x!\langle P' \rangle, x \in \meaningof{a} P' \in \meaningof{E}\} }
\end{mathpar}

\begin{mathpar}
 \inferrule* [lab=nominal] {} {\meaningof{\quotep{E}} = \{ \quotep{P} \in \quotep{\pi} | P \in \meaningof{E} \}, \and \meaningof{\quotep{P}} = \{ \quotep{Q} \in \quotep{\pi} | P \equiv Q \} \and \\ \meaningof{@\quotep{E}} = \{ P \in \pi | P \equiv @x, x \in \meaningof{E} \}}
\end{mathpar}

\begin{eqnarray*}
  \\
  \meaningof{-} : TS \to ST
\end{eqnarray*}

\begin{eqnarray*}
  \\
  L : TS \to ST
\end{eqnarray*}

\begin{eqnarray*}
  \\
  P \models E \iff P \in \meaningof{E}
\end{eqnarray*}

\begin{eqnarray*}
  P \approx_{L} Q \iff \forall E \in L. P \models E \iff Q \models E
\end{eqnarray*}

\begin{eqnarray*}
  P \approx_{K} Q
\end{eqnarray*}

\begin{eqnarray*}
  P \approx Q
\end{eqnarray*}

$\approx_{K} = \approx = \approx_{L}$

\subsubsection{Contextual duality}

Note that contexts extend the quotation operation to a family of
operations from processes to names. Given a context, $M$, we can
define a \emph{nominal context}, $\quotep{M}$ by $\quotep{M}[P] :=
\quotep{M[P]}$. To foreshadow what is to come we observe that these
operations enjoy a duality with processes very much like the duality
between vectors and maps from vectors to scalars.

Further, because the calculus is essentially higher-order, we have a
correspondence between contexts and processes. More specifically,
given a name $x$ and a context $M$ we can construct $M^{*}_{x}$ such
that 

\begin{mathpar}
  M^{*}_{x} | \lift{x}{P} \red M[P]
\end{mathpar}

namely,

\begin{mathpar}
  M^{*}_{x} := x?(u).M[\dropn{u}]
\end{mathpar}

The dependence of $M^{*}_{x}$ on a name makes it an abstraction, 

\begin{mathpar}
  M^{*} := (x)x?(u).M[\dropn{u}]
\end{mathpar}

\subsection{Additional notation}

It will sometimes be convenient to denote the process a name
quotes. We already have the notation $x = \quotep{P}$, but it will be
convenient to introduce an alternate notation, $\procn{x}$, when we
want to emphasize the connection to the use of the name. Note that, by
virtue of name equivalence, $\quotep{\procn{x}} \nameeq x$; so, the
notation is consistent with previous definitions.

Further, because names have structure it is possible to effect
substitutions on the basis of that structure. This means we need to
upgrade our notation for substitutions, which we accomplish by
adapting comprehension notation. Thus,

\begin{mathpar}
  P\{ y / x : x \in S \}
\end{mathpar}

is interpreted to mean the process derived from P by replacing (in a
capture-avoiding manner) each occurrence of $x$ in $S$ by $y$. For example,

\begin{mathpar}
  P\{ \quotep{\procn{x}|\procn{x}} / x : x \in \freenames{P} \}
\end{mathpar}

will replace each (occurrence) of a free name $x$ in $P$ by
$\quotep{\procn{x}|\procn{x}}$.

Also, we will avail ourselves of the notation $x^{L}$ and $x^{R}$ to
denote injections of a name into disjoint copies of the name
space. There are numerous ways to accomplish this. One example can be
found in \cite{MeredithR05}. This notation overloads to vectors of
names: $\vec{x}^{\pi} := (x_{i}^{\pi} \; : \; 0 \leq i < |\vec{x}| )$ where $\pi \in \{L,R\}$.

We also use $P^{\Box} := P|\Box$.

In \cite{MeredithR05} an interpretation of the new operator is
given. It turns out that there are several possible interpretations
all enjoying the requisite algebraic properties of the operator (see
\cite{milner91polyadicpi}). We will therefore make liberal use of
$(\nu\; \vec{x})P$.

% subsection the_syntax_and_semantics_of_the_notation_system (end)   

\input{qm2pi.qmops} 

\input{qm2pi.sterngerlach} 

\input{qm2pi.metric} 

% section concurrent_process_calculi (end)

%\input{qm2pi.proofsketch}

% section proof sketch (end)

%\input{qm2pi.slviaknots} 

% section spatial logic via knots (end)

\input{qm2pi.conclusion}

% section conclusion (end)

%\input{qm2pi.dtcodes} 

% section wiring algorithm (end)

\input{qm2pi.ack} 

% section acknowledgments (end)

\newpage


\bibliographystyle{plain}   
\bibliography{../../biblios/main.bib}

\input{qm2pi.rhodetails}

\end{document}



% section proof sketch (end)

%\section{Unlikely characters: spatial logic for
  knots}\label{sub:characteristic_formulae} % (fold)

Associated to the mobile process calculi are a family of logics known
as the Hennessy-Milner logics. These logics typically enjoy a
semantics interpreting formulae as sets of processes that when
factored through the encoding outlined above allows an identification
of classes of knots with logical formulae. In the context of this
encoding the sub-family known as the spatial logics \cite{CairesC03}
\cite{CairesC04} \cite{Caires04} are of particular interest providing
several important features for expressing and reasoning about
properties (i.e. classes) of knots. We hint here at how this may be done.

%\begin{description}
%\item [structural connectives] 
\subsubsection{Structural connectives} The spatial logics enjoy
structural connectives corresponding, at the logical level, to the
parallel composition ($P | Q$) and new name ($(\nu \; x)P$)
connectives for processes. As illustrated in the examples below, these
connectives are extremely expressive given the shape of our encoding.
%\item [decideable satisfaction]

\subsubsection{Decideable satisfaction}
In \cite{Caires04} the satisfaction relation is shown to be decideable
for a rich class of processes. It further turns out that the image of
the our encoding is a proper subset of that class. This result
provides the basis for an algorithm by which to search for knots
enjoying a given property.
%\item [characteristic formulae]

\subsubsection{Characteristic formulae}
In the same paper \cite{Caires04} , Caires presents a means of calculating
characteristic formulae, selecting equivalence classes of processes
up to a pre--specified depth limit on the support set of names. Composed with our
encoding, this characteristic formula can be used to select
characteristic formulae for knots.
%\end{description}

\subsubsection{Spatial logic formulae}

The grammar below (segmented for comprehension) summarizes the syntax
of spatial logic formulae. We employ illustrative examples in the
sequel to provide an intuitive understanding of their meaning
referring the reader to \cite{Caires04} for a more detailed explication
of the semantics.

\begin{mathpar}
  \inferrule* [lab=boolean] {} {{A,B} \bc T \;|\; \neg A \;|\; A \wedge B \;|\; \eta = \eta'}
  \and
  \inferrule* [lab=spatial] {} {|\; \pzero \;|\; A | B \;|\; x \text{\textregistered} A \;|\; \forall x . A \;|\;  H x . A}
  \and
  \inferrule* [lab=behavioral] {} {|\; \alpha . A}
  \and 
  \inferrule* [lab=recursion] {} {|\; X(\vec{u}) \;|\; \mu X(\vec{u}) . A}
  \and
  \inferrule* [lab=action] {} {\alpha \bc \langle x?(\vec{y}) \rangle \;|\; \langle x!(\vec{y}) \rangle \;|\; \langle \tau \rangle}
  \and 
  \inferrule* [lab=name] {} {\eta \bc x \;|\; \tau}
\end{mathpar} 

% subsection characteristic_formulae (end)   	 

\subsection{Example formulae}\label{sub:example_formulae_} % (fold)

\subsubsection{Crossing as formula.}
% 
% \begin{align*}
%   \frac{d}{dx} \sin x &= \cos x 
%   & \frac{d}{dx} e^x &= e^x \\
%   \frac{d}{dx} \cos x &= - \sin x 
%   & \frac{d}{dx} \log x &= \frac{1}{x} \\
% \end{align*} 

\begin{align*}
 \mu C(x_{0},x_{1},y_{0},y_{1},u).&(\langle x_{0}?(z) \rangle(\langle u! \rangle\langle y_{1}!z \rangle C(x_{0},x_{1},y_{0},y_{1},u)) & \\
  & \wedge \langle y_{1}?(z) \rangle (\langle u! \rangle \langle x_{0}!z \rangle C(x_{0},x_{1},y_{0},y_{1},u)) & \\
  & \wedge \langle x_{1}?(z) \rangle (\langle u? \rangle \langle y_{0}!z \rangle C(x_{0},x_{1},y_{0},y_{1},u)) & \\
  & \wedge \langle y_{0}?(z) \rangle (\langle u? \rangle \langle x_{1}!z \rangle C(x_{0},x_{1},y_{0},y_{1},u))) &
\end{align*}

The lexicographical similarity between the shape of this formulae and
the shape of definition of the process representing a crossing reveals
the intuitive meaning of this formulae. It describes the capabilities
of a process that has the right to represent a crossing. For example
it picks out processes that may perform an input on the port $x_0$ in
its initial menu of capabilities. What differentiates the formula
from the process, however, is that the crossing process is the
smallest candidate to satisfy the formula. Infinitely many other
processes -- with internal behavior hidden behind this interface, so
to speak -- also satisfy this formula. Even this simple formula,
then, can be seen to open a new view onto knots, providing a
computational interpretation of \emph{virtual} knots.

Note that this formula is derived by hand. A similar formula can be
derived by employing Caires' calculation of characteristic formula
\cite{Caires04} to the process representing a crossing. In light of
this discussion, we let
$\meaningof{C}_{\phi}(x0,x1,y0,y1,u)$ denote a formula specifying the
dynamics we wish to capture of a crossing. To guarantee we preserve
the shape of the interface and minimal semantics we demand that
$\meaningof{C}_{\phi}(x0,x1,y0,y1,u) \Rightarrow
\textbf{C}(x0,x1,y0,y1,u)$ where $\textbf{C}(x0,x1,y0,y1,u)$ denotes
the formula above.
                            
\subsubsection{Crossing number constraints.}
The moral content of the context lemma (Lemma \ref{context}) is that the notion of
``locality'' in the Reidemeister moves is effectively captured by the
parallel composition operator of the process calculus. This intuition
extends through the logic. Given a formula,
$\meaningof{C}_{\phi}(x0,x1,y0,y1,u)$, we can use the structural
connectives to specify constraints on crossing numbers, such as at
least $n$ crossings, or exactly $n$ crossings.
\begin{mathpar}
  \inferrule* [lab=at-least-n] {} { K^{\geq n}_{\phi}(\vec{xs},\vec{ys}) := \Pi_{i=0}^{n-1} Hu . \meaningof{C}_{\phi}(xs_i,ys_i,u) | T }
  \and 
  \inferrule* [lab=exactly-n] {} { K^{= n}_{\phi}(\vec{xs},\vec{ys}) := \Pi_{i=0}^{n-1} Hu . \meaningof{C}_{\phi}(xs_i,ys_i,u) | \neg (\forall x_0,y_0,x_1,y_1,u . \meaningof{C}_{\phi}(x_0,y_0,x_1,y_1,u) | T) }
\end{mathpar}

To round out this section, recall that the encoding of an $n$-crossing
knot decomposes into a parallel composition of $n$ \emph{copies} of a
crossing process together with a wiring harness. To specify different
knot classes with the same crossing number amounts to specifying
logical constraints on the wiring harness. In the interest of space,
we defer examples to a forthcoming paper. Suffice it to say that both
the conditions ``alternating knot'' and ``contains the tangle
corresponding to 5/3'' are expressible. For example, it is possible to
calculate the characteristic formula of a process corresponding to the
tangle 5/3 and conjoin it into the classifying formula via the
composition connective of the logic.

Finally, we wish to observe that it is entirely within reason to
contemplate a more domain-specific version of spatial logic tailored
to the shape of processes in the image of the encoding. Such a
domain-specific logic would have a better claim to the title formal
language of knot properties.

% subsection example_formulae_ (end)

% section knots_as_processes (end) 

% section spatial logic via knots (end)

\section{Conclusions and future work}

\paragraph{Testing physical space}
You, gentle reader, may wonder why of all the theorems to be proved
given this set up we pick the one above. In some sense it's hardly
central to quantum mechanics. We see it as central in the sense that
it firmly establishes a notion of physical space arising from a notion
of the equivalence of behavior. Relating bisimulation to a metric is a
big step forward, but one is faced with interpreting the relationship
of that metric space to something more physical. Quantum mechanical
notions of ``physical'' space are still far from intuitive, but by
relating this idea of distance as testing to calculations that predict
physical circumstances we are making a not insignificant step forward
toward an understanding of the physical space we inhabit as
essentially dynamic.

\paragraph{Effectivity and simulation}
One of the observations we have yet to make is that the entire program
spelled out here is effective. We have built various interpreters for
the reflective calculus at work in this interpretation. In principle,
then, we can simulate quantum mechanics on a computer. The place where
the simulation may lose fidelity is the infinitely branching summation
for the annihilator.

In this connection i also want to point out that the evaluation style
calculation of the inner product puts the non-determinism of the
summation right at the heart of measurement. This suggests that
Milner's original reduction-based formulation of the dynamics of his
calculi in terms of sums was not just notationally suggestive of a
notion of measure-and-continue but captured some significant part of
the physics.

\paragraph{Quantum continuations}
In light of this last observation i want to point out that the
predominant account of quantum mechanics is missing a key aspect of a
truly compositional story of the physical situation. In a real lab,
when a measurement is made the observation can be made to feed into
another device that then makes another measurement conditioned on the
results of the first. This means that after the superposition was
collapsed the entire experimental set up remained in
superposition. While QM offers a means of writing this down it doesn't
quite line up well with the well-trodden formulation of computation
and continuation that we see so succinctly expressed in Milner's
calculi. This suggests that there might be advantages to this account
of dynamics waiting to be explored.

\paragraph{Quantum logic}
In this connection, we also note that by virtue of having the
Hennessy-Milner construction, we can pull the construction through the
interpretation of QM. This gives us a natural candidate for a quantum
logic that enjoys an extremely tight connection with it's domain of
interpretation, making the construction much less ad hoc (rather it is
the image of functor!).

\paragraph{Quantum probabiity}
i have questions about the basis of the interpretation of inner
product as probability amplitude. In particular, using which
axiomatization of probability theory does the notion of probability
amplitude earn the right to be so dubbed? In other words, where is the
proof that the operation for calculating a probability amplitude (and
then squaring) satisfies the axioms of what it means to calculate a
probability? Even if such a proof exists (i have yet to find it in the
literature), i wonder if it might not be possible to turn things on
their heads. Can we view the calculation of the probability amplitude
as an axiomatization of probability? If so, then the definition we
give for calculating probability amplitude may provide the basis for
an \emph{effective} theory of probability.

\paragraph{Quantum vs ``biological'' information}
Finally, i want to conclude with a more philosophical observation. At
a recent workshop in which QM was a predominant topic i noticed
something about quantum information. The speaker was giving a riveting
discussion of axiomatic QM and showing how properties of ``no
cloning'' and ``no deleting'' emerged as consequences of the
axiomatization. Theorems of this form are necessary to give us a sense
of confidence that our axioms characterize the physical theory. What
struck me, though, was that if quantum information is neither erasable
nor replicable it is markedly different from \emph{life}. Two of the
things we know about life is that

\begin{itemize}
  \item it ends;
  \item to gain some measure of persistence, to transcend it's
    finitude it is imminently copyable.
\end{itemize}

Both of these qualities are summarized succinctly in the aphorism: all
flesh is grass. For me these two kinds of ``information'' -- call them
quantum and biological -- are end points on a spectrum of strategies
for persistence. At one end, we have those curious entities that enjoy
uniqueness and permanence; at the other, we have those who in the face
of a certain end and an uncertain present make a go of passing
something on. To me one of the more remarkable aspects of the latter
strategy is that in the presence of noise (and certain features of
copying) we get a kind of dynamism, a chance for improvement against a
given persistent condition.

% subsection other_calculi_other_bisimulations_and_geometry_as_behavior (end)




% section conclusion (end)

%\documentclass[12pt]{llncs}
%\documentclass{jktr}

\usepackage[pdftex]{hyperref}                   
\usepackage {listings}
\usepackage {mathpartir}
\usepackage{bcprules}
%\usepackage{listings}
                       
\usepackage{graphicx} 
%\usepackage[margins=2.5cm,nohead,nofoot]{geometry}
%\usepackage{geometry}
\usepackage{amsfonts}
\usepackage{amstext}
\usepackage{latexsym}
\usepackage{amssymb}
\usepackage{color}


%\include{myPreamble}
\include{qm2pi.local} 

%\ifpdf
%\usepackage[pdftex]{graphicx}
%\else
%\usepackage{graphicx}
%\fi

 % \ifpdf
%  \usepackage{pdfsync}
%  \if


%\title{Brief Article}
%\author{David F. Snyder}
%\author{L.G. Meredith}

%\address{Dept. of Math., Texas State University--San Marcos, San Marcos, TX 78666}
       
\pagestyle{empty}


\begin{document}

\lstset{language=[Objective]Caml,frame=shadowbox}

\input{qm2pi.front}

% section front matter (end)

\input{qm2pi.intro} 
 
% section introduction (end)

% \input{qm2pi.knotations} 

% section notation (end)

\input{qm2pi.process.calculi} 

% section concurrent_process_calculi_and_spatial_logics_ (end)
    
%\input{qm2pi.knots2pi} 

%\input{qm2pi.trefoil} 

%\input{qm2pi.mainthm} 

% subsection basic_interpretation (end)

%\input{qm2pi.rho.presentation} 
\subsection{The syntax and semantics of the notation system}\label{sub:the_syntax_and_semantics_of_the_notation_system} % (fold)

We now summarize a technical presentation of the calculus that
embodies our theory of dynamics. The typical presentation of such a
calculus follows the style of giving generators and relations on
them. The grammar, below, describing term constructors, freely
generates the set of processes, $\Proc$. This set is then quotiented
by a relation known as structural congruence and it is over this set
that the notion of dynamics is expressed. This presentation is
essentially that of \cite{MeredithR05} with the addition of
polyadicity and summation. For readability we have relegated some of
the technical subtleties to an appendix.

\subsubsection{Process grammar}\label{subsub:process_grammar}

\begin{mathpar}
  \inferrule* [lab=synchronization] {} {{M} \bc \pzero \;|\; x?F \;|\; x!C }
  \and
  \inferrule* [lab=abstraction] {} {{F} \bc (x)P}
  \and
  \inferrule* [lab=concretion] {} {{C} \bc \langle Q \rangle}
  \and
  \inferrule* [lab=process] {} {{P,Q} \bc M \;| \;P|Q \;|\; @{x}}
  \and
  \inferrule* [lab=name] {} {{x} \bc \quotep{P}}
\end{mathpar} 

Note that $\vec{x}$ (resp. $\vec{P}$) denotes a vector of names
(resp. processes) of length $|\vec{x}|$ (resp. $|\vec{P}|$). We adopt
the following useful abbreviations.

\begin{mathpar}
   x?(\vec{y}).P := x.(\vec{y})P \and  x\clift{\vec{P}} := x.\clift{\vec{P}}
   \and x!(y) := \lift{x}{\dropn{y}}
   \and \Pi_{i=0}^{n-1}P_i := P_0 | \ldots | P_{n-1}
\end{mathpar}

\subsubsection{Structural congruence}

\paragraph{Free and bound names and alpha-equivalence.} At the
core of structural equivalence is alpha-equivalence which identifies
process that are the same up to a change of variable. Formally, we
recognize the distinction between free and bound names. The free names
of a process, $\freenames{P}$, may be calculated recursively as
follows:

\begin{mathpar}
\freenames{\pzero} := \emptyset
  \and \\
  \freenames{x?(y).P} := \{ x \} \cup (\freenames{P} \setminus \{ y \})
  \and 
  \freenames{x!\langle P \rangle} := \{ x \} \cup \{ P \} 
  \and \\
  \freenames{P|Q} := \freenames{P} \cup \freenames{Q}
  \and \\
  \freenames{@{x}} := \{ x \}
\end{mathpar}

$\pi$
$\quotep{\pi}$

$\freenames{-} : \pi \to \mathcal{P}(\quotep{\pi})$

\begin{eqnarray*}
  \freenames{\pzero} & := & \emptyset \\
  \freenames{x?(y).P} & := & \{ x \} \cup (\freenames{P} \setminus \{ y \}) \\
  \freenames{x!\langle P \rangle} & := & \{ x \} \cup \{ P \} \\
  \freenames{P|Q} & := & \freenames{P} \cup \freenames{Q} \\
  \freenames{\dropn{x}} & := & \{ x \}
\end{eqnarray*}

The bound names of a process, $\boundnames{P}$, are those names occurring in $P$
that are not free. For example, in $x?(y).0$, the name $x$ is free, while $y$ is bound.

\begin{mathpar}
  \inferrule* [lab=monoidal-laws] {} { P|Q \equiv Q|P \and P|0 \equiv P \and P|(Q|R) \equiv (P|Q)|R }
\end{mathpar}

\begin{mathpar}
  \inferrule* [lab=alpha-equivalence] {} { (x)P \equiv (y)P\{y/x\} \and y \not\in \freenames{P} }
\end{mathpar}

\begin{definition}
Then two processes, $P,Q$, are alpha-equivalent if $P = Q\{\vec{y}/\vec{x}\}$ for
some $\vec{x} \in \boundnames{Q},\vec{y} \in \boundnames{P}$, where $Q\{\vec{y}/\vec{x}\}$
denotes the capture-avoiding substitution of $\vec{y}$ for $\vec{x}$ in $Q$.
\end{definition}

\begin{definition}
  The {\em structural congruence} \cite{SangiorgiWalker} , $\equiv$,
  between processes is the least congruence containing
  alpha-equivalence, satisfying the abelian monoid laws
  (associativity, commutativity and $\pzero$ as identity) for parallel
  composition $|$ and for summation $+$.
\end{definition}

\subsection{Name equivalence}

We take name equivalence, written $\nameeq$, to be the smallest
equivalence relation generated by the following rules.

\begin{mathpar}
\inferrule*[lab=Quote-drop]
{ }
{ \quotep{@{x}} \nameeq x }

\inferrule*[lab=Struct-equiv]
{ P \scong Q }
{ \quotep{P} \nameeq \quotep{Q} }
\end{mathpar}

The astute reader will have noticed that the mutual recursion of names
and processes imposes a mutual recursion on alpha-equivalence and
structural equivalence via name-equivalence. Fortunately, all of this
works out pleasantly and we may calculate in the natural way, free of
concern. The reader interested in the details is referred to the
appendix \ref{appendix:rho_details}.

\subsection{Substitution}

We use $\Proc$ for the set of processes, $\QProc$ for the set of
names, and $\id{\{}\vec{y} / \vec{x} \id{\}}$ to denote partial maps,
$s : \QProc \rightarrow \QProc$. A map, $s$ lifts, uniquely, to a map
on process terms, $\widehat{s} : \Proc \rightarrow \Proc$ by the
following equations.

\begin{mathpar}
  (0) \psubstp{Q}{P} := 0 \\
  (R \juxtap S) \psubstp{Q}{P}
  :=    
  (R)\psubstp{Q}{P} \juxtap (S) \psubstp{Q}{P} \\
  (x?(y).R) \psubstp{Q}{P}    
  :=    
  (x)\substp{Q}{P} (z)\concat( (R \psubstn{z}{y}) \psubstp{Q}{P} ) \\
  (\lift{x}{R}) \psubstp{Q}{P}  
  :=
  \lift{(x)\substp{Q}{P}}{ R \psubstp{Q}{P} } \\
%   (\dropn{x})  \psubstp{Q}{P}       
%   := 
%   \left\{ 
%     \begin{array}{ccc} 
%       \dropn{\quotep{Q}} & & x \nameeq \quotep{P} \\
%       \dropn{x} & & otherwise \\
%     \end{array}
%   \right. 
  (\dropn{x})  \psubstp{Q}{P}       
  := 
  \left\{ 
    \begin{array}{ccc} 
      Q & & x \nameeq \quotep{P} \\
      \dropn{x} & & otherwise \\
    \end{array}
  \right.
\end{mathpar}
 

where

\begin{eqnarray}
  (x)\id{\{} \lpquote Q \rpquote / \lpquote P \rpquote \id{\}}            = 
  \left\{ 
    \begin{array}{ccc}
      \lpquote Q \rpquote & & x \nameeq \lpquote P \rpquote \\
      x & & otherwise \\
    \end{array}
  \right. \nonumber
\end{eqnarray}

and $z$ is chosen distinct from $\quotep{P}$, $\quotep{Q}$, the free
names in $Q$, and all the names in $R$. Our $\alpha$-equivalence will
be built in the standard way from this substitution.

\begin{remark}\label{rem:no_self_referential_names}
  One consequence of these definitions is that $\forall P. \quotep{P}
  \not\in \freenames{P}$.
\end{remark}

\subsection{ Dynamic quote: an example }

Anticipating something of what's to come, consider applying the
substitution, $\widehat{\id{\{}u / z \id{\}}}$, to the following pair
of processes, $\lift{w}{y!(z)}$ and $w[ \lpquote y!(z) \rpquote ]$.

\begin{eqnarray}
	\lift{w}{y!(z)}\widehat{\id{\{}u / z \id{\}}}
		& = &
		\lift{w}{y!(u)} \nonumber\\
	w[ \lpquote y!(z) \rpquote ] \widehat{ \id{\{}u / z \id{\}} }
		& = &
		w[ \lpquote y!(z) \rpquote ] \nonumber
\end{eqnarray}

Because the body of the process between quotes is impervious to
substitution, we get radically different answers. In fact, by
examining the first process in an input context,
e.g. $x?(z).\lift{w}{y!(z)}$, we see that the process under the lift
operator may be shaped by prefixed inputs binding a name inside it. In
this sense, the lift operator will be seen as a way to dynamically
construct processes before reifying them as names.

Finally equipped with these standard features we can present the
dynamics of the calculus.

\subsubsection{Operational semantics} 

Finally, we introduce the computational dynamics. What marks these
algebras as distinct from other more traditionally studied algebraic
structures, e.g. vector spaces or polynomial rings, is the manner in
which dynamics is captured. In traditional structures, dynamics is typically
expressed through morphisms between such structures, as in linear maps
between vector spaces or morphisms between rings. In algebras
associated with the semantics of computation, the dynamics is
expressed as part of the algebraic structure itself, through a
reduction reduction relation typically denoted by $\red$. Below, we
give a recursive presentation of this relation for the calculus used
in the encoding.

$\red \subseteq \pi \times \pi$
$\red : \pi \to \mathcal{P}(\pi)$

\begin{mathpar}
  \inferrule* [lab=Comm] { \textsf{match}( x_{src}, x_{trgt} ) } { x_{trgt}?(y)P \; | \; x_{src}!\langle {Q} \rangle \red P\{\quotep{Q}/y}\} }
  \and \\
  \inferrule* [lab=Par] {{P} \red {P}'} {{{P} | {Q}} \red {{P}' | {Q}}}
  \and
  \inferrule* [lab=Equiv]{{{P} \scong {P}'} \andalso {{P}' \red {Q}'} \andalso {{Q}' \scong {Q}}}{{P} \red {Q}}
\end{mathpar}

\begin{eqnarray*}
  match_{\equiv} (\quotep{P},\quotep{Q}) & := & P \equiv Q \\
  match_{\dagger}(\quotep{P},\quotep{Q}) & := & \forall R. P|Q \red^{*} R => R \red^{*} 0 \\
  match_{K}(\quotep{P},\quotep{Q}) & := & K \mbox{ for some context } K
\end{eqnarray*}

$u?(x)P | u!\langle Q \rangle \red P\{\quotep{Q}/x\}$

%We write $\wred$ for $\red^*$, and $P\red$ if $\exists Q $ such that $ P \red Q$.
We write $P\red$ if $\exists Q $ such that $ P \red Q$ and $P\not\red$, otherwise.

\section{Replication}

As mentioned before, it is known that replication (and hence
recursion) can be implemented in a higher-order process algebra
\cite{SangiorgiWalker}. As our first example of calculation with the
machinery thus far presented we give the construction explicitly in
the {\rhoc}.

\begin{eqnarray}
	D_{x} & := & \prefix{x}{y}{(\binpar{\outputp{x}{y}}{@{y}})} \nonumber\\
	\bangp_{x}{P} & := & \binpar{{x}!\langle{\binpar{D_{x}}{P}}\rangle}{D_{x}} \nonumber
\end{eqnarray}

\begin{eqnarray}
	\bangp_{x}{P} & & \nonumber\\
	=
	& {x}!\langle{(\prefix{x}{y}{(\outputp{x}{y} | @{y})) | P}}\rangle 
	      | \prefix{x}{y}{(\outputp{x}{y} | @{y})} & \nonumber\\
	\red
	& (\outputp{x}{y} | @{y})\substn{\quotep{(\prefix{x}{y}{(@{y} | \outputp{x}{y})) | P}}}{y} & \nonumber\\
	=
	& \outputp{x}{\quotep{(\prefix{x}{y}{(\outputp{x}{y} | @{y})) | P}}}
	  | {(\prefix{x}{y}{(\outputp{x}{y} | @{y})) | P}} & \nonumber\\
	\red
	& \ldots & \nonumber\\
	\red^*
	& P | P | \ldots & \nonumber
\end{eqnarray}

Of course, this encoding, as an implementation, runs away, unfolding
$\bangp{P}$ eagerly. A lazier and more implementable replication
operator, restricted to input-guarded processes, may be obtained as follows.

\begin{eqnarray}
\bangp{\prefix{u}{v}{P}} 
	:= 
	\binpar{\lift{x}{\prefix{u}{v}{(\binpar{D(x)}{P})}}}{D(x)} \nonumber
\end{eqnarray}

\begin{remark}
  Note that the lazier definition still does not deal with summation
  or mixed summation (i.e. sums over input and output). The reader is
  invited to construct definitions of replication that deal with these
  features. 

  Further, the definitions are parameterized in a name, $x$. Can you,
  gentle reader, make a definition that eliminates this parameter and
  guarantees no accidental interaction between the replication
  machinery and the process being replicated -- i.e. no accidental
  sharing of names used by the process to get its work done and the
  name(s) used by the replication to effect copying. This latter
  revision of the definition of replication is crucial to obtaining
  the expected identity $!!P \sim !P$.
\end{remark}

\begin{remark}\label{rem:paradoxical_combinator}
  The reader familiar with the lambda calculus will have noticed the
  similarity between $D$ and the paradoxical combinator.

  [Ed. note: the existence of this seems to suggest we have to be more
  restrictive on the set of processes and names we admit if we are to
  support no-cloning.]
\end{remark}

\subsubsection{Bisimulation}

The computational dynamics gives rise to another kind of equivalence,
the equivalence of computational behavior. As previously mentioned
this is typically captured \emph{via} some form of bisimulation.

% The notion we use in this paper is weak barbed bisimulation
% \cite{milner91polyadicpi}.

The notion we use in this paper is derived from weak barbed
bisimulation \cite{milner91polyadicpi}. 

\begin{definition}
An \emph{observation relation}, $\downarrow_{\mathcal N}$, over a set
of names, $\mathcal N$, is the smallest relation satisfying the rules
below.

\infrule[Out-barb]{y \in {\mathcal N}, \; x \nameeq y}
		  {\outputp{x}{v} \downarrow_{\mathcal N} x}
\infrule[Par-barb]{\mbox{$P\downarrow_{\mathcal N} x$ or $Q\downarrow_{\mathcal N} x$}}
		  {\binpar{P}{Q} \downarrow_{\mathcal N} x}

We write $P \Downarrow_{\mathcal N} x$ if there is $Q$ such that 
$P \wred Q$ and $Q \downarrow_{\mathcal N} x$.
\end{definition}

\begin{definition}
%\label{def.bbisim}
An  ${\mathcal N}$-\emph{barbed bisimulation} over a set of names, ${\mathcal N}$, is a symmetric binary relation 
${\mathcal S}_{\mathcal N}$ between agents such that $P\rel{S}_{\mathcal N}Q$ implies:
\begin{enumerate}
\item If $P \red P'$ then $Q \wred Q'$ and $P'\rel{S}_{\mathcal N} Q'$.
\item If $P\downarrow_{\mathcal N} x$, then $Q\Downarrow_{\mathcal N} x$.
\end{enumerate}
$P$ is ${\mathcal N}$-barbed bisimilar to $Q$, written
$P \wbbisim_{\mathcal N} Q$, if $P \rel{S}_{\mathcal N} Q$ for some ${\mathcal N}$-barbed bisimulation ${\mathcal S}_{\mathcal N}$.
\end{definition}

$\mathcal{R} \subseteq \pi \times \pi$

$P \mathcal{R} Q => \forall P'. P \red P' \Rightarrow \exists Q'. Q \red Q', P' \mathcal{R} Q'$

$P \vdash x \Rightarrow Q \vdash x$

\begin{mathpar}
  \inferrule*[lab=Out-barb]{x \nameeq y}{{y}!\langle{Q}\rangle \vdash x}
  \and
  \inferrule*[lab=Par-barb]{\mbox{$P\vdash x$ or $Q\vdash x$}}{\binpar{P}{Q} \vdash x}
\end{mathpar}

\subsubsection{Contexts}

One of the principle advantages of computational calculi like the
$\pi$-calculus is a well-defined notion of context,
contextual-equivalence and a correlation between
contextual-equivalence and notions of bisimulation. The notion of
context allows the decomposition of a process into (sub-)process and
its syntactic environment, its context. Thus, a context may be
thought of as a process with a ``hole'' (written $\Box$) in it. The
application of a context $M$ to a process $P$, written $M[P]$, is
tantamount to filling the hole in $M$ with $P$. In this paper we do
not need the full weight of this theory, but do make use of the notion
of context in the proof the main theorem. 

\begin{mathpar}
  \inferrule* [lab=summation] {} {{M_{M},M_{N}} \bc \Box \;|\; x.M_{A} \;|\; M_{M}+M_{N}}
  \and
  \inferrule* [lab=agent] {} {{M_{A}} \bc (\vec{x})M_{P} \;| \; \clift{P_0,\ldots,M_{P},\ldots,P_N}}
  \and \\
  \inferrule* [lab=process] {} {{M_{P}} \bc M_{N} \;| \;P|M_{P} }
\end{mathpar} 

\begin{mathpar}
  \inferrule* [lab=sychronization] {} {M_{N} \bc \Box \;|\; x?M_{F} \;|\; x!M_{C}}
  \and
  \inferrule* [lab=abstraction] {} {{M_{F}} \bc (x)M_{P} }
  \and
  \inferrule* [lab=concretion] {} {{M_{C}} \bc \langle M_{P} \rangle }
  \and \\
  \inferrule* [lab=process] {} {{M_{P}} \bc M_{N} \;| \;P|M_{P} }
\end{mathpar}

\begin{definition}[contextual application] Given a context $M$, and
  process $P$, we define the \emph{contextual application}, $M[P] :=
  M\{P/\Box\}$. That is, the contextual application of M to P is the
  substitution of $P$ for $\Box$ in $M$.
\end{definition}

$\meaningof{-} : L \to \mathcal{P}(\pi)$

\begin{mathpar}
  \inferrule* [lab=collection] {} {\meaningof{true} = \pi, \and \meaningof{~E} = \pi \setminus \meaningof{E}, \and \meaningof{E_{1} \& E_{2}} = \meaningof{E_{1}} \cap \meaningof{E_{2}}}
\end{mathpar}

\begin{mathpar}
  \inferrule* [lab=structure] {} {\meaningof{0} = \{ P \in \pi | P \equiv 0 \}, \and \\ \meaningof{E_1 | E_2} = \{ P \in \pi | P \equiv P_{1} | P_{2}, P_{1} \in \meaningof{E_{1}}, P_{2} \in \meaningof{E_2}\} }
\end{mathpar}

\begin{mathpar}
 \inferrule* [lab=behavior] {} {\meaningof{\langle a?b \rangle E} = \{ P \in \pi | P \equiv Q | u?(y)P', \\ \and \\\\ \and \\ \;\;\; u \in \meaningof{a}, \forall z.P'\{z/y\} \in \meaningof{E\{z/b\}}\}, \and \\ \meaningof{a!E} = \{ P \in \pi | P \equiv Q | x!\langle P' \rangle, x \in \meaningof{a} P' \in \meaningof{E}\} }
\end{mathpar}

\begin{mathpar}
 \inferrule* [lab=nominal] {} {\meaningof{\quotep{E}} = \{ \quotep{P} \in \quotep{\pi} | P \in \meaningof{E} \}, \and \meaningof{\quotep{P}} = \{ \quotep{Q} \in \quotep{\pi} | P \equiv Q \} \and \\ \meaningof{@\quotep{E}} = \{ P \in \pi | P \equiv @x, x \in \meaningof{E} \}}
\end{mathpar}

\begin{eqnarray*}
  \\
  \meaningof{-} : TS \to ST
\end{eqnarray*}

\begin{eqnarray*}
  \\
  L : TS \to ST
\end{eqnarray*}

\begin{eqnarray*}
  \\
  P \models E \iff P \in \meaningof{E}
\end{eqnarray*}

\begin{eqnarray*}
  P \approx_{L} Q \iff \forall E \in L. P \models E \iff Q \models E
\end{eqnarray*}

\begin{eqnarray*}
  P \approx_{K} Q
\end{eqnarray*}

\begin{eqnarray*}
  P \approx Q
\end{eqnarray*}

$\approx_{K} = \approx = \approx_{L}$

\subsubsection{Contextual duality}

Note that contexts extend the quotation operation to a family of
operations from processes to names. Given a context, $M$, we can
define a \emph{nominal context}, $\quotep{M}$ by $\quotep{M}[P] :=
\quotep{M[P]}$. To foreshadow what is to come we observe that these
operations enjoy a duality with processes very much like the duality
between vectors and maps from vectors to scalars.

Further, because the calculus is essentially higher-order, we have a
correspondence between contexts and processes. More specifically,
given a name $x$ and a context $M$ we can construct $M^{*}_{x}$ such
that 

\begin{mathpar}
  M^{*}_{x} | \lift{x}{P} \red M[P]
\end{mathpar}

namely,

\begin{mathpar}
  M^{*}_{x} := x?(u).M[\dropn{u}]
\end{mathpar}

The dependence of $M^{*}_{x}$ on a name makes it an abstraction, 

\begin{mathpar}
  M^{*} := (x)x?(u).M[\dropn{u}]
\end{mathpar}

\subsection{Additional notation}

It will sometimes be convenient to denote the process a name
quotes. We already have the notation $x = \quotep{P}$, but it will be
convenient to introduce an alternate notation, $\procn{x}$, when we
want to emphasize the connection to the use of the name. Note that, by
virtue of name equivalence, $\quotep{\procn{x}} \nameeq x$; so, the
notation is consistent with previous definitions.

Further, because names have structure it is possible to effect
substitutions on the basis of that structure. This means we need to
upgrade our notation for substitutions, which we accomplish by
adapting comprehension notation. Thus,

\begin{mathpar}
  P\{ y / x : x \in S \}
\end{mathpar}

is interpreted to mean the process derived from P by replacing (in a
capture-avoiding manner) each occurrence of $x$ in $S$ by $y$. For example,

\begin{mathpar}
  P\{ \quotep{\procn{x}|\procn{x}} / x : x \in \freenames{P} \}
\end{mathpar}

will replace each (occurrence) of a free name $x$ in $P$ by
$\quotep{\procn{x}|\procn{x}}$.

Also, we will avail ourselves of the notation $x^{L}$ and $x^{R}$ to
denote injections of a name into disjoint copies of the name
space. There are numerous ways to accomplish this. One example can be
found in \cite{MeredithR05}. This notation overloads to vectors of
names: $\vec{x}^{\pi} := (x_{i}^{\pi} \; : \; 0 \leq i < |\vec{x}| )$ where $\pi \in \{L,R\}$.

We also use $P^{\Box} := P|\Box$.

In \cite{MeredithR05} an interpretation of the new operator is
given. It turns out that there are several possible interpretations
all enjoying the requisite algebraic properties of the operator (see
\cite{milner91polyadicpi}). We will therefore make liberal use of
$(\nu\; \vec{x})P$.

% subsection the_syntax_and_semantics_of_the_notation_system (end)   

\input{qm2pi.qmops} 

\input{qm2pi.sterngerlach} 

\input{qm2pi.metric} 

% section concurrent_process_calculi (end)

%\input{qm2pi.proofsketch}

% section proof sketch (end)

%\input{qm2pi.slviaknots} 

% section spatial logic via knots (end)

\input{qm2pi.conclusion}

% section conclusion (end)

%\input{qm2pi.dtcodes} 

% section wiring algorithm (end)

\input{qm2pi.ack} 

% section acknowledgments (end)

\newpage


\bibliographystyle{plain}   
\bibliography{../../biblios/main.bib}

\input{qm2pi.rhodetails}

\end{document}

 

% section wiring algorithm (end)

\documentclass[12pt]{llncs}
%\documentclass{jktr}

\usepackage[pdftex]{hyperref}                   
\usepackage {listings}
\usepackage {mathpartir}
\usepackage{bcprules}
%\usepackage{listings}
                       
\usepackage{graphicx} 
%\usepackage[margins=2.5cm,nohead,nofoot]{geometry}
%\usepackage{geometry}
\usepackage{amsfonts}
\usepackage{amstext}
\usepackage{latexsym}
\usepackage{amssymb}
\usepackage{color}


%\include{myPreamble}
\include{qm2pi.local} 

%\ifpdf
%\usepackage[pdftex]{graphicx}
%\else
%\usepackage{graphicx}
%\fi

 % \ifpdf
%  \usepackage{pdfsync}
%  \if


%\title{Brief Article}
%\author{David F. Snyder}
%\author{L.G. Meredith}

%\address{Dept. of Math., Texas State University--San Marcos, San Marcos, TX 78666}
       
\pagestyle{empty}


\begin{document}

\lstset{language=[Objective]Caml,frame=shadowbox}

\input{qm2pi.front}

% section front matter (end)

\input{qm2pi.intro} 
 
% section introduction (end)

% \input{qm2pi.knotations} 

% section notation (end)

\input{qm2pi.process.calculi} 

% section concurrent_process_calculi_and_spatial_logics_ (end)
    
%\input{qm2pi.knots2pi} 

%\input{qm2pi.trefoil} 

%\input{qm2pi.mainthm} 

% subsection basic_interpretation (end)

%\input{qm2pi.rho.presentation} 
\subsection{The syntax and semantics of the notation system}\label{sub:the_syntax_and_semantics_of_the_notation_system} % (fold)

We now summarize a technical presentation of the calculus that
embodies our theory of dynamics. The typical presentation of such a
calculus follows the style of giving generators and relations on
them. The grammar, below, describing term constructors, freely
generates the set of processes, $\Proc$. This set is then quotiented
by a relation known as structural congruence and it is over this set
that the notion of dynamics is expressed. This presentation is
essentially that of \cite{MeredithR05} with the addition of
polyadicity and summation. For readability we have relegated some of
the technical subtleties to an appendix.

\subsubsection{Process grammar}\label{subsub:process_grammar}

\begin{mathpar}
  \inferrule* [lab=synchronization] {} {{M} \bc \pzero \;|\; x?F \;|\; x!C }
  \and
  \inferrule* [lab=abstraction] {} {{F} \bc (x)P}
  \and
  \inferrule* [lab=concretion] {} {{C} \bc \langle Q \rangle}
  \and
  \inferrule* [lab=process] {} {{P,Q} \bc M \;| \;P|Q \;|\; @{x}}
  \and
  \inferrule* [lab=name] {} {{x} \bc \quotep{P}}
\end{mathpar} 

Note that $\vec{x}$ (resp. $\vec{P}$) denotes a vector of names
(resp. processes) of length $|\vec{x}|$ (resp. $|\vec{P}|$). We adopt
the following useful abbreviations.

\begin{mathpar}
   x?(\vec{y}).P := x.(\vec{y})P \and  x\clift{\vec{P}} := x.\clift{\vec{P}}
   \and x!(y) := \lift{x}{\dropn{y}}
   \and \Pi_{i=0}^{n-1}P_i := P_0 | \ldots | P_{n-1}
\end{mathpar}

\subsubsection{Structural congruence}

\paragraph{Free and bound names and alpha-equivalence.} At the
core of structural equivalence is alpha-equivalence which identifies
process that are the same up to a change of variable. Formally, we
recognize the distinction between free and bound names. The free names
of a process, $\freenames{P}$, may be calculated recursively as
follows:

\begin{mathpar}
\freenames{\pzero} := \emptyset
  \and \\
  \freenames{x?(y).P} := \{ x \} \cup (\freenames{P} \setminus \{ y \})
  \and 
  \freenames{x!\langle P \rangle} := \{ x \} \cup \{ P \} 
  \and \\
  \freenames{P|Q} := \freenames{P} \cup \freenames{Q}
  \and \\
  \freenames{@{x}} := \{ x \}
\end{mathpar}

$\pi$
$\quotep{\pi}$

$\freenames{-} : \pi \to \mathcal{P}(\quotep{\pi})$

\begin{eqnarray*}
  \freenames{\pzero} & := & \emptyset \\
  \freenames{x?(y).P} & := & \{ x \} \cup (\freenames{P} \setminus \{ y \}) \\
  \freenames{x!\langle P \rangle} & := & \{ x \} \cup \{ P \} \\
  \freenames{P|Q} & := & \freenames{P} \cup \freenames{Q} \\
  \freenames{\dropn{x}} & := & \{ x \}
\end{eqnarray*}

The bound names of a process, $\boundnames{P}$, are those names occurring in $P$
that are not free. For example, in $x?(y).0$, the name $x$ is free, while $y$ is bound.

\begin{mathpar}
  \inferrule* [lab=monoidal-laws] {} { P|Q \equiv Q|P \and P|0 \equiv P \and P|(Q|R) \equiv (P|Q)|R }
\end{mathpar}

\begin{mathpar}
  \inferrule* [lab=alpha-equivalence] {} { (x)P \equiv (y)P\{y/x\} \and y \not\in \freenames{P} }
\end{mathpar}

\begin{definition}
Then two processes, $P,Q$, are alpha-equivalent if $P = Q\{\vec{y}/\vec{x}\}$ for
some $\vec{x} \in \boundnames{Q},\vec{y} \in \boundnames{P}$, where $Q\{\vec{y}/\vec{x}\}$
denotes the capture-avoiding substitution of $\vec{y}$ for $\vec{x}$ in $Q$.
\end{definition}

\begin{definition}
  The {\em structural congruence} \cite{SangiorgiWalker} , $\equiv$,
  between processes is the least congruence containing
  alpha-equivalence, satisfying the abelian monoid laws
  (associativity, commutativity and $\pzero$ as identity) for parallel
  composition $|$ and for summation $+$.
\end{definition}

\subsection{Name equivalence}

We take name equivalence, written $\nameeq$, to be the smallest
equivalence relation generated by the following rules.

\begin{mathpar}
\inferrule*[lab=Quote-drop]
{ }
{ \quotep{@{x}} \nameeq x }

\inferrule*[lab=Struct-equiv]
{ P \scong Q }
{ \quotep{P} \nameeq \quotep{Q} }
\end{mathpar}

The astute reader will have noticed that the mutual recursion of names
and processes imposes a mutual recursion on alpha-equivalence and
structural equivalence via name-equivalence. Fortunately, all of this
works out pleasantly and we may calculate in the natural way, free of
concern. The reader interested in the details is referred to the
appendix \ref{appendix:rho_details}.

\subsection{Substitution}

We use $\Proc$ for the set of processes, $\QProc$ for the set of
names, and $\id{\{}\vec{y} / \vec{x} \id{\}}$ to denote partial maps,
$s : \QProc \rightarrow \QProc$. A map, $s$ lifts, uniquely, to a map
on process terms, $\widehat{s} : \Proc \rightarrow \Proc$ by the
following equations.

\begin{mathpar}
  (0) \psubstp{Q}{P} := 0 \\
  (R \juxtap S) \psubstp{Q}{P}
  :=    
  (R)\psubstp{Q}{P} \juxtap (S) \psubstp{Q}{P} \\
  (x?(y).R) \psubstp{Q}{P}    
  :=    
  (x)\substp{Q}{P} (z)\concat( (R \psubstn{z}{y}) \psubstp{Q}{P} ) \\
  (\lift{x}{R}) \psubstp{Q}{P}  
  :=
  \lift{(x)\substp{Q}{P}}{ R \psubstp{Q}{P} } \\
%   (\dropn{x})  \psubstp{Q}{P}       
%   := 
%   \left\{ 
%     \begin{array}{ccc} 
%       \dropn{\quotep{Q}} & & x \nameeq \quotep{P} \\
%       \dropn{x} & & otherwise \\
%     \end{array}
%   \right. 
  (\dropn{x})  \psubstp{Q}{P}       
  := 
  \left\{ 
    \begin{array}{ccc} 
      Q & & x \nameeq \quotep{P} \\
      \dropn{x} & & otherwise \\
    \end{array}
  \right.
\end{mathpar}
 

where

\begin{eqnarray}
  (x)\id{\{} \lpquote Q \rpquote / \lpquote P \rpquote \id{\}}            = 
  \left\{ 
    \begin{array}{ccc}
      \lpquote Q \rpquote & & x \nameeq \lpquote P \rpquote \\
      x & & otherwise \\
    \end{array}
  \right. \nonumber
\end{eqnarray}

and $z$ is chosen distinct from $\quotep{P}$, $\quotep{Q}$, the free
names in $Q$, and all the names in $R$. Our $\alpha$-equivalence will
be built in the standard way from this substitution.

\begin{remark}\label{rem:no_self_referential_names}
  One consequence of these definitions is that $\forall P. \quotep{P}
  \not\in \freenames{P}$.
\end{remark}

\subsection{ Dynamic quote: an example }

Anticipating something of what's to come, consider applying the
substitution, $\widehat{\id{\{}u / z \id{\}}}$, to the following pair
of processes, $\lift{w}{y!(z)}$ and $w[ \lpquote y!(z) \rpquote ]$.

\begin{eqnarray}
	\lift{w}{y!(z)}\widehat{\id{\{}u / z \id{\}}}
		& = &
		\lift{w}{y!(u)} \nonumber\\
	w[ \lpquote y!(z) \rpquote ] \widehat{ \id{\{}u / z \id{\}} }
		& = &
		w[ \lpquote y!(z) \rpquote ] \nonumber
\end{eqnarray}

Because the body of the process between quotes is impervious to
substitution, we get radically different answers. In fact, by
examining the first process in an input context,
e.g. $x?(z).\lift{w}{y!(z)}$, we see that the process under the lift
operator may be shaped by prefixed inputs binding a name inside it. In
this sense, the lift operator will be seen as a way to dynamically
construct processes before reifying them as names.

Finally equipped with these standard features we can present the
dynamics of the calculus.

\subsubsection{Operational semantics} 

Finally, we introduce the computational dynamics. What marks these
algebras as distinct from other more traditionally studied algebraic
structures, e.g. vector spaces or polynomial rings, is the manner in
which dynamics is captured. In traditional structures, dynamics is typically
expressed through morphisms between such structures, as in linear maps
between vector spaces or morphisms between rings. In algebras
associated with the semantics of computation, the dynamics is
expressed as part of the algebraic structure itself, through a
reduction reduction relation typically denoted by $\red$. Below, we
give a recursive presentation of this relation for the calculus used
in the encoding.

$\red \subseteq \pi \times \pi$
$\red : \pi \to \mathcal{P}(\pi)$

\begin{mathpar}
  \inferrule* [lab=Comm] { \textsf{match}( x_{src}, x_{trgt} ) } { x_{trgt}?(y)P \; | \; x_{src}!\langle {Q} \rangle \red P\{\quotep{Q}/y}\} }
  \and \\
  \inferrule* [lab=Par] {{P} \red {P}'} {{{P} | {Q}} \red {{P}' | {Q}}}
  \and
  \inferrule* [lab=Equiv]{{{P} \scong {P}'} \andalso {{P}' \red {Q}'} \andalso {{Q}' \scong {Q}}}{{P} \red {Q}}
\end{mathpar}

\begin{eqnarray*}
  match_{\equiv} (\quotep{P},\quotep{Q}) & := & P \equiv Q \\
  match_{\dagger}(\quotep{P},\quotep{Q}) & := & \forall R. P|Q \red^{*} R => R \red^{*} 0 \\
  match_{K}(\quotep{P},\quotep{Q}) & := & K \mbox{ for some context } K
\end{eqnarray*}

$u?(x)P | u!\langle Q \rangle \red P\{\quotep{Q}/x\}$

%We write $\wred$ for $\red^*$, and $P\red$ if $\exists Q $ such that $ P \red Q$.
We write $P\red$ if $\exists Q $ such that $ P \red Q$ and $P\not\red$, otherwise.

\section{Replication}

As mentioned before, it is known that replication (and hence
recursion) can be implemented in a higher-order process algebra
\cite{SangiorgiWalker}. As our first example of calculation with the
machinery thus far presented we give the construction explicitly in
the {\rhoc}.

\begin{eqnarray}
	D_{x} & := & \prefix{x}{y}{(\binpar{\outputp{x}{y}}{@{y}})} \nonumber\\
	\bangp_{x}{P} & := & \binpar{{x}!\langle{\binpar{D_{x}}{P}}\rangle}{D_{x}} \nonumber
\end{eqnarray}

\begin{eqnarray}
	\bangp_{x}{P} & & \nonumber\\
	=
	& {x}!\langle{(\prefix{x}{y}{(\outputp{x}{y} | @{y})) | P}}\rangle 
	      | \prefix{x}{y}{(\outputp{x}{y} | @{y})} & \nonumber\\
	\red
	& (\outputp{x}{y} | @{y})\substn{\quotep{(\prefix{x}{y}{(@{y} | \outputp{x}{y})) | P}}}{y} & \nonumber\\
	=
	& \outputp{x}{\quotep{(\prefix{x}{y}{(\outputp{x}{y} | @{y})) | P}}}
	  | {(\prefix{x}{y}{(\outputp{x}{y} | @{y})) | P}} & \nonumber\\
	\red
	& \ldots & \nonumber\\
	\red^*
	& P | P | \ldots & \nonumber
\end{eqnarray}

Of course, this encoding, as an implementation, runs away, unfolding
$\bangp{P}$ eagerly. A lazier and more implementable replication
operator, restricted to input-guarded processes, may be obtained as follows.

\begin{eqnarray}
\bangp{\prefix{u}{v}{P}} 
	:= 
	\binpar{\lift{x}{\prefix{u}{v}{(\binpar{D(x)}{P})}}}{D(x)} \nonumber
\end{eqnarray}

\begin{remark}
  Note that the lazier definition still does not deal with summation
  or mixed summation (i.e. sums over input and output). The reader is
  invited to construct definitions of replication that deal with these
  features. 

  Further, the definitions are parameterized in a name, $x$. Can you,
  gentle reader, make a definition that eliminates this parameter and
  guarantees no accidental interaction between the replication
  machinery and the process being replicated -- i.e. no accidental
  sharing of names used by the process to get its work done and the
  name(s) used by the replication to effect copying. This latter
  revision of the definition of replication is crucial to obtaining
  the expected identity $!!P \sim !P$.
\end{remark}

\begin{remark}\label{rem:paradoxical_combinator}
  The reader familiar with the lambda calculus will have noticed the
  similarity between $D$ and the paradoxical combinator.

  [Ed. note: the existence of this seems to suggest we have to be more
  restrictive on the set of processes and names we admit if we are to
  support no-cloning.]
\end{remark}

\subsubsection{Bisimulation}

The computational dynamics gives rise to another kind of equivalence,
the equivalence of computational behavior. As previously mentioned
this is typically captured \emph{via} some form of bisimulation.

% The notion we use in this paper is weak barbed bisimulation
% \cite{milner91polyadicpi}.

The notion we use in this paper is derived from weak barbed
bisimulation \cite{milner91polyadicpi}. 

\begin{definition}
An \emph{observation relation}, $\downarrow_{\mathcal N}$, over a set
of names, $\mathcal N$, is the smallest relation satisfying the rules
below.

\infrule[Out-barb]{y \in {\mathcal N}, \; x \nameeq y}
		  {\outputp{x}{v} \downarrow_{\mathcal N} x}
\infrule[Par-barb]{\mbox{$P\downarrow_{\mathcal N} x$ or $Q\downarrow_{\mathcal N} x$}}
		  {\binpar{P}{Q} \downarrow_{\mathcal N} x}

We write $P \Downarrow_{\mathcal N} x$ if there is $Q$ such that 
$P \wred Q$ and $Q \downarrow_{\mathcal N} x$.
\end{definition}

\begin{definition}
%\label{def.bbisim}
An  ${\mathcal N}$-\emph{barbed bisimulation} over a set of names, ${\mathcal N}$, is a symmetric binary relation 
${\mathcal S}_{\mathcal N}$ between agents such that $P\rel{S}_{\mathcal N}Q$ implies:
\begin{enumerate}
\item If $P \red P'$ then $Q \wred Q'$ and $P'\rel{S}_{\mathcal N} Q'$.
\item If $P\downarrow_{\mathcal N} x$, then $Q\Downarrow_{\mathcal N} x$.
\end{enumerate}
$P$ is ${\mathcal N}$-barbed bisimilar to $Q$, written
$P \wbbisim_{\mathcal N} Q$, if $P \rel{S}_{\mathcal N} Q$ for some ${\mathcal N}$-barbed bisimulation ${\mathcal S}_{\mathcal N}$.
\end{definition}

$\mathcal{R} \subseteq \pi \times \pi$

$P \mathcal{R} Q => \forall P'. P \red P' \Rightarrow \exists Q'. Q \red Q', P' \mathcal{R} Q'$

$P \vdash x \Rightarrow Q \vdash x$

\begin{mathpar}
  \inferrule*[lab=Out-barb]{x \nameeq y}{{y}!\langle{Q}\rangle \vdash x}
  \and
  \inferrule*[lab=Par-barb]{\mbox{$P\vdash x$ or $Q\vdash x$}}{\binpar{P}{Q} \vdash x}
\end{mathpar}

\subsubsection{Contexts}

One of the principle advantages of computational calculi like the
$\pi$-calculus is a well-defined notion of context,
contextual-equivalence and a correlation between
contextual-equivalence and notions of bisimulation. The notion of
context allows the decomposition of a process into (sub-)process and
its syntactic environment, its context. Thus, a context may be
thought of as a process with a ``hole'' (written $\Box$) in it. The
application of a context $M$ to a process $P$, written $M[P]$, is
tantamount to filling the hole in $M$ with $P$. In this paper we do
not need the full weight of this theory, but do make use of the notion
of context in the proof the main theorem. 

\begin{mathpar}
  \inferrule* [lab=summation] {} {{M_{M},M_{N}} \bc \Box \;|\; x.M_{A} \;|\; M_{M}+M_{N}}
  \and
  \inferrule* [lab=agent] {} {{M_{A}} \bc (\vec{x})M_{P} \;| \; \clift{P_0,\ldots,M_{P},\ldots,P_N}}
  \and \\
  \inferrule* [lab=process] {} {{M_{P}} \bc M_{N} \;| \;P|M_{P} }
\end{mathpar} 

\begin{mathpar}
  \inferrule* [lab=sychronization] {} {M_{N} \bc \Box \;|\; x?M_{F} \;|\; x!M_{C}}
  \and
  \inferrule* [lab=abstraction] {} {{M_{F}} \bc (x)M_{P} }
  \and
  \inferrule* [lab=concretion] {} {{M_{C}} \bc \langle M_{P} \rangle }
  \and \\
  \inferrule* [lab=process] {} {{M_{P}} \bc M_{N} \;| \;P|M_{P} }
\end{mathpar}

\begin{definition}[contextual application] Given a context $M$, and
  process $P$, we define the \emph{contextual application}, $M[P] :=
  M\{P/\Box\}$. That is, the contextual application of M to P is the
  substitution of $P$ for $\Box$ in $M$.
\end{definition}

$\meaningof{-} : L \to \mathcal{P}(\pi)$

\begin{mathpar}
  \inferrule* [lab=collection] {} {\meaningof{true} = \pi, \and \meaningof{~E} = \pi \setminus \meaningof{E}, \and \meaningof{E_{1} \& E_{2}} = \meaningof{E_{1}} \cap \meaningof{E_{2}}}
\end{mathpar}

\begin{mathpar}
  \inferrule* [lab=structure] {} {\meaningof{0} = \{ P \in \pi | P \equiv 0 \}, \and \\ \meaningof{E_1 | E_2} = \{ P \in \pi | P \equiv P_{1} | P_{2}, P_{1} \in \meaningof{E_{1}}, P_{2} \in \meaningof{E_2}\} }
\end{mathpar}

\begin{mathpar}
 \inferrule* [lab=behavior] {} {\meaningof{\langle a?b \rangle E} = \{ P \in \pi | P \equiv Q | u?(y)P', \\ \and \\\\ \and \\ \;\;\; u \in \meaningof{a}, \forall z.P'\{z/y\} \in \meaningof{E\{z/b\}}\}, \and \\ \meaningof{a!E} = \{ P \in \pi | P \equiv Q | x!\langle P' \rangle, x \in \meaningof{a} P' \in \meaningof{E}\} }
\end{mathpar}

\begin{mathpar}
 \inferrule* [lab=nominal] {} {\meaningof{\quotep{E}} = \{ \quotep{P} \in \quotep{\pi} | P \in \meaningof{E} \}, \and \meaningof{\quotep{P}} = \{ \quotep{Q} \in \quotep{\pi} | P \equiv Q \} \and \\ \meaningof{@\quotep{E}} = \{ P \in \pi | P \equiv @x, x \in \meaningof{E} \}}
\end{mathpar}

\begin{eqnarray*}
  \\
  \meaningof{-} : TS \to ST
\end{eqnarray*}

\begin{eqnarray*}
  \\
  L : TS \to ST
\end{eqnarray*}

\begin{eqnarray*}
  \\
  P \models E \iff P \in \meaningof{E}
\end{eqnarray*}

\begin{eqnarray*}
  P \approx_{L} Q \iff \forall E \in L. P \models E \iff Q \models E
\end{eqnarray*}

\begin{eqnarray*}
  P \approx_{K} Q
\end{eqnarray*}

\begin{eqnarray*}
  P \approx Q
\end{eqnarray*}

$\approx_{K} = \approx = \approx_{L}$

\subsubsection{Contextual duality}

Note that contexts extend the quotation operation to a family of
operations from processes to names. Given a context, $M$, we can
define a \emph{nominal context}, $\quotep{M}$ by $\quotep{M}[P] :=
\quotep{M[P]}$. To foreshadow what is to come we observe that these
operations enjoy a duality with processes very much like the duality
between vectors and maps from vectors to scalars.

Further, because the calculus is essentially higher-order, we have a
correspondence between contexts and processes. More specifically,
given a name $x$ and a context $M$ we can construct $M^{*}_{x}$ such
that 

\begin{mathpar}
  M^{*}_{x} | \lift{x}{P} \red M[P]
\end{mathpar}

namely,

\begin{mathpar}
  M^{*}_{x} := x?(u).M[\dropn{u}]
\end{mathpar}

The dependence of $M^{*}_{x}$ on a name makes it an abstraction, 

\begin{mathpar}
  M^{*} := (x)x?(u).M[\dropn{u}]
\end{mathpar}

\subsection{Additional notation}

It will sometimes be convenient to denote the process a name
quotes. We already have the notation $x = \quotep{P}$, but it will be
convenient to introduce an alternate notation, $\procn{x}$, when we
want to emphasize the connection to the use of the name. Note that, by
virtue of name equivalence, $\quotep{\procn{x}} \nameeq x$; so, the
notation is consistent with previous definitions.

Further, because names have structure it is possible to effect
substitutions on the basis of that structure. This means we need to
upgrade our notation for substitutions, which we accomplish by
adapting comprehension notation. Thus,

\begin{mathpar}
  P\{ y / x : x \in S \}
\end{mathpar}

is interpreted to mean the process derived from P by replacing (in a
capture-avoiding manner) each occurrence of $x$ in $S$ by $y$. For example,

\begin{mathpar}
  P\{ \quotep{\procn{x}|\procn{x}} / x : x \in \freenames{P} \}
\end{mathpar}

will replace each (occurrence) of a free name $x$ in $P$ by
$\quotep{\procn{x}|\procn{x}}$.

Also, we will avail ourselves of the notation $x^{L}$ and $x^{R}$ to
denote injections of a name into disjoint copies of the name
space. There are numerous ways to accomplish this. One example can be
found in \cite{MeredithR05}. This notation overloads to vectors of
names: $\vec{x}^{\pi} := (x_{i}^{\pi} \; : \; 0 \leq i < |\vec{x}| )$ where $\pi \in \{L,R\}$.

We also use $P^{\Box} := P|\Box$.

In \cite{MeredithR05} an interpretation of the new operator is
given. It turns out that there are several possible interpretations
all enjoying the requisite algebraic properties of the operator (see
\cite{milner91polyadicpi}). We will therefore make liberal use of
$(\nu\; \vec{x})P$.

% subsection the_syntax_and_semantics_of_the_notation_system (end)   

\input{qm2pi.qmops} 

\input{qm2pi.sterngerlach} 

\input{qm2pi.metric} 

% section concurrent_process_calculi (end)

%\input{qm2pi.proofsketch}

% section proof sketch (end)

%\input{qm2pi.slviaknots} 

% section spatial logic via knots (end)

\input{qm2pi.conclusion}

% section conclusion (end)

%\input{qm2pi.dtcodes} 

% section wiring algorithm (end)

\input{qm2pi.ack} 

% section acknowledgments (end)

\newpage


\bibliographystyle{plain}   
\bibliography{../../biblios/main.bib}

\input{qm2pi.rhodetails}

\end{document}

 

% section acknowledgments (end)

\newpage


\bibliographystyle{plain}   
\bibliography{../../biblios/main.bib}

\documentclass[12pt]{llncs}
%\documentclass{jktr}

\usepackage[pdftex]{hyperref}                   
\usepackage {listings}
\usepackage {mathpartir}
\usepackage{bcprules}
%\usepackage{listings}
                       
\usepackage{graphicx} 
%\usepackage[margins=2.5cm,nohead,nofoot]{geometry}
%\usepackage{geometry}
\usepackage{amsfonts}
\usepackage{amstext}
\usepackage{latexsym}
\usepackage{amssymb}
\usepackage{color}


%\include{myPreamble}
\include{qm2pi.local} 

%\ifpdf
%\usepackage[pdftex]{graphicx}
%\else
%\usepackage{graphicx}
%\fi

 % \ifpdf
%  \usepackage{pdfsync}
%  \if


%\title{Brief Article}
%\author{David F. Snyder}
%\author{L.G. Meredith}

%\address{Dept. of Math., Texas State University--San Marcos, San Marcos, TX 78666}
       
\pagestyle{empty}


\begin{document}

\lstset{language=[Objective]Caml,frame=shadowbox}

\input{qm2pi.front}

% section front matter (end)

\input{qm2pi.intro} 
 
% section introduction (end)

% \input{qm2pi.knotations} 

% section notation (end)

\input{qm2pi.process.calculi} 

% section concurrent_process_calculi_and_spatial_logics_ (end)
    
%\input{qm2pi.knots2pi} 

%\input{qm2pi.trefoil} 

%\input{qm2pi.mainthm} 

% subsection basic_interpretation (end)

%\input{qm2pi.rho.presentation} 
\subsection{The syntax and semantics of the notation system}\label{sub:the_syntax_and_semantics_of_the_notation_system} % (fold)

We now summarize a technical presentation of the calculus that
embodies our theory of dynamics. The typical presentation of such a
calculus follows the style of giving generators and relations on
them. The grammar, below, describing term constructors, freely
generates the set of processes, $\Proc$. This set is then quotiented
by a relation known as structural congruence and it is over this set
that the notion of dynamics is expressed. This presentation is
essentially that of \cite{MeredithR05} with the addition of
polyadicity and summation. For readability we have relegated some of
the technical subtleties to an appendix.

\subsubsection{Process grammar}\label{subsub:process_grammar}

\begin{mathpar}
  \inferrule* [lab=synchronization] {} {{M} \bc \pzero \;|\; x?F \;|\; x!C }
  \and
  \inferrule* [lab=abstraction] {} {{F} \bc (x)P}
  \and
  \inferrule* [lab=concretion] {} {{C} \bc \langle Q \rangle}
  \and
  \inferrule* [lab=process] {} {{P,Q} \bc M \;| \;P|Q \;|\; @{x}}
  \and
  \inferrule* [lab=name] {} {{x} \bc \quotep{P}}
\end{mathpar} 

Note that $\vec{x}$ (resp. $\vec{P}$) denotes a vector of names
(resp. processes) of length $|\vec{x}|$ (resp. $|\vec{P}|$). We adopt
the following useful abbreviations.

\begin{mathpar}
   x?(\vec{y}).P := x.(\vec{y})P \and  x\clift{\vec{P}} := x.\clift{\vec{P}}
   \and x!(y) := \lift{x}{\dropn{y}}
   \and \Pi_{i=0}^{n-1}P_i := P_0 | \ldots | P_{n-1}
\end{mathpar}

\subsubsection{Structural congruence}

\paragraph{Free and bound names and alpha-equivalence.} At the
core of structural equivalence is alpha-equivalence which identifies
process that are the same up to a change of variable. Formally, we
recognize the distinction between free and bound names. The free names
of a process, $\freenames{P}$, may be calculated recursively as
follows:

\begin{mathpar}
\freenames{\pzero} := \emptyset
  \and \\
  \freenames{x?(y).P} := \{ x \} \cup (\freenames{P} \setminus \{ y \})
  \and 
  \freenames{x!\langle P \rangle} := \{ x \} \cup \{ P \} 
  \and \\
  \freenames{P|Q} := \freenames{P} \cup \freenames{Q}
  \and \\
  \freenames{@{x}} := \{ x \}
\end{mathpar}

$\pi$
$\quotep{\pi}$

$\freenames{-} : \pi \to \mathcal{P}(\quotep{\pi})$

\begin{eqnarray*}
  \freenames{\pzero} & := & \emptyset \\
  \freenames{x?(y).P} & := & \{ x \} \cup (\freenames{P} \setminus \{ y \}) \\
  \freenames{x!\langle P \rangle} & := & \{ x \} \cup \{ P \} \\
  \freenames{P|Q} & := & \freenames{P} \cup \freenames{Q} \\
  \freenames{\dropn{x}} & := & \{ x \}
\end{eqnarray*}

The bound names of a process, $\boundnames{P}$, are those names occurring in $P$
that are not free. For example, in $x?(y).0$, the name $x$ is free, while $y$ is bound.

\begin{mathpar}
  \inferrule* [lab=monoidal-laws] {} { P|Q \equiv Q|P \and P|0 \equiv P \and P|(Q|R) \equiv (P|Q)|R }
\end{mathpar}

\begin{mathpar}
  \inferrule* [lab=alpha-equivalence] {} { (x)P \equiv (y)P\{y/x\} \and y \not\in \freenames{P} }
\end{mathpar}

\begin{definition}
Then two processes, $P,Q$, are alpha-equivalent if $P = Q\{\vec{y}/\vec{x}\}$ for
some $\vec{x} \in \boundnames{Q},\vec{y} \in \boundnames{P}$, where $Q\{\vec{y}/\vec{x}\}$
denotes the capture-avoiding substitution of $\vec{y}$ for $\vec{x}$ in $Q$.
\end{definition}

\begin{definition}
  The {\em structural congruence} \cite{SangiorgiWalker} , $\equiv$,
  between processes is the least congruence containing
  alpha-equivalence, satisfying the abelian monoid laws
  (associativity, commutativity and $\pzero$ as identity) for parallel
  composition $|$ and for summation $+$.
\end{definition}

\subsection{Name equivalence}

We take name equivalence, written $\nameeq$, to be the smallest
equivalence relation generated by the following rules.

\begin{mathpar}
\inferrule*[lab=Quote-drop]
{ }
{ \quotep{@{x}} \nameeq x }

\inferrule*[lab=Struct-equiv]
{ P \scong Q }
{ \quotep{P} \nameeq \quotep{Q} }
\end{mathpar}

The astute reader will have noticed that the mutual recursion of names
and processes imposes a mutual recursion on alpha-equivalence and
structural equivalence via name-equivalence. Fortunately, all of this
works out pleasantly and we may calculate in the natural way, free of
concern. The reader interested in the details is referred to the
appendix \ref{appendix:rho_details}.

\subsection{Substitution}

We use $\Proc$ for the set of processes, $\QProc$ for the set of
names, and $\id{\{}\vec{y} / \vec{x} \id{\}}$ to denote partial maps,
$s : \QProc \rightarrow \QProc$. A map, $s$ lifts, uniquely, to a map
on process terms, $\widehat{s} : \Proc \rightarrow \Proc$ by the
following equations.

\begin{mathpar}
  (0) \psubstp{Q}{P} := 0 \\
  (R \juxtap S) \psubstp{Q}{P}
  :=    
  (R)\psubstp{Q}{P} \juxtap (S) \psubstp{Q}{P} \\
  (x?(y).R) \psubstp{Q}{P}    
  :=    
  (x)\substp{Q}{P} (z)\concat( (R \psubstn{z}{y}) \psubstp{Q}{P} ) \\
  (\lift{x}{R}) \psubstp{Q}{P}  
  :=
  \lift{(x)\substp{Q}{P}}{ R \psubstp{Q}{P} } \\
%   (\dropn{x})  \psubstp{Q}{P}       
%   := 
%   \left\{ 
%     \begin{array}{ccc} 
%       \dropn{\quotep{Q}} & & x \nameeq \quotep{P} \\
%       \dropn{x} & & otherwise \\
%     \end{array}
%   \right. 
  (\dropn{x})  \psubstp{Q}{P}       
  := 
  \left\{ 
    \begin{array}{ccc} 
      Q & & x \nameeq \quotep{P} \\
      \dropn{x} & & otherwise \\
    \end{array}
  \right.
\end{mathpar}
 

where

\begin{eqnarray}
  (x)\id{\{} \lpquote Q \rpquote / \lpquote P \rpquote \id{\}}            = 
  \left\{ 
    \begin{array}{ccc}
      \lpquote Q \rpquote & & x \nameeq \lpquote P \rpquote \\
      x & & otherwise \\
    \end{array}
  \right. \nonumber
\end{eqnarray}

and $z$ is chosen distinct from $\quotep{P}$, $\quotep{Q}$, the free
names in $Q$, and all the names in $R$. Our $\alpha$-equivalence will
be built in the standard way from this substitution.

\begin{remark}\label{rem:no_self_referential_names}
  One consequence of these definitions is that $\forall P. \quotep{P}
  \not\in \freenames{P}$.
\end{remark}

\subsection{ Dynamic quote: an example }

Anticipating something of what's to come, consider applying the
substitution, $\widehat{\id{\{}u / z \id{\}}}$, to the following pair
of processes, $\lift{w}{y!(z)}$ and $w[ \lpquote y!(z) \rpquote ]$.

\begin{eqnarray}
	\lift{w}{y!(z)}\widehat{\id{\{}u / z \id{\}}}
		& = &
		\lift{w}{y!(u)} \nonumber\\
	w[ \lpquote y!(z) \rpquote ] \widehat{ \id{\{}u / z \id{\}} }
		& = &
		w[ \lpquote y!(z) \rpquote ] \nonumber
\end{eqnarray}

Because the body of the process between quotes is impervious to
substitution, we get radically different answers. In fact, by
examining the first process in an input context,
e.g. $x?(z).\lift{w}{y!(z)}$, we see that the process under the lift
operator may be shaped by prefixed inputs binding a name inside it. In
this sense, the lift operator will be seen as a way to dynamically
construct processes before reifying them as names.

Finally equipped with these standard features we can present the
dynamics of the calculus.

\subsubsection{Operational semantics} 

Finally, we introduce the computational dynamics. What marks these
algebras as distinct from other more traditionally studied algebraic
structures, e.g. vector spaces or polynomial rings, is the manner in
which dynamics is captured. In traditional structures, dynamics is typically
expressed through morphisms between such structures, as in linear maps
between vector spaces or morphisms between rings. In algebras
associated with the semantics of computation, the dynamics is
expressed as part of the algebraic structure itself, through a
reduction reduction relation typically denoted by $\red$. Below, we
give a recursive presentation of this relation for the calculus used
in the encoding.

$\red \subseteq \pi \times \pi$
$\red : \pi \to \mathcal{P}(\pi)$

\begin{mathpar}
  \inferrule* [lab=Comm] { \textsf{match}( x_{src}, x_{trgt} ) } { x_{trgt}?(y)P \; | \; x_{src}!\langle {Q} \rangle \red P\{\quotep{Q}/y}\} }
  \and \\
  \inferrule* [lab=Par] {{P} \red {P}'} {{{P} | {Q}} \red {{P}' | {Q}}}
  \and
  \inferrule* [lab=Equiv]{{{P} \scong {P}'} \andalso {{P}' \red {Q}'} \andalso {{Q}' \scong {Q}}}{{P} \red {Q}}
\end{mathpar}

\begin{eqnarray*}
  match_{\equiv} (\quotep{P},\quotep{Q}) & := & P \equiv Q \\
  match_{\dagger}(\quotep{P},\quotep{Q}) & := & \forall R. P|Q \red^{*} R => R \red^{*} 0 \\
  match_{K}(\quotep{P},\quotep{Q}) & := & K \mbox{ for some context } K
\end{eqnarray*}

$u?(x)P | u!\langle Q \rangle \red P\{\quotep{Q}/x\}$

%We write $\wred$ for $\red^*$, and $P\red$ if $\exists Q $ such that $ P \red Q$.
We write $P\red$ if $\exists Q $ such that $ P \red Q$ and $P\not\red$, otherwise.

\section{Replication}

As mentioned before, it is known that replication (and hence
recursion) can be implemented in a higher-order process algebra
\cite{SangiorgiWalker}. As our first example of calculation with the
machinery thus far presented we give the construction explicitly in
the {\rhoc}.

\begin{eqnarray}
	D_{x} & := & \prefix{x}{y}{(\binpar{\outputp{x}{y}}{@{y}})} \nonumber\\
	\bangp_{x}{P} & := & \binpar{{x}!\langle{\binpar{D_{x}}{P}}\rangle}{D_{x}} \nonumber
\end{eqnarray}

\begin{eqnarray}
	\bangp_{x}{P} & & \nonumber\\
	=
	& {x}!\langle{(\prefix{x}{y}{(\outputp{x}{y} | @{y})) | P}}\rangle 
	      | \prefix{x}{y}{(\outputp{x}{y} | @{y})} & \nonumber\\
	\red
	& (\outputp{x}{y} | @{y})\substn{\quotep{(\prefix{x}{y}{(@{y} | \outputp{x}{y})) | P}}}{y} & \nonumber\\
	=
	& \outputp{x}{\quotep{(\prefix{x}{y}{(\outputp{x}{y} | @{y})) | P}}}
	  | {(\prefix{x}{y}{(\outputp{x}{y} | @{y})) | P}} & \nonumber\\
	\red
	& \ldots & \nonumber\\
	\red^*
	& P | P | \ldots & \nonumber
\end{eqnarray}

Of course, this encoding, as an implementation, runs away, unfolding
$\bangp{P}$ eagerly. A lazier and more implementable replication
operator, restricted to input-guarded processes, may be obtained as follows.

\begin{eqnarray}
\bangp{\prefix{u}{v}{P}} 
	:= 
	\binpar{\lift{x}{\prefix{u}{v}{(\binpar{D(x)}{P})}}}{D(x)} \nonumber
\end{eqnarray}

\begin{remark}
  Note that the lazier definition still does not deal with summation
  or mixed summation (i.e. sums over input and output). The reader is
  invited to construct definitions of replication that deal with these
  features. 

  Further, the definitions are parameterized in a name, $x$. Can you,
  gentle reader, make a definition that eliminates this parameter and
  guarantees no accidental interaction between the replication
  machinery and the process being replicated -- i.e. no accidental
  sharing of names used by the process to get its work done and the
  name(s) used by the replication to effect copying. This latter
  revision of the definition of replication is crucial to obtaining
  the expected identity $!!P \sim !P$.
\end{remark}

\begin{remark}\label{rem:paradoxical_combinator}
  The reader familiar with the lambda calculus will have noticed the
  similarity between $D$ and the paradoxical combinator.

  [Ed. note: the existence of this seems to suggest we have to be more
  restrictive on the set of processes and names we admit if we are to
  support no-cloning.]
\end{remark}

\subsubsection{Bisimulation}

The computational dynamics gives rise to another kind of equivalence,
the equivalence of computational behavior. As previously mentioned
this is typically captured \emph{via} some form of bisimulation.

% The notion we use in this paper is weak barbed bisimulation
% \cite{milner91polyadicpi}.

The notion we use in this paper is derived from weak barbed
bisimulation \cite{milner91polyadicpi}. 

\begin{definition}
An \emph{observation relation}, $\downarrow_{\mathcal N}$, over a set
of names, $\mathcal N$, is the smallest relation satisfying the rules
below.

\infrule[Out-barb]{y \in {\mathcal N}, \; x \nameeq y}
		  {\outputp{x}{v} \downarrow_{\mathcal N} x}
\infrule[Par-barb]{\mbox{$P\downarrow_{\mathcal N} x$ or $Q\downarrow_{\mathcal N} x$}}
		  {\binpar{P}{Q} \downarrow_{\mathcal N} x}

We write $P \Downarrow_{\mathcal N} x$ if there is $Q$ such that 
$P \wred Q$ and $Q \downarrow_{\mathcal N} x$.
\end{definition}

\begin{definition}
%\label{def.bbisim}
An  ${\mathcal N}$-\emph{barbed bisimulation} over a set of names, ${\mathcal N}$, is a symmetric binary relation 
${\mathcal S}_{\mathcal N}$ between agents such that $P\rel{S}_{\mathcal N}Q$ implies:
\begin{enumerate}
\item If $P \red P'$ then $Q \wred Q'$ and $P'\rel{S}_{\mathcal N} Q'$.
\item If $P\downarrow_{\mathcal N} x$, then $Q\Downarrow_{\mathcal N} x$.
\end{enumerate}
$P$ is ${\mathcal N}$-barbed bisimilar to $Q$, written
$P \wbbisim_{\mathcal N} Q$, if $P \rel{S}_{\mathcal N} Q$ for some ${\mathcal N}$-barbed bisimulation ${\mathcal S}_{\mathcal N}$.
\end{definition}

$\mathcal{R} \subseteq \pi \times \pi$

$P \mathcal{R} Q => \forall P'. P \red P' \Rightarrow \exists Q'. Q \red Q', P' \mathcal{R} Q'$

$P \vdash x \Rightarrow Q \vdash x$

\begin{mathpar}
  \inferrule*[lab=Out-barb]{x \nameeq y}{{y}!\langle{Q}\rangle \vdash x}
  \and
  \inferrule*[lab=Par-barb]{\mbox{$P\vdash x$ or $Q\vdash x$}}{\binpar{P}{Q} \vdash x}
\end{mathpar}

\subsubsection{Contexts}

One of the principle advantages of computational calculi like the
$\pi$-calculus is a well-defined notion of context,
contextual-equivalence and a correlation between
contextual-equivalence and notions of bisimulation. The notion of
context allows the decomposition of a process into (sub-)process and
its syntactic environment, its context. Thus, a context may be
thought of as a process with a ``hole'' (written $\Box$) in it. The
application of a context $M$ to a process $P$, written $M[P]$, is
tantamount to filling the hole in $M$ with $P$. In this paper we do
not need the full weight of this theory, but do make use of the notion
of context in the proof the main theorem. 

\begin{mathpar}
  \inferrule* [lab=summation] {} {{M_{M},M_{N}} \bc \Box \;|\; x.M_{A} \;|\; M_{M}+M_{N}}
  \and
  \inferrule* [lab=agent] {} {{M_{A}} \bc (\vec{x})M_{P} \;| \; \clift{P_0,\ldots,M_{P},\ldots,P_N}}
  \and \\
  \inferrule* [lab=process] {} {{M_{P}} \bc M_{N} \;| \;P|M_{P} }
\end{mathpar} 

\begin{mathpar}
  \inferrule* [lab=sychronization] {} {M_{N} \bc \Box \;|\; x?M_{F} \;|\; x!M_{C}}
  \and
  \inferrule* [lab=abstraction] {} {{M_{F}} \bc (x)M_{P} }
  \and
  \inferrule* [lab=concretion] {} {{M_{C}} \bc \langle M_{P} \rangle }
  \and \\
  \inferrule* [lab=process] {} {{M_{P}} \bc M_{N} \;| \;P|M_{P} }
\end{mathpar}

\begin{definition}[contextual application] Given a context $M$, and
  process $P$, we define the \emph{contextual application}, $M[P] :=
  M\{P/\Box\}$. That is, the contextual application of M to P is the
  substitution of $P$ for $\Box$ in $M$.
\end{definition}

$\meaningof{-} : L \to \mathcal{P}(\pi)$

\begin{mathpar}
  \inferrule* [lab=collection] {} {\meaningof{true} = \pi, \and \meaningof{~E} = \pi \setminus \meaningof{E}, \and \meaningof{E_{1} \& E_{2}} = \meaningof{E_{1}} \cap \meaningof{E_{2}}}
\end{mathpar}

\begin{mathpar}
  \inferrule* [lab=structure] {} {\meaningof{0} = \{ P \in \pi | P \equiv 0 \}, \and \\ \meaningof{E_1 | E_2} = \{ P \in \pi | P \equiv P_{1} | P_{2}, P_{1} \in \meaningof{E_{1}}, P_{2} \in \meaningof{E_2}\} }
\end{mathpar}

\begin{mathpar}
 \inferrule* [lab=behavior] {} {\meaningof{\langle a?b \rangle E} = \{ P \in \pi | P \equiv Q | u?(y)P', \\ \and \\\\ \and \\ \;\;\; u \in \meaningof{a}, \forall z.P'\{z/y\} \in \meaningof{E\{z/b\}}\}, \and \\ \meaningof{a!E} = \{ P \in \pi | P \equiv Q | x!\langle P' \rangle, x \in \meaningof{a} P' \in \meaningof{E}\} }
\end{mathpar}

\begin{mathpar}
 \inferrule* [lab=nominal] {} {\meaningof{\quotep{E}} = \{ \quotep{P} \in \quotep{\pi} | P \in \meaningof{E} \}, \and \meaningof{\quotep{P}} = \{ \quotep{Q} \in \quotep{\pi} | P \equiv Q \} \and \\ \meaningof{@\quotep{E}} = \{ P \in \pi | P \equiv @x, x \in \meaningof{E} \}}
\end{mathpar}

\begin{eqnarray*}
  \\
  \meaningof{-} : TS \to ST
\end{eqnarray*}

\begin{eqnarray*}
  \\
  L : TS \to ST
\end{eqnarray*}

\begin{eqnarray*}
  \\
  P \models E \iff P \in \meaningof{E}
\end{eqnarray*}

\begin{eqnarray*}
  P \approx_{L} Q \iff \forall E \in L. P \models E \iff Q \models E
\end{eqnarray*}

\begin{eqnarray*}
  P \approx_{K} Q
\end{eqnarray*}

\begin{eqnarray*}
  P \approx Q
\end{eqnarray*}

$\approx_{K} = \approx = \approx_{L}$

\subsubsection{Contextual duality}

Note that contexts extend the quotation operation to a family of
operations from processes to names. Given a context, $M$, we can
define a \emph{nominal context}, $\quotep{M}$ by $\quotep{M}[P] :=
\quotep{M[P]}$. To foreshadow what is to come we observe that these
operations enjoy a duality with processes very much like the duality
between vectors and maps from vectors to scalars.

Further, because the calculus is essentially higher-order, we have a
correspondence between contexts and processes. More specifically,
given a name $x$ and a context $M$ we can construct $M^{*}_{x}$ such
that 

\begin{mathpar}
  M^{*}_{x} | \lift{x}{P} \red M[P]
\end{mathpar}

namely,

\begin{mathpar}
  M^{*}_{x} := x?(u).M[\dropn{u}]
\end{mathpar}

The dependence of $M^{*}_{x}$ on a name makes it an abstraction, 

\begin{mathpar}
  M^{*} := (x)x?(u).M[\dropn{u}]
\end{mathpar}

\subsection{Additional notation}

It will sometimes be convenient to denote the process a name
quotes. We already have the notation $x = \quotep{P}$, but it will be
convenient to introduce an alternate notation, $\procn{x}$, when we
want to emphasize the connection to the use of the name. Note that, by
virtue of name equivalence, $\quotep{\procn{x}} \nameeq x$; so, the
notation is consistent with previous definitions.

Further, because names have structure it is possible to effect
substitutions on the basis of that structure. This means we need to
upgrade our notation for substitutions, which we accomplish by
adapting comprehension notation. Thus,

\begin{mathpar}
  P\{ y / x : x \in S \}
\end{mathpar}

is interpreted to mean the process derived from P by replacing (in a
capture-avoiding manner) each occurrence of $x$ in $S$ by $y$. For example,

\begin{mathpar}
  P\{ \quotep{\procn{x}|\procn{x}} / x : x \in \freenames{P} \}
\end{mathpar}

will replace each (occurrence) of a free name $x$ in $P$ by
$\quotep{\procn{x}|\procn{x}}$.

Also, we will avail ourselves of the notation $x^{L}$ and $x^{R}$ to
denote injections of a name into disjoint copies of the name
space. There are numerous ways to accomplish this. One example can be
found in \cite{MeredithR05}. This notation overloads to vectors of
names: $\vec{x}^{\pi} := (x_{i}^{\pi} \; : \; 0 \leq i < |\vec{x}| )$ where $\pi \in \{L,R\}$.

We also use $P^{\Box} := P|\Box$.

In \cite{MeredithR05} an interpretation of the new operator is
given. It turns out that there are several possible interpretations
all enjoying the requisite algebraic properties of the operator (see
\cite{milner91polyadicpi}). We will therefore make liberal use of
$(\nu\; \vec{x})P$.

% subsection the_syntax_and_semantics_of_the_notation_system (end)   

\input{qm2pi.qmops} 

\input{qm2pi.sterngerlach} 

\input{qm2pi.metric} 

% section concurrent_process_calculi (end)

%\input{qm2pi.proofsketch}

% section proof sketch (end)

%\input{qm2pi.slviaknots} 

% section spatial logic via knots (end)

\input{qm2pi.conclusion}

% section conclusion (end)

%\input{qm2pi.dtcodes} 

% section wiring algorithm (end)

\input{qm2pi.ack} 

% section acknowledgments (end)

\newpage


\bibliographystyle{plain}   
\bibliography{../../biblios/main.bib}

\input{qm2pi.rhodetails}

\end{document}



\end{document}

 

% section notation (end)

\input{qm2pi.process.calculi} 

% section concurrent_process_calculi_and_spatial_logics_ (end)
    
%\documentclass[12pt]{llncs}
%\documentclass{jktr}

\usepackage[pdftex]{hyperref}                   
\usepackage {listings}
\usepackage {mathpartir}
\usepackage{bcprules}
%\usepackage{listings}
                       
\usepackage{graphicx} 
%\usepackage[margins=2.5cm,nohead,nofoot]{geometry}
%\usepackage{geometry}
\usepackage{amsfonts}
\usepackage{amstext}
\usepackage{latexsym}
\usepackage{amssymb}
\usepackage{color}


%\include{myPreamble}
\documentclass[12pt]{llncs}
%\documentclass{jktr}

\usepackage[pdftex]{hyperref}                   
\usepackage {listings}
\usepackage {mathpartir}
\usepackage{bcprules}
%\usepackage{listings}
                       
\usepackage{graphicx} 
%\usepackage[margins=2.5cm,nohead,nofoot]{geometry}
%\usepackage{geometry}
\usepackage{amsfonts}
\usepackage{amstext}
\usepackage{latexsym}
\usepackage{amssymb}
\usepackage{color}


%\include{myPreamble}
\include{qm2pi.local} 

%\ifpdf
%\usepackage[pdftex]{graphicx}
%\else
%\usepackage{graphicx}
%\fi

 % \ifpdf
%  \usepackage{pdfsync}
%  \if


%\title{Brief Article}
%\author{David F. Snyder}
%\author{L.G. Meredith}

%\address{Dept. of Math., Texas State University--San Marcos, San Marcos, TX 78666}
       
\pagestyle{empty}


\begin{document}

\lstset{language=[Objective]Caml,frame=shadowbox}

\input{qm2pi.front}

% section front matter (end)

\input{qm2pi.intro} 
 
% section introduction (end)

% \input{qm2pi.knotations} 

% section notation (end)

\input{qm2pi.process.calculi} 

% section concurrent_process_calculi_and_spatial_logics_ (end)
    
%\input{qm2pi.knots2pi} 

%\input{qm2pi.trefoil} 

%\input{qm2pi.mainthm} 

% subsection basic_interpretation (end)

%\input{qm2pi.rho.presentation} 
\subsection{The syntax and semantics of the notation system}\label{sub:the_syntax_and_semantics_of_the_notation_system} % (fold)

We now summarize a technical presentation of the calculus that
embodies our theory of dynamics. The typical presentation of such a
calculus follows the style of giving generators and relations on
them. The grammar, below, describing term constructors, freely
generates the set of processes, $\Proc$. This set is then quotiented
by a relation known as structural congruence and it is over this set
that the notion of dynamics is expressed. This presentation is
essentially that of \cite{MeredithR05} with the addition of
polyadicity and summation. For readability we have relegated some of
the technical subtleties to an appendix.

\subsubsection{Process grammar}\label{subsub:process_grammar}

\begin{mathpar}
  \inferrule* [lab=synchronization] {} {{M} \bc \pzero \;|\; x?F \;|\; x!C }
  \and
  \inferrule* [lab=abstraction] {} {{F} \bc (x)P}
  \and
  \inferrule* [lab=concretion] {} {{C} \bc \langle Q \rangle}
  \and
  \inferrule* [lab=process] {} {{P,Q} \bc M \;| \;P|Q \;|\; @{x}}
  \and
  \inferrule* [lab=name] {} {{x} \bc \quotep{P}}
\end{mathpar} 

Note that $\vec{x}$ (resp. $\vec{P}$) denotes a vector of names
(resp. processes) of length $|\vec{x}|$ (resp. $|\vec{P}|$). We adopt
the following useful abbreviations.

\begin{mathpar}
   x?(\vec{y}).P := x.(\vec{y})P \and  x\clift{\vec{P}} := x.\clift{\vec{P}}
   \and x!(y) := \lift{x}{\dropn{y}}
   \and \Pi_{i=0}^{n-1}P_i := P_0 | \ldots | P_{n-1}
\end{mathpar}

\subsubsection{Structural congruence}

\paragraph{Free and bound names and alpha-equivalence.} At the
core of structural equivalence is alpha-equivalence which identifies
process that are the same up to a change of variable. Formally, we
recognize the distinction between free and bound names. The free names
of a process, $\freenames{P}$, may be calculated recursively as
follows:

\begin{mathpar}
\freenames{\pzero} := \emptyset
  \and \\
  \freenames{x?(y).P} := \{ x \} \cup (\freenames{P} \setminus \{ y \})
  \and 
  \freenames{x!\langle P \rangle} := \{ x \} \cup \{ P \} 
  \and \\
  \freenames{P|Q} := \freenames{P} \cup \freenames{Q}
  \and \\
  \freenames{@{x}} := \{ x \}
\end{mathpar}

$\pi$
$\quotep{\pi}$

$\freenames{-} : \pi \to \mathcal{P}(\quotep{\pi})$

\begin{eqnarray*}
  \freenames{\pzero} & := & \emptyset \\
  \freenames{x?(y).P} & := & \{ x \} \cup (\freenames{P} \setminus \{ y \}) \\
  \freenames{x!\langle P \rangle} & := & \{ x \} \cup \{ P \} \\
  \freenames{P|Q} & := & \freenames{P} \cup \freenames{Q} \\
  \freenames{\dropn{x}} & := & \{ x \}
\end{eqnarray*}

The bound names of a process, $\boundnames{P}$, are those names occurring in $P$
that are not free. For example, in $x?(y).0$, the name $x$ is free, while $y$ is bound.

\begin{mathpar}
  \inferrule* [lab=monoidal-laws] {} { P|Q \equiv Q|P \and P|0 \equiv P \and P|(Q|R) \equiv (P|Q)|R }
\end{mathpar}

\begin{mathpar}
  \inferrule* [lab=alpha-equivalence] {} { (x)P \equiv (y)P\{y/x\} \and y \not\in \freenames{P} }
\end{mathpar}

\begin{definition}
Then two processes, $P,Q$, are alpha-equivalent if $P = Q\{\vec{y}/\vec{x}\}$ for
some $\vec{x} \in \boundnames{Q},\vec{y} \in \boundnames{P}$, where $Q\{\vec{y}/\vec{x}\}$
denotes the capture-avoiding substitution of $\vec{y}$ for $\vec{x}$ in $Q$.
\end{definition}

\begin{definition}
  The {\em structural congruence} \cite{SangiorgiWalker} , $\equiv$,
  between processes is the least congruence containing
  alpha-equivalence, satisfying the abelian monoid laws
  (associativity, commutativity and $\pzero$ as identity) for parallel
  composition $|$ and for summation $+$.
\end{definition}

\subsection{Name equivalence}

We take name equivalence, written $\nameeq$, to be the smallest
equivalence relation generated by the following rules.

\begin{mathpar}
\inferrule*[lab=Quote-drop]
{ }
{ \quotep{@{x}} \nameeq x }

\inferrule*[lab=Struct-equiv]
{ P \scong Q }
{ \quotep{P} \nameeq \quotep{Q} }
\end{mathpar}

The astute reader will have noticed that the mutual recursion of names
and processes imposes a mutual recursion on alpha-equivalence and
structural equivalence via name-equivalence. Fortunately, all of this
works out pleasantly and we may calculate in the natural way, free of
concern. The reader interested in the details is referred to the
appendix \ref{appendix:rho_details}.

\subsection{Substitution}

We use $\Proc$ for the set of processes, $\QProc$ for the set of
names, and $\id{\{}\vec{y} / \vec{x} \id{\}}$ to denote partial maps,
$s : \QProc \rightarrow \QProc$. A map, $s$ lifts, uniquely, to a map
on process terms, $\widehat{s} : \Proc \rightarrow \Proc$ by the
following equations.

\begin{mathpar}
  (0) \psubstp{Q}{P} := 0 \\
  (R \juxtap S) \psubstp{Q}{P}
  :=    
  (R)\psubstp{Q}{P} \juxtap (S) \psubstp{Q}{P} \\
  (x?(y).R) \psubstp{Q}{P}    
  :=    
  (x)\substp{Q}{P} (z)\concat( (R \psubstn{z}{y}) \psubstp{Q}{P} ) \\
  (\lift{x}{R}) \psubstp{Q}{P}  
  :=
  \lift{(x)\substp{Q}{P}}{ R \psubstp{Q}{P} } \\
%   (\dropn{x})  \psubstp{Q}{P}       
%   := 
%   \left\{ 
%     \begin{array}{ccc} 
%       \dropn{\quotep{Q}} & & x \nameeq \quotep{P} \\
%       \dropn{x} & & otherwise \\
%     \end{array}
%   \right. 
  (\dropn{x})  \psubstp{Q}{P}       
  := 
  \left\{ 
    \begin{array}{ccc} 
      Q & & x \nameeq \quotep{P} \\
      \dropn{x} & & otherwise \\
    \end{array}
  \right.
\end{mathpar}
 

where

\begin{eqnarray}
  (x)\id{\{} \lpquote Q \rpquote / \lpquote P \rpquote \id{\}}            = 
  \left\{ 
    \begin{array}{ccc}
      \lpquote Q \rpquote & & x \nameeq \lpquote P \rpquote \\
      x & & otherwise \\
    \end{array}
  \right. \nonumber
\end{eqnarray}

and $z$ is chosen distinct from $\quotep{P}$, $\quotep{Q}$, the free
names in $Q$, and all the names in $R$. Our $\alpha$-equivalence will
be built in the standard way from this substitution.

\begin{remark}\label{rem:no_self_referential_names}
  One consequence of these definitions is that $\forall P. \quotep{P}
  \not\in \freenames{P}$.
\end{remark}

\subsection{ Dynamic quote: an example }

Anticipating something of what's to come, consider applying the
substitution, $\widehat{\id{\{}u / z \id{\}}}$, to the following pair
of processes, $\lift{w}{y!(z)}$ and $w[ \lpquote y!(z) \rpquote ]$.

\begin{eqnarray}
	\lift{w}{y!(z)}\widehat{\id{\{}u / z \id{\}}}
		& = &
		\lift{w}{y!(u)} \nonumber\\
	w[ \lpquote y!(z) \rpquote ] \widehat{ \id{\{}u / z \id{\}} }
		& = &
		w[ \lpquote y!(z) \rpquote ] \nonumber
\end{eqnarray}

Because the body of the process between quotes is impervious to
substitution, we get radically different answers. In fact, by
examining the first process in an input context,
e.g. $x?(z).\lift{w}{y!(z)}$, we see that the process under the lift
operator may be shaped by prefixed inputs binding a name inside it. In
this sense, the lift operator will be seen as a way to dynamically
construct processes before reifying them as names.

Finally equipped with these standard features we can present the
dynamics of the calculus.

\subsubsection{Operational semantics} 

Finally, we introduce the computational dynamics. What marks these
algebras as distinct from other more traditionally studied algebraic
structures, e.g. vector spaces or polynomial rings, is the manner in
which dynamics is captured. In traditional structures, dynamics is typically
expressed through morphisms between such structures, as in linear maps
between vector spaces or morphisms between rings. In algebras
associated with the semantics of computation, the dynamics is
expressed as part of the algebraic structure itself, through a
reduction reduction relation typically denoted by $\red$. Below, we
give a recursive presentation of this relation for the calculus used
in the encoding.

$\red \subseteq \pi \times \pi$
$\red : \pi \to \mathcal{P}(\pi)$

\begin{mathpar}
  \inferrule* [lab=Comm] { \textsf{match}( x_{src}, x_{trgt} ) } { x_{trgt}?(y)P \; | \; x_{src}!\langle {Q} \rangle \red P\{\quotep{Q}/y}\} }
  \and \\
  \inferrule* [lab=Par] {{P} \red {P}'} {{{P} | {Q}} \red {{P}' | {Q}}}
  \and
  \inferrule* [lab=Equiv]{{{P} \scong {P}'} \andalso {{P}' \red {Q}'} \andalso {{Q}' \scong {Q}}}{{P} \red {Q}}
\end{mathpar}

\begin{eqnarray*}
  match_{\equiv} (\quotep{P},\quotep{Q}) & := & P \equiv Q \\
  match_{\dagger}(\quotep{P},\quotep{Q}) & := & \forall R. P|Q \red^{*} R => R \red^{*} 0 \\
  match_{K}(\quotep{P},\quotep{Q}) & := & K \mbox{ for some context } K
\end{eqnarray*}

$u?(x)P | u!\langle Q \rangle \red P\{\quotep{Q}/x\}$

%We write $\wred$ for $\red^*$, and $P\red$ if $\exists Q $ such that $ P \red Q$.
We write $P\red$ if $\exists Q $ such that $ P \red Q$ and $P\not\red$, otherwise.

\section{Replication}

As mentioned before, it is known that replication (and hence
recursion) can be implemented in a higher-order process algebra
\cite{SangiorgiWalker}. As our first example of calculation with the
machinery thus far presented we give the construction explicitly in
the {\rhoc}.

\begin{eqnarray}
	D_{x} & := & \prefix{x}{y}{(\binpar{\outputp{x}{y}}{@{y}})} \nonumber\\
	\bangp_{x}{P} & := & \binpar{{x}!\langle{\binpar{D_{x}}{P}}\rangle}{D_{x}} \nonumber
\end{eqnarray}

\begin{eqnarray}
	\bangp_{x}{P} & & \nonumber\\
	=
	& {x}!\langle{(\prefix{x}{y}{(\outputp{x}{y} | @{y})) | P}}\rangle 
	      | \prefix{x}{y}{(\outputp{x}{y} | @{y})} & \nonumber\\
	\red
	& (\outputp{x}{y} | @{y})\substn{\quotep{(\prefix{x}{y}{(@{y} | \outputp{x}{y})) | P}}}{y} & \nonumber\\
	=
	& \outputp{x}{\quotep{(\prefix{x}{y}{(\outputp{x}{y} | @{y})) | P}}}
	  | {(\prefix{x}{y}{(\outputp{x}{y} | @{y})) | P}} & \nonumber\\
	\red
	& \ldots & \nonumber\\
	\red^*
	& P | P | \ldots & \nonumber
\end{eqnarray}

Of course, this encoding, as an implementation, runs away, unfolding
$\bangp{P}$ eagerly. A lazier and more implementable replication
operator, restricted to input-guarded processes, may be obtained as follows.

\begin{eqnarray}
\bangp{\prefix{u}{v}{P}} 
	:= 
	\binpar{\lift{x}{\prefix{u}{v}{(\binpar{D(x)}{P})}}}{D(x)} \nonumber
\end{eqnarray}

\begin{remark}
  Note that the lazier definition still does not deal with summation
  or mixed summation (i.e. sums over input and output). The reader is
  invited to construct definitions of replication that deal with these
  features. 

  Further, the definitions are parameterized in a name, $x$. Can you,
  gentle reader, make a definition that eliminates this parameter and
  guarantees no accidental interaction between the replication
  machinery and the process being replicated -- i.e. no accidental
  sharing of names used by the process to get its work done and the
  name(s) used by the replication to effect copying. This latter
  revision of the definition of replication is crucial to obtaining
  the expected identity $!!P \sim !P$.
\end{remark}

\begin{remark}\label{rem:paradoxical_combinator}
  The reader familiar with the lambda calculus will have noticed the
  similarity between $D$ and the paradoxical combinator.

  [Ed. note: the existence of this seems to suggest we have to be more
  restrictive on the set of processes and names we admit if we are to
  support no-cloning.]
\end{remark}

\subsubsection{Bisimulation}

The computational dynamics gives rise to another kind of equivalence,
the equivalence of computational behavior. As previously mentioned
this is typically captured \emph{via} some form of bisimulation.

% The notion we use in this paper is weak barbed bisimulation
% \cite{milner91polyadicpi}.

The notion we use in this paper is derived from weak barbed
bisimulation \cite{milner91polyadicpi}. 

\begin{definition}
An \emph{observation relation}, $\downarrow_{\mathcal N}$, over a set
of names, $\mathcal N$, is the smallest relation satisfying the rules
below.

\infrule[Out-barb]{y \in {\mathcal N}, \; x \nameeq y}
		  {\outputp{x}{v} \downarrow_{\mathcal N} x}
\infrule[Par-barb]{\mbox{$P\downarrow_{\mathcal N} x$ or $Q\downarrow_{\mathcal N} x$}}
		  {\binpar{P}{Q} \downarrow_{\mathcal N} x}

We write $P \Downarrow_{\mathcal N} x$ if there is $Q$ such that 
$P \wred Q$ and $Q \downarrow_{\mathcal N} x$.
\end{definition}

\begin{definition}
%\label{def.bbisim}
An  ${\mathcal N}$-\emph{barbed bisimulation} over a set of names, ${\mathcal N}$, is a symmetric binary relation 
${\mathcal S}_{\mathcal N}$ between agents such that $P\rel{S}_{\mathcal N}Q$ implies:
\begin{enumerate}
\item If $P \red P'$ then $Q \wred Q'$ and $P'\rel{S}_{\mathcal N} Q'$.
\item If $P\downarrow_{\mathcal N} x$, then $Q\Downarrow_{\mathcal N} x$.
\end{enumerate}
$P$ is ${\mathcal N}$-barbed bisimilar to $Q$, written
$P \wbbisim_{\mathcal N} Q$, if $P \rel{S}_{\mathcal N} Q$ for some ${\mathcal N}$-barbed bisimulation ${\mathcal S}_{\mathcal N}$.
\end{definition}

$\mathcal{R} \subseteq \pi \times \pi$

$P \mathcal{R} Q => \forall P'. P \red P' \Rightarrow \exists Q'. Q \red Q', P' \mathcal{R} Q'$

$P \vdash x \Rightarrow Q \vdash x$

\begin{mathpar}
  \inferrule*[lab=Out-barb]{x \nameeq y}{{y}!\langle{Q}\rangle \vdash x}
  \and
  \inferrule*[lab=Par-barb]{\mbox{$P\vdash x$ or $Q\vdash x$}}{\binpar{P}{Q} \vdash x}
\end{mathpar}

\subsubsection{Contexts}

One of the principle advantages of computational calculi like the
$\pi$-calculus is a well-defined notion of context,
contextual-equivalence and a correlation between
contextual-equivalence and notions of bisimulation. The notion of
context allows the decomposition of a process into (sub-)process and
its syntactic environment, its context. Thus, a context may be
thought of as a process with a ``hole'' (written $\Box$) in it. The
application of a context $M$ to a process $P$, written $M[P]$, is
tantamount to filling the hole in $M$ with $P$. In this paper we do
not need the full weight of this theory, but do make use of the notion
of context in the proof the main theorem. 

\begin{mathpar}
  \inferrule* [lab=summation] {} {{M_{M},M_{N}} \bc \Box \;|\; x.M_{A} \;|\; M_{M}+M_{N}}
  \and
  \inferrule* [lab=agent] {} {{M_{A}} \bc (\vec{x})M_{P} \;| \; \clift{P_0,\ldots,M_{P},\ldots,P_N}}
  \and \\
  \inferrule* [lab=process] {} {{M_{P}} \bc M_{N} \;| \;P|M_{P} }
\end{mathpar} 

\begin{mathpar}
  \inferrule* [lab=sychronization] {} {M_{N} \bc \Box \;|\; x?M_{F} \;|\; x!M_{C}}
  \and
  \inferrule* [lab=abstraction] {} {{M_{F}} \bc (x)M_{P} }
  \and
  \inferrule* [lab=concretion] {} {{M_{C}} \bc \langle M_{P} \rangle }
  \and \\
  \inferrule* [lab=process] {} {{M_{P}} \bc M_{N} \;| \;P|M_{P} }
\end{mathpar}

\begin{definition}[contextual application] Given a context $M$, and
  process $P$, we define the \emph{contextual application}, $M[P] :=
  M\{P/\Box\}$. That is, the contextual application of M to P is the
  substitution of $P$ for $\Box$ in $M$.
\end{definition}

$\meaningof{-} : L \to \mathcal{P}(\pi)$

\begin{mathpar}
  \inferrule* [lab=collection] {} {\meaningof{true} = \pi, \and \meaningof{~E} = \pi \setminus \meaningof{E}, \and \meaningof{E_{1} \& E_{2}} = \meaningof{E_{1}} \cap \meaningof{E_{2}}}
\end{mathpar}

\begin{mathpar}
  \inferrule* [lab=structure] {} {\meaningof{0} = \{ P \in \pi | P \equiv 0 \}, \and \\ \meaningof{E_1 | E_2} = \{ P \in \pi | P \equiv P_{1} | P_{2}, P_{1} \in \meaningof{E_{1}}, P_{2} \in \meaningof{E_2}\} }
\end{mathpar}

\begin{mathpar}
 \inferrule* [lab=behavior] {} {\meaningof{\langle a?b \rangle E} = \{ P \in \pi | P \equiv Q | u?(y)P', \\ \and \\\\ \and \\ \;\;\; u \in \meaningof{a}, \forall z.P'\{z/y\} \in \meaningof{E\{z/b\}}\}, \and \\ \meaningof{a!E} = \{ P \in \pi | P \equiv Q | x!\langle P' \rangle, x \in \meaningof{a} P' \in \meaningof{E}\} }
\end{mathpar}

\begin{mathpar}
 \inferrule* [lab=nominal] {} {\meaningof{\quotep{E}} = \{ \quotep{P} \in \quotep{\pi} | P \in \meaningof{E} \}, \and \meaningof{\quotep{P}} = \{ \quotep{Q} \in \quotep{\pi} | P \equiv Q \} \and \\ \meaningof{@\quotep{E}} = \{ P \in \pi | P \equiv @x, x \in \meaningof{E} \}}
\end{mathpar}

\begin{eqnarray*}
  \\
  \meaningof{-} : TS \to ST
\end{eqnarray*}

\begin{eqnarray*}
  \\
  L : TS \to ST
\end{eqnarray*}

\begin{eqnarray*}
  \\
  P \models E \iff P \in \meaningof{E}
\end{eqnarray*}

\begin{eqnarray*}
  P \approx_{L} Q \iff \forall E \in L. P \models E \iff Q \models E
\end{eqnarray*}

\begin{eqnarray*}
  P \approx_{K} Q
\end{eqnarray*}

\begin{eqnarray*}
  P \approx Q
\end{eqnarray*}

$\approx_{K} = \approx = \approx_{L}$

\subsubsection{Contextual duality}

Note that contexts extend the quotation operation to a family of
operations from processes to names. Given a context, $M$, we can
define a \emph{nominal context}, $\quotep{M}$ by $\quotep{M}[P] :=
\quotep{M[P]}$. To foreshadow what is to come we observe that these
operations enjoy a duality with processes very much like the duality
between vectors and maps from vectors to scalars.

Further, because the calculus is essentially higher-order, we have a
correspondence between contexts and processes. More specifically,
given a name $x$ and a context $M$ we can construct $M^{*}_{x}$ such
that 

\begin{mathpar}
  M^{*}_{x} | \lift{x}{P} \red M[P]
\end{mathpar}

namely,

\begin{mathpar}
  M^{*}_{x} := x?(u).M[\dropn{u}]
\end{mathpar}

The dependence of $M^{*}_{x}$ on a name makes it an abstraction, 

\begin{mathpar}
  M^{*} := (x)x?(u).M[\dropn{u}]
\end{mathpar}

\subsection{Additional notation}

It will sometimes be convenient to denote the process a name
quotes. We already have the notation $x = \quotep{P}$, but it will be
convenient to introduce an alternate notation, $\procn{x}$, when we
want to emphasize the connection to the use of the name. Note that, by
virtue of name equivalence, $\quotep{\procn{x}} \nameeq x$; so, the
notation is consistent with previous definitions.

Further, because names have structure it is possible to effect
substitutions on the basis of that structure. This means we need to
upgrade our notation for substitutions, which we accomplish by
adapting comprehension notation. Thus,

\begin{mathpar}
  P\{ y / x : x \in S \}
\end{mathpar}

is interpreted to mean the process derived from P by replacing (in a
capture-avoiding manner) each occurrence of $x$ in $S$ by $y$. For example,

\begin{mathpar}
  P\{ \quotep{\procn{x}|\procn{x}} / x : x \in \freenames{P} \}
\end{mathpar}

will replace each (occurrence) of a free name $x$ in $P$ by
$\quotep{\procn{x}|\procn{x}}$.

Also, we will avail ourselves of the notation $x^{L}$ and $x^{R}$ to
denote injections of a name into disjoint copies of the name
space. There are numerous ways to accomplish this. One example can be
found in \cite{MeredithR05}. This notation overloads to vectors of
names: $\vec{x}^{\pi} := (x_{i}^{\pi} \; : \; 0 \leq i < |\vec{x}| )$ where $\pi \in \{L,R\}$.

We also use $P^{\Box} := P|\Box$.

In \cite{MeredithR05} an interpretation of the new operator is
given. It turns out that there are several possible interpretations
all enjoying the requisite algebraic properties of the operator (see
\cite{milner91polyadicpi}). We will therefore make liberal use of
$(\nu\; \vec{x})P$.

% subsection the_syntax_and_semantics_of_the_notation_system (end)   

\input{qm2pi.qmops} 

\input{qm2pi.sterngerlach} 

\input{qm2pi.metric} 

% section concurrent_process_calculi (end)

%\input{qm2pi.proofsketch}

% section proof sketch (end)

%\input{qm2pi.slviaknots} 

% section spatial logic via knots (end)

\input{qm2pi.conclusion}

% section conclusion (end)

%\input{qm2pi.dtcodes} 

% section wiring algorithm (end)

\input{qm2pi.ack} 

% section acknowledgments (end)

\newpage


\bibliographystyle{plain}   
\bibliography{../../biblios/main.bib}

\input{qm2pi.rhodetails}

\end{document}

 

%\ifpdf
%\usepackage[pdftex]{graphicx}
%\else
%\usepackage{graphicx}
%\fi

 % \ifpdf
%  \usepackage{pdfsync}
%  \if


%\title{Brief Article}
%\author{David F. Snyder}
%\author{L.G. Meredith}

%\address{Dept. of Math., Texas State University--San Marcos, San Marcos, TX 78666}
       
\pagestyle{empty}


\begin{document}

\lstset{language=[Objective]Caml,frame=shadowbox}

\documentclass[12pt]{llncs}
%\documentclass{jktr}

\usepackage[pdftex]{hyperref}                   
\usepackage {listings}
\usepackage {mathpartir}
\usepackage{bcprules}
%\usepackage{listings}
                       
\usepackage{graphicx} 
%\usepackage[margins=2.5cm,nohead,nofoot]{geometry}
%\usepackage{geometry}
\usepackage{amsfonts}
\usepackage{amstext}
\usepackage{latexsym}
\usepackage{amssymb}
\usepackage{color}


%\include{myPreamble}
\include{qm2pi.local} 

%\ifpdf
%\usepackage[pdftex]{graphicx}
%\else
%\usepackage{graphicx}
%\fi

 % \ifpdf
%  \usepackage{pdfsync}
%  \if


%\title{Brief Article}
%\author{David F. Snyder}
%\author{L.G. Meredith}

%\address{Dept. of Math., Texas State University--San Marcos, San Marcos, TX 78666}
       
\pagestyle{empty}


\begin{document}

\lstset{language=[Objective]Caml,frame=shadowbox}

\input{qm2pi.front}

% section front matter (end)

\input{qm2pi.intro} 
 
% section introduction (end)

% \input{qm2pi.knotations} 

% section notation (end)

\input{qm2pi.process.calculi} 

% section concurrent_process_calculi_and_spatial_logics_ (end)
    
%\input{qm2pi.knots2pi} 

%\input{qm2pi.trefoil} 

%\input{qm2pi.mainthm} 

% subsection basic_interpretation (end)

%\input{qm2pi.rho.presentation} 
\subsection{The syntax and semantics of the notation system}\label{sub:the_syntax_and_semantics_of_the_notation_system} % (fold)

We now summarize a technical presentation of the calculus that
embodies our theory of dynamics. The typical presentation of such a
calculus follows the style of giving generators and relations on
them. The grammar, below, describing term constructors, freely
generates the set of processes, $\Proc$. This set is then quotiented
by a relation known as structural congruence and it is over this set
that the notion of dynamics is expressed. This presentation is
essentially that of \cite{MeredithR05} with the addition of
polyadicity and summation. For readability we have relegated some of
the technical subtleties to an appendix.

\subsubsection{Process grammar}\label{subsub:process_grammar}

\begin{mathpar}
  \inferrule* [lab=synchronization] {} {{M} \bc \pzero \;|\; x?F \;|\; x!C }
  \and
  \inferrule* [lab=abstraction] {} {{F} \bc (x)P}
  \and
  \inferrule* [lab=concretion] {} {{C} \bc \langle Q \rangle}
  \and
  \inferrule* [lab=process] {} {{P,Q} \bc M \;| \;P|Q \;|\; @{x}}
  \and
  \inferrule* [lab=name] {} {{x} \bc \quotep{P}}
\end{mathpar} 

Note that $\vec{x}$ (resp. $\vec{P}$) denotes a vector of names
(resp. processes) of length $|\vec{x}|$ (resp. $|\vec{P}|$). We adopt
the following useful abbreviations.

\begin{mathpar}
   x?(\vec{y}).P := x.(\vec{y})P \and  x\clift{\vec{P}} := x.\clift{\vec{P}}
   \and x!(y) := \lift{x}{\dropn{y}}
   \and \Pi_{i=0}^{n-1}P_i := P_0 | \ldots | P_{n-1}
\end{mathpar}

\subsubsection{Structural congruence}

\paragraph{Free and bound names and alpha-equivalence.} At the
core of structural equivalence is alpha-equivalence which identifies
process that are the same up to a change of variable. Formally, we
recognize the distinction between free and bound names. The free names
of a process, $\freenames{P}$, may be calculated recursively as
follows:

\begin{mathpar}
\freenames{\pzero} := \emptyset
  \and \\
  \freenames{x?(y).P} := \{ x \} \cup (\freenames{P} \setminus \{ y \})
  \and 
  \freenames{x!\langle P \rangle} := \{ x \} \cup \{ P \} 
  \and \\
  \freenames{P|Q} := \freenames{P} \cup \freenames{Q}
  \and \\
  \freenames{@{x}} := \{ x \}
\end{mathpar}

$\pi$
$\quotep{\pi}$

$\freenames{-} : \pi \to \mathcal{P}(\quotep{\pi})$

\begin{eqnarray*}
  \freenames{\pzero} & := & \emptyset \\
  \freenames{x?(y).P} & := & \{ x \} \cup (\freenames{P} \setminus \{ y \}) \\
  \freenames{x!\langle P \rangle} & := & \{ x \} \cup \{ P \} \\
  \freenames{P|Q} & := & \freenames{P} \cup \freenames{Q} \\
  \freenames{\dropn{x}} & := & \{ x \}
\end{eqnarray*}

The bound names of a process, $\boundnames{P}$, are those names occurring in $P$
that are not free. For example, in $x?(y).0$, the name $x$ is free, while $y$ is bound.

\begin{mathpar}
  \inferrule* [lab=monoidal-laws] {} { P|Q \equiv Q|P \and P|0 \equiv P \and P|(Q|R) \equiv (P|Q)|R }
\end{mathpar}

\begin{mathpar}
  \inferrule* [lab=alpha-equivalence] {} { (x)P \equiv (y)P\{y/x\} \and y \not\in \freenames{P} }
\end{mathpar}

\begin{definition}
Then two processes, $P,Q$, are alpha-equivalent if $P = Q\{\vec{y}/\vec{x}\}$ for
some $\vec{x} \in \boundnames{Q},\vec{y} \in \boundnames{P}$, where $Q\{\vec{y}/\vec{x}\}$
denotes the capture-avoiding substitution of $\vec{y}$ for $\vec{x}$ in $Q$.
\end{definition}

\begin{definition}
  The {\em structural congruence} \cite{SangiorgiWalker} , $\equiv$,
  between processes is the least congruence containing
  alpha-equivalence, satisfying the abelian monoid laws
  (associativity, commutativity and $\pzero$ as identity) for parallel
  composition $|$ and for summation $+$.
\end{definition}

\subsection{Name equivalence}

We take name equivalence, written $\nameeq$, to be the smallest
equivalence relation generated by the following rules.

\begin{mathpar}
\inferrule*[lab=Quote-drop]
{ }
{ \quotep{@{x}} \nameeq x }

\inferrule*[lab=Struct-equiv]
{ P \scong Q }
{ \quotep{P} \nameeq \quotep{Q} }
\end{mathpar}

The astute reader will have noticed that the mutual recursion of names
and processes imposes a mutual recursion on alpha-equivalence and
structural equivalence via name-equivalence. Fortunately, all of this
works out pleasantly and we may calculate in the natural way, free of
concern. The reader interested in the details is referred to the
appendix \ref{appendix:rho_details}.

\subsection{Substitution}

We use $\Proc$ for the set of processes, $\QProc$ for the set of
names, and $\id{\{}\vec{y} / \vec{x} \id{\}}$ to denote partial maps,
$s : \QProc \rightarrow \QProc$. A map, $s$ lifts, uniquely, to a map
on process terms, $\widehat{s} : \Proc \rightarrow \Proc$ by the
following equations.

\begin{mathpar}
  (0) \psubstp{Q}{P} := 0 \\
  (R \juxtap S) \psubstp{Q}{P}
  :=    
  (R)\psubstp{Q}{P} \juxtap (S) \psubstp{Q}{P} \\
  (x?(y).R) \psubstp{Q}{P}    
  :=    
  (x)\substp{Q}{P} (z)\concat( (R \psubstn{z}{y}) \psubstp{Q}{P} ) \\
  (\lift{x}{R}) \psubstp{Q}{P}  
  :=
  \lift{(x)\substp{Q}{P}}{ R \psubstp{Q}{P} } \\
%   (\dropn{x})  \psubstp{Q}{P}       
%   := 
%   \left\{ 
%     \begin{array}{ccc} 
%       \dropn{\quotep{Q}} & & x \nameeq \quotep{P} \\
%       \dropn{x} & & otherwise \\
%     \end{array}
%   \right. 
  (\dropn{x})  \psubstp{Q}{P}       
  := 
  \left\{ 
    \begin{array}{ccc} 
      Q & & x \nameeq \quotep{P} \\
      \dropn{x} & & otherwise \\
    \end{array}
  \right.
\end{mathpar}
 

where

\begin{eqnarray}
  (x)\id{\{} \lpquote Q \rpquote / \lpquote P \rpquote \id{\}}            = 
  \left\{ 
    \begin{array}{ccc}
      \lpquote Q \rpquote & & x \nameeq \lpquote P \rpquote \\
      x & & otherwise \\
    \end{array}
  \right. \nonumber
\end{eqnarray}

and $z$ is chosen distinct from $\quotep{P}$, $\quotep{Q}$, the free
names in $Q$, and all the names in $R$. Our $\alpha$-equivalence will
be built in the standard way from this substitution.

\begin{remark}\label{rem:no_self_referential_names}
  One consequence of these definitions is that $\forall P. \quotep{P}
  \not\in \freenames{P}$.
\end{remark}

\subsection{ Dynamic quote: an example }

Anticipating something of what's to come, consider applying the
substitution, $\widehat{\id{\{}u / z \id{\}}}$, to the following pair
of processes, $\lift{w}{y!(z)}$ and $w[ \lpquote y!(z) \rpquote ]$.

\begin{eqnarray}
	\lift{w}{y!(z)}\widehat{\id{\{}u / z \id{\}}}
		& = &
		\lift{w}{y!(u)} \nonumber\\
	w[ \lpquote y!(z) \rpquote ] \widehat{ \id{\{}u / z \id{\}} }
		& = &
		w[ \lpquote y!(z) \rpquote ] \nonumber
\end{eqnarray}

Because the body of the process between quotes is impervious to
substitution, we get radically different answers. In fact, by
examining the first process in an input context,
e.g. $x?(z).\lift{w}{y!(z)}$, we see that the process under the lift
operator may be shaped by prefixed inputs binding a name inside it. In
this sense, the lift operator will be seen as a way to dynamically
construct processes before reifying them as names.

Finally equipped with these standard features we can present the
dynamics of the calculus.

\subsubsection{Operational semantics} 

Finally, we introduce the computational dynamics. What marks these
algebras as distinct from other more traditionally studied algebraic
structures, e.g. vector spaces or polynomial rings, is the manner in
which dynamics is captured. In traditional structures, dynamics is typically
expressed through morphisms between such structures, as in linear maps
between vector spaces or morphisms between rings. In algebras
associated with the semantics of computation, the dynamics is
expressed as part of the algebraic structure itself, through a
reduction reduction relation typically denoted by $\red$. Below, we
give a recursive presentation of this relation for the calculus used
in the encoding.

$\red \subseteq \pi \times \pi$
$\red : \pi \to \mathcal{P}(\pi)$

\begin{mathpar}
  \inferrule* [lab=Comm] { \textsf{match}( x_{src}, x_{trgt} ) } { x_{trgt}?(y)P \; | \; x_{src}!\langle {Q} \rangle \red P\{\quotep{Q}/y}\} }
  \and \\
  \inferrule* [lab=Par] {{P} \red {P}'} {{{P} | {Q}} \red {{P}' | {Q}}}
  \and
  \inferrule* [lab=Equiv]{{{P} \scong {P}'} \andalso {{P}' \red {Q}'} \andalso {{Q}' \scong {Q}}}{{P} \red {Q}}
\end{mathpar}

\begin{eqnarray*}
  match_{\equiv} (\quotep{P},\quotep{Q}) & := & P \equiv Q \\
  match_{\dagger}(\quotep{P},\quotep{Q}) & := & \forall R. P|Q \red^{*} R => R \red^{*} 0 \\
  match_{K}(\quotep{P},\quotep{Q}) & := & K \mbox{ for some context } K
\end{eqnarray*}

$u?(x)P | u!\langle Q \rangle \red P\{\quotep{Q}/x\}$

%We write $\wred$ for $\red^*$, and $P\red$ if $\exists Q $ such that $ P \red Q$.
We write $P\red$ if $\exists Q $ such that $ P \red Q$ and $P\not\red$, otherwise.

\section{Replication}

As mentioned before, it is known that replication (and hence
recursion) can be implemented in a higher-order process algebra
\cite{SangiorgiWalker}. As our first example of calculation with the
machinery thus far presented we give the construction explicitly in
the {\rhoc}.

\begin{eqnarray}
	D_{x} & := & \prefix{x}{y}{(\binpar{\outputp{x}{y}}{@{y}})} \nonumber\\
	\bangp_{x}{P} & := & \binpar{{x}!\langle{\binpar{D_{x}}{P}}\rangle}{D_{x}} \nonumber
\end{eqnarray}

\begin{eqnarray}
	\bangp_{x}{P} & & \nonumber\\
	=
	& {x}!\langle{(\prefix{x}{y}{(\outputp{x}{y} | @{y})) | P}}\rangle 
	      | \prefix{x}{y}{(\outputp{x}{y} | @{y})} & \nonumber\\
	\red
	& (\outputp{x}{y} | @{y})\substn{\quotep{(\prefix{x}{y}{(@{y} | \outputp{x}{y})) | P}}}{y} & \nonumber\\
	=
	& \outputp{x}{\quotep{(\prefix{x}{y}{(\outputp{x}{y} | @{y})) | P}}}
	  | {(\prefix{x}{y}{(\outputp{x}{y} | @{y})) | P}} & \nonumber\\
	\red
	& \ldots & \nonumber\\
	\red^*
	& P | P | \ldots & \nonumber
\end{eqnarray}

Of course, this encoding, as an implementation, runs away, unfolding
$\bangp{P}$ eagerly. A lazier and more implementable replication
operator, restricted to input-guarded processes, may be obtained as follows.

\begin{eqnarray}
\bangp{\prefix{u}{v}{P}} 
	:= 
	\binpar{\lift{x}{\prefix{u}{v}{(\binpar{D(x)}{P})}}}{D(x)} \nonumber
\end{eqnarray}

\begin{remark}
  Note that the lazier definition still does not deal with summation
  or mixed summation (i.e. sums over input and output). The reader is
  invited to construct definitions of replication that deal with these
  features. 

  Further, the definitions are parameterized in a name, $x$. Can you,
  gentle reader, make a definition that eliminates this parameter and
  guarantees no accidental interaction between the replication
  machinery and the process being replicated -- i.e. no accidental
  sharing of names used by the process to get its work done and the
  name(s) used by the replication to effect copying. This latter
  revision of the definition of replication is crucial to obtaining
  the expected identity $!!P \sim !P$.
\end{remark}

\begin{remark}\label{rem:paradoxical_combinator}
  The reader familiar with the lambda calculus will have noticed the
  similarity between $D$ and the paradoxical combinator.

  [Ed. note: the existence of this seems to suggest we have to be more
  restrictive on the set of processes and names we admit if we are to
  support no-cloning.]
\end{remark}

\subsubsection{Bisimulation}

The computational dynamics gives rise to another kind of equivalence,
the equivalence of computational behavior. As previously mentioned
this is typically captured \emph{via} some form of bisimulation.

% The notion we use in this paper is weak barbed bisimulation
% \cite{milner91polyadicpi}.

The notion we use in this paper is derived from weak barbed
bisimulation \cite{milner91polyadicpi}. 

\begin{definition}
An \emph{observation relation}, $\downarrow_{\mathcal N}$, over a set
of names, $\mathcal N$, is the smallest relation satisfying the rules
below.

\infrule[Out-barb]{y \in {\mathcal N}, \; x \nameeq y}
		  {\outputp{x}{v} \downarrow_{\mathcal N} x}
\infrule[Par-barb]{\mbox{$P\downarrow_{\mathcal N} x$ or $Q\downarrow_{\mathcal N} x$}}
		  {\binpar{P}{Q} \downarrow_{\mathcal N} x}

We write $P \Downarrow_{\mathcal N} x$ if there is $Q$ such that 
$P \wred Q$ and $Q \downarrow_{\mathcal N} x$.
\end{definition}

\begin{definition}
%\label{def.bbisim}
An  ${\mathcal N}$-\emph{barbed bisimulation} over a set of names, ${\mathcal N}$, is a symmetric binary relation 
${\mathcal S}_{\mathcal N}$ between agents such that $P\rel{S}_{\mathcal N}Q$ implies:
\begin{enumerate}
\item If $P \red P'$ then $Q \wred Q'$ and $P'\rel{S}_{\mathcal N} Q'$.
\item If $P\downarrow_{\mathcal N} x$, then $Q\Downarrow_{\mathcal N} x$.
\end{enumerate}
$P$ is ${\mathcal N}$-barbed bisimilar to $Q$, written
$P \wbbisim_{\mathcal N} Q$, if $P \rel{S}_{\mathcal N} Q$ for some ${\mathcal N}$-barbed bisimulation ${\mathcal S}_{\mathcal N}$.
\end{definition}

$\mathcal{R} \subseteq \pi \times \pi$

$P \mathcal{R} Q => \forall P'. P \red P' \Rightarrow \exists Q'. Q \red Q', P' \mathcal{R} Q'$

$P \vdash x \Rightarrow Q \vdash x$

\begin{mathpar}
  \inferrule*[lab=Out-barb]{x \nameeq y}{{y}!\langle{Q}\rangle \vdash x}
  \and
  \inferrule*[lab=Par-barb]{\mbox{$P\vdash x$ or $Q\vdash x$}}{\binpar{P}{Q} \vdash x}
\end{mathpar}

\subsubsection{Contexts}

One of the principle advantages of computational calculi like the
$\pi$-calculus is a well-defined notion of context,
contextual-equivalence and a correlation between
contextual-equivalence and notions of bisimulation. The notion of
context allows the decomposition of a process into (sub-)process and
its syntactic environment, its context. Thus, a context may be
thought of as a process with a ``hole'' (written $\Box$) in it. The
application of a context $M$ to a process $P$, written $M[P]$, is
tantamount to filling the hole in $M$ with $P$. In this paper we do
not need the full weight of this theory, but do make use of the notion
of context in the proof the main theorem. 

\begin{mathpar}
  \inferrule* [lab=summation] {} {{M_{M},M_{N}} \bc \Box \;|\; x.M_{A} \;|\; M_{M}+M_{N}}
  \and
  \inferrule* [lab=agent] {} {{M_{A}} \bc (\vec{x})M_{P} \;| \; \clift{P_0,\ldots,M_{P},\ldots,P_N}}
  \and \\
  \inferrule* [lab=process] {} {{M_{P}} \bc M_{N} \;| \;P|M_{P} }
\end{mathpar} 

\begin{mathpar}
  \inferrule* [lab=sychronization] {} {M_{N} \bc \Box \;|\; x?M_{F} \;|\; x!M_{C}}
  \and
  \inferrule* [lab=abstraction] {} {{M_{F}} \bc (x)M_{P} }
  \and
  \inferrule* [lab=concretion] {} {{M_{C}} \bc \langle M_{P} \rangle }
  \and \\
  \inferrule* [lab=process] {} {{M_{P}} \bc M_{N} \;| \;P|M_{P} }
\end{mathpar}

\begin{definition}[contextual application] Given a context $M$, and
  process $P$, we define the \emph{contextual application}, $M[P] :=
  M\{P/\Box\}$. That is, the contextual application of M to P is the
  substitution of $P$ for $\Box$ in $M$.
\end{definition}

$\meaningof{-} : L \to \mathcal{P}(\pi)$

\begin{mathpar}
  \inferrule* [lab=collection] {} {\meaningof{true} = \pi, \and \meaningof{~E} = \pi \setminus \meaningof{E}, \and \meaningof{E_{1} \& E_{2}} = \meaningof{E_{1}} \cap \meaningof{E_{2}}}
\end{mathpar}

\begin{mathpar}
  \inferrule* [lab=structure] {} {\meaningof{0} = \{ P \in \pi | P \equiv 0 \}, \and \\ \meaningof{E_1 | E_2} = \{ P \in \pi | P \equiv P_{1} | P_{2}, P_{1} \in \meaningof{E_{1}}, P_{2} \in \meaningof{E_2}\} }
\end{mathpar}

\begin{mathpar}
 \inferrule* [lab=behavior] {} {\meaningof{\langle a?b \rangle E} = \{ P \in \pi | P \equiv Q | u?(y)P', \\ \and \\\\ \and \\ \;\;\; u \in \meaningof{a}, \forall z.P'\{z/y\} \in \meaningof{E\{z/b\}}\}, \and \\ \meaningof{a!E} = \{ P \in \pi | P \equiv Q | x!\langle P' \rangle, x \in \meaningof{a} P' \in \meaningof{E}\} }
\end{mathpar}

\begin{mathpar}
 \inferrule* [lab=nominal] {} {\meaningof{\quotep{E}} = \{ \quotep{P} \in \quotep{\pi} | P \in \meaningof{E} \}, \and \meaningof{\quotep{P}} = \{ \quotep{Q} \in \quotep{\pi} | P \equiv Q \} \and \\ \meaningof{@\quotep{E}} = \{ P \in \pi | P \equiv @x, x \in \meaningof{E} \}}
\end{mathpar}

\begin{eqnarray*}
  \\
  \meaningof{-} : TS \to ST
\end{eqnarray*}

\begin{eqnarray*}
  \\
  L : TS \to ST
\end{eqnarray*}

\begin{eqnarray*}
  \\
  P \models E \iff P \in \meaningof{E}
\end{eqnarray*}

\begin{eqnarray*}
  P \approx_{L} Q \iff \forall E \in L. P \models E \iff Q \models E
\end{eqnarray*}

\begin{eqnarray*}
  P \approx_{K} Q
\end{eqnarray*}

\begin{eqnarray*}
  P \approx Q
\end{eqnarray*}

$\approx_{K} = \approx = \approx_{L}$

\subsubsection{Contextual duality}

Note that contexts extend the quotation operation to a family of
operations from processes to names. Given a context, $M$, we can
define a \emph{nominal context}, $\quotep{M}$ by $\quotep{M}[P] :=
\quotep{M[P]}$. To foreshadow what is to come we observe that these
operations enjoy a duality with processes very much like the duality
between vectors and maps from vectors to scalars.

Further, because the calculus is essentially higher-order, we have a
correspondence between contexts and processes. More specifically,
given a name $x$ and a context $M$ we can construct $M^{*}_{x}$ such
that 

\begin{mathpar}
  M^{*}_{x} | \lift{x}{P} \red M[P]
\end{mathpar}

namely,

\begin{mathpar}
  M^{*}_{x} := x?(u).M[\dropn{u}]
\end{mathpar}

The dependence of $M^{*}_{x}$ on a name makes it an abstraction, 

\begin{mathpar}
  M^{*} := (x)x?(u).M[\dropn{u}]
\end{mathpar}

\subsection{Additional notation}

It will sometimes be convenient to denote the process a name
quotes. We already have the notation $x = \quotep{P}$, but it will be
convenient to introduce an alternate notation, $\procn{x}$, when we
want to emphasize the connection to the use of the name. Note that, by
virtue of name equivalence, $\quotep{\procn{x}} \nameeq x$; so, the
notation is consistent with previous definitions.

Further, because names have structure it is possible to effect
substitutions on the basis of that structure. This means we need to
upgrade our notation for substitutions, which we accomplish by
adapting comprehension notation. Thus,

\begin{mathpar}
  P\{ y / x : x \in S \}
\end{mathpar}

is interpreted to mean the process derived from P by replacing (in a
capture-avoiding manner) each occurrence of $x$ in $S$ by $y$. For example,

\begin{mathpar}
  P\{ \quotep{\procn{x}|\procn{x}} / x : x \in \freenames{P} \}
\end{mathpar}

will replace each (occurrence) of a free name $x$ in $P$ by
$\quotep{\procn{x}|\procn{x}}$.

Also, we will avail ourselves of the notation $x^{L}$ and $x^{R}$ to
denote injections of a name into disjoint copies of the name
space. There are numerous ways to accomplish this. One example can be
found in \cite{MeredithR05}. This notation overloads to vectors of
names: $\vec{x}^{\pi} := (x_{i}^{\pi} \; : \; 0 \leq i < |\vec{x}| )$ where $\pi \in \{L,R\}$.

We also use $P^{\Box} := P|\Box$.

In \cite{MeredithR05} an interpretation of the new operator is
given. It turns out that there are several possible interpretations
all enjoying the requisite algebraic properties of the operator (see
\cite{milner91polyadicpi}). We will therefore make liberal use of
$(\nu\; \vec{x})P$.

% subsection the_syntax_and_semantics_of_the_notation_system (end)   

\input{qm2pi.qmops} 

\input{qm2pi.sterngerlach} 

\input{qm2pi.metric} 

% section concurrent_process_calculi (end)

%\input{qm2pi.proofsketch}

% section proof sketch (end)

%\input{qm2pi.slviaknots} 

% section spatial logic via knots (end)

\input{qm2pi.conclusion}

% section conclusion (end)

%\input{qm2pi.dtcodes} 

% section wiring algorithm (end)

\input{qm2pi.ack} 

% section acknowledgments (end)

\newpage


\bibliographystyle{plain}   
\bibliography{../../biblios/main.bib}

\input{qm2pi.rhodetails}

\end{document}



% section front matter (end)

\section{Introduction}\label{sec:introduction} % (fold)
In this draft of the material i am going to have to dispense with the
usual writing conventions adopted in papers on these topics. i'm going
to have adopt whatever tone i need at the time i'm writing up the
calculations. Sometimes this may be very conversational; others it may
be the barest mathematical grunts; others still it may be that i have
lifted text from one of my other papers because the exposition of some
point was better said there. i hope that my readers are not unduly put
out by this decision. i'm not doing this to flout convention or be
rebellious. i find these calculations very technically challenging. To
keep everything going technically, something has to give; i have to
let go of some cognitive burden. So, the academic writing style --
with all of its trade-offs in terms of facilitating technical
communication -- is what i'm letting go of. Perhaps subsequent drafts
can be tightened and polished, but for now, i'm going to speak as if
we were sitting together in a coffee shop with a laptop, wifi and a
pad of paper and a pencil.

So, here's what i have to say. We -- you and i, comfortably ensconced
in our coffee shop and well-equipped with our tools -- can realize and
carry out the calculations of quantum mechanics over a very different
formal theory of dynamics, a formal theory of dynamics that
corresponds to a theory of concurrent computation with
\emph{reflection}. It has the advantage that the underlying theory is
already `quantized', but supports analogues all of the continuuous
operations. Strikingly, this underlying theory has recently been
connected with a notion of metric that we can show, by calculating
together, coincides with the metric induced by the inner product.

There are a lot of reasons why you might be interested in seeing
calculations of this form. Here's why i'm interested. For the past
several centuries there has been no competitor to the ``Newtonian''
account of dynamics. As a result the predominant share of accounts of
dynamical systems and situations have had to be formulated in terms of
the Newtonian machinery. i view this as an intellectually dangerous
position to occupy. Everything, despite it's intrinsic shape, turns
into a nail to be hit with this hammer. Recently, however, the theory
of computation has matured to the point where we have candidates for
theories of dynamics that offer very different perspective on
reasoning about dynamical systems and situations. Testing these
candidates against very successful accounts of dynamical situations,
like quantum mechanics, is going to give us some sense of how mature
they are and some measure of the quality of these accounts of
dynamics.

\subsection{Summary of contributions and outline of paper}

So, we're going to develop an interpretation of the operations of
quantum mechanics normally interpreted by Hilbert spaces and
operators. We're going to do this over a theory of computation. Note
that this is very different than the usual quantum computation program
which develops notions of computation over quantum mechanics. Rather,
we are developing a story that aligns with Wheeler's slogan: It from
Bit. To do this we will first provide an account of the theory of
computation at play here. Then we will dive into a calculation-driven
interpretation of the operations of quantum mechanics.

The reason we take this approach is that -- until very recently --
there hasn't been an axiomatic account of quantum mechanics. As a
result there has been no sharp delineation of the mathematical theory
supporting interpretation of the physical theory and the physical
theory, itself. So, ambient features of the maths are free to be
exploited (or supressed) without a real accounting of their physical
relevance. There is no sharp statement ``here's the physical theory''
qua \emph{theory} and ``here's the mathematical interpretation''
enabling a judgment of how faithful the interpretation is -- apart
from experimental observation. When there is an axiomatic account we
can judge how well a given mathematical formalism supports an
interpretation of the axioms, independent of
experimentation. Likewise, we can judge how well we have captured our
physical evidence and experience with our axiomatics, independent of
any specific mathematical implementation, with accidental detail that
may or may not have physical significance. 

In lieu of a fully fleshed out and vetted axiomatic account of quantum
mechanics, interpreting the operational notions in service of modeling
physical systems will have to suffice. In other words, we are not in
the business of providing a model of Hilbert spaces and operators. We
are in the business of providing a model of quantum mechanics because
we are motivated by testing our notions of dynamics against physical
theory; and, the predictive calculations of the physical theory must
serve as the best formulation -- shy of a fully fleshed out axiomatic
account -- of the physical theory itself (as they have for scientific
theories since time immemorial). Put another way, despite a
whole-hearted commitment to an It-from-Bit ontology, we are firmly
aligned with the shut-up-and-calculate camp as the best way to obtain
results either from the physical perspective or as a quality assurance
measure of our fledgling theory of dynamics.

In detail, we present a reflective process calculus. Then we develop
intuitive correspondences between the notions available in this
calculus and the usual physical notions supporting quantum mechanical
calculations. Thus, 

\begin{table}[htp]
  \center{
    \fbox{
      \begin{tabular}{c|c}
        quantum mechanics & process calculus \\
        \hline
        scalar & name \\
        state vector & process \\
        dual & contextual duals \\
        matrix & formal sums of process-context-dual pairs \\
        orthogonality & process annihilation \\
        inner product & execution-formula + quoting
      \end{tabular}
    }
  }
  \caption{QM - process calculi correspondences}
\end{table}

Then we tighten up these intuitions to operational definitions. We
employ the Dirac notation as the best proxy we can find for an
abstract syntax of the quantum mechanical notions. The definitions we
develop put us in contact with equational constraints coming from the
theory that we demonstrate the definitions and calculations satisfy.

This puts us in a position to shut up and calculate for the
Stern-Gerlach experimental set up, showing how these predictive
calculations become calculations on processes in our theory of a
reflective process calculus.

Penultimately, we demonstrate that the notion of metric coming from
the inner product coincides with the notion of metric available from
the theory of bisimulation. This demonstration gives us the right to
think of space as arising from behavior. Finally, we consider where we
might go from the new vantage point we have obtained.

% section introduction (end) 
 
% section introduction (end)

% \documentclass[12pt]{llncs}
%\documentclass{jktr}

\usepackage[pdftex]{hyperref}                   
\usepackage {listings}
\usepackage {mathpartir}
\usepackage{bcprules}
%\usepackage{listings}
                       
\usepackage{graphicx} 
%\usepackage[margins=2.5cm,nohead,nofoot]{geometry}
%\usepackage{geometry}
\usepackage{amsfonts}
\usepackage{amstext}
\usepackage{latexsym}
\usepackage{amssymb}
\usepackage{color}


%\include{myPreamble}
\include{qm2pi.local} 

%\ifpdf
%\usepackage[pdftex]{graphicx}
%\else
%\usepackage{graphicx}
%\fi

 % \ifpdf
%  \usepackage{pdfsync}
%  \if


%\title{Brief Article}
%\author{David F. Snyder}
%\author{L.G. Meredith}

%\address{Dept. of Math., Texas State University--San Marcos, San Marcos, TX 78666}
       
\pagestyle{empty}


\begin{document}

\lstset{language=[Objective]Caml,frame=shadowbox}

\input{qm2pi.front}

% section front matter (end)

\input{qm2pi.intro} 
 
% section introduction (end)

% \input{qm2pi.knotations} 

% section notation (end)

\input{qm2pi.process.calculi} 

% section concurrent_process_calculi_and_spatial_logics_ (end)
    
%\input{qm2pi.knots2pi} 

%\input{qm2pi.trefoil} 

%\input{qm2pi.mainthm} 

% subsection basic_interpretation (end)

%\input{qm2pi.rho.presentation} 
\subsection{The syntax and semantics of the notation system}\label{sub:the_syntax_and_semantics_of_the_notation_system} % (fold)

We now summarize a technical presentation of the calculus that
embodies our theory of dynamics. The typical presentation of such a
calculus follows the style of giving generators and relations on
them. The grammar, below, describing term constructors, freely
generates the set of processes, $\Proc$. This set is then quotiented
by a relation known as structural congruence and it is over this set
that the notion of dynamics is expressed. This presentation is
essentially that of \cite{MeredithR05} with the addition of
polyadicity and summation. For readability we have relegated some of
the technical subtleties to an appendix.

\subsubsection{Process grammar}\label{subsub:process_grammar}

\begin{mathpar}
  \inferrule* [lab=synchronization] {} {{M} \bc \pzero \;|\; x?F \;|\; x!C }
  \and
  \inferrule* [lab=abstraction] {} {{F} \bc (x)P}
  \and
  \inferrule* [lab=concretion] {} {{C} \bc \langle Q \rangle}
  \and
  \inferrule* [lab=process] {} {{P,Q} \bc M \;| \;P|Q \;|\; @{x}}
  \and
  \inferrule* [lab=name] {} {{x} \bc \quotep{P}}
\end{mathpar} 

Note that $\vec{x}$ (resp. $\vec{P}$) denotes a vector of names
(resp. processes) of length $|\vec{x}|$ (resp. $|\vec{P}|$). We adopt
the following useful abbreviations.

\begin{mathpar}
   x?(\vec{y}).P := x.(\vec{y})P \and  x\clift{\vec{P}} := x.\clift{\vec{P}}
   \and x!(y) := \lift{x}{\dropn{y}}
   \and \Pi_{i=0}^{n-1}P_i := P_0 | \ldots | P_{n-1}
\end{mathpar}

\subsubsection{Structural congruence}

\paragraph{Free and bound names and alpha-equivalence.} At the
core of structural equivalence is alpha-equivalence which identifies
process that are the same up to a change of variable. Formally, we
recognize the distinction between free and bound names. The free names
of a process, $\freenames{P}$, may be calculated recursively as
follows:

\begin{mathpar}
\freenames{\pzero} := \emptyset
  \and \\
  \freenames{x?(y).P} := \{ x \} \cup (\freenames{P} \setminus \{ y \})
  \and 
  \freenames{x!\langle P \rangle} := \{ x \} \cup \{ P \} 
  \and \\
  \freenames{P|Q} := \freenames{P} \cup \freenames{Q}
  \and \\
  \freenames{@{x}} := \{ x \}
\end{mathpar}

$\pi$
$\quotep{\pi}$

$\freenames{-} : \pi \to \mathcal{P}(\quotep{\pi})$

\begin{eqnarray*}
  \freenames{\pzero} & := & \emptyset \\
  \freenames{x?(y).P} & := & \{ x \} \cup (\freenames{P} \setminus \{ y \}) \\
  \freenames{x!\langle P \rangle} & := & \{ x \} \cup \{ P \} \\
  \freenames{P|Q} & := & \freenames{P} \cup \freenames{Q} \\
  \freenames{\dropn{x}} & := & \{ x \}
\end{eqnarray*}

The bound names of a process, $\boundnames{P}$, are those names occurring in $P$
that are not free. For example, in $x?(y).0$, the name $x$ is free, while $y$ is bound.

\begin{mathpar}
  \inferrule* [lab=monoidal-laws] {} { P|Q \equiv Q|P \and P|0 \equiv P \and P|(Q|R) \equiv (P|Q)|R }
\end{mathpar}

\begin{mathpar}
  \inferrule* [lab=alpha-equivalence] {} { (x)P \equiv (y)P\{y/x\} \and y \not\in \freenames{P} }
\end{mathpar}

\begin{definition}
Then two processes, $P,Q$, are alpha-equivalent if $P = Q\{\vec{y}/\vec{x}\}$ for
some $\vec{x} \in \boundnames{Q},\vec{y} \in \boundnames{P}$, where $Q\{\vec{y}/\vec{x}\}$
denotes the capture-avoiding substitution of $\vec{y}$ for $\vec{x}$ in $Q$.
\end{definition}

\begin{definition}
  The {\em structural congruence} \cite{SangiorgiWalker} , $\equiv$,
  between processes is the least congruence containing
  alpha-equivalence, satisfying the abelian monoid laws
  (associativity, commutativity and $\pzero$ as identity) for parallel
  composition $|$ and for summation $+$.
\end{definition}

\subsection{Name equivalence}

We take name equivalence, written $\nameeq$, to be the smallest
equivalence relation generated by the following rules.

\begin{mathpar}
\inferrule*[lab=Quote-drop]
{ }
{ \quotep{@{x}} \nameeq x }

\inferrule*[lab=Struct-equiv]
{ P \scong Q }
{ \quotep{P} \nameeq \quotep{Q} }
\end{mathpar}

The astute reader will have noticed that the mutual recursion of names
and processes imposes a mutual recursion on alpha-equivalence and
structural equivalence via name-equivalence. Fortunately, all of this
works out pleasantly and we may calculate in the natural way, free of
concern. The reader interested in the details is referred to the
appendix \ref{appendix:rho_details}.

\subsection{Substitution}

We use $\Proc$ for the set of processes, $\QProc$ for the set of
names, and $\id{\{}\vec{y} / \vec{x} \id{\}}$ to denote partial maps,
$s : \QProc \rightarrow \QProc$. A map, $s$ lifts, uniquely, to a map
on process terms, $\widehat{s} : \Proc \rightarrow \Proc$ by the
following equations.

\begin{mathpar}
  (0) \psubstp{Q}{P} := 0 \\
  (R \juxtap S) \psubstp{Q}{P}
  :=    
  (R)\psubstp{Q}{P} \juxtap (S) \psubstp{Q}{P} \\
  (x?(y).R) \psubstp{Q}{P}    
  :=    
  (x)\substp{Q}{P} (z)\concat( (R \psubstn{z}{y}) \psubstp{Q}{P} ) \\
  (\lift{x}{R}) \psubstp{Q}{P}  
  :=
  \lift{(x)\substp{Q}{P}}{ R \psubstp{Q}{P} } \\
%   (\dropn{x})  \psubstp{Q}{P}       
%   := 
%   \left\{ 
%     \begin{array}{ccc} 
%       \dropn{\quotep{Q}} & & x \nameeq \quotep{P} \\
%       \dropn{x} & & otherwise \\
%     \end{array}
%   \right. 
  (\dropn{x})  \psubstp{Q}{P}       
  := 
  \left\{ 
    \begin{array}{ccc} 
      Q & & x \nameeq \quotep{P} \\
      \dropn{x} & & otherwise \\
    \end{array}
  \right.
\end{mathpar}
 

where

\begin{eqnarray}
  (x)\id{\{} \lpquote Q \rpquote / \lpquote P \rpquote \id{\}}            = 
  \left\{ 
    \begin{array}{ccc}
      \lpquote Q \rpquote & & x \nameeq \lpquote P \rpquote \\
      x & & otherwise \\
    \end{array}
  \right. \nonumber
\end{eqnarray}

and $z$ is chosen distinct from $\quotep{P}$, $\quotep{Q}$, the free
names in $Q$, and all the names in $R$. Our $\alpha$-equivalence will
be built in the standard way from this substitution.

\begin{remark}\label{rem:no_self_referential_names}
  One consequence of these definitions is that $\forall P. \quotep{P}
  \not\in \freenames{P}$.
\end{remark}

\subsection{ Dynamic quote: an example }

Anticipating something of what's to come, consider applying the
substitution, $\widehat{\id{\{}u / z \id{\}}}$, to the following pair
of processes, $\lift{w}{y!(z)}$ and $w[ \lpquote y!(z) \rpquote ]$.

\begin{eqnarray}
	\lift{w}{y!(z)}\widehat{\id{\{}u / z \id{\}}}
		& = &
		\lift{w}{y!(u)} \nonumber\\
	w[ \lpquote y!(z) \rpquote ] \widehat{ \id{\{}u / z \id{\}} }
		& = &
		w[ \lpquote y!(z) \rpquote ] \nonumber
\end{eqnarray}

Because the body of the process between quotes is impervious to
substitution, we get radically different answers. In fact, by
examining the first process in an input context,
e.g. $x?(z).\lift{w}{y!(z)}$, we see that the process under the lift
operator may be shaped by prefixed inputs binding a name inside it. In
this sense, the lift operator will be seen as a way to dynamically
construct processes before reifying them as names.

Finally equipped with these standard features we can present the
dynamics of the calculus.

\subsubsection{Operational semantics} 

Finally, we introduce the computational dynamics. What marks these
algebras as distinct from other more traditionally studied algebraic
structures, e.g. vector spaces or polynomial rings, is the manner in
which dynamics is captured. In traditional structures, dynamics is typically
expressed through morphisms between such structures, as in linear maps
between vector spaces or morphisms between rings. In algebras
associated with the semantics of computation, the dynamics is
expressed as part of the algebraic structure itself, through a
reduction reduction relation typically denoted by $\red$. Below, we
give a recursive presentation of this relation for the calculus used
in the encoding.

$\red \subseteq \pi \times \pi$
$\red : \pi \to \mathcal{P}(\pi)$

\begin{mathpar}
  \inferrule* [lab=Comm] { \textsf{match}( x_{src}, x_{trgt} ) } { x_{trgt}?(y)P \; | \; x_{src}!\langle {Q} \rangle \red P\{\quotep{Q}/y}\} }
  \and \\
  \inferrule* [lab=Par] {{P} \red {P}'} {{{P} | {Q}} \red {{P}' | {Q}}}
  \and
  \inferrule* [lab=Equiv]{{{P} \scong {P}'} \andalso {{P}' \red {Q}'} \andalso {{Q}' \scong {Q}}}{{P} \red {Q}}
\end{mathpar}

\begin{eqnarray*}
  match_{\equiv} (\quotep{P},\quotep{Q}) & := & P \equiv Q \\
  match_{\dagger}(\quotep{P},\quotep{Q}) & := & \forall R. P|Q \red^{*} R => R \red^{*} 0 \\
  match_{K}(\quotep{P},\quotep{Q}) & := & K \mbox{ for some context } K
\end{eqnarray*}

$u?(x)P | u!\langle Q \rangle \red P\{\quotep{Q}/x\}$

%We write $\wred$ for $\red^*$, and $P\red$ if $\exists Q $ such that $ P \red Q$.
We write $P\red$ if $\exists Q $ such that $ P \red Q$ and $P\not\red$, otherwise.

\section{Replication}

As mentioned before, it is known that replication (and hence
recursion) can be implemented in a higher-order process algebra
\cite{SangiorgiWalker}. As our first example of calculation with the
machinery thus far presented we give the construction explicitly in
the {\rhoc}.

\begin{eqnarray}
	D_{x} & := & \prefix{x}{y}{(\binpar{\outputp{x}{y}}{@{y}})} \nonumber\\
	\bangp_{x}{P} & := & \binpar{{x}!\langle{\binpar{D_{x}}{P}}\rangle}{D_{x}} \nonumber
\end{eqnarray}

\begin{eqnarray}
	\bangp_{x}{P} & & \nonumber\\
	=
	& {x}!\langle{(\prefix{x}{y}{(\outputp{x}{y} | @{y})) | P}}\rangle 
	      | \prefix{x}{y}{(\outputp{x}{y} | @{y})} & \nonumber\\
	\red
	& (\outputp{x}{y} | @{y})\substn{\quotep{(\prefix{x}{y}{(@{y} | \outputp{x}{y})) | P}}}{y} & \nonumber\\
	=
	& \outputp{x}{\quotep{(\prefix{x}{y}{(\outputp{x}{y} | @{y})) | P}}}
	  | {(\prefix{x}{y}{(\outputp{x}{y} | @{y})) | P}} & \nonumber\\
	\red
	& \ldots & \nonumber\\
	\red^*
	& P | P | \ldots & \nonumber
\end{eqnarray}

Of course, this encoding, as an implementation, runs away, unfolding
$\bangp{P}$ eagerly. A lazier and more implementable replication
operator, restricted to input-guarded processes, may be obtained as follows.

\begin{eqnarray}
\bangp{\prefix{u}{v}{P}} 
	:= 
	\binpar{\lift{x}{\prefix{u}{v}{(\binpar{D(x)}{P})}}}{D(x)} \nonumber
\end{eqnarray}

\begin{remark}
  Note that the lazier definition still does not deal with summation
  or mixed summation (i.e. sums over input and output). The reader is
  invited to construct definitions of replication that deal with these
  features. 

  Further, the definitions are parameterized in a name, $x$. Can you,
  gentle reader, make a definition that eliminates this parameter and
  guarantees no accidental interaction between the replication
  machinery and the process being replicated -- i.e. no accidental
  sharing of names used by the process to get its work done and the
  name(s) used by the replication to effect copying. This latter
  revision of the definition of replication is crucial to obtaining
  the expected identity $!!P \sim !P$.
\end{remark}

\begin{remark}\label{rem:paradoxical_combinator}
  The reader familiar with the lambda calculus will have noticed the
  similarity between $D$ and the paradoxical combinator.

  [Ed. note: the existence of this seems to suggest we have to be more
  restrictive on the set of processes and names we admit if we are to
  support no-cloning.]
\end{remark}

\subsubsection{Bisimulation}

The computational dynamics gives rise to another kind of equivalence,
the equivalence of computational behavior. As previously mentioned
this is typically captured \emph{via} some form of bisimulation.

% The notion we use in this paper is weak barbed bisimulation
% \cite{milner91polyadicpi}.

The notion we use in this paper is derived from weak barbed
bisimulation \cite{milner91polyadicpi}. 

\begin{definition}
An \emph{observation relation}, $\downarrow_{\mathcal N}$, over a set
of names, $\mathcal N$, is the smallest relation satisfying the rules
below.

\infrule[Out-barb]{y \in {\mathcal N}, \; x \nameeq y}
		  {\outputp{x}{v} \downarrow_{\mathcal N} x}
\infrule[Par-barb]{\mbox{$P\downarrow_{\mathcal N} x$ or $Q\downarrow_{\mathcal N} x$}}
		  {\binpar{P}{Q} \downarrow_{\mathcal N} x}

We write $P \Downarrow_{\mathcal N} x$ if there is $Q$ such that 
$P \wred Q$ and $Q \downarrow_{\mathcal N} x$.
\end{definition}

\begin{definition}
%\label{def.bbisim}
An  ${\mathcal N}$-\emph{barbed bisimulation} over a set of names, ${\mathcal N}$, is a symmetric binary relation 
${\mathcal S}_{\mathcal N}$ between agents such that $P\rel{S}_{\mathcal N}Q$ implies:
\begin{enumerate}
\item If $P \red P'$ then $Q \wred Q'$ and $P'\rel{S}_{\mathcal N} Q'$.
\item If $P\downarrow_{\mathcal N} x$, then $Q\Downarrow_{\mathcal N} x$.
\end{enumerate}
$P$ is ${\mathcal N}$-barbed bisimilar to $Q$, written
$P \wbbisim_{\mathcal N} Q$, if $P \rel{S}_{\mathcal N} Q$ for some ${\mathcal N}$-barbed bisimulation ${\mathcal S}_{\mathcal N}$.
\end{definition}

$\mathcal{R} \subseteq \pi \times \pi$

$P \mathcal{R} Q => \forall P'. P \red P' \Rightarrow \exists Q'. Q \red Q', P' \mathcal{R} Q'$

$P \vdash x \Rightarrow Q \vdash x$

\begin{mathpar}
  \inferrule*[lab=Out-barb]{x \nameeq y}{{y}!\langle{Q}\rangle \vdash x}
  \and
  \inferrule*[lab=Par-barb]{\mbox{$P\vdash x$ or $Q\vdash x$}}{\binpar{P}{Q} \vdash x}
\end{mathpar}

\subsubsection{Contexts}

One of the principle advantages of computational calculi like the
$\pi$-calculus is a well-defined notion of context,
contextual-equivalence and a correlation between
contextual-equivalence and notions of bisimulation. The notion of
context allows the decomposition of a process into (sub-)process and
its syntactic environment, its context. Thus, a context may be
thought of as a process with a ``hole'' (written $\Box$) in it. The
application of a context $M$ to a process $P$, written $M[P]$, is
tantamount to filling the hole in $M$ with $P$. In this paper we do
not need the full weight of this theory, but do make use of the notion
of context in the proof the main theorem. 

\begin{mathpar}
  \inferrule* [lab=summation] {} {{M_{M},M_{N}} \bc \Box \;|\; x.M_{A} \;|\; M_{M}+M_{N}}
  \and
  \inferrule* [lab=agent] {} {{M_{A}} \bc (\vec{x})M_{P} \;| \; \clift{P_0,\ldots,M_{P},\ldots,P_N}}
  \and \\
  \inferrule* [lab=process] {} {{M_{P}} \bc M_{N} \;| \;P|M_{P} }
\end{mathpar} 

\begin{mathpar}
  \inferrule* [lab=sychronization] {} {M_{N} \bc \Box \;|\; x?M_{F} \;|\; x!M_{C}}
  \and
  \inferrule* [lab=abstraction] {} {{M_{F}} \bc (x)M_{P} }
  \and
  \inferrule* [lab=concretion] {} {{M_{C}} \bc \langle M_{P} \rangle }
  \and \\
  \inferrule* [lab=process] {} {{M_{P}} \bc M_{N} \;| \;P|M_{P} }
\end{mathpar}

\begin{definition}[contextual application] Given a context $M$, and
  process $P$, we define the \emph{contextual application}, $M[P] :=
  M\{P/\Box\}$. That is, the contextual application of M to P is the
  substitution of $P$ for $\Box$ in $M$.
\end{definition}

$\meaningof{-} : L \to \mathcal{P}(\pi)$

\begin{mathpar}
  \inferrule* [lab=collection] {} {\meaningof{true} = \pi, \and \meaningof{~E} = \pi \setminus \meaningof{E}, \and \meaningof{E_{1} \& E_{2}} = \meaningof{E_{1}} \cap \meaningof{E_{2}}}
\end{mathpar}

\begin{mathpar}
  \inferrule* [lab=structure] {} {\meaningof{0} = \{ P \in \pi | P \equiv 0 \}, \and \\ \meaningof{E_1 | E_2} = \{ P \in \pi | P \equiv P_{1} | P_{2}, P_{1} \in \meaningof{E_{1}}, P_{2} \in \meaningof{E_2}\} }
\end{mathpar}

\begin{mathpar}
 \inferrule* [lab=behavior] {} {\meaningof{\langle a?b \rangle E} = \{ P \in \pi | P \equiv Q | u?(y)P', \\ \and \\\\ \and \\ \;\;\; u \in \meaningof{a}, \forall z.P'\{z/y\} \in \meaningof{E\{z/b\}}\}, \and \\ \meaningof{a!E} = \{ P \in \pi | P \equiv Q | x!\langle P' \rangle, x \in \meaningof{a} P' \in \meaningof{E}\} }
\end{mathpar}

\begin{mathpar}
 \inferrule* [lab=nominal] {} {\meaningof{\quotep{E}} = \{ \quotep{P} \in \quotep{\pi} | P \in \meaningof{E} \}, \and \meaningof{\quotep{P}} = \{ \quotep{Q} \in \quotep{\pi} | P \equiv Q \} \and \\ \meaningof{@\quotep{E}} = \{ P \in \pi | P \equiv @x, x \in \meaningof{E} \}}
\end{mathpar}

\begin{eqnarray*}
  \\
  \meaningof{-} : TS \to ST
\end{eqnarray*}

\begin{eqnarray*}
  \\
  L : TS \to ST
\end{eqnarray*}

\begin{eqnarray*}
  \\
  P \models E \iff P \in \meaningof{E}
\end{eqnarray*}

\begin{eqnarray*}
  P \approx_{L} Q \iff \forall E \in L. P \models E \iff Q \models E
\end{eqnarray*}

\begin{eqnarray*}
  P \approx_{K} Q
\end{eqnarray*}

\begin{eqnarray*}
  P \approx Q
\end{eqnarray*}

$\approx_{K} = \approx = \approx_{L}$

\subsubsection{Contextual duality}

Note that contexts extend the quotation operation to a family of
operations from processes to names. Given a context, $M$, we can
define a \emph{nominal context}, $\quotep{M}$ by $\quotep{M}[P] :=
\quotep{M[P]}$. To foreshadow what is to come we observe that these
operations enjoy a duality with processes very much like the duality
between vectors and maps from vectors to scalars.

Further, because the calculus is essentially higher-order, we have a
correspondence between contexts and processes. More specifically,
given a name $x$ and a context $M$ we can construct $M^{*}_{x}$ such
that 

\begin{mathpar}
  M^{*}_{x} | \lift{x}{P} \red M[P]
\end{mathpar}

namely,

\begin{mathpar}
  M^{*}_{x} := x?(u).M[\dropn{u}]
\end{mathpar}

The dependence of $M^{*}_{x}$ on a name makes it an abstraction, 

\begin{mathpar}
  M^{*} := (x)x?(u).M[\dropn{u}]
\end{mathpar}

\subsection{Additional notation}

It will sometimes be convenient to denote the process a name
quotes. We already have the notation $x = \quotep{P}$, but it will be
convenient to introduce an alternate notation, $\procn{x}$, when we
want to emphasize the connection to the use of the name. Note that, by
virtue of name equivalence, $\quotep{\procn{x}} \nameeq x$; so, the
notation is consistent with previous definitions.

Further, because names have structure it is possible to effect
substitutions on the basis of that structure. This means we need to
upgrade our notation for substitutions, which we accomplish by
adapting comprehension notation. Thus,

\begin{mathpar}
  P\{ y / x : x \in S \}
\end{mathpar}

is interpreted to mean the process derived from P by replacing (in a
capture-avoiding manner) each occurrence of $x$ in $S$ by $y$. For example,

\begin{mathpar}
  P\{ \quotep{\procn{x}|\procn{x}} / x : x \in \freenames{P} \}
\end{mathpar}

will replace each (occurrence) of a free name $x$ in $P$ by
$\quotep{\procn{x}|\procn{x}}$.

Also, we will avail ourselves of the notation $x^{L}$ and $x^{R}$ to
denote injections of a name into disjoint copies of the name
space. There are numerous ways to accomplish this. One example can be
found in \cite{MeredithR05}. This notation overloads to vectors of
names: $\vec{x}^{\pi} := (x_{i}^{\pi} \; : \; 0 \leq i < |\vec{x}| )$ where $\pi \in \{L,R\}$.

We also use $P^{\Box} := P|\Box$.

In \cite{MeredithR05} an interpretation of the new operator is
given. It turns out that there are several possible interpretations
all enjoying the requisite algebraic properties of the operator (see
\cite{milner91polyadicpi}). We will therefore make liberal use of
$(\nu\; \vec{x})P$.

% subsection the_syntax_and_semantics_of_the_notation_system (end)   

\input{qm2pi.qmops} 

\input{qm2pi.sterngerlach} 

\input{qm2pi.metric} 

% section concurrent_process_calculi (end)

%\input{qm2pi.proofsketch}

% section proof sketch (end)

%\input{qm2pi.slviaknots} 

% section spatial logic via knots (end)

\input{qm2pi.conclusion}

% section conclusion (end)

%\input{qm2pi.dtcodes} 

% section wiring algorithm (end)

\input{qm2pi.ack} 

% section acknowledgments (end)

\newpage


\bibliographystyle{plain}   
\bibliography{../../biblios/main.bib}

\input{qm2pi.rhodetails}

\end{document}

 

% section notation (end)

\input{qm2pi.process.calculi} 

% section concurrent_process_calculi_and_spatial_logics_ (end)
    
%\documentclass[12pt]{llncs}
%\documentclass{jktr}

\usepackage[pdftex]{hyperref}                   
\usepackage {listings}
\usepackage {mathpartir}
\usepackage{bcprules}
%\usepackage{listings}
                       
\usepackage{graphicx} 
%\usepackage[margins=2.5cm,nohead,nofoot]{geometry}
%\usepackage{geometry}
\usepackage{amsfonts}
\usepackage{amstext}
\usepackage{latexsym}
\usepackage{amssymb}
\usepackage{color}


%\include{myPreamble}
\include{qm2pi.local} 

%\ifpdf
%\usepackage[pdftex]{graphicx}
%\else
%\usepackage{graphicx}
%\fi

 % \ifpdf
%  \usepackage{pdfsync}
%  \if


%\title{Brief Article}
%\author{David F. Snyder}
%\author{L.G. Meredith}

%\address{Dept. of Math., Texas State University--San Marcos, San Marcos, TX 78666}
       
\pagestyle{empty}


\begin{document}

\lstset{language=[Objective]Caml,frame=shadowbox}

\input{qm2pi.front}

% section front matter (end)

\input{qm2pi.intro} 
 
% section introduction (end)

% \input{qm2pi.knotations} 

% section notation (end)

\input{qm2pi.process.calculi} 

% section concurrent_process_calculi_and_spatial_logics_ (end)
    
%\input{qm2pi.knots2pi} 

%\input{qm2pi.trefoil} 

%\input{qm2pi.mainthm} 

% subsection basic_interpretation (end)

%\input{qm2pi.rho.presentation} 
\subsection{The syntax and semantics of the notation system}\label{sub:the_syntax_and_semantics_of_the_notation_system} % (fold)

We now summarize a technical presentation of the calculus that
embodies our theory of dynamics. The typical presentation of such a
calculus follows the style of giving generators and relations on
them. The grammar, below, describing term constructors, freely
generates the set of processes, $\Proc$. This set is then quotiented
by a relation known as structural congruence and it is over this set
that the notion of dynamics is expressed. This presentation is
essentially that of \cite{MeredithR05} with the addition of
polyadicity and summation. For readability we have relegated some of
the technical subtleties to an appendix.

\subsubsection{Process grammar}\label{subsub:process_grammar}

\begin{mathpar}
  \inferrule* [lab=synchronization] {} {{M} \bc \pzero \;|\; x?F \;|\; x!C }
  \and
  \inferrule* [lab=abstraction] {} {{F} \bc (x)P}
  \and
  \inferrule* [lab=concretion] {} {{C} \bc \langle Q \rangle}
  \and
  \inferrule* [lab=process] {} {{P,Q} \bc M \;| \;P|Q \;|\; @{x}}
  \and
  \inferrule* [lab=name] {} {{x} \bc \quotep{P}}
\end{mathpar} 

Note that $\vec{x}$ (resp. $\vec{P}$) denotes a vector of names
(resp. processes) of length $|\vec{x}|$ (resp. $|\vec{P}|$). We adopt
the following useful abbreviations.

\begin{mathpar}
   x?(\vec{y}).P := x.(\vec{y})P \and  x\clift{\vec{P}} := x.\clift{\vec{P}}
   \and x!(y) := \lift{x}{\dropn{y}}
   \and \Pi_{i=0}^{n-1}P_i := P_0 | \ldots | P_{n-1}
\end{mathpar}

\subsubsection{Structural congruence}

\paragraph{Free and bound names and alpha-equivalence.} At the
core of structural equivalence is alpha-equivalence which identifies
process that are the same up to a change of variable. Formally, we
recognize the distinction between free and bound names. The free names
of a process, $\freenames{P}$, may be calculated recursively as
follows:

\begin{mathpar}
\freenames{\pzero} := \emptyset
  \and \\
  \freenames{x?(y).P} := \{ x \} \cup (\freenames{P} \setminus \{ y \})
  \and 
  \freenames{x!\langle P \rangle} := \{ x \} \cup \{ P \} 
  \and \\
  \freenames{P|Q} := \freenames{P} \cup \freenames{Q}
  \and \\
  \freenames{@{x}} := \{ x \}
\end{mathpar}

$\pi$
$\quotep{\pi}$

$\freenames{-} : \pi \to \mathcal{P}(\quotep{\pi})$

\begin{eqnarray*}
  \freenames{\pzero} & := & \emptyset \\
  \freenames{x?(y).P} & := & \{ x \} \cup (\freenames{P} \setminus \{ y \}) \\
  \freenames{x!\langle P \rangle} & := & \{ x \} \cup \{ P \} \\
  \freenames{P|Q} & := & \freenames{P} \cup \freenames{Q} \\
  \freenames{\dropn{x}} & := & \{ x \}
\end{eqnarray*}

The bound names of a process, $\boundnames{P}$, are those names occurring in $P$
that are not free. For example, in $x?(y).0$, the name $x$ is free, while $y$ is bound.

\begin{mathpar}
  \inferrule* [lab=monoidal-laws] {} { P|Q \equiv Q|P \and P|0 \equiv P \and P|(Q|R) \equiv (P|Q)|R }
\end{mathpar}

\begin{mathpar}
  \inferrule* [lab=alpha-equivalence] {} { (x)P \equiv (y)P\{y/x\} \and y \not\in \freenames{P} }
\end{mathpar}

\begin{definition}
Then two processes, $P,Q$, are alpha-equivalent if $P = Q\{\vec{y}/\vec{x}\}$ for
some $\vec{x} \in \boundnames{Q},\vec{y} \in \boundnames{P}$, where $Q\{\vec{y}/\vec{x}\}$
denotes the capture-avoiding substitution of $\vec{y}$ for $\vec{x}$ in $Q$.
\end{definition}

\begin{definition}
  The {\em structural congruence} \cite{SangiorgiWalker} , $\equiv$,
  between processes is the least congruence containing
  alpha-equivalence, satisfying the abelian monoid laws
  (associativity, commutativity and $\pzero$ as identity) for parallel
  composition $|$ and for summation $+$.
\end{definition}

\subsection{Name equivalence}

We take name equivalence, written $\nameeq$, to be the smallest
equivalence relation generated by the following rules.

\begin{mathpar}
\inferrule*[lab=Quote-drop]
{ }
{ \quotep{@{x}} \nameeq x }

\inferrule*[lab=Struct-equiv]
{ P \scong Q }
{ \quotep{P} \nameeq \quotep{Q} }
\end{mathpar}

The astute reader will have noticed that the mutual recursion of names
and processes imposes a mutual recursion on alpha-equivalence and
structural equivalence via name-equivalence. Fortunately, all of this
works out pleasantly and we may calculate in the natural way, free of
concern. The reader interested in the details is referred to the
appendix \ref{appendix:rho_details}.

\subsection{Substitution}

We use $\Proc$ for the set of processes, $\QProc$ for the set of
names, and $\id{\{}\vec{y} / \vec{x} \id{\}}$ to denote partial maps,
$s : \QProc \rightarrow \QProc$. A map, $s$ lifts, uniquely, to a map
on process terms, $\widehat{s} : \Proc \rightarrow \Proc$ by the
following equations.

\begin{mathpar}
  (0) \psubstp{Q}{P} := 0 \\
  (R \juxtap S) \psubstp{Q}{P}
  :=    
  (R)\psubstp{Q}{P} \juxtap (S) \psubstp{Q}{P} \\
  (x?(y).R) \psubstp{Q}{P}    
  :=    
  (x)\substp{Q}{P} (z)\concat( (R \psubstn{z}{y}) \psubstp{Q}{P} ) \\
  (\lift{x}{R}) \psubstp{Q}{P}  
  :=
  \lift{(x)\substp{Q}{P}}{ R \psubstp{Q}{P} } \\
%   (\dropn{x})  \psubstp{Q}{P}       
%   := 
%   \left\{ 
%     \begin{array}{ccc} 
%       \dropn{\quotep{Q}} & & x \nameeq \quotep{P} \\
%       \dropn{x} & & otherwise \\
%     \end{array}
%   \right. 
  (\dropn{x})  \psubstp{Q}{P}       
  := 
  \left\{ 
    \begin{array}{ccc} 
      Q & & x \nameeq \quotep{P} \\
      \dropn{x} & & otherwise \\
    \end{array}
  \right.
\end{mathpar}
 

where

\begin{eqnarray}
  (x)\id{\{} \lpquote Q \rpquote / \lpquote P \rpquote \id{\}}            = 
  \left\{ 
    \begin{array}{ccc}
      \lpquote Q \rpquote & & x \nameeq \lpquote P \rpquote \\
      x & & otherwise \\
    \end{array}
  \right. \nonumber
\end{eqnarray}

and $z$ is chosen distinct from $\quotep{P}$, $\quotep{Q}$, the free
names in $Q$, and all the names in $R$. Our $\alpha$-equivalence will
be built in the standard way from this substitution.

\begin{remark}\label{rem:no_self_referential_names}
  One consequence of these definitions is that $\forall P. \quotep{P}
  \not\in \freenames{P}$.
\end{remark}

\subsection{ Dynamic quote: an example }

Anticipating something of what's to come, consider applying the
substitution, $\widehat{\id{\{}u / z \id{\}}}$, to the following pair
of processes, $\lift{w}{y!(z)}$ and $w[ \lpquote y!(z) \rpquote ]$.

\begin{eqnarray}
	\lift{w}{y!(z)}\widehat{\id{\{}u / z \id{\}}}
		& = &
		\lift{w}{y!(u)} \nonumber\\
	w[ \lpquote y!(z) \rpquote ] \widehat{ \id{\{}u / z \id{\}} }
		& = &
		w[ \lpquote y!(z) \rpquote ] \nonumber
\end{eqnarray}

Because the body of the process between quotes is impervious to
substitution, we get radically different answers. In fact, by
examining the first process in an input context,
e.g. $x?(z).\lift{w}{y!(z)}$, we see that the process under the lift
operator may be shaped by prefixed inputs binding a name inside it. In
this sense, the lift operator will be seen as a way to dynamically
construct processes before reifying them as names.

Finally equipped with these standard features we can present the
dynamics of the calculus.

\subsubsection{Operational semantics} 

Finally, we introduce the computational dynamics. What marks these
algebras as distinct from other more traditionally studied algebraic
structures, e.g. vector spaces or polynomial rings, is the manner in
which dynamics is captured. In traditional structures, dynamics is typically
expressed through morphisms between such structures, as in linear maps
between vector spaces or morphisms between rings. In algebras
associated with the semantics of computation, the dynamics is
expressed as part of the algebraic structure itself, through a
reduction reduction relation typically denoted by $\red$. Below, we
give a recursive presentation of this relation for the calculus used
in the encoding.

$\red \subseteq \pi \times \pi$
$\red : \pi \to \mathcal{P}(\pi)$

\begin{mathpar}
  \inferrule* [lab=Comm] { \textsf{match}( x_{src}, x_{trgt} ) } { x_{trgt}?(y)P \; | \; x_{src}!\langle {Q} \rangle \red P\{\quotep{Q}/y}\} }
  \and \\
  \inferrule* [lab=Par] {{P} \red {P}'} {{{P} | {Q}} \red {{P}' | {Q}}}
  \and
  \inferrule* [lab=Equiv]{{{P} \scong {P}'} \andalso {{P}' \red {Q}'} \andalso {{Q}' \scong {Q}}}{{P} \red {Q}}
\end{mathpar}

\begin{eqnarray*}
  match_{\equiv} (\quotep{P},\quotep{Q}) & := & P \equiv Q \\
  match_{\dagger}(\quotep{P},\quotep{Q}) & := & \forall R. P|Q \red^{*} R => R \red^{*} 0 \\
  match_{K}(\quotep{P},\quotep{Q}) & := & K \mbox{ for some context } K
\end{eqnarray*}

$u?(x)P | u!\langle Q \rangle \red P\{\quotep{Q}/x\}$

%We write $\wred$ for $\red^*$, and $P\red$ if $\exists Q $ such that $ P \red Q$.
We write $P\red$ if $\exists Q $ such that $ P \red Q$ and $P\not\red$, otherwise.

\section{Replication}

As mentioned before, it is known that replication (and hence
recursion) can be implemented in a higher-order process algebra
\cite{SangiorgiWalker}. As our first example of calculation with the
machinery thus far presented we give the construction explicitly in
the {\rhoc}.

\begin{eqnarray}
	D_{x} & := & \prefix{x}{y}{(\binpar{\outputp{x}{y}}{@{y}})} \nonumber\\
	\bangp_{x}{P} & := & \binpar{{x}!\langle{\binpar{D_{x}}{P}}\rangle}{D_{x}} \nonumber
\end{eqnarray}

\begin{eqnarray}
	\bangp_{x}{P} & & \nonumber\\
	=
	& {x}!\langle{(\prefix{x}{y}{(\outputp{x}{y} | @{y})) | P}}\rangle 
	      | \prefix{x}{y}{(\outputp{x}{y} | @{y})} & \nonumber\\
	\red
	& (\outputp{x}{y} | @{y})\substn{\quotep{(\prefix{x}{y}{(@{y} | \outputp{x}{y})) | P}}}{y} & \nonumber\\
	=
	& \outputp{x}{\quotep{(\prefix{x}{y}{(\outputp{x}{y} | @{y})) | P}}}
	  | {(\prefix{x}{y}{(\outputp{x}{y} | @{y})) | P}} & \nonumber\\
	\red
	& \ldots & \nonumber\\
	\red^*
	& P | P | \ldots & \nonumber
\end{eqnarray}

Of course, this encoding, as an implementation, runs away, unfolding
$\bangp{P}$ eagerly. A lazier and more implementable replication
operator, restricted to input-guarded processes, may be obtained as follows.

\begin{eqnarray}
\bangp{\prefix{u}{v}{P}} 
	:= 
	\binpar{\lift{x}{\prefix{u}{v}{(\binpar{D(x)}{P})}}}{D(x)} \nonumber
\end{eqnarray}

\begin{remark}
  Note that the lazier definition still does not deal with summation
  or mixed summation (i.e. sums over input and output). The reader is
  invited to construct definitions of replication that deal with these
  features. 

  Further, the definitions are parameterized in a name, $x$. Can you,
  gentle reader, make a definition that eliminates this parameter and
  guarantees no accidental interaction between the replication
  machinery and the process being replicated -- i.e. no accidental
  sharing of names used by the process to get its work done and the
  name(s) used by the replication to effect copying. This latter
  revision of the definition of replication is crucial to obtaining
  the expected identity $!!P \sim !P$.
\end{remark}

\begin{remark}\label{rem:paradoxical_combinator}
  The reader familiar with the lambda calculus will have noticed the
  similarity between $D$ and the paradoxical combinator.

  [Ed. note: the existence of this seems to suggest we have to be more
  restrictive on the set of processes and names we admit if we are to
  support no-cloning.]
\end{remark}

\subsubsection{Bisimulation}

The computational dynamics gives rise to another kind of equivalence,
the equivalence of computational behavior. As previously mentioned
this is typically captured \emph{via} some form of bisimulation.

% The notion we use in this paper is weak barbed bisimulation
% \cite{milner91polyadicpi}.

The notion we use in this paper is derived from weak barbed
bisimulation \cite{milner91polyadicpi}. 

\begin{definition}
An \emph{observation relation}, $\downarrow_{\mathcal N}$, over a set
of names, $\mathcal N$, is the smallest relation satisfying the rules
below.

\infrule[Out-barb]{y \in {\mathcal N}, \; x \nameeq y}
		  {\outputp{x}{v} \downarrow_{\mathcal N} x}
\infrule[Par-barb]{\mbox{$P\downarrow_{\mathcal N} x$ or $Q\downarrow_{\mathcal N} x$}}
		  {\binpar{P}{Q} \downarrow_{\mathcal N} x}

We write $P \Downarrow_{\mathcal N} x$ if there is $Q$ such that 
$P \wred Q$ and $Q \downarrow_{\mathcal N} x$.
\end{definition}

\begin{definition}
%\label{def.bbisim}
An  ${\mathcal N}$-\emph{barbed bisimulation} over a set of names, ${\mathcal N}$, is a symmetric binary relation 
${\mathcal S}_{\mathcal N}$ between agents such that $P\rel{S}_{\mathcal N}Q$ implies:
\begin{enumerate}
\item If $P \red P'$ then $Q \wred Q'$ and $P'\rel{S}_{\mathcal N} Q'$.
\item If $P\downarrow_{\mathcal N} x$, then $Q\Downarrow_{\mathcal N} x$.
\end{enumerate}
$P$ is ${\mathcal N}$-barbed bisimilar to $Q$, written
$P \wbbisim_{\mathcal N} Q$, if $P \rel{S}_{\mathcal N} Q$ for some ${\mathcal N}$-barbed bisimulation ${\mathcal S}_{\mathcal N}$.
\end{definition}

$\mathcal{R} \subseteq \pi \times \pi$

$P \mathcal{R} Q => \forall P'. P \red P' \Rightarrow \exists Q'. Q \red Q', P' \mathcal{R} Q'$

$P \vdash x \Rightarrow Q \vdash x$

\begin{mathpar}
  \inferrule*[lab=Out-barb]{x \nameeq y}{{y}!\langle{Q}\rangle \vdash x}
  \and
  \inferrule*[lab=Par-barb]{\mbox{$P\vdash x$ or $Q\vdash x$}}{\binpar{P}{Q} \vdash x}
\end{mathpar}

\subsubsection{Contexts}

One of the principle advantages of computational calculi like the
$\pi$-calculus is a well-defined notion of context,
contextual-equivalence and a correlation between
contextual-equivalence and notions of bisimulation. The notion of
context allows the decomposition of a process into (sub-)process and
its syntactic environment, its context. Thus, a context may be
thought of as a process with a ``hole'' (written $\Box$) in it. The
application of a context $M$ to a process $P$, written $M[P]$, is
tantamount to filling the hole in $M$ with $P$. In this paper we do
not need the full weight of this theory, but do make use of the notion
of context in the proof the main theorem. 

\begin{mathpar}
  \inferrule* [lab=summation] {} {{M_{M},M_{N}} \bc \Box \;|\; x.M_{A} \;|\; M_{M}+M_{N}}
  \and
  \inferrule* [lab=agent] {} {{M_{A}} \bc (\vec{x})M_{P} \;| \; \clift{P_0,\ldots,M_{P},\ldots,P_N}}
  \and \\
  \inferrule* [lab=process] {} {{M_{P}} \bc M_{N} \;| \;P|M_{P} }
\end{mathpar} 

\begin{mathpar}
  \inferrule* [lab=sychronization] {} {M_{N} \bc \Box \;|\; x?M_{F} \;|\; x!M_{C}}
  \and
  \inferrule* [lab=abstraction] {} {{M_{F}} \bc (x)M_{P} }
  \and
  \inferrule* [lab=concretion] {} {{M_{C}} \bc \langle M_{P} \rangle }
  \and \\
  \inferrule* [lab=process] {} {{M_{P}} \bc M_{N} \;| \;P|M_{P} }
\end{mathpar}

\begin{definition}[contextual application] Given a context $M$, and
  process $P$, we define the \emph{contextual application}, $M[P] :=
  M\{P/\Box\}$. That is, the contextual application of M to P is the
  substitution of $P$ for $\Box$ in $M$.
\end{definition}

$\meaningof{-} : L \to \mathcal{P}(\pi)$

\begin{mathpar}
  \inferrule* [lab=collection] {} {\meaningof{true} = \pi, \and \meaningof{~E} = \pi \setminus \meaningof{E}, \and \meaningof{E_{1} \& E_{2}} = \meaningof{E_{1}} \cap \meaningof{E_{2}}}
\end{mathpar}

\begin{mathpar}
  \inferrule* [lab=structure] {} {\meaningof{0} = \{ P \in \pi | P \equiv 0 \}, \and \\ \meaningof{E_1 | E_2} = \{ P \in \pi | P \equiv P_{1} | P_{2}, P_{1} \in \meaningof{E_{1}}, P_{2} \in \meaningof{E_2}\} }
\end{mathpar}

\begin{mathpar}
 \inferrule* [lab=behavior] {} {\meaningof{\langle a?b \rangle E} = \{ P \in \pi | P \equiv Q | u?(y)P', \\ \and \\\\ \and \\ \;\;\; u \in \meaningof{a}, \forall z.P'\{z/y\} \in \meaningof{E\{z/b\}}\}, \and \\ \meaningof{a!E} = \{ P \in \pi | P \equiv Q | x!\langle P' \rangle, x \in \meaningof{a} P' \in \meaningof{E}\} }
\end{mathpar}

\begin{mathpar}
 \inferrule* [lab=nominal] {} {\meaningof{\quotep{E}} = \{ \quotep{P} \in \quotep{\pi} | P \in \meaningof{E} \}, \and \meaningof{\quotep{P}} = \{ \quotep{Q} \in \quotep{\pi} | P \equiv Q \} \and \\ \meaningof{@\quotep{E}} = \{ P \in \pi | P \equiv @x, x \in \meaningof{E} \}}
\end{mathpar}

\begin{eqnarray*}
  \\
  \meaningof{-} : TS \to ST
\end{eqnarray*}

\begin{eqnarray*}
  \\
  L : TS \to ST
\end{eqnarray*}

\begin{eqnarray*}
  \\
  P \models E \iff P \in \meaningof{E}
\end{eqnarray*}

\begin{eqnarray*}
  P \approx_{L} Q \iff \forall E \in L. P \models E \iff Q \models E
\end{eqnarray*}

\begin{eqnarray*}
  P \approx_{K} Q
\end{eqnarray*}

\begin{eqnarray*}
  P \approx Q
\end{eqnarray*}

$\approx_{K} = \approx = \approx_{L}$

\subsubsection{Contextual duality}

Note that contexts extend the quotation operation to a family of
operations from processes to names. Given a context, $M$, we can
define a \emph{nominal context}, $\quotep{M}$ by $\quotep{M}[P] :=
\quotep{M[P]}$. To foreshadow what is to come we observe that these
operations enjoy a duality with processes very much like the duality
between vectors and maps from vectors to scalars.

Further, because the calculus is essentially higher-order, we have a
correspondence between contexts and processes. More specifically,
given a name $x$ and a context $M$ we can construct $M^{*}_{x}$ such
that 

\begin{mathpar}
  M^{*}_{x} | \lift{x}{P} \red M[P]
\end{mathpar}

namely,

\begin{mathpar}
  M^{*}_{x} := x?(u).M[\dropn{u}]
\end{mathpar}

The dependence of $M^{*}_{x}$ on a name makes it an abstraction, 

\begin{mathpar}
  M^{*} := (x)x?(u).M[\dropn{u}]
\end{mathpar}

\subsection{Additional notation}

It will sometimes be convenient to denote the process a name
quotes. We already have the notation $x = \quotep{P}$, but it will be
convenient to introduce an alternate notation, $\procn{x}$, when we
want to emphasize the connection to the use of the name. Note that, by
virtue of name equivalence, $\quotep{\procn{x}} \nameeq x$; so, the
notation is consistent with previous definitions.

Further, because names have structure it is possible to effect
substitutions on the basis of that structure. This means we need to
upgrade our notation for substitutions, which we accomplish by
adapting comprehension notation. Thus,

\begin{mathpar}
  P\{ y / x : x \in S \}
\end{mathpar}

is interpreted to mean the process derived from P by replacing (in a
capture-avoiding manner) each occurrence of $x$ in $S$ by $y$. For example,

\begin{mathpar}
  P\{ \quotep{\procn{x}|\procn{x}} / x : x \in \freenames{P} \}
\end{mathpar}

will replace each (occurrence) of a free name $x$ in $P$ by
$\quotep{\procn{x}|\procn{x}}$.

Also, we will avail ourselves of the notation $x^{L}$ and $x^{R}$ to
denote injections of a name into disjoint copies of the name
space. There are numerous ways to accomplish this. One example can be
found in \cite{MeredithR05}. This notation overloads to vectors of
names: $\vec{x}^{\pi} := (x_{i}^{\pi} \; : \; 0 \leq i < |\vec{x}| )$ where $\pi \in \{L,R\}$.

We also use $P^{\Box} := P|\Box$.

In \cite{MeredithR05} an interpretation of the new operator is
given. It turns out that there are several possible interpretations
all enjoying the requisite algebraic properties of the operator (see
\cite{milner91polyadicpi}). We will therefore make liberal use of
$(\nu\; \vec{x})P$.

% subsection the_syntax_and_semantics_of_the_notation_system (end)   

\input{qm2pi.qmops} 

\input{qm2pi.sterngerlach} 

\input{qm2pi.metric} 

% section concurrent_process_calculi (end)

%\input{qm2pi.proofsketch}

% section proof sketch (end)

%\input{qm2pi.slviaknots} 

% section spatial logic via knots (end)

\input{qm2pi.conclusion}

% section conclusion (end)

%\input{qm2pi.dtcodes} 

% section wiring algorithm (end)

\input{qm2pi.ack} 

% section acknowledgments (end)

\newpage


\bibliographystyle{plain}   
\bibliography{../../biblios/main.bib}

\input{qm2pi.rhodetails}

\end{document}

 

%\documentclass[12pt]{llncs}
%\documentclass{jktr}

\usepackage[pdftex]{hyperref}                   
\usepackage {listings}
\usepackage {mathpartir}
\usepackage{bcprules}
%\usepackage{listings}
                       
\usepackage{graphicx} 
%\usepackage[margins=2.5cm,nohead,nofoot]{geometry}
%\usepackage{geometry}
\usepackage{amsfonts}
\usepackage{amstext}
\usepackage{latexsym}
\usepackage{amssymb}
\usepackage{color}


%\include{myPreamble}
\include{qm2pi.local} 

%\ifpdf
%\usepackage[pdftex]{graphicx}
%\else
%\usepackage{graphicx}
%\fi

 % \ifpdf
%  \usepackage{pdfsync}
%  \if


%\title{Brief Article}
%\author{David F. Snyder}
%\author{L.G. Meredith}

%\address{Dept. of Math., Texas State University--San Marcos, San Marcos, TX 78666}
       
\pagestyle{empty}


\begin{document}

\lstset{language=[Objective]Caml,frame=shadowbox}

\input{qm2pi.front}

% section front matter (end)

\input{qm2pi.intro} 
 
% section introduction (end)

% \input{qm2pi.knotations} 

% section notation (end)

\input{qm2pi.process.calculi} 

% section concurrent_process_calculi_and_spatial_logics_ (end)
    
%\input{qm2pi.knots2pi} 

%\input{qm2pi.trefoil} 

%\input{qm2pi.mainthm} 

% subsection basic_interpretation (end)

%\input{qm2pi.rho.presentation} 
\subsection{The syntax and semantics of the notation system}\label{sub:the_syntax_and_semantics_of_the_notation_system} % (fold)

We now summarize a technical presentation of the calculus that
embodies our theory of dynamics. The typical presentation of such a
calculus follows the style of giving generators and relations on
them. The grammar, below, describing term constructors, freely
generates the set of processes, $\Proc$. This set is then quotiented
by a relation known as structural congruence and it is over this set
that the notion of dynamics is expressed. This presentation is
essentially that of \cite{MeredithR05} with the addition of
polyadicity and summation. For readability we have relegated some of
the technical subtleties to an appendix.

\subsubsection{Process grammar}\label{subsub:process_grammar}

\begin{mathpar}
  \inferrule* [lab=synchronization] {} {{M} \bc \pzero \;|\; x?F \;|\; x!C }
  \and
  \inferrule* [lab=abstraction] {} {{F} \bc (x)P}
  \and
  \inferrule* [lab=concretion] {} {{C} \bc \langle Q \rangle}
  \and
  \inferrule* [lab=process] {} {{P,Q} \bc M \;| \;P|Q \;|\; @{x}}
  \and
  \inferrule* [lab=name] {} {{x} \bc \quotep{P}}
\end{mathpar} 

Note that $\vec{x}$ (resp. $\vec{P}$) denotes a vector of names
(resp. processes) of length $|\vec{x}|$ (resp. $|\vec{P}|$). We adopt
the following useful abbreviations.

\begin{mathpar}
   x?(\vec{y}).P := x.(\vec{y})P \and  x\clift{\vec{P}} := x.\clift{\vec{P}}
   \and x!(y) := \lift{x}{\dropn{y}}
   \and \Pi_{i=0}^{n-1}P_i := P_0 | \ldots | P_{n-1}
\end{mathpar}

\subsubsection{Structural congruence}

\paragraph{Free and bound names and alpha-equivalence.} At the
core of structural equivalence is alpha-equivalence which identifies
process that are the same up to a change of variable. Formally, we
recognize the distinction between free and bound names. The free names
of a process, $\freenames{P}$, may be calculated recursively as
follows:

\begin{mathpar}
\freenames{\pzero} := \emptyset
  \and \\
  \freenames{x?(y).P} := \{ x \} \cup (\freenames{P} \setminus \{ y \})
  \and 
  \freenames{x!\langle P \rangle} := \{ x \} \cup \{ P \} 
  \and \\
  \freenames{P|Q} := \freenames{P} \cup \freenames{Q}
  \and \\
  \freenames{@{x}} := \{ x \}
\end{mathpar}

$\pi$
$\quotep{\pi}$

$\freenames{-} : \pi \to \mathcal{P}(\quotep{\pi})$

\begin{eqnarray*}
  \freenames{\pzero} & := & \emptyset \\
  \freenames{x?(y).P} & := & \{ x \} \cup (\freenames{P} \setminus \{ y \}) \\
  \freenames{x!\langle P \rangle} & := & \{ x \} \cup \{ P \} \\
  \freenames{P|Q} & := & \freenames{P} \cup \freenames{Q} \\
  \freenames{\dropn{x}} & := & \{ x \}
\end{eqnarray*}

The bound names of a process, $\boundnames{P}$, are those names occurring in $P$
that are not free. For example, in $x?(y).0$, the name $x$ is free, while $y$ is bound.

\begin{mathpar}
  \inferrule* [lab=monoidal-laws] {} { P|Q \equiv Q|P \and P|0 \equiv P \and P|(Q|R) \equiv (P|Q)|R }
\end{mathpar}

\begin{mathpar}
  \inferrule* [lab=alpha-equivalence] {} { (x)P \equiv (y)P\{y/x\} \and y \not\in \freenames{P} }
\end{mathpar}

\begin{definition}
Then two processes, $P,Q$, are alpha-equivalent if $P = Q\{\vec{y}/\vec{x}\}$ for
some $\vec{x} \in \boundnames{Q},\vec{y} \in \boundnames{P}$, where $Q\{\vec{y}/\vec{x}\}$
denotes the capture-avoiding substitution of $\vec{y}$ for $\vec{x}$ in $Q$.
\end{definition}

\begin{definition}
  The {\em structural congruence} \cite{SangiorgiWalker} , $\equiv$,
  between processes is the least congruence containing
  alpha-equivalence, satisfying the abelian monoid laws
  (associativity, commutativity and $\pzero$ as identity) for parallel
  composition $|$ and for summation $+$.
\end{definition}

\subsection{Name equivalence}

We take name equivalence, written $\nameeq$, to be the smallest
equivalence relation generated by the following rules.

\begin{mathpar}
\inferrule*[lab=Quote-drop]
{ }
{ \quotep{@{x}} \nameeq x }

\inferrule*[lab=Struct-equiv]
{ P \scong Q }
{ \quotep{P} \nameeq \quotep{Q} }
\end{mathpar}

The astute reader will have noticed that the mutual recursion of names
and processes imposes a mutual recursion on alpha-equivalence and
structural equivalence via name-equivalence. Fortunately, all of this
works out pleasantly and we may calculate in the natural way, free of
concern. The reader interested in the details is referred to the
appendix \ref{appendix:rho_details}.

\subsection{Substitution}

We use $\Proc$ for the set of processes, $\QProc$ for the set of
names, and $\id{\{}\vec{y} / \vec{x} \id{\}}$ to denote partial maps,
$s : \QProc \rightarrow \QProc$. A map, $s$ lifts, uniquely, to a map
on process terms, $\widehat{s} : \Proc \rightarrow \Proc$ by the
following equations.

\begin{mathpar}
  (0) \psubstp{Q}{P} := 0 \\
  (R \juxtap S) \psubstp{Q}{P}
  :=    
  (R)\psubstp{Q}{P} \juxtap (S) \psubstp{Q}{P} \\
  (x?(y).R) \psubstp{Q}{P}    
  :=    
  (x)\substp{Q}{P} (z)\concat( (R \psubstn{z}{y}) \psubstp{Q}{P} ) \\
  (\lift{x}{R}) \psubstp{Q}{P}  
  :=
  \lift{(x)\substp{Q}{P}}{ R \psubstp{Q}{P} } \\
%   (\dropn{x})  \psubstp{Q}{P}       
%   := 
%   \left\{ 
%     \begin{array}{ccc} 
%       \dropn{\quotep{Q}} & & x \nameeq \quotep{P} \\
%       \dropn{x} & & otherwise \\
%     \end{array}
%   \right. 
  (\dropn{x})  \psubstp{Q}{P}       
  := 
  \left\{ 
    \begin{array}{ccc} 
      Q & & x \nameeq \quotep{P} \\
      \dropn{x} & & otherwise \\
    \end{array}
  \right.
\end{mathpar}
 

where

\begin{eqnarray}
  (x)\id{\{} \lpquote Q \rpquote / \lpquote P \rpquote \id{\}}            = 
  \left\{ 
    \begin{array}{ccc}
      \lpquote Q \rpquote & & x \nameeq \lpquote P \rpquote \\
      x & & otherwise \\
    \end{array}
  \right. \nonumber
\end{eqnarray}

and $z$ is chosen distinct from $\quotep{P}$, $\quotep{Q}$, the free
names in $Q$, and all the names in $R$. Our $\alpha$-equivalence will
be built in the standard way from this substitution.

\begin{remark}\label{rem:no_self_referential_names}
  One consequence of these definitions is that $\forall P. \quotep{P}
  \not\in \freenames{P}$.
\end{remark}

\subsection{ Dynamic quote: an example }

Anticipating something of what's to come, consider applying the
substitution, $\widehat{\id{\{}u / z \id{\}}}$, to the following pair
of processes, $\lift{w}{y!(z)}$ and $w[ \lpquote y!(z) \rpquote ]$.

\begin{eqnarray}
	\lift{w}{y!(z)}\widehat{\id{\{}u / z \id{\}}}
		& = &
		\lift{w}{y!(u)} \nonumber\\
	w[ \lpquote y!(z) \rpquote ] \widehat{ \id{\{}u / z \id{\}} }
		& = &
		w[ \lpquote y!(z) \rpquote ] \nonumber
\end{eqnarray}

Because the body of the process between quotes is impervious to
substitution, we get radically different answers. In fact, by
examining the first process in an input context,
e.g. $x?(z).\lift{w}{y!(z)}$, we see that the process under the lift
operator may be shaped by prefixed inputs binding a name inside it. In
this sense, the lift operator will be seen as a way to dynamically
construct processes before reifying them as names.

Finally equipped with these standard features we can present the
dynamics of the calculus.

\subsubsection{Operational semantics} 

Finally, we introduce the computational dynamics. What marks these
algebras as distinct from other more traditionally studied algebraic
structures, e.g. vector spaces or polynomial rings, is the manner in
which dynamics is captured. In traditional structures, dynamics is typically
expressed through morphisms between such structures, as in linear maps
between vector spaces or morphisms between rings. In algebras
associated with the semantics of computation, the dynamics is
expressed as part of the algebraic structure itself, through a
reduction reduction relation typically denoted by $\red$. Below, we
give a recursive presentation of this relation for the calculus used
in the encoding.

$\red \subseteq \pi \times \pi$
$\red : \pi \to \mathcal{P}(\pi)$

\begin{mathpar}
  \inferrule* [lab=Comm] { \textsf{match}( x_{src}, x_{trgt} ) } { x_{trgt}?(y)P \; | \; x_{src}!\langle {Q} \rangle \red P\{\quotep{Q}/y}\} }
  \and \\
  \inferrule* [lab=Par] {{P} \red {P}'} {{{P} | {Q}} \red {{P}' | {Q}}}
  \and
  \inferrule* [lab=Equiv]{{{P} \scong {P}'} \andalso {{P}' \red {Q}'} \andalso {{Q}' \scong {Q}}}{{P} \red {Q}}
\end{mathpar}

\begin{eqnarray*}
  match_{\equiv} (\quotep{P},\quotep{Q}) & := & P \equiv Q \\
  match_{\dagger}(\quotep{P},\quotep{Q}) & := & \forall R. P|Q \red^{*} R => R \red^{*} 0 \\
  match_{K}(\quotep{P},\quotep{Q}) & := & K \mbox{ for some context } K
\end{eqnarray*}

$u?(x)P | u!\langle Q \rangle \red P\{\quotep{Q}/x\}$

%We write $\wred$ for $\red^*$, and $P\red$ if $\exists Q $ such that $ P \red Q$.
We write $P\red$ if $\exists Q $ such that $ P \red Q$ and $P\not\red$, otherwise.

\section{Replication}

As mentioned before, it is known that replication (and hence
recursion) can be implemented in a higher-order process algebra
\cite{SangiorgiWalker}. As our first example of calculation with the
machinery thus far presented we give the construction explicitly in
the {\rhoc}.

\begin{eqnarray}
	D_{x} & := & \prefix{x}{y}{(\binpar{\outputp{x}{y}}{@{y}})} \nonumber\\
	\bangp_{x}{P} & := & \binpar{{x}!\langle{\binpar{D_{x}}{P}}\rangle}{D_{x}} \nonumber
\end{eqnarray}

\begin{eqnarray}
	\bangp_{x}{P} & & \nonumber\\
	=
	& {x}!\langle{(\prefix{x}{y}{(\outputp{x}{y} | @{y})) | P}}\rangle 
	      | \prefix{x}{y}{(\outputp{x}{y} | @{y})} & \nonumber\\
	\red
	& (\outputp{x}{y} | @{y})\substn{\quotep{(\prefix{x}{y}{(@{y} | \outputp{x}{y})) | P}}}{y} & \nonumber\\
	=
	& \outputp{x}{\quotep{(\prefix{x}{y}{(\outputp{x}{y} | @{y})) | P}}}
	  | {(\prefix{x}{y}{(\outputp{x}{y} | @{y})) | P}} & \nonumber\\
	\red
	& \ldots & \nonumber\\
	\red^*
	& P | P | \ldots & \nonumber
\end{eqnarray}

Of course, this encoding, as an implementation, runs away, unfolding
$\bangp{P}$ eagerly. A lazier and more implementable replication
operator, restricted to input-guarded processes, may be obtained as follows.

\begin{eqnarray}
\bangp{\prefix{u}{v}{P}} 
	:= 
	\binpar{\lift{x}{\prefix{u}{v}{(\binpar{D(x)}{P})}}}{D(x)} \nonumber
\end{eqnarray}

\begin{remark}
  Note that the lazier definition still does not deal with summation
  or mixed summation (i.e. sums over input and output). The reader is
  invited to construct definitions of replication that deal with these
  features. 

  Further, the definitions are parameterized in a name, $x$. Can you,
  gentle reader, make a definition that eliminates this parameter and
  guarantees no accidental interaction between the replication
  machinery and the process being replicated -- i.e. no accidental
  sharing of names used by the process to get its work done and the
  name(s) used by the replication to effect copying. This latter
  revision of the definition of replication is crucial to obtaining
  the expected identity $!!P \sim !P$.
\end{remark}

\begin{remark}\label{rem:paradoxical_combinator}
  The reader familiar with the lambda calculus will have noticed the
  similarity between $D$ and the paradoxical combinator.

  [Ed. note: the existence of this seems to suggest we have to be more
  restrictive on the set of processes and names we admit if we are to
  support no-cloning.]
\end{remark}

\subsubsection{Bisimulation}

The computational dynamics gives rise to another kind of equivalence,
the equivalence of computational behavior. As previously mentioned
this is typically captured \emph{via} some form of bisimulation.

% The notion we use in this paper is weak barbed bisimulation
% \cite{milner91polyadicpi}.

The notion we use in this paper is derived from weak barbed
bisimulation \cite{milner91polyadicpi}. 

\begin{definition}
An \emph{observation relation}, $\downarrow_{\mathcal N}$, over a set
of names, $\mathcal N$, is the smallest relation satisfying the rules
below.

\infrule[Out-barb]{y \in {\mathcal N}, \; x \nameeq y}
		  {\outputp{x}{v} \downarrow_{\mathcal N} x}
\infrule[Par-barb]{\mbox{$P\downarrow_{\mathcal N} x$ or $Q\downarrow_{\mathcal N} x$}}
		  {\binpar{P}{Q} \downarrow_{\mathcal N} x}

We write $P \Downarrow_{\mathcal N} x$ if there is $Q$ such that 
$P \wred Q$ and $Q \downarrow_{\mathcal N} x$.
\end{definition}

\begin{definition}
%\label{def.bbisim}
An  ${\mathcal N}$-\emph{barbed bisimulation} over a set of names, ${\mathcal N}$, is a symmetric binary relation 
${\mathcal S}_{\mathcal N}$ between agents such that $P\rel{S}_{\mathcal N}Q$ implies:
\begin{enumerate}
\item If $P \red P'$ then $Q \wred Q'$ and $P'\rel{S}_{\mathcal N} Q'$.
\item If $P\downarrow_{\mathcal N} x$, then $Q\Downarrow_{\mathcal N} x$.
\end{enumerate}
$P$ is ${\mathcal N}$-barbed bisimilar to $Q$, written
$P \wbbisim_{\mathcal N} Q$, if $P \rel{S}_{\mathcal N} Q$ for some ${\mathcal N}$-barbed bisimulation ${\mathcal S}_{\mathcal N}$.
\end{definition}

$\mathcal{R} \subseteq \pi \times \pi$

$P \mathcal{R} Q => \forall P'. P \red P' \Rightarrow \exists Q'. Q \red Q', P' \mathcal{R} Q'$

$P \vdash x \Rightarrow Q \vdash x$

\begin{mathpar}
  \inferrule*[lab=Out-barb]{x \nameeq y}{{y}!\langle{Q}\rangle \vdash x}
  \and
  \inferrule*[lab=Par-barb]{\mbox{$P\vdash x$ or $Q\vdash x$}}{\binpar{P}{Q} \vdash x}
\end{mathpar}

\subsubsection{Contexts}

One of the principle advantages of computational calculi like the
$\pi$-calculus is a well-defined notion of context,
contextual-equivalence and a correlation between
contextual-equivalence and notions of bisimulation. The notion of
context allows the decomposition of a process into (sub-)process and
its syntactic environment, its context. Thus, a context may be
thought of as a process with a ``hole'' (written $\Box$) in it. The
application of a context $M$ to a process $P$, written $M[P]$, is
tantamount to filling the hole in $M$ with $P$. In this paper we do
not need the full weight of this theory, but do make use of the notion
of context in the proof the main theorem. 

\begin{mathpar}
  \inferrule* [lab=summation] {} {{M_{M},M_{N}} \bc \Box \;|\; x.M_{A} \;|\; M_{M}+M_{N}}
  \and
  \inferrule* [lab=agent] {} {{M_{A}} \bc (\vec{x})M_{P} \;| \; \clift{P_0,\ldots,M_{P},\ldots,P_N}}
  \and \\
  \inferrule* [lab=process] {} {{M_{P}} \bc M_{N} \;| \;P|M_{P} }
\end{mathpar} 

\begin{mathpar}
  \inferrule* [lab=sychronization] {} {M_{N} \bc \Box \;|\; x?M_{F} \;|\; x!M_{C}}
  \and
  \inferrule* [lab=abstraction] {} {{M_{F}} \bc (x)M_{P} }
  \and
  \inferrule* [lab=concretion] {} {{M_{C}} \bc \langle M_{P} \rangle }
  \and \\
  \inferrule* [lab=process] {} {{M_{P}} \bc M_{N} \;| \;P|M_{P} }
\end{mathpar}

\begin{definition}[contextual application] Given a context $M$, and
  process $P$, we define the \emph{contextual application}, $M[P] :=
  M\{P/\Box\}$. That is, the contextual application of M to P is the
  substitution of $P$ for $\Box$ in $M$.
\end{definition}

$\meaningof{-} : L \to \mathcal{P}(\pi)$

\begin{mathpar}
  \inferrule* [lab=collection] {} {\meaningof{true} = \pi, \and \meaningof{~E} = \pi \setminus \meaningof{E}, \and \meaningof{E_{1} \& E_{2}} = \meaningof{E_{1}} \cap \meaningof{E_{2}}}
\end{mathpar}

\begin{mathpar}
  \inferrule* [lab=structure] {} {\meaningof{0} = \{ P \in \pi | P \equiv 0 \}, \and \\ \meaningof{E_1 | E_2} = \{ P \in \pi | P \equiv P_{1} | P_{2}, P_{1} \in \meaningof{E_{1}}, P_{2} \in \meaningof{E_2}\} }
\end{mathpar}

\begin{mathpar}
 \inferrule* [lab=behavior] {} {\meaningof{\langle a?b \rangle E} = \{ P \in \pi | P \equiv Q | u?(y)P', \\ \and \\\\ \and \\ \;\;\; u \in \meaningof{a}, \forall z.P'\{z/y\} \in \meaningof{E\{z/b\}}\}, \and \\ \meaningof{a!E} = \{ P \in \pi | P \equiv Q | x!\langle P' \rangle, x \in \meaningof{a} P' \in \meaningof{E}\} }
\end{mathpar}

\begin{mathpar}
 \inferrule* [lab=nominal] {} {\meaningof{\quotep{E}} = \{ \quotep{P} \in \quotep{\pi} | P \in \meaningof{E} \}, \and \meaningof{\quotep{P}} = \{ \quotep{Q} \in \quotep{\pi} | P \equiv Q \} \and \\ \meaningof{@\quotep{E}} = \{ P \in \pi | P \equiv @x, x \in \meaningof{E} \}}
\end{mathpar}

\begin{eqnarray*}
  \\
  \meaningof{-} : TS \to ST
\end{eqnarray*}

\begin{eqnarray*}
  \\
  L : TS \to ST
\end{eqnarray*}

\begin{eqnarray*}
  \\
  P \models E \iff P \in \meaningof{E}
\end{eqnarray*}

\begin{eqnarray*}
  P \approx_{L} Q \iff \forall E \in L. P \models E \iff Q \models E
\end{eqnarray*}

\begin{eqnarray*}
  P \approx_{K} Q
\end{eqnarray*}

\begin{eqnarray*}
  P \approx Q
\end{eqnarray*}

$\approx_{K} = \approx = \approx_{L}$

\subsubsection{Contextual duality}

Note that contexts extend the quotation operation to a family of
operations from processes to names. Given a context, $M$, we can
define a \emph{nominal context}, $\quotep{M}$ by $\quotep{M}[P] :=
\quotep{M[P]}$. To foreshadow what is to come we observe that these
operations enjoy a duality with processes very much like the duality
between vectors and maps from vectors to scalars.

Further, because the calculus is essentially higher-order, we have a
correspondence between contexts and processes. More specifically,
given a name $x$ and a context $M$ we can construct $M^{*}_{x}$ such
that 

\begin{mathpar}
  M^{*}_{x} | \lift{x}{P} \red M[P]
\end{mathpar}

namely,

\begin{mathpar}
  M^{*}_{x} := x?(u).M[\dropn{u}]
\end{mathpar}

The dependence of $M^{*}_{x}$ on a name makes it an abstraction, 

\begin{mathpar}
  M^{*} := (x)x?(u).M[\dropn{u}]
\end{mathpar}

\subsection{Additional notation}

It will sometimes be convenient to denote the process a name
quotes. We already have the notation $x = \quotep{P}$, but it will be
convenient to introduce an alternate notation, $\procn{x}$, when we
want to emphasize the connection to the use of the name. Note that, by
virtue of name equivalence, $\quotep{\procn{x}} \nameeq x$; so, the
notation is consistent with previous definitions.

Further, because names have structure it is possible to effect
substitutions on the basis of that structure. This means we need to
upgrade our notation for substitutions, which we accomplish by
adapting comprehension notation. Thus,

\begin{mathpar}
  P\{ y / x : x \in S \}
\end{mathpar}

is interpreted to mean the process derived from P by replacing (in a
capture-avoiding manner) each occurrence of $x$ in $S$ by $y$. For example,

\begin{mathpar}
  P\{ \quotep{\procn{x}|\procn{x}} / x : x \in \freenames{P} \}
\end{mathpar}

will replace each (occurrence) of a free name $x$ in $P$ by
$\quotep{\procn{x}|\procn{x}}$.

Also, we will avail ourselves of the notation $x^{L}$ and $x^{R}$ to
denote injections of a name into disjoint copies of the name
space. There are numerous ways to accomplish this. One example can be
found in \cite{MeredithR05}. This notation overloads to vectors of
names: $\vec{x}^{\pi} := (x_{i}^{\pi} \; : \; 0 \leq i < |\vec{x}| )$ where $\pi \in \{L,R\}$.

We also use $P^{\Box} := P|\Box$.

In \cite{MeredithR05} an interpretation of the new operator is
given. It turns out that there are several possible interpretations
all enjoying the requisite algebraic properties of the operator (see
\cite{milner91polyadicpi}). We will therefore make liberal use of
$(\nu\; \vec{x})P$.

% subsection the_syntax_and_semantics_of_the_notation_system (end)   

\input{qm2pi.qmops} 

\input{qm2pi.sterngerlach} 

\input{qm2pi.metric} 

% section concurrent_process_calculi (end)

%\input{qm2pi.proofsketch}

% section proof sketch (end)

%\input{qm2pi.slviaknots} 

% section spatial logic via knots (end)

\input{qm2pi.conclusion}

% section conclusion (end)

%\input{qm2pi.dtcodes} 

% section wiring algorithm (end)

\input{qm2pi.ack} 

% section acknowledgments (end)

\newpage


\bibliographystyle{plain}   
\bibliography{../../biblios/main.bib}

\input{qm2pi.rhodetails}

\end{document}

 

%\documentclass[12pt]{llncs}
%\documentclass{jktr}

\usepackage[pdftex]{hyperref}                   
\usepackage {listings}
\usepackage {mathpartir}
\usepackage{bcprules}
%\usepackage{listings}
                       
\usepackage{graphicx} 
%\usepackage[margins=2.5cm,nohead,nofoot]{geometry}
%\usepackage{geometry}
\usepackage{amsfonts}
\usepackage{amstext}
\usepackage{latexsym}
\usepackage{amssymb}
\usepackage{color}


%\include{myPreamble}
\include{qm2pi.local} 

%\ifpdf
%\usepackage[pdftex]{graphicx}
%\else
%\usepackage{graphicx}
%\fi

 % \ifpdf
%  \usepackage{pdfsync}
%  \if


%\title{Brief Article}
%\author{David F. Snyder}
%\author{L.G. Meredith}

%\address{Dept. of Math., Texas State University--San Marcos, San Marcos, TX 78666}
       
\pagestyle{empty}


\begin{document}

\lstset{language=[Objective]Caml,frame=shadowbox}

\input{qm2pi.front}

% section front matter (end)

\input{qm2pi.intro} 
 
% section introduction (end)

% \input{qm2pi.knotations} 

% section notation (end)

\input{qm2pi.process.calculi} 

% section concurrent_process_calculi_and_spatial_logics_ (end)
    
%\input{qm2pi.knots2pi} 

%\input{qm2pi.trefoil} 

%\input{qm2pi.mainthm} 

% subsection basic_interpretation (end)

%\input{qm2pi.rho.presentation} 
\subsection{The syntax and semantics of the notation system}\label{sub:the_syntax_and_semantics_of_the_notation_system} % (fold)

We now summarize a technical presentation of the calculus that
embodies our theory of dynamics. The typical presentation of such a
calculus follows the style of giving generators and relations on
them. The grammar, below, describing term constructors, freely
generates the set of processes, $\Proc$. This set is then quotiented
by a relation known as structural congruence and it is over this set
that the notion of dynamics is expressed. This presentation is
essentially that of \cite{MeredithR05} with the addition of
polyadicity and summation. For readability we have relegated some of
the technical subtleties to an appendix.

\subsubsection{Process grammar}\label{subsub:process_grammar}

\begin{mathpar}
  \inferrule* [lab=synchronization] {} {{M} \bc \pzero \;|\; x?F \;|\; x!C }
  \and
  \inferrule* [lab=abstraction] {} {{F} \bc (x)P}
  \and
  \inferrule* [lab=concretion] {} {{C} \bc \langle Q \rangle}
  \and
  \inferrule* [lab=process] {} {{P,Q} \bc M \;| \;P|Q \;|\; @{x}}
  \and
  \inferrule* [lab=name] {} {{x} \bc \quotep{P}}
\end{mathpar} 

Note that $\vec{x}$ (resp. $\vec{P}$) denotes a vector of names
(resp. processes) of length $|\vec{x}|$ (resp. $|\vec{P}|$). We adopt
the following useful abbreviations.

\begin{mathpar}
   x?(\vec{y}).P := x.(\vec{y})P \and  x\clift{\vec{P}} := x.\clift{\vec{P}}
   \and x!(y) := \lift{x}{\dropn{y}}
   \and \Pi_{i=0}^{n-1}P_i := P_0 | \ldots | P_{n-1}
\end{mathpar}

\subsubsection{Structural congruence}

\paragraph{Free and bound names and alpha-equivalence.} At the
core of structural equivalence is alpha-equivalence which identifies
process that are the same up to a change of variable. Formally, we
recognize the distinction between free and bound names. The free names
of a process, $\freenames{P}$, may be calculated recursively as
follows:

\begin{mathpar}
\freenames{\pzero} := \emptyset
  \and \\
  \freenames{x?(y).P} := \{ x \} \cup (\freenames{P} \setminus \{ y \})
  \and 
  \freenames{x!\langle P \rangle} := \{ x \} \cup \{ P \} 
  \and \\
  \freenames{P|Q} := \freenames{P} \cup \freenames{Q}
  \and \\
  \freenames{@{x}} := \{ x \}
\end{mathpar}

$\pi$
$\quotep{\pi}$

$\freenames{-} : \pi \to \mathcal{P}(\quotep{\pi})$

\begin{eqnarray*}
  \freenames{\pzero} & := & \emptyset \\
  \freenames{x?(y).P} & := & \{ x \} \cup (\freenames{P} \setminus \{ y \}) \\
  \freenames{x!\langle P \rangle} & := & \{ x \} \cup \{ P \} \\
  \freenames{P|Q} & := & \freenames{P} \cup \freenames{Q} \\
  \freenames{\dropn{x}} & := & \{ x \}
\end{eqnarray*}

The bound names of a process, $\boundnames{P}$, are those names occurring in $P$
that are not free. For example, in $x?(y).0$, the name $x$ is free, while $y$ is bound.

\begin{mathpar}
  \inferrule* [lab=monoidal-laws] {} { P|Q \equiv Q|P \and P|0 \equiv P \and P|(Q|R) \equiv (P|Q)|R }
\end{mathpar}

\begin{mathpar}
  \inferrule* [lab=alpha-equivalence] {} { (x)P \equiv (y)P\{y/x\} \and y \not\in \freenames{P} }
\end{mathpar}

\begin{definition}
Then two processes, $P,Q$, are alpha-equivalent if $P = Q\{\vec{y}/\vec{x}\}$ for
some $\vec{x} \in \boundnames{Q},\vec{y} \in \boundnames{P}$, where $Q\{\vec{y}/\vec{x}\}$
denotes the capture-avoiding substitution of $\vec{y}$ for $\vec{x}$ in $Q$.
\end{definition}

\begin{definition}
  The {\em structural congruence} \cite{SangiorgiWalker} , $\equiv$,
  between processes is the least congruence containing
  alpha-equivalence, satisfying the abelian monoid laws
  (associativity, commutativity and $\pzero$ as identity) for parallel
  composition $|$ and for summation $+$.
\end{definition}

\subsection{Name equivalence}

We take name equivalence, written $\nameeq$, to be the smallest
equivalence relation generated by the following rules.

\begin{mathpar}
\inferrule*[lab=Quote-drop]
{ }
{ \quotep{@{x}} \nameeq x }

\inferrule*[lab=Struct-equiv]
{ P \scong Q }
{ \quotep{P} \nameeq \quotep{Q} }
\end{mathpar}

The astute reader will have noticed that the mutual recursion of names
and processes imposes a mutual recursion on alpha-equivalence and
structural equivalence via name-equivalence. Fortunately, all of this
works out pleasantly and we may calculate in the natural way, free of
concern. The reader interested in the details is referred to the
appendix \ref{appendix:rho_details}.

\subsection{Substitution}

We use $\Proc$ for the set of processes, $\QProc$ for the set of
names, and $\id{\{}\vec{y} / \vec{x} \id{\}}$ to denote partial maps,
$s : \QProc \rightarrow \QProc$. A map, $s$ lifts, uniquely, to a map
on process terms, $\widehat{s} : \Proc \rightarrow \Proc$ by the
following equations.

\begin{mathpar}
  (0) \psubstp{Q}{P} := 0 \\
  (R \juxtap S) \psubstp{Q}{P}
  :=    
  (R)\psubstp{Q}{P} \juxtap (S) \psubstp{Q}{P} \\
  (x?(y).R) \psubstp{Q}{P}    
  :=    
  (x)\substp{Q}{P} (z)\concat( (R \psubstn{z}{y}) \psubstp{Q}{P} ) \\
  (\lift{x}{R}) \psubstp{Q}{P}  
  :=
  \lift{(x)\substp{Q}{P}}{ R \psubstp{Q}{P} } \\
%   (\dropn{x})  \psubstp{Q}{P}       
%   := 
%   \left\{ 
%     \begin{array}{ccc} 
%       \dropn{\quotep{Q}} & & x \nameeq \quotep{P} \\
%       \dropn{x} & & otherwise \\
%     \end{array}
%   \right. 
  (\dropn{x})  \psubstp{Q}{P}       
  := 
  \left\{ 
    \begin{array}{ccc} 
      Q & & x \nameeq \quotep{P} \\
      \dropn{x} & & otherwise \\
    \end{array}
  \right.
\end{mathpar}
 

where

\begin{eqnarray}
  (x)\id{\{} \lpquote Q \rpquote / \lpquote P \rpquote \id{\}}            = 
  \left\{ 
    \begin{array}{ccc}
      \lpquote Q \rpquote & & x \nameeq \lpquote P \rpquote \\
      x & & otherwise \\
    \end{array}
  \right. \nonumber
\end{eqnarray}

and $z$ is chosen distinct from $\quotep{P}$, $\quotep{Q}$, the free
names in $Q$, and all the names in $R$. Our $\alpha$-equivalence will
be built in the standard way from this substitution.

\begin{remark}\label{rem:no_self_referential_names}
  One consequence of these definitions is that $\forall P. \quotep{P}
  \not\in \freenames{P}$.
\end{remark}

\subsection{ Dynamic quote: an example }

Anticipating something of what's to come, consider applying the
substitution, $\widehat{\id{\{}u / z \id{\}}}$, to the following pair
of processes, $\lift{w}{y!(z)}$ and $w[ \lpquote y!(z) \rpquote ]$.

\begin{eqnarray}
	\lift{w}{y!(z)}\widehat{\id{\{}u / z \id{\}}}
		& = &
		\lift{w}{y!(u)} \nonumber\\
	w[ \lpquote y!(z) \rpquote ] \widehat{ \id{\{}u / z \id{\}} }
		& = &
		w[ \lpquote y!(z) \rpquote ] \nonumber
\end{eqnarray}

Because the body of the process between quotes is impervious to
substitution, we get radically different answers. In fact, by
examining the first process in an input context,
e.g. $x?(z).\lift{w}{y!(z)}$, we see that the process under the lift
operator may be shaped by prefixed inputs binding a name inside it. In
this sense, the lift operator will be seen as a way to dynamically
construct processes before reifying them as names.

Finally equipped with these standard features we can present the
dynamics of the calculus.

\subsubsection{Operational semantics} 

Finally, we introduce the computational dynamics. What marks these
algebras as distinct from other more traditionally studied algebraic
structures, e.g. vector spaces or polynomial rings, is the manner in
which dynamics is captured. In traditional structures, dynamics is typically
expressed through morphisms between such structures, as in linear maps
between vector spaces or morphisms between rings. In algebras
associated with the semantics of computation, the dynamics is
expressed as part of the algebraic structure itself, through a
reduction reduction relation typically denoted by $\red$. Below, we
give a recursive presentation of this relation for the calculus used
in the encoding.

$\red \subseteq \pi \times \pi$
$\red : \pi \to \mathcal{P}(\pi)$

\begin{mathpar}
  \inferrule* [lab=Comm] { \textsf{match}( x_{src}, x_{trgt} ) } { x_{trgt}?(y)P \; | \; x_{src}!\langle {Q} \rangle \red P\{\quotep{Q}/y}\} }
  \and \\
  \inferrule* [lab=Par] {{P} \red {P}'} {{{P} | {Q}} \red {{P}' | {Q}}}
  \and
  \inferrule* [lab=Equiv]{{{P} \scong {P}'} \andalso {{P}' \red {Q}'} \andalso {{Q}' \scong {Q}}}{{P} \red {Q}}
\end{mathpar}

\begin{eqnarray*}
  match_{\equiv} (\quotep{P},\quotep{Q}) & := & P \equiv Q \\
  match_{\dagger}(\quotep{P},\quotep{Q}) & := & \forall R. P|Q \red^{*} R => R \red^{*} 0 \\
  match_{K}(\quotep{P},\quotep{Q}) & := & K \mbox{ for some context } K
\end{eqnarray*}

$u?(x)P | u!\langle Q \rangle \red P\{\quotep{Q}/x\}$

%We write $\wred$ for $\red^*$, and $P\red$ if $\exists Q $ such that $ P \red Q$.
We write $P\red$ if $\exists Q $ such that $ P \red Q$ and $P\not\red$, otherwise.

\section{Replication}

As mentioned before, it is known that replication (and hence
recursion) can be implemented in a higher-order process algebra
\cite{SangiorgiWalker}. As our first example of calculation with the
machinery thus far presented we give the construction explicitly in
the {\rhoc}.

\begin{eqnarray}
	D_{x} & := & \prefix{x}{y}{(\binpar{\outputp{x}{y}}{@{y}})} \nonumber\\
	\bangp_{x}{P} & := & \binpar{{x}!\langle{\binpar{D_{x}}{P}}\rangle}{D_{x}} \nonumber
\end{eqnarray}

\begin{eqnarray}
	\bangp_{x}{P} & & \nonumber\\
	=
	& {x}!\langle{(\prefix{x}{y}{(\outputp{x}{y} | @{y})) | P}}\rangle 
	      | \prefix{x}{y}{(\outputp{x}{y} | @{y})} & \nonumber\\
	\red
	& (\outputp{x}{y} | @{y})\substn{\quotep{(\prefix{x}{y}{(@{y} | \outputp{x}{y})) | P}}}{y} & \nonumber\\
	=
	& \outputp{x}{\quotep{(\prefix{x}{y}{(\outputp{x}{y} | @{y})) | P}}}
	  | {(\prefix{x}{y}{(\outputp{x}{y} | @{y})) | P}} & \nonumber\\
	\red
	& \ldots & \nonumber\\
	\red^*
	& P | P | \ldots & \nonumber
\end{eqnarray}

Of course, this encoding, as an implementation, runs away, unfolding
$\bangp{P}$ eagerly. A lazier and more implementable replication
operator, restricted to input-guarded processes, may be obtained as follows.

\begin{eqnarray}
\bangp{\prefix{u}{v}{P}} 
	:= 
	\binpar{\lift{x}{\prefix{u}{v}{(\binpar{D(x)}{P})}}}{D(x)} \nonumber
\end{eqnarray}

\begin{remark}
  Note that the lazier definition still does not deal with summation
  or mixed summation (i.e. sums over input and output). The reader is
  invited to construct definitions of replication that deal with these
  features. 

  Further, the definitions are parameterized in a name, $x$. Can you,
  gentle reader, make a definition that eliminates this parameter and
  guarantees no accidental interaction between the replication
  machinery and the process being replicated -- i.e. no accidental
  sharing of names used by the process to get its work done and the
  name(s) used by the replication to effect copying. This latter
  revision of the definition of replication is crucial to obtaining
  the expected identity $!!P \sim !P$.
\end{remark}

\begin{remark}\label{rem:paradoxical_combinator}
  The reader familiar with the lambda calculus will have noticed the
  similarity between $D$ and the paradoxical combinator.

  [Ed. note: the existence of this seems to suggest we have to be more
  restrictive on the set of processes and names we admit if we are to
  support no-cloning.]
\end{remark}

\subsubsection{Bisimulation}

The computational dynamics gives rise to another kind of equivalence,
the equivalence of computational behavior. As previously mentioned
this is typically captured \emph{via} some form of bisimulation.

% The notion we use in this paper is weak barbed bisimulation
% \cite{milner91polyadicpi}.

The notion we use in this paper is derived from weak barbed
bisimulation \cite{milner91polyadicpi}. 

\begin{definition}
An \emph{observation relation}, $\downarrow_{\mathcal N}$, over a set
of names, $\mathcal N$, is the smallest relation satisfying the rules
below.

\infrule[Out-barb]{y \in {\mathcal N}, \; x \nameeq y}
		  {\outputp{x}{v} \downarrow_{\mathcal N} x}
\infrule[Par-barb]{\mbox{$P\downarrow_{\mathcal N} x$ or $Q\downarrow_{\mathcal N} x$}}
		  {\binpar{P}{Q} \downarrow_{\mathcal N} x}

We write $P \Downarrow_{\mathcal N} x$ if there is $Q$ such that 
$P \wred Q$ and $Q \downarrow_{\mathcal N} x$.
\end{definition}

\begin{definition}
%\label{def.bbisim}
An  ${\mathcal N}$-\emph{barbed bisimulation} over a set of names, ${\mathcal N}$, is a symmetric binary relation 
${\mathcal S}_{\mathcal N}$ between agents such that $P\rel{S}_{\mathcal N}Q$ implies:
\begin{enumerate}
\item If $P \red P'$ then $Q \wred Q'$ and $P'\rel{S}_{\mathcal N} Q'$.
\item If $P\downarrow_{\mathcal N} x$, then $Q\Downarrow_{\mathcal N} x$.
\end{enumerate}
$P$ is ${\mathcal N}$-barbed bisimilar to $Q$, written
$P \wbbisim_{\mathcal N} Q$, if $P \rel{S}_{\mathcal N} Q$ for some ${\mathcal N}$-barbed bisimulation ${\mathcal S}_{\mathcal N}$.
\end{definition}

$\mathcal{R} \subseteq \pi \times \pi$

$P \mathcal{R} Q => \forall P'. P \red P' \Rightarrow \exists Q'. Q \red Q', P' \mathcal{R} Q'$

$P \vdash x \Rightarrow Q \vdash x$

\begin{mathpar}
  \inferrule*[lab=Out-barb]{x \nameeq y}{{y}!\langle{Q}\rangle \vdash x}
  \and
  \inferrule*[lab=Par-barb]{\mbox{$P\vdash x$ or $Q\vdash x$}}{\binpar{P}{Q} \vdash x}
\end{mathpar}

\subsubsection{Contexts}

One of the principle advantages of computational calculi like the
$\pi$-calculus is a well-defined notion of context,
contextual-equivalence and a correlation between
contextual-equivalence and notions of bisimulation. The notion of
context allows the decomposition of a process into (sub-)process and
its syntactic environment, its context. Thus, a context may be
thought of as a process with a ``hole'' (written $\Box$) in it. The
application of a context $M$ to a process $P$, written $M[P]$, is
tantamount to filling the hole in $M$ with $P$. In this paper we do
not need the full weight of this theory, but do make use of the notion
of context in the proof the main theorem. 

\begin{mathpar}
  \inferrule* [lab=summation] {} {{M_{M},M_{N}} \bc \Box \;|\; x.M_{A} \;|\; M_{M}+M_{N}}
  \and
  \inferrule* [lab=agent] {} {{M_{A}} \bc (\vec{x})M_{P} \;| \; \clift{P_0,\ldots,M_{P},\ldots,P_N}}
  \and \\
  \inferrule* [lab=process] {} {{M_{P}} \bc M_{N} \;| \;P|M_{P} }
\end{mathpar} 

\begin{mathpar}
  \inferrule* [lab=sychronization] {} {M_{N} \bc \Box \;|\; x?M_{F} \;|\; x!M_{C}}
  \and
  \inferrule* [lab=abstraction] {} {{M_{F}} \bc (x)M_{P} }
  \and
  \inferrule* [lab=concretion] {} {{M_{C}} \bc \langle M_{P} \rangle }
  \and \\
  \inferrule* [lab=process] {} {{M_{P}} \bc M_{N} \;| \;P|M_{P} }
\end{mathpar}

\begin{definition}[contextual application] Given a context $M$, and
  process $P$, we define the \emph{contextual application}, $M[P] :=
  M\{P/\Box\}$. That is, the contextual application of M to P is the
  substitution of $P$ for $\Box$ in $M$.
\end{definition}

$\meaningof{-} : L \to \mathcal{P}(\pi)$

\begin{mathpar}
  \inferrule* [lab=collection] {} {\meaningof{true} = \pi, \and \meaningof{~E} = \pi \setminus \meaningof{E}, \and \meaningof{E_{1} \& E_{2}} = \meaningof{E_{1}} \cap \meaningof{E_{2}}}
\end{mathpar}

\begin{mathpar}
  \inferrule* [lab=structure] {} {\meaningof{0} = \{ P \in \pi | P \equiv 0 \}, \and \\ \meaningof{E_1 | E_2} = \{ P \in \pi | P \equiv P_{1} | P_{2}, P_{1} \in \meaningof{E_{1}}, P_{2} \in \meaningof{E_2}\} }
\end{mathpar}

\begin{mathpar}
 \inferrule* [lab=behavior] {} {\meaningof{\langle a?b \rangle E} = \{ P \in \pi | P \equiv Q | u?(y)P', \\ \and \\\\ \and \\ \;\;\; u \in \meaningof{a}, \forall z.P'\{z/y\} \in \meaningof{E\{z/b\}}\}, \and \\ \meaningof{a!E} = \{ P \in \pi | P \equiv Q | x!\langle P' \rangle, x \in \meaningof{a} P' \in \meaningof{E}\} }
\end{mathpar}

\begin{mathpar}
 \inferrule* [lab=nominal] {} {\meaningof{\quotep{E}} = \{ \quotep{P} \in \quotep{\pi} | P \in \meaningof{E} \}, \and \meaningof{\quotep{P}} = \{ \quotep{Q} \in \quotep{\pi} | P \equiv Q \} \and \\ \meaningof{@\quotep{E}} = \{ P \in \pi | P \equiv @x, x \in \meaningof{E} \}}
\end{mathpar}

\begin{eqnarray*}
  \\
  \meaningof{-} : TS \to ST
\end{eqnarray*}

\begin{eqnarray*}
  \\
  L : TS \to ST
\end{eqnarray*}

\begin{eqnarray*}
  \\
  P \models E \iff P \in \meaningof{E}
\end{eqnarray*}

\begin{eqnarray*}
  P \approx_{L} Q \iff \forall E \in L. P \models E \iff Q \models E
\end{eqnarray*}

\begin{eqnarray*}
  P \approx_{K} Q
\end{eqnarray*}

\begin{eqnarray*}
  P \approx Q
\end{eqnarray*}

$\approx_{K} = \approx = \approx_{L}$

\subsubsection{Contextual duality}

Note that contexts extend the quotation operation to a family of
operations from processes to names. Given a context, $M$, we can
define a \emph{nominal context}, $\quotep{M}$ by $\quotep{M}[P] :=
\quotep{M[P]}$. To foreshadow what is to come we observe that these
operations enjoy a duality with processes very much like the duality
between vectors and maps from vectors to scalars.

Further, because the calculus is essentially higher-order, we have a
correspondence between contexts and processes. More specifically,
given a name $x$ and a context $M$ we can construct $M^{*}_{x}$ such
that 

\begin{mathpar}
  M^{*}_{x} | \lift{x}{P} \red M[P]
\end{mathpar}

namely,

\begin{mathpar}
  M^{*}_{x} := x?(u).M[\dropn{u}]
\end{mathpar}

The dependence of $M^{*}_{x}$ on a name makes it an abstraction, 

\begin{mathpar}
  M^{*} := (x)x?(u).M[\dropn{u}]
\end{mathpar}

\subsection{Additional notation}

It will sometimes be convenient to denote the process a name
quotes. We already have the notation $x = \quotep{P}$, but it will be
convenient to introduce an alternate notation, $\procn{x}$, when we
want to emphasize the connection to the use of the name. Note that, by
virtue of name equivalence, $\quotep{\procn{x}} \nameeq x$; so, the
notation is consistent with previous definitions.

Further, because names have structure it is possible to effect
substitutions on the basis of that structure. This means we need to
upgrade our notation for substitutions, which we accomplish by
adapting comprehension notation. Thus,

\begin{mathpar}
  P\{ y / x : x \in S \}
\end{mathpar}

is interpreted to mean the process derived from P by replacing (in a
capture-avoiding manner) each occurrence of $x$ in $S$ by $y$. For example,

\begin{mathpar}
  P\{ \quotep{\procn{x}|\procn{x}} / x : x \in \freenames{P} \}
\end{mathpar}

will replace each (occurrence) of a free name $x$ in $P$ by
$\quotep{\procn{x}|\procn{x}}$.

Also, we will avail ourselves of the notation $x^{L}$ and $x^{R}$ to
denote injections of a name into disjoint copies of the name
space. There are numerous ways to accomplish this. One example can be
found in \cite{MeredithR05}. This notation overloads to vectors of
names: $\vec{x}^{\pi} := (x_{i}^{\pi} \; : \; 0 \leq i < |\vec{x}| )$ where $\pi \in \{L,R\}$.

We also use $P^{\Box} := P|\Box$.

In \cite{MeredithR05} an interpretation of the new operator is
given. It turns out that there are several possible interpretations
all enjoying the requisite algebraic properties of the operator (see
\cite{milner91polyadicpi}). We will therefore make liberal use of
$(\nu\; \vec{x})P$.

% subsection the_syntax_and_semantics_of_the_notation_system (end)   

\input{qm2pi.qmops} 

\input{qm2pi.sterngerlach} 

\input{qm2pi.metric} 

% section concurrent_process_calculi (end)

%\input{qm2pi.proofsketch}

% section proof sketch (end)

%\input{qm2pi.slviaknots} 

% section spatial logic via knots (end)

\input{qm2pi.conclusion}

% section conclusion (end)

%\input{qm2pi.dtcodes} 

% section wiring algorithm (end)

\input{qm2pi.ack} 

% section acknowledgments (end)

\newpage


\bibliographystyle{plain}   
\bibliography{../../biblios/main.bib}

\input{qm2pi.rhodetails}

\end{document}

 

% subsection basic_interpretation (end)

%\input{qm2pi.rho.presentation} 
\subsection{The syntax and semantics of the notation system}\label{sub:the_syntax_and_semantics_of_the_notation_system} % (fold)

We now summarize a technical presentation of the calculus that
embodies our theory of dynamics. The typical presentation of such a
calculus follows the style of giving generators and relations on
them. The grammar, below, describing term constructors, freely
generates the set of processes, $\Proc$. This set is then quotiented
by a relation known as structural congruence and it is over this set
that the notion of dynamics is expressed. This presentation is
essentially that of \cite{MeredithR05} with the addition of
polyadicity and summation. For readability we have relegated some of
the technical subtleties to an appendix.

\subsubsection{Process grammar}\label{subsub:process_grammar}

\begin{mathpar}
  \inferrule* [lab=synchronization] {} {{M} \bc \pzero \;|\; x?F \;|\; x!C }
  \and
  \inferrule* [lab=abstraction] {} {{F} \bc (x)P}
  \and
  \inferrule* [lab=concretion] {} {{C} \bc \langle Q \rangle}
  \and
  \inferrule* [lab=process] {} {{P,Q} \bc M \;| \;P|Q \;|\; @{x}}
  \and
  \inferrule* [lab=name] {} {{x} \bc \quotep{P}}
\end{mathpar} 

Note that $\vec{x}$ (resp. $\vec{P}$) denotes a vector of names
(resp. processes) of length $|\vec{x}|$ (resp. $|\vec{P}|$). We adopt
the following useful abbreviations.

\begin{mathpar}
   x?(\vec{y}).P := x.(\vec{y})P \and  x\clift{\vec{P}} := x.\clift{\vec{P}}
   \and x!(y) := \lift{x}{\dropn{y}}
   \and \Pi_{i=0}^{n-1}P_i := P_0 | \ldots | P_{n-1}
\end{mathpar}

\subsubsection{Structural congruence}

\paragraph{Free and bound names and alpha-equivalence.} At the
core of structural equivalence is alpha-equivalence which identifies
process that are the same up to a change of variable. Formally, we
recognize the distinction between free and bound names. The free names
of a process, $\freenames{P}$, may be calculated recursively as
follows:

\begin{mathpar}
\freenames{\pzero} := \emptyset
  \and \\
  \freenames{x?(y).P} := \{ x \} \cup (\freenames{P} \setminus \{ y \})
  \and 
  \freenames{x!\langle P \rangle} := \{ x \} \cup \{ P \} 
  \and \\
  \freenames{P|Q} := \freenames{P} \cup \freenames{Q}
  \and \\
  \freenames{@{x}} := \{ x \}
\end{mathpar}

$\pi$
$\quotep{\pi}$

$\freenames{-} : \pi \to \mathcal{P}(\quotep{\pi})$

\begin{eqnarray*}
  \freenames{\pzero} & := & \emptyset \\
  \freenames{x?(y).P} & := & \{ x \} \cup (\freenames{P} \setminus \{ y \}) \\
  \freenames{x!\langle P \rangle} & := & \{ x \} \cup \{ P \} \\
  \freenames{P|Q} & := & \freenames{P} \cup \freenames{Q} \\
  \freenames{\dropn{x}} & := & \{ x \}
\end{eqnarray*}

The bound names of a process, $\boundnames{P}$, are those names occurring in $P$
that are not free. For example, in $x?(y).0$, the name $x$ is free, while $y$ is bound.

\begin{mathpar}
  \inferrule* [lab=monoidal-laws] {} { P|Q \equiv Q|P \and P|0 \equiv P \and P|(Q|R) \equiv (P|Q)|R }
\end{mathpar}

\begin{mathpar}
  \inferrule* [lab=alpha-equivalence] {} { (x)P \equiv (y)P\{y/x\} \and y \not\in \freenames{P} }
\end{mathpar}

\begin{definition}
Then two processes, $P,Q$, are alpha-equivalent if $P = Q\{\vec{y}/\vec{x}\}$ for
some $\vec{x} \in \boundnames{Q},\vec{y} \in \boundnames{P}$, where $Q\{\vec{y}/\vec{x}\}$
denotes the capture-avoiding substitution of $\vec{y}$ for $\vec{x}$ in $Q$.
\end{definition}

\begin{definition}
  The {\em structural congruence} \cite{SangiorgiWalker} , $\equiv$,
  between processes is the least congruence containing
  alpha-equivalence, satisfying the abelian monoid laws
  (associativity, commutativity and $\pzero$ as identity) for parallel
  composition $|$ and for summation $+$.
\end{definition}

\subsection{Name equivalence}

We take name equivalence, written $\nameeq$, to be the smallest
equivalence relation generated by the following rules.

\begin{mathpar}
\inferrule*[lab=Quote-drop]
{ }
{ \quotep{@{x}} \nameeq x }

\inferrule*[lab=Struct-equiv]
{ P \scong Q }
{ \quotep{P} \nameeq \quotep{Q} }
\end{mathpar}

The astute reader will have noticed that the mutual recursion of names
and processes imposes a mutual recursion on alpha-equivalence and
structural equivalence via name-equivalence. Fortunately, all of this
works out pleasantly and we may calculate in the natural way, free of
concern. The reader interested in the details is referred to the
appendix \ref{appendix:rho_details}.

\subsection{Substitution}

We use $\Proc$ for the set of processes, $\QProc$ for the set of
names, and $\id{\{}\vec{y} / \vec{x} \id{\}}$ to denote partial maps,
$s : \QProc \rightarrow \QProc$. A map, $s$ lifts, uniquely, to a map
on process terms, $\widehat{s} : \Proc \rightarrow \Proc$ by the
following equations.

\begin{mathpar}
  (0) \psubstp{Q}{P} := 0 \\
  (R \juxtap S) \psubstp{Q}{P}
  :=    
  (R)\psubstp{Q}{P} \juxtap (S) \psubstp{Q}{P} \\
  (x?(y).R) \psubstp{Q}{P}    
  :=    
  (x)\substp{Q}{P} (z)\concat( (R \psubstn{z}{y}) \psubstp{Q}{P} ) \\
  (\lift{x}{R}) \psubstp{Q}{P}  
  :=
  \lift{(x)\substp{Q}{P}}{ R \psubstp{Q}{P} } \\
%   (\dropn{x})  \psubstp{Q}{P}       
%   := 
%   \left\{ 
%     \begin{array}{ccc} 
%       \dropn{\quotep{Q}} & & x \nameeq \quotep{P} \\
%       \dropn{x} & & otherwise \\
%     \end{array}
%   \right. 
  (\dropn{x})  \psubstp{Q}{P}       
  := 
  \left\{ 
    \begin{array}{ccc} 
      Q & & x \nameeq \quotep{P} \\
      \dropn{x} & & otherwise \\
    \end{array}
  \right.
\end{mathpar}
 

where

\begin{eqnarray}
  (x)\id{\{} \lpquote Q \rpquote / \lpquote P \rpquote \id{\}}            = 
  \left\{ 
    \begin{array}{ccc}
      \lpquote Q \rpquote & & x \nameeq \lpquote P \rpquote \\
      x & & otherwise \\
    \end{array}
  \right. \nonumber
\end{eqnarray}

and $z$ is chosen distinct from $\quotep{P}$, $\quotep{Q}$, the free
names in $Q$, and all the names in $R$. Our $\alpha$-equivalence will
be built in the standard way from this substitution.

\begin{remark}\label{rem:no_self_referential_names}
  One consequence of these definitions is that $\forall P. \quotep{P}
  \not\in \freenames{P}$.
\end{remark}

\subsection{ Dynamic quote: an example }

Anticipating something of what's to come, consider applying the
substitution, $\widehat{\id{\{}u / z \id{\}}}$, to the following pair
of processes, $\lift{w}{y!(z)}$ and $w[ \lpquote y!(z) \rpquote ]$.

\begin{eqnarray}
	\lift{w}{y!(z)}\widehat{\id{\{}u / z \id{\}}}
		& = &
		\lift{w}{y!(u)} \nonumber\\
	w[ \lpquote y!(z) \rpquote ] \widehat{ \id{\{}u / z \id{\}} }
		& = &
		w[ \lpquote y!(z) \rpquote ] \nonumber
\end{eqnarray}

Because the body of the process between quotes is impervious to
substitution, we get radically different answers. In fact, by
examining the first process in an input context,
e.g. $x?(z).\lift{w}{y!(z)}$, we see that the process under the lift
operator may be shaped by prefixed inputs binding a name inside it. In
this sense, the lift operator will be seen as a way to dynamically
construct processes before reifying them as names.

Finally equipped with these standard features we can present the
dynamics of the calculus.

\subsubsection{Operational semantics} 

Finally, we introduce the computational dynamics. What marks these
algebras as distinct from other more traditionally studied algebraic
structures, e.g. vector spaces or polynomial rings, is the manner in
which dynamics is captured. In traditional structures, dynamics is typically
expressed through morphisms between such structures, as in linear maps
between vector spaces or morphisms between rings. In algebras
associated with the semantics of computation, the dynamics is
expressed as part of the algebraic structure itself, through a
reduction reduction relation typically denoted by $\red$. Below, we
give a recursive presentation of this relation for the calculus used
in the encoding.

$\red \subseteq \pi \times \pi$
$\red : \pi \to \mathcal{P}(\pi)$

\begin{mathpar}
  \inferrule* [lab=Comm] { \textsf{match}( x_{src}, x_{trgt} ) } { x_{trgt}?(y)P \; | \; x_{src}!\langle {Q} \rangle \red P\{\quotep{Q}/y}\} }
  \and \\
  \inferrule* [lab=Par] {{P} \red {P}'} {{{P} | {Q}} \red {{P}' | {Q}}}
  \and
  \inferrule* [lab=Equiv]{{{P} \scong {P}'} \andalso {{P}' \red {Q}'} \andalso {{Q}' \scong {Q}}}{{P} \red {Q}}
\end{mathpar}

\begin{eqnarray*}
  match_{\equiv} (\quotep{P},\quotep{Q}) & := & P \equiv Q \\
  match_{\dagger}(\quotep{P},\quotep{Q}) & := & \forall R. P|Q \red^{*} R => R \red^{*} 0 \\
  match_{K}(\quotep{P},\quotep{Q}) & := & K \mbox{ for some context } K
\end{eqnarray*}

$u?(x)P | u!\langle Q \rangle \red P\{\quotep{Q}/x\}$

%We write $\wred$ for $\red^*$, and $P\red$ if $\exists Q $ such that $ P \red Q$.
We write $P\red$ if $\exists Q $ such that $ P \red Q$ and $P\not\red$, otherwise.

\section{Replication}

As mentioned before, it is known that replication (and hence
recursion) can be implemented in a higher-order process algebra
\cite{SangiorgiWalker}. As our first example of calculation with the
machinery thus far presented we give the construction explicitly in
the {\rhoc}.

\begin{eqnarray}
	D_{x} & := & \prefix{x}{y}{(\binpar{\outputp{x}{y}}{@{y}})} \nonumber\\
	\bangp_{x}{P} & := & \binpar{{x}!\langle{\binpar{D_{x}}{P}}\rangle}{D_{x}} \nonumber
\end{eqnarray}

\begin{eqnarray}
	\bangp_{x}{P} & & \nonumber\\
	=
	& {x}!\langle{(\prefix{x}{y}{(\outputp{x}{y} | @{y})) | P}}\rangle 
	      | \prefix{x}{y}{(\outputp{x}{y} | @{y})} & \nonumber\\
	\red
	& (\outputp{x}{y} | @{y})\substn{\quotep{(\prefix{x}{y}{(@{y} | \outputp{x}{y})) | P}}}{y} & \nonumber\\
	=
	& \outputp{x}{\quotep{(\prefix{x}{y}{(\outputp{x}{y} | @{y})) | P}}}
	  | {(\prefix{x}{y}{(\outputp{x}{y} | @{y})) | P}} & \nonumber\\
	\red
	& \ldots & \nonumber\\
	\red^*
	& P | P | \ldots & \nonumber
\end{eqnarray}

Of course, this encoding, as an implementation, runs away, unfolding
$\bangp{P}$ eagerly. A lazier and more implementable replication
operator, restricted to input-guarded processes, may be obtained as follows.

\begin{eqnarray}
\bangp{\prefix{u}{v}{P}} 
	:= 
	\binpar{\lift{x}{\prefix{u}{v}{(\binpar{D(x)}{P})}}}{D(x)} \nonumber
\end{eqnarray}

\begin{remark}
  Note that the lazier definition still does not deal with summation
  or mixed summation (i.e. sums over input and output). The reader is
  invited to construct definitions of replication that deal with these
  features. 

  Further, the definitions are parameterized in a name, $x$. Can you,
  gentle reader, make a definition that eliminates this parameter and
  guarantees no accidental interaction between the replication
  machinery and the process being replicated -- i.e. no accidental
  sharing of names used by the process to get its work done and the
  name(s) used by the replication to effect copying. This latter
  revision of the definition of replication is crucial to obtaining
  the expected identity $!!P \sim !P$.
\end{remark}

\begin{remark}\label{rem:paradoxical_combinator}
  The reader familiar with the lambda calculus will have noticed the
  similarity between $D$ and the paradoxical combinator.

  [Ed. note: the existence of this seems to suggest we have to be more
  restrictive on the set of processes and names we admit if we are to
  support no-cloning.]
\end{remark}

\subsubsection{Bisimulation}

The computational dynamics gives rise to another kind of equivalence,
the equivalence of computational behavior. As previously mentioned
this is typically captured \emph{via} some form of bisimulation.

% The notion we use in this paper is weak barbed bisimulation
% \cite{milner91polyadicpi}.

The notion we use in this paper is derived from weak barbed
bisimulation \cite{milner91polyadicpi}. 

\begin{definition}
An \emph{observation relation}, $\downarrow_{\mathcal N}$, over a set
of names, $\mathcal N$, is the smallest relation satisfying the rules
below.

\infrule[Out-barb]{y \in {\mathcal N}, \; x \nameeq y}
		  {\outputp{x}{v} \downarrow_{\mathcal N} x}
\infrule[Par-barb]{\mbox{$P\downarrow_{\mathcal N} x$ or $Q\downarrow_{\mathcal N} x$}}
		  {\binpar{P}{Q} \downarrow_{\mathcal N} x}

We write $P \Downarrow_{\mathcal N} x$ if there is $Q$ such that 
$P \wred Q$ and $Q \downarrow_{\mathcal N} x$.
\end{definition}

\begin{definition}
%\label{def.bbisim}
An  ${\mathcal N}$-\emph{barbed bisimulation} over a set of names, ${\mathcal N}$, is a symmetric binary relation 
${\mathcal S}_{\mathcal N}$ between agents such that $P\rel{S}_{\mathcal N}Q$ implies:
\begin{enumerate}
\item If $P \red P'$ then $Q \wred Q'$ and $P'\rel{S}_{\mathcal N} Q'$.
\item If $P\downarrow_{\mathcal N} x$, then $Q\Downarrow_{\mathcal N} x$.
\end{enumerate}
$P$ is ${\mathcal N}$-barbed bisimilar to $Q$, written
$P \wbbisim_{\mathcal N} Q$, if $P \rel{S}_{\mathcal N} Q$ for some ${\mathcal N}$-barbed bisimulation ${\mathcal S}_{\mathcal N}$.
\end{definition}

$\mathcal{R} \subseteq \pi \times \pi$

$P \mathcal{R} Q => \forall P'. P \red P' \Rightarrow \exists Q'. Q \red Q', P' \mathcal{R} Q'$

$P \vdash x \Rightarrow Q \vdash x$

\begin{mathpar}
  \inferrule*[lab=Out-barb]{x \nameeq y}{{y}!\langle{Q}\rangle \vdash x}
  \and
  \inferrule*[lab=Par-barb]{\mbox{$P\vdash x$ or $Q\vdash x$}}{\binpar{P}{Q} \vdash x}
\end{mathpar}

\subsubsection{Contexts}

One of the principle advantages of computational calculi like the
$\pi$-calculus is a well-defined notion of context,
contextual-equivalence and a correlation between
contextual-equivalence and notions of bisimulation. The notion of
context allows the decomposition of a process into (sub-)process and
its syntactic environment, its context. Thus, a context may be
thought of as a process with a ``hole'' (written $\Box$) in it. The
application of a context $M$ to a process $P$, written $M[P]$, is
tantamount to filling the hole in $M$ with $P$. In this paper we do
not need the full weight of this theory, but do make use of the notion
of context in the proof the main theorem. 

\begin{mathpar}
  \inferrule* [lab=summation] {} {{M_{M},M_{N}} \bc \Box \;|\; x.M_{A} \;|\; M_{M}+M_{N}}
  \and
  \inferrule* [lab=agent] {} {{M_{A}} \bc (\vec{x})M_{P} \;| \; \clift{P_0,\ldots,M_{P},\ldots,P_N}}
  \and \\
  \inferrule* [lab=process] {} {{M_{P}} \bc M_{N} \;| \;P|M_{P} }
\end{mathpar} 

\begin{mathpar}
  \inferrule* [lab=sychronization] {} {M_{N} \bc \Box \;|\; x?M_{F} \;|\; x!M_{C}}
  \and
  \inferrule* [lab=abstraction] {} {{M_{F}} \bc (x)M_{P} }
  \and
  \inferrule* [lab=concretion] {} {{M_{C}} \bc \langle M_{P} \rangle }
  \and \\
  \inferrule* [lab=process] {} {{M_{P}} \bc M_{N} \;| \;P|M_{P} }
\end{mathpar}

\begin{definition}[contextual application] Given a context $M$, and
  process $P$, we define the \emph{contextual application}, $M[P] :=
  M\{P/\Box\}$. That is, the contextual application of M to P is the
  substitution of $P$ for $\Box$ in $M$.
\end{definition}

$\meaningof{-} : L \to \mathcal{P}(\pi)$

\begin{mathpar}
  \inferrule* [lab=collection] {} {\meaningof{true} = \pi, \and \meaningof{~E} = \pi \setminus \meaningof{E}, \and \meaningof{E_{1} \& E_{2}} = \meaningof{E_{1}} \cap \meaningof{E_{2}}}
\end{mathpar}

\begin{mathpar}
  \inferrule* [lab=structure] {} {\meaningof{0} = \{ P \in \pi | P \equiv 0 \}, \and \\ \meaningof{E_1 | E_2} = \{ P \in \pi | P \equiv P_{1} | P_{2}, P_{1} \in \meaningof{E_{1}}, P_{2} \in \meaningof{E_2}\} }
\end{mathpar}

\begin{mathpar}
 \inferrule* [lab=behavior] {} {\meaningof{\langle a?b \rangle E} = \{ P \in \pi | P \equiv Q | u?(y)P', \\ \and \\\\ \and \\ \;\;\; u \in \meaningof{a}, \forall z.P'\{z/y\} \in \meaningof{E\{z/b\}}\}, \and \\ \meaningof{a!E} = \{ P \in \pi | P \equiv Q | x!\langle P' \rangle, x \in \meaningof{a} P' \in \meaningof{E}\} }
\end{mathpar}

\begin{mathpar}
 \inferrule* [lab=nominal] {} {\meaningof{\quotep{E}} = \{ \quotep{P} \in \quotep{\pi} | P \in \meaningof{E} \}, \and \meaningof{\quotep{P}} = \{ \quotep{Q} \in \quotep{\pi} | P \equiv Q \} \and \\ \meaningof{@\quotep{E}} = \{ P \in \pi | P \equiv @x, x \in \meaningof{E} \}}
\end{mathpar}

\begin{eqnarray*}
  \\
  \meaningof{-} : TS \to ST
\end{eqnarray*}

\begin{eqnarray*}
  \\
  L : TS \to ST
\end{eqnarray*}

\begin{eqnarray*}
  \\
  P \models E \iff P \in \meaningof{E}
\end{eqnarray*}

\begin{eqnarray*}
  P \approx_{L} Q \iff \forall E \in L. P \models E \iff Q \models E
\end{eqnarray*}

\begin{eqnarray*}
  P \approx_{K} Q
\end{eqnarray*}

\begin{eqnarray*}
  P \approx Q
\end{eqnarray*}

$\approx_{K} = \approx = \approx_{L}$

\subsubsection{Contextual duality}

Note that contexts extend the quotation operation to a family of
operations from processes to names. Given a context, $M$, we can
define a \emph{nominal context}, $\quotep{M}$ by $\quotep{M}[P] :=
\quotep{M[P]}$. To foreshadow what is to come we observe that these
operations enjoy a duality with processes very much like the duality
between vectors and maps from vectors to scalars.

Further, because the calculus is essentially higher-order, we have a
correspondence between contexts and processes. More specifically,
given a name $x$ and a context $M$ we can construct $M^{*}_{x}$ such
that 

\begin{mathpar}
  M^{*}_{x} | \lift{x}{P} \red M[P]
\end{mathpar}

namely,

\begin{mathpar}
  M^{*}_{x} := x?(u).M[\dropn{u}]
\end{mathpar}

The dependence of $M^{*}_{x}$ on a name makes it an abstraction, 

\begin{mathpar}
  M^{*} := (x)x?(u).M[\dropn{u}]
\end{mathpar}

\subsection{Additional notation}

It will sometimes be convenient to denote the process a name
quotes. We already have the notation $x = \quotep{P}$, but it will be
convenient to introduce an alternate notation, $\procn{x}$, when we
want to emphasize the connection to the use of the name. Note that, by
virtue of name equivalence, $\quotep{\procn{x}} \nameeq x$; so, the
notation is consistent with previous definitions.

Further, because names have structure it is possible to effect
substitutions on the basis of that structure. This means we need to
upgrade our notation for substitutions, which we accomplish by
adapting comprehension notation. Thus,

\begin{mathpar}
  P\{ y / x : x \in S \}
\end{mathpar}

is interpreted to mean the process derived from P by replacing (in a
capture-avoiding manner) each occurrence of $x$ in $S$ by $y$. For example,

\begin{mathpar}
  P\{ \quotep{\procn{x}|\procn{x}} / x : x \in \freenames{P} \}
\end{mathpar}

will replace each (occurrence) of a free name $x$ in $P$ by
$\quotep{\procn{x}|\procn{x}}$.

Also, we will avail ourselves of the notation $x^{L}$ and $x^{R}$ to
denote injections of a name into disjoint copies of the name
space. There are numerous ways to accomplish this. One example can be
found in \cite{MeredithR05}. This notation overloads to vectors of
names: $\vec{x}^{\pi} := (x_{i}^{\pi} \; : \; 0 \leq i < |\vec{x}| )$ where $\pi \in \{L,R\}$.

We also use $P^{\Box} := P|\Box$.

In \cite{MeredithR05} an interpretation of the new operator is
given. It turns out that there are several possible interpretations
all enjoying the requisite algebraic properties of the operator (see
\cite{milner91polyadicpi}). We will therefore make liberal use of
$(\nu\; \vec{x})P$.

% subsection the_syntax_and_semantics_of_the_notation_system (end)   

\section{Interpretation of QM}
\subsection{Supporting definitions}
\subsubsection{Multiplication}
\begin{mathpar}
  \quotep{Q} \cdot \quotep{R} := \quotep{Q|R}
  \and \\
  \quotep{Q} \cdot P := P\{ \quotep{Q|R} / \quotep{R} : \quotep{R} \in \freenames{P} \}
\end{mathpar}

\paragraph{Discussion}
The first line needs little explanation. The second line says that
each free name of the process is replaced with the multiplication of
that name by the scalar. Multiplication of a scalar (name) by a state
(process) results in a process all the names of which have been `moved
over' by parallel composition with the process the scalar
quotes. There is a subtlety that the bound names have to be
manipulated so that multiplied names aren't accidentally
captured. There are many ways to achieve this.

\begin{remark}\label{rem:multiplication_identities}
  The reader is invited to verify that for all $x,y,z \in \QProc$ and $P \in \Proc$
  \begin{mathpar}
    x \cdot \quotep{0} \equiv x 
    \and
    x \cdot y \equiv y \cdot x
    \and
    x \cdot (y \cdot z) \equiv (x \cdot y) \cdot z
    \and \\
    \quotep{0} \cdot P \equiv P
    \and \\
    x \cdot (y \cdot P) \equiv (x \cdot y) \cdot P
    \and \\
    x \cdot (P|Q) \equiv (x \cdot P) | (x \cdot Q)
    \and \\    
  \end{mathpar}
\end{remark}

\subsubsection{Tensor product}

We define a tensor product on processes by structural induction.

\paragraph{Tensor of sums} First note that all summations, including
$\pzero$ and sequence, can be written $\Sigma_{i} x_{i}.A_{i} +
\Sigma_{j} x_{j}.C_{j}$, where we have grouped input-guarded processes
together and output-guarded processes together.

Thus, we can define the tensor product of two summations, $N_{1}\otimes N_{2}$, where

\begin{mathpar}
  N_{1} := \Sigma_{i} x_{i}.A_{i} + \Sigma_{j} x_{j}.C_{j}
  \and
  N_{2} := \Sigma_{i'} y_{i'}.B_{i'} + \Sigma_{j'} y_{j'}.D_{j'} 
\end{mathpar}

as follows.

\begin{mathpar}
  \Sigma_{i} x_{i}.A_{i} + \Sigma_{j} x_{j}.C_{j} \otimes \Sigma_{i'}
  y_{i'}.B_{i'} + \Sigma_{j'} y_{j'}.D_{j'} 
  \and \\
  := \; \Sigma_{i} \Sigma_{i'} \quotep{\stackrel{\vee}{x_{i}}| \stackrel{\vee}{y_{i'}}}.(A_{i}\otimes B_{i'}) \; | \; \Sigma_{i'} \Sigma_{i} \quotep{\stackrel{\vee}{y_{i'}}|\stackrel{\vee}{x_{i}}}.(B_{i'}\otimes A_{i})
  \and
  \;\; | \;\; \Sigma_{j} \Sigma_{j'} \quotep{\stackrel{\vee}{x_{j}}|\stackrel{\vee}{y_{j'}}}.(A_{j}\otimes B_{j'}) \; | \; \Sigma_{j'} \Sigma_{j} \quotep{\stackrel{\vee}{y_{j'}}|\stackrel{\vee}{x_{j}}}.(B_{j'}\otimes A_{j})
\end{mathpar}

\begin{remark}
  Do we need to $x^{L}$ and $y^{R}$ for this construction as well?
\end{remark}

\paragraph{Tensor of parallel compositions} Next, we distribute tensor
over par.

\begin{mathpar}
  P_{1}|P_{2} \otimes Q_{1}|Q_{2} := (P_{1} \otimes Q_{1}) | (P_{1}
  \otimes Q_{2}) | (P_{2} \otimes Q_{1}) | (P_{2} \otimes Q_{2})
\end{mathpar}

\paragraph{Tensor with dropped names} We treat tensor of a
process with a dropped name as parallel composition.

\begin{mathpar}
  P \otimes \dropn{x} := P | \dropn{x}
\end{mathpar}

\paragraph{Tensor of agents}

Finally, we need to define tensor on agents. Note that the definition
of tensor on normal products only tensors inputs with inputs and
outputs with outputs. Thus, we only have to define the operation on
``homogeneous'' pairings.

\begin{mathpar}
  (\vec{x})P \otimes (\vec{y})Q
  \and \\
  := (x_{0}^{L}|y_{0}^{R},\ldots,x_{0}^{L}|y_{n}^{R},\ldots,x_{m}^{L}|y_{0}^{R},\ldots,x_{m}^{L}|y_{n}^R)(P\{ \vec{x}^{L}/\vec{x}\} \otimes Q \{ \vec{y}^{R}/\vec{y}\})
  \and \\
  \clift{\vec{P}} \otimes \clift{\vec{Q}}
  \and \\
  := \clift{P_{0}\otimes Q_{0},\ldots,P_{0}\otimes Q_{n},\ldots,P_{m}\otimes Q_{0},\ldots,P_{m}\otimes Q_{n}}
\end{mathpar}

\begin{remark}
  Observe that arities of tensored abstractions matches arities of
  tensored concretions if the original arities matched. Note also that
  the length of the arities corresponds to the increase in dimension
  we see in ordinary vector space tensor product.
\end{remark}

\begin{remark}
  Operationally, this definition distributes the tensor down to
  components ``linked'' by summation. Tensor over summation is
  intriguing in that it mixes names. Moreover, as a consequence of the
  way it mixes names we have the identities for all $x \in \QProc$ and
  $P,Q \in \Proc$

  \begin{mathpar}
    (x \cdot P) \otimes Q \equiv x \cdot (P \otimes Q) \equiv P \otimes (x \cdot Q)
    \and
    P \otimes \pzero \equiv P
  \end{mathpar}

  that the reader is invited to verify.
\end{remark}

\subsubsection{Annihilation}
\begin{mathpar}
  P^{\perp} := \{ Q | \forall R. P|Q \red^{*} R \Rightarrow R \red^{*} \pzero \}
  \and \\
  P^{\underline{\perp}} := \Sigma_{Q \in P^{\perp}} \quotep{Q}?(y).(\dropn{y}|Q) | \Sigma_{Q \in P^{\perp}} \quotep{Q}\clift{\Box}
\end{mathpar}

\paragraph{Discussion} The reader will note that $P^{\perp}$ is a
\emph{set} of processes, while $P^{\underline{\perp}}$ is a
\emph{context}. We call the set $P^{\perp}$ the \emph{annihilators} of
$P$. The parallel composition of a process in the annihilators of $P$
with $P$ will result in a process, the state space of which has all
paths eventually leading to $\pzero$. Execution may endure loops; but
under reasonable conditions of fairness (naturally guaranteed under
most notions of bisimulation) such a composite process cannot get
stuck in such a loop and will, eventually pop out and terminate.

The context $P^{\underline{\perp}}$ is ready and willing to ``take the
$P$ out of'' the process to which it is applied. It will effectively
transmit the code of the process to which it is applied to one of the
annihilators and run the process against it.

\subsubsection{Evaluation}
We fix $M$ a domain of fully abstract interpretation with an equality
coincident with bisimulation. We take $\meaningof{\cdot} : \Proc \to
M$ to be the map interpreting processes and $\nmeaningof{\cdot} : \M
\to Proc$ to be the map running the other way. Then we define

\begin{mathpar}
  \int P := \nmeaningof{\meaningof{P}}
\end{mathpar}

\paragraph{Discussion}
There are many fully abstract interpretations of Milner's
$\pi$-calculus. Any of them can be used as a basis for interpreting
the reflective calculus here. Equipped with such a domain it is
largely a matter of grinding through to check that the Yoneda
construction for the normalization-by-evaluation program can be
extended to this setting.

\begin{remark}
  The reader is invited to verify that $\int (P^{\underline{\perp}}[P]) = 0$.
\end{remark}

\subsection{Quantum mechanics}

Table \ref{tbl:core_qm_op_defns} gives the core operational definitions

\begin{table}[htp]\label{tbl:core_qm_op_defns}
  \center{
    \fbox{
      \begin{tabular}{c|c}
        quantum mechanics & process calculus \\
        \hline
        scalar & $x := \quotep{P}$ \\
        state vector & $\state{P} := P$ \\
        dual & $\state{P}^{*} := \event{P^{\underline{\perp}}} := \quotep{P^{\underline{\perp}}}[-]$ \\
        matrix & $ \Sigma_{\alpha} \state{P_{\alpha}}x_{\alpha}\event{Q_{\alpha}}$ \\
        vector addition & $\state{P} + \state{Q} := \state{P | Q}$ \\
        tensor product & $\state{P} \otimes \state{Q} := \state{P \otimes Q}$ \\
        inner product & $\innerprod{P}{Q} := \quotep{\int P^{\underline{\perp}}[Q]}$ \\
      \end{tabular}
    }
  }
  \caption{QM - operational definitions}
\end{table}

where

\begin{mathpar}
  \prmatrix{P}{Q} := \fprmatrix{P}{\quotep{\pzero}}{Q}
  \and
  \fprmatrix{P}{x}{Q} := (\state{P},x,\event{Q})
  \and
  (\fprmatrix{P}{x}{Q})(\state{R}) := x \cdot \innerprod{Q}{R} \cdot \state{P}
  \and
  (\fprmatrix{P}{x}{Q})(\event{R}) := x \cdot \innerprod{R}{P} \cdot \event{Q}
\end{mathpar}

\paragraph{Discussion}
As promised: vectors (aka states) are represented as processes; duals
as contextual duals; inner product definition should be compared with
standard inner product definition for ....

\begin{remark}
  Assuming $\int (P^{\underline{\perp}}[P]) = 0$, the reader is
  invited to verify that $(\fprmatrix{P}{x}{P})(\state{P}) = x \cdot \state{P}$.
\end{remark}

\begin{remark}
  The reader is invited to verify that $\innerprod{P}{Q}$ could
  equally well have been written $\quotep{\int \stackrel{\vee}{x}}$
  where $x = \event{P^{\underline{\perp}}}(Q)$.

  One of the motivations for this remark is that there is another way
  to factor these operations. We could package up evaluation in the dual:

  \begin{mathpar}
    \state{P}^{*} := \event{\int P^{\underline{\perp}}} := \quotep{\int P^{\underline{\perp}}}[-]
  \end{mathpar}

  and then have inner product defined by
  
  \begin{mathpar}
    \innerprod{P}{Q} := \event{P}(Q)
  \end{mathpar}

  Hopefully, experience with the calculations will provide guidance on
  the best factoring.
\end{remark}

\begin{remark}
  Assuming $\int (P^{\underline{\perp}}[P]) = 0$, the reader is
  invited to verify that $\forall P,Q. (\prmatrix{0}{Q})(\state{0}) =
  \state{0}$ and dually $(\prmatrix{P}{0})(\event{0}) = \event{0}$.
\end{remark}

\begin{remark}
  i'm a little worried that i don't (yet) have proper support for
  complex conjugacy. But, the observation above may give us a
  clue. According to Abramsky, it must be the case that the scalars
  are iso to the homset of the identity for the tensor -- which the
  observation above characterizes. 

  For now, we will simply bookmark the notion with $\overline{x}$.
\end{remark}

\subsubsection{Adjointness}

We need to give a definition of $(\cdot)^{\dagger}$ for matrices. The
obvious candidate definition is
\begin{mathpar}
(\Sigma_{\alpha}\fprmatrix{P_{\alpha}}{x_{\alpha}}{Q_{\alpha}})^{\dagger}
= \Sigma_{\alpha}\fprmatrix{(Q_{\alpha}^{\underline{\perp}})^{*}}{\overline{x}_{\alpha}}{P_{\alpha}^{\underline{\perp}}} 
\end{mathpar}

But, $(Q_{\alpha}^{\underline{\perp}})^{*}$ requires a name along
which to communicate the process to achieve the context application.

\subsubsection{Basis for a basis}
If processes label states and ``addition'' of states (a.k.a. vector
addition) is interpreted as parallel composition, what corresponds to
notions of linear independence and basis? Here, we recall that Yoshida
has developed a set of \emph{combinators} for an asynchronous verison
of Milner's $\pi$-calculus. These are a finite set of processes such
any process can be expressed as parallel composition of these
combinators together with liberal uses of the new operator and
replication. We can simply give a translation of these into the
present calculus and have reasonable expectation that the property
carries over. That is, that the resultant set allows to express all
processes via parallel composition. Note, however, that there is no
new operator or replication in this calculus. As a result, we expect
that the corresponding set is actually infinite. That is, we expect
that the space is actually infinite dimensional.

\begin{remark}
  The attentive reader may be a bit concerned. Certainly, the
  collection $S$, $K$ and $I$ is a finite set of
  combinators. Shouldn't we expect to see a finite set of combinators
  for an effectively equivalent system? i am very sympathetic to this
  critique and feel it warrants full attention. On the other hand, i
  also have in mind the following analogy. The natural numbers, as a
  monoid under addition, has exactly $1$ generator, while the natural
  numbers, as a monoid under multiplication, has countably many
  generators (the primes). We observe that the application of the
  lambda calculus is much less resource sensitive than the parallel
  composition of the $\pi$-calculus. Could it be the case that we have
  an analogy of the form
  
  \begin{mathpar}
    m + n : MN :: m*n : M|N
  \end{mathpar}

  giving a similar blow up in the set of ``primes''?  This is such a
  wonderful thought that, even if it's not true, i think it's worth
  writing down.
\end{remark}
 

\documentclass[12pt]{llncs}
%\documentclass{jktr}

\usepackage[pdftex]{hyperref}                   
\usepackage {listings}
\usepackage {mathpartir}
\usepackage{bcprules}
%\usepackage{listings}
                       
\usepackage{graphicx} 
%\usepackage[margins=2.5cm,nohead,nofoot]{geometry}
%\usepackage{geometry}
\usepackage{amsfonts}
\usepackage{amstext}
\usepackage{latexsym}
\usepackage{amssymb}
\usepackage{color}


%\include{myPreamble}
\include{qm2pi.local} 

%\ifpdf
%\usepackage[pdftex]{graphicx}
%\else
%\usepackage{graphicx}
%\fi

 % \ifpdf
%  \usepackage{pdfsync}
%  \if


%\title{Brief Article}
%\author{David F. Snyder}
%\author{L.G. Meredith}

%\address{Dept. of Math., Texas State University--San Marcos, San Marcos, TX 78666}
       
\pagestyle{empty}


\begin{document}

\lstset{language=[Objective]Caml,frame=shadowbox}

\input{qm2pi.front}

% section front matter (end)

\input{qm2pi.intro} 
 
% section introduction (end)

% \input{qm2pi.knotations} 

% section notation (end)

\input{qm2pi.process.calculi} 

% section concurrent_process_calculi_and_spatial_logics_ (end)
    
%\input{qm2pi.knots2pi} 

%\input{qm2pi.trefoil} 

%\input{qm2pi.mainthm} 

% subsection basic_interpretation (end)

%\input{qm2pi.rho.presentation} 
\subsection{The syntax and semantics of the notation system}\label{sub:the_syntax_and_semantics_of_the_notation_system} % (fold)

We now summarize a technical presentation of the calculus that
embodies our theory of dynamics. The typical presentation of such a
calculus follows the style of giving generators and relations on
them. The grammar, below, describing term constructors, freely
generates the set of processes, $\Proc$. This set is then quotiented
by a relation known as structural congruence and it is over this set
that the notion of dynamics is expressed. This presentation is
essentially that of \cite{MeredithR05} with the addition of
polyadicity and summation. For readability we have relegated some of
the technical subtleties to an appendix.

\subsubsection{Process grammar}\label{subsub:process_grammar}

\begin{mathpar}
  \inferrule* [lab=synchronization] {} {{M} \bc \pzero \;|\; x?F \;|\; x!C }
  \and
  \inferrule* [lab=abstraction] {} {{F} \bc (x)P}
  \and
  \inferrule* [lab=concretion] {} {{C} \bc \langle Q \rangle}
  \and
  \inferrule* [lab=process] {} {{P,Q} \bc M \;| \;P|Q \;|\; @{x}}
  \and
  \inferrule* [lab=name] {} {{x} \bc \quotep{P}}
\end{mathpar} 

Note that $\vec{x}$ (resp. $\vec{P}$) denotes a vector of names
(resp. processes) of length $|\vec{x}|$ (resp. $|\vec{P}|$). We adopt
the following useful abbreviations.

\begin{mathpar}
   x?(\vec{y}).P := x.(\vec{y})P \and  x\clift{\vec{P}} := x.\clift{\vec{P}}
   \and x!(y) := \lift{x}{\dropn{y}}
   \and \Pi_{i=0}^{n-1}P_i := P_0 | \ldots | P_{n-1}
\end{mathpar}

\subsubsection{Structural congruence}

\paragraph{Free and bound names and alpha-equivalence.} At the
core of structural equivalence is alpha-equivalence which identifies
process that are the same up to a change of variable. Formally, we
recognize the distinction between free and bound names. The free names
of a process, $\freenames{P}$, may be calculated recursively as
follows:

\begin{mathpar}
\freenames{\pzero} := \emptyset
  \and \\
  \freenames{x?(y).P} := \{ x \} \cup (\freenames{P} \setminus \{ y \})
  \and 
  \freenames{x!\langle P \rangle} := \{ x \} \cup \{ P \} 
  \and \\
  \freenames{P|Q} := \freenames{P} \cup \freenames{Q}
  \and \\
  \freenames{@{x}} := \{ x \}
\end{mathpar}

$\pi$
$\quotep{\pi}$

$\freenames{-} : \pi \to \mathcal{P}(\quotep{\pi})$

\begin{eqnarray*}
  \freenames{\pzero} & := & \emptyset \\
  \freenames{x?(y).P} & := & \{ x \} \cup (\freenames{P} \setminus \{ y \}) \\
  \freenames{x!\langle P \rangle} & := & \{ x \} \cup \{ P \} \\
  \freenames{P|Q} & := & \freenames{P} \cup \freenames{Q} \\
  \freenames{\dropn{x}} & := & \{ x \}
\end{eqnarray*}

The bound names of a process, $\boundnames{P}$, are those names occurring in $P$
that are not free. For example, in $x?(y).0$, the name $x$ is free, while $y$ is bound.

\begin{mathpar}
  \inferrule* [lab=monoidal-laws] {} { P|Q \equiv Q|P \and P|0 \equiv P \and P|(Q|R) \equiv (P|Q)|R }
\end{mathpar}

\begin{mathpar}
  \inferrule* [lab=alpha-equivalence] {} { (x)P \equiv (y)P\{y/x\} \and y \not\in \freenames{P} }
\end{mathpar}

\begin{definition}
Then two processes, $P,Q$, are alpha-equivalent if $P = Q\{\vec{y}/\vec{x}\}$ for
some $\vec{x} \in \boundnames{Q},\vec{y} \in \boundnames{P}$, where $Q\{\vec{y}/\vec{x}\}$
denotes the capture-avoiding substitution of $\vec{y}$ for $\vec{x}$ in $Q$.
\end{definition}

\begin{definition}
  The {\em structural congruence} \cite{SangiorgiWalker} , $\equiv$,
  between processes is the least congruence containing
  alpha-equivalence, satisfying the abelian monoid laws
  (associativity, commutativity and $\pzero$ as identity) for parallel
  composition $|$ and for summation $+$.
\end{definition}

\subsection{Name equivalence}

We take name equivalence, written $\nameeq$, to be the smallest
equivalence relation generated by the following rules.

\begin{mathpar}
\inferrule*[lab=Quote-drop]
{ }
{ \quotep{@{x}} \nameeq x }

\inferrule*[lab=Struct-equiv]
{ P \scong Q }
{ \quotep{P} \nameeq \quotep{Q} }
\end{mathpar}

The astute reader will have noticed that the mutual recursion of names
and processes imposes a mutual recursion on alpha-equivalence and
structural equivalence via name-equivalence. Fortunately, all of this
works out pleasantly and we may calculate in the natural way, free of
concern. The reader interested in the details is referred to the
appendix \ref{appendix:rho_details}.

\subsection{Substitution}

We use $\Proc$ for the set of processes, $\QProc$ for the set of
names, and $\id{\{}\vec{y} / \vec{x} \id{\}}$ to denote partial maps,
$s : \QProc \rightarrow \QProc$. A map, $s$ lifts, uniquely, to a map
on process terms, $\widehat{s} : \Proc \rightarrow \Proc$ by the
following equations.

\begin{mathpar}
  (0) \psubstp{Q}{P} := 0 \\
  (R \juxtap S) \psubstp{Q}{P}
  :=    
  (R)\psubstp{Q}{P} \juxtap (S) \psubstp{Q}{P} \\
  (x?(y).R) \psubstp{Q}{P}    
  :=    
  (x)\substp{Q}{P} (z)\concat( (R \psubstn{z}{y}) \psubstp{Q}{P} ) \\
  (\lift{x}{R}) \psubstp{Q}{P}  
  :=
  \lift{(x)\substp{Q}{P}}{ R \psubstp{Q}{P} } \\
%   (\dropn{x})  \psubstp{Q}{P}       
%   := 
%   \left\{ 
%     \begin{array}{ccc} 
%       \dropn{\quotep{Q}} & & x \nameeq \quotep{P} \\
%       \dropn{x} & & otherwise \\
%     \end{array}
%   \right. 
  (\dropn{x})  \psubstp{Q}{P}       
  := 
  \left\{ 
    \begin{array}{ccc} 
      Q & & x \nameeq \quotep{P} \\
      \dropn{x} & & otherwise \\
    \end{array}
  \right.
\end{mathpar}
 

where

\begin{eqnarray}
  (x)\id{\{} \lpquote Q \rpquote / \lpquote P \rpquote \id{\}}            = 
  \left\{ 
    \begin{array}{ccc}
      \lpquote Q \rpquote & & x \nameeq \lpquote P \rpquote \\
      x & & otherwise \\
    \end{array}
  \right. \nonumber
\end{eqnarray}

and $z$ is chosen distinct from $\quotep{P}$, $\quotep{Q}$, the free
names in $Q$, and all the names in $R$. Our $\alpha$-equivalence will
be built in the standard way from this substitution.

\begin{remark}\label{rem:no_self_referential_names}
  One consequence of these definitions is that $\forall P. \quotep{P}
  \not\in \freenames{P}$.
\end{remark}

\subsection{ Dynamic quote: an example }

Anticipating something of what's to come, consider applying the
substitution, $\widehat{\id{\{}u / z \id{\}}}$, to the following pair
of processes, $\lift{w}{y!(z)}$ and $w[ \lpquote y!(z) \rpquote ]$.

\begin{eqnarray}
	\lift{w}{y!(z)}\widehat{\id{\{}u / z \id{\}}}
		& = &
		\lift{w}{y!(u)} \nonumber\\
	w[ \lpquote y!(z) \rpquote ] \widehat{ \id{\{}u / z \id{\}} }
		& = &
		w[ \lpquote y!(z) \rpquote ] \nonumber
\end{eqnarray}

Because the body of the process between quotes is impervious to
substitution, we get radically different answers. In fact, by
examining the first process in an input context,
e.g. $x?(z).\lift{w}{y!(z)}$, we see that the process under the lift
operator may be shaped by prefixed inputs binding a name inside it. In
this sense, the lift operator will be seen as a way to dynamically
construct processes before reifying them as names.

Finally equipped with these standard features we can present the
dynamics of the calculus.

\subsubsection{Operational semantics} 

Finally, we introduce the computational dynamics. What marks these
algebras as distinct from other more traditionally studied algebraic
structures, e.g. vector spaces or polynomial rings, is the manner in
which dynamics is captured. In traditional structures, dynamics is typically
expressed through morphisms between such structures, as in linear maps
between vector spaces or morphisms between rings. In algebras
associated with the semantics of computation, the dynamics is
expressed as part of the algebraic structure itself, through a
reduction reduction relation typically denoted by $\red$. Below, we
give a recursive presentation of this relation for the calculus used
in the encoding.

$\red \subseteq \pi \times \pi$
$\red : \pi \to \mathcal{P}(\pi)$

\begin{mathpar}
  \inferrule* [lab=Comm] { \textsf{match}( x_{src}, x_{trgt} ) } { x_{trgt}?(y)P \; | \; x_{src}!\langle {Q} \rangle \red P\{\quotep{Q}/y}\} }
  \and \\
  \inferrule* [lab=Par] {{P} \red {P}'} {{{P} | {Q}} \red {{P}' | {Q}}}
  \and
  \inferrule* [lab=Equiv]{{{P} \scong {P}'} \andalso {{P}' \red {Q}'} \andalso {{Q}' \scong {Q}}}{{P} \red {Q}}
\end{mathpar}

\begin{eqnarray*}
  match_{\equiv} (\quotep{P},\quotep{Q}) & := & P \equiv Q \\
  match_{\dagger}(\quotep{P},\quotep{Q}) & := & \forall R. P|Q \red^{*} R => R \red^{*} 0 \\
  match_{K}(\quotep{P},\quotep{Q}) & := & K \mbox{ for some context } K
\end{eqnarray*}

$u?(x)P | u!\langle Q \rangle \red P\{\quotep{Q}/x\}$

%We write $\wred$ for $\red^*$, and $P\red$ if $\exists Q $ such that $ P \red Q$.
We write $P\red$ if $\exists Q $ such that $ P \red Q$ and $P\not\red$, otherwise.

\section{Replication}

As mentioned before, it is known that replication (and hence
recursion) can be implemented in a higher-order process algebra
\cite{SangiorgiWalker}. As our first example of calculation with the
machinery thus far presented we give the construction explicitly in
the {\rhoc}.

\begin{eqnarray}
	D_{x} & := & \prefix{x}{y}{(\binpar{\outputp{x}{y}}{@{y}})} \nonumber\\
	\bangp_{x}{P} & := & \binpar{{x}!\langle{\binpar{D_{x}}{P}}\rangle}{D_{x}} \nonumber
\end{eqnarray}

\begin{eqnarray}
	\bangp_{x}{P} & & \nonumber\\
	=
	& {x}!\langle{(\prefix{x}{y}{(\outputp{x}{y} | @{y})) | P}}\rangle 
	      | \prefix{x}{y}{(\outputp{x}{y} | @{y})} & \nonumber\\
	\red
	& (\outputp{x}{y} | @{y})\substn{\quotep{(\prefix{x}{y}{(@{y} | \outputp{x}{y})) | P}}}{y} & \nonumber\\
	=
	& \outputp{x}{\quotep{(\prefix{x}{y}{(\outputp{x}{y} | @{y})) | P}}}
	  | {(\prefix{x}{y}{(\outputp{x}{y} | @{y})) | P}} & \nonumber\\
	\red
	& \ldots & \nonumber\\
	\red^*
	& P | P | \ldots & \nonumber
\end{eqnarray}

Of course, this encoding, as an implementation, runs away, unfolding
$\bangp{P}$ eagerly. A lazier and more implementable replication
operator, restricted to input-guarded processes, may be obtained as follows.

\begin{eqnarray}
\bangp{\prefix{u}{v}{P}} 
	:= 
	\binpar{\lift{x}{\prefix{u}{v}{(\binpar{D(x)}{P})}}}{D(x)} \nonumber
\end{eqnarray}

\begin{remark}
  Note that the lazier definition still does not deal with summation
  or mixed summation (i.e. sums over input and output). The reader is
  invited to construct definitions of replication that deal with these
  features. 

  Further, the definitions are parameterized in a name, $x$. Can you,
  gentle reader, make a definition that eliminates this parameter and
  guarantees no accidental interaction between the replication
  machinery and the process being replicated -- i.e. no accidental
  sharing of names used by the process to get its work done and the
  name(s) used by the replication to effect copying. This latter
  revision of the definition of replication is crucial to obtaining
  the expected identity $!!P \sim !P$.
\end{remark}

\begin{remark}\label{rem:paradoxical_combinator}
  The reader familiar with the lambda calculus will have noticed the
  similarity between $D$ and the paradoxical combinator.

  [Ed. note: the existence of this seems to suggest we have to be more
  restrictive on the set of processes and names we admit if we are to
  support no-cloning.]
\end{remark}

\subsubsection{Bisimulation}

The computational dynamics gives rise to another kind of equivalence,
the equivalence of computational behavior. As previously mentioned
this is typically captured \emph{via} some form of bisimulation.

% The notion we use in this paper is weak barbed bisimulation
% \cite{milner91polyadicpi}.

The notion we use in this paper is derived from weak barbed
bisimulation \cite{milner91polyadicpi}. 

\begin{definition}
An \emph{observation relation}, $\downarrow_{\mathcal N}$, over a set
of names, $\mathcal N$, is the smallest relation satisfying the rules
below.

\infrule[Out-barb]{y \in {\mathcal N}, \; x \nameeq y}
		  {\outputp{x}{v} \downarrow_{\mathcal N} x}
\infrule[Par-barb]{\mbox{$P\downarrow_{\mathcal N} x$ or $Q\downarrow_{\mathcal N} x$}}
		  {\binpar{P}{Q} \downarrow_{\mathcal N} x}

We write $P \Downarrow_{\mathcal N} x$ if there is $Q$ such that 
$P \wred Q$ and $Q \downarrow_{\mathcal N} x$.
\end{definition}

\begin{definition}
%\label{def.bbisim}
An  ${\mathcal N}$-\emph{barbed bisimulation} over a set of names, ${\mathcal N}$, is a symmetric binary relation 
${\mathcal S}_{\mathcal N}$ between agents such that $P\rel{S}_{\mathcal N}Q$ implies:
\begin{enumerate}
\item If $P \red P'$ then $Q \wred Q'$ and $P'\rel{S}_{\mathcal N} Q'$.
\item If $P\downarrow_{\mathcal N} x$, then $Q\Downarrow_{\mathcal N} x$.
\end{enumerate}
$P$ is ${\mathcal N}$-barbed bisimilar to $Q$, written
$P \wbbisim_{\mathcal N} Q$, if $P \rel{S}_{\mathcal N} Q$ for some ${\mathcal N}$-barbed bisimulation ${\mathcal S}_{\mathcal N}$.
\end{definition}

$\mathcal{R} \subseteq \pi \times \pi$

$P \mathcal{R} Q => \forall P'. P \red P' \Rightarrow \exists Q'. Q \red Q', P' \mathcal{R} Q'$

$P \vdash x \Rightarrow Q \vdash x$

\begin{mathpar}
  \inferrule*[lab=Out-barb]{x \nameeq y}{{y}!\langle{Q}\rangle \vdash x}
  \and
  \inferrule*[lab=Par-barb]{\mbox{$P\vdash x$ or $Q\vdash x$}}{\binpar{P}{Q} \vdash x}
\end{mathpar}

\subsubsection{Contexts}

One of the principle advantages of computational calculi like the
$\pi$-calculus is a well-defined notion of context,
contextual-equivalence and a correlation between
contextual-equivalence and notions of bisimulation. The notion of
context allows the decomposition of a process into (sub-)process and
its syntactic environment, its context. Thus, a context may be
thought of as a process with a ``hole'' (written $\Box$) in it. The
application of a context $M$ to a process $P$, written $M[P]$, is
tantamount to filling the hole in $M$ with $P$. In this paper we do
not need the full weight of this theory, but do make use of the notion
of context in the proof the main theorem. 

\begin{mathpar}
  \inferrule* [lab=summation] {} {{M_{M},M_{N}} \bc \Box \;|\; x.M_{A} \;|\; M_{M}+M_{N}}
  \and
  \inferrule* [lab=agent] {} {{M_{A}} \bc (\vec{x})M_{P} \;| \; \clift{P_0,\ldots,M_{P},\ldots,P_N}}
  \and \\
  \inferrule* [lab=process] {} {{M_{P}} \bc M_{N} \;| \;P|M_{P} }
\end{mathpar} 

\begin{mathpar}
  \inferrule* [lab=sychronization] {} {M_{N} \bc \Box \;|\; x?M_{F} \;|\; x!M_{C}}
  \and
  \inferrule* [lab=abstraction] {} {{M_{F}} \bc (x)M_{P} }
  \and
  \inferrule* [lab=concretion] {} {{M_{C}} \bc \langle M_{P} \rangle }
  \and \\
  \inferrule* [lab=process] {} {{M_{P}} \bc M_{N} \;| \;P|M_{P} }
\end{mathpar}

\begin{definition}[contextual application] Given a context $M$, and
  process $P$, we define the \emph{contextual application}, $M[P] :=
  M\{P/\Box\}$. That is, the contextual application of M to P is the
  substitution of $P$ for $\Box$ in $M$.
\end{definition}

$\meaningof{-} : L \to \mathcal{P}(\pi)$

\begin{mathpar}
  \inferrule* [lab=collection] {} {\meaningof{true} = \pi, \and \meaningof{~E} = \pi \setminus \meaningof{E}, \and \meaningof{E_{1} \& E_{2}} = \meaningof{E_{1}} \cap \meaningof{E_{2}}}
\end{mathpar}

\begin{mathpar}
  \inferrule* [lab=structure] {} {\meaningof{0} = \{ P \in \pi | P \equiv 0 \}, \and \\ \meaningof{E_1 | E_2} = \{ P \in \pi | P \equiv P_{1} | P_{2}, P_{1} \in \meaningof{E_{1}}, P_{2} \in \meaningof{E_2}\} }
\end{mathpar}

\begin{mathpar}
 \inferrule* [lab=behavior] {} {\meaningof{\langle a?b \rangle E} = \{ P \in \pi | P \equiv Q | u?(y)P', \\ \and \\\\ \and \\ \;\;\; u \in \meaningof{a}, \forall z.P'\{z/y\} \in \meaningof{E\{z/b\}}\}, \and \\ \meaningof{a!E} = \{ P \in \pi | P \equiv Q | x!\langle P' \rangle, x \in \meaningof{a} P' \in \meaningof{E}\} }
\end{mathpar}

\begin{mathpar}
 \inferrule* [lab=nominal] {} {\meaningof{\quotep{E}} = \{ \quotep{P} \in \quotep{\pi} | P \in \meaningof{E} \}, \and \meaningof{\quotep{P}} = \{ \quotep{Q} \in \quotep{\pi} | P \equiv Q \} \and \\ \meaningof{@\quotep{E}} = \{ P \in \pi | P \equiv @x, x \in \meaningof{E} \}}
\end{mathpar}

\begin{eqnarray*}
  \\
  \meaningof{-} : TS \to ST
\end{eqnarray*}

\begin{eqnarray*}
  \\
  L : TS \to ST
\end{eqnarray*}

\begin{eqnarray*}
  \\
  P \models E \iff P \in \meaningof{E}
\end{eqnarray*}

\begin{eqnarray*}
  P \approx_{L} Q \iff \forall E \in L. P \models E \iff Q \models E
\end{eqnarray*}

\begin{eqnarray*}
  P \approx_{K} Q
\end{eqnarray*}

\begin{eqnarray*}
  P \approx Q
\end{eqnarray*}

$\approx_{K} = \approx = \approx_{L}$

\subsubsection{Contextual duality}

Note that contexts extend the quotation operation to a family of
operations from processes to names. Given a context, $M$, we can
define a \emph{nominal context}, $\quotep{M}$ by $\quotep{M}[P] :=
\quotep{M[P]}$. To foreshadow what is to come we observe that these
operations enjoy a duality with processes very much like the duality
between vectors and maps from vectors to scalars.

Further, because the calculus is essentially higher-order, we have a
correspondence between contexts and processes. More specifically,
given a name $x$ and a context $M$ we can construct $M^{*}_{x}$ such
that 

\begin{mathpar}
  M^{*}_{x} | \lift{x}{P} \red M[P]
\end{mathpar}

namely,

\begin{mathpar}
  M^{*}_{x} := x?(u).M[\dropn{u}]
\end{mathpar}

The dependence of $M^{*}_{x}$ on a name makes it an abstraction, 

\begin{mathpar}
  M^{*} := (x)x?(u).M[\dropn{u}]
\end{mathpar}

\subsection{Additional notation}

It will sometimes be convenient to denote the process a name
quotes. We already have the notation $x = \quotep{P}$, but it will be
convenient to introduce an alternate notation, $\procn{x}$, when we
want to emphasize the connection to the use of the name. Note that, by
virtue of name equivalence, $\quotep{\procn{x}} \nameeq x$; so, the
notation is consistent with previous definitions.

Further, because names have structure it is possible to effect
substitutions on the basis of that structure. This means we need to
upgrade our notation for substitutions, which we accomplish by
adapting comprehension notation. Thus,

\begin{mathpar}
  P\{ y / x : x \in S \}
\end{mathpar}

is interpreted to mean the process derived from P by replacing (in a
capture-avoiding manner) each occurrence of $x$ in $S$ by $y$. For example,

\begin{mathpar}
  P\{ \quotep{\procn{x}|\procn{x}} / x : x \in \freenames{P} \}
\end{mathpar}

will replace each (occurrence) of a free name $x$ in $P$ by
$\quotep{\procn{x}|\procn{x}}$.

Also, we will avail ourselves of the notation $x^{L}$ and $x^{R}$ to
denote injections of a name into disjoint copies of the name
space. There are numerous ways to accomplish this. One example can be
found in \cite{MeredithR05}. This notation overloads to vectors of
names: $\vec{x}^{\pi} := (x_{i}^{\pi} \; : \; 0 \leq i < |\vec{x}| )$ where $\pi \in \{L,R\}$.

We also use $P^{\Box} := P|\Box$.

In \cite{MeredithR05} an interpretation of the new operator is
given. It turns out that there are several possible interpretations
all enjoying the requisite algebraic properties of the operator (see
\cite{milner91polyadicpi}). We will therefore make liberal use of
$(\nu\; \vec{x})P$.

% subsection the_syntax_and_semantics_of_the_notation_system (end)   

\input{qm2pi.qmops} 

\input{qm2pi.sterngerlach} 

\input{qm2pi.metric} 

% section concurrent_process_calculi (end)

%\input{qm2pi.proofsketch}

% section proof sketch (end)

%\input{qm2pi.slviaknots} 

% section spatial logic via knots (end)

\input{qm2pi.conclusion}

% section conclusion (end)

%\input{qm2pi.dtcodes} 

% section wiring algorithm (end)

\input{qm2pi.ack} 

% section acknowledgments (end)

\newpage


\bibliographystyle{plain}   
\bibliography{../../biblios/main.bib}

\input{qm2pi.rhodetails}

\end{document}

 

\documentclass[12pt]{llncs}
%\documentclass{jktr}

\usepackage[pdftex]{hyperref}                   
\usepackage {listings}
\usepackage {mathpartir}
\usepackage{bcprules}
%\usepackage{listings}
                       
\usepackage{graphicx} 
%\usepackage[margins=2.5cm,nohead,nofoot]{geometry}
%\usepackage{geometry}
\usepackage{amsfonts}
\usepackage{amstext}
\usepackage{latexsym}
\usepackage{amssymb}
\usepackage{color}


%\include{myPreamble}
\include{qm2pi.local} 

%\ifpdf
%\usepackage[pdftex]{graphicx}
%\else
%\usepackage{graphicx}
%\fi

 % \ifpdf
%  \usepackage{pdfsync}
%  \if


%\title{Brief Article}
%\author{David F. Snyder}
%\author{L.G. Meredith}

%\address{Dept. of Math., Texas State University--San Marcos, San Marcos, TX 78666}
       
\pagestyle{empty}


\begin{document}

\lstset{language=[Objective]Caml,frame=shadowbox}

\input{qm2pi.front}

% section front matter (end)

\input{qm2pi.intro} 
 
% section introduction (end)

% \input{qm2pi.knotations} 

% section notation (end)

\input{qm2pi.process.calculi} 

% section concurrent_process_calculi_and_spatial_logics_ (end)
    
%\input{qm2pi.knots2pi} 

%\input{qm2pi.trefoil} 

%\input{qm2pi.mainthm} 

% subsection basic_interpretation (end)

%\input{qm2pi.rho.presentation} 
\subsection{The syntax and semantics of the notation system}\label{sub:the_syntax_and_semantics_of_the_notation_system} % (fold)

We now summarize a technical presentation of the calculus that
embodies our theory of dynamics. The typical presentation of such a
calculus follows the style of giving generators and relations on
them. The grammar, below, describing term constructors, freely
generates the set of processes, $\Proc$. This set is then quotiented
by a relation known as structural congruence and it is over this set
that the notion of dynamics is expressed. This presentation is
essentially that of \cite{MeredithR05} with the addition of
polyadicity and summation. For readability we have relegated some of
the technical subtleties to an appendix.

\subsubsection{Process grammar}\label{subsub:process_grammar}

\begin{mathpar}
  \inferrule* [lab=synchronization] {} {{M} \bc \pzero \;|\; x?F \;|\; x!C }
  \and
  \inferrule* [lab=abstraction] {} {{F} \bc (x)P}
  \and
  \inferrule* [lab=concretion] {} {{C} \bc \langle Q \rangle}
  \and
  \inferrule* [lab=process] {} {{P,Q} \bc M \;| \;P|Q \;|\; @{x}}
  \and
  \inferrule* [lab=name] {} {{x} \bc \quotep{P}}
\end{mathpar} 

Note that $\vec{x}$ (resp. $\vec{P}$) denotes a vector of names
(resp. processes) of length $|\vec{x}|$ (resp. $|\vec{P}|$). We adopt
the following useful abbreviations.

\begin{mathpar}
   x?(\vec{y}).P := x.(\vec{y})P \and  x\clift{\vec{P}} := x.\clift{\vec{P}}
   \and x!(y) := \lift{x}{\dropn{y}}
   \and \Pi_{i=0}^{n-1}P_i := P_0 | \ldots | P_{n-1}
\end{mathpar}

\subsubsection{Structural congruence}

\paragraph{Free and bound names and alpha-equivalence.} At the
core of structural equivalence is alpha-equivalence which identifies
process that are the same up to a change of variable. Formally, we
recognize the distinction between free and bound names. The free names
of a process, $\freenames{P}$, may be calculated recursively as
follows:

\begin{mathpar}
\freenames{\pzero} := \emptyset
  \and \\
  \freenames{x?(y).P} := \{ x \} \cup (\freenames{P} \setminus \{ y \})
  \and 
  \freenames{x!\langle P \rangle} := \{ x \} \cup \{ P \} 
  \and \\
  \freenames{P|Q} := \freenames{P} \cup \freenames{Q}
  \and \\
  \freenames{@{x}} := \{ x \}
\end{mathpar}

$\pi$
$\quotep{\pi}$

$\freenames{-} : \pi \to \mathcal{P}(\quotep{\pi})$

\begin{eqnarray*}
  \freenames{\pzero} & := & \emptyset \\
  \freenames{x?(y).P} & := & \{ x \} \cup (\freenames{P} \setminus \{ y \}) \\
  \freenames{x!\langle P \rangle} & := & \{ x \} \cup \{ P \} \\
  \freenames{P|Q} & := & \freenames{P} \cup \freenames{Q} \\
  \freenames{\dropn{x}} & := & \{ x \}
\end{eqnarray*}

The bound names of a process, $\boundnames{P}$, are those names occurring in $P$
that are not free. For example, in $x?(y).0$, the name $x$ is free, while $y$ is bound.

\begin{mathpar}
  \inferrule* [lab=monoidal-laws] {} { P|Q \equiv Q|P \and P|0 \equiv P \and P|(Q|R) \equiv (P|Q)|R }
\end{mathpar}

\begin{mathpar}
  \inferrule* [lab=alpha-equivalence] {} { (x)P \equiv (y)P\{y/x\} \and y \not\in \freenames{P} }
\end{mathpar}

\begin{definition}
Then two processes, $P,Q$, are alpha-equivalent if $P = Q\{\vec{y}/\vec{x}\}$ for
some $\vec{x} \in \boundnames{Q},\vec{y} \in \boundnames{P}$, where $Q\{\vec{y}/\vec{x}\}$
denotes the capture-avoiding substitution of $\vec{y}$ for $\vec{x}$ in $Q$.
\end{definition}

\begin{definition}
  The {\em structural congruence} \cite{SangiorgiWalker} , $\equiv$,
  between processes is the least congruence containing
  alpha-equivalence, satisfying the abelian monoid laws
  (associativity, commutativity and $\pzero$ as identity) for parallel
  composition $|$ and for summation $+$.
\end{definition}

\subsection{Name equivalence}

We take name equivalence, written $\nameeq$, to be the smallest
equivalence relation generated by the following rules.

\begin{mathpar}
\inferrule*[lab=Quote-drop]
{ }
{ \quotep{@{x}} \nameeq x }

\inferrule*[lab=Struct-equiv]
{ P \scong Q }
{ \quotep{P} \nameeq \quotep{Q} }
\end{mathpar}

The astute reader will have noticed that the mutual recursion of names
and processes imposes a mutual recursion on alpha-equivalence and
structural equivalence via name-equivalence. Fortunately, all of this
works out pleasantly and we may calculate in the natural way, free of
concern. The reader interested in the details is referred to the
appendix \ref{appendix:rho_details}.

\subsection{Substitution}

We use $\Proc$ for the set of processes, $\QProc$ for the set of
names, and $\id{\{}\vec{y} / \vec{x} \id{\}}$ to denote partial maps,
$s : \QProc \rightarrow \QProc$. A map, $s$ lifts, uniquely, to a map
on process terms, $\widehat{s} : \Proc \rightarrow \Proc$ by the
following equations.

\begin{mathpar}
  (0) \psubstp{Q}{P} := 0 \\
  (R \juxtap S) \psubstp{Q}{P}
  :=    
  (R)\psubstp{Q}{P} \juxtap (S) \psubstp{Q}{P} \\
  (x?(y).R) \psubstp{Q}{P}    
  :=    
  (x)\substp{Q}{P} (z)\concat( (R \psubstn{z}{y}) \psubstp{Q}{P} ) \\
  (\lift{x}{R}) \psubstp{Q}{P}  
  :=
  \lift{(x)\substp{Q}{P}}{ R \psubstp{Q}{P} } \\
%   (\dropn{x})  \psubstp{Q}{P}       
%   := 
%   \left\{ 
%     \begin{array}{ccc} 
%       \dropn{\quotep{Q}} & & x \nameeq \quotep{P} \\
%       \dropn{x} & & otherwise \\
%     \end{array}
%   \right. 
  (\dropn{x})  \psubstp{Q}{P}       
  := 
  \left\{ 
    \begin{array}{ccc} 
      Q & & x \nameeq \quotep{P} \\
      \dropn{x} & & otherwise \\
    \end{array}
  \right.
\end{mathpar}
 

where

\begin{eqnarray}
  (x)\id{\{} \lpquote Q \rpquote / \lpquote P \rpquote \id{\}}            = 
  \left\{ 
    \begin{array}{ccc}
      \lpquote Q \rpquote & & x \nameeq \lpquote P \rpquote \\
      x & & otherwise \\
    \end{array}
  \right. \nonumber
\end{eqnarray}

and $z$ is chosen distinct from $\quotep{P}$, $\quotep{Q}$, the free
names in $Q$, and all the names in $R$. Our $\alpha$-equivalence will
be built in the standard way from this substitution.

\begin{remark}\label{rem:no_self_referential_names}
  One consequence of these definitions is that $\forall P. \quotep{P}
  \not\in \freenames{P}$.
\end{remark}

\subsection{ Dynamic quote: an example }

Anticipating something of what's to come, consider applying the
substitution, $\widehat{\id{\{}u / z \id{\}}}$, to the following pair
of processes, $\lift{w}{y!(z)}$ and $w[ \lpquote y!(z) \rpquote ]$.

\begin{eqnarray}
	\lift{w}{y!(z)}\widehat{\id{\{}u / z \id{\}}}
		& = &
		\lift{w}{y!(u)} \nonumber\\
	w[ \lpquote y!(z) \rpquote ] \widehat{ \id{\{}u / z \id{\}} }
		& = &
		w[ \lpquote y!(z) \rpquote ] \nonumber
\end{eqnarray}

Because the body of the process between quotes is impervious to
substitution, we get radically different answers. In fact, by
examining the first process in an input context,
e.g. $x?(z).\lift{w}{y!(z)}$, we see that the process under the lift
operator may be shaped by prefixed inputs binding a name inside it. In
this sense, the lift operator will be seen as a way to dynamically
construct processes before reifying them as names.

Finally equipped with these standard features we can present the
dynamics of the calculus.

\subsubsection{Operational semantics} 

Finally, we introduce the computational dynamics. What marks these
algebras as distinct from other more traditionally studied algebraic
structures, e.g. vector spaces or polynomial rings, is the manner in
which dynamics is captured. In traditional structures, dynamics is typically
expressed through morphisms between such structures, as in linear maps
between vector spaces or morphisms between rings. In algebras
associated with the semantics of computation, the dynamics is
expressed as part of the algebraic structure itself, through a
reduction reduction relation typically denoted by $\red$. Below, we
give a recursive presentation of this relation for the calculus used
in the encoding.

$\red \subseteq \pi \times \pi$
$\red : \pi \to \mathcal{P}(\pi)$

\begin{mathpar}
  \inferrule* [lab=Comm] { \textsf{match}( x_{src}, x_{trgt} ) } { x_{trgt}?(y)P \; | \; x_{src}!\langle {Q} \rangle \red P\{\quotep{Q}/y}\} }
  \and \\
  \inferrule* [lab=Par] {{P} \red {P}'} {{{P} | {Q}} \red {{P}' | {Q}}}
  \and
  \inferrule* [lab=Equiv]{{{P} \scong {P}'} \andalso {{P}' \red {Q}'} \andalso {{Q}' \scong {Q}}}{{P} \red {Q}}
\end{mathpar}

\begin{eqnarray*}
  match_{\equiv} (\quotep{P},\quotep{Q}) & := & P \equiv Q \\
  match_{\dagger}(\quotep{P},\quotep{Q}) & := & \forall R. P|Q \red^{*} R => R \red^{*} 0 \\
  match_{K}(\quotep{P},\quotep{Q}) & := & K \mbox{ for some context } K
\end{eqnarray*}

$u?(x)P | u!\langle Q \rangle \red P\{\quotep{Q}/x\}$

%We write $\wred$ for $\red^*$, and $P\red$ if $\exists Q $ such that $ P \red Q$.
We write $P\red$ if $\exists Q $ such that $ P \red Q$ and $P\not\red$, otherwise.

\section{Replication}

As mentioned before, it is known that replication (and hence
recursion) can be implemented in a higher-order process algebra
\cite{SangiorgiWalker}. As our first example of calculation with the
machinery thus far presented we give the construction explicitly in
the {\rhoc}.

\begin{eqnarray}
	D_{x} & := & \prefix{x}{y}{(\binpar{\outputp{x}{y}}{@{y}})} \nonumber\\
	\bangp_{x}{P} & := & \binpar{{x}!\langle{\binpar{D_{x}}{P}}\rangle}{D_{x}} \nonumber
\end{eqnarray}

\begin{eqnarray}
	\bangp_{x}{P} & & \nonumber\\
	=
	& {x}!\langle{(\prefix{x}{y}{(\outputp{x}{y} | @{y})) | P}}\rangle 
	      | \prefix{x}{y}{(\outputp{x}{y} | @{y})} & \nonumber\\
	\red
	& (\outputp{x}{y} | @{y})\substn{\quotep{(\prefix{x}{y}{(@{y} | \outputp{x}{y})) | P}}}{y} & \nonumber\\
	=
	& \outputp{x}{\quotep{(\prefix{x}{y}{(\outputp{x}{y} | @{y})) | P}}}
	  | {(\prefix{x}{y}{(\outputp{x}{y} | @{y})) | P}} & \nonumber\\
	\red
	& \ldots & \nonumber\\
	\red^*
	& P | P | \ldots & \nonumber
\end{eqnarray}

Of course, this encoding, as an implementation, runs away, unfolding
$\bangp{P}$ eagerly. A lazier and more implementable replication
operator, restricted to input-guarded processes, may be obtained as follows.

\begin{eqnarray}
\bangp{\prefix{u}{v}{P}} 
	:= 
	\binpar{\lift{x}{\prefix{u}{v}{(\binpar{D(x)}{P})}}}{D(x)} \nonumber
\end{eqnarray}

\begin{remark}
  Note that the lazier definition still does not deal with summation
  or mixed summation (i.e. sums over input and output). The reader is
  invited to construct definitions of replication that deal with these
  features. 

  Further, the definitions are parameterized in a name, $x$. Can you,
  gentle reader, make a definition that eliminates this parameter and
  guarantees no accidental interaction between the replication
  machinery and the process being replicated -- i.e. no accidental
  sharing of names used by the process to get its work done and the
  name(s) used by the replication to effect copying. This latter
  revision of the definition of replication is crucial to obtaining
  the expected identity $!!P \sim !P$.
\end{remark}

\begin{remark}\label{rem:paradoxical_combinator}
  The reader familiar with the lambda calculus will have noticed the
  similarity between $D$ and the paradoxical combinator.

  [Ed. note: the existence of this seems to suggest we have to be more
  restrictive on the set of processes and names we admit if we are to
  support no-cloning.]
\end{remark}

\subsubsection{Bisimulation}

The computational dynamics gives rise to another kind of equivalence,
the equivalence of computational behavior. As previously mentioned
this is typically captured \emph{via} some form of bisimulation.

% The notion we use in this paper is weak barbed bisimulation
% \cite{milner91polyadicpi}.

The notion we use in this paper is derived from weak barbed
bisimulation \cite{milner91polyadicpi}. 

\begin{definition}
An \emph{observation relation}, $\downarrow_{\mathcal N}$, over a set
of names, $\mathcal N$, is the smallest relation satisfying the rules
below.

\infrule[Out-barb]{y \in {\mathcal N}, \; x \nameeq y}
		  {\outputp{x}{v} \downarrow_{\mathcal N} x}
\infrule[Par-barb]{\mbox{$P\downarrow_{\mathcal N} x$ or $Q\downarrow_{\mathcal N} x$}}
		  {\binpar{P}{Q} \downarrow_{\mathcal N} x}

We write $P \Downarrow_{\mathcal N} x$ if there is $Q$ such that 
$P \wred Q$ and $Q \downarrow_{\mathcal N} x$.
\end{definition}

\begin{definition}
%\label{def.bbisim}
An  ${\mathcal N}$-\emph{barbed bisimulation} over a set of names, ${\mathcal N}$, is a symmetric binary relation 
${\mathcal S}_{\mathcal N}$ between agents such that $P\rel{S}_{\mathcal N}Q$ implies:
\begin{enumerate}
\item If $P \red P'$ then $Q \wred Q'$ and $P'\rel{S}_{\mathcal N} Q'$.
\item If $P\downarrow_{\mathcal N} x$, then $Q\Downarrow_{\mathcal N} x$.
\end{enumerate}
$P$ is ${\mathcal N}$-barbed bisimilar to $Q$, written
$P \wbbisim_{\mathcal N} Q$, if $P \rel{S}_{\mathcal N} Q$ for some ${\mathcal N}$-barbed bisimulation ${\mathcal S}_{\mathcal N}$.
\end{definition}

$\mathcal{R} \subseteq \pi \times \pi$

$P \mathcal{R} Q => \forall P'. P \red P' \Rightarrow \exists Q'. Q \red Q', P' \mathcal{R} Q'$

$P \vdash x \Rightarrow Q \vdash x$

\begin{mathpar}
  \inferrule*[lab=Out-barb]{x \nameeq y}{{y}!\langle{Q}\rangle \vdash x}
  \and
  \inferrule*[lab=Par-barb]{\mbox{$P\vdash x$ or $Q\vdash x$}}{\binpar{P}{Q} \vdash x}
\end{mathpar}

\subsubsection{Contexts}

One of the principle advantages of computational calculi like the
$\pi$-calculus is a well-defined notion of context,
contextual-equivalence and a correlation between
contextual-equivalence and notions of bisimulation. The notion of
context allows the decomposition of a process into (sub-)process and
its syntactic environment, its context. Thus, a context may be
thought of as a process with a ``hole'' (written $\Box$) in it. The
application of a context $M$ to a process $P$, written $M[P]$, is
tantamount to filling the hole in $M$ with $P$. In this paper we do
not need the full weight of this theory, but do make use of the notion
of context in the proof the main theorem. 

\begin{mathpar}
  \inferrule* [lab=summation] {} {{M_{M},M_{N}} \bc \Box \;|\; x.M_{A} \;|\; M_{M}+M_{N}}
  \and
  \inferrule* [lab=agent] {} {{M_{A}} \bc (\vec{x})M_{P} \;| \; \clift{P_0,\ldots,M_{P},\ldots,P_N}}
  \and \\
  \inferrule* [lab=process] {} {{M_{P}} \bc M_{N} \;| \;P|M_{P} }
\end{mathpar} 

\begin{mathpar}
  \inferrule* [lab=sychronization] {} {M_{N} \bc \Box \;|\; x?M_{F} \;|\; x!M_{C}}
  \and
  \inferrule* [lab=abstraction] {} {{M_{F}} \bc (x)M_{P} }
  \and
  \inferrule* [lab=concretion] {} {{M_{C}} \bc \langle M_{P} \rangle }
  \and \\
  \inferrule* [lab=process] {} {{M_{P}} \bc M_{N} \;| \;P|M_{P} }
\end{mathpar}

\begin{definition}[contextual application] Given a context $M$, and
  process $P$, we define the \emph{contextual application}, $M[P] :=
  M\{P/\Box\}$. That is, the contextual application of M to P is the
  substitution of $P$ for $\Box$ in $M$.
\end{definition}

$\meaningof{-} : L \to \mathcal{P}(\pi)$

\begin{mathpar}
  \inferrule* [lab=collection] {} {\meaningof{true} = \pi, \and \meaningof{~E} = \pi \setminus \meaningof{E}, \and \meaningof{E_{1} \& E_{2}} = \meaningof{E_{1}} \cap \meaningof{E_{2}}}
\end{mathpar}

\begin{mathpar}
  \inferrule* [lab=structure] {} {\meaningof{0} = \{ P \in \pi | P \equiv 0 \}, \and \\ \meaningof{E_1 | E_2} = \{ P \in \pi | P \equiv P_{1} | P_{2}, P_{1} \in \meaningof{E_{1}}, P_{2} \in \meaningof{E_2}\} }
\end{mathpar}

\begin{mathpar}
 \inferrule* [lab=behavior] {} {\meaningof{\langle a?b \rangle E} = \{ P \in \pi | P \equiv Q | u?(y)P', \\ \and \\\\ \and \\ \;\;\; u \in \meaningof{a}, \forall z.P'\{z/y\} \in \meaningof{E\{z/b\}}\}, \and \\ \meaningof{a!E} = \{ P \in \pi | P \equiv Q | x!\langle P' \rangle, x \in \meaningof{a} P' \in \meaningof{E}\} }
\end{mathpar}

\begin{mathpar}
 \inferrule* [lab=nominal] {} {\meaningof{\quotep{E}} = \{ \quotep{P} \in \quotep{\pi} | P \in \meaningof{E} \}, \and \meaningof{\quotep{P}} = \{ \quotep{Q} \in \quotep{\pi} | P \equiv Q \} \and \\ \meaningof{@\quotep{E}} = \{ P \in \pi | P \equiv @x, x \in \meaningof{E} \}}
\end{mathpar}

\begin{eqnarray*}
  \\
  \meaningof{-} : TS \to ST
\end{eqnarray*}

\begin{eqnarray*}
  \\
  L : TS \to ST
\end{eqnarray*}

\begin{eqnarray*}
  \\
  P \models E \iff P \in \meaningof{E}
\end{eqnarray*}

\begin{eqnarray*}
  P \approx_{L} Q \iff \forall E \in L. P \models E \iff Q \models E
\end{eqnarray*}

\begin{eqnarray*}
  P \approx_{K} Q
\end{eqnarray*}

\begin{eqnarray*}
  P \approx Q
\end{eqnarray*}

$\approx_{K} = \approx = \approx_{L}$

\subsubsection{Contextual duality}

Note that contexts extend the quotation operation to a family of
operations from processes to names. Given a context, $M$, we can
define a \emph{nominal context}, $\quotep{M}$ by $\quotep{M}[P] :=
\quotep{M[P]}$. To foreshadow what is to come we observe that these
operations enjoy a duality with processes very much like the duality
between vectors and maps from vectors to scalars.

Further, because the calculus is essentially higher-order, we have a
correspondence between contexts and processes. More specifically,
given a name $x$ and a context $M$ we can construct $M^{*}_{x}$ such
that 

\begin{mathpar}
  M^{*}_{x} | \lift{x}{P} \red M[P]
\end{mathpar}

namely,

\begin{mathpar}
  M^{*}_{x} := x?(u).M[\dropn{u}]
\end{mathpar}

The dependence of $M^{*}_{x}$ on a name makes it an abstraction, 

\begin{mathpar}
  M^{*} := (x)x?(u).M[\dropn{u}]
\end{mathpar}

\subsection{Additional notation}

It will sometimes be convenient to denote the process a name
quotes. We already have the notation $x = \quotep{P}$, but it will be
convenient to introduce an alternate notation, $\procn{x}$, when we
want to emphasize the connection to the use of the name. Note that, by
virtue of name equivalence, $\quotep{\procn{x}} \nameeq x$; so, the
notation is consistent with previous definitions.

Further, because names have structure it is possible to effect
substitutions on the basis of that structure. This means we need to
upgrade our notation for substitutions, which we accomplish by
adapting comprehension notation. Thus,

\begin{mathpar}
  P\{ y / x : x \in S \}
\end{mathpar}

is interpreted to mean the process derived from P by replacing (in a
capture-avoiding manner) each occurrence of $x$ in $S$ by $y$. For example,

\begin{mathpar}
  P\{ \quotep{\procn{x}|\procn{x}} / x : x \in \freenames{P} \}
\end{mathpar}

will replace each (occurrence) of a free name $x$ in $P$ by
$\quotep{\procn{x}|\procn{x}}$.

Also, we will avail ourselves of the notation $x^{L}$ and $x^{R}$ to
denote injections of a name into disjoint copies of the name
space. There are numerous ways to accomplish this. One example can be
found in \cite{MeredithR05}. This notation overloads to vectors of
names: $\vec{x}^{\pi} := (x_{i}^{\pi} \; : \; 0 \leq i < |\vec{x}| )$ where $\pi \in \{L,R\}$.

We also use $P^{\Box} := P|\Box$.

In \cite{MeredithR05} an interpretation of the new operator is
given. It turns out that there are several possible interpretations
all enjoying the requisite algebraic properties of the operator (see
\cite{milner91polyadicpi}). We will therefore make liberal use of
$(\nu\; \vec{x})P$.

% subsection the_syntax_and_semantics_of_the_notation_system (end)   

\input{qm2pi.qmops} 

\input{qm2pi.sterngerlach} 

\input{qm2pi.metric} 

% section concurrent_process_calculi (end)

%\input{qm2pi.proofsketch}

% section proof sketch (end)

%\input{qm2pi.slviaknots} 

% section spatial logic via knots (end)

\input{qm2pi.conclusion}

% section conclusion (end)

%\input{qm2pi.dtcodes} 

% section wiring algorithm (end)

\input{qm2pi.ack} 

% section acknowledgments (end)

\newpage


\bibliographystyle{plain}   
\bibliography{../../biblios/main.bib}

\input{qm2pi.rhodetails}

\end{document}

 

% section concurrent_process_calculi (end)

%\documentclass[12pt]{llncs}
%\documentclass{jktr}

\usepackage[pdftex]{hyperref}                   
\usepackage {listings}
\usepackage {mathpartir}
\usepackage{bcprules}
%\usepackage{listings}
                       
\usepackage{graphicx} 
%\usepackage[margins=2.5cm,nohead,nofoot]{geometry}
%\usepackage{geometry}
\usepackage{amsfonts}
\usepackage{amstext}
\usepackage{latexsym}
\usepackage{amssymb}
\usepackage{color}


%\include{myPreamble}
\include{qm2pi.local} 

%\ifpdf
%\usepackage[pdftex]{graphicx}
%\else
%\usepackage{graphicx}
%\fi

 % \ifpdf
%  \usepackage{pdfsync}
%  \if


%\title{Brief Article}
%\author{David F. Snyder}
%\author{L.G. Meredith}

%\address{Dept. of Math., Texas State University--San Marcos, San Marcos, TX 78666}
       
\pagestyle{empty}


\begin{document}

\lstset{language=[Objective]Caml,frame=shadowbox}

\input{qm2pi.front}

% section front matter (end)

\input{qm2pi.intro} 
 
% section introduction (end)

% \input{qm2pi.knotations} 

% section notation (end)

\input{qm2pi.process.calculi} 

% section concurrent_process_calculi_and_spatial_logics_ (end)
    
%\input{qm2pi.knots2pi} 

%\input{qm2pi.trefoil} 

%\input{qm2pi.mainthm} 

% subsection basic_interpretation (end)

%\input{qm2pi.rho.presentation} 
\subsection{The syntax and semantics of the notation system}\label{sub:the_syntax_and_semantics_of_the_notation_system} % (fold)

We now summarize a technical presentation of the calculus that
embodies our theory of dynamics. The typical presentation of such a
calculus follows the style of giving generators and relations on
them. The grammar, below, describing term constructors, freely
generates the set of processes, $\Proc$. This set is then quotiented
by a relation known as structural congruence and it is over this set
that the notion of dynamics is expressed. This presentation is
essentially that of \cite{MeredithR05} with the addition of
polyadicity and summation. For readability we have relegated some of
the technical subtleties to an appendix.

\subsubsection{Process grammar}\label{subsub:process_grammar}

\begin{mathpar}
  \inferrule* [lab=synchronization] {} {{M} \bc \pzero \;|\; x?F \;|\; x!C }
  \and
  \inferrule* [lab=abstraction] {} {{F} \bc (x)P}
  \and
  \inferrule* [lab=concretion] {} {{C} \bc \langle Q \rangle}
  \and
  \inferrule* [lab=process] {} {{P,Q} \bc M \;| \;P|Q \;|\; @{x}}
  \and
  \inferrule* [lab=name] {} {{x} \bc \quotep{P}}
\end{mathpar} 

Note that $\vec{x}$ (resp. $\vec{P}$) denotes a vector of names
(resp. processes) of length $|\vec{x}|$ (resp. $|\vec{P}|$). We adopt
the following useful abbreviations.

\begin{mathpar}
   x?(\vec{y}).P := x.(\vec{y})P \and  x\clift{\vec{P}} := x.\clift{\vec{P}}
   \and x!(y) := \lift{x}{\dropn{y}}
   \and \Pi_{i=0}^{n-1}P_i := P_0 | \ldots | P_{n-1}
\end{mathpar}

\subsubsection{Structural congruence}

\paragraph{Free and bound names and alpha-equivalence.} At the
core of structural equivalence is alpha-equivalence which identifies
process that are the same up to a change of variable. Formally, we
recognize the distinction between free and bound names. The free names
of a process, $\freenames{P}$, may be calculated recursively as
follows:

\begin{mathpar}
\freenames{\pzero} := \emptyset
  \and \\
  \freenames{x?(y).P} := \{ x \} \cup (\freenames{P} \setminus \{ y \})
  \and 
  \freenames{x!\langle P \rangle} := \{ x \} \cup \{ P \} 
  \and \\
  \freenames{P|Q} := \freenames{P} \cup \freenames{Q}
  \and \\
  \freenames{@{x}} := \{ x \}
\end{mathpar}

$\pi$
$\quotep{\pi}$

$\freenames{-} : \pi \to \mathcal{P}(\quotep{\pi})$

\begin{eqnarray*}
  \freenames{\pzero} & := & \emptyset \\
  \freenames{x?(y).P} & := & \{ x \} \cup (\freenames{P} \setminus \{ y \}) \\
  \freenames{x!\langle P \rangle} & := & \{ x \} \cup \{ P \} \\
  \freenames{P|Q} & := & \freenames{P} \cup \freenames{Q} \\
  \freenames{\dropn{x}} & := & \{ x \}
\end{eqnarray*}

The bound names of a process, $\boundnames{P}$, are those names occurring in $P$
that are not free. For example, in $x?(y).0$, the name $x$ is free, while $y$ is bound.

\begin{mathpar}
  \inferrule* [lab=monoidal-laws] {} { P|Q \equiv Q|P \and P|0 \equiv P \and P|(Q|R) \equiv (P|Q)|R }
\end{mathpar}

\begin{mathpar}
  \inferrule* [lab=alpha-equivalence] {} { (x)P \equiv (y)P\{y/x\} \and y \not\in \freenames{P} }
\end{mathpar}

\begin{definition}
Then two processes, $P,Q$, are alpha-equivalent if $P = Q\{\vec{y}/\vec{x}\}$ for
some $\vec{x} \in \boundnames{Q},\vec{y} \in \boundnames{P}$, where $Q\{\vec{y}/\vec{x}\}$
denotes the capture-avoiding substitution of $\vec{y}$ for $\vec{x}$ in $Q$.
\end{definition}

\begin{definition}
  The {\em structural congruence} \cite{SangiorgiWalker} , $\equiv$,
  between processes is the least congruence containing
  alpha-equivalence, satisfying the abelian monoid laws
  (associativity, commutativity and $\pzero$ as identity) for parallel
  composition $|$ and for summation $+$.
\end{definition}

\subsection{Name equivalence}

We take name equivalence, written $\nameeq$, to be the smallest
equivalence relation generated by the following rules.

\begin{mathpar}
\inferrule*[lab=Quote-drop]
{ }
{ \quotep{@{x}} \nameeq x }

\inferrule*[lab=Struct-equiv]
{ P \scong Q }
{ \quotep{P} \nameeq \quotep{Q} }
\end{mathpar}

The astute reader will have noticed that the mutual recursion of names
and processes imposes a mutual recursion on alpha-equivalence and
structural equivalence via name-equivalence. Fortunately, all of this
works out pleasantly and we may calculate in the natural way, free of
concern. The reader interested in the details is referred to the
appendix \ref{appendix:rho_details}.

\subsection{Substitution}

We use $\Proc$ for the set of processes, $\QProc$ for the set of
names, and $\id{\{}\vec{y} / \vec{x} \id{\}}$ to denote partial maps,
$s : \QProc \rightarrow \QProc$. A map, $s$ lifts, uniquely, to a map
on process terms, $\widehat{s} : \Proc \rightarrow \Proc$ by the
following equations.

\begin{mathpar}
  (0) \psubstp{Q}{P} := 0 \\
  (R \juxtap S) \psubstp{Q}{P}
  :=    
  (R)\psubstp{Q}{P} \juxtap (S) \psubstp{Q}{P} \\
  (x?(y).R) \psubstp{Q}{P}    
  :=    
  (x)\substp{Q}{P} (z)\concat( (R \psubstn{z}{y}) \psubstp{Q}{P} ) \\
  (\lift{x}{R}) \psubstp{Q}{P}  
  :=
  \lift{(x)\substp{Q}{P}}{ R \psubstp{Q}{P} } \\
%   (\dropn{x})  \psubstp{Q}{P}       
%   := 
%   \left\{ 
%     \begin{array}{ccc} 
%       \dropn{\quotep{Q}} & & x \nameeq \quotep{P} \\
%       \dropn{x} & & otherwise \\
%     \end{array}
%   \right. 
  (\dropn{x})  \psubstp{Q}{P}       
  := 
  \left\{ 
    \begin{array}{ccc} 
      Q & & x \nameeq \quotep{P} \\
      \dropn{x} & & otherwise \\
    \end{array}
  \right.
\end{mathpar}
 

where

\begin{eqnarray}
  (x)\id{\{} \lpquote Q \rpquote / \lpquote P \rpquote \id{\}}            = 
  \left\{ 
    \begin{array}{ccc}
      \lpquote Q \rpquote & & x \nameeq \lpquote P \rpquote \\
      x & & otherwise \\
    \end{array}
  \right. \nonumber
\end{eqnarray}

and $z$ is chosen distinct from $\quotep{P}$, $\quotep{Q}$, the free
names in $Q$, and all the names in $R$. Our $\alpha$-equivalence will
be built in the standard way from this substitution.

\begin{remark}\label{rem:no_self_referential_names}
  One consequence of these definitions is that $\forall P. \quotep{P}
  \not\in \freenames{P}$.
\end{remark}

\subsection{ Dynamic quote: an example }

Anticipating something of what's to come, consider applying the
substitution, $\widehat{\id{\{}u / z \id{\}}}$, to the following pair
of processes, $\lift{w}{y!(z)}$ and $w[ \lpquote y!(z) \rpquote ]$.

\begin{eqnarray}
	\lift{w}{y!(z)}\widehat{\id{\{}u / z \id{\}}}
		& = &
		\lift{w}{y!(u)} \nonumber\\
	w[ \lpquote y!(z) \rpquote ] \widehat{ \id{\{}u / z \id{\}} }
		& = &
		w[ \lpquote y!(z) \rpquote ] \nonumber
\end{eqnarray}

Because the body of the process between quotes is impervious to
substitution, we get radically different answers. In fact, by
examining the first process in an input context,
e.g. $x?(z).\lift{w}{y!(z)}$, we see that the process under the lift
operator may be shaped by prefixed inputs binding a name inside it. In
this sense, the lift operator will be seen as a way to dynamically
construct processes before reifying them as names.

Finally equipped with these standard features we can present the
dynamics of the calculus.

\subsubsection{Operational semantics} 

Finally, we introduce the computational dynamics. What marks these
algebras as distinct from other more traditionally studied algebraic
structures, e.g. vector spaces or polynomial rings, is the manner in
which dynamics is captured. In traditional structures, dynamics is typically
expressed through morphisms between such structures, as in linear maps
between vector spaces or morphisms between rings. In algebras
associated with the semantics of computation, the dynamics is
expressed as part of the algebraic structure itself, through a
reduction reduction relation typically denoted by $\red$. Below, we
give a recursive presentation of this relation for the calculus used
in the encoding.

$\red \subseteq \pi \times \pi$
$\red : \pi \to \mathcal{P}(\pi)$

\begin{mathpar}
  \inferrule* [lab=Comm] { \textsf{match}( x_{src}, x_{trgt} ) } { x_{trgt}?(y)P \; | \; x_{src}!\langle {Q} \rangle \red P\{\quotep{Q}/y}\} }
  \and \\
  \inferrule* [lab=Par] {{P} \red {P}'} {{{P} | {Q}} \red {{P}' | {Q}}}
  \and
  \inferrule* [lab=Equiv]{{{P} \scong {P}'} \andalso {{P}' \red {Q}'} \andalso {{Q}' \scong {Q}}}{{P} \red {Q}}
\end{mathpar}

\begin{eqnarray*}
  match_{\equiv} (\quotep{P},\quotep{Q}) & := & P \equiv Q \\
  match_{\dagger}(\quotep{P},\quotep{Q}) & := & \forall R. P|Q \red^{*} R => R \red^{*} 0 \\
  match_{K}(\quotep{P},\quotep{Q}) & := & K \mbox{ for some context } K
\end{eqnarray*}

$u?(x)P | u!\langle Q \rangle \red P\{\quotep{Q}/x\}$

%We write $\wred$ for $\red^*$, and $P\red$ if $\exists Q $ such that $ P \red Q$.
We write $P\red$ if $\exists Q $ such that $ P \red Q$ and $P\not\red$, otherwise.

\section{Replication}

As mentioned before, it is known that replication (and hence
recursion) can be implemented in a higher-order process algebra
\cite{SangiorgiWalker}. As our first example of calculation with the
machinery thus far presented we give the construction explicitly in
the {\rhoc}.

\begin{eqnarray}
	D_{x} & := & \prefix{x}{y}{(\binpar{\outputp{x}{y}}{@{y}})} \nonumber\\
	\bangp_{x}{P} & := & \binpar{{x}!\langle{\binpar{D_{x}}{P}}\rangle}{D_{x}} \nonumber
\end{eqnarray}

\begin{eqnarray}
	\bangp_{x}{P} & & \nonumber\\
	=
	& {x}!\langle{(\prefix{x}{y}{(\outputp{x}{y} | @{y})) | P}}\rangle 
	      | \prefix{x}{y}{(\outputp{x}{y} | @{y})} & \nonumber\\
	\red
	& (\outputp{x}{y} | @{y})\substn{\quotep{(\prefix{x}{y}{(@{y} | \outputp{x}{y})) | P}}}{y} & \nonumber\\
	=
	& \outputp{x}{\quotep{(\prefix{x}{y}{(\outputp{x}{y} | @{y})) | P}}}
	  | {(\prefix{x}{y}{(\outputp{x}{y} | @{y})) | P}} & \nonumber\\
	\red
	& \ldots & \nonumber\\
	\red^*
	& P | P | \ldots & \nonumber
\end{eqnarray}

Of course, this encoding, as an implementation, runs away, unfolding
$\bangp{P}$ eagerly. A lazier and more implementable replication
operator, restricted to input-guarded processes, may be obtained as follows.

\begin{eqnarray}
\bangp{\prefix{u}{v}{P}} 
	:= 
	\binpar{\lift{x}{\prefix{u}{v}{(\binpar{D(x)}{P})}}}{D(x)} \nonumber
\end{eqnarray}

\begin{remark}
  Note that the lazier definition still does not deal with summation
  or mixed summation (i.e. sums over input and output). The reader is
  invited to construct definitions of replication that deal with these
  features. 

  Further, the definitions are parameterized in a name, $x$. Can you,
  gentle reader, make a definition that eliminates this parameter and
  guarantees no accidental interaction between the replication
  machinery and the process being replicated -- i.e. no accidental
  sharing of names used by the process to get its work done and the
  name(s) used by the replication to effect copying. This latter
  revision of the definition of replication is crucial to obtaining
  the expected identity $!!P \sim !P$.
\end{remark}

\begin{remark}\label{rem:paradoxical_combinator}
  The reader familiar with the lambda calculus will have noticed the
  similarity between $D$ and the paradoxical combinator.

  [Ed. note: the existence of this seems to suggest we have to be more
  restrictive on the set of processes and names we admit if we are to
  support no-cloning.]
\end{remark}

\subsubsection{Bisimulation}

The computational dynamics gives rise to another kind of equivalence,
the equivalence of computational behavior. As previously mentioned
this is typically captured \emph{via} some form of bisimulation.

% The notion we use in this paper is weak barbed bisimulation
% \cite{milner91polyadicpi}.

The notion we use in this paper is derived from weak barbed
bisimulation \cite{milner91polyadicpi}. 

\begin{definition}
An \emph{observation relation}, $\downarrow_{\mathcal N}$, over a set
of names, $\mathcal N$, is the smallest relation satisfying the rules
below.

\infrule[Out-barb]{y \in {\mathcal N}, \; x \nameeq y}
		  {\outputp{x}{v} \downarrow_{\mathcal N} x}
\infrule[Par-barb]{\mbox{$P\downarrow_{\mathcal N} x$ or $Q\downarrow_{\mathcal N} x$}}
		  {\binpar{P}{Q} \downarrow_{\mathcal N} x}

We write $P \Downarrow_{\mathcal N} x$ if there is $Q$ such that 
$P \wred Q$ and $Q \downarrow_{\mathcal N} x$.
\end{definition}

\begin{definition}
%\label{def.bbisim}
An  ${\mathcal N}$-\emph{barbed bisimulation} over a set of names, ${\mathcal N}$, is a symmetric binary relation 
${\mathcal S}_{\mathcal N}$ between agents such that $P\rel{S}_{\mathcal N}Q$ implies:
\begin{enumerate}
\item If $P \red P'$ then $Q \wred Q'$ and $P'\rel{S}_{\mathcal N} Q'$.
\item If $P\downarrow_{\mathcal N} x$, then $Q\Downarrow_{\mathcal N} x$.
\end{enumerate}
$P$ is ${\mathcal N}$-barbed bisimilar to $Q$, written
$P \wbbisim_{\mathcal N} Q$, if $P \rel{S}_{\mathcal N} Q$ for some ${\mathcal N}$-barbed bisimulation ${\mathcal S}_{\mathcal N}$.
\end{definition}

$\mathcal{R} \subseteq \pi \times \pi$

$P \mathcal{R} Q => \forall P'. P \red P' \Rightarrow \exists Q'. Q \red Q', P' \mathcal{R} Q'$

$P \vdash x \Rightarrow Q \vdash x$

\begin{mathpar}
  \inferrule*[lab=Out-barb]{x \nameeq y}{{y}!\langle{Q}\rangle \vdash x}
  \and
  \inferrule*[lab=Par-barb]{\mbox{$P\vdash x$ or $Q\vdash x$}}{\binpar{P}{Q} \vdash x}
\end{mathpar}

\subsubsection{Contexts}

One of the principle advantages of computational calculi like the
$\pi$-calculus is a well-defined notion of context,
contextual-equivalence and a correlation between
contextual-equivalence and notions of bisimulation. The notion of
context allows the decomposition of a process into (sub-)process and
its syntactic environment, its context. Thus, a context may be
thought of as a process with a ``hole'' (written $\Box$) in it. The
application of a context $M$ to a process $P$, written $M[P]$, is
tantamount to filling the hole in $M$ with $P$. In this paper we do
not need the full weight of this theory, but do make use of the notion
of context in the proof the main theorem. 

\begin{mathpar}
  \inferrule* [lab=summation] {} {{M_{M},M_{N}} \bc \Box \;|\; x.M_{A} \;|\; M_{M}+M_{N}}
  \and
  \inferrule* [lab=agent] {} {{M_{A}} \bc (\vec{x})M_{P} \;| \; \clift{P_0,\ldots,M_{P},\ldots,P_N}}
  \and \\
  \inferrule* [lab=process] {} {{M_{P}} \bc M_{N} \;| \;P|M_{P} }
\end{mathpar} 

\begin{mathpar}
  \inferrule* [lab=sychronization] {} {M_{N} \bc \Box \;|\; x?M_{F} \;|\; x!M_{C}}
  \and
  \inferrule* [lab=abstraction] {} {{M_{F}} \bc (x)M_{P} }
  \and
  \inferrule* [lab=concretion] {} {{M_{C}} \bc \langle M_{P} \rangle }
  \and \\
  \inferrule* [lab=process] {} {{M_{P}} \bc M_{N} \;| \;P|M_{P} }
\end{mathpar}

\begin{definition}[contextual application] Given a context $M$, and
  process $P$, we define the \emph{contextual application}, $M[P] :=
  M\{P/\Box\}$. That is, the contextual application of M to P is the
  substitution of $P$ for $\Box$ in $M$.
\end{definition}

$\meaningof{-} : L \to \mathcal{P}(\pi)$

\begin{mathpar}
  \inferrule* [lab=collection] {} {\meaningof{true} = \pi, \and \meaningof{~E} = \pi \setminus \meaningof{E}, \and \meaningof{E_{1} \& E_{2}} = \meaningof{E_{1}} \cap \meaningof{E_{2}}}
\end{mathpar}

\begin{mathpar}
  \inferrule* [lab=structure] {} {\meaningof{0} = \{ P \in \pi | P \equiv 0 \}, \and \\ \meaningof{E_1 | E_2} = \{ P \in \pi | P \equiv P_{1} | P_{2}, P_{1} \in \meaningof{E_{1}}, P_{2} \in \meaningof{E_2}\} }
\end{mathpar}

\begin{mathpar}
 \inferrule* [lab=behavior] {} {\meaningof{\langle a?b \rangle E} = \{ P \in \pi | P \equiv Q | u?(y)P', \\ \and \\\\ \and \\ \;\;\; u \in \meaningof{a}, \forall z.P'\{z/y\} \in \meaningof{E\{z/b\}}\}, \and \\ \meaningof{a!E} = \{ P \in \pi | P \equiv Q | x!\langle P' \rangle, x \in \meaningof{a} P' \in \meaningof{E}\} }
\end{mathpar}

\begin{mathpar}
 \inferrule* [lab=nominal] {} {\meaningof{\quotep{E}} = \{ \quotep{P} \in \quotep{\pi} | P \in \meaningof{E} \}, \and \meaningof{\quotep{P}} = \{ \quotep{Q} \in \quotep{\pi} | P \equiv Q \} \and \\ \meaningof{@\quotep{E}} = \{ P \in \pi | P \equiv @x, x \in \meaningof{E} \}}
\end{mathpar}

\begin{eqnarray*}
  \\
  \meaningof{-} : TS \to ST
\end{eqnarray*}

\begin{eqnarray*}
  \\
  L : TS \to ST
\end{eqnarray*}

\begin{eqnarray*}
  \\
  P \models E \iff P \in \meaningof{E}
\end{eqnarray*}

\begin{eqnarray*}
  P \approx_{L} Q \iff \forall E \in L. P \models E \iff Q \models E
\end{eqnarray*}

\begin{eqnarray*}
  P \approx_{K} Q
\end{eqnarray*}

\begin{eqnarray*}
  P \approx Q
\end{eqnarray*}

$\approx_{K} = \approx = \approx_{L}$

\subsubsection{Contextual duality}

Note that contexts extend the quotation operation to a family of
operations from processes to names. Given a context, $M$, we can
define a \emph{nominal context}, $\quotep{M}$ by $\quotep{M}[P] :=
\quotep{M[P]}$. To foreshadow what is to come we observe that these
operations enjoy a duality with processes very much like the duality
between vectors and maps from vectors to scalars.

Further, because the calculus is essentially higher-order, we have a
correspondence between contexts and processes. More specifically,
given a name $x$ and a context $M$ we can construct $M^{*}_{x}$ such
that 

\begin{mathpar}
  M^{*}_{x} | \lift{x}{P} \red M[P]
\end{mathpar}

namely,

\begin{mathpar}
  M^{*}_{x} := x?(u).M[\dropn{u}]
\end{mathpar}

The dependence of $M^{*}_{x}$ on a name makes it an abstraction, 

\begin{mathpar}
  M^{*} := (x)x?(u).M[\dropn{u}]
\end{mathpar}

\subsection{Additional notation}

It will sometimes be convenient to denote the process a name
quotes. We already have the notation $x = \quotep{P}$, but it will be
convenient to introduce an alternate notation, $\procn{x}$, when we
want to emphasize the connection to the use of the name. Note that, by
virtue of name equivalence, $\quotep{\procn{x}} \nameeq x$; so, the
notation is consistent with previous definitions.

Further, because names have structure it is possible to effect
substitutions on the basis of that structure. This means we need to
upgrade our notation for substitutions, which we accomplish by
adapting comprehension notation. Thus,

\begin{mathpar}
  P\{ y / x : x \in S \}
\end{mathpar}

is interpreted to mean the process derived from P by replacing (in a
capture-avoiding manner) each occurrence of $x$ in $S$ by $y$. For example,

\begin{mathpar}
  P\{ \quotep{\procn{x}|\procn{x}} / x : x \in \freenames{P} \}
\end{mathpar}

will replace each (occurrence) of a free name $x$ in $P$ by
$\quotep{\procn{x}|\procn{x}}$.

Also, we will avail ourselves of the notation $x^{L}$ and $x^{R}$ to
denote injections of a name into disjoint copies of the name
space. There are numerous ways to accomplish this. One example can be
found in \cite{MeredithR05}. This notation overloads to vectors of
names: $\vec{x}^{\pi} := (x_{i}^{\pi} \; : \; 0 \leq i < |\vec{x}| )$ where $\pi \in \{L,R\}$.

We also use $P^{\Box} := P|\Box$.

In \cite{MeredithR05} an interpretation of the new operator is
given. It turns out that there are several possible interpretations
all enjoying the requisite algebraic properties of the operator (see
\cite{milner91polyadicpi}). We will therefore make liberal use of
$(\nu\; \vec{x})P$.

% subsection the_syntax_and_semantics_of_the_notation_system (end)   

\input{qm2pi.qmops} 

\input{qm2pi.sterngerlach} 

\input{qm2pi.metric} 

% section concurrent_process_calculi (end)

%\input{qm2pi.proofsketch}

% section proof sketch (end)

%\input{qm2pi.slviaknots} 

% section spatial logic via knots (end)

\input{qm2pi.conclusion}

% section conclusion (end)

%\input{qm2pi.dtcodes} 

% section wiring algorithm (end)

\input{qm2pi.ack} 

% section acknowledgments (end)

\newpage


\bibliographystyle{plain}   
\bibliography{../../biblios/main.bib}

\input{qm2pi.rhodetails}

\end{document}



% section proof sketch (end)

%\section{Unlikely characters: spatial logic for
  knots}\label{sub:characteristic_formulae} % (fold)

Associated to the mobile process calculi are a family of logics known
as the Hennessy-Milner logics. These logics typically enjoy a
semantics interpreting formulae as sets of processes that when
factored through the encoding outlined above allows an identification
of classes of knots with logical formulae. In the context of this
encoding the sub-family known as the spatial logics \cite{CairesC03}
\cite{CairesC04} \cite{Caires04} are of particular interest providing
several important features for expressing and reasoning about
properties (i.e. classes) of knots. We hint here at how this may be done.

%\begin{description}
%\item [structural connectives] 
\subsubsection{Structural connectives} The spatial logics enjoy
structural connectives corresponding, at the logical level, to the
parallel composition ($P | Q$) and new name ($(\nu \; x)P$)
connectives for processes. As illustrated in the examples below, these
connectives are extremely expressive given the shape of our encoding.
%\item [decideable satisfaction]

\subsubsection{Decideable satisfaction}
In \cite{Caires04} the satisfaction relation is shown to be decideable
for a rich class of processes. It further turns out that the image of
the our encoding is a proper subset of that class. This result
provides the basis for an algorithm by which to search for knots
enjoying a given property.
%\item [characteristic formulae]

\subsubsection{Characteristic formulae}
In the same paper \cite{Caires04} , Caires presents a means of calculating
characteristic formulae, selecting equivalence classes of processes
up to a pre--specified depth limit on the support set of names. Composed with our
encoding, this characteristic formula can be used to select
characteristic formulae for knots.
%\end{description}

\subsubsection{Spatial logic formulae}

The grammar below (segmented for comprehension) summarizes the syntax
of spatial logic formulae. We employ illustrative examples in the
sequel to provide an intuitive understanding of their meaning
referring the reader to \cite{Caires04} for a more detailed explication
of the semantics.

\begin{mathpar}
  \inferrule* [lab=boolean] {} {{A,B} \bc T \;|\; \neg A \;|\; A \wedge B \;|\; \eta = \eta'}
  \and
  \inferrule* [lab=spatial] {} {|\; \pzero \;|\; A | B \;|\; x \text{\textregistered} A \;|\; \forall x . A \;|\;  H x . A}
  \and
  \inferrule* [lab=behavioral] {} {|\; \alpha . A}
  \and 
  \inferrule* [lab=recursion] {} {|\; X(\vec{u}) \;|\; \mu X(\vec{u}) . A}
  \and
  \inferrule* [lab=action] {} {\alpha \bc \langle x?(\vec{y}) \rangle \;|\; \langle x!(\vec{y}) \rangle \;|\; \langle \tau \rangle}
  \and 
  \inferrule* [lab=name] {} {\eta \bc x \;|\; \tau}
\end{mathpar} 

% subsection characteristic_formulae (end)   	 

\subsection{Example formulae}\label{sub:example_formulae_} % (fold)

\subsubsection{Crossing as formula.}
% 
% \begin{align*}
%   \frac{d}{dx} \sin x &= \cos x 
%   & \frac{d}{dx} e^x &= e^x \\
%   \frac{d}{dx} \cos x &= - \sin x 
%   & \frac{d}{dx} \log x &= \frac{1}{x} \\
% \end{align*} 

\begin{align*}
 \mu C(x_{0},x_{1},y_{0},y_{1},u).&(\langle x_{0}?(z) \rangle(\langle u! \rangle\langle y_{1}!z \rangle C(x_{0},x_{1},y_{0},y_{1},u)) & \\
  & \wedge \langle y_{1}?(z) \rangle (\langle u! \rangle \langle x_{0}!z \rangle C(x_{0},x_{1},y_{0},y_{1},u)) & \\
  & \wedge \langle x_{1}?(z) \rangle (\langle u? \rangle \langle y_{0}!z \rangle C(x_{0},x_{1},y_{0},y_{1},u)) & \\
  & \wedge \langle y_{0}?(z) \rangle (\langle u? \rangle \langle x_{1}!z \rangle C(x_{0},x_{1},y_{0},y_{1},u))) &
\end{align*}

The lexicographical similarity between the shape of this formulae and
the shape of definition of the process representing a crossing reveals
the intuitive meaning of this formulae. It describes the capabilities
of a process that has the right to represent a crossing. For example
it picks out processes that may perform an input on the port $x_0$ in
its initial menu of capabilities. What differentiates the formula
from the process, however, is that the crossing process is the
smallest candidate to satisfy the formula. Infinitely many other
processes -- with internal behavior hidden behind this interface, so
to speak -- also satisfy this formula. Even this simple formula,
then, can be seen to open a new view onto knots, providing a
computational interpretation of \emph{virtual} knots.

Note that this formula is derived by hand. A similar formula can be
derived by employing Caires' calculation of characteristic formula
\cite{Caires04} to the process representing a crossing. In light of
this discussion, we let
$\meaningof{C}_{\phi}(x0,x1,y0,y1,u)$ denote a formula specifying the
dynamics we wish to capture of a crossing. To guarantee we preserve
the shape of the interface and minimal semantics we demand that
$\meaningof{C}_{\phi}(x0,x1,y0,y1,u) \Rightarrow
\textbf{C}(x0,x1,y0,y1,u)$ where $\textbf{C}(x0,x1,y0,y1,u)$ denotes
the formula above.
                            
\subsubsection{Crossing number constraints.}
The moral content of the context lemma (Lemma \ref{context}) is that the notion of
``locality'' in the Reidemeister moves is effectively captured by the
parallel composition operator of the process calculus. This intuition
extends through the logic. Given a formula,
$\meaningof{C}_{\phi}(x0,x1,y0,y1,u)$, we can use the structural
connectives to specify constraints on crossing numbers, such as at
least $n$ crossings, or exactly $n$ crossings.
\begin{mathpar}
  \inferrule* [lab=at-least-n] {} { K^{\geq n}_{\phi}(\vec{xs},\vec{ys}) := \Pi_{i=0}^{n-1} Hu . \meaningof{C}_{\phi}(xs_i,ys_i,u) | T }
  \and 
  \inferrule* [lab=exactly-n] {} { K^{= n}_{\phi}(\vec{xs},\vec{ys}) := \Pi_{i=0}^{n-1} Hu . \meaningof{C}_{\phi}(xs_i,ys_i,u) | \neg (\forall x_0,y_0,x_1,y_1,u . \meaningof{C}_{\phi}(x_0,y_0,x_1,y_1,u) | T) }
\end{mathpar}

To round out this section, recall that the encoding of an $n$-crossing
knot decomposes into a parallel composition of $n$ \emph{copies} of a
crossing process together with a wiring harness. To specify different
knot classes with the same crossing number amounts to specifying
logical constraints on the wiring harness. In the interest of space,
we defer examples to a forthcoming paper. Suffice it to say that both
the conditions ``alternating knot'' and ``contains the tangle
corresponding to 5/3'' are expressible. For example, it is possible to
calculate the characteristic formula of a process corresponding to the
tangle 5/3 and conjoin it into the classifying formula via the
composition connective of the logic.

Finally, we wish to observe that it is entirely within reason to
contemplate a more domain-specific version of spatial logic tailored
to the shape of processes in the image of the encoding. Such a
domain-specific logic would have a better claim to the title formal
language of knot properties.

% subsection example_formulae_ (end)

% section knots_as_processes (end) 

% section spatial logic via knots (end)

\section{Conclusions and future work}

\paragraph{Testing physical space}
You, gentle reader, may wonder why of all the theorems to be proved
given this set up we pick the one above. In some sense it's hardly
central to quantum mechanics. We see it as central in the sense that
it firmly establishes a notion of physical space arising from a notion
of the equivalence of behavior. Relating bisimulation to a metric is a
big step forward, but one is faced with interpreting the relationship
of that metric space to something more physical. Quantum mechanical
notions of ``physical'' space are still far from intuitive, but by
relating this idea of distance as testing to calculations that predict
physical circumstances we are making a not insignificant step forward
toward an understanding of the physical space we inhabit as
essentially dynamic.

\paragraph{Effectivity and simulation}
One of the observations we have yet to make is that the entire program
spelled out here is effective. We have built various interpreters for
the reflective calculus at work in this interpretation. In principle,
then, we can simulate quantum mechanics on a computer. The place where
the simulation may lose fidelity is the infinitely branching summation
for the annihilator.

In this connection i also want to point out that the evaluation style
calculation of the inner product puts the non-determinism of the
summation right at the heart of measurement. This suggests that
Milner's original reduction-based formulation of the dynamics of his
calculi in terms of sums was not just notationally suggestive of a
notion of measure-and-continue but captured some significant part of
the physics.

\paragraph{Quantum continuations}
In light of this last observation i want to point out that the
predominant account of quantum mechanics is missing a key aspect of a
truly compositional story of the physical situation. In a real lab,
when a measurement is made the observation can be made to feed into
another device that then makes another measurement conditioned on the
results of the first. This means that after the superposition was
collapsed the entire experimental set up remained in
superposition. While QM offers a means of writing this down it doesn't
quite line up well with the well-trodden formulation of computation
and continuation that we see so succinctly expressed in Milner's
calculi. This suggests that there might be advantages to this account
of dynamics waiting to be explored.

\paragraph{Quantum logic}
In this connection, we also note that by virtue of having the
Hennessy-Milner construction, we can pull the construction through the
interpretation of QM. This gives us a natural candidate for a quantum
logic that enjoys an extremely tight connection with it's domain of
interpretation, making the construction much less ad hoc (rather it is
the image of functor!).

\paragraph{Quantum probabiity}
i have questions about the basis of the interpretation of inner
product as probability amplitude. In particular, using which
axiomatization of probability theory does the notion of probability
amplitude earn the right to be so dubbed? In other words, where is the
proof that the operation for calculating a probability amplitude (and
then squaring) satisfies the axioms of what it means to calculate a
probability? Even if such a proof exists (i have yet to find it in the
literature), i wonder if it might not be possible to turn things on
their heads. Can we view the calculation of the probability amplitude
as an axiomatization of probability? If so, then the definition we
give for calculating probability amplitude may provide the basis for
an \emph{effective} theory of probability.

\paragraph{Quantum vs ``biological'' information}
Finally, i want to conclude with a more philosophical observation. At
a recent workshop in which QM was a predominant topic i noticed
something about quantum information. The speaker was giving a riveting
discussion of axiomatic QM and showing how properties of ``no
cloning'' and ``no deleting'' emerged as consequences of the
axiomatization. Theorems of this form are necessary to give us a sense
of confidence that our axioms characterize the physical theory. What
struck me, though, was that if quantum information is neither erasable
nor replicable it is markedly different from \emph{life}. Two of the
things we know about life is that

\begin{itemize}
  \item it ends;
  \item to gain some measure of persistence, to transcend it's
    finitude it is imminently copyable.
\end{itemize}

Both of these qualities are summarized succinctly in the aphorism: all
flesh is grass. For me these two kinds of ``information'' -- call them
quantum and biological -- are end points on a spectrum of strategies
for persistence. At one end, we have those curious entities that enjoy
uniqueness and permanence; at the other, we have those who in the face
of a certain end and an uncertain present make a go of passing
something on. To me one of the more remarkable aspects of the latter
strategy is that in the presence of noise (and certain features of
copying) we get a kind of dynamism, a chance for improvement against a
given persistent condition.

% subsection other_calculi_other_bisimulations_and_geometry_as_behavior (end)




% section conclusion (end)

%\documentclass[12pt]{llncs}
%\documentclass{jktr}

\usepackage[pdftex]{hyperref}                   
\usepackage {listings}
\usepackage {mathpartir}
\usepackage{bcprules}
%\usepackage{listings}
                       
\usepackage{graphicx} 
%\usepackage[margins=2.5cm,nohead,nofoot]{geometry}
%\usepackage{geometry}
\usepackage{amsfonts}
\usepackage{amstext}
\usepackage{latexsym}
\usepackage{amssymb}
\usepackage{color}


%\include{myPreamble}
\include{qm2pi.local} 

%\ifpdf
%\usepackage[pdftex]{graphicx}
%\else
%\usepackage{graphicx}
%\fi

 % \ifpdf
%  \usepackage{pdfsync}
%  \if


%\title{Brief Article}
%\author{David F. Snyder}
%\author{L.G. Meredith}

%\address{Dept. of Math., Texas State University--San Marcos, San Marcos, TX 78666}
       
\pagestyle{empty}


\begin{document}

\lstset{language=[Objective]Caml,frame=shadowbox}

\input{qm2pi.front}

% section front matter (end)

\input{qm2pi.intro} 
 
% section introduction (end)

% \input{qm2pi.knotations} 

% section notation (end)

\input{qm2pi.process.calculi} 

% section concurrent_process_calculi_and_spatial_logics_ (end)
    
%\input{qm2pi.knots2pi} 

%\input{qm2pi.trefoil} 

%\input{qm2pi.mainthm} 

% subsection basic_interpretation (end)

%\input{qm2pi.rho.presentation} 
\subsection{The syntax and semantics of the notation system}\label{sub:the_syntax_and_semantics_of_the_notation_system} % (fold)

We now summarize a technical presentation of the calculus that
embodies our theory of dynamics. The typical presentation of such a
calculus follows the style of giving generators and relations on
them. The grammar, below, describing term constructors, freely
generates the set of processes, $\Proc$. This set is then quotiented
by a relation known as structural congruence and it is over this set
that the notion of dynamics is expressed. This presentation is
essentially that of \cite{MeredithR05} with the addition of
polyadicity and summation. For readability we have relegated some of
the technical subtleties to an appendix.

\subsubsection{Process grammar}\label{subsub:process_grammar}

\begin{mathpar}
  \inferrule* [lab=synchronization] {} {{M} \bc \pzero \;|\; x?F \;|\; x!C }
  \and
  \inferrule* [lab=abstraction] {} {{F} \bc (x)P}
  \and
  \inferrule* [lab=concretion] {} {{C} \bc \langle Q \rangle}
  \and
  \inferrule* [lab=process] {} {{P,Q} \bc M \;| \;P|Q \;|\; @{x}}
  \and
  \inferrule* [lab=name] {} {{x} \bc \quotep{P}}
\end{mathpar} 

Note that $\vec{x}$ (resp. $\vec{P}$) denotes a vector of names
(resp. processes) of length $|\vec{x}|$ (resp. $|\vec{P}|$). We adopt
the following useful abbreviations.

\begin{mathpar}
   x?(\vec{y}).P := x.(\vec{y})P \and  x\clift{\vec{P}} := x.\clift{\vec{P}}
   \and x!(y) := \lift{x}{\dropn{y}}
   \and \Pi_{i=0}^{n-1}P_i := P_0 | \ldots | P_{n-1}
\end{mathpar}

\subsubsection{Structural congruence}

\paragraph{Free and bound names and alpha-equivalence.} At the
core of structural equivalence is alpha-equivalence which identifies
process that are the same up to a change of variable. Formally, we
recognize the distinction between free and bound names. The free names
of a process, $\freenames{P}$, may be calculated recursively as
follows:

\begin{mathpar}
\freenames{\pzero} := \emptyset
  \and \\
  \freenames{x?(y).P} := \{ x \} \cup (\freenames{P} \setminus \{ y \})
  \and 
  \freenames{x!\langle P \rangle} := \{ x \} \cup \{ P \} 
  \and \\
  \freenames{P|Q} := \freenames{P} \cup \freenames{Q}
  \and \\
  \freenames{@{x}} := \{ x \}
\end{mathpar}

$\pi$
$\quotep{\pi}$

$\freenames{-} : \pi \to \mathcal{P}(\quotep{\pi})$

\begin{eqnarray*}
  \freenames{\pzero} & := & \emptyset \\
  \freenames{x?(y).P} & := & \{ x \} \cup (\freenames{P} \setminus \{ y \}) \\
  \freenames{x!\langle P \rangle} & := & \{ x \} \cup \{ P \} \\
  \freenames{P|Q} & := & \freenames{P} \cup \freenames{Q} \\
  \freenames{\dropn{x}} & := & \{ x \}
\end{eqnarray*}

The bound names of a process, $\boundnames{P}$, are those names occurring in $P$
that are not free. For example, in $x?(y).0$, the name $x$ is free, while $y$ is bound.

\begin{mathpar}
  \inferrule* [lab=monoidal-laws] {} { P|Q \equiv Q|P \and P|0 \equiv P \and P|(Q|R) \equiv (P|Q)|R }
\end{mathpar}

\begin{mathpar}
  \inferrule* [lab=alpha-equivalence] {} { (x)P \equiv (y)P\{y/x\} \and y \not\in \freenames{P} }
\end{mathpar}

\begin{definition}
Then two processes, $P,Q$, are alpha-equivalent if $P = Q\{\vec{y}/\vec{x}\}$ for
some $\vec{x} \in \boundnames{Q},\vec{y} \in \boundnames{P}$, where $Q\{\vec{y}/\vec{x}\}$
denotes the capture-avoiding substitution of $\vec{y}$ for $\vec{x}$ in $Q$.
\end{definition}

\begin{definition}
  The {\em structural congruence} \cite{SangiorgiWalker} , $\equiv$,
  between processes is the least congruence containing
  alpha-equivalence, satisfying the abelian monoid laws
  (associativity, commutativity and $\pzero$ as identity) for parallel
  composition $|$ and for summation $+$.
\end{definition}

\subsection{Name equivalence}

We take name equivalence, written $\nameeq$, to be the smallest
equivalence relation generated by the following rules.

\begin{mathpar}
\inferrule*[lab=Quote-drop]
{ }
{ \quotep{@{x}} \nameeq x }

\inferrule*[lab=Struct-equiv]
{ P \scong Q }
{ \quotep{P} \nameeq \quotep{Q} }
\end{mathpar}

The astute reader will have noticed that the mutual recursion of names
and processes imposes a mutual recursion on alpha-equivalence and
structural equivalence via name-equivalence. Fortunately, all of this
works out pleasantly and we may calculate in the natural way, free of
concern. The reader interested in the details is referred to the
appendix \ref{appendix:rho_details}.

\subsection{Substitution}

We use $\Proc$ for the set of processes, $\QProc$ for the set of
names, and $\id{\{}\vec{y} / \vec{x} \id{\}}$ to denote partial maps,
$s : \QProc \rightarrow \QProc$. A map, $s$ lifts, uniquely, to a map
on process terms, $\widehat{s} : \Proc \rightarrow \Proc$ by the
following equations.

\begin{mathpar}
  (0) \psubstp{Q}{P} := 0 \\
  (R \juxtap S) \psubstp{Q}{P}
  :=    
  (R)\psubstp{Q}{P} \juxtap (S) \psubstp{Q}{P} \\
  (x?(y).R) \psubstp{Q}{P}    
  :=    
  (x)\substp{Q}{P} (z)\concat( (R \psubstn{z}{y}) \psubstp{Q}{P} ) \\
  (\lift{x}{R}) \psubstp{Q}{P}  
  :=
  \lift{(x)\substp{Q}{P}}{ R \psubstp{Q}{P} } \\
%   (\dropn{x})  \psubstp{Q}{P}       
%   := 
%   \left\{ 
%     \begin{array}{ccc} 
%       \dropn{\quotep{Q}} & & x \nameeq \quotep{P} \\
%       \dropn{x} & & otherwise \\
%     \end{array}
%   \right. 
  (\dropn{x})  \psubstp{Q}{P}       
  := 
  \left\{ 
    \begin{array}{ccc} 
      Q & & x \nameeq \quotep{P} \\
      \dropn{x} & & otherwise \\
    \end{array}
  \right.
\end{mathpar}
 

where

\begin{eqnarray}
  (x)\id{\{} \lpquote Q \rpquote / \lpquote P \rpquote \id{\}}            = 
  \left\{ 
    \begin{array}{ccc}
      \lpquote Q \rpquote & & x \nameeq \lpquote P \rpquote \\
      x & & otherwise \\
    \end{array}
  \right. \nonumber
\end{eqnarray}

and $z$ is chosen distinct from $\quotep{P}$, $\quotep{Q}$, the free
names in $Q$, and all the names in $R$. Our $\alpha$-equivalence will
be built in the standard way from this substitution.

\begin{remark}\label{rem:no_self_referential_names}
  One consequence of these definitions is that $\forall P. \quotep{P}
  \not\in \freenames{P}$.
\end{remark}

\subsection{ Dynamic quote: an example }

Anticipating something of what's to come, consider applying the
substitution, $\widehat{\id{\{}u / z \id{\}}}$, to the following pair
of processes, $\lift{w}{y!(z)}$ and $w[ \lpquote y!(z) \rpquote ]$.

\begin{eqnarray}
	\lift{w}{y!(z)}\widehat{\id{\{}u / z \id{\}}}
		& = &
		\lift{w}{y!(u)} \nonumber\\
	w[ \lpquote y!(z) \rpquote ] \widehat{ \id{\{}u / z \id{\}} }
		& = &
		w[ \lpquote y!(z) \rpquote ] \nonumber
\end{eqnarray}

Because the body of the process between quotes is impervious to
substitution, we get radically different answers. In fact, by
examining the first process in an input context,
e.g. $x?(z).\lift{w}{y!(z)}$, we see that the process under the lift
operator may be shaped by prefixed inputs binding a name inside it. In
this sense, the lift operator will be seen as a way to dynamically
construct processes before reifying them as names.

Finally equipped with these standard features we can present the
dynamics of the calculus.

\subsubsection{Operational semantics} 

Finally, we introduce the computational dynamics. What marks these
algebras as distinct from other more traditionally studied algebraic
structures, e.g. vector spaces or polynomial rings, is the manner in
which dynamics is captured. In traditional structures, dynamics is typically
expressed through morphisms between such structures, as in linear maps
between vector spaces or morphisms between rings. In algebras
associated with the semantics of computation, the dynamics is
expressed as part of the algebraic structure itself, through a
reduction reduction relation typically denoted by $\red$. Below, we
give a recursive presentation of this relation for the calculus used
in the encoding.

$\red \subseteq \pi \times \pi$
$\red : \pi \to \mathcal{P}(\pi)$

\begin{mathpar}
  \inferrule* [lab=Comm] { \textsf{match}( x_{src}, x_{trgt} ) } { x_{trgt}?(y)P \; | \; x_{src}!\langle {Q} \rangle \red P\{\quotep{Q}/y}\} }
  \and \\
  \inferrule* [lab=Par] {{P} \red {P}'} {{{P} | {Q}} \red {{P}' | {Q}}}
  \and
  \inferrule* [lab=Equiv]{{{P} \scong {P}'} \andalso {{P}' \red {Q}'} \andalso {{Q}' \scong {Q}}}{{P} \red {Q}}
\end{mathpar}

\begin{eqnarray*}
  match_{\equiv} (\quotep{P},\quotep{Q}) & := & P \equiv Q \\
  match_{\dagger}(\quotep{P},\quotep{Q}) & := & \forall R. P|Q \red^{*} R => R \red^{*} 0 \\
  match_{K}(\quotep{P},\quotep{Q}) & := & K \mbox{ for some context } K
\end{eqnarray*}

$u?(x)P | u!\langle Q \rangle \red P\{\quotep{Q}/x\}$

%We write $\wred$ for $\red^*$, and $P\red$ if $\exists Q $ such that $ P \red Q$.
We write $P\red$ if $\exists Q $ such that $ P \red Q$ and $P\not\red$, otherwise.

\section{Replication}

As mentioned before, it is known that replication (and hence
recursion) can be implemented in a higher-order process algebra
\cite{SangiorgiWalker}. As our first example of calculation with the
machinery thus far presented we give the construction explicitly in
the {\rhoc}.

\begin{eqnarray}
	D_{x} & := & \prefix{x}{y}{(\binpar{\outputp{x}{y}}{@{y}})} \nonumber\\
	\bangp_{x}{P} & := & \binpar{{x}!\langle{\binpar{D_{x}}{P}}\rangle}{D_{x}} \nonumber
\end{eqnarray}

\begin{eqnarray}
	\bangp_{x}{P} & & \nonumber\\
	=
	& {x}!\langle{(\prefix{x}{y}{(\outputp{x}{y} | @{y})) | P}}\rangle 
	      | \prefix{x}{y}{(\outputp{x}{y} | @{y})} & \nonumber\\
	\red
	& (\outputp{x}{y} | @{y})\substn{\quotep{(\prefix{x}{y}{(@{y} | \outputp{x}{y})) | P}}}{y} & \nonumber\\
	=
	& \outputp{x}{\quotep{(\prefix{x}{y}{(\outputp{x}{y} | @{y})) | P}}}
	  | {(\prefix{x}{y}{(\outputp{x}{y} | @{y})) | P}} & \nonumber\\
	\red
	& \ldots & \nonumber\\
	\red^*
	& P | P | \ldots & \nonumber
\end{eqnarray}

Of course, this encoding, as an implementation, runs away, unfolding
$\bangp{P}$ eagerly. A lazier and more implementable replication
operator, restricted to input-guarded processes, may be obtained as follows.

\begin{eqnarray}
\bangp{\prefix{u}{v}{P}} 
	:= 
	\binpar{\lift{x}{\prefix{u}{v}{(\binpar{D(x)}{P})}}}{D(x)} \nonumber
\end{eqnarray}

\begin{remark}
  Note that the lazier definition still does not deal with summation
  or mixed summation (i.e. sums over input and output). The reader is
  invited to construct definitions of replication that deal with these
  features. 

  Further, the definitions are parameterized in a name, $x$. Can you,
  gentle reader, make a definition that eliminates this parameter and
  guarantees no accidental interaction between the replication
  machinery and the process being replicated -- i.e. no accidental
  sharing of names used by the process to get its work done and the
  name(s) used by the replication to effect copying. This latter
  revision of the definition of replication is crucial to obtaining
  the expected identity $!!P \sim !P$.
\end{remark}

\begin{remark}\label{rem:paradoxical_combinator}
  The reader familiar with the lambda calculus will have noticed the
  similarity between $D$ and the paradoxical combinator.

  [Ed. note: the existence of this seems to suggest we have to be more
  restrictive on the set of processes and names we admit if we are to
  support no-cloning.]
\end{remark}

\subsubsection{Bisimulation}

The computational dynamics gives rise to another kind of equivalence,
the equivalence of computational behavior. As previously mentioned
this is typically captured \emph{via} some form of bisimulation.

% The notion we use in this paper is weak barbed bisimulation
% \cite{milner91polyadicpi}.

The notion we use in this paper is derived from weak barbed
bisimulation \cite{milner91polyadicpi}. 

\begin{definition}
An \emph{observation relation}, $\downarrow_{\mathcal N}$, over a set
of names, $\mathcal N$, is the smallest relation satisfying the rules
below.

\infrule[Out-barb]{y \in {\mathcal N}, \; x \nameeq y}
		  {\outputp{x}{v} \downarrow_{\mathcal N} x}
\infrule[Par-barb]{\mbox{$P\downarrow_{\mathcal N} x$ or $Q\downarrow_{\mathcal N} x$}}
		  {\binpar{P}{Q} \downarrow_{\mathcal N} x}

We write $P \Downarrow_{\mathcal N} x$ if there is $Q$ such that 
$P \wred Q$ and $Q \downarrow_{\mathcal N} x$.
\end{definition}

\begin{definition}
%\label{def.bbisim}
An  ${\mathcal N}$-\emph{barbed bisimulation} over a set of names, ${\mathcal N}$, is a symmetric binary relation 
${\mathcal S}_{\mathcal N}$ between agents such that $P\rel{S}_{\mathcal N}Q$ implies:
\begin{enumerate}
\item If $P \red P'$ then $Q \wred Q'$ and $P'\rel{S}_{\mathcal N} Q'$.
\item If $P\downarrow_{\mathcal N} x$, then $Q\Downarrow_{\mathcal N} x$.
\end{enumerate}
$P$ is ${\mathcal N}$-barbed bisimilar to $Q$, written
$P \wbbisim_{\mathcal N} Q$, if $P \rel{S}_{\mathcal N} Q$ for some ${\mathcal N}$-barbed bisimulation ${\mathcal S}_{\mathcal N}$.
\end{definition}

$\mathcal{R} \subseteq \pi \times \pi$

$P \mathcal{R} Q => \forall P'. P \red P' \Rightarrow \exists Q'. Q \red Q', P' \mathcal{R} Q'$

$P \vdash x \Rightarrow Q \vdash x$

\begin{mathpar}
  \inferrule*[lab=Out-barb]{x \nameeq y}{{y}!\langle{Q}\rangle \vdash x}
  \and
  \inferrule*[lab=Par-barb]{\mbox{$P\vdash x$ or $Q\vdash x$}}{\binpar{P}{Q} \vdash x}
\end{mathpar}

\subsubsection{Contexts}

One of the principle advantages of computational calculi like the
$\pi$-calculus is a well-defined notion of context,
contextual-equivalence and a correlation between
contextual-equivalence and notions of bisimulation. The notion of
context allows the decomposition of a process into (sub-)process and
its syntactic environment, its context. Thus, a context may be
thought of as a process with a ``hole'' (written $\Box$) in it. The
application of a context $M$ to a process $P$, written $M[P]$, is
tantamount to filling the hole in $M$ with $P$. In this paper we do
not need the full weight of this theory, but do make use of the notion
of context in the proof the main theorem. 

\begin{mathpar}
  \inferrule* [lab=summation] {} {{M_{M},M_{N}} \bc \Box \;|\; x.M_{A} \;|\; M_{M}+M_{N}}
  \and
  \inferrule* [lab=agent] {} {{M_{A}} \bc (\vec{x})M_{P} \;| \; \clift{P_0,\ldots,M_{P},\ldots,P_N}}
  \and \\
  \inferrule* [lab=process] {} {{M_{P}} \bc M_{N} \;| \;P|M_{P} }
\end{mathpar} 

\begin{mathpar}
  \inferrule* [lab=sychronization] {} {M_{N} \bc \Box \;|\; x?M_{F} \;|\; x!M_{C}}
  \and
  \inferrule* [lab=abstraction] {} {{M_{F}} \bc (x)M_{P} }
  \and
  \inferrule* [lab=concretion] {} {{M_{C}} \bc \langle M_{P} \rangle }
  \and \\
  \inferrule* [lab=process] {} {{M_{P}} \bc M_{N} \;| \;P|M_{P} }
\end{mathpar}

\begin{definition}[contextual application] Given a context $M$, and
  process $P$, we define the \emph{contextual application}, $M[P] :=
  M\{P/\Box\}$. That is, the contextual application of M to P is the
  substitution of $P$ for $\Box$ in $M$.
\end{definition}

$\meaningof{-} : L \to \mathcal{P}(\pi)$

\begin{mathpar}
  \inferrule* [lab=collection] {} {\meaningof{true} = \pi, \and \meaningof{~E} = \pi \setminus \meaningof{E}, \and \meaningof{E_{1} \& E_{2}} = \meaningof{E_{1}} \cap \meaningof{E_{2}}}
\end{mathpar}

\begin{mathpar}
  \inferrule* [lab=structure] {} {\meaningof{0} = \{ P \in \pi | P \equiv 0 \}, \and \\ \meaningof{E_1 | E_2} = \{ P \in \pi | P \equiv P_{1} | P_{2}, P_{1} \in \meaningof{E_{1}}, P_{2} \in \meaningof{E_2}\} }
\end{mathpar}

\begin{mathpar}
 \inferrule* [lab=behavior] {} {\meaningof{\langle a?b \rangle E} = \{ P \in \pi | P \equiv Q | u?(y)P', \\ \and \\\\ \and \\ \;\;\; u \in \meaningof{a}, \forall z.P'\{z/y\} \in \meaningof{E\{z/b\}}\}, \and \\ \meaningof{a!E} = \{ P \in \pi | P \equiv Q | x!\langle P' \rangle, x \in \meaningof{a} P' \in \meaningof{E}\} }
\end{mathpar}

\begin{mathpar}
 \inferrule* [lab=nominal] {} {\meaningof{\quotep{E}} = \{ \quotep{P} \in \quotep{\pi} | P \in \meaningof{E} \}, \and \meaningof{\quotep{P}} = \{ \quotep{Q} \in \quotep{\pi} | P \equiv Q \} \and \\ \meaningof{@\quotep{E}} = \{ P \in \pi | P \equiv @x, x \in \meaningof{E} \}}
\end{mathpar}

\begin{eqnarray*}
  \\
  \meaningof{-} : TS \to ST
\end{eqnarray*}

\begin{eqnarray*}
  \\
  L : TS \to ST
\end{eqnarray*}

\begin{eqnarray*}
  \\
  P \models E \iff P \in \meaningof{E}
\end{eqnarray*}

\begin{eqnarray*}
  P \approx_{L} Q \iff \forall E \in L. P \models E \iff Q \models E
\end{eqnarray*}

\begin{eqnarray*}
  P \approx_{K} Q
\end{eqnarray*}

\begin{eqnarray*}
  P \approx Q
\end{eqnarray*}

$\approx_{K} = \approx = \approx_{L}$

\subsubsection{Contextual duality}

Note that contexts extend the quotation operation to a family of
operations from processes to names. Given a context, $M$, we can
define a \emph{nominal context}, $\quotep{M}$ by $\quotep{M}[P] :=
\quotep{M[P]}$. To foreshadow what is to come we observe that these
operations enjoy a duality with processes very much like the duality
between vectors and maps from vectors to scalars.

Further, because the calculus is essentially higher-order, we have a
correspondence between contexts and processes. More specifically,
given a name $x$ and a context $M$ we can construct $M^{*}_{x}$ such
that 

\begin{mathpar}
  M^{*}_{x} | \lift{x}{P} \red M[P]
\end{mathpar}

namely,

\begin{mathpar}
  M^{*}_{x} := x?(u).M[\dropn{u}]
\end{mathpar}

The dependence of $M^{*}_{x}$ on a name makes it an abstraction, 

\begin{mathpar}
  M^{*} := (x)x?(u).M[\dropn{u}]
\end{mathpar}

\subsection{Additional notation}

It will sometimes be convenient to denote the process a name
quotes. We already have the notation $x = \quotep{P}$, but it will be
convenient to introduce an alternate notation, $\procn{x}$, when we
want to emphasize the connection to the use of the name. Note that, by
virtue of name equivalence, $\quotep{\procn{x}} \nameeq x$; so, the
notation is consistent with previous definitions.

Further, because names have structure it is possible to effect
substitutions on the basis of that structure. This means we need to
upgrade our notation for substitutions, which we accomplish by
adapting comprehension notation. Thus,

\begin{mathpar}
  P\{ y / x : x \in S \}
\end{mathpar}

is interpreted to mean the process derived from P by replacing (in a
capture-avoiding manner) each occurrence of $x$ in $S$ by $y$. For example,

\begin{mathpar}
  P\{ \quotep{\procn{x}|\procn{x}} / x : x \in \freenames{P} \}
\end{mathpar}

will replace each (occurrence) of a free name $x$ in $P$ by
$\quotep{\procn{x}|\procn{x}}$.

Also, we will avail ourselves of the notation $x^{L}$ and $x^{R}$ to
denote injections of a name into disjoint copies of the name
space. There are numerous ways to accomplish this. One example can be
found in \cite{MeredithR05}. This notation overloads to vectors of
names: $\vec{x}^{\pi} := (x_{i}^{\pi} \; : \; 0 \leq i < |\vec{x}| )$ where $\pi \in \{L,R\}$.

We also use $P^{\Box} := P|\Box$.

In \cite{MeredithR05} an interpretation of the new operator is
given. It turns out that there are several possible interpretations
all enjoying the requisite algebraic properties of the operator (see
\cite{milner91polyadicpi}). We will therefore make liberal use of
$(\nu\; \vec{x})P$.

% subsection the_syntax_and_semantics_of_the_notation_system (end)   

\input{qm2pi.qmops} 

\input{qm2pi.sterngerlach} 

\input{qm2pi.metric} 

% section concurrent_process_calculi (end)

%\input{qm2pi.proofsketch}

% section proof sketch (end)

%\input{qm2pi.slviaknots} 

% section spatial logic via knots (end)

\input{qm2pi.conclusion}

% section conclusion (end)

%\input{qm2pi.dtcodes} 

% section wiring algorithm (end)

\input{qm2pi.ack} 

% section acknowledgments (end)

\newpage


\bibliographystyle{plain}   
\bibliography{../../biblios/main.bib}

\input{qm2pi.rhodetails}

\end{document}

 

% section wiring algorithm (end)

\documentclass[12pt]{llncs}
%\documentclass{jktr}

\usepackage[pdftex]{hyperref}                   
\usepackage {listings}
\usepackage {mathpartir}
\usepackage{bcprules}
%\usepackage{listings}
                       
\usepackage{graphicx} 
%\usepackage[margins=2.5cm,nohead,nofoot]{geometry}
%\usepackage{geometry}
\usepackage{amsfonts}
\usepackage{amstext}
\usepackage{latexsym}
\usepackage{amssymb}
\usepackage{color}


%\include{myPreamble}
\include{qm2pi.local} 

%\ifpdf
%\usepackage[pdftex]{graphicx}
%\else
%\usepackage{graphicx}
%\fi

 % \ifpdf
%  \usepackage{pdfsync}
%  \if


%\title{Brief Article}
%\author{David F. Snyder}
%\author{L.G. Meredith}

%\address{Dept. of Math., Texas State University--San Marcos, San Marcos, TX 78666}
       
\pagestyle{empty}


\begin{document}

\lstset{language=[Objective]Caml,frame=shadowbox}

\input{qm2pi.front}

% section front matter (end)

\input{qm2pi.intro} 
 
% section introduction (end)

% \input{qm2pi.knotations} 

% section notation (end)

\input{qm2pi.process.calculi} 

% section concurrent_process_calculi_and_spatial_logics_ (end)
    
%\input{qm2pi.knots2pi} 

%\input{qm2pi.trefoil} 

%\input{qm2pi.mainthm} 

% subsection basic_interpretation (end)

%\input{qm2pi.rho.presentation} 
\subsection{The syntax and semantics of the notation system}\label{sub:the_syntax_and_semantics_of_the_notation_system} % (fold)

We now summarize a technical presentation of the calculus that
embodies our theory of dynamics. The typical presentation of such a
calculus follows the style of giving generators and relations on
them. The grammar, below, describing term constructors, freely
generates the set of processes, $\Proc$. This set is then quotiented
by a relation known as structural congruence and it is over this set
that the notion of dynamics is expressed. This presentation is
essentially that of \cite{MeredithR05} with the addition of
polyadicity and summation. For readability we have relegated some of
the technical subtleties to an appendix.

\subsubsection{Process grammar}\label{subsub:process_grammar}

\begin{mathpar}
  \inferrule* [lab=synchronization] {} {{M} \bc \pzero \;|\; x?F \;|\; x!C }
  \and
  \inferrule* [lab=abstraction] {} {{F} \bc (x)P}
  \and
  \inferrule* [lab=concretion] {} {{C} \bc \langle Q \rangle}
  \and
  \inferrule* [lab=process] {} {{P,Q} \bc M \;| \;P|Q \;|\; @{x}}
  \and
  \inferrule* [lab=name] {} {{x} \bc \quotep{P}}
\end{mathpar} 

Note that $\vec{x}$ (resp. $\vec{P}$) denotes a vector of names
(resp. processes) of length $|\vec{x}|$ (resp. $|\vec{P}|$). We adopt
the following useful abbreviations.

\begin{mathpar}
   x?(\vec{y}).P := x.(\vec{y})P \and  x\clift{\vec{P}} := x.\clift{\vec{P}}
   \and x!(y) := \lift{x}{\dropn{y}}
   \and \Pi_{i=0}^{n-1}P_i := P_0 | \ldots | P_{n-1}
\end{mathpar}

\subsubsection{Structural congruence}

\paragraph{Free and bound names and alpha-equivalence.} At the
core of structural equivalence is alpha-equivalence which identifies
process that are the same up to a change of variable. Formally, we
recognize the distinction between free and bound names. The free names
of a process, $\freenames{P}$, may be calculated recursively as
follows:

\begin{mathpar}
\freenames{\pzero} := \emptyset
  \and \\
  \freenames{x?(y).P} := \{ x \} \cup (\freenames{P} \setminus \{ y \})
  \and 
  \freenames{x!\langle P \rangle} := \{ x \} \cup \{ P \} 
  \and \\
  \freenames{P|Q} := \freenames{P} \cup \freenames{Q}
  \and \\
  \freenames{@{x}} := \{ x \}
\end{mathpar}

$\pi$
$\quotep{\pi}$

$\freenames{-} : \pi \to \mathcal{P}(\quotep{\pi})$

\begin{eqnarray*}
  \freenames{\pzero} & := & \emptyset \\
  \freenames{x?(y).P} & := & \{ x \} \cup (\freenames{P} \setminus \{ y \}) \\
  \freenames{x!\langle P \rangle} & := & \{ x \} \cup \{ P \} \\
  \freenames{P|Q} & := & \freenames{P} \cup \freenames{Q} \\
  \freenames{\dropn{x}} & := & \{ x \}
\end{eqnarray*}

The bound names of a process, $\boundnames{P}$, are those names occurring in $P$
that are not free. For example, in $x?(y).0$, the name $x$ is free, while $y$ is bound.

\begin{mathpar}
  \inferrule* [lab=monoidal-laws] {} { P|Q \equiv Q|P \and P|0 \equiv P \and P|(Q|R) \equiv (P|Q)|R }
\end{mathpar}

\begin{mathpar}
  \inferrule* [lab=alpha-equivalence] {} { (x)P \equiv (y)P\{y/x\} \and y \not\in \freenames{P} }
\end{mathpar}

\begin{definition}
Then two processes, $P,Q$, are alpha-equivalent if $P = Q\{\vec{y}/\vec{x}\}$ for
some $\vec{x} \in \boundnames{Q},\vec{y} \in \boundnames{P}$, where $Q\{\vec{y}/\vec{x}\}$
denotes the capture-avoiding substitution of $\vec{y}$ for $\vec{x}$ in $Q$.
\end{definition}

\begin{definition}
  The {\em structural congruence} \cite{SangiorgiWalker} , $\equiv$,
  between processes is the least congruence containing
  alpha-equivalence, satisfying the abelian monoid laws
  (associativity, commutativity and $\pzero$ as identity) for parallel
  composition $|$ and for summation $+$.
\end{definition}

\subsection{Name equivalence}

We take name equivalence, written $\nameeq$, to be the smallest
equivalence relation generated by the following rules.

\begin{mathpar}
\inferrule*[lab=Quote-drop]
{ }
{ \quotep{@{x}} \nameeq x }

\inferrule*[lab=Struct-equiv]
{ P \scong Q }
{ \quotep{P} \nameeq \quotep{Q} }
\end{mathpar}

The astute reader will have noticed that the mutual recursion of names
and processes imposes a mutual recursion on alpha-equivalence and
structural equivalence via name-equivalence. Fortunately, all of this
works out pleasantly and we may calculate in the natural way, free of
concern. The reader interested in the details is referred to the
appendix \ref{appendix:rho_details}.

\subsection{Substitution}

We use $\Proc$ for the set of processes, $\QProc$ for the set of
names, and $\id{\{}\vec{y} / \vec{x} \id{\}}$ to denote partial maps,
$s : \QProc \rightarrow \QProc$. A map, $s$ lifts, uniquely, to a map
on process terms, $\widehat{s} : \Proc \rightarrow \Proc$ by the
following equations.

\begin{mathpar}
  (0) \psubstp{Q}{P} := 0 \\
  (R \juxtap S) \psubstp{Q}{P}
  :=    
  (R)\psubstp{Q}{P} \juxtap (S) \psubstp{Q}{P} \\
  (x?(y).R) \psubstp{Q}{P}    
  :=    
  (x)\substp{Q}{P} (z)\concat( (R \psubstn{z}{y}) \psubstp{Q}{P} ) \\
  (\lift{x}{R}) \psubstp{Q}{P}  
  :=
  \lift{(x)\substp{Q}{P}}{ R \psubstp{Q}{P} } \\
%   (\dropn{x})  \psubstp{Q}{P}       
%   := 
%   \left\{ 
%     \begin{array}{ccc} 
%       \dropn{\quotep{Q}} & & x \nameeq \quotep{P} \\
%       \dropn{x} & & otherwise \\
%     \end{array}
%   \right. 
  (\dropn{x})  \psubstp{Q}{P}       
  := 
  \left\{ 
    \begin{array}{ccc} 
      Q & & x \nameeq \quotep{P} \\
      \dropn{x} & & otherwise \\
    \end{array}
  \right.
\end{mathpar}
 

where

\begin{eqnarray}
  (x)\id{\{} \lpquote Q \rpquote / \lpquote P \rpquote \id{\}}            = 
  \left\{ 
    \begin{array}{ccc}
      \lpquote Q \rpquote & & x \nameeq \lpquote P \rpquote \\
      x & & otherwise \\
    \end{array}
  \right. \nonumber
\end{eqnarray}

and $z$ is chosen distinct from $\quotep{P}$, $\quotep{Q}$, the free
names in $Q$, and all the names in $R$. Our $\alpha$-equivalence will
be built in the standard way from this substitution.

\begin{remark}\label{rem:no_self_referential_names}
  One consequence of these definitions is that $\forall P. \quotep{P}
  \not\in \freenames{P}$.
\end{remark}

\subsection{ Dynamic quote: an example }

Anticipating something of what's to come, consider applying the
substitution, $\widehat{\id{\{}u / z \id{\}}}$, to the following pair
of processes, $\lift{w}{y!(z)}$ and $w[ \lpquote y!(z) \rpquote ]$.

\begin{eqnarray}
	\lift{w}{y!(z)}\widehat{\id{\{}u / z \id{\}}}
		& = &
		\lift{w}{y!(u)} \nonumber\\
	w[ \lpquote y!(z) \rpquote ] \widehat{ \id{\{}u / z \id{\}} }
		& = &
		w[ \lpquote y!(z) \rpquote ] \nonumber
\end{eqnarray}

Because the body of the process between quotes is impervious to
substitution, we get radically different answers. In fact, by
examining the first process in an input context,
e.g. $x?(z).\lift{w}{y!(z)}$, we see that the process under the lift
operator may be shaped by prefixed inputs binding a name inside it. In
this sense, the lift operator will be seen as a way to dynamically
construct processes before reifying them as names.

Finally equipped with these standard features we can present the
dynamics of the calculus.

\subsubsection{Operational semantics} 

Finally, we introduce the computational dynamics. What marks these
algebras as distinct from other more traditionally studied algebraic
structures, e.g. vector spaces or polynomial rings, is the manner in
which dynamics is captured. In traditional structures, dynamics is typically
expressed through morphisms between such structures, as in linear maps
between vector spaces or morphisms between rings. In algebras
associated with the semantics of computation, the dynamics is
expressed as part of the algebraic structure itself, through a
reduction reduction relation typically denoted by $\red$. Below, we
give a recursive presentation of this relation for the calculus used
in the encoding.

$\red \subseteq \pi \times \pi$
$\red : \pi \to \mathcal{P}(\pi)$

\begin{mathpar}
  \inferrule* [lab=Comm] { \textsf{match}( x_{src}, x_{trgt} ) } { x_{trgt}?(y)P \; | \; x_{src}!\langle {Q} \rangle \red P\{\quotep{Q}/y}\} }
  \and \\
  \inferrule* [lab=Par] {{P} \red {P}'} {{{P} | {Q}} \red {{P}' | {Q}}}
  \and
  \inferrule* [lab=Equiv]{{{P} \scong {P}'} \andalso {{P}' \red {Q}'} \andalso {{Q}' \scong {Q}}}{{P} \red {Q}}
\end{mathpar}

\begin{eqnarray*}
  match_{\equiv} (\quotep{P},\quotep{Q}) & := & P \equiv Q \\
  match_{\dagger}(\quotep{P},\quotep{Q}) & := & \forall R. P|Q \red^{*} R => R \red^{*} 0 \\
  match_{K}(\quotep{P},\quotep{Q}) & := & K \mbox{ for some context } K
\end{eqnarray*}

$u?(x)P | u!\langle Q \rangle \red P\{\quotep{Q}/x\}$

%We write $\wred$ for $\red^*$, and $P\red$ if $\exists Q $ such that $ P \red Q$.
We write $P\red$ if $\exists Q $ such that $ P \red Q$ and $P\not\red$, otherwise.

\section{Replication}

As mentioned before, it is known that replication (and hence
recursion) can be implemented in a higher-order process algebra
\cite{SangiorgiWalker}. As our first example of calculation with the
machinery thus far presented we give the construction explicitly in
the {\rhoc}.

\begin{eqnarray}
	D_{x} & := & \prefix{x}{y}{(\binpar{\outputp{x}{y}}{@{y}})} \nonumber\\
	\bangp_{x}{P} & := & \binpar{{x}!\langle{\binpar{D_{x}}{P}}\rangle}{D_{x}} \nonumber
\end{eqnarray}

\begin{eqnarray}
	\bangp_{x}{P} & & \nonumber\\
	=
	& {x}!\langle{(\prefix{x}{y}{(\outputp{x}{y} | @{y})) | P}}\rangle 
	      | \prefix{x}{y}{(\outputp{x}{y} | @{y})} & \nonumber\\
	\red
	& (\outputp{x}{y} | @{y})\substn{\quotep{(\prefix{x}{y}{(@{y} | \outputp{x}{y})) | P}}}{y} & \nonumber\\
	=
	& \outputp{x}{\quotep{(\prefix{x}{y}{(\outputp{x}{y} | @{y})) | P}}}
	  | {(\prefix{x}{y}{(\outputp{x}{y} | @{y})) | P}} & \nonumber\\
	\red
	& \ldots & \nonumber\\
	\red^*
	& P | P | \ldots & \nonumber
\end{eqnarray}

Of course, this encoding, as an implementation, runs away, unfolding
$\bangp{P}$ eagerly. A lazier and more implementable replication
operator, restricted to input-guarded processes, may be obtained as follows.

\begin{eqnarray}
\bangp{\prefix{u}{v}{P}} 
	:= 
	\binpar{\lift{x}{\prefix{u}{v}{(\binpar{D(x)}{P})}}}{D(x)} \nonumber
\end{eqnarray}

\begin{remark}
  Note that the lazier definition still does not deal with summation
  or mixed summation (i.e. sums over input and output). The reader is
  invited to construct definitions of replication that deal with these
  features. 

  Further, the definitions are parameterized in a name, $x$. Can you,
  gentle reader, make a definition that eliminates this parameter and
  guarantees no accidental interaction between the replication
  machinery and the process being replicated -- i.e. no accidental
  sharing of names used by the process to get its work done and the
  name(s) used by the replication to effect copying. This latter
  revision of the definition of replication is crucial to obtaining
  the expected identity $!!P \sim !P$.
\end{remark}

\begin{remark}\label{rem:paradoxical_combinator}
  The reader familiar with the lambda calculus will have noticed the
  similarity between $D$ and the paradoxical combinator.

  [Ed. note: the existence of this seems to suggest we have to be more
  restrictive on the set of processes and names we admit if we are to
  support no-cloning.]
\end{remark}

\subsubsection{Bisimulation}

The computational dynamics gives rise to another kind of equivalence,
the equivalence of computational behavior. As previously mentioned
this is typically captured \emph{via} some form of bisimulation.

% The notion we use in this paper is weak barbed bisimulation
% \cite{milner91polyadicpi}.

The notion we use in this paper is derived from weak barbed
bisimulation \cite{milner91polyadicpi}. 

\begin{definition}
An \emph{observation relation}, $\downarrow_{\mathcal N}$, over a set
of names, $\mathcal N$, is the smallest relation satisfying the rules
below.

\infrule[Out-barb]{y \in {\mathcal N}, \; x \nameeq y}
		  {\outputp{x}{v} \downarrow_{\mathcal N} x}
\infrule[Par-barb]{\mbox{$P\downarrow_{\mathcal N} x$ or $Q\downarrow_{\mathcal N} x$}}
		  {\binpar{P}{Q} \downarrow_{\mathcal N} x}

We write $P \Downarrow_{\mathcal N} x$ if there is $Q$ such that 
$P \wred Q$ and $Q \downarrow_{\mathcal N} x$.
\end{definition}

\begin{definition}
%\label{def.bbisim}
An  ${\mathcal N}$-\emph{barbed bisimulation} over a set of names, ${\mathcal N}$, is a symmetric binary relation 
${\mathcal S}_{\mathcal N}$ between agents such that $P\rel{S}_{\mathcal N}Q$ implies:
\begin{enumerate}
\item If $P \red P'$ then $Q \wred Q'$ and $P'\rel{S}_{\mathcal N} Q'$.
\item If $P\downarrow_{\mathcal N} x$, then $Q\Downarrow_{\mathcal N} x$.
\end{enumerate}
$P$ is ${\mathcal N}$-barbed bisimilar to $Q$, written
$P \wbbisim_{\mathcal N} Q$, if $P \rel{S}_{\mathcal N} Q$ for some ${\mathcal N}$-barbed bisimulation ${\mathcal S}_{\mathcal N}$.
\end{definition}

$\mathcal{R} \subseteq \pi \times \pi$

$P \mathcal{R} Q => \forall P'. P \red P' \Rightarrow \exists Q'. Q \red Q', P' \mathcal{R} Q'$

$P \vdash x \Rightarrow Q \vdash x$

\begin{mathpar}
  \inferrule*[lab=Out-barb]{x \nameeq y}{{y}!\langle{Q}\rangle \vdash x}
  \and
  \inferrule*[lab=Par-barb]{\mbox{$P\vdash x$ or $Q\vdash x$}}{\binpar{P}{Q} \vdash x}
\end{mathpar}

\subsubsection{Contexts}

One of the principle advantages of computational calculi like the
$\pi$-calculus is a well-defined notion of context,
contextual-equivalence and a correlation between
contextual-equivalence and notions of bisimulation. The notion of
context allows the decomposition of a process into (sub-)process and
its syntactic environment, its context. Thus, a context may be
thought of as a process with a ``hole'' (written $\Box$) in it. The
application of a context $M$ to a process $P$, written $M[P]$, is
tantamount to filling the hole in $M$ with $P$. In this paper we do
not need the full weight of this theory, but do make use of the notion
of context in the proof the main theorem. 

\begin{mathpar}
  \inferrule* [lab=summation] {} {{M_{M},M_{N}} \bc \Box \;|\; x.M_{A} \;|\; M_{M}+M_{N}}
  \and
  \inferrule* [lab=agent] {} {{M_{A}} \bc (\vec{x})M_{P} \;| \; \clift{P_0,\ldots,M_{P},\ldots,P_N}}
  \and \\
  \inferrule* [lab=process] {} {{M_{P}} \bc M_{N} \;| \;P|M_{P} }
\end{mathpar} 

\begin{mathpar}
  \inferrule* [lab=sychronization] {} {M_{N} \bc \Box \;|\; x?M_{F} \;|\; x!M_{C}}
  \and
  \inferrule* [lab=abstraction] {} {{M_{F}} \bc (x)M_{P} }
  \and
  \inferrule* [lab=concretion] {} {{M_{C}} \bc \langle M_{P} \rangle }
  \and \\
  \inferrule* [lab=process] {} {{M_{P}} \bc M_{N} \;| \;P|M_{P} }
\end{mathpar}

\begin{definition}[contextual application] Given a context $M$, and
  process $P$, we define the \emph{contextual application}, $M[P] :=
  M\{P/\Box\}$. That is, the contextual application of M to P is the
  substitution of $P$ for $\Box$ in $M$.
\end{definition}

$\meaningof{-} : L \to \mathcal{P}(\pi)$

\begin{mathpar}
  \inferrule* [lab=collection] {} {\meaningof{true} = \pi, \and \meaningof{~E} = \pi \setminus \meaningof{E}, \and \meaningof{E_{1} \& E_{2}} = \meaningof{E_{1}} \cap \meaningof{E_{2}}}
\end{mathpar}

\begin{mathpar}
  \inferrule* [lab=structure] {} {\meaningof{0} = \{ P \in \pi | P \equiv 0 \}, \and \\ \meaningof{E_1 | E_2} = \{ P \in \pi | P \equiv P_{1} | P_{2}, P_{1} \in \meaningof{E_{1}}, P_{2} \in \meaningof{E_2}\} }
\end{mathpar}

\begin{mathpar}
 \inferrule* [lab=behavior] {} {\meaningof{\langle a?b \rangle E} = \{ P \in \pi | P \equiv Q | u?(y)P', \\ \and \\\\ \and \\ \;\;\; u \in \meaningof{a}, \forall z.P'\{z/y\} \in \meaningof{E\{z/b\}}\}, \and \\ \meaningof{a!E} = \{ P \in \pi | P \equiv Q | x!\langle P' \rangle, x \in \meaningof{a} P' \in \meaningof{E}\} }
\end{mathpar}

\begin{mathpar}
 \inferrule* [lab=nominal] {} {\meaningof{\quotep{E}} = \{ \quotep{P} \in \quotep{\pi} | P \in \meaningof{E} \}, \and \meaningof{\quotep{P}} = \{ \quotep{Q} \in \quotep{\pi} | P \equiv Q \} \and \\ \meaningof{@\quotep{E}} = \{ P \in \pi | P \equiv @x, x \in \meaningof{E} \}}
\end{mathpar}

\begin{eqnarray*}
  \\
  \meaningof{-} : TS \to ST
\end{eqnarray*}

\begin{eqnarray*}
  \\
  L : TS \to ST
\end{eqnarray*}

\begin{eqnarray*}
  \\
  P \models E \iff P \in \meaningof{E}
\end{eqnarray*}

\begin{eqnarray*}
  P \approx_{L} Q \iff \forall E \in L. P \models E \iff Q \models E
\end{eqnarray*}

\begin{eqnarray*}
  P \approx_{K} Q
\end{eqnarray*}

\begin{eqnarray*}
  P \approx Q
\end{eqnarray*}

$\approx_{K} = \approx = \approx_{L}$

\subsubsection{Contextual duality}

Note that contexts extend the quotation operation to a family of
operations from processes to names. Given a context, $M$, we can
define a \emph{nominal context}, $\quotep{M}$ by $\quotep{M}[P] :=
\quotep{M[P]}$. To foreshadow what is to come we observe that these
operations enjoy a duality with processes very much like the duality
between vectors and maps from vectors to scalars.

Further, because the calculus is essentially higher-order, we have a
correspondence between contexts and processes. More specifically,
given a name $x$ and a context $M$ we can construct $M^{*}_{x}$ such
that 

\begin{mathpar}
  M^{*}_{x} | \lift{x}{P} \red M[P]
\end{mathpar}

namely,

\begin{mathpar}
  M^{*}_{x} := x?(u).M[\dropn{u}]
\end{mathpar}

The dependence of $M^{*}_{x}$ on a name makes it an abstraction, 

\begin{mathpar}
  M^{*} := (x)x?(u).M[\dropn{u}]
\end{mathpar}

\subsection{Additional notation}

It will sometimes be convenient to denote the process a name
quotes. We already have the notation $x = \quotep{P}$, but it will be
convenient to introduce an alternate notation, $\procn{x}$, when we
want to emphasize the connection to the use of the name. Note that, by
virtue of name equivalence, $\quotep{\procn{x}} \nameeq x$; so, the
notation is consistent with previous definitions.

Further, because names have structure it is possible to effect
substitutions on the basis of that structure. This means we need to
upgrade our notation for substitutions, which we accomplish by
adapting comprehension notation. Thus,

\begin{mathpar}
  P\{ y / x : x \in S \}
\end{mathpar}

is interpreted to mean the process derived from P by replacing (in a
capture-avoiding manner) each occurrence of $x$ in $S$ by $y$. For example,

\begin{mathpar}
  P\{ \quotep{\procn{x}|\procn{x}} / x : x \in \freenames{P} \}
\end{mathpar}

will replace each (occurrence) of a free name $x$ in $P$ by
$\quotep{\procn{x}|\procn{x}}$.

Also, we will avail ourselves of the notation $x^{L}$ and $x^{R}$ to
denote injections of a name into disjoint copies of the name
space. There are numerous ways to accomplish this. One example can be
found in \cite{MeredithR05}. This notation overloads to vectors of
names: $\vec{x}^{\pi} := (x_{i}^{\pi} \; : \; 0 \leq i < |\vec{x}| )$ where $\pi \in \{L,R\}$.

We also use $P^{\Box} := P|\Box$.

In \cite{MeredithR05} an interpretation of the new operator is
given. It turns out that there are several possible interpretations
all enjoying the requisite algebraic properties of the operator (see
\cite{milner91polyadicpi}). We will therefore make liberal use of
$(\nu\; \vec{x})P$.

% subsection the_syntax_and_semantics_of_the_notation_system (end)   

\input{qm2pi.qmops} 

\input{qm2pi.sterngerlach} 

\input{qm2pi.metric} 

% section concurrent_process_calculi (end)

%\input{qm2pi.proofsketch}

% section proof sketch (end)

%\input{qm2pi.slviaknots} 

% section spatial logic via knots (end)

\input{qm2pi.conclusion}

% section conclusion (end)

%\input{qm2pi.dtcodes} 

% section wiring algorithm (end)

\input{qm2pi.ack} 

% section acknowledgments (end)

\newpage


\bibliographystyle{plain}   
\bibliography{../../biblios/main.bib}

\input{qm2pi.rhodetails}

\end{document}

 

% section acknowledgments (end)

\newpage


\bibliographystyle{plain}   
\bibliography{../../biblios/main.bib}

\documentclass[12pt]{llncs}
%\documentclass{jktr}

\usepackage[pdftex]{hyperref}                   
\usepackage {listings}
\usepackage {mathpartir}
\usepackage{bcprules}
%\usepackage{listings}
                       
\usepackage{graphicx} 
%\usepackage[margins=2.5cm,nohead,nofoot]{geometry}
%\usepackage{geometry}
\usepackage{amsfonts}
\usepackage{amstext}
\usepackage{latexsym}
\usepackage{amssymb}
\usepackage{color}


%\include{myPreamble}
\include{qm2pi.local} 

%\ifpdf
%\usepackage[pdftex]{graphicx}
%\else
%\usepackage{graphicx}
%\fi

 % \ifpdf
%  \usepackage{pdfsync}
%  \if


%\title{Brief Article}
%\author{David F. Snyder}
%\author{L.G. Meredith}

%\address{Dept. of Math., Texas State University--San Marcos, San Marcos, TX 78666}
       
\pagestyle{empty}


\begin{document}

\lstset{language=[Objective]Caml,frame=shadowbox}

\input{qm2pi.front}

% section front matter (end)

\input{qm2pi.intro} 
 
% section introduction (end)

% \input{qm2pi.knotations} 

% section notation (end)

\input{qm2pi.process.calculi} 

% section concurrent_process_calculi_and_spatial_logics_ (end)
    
%\input{qm2pi.knots2pi} 

%\input{qm2pi.trefoil} 

%\input{qm2pi.mainthm} 

% subsection basic_interpretation (end)

%\input{qm2pi.rho.presentation} 
\subsection{The syntax and semantics of the notation system}\label{sub:the_syntax_and_semantics_of_the_notation_system} % (fold)

We now summarize a technical presentation of the calculus that
embodies our theory of dynamics. The typical presentation of such a
calculus follows the style of giving generators and relations on
them. The grammar, below, describing term constructors, freely
generates the set of processes, $\Proc$. This set is then quotiented
by a relation known as structural congruence and it is over this set
that the notion of dynamics is expressed. This presentation is
essentially that of \cite{MeredithR05} with the addition of
polyadicity and summation. For readability we have relegated some of
the technical subtleties to an appendix.

\subsubsection{Process grammar}\label{subsub:process_grammar}

\begin{mathpar}
  \inferrule* [lab=synchronization] {} {{M} \bc \pzero \;|\; x?F \;|\; x!C }
  \and
  \inferrule* [lab=abstraction] {} {{F} \bc (x)P}
  \and
  \inferrule* [lab=concretion] {} {{C} \bc \langle Q \rangle}
  \and
  \inferrule* [lab=process] {} {{P,Q} \bc M \;| \;P|Q \;|\; @{x}}
  \and
  \inferrule* [lab=name] {} {{x} \bc \quotep{P}}
\end{mathpar} 

Note that $\vec{x}$ (resp. $\vec{P}$) denotes a vector of names
(resp. processes) of length $|\vec{x}|$ (resp. $|\vec{P}|$). We adopt
the following useful abbreviations.

\begin{mathpar}
   x?(\vec{y}).P := x.(\vec{y})P \and  x\clift{\vec{P}} := x.\clift{\vec{P}}
   \and x!(y) := \lift{x}{\dropn{y}}
   \and \Pi_{i=0}^{n-1}P_i := P_0 | \ldots | P_{n-1}
\end{mathpar}

\subsubsection{Structural congruence}

\paragraph{Free and bound names and alpha-equivalence.} At the
core of structural equivalence is alpha-equivalence which identifies
process that are the same up to a change of variable. Formally, we
recognize the distinction between free and bound names. The free names
of a process, $\freenames{P}$, may be calculated recursively as
follows:

\begin{mathpar}
\freenames{\pzero} := \emptyset
  \and \\
  \freenames{x?(y).P} := \{ x \} \cup (\freenames{P} \setminus \{ y \})
  \and 
  \freenames{x!\langle P \rangle} := \{ x \} \cup \{ P \} 
  \and \\
  \freenames{P|Q} := \freenames{P} \cup \freenames{Q}
  \and \\
  \freenames{@{x}} := \{ x \}
\end{mathpar}

$\pi$
$\quotep{\pi}$

$\freenames{-} : \pi \to \mathcal{P}(\quotep{\pi})$

\begin{eqnarray*}
  \freenames{\pzero} & := & \emptyset \\
  \freenames{x?(y).P} & := & \{ x \} \cup (\freenames{P} \setminus \{ y \}) \\
  \freenames{x!\langle P \rangle} & := & \{ x \} \cup \{ P \} \\
  \freenames{P|Q} & := & \freenames{P} \cup \freenames{Q} \\
  \freenames{\dropn{x}} & := & \{ x \}
\end{eqnarray*}

The bound names of a process, $\boundnames{P}$, are those names occurring in $P$
that are not free. For example, in $x?(y).0$, the name $x$ is free, while $y$ is bound.

\begin{mathpar}
  \inferrule* [lab=monoidal-laws] {} { P|Q \equiv Q|P \and P|0 \equiv P \and P|(Q|R) \equiv (P|Q)|R }
\end{mathpar}

\begin{mathpar}
  \inferrule* [lab=alpha-equivalence] {} { (x)P \equiv (y)P\{y/x\} \and y \not\in \freenames{P} }
\end{mathpar}

\begin{definition}
Then two processes, $P,Q$, are alpha-equivalent if $P = Q\{\vec{y}/\vec{x}\}$ for
some $\vec{x} \in \boundnames{Q},\vec{y} \in \boundnames{P}$, where $Q\{\vec{y}/\vec{x}\}$
denotes the capture-avoiding substitution of $\vec{y}$ for $\vec{x}$ in $Q$.
\end{definition}

\begin{definition}
  The {\em structural congruence} \cite{SangiorgiWalker} , $\equiv$,
  between processes is the least congruence containing
  alpha-equivalence, satisfying the abelian monoid laws
  (associativity, commutativity and $\pzero$ as identity) for parallel
  composition $|$ and for summation $+$.
\end{definition}

\subsection{Name equivalence}

We take name equivalence, written $\nameeq$, to be the smallest
equivalence relation generated by the following rules.

\begin{mathpar}
\inferrule*[lab=Quote-drop]
{ }
{ \quotep{@{x}} \nameeq x }

\inferrule*[lab=Struct-equiv]
{ P \scong Q }
{ \quotep{P} \nameeq \quotep{Q} }
\end{mathpar}

The astute reader will have noticed that the mutual recursion of names
and processes imposes a mutual recursion on alpha-equivalence and
structural equivalence via name-equivalence. Fortunately, all of this
works out pleasantly and we may calculate in the natural way, free of
concern. The reader interested in the details is referred to the
appendix \ref{appendix:rho_details}.

\subsection{Substitution}

We use $\Proc$ for the set of processes, $\QProc$ for the set of
names, and $\id{\{}\vec{y} / \vec{x} \id{\}}$ to denote partial maps,
$s : \QProc \rightarrow \QProc$. A map, $s$ lifts, uniquely, to a map
on process terms, $\widehat{s} : \Proc \rightarrow \Proc$ by the
following equations.

\begin{mathpar}
  (0) \psubstp{Q}{P} := 0 \\
  (R \juxtap S) \psubstp{Q}{P}
  :=    
  (R)\psubstp{Q}{P} \juxtap (S) \psubstp{Q}{P} \\
  (x?(y).R) \psubstp{Q}{P}    
  :=    
  (x)\substp{Q}{P} (z)\concat( (R \psubstn{z}{y}) \psubstp{Q}{P} ) \\
  (\lift{x}{R}) \psubstp{Q}{P}  
  :=
  \lift{(x)\substp{Q}{P}}{ R \psubstp{Q}{P} } \\
%   (\dropn{x})  \psubstp{Q}{P}       
%   := 
%   \left\{ 
%     \begin{array}{ccc} 
%       \dropn{\quotep{Q}} & & x \nameeq \quotep{P} \\
%       \dropn{x} & & otherwise \\
%     \end{array}
%   \right. 
  (\dropn{x})  \psubstp{Q}{P}       
  := 
  \left\{ 
    \begin{array}{ccc} 
      Q & & x \nameeq \quotep{P} \\
      \dropn{x} & & otherwise \\
    \end{array}
  \right.
\end{mathpar}
 

where

\begin{eqnarray}
  (x)\id{\{} \lpquote Q \rpquote / \lpquote P \rpquote \id{\}}            = 
  \left\{ 
    \begin{array}{ccc}
      \lpquote Q \rpquote & & x \nameeq \lpquote P \rpquote \\
      x & & otherwise \\
    \end{array}
  \right. \nonumber
\end{eqnarray}

and $z$ is chosen distinct from $\quotep{P}$, $\quotep{Q}$, the free
names in $Q$, and all the names in $R$. Our $\alpha$-equivalence will
be built in the standard way from this substitution.

\begin{remark}\label{rem:no_self_referential_names}
  One consequence of these definitions is that $\forall P. \quotep{P}
  \not\in \freenames{P}$.
\end{remark}

\subsection{ Dynamic quote: an example }

Anticipating something of what's to come, consider applying the
substitution, $\widehat{\id{\{}u / z \id{\}}}$, to the following pair
of processes, $\lift{w}{y!(z)}$ and $w[ \lpquote y!(z) \rpquote ]$.

\begin{eqnarray}
	\lift{w}{y!(z)}\widehat{\id{\{}u / z \id{\}}}
		& = &
		\lift{w}{y!(u)} \nonumber\\
	w[ \lpquote y!(z) \rpquote ] \widehat{ \id{\{}u / z \id{\}} }
		& = &
		w[ \lpquote y!(z) \rpquote ] \nonumber
\end{eqnarray}

Because the body of the process between quotes is impervious to
substitution, we get radically different answers. In fact, by
examining the first process in an input context,
e.g. $x?(z).\lift{w}{y!(z)}$, we see that the process under the lift
operator may be shaped by prefixed inputs binding a name inside it. In
this sense, the lift operator will be seen as a way to dynamically
construct processes before reifying them as names.

Finally equipped with these standard features we can present the
dynamics of the calculus.

\subsubsection{Operational semantics} 

Finally, we introduce the computational dynamics. What marks these
algebras as distinct from other more traditionally studied algebraic
structures, e.g. vector spaces or polynomial rings, is the manner in
which dynamics is captured. In traditional structures, dynamics is typically
expressed through morphisms between such structures, as in linear maps
between vector spaces or morphisms between rings. In algebras
associated with the semantics of computation, the dynamics is
expressed as part of the algebraic structure itself, through a
reduction reduction relation typically denoted by $\red$. Below, we
give a recursive presentation of this relation for the calculus used
in the encoding.

$\red \subseteq \pi \times \pi$
$\red : \pi \to \mathcal{P}(\pi)$

\begin{mathpar}
  \inferrule* [lab=Comm] { \textsf{match}( x_{src}, x_{trgt} ) } { x_{trgt}?(y)P \; | \; x_{src}!\langle {Q} \rangle \red P\{\quotep{Q}/y}\} }
  \and \\
  \inferrule* [lab=Par] {{P} \red {P}'} {{{P} | {Q}} \red {{P}' | {Q}}}
  \and
  \inferrule* [lab=Equiv]{{{P} \scong {P}'} \andalso {{P}' \red {Q}'} \andalso {{Q}' \scong {Q}}}{{P} \red {Q}}
\end{mathpar}

\begin{eqnarray*}
  match_{\equiv} (\quotep{P},\quotep{Q}) & := & P \equiv Q \\
  match_{\dagger}(\quotep{P},\quotep{Q}) & := & \forall R. P|Q \red^{*} R => R \red^{*} 0 \\
  match_{K}(\quotep{P},\quotep{Q}) & := & K \mbox{ for some context } K
\end{eqnarray*}

$u?(x)P | u!\langle Q \rangle \red P\{\quotep{Q}/x\}$

%We write $\wred$ for $\red^*$, and $P\red$ if $\exists Q $ such that $ P \red Q$.
We write $P\red$ if $\exists Q $ such that $ P \red Q$ and $P\not\red$, otherwise.

\section{Replication}

As mentioned before, it is known that replication (and hence
recursion) can be implemented in a higher-order process algebra
\cite{SangiorgiWalker}. As our first example of calculation with the
machinery thus far presented we give the construction explicitly in
the {\rhoc}.

\begin{eqnarray}
	D_{x} & := & \prefix{x}{y}{(\binpar{\outputp{x}{y}}{@{y}})} \nonumber\\
	\bangp_{x}{P} & := & \binpar{{x}!\langle{\binpar{D_{x}}{P}}\rangle}{D_{x}} \nonumber
\end{eqnarray}

\begin{eqnarray}
	\bangp_{x}{P} & & \nonumber\\
	=
	& {x}!\langle{(\prefix{x}{y}{(\outputp{x}{y} | @{y})) | P}}\rangle 
	      | \prefix{x}{y}{(\outputp{x}{y} | @{y})} & \nonumber\\
	\red
	& (\outputp{x}{y} | @{y})\substn{\quotep{(\prefix{x}{y}{(@{y} | \outputp{x}{y})) | P}}}{y} & \nonumber\\
	=
	& \outputp{x}{\quotep{(\prefix{x}{y}{(\outputp{x}{y} | @{y})) | P}}}
	  | {(\prefix{x}{y}{(\outputp{x}{y} | @{y})) | P}} & \nonumber\\
	\red
	& \ldots & \nonumber\\
	\red^*
	& P | P | \ldots & \nonumber
\end{eqnarray}

Of course, this encoding, as an implementation, runs away, unfolding
$\bangp{P}$ eagerly. A lazier and more implementable replication
operator, restricted to input-guarded processes, may be obtained as follows.

\begin{eqnarray}
\bangp{\prefix{u}{v}{P}} 
	:= 
	\binpar{\lift{x}{\prefix{u}{v}{(\binpar{D(x)}{P})}}}{D(x)} \nonumber
\end{eqnarray}

\begin{remark}
  Note that the lazier definition still does not deal with summation
  or mixed summation (i.e. sums over input and output). The reader is
  invited to construct definitions of replication that deal with these
  features. 

  Further, the definitions are parameterized in a name, $x$. Can you,
  gentle reader, make a definition that eliminates this parameter and
  guarantees no accidental interaction between the replication
  machinery and the process being replicated -- i.e. no accidental
  sharing of names used by the process to get its work done and the
  name(s) used by the replication to effect copying. This latter
  revision of the definition of replication is crucial to obtaining
  the expected identity $!!P \sim !P$.
\end{remark}

\begin{remark}\label{rem:paradoxical_combinator}
  The reader familiar with the lambda calculus will have noticed the
  similarity between $D$ and the paradoxical combinator.

  [Ed. note: the existence of this seems to suggest we have to be more
  restrictive on the set of processes and names we admit if we are to
  support no-cloning.]
\end{remark}

\subsubsection{Bisimulation}

The computational dynamics gives rise to another kind of equivalence,
the equivalence of computational behavior. As previously mentioned
this is typically captured \emph{via} some form of bisimulation.

% The notion we use in this paper is weak barbed bisimulation
% \cite{milner91polyadicpi}.

The notion we use in this paper is derived from weak barbed
bisimulation \cite{milner91polyadicpi}. 

\begin{definition}
An \emph{observation relation}, $\downarrow_{\mathcal N}$, over a set
of names, $\mathcal N$, is the smallest relation satisfying the rules
below.

\infrule[Out-barb]{y \in {\mathcal N}, \; x \nameeq y}
		  {\outputp{x}{v} \downarrow_{\mathcal N} x}
\infrule[Par-barb]{\mbox{$P\downarrow_{\mathcal N} x$ or $Q\downarrow_{\mathcal N} x$}}
		  {\binpar{P}{Q} \downarrow_{\mathcal N} x}

We write $P \Downarrow_{\mathcal N} x$ if there is $Q$ such that 
$P \wred Q$ and $Q \downarrow_{\mathcal N} x$.
\end{definition}

\begin{definition}
%\label{def.bbisim}
An  ${\mathcal N}$-\emph{barbed bisimulation} over a set of names, ${\mathcal N}$, is a symmetric binary relation 
${\mathcal S}_{\mathcal N}$ between agents such that $P\rel{S}_{\mathcal N}Q$ implies:
\begin{enumerate}
\item If $P \red P'$ then $Q \wred Q'$ and $P'\rel{S}_{\mathcal N} Q'$.
\item If $P\downarrow_{\mathcal N} x$, then $Q\Downarrow_{\mathcal N} x$.
\end{enumerate}
$P$ is ${\mathcal N}$-barbed bisimilar to $Q$, written
$P \wbbisim_{\mathcal N} Q$, if $P \rel{S}_{\mathcal N} Q$ for some ${\mathcal N}$-barbed bisimulation ${\mathcal S}_{\mathcal N}$.
\end{definition}

$\mathcal{R} \subseteq \pi \times \pi$

$P \mathcal{R} Q => \forall P'. P \red P' \Rightarrow \exists Q'. Q \red Q', P' \mathcal{R} Q'$

$P \vdash x \Rightarrow Q \vdash x$

\begin{mathpar}
  \inferrule*[lab=Out-barb]{x \nameeq y}{{y}!\langle{Q}\rangle \vdash x}
  \and
  \inferrule*[lab=Par-barb]{\mbox{$P\vdash x$ or $Q\vdash x$}}{\binpar{P}{Q} \vdash x}
\end{mathpar}

\subsubsection{Contexts}

One of the principle advantages of computational calculi like the
$\pi$-calculus is a well-defined notion of context,
contextual-equivalence and a correlation between
contextual-equivalence and notions of bisimulation. The notion of
context allows the decomposition of a process into (sub-)process and
its syntactic environment, its context. Thus, a context may be
thought of as a process with a ``hole'' (written $\Box$) in it. The
application of a context $M$ to a process $P$, written $M[P]$, is
tantamount to filling the hole in $M$ with $P$. In this paper we do
not need the full weight of this theory, but do make use of the notion
of context in the proof the main theorem. 

\begin{mathpar}
  \inferrule* [lab=summation] {} {{M_{M},M_{N}} \bc \Box \;|\; x.M_{A} \;|\; M_{M}+M_{N}}
  \and
  \inferrule* [lab=agent] {} {{M_{A}} \bc (\vec{x})M_{P} \;| \; \clift{P_0,\ldots,M_{P},\ldots,P_N}}
  \and \\
  \inferrule* [lab=process] {} {{M_{P}} \bc M_{N} \;| \;P|M_{P} }
\end{mathpar} 

\begin{mathpar}
  \inferrule* [lab=sychronization] {} {M_{N} \bc \Box \;|\; x?M_{F} \;|\; x!M_{C}}
  \and
  \inferrule* [lab=abstraction] {} {{M_{F}} \bc (x)M_{P} }
  \and
  \inferrule* [lab=concretion] {} {{M_{C}} \bc \langle M_{P} \rangle }
  \and \\
  \inferrule* [lab=process] {} {{M_{P}} \bc M_{N} \;| \;P|M_{P} }
\end{mathpar}

\begin{definition}[contextual application] Given a context $M$, and
  process $P$, we define the \emph{contextual application}, $M[P] :=
  M\{P/\Box\}$. That is, the contextual application of M to P is the
  substitution of $P$ for $\Box$ in $M$.
\end{definition}

$\meaningof{-} : L \to \mathcal{P}(\pi)$

\begin{mathpar}
  \inferrule* [lab=collection] {} {\meaningof{true} = \pi, \and \meaningof{~E} = \pi \setminus \meaningof{E}, \and \meaningof{E_{1} \& E_{2}} = \meaningof{E_{1}} \cap \meaningof{E_{2}}}
\end{mathpar}

\begin{mathpar}
  \inferrule* [lab=structure] {} {\meaningof{0} = \{ P \in \pi | P \equiv 0 \}, \and \\ \meaningof{E_1 | E_2} = \{ P \in \pi | P \equiv P_{1} | P_{2}, P_{1} \in \meaningof{E_{1}}, P_{2} \in \meaningof{E_2}\} }
\end{mathpar}

\begin{mathpar}
 \inferrule* [lab=behavior] {} {\meaningof{\langle a?b \rangle E} = \{ P \in \pi | P \equiv Q | u?(y)P', \\ \and \\\\ \and \\ \;\;\; u \in \meaningof{a}, \forall z.P'\{z/y\} \in \meaningof{E\{z/b\}}\}, \and \\ \meaningof{a!E} = \{ P \in \pi | P \equiv Q | x!\langle P' \rangle, x \in \meaningof{a} P' \in \meaningof{E}\} }
\end{mathpar}

\begin{mathpar}
 \inferrule* [lab=nominal] {} {\meaningof{\quotep{E}} = \{ \quotep{P} \in \quotep{\pi} | P \in \meaningof{E} \}, \and \meaningof{\quotep{P}} = \{ \quotep{Q} \in \quotep{\pi} | P \equiv Q \} \and \\ \meaningof{@\quotep{E}} = \{ P \in \pi | P \equiv @x, x \in \meaningof{E} \}}
\end{mathpar}

\begin{eqnarray*}
  \\
  \meaningof{-} : TS \to ST
\end{eqnarray*}

\begin{eqnarray*}
  \\
  L : TS \to ST
\end{eqnarray*}

\begin{eqnarray*}
  \\
  P \models E \iff P \in \meaningof{E}
\end{eqnarray*}

\begin{eqnarray*}
  P \approx_{L} Q \iff \forall E \in L. P \models E \iff Q \models E
\end{eqnarray*}

\begin{eqnarray*}
  P \approx_{K} Q
\end{eqnarray*}

\begin{eqnarray*}
  P \approx Q
\end{eqnarray*}

$\approx_{K} = \approx = \approx_{L}$

\subsubsection{Contextual duality}

Note that contexts extend the quotation operation to a family of
operations from processes to names. Given a context, $M$, we can
define a \emph{nominal context}, $\quotep{M}$ by $\quotep{M}[P] :=
\quotep{M[P]}$. To foreshadow what is to come we observe that these
operations enjoy a duality with processes very much like the duality
between vectors and maps from vectors to scalars.

Further, because the calculus is essentially higher-order, we have a
correspondence between contexts and processes. More specifically,
given a name $x$ and a context $M$ we can construct $M^{*}_{x}$ such
that 

\begin{mathpar}
  M^{*}_{x} | \lift{x}{P} \red M[P]
\end{mathpar}

namely,

\begin{mathpar}
  M^{*}_{x} := x?(u).M[\dropn{u}]
\end{mathpar}

The dependence of $M^{*}_{x}$ on a name makes it an abstraction, 

\begin{mathpar}
  M^{*} := (x)x?(u).M[\dropn{u}]
\end{mathpar}

\subsection{Additional notation}

It will sometimes be convenient to denote the process a name
quotes. We already have the notation $x = \quotep{P}$, but it will be
convenient to introduce an alternate notation, $\procn{x}$, when we
want to emphasize the connection to the use of the name. Note that, by
virtue of name equivalence, $\quotep{\procn{x}} \nameeq x$; so, the
notation is consistent with previous definitions.

Further, because names have structure it is possible to effect
substitutions on the basis of that structure. This means we need to
upgrade our notation for substitutions, which we accomplish by
adapting comprehension notation. Thus,

\begin{mathpar}
  P\{ y / x : x \in S \}
\end{mathpar}

is interpreted to mean the process derived from P by replacing (in a
capture-avoiding manner) each occurrence of $x$ in $S$ by $y$. For example,

\begin{mathpar}
  P\{ \quotep{\procn{x}|\procn{x}} / x : x \in \freenames{P} \}
\end{mathpar}

will replace each (occurrence) of a free name $x$ in $P$ by
$\quotep{\procn{x}|\procn{x}}$.

Also, we will avail ourselves of the notation $x^{L}$ and $x^{R}$ to
denote injections of a name into disjoint copies of the name
space. There are numerous ways to accomplish this. One example can be
found in \cite{MeredithR05}. This notation overloads to vectors of
names: $\vec{x}^{\pi} := (x_{i}^{\pi} \; : \; 0 \leq i < |\vec{x}| )$ where $\pi \in \{L,R\}$.

We also use $P^{\Box} := P|\Box$.

In \cite{MeredithR05} an interpretation of the new operator is
given. It turns out that there are several possible interpretations
all enjoying the requisite algebraic properties of the operator (see
\cite{milner91polyadicpi}). We will therefore make liberal use of
$(\nu\; \vec{x})P$.

% subsection the_syntax_and_semantics_of_the_notation_system (end)   

\input{qm2pi.qmops} 

\input{qm2pi.sterngerlach} 

\input{qm2pi.metric} 

% section concurrent_process_calculi (end)

%\input{qm2pi.proofsketch}

% section proof sketch (end)

%\input{qm2pi.slviaknots} 

% section spatial logic via knots (end)

\input{qm2pi.conclusion}

% section conclusion (end)

%\input{qm2pi.dtcodes} 

% section wiring algorithm (end)

\input{qm2pi.ack} 

% section acknowledgments (end)

\newpage


\bibliographystyle{plain}   
\bibliography{../../biblios/main.bib}

\input{qm2pi.rhodetails}

\end{document}



\end{document}

 

%\documentclass[12pt]{llncs}
%\documentclass{jktr}

\usepackage[pdftex]{hyperref}                   
\usepackage {listings}
\usepackage {mathpartir}
\usepackage{bcprules}
%\usepackage{listings}
                       
\usepackage{graphicx} 
%\usepackage[margins=2.5cm,nohead,nofoot]{geometry}
%\usepackage{geometry}
\usepackage{amsfonts}
\usepackage{amstext}
\usepackage{latexsym}
\usepackage{amssymb}
\usepackage{color}


%\include{myPreamble}
\documentclass[12pt]{llncs}
%\documentclass{jktr}

\usepackage[pdftex]{hyperref}                   
\usepackage {listings}
\usepackage {mathpartir}
\usepackage{bcprules}
%\usepackage{listings}
                       
\usepackage{graphicx} 
%\usepackage[margins=2.5cm,nohead,nofoot]{geometry}
%\usepackage{geometry}
\usepackage{amsfonts}
\usepackage{amstext}
\usepackage{latexsym}
\usepackage{amssymb}
\usepackage{color}


%\include{myPreamble}
\include{qm2pi.local} 

%\ifpdf
%\usepackage[pdftex]{graphicx}
%\else
%\usepackage{graphicx}
%\fi

 % \ifpdf
%  \usepackage{pdfsync}
%  \if


%\title{Brief Article}
%\author{David F. Snyder}
%\author{L.G. Meredith}

%\address{Dept. of Math., Texas State University--San Marcos, San Marcos, TX 78666}
       
\pagestyle{empty}


\begin{document}

\lstset{language=[Objective]Caml,frame=shadowbox}

\input{qm2pi.front}

% section front matter (end)

\input{qm2pi.intro} 
 
% section introduction (end)

% \input{qm2pi.knotations} 

% section notation (end)

\input{qm2pi.process.calculi} 

% section concurrent_process_calculi_and_spatial_logics_ (end)
    
%\input{qm2pi.knots2pi} 

%\input{qm2pi.trefoil} 

%\input{qm2pi.mainthm} 

% subsection basic_interpretation (end)

%\input{qm2pi.rho.presentation} 
\subsection{The syntax and semantics of the notation system}\label{sub:the_syntax_and_semantics_of_the_notation_system} % (fold)

We now summarize a technical presentation of the calculus that
embodies our theory of dynamics. The typical presentation of such a
calculus follows the style of giving generators and relations on
them. The grammar, below, describing term constructors, freely
generates the set of processes, $\Proc$. This set is then quotiented
by a relation known as structural congruence and it is over this set
that the notion of dynamics is expressed. This presentation is
essentially that of \cite{MeredithR05} with the addition of
polyadicity and summation. For readability we have relegated some of
the technical subtleties to an appendix.

\subsubsection{Process grammar}\label{subsub:process_grammar}

\begin{mathpar}
  \inferrule* [lab=synchronization] {} {{M} \bc \pzero \;|\; x?F \;|\; x!C }
  \and
  \inferrule* [lab=abstraction] {} {{F} \bc (x)P}
  \and
  \inferrule* [lab=concretion] {} {{C} \bc \langle Q \rangle}
  \and
  \inferrule* [lab=process] {} {{P,Q} \bc M \;| \;P|Q \;|\; @{x}}
  \and
  \inferrule* [lab=name] {} {{x} \bc \quotep{P}}
\end{mathpar} 

Note that $\vec{x}$ (resp. $\vec{P}$) denotes a vector of names
(resp. processes) of length $|\vec{x}|$ (resp. $|\vec{P}|$). We adopt
the following useful abbreviations.

\begin{mathpar}
   x?(\vec{y}).P := x.(\vec{y})P \and  x\clift{\vec{P}} := x.\clift{\vec{P}}
   \and x!(y) := \lift{x}{\dropn{y}}
   \and \Pi_{i=0}^{n-1}P_i := P_0 | \ldots | P_{n-1}
\end{mathpar}

\subsubsection{Structural congruence}

\paragraph{Free and bound names and alpha-equivalence.} At the
core of structural equivalence is alpha-equivalence which identifies
process that are the same up to a change of variable. Formally, we
recognize the distinction between free and bound names. The free names
of a process, $\freenames{P}$, may be calculated recursively as
follows:

\begin{mathpar}
\freenames{\pzero} := \emptyset
  \and \\
  \freenames{x?(y).P} := \{ x \} \cup (\freenames{P} \setminus \{ y \})
  \and 
  \freenames{x!\langle P \rangle} := \{ x \} \cup \{ P \} 
  \and \\
  \freenames{P|Q} := \freenames{P} \cup \freenames{Q}
  \and \\
  \freenames{@{x}} := \{ x \}
\end{mathpar}

$\pi$
$\quotep{\pi}$

$\freenames{-} : \pi \to \mathcal{P}(\quotep{\pi})$

\begin{eqnarray*}
  \freenames{\pzero} & := & \emptyset \\
  \freenames{x?(y).P} & := & \{ x \} \cup (\freenames{P} \setminus \{ y \}) \\
  \freenames{x!\langle P \rangle} & := & \{ x \} \cup \{ P \} \\
  \freenames{P|Q} & := & \freenames{P} \cup \freenames{Q} \\
  \freenames{\dropn{x}} & := & \{ x \}
\end{eqnarray*}

The bound names of a process, $\boundnames{P}$, are those names occurring in $P$
that are not free. For example, in $x?(y).0$, the name $x$ is free, while $y$ is bound.

\begin{mathpar}
  \inferrule* [lab=monoidal-laws] {} { P|Q \equiv Q|P \and P|0 \equiv P \and P|(Q|R) \equiv (P|Q)|R }
\end{mathpar}

\begin{mathpar}
  \inferrule* [lab=alpha-equivalence] {} { (x)P \equiv (y)P\{y/x\} \and y \not\in \freenames{P} }
\end{mathpar}

\begin{definition}
Then two processes, $P,Q$, are alpha-equivalent if $P = Q\{\vec{y}/\vec{x}\}$ for
some $\vec{x} \in \boundnames{Q},\vec{y} \in \boundnames{P}$, where $Q\{\vec{y}/\vec{x}\}$
denotes the capture-avoiding substitution of $\vec{y}$ for $\vec{x}$ in $Q$.
\end{definition}

\begin{definition}
  The {\em structural congruence} \cite{SangiorgiWalker} , $\equiv$,
  between processes is the least congruence containing
  alpha-equivalence, satisfying the abelian monoid laws
  (associativity, commutativity and $\pzero$ as identity) for parallel
  composition $|$ and for summation $+$.
\end{definition}

\subsection{Name equivalence}

We take name equivalence, written $\nameeq$, to be the smallest
equivalence relation generated by the following rules.

\begin{mathpar}
\inferrule*[lab=Quote-drop]
{ }
{ \quotep{@{x}} \nameeq x }

\inferrule*[lab=Struct-equiv]
{ P \scong Q }
{ \quotep{P} \nameeq \quotep{Q} }
\end{mathpar}

The astute reader will have noticed that the mutual recursion of names
and processes imposes a mutual recursion on alpha-equivalence and
structural equivalence via name-equivalence. Fortunately, all of this
works out pleasantly and we may calculate in the natural way, free of
concern. The reader interested in the details is referred to the
appendix \ref{appendix:rho_details}.

\subsection{Substitution}

We use $\Proc$ for the set of processes, $\QProc$ for the set of
names, and $\id{\{}\vec{y} / \vec{x} \id{\}}$ to denote partial maps,
$s : \QProc \rightarrow \QProc$. A map, $s$ lifts, uniquely, to a map
on process terms, $\widehat{s} : \Proc \rightarrow \Proc$ by the
following equations.

\begin{mathpar}
  (0) \psubstp{Q}{P} := 0 \\
  (R \juxtap S) \psubstp{Q}{P}
  :=    
  (R)\psubstp{Q}{P} \juxtap (S) \psubstp{Q}{P} \\
  (x?(y).R) \psubstp{Q}{P}    
  :=    
  (x)\substp{Q}{P} (z)\concat( (R \psubstn{z}{y}) \psubstp{Q}{P} ) \\
  (\lift{x}{R}) \psubstp{Q}{P}  
  :=
  \lift{(x)\substp{Q}{P}}{ R \psubstp{Q}{P} } \\
%   (\dropn{x})  \psubstp{Q}{P}       
%   := 
%   \left\{ 
%     \begin{array}{ccc} 
%       \dropn{\quotep{Q}} & & x \nameeq \quotep{P} \\
%       \dropn{x} & & otherwise \\
%     \end{array}
%   \right. 
  (\dropn{x})  \psubstp{Q}{P}       
  := 
  \left\{ 
    \begin{array}{ccc} 
      Q & & x \nameeq \quotep{P} \\
      \dropn{x} & & otherwise \\
    \end{array}
  \right.
\end{mathpar}
 

where

\begin{eqnarray}
  (x)\id{\{} \lpquote Q \rpquote / \lpquote P \rpquote \id{\}}            = 
  \left\{ 
    \begin{array}{ccc}
      \lpquote Q \rpquote & & x \nameeq \lpquote P \rpquote \\
      x & & otherwise \\
    \end{array}
  \right. \nonumber
\end{eqnarray}

and $z$ is chosen distinct from $\quotep{P}$, $\quotep{Q}$, the free
names in $Q$, and all the names in $R$. Our $\alpha$-equivalence will
be built in the standard way from this substitution.

\begin{remark}\label{rem:no_self_referential_names}
  One consequence of these definitions is that $\forall P. \quotep{P}
  \not\in \freenames{P}$.
\end{remark}

\subsection{ Dynamic quote: an example }

Anticipating something of what's to come, consider applying the
substitution, $\widehat{\id{\{}u / z \id{\}}}$, to the following pair
of processes, $\lift{w}{y!(z)}$ and $w[ \lpquote y!(z) \rpquote ]$.

\begin{eqnarray}
	\lift{w}{y!(z)}\widehat{\id{\{}u / z \id{\}}}
		& = &
		\lift{w}{y!(u)} \nonumber\\
	w[ \lpquote y!(z) \rpquote ] \widehat{ \id{\{}u / z \id{\}} }
		& = &
		w[ \lpquote y!(z) \rpquote ] \nonumber
\end{eqnarray}

Because the body of the process between quotes is impervious to
substitution, we get radically different answers. In fact, by
examining the first process in an input context,
e.g. $x?(z).\lift{w}{y!(z)}$, we see that the process under the lift
operator may be shaped by prefixed inputs binding a name inside it. In
this sense, the lift operator will be seen as a way to dynamically
construct processes before reifying them as names.

Finally equipped with these standard features we can present the
dynamics of the calculus.

\subsubsection{Operational semantics} 

Finally, we introduce the computational dynamics. What marks these
algebras as distinct from other more traditionally studied algebraic
structures, e.g. vector spaces or polynomial rings, is the manner in
which dynamics is captured. In traditional structures, dynamics is typically
expressed through morphisms between such structures, as in linear maps
between vector spaces or morphisms between rings. In algebras
associated with the semantics of computation, the dynamics is
expressed as part of the algebraic structure itself, through a
reduction reduction relation typically denoted by $\red$. Below, we
give a recursive presentation of this relation for the calculus used
in the encoding.

$\red \subseteq \pi \times \pi$
$\red : \pi \to \mathcal{P}(\pi)$

\begin{mathpar}
  \inferrule* [lab=Comm] { \textsf{match}( x_{src}, x_{trgt} ) } { x_{trgt}?(y)P \; | \; x_{src}!\langle {Q} \rangle \red P\{\quotep{Q}/y}\} }
  \and \\
  \inferrule* [lab=Par] {{P} \red {P}'} {{{P} | {Q}} \red {{P}' | {Q}}}
  \and
  \inferrule* [lab=Equiv]{{{P} \scong {P}'} \andalso {{P}' \red {Q}'} \andalso {{Q}' \scong {Q}}}{{P} \red {Q}}
\end{mathpar}

\begin{eqnarray*}
  match_{\equiv} (\quotep{P},\quotep{Q}) & := & P \equiv Q \\
  match_{\dagger}(\quotep{P},\quotep{Q}) & := & \forall R. P|Q \red^{*} R => R \red^{*} 0 \\
  match_{K}(\quotep{P},\quotep{Q}) & := & K \mbox{ for some context } K
\end{eqnarray*}

$u?(x)P | u!\langle Q \rangle \red P\{\quotep{Q}/x\}$

%We write $\wred$ for $\red^*$, and $P\red$ if $\exists Q $ such that $ P \red Q$.
We write $P\red$ if $\exists Q $ such that $ P \red Q$ and $P\not\red$, otherwise.

\section{Replication}

As mentioned before, it is known that replication (and hence
recursion) can be implemented in a higher-order process algebra
\cite{SangiorgiWalker}. As our first example of calculation with the
machinery thus far presented we give the construction explicitly in
the {\rhoc}.

\begin{eqnarray}
	D_{x} & := & \prefix{x}{y}{(\binpar{\outputp{x}{y}}{@{y}})} \nonumber\\
	\bangp_{x}{P} & := & \binpar{{x}!\langle{\binpar{D_{x}}{P}}\rangle}{D_{x}} \nonumber
\end{eqnarray}

\begin{eqnarray}
	\bangp_{x}{P} & & \nonumber\\
	=
	& {x}!\langle{(\prefix{x}{y}{(\outputp{x}{y} | @{y})) | P}}\rangle 
	      | \prefix{x}{y}{(\outputp{x}{y} | @{y})} & \nonumber\\
	\red
	& (\outputp{x}{y} | @{y})\substn{\quotep{(\prefix{x}{y}{(@{y} | \outputp{x}{y})) | P}}}{y} & \nonumber\\
	=
	& \outputp{x}{\quotep{(\prefix{x}{y}{(\outputp{x}{y} | @{y})) | P}}}
	  | {(\prefix{x}{y}{(\outputp{x}{y} | @{y})) | P}} & \nonumber\\
	\red
	& \ldots & \nonumber\\
	\red^*
	& P | P | \ldots & \nonumber
\end{eqnarray}

Of course, this encoding, as an implementation, runs away, unfolding
$\bangp{P}$ eagerly. A lazier and more implementable replication
operator, restricted to input-guarded processes, may be obtained as follows.

\begin{eqnarray}
\bangp{\prefix{u}{v}{P}} 
	:= 
	\binpar{\lift{x}{\prefix{u}{v}{(\binpar{D(x)}{P})}}}{D(x)} \nonumber
\end{eqnarray}

\begin{remark}
  Note that the lazier definition still does not deal with summation
  or mixed summation (i.e. sums over input and output). The reader is
  invited to construct definitions of replication that deal with these
  features. 

  Further, the definitions are parameterized in a name, $x$. Can you,
  gentle reader, make a definition that eliminates this parameter and
  guarantees no accidental interaction between the replication
  machinery and the process being replicated -- i.e. no accidental
  sharing of names used by the process to get its work done and the
  name(s) used by the replication to effect copying. This latter
  revision of the definition of replication is crucial to obtaining
  the expected identity $!!P \sim !P$.
\end{remark}

\begin{remark}\label{rem:paradoxical_combinator}
  The reader familiar with the lambda calculus will have noticed the
  similarity between $D$ and the paradoxical combinator.

  [Ed. note: the existence of this seems to suggest we have to be more
  restrictive on the set of processes and names we admit if we are to
  support no-cloning.]
\end{remark}

\subsubsection{Bisimulation}

The computational dynamics gives rise to another kind of equivalence,
the equivalence of computational behavior. As previously mentioned
this is typically captured \emph{via} some form of bisimulation.

% The notion we use in this paper is weak barbed bisimulation
% \cite{milner91polyadicpi}.

The notion we use in this paper is derived from weak barbed
bisimulation \cite{milner91polyadicpi}. 

\begin{definition}
An \emph{observation relation}, $\downarrow_{\mathcal N}$, over a set
of names, $\mathcal N$, is the smallest relation satisfying the rules
below.

\infrule[Out-barb]{y \in {\mathcal N}, \; x \nameeq y}
		  {\outputp{x}{v} \downarrow_{\mathcal N} x}
\infrule[Par-barb]{\mbox{$P\downarrow_{\mathcal N} x$ or $Q\downarrow_{\mathcal N} x$}}
		  {\binpar{P}{Q} \downarrow_{\mathcal N} x}

We write $P \Downarrow_{\mathcal N} x$ if there is $Q$ such that 
$P \wred Q$ and $Q \downarrow_{\mathcal N} x$.
\end{definition}

\begin{definition}
%\label{def.bbisim}
An  ${\mathcal N}$-\emph{barbed bisimulation} over a set of names, ${\mathcal N}$, is a symmetric binary relation 
${\mathcal S}_{\mathcal N}$ between agents such that $P\rel{S}_{\mathcal N}Q$ implies:
\begin{enumerate}
\item If $P \red P'$ then $Q \wred Q'$ and $P'\rel{S}_{\mathcal N} Q'$.
\item If $P\downarrow_{\mathcal N} x$, then $Q\Downarrow_{\mathcal N} x$.
\end{enumerate}
$P$ is ${\mathcal N}$-barbed bisimilar to $Q$, written
$P \wbbisim_{\mathcal N} Q$, if $P \rel{S}_{\mathcal N} Q$ for some ${\mathcal N}$-barbed bisimulation ${\mathcal S}_{\mathcal N}$.
\end{definition}

$\mathcal{R} \subseteq \pi \times \pi$

$P \mathcal{R} Q => \forall P'. P \red P' \Rightarrow \exists Q'. Q \red Q', P' \mathcal{R} Q'$

$P \vdash x \Rightarrow Q \vdash x$

\begin{mathpar}
  \inferrule*[lab=Out-barb]{x \nameeq y}{{y}!\langle{Q}\rangle \vdash x}
  \and
  \inferrule*[lab=Par-barb]{\mbox{$P\vdash x$ or $Q\vdash x$}}{\binpar{P}{Q} \vdash x}
\end{mathpar}

\subsubsection{Contexts}

One of the principle advantages of computational calculi like the
$\pi$-calculus is a well-defined notion of context,
contextual-equivalence and a correlation between
contextual-equivalence and notions of bisimulation. The notion of
context allows the decomposition of a process into (sub-)process and
its syntactic environment, its context. Thus, a context may be
thought of as a process with a ``hole'' (written $\Box$) in it. The
application of a context $M$ to a process $P$, written $M[P]$, is
tantamount to filling the hole in $M$ with $P$. In this paper we do
not need the full weight of this theory, but do make use of the notion
of context in the proof the main theorem. 

\begin{mathpar}
  \inferrule* [lab=summation] {} {{M_{M},M_{N}} \bc \Box \;|\; x.M_{A} \;|\; M_{M}+M_{N}}
  \and
  \inferrule* [lab=agent] {} {{M_{A}} \bc (\vec{x})M_{P} \;| \; \clift{P_0,\ldots,M_{P},\ldots,P_N}}
  \and \\
  \inferrule* [lab=process] {} {{M_{P}} \bc M_{N} \;| \;P|M_{P} }
\end{mathpar} 

\begin{mathpar}
  \inferrule* [lab=sychronization] {} {M_{N} \bc \Box \;|\; x?M_{F} \;|\; x!M_{C}}
  \and
  \inferrule* [lab=abstraction] {} {{M_{F}} \bc (x)M_{P} }
  \and
  \inferrule* [lab=concretion] {} {{M_{C}} \bc \langle M_{P} \rangle }
  \and \\
  \inferrule* [lab=process] {} {{M_{P}} \bc M_{N} \;| \;P|M_{P} }
\end{mathpar}

\begin{definition}[contextual application] Given a context $M$, and
  process $P$, we define the \emph{contextual application}, $M[P] :=
  M\{P/\Box\}$. That is, the contextual application of M to P is the
  substitution of $P$ for $\Box$ in $M$.
\end{definition}

$\meaningof{-} : L \to \mathcal{P}(\pi)$

\begin{mathpar}
  \inferrule* [lab=collection] {} {\meaningof{true} = \pi, \and \meaningof{~E} = \pi \setminus \meaningof{E}, \and \meaningof{E_{1} \& E_{2}} = \meaningof{E_{1}} \cap \meaningof{E_{2}}}
\end{mathpar}

\begin{mathpar}
  \inferrule* [lab=structure] {} {\meaningof{0} = \{ P \in \pi | P \equiv 0 \}, \and \\ \meaningof{E_1 | E_2} = \{ P \in \pi | P \equiv P_{1} | P_{2}, P_{1} \in \meaningof{E_{1}}, P_{2} \in \meaningof{E_2}\} }
\end{mathpar}

\begin{mathpar}
 \inferrule* [lab=behavior] {} {\meaningof{\langle a?b \rangle E} = \{ P \in \pi | P \equiv Q | u?(y)P', \\ \and \\\\ \and \\ \;\;\; u \in \meaningof{a}, \forall z.P'\{z/y\} \in \meaningof{E\{z/b\}}\}, \and \\ \meaningof{a!E} = \{ P \in \pi | P \equiv Q | x!\langle P' \rangle, x \in \meaningof{a} P' \in \meaningof{E}\} }
\end{mathpar}

\begin{mathpar}
 \inferrule* [lab=nominal] {} {\meaningof{\quotep{E}} = \{ \quotep{P} \in \quotep{\pi} | P \in \meaningof{E} \}, \and \meaningof{\quotep{P}} = \{ \quotep{Q} \in \quotep{\pi} | P \equiv Q \} \and \\ \meaningof{@\quotep{E}} = \{ P \in \pi | P \equiv @x, x \in \meaningof{E} \}}
\end{mathpar}

\begin{eqnarray*}
  \\
  \meaningof{-} : TS \to ST
\end{eqnarray*}

\begin{eqnarray*}
  \\
  L : TS \to ST
\end{eqnarray*}

\begin{eqnarray*}
  \\
  P \models E \iff P \in \meaningof{E}
\end{eqnarray*}

\begin{eqnarray*}
  P \approx_{L} Q \iff \forall E \in L. P \models E \iff Q \models E
\end{eqnarray*}

\begin{eqnarray*}
  P \approx_{K} Q
\end{eqnarray*}

\begin{eqnarray*}
  P \approx Q
\end{eqnarray*}

$\approx_{K} = \approx = \approx_{L}$

\subsubsection{Contextual duality}

Note that contexts extend the quotation operation to a family of
operations from processes to names. Given a context, $M$, we can
define a \emph{nominal context}, $\quotep{M}$ by $\quotep{M}[P] :=
\quotep{M[P]}$. To foreshadow what is to come we observe that these
operations enjoy a duality with processes very much like the duality
between vectors and maps from vectors to scalars.

Further, because the calculus is essentially higher-order, we have a
correspondence between contexts and processes. More specifically,
given a name $x$ and a context $M$ we can construct $M^{*}_{x}$ such
that 

\begin{mathpar}
  M^{*}_{x} | \lift{x}{P} \red M[P]
\end{mathpar}

namely,

\begin{mathpar}
  M^{*}_{x} := x?(u).M[\dropn{u}]
\end{mathpar}

The dependence of $M^{*}_{x}$ on a name makes it an abstraction, 

\begin{mathpar}
  M^{*} := (x)x?(u).M[\dropn{u}]
\end{mathpar}

\subsection{Additional notation}

It will sometimes be convenient to denote the process a name
quotes. We already have the notation $x = \quotep{P}$, but it will be
convenient to introduce an alternate notation, $\procn{x}$, when we
want to emphasize the connection to the use of the name. Note that, by
virtue of name equivalence, $\quotep{\procn{x}} \nameeq x$; so, the
notation is consistent with previous definitions.

Further, because names have structure it is possible to effect
substitutions on the basis of that structure. This means we need to
upgrade our notation for substitutions, which we accomplish by
adapting comprehension notation. Thus,

\begin{mathpar}
  P\{ y / x : x \in S \}
\end{mathpar}

is interpreted to mean the process derived from P by replacing (in a
capture-avoiding manner) each occurrence of $x$ in $S$ by $y$. For example,

\begin{mathpar}
  P\{ \quotep{\procn{x}|\procn{x}} / x : x \in \freenames{P} \}
\end{mathpar}

will replace each (occurrence) of a free name $x$ in $P$ by
$\quotep{\procn{x}|\procn{x}}$.

Also, we will avail ourselves of the notation $x^{L}$ and $x^{R}$ to
denote injections of a name into disjoint copies of the name
space. There are numerous ways to accomplish this. One example can be
found in \cite{MeredithR05}. This notation overloads to vectors of
names: $\vec{x}^{\pi} := (x_{i}^{\pi} \; : \; 0 \leq i < |\vec{x}| )$ where $\pi \in \{L,R\}$.

We also use $P^{\Box} := P|\Box$.

In \cite{MeredithR05} an interpretation of the new operator is
given. It turns out that there are several possible interpretations
all enjoying the requisite algebraic properties of the operator (see
\cite{milner91polyadicpi}). We will therefore make liberal use of
$(\nu\; \vec{x})P$.

% subsection the_syntax_and_semantics_of_the_notation_system (end)   

\input{qm2pi.qmops} 

\input{qm2pi.sterngerlach} 

\input{qm2pi.metric} 

% section concurrent_process_calculi (end)

%\input{qm2pi.proofsketch}

% section proof sketch (end)

%\input{qm2pi.slviaknots} 

% section spatial logic via knots (end)

\input{qm2pi.conclusion}

% section conclusion (end)

%\input{qm2pi.dtcodes} 

% section wiring algorithm (end)

\input{qm2pi.ack} 

% section acknowledgments (end)

\newpage


\bibliographystyle{plain}   
\bibliography{../../biblios/main.bib}

\input{qm2pi.rhodetails}

\end{document}

 

%\ifpdf
%\usepackage[pdftex]{graphicx}
%\else
%\usepackage{graphicx}
%\fi

 % \ifpdf
%  \usepackage{pdfsync}
%  \if


%\title{Brief Article}
%\author{David F. Snyder}
%\author{L.G. Meredith}

%\address{Dept. of Math., Texas State University--San Marcos, San Marcos, TX 78666}
       
\pagestyle{empty}


\begin{document}

\lstset{language=[Objective]Caml,frame=shadowbox}

\documentclass[12pt]{llncs}
%\documentclass{jktr}

\usepackage[pdftex]{hyperref}                   
\usepackage {listings}
\usepackage {mathpartir}
\usepackage{bcprules}
%\usepackage{listings}
                       
\usepackage{graphicx} 
%\usepackage[margins=2.5cm,nohead,nofoot]{geometry}
%\usepackage{geometry}
\usepackage{amsfonts}
\usepackage{amstext}
\usepackage{latexsym}
\usepackage{amssymb}
\usepackage{color}


%\include{myPreamble}
\include{qm2pi.local} 

%\ifpdf
%\usepackage[pdftex]{graphicx}
%\else
%\usepackage{graphicx}
%\fi

 % \ifpdf
%  \usepackage{pdfsync}
%  \if


%\title{Brief Article}
%\author{David F. Snyder}
%\author{L.G. Meredith}

%\address{Dept. of Math., Texas State University--San Marcos, San Marcos, TX 78666}
       
\pagestyle{empty}


\begin{document}

\lstset{language=[Objective]Caml,frame=shadowbox}

\input{qm2pi.front}

% section front matter (end)

\input{qm2pi.intro} 
 
% section introduction (end)

% \input{qm2pi.knotations} 

% section notation (end)

\input{qm2pi.process.calculi} 

% section concurrent_process_calculi_and_spatial_logics_ (end)
    
%\input{qm2pi.knots2pi} 

%\input{qm2pi.trefoil} 

%\input{qm2pi.mainthm} 

% subsection basic_interpretation (end)

%\input{qm2pi.rho.presentation} 
\subsection{The syntax and semantics of the notation system}\label{sub:the_syntax_and_semantics_of_the_notation_system} % (fold)

We now summarize a technical presentation of the calculus that
embodies our theory of dynamics. The typical presentation of such a
calculus follows the style of giving generators and relations on
them. The grammar, below, describing term constructors, freely
generates the set of processes, $\Proc$. This set is then quotiented
by a relation known as structural congruence and it is over this set
that the notion of dynamics is expressed. This presentation is
essentially that of \cite{MeredithR05} with the addition of
polyadicity and summation. For readability we have relegated some of
the technical subtleties to an appendix.

\subsubsection{Process grammar}\label{subsub:process_grammar}

\begin{mathpar}
  \inferrule* [lab=synchronization] {} {{M} \bc \pzero \;|\; x?F \;|\; x!C }
  \and
  \inferrule* [lab=abstraction] {} {{F} \bc (x)P}
  \and
  \inferrule* [lab=concretion] {} {{C} \bc \langle Q \rangle}
  \and
  \inferrule* [lab=process] {} {{P,Q} \bc M \;| \;P|Q \;|\; @{x}}
  \and
  \inferrule* [lab=name] {} {{x} \bc \quotep{P}}
\end{mathpar} 

Note that $\vec{x}$ (resp. $\vec{P}$) denotes a vector of names
(resp. processes) of length $|\vec{x}|$ (resp. $|\vec{P}|$). We adopt
the following useful abbreviations.

\begin{mathpar}
   x?(\vec{y}).P := x.(\vec{y})P \and  x\clift{\vec{P}} := x.\clift{\vec{P}}
   \and x!(y) := \lift{x}{\dropn{y}}
   \and \Pi_{i=0}^{n-1}P_i := P_0 | \ldots | P_{n-1}
\end{mathpar}

\subsubsection{Structural congruence}

\paragraph{Free and bound names and alpha-equivalence.} At the
core of structural equivalence is alpha-equivalence which identifies
process that are the same up to a change of variable. Formally, we
recognize the distinction between free and bound names. The free names
of a process, $\freenames{P}$, may be calculated recursively as
follows:

\begin{mathpar}
\freenames{\pzero} := \emptyset
  \and \\
  \freenames{x?(y).P} := \{ x \} \cup (\freenames{P} \setminus \{ y \})
  \and 
  \freenames{x!\langle P \rangle} := \{ x \} \cup \{ P \} 
  \and \\
  \freenames{P|Q} := \freenames{P} \cup \freenames{Q}
  \and \\
  \freenames{@{x}} := \{ x \}
\end{mathpar}

$\pi$
$\quotep{\pi}$

$\freenames{-} : \pi \to \mathcal{P}(\quotep{\pi})$

\begin{eqnarray*}
  \freenames{\pzero} & := & \emptyset \\
  \freenames{x?(y).P} & := & \{ x \} \cup (\freenames{P} \setminus \{ y \}) \\
  \freenames{x!\langle P \rangle} & := & \{ x \} \cup \{ P \} \\
  \freenames{P|Q} & := & \freenames{P} \cup \freenames{Q} \\
  \freenames{\dropn{x}} & := & \{ x \}
\end{eqnarray*}

The bound names of a process, $\boundnames{P}$, are those names occurring in $P$
that are not free. For example, in $x?(y).0$, the name $x$ is free, while $y$ is bound.

\begin{mathpar}
  \inferrule* [lab=monoidal-laws] {} { P|Q \equiv Q|P \and P|0 \equiv P \and P|(Q|R) \equiv (P|Q)|R }
\end{mathpar}

\begin{mathpar}
  \inferrule* [lab=alpha-equivalence] {} { (x)P \equiv (y)P\{y/x\} \and y \not\in \freenames{P} }
\end{mathpar}

\begin{definition}
Then two processes, $P,Q$, are alpha-equivalent if $P = Q\{\vec{y}/\vec{x}\}$ for
some $\vec{x} \in \boundnames{Q},\vec{y} \in \boundnames{P}$, where $Q\{\vec{y}/\vec{x}\}$
denotes the capture-avoiding substitution of $\vec{y}$ for $\vec{x}$ in $Q$.
\end{definition}

\begin{definition}
  The {\em structural congruence} \cite{SangiorgiWalker} , $\equiv$,
  between processes is the least congruence containing
  alpha-equivalence, satisfying the abelian monoid laws
  (associativity, commutativity and $\pzero$ as identity) for parallel
  composition $|$ and for summation $+$.
\end{definition}

\subsection{Name equivalence}

We take name equivalence, written $\nameeq$, to be the smallest
equivalence relation generated by the following rules.

\begin{mathpar}
\inferrule*[lab=Quote-drop]
{ }
{ \quotep{@{x}} \nameeq x }

\inferrule*[lab=Struct-equiv]
{ P \scong Q }
{ \quotep{P} \nameeq \quotep{Q} }
\end{mathpar}

The astute reader will have noticed that the mutual recursion of names
and processes imposes a mutual recursion on alpha-equivalence and
structural equivalence via name-equivalence. Fortunately, all of this
works out pleasantly and we may calculate in the natural way, free of
concern. The reader interested in the details is referred to the
appendix \ref{appendix:rho_details}.

\subsection{Substitution}

We use $\Proc$ for the set of processes, $\QProc$ for the set of
names, and $\id{\{}\vec{y} / \vec{x} \id{\}}$ to denote partial maps,
$s : \QProc \rightarrow \QProc$. A map, $s$ lifts, uniquely, to a map
on process terms, $\widehat{s} : \Proc \rightarrow \Proc$ by the
following equations.

\begin{mathpar}
  (0) \psubstp{Q}{P} := 0 \\
  (R \juxtap S) \psubstp{Q}{P}
  :=    
  (R)\psubstp{Q}{P} \juxtap (S) \psubstp{Q}{P} \\
  (x?(y).R) \psubstp{Q}{P}    
  :=    
  (x)\substp{Q}{P} (z)\concat( (R \psubstn{z}{y}) \psubstp{Q}{P} ) \\
  (\lift{x}{R}) \psubstp{Q}{P}  
  :=
  \lift{(x)\substp{Q}{P}}{ R \psubstp{Q}{P} } \\
%   (\dropn{x})  \psubstp{Q}{P}       
%   := 
%   \left\{ 
%     \begin{array}{ccc} 
%       \dropn{\quotep{Q}} & & x \nameeq \quotep{P} \\
%       \dropn{x} & & otherwise \\
%     \end{array}
%   \right. 
  (\dropn{x})  \psubstp{Q}{P}       
  := 
  \left\{ 
    \begin{array}{ccc} 
      Q & & x \nameeq \quotep{P} \\
      \dropn{x} & & otherwise \\
    \end{array}
  \right.
\end{mathpar}
 

where

\begin{eqnarray}
  (x)\id{\{} \lpquote Q \rpquote / \lpquote P \rpquote \id{\}}            = 
  \left\{ 
    \begin{array}{ccc}
      \lpquote Q \rpquote & & x \nameeq \lpquote P \rpquote \\
      x & & otherwise \\
    \end{array}
  \right. \nonumber
\end{eqnarray}

and $z$ is chosen distinct from $\quotep{P}$, $\quotep{Q}$, the free
names in $Q$, and all the names in $R$. Our $\alpha$-equivalence will
be built in the standard way from this substitution.

\begin{remark}\label{rem:no_self_referential_names}
  One consequence of these definitions is that $\forall P. \quotep{P}
  \not\in \freenames{P}$.
\end{remark}

\subsection{ Dynamic quote: an example }

Anticipating something of what's to come, consider applying the
substitution, $\widehat{\id{\{}u / z \id{\}}}$, to the following pair
of processes, $\lift{w}{y!(z)}$ and $w[ \lpquote y!(z) \rpquote ]$.

\begin{eqnarray}
	\lift{w}{y!(z)}\widehat{\id{\{}u / z \id{\}}}
		& = &
		\lift{w}{y!(u)} \nonumber\\
	w[ \lpquote y!(z) \rpquote ] \widehat{ \id{\{}u / z \id{\}} }
		& = &
		w[ \lpquote y!(z) \rpquote ] \nonumber
\end{eqnarray}

Because the body of the process between quotes is impervious to
substitution, we get radically different answers. In fact, by
examining the first process in an input context,
e.g. $x?(z).\lift{w}{y!(z)}$, we see that the process under the lift
operator may be shaped by prefixed inputs binding a name inside it. In
this sense, the lift operator will be seen as a way to dynamically
construct processes before reifying them as names.

Finally equipped with these standard features we can present the
dynamics of the calculus.

\subsubsection{Operational semantics} 

Finally, we introduce the computational dynamics. What marks these
algebras as distinct from other more traditionally studied algebraic
structures, e.g. vector spaces or polynomial rings, is the manner in
which dynamics is captured. In traditional structures, dynamics is typically
expressed through morphisms between such structures, as in linear maps
between vector spaces or morphisms between rings. In algebras
associated with the semantics of computation, the dynamics is
expressed as part of the algebraic structure itself, through a
reduction reduction relation typically denoted by $\red$. Below, we
give a recursive presentation of this relation for the calculus used
in the encoding.

$\red \subseteq \pi \times \pi$
$\red : \pi \to \mathcal{P}(\pi)$

\begin{mathpar}
  \inferrule* [lab=Comm] { \textsf{match}( x_{src}, x_{trgt} ) } { x_{trgt}?(y)P \; | \; x_{src}!\langle {Q} \rangle \red P\{\quotep{Q}/y}\} }
  \and \\
  \inferrule* [lab=Par] {{P} \red {P}'} {{{P} | {Q}} \red {{P}' | {Q}}}
  \and
  \inferrule* [lab=Equiv]{{{P} \scong {P}'} \andalso {{P}' \red {Q}'} \andalso {{Q}' \scong {Q}}}{{P} \red {Q}}
\end{mathpar}

\begin{eqnarray*}
  match_{\equiv} (\quotep{P},\quotep{Q}) & := & P \equiv Q \\
  match_{\dagger}(\quotep{P},\quotep{Q}) & := & \forall R. P|Q \red^{*} R => R \red^{*} 0 \\
  match_{K}(\quotep{P},\quotep{Q}) & := & K \mbox{ for some context } K
\end{eqnarray*}

$u?(x)P | u!\langle Q \rangle \red P\{\quotep{Q}/x\}$

%We write $\wred$ for $\red^*$, and $P\red$ if $\exists Q $ such that $ P \red Q$.
We write $P\red$ if $\exists Q $ such that $ P \red Q$ and $P\not\red$, otherwise.

\section{Replication}

As mentioned before, it is known that replication (and hence
recursion) can be implemented in a higher-order process algebra
\cite{SangiorgiWalker}. As our first example of calculation with the
machinery thus far presented we give the construction explicitly in
the {\rhoc}.

\begin{eqnarray}
	D_{x} & := & \prefix{x}{y}{(\binpar{\outputp{x}{y}}{@{y}})} \nonumber\\
	\bangp_{x}{P} & := & \binpar{{x}!\langle{\binpar{D_{x}}{P}}\rangle}{D_{x}} \nonumber
\end{eqnarray}

\begin{eqnarray}
	\bangp_{x}{P} & & \nonumber\\
	=
	& {x}!\langle{(\prefix{x}{y}{(\outputp{x}{y} | @{y})) | P}}\rangle 
	      | \prefix{x}{y}{(\outputp{x}{y} | @{y})} & \nonumber\\
	\red
	& (\outputp{x}{y} | @{y})\substn{\quotep{(\prefix{x}{y}{(@{y} | \outputp{x}{y})) | P}}}{y} & \nonumber\\
	=
	& \outputp{x}{\quotep{(\prefix{x}{y}{(\outputp{x}{y} | @{y})) | P}}}
	  | {(\prefix{x}{y}{(\outputp{x}{y} | @{y})) | P}} & \nonumber\\
	\red
	& \ldots & \nonumber\\
	\red^*
	& P | P | \ldots & \nonumber
\end{eqnarray}

Of course, this encoding, as an implementation, runs away, unfolding
$\bangp{P}$ eagerly. A lazier and more implementable replication
operator, restricted to input-guarded processes, may be obtained as follows.

\begin{eqnarray}
\bangp{\prefix{u}{v}{P}} 
	:= 
	\binpar{\lift{x}{\prefix{u}{v}{(\binpar{D(x)}{P})}}}{D(x)} \nonumber
\end{eqnarray}

\begin{remark}
  Note that the lazier definition still does not deal with summation
  or mixed summation (i.e. sums over input and output). The reader is
  invited to construct definitions of replication that deal with these
  features. 

  Further, the definitions are parameterized in a name, $x$. Can you,
  gentle reader, make a definition that eliminates this parameter and
  guarantees no accidental interaction between the replication
  machinery and the process being replicated -- i.e. no accidental
  sharing of names used by the process to get its work done and the
  name(s) used by the replication to effect copying. This latter
  revision of the definition of replication is crucial to obtaining
  the expected identity $!!P \sim !P$.
\end{remark}

\begin{remark}\label{rem:paradoxical_combinator}
  The reader familiar with the lambda calculus will have noticed the
  similarity between $D$ and the paradoxical combinator.

  [Ed. note: the existence of this seems to suggest we have to be more
  restrictive on the set of processes and names we admit if we are to
  support no-cloning.]
\end{remark}

\subsubsection{Bisimulation}

The computational dynamics gives rise to another kind of equivalence,
the equivalence of computational behavior. As previously mentioned
this is typically captured \emph{via} some form of bisimulation.

% The notion we use in this paper is weak barbed bisimulation
% \cite{milner91polyadicpi}.

The notion we use in this paper is derived from weak barbed
bisimulation \cite{milner91polyadicpi}. 

\begin{definition}
An \emph{observation relation}, $\downarrow_{\mathcal N}$, over a set
of names, $\mathcal N$, is the smallest relation satisfying the rules
below.

\infrule[Out-barb]{y \in {\mathcal N}, \; x \nameeq y}
		  {\outputp{x}{v} \downarrow_{\mathcal N} x}
\infrule[Par-barb]{\mbox{$P\downarrow_{\mathcal N} x$ or $Q\downarrow_{\mathcal N} x$}}
		  {\binpar{P}{Q} \downarrow_{\mathcal N} x}

We write $P \Downarrow_{\mathcal N} x$ if there is $Q$ such that 
$P \wred Q$ and $Q \downarrow_{\mathcal N} x$.
\end{definition}

\begin{definition}
%\label{def.bbisim}
An  ${\mathcal N}$-\emph{barbed bisimulation} over a set of names, ${\mathcal N}$, is a symmetric binary relation 
${\mathcal S}_{\mathcal N}$ between agents such that $P\rel{S}_{\mathcal N}Q$ implies:
\begin{enumerate}
\item If $P \red P'$ then $Q \wred Q'$ and $P'\rel{S}_{\mathcal N} Q'$.
\item If $P\downarrow_{\mathcal N} x$, then $Q\Downarrow_{\mathcal N} x$.
\end{enumerate}
$P$ is ${\mathcal N}$-barbed bisimilar to $Q$, written
$P \wbbisim_{\mathcal N} Q$, if $P \rel{S}_{\mathcal N} Q$ for some ${\mathcal N}$-barbed bisimulation ${\mathcal S}_{\mathcal N}$.
\end{definition}

$\mathcal{R} \subseteq \pi \times \pi$

$P \mathcal{R} Q => \forall P'. P \red P' \Rightarrow \exists Q'. Q \red Q', P' \mathcal{R} Q'$

$P \vdash x \Rightarrow Q \vdash x$

\begin{mathpar}
  \inferrule*[lab=Out-barb]{x \nameeq y}{{y}!\langle{Q}\rangle \vdash x}
  \and
  \inferrule*[lab=Par-barb]{\mbox{$P\vdash x$ or $Q\vdash x$}}{\binpar{P}{Q} \vdash x}
\end{mathpar}

\subsubsection{Contexts}

One of the principle advantages of computational calculi like the
$\pi$-calculus is a well-defined notion of context,
contextual-equivalence and a correlation between
contextual-equivalence and notions of bisimulation. The notion of
context allows the decomposition of a process into (sub-)process and
its syntactic environment, its context. Thus, a context may be
thought of as a process with a ``hole'' (written $\Box$) in it. The
application of a context $M$ to a process $P$, written $M[P]$, is
tantamount to filling the hole in $M$ with $P$. In this paper we do
not need the full weight of this theory, but do make use of the notion
of context in the proof the main theorem. 

\begin{mathpar}
  \inferrule* [lab=summation] {} {{M_{M},M_{N}} \bc \Box \;|\; x.M_{A} \;|\; M_{M}+M_{N}}
  \and
  \inferrule* [lab=agent] {} {{M_{A}} \bc (\vec{x})M_{P} \;| \; \clift{P_0,\ldots,M_{P},\ldots,P_N}}
  \and \\
  \inferrule* [lab=process] {} {{M_{P}} \bc M_{N} \;| \;P|M_{P} }
\end{mathpar} 

\begin{mathpar}
  \inferrule* [lab=sychronization] {} {M_{N} \bc \Box \;|\; x?M_{F} \;|\; x!M_{C}}
  \and
  \inferrule* [lab=abstraction] {} {{M_{F}} \bc (x)M_{P} }
  \and
  \inferrule* [lab=concretion] {} {{M_{C}} \bc \langle M_{P} \rangle }
  \and \\
  \inferrule* [lab=process] {} {{M_{P}} \bc M_{N} \;| \;P|M_{P} }
\end{mathpar}

\begin{definition}[contextual application] Given a context $M$, and
  process $P$, we define the \emph{contextual application}, $M[P] :=
  M\{P/\Box\}$. That is, the contextual application of M to P is the
  substitution of $P$ for $\Box$ in $M$.
\end{definition}

$\meaningof{-} : L \to \mathcal{P}(\pi)$

\begin{mathpar}
  \inferrule* [lab=collection] {} {\meaningof{true} = \pi, \and \meaningof{~E} = \pi \setminus \meaningof{E}, \and \meaningof{E_{1} \& E_{2}} = \meaningof{E_{1}} \cap \meaningof{E_{2}}}
\end{mathpar}

\begin{mathpar}
  \inferrule* [lab=structure] {} {\meaningof{0} = \{ P \in \pi | P \equiv 0 \}, \and \\ \meaningof{E_1 | E_2} = \{ P \in \pi | P \equiv P_{1} | P_{2}, P_{1} \in \meaningof{E_{1}}, P_{2} \in \meaningof{E_2}\} }
\end{mathpar}

\begin{mathpar}
 \inferrule* [lab=behavior] {} {\meaningof{\langle a?b \rangle E} = \{ P \in \pi | P \equiv Q | u?(y)P', \\ \and \\\\ \and \\ \;\;\; u \in \meaningof{a}, \forall z.P'\{z/y\} \in \meaningof{E\{z/b\}}\}, \and \\ \meaningof{a!E} = \{ P \in \pi | P \equiv Q | x!\langle P' \rangle, x \in \meaningof{a} P' \in \meaningof{E}\} }
\end{mathpar}

\begin{mathpar}
 \inferrule* [lab=nominal] {} {\meaningof{\quotep{E}} = \{ \quotep{P} \in \quotep{\pi} | P \in \meaningof{E} \}, \and \meaningof{\quotep{P}} = \{ \quotep{Q} \in \quotep{\pi} | P \equiv Q \} \and \\ \meaningof{@\quotep{E}} = \{ P \in \pi | P \equiv @x, x \in \meaningof{E} \}}
\end{mathpar}

\begin{eqnarray*}
  \\
  \meaningof{-} : TS \to ST
\end{eqnarray*}

\begin{eqnarray*}
  \\
  L : TS \to ST
\end{eqnarray*}

\begin{eqnarray*}
  \\
  P \models E \iff P \in \meaningof{E}
\end{eqnarray*}

\begin{eqnarray*}
  P \approx_{L} Q \iff \forall E \in L. P \models E \iff Q \models E
\end{eqnarray*}

\begin{eqnarray*}
  P \approx_{K} Q
\end{eqnarray*}

\begin{eqnarray*}
  P \approx Q
\end{eqnarray*}

$\approx_{K} = \approx = \approx_{L}$

\subsubsection{Contextual duality}

Note that contexts extend the quotation operation to a family of
operations from processes to names. Given a context, $M$, we can
define a \emph{nominal context}, $\quotep{M}$ by $\quotep{M}[P] :=
\quotep{M[P]}$. To foreshadow what is to come we observe that these
operations enjoy a duality with processes very much like the duality
between vectors and maps from vectors to scalars.

Further, because the calculus is essentially higher-order, we have a
correspondence between contexts and processes. More specifically,
given a name $x$ and a context $M$ we can construct $M^{*}_{x}$ such
that 

\begin{mathpar}
  M^{*}_{x} | \lift{x}{P} \red M[P]
\end{mathpar}

namely,

\begin{mathpar}
  M^{*}_{x} := x?(u).M[\dropn{u}]
\end{mathpar}

The dependence of $M^{*}_{x}$ on a name makes it an abstraction, 

\begin{mathpar}
  M^{*} := (x)x?(u).M[\dropn{u}]
\end{mathpar}

\subsection{Additional notation}

It will sometimes be convenient to denote the process a name
quotes. We already have the notation $x = \quotep{P}$, but it will be
convenient to introduce an alternate notation, $\procn{x}$, when we
want to emphasize the connection to the use of the name. Note that, by
virtue of name equivalence, $\quotep{\procn{x}} \nameeq x$; so, the
notation is consistent with previous definitions.

Further, because names have structure it is possible to effect
substitutions on the basis of that structure. This means we need to
upgrade our notation for substitutions, which we accomplish by
adapting comprehension notation. Thus,

\begin{mathpar}
  P\{ y / x : x \in S \}
\end{mathpar}

is interpreted to mean the process derived from P by replacing (in a
capture-avoiding manner) each occurrence of $x$ in $S$ by $y$. For example,

\begin{mathpar}
  P\{ \quotep{\procn{x}|\procn{x}} / x : x \in \freenames{P} \}
\end{mathpar}

will replace each (occurrence) of a free name $x$ in $P$ by
$\quotep{\procn{x}|\procn{x}}$.

Also, we will avail ourselves of the notation $x^{L}$ and $x^{R}$ to
denote injections of a name into disjoint copies of the name
space. There are numerous ways to accomplish this. One example can be
found in \cite{MeredithR05}. This notation overloads to vectors of
names: $\vec{x}^{\pi} := (x_{i}^{\pi} \; : \; 0 \leq i < |\vec{x}| )$ where $\pi \in \{L,R\}$.

We also use $P^{\Box} := P|\Box$.

In \cite{MeredithR05} an interpretation of the new operator is
given. It turns out that there are several possible interpretations
all enjoying the requisite algebraic properties of the operator (see
\cite{milner91polyadicpi}). We will therefore make liberal use of
$(\nu\; \vec{x})P$.

% subsection the_syntax_and_semantics_of_the_notation_system (end)   

\input{qm2pi.qmops} 

\input{qm2pi.sterngerlach} 

\input{qm2pi.metric} 

% section concurrent_process_calculi (end)

%\input{qm2pi.proofsketch}

% section proof sketch (end)

%\input{qm2pi.slviaknots} 

% section spatial logic via knots (end)

\input{qm2pi.conclusion}

% section conclusion (end)

%\input{qm2pi.dtcodes} 

% section wiring algorithm (end)

\input{qm2pi.ack} 

% section acknowledgments (end)

\newpage


\bibliographystyle{plain}   
\bibliography{../../biblios/main.bib}

\input{qm2pi.rhodetails}

\end{document}



% section front matter (end)

\section{Introduction}\label{sec:introduction} % (fold)
In this draft of the material i am going to have to dispense with the
usual writing conventions adopted in papers on these topics. i'm going
to have adopt whatever tone i need at the time i'm writing up the
calculations. Sometimes this may be very conversational; others it may
be the barest mathematical grunts; others still it may be that i have
lifted text from one of my other papers because the exposition of some
point was better said there. i hope that my readers are not unduly put
out by this decision. i'm not doing this to flout convention or be
rebellious. i find these calculations very technically challenging. To
keep everything going technically, something has to give; i have to
let go of some cognitive burden. So, the academic writing style --
with all of its trade-offs in terms of facilitating technical
communication -- is what i'm letting go of. Perhaps subsequent drafts
can be tightened and polished, but for now, i'm going to speak as if
we were sitting together in a coffee shop with a laptop, wifi and a
pad of paper and a pencil.

So, here's what i have to say. We -- you and i, comfortably ensconced
in our coffee shop and well-equipped with our tools -- can realize and
carry out the calculations of quantum mechanics over a very different
formal theory of dynamics, a formal theory of dynamics that
corresponds to a theory of concurrent computation with
\emph{reflection}. It has the advantage that the underlying theory is
already `quantized', but supports analogues all of the continuuous
operations. Strikingly, this underlying theory has recently been
connected with a notion of metric that we can show, by calculating
together, coincides with the metric induced by the inner product.

There are a lot of reasons why you might be interested in seeing
calculations of this form. Here's why i'm interested. For the past
several centuries there has been no competitor to the ``Newtonian''
account of dynamics. As a result the predominant share of accounts of
dynamical systems and situations have had to be formulated in terms of
the Newtonian machinery. i view this as an intellectually dangerous
position to occupy. Everything, despite it's intrinsic shape, turns
into a nail to be hit with this hammer. Recently, however, the theory
of computation has matured to the point where we have candidates for
theories of dynamics that offer very different perspective on
reasoning about dynamical systems and situations. Testing these
candidates against very successful accounts of dynamical situations,
like quantum mechanics, is going to give us some sense of how mature
they are and some measure of the quality of these accounts of
dynamics.

\subsection{Summary of contributions and outline of paper}

So, we're going to develop an interpretation of the operations of
quantum mechanics normally interpreted by Hilbert spaces and
operators. We're going to do this over a theory of computation. Note
that this is very different than the usual quantum computation program
which develops notions of computation over quantum mechanics. Rather,
we are developing a story that aligns with Wheeler's slogan: It from
Bit. To do this we will first provide an account of the theory of
computation at play here. Then we will dive into a calculation-driven
interpretation of the operations of quantum mechanics.

The reason we take this approach is that -- until very recently --
there hasn't been an axiomatic account of quantum mechanics. As a
result there has been no sharp delineation of the mathematical theory
supporting interpretation of the physical theory and the physical
theory, itself. So, ambient features of the maths are free to be
exploited (or supressed) without a real accounting of their physical
relevance. There is no sharp statement ``here's the physical theory''
qua \emph{theory} and ``here's the mathematical interpretation''
enabling a judgment of how faithful the interpretation is -- apart
from experimental observation. When there is an axiomatic account we
can judge how well a given mathematical formalism supports an
interpretation of the axioms, independent of
experimentation. Likewise, we can judge how well we have captured our
physical evidence and experience with our axiomatics, independent of
any specific mathematical implementation, with accidental detail that
may or may not have physical significance. 

In lieu of a fully fleshed out and vetted axiomatic account of quantum
mechanics, interpreting the operational notions in service of modeling
physical systems will have to suffice. In other words, we are not in
the business of providing a model of Hilbert spaces and operators. We
are in the business of providing a model of quantum mechanics because
we are motivated by testing our notions of dynamics against physical
theory; and, the predictive calculations of the physical theory must
serve as the best formulation -- shy of a fully fleshed out axiomatic
account -- of the physical theory itself (as they have for scientific
theories since time immemorial). Put another way, despite a
whole-hearted commitment to an It-from-Bit ontology, we are firmly
aligned with the shut-up-and-calculate camp as the best way to obtain
results either from the physical perspective or as a quality assurance
measure of our fledgling theory of dynamics.

In detail, we present a reflective process calculus. Then we develop
intuitive correspondences between the notions available in this
calculus and the usual physical notions supporting quantum mechanical
calculations. Thus, 

\begin{table}[htp]
  \center{
    \fbox{
      \begin{tabular}{c|c}
        quantum mechanics & process calculus \\
        \hline
        scalar & name \\
        state vector & process \\
        dual & contextual duals \\
        matrix & formal sums of process-context-dual pairs \\
        orthogonality & process annihilation \\
        inner product & execution-formula + quoting
      \end{tabular}
    }
  }
  \caption{QM - process calculi correspondences}
\end{table}

Then we tighten up these intuitions to operational definitions. We
employ the Dirac notation as the best proxy we can find for an
abstract syntax of the quantum mechanical notions. The definitions we
develop put us in contact with equational constraints coming from the
theory that we demonstrate the definitions and calculations satisfy.

This puts us in a position to shut up and calculate for the
Stern-Gerlach experimental set up, showing how these predictive
calculations become calculations on processes in our theory of a
reflective process calculus.

Penultimately, we demonstrate that the notion of metric coming from
the inner product coincides with the notion of metric available from
the theory of bisimulation. This demonstration gives us the right to
think of space as arising from behavior. Finally, we consider where we
might go from the new vantage point we have obtained.

% section introduction (end) 
 
% section introduction (end)

% \documentclass[12pt]{llncs}
%\documentclass{jktr}

\usepackage[pdftex]{hyperref}                   
\usepackage {listings}
\usepackage {mathpartir}
\usepackage{bcprules}
%\usepackage{listings}
                       
\usepackage{graphicx} 
%\usepackage[margins=2.5cm,nohead,nofoot]{geometry}
%\usepackage{geometry}
\usepackage{amsfonts}
\usepackage{amstext}
\usepackage{latexsym}
\usepackage{amssymb}
\usepackage{color}


%\include{myPreamble}
\include{qm2pi.local} 

%\ifpdf
%\usepackage[pdftex]{graphicx}
%\else
%\usepackage{graphicx}
%\fi

 % \ifpdf
%  \usepackage{pdfsync}
%  \if


%\title{Brief Article}
%\author{David F. Snyder}
%\author{L.G. Meredith}

%\address{Dept. of Math., Texas State University--San Marcos, San Marcos, TX 78666}
       
\pagestyle{empty}


\begin{document}

\lstset{language=[Objective]Caml,frame=shadowbox}

\input{qm2pi.front}

% section front matter (end)

\input{qm2pi.intro} 
 
% section introduction (end)

% \input{qm2pi.knotations} 

% section notation (end)

\input{qm2pi.process.calculi} 

% section concurrent_process_calculi_and_spatial_logics_ (end)
    
%\input{qm2pi.knots2pi} 

%\input{qm2pi.trefoil} 

%\input{qm2pi.mainthm} 

% subsection basic_interpretation (end)

%\input{qm2pi.rho.presentation} 
\subsection{The syntax and semantics of the notation system}\label{sub:the_syntax_and_semantics_of_the_notation_system} % (fold)

We now summarize a technical presentation of the calculus that
embodies our theory of dynamics. The typical presentation of such a
calculus follows the style of giving generators and relations on
them. The grammar, below, describing term constructors, freely
generates the set of processes, $\Proc$. This set is then quotiented
by a relation known as structural congruence and it is over this set
that the notion of dynamics is expressed. This presentation is
essentially that of \cite{MeredithR05} with the addition of
polyadicity and summation. For readability we have relegated some of
the technical subtleties to an appendix.

\subsubsection{Process grammar}\label{subsub:process_grammar}

\begin{mathpar}
  \inferrule* [lab=synchronization] {} {{M} \bc \pzero \;|\; x?F \;|\; x!C }
  \and
  \inferrule* [lab=abstraction] {} {{F} \bc (x)P}
  \and
  \inferrule* [lab=concretion] {} {{C} \bc \langle Q \rangle}
  \and
  \inferrule* [lab=process] {} {{P,Q} \bc M \;| \;P|Q \;|\; @{x}}
  \and
  \inferrule* [lab=name] {} {{x} \bc \quotep{P}}
\end{mathpar} 

Note that $\vec{x}$ (resp. $\vec{P}$) denotes a vector of names
(resp. processes) of length $|\vec{x}|$ (resp. $|\vec{P}|$). We adopt
the following useful abbreviations.

\begin{mathpar}
   x?(\vec{y}).P := x.(\vec{y})P \and  x\clift{\vec{P}} := x.\clift{\vec{P}}
   \and x!(y) := \lift{x}{\dropn{y}}
   \and \Pi_{i=0}^{n-1}P_i := P_0 | \ldots | P_{n-1}
\end{mathpar}

\subsubsection{Structural congruence}

\paragraph{Free and bound names and alpha-equivalence.} At the
core of structural equivalence is alpha-equivalence which identifies
process that are the same up to a change of variable. Formally, we
recognize the distinction between free and bound names. The free names
of a process, $\freenames{P}$, may be calculated recursively as
follows:

\begin{mathpar}
\freenames{\pzero} := \emptyset
  \and \\
  \freenames{x?(y).P} := \{ x \} \cup (\freenames{P} \setminus \{ y \})
  \and 
  \freenames{x!\langle P \rangle} := \{ x \} \cup \{ P \} 
  \and \\
  \freenames{P|Q} := \freenames{P} \cup \freenames{Q}
  \and \\
  \freenames{@{x}} := \{ x \}
\end{mathpar}

$\pi$
$\quotep{\pi}$

$\freenames{-} : \pi \to \mathcal{P}(\quotep{\pi})$

\begin{eqnarray*}
  \freenames{\pzero} & := & \emptyset \\
  \freenames{x?(y).P} & := & \{ x \} \cup (\freenames{P} \setminus \{ y \}) \\
  \freenames{x!\langle P \rangle} & := & \{ x \} \cup \{ P \} \\
  \freenames{P|Q} & := & \freenames{P} \cup \freenames{Q} \\
  \freenames{\dropn{x}} & := & \{ x \}
\end{eqnarray*}

The bound names of a process, $\boundnames{P}$, are those names occurring in $P$
that are not free. For example, in $x?(y).0$, the name $x$ is free, while $y$ is bound.

\begin{mathpar}
  \inferrule* [lab=monoidal-laws] {} { P|Q \equiv Q|P \and P|0 \equiv P \and P|(Q|R) \equiv (P|Q)|R }
\end{mathpar}

\begin{mathpar}
  \inferrule* [lab=alpha-equivalence] {} { (x)P \equiv (y)P\{y/x\} \and y \not\in \freenames{P} }
\end{mathpar}

\begin{definition}
Then two processes, $P,Q$, are alpha-equivalent if $P = Q\{\vec{y}/\vec{x}\}$ for
some $\vec{x} \in \boundnames{Q},\vec{y} \in \boundnames{P}$, where $Q\{\vec{y}/\vec{x}\}$
denotes the capture-avoiding substitution of $\vec{y}$ for $\vec{x}$ in $Q$.
\end{definition}

\begin{definition}
  The {\em structural congruence} \cite{SangiorgiWalker} , $\equiv$,
  between processes is the least congruence containing
  alpha-equivalence, satisfying the abelian monoid laws
  (associativity, commutativity and $\pzero$ as identity) for parallel
  composition $|$ and for summation $+$.
\end{definition}

\subsection{Name equivalence}

We take name equivalence, written $\nameeq$, to be the smallest
equivalence relation generated by the following rules.

\begin{mathpar}
\inferrule*[lab=Quote-drop]
{ }
{ \quotep{@{x}} \nameeq x }

\inferrule*[lab=Struct-equiv]
{ P \scong Q }
{ \quotep{P} \nameeq \quotep{Q} }
\end{mathpar}

The astute reader will have noticed that the mutual recursion of names
and processes imposes a mutual recursion on alpha-equivalence and
structural equivalence via name-equivalence. Fortunately, all of this
works out pleasantly and we may calculate in the natural way, free of
concern. The reader interested in the details is referred to the
appendix \ref{appendix:rho_details}.

\subsection{Substitution}

We use $\Proc$ for the set of processes, $\QProc$ for the set of
names, and $\id{\{}\vec{y} / \vec{x} \id{\}}$ to denote partial maps,
$s : \QProc \rightarrow \QProc$. A map, $s$ lifts, uniquely, to a map
on process terms, $\widehat{s} : \Proc \rightarrow \Proc$ by the
following equations.

\begin{mathpar}
  (0) \psubstp{Q}{P} := 0 \\
  (R \juxtap S) \psubstp{Q}{P}
  :=    
  (R)\psubstp{Q}{P} \juxtap (S) \psubstp{Q}{P} \\
  (x?(y).R) \psubstp{Q}{P}    
  :=    
  (x)\substp{Q}{P} (z)\concat( (R \psubstn{z}{y}) \psubstp{Q}{P} ) \\
  (\lift{x}{R}) \psubstp{Q}{P}  
  :=
  \lift{(x)\substp{Q}{P}}{ R \psubstp{Q}{P} } \\
%   (\dropn{x})  \psubstp{Q}{P}       
%   := 
%   \left\{ 
%     \begin{array}{ccc} 
%       \dropn{\quotep{Q}} & & x \nameeq \quotep{P} \\
%       \dropn{x} & & otherwise \\
%     \end{array}
%   \right. 
  (\dropn{x})  \psubstp{Q}{P}       
  := 
  \left\{ 
    \begin{array}{ccc} 
      Q & & x \nameeq \quotep{P} \\
      \dropn{x} & & otherwise \\
    \end{array}
  \right.
\end{mathpar}
 

where

\begin{eqnarray}
  (x)\id{\{} \lpquote Q \rpquote / \lpquote P \rpquote \id{\}}            = 
  \left\{ 
    \begin{array}{ccc}
      \lpquote Q \rpquote & & x \nameeq \lpquote P \rpquote \\
      x & & otherwise \\
    \end{array}
  \right. \nonumber
\end{eqnarray}

and $z$ is chosen distinct from $\quotep{P}$, $\quotep{Q}$, the free
names in $Q$, and all the names in $R$. Our $\alpha$-equivalence will
be built in the standard way from this substitution.

\begin{remark}\label{rem:no_self_referential_names}
  One consequence of these definitions is that $\forall P. \quotep{P}
  \not\in \freenames{P}$.
\end{remark}

\subsection{ Dynamic quote: an example }

Anticipating something of what's to come, consider applying the
substitution, $\widehat{\id{\{}u / z \id{\}}}$, to the following pair
of processes, $\lift{w}{y!(z)}$ and $w[ \lpquote y!(z) \rpquote ]$.

\begin{eqnarray}
	\lift{w}{y!(z)}\widehat{\id{\{}u / z \id{\}}}
		& = &
		\lift{w}{y!(u)} \nonumber\\
	w[ \lpquote y!(z) \rpquote ] \widehat{ \id{\{}u / z \id{\}} }
		& = &
		w[ \lpquote y!(z) \rpquote ] \nonumber
\end{eqnarray}

Because the body of the process between quotes is impervious to
substitution, we get radically different answers. In fact, by
examining the first process in an input context,
e.g. $x?(z).\lift{w}{y!(z)}$, we see that the process under the lift
operator may be shaped by prefixed inputs binding a name inside it. In
this sense, the lift operator will be seen as a way to dynamically
construct processes before reifying them as names.

Finally equipped with these standard features we can present the
dynamics of the calculus.

\subsubsection{Operational semantics} 

Finally, we introduce the computational dynamics. What marks these
algebras as distinct from other more traditionally studied algebraic
structures, e.g. vector spaces or polynomial rings, is the manner in
which dynamics is captured. In traditional structures, dynamics is typically
expressed through morphisms between such structures, as in linear maps
between vector spaces or morphisms between rings. In algebras
associated with the semantics of computation, the dynamics is
expressed as part of the algebraic structure itself, through a
reduction reduction relation typically denoted by $\red$. Below, we
give a recursive presentation of this relation for the calculus used
in the encoding.

$\red \subseteq \pi \times \pi$
$\red : \pi \to \mathcal{P}(\pi)$

\begin{mathpar}
  \inferrule* [lab=Comm] { \textsf{match}( x_{src}, x_{trgt} ) } { x_{trgt}?(y)P \; | \; x_{src}!\langle {Q} \rangle \red P\{\quotep{Q}/y}\} }
  \and \\
  \inferrule* [lab=Par] {{P} \red {P}'} {{{P} | {Q}} \red {{P}' | {Q}}}
  \and
  \inferrule* [lab=Equiv]{{{P} \scong {P}'} \andalso {{P}' \red {Q}'} \andalso {{Q}' \scong {Q}}}{{P} \red {Q}}
\end{mathpar}

\begin{eqnarray*}
  match_{\equiv} (\quotep{P},\quotep{Q}) & := & P \equiv Q \\
  match_{\dagger}(\quotep{P},\quotep{Q}) & := & \forall R. P|Q \red^{*} R => R \red^{*} 0 \\
  match_{K}(\quotep{P},\quotep{Q}) & := & K \mbox{ for some context } K
\end{eqnarray*}

$u?(x)P | u!\langle Q \rangle \red P\{\quotep{Q}/x\}$

%We write $\wred$ for $\red^*$, and $P\red$ if $\exists Q $ such that $ P \red Q$.
We write $P\red$ if $\exists Q $ such that $ P \red Q$ and $P\not\red$, otherwise.

\section{Replication}

As mentioned before, it is known that replication (and hence
recursion) can be implemented in a higher-order process algebra
\cite{SangiorgiWalker}. As our first example of calculation with the
machinery thus far presented we give the construction explicitly in
the {\rhoc}.

\begin{eqnarray}
	D_{x} & := & \prefix{x}{y}{(\binpar{\outputp{x}{y}}{@{y}})} \nonumber\\
	\bangp_{x}{P} & := & \binpar{{x}!\langle{\binpar{D_{x}}{P}}\rangle}{D_{x}} \nonumber
\end{eqnarray}

\begin{eqnarray}
	\bangp_{x}{P} & & \nonumber\\
	=
	& {x}!\langle{(\prefix{x}{y}{(\outputp{x}{y} | @{y})) | P}}\rangle 
	      | \prefix{x}{y}{(\outputp{x}{y} | @{y})} & \nonumber\\
	\red
	& (\outputp{x}{y} | @{y})\substn{\quotep{(\prefix{x}{y}{(@{y} | \outputp{x}{y})) | P}}}{y} & \nonumber\\
	=
	& \outputp{x}{\quotep{(\prefix{x}{y}{(\outputp{x}{y} | @{y})) | P}}}
	  | {(\prefix{x}{y}{(\outputp{x}{y} | @{y})) | P}} & \nonumber\\
	\red
	& \ldots & \nonumber\\
	\red^*
	& P | P | \ldots & \nonumber
\end{eqnarray}

Of course, this encoding, as an implementation, runs away, unfolding
$\bangp{P}$ eagerly. A lazier and more implementable replication
operator, restricted to input-guarded processes, may be obtained as follows.

\begin{eqnarray}
\bangp{\prefix{u}{v}{P}} 
	:= 
	\binpar{\lift{x}{\prefix{u}{v}{(\binpar{D(x)}{P})}}}{D(x)} \nonumber
\end{eqnarray}

\begin{remark}
  Note that the lazier definition still does not deal with summation
  or mixed summation (i.e. sums over input and output). The reader is
  invited to construct definitions of replication that deal with these
  features. 

  Further, the definitions are parameterized in a name, $x$. Can you,
  gentle reader, make a definition that eliminates this parameter and
  guarantees no accidental interaction between the replication
  machinery and the process being replicated -- i.e. no accidental
  sharing of names used by the process to get its work done and the
  name(s) used by the replication to effect copying. This latter
  revision of the definition of replication is crucial to obtaining
  the expected identity $!!P \sim !P$.
\end{remark}

\begin{remark}\label{rem:paradoxical_combinator}
  The reader familiar with the lambda calculus will have noticed the
  similarity between $D$ and the paradoxical combinator.

  [Ed. note: the existence of this seems to suggest we have to be more
  restrictive on the set of processes and names we admit if we are to
  support no-cloning.]
\end{remark}

\subsubsection{Bisimulation}

The computational dynamics gives rise to another kind of equivalence,
the equivalence of computational behavior. As previously mentioned
this is typically captured \emph{via} some form of bisimulation.

% The notion we use in this paper is weak barbed bisimulation
% \cite{milner91polyadicpi}.

The notion we use in this paper is derived from weak barbed
bisimulation \cite{milner91polyadicpi}. 

\begin{definition}
An \emph{observation relation}, $\downarrow_{\mathcal N}$, over a set
of names, $\mathcal N$, is the smallest relation satisfying the rules
below.

\infrule[Out-barb]{y \in {\mathcal N}, \; x \nameeq y}
		  {\outputp{x}{v} \downarrow_{\mathcal N} x}
\infrule[Par-barb]{\mbox{$P\downarrow_{\mathcal N} x$ or $Q\downarrow_{\mathcal N} x$}}
		  {\binpar{P}{Q} \downarrow_{\mathcal N} x}

We write $P \Downarrow_{\mathcal N} x$ if there is $Q$ such that 
$P \wred Q$ and $Q \downarrow_{\mathcal N} x$.
\end{definition}

\begin{definition}
%\label{def.bbisim}
An  ${\mathcal N}$-\emph{barbed bisimulation} over a set of names, ${\mathcal N}$, is a symmetric binary relation 
${\mathcal S}_{\mathcal N}$ between agents such that $P\rel{S}_{\mathcal N}Q$ implies:
\begin{enumerate}
\item If $P \red P'$ then $Q \wred Q'$ and $P'\rel{S}_{\mathcal N} Q'$.
\item If $P\downarrow_{\mathcal N} x$, then $Q\Downarrow_{\mathcal N} x$.
\end{enumerate}
$P$ is ${\mathcal N}$-barbed bisimilar to $Q$, written
$P \wbbisim_{\mathcal N} Q$, if $P \rel{S}_{\mathcal N} Q$ for some ${\mathcal N}$-barbed bisimulation ${\mathcal S}_{\mathcal N}$.
\end{definition}

$\mathcal{R} \subseteq \pi \times \pi$

$P \mathcal{R} Q => \forall P'. P \red P' \Rightarrow \exists Q'. Q \red Q', P' \mathcal{R} Q'$

$P \vdash x \Rightarrow Q \vdash x$

\begin{mathpar}
  \inferrule*[lab=Out-barb]{x \nameeq y}{{y}!\langle{Q}\rangle \vdash x}
  \and
  \inferrule*[lab=Par-barb]{\mbox{$P\vdash x$ or $Q\vdash x$}}{\binpar{P}{Q} \vdash x}
\end{mathpar}

\subsubsection{Contexts}

One of the principle advantages of computational calculi like the
$\pi$-calculus is a well-defined notion of context,
contextual-equivalence and a correlation between
contextual-equivalence and notions of bisimulation. The notion of
context allows the decomposition of a process into (sub-)process and
its syntactic environment, its context. Thus, a context may be
thought of as a process with a ``hole'' (written $\Box$) in it. The
application of a context $M$ to a process $P$, written $M[P]$, is
tantamount to filling the hole in $M$ with $P$. In this paper we do
not need the full weight of this theory, but do make use of the notion
of context in the proof the main theorem. 

\begin{mathpar}
  \inferrule* [lab=summation] {} {{M_{M},M_{N}} \bc \Box \;|\; x.M_{A} \;|\; M_{M}+M_{N}}
  \and
  \inferrule* [lab=agent] {} {{M_{A}} \bc (\vec{x})M_{P} \;| \; \clift{P_0,\ldots,M_{P},\ldots,P_N}}
  \and \\
  \inferrule* [lab=process] {} {{M_{P}} \bc M_{N} \;| \;P|M_{P} }
\end{mathpar} 

\begin{mathpar}
  \inferrule* [lab=sychronization] {} {M_{N} \bc \Box \;|\; x?M_{F} \;|\; x!M_{C}}
  \and
  \inferrule* [lab=abstraction] {} {{M_{F}} \bc (x)M_{P} }
  \and
  \inferrule* [lab=concretion] {} {{M_{C}} \bc \langle M_{P} \rangle }
  \and \\
  \inferrule* [lab=process] {} {{M_{P}} \bc M_{N} \;| \;P|M_{P} }
\end{mathpar}

\begin{definition}[contextual application] Given a context $M$, and
  process $P$, we define the \emph{contextual application}, $M[P] :=
  M\{P/\Box\}$. That is, the contextual application of M to P is the
  substitution of $P$ for $\Box$ in $M$.
\end{definition}

$\meaningof{-} : L \to \mathcal{P}(\pi)$

\begin{mathpar}
  \inferrule* [lab=collection] {} {\meaningof{true} = \pi, \and \meaningof{~E} = \pi \setminus \meaningof{E}, \and \meaningof{E_{1} \& E_{2}} = \meaningof{E_{1}} \cap \meaningof{E_{2}}}
\end{mathpar}

\begin{mathpar}
  \inferrule* [lab=structure] {} {\meaningof{0} = \{ P \in \pi | P \equiv 0 \}, \and \\ \meaningof{E_1 | E_2} = \{ P \in \pi | P \equiv P_{1} | P_{2}, P_{1} \in \meaningof{E_{1}}, P_{2} \in \meaningof{E_2}\} }
\end{mathpar}

\begin{mathpar}
 \inferrule* [lab=behavior] {} {\meaningof{\langle a?b \rangle E} = \{ P \in \pi | P \equiv Q | u?(y)P', \\ \and \\\\ \and \\ \;\;\; u \in \meaningof{a}, \forall z.P'\{z/y\} \in \meaningof{E\{z/b\}}\}, \and \\ \meaningof{a!E} = \{ P \in \pi | P \equiv Q | x!\langle P' \rangle, x \in \meaningof{a} P' \in \meaningof{E}\} }
\end{mathpar}

\begin{mathpar}
 \inferrule* [lab=nominal] {} {\meaningof{\quotep{E}} = \{ \quotep{P} \in \quotep{\pi} | P \in \meaningof{E} \}, \and \meaningof{\quotep{P}} = \{ \quotep{Q} \in \quotep{\pi} | P \equiv Q \} \and \\ \meaningof{@\quotep{E}} = \{ P \in \pi | P \equiv @x, x \in \meaningof{E} \}}
\end{mathpar}

\begin{eqnarray*}
  \\
  \meaningof{-} : TS \to ST
\end{eqnarray*}

\begin{eqnarray*}
  \\
  L : TS \to ST
\end{eqnarray*}

\begin{eqnarray*}
  \\
  P \models E \iff P \in \meaningof{E}
\end{eqnarray*}

\begin{eqnarray*}
  P \approx_{L} Q \iff \forall E \in L. P \models E \iff Q \models E
\end{eqnarray*}

\begin{eqnarray*}
  P \approx_{K} Q
\end{eqnarray*}

\begin{eqnarray*}
  P \approx Q
\end{eqnarray*}

$\approx_{K} = \approx = \approx_{L}$

\subsubsection{Contextual duality}

Note that contexts extend the quotation operation to a family of
operations from processes to names. Given a context, $M$, we can
define a \emph{nominal context}, $\quotep{M}$ by $\quotep{M}[P] :=
\quotep{M[P]}$. To foreshadow what is to come we observe that these
operations enjoy a duality with processes very much like the duality
between vectors and maps from vectors to scalars.

Further, because the calculus is essentially higher-order, we have a
correspondence between contexts and processes. More specifically,
given a name $x$ and a context $M$ we can construct $M^{*}_{x}$ such
that 

\begin{mathpar}
  M^{*}_{x} | \lift{x}{P} \red M[P]
\end{mathpar}

namely,

\begin{mathpar}
  M^{*}_{x} := x?(u).M[\dropn{u}]
\end{mathpar}

The dependence of $M^{*}_{x}$ on a name makes it an abstraction, 

\begin{mathpar}
  M^{*} := (x)x?(u).M[\dropn{u}]
\end{mathpar}

\subsection{Additional notation}

It will sometimes be convenient to denote the process a name
quotes. We already have the notation $x = \quotep{P}$, but it will be
convenient to introduce an alternate notation, $\procn{x}$, when we
want to emphasize the connection to the use of the name. Note that, by
virtue of name equivalence, $\quotep{\procn{x}} \nameeq x$; so, the
notation is consistent with previous definitions.

Further, because names have structure it is possible to effect
substitutions on the basis of that structure. This means we need to
upgrade our notation for substitutions, which we accomplish by
adapting comprehension notation. Thus,

\begin{mathpar}
  P\{ y / x : x \in S \}
\end{mathpar}

is interpreted to mean the process derived from P by replacing (in a
capture-avoiding manner) each occurrence of $x$ in $S$ by $y$. For example,

\begin{mathpar}
  P\{ \quotep{\procn{x}|\procn{x}} / x : x \in \freenames{P} \}
\end{mathpar}

will replace each (occurrence) of a free name $x$ in $P$ by
$\quotep{\procn{x}|\procn{x}}$.

Also, we will avail ourselves of the notation $x^{L}$ and $x^{R}$ to
denote injections of a name into disjoint copies of the name
space. There are numerous ways to accomplish this. One example can be
found in \cite{MeredithR05}. This notation overloads to vectors of
names: $\vec{x}^{\pi} := (x_{i}^{\pi} \; : \; 0 \leq i < |\vec{x}| )$ where $\pi \in \{L,R\}$.

We also use $P^{\Box} := P|\Box$.

In \cite{MeredithR05} an interpretation of the new operator is
given. It turns out that there are several possible interpretations
all enjoying the requisite algebraic properties of the operator (see
\cite{milner91polyadicpi}). We will therefore make liberal use of
$(\nu\; \vec{x})P$.

% subsection the_syntax_and_semantics_of_the_notation_system (end)   

\input{qm2pi.qmops} 

\input{qm2pi.sterngerlach} 

\input{qm2pi.metric} 

% section concurrent_process_calculi (end)

%\input{qm2pi.proofsketch}

% section proof sketch (end)

%\input{qm2pi.slviaknots} 

% section spatial logic via knots (end)

\input{qm2pi.conclusion}

% section conclusion (end)

%\input{qm2pi.dtcodes} 

% section wiring algorithm (end)

\input{qm2pi.ack} 

% section acknowledgments (end)

\newpage


\bibliographystyle{plain}   
\bibliography{../../biblios/main.bib}

\input{qm2pi.rhodetails}

\end{document}

 

% section notation (end)

\input{qm2pi.process.calculi} 

% section concurrent_process_calculi_and_spatial_logics_ (end)
    
%\documentclass[12pt]{llncs}
%\documentclass{jktr}

\usepackage[pdftex]{hyperref}                   
\usepackage {listings}
\usepackage {mathpartir}
\usepackage{bcprules}
%\usepackage{listings}
                       
\usepackage{graphicx} 
%\usepackage[margins=2.5cm,nohead,nofoot]{geometry}
%\usepackage{geometry}
\usepackage{amsfonts}
\usepackage{amstext}
\usepackage{latexsym}
\usepackage{amssymb}
\usepackage{color}


%\include{myPreamble}
\include{qm2pi.local} 

%\ifpdf
%\usepackage[pdftex]{graphicx}
%\else
%\usepackage{graphicx}
%\fi

 % \ifpdf
%  \usepackage{pdfsync}
%  \if


%\title{Brief Article}
%\author{David F. Snyder}
%\author{L.G. Meredith}

%\address{Dept. of Math., Texas State University--San Marcos, San Marcos, TX 78666}
       
\pagestyle{empty}


\begin{document}

\lstset{language=[Objective]Caml,frame=shadowbox}

\input{qm2pi.front}

% section front matter (end)

\input{qm2pi.intro} 
 
% section introduction (end)

% \input{qm2pi.knotations} 

% section notation (end)

\input{qm2pi.process.calculi} 

% section concurrent_process_calculi_and_spatial_logics_ (end)
    
%\input{qm2pi.knots2pi} 

%\input{qm2pi.trefoil} 

%\input{qm2pi.mainthm} 

% subsection basic_interpretation (end)

%\input{qm2pi.rho.presentation} 
\subsection{The syntax and semantics of the notation system}\label{sub:the_syntax_and_semantics_of_the_notation_system} % (fold)

We now summarize a technical presentation of the calculus that
embodies our theory of dynamics. The typical presentation of such a
calculus follows the style of giving generators and relations on
them. The grammar, below, describing term constructors, freely
generates the set of processes, $\Proc$. This set is then quotiented
by a relation known as structural congruence and it is over this set
that the notion of dynamics is expressed. This presentation is
essentially that of \cite{MeredithR05} with the addition of
polyadicity and summation. For readability we have relegated some of
the technical subtleties to an appendix.

\subsubsection{Process grammar}\label{subsub:process_grammar}

\begin{mathpar}
  \inferrule* [lab=synchronization] {} {{M} \bc \pzero \;|\; x?F \;|\; x!C }
  \and
  \inferrule* [lab=abstraction] {} {{F} \bc (x)P}
  \and
  \inferrule* [lab=concretion] {} {{C} \bc \langle Q \rangle}
  \and
  \inferrule* [lab=process] {} {{P,Q} \bc M \;| \;P|Q \;|\; @{x}}
  \and
  \inferrule* [lab=name] {} {{x} \bc \quotep{P}}
\end{mathpar} 

Note that $\vec{x}$ (resp. $\vec{P}$) denotes a vector of names
(resp. processes) of length $|\vec{x}|$ (resp. $|\vec{P}|$). We adopt
the following useful abbreviations.

\begin{mathpar}
   x?(\vec{y}).P := x.(\vec{y})P \and  x\clift{\vec{P}} := x.\clift{\vec{P}}
   \and x!(y) := \lift{x}{\dropn{y}}
   \and \Pi_{i=0}^{n-1}P_i := P_0 | \ldots | P_{n-1}
\end{mathpar}

\subsubsection{Structural congruence}

\paragraph{Free and bound names and alpha-equivalence.} At the
core of structural equivalence is alpha-equivalence which identifies
process that are the same up to a change of variable. Formally, we
recognize the distinction between free and bound names. The free names
of a process, $\freenames{P}$, may be calculated recursively as
follows:

\begin{mathpar}
\freenames{\pzero} := \emptyset
  \and \\
  \freenames{x?(y).P} := \{ x \} \cup (\freenames{P} \setminus \{ y \})
  \and 
  \freenames{x!\langle P \rangle} := \{ x \} \cup \{ P \} 
  \and \\
  \freenames{P|Q} := \freenames{P} \cup \freenames{Q}
  \and \\
  \freenames{@{x}} := \{ x \}
\end{mathpar}

$\pi$
$\quotep{\pi}$

$\freenames{-} : \pi \to \mathcal{P}(\quotep{\pi})$

\begin{eqnarray*}
  \freenames{\pzero} & := & \emptyset \\
  \freenames{x?(y).P} & := & \{ x \} \cup (\freenames{P} \setminus \{ y \}) \\
  \freenames{x!\langle P \rangle} & := & \{ x \} \cup \{ P \} \\
  \freenames{P|Q} & := & \freenames{P} \cup \freenames{Q} \\
  \freenames{\dropn{x}} & := & \{ x \}
\end{eqnarray*}

The bound names of a process, $\boundnames{P}$, are those names occurring in $P$
that are not free. For example, in $x?(y).0$, the name $x$ is free, while $y$ is bound.

\begin{mathpar}
  \inferrule* [lab=monoidal-laws] {} { P|Q \equiv Q|P \and P|0 \equiv P \and P|(Q|R) \equiv (P|Q)|R }
\end{mathpar}

\begin{mathpar}
  \inferrule* [lab=alpha-equivalence] {} { (x)P \equiv (y)P\{y/x\} \and y \not\in \freenames{P} }
\end{mathpar}

\begin{definition}
Then two processes, $P,Q$, are alpha-equivalent if $P = Q\{\vec{y}/\vec{x}\}$ for
some $\vec{x} \in \boundnames{Q},\vec{y} \in \boundnames{P}$, where $Q\{\vec{y}/\vec{x}\}$
denotes the capture-avoiding substitution of $\vec{y}$ for $\vec{x}$ in $Q$.
\end{definition}

\begin{definition}
  The {\em structural congruence} \cite{SangiorgiWalker} , $\equiv$,
  between processes is the least congruence containing
  alpha-equivalence, satisfying the abelian monoid laws
  (associativity, commutativity and $\pzero$ as identity) for parallel
  composition $|$ and for summation $+$.
\end{definition}

\subsection{Name equivalence}

We take name equivalence, written $\nameeq$, to be the smallest
equivalence relation generated by the following rules.

\begin{mathpar}
\inferrule*[lab=Quote-drop]
{ }
{ \quotep{@{x}} \nameeq x }

\inferrule*[lab=Struct-equiv]
{ P \scong Q }
{ \quotep{P} \nameeq \quotep{Q} }
\end{mathpar}

The astute reader will have noticed that the mutual recursion of names
and processes imposes a mutual recursion on alpha-equivalence and
structural equivalence via name-equivalence. Fortunately, all of this
works out pleasantly and we may calculate in the natural way, free of
concern. The reader interested in the details is referred to the
appendix \ref{appendix:rho_details}.

\subsection{Substitution}

We use $\Proc$ for the set of processes, $\QProc$ for the set of
names, and $\id{\{}\vec{y} / \vec{x} \id{\}}$ to denote partial maps,
$s : \QProc \rightarrow \QProc$. A map, $s$ lifts, uniquely, to a map
on process terms, $\widehat{s} : \Proc \rightarrow \Proc$ by the
following equations.

\begin{mathpar}
  (0) \psubstp{Q}{P} := 0 \\
  (R \juxtap S) \psubstp{Q}{P}
  :=    
  (R)\psubstp{Q}{P} \juxtap (S) \psubstp{Q}{P} \\
  (x?(y).R) \psubstp{Q}{P}    
  :=    
  (x)\substp{Q}{P} (z)\concat( (R \psubstn{z}{y}) \psubstp{Q}{P} ) \\
  (\lift{x}{R}) \psubstp{Q}{P}  
  :=
  \lift{(x)\substp{Q}{P}}{ R \psubstp{Q}{P} } \\
%   (\dropn{x})  \psubstp{Q}{P}       
%   := 
%   \left\{ 
%     \begin{array}{ccc} 
%       \dropn{\quotep{Q}} & & x \nameeq \quotep{P} \\
%       \dropn{x} & & otherwise \\
%     \end{array}
%   \right. 
  (\dropn{x})  \psubstp{Q}{P}       
  := 
  \left\{ 
    \begin{array}{ccc} 
      Q & & x \nameeq \quotep{P} \\
      \dropn{x} & & otherwise \\
    \end{array}
  \right.
\end{mathpar}
 

where

\begin{eqnarray}
  (x)\id{\{} \lpquote Q \rpquote / \lpquote P \rpquote \id{\}}            = 
  \left\{ 
    \begin{array}{ccc}
      \lpquote Q \rpquote & & x \nameeq \lpquote P \rpquote \\
      x & & otherwise \\
    \end{array}
  \right. \nonumber
\end{eqnarray}

and $z$ is chosen distinct from $\quotep{P}$, $\quotep{Q}$, the free
names in $Q$, and all the names in $R$. Our $\alpha$-equivalence will
be built in the standard way from this substitution.

\begin{remark}\label{rem:no_self_referential_names}
  One consequence of these definitions is that $\forall P. \quotep{P}
  \not\in \freenames{P}$.
\end{remark}

\subsection{ Dynamic quote: an example }

Anticipating something of what's to come, consider applying the
substitution, $\widehat{\id{\{}u / z \id{\}}}$, to the following pair
of processes, $\lift{w}{y!(z)}$ and $w[ \lpquote y!(z) \rpquote ]$.

\begin{eqnarray}
	\lift{w}{y!(z)}\widehat{\id{\{}u / z \id{\}}}
		& = &
		\lift{w}{y!(u)} \nonumber\\
	w[ \lpquote y!(z) \rpquote ] \widehat{ \id{\{}u / z \id{\}} }
		& = &
		w[ \lpquote y!(z) \rpquote ] \nonumber
\end{eqnarray}

Because the body of the process between quotes is impervious to
substitution, we get radically different answers. In fact, by
examining the first process in an input context,
e.g. $x?(z).\lift{w}{y!(z)}$, we see that the process under the lift
operator may be shaped by prefixed inputs binding a name inside it. In
this sense, the lift operator will be seen as a way to dynamically
construct processes before reifying them as names.

Finally equipped with these standard features we can present the
dynamics of the calculus.

\subsubsection{Operational semantics} 

Finally, we introduce the computational dynamics. What marks these
algebras as distinct from other more traditionally studied algebraic
structures, e.g. vector spaces or polynomial rings, is the manner in
which dynamics is captured. In traditional structures, dynamics is typically
expressed through morphisms between such structures, as in linear maps
between vector spaces or morphisms between rings. In algebras
associated with the semantics of computation, the dynamics is
expressed as part of the algebraic structure itself, through a
reduction reduction relation typically denoted by $\red$. Below, we
give a recursive presentation of this relation for the calculus used
in the encoding.

$\red \subseteq \pi \times \pi$
$\red : \pi \to \mathcal{P}(\pi)$

\begin{mathpar}
  \inferrule* [lab=Comm] { \textsf{match}( x_{src}, x_{trgt} ) } { x_{trgt}?(y)P \; | \; x_{src}!\langle {Q} \rangle \red P\{\quotep{Q}/y}\} }
  \and \\
  \inferrule* [lab=Par] {{P} \red {P}'} {{{P} | {Q}} \red {{P}' | {Q}}}
  \and
  \inferrule* [lab=Equiv]{{{P} \scong {P}'} \andalso {{P}' \red {Q}'} \andalso {{Q}' \scong {Q}}}{{P} \red {Q}}
\end{mathpar}

\begin{eqnarray*}
  match_{\equiv} (\quotep{P},\quotep{Q}) & := & P \equiv Q \\
  match_{\dagger}(\quotep{P},\quotep{Q}) & := & \forall R. P|Q \red^{*} R => R \red^{*} 0 \\
  match_{K}(\quotep{P},\quotep{Q}) & := & K \mbox{ for some context } K
\end{eqnarray*}

$u?(x)P | u!\langle Q \rangle \red P\{\quotep{Q}/x\}$

%We write $\wred$ for $\red^*$, and $P\red$ if $\exists Q $ such that $ P \red Q$.
We write $P\red$ if $\exists Q $ such that $ P \red Q$ and $P\not\red$, otherwise.

\section{Replication}

As mentioned before, it is known that replication (and hence
recursion) can be implemented in a higher-order process algebra
\cite{SangiorgiWalker}. As our first example of calculation with the
machinery thus far presented we give the construction explicitly in
the {\rhoc}.

\begin{eqnarray}
	D_{x} & := & \prefix{x}{y}{(\binpar{\outputp{x}{y}}{@{y}})} \nonumber\\
	\bangp_{x}{P} & := & \binpar{{x}!\langle{\binpar{D_{x}}{P}}\rangle}{D_{x}} \nonumber
\end{eqnarray}

\begin{eqnarray}
	\bangp_{x}{P} & & \nonumber\\
	=
	& {x}!\langle{(\prefix{x}{y}{(\outputp{x}{y} | @{y})) | P}}\rangle 
	      | \prefix{x}{y}{(\outputp{x}{y} | @{y})} & \nonumber\\
	\red
	& (\outputp{x}{y} | @{y})\substn{\quotep{(\prefix{x}{y}{(@{y} | \outputp{x}{y})) | P}}}{y} & \nonumber\\
	=
	& \outputp{x}{\quotep{(\prefix{x}{y}{(\outputp{x}{y} | @{y})) | P}}}
	  | {(\prefix{x}{y}{(\outputp{x}{y} | @{y})) | P}} & \nonumber\\
	\red
	& \ldots & \nonumber\\
	\red^*
	& P | P | \ldots & \nonumber
\end{eqnarray}

Of course, this encoding, as an implementation, runs away, unfolding
$\bangp{P}$ eagerly. A lazier and more implementable replication
operator, restricted to input-guarded processes, may be obtained as follows.

\begin{eqnarray}
\bangp{\prefix{u}{v}{P}} 
	:= 
	\binpar{\lift{x}{\prefix{u}{v}{(\binpar{D(x)}{P})}}}{D(x)} \nonumber
\end{eqnarray}

\begin{remark}
  Note that the lazier definition still does not deal with summation
  or mixed summation (i.e. sums over input and output). The reader is
  invited to construct definitions of replication that deal with these
  features. 

  Further, the definitions are parameterized in a name, $x$. Can you,
  gentle reader, make a definition that eliminates this parameter and
  guarantees no accidental interaction between the replication
  machinery and the process being replicated -- i.e. no accidental
  sharing of names used by the process to get its work done and the
  name(s) used by the replication to effect copying. This latter
  revision of the definition of replication is crucial to obtaining
  the expected identity $!!P \sim !P$.
\end{remark}

\begin{remark}\label{rem:paradoxical_combinator}
  The reader familiar with the lambda calculus will have noticed the
  similarity between $D$ and the paradoxical combinator.

  [Ed. note: the existence of this seems to suggest we have to be more
  restrictive on the set of processes and names we admit if we are to
  support no-cloning.]
\end{remark}

\subsubsection{Bisimulation}

The computational dynamics gives rise to another kind of equivalence,
the equivalence of computational behavior. As previously mentioned
this is typically captured \emph{via} some form of bisimulation.

% The notion we use in this paper is weak barbed bisimulation
% \cite{milner91polyadicpi}.

The notion we use in this paper is derived from weak barbed
bisimulation \cite{milner91polyadicpi}. 

\begin{definition}
An \emph{observation relation}, $\downarrow_{\mathcal N}$, over a set
of names, $\mathcal N$, is the smallest relation satisfying the rules
below.

\infrule[Out-barb]{y \in {\mathcal N}, \; x \nameeq y}
		  {\outputp{x}{v} \downarrow_{\mathcal N} x}
\infrule[Par-barb]{\mbox{$P\downarrow_{\mathcal N} x$ or $Q\downarrow_{\mathcal N} x$}}
		  {\binpar{P}{Q} \downarrow_{\mathcal N} x}

We write $P \Downarrow_{\mathcal N} x$ if there is $Q$ such that 
$P \wred Q$ and $Q \downarrow_{\mathcal N} x$.
\end{definition}

\begin{definition}
%\label{def.bbisim}
An  ${\mathcal N}$-\emph{barbed bisimulation} over a set of names, ${\mathcal N}$, is a symmetric binary relation 
${\mathcal S}_{\mathcal N}$ between agents such that $P\rel{S}_{\mathcal N}Q$ implies:
\begin{enumerate}
\item If $P \red P'$ then $Q \wred Q'$ and $P'\rel{S}_{\mathcal N} Q'$.
\item If $P\downarrow_{\mathcal N} x$, then $Q\Downarrow_{\mathcal N} x$.
\end{enumerate}
$P$ is ${\mathcal N}$-barbed bisimilar to $Q$, written
$P \wbbisim_{\mathcal N} Q$, if $P \rel{S}_{\mathcal N} Q$ for some ${\mathcal N}$-barbed bisimulation ${\mathcal S}_{\mathcal N}$.
\end{definition}

$\mathcal{R} \subseteq \pi \times \pi$

$P \mathcal{R} Q => \forall P'. P \red P' \Rightarrow \exists Q'. Q \red Q', P' \mathcal{R} Q'$

$P \vdash x \Rightarrow Q \vdash x$

\begin{mathpar}
  \inferrule*[lab=Out-barb]{x \nameeq y}{{y}!\langle{Q}\rangle \vdash x}
  \and
  \inferrule*[lab=Par-barb]{\mbox{$P\vdash x$ or $Q\vdash x$}}{\binpar{P}{Q} \vdash x}
\end{mathpar}

\subsubsection{Contexts}

One of the principle advantages of computational calculi like the
$\pi$-calculus is a well-defined notion of context,
contextual-equivalence and a correlation between
contextual-equivalence and notions of bisimulation. The notion of
context allows the decomposition of a process into (sub-)process and
its syntactic environment, its context. Thus, a context may be
thought of as a process with a ``hole'' (written $\Box$) in it. The
application of a context $M$ to a process $P$, written $M[P]$, is
tantamount to filling the hole in $M$ with $P$. In this paper we do
not need the full weight of this theory, but do make use of the notion
of context in the proof the main theorem. 

\begin{mathpar}
  \inferrule* [lab=summation] {} {{M_{M},M_{N}} \bc \Box \;|\; x.M_{A} \;|\; M_{M}+M_{N}}
  \and
  \inferrule* [lab=agent] {} {{M_{A}} \bc (\vec{x})M_{P} \;| \; \clift{P_0,\ldots,M_{P},\ldots,P_N}}
  \and \\
  \inferrule* [lab=process] {} {{M_{P}} \bc M_{N} \;| \;P|M_{P} }
\end{mathpar} 

\begin{mathpar}
  \inferrule* [lab=sychronization] {} {M_{N} \bc \Box \;|\; x?M_{F} \;|\; x!M_{C}}
  \and
  \inferrule* [lab=abstraction] {} {{M_{F}} \bc (x)M_{P} }
  \and
  \inferrule* [lab=concretion] {} {{M_{C}} \bc \langle M_{P} \rangle }
  \and \\
  \inferrule* [lab=process] {} {{M_{P}} \bc M_{N} \;| \;P|M_{P} }
\end{mathpar}

\begin{definition}[contextual application] Given a context $M$, and
  process $P$, we define the \emph{contextual application}, $M[P] :=
  M\{P/\Box\}$. That is, the contextual application of M to P is the
  substitution of $P$ for $\Box$ in $M$.
\end{definition}

$\meaningof{-} : L \to \mathcal{P}(\pi)$

\begin{mathpar}
  \inferrule* [lab=collection] {} {\meaningof{true} = \pi, \and \meaningof{~E} = \pi \setminus \meaningof{E}, \and \meaningof{E_{1} \& E_{2}} = \meaningof{E_{1}} \cap \meaningof{E_{2}}}
\end{mathpar}

\begin{mathpar}
  \inferrule* [lab=structure] {} {\meaningof{0} = \{ P \in \pi | P \equiv 0 \}, \and \\ \meaningof{E_1 | E_2} = \{ P \in \pi | P \equiv P_{1} | P_{2}, P_{1} \in \meaningof{E_{1}}, P_{2} \in \meaningof{E_2}\} }
\end{mathpar}

\begin{mathpar}
 \inferrule* [lab=behavior] {} {\meaningof{\langle a?b \rangle E} = \{ P \in \pi | P \equiv Q | u?(y)P', \\ \and \\\\ \and \\ \;\;\; u \in \meaningof{a}, \forall z.P'\{z/y\} \in \meaningof{E\{z/b\}}\}, \and \\ \meaningof{a!E} = \{ P \in \pi | P \equiv Q | x!\langle P' \rangle, x \in \meaningof{a} P' \in \meaningof{E}\} }
\end{mathpar}

\begin{mathpar}
 \inferrule* [lab=nominal] {} {\meaningof{\quotep{E}} = \{ \quotep{P} \in \quotep{\pi} | P \in \meaningof{E} \}, \and \meaningof{\quotep{P}} = \{ \quotep{Q} \in \quotep{\pi} | P \equiv Q \} \and \\ \meaningof{@\quotep{E}} = \{ P \in \pi | P \equiv @x, x \in \meaningof{E} \}}
\end{mathpar}

\begin{eqnarray*}
  \\
  \meaningof{-} : TS \to ST
\end{eqnarray*}

\begin{eqnarray*}
  \\
  L : TS \to ST
\end{eqnarray*}

\begin{eqnarray*}
  \\
  P \models E \iff P \in \meaningof{E}
\end{eqnarray*}

\begin{eqnarray*}
  P \approx_{L} Q \iff \forall E \in L. P \models E \iff Q \models E
\end{eqnarray*}

\begin{eqnarray*}
  P \approx_{K} Q
\end{eqnarray*}

\begin{eqnarray*}
  P \approx Q
\end{eqnarray*}

$\approx_{K} = \approx = \approx_{L}$

\subsubsection{Contextual duality}

Note that contexts extend the quotation operation to a family of
operations from processes to names. Given a context, $M$, we can
define a \emph{nominal context}, $\quotep{M}$ by $\quotep{M}[P] :=
\quotep{M[P]}$. To foreshadow what is to come we observe that these
operations enjoy a duality with processes very much like the duality
between vectors and maps from vectors to scalars.

Further, because the calculus is essentially higher-order, we have a
correspondence between contexts and processes. More specifically,
given a name $x$ and a context $M$ we can construct $M^{*}_{x}$ such
that 

\begin{mathpar}
  M^{*}_{x} | \lift{x}{P} \red M[P]
\end{mathpar}

namely,

\begin{mathpar}
  M^{*}_{x} := x?(u).M[\dropn{u}]
\end{mathpar}

The dependence of $M^{*}_{x}$ on a name makes it an abstraction, 

\begin{mathpar}
  M^{*} := (x)x?(u).M[\dropn{u}]
\end{mathpar}

\subsection{Additional notation}

It will sometimes be convenient to denote the process a name
quotes. We already have the notation $x = \quotep{P}$, but it will be
convenient to introduce an alternate notation, $\procn{x}$, when we
want to emphasize the connection to the use of the name. Note that, by
virtue of name equivalence, $\quotep{\procn{x}} \nameeq x$; so, the
notation is consistent with previous definitions.

Further, because names have structure it is possible to effect
substitutions on the basis of that structure. This means we need to
upgrade our notation for substitutions, which we accomplish by
adapting comprehension notation. Thus,

\begin{mathpar}
  P\{ y / x : x \in S \}
\end{mathpar}

is interpreted to mean the process derived from P by replacing (in a
capture-avoiding manner) each occurrence of $x$ in $S$ by $y$. For example,

\begin{mathpar}
  P\{ \quotep{\procn{x}|\procn{x}} / x : x \in \freenames{P} \}
\end{mathpar}

will replace each (occurrence) of a free name $x$ in $P$ by
$\quotep{\procn{x}|\procn{x}}$.

Also, we will avail ourselves of the notation $x^{L}$ and $x^{R}$ to
denote injections of a name into disjoint copies of the name
space. There are numerous ways to accomplish this. One example can be
found in \cite{MeredithR05}. This notation overloads to vectors of
names: $\vec{x}^{\pi} := (x_{i}^{\pi} \; : \; 0 \leq i < |\vec{x}| )$ where $\pi \in \{L,R\}$.

We also use $P^{\Box} := P|\Box$.

In \cite{MeredithR05} an interpretation of the new operator is
given. It turns out that there are several possible interpretations
all enjoying the requisite algebraic properties of the operator (see
\cite{milner91polyadicpi}). We will therefore make liberal use of
$(\nu\; \vec{x})P$.

% subsection the_syntax_and_semantics_of_the_notation_system (end)   

\input{qm2pi.qmops} 

\input{qm2pi.sterngerlach} 

\input{qm2pi.metric} 

% section concurrent_process_calculi (end)

%\input{qm2pi.proofsketch}

% section proof sketch (end)

%\input{qm2pi.slviaknots} 

% section spatial logic via knots (end)

\input{qm2pi.conclusion}

% section conclusion (end)

%\input{qm2pi.dtcodes} 

% section wiring algorithm (end)

\input{qm2pi.ack} 

% section acknowledgments (end)

\newpage


\bibliographystyle{plain}   
\bibliography{../../biblios/main.bib}

\input{qm2pi.rhodetails}

\end{document}

 

%\documentclass[12pt]{llncs}
%\documentclass{jktr}

\usepackage[pdftex]{hyperref}                   
\usepackage {listings}
\usepackage {mathpartir}
\usepackage{bcprules}
%\usepackage{listings}
                       
\usepackage{graphicx} 
%\usepackage[margins=2.5cm,nohead,nofoot]{geometry}
%\usepackage{geometry}
\usepackage{amsfonts}
\usepackage{amstext}
\usepackage{latexsym}
\usepackage{amssymb}
\usepackage{color}


%\include{myPreamble}
\include{qm2pi.local} 

%\ifpdf
%\usepackage[pdftex]{graphicx}
%\else
%\usepackage{graphicx}
%\fi

 % \ifpdf
%  \usepackage{pdfsync}
%  \if


%\title{Brief Article}
%\author{David F. Snyder}
%\author{L.G. Meredith}

%\address{Dept. of Math., Texas State University--San Marcos, San Marcos, TX 78666}
       
\pagestyle{empty}


\begin{document}

\lstset{language=[Objective]Caml,frame=shadowbox}

\input{qm2pi.front}

% section front matter (end)

\input{qm2pi.intro} 
 
% section introduction (end)

% \input{qm2pi.knotations} 

% section notation (end)

\input{qm2pi.process.calculi} 

% section concurrent_process_calculi_and_spatial_logics_ (end)
    
%\input{qm2pi.knots2pi} 

%\input{qm2pi.trefoil} 

%\input{qm2pi.mainthm} 

% subsection basic_interpretation (end)

%\input{qm2pi.rho.presentation} 
\subsection{The syntax and semantics of the notation system}\label{sub:the_syntax_and_semantics_of_the_notation_system} % (fold)

We now summarize a technical presentation of the calculus that
embodies our theory of dynamics. The typical presentation of such a
calculus follows the style of giving generators and relations on
them. The grammar, below, describing term constructors, freely
generates the set of processes, $\Proc$. This set is then quotiented
by a relation known as structural congruence and it is over this set
that the notion of dynamics is expressed. This presentation is
essentially that of \cite{MeredithR05} with the addition of
polyadicity and summation. For readability we have relegated some of
the technical subtleties to an appendix.

\subsubsection{Process grammar}\label{subsub:process_grammar}

\begin{mathpar}
  \inferrule* [lab=synchronization] {} {{M} \bc \pzero \;|\; x?F \;|\; x!C }
  \and
  \inferrule* [lab=abstraction] {} {{F} \bc (x)P}
  \and
  \inferrule* [lab=concretion] {} {{C} \bc \langle Q \rangle}
  \and
  \inferrule* [lab=process] {} {{P,Q} \bc M \;| \;P|Q \;|\; @{x}}
  \and
  \inferrule* [lab=name] {} {{x} \bc \quotep{P}}
\end{mathpar} 

Note that $\vec{x}$ (resp. $\vec{P}$) denotes a vector of names
(resp. processes) of length $|\vec{x}|$ (resp. $|\vec{P}|$). We adopt
the following useful abbreviations.

\begin{mathpar}
   x?(\vec{y}).P := x.(\vec{y})P \and  x\clift{\vec{P}} := x.\clift{\vec{P}}
   \and x!(y) := \lift{x}{\dropn{y}}
   \and \Pi_{i=0}^{n-1}P_i := P_0 | \ldots | P_{n-1}
\end{mathpar}

\subsubsection{Structural congruence}

\paragraph{Free and bound names and alpha-equivalence.} At the
core of structural equivalence is alpha-equivalence which identifies
process that are the same up to a change of variable. Formally, we
recognize the distinction between free and bound names. The free names
of a process, $\freenames{P}$, may be calculated recursively as
follows:

\begin{mathpar}
\freenames{\pzero} := \emptyset
  \and \\
  \freenames{x?(y).P} := \{ x \} \cup (\freenames{P} \setminus \{ y \})
  \and 
  \freenames{x!\langle P \rangle} := \{ x \} \cup \{ P \} 
  \and \\
  \freenames{P|Q} := \freenames{P} \cup \freenames{Q}
  \and \\
  \freenames{@{x}} := \{ x \}
\end{mathpar}

$\pi$
$\quotep{\pi}$

$\freenames{-} : \pi \to \mathcal{P}(\quotep{\pi})$

\begin{eqnarray*}
  \freenames{\pzero} & := & \emptyset \\
  \freenames{x?(y).P} & := & \{ x \} \cup (\freenames{P} \setminus \{ y \}) \\
  \freenames{x!\langle P \rangle} & := & \{ x \} \cup \{ P \} \\
  \freenames{P|Q} & := & \freenames{P} \cup \freenames{Q} \\
  \freenames{\dropn{x}} & := & \{ x \}
\end{eqnarray*}

The bound names of a process, $\boundnames{P}$, are those names occurring in $P$
that are not free. For example, in $x?(y).0$, the name $x$ is free, while $y$ is bound.

\begin{mathpar}
  \inferrule* [lab=monoidal-laws] {} { P|Q \equiv Q|P \and P|0 \equiv P \and P|(Q|R) \equiv (P|Q)|R }
\end{mathpar}

\begin{mathpar}
  \inferrule* [lab=alpha-equivalence] {} { (x)P \equiv (y)P\{y/x\} \and y \not\in \freenames{P} }
\end{mathpar}

\begin{definition}
Then two processes, $P,Q$, are alpha-equivalent if $P = Q\{\vec{y}/\vec{x}\}$ for
some $\vec{x} \in \boundnames{Q},\vec{y} \in \boundnames{P}$, where $Q\{\vec{y}/\vec{x}\}$
denotes the capture-avoiding substitution of $\vec{y}$ for $\vec{x}$ in $Q$.
\end{definition}

\begin{definition}
  The {\em structural congruence} \cite{SangiorgiWalker} , $\equiv$,
  between processes is the least congruence containing
  alpha-equivalence, satisfying the abelian monoid laws
  (associativity, commutativity and $\pzero$ as identity) for parallel
  composition $|$ and for summation $+$.
\end{definition}

\subsection{Name equivalence}

We take name equivalence, written $\nameeq$, to be the smallest
equivalence relation generated by the following rules.

\begin{mathpar}
\inferrule*[lab=Quote-drop]
{ }
{ \quotep{@{x}} \nameeq x }

\inferrule*[lab=Struct-equiv]
{ P \scong Q }
{ \quotep{P} \nameeq \quotep{Q} }
\end{mathpar}

The astute reader will have noticed that the mutual recursion of names
and processes imposes a mutual recursion on alpha-equivalence and
structural equivalence via name-equivalence. Fortunately, all of this
works out pleasantly and we may calculate in the natural way, free of
concern. The reader interested in the details is referred to the
appendix \ref{appendix:rho_details}.

\subsection{Substitution}

We use $\Proc$ for the set of processes, $\QProc$ for the set of
names, and $\id{\{}\vec{y} / \vec{x} \id{\}}$ to denote partial maps,
$s : \QProc \rightarrow \QProc$. A map, $s$ lifts, uniquely, to a map
on process terms, $\widehat{s} : \Proc \rightarrow \Proc$ by the
following equations.

\begin{mathpar}
  (0) \psubstp{Q}{P} := 0 \\
  (R \juxtap S) \psubstp{Q}{P}
  :=    
  (R)\psubstp{Q}{P} \juxtap (S) \psubstp{Q}{P} \\
  (x?(y).R) \psubstp{Q}{P}    
  :=    
  (x)\substp{Q}{P} (z)\concat( (R \psubstn{z}{y}) \psubstp{Q}{P} ) \\
  (\lift{x}{R}) \psubstp{Q}{P}  
  :=
  \lift{(x)\substp{Q}{P}}{ R \psubstp{Q}{P} } \\
%   (\dropn{x})  \psubstp{Q}{P}       
%   := 
%   \left\{ 
%     \begin{array}{ccc} 
%       \dropn{\quotep{Q}} & & x \nameeq \quotep{P} \\
%       \dropn{x} & & otherwise \\
%     \end{array}
%   \right. 
  (\dropn{x})  \psubstp{Q}{P}       
  := 
  \left\{ 
    \begin{array}{ccc} 
      Q & & x \nameeq \quotep{P} \\
      \dropn{x} & & otherwise \\
    \end{array}
  \right.
\end{mathpar}
 

where

\begin{eqnarray}
  (x)\id{\{} \lpquote Q \rpquote / \lpquote P \rpquote \id{\}}            = 
  \left\{ 
    \begin{array}{ccc}
      \lpquote Q \rpquote & & x \nameeq \lpquote P \rpquote \\
      x & & otherwise \\
    \end{array}
  \right. \nonumber
\end{eqnarray}

and $z$ is chosen distinct from $\quotep{P}$, $\quotep{Q}$, the free
names in $Q$, and all the names in $R$. Our $\alpha$-equivalence will
be built in the standard way from this substitution.

\begin{remark}\label{rem:no_self_referential_names}
  One consequence of these definitions is that $\forall P. \quotep{P}
  \not\in \freenames{P}$.
\end{remark}

\subsection{ Dynamic quote: an example }

Anticipating something of what's to come, consider applying the
substitution, $\widehat{\id{\{}u / z \id{\}}}$, to the following pair
of processes, $\lift{w}{y!(z)}$ and $w[ \lpquote y!(z) \rpquote ]$.

\begin{eqnarray}
	\lift{w}{y!(z)}\widehat{\id{\{}u / z \id{\}}}
		& = &
		\lift{w}{y!(u)} \nonumber\\
	w[ \lpquote y!(z) \rpquote ] \widehat{ \id{\{}u / z \id{\}} }
		& = &
		w[ \lpquote y!(z) \rpquote ] \nonumber
\end{eqnarray}

Because the body of the process between quotes is impervious to
substitution, we get radically different answers. In fact, by
examining the first process in an input context,
e.g. $x?(z).\lift{w}{y!(z)}$, we see that the process under the lift
operator may be shaped by prefixed inputs binding a name inside it. In
this sense, the lift operator will be seen as a way to dynamically
construct processes before reifying them as names.

Finally equipped with these standard features we can present the
dynamics of the calculus.

\subsubsection{Operational semantics} 

Finally, we introduce the computational dynamics. What marks these
algebras as distinct from other more traditionally studied algebraic
structures, e.g. vector spaces or polynomial rings, is the manner in
which dynamics is captured. In traditional structures, dynamics is typically
expressed through morphisms between such structures, as in linear maps
between vector spaces or morphisms between rings. In algebras
associated with the semantics of computation, the dynamics is
expressed as part of the algebraic structure itself, through a
reduction reduction relation typically denoted by $\red$. Below, we
give a recursive presentation of this relation for the calculus used
in the encoding.

$\red \subseteq \pi \times \pi$
$\red : \pi \to \mathcal{P}(\pi)$

\begin{mathpar}
  \inferrule* [lab=Comm] { \textsf{match}( x_{src}, x_{trgt} ) } { x_{trgt}?(y)P \; | \; x_{src}!\langle {Q} \rangle \red P\{\quotep{Q}/y}\} }
  \and \\
  \inferrule* [lab=Par] {{P} \red {P}'} {{{P} | {Q}} \red {{P}' | {Q}}}
  \and
  \inferrule* [lab=Equiv]{{{P} \scong {P}'} \andalso {{P}' \red {Q}'} \andalso {{Q}' \scong {Q}}}{{P} \red {Q}}
\end{mathpar}

\begin{eqnarray*}
  match_{\equiv} (\quotep{P},\quotep{Q}) & := & P \equiv Q \\
  match_{\dagger}(\quotep{P},\quotep{Q}) & := & \forall R. P|Q \red^{*} R => R \red^{*} 0 \\
  match_{K}(\quotep{P},\quotep{Q}) & := & K \mbox{ for some context } K
\end{eqnarray*}

$u?(x)P | u!\langle Q \rangle \red P\{\quotep{Q}/x\}$

%We write $\wred$ for $\red^*$, and $P\red$ if $\exists Q $ such that $ P \red Q$.
We write $P\red$ if $\exists Q $ such that $ P \red Q$ and $P\not\red$, otherwise.

\section{Replication}

As mentioned before, it is known that replication (and hence
recursion) can be implemented in a higher-order process algebra
\cite{SangiorgiWalker}. As our first example of calculation with the
machinery thus far presented we give the construction explicitly in
the {\rhoc}.

\begin{eqnarray}
	D_{x} & := & \prefix{x}{y}{(\binpar{\outputp{x}{y}}{@{y}})} \nonumber\\
	\bangp_{x}{P} & := & \binpar{{x}!\langle{\binpar{D_{x}}{P}}\rangle}{D_{x}} \nonumber
\end{eqnarray}

\begin{eqnarray}
	\bangp_{x}{P} & & \nonumber\\
	=
	& {x}!\langle{(\prefix{x}{y}{(\outputp{x}{y} | @{y})) | P}}\rangle 
	      | \prefix{x}{y}{(\outputp{x}{y} | @{y})} & \nonumber\\
	\red
	& (\outputp{x}{y} | @{y})\substn{\quotep{(\prefix{x}{y}{(@{y} | \outputp{x}{y})) | P}}}{y} & \nonumber\\
	=
	& \outputp{x}{\quotep{(\prefix{x}{y}{(\outputp{x}{y} | @{y})) | P}}}
	  | {(\prefix{x}{y}{(\outputp{x}{y} | @{y})) | P}} & \nonumber\\
	\red
	& \ldots & \nonumber\\
	\red^*
	& P | P | \ldots & \nonumber
\end{eqnarray}

Of course, this encoding, as an implementation, runs away, unfolding
$\bangp{P}$ eagerly. A lazier and more implementable replication
operator, restricted to input-guarded processes, may be obtained as follows.

\begin{eqnarray}
\bangp{\prefix{u}{v}{P}} 
	:= 
	\binpar{\lift{x}{\prefix{u}{v}{(\binpar{D(x)}{P})}}}{D(x)} \nonumber
\end{eqnarray}

\begin{remark}
  Note that the lazier definition still does not deal with summation
  or mixed summation (i.e. sums over input and output). The reader is
  invited to construct definitions of replication that deal with these
  features. 

  Further, the definitions are parameterized in a name, $x$. Can you,
  gentle reader, make a definition that eliminates this parameter and
  guarantees no accidental interaction between the replication
  machinery and the process being replicated -- i.e. no accidental
  sharing of names used by the process to get its work done and the
  name(s) used by the replication to effect copying. This latter
  revision of the definition of replication is crucial to obtaining
  the expected identity $!!P \sim !P$.
\end{remark}

\begin{remark}\label{rem:paradoxical_combinator}
  The reader familiar with the lambda calculus will have noticed the
  similarity between $D$ and the paradoxical combinator.

  [Ed. note: the existence of this seems to suggest we have to be more
  restrictive on the set of processes and names we admit if we are to
  support no-cloning.]
\end{remark}

\subsubsection{Bisimulation}

The computational dynamics gives rise to another kind of equivalence,
the equivalence of computational behavior. As previously mentioned
this is typically captured \emph{via} some form of bisimulation.

% The notion we use in this paper is weak barbed bisimulation
% \cite{milner91polyadicpi}.

The notion we use in this paper is derived from weak barbed
bisimulation \cite{milner91polyadicpi}. 

\begin{definition}
An \emph{observation relation}, $\downarrow_{\mathcal N}$, over a set
of names, $\mathcal N$, is the smallest relation satisfying the rules
below.

\infrule[Out-barb]{y \in {\mathcal N}, \; x \nameeq y}
		  {\outputp{x}{v} \downarrow_{\mathcal N} x}
\infrule[Par-barb]{\mbox{$P\downarrow_{\mathcal N} x$ or $Q\downarrow_{\mathcal N} x$}}
		  {\binpar{P}{Q} \downarrow_{\mathcal N} x}

We write $P \Downarrow_{\mathcal N} x$ if there is $Q$ such that 
$P \wred Q$ and $Q \downarrow_{\mathcal N} x$.
\end{definition}

\begin{definition}
%\label{def.bbisim}
An  ${\mathcal N}$-\emph{barbed bisimulation} over a set of names, ${\mathcal N}$, is a symmetric binary relation 
${\mathcal S}_{\mathcal N}$ between agents such that $P\rel{S}_{\mathcal N}Q$ implies:
\begin{enumerate}
\item If $P \red P'$ then $Q \wred Q'$ and $P'\rel{S}_{\mathcal N} Q'$.
\item If $P\downarrow_{\mathcal N} x$, then $Q\Downarrow_{\mathcal N} x$.
\end{enumerate}
$P$ is ${\mathcal N}$-barbed bisimilar to $Q$, written
$P \wbbisim_{\mathcal N} Q$, if $P \rel{S}_{\mathcal N} Q$ for some ${\mathcal N}$-barbed bisimulation ${\mathcal S}_{\mathcal N}$.
\end{definition}

$\mathcal{R} \subseteq \pi \times \pi$

$P \mathcal{R} Q => \forall P'. P \red P' \Rightarrow \exists Q'. Q \red Q', P' \mathcal{R} Q'$

$P \vdash x \Rightarrow Q \vdash x$

\begin{mathpar}
  \inferrule*[lab=Out-barb]{x \nameeq y}{{y}!\langle{Q}\rangle \vdash x}
  \and
  \inferrule*[lab=Par-barb]{\mbox{$P\vdash x$ or $Q\vdash x$}}{\binpar{P}{Q} \vdash x}
\end{mathpar}

\subsubsection{Contexts}

One of the principle advantages of computational calculi like the
$\pi$-calculus is a well-defined notion of context,
contextual-equivalence and a correlation between
contextual-equivalence and notions of bisimulation. The notion of
context allows the decomposition of a process into (sub-)process and
its syntactic environment, its context. Thus, a context may be
thought of as a process with a ``hole'' (written $\Box$) in it. The
application of a context $M$ to a process $P$, written $M[P]$, is
tantamount to filling the hole in $M$ with $P$. In this paper we do
not need the full weight of this theory, but do make use of the notion
of context in the proof the main theorem. 

\begin{mathpar}
  \inferrule* [lab=summation] {} {{M_{M},M_{N}} \bc \Box \;|\; x.M_{A} \;|\; M_{M}+M_{N}}
  \and
  \inferrule* [lab=agent] {} {{M_{A}} \bc (\vec{x})M_{P} \;| \; \clift{P_0,\ldots,M_{P},\ldots,P_N}}
  \and \\
  \inferrule* [lab=process] {} {{M_{P}} \bc M_{N} \;| \;P|M_{P} }
\end{mathpar} 

\begin{mathpar}
  \inferrule* [lab=sychronization] {} {M_{N} \bc \Box \;|\; x?M_{F} \;|\; x!M_{C}}
  \and
  \inferrule* [lab=abstraction] {} {{M_{F}} \bc (x)M_{P} }
  \and
  \inferrule* [lab=concretion] {} {{M_{C}} \bc \langle M_{P} \rangle }
  \and \\
  \inferrule* [lab=process] {} {{M_{P}} \bc M_{N} \;| \;P|M_{P} }
\end{mathpar}

\begin{definition}[contextual application] Given a context $M$, and
  process $P$, we define the \emph{contextual application}, $M[P] :=
  M\{P/\Box\}$. That is, the contextual application of M to P is the
  substitution of $P$ for $\Box$ in $M$.
\end{definition}

$\meaningof{-} : L \to \mathcal{P}(\pi)$

\begin{mathpar}
  \inferrule* [lab=collection] {} {\meaningof{true} = \pi, \and \meaningof{~E} = \pi \setminus \meaningof{E}, \and \meaningof{E_{1} \& E_{2}} = \meaningof{E_{1}} \cap \meaningof{E_{2}}}
\end{mathpar}

\begin{mathpar}
  \inferrule* [lab=structure] {} {\meaningof{0} = \{ P \in \pi | P \equiv 0 \}, \and \\ \meaningof{E_1 | E_2} = \{ P \in \pi | P \equiv P_{1} | P_{2}, P_{1} \in \meaningof{E_{1}}, P_{2} \in \meaningof{E_2}\} }
\end{mathpar}

\begin{mathpar}
 \inferrule* [lab=behavior] {} {\meaningof{\langle a?b \rangle E} = \{ P \in \pi | P \equiv Q | u?(y)P', \\ \and \\\\ \and \\ \;\;\; u \in \meaningof{a}, \forall z.P'\{z/y\} \in \meaningof{E\{z/b\}}\}, \and \\ \meaningof{a!E} = \{ P \in \pi | P \equiv Q | x!\langle P' \rangle, x \in \meaningof{a} P' \in \meaningof{E}\} }
\end{mathpar}

\begin{mathpar}
 \inferrule* [lab=nominal] {} {\meaningof{\quotep{E}} = \{ \quotep{P} \in \quotep{\pi} | P \in \meaningof{E} \}, \and \meaningof{\quotep{P}} = \{ \quotep{Q} \in \quotep{\pi} | P \equiv Q \} \and \\ \meaningof{@\quotep{E}} = \{ P \in \pi | P \equiv @x, x \in \meaningof{E} \}}
\end{mathpar}

\begin{eqnarray*}
  \\
  \meaningof{-} : TS \to ST
\end{eqnarray*}

\begin{eqnarray*}
  \\
  L : TS \to ST
\end{eqnarray*}

\begin{eqnarray*}
  \\
  P \models E \iff P \in \meaningof{E}
\end{eqnarray*}

\begin{eqnarray*}
  P \approx_{L} Q \iff \forall E \in L. P \models E \iff Q \models E
\end{eqnarray*}

\begin{eqnarray*}
  P \approx_{K} Q
\end{eqnarray*}

\begin{eqnarray*}
  P \approx Q
\end{eqnarray*}

$\approx_{K} = \approx = \approx_{L}$

\subsubsection{Contextual duality}

Note that contexts extend the quotation operation to a family of
operations from processes to names. Given a context, $M$, we can
define a \emph{nominal context}, $\quotep{M}$ by $\quotep{M}[P] :=
\quotep{M[P]}$. To foreshadow what is to come we observe that these
operations enjoy a duality with processes very much like the duality
between vectors and maps from vectors to scalars.

Further, because the calculus is essentially higher-order, we have a
correspondence between contexts and processes. More specifically,
given a name $x$ and a context $M$ we can construct $M^{*}_{x}$ such
that 

\begin{mathpar}
  M^{*}_{x} | \lift{x}{P} \red M[P]
\end{mathpar}

namely,

\begin{mathpar}
  M^{*}_{x} := x?(u).M[\dropn{u}]
\end{mathpar}

The dependence of $M^{*}_{x}$ on a name makes it an abstraction, 

\begin{mathpar}
  M^{*} := (x)x?(u).M[\dropn{u}]
\end{mathpar}

\subsection{Additional notation}

It will sometimes be convenient to denote the process a name
quotes. We already have the notation $x = \quotep{P}$, but it will be
convenient to introduce an alternate notation, $\procn{x}$, when we
want to emphasize the connection to the use of the name. Note that, by
virtue of name equivalence, $\quotep{\procn{x}} \nameeq x$; so, the
notation is consistent with previous definitions.

Further, because names have structure it is possible to effect
substitutions on the basis of that structure. This means we need to
upgrade our notation for substitutions, which we accomplish by
adapting comprehension notation. Thus,

\begin{mathpar}
  P\{ y / x : x \in S \}
\end{mathpar}

is interpreted to mean the process derived from P by replacing (in a
capture-avoiding manner) each occurrence of $x$ in $S$ by $y$. For example,

\begin{mathpar}
  P\{ \quotep{\procn{x}|\procn{x}} / x : x \in \freenames{P} \}
\end{mathpar}

will replace each (occurrence) of a free name $x$ in $P$ by
$\quotep{\procn{x}|\procn{x}}$.

Also, we will avail ourselves of the notation $x^{L}$ and $x^{R}$ to
denote injections of a name into disjoint copies of the name
space. There are numerous ways to accomplish this. One example can be
found in \cite{MeredithR05}. This notation overloads to vectors of
names: $\vec{x}^{\pi} := (x_{i}^{\pi} \; : \; 0 \leq i < |\vec{x}| )$ where $\pi \in \{L,R\}$.

We also use $P^{\Box} := P|\Box$.

In \cite{MeredithR05} an interpretation of the new operator is
given. It turns out that there are several possible interpretations
all enjoying the requisite algebraic properties of the operator (see
\cite{milner91polyadicpi}). We will therefore make liberal use of
$(\nu\; \vec{x})P$.

% subsection the_syntax_and_semantics_of_the_notation_system (end)   

\input{qm2pi.qmops} 

\input{qm2pi.sterngerlach} 

\input{qm2pi.metric} 

% section concurrent_process_calculi (end)

%\input{qm2pi.proofsketch}

% section proof sketch (end)

%\input{qm2pi.slviaknots} 

% section spatial logic via knots (end)

\input{qm2pi.conclusion}

% section conclusion (end)

%\input{qm2pi.dtcodes} 

% section wiring algorithm (end)

\input{qm2pi.ack} 

% section acknowledgments (end)

\newpage


\bibliographystyle{plain}   
\bibliography{../../biblios/main.bib}

\input{qm2pi.rhodetails}

\end{document}

 

%\documentclass[12pt]{llncs}
%\documentclass{jktr}

\usepackage[pdftex]{hyperref}                   
\usepackage {listings}
\usepackage {mathpartir}
\usepackage{bcprules}
%\usepackage{listings}
                       
\usepackage{graphicx} 
%\usepackage[margins=2.5cm,nohead,nofoot]{geometry}
%\usepackage{geometry}
\usepackage{amsfonts}
\usepackage{amstext}
\usepackage{latexsym}
\usepackage{amssymb}
\usepackage{color}


%\include{myPreamble}
\include{qm2pi.local} 

%\ifpdf
%\usepackage[pdftex]{graphicx}
%\else
%\usepackage{graphicx}
%\fi

 % \ifpdf
%  \usepackage{pdfsync}
%  \if


%\title{Brief Article}
%\author{David F. Snyder}
%\author{L.G. Meredith}

%\address{Dept. of Math., Texas State University--San Marcos, San Marcos, TX 78666}
       
\pagestyle{empty}


\begin{document}

\lstset{language=[Objective]Caml,frame=shadowbox}

\input{qm2pi.front}

% section front matter (end)

\input{qm2pi.intro} 
 
% section introduction (end)

% \input{qm2pi.knotations} 

% section notation (end)

\input{qm2pi.process.calculi} 

% section concurrent_process_calculi_and_spatial_logics_ (end)
    
%\input{qm2pi.knots2pi} 

%\input{qm2pi.trefoil} 

%\input{qm2pi.mainthm} 

% subsection basic_interpretation (end)

%\input{qm2pi.rho.presentation} 
\subsection{The syntax and semantics of the notation system}\label{sub:the_syntax_and_semantics_of_the_notation_system} % (fold)

We now summarize a technical presentation of the calculus that
embodies our theory of dynamics. The typical presentation of such a
calculus follows the style of giving generators and relations on
them. The grammar, below, describing term constructors, freely
generates the set of processes, $\Proc$. This set is then quotiented
by a relation known as structural congruence and it is over this set
that the notion of dynamics is expressed. This presentation is
essentially that of \cite{MeredithR05} with the addition of
polyadicity and summation. For readability we have relegated some of
the technical subtleties to an appendix.

\subsubsection{Process grammar}\label{subsub:process_grammar}

\begin{mathpar}
  \inferrule* [lab=synchronization] {} {{M} \bc \pzero \;|\; x?F \;|\; x!C }
  \and
  \inferrule* [lab=abstraction] {} {{F} \bc (x)P}
  \and
  \inferrule* [lab=concretion] {} {{C} \bc \langle Q \rangle}
  \and
  \inferrule* [lab=process] {} {{P,Q} \bc M \;| \;P|Q \;|\; @{x}}
  \and
  \inferrule* [lab=name] {} {{x} \bc \quotep{P}}
\end{mathpar} 

Note that $\vec{x}$ (resp. $\vec{P}$) denotes a vector of names
(resp. processes) of length $|\vec{x}|$ (resp. $|\vec{P}|$). We adopt
the following useful abbreviations.

\begin{mathpar}
   x?(\vec{y}).P := x.(\vec{y})P \and  x\clift{\vec{P}} := x.\clift{\vec{P}}
   \and x!(y) := \lift{x}{\dropn{y}}
   \and \Pi_{i=0}^{n-1}P_i := P_0 | \ldots | P_{n-1}
\end{mathpar}

\subsubsection{Structural congruence}

\paragraph{Free and bound names and alpha-equivalence.} At the
core of structural equivalence is alpha-equivalence which identifies
process that are the same up to a change of variable. Formally, we
recognize the distinction between free and bound names. The free names
of a process, $\freenames{P}$, may be calculated recursively as
follows:

\begin{mathpar}
\freenames{\pzero} := \emptyset
  \and \\
  \freenames{x?(y).P} := \{ x \} \cup (\freenames{P} \setminus \{ y \})
  \and 
  \freenames{x!\langle P \rangle} := \{ x \} \cup \{ P \} 
  \and \\
  \freenames{P|Q} := \freenames{P} \cup \freenames{Q}
  \and \\
  \freenames{@{x}} := \{ x \}
\end{mathpar}

$\pi$
$\quotep{\pi}$

$\freenames{-} : \pi \to \mathcal{P}(\quotep{\pi})$

\begin{eqnarray*}
  \freenames{\pzero} & := & \emptyset \\
  \freenames{x?(y).P} & := & \{ x \} \cup (\freenames{P} \setminus \{ y \}) \\
  \freenames{x!\langle P \rangle} & := & \{ x \} \cup \{ P \} \\
  \freenames{P|Q} & := & \freenames{P} \cup \freenames{Q} \\
  \freenames{\dropn{x}} & := & \{ x \}
\end{eqnarray*}

The bound names of a process, $\boundnames{P}$, are those names occurring in $P$
that are not free. For example, in $x?(y).0$, the name $x$ is free, while $y$ is bound.

\begin{mathpar}
  \inferrule* [lab=monoidal-laws] {} { P|Q \equiv Q|P \and P|0 \equiv P \and P|(Q|R) \equiv (P|Q)|R }
\end{mathpar}

\begin{mathpar}
  \inferrule* [lab=alpha-equivalence] {} { (x)P \equiv (y)P\{y/x\} \and y \not\in \freenames{P} }
\end{mathpar}

\begin{definition}
Then two processes, $P,Q$, are alpha-equivalent if $P = Q\{\vec{y}/\vec{x}\}$ for
some $\vec{x} \in \boundnames{Q},\vec{y} \in \boundnames{P}$, where $Q\{\vec{y}/\vec{x}\}$
denotes the capture-avoiding substitution of $\vec{y}$ for $\vec{x}$ in $Q$.
\end{definition}

\begin{definition}
  The {\em structural congruence} \cite{SangiorgiWalker} , $\equiv$,
  between processes is the least congruence containing
  alpha-equivalence, satisfying the abelian monoid laws
  (associativity, commutativity and $\pzero$ as identity) for parallel
  composition $|$ and for summation $+$.
\end{definition}

\subsection{Name equivalence}

We take name equivalence, written $\nameeq$, to be the smallest
equivalence relation generated by the following rules.

\begin{mathpar}
\inferrule*[lab=Quote-drop]
{ }
{ \quotep{@{x}} \nameeq x }

\inferrule*[lab=Struct-equiv]
{ P \scong Q }
{ \quotep{P} \nameeq \quotep{Q} }
\end{mathpar}

The astute reader will have noticed that the mutual recursion of names
and processes imposes a mutual recursion on alpha-equivalence and
structural equivalence via name-equivalence. Fortunately, all of this
works out pleasantly and we may calculate in the natural way, free of
concern. The reader interested in the details is referred to the
appendix \ref{appendix:rho_details}.

\subsection{Substitution}

We use $\Proc$ for the set of processes, $\QProc$ for the set of
names, and $\id{\{}\vec{y} / \vec{x} \id{\}}$ to denote partial maps,
$s : \QProc \rightarrow \QProc$. A map, $s$ lifts, uniquely, to a map
on process terms, $\widehat{s} : \Proc \rightarrow \Proc$ by the
following equations.

\begin{mathpar}
  (0) \psubstp{Q}{P} := 0 \\
  (R \juxtap S) \psubstp{Q}{P}
  :=    
  (R)\psubstp{Q}{P} \juxtap (S) \psubstp{Q}{P} \\
  (x?(y).R) \psubstp{Q}{P}    
  :=    
  (x)\substp{Q}{P} (z)\concat( (R \psubstn{z}{y}) \psubstp{Q}{P} ) \\
  (\lift{x}{R}) \psubstp{Q}{P}  
  :=
  \lift{(x)\substp{Q}{P}}{ R \psubstp{Q}{P} } \\
%   (\dropn{x})  \psubstp{Q}{P}       
%   := 
%   \left\{ 
%     \begin{array}{ccc} 
%       \dropn{\quotep{Q}} & & x \nameeq \quotep{P} \\
%       \dropn{x} & & otherwise \\
%     \end{array}
%   \right. 
  (\dropn{x})  \psubstp{Q}{P}       
  := 
  \left\{ 
    \begin{array}{ccc} 
      Q & & x \nameeq \quotep{P} \\
      \dropn{x} & & otherwise \\
    \end{array}
  \right.
\end{mathpar}
 

where

\begin{eqnarray}
  (x)\id{\{} \lpquote Q \rpquote / \lpquote P \rpquote \id{\}}            = 
  \left\{ 
    \begin{array}{ccc}
      \lpquote Q \rpquote & & x \nameeq \lpquote P \rpquote \\
      x & & otherwise \\
    \end{array}
  \right. \nonumber
\end{eqnarray}

and $z$ is chosen distinct from $\quotep{P}$, $\quotep{Q}$, the free
names in $Q$, and all the names in $R$. Our $\alpha$-equivalence will
be built in the standard way from this substitution.

\begin{remark}\label{rem:no_self_referential_names}
  One consequence of these definitions is that $\forall P. \quotep{P}
  \not\in \freenames{P}$.
\end{remark}

\subsection{ Dynamic quote: an example }

Anticipating something of what's to come, consider applying the
substitution, $\widehat{\id{\{}u / z \id{\}}}$, to the following pair
of processes, $\lift{w}{y!(z)}$ and $w[ \lpquote y!(z) \rpquote ]$.

\begin{eqnarray}
	\lift{w}{y!(z)}\widehat{\id{\{}u / z \id{\}}}
		& = &
		\lift{w}{y!(u)} \nonumber\\
	w[ \lpquote y!(z) \rpquote ] \widehat{ \id{\{}u / z \id{\}} }
		& = &
		w[ \lpquote y!(z) \rpquote ] \nonumber
\end{eqnarray}

Because the body of the process between quotes is impervious to
substitution, we get radically different answers. In fact, by
examining the first process in an input context,
e.g. $x?(z).\lift{w}{y!(z)}$, we see that the process under the lift
operator may be shaped by prefixed inputs binding a name inside it. In
this sense, the lift operator will be seen as a way to dynamically
construct processes before reifying them as names.

Finally equipped with these standard features we can present the
dynamics of the calculus.

\subsubsection{Operational semantics} 

Finally, we introduce the computational dynamics. What marks these
algebras as distinct from other more traditionally studied algebraic
structures, e.g. vector spaces or polynomial rings, is the manner in
which dynamics is captured. In traditional structures, dynamics is typically
expressed through morphisms between such structures, as in linear maps
between vector spaces or morphisms between rings. In algebras
associated with the semantics of computation, the dynamics is
expressed as part of the algebraic structure itself, through a
reduction reduction relation typically denoted by $\red$. Below, we
give a recursive presentation of this relation for the calculus used
in the encoding.

$\red \subseteq \pi \times \pi$
$\red : \pi \to \mathcal{P}(\pi)$

\begin{mathpar}
  \inferrule* [lab=Comm] { \textsf{match}( x_{src}, x_{trgt} ) } { x_{trgt}?(y)P \; | \; x_{src}!\langle {Q} \rangle \red P\{\quotep{Q}/y}\} }
  \and \\
  \inferrule* [lab=Par] {{P} \red {P}'} {{{P} | {Q}} \red {{P}' | {Q}}}
  \and
  \inferrule* [lab=Equiv]{{{P} \scong {P}'} \andalso {{P}' \red {Q}'} \andalso {{Q}' \scong {Q}}}{{P} \red {Q}}
\end{mathpar}

\begin{eqnarray*}
  match_{\equiv} (\quotep{P},\quotep{Q}) & := & P \equiv Q \\
  match_{\dagger}(\quotep{P},\quotep{Q}) & := & \forall R. P|Q \red^{*} R => R \red^{*} 0 \\
  match_{K}(\quotep{P},\quotep{Q}) & := & K \mbox{ for some context } K
\end{eqnarray*}

$u?(x)P | u!\langle Q \rangle \red P\{\quotep{Q}/x\}$

%We write $\wred$ for $\red^*$, and $P\red$ if $\exists Q $ such that $ P \red Q$.
We write $P\red$ if $\exists Q $ such that $ P \red Q$ and $P\not\red$, otherwise.

\section{Replication}

As mentioned before, it is known that replication (and hence
recursion) can be implemented in a higher-order process algebra
\cite{SangiorgiWalker}. As our first example of calculation with the
machinery thus far presented we give the construction explicitly in
the {\rhoc}.

\begin{eqnarray}
	D_{x} & := & \prefix{x}{y}{(\binpar{\outputp{x}{y}}{@{y}})} \nonumber\\
	\bangp_{x}{P} & := & \binpar{{x}!\langle{\binpar{D_{x}}{P}}\rangle}{D_{x}} \nonumber
\end{eqnarray}

\begin{eqnarray}
	\bangp_{x}{P} & & \nonumber\\
	=
	& {x}!\langle{(\prefix{x}{y}{(\outputp{x}{y} | @{y})) | P}}\rangle 
	      | \prefix{x}{y}{(\outputp{x}{y} | @{y})} & \nonumber\\
	\red
	& (\outputp{x}{y} | @{y})\substn{\quotep{(\prefix{x}{y}{(@{y} | \outputp{x}{y})) | P}}}{y} & \nonumber\\
	=
	& \outputp{x}{\quotep{(\prefix{x}{y}{(\outputp{x}{y} | @{y})) | P}}}
	  | {(\prefix{x}{y}{(\outputp{x}{y} | @{y})) | P}} & \nonumber\\
	\red
	& \ldots & \nonumber\\
	\red^*
	& P | P | \ldots & \nonumber
\end{eqnarray}

Of course, this encoding, as an implementation, runs away, unfolding
$\bangp{P}$ eagerly. A lazier and more implementable replication
operator, restricted to input-guarded processes, may be obtained as follows.

\begin{eqnarray}
\bangp{\prefix{u}{v}{P}} 
	:= 
	\binpar{\lift{x}{\prefix{u}{v}{(\binpar{D(x)}{P})}}}{D(x)} \nonumber
\end{eqnarray}

\begin{remark}
  Note that the lazier definition still does not deal with summation
  or mixed summation (i.e. sums over input and output). The reader is
  invited to construct definitions of replication that deal with these
  features. 

  Further, the definitions are parameterized in a name, $x$. Can you,
  gentle reader, make a definition that eliminates this parameter and
  guarantees no accidental interaction between the replication
  machinery and the process being replicated -- i.e. no accidental
  sharing of names used by the process to get its work done and the
  name(s) used by the replication to effect copying. This latter
  revision of the definition of replication is crucial to obtaining
  the expected identity $!!P \sim !P$.
\end{remark}

\begin{remark}\label{rem:paradoxical_combinator}
  The reader familiar with the lambda calculus will have noticed the
  similarity between $D$ and the paradoxical combinator.

  [Ed. note: the existence of this seems to suggest we have to be more
  restrictive on the set of processes and names we admit if we are to
  support no-cloning.]
\end{remark}

\subsubsection{Bisimulation}

The computational dynamics gives rise to another kind of equivalence,
the equivalence of computational behavior. As previously mentioned
this is typically captured \emph{via} some form of bisimulation.

% The notion we use in this paper is weak barbed bisimulation
% \cite{milner91polyadicpi}.

The notion we use in this paper is derived from weak barbed
bisimulation \cite{milner91polyadicpi}. 

\begin{definition}
An \emph{observation relation}, $\downarrow_{\mathcal N}$, over a set
of names, $\mathcal N$, is the smallest relation satisfying the rules
below.

\infrule[Out-barb]{y \in {\mathcal N}, \; x \nameeq y}
		  {\outputp{x}{v} \downarrow_{\mathcal N} x}
\infrule[Par-barb]{\mbox{$P\downarrow_{\mathcal N} x$ or $Q\downarrow_{\mathcal N} x$}}
		  {\binpar{P}{Q} \downarrow_{\mathcal N} x}

We write $P \Downarrow_{\mathcal N} x$ if there is $Q$ such that 
$P \wred Q$ and $Q \downarrow_{\mathcal N} x$.
\end{definition}

\begin{definition}
%\label{def.bbisim}
An  ${\mathcal N}$-\emph{barbed bisimulation} over a set of names, ${\mathcal N}$, is a symmetric binary relation 
${\mathcal S}_{\mathcal N}$ between agents such that $P\rel{S}_{\mathcal N}Q$ implies:
\begin{enumerate}
\item If $P \red P'$ then $Q \wred Q'$ and $P'\rel{S}_{\mathcal N} Q'$.
\item If $P\downarrow_{\mathcal N} x$, then $Q\Downarrow_{\mathcal N} x$.
\end{enumerate}
$P$ is ${\mathcal N}$-barbed bisimilar to $Q$, written
$P \wbbisim_{\mathcal N} Q$, if $P \rel{S}_{\mathcal N} Q$ for some ${\mathcal N}$-barbed bisimulation ${\mathcal S}_{\mathcal N}$.
\end{definition}

$\mathcal{R} \subseteq \pi \times \pi$

$P \mathcal{R} Q => \forall P'. P \red P' \Rightarrow \exists Q'. Q \red Q', P' \mathcal{R} Q'$

$P \vdash x \Rightarrow Q \vdash x$

\begin{mathpar}
  \inferrule*[lab=Out-barb]{x \nameeq y}{{y}!\langle{Q}\rangle \vdash x}
  \and
  \inferrule*[lab=Par-barb]{\mbox{$P\vdash x$ or $Q\vdash x$}}{\binpar{P}{Q} \vdash x}
\end{mathpar}

\subsubsection{Contexts}

One of the principle advantages of computational calculi like the
$\pi$-calculus is a well-defined notion of context,
contextual-equivalence and a correlation between
contextual-equivalence and notions of bisimulation. The notion of
context allows the decomposition of a process into (sub-)process and
its syntactic environment, its context. Thus, a context may be
thought of as a process with a ``hole'' (written $\Box$) in it. The
application of a context $M$ to a process $P$, written $M[P]$, is
tantamount to filling the hole in $M$ with $P$. In this paper we do
not need the full weight of this theory, but do make use of the notion
of context in the proof the main theorem. 

\begin{mathpar}
  \inferrule* [lab=summation] {} {{M_{M},M_{N}} \bc \Box \;|\; x.M_{A} \;|\; M_{M}+M_{N}}
  \and
  \inferrule* [lab=agent] {} {{M_{A}} \bc (\vec{x})M_{P} \;| \; \clift{P_0,\ldots,M_{P},\ldots,P_N}}
  \and \\
  \inferrule* [lab=process] {} {{M_{P}} \bc M_{N} \;| \;P|M_{P} }
\end{mathpar} 

\begin{mathpar}
  \inferrule* [lab=sychronization] {} {M_{N} \bc \Box \;|\; x?M_{F} \;|\; x!M_{C}}
  \and
  \inferrule* [lab=abstraction] {} {{M_{F}} \bc (x)M_{P} }
  \and
  \inferrule* [lab=concretion] {} {{M_{C}} \bc \langle M_{P} \rangle }
  \and \\
  \inferrule* [lab=process] {} {{M_{P}} \bc M_{N} \;| \;P|M_{P} }
\end{mathpar}

\begin{definition}[contextual application] Given a context $M$, and
  process $P$, we define the \emph{contextual application}, $M[P] :=
  M\{P/\Box\}$. That is, the contextual application of M to P is the
  substitution of $P$ for $\Box$ in $M$.
\end{definition}

$\meaningof{-} : L \to \mathcal{P}(\pi)$

\begin{mathpar}
  \inferrule* [lab=collection] {} {\meaningof{true} = \pi, \and \meaningof{~E} = \pi \setminus \meaningof{E}, \and \meaningof{E_{1} \& E_{2}} = \meaningof{E_{1}} \cap \meaningof{E_{2}}}
\end{mathpar}

\begin{mathpar}
  \inferrule* [lab=structure] {} {\meaningof{0} = \{ P \in \pi | P \equiv 0 \}, \and \\ \meaningof{E_1 | E_2} = \{ P \in \pi | P \equiv P_{1} | P_{2}, P_{1} \in \meaningof{E_{1}}, P_{2} \in \meaningof{E_2}\} }
\end{mathpar}

\begin{mathpar}
 \inferrule* [lab=behavior] {} {\meaningof{\langle a?b \rangle E} = \{ P \in \pi | P \equiv Q | u?(y)P', \\ \and \\\\ \and \\ \;\;\; u \in \meaningof{a}, \forall z.P'\{z/y\} \in \meaningof{E\{z/b\}}\}, \and \\ \meaningof{a!E} = \{ P \in \pi | P \equiv Q | x!\langle P' \rangle, x \in \meaningof{a} P' \in \meaningof{E}\} }
\end{mathpar}

\begin{mathpar}
 \inferrule* [lab=nominal] {} {\meaningof{\quotep{E}} = \{ \quotep{P} \in \quotep{\pi} | P \in \meaningof{E} \}, \and \meaningof{\quotep{P}} = \{ \quotep{Q} \in \quotep{\pi} | P \equiv Q \} \and \\ \meaningof{@\quotep{E}} = \{ P \in \pi | P \equiv @x, x \in \meaningof{E} \}}
\end{mathpar}

\begin{eqnarray*}
  \\
  \meaningof{-} : TS \to ST
\end{eqnarray*}

\begin{eqnarray*}
  \\
  L : TS \to ST
\end{eqnarray*}

\begin{eqnarray*}
  \\
  P \models E \iff P \in \meaningof{E}
\end{eqnarray*}

\begin{eqnarray*}
  P \approx_{L} Q \iff \forall E \in L. P \models E \iff Q \models E
\end{eqnarray*}

\begin{eqnarray*}
  P \approx_{K} Q
\end{eqnarray*}

\begin{eqnarray*}
  P \approx Q
\end{eqnarray*}

$\approx_{K} = \approx = \approx_{L}$

\subsubsection{Contextual duality}

Note that contexts extend the quotation operation to a family of
operations from processes to names. Given a context, $M$, we can
define a \emph{nominal context}, $\quotep{M}$ by $\quotep{M}[P] :=
\quotep{M[P]}$. To foreshadow what is to come we observe that these
operations enjoy a duality with processes very much like the duality
between vectors and maps from vectors to scalars.

Further, because the calculus is essentially higher-order, we have a
correspondence between contexts and processes. More specifically,
given a name $x$ and a context $M$ we can construct $M^{*}_{x}$ such
that 

\begin{mathpar}
  M^{*}_{x} | \lift{x}{P} \red M[P]
\end{mathpar}

namely,

\begin{mathpar}
  M^{*}_{x} := x?(u).M[\dropn{u}]
\end{mathpar}

The dependence of $M^{*}_{x}$ on a name makes it an abstraction, 

\begin{mathpar}
  M^{*} := (x)x?(u).M[\dropn{u}]
\end{mathpar}

\subsection{Additional notation}

It will sometimes be convenient to denote the process a name
quotes. We already have the notation $x = \quotep{P}$, but it will be
convenient to introduce an alternate notation, $\procn{x}$, when we
want to emphasize the connection to the use of the name. Note that, by
virtue of name equivalence, $\quotep{\procn{x}} \nameeq x$; so, the
notation is consistent with previous definitions.

Further, because names have structure it is possible to effect
substitutions on the basis of that structure. This means we need to
upgrade our notation for substitutions, which we accomplish by
adapting comprehension notation. Thus,

\begin{mathpar}
  P\{ y / x : x \in S \}
\end{mathpar}

is interpreted to mean the process derived from P by replacing (in a
capture-avoiding manner) each occurrence of $x$ in $S$ by $y$. For example,

\begin{mathpar}
  P\{ \quotep{\procn{x}|\procn{x}} / x : x \in \freenames{P} \}
\end{mathpar}

will replace each (occurrence) of a free name $x$ in $P$ by
$\quotep{\procn{x}|\procn{x}}$.

Also, we will avail ourselves of the notation $x^{L}$ and $x^{R}$ to
denote injections of a name into disjoint copies of the name
space. There are numerous ways to accomplish this. One example can be
found in \cite{MeredithR05}. This notation overloads to vectors of
names: $\vec{x}^{\pi} := (x_{i}^{\pi} \; : \; 0 \leq i < |\vec{x}| )$ where $\pi \in \{L,R\}$.

We also use $P^{\Box} := P|\Box$.

In \cite{MeredithR05} an interpretation of the new operator is
given. It turns out that there are several possible interpretations
all enjoying the requisite algebraic properties of the operator (see
\cite{milner91polyadicpi}). We will therefore make liberal use of
$(\nu\; \vec{x})P$.

% subsection the_syntax_and_semantics_of_the_notation_system (end)   

\input{qm2pi.qmops} 

\input{qm2pi.sterngerlach} 

\input{qm2pi.metric} 

% section concurrent_process_calculi (end)

%\input{qm2pi.proofsketch}

% section proof sketch (end)

%\input{qm2pi.slviaknots} 

% section spatial logic via knots (end)

\input{qm2pi.conclusion}

% section conclusion (end)

%\input{qm2pi.dtcodes} 

% section wiring algorithm (end)

\input{qm2pi.ack} 

% section acknowledgments (end)

\newpage


\bibliographystyle{plain}   
\bibliography{../../biblios/main.bib}

\input{qm2pi.rhodetails}

\end{document}

 

% subsection basic_interpretation (end)

%\input{qm2pi.rho.presentation} 
\subsection{The syntax and semantics of the notation system}\label{sub:the_syntax_and_semantics_of_the_notation_system} % (fold)

We now summarize a technical presentation of the calculus that
embodies our theory of dynamics. The typical presentation of such a
calculus follows the style of giving generators and relations on
them. The grammar, below, describing term constructors, freely
generates the set of processes, $\Proc$. This set is then quotiented
by a relation known as structural congruence and it is over this set
that the notion of dynamics is expressed. This presentation is
essentially that of \cite{MeredithR05} with the addition of
polyadicity and summation. For readability we have relegated some of
the technical subtleties to an appendix.

\subsubsection{Process grammar}\label{subsub:process_grammar}

\begin{mathpar}
  \inferrule* [lab=synchronization] {} {{M} \bc \pzero \;|\; x?F \;|\; x!C }
  \and
  \inferrule* [lab=abstraction] {} {{F} \bc (x)P}
  \and
  \inferrule* [lab=concretion] {} {{C} \bc \langle Q \rangle}
  \and
  \inferrule* [lab=process] {} {{P,Q} \bc M \;| \;P|Q \;|\; @{x}}
  \and
  \inferrule* [lab=name] {} {{x} \bc \quotep{P}}
\end{mathpar} 

Note that $\vec{x}$ (resp. $\vec{P}$) denotes a vector of names
(resp. processes) of length $|\vec{x}|$ (resp. $|\vec{P}|$). We adopt
the following useful abbreviations.

\begin{mathpar}
   x?(\vec{y}).P := x.(\vec{y})P \and  x\clift{\vec{P}} := x.\clift{\vec{P}}
   \and x!(y) := \lift{x}{\dropn{y}}
   \and \Pi_{i=0}^{n-1}P_i := P_0 | \ldots | P_{n-1}
\end{mathpar}

\subsubsection{Structural congruence}

\paragraph{Free and bound names and alpha-equivalence.} At the
core of structural equivalence is alpha-equivalence which identifies
process that are the same up to a change of variable. Formally, we
recognize the distinction between free and bound names. The free names
of a process, $\freenames{P}$, may be calculated recursively as
follows:

\begin{mathpar}
\freenames{\pzero} := \emptyset
  \and \\
  \freenames{x?(y).P} := \{ x \} \cup (\freenames{P} \setminus \{ y \})
  \and 
  \freenames{x!\langle P \rangle} := \{ x \} \cup \{ P \} 
  \and \\
  \freenames{P|Q} := \freenames{P} \cup \freenames{Q}
  \and \\
  \freenames{@{x}} := \{ x \}
\end{mathpar}

$\pi$
$\quotep{\pi}$

$\freenames{-} : \pi \to \mathcal{P}(\quotep{\pi})$

\begin{eqnarray*}
  \freenames{\pzero} & := & \emptyset \\
  \freenames{x?(y).P} & := & \{ x \} \cup (\freenames{P} \setminus \{ y \}) \\
  \freenames{x!\langle P \rangle} & := & \{ x \} \cup \{ P \} \\
  \freenames{P|Q} & := & \freenames{P} \cup \freenames{Q} \\
  \freenames{\dropn{x}} & := & \{ x \}
\end{eqnarray*}

The bound names of a process, $\boundnames{P}$, are those names occurring in $P$
that are not free. For example, in $x?(y).0$, the name $x$ is free, while $y$ is bound.

\begin{mathpar}
  \inferrule* [lab=monoidal-laws] {} { P|Q \equiv Q|P \and P|0 \equiv P \and P|(Q|R) \equiv (P|Q)|R }
\end{mathpar}

\begin{mathpar}
  \inferrule* [lab=alpha-equivalence] {} { (x)P \equiv (y)P\{y/x\} \and y \not\in \freenames{P} }
\end{mathpar}

\begin{definition}
Then two processes, $P,Q$, are alpha-equivalent if $P = Q\{\vec{y}/\vec{x}\}$ for
some $\vec{x} \in \boundnames{Q},\vec{y} \in \boundnames{P}$, where $Q\{\vec{y}/\vec{x}\}$
denotes the capture-avoiding substitution of $\vec{y}$ for $\vec{x}$ in $Q$.
\end{definition}

\begin{definition}
  The {\em structural congruence} \cite{SangiorgiWalker} , $\equiv$,
  between processes is the least congruence containing
  alpha-equivalence, satisfying the abelian monoid laws
  (associativity, commutativity and $\pzero$ as identity) for parallel
  composition $|$ and for summation $+$.
\end{definition}

\subsection{Name equivalence}

We take name equivalence, written $\nameeq$, to be the smallest
equivalence relation generated by the following rules.

\begin{mathpar}
\inferrule*[lab=Quote-drop]
{ }
{ \quotep{@{x}} \nameeq x }

\inferrule*[lab=Struct-equiv]
{ P \scong Q }
{ \quotep{P} \nameeq \quotep{Q} }
\end{mathpar}

The astute reader will have noticed that the mutual recursion of names
and processes imposes a mutual recursion on alpha-equivalence and
structural equivalence via name-equivalence. Fortunately, all of this
works out pleasantly and we may calculate in the natural way, free of
concern. The reader interested in the details is referred to the
appendix \ref{appendix:rho_details}.

\subsection{Substitution}

We use $\Proc$ for the set of processes, $\QProc$ for the set of
names, and $\id{\{}\vec{y} / \vec{x} \id{\}}$ to denote partial maps,
$s : \QProc \rightarrow \QProc$. A map, $s$ lifts, uniquely, to a map
on process terms, $\widehat{s} : \Proc \rightarrow \Proc$ by the
following equations.

\begin{mathpar}
  (0) \psubstp{Q}{P} := 0 \\
  (R \juxtap S) \psubstp{Q}{P}
  :=    
  (R)\psubstp{Q}{P} \juxtap (S) \psubstp{Q}{P} \\
  (x?(y).R) \psubstp{Q}{P}    
  :=    
  (x)\substp{Q}{P} (z)\concat( (R \psubstn{z}{y}) \psubstp{Q}{P} ) \\
  (\lift{x}{R}) \psubstp{Q}{P}  
  :=
  \lift{(x)\substp{Q}{P}}{ R \psubstp{Q}{P} } \\
%   (\dropn{x})  \psubstp{Q}{P}       
%   := 
%   \left\{ 
%     \begin{array}{ccc} 
%       \dropn{\quotep{Q}} & & x \nameeq \quotep{P} \\
%       \dropn{x} & & otherwise \\
%     \end{array}
%   \right. 
  (\dropn{x})  \psubstp{Q}{P}       
  := 
  \left\{ 
    \begin{array}{ccc} 
      Q & & x \nameeq \quotep{P} \\
      \dropn{x} & & otherwise \\
    \end{array}
  \right.
\end{mathpar}
 

where

\begin{eqnarray}
  (x)\id{\{} \lpquote Q \rpquote / \lpquote P \rpquote \id{\}}            = 
  \left\{ 
    \begin{array}{ccc}
      \lpquote Q \rpquote & & x \nameeq \lpquote P \rpquote \\
      x & & otherwise \\
    \end{array}
  \right. \nonumber
\end{eqnarray}

and $z$ is chosen distinct from $\quotep{P}$, $\quotep{Q}$, the free
names in $Q$, and all the names in $R$. Our $\alpha$-equivalence will
be built in the standard way from this substitution.

\begin{remark}\label{rem:no_self_referential_names}
  One consequence of these definitions is that $\forall P. \quotep{P}
  \not\in \freenames{P}$.
\end{remark}

\subsection{ Dynamic quote: an example }

Anticipating something of what's to come, consider applying the
substitution, $\widehat{\id{\{}u / z \id{\}}}$, to the following pair
of processes, $\lift{w}{y!(z)}$ and $w[ \lpquote y!(z) \rpquote ]$.

\begin{eqnarray}
	\lift{w}{y!(z)}\widehat{\id{\{}u / z \id{\}}}
		& = &
		\lift{w}{y!(u)} \nonumber\\
	w[ \lpquote y!(z) \rpquote ] \widehat{ \id{\{}u / z \id{\}} }
		& = &
		w[ \lpquote y!(z) \rpquote ] \nonumber
\end{eqnarray}

Because the body of the process between quotes is impervious to
substitution, we get radically different answers. In fact, by
examining the first process in an input context,
e.g. $x?(z).\lift{w}{y!(z)}$, we see that the process under the lift
operator may be shaped by prefixed inputs binding a name inside it. In
this sense, the lift operator will be seen as a way to dynamically
construct processes before reifying them as names.

Finally equipped with these standard features we can present the
dynamics of the calculus.

\subsubsection{Operational semantics} 

Finally, we introduce the computational dynamics. What marks these
algebras as distinct from other more traditionally studied algebraic
structures, e.g. vector spaces or polynomial rings, is the manner in
which dynamics is captured. In traditional structures, dynamics is typically
expressed through morphisms between such structures, as in linear maps
between vector spaces or morphisms between rings. In algebras
associated with the semantics of computation, the dynamics is
expressed as part of the algebraic structure itself, through a
reduction reduction relation typically denoted by $\red$. Below, we
give a recursive presentation of this relation for the calculus used
in the encoding.

$\red \subseteq \pi \times \pi$
$\red : \pi \to \mathcal{P}(\pi)$

\begin{mathpar}
  \inferrule* [lab=Comm] { \textsf{match}( x_{src}, x_{trgt} ) } { x_{trgt}?(y)P \; | \; x_{src}!\langle {Q} \rangle \red P\{\quotep{Q}/y}\} }
  \and \\
  \inferrule* [lab=Par] {{P} \red {P}'} {{{P} | {Q}} \red {{P}' | {Q}}}
  \and
  \inferrule* [lab=Equiv]{{{P} \scong {P}'} \andalso {{P}' \red {Q}'} \andalso {{Q}' \scong {Q}}}{{P} \red {Q}}
\end{mathpar}

\begin{eqnarray*}
  match_{\equiv} (\quotep{P},\quotep{Q}) & := & P \equiv Q \\
  match_{\dagger}(\quotep{P},\quotep{Q}) & := & \forall R. P|Q \red^{*} R => R \red^{*} 0 \\
  match_{K}(\quotep{P},\quotep{Q}) & := & K \mbox{ for some context } K
\end{eqnarray*}

$u?(x)P | u!\langle Q \rangle \red P\{\quotep{Q}/x\}$

%We write $\wred$ for $\red^*$, and $P\red$ if $\exists Q $ such that $ P \red Q$.
We write $P\red$ if $\exists Q $ such that $ P \red Q$ and $P\not\red$, otherwise.

\section{Replication}

As mentioned before, it is known that replication (and hence
recursion) can be implemented in a higher-order process algebra
\cite{SangiorgiWalker}. As our first example of calculation with the
machinery thus far presented we give the construction explicitly in
the {\rhoc}.

\begin{eqnarray}
	D_{x} & := & \prefix{x}{y}{(\binpar{\outputp{x}{y}}{@{y}})} \nonumber\\
	\bangp_{x}{P} & := & \binpar{{x}!\langle{\binpar{D_{x}}{P}}\rangle}{D_{x}} \nonumber
\end{eqnarray}

\begin{eqnarray}
	\bangp_{x}{P} & & \nonumber\\
	=
	& {x}!\langle{(\prefix{x}{y}{(\outputp{x}{y} | @{y})) | P}}\rangle 
	      | \prefix{x}{y}{(\outputp{x}{y} | @{y})} & \nonumber\\
	\red
	& (\outputp{x}{y} | @{y})\substn{\quotep{(\prefix{x}{y}{(@{y} | \outputp{x}{y})) | P}}}{y} & \nonumber\\
	=
	& \outputp{x}{\quotep{(\prefix{x}{y}{(\outputp{x}{y} | @{y})) | P}}}
	  | {(\prefix{x}{y}{(\outputp{x}{y} | @{y})) | P}} & \nonumber\\
	\red
	& \ldots & \nonumber\\
	\red^*
	& P | P | \ldots & \nonumber
\end{eqnarray}

Of course, this encoding, as an implementation, runs away, unfolding
$\bangp{P}$ eagerly. A lazier and more implementable replication
operator, restricted to input-guarded processes, may be obtained as follows.

\begin{eqnarray}
\bangp{\prefix{u}{v}{P}} 
	:= 
	\binpar{\lift{x}{\prefix{u}{v}{(\binpar{D(x)}{P})}}}{D(x)} \nonumber
\end{eqnarray}

\begin{remark}
  Note that the lazier definition still does not deal with summation
  or mixed summation (i.e. sums over input and output). The reader is
  invited to construct definitions of replication that deal with these
  features. 

  Further, the definitions are parameterized in a name, $x$. Can you,
  gentle reader, make a definition that eliminates this parameter and
  guarantees no accidental interaction between the replication
  machinery and the process being replicated -- i.e. no accidental
  sharing of names used by the process to get its work done and the
  name(s) used by the replication to effect copying. This latter
  revision of the definition of replication is crucial to obtaining
  the expected identity $!!P \sim !P$.
\end{remark}

\begin{remark}\label{rem:paradoxical_combinator}
  The reader familiar with the lambda calculus will have noticed the
  similarity between $D$ and the paradoxical combinator.

  [Ed. note: the existence of this seems to suggest we have to be more
  restrictive on the set of processes and names we admit if we are to
  support no-cloning.]
\end{remark}

\subsubsection{Bisimulation}

The computational dynamics gives rise to another kind of equivalence,
the equivalence of computational behavior. As previously mentioned
this is typically captured \emph{via} some form of bisimulation.

% The notion we use in this paper is weak barbed bisimulation
% \cite{milner91polyadicpi}.

The notion we use in this paper is derived from weak barbed
bisimulation \cite{milner91polyadicpi}. 

\begin{definition}
An \emph{observation relation}, $\downarrow_{\mathcal N}$, over a set
of names, $\mathcal N$, is the smallest relation satisfying the rules
below.

\infrule[Out-barb]{y \in {\mathcal N}, \; x \nameeq y}
		  {\outputp{x}{v} \downarrow_{\mathcal N} x}
\infrule[Par-barb]{\mbox{$P\downarrow_{\mathcal N} x$ or $Q\downarrow_{\mathcal N} x$}}
		  {\binpar{P}{Q} \downarrow_{\mathcal N} x}

We write $P \Downarrow_{\mathcal N} x$ if there is $Q$ such that 
$P \wred Q$ and $Q \downarrow_{\mathcal N} x$.
\end{definition}

\begin{definition}
%\label{def.bbisim}
An  ${\mathcal N}$-\emph{barbed bisimulation} over a set of names, ${\mathcal N}$, is a symmetric binary relation 
${\mathcal S}_{\mathcal N}$ between agents such that $P\rel{S}_{\mathcal N}Q$ implies:
\begin{enumerate}
\item If $P \red P'$ then $Q \wred Q'$ and $P'\rel{S}_{\mathcal N} Q'$.
\item If $P\downarrow_{\mathcal N} x$, then $Q\Downarrow_{\mathcal N} x$.
\end{enumerate}
$P$ is ${\mathcal N}$-barbed bisimilar to $Q$, written
$P \wbbisim_{\mathcal N} Q$, if $P \rel{S}_{\mathcal N} Q$ for some ${\mathcal N}$-barbed bisimulation ${\mathcal S}_{\mathcal N}$.
\end{definition}

$\mathcal{R} \subseteq \pi \times \pi$

$P \mathcal{R} Q => \forall P'. P \red P' \Rightarrow \exists Q'. Q \red Q', P' \mathcal{R} Q'$

$P \vdash x \Rightarrow Q \vdash x$

\begin{mathpar}
  \inferrule*[lab=Out-barb]{x \nameeq y}{{y}!\langle{Q}\rangle \vdash x}
  \and
  \inferrule*[lab=Par-barb]{\mbox{$P\vdash x$ or $Q\vdash x$}}{\binpar{P}{Q} \vdash x}
\end{mathpar}

\subsubsection{Contexts}

One of the principle advantages of computational calculi like the
$\pi$-calculus is a well-defined notion of context,
contextual-equivalence and a correlation between
contextual-equivalence and notions of bisimulation. The notion of
context allows the decomposition of a process into (sub-)process and
its syntactic environment, its context. Thus, a context may be
thought of as a process with a ``hole'' (written $\Box$) in it. The
application of a context $M$ to a process $P$, written $M[P]$, is
tantamount to filling the hole in $M$ with $P$. In this paper we do
not need the full weight of this theory, but do make use of the notion
of context in the proof the main theorem. 

\begin{mathpar}
  \inferrule* [lab=summation] {} {{M_{M},M_{N}} \bc \Box \;|\; x.M_{A} \;|\; M_{M}+M_{N}}
  \and
  \inferrule* [lab=agent] {} {{M_{A}} \bc (\vec{x})M_{P} \;| \; \clift{P_0,\ldots,M_{P},\ldots,P_N}}
  \and \\
  \inferrule* [lab=process] {} {{M_{P}} \bc M_{N} \;| \;P|M_{P} }
\end{mathpar} 

\begin{mathpar}
  \inferrule* [lab=sychronization] {} {M_{N} \bc \Box \;|\; x?M_{F} \;|\; x!M_{C}}
  \and
  \inferrule* [lab=abstraction] {} {{M_{F}} \bc (x)M_{P} }
  \and
  \inferrule* [lab=concretion] {} {{M_{C}} \bc \langle M_{P} \rangle }
  \and \\
  \inferrule* [lab=process] {} {{M_{P}} \bc M_{N} \;| \;P|M_{P} }
\end{mathpar}

\begin{definition}[contextual application] Given a context $M$, and
  process $P$, we define the \emph{contextual application}, $M[P] :=
  M\{P/\Box\}$. That is, the contextual application of M to P is the
  substitution of $P$ for $\Box$ in $M$.
\end{definition}

$\meaningof{-} : L \to \mathcal{P}(\pi)$

\begin{mathpar}
  \inferrule* [lab=collection] {} {\meaningof{true} = \pi, \and \meaningof{~E} = \pi \setminus \meaningof{E}, \and \meaningof{E_{1} \& E_{2}} = \meaningof{E_{1}} \cap \meaningof{E_{2}}}
\end{mathpar}

\begin{mathpar}
  \inferrule* [lab=structure] {} {\meaningof{0} = \{ P \in \pi | P \equiv 0 \}, \and \\ \meaningof{E_1 | E_2} = \{ P \in \pi | P \equiv P_{1} | P_{2}, P_{1} \in \meaningof{E_{1}}, P_{2} \in \meaningof{E_2}\} }
\end{mathpar}

\begin{mathpar}
 \inferrule* [lab=behavior] {} {\meaningof{\langle a?b \rangle E} = \{ P \in \pi | P \equiv Q | u?(y)P', \\ \and \\\\ \and \\ \;\;\; u \in \meaningof{a}, \forall z.P'\{z/y\} \in \meaningof{E\{z/b\}}\}, \and \\ \meaningof{a!E} = \{ P \in \pi | P \equiv Q | x!\langle P' \rangle, x \in \meaningof{a} P' \in \meaningof{E}\} }
\end{mathpar}

\begin{mathpar}
 \inferrule* [lab=nominal] {} {\meaningof{\quotep{E}} = \{ \quotep{P} \in \quotep{\pi} | P \in \meaningof{E} \}, \and \meaningof{\quotep{P}} = \{ \quotep{Q} \in \quotep{\pi} | P \equiv Q \} \and \\ \meaningof{@\quotep{E}} = \{ P \in \pi | P \equiv @x, x \in \meaningof{E} \}}
\end{mathpar}

\begin{eqnarray*}
  \\
  \meaningof{-} : TS \to ST
\end{eqnarray*}

\begin{eqnarray*}
  \\
  L : TS \to ST
\end{eqnarray*}

\begin{eqnarray*}
  \\
  P \models E \iff P \in \meaningof{E}
\end{eqnarray*}

\begin{eqnarray*}
  P \approx_{L} Q \iff \forall E \in L. P \models E \iff Q \models E
\end{eqnarray*}

\begin{eqnarray*}
  P \approx_{K} Q
\end{eqnarray*}

\begin{eqnarray*}
  P \approx Q
\end{eqnarray*}

$\approx_{K} = \approx = \approx_{L}$

\subsubsection{Contextual duality}

Note that contexts extend the quotation operation to a family of
operations from processes to names. Given a context, $M$, we can
define a \emph{nominal context}, $\quotep{M}$ by $\quotep{M}[P] :=
\quotep{M[P]}$. To foreshadow what is to come we observe that these
operations enjoy a duality with processes very much like the duality
between vectors and maps from vectors to scalars.

Further, because the calculus is essentially higher-order, we have a
correspondence between contexts and processes. More specifically,
given a name $x$ and a context $M$ we can construct $M^{*}_{x}$ such
that 

\begin{mathpar}
  M^{*}_{x} | \lift{x}{P} \red M[P]
\end{mathpar}

namely,

\begin{mathpar}
  M^{*}_{x} := x?(u).M[\dropn{u}]
\end{mathpar}

The dependence of $M^{*}_{x}$ on a name makes it an abstraction, 

\begin{mathpar}
  M^{*} := (x)x?(u).M[\dropn{u}]
\end{mathpar}

\subsection{Additional notation}

It will sometimes be convenient to denote the process a name
quotes. We already have the notation $x = \quotep{P}$, but it will be
convenient to introduce an alternate notation, $\procn{x}$, when we
want to emphasize the connection to the use of the name. Note that, by
virtue of name equivalence, $\quotep{\procn{x}} \nameeq x$; so, the
notation is consistent with previous definitions.

Further, because names have structure it is possible to effect
substitutions on the basis of that structure. This means we need to
upgrade our notation for substitutions, which we accomplish by
adapting comprehension notation. Thus,

\begin{mathpar}
  P\{ y / x : x \in S \}
\end{mathpar}

is interpreted to mean the process derived from P by replacing (in a
capture-avoiding manner) each occurrence of $x$ in $S$ by $y$. For example,

\begin{mathpar}
  P\{ \quotep{\procn{x}|\procn{x}} / x : x \in \freenames{P} \}
\end{mathpar}

will replace each (occurrence) of a free name $x$ in $P$ by
$\quotep{\procn{x}|\procn{x}}$.

Also, we will avail ourselves of the notation $x^{L}$ and $x^{R}$ to
denote injections of a name into disjoint copies of the name
space. There are numerous ways to accomplish this. One example can be
found in \cite{MeredithR05}. This notation overloads to vectors of
names: $\vec{x}^{\pi} := (x_{i}^{\pi} \; : \; 0 \leq i < |\vec{x}| )$ where $\pi \in \{L,R\}$.

We also use $P^{\Box} := P|\Box$.

In \cite{MeredithR05} an interpretation of the new operator is
given. It turns out that there are several possible interpretations
all enjoying the requisite algebraic properties of the operator (see
\cite{milner91polyadicpi}). We will therefore make liberal use of
$(\nu\; \vec{x})P$.

% subsection the_syntax_and_semantics_of_the_notation_system (end)   

\section{Interpretation of QM}
\subsection{Supporting definitions}
\subsubsection{Multiplication}
\begin{mathpar}
  \quotep{Q} \cdot \quotep{R} := \quotep{Q|R}
  \and \\
  \quotep{Q} \cdot P := P\{ \quotep{Q|R} / \quotep{R} : \quotep{R} \in \freenames{P} \}
\end{mathpar}

\paragraph{Discussion}
The first line needs little explanation. The second line says that
each free name of the process is replaced with the multiplication of
that name by the scalar. Multiplication of a scalar (name) by a state
(process) results in a process all the names of which have been `moved
over' by parallel composition with the process the scalar
quotes. There is a subtlety that the bound names have to be
manipulated so that multiplied names aren't accidentally
captured. There are many ways to achieve this.

\begin{remark}\label{rem:multiplication_identities}
  The reader is invited to verify that for all $x,y,z \in \QProc$ and $P \in \Proc$
  \begin{mathpar}
    x \cdot \quotep{0} \equiv x 
    \and
    x \cdot y \equiv y \cdot x
    \and
    x \cdot (y \cdot z) \equiv (x \cdot y) \cdot z
    \and \\
    \quotep{0} \cdot P \equiv P
    \and \\
    x \cdot (y \cdot P) \equiv (x \cdot y) \cdot P
    \and \\
    x \cdot (P|Q) \equiv (x \cdot P) | (x \cdot Q)
    \and \\    
  \end{mathpar}
\end{remark}

\subsubsection{Tensor product}

We define a tensor product on processes by structural induction.

\paragraph{Tensor of sums} First note that all summations, including
$\pzero$ and sequence, can be written $\Sigma_{i} x_{i}.A_{i} +
\Sigma_{j} x_{j}.C_{j}$, where we have grouped input-guarded processes
together and output-guarded processes together.

Thus, we can define the tensor product of two summations, $N_{1}\otimes N_{2}$, where

\begin{mathpar}
  N_{1} := \Sigma_{i} x_{i}.A_{i} + \Sigma_{j} x_{j}.C_{j}
  \and
  N_{2} := \Sigma_{i'} y_{i'}.B_{i'} + \Sigma_{j'} y_{j'}.D_{j'} 
\end{mathpar}

as follows.

\begin{mathpar}
  \Sigma_{i} x_{i}.A_{i} + \Sigma_{j} x_{j}.C_{j} \otimes \Sigma_{i'}
  y_{i'}.B_{i'} + \Sigma_{j'} y_{j'}.D_{j'} 
  \and \\
  := \; \Sigma_{i} \Sigma_{i'} \quotep{\stackrel{\vee}{x_{i}}| \stackrel{\vee}{y_{i'}}}.(A_{i}\otimes B_{i'}) \; | \; \Sigma_{i'} \Sigma_{i} \quotep{\stackrel{\vee}{y_{i'}}|\stackrel{\vee}{x_{i}}}.(B_{i'}\otimes A_{i})
  \and
  \;\; | \;\; \Sigma_{j} \Sigma_{j'} \quotep{\stackrel{\vee}{x_{j}}|\stackrel{\vee}{y_{j'}}}.(A_{j}\otimes B_{j'}) \; | \; \Sigma_{j'} \Sigma_{j} \quotep{\stackrel{\vee}{y_{j'}}|\stackrel{\vee}{x_{j}}}.(B_{j'}\otimes A_{j})
\end{mathpar}

\begin{remark}
  Do we need to $x^{L}$ and $y^{R}$ for this construction as well?
\end{remark}

\paragraph{Tensor of parallel compositions} Next, we distribute tensor
over par.

\begin{mathpar}
  P_{1}|P_{2} \otimes Q_{1}|Q_{2} := (P_{1} \otimes Q_{1}) | (P_{1}
  \otimes Q_{2}) | (P_{2} \otimes Q_{1}) | (P_{2} \otimes Q_{2})
\end{mathpar}

\paragraph{Tensor with dropped names} We treat tensor of a
process with a dropped name as parallel composition.

\begin{mathpar}
  P \otimes \dropn{x} := P | \dropn{x}
\end{mathpar}

\paragraph{Tensor of agents}

Finally, we need to define tensor on agents. Note that the definition
of tensor on normal products only tensors inputs with inputs and
outputs with outputs. Thus, we only have to define the operation on
``homogeneous'' pairings.

\begin{mathpar}
  (\vec{x})P \otimes (\vec{y})Q
  \and \\
  := (x_{0}^{L}|y_{0}^{R},\ldots,x_{0}^{L}|y_{n}^{R},\ldots,x_{m}^{L}|y_{0}^{R},\ldots,x_{m}^{L}|y_{n}^R)(P\{ \vec{x}^{L}/\vec{x}\} \otimes Q \{ \vec{y}^{R}/\vec{y}\})
  \and \\
  \clift{\vec{P}} \otimes \clift{\vec{Q}}
  \and \\
  := \clift{P_{0}\otimes Q_{0},\ldots,P_{0}\otimes Q_{n},\ldots,P_{m}\otimes Q_{0},\ldots,P_{m}\otimes Q_{n}}
\end{mathpar}

\begin{remark}
  Observe that arities of tensored abstractions matches arities of
  tensored concretions if the original arities matched. Note also that
  the length of the arities corresponds to the increase in dimension
  we see in ordinary vector space tensor product.
\end{remark}

\begin{remark}
  Operationally, this definition distributes the tensor down to
  components ``linked'' by summation. Tensor over summation is
  intriguing in that it mixes names. Moreover, as a consequence of the
  way it mixes names we have the identities for all $x \in \QProc$ and
  $P,Q \in \Proc$

  \begin{mathpar}
    (x \cdot P) \otimes Q \equiv x \cdot (P \otimes Q) \equiv P \otimes (x \cdot Q)
    \and
    P \otimes \pzero \equiv P
  \end{mathpar}

  that the reader is invited to verify.
\end{remark}

\subsubsection{Annihilation}
\begin{mathpar}
  P^{\perp} := \{ Q | \forall R. P|Q \red^{*} R \Rightarrow R \red^{*} \pzero \}
  \and \\
  P^{\underline{\perp}} := \Sigma_{Q \in P^{\perp}} \quotep{Q}?(y).(\dropn{y}|Q) | \Sigma_{Q \in P^{\perp}} \quotep{Q}\clift{\Box}
\end{mathpar}

\paragraph{Discussion} The reader will note that $P^{\perp}$ is a
\emph{set} of processes, while $P^{\underline{\perp}}$ is a
\emph{context}. We call the set $P^{\perp}$ the \emph{annihilators} of
$P$. The parallel composition of a process in the annihilators of $P$
with $P$ will result in a process, the state space of which has all
paths eventually leading to $\pzero$. Execution may endure loops; but
under reasonable conditions of fairness (naturally guaranteed under
most notions of bisimulation) such a composite process cannot get
stuck in such a loop and will, eventually pop out and terminate.

The context $P^{\underline{\perp}}$ is ready and willing to ``take the
$P$ out of'' the process to which it is applied. It will effectively
transmit the code of the process to which it is applied to one of the
annihilators and run the process against it.

\subsubsection{Evaluation}
We fix $M$ a domain of fully abstract interpretation with an equality
coincident with bisimulation. We take $\meaningof{\cdot} : \Proc \to
M$ to be the map interpreting processes and $\nmeaningof{\cdot} : \M
\to Proc$ to be the map running the other way. Then we define

\begin{mathpar}
  \int P := \nmeaningof{\meaningof{P}}
\end{mathpar}

\paragraph{Discussion}
There are many fully abstract interpretations of Milner's
$\pi$-calculus. Any of them can be used as a basis for interpreting
the reflective calculus here. Equipped with such a domain it is
largely a matter of grinding through to check that the Yoneda
construction for the normalization-by-evaluation program can be
extended to this setting.

\begin{remark}
  The reader is invited to verify that $\int (P^{\underline{\perp}}[P]) = 0$.
\end{remark}

\subsection{Quantum mechanics}

Table \ref{tbl:core_qm_op_defns} gives the core operational definitions

\begin{table}[htp]\label{tbl:core_qm_op_defns}
  \center{
    \fbox{
      \begin{tabular}{c|c}
        quantum mechanics & process calculus \\
        \hline
        scalar & $x := \quotep{P}$ \\
        state vector & $\state{P} := P$ \\
        dual & $\state{P}^{*} := \event{P^{\underline{\perp}}} := \quotep{P^{\underline{\perp}}}[-]$ \\
        matrix & $ \Sigma_{\alpha} \state{P_{\alpha}}x_{\alpha}\event{Q_{\alpha}}$ \\
        vector addition & $\state{P} + \state{Q} := \state{P | Q}$ \\
        tensor product & $\state{P} \otimes \state{Q} := \state{P \otimes Q}$ \\
        inner product & $\innerprod{P}{Q} := \quotep{\int P^{\underline{\perp}}[Q]}$ \\
      \end{tabular}
    }
  }
  \caption{QM - operational definitions}
\end{table}

where

\begin{mathpar}
  \prmatrix{P}{Q} := \fprmatrix{P}{\quotep{\pzero}}{Q}
  \and
  \fprmatrix{P}{x}{Q} := (\state{P},x,\event{Q})
  \and
  (\fprmatrix{P}{x}{Q})(\state{R}) := x \cdot \innerprod{Q}{R} \cdot \state{P}
  \and
  (\fprmatrix{P}{x}{Q})(\event{R}) := x \cdot \innerprod{R}{P} \cdot \event{Q}
\end{mathpar}

\paragraph{Discussion}
As promised: vectors (aka states) are represented as processes; duals
as contextual duals; inner product definition should be compared with
standard inner product definition for ....

\begin{remark}
  Assuming $\int (P^{\underline{\perp}}[P]) = 0$, the reader is
  invited to verify that $(\fprmatrix{P}{x}{P})(\state{P}) = x \cdot \state{P}$.
\end{remark}

\begin{remark}
  The reader is invited to verify that $\innerprod{P}{Q}$ could
  equally well have been written $\quotep{\int \stackrel{\vee}{x}}$
  where $x = \event{P^{\underline{\perp}}}(Q)$.

  One of the motivations for this remark is that there is another way
  to factor these operations. We could package up evaluation in the dual:

  \begin{mathpar}
    \state{P}^{*} := \event{\int P^{\underline{\perp}}} := \quotep{\int P^{\underline{\perp}}}[-]
  \end{mathpar}

  and then have inner product defined by
  
  \begin{mathpar}
    \innerprod{P}{Q} := \event{P}(Q)
  \end{mathpar}

  Hopefully, experience with the calculations will provide guidance on
  the best factoring.
\end{remark}

\begin{remark}
  Assuming $\int (P^{\underline{\perp}}[P]) = 0$, the reader is
  invited to verify that $\forall P,Q. (\prmatrix{0}{Q})(\state{0}) =
  \state{0}$ and dually $(\prmatrix{P}{0})(\event{0}) = \event{0}$.
\end{remark}

\begin{remark}
  i'm a little worried that i don't (yet) have proper support for
  complex conjugacy. But, the observation above may give us a
  clue. According to Abramsky, it must be the case that the scalars
  are iso to the homset of the identity for the tensor -- which the
  observation above characterizes. 

  For now, we will simply bookmark the notion with $\overline{x}$.
\end{remark}

\subsubsection{Adjointness}

We need to give a definition of $(\cdot)^{\dagger}$ for matrices. The
obvious candidate definition is
\begin{mathpar}
(\Sigma_{\alpha}\fprmatrix{P_{\alpha}}{x_{\alpha}}{Q_{\alpha}})^{\dagger}
= \Sigma_{\alpha}\fprmatrix{(Q_{\alpha}^{\underline{\perp}})^{*}}{\overline{x}_{\alpha}}{P_{\alpha}^{\underline{\perp}}} 
\end{mathpar}

But, $(Q_{\alpha}^{\underline{\perp}})^{*}$ requires a name along
which to communicate the process to achieve the context application.

\subsubsection{Basis for a basis}
If processes label states and ``addition'' of states (a.k.a. vector
addition) is interpreted as parallel composition, what corresponds to
notions of linear independence and basis? Here, we recall that Yoshida
has developed a set of \emph{combinators} for an asynchronous verison
of Milner's $\pi$-calculus. These are a finite set of processes such
any process can be expressed as parallel composition of these
combinators together with liberal uses of the new operator and
replication. We can simply give a translation of these into the
present calculus and have reasonable expectation that the property
carries over. That is, that the resultant set allows to express all
processes via parallel composition. Note, however, that there is no
new operator or replication in this calculus. As a result, we expect
that the corresponding set is actually infinite. That is, we expect
that the space is actually infinite dimensional.

\begin{remark}
  The attentive reader may be a bit concerned. Certainly, the
  collection $S$, $K$ and $I$ is a finite set of
  combinators. Shouldn't we expect to see a finite set of combinators
  for an effectively equivalent system? i am very sympathetic to this
  critique and feel it warrants full attention. On the other hand, i
  also have in mind the following analogy. The natural numbers, as a
  monoid under addition, has exactly $1$ generator, while the natural
  numbers, as a monoid under multiplication, has countably many
  generators (the primes). We observe that the application of the
  lambda calculus is much less resource sensitive than the parallel
  composition of the $\pi$-calculus. Could it be the case that we have
  an analogy of the form
  
  \begin{mathpar}
    m + n : MN :: m*n : M|N
  \end{mathpar}

  giving a similar blow up in the set of ``primes''?  This is such a
  wonderful thought that, even if it's not true, i think it's worth
  writing down.
\end{remark}
 

\documentclass[12pt]{llncs}
%\documentclass{jktr}

\usepackage[pdftex]{hyperref}                   
\usepackage {listings}
\usepackage {mathpartir}
\usepackage{bcprules}
%\usepackage{listings}
                       
\usepackage{graphicx} 
%\usepackage[margins=2.5cm,nohead,nofoot]{geometry}
%\usepackage{geometry}
\usepackage{amsfonts}
\usepackage{amstext}
\usepackage{latexsym}
\usepackage{amssymb}
\usepackage{color}


%\include{myPreamble}
\include{qm2pi.local} 

%\ifpdf
%\usepackage[pdftex]{graphicx}
%\else
%\usepackage{graphicx}
%\fi

 % \ifpdf
%  \usepackage{pdfsync}
%  \if


%\title{Brief Article}
%\author{David F. Snyder}
%\author{L.G. Meredith}

%\address{Dept. of Math., Texas State University--San Marcos, San Marcos, TX 78666}
       
\pagestyle{empty}


\begin{document}

\lstset{language=[Objective]Caml,frame=shadowbox}

\input{qm2pi.front}

% section front matter (end)

\input{qm2pi.intro} 
 
% section introduction (end)

% \input{qm2pi.knotations} 

% section notation (end)

\input{qm2pi.process.calculi} 

% section concurrent_process_calculi_and_spatial_logics_ (end)
    
%\input{qm2pi.knots2pi} 

%\input{qm2pi.trefoil} 

%\input{qm2pi.mainthm} 

% subsection basic_interpretation (end)

%\input{qm2pi.rho.presentation} 
\subsection{The syntax and semantics of the notation system}\label{sub:the_syntax_and_semantics_of_the_notation_system} % (fold)

We now summarize a technical presentation of the calculus that
embodies our theory of dynamics. The typical presentation of such a
calculus follows the style of giving generators and relations on
them. The grammar, below, describing term constructors, freely
generates the set of processes, $\Proc$. This set is then quotiented
by a relation known as structural congruence and it is over this set
that the notion of dynamics is expressed. This presentation is
essentially that of \cite{MeredithR05} with the addition of
polyadicity and summation. For readability we have relegated some of
the technical subtleties to an appendix.

\subsubsection{Process grammar}\label{subsub:process_grammar}

\begin{mathpar}
  \inferrule* [lab=synchronization] {} {{M} \bc \pzero \;|\; x?F \;|\; x!C }
  \and
  \inferrule* [lab=abstraction] {} {{F} \bc (x)P}
  \and
  \inferrule* [lab=concretion] {} {{C} \bc \langle Q \rangle}
  \and
  \inferrule* [lab=process] {} {{P,Q} \bc M \;| \;P|Q \;|\; @{x}}
  \and
  \inferrule* [lab=name] {} {{x} \bc \quotep{P}}
\end{mathpar} 

Note that $\vec{x}$ (resp. $\vec{P}$) denotes a vector of names
(resp. processes) of length $|\vec{x}|$ (resp. $|\vec{P}|$). We adopt
the following useful abbreviations.

\begin{mathpar}
   x?(\vec{y}).P := x.(\vec{y})P \and  x\clift{\vec{P}} := x.\clift{\vec{P}}
   \and x!(y) := \lift{x}{\dropn{y}}
   \and \Pi_{i=0}^{n-1}P_i := P_0 | \ldots | P_{n-1}
\end{mathpar}

\subsubsection{Structural congruence}

\paragraph{Free and bound names and alpha-equivalence.} At the
core of structural equivalence is alpha-equivalence which identifies
process that are the same up to a change of variable. Formally, we
recognize the distinction between free and bound names. The free names
of a process, $\freenames{P}$, may be calculated recursively as
follows:

\begin{mathpar}
\freenames{\pzero} := \emptyset
  \and \\
  \freenames{x?(y).P} := \{ x \} \cup (\freenames{P} \setminus \{ y \})
  \and 
  \freenames{x!\langle P \rangle} := \{ x \} \cup \{ P \} 
  \and \\
  \freenames{P|Q} := \freenames{P} \cup \freenames{Q}
  \and \\
  \freenames{@{x}} := \{ x \}
\end{mathpar}

$\pi$
$\quotep{\pi}$

$\freenames{-} : \pi \to \mathcal{P}(\quotep{\pi})$

\begin{eqnarray*}
  \freenames{\pzero} & := & \emptyset \\
  \freenames{x?(y).P} & := & \{ x \} \cup (\freenames{P} \setminus \{ y \}) \\
  \freenames{x!\langle P \rangle} & := & \{ x \} \cup \{ P \} \\
  \freenames{P|Q} & := & \freenames{P} \cup \freenames{Q} \\
  \freenames{\dropn{x}} & := & \{ x \}
\end{eqnarray*}

The bound names of a process, $\boundnames{P}$, are those names occurring in $P$
that are not free. For example, in $x?(y).0$, the name $x$ is free, while $y$ is bound.

\begin{mathpar}
  \inferrule* [lab=monoidal-laws] {} { P|Q \equiv Q|P \and P|0 \equiv P \and P|(Q|R) \equiv (P|Q)|R }
\end{mathpar}

\begin{mathpar}
  \inferrule* [lab=alpha-equivalence] {} { (x)P \equiv (y)P\{y/x\} \and y \not\in \freenames{P} }
\end{mathpar}

\begin{definition}
Then two processes, $P,Q$, are alpha-equivalent if $P = Q\{\vec{y}/\vec{x}\}$ for
some $\vec{x} \in \boundnames{Q},\vec{y} \in \boundnames{P}$, where $Q\{\vec{y}/\vec{x}\}$
denotes the capture-avoiding substitution of $\vec{y}$ for $\vec{x}$ in $Q$.
\end{definition}

\begin{definition}
  The {\em structural congruence} \cite{SangiorgiWalker} , $\equiv$,
  between processes is the least congruence containing
  alpha-equivalence, satisfying the abelian monoid laws
  (associativity, commutativity and $\pzero$ as identity) for parallel
  composition $|$ and for summation $+$.
\end{definition}

\subsection{Name equivalence}

We take name equivalence, written $\nameeq$, to be the smallest
equivalence relation generated by the following rules.

\begin{mathpar}
\inferrule*[lab=Quote-drop]
{ }
{ \quotep{@{x}} \nameeq x }

\inferrule*[lab=Struct-equiv]
{ P \scong Q }
{ \quotep{P} \nameeq \quotep{Q} }
\end{mathpar}

The astute reader will have noticed that the mutual recursion of names
and processes imposes a mutual recursion on alpha-equivalence and
structural equivalence via name-equivalence. Fortunately, all of this
works out pleasantly and we may calculate in the natural way, free of
concern. The reader interested in the details is referred to the
appendix \ref{appendix:rho_details}.

\subsection{Substitution}

We use $\Proc$ for the set of processes, $\QProc$ for the set of
names, and $\id{\{}\vec{y} / \vec{x} \id{\}}$ to denote partial maps,
$s : \QProc \rightarrow \QProc$. A map, $s$ lifts, uniquely, to a map
on process terms, $\widehat{s} : \Proc \rightarrow \Proc$ by the
following equations.

\begin{mathpar}
  (0) \psubstp{Q}{P} := 0 \\
  (R \juxtap S) \psubstp{Q}{P}
  :=    
  (R)\psubstp{Q}{P} \juxtap (S) \psubstp{Q}{P} \\
  (x?(y).R) \psubstp{Q}{P}    
  :=    
  (x)\substp{Q}{P} (z)\concat( (R \psubstn{z}{y}) \psubstp{Q}{P} ) \\
  (\lift{x}{R}) \psubstp{Q}{P}  
  :=
  \lift{(x)\substp{Q}{P}}{ R \psubstp{Q}{P} } \\
%   (\dropn{x})  \psubstp{Q}{P}       
%   := 
%   \left\{ 
%     \begin{array}{ccc} 
%       \dropn{\quotep{Q}} & & x \nameeq \quotep{P} \\
%       \dropn{x} & & otherwise \\
%     \end{array}
%   \right. 
  (\dropn{x})  \psubstp{Q}{P}       
  := 
  \left\{ 
    \begin{array}{ccc} 
      Q & & x \nameeq \quotep{P} \\
      \dropn{x} & & otherwise \\
    \end{array}
  \right.
\end{mathpar}
 

where

\begin{eqnarray}
  (x)\id{\{} \lpquote Q \rpquote / \lpquote P \rpquote \id{\}}            = 
  \left\{ 
    \begin{array}{ccc}
      \lpquote Q \rpquote & & x \nameeq \lpquote P \rpquote \\
      x & & otherwise \\
    \end{array}
  \right. \nonumber
\end{eqnarray}

and $z$ is chosen distinct from $\quotep{P}$, $\quotep{Q}$, the free
names in $Q$, and all the names in $R$. Our $\alpha$-equivalence will
be built in the standard way from this substitution.

\begin{remark}\label{rem:no_self_referential_names}
  One consequence of these definitions is that $\forall P. \quotep{P}
  \not\in \freenames{P}$.
\end{remark}

\subsection{ Dynamic quote: an example }

Anticipating something of what's to come, consider applying the
substitution, $\widehat{\id{\{}u / z \id{\}}}$, to the following pair
of processes, $\lift{w}{y!(z)}$ and $w[ \lpquote y!(z) \rpquote ]$.

\begin{eqnarray}
	\lift{w}{y!(z)}\widehat{\id{\{}u / z \id{\}}}
		& = &
		\lift{w}{y!(u)} \nonumber\\
	w[ \lpquote y!(z) \rpquote ] \widehat{ \id{\{}u / z \id{\}} }
		& = &
		w[ \lpquote y!(z) \rpquote ] \nonumber
\end{eqnarray}

Because the body of the process between quotes is impervious to
substitution, we get radically different answers. In fact, by
examining the first process in an input context,
e.g. $x?(z).\lift{w}{y!(z)}$, we see that the process under the lift
operator may be shaped by prefixed inputs binding a name inside it. In
this sense, the lift operator will be seen as a way to dynamically
construct processes before reifying them as names.

Finally equipped with these standard features we can present the
dynamics of the calculus.

\subsubsection{Operational semantics} 

Finally, we introduce the computational dynamics. What marks these
algebras as distinct from other more traditionally studied algebraic
structures, e.g. vector spaces or polynomial rings, is the manner in
which dynamics is captured. In traditional structures, dynamics is typically
expressed through morphisms between such structures, as in linear maps
between vector spaces or morphisms between rings. In algebras
associated with the semantics of computation, the dynamics is
expressed as part of the algebraic structure itself, through a
reduction reduction relation typically denoted by $\red$. Below, we
give a recursive presentation of this relation for the calculus used
in the encoding.

$\red \subseteq \pi \times \pi$
$\red : \pi \to \mathcal{P}(\pi)$

\begin{mathpar}
  \inferrule* [lab=Comm] { \textsf{match}( x_{src}, x_{trgt} ) } { x_{trgt}?(y)P \; | \; x_{src}!\langle {Q} \rangle \red P\{\quotep{Q}/y}\} }
  \and \\
  \inferrule* [lab=Par] {{P} \red {P}'} {{{P} | {Q}} \red {{P}' | {Q}}}
  \and
  \inferrule* [lab=Equiv]{{{P} \scong {P}'} \andalso {{P}' \red {Q}'} \andalso {{Q}' \scong {Q}}}{{P} \red {Q}}
\end{mathpar}

\begin{eqnarray*}
  match_{\equiv} (\quotep{P},\quotep{Q}) & := & P \equiv Q \\
  match_{\dagger}(\quotep{P},\quotep{Q}) & := & \forall R. P|Q \red^{*} R => R \red^{*} 0 \\
  match_{K}(\quotep{P},\quotep{Q}) & := & K \mbox{ for some context } K
\end{eqnarray*}

$u?(x)P | u!\langle Q \rangle \red P\{\quotep{Q}/x\}$

%We write $\wred$ for $\red^*$, and $P\red$ if $\exists Q $ such that $ P \red Q$.
We write $P\red$ if $\exists Q $ such that $ P \red Q$ and $P\not\red$, otherwise.

\section{Replication}

As mentioned before, it is known that replication (and hence
recursion) can be implemented in a higher-order process algebra
\cite{SangiorgiWalker}. As our first example of calculation with the
machinery thus far presented we give the construction explicitly in
the {\rhoc}.

\begin{eqnarray}
	D_{x} & := & \prefix{x}{y}{(\binpar{\outputp{x}{y}}{@{y}})} \nonumber\\
	\bangp_{x}{P} & := & \binpar{{x}!\langle{\binpar{D_{x}}{P}}\rangle}{D_{x}} \nonumber
\end{eqnarray}

\begin{eqnarray}
	\bangp_{x}{P} & & \nonumber\\
	=
	& {x}!\langle{(\prefix{x}{y}{(\outputp{x}{y} | @{y})) | P}}\rangle 
	      | \prefix{x}{y}{(\outputp{x}{y} | @{y})} & \nonumber\\
	\red
	& (\outputp{x}{y} | @{y})\substn{\quotep{(\prefix{x}{y}{(@{y} | \outputp{x}{y})) | P}}}{y} & \nonumber\\
	=
	& \outputp{x}{\quotep{(\prefix{x}{y}{(\outputp{x}{y} | @{y})) | P}}}
	  | {(\prefix{x}{y}{(\outputp{x}{y} | @{y})) | P}} & \nonumber\\
	\red
	& \ldots & \nonumber\\
	\red^*
	& P | P | \ldots & \nonumber
\end{eqnarray}

Of course, this encoding, as an implementation, runs away, unfolding
$\bangp{P}$ eagerly. A lazier and more implementable replication
operator, restricted to input-guarded processes, may be obtained as follows.

\begin{eqnarray}
\bangp{\prefix{u}{v}{P}} 
	:= 
	\binpar{\lift{x}{\prefix{u}{v}{(\binpar{D(x)}{P})}}}{D(x)} \nonumber
\end{eqnarray}

\begin{remark}
  Note that the lazier definition still does not deal with summation
  or mixed summation (i.e. sums over input and output). The reader is
  invited to construct definitions of replication that deal with these
  features. 

  Further, the definitions are parameterized in a name, $x$. Can you,
  gentle reader, make a definition that eliminates this parameter and
  guarantees no accidental interaction between the replication
  machinery and the process being replicated -- i.e. no accidental
  sharing of names used by the process to get its work done and the
  name(s) used by the replication to effect copying. This latter
  revision of the definition of replication is crucial to obtaining
  the expected identity $!!P \sim !P$.
\end{remark}

\begin{remark}\label{rem:paradoxical_combinator}
  The reader familiar with the lambda calculus will have noticed the
  similarity between $D$ and the paradoxical combinator.

  [Ed. note: the existence of this seems to suggest we have to be more
  restrictive on the set of processes and names we admit if we are to
  support no-cloning.]
\end{remark}

\subsubsection{Bisimulation}

The computational dynamics gives rise to another kind of equivalence,
the equivalence of computational behavior. As previously mentioned
this is typically captured \emph{via} some form of bisimulation.

% The notion we use in this paper is weak barbed bisimulation
% \cite{milner91polyadicpi}.

The notion we use in this paper is derived from weak barbed
bisimulation \cite{milner91polyadicpi}. 

\begin{definition}
An \emph{observation relation}, $\downarrow_{\mathcal N}$, over a set
of names, $\mathcal N$, is the smallest relation satisfying the rules
below.

\infrule[Out-barb]{y \in {\mathcal N}, \; x \nameeq y}
		  {\outputp{x}{v} \downarrow_{\mathcal N} x}
\infrule[Par-barb]{\mbox{$P\downarrow_{\mathcal N} x$ or $Q\downarrow_{\mathcal N} x$}}
		  {\binpar{P}{Q} \downarrow_{\mathcal N} x}

We write $P \Downarrow_{\mathcal N} x$ if there is $Q$ such that 
$P \wred Q$ and $Q \downarrow_{\mathcal N} x$.
\end{definition}

\begin{definition}
%\label{def.bbisim}
An  ${\mathcal N}$-\emph{barbed bisimulation} over a set of names, ${\mathcal N}$, is a symmetric binary relation 
${\mathcal S}_{\mathcal N}$ between agents such that $P\rel{S}_{\mathcal N}Q$ implies:
\begin{enumerate}
\item If $P \red P'$ then $Q \wred Q'$ and $P'\rel{S}_{\mathcal N} Q'$.
\item If $P\downarrow_{\mathcal N} x$, then $Q\Downarrow_{\mathcal N} x$.
\end{enumerate}
$P$ is ${\mathcal N}$-barbed bisimilar to $Q$, written
$P \wbbisim_{\mathcal N} Q$, if $P \rel{S}_{\mathcal N} Q$ for some ${\mathcal N}$-barbed bisimulation ${\mathcal S}_{\mathcal N}$.
\end{definition}

$\mathcal{R} \subseteq \pi \times \pi$

$P \mathcal{R} Q => \forall P'. P \red P' \Rightarrow \exists Q'. Q \red Q', P' \mathcal{R} Q'$

$P \vdash x \Rightarrow Q \vdash x$

\begin{mathpar}
  \inferrule*[lab=Out-barb]{x \nameeq y}{{y}!\langle{Q}\rangle \vdash x}
  \and
  \inferrule*[lab=Par-barb]{\mbox{$P\vdash x$ or $Q\vdash x$}}{\binpar{P}{Q} \vdash x}
\end{mathpar}

\subsubsection{Contexts}

One of the principle advantages of computational calculi like the
$\pi$-calculus is a well-defined notion of context,
contextual-equivalence and a correlation between
contextual-equivalence and notions of bisimulation. The notion of
context allows the decomposition of a process into (sub-)process and
its syntactic environment, its context. Thus, a context may be
thought of as a process with a ``hole'' (written $\Box$) in it. The
application of a context $M$ to a process $P$, written $M[P]$, is
tantamount to filling the hole in $M$ with $P$. In this paper we do
not need the full weight of this theory, but do make use of the notion
of context in the proof the main theorem. 

\begin{mathpar}
  \inferrule* [lab=summation] {} {{M_{M},M_{N}} \bc \Box \;|\; x.M_{A} \;|\; M_{M}+M_{N}}
  \and
  \inferrule* [lab=agent] {} {{M_{A}} \bc (\vec{x})M_{P} \;| \; \clift{P_0,\ldots,M_{P},\ldots,P_N}}
  \and \\
  \inferrule* [lab=process] {} {{M_{P}} \bc M_{N} \;| \;P|M_{P} }
\end{mathpar} 

\begin{mathpar}
  \inferrule* [lab=sychronization] {} {M_{N} \bc \Box \;|\; x?M_{F} \;|\; x!M_{C}}
  \and
  \inferrule* [lab=abstraction] {} {{M_{F}} \bc (x)M_{P} }
  \and
  \inferrule* [lab=concretion] {} {{M_{C}} \bc \langle M_{P} \rangle }
  \and \\
  \inferrule* [lab=process] {} {{M_{P}} \bc M_{N} \;| \;P|M_{P} }
\end{mathpar}

\begin{definition}[contextual application] Given a context $M$, and
  process $P$, we define the \emph{contextual application}, $M[P] :=
  M\{P/\Box\}$. That is, the contextual application of M to P is the
  substitution of $P$ for $\Box$ in $M$.
\end{definition}

$\meaningof{-} : L \to \mathcal{P}(\pi)$

\begin{mathpar}
  \inferrule* [lab=collection] {} {\meaningof{true} = \pi, \and \meaningof{~E} = \pi \setminus \meaningof{E}, \and \meaningof{E_{1} \& E_{2}} = \meaningof{E_{1}} \cap \meaningof{E_{2}}}
\end{mathpar}

\begin{mathpar}
  \inferrule* [lab=structure] {} {\meaningof{0} = \{ P \in \pi | P \equiv 0 \}, \and \\ \meaningof{E_1 | E_2} = \{ P \in \pi | P \equiv P_{1} | P_{2}, P_{1} \in \meaningof{E_{1}}, P_{2} \in \meaningof{E_2}\} }
\end{mathpar}

\begin{mathpar}
 \inferrule* [lab=behavior] {} {\meaningof{\langle a?b \rangle E} = \{ P \in \pi | P \equiv Q | u?(y)P', \\ \and \\\\ \and \\ \;\;\; u \in \meaningof{a}, \forall z.P'\{z/y\} \in \meaningof{E\{z/b\}}\}, \and \\ \meaningof{a!E} = \{ P \in \pi | P \equiv Q | x!\langle P' \rangle, x \in \meaningof{a} P' \in \meaningof{E}\} }
\end{mathpar}

\begin{mathpar}
 \inferrule* [lab=nominal] {} {\meaningof{\quotep{E}} = \{ \quotep{P} \in \quotep{\pi} | P \in \meaningof{E} \}, \and \meaningof{\quotep{P}} = \{ \quotep{Q} \in \quotep{\pi} | P \equiv Q \} \and \\ \meaningof{@\quotep{E}} = \{ P \in \pi | P \equiv @x, x \in \meaningof{E} \}}
\end{mathpar}

\begin{eqnarray*}
  \\
  \meaningof{-} : TS \to ST
\end{eqnarray*}

\begin{eqnarray*}
  \\
  L : TS \to ST
\end{eqnarray*}

\begin{eqnarray*}
  \\
  P \models E \iff P \in \meaningof{E}
\end{eqnarray*}

\begin{eqnarray*}
  P \approx_{L} Q \iff \forall E \in L. P \models E \iff Q \models E
\end{eqnarray*}

\begin{eqnarray*}
  P \approx_{K} Q
\end{eqnarray*}

\begin{eqnarray*}
  P \approx Q
\end{eqnarray*}

$\approx_{K} = \approx = \approx_{L}$

\subsubsection{Contextual duality}

Note that contexts extend the quotation operation to a family of
operations from processes to names. Given a context, $M$, we can
define a \emph{nominal context}, $\quotep{M}$ by $\quotep{M}[P] :=
\quotep{M[P]}$. To foreshadow what is to come we observe that these
operations enjoy a duality with processes very much like the duality
between vectors and maps from vectors to scalars.

Further, because the calculus is essentially higher-order, we have a
correspondence between contexts and processes. More specifically,
given a name $x$ and a context $M$ we can construct $M^{*}_{x}$ such
that 

\begin{mathpar}
  M^{*}_{x} | \lift{x}{P} \red M[P]
\end{mathpar}

namely,

\begin{mathpar}
  M^{*}_{x} := x?(u).M[\dropn{u}]
\end{mathpar}

The dependence of $M^{*}_{x}$ on a name makes it an abstraction, 

\begin{mathpar}
  M^{*} := (x)x?(u).M[\dropn{u}]
\end{mathpar}

\subsection{Additional notation}

It will sometimes be convenient to denote the process a name
quotes. We already have the notation $x = \quotep{P}$, but it will be
convenient to introduce an alternate notation, $\procn{x}$, when we
want to emphasize the connection to the use of the name. Note that, by
virtue of name equivalence, $\quotep{\procn{x}} \nameeq x$; so, the
notation is consistent with previous definitions.

Further, because names have structure it is possible to effect
substitutions on the basis of that structure. This means we need to
upgrade our notation for substitutions, which we accomplish by
adapting comprehension notation. Thus,

\begin{mathpar}
  P\{ y / x : x \in S \}
\end{mathpar}

is interpreted to mean the process derived from P by replacing (in a
capture-avoiding manner) each occurrence of $x$ in $S$ by $y$. For example,

\begin{mathpar}
  P\{ \quotep{\procn{x}|\procn{x}} / x : x \in \freenames{P} \}
\end{mathpar}

will replace each (occurrence) of a free name $x$ in $P$ by
$\quotep{\procn{x}|\procn{x}}$.

Also, we will avail ourselves of the notation $x^{L}$ and $x^{R}$ to
denote injections of a name into disjoint copies of the name
space. There are numerous ways to accomplish this. One example can be
found in \cite{MeredithR05}. This notation overloads to vectors of
names: $\vec{x}^{\pi} := (x_{i}^{\pi} \; : \; 0 \leq i < |\vec{x}| )$ where $\pi \in \{L,R\}$.

We also use $P^{\Box} := P|\Box$.

In \cite{MeredithR05} an interpretation of the new operator is
given. It turns out that there are several possible interpretations
all enjoying the requisite algebraic properties of the operator (see
\cite{milner91polyadicpi}). We will therefore make liberal use of
$(\nu\; \vec{x})P$.

% subsection the_syntax_and_semantics_of_the_notation_system (end)   

\input{qm2pi.qmops} 

\input{qm2pi.sterngerlach} 

\input{qm2pi.metric} 

% section concurrent_process_calculi (end)

%\input{qm2pi.proofsketch}

% section proof sketch (end)

%\input{qm2pi.slviaknots} 

% section spatial logic via knots (end)

\input{qm2pi.conclusion}

% section conclusion (end)

%\input{qm2pi.dtcodes} 

% section wiring algorithm (end)

\input{qm2pi.ack} 

% section acknowledgments (end)

\newpage


\bibliographystyle{plain}   
\bibliography{../../biblios/main.bib}

\input{qm2pi.rhodetails}

\end{document}

 

\documentclass[12pt]{llncs}
%\documentclass{jktr}

\usepackage[pdftex]{hyperref}                   
\usepackage {listings}
\usepackage {mathpartir}
\usepackage{bcprules}
%\usepackage{listings}
                       
\usepackage{graphicx} 
%\usepackage[margins=2.5cm,nohead,nofoot]{geometry}
%\usepackage{geometry}
\usepackage{amsfonts}
\usepackage{amstext}
\usepackage{latexsym}
\usepackage{amssymb}
\usepackage{color}


%\include{myPreamble}
\include{qm2pi.local} 

%\ifpdf
%\usepackage[pdftex]{graphicx}
%\else
%\usepackage{graphicx}
%\fi

 % \ifpdf
%  \usepackage{pdfsync}
%  \if


%\title{Brief Article}
%\author{David F. Snyder}
%\author{L.G. Meredith}

%\address{Dept. of Math., Texas State University--San Marcos, San Marcos, TX 78666}
       
\pagestyle{empty}


\begin{document}

\lstset{language=[Objective]Caml,frame=shadowbox}

\input{qm2pi.front}

% section front matter (end)

\input{qm2pi.intro} 
 
% section introduction (end)

% \input{qm2pi.knotations} 

% section notation (end)

\input{qm2pi.process.calculi} 

% section concurrent_process_calculi_and_spatial_logics_ (end)
    
%\input{qm2pi.knots2pi} 

%\input{qm2pi.trefoil} 

%\input{qm2pi.mainthm} 

% subsection basic_interpretation (end)

%\input{qm2pi.rho.presentation} 
\subsection{The syntax and semantics of the notation system}\label{sub:the_syntax_and_semantics_of_the_notation_system} % (fold)

We now summarize a technical presentation of the calculus that
embodies our theory of dynamics. The typical presentation of such a
calculus follows the style of giving generators and relations on
them. The grammar, below, describing term constructors, freely
generates the set of processes, $\Proc$. This set is then quotiented
by a relation known as structural congruence and it is over this set
that the notion of dynamics is expressed. This presentation is
essentially that of \cite{MeredithR05} with the addition of
polyadicity and summation. For readability we have relegated some of
the technical subtleties to an appendix.

\subsubsection{Process grammar}\label{subsub:process_grammar}

\begin{mathpar}
  \inferrule* [lab=synchronization] {} {{M} \bc \pzero \;|\; x?F \;|\; x!C }
  \and
  \inferrule* [lab=abstraction] {} {{F} \bc (x)P}
  \and
  \inferrule* [lab=concretion] {} {{C} \bc \langle Q \rangle}
  \and
  \inferrule* [lab=process] {} {{P,Q} \bc M \;| \;P|Q \;|\; @{x}}
  \and
  \inferrule* [lab=name] {} {{x} \bc \quotep{P}}
\end{mathpar} 

Note that $\vec{x}$ (resp. $\vec{P}$) denotes a vector of names
(resp. processes) of length $|\vec{x}|$ (resp. $|\vec{P}|$). We adopt
the following useful abbreviations.

\begin{mathpar}
   x?(\vec{y}).P := x.(\vec{y})P \and  x\clift{\vec{P}} := x.\clift{\vec{P}}
   \and x!(y) := \lift{x}{\dropn{y}}
   \and \Pi_{i=0}^{n-1}P_i := P_0 | \ldots | P_{n-1}
\end{mathpar}

\subsubsection{Structural congruence}

\paragraph{Free and bound names and alpha-equivalence.} At the
core of structural equivalence is alpha-equivalence which identifies
process that are the same up to a change of variable. Formally, we
recognize the distinction between free and bound names. The free names
of a process, $\freenames{P}$, may be calculated recursively as
follows:

\begin{mathpar}
\freenames{\pzero} := \emptyset
  \and \\
  \freenames{x?(y).P} := \{ x \} \cup (\freenames{P} \setminus \{ y \})
  \and 
  \freenames{x!\langle P \rangle} := \{ x \} \cup \{ P \} 
  \and \\
  \freenames{P|Q} := \freenames{P} \cup \freenames{Q}
  \and \\
  \freenames{@{x}} := \{ x \}
\end{mathpar}

$\pi$
$\quotep{\pi}$

$\freenames{-} : \pi \to \mathcal{P}(\quotep{\pi})$

\begin{eqnarray*}
  \freenames{\pzero} & := & \emptyset \\
  \freenames{x?(y).P} & := & \{ x \} \cup (\freenames{P} \setminus \{ y \}) \\
  \freenames{x!\langle P \rangle} & := & \{ x \} \cup \{ P \} \\
  \freenames{P|Q} & := & \freenames{P} \cup \freenames{Q} \\
  \freenames{\dropn{x}} & := & \{ x \}
\end{eqnarray*}

The bound names of a process, $\boundnames{P}$, are those names occurring in $P$
that are not free. For example, in $x?(y).0$, the name $x$ is free, while $y$ is bound.

\begin{mathpar}
  \inferrule* [lab=monoidal-laws] {} { P|Q \equiv Q|P \and P|0 \equiv P \and P|(Q|R) \equiv (P|Q)|R }
\end{mathpar}

\begin{mathpar}
  \inferrule* [lab=alpha-equivalence] {} { (x)P \equiv (y)P\{y/x\} \and y \not\in \freenames{P} }
\end{mathpar}

\begin{definition}
Then two processes, $P,Q$, are alpha-equivalent if $P = Q\{\vec{y}/\vec{x}\}$ for
some $\vec{x} \in \boundnames{Q},\vec{y} \in \boundnames{P}$, where $Q\{\vec{y}/\vec{x}\}$
denotes the capture-avoiding substitution of $\vec{y}$ for $\vec{x}$ in $Q$.
\end{definition}

\begin{definition}
  The {\em structural congruence} \cite{SangiorgiWalker} , $\equiv$,
  between processes is the least congruence containing
  alpha-equivalence, satisfying the abelian monoid laws
  (associativity, commutativity and $\pzero$ as identity) for parallel
  composition $|$ and for summation $+$.
\end{definition}

\subsection{Name equivalence}

We take name equivalence, written $\nameeq$, to be the smallest
equivalence relation generated by the following rules.

\begin{mathpar}
\inferrule*[lab=Quote-drop]
{ }
{ \quotep{@{x}} \nameeq x }

\inferrule*[lab=Struct-equiv]
{ P \scong Q }
{ \quotep{P} \nameeq \quotep{Q} }
\end{mathpar}

The astute reader will have noticed that the mutual recursion of names
and processes imposes a mutual recursion on alpha-equivalence and
structural equivalence via name-equivalence. Fortunately, all of this
works out pleasantly and we may calculate in the natural way, free of
concern. The reader interested in the details is referred to the
appendix \ref{appendix:rho_details}.

\subsection{Substitution}

We use $\Proc$ for the set of processes, $\QProc$ for the set of
names, and $\id{\{}\vec{y} / \vec{x} \id{\}}$ to denote partial maps,
$s : \QProc \rightarrow \QProc$. A map, $s$ lifts, uniquely, to a map
on process terms, $\widehat{s} : \Proc \rightarrow \Proc$ by the
following equations.

\begin{mathpar}
  (0) \psubstp{Q}{P} := 0 \\
  (R \juxtap S) \psubstp{Q}{P}
  :=    
  (R)\psubstp{Q}{P} \juxtap (S) \psubstp{Q}{P} \\
  (x?(y).R) \psubstp{Q}{P}    
  :=    
  (x)\substp{Q}{P} (z)\concat( (R \psubstn{z}{y}) \psubstp{Q}{P} ) \\
  (\lift{x}{R}) \psubstp{Q}{P}  
  :=
  \lift{(x)\substp{Q}{P}}{ R \psubstp{Q}{P} } \\
%   (\dropn{x})  \psubstp{Q}{P}       
%   := 
%   \left\{ 
%     \begin{array}{ccc} 
%       \dropn{\quotep{Q}} & & x \nameeq \quotep{P} \\
%       \dropn{x} & & otherwise \\
%     \end{array}
%   \right. 
  (\dropn{x})  \psubstp{Q}{P}       
  := 
  \left\{ 
    \begin{array}{ccc} 
      Q & & x \nameeq \quotep{P} \\
      \dropn{x} & & otherwise \\
    \end{array}
  \right.
\end{mathpar}
 

where

\begin{eqnarray}
  (x)\id{\{} \lpquote Q \rpquote / \lpquote P \rpquote \id{\}}            = 
  \left\{ 
    \begin{array}{ccc}
      \lpquote Q \rpquote & & x \nameeq \lpquote P \rpquote \\
      x & & otherwise \\
    \end{array}
  \right. \nonumber
\end{eqnarray}

and $z$ is chosen distinct from $\quotep{P}$, $\quotep{Q}$, the free
names in $Q$, and all the names in $R$. Our $\alpha$-equivalence will
be built in the standard way from this substitution.

\begin{remark}\label{rem:no_self_referential_names}
  One consequence of these definitions is that $\forall P. \quotep{P}
  \not\in \freenames{P}$.
\end{remark}

\subsection{ Dynamic quote: an example }

Anticipating something of what's to come, consider applying the
substitution, $\widehat{\id{\{}u / z \id{\}}}$, to the following pair
of processes, $\lift{w}{y!(z)}$ and $w[ \lpquote y!(z) \rpquote ]$.

\begin{eqnarray}
	\lift{w}{y!(z)}\widehat{\id{\{}u / z \id{\}}}
		& = &
		\lift{w}{y!(u)} \nonumber\\
	w[ \lpquote y!(z) \rpquote ] \widehat{ \id{\{}u / z \id{\}} }
		& = &
		w[ \lpquote y!(z) \rpquote ] \nonumber
\end{eqnarray}

Because the body of the process between quotes is impervious to
substitution, we get radically different answers. In fact, by
examining the first process in an input context,
e.g. $x?(z).\lift{w}{y!(z)}$, we see that the process under the lift
operator may be shaped by prefixed inputs binding a name inside it. In
this sense, the lift operator will be seen as a way to dynamically
construct processes before reifying them as names.

Finally equipped with these standard features we can present the
dynamics of the calculus.

\subsubsection{Operational semantics} 

Finally, we introduce the computational dynamics. What marks these
algebras as distinct from other more traditionally studied algebraic
structures, e.g. vector spaces or polynomial rings, is the manner in
which dynamics is captured. In traditional structures, dynamics is typically
expressed through morphisms between such structures, as in linear maps
between vector spaces or morphisms between rings. In algebras
associated with the semantics of computation, the dynamics is
expressed as part of the algebraic structure itself, through a
reduction reduction relation typically denoted by $\red$. Below, we
give a recursive presentation of this relation for the calculus used
in the encoding.

$\red \subseteq \pi \times \pi$
$\red : \pi \to \mathcal{P}(\pi)$

\begin{mathpar}
  \inferrule* [lab=Comm] { \textsf{match}( x_{src}, x_{trgt} ) } { x_{trgt}?(y)P \; | \; x_{src}!\langle {Q} \rangle \red P\{\quotep{Q}/y}\} }
  \and \\
  \inferrule* [lab=Par] {{P} \red {P}'} {{{P} | {Q}} \red {{P}' | {Q}}}
  \and
  \inferrule* [lab=Equiv]{{{P} \scong {P}'} \andalso {{P}' \red {Q}'} \andalso {{Q}' \scong {Q}}}{{P} \red {Q}}
\end{mathpar}

\begin{eqnarray*}
  match_{\equiv} (\quotep{P},\quotep{Q}) & := & P \equiv Q \\
  match_{\dagger}(\quotep{P},\quotep{Q}) & := & \forall R. P|Q \red^{*} R => R \red^{*} 0 \\
  match_{K}(\quotep{P},\quotep{Q}) & := & K \mbox{ for some context } K
\end{eqnarray*}

$u?(x)P | u!\langle Q \rangle \red P\{\quotep{Q}/x\}$

%We write $\wred$ for $\red^*$, and $P\red$ if $\exists Q $ such that $ P \red Q$.
We write $P\red$ if $\exists Q $ such that $ P \red Q$ and $P\not\red$, otherwise.

\section{Replication}

As mentioned before, it is known that replication (and hence
recursion) can be implemented in a higher-order process algebra
\cite{SangiorgiWalker}. As our first example of calculation with the
machinery thus far presented we give the construction explicitly in
the {\rhoc}.

\begin{eqnarray}
	D_{x} & := & \prefix{x}{y}{(\binpar{\outputp{x}{y}}{@{y}})} \nonumber\\
	\bangp_{x}{P} & := & \binpar{{x}!\langle{\binpar{D_{x}}{P}}\rangle}{D_{x}} \nonumber
\end{eqnarray}

\begin{eqnarray}
	\bangp_{x}{P} & & \nonumber\\
	=
	& {x}!\langle{(\prefix{x}{y}{(\outputp{x}{y} | @{y})) | P}}\rangle 
	      | \prefix{x}{y}{(\outputp{x}{y} | @{y})} & \nonumber\\
	\red
	& (\outputp{x}{y} | @{y})\substn{\quotep{(\prefix{x}{y}{(@{y} | \outputp{x}{y})) | P}}}{y} & \nonumber\\
	=
	& \outputp{x}{\quotep{(\prefix{x}{y}{(\outputp{x}{y} | @{y})) | P}}}
	  | {(\prefix{x}{y}{(\outputp{x}{y} | @{y})) | P}} & \nonumber\\
	\red
	& \ldots & \nonumber\\
	\red^*
	& P | P | \ldots & \nonumber
\end{eqnarray}

Of course, this encoding, as an implementation, runs away, unfolding
$\bangp{P}$ eagerly. A lazier and more implementable replication
operator, restricted to input-guarded processes, may be obtained as follows.

\begin{eqnarray}
\bangp{\prefix{u}{v}{P}} 
	:= 
	\binpar{\lift{x}{\prefix{u}{v}{(\binpar{D(x)}{P})}}}{D(x)} \nonumber
\end{eqnarray}

\begin{remark}
  Note that the lazier definition still does not deal with summation
  or mixed summation (i.e. sums over input and output). The reader is
  invited to construct definitions of replication that deal with these
  features. 

  Further, the definitions are parameterized in a name, $x$. Can you,
  gentle reader, make a definition that eliminates this parameter and
  guarantees no accidental interaction between the replication
  machinery and the process being replicated -- i.e. no accidental
  sharing of names used by the process to get its work done and the
  name(s) used by the replication to effect copying. This latter
  revision of the definition of replication is crucial to obtaining
  the expected identity $!!P \sim !P$.
\end{remark}

\begin{remark}\label{rem:paradoxical_combinator}
  The reader familiar with the lambda calculus will have noticed the
  similarity between $D$ and the paradoxical combinator.

  [Ed. note: the existence of this seems to suggest we have to be more
  restrictive on the set of processes and names we admit if we are to
  support no-cloning.]
\end{remark}

\subsubsection{Bisimulation}

The computational dynamics gives rise to another kind of equivalence,
the equivalence of computational behavior. As previously mentioned
this is typically captured \emph{via} some form of bisimulation.

% The notion we use in this paper is weak barbed bisimulation
% \cite{milner91polyadicpi}.

The notion we use in this paper is derived from weak barbed
bisimulation \cite{milner91polyadicpi}. 

\begin{definition}
An \emph{observation relation}, $\downarrow_{\mathcal N}$, over a set
of names, $\mathcal N$, is the smallest relation satisfying the rules
below.

\infrule[Out-barb]{y \in {\mathcal N}, \; x \nameeq y}
		  {\outputp{x}{v} \downarrow_{\mathcal N} x}
\infrule[Par-barb]{\mbox{$P\downarrow_{\mathcal N} x$ or $Q\downarrow_{\mathcal N} x$}}
		  {\binpar{P}{Q} \downarrow_{\mathcal N} x}

We write $P \Downarrow_{\mathcal N} x$ if there is $Q$ such that 
$P \wred Q$ and $Q \downarrow_{\mathcal N} x$.
\end{definition}

\begin{definition}
%\label{def.bbisim}
An  ${\mathcal N}$-\emph{barbed bisimulation} over a set of names, ${\mathcal N}$, is a symmetric binary relation 
${\mathcal S}_{\mathcal N}$ between agents such that $P\rel{S}_{\mathcal N}Q$ implies:
\begin{enumerate}
\item If $P \red P'$ then $Q \wred Q'$ and $P'\rel{S}_{\mathcal N} Q'$.
\item If $P\downarrow_{\mathcal N} x$, then $Q\Downarrow_{\mathcal N} x$.
\end{enumerate}
$P$ is ${\mathcal N}$-barbed bisimilar to $Q$, written
$P \wbbisim_{\mathcal N} Q$, if $P \rel{S}_{\mathcal N} Q$ for some ${\mathcal N}$-barbed bisimulation ${\mathcal S}_{\mathcal N}$.
\end{definition}

$\mathcal{R} \subseteq \pi \times \pi$

$P \mathcal{R} Q => \forall P'. P \red P' \Rightarrow \exists Q'. Q \red Q', P' \mathcal{R} Q'$

$P \vdash x \Rightarrow Q \vdash x$

\begin{mathpar}
  \inferrule*[lab=Out-barb]{x \nameeq y}{{y}!\langle{Q}\rangle \vdash x}
  \and
  \inferrule*[lab=Par-barb]{\mbox{$P\vdash x$ or $Q\vdash x$}}{\binpar{P}{Q} \vdash x}
\end{mathpar}

\subsubsection{Contexts}

One of the principle advantages of computational calculi like the
$\pi$-calculus is a well-defined notion of context,
contextual-equivalence and a correlation between
contextual-equivalence and notions of bisimulation. The notion of
context allows the decomposition of a process into (sub-)process and
its syntactic environment, its context. Thus, a context may be
thought of as a process with a ``hole'' (written $\Box$) in it. The
application of a context $M$ to a process $P$, written $M[P]$, is
tantamount to filling the hole in $M$ with $P$. In this paper we do
not need the full weight of this theory, but do make use of the notion
of context in the proof the main theorem. 

\begin{mathpar}
  \inferrule* [lab=summation] {} {{M_{M},M_{N}} \bc \Box \;|\; x.M_{A} \;|\; M_{M}+M_{N}}
  \and
  \inferrule* [lab=agent] {} {{M_{A}} \bc (\vec{x})M_{P} \;| \; \clift{P_0,\ldots,M_{P},\ldots,P_N}}
  \and \\
  \inferrule* [lab=process] {} {{M_{P}} \bc M_{N} \;| \;P|M_{P} }
\end{mathpar} 

\begin{mathpar}
  \inferrule* [lab=sychronization] {} {M_{N} \bc \Box \;|\; x?M_{F} \;|\; x!M_{C}}
  \and
  \inferrule* [lab=abstraction] {} {{M_{F}} \bc (x)M_{P} }
  \and
  \inferrule* [lab=concretion] {} {{M_{C}} \bc \langle M_{P} \rangle }
  \and \\
  \inferrule* [lab=process] {} {{M_{P}} \bc M_{N} \;| \;P|M_{P} }
\end{mathpar}

\begin{definition}[contextual application] Given a context $M$, and
  process $P$, we define the \emph{contextual application}, $M[P] :=
  M\{P/\Box\}$. That is, the contextual application of M to P is the
  substitution of $P$ for $\Box$ in $M$.
\end{definition}

$\meaningof{-} : L \to \mathcal{P}(\pi)$

\begin{mathpar}
  \inferrule* [lab=collection] {} {\meaningof{true} = \pi, \and \meaningof{~E} = \pi \setminus \meaningof{E}, \and \meaningof{E_{1} \& E_{2}} = \meaningof{E_{1}} \cap \meaningof{E_{2}}}
\end{mathpar}

\begin{mathpar}
  \inferrule* [lab=structure] {} {\meaningof{0} = \{ P \in \pi | P \equiv 0 \}, \and \\ \meaningof{E_1 | E_2} = \{ P \in \pi | P \equiv P_{1} | P_{2}, P_{1} \in \meaningof{E_{1}}, P_{2} \in \meaningof{E_2}\} }
\end{mathpar}

\begin{mathpar}
 \inferrule* [lab=behavior] {} {\meaningof{\langle a?b \rangle E} = \{ P \in \pi | P \equiv Q | u?(y)P', \\ \and \\\\ \and \\ \;\;\; u \in \meaningof{a}, \forall z.P'\{z/y\} \in \meaningof{E\{z/b\}}\}, \and \\ \meaningof{a!E} = \{ P \in \pi | P \equiv Q | x!\langle P' \rangle, x \in \meaningof{a} P' \in \meaningof{E}\} }
\end{mathpar}

\begin{mathpar}
 \inferrule* [lab=nominal] {} {\meaningof{\quotep{E}} = \{ \quotep{P} \in \quotep{\pi} | P \in \meaningof{E} \}, \and \meaningof{\quotep{P}} = \{ \quotep{Q} \in \quotep{\pi} | P \equiv Q \} \and \\ \meaningof{@\quotep{E}} = \{ P \in \pi | P \equiv @x, x \in \meaningof{E} \}}
\end{mathpar}

\begin{eqnarray*}
  \\
  \meaningof{-} : TS \to ST
\end{eqnarray*}

\begin{eqnarray*}
  \\
  L : TS \to ST
\end{eqnarray*}

\begin{eqnarray*}
  \\
  P \models E \iff P \in \meaningof{E}
\end{eqnarray*}

\begin{eqnarray*}
  P \approx_{L} Q \iff \forall E \in L. P \models E \iff Q \models E
\end{eqnarray*}

\begin{eqnarray*}
  P \approx_{K} Q
\end{eqnarray*}

\begin{eqnarray*}
  P \approx Q
\end{eqnarray*}

$\approx_{K} = \approx = \approx_{L}$

\subsubsection{Contextual duality}

Note that contexts extend the quotation operation to a family of
operations from processes to names. Given a context, $M$, we can
define a \emph{nominal context}, $\quotep{M}$ by $\quotep{M}[P] :=
\quotep{M[P]}$. To foreshadow what is to come we observe that these
operations enjoy a duality with processes very much like the duality
between vectors and maps from vectors to scalars.

Further, because the calculus is essentially higher-order, we have a
correspondence between contexts and processes. More specifically,
given a name $x$ and a context $M$ we can construct $M^{*}_{x}$ such
that 

\begin{mathpar}
  M^{*}_{x} | \lift{x}{P} \red M[P]
\end{mathpar}

namely,

\begin{mathpar}
  M^{*}_{x} := x?(u).M[\dropn{u}]
\end{mathpar}

The dependence of $M^{*}_{x}$ on a name makes it an abstraction, 

\begin{mathpar}
  M^{*} := (x)x?(u).M[\dropn{u}]
\end{mathpar}

\subsection{Additional notation}

It will sometimes be convenient to denote the process a name
quotes. We already have the notation $x = \quotep{P}$, but it will be
convenient to introduce an alternate notation, $\procn{x}$, when we
want to emphasize the connection to the use of the name. Note that, by
virtue of name equivalence, $\quotep{\procn{x}} \nameeq x$; so, the
notation is consistent with previous definitions.

Further, because names have structure it is possible to effect
substitutions on the basis of that structure. This means we need to
upgrade our notation for substitutions, which we accomplish by
adapting comprehension notation. Thus,

\begin{mathpar}
  P\{ y / x : x \in S \}
\end{mathpar}

is interpreted to mean the process derived from P by replacing (in a
capture-avoiding manner) each occurrence of $x$ in $S$ by $y$. For example,

\begin{mathpar}
  P\{ \quotep{\procn{x}|\procn{x}} / x : x \in \freenames{P} \}
\end{mathpar}

will replace each (occurrence) of a free name $x$ in $P$ by
$\quotep{\procn{x}|\procn{x}}$.

Also, we will avail ourselves of the notation $x^{L}$ and $x^{R}$ to
denote injections of a name into disjoint copies of the name
space. There are numerous ways to accomplish this. One example can be
found in \cite{MeredithR05}. This notation overloads to vectors of
names: $\vec{x}^{\pi} := (x_{i}^{\pi} \; : \; 0 \leq i < |\vec{x}| )$ where $\pi \in \{L,R\}$.

We also use $P^{\Box} := P|\Box$.

In \cite{MeredithR05} an interpretation of the new operator is
given. It turns out that there are several possible interpretations
all enjoying the requisite algebraic properties of the operator (see
\cite{milner91polyadicpi}). We will therefore make liberal use of
$(\nu\; \vec{x})P$.

% subsection the_syntax_and_semantics_of_the_notation_system (end)   

\input{qm2pi.qmops} 

\input{qm2pi.sterngerlach} 

\input{qm2pi.metric} 

% section concurrent_process_calculi (end)

%\input{qm2pi.proofsketch}

% section proof sketch (end)

%\input{qm2pi.slviaknots} 

% section spatial logic via knots (end)

\input{qm2pi.conclusion}

% section conclusion (end)

%\input{qm2pi.dtcodes} 

% section wiring algorithm (end)

\input{qm2pi.ack} 

% section acknowledgments (end)

\newpage


\bibliographystyle{plain}   
\bibliography{../../biblios/main.bib}

\input{qm2pi.rhodetails}

\end{document}

 

% section concurrent_process_calculi (end)

%\documentclass[12pt]{llncs}
%\documentclass{jktr}

\usepackage[pdftex]{hyperref}                   
\usepackage {listings}
\usepackage {mathpartir}
\usepackage{bcprules}
%\usepackage{listings}
                       
\usepackage{graphicx} 
%\usepackage[margins=2.5cm,nohead,nofoot]{geometry}
%\usepackage{geometry}
\usepackage{amsfonts}
\usepackage{amstext}
\usepackage{latexsym}
\usepackage{amssymb}
\usepackage{color}


%\include{myPreamble}
\include{qm2pi.local} 

%\ifpdf
%\usepackage[pdftex]{graphicx}
%\else
%\usepackage{graphicx}
%\fi

 % \ifpdf
%  \usepackage{pdfsync}
%  \if


%\title{Brief Article}
%\author{David F. Snyder}
%\author{L.G. Meredith}

%\address{Dept. of Math., Texas State University--San Marcos, San Marcos, TX 78666}
       
\pagestyle{empty}


\begin{document}

\lstset{language=[Objective]Caml,frame=shadowbox}

\input{qm2pi.front}

% section front matter (end)

\input{qm2pi.intro} 
 
% section introduction (end)

% \input{qm2pi.knotations} 

% section notation (end)

\input{qm2pi.process.calculi} 

% section concurrent_process_calculi_and_spatial_logics_ (end)
    
%\input{qm2pi.knots2pi} 

%\input{qm2pi.trefoil} 

%\input{qm2pi.mainthm} 

% subsection basic_interpretation (end)

%\input{qm2pi.rho.presentation} 
\subsection{The syntax and semantics of the notation system}\label{sub:the_syntax_and_semantics_of_the_notation_system} % (fold)

We now summarize a technical presentation of the calculus that
embodies our theory of dynamics. The typical presentation of such a
calculus follows the style of giving generators and relations on
them. The grammar, below, describing term constructors, freely
generates the set of processes, $\Proc$. This set is then quotiented
by a relation known as structural congruence and it is over this set
that the notion of dynamics is expressed. This presentation is
essentially that of \cite{MeredithR05} with the addition of
polyadicity and summation. For readability we have relegated some of
the technical subtleties to an appendix.

\subsubsection{Process grammar}\label{subsub:process_grammar}

\begin{mathpar}
  \inferrule* [lab=synchronization] {} {{M} \bc \pzero \;|\; x?F \;|\; x!C }
  \and
  \inferrule* [lab=abstraction] {} {{F} \bc (x)P}
  \and
  \inferrule* [lab=concretion] {} {{C} \bc \langle Q \rangle}
  \and
  \inferrule* [lab=process] {} {{P,Q} \bc M \;| \;P|Q \;|\; @{x}}
  \and
  \inferrule* [lab=name] {} {{x} \bc \quotep{P}}
\end{mathpar} 

Note that $\vec{x}$ (resp. $\vec{P}$) denotes a vector of names
(resp. processes) of length $|\vec{x}|$ (resp. $|\vec{P}|$). We adopt
the following useful abbreviations.

\begin{mathpar}
   x?(\vec{y}).P := x.(\vec{y})P \and  x\clift{\vec{P}} := x.\clift{\vec{P}}
   \and x!(y) := \lift{x}{\dropn{y}}
   \and \Pi_{i=0}^{n-1}P_i := P_0 | \ldots | P_{n-1}
\end{mathpar}

\subsubsection{Structural congruence}

\paragraph{Free and bound names and alpha-equivalence.} At the
core of structural equivalence is alpha-equivalence which identifies
process that are the same up to a change of variable. Formally, we
recognize the distinction between free and bound names. The free names
of a process, $\freenames{P}$, may be calculated recursively as
follows:

\begin{mathpar}
\freenames{\pzero} := \emptyset
  \and \\
  \freenames{x?(y).P} := \{ x \} \cup (\freenames{P} \setminus \{ y \})
  \and 
  \freenames{x!\langle P \rangle} := \{ x \} \cup \{ P \} 
  \and \\
  \freenames{P|Q} := \freenames{P} \cup \freenames{Q}
  \and \\
  \freenames{@{x}} := \{ x \}
\end{mathpar}

$\pi$
$\quotep{\pi}$

$\freenames{-} : \pi \to \mathcal{P}(\quotep{\pi})$

\begin{eqnarray*}
  \freenames{\pzero} & := & \emptyset \\
  \freenames{x?(y).P} & := & \{ x \} \cup (\freenames{P} \setminus \{ y \}) \\
  \freenames{x!\langle P \rangle} & := & \{ x \} \cup \{ P \} \\
  \freenames{P|Q} & := & \freenames{P} \cup \freenames{Q} \\
  \freenames{\dropn{x}} & := & \{ x \}
\end{eqnarray*}

The bound names of a process, $\boundnames{P}$, are those names occurring in $P$
that are not free. For example, in $x?(y).0$, the name $x$ is free, while $y$ is bound.

\begin{mathpar}
  \inferrule* [lab=monoidal-laws] {} { P|Q \equiv Q|P \and P|0 \equiv P \and P|(Q|R) \equiv (P|Q)|R }
\end{mathpar}

\begin{mathpar}
  \inferrule* [lab=alpha-equivalence] {} { (x)P \equiv (y)P\{y/x\} \and y \not\in \freenames{P} }
\end{mathpar}

\begin{definition}
Then two processes, $P,Q$, are alpha-equivalent if $P = Q\{\vec{y}/\vec{x}\}$ for
some $\vec{x} \in \boundnames{Q},\vec{y} \in \boundnames{P}$, where $Q\{\vec{y}/\vec{x}\}$
denotes the capture-avoiding substitution of $\vec{y}$ for $\vec{x}$ in $Q$.
\end{definition}

\begin{definition}
  The {\em structural congruence} \cite{SangiorgiWalker} , $\equiv$,
  between processes is the least congruence containing
  alpha-equivalence, satisfying the abelian monoid laws
  (associativity, commutativity and $\pzero$ as identity) for parallel
  composition $|$ and for summation $+$.
\end{definition}

\subsection{Name equivalence}

We take name equivalence, written $\nameeq$, to be the smallest
equivalence relation generated by the following rules.

\begin{mathpar}
\inferrule*[lab=Quote-drop]
{ }
{ \quotep{@{x}} \nameeq x }

\inferrule*[lab=Struct-equiv]
{ P \scong Q }
{ \quotep{P} \nameeq \quotep{Q} }
\end{mathpar}

The astute reader will have noticed that the mutual recursion of names
and processes imposes a mutual recursion on alpha-equivalence and
structural equivalence via name-equivalence. Fortunately, all of this
works out pleasantly and we may calculate in the natural way, free of
concern. The reader interested in the details is referred to the
appendix \ref{appendix:rho_details}.

\subsection{Substitution}

We use $\Proc$ for the set of processes, $\QProc$ for the set of
names, and $\id{\{}\vec{y} / \vec{x} \id{\}}$ to denote partial maps,
$s : \QProc \rightarrow \QProc$. A map, $s$ lifts, uniquely, to a map
on process terms, $\widehat{s} : \Proc \rightarrow \Proc$ by the
following equations.

\begin{mathpar}
  (0) \psubstp{Q}{P} := 0 \\
  (R \juxtap S) \psubstp{Q}{P}
  :=    
  (R)\psubstp{Q}{P} \juxtap (S) \psubstp{Q}{P} \\
  (x?(y).R) \psubstp{Q}{P}    
  :=    
  (x)\substp{Q}{P} (z)\concat( (R \psubstn{z}{y}) \psubstp{Q}{P} ) \\
  (\lift{x}{R}) \psubstp{Q}{P}  
  :=
  \lift{(x)\substp{Q}{P}}{ R \psubstp{Q}{P} } \\
%   (\dropn{x})  \psubstp{Q}{P}       
%   := 
%   \left\{ 
%     \begin{array}{ccc} 
%       \dropn{\quotep{Q}} & & x \nameeq \quotep{P} \\
%       \dropn{x} & & otherwise \\
%     \end{array}
%   \right. 
  (\dropn{x})  \psubstp{Q}{P}       
  := 
  \left\{ 
    \begin{array}{ccc} 
      Q & & x \nameeq \quotep{P} \\
      \dropn{x} & & otherwise \\
    \end{array}
  \right.
\end{mathpar}
 

where

\begin{eqnarray}
  (x)\id{\{} \lpquote Q \rpquote / \lpquote P \rpquote \id{\}}            = 
  \left\{ 
    \begin{array}{ccc}
      \lpquote Q \rpquote & & x \nameeq \lpquote P \rpquote \\
      x & & otherwise \\
    \end{array}
  \right. \nonumber
\end{eqnarray}

and $z$ is chosen distinct from $\quotep{P}$, $\quotep{Q}$, the free
names in $Q$, and all the names in $R$. Our $\alpha$-equivalence will
be built in the standard way from this substitution.

\begin{remark}\label{rem:no_self_referential_names}
  One consequence of these definitions is that $\forall P. \quotep{P}
  \not\in \freenames{P}$.
\end{remark}

\subsection{ Dynamic quote: an example }

Anticipating something of what's to come, consider applying the
substitution, $\widehat{\id{\{}u / z \id{\}}}$, to the following pair
of processes, $\lift{w}{y!(z)}$ and $w[ \lpquote y!(z) \rpquote ]$.

\begin{eqnarray}
	\lift{w}{y!(z)}\widehat{\id{\{}u / z \id{\}}}
		& = &
		\lift{w}{y!(u)} \nonumber\\
	w[ \lpquote y!(z) \rpquote ] \widehat{ \id{\{}u / z \id{\}} }
		& = &
		w[ \lpquote y!(z) \rpquote ] \nonumber
\end{eqnarray}

Because the body of the process between quotes is impervious to
substitution, we get radically different answers. In fact, by
examining the first process in an input context,
e.g. $x?(z).\lift{w}{y!(z)}$, we see that the process under the lift
operator may be shaped by prefixed inputs binding a name inside it. In
this sense, the lift operator will be seen as a way to dynamically
construct processes before reifying them as names.

Finally equipped with these standard features we can present the
dynamics of the calculus.

\subsubsection{Operational semantics} 

Finally, we introduce the computational dynamics. What marks these
algebras as distinct from other more traditionally studied algebraic
structures, e.g. vector spaces or polynomial rings, is the manner in
which dynamics is captured. In traditional structures, dynamics is typically
expressed through morphisms between such structures, as in linear maps
between vector spaces or morphisms between rings. In algebras
associated with the semantics of computation, the dynamics is
expressed as part of the algebraic structure itself, through a
reduction reduction relation typically denoted by $\red$. Below, we
give a recursive presentation of this relation for the calculus used
in the encoding.

$\red \subseteq \pi \times \pi$
$\red : \pi \to \mathcal{P}(\pi)$

\begin{mathpar}
  \inferrule* [lab=Comm] { \textsf{match}( x_{src}, x_{trgt} ) } { x_{trgt}?(y)P \; | \; x_{src}!\langle {Q} \rangle \red P\{\quotep{Q}/y}\} }
  \and \\
  \inferrule* [lab=Par] {{P} \red {P}'} {{{P} | {Q}} \red {{P}' | {Q}}}
  \and
  \inferrule* [lab=Equiv]{{{P} \scong {P}'} \andalso {{P}' \red {Q}'} \andalso {{Q}' \scong {Q}}}{{P} \red {Q}}
\end{mathpar}

\begin{eqnarray*}
  match_{\equiv} (\quotep{P},\quotep{Q}) & := & P \equiv Q \\
  match_{\dagger}(\quotep{P},\quotep{Q}) & := & \forall R. P|Q \red^{*} R => R \red^{*} 0 \\
  match_{K}(\quotep{P},\quotep{Q}) & := & K \mbox{ for some context } K
\end{eqnarray*}

$u?(x)P | u!\langle Q \rangle \red P\{\quotep{Q}/x\}$

%We write $\wred$ for $\red^*$, and $P\red$ if $\exists Q $ such that $ P \red Q$.
We write $P\red$ if $\exists Q $ such that $ P \red Q$ and $P\not\red$, otherwise.

\section{Replication}

As mentioned before, it is known that replication (and hence
recursion) can be implemented in a higher-order process algebra
\cite{SangiorgiWalker}. As our first example of calculation with the
machinery thus far presented we give the construction explicitly in
the {\rhoc}.

\begin{eqnarray}
	D_{x} & := & \prefix{x}{y}{(\binpar{\outputp{x}{y}}{@{y}})} \nonumber\\
	\bangp_{x}{P} & := & \binpar{{x}!\langle{\binpar{D_{x}}{P}}\rangle}{D_{x}} \nonumber
\end{eqnarray}

\begin{eqnarray}
	\bangp_{x}{P} & & \nonumber\\
	=
	& {x}!\langle{(\prefix{x}{y}{(\outputp{x}{y} | @{y})) | P}}\rangle 
	      | \prefix{x}{y}{(\outputp{x}{y} | @{y})} & \nonumber\\
	\red
	& (\outputp{x}{y} | @{y})\substn{\quotep{(\prefix{x}{y}{(@{y} | \outputp{x}{y})) | P}}}{y} & \nonumber\\
	=
	& \outputp{x}{\quotep{(\prefix{x}{y}{(\outputp{x}{y} | @{y})) | P}}}
	  | {(\prefix{x}{y}{(\outputp{x}{y} | @{y})) | P}} & \nonumber\\
	\red
	& \ldots & \nonumber\\
	\red^*
	& P | P | \ldots & \nonumber
\end{eqnarray}

Of course, this encoding, as an implementation, runs away, unfolding
$\bangp{P}$ eagerly. A lazier and more implementable replication
operator, restricted to input-guarded processes, may be obtained as follows.

\begin{eqnarray}
\bangp{\prefix{u}{v}{P}} 
	:= 
	\binpar{\lift{x}{\prefix{u}{v}{(\binpar{D(x)}{P})}}}{D(x)} \nonumber
\end{eqnarray}

\begin{remark}
  Note that the lazier definition still does not deal with summation
  or mixed summation (i.e. sums over input and output). The reader is
  invited to construct definitions of replication that deal with these
  features. 

  Further, the definitions are parameterized in a name, $x$. Can you,
  gentle reader, make a definition that eliminates this parameter and
  guarantees no accidental interaction between the replication
  machinery and the process being replicated -- i.e. no accidental
  sharing of names used by the process to get its work done and the
  name(s) used by the replication to effect copying. This latter
  revision of the definition of replication is crucial to obtaining
  the expected identity $!!P \sim !P$.
\end{remark}

\begin{remark}\label{rem:paradoxical_combinator}
  The reader familiar with the lambda calculus will have noticed the
  similarity between $D$ and the paradoxical combinator.

  [Ed. note: the existence of this seems to suggest we have to be more
  restrictive on the set of processes and names we admit if we are to
  support no-cloning.]
\end{remark}

\subsubsection{Bisimulation}

The computational dynamics gives rise to another kind of equivalence,
the equivalence of computational behavior. As previously mentioned
this is typically captured \emph{via} some form of bisimulation.

% The notion we use in this paper is weak barbed bisimulation
% \cite{milner91polyadicpi}.

The notion we use in this paper is derived from weak barbed
bisimulation \cite{milner91polyadicpi}. 

\begin{definition}
An \emph{observation relation}, $\downarrow_{\mathcal N}$, over a set
of names, $\mathcal N$, is the smallest relation satisfying the rules
below.

\infrule[Out-barb]{y \in {\mathcal N}, \; x \nameeq y}
		  {\outputp{x}{v} \downarrow_{\mathcal N} x}
\infrule[Par-barb]{\mbox{$P\downarrow_{\mathcal N} x$ or $Q\downarrow_{\mathcal N} x$}}
		  {\binpar{P}{Q} \downarrow_{\mathcal N} x}

We write $P \Downarrow_{\mathcal N} x$ if there is $Q$ such that 
$P \wred Q$ and $Q \downarrow_{\mathcal N} x$.
\end{definition}

\begin{definition}
%\label{def.bbisim}
An  ${\mathcal N}$-\emph{barbed bisimulation} over a set of names, ${\mathcal N}$, is a symmetric binary relation 
${\mathcal S}_{\mathcal N}$ between agents such that $P\rel{S}_{\mathcal N}Q$ implies:
\begin{enumerate}
\item If $P \red P'$ then $Q \wred Q'$ and $P'\rel{S}_{\mathcal N} Q'$.
\item If $P\downarrow_{\mathcal N} x$, then $Q\Downarrow_{\mathcal N} x$.
\end{enumerate}
$P$ is ${\mathcal N}$-barbed bisimilar to $Q$, written
$P \wbbisim_{\mathcal N} Q$, if $P \rel{S}_{\mathcal N} Q$ for some ${\mathcal N}$-barbed bisimulation ${\mathcal S}_{\mathcal N}$.
\end{definition}

$\mathcal{R} \subseteq \pi \times \pi$

$P \mathcal{R} Q => \forall P'. P \red P' \Rightarrow \exists Q'. Q \red Q', P' \mathcal{R} Q'$

$P \vdash x \Rightarrow Q \vdash x$

\begin{mathpar}
  \inferrule*[lab=Out-barb]{x \nameeq y}{{y}!\langle{Q}\rangle \vdash x}
  \and
  \inferrule*[lab=Par-barb]{\mbox{$P\vdash x$ or $Q\vdash x$}}{\binpar{P}{Q} \vdash x}
\end{mathpar}

\subsubsection{Contexts}

One of the principle advantages of computational calculi like the
$\pi$-calculus is a well-defined notion of context,
contextual-equivalence and a correlation between
contextual-equivalence and notions of bisimulation. The notion of
context allows the decomposition of a process into (sub-)process and
its syntactic environment, its context. Thus, a context may be
thought of as a process with a ``hole'' (written $\Box$) in it. The
application of a context $M$ to a process $P$, written $M[P]$, is
tantamount to filling the hole in $M$ with $P$. In this paper we do
not need the full weight of this theory, but do make use of the notion
of context in the proof the main theorem. 

\begin{mathpar}
  \inferrule* [lab=summation] {} {{M_{M},M_{N}} \bc \Box \;|\; x.M_{A} \;|\; M_{M}+M_{N}}
  \and
  \inferrule* [lab=agent] {} {{M_{A}} \bc (\vec{x})M_{P} \;| \; \clift{P_0,\ldots,M_{P},\ldots,P_N}}
  \and \\
  \inferrule* [lab=process] {} {{M_{P}} \bc M_{N} \;| \;P|M_{P} }
\end{mathpar} 

\begin{mathpar}
  \inferrule* [lab=sychronization] {} {M_{N} \bc \Box \;|\; x?M_{F} \;|\; x!M_{C}}
  \and
  \inferrule* [lab=abstraction] {} {{M_{F}} \bc (x)M_{P} }
  \and
  \inferrule* [lab=concretion] {} {{M_{C}} \bc \langle M_{P} \rangle }
  \and \\
  \inferrule* [lab=process] {} {{M_{P}} \bc M_{N} \;| \;P|M_{P} }
\end{mathpar}

\begin{definition}[contextual application] Given a context $M$, and
  process $P$, we define the \emph{contextual application}, $M[P] :=
  M\{P/\Box\}$. That is, the contextual application of M to P is the
  substitution of $P$ for $\Box$ in $M$.
\end{definition}

$\meaningof{-} : L \to \mathcal{P}(\pi)$

\begin{mathpar}
  \inferrule* [lab=collection] {} {\meaningof{true} = \pi, \and \meaningof{~E} = \pi \setminus \meaningof{E}, \and \meaningof{E_{1} \& E_{2}} = \meaningof{E_{1}} \cap \meaningof{E_{2}}}
\end{mathpar}

\begin{mathpar}
  \inferrule* [lab=structure] {} {\meaningof{0} = \{ P \in \pi | P \equiv 0 \}, \and \\ \meaningof{E_1 | E_2} = \{ P \in \pi | P \equiv P_{1} | P_{2}, P_{1} \in \meaningof{E_{1}}, P_{2} \in \meaningof{E_2}\} }
\end{mathpar}

\begin{mathpar}
 \inferrule* [lab=behavior] {} {\meaningof{\langle a?b \rangle E} = \{ P \in \pi | P \equiv Q | u?(y)P', \\ \and \\\\ \and \\ \;\;\; u \in \meaningof{a}, \forall z.P'\{z/y\} \in \meaningof{E\{z/b\}}\}, \and \\ \meaningof{a!E} = \{ P \in \pi | P \equiv Q | x!\langle P' \rangle, x \in \meaningof{a} P' \in \meaningof{E}\} }
\end{mathpar}

\begin{mathpar}
 \inferrule* [lab=nominal] {} {\meaningof{\quotep{E}} = \{ \quotep{P} \in \quotep{\pi} | P \in \meaningof{E} \}, \and \meaningof{\quotep{P}} = \{ \quotep{Q} \in \quotep{\pi} | P \equiv Q \} \and \\ \meaningof{@\quotep{E}} = \{ P \in \pi | P \equiv @x, x \in \meaningof{E} \}}
\end{mathpar}

\begin{eqnarray*}
  \\
  \meaningof{-} : TS \to ST
\end{eqnarray*}

\begin{eqnarray*}
  \\
  L : TS \to ST
\end{eqnarray*}

\begin{eqnarray*}
  \\
  P \models E \iff P \in \meaningof{E}
\end{eqnarray*}

\begin{eqnarray*}
  P \approx_{L} Q \iff \forall E \in L. P \models E \iff Q \models E
\end{eqnarray*}

\begin{eqnarray*}
  P \approx_{K} Q
\end{eqnarray*}

\begin{eqnarray*}
  P \approx Q
\end{eqnarray*}

$\approx_{K} = \approx = \approx_{L}$

\subsubsection{Contextual duality}

Note that contexts extend the quotation operation to a family of
operations from processes to names. Given a context, $M$, we can
define a \emph{nominal context}, $\quotep{M}$ by $\quotep{M}[P] :=
\quotep{M[P]}$. To foreshadow what is to come we observe that these
operations enjoy a duality with processes very much like the duality
between vectors and maps from vectors to scalars.

Further, because the calculus is essentially higher-order, we have a
correspondence between contexts and processes. More specifically,
given a name $x$ and a context $M$ we can construct $M^{*}_{x}$ such
that 

\begin{mathpar}
  M^{*}_{x} | \lift{x}{P} \red M[P]
\end{mathpar}

namely,

\begin{mathpar}
  M^{*}_{x} := x?(u).M[\dropn{u}]
\end{mathpar}

The dependence of $M^{*}_{x}$ on a name makes it an abstraction, 

\begin{mathpar}
  M^{*} := (x)x?(u).M[\dropn{u}]
\end{mathpar}

\subsection{Additional notation}

It will sometimes be convenient to denote the process a name
quotes. We already have the notation $x = \quotep{P}$, but it will be
convenient to introduce an alternate notation, $\procn{x}$, when we
want to emphasize the connection to the use of the name. Note that, by
virtue of name equivalence, $\quotep{\procn{x}} \nameeq x$; so, the
notation is consistent with previous definitions.

Further, because names have structure it is possible to effect
substitutions on the basis of that structure. This means we need to
upgrade our notation for substitutions, which we accomplish by
adapting comprehension notation. Thus,

\begin{mathpar}
  P\{ y / x : x \in S \}
\end{mathpar}

is interpreted to mean the process derived from P by replacing (in a
capture-avoiding manner) each occurrence of $x$ in $S$ by $y$. For example,

\begin{mathpar}
  P\{ \quotep{\procn{x}|\procn{x}} / x : x \in \freenames{P} \}
\end{mathpar}

will replace each (occurrence) of a free name $x$ in $P$ by
$\quotep{\procn{x}|\procn{x}}$.

Also, we will avail ourselves of the notation $x^{L}$ and $x^{R}$ to
denote injections of a name into disjoint copies of the name
space. There are numerous ways to accomplish this. One example can be
found in \cite{MeredithR05}. This notation overloads to vectors of
names: $\vec{x}^{\pi} := (x_{i}^{\pi} \; : \; 0 \leq i < |\vec{x}| )$ where $\pi \in \{L,R\}$.

We also use $P^{\Box} := P|\Box$.

In \cite{MeredithR05} an interpretation of the new operator is
given. It turns out that there are several possible interpretations
all enjoying the requisite algebraic properties of the operator (see
\cite{milner91polyadicpi}). We will therefore make liberal use of
$(\nu\; \vec{x})P$.

% subsection the_syntax_and_semantics_of_the_notation_system (end)   

\input{qm2pi.qmops} 

\input{qm2pi.sterngerlach} 

\input{qm2pi.metric} 

% section concurrent_process_calculi (end)

%\input{qm2pi.proofsketch}

% section proof sketch (end)

%\input{qm2pi.slviaknots} 

% section spatial logic via knots (end)

\input{qm2pi.conclusion}

% section conclusion (end)

%\input{qm2pi.dtcodes} 

% section wiring algorithm (end)

\input{qm2pi.ack} 

% section acknowledgments (end)

\newpage


\bibliographystyle{plain}   
\bibliography{../../biblios/main.bib}

\input{qm2pi.rhodetails}

\end{document}



% section proof sketch (end)

%\section{Unlikely characters: spatial logic for
  knots}\label{sub:characteristic_formulae} % (fold)

Associated to the mobile process calculi are a family of logics known
as the Hennessy-Milner logics. These logics typically enjoy a
semantics interpreting formulae as sets of processes that when
factored through the encoding outlined above allows an identification
of classes of knots with logical formulae. In the context of this
encoding the sub-family known as the spatial logics \cite{CairesC03}
\cite{CairesC04} \cite{Caires04} are of particular interest providing
several important features for expressing and reasoning about
properties (i.e. classes) of knots. We hint here at how this may be done.

%\begin{description}
%\item [structural connectives] 
\subsubsection{Structural connectives} The spatial logics enjoy
structural connectives corresponding, at the logical level, to the
parallel composition ($P | Q$) and new name ($(\nu \; x)P$)
connectives for processes. As illustrated in the examples below, these
connectives are extremely expressive given the shape of our encoding.
%\item [decideable satisfaction]

\subsubsection{Decideable satisfaction}
In \cite{Caires04} the satisfaction relation is shown to be decideable
for a rich class of processes. It further turns out that the image of
the our encoding is a proper subset of that class. This result
provides the basis for an algorithm by which to search for knots
enjoying a given property.
%\item [characteristic formulae]

\subsubsection{Characteristic formulae}
In the same paper \cite{Caires04} , Caires presents a means of calculating
characteristic formulae, selecting equivalence classes of processes
up to a pre--specified depth limit on the support set of names. Composed with our
encoding, this characteristic formula can be used to select
characteristic formulae for knots.
%\end{description}

\subsubsection{Spatial logic formulae}

The grammar below (segmented for comprehension) summarizes the syntax
of spatial logic formulae. We employ illustrative examples in the
sequel to provide an intuitive understanding of their meaning
referring the reader to \cite{Caires04} for a more detailed explication
of the semantics.

\begin{mathpar}
  \inferrule* [lab=boolean] {} {{A,B} \bc T \;|\; \neg A \;|\; A \wedge B \;|\; \eta = \eta'}
  \and
  \inferrule* [lab=spatial] {} {|\; \pzero \;|\; A | B \;|\; x \text{\textregistered} A \;|\; \forall x . A \;|\;  H x . A}
  \and
  \inferrule* [lab=behavioral] {} {|\; \alpha . A}
  \and 
  \inferrule* [lab=recursion] {} {|\; X(\vec{u}) \;|\; \mu X(\vec{u}) . A}
  \and
  \inferrule* [lab=action] {} {\alpha \bc \langle x?(\vec{y}) \rangle \;|\; \langle x!(\vec{y}) \rangle \;|\; \langle \tau \rangle}
  \and 
  \inferrule* [lab=name] {} {\eta \bc x \;|\; \tau}
\end{mathpar} 

% subsection characteristic_formulae (end)   	 

\subsection{Example formulae}\label{sub:example_formulae_} % (fold)

\subsubsection{Crossing as formula.}
% 
% \begin{align*}
%   \frac{d}{dx} \sin x &= \cos x 
%   & \frac{d}{dx} e^x &= e^x \\
%   \frac{d}{dx} \cos x &= - \sin x 
%   & \frac{d}{dx} \log x &= \frac{1}{x} \\
% \end{align*} 

\begin{align*}
 \mu C(x_{0},x_{1},y_{0},y_{1},u).&(\langle x_{0}?(z) \rangle(\langle u! \rangle\langle y_{1}!z \rangle C(x_{0},x_{1},y_{0},y_{1},u)) & \\
  & \wedge \langle y_{1}?(z) \rangle (\langle u! \rangle \langle x_{0}!z \rangle C(x_{0},x_{1},y_{0},y_{1},u)) & \\
  & \wedge \langle x_{1}?(z) \rangle (\langle u? \rangle \langle y_{0}!z \rangle C(x_{0},x_{1},y_{0},y_{1},u)) & \\
  & \wedge \langle y_{0}?(z) \rangle (\langle u? \rangle \langle x_{1}!z \rangle C(x_{0},x_{1},y_{0},y_{1},u))) &
\end{align*}

The lexicographical similarity between the shape of this formulae and
the shape of definition of the process representing a crossing reveals
the intuitive meaning of this formulae. It describes the capabilities
of a process that has the right to represent a crossing. For example
it picks out processes that may perform an input on the port $x_0$ in
its initial menu of capabilities. What differentiates the formula
from the process, however, is that the crossing process is the
smallest candidate to satisfy the formula. Infinitely many other
processes -- with internal behavior hidden behind this interface, so
to speak -- also satisfy this formula. Even this simple formula,
then, can be seen to open a new view onto knots, providing a
computational interpretation of \emph{virtual} knots.

Note that this formula is derived by hand. A similar formula can be
derived by employing Caires' calculation of characteristic formula
\cite{Caires04} to the process representing a crossing. In light of
this discussion, we let
$\meaningof{C}_{\phi}(x0,x1,y0,y1,u)$ denote a formula specifying the
dynamics we wish to capture of a crossing. To guarantee we preserve
the shape of the interface and minimal semantics we demand that
$\meaningof{C}_{\phi}(x0,x1,y0,y1,u) \Rightarrow
\textbf{C}(x0,x1,y0,y1,u)$ where $\textbf{C}(x0,x1,y0,y1,u)$ denotes
the formula above.
                            
\subsubsection{Crossing number constraints.}
The moral content of the context lemma (Lemma \ref{context}) is that the notion of
``locality'' in the Reidemeister moves is effectively captured by the
parallel composition operator of the process calculus. This intuition
extends through the logic. Given a formula,
$\meaningof{C}_{\phi}(x0,x1,y0,y1,u)$, we can use the structural
connectives to specify constraints on crossing numbers, such as at
least $n$ crossings, or exactly $n$ crossings.
\begin{mathpar}
  \inferrule* [lab=at-least-n] {} { K^{\geq n}_{\phi}(\vec{xs},\vec{ys}) := \Pi_{i=0}^{n-1} Hu . \meaningof{C}_{\phi}(xs_i,ys_i,u) | T }
  \and 
  \inferrule* [lab=exactly-n] {} { K^{= n}_{\phi}(\vec{xs},\vec{ys}) := \Pi_{i=0}^{n-1} Hu . \meaningof{C}_{\phi}(xs_i,ys_i,u) | \neg (\forall x_0,y_0,x_1,y_1,u . \meaningof{C}_{\phi}(x_0,y_0,x_1,y_1,u) | T) }
\end{mathpar}

To round out this section, recall that the encoding of an $n$-crossing
knot decomposes into a parallel composition of $n$ \emph{copies} of a
crossing process together with a wiring harness. To specify different
knot classes with the same crossing number amounts to specifying
logical constraints on the wiring harness. In the interest of space,
we defer examples to a forthcoming paper. Suffice it to say that both
the conditions ``alternating knot'' and ``contains the tangle
corresponding to 5/3'' are expressible. For example, it is possible to
calculate the characteristic formula of a process corresponding to the
tangle 5/3 and conjoin it into the classifying formula via the
composition connective of the logic.

Finally, we wish to observe that it is entirely within reason to
contemplate a more domain-specific version of spatial logic tailored
to the shape of processes in the image of the encoding. Such a
domain-specific logic would have a better claim to the title formal
language of knot properties.

% subsection example_formulae_ (end)

% section knots_as_processes (end) 

% section spatial logic via knots (end)

\section{Conclusions and future work}

\paragraph{Testing physical space}
You, gentle reader, may wonder why of all the theorems to be proved
given this set up we pick the one above. In some sense it's hardly
central to quantum mechanics. We see it as central in the sense that
it firmly establishes a notion of physical space arising from a notion
of the equivalence of behavior. Relating bisimulation to a metric is a
big step forward, but one is faced with interpreting the relationship
of that metric space to something more physical. Quantum mechanical
notions of ``physical'' space are still far from intuitive, but by
relating this idea of distance as testing to calculations that predict
physical circumstances we are making a not insignificant step forward
toward an understanding of the physical space we inhabit as
essentially dynamic.

\paragraph{Effectivity and simulation}
One of the observations we have yet to make is that the entire program
spelled out here is effective. We have built various interpreters for
the reflective calculus at work in this interpretation. In principle,
then, we can simulate quantum mechanics on a computer. The place where
the simulation may lose fidelity is the infinitely branching summation
for the annihilator.

In this connection i also want to point out that the evaluation style
calculation of the inner product puts the non-determinism of the
summation right at the heart of measurement. This suggests that
Milner's original reduction-based formulation of the dynamics of his
calculi in terms of sums was not just notationally suggestive of a
notion of measure-and-continue but captured some significant part of
the physics.

\paragraph{Quantum continuations}
In light of this last observation i want to point out that the
predominant account of quantum mechanics is missing a key aspect of a
truly compositional story of the physical situation. In a real lab,
when a measurement is made the observation can be made to feed into
another device that then makes another measurement conditioned on the
results of the first. This means that after the superposition was
collapsed the entire experimental set up remained in
superposition. While QM offers a means of writing this down it doesn't
quite line up well with the well-trodden formulation of computation
and continuation that we see so succinctly expressed in Milner's
calculi. This suggests that there might be advantages to this account
of dynamics waiting to be explored.

\paragraph{Quantum logic}
In this connection, we also note that by virtue of having the
Hennessy-Milner construction, we can pull the construction through the
interpretation of QM. This gives us a natural candidate for a quantum
logic that enjoys an extremely tight connection with it's domain of
interpretation, making the construction much less ad hoc (rather it is
the image of functor!).

\paragraph{Quantum probabiity}
i have questions about the basis of the interpretation of inner
product as probability amplitude. In particular, using which
axiomatization of probability theory does the notion of probability
amplitude earn the right to be so dubbed? In other words, where is the
proof that the operation for calculating a probability amplitude (and
then squaring) satisfies the axioms of what it means to calculate a
probability? Even if such a proof exists (i have yet to find it in the
literature), i wonder if it might not be possible to turn things on
their heads. Can we view the calculation of the probability amplitude
as an axiomatization of probability? If so, then the definition we
give for calculating probability amplitude may provide the basis for
an \emph{effective} theory of probability.

\paragraph{Quantum vs ``biological'' information}
Finally, i want to conclude with a more philosophical observation. At
a recent workshop in which QM was a predominant topic i noticed
something about quantum information. The speaker was giving a riveting
discussion of axiomatic QM and showing how properties of ``no
cloning'' and ``no deleting'' emerged as consequences of the
axiomatization. Theorems of this form are necessary to give us a sense
of confidence that our axioms characterize the physical theory. What
struck me, though, was that if quantum information is neither erasable
nor replicable it is markedly different from \emph{life}. Two of the
things we know about life is that

\begin{itemize}
  \item it ends;
  \item to gain some measure of persistence, to transcend it's
    finitude it is imminently copyable.
\end{itemize}

Both of these qualities are summarized succinctly in the aphorism: all
flesh is grass. For me these two kinds of ``information'' -- call them
quantum and biological -- are end points on a spectrum of strategies
for persistence. At one end, we have those curious entities that enjoy
uniqueness and permanence; at the other, we have those who in the face
of a certain end and an uncertain present make a go of passing
something on. To me one of the more remarkable aspects of the latter
strategy is that in the presence of noise (and certain features of
copying) we get a kind of dynamism, a chance for improvement against a
given persistent condition.

% subsection other_calculi_other_bisimulations_and_geometry_as_behavior (end)




% section conclusion (end)

%\documentclass[12pt]{llncs}
%\documentclass{jktr}

\usepackage[pdftex]{hyperref}                   
\usepackage {listings}
\usepackage {mathpartir}
\usepackage{bcprules}
%\usepackage{listings}
                       
\usepackage{graphicx} 
%\usepackage[margins=2.5cm,nohead,nofoot]{geometry}
%\usepackage{geometry}
\usepackage{amsfonts}
\usepackage{amstext}
\usepackage{latexsym}
\usepackage{amssymb}
\usepackage{color}


%\include{myPreamble}
\include{qm2pi.local} 

%\ifpdf
%\usepackage[pdftex]{graphicx}
%\else
%\usepackage{graphicx}
%\fi

 % \ifpdf
%  \usepackage{pdfsync}
%  \if


%\title{Brief Article}
%\author{David F. Snyder}
%\author{L.G. Meredith}

%\address{Dept. of Math., Texas State University--San Marcos, San Marcos, TX 78666}
       
\pagestyle{empty}


\begin{document}

\lstset{language=[Objective]Caml,frame=shadowbox}

\input{qm2pi.front}

% section front matter (end)

\input{qm2pi.intro} 
 
% section introduction (end)

% \input{qm2pi.knotations} 

% section notation (end)

\input{qm2pi.process.calculi} 

% section concurrent_process_calculi_and_spatial_logics_ (end)
    
%\input{qm2pi.knots2pi} 

%\input{qm2pi.trefoil} 

%\input{qm2pi.mainthm} 

% subsection basic_interpretation (end)

%\input{qm2pi.rho.presentation} 
\subsection{The syntax and semantics of the notation system}\label{sub:the_syntax_and_semantics_of_the_notation_system} % (fold)

We now summarize a technical presentation of the calculus that
embodies our theory of dynamics. The typical presentation of such a
calculus follows the style of giving generators and relations on
them. The grammar, below, describing term constructors, freely
generates the set of processes, $\Proc$. This set is then quotiented
by a relation known as structural congruence and it is over this set
that the notion of dynamics is expressed. This presentation is
essentially that of \cite{MeredithR05} with the addition of
polyadicity and summation. For readability we have relegated some of
the technical subtleties to an appendix.

\subsubsection{Process grammar}\label{subsub:process_grammar}

\begin{mathpar}
  \inferrule* [lab=synchronization] {} {{M} \bc \pzero \;|\; x?F \;|\; x!C }
  \and
  \inferrule* [lab=abstraction] {} {{F} \bc (x)P}
  \and
  \inferrule* [lab=concretion] {} {{C} \bc \langle Q \rangle}
  \and
  \inferrule* [lab=process] {} {{P,Q} \bc M \;| \;P|Q \;|\; @{x}}
  \and
  \inferrule* [lab=name] {} {{x} \bc \quotep{P}}
\end{mathpar} 

Note that $\vec{x}$ (resp. $\vec{P}$) denotes a vector of names
(resp. processes) of length $|\vec{x}|$ (resp. $|\vec{P}|$). We adopt
the following useful abbreviations.

\begin{mathpar}
   x?(\vec{y}).P := x.(\vec{y})P \and  x\clift{\vec{P}} := x.\clift{\vec{P}}
   \and x!(y) := \lift{x}{\dropn{y}}
   \and \Pi_{i=0}^{n-1}P_i := P_0 | \ldots | P_{n-1}
\end{mathpar}

\subsubsection{Structural congruence}

\paragraph{Free and bound names and alpha-equivalence.} At the
core of structural equivalence is alpha-equivalence which identifies
process that are the same up to a change of variable. Formally, we
recognize the distinction between free and bound names. The free names
of a process, $\freenames{P}$, may be calculated recursively as
follows:

\begin{mathpar}
\freenames{\pzero} := \emptyset
  \and \\
  \freenames{x?(y).P} := \{ x \} \cup (\freenames{P} \setminus \{ y \})
  \and 
  \freenames{x!\langle P \rangle} := \{ x \} \cup \{ P \} 
  \and \\
  \freenames{P|Q} := \freenames{P} \cup \freenames{Q}
  \and \\
  \freenames{@{x}} := \{ x \}
\end{mathpar}

$\pi$
$\quotep{\pi}$

$\freenames{-} : \pi \to \mathcal{P}(\quotep{\pi})$

\begin{eqnarray*}
  \freenames{\pzero} & := & \emptyset \\
  \freenames{x?(y).P} & := & \{ x \} \cup (\freenames{P} \setminus \{ y \}) \\
  \freenames{x!\langle P \rangle} & := & \{ x \} \cup \{ P \} \\
  \freenames{P|Q} & := & \freenames{P} \cup \freenames{Q} \\
  \freenames{\dropn{x}} & := & \{ x \}
\end{eqnarray*}

The bound names of a process, $\boundnames{P}$, are those names occurring in $P$
that are not free. For example, in $x?(y).0$, the name $x$ is free, while $y$ is bound.

\begin{mathpar}
  \inferrule* [lab=monoidal-laws] {} { P|Q \equiv Q|P \and P|0 \equiv P \and P|(Q|R) \equiv (P|Q)|R }
\end{mathpar}

\begin{mathpar}
  \inferrule* [lab=alpha-equivalence] {} { (x)P \equiv (y)P\{y/x\} \and y \not\in \freenames{P} }
\end{mathpar}

\begin{definition}
Then two processes, $P,Q$, are alpha-equivalent if $P = Q\{\vec{y}/\vec{x}\}$ for
some $\vec{x} \in \boundnames{Q},\vec{y} \in \boundnames{P}$, where $Q\{\vec{y}/\vec{x}\}$
denotes the capture-avoiding substitution of $\vec{y}$ for $\vec{x}$ in $Q$.
\end{definition}

\begin{definition}
  The {\em structural congruence} \cite{SangiorgiWalker} , $\equiv$,
  between processes is the least congruence containing
  alpha-equivalence, satisfying the abelian monoid laws
  (associativity, commutativity and $\pzero$ as identity) for parallel
  composition $|$ and for summation $+$.
\end{definition}

\subsection{Name equivalence}

We take name equivalence, written $\nameeq$, to be the smallest
equivalence relation generated by the following rules.

\begin{mathpar}
\inferrule*[lab=Quote-drop]
{ }
{ \quotep{@{x}} \nameeq x }

\inferrule*[lab=Struct-equiv]
{ P \scong Q }
{ \quotep{P} \nameeq \quotep{Q} }
\end{mathpar}

The astute reader will have noticed that the mutual recursion of names
and processes imposes a mutual recursion on alpha-equivalence and
structural equivalence via name-equivalence. Fortunately, all of this
works out pleasantly and we may calculate in the natural way, free of
concern. The reader interested in the details is referred to the
appendix \ref{appendix:rho_details}.

\subsection{Substitution}

We use $\Proc$ for the set of processes, $\QProc$ for the set of
names, and $\id{\{}\vec{y} / \vec{x} \id{\}}$ to denote partial maps,
$s : \QProc \rightarrow \QProc$. A map, $s$ lifts, uniquely, to a map
on process terms, $\widehat{s} : \Proc \rightarrow \Proc$ by the
following equations.

\begin{mathpar}
  (0) \psubstp{Q}{P} := 0 \\
  (R \juxtap S) \psubstp{Q}{P}
  :=    
  (R)\psubstp{Q}{P} \juxtap (S) \psubstp{Q}{P} \\
  (x?(y).R) \psubstp{Q}{P}    
  :=    
  (x)\substp{Q}{P} (z)\concat( (R \psubstn{z}{y}) \psubstp{Q}{P} ) \\
  (\lift{x}{R}) \psubstp{Q}{P}  
  :=
  \lift{(x)\substp{Q}{P}}{ R \psubstp{Q}{P} } \\
%   (\dropn{x})  \psubstp{Q}{P}       
%   := 
%   \left\{ 
%     \begin{array}{ccc} 
%       \dropn{\quotep{Q}} & & x \nameeq \quotep{P} \\
%       \dropn{x} & & otherwise \\
%     \end{array}
%   \right. 
  (\dropn{x})  \psubstp{Q}{P}       
  := 
  \left\{ 
    \begin{array}{ccc} 
      Q & & x \nameeq \quotep{P} \\
      \dropn{x} & & otherwise \\
    \end{array}
  \right.
\end{mathpar}
 

where

\begin{eqnarray}
  (x)\id{\{} \lpquote Q \rpquote / \lpquote P \rpquote \id{\}}            = 
  \left\{ 
    \begin{array}{ccc}
      \lpquote Q \rpquote & & x \nameeq \lpquote P \rpquote \\
      x & & otherwise \\
    \end{array}
  \right. \nonumber
\end{eqnarray}

and $z$ is chosen distinct from $\quotep{P}$, $\quotep{Q}$, the free
names in $Q$, and all the names in $R$. Our $\alpha$-equivalence will
be built in the standard way from this substitution.

\begin{remark}\label{rem:no_self_referential_names}
  One consequence of these definitions is that $\forall P. \quotep{P}
  \not\in \freenames{P}$.
\end{remark}

\subsection{ Dynamic quote: an example }

Anticipating something of what's to come, consider applying the
substitution, $\widehat{\id{\{}u / z \id{\}}}$, to the following pair
of processes, $\lift{w}{y!(z)}$ and $w[ \lpquote y!(z) \rpquote ]$.

\begin{eqnarray}
	\lift{w}{y!(z)}\widehat{\id{\{}u / z \id{\}}}
		& = &
		\lift{w}{y!(u)} \nonumber\\
	w[ \lpquote y!(z) \rpquote ] \widehat{ \id{\{}u / z \id{\}} }
		& = &
		w[ \lpquote y!(z) \rpquote ] \nonumber
\end{eqnarray}

Because the body of the process between quotes is impervious to
substitution, we get radically different answers. In fact, by
examining the first process in an input context,
e.g. $x?(z).\lift{w}{y!(z)}$, we see that the process under the lift
operator may be shaped by prefixed inputs binding a name inside it. In
this sense, the lift operator will be seen as a way to dynamically
construct processes before reifying them as names.

Finally equipped with these standard features we can present the
dynamics of the calculus.

\subsubsection{Operational semantics} 

Finally, we introduce the computational dynamics. What marks these
algebras as distinct from other more traditionally studied algebraic
structures, e.g. vector spaces or polynomial rings, is the manner in
which dynamics is captured. In traditional structures, dynamics is typically
expressed through morphisms between such structures, as in linear maps
between vector spaces or morphisms between rings. In algebras
associated with the semantics of computation, the dynamics is
expressed as part of the algebraic structure itself, through a
reduction reduction relation typically denoted by $\red$. Below, we
give a recursive presentation of this relation for the calculus used
in the encoding.

$\red \subseteq \pi \times \pi$
$\red : \pi \to \mathcal{P}(\pi)$

\begin{mathpar}
  \inferrule* [lab=Comm] { \textsf{match}( x_{src}, x_{trgt} ) } { x_{trgt}?(y)P \; | \; x_{src}!\langle {Q} \rangle \red P\{\quotep{Q}/y}\} }
  \and \\
  \inferrule* [lab=Par] {{P} \red {P}'} {{{P} | {Q}} \red {{P}' | {Q}}}
  \and
  \inferrule* [lab=Equiv]{{{P} \scong {P}'} \andalso {{P}' \red {Q}'} \andalso {{Q}' \scong {Q}}}{{P} \red {Q}}
\end{mathpar}

\begin{eqnarray*}
  match_{\equiv} (\quotep{P},\quotep{Q}) & := & P \equiv Q \\
  match_{\dagger}(\quotep{P},\quotep{Q}) & := & \forall R. P|Q \red^{*} R => R \red^{*} 0 \\
  match_{K}(\quotep{P},\quotep{Q}) & := & K \mbox{ for some context } K
\end{eqnarray*}

$u?(x)P | u!\langle Q \rangle \red P\{\quotep{Q}/x\}$

%We write $\wred$ for $\red^*$, and $P\red$ if $\exists Q $ such that $ P \red Q$.
We write $P\red$ if $\exists Q $ such that $ P \red Q$ and $P\not\red$, otherwise.

\section{Replication}

As mentioned before, it is known that replication (and hence
recursion) can be implemented in a higher-order process algebra
\cite{SangiorgiWalker}. As our first example of calculation with the
machinery thus far presented we give the construction explicitly in
the {\rhoc}.

\begin{eqnarray}
	D_{x} & := & \prefix{x}{y}{(\binpar{\outputp{x}{y}}{@{y}})} \nonumber\\
	\bangp_{x}{P} & := & \binpar{{x}!\langle{\binpar{D_{x}}{P}}\rangle}{D_{x}} \nonumber
\end{eqnarray}

\begin{eqnarray}
	\bangp_{x}{P} & & \nonumber\\
	=
	& {x}!\langle{(\prefix{x}{y}{(\outputp{x}{y} | @{y})) | P}}\rangle 
	      | \prefix{x}{y}{(\outputp{x}{y} | @{y})} & \nonumber\\
	\red
	& (\outputp{x}{y} | @{y})\substn{\quotep{(\prefix{x}{y}{(@{y} | \outputp{x}{y})) | P}}}{y} & \nonumber\\
	=
	& \outputp{x}{\quotep{(\prefix{x}{y}{(\outputp{x}{y} | @{y})) | P}}}
	  | {(\prefix{x}{y}{(\outputp{x}{y} | @{y})) | P}} & \nonumber\\
	\red
	& \ldots & \nonumber\\
	\red^*
	& P | P | \ldots & \nonumber
\end{eqnarray}

Of course, this encoding, as an implementation, runs away, unfolding
$\bangp{P}$ eagerly. A lazier and more implementable replication
operator, restricted to input-guarded processes, may be obtained as follows.

\begin{eqnarray}
\bangp{\prefix{u}{v}{P}} 
	:= 
	\binpar{\lift{x}{\prefix{u}{v}{(\binpar{D(x)}{P})}}}{D(x)} \nonumber
\end{eqnarray}

\begin{remark}
  Note that the lazier definition still does not deal with summation
  or mixed summation (i.e. sums over input and output). The reader is
  invited to construct definitions of replication that deal with these
  features. 

  Further, the definitions are parameterized in a name, $x$. Can you,
  gentle reader, make a definition that eliminates this parameter and
  guarantees no accidental interaction between the replication
  machinery and the process being replicated -- i.e. no accidental
  sharing of names used by the process to get its work done and the
  name(s) used by the replication to effect copying. This latter
  revision of the definition of replication is crucial to obtaining
  the expected identity $!!P \sim !P$.
\end{remark}

\begin{remark}\label{rem:paradoxical_combinator}
  The reader familiar with the lambda calculus will have noticed the
  similarity between $D$ and the paradoxical combinator.

  [Ed. note: the existence of this seems to suggest we have to be more
  restrictive on the set of processes and names we admit if we are to
  support no-cloning.]
\end{remark}

\subsubsection{Bisimulation}

The computational dynamics gives rise to another kind of equivalence,
the equivalence of computational behavior. As previously mentioned
this is typically captured \emph{via} some form of bisimulation.

% The notion we use in this paper is weak barbed bisimulation
% \cite{milner91polyadicpi}.

The notion we use in this paper is derived from weak barbed
bisimulation \cite{milner91polyadicpi}. 

\begin{definition}
An \emph{observation relation}, $\downarrow_{\mathcal N}$, over a set
of names, $\mathcal N$, is the smallest relation satisfying the rules
below.

\infrule[Out-barb]{y \in {\mathcal N}, \; x \nameeq y}
		  {\outputp{x}{v} \downarrow_{\mathcal N} x}
\infrule[Par-barb]{\mbox{$P\downarrow_{\mathcal N} x$ or $Q\downarrow_{\mathcal N} x$}}
		  {\binpar{P}{Q} \downarrow_{\mathcal N} x}

We write $P \Downarrow_{\mathcal N} x$ if there is $Q$ such that 
$P \wred Q$ and $Q \downarrow_{\mathcal N} x$.
\end{definition}

\begin{definition}
%\label{def.bbisim}
An  ${\mathcal N}$-\emph{barbed bisimulation} over a set of names, ${\mathcal N}$, is a symmetric binary relation 
${\mathcal S}_{\mathcal N}$ between agents such that $P\rel{S}_{\mathcal N}Q$ implies:
\begin{enumerate}
\item If $P \red P'$ then $Q \wred Q'$ and $P'\rel{S}_{\mathcal N} Q'$.
\item If $P\downarrow_{\mathcal N} x$, then $Q\Downarrow_{\mathcal N} x$.
\end{enumerate}
$P$ is ${\mathcal N}$-barbed bisimilar to $Q$, written
$P \wbbisim_{\mathcal N} Q$, if $P \rel{S}_{\mathcal N} Q$ for some ${\mathcal N}$-barbed bisimulation ${\mathcal S}_{\mathcal N}$.
\end{definition}

$\mathcal{R} \subseteq \pi \times \pi$

$P \mathcal{R} Q => \forall P'. P \red P' \Rightarrow \exists Q'. Q \red Q', P' \mathcal{R} Q'$

$P \vdash x \Rightarrow Q \vdash x$

\begin{mathpar}
  \inferrule*[lab=Out-barb]{x \nameeq y}{{y}!\langle{Q}\rangle \vdash x}
  \and
  \inferrule*[lab=Par-barb]{\mbox{$P\vdash x$ or $Q\vdash x$}}{\binpar{P}{Q} \vdash x}
\end{mathpar}

\subsubsection{Contexts}

One of the principle advantages of computational calculi like the
$\pi$-calculus is a well-defined notion of context,
contextual-equivalence and a correlation between
contextual-equivalence and notions of bisimulation. The notion of
context allows the decomposition of a process into (sub-)process and
its syntactic environment, its context. Thus, a context may be
thought of as a process with a ``hole'' (written $\Box$) in it. The
application of a context $M$ to a process $P$, written $M[P]$, is
tantamount to filling the hole in $M$ with $P$. In this paper we do
not need the full weight of this theory, but do make use of the notion
of context in the proof the main theorem. 

\begin{mathpar}
  \inferrule* [lab=summation] {} {{M_{M},M_{N}} \bc \Box \;|\; x.M_{A} \;|\; M_{M}+M_{N}}
  \and
  \inferrule* [lab=agent] {} {{M_{A}} \bc (\vec{x})M_{P} \;| \; \clift{P_0,\ldots,M_{P},\ldots,P_N}}
  \and \\
  \inferrule* [lab=process] {} {{M_{P}} \bc M_{N} \;| \;P|M_{P} }
\end{mathpar} 

\begin{mathpar}
  \inferrule* [lab=sychronization] {} {M_{N} \bc \Box \;|\; x?M_{F} \;|\; x!M_{C}}
  \and
  \inferrule* [lab=abstraction] {} {{M_{F}} \bc (x)M_{P} }
  \and
  \inferrule* [lab=concretion] {} {{M_{C}} \bc \langle M_{P} \rangle }
  \and \\
  \inferrule* [lab=process] {} {{M_{P}} \bc M_{N} \;| \;P|M_{P} }
\end{mathpar}

\begin{definition}[contextual application] Given a context $M$, and
  process $P$, we define the \emph{contextual application}, $M[P] :=
  M\{P/\Box\}$. That is, the contextual application of M to P is the
  substitution of $P$ for $\Box$ in $M$.
\end{definition}

$\meaningof{-} : L \to \mathcal{P}(\pi)$

\begin{mathpar}
  \inferrule* [lab=collection] {} {\meaningof{true} = \pi, \and \meaningof{~E} = \pi \setminus \meaningof{E}, \and \meaningof{E_{1} \& E_{2}} = \meaningof{E_{1}} \cap \meaningof{E_{2}}}
\end{mathpar}

\begin{mathpar}
  \inferrule* [lab=structure] {} {\meaningof{0} = \{ P \in \pi | P \equiv 0 \}, \and \\ \meaningof{E_1 | E_2} = \{ P \in \pi | P \equiv P_{1} | P_{2}, P_{1} \in \meaningof{E_{1}}, P_{2} \in \meaningof{E_2}\} }
\end{mathpar}

\begin{mathpar}
 \inferrule* [lab=behavior] {} {\meaningof{\langle a?b \rangle E} = \{ P \in \pi | P \equiv Q | u?(y)P', \\ \and \\\\ \and \\ \;\;\; u \in \meaningof{a}, \forall z.P'\{z/y\} \in \meaningof{E\{z/b\}}\}, \and \\ \meaningof{a!E} = \{ P \in \pi | P \equiv Q | x!\langle P' \rangle, x \in \meaningof{a} P' \in \meaningof{E}\} }
\end{mathpar}

\begin{mathpar}
 \inferrule* [lab=nominal] {} {\meaningof{\quotep{E}} = \{ \quotep{P} \in \quotep{\pi} | P \in \meaningof{E} \}, \and \meaningof{\quotep{P}} = \{ \quotep{Q} \in \quotep{\pi} | P \equiv Q \} \and \\ \meaningof{@\quotep{E}} = \{ P \in \pi | P \equiv @x, x \in \meaningof{E} \}}
\end{mathpar}

\begin{eqnarray*}
  \\
  \meaningof{-} : TS \to ST
\end{eqnarray*}

\begin{eqnarray*}
  \\
  L : TS \to ST
\end{eqnarray*}

\begin{eqnarray*}
  \\
  P \models E \iff P \in \meaningof{E}
\end{eqnarray*}

\begin{eqnarray*}
  P \approx_{L} Q \iff \forall E \in L. P \models E \iff Q \models E
\end{eqnarray*}

\begin{eqnarray*}
  P \approx_{K} Q
\end{eqnarray*}

\begin{eqnarray*}
  P \approx Q
\end{eqnarray*}

$\approx_{K} = \approx = \approx_{L}$

\subsubsection{Contextual duality}

Note that contexts extend the quotation operation to a family of
operations from processes to names. Given a context, $M$, we can
define a \emph{nominal context}, $\quotep{M}$ by $\quotep{M}[P] :=
\quotep{M[P]}$. To foreshadow what is to come we observe that these
operations enjoy a duality with processes very much like the duality
between vectors and maps from vectors to scalars.

Further, because the calculus is essentially higher-order, we have a
correspondence between contexts and processes. More specifically,
given a name $x$ and a context $M$ we can construct $M^{*}_{x}$ such
that 

\begin{mathpar}
  M^{*}_{x} | \lift{x}{P} \red M[P]
\end{mathpar}

namely,

\begin{mathpar}
  M^{*}_{x} := x?(u).M[\dropn{u}]
\end{mathpar}

The dependence of $M^{*}_{x}$ on a name makes it an abstraction, 

\begin{mathpar}
  M^{*} := (x)x?(u).M[\dropn{u}]
\end{mathpar}

\subsection{Additional notation}

It will sometimes be convenient to denote the process a name
quotes. We already have the notation $x = \quotep{P}$, but it will be
convenient to introduce an alternate notation, $\procn{x}$, when we
want to emphasize the connection to the use of the name. Note that, by
virtue of name equivalence, $\quotep{\procn{x}} \nameeq x$; so, the
notation is consistent with previous definitions.

Further, because names have structure it is possible to effect
substitutions on the basis of that structure. This means we need to
upgrade our notation for substitutions, which we accomplish by
adapting comprehension notation. Thus,

\begin{mathpar}
  P\{ y / x : x \in S \}
\end{mathpar}

is interpreted to mean the process derived from P by replacing (in a
capture-avoiding manner) each occurrence of $x$ in $S$ by $y$. For example,

\begin{mathpar}
  P\{ \quotep{\procn{x}|\procn{x}} / x : x \in \freenames{P} \}
\end{mathpar}

will replace each (occurrence) of a free name $x$ in $P$ by
$\quotep{\procn{x}|\procn{x}}$.

Also, we will avail ourselves of the notation $x^{L}$ and $x^{R}$ to
denote injections of a name into disjoint copies of the name
space. There are numerous ways to accomplish this. One example can be
found in \cite{MeredithR05}. This notation overloads to vectors of
names: $\vec{x}^{\pi} := (x_{i}^{\pi} \; : \; 0 \leq i < |\vec{x}| )$ where $\pi \in \{L,R\}$.

We also use $P^{\Box} := P|\Box$.

In \cite{MeredithR05} an interpretation of the new operator is
given. It turns out that there are several possible interpretations
all enjoying the requisite algebraic properties of the operator (see
\cite{milner91polyadicpi}). We will therefore make liberal use of
$(\nu\; \vec{x})P$.

% subsection the_syntax_and_semantics_of_the_notation_system (end)   

\input{qm2pi.qmops} 

\input{qm2pi.sterngerlach} 

\input{qm2pi.metric} 

% section concurrent_process_calculi (end)

%\input{qm2pi.proofsketch}

% section proof sketch (end)

%\input{qm2pi.slviaknots} 

% section spatial logic via knots (end)

\input{qm2pi.conclusion}

% section conclusion (end)

%\input{qm2pi.dtcodes} 

% section wiring algorithm (end)

\input{qm2pi.ack} 

% section acknowledgments (end)

\newpage


\bibliographystyle{plain}   
\bibliography{../../biblios/main.bib}

\input{qm2pi.rhodetails}

\end{document}

 

% section wiring algorithm (end)

\documentclass[12pt]{llncs}
%\documentclass{jktr}

\usepackage[pdftex]{hyperref}                   
\usepackage {listings}
\usepackage {mathpartir}
\usepackage{bcprules}
%\usepackage{listings}
                       
\usepackage{graphicx} 
%\usepackage[margins=2.5cm,nohead,nofoot]{geometry}
%\usepackage{geometry}
\usepackage{amsfonts}
\usepackage{amstext}
\usepackage{latexsym}
\usepackage{amssymb}
\usepackage{color}


%\include{myPreamble}
\include{qm2pi.local} 

%\ifpdf
%\usepackage[pdftex]{graphicx}
%\else
%\usepackage{graphicx}
%\fi

 % \ifpdf
%  \usepackage{pdfsync}
%  \if


%\title{Brief Article}
%\author{David F. Snyder}
%\author{L.G. Meredith}

%\address{Dept. of Math., Texas State University--San Marcos, San Marcos, TX 78666}
       
\pagestyle{empty}


\begin{document}

\lstset{language=[Objective]Caml,frame=shadowbox}

\input{qm2pi.front}

% section front matter (end)

\input{qm2pi.intro} 
 
% section introduction (end)

% \input{qm2pi.knotations} 

% section notation (end)

\input{qm2pi.process.calculi} 

% section concurrent_process_calculi_and_spatial_logics_ (end)
    
%\input{qm2pi.knots2pi} 

%\input{qm2pi.trefoil} 

%\input{qm2pi.mainthm} 

% subsection basic_interpretation (end)

%\input{qm2pi.rho.presentation} 
\subsection{The syntax and semantics of the notation system}\label{sub:the_syntax_and_semantics_of_the_notation_system} % (fold)

We now summarize a technical presentation of the calculus that
embodies our theory of dynamics. The typical presentation of such a
calculus follows the style of giving generators and relations on
them. The grammar, below, describing term constructors, freely
generates the set of processes, $\Proc$. This set is then quotiented
by a relation known as structural congruence and it is over this set
that the notion of dynamics is expressed. This presentation is
essentially that of \cite{MeredithR05} with the addition of
polyadicity and summation. For readability we have relegated some of
the technical subtleties to an appendix.

\subsubsection{Process grammar}\label{subsub:process_grammar}

\begin{mathpar}
  \inferrule* [lab=synchronization] {} {{M} \bc \pzero \;|\; x?F \;|\; x!C }
  \and
  \inferrule* [lab=abstraction] {} {{F} \bc (x)P}
  \and
  \inferrule* [lab=concretion] {} {{C} \bc \langle Q \rangle}
  \and
  \inferrule* [lab=process] {} {{P,Q} \bc M \;| \;P|Q \;|\; @{x}}
  \and
  \inferrule* [lab=name] {} {{x} \bc \quotep{P}}
\end{mathpar} 

Note that $\vec{x}$ (resp. $\vec{P}$) denotes a vector of names
(resp. processes) of length $|\vec{x}|$ (resp. $|\vec{P}|$). We adopt
the following useful abbreviations.

\begin{mathpar}
   x?(\vec{y}).P := x.(\vec{y})P \and  x\clift{\vec{P}} := x.\clift{\vec{P}}
   \and x!(y) := \lift{x}{\dropn{y}}
   \and \Pi_{i=0}^{n-1}P_i := P_0 | \ldots | P_{n-1}
\end{mathpar}

\subsubsection{Structural congruence}

\paragraph{Free and bound names and alpha-equivalence.} At the
core of structural equivalence is alpha-equivalence which identifies
process that are the same up to a change of variable. Formally, we
recognize the distinction between free and bound names. The free names
of a process, $\freenames{P}$, may be calculated recursively as
follows:

\begin{mathpar}
\freenames{\pzero} := \emptyset
  \and \\
  \freenames{x?(y).P} := \{ x \} \cup (\freenames{P} \setminus \{ y \})
  \and 
  \freenames{x!\langle P \rangle} := \{ x \} \cup \{ P \} 
  \and \\
  \freenames{P|Q} := \freenames{P} \cup \freenames{Q}
  \and \\
  \freenames{@{x}} := \{ x \}
\end{mathpar}

$\pi$
$\quotep{\pi}$

$\freenames{-} : \pi \to \mathcal{P}(\quotep{\pi})$

\begin{eqnarray*}
  \freenames{\pzero} & := & \emptyset \\
  \freenames{x?(y).P} & := & \{ x \} \cup (\freenames{P} \setminus \{ y \}) \\
  \freenames{x!\langle P \rangle} & := & \{ x \} \cup \{ P \} \\
  \freenames{P|Q} & := & \freenames{P} \cup \freenames{Q} \\
  \freenames{\dropn{x}} & := & \{ x \}
\end{eqnarray*}

The bound names of a process, $\boundnames{P}$, are those names occurring in $P$
that are not free. For example, in $x?(y).0$, the name $x$ is free, while $y$ is bound.

\begin{mathpar}
  \inferrule* [lab=monoidal-laws] {} { P|Q \equiv Q|P \and P|0 \equiv P \and P|(Q|R) \equiv (P|Q)|R }
\end{mathpar}

\begin{mathpar}
  \inferrule* [lab=alpha-equivalence] {} { (x)P \equiv (y)P\{y/x\} \and y \not\in \freenames{P} }
\end{mathpar}

\begin{definition}
Then two processes, $P,Q$, are alpha-equivalent if $P = Q\{\vec{y}/\vec{x}\}$ for
some $\vec{x} \in \boundnames{Q},\vec{y} \in \boundnames{P}$, where $Q\{\vec{y}/\vec{x}\}$
denotes the capture-avoiding substitution of $\vec{y}$ for $\vec{x}$ in $Q$.
\end{definition}

\begin{definition}
  The {\em structural congruence} \cite{SangiorgiWalker} , $\equiv$,
  between processes is the least congruence containing
  alpha-equivalence, satisfying the abelian monoid laws
  (associativity, commutativity and $\pzero$ as identity) for parallel
  composition $|$ and for summation $+$.
\end{definition}

\subsection{Name equivalence}

We take name equivalence, written $\nameeq$, to be the smallest
equivalence relation generated by the following rules.

\begin{mathpar}
\inferrule*[lab=Quote-drop]
{ }
{ \quotep{@{x}} \nameeq x }

\inferrule*[lab=Struct-equiv]
{ P \scong Q }
{ \quotep{P} \nameeq \quotep{Q} }
\end{mathpar}

The astute reader will have noticed that the mutual recursion of names
and processes imposes a mutual recursion on alpha-equivalence and
structural equivalence via name-equivalence. Fortunately, all of this
works out pleasantly and we may calculate in the natural way, free of
concern. The reader interested in the details is referred to the
appendix \ref{appendix:rho_details}.

\subsection{Substitution}

We use $\Proc$ for the set of processes, $\QProc$ for the set of
names, and $\id{\{}\vec{y} / \vec{x} \id{\}}$ to denote partial maps,
$s : \QProc \rightarrow \QProc$. A map, $s$ lifts, uniquely, to a map
on process terms, $\widehat{s} : \Proc \rightarrow \Proc$ by the
following equations.

\begin{mathpar}
  (0) \psubstp{Q}{P} := 0 \\
  (R \juxtap S) \psubstp{Q}{P}
  :=    
  (R)\psubstp{Q}{P} \juxtap (S) \psubstp{Q}{P} \\
  (x?(y).R) \psubstp{Q}{P}    
  :=    
  (x)\substp{Q}{P} (z)\concat( (R \psubstn{z}{y}) \psubstp{Q}{P} ) \\
  (\lift{x}{R}) \psubstp{Q}{P}  
  :=
  \lift{(x)\substp{Q}{P}}{ R \psubstp{Q}{P} } \\
%   (\dropn{x})  \psubstp{Q}{P}       
%   := 
%   \left\{ 
%     \begin{array}{ccc} 
%       \dropn{\quotep{Q}} & & x \nameeq \quotep{P} \\
%       \dropn{x} & & otherwise \\
%     \end{array}
%   \right. 
  (\dropn{x})  \psubstp{Q}{P}       
  := 
  \left\{ 
    \begin{array}{ccc} 
      Q & & x \nameeq \quotep{P} \\
      \dropn{x} & & otherwise \\
    \end{array}
  \right.
\end{mathpar}
 

where

\begin{eqnarray}
  (x)\id{\{} \lpquote Q \rpquote / \lpquote P \rpquote \id{\}}            = 
  \left\{ 
    \begin{array}{ccc}
      \lpquote Q \rpquote & & x \nameeq \lpquote P \rpquote \\
      x & & otherwise \\
    \end{array}
  \right. \nonumber
\end{eqnarray}

and $z$ is chosen distinct from $\quotep{P}$, $\quotep{Q}$, the free
names in $Q$, and all the names in $R$. Our $\alpha$-equivalence will
be built in the standard way from this substitution.

\begin{remark}\label{rem:no_self_referential_names}
  One consequence of these definitions is that $\forall P. \quotep{P}
  \not\in \freenames{P}$.
\end{remark}

\subsection{ Dynamic quote: an example }

Anticipating something of what's to come, consider applying the
substitution, $\widehat{\id{\{}u / z \id{\}}}$, to the following pair
of processes, $\lift{w}{y!(z)}$ and $w[ \lpquote y!(z) \rpquote ]$.

\begin{eqnarray}
	\lift{w}{y!(z)}\widehat{\id{\{}u / z \id{\}}}
		& = &
		\lift{w}{y!(u)} \nonumber\\
	w[ \lpquote y!(z) \rpquote ] \widehat{ \id{\{}u / z \id{\}} }
		& = &
		w[ \lpquote y!(z) \rpquote ] \nonumber
\end{eqnarray}

Because the body of the process between quotes is impervious to
substitution, we get radically different answers. In fact, by
examining the first process in an input context,
e.g. $x?(z).\lift{w}{y!(z)}$, we see that the process under the lift
operator may be shaped by prefixed inputs binding a name inside it. In
this sense, the lift operator will be seen as a way to dynamically
construct processes before reifying them as names.

Finally equipped with these standard features we can present the
dynamics of the calculus.

\subsubsection{Operational semantics} 

Finally, we introduce the computational dynamics. What marks these
algebras as distinct from other more traditionally studied algebraic
structures, e.g. vector spaces or polynomial rings, is the manner in
which dynamics is captured. In traditional structures, dynamics is typically
expressed through morphisms between such structures, as in linear maps
between vector spaces or morphisms between rings. In algebras
associated with the semantics of computation, the dynamics is
expressed as part of the algebraic structure itself, through a
reduction reduction relation typically denoted by $\red$. Below, we
give a recursive presentation of this relation for the calculus used
in the encoding.

$\red \subseteq \pi \times \pi$
$\red : \pi \to \mathcal{P}(\pi)$

\begin{mathpar}
  \inferrule* [lab=Comm] { \textsf{match}( x_{src}, x_{trgt} ) } { x_{trgt}?(y)P \; | \; x_{src}!\langle {Q} \rangle \red P\{\quotep{Q}/y}\} }
  \and \\
  \inferrule* [lab=Par] {{P} \red {P}'} {{{P} | {Q}} \red {{P}' | {Q}}}
  \and
  \inferrule* [lab=Equiv]{{{P} \scong {P}'} \andalso {{P}' \red {Q}'} \andalso {{Q}' \scong {Q}}}{{P} \red {Q}}
\end{mathpar}

\begin{eqnarray*}
  match_{\equiv} (\quotep{P},\quotep{Q}) & := & P \equiv Q \\
  match_{\dagger}(\quotep{P},\quotep{Q}) & := & \forall R. P|Q \red^{*} R => R \red^{*} 0 \\
  match_{K}(\quotep{P},\quotep{Q}) & := & K \mbox{ for some context } K
\end{eqnarray*}

$u?(x)P | u!\langle Q \rangle \red P\{\quotep{Q}/x\}$

%We write $\wred$ for $\red^*$, and $P\red$ if $\exists Q $ such that $ P \red Q$.
We write $P\red$ if $\exists Q $ such that $ P \red Q$ and $P\not\red$, otherwise.

\section{Replication}

As mentioned before, it is known that replication (and hence
recursion) can be implemented in a higher-order process algebra
\cite{SangiorgiWalker}. As our first example of calculation with the
machinery thus far presented we give the construction explicitly in
the {\rhoc}.

\begin{eqnarray}
	D_{x} & := & \prefix{x}{y}{(\binpar{\outputp{x}{y}}{@{y}})} \nonumber\\
	\bangp_{x}{P} & := & \binpar{{x}!\langle{\binpar{D_{x}}{P}}\rangle}{D_{x}} \nonumber
\end{eqnarray}

\begin{eqnarray}
	\bangp_{x}{P} & & \nonumber\\
	=
	& {x}!\langle{(\prefix{x}{y}{(\outputp{x}{y} | @{y})) | P}}\rangle 
	      | \prefix{x}{y}{(\outputp{x}{y} | @{y})} & \nonumber\\
	\red
	& (\outputp{x}{y} | @{y})\substn{\quotep{(\prefix{x}{y}{(@{y} | \outputp{x}{y})) | P}}}{y} & \nonumber\\
	=
	& \outputp{x}{\quotep{(\prefix{x}{y}{(\outputp{x}{y} | @{y})) | P}}}
	  | {(\prefix{x}{y}{(\outputp{x}{y} | @{y})) | P}} & \nonumber\\
	\red
	& \ldots & \nonumber\\
	\red^*
	& P | P | \ldots & \nonumber
\end{eqnarray}

Of course, this encoding, as an implementation, runs away, unfolding
$\bangp{P}$ eagerly. A lazier and more implementable replication
operator, restricted to input-guarded processes, may be obtained as follows.

\begin{eqnarray}
\bangp{\prefix{u}{v}{P}} 
	:= 
	\binpar{\lift{x}{\prefix{u}{v}{(\binpar{D(x)}{P})}}}{D(x)} \nonumber
\end{eqnarray}

\begin{remark}
  Note that the lazier definition still does not deal with summation
  or mixed summation (i.e. sums over input and output). The reader is
  invited to construct definitions of replication that deal with these
  features. 

  Further, the definitions are parameterized in a name, $x$. Can you,
  gentle reader, make a definition that eliminates this parameter and
  guarantees no accidental interaction between the replication
  machinery and the process being replicated -- i.e. no accidental
  sharing of names used by the process to get its work done and the
  name(s) used by the replication to effect copying. This latter
  revision of the definition of replication is crucial to obtaining
  the expected identity $!!P \sim !P$.
\end{remark}

\begin{remark}\label{rem:paradoxical_combinator}
  The reader familiar with the lambda calculus will have noticed the
  similarity between $D$ and the paradoxical combinator.

  [Ed. note: the existence of this seems to suggest we have to be more
  restrictive on the set of processes and names we admit if we are to
  support no-cloning.]
\end{remark}

\subsubsection{Bisimulation}

The computational dynamics gives rise to another kind of equivalence,
the equivalence of computational behavior. As previously mentioned
this is typically captured \emph{via} some form of bisimulation.

% The notion we use in this paper is weak barbed bisimulation
% \cite{milner91polyadicpi}.

The notion we use in this paper is derived from weak barbed
bisimulation \cite{milner91polyadicpi}. 

\begin{definition}
An \emph{observation relation}, $\downarrow_{\mathcal N}$, over a set
of names, $\mathcal N$, is the smallest relation satisfying the rules
below.

\infrule[Out-barb]{y \in {\mathcal N}, \; x \nameeq y}
		  {\outputp{x}{v} \downarrow_{\mathcal N} x}
\infrule[Par-barb]{\mbox{$P\downarrow_{\mathcal N} x$ or $Q\downarrow_{\mathcal N} x$}}
		  {\binpar{P}{Q} \downarrow_{\mathcal N} x}

We write $P \Downarrow_{\mathcal N} x$ if there is $Q$ such that 
$P \wred Q$ and $Q \downarrow_{\mathcal N} x$.
\end{definition}

\begin{definition}
%\label{def.bbisim}
An  ${\mathcal N}$-\emph{barbed bisimulation} over a set of names, ${\mathcal N}$, is a symmetric binary relation 
${\mathcal S}_{\mathcal N}$ between agents such that $P\rel{S}_{\mathcal N}Q$ implies:
\begin{enumerate}
\item If $P \red P'$ then $Q \wred Q'$ and $P'\rel{S}_{\mathcal N} Q'$.
\item If $P\downarrow_{\mathcal N} x$, then $Q\Downarrow_{\mathcal N} x$.
\end{enumerate}
$P$ is ${\mathcal N}$-barbed bisimilar to $Q$, written
$P \wbbisim_{\mathcal N} Q$, if $P \rel{S}_{\mathcal N} Q$ for some ${\mathcal N}$-barbed bisimulation ${\mathcal S}_{\mathcal N}$.
\end{definition}

$\mathcal{R} \subseteq \pi \times \pi$

$P \mathcal{R} Q => \forall P'. P \red P' \Rightarrow \exists Q'. Q \red Q', P' \mathcal{R} Q'$

$P \vdash x \Rightarrow Q \vdash x$

\begin{mathpar}
  \inferrule*[lab=Out-barb]{x \nameeq y}{{y}!\langle{Q}\rangle \vdash x}
  \and
  \inferrule*[lab=Par-barb]{\mbox{$P\vdash x$ or $Q\vdash x$}}{\binpar{P}{Q} \vdash x}
\end{mathpar}

\subsubsection{Contexts}

One of the principle advantages of computational calculi like the
$\pi$-calculus is a well-defined notion of context,
contextual-equivalence and a correlation between
contextual-equivalence and notions of bisimulation. The notion of
context allows the decomposition of a process into (sub-)process and
its syntactic environment, its context. Thus, a context may be
thought of as a process with a ``hole'' (written $\Box$) in it. The
application of a context $M$ to a process $P$, written $M[P]$, is
tantamount to filling the hole in $M$ with $P$. In this paper we do
not need the full weight of this theory, but do make use of the notion
of context in the proof the main theorem. 

\begin{mathpar}
  \inferrule* [lab=summation] {} {{M_{M},M_{N}} \bc \Box \;|\; x.M_{A} \;|\; M_{M}+M_{N}}
  \and
  \inferrule* [lab=agent] {} {{M_{A}} \bc (\vec{x})M_{P} \;| \; \clift{P_0,\ldots,M_{P},\ldots,P_N}}
  \and \\
  \inferrule* [lab=process] {} {{M_{P}} \bc M_{N} \;| \;P|M_{P} }
\end{mathpar} 

\begin{mathpar}
  \inferrule* [lab=sychronization] {} {M_{N} \bc \Box \;|\; x?M_{F} \;|\; x!M_{C}}
  \and
  \inferrule* [lab=abstraction] {} {{M_{F}} \bc (x)M_{P} }
  \and
  \inferrule* [lab=concretion] {} {{M_{C}} \bc \langle M_{P} \rangle }
  \and \\
  \inferrule* [lab=process] {} {{M_{P}} \bc M_{N} \;| \;P|M_{P} }
\end{mathpar}

\begin{definition}[contextual application] Given a context $M$, and
  process $P$, we define the \emph{contextual application}, $M[P] :=
  M\{P/\Box\}$. That is, the contextual application of M to P is the
  substitution of $P$ for $\Box$ in $M$.
\end{definition}

$\meaningof{-} : L \to \mathcal{P}(\pi)$

\begin{mathpar}
  \inferrule* [lab=collection] {} {\meaningof{true} = \pi, \and \meaningof{~E} = \pi \setminus \meaningof{E}, \and \meaningof{E_{1} \& E_{2}} = \meaningof{E_{1}} \cap \meaningof{E_{2}}}
\end{mathpar}

\begin{mathpar}
  \inferrule* [lab=structure] {} {\meaningof{0} = \{ P \in \pi | P \equiv 0 \}, \and \\ \meaningof{E_1 | E_2} = \{ P \in \pi | P \equiv P_{1} | P_{2}, P_{1} \in \meaningof{E_{1}}, P_{2} \in \meaningof{E_2}\} }
\end{mathpar}

\begin{mathpar}
 \inferrule* [lab=behavior] {} {\meaningof{\langle a?b \rangle E} = \{ P \in \pi | P \equiv Q | u?(y)P', \\ \and \\\\ \and \\ \;\;\; u \in \meaningof{a}, \forall z.P'\{z/y\} \in \meaningof{E\{z/b\}}\}, \and \\ \meaningof{a!E} = \{ P \in \pi | P \equiv Q | x!\langle P' \rangle, x \in \meaningof{a} P' \in \meaningof{E}\} }
\end{mathpar}

\begin{mathpar}
 \inferrule* [lab=nominal] {} {\meaningof{\quotep{E}} = \{ \quotep{P} \in \quotep{\pi} | P \in \meaningof{E} \}, \and \meaningof{\quotep{P}} = \{ \quotep{Q} \in \quotep{\pi} | P \equiv Q \} \and \\ \meaningof{@\quotep{E}} = \{ P \in \pi | P \equiv @x, x \in \meaningof{E} \}}
\end{mathpar}

\begin{eqnarray*}
  \\
  \meaningof{-} : TS \to ST
\end{eqnarray*}

\begin{eqnarray*}
  \\
  L : TS \to ST
\end{eqnarray*}

\begin{eqnarray*}
  \\
  P \models E \iff P \in \meaningof{E}
\end{eqnarray*}

\begin{eqnarray*}
  P \approx_{L} Q \iff \forall E \in L. P \models E \iff Q \models E
\end{eqnarray*}

\begin{eqnarray*}
  P \approx_{K} Q
\end{eqnarray*}

\begin{eqnarray*}
  P \approx Q
\end{eqnarray*}

$\approx_{K} = \approx = \approx_{L}$

\subsubsection{Contextual duality}

Note that contexts extend the quotation operation to a family of
operations from processes to names. Given a context, $M$, we can
define a \emph{nominal context}, $\quotep{M}$ by $\quotep{M}[P] :=
\quotep{M[P]}$. To foreshadow what is to come we observe that these
operations enjoy a duality with processes very much like the duality
between vectors and maps from vectors to scalars.

Further, because the calculus is essentially higher-order, we have a
correspondence between contexts and processes. More specifically,
given a name $x$ and a context $M$ we can construct $M^{*}_{x}$ such
that 

\begin{mathpar}
  M^{*}_{x} | \lift{x}{P} \red M[P]
\end{mathpar}

namely,

\begin{mathpar}
  M^{*}_{x} := x?(u).M[\dropn{u}]
\end{mathpar}

The dependence of $M^{*}_{x}$ on a name makes it an abstraction, 

\begin{mathpar}
  M^{*} := (x)x?(u).M[\dropn{u}]
\end{mathpar}

\subsection{Additional notation}

It will sometimes be convenient to denote the process a name
quotes. We already have the notation $x = \quotep{P}$, but it will be
convenient to introduce an alternate notation, $\procn{x}$, when we
want to emphasize the connection to the use of the name. Note that, by
virtue of name equivalence, $\quotep{\procn{x}} \nameeq x$; so, the
notation is consistent with previous definitions.

Further, because names have structure it is possible to effect
substitutions on the basis of that structure. This means we need to
upgrade our notation for substitutions, which we accomplish by
adapting comprehension notation. Thus,

\begin{mathpar}
  P\{ y / x : x \in S \}
\end{mathpar}

is interpreted to mean the process derived from P by replacing (in a
capture-avoiding manner) each occurrence of $x$ in $S$ by $y$. For example,

\begin{mathpar}
  P\{ \quotep{\procn{x}|\procn{x}} / x : x \in \freenames{P} \}
\end{mathpar}

will replace each (occurrence) of a free name $x$ in $P$ by
$\quotep{\procn{x}|\procn{x}}$.

Also, we will avail ourselves of the notation $x^{L}$ and $x^{R}$ to
denote injections of a name into disjoint copies of the name
space. There are numerous ways to accomplish this. One example can be
found in \cite{MeredithR05}. This notation overloads to vectors of
names: $\vec{x}^{\pi} := (x_{i}^{\pi} \; : \; 0 \leq i < |\vec{x}| )$ where $\pi \in \{L,R\}$.

We also use $P^{\Box} := P|\Box$.

In \cite{MeredithR05} an interpretation of the new operator is
given. It turns out that there are several possible interpretations
all enjoying the requisite algebraic properties of the operator (see
\cite{milner91polyadicpi}). We will therefore make liberal use of
$(\nu\; \vec{x})P$.

% subsection the_syntax_and_semantics_of_the_notation_system (end)   

\input{qm2pi.qmops} 

\input{qm2pi.sterngerlach} 

\input{qm2pi.metric} 

% section concurrent_process_calculi (end)

%\input{qm2pi.proofsketch}

% section proof sketch (end)

%\input{qm2pi.slviaknots} 

% section spatial logic via knots (end)

\input{qm2pi.conclusion}

% section conclusion (end)

%\input{qm2pi.dtcodes} 

% section wiring algorithm (end)

\input{qm2pi.ack} 

% section acknowledgments (end)

\newpage


\bibliographystyle{plain}   
\bibliography{../../biblios/main.bib}

\input{qm2pi.rhodetails}

\end{document}

 

% section acknowledgments (end)

\newpage


\bibliographystyle{plain}   
\bibliography{../../biblios/main.bib}

\documentclass[12pt]{llncs}
%\documentclass{jktr}

\usepackage[pdftex]{hyperref}                   
\usepackage {listings}
\usepackage {mathpartir}
\usepackage{bcprules}
%\usepackage{listings}
                       
\usepackage{graphicx} 
%\usepackage[margins=2.5cm,nohead,nofoot]{geometry}
%\usepackage{geometry}
\usepackage{amsfonts}
\usepackage{amstext}
\usepackage{latexsym}
\usepackage{amssymb}
\usepackage{color}


%\include{myPreamble}
\include{qm2pi.local} 

%\ifpdf
%\usepackage[pdftex]{graphicx}
%\else
%\usepackage{graphicx}
%\fi

 % \ifpdf
%  \usepackage{pdfsync}
%  \if


%\title{Brief Article}
%\author{David F. Snyder}
%\author{L.G. Meredith}

%\address{Dept. of Math., Texas State University--San Marcos, San Marcos, TX 78666}
       
\pagestyle{empty}


\begin{document}

\lstset{language=[Objective]Caml,frame=shadowbox}

\input{qm2pi.front}

% section front matter (end)

\input{qm2pi.intro} 
 
% section introduction (end)

% \input{qm2pi.knotations} 

% section notation (end)

\input{qm2pi.process.calculi} 

% section concurrent_process_calculi_and_spatial_logics_ (end)
    
%\input{qm2pi.knots2pi} 

%\input{qm2pi.trefoil} 

%\input{qm2pi.mainthm} 

% subsection basic_interpretation (end)

%\input{qm2pi.rho.presentation} 
\subsection{The syntax and semantics of the notation system}\label{sub:the_syntax_and_semantics_of_the_notation_system} % (fold)

We now summarize a technical presentation of the calculus that
embodies our theory of dynamics. The typical presentation of such a
calculus follows the style of giving generators and relations on
them. The grammar, below, describing term constructors, freely
generates the set of processes, $\Proc$. This set is then quotiented
by a relation known as structural congruence and it is over this set
that the notion of dynamics is expressed. This presentation is
essentially that of \cite{MeredithR05} with the addition of
polyadicity and summation. For readability we have relegated some of
the technical subtleties to an appendix.

\subsubsection{Process grammar}\label{subsub:process_grammar}

\begin{mathpar}
  \inferrule* [lab=synchronization] {} {{M} \bc \pzero \;|\; x?F \;|\; x!C }
  \and
  \inferrule* [lab=abstraction] {} {{F} \bc (x)P}
  \and
  \inferrule* [lab=concretion] {} {{C} \bc \langle Q \rangle}
  \and
  \inferrule* [lab=process] {} {{P,Q} \bc M \;| \;P|Q \;|\; @{x}}
  \and
  \inferrule* [lab=name] {} {{x} \bc \quotep{P}}
\end{mathpar} 

Note that $\vec{x}$ (resp. $\vec{P}$) denotes a vector of names
(resp. processes) of length $|\vec{x}|$ (resp. $|\vec{P}|$). We adopt
the following useful abbreviations.

\begin{mathpar}
   x?(\vec{y}).P := x.(\vec{y})P \and  x\clift{\vec{P}} := x.\clift{\vec{P}}
   \and x!(y) := \lift{x}{\dropn{y}}
   \and \Pi_{i=0}^{n-1}P_i := P_0 | \ldots | P_{n-1}
\end{mathpar}

\subsubsection{Structural congruence}

\paragraph{Free and bound names and alpha-equivalence.} At the
core of structural equivalence is alpha-equivalence which identifies
process that are the same up to a change of variable. Formally, we
recognize the distinction between free and bound names. The free names
of a process, $\freenames{P}$, may be calculated recursively as
follows:

\begin{mathpar}
\freenames{\pzero} := \emptyset
  \and \\
  \freenames{x?(y).P} := \{ x \} \cup (\freenames{P} \setminus \{ y \})
  \and 
  \freenames{x!\langle P \rangle} := \{ x \} \cup \{ P \} 
  \and \\
  \freenames{P|Q} := \freenames{P} \cup \freenames{Q}
  \and \\
  \freenames{@{x}} := \{ x \}
\end{mathpar}

$\pi$
$\quotep{\pi}$

$\freenames{-} : \pi \to \mathcal{P}(\quotep{\pi})$

\begin{eqnarray*}
  \freenames{\pzero} & := & \emptyset \\
  \freenames{x?(y).P} & := & \{ x \} \cup (\freenames{P} \setminus \{ y \}) \\
  \freenames{x!\langle P \rangle} & := & \{ x \} \cup \{ P \} \\
  \freenames{P|Q} & := & \freenames{P} \cup \freenames{Q} \\
  \freenames{\dropn{x}} & := & \{ x \}
\end{eqnarray*}

The bound names of a process, $\boundnames{P}$, are those names occurring in $P$
that are not free. For example, in $x?(y).0$, the name $x$ is free, while $y$ is bound.

\begin{mathpar}
  \inferrule* [lab=monoidal-laws] {} { P|Q \equiv Q|P \and P|0 \equiv P \and P|(Q|R) \equiv (P|Q)|R }
\end{mathpar}

\begin{mathpar}
  \inferrule* [lab=alpha-equivalence] {} { (x)P \equiv (y)P\{y/x\} \and y \not\in \freenames{P} }
\end{mathpar}

\begin{definition}
Then two processes, $P,Q$, are alpha-equivalent if $P = Q\{\vec{y}/\vec{x}\}$ for
some $\vec{x} \in \boundnames{Q},\vec{y} \in \boundnames{P}$, where $Q\{\vec{y}/\vec{x}\}$
denotes the capture-avoiding substitution of $\vec{y}$ for $\vec{x}$ in $Q$.
\end{definition}

\begin{definition}
  The {\em structural congruence} \cite{SangiorgiWalker} , $\equiv$,
  between processes is the least congruence containing
  alpha-equivalence, satisfying the abelian monoid laws
  (associativity, commutativity and $\pzero$ as identity) for parallel
  composition $|$ and for summation $+$.
\end{definition}

\subsection{Name equivalence}

We take name equivalence, written $\nameeq$, to be the smallest
equivalence relation generated by the following rules.

\begin{mathpar}
\inferrule*[lab=Quote-drop]
{ }
{ \quotep{@{x}} \nameeq x }

\inferrule*[lab=Struct-equiv]
{ P \scong Q }
{ \quotep{P} \nameeq \quotep{Q} }
\end{mathpar}

The astute reader will have noticed that the mutual recursion of names
and processes imposes a mutual recursion on alpha-equivalence and
structural equivalence via name-equivalence. Fortunately, all of this
works out pleasantly and we may calculate in the natural way, free of
concern. The reader interested in the details is referred to the
appendix \ref{appendix:rho_details}.

\subsection{Substitution}

We use $\Proc$ for the set of processes, $\QProc$ for the set of
names, and $\id{\{}\vec{y} / \vec{x} \id{\}}$ to denote partial maps,
$s : \QProc \rightarrow \QProc$. A map, $s$ lifts, uniquely, to a map
on process terms, $\widehat{s} : \Proc \rightarrow \Proc$ by the
following equations.

\begin{mathpar}
  (0) \psubstp{Q}{P} := 0 \\
  (R \juxtap S) \psubstp{Q}{P}
  :=    
  (R)\psubstp{Q}{P} \juxtap (S) \psubstp{Q}{P} \\
  (x?(y).R) \psubstp{Q}{P}    
  :=    
  (x)\substp{Q}{P} (z)\concat( (R \psubstn{z}{y}) \psubstp{Q}{P} ) \\
  (\lift{x}{R}) \psubstp{Q}{P}  
  :=
  \lift{(x)\substp{Q}{P}}{ R \psubstp{Q}{P} } \\
%   (\dropn{x})  \psubstp{Q}{P}       
%   := 
%   \left\{ 
%     \begin{array}{ccc} 
%       \dropn{\quotep{Q}} & & x \nameeq \quotep{P} \\
%       \dropn{x} & & otherwise \\
%     \end{array}
%   \right. 
  (\dropn{x})  \psubstp{Q}{P}       
  := 
  \left\{ 
    \begin{array}{ccc} 
      Q & & x \nameeq \quotep{P} \\
      \dropn{x} & & otherwise \\
    \end{array}
  \right.
\end{mathpar}
 

where

\begin{eqnarray}
  (x)\id{\{} \lpquote Q \rpquote / \lpquote P \rpquote \id{\}}            = 
  \left\{ 
    \begin{array}{ccc}
      \lpquote Q \rpquote & & x \nameeq \lpquote P \rpquote \\
      x & & otherwise \\
    \end{array}
  \right. \nonumber
\end{eqnarray}

and $z$ is chosen distinct from $\quotep{P}$, $\quotep{Q}$, the free
names in $Q$, and all the names in $R$. Our $\alpha$-equivalence will
be built in the standard way from this substitution.

\begin{remark}\label{rem:no_self_referential_names}
  One consequence of these definitions is that $\forall P. \quotep{P}
  \not\in \freenames{P}$.
\end{remark}

\subsection{ Dynamic quote: an example }

Anticipating something of what's to come, consider applying the
substitution, $\widehat{\id{\{}u / z \id{\}}}$, to the following pair
of processes, $\lift{w}{y!(z)}$ and $w[ \lpquote y!(z) \rpquote ]$.

\begin{eqnarray}
	\lift{w}{y!(z)}\widehat{\id{\{}u / z \id{\}}}
		& = &
		\lift{w}{y!(u)} \nonumber\\
	w[ \lpquote y!(z) \rpquote ] \widehat{ \id{\{}u / z \id{\}} }
		& = &
		w[ \lpquote y!(z) \rpquote ] \nonumber
\end{eqnarray}

Because the body of the process between quotes is impervious to
substitution, we get radically different answers. In fact, by
examining the first process in an input context,
e.g. $x?(z).\lift{w}{y!(z)}$, we see that the process under the lift
operator may be shaped by prefixed inputs binding a name inside it. In
this sense, the lift operator will be seen as a way to dynamically
construct processes before reifying them as names.

Finally equipped with these standard features we can present the
dynamics of the calculus.

\subsubsection{Operational semantics} 

Finally, we introduce the computational dynamics. What marks these
algebras as distinct from other more traditionally studied algebraic
structures, e.g. vector spaces or polynomial rings, is the manner in
which dynamics is captured. In traditional structures, dynamics is typically
expressed through morphisms between such structures, as in linear maps
between vector spaces or morphisms between rings. In algebras
associated with the semantics of computation, the dynamics is
expressed as part of the algebraic structure itself, through a
reduction reduction relation typically denoted by $\red$. Below, we
give a recursive presentation of this relation for the calculus used
in the encoding.

$\red \subseteq \pi \times \pi$
$\red : \pi \to \mathcal{P}(\pi)$

\begin{mathpar}
  \inferrule* [lab=Comm] { \textsf{match}( x_{src}, x_{trgt} ) } { x_{trgt}?(y)P \; | \; x_{src}!\langle {Q} \rangle \red P\{\quotep{Q}/y}\} }
  \and \\
  \inferrule* [lab=Par] {{P} \red {P}'} {{{P} | {Q}} \red {{P}' | {Q}}}
  \and
  \inferrule* [lab=Equiv]{{{P} \scong {P}'} \andalso {{P}' \red {Q}'} \andalso {{Q}' \scong {Q}}}{{P} \red {Q}}
\end{mathpar}

\begin{eqnarray*}
  match_{\equiv} (\quotep{P},\quotep{Q}) & := & P \equiv Q \\
  match_{\dagger}(\quotep{P},\quotep{Q}) & := & \forall R. P|Q \red^{*} R => R \red^{*} 0 \\
  match_{K}(\quotep{P},\quotep{Q}) & := & K \mbox{ for some context } K
\end{eqnarray*}

$u?(x)P | u!\langle Q \rangle \red P\{\quotep{Q}/x\}$

%We write $\wred$ for $\red^*$, and $P\red$ if $\exists Q $ such that $ P \red Q$.
We write $P\red$ if $\exists Q $ such that $ P \red Q$ and $P\not\red$, otherwise.

\section{Replication}

As mentioned before, it is known that replication (and hence
recursion) can be implemented in a higher-order process algebra
\cite{SangiorgiWalker}. As our first example of calculation with the
machinery thus far presented we give the construction explicitly in
the {\rhoc}.

\begin{eqnarray}
	D_{x} & := & \prefix{x}{y}{(\binpar{\outputp{x}{y}}{@{y}})} \nonumber\\
	\bangp_{x}{P} & := & \binpar{{x}!\langle{\binpar{D_{x}}{P}}\rangle}{D_{x}} \nonumber
\end{eqnarray}

\begin{eqnarray}
	\bangp_{x}{P} & & \nonumber\\
	=
	& {x}!\langle{(\prefix{x}{y}{(\outputp{x}{y} | @{y})) | P}}\rangle 
	      | \prefix{x}{y}{(\outputp{x}{y} | @{y})} & \nonumber\\
	\red
	& (\outputp{x}{y} | @{y})\substn{\quotep{(\prefix{x}{y}{(@{y} | \outputp{x}{y})) | P}}}{y} & \nonumber\\
	=
	& \outputp{x}{\quotep{(\prefix{x}{y}{(\outputp{x}{y} | @{y})) | P}}}
	  | {(\prefix{x}{y}{(\outputp{x}{y} | @{y})) | P}} & \nonumber\\
	\red
	& \ldots & \nonumber\\
	\red^*
	& P | P | \ldots & \nonumber
\end{eqnarray}

Of course, this encoding, as an implementation, runs away, unfolding
$\bangp{P}$ eagerly. A lazier and more implementable replication
operator, restricted to input-guarded processes, may be obtained as follows.

\begin{eqnarray}
\bangp{\prefix{u}{v}{P}} 
	:= 
	\binpar{\lift{x}{\prefix{u}{v}{(\binpar{D(x)}{P})}}}{D(x)} \nonumber
\end{eqnarray}

\begin{remark}
  Note that the lazier definition still does not deal with summation
  or mixed summation (i.e. sums over input and output). The reader is
  invited to construct definitions of replication that deal with these
  features. 

  Further, the definitions are parameterized in a name, $x$. Can you,
  gentle reader, make a definition that eliminates this parameter and
  guarantees no accidental interaction between the replication
  machinery and the process being replicated -- i.e. no accidental
  sharing of names used by the process to get its work done and the
  name(s) used by the replication to effect copying. This latter
  revision of the definition of replication is crucial to obtaining
  the expected identity $!!P \sim !P$.
\end{remark}

\begin{remark}\label{rem:paradoxical_combinator}
  The reader familiar with the lambda calculus will have noticed the
  similarity between $D$ and the paradoxical combinator.

  [Ed. note: the existence of this seems to suggest we have to be more
  restrictive on the set of processes and names we admit if we are to
  support no-cloning.]
\end{remark}

\subsubsection{Bisimulation}

The computational dynamics gives rise to another kind of equivalence,
the equivalence of computational behavior. As previously mentioned
this is typically captured \emph{via} some form of bisimulation.

% The notion we use in this paper is weak barbed bisimulation
% \cite{milner91polyadicpi}.

The notion we use in this paper is derived from weak barbed
bisimulation \cite{milner91polyadicpi}. 

\begin{definition}
An \emph{observation relation}, $\downarrow_{\mathcal N}$, over a set
of names, $\mathcal N$, is the smallest relation satisfying the rules
below.

\infrule[Out-barb]{y \in {\mathcal N}, \; x \nameeq y}
		  {\outputp{x}{v} \downarrow_{\mathcal N} x}
\infrule[Par-barb]{\mbox{$P\downarrow_{\mathcal N} x$ or $Q\downarrow_{\mathcal N} x$}}
		  {\binpar{P}{Q} \downarrow_{\mathcal N} x}

We write $P \Downarrow_{\mathcal N} x$ if there is $Q$ such that 
$P \wred Q$ and $Q \downarrow_{\mathcal N} x$.
\end{definition}

\begin{definition}
%\label{def.bbisim}
An  ${\mathcal N}$-\emph{barbed bisimulation} over a set of names, ${\mathcal N}$, is a symmetric binary relation 
${\mathcal S}_{\mathcal N}$ between agents such that $P\rel{S}_{\mathcal N}Q$ implies:
\begin{enumerate}
\item If $P \red P'$ then $Q \wred Q'$ and $P'\rel{S}_{\mathcal N} Q'$.
\item If $P\downarrow_{\mathcal N} x$, then $Q\Downarrow_{\mathcal N} x$.
\end{enumerate}
$P$ is ${\mathcal N}$-barbed bisimilar to $Q$, written
$P \wbbisim_{\mathcal N} Q$, if $P \rel{S}_{\mathcal N} Q$ for some ${\mathcal N}$-barbed bisimulation ${\mathcal S}_{\mathcal N}$.
\end{definition}

$\mathcal{R} \subseteq \pi \times \pi$

$P \mathcal{R} Q => \forall P'. P \red P' \Rightarrow \exists Q'. Q \red Q', P' \mathcal{R} Q'$

$P \vdash x \Rightarrow Q \vdash x$

\begin{mathpar}
  \inferrule*[lab=Out-barb]{x \nameeq y}{{y}!\langle{Q}\rangle \vdash x}
  \and
  \inferrule*[lab=Par-barb]{\mbox{$P\vdash x$ or $Q\vdash x$}}{\binpar{P}{Q} \vdash x}
\end{mathpar}

\subsubsection{Contexts}

One of the principle advantages of computational calculi like the
$\pi$-calculus is a well-defined notion of context,
contextual-equivalence and a correlation between
contextual-equivalence and notions of bisimulation. The notion of
context allows the decomposition of a process into (sub-)process and
its syntactic environment, its context. Thus, a context may be
thought of as a process with a ``hole'' (written $\Box$) in it. The
application of a context $M$ to a process $P$, written $M[P]$, is
tantamount to filling the hole in $M$ with $P$. In this paper we do
not need the full weight of this theory, but do make use of the notion
of context in the proof the main theorem. 

\begin{mathpar}
  \inferrule* [lab=summation] {} {{M_{M},M_{N}} \bc \Box \;|\; x.M_{A} \;|\; M_{M}+M_{N}}
  \and
  \inferrule* [lab=agent] {} {{M_{A}} \bc (\vec{x})M_{P} \;| \; \clift{P_0,\ldots,M_{P},\ldots,P_N}}
  \and \\
  \inferrule* [lab=process] {} {{M_{P}} \bc M_{N} \;| \;P|M_{P} }
\end{mathpar} 

\begin{mathpar}
  \inferrule* [lab=sychronization] {} {M_{N} \bc \Box \;|\; x?M_{F} \;|\; x!M_{C}}
  \and
  \inferrule* [lab=abstraction] {} {{M_{F}} \bc (x)M_{P} }
  \and
  \inferrule* [lab=concretion] {} {{M_{C}} \bc \langle M_{P} \rangle }
  \and \\
  \inferrule* [lab=process] {} {{M_{P}} \bc M_{N} \;| \;P|M_{P} }
\end{mathpar}

\begin{definition}[contextual application] Given a context $M$, and
  process $P$, we define the \emph{contextual application}, $M[P] :=
  M\{P/\Box\}$. That is, the contextual application of M to P is the
  substitution of $P$ for $\Box$ in $M$.
\end{definition}

$\meaningof{-} : L \to \mathcal{P}(\pi)$

\begin{mathpar}
  \inferrule* [lab=collection] {} {\meaningof{true} = \pi, \and \meaningof{~E} = \pi \setminus \meaningof{E}, \and \meaningof{E_{1} \& E_{2}} = \meaningof{E_{1}} \cap \meaningof{E_{2}}}
\end{mathpar}

\begin{mathpar}
  \inferrule* [lab=structure] {} {\meaningof{0} = \{ P \in \pi | P \equiv 0 \}, \and \\ \meaningof{E_1 | E_2} = \{ P \in \pi | P \equiv P_{1} | P_{2}, P_{1} \in \meaningof{E_{1}}, P_{2} \in \meaningof{E_2}\} }
\end{mathpar}

\begin{mathpar}
 \inferrule* [lab=behavior] {} {\meaningof{\langle a?b \rangle E} = \{ P \in \pi | P \equiv Q | u?(y)P', \\ \and \\\\ \and \\ \;\;\; u \in \meaningof{a}, \forall z.P'\{z/y\} \in \meaningof{E\{z/b\}}\}, \and \\ \meaningof{a!E} = \{ P \in \pi | P \equiv Q | x!\langle P' \rangle, x \in \meaningof{a} P' \in \meaningof{E}\} }
\end{mathpar}

\begin{mathpar}
 \inferrule* [lab=nominal] {} {\meaningof{\quotep{E}} = \{ \quotep{P} \in \quotep{\pi} | P \in \meaningof{E} \}, \and \meaningof{\quotep{P}} = \{ \quotep{Q} \in \quotep{\pi} | P \equiv Q \} \and \\ \meaningof{@\quotep{E}} = \{ P \in \pi | P \equiv @x, x \in \meaningof{E} \}}
\end{mathpar}

\begin{eqnarray*}
  \\
  \meaningof{-} : TS \to ST
\end{eqnarray*}

\begin{eqnarray*}
  \\
  L : TS \to ST
\end{eqnarray*}

\begin{eqnarray*}
  \\
  P \models E \iff P \in \meaningof{E}
\end{eqnarray*}

\begin{eqnarray*}
  P \approx_{L} Q \iff \forall E \in L. P \models E \iff Q \models E
\end{eqnarray*}

\begin{eqnarray*}
  P \approx_{K} Q
\end{eqnarray*}

\begin{eqnarray*}
  P \approx Q
\end{eqnarray*}

$\approx_{K} = \approx = \approx_{L}$

\subsubsection{Contextual duality}

Note that contexts extend the quotation operation to a family of
operations from processes to names. Given a context, $M$, we can
define a \emph{nominal context}, $\quotep{M}$ by $\quotep{M}[P] :=
\quotep{M[P]}$. To foreshadow what is to come we observe that these
operations enjoy a duality with processes very much like the duality
between vectors and maps from vectors to scalars.

Further, because the calculus is essentially higher-order, we have a
correspondence between contexts and processes. More specifically,
given a name $x$ and a context $M$ we can construct $M^{*}_{x}$ such
that 

\begin{mathpar}
  M^{*}_{x} | \lift{x}{P} \red M[P]
\end{mathpar}

namely,

\begin{mathpar}
  M^{*}_{x} := x?(u).M[\dropn{u}]
\end{mathpar}

The dependence of $M^{*}_{x}$ on a name makes it an abstraction, 

\begin{mathpar}
  M^{*} := (x)x?(u).M[\dropn{u}]
\end{mathpar}

\subsection{Additional notation}

It will sometimes be convenient to denote the process a name
quotes. We already have the notation $x = \quotep{P}$, but it will be
convenient to introduce an alternate notation, $\procn{x}$, when we
want to emphasize the connection to the use of the name. Note that, by
virtue of name equivalence, $\quotep{\procn{x}} \nameeq x$; so, the
notation is consistent with previous definitions.

Further, because names have structure it is possible to effect
substitutions on the basis of that structure. This means we need to
upgrade our notation for substitutions, which we accomplish by
adapting comprehension notation. Thus,

\begin{mathpar}
  P\{ y / x : x \in S \}
\end{mathpar}

is interpreted to mean the process derived from P by replacing (in a
capture-avoiding manner) each occurrence of $x$ in $S$ by $y$. For example,

\begin{mathpar}
  P\{ \quotep{\procn{x}|\procn{x}} / x : x \in \freenames{P} \}
\end{mathpar}

will replace each (occurrence) of a free name $x$ in $P$ by
$\quotep{\procn{x}|\procn{x}}$.

Also, we will avail ourselves of the notation $x^{L}$ and $x^{R}$ to
denote injections of a name into disjoint copies of the name
space. There are numerous ways to accomplish this. One example can be
found in \cite{MeredithR05}. This notation overloads to vectors of
names: $\vec{x}^{\pi} := (x_{i}^{\pi} \; : \; 0 \leq i < |\vec{x}| )$ where $\pi \in \{L,R\}$.

We also use $P^{\Box} := P|\Box$.

In \cite{MeredithR05} an interpretation of the new operator is
given. It turns out that there are several possible interpretations
all enjoying the requisite algebraic properties of the operator (see
\cite{milner91polyadicpi}). We will therefore make liberal use of
$(\nu\; \vec{x})P$.

% subsection the_syntax_and_semantics_of_the_notation_system (end)   

\input{qm2pi.qmops} 

\input{qm2pi.sterngerlach} 

\input{qm2pi.metric} 

% section concurrent_process_calculi (end)

%\input{qm2pi.proofsketch}

% section proof sketch (end)

%\input{qm2pi.slviaknots} 

% section spatial logic via knots (end)

\input{qm2pi.conclusion}

% section conclusion (end)

%\input{qm2pi.dtcodes} 

% section wiring algorithm (end)

\input{qm2pi.ack} 

% section acknowledgments (end)

\newpage


\bibliographystyle{plain}   
\bibliography{../../biblios/main.bib}

\input{qm2pi.rhodetails}

\end{document}



\end{document}

 

%\documentclass[12pt]{llncs}
%\documentclass{jktr}

\usepackage[pdftex]{hyperref}                   
\usepackage {listings}
\usepackage {mathpartir}
\usepackage{bcprules}
%\usepackage{listings}
                       
\usepackage{graphicx} 
%\usepackage[margins=2.5cm,nohead,nofoot]{geometry}
%\usepackage{geometry}
\usepackage{amsfonts}
\usepackage{amstext}
\usepackage{latexsym}
\usepackage{amssymb}
\usepackage{color}


%\include{myPreamble}
\documentclass[12pt]{llncs}
%\documentclass{jktr}

\usepackage[pdftex]{hyperref}                   
\usepackage {listings}
\usepackage {mathpartir}
\usepackage{bcprules}
%\usepackage{listings}
                       
\usepackage{graphicx} 
%\usepackage[margins=2.5cm,nohead,nofoot]{geometry}
%\usepackage{geometry}
\usepackage{amsfonts}
\usepackage{amstext}
\usepackage{latexsym}
\usepackage{amssymb}
\usepackage{color}


%\include{myPreamble}
\include{qm2pi.local} 

%\ifpdf
%\usepackage[pdftex]{graphicx}
%\else
%\usepackage{graphicx}
%\fi

 % \ifpdf
%  \usepackage{pdfsync}
%  \if


%\title{Brief Article}
%\author{David F. Snyder}
%\author{L.G. Meredith}

%\address{Dept. of Math., Texas State University--San Marcos, San Marcos, TX 78666}
       
\pagestyle{empty}


\begin{document}

\lstset{language=[Objective]Caml,frame=shadowbox}

\input{qm2pi.front}

% section front matter (end)

\input{qm2pi.intro} 
 
% section introduction (end)

% \input{qm2pi.knotations} 

% section notation (end)

\input{qm2pi.process.calculi} 

% section concurrent_process_calculi_and_spatial_logics_ (end)
    
%\input{qm2pi.knots2pi} 

%\input{qm2pi.trefoil} 

%\input{qm2pi.mainthm} 

% subsection basic_interpretation (end)

%\input{qm2pi.rho.presentation} 
\subsection{The syntax and semantics of the notation system}\label{sub:the_syntax_and_semantics_of_the_notation_system} % (fold)

We now summarize a technical presentation of the calculus that
embodies our theory of dynamics. The typical presentation of such a
calculus follows the style of giving generators and relations on
them. The grammar, below, describing term constructors, freely
generates the set of processes, $\Proc$. This set is then quotiented
by a relation known as structural congruence and it is over this set
that the notion of dynamics is expressed. This presentation is
essentially that of \cite{MeredithR05} with the addition of
polyadicity and summation. For readability we have relegated some of
the technical subtleties to an appendix.

\subsubsection{Process grammar}\label{subsub:process_grammar}

\begin{mathpar}
  \inferrule* [lab=synchronization] {} {{M} \bc \pzero \;|\; x?F \;|\; x!C }
  \and
  \inferrule* [lab=abstraction] {} {{F} \bc (x)P}
  \and
  \inferrule* [lab=concretion] {} {{C} \bc \langle Q \rangle}
  \and
  \inferrule* [lab=process] {} {{P,Q} \bc M \;| \;P|Q \;|\; @{x}}
  \and
  \inferrule* [lab=name] {} {{x} \bc \quotep{P}}
\end{mathpar} 

Note that $\vec{x}$ (resp. $\vec{P}$) denotes a vector of names
(resp. processes) of length $|\vec{x}|$ (resp. $|\vec{P}|$). We adopt
the following useful abbreviations.

\begin{mathpar}
   x?(\vec{y}).P := x.(\vec{y})P \and  x\clift{\vec{P}} := x.\clift{\vec{P}}
   \and x!(y) := \lift{x}{\dropn{y}}
   \and \Pi_{i=0}^{n-1}P_i := P_0 | \ldots | P_{n-1}
\end{mathpar}

\subsubsection{Structural congruence}

\paragraph{Free and bound names and alpha-equivalence.} At the
core of structural equivalence is alpha-equivalence which identifies
process that are the same up to a change of variable. Formally, we
recognize the distinction between free and bound names. The free names
of a process, $\freenames{P}$, may be calculated recursively as
follows:

\begin{mathpar}
\freenames{\pzero} := \emptyset
  \and \\
  \freenames{x?(y).P} := \{ x \} \cup (\freenames{P} \setminus \{ y \})
  \and 
  \freenames{x!\langle P \rangle} := \{ x \} \cup \{ P \} 
  \and \\
  \freenames{P|Q} := \freenames{P} \cup \freenames{Q}
  \and \\
  \freenames{@{x}} := \{ x \}
\end{mathpar}

$\pi$
$\quotep{\pi}$

$\freenames{-} : \pi \to \mathcal{P}(\quotep{\pi})$

\begin{eqnarray*}
  \freenames{\pzero} & := & \emptyset \\
  \freenames{x?(y).P} & := & \{ x \} \cup (\freenames{P} \setminus \{ y \}) \\
  \freenames{x!\langle P \rangle} & := & \{ x \} \cup \{ P \} \\
  \freenames{P|Q} & := & \freenames{P} \cup \freenames{Q} \\
  \freenames{\dropn{x}} & := & \{ x \}
\end{eqnarray*}

The bound names of a process, $\boundnames{P}$, are those names occurring in $P$
that are not free. For example, in $x?(y).0$, the name $x$ is free, while $y$ is bound.

\begin{mathpar}
  \inferrule* [lab=monoidal-laws] {} { P|Q \equiv Q|P \and P|0 \equiv P \and P|(Q|R) \equiv (P|Q)|R }
\end{mathpar}

\begin{mathpar}
  \inferrule* [lab=alpha-equivalence] {} { (x)P \equiv (y)P\{y/x\} \and y \not\in \freenames{P} }
\end{mathpar}

\begin{definition}
Then two processes, $P,Q$, are alpha-equivalent if $P = Q\{\vec{y}/\vec{x}\}$ for
some $\vec{x} \in \boundnames{Q},\vec{y} \in \boundnames{P}$, where $Q\{\vec{y}/\vec{x}\}$
denotes the capture-avoiding substitution of $\vec{y}$ for $\vec{x}$ in $Q$.
\end{definition}

\begin{definition}
  The {\em structural congruence} \cite{SangiorgiWalker} , $\equiv$,
  between processes is the least congruence containing
  alpha-equivalence, satisfying the abelian monoid laws
  (associativity, commutativity and $\pzero$ as identity) for parallel
  composition $|$ and for summation $+$.
\end{definition}

\subsection{Name equivalence}

We take name equivalence, written $\nameeq$, to be the smallest
equivalence relation generated by the following rules.

\begin{mathpar}
\inferrule*[lab=Quote-drop]
{ }
{ \quotep{@{x}} \nameeq x }

\inferrule*[lab=Struct-equiv]
{ P \scong Q }
{ \quotep{P} \nameeq \quotep{Q} }
\end{mathpar}

The astute reader will have noticed that the mutual recursion of names
and processes imposes a mutual recursion on alpha-equivalence and
structural equivalence via name-equivalence. Fortunately, all of this
works out pleasantly and we may calculate in the natural way, free of
concern. The reader interested in the details is referred to the
appendix \ref{appendix:rho_details}.

\subsection{Substitution}

We use $\Proc$ for the set of processes, $\QProc$ for the set of
names, and $\id{\{}\vec{y} / \vec{x} \id{\}}$ to denote partial maps,
$s : \QProc \rightarrow \QProc$. A map, $s$ lifts, uniquely, to a map
on process terms, $\widehat{s} : \Proc \rightarrow \Proc$ by the
following equations.

\begin{mathpar}
  (0) \psubstp{Q}{P} := 0 \\
  (R \juxtap S) \psubstp{Q}{P}
  :=    
  (R)\psubstp{Q}{P} \juxtap (S) \psubstp{Q}{P} \\
  (x?(y).R) \psubstp{Q}{P}    
  :=    
  (x)\substp{Q}{P} (z)\concat( (R \psubstn{z}{y}) \psubstp{Q}{P} ) \\
  (\lift{x}{R}) \psubstp{Q}{P}  
  :=
  \lift{(x)\substp{Q}{P}}{ R \psubstp{Q}{P} } \\
%   (\dropn{x})  \psubstp{Q}{P}       
%   := 
%   \left\{ 
%     \begin{array}{ccc} 
%       \dropn{\quotep{Q}} & & x \nameeq \quotep{P} \\
%       \dropn{x} & & otherwise \\
%     \end{array}
%   \right. 
  (\dropn{x})  \psubstp{Q}{P}       
  := 
  \left\{ 
    \begin{array}{ccc} 
      Q & & x \nameeq \quotep{P} \\
      \dropn{x} & & otherwise \\
    \end{array}
  \right.
\end{mathpar}
 

where

\begin{eqnarray}
  (x)\id{\{} \lpquote Q \rpquote / \lpquote P \rpquote \id{\}}            = 
  \left\{ 
    \begin{array}{ccc}
      \lpquote Q \rpquote & & x \nameeq \lpquote P \rpquote \\
      x & & otherwise \\
    \end{array}
  \right. \nonumber
\end{eqnarray}

and $z$ is chosen distinct from $\quotep{P}$, $\quotep{Q}$, the free
names in $Q$, and all the names in $R$. Our $\alpha$-equivalence will
be built in the standard way from this substitution.

\begin{remark}\label{rem:no_self_referential_names}
  One consequence of these definitions is that $\forall P. \quotep{P}
  \not\in \freenames{P}$.
\end{remark}

\subsection{ Dynamic quote: an example }

Anticipating something of what's to come, consider applying the
substitution, $\widehat{\id{\{}u / z \id{\}}}$, to the following pair
of processes, $\lift{w}{y!(z)}$ and $w[ \lpquote y!(z) \rpquote ]$.

\begin{eqnarray}
	\lift{w}{y!(z)}\widehat{\id{\{}u / z \id{\}}}
		& = &
		\lift{w}{y!(u)} \nonumber\\
	w[ \lpquote y!(z) \rpquote ] \widehat{ \id{\{}u / z \id{\}} }
		& = &
		w[ \lpquote y!(z) \rpquote ] \nonumber
\end{eqnarray}

Because the body of the process between quotes is impervious to
substitution, we get radically different answers. In fact, by
examining the first process in an input context,
e.g. $x?(z).\lift{w}{y!(z)}$, we see that the process under the lift
operator may be shaped by prefixed inputs binding a name inside it. In
this sense, the lift operator will be seen as a way to dynamically
construct processes before reifying them as names.

Finally equipped with these standard features we can present the
dynamics of the calculus.

\subsubsection{Operational semantics} 

Finally, we introduce the computational dynamics. What marks these
algebras as distinct from other more traditionally studied algebraic
structures, e.g. vector spaces or polynomial rings, is the manner in
which dynamics is captured. In traditional structures, dynamics is typically
expressed through morphisms between such structures, as in linear maps
between vector spaces or morphisms between rings. In algebras
associated with the semantics of computation, the dynamics is
expressed as part of the algebraic structure itself, through a
reduction reduction relation typically denoted by $\red$. Below, we
give a recursive presentation of this relation for the calculus used
in the encoding.

$\red \subseteq \pi \times \pi$
$\red : \pi \to \mathcal{P}(\pi)$

\begin{mathpar}
  \inferrule* [lab=Comm] { \textsf{match}( x_{src}, x_{trgt} ) } { x_{trgt}?(y)P \; | \; x_{src}!\langle {Q} \rangle \red P\{\quotep{Q}/y}\} }
  \and \\
  \inferrule* [lab=Par] {{P} \red {P}'} {{{P} | {Q}} \red {{P}' | {Q}}}
  \and
  \inferrule* [lab=Equiv]{{{P} \scong {P}'} \andalso {{P}' \red {Q}'} \andalso {{Q}' \scong {Q}}}{{P} \red {Q}}
\end{mathpar}

\begin{eqnarray*}
  match_{\equiv} (\quotep{P},\quotep{Q}) & := & P \equiv Q \\
  match_{\dagger}(\quotep{P},\quotep{Q}) & := & \forall R. P|Q \red^{*} R => R \red^{*} 0 \\
  match_{K}(\quotep{P},\quotep{Q}) & := & K \mbox{ for some context } K
\end{eqnarray*}

$u?(x)P | u!\langle Q \rangle \red P\{\quotep{Q}/x\}$

%We write $\wred$ for $\red^*$, and $P\red$ if $\exists Q $ such that $ P \red Q$.
We write $P\red$ if $\exists Q $ such that $ P \red Q$ and $P\not\red$, otherwise.

\section{Replication}

As mentioned before, it is known that replication (and hence
recursion) can be implemented in a higher-order process algebra
\cite{SangiorgiWalker}. As our first example of calculation with the
machinery thus far presented we give the construction explicitly in
the {\rhoc}.

\begin{eqnarray}
	D_{x} & := & \prefix{x}{y}{(\binpar{\outputp{x}{y}}{@{y}})} \nonumber\\
	\bangp_{x}{P} & := & \binpar{{x}!\langle{\binpar{D_{x}}{P}}\rangle}{D_{x}} \nonumber
\end{eqnarray}

\begin{eqnarray}
	\bangp_{x}{P} & & \nonumber\\
	=
	& {x}!\langle{(\prefix{x}{y}{(\outputp{x}{y} | @{y})) | P}}\rangle 
	      | \prefix{x}{y}{(\outputp{x}{y} | @{y})} & \nonumber\\
	\red
	& (\outputp{x}{y} | @{y})\substn{\quotep{(\prefix{x}{y}{(@{y} | \outputp{x}{y})) | P}}}{y} & \nonumber\\
	=
	& \outputp{x}{\quotep{(\prefix{x}{y}{(\outputp{x}{y} | @{y})) | P}}}
	  | {(\prefix{x}{y}{(\outputp{x}{y} | @{y})) | P}} & \nonumber\\
	\red
	& \ldots & \nonumber\\
	\red^*
	& P | P | \ldots & \nonumber
\end{eqnarray}

Of course, this encoding, as an implementation, runs away, unfolding
$\bangp{P}$ eagerly. A lazier and more implementable replication
operator, restricted to input-guarded processes, may be obtained as follows.

\begin{eqnarray}
\bangp{\prefix{u}{v}{P}} 
	:= 
	\binpar{\lift{x}{\prefix{u}{v}{(\binpar{D(x)}{P})}}}{D(x)} \nonumber
\end{eqnarray}

\begin{remark}
  Note that the lazier definition still does not deal with summation
  or mixed summation (i.e. sums over input and output). The reader is
  invited to construct definitions of replication that deal with these
  features. 

  Further, the definitions are parameterized in a name, $x$. Can you,
  gentle reader, make a definition that eliminates this parameter and
  guarantees no accidental interaction between the replication
  machinery and the process being replicated -- i.e. no accidental
  sharing of names used by the process to get its work done and the
  name(s) used by the replication to effect copying. This latter
  revision of the definition of replication is crucial to obtaining
  the expected identity $!!P \sim !P$.
\end{remark}

\begin{remark}\label{rem:paradoxical_combinator}
  The reader familiar with the lambda calculus will have noticed the
  similarity between $D$ and the paradoxical combinator.

  [Ed. note: the existence of this seems to suggest we have to be more
  restrictive on the set of processes and names we admit if we are to
  support no-cloning.]
\end{remark}

\subsubsection{Bisimulation}

The computational dynamics gives rise to another kind of equivalence,
the equivalence of computational behavior. As previously mentioned
this is typically captured \emph{via} some form of bisimulation.

% The notion we use in this paper is weak barbed bisimulation
% \cite{milner91polyadicpi}.

The notion we use in this paper is derived from weak barbed
bisimulation \cite{milner91polyadicpi}. 

\begin{definition}
An \emph{observation relation}, $\downarrow_{\mathcal N}$, over a set
of names, $\mathcal N$, is the smallest relation satisfying the rules
below.

\infrule[Out-barb]{y \in {\mathcal N}, \; x \nameeq y}
		  {\outputp{x}{v} \downarrow_{\mathcal N} x}
\infrule[Par-barb]{\mbox{$P\downarrow_{\mathcal N} x$ or $Q\downarrow_{\mathcal N} x$}}
		  {\binpar{P}{Q} \downarrow_{\mathcal N} x}

We write $P \Downarrow_{\mathcal N} x$ if there is $Q$ such that 
$P \wred Q$ and $Q \downarrow_{\mathcal N} x$.
\end{definition}

\begin{definition}
%\label{def.bbisim}
An  ${\mathcal N}$-\emph{barbed bisimulation} over a set of names, ${\mathcal N}$, is a symmetric binary relation 
${\mathcal S}_{\mathcal N}$ between agents such that $P\rel{S}_{\mathcal N}Q$ implies:
\begin{enumerate}
\item If $P \red P'$ then $Q \wred Q'$ and $P'\rel{S}_{\mathcal N} Q'$.
\item If $P\downarrow_{\mathcal N} x$, then $Q\Downarrow_{\mathcal N} x$.
\end{enumerate}
$P$ is ${\mathcal N}$-barbed bisimilar to $Q$, written
$P \wbbisim_{\mathcal N} Q$, if $P \rel{S}_{\mathcal N} Q$ for some ${\mathcal N}$-barbed bisimulation ${\mathcal S}_{\mathcal N}$.
\end{definition}

$\mathcal{R} \subseteq \pi \times \pi$

$P \mathcal{R} Q => \forall P'. P \red P' \Rightarrow \exists Q'. Q \red Q', P' \mathcal{R} Q'$

$P \vdash x \Rightarrow Q \vdash x$

\begin{mathpar}
  \inferrule*[lab=Out-barb]{x \nameeq y}{{y}!\langle{Q}\rangle \vdash x}
  \and
  \inferrule*[lab=Par-barb]{\mbox{$P\vdash x$ or $Q\vdash x$}}{\binpar{P}{Q} \vdash x}
\end{mathpar}

\subsubsection{Contexts}

One of the principle advantages of computational calculi like the
$\pi$-calculus is a well-defined notion of context,
contextual-equivalence and a correlation between
contextual-equivalence and notions of bisimulation. The notion of
context allows the decomposition of a process into (sub-)process and
its syntactic environment, its context. Thus, a context may be
thought of as a process with a ``hole'' (written $\Box$) in it. The
application of a context $M$ to a process $P$, written $M[P]$, is
tantamount to filling the hole in $M$ with $P$. In this paper we do
not need the full weight of this theory, but do make use of the notion
of context in the proof the main theorem. 

\begin{mathpar}
  \inferrule* [lab=summation] {} {{M_{M},M_{N}} \bc \Box \;|\; x.M_{A} \;|\; M_{M}+M_{N}}
  \and
  \inferrule* [lab=agent] {} {{M_{A}} \bc (\vec{x})M_{P} \;| \; \clift{P_0,\ldots,M_{P},\ldots,P_N}}
  \and \\
  \inferrule* [lab=process] {} {{M_{P}} \bc M_{N} \;| \;P|M_{P} }
\end{mathpar} 

\begin{mathpar}
  \inferrule* [lab=sychronization] {} {M_{N} \bc \Box \;|\; x?M_{F} \;|\; x!M_{C}}
  \and
  \inferrule* [lab=abstraction] {} {{M_{F}} \bc (x)M_{P} }
  \and
  \inferrule* [lab=concretion] {} {{M_{C}} \bc \langle M_{P} \rangle }
  \and \\
  \inferrule* [lab=process] {} {{M_{P}} \bc M_{N} \;| \;P|M_{P} }
\end{mathpar}

\begin{definition}[contextual application] Given a context $M$, and
  process $P$, we define the \emph{contextual application}, $M[P] :=
  M\{P/\Box\}$. That is, the contextual application of M to P is the
  substitution of $P$ for $\Box$ in $M$.
\end{definition}

$\meaningof{-} : L \to \mathcal{P}(\pi)$

\begin{mathpar}
  \inferrule* [lab=collection] {} {\meaningof{true} = \pi, \and \meaningof{~E} = \pi \setminus \meaningof{E}, \and \meaningof{E_{1} \& E_{2}} = \meaningof{E_{1}} \cap \meaningof{E_{2}}}
\end{mathpar}

\begin{mathpar}
  \inferrule* [lab=structure] {} {\meaningof{0} = \{ P \in \pi | P \equiv 0 \}, \and \\ \meaningof{E_1 | E_2} = \{ P \in \pi | P \equiv P_{1} | P_{2}, P_{1} \in \meaningof{E_{1}}, P_{2} \in \meaningof{E_2}\} }
\end{mathpar}

\begin{mathpar}
 \inferrule* [lab=behavior] {} {\meaningof{\langle a?b \rangle E} = \{ P \in \pi | P \equiv Q | u?(y)P', \\ \and \\\\ \and \\ \;\;\; u \in \meaningof{a}, \forall z.P'\{z/y\} \in \meaningof{E\{z/b\}}\}, \and \\ \meaningof{a!E} = \{ P \in \pi | P \equiv Q | x!\langle P' \rangle, x \in \meaningof{a} P' \in \meaningof{E}\} }
\end{mathpar}

\begin{mathpar}
 \inferrule* [lab=nominal] {} {\meaningof{\quotep{E}} = \{ \quotep{P} \in \quotep{\pi} | P \in \meaningof{E} \}, \and \meaningof{\quotep{P}} = \{ \quotep{Q} \in \quotep{\pi} | P \equiv Q \} \and \\ \meaningof{@\quotep{E}} = \{ P \in \pi | P \equiv @x, x \in \meaningof{E} \}}
\end{mathpar}

\begin{eqnarray*}
  \\
  \meaningof{-} : TS \to ST
\end{eqnarray*}

\begin{eqnarray*}
  \\
  L : TS \to ST
\end{eqnarray*}

\begin{eqnarray*}
  \\
  P \models E \iff P \in \meaningof{E}
\end{eqnarray*}

\begin{eqnarray*}
  P \approx_{L} Q \iff \forall E \in L. P \models E \iff Q \models E
\end{eqnarray*}

\begin{eqnarray*}
  P \approx_{K} Q
\end{eqnarray*}

\begin{eqnarray*}
  P \approx Q
\end{eqnarray*}

$\approx_{K} = \approx = \approx_{L}$

\subsubsection{Contextual duality}

Note that contexts extend the quotation operation to a family of
operations from processes to names. Given a context, $M$, we can
define a \emph{nominal context}, $\quotep{M}$ by $\quotep{M}[P] :=
\quotep{M[P]}$. To foreshadow what is to come we observe that these
operations enjoy a duality with processes very much like the duality
between vectors and maps from vectors to scalars.

Further, because the calculus is essentially higher-order, we have a
correspondence between contexts and processes. More specifically,
given a name $x$ and a context $M$ we can construct $M^{*}_{x}$ such
that 

\begin{mathpar}
  M^{*}_{x} | \lift{x}{P} \red M[P]
\end{mathpar}

namely,

\begin{mathpar}
  M^{*}_{x} := x?(u).M[\dropn{u}]
\end{mathpar}

The dependence of $M^{*}_{x}$ on a name makes it an abstraction, 

\begin{mathpar}
  M^{*} := (x)x?(u).M[\dropn{u}]
\end{mathpar}

\subsection{Additional notation}

It will sometimes be convenient to denote the process a name
quotes. We already have the notation $x = \quotep{P}$, but it will be
convenient to introduce an alternate notation, $\procn{x}$, when we
want to emphasize the connection to the use of the name. Note that, by
virtue of name equivalence, $\quotep{\procn{x}} \nameeq x$; so, the
notation is consistent with previous definitions.

Further, because names have structure it is possible to effect
substitutions on the basis of that structure. This means we need to
upgrade our notation for substitutions, which we accomplish by
adapting comprehension notation. Thus,

\begin{mathpar}
  P\{ y / x : x \in S \}
\end{mathpar}

is interpreted to mean the process derived from P by replacing (in a
capture-avoiding manner) each occurrence of $x$ in $S$ by $y$. For example,

\begin{mathpar}
  P\{ \quotep{\procn{x}|\procn{x}} / x : x \in \freenames{P} \}
\end{mathpar}

will replace each (occurrence) of a free name $x$ in $P$ by
$\quotep{\procn{x}|\procn{x}}$.

Also, we will avail ourselves of the notation $x^{L}$ and $x^{R}$ to
denote injections of a name into disjoint copies of the name
space. There are numerous ways to accomplish this. One example can be
found in \cite{MeredithR05}. This notation overloads to vectors of
names: $\vec{x}^{\pi} := (x_{i}^{\pi} \; : \; 0 \leq i < |\vec{x}| )$ where $\pi \in \{L,R\}$.

We also use $P^{\Box} := P|\Box$.

In \cite{MeredithR05} an interpretation of the new operator is
given. It turns out that there are several possible interpretations
all enjoying the requisite algebraic properties of the operator (see
\cite{milner91polyadicpi}). We will therefore make liberal use of
$(\nu\; \vec{x})P$.

% subsection the_syntax_and_semantics_of_the_notation_system (end)   

\input{qm2pi.qmops} 

\input{qm2pi.sterngerlach} 

\input{qm2pi.metric} 

% section concurrent_process_calculi (end)

%\input{qm2pi.proofsketch}

% section proof sketch (end)

%\input{qm2pi.slviaknots} 

% section spatial logic via knots (end)

\input{qm2pi.conclusion}

% section conclusion (end)

%\input{qm2pi.dtcodes} 

% section wiring algorithm (end)

\input{qm2pi.ack} 

% section acknowledgments (end)

\newpage


\bibliographystyle{plain}   
\bibliography{../../biblios/main.bib}

\input{qm2pi.rhodetails}

\end{document}

 

%\ifpdf
%\usepackage[pdftex]{graphicx}
%\else
%\usepackage{graphicx}
%\fi

 % \ifpdf
%  \usepackage{pdfsync}
%  \if


%\title{Brief Article}
%\author{David F. Snyder}
%\author{L.G. Meredith}

%\address{Dept. of Math., Texas State University--San Marcos, San Marcos, TX 78666}
       
\pagestyle{empty}


\begin{document}

\lstset{language=[Objective]Caml,frame=shadowbox}

\documentclass[12pt]{llncs}
%\documentclass{jktr}

\usepackage[pdftex]{hyperref}                   
\usepackage {listings}
\usepackage {mathpartir}
\usepackage{bcprules}
%\usepackage{listings}
                       
\usepackage{graphicx} 
%\usepackage[margins=2.5cm,nohead,nofoot]{geometry}
%\usepackage{geometry}
\usepackage{amsfonts}
\usepackage{amstext}
\usepackage{latexsym}
\usepackage{amssymb}
\usepackage{color}


%\include{myPreamble}
\include{qm2pi.local} 

%\ifpdf
%\usepackage[pdftex]{graphicx}
%\else
%\usepackage{graphicx}
%\fi

 % \ifpdf
%  \usepackage{pdfsync}
%  \if


%\title{Brief Article}
%\author{David F. Snyder}
%\author{L.G. Meredith}

%\address{Dept. of Math., Texas State University--San Marcos, San Marcos, TX 78666}
       
\pagestyle{empty}


\begin{document}

\lstset{language=[Objective]Caml,frame=shadowbox}

\input{qm2pi.front}

% section front matter (end)

\input{qm2pi.intro} 
 
% section introduction (end)

% \input{qm2pi.knotations} 

% section notation (end)

\input{qm2pi.process.calculi} 

% section concurrent_process_calculi_and_spatial_logics_ (end)
    
%\input{qm2pi.knots2pi} 

%\input{qm2pi.trefoil} 

%\input{qm2pi.mainthm} 

% subsection basic_interpretation (end)

%\input{qm2pi.rho.presentation} 
\subsection{The syntax and semantics of the notation system}\label{sub:the_syntax_and_semantics_of_the_notation_system} % (fold)

We now summarize a technical presentation of the calculus that
embodies our theory of dynamics. The typical presentation of such a
calculus follows the style of giving generators and relations on
them. The grammar, below, describing term constructors, freely
generates the set of processes, $\Proc$. This set is then quotiented
by a relation known as structural congruence and it is over this set
that the notion of dynamics is expressed. This presentation is
essentially that of \cite{MeredithR05} with the addition of
polyadicity and summation. For readability we have relegated some of
the technical subtleties to an appendix.

\subsubsection{Process grammar}\label{subsub:process_grammar}

\begin{mathpar}
  \inferrule* [lab=synchronization] {} {{M} \bc \pzero \;|\; x?F \;|\; x!C }
  \and
  \inferrule* [lab=abstraction] {} {{F} \bc (x)P}
  \and
  \inferrule* [lab=concretion] {} {{C} \bc \langle Q \rangle}
  \and
  \inferrule* [lab=process] {} {{P,Q} \bc M \;| \;P|Q \;|\; @{x}}
  \and
  \inferrule* [lab=name] {} {{x} \bc \quotep{P}}
\end{mathpar} 

Note that $\vec{x}$ (resp. $\vec{P}$) denotes a vector of names
(resp. processes) of length $|\vec{x}|$ (resp. $|\vec{P}|$). We adopt
the following useful abbreviations.

\begin{mathpar}
   x?(\vec{y}).P := x.(\vec{y})P \and  x\clift{\vec{P}} := x.\clift{\vec{P}}
   \and x!(y) := \lift{x}{\dropn{y}}
   \and \Pi_{i=0}^{n-1}P_i := P_0 | \ldots | P_{n-1}
\end{mathpar}

\subsubsection{Structural congruence}

\paragraph{Free and bound names and alpha-equivalence.} At the
core of structural equivalence is alpha-equivalence which identifies
process that are the same up to a change of variable. Formally, we
recognize the distinction between free and bound names. The free names
of a process, $\freenames{P}$, may be calculated recursively as
follows:

\begin{mathpar}
\freenames{\pzero} := \emptyset
  \and \\
  \freenames{x?(y).P} := \{ x \} \cup (\freenames{P} \setminus \{ y \})
  \and 
  \freenames{x!\langle P \rangle} := \{ x \} \cup \{ P \} 
  \and \\
  \freenames{P|Q} := \freenames{P} \cup \freenames{Q}
  \and \\
  \freenames{@{x}} := \{ x \}
\end{mathpar}

$\pi$
$\quotep{\pi}$

$\freenames{-} : \pi \to \mathcal{P}(\quotep{\pi})$

\begin{eqnarray*}
  \freenames{\pzero} & := & \emptyset \\
  \freenames{x?(y).P} & := & \{ x \} \cup (\freenames{P} \setminus \{ y \}) \\
  \freenames{x!\langle P \rangle} & := & \{ x \} \cup \{ P \} \\
  \freenames{P|Q} & := & \freenames{P} \cup \freenames{Q} \\
  \freenames{\dropn{x}} & := & \{ x \}
\end{eqnarray*}

The bound names of a process, $\boundnames{P}$, are those names occurring in $P$
that are not free. For example, in $x?(y).0$, the name $x$ is free, while $y$ is bound.

\begin{mathpar}
  \inferrule* [lab=monoidal-laws] {} { P|Q \equiv Q|P \and P|0 \equiv P \and P|(Q|R) \equiv (P|Q)|R }
\end{mathpar}

\begin{mathpar}
  \inferrule* [lab=alpha-equivalence] {} { (x)P \equiv (y)P\{y/x\} \and y \not\in \freenames{P} }
\end{mathpar}

\begin{definition}
Then two processes, $P,Q$, are alpha-equivalent if $P = Q\{\vec{y}/\vec{x}\}$ for
some $\vec{x} \in \boundnames{Q},\vec{y} \in \boundnames{P}$, where $Q\{\vec{y}/\vec{x}\}$
denotes the capture-avoiding substitution of $\vec{y}$ for $\vec{x}$ in $Q$.
\end{definition}

\begin{definition}
  The {\em structural congruence} \cite{SangiorgiWalker} , $\equiv$,
  between processes is the least congruence containing
  alpha-equivalence, satisfying the abelian monoid laws
  (associativity, commutativity and $\pzero$ as identity) for parallel
  composition $|$ and for summation $+$.
\end{definition}

\subsection{Name equivalence}

We take name equivalence, written $\nameeq$, to be the smallest
equivalence relation generated by the following rules.

\begin{mathpar}
\inferrule*[lab=Quote-drop]
{ }
{ \quotep{@{x}} \nameeq x }

\inferrule*[lab=Struct-equiv]
{ P \scong Q }
{ \quotep{P} \nameeq \quotep{Q} }
\end{mathpar}

The astute reader will have noticed that the mutual recursion of names
and processes imposes a mutual recursion on alpha-equivalence and
structural equivalence via name-equivalence. Fortunately, all of this
works out pleasantly and we may calculate in the natural way, free of
concern. The reader interested in the details is referred to the
appendix \ref{appendix:rho_details}.

\subsection{Substitution}

We use $\Proc$ for the set of processes, $\QProc$ for the set of
names, and $\id{\{}\vec{y} / \vec{x} \id{\}}$ to denote partial maps,
$s : \QProc \rightarrow \QProc$. A map, $s$ lifts, uniquely, to a map
on process terms, $\widehat{s} : \Proc \rightarrow \Proc$ by the
following equations.

\begin{mathpar}
  (0) \psubstp{Q}{P} := 0 \\
  (R \juxtap S) \psubstp{Q}{P}
  :=    
  (R)\psubstp{Q}{P} \juxtap (S) \psubstp{Q}{P} \\
  (x?(y).R) \psubstp{Q}{P}    
  :=    
  (x)\substp{Q}{P} (z)\concat( (R \psubstn{z}{y}) \psubstp{Q}{P} ) \\
  (\lift{x}{R}) \psubstp{Q}{P}  
  :=
  \lift{(x)\substp{Q}{P}}{ R \psubstp{Q}{P} } \\
%   (\dropn{x})  \psubstp{Q}{P}       
%   := 
%   \left\{ 
%     \begin{array}{ccc} 
%       \dropn{\quotep{Q}} & & x \nameeq \quotep{P} \\
%       \dropn{x} & & otherwise \\
%     \end{array}
%   \right. 
  (\dropn{x})  \psubstp{Q}{P}       
  := 
  \left\{ 
    \begin{array}{ccc} 
      Q & & x \nameeq \quotep{P} \\
      \dropn{x} & & otherwise \\
    \end{array}
  \right.
\end{mathpar}
 

where

\begin{eqnarray}
  (x)\id{\{} \lpquote Q \rpquote / \lpquote P \rpquote \id{\}}            = 
  \left\{ 
    \begin{array}{ccc}
      \lpquote Q \rpquote & & x \nameeq \lpquote P \rpquote \\
      x & & otherwise \\
    \end{array}
  \right. \nonumber
\end{eqnarray}

and $z$ is chosen distinct from $\quotep{P}$, $\quotep{Q}$, the free
names in $Q$, and all the names in $R$. Our $\alpha$-equivalence will
be built in the standard way from this substitution.

\begin{remark}\label{rem:no_self_referential_names}
  One consequence of these definitions is that $\forall P. \quotep{P}
  \not\in \freenames{P}$.
\end{remark}

\subsection{ Dynamic quote: an example }

Anticipating something of what's to come, consider applying the
substitution, $\widehat{\id{\{}u / z \id{\}}}$, to the following pair
of processes, $\lift{w}{y!(z)}$ and $w[ \lpquote y!(z) \rpquote ]$.

\begin{eqnarray}
	\lift{w}{y!(z)}\widehat{\id{\{}u / z \id{\}}}
		& = &
		\lift{w}{y!(u)} \nonumber\\
	w[ \lpquote y!(z) \rpquote ] \widehat{ \id{\{}u / z \id{\}} }
		& = &
		w[ \lpquote y!(z) \rpquote ] \nonumber
\end{eqnarray}

Because the body of the process between quotes is impervious to
substitution, we get radically different answers. In fact, by
examining the first process in an input context,
e.g. $x?(z).\lift{w}{y!(z)}$, we see that the process under the lift
operator may be shaped by prefixed inputs binding a name inside it. In
this sense, the lift operator will be seen as a way to dynamically
construct processes before reifying them as names.

Finally equipped with these standard features we can present the
dynamics of the calculus.

\subsubsection{Operational semantics} 

Finally, we introduce the computational dynamics. What marks these
algebras as distinct from other more traditionally studied algebraic
structures, e.g. vector spaces or polynomial rings, is the manner in
which dynamics is captured. In traditional structures, dynamics is typically
expressed through morphisms between such structures, as in linear maps
between vector spaces or morphisms between rings. In algebras
associated with the semantics of computation, the dynamics is
expressed as part of the algebraic structure itself, through a
reduction reduction relation typically denoted by $\red$. Below, we
give a recursive presentation of this relation for the calculus used
in the encoding.

$\red \subseteq \pi \times \pi$
$\red : \pi \to \mathcal{P}(\pi)$

\begin{mathpar}
  \inferrule* [lab=Comm] { \textsf{match}( x_{src}, x_{trgt} ) } { x_{trgt}?(y)P \; | \; x_{src}!\langle {Q} \rangle \red P\{\quotep{Q}/y}\} }
  \and \\
  \inferrule* [lab=Par] {{P} \red {P}'} {{{P} | {Q}} \red {{P}' | {Q}}}
  \and
  \inferrule* [lab=Equiv]{{{P} \scong {P}'} \andalso {{P}' \red {Q}'} \andalso {{Q}' \scong {Q}}}{{P} \red {Q}}
\end{mathpar}

\begin{eqnarray*}
  match_{\equiv} (\quotep{P},\quotep{Q}) & := & P \equiv Q \\
  match_{\dagger}(\quotep{P},\quotep{Q}) & := & \forall R. P|Q \red^{*} R => R \red^{*} 0 \\
  match_{K}(\quotep{P},\quotep{Q}) & := & K \mbox{ for some context } K
\end{eqnarray*}

$u?(x)P | u!\langle Q \rangle \red P\{\quotep{Q}/x\}$

%We write $\wred$ for $\red^*$, and $P\red$ if $\exists Q $ such that $ P \red Q$.
We write $P\red$ if $\exists Q $ such that $ P \red Q$ and $P\not\red$, otherwise.

\section{Replication}

As mentioned before, it is known that replication (and hence
recursion) can be implemented in a higher-order process algebra
\cite{SangiorgiWalker}. As our first example of calculation with the
machinery thus far presented we give the construction explicitly in
the {\rhoc}.

\begin{eqnarray}
	D_{x} & := & \prefix{x}{y}{(\binpar{\outputp{x}{y}}{@{y}})} \nonumber\\
	\bangp_{x}{P} & := & \binpar{{x}!\langle{\binpar{D_{x}}{P}}\rangle}{D_{x}} \nonumber
\end{eqnarray}

\begin{eqnarray}
	\bangp_{x}{P} & & \nonumber\\
	=
	& {x}!\langle{(\prefix{x}{y}{(\outputp{x}{y} | @{y})) | P}}\rangle 
	      | \prefix{x}{y}{(\outputp{x}{y} | @{y})} & \nonumber\\
	\red
	& (\outputp{x}{y} | @{y})\substn{\quotep{(\prefix{x}{y}{(@{y} | \outputp{x}{y})) | P}}}{y} & \nonumber\\
	=
	& \outputp{x}{\quotep{(\prefix{x}{y}{(\outputp{x}{y} | @{y})) | P}}}
	  | {(\prefix{x}{y}{(\outputp{x}{y} | @{y})) | P}} & \nonumber\\
	\red
	& \ldots & \nonumber\\
	\red^*
	& P | P | \ldots & \nonumber
\end{eqnarray}

Of course, this encoding, as an implementation, runs away, unfolding
$\bangp{P}$ eagerly. A lazier and more implementable replication
operator, restricted to input-guarded processes, may be obtained as follows.

\begin{eqnarray}
\bangp{\prefix{u}{v}{P}} 
	:= 
	\binpar{\lift{x}{\prefix{u}{v}{(\binpar{D(x)}{P})}}}{D(x)} \nonumber
\end{eqnarray}

\begin{remark}
  Note that the lazier definition still does not deal with summation
  or mixed summation (i.e. sums over input and output). The reader is
  invited to construct definitions of replication that deal with these
  features. 

  Further, the definitions are parameterized in a name, $x$. Can you,
  gentle reader, make a definition that eliminates this parameter and
  guarantees no accidental interaction between the replication
  machinery and the process being replicated -- i.e. no accidental
  sharing of names used by the process to get its work done and the
  name(s) used by the replication to effect copying. This latter
  revision of the definition of replication is crucial to obtaining
  the expected identity $!!P \sim !P$.
\end{remark}

\begin{remark}\label{rem:paradoxical_combinator}
  The reader familiar with the lambda calculus will have noticed the
  similarity between $D$ and the paradoxical combinator.

  [Ed. note: the existence of this seems to suggest we have to be more
  restrictive on the set of processes and names we admit if we are to
  support no-cloning.]
\end{remark}

\subsubsection{Bisimulation}

The computational dynamics gives rise to another kind of equivalence,
the equivalence of computational behavior. As previously mentioned
this is typically captured \emph{via} some form of bisimulation.

% The notion we use in this paper is weak barbed bisimulation
% \cite{milner91polyadicpi}.

The notion we use in this paper is derived from weak barbed
bisimulation \cite{milner91polyadicpi}. 

\begin{definition}
An \emph{observation relation}, $\downarrow_{\mathcal N}$, over a set
of names, $\mathcal N$, is the smallest relation satisfying the rules
below.

\infrule[Out-barb]{y \in {\mathcal N}, \; x \nameeq y}
		  {\outputp{x}{v} \downarrow_{\mathcal N} x}
\infrule[Par-barb]{\mbox{$P\downarrow_{\mathcal N} x$ or $Q\downarrow_{\mathcal N} x$}}
		  {\binpar{P}{Q} \downarrow_{\mathcal N} x}

We write $P \Downarrow_{\mathcal N} x$ if there is $Q$ such that 
$P \wred Q$ and $Q \downarrow_{\mathcal N} x$.
\end{definition}

\begin{definition}
%\label{def.bbisim}
An  ${\mathcal N}$-\emph{barbed bisimulation} over a set of names, ${\mathcal N}$, is a symmetric binary relation 
${\mathcal S}_{\mathcal N}$ between agents such that $P\rel{S}_{\mathcal N}Q$ implies:
\begin{enumerate}
\item If $P \red P'$ then $Q \wred Q'$ and $P'\rel{S}_{\mathcal N} Q'$.
\item If $P\downarrow_{\mathcal N} x$, then $Q\Downarrow_{\mathcal N} x$.
\end{enumerate}
$P$ is ${\mathcal N}$-barbed bisimilar to $Q$, written
$P \wbbisim_{\mathcal N} Q$, if $P \rel{S}_{\mathcal N} Q$ for some ${\mathcal N}$-barbed bisimulation ${\mathcal S}_{\mathcal N}$.
\end{definition}

$\mathcal{R} \subseteq \pi \times \pi$

$P \mathcal{R} Q => \forall P'. P \red P' \Rightarrow \exists Q'. Q \red Q', P' \mathcal{R} Q'$

$P \vdash x \Rightarrow Q \vdash x$

\begin{mathpar}
  \inferrule*[lab=Out-barb]{x \nameeq y}{{y}!\langle{Q}\rangle \vdash x}
  \and
  \inferrule*[lab=Par-barb]{\mbox{$P\vdash x$ or $Q\vdash x$}}{\binpar{P}{Q} \vdash x}
\end{mathpar}

\subsubsection{Contexts}

One of the principle advantages of computational calculi like the
$\pi$-calculus is a well-defined notion of context,
contextual-equivalence and a correlation between
contextual-equivalence and notions of bisimulation. The notion of
context allows the decomposition of a process into (sub-)process and
its syntactic environment, its context. Thus, a context may be
thought of as a process with a ``hole'' (written $\Box$) in it. The
application of a context $M$ to a process $P$, written $M[P]$, is
tantamount to filling the hole in $M$ with $P$. In this paper we do
not need the full weight of this theory, but do make use of the notion
of context in the proof the main theorem. 

\begin{mathpar}
  \inferrule* [lab=summation] {} {{M_{M},M_{N}} \bc \Box \;|\; x.M_{A} \;|\; M_{M}+M_{N}}
  \and
  \inferrule* [lab=agent] {} {{M_{A}} \bc (\vec{x})M_{P} \;| \; \clift{P_0,\ldots,M_{P},\ldots,P_N}}
  \and \\
  \inferrule* [lab=process] {} {{M_{P}} \bc M_{N} \;| \;P|M_{P} }
\end{mathpar} 

\begin{mathpar}
  \inferrule* [lab=sychronization] {} {M_{N} \bc \Box \;|\; x?M_{F} \;|\; x!M_{C}}
  \and
  \inferrule* [lab=abstraction] {} {{M_{F}} \bc (x)M_{P} }
  \and
  \inferrule* [lab=concretion] {} {{M_{C}} \bc \langle M_{P} \rangle }
  \and \\
  \inferrule* [lab=process] {} {{M_{P}} \bc M_{N} \;| \;P|M_{P} }
\end{mathpar}

\begin{definition}[contextual application] Given a context $M$, and
  process $P$, we define the \emph{contextual application}, $M[P] :=
  M\{P/\Box\}$. That is, the contextual application of M to P is the
  substitution of $P$ for $\Box$ in $M$.
\end{definition}

$\meaningof{-} : L \to \mathcal{P}(\pi)$

\begin{mathpar}
  \inferrule* [lab=collection] {} {\meaningof{true} = \pi, \and \meaningof{~E} = \pi \setminus \meaningof{E}, \and \meaningof{E_{1} \& E_{2}} = \meaningof{E_{1}} \cap \meaningof{E_{2}}}
\end{mathpar}

\begin{mathpar}
  \inferrule* [lab=structure] {} {\meaningof{0} = \{ P \in \pi | P \equiv 0 \}, \and \\ \meaningof{E_1 | E_2} = \{ P \in \pi | P \equiv P_{1} | P_{2}, P_{1} \in \meaningof{E_{1}}, P_{2} \in \meaningof{E_2}\} }
\end{mathpar}

\begin{mathpar}
 \inferrule* [lab=behavior] {} {\meaningof{\langle a?b \rangle E} = \{ P \in \pi | P \equiv Q | u?(y)P', \\ \and \\\\ \and \\ \;\;\; u \in \meaningof{a}, \forall z.P'\{z/y\} \in \meaningof{E\{z/b\}}\}, \and \\ \meaningof{a!E} = \{ P \in \pi | P \equiv Q | x!\langle P' \rangle, x \in \meaningof{a} P' \in \meaningof{E}\} }
\end{mathpar}

\begin{mathpar}
 \inferrule* [lab=nominal] {} {\meaningof{\quotep{E}} = \{ \quotep{P} \in \quotep{\pi} | P \in \meaningof{E} \}, \and \meaningof{\quotep{P}} = \{ \quotep{Q} \in \quotep{\pi} | P \equiv Q \} \and \\ \meaningof{@\quotep{E}} = \{ P \in \pi | P \equiv @x, x \in \meaningof{E} \}}
\end{mathpar}

\begin{eqnarray*}
  \\
  \meaningof{-} : TS \to ST
\end{eqnarray*}

\begin{eqnarray*}
  \\
  L : TS \to ST
\end{eqnarray*}

\begin{eqnarray*}
  \\
  P \models E \iff P \in \meaningof{E}
\end{eqnarray*}

\begin{eqnarray*}
  P \approx_{L} Q \iff \forall E \in L. P \models E \iff Q \models E
\end{eqnarray*}

\begin{eqnarray*}
  P \approx_{K} Q
\end{eqnarray*}

\begin{eqnarray*}
  P \approx Q
\end{eqnarray*}

$\approx_{K} = \approx = \approx_{L}$

\subsubsection{Contextual duality}

Note that contexts extend the quotation operation to a family of
operations from processes to names. Given a context, $M$, we can
define a \emph{nominal context}, $\quotep{M}$ by $\quotep{M}[P] :=
\quotep{M[P]}$. To foreshadow what is to come we observe that these
operations enjoy a duality with processes very much like the duality
between vectors and maps from vectors to scalars.

Further, because the calculus is essentially higher-order, we have a
correspondence between contexts and processes. More specifically,
given a name $x$ and a context $M$ we can construct $M^{*}_{x}$ such
that 

\begin{mathpar}
  M^{*}_{x} | \lift{x}{P} \red M[P]
\end{mathpar}

namely,

\begin{mathpar}
  M^{*}_{x} := x?(u).M[\dropn{u}]
\end{mathpar}

The dependence of $M^{*}_{x}$ on a name makes it an abstraction, 

\begin{mathpar}
  M^{*} := (x)x?(u).M[\dropn{u}]
\end{mathpar}

\subsection{Additional notation}

It will sometimes be convenient to denote the process a name
quotes. We already have the notation $x = \quotep{P}$, but it will be
convenient to introduce an alternate notation, $\procn{x}$, when we
want to emphasize the connection to the use of the name. Note that, by
virtue of name equivalence, $\quotep{\procn{x}} \nameeq x$; so, the
notation is consistent with previous definitions.

Further, because names have structure it is possible to effect
substitutions on the basis of that structure. This means we need to
upgrade our notation for substitutions, which we accomplish by
adapting comprehension notation. Thus,

\begin{mathpar}
  P\{ y / x : x \in S \}
\end{mathpar}

is interpreted to mean the process derived from P by replacing (in a
capture-avoiding manner) each occurrence of $x$ in $S$ by $y$. For example,

\begin{mathpar}
  P\{ \quotep{\procn{x}|\procn{x}} / x : x \in \freenames{P} \}
\end{mathpar}

will replace each (occurrence) of a free name $x$ in $P$ by
$\quotep{\procn{x}|\procn{x}}$.

Also, we will avail ourselves of the notation $x^{L}$ and $x^{R}$ to
denote injections of a name into disjoint copies of the name
space. There are numerous ways to accomplish this. One example can be
found in \cite{MeredithR05}. This notation overloads to vectors of
names: $\vec{x}^{\pi} := (x_{i}^{\pi} \; : \; 0 \leq i < |\vec{x}| )$ where $\pi \in \{L,R\}$.

We also use $P^{\Box} := P|\Box$.

In \cite{MeredithR05} an interpretation of the new operator is
given. It turns out that there are several possible interpretations
all enjoying the requisite algebraic properties of the operator (see
\cite{milner91polyadicpi}). We will therefore make liberal use of
$(\nu\; \vec{x})P$.

% subsection the_syntax_and_semantics_of_the_notation_system (end)   

\input{qm2pi.qmops} 

\input{qm2pi.sterngerlach} 

\input{qm2pi.metric} 

% section concurrent_process_calculi (end)

%\input{qm2pi.proofsketch}

% section proof sketch (end)

%\input{qm2pi.slviaknots} 

% section spatial logic via knots (end)

\input{qm2pi.conclusion}

% section conclusion (end)

%\input{qm2pi.dtcodes} 

% section wiring algorithm (end)

\input{qm2pi.ack} 

% section acknowledgments (end)

\newpage


\bibliographystyle{plain}   
\bibliography{../../biblios/main.bib}

\input{qm2pi.rhodetails}

\end{document}



% section front matter (end)

\section{Introduction}\label{sec:introduction} % (fold)
In this draft of the material i am going to have to dispense with the
usual writing conventions adopted in papers on these topics. i'm going
to have adopt whatever tone i need at the time i'm writing up the
calculations. Sometimes this may be very conversational; others it may
be the barest mathematical grunts; others still it may be that i have
lifted text from one of my other papers because the exposition of some
point was better said there. i hope that my readers are not unduly put
out by this decision. i'm not doing this to flout convention or be
rebellious. i find these calculations very technically challenging. To
keep everything going technically, something has to give; i have to
let go of some cognitive burden. So, the academic writing style --
with all of its trade-offs in terms of facilitating technical
communication -- is what i'm letting go of. Perhaps subsequent drafts
can be tightened and polished, but for now, i'm going to speak as if
we were sitting together in a coffee shop with a laptop, wifi and a
pad of paper and a pencil.

So, here's what i have to say. We -- you and i, comfortably ensconced
in our coffee shop and well-equipped with our tools -- can realize and
carry out the calculations of quantum mechanics over a very different
formal theory of dynamics, a formal theory of dynamics that
corresponds to a theory of concurrent computation with
\emph{reflection}. It has the advantage that the underlying theory is
already `quantized', but supports analogues all of the continuuous
operations. Strikingly, this underlying theory has recently been
connected with a notion of metric that we can show, by calculating
together, coincides with the metric induced by the inner product.

There are a lot of reasons why you might be interested in seeing
calculations of this form. Here's why i'm interested. For the past
several centuries there has been no competitor to the ``Newtonian''
account of dynamics. As a result the predominant share of accounts of
dynamical systems and situations have had to be formulated in terms of
the Newtonian machinery. i view this as an intellectually dangerous
position to occupy. Everything, despite it's intrinsic shape, turns
into a nail to be hit with this hammer. Recently, however, the theory
of computation has matured to the point where we have candidates for
theories of dynamics that offer very different perspective on
reasoning about dynamical systems and situations. Testing these
candidates against very successful accounts of dynamical situations,
like quantum mechanics, is going to give us some sense of how mature
they are and some measure of the quality of these accounts of
dynamics.

\subsection{Summary of contributions and outline of paper}

So, we're going to develop an interpretation of the operations of
quantum mechanics normally interpreted by Hilbert spaces and
operators. We're going to do this over a theory of computation. Note
that this is very different than the usual quantum computation program
which develops notions of computation over quantum mechanics. Rather,
we are developing a story that aligns with Wheeler's slogan: It from
Bit. To do this we will first provide an account of the theory of
computation at play here. Then we will dive into a calculation-driven
interpretation of the operations of quantum mechanics.

The reason we take this approach is that -- until very recently --
there hasn't been an axiomatic account of quantum mechanics. As a
result there has been no sharp delineation of the mathematical theory
supporting interpretation of the physical theory and the physical
theory, itself. So, ambient features of the maths are free to be
exploited (or supressed) without a real accounting of their physical
relevance. There is no sharp statement ``here's the physical theory''
qua \emph{theory} and ``here's the mathematical interpretation''
enabling a judgment of how faithful the interpretation is -- apart
from experimental observation. When there is an axiomatic account we
can judge how well a given mathematical formalism supports an
interpretation of the axioms, independent of
experimentation. Likewise, we can judge how well we have captured our
physical evidence and experience with our axiomatics, independent of
any specific mathematical implementation, with accidental detail that
may or may not have physical significance. 

In lieu of a fully fleshed out and vetted axiomatic account of quantum
mechanics, interpreting the operational notions in service of modeling
physical systems will have to suffice. In other words, we are not in
the business of providing a model of Hilbert spaces and operators. We
are in the business of providing a model of quantum mechanics because
we are motivated by testing our notions of dynamics against physical
theory; and, the predictive calculations of the physical theory must
serve as the best formulation -- shy of a fully fleshed out axiomatic
account -- of the physical theory itself (as they have for scientific
theories since time immemorial). Put another way, despite a
whole-hearted commitment to an It-from-Bit ontology, we are firmly
aligned with the shut-up-and-calculate camp as the best way to obtain
results either from the physical perspective or as a quality assurance
measure of our fledgling theory of dynamics.

In detail, we present a reflective process calculus. Then we develop
intuitive correspondences between the notions available in this
calculus and the usual physical notions supporting quantum mechanical
calculations. Thus, 

\begin{table}[htp]
  \center{
    \fbox{
      \begin{tabular}{c|c}
        quantum mechanics & process calculus \\
        \hline
        scalar & name \\
        state vector & process \\
        dual & contextual duals \\
        matrix & formal sums of process-context-dual pairs \\
        orthogonality & process annihilation \\
        inner product & execution-formula + quoting
      \end{tabular}
    }
  }
  \caption{QM - process calculi correspondences}
\end{table}

Then we tighten up these intuitions to operational definitions. We
employ the Dirac notation as the best proxy we can find for an
abstract syntax of the quantum mechanical notions. The definitions we
develop put us in contact with equational constraints coming from the
theory that we demonstrate the definitions and calculations satisfy.

This puts us in a position to shut up and calculate for the
Stern-Gerlach experimental set up, showing how these predictive
calculations become calculations on processes in our theory of a
reflective process calculus.

Penultimately, we demonstrate that the notion of metric coming from
the inner product coincides with the notion of metric available from
the theory of bisimulation. This demonstration gives us the right to
think of space as arising from behavior. Finally, we consider where we
might go from the new vantage point we have obtained.

% section introduction (end) 
 
% section introduction (end)

% \documentclass[12pt]{llncs}
%\documentclass{jktr}

\usepackage[pdftex]{hyperref}                   
\usepackage {listings}
\usepackage {mathpartir}
\usepackage{bcprules}
%\usepackage{listings}
                       
\usepackage{graphicx} 
%\usepackage[margins=2.5cm,nohead,nofoot]{geometry}
%\usepackage{geometry}
\usepackage{amsfonts}
\usepackage{amstext}
\usepackage{latexsym}
\usepackage{amssymb}
\usepackage{color}


%\include{myPreamble}
\include{qm2pi.local} 

%\ifpdf
%\usepackage[pdftex]{graphicx}
%\else
%\usepackage{graphicx}
%\fi

 % \ifpdf
%  \usepackage{pdfsync}
%  \if


%\title{Brief Article}
%\author{David F. Snyder}
%\author{L.G. Meredith}

%\address{Dept. of Math., Texas State University--San Marcos, San Marcos, TX 78666}
       
\pagestyle{empty}


\begin{document}

\lstset{language=[Objective]Caml,frame=shadowbox}

\input{qm2pi.front}

% section front matter (end)

\input{qm2pi.intro} 
 
% section introduction (end)

% \input{qm2pi.knotations} 

% section notation (end)

\input{qm2pi.process.calculi} 

% section concurrent_process_calculi_and_spatial_logics_ (end)
    
%\input{qm2pi.knots2pi} 

%\input{qm2pi.trefoil} 

%\input{qm2pi.mainthm} 

% subsection basic_interpretation (end)

%\input{qm2pi.rho.presentation} 
\subsection{The syntax and semantics of the notation system}\label{sub:the_syntax_and_semantics_of_the_notation_system} % (fold)

We now summarize a technical presentation of the calculus that
embodies our theory of dynamics. The typical presentation of such a
calculus follows the style of giving generators and relations on
them. The grammar, below, describing term constructors, freely
generates the set of processes, $\Proc$. This set is then quotiented
by a relation known as structural congruence and it is over this set
that the notion of dynamics is expressed. This presentation is
essentially that of \cite{MeredithR05} with the addition of
polyadicity and summation. For readability we have relegated some of
the technical subtleties to an appendix.

\subsubsection{Process grammar}\label{subsub:process_grammar}

\begin{mathpar}
  \inferrule* [lab=synchronization] {} {{M} \bc \pzero \;|\; x?F \;|\; x!C }
  \and
  \inferrule* [lab=abstraction] {} {{F} \bc (x)P}
  \and
  \inferrule* [lab=concretion] {} {{C} \bc \langle Q \rangle}
  \and
  \inferrule* [lab=process] {} {{P,Q} \bc M \;| \;P|Q \;|\; @{x}}
  \and
  \inferrule* [lab=name] {} {{x} \bc \quotep{P}}
\end{mathpar} 

Note that $\vec{x}$ (resp. $\vec{P}$) denotes a vector of names
(resp. processes) of length $|\vec{x}|$ (resp. $|\vec{P}|$). We adopt
the following useful abbreviations.

\begin{mathpar}
   x?(\vec{y}).P := x.(\vec{y})P \and  x\clift{\vec{P}} := x.\clift{\vec{P}}
   \and x!(y) := \lift{x}{\dropn{y}}
   \and \Pi_{i=0}^{n-1}P_i := P_0 | \ldots | P_{n-1}
\end{mathpar}

\subsubsection{Structural congruence}

\paragraph{Free and bound names and alpha-equivalence.} At the
core of structural equivalence is alpha-equivalence which identifies
process that are the same up to a change of variable. Formally, we
recognize the distinction between free and bound names. The free names
of a process, $\freenames{P}$, may be calculated recursively as
follows:

\begin{mathpar}
\freenames{\pzero} := \emptyset
  \and \\
  \freenames{x?(y).P} := \{ x \} \cup (\freenames{P} \setminus \{ y \})
  \and 
  \freenames{x!\langle P \rangle} := \{ x \} \cup \{ P \} 
  \and \\
  \freenames{P|Q} := \freenames{P} \cup \freenames{Q}
  \and \\
  \freenames{@{x}} := \{ x \}
\end{mathpar}

$\pi$
$\quotep{\pi}$

$\freenames{-} : \pi \to \mathcal{P}(\quotep{\pi})$

\begin{eqnarray*}
  \freenames{\pzero} & := & \emptyset \\
  \freenames{x?(y).P} & := & \{ x \} \cup (\freenames{P} \setminus \{ y \}) \\
  \freenames{x!\langle P \rangle} & := & \{ x \} \cup \{ P \} \\
  \freenames{P|Q} & := & \freenames{P} \cup \freenames{Q} \\
  \freenames{\dropn{x}} & := & \{ x \}
\end{eqnarray*}

The bound names of a process, $\boundnames{P}$, are those names occurring in $P$
that are not free. For example, in $x?(y).0$, the name $x$ is free, while $y$ is bound.

\begin{mathpar}
  \inferrule* [lab=monoidal-laws] {} { P|Q \equiv Q|P \and P|0 \equiv P \and P|(Q|R) \equiv (P|Q)|R }
\end{mathpar}

\begin{mathpar}
  \inferrule* [lab=alpha-equivalence] {} { (x)P \equiv (y)P\{y/x\} \and y \not\in \freenames{P} }
\end{mathpar}

\begin{definition}
Then two processes, $P,Q$, are alpha-equivalent if $P = Q\{\vec{y}/\vec{x}\}$ for
some $\vec{x} \in \boundnames{Q},\vec{y} \in \boundnames{P}$, where $Q\{\vec{y}/\vec{x}\}$
denotes the capture-avoiding substitution of $\vec{y}$ for $\vec{x}$ in $Q$.
\end{definition}

\begin{definition}
  The {\em structural congruence} \cite{SangiorgiWalker} , $\equiv$,
  between processes is the least congruence containing
  alpha-equivalence, satisfying the abelian monoid laws
  (associativity, commutativity and $\pzero$ as identity) for parallel
  composition $|$ and for summation $+$.
\end{definition}

\subsection{Name equivalence}

We take name equivalence, written $\nameeq$, to be the smallest
equivalence relation generated by the following rules.

\begin{mathpar}
\inferrule*[lab=Quote-drop]
{ }
{ \quotep{@{x}} \nameeq x }

\inferrule*[lab=Struct-equiv]
{ P \scong Q }
{ \quotep{P} \nameeq \quotep{Q} }
\end{mathpar}

The astute reader will have noticed that the mutual recursion of names
and processes imposes a mutual recursion on alpha-equivalence and
structural equivalence via name-equivalence. Fortunately, all of this
works out pleasantly and we may calculate in the natural way, free of
concern. The reader interested in the details is referred to the
appendix \ref{appendix:rho_details}.

\subsection{Substitution}

We use $\Proc$ for the set of processes, $\QProc$ for the set of
names, and $\id{\{}\vec{y} / \vec{x} \id{\}}$ to denote partial maps,
$s : \QProc \rightarrow \QProc$. A map, $s$ lifts, uniquely, to a map
on process terms, $\widehat{s} : \Proc \rightarrow \Proc$ by the
following equations.

\begin{mathpar}
  (0) \psubstp{Q}{P} := 0 \\
  (R \juxtap S) \psubstp{Q}{P}
  :=    
  (R)\psubstp{Q}{P} \juxtap (S) \psubstp{Q}{P} \\
  (x?(y).R) \psubstp{Q}{P}    
  :=    
  (x)\substp{Q}{P} (z)\concat( (R \psubstn{z}{y}) \psubstp{Q}{P} ) \\
  (\lift{x}{R}) \psubstp{Q}{P}  
  :=
  \lift{(x)\substp{Q}{P}}{ R \psubstp{Q}{P} } \\
%   (\dropn{x})  \psubstp{Q}{P}       
%   := 
%   \left\{ 
%     \begin{array}{ccc} 
%       \dropn{\quotep{Q}} & & x \nameeq \quotep{P} \\
%       \dropn{x} & & otherwise \\
%     \end{array}
%   \right. 
  (\dropn{x})  \psubstp{Q}{P}       
  := 
  \left\{ 
    \begin{array}{ccc} 
      Q & & x \nameeq \quotep{P} \\
      \dropn{x} & & otherwise \\
    \end{array}
  \right.
\end{mathpar}
 

where

\begin{eqnarray}
  (x)\id{\{} \lpquote Q \rpquote / \lpquote P \rpquote \id{\}}            = 
  \left\{ 
    \begin{array}{ccc}
      \lpquote Q \rpquote & & x \nameeq \lpquote P \rpquote \\
      x & & otherwise \\
    \end{array}
  \right. \nonumber
\end{eqnarray}

and $z$ is chosen distinct from $\quotep{P}$, $\quotep{Q}$, the free
names in $Q$, and all the names in $R$. Our $\alpha$-equivalence will
be built in the standard way from this substitution.

\begin{remark}\label{rem:no_self_referential_names}
  One consequence of these definitions is that $\forall P. \quotep{P}
  \not\in \freenames{P}$.
\end{remark}

\subsection{ Dynamic quote: an example }

Anticipating something of what's to come, consider applying the
substitution, $\widehat{\id{\{}u / z \id{\}}}$, to the following pair
of processes, $\lift{w}{y!(z)}$ and $w[ \lpquote y!(z) \rpquote ]$.

\begin{eqnarray}
	\lift{w}{y!(z)}\widehat{\id{\{}u / z \id{\}}}
		& = &
		\lift{w}{y!(u)} \nonumber\\
	w[ \lpquote y!(z) \rpquote ] \widehat{ \id{\{}u / z \id{\}} }
		& = &
		w[ \lpquote y!(z) \rpquote ] \nonumber
\end{eqnarray}

Because the body of the process between quotes is impervious to
substitution, we get radically different answers. In fact, by
examining the first process in an input context,
e.g. $x?(z).\lift{w}{y!(z)}$, we see that the process under the lift
operator may be shaped by prefixed inputs binding a name inside it. In
this sense, the lift operator will be seen as a way to dynamically
construct processes before reifying them as names.

Finally equipped with these standard features we can present the
dynamics of the calculus.

\subsubsection{Operational semantics} 

Finally, we introduce the computational dynamics. What marks these
algebras as distinct from other more traditionally studied algebraic
structures, e.g. vector spaces or polynomial rings, is the manner in
which dynamics is captured. In traditional structures, dynamics is typically
expressed through morphisms between such structures, as in linear maps
between vector spaces or morphisms between rings. In algebras
associated with the semantics of computation, the dynamics is
expressed as part of the algebraic structure itself, through a
reduction reduction relation typically denoted by $\red$. Below, we
give a recursive presentation of this relation for the calculus used
in the encoding.

$\red \subseteq \pi \times \pi$
$\red : \pi \to \mathcal{P}(\pi)$

\begin{mathpar}
  \inferrule* [lab=Comm] { \textsf{match}( x_{src}, x_{trgt} ) } { x_{trgt}?(y)P \; | \; x_{src}!\langle {Q} \rangle \red P\{\quotep{Q}/y}\} }
  \and \\
  \inferrule* [lab=Par] {{P} \red {P}'} {{{P} | {Q}} \red {{P}' | {Q}}}
  \and
  \inferrule* [lab=Equiv]{{{P} \scong {P}'} \andalso {{P}' \red {Q}'} \andalso {{Q}' \scong {Q}}}{{P} \red {Q}}
\end{mathpar}

\begin{eqnarray*}
  match_{\equiv} (\quotep{P},\quotep{Q}) & := & P \equiv Q \\
  match_{\dagger}(\quotep{P},\quotep{Q}) & := & \forall R. P|Q \red^{*} R => R \red^{*} 0 \\
  match_{K}(\quotep{P},\quotep{Q}) & := & K \mbox{ for some context } K
\end{eqnarray*}

$u?(x)P | u!\langle Q \rangle \red P\{\quotep{Q}/x\}$

%We write $\wred$ for $\red^*$, and $P\red$ if $\exists Q $ such that $ P \red Q$.
We write $P\red$ if $\exists Q $ such that $ P \red Q$ and $P\not\red$, otherwise.

\section{Replication}

As mentioned before, it is known that replication (and hence
recursion) can be implemented in a higher-order process algebra
\cite{SangiorgiWalker}. As our first example of calculation with the
machinery thus far presented we give the construction explicitly in
the {\rhoc}.

\begin{eqnarray}
	D_{x} & := & \prefix{x}{y}{(\binpar{\outputp{x}{y}}{@{y}})} \nonumber\\
	\bangp_{x}{P} & := & \binpar{{x}!\langle{\binpar{D_{x}}{P}}\rangle}{D_{x}} \nonumber
\end{eqnarray}

\begin{eqnarray}
	\bangp_{x}{P} & & \nonumber\\
	=
	& {x}!\langle{(\prefix{x}{y}{(\outputp{x}{y} | @{y})) | P}}\rangle 
	      | \prefix{x}{y}{(\outputp{x}{y} | @{y})} & \nonumber\\
	\red
	& (\outputp{x}{y} | @{y})\substn{\quotep{(\prefix{x}{y}{(@{y} | \outputp{x}{y})) | P}}}{y} & \nonumber\\
	=
	& \outputp{x}{\quotep{(\prefix{x}{y}{(\outputp{x}{y} | @{y})) | P}}}
	  | {(\prefix{x}{y}{(\outputp{x}{y} | @{y})) | P}} & \nonumber\\
	\red
	& \ldots & \nonumber\\
	\red^*
	& P | P | \ldots & \nonumber
\end{eqnarray}

Of course, this encoding, as an implementation, runs away, unfolding
$\bangp{P}$ eagerly. A lazier and more implementable replication
operator, restricted to input-guarded processes, may be obtained as follows.

\begin{eqnarray}
\bangp{\prefix{u}{v}{P}} 
	:= 
	\binpar{\lift{x}{\prefix{u}{v}{(\binpar{D(x)}{P})}}}{D(x)} \nonumber
\end{eqnarray}

\begin{remark}
  Note that the lazier definition still does not deal with summation
  or mixed summation (i.e. sums over input and output). The reader is
  invited to construct definitions of replication that deal with these
  features. 

  Further, the definitions are parameterized in a name, $x$. Can you,
  gentle reader, make a definition that eliminates this parameter and
  guarantees no accidental interaction between the replication
  machinery and the process being replicated -- i.e. no accidental
  sharing of names used by the process to get its work done and the
  name(s) used by the replication to effect copying. This latter
  revision of the definition of replication is crucial to obtaining
  the expected identity $!!P \sim !P$.
\end{remark}

\begin{remark}\label{rem:paradoxical_combinator}
  The reader familiar with the lambda calculus will have noticed the
  similarity between $D$ and the paradoxical combinator.

  [Ed. note: the existence of this seems to suggest we have to be more
  restrictive on the set of processes and names we admit if we are to
  support no-cloning.]
\end{remark}

\subsubsection{Bisimulation}

The computational dynamics gives rise to another kind of equivalence,
the equivalence of computational behavior. As previously mentioned
this is typically captured \emph{via} some form of bisimulation.

% The notion we use in this paper is weak barbed bisimulation
% \cite{milner91polyadicpi}.

The notion we use in this paper is derived from weak barbed
bisimulation \cite{milner91polyadicpi}. 

\begin{definition}
An \emph{observation relation}, $\downarrow_{\mathcal N}$, over a set
of names, $\mathcal N$, is the smallest relation satisfying the rules
below.

\infrule[Out-barb]{y \in {\mathcal N}, \; x \nameeq y}
		  {\outputp{x}{v} \downarrow_{\mathcal N} x}
\infrule[Par-barb]{\mbox{$P\downarrow_{\mathcal N} x$ or $Q\downarrow_{\mathcal N} x$}}
		  {\binpar{P}{Q} \downarrow_{\mathcal N} x}

We write $P \Downarrow_{\mathcal N} x$ if there is $Q$ such that 
$P \wred Q$ and $Q \downarrow_{\mathcal N} x$.
\end{definition}

\begin{definition}
%\label{def.bbisim}
An  ${\mathcal N}$-\emph{barbed bisimulation} over a set of names, ${\mathcal N}$, is a symmetric binary relation 
${\mathcal S}_{\mathcal N}$ between agents such that $P\rel{S}_{\mathcal N}Q$ implies:
\begin{enumerate}
\item If $P \red P'$ then $Q \wred Q'$ and $P'\rel{S}_{\mathcal N} Q'$.
\item If $P\downarrow_{\mathcal N} x$, then $Q\Downarrow_{\mathcal N} x$.
\end{enumerate}
$P$ is ${\mathcal N}$-barbed bisimilar to $Q$, written
$P \wbbisim_{\mathcal N} Q$, if $P \rel{S}_{\mathcal N} Q$ for some ${\mathcal N}$-barbed bisimulation ${\mathcal S}_{\mathcal N}$.
\end{definition}

$\mathcal{R} \subseteq \pi \times \pi$

$P \mathcal{R} Q => \forall P'. P \red P' \Rightarrow \exists Q'. Q \red Q', P' \mathcal{R} Q'$

$P \vdash x \Rightarrow Q \vdash x$

\begin{mathpar}
  \inferrule*[lab=Out-barb]{x \nameeq y}{{y}!\langle{Q}\rangle \vdash x}
  \and
  \inferrule*[lab=Par-barb]{\mbox{$P\vdash x$ or $Q\vdash x$}}{\binpar{P}{Q} \vdash x}
\end{mathpar}

\subsubsection{Contexts}

One of the principle advantages of computational calculi like the
$\pi$-calculus is a well-defined notion of context,
contextual-equivalence and a correlation between
contextual-equivalence and notions of bisimulation. The notion of
context allows the decomposition of a process into (sub-)process and
its syntactic environment, its context. Thus, a context may be
thought of as a process with a ``hole'' (written $\Box$) in it. The
application of a context $M$ to a process $P$, written $M[P]$, is
tantamount to filling the hole in $M$ with $P$. In this paper we do
not need the full weight of this theory, but do make use of the notion
of context in the proof the main theorem. 

\begin{mathpar}
  \inferrule* [lab=summation] {} {{M_{M},M_{N}} \bc \Box \;|\; x.M_{A} \;|\; M_{M}+M_{N}}
  \and
  \inferrule* [lab=agent] {} {{M_{A}} \bc (\vec{x})M_{P} \;| \; \clift{P_0,\ldots,M_{P},\ldots,P_N}}
  \and \\
  \inferrule* [lab=process] {} {{M_{P}} \bc M_{N} \;| \;P|M_{P} }
\end{mathpar} 

\begin{mathpar}
  \inferrule* [lab=sychronization] {} {M_{N} \bc \Box \;|\; x?M_{F} \;|\; x!M_{C}}
  \and
  \inferrule* [lab=abstraction] {} {{M_{F}} \bc (x)M_{P} }
  \and
  \inferrule* [lab=concretion] {} {{M_{C}} \bc \langle M_{P} \rangle }
  \and \\
  \inferrule* [lab=process] {} {{M_{P}} \bc M_{N} \;| \;P|M_{P} }
\end{mathpar}

\begin{definition}[contextual application] Given a context $M$, and
  process $P$, we define the \emph{contextual application}, $M[P] :=
  M\{P/\Box\}$. That is, the contextual application of M to P is the
  substitution of $P$ for $\Box$ in $M$.
\end{definition}

$\meaningof{-} : L \to \mathcal{P}(\pi)$

\begin{mathpar}
  \inferrule* [lab=collection] {} {\meaningof{true} = \pi, \and \meaningof{~E} = \pi \setminus \meaningof{E}, \and \meaningof{E_{1} \& E_{2}} = \meaningof{E_{1}} \cap \meaningof{E_{2}}}
\end{mathpar}

\begin{mathpar}
  \inferrule* [lab=structure] {} {\meaningof{0} = \{ P \in \pi | P \equiv 0 \}, \and \\ \meaningof{E_1 | E_2} = \{ P \in \pi | P \equiv P_{1} | P_{2}, P_{1} \in \meaningof{E_{1}}, P_{2} \in \meaningof{E_2}\} }
\end{mathpar}

\begin{mathpar}
 \inferrule* [lab=behavior] {} {\meaningof{\langle a?b \rangle E} = \{ P \in \pi | P \equiv Q | u?(y)P', \\ \and \\\\ \and \\ \;\;\; u \in \meaningof{a}, \forall z.P'\{z/y\} \in \meaningof{E\{z/b\}}\}, \and \\ \meaningof{a!E} = \{ P \in \pi | P \equiv Q | x!\langle P' \rangle, x \in \meaningof{a} P' \in \meaningof{E}\} }
\end{mathpar}

\begin{mathpar}
 \inferrule* [lab=nominal] {} {\meaningof{\quotep{E}} = \{ \quotep{P} \in \quotep{\pi} | P \in \meaningof{E} \}, \and \meaningof{\quotep{P}} = \{ \quotep{Q} \in \quotep{\pi} | P \equiv Q \} \and \\ \meaningof{@\quotep{E}} = \{ P \in \pi | P \equiv @x, x \in \meaningof{E} \}}
\end{mathpar}

\begin{eqnarray*}
  \\
  \meaningof{-} : TS \to ST
\end{eqnarray*}

\begin{eqnarray*}
  \\
  L : TS \to ST
\end{eqnarray*}

\begin{eqnarray*}
  \\
  P \models E \iff P \in \meaningof{E}
\end{eqnarray*}

\begin{eqnarray*}
  P \approx_{L} Q \iff \forall E \in L. P \models E \iff Q \models E
\end{eqnarray*}

\begin{eqnarray*}
  P \approx_{K} Q
\end{eqnarray*}

\begin{eqnarray*}
  P \approx Q
\end{eqnarray*}

$\approx_{K} = \approx = \approx_{L}$

\subsubsection{Contextual duality}

Note that contexts extend the quotation operation to a family of
operations from processes to names. Given a context, $M$, we can
define a \emph{nominal context}, $\quotep{M}$ by $\quotep{M}[P] :=
\quotep{M[P]}$. To foreshadow what is to come we observe that these
operations enjoy a duality with processes very much like the duality
between vectors and maps from vectors to scalars.

Further, because the calculus is essentially higher-order, we have a
correspondence between contexts and processes. More specifically,
given a name $x$ and a context $M$ we can construct $M^{*}_{x}$ such
that 

\begin{mathpar}
  M^{*}_{x} | \lift{x}{P} \red M[P]
\end{mathpar}

namely,

\begin{mathpar}
  M^{*}_{x} := x?(u).M[\dropn{u}]
\end{mathpar}

The dependence of $M^{*}_{x}$ on a name makes it an abstraction, 

\begin{mathpar}
  M^{*} := (x)x?(u).M[\dropn{u}]
\end{mathpar}

\subsection{Additional notation}

It will sometimes be convenient to denote the process a name
quotes. We already have the notation $x = \quotep{P}$, but it will be
convenient to introduce an alternate notation, $\procn{x}$, when we
want to emphasize the connection to the use of the name. Note that, by
virtue of name equivalence, $\quotep{\procn{x}} \nameeq x$; so, the
notation is consistent with previous definitions.

Further, because names have structure it is possible to effect
substitutions on the basis of that structure. This means we need to
upgrade our notation for substitutions, which we accomplish by
adapting comprehension notation. Thus,

\begin{mathpar}
  P\{ y / x : x \in S \}
\end{mathpar}

is interpreted to mean the process derived from P by replacing (in a
capture-avoiding manner) each occurrence of $x$ in $S$ by $y$. For example,

\begin{mathpar}
  P\{ \quotep{\procn{x}|\procn{x}} / x : x \in \freenames{P} \}
\end{mathpar}

will replace each (occurrence) of a free name $x$ in $P$ by
$\quotep{\procn{x}|\procn{x}}$.

Also, we will avail ourselves of the notation $x^{L}$ and $x^{R}$ to
denote injections of a name into disjoint copies of the name
space. There are numerous ways to accomplish this. One example can be
found in \cite{MeredithR05}. This notation overloads to vectors of
names: $\vec{x}^{\pi} := (x_{i}^{\pi} \; : \; 0 \leq i < |\vec{x}| )$ where $\pi \in \{L,R\}$.

We also use $P^{\Box} := P|\Box$.

In \cite{MeredithR05} an interpretation of the new operator is
given. It turns out that there are several possible interpretations
all enjoying the requisite algebraic properties of the operator (see
\cite{milner91polyadicpi}). We will therefore make liberal use of
$(\nu\; \vec{x})P$.

% subsection the_syntax_and_semantics_of_the_notation_system (end)   

\input{qm2pi.qmops} 

\input{qm2pi.sterngerlach} 

\input{qm2pi.metric} 

% section concurrent_process_calculi (end)

%\input{qm2pi.proofsketch}

% section proof sketch (end)

%\input{qm2pi.slviaknots} 

% section spatial logic via knots (end)

\input{qm2pi.conclusion}

% section conclusion (end)

%\input{qm2pi.dtcodes} 

% section wiring algorithm (end)

\input{qm2pi.ack} 

% section acknowledgments (end)

\newpage


\bibliographystyle{plain}   
\bibliography{../../biblios/main.bib}

\input{qm2pi.rhodetails}

\end{document}

 

% section notation (end)

\input{qm2pi.process.calculi} 

% section concurrent_process_calculi_and_spatial_logics_ (end)
    
%\documentclass[12pt]{llncs}
%\documentclass{jktr}

\usepackage[pdftex]{hyperref}                   
\usepackage {listings}
\usepackage {mathpartir}
\usepackage{bcprules}
%\usepackage{listings}
                       
\usepackage{graphicx} 
%\usepackage[margins=2.5cm,nohead,nofoot]{geometry}
%\usepackage{geometry}
\usepackage{amsfonts}
\usepackage{amstext}
\usepackage{latexsym}
\usepackage{amssymb}
\usepackage{color}


%\include{myPreamble}
\include{qm2pi.local} 

%\ifpdf
%\usepackage[pdftex]{graphicx}
%\else
%\usepackage{graphicx}
%\fi

 % \ifpdf
%  \usepackage{pdfsync}
%  \if


%\title{Brief Article}
%\author{David F. Snyder}
%\author{L.G. Meredith}

%\address{Dept. of Math., Texas State University--San Marcos, San Marcos, TX 78666}
       
\pagestyle{empty}


\begin{document}

\lstset{language=[Objective]Caml,frame=shadowbox}

\input{qm2pi.front}

% section front matter (end)

\input{qm2pi.intro} 
 
% section introduction (end)

% \input{qm2pi.knotations} 

% section notation (end)

\input{qm2pi.process.calculi} 

% section concurrent_process_calculi_and_spatial_logics_ (end)
    
%\input{qm2pi.knots2pi} 

%\input{qm2pi.trefoil} 

%\input{qm2pi.mainthm} 

% subsection basic_interpretation (end)

%\input{qm2pi.rho.presentation} 
\subsection{The syntax and semantics of the notation system}\label{sub:the_syntax_and_semantics_of_the_notation_system} % (fold)

We now summarize a technical presentation of the calculus that
embodies our theory of dynamics. The typical presentation of such a
calculus follows the style of giving generators and relations on
them. The grammar, below, describing term constructors, freely
generates the set of processes, $\Proc$. This set is then quotiented
by a relation known as structural congruence and it is over this set
that the notion of dynamics is expressed. This presentation is
essentially that of \cite{MeredithR05} with the addition of
polyadicity and summation. For readability we have relegated some of
the technical subtleties to an appendix.

\subsubsection{Process grammar}\label{subsub:process_grammar}

\begin{mathpar}
  \inferrule* [lab=synchronization] {} {{M} \bc \pzero \;|\; x?F \;|\; x!C }
  \and
  \inferrule* [lab=abstraction] {} {{F} \bc (x)P}
  \and
  \inferrule* [lab=concretion] {} {{C} \bc \langle Q \rangle}
  \and
  \inferrule* [lab=process] {} {{P,Q} \bc M \;| \;P|Q \;|\; @{x}}
  \and
  \inferrule* [lab=name] {} {{x} \bc \quotep{P}}
\end{mathpar} 

Note that $\vec{x}$ (resp. $\vec{P}$) denotes a vector of names
(resp. processes) of length $|\vec{x}|$ (resp. $|\vec{P}|$). We adopt
the following useful abbreviations.

\begin{mathpar}
   x?(\vec{y}).P := x.(\vec{y})P \and  x\clift{\vec{P}} := x.\clift{\vec{P}}
   \and x!(y) := \lift{x}{\dropn{y}}
   \and \Pi_{i=0}^{n-1}P_i := P_0 | \ldots | P_{n-1}
\end{mathpar}

\subsubsection{Structural congruence}

\paragraph{Free and bound names and alpha-equivalence.} At the
core of structural equivalence is alpha-equivalence which identifies
process that are the same up to a change of variable. Formally, we
recognize the distinction between free and bound names. The free names
of a process, $\freenames{P}$, may be calculated recursively as
follows:

\begin{mathpar}
\freenames{\pzero} := \emptyset
  \and \\
  \freenames{x?(y).P} := \{ x \} \cup (\freenames{P} \setminus \{ y \})
  \and 
  \freenames{x!\langle P \rangle} := \{ x \} \cup \{ P \} 
  \and \\
  \freenames{P|Q} := \freenames{P} \cup \freenames{Q}
  \and \\
  \freenames{@{x}} := \{ x \}
\end{mathpar}

$\pi$
$\quotep{\pi}$

$\freenames{-} : \pi \to \mathcal{P}(\quotep{\pi})$

\begin{eqnarray*}
  \freenames{\pzero} & := & \emptyset \\
  \freenames{x?(y).P} & := & \{ x \} \cup (\freenames{P} \setminus \{ y \}) \\
  \freenames{x!\langle P \rangle} & := & \{ x \} \cup \{ P \} \\
  \freenames{P|Q} & := & \freenames{P} \cup \freenames{Q} \\
  \freenames{\dropn{x}} & := & \{ x \}
\end{eqnarray*}

The bound names of a process, $\boundnames{P}$, are those names occurring in $P$
that are not free. For example, in $x?(y).0$, the name $x$ is free, while $y$ is bound.

\begin{mathpar}
  \inferrule* [lab=monoidal-laws] {} { P|Q \equiv Q|P \and P|0 \equiv P \and P|(Q|R) \equiv (P|Q)|R }
\end{mathpar}

\begin{mathpar}
  \inferrule* [lab=alpha-equivalence] {} { (x)P \equiv (y)P\{y/x\} \and y \not\in \freenames{P} }
\end{mathpar}

\begin{definition}
Then two processes, $P,Q$, are alpha-equivalent if $P = Q\{\vec{y}/\vec{x}\}$ for
some $\vec{x} \in \boundnames{Q},\vec{y} \in \boundnames{P}$, where $Q\{\vec{y}/\vec{x}\}$
denotes the capture-avoiding substitution of $\vec{y}$ for $\vec{x}$ in $Q$.
\end{definition}

\begin{definition}
  The {\em structural congruence} \cite{SangiorgiWalker} , $\equiv$,
  between processes is the least congruence containing
  alpha-equivalence, satisfying the abelian monoid laws
  (associativity, commutativity and $\pzero$ as identity) for parallel
  composition $|$ and for summation $+$.
\end{definition}

\subsection{Name equivalence}

We take name equivalence, written $\nameeq$, to be the smallest
equivalence relation generated by the following rules.

\begin{mathpar}
\inferrule*[lab=Quote-drop]
{ }
{ \quotep{@{x}} \nameeq x }

\inferrule*[lab=Struct-equiv]
{ P \scong Q }
{ \quotep{P} \nameeq \quotep{Q} }
\end{mathpar}

The astute reader will have noticed that the mutual recursion of names
and processes imposes a mutual recursion on alpha-equivalence and
structural equivalence via name-equivalence. Fortunately, all of this
works out pleasantly and we may calculate in the natural way, free of
concern. The reader interested in the details is referred to the
appendix \ref{appendix:rho_details}.

\subsection{Substitution}

We use $\Proc$ for the set of processes, $\QProc$ for the set of
names, and $\id{\{}\vec{y} / \vec{x} \id{\}}$ to denote partial maps,
$s : \QProc \rightarrow \QProc$. A map, $s$ lifts, uniquely, to a map
on process terms, $\widehat{s} : \Proc \rightarrow \Proc$ by the
following equations.

\begin{mathpar}
  (0) \psubstp{Q}{P} := 0 \\
  (R \juxtap S) \psubstp{Q}{P}
  :=    
  (R)\psubstp{Q}{P} \juxtap (S) \psubstp{Q}{P} \\
  (x?(y).R) \psubstp{Q}{P}    
  :=    
  (x)\substp{Q}{P} (z)\concat( (R \psubstn{z}{y}) \psubstp{Q}{P} ) \\
  (\lift{x}{R}) \psubstp{Q}{P}  
  :=
  \lift{(x)\substp{Q}{P}}{ R \psubstp{Q}{P} } \\
%   (\dropn{x})  \psubstp{Q}{P}       
%   := 
%   \left\{ 
%     \begin{array}{ccc} 
%       \dropn{\quotep{Q}} & & x \nameeq \quotep{P} \\
%       \dropn{x} & & otherwise \\
%     \end{array}
%   \right. 
  (\dropn{x})  \psubstp{Q}{P}       
  := 
  \left\{ 
    \begin{array}{ccc} 
      Q & & x \nameeq \quotep{P} \\
      \dropn{x} & & otherwise \\
    \end{array}
  \right.
\end{mathpar}
 

where

\begin{eqnarray}
  (x)\id{\{} \lpquote Q \rpquote / \lpquote P \rpquote \id{\}}            = 
  \left\{ 
    \begin{array}{ccc}
      \lpquote Q \rpquote & & x \nameeq \lpquote P \rpquote \\
      x & & otherwise \\
    \end{array}
  \right. \nonumber
\end{eqnarray}

and $z$ is chosen distinct from $\quotep{P}$, $\quotep{Q}$, the free
names in $Q$, and all the names in $R$. Our $\alpha$-equivalence will
be built in the standard way from this substitution.

\begin{remark}\label{rem:no_self_referential_names}
  One consequence of these definitions is that $\forall P. \quotep{P}
  \not\in \freenames{P}$.
\end{remark}

\subsection{ Dynamic quote: an example }

Anticipating something of what's to come, consider applying the
substitution, $\widehat{\id{\{}u / z \id{\}}}$, to the following pair
of processes, $\lift{w}{y!(z)}$ and $w[ \lpquote y!(z) \rpquote ]$.

\begin{eqnarray}
	\lift{w}{y!(z)}\widehat{\id{\{}u / z \id{\}}}
		& = &
		\lift{w}{y!(u)} \nonumber\\
	w[ \lpquote y!(z) \rpquote ] \widehat{ \id{\{}u / z \id{\}} }
		& = &
		w[ \lpquote y!(z) \rpquote ] \nonumber
\end{eqnarray}

Because the body of the process between quotes is impervious to
substitution, we get radically different answers. In fact, by
examining the first process in an input context,
e.g. $x?(z).\lift{w}{y!(z)}$, we see that the process under the lift
operator may be shaped by prefixed inputs binding a name inside it. In
this sense, the lift operator will be seen as a way to dynamically
construct processes before reifying them as names.

Finally equipped with these standard features we can present the
dynamics of the calculus.

\subsubsection{Operational semantics} 

Finally, we introduce the computational dynamics. What marks these
algebras as distinct from other more traditionally studied algebraic
structures, e.g. vector spaces or polynomial rings, is the manner in
which dynamics is captured. In traditional structures, dynamics is typically
expressed through morphisms between such structures, as in linear maps
between vector spaces or morphisms between rings. In algebras
associated with the semantics of computation, the dynamics is
expressed as part of the algebraic structure itself, through a
reduction reduction relation typically denoted by $\red$. Below, we
give a recursive presentation of this relation for the calculus used
in the encoding.

$\red \subseteq \pi \times \pi$
$\red : \pi \to \mathcal{P}(\pi)$

\begin{mathpar}
  \inferrule* [lab=Comm] { \textsf{match}( x_{src}, x_{trgt} ) } { x_{trgt}?(y)P \; | \; x_{src}!\langle {Q} \rangle \red P\{\quotep{Q}/y}\} }
  \and \\
  \inferrule* [lab=Par] {{P} \red {P}'} {{{P} | {Q}} \red {{P}' | {Q}}}
  \and
  \inferrule* [lab=Equiv]{{{P} \scong {P}'} \andalso {{P}' \red {Q}'} \andalso {{Q}' \scong {Q}}}{{P} \red {Q}}
\end{mathpar}

\begin{eqnarray*}
  match_{\equiv} (\quotep{P},\quotep{Q}) & := & P \equiv Q \\
  match_{\dagger}(\quotep{P},\quotep{Q}) & := & \forall R. P|Q \red^{*} R => R \red^{*} 0 \\
  match_{K}(\quotep{P},\quotep{Q}) & := & K \mbox{ for some context } K
\end{eqnarray*}

$u?(x)P | u!\langle Q \rangle \red P\{\quotep{Q}/x\}$

%We write $\wred$ for $\red^*$, and $P\red$ if $\exists Q $ such that $ P \red Q$.
We write $P\red$ if $\exists Q $ such that $ P \red Q$ and $P\not\red$, otherwise.

\section{Replication}

As mentioned before, it is known that replication (and hence
recursion) can be implemented in a higher-order process algebra
\cite{SangiorgiWalker}. As our first example of calculation with the
machinery thus far presented we give the construction explicitly in
the {\rhoc}.

\begin{eqnarray}
	D_{x} & := & \prefix{x}{y}{(\binpar{\outputp{x}{y}}{@{y}})} \nonumber\\
	\bangp_{x}{P} & := & \binpar{{x}!\langle{\binpar{D_{x}}{P}}\rangle}{D_{x}} \nonumber
\end{eqnarray}

\begin{eqnarray}
	\bangp_{x}{P} & & \nonumber\\
	=
	& {x}!\langle{(\prefix{x}{y}{(\outputp{x}{y} | @{y})) | P}}\rangle 
	      | \prefix{x}{y}{(\outputp{x}{y} | @{y})} & \nonumber\\
	\red
	& (\outputp{x}{y} | @{y})\substn{\quotep{(\prefix{x}{y}{(@{y} | \outputp{x}{y})) | P}}}{y} & \nonumber\\
	=
	& \outputp{x}{\quotep{(\prefix{x}{y}{(\outputp{x}{y} | @{y})) | P}}}
	  | {(\prefix{x}{y}{(\outputp{x}{y} | @{y})) | P}} & \nonumber\\
	\red
	& \ldots & \nonumber\\
	\red^*
	& P | P | \ldots & \nonumber
\end{eqnarray}

Of course, this encoding, as an implementation, runs away, unfolding
$\bangp{P}$ eagerly. A lazier and more implementable replication
operator, restricted to input-guarded processes, may be obtained as follows.

\begin{eqnarray}
\bangp{\prefix{u}{v}{P}} 
	:= 
	\binpar{\lift{x}{\prefix{u}{v}{(\binpar{D(x)}{P})}}}{D(x)} \nonumber
\end{eqnarray}

\begin{remark}
  Note that the lazier definition still does not deal with summation
  or mixed summation (i.e. sums over input and output). The reader is
  invited to construct definitions of replication that deal with these
  features. 

  Further, the definitions are parameterized in a name, $x$. Can you,
  gentle reader, make a definition that eliminates this parameter and
  guarantees no accidental interaction between the replication
  machinery and the process being replicated -- i.e. no accidental
  sharing of names used by the process to get its work done and the
  name(s) used by the replication to effect copying. This latter
  revision of the definition of replication is crucial to obtaining
  the expected identity $!!P \sim !P$.
\end{remark}

\begin{remark}\label{rem:paradoxical_combinator}
  The reader familiar with the lambda calculus will have noticed the
  similarity between $D$ and the paradoxical combinator.

  [Ed. note: the existence of this seems to suggest we have to be more
  restrictive on the set of processes and names we admit if we are to
  support no-cloning.]
\end{remark}

\subsubsection{Bisimulation}

The computational dynamics gives rise to another kind of equivalence,
the equivalence of computational behavior. As previously mentioned
this is typically captured \emph{via} some form of bisimulation.

% The notion we use in this paper is weak barbed bisimulation
% \cite{milner91polyadicpi}.

The notion we use in this paper is derived from weak barbed
bisimulation \cite{milner91polyadicpi}. 

\begin{definition}
An \emph{observation relation}, $\downarrow_{\mathcal N}$, over a set
of names, $\mathcal N$, is the smallest relation satisfying the rules
below.

\infrule[Out-barb]{y \in {\mathcal N}, \; x \nameeq y}
		  {\outputp{x}{v} \downarrow_{\mathcal N} x}
\infrule[Par-barb]{\mbox{$P\downarrow_{\mathcal N} x$ or $Q\downarrow_{\mathcal N} x$}}
		  {\binpar{P}{Q} \downarrow_{\mathcal N} x}

We write $P \Downarrow_{\mathcal N} x$ if there is $Q$ such that 
$P \wred Q$ and $Q \downarrow_{\mathcal N} x$.
\end{definition}

\begin{definition}
%\label{def.bbisim}
An  ${\mathcal N}$-\emph{barbed bisimulation} over a set of names, ${\mathcal N}$, is a symmetric binary relation 
${\mathcal S}_{\mathcal N}$ between agents such that $P\rel{S}_{\mathcal N}Q$ implies:
\begin{enumerate}
\item If $P \red P'$ then $Q \wred Q'$ and $P'\rel{S}_{\mathcal N} Q'$.
\item If $P\downarrow_{\mathcal N} x$, then $Q\Downarrow_{\mathcal N} x$.
\end{enumerate}
$P$ is ${\mathcal N}$-barbed bisimilar to $Q$, written
$P \wbbisim_{\mathcal N} Q$, if $P \rel{S}_{\mathcal N} Q$ for some ${\mathcal N}$-barbed bisimulation ${\mathcal S}_{\mathcal N}$.
\end{definition}

$\mathcal{R} \subseteq \pi \times \pi$

$P \mathcal{R} Q => \forall P'. P \red P' \Rightarrow \exists Q'. Q \red Q', P' \mathcal{R} Q'$

$P \vdash x \Rightarrow Q \vdash x$

\begin{mathpar}
  \inferrule*[lab=Out-barb]{x \nameeq y}{{y}!\langle{Q}\rangle \vdash x}
  \and
  \inferrule*[lab=Par-barb]{\mbox{$P\vdash x$ or $Q\vdash x$}}{\binpar{P}{Q} \vdash x}
\end{mathpar}

\subsubsection{Contexts}

One of the principle advantages of computational calculi like the
$\pi$-calculus is a well-defined notion of context,
contextual-equivalence and a correlation between
contextual-equivalence and notions of bisimulation. The notion of
context allows the decomposition of a process into (sub-)process and
its syntactic environment, its context. Thus, a context may be
thought of as a process with a ``hole'' (written $\Box$) in it. The
application of a context $M$ to a process $P$, written $M[P]$, is
tantamount to filling the hole in $M$ with $P$. In this paper we do
not need the full weight of this theory, but do make use of the notion
of context in the proof the main theorem. 

\begin{mathpar}
  \inferrule* [lab=summation] {} {{M_{M},M_{N}} \bc \Box \;|\; x.M_{A} \;|\; M_{M}+M_{N}}
  \and
  \inferrule* [lab=agent] {} {{M_{A}} \bc (\vec{x})M_{P} \;| \; \clift{P_0,\ldots,M_{P},\ldots,P_N}}
  \and \\
  \inferrule* [lab=process] {} {{M_{P}} \bc M_{N} \;| \;P|M_{P} }
\end{mathpar} 

\begin{mathpar}
  \inferrule* [lab=sychronization] {} {M_{N} \bc \Box \;|\; x?M_{F} \;|\; x!M_{C}}
  \and
  \inferrule* [lab=abstraction] {} {{M_{F}} \bc (x)M_{P} }
  \and
  \inferrule* [lab=concretion] {} {{M_{C}} \bc \langle M_{P} \rangle }
  \and \\
  \inferrule* [lab=process] {} {{M_{P}} \bc M_{N} \;| \;P|M_{P} }
\end{mathpar}

\begin{definition}[contextual application] Given a context $M$, and
  process $P$, we define the \emph{contextual application}, $M[P] :=
  M\{P/\Box\}$. That is, the contextual application of M to P is the
  substitution of $P$ for $\Box$ in $M$.
\end{definition}

$\meaningof{-} : L \to \mathcal{P}(\pi)$

\begin{mathpar}
  \inferrule* [lab=collection] {} {\meaningof{true} = \pi, \and \meaningof{~E} = \pi \setminus \meaningof{E}, \and \meaningof{E_{1} \& E_{2}} = \meaningof{E_{1}} \cap \meaningof{E_{2}}}
\end{mathpar}

\begin{mathpar}
  \inferrule* [lab=structure] {} {\meaningof{0} = \{ P \in \pi | P \equiv 0 \}, \and \\ \meaningof{E_1 | E_2} = \{ P \in \pi | P \equiv P_{1} | P_{2}, P_{1} \in \meaningof{E_{1}}, P_{2} \in \meaningof{E_2}\} }
\end{mathpar}

\begin{mathpar}
 \inferrule* [lab=behavior] {} {\meaningof{\langle a?b \rangle E} = \{ P \in \pi | P \equiv Q | u?(y)P', \\ \and \\\\ \and \\ \;\;\; u \in \meaningof{a}, \forall z.P'\{z/y\} \in \meaningof{E\{z/b\}}\}, \and \\ \meaningof{a!E} = \{ P \in \pi | P \equiv Q | x!\langle P' \rangle, x \in \meaningof{a} P' \in \meaningof{E}\} }
\end{mathpar}

\begin{mathpar}
 \inferrule* [lab=nominal] {} {\meaningof{\quotep{E}} = \{ \quotep{P} \in \quotep{\pi} | P \in \meaningof{E} \}, \and \meaningof{\quotep{P}} = \{ \quotep{Q} \in \quotep{\pi} | P \equiv Q \} \and \\ \meaningof{@\quotep{E}} = \{ P \in \pi | P \equiv @x, x \in \meaningof{E} \}}
\end{mathpar}

\begin{eqnarray*}
  \\
  \meaningof{-} : TS \to ST
\end{eqnarray*}

\begin{eqnarray*}
  \\
  L : TS \to ST
\end{eqnarray*}

\begin{eqnarray*}
  \\
  P \models E \iff P \in \meaningof{E}
\end{eqnarray*}

\begin{eqnarray*}
  P \approx_{L} Q \iff \forall E \in L. P \models E \iff Q \models E
\end{eqnarray*}

\begin{eqnarray*}
  P \approx_{K} Q
\end{eqnarray*}

\begin{eqnarray*}
  P \approx Q
\end{eqnarray*}

$\approx_{K} = \approx = \approx_{L}$

\subsubsection{Contextual duality}

Note that contexts extend the quotation operation to a family of
operations from processes to names. Given a context, $M$, we can
define a \emph{nominal context}, $\quotep{M}$ by $\quotep{M}[P] :=
\quotep{M[P]}$. To foreshadow what is to come we observe that these
operations enjoy a duality with processes very much like the duality
between vectors and maps from vectors to scalars.

Further, because the calculus is essentially higher-order, we have a
correspondence between contexts and processes. More specifically,
given a name $x$ and a context $M$ we can construct $M^{*}_{x}$ such
that 

\begin{mathpar}
  M^{*}_{x} | \lift{x}{P} \red M[P]
\end{mathpar}

namely,

\begin{mathpar}
  M^{*}_{x} := x?(u).M[\dropn{u}]
\end{mathpar}

The dependence of $M^{*}_{x}$ on a name makes it an abstraction, 

\begin{mathpar}
  M^{*} := (x)x?(u).M[\dropn{u}]
\end{mathpar}

\subsection{Additional notation}

It will sometimes be convenient to denote the process a name
quotes. We already have the notation $x = \quotep{P}$, but it will be
convenient to introduce an alternate notation, $\procn{x}$, when we
want to emphasize the connection to the use of the name. Note that, by
virtue of name equivalence, $\quotep{\procn{x}} \nameeq x$; so, the
notation is consistent with previous definitions.

Further, because names have structure it is possible to effect
substitutions on the basis of that structure. This means we need to
upgrade our notation for substitutions, which we accomplish by
adapting comprehension notation. Thus,

\begin{mathpar}
  P\{ y / x : x \in S \}
\end{mathpar}

is interpreted to mean the process derived from P by replacing (in a
capture-avoiding manner) each occurrence of $x$ in $S$ by $y$. For example,

\begin{mathpar}
  P\{ \quotep{\procn{x}|\procn{x}} / x : x \in \freenames{P} \}
\end{mathpar}

will replace each (occurrence) of a free name $x$ in $P$ by
$\quotep{\procn{x}|\procn{x}}$.

Also, we will avail ourselves of the notation $x^{L}$ and $x^{R}$ to
denote injections of a name into disjoint copies of the name
space. There are numerous ways to accomplish this. One example can be
found in \cite{MeredithR05}. This notation overloads to vectors of
names: $\vec{x}^{\pi} := (x_{i}^{\pi} \; : \; 0 \leq i < |\vec{x}| )$ where $\pi \in \{L,R\}$.

We also use $P^{\Box} := P|\Box$.

In \cite{MeredithR05} an interpretation of the new operator is
given. It turns out that there are several possible interpretations
all enjoying the requisite algebraic properties of the operator (see
\cite{milner91polyadicpi}). We will therefore make liberal use of
$(\nu\; \vec{x})P$.

% subsection the_syntax_and_semantics_of_the_notation_system (end)   

\input{qm2pi.qmops} 

\input{qm2pi.sterngerlach} 

\input{qm2pi.metric} 

% section concurrent_process_calculi (end)

%\input{qm2pi.proofsketch}

% section proof sketch (end)

%\input{qm2pi.slviaknots} 

% section spatial logic via knots (end)

\input{qm2pi.conclusion}

% section conclusion (end)

%\input{qm2pi.dtcodes} 

% section wiring algorithm (end)

\input{qm2pi.ack} 

% section acknowledgments (end)

\newpage


\bibliographystyle{plain}   
\bibliography{../../biblios/main.bib}

\input{qm2pi.rhodetails}

\end{document}

 

%\documentclass[12pt]{llncs}
%\documentclass{jktr}

\usepackage[pdftex]{hyperref}                   
\usepackage {listings}
\usepackage {mathpartir}
\usepackage{bcprules}
%\usepackage{listings}
                       
\usepackage{graphicx} 
%\usepackage[margins=2.5cm,nohead,nofoot]{geometry}
%\usepackage{geometry}
\usepackage{amsfonts}
\usepackage{amstext}
\usepackage{latexsym}
\usepackage{amssymb}
\usepackage{color}


%\include{myPreamble}
\include{qm2pi.local} 

%\ifpdf
%\usepackage[pdftex]{graphicx}
%\else
%\usepackage{graphicx}
%\fi

 % \ifpdf
%  \usepackage{pdfsync}
%  \if


%\title{Brief Article}
%\author{David F. Snyder}
%\author{L.G. Meredith}

%\address{Dept. of Math., Texas State University--San Marcos, San Marcos, TX 78666}
       
\pagestyle{empty}


\begin{document}

\lstset{language=[Objective]Caml,frame=shadowbox}

\input{qm2pi.front}

% section front matter (end)

\input{qm2pi.intro} 
 
% section introduction (end)

% \input{qm2pi.knotations} 

% section notation (end)

\input{qm2pi.process.calculi} 

% section concurrent_process_calculi_and_spatial_logics_ (end)
    
%\input{qm2pi.knots2pi} 

%\input{qm2pi.trefoil} 

%\input{qm2pi.mainthm} 

% subsection basic_interpretation (end)

%\input{qm2pi.rho.presentation} 
\subsection{The syntax and semantics of the notation system}\label{sub:the_syntax_and_semantics_of_the_notation_system} % (fold)

We now summarize a technical presentation of the calculus that
embodies our theory of dynamics. The typical presentation of such a
calculus follows the style of giving generators and relations on
them. The grammar, below, describing term constructors, freely
generates the set of processes, $\Proc$. This set is then quotiented
by a relation known as structural congruence and it is over this set
that the notion of dynamics is expressed. This presentation is
essentially that of \cite{MeredithR05} with the addition of
polyadicity and summation. For readability we have relegated some of
the technical subtleties to an appendix.

\subsubsection{Process grammar}\label{subsub:process_grammar}

\begin{mathpar}
  \inferrule* [lab=synchronization] {} {{M} \bc \pzero \;|\; x?F \;|\; x!C }
  \and
  \inferrule* [lab=abstraction] {} {{F} \bc (x)P}
  \and
  \inferrule* [lab=concretion] {} {{C} \bc \langle Q \rangle}
  \and
  \inferrule* [lab=process] {} {{P,Q} \bc M \;| \;P|Q \;|\; @{x}}
  \and
  \inferrule* [lab=name] {} {{x} \bc \quotep{P}}
\end{mathpar} 

Note that $\vec{x}$ (resp. $\vec{P}$) denotes a vector of names
(resp. processes) of length $|\vec{x}|$ (resp. $|\vec{P}|$). We adopt
the following useful abbreviations.

\begin{mathpar}
   x?(\vec{y}).P := x.(\vec{y})P \and  x\clift{\vec{P}} := x.\clift{\vec{P}}
   \and x!(y) := \lift{x}{\dropn{y}}
   \and \Pi_{i=0}^{n-1}P_i := P_0 | \ldots | P_{n-1}
\end{mathpar}

\subsubsection{Structural congruence}

\paragraph{Free and bound names and alpha-equivalence.} At the
core of structural equivalence is alpha-equivalence which identifies
process that are the same up to a change of variable. Formally, we
recognize the distinction between free and bound names. The free names
of a process, $\freenames{P}$, may be calculated recursively as
follows:

\begin{mathpar}
\freenames{\pzero} := \emptyset
  \and \\
  \freenames{x?(y).P} := \{ x \} \cup (\freenames{P} \setminus \{ y \})
  \and 
  \freenames{x!\langle P \rangle} := \{ x \} \cup \{ P \} 
  \and \\
  \freenames{P|Q} := \freenames{P} \cup \freenames{Q}
  \and \\
  \freenames{@{x}} := \{ x \}
\end{mathpar}

$\pi$
$\quotep{\pi}$

$\freenames{-} : \pi \to \mathcal{P}(\quotep{\pi})$

\begin{eqnarray*}
  \freenames{\pzero} & := & \emptyset \\
  \freenames{x?(y).P} & := & \{ x \} \cup (\freenames{P} \setminus \{ y \}) \\
  \freenames{x!\langle P \rangle} & := & \{ x \} \cup \{ P \} \\
  \freenames{P|Q} & := & \freenames{P} \cup \freenames{Q} \\
  \freenames{\dropn{x}} & := & \{ x \}
\end{eqnarray*}

The bound names of a process, $\boundnames{P}$, are those names occurring in $P$
that are not free. For example, in $x?(y).0$, the name $x$ is free, while $y$ is bound.

\begin{mathpar}
  \inferrule* [lab=monoidal-laws] {} { P|Q \equiv Q|P \and P|0 \equiv P \and P|(Q|R) \equiv (P|Q)|R }
\end{mathpar}

\begin{mathpar}
  \inferrule* [lab=alpha-equivalence] {} { (x)P \equiv (y)P\{y/x\} \and y \not\in \freenames{P} }
\end{mathpar}

\begin{definition}
Then two processes, $P,Q$, are alpha-equivalent if $P = Q\{\vec{y}/\vec{x}\}$ for
some $\vec{x} \in \boundnames{Q},\vec{y} \in \boundnames{P}$, where $Q\{\vec{y}/\vec{x}\}$
denotes the capture-avoiding substitution of $\vec{y}$ for $\vec{x}$ in $Q$.
\end{definition}

\begin{definition}
  The {\em structural congruence} \cite{SangiorgiWalker} , $\equiv$,
  between processes is the least congruence containing
  alpha-equivalence, satisfying the abelian monoid laws
  (associativity, commutativity and $\pzero$ as identity) for parallel
  composition $|$ and for summation $+$.
\end{definition}

\subsection{Name equivalence}

We take name equivalence, written $\nameeq$, to be the smallest
equivalence relation generated by the following rules.

\begin{mathpar}
\inferrule*[lab=Quote-drop]
{ }
{ \quotep{@{x}} \nameeq x }

\inferrule*[lab=Struct-equiv]
{ P \scong Q }
{ \quotep{P} \nameeq \quotep{Q} }
\end{mathpar}

The astute reader will have noticed that the mutual recursion of names
and processes imposes a mutual recursion on alpha-equivalence and
structural equivalence via name-equivalence. Fortunately, all of this
works out pleasantly and we may calculate in the natural way, free of
concern. The reader interested in the details is referred to the
appendix \ref{appendix:rho_details}.

\subsection{Substitution}

We use $\Proc$ for the set of processes, $\QProc$ for the set of
names, and $\id{\{}\vec{y} / \vec{x} \id{\}}$ to denote partial maps,
$s : \QProc \rightarrow \QProc$. A map, $s$ lifts, uniquely, to a map
on process terms, $\widehat{s} : \Proc \rightarrow \Proc$ by the
following equations.

\begin{mathpar}
  (0) \psubstp{Q}{P} := 0 \\
  (R \juxtap S) \psubstp{Q}{P}
  :=    
  (R)\psubstp{Q}{P} \juxtap (S) \psubstp{Q}{P} \\
  (x?(y).R) \psubstp{Q}{P}    
  :=    
  (x)\substp{Q}{P} (z)\concat( (R \psubstn{z}{y}) \psubstp{Q}{P} ) \\
  (\lift{x}{R}) \psubstp{Q}{P}  
  :=
  \lift{(x)\substp{Q}{P}}{ R \psubstp{Q}{P} } \\
%   (\dropn{x})  \psubstp{Q}{P}       
%   := 
%   \left\{ 
%     \begin{array}{ccc} 
%       \dropn{\quotep{Q}} & & x \nameeq \quotep{P} \\
%       \dropn{x} & & otherwise \\
%     \end{array}
%   \right. 
  (\dropn{x})  \psubstp{Q}{P}       
  := 
  \left\{ 
    \begin{array}{ccc} 
      Q & & x \nameeq \quotep{P} \\
      \dropn{x} & & otherwise \\
    \end{array}
  \right.
\end{mathpar}
 

where

\begin{eqnarray}
  (x)\id{\{} \lpquote Q \rpquote / \lpquote P \rpquote \id{\}}            = 
  \left\{ 
    \begin{array}{ccc}
      \lpquote Q \rpquote & & x \nameeq \lpquote P \rpquote \\
      x & & otherwise \\
    \end{array}
  \right. \nonumber
\end{eqnarray}

and $z$ is chosen distinct from $\quotep{P}$, $\quotep{Q}$, the free
names in $Q$, and all the names in $R$. Our $\alpha$-equivalence will
be built in the standard way from this substitution.

\begin{remark}\label{rem:no_self_referential_names}
  One consequence of these definitions is that $\forall P. \quotep{P}
  \not\in \freenames{P}$.
\end{remark}

\subsection{ Dynamic quote: an example }

Anticipating something of what's to come, consider applying the
substitution, $\widehat{\id{\{}u / z \id{\}}}$, to the following pair
of processes, $\lift{w}{y!(z)}$ and $w[ \lpquote y!(z) \rpquote ]$.

\begin{eqnarray}
	\lift{w}{y!(z)}\widehat{\id{\{}u / z \id{\}}}
		& = &
		\lift{w}{y!(u)} \nonumber\\
	w[ \lpquote y!(z) \rpquote ] \widehat{ \id{\{}u / z \id{\}} }
		& = &
		w[ \lpquote y!(z) \rpquote ] \nonumber
\end{eqnarray}

Because the body of the process between quotes is impervious to
substitution, we get radically different answers. In fact, by
examining the first process in an input context,
e.g. $x?(z).\lift{w}{y!(z)}$, we see that the process under the lift
operator may be shaped by prefixed inputs binding a name inside it. In
this sense, the lift operator will be seen as a way to dynamically
construct processes before reifying them as names.

Finally equipped with these standard features we can present the
dynamics of the calculus.

\subsubsection{Operational semantics} 

Finally, we introduce the computational dynamics. What marks these
algebras as distinct from other more traditionally studied algebraic
structures, e.g. vector spaces or polynomial rings, is the manner in
which dynamics is captured. In traditional structures, dynamics is typically
expressed through morphisms between such structures, as in linear maps
between vector spaces or morphisms between rings. In algebras
associated with the semantics of computation, the dynamics is
expressed as part of the algebraic structure itself, through a
reduction reduction relation typically denoted by $\red$. Below, we
give a recursive presentation of this relation for the calculus used
in the encoding.

$\red \subseteq \pi \times \pi$
$\red : \pi \to \mathcal{P}(\pi)$

\begin{mathpar}
  \inferrule* [lab=Comm] { \textsf{match}( x_{src}, x_{trgt} ) } { x_{trgt}?(y)P \; | \; x_{src}!\langle {Q} \rangle \red P\{\quotep{Q}/y}\} }
  \and \\
  \inferrule* [lab=Par] {{P} \red {P}'} {{{P} | {Q}} \red {{P}' | {Q}}}
  \and
  \inferrule* [lab=Equiv]{{{P} \scong {P}'} \andalso {{P}' \red {Q}'} \andalso {{Q}' \scong {Q}}}{{P} \red {Q}}
\end{mathpar}

\begin{eqnarray*}
  match_{\equiv} (\quotep{P},\quotep{Q}) & := & P \equiv Q \\
  match_{\dagger}(\quotep{P},\quotep{Q}) & := & \forall R. P|Q \red^{*} R => R \red^{*} 0 \\
  match_{K}(\quotep{P},\quotep{Q}) & := & K \mbox{ for some context } K
\end{eqnarray*}

$u?(x)P | u!\langle Q \rangle \red P\{\quotep{Q}/x\}$

%We write $\wred$ for $\red^*$, and $P\red$ if $\exists Q $ such that $ P \red Q$.
We write $P\red$ if $\exists Q $ such that $ P \red Q$ and $P\not\red$, otherwise.

\section{Replication}

As mentioned before, it is known that replication (and hence
recursion) can be implemented in a higher-order process algebra
\cite{SangiorgiWalker}. As our first example of calculation with the
machinery thus far presented we give the construction explicitly in
the {\rhoc}.

\begin{eqnarray}
	D_{x} & := & \prefix{x}{y}{(\binpar{\outputp{x}{y}}{@{y}})} \nonumber\\
	\bangp_{x}{P} & := & \binpar{{x}!\langle{\binpar{D_{x}}{P}}\rangle}{D_{x}} \nonumber
\end{eqnarray}

\begin{eqnarray}
	\bangp_{x}{P} & & \nonumber\\
	=
	& {x}!\langle{(\prefix{x}{y}{(\outputp{x}{y} | @{y})) | P}}\rangle 
	      | \prefix{x}{y}{(\outputp{x}{y} | @{y})} & \nonumber\\
	\red
	& (\outputp{x}{y} | @{y})\substn{\quotep{(\prefix{x}{y}{(@{y} | \outputp{x}{y})) | P}}}{y} & \nonumber\\
	=
	& \outputp{x}{\quotep{(\prefix{x}{y}{(\outputp{x}{y} | @{y})) | P}}}
	  | {(\prefix{x}{y}{(\outputp{x}{y} | @{y})) | P}} & \nonumber\\
	\red
	& \ldots & \nonumber\\
	\red^*
	& P | P | \ldots & \nonumber
\end{eqnarray}

Of course, this encoding, as an implementation, runs away, unfolding
$\bangp{P}$ eagerly. A lazier and more implementable replication
operator, restricted to input-guarded processes, may be obtained as follows.

\begin{eqnarray}
\bangp{\prefix{u}{v}{P}} 
	:= 
	\binpar{\lift{x}{\prefix{u}{v}{(\binpar{D(x)}{P})}}}{D(x)} \nonumber
\end{eqnarray}

\begin{remark}
  Note that the lazier definition still does not deal with summation
  or mixed summation (i.e. sums over input and output). The reader is
  invited to construct definitions of replication that deal with these
  features. 

  Further, the definitions are parameterized in a name, $x$. Can you,
  gentle reader, make a definition that eliminates this parameter and
  guarantees no accidental interaction between the replication
  machinery and the process being replicated -- i.e. no accidental
  sharing of names used by the process to get its work done and the
  name(s) used by the replication to effect copying. This latter
  revision of the definition of replication is crucial to obtaining
  the expected identity $!!P \sim !P$.
\end{remark}

\begin{remark}\label{rem:paradoxical_combinator}
  The reader familiar with the lambda calculus will have noticed the
  similarity between $D$ and the paradoxical combinator.

  [Ed. note: the existence of this seems to suggest we have to be more
  restrictive on the set of processes and names we admit if we are to
  support no-cloning.]
\end{remark}

\subsubsection{Bisimulation}

The computational dynamics gives rise to another kind of equivalence,
the equivalence of computational behavior. As previously mentioned
this is typically captured \emph{via} some form of bisimulation.

% The notion we use in this paper is weak barbed bisimulation
% \cite{milner91polyadicpi}.

The notion we use in this paper is derived from weak barbed
bisimulation \cite{milner91polyadicpi}. 

\begin{definition}
An \emph{observation relation}, $\downarrow_{\mathcal N}$, over a set
of names, $\mathcal N$, is the smallest relation satisfying the rules
below.

\infrule[Out-barb]{y \in {\mathcal N}, \; x \nameeq y}
		  {\outputp{x}{v} \downarrow_{\mathcal N} x}
\infrule[Par-barb]{\mbox{$P\downarrow_{\mathcal N} x$ or $Q\downarrow_{\mathcal N} x$}}
		  {\binpar{P}{Q} \downarrow_{\mathcal N} x}

We write $P \Downarrow_{\mathcal N} x$ if there is $Q$ such that 
$P \wred Q$ and $Q \downarrow_{\mathcal N} x$.
\end{definition}

\begin{definition}
%\label{def.bbisim}
An  ${\mathcal N}$-\emph{barbed bisimulation} over a set of names, ${\mathcal N}$, is a symmetric binary relation 
${\mathcal S}_{\mathcal N}$ between agents such that $P\rel{S}_{\mathcal N}Q$ implies:
\begin{enumerate}
\item If $P \red P'$ then $Q \wred Q'$ and $P'\rel{S}_{\mathcal N} Q'$.
\item If $P\downarrow_{\mathcal N} x$, then $Q\Downarrow_{\mathcal N} x$.
\end{enumerate}
$P$ is ${\mathcal N}$-barbed bisimilar to $Q$, written
$P \wbbisim_{\mathcal N} Q$, if $P \rel{S}_{\mathcal N} Q$ for some ${\mathcal N}$-barbed bisimulation ${\mathcal S}_{\mathcal N}$.
\end{definition}

$\mathcal{R} \subseteq \pi \times \pi$

$P \mathcal{R} Q => \forall P'. P \red P' \Rightarrow \exists Q'. Q \red Q', P' \mathcal{R} Q'$

$P \vdash x \Rightarrow Q \vdash x$

\begin{mathpar}
  \inferrule*[lab=Out-barb]{x \nameeq y}{{y}!\langle{Q}\rangle \vdash x}
  \and
  \inferrule*[lab=Par-barb]{\mbox{$P\vdash x$ or $Q\vdash x$}}{\binpar{P}{Q} \vdash x}
\end{mathpar}

\subsubsection{Contexts}

One of the principle advantages of computational calculi like the
$\pi$-calculus is a well-defined notion of context,
contextual-equivalence and a correlation between
contextual-equivalence and notions of bisimulation. The notion of
context allows the decomposition of a process into (sub-)process and
its syntactic environment, its context. Thus, a context may be
thought of as a process with a ``hole'' (written $\Box$) in it. The
application of a context $M$ to a process $P$, written $M[P]$, is
tantamount to filling the hole in $M$ with $P$. In this paper we do
not need the full weight of this theory, but do make use of the notion
of context in the proof the main theorem. 

\begin{mathpar}
  \inferrule* [lab=summation] {} {{M_{M},M_{N}} \bc \Box \;|\; x.M_{A} \;|\; M_{M}+M_{N}}
  \and
  \inferrule* [lab=agent] {} {{M_{A}} \bc (\vec{x})M_{P} \;| \; \clift{P_0,\ldots,M_{P},\ldots,P_N}}
  \and \\
  \inferrule* [lab=process] {} {{M_{P}} \bc M_{N} \;| \;P|M_{P} }
\end{mathpar} 

\begin{mathpar}
  \inferrule* [lab=sychronization] {} {M_{N} \bc \Box \;|\; x?M_{F} \;|\; x!M_{C}}
  \and
  \inferrule* [lab=abstraction] {} {{M_{F}} \bc (x)M_{P} }
  \and
  \inferrule* [lab=concretion] {} {{M_{C}} \bc \langle M_{P} \rangle }
  \and \\
  \inferrule* [lab=process] {} {{M_{P}} \bc M_{N} \;| \;P|M_{P} }
\end{mathpar}

\begin{definition}[contextual application] Given a context $M$, and
  process $P$, we define the \emph{contextual application}, $M[P] :=
  M\{P/\Box\}$. That is, the contextual application of M to P is the
  substitution of $P$ for $\Box$ in $M$.
\end{definition}

$\meaningof{-} : L \to \mathcal{P}(\pi)$

\begin{mathpar}
  \inferrule* [lab=collection] {} {\meaningof{true} = \pi, \and \meaningof{~E} = \pi \setminus \meaningof{E}, \and \meaningof{E_{1} \& E_{2}} = \meaningof{E_{1}} \cap \meaningof{E_{2}}}
\end{mathpar}

\begin{mathpar}
  \inferrule* [lab=structure] {} {\meaningof{0} = \{ P \in \pi | P \equiv 0 \}, \and \\ \meaningof{E_1 | E_2} = \{ P \in \pi | P \equiv P_{1} | P_{2}, P_{1} \in \meaningof{E_{1}}, P_{2} \in \meaningof{E_2}\} }
\end{mathpar}

\begin{mathpar}
 \inferrule* [lab=behavior] {} {\meaningof{\langle a?b \rangle E} = \{ P \in \pi | P \equiv Q | u?(y)P', \\ \and \\\\ \and \\ \;\;\; u \in \meaningof{a}, \forall z.P'\{z/y\} \in \meaningof{E\{z/b\}}\}, \and \\ \meaningof{a!E} = \{ P \in \pi | P \equiv Q | x!\langle P' \rangle, x \in \meaningof{a} P' \in \meaningof{E}\} }
\end{mathpar}

\begin{mathpar}
 \inferrule* [lab=nominal] {} {\meaningof{\quotep{E}} = \{ \quotep{P} \in \quotep{\pi} | P \in \meaningof{E} \}, \and \meaningof{\quotep{P}} = \{ \quotep{Q} \in \quotep{\pi} | P \equiv Q \} \and \\ \meaningof{@\quotep{E}} = \{ P \in \pi | P \equiv @x, x \in \meaningof{E} \}}
\end{mathpar}

\begin{eqnarray*}
  \\
  \meaningof{-} : TS \to ST
\end{eqnarray*}

\begin{eqnarray*}
  \\
  L : TS \to ST
\end{eqnarray*}

\begin{eqnarray*}
  \\
  P \models E \iff P \in \meaningof{E}
\end{eqnarray*}

\begin{eqnarray*}
  P \approx_{L} Q \iff \forall E \in L. P \models E \iff Q \models E
\end{eqnarray*}

\begin{eqnarray*}
  P \approx_{K} Q
\end{eqnarray*}

\begin{eqnarray*}
  P \approx Q
\end{eqnarray*}

$\approx_{K} = \approx = \approx_{L}$

\subsubsection{Contextual duality}

Note that contexts extend the quotation operation to a family of
operations from processes to names. Given a context, $M$, we can
define a \emph{nominal context}, $\quotep{M}$ by $\quotep{M}[P] :=
\quotep{M[P]}$. To foreshadow what is to come we observe that these
operations enjoy a duality with processes very much like the duality
between vectors and maps from vectors to scalars.

Further, because the calculus is essentially higher-order, we have a
correspondence between contexts and processes. More specifically,
given a name $x$ and a context $M$ we can construct $M^{*}_{x}$ such
that 

\begin{mathpar}
  M^{*}_{x} | \lift{x}{P} \red M[P]
\end{mathpar}

namely,

\begin{mathpar}
  M^{*}_{x} := x?(u).M[\dropn{u}]
\end{mathpar}

The dependence of $M^{*}_{x}$ on a name makes it an abstraction, 

\begin{mathpar}
  M^{*} := (x)x?(u).M[\dropn{u}]
\end{mathpar}

\subsection{Additional notation}

It will sometimes be convenient to denote the process a name
quotes. We already have the notation $x = \quotep{P}$, but it will be
convenient to introduce an alternate notation, $\procn{x}$, when we
want to emphasize the connection to the use of the name. Note that, by
virtue of name equivalence, $\quotep{\procn{x}} \nameeq x$; so, the
notation is consistent with previous definitions.

Further, because names have structure it is possible to effect
substitutions on the basis of that structure. This means we need to
upgrade our notation for substitutions, which we accomplish by
adapting comprehension notation. Thus,

\begin{mathpar}
  P\{ y / x : x \in S \}
\end{mathpar}

is interpreted to mean the process derived from P by replacing (in a
capture-avoiding manner) each occurrence of $x$ in $S$ by $y$. For example,

\begin{mathpar}
  P\{ \quotep{\procn{x}|\procn{x}} / x : x \in \freenames{P} \}
\end{mathpar}

will replace each (occurrence) of a free name $x$ in $P$ by
$\quotep{\procn{x}|\procn{x}}$.

Also, we will avail ourselves of the notation $x^{L}$ and $x^{R}$ to
denote injections of a name into disjoint copies of the name
space. There are numerous ways to accomplish this. One example can be
found in \cite{MeredithR05}. This notation overloads to vectors of
names: $\vec{x}^{\pi} := (x_{i}^{\pi} \; : \; 0 \leq i < |\vec{x}| )$ where $\pi \in \{L,R\}$.

We also use $P^{\Box} := P|\Box$.

In \cite{MeredithR05} an interpretation of the new operator is
given. It turns out that there are several possible interpretations
all enjoying the requisite algebraic properties of the operator (see
\cite{milner91polyadicpi}). We will therefore make liberal use of
$(\nu\; \vec{x})P$.

% subsection the_syntax_and_semantics_of_the_notation_system (end)   

\input{qm2pi.qmops} 

\input{qm2pi.sterngerlach} 

\input{qm2pi.metric} 

% section concurrent_process_calculi (end)

%\input{qm2pi.proofsketch}

% section proof sketch (end)

%\input{qm2pi.slviaknots} 

% section spatial logic via knots (end)

\input{qm2pi.conclusion}

% section conclusion (end)

%\input{qm2pi.dtcodes} 

% section wiring algorithm (end)

\input{qm2pi.ack} 

% section acknowledgments (end)

\newpage


\bibliographystyle{plain}   
\bibliography{../../biblios/main.bib}

\input{qm2pi.rhodetails}

\end{document}

 

%\documentclass[12pt]{llncs}
%\documentclass{jktr}

\usepackage[pdftex]{hyperref}                   
\usepackage {listings}
\usepackage {mathpartir}
\usepackage{bcprules}
%\usepackage{listings}
                       
\usepackage{graphicx} 
%\usepackage[margins=2.5cm,nohead,nofoot]{geometry}
%\usepackage{geometry}
\usepackage{amsfonts}
\usepackage{amstext}
\usepackage{latexsym}
\usepackage{amssymb}
\usepackage{color}


%\include{myPreamble}
\include{qm2pi.local} 

%\ifpdf
%\usepackage[pdftex]{graphicx}
%\else
%\usepackage{graphicx}
%\fi

 % \ifpdf
%  \usepackage{pdfsync}
%  \if


%\title{Brief Article}
%\author{David F. Snyder}
%\author{L.G. Meredith}

%\address{Dept. of Math., Texas State University--San Marcos, San Marcos, TX 78666}
       
\pagestyle{empty}


\begin{document}

\lstset{language=[Objective]Caml,frame=shadowbox}

\input{qm2pi.front}

% section front matter (end)

\input{qm2pi.intro} 
 
% section introduction (end)

% \input{qm2pi.knotations} 

% section notation (end)

\input{qm2pi.process.calculi} 

% section concurrent_process_calculi_and_spatial_logics_ (end)
    
%\input{qm2pi.knots2pi} 

%\input{qm2pi.trefoil} 

%\input{qm2pi.mainthm} 

% subsection basic_interpretation (end)

%\input{qm2pi.rho.presentation} 
\subsection{The syntax and semantics of the notation system}\label{sub:the_syntax_and_semantics_of_the_notation_system} % (fold)

We now summarize a technical presentation of the calculus that
embodies our theory of dynamics. The typical presentation of such a
calculus follows the style of giving generators and relations on
them. The grammar, below, describing term constructors, freely
generates the set of processes, $\Proc$. This set is then quotiented
by a relation known as structural congruence and it is over this set
that the notion of dynamics is expressed. This presentation is
essentially that of \cite{MeredithR05} with the addition of
polyadicity and summation. For readability we have relegated some of
the technical subtleties to an appendix.

\subsubsection{Process grammar}\label{subsub:process_grammar}

\begin{mathpar}
  \inferrule* [lab=synchronization] {} {{M} \bc \pzero \;|\; x?F \;|\; x!C }
  \and
  \inferrule* [lab=abstraction] {} {{F} \bc (x)P}
  \and
  \inferrule* [lab=concretion] {} {{C} \bc \langle Q \rangle}
  \and
  \inferrule* [lab=process] {} {{P,Q} \bc M \;| \;P|Q \;|\; @{x}}
  \and
  \inferrule* [lab=name] {} {{x} \bc \quotep{P}}
\end{mathpar} 

Note that $\vec{x}$ (resp. $\vec{P}$) denotes a vector of names
(resp. processes) of length $|\vec{x}|$ (resp. $|\vec{P}|$). We adopt
the following useful abbreviations.

\begin{mathpar}
   x?(\vec{y}).P := x.(\vec{y})P \and  x\clift{\vec{P}} := x.\clift{\vec{P}}
   \and x!(y) := \lift{x}{\dropn{y}}
   \and \Pi_{i=0}^{n-1}P_i := P_0 | \ldots | P_{n-1}
\end{mathpar}

\subsubsection{Structural congruence}

\paragraph{Free and bound names and alpha-equivalence.} At the
core of structural equivalence is alpha-equivalence which identifies
process that are the same up to a change of variable. Formally, we
recognize the distinction between free and bound names. The free names
of a process, $\freenames{P}$, may be calculated recursively as
follows:

\begin{mathpar}
\freenames{\pzero} := \emptyset
  \and \\
  \freenames{x?(y).P} := \{ x \} \cup (\freenames{P} \setminus \{ y \})
  \and 
  \freenames{x!\langle P \rangle} := \{ x \} \cup \{ P \} 
  \and \\
  \freenames{P|Q} := \freenames{P} \cup \freenames{Q}
  \and \\
  \freenames{@{x}} := \{ x \}
\end{mathpar}

$\pi$
$\quotep{\pi}$

$\freenames{-} : \pi \to \mathcal{P}(\quotep{\pi})$

\begin{eqnarray*}
  \freenames{\pzero} & := & \emptyset \\
  \freenames{x?(y).P} & := & \{ x \} \cup (\freenames{P} \setminus \{ y \}) \\
  \freenames{x!\langle P \rangle} & := & \{ x \} \cup \{ P \} \\
  \freenames{P|Q} & := & \freenames{P} \cup \freenames{Q} \\
  \freenames{\dropn{x}} & := & \{ x \}
\end{eqnarray*}

The bound names of a process, $\boundnames{P}$, are those names occurring in $P$
that are not free. For example, in $x?(y).0$, the name $x$ is free, while $y$ is bound.

\begin{mathpar}
  \inferrule* [lab=monoidal-laws] {} { P|Q \equiv Q|P \and P|0 \equiv P \and P|(Q|R) \equiv (P|Q)|R }
\end{mathpar}

\begin{mathpar}
  \inferrule* [lab=alpha-equivalence] {} { (x)P \equiv (y)P\{y/x\} \and y \not\in \freenames{P} }
\end{mathpar}

\begin{definition}
Then two processes, $P,Q$, are alpha-equivalent if $P = Q\{\vec{y}/\vec{x}\}$ for
some $\vec{x} \in \boundnames{Q},\vec{y} \in \boundnames{P}$, where $Q\{\vec{y}/\vec{x}\}$
denotes the capture-avoiding substitution of $\vec{y}$ for $\vec{x}$ in $Q$.
\end{definition}

\begin{definition}
  The {\em structural congruence} \cite{SangiorgiWalker} , $\equiv$,
  between processes is the least congruence containing
  alpha-equivalence, satisfying the abelian monoid laws
  (associativity, commutativity and $\pzero$ as identity) for parallel
  composition $|$ and for summation $+$.
\end{definition}

\subsection{Name equivalence}

We take name equivalence, written $\nameeq$, to be the smallest
equivalence relation generated by the following rules.

\begin{mathpar}
\inferrule*[lab=Quote-drop]
{ }
{ \quotep{@{x}} \nameeq x }

\inferrule*[lab=Struct-equiv]
{ P \scong Q }
{ \quotep{P} \nameeq \quotep{Q} }
\end{mathpar}

The astute reader will have noticed that the mutual recursion of names
and processes imposes a mutual recursion on alpha-equivalence and
structural equivalence via name-equivalence. Fortunately, all of this
works out pleasantly and we may calculate in the natural way, free of
concern. The reader interested in the details is referred to the
appendix \ref{appendix:rho_details}.

\subsection{Substitution}

We use $\Proc$ for the set of processes, $\QProc$ for the set of
names, and $\id{\{}\vec{y} / \vec{x} \id{\}}$ to denote partial maps,
$s : \QProc \rightarrow \QProc$. A map, $s$ lifts, uniquely, to a map
on process terms, $\widehat{s} : \Proc \rightarrow \Proc$ by the
following equations.

\begin{mathpar}
  (0) \psubstp{Q}{P} := 0 \\
  (R \juxtap S) \psubstp{Q}{P}
  :=    
  (R)\psubstp{Q}{P} \juxtap (S) \psubstp{Q}{P} \\
  (x?(y).R) \psubstp{Q}{P}    
  :=    
  (x)\substp{Q}{P} (z)\concat( (R \psubstn{z}{y}) \psubstp{Q}{P} ) \\
  (\lift{x}{R}) \psubstp{Q}{P}  
  :=
  \lift{(x)\substp{Q}{P}}{ R \psubstp{Q}{P} } \\
%   (\dropn{x})  \psubstp{Q}{P}       
%   := 
%   \left\{ 
%     \begin{array}{ccc} 
%       \dropn{\quotep{Q}} & & x \nameeq \quotep{P} \\
%       \dropn{x} & & otherwise \\
%     \end{array}
%   \right. 
  (\dropn{x})  \psubstp{Q}{P}       
  := 
  \left\{ 
    \begin{array}{ccc} 
      Q & & x \nameeq \quotep{P} \\
      \dropn{x} & & otherwise \\
    \end{array}
  \right.
\end{mathpar}
 

where

\begin{eqnarray}
  (x)\id{\{} \lpquote Q \rpquote / \lpquote P \rpquote \id{\}}            = 
  \left\{ 
    \begin{array}{ccc}
      \lpquote Q \rpquote & & x \nameeq \lpquote P \rpquote \\
      x & & otherwise \\
    \end{array}
  \right. \nonumber
\end{eqnarray}

and $z$ is chosen distinct from $\quotep{P}$, $\quotep{Q}$, the free
names in $Q$, and all the names in $R$. Our $\alpha$-equivalence will
be built in the standard way from this substitution.

\begin{remark}\label{rem:no_self_referential_names}
  One consequence of these definitions is that $\forall P. \quotep{P}
  \not\in \freenames{P}$.
\end{remark}

\subsection{ Dynamic quote: an example }

Anticipating something of what's to come, consider applying the
substitution, $\widehat{\id{\{}u / z \id{\}}}$, to the following pair
of processes, $\lift{w}{y!(z)}$ and $w[ \lpquote y!(z) \rpquote ]$.

\begin{eqnarray}
	\lift{w}{y!(z)}\widehat{\id{\{}u / z \id{\}}}
		& = &
		\lift{w}{y!(u)} \nonumber\\
	w[ \lpquote y!(z) \rpquote ] \widehat{ \id{\{}u / z \id{\}} }
		& = &
		w[ \lpquote y!(z) \rpquote ] \nonumber
\end{eqnarray}

Because the body of the process between quotes is impervious to
substitution, we get radically different answers. In fact, by
examining the first process in an input context,
e.g. $x?(z).\lift{w}{y!(z)}$, we see that the process under the lift
operator may be shaped by prefixed inputs binding a name inside it. In
this sense, the lift operator will be seen as a way to dynamically
construct processes before reifying them as names.

Finally equipped with these standard features we can present the
dynamics of the calculus.

\subsubsection{Operational semantics} 

Finally, we introduce the computational dynamics. What marks these
algebras as distinct from other more traditionally studied algebraic
structures, e.g. vector spaces or polynomial rings, is the manner in
which dynamics is captured. In traditional structures, dynamics is typically
expressed through morphisms between such structures, as in linear maps
between vector spaces or morphisms between rings. In algebras
associated with the semantics of computation, the dynamics is
expressed as part of the algebraic structure itself, through a
reduction reduction relation typically denoted by $\red$. Below, we
give a recursive presentation of this relation for the calculus used
in the encoding.

$\red \subseteq \pi \times \pi$
$\red : \pi \to \mathcal{P}(\pi)$

\begin{mathpar}
  \inferrule* [lab=Comm] { \textsf{match}( x_{src}, x_{trgt} ) } { x_{trgt}?(y)P \; | \; x_{src}!\langle {Q} \rangle \red P\{\quotep{Q}/y}\} }
  \and \\
  \inferrule* [lab=Par] {{P} \red {P}'} {{{P} | {Q}} \red {{P}' | {Q}}}
  \and
  \inferrule* [lab=Equiv]{{{P} \scong {P}'} \andalso {{P}' \red {Q}'} \andalso {{Q}' \scong {Q}}}{{P} \red {Q}}
\end{mathpar}

\begin{eqnarray*}
  match_{\equiv} (\quotep{P},\quotep{Q}) & := & P \equiv Q \\
  match_{\dagger}(\quotep{P},\quotep{Q}) & := & \forall R. P|Q \red^{*} R => R \red^{*} 0 \\
  match_{K}(\quotep{P},\quotep{Q}) & := & K \mbox{ for some context } K
\end{eqnarray*}

$u?(x)P | u!\langle Q \rangle \red P\{\quotep{Q}/x\}$

%We write $\wred$ for $\red^*$, and $P\red$ if $\exists Q $ such that $ P \red Q$.
We write $P\red$ if $\exists Q $ such that $ P \red Q$ and $P\not\red$, otherwise.

\section{Replication}

As mentioned before, it is known that replication (and hence
recursion) can be implemented in a higher-order process algebra
\cite{SangiorgiWalker}. As our first example of calculation with the
machinery thus far presented we give the construction explicitly in
the {\rhoc}.

\begin{eqnarray}
	D_{x} & := & \prefix{x}{y}{(\binpar{\outputp{x}{y}}{@{y}})} \nonumber\\
	\bangp_{x}{P} & := & \binpar{{x}!\langle{\binpar{D_{x}}{P}}\rangle}{D_{x}} \nonumber
\end{eqnarray}

\begin{eqnarray}
	\bangp_{x}{P} & & \nonumber\\
	=
	& {x}!\langle{(\prefix{x}{y}{(\outputp{x}{y} | @{y})) | P}}\rangle 
	      | \prefix{x}{y}{(\outputp{x}{y} | @{y})} & \nonumber\\
	\red
	& (\outputp{x}{y} | @{y})\substn{\quotep{(\prefix{x}{y}{(@{y} | \outputp{x}{y})) | P}}}{y} & \nonumber\\
	=
	& \outputp{x}{\quotep{(\prefix{x}{y}{(\outputp{x}{y} | @{y})) | P}}}
	  | {(\prefix{x}{y}{(\outputp{x}{y} | @{y})) | P}} & \nonumber\\
	\red
	& \ldots & \nonumber\\
	\red^*
	& P | P | \ldots & \nonumber
\end{eqnarray}

Of course, this encoding, as an implementation, runs away, unfolding
$\bangp{P}$ eagerly. A lazier and more implementable replication
operator, restricted to input-guarded processes, may be obtained as follows.

\begin{eqnarray}
\bangp{\prefix{u}{v}{P}} 
	:= 
	\binpar{\lift{x}{\prefix{u}{v}{(\binpar{D(x)}{P})}}}{D(x)} \nonumber
\end{eqnarray}

\begin{remark}
  Note that the lazier definition still does not deal with summation
  or mixed summation (i.e. sums over input and output). The reader is
  invited to construct definitions of replication that deal with these
  features. 

  Further, the definitions are parameterized in a name, $x$. Can you,
  gentle reader, make a definition that eliminates this parameter and
  guarantees no accidental interaction between the replication
  machinery and the process being replicated -- i.e. no accidental
  sharing of names used by the process to get its work done and the
  name(s) used by the replication to effect copying. This latter
  revision of the definition of replication is crucial to obtaining
  the expected identity $!!P \sim !P$.
\end{remark}

\begin{remark}\label{rem:paradoxical_combinator}
  The reader familiar with the lambda calculus will have noticed the
  similarity between $D$ and the paradoxical combinator.

  [Ed. note: the existence of this seems to suggest we have to be more
  restrictive on the set of processes and names we admit if we are to
  support no-cloning.]
\end{remark}

\subsubsection{Bisimulation}

The computational dynamics gives rise to another kind of equivalence,
the equivalence of computational behavior. As previously mentioned
this is typically captured \emph{via} some form of bisimulation.

% The notion we use in this paper is weak barbed bisimulation
% \cite{milner91polyadicpi}.

The notion we use in this paper is derived from weak barbed
bisimulation \cite{milner91polyadicpi}. 

\begin{definition}
An \emph{observation relation}, $\downarrow_{\mathcal N}$, over a set
of names, $\mathcal N$, is the smallest relation satisfying the rules
below.

\infrule[Out-barb]{y \in {\mathcal N}, \; x \nameeq y}
		  {\outputp{x}{v} \downarrow_{\mathcal N} x}
\infrule[Par-barb]{\mbox{$P\downarrow_{\mathcal N} x$ or $Q\downarrow_{\mathcal N} x$}}
		  {\binpar{P}{Q} \downarrow_{\mathcal N} x}

We write $P \Downarrow_{\mathcal N} x$ if there is $Q$ such that 
$P \wred Q$ and $Q \downarrow_{\mathcal N} x$.
\end{definition}

\begin{definition}
%\label{def.bbisim}
An  ${\mathcal N}$-\emph{barbed bisimulation} over a set of names, ${\mathcal N}$, is a symmetric binary relation 
${\mathcal S}_{\mathcal N}$ between agents such that $P\rel{S}_{\mathcal N}Q$ implies:
\begin{enumerate}
\item If $P \red P'$ then $Q \wred Q'$ and $P'\rel{S}_{\mathcal N} Q'$.
\item If $P\downarrow_{\mathcal N} x$, then $Q\Downarrow_{\mathcal N} x$.
\end{enumerate}
$P$ is ${\mathcal N}$-barbed bisimilar to $Q$, written
$P \wbbisim_{\mathcal N} Q$, if $P \rel{S}_{\mathcal N} Q$ for some ${\mathcal N}$-barbed bisimulation ${\mathcal S}_{\mathcal N}$.
\end{definition}

$\mathcal{R} \subseteq \pi \times \pi$

$P \mathcal{R} Q => \forall P'. P \red P' \Rightarrow \exists Q'. Q \red Q', P' \mathcal{R} Q'$

$P \vdash x \Rightarrow Q \vdash x$

\begin{mathpar}
  \inferrule*[lab=Out-barb]{x \nameeq y}{{y}!\langle{Q}\rangle \vdash x}
  \and
  \inferrule*[lab=Par-barb]{\mbox{$P\vdash x$ or $Q\vdash x$}}{\binpar{P}{Q} \vdash x}
\end{mathpar}

\subsubsection{Contexts}

One of the principle advantages of computational calculi like the
$\pi$-calculus is a well-defined notion of context,
contextual-equivalence and a correlation between
contextual-equivalence and notions of bisimulation. The notion of
context allows the decomposition of a process into (sub-)process and
its syntactic environment, its context. Thus, a context may be
thought of as a process with a ``hole'' (written $\Box$) in it. The
application of a context $M$ to a process $P$, written $M[P]$, is
tantamount to filling the hole in $M$ with $P$. In this paper we do
not need the full weight of this theory, but do make use of the notion
of context in the proof the main theorem. 

\begin{mathpar}
  \inferrule* [lab=summation] {} {{M_{M},M_{N}} \bc \Box \;|\; x.M_{A} \;|\; M_{M}+M_{N}}
  \and
  \inferrule* [lab=agent] {} {{M_{A}} \bc (\vec{x})M_{P} \;| \; \clift{P_0,\ldots,M_{P},\ldots,P_N}}
  \and \\
  \inferrule* [lab=process] {} {{M_{P}} \bc M_{N} \;| \;P|M_{P} }
\end{mathpar} 

\begin{mathpar}
  \inferrule* [lab=sychronization] {} {M_{N} \bc \Box \;|\; x?M_{F} \;|\; x!M_{C}}
  \and
  \inferrule* [lab=abstraction] {} {{M_{F}} \bc (x)M_{P} }
  \and
  \inferrule* [lab=concretion] {} {{M_{C}} \bc \langle M_{P} \rangle }
  \and \\
  \inferrule* [lab=process] {} {{M_{P}} \bc M_{N} \;| \;P|M_{P} }
\end{mathpar}

\begin{definition}[contextual application] Given a context $M$, and
  process $P$, we define the \emph{contextual application}, $M[P] :=
  M\{P/\Box\}$. That is, the contextual application of M to P is the
  substitution of $P$ for $\Box$ in $M$.
\end{definition}

$\meaningof{-} : L \to \mathcal{P}(\pi)$

\begin{mathpar}
  \inferrule* [lab=collection] {} {\meaningof{true} = \pi, \and \meaningof{~E} = \pi \setminus \meaningof{E}, \and \meaningof{E_{1} \& E_{2}} = \meaningof{E_{1}} \cap \meaningof{E_{2}}}
\end{mathpar}

\begin{mathpar}
  \inferrule* [lab=structure] {} {\meaningof{0} = \{ P \in \pi | P \equiv 0 \}, \and \\ \meaningof{E_1 | E_2} = \{ P \in \pi | P \equiv P_{1} | P_{2}, P_{1} \in \meaningof{E_{1}}, P_{2} \in \meaningof{E_2}\} }
\end{mathpar}

\begin{mathpar}
 \inferrule* [lab=behavior] {} {\meaningof{\langle a?b \rangle E} = \{ P \in \pi | P \equiv Q | u?(y)P', \\ \and \\\\ \and \\ \;\;\; u \in \meaningof{a}, \forall z.P'\{z/y\} \in \meaningof{E\{z/b\}}\}, \and \\ \meaningof{a!E} = \{ P \in \pi | P \equiv Q | x!\langle P' \rangle, x \in \meaningof{a} P' \in \meaningof{E}\} }
\end{mathpar}

\begin{mathpar}
 \inferrule* [lab=nominal] {} {\meaningof{\quotep{E}} = \{ \quotep{P} \in \quotep{\pi} | P \in \meaningof{E} \}, \and \meaningof{\quotep{P}} = \{ \quotep{Q} \in \quotep{\pi} | P \equiv Q \} \and \\ \meaningof{@\quotep{E}} = \{ P \in \pi | P \equiv @x, x \in \meaningof{E} \}}
\end{mathpar}

\begin{eqnarray*}
  \\
  \meaningof{-} : TS \to ST
\end{eqnarray*}

\begin{eqnarray*}
  \\
  L : TS \to ST
\end{eqnarray*}

\begin{eqnarray*}
  \\
  P \models E \iff P \in \meaningof{E}
\end{eqnarray*}

\begin{eqnarray*}
  P \approx_{L} Q \iff \forall E \in L. P \models E \iff Q \models E
\end{eqnarray*}

\begin{eqnarray*}
  P \approx_{K} Q
\end{eqnarray*}

\begin{eqnarray*}
  P \approx Q
\end{eqnarray*}

$\approx_{K} = \approx = \approx_{L}$

\subsubsection{Contextual duality}

Note that contexts extend the quotation operation to a family of
operations from processes to names. Given a context, $M$, we can
define a \emph{nominal context}, $\quotep{M}$ by $\quotep{M}[P] :=
\quotep{M[P]}$. To foreshadow what is to come we observe that these
operations enjoy a duality with processes very much like the duality
between vectors and maps from vectors to scalars.

Further, because the calculus is essentially higher-order, we have a
correspondence between contexts and processes. More specifically,
given a name $x$ and a context $M$ we can construct $M^{*}_{x}$ such
that 

\begin{mathpar}
  M^{*}_{x} | \lift{x}{P} \red M[P]
\end{mathpar}

namely,

\begin{mathpar}
  M^{*}_{x} := x?(u).M[\dropn{u}]
\end{mathpar}

The dependence of $M^{*}_{x}$ on a name makes it an abstraction, 

\begin{mathpar}
  M^{*} := (x)x?(u).M[\dropn{u}]
\end{mathpar}

\subsection{Additional notation}

It will sometimes be convenient to denote the process a name
quotes. We already have the notation $x = \quotep{P}$, but it will be
convenient to introduce an alternate notation, $\procn{x}$, when we
want to emphasize the connection to the use of the name. Note that, by
virtue of name equivalence, $\quotep{\procn{x}} \nameeq x$; so, the
notation is consistent with previous definitions.

Further, because names have structure it is possible to effect
substitutions on the basis of that structure. This means we need to
upgrade our notation for substitutions, which we accomplish by
adapting comprehension notation. Thus,

\begin{mathpar}
  P\{ y / x : x \in S \}
\end{mathpar}

is interpreted to mean the process derived from P by replacing (in a
capture-avoiding manner) each occurrence of $x$ in $S$ by $y$. For example,

\begin{mathpar}
  P\{ \quotep{\procn{x}|\procn{x}} / x : x \in \freenames{P} \}
\end{mathpar}

will replace each (occurrence) of a free name $x$ in $P$ by
$\quotep{\procn{x}|\procn{x}}$.

Also, we will avail ourselves of the notation $x^{L}$ and $x^{R}$ to
denote injections of a name into disjoint copies of the name
space. There are numerous ways to accomplish this. One example can be
found in \cite{MeredithR05}. This notation overloads to vectors of
names: $\vec{x}^{\pi} := (x_{i}^{\pi} \; : \; 0 \leq i < |\vec{x}| )$ where $\pi \in \{L,R\}$.

We also use $P^{\Box} := P|\Box$.

In \cite{MeredithR05} an interpretation of the new operator is
given. It turns out that there are several possible interpretations
all enjoying the requisite algebraic properties of the operator (see
\cite{milner91polyadicpi}). We will therefore make liberal use of
$(\nu\; \vec{x})P$.

% subsection the_syntax_and_semantics_of_the_notation_system (end)   

\input{qm2pi.qmops} 

\input{qm2pi.sterngerlach} 

\input{qm2pi.metric} 

% section concurrent_process_calculi (end)

%\input{qm2pi.proofsketch}

% section proof sketch (end)

%\input{qm2pi.slviaknots} 

% section spatial logic via knots (end)

\input{qm2pi.conclusion}

% section conclusion (end)

%\input{qm2pi.dtcodes} 

% section wiring algorithm (end)

\input{qm2pi.ack} 

% section acknowledgments (end)

\newpage


\bibliographystyle{plain}   
\bibliography{../../biblios/main.bib}

\input{qm2pi.rhodetails}

\end{document}

 

% subsection basic_interpretation (end)

%\input{qm2pi.rho.presentation} 
\subsection{The syntax and semantics of the notation system}\label{sub:the_syntax_and_semantics_of_the_notation_system} % (fold)

We now summarize a technical presentation of the calculus that
embodies our theory of dynamics. The typical presentation of such a
calculus follows the style of giving generators and relations on
them. The grammar, below, describing term constructors, freely
generates the set of processes, $\Proc$. This set is then quotiented
by a relation known as structural congruence and it is over this set
that the notion of dynamics is expressed. This presentation is
essentially that of \cite{MeredithR05} with the addition of
polyadicity and summation. For readability we have relegated some of
the technical subtleties to an appendix.

\subsubsection{Process grammar}\label{subsub:process_grammar}

\begin{mathpar}
  \inferrule* [lab=synchronization] {} {{M} \bc \pzero \;|\; x?F \;|\; x!C }
  \and
  \inferrule* [lab=abstraction] {} {{F} \bc (x)P}
  \and
  \inferrule* [lab=concretion] {} {{C} \bc \langle Q \rangle}
  \and
  \inferrule* [lab=process] {} {{P,Q} \bc M \;| \;P|Q \;|\; @{x}}
  \and
  \inferrule* [lab=name] {} {{x} \bc \quotep{P}}
\end{mathpar} 

Note that $\vec{x}$ (resp. $\vec{P}$) denotes a vector of names
(resp. processes) of length $|\vec{x}|$ (resp. $|\vec{P}|$). We adopt
the following useful abbreviations.

\begin{mathpar}
   x?(\vec{y}).P := x.(\vec{y})P \and  x\clift{\vec{P}} := x.\clift{\vec{P}}
   \and x!(y) := \lift{x}{\dropn{y}}
   \and \Pi_{i=0}^{n-1}P_i := P_0 | \ldots | P_{n-1}
\end{mathpar}

\subsubsection{Structural congruence}

\paragraph{Free and bound names and alpha-equivalence.} At the
core of structural equivalence is alpha-equivalence which identifies
process that are the same up to a change of variable. Formally, we
recognize the distinction between free and bound names. The free names
of a process, $\freenames{P}$, may be calculated recursively as
follows:

\begin{mathpar}
\freenames{\pzero} := \emptyset
  \and \\
  \freenames{x?(y).P} := \{ x \} \cup (\freenames{P} \setminus \{ y \})
  \and 
  \freenames{x!\langle P \rangle} := \{ x \} \cup \{ P \} 
  \and \\
  \freenames{P|Q} := \freenames{P} \cup \freenames{Q}
  \and \\
  \freenames{@{x}} := \{ x \}
\end{mathpar}

$\pi$
$\quotep{\pi}$

$\freenames{-} : \pi \to \mathcal{P}(\quotep{\pi})$

\begin{eqnarray*}
  \freenames{\pzero} & := & \emptyset \\
  \freenames{x?(y).P} & := & \{ x \} \cup (\freenames{P} \setminus \{ y \}) \\
  \freenames{x!\langle P \rangle} & := & \{ x \} \cup \{ P \} \\
  \freenames{P|Q} & := & \freenames{P} \cup \freenames{Q} \\
  \freenames{\dropn{x}} & := & \{ x \}
\end{eqnarray*}

The bound names of a process, $\boundnames{P}$, are those names occurring in $P$
that are not free. For example, in $x?(y).0$, the name $x$ is free, while $y$ is bound.

\begin{mathpar}
  \inferrule* [lab=monoidal-laws] {} { P|Q \equiv Q|P \and P|0 \equiv P \and P|(Q|R) \equiv (P|Q)|R }
\end{mathpar}

\begin{mathpar}
  \inferrule* [lab=alpha-equivalence] {} { (x)P \equiv (y)P\{y/x\} \and y \not\in \freenames{P} }
\end{mathpar}

\begin{definition}
Then two processes, $P,Q$, are alpha-equivalent if $P = Q\{\vec{y}/\vec{x}\}$ for
some $\vec{x} \in \boundnames{Q},\vec{y} \in \boundnames{P}$, where $Q\{\vec{y}/\vec{x}\}$
denotes the capture-avoiding substitution of $\vec{y}$ for $\vec{x}$ in $Q$.
\end{definition}

\begin{definition}
  The {\em structural congruence} \cite{SangiorgiWalker} , $\equiv$,
  between processes is the least congruence containing
  alpha-equivalence, satisfying the abelian monoid laws
  (associativity, commutativity and $\pzero$ as identity) for parallel
  composition $|$ and for summation $+$.
\end{definition}

\subsection{Name equivalence}

We take name equivalence, written $\nameeq$, to be the smallest
equivalence relation generated by the following rules.

\begin{mathpar}
\inferrule*[lab=Quote-drop]
{ }
{ \quotep{@{x}} \nameeq x }

\inferrule*[lab=Struct-equiv]
{ P \scong Q }
{ \quotep{P} \nameeq \quotep{Q} }
\end{mathpar}

The astute reader will have noticed that the mutual recursion of names
and processes imposes a mutual recursion on alpha-equivalence and
structural equivalence via name-equivalence. Fortunately, all of this
works out pleasantly and we may calculate in the natural way, free of
concern. The reader interested in the details is referred to the
appendix \ref{appendix:rho_details}.

\subsection{Substitution}

We use $\Proc$ for the set of processes, $\QProc$ for the set of
names, and $\id{\{}\vec{y} / \vec{x} \id{\}}$ to denote partial maps,
$s : \QProc \rightarrow \QProc$. A map, $s$ lifts, uniquely, to a map
on process terms, $\widehat{s} : \Proc \rightarrow \Proc$ by the
following equations.

\begin{mathpar}
  (0) \psubstp{Q}{P} := 0 \\
  (R \juxtap S) \psubstp{Q}{P}
  :=    
  (R)\psubstp{Q}{P} \juxtap (S) \psubstp{Q}{P} \\
  (x?(y).R) \psubstp{Q}{P}    
  :=    
  (x)\substp{Q}{P} (z)\concat( (R \psubstn{z}{y}) \psubstp{Q}{P} ) \\
  (\lift{x}{R}) \psubstp{Q}{P}  
  :=
  \lift{(x)\substp{Q}{P}}{ R \psubstp{Q}{P} } \\
%   (\dropn{x})  \psubstp{Q}{P}       
%   := 
%   \left\{ 
%     \begin{array}{ccc} 
%       \dropn{\quotep{Q}} & & x \nameeq \quotep{P} \\
%       \dropn{x} & & otherwise \\
%     \end{array}
%   \right. 
  (\dropn{x})  \psubstp{Q}{P}       
  := 
  \left\{ 
    \begin{array}{ccc} 
      Q & & x \nameeq \quotep{P} \\
      \dropn{x} & & otherwise \\
    \end{array}
  \right.
\end{mathpar}
 

where

\begin{eqnarray}
  (x)\id{\{} \lpquote Q \rpquote / \lpquote P \rpquote \id{\}}            = 
  \left\{ 
    \begin{array}{ccc}
      \lpquote Q \rpquote & & x \nameeq \lpquote P \rpquote \\
      x & & otherwise \\
    \end{array}
  \right. \nonumber
\end{eqnarray}

and $z$ is chosen distinct from $\quotep{P}$, $\quotep{Q}$, the free
names in $Q$, and all the names in $R$. Our $\alpha$-equivalence will
be built in the standard way from this substitution.

\begin{remark}\label{rem:no_self_referential_names}
  One consequence of these definitions is that $\forall P. \quotep{P}
  \not\in \freenames{P}$.
\end{remark}

\subsection{ Dynamic quote: an example }

Anticipating something of what's to come, consider applying the
substitution, $\widehat{\id{\{}u / z \id{\}}}$, to the following pair
of processes, $\lift{w}{y!(z)}$ and $w[ \lpquote y!(z) \rpquote ]$.

\begin{eqnarray}
	\lift{w}{y!(z)}\widehat{\id{\{}u / z \id{\}}}
		& = &
		\lift{w}{y!(u)} \nonumber\\
	w[ \lpquote y!(z) \rpquote ] \widehat{ \id{\{}u / z \id{\}} }
		& = &
		w[ \lpquote y!(z) \rpquote ] \nonumber
\end{eqnarray}

Because the body of the process between quotes is impervious to
substitution, we get radically different answers. In fact, by
examining the first process in an input context,
e.g. $x?(z).\lift{w}{y!(z)}$, we see that the process under the lift
operator may be shaped by prefixed inputs binding a name inside it. In
this sense, the lift operator will be seen as a way to dynamically
construct processes before reifying them as names.

Finally equipped with these standard features we can present the
dynamics of the calculus.

\subsubsection{Operational semantics} 

Finally, we introduce the computational dynamics. What marks these
algebras as distinct from other more traditionally studied algebraic
structures, e.g. vector spaces or polynomial rings, is the manner in
which dynamics is captured. In traditional structures, dynamics is typically
expressed through morphisms between such structures, as in linear maps
between vector spaces or morphisms between rings. In algebras
associated with the semantics of computation, the dynamics is
expressed as part of the algebraic structure itself, through a
reduction reduction relation typically denoted by $\red$. Below, we
give a recursive presentation of this relation for the calculus used
in the encoding.

$\red \subseteq \pi \times \pi$
$\red : \pi \to \mathcal{P}(\pi)$

\begin{mathpar}
  \inferrule* [lab=Comm] { \textsf{match}( x_{src}, x_{trgt} ) } { x_{trgt}?(y)P \; | \; x_{src}!\langle {Q} \rangle \red P\{\quotep{Q}/y}\} }
  \and \\
  \inferrule* [lab=Par] {{P} \red {P}'} {{{P} | {Q}} \red {{P}' | {Q}}}
  \and
  \inferrule* [lab=Equiv]{{{P} \scong {P}'} \andalso {{P}' \red {Q}'} \andalso {{Q}' \scong {Q}}}{{P} \red {Q}}
\end{mathpar}

\begin{eqnarray*}
  match_{\equiv} (\quotep{P},\quotep{Q}) & := & P \equiv Q \\
  match_{\dagger}(\quotep{P},\quotep{Q}) & := & \forall R. P|Q \red^{*} R => R \red^{*} 0 \\
  match_{K}(\quotep{P},\quotep{Q}) & := & K \mbox{ for some context } K
\end{eqnarray*}

$u?(x)P | u!\langle Q \rangle \red P\{\quotep{Q}/x\}$

%We write $\wred$ for $\red^*$, and $P\red$ if $\exists Q $ such that $ P \red Q$.
We write $P\red$ if $\exists Q $ such that $ P \red Q$ and $P\not\red$, otherwise.

\section{Replication}

As mentioned before, it is known that replication (and hence
recursion) can be implemented in a higher-order process algebra
\cite{SangiorgiWalker}. As our first example of calculation with the
machinery thus far presented we give the construction explicitly in
the {\rhoc}.

\begin{eqnarray}
	D_{x} & := & \prefix{x}{y}{(\binpar{\outputp{x}{y}}{@{y}})} \nonumber\\
	\bangp_{x}{P} & := & \binpar{{x}!\langle{\binpar{D_{x}}{P}}\rangle}{D_{x}} \nonumber
\end{eqnarray}

\begin{eqnarray}
	\bangp_{x}{P} & & \nonumber\\
	=
	& {x}!\langle{(\prefix{x}{y}{(\outputp{x}{y} | @{y})) | P}}\rangle 
	      | \prefix{x}{y}{(\outputp{x}{y} | @{y})} & \nonumber\\
	\red
	& (\outputp{x}{y} | @{y})\substn{\quotep{(\prefix{x}{y}{(@{y} | \outputp{x}{y})) | P}}}{y} & \nonumber\\
	=
	& \outputp{x}{\quotep{(\prefix{x}{y}{(\outputp{x}{y} | @{y})) | P}}}
	  | {(\prefix{x}{y}{(\outputp{x}{y} | @{y})) | P}} & \nonumber\\
	\red
	& \ldots & \nonumber\\
	\red^*
	& P | P | \ldots & \nonumber
\end{eqnarray}

Of course, this encoding, as an implementation, runs away, unfolding
$\bangp{P}$ eagerly. A lazier and more implementable replication
operator, restricted to input-guarded processes, may be obtained as follows.

\begin{eqnarray}
\bangp{\prefix{u}{v}{P}} 
	:= 
	\binpar{\lift{x}{\prefix{u}{v}{(\binpar{D(x)}{P})}}}{D(x)} \nonumber
\end{eqnarray}

\begin{remark}
  Note that the lazier definition still does not deal with summation
  or mixed summation (i.e. sums over input and output). The reader is
  invited to construct definitions of replication that deal with these
  features. 

  Further, the definitions are parameterized in a name, $x$. Can you,
  gentle reader, make a definition that eliminates this parameter and
  guarantees no accidental interaction between the replication
  machinery and the process being replicated -- i.e. no accidental
  sharing of names used by the process to get its work done and the
  name(s) used by the replication to effect copying. This latter
  revision of the definition of replication is crucial to obtaining
  the expected identity $!!P \sim !P$.
\end{remark}

\begin{remark}\label{rem:paradoxical_combinator}
  The reader familiar with the lambda calculus will have noticed the
  similarity between $D$ and the paradoxical combinator.

  [Ed. note: the existence of this seems to suggest we have to be more
  restrictive on the set of processes and names we admit if we are to
  support no-cloning.]
\end{remark}

\subsubsection{Bisimulation}

The computational dynamics gives rise to another kind of equivalence,
the equivalence of computational behavior. As previously mentioned
this is typically captured \emph{via} some form of bisimulation.

% The notion we use in this paper is weak barbed bisimulation
% \cite{milner91polyadicpi}.

The notion we use in this paper is derived from weak barbed
bisimulation \cite{milner91polyadicpi}. 

\begin{definition}
An \emph{observation relation}, $\downarrow_{\mathcal N}$, over a set
of names, $\mathcal N$, is the smallest relation satisfying the rules
below.

\infrule[Out-barb]{y \in {\mathcal N}, \; x \nameeq y}
		  {\outputp{x}{v} \downarrow_{\mathcal N} x}
\infrule[Par-barb]{\mbox{$P\downarrow_{\mathcal N} x$ or $Q\downarrow_{\mathcal N} x$}}
		  {\binpar{P}{Q} \downarrow_{\mathcal N} x}

We write $P \Downarrow_{\mathcal N} x$ if there is $Q$ such that 
$P \wred Q$ and $Q \downarrow_{\mathcal N} x$.
\end{definition}

\begin{definition}
%\label{def.bbisim}
An  ${\mathcal N}$-\emph{barbed bisimulation} over a set of names, ${\mathcal N}$, is a symmetric binary relation 
${\mathcal S}_{\mathcal N}$ between agents such that $P\rel{S}_{\mathcal N}Q$ implies:
\begin{enumerate}
\item If $P \red P'$ then $Q \wred Q'$ and $P'\rel{S}_{\mathcal N} Q'$.
\item If $P\downarrow_{\mathcal N} x$, then $Q\Downarrow_{\mathcal N} x$.
\end{enumerate}
$P$ is ${\mathcal N}$-barbed bisimilar to $Q$, written
$P \wbbisim_{\mathcal N} Q$, if $P \rel{S}_{\mathcal N} Q$ for some ${\mathcal N}$-barbed bisimulation ${\mathcal S}_{\mathcal N}$.
\end{definition}

$\mathcal{R} \subseteq \pi \times \pi$

$P \mathcal{R} Q => \forall P'. P \red P' \Rightarrow \exists Q'. Q \red Q', P' \mathcal{R} Q'$

$P \vdash x \Rightarrow Q \vdash x$

\begin{mathpar}
  \inferrule*[lab=Out-barb]{x \nameeq y}{{y}!\langle{Q}\rangle \vdash x}
  \and
  \inferrule*[lab=Par-barb]{\mbox{$P\vdash x$ or $Q\vdash x$}}{\binpar{P}{Q} \vdash x}
\end{mathpar}

\subsubsection{Contexts}

One of the principle advantages of computational calculi like the
$\pi$-calculus is a well-defined notion of context,
contextual-equivalence and a correlation between
contextual-equivalence and notions of bisimulation. The notion of
context allows the decomposition of a process into (sub-)process and
its syntactic environment, its context. Thus, a context may be
thought of as a process with a ``hole'' (written $\Box$) in it. The
application of a context $M$ to a process $P$, written $M[P]$, is
tantamount to filling the hole in $M$ with $P$. In this paper we do
not need the full weight of this theory, but do make use of the notion
of context in the proof the main theorem. 

\begin{mathpar}
  \inferrule* [lab=summation] {} {{M_{M},M_{N}} \bc \Box \;|\; x.M_{A} \;|\; M_{M}+M_{N}}
  \and
  \inferrule* [lab=agent] {} {{M_{A}} \bc (\vec{x})M_{P} \;| \; \clift{P_0,\ldots,M_{P},\ldots,P_N}}
  \and \\
  \inferrule* [lab=process] {} {{M_{P}} \bc M_{N} \;| \;P|M_{P} }
\end{mathpar} 

\begin{mathpar}
  \inferrule* [lab=sychronization] {} {M_{N} \bc \Box \;|\; x?M_{F} \;|\; x!M_{C}}
  \and
  \inferrule* [lab=abstraction] {} {{M_{F}} \bc (x)M_{P} }
  \and
  \inferrule* [lab=concretion] {} {{M_{C}} \bc \langle M_{P} \rangle }
  \and \\
  \inferrule* [lab=process] {} {{M_{P}} \bc M_{N} \;| \;P|M_{P} }
\end{mathpar}

\begin{definition}[contextual application] Given a context $M$, and
  process $P$, we define the \emph{contextual application}, $M[P] :=
  M\{P/\Box\}$. That is, the contextual application of M to P is the
  substitution of $P$ for $\Box$ in $M$.
\end{definition}

$\meaningof{-} : L \to \mathcal{P}(\pi)$

\begin{mathpar}
  \inferrule* [lab=collection] {} {\meaningof{true} = \pi, \and \meaningof{~E} = \pi \setminus \meaningof{E}, \and \meaningof{E_{1} \& E_{2}} = \meaningof{E_{1}} \cap \meaningof{E_{2}}}
\end{mathpar}

\begin{mathpar}
  \inferrule* [lab=structure] {} {\meaningof{0} = \{ P \in \pi | P \equiv 0 \}, \and \\ \meaningof{E_1 | E_2} = \{ P \in \pi | P \equiv P_{1} | P_{2}, P_{1} \in \meaningof{E_{1}}, P_{2} \in \meaningof{E_2}\} }
\end{mathpar}

\begin{mathpar}
 \inferrule* [lab=behavior] {} {\meaningof{\langle a?b \rangle E} = \{ P \in \pi | P \equiv Q | u?(y)P', \\ \and \\\\ \and \\ \;\;\; u \in \meaningof{a}, \forall z.P'\{z/y\} \in \meaningof{E\{z/b\}}\}, \and \\ \meaningof{a!E} = \{ P \in \pi | P \equiv Q | x!\langle P' \rangle, x \in \meaningof{a} P' \in \meaningof{E}\} }
\end{mathpar}

\begin{mathpar}
 \inferrule* [lab=nominal] {} {\meaningof{\quotep{E}} = \{ \quotep{P} \in \quotep{\pi} | P \in \meaningof{E} \}, \and \meaningof{\quotep{P}} = \{ \quotep{Q} \in \quotep{\pi} | P \equiv Q \} \and \\ \meaningof{@\quotep{E}} = \{ P \in \pi | P \equiv @x, x \in \meaningof{E} \}}
\end{mathpar}

\begin{eqnarray*}
  \\
  \meaningof{-} : TS \to ST
\end{eqnarray*}

\begin{eqnarray*}
  \\
  L : TS \to ST
\end{eqnarray*}

\begin{eqnarray*}
  \\
  P \models E \iff P \in \meaningof{E}
\end{eqnarray*}

\begin{eqnarray*}
  P \approx_{L} Q \iff \forall E \in L. P \models E \iff Q \models E
\end{eqnarray*}

\begin{eqnarray*}
  P \approx_{K} Q
\end{eqnarray*}

\begin{eqnarray*}
  P \approx Q
\end{eqnarray*}

$\approx_{K} = \approx = \approx_{L}$

\subsubsection{Contextual duality}

Note that contexts extend the quotation operation to a family of
operations from processes to names. Given a context, $M$, we can
define a \emph{nominal context}, $\quotep{M}$ by $\quotep{M}[P] :=
\quotep{M[P]}$. To foreshadow what is to come we observe that these
operations enjoy a duality with processes very much like the duality
between vectors and maps from vectors to scalars.

Further, because the calculus is essentially higher-order, we have a
correspondence between contexts and processes. More specifically,
given a name $x$ and a context $M$ we can construct $M^{*}_{x}$ such
that 

\begin{mathpar}
  M^{*}_{x} | \lift{x}{P} \red M[P]
\end{mathpar}

namely,

\begin{mathpar}
  M^{*}_{x} := x?(u).M[\dropn{u}]
\end{mathpar}

The dependence of $M^{*}_{x}$ on a name makes it an abstraction, 

\begin{mathpar}
  M^{*} := (x)x?(u).M[\dropn{u}]
\end{mathpar}

\subsection{Additional notation}

It will sometimes be convenient to denote the process a name
quotes. We already have the notation $x = \quotep{P}$, but it will be
convenient to introduce an alternate notation, $\procn{x}$, when we
want to emphasize the connection to the use of the name. Note that, by
virtue of name equivalence, $\quotep{\procn{x}} \nameeq x$; so, the
notation is consistent with previous definitions.

Further, because names have structure it is possible to effect
substitutions on the basis of that structure. This means we need to
upgrade our notation for substitutions, which we accomplish by
adapting comprehension notation. Thus,

\begin{mathpar}
  P\{ y / x : x \in S \}
\end{mathpar}

is interpreted to mean the process derived from P by replacing (in a
capture-avoiding manner) each occurrence of $x$ in $S$ by $y$. For example,

\begin{mathpar}
  P\{ \quotep{\procn{x}|\procn{x}} / x : x \in \freenames{P} \}
\end{mathpar}

will replace each (occurrence) of a free name $x$ in $P$ by
$\quotep{\procn{x}|\procn{x}}$.

Also, we will avail ourselves of the notation $x^{L}$ and $x^{R}$ to
denote injections of a name into disjoint copies of the name
space. There are numerous ways to accomplish this. One example can be
found in \cite{MeredithR05}. This notation overloads to vectors of
names: $\vec{x}^{\pi} := (x_{i}^{\pi} \; : \; 0 \leq i < |\vec{x}| )$ where $\pi \in \{L,R\}$.

We also use $P^{\Box} := P|\Box$.

In \cite{MeredithR05} an interpretation of the new operator is
given. It turns out that there are several possible interpretations
all enjoying the requisite algebraic properties of the operator (see
\cite{milner91polyadicpi}). We will therefore make liberal use of
$(\nu\; \vec{x})P$.

% subsection the_syntax_and_semantics_of_the_notation_system (end)   

\section{Interpretation of QM}
\subsection{Supporting definitions}
\subsubsection{Multiplication}
\begin{mathpar}
  \quotep{Q} \cdot \quotep{R} := \quotep{Q|R}
  \and \\
  \quotep{Q} \cdot P := P\{ \quotep{Q|R} / \quotep{R} : \quotep{R} \in \freenames{P} \}
\end{mathpar}

\paragraph{Discussion}
The first line needs little explanation. The second line says that
each free name of the process is replaced with the multiplication of
that name by the scalar. Multiplication of a scalar (name) by a state
(process) results in a process all the names of which have been `moved
over' by parallel composition with the process the scalar
quotes. There is a subtlety that the bound names have to be
manipulated so that multiplied names aren't accidentally
captured. There are many ways to achieve this.

\begin{remark}\label{rem:multiplication_identities}
  The reader is invited to verify that for all $x,y,z \in \QProc$ and $P \in \Proc$
  \begin{mathpar}
    x \cdot \quotep{0} \equiv x 
    \and
    x \cdot y \equiv y \cdot x
    \and
    x \cdot (y \cdot z) \equiv (x \cdot y) \cdot z
    \and \\
    \quotep{0} \cdot P \equiv P
    \and \\
    x \cdot (y \cdot P) \equiv (x \cdot y) \cdot P
    \and \\
    x \cdot (P|Q) \equiv (x \cdot P) | (x \cdot Q)
    \and \\    
  \end{mathpar}
\end{remark}

\subsubsection{Tensor product}

We define a tensor product on processes by structural induction.

\paragraph{Tensor of sums} First note that all summations, including
$\pzero$ and sequence, can be written $\Sigma_{i} x_{i}.A_{i} +
\Sigma_{j} x_{j}.C_{j}$, where we have grouped input-guarded processes
together and output-guarded processes together.

Thus, we can define the tensor product of two summations, $N_{1}\otimes N_{2}$, where

\begin{mathpar}
  N_{1} := \Sigma_{i} x_{i}.A_{i} + \Sigma_{j} x_{j}.C_{j}
  \and
  N_{2} := \Sigma_{i'} y_{i'}.B_{i'} + \Sigma_{j'} y_{j'}.D_{j'} 
\end{mathpar}

as follows.

\begin{mathpar}
  \Sigma_{i} x_{i}.A_{i} + \Sigma_{j} x_{j}.C_{j} \otimes \Sigma_{i'}
  y_{i'}.B_{i'} + \Sigma_{j'} y_{j'}.D_{j'} 
  \and \\
  := \; \Sigma_{i} \Sigma_{i'} \quotep{\stackrel{\vee}{x_{i}}| \stackrel{\vee}{y_{i'}}}.(A_{i}\otimes B_{i'}) \; | \; \Sigma_{i'} \Sigma_{i} \quotep{\stackrel{\vee}{y_{i'}}|\stackrel{\vee}{x_{i}}}.(B_{i'}\otimes A_{i})
  \and
  \;\; | \;\; \Sigma_{j} \Sigma_{j'} \quotep{\stackrel{\vee}{x_{j}}|\stackrel{\vee}{y_{j'}}}.(A_{j}\otimes B_{j'}) \; | \; \Sigma_{j'} \Sigma_{j} \quotep{\stackrel{\vee}{y_{j'}}|\stackrel{\vee}{x_{j}}}.(B_{j'}\otimes A_{j})
\end{mathpar}

\begin{remark}
  Do we need to $x^{L}$ and $y^{R}$ for this construction as well?
\end{remark}

\paragraph{Tensor of parallel compositions} Next, we distribute tensor
over par.

\begin{mathpar}
  P_{1}|P_{2} \otimes Q_{1}|Q_{2} := (P_{1} \otimes Q_{1}) | (P_{1}
  \otimes Q_{2}) | (P_{2} \otimes Q_{1}) | (P_{2} \otimes Q_{2})
\end{mathpar}

\paragraph{Tensor with dropped names} We treat tensor of a
process with a dropped name as parallel composition.

\begin{mathpar}
  P \otimes \dropn{x} := P | \dropn{x}
\end{mathpar}

\paragraph{Tensor of agents}

Finally, we need to define tensor on agents. Note that the definition
of tensor on normal products only tensors inputs with inputs and
outputs with outputs. Thus, we only have to define the operation on
``homogeneous'' pairings.

\begin{mathpar}
  (\vec{x})P \otimes (\vec{y})Q
  \and \\
  := (x_{0}^{L}|y_{0}^{R},\ldots,x_{0}^{L}|y_{n}^{R},\ldots,x_{m}^{L}|y_{0}^{R},\ldots,x_{m}^{L}|y_{n}^R)(P\{ \vec{x}^{L}/\vec{x}\} \otimes Q \{ \vec{y}^{R}/\vec{y}\})
  \and \\
  \clift{\vec{P}} \otimes \clift{\vec{Q}}
  \and \\
  := \clift{P_{0}\otimes Q_{0},\ldots,P_{0}\otimes Q_{n},\ldots,P_{m}\otimes Q_{0},\ldots,P_{m}\otimes Q_{n}}
\end{mathpar}

\begin{remark}
  Observe that arities of tensored abstractions matches arities of
  tensored concretions if the original arities matched. Note also that
  the length of the arities corresponds to the increase in dimension
  we see in ordinary vector space tensor product.
\end{remark}

\begin{remark}
  Operationally, this definition distributes the tensor down to
  components ``linked'' by summation. Tensor over summation is
  intriguing in that it mixes names. Moreover, as a consequence of the
  way it mixes names we have the identities for all $x \in \QProc$ and
  $P,Q \in \Proc$

  \begin{mathpar}
    (x \cdot P) \otimes Q \equiv x \cdot (P \otimes Q) \equiv P \otimes (x \cdot Q)
    \and
    P \otimes \pzero \equiv P
  \end{mathpar}

  that the reader is invited to verify.
\end{remark}

\subsubsection{Annihilation}
\begin{mathpar}
  P^{\perp} := \{ Q | \forall R. P|Q \red^{*} R \Rightarrow R \red^{*} \pzero \}
  \and \\
  P^{\underline{\perp}} := \Sigma_{Q \in P^{\perp}} \quotep{Q}?(y).(\dropn{y}|Q) | \Sigma_{Q \in P^{\perp}} \quotep{Q}\clift{\Box}
\end{mathpar}

\paragraph{Discussion} The reader will note that $P^{\perp}$ is a
\emph{set} of processes, while $P^{\underline{\perp}}$ is a
\emph{context}. We call the set $P^{\perp}$ the \emph{annihilators} of
$P$. The parallel composition of a process in the annihilators of $P$
with $P$ will result in a process, the state space of which has all
paths eventually leading to $\pzero$. Execution may endure loops; but
under reasonable conditions of fairness (naturally guaranteed under
most notions of bisimulation) such a composite process cannot get
stuck in such a loop and will, eventually pop out and terminate.

The context $P^{\underline{\perp}}$ is ready and willing to ``take the
$P$ out of'' the process to which it is applied. It will effectively
transmit the code of the process to which it is applied to one of the
annihilators and run the process against it.

\subsubsection{Evaluation}
We fix $M$ a domain of fully abstract interpretation with an equality
coincident with bisimulation. We take $\meaningof{\cdot} : \Proc \to
M$ to be the map interpreting processes and $\nmeaningof{\cdot} : \M
\to Proc$ to be the map running the other way. Then we define

\begin{mathpar}
  \int P := \nmeaningof{\meaningof{P}}
\end{mathpar}

\paragraph{Discussion}
There are many fully abstract interpretations of Milner's
$\pi$-calculus. Any of them can be used as a basis for interpreting
the reflective calculus here. Equipped with such a domain it is
largely a matter of grinding through to check that the Yoneda
construction for the normalization-by-evaluation program can be
extended to this setting.

\begin{remark}
  The reader is invited to verify that $\int (P^{\underline{\perp}}[P]) = 0$.
\end{remark}

\subsection{Quantum mechanics}

Table \ref{tbl:core_qm_op_defns} gives the core operational definitions

\begin{table}[htp]\label{tbl:core_qm_op_defns}
  \center{
    \fbox{
      \begin{tabular}{c|c}
        quantum mechanics & process calculus \\
        \hline
        scalar & $x := \quotep{P}$ \\
        state vector & $\state{P} := P$ \\
        dual & $\state{P}^{*} := \event{P^{\underline{\perp}}} := \quotep{P^{\underline{\perp}}}[-]$ \\
        matrix & $ \Sigma_{\alpha} \state{P_{\alpha}}x_{\alpha}\event{Q_{\alpha}}$ \\
        vector addition & $\state{P} + \state{Q} := \state{P | Q}$ \\
        tensor product & $\state{P} \otimes \state{Q} := \state{P \otimes Q}$ \\
        inner product & $\innerprod{P}{Q} := \quotep{\int P^{\underline{\perp}}[Q]}$ \\
      \end{tabular}
    }
  }
  \caption{QM - operational definitions}
\end{table}

where

\begin{mathpar}
  \prmatrix{P}{Q} := \fprmatrix{P}{\quotep{\pzero}}{Q}
  \and
  \fprmatrix{P}{x}{Q} := (\state{P},x,\event{Q})
  \and
  (\fprmatrix{P}{x}{Q})(\state{R}) := x \cdot \innerprod{Q}{R} \cdot \state{P}
  \and
  (\fprmatrix{P}{x}{Q})(\event{R}) := x \cdot \innerprod{R}{P} \cdot \event{Q}
\end{mathpar}

\paragraph{Discussion}
As promised: vectors (aka states) are represented as processes; duals
as contextual duals; inner product definition should be compared with
standard inner product definition for ....

\begin{remark}
  Assuming $\int (P^{\underline{\perp}}[P]) = 0$, the reader is
  invited to verify that $(\fprmatrix{P}{x}{P})(\state{P}) = x \cdot \state{P}$.
\end{remark}

\begin{remark}
  The reader is invited to verify that $\innerprod{P}{Q}$ could
  equally well have been written $\quotep{\int \stackrel{\vee}{x}}$
  where $x = \event{P^{\underline{\perp}}}(Q)$.

  One of the motivations for this remark is that there is another way
  to factor these operations. We could package up evaluation in the dual:

  \begin{mathpar}
    \state{P}^{*} := \event{\int P^{\underline{\perp}}} := \quotep{\int P^{\underline{\perp}}}[-]
  \end{mathpar}

  and then have inner product defined by
  
  \begin{mathpar}
    \innerprod{P}{Q} := \event{P}(Q)
  \end{mathpar}

  Hopefully, experience with the calculations will provide guidance on
  the best factoring.
\end{remark}

\begin{remark}
  Assuming $\int (P^{\underline{\perp}}[P]) = 0$, the reader is
  invited to verify that $\forall P,Q. (\prmatrix{0}{Q})(\state{0}) =
  \state{0}$ and dually $(\prmatrix{P}{0})(\event{0}) = \event{0}$.
\end{remark}

\begin{remark}
  i'm a little worried that i don't (yet) have proper support for
  complex conjugacy. But, the observation above may give us a
  clue. According to Abramsky, it must be the case that the scalars
  are iso to the homset of the identity for the tensor -- which the
  observation above characterizes. 

  For now, we will simply bookmark the notion with $\overline{x}$.
\end{remark}

\subsubsection{Adjointness}

We need to give a definition of $(\cdot)^{\dagger}$ for matrices. The
obvious candidate definition is
\begin{mathpar}
(\Sigma_{\alpha}\fprmatrix{P_{\alpha}}{x_{\alpha}}{Q_{\alpha}})^{\dagger}
= \Sigma_{\alpha}\fprmatrix{(Q_{\alpha}^{\underline{\perp}})^{*}}{\overline{x}_{\alpha}}{P_{\alpha}^{\underline{\perp}}} 
\end{mathpar}

But, $(Q_{\alpha}^{\underline{\perp}})^{*}$ requires a name along
which to communicate the process to achieve the context application.

\subsubsection{Basis for a basis}
If processes label states and ``addition'' of states (a.k.a. vector
addition) is interpreted as parallel composition, what corresponds to
notions of linear independence and basis? Here, we recall that Yoshida
has developed a set of \emph{combinators} for an asynchronous verison
of Milner's $\pi$-calculus. These are a finite set of processes such
any process can be expressed as parallel composition of these
combinators together with liberal uses of the new operator and
replication. We can simply give a translation of these into the
present calculus and have reasonable expectation that the property
carries over. That is, that the resultant set allows to express all
processes via parallel composition. Note, however, that there is no
new operator or replication in this calculus. As a result, we expect
that the corresponding set is actually infinite. That is, we expect
that the space is actually infinite dimensional.

\begin{remark}
  The attentive reader may be a bit concerned. Certainly, the
  collection $S$, $K$ and $I$ is a finite set of
  combinators. Shouldn't we expect to see a finite set of combinators
  for an effectively equivalent system? i am very sympathetic to this
  critique and feel it warrants full attention. On the other hand, i
  also have in mind the following analogy. The natural numbers, as a
  monoid under addition, has exactly $1$ generator, while the natural
  numbers, as a monoid under multiplication, has countably many
  generators (the primes). We observe that the application of the
  lambda calculus is much less resource sensitive than the parallel
  composition of the $\pi$-calculus. Could it be the case that we have
  an analogy of the form
  
  \begin{mathpar}
    m + n : MN :: m*n : M|N
  \end{mathpar}

  giving a similar blow up in the set of ``primes''?  This is such a
  wonderful thought that, even if it's not true, i think it's worth
  writing down.
\end{remark}
 

\documentclass[12pt]{llncs}
%\documentclass{jktr}

\usepackage[pdftex]{hyperref}                   
\usepackage {listings}
\usepackage {mathpartir}
\usepackage{bcprules}
%\usepackage{listings}
                       
\usepackage{graphicx} 
%\usepackage[margins=2.5cm,nohead,nofoot]{geometry}
%\usepackage{geometry}
\usepackage{amsfonts}
\usepackage{amstext}
\usepackage{latexsym}
\usepackage{amssymb}
\usepackage{color}


%\include{myPreamble}
\include{qm2pi.local} 

%\ifpdf
%\usepackage[pdftex]{graphicx}
%\else
%\usepackage{graphicx}
%\fi

 % \ifpdf
%  \usepackage{pdfsync}
%  \if


%\title{Brief Article}
%\author{David F. Snyder}
%\author{L.G. Meredith}

%\address{Dept. of Math., Texas State University--San Marcos, San Marcos, TX 78666}
       
\pagestyle{empty}


\begin{document}

\lstset{language=[Objective]Caml,frame=shadowbox}

\input{qm2pi.front}

% section front matter (end)

\input{qm2pi.intro} 
 
% section introduction (end)

% \input{qm2pi.knotations} 

% section notation (end)

\input{qm2pi.process.calculi} 

% section concurrent_process_calculi_and_spatial_logics_ (end)
    
%\input{qm2pi.knots2pi} 

%\input{qm2pi.trefoil} 

%\input{qm2pi.mainthm} 

% subsection basic_interpretation (end)

%\input{qm2pi.rho.presentation} 
\subsection{The syntax and semantics of the notation system}\label{sub:the_syntax_and_semantics_of_the_notation_system} % (fold)

We now summarize a technical presentation of the calculus that
embodies our theory of dynamics. The typical presentation of such a
calculus follows the style of giving generators and relations on
them. The grammar, below, describing term constructors, freely
generates the set of processes, $\Proc$. This set is then quotiented
by a relation known as structural congruence and it is over this set
that the notion of dynamics is expressed. This presentation is
essentially that of \cite{MeredithR05} with the addition of
polyadicity and summation. For readability we have relegated some of
the technical subtleties to an appendix.

\subsubsection{Process grammar}\label{subsub:process_grammar}

\begin{mathpar}
  \inferrule* [lab=synchronization] {} {{M} \bc \pzero \;|\; x?F \;|\; x!C }
  \and
  \inferrule* [lab=abstraction] {} {{F} \bc (x)P}
  \and
  \inferrule* [lab=concretion] {} {{C} \bc \langle Q \rangle}
  \and
  \inferrule* [lab=process] {} {{P,Q} \bc M \;| \;P|Q \;|\; @{x}}
  \and
  \inferrule* [lab=name] {} {{x} \bc \quotep{P}}
\end{mathpar} 

Note that $\vec{x}$ (resp. $\vec{P}$) denotes a vector of names
(resp. processes) of length $|\vec{x}|$ (resp. $|\vec{P}|$). We adopt
the following useful abbreviations.

\begin{mathpar}
   x?(\vec{y}).P := x.(\vec{y})P \and  x\clift{\vec{P}} := x.\clift{\vec{P}}
   \and x!(y) := \lift{x}{\dropn{y}}
   \and \Pi_{i=0}^{n-1}P_i := P_0 | \ldots | P_{n-1}
\end{mathpar}

\subsubsection{Structural congruence}

\paragraph{Free and bound names and alpha-equivalence.} At the
core of structural equivalence is alpha-equivalence which identifies
process that are the same up to a change of variable. Formally, we
recognize the distinction between free and bound names. The free names
of a process, $\freenames{P}$, may be calculated recursively as
follows:

\begin{mathpar}
\freenames{\pzero} := \emptyset
  \and \\
  \freenames{x?(y).P} := \{ x \} \cup (\freenames{P} \setminus \{ y \})
  \and 
  \freenames{x!\langle P \rangle} := \{ x \} \cup \{ P \} 
  \and \\
  \freenames{P|Q} := \freenames{P} \cup \freenames{Q}
  \and \\
  \freenames{@{x}} := \{ x \}
\end{mathpar}

$\pi$
$\quotep{\pi}$

$\freenames{-} : \pi \to \mathcal{P}(\quotep{\pi})$

\begin{eqnarray*}
  \freenames{\pzero} & := & \emptyset \\
  \freenames{x?(y).P} & := & \{ x \} \cup (\freenames{P} \setminus \{ y \}) \\
  \freenames{x!\langle P \rangle} & := & \{ x \} \cup \{ P \} \\
  \freenames{P|Q} & := & \freenames{P} \cup \freenames{Q} \\
  \freenames{\dropn{x}} & := & \{ x \}
\end{eqnarray*}

The bound names of a process, $\boundnames{P}$, are those names occurring in $P$
that are not free. For example, in $x?(y).0$, the name $x$ is free, while $y$ is bound.

\begin{mathpar}
  \inferrule* [lab=monoidal-laws] {} { P|Q \equiv Q|P \and P|0 \equiv P \and P|(Q|R) \equiv (P|Q)|R }
\end{mathpar}

\begin{mathpar}
  \inferrule* [lab=alpha-equivalence] {} { (x)P \equiv (y)P\{y/x\} \and y \not\in \freenames{P} }
\end{mathpar}

\begin{definition}
Then two processes, $P,Q$, are alpha-equivalent if $P = Q\{\vec{y}/\vec{x}\}$ for
some $\vec{x} \in \boundnames{Q},\vec{y} \in \boundnames{P}$, where $Q\{\vec{y}/\vec{x}\}$
denotes the capture-avoiding substitution of $\vec{y}$ for $\vec{x}$ in $Q$.
\end{definition}

\begin{definition}
  The {\em structural congruence} \cite{SangiorgiWalker} , $\equiv$,
  between processes is the least congruence containing
  alpha-equivalence, satisfying the abelian monoid laws
  (associativity, commutativity and $\pzero$ as identity) for parallel
  composition $|$ and for summation $+$.
\end{definition}

\subsection{Name equivalence}

We take name equivalence, written $\nameeq$, to be the smallest
equivalence relation generated by the following rules.

\begin{mathpar}
\inferrule*[lab=Quote-drop]
{ }
{ \quotep{@{x}} \nameeq x }

\inferrule*[lab=Struct-equiv]
{ P \scong Q }
{ \quotep{P} \nameeq \quotep{Q} }
\end{mathpar}

The astute reader will have noticed that the mutual recursion of names
and processes imposes a mutual recursion on alpha-equivalence and
structural equivalence via name-equivalence. Fortunately, all of this
works out pleasantly and we may calculate in the natural way, free of
concern. The reader interested in the details is referred to the
appendix \ref{appendix:rho_details}.

\subsection{Substitution}

We use $\Proc$ for the set of processes, $\QProc$ for the set of
names, and $\id{\{}\vec{y} / \vec{x} \id{\}}$ to denote partial maps,
$s : \QProc \rightarrow \QProc$. A map, $s$ lifts, uniquely, to a map
on process terms, $\widehat{s} : \Proc \rightarrow \Proc$ by the
following equations.

\begin{mathpar}
  (0) \psubstp{Q}{P} := 0 \\
  (R \juxtap S) \psubstp{Q}{P}
  :=    
  (R)\psubstp{Q}{P} \juxtap (S) \psubstp{Q}{P} \\
  (x?(y).R) \psubstp{Q}{P}    
  :=    
  (x)\substp{Q}{P} (z)\concat( (R \psubstn{z}{y}) \psubstp{Q}{P} ) \\
  (\lift{x}{R}) \psubstp{Q}{P}  
  :=
  \lift{(x)\substp{Q}{P}}{ R \psubstp{Q}{P} } \\
%   (\dropn{x})  \psubstp{Q}{P}       
%   := 
%   \left\{ 
%     \begin{array}{ccc} 
%       \dropn{\quotep{Q}} & & x \nameeq \quotep{P} \\
%       \dropn{x} & & otherwise \\
%     \end{array}
%   \right. 
  (\dropn{x})  \psubstp{Q}{P}       
  := 
  \left\{ 
    \begin{array}{ccc} 
      Q & & x \nameeq \quotep{P} \\
      \dropn{x} & & otherwise \\
    \end{array}
  \right.
\end{mathpar}
 

where

\begin{eqnarray}
  (x)\id{\{} \lpquote Q \rpquote / \lpquote P \rpquote \id{\}}            = 
  \left\{ 
    \begin{array}{ccc}
      \lpquote Q \rpquote & & x \nameeq \lpquote P \rpquote \\
      x & & otherwise \\
    \end{array}
  \right. \nonumber
\end{eqnarray}

and $z$ is chosen distinct from $\quotep{P}$, $\quotep{Q}$, the free
names in $Q$, and all the names in $R$. Our $\alpha$-equivalence will
be built in the standard way from this substitution.

\begin{remark}\label{rem:no_self_referential_names}
  One consequence of these definitions is that $\forall P. \quotep{P}
  \not\in \freenames{P}$.
\end{remark}

\subsection{ Dynamic quote: an example }

Anticipating something of what's to come, consider applying the
substitution, $\widehat{\id{\{}u / z \id{\}}}$, to the following pair
of processes, $\lift{w}{y!(z)}$ and $w[ \lpquote y!(z) \rpquote ]$.

\begin{eqnarray}
	\lift{w}{y!(z)}\widehat{\id{\{}u / z \id{\}}}
		& = &
		\lift{w}{y!(u)} \nonumber\\
	w[ \lpquote y!(z) \rpquote ] \widehat{ \id{\{}u / z \id{\}} }
		& = &
		w[ \lpquote y!(z) \rpquote ] \nonumber
\end{eqnarray}

Because the body of the process between quotes is impervious to
substitution, we get radically different answers. In fact, by
examining the first process in an input context,
e.g. $x?(z).\lift{w}{y!(z)}$, we see that the process under the lift
operator may be shaped by prefixed inputs binding a name inside it. In
this sense, the lift operator will be seen as a way to dynamically
construct processes before reifying them as names.

Finally equipped with these standard features we can present the
dynamics of the calculus.

\subsubsection{Operational semantics} 

Finally, we introduce the computational dynamics. What marks these
algebras as distinct from other more traditionally studied algebraic
structures, e.g. vector spaces or polynomial rings, is the manner in
which dynamics is captured. In traditional structures, dynamics is typically
expressed through morphisms between such structures, as in linear maps
between vector spaces or morphisms between rings. In algebras
associated with the semantics of computation, the dynamics is
expressed as part of the algebraic structure itself, through a
reduction reduction relation typically denoted by $\red$. Below, we
give a recursive presentation of this relation for the calculus used
in the encoding.

$\red \subseteq \pi \times \pi$
$\red : \pi \to \mathcal{P}(\pi)$

\begin{mathpar}
  \inferrule* [lab=Comm] { \textsf{match}( x_{src}, x_{trgt} ) } { x_{trgt}?(y)P \; | \; x_{src}!\langle {Q} \rangle \red P\{\quotep{Q}/y}\} }
  \and \\
  \inferrule* [lab=Par] {{P} \red {P}'} {{{P} | {Q}} \red {{P}' | {Q}}}
  \and
  \inferrule* [lab=Equiv]{{{P} \scong {P}'} \andalso {{P}' \red {Q}'} \andalso {{Q}' \scong {Q}}}{{P} \red {Q}}
\end{mathpar}

\begin{eqnarray*}
  match_{\equiv} (\quotep{P},\quotep{Q}) & := & P \equiv Q \\
  match_{\dagger}(\quotep{P},\quotep{Q}) & := & \forall R. P|Q \red^{*} R => R \red^{*} 0 \\
  match_{K}(\quotep{P},\quotep{Q}) & := & K \mbox{ for some context } K
\end{eqnarray*}

$u?(x)P | u!\langle Q \rangle \red P\{\quotep{Q}/x\}$

%We write $\wred$ for $\red^*$, and $P\red$ if $\exists Q $ such that $ P \red Q$.
We write $P\red$ if $\exists Q $ such that $ P \red Q$ and $P\not\red$, otherwise.

\section{Replication}

As mentioned before, it is known that replication (and hence
recursion) can be implemented in a higher-order process algebra
\cite{SangiorgiWalker}. As our first example of calculation with the
machinery thus far presented we give the construction explicitly in
the {\rhoc}.

\begin{eqnarray}
	D_{x} & := & \prefix{x}{y}{(\binpar{\outputp{x}{y}}{@{y}})} \nonumber\\
	\bangp_{x}{P} & := & \binpar{{x}!\langle{\binpar{D_{x}}{P}}\rangle}{D_{x}} \nonumber
\end{eqnarray}

\begin{eqnarray}
	\bangp_{x}{P} & & \nonumber\\
	=
	& {x}!\langle{(\prefix{x}{y}{(\outputp{x}{y} | @{y})) | P}}\rangle 
	      | \prefix{x}{y}{(\outputp{x}{y} | @{y})} & \nonumber\\
	\red
	& (\outputp{x}{y} | @{y})\substn{\quotep{(\prefix{x}{y}{(@{y} | \outputp{x}{y})) | P}}}{y} & \nonumber\\
	=
	& \outputp{x}{\quotep{(\prefix{x}{y}{(\outputp{x}{y} | @{y})) | P}}}
	  | {(\prefix{x}{y}{(\outputp{x}{y} | @{y})) | P}} & \nonumber\\
	\red
	& \ldots & \nonumber\\
	\red^*
	& P | P | \ldots & \nonumber
\end{eqnarray}

Of course, this encoding, as an implementation, runs away, unfolding
$\bangp{P}$ eagerly. A lazier and more implementable replication
operator, restricted to input-guarded processes, may be obtained as follows.

\begin{eqnarray}
\bangp{\prefix{u}{v}{P}} 
	:= 
	\binpar{\lift{x}{\prefix{u}{v}{(\binpar{D(x)}{P})}}}{D(x)} \nonumber
\end{eqnarray}

\begin{remark}
  Note that the lazier definition still does not deal with summation
  or mixed summation (i.e. sums over input and output). The reader is
  invited to construct definitions of replication that deal with these
  features. 

  Further, the definitions are parameterized in a name, $x$. Can you,
  gentle reader, make a definition that eliminates this parameter and
  guarantees no accidental interaction between the replication
  machinery and the process being replicated -- i.e. no accidental
  sharing of names used by the process to get its work done and the
  name(s) used by the replication to effect copying. This latter
  revision of the definition of replication is crucial to obtaining
  the expected identity $!!P \sim !P$.
\end{remark}

\begin{remark}\label{rem:paradoxical_combinator}
  The reader familiar with the lambda calculus will have noticed the
  similarity between $D$ and the paradoxical combinator.

  [Ed. note: the existence of this seems to suggest we have to be more
  restrictive on the set of processes and names we admit if we are to
  support no-cloning.]
\end{remark}

\subsubsection{Bisimulation}

The computational dynamics gives rise to another kind of equivalence,
the equivalence of computational behavior. As previously mentioned
this is typically captured \emph{via} some form of bisimulation.

% The notion we use in this paper is weak barbed bisimulation
% \cite{milner91polyadicpi}.

The notion we use in this paper is derived from weak barbed
bisimulation \cite{milner91polyadicpi}. 

\begin{definition}
An \emph{observation relation}, $\downarrow_{\mathcal N}$, over a set
of names, $\mathcal N$, is the smallest relation satisfying the rules
below.

\infrule[Out-barb]{y \in {\mathcal N}, \; x \nameeq y}
		  {\outputp{x}{v} \downarrow_{\mathcal N} x}
\infrule[Par-barb]{\mbox{$P\downarrow_{\mathcal N} x$ or $Q\downarrow_{\mathcal N} x$}}
		  {\binpar{P}{Q} \downarrow_{\mathcal N} x}

We write $P \Downarrow_{\mathcal N} x$ if there is $Q$ such that 
$P \wred Q$ and $Q \downarrow_{\mathcal N} x$.
\end{definition}

\begin{definition}
%\label{def.bbisim}
An  ${\mathcal N}$-\emph{barbed bisimulation} over a set of names, ${\mathcal N}$, is a symmetric binary relation 
${\mathcal S}_{\mathcal N}$ between agents such that $P\rel{S}_{\mathcal N}Q$ implies:
\begin{enumerate}
\item If $P \red P'$ then $Q \wred Q'$ and $P'\rel{S}_{\mathcal N} Q'$.
\item If $P\downarrow_{\mathcal N} x$, then $Q\Downarrow_{\mathcal N} x$.
\end{enumerate}
$P$ is ${\mathcal N}$-barbed bisimilar to $Q$, written
$P \wbbisim_{\mathcal N} Q$, if $P \rel{S}_{\mathcal N} Q$ for some ${\mathcal N}$-barbed bisimulation ${\mathcal S}_{\mathcal N}$.
\end{definition}

$\mathcal{R} \subseteq \pi \times \pi$

$P \mathcal{R} Q => \forall P'. P \red P' \Rightarrow \exists Q'. Q \red Q', P' \mathcal{R} Q'$

$P \vdash x \Rightarrow Q \vdash x$

\begin{mathpar}
  \inferrule*[lab=Out-barb]{x \nameeq y}{{y}!\langle{Q}\rangle \vdash x}
  \and
  \inferrule*[lab=Par-barb]{\mbox{$P\vdash x$ or $Q\vdash x$}}{\binpar{P}{Q} \vdash x}
\end{mathpar}

\subsubsection{Contexts}

One of the principle advantages of computational calculi like the
$\pi$-calculus is a well-defined notion of context,
contextual-equivalence and a correlation between
contextual-equivalence and notions of bisimulation. The notion of
context allows the decomposition of a process into (sub-)process and
its syntactic environment, its context. Thus, a context may be
thought of as a process with a ``hole'' (written $\Box$) in it. The
application of a context $M$ to a process $P$, written $M[P]$, is
tantamount to filling the hole in $M$ with $P$. In this paper we do
not need the full weight of this theory, but do make use of the notion
of context in the proof the main theorem. 

\begin{mathpar}
  \inferrule* [lab=summation] {} {{M_{M},M_{N}} \bc \Box \;|\; x.M_{A} \;|\; M_{M}+M_{N}}
  \and
  \inferrule* [lab=agent] {} {{M_{A}} \bc (\vec{x})M_{P} \;| \; \clift{P_0,\ldots,M_{P},\ldots,P_N}}
  \and \\
  \inferrule* [lab=process] {} {{M_{P}} \bc M_{N} \;| \;P|M_{P} }
\end{mathpar} 

\begin{mathpar}
  \inferrule* [lab=sychronization] {} {M_{N} \bc \Box \;|\; x?M_{F} \;|\; x!M_{C}}
  \and
  \inferrule* [lab=abstraction] {} {{M_{F}} \bc (x)M_{P} }
  \and
  \inferrule* [lab=concretion] {} {{M_{C}} \bc \langle M_{P} \rangle }
  \and \\
  \inferrule* [lab=process] {} {{M_{P}} \bc M_{N} \;| \;P|M_{P} }
\end{mathpar}

\begin{definition}[contextual application] Given a context $M$, and
  process $P$, we define the \emph{contextual application}, $M[P] :=
  M\{P/\Box\}$. That is, the contextual application of M to P is the
  substitution of $P$ for $\Box$ in $M$.
\end{definition}

$\meaningof{-} : L \to \mathcal{P}(\pi)$

\begin{mathpar}
  \inferrule* [lab=collection] {} {\meaningof{true} = \pi, \and \meaningof{~E} = \pi \setminus \meaningof{E}, \and \meaningof{E_{1} \& E_{2}} = \meaningof{E_{1}} \cap \meaningof{E_{2}}}
\end{mathpar}

\begin{mathpar}
  \inferrule* [lab=structure] {} {\meaningof{0} = \{ P \in \pi | P \equiv 0 \}, \and \\ \meaningof{E_1 | E_2} = \{ P \in \pi | P \equiv P_{1} | P_{2}, P_{1} \in \meaningof{E_{1}}, P_{2} \in \meaningof{E_2}\} }
\end{mathpar}

\begin{mathpar}
 \inferrule* [lab=behavior] {} {\meaningof{\langle a?b \rangle E} = \{ P \in \pi | P \equiv Q | u?(y)P', \\ \and \\\\ \and \\ \;\;\; u \in \meaningof{a}, \forall z.P'\{z/y\} \in \meaningof{E\{z/b\}}\}, \and \\ \meaningof{a!E} = \{ P \in \pi | P \equiv Q | x!\langle P' \rangle, x \in \meaningof{a} P' \in \meaningof{E}\} }
\end{mathpar}

\begin{mathpar}
 \inferrule* [lab=nominal] {} {\meaningof{\quotep{E}} = \{ \quotep{P} \in \quotep{\pi} | P \in \meaningof{E} \}, \and \meaningof{\quotep{P}} = \{ \quotep{Q} \in \quotep{\pi} | P \equiv Q \} \and \\ \meaningof{@\quotep{E}} = \{ P \in \pi | P \equiv @x, x \in \meaningof{E} \}}
\end{mathpar}

\begin{eqnarray*}
  \\
  \meaningof{-} : TS \to ST
\end{eqnarray*}

\begin{eqnarray*}
  \\
  L : TS \to ST
\end{eqnarray*}

\begin{eqnarray*}
  \\
  P \models E \iff P \in \meaningof{E}
\end{eqnarray*}

\begin{eqnarray*}
  P \approx_{L} Q \iff \forall E \in L. P \models E \iff Q \models E
\end{eqnarray*}

\begin{eqnarray*}
  P \approx_{K} Q
\end{eqnarray*}

\begin{eqnarray*}
  P \approx Q
\end{eqnarray*}

$\approx_{K} = \approx = \approx_{L}$

\subsubsection{Contextual duality}

Note that contexts extend the quotation operation to a family of
operations from processes to names. Given a context, $M$, we can
define a \emph{nominal context}, $\quotep{M}$ by $\quotep{M}[P] :=
\quotep{M[P]}$. To foreshadow what is to come we observe that these
operations enjoy a duality with processes very much like the duality
between vectors and maps from vectors to scalars.

Further, because the calculus is essentially higher-order, we have a
correspondence between contexts and processes. More specifically,
given a name $x$ and a context $M$ we can construct $M^{*}_{x}$ such
that 

\begin{mathpar}
  M^{*}_{x} | \lift{x}{P} \red M[P]
\end{mathpar}

namely,

\begin{mathpar}
  M^{*}_{x} := x?(u).M[\dropn{u}]
\end{mathpar}

The dependence of $M^{*}_{x}$ on a name makes it an abstraction, 

\begin{mathpar}
  M^{*} := (x)x?(u).M[\dropn{u}]
\end{mathpar}

\subsection{Additional notation}

It will sometimes be convenient to denote the process a name
quotes. We already have the notation $x = \quotep{P}$, but it will be
convenient to introduce an alternate notation, $\procn{x}$, when we
want to emphasize the connection to the use of the name. Note that, by
virtue of name equivalence, $\quotep{\procn{x}} \nameeq x$; so, the
notation is consistent with previous definitions.

Further, because names have structure it is possible to effect
substitutions on the basis of that structure. This means we need to
upgrade our notation for substitutions, which we accomplish by
adapting comprehension notation. Thus,

\begin{mathpar}
  P\{ y / x : x \in S \}
\end{mathpar}

is interpreted to mean the process derived from P by replacing (in a
capture-avoiding manner) each occurrence of $x$ in $S$ by $y$. For example,

\begin{mathpar}
  P\{ \quotep{\procn{x}|\procn{x}} / x : x \in \freenames{P} \}
\end{mathpar}

will replace each (occurrence) of a free name $x$ in $P$ by
$\quotep{\procn{x}|\procn{x}}$.

Also, we will avail ourselves of the notation $x^{L}$ and $x^{R}$ to
denote injections of a name into disjoint copies of the name
space. There are numerous ways to accomplish this. One example can be
found in \cite{MeredithR05}. This notation overloads to vectors of
names: $\vec{x}^{\pi} := (x_{i}^{\pi} \; : \; 0 \leq i < |\vec{x}| )$ where $\pi \in \{L,R\}$.

We also use $P^{\Box} := P|\Box$.

In \cite{MeredithR05} an interpretation of the new operator is
given. It turns out that there are several possible interpretations
all enjoying the requisite algebraic properties of the operator (see
\cite{milner91polyadicpi}). We will therefore make liberal use of
$(\nu\; \vec{x})P$.

% subsection the_syntax_and_semantics_of_the_notation_system (end)   

\input{qm2pi.qmops} 

\input{qm2pi.sterngerlach} 

\input{qm2pi.metric} 

% section concurrent_process_calculi (end)

%\input{qm2pi.proofsketch}

% section proof sketch (end)

%\input{qm2pi.slviaknots} 

% section spatial logic via knots (end)

\input{qm2pi.conclusion}

% section conclusion (end)

%\input{qm2pi.dtcodes} 

% section wiring algorithm (end)

\input{qm2pi.ack} 

% section acknowledgments (end)

\newpage


\bibliographystyle{plain}   
\bibliography{../../biblios/main.bib}

\input{qm2pi.rhodetails}

\end{document}

 

\documentclass[12pt]{llncs}
%\documentclass{jktr}

\usepackage[pdftex]{hyperref}                   
\usepackage {listings}
\usepackage {mathpartir}
\usepackage{bcprules}
%\usepackage{listings}
                       
\usepackage{graphicx} 
%\usepackage[margins=2.5cm,nohead,nofoot]{geometry}
%\usepackage{geometry}
\usepackage{amsfonts}
\usepackage{amstext}
\usepackage{latexsym}
\usepackage{amssymb}
\usepackage{color}


%\include{myPreamble}
\include{qm2pi.local} 

%\ifpdf
%\usepackage[pdftex]{graphicx}
%\else
%\usepackage{graphicx}
%\fi

 % \ifpdf
%  \usepackage{pdfsync}
%  \if


%\title{Brief Article}
%\author{David F. Snyder}
%\author{L.G. Meredith}

%\address{Dept. of Math., Texas State University--San Marcos, San Marcos, TX 78666}
       
\pagestyle{empty}


\begin{document}

\lstset{language=[Objective]Caml,frame=shadowbox}

\input{qm2pi.front}

% section front matter (end)

\input{qm2pi.intro} 
 
% section introduction (end)

% \input{qm2pi.knotations} 

% section notation (end)

\input{qm2pi.process.calculi} 

% section concurrent_process_calculi_and_spatial_logics_ (end)
    
%\input{qm2pi.knots2pi} 

%\input{qm2pi.trefoil} 

%\input{qm2pi.mainthm} 

% subsection basic_interpretation (end)

%\input{qm2pi.rho.presentation} 
\subsection{The syntax and semantics of the notation system}\label{sub:the_syntax_and_semantics_of_the_notation_system} % (fold)

We now summarize a technical presentation of the calculus that
embodies our theory of dynamics. The typical presentation of such a
calculus follows the style of giving generators and relations on
them. The grammar, below, describing term constructors, freely
generates the set of processes, $\Proc$. This set is then quotiented
by a relation known as structural congruence and it is over this set
that the notion of dynamics is expressed. This presentation is
essentially that of \cite{MeredithR05} with the addition of
polyadicity and summation. For readability we have relegated some of
the technical subtleties to an appendix.

\subsubsection{Process grammar}\label{subsub:process_grammar}

\begin{mathpar}
  \inferrule* [lab=synchronization] {} {{M} \bc \pzero \;|\; x?F \;|\; x!C }
  \and
  \inferrule* [lab=abstraction] {} {{F} \bc (x)P}
  \and
  \inferrule* [lab=concretion] {} {{C} \bc \langle Q \rangle}
  \and
  \inferrule* [lab=process] {} {{P,Q} \bc M \;| \;P|Q \;|\; @{x}}
  \and
  \inferrule* [lab=name] {} {{x} \bc \quotep{P}}
\end{mathpar} 

Note that $\vec{x}$ (resp. $\vec{P}$) denotes a vector of names
(resp. processes) of length $|\vec{x}|$ (resp. $|\vec{P}|$). We adopt
the following useful abbreviations.

\begin{mathpar}
   x?(\vec{y}).P := x.(\vec{y})P \and  x\clift{\vec{P}} := x.\clift{\vec{P}}
   \and x!(y) := \lift{x}{\dropn{y}}
   \and \Pi_{i=0}^{n-1}P_i := P_0 | \ldots | P_{n-1}
\end{mathpar}

\subsubsection{Structural congruence}

\paragraph{Free and bound names and alpha-equivalence.} At the
core of structural equivalence is alpha-equivalence which identifies
process that are the same up to a change of variable. Formally, we
recognize the distinction between free and bound names. The free names
of a process, $\freenames{P}$, may be calculated recursively as
follows:

\begin{mathpar}
\freenames{\pzero} := \emptyset
  \and \\
  \freenames{x?(y).P} := \{ x \} \cup (\freenames{P} \setminus \{ y \})
  \and 
  \freenames{x!\langle P \rangle} := \{ x \} \cup \{ P \} 
  \and \\
  \freenames{P|Q} := \freenames{P} \cup \freenames{Q}
  \and \\
  \freenames{@{x}} := \{ x \}
\end{mathpar}

$\pi$
$\quotep{\pi}$

$\freenames{-} : \pi \to \mathcal{P}(\quotep{\pi})$

\begin{eqnarray*}
  \freenames{\pzero} & := & \emptyset \\
  \freenames{x?(y).P} & := & \{ x \} \cup (\freenames{P} \setminus \{ y \}) \\
  \freenames{x!\langle P \rangle} & := & \{ x \} \cup \{ P \} \\
  \freenames{P|Q} & := & \freenames{P} \cup \freenames{Q} \\
  \freenames{\dropn{x}} & := & \{ x \}
\end{eqnarray*}

The bound names of a process, $\boundnames{P}$, are those names occurring in $P$
that are not free. For example, in $x?(y).0$, the name $x$ is free, while $y$ is bound.

\begin{mathpar}
  \inferrule* [lab=monoidal-laws] {} { P|Q \equiv Q|P \and P|0 \equiv P \and P|(Q|R) \equiv (P|Q)|R }
\end{mathpar}

\begin{mathpar}
  \inferrule* [lab=alpha-equivalence] {} { (x)P \equiv (y)P\{y/x\} \and y \not\in \freenames{P} }
\end{mathpar}

\begin{definition}
Then two processes, $P,Q$, are alpha-equivalent if $P = Q\{\vec{y}/\vec{x}\}$ for
some $\vec{x} \in \boundnames{Q},\vec{y} \in \boundnames{P}$, where $Q\{\vec{y}/\vec{x}\}$
denotes the capture-avoiding substitution of $\vec{y}$ for $\vec{x}$ in $Q$.
\end{definition}

\begin{definition}
  The {\em structural congruence} \cite{SangiorgiWalker} , $\equiv$,
  between processes is the least congruence containing
  alpha-equivalence, satisfying the abelian monoid laws
  (associativity, commutativity and $\pzero$ as identity) for parallel
  composition $|$ and for summation $+$.
\end{definition}

\subsection{Name equivalence}

We take name equivalence, written $\nameeq$, to be the smallest
equivalence relation generated by the following rules.

\begin{mathpar}
\inferrule*[lab=Quote-drop]
{ }
{ \quotep{@{x}} \nameeq x }

\inferrule*[lab=Struct-equiv]
{ P \scong Q }
{ \quotep{P} \nameeq \quotep{Q} }
\end{mathpar}

The astute reader will have noticed that the mutual recursion of names
and processes imposes a mutual recursion on alpha-equivalence and
structural equivalence via name-equivalence. Fortunately, all of this
works out pleasantly and we may calculate in the natural way, free of
concern. The reader interested in the details is referred to the
appendix \ref{appendix:rho_details}.

\subsection{Substitution}

We use $\Proc$ for the set of processes, $\QProc$ for the set of
names, and $\id{\{}\vec{y} / \vec{x} \id{\}}$ to denote partial maps,
$s : \QProc \rightarrow \QProc$. A map, $s$ lifts, uniquely, to a map
on process terms, $\widehat{s} : \Proc \rightarrow \Proc$ by the
following equations.

\begin{mathpar}
  (0) \psubstp{Q}{P} := 0 \\
  (R \juxtap S) \psubstp{Q}{P}
  :=    
  (R)\psubstp{Q}{P} \juxtap (S) \psubstp{Q}{P} \\
  (x?(y).R) \psubstp{Q}{P}    
  :=    
  (x)\substp{Q}{P} (z)\concat( (R \psubstn{z}{y}) \psubstp{Q}{P} ) \\
  (\lift{x}{R}) \psubstp{Q}{P}  
  :=
  \lift{(x)\substp{Q}{P}}{ R \psubstp{Q}{P} } \\
%   (\dropn{x})  \psubstp{Q}{P}       
%   := 
%   \left\{ 
%     \begin{array}{ccc} 
%       \dropn{\quotep{Q}} & & x \nameeq \quotep{P} \\
%       \dropn{x} & & otherwise \\
%     \end{array}
%   \right. 
  (\dropn{x})  \psubstp{Q}{P}       
  := 
  \left\{ 
    \begin{array}{ccc} 
      Q & & x \nameeq \quotep{P} \\
      \dropn{x} & & otherwise \\
    \end{array}
  \right.
\end{mathpar}
 

where

\begin{eqnarray}
  (x)\id{\{} \lpquote Q \rpquote / \lpquote P \rpquote \id{\}}            = 
  \left\{ 
    \begin{array}{ccc}
      \lpquote Q \rpquote & & x \nameeq \lpquote P \rpquote \\
      x & & otherwise \\
    \end{array}
  \right. \nonumber
\end{eqnarray}

and $z$ is chosen distinct from $\quotep{P}$, $\quotep{Q}$, the free
names in $Q$, and all the names in $R$. Our $\alpha$-equivalence will
be built in the standard way from this substitution.

\begin{remark}\label{rem:no_self_referential_names}
  One consequence of these definitions is that $\forall P. \quotep{P}
  \not\in \freenames{P}$.
\end{remark}

\subsection{ Dynamic quote: an example }

Anticipating something of what's to come, consider applying the
substitution, $\widehat{\id{\{}u / z \id{\}}}$, to the following pair
of processes, $\lift{w}{y!(z)}$ and $w[ \lpquote y!(z) \rpquote ]$.

\begin{eqnarray}
	\lift{w}{y!(z)}\widehat{\id{\{}u / z \id{\}}}
		& = &
		\lift{w}{y!(u)} \nonumber\\
	w[ \lpquote y!(z) \rpquote ] \widehat{ \id{\{}u / z \id{\}} }
		& = &
		w[ \lpquote y!(z) \rpquote ] \nonumber
\end{eqnarray}

Because the body of the process between quotes is impervious to
substitution, we get radically different answers. In fact, by
examining the first process in an input context,
e.g. $x?(z).\lift{w}{y!(z)}$, we see that the process under the lift
operator may be shaped by prefixed inputs binding a name inside it. In
this sense, the lift operator will be seen as a way to dynamically
construct processes before reifying them as names.

Finally equipped with these standard features we can present the
dynamics of the calculus.

\subsubsection{Operational semantics} 

Finally, we introduce the computational dynamics. What marks these
algebras as distinct from other more traditionally studied algebraic
structures, e.g. vector spaces or polynomial rings, is the manner in
which dynamics is captured. In traditional structures, dynamics is typically
expressed through morphisms between such structures, as in linear maps
between vector spaces or morphisms between rings. In algebras
associated with the semantics of computation, the dynamics is
expressed as part of the algebraic structure itself, through a
reduction reduction relation typically denoted by $\red$. Below, we
give a recursive presentation of this relation for the calculus used
in the encoding.

$\red \subseteq \pi \times \pi$
$\red : \pi \to \mathcal{P}(\pi)$

\begin{mathpar}
  \inferrule* [lab=Comm] { \textsf{match}( x_{src}, x_{trgt} ) } { x_{trgt}?(y)P \; | \; x_{src}!\langle {Q} \rangle \red P\{\quotep{Q}/y}\} }
  \and \\
  \inferrule* [lab=Par] {{P} \red {P}'} {{{P} | {Q}} \red {{P}' | {Q}}}
  \and
  \inferrule* [lab=Equiv]{{{P} \scong {P}'} \andalso {{P}' \red {Q}'} \andalso {{Q}' \scong {Q}}}{{P} \red {Q}}
\end{mathpar}

\begin{eqnarray*}
  match_{\equiv} (\quotep{P},\quotep{Q}) & := & P \equiv Q \\
  match_{\dagger}(\quotep{P},\quotep{Q}) & := & \forall R. P|Q \red^{*} R => R \red^{*} 0 \\
  match_{K}(\quotep{P},\quotep{Q}) & := & K \mbox{ for some context } K
\end{eqnarray*}

$u?(x)P | u!\langle Q \rangle \red P\{\quotep{Q}/x\}$

%We write $\wred$ for $\red^*$, and $P\red$ if $\exists Q $ such that $ P \red Q$.
We write $P\red$ if $\exists Q $ such that $ P \red Q$ and $P\not\red$, otherwise.

\section{Replication}

As mentioned before, it is known that replication (and hence
recursion) can be implemented in a higher-order process algebra
\cite{SangiorgiWalker}. As our first example of calculation with the
machinery thus far presented we give the construction explicitly in
the {\rhoc}.

\begin{eqnarray}
	D_{x} & := & \prefix{x}{y}{(\binpar{\outputp{x}{y}}{@{y}})} \nonumber\\
	\bangp_{x}{P} & := & \binpar{{x}!\langle{\binpar{D_{x}}{P}}\rangle}{D_{x}} \nonumber
\end{eqnarray}

\begin{eqnarray}
	\bangp_{x}{P} & & \nonumber\\
	=
	& {x}!\langle{(\prefix{x}{y}{(\outputp{x}{y} | @{y})) | P}}\rangle 
	      | \prefix{x}{y}{(\outputp{x}{y} | @{y})} & \nonumber\\
	\red
	& (\outputp{x}{y} | @{y})\substn{\quotep{(\prefix{x}{y}{(@{y} | \outputp{x}{y})) | P}}}{y} & \nonumber\\
	=
	& \outputp{x}{\quotep{(\prefix{x}{y}{(\outputp{x}{y} | @{y})) | P}}}
	  | {(\prefix{x}{y}{(\outputp{x}{y} | @{y})) | P}} & \nonumber\\
	\red
	& \ldots & \nonumber\\
	\red^*
	& P | P | \ldots & \nonumber
\end{eqnarray}

Of course, this encoding, as an implementation, runs away, unfolding
$\bangp{P}$ eagerly. A lazier and more implementable replication
operator, restricted to input-guarded processes, may be obtained as follows.

\begin{eqnarray}
\bangp{\prefix{u}{v}{P}} 
	:= 
	\binpar{\lift{x}{\prefix{u}{v}{(\binpar{D(x)}{P})}}}{D(x)} \nonumber
\end{eqnarray}

\begin{remark}
  Note that the lazier definition still does not deal with summation
  or mixed summation (i.e. sums over input and output). The reader is
  invited to construct definitions of replication that deal with these
  features. 

  Further, the definitions are parameterized in a name, $x$. Can you,
  gentle reader, make a definition that eliminates this parameter and
  guarantees no accidental interaction between the replication
  machinery and the process being replicated -- i.e. no accidental
  sharing of names used by the process to get its work done and the
  name(s) used by the replication to effect copying. This latter
  revision of the definition of replication is crucial to obtaining
  the expected identity $!!P \sim !P$.
\end{remark}

\begin{remark}\label{rem:paradoxical_combinator}
  The reader familiar with the lambda calculus will have noticed the
  similarity between $D$ and the paradoxical combinator.

  [Ed. note: the existence of this seems to suggest we have to be more
  restrictive on the set of processes and names we admit if we are to
  support no-cloning.]
\end{remark}

\subsubsection{Bisimulation}

The computational dynamics gives rise to another kind of equivalence,
the equivalence of computational behavior. As previously mentioned
this is typically captured \emph{via} some form of bisimulation.

% The notion we use in this paper is weak barbed bisimulation
% \cite{milner91polyadicpi}.

The notion we use in this paper is derived from weak barbed
bisimulation \cite{milner91polyadicpi}. 

\begin{definition}
An \emph{observation relation}, $\downarrow_{\mathcal N}$, over a set
of names, $\mathcal N$, is the smallest relation satisfying the rules
below.

\infrule[Out-barb]{y \in {\mathcal N}, \; x \nameeq y}
		  {\outputp{x}{v} \downarrow_{\mathcal N} x}
\infrule[Par-barb]{\mbox{$P\downarrow_{\mathcal N} x$ or $Q\downarrow_{\mathcal N} x$}}
		  {\binpar{P}{Q} \downarrow_{\mathcal N} x}

We write $P \Downarrow_{\mathcal N} x$ if there is $Q$ such that 
$P \wred Q$ and $Q \downarrow_{\mathcal N} x$.
\end{definition}

\begin{definition}
%\label{def.bbisim}
An  ${\mathcal N}$-\emph{barbed bisimulation} over a set of names, ${\mathcal N}$, is a symmetric binary relation 
${\mathcal S}_{\mathcal N}$ between agents such that $P\rel{S}_{\mathcal N}Q$ implies:
\begin{enumerate}
\item If $P \red P'$ then $Q \wred Q'$ and $P'\rel{S}_{\mathcal N} Q'$.
\item If $P\downarrow_{\mathcal N} x$, then $Q\Downarrow_{\mathcal N} x$.
\end{enumerate}
$P$ is ${\mathcal N}$-barbed bisimilar to $Q$, written
$P \wbbisim_{\mathcal N} Q$, if $P \rel{S}_{\mathcal N} Q$ for some ${\mathcal N}$-barbed bisimulation ${\mathcal S}_{\mathcal N}$.
\end{definition}

$\mathcal{R} \subseteq \pi \times \pi$

$P \mathcal{R} Q => \forall P'. P \red P' \Rightarrow \exists Q'. Q \red Q', P' \mathcal{R} Q'$

$P \vdash x \Rightarrow Q \vdash x$

\begin{mathpar}
  \inferrule*[lab=Out-barb]{x \nameeq y}{{y}!\langle{Q}\rangle \vdash x}
  \and
  \inferrule*[lab=Par-barb]{\mbox{$P\vdash x$ or $Q\vdash x$}}{\binpar{P}{Q} \vdash x}
\end{mathpar}

\subsubsection{Contexts}

One of the principle advantages of computational calculi like the
$\pi$-calculus is a well-defined notion of context,
contextual-equivalence and a correlation between
contextual-equivalence and notions of bisimulation. The notion of
context allows the decomposition of a process into (sub-)process and
its syntactic environment, its context. Thus, a context may be
thought of as a process with a ``hole'' (written $\Box$) in it. The
application of a context $M$ to a process $P$, written $M[P]$, is
tantamount to filling the hole in $M$ with $P$. In this paper we do
not need the full weight of this theory, but do make use of the notion
of context in the proof the main theorem. 

\begin{mathpar}
  \inferrule* [lab=summation] {} {{M_{M},M_{N}} \bc \Box \;|\; x.M_{A} \;|\; M_{M}+M_{N}}
  \and
  \inferrule* [lab=agent] {} {{M_{A}} \bc (\vec{x})M_{P} \;| \; \clift{P_0,\ldots,M_{P},\ldots,P_N}}
  \and \\
  \inferrule* [lab=process] {} {{M_{P}} \bc M_{N} \;| \;P|M_{P} }
\end{mathpar} 

\begin{mathpar}
  \inferrule* [lab=sychronization] {} {M_{N} \bc \Box \;|\; x?M_{F} \;|\; x!M_{C}}
  \and
  \inferrule* [lab=abstraction] {} {{M_{F}} \bc (x)M_{P} }
  \and
  \inferrule* [lab=concretion] {} {{M_{C}} \bc \langle M_{P} \rangle }
  \and \\
  \inferrule* [lab=process] {} {{M_{P}} \bc M_{N} \;| \;P|M_{P} }
\end{mathpar}

\begin{definition}[contextual application] Given a context $M$, and
  process $P$, we define the \emph{contextual application}, $M[P] :=
  M\{P/\Box\}$. That is, the contextual application of M to P is the
  substitution of $P$ for $\Box$ in $M$.
\end{definition}

$\meaningof{-} : L \to \mathcal{P}(\pi)$

\begin{mathpar}
  \inferrule* [lab=collection] {} {\meaningof{true} = \pi, \and \meaningof{~E} = \pi \setminus \meaningof{E}, \and \meaningof{E_{1} \& E_{2}} = \meaningof{E_{1}} \cap \meaningof{E_{2}}}
\end{mathpar}

\begin{mathpar}
  \inferrule* [lab=structure] {} {\meaningof{0} = \{ P \in \pi | P \equiv 0 \}, \and \\ \meaningof{E_1 | E_2} = \{ P \in \pi | P \equiv P_{1} | P_{2}, P_{1} \in \meaningof{E_{1}}, P_{2} \in \meaningof{E_2}\} }
\end{mathpar}

\begin{mathpar}
 \inferrule* [lab=behavior] {} {\meaningof{\langle a?b \rangle E} = \{ P \in \pi | P \equiv Q | u?(y)P', \\ \and \\\\ \and \\ \;\;\; u \in \meaningof{a}, \forall z.P'\{z/y\} \in \meaningof{E\{z/b\}}\}, \and \\ \meaningof{a!E} = \{ P \in \pi | P \equiv Q | x!\langle P' \rangle, x \in \meaningof{a} P' \in \meaningof{E}\} }
\end{mathpar}

\begin{mathpar}
 \inferrule* [lab=nominal] {} {\meaningof{\quotep{E}} = \{ \quotep{P} \in \quotep{\pi} | P \in \meaningof{E} \}, \and \meaningof{\quotep{P}} = \{ \quotep{Q} \in \quotep{\pi} | P \equiv Q \} \and \\ \meaningof{@\quotep{E}} = \{ P \in \pi | P \equiv @x, x \in \meaningof{E} \}}
\end{mathpar}

\begin{eqnarray*}
  \\
  \meaningof{-} : TS \to ST
\end{eqnarray*}

\begin{eqnarray*}
  \\
  L : TS \to ST
\end{eqnarray*}

\begin{eqnarray*}
  \\
  P \models E \iff P \in \meaningof{E}
\end{eqnarray*}

\begin{eqnarray*}
  P \approx_{L} Q \iff \forall E \in L. P \models E \iff Q \models E
\end{eqnarray*}

\begin{eqnarray*}
  P \approx_{K} Q
\end{eqnarray*}

\begin{eqnarray*}
  P \approx Q
\end{eqnarray*}

$\approx_{K} = \approx = \approx_{L}$

\subsubsection{Contextual duality}

Note that contexts extend the quotation operation to a family of
operations from processes to names. Given a context, $M$, we can
define a \emph{nominal context}, $\quotep{M}$ by $\quotep{M}[P] :=
\quotep{M[P]}$. To foreshadow what is to come we observe that these
operations enjoy a duality with processes very much like the duality
between vectors and maps from vectors to scalars.

Further, because the calculus is essentially higher-order, we have a
correspondence between contexts and processes. More specifically,
given a name $x$ and a context $M$ we can construct $M^{*}_{x}$ such
that 

\begin{mathpar}
  M^{*}_{x} | \lift{x}{P} \red M[P]
\end{mathpar}

namely,

\begin{mathpar}
  M^{*}_{x} := x?(u).M[\dropn{u}]
\end{mathpar}

The dependence of $M^{*}_{x}$ on a name makes it an abstraction, 

\begin{mathpar}
  M^{*} := (x)x?(u).M[\dropn{u}]
\end{mathpar}

\subsection{Additional notation}

It will sometimes be convenient to denote the process a name
quotes. We already have the notation $x = \quotep{P}$, but it will be
convenient to introduce an alternate notation, $\procn{x}$, when we
want to emphasize the connection to the use of the name. Note that, by
virtue of name equivalence, $\quotep{\procn{x}} \nameeq x$; so, the
notation is consistent with previous definitions.

Further, because names have structure it is possible to effect
substitutions on the basis of that structure. This means we need to
upgrade our notation for substitutions, which we accomplish by
adapting comprehension notation. Thus,

\begin{mathpar}
  P\{ y / x : x \in S \}
\end{mathpar}

is interpreted to mean the process derived from P by replacing (in a
capture-avoiding manner) each occurrence of $x$ in $S$ by $y$. For example,

\begin{mathpar}
  P\{ \quotep{\procn{x}|\procn{x}} / x : x \in \freenames{P} \}
\end{mathpar}

will replace each (occurrence) of a free name $x$ in $P$ by
$\quotep{\procn{x}|\procn{x}}$.

Also, we will avail ourselves of the notation $x^{L}$ and $x^{R}$ to
denote injections of a name into disjoint copies of the name
space. There are numerous ways to accomplish this. One example can be
found in \cite{MeredithR05}. This notation overloads to vectors of
names: $\vec{x}^{\pi} := (x_{i}^{\pi} \; : \; 0 \leq i < |\vec{x}| )$ where $\pi \in \{L,R\}$.

We also use $P^{\Box} := P|\Box$.

In \cite{MeredithR05} an interpretation of the new operator is
given. It turns out that there are several possible interpretations
all enjoying the requisite algebraic properties of the operator (see
\cite{milner91polyadicpi}). We will therefore make liberal use of
$(\nu\; \vec{x})P$.

% subsection the_syntax_and_semantics_of_the_notation_system (end)   

\input{qm2pi.qmops} 

\input{qm2pi.sterngerlach} 

\input{qm2pi.metric} 

% section concurrent_process_calculi (end)

%\input{qm2pi.proofsketch}

% section proof sketch (end)

%\input{qm2pi.slviaknots} 

% section spatial logic via knots (end)

\input{qm2pi.conclusion}

% section conclusion (end)

%\input{qm2pi.dtcodes} 

% section wiring algorithm (end)

\input{qm2pi.ack} 

% section acknowledgments (end)

\newpage


\bibliographystyle{plain}   
\bibliography{../../biblios/main.bib}

\input{qm2pi.rhodetails}

\end{document}

 

% section concurrent_process_calculi (end)

%\documentclass[12pt]{llncs}
%\documentclass{jktr}

\usepackage[pdftex]{hyperref}                   
\usepackage {listings}
\usepackage {mathpartir}
\usepackage{bcprules}
%\usepackage{listings}
                       
\usepackage{graphicx} 
%\usepackage[margins=2.5cm,nohead,nofoot]{geometry}
%\usepackage{geometry}
\usepackage{amsfonts}
\usepackage{amstext}
\usepackage{latexsym}
\usepackage{amssymb}
\usepackage{color}


%\include{myPreamble}
\include{qm2pi.local} 

%\ifpdf
%\usepackage[pdftex]{graphicx}
%\else
%\usepackage{graphicx}
%\fi

 % \ifpdf
%  \usepackage{pdfsync}
%  \if


%\title{Brief Article}
%\author{David F. Snyder}
%\author{L.G. Meredith}

%\address{Dept. of Math., Texas State University--San Marcos, San Marcos, TX 78666}
       
\pagestyle{empty}


\begin{document}

\lstset{language=[Objective]Caml,frame=shadowbox}

\input{qm2pi.front}

% section front matter (end)

\input{qm2pi.intro} 
 
% section introduction (end)

% \input{qm2pi.knotations} 

% section notation (end)

\input{qm2pi.process.calculi} 

% section concurrent_process_calculi_and_spatial_logics_ (end)
    
%\input{qm2pi.knots2pi} 

%\input{qm2pi.trefoil} 

%\input{qm2pi.mainthm} 

% subsection basic_interpretation (end)

%\input{qm2pi.rho.presentation} 
\subsection{The syntax and semantics of the notation system}\label{sub:the_syntax_and_semantics_of_the_notation_system} % (fold)

We now summarize a technical presentation of the calculus that
embodies our theory of dynamics. The typical presentation of such a
calculus follows the style of giving generators and relations on
them. The grammar, below, describing term constructors, freely
generates the set of processes, $\Proc$. This set is then quotiented
by a relation known as structural congruence and it is over this set
that the notion of dynamics is expressed. This presentation is
essentially that of \cite{MeredithR05} with the addition of
polyadicity and summation. For readability we have relegated some of
the technical subtleties to an appendix.

\subsubsection{Process grammar}\label{subsub:process_grammar}

\begin{mathpar}
  \inferrule* [lab=synchronization] {} {{M} \bc \pzero \;|\; x?F \;|\; x!C }
  \and
  \inferrule* [lab=abstraction] {} {{F} \bc (x)P}
  \and
  \inferrule* [lab=concretion] {} {{C} \bc \langle Q \rangle}
  \and
  \inferrule* [lab=process] {} {{P,Q} \bc M \;| \;P|Q \;|\; @{x}}
  \and
  \inferrule* [lab=name] {} {{x} \bc \quotep{P}}
\end{mathpar} 

Note that $\vec{x}$ (resp. $\vec{P}$) denotes a vector of names
(resp. processes) of length $|\vec{x}|$ (resp. $|\vec{P}|$). We adopt
the following useful abbreviations.

\begin{mathpar}
   x?(\vec{y}).P := x.(\vec{y})P \and  x\clift{\vec{P}} := x.\clift{\vec{P}}
   \and x!(y) := \lift{x}{\dropn{y}}
   \and \Pi_{i=0}^{n-1}P_i := P_0 | \ldots | P_{n-1}
\end{mathpar}

\subsubsection{Structural congruence}

\paragraph{Free and bound names and alpha-equivalence.} At the
core of structural equivalence is alpha-equivalence which identifies
process that are the same up to a change of variable. Formally, we
recognize the distinction between free and bound names. The free names
of a process, $\freenames{P}$, may be calculated recursively as
follows:

\begin{mathpar}
\freenames{\pzero} := \emptyset
  \and \\
  \freenames{x?(y).P} := \{ x \} \cup (\freenames{P} \setminus \{ y \})
  \and 
  \freenames{x!\langle P \rangle} := \{ x \} \cup \{ P \} 
  \and \\
  \freenames{P|Q} := \freenames{P} \cup \freenames{Q}
  \and \\
  \freenames{@{x}} := \{ x \}
\end{mathpar}

$\pi$
$\quotep{\pi}$

$\freenames{-} : \pi \to \mathcal{P}(\quotep{\pi})$

\begin{eqnarray*}
  \freenames{\pzero} & := & \emptyset \\
  \freenames{x?(y).P} & := & \{ x \} \cup (\freenames{P} \setminus \{ y \}) \\
  \freenames{x!\langle P \rangle} & := & \{ x \} \cup \{ P \} \\
  \freenames{P|Q} & := & \freenames{P} \cup \freenames{Q} \\
  \freenames{\dropn{x}} & := & \{ x \}
\end{eqnarray*}

The bound names of a process, $\boundnames{P}$, are those names occurring in $P$
that are not free. For example, in $x?(y).0$, the name $x$ is free, while $y$ is bound.

\begin{mathpar}
  \inferrule* [lab=monoidal-laws] {} { P|Q \equiv Q|P \and P|0 \equiv P \and P|(Q|R) \equiv (P|Q)|R }
\end{mathpar}

\begin{mathpar}
  \inferrule* [lab=alpha-equivalence] {} { (x)P \equiv (y)P\{y/x\} \and y \not\in \freenames{P} }
\end{mathpar}

\begin{definition}
Then two processes, $P,Q$, are alpha-equivalent if $P = Q\{\vec{y}/\vec{x}\}$ for
some $\vec{x} \in \boundnames{Q},\vec{y} \in \boundnames{P}$, where $Q\{\vec{y}/\vec{x}\}$
denotes the capture-avoiding substitution of $\vec{y}$ for $\vec{x}$ in $Q$.
\end{definition}

\begin{definition}
  The {\em structural congruence} \cite{SangiorgiWalker} , $\equiv$,
  between processes is the least congruence containing
  alpha-equivalence, satisfying the abelian monoid laws
  (associativity, commutativity and $\pzero$ as identity) for parallel
  composition $|$ and for summation $+$.
\end{definition}

\subsection{Name equivalence}

We take name equivalence, written $\nameeq$, to be the smallest
equivalence relation generated by the following rules.

\begin{mathpar}
\inferrule*[lab=Quote-drop]
{ }
{ \quotep{@{x}} \nameeq x }

\inferrule*[lab=Struct-equiv]
{ P \scong Q }
{ \quotep{P} \nameeq \quotep{Q} }
\end{mathpar}

The astute reader will have noticed that the mutual recursion of names
and processes imposes a mutual recursion on alpha-equivalence and
structural equivalence via name-equivalence. Fortunately, all of this
works out pleasantly and we may calculate in the natural way, free of
concern. The reader interested in the details is referred to the
appendix \ref{appendix:rho_details}.

\subsection{Substitution}

We use $\Proc$ for the set of processes, $\QProc$ for the set of
names, and $\id{\{}\vec{y} / \vec{x} \id{\}}$ to denote partial maps,
$s : \QProc \rightarrow \QProc$. A map, $s$ lifts, uniquely, to a map
on process terms, $\widehat{s} : \Proc \rightarrow \Proc$ by the
following equations.

\begin{mathpar}
  (0) \psubstp{Q}{P} := 0 \\
  (R \juxtap S) \psubstp{Q}{P}
  :=    
  (R)\psubstp{Q}{P} \juxtap (S) \psubstp{Q}{P} \\
  (x?(y).R) \psubstp{Q}{P}    
  :=    
  (x)\substp{Q}{P} (z)\concat( (R \psubstn{z}{y}) \psubstp{Q}{P} ) \\
  (\lift{x}{R}) \psubstp{Q}{P}  
  :=
  \lift{(x)\substp{Q}{P}}{ R \psubstp{Q}{P} } \\
%   (\dropn{x})  \psubstp{Q}{P}       
%   := 
%   \left\{ 
%     \begin{array}{ccc} 
%       \dropn{\quotep{Q}} & & x \nameeq \quotep{P} \\
%       \dropn{x} & & otherwise \\
%     \end{array}
%   \right. 
  (\dropn{x})  \psubstp{Q}{P}       
  := 
  \left\{ 
    \begin{array}{ccc} 
      Q & & x \nameeq \quotep{P} \\
      \dropn{x} & & otherwise \\
    \end{array}
  \right.
\end{mathpar}
 

where

\begin{eqnarray}
  (x)\id{\{} \lpquote Q \rpquote / \lpquote P \rpquote \id{\}}            = 
  \left\{ 
    \begin{array}{ccc}
      \lpquote Q \rpquote & & x \nameeq \lpquote P \rpquote \\
      x & & otherwise \\
    \end{array}
  \right. \nonumber
\end{eqnarray}

and $z$ is chosen distinct from $\quotep{P}$, $\quotep{Q}$, the free
names in $Q$, and all the names in $R$. Our $\alpha$-equivalence will
be built in the standard way from this substitution.

\begin{remark}\label{rem:no_self_referential_names}
  One consequence of these definitions is that $\forall P. \quotep{P}
  \not\in \freenames{P}$.
\end{remark}

\subsection{ Dynamic quote: an example }

Anticipating something of what's to come, consider applying the
substitution, $\widehat{\id{\{}u / z \id{\}}}$, to the following pair
of processes, $\lift{w}{y!(z)}$ and $w[ \lpquote y!(z) \rpquote ]$.

\begin{eqnarray}
	\lift{w}{y!(z)}\widehat{\id{\{}u / z \id{\}}}
		& = &
		\lift{w}{y!(u)} \nonumber\\
	w[ \lpquote y!(z) \rpquote ] \widehat{ \id{\{}u / z \id{\}} }
		& = &
		w[ \lpquote y!(z) \rpquote ] \nonumber
\end{eqnarray}

Because the body of the process between quotes is impervious to
substitution, we get radically different answers. In fact, by
examining the first process in an input context,
e.g. $x?(z).\lift{w}{y!(z)}$, we see that the process under the lift
operator may be shaped by prefixed inputs binding a name inside it. In
this sense, the lift operator will be seen as a way to dynamically
construct processes before reifying them as names.

Finally equipped with these standard features we can present the
dynamics of the calculus.

\subsubsection{Operational semantics} 

Finally, we introduce the computational dynamics. What marks these
algebras as distinct from other more traditionally studied algebraic
structures, e.g. vector spaces or polynomial rings, is the manner in
which dynamics is captured. In traditional structures, dynamics is typically
expressed through morphisms between such structures, as in linear maps
between vector spaces or morphisms between rings. In algebras
associated with the semantics of computation, the dynamics is
expressed as part of the algebraic structure itself, through a
reduction reduction relation typically denoted by $\red$. Below, we
give a recursive presentation of this relation for the calculus used
in the encoding.

$\red \subseteq \pi \times \pi$
$\red : \pi \to \mathcal{P}(\pi)$

\begin{mathpar}
  \inferrule* [lab=Comm] { \textsf{match}( x_{src}, x_{trgt} ) } { x_{trgt}?(y)P \; | \; x_{src}!\langle {Q} \rangle \red P\{\quotep{Q}/y}\} }
  \and \\
  \inferrule* [lab=Par] {{P} \red {P}'} {{{P} | {Q}} \red {{P}' | {Q}}}
  \and
  \inferrule* [lab=Equiv]{{{P} \scong {P}'} \andalso {{P}' \red {Q}'} \andalso {{Q}' \scong {Q}}}{{P} \red {Q}}
\end{mathpar}

\begin{eqnarray*}
  match_{\equiv} (\quotep{P},\quotep{Q}) & := & P \equiv Q \\
  match_{\dagger}(\quotep{P},\quotep{Q}) & := & \forall R. P|Q \red^{*} R => R \red^{*} 0 \\
  match_{K}(\quotep{P},\quotep{Q}) & := & K \mbox{ for some context } K
\end{eqnarray*}

$u?(x)P | u!\langle Q \rangle \red P\{\quotep{Q}/x\}$

%We write $\wred$ for $\red^*$, and $P\red$ if $\exists Q $ such that $ P \red Q$.
We write $P\red$ if $\exists Q $ such that $ P \red Q$ and $P\not\red$, otherwise.

\section{Replication}

As mentioned before, it is known that replication (and hence
recursion) can be implemented in a higher-order process algebra
\cite{SangiorgiWalker}. As our first example of calculation with the
machinery thus far presented we give the construction explicitly in
the {\rhoc}.

\begin{eqnarray}
	D_{x} & := & \prefix{x}{y}{(\binpar{\outputp{x}{y}}{@{y}})} \nonumber\\
	\bangp_{x}{P} & := & \binpar{{x}!\langle{\binpar{D_{x}}{P}}\rangle}{D_{x}} \nonumber
\end{eqnarray}

\begin{eqnarray}
	\bangp_{x}{P} & & \nonumber\\
	=
	& {x}!\langle{(\prefix{x}{y}{(\outputp{x}{y} | @{y})) | P}}\rangle 
	      | \prefix{x}{y}{(\outputp{x}{y} | @{y})} & \nonumber\\
	\red
	& (\outputp{x}{y} | @{y})\substn{\quotep{(\prefix{x}{y}{(@{y} | \outputp{x}{y})) | P}}}{y} & \nonumber\\
	=
	& \outputp{x}{\quotep{(\prefix{x}{y}{(\outputp{x}{y} | @{y})) | P}}}
	  | {(\prefix{x}{y}{(\outputp{x}{y} | @{y})) | P}} & \nonumber\\
	\red
	& \ldots & \nonumber\\
	\red^*
	& P | P | \ldots & \nonumber
\end{eqnarray}

Of course, this encoding, as an implementation, runs away, unfolding
$\bangp{P}$ eagerly. A lazier and more implementable replication
operator, restricted to input-guarded processes, may be obtained as follows.

\begin{eqnarray}
\bangp{\prefix{u}{v}{P}} 
	:= 
	\binpar{\lift{x}{\prefix{u}{v}{(\binpar{D(x)}{P})}}}{D(x)} \nonumber
\end{eqnarray}

\begin{remark}
  Note that the lazier definition still does not deal with summation
  or mixed summation (i.e. sums over input and output). The reader is
  invited to construct definitions of replication that deal with these
  features. 

  Further, the definitions are parameterized in a name, $x$. Can you,
  gentle reader, make a definition that eliminates this parameter and
  guarantees no accidental interaction between the replication
  machinery and the process being replicated -- i.e. no accidental
  sharing of names used by the process to get its work done and the
  name(s) used by the replication to effect copying. This latter
  revision of the definition of replication is crucial to obtaining
  the expected identity $!!P \sim !P$.
\end{remark}

\begin{remark}\label{rem:paradoxical_combinator}
  The reader familiar with the lambda calculus will have noticed the
  similarity between $D$ and the paradoxical combinator.

  [Ed. note: the existence of this seems to suggest we have to be more
  restrictive on the set of processes and names we admit if we are to
  support no-cloning.]
\end{remark}

\subsubsection{Bisimulation}

The computational dynamics gives rise to another kind of equivalence,
the equivalence of computational behavior. As previously mentioned
this is typically captured \emph{via} some form of bisimulation.

% The notion we use in this paper is weak barbed bisimulation
% \cite{milner91polyadicpi}.

The notion we use in this paper is derived from weak barbed
bisimulation \cite{milner91polyadicpi}. 

\begin{definition}
An \emph{observation relation}, $\downarrow_{\mathcal N}$, over a set
of names, $\mathcal N$, is the smallest relation satisfying the rules
below.

\infrule[Out-barb]{y \in {\mathcal N}, \; x \nameeq y}
		  {\outputp{x}{v} \downarrow_{\mathcal N} x}
\infrule[Par-barb]{\mbox{$P\downarrow_{\mathcal N} x$ or $Q\downarrow_{\mathcal N} x$}}
		  {\binpar{P}{Q} \downarrow_{\mathcal N} x}

We write $P \Downarrow_{\mathcal N} x$ if there is $Q$ such that 
$P \wred Q$ and $Q \downarrow_{\mathcal N} x$.
\end{definition}

\begin{definition}
%\label{def.bbisim}
An  ${\mathcal N}$-\emph{barbed bisimulation} over a set of names, ${\mathcal N}$, is a symmetric binary relation 
${\mathcal S}_{\mathcal N}$ between agents such that $P\rel{S}_{\mathcal N}Q$ implies:
\begin{enumerate}
\item If $P \red P'$ then $Q \wred Q'$ and $P'\rel{S}_{\mathcal N} Q'$.
\item If $P\downarrow_{\mathcal N} x$, then $Q\Downarrow_{\mathcal N} x$.
\end{enumerate}
$P$ is ${\mathcal N}$-barbed bisimilar to $Q$, written
$P \wbbisim_{\mathcal N} Q$, if $P \rel{S}_{\mathcal N} Q$ for some ${\mathcal N}$-barbed bisimulation ${\mathcal S}_{\mathcal N}$.
\end{definition}

$\mathcal{R} \subseteq \pi \times \pi$

$P \mathcal{R} Q => \forall P'. P \red P' \Rightarrow \exists Q'. Q \red Q', P' \mathcal{R} Q'$

$P \vdash x \Rightarrow Q \vdash x$

\begin{mathpar}
  \inferrule*[lab=Out-barb]{x \nameeq y}{{y}!\langle{Q}\rangle \vdash x}
  \and
  \inferrule*[lab=Par-barb]{\mbox{$P\vdash x$ or $Q\vdash x$}}{\binpar{P}{Q} \vdash x}
\end{mathpar}

\subsubsection{Contexts}

One of the principle advantages of computational calculi like the
$\pi$-calculus is a well-defined notion of context,
contextual-equivalence and a correlation between
contextual-equivalence and notions of bisimulation. The notion of
context allows the decomposition of a process into (sub-)process and
its syntactic environment, its context. Thus, a context may be
thought of as a process with a ``hole'' (written $\Box$) in it. The
application of a context $M$ to a process $P$, written $M[P]$, is
tantamount to filling the hole in $M$ with $P$. In this paper we do
not need the full weight of this theory, but do make use of the notion
of context in the proof the main theorem. 

\begin{mathpar}
  \inferrule* [lab=summation] {} {{M_{M},M_{N}} \bc \Box \;|\; x.M_{A} \;|\; M_{M}+M_{N}}
  \and
  \inferrule* [lab=agent] {} {{M_{A}} \bc (\vec{x})M_{P} \;| \; \clift{P_0,\ldots,M_{P},\ldots,P_N}}
  \and \\
  \inferrule* [lab=process] {} {{M_{P}} \bc M_{N} \;| \;P|M_{P} }
\end{mathpar} 

\begin{mathpar}
  \inferrule* [lab=sychronization] {} {M_{N} \bc \Box \;|\; x?M_{F} \;|\; x!M_{C}}
  \and
  \inferrule* [lab=abstraction] {} {{M_{F}} \bc (x)M_{P} }
  \and
  \inferrule* [lab=concretion] {} {{M_{C}} \bc \langle M_{P} \rangle }
  \and \\
  \inferrule* [lab=process] {} {{M_{P}} \bc M_{N} \;| \;P|M_{P} }
\end{mathpar}

\begin{definition}[contextual application] Given a context $M$, and
  process $P$, we define the \emph{contextual application}, $M[P] :=
  M\{P/\Box\}$. That is, the contextual application of M to P is the
  substitution of $P$ for $\Box$ in $M$.
\end{definition}

$\meaningof{-} : L \to \mathcal{P}(\pi)$

\begin{mathpar}
  \inferrule* [lab=collection] {} {\meaningof{true} = \pi, \and \meaningof{~E} = \pi \setminus \meaningof{E}, \and \meaningof{E_{1} \& E_{2}} = \meaningof{E_{1}} \cap \meaningof{E_{2}}}
\end{mathpar}

\begin{mathpar}
  \inferrule* [lab=structure] {} {\meaningof{0} = \{ P \in \pi | P \equiv 0 \}, \and \\ \meaningof{E_1 | E_2} = \{ P \in \pi | P \equiv P_{1} | P_{2}, P_{1} \in \meaningof{E_{1}}, P_{2} \in \meaningof{E_2}\} }
\end{mathpar}

\begin{mathpar}
 \inferrule* [lab=behavior] {} {\meaningof{\langle a?b \rangle E} = \{ P \in \pi | P \equiv Q | u?(y)P', \\ \and \\\\ \and \\ \;\;\; u \in \meaningof{a}, \forall z.P'\{z/y\} \in \meaningof{E\{z/b\}}\}, \and \\ \meaningof{a!E} = \{ P \in \pi | P \equiv Q | x!\langle P' \rangle, x \in \meaningof{a} P' \in \meaningof{E}\} }
\end{mathpar}

\begin{mathpar}
 \inferrule* [lab=nominal] {} {\meaningof{\quotep{E}} = \{ \quotep{P} \in \quotep{\pi} | P \in \meaningof{E} \}, \and \meaningof{\quotep{P}} = \{ \quotep{Q} \in \quotep{\pi} | P \equiv Q \} \and \\ \meaningof{@\quotep{E}} = \{ P \in \pi | P \equiv @x, x \in \meaningof{E} \}}
\end{mathpar}

\begin{eqnarray*}
  \\
  \meaningof{-} : TS \to ST
\end{eqnarray*}

\begin{eqnarray*}
  \\
  L : TS \to ST
\end{eqnarray*}

\begin{eqnarray*}
  \\
  P \models E \iff P \in \meaningof{E}
\end{eqnarray*}

\begin{eqnarray*}
  P \approx_{L} Q \iff \forall E \in L. P \models E \iff Q \models E
\end{eqnarray*}

\begin{eqnarray*}
  P \approx_{K} Q
\end{eqnarray*}

\begin{eqnarray*}
  P \approx Q
\end{eqnarray*}

$\approx_{K} = \approx = \approx_{L}$

\subsubsection{Contextual duality}

Note that contexts extend the quotation operation to a family of
operations from processes to names. Given a context, $M$, we can
define a \emph{nominal context}, $\quotep{M}$ by $\quotep{M}[P] :=
\quotep{M[P]}$. To foreshadow what is to come we observe that these
operations enjoy a duality with processes very much like the duality
between vectors and maps from vectors to scalars.

Further, because the calculus is essentially higher-order, we have a
correspondence between contexts and processes. More specifically,
given a name $x$ and a context $M$ we can construct $M^{*}_{x}$ such
that 

\begin{mathpar}
  M^{*}_{x} | \lift{x}{P} \red M[P]
\end{mathpar}

namely,

\begin{mathpar}
  M^{*}_{x} := x?(u).M[\dropn{u}]
\end{mathpar}

The dependence of $M^{*}_{x}$ on a name makes it an abstraction, 

\begin{mathpar}
  M^{*} := (x)x?(u).M[\dropn{u}]
\end{mathpar}

\subsection{Additional notation}

It will sometimes be convenient to denote the process a name
quotes. We already have the notation $x = \quotep{P}$, but it will be
convenient to introduce an alternate notation, $\procn{x}$, when we
want to emphasize the connection to the use of the name. Note that, by
virtue of name equivalence, $\quotep{\procn{x}} \nameeq x$; so, the
notation is consistent with previous definitions.

Further, because names have structure it is possible to effect
substitutions on the basis of that structure. This means we need to
upgrade our notation for substitutions, which we accomplish by
adapting comprehension notation. Thus,

\begin{mathpar}
  P\{ y / x : x \in S \}
\end{mathpar}

is interpreted to mean the process derived from P by replacing (in a
capture-avoiding manner) each occurrence of $x$ in $S$ by $y$. For example,

\begin{mathpar}
  P\{ \quotep{\procn{x}|\procn{x}} / x : x \in \freenames{P} \}
\end{mathpar}

will replace each (occurrence) of a free name $x$ in $P$ by
$\quotep{\procn{x}|\procn{x}}$.

Also, we will avail ourselves of the notation $x^{L}$ and $x^{R}$ to
denote injections of a name into disjoint copies of the name
space. There are numerous ways to accomplish this. One example can be
found in \cite{MeredithR05}. This notation overloads to vectors of
names: $\vec{x}^{\pi} := (x_{i}^{\pi} \; : \; 0 \leq i < |\vec{x}| )$ where $\pi \in \{L,R\}$.

We also use $P^{\Box} := P|\Box$.

In \cite{MeredithR05} an interpretation of the new operator is
given. It turns out that there are several possible interpretations
all enjoying the requisite algebraic properties of the operator (see
\cite{milner91polyadicpi}). We will therefore make liberal use of
$(\nu\; \vec{x})P$.

% subsection the_syntax_and_semantics_of_the_notation_system (end)   

\input{qm2pi.qmops} 

\input{qm2pi.sterngerlach} 

\input{qm2pi.metric} 

% section concurrent_process_calculi (end)

%\input{qm2pi.proofsketch}

% section proof sketch (end)

%\input{qm2pi.slviaknots} 

% section spatial logic via knots (end)

\input{qm2pi.conclusion}

% section conclusion (end)

%\input{qm2pi.dtcodes} 

% section wiring algorithm (end)

\input{qm2pi.ack} 

% section acknowledgments (end)

\newpage


\bibliographystyle{plain}   
\bibliography{../../biblios/main.bib}

\input{qm2pi.rhodetails}

\end{document}



% section proof sketch (end)

%\section{Unlikely characters: spatial logic for
  knots}\label{sub:characteristic_formulae} % (fold)

Associated to the mobile process calculi are a family of logics known
as the Hennessy-Milner logics. These logics typically enjoy a
semantics interpreting formulae as sets of processes that when
factored through the encoding outlined above allows an identification
of classes of knots with logical formulae. In the context of this
encoding the sub-family known as the spatial logics \cite{CairesC03}
\cite{CairesC04} \cite{Caires04} are of particular interest providing
several important features for expressing and reasoning about
properties (i.e. classes) of knots. We hint here at how this may be done.

%\begin{description}
%\item [structural connectives] 
\subsubsection{Structural connectives} The spatial logics enjoy
structural connectives corresponding, at the logical level, to the
parallel composition ($P | Q$) and new name ($(\nu \; x)P$)
connectives for processes. As illustrated in the examples below, these
connectives are extremely expressive given the shape of our encoding.
%\item [decideable satisfaction]

\subsubsection{Decideable satisfaction}
In \cite{Caires04} the satisfaction relation is shown to be decideable
for a rich class of processes. It further turns out that the image of
the our encoding is a proper subset of that class. This result
provides the basis for an algorithm by which to search for knots
enjoying a given property.
%\item [characteristic formulae]

\subsubsection{Characteristic formulae}
In the same paper \cite{Caires04} , Caires presents a means of calculating
characteristic formulae, selecting equivalence classes of processes
up to a pre--specified depth limit on the support set of names. Composed with our
encoding, this characteristic formula can be used to select
characteristic formulae for knots.
%\end{description}

\subsubsection{Spatial logic formulae}

The grammar below (segmented for comprehension) summarizes the syntax
of spatial logic formulae. We employ illustrative examples in the
sequel to provide an intuitive understanding of their meaning
referring the reader to \cite{Caires04} for a more detailed explication
of the semantics.

\begin{mathpar}
  \inferrule* [lab=boolean] {} {{A,B} \bc T \;|\; \neg A \;|\; A \wedge B \;|\; \eta = \eta'}
  \and
  \inferrule* [lab=spatial] {} {|\; \pzero \;|\; A | B \;|\; x \text{\textregistered} A \;|\; \forall x . A \;|\;  H x . A}
  \and
  \inferrule* [lab=behavioral] {} {|\; \alpha . A}
  \and 
  \inferrule* [lab=recursion] {} {|\; X(\vec{u}) \;|\; \mu X(\vec{u}) . A}
  \and
  \inferrule* [lab=action] {} {\alpha \bc \langle x?(\vec{y}) \rangle \;|\; \langle x!(\vec{y}) \rangle \;|\; \langle \tau \rangle}
  \and 
  \inferrule* [lab=name] {} {\eta \bc x \;|\; \tau}
\end{mathpar} 

% subsection characteristic_formulae (end)   	 

\subsection{Example formulae}\label{sub:example_formulae_} % (fold)

\subsubsection{Crossing as formula.}
% 
% \begin{align*}
%   \frac{d}{dx} \sin x &= \cos x 
%   & \frac{d}{dx} e^x &= e^x \\
%   \frac{d}{dx} \cos x &= - \sin x 
%   & \frac{d}{dx} \log x &= \frac{1}{x} \\
% \end{align*} 

\begin{align*}
 \mu C(x_{0},x_{1},y_{0},y_{1},u).&(\langle x_{0}?(z) \rangle(\langle u! \rangle\langle y_{1}!z \rangle C(x_{0},x_{1},y_{0},y_{1},u)) & \\
  & \wedge \langle y_{1}?(z) \rangle (\langle u! \rangle \langle x_{0}!z \rangle C(x_{0},x_{1},y_{0},y_{1},u)) & \\
  & \wedge \langle x_{1}?(z) \rangle (\langle u? \rangle \langle y_{0}!z \rangle C(x_{0},x_{1},y_{0},y_{1},u)) & \\
  & \wedge \langle y_{0}?(z) \rangle (\langle u? \rangle \langle x_{1}!z \rangle C(x_{0},x_{1},y_{0},y_{1},u))) &
\end{align*}

The lexicographical similarity between the shape of this formulae and
the shape of definition of the process representing a crossing reveals
the intuitive meaning of this formulae. It describes the capabilities
of a process that has the right to represent a crossing. For example
it picks out processes that may perform an input on the port $x_0$ in
its initial menu of capabilities. What differentiates the formula
from the process, however, is that the crossing process is the
smallest candidate to satisfy the formula. Infinitely many other
processes -- with internal behavior hidden behind this interface, so
to speak -- also satisfy this formula. Even this simple formula,
then, can be seen to open a new view onto knots, providing a
computational interpretation of \emph{virtual} knots.

Note that this formula is derived by hand. A similar formula can be
derived by employing Caires' calculation of characteristic formula
\cite{Caires04} to the process representing a crossing. In light of
this discussion, we let
$\meaningof{C}_{\phi}(x0,x1,y0,y1,u)$ denote a formula specifying the
dynamics we wish to capture of a crossing. To guarantee we preserve
the shape of the interface and minimal semantics we demand that
$\meaningof{C}_{\phi}(x0,x1,y0,y1,u) \Rightarrow
\textbf{C}(x0,x1,y0,y1,u)$ where $\textbf{C}(x0,x1,y0,y1,u)$ denotes
the formula above.
                            
\subsubsection{Crossing number constraints.}
The moral content of the context lemma (Lemma \ref{context}) is that the notion of
``locality'' in the Reidemeister moves is effectively captured by the
parallel composition operator of the process calculus. This intuition
extends through the logic. Given a formula,
$\meaningof{C}_{\phi}(x0,x1,y0,y1,u)$, we can use the structural
connectives to specify constraints on crossing numbers, such as at
least $n$ crossings, or exactly $n$ crossings.
\begin{mathpar}
  \inferrule* [lab=at-least-n] {} { K^{\geq n}_{\phi}(\vec{xs},\vec{ys}) := \Pi_{i=0}^{n-1} Hu . \meaningof{C}_{\phi}(xs_i,ys_i,u) | T }
  \and 
  \inferrule* [lab=exactly-n] {} { K^{= n}_{\phi}(\vec{xs},\vec{ys}) := \Pi_{i=0}^{n-1} Hu . \meaningof{C}_{\phi}(xs_i,ys_i,u) | \neg (\forall x_0,y_0,x_1,y_1,u . \meaningof{C}_{\phi}(x_0,y_0,x_1,y_1,u) | T) }
\end{mathpar}

To round out this section, recall that the encoding of an $n$-crossing
knot decomposes into a parallel composition of $n$ \emph{copies} of a
crossing process together with a wiring harness. To specify different
knot classes with the same crossing number amounts to specifying
logical constraints on the wiring harness. In the interest of space,
we defer examples to a forthcoming paper. Suffice it to say that both
the conditions ``alternating knot'' and ``contains the tangle
corresponding to 5/3'' are expressible. For example, it is possible to
calculate the characteristic formula of a process corresponding to the
tangle 5/3 and conjoin it into the classifying formula via the
composition connective of the logic.

Finally, we wish to observe that it is entirely within reason to
contemplate a more domain-specific version of spatial logic tailored
to the shape of processes in the image of the encoding. Such a
domain-specific logic would have a better claim to the title formal
language of knot properties.

% subsection example_formulae_ (end)

% section knots_as_processes (end) 

% section spatial logic via knots (end)

\section{Conclusions and future work}

\paragraph{Testing physical space}
You, gentle reader, may wonder why of all the theorems to be proved
given this set up we pick the one above. In some sense it's hardly
central to quantum mechanics. We see it as central in the sense that
it firmly establishes a notion of physical space arising from a notion
of the equivalence of behavior. Relating bisimulation to a metric is a
big step forward, but one is faced with interpreting the relationship
of that metric space to something more physical. Quantum mechanical
notions of ``physical'' space are still far from intuitive, but by
relating this idea of distance as testing to calculations that predict
physical circumstances we are making a not insignificant step forward
toward an understanding of the physical space we inhabit as
essentially dynamic.

\paragraph{Effectivity and simulation}
One of the observations we have yet to make is that the entire program
spelled out here is effective. We have built various interpreters for
the reflective calculus at work in this interpretation. In principle,
then, we can simulate quantum mechanics on a computer. The place where
the simulation may lose fidelity is the infinitely branching summation
for the annihilator.

In this connection i also want to point out that the evaluation style
calculation of the inner product puts the non-determinism of the
summation right at the heart of measurement. This suggests that
Milner's original reduction-based formulation of the dynamics of his
calculi in terms of sums was not just notationally suggestive of a
notion of measure-and-continue but captured some significant part of
the physics.

\paragraph{Quantum continuations}
In light of this last observation i want to point out that the
predominant account of quantum mechanics is missing a key aspect of a
truly compositional story of the physical situation. In a real lab,
when a measurement is made the observation can be made to feed into
another device that then makes another measurement conditioned on the
results of the first. This means that after the superposition was
collapsed the entire experimental set up remained in
superposition. While QM offers a means of writing this down it doesn't
quite line up well with the well-trodden formulation of computation
and continuation that we see so succinctly expressed in Milner's
calculi. This suggests that there might be advantages to this account
of dynamics waiting to be explored.

\paragraph{Quantum logic}
In this connection, we also note that by virtue of having the
Hennessy-Milner construction, we can pull the construction through the
interpretation of QM. This gives us a natural candidate for a quantum
logic that enjoys an extremely tight connection with it's domain of
interpretation, making the construction much less ad hoc (rather it is
the image of functor!).

\paragraph{Quantum probabiity}
i have questions about the basis of the interpretation of inner
product as probability amplitude. In particular, using which
axiomatization of probability theory does the notion of probability
amplitude earn the right to be so dubbed? In other words, where is the
proof that the operation for calculating a probability amplitude (and
then squaring) satisfies the axioms of what it means to calculate a
probability? Even if such a proof exists (i have yet to find it in the
literature), i wonder if it might not be possible to turn things on
their heads. Can we view the calculation of the probability amplitude
as an axiomatization of probability? If so, then the definition we
give for calculating probability amplitude may provide the basis for
an \emph{effective} theory of probability.

\paragraph{Quantum vs ``biological'' information}
Finally, i want to conclude with a more philosophical observation. At
a recent workshop in which QM was a predominant topic i noticed
something about quantum information. The speaker was giving a riveting
discussion of axiomatic QM and showing how properties of ``no
cloning'' and ``no deleting'' emerged as consequences of the
axiomatization. Theorems of this form are necessary to give us a sense
of confidence that our axioms characterize the physical theory. What
struck me, though, was that if quantum information is neither erasable
nor replicable it is markedly different from \emph{life}. Two of the
things we know about life is that

\begin{itemize}
  \item it ends;
  \item to gain some measure of persistence, to transcend it's
    finitude it is imminently copyable.
\end{itemize}

Both of these qualities are summarized succinctly in the aphorism: all
flesh is grass. For me these two kinds of ``information'' -- call them
quantum and biological -- are end points on a spectrum of strategies
for persistence. At one end, we have those curious entities that enjoy
uniqueness and permanence; at the other, we have those who in the face
of a certain end and an uncertain present make a go of passing
something on. To me one of the more remarkable aspects of the latter
strategy is that in the presence of noise (and certain features of
copying) we get a kind of dynamism, a chance for improvement against a
given persistent condition.

% subsection other_calculi_other_bisimulations_and_geometry_as_behavior (end)




% section conclusion (end)

%\documentclass[12pt]{llncs}
%\documentclass{jktr}

\usepackage[pdftex]{hyperref}                   
\usepackage {listings}
\usepackage {mathpartir}
\usepackage{bcprules}
%\usepackage{listings}
                       
\usepackage{graphicx} 
%\usepackage[margins=2.5cm,nohead,nofoot]{geometry}
%\usepackage{geometry}
\usepackage{amsfonts}
\usepackage{amstext}
\usepackage{latexsym}
\usepackage{amssymb}
\usepackage{color}


%\include{myPreamble}
\include{qm2pi.local} 

%\ifpdf
%\usepackage[pdftex]{graphicx}
%\else
%\usepackage{graphicx}
%\fi

 % \ifpdf
%  \usepackage{pdfsync}
%  \if


%\title{Brief Article}
%\author{David F. Snyder}
%\author{L.G. Meredith}

%\address{Dept. of Math., Texas State University--San Marcos, San Marcos, TX 78666}
       
\pagestyle{empty}


\begin{document}

\lstset{language=[Objective]Caml,frame=shadowbox}

\input{qm2pi.front}

% section front matter (end)

\input{qm2pi.intro} 
 
% section introduction (end)

% \input{qm2pi.knotations} 

% section notation (end)

\input{qm2pi.process.calculi} 

% section concurrent_process_calculi_and_spatial_logics_ (end)
    
%\input{qm2pi.knots2pi} 

%\input{qm2pi.trefoil} 

%\input{qm2pi.mainthm} 

% subsection basic_interpretation (end)

%\input{qm2pi.rho.presentation} 
\subsection{The syntax and semantics of the notation system}\label{sub:the_syntax_and_semantics_of_the_notation_system} % (fold)

We now summarize a technical presentation of the calculus that
embodies our theory of dynamics. The typical presentation of such a
calculus follows the style of giving generators and relations on
them. The grammar, below, describing term constructors, freely
generates the set of processes, $\Proc$. This set is then quotiented
by a relation known as structural congruence and it is over this set
that the notion of dynamics is expressed. This presentation is
essentially that of \cite{MeredithR05} with the addition of
polyadicity and summation. For readability we have relegated some of
the technical subtleties to an appendix.

\subsubsection{Process grammar}\label{subsub:process_grammar}

\begin{mathpar}
  \inferrule* [lab=synchronization] {} {{M} \bc \pzero \;|\; x?F \;|\; x!C }
  \and
  \inferrule* [lab=abstraction] {} {{F} \bc (x)P}
  \and
  \inferrule* [lab=concretion] {} {{C} \bc \langle Q \rangle}
  \and
  \inferrule* [lab=process] {} {{P,Q} \bc M \;| \;P|Q \;|\; @{x}}
  \and
  \inferrule* [lab=name] {} {{x} \bc \quotep{P}}
\end{mathpar} 

Note that $\vec{x}$ (resp. $\vec{P}$) denotes a vector of names
(resp. processes) of length $|\vec{x}|$ (resp. $|\vec{P}|$). We adopt
the following useful abbreviations.

\begin{mathpar}
   x?(\vec{y}).P := x.(\vec{y})P \and  x\clift{\vec{P}} := x.\clift{\vec{P}}
   \and x!(y) := \lift{x}{\dropn{y}}
   \and \Pi_{i=0}^{n-1}P_i := P_0 | \ldots | P_{n-1}
\end{mathpar}

\subsubsection{Structural congruence}

\paragraph{Free and bound names and alpha-equivalence.} At the
core of structural equivalence is alpha-equivalence which identifies
process that are the same up to a change of variable. Formally, we
recognize the distinction between free and bound names. The free names
of a process, $\freenames{P}$, may be calculated recursively as
follows:

\begin{mathpar}
\freenames{\pzero} := \emptyset
  \and \\
  \freenames{x?(y).P} := \{ x \} \cup (\freenames{P} \setminus \{ y \})
  \and 
  \freenames{x!\langle P \rangle} := \{ x \} \cup \{ P \} 
  \and \\
  \freenames{P|Q} := \freenames{P} \cup \freenames{Q}
  \and \\
  \freenames{@{x}} := \{ x \}
\end{mathpar}

$\pi$
$\quotep{\pi}$

$\freenames{-} : \pi \to \mathcal{P}(\quotep{\pi})$

\begin{eqnarray*}
  \freenames{\pzero} & := & \emptyset \\
  \freenames{x?(y).P} & := & \{ x \} \cup (\freenames{P} \setminus \{ y \}) \\
  \freenames{x!\langle P \rangle} & := & \{ x \} \cup \{ P \} \\
  \freenames{P|Q} & := & \freenames{P} \cup \freenames{Q} \\
  \freenames{\dropn{x}} & := & \{ x \}
\end{eqnarray*}

The bound names of a process, $\boundnames{P}$, are those names occurring in $P$
that are not free. For example, in $x?(y).0$, the name $x$ is free, while $y$ is bound.

\begin{mathpar}
  \inferrule* [lab=monoidal-laws] {} { P|Q \equiv Q|P \and P|0 \equiv P \and P|(Q|R) \equiv (P|Q)|R }
\end{mathpar}

\begin{mathpar}
  \inferrule* [lab=alpha-equivalence] {} { (x)P \equiv (y)P\{y/x\} \and y \not\in \freenames{P} }
\end{mathpar}

\begin{definition}
Then two processes, $P,Q$, are alpha-equivalent if $P = Q\{\vec{y}/\vec{x}\}$ for
some $\vec{x} \in \boundnames{Q},\vec{y} \in \boundnames{P}$, where $Q\{\vec{y}/\vec{x}\}$
denotes the capture-avoiding substitution of $\vec{y}$ for $\vec{x}$ in $Q$.
\end{definition}

\begin{definition}
  The {\em structural congruence} \cite{SangiorgiWalker} , $\equiv$,
  between processes is the least congruence containing
  alpha-equivalence, satisfying the abelian monoid laws
  (associativity, commutativity and $\pzero$ as identity) for parallel
  composition $|$ and for summation $+$.
\end{definition}

\subsection{Name equivalence}

We take name equivalence, written $\nameeq$, to be the smallest
equivalence relation generated by the following rules.

\begin{mathpar}
\inferrule*[lab=Quote-drop]
{ }
{ \quotep{@{x}} \nameeq x }

\inferrule*[lab=Struct-equiv]
{ P \scong Q }
{ \quotep{P} \nameeq \quotep{Q} }
\end{mathpar}

The astute reader will have noticed that the mutual recursion of names
and processes imposes a mutual recursion on alpha-equivalence and
structural equivalence via name-equivalence. Fortunately, all of this
works out pleasantly and we may calculate in the natural way, free of
concern. The reader interested in the details is referred to the
appendix \ref{appendix:rho_details}.

\subsection{Substitution}

We use $\Proc$ for the set of processes, $\QProc$ for the set of
names, and $\id{\{}\vec{y} / \vec{x} \id{\}}$ to denote partial maps,
$s : \QProc \rightarrow \QProc$. A map, $s$ lifts, uniquely, to a map
on process terms, $\widehat{s} : \Proc \rightarrow \Proc$ by the
following equations.

\begin{mathpar}
  (0) \psubstp{Q}{P} := 0 \\
  (R \juxtap S) \psubstp{Q}{P}
  :=    
  (R)\psubstp{Q}{P} \juxtap (S) \psubstp{Q}{P} \\
  (x?(y).R) \psubstp{Q}{P}    
  :=    
  (x)\substp{Q}{P} (z)\concat( (R \psubstn{z}{y}) \psubstp{Q}{P} ) \\
  (\lift{x}{R}) \psubstp{Q}{P}  
  :=
  \lift{(x)\substp{Q}{P}}{ R \psubstp{Q}{P} } \\
%   (\dropn{x})  \psubstp{Q}{P}       
%   := 
%   \left\{ 
%     \begin{array}{ccc} 
%       \dropn{\quotep{Q}} & & x \nameeq \quotep{P} \\
%       \dropn{x} & & otherwise \\
%     \end{array}
%   \right. 
  (\dropn{x})  \psubstp{Q}{P}       
  := 
  \left\{ 
    \begin{array}{ccc} 
      Q & & x \nameeq \quotep{P} \\
      \dropn{x} & & otherwise \\
    \end{array}
  \right.
\end{mathpar}
 

where

\begin{eqnarray}
  (x)\id{\{} \lpquote Q \rpquote / \lpquote P \rpquote \id{\}}            = 
  \left\{ 
    \begin{array}{ccc}
      \lpquote Q \rpquote & & x \nameeq \lpquote P \rpquote \\
      x & & otherwise \\
    \end{array}
  \right. \nonumber
\end{eqnarray}

and $z$ is chosen distinct from $\quotep{P}$, $\quotep{Q}$, the free
names in $Q$, and all the names in $R$. Our $\alpha$-equivalence will
be built in the standard way from this substitution.

\begin{remark}\label{rem:no_self_referential_names}
  One consequence of these definitions is that $\forall P. \quotep{P}
  \not\in \freenames{P}$.
\end{remark}

\subsection{ Dynamic quote: an example }

Anticipating something of what's to come, consider applying the
substitution, $\widehat{\id{\{}u / z \id{\}}}$, to the following pair
of processes, $\lift{w}{y!(z)}$ and $w[ \lpquote y!(z) \rpquote ]$.

\begin{eqnarray}
	\lift{w}{y!(z)}\widehat{\id{\{}u / z \id{\}}}
		& = &
		\lift{w}{y!(u)} \nonumber\\
	w[ \lpquote y!(z) \rpquote ] \widehat{ \id{\{}u / z \id{\}} }
		& = &
		w[ \lpquote y!(z) \rpquote ] \nonumber
\end{eqnarray}

Because the body of the process between quotes is impervious to
substitution, we get radically different answers. In fact, by
examining the first process in an input context,
e.g. $x?(z).\lift{w}{y!(z)}$, we see that the process under the lift
operator may be shaped by prefixed inputs binding a name inside it. In
this sense, the lift operator will be seen as a way to dynamically
construct processes before reifying them as names.

Finally equipped with these standard features we can present the
dynamics of the calculus.

\subsubsection{Operational semantics} 

Finally, we introduce the computational dynamics. What marks these
algebras as distinct from other more traditionally studied algebraic
structures, e.g. vector spaces or polynomial rings, is the manner in
which dynamics is captured. In traditional structures, dynamics is typically
expressed through morphisms between such structures, as in linear maps
between vector spaces or morphisms between rings. In algebras
associated with the semantics of computation, the dynamics is
expressed as part of the algebraic structure itself, through a
reduction reduction relation typically denoted by $\red$. Below, we
give a recursive presentation of this relation for the calculus used
in the encoding.

$\red \subseteq \pi \times \pi$
$\red : \pi \to \mathcal{P}(\pi)$

\begin{mathpar}
  \inferrule* [lab=Comm] { \textsf{match}( x_{src}, x_{trgt} ) } { x_{trgt}?(y)P \; | \; x_{src}!\langle {Q} \rangle \red P\{\quotep{Q}/y}\} }
  \and \\
  \inferrule* [lab=Par] {{P} \red {P}'} {{{P} | {Q}} \red {{P}' | {Q}}}
  \and
  \inferrule* [lab=Equiv]{{{P} \scong {P}'} \andalso {{P}' \red {Q}'} \andalso {{Q}' \scong {Q}}}{{P} \red {Q}}
\end{mathpar}

\begin{eqnarray*}
  match_{\equiv} (\quotep{P},\quotep{Q}) & := & P \equiv Q \\
  match_{\dagger}(\quotep{P},\quotep{Q}) & := & \forall R. P|Q \red^{*} R => R \red^{*} 0 \\
  match_{K}(\quotep{P},\quotep{Q}) & := & K \mbox{ for some context } K
\end{eqnarray*}

$u?(x)P | u!\langle Q \rangle \red P\{\quotep{Q}/x\}$

%We write $\wred$ for $\red^*$, and $P\red$ if $\exists Q $ such that $ P \red Q$.
We write $P\red$ if $\exists Q $ such that $ P \red Q$ and $P\not\red$, otherwise.

\section{Replication}

As mentioned before, it is known that replication (and hence
recursion) can be implemented in a higher-order process algebra
\cite{SangiorgiWalker}. As our first example of calculation with the
machinery thus far presented we give the construction explicitly in
the {\rhoc}.

\begin{eqnarray}
	D_{x} & := & \prefix{x}{y}{(\binpar{\outputp{x}{y}}{@{y}})} \nonumber\\
	\bangp_{x}{P} & := & \binpar{{x}!\langle{\binpar{D_{x}}{P}}\rangle}{D_{x}} \nonumber
\end{eqnarray}

\begin{eqnarray}
	\bangp_{x}{P} & & \nonumber\\
	=
	& {x}!\langle{(\prefix{x}{y}{(\outputp{x}{y} | @{y})) | P}}\rangle 
	      | \prefix{x}{y}{(\outputp{x}{y} | @{y})} & \nonumber\\
	\red
	& (\outputp{x}{y} | @{y})\substn{\quotep{(\prefix{x}{y}{(@{y} | \outputp{x}{y})) | P}}}{y} & \nonumber\\
	=
	& \outputp{x}{\quotep{(\prefix{x}{y}{(\outputp{x}{y} | @{y})) | P}}}
	  | {(\prefix{x}{y}{(\outputp{x}{y} | @{y})) | P}} & \nonumber\\
	\red
	& \ldots & \nonumber\\
	\red^*
	& P | P | \ldots & \nonumber
\end{eqnarray}

Of course, this encoding, as an implementation, runs away, unfolding
$\bangp{P}$ eagerly. A lazier and more implementable replication
operator, restricted to input-guarded processes, may be obtained as follows.

\begin{eqnarray}
\bangp{\prefix{u}{v}{P}} 
	:= 
	\binpar{\lift{x}{\prefix{u}{v}{(\binpar{D(x)}{P})}}}{D(x)} \nonumber
\end{eqnarray}

\begin{remark}
  Note that the lazier definition still does not deal with summation
  or mixed summation (i.e. sums over input and output). The reader is
  invited to construct definitions of replication that deal with these
  features. 

  Further, the definitions are parameterized in a name, $x$. Can you,
  gentle reader, make a definition that eliminates this parameter and
  guarantees no accidental interaction between the replication
  machinery and the process being replicated -- i.e. no accidental
  sharing of names used by the process to get its work done and the
  name(s) used by the replication to effect copying. This latter
  revision of the definition of replication is crucial to obtaining
  the expected identity $!!P \sim !P$.
\end{remark}

\begin{remark}\label{rem:paradoxical_combinator}
  The reader familiar with the lambda calculus will have noticed the
  similarity between $D$ and the paradoxical combinator.

  [Ed. note: the existence of this seems to suggest we have to be more
  restrictive on the set of processes and names we admit if we are to
  support no-cloning.]
\end{remark}

\subsubsection{Bisimulation}

The computational dynamics gives rise to another kind of equivalence,
the equivalence of computational behavior. As previously mentioned
this is typically captured \emph{via} some form of bisimulation.

% The notion we use in this paper is weak barbed bisimulation
% \cite{milner91polyadicpi}.

The notion we use in this paper is derived from weak barbed
bisimulation \cite{milner91polyadicpi}. 

\begin{definition}
An \emph{observation relation}, $\downarrow_{\mathcal N}$, over a set
of names, $\mathcal N$, is the smallest relation satisfying the rules
below.

\infrule[Out-barb]{y \in {\mathcal N}, \; x \nameeq y}
		  {\outputp{x}{v} \downarrow_{\mathcal N} x}
\infrule[Par-barb]{\mbox{$P\downarrow_{\mathcal N} x$ or $Q\downarrow_{\mathcal N} x$}}
		  {\binpar{P}{Q} \downarrow_{\mathcal N} x}

We write $P \Downarrow_{\mathcal N} x$ if there is $Q$ such that 
$P \wred Q$ and $Q \downarrow_{\mathcal N} x$.
\end{definition}

\begin{definition}
%\label{def.bbisim}
An  ${\mathcal N}$-\emph{barbed bisimulation} over a set of names, ${\mathcal N}$, is a symmetric binary relation 
${\mathcal S}_{\mathcal N}$ between agents such that $P\rel{S}_{\mathcal N}Q$ implies:
\begin{enumerate}
\item If $P \red P'$ then $Q \wred Q'$ and $P'\rel{S}_{\mathcal N} Q'$.
\item If $P\downarrow_{\mathcal N} x$, then $Q\Downarrow_{\mathcal N} x$.
\end{enumerate}
$P$ is ${\mathcal N}$-barbed bisimilar to $Q$, written
$P \wbbisim_{\mathcal N} Q$, if $P \rel{S}_{\mathcal N} Q$ for some ${\mathcal N}$-barbed bisimulation ${\mathcal S}_{\mathcal N}$.
\end{definition}

$\mathcal{R} \subseteq \pi \times \pi$

$P \mathcal{R} Q => \forall P'. P \red P' \Rightarrow \exists Q'. Q \red Q', P' \mathcal{R} Q'$

$P \vdash x \Rightarrow Q \vdash x$

\begin{mathpar}
  \inferrule*[lab=Out-barb]{x \nameeq y}{{y}!\langle{Q}\rangle \vdash x}
  \and
  \inferrule*[lab=Par-barb]{\mbox{$P\vdash x$ or $Q\vdash x$}}{\binpar{P}{Q} \vdash x}
\end{mathpar}

\subsubsection{Contexts}

One of the principle advantages of computational calculi like the
$\pi$-calculus is a well-defined notion of context,
contextual-equivalence and a correlation between
contextual-equivalence and notions of bisimulation. The notion of
context allows the decomposition of a process into (sub-)process and
its syntactic environment, its context. Thus, a context may be
thought of as a process with a ``hole'' (written $\Box$) in it. The
application of a context $M$ to a process $P$, written $M[P]$, is
tantamount to filling the hole in $M$ with $P$. In this paper we do
not need the full weight of this theory, but do make use of the notion
of context in the proof the main theorem. 

\begin{mathpar}
  \inferrule* [lab=summation] {} {{M_{M},M_{N}} \bc \Box \;|\; x.M_{A} \;|\; M_{M}+M_{N}}
  \and
  \inferrule* [lab=agent] {} {{M_{A}} \bc (\vec{x})M_{P} \;| \; \clift{P_0,\ldots,M_{P},\ldots,P_N}}
  \and \\
  \inferrule* [lab=process] {} {{M_{P}} \bc M_{N} \;| \;P|M_{P} }
\end{mathpar} 

\begin{mathpar}
  \inferrule* [lab=sychronization] {} {M_{N} \bc \Box \;|\; x?M_{F} \;|\; x!M_{C}}
  \and
  \inferrule* [lab=abstraction] {} {{M_{F}} \bc (x)M_{P} }
  \and
  \inferrule* [lab=concretion] {} {{M_{C}} \bc \langle M_{P} \rangle }
  \and \\
  \inferrule* [lab=process] {} {{M_{P}} \bc M_{N} \;| \;P|M_{P} }
\end{mathpar}

\begin{definition}[contextual application] Given a context $M$, and
  process $P$, we define the \emph{contextual application}, $M[P] :=
  M\{P/\Box\}$. That is, the contextual application of M to P is the
  substitution of $P$ for $\Box$ in $M$.
\end{definition}

$\meaningof{-} : L \to \mathcal{P}(\pi)$

\begin{mathpar}
  \inferrule* [lab=collection] {} {\meaningof{true} = \pi, \and \meaningof{~E} = \pi \setminus \meaningof{E}, \and \meaningof{E_{1} \& E_{2}} = \meaningof{E_{1}} \cap \meaningof{E_{2}}}
\end{mathpar}

\begin{mathpar}
  \inferrule* [lab=structure] {} {\meaningof{0} = \{ P \in \pi | P \equiv 0 \}, \and \\ \meaningof{E_1 | E_2} = \{ P \in \pi | P \equiv P_{1} | P_{2}, P_{1} \in \meaningof{E_{1}}, P_{2} \in \meaningof{E_2}\} }
\end{mathpar}

\begin{mathpar}
 \inferrule* [lab=behavior] {} {\meaningof{\langle a?b \rangle E} = \{ P \in \pi | P \equiv Q | u?(y)P', \\ \and \\\\ \and \\ \;\;\; u \in \meaningof{a}, \forall z.P'\{z/y\} \in \meaningof{E\{z/b\}}\}, \and \\ \meaningof{a!E} = \{ P \in \pi | P \equiv Q | x!\langle P' \rangle, x \in \meaningof{a} P' \in \meaningof{E}\} }
\end{mathpar}

\begin{mathpar}
 \inferrule* [lab=nominal] {} {\meaningof{\quotep{E}} = \{ \quotep{P} \in \quotep{\pi} | P \in \meaningof{E} \}, \and \meaningof{\quotep{P}} = \{ \quotep{Q} \in \quotep{\pi} | P \equiv Q \} \and \\ \meaningof{@\quotep{E}} = \{ P \in \pi | P \equiv @x, x \in \meaningof{E} \}}
\end{mathpar}

\begin{eqnarray*}
  \\
  \meaningof{-} : TS \to ST
\end{eqnarray*}

\begin{eqnarray*}
  \\
  L : TS \to ST
\end{eqnarray*}

\begin{eqnarray*}
  \\
  P \models E \iff P \in \meaningof{E}
\end{eqnarray*}

\begin{eqnarray*}
  P \approx_{L} Q \iff \forall E \in L. P \models E \iff Q \models E
\end{eqnarray*}

\begin{eqnarray*}
  P \approx_{K} Q
\end{eqnarray*}

\begin{eqnarray*}
  P \approx Q
\end{eqnarray*}

$\approx_{K} = \approx = \approx_{L}$

\subsubsection{Contextual duality}

Note that contexts extend the quotation operation to a family of
operations from processes to names. Given a context, $M$, we can
define a \emph{nominal context}, $\quotep{M}$ by $\quotep{M}[P] :=
\quotep{M[P]}$. To foreshadow what is to come we observe that these
operations enjoy a duality with processes very much like the duality
between vectors and maps from vectors to scalars.

Further, because the calculus is essentially higher-order, we have a
correspondence between contexts and processes. More specifically,
given a name $x$ and a context $M$ we can construct $M^{*}_{x}$ such
that 

\begin{mathpar}
  M^{*}_{x} | \lift{x}{P} \red M[P]
\end{mathpar}

namely,

\begin{mathpar}
  M^{*}_{x} := x?(u).M[\dropn{u}]
\end{mathpar}

The dependence of $M^{*}_{x}$ on a name makes it an abstraction, 

\begin{mathpar}
  M^{*} := (x)x?(u).M[\dropn{u}]
\end{mathpar}

\subsection{Additional notation}

It will sometimes be convenient to denote the process a name
quotes. We already have the notation $x = \quotep{P}$, but it will be
convenient to introduce an alternate notation, $\procn{x}$, when we
want to emphasize the connection to the use of the name. Note that, by
virtue of name equivalence, $\quotep{\procn{x}} \nameeq x$; so, the
notation is consistent with previous definitions.

Further, because names have structure it is possible to effect
substitutions on the basis of that structure. This means we need to
upgrade our notation for substitutions, which we accomplish by
adapting comprehension notation. Thus,

\begin{mathpar}
  P\{ y / x : x \in S \}
\end{mathpar}

is interpreted to mean the process derived from P by replacing (in a
capture-avoiding manner) each occurrence of $x$ in $S$ by $y$. For example,

\begin{mathpar}
  P\{ \quotep{\procn{x}|\procn{x}} / x : x \in \freenames{P} \}
\end{mathpar}

will replace each (occurrence) of a free name $x$ in $P$ by
$\quotep{\procn{x}|\procn{x}}$.

Also, we will avail ourselves of the notation $x^{L}$ and $x^{R}$ to
denote injections of a name into disjoint copies of the name
space. There are numerous ways to accomplish this. One example can be
found in \cite{MeredithR05}. This notation overloads to vectors of
names: $\vec{x}^{\pi} := (x_{i}^{\pi} \; : \; 0 \leq i < |\vec{x}| )$ where $\pi \in \{L,R\}$.

We also use $P^{\Box} := P|\Box$.

In \cite{MeredithR05} an interpretation of the new operator is
given. It turns out that there are several possible interpretations
all enjoying the requisite algebraic properties of the operator (see
\cite{milner91polyadicpi}). We will therefore make liberal use of
$(\nu\; \vec{x})P$.

% subsection the_syntax_and_semantics_of_the_notation_system (end)   

\input{qm2pi.qmops} 

\input{qm2pi.sterngerlach} 

\input{qm2pi.metric} 

% section concurrent_process_calculi (end)

%\input{qm2pi.proofsketch}

% section proof sketch (end)

%\input{qm2pi.slviaknots} 

% section spatial logic via knots (end)

\input{qm2pi.conclusion}

% section conclusion (end)

%\input{qm2pi.dtcodes} 

% section wiring algorithm (end)

\input{qm2pi.ack} 

% section acknowledgments (end)

\newpage


\bibliographystyle{plain}   
\bibliography{../../biblios/main.bib}

\input{qm2pi.rhodetails}

\end{document}

 

% section wiring algorithm (end)

\documentclass[12pt]{llncs}
%\documentclass{jktr}

\usepackage[pdftex]{hyperref}                   
\usepackage {listings}
\usepackage {mathpartir}
\usepackage{bcprules}
%\usepackage{listings}
                       
\usepackage{graphicx} 
%\usepackage[margins=2.5cm,nohead,nofoot]{geometry}
%\usepackage{geometry}
\usepackage{amsfonts}
\usepackage{amstext}
\usepackage{latexsym}
\usepackage{amssymb}
\usepackage{color}


%\include{myPreamble}
\include{qm2pi.local} 

%\ifpdf
%\usepackage[pdftex]{graphicx}
%\else
%\usepackage{graphicx}
%\fi

 % \ifpdf
%  \usepackage{pdfsync}
%  \if


%\title{Brief Article}
%\author{David F. Snyder}
%\author{L.G. Meredith}

%\address{Dept. of Math., Texas State University--San Marcos, San Marcos, TX 78666}
       
\pagestyle{empty}


\begin{document}

\lstset{language=[Objective]Caml,frame=shadowbox}

\input{qm2pi.front}

% section front matter (end)

\input{qm2pi.intro} 
 
% section introduction (end)

% \input{qm2pi.knotations} 

% section notation (end)

\input{qm2pi.process.calculi} 

% section concurrent_process_calculi_and_spatial_logics_ (end)
    
%\input{qm2pi.knots2pi} 

%\input{qm2pi.trefoil} 

%\input{qm2pi.mainthm} 

% subsection basic_interpretation (end)

%\input{qm2pi.rho.presentation} 
\subsection{The syntax and semantics of the notation system}\label{sub:the_syntax_and_semantics_of_the_notation_system} % (fold)

We now summarize a technical presentation of the calculus that
embodies our theory of dynamics. The typical presentation of such a
calculus follows the style of giving generators and relations on
them. The grammar, below, describing term constructors, freely
generates the set of processes, $\Proc$. This set is then quotiented
by a relation known as structural congruence and it is over this set
that the notion of dynamics is expressed. This presentation is
essentially that of \cite{MeredithR05} with the addition of
polyadicity and summation. For readability we have relegated some of
the technical subtleties to an appendix.

\subsubsection{Process grammar}\label{subsub:process_grammar}

\begin{mathpar}
  \inferrule* [lab=synchronization] {} {{M} \bc \pzero \;|\; x?F \;|\; x!C }
  \and
  \inferrule* [lab=abstraction] {} {{F} \bc (x)P}
  \and
  \inferrule* [lab=concretion] {} {{C} \bc \langle Q \rangle}
  \and
  \inferrule* [lab=process] {} {{P,Q} \bc M \;| \;P|Q \;|\; @{x}}
  \and
  \inferrule* [lab=name] {} {{x} \bc \quotep{P}}
\end{mathpar} 

Note that $\vec{x}$ (resp. $\vec{P}$) denotes a vector of names
(resp. processes) of length $|\vec{x}|$ (resp. $|\vec{P}|$). We adopt
the following useful abbreviations.

\begin{mathpar}
   x?(\vec{y}).P := x.(\vec{y})P \and  x\clift{\vec{P}} := x.\clift{\vec{P}}
   \and x!(y) := \lift{x}{\dropn{y}}
   \and \Pi_{i=0}^{n-1}P_i := P_0 | \ldots | P_{n-1}
\end{mathpar}

\subsubsection{Structural congruence}

\paragraph{Free and bound names and alpha-equivalence.} At the
core of structural equivalence is alpha-equivalence which identifies
process that are the same up to a change of variable. Formally, we
recognize the distinction between free and bound names. The free names
of a process, $\freenames{P}$, may be calculated recursively as
follows:

\begin{mathpar}
\freenames{\pzero} := \emptyset
  \and \\
  \freenames{x?(y).P} := \{ x \} \cup (\freenames{P} \setminus \{ y \})
  \and 
  \freenames{x!\langle P \rangle} := \{ x \} \cup \{ P \} 
  \and \\
  \freenames{P|Q} := \freenames{P} \cup \freenames{Q}
  \and \\
  \freenames{@{x}} := \{ x \}
\end{mathpar}

$\pi$
$\quotep{\pi}$

$\freenames{-} : \pi \to \mathcal{P}(\quotep{\pi})$

\begin{eqnarray*}
  \freenames{\pzero} & := & \emptyset \\
  \freenames{x?(y).P} & := & \{ x \} \cup (\freenames{P} \setminus \{ y \}) \\
  \freenames{x!\langle P \rangle} & := & \{ x \} \cup \{ P \} \\
  \freenames{P|Q} & := & \freenames{P} \cup \freenames{Q} \\
  \freenames{\dropn{x}} & := & \{ x \}
\end{eqnarray*}

The bound names of a process, $\boundnames{P}$, are those names occurring in $P$
that are not free. For example, in $x?(y).0$, the name $x$ is free, while $y$ is bound.

\begin{mathpar}
  \inferrule* [lab=monoidal-laws] {} { P|Q \equiv Q|P \and P|0 \equiv P \and P|(Q|R) \equiv (P|Q)|R }
\end{mathpar}

\begin{mathpar}
  \inferrule* [lab=alpha-equivalence] {} { (x)P \equiv (y)P\{y/x\} \and y \not\in \freenames{P} }
\end{mathpar}

\begin{definition}
Then two processes, $P,Q$, are alpha-equivalent if $P = Q\{\vec{y}/\vec{x}\}$ for
some $\vec{x} \in \boundnames{Q},\vec{y} \in \boundnames{P}$, where $Q\{\vec{y}/\vec{x}\}$
denotes the capture-avoiding substitution of $\vec{y}$ for $\vec{x}$ in $Q$.
\end{definition}

\begin{definition}
  The {\em structural congruence} \cite{SangiorgiWalker} , $\equiv$,
  between processes is the least congruence containing
  alpha-equivalence, satisfying the abelian monoid laws
  (associativity, commutativity and $\pzero$ as identity) for parallel
  composition $|$ and for summation $+$.
\end{definition}

\subsection{Name equivalence}

We take name equivalence, written $\nameeq$, to be the smallest
equivalence relation generated by the following rules.

\begin{mathpar}
\inferrule*[lab=Quote-drop]
{ }
{ \quotep{@{x}} \nameeq x }

\inferrule*[lab=Struct-equiv]
{ P \scong Q }
{ \quotep{P} \nameeq \quotep{Q} }
\end{mathpar}

The astute reader will have noticed that the mutual recursion of names
and processes imposes a mutual recursion on alpha-equivalence and
structural equivalence via name-equivalence. Fortunately, all of this
works out pleasantly and we may calculate in the natural way, free of
concern. The reader interested in the details is referred to the
appendix \ref{appendix:rho_details}.

\subsection{Substitution}

We use $\Proc$ for the set of processes, $\QProc$ for the set of
names, and $\id{\{}\vec{y} / \vec{x} \id{\}}$ to denote partial maps,
$s : \QProc \rightarrow \QProc$. A map, $s$ lifts, uniquely, to a map
on process terms, $\widehat{s} : \Proc \rightarrow \Proc$ by the
following equations.

\begin{mathpar}
  (0) \psubstp{Q}{P} := 0 \\
  (R \juxtap S) \psubstp{Q}{P}
  :=    
  (R)\psubstp{Q}{P} \juxtap (S) \psubstp{Q}{P} \\
  (x?(y).R) \psubstp{Q}{P}    
  :=    
  (x)\substp{Q}{P} (z)\concat( (R \psubstn{z}{y}) \psubstp{Q}{P} ) \\
  (\lift{x}{R}) \psubstp{Q}{P}  
  :=
  \lift{(x)\substp{Q}{P}}{ R \psubstp{Q}{P} } \\
%   (\dropn{x})  \psubstp{Q}{P}       
%   := 
%   \left\{ 
%     \begin{array}{ccc} 
%       \dropn{\quotep{Q}} & & x \nameeq \quotep{P} \\
%       \dropn{x} & & otherwise \\
%     \end{array}
%   \right. 
  (\dropn{x})  \psubstp{Q}{P}       
  := 
  \left\{ 
    \begin{array}{ccc} 
      Q & & x \nameeq \quotep{P} \\
      \dropn{x} & & otherwise \\
    \end{array}
  \right.
\end{mathpar}
 

where

\begin{eqnarray}
  (x)\id{\{} \lpquote Q \rpquote / \lpquote P \rpquote \id{\}}            = 
  \left\{ 
    \begin{array}{ccc}
      \lpquote Q \rpquote & & x \nameeq \lpquote P \rpquote \\
      x & & otherwise \\
    \end{array}
  \right. \nonumber
\end{eqnarray}

and $z$ is chosen distinct from $\quotep{P}$, $\quotep{Q}$, the free
names in $Q$, and all the names in $R$. Our $\alpha$-equivalence will
be built in the standard way from this substitution.

\begin{remark}\label{rem:no_self_referential_names}
  One consequence of these definitions is that $\forall P. \quotep{P}
  \not\in \freenames{P}$.
\end{remark}

\subsection{ Dynamic quote: an example }

Anticipating something of what's to come, consider applying the
substitution, $\widehat{\id{\{}u / z \id{\}}}$, to the following pair
of processes, $\lift{w}{y!(z)}$ and $w[ \lpquote y!(z) \rpquote ]$.

\begin{eqnarray}
	\lift{w}{y!(z)}\widehat{\id{\{}u / z \id{\}}}
		& = &
		\lift{w}{y!(u)} \nonumber\\
	w[ \lpquote y!(z) \rpquote ] \widehat{ \id{\{}u / z \id{\}} }
		& = &
		w[ \lpquote y!(z) \rpquote ] \nonumber
\end{eqnarray}

Because the body of the process between quotes is impervious to
substitution, we get radically different answers. In fact, by
examining the first process in an input context,
e.g. $x?(z).\lift{w}{y!(z)}$, we see that the process under the lift
operator may be shaped by prefixed inputs binding a name inside it. In
this sense, the lift operator will be seen as a way to dynamically
construct processes before reifying them as names.

Finally equipped with these standard features we can present the
dynamics of the calculus.

\subsubsection{Operational semantics} 

Finally, we introduce the computational dynamics. What marks these
algebras as distinct from other more traditionally studied algebraic
structures, e.g. vector spaces or polynomial rings, is the manner in
which dynamics is captured. In traditional structures, dynamics is typically
expressed through morphisms between such structures, as in linear maps
between vector spaces or morphisms between rings. In algebras
associated with the semantics of computation, the dynamics is
expressed as part of the algebraic structure itself, through a
reduction reduction relation typically denoted by $\red$. Below, we
give a recursive presentation of this relation for the calculus used
in the encoding.

$\red \subseteq \pi \times \pi$
$\red : \pi \to \mathcal{P}(\pi)$

\begin{mathpar}
  \inferrule* [lab=Comm] { \textsf{match}( x_{src}, x_{trgt} ) } { x_{trgt}?(y)P \; | \; x_{src}!\langle {Q} \rangle \red P\{\quotep{Q}/y}\} }
  \and \\
  \inferrule* [lab=Par] {{P} \red {P}'} {{{P} | {Q}} \red {{P}' | {Q}}}
  \and
  \inferrule* [lab=Equiv]{{{P} \scong {P}'} \andalso {{P}' \red {Q}'} \andalso {{Q}' \scong {Q}}}{{P} \red {Q}}
\end{mathpar}

\begin{eqnarray*}
  match_{\equiv} (\quotep{P},\quotep{Q}) & := & P \equiv Q \\
  match_{\dagger}(\quotep{P},\quotep{Q}) & := & \forall R. P|Q \red^{*} R => R \red^{*} 0 \\
  match_{K}(\quotep{P},\quotep{Q}) & := & K \mbox{ for some context } K
\end{eqnarray*}

$u?(x)P | u!\langle Q \rangle \red P\{\quotep{Q}/x\}$

%We write $\wred$ for $\red^*$, and $P\red$ if $\exists Q $ such that $ P \red Q$.
We write $P\red$ if $\exists Q $ such that $ P \red Q$ and $P\not\red$, otherwise.

\section{Replication}

As mentioned before, it is known that replication (and hence
recursion) can be implemented in a higher-order process algebra
\cite{SangiorgiWalker}. As our first example of calculation with the
machinery thus far presented we give the construction explicitly in
the {\rhoc}.

\begin{eqnarray}
	D_{x} & := & \prefix{x}{y}{(\binpar{\outputp{x}{y}}{@{y}})} \nonumber\\
	\bangp_{x}{P} & := & \binpar{{x}!\langle{\binpar{D_{x}}{P}}\rangle}{D_{x}} \nonumber
\end{eqnarray}

\begin{eqnarray}
	\bangp_{x}{P} & & \nonumber\\
	=
	& {x}!\langle{(\prefix{x}{y}{(\outputp{x}{y} | @{y})) | P}}\rangle 
	      | \prefix{x}{y}{(\outputp{x}{y} | @{y})} & \nonumber\\
	\red
	& (\outputp{x}{y} | @{y})\substn{\quotep{(\prefix{x}{y}{(@{y} | \outputp{x}{y})) | P}}}{y} & \nonumber\\
	=
	& \outputp{x}{\quotep{(\prefix{x}{y}{(\outputp{x}{y} | @{y})) | P}}}
	  | {(\prefix{x}{y}{(\outputp{x}{y} | @{y})) | P}} & \nonumber\\
	\red
	& \ldots & \nonumber\\
	\red^*
	& P | P | \ldots & \nonumber
\end{eqnarray}

Of course, this encoding, as an implementation, runs away, unfolding
$\bangp{P}$ eagerly. A lazier and more implementable replication
operator, restricted to input-guarded processes, may be obtained as follows.

\begin{eqnarray}
\bangp{\prefix{u}{v}{P}} 
	:= 
	\binpar{\lift{x}{\prefix{u}{v}{(\binpar{D(x)}{P})}}}{D(x)} \nonumber
\end{eqnarray}

\begin{remark}
  Note that the lazier definition still does not deal with summation
  or mixed summation (i.e. sums over input and output). The reader is
  invited to construct definitions of replication that deal with these
  features. 

  Further, the definitions are parameterized in a name, $x$. Can you,
  gentle reader, make a definition that eliminates this parameter and
  guarantees no accidental interaction between the replication
  machinery and the process being replicated -- i.e. no accidental
  sharing of names used by the process to get its work done and the
  name(s) used by the replication to effect copying. This latter
  revision of the definition of replication is crucial to obtaining
  the expected identity $!!P \sim !P$.
\end{remark}

\begin{remark}\label{rem:paradoxical_combinator}
  The reader familiar with the lambda calculus will have noticed the
  similarity between $D$ and the paradoxical combinator.

  [Ed. note: the existence of this seems to suggest we have to be more
  restrictive on the set of processes and names we admit if we are to
  support no-cloning.]
\end{remark}

\subsubsection{Bisimulation}

The computational dynamics gives rise to another kind of equivalence,
the equivalence of computational behavior. As previously mentioned
this is typically captured \emph{via} some form of bisimulation.

% The notion we use in this paper is weak barbed bisimulation
% \cite{milner91polyadicpi}.

The notion we use in this paper is derived from weak barbed
bisimulation \cite{milner91polyadicpi}. 

\begin{definition}
An \emph{observation relation}, $\downarrow_{\mathcal N}$, over a set
of names, $\mathcal N$, is the smallest relation satisfying the rules
below.

\infrule[Out-barb]{y \in {\mathcal N}, \; x \nameeq y}
		  {\outputp{x}{v} \downarrow_{\mathcal N} x}
\infrule[Par-barb]{\mbox{$P\downarrow_{\mathcal N} x$ or $Q\downarrow_{\mathcal N} x$}}
		  {\binpar{P}{Q} \downarrow_{\mathcal N} x}

We write $P \Downarrow_{\mathcal N} x$ if there is $Q$ such that 
$P \wred Q$ and $Q \downarrow_{\mathcal N} x$.
\end{definition}

\begin{definition}
%\label{def.bbisim}
An  ${\mathcal N}$-\emph{barbed bisimulation} over a set of names, ${\mathcal N}$, is a symmetric binary relation 
${\mathcal S}_{\mathcal N}$ between agents such that $P\rel{S}_{\mathcal N}Q$ implies:
\begin{enumerate}
\item If $P \red P'$ then $Q \wred Q'$ and $P'\rel{S}_{\mathcal N} Q'$.
\item If $P\downarrow_{\mathcal N} x$, then $Q\Downarrow_{\mathcal N} x$.
\end{enumerate}
$P$ is ${\mathcal N}$-barbed bisimilar to $Q$, written
$P \wbbisim_{\mathcal N} Q$, if $P \rel{S}_{\mathcal N} Q$ for some ${\mathcal N}$-barbed bisimulation ${\mathcal S}_{\mathcal N}$.
\end{definition}

$\mathcal{R} \subseteq \pi \times \pi$

$P \mathcal{R} Q => \forall P'. P \red P' \Rightarrow \exists Q'. Q \red Q', P' \mathcal{R} Q'$

$P \vdash x \Rightarrow Q \vdash x$

\begin{mathpar}
  \inferrule*[lab=Out-barb]{x \nameeq y}{{y}!\langle{Q}\rangle \vdash x}
  \and
  \inferrule*[lab=Par-barb]{\mbox{$P\vdash x$ or $Q\vdash x$}}{\binpar{P}{Q} \vdash x}
\end{mathpar}

\subsubsection{Contexts}

One of the principle advantages of computational calculi like the
$\pi$-calculus is a well-defined notion of context,
contextual-equivalence and a correlation between
contextual-equivalence and notions of bisimulation. The notion of
context allows the decomposition of a process into (sub-)process and
its syntactic environment, its context. Thus, a context may be
thought of as a process with a ``hole'' (written $\Box$) in it. The
application of a context $M$ to a process $P$, written $M[P]$, is
tantamount to filling the hole in $M$ with $P$. In this paper we do
not need the full weight of this theory, but do make use of the notion
of context in the proof the main theorem. 

\begin{mathpar}
  \inferrule* [lab=summation] {} {{M_{M},M_{N}} \bc \Box \;|\; x.M_{A} \;|\; M_{M}+M_{N}}
  \and
  \inferrule* [lab=agent] {} {{M_{A}} \bc (\vec{x})M_{P} \;| \; \clift{P_0,\ldots,M_{P},\ldots,P_N}}
  \and \\
  \inferrule* [lab=process] {} {{M_{P}} \bc M_{N} \;| \;P|M_{P} }
\end{mathpar} 

\begin{mathpar}
  \inferrule* [lab=sychronization] {} {M_{N} \bc \Box \;|\; x?M_{F} \;|\; x!M_{C}}
  \and
  \inferrule* [lab=abstraction] {} {{M_{F}} \bc (x)M_{P} }
  \and
  \inferrule* [lab=concretion] {} {{M_{C}} \bc \langle M_{P} \rangle }
  \and \\
  \inferrule* [lab=process] {} {{M_{P}} \bc M_{N} \;| \;P|M_{P} }
\end{mathpar}

\begin{definition}[contextual application] Given a context $M$, and
  process $P$, we define the \emph{contextual application}, $M[P] :=
  M\{P/\Box\}$. That is, the contextual application of M to P is the
  substitution of $P$ for $\Box$ in $M$.
\end{definition}

$\meaningof{-} : L \to \mathcal{P}(\pi)$

\begin{mathpar}
  \inferrule* [lab=collection] {} {\meaningof{true} = \pi, \and \meaningof{~E} = \pi \setminus \meaningof{E}, \and \meaningof{E_{1} \& E_{2}} = \meaningof{E_{1}} \cap \meaningof{E_{2}}}
\end{mathpar}

\begin{mathpar}
  \inferrule* [lab=structure] {} {\meaningof{0} = \{ P \in \pi | P \equiv 0 \}, \and \\ \meaningof{E_1 | E_2} = \{ P \in \pi | P \equiv P_{1} | P_{2}, P_{1} \in \meaningof{E_{1}}, P_{2} \in \meaningof{E_2}\} }
\end{mathpar}

\begin{mathpar}
 \inferrule* [lab=behavior] {} {\meaningof{\langle a?b \rangle E} = \{ P \in \pi | P \equiv Q | u?(y)P', \\ \and \\\\ \and \\ \;\;\; u \in \meaningof{a}, \forall z.P'\{z/y\} \in \meaningof{E\{z/b\}}\}, \and \\ \meaningof{a!E} = \{ P \in \pi | P \equiv Q | x!\langle P' \rangle, x \in \meaningof{a} P' \in \meaningof{E}\} }
\end{mathpar}

\begin{mathpar}
 \inferrule* [lab=nominal] {} {\meaningof{\quotep{E}} = \{ \quotep{P} \in \quotep{\pi} | P \in \meaningof{E} \}, \and \meaningof{\quotep{P}} = \{ \quotep{Q} \in \quotep{\pi} | P \equiv Q \} \and \\ \meaningof{@\quotep{E}} = \{ P \in \pi | P \equiv @x, x \in \meaningof{E} \}}
\end{mathpar}

\begin{eqnarray*}
  \\
  \meaningof{-} : TS \to ST
\end{eqnarray*}

\begin{eqnarray*}
  \\
  L : TS \to ST
\end{eqnarray*}

\begin{eqnarray*}
  \\
  P \models E \iff P \in \meaningof{E}
\end{eqnarray*}

\begin{eqnarray*}
  P \approx_{L} Q \iff \forall E \in L. P \models E \iff Q \models E
\end{eqnarray*}

\begin{eqnarray*}
  P \approx_{K} Q
\end{eqnarray*}

\begin{eqnarray*}
  P \approx Q
\end{eqnarray*}

$\approx_{K} = \approx = \approx_{L}$

\subsubsection{Contextual duality}

Note that contexts extend the quotation operation to a family of
operations from processes to names. Given a context, $M$, we can
define a \emph{nominal context}, $\quotep{M}$ by $\quotep{M}[P] :=
\quotep{M[P]}$. To foreshadow what is to come we observe that these
operations enjoy a duality with processes very much like the duality
between vectors and maps from vectors to scalars.

Further, because the calculus is essentially higher-order, we have a
correspondence between contexts and processes. More specifically,
given a name $x$ and a context $M$ we can construct $M^{*}_{x}$ such
that 

\begin{mathpar}
  M^{*}_{x} | \lift{x}{P} \red M[P]
\end{mathpar}

namely,

\begin{mathpar}
  M^{*}_{x} := x?(u).M[\dropn{u}]
\end{mathpar}

The dependence of $M^{*}_{x}$ on a name makes it an abstraction, 

\begin{mathpar}
  M^{*} := (x)x?(u).M[\dropn{u}]
\end{mathpar}

\subsection{Additional notation}

It will sometimes be convenient to denote the process a name
quotes. We already have the notation $x = \quotep{P}$, but it will be
convenient to introduce an alternate notation, $\procn{x}$, when we
want to emphasize the connection to the use of the name. Note that, by
virtue of name equivalence, $\quotep{\procn{x}} \nameeq x$; so, the
notation is consistent with previous definitions.

Further, because names have structure it is possible to effect
substitutions on the basis of that structure. This means we need to
upgrade our notation for substitutions, which we accomplish by
adapting comprehension notation. Thus,

\begin{mathpar}
  P\{ y / x : x \in S \}
\end{mathpar}

is interpreted to mean the process derived from P by replacing (in a
capture-avoiding manner) each occurrence of $x$ in $S$ by $y$. For example,

\begin{mathpar}
  P\{ \quotep{\procn{x}|\procn{x}} / x : x \in \freenames{P} \}
\end{mathpar}

will replace each (occurrence) of a free name $x$ in $P$ by
$\quotep{\procn{x}|\procn{x}}$.

Also, we will avail ourselves of the notation $x^{L}$ and $x^{R}$ to
denote injections of a name into disjoint copies of the name
space. There are numerous ways to accomplish this. One example can be
found in \cite{MeredithR05}. This notation overloads to vectors of
names: $\vec{x}^{\pi} := (x_{i}^{\pi} \; : \; 0 \leq i < |\vec{x}| )$ where $\pi \in \{L,R\}$.

We also use $P^{\Box} := P|\Box$.

In \cite{MeredithR05} an interpretation of the new operator is
given. It turns out that there are several possible interpretations
all enjoying the requisite algebraic properties of the operator (see
\cite{milner91polyadicpi}). We will therefore make liberal use of
$(\nu\; \vec{x})P$.

% subsection the_syntax_and_semantics_of_the_notation_system (end)   

\input{qm2pi.qmops} 

\input{qm2pi.sterngerlach} 

\input{qm2pi.metric} 

% section concurrent_process_calculi (end)

%\input{qm2pi.proofsketch}

% section proof sketch (end)

%\input{qm2pi.slviaknots} 

% section spatial logic via knots (end)

\input{qm2pi.conclusion}

% section conclusion (end)

%\input{qm2pi.dtcodes} 

% section wiring algorithm (end)

\input{qm2pi.ack} 

% section acknowledgments (end)

\newpage


\bibliographystyle{plain}   
\bibliography{../../biblios/main.bib}

\input{qm2pi.rhodetails}

\end{document}

 

% section acknowledgments (end)

\newpage


\bibliographystyle{plain}   
\bibliography{../../biblios/main.bib}

\documentclass[12pt]{llncs}
%\documentclass{jktr}

\usepackage[pdftex]{hyperref}                   
\usepackage {listings}
\usepackage {mathpartir}
\usepackage{bcprules}
%\usepackage{listings}
                       
\usepackage{graphicx} 
%\usepackage[margins=2.5cm,nohead,nofoot]{geometry}
%\usepackage{geometry}
\usepackage{amsfonts}
\usepackage{amstext}
\usepackage{latexsym}
\usepackage{amssymb}
\usepackage{color}


%\include{myPreamble}
\include{qm2pi.local} 

%\ifpdf
%\usepackage[pdftex]{graphicx}
%\else
%\usepackage{graphicx}
%\fi

 % \ifpdf
%  \usepackage{pdfsync}
%  \if


%\title{Brief Article}
%\author{David F. Snyder}
%\author{L.G. Meredith}

%\address{Dept. of Math., Texas State University--San Marcos, San Marcos, TX 78666}
       
\pagestyle{empty}


\begin{document}

\lstset{language=[Objective]Caml,frame=shadowbox}

\input{qm2pi.front}

% section front matter (end)

\input{qm2pi.intro} 
 
% section introduction (end)

% \input{qm2pi.knotations} 

% section notation (end)

\input{qm2pi.process.calculi} 

% section concurrent_process_calculi_and_spatial_logics_ (end)
    
%\input{qm2pi.knots2pi} 

%\input{qm2pi.trefoil} 

%\input{qm2pi.mainthm} 

% subsection basic_interpretation (end)

%\input{qm2pi.rho.presentation} 
\subsection{The syntax and semantics of the notation system}\label{sub:the_syntax_and_semantics_of_the_notation_system} % (fold)

We now summarize a technical presentation of the calculus that
embodies our theory of dynamics. The typical presentation of such a
calculus follows the style of giving generators and relations on
them. The grammar, below, describing term constructors, freely
generates the set of processes, $\Proc$. This set is then quotiented
by a relation known as structural congruence and it is over this set
that the notion of dynamics is expressed. This presentation is
essentially that of \cite{MeredithR05} with the addition of
polyadicity and summation. For readability we have relegated some of
the technical subtleties to an appendix.

\subsubsection{Process grammar}\label{subsub:process_grammar}

\begin{mathpar}
  \inferrule* [lab=synchronization] {} {{M} \bc \pzero \;|\; x?F \;|\; x!C }
  \and
  \inferrule* [lab=abstraction] {} {{F} \bc (x)P}
  \and
  \inferrule* [lab=concretion] {} {{C} \bc \langle Q \rangle}
  \and
  \inferrule* [lab=process] {} {{P,Q} \bc M \;| \;P|Q \;|\; @{x}}
  \and
  \inferrule* [lab=name] {} {{x} \bc \quotep{P}}
\end{mathpar} 

Note that $\vec{x}$ (resp. $\vec{P}$) denotes a vector of names
(resp. processes) of length $|\vec{x}|$ (resp. $|\vec{P}|$). We adopt
the following useful abbreviations.

\begin{mathpar}
   x?(\vec{y}).P := x.(\vec{y})P \and  x\clift{\vec{P}} := x.\clift{\vec{P}}
   \and x!(y) := \lift{x}{\dropn{y}}
   \and \Pi_{i=0}^{n-1}P_i := P_0 | \ldots | P_{n-1}
\end{mathpar}

\subsubsection{Structural congruence}

\paragraph{Free and bound names and alpha-equivalence.} At the
core of structural equivalence is alpha-equivalence which identifies
process that are the same up to a change of variable. Formally, we
recognize the distinction between free and bound names. The free names
of a process, $\freenames{P}$, may be calculated recursively as
follows:

\begin{mathpar}
\freenames{\pzero} := \emptyset
  \and \\
  \freenames{x?(y).P} := \{ x \} \cup (\freenames{P} \setminus \{ y \})
  \and 
  \freenames{x!\langle P \rangle} := \{ x \} \cup \{ P \} 
  \and \\
  \freenames{P|Q} := \freenames{P} \cup \freenames{Q}
  \and \\
  \freenames{@{x}} := \{ x \}
\end{mathpar}

$\pi$
$\quotep{\pi}$

$\freenames{-} : \pi \to \mathcal{P}(\quotep{\pi})$

\begin{eqnarray*}
  \freenames{\pzero} & := & \emptyset \\
  \freenames{x?(y).P} & := & \{ x \} \cup (\freenames{P} \setminus \{ y \}) \\
  \freenames{x!\langle P \rangle} & := & \{ x \} \cup \{ P \} \\
  \freenames{P|Q} & := & \freenames{P} \cup \freenames{Q} \\
  \freenames{\dropn{x}} & := & \{ x \}
\end{eqnarray*}

The bound names of a process, $\boundnames{P}$, are those names occurring in $P$
that are not free. For example, in $x?(y).0$, the name $x$ is free, while $y$ is bound.

\begin{mathpar}
  \inferrule* [lab=monoidal-laws] {} { P|Q \equiv Q|P \and P|0 \equiv P \and P|(Q|R) \equiv (P|Q)|R }
\end{mathpar}

\begin{mathpar}
  \inferrule* [lab=alpha-equivalence] {} { (x)P \equiv (y)P\{y/x\} \and y \not\in \freenames{P} }
\end{mathpar}

\begin{definition}
Then two processes, $P,Q$, are alpha-equivalent if $P = Q\{\vec{y}/\vec{x}\}$ for
some $\vec{x} \in \boundnames{Q},\vec{y} \in \boundnames{P}$, where $Q\{\vec{y}/\vec{x}\}$
denotes the capture-avoiding substitution of $\vec{y}$ for $\vec{x}$ in $Q$.
\end{definition}

\begin{definition}
  The {\em structural congruence} \cite{SangiorgiWalker} , $\equiv$,
  between processes is the least congruence containing
  alpha-equivalence, satisfying the abelian monoid laws
  (associativity, commutativity and $\pzero$ as identity) for parallel
  composition $|$ and for summation $+$.
\end{definition}

\subsection{Name equivalence}

We take name equivalence, written $\nameeq$, to be the smallest
equivalence relation generated by the following rules.

\begin{mathpar}
\inferrule*[lab=Quote-drop]
{ }
{ \quotep{@{x}} \nameeq x }

\inferrule*[lab=Struct-equiv]
{ P \scong Q }
{ \quotep{P} \nameeq \quotep{Q} }
\end{mathpar}

The astute reader will have noticed that the mutual recursion of names
and processes imposes a mutual recursion on alpha-equivalence and
structural equivalence via name-equivalence. Fortunately, all of this
works out pleasantly and we may calculate in the natural way, free of
concern. The reader interested in the details is referred to the
appendix \ref{appendix:rho_details}.

\subsection{Substitution}

We use $\Proc$ for the set of processes, $\QProc$ for the set of
names, and $\id{\{}\vec{y} / \vec{x} \id{\}}$ to denote partial maps,
$s : \QProc \rightarrow \QProc$. A map, $s$ lifts, uniquely, to a map
on process terms, $\widehat{s} : \Proc \rightarrow \Proc$ by the
following equations.

\begin{mathpar}
  (0) \psubstp{Q}{P} := 0 \\
  (R \juxtap S) \psubstp{Q}{P}
  :=    
  (R)\psubstp{Q}{P} \juxtap (S) \psubstp{Q}{P} \\
  (x?(y).R) \psubstp{Q}{P}    
  :=    
  (x)\substp{Q}{P} (z)\concat( (R \psubstn{z}{y}) \psubstp{Q}{P} ) \\
  (\lift{x}{R}) \psubstp{Q}{P}  
  :=
  \lift{(x)\substp{Q}{P}}{ R \psubstp{Q}{P} } \\
%   (\dropn{x})  \psubstp{Q}{P}       
%   := 
%   \left\{ 
%     \begin{array}{ccc} 
%       \dropn{\quotep{Q}} & & x \nameeq \quotep{P} \\
%       \dropn{x} & & otherwise \\
%     \end{array}
%   \right. 
  (\dropn{x})  \psubstp{Q}{P}       
  := 
  \left\{ 
    \begin{array}{ccc} 
      Q & & x \nameeq \quotep{P} \\
      \dropn{x} & & otherwise \\
    \end{array}
  \right.
\end{mathpar}
 

where

\begin{eqnarray}
  (x)\id{\{} \lpquote Q \rpquote / \lpquote P \rpquote \id{\}}            = 
  \left\{ 
    \begin{array}{ccc}
      \lpquote Q \rpquote & & x \nameeq \lpquote P \rpquote \\
      x & & otherwise \\
    \end{array}
  \right. \nonumber
\end{eqnarray}

and $z$ is chosen distinct from $\quotep{P}$, $\quotep{Q}$, the free
names in $Q$, and all the names in $R$. Our $\alpha$-equivalence will
be built in the standard way from this substitution.

\begin{remark}\label{rem:no_self_referential_names}
  One consequence of these definitions is that $\forall P. \quotep{P}
  \not\in \freenames{P}$.
\end{remark}

\subsection{ Dynamic quote: an example }

Anticipating something of what's to come, consider applying the
substitution, $\widehat{\id{\{}u / z \id{\}}}$, to the following pair
of processes, $\lift{w}{y!(z)}$ and $w[ \lpquote y!(z) \rpquote ]$.

\begin{eqnarray}
	\lift{w}{y!(z)}\widehat{\id{\{}u / z \id{\}}}
		& = &
		\lift{w}{y!(u)} \nonumber\\
	w[ \lpquote y!(z) \rpquote ] \widehat{ \id{\{}u / z \id{\}} }
		& = &
		w[ \lpquote y!(z) \rpquote ] \nonumber
\end{eqnarray}

Because the body of the process between quotes is impervious to
substitution, we get radically different answers. In fact, by
examining the first process in an input context,
e.g. $x?(z).\lift{w}{y!(z)}$, we see that the process under the lift
operator may be shaped by prefixed inputs binding a name inside it. In
this sense, the lift operator will be seen as a way to dynamically
construct processes before reifying them as names.

Finally equipped with these standard features we can present the
dynamics of the calculus.

\subsubsection{Operational semantics} 

Finally, we introduce the computational dynamics. What marks these
algebras as distinct from other more traditionally studied algebraic
structures, e.g. vector spaces or polynomial rings, is the manner in
which dynamics is captured. In traditional structures, dynamics is typically
expressed through morphisms between such structures, as in linear maps
between vector spaces or morphisms between rings. In algebras
associated with the semantics of computation, the dynamics is
expressed as part of the algebraic structure itself, through a
reduction reduction relation typically denoted by $\red$. Below, we
give a recursive presentation of this relation for the calculus used
in the encoding.

$\red \subseteq \pi \times \pi$
$\red : \pi \to \mathcal{P}(\pi)$

\begin{mathpar}
  \inferrule* [lab=Comm] { \textsf{match}( x_{src}, x_{trgt} ) } { x_{trgt}?(y)P \; | \; x_{src}!\langle {Q} \rangle \red P\{\quotep{Q}/y}\} }
  \and \\
  \inferrule* [lab=Par] {{P} \red {P}'} {{{P} | {Q}} \red {{P}' | {Q}}}
  \and
  \inferrule* [lab=Equiv]{{{P} \scong {P}'} \andalso {{P}' \red {Q}'} \andalso {{Q}' \scong {Q}}}{{P} \red {Q}}
\end{mathpar}

\begin{eqnarray*}
  match_{\equiv} (\quotep{P},\quotep{Q}) & := & P \equiv Q \\
  match_{\dagger}(\quotep{P},\quotep{Q}) & := & \forall R. P|Q \red^{*} R => R \red^{*} 0 \\
  match_{K}(\quotep{P},\quotep{Q}) & := & K \mbox{ for some context } K
\end{eqnarray*}

$u?(x)P | u!\langle Q \rangle \red P\{\quotep{Q}/x\}$

%We write $\wred$ for $\red^*$, and $P\red$ if $\exists Q $ such that $ P \red Q$.
We write $P\red$ if $\exists Q $ such that $ P \red Q$ and $P\not\red$, otherwise.

\section{Replication}

As mentioned before, it is known that replication (and hence
recursion) can be implemented in a higher-order process algebra
\cite{SangiorgiWalker}. As our first example of calculation with the
machinery thus far presented we give the construction explicitly in
the {\rhoc}.

\begin{eqnarray}
	D_{x} & := & \prefix{x}{y}{(\binpar{\outputp{x}{y}}{@{y}})} \nonumber\\
	\bangp_{x}{P} & := & \binpar{{x}!\langle{\binpar{D_{x}}{P}}\rangle}{D_{x}} \nonumber
\end{eqnarray}

\begin{eqnarray}
	\bangp_{x}{P} & & \nonumber\\
	=
	& {x}!\langle{(\prefix{x}{y}{(\outputp{x}{y} | @{y})) | P}}\rangle 
	      | \prefix{x}{y}{(\outputp{x}{y} | @{y})} & \nonumber\\
	\red
	& (\outputp{x}{y} | @{y})\substn{\quotep{(\prefix{x}{y}{(@{y} | \outputp{x}{y})) | P}}}{y} & \nonumber\\
	=
	& \outputp{x}{\quotep{(\prefix{x}{y}{(\outputp{x}{y} | @{y})) | P}}}
	  | {(\prefix{x}{y}{(\outputp{x}{y} | @{y})) | P}} & \nonumber\\
	\red
	& \ldots & \nonumber\\
	\red^*
	& P | P | \ldots & \nonumber
\end{eqnarray}

Of course, this encoding, as an implementation, runs away, unfolding
$\bangp{P}$ eagerly. A lazier and more implementable replication
operator, restricted to input-guarded processes, may be obtained as follows.

\begin{eqnarray}
\bangp{\prefix{u}{v}{P}} 
	:= 
	\binpar{\lift{x}{\prefix{u}{v}{(\binpar{D(x)}{P})}}}{D(x)} \nonumber
\end{eqnarray}

\begin{remark}
  Note that the lazier definition still does not deal with summation
  or mixed summation (i.e. sums over input and output). The reader is
  invited to construct definitions of replication that deal with these
  features. 

  Further, the definitions are parameterized in a name, $x$. Can you,
  gentle reader, make a definition that eliminates this parameter and
  guarantees no accidental interaction between the replication
  machinery and the process being replicated -- i.e. no accidental
  sharing of names used by the process to get its work done and the
  name(s) used by the replication to effect copying. This latter
  revision of the definition of replication is crucial to obtaining
  the expected identity $!!P \sim !P$.
\end{remark}

\begin{remark}\label{rem:paradoxical_combinator}
  The reader familiar with the lambda calculus will have noticed the
  similarity between $D$ and the paradoxical combinator.

  [Ed. note: the existence of this seems to suggest we have to be more
  restrictive on the set of processes and names we admit if we are to
  support no-cloning.]
\end{remark}

\subsubsection{Bisimulation}

The computational dynamics gives rise to another kind of equivalence,
the equivalence of computational behavior. As previously mentioned
this is typically captured \emph{via} some form of bisimulation.

% The notion we use in this paper is weak barbed bisimulation
% \cite{milner91polyadicpi}.

The notion we use in this paper is derived from weak barbed
bisimulation \cite{milner91polyadicpi}. 

\begin{definition}
An \emph{observation relation}, $\downarrow_{\mathcal N}$, over a set
of names, $\mathcal N$, is the smallest relation satisfying the rules
below.

\infrule[Out-barb]{y \in {\mathcal N}, \; x \nameeq y}
		  {\outputp{x}{v} \downarrow_{\mathcal N} x}
\infrule[Par-barb]{\mbox{$P\downarrow_{\mathcal N} x$ or $Q\downarrow_{\mathcal N} x$}}
		  {\binpar{P}{Q} \downarrow_{\mathcal N} x}

We write $P \Downarrow_{\mathcal N} x$ if there is $Q$ such that 
$P \wred Q$ and $Q \downarrow_{\mathcal N} x$.
\end{definition}

\begin{definition}
%\label{def.bbisim}
An  ${\mathcal N}$-\emph{barbed bisimulation} over a set of names, ${\mathcal N}$, is a symmetric binary relation 
${\mathcal S}_{\mathcal N}$ between agents such that $P\rel{S}_{\mathcal N}Q$ implies:
\begin{enumerate}
\item If $P \red P'$ then $Q \wred Q'$ and $P'\rel{S}_{\mathcal N} Q'$.
\item If $P\downarrow_{\mathcal N} x$, then $Q\Downarrow_{\mathcal N} x$.
\end{enumerate}
$P$ is ${\mathcal N}$-barbed bisimilar to $Q$, written
$P \wbbisim_{\mathcal N} Q$, if $P \rel{S}_{\mathcal N} Q$ for some ${\mathcal N}$-barbed bisimulation ${\mathcal S}_{\mathcal N}$.
\end{definition}

$\mathcal{R} \subseteq \pi \times \pi$

$P \mathcal{R} Q => \forall P'. P \red P' \Rightarrow \exists Q'. Q \red Q', P' \mathcal{R} Q'$

$P \vdash x \Rightarrow Q \vdash x$

\begin{mathpar}
  \inferrule*[lab=Out-barb]{x \nameeq y}{{y}!\langle{Q}\rangle \vdash x}
  \and
  \inferrule*[lab=Par-barb]{\mbox{$P\vdash x$ or $Q\vdash x$}}{\binpar{P}{Q} \vdash x}
\end{mathpar}

\subsubsection{Contexts}

One of the principle advantages of computational calculi like the
$\pi$-calculus is a well-defined notion of context,
contextual-equivalence and a correlation between
contextual-equivalence and notions of bisimulation. The notion of
context allows the decomposition of a process into (sub-)process and
its syntactic environment, its context. Thus, a context may be
thought of as a process with a ``hole'' (written $\Box$) in it. The
application of a context $M$ to a process $P$, written $M[P]$, is
tantamount to filling the hole in $M$ with $P$. In this paper we do
not need the full weight of this theory, but do make use of the notion
of context in the proof the main theorem. 

\begin{mathpar}
  \inferrule* [lab=summation] {} {{M_{M},M_{N}} \bc \Box \;|\; x.M_{A} \;|\; M_{M}+M_{N}}
  \and
  \inferrule* [lab=agent] {} {{M_{A}} \bc (\vec{x})M_{P} \;| \; \clift{P_0,\ldots,M_{P},\ldots,P_N}}
  \and \\
  \inferrule* [lab=process] {} {{M_{P}} \bc M_{N} \;| \;P|M_{P} }
\end{mathpar} 

\begin{mathpar}
  \inferrule* [lab=sychronization] {} {M_{N} \bc \Box \;|\; x?M_{F} \;|\; x!M_{C}}
  \and
  \inferrule* [lab=abstraction] {} {{M_{F}} \bc (x)M_{P} }
  \and
  \inferrule* [lab=concretion] {} {{M_{C}} \bc \langle M_{P} \rangle }
  \and \\
  \inferrule* [lab=process] {} {{M_{P}} \bc M_{N} \;| \;P|M_{P} }
\end{mathpar}

\begin{definition}[contextual application] Given a context $M$, and
  process $P$, we define the \emph{contextual application}, $M[P] :=
  M\{P/\Box\}$. That is, the contextual application of M to P is the
  substitution of $P$ for $\Box$ in $M$.
\end{definition}

$\meaningof{-} : L \to \mathcal{P}(\pi)$

\begin{mathpar}
  \inferrule* [lab=collection] {} {\meaningof{true} = \pi, \and \meaningof{~E} = \pi \setminus \meaningof{E}, \and \meaningof{E_{1} \& E_{2}} = \meaningof{E_{1}} \cap \meaningof{E_{2}}}
\end{mathpar}

\begin{mathpar}
  \inferrule* [lab=structure] {} {\meaningof{0} = \{ P \in \pi | P \equiv 0 \}, \and \\ \meaningof{E_1 | E_2} = \{ P \in \pi | P \equiv P_{1} | P_{2}, P_{1} \in \meaningof{E_{1}}, P_{2} \in \meaningof{E_2}\} }
\end{mathpar}

\begin{mathpar}
 \inferrule* [lab=behavior] {} {\meaningof{\langle a?b \rangle E} = \{ P \in \pi | P \equiv Q | u?(y)P', \\ \and \\\\ \and \\ \;\;\; u \in \meaningof{a}, \forall z.P'\{z/y\} \in \meaningof{E\{z/b\}}\}, \and \\ \meaningof{a!E} = \{ P \in \pi | P \equiv Q | x!\langle P' \rangle, x \in \meaningof{a} P' \in \meaningof{E}\} }
\end{mathpar}

\begin{mathpar}
 \inferrule* [lab=nominal] {} {\meaningof{\quotep{E}} = \{ \quotep{P} \in \quotep{\pi} | P \in \meaningof{E} \}, \and \meaningof{\quotep{P}} = \{ \quotep{Q} \in \quotep{\pi} | P \equiv Q \} \and \\ \meaningof{@\quotep{E}} = \{ P \in \pi | P \equiv @x, x \in \meaningof{E} \}}
\end{mathpar}

\begin{eqnarray*}
  \\
  \meaningof{-} : TS \to ST
\end{eqnarray*}

\begin{eqnarray*}
  \\
  L : TS \to ST
\end{eqnarray*}

\begin{eqnarray*}
  \\
  P \models E \iff P \in \meaningof{E}
\end{eqnarray*}

\begin{eqnarray*}
  P \approx_{L} Q \iff \forall E \in L. P \models E \iff Q \models E
\end{eqnarray*}

\begin{eqnarray*}
  P \approx_{K} Q
\end{eqnarray*}

\begin{eqnarray*}
  P \approx Q
\end{eqnarray*}

$\approx_{K} = \approx = \approx_{L}$

\subsubsection{Contextual duality}

Note that contexts extend the quotation operation to a family of
operations from processes to names. Given a context, $M$, we can
define a \emph{nominal context}, $\quotep{M}$ by $\quotep{M}[P] :=
\quotep{M[P]}$. To foreshadow what is to come we observe that these
operations enjoy a duality with processes very much like the duality
between vectors and maps from vectors to scalars.

Further, because the calculus is essentially higher-order, we have a
correspondence between contexts and processes. More specifically,
given a name $x$ and a context $M$ we can construct $M^{*}_{x}$ such
that 

\begin{mathpar}
  M^{*}_{x} | \lift{x}{P} \red M[P]
\end{mathpar}

namely,

\begin{mathpar}
  M^{*}_{x} := x?(u).M[\dropn{u}]
\end{mathpar}

The dependence of $M^{*}_{x}$ on a name makes it an abstraction, 

\begin{mathpar}
  M^{*} := (x)x?(u).M[\dropn{u}]
\end{mathpar}

\subsection{Additional notation}

It will sometimes be convenient to denote the process a name
quotes. We already have the notation $x = \quotep{P}$, but it will be
convenient to introduce an alternate notation, $\procn{x}$, when we
want to emphasize the connection to the use of the name. Note that, by
virtue of name equivalence, $\quotep{\procn{x}} \nameeq x$; so, the
notation is consistent with previous definitions.

Further, because names have structure it is possible to effect
substitutions on the basis of that structure. This means we need to
upgrade our notation for substitutions, which we accomplish by
adapting comprehension notation. Thus,

\begin{mathpar}
  P\{ y / x : x \in S \}
\end{mathpar}

is interpreted to mean the process derived from P by replacing (in a
capture-avoiding manner) each occurrence of $x$ in $S$ by $y$. For example,

\begin{mathpar}
  P\{ \quotep{\procn{x}|\procn{x}} / x : x \in \freenames{P} \}
\end{mathpar}

will replace each (occurrence) of a free name $x$ in $P$ by
$\quotep{\procn{x}|\procn{x}}$.

Also, we will avail ourselves of the notation $x^{L}$ and $x^{R}$ to
denote injections of a name into disjoint copies of the name
space. There are numerous ways to accomplish this. One example can be
found in \cite{MeredithR05}. This notation overloads to vectors of
names: $\vec{x}^{\pi} := (x_{i}^{\pi} \; : \; 0 \leq i < |\vec{x}| )$ where $\pi \in \{L,R\}$.

We also use $P^{\Box} := P|\Box$.

In \cite{MeredithR05} an interpretation of the new operator is
given. It turns out that there are several possible interpretations
all enjoying the requisite algebraic properties of the operator (see
\cite{milner91polyadicpi}). We will therefore make liberal use of
$(\nu\; \vec{x})P$.

% subsection the_syntax_and_semantics_of_the_notation_system (end)   

\input{qm2pi.qmops} 

\input{qm2pi.sterngerlach} 

\input{qm2pi.metric} 

% section concurrent_process_calculi (end)

%\input{qm2pi.proofsketch}

% section proof sketch (end)

%\input{qm2pi.slviaknots} 

% section spatial logic via knots (end)

\input{qm2pi.conclusion}

% section conclusion (end)

%\input{qm2pi.dtcodes} 

% section wiring algorithm (end)

\input{qm2pi.ack} 

% section acknowledgments (end)

\newpage


\bibliographystyle{plain}   
\bibliography{../../biblios/main.bib}

\input{qm2pi.rhodetails}

\end{document}



\end{document}

 

% subsection basic_interpretation (end)

%\input{qm2pi.rho.presentation} 
\subsection{The syntax and semantics of the notation system}\label{sub:the_syntax_and_semantics_of_the_notation_system} % (fold)

We now summarize a technical presentation of the calculus that
embodies our theory of dynamics. The typical presentation of such a
calculus follows the style of giving generators and relations on
them. The grammar, below, describing term constructors, freely
generates the set of processes, $\Proc$. This set is then quotiented
by a relation known as structural congruence and it is over this set
that the notion of dynamics is expressed. This presentation is
essentially that of \cite{MeredithR05} with the addition of
polyadicity and summation. For readability we have relegated some of
the technical subtleties to an appendix.

\subsubsection{Process grammar}\label{subsub:process_grammar}

\begin{mathpar}
  \inferrule* [lab=synchronization] {} {{M} \bc \pzero \;|\; x?F \;|\; x!C }
  \and
  \inferrule* [lab=abstraction] {} {{F} \bc (x)P}
  \and
  \inferrule* [lab=concretion] {} {{C} \bc \langle Q \rangle}
  \and
  \inferrule* [lab=process] {} {{P,Q} \bc M \;| \;P|Q \;|\; @{x}}
  \and
  \inferrule* [lab=name] {} {{x} \bc \quotep{P}}
\end{mathpar} 

Note that $\vec{x}$ (resp. $\vec{P}$) denotes a vector of names
(resp. processes) of length $|\vec{x}|$ (resp. $|\vec{P}|$). We adopt
the following useful abbreviations.

\begin{mathpar}
   x?(\vec{y}).P := x.(\vec{y})P \and  x\clift{\vec{P}} := x.\clift{\vec{P}}
   \and x!(y) := \lift{x}{\dropn{y}}
   \and \Pi_{i=0}^{n-1}P_i := P_0 | \ldots | P_{n-1}
\end{mathpar}

\subsubsection{Structural congruence}

\paragraph{Free and bound names and alpha-equivalence.} At the
core of structural equivalence is alpha-equivalence which identifies
process that are the same up to a change of variable. Formally, we
recognize the distinction between free and bound names. The free names
of a process, $\freenames{P}$, may be calculated recursively as
follows:

\begin{mathpar}
\freenames{\pzero} := \emptyset
  \and \\
  \freenames{x?(y).P} := \{ x \} \cup (\freenames{P} \setminus \{ y \})
  \and 
  \freenames{x!\langle P \rangle} := \{ x \} \cup \{ P \} 
  \and \\
  \freenames{P|Q} := \freenames{P} \cup \freenames{Q}
  \and \\
  \freenames{@{x}} := \{ x \}
\end{mathpar}

$\pi$
$\quotep{\pi}$

$\freenames{-} : \pi \to \mathcal{P}(\quotep{\pi})$

\begin{eqnarray*}
  \freenames{\pzero} & := & \emptyset \\
  \freenames{x?(y).P} & := & \{ x \} \cup (\freenames{P} \setminus \{ y \}) \\
  \freenames{x!\langle P \rangle} & := & \{ x \} \cup \{ P \} \\
  \freenames{P|Q} & := & \freenames{P} \cup \freenames{Q} \\
  \freenames{\dropn{x}} & := & \{ x \}
\end{eqnarray*}

The bound names of a process, $\boundnames{P}$, are those names occurring in $P$
that are not free. For example, in $x?(y).0$, the name $x$ is free, while $y$ is bound.

\begin{mathpar}
  \inferrule* [lab=monoidal-laws] {} { P|Q \equiv Q|P \and P|0 \equiv P \and P|(Q|R) \equiv (P|Q)|R }
\end{mathpar}

\begin{mathpar}
  \inferrule* [lab=alpha-equivalence] {} { (x)P \equiv (y)P\{y/x\} \and y \not\in \freenames{P} }
\end{mathpar}

\begin{definition}
Then two processes, $P,Q$, are alpha-equivalent if $P = Q\{\vec{y}/\vec{x}\}$ for
some $\vec{x} \in \boundnames{Q},\vec{y} \in \boundnames{P}$, where $Q\{\vec{y}/\vec{x}\}$
denotes the capture-avoiding substitution of $\vec{y}$ for $\vec{x}$ in $Q$.
\end{definition}

\begin{definition}
  The {\em structural congruence} \cite{SangiorgiWalker} , $\equiv$,
  between processes is the least congruence containing
  alpha-equivalence, satisfying the abelian monoid laws
  (associativity, commutativity and $\pzero$ as identity) for parallel
  composition $|$ and for summation $+$.
\end{definition}

\subsection{Name equivalence}

We take name equivalence, written $\nameeq$, to be the smallest
equivalence relation generated by the following rules.

\begin{mathpar}
\inferrule*[lab=Quote-drop]
{ }
{ \quotep{@{x}} \nameeq x }

\inferrule*[lab=Struct-equiv]
{ P \scong Q }
{ \quotep{P} \nameeq \quotep{Q} }
\end{mathpar}

The astute reader will have noticed that the mutual recursion of names
and processes imposes a mutual recursion on alpha-equivalence and
structural equivalence via name-equivalence. Fortunately, all of this
works out pleasantly and we may calculate in the natural way, free of
concern. The reader interested in the details is referred to the
appendix \ref{appendix:rho_details}.

\subsection{Substitution}

We use $\Proc$ for the set of processes, $\QProc$ for the set of
names, and $\id{\{}\vec{y} / \vec{x} \id{\}}$ to denote partial maps,
$s : \QProc \rightarrow \QProc$. A map, $s$ lifts, uniquely, to a map
on process terms, $\widehat{s} : \Proc \rightarrow \Proc$ by the
following equations.

\begin{mathpar}
  (0) \psubstp{Q}{P} := 0 \\
  (R \juxtap S) \psubstp{Q}{P}
  :=    
  (R)\psubstp{Q}{P} \juxtap (S) \psubstp{Q}{P} \\
  (x?(y).R) \psubstp{Q}{P}    
  :=    
  (x)\substp{Q}{P} (z)\concat( (R \psubstn{z}{y}) \psubstp{Q}{P} ) \\
  (\lift{x}{R}) \psubstp{Q}{P}  
  :=
  \lift{(x)\substp{Q}{P}}{ R \psubstp{Q}{P} } \\
%   (\dropn{x})  \psubstp{Q}{P}       
%   := 
%   \left\{ 
%     \begin{array}{ccc} 
%       \dropn{\quotep{Q}} & & x \nameeq \quotep{P} \\
%       \dropn{x} & & otherwise \\
%     \end{array}
%   \right. 
  (\dropn{x})  \psubstp{Q}{P}       
  := 
  \left\{ 
    \begin{array}{ccc} 
      Q & & x \nameeq \quotep{P} \\
      \dropn{x} & & otherwise \\
    \end{array}
  \right.
\end{mathpar}
 

where

\begin{eqnarray}
  (x)\id{\{} \lpquote Q \rpquote / \lpquote P \rpquote \id{\}}            = 
  \left\{ 
    \begin{array}{ccc}
      \lpquote Q \rpquote & & x \nameeq \lpquote P \rpquote \\
      x & & otherwise \\
    \end{array}
  \right. \nonumber
\end{eqnarray}

and $z$ is chosen distinct from $\quotep{P}$, $\quotep{Q}$, the free
names in $Q$, and all the names in $R$. Our $\alpha$-equivalence will
be built in the standard way from this substitution.

\begin{remark}\label{rem:no_self_referential_names}
  One consequence of these definitions is that $\forall P. \quotep{P}
  \not\in \freenames{P}$.
\end{remark}

\subsection{ Dynamic quote: an example }

Anticipating something of what's to come, consider applying the
substitution, $\widehat{\id{\{}u / z \id{\}}}$, to the following pair
of processes, $\lift{w}{y!(z)}$ and $w[ \lpquote y!(z) \rpquote ]$.

\begin{eqnarray}
	\lift{w}{y!(z)}\widehat{\id{\{}u / z \id{\}}}
		& = &
		\lift{w}{y!(u)} \nonumber\\
	w[ \lpquote y!(z) \rpquote ] \widehat{ \id{\{}u / z \id{\}} }
		& = &
		w[ \lpquote y!(z) \rpquote ] \nonumber
\end{eqnarray}

Because the body of the process between quotes is impervious to
substitution, we get radically different answers. In fact, by
examining the first process in an input context,
e.g. $x?(z).\lift{w}{y!(z)}$, we see that the process under the lift
operator may be shaped by prefixed inputs binding a name inside it. In
this sense, the lift operator will be seen as a way to dynamically
construct processes before reifying them as names.

Finally equipped with these standard features we can present the
dynamics of the calculus.

\subsubsection{Operational semantics} 

Finally, we introduce the computational dynamics. What marks these
algebras as distinct from other more traditionally studied algebraic
structures, e.g. vector spaces or polynomial rings, is the manner in
which dynamics is captured. In traditional structures, dynamics is typically
expressed through morphisms between such structures, as in linear maps
between vector spaces or morphisms between rings. In algebras
associated with the semantics of computation, the dynamics is
expressed as part of the algebraic structure itself, through a
reduction reduction relation typically denoted by $\red$. Below, we
give a recursive presentation of this relation for the calculus used
in the encoding.

$\red \subseteq \pi \times \pi$
$\red : \pi \to \mathcal{P}(\pi)$

\begin{mathpar}
  \inferrule* [lab=Comm] { \textsf{match}( x_{src}, x_{trgt} ) } { x_{trgt}?(y)P \; | \; x_{src}!\langle {Q} \rangle \red P\{\quotep{Q}/y}\} }
  \and \\
  \inferrule* [lab=Par] {{P} \red {P}'} {{{P} | {Q}} \red {{P}' | {Q}}}
  \and
  \inferrule* [lab=Equiv]{{{P} \scong {P}'} \andalso {{P}' \red {Q}'} \andalso {{Q}' \scong {Q}}}{{P} \red {Q}}
\end{mathpar}

\begin{eqnarray*}
  match_{\equiv} (\quotep{P},\quotep{Q}) & := & P \equiv Q \\
  match_{\dagger}(\quotep{P},\quotep{Q}) & := & \forall R. P|Q \red^{*} R => R \red^{*} 0 \\
  match_{K}(\quotep{P},\quotep{Q}) & := & K \mbox{ for some context } K
\end{eqnarray*}

$u?(x)P | u!\langle Q \rangle \red P\{\quotep{Q}/x\}$

%We write $\wred$ for $\red^*$, and $P\red$ if $\exists Q $ such that $ P \red Q$.
We write $P\red$ if $\exists Q $ such that $ P \red Q$ and $P\not\red$, otherwise.

\section{Replication}

As mentioned before, it is known that replication (and hence
recursion) can be implemented in a higher-order process algebra
\cite{SangiorgiWalker}. As our first example of calculation with the
machinery thus far presented we give the construction explicitly in
the {\rhoc}.

\begin{eqnarray}
	D_{x} & := & \prefix{x}{y}{(\binpar{\outputp{x}{y}}{@{y}})} \nonumber\\
	\bangp_{x}{P} & := & \binpar{{x}!\langle{\binpar{D_{x}}{P}}\rangle}{D_{x}} \nonumber
\end{eqnarray}

\begin{eqnarray}
	\bangp_{x}{P} & & \nonumber\\
	=
	& {x}!\langle{(\prefix{x}{y}{(\outputp{x}{y} | @{y})) | P}}\rangle 
	      | \prefix{x}{y}{(\outputp{x}{y} | @{y})} & \nonumber\\
	\red
	& (\outputp{x}{y} | @{y})\substn{\quotep{(\prefix{x}{y}{(@{y} | \outputp{x}{y})) | P}}}{y} & \nonumber\\
	=
	& \outputp{x}{\quotep{(\prefix{x}{y}{(\outputp{x}{y} | @{y})) | P}}}
	  | {(\prefix{x}{y}{(\outputp{x}{y} | @{y})) | P}} & \nonumber\\
	\red
	& \ldots & \nonumber\\
	\red^*
	& P | P | \ldots & \nonumber
\end{eqnarray}

Of course, this encoding, as an implementation, runs away, unfolding
$\bangp{P}$ eagerly. A lazier and more implementable replication
operator, restricted to input-guarded processes, may be obtained as follows.

\begin{eqnarray}
\bangp{\prefix{u}{v}{P}} 
	:= 
	\binpar{\lift{x}{\prefix{u}{v}{(\binpar{D(x)}{P})}}}{D(x)} \nonumber
\end{eqnarray}

\begin{remark}
  Note that the lazier definition still does not deal with summation
  or mixed summation (i.e. sums over input and output). The reader is
  invited to construct definitions of replication that deal with these
  features. 

  Further, the definitions are parameterized in a name, $x$. Can you,
  gentle reader, make a definition that eliminates this parameter and
  guarantees no accidental interaction between the replication
  machinery and the process being replicated -- i.e. no accidental
  sharing of names used by the process to get its work done and the
  name(s) used by the replication to effect copying. This latter
  revision of the definition of replication is crucial to obtaining
  the expected identity $!!P \sim !P$.
\end{remark}

\begin{remark}\label{rem:paradoxical_combinator}
  The reader familiar with the lambda calculus will have noticed the
  similarity between $D$ and the paradoxical combinator.

  [Ed. note: the existence of this seems to suggest we have to be more
  restrictive on the set of processes and names we admit if we are to
  support no-cloning.]
\end{remark}

\subsubsection{Bisimulation}

The computational dynamics gives rise to another kind of equivalence,
the equivalence of computational behavior. As previously mentioned
this is typically captured \emph{via} some form of bisimulation.

% The notion we use in this paper is weak barbed bisimulation
% \cite{milner91polyadicpi}.

The notion we use in this paper is derived from weak barbed
bisimulation \cite{milner91polyadicpi}. 

\begin{definition}
An \emph{observation relation}, $\downarrow_{\mathcal N}$, over a set
of names, $\mathcal N$, is the smallest relation satisfying the rules
below.

\infrule[Out-barb]{y \in {\mathcal N}, \; x \nameeq y}
		  {\outputp{x}{v} \downarrow_{\mathcal N} x}
\infrule[Par-barb]{\mbox{$P\downarrow_{\mathcal N} x$ or $Q\downarrow_{\mathcal N} x$}}
		  {\binpar{P}{Q} \downarrow_{\mathcal N} x}

We write $P \Downarrow_{\mathcal N} x$ if there is $Q$ such that 
$P \wred Q$ and $Q \downarrow_{\mathcal N} x$.
\end{definition}

\begin{definition}
%\label{def.bbisim}
An  ${\mathcal N}$-\emph{barbed bisimulation} over a set of names, ${\mathcal N}$, is a symmetric binary relation 
${\mathcal S}_{\mathcal N}$ between agents such that $P\rel{S}_{\mathcal N}Q$ implies:
\begin{enumerate}
\item If $P \red P'$ then $Q \wred Q'$ and $P'\rel{S}_{\mathcal N} Q'$.
\item If $P\downarrow_{\mathcal N} x$, then $Q\Downarrow_{\mathcal N} x$.
\end{enumerate}
$P$ is ${\mathcal N}$-barbed bisimilar to $Q$, written
$P \wbbisim_{\mathcal N} Q$, if $P \rel{S}_{\mathcal N} Q$ for some ${\mathcal N}$-barbed bisimulation ${\mathcal S}_{\mathcal N}$.
\end{definition}

$\mathcal{R} \subseteq \pi \times \pi$

$P \mathcal{R} Q => \forall P'. P \red P' \Rightarrow \exists Q'. Q \red Q', P' \mathcal{R} Q'$

$P \vdash x \Rightarrow Q \vdash x$

\begin{mathpar}
  \inferrule*[lab=Out-barb]{x \nameeq y}{{y}!\langle{Q}\rangle \vdash x}
  \and
  \inferrule*[lab=Par-barb]{\mbox{$P\vdash x$ or $Q\vdash x$}}{\binpar{P}{Q} \vdash x}
\end{mathpar}

\subsubsection{Contexts}

One of the principle advantages of computational calculi like the
$\pi$-calculus is a well-defined notion of context,
contextual-equivalence and a correlation between
contextual-equivalence and notions of bisimulation. The notion of
context allows the decomposition of a process into (sub-)process and
its syntactic environment, its context. Thus, a context may be
thought of as a process with a ``hole'' (written $\Box$) in it. The
application of a context $M$ to a process $P$, written $M[P]$, is
tantamount to filling the hole in $M$ with $P$. In this paper we do
not need the full weight of this theory, but do make use of the notion
of context in the proof the main theorem. 

\begin{mathpar}
  \inferrule* [lab=summation] {} {{M_{M},M_{N}} \bc \Box \;|\; x.M_{A} \;|\; M_{M}+M_{N}}
  \and
  \inferrule* [lab=agent] {} {{M_{A}} \bc (\vec{x})M_{P} \;| \; \clift{P_0,\ldots,M_{P},\ldots,P_N}}
  \and \\
  \inferrule* [lab=process] {} {{M_{P}} \bc M_{N} \;| \;P|M_{P} }
\end{mathpar} 

\begin{mathpar}
  \inferrule* [lab=sychronization] {} {M_{N} \bc \Box \;|\; x?M_{F} \;|\; x!M_{C}}
  \and
  \inferrule* [lab=abstraction] {} {{M_{F}} \bc (x)M_{P} }
  \and
  \inferrule* [lab=concretion] {} {{M_{C}} \bc \langle M_{P} \rangle }
  \and \\
  \inferrule* [lab=process] {} {{M_{P}} \bc M_{N} \;| \;P|M_{P} }
\end{mathpar}

\begin{definition}[contextual application] Given a context $M$, and
  process $P$, we define the \emph{contextual application}, $M[P] :=
  M\{P/\Box\}$. That is, the contextual application of M to P is the
  substitution of $P$ for $\Box$ in $M$.
\end{definition}

$\meaningof{-} : L \to \mathcal{P}(\pi)$

\begin{mathpar}
  \inferrule* [lab=collection] {} {\meaningof{true} = \pi, \and \meaningof{~E} = \pi \setminus \meaningof{E}, \and \meaningof{E_{1} \& E_{2}} = \meaningof{E_{1}} \cap \meaningof{E_{2}}}
\end{mathpar}

\begin{mathpar}
  \inferrule* [lab=structure] {} {\meaningof{0} = \{ P \in \pi | P \equiv 0 \}, \and \\ \meaningof{E_1 | E_2} = \{ P \in \pi | P \equiv P_{1} | P_{2}, P_{1} \in \meaningof{E_{1}}, P_{2} \in \meaningof{E_2}\} }
\end{mathpar}

\begin{mathpar}
 \inferrule* [lab=behavior] {} {\meaningof{\langle a?b \rangle E} = \{ P \in \pi | P \equiv Q | u?(y)P', \\ \and \\\\ \and \\ \;\;\; u \in \meaningof{a}, \forall z.P'\{z/y\} \in \meaningof{E\{z/b\}}\}, \and \\ \meaningof{a!E} = \{ P \in \pi | P \equiv Q | x!\langle P' \rangle, x \in \meaningof{a} P' \in \meaningof{E}\} }
\end{mathpar}

\begin{mathpar}
 \inferrule* [lab=nominal] {} {\meaningof{\quotep{E}} = \{ \quotep{P} \in \quotep{\pi} | P \in \meaningof{E} \}, \and \meaningof{\quotep{P}} = \{ \quotep{Q} \in \quotep{\pi} | P \equiv Q \} \and \\ \meaningof{@\quotep{E}} = \{ P \in \pi | P \equiv @x, x \in \meaningof{E} \}}
\end{mathpar}

\begin{eqnarray*}
  \\
  \meaningof{-} : TS \to ST
\end{eqnarray*}

\begin{eqnarray*}
  \\
  L : TS \to ST
\end{eqnarray*}

\begin{eqnarray*}
  \\
  P \models E \iff P \in \meaningof{E}
\end{eqnarray*}

\begin{eqnarray*}
  P \approx_{L} Q \iff \forall E \in L. P \models E \iff Q \models E
\end{eqnarray*}

\begin{eqnarray*}
  P \approx_{K} Q
\end{eqnarray*}

\begin{eqnarray*}
  P \approx Q
\end{eqnarray*}

$\approx_{K} = \approx = \approx_{L}$

\subsubsection{Contextual duality}

Note that contexts extend the quotation operation to a family of
operations from processes to names. Given a context, $M$, we can
define a \emph{nominal context}, $\quotep{M}$ by $\quotep{M}[P] :=
\quotep{M[P]}$. To foreshadow what is to come we observe that these
operations enjoy a duality with processes very much like the duality
between vectors and maps from vectors to scalars.

Further, because the calculus is essentially higher-order, we have a
correspondence between contexts and processes. More specifically,
given a name $x$ and a context $M$ we can construct $M^{*}_{x}$ such
that 

\begin{mathpar}
  M^{*}_{x} | \lift{x}{P} \red M[P]
\end{mathpar}

namely,

\begin{mathpar}
  M^{*}_{x} := x?(u).M[\dropn{u}]
\end{mathpar}

The dependence of $M^{*}_{x}$ on a name makes it an abstraction, 

\begin{mathpar}
  M^{*} := (x)x?(u).M[\dropn{u}]
\end{mathpar}

\subsection{Additional notation}

It will sometimes be convenient to denote the process a name
quotes. We already have the notation $x = \quotep{P}$, but it will be
convenient to introduce an alternate notation, $\procn{x}$, when we
want to emphasize the connection to the use of the name. Note that, by
virtue of name equivalence, $\quotep{\procn{x}} \nameeq x$; so, the
notation is consistent with previous definitions.

Further, because names have structure it is possible to effect
substitutions on the basis of that structure. This means we need to
upgrade our notation for substitutions, which we accomplish by
adapting comprehension notation. Thus,

\begin{mathpar}
  P\{ y / x : x \in S \}
\end{mathpar}

is interpreted to mean the process derived from P by replacing (in a
capture-avoiding manner) each occurrence of $x$ in $S$ by $y$. For example,

\begin{mathpar}
  P\{ \quotep{\procn{x}|\procn{x}} / x : x \in \freenames{P} \}
\end{mathpar}

will replace each (occurrence) of a free name $x$ in $P$ by
$\quotep{\procn{x}|\procn{x}}$.

Also, we will avail ourselves of the notation $x^{L}$ and $x^{R}$ to
denote injections of a name into disjoint copies of the name
space. There are numerous ways to accomplish this. One example can be
found in \cite{MeredithR05}. This notation overloads to vectors of
names: $\vec{x}^{\pi} := (x_{i}^{\pi} \; : \; 0 \leq i < |\vec{x}| )$ where $\pi \in \{L,R\}$.

We also use $P^{\Box} := P|\Box$.

In \cite{MeredithR05} an interpretation of the new operator is
given. It turns out that there are several possible interpretations
all enjoying the requisite algebraic properties of the operator (see
\cite{milner91polyadicpi}). We will therefore make liberal use of
$(\nu\; \vec{x})P$.

% subsection the_syntax_and_semantics_of_the_notation_system (end)   

\section{Interpretation of QM}
\subsection{Supporting definitions}
\subsubsection{Multiplication}
\begin{mathpar}
  \quotep{Q} \cdot \quotep{R} := \quotep{Q|R}
  \and \\
  \quotep{Q} \cdot P := P\{ \quotep{Q|R} / \quotep{R} : \quotep{R} \in \freenames{P} \}
\end{mathpar}

\paragraph{Discussion}
The first line needs little explanation. The second line says that
each free name of the process is replaced with the multiplication of
that name by the scalar. Multiplication of a scalar (name) by a state
(process) results in a process all the names of which have been `moved
over' by parallel composition with the process the scalar
quotes. There is a subtlety that the bound names have to be
manipulated so that multiplied names aren't accidentally
captured. There are many ways to achieve this.

\begin{remark}\label{rem:multiplication_identities}
  The reader is invited to verify that for all $x,y,z \in \QProc$ and $P \in \Proc$
  \begin{mathpar}
    x \cdot \quotep{0} \equiv x 
    \and
    x \cdot y \equiv y \cdot x
    \and
    x \cdot (y \cdot z) \equiv (x \cdot y) \cdot z
    \and \\
    \quotep{0} \cdot P \equiv P
    \and \\
    x \cdot (y \cdot P) \equiv (x \cdot y) \cdot P
    \and \\
    x \cdot (P|Q) \equiv (x \cdot P) | (x \cdot Q)
    \and \\    
  \end{mathpar}
\end{remark}

\subsubsection{Tensor product}

We define a tensor product on processes by structural induction.

\paragraph{Tensor of sums} First note that all summations, including
$\pzero$ and sequence, can be written $\Sigma_{i} x_{i}.A_{i} +
\Sigma_{j} x_{j}.C_{j}$, where we have grouped input-guarded processes
together and output-guarded processes together.

Thus, we can define the tensor product of two summations, $N_{1}\otimes N_{2}$, where

\begin{mathpar}
  N_{1} := \Sigma_{i} x_{i}.A_{i} + \Sigma_{j} x_{j}.C_{j}
  \and
  N_{2} := \Sigma_{i'} y_{i'}.B_{i'} + \Sigma_{j'} y_{j'}.D_{j'} 
\end{mathpar}

as follows.

\begin{mathpar}
  \Sigma_{i} x_{i}.A_{i} + \Sigma_{j} x_{j}.C_{j} \otimes \Sigma_{i'}
  y_{i'}.B_{i'} + \Sigma_{j'} y_{j'}.D_{j'} 
  \and \\
  := \; \Sigma_{i} \Sigma_{i'} \quotep{\stackrel{\vee}{x_{i}}| \stackrel{\vee}{y_{i'}}}.(A_{i}\otimes B_{i'}) \; | \; \Sigma_{i'} \Sigma_{i} \quotep{\stackrel{\vee}{y_{i'}}|\stackrel{\vee}{x_{i}}}.(B_{i'}\otimes A_{i})
  \and
  \;\; | \;\; \Sigma_{j} \Sigma_{j'} \quotep{\stackrel{\vee}{x_{j}}|\stackrel{\vee}{y_{j'}}}.(A_{j}\otimes B_{j'}) \; | \; \Sigma_{j'} \Sigma_{j} \quotep{\stackrel{\vee}{y_{j'}}|\stackrel{\vee}{x_{j}}}.(B_{j'}\otimes A_{j})
\end{mathpar}

\begin{remark}
  Do we need to $x^{L}$ and $y^{R}$ for this construction as well?
\end{remark}

\paragraph{Tensor of parallel compositions} Next, we distribute tensor
over par.

\begin{mathpar}
  P_{1}|P_{2} \otimes Q_{1}|Q_{2} := (P_{1} \otimes Q_{1}) | (P_{1}
  \otimes Q_{2}) | (P_{2} \otimes Q_{1}) | (P_{2} \otimes Q_{2})
\end{mathpar}

\paragraph{Tensor with dropped names} We treat tensor of a
process with a dropped name as parallel composition.

\begin{mathpar}
  P \otimes \dropn{x} := P | \dropn{x}
\end{mathpar}

\paragraph{Tensor of agents}

Finally, we need to define tensor on agents. Note that the definition
of tensor on normal products only tensors inputs with inputs and
outputs with outputs. Thus, we only have to define the operation on
``homogeneous'' pairings.

\begin{mathpar}
  (\vec{x})P \otimes (\vec{y})Q
  \and \\
  := (x_{0}^{L}|y_{0}^{R},\ldots,x_{0}^{L}|y_{n}^{R},\ldots,x_{m}^{L}|y_{0}^{R},\ldots,x_{m}^{L}|y_{n}^R)(P\{ \vec{x}^{L}/\vec{x}\} \otimes Q \{ \vec{y}^{R}/\vec{y}\})
  \and \\
  \clift{\vec{P}} \otimes \clift{\vec{Q}}
  \and \\
  := \clift{P_{0}\otimes Q_{0},\ldots,P_{0}\otimes Q_{n},\ldots,P_{m}\otimes Q_{0},\ldots,P_{m}\otimes Q_{n}}
\end{mathpar}

\begin{remark}
  Observe that arities of tensored abstractions matches arities of
  tensored concretions if the original arities matched. Note also that
  the length of the arities corresponds to the increase in dimension
  we see in ordinary vector space tensor product.
\end{remark}

\begin{remark}
  Operationally, this definition distributes the tensor down to
  components ``linked'' by summation. Tensor over summation is
  intriguing in that it mixes names. Moreover, as a consequence of the
  way it mixes names we have the identities for all $x \in \QProc$ and
  $P,Q \in \Proc$

  \begin{mathpar}
    (x \cdot P) \otimes Q \equiv x \cdot (P \otimes Q) \equiv P \otimes (x \cdot Q)
    \and
    P \otimes \pzero \equiv P
  \end{mathpar}

  that the reader is invited to verify.
\end{remark}

\subsubsection{Annihilation}
\begin{mathpar}
  P^{\perp} := \{ Q | \forall R. P|Q \red^{*} R \Rightarrow R \red^{*} \pzero \}
  \and \\
  P^{\underline{\perp}} := \Sigma_{Q \in P^{\perp}} \quotep{Q}?(y).(\dropn{y}|Q) | \Sigma_{Q \in P^{\perp}} \quotep{Q}\clift{\Box}
\end{mathpar}

\paragraph{Discussion} The reader will note that $P^{\perp}$ is a
\emph{set} of processes, while $P^{\underline{\perp}}$ is a
\emph{context}. We call the set $P^{\perp}$ the \emph{annihilators} of
$P$. The parallel composition of a process in the annihilators of $P$
with $P$ will result in a process, the state space of which has all
paths eventually leading to $\pzero$. Execution may endure loops; but
under reasonable conditions of fairness (naturally guaranteed under
most notions of bisimulation) such a composite process cannot get
stuck in such a loop and will, eventually pop out and terminate.

The context $P^{\underline{\perp}}$ is ready and willing to ``take the
$P$ out of'' the process to which it is applied. It will effectively
transmit the code of the process to which it is applied to one of the
annihilators and run the process against it.

\subsubsection{Evaluation}
We fix $M$ a domain of fully abstract interpretation with an equality
coincident with bisimulation. We take $\meaningof{\cdot} : \Proc \to
M$ to be the map interpreting processes and $\nmeaningof{\cdot} : \M
\to Proc$ to be the map running the other way. Then we define

\begin{mathpar}
  \int P := \nmeaningof{\meaningof{P}}
\end{mathpar}

\paragraph{Discussion}
There are many fully abstract interpretations of Milner's
$\pi$-calculus. Any of them can be used as a basis for interpreting
the reflective calculus here. Equipped with such a domain it is
largely a matter of grinding through to check that the Yoneda
construction for the normalization-by-evaluation program can be
extended to this setting.

\begin{remark}
  The reader is invited to verify that $\int (P^{\underline{\perp}}[P]) = 0$.
\end{remark}

\subsection{Quantum mechanics}

Table \ref{tbl:core_qm_op_defns} gives the core operational definitions

\begin{table}[htp]\label{tbl:core_qm_op_defns}
  \center{
    \fbox{
      \begin{tabular}{c|c}
        quantum mechanics & process calculus \\
        \hline
        scalar & $x := \quotep{P}$ \\
        state vector & $\state{P} := P$ \\
        dual & $\state{P}^{*} := \event{P^{\underline{\perp}}} := \quotep{P^{\underline{\perp}}}[-]$ \\
        matrix & $ \Sigma_{\alpha} \state{P_{\alpha}}x_{\alpha}\event{Q_{\alpha}}$ \\
        vector addition & $\state{P} + \state{Q} := \state{P | Q}$ \\
        tensor product & $\state{P} \otimes \state{Q} := \state{P \otimes Q}$ \\
        inner product & $\innerprod{P}{Q} := \quotep{\int P^{\underline{\perp}}[Q]}$ \\
      \end{tabular}
    }
  }
  \caption{QM - operational definitions}
\end{table}

where

\begin{mathpar}
  \prmatrix{P}{Q} := \fprmatrix{P}{\quotep{\pzero}}{Q}
  \and
  \fprmatrix{P}{x}{Q} := (\state{P},x,\event{Q})
  \and
  (\fprmatrix{P}{x}{Q})(\state{R}) := x \cdot \innerprod{Q}{R} \cdot \state{P}
  \and
  (\fprmatrix{P}{x}{Q})(\event{R}) := x \cdot \innerprod{R}{P} \cdot \event{Q}
\end{mathpar}

\paragraph{Discussion}
As promised: vectors (aka states) are represented as processes; duals
as contextual duals; inner product definition should be compared with
standard inner product definition for ....

\begin{remark}
  Assuming $\int (P^{\underline{\perp}}[P]) = 0$, the reader is
  invited to verify that $(\fprmatrix{P}{x}{P})(\state{P}) = x \cdot \state{P}$.
\end{remark}

\begin{remark}
  The reader is invited to verify that $\innerprod{P}{Q}$ could
  equally well have been written $\quotep{\int \stackrel{\vee}{x}}$
  where $x = \event{P^{\underline{\perp}}}(Q)$.

  One of the motivations for this remark is that there is another way
  to factor these operations. We could package up evaluation in the dual:

  \begin{mathpar}
    \state{P}^{*} := \event{\int P^{\underline{\perp}}} := \quotep{\int P^{\underline{\perp}}}[-]
  \end{mathpar}

  and then have inner product defined by
  
  \begin{mathpar}
    \innerprod{P}{Q} := \event{P}(Q)
  \end{mathpar}

  Hopefully, experience with the calculations will provide guidance on
  the best factoring.
\end{remark}

\begin{remark}
  Assuming $\int (P^{\underline{\perp}}[P]) = 0$, the reader is
  invited to verify that $\forall P,Q. (\prmatrix{0}{Q})(\state{0}) =
  \state{0}$ and dually $(\prmatrix{P}{0})(\event{0}) = \event{0}$.
\end{remark}

\begin{remark}
  i'm a little worried that i don't (yet) have proper support for
  complex conjugacy. But, the observation above may give us a
  clue. According to Abramsky, it must be the case that the scalars
  are iso to the homset of the identity for the tensor -- which the
  observation above characterizes. 

  For now, we will simply bookmark the notion with $\overline{x}$.
\end{remark}

\subsubsection{Adjointness}

We need to give a definition of $(\cdot)^{\dagger}$ for matrices. The
obvious candidate definition is
\begin{mathpar}
(\Sigma_{\alpha}\fprmatrix{P_{\alpha}}{x_{\alpha}}{Q_{\alpha}})^{\dagger}
= \Sigma_{\alpha}\fprmatrix{(Q_{\alpha}^{\underline{\perp}})^{*}}{\overline{x}_{\alpha}}{P_{\alpha}^{\underline{\perp}}} 
\end{mathpar}

But, $(Q_{\alpha}^{\underline{\perp}})^{*}$ requires a name along
which to communicate the process to achieve the context application.

\subsubsection{Basis for a basis}
If processes label states and ``addition'' of states (a.k.a. vector
addition) is interpreted as parallel composition, what corresponds to
notions of linear independence and basis? Here, we recall that Yoshida
has developed a set of \emph{combinators} for an asynchronous verison
of Milner's $\pi$-calculus. These are a finite set of processes such
any process can be expressed as parallel composition of these
combinators together with liberal uses of the new operator and
replication. We can simply give a translation of these into the
present calculus and have reasonable expectation that the property
carries over. That is, that the resultant set allows to express all
processes via parallel composition. Note, however, that there is no
new operator or replication in this calculus. As a result, we expect
that the corresponding set is actually infinite. That is, we expect
that the space is actually infinite dimensional.

\begin{remark}
  The attentive reader may be a bit concerned. Certainly, the
  collection $S$, $K$ and $I$ is a finite set of
  combinators. Shouldn't we expect to see a finite set of combinators
  for an effectively equivalent system? i am very sympathetic to this
  critique and feel it warrants full attention. On the other hand, i
  also have in mind the following analogy. The natural numbers, as a
  monoid under addition, has exactly $1$ generator, while the natural
  numbers, as a monoid under multiplication, has countably many
  generators (the primes). We observe that the application of the
  lambda calculus is much less resource sensitive than the parallel
  composition of the $\pi$-calculus. Could it be the case that we have
  an analogy of the form
  
  \begin{mathpar}
    m + n : MN :: m*n : M|N
  \end{mathpar}

  giving a similar blow up in the set of ``primes''?  This is such a
  wonderful thought that, even if it's not true, i think it's worth
  writing down.
\end{remark}
 

\documentclass[12pt]{llncs}
%\documentclass{jktr}

\usepackage[pdftex]{hyperref}                   
\usepackage {listings}
\usepackage {mathpartir}
\usepackage{bcprules}
%\usepackage{listings}
                       
\usepackage{graphicx} 
%\usepackage[margins=2.5cm,nohead,nofoot]{geometry}
%\usepackage{geometry}
\usepackage{amsfonts}
\usepackage{amstext}
\usepackage{latexsym}
\usepackage{amssymb}
\usepackage{color}


%\include{myPreamble}
\documentclass[12pt]{llncs}
%\documentclass{jktr}

\usepackage[pdftex]{hyperref}                   
\usepackage {listings}
\usepackage {mathpartir}
\usepackage{bcprules}
%\usepackage{listings}
                       
\usepackage{graphicx} 
%\usepackage[margins=2.5cm,nohead,nofoot]{geometry}
%\usepackage{geometry}
\usepackage{amsfonts}
\usepackage{amstext}
\usepackage{latexsym}
\usepackage{amssymb}
\usepackage{color}


%\include{myPreamble}
\include{qm2pi.local} 

%\ifpdf
%\usepackage[pdftex]{graphicx}
%\else
%\usepackage{graphicx}
%\fi

 % \ifpdf
%  \usepackage{pdfsync}
%  \if


%\title{Brief Article}
%\author{David F. Snyder}
%\author{L.G. Meredith}

%\address{Dept. of Math., Texas State University--San Marcos, San Marcos, TX 78666}
       
\pagestyle{empty}


\begin{document}

\lstset{language=[Objective]Caml,frame=shadowbox}

\input{qm2pi.front}

% section front matter (end)

\input{qm2pi.intro} 
 
% section introduction (end)

% \input{qm2pi.knotations} 

% section notation (end)

\input{qm2pi.process.calculi} 

% section concurrent_process_calculi_and_spatial_logics_ (end)
    
%\input{qm2pi.knots2pi} 

%\input{qm2pi.trefoil} 

%\input{qm2pi.mainthm} 

% subsection basic_interpretation (end)

%\input{qm2pi.rho.presentation} 
\subsection{The syntax and semantics of the notation system}\label{sub:the_syntax_and_semantics_of_the_notation_system} % (fold)

We now summarize a technical presentation of the calculus that
embodies our theory of dynamics. The typical presentation of such a
calculus follows the style of giving generators and relations on
them. The grammar, below, describing term constructors, freely
generates the set of processes, $\Proc$. This set is then quotiented
by a relation known as structural congruence and it is over this set
that the notion of dynamics is expressed. This presentation is
essentially that of \cite{MeredithR05} with the addition of
polyadicity and summation. For readability we have relegated some of
the technical subtleties to an appendix.

\subsubsection{Process grammar}\label{subsub:process_grammar}

\begin{mathpar}
  \inferrule* [lab=synchronization] {} {{M} \bc \pzero \;|\; x?F \;|\; x!C }
  \and
  \inferrule* [lab=abstraction] {} {{F} \bc (x)P}
  \and
  \inferrule* [lab=concretion] {} {{C} \bc \langle Q \rangle}
  \and
  \inferrule* [lab=process] {} {{P,Q} \bc M \;| \;P|Q \;|\; @{x}}
  \and
  \inferrule* [lab=name] {} {{x} \bc \quotep{P}}
\end{mathpar} 

Note that $\vec{x}$ (resp. $\vec{P}$) denotes a vector of names
(resp. processes) of length $|\vec{x}|$ (resp. $|\vec{P}|$). We adopt
the following useful abbreviations.

\begin{mathpar}
   x?(\vec{y}).P := x.(\vec{y})P \and  x\clift{\vec{P}} := x.\clift{\vec{P}}
   \and x!(y) := \lift{x}{\dropn{y}}
   \and \Pi_{i=0}^{n-1}P_i := P_0 | \ldots | P_{n-1}
\end{mathpar}

\subsubsection{Structural congruence}

\paragraph{Free and bound names and alpha-equivalence.} At the
core of structural equivalence is alpha-equivalence which identifies
process that are the same up to a change of variable. Formally, we
recognize the distinction between free and bound names. The free names
of a process, $\freenames{P}$, may be calculated recursively as
follows:

\begin{mathpar}
\freenames{\pzero} := \emptyset
  \and \\
  \freenames{x?(y).P} := \{ x \} \cup (\freenames{P} \setminus \{ y \})
  \and 
  \freenames{x!\langle P \rangle} := \{ x \} \cup \{ P \} 
  \and \\
  \freenames{P|Q} := \freenames{P} \cup \freenames{Q}
  \and \\
  \freenames{@{x}} := \{ x \}
\end{mathpar}

$\pi$
$\quotep{\pi}$

$\freenames{-} : \pi \to \mathcal{P}(\quotep{\pi})$

\begin{eqnarray*}
  \freenames{\pzero} & := & \emptyset \\
  \freenames{x?(y).P} & := & \{ x \} \cup (\freenames{P} \setminus \{ y \}) \\
  \freenames{x!\langle P \rangle} & := & \{ x \} \cup \{ P \} \\
  \freenames{P|Q} & := & \freenames{P} \cup \freenames{Q} \\
  \freenames{\dropn{x}} & := & \{ x \}
\end{eqnarray*}

The bound names of a process, $\boundnames{P}$, are those names occurring in $P$
that are not free. For example, in $x?(y).0$, the name $x$ is free, while $y$ is bound.

\begin{mathpar}
  \inferrule* [lab=monoidal-laws] {} { P|Q \equiv Q|P \and P|0 \equiv P \and P|(Q|R) \equiv (P|Q)|R }
\end{mathpar}

\begin{mathpar}
  \inferrule* [lab=alpha-equivalence] {} { (x)P \equiv (y)P\{y/x\} \and y \not\in \freenames{P} }
\end{mathpar}

\begin{definition}
Then two processes, $P,Q$, are alpha-equivalent if $P = Q\{\vec{y}/\vec{x}\}$ for
some $\vec{x} \in \boundnames{Q},\vec{y} \in \boundnames{P}$, where $Q\{\vec{y}/\vec{x}\}$
denotes the capture-avoiding substitution of $\vec{y}$ for $\vec{x}$ in $Q$.
\end{definition}

\begin{definition}
  The {\em structural congruence} \cite{SangiorgiWalker} , $\equiv$,
  between processes is the least congruence containing
  alpha-equivalence, satisfying the abelian monoid laws
  (associativity, commutativity and $\pzero$ as identity) for parallel
  composition $|$ and for summation $+$.
\end{definition}

\subsection{Name equivalence}

We take name equivalence, written $\nameeq$, to be the smallest
equivalence relation generated by the following rules.

\begin{mathpar}
\inferrule*[lab=Quote-drop]
{ }
{ \quotep{@{x}} \nameeq x }

\inferrule*[lab=Struct-equiv]
{ P \scong Q }
{ \quotep{P} \nameeq \quotep{Q} }
\end{mathpar}

The astute reader will have noticed that the mutual recursion of names
and processes imposes a mutual recursion on alpha-equivalence and
structural equivalence via name-equivalence. Fortunately, all of this
works out pleasantly and we may calculate in the natural way, free of
concern. The reader interested in the details is referred to the
appendix \ref{appendix:rho_details}.

\subsection{Substitution}

We use $\Proc$ for the set of processes, $\QProc$ for the set of
names, and $\id{\{}\vec{y} / \vec{x} \id{\}}$ to denote partial maps,
$s : \QProc \rightarrow \QProc$. A map, $s$ lifts, uniquely, to a map
on process terms, $\widehat{s} : \Proc \rightarrow \Proc$ by the
following equations.

\begin{mathpar}
  (0) \psubstp{Q}{P} := 0 \\
  (R \juxtap S) \psubstp{Q}{P}
  :=    
  (R)\psubstp{Q}{P} \juxtap (S) \psubstp{Q}{P} \\
  (x?(y).R) \psubstp{Q}{P}    
  :=    
  (x)\substp{Q}{P} (z)\concat( (R \psubstn{z}{y}) \psubstp{Q}{P} ) \\
  (\lift{x}{R}) \psubstp{Q}{P}  
  :=
  \lift{(x)\substp{Q}{P}}{ R \psubstp{Q}{P} } \\
%   (\dropn{x})  \psubstp{Q}{P}       
%   := 
%   \left\{ 
%     \begin{array}{ccc} 
%       \dropn{\quotep{Q}} & & x \nameeq \quotep{P} \\
%       \dropn{x} & & otherwise \\
%     \end{array}
%   \right. 
  (\dropn{x})  \psubstp{Q}{P}       
  := 
  \left\{ 
    \begin{array}{ccc} 
      Q & & x \nameeq \quotep{P} \\
      \dropn{x} & & otherwise \\
    \end{array}
  \right.
\end{mathpar}
 

where

\begin{eqnarray}
  (x)\id{\{} \lpquote Q \rpquote / \lpquote P \rpquote \id{\}}            = 
  \left\{ 
    \begin{array}{ccc}
      \lpquote Q \rpquote & & x \nameeq \lpquote P \rpquote \\
      x & & otherwise \\
    \end{array}
  \right. \nonumber
\end{eqnarray}

and $z$ is chosen distinct from $\quotep{P}$, $\quotep{Q}$, the free
names in $Q$, and all the names in $R$. Our $\alpha$-equivalence will
be built in the standard way from this substitution.

\begin{remark}\label{rem:no_self_referential_names}
  One consequence of these definitions is that $\forall P. \quotep{P}
  \not\in \freenames{P}$.
\end{remark}

\subsection{ Dynamic quote: an example }

Anticipating something of what's to come, consider applying the
substitution, $\widehat{\id{\{}u / z \id{\}}}$, to the following pair
of processes, $\lift{w}{y!(z)}$ and $w[ \lpquote y!(z) \rpquote ]$.

\begin{eqnarray}
	\lift{w}{y!(z)}\widehat{\id{\{}u / z \id{\}}}
		& = &
		\lift{w}{y!(u)} \nonumber\\
	w[ \lpquote y!(z) \rpquote ] \widehat{ \id{\{}u / z \id{\}} }
		& = &
		w[ \lpquote y!(z) \rpquote ] \nonumber
\end{eqnarray}

Because the body of the process between quotes is impervious to
substitution, we get radically different answers. In fact, by
examining the first process in an input context,
e.g. $x?(z).\lift{w}{y!(z)}$, we see that the process under the lift
operator may be shaped by prefixed inputs binding a name inside it. In
this sense, the lift operator will be seen as a way to dynamically
construct processes before reifying them as names.

Finally equipped with these standard features we can present the
dynamics of the calculus.

\subsubsection{Operational semantics} 

Finally, we introduce the computational dynamics. What marks these
algebras as distinct from other more traditionally studied algebraic
structures, e.g. vector spaces or polynomial rings, is the manner in
which dynamics is captured. In traditional structures, dynamics is typically
expressed through morphisms between such structures, as in linear maps
between vector spaces or morphisms between rings. In algebras
associated with the semantics of computation, the dynamics is
expressed as part of the algebraic structure itself, through a
reduction reduction relation typically denoted by $\red$. Below, we
give a recursive presentation of this relation for the calculus used
in the encoding.

$\red \subseteq \pi \times \pi$
$\red : \pi \to \mathcal{P}(\pi)$

\begin{mathpar}
  \inferrule* [lab=Comm] { \textsf{match}( x_{src}, x_{trgt} ) } { x_{trgt}?(y)P \; | \; x_{src}!\langle {Q} \rangle \red P\{\quotep{Q}/y}\} }
  \and \\
  \inferrule* [lab=Par] {{P} \red {P}'} {{{P} | {Q}} \red {{P}' | {Q}}}
  \and
  \inferrule* [lab=Equiv]{{{P} \scong {P}'} \andalso {{P}' \red {Q}'} \andalso {{Q}' \scong {Q}}}{{P} \red {Q}}
\end{mathpar}

\begin{eqnarray*}
  match_{\equiv} (\quotep{P},\quotep{Q}) & := & P \equiv Q \\
  match_{\dagger}(\quotep{P},\quotep{Q}) & := & \forall R. P|Q \red^{*} R => R \red^{*} 0 \\
  match_{K}(\quotep{P},\quotep{Q}) & := & K \mbox{ for some context } K
\end{eqnarray*}

$u?(x)P | u!\langle Q \rangle \red P\{\quotep{Q}/x\}$

%We write $\wred$ for $\red^*$, and $P\red$ if $\exists Q $ such that $ P \red Q$.
We write $P\red$ if $\exists Q $ such that $ P \red Q$ and $P\not\red$, otherwise.

\section{Replication}

As mentioned before, it is known that replication (and hence
recursion) can be implemented in a higher-order process algebra
\cite{SangiorgiWalker}. As our first example of calculation with the
machinery thus far presented we give the construction explicitly in
the {\rhoc}.

\begin{eqnarray}
	D_{x} & := & \prefix{x}{y}{(\binpar{\outputp{x}{y}}{@{y}})} \nonumber\\
	\bangp_{x}{P} & := & \binpar{{x}!\langle{\binpar{D_{x}}{P}}\rangle}{D_{x}} \nonumber
\end{eqnarray}

\begin{eqnarray}
	\bangp_{x}{P} & & \nonumber\\
	=
	& {x}!\langle{(\prefix{x}{y}{(\outputp{x}{y} | @{y})) | P}}\rangle 
	      | \prefix{x}{y}{(\outputp{x}{y} | @{y})} & \nonumber\\
	\red
	& (\outputp{x}{y} | @{y})\substn{\quotep{(\prefix{x}{y}{(@{y} | \outputp{x}{y})) | P}}}{y} & \nonumber\\
	=
	& \outputp{x}{\quotep{(\prefix{x}{y}{(\outputp{x}{y} | @{y})) | P}}}
	  | {(\prefix{x}{y}{(\outputp{x}{y} | @{y})) | P}} & \nonumber\\
	\red
	& \ldots & \nonumber\\
	\red^*
	& P | P | \ldots & \nonumber
\end{eqnarray}

Of course, this encoding, as an implementation, runs away, unfolding
$\bangp{P}$ eagerly. A lazier and more implementable replication
operator, restricted to input-guarded processes, may be obtained as follows.

\begin{eqnarray}
\bangp{\prefix{u}{v}{P}} 
	:= 
	\binpar{\lift{x}{\prefix{u}{v}{(\binpar{D(x)}{P})}}}{D(x)} \nonumber
\end{eqnarray}

\begin{remark}
  Note that the lazier definition still does not deal with summation
  or mixed summation (i.e. sums over input and output). The reader is
  invited to construct definitions of replication that deal with these
  features. 

  Further, the definitions are parameterized in a name, $x$. Can you,
  gentle reader, make a definition that eliminates this parameter and
  guarantees no accidental interaction between the replication
  machinery and the process being replicated -- i.e. no accidental
  sharing of names used by the process to get its work done and the
  name(s) used by the replication to effect copying. This latter
  revision of the definition of replication is crucial to obtaining
  the expected identity $!!P \sim !P$.
\end{remark}

\begin{remark}\label{rem:paradoxical_combinator}
  The reader familiar with the lambda calculus will have noticed the
  similarity between $D$ and the paradoxical combinator.

  [Ed. note: the existence of this seems to suggest we have to be more
  restrictive on the set of processes and names we admit if we are to
  support no-cloning.]
\end{remark}

\subsubsection{Bisimulation}

The computational dynamics gives rise to another kind of equivalence,
the equivalence of computational behavior. As previously mentioned
this is typically captured \emph{via} some form of bisimulation.

% The notion we use in this paper is weak barbed bisimulation
% \cite{milner91polyadicpi}.

The notion we use in this paper is derived from weak barbed
bisimulation \cite{milner91polyadicpi}. 

\begin{definition}
An \emph{observation relation}, $\downarrow_{\mathcal N}$, over a set
of names, $\mathcal N$, is the smallest relation satisfying the rules
below.

\infrule[Out-barb]{y \in {\mathcal N}, \; x \nameeq y}
		  {\outputp{x}{v} \downarrow_{\mathcal N} x}
\infrule[Par-barb]{\mbox{$P\downarrow_{\mathcal N} x$ or $Q\downarrow_{\mathcal N} x$}}
		  {\binpar{P}{Q} \downarrow_{\mathcal N} x}

We write $P \Downarrow_{\mathcal N} x$ if there is $Q$ such that 
$P \wred Q$ and $Q \downarrow_{\mathcal N} x$.
\end{definition}

\begin{definition}
%\label{def.bbisim}
An  ${\mathcal N}$-\emph{barbed bisimulation} over a set of names, ${\mathcal N}$, is a symmetric binary relation 
${\mathcal S}_{\mathcal N}$ between agents such that $P\rel{S}_{\mathcal N}Q$ implies:
\begin{enumerate}
\item If $P \red P'$ then $Q \wred Q'$ and $P'\rel{S}_{\mathcal N} Q'$.
\item If $P\downarrow_{\mathcal N} x$, then $Q\Downarrow_{\mathcal N} x$.
\end{enumerate}
$P$ is ${\mathcal N}$-barbed bisimilar to $Q$, written
$P \wbbisim_{\mathcal N} Q$, if $P \rel{S}_{\mathcal N} Q$ for some ${\mathcal N}$-barbed bisimulation ${\mathcal S}_{\mathcal N}$.
\end{definition}

$\mathcal{R} \subseteq \pi \times \pi$

$P \mathcal{R} Q => \forall P'. P \red P' \Rightarrow \exists Q'. Q \red Q', P' \mathcal{R} Q'$

$P \vdash x \Rightarrow Q \vdash x$

\begin{mathpar}
  \inferrule*[lab=Out-barb]{x \nameeq y}{{y}!\langle{Q}\rangle \vdash x}
  \and
  \inferrule*[lab=Par-barb]{\mbox{$P\vdash x$ or $Q\vdash x$}}{\binpar{P}{Q} \vdash x}
\end{mathpar}

\subsubsection{Contexts}

One of the principle advantages of computational calculi like the
$\pi$-calculus is a well-defined notion of context,
contextual-equivalence and a correlation between
contextual-equivalence and notions of bisimulation. The notion of
context allows the decomposition of a process into (sub-)process and
its syntactic environment, its context. Thus, a context may be
thought of as a process with a ``hole'' (written $\Box$) in it. The
application of a context $M$ to a process $P$, written $M[P]$, is
tantamount to filling the hole in $M$ with $P$. In this paper we do
not need the full weight of this theory, but do make use of the notion
of context in the proof the main theorem. 

\begin{mathpar}
  \inferrule* [lab=summation] {} {{M_{M},M_{N}} \bc \Box \;|\; x.M_{A} \;|\; M_{M}+M_{N}}
  \and
  \inferrule* [lab=agent] {} {{M_{A}} \bc (\vec{x})M_{P} \;| \; \clift{P_0,\ldots,M_{P},\ldots,P_N}}
  \and \\
  \inferrule* [lab=process] {} {{M_{P}} \bc M_{N} \;| \;P|M_{P} }
\end{mathpar} 

\begin{mathpar}
  \inferrule* [lab=sychronization] {} {M_{N} \bc \Box \;|\; x?M_{F} \;|\; x!M_{C}}
  \and
  \inferrule* [lab=abstraction] {} {{M_{F}} \bc (x)M_{P} }
  \and
  \inferrule* [lab=concretion] {} {{M_{C}} \bc \langle M_{P} \rangle }
  \and \\
  \inferrule* [lab=process] {} {{M_{P}} \bc M_{N} \;| \;P|M_{P} }
\end{mathpar}

\begin{definition}[contextual application] Given a context $M$, and
  process $P$, we define the \emph{contextual application}, $M[P] :=
  M\{P/\Box\}$. That is, the contextual application of M to P is the
  substitution of $P$ for $\Box$ in $M$.
\end{definition}

$\meaningof{-} : L \to \mathcal{P}(\pi)$

\begin{mathpar}
  \inferrule* [lab=collection] {} {\meaningof{true} = \pi, \and \meaningof{~E} = \pi \setminus \meaningof{E}, \and \meaningof{E_{1} \& E_{2}} = \meaningof{E_{1}} \cap \meaningof{E_{2}}}
\end{mathpar}

\begin{mathpar}
  \inferrule* [lab=structure] {} {\meaningof{0} = \{ P \in \pi | P \equiv 0 \}, \and \\ \meaningof{E_1 | E_2} = \{ P \in \pi | P \equiv P_{1} | P_{2}, P_{1} \in \meaningof{E_{1}}, P_{2} \in \meaningof{E_2}\} }
\end{mathpar}

\begin{mathpar}
 \inferrule* [lab=behavior] {} {\meaningof{\langle a?b \rangle E} = \{ P \in \pi | P \equiv Q | u?(y)P', \\ \and \\\\ \and \\ \;\;\; u \in \meaningof{a}, \forall z.P'\{z/y\} \in \meaningof{E\{z/b\}}\}, \and \\ \meaningof{a!E} = \{ P \in \pi | P \equiv Q | x!\langle P' \rangle, x \in \meaningof{a} P' \in \meaningof{E}\} }
\end{mathpar}

\begin{mathpar}
 \inferrule* [lab=nominal] {} {\meaningof{\quotep{E}} = \{ \quotep{P} \in \quotep{\pi} | P \in \meaningof{E} \}, \and \meaningof{\quotep{P}} = \{ \quotep{Q} \in \quotep{\pi} | P \equiv Q \} \and \\ \meaningof{@\quotep{E}} = \{ P \in \pi | P \equiv @x, x \in \meaningof{E} \}}
\end{mathpar}

\begin{eqnarray*}
  \\
  \meaningof{-} : TS \to ST
\end{eqnarray*}

\begin{eqnarray*}
  \\
  L : TS \to ST
\end{eqnarray*}

\begin{eqnarray*}
  \\
  P \models E \iff P \in \meaningof{E}
\end{eqnarray*}

\begin{eqnarray*}
  P \approx_{L} Q \iff \forall E \in L. P \models E \iff Q \models E
\end{eqnarray*}

\begin{eqnarray*}
  P \approx_{K} Q
\end{eqnarray*}

\begin{eqnarray*}
  P \approx Q
\end{eqnarray*}

$\approx_{K} = \approx = \approx_{L}$

\subsubsection{Contextual duality}

Note that contexts extend the quotation operation to a family of
operations from processes to names. Given a context, $M$, we can
define a \emph{nominal context}, $\quotep{M}$ by $\quotep{M}[P] :=
\quotep{M[P]}$. To foreshadow what is to come we observe that these
operations enjoy a duality with processes very much like the duality
between vectors and maps from vectors to scalars.

Further, because the calculus is essentially higher-order, we have a
correspondence between contexts and processes. More specifically,
given a name $x$ and a context $M$ we can construct $M^{*}_{x}$ such
that 

\begin{mathpar}
  M^{*}_{x} | \lift{x}{P} \red M[P]
\end{mathpar}

namely,

\begin{mathpar}
  M^{*}_{x} := x?(u).M[\dropn{u}]
\end{mathpar}

The dependence of $M^{*}_{x}$ on a name makes it an abstraction, 

\begin{mathpar}
  M^{*} := (x)x?(u).M[\dropn{u}]
\end{mathpar}

\subsection{Additional notation}

It will sometimes be convenient to denote the process a name
quotes. We already have the notation $x = \quotep{P}$, but it will be
convenient to introduce an alternate notation, $\procn{x}$, when we
want to emphasize the connection to the use of the name. Note that, by
virtue of name equivalence, $\quotep{\procn{x}} \nameeq x$; so, the
notation is consistent with previous definitions.

Further, because names have structure it is possible to effect
substitutions on the basis of that structure. This means we need to
upgrade our notation for substitutions, which we accomplish by
adapting comprehension notation. Thus,

\begin{mathpar}
  P\{ y / x : x \in S \}
\end{mathpar}

is interpreted to mean the process derived from P by replacing (in a
capture-avoiding manner) each occurrence of $x$ in $S$ by $y$. For example,

\begin{mathpar}
  P\{ \quotep{\procn{x}|\procn{x}} / x : x \in \freenames{P} \}
\end{mathpar}

will replace each (occurrence) of a free name $x$ in $P$ by
$\quotep{\procn{x}|\procn{x}}$.

Also, we will avail ourselves of the notation $x^{L}$ and $x^{R}$ to
denote injections of a name into disjoint copies of the name
space. There are numerous ways to accomplish this. One example can be
found in \cite{MeredithR05}. This notation overloads to vectors of
names: $\vec{x}^{\pi} := (x_{i}^{\pi} \; : \; 0 \leq i < |\vec{x}| )$ where $\pi \in \{L,R\}$.

We also use $P^{\Box} := P|\Box$.

In \cite{MeredithR05} an interpretation of the new operator is
given. It turns out that there are several possible interpretations
all enjoying the requisite algebraic properties of the operator (see
\cite{milner91polyadicpi}). We will therefore make liberal use of
$(\nu\; \vec{x})P$.

% subsection the_syntax_and_semantics_of_the_notation_system (end)   

\input{qm2pi.qmops} 

\input{qm2pi.sterngerlach} 

\input{qm2pi.metric} 

% section concurrent_process_calculi (end)

%\input{qm2pi.proofsketch}

% section proof sketch (end)

%\input{qm2pi.slviaknots} 

% section spatial logic via knots (end)

\input{qm2pi.conclusion}

% section conclusion (end)

%\input{qm2pi.dtcodes} 

% section wiring algorithm (end)

\input{qm2pi.ack} 

% section acknowledgments (end)

\newpage


\bibliographystyle{plain}   
\bibliography{../../biblios/main.bib}

\input{qm2pi.rhodetails}

\end{document}

 

%\ifpdf
%\usepackage[pdftex]{graphicx}
%\else
%\usepackage{graphicx}
%\fi

 % \ifpdf
%  \usepackage{pdfsync}
%  \if


%\title{Brief Article}
%\author{David F. Snyder}
%\author{L.G. Meredith}

%\address{Dept. of Math., Texas State University--San Marcos, San Marcos, TX 78666}
       
\pagestyle{empty}


\begin{document}

\lstset{language=[Objective]Caml,frame=shadowbox}

\documentclass[12pt]{llncs}
%\documentclass{jktr}

\usepackage[pdftex]{hyperref}                   
\usepackage {listings}
\usepackage {mathpartir}
\usepackage{bcprules}
%\usepackage{listings}
                       
\usepackage{graphicx} 
%\usepackage[margins=2.5cm,nohead,nofoot]{geometry}
%\usepackage{geometry}
\usepackage{amsfonts}
\usepackage{amstext}
\usepackage{latexsym}
\usepackage{amssymb}
\usepackage{color}


%\include{myPreamble}
\include{qm2pi.local} 

%\ifpdf
%\usepackage[pdftex]{graphicx}
%\else
%\usepackage{graphicx}
%\fi

 % \ifpdf
%  \usepackage{pdfsync}
%  \if


%\title{Brief Article}
%\author{David F. Snyder}
%\author{L.G. Meredith}

%\address{Dept. of Math., Texas State University--San Marcos, San Marcos, TX 78666}
       
\pagestyle{empty}


\begin{document}

\lstset{language=[Objective]Caml,frame=shadowbox}

\input{qm2pi.front}

% section front matter (end)

\input{qm2pi.intro} 
 
% section introduction (end)

% \input{qm2pi.knotations} 

% section notation (end)

\input{qm2pi.process.calculi} 

% section concurrent_process_calculi_and_spatial_logics_ (end)
    
%\input{qm2pi.knots2pi} 

%\input{qm2pi.trefoil} 

%\input{qm2pi.mainthm} 

% subsection basic_interpretation (end)

%\input{qm2pi.rho.presentation} 
\subsection{The syntax and semantics of the notation system}\label{sub:the_syntax_and_semantics_of_the_notation_system} % (fold)

We now summarize a technical presentation of the calculus that
embodies our theory of dynamics. The typical presentation of such a
calculus follows the style of giving generators and relations on
them. The grammar, below, describing term constructors, freely
generates the set of processes, $\Proc$. This set is then quotiented
by a relation known as structural congruence and it is over this set
that the notion of dynamics is expressed. This presentation is
essentially that of \cite{MeredithR05} with the addition of
polyadicity and summation. For readability we have relegated some of
the technical subtleties to an appendix.

\subsubsection{Process grammar}\label{subsub:process_grammar}

\begin{mathpar}
  \inferrule* [lab=synchronization] {} {{M} \bc \pzero \;|\; x?F \;|\; x!C }
  \and
  \inferrule* [lab=abstraction] {} {{F} \bc (x)P}
  \and
  \inferrule* [lab=concretion] {} {{C} \bc \langle Q \rangle}
  \and
  \inferrule* [lab=process] {} {{P,Q} \bc M \;| \;P|Q \;|\; @{x}}
  \and
  \inferrule* [lab=name] {} {{x} \bc \quotep{P}}
\end{mathpar} 

Note that $\vec{x}$ (resp. $\vec{P}$) denotes a vector of names
(resp. processes) of length $|\vec{x}|$ (resp. $|\vec{P}|$). We adopt
the following useful abbreviations.

\begin{mathpar}
   x?(\vec{y}).P := x.(\vec{y})P \and  x\clift{\vec{P}} := x.\clift{\vec{P}}
   \and x!(y) := \lift{x}{\dropn{y}}
   \and \Pi_{i=0}^{n-1}P_i := P_0 | \ldots | P_{n-1}
\end{mathpar}

\subsubsection{Structural congruence}

\paragraph{Free and bound names and alpha-equivalence.} At the
core of structural equivalence is alpha-equivalence which identifies
process that are the same up to a change of variable. Formally, we
recognize the distinction between free and bound names. The free names
of a process, $\freenames{P}$, may be calculated recursively as
follows:

\begin{mathpar}
\freenames{\pzero} := \emptyset
  \and \\
  \freenames{x?(y).P} := \{ x \} \cup (\freenames{P} \setminus \{ y \})
  \and 
  \freenames{x!\langle P \rangle} := \{ x \} \cup \{ P \} 
  \and \\
  \freenames{P|Q} := \freenames{P} \cup \freenames{Q}
  \and \\
  \freenames{@{x}} := \{ x \}
\end{mathpar}

$\pi$
$\quotep{\pi}$

$\freenames{-} : \pi \to \mathcal{P}(\quotep{\pi})$

\begin{eqnarray*}
  \freenames{\pzero} & := & \emptyset \\
  \freenames{x?(y).P} & := & \{ x \} \cup (\freenames{P} \setminus \{ y \}) \\
  \freenames{x!\langle P \rangle} & := & \{ x \} \cup \{ P \} \\
  \freenames{P|Q} & := & \freenames{P} \cup \freenames{Q} \\
  \freenames{\dropn{x}} & := & \{ x \}
\end{eqnarray*}

The bound names of a process, $\boundnames{P}$, are those names occurring in $P$
that are not free. For example, in $x?(y).0$, the name $x$ is free, while $y$ is bound.

\begin{mathpar}
  \inferrule* [lab=monoidal-laws] {} { P|Q \equiv Q|P \and P|0 \equiv P \and P|(Q|R) \equiv (P|Q)|R }
\end{mathpar}

\begin{mathpar}
  \inferrule* [lab=alpha-equivalence] {} { (x)P \equiv (y)P\{y/x\} \and y \not\in \freenames{P} }
\end{mathpar}

\begin{definition}
Then two processes, $P,Q$, are alpha-equivalent if $P = Q\{\vec{y}/\vec{x}\}$ for
some $\vec{x} \in \boundnames{Q},\vec{y} \in \boundnames{P}$, where $Q\{\vec{y}/\vec{x}\}$
denotes the capture-avoiding substitution of $\vec{y}$ for $\vec{x}$ in $Q$.
\end{definition}

\begin{definition}
  The {\em structural congruence} \cite{SangiorgiWalker} , $\equiv$,
  between processes is the least congruence containing
  alpha-equivalence, satisfying the abelian monoid laws
  (associativity, commutativity and $\pzero$ as identity) for parallel
  composition $|$ and for summation $+$.
\end{definition}

\subsection{Name equivalence}

We take name equivalence, written $\nameeq$, to be the smallest
equivalence relation generated by the following rules.

\begin{mathpar}
\inferrule*[lab=Quote-drop]
{ }
{ \quotep{@{x}} \nameeq x }

\inferrule*[lab=Struct-equiv]
{ P \scong Q }
{ \quotep{P} \nameeq \quotep{Q} }
\end{mathpar}

The astute reader will have noticed that the mutual recursion of names
and processes imposes a mutual recursion on alpha-equivalence and
structural equivalence via name-equivalence. Fortunately, all of this
works out pleasantly and we may calculate in the natural way, free of
concern. The reader interested in the details is referred to the
appendix \ref{appendix:rho_details}.

\subsection{Substitution}

We use $\Proc$ for the set of processes, $\QProc$ for the set of
names, and $\id{\{}\vec{y} / \vec{x} \id{\}}$ to denote partial maps,
$s : \QProc \rightarrow \QProc$. A map, $s$ lifts, uniquely, to a map
on process terms, $\widehat{s} : \Proc \rightarrow \Proc$ by the
following equations.

\begin{mathpar}
  (0) \psubstp{Q}{P} := 0 \\
  (R \juxtap S) \psubstp{Q}{P}
  :=    
  (R)\psubstp{Q}{P} \juxtap (S) \psubstp{Q}{P} \\
  (x?(y).R) \psubstp{Q}{P}    
  :=    
  (x)\substp{Q}{P} (z)\concat( (R \psubstn{z}{y}) \psubstp{Q}{P} ) \\
  (\lift{x}{R}) \psubstp{Q}{P}  
  :=
  \lift{(x)\substp{Q}{P}}{ R \psubstp{Q}{P} } \\
%   (\dropn{x})  \psubstp{Q}{P}       
%   := 
%   \left\{ 
%     \begin{array}{ccc} 
%       \dropn{\quotep{Q}} & & x \nameeq \quotep{P} \\
%       \dropn{x} & & otherwise \\
%     \end{array}
%   \right. 
  (\dropn{x})  \psubstp{Q}{P}       
  := 
  \left\{ 
    \begin{array}{ccc} 
      Q & & x \nameeq \quotep{P} \\
      \dropn{x} & & otherwise \\
    \end{array}
  \right.
\end{mathpar}
 

where

\begin{eqnarray}
  (x)\id{\{} \lpquote Q \rpquote / \lpquote P \rpquote \id{\}}            = 
  \left\{ 
    \begin{array}{ccc}
      \lpquote Q \rpquote & & x \nameeq \lpquote P \rpquote \\
      x & & otherwise \\
    \end{array}
  \right. \nonumber
\end{eqnarray}

and $z$ is chosen distinct from $\quotep{P}$, $\quotep{Q}$, the free
names in $Q$, and all the names in $R$. Our $\alpha$-equivalence will
be built in the standard way from this substitution.

\begin{remark}\label{rem:no_self_referential_names}
  One consequence of these definitions is that $\forall P. \quotep{P}
  \not\in \freenames{P}$.
\end{remark}

\subsection{ Dynamic quote: an example }

Anticipating something of what's to come, consider applying the
substitution, $\widehat{\id{\{}u / z \id{\}}}$, to the following pair
of processes, $\lift{w}{y!(z)}$ and $w[ \lpquote y!(z) \rpquote ]$.

\begin{eqnarray}
	\lift{w}{y!(z)}\widehat{\id{\{}u / z \id{\}}}
		& = &
		\lift{w}{y!(u)} \nonumber\\
	w[ \lpquote y!(z) \rpquote ] \widehat{ \id{\{}u / z \id{\}} }
		& = &
		w[ \lpquote y!(z) \rpquote ] \nonumber
\end{eqnarray}

Because the body of the process between quotes is impervious to
substitution, we get radically different answers. In fact, by
examining the first process in an input context,
e.g. $x?(z).\lift{w}{y!(z)}$, we see that the process under the lift
operator may be shaped by prefixed inputs binding a name inside it. In
this sense, the lift operator will be seen as a way to dynamically
construct processes before reifying them as names.

Finally equipped with these standard features we can present the
dynamics of the calculus.

\subsubsection{Operational semantics} 

Finally, we introduce the computational dynamics. What marks these
algebras as distinct from other more traditionally studied algebraic
structures, e.g. vector spaces or polynomial rings, is the manner in
which dynamics is captured. In traditional structures, dynamics is typically
expressed through morphisms between such structures, as in linear maps
between vector spaces or morphisms between rings. In algebras
associated with the semantics of computation, the dynamics is
expressed as part of the algebraic structure itself, through a
reduction reduction relation typically denoted by $\red$. Below, we
give a recursive presentation of this relation for the calculus used
in the encoding.

$\red \subseteq \pi \times \pi$
$\red : \pi \to \mathcal{P}(\pi)$

\begin{mathpar}
  \inferrule* [lab=Comm] { \textsf{match}( x_{src}, x_{trgt} ) } { x_{trgt}?(y)P \; | \; x_{src}!\langle {Q} \rangle \red P\{\quotep{Q}/y}\} }
  \and \\
  \inferrule* [lab=Par] {{P} \red {P}'} {{{P} | {Q}} \red {{P}' | {Q}}}
  \and
  \inferrule* [lab=Equiv]{{{P} \scong {P}'} \andalso {{P}' \red {Q}'} \andalso {{Q}' \scong {Q}}}{{P} \red {Q}}
\end{mathpar}

\begin{eqnarray*}
  match_{\equiv} (\quotep{P},\quotep{Q}) & := & P \equiv Q \\
  match_{\dagger}(\quotep{P},\quotep{Q}) & := & \forall R. P|Q \red^{*} R => R \red^{*} 0 \\
  match_{K}(\quotep{P},\quotep{Q}) & := & K \mbox{ for some context } K
\end{eqnarray*}

$u?(x)P | u!\langle Q \rangle \red P\{\quotep{Q}/x\}$

%We write $\wred$ for $\red^*$, and $P\red$ if $\exists Q $ such that $ P \red Q$.
We write $P\red$ if $\exists Q $ such that $ P \red Q$ and $P\not\red$, otherwise.

\section{Replication}

As mentioned before, it is known that replication (and hence
recursion) can be implemented in a higher-order process algebra
\cite{SangiorgiWalker}. As our first example of calculation with the
machinery thus far presented we give the construction explicitly in
the {\rhoc}.

\begin{eqnarray}
	D_{x} & := & \prefix{x}{y}{(\binpar{\outputp{x}{y}}{@{y}})} \nonumber\\
	\bangp_{x}{P} & := & \binpar{{x}!\langle{\binpar{D_{x}}{P}}\rangle}{D_{x}} \nonumber
\end{eqnarray}

\begin{eqnarray}
	\bangp_{x}{P} & & \nonumber\\
	=
	& {x}!\langle{(\prefix{x}{y}{(\outputp{x}{y} | @{y})) | P}}\rangle 
	      | \prefix{x}{y}{(\outputp{x}{y} | @{y})} & \nonumber\\
	\red
	& (\outputp{x}{y} | @{y})\substn{\quotep{(\prefix{x}{y}{(@{y} | \outputp{x}{y})) | P}}}{y} & \nonumber\\
	=
	& \outputp{x}{\quotep{(\prefix{x}{y}{(\outputp{x}{y} | @{y})) | P}}}
	  | {(\prefix{x}{y}{(\outputp{x}{y} | @{y})) | P}} & \nonumber\\
	\red
	& \ldots & \nonumber\\
	\red^*
	& P | P | \ldots & \nonumber
\end{eqnarray}

Of course, this encoding, as an implementation, runs away, unfolding
$\bangp{P}$ eagerly. A lazier and more implementable replication
operator, restricted to input-guarded processes, may be obtained as follows.

\begin{eqnarray}
\bangp{\prefix{u}{v}{P}} 
	:= 
	\binpar{\lift{x}{\prefix{u}{v}{(\binpar{D(x)}{P})}}}{D(x)} \nonumber
\end{eqnarray}

\begin{remark}
  Note that the lazier definition still does not deal with summation
  or mixed summation (i.e. sums over input and output). The reader is
  invited to construct definitions of replication that deal with these
  features. 

  Further, the definitions are parameterized in a name, $x$. Can you,
  gentle reader, make a definition that eliminates this parameter and
  guarantees no accidental interaction between the replication
  machinery and the process being replicated -- i.e. no accidental
  sharing of names used by the process to get its work done and the
  name(s) used by the replication to effect copying. This latter
  revision of the definition of replication is crucial to obtaining
  the expected identity $!!P \sim !P$.
\end{remark}

\begin{remark}\label{rem:paradoxical_combinator}
  The reader familiar with the lambda calculus will have noticed the
  similarity between $D$ and the paradoxical combinator.

  [Ed. note: the existence of this seems to suggest we have to be more
  restrictive on the set of processes and names we admit if we are to
  support no-cloning.]
\end{remark}

\subsubsection{Bisimulation}

The computational dynamics gives rise to another kind of equivalence,
the equivalence of computational behavior. As previously mentioned
this is typically captured \emph{via} some form of bisimulation.

% The notion we use in this paper is weak barbed bisimulation
% \cite{milner91polyadicpi}.

The notion we use in this paper is derived from weak barbed
bisimulation \cite{milner91polyadicpi}. 

\begin{definition}
An \emph{observation relation}, $\downarrow_{\mathcal N}$, over a set
of names, $\mathcal N$, is the smallest relation satisfying the rules
below.

\infrule[Out-barb]{y \in {\mathcal N}, \; x \nameeq y}
		  {\outputp{x}{v} \downarrow_{\mathcal N} x}
\infrule[Par-barb]{\mbox{$P\downarrow_{\mathcal N} x$ or $Q\downarrow_{\mathcal N} x$}}
		  {\binpar{P}{Q} \downarrow_{\mathcal N} x}

We write $P \Downarrow_{\mathcal N} x$ if there is $Q$ such that 
$P \wred Q$ and $Q \downarrow_{\mathcal N} x$.
\end{definition}

\begin{definition}
%\label{def.bbisim}
An  ${\mathcal N}$-\emph{barbed bisimulation} over a set of names, ${\mathcal N}$, is a symmetric binary relation 
${\mathcal S}_{\mathcal N}$ between agents such that $P\rel{S}_{\mathcal N}Q$ implies:
\begin{enumerate}
\item If $P \red P'$ then $Q \wred Q'$ and $P'\rel{S}_{\mathcal N} Q'$.
\item If $P\downarrow_{\mathcal N} x$, then $Q\Downarrow_{\mathcal N} x$.
\end{enumerate}
$P$ is ${\mathcal N}$-barbed bisimilar to $Q$, written
$P \wbbisim_{\mathcal N} Q$, if $P \rel{S}_{\mathcal N} Q$ for some ${\mathcal N}$-barbed bisimulation ${\mathcal S}_{\mathcal N}$.
\end{definition}

$\mathcal{R} \subseteq \pi \times \pi$

$P \mathcal{R} Q => \forall P'. P \red P' \Rightarrow \exists Q'. Q \red Q', P' \mathcal{R} Q'$

$P \vdash x \Rightarrow Q \vdash x$

\begin{mathpar}
  \inferrule*[lab=Out-barb]{x \nameeq y}{{y}!\langle{Q}\rangle \vdash x}
  \and
  \inferrule*[lab=Par-barb]{\mbox{$P\vdash x$ or $Q\vdash x$}}{\binpar{P}{Q} \vdash x}
\end{mathpar}

\subsubsection{Contexts}

One of the principle advantages of computational calculi like the
$\pi$-calculus is a well-defined notion of context,
contextual-equivalence and a correlation between
contextual-equivalence and notions of bisimulation. The notion of
context allows the decomposition of a process into (sub-)process and
its syntactic environment, its context. Thus, a context may be
thought of as a process with a ``hole'' (written $\Box$) in it. The
application of a context $M$ to a process $P$, written $M[P]$, is
tantamount to filling the hole in $M$ with $P$. In this paper we do
not need the full weight of this theory, but do make use of the notion
of context in the proof the main theorem. 

\begin{mathpar}
  \inferrule* [lab=summation] {} {{M_{M},M_{N}} \bc \Box \;|\; x.M_{A} \;|\; M_{M}+M_{N}}
  \and
  \inferrule* [lab=agent] {} {{M_{A}} \bc (\vec{x})M_{P} \;| \; \clift{P_0,\ldots,M_{P},\ldots,P_N}}
  \and \\
  \inferrule* [lab=process] {} {{M_{P}} \bc M_{N} \;| \;P|M_{P} }
\end{mathpar} 

\begin{mathpar}
  \inferrule* [lab=sychronization] {} {M_{N} \bc \Box \;|\; x?M_{F} \;|\; x!M_{C}}
  \and
  \inferrule* [lab=abstraction] {} {{M_{F}} \bc (x)M_{P} }
  \and
  \inferrule* [lab=concretion] {} {{M_{C}} \bc \langle M_{P} \rangle }
  \and \\
  \inferrule* [lab=process] {} {{M_{P}} \bc M_{N} \;| \;P|M_{P} }
\end{mathpar}

\begin{definition}[contextual application] Given a context $M$, and
  process $P$, we define the \emph{contextual application}, $M[P] :=
  M\{P/\Box\}$. That is, the contextual application of M to P is the
  substitution of $P$ for $\Box$ in $M$.
\end{definition}

$\meaningof{-} : L \to \mathcal{P}(\pi)$

\begin{mathpar}
  \inferrule* [lab=collection] {} {\meaningof{true} = \pi, \and \meaningof{~E} = \pi \setminus \meaningof{E}, \and \meaningof{E_{1} \& E_{2}} = \meaningof{E_{1}} \cap \meaningof{E_{2}}}
\end{mathpar}

\begin{mathpar}
  \inferrule* [lab=structure] {} {\meaningof{0} = \{ P \in \pi | P \equiv 0 \}, \and \\ \meaningof{E_1 | E_2} = \{ P \in \pi | P \equiv P_{1} | P_{2}, P_{1} \in \meaningof{E_{1}}, P_{2} \in \meaningof{E_2}\} }
\end{mathpar}

\begin{mathpar}
 \inferrule* [lab=behavior] {} {\meaningof{\langle a?b \rangle E} = \{ P \in \pi | P \equiv Q | u?(y)P', \\ \and \\\\ \and \\ \;\;\; u \in \meaningof{a}, \forall z.P'\{z/y\} \in \meaningof{E\{z/b\}}\}, \and \\ \meaningof{a!E} = \{ P \in \pi | P \equiv Q | x!\langle P' \rangle, x \in \meaningof{a} P' \in \meaningof{E}\} }
\end{mathpar}

\begin{mathpar}
 \inferrule* [lab=nominal] {} {\meaningof{\quotep{E}} = \{ \quotep{P} \in \quotep{\pi} | P \in \meaningof{E} \}, \and \meaningof{\quotep{P}} = \{ \quotep{Q} \in \quotep{\pi} | P \equiv Q \} \and \\ \meaningof{@\quotep{E}} = \{ P \in \pi | P \equiv @x, x \in \meaningof{E} \}}
\end{mathpar}

\begin{eqnarray*}
  \\
  \meaningof{-} : TS \to ST
\end{eqnarray*}

\begin{eqnarray*}
  \\
  L : TS \to ST
\end{eqnarray*}

\begin{eqnarray*}
  \\
  P \models E \iff P \in \meaningof{E}
\end{eqnarray*}

\begin{eqnarray*}
  P \approx_{L} Q \iff \forall E \in L. P \models E \iff Q \models E
\end{eqnarray*}

\begin{eqnarray*}
  P \approx_{K} Q
\end{eqnarray*}

\begin{eqnarray*}
  P \approx Q
\end{eqnarray*}

$\approx_{K} = \approx = \approx_{L}$

\subsubsection{Contextual duality}

Note that contexts extend the quotation operation to a family of
operations from processes to names. Given a context, $M$, we can
define a \emph{nominal context}, $\quotep{M}$ by $\quotep{M}[P] :=
\quotep{M[P]}$. To foreshadow what is to come we observe that these
operations enjoy a duality with processes very much like the duality
between vectors and maps from vectors to scalars.

Further, because the calculus is essentially higher-order, we have a
correspondence between contexts and processes. More specifically,
given a name $x$ and a context $M$ we can construct $M^{*}_{x}$ such
that 

\begin{mathpar}
  M^{*}_{x} | \lift{x}{P} \red M[P]
\end{mathpar}

namely,

\begin{mathpar}
  M^{*}_{x} := x?(u).M[\dropn{u}]
\end{mathpar}

The dependence of $M^{*}_{x}$ on a name makes it an abstraction, 

\begin{mathpar}
  M^{*} := (x)x?(u).M[\dropn{u}]
\end{mathpar}

\subsection{Additional notation}

It will sometimes be convenient to denote the process a name
quotes. We already have the notation $x = \quotep{P}$, but it will be
convenient to introduce an alternate notation, $\procn{x}$, when we
want to emphasize the connection to the use of the name. Note that, by
virtue of name equivalence, $\quotep{\procn{x}} \nameeq x$; so, the
notation is consistent with previous definitions.

Further, because names have structure it is possible to effect
substitutions on the basis of that structure. This means we need to
upgrade our notation for substitutions, which we accomplish by
adapting comprehension notation. Thus,

\begin{mathpar}
  P\{ y / x : x \in S \}
\end{mathpar}

is interpreted to mean the process derived from P by replacing (in a
capture-avoiding manner) each occurrence of $x$ in $S$ by $y$. For example,

\begin{mathpar}
  P\{ \quotep{\procn{x}|\procn{x}} / x : x \in \freenames{P} \}
\end{mathpar}

will replace each (occurrence) of a free name $x$ in $P$ by
$\quotep{\procn{x}|\procn{x}}$.

Also, we will avail ourselves of the notation $x^{L}$ and $x^{R}$ to
denote injections of a name into disjoint copies of the name
space. There are numerous ways to accomplish this. One example can be
found in \cite{MeredithR05}. This notation overloads to vectors of
names: $\vec{x}^{\pi} := (x_{i}^{\pi} \; : \; 0 \leq i < |\vec{x}| )$ where $\pi \in \{L,R\}$.

We also use $P^{\Box} := P|\Box$.

In \cite{MeredithR05} an interpretation of the new operator is
given. It turns out that there are several possible interpretations
all enjoying the requisite algebraic properties of the operator (see
\cite{milner91polyadicpi}). We will therefore make liberal use of
$(\nu\; \vec{x})P$.

% subsection the_syntax_and_semantics_of_the_notation_system (end)   

\input{qm2pi.qmops} 

\input{qm2pi.sterngerlach} 

\input{qm2pi.metric} 

% section concurrent_process_calculi (end)

%\input{qm2pi.proofsketch}

% section proof sketch (end)

%\input{qm2pi.slviaknots} 

% section spatial logic via knots (end)

\input{qm2pi.conclusion}

% section conclusion (end)

%\input{qm2pi.dtcodes} 

% section wiring algorithm (end)

\input{qm2pi.ack} 

% section acknowledgments (end)

\newpage


\bibliographystyle{plain}   
\bibliography{../../biblios/main.bib}

\input{qm2pi.rhodetails}

\end{document}



% section front matter (end)

\section{Introduction}\label{sec:introduction} % (fold)
In this draft of the material i am going to have to dispense with the
usual writing conventions adopted in papers on these topics. i'm going
to have adopt whatever tone i need at the time i'm writing up the
calculations. Sometimes this may be very conversational; others it may
be the barest mathematical grunts; others still it may be that i have
lifted text from one of my other papers because the exposition of some
point was better said there. i hope that my readers are not unduly put
out by this decision. i'm not doing this to flout convention or be
rebellious. i find these calculations very technically challenging. To
keep everything going technically, something has to give; i have to
let go of some cognitive burden. So, the academic writing style --
with all of its trade-offs in terms of facilitating technical
communication -- is what i'm letting go of. Perhaps subsequent drafts
can be tightened and polished, but for now, i'm going to speak as if
we were sitting together in a coffee shop with a laptop, wifi and a
pad of paper and a pencil.

So, here's what i have to say. We -- you and i, comfortably ensconced
in our coffee shop and well-equipped with our tools -- can realize and
carry out the calculations of quantum mechanics over a very different
formal theory of dynamics, a formal theory of dynamics that
corresponds to a theory of concurrent computation with
\emph{reflection}. It has the advantage that the underlying theory is
already `quantized', but supports analogues all of the continuuous
operations. Strikingly, this underlying theory has recently been
connected with a notion of metric that we can show, by calculating
together, coincides with the metric induced by the inner product.

There are a lot of reasons why you might be interested in seeing
calculations of this form. Here's why i'm interested. For the past
several centuries there has been no competitor to the ``Newtonian''
account of dynamics. As a result the predominant share of accounts of
dynamical systems and situations have had to be formulated in terms of
the Newtonian machinery. i view this as an intellectually dangerous
position to occupy. Everything, despite it's intrinsic shape, turns
into a nail to be hit with this hammer. Recently, however, the theory
of computation has matured to the point where we have candidates for
theories of dynamics that offer very different perspective on
reasoning about dynamical systems and situations. Testing these
candidates against very successful accounts of dynamical situations,
like quantum mechanics, is going to give us some sense of how mature
they are and some measure of the quality of these accounts of
dynamics.

\subsection{Summary of contributions and outline of paper}

So, we're going to develop an interpretation of the operations of
quantum mechanics normally interpreted by Hilbert spaces and
operators. We're going to do this over a theory of computation. Note
that this is very different than the usual quantum computation program
which develops notions of computation over quantum mechanics. Rather,
we are developing a story that aligns with Wheeler's slogan: It from
Bit. To do this we will first provide an account of the theory of
computation at play here. Then we will dive into a calculation-driven
interpretation of the operations of quantum mechanics.

The reason we take this approach is that -- until very recently --
there hasn't been an axiomatic account of quantum mechanics. As a
result there has been no sharp delineation of the mathematical theory
supporting interpretation of the physical theory and the physical
theory, itself. So, ambient features of the maths are free to be
exploited (or supressed) without a real accounting of their physical
relevance. There is no sharp statement ``here's the physical theory''
qua \emph{theory} and ``here's the mathematical interpretation''
enabling a judgment of how faithful the interpretation is -- apart
from experimental observation. When there is an axiomatic account we
can judge how well a given mathematical formalism supports an
interpretation of the axioms, independent of
experimentation. Likewise, we can judge how well we have captured our
physical evidence and experience with our axiomatics, independent of
any specific mathematical implementation, with accidental detail that
may or may not have physical significance. 

In lieu of a fully fleshed out and vetted axiomatic account of quantum
mechanics, interpreting the operational notions in service of modeling
physical systems will have to suffice. In other words, we are not in
the business of providing a model of Hilbert spaces and operators. We
are in the business of providing a model of quantum mechanics because
we are motivated by testing our notions of dynamics against physical
theory; and, the predictive calculations of the physical theory must
serve as the best formulation -- shy of a fully fleshed out axiomatic
account -- of the physical theory itself (as they have for scientific
theories since time immemorial). Put another way, despite a
whole-hearted commitment to an It-from-Bit ontology, we are firmly
aligned with the shut-up-and-calculate camp as the best way to obtain
results either from the physical perspective or as a quality assurance
measure of our fledgling theory of dynamics.

In detail, we present a reflective process calculus. Then we develop
intuitive correspondences between the notions available in this
calculus and the usual physical notions supporting quantum mechanical
calculations. Thus, 

\begin{table}[htp]
  \center{
    \fbox{
      \begin{tabular}{c|c}
        quantum mechanics & process calculus \\
        \hline
        scalar & name \\
        state vector & process \\
        dual & contextual duals \\
        matrix & formal sums of process-context-dual pairs \\
        orthogonality & process annihilation \\
        inner product & execution-formula + quoting
      \end{tabular}
    }
  }
  \caption{QM - process calculi correspondences}
\end{table}

Then we tighten up these intuitions to operational definitions. We
employ the Dirac notation as the best proxy we can find for an
abstract syntax of the quantum mechanical notions. The definitions we
develop put us in contact with equational constraints coming from the
theory that we demonstrate the definitions and calculations satisfy.

This puts us in a position to shut up and calculate for the
Stern-Gerlach experimental set up, showing how these predictive
calculations become calculations on processes in our theory of a
reflective process calculus.

Penultimately, we demonstrate that the notion of metric coming from
the inner product coincides with the notion of metric available from
the theory of bisimulation. This demonstration gives us the right to
think of space as arising from behavior. Finally, we consider where we
might go from the new vantage point we have obtained.

% section introduction (end) 
 
% section introduction (end)

% \documentclass[12pt]{llncs}
%\documentclass{jktr}

\usepackage[pdftex]{hyperref}                   
\usepackage {listings}
\usepackage {mathpartir}
\usepackage{bcprules}
%\usepackage{listings}
                       
\usepackage{graphicx} 
%\usepackage[margins=2.5cm,nohead,nofoot]{geometry}
%\usepackage{geometry}
\usepackage{amsfonts}
\usepackage{amstext}
\usepackage{latexsym}
\usepackage{amssymb}
\usepackage{color}


%\include{myPreamble}
\include{qm2pi.local} 

%\ifpdf
%\usepackage[pdftex]{graphicx}
%\else
%\usepackage{graphicx}
%\fi

 % \ifpdf
%  \usepackage{pdfsync}
%  \if


%\title{Brief Article}
%\author{David F. Snyder}
%\author{L.G. Meredith}

%\address{Dept. of Math., Texas State University--San Marcos, San Marcos, TX 78666}
       
\pagestyle{empty}


\begin{document}

\lstset{language=[Objective]Caml,frame=shadowbox}

\input{qm2pi.front}

% section front matter (end)

\input{qm2pi.intro} 
 
% section introduction (end)

% \input{qm2pi.knotations} 

% section notation (end)

\input{qm2pi.process.calculi} 

% section concurrent_process_calculi_and_spatial_logics_ (end)
    
%\input{qm2pi.knots2pi} 

%\input{qm2pi.trefoil} 

%\input{qm2pi.mainthm} 

% subsection basic_interpretation (end)

%\input{qm2pi.rho.presentation} 
\subsection{The syntax and semantics of the notation system}\label{sub:the_syntax_and_semantics_of_the_notation_system} % (fold)

We now summarize a technical presentation of the calculus that
embodies our theory of dynamics. The typical presentation of such a
calculus follows the style of giving generators and relations on
them. The grammar, below, describing term constructors, freely
generates the set of processes, $\Proc$. This set is then quotiented
by a relation known as structural congruence and it is over this set
that the notion of dynamics is expressed. This presentation is
essentially that of \cite{MeredithR05} with the addition of
polyadicity and summation. For readability we have relegated some of
the technical subtleties to an appendix.

\subsubsection{Process grammar}\label{subsub:process_grammar}

\begin{mathpar}
  \inferrule* [lab=synchronization] {} {{M} \bc \pzero \;|\; x?F \;|\; x!C }
  \and
  \inferrule* [lab=abstraction] {} {{F} \bc (x)P}
  \and
  \inferrule* [lab=concretion] {} {{C} \bc \langle Q \rangle}
  \and
  \inferrule* [lab=process] {} {{P,Q} \bc M \;| \;P|Q \;|\; @{x}}
  \and
  \inferrule* [lab=name] {} {{x} \bc \quotep{P}}
\end{mathpar} 

Note that $\vec{x}$ (resp. $\vec{P}$) denotes a vector of names
(resp. processes) of length $|\vec{x}|$ (resp. $|\vec{P}|$). We adopt
the following useful abbreviations.

\begin{mathpar}
   x?(\vec{y}).P := x.(\vec{y})P \and  x\clift{\vec{P}} := x.\clift{\vec{P}}
   \and x!(y) := \lift{x}{\dropn{y}}
   \and \Pi_{i=0}^{n-1}P_i := P_0 | \ldots | P_{n-1}
\end{mathpar}

\subsubsection{Structural congruence}

\paragraph{Free and bound names and alpha-equivalence.} At the
core of structural equivalence is alpha-equivalence which identifies
process that are the same up to a change of variable. Formally, we
recognize the distinction between free and bound names. The free names
of a process, $\freenames{P}$, may be calculated recursively as
follows:

\begin{mathpar}
\freenames{\pzero} := \emptyset
  \and \\
  \freenames{x?(y).P} := \{ x \} \cup (\freenames{P} \setminus \{ y \})
  \and 
  \freenames{x!\langle P \rangle} := \{ x \} \cup \{ P \} 
  \and \\
  \freenames{P|Q} := \freenames{P} \cup \freenames{Q}
  \and \\
  \freenames{@{x}} := \{ x \}
\end{mathpar}

$\pi$
$\quotep{\pi}$

$\freenames{-} : \pi \to \mathcal{P}(\quotep{\pi})$

\begin{eqnarray*}
  \freenames{\pzero} & := & \emptyset \\
  \freenames{x?(y).P} & := & \{ x \} \cup (\freenames{P} \setminus \{ y \}) \\
  \freenames{x!\langle P \rangle} & := & \{ x \} \cup \{ P \} \\
  \freenames{P|Q} & := & \freenames{P} \cup \freenames{Q} \\
  \freenames{\dropn{x}} & := & \{ x \}
\end{eqnarray*}

The bound names of a process, $\boundnames{P}$, are those names occurring in $P$
that are not free. For example, in $x?(y).0$, the name $x$ is free, while $y$ is bound.

\begin{mathpar}
  \inferrule* [lab=monoidal-laws] {} { P|Q \equiv Q|P \and P|0 \equiv P \and P|(Q|R) \equiv (P|Q)|R }
\end{mathpar}

\begin{mathpar}
  \inferrule* [lab=alpha-equivalence] {} { (x)P \equiv (y)P\{y/x\} \and y \not\in \freenames{P} }
\end{mathpar}

\begin{definition}
Then two processes, $P,Q$, are alpha-equivalent if $P = Q\{\vec{y}/\vec{x}\}$ for
some $\vec{x} \in \boundnames{Q},\vec{y} \in \boundnames{P}$, where $Q\{\vec{y}/\vec{x}\}$
denotes the capture-avoiding substitution of $\vec{y}$ for $\vec{x}$ in $Q$.
\end{definition}

\begin{definition}
  The {\em structural congruence} \cite{SangiorgiWalker} , $\equiv$,
  between processes is the least congruence containing
  alpha-equivalence, satisfying the abelian monoid laws
  (associativity, commutativity and $\pzero$ as identity) for parallel
  composition $|$ and for summation $+$.
\end{definition}

\subsection{Name equivalence}

We take name equivalence, written $\nameeq$, to be the smallest
equivalence relation generated by the following rules.

\begin{mathpar}
\inferrule*[lab=Quote-drop]
{ }
{ \quotep{@{x}} \nameeq x }

\inferrule*[lab=Struct-equiv]
{ P \scong Q }
{ \quotep{P} \nameeq \quotep{Q} }
\end{mathpar}

The astute reader will have noticed that the mutual recursion of names
and processes imposes a mutual recursion on alpha-equivalence and
structural equivalence via name-equivalence. Fortunately, all of this
works out pleasantly and we may calculate in the natural way, free of
concern. The reader interested in the details is referred to the
appendix \ref{appendix:rho_details}.

\subsection{Substitution}

We use $\Proc$ for the set of processes, $\QProc$ for the set of
names, and $\id{\{}\vec{y} / \vec{x} \id{\}}$ to denote partial maps,
$s : \QProc \rightarrow \QProc$. A map, $s$ lifts, uniquely, to a map
on process terms, $\widehat{s} : \Proc \rightarrow \Proc$ by the
following equations.

\begin{mathpar}
  (0) \psubstp{Q}{P} := 0 \\
  (R \juxtap S) \psubstp{Q}{P}
  :=    
  (R)\psubstp{Q}{P} \juxtap (S) \psubstp{Q}{P} \\
  (x?(y).R) \psubstp{Q}{P}    
  :=    
  (x)\substp{Q}{P} (z)\concat( (R \psubstn{z}{y}) \psubstp{Q}{P} ) \\
  (\lift{x}{R}) \psubstp{Q}{P}  
  :=
  \lift{(x)\substp{Q}{P}}{ R \psubstp{Q}{P} } \\
%   (\dropn{x})  \psubstp{Q}{P}       
%   := 
%   \left\{ 
%     \begin{array}{ccc} 
%       \dropn{\quotep{Q}} & & x \nameeq \quotep{P} \\
%       \dropn{x} & & otherwise \\
%     \end{array}
%   \right. 
  (\dropn{x})  \psubstp{Q}{P}       
  := 
  \left\{ 
    \begin{array}{ccc} 
      Q & & x \nameeq \quotep{P} \\
      \dropn{x} & & otherwise \\
    \end{array}
  \right.
\end{mathpar}
 

where

\begin{eqnarray}
  (x)\id{\{} \lpquote Q \rpquote / \lpquote P \rpquote \id{\}}            = 
  \left\{ 
    \begin{array}{ccc}
      \lpquote Q \rpquote & & x \nameeq \lpquote P \rpquote \\
      x & & otherwise \\
    \end{array}
  \right. \nonumber
\end{eqnarray}

and $z$ is chosen distinct from $\quotep{P}$, $\quotep{Q}$, the free
names in $Q$, and all the names in $R$. Our $\alpha$-equivalence will
be built in the standard way from this substitution.

\begin{remark}\label{rem:no_self_referential_names}
  One consequence of these definitions is that $\forall P. \quotep{P}
  \not\in \freenames{P}$.
\end{remark}

\subsection{ Dynamic quote: an example }

Anticipating something of what's to come, consider applying the
substitution, $\widehat{\id{\{}u / z \id{\}}}$, to the following pair
of processes, $\lift{w}{y!(z)}$ and $w[ \lpquote y!(z) \rpquote ]$.

\begin{eqnarray}
	\lift{w}{y!(z)}\widehat{\id{\{}u / z \id{\}}}
		& = &
		\lift{w}{y!(u)} \nonumber\\
	w[ \lpquote y!(z) \rpquote ] \widehat{ \id{\{}u / z \id{\}} }
		& = &
		w[ \lpquote y!(z) \rpquote ] \nonumber
\end{eqnarray}

Because the body of the process between quotes is impervious to
substitution, we get radically different answers. In fact, by
examining the first process in an input context,
e.g. $x?(z).\lift{w}{y!(z)}$, we see that the process under the lift
operator may be shaped by prefixed inputs binding a name inside it. In
this sense, the lift operator will be seen as a way to dynamically
construct processes before reifying them as names.

Finally equipped with these standard features we can present the
dynamics of the calculus.

\subsubsection{Operational semantics} 

Finally, we introduce the computational dynamics. What marks these
algebras as distinct from other more traditionally studied algebraic
structures, e.g. vector spaces or polynomial rings, is the manner in
which dynamics is captured. In traditional structures, dynamics is typically
expressed through morphisms between such structures, as in linear maps
between vector spaces or morphisms between rings. In algebras
associated with the semantics of computation, the dynamics is
expressed as part of the algebraic structure itself, through a
reduction reduction relation typically denoted by $\red$. Below, we
give a recursive presentation of this relation for the calculus used
in the encoding.

$\red \subseteq \pi \times \pi$
$\red : \pi \to \mathcal{P}(\pi)$

\begin{mathpar}
  \inferrule* [lab=Comm] { \textsf{match}( x_{src}, x_{trgt} ) } { x_{trgt}?(y)P \; | \; x_{src}!\langle {Q} \rangle \red P\{\quotep{Q}/y}\} }
  \and \\
  \inferrule* [lab=Par] {{P} \red {P}'} {{{P} | {Q}} \red {{P}' | {Q}}}
  \and
  \inferrule* [lab=Equiv]{{{P} \scong {P}'} \andalso {{P}' \red {Q}'} \andalso {{Q}' \scong {Q}}}{{P} \red {Q}}
\end{mathpar}

\begin{eqnarray*}
  match_{\equiv} (\quotep{P},\quotep{Q}) & := & P \equiv Q \\
  match_{\dagger}(\quotep{P},\quotep{Q}) & := & \forall R. P|Q \red^{*} R => R \red^{*} 0 \\
  match_{K}(\quotep{P},\quotep{Q}) & := & K \mbox{ for some context } K
\end{eqnarray*}

$u?(x)P | u!\langle Q \rangle \red P\{\quotep{Q}/x\}$

%We write $\wred$ for $\red^*$, and $P\red$ if $\exists Q $ such that $ P \red Q$.
We write $P\red$ if $\exists Q $ such that $ P \red Q$ and $P\not\red$, otherwise.

\section{Replication}

As mentioned before, it is known that replication (and hence
recursion) can be implemented in a higher-order process algebra
\cite{SangiorgiWalker}. As our first example of calculation with the
machinery thus far presented we give the construction explicitly in
the {\rhoc}.

\begin{eqnarray}
	D_{x} & := & \prefix{x}{y}{(\binpar{\outputp{x}{y}}{@{y}})} \nonumber\\
	\bangp_{x}{P} & := & \binpar{{x}!\langle{\binpar{D_{x}}{P}}\rangle}{D_{x}} \nonumber
\end{eqnarray}

\begin{eqnarray}
	\bangp_{x}{P} & & \nonumber\\
	=
	& {x}!\langle{(\prefix{x}{y}{(\outputp{x}{y} | @{y})) | P}}\rangle 
	      | \prefix{x}{y}{(\outputp{x}{y} | @{y})} & \nonumber\\
	\red
	& (\outputp{x}{y} | @{y})\substn{\quotep{(\prefix{x}{y}{(@{y} | \outputp{x}{y})) | P}}}{y} & \nonumber\\
	=
	& \outputp{x}{\quotep{(\prefix{x}{y}{(\outputp{x}{y} | @{y})) | P}}}
	  | {(\prefix{x}{y}{(\outputp{x}{y} | @{y})) | P}} & \nonumber\\
	\red
	& \ldots & \nonumber\\
	\red^*
	& P | P | \ldots & \nonumber
\end{eqnarray}

Of course, this encoding, as an implementation, runs away, unfolding
$\bangp{P}$ eagerly. A lazier and more implementable replication
operator, restricted to input-guarded processes, may be obtained as follows.

\begin{eqnarray}
\bangp{\prefix{u}{v}{P}} 
	:= 
	\binpar{\lift{x}{\prefix{u}{v}{(\binpar{D(x)}{P})}}}{D(x)} \nonumber
\end{eqnarray}

\begin{remark}
  Note that the lazier definition still does not deal with summation
  or mixed summation (i.e. sums over input and output). The reader is
  invited to construct definitions of replication that deal with these
  features. 

  Further, the definitions are parameterized in a name, $x$. Can you,
  gentle reader, make a definition that eliminates this parameter and
  guarantees no accidental interaction between the replication
  machinery and the process being replicated -- i.e. no accidental
  sharing of names used by the process to get its work done and the
  name(s) used by the replication to effect copying. This latter
  revision of the definition of replication is crucial to obtaining
  the expected identity $!!P \sim !P$.
\end{remark}

\begin{remark}\label{rem:paradoxical_combinator}
  The reader familiar with the lambda calculus will have noticed the
  similarity between $D$ and the paradoxical combinator.

  [Ed. note: the existence of this seems to suggest we have to be more
  restrictive on the set of processes and names we admit if we are to
  support no-cloning.]
\end{remark}

\subsubsection{Bisimulation}

The computational dynamics gives rise to another kind of equivalence,
the equivalence of computational behavior. As previously mentioned
this is typically captured \emph{via} some form of bisimulation.

% The notion we use in this paper is weak barbed bisimulation
% \cite{milner91polyadicpi}.

The notion we use in this paper is derived from weak barbed
bisimulation \cite{milner91polyadicpi}. 

\begin{definition}
An \emph{observation relation}, $\downarrow_{\mathcal N}$, over a set
of names, $\mathcal N$, is the smallest relation satisfying the rules
below.

\infrule[Out-barb]{y \in {\mathcal N}, \; x \nameeq y}
		  {\outputp{x}{v} \downarrow_{\mathcal N} x}
\infrule[Par-barb]{\mbox{$P\downarrow_{\mathcal N} x$ or $Q\downarrow_{\mathcal N} x$}}
		  {\binpar{P}{Q} \downarrow_{\mathcal N} x}

We write $P \Downarrow_{\mathcal N} x$ if there is $Q$ such that 
$P \wred Q$ and $Q \downarrow_{\mathcal N} x$.
\end{definition}

\begin{definition}
%\label{def.bbisim}
An  ${\mathcal N}$-\emph{barbed bisimulation} over a set of names, ${\mathcal N}$, is a symmetric binary relation 
${\mathcal S}_{\mathcal N}$ between agents such that $P\rel{S}_{\mathcal N}Q$ implies:
\begin{enumerate}
\item If $P \red P'$ then $Q \wred Q'$ and $P'\rel{S}_{\mathcal N} Q'$.
\item If $P\downarrow_{\mathcal N} x$, then $Q\Downarrow_{\mathcal N} x$.
\end{enumerate}
$P$ is ${\mathcal N}$-barbed bisimilar to $Q$, written
$P \wbbisim_{\mathcal N} Q$, if $P \rel{S}_{\mathcal N} Q$ for some ${\mathcal N}$-barbed bisimulation ${\mathcal S}_{\mathcal N}$.
\end{definition}

$\mathcal{R} \subseteq \pi \times \pi$

$P \mathcal{R} Q => \forall P'. P \red P' \Rightarrow \exists Q'. Q \red Q', P' \mathcal{R} Q'$

$P \vdash x \Rightarrow Q \vdash x$

\begin{mathpar}
  \inferrule*[lab=Out-barb]{x \nameeq y}{{y}!\langle{Q}\rangle \vdash x}
  \and
  \inferrule*[lab=Par-barb]{\mbox{$P\vdash x$ or $Q\vdash x$}}{\binpar{P}{Q} \vdash x}
\end{mathpar}

\subsubsection{Contexts}

One of the principle advantages of computational calculi like the
$\pi$-calculus is a well-defined notion of context,
contextual-equivalence and a correlation between
contextual-equivalence and notions of bisimulation. The notion of
context allows the decomposition of a process into (sub-)process and
its syntactic environment, its context. Thus, a context may be
thought of as a process with a ``hole'' (written $\Box$) in it. The
application of a context $M$ to a process $P$, written $M[P]$, is
tantamount to filling the hole in $M$ with $P$. In this paper we do
not need the full weight of this theory, but do make use of the notion
of context in the proof the main theorem. 

\begin{mathpar}
  \inferrule* [lab=summation] {} {{M_{M},M_{N}} \bc \Box \;|\; x.M_{A} \;|\; M_{M}+M_{N}}
  \and
  \inferrule* [lab=agent] {} {{M_{A}} \bc (\vec{x})M_{P} \;| \; \clift{P_0,\ldots,M_{P},\ldots,P_N}}
  \and \\
  \inferrule* [lab=process] {} {{M_{P}} \bc M_{N} \;| \;P|M_{P} }
\end{mathpar} 

\begin{mathpar}
  \inferrule* [lab=sychronization] {} {M_{N} \bc \Box \;|\; x?M_{F} \;|\; x!M_{C}}
  \and
  \inferrule* [lab=abstraction] {} {{M_{F}} \bc (x)M_{P} }
  \and
  \inferrule* [lab=concretion] {} {{M_{C}} \bc \langle M_{P} \rangle }
  \and \\
  \inferrule* [lab=process] {} {{M_{P}} \bc M_{N} \;| \;P|M_{P} }
\end{mathpar}

\begin{definition}[contextual application] Given a context $M$, and
  process $P$, we define the \emph{contextual application}, $M[P] :=
  M\{P/\Box\}$. That is, the contextual application of M to P is the
  substitution of $P$ for $\Box$ in $M$.
\end{definition}

$\meaningof{-} : L \to \mathcal{P}(\pi)$

\begin{mathpar}
  \inferrule* [lab=collection] {} {\meaningof{true} = \pi, \and \meaningof{~E} = \pi \setminus \meaningof{E}, \and \meaningof{E_{1} \& E_{2}} = \meaningof{E_{1}} \cap \meaningof{E_{2}}}
\end{mathpar}

\begin{mathpar}
  \inferrule* [lab=structure] {} {\meaningof{0} = \{ P \in \pi | P \equiv 0 \}, \and \\ \meaningof{E_1 | E_2} = \{ P \in \pi | P \equiv P_{1} | P_{2}, P_{1} \in \meaningof{E_{1}}, P_{2} \in \meaningof{E_2}\} }
\end{mathpar}

\begin{mathpar}
 \inferrule* [lab=behavior] {} {\meaningof{\langle a?b \rangle E} = \{ P \in \pi | P \equiv Q | u?(y)P', \\ \and \\\\ \and \\ \;\;\; u \in \meaningof{a}, \forall z.P'\{z/y\} \in \meaningof{E\{z/b\}}\}, \and \\ \meaningof{a!E} = \{ P \in \pi | P \equiv Q | x!\langle P' \rangle, x \in \meaningof{a} P' \in \meaningof{E}\} }
\end{mathpar}

\begin{mathpar}
 \inferrule* [lab=nominal] {} {\meaningof{\quotep{E}} = \{ \quotep{P} \in \quotep{\pi} | P \in \meaningof{E} \}, \and \meaningof{\quotep{P}} = \{ \quotep{Q} \in \quotep{\pi} | P \equiv Q \} \and \\ \meaningof{@\quotep{E}} = \{ P \in \pi | P \equiv @x, x \in \meaningof{E} \}}
\end{mathpar}

\begin{eqnarray*}
  \\
  \meaningof{-} : TS \to ST
\end{eqnarray*}

\begin{eqnarray*}
  \\
  L : TS \to ST
\end{eqnarray*}

\begin{eqnarray*}
  \\
  P \models E \iff P \in \meaningof{E}
\end{eqnarray*}

\begin{eqnarray*}
  P \approx_{L} Q \iff \forall E \in L. P \models E \iff Q \models E
\end{eqnarray*}

\begin{eqnarray*}
  P \approx_{K} Q
\end{eqnarray*}

\begin{eqnarray*}
  P \approx Q
\end{eqnarray*}

$\approx_{K} = \approx = \approx_{L}$

\subsubsection{Contextual duality}

Note that contexts extend the quotation operation to a family of
operations from processes to names. Given a context, $M$, we can
define a \emph{nominal context}, $\quotep{M}$ by $\quotep{M}[P] :=
\quotep{M[P]}$. To foreshadow what is to come we observe that these
operations enjoy a duality with processes very much like the duality
between vectors and maps from vectors to scalars.

Further, because the calculus is essentially higher-order, we have a
correspondence between contexts and processes. More specifically,
given a name $x$ and a context $M$ we can construct $M^{*}_{x}$ such
that 

\begin{mathpar}
  M^{*}_{x} | \lift{x}{P} \red M[P]
\end{mathpar}

namely,

\begin{mathpar}
  M^{*}_{x} := x?(u).M[\dropn{u}]
\end{mathpar}

The dependence of $M^{*}_{x}$ on a name makes it an abstraction, 

\begin{mathpar}
  M^{*} := (x)x?(u).M[\dropn{u}]
\end{mathpar}

\subsection{Additional notation}

It will sometimes be convenient to denote the process a name
quotes. We already have the notation $x = \quotep{P}$, but it will be
convenient to introduce an alternate notation, $\procn{x}$, when we
want to emphasize the connection to the use of the name. Note that, by
virtue of name equivalence, $\quotep{\procn{x}} \nameeq x$; so, the
notation is consistent with previous definitions.

Further, because names have structure it is possible to effect
substitutions on the basis of that structure. This means we need to
upgrade our notation for substitutions, which we accomplish by
adapting comprehension notation. Thus,

\begin{mathpar}
  P\{ y / x : x \in S \}
\end{mathpar}

is interpreted to mean the process derived from P by replacing (in a
capture-avoiding manner) each occurrence of $x$ in $S$ by $y$. For example,

\begin{mathpar}
  P\{ \quotep{\procn{x}|\procn{x}} / x : x \in \freenames{P} \}
\end{mathpar}

will replace each (occurrence) of a free name $x$ in $P$ by
$\quotep{\procn{x}|\procn{x}}$.

Also, we will avail ourselves of the notation $x^{L}$ and $x^{R}$ to
denote injections of a name into disjoint copies of the name
space. There are numerous ways to accomplish this. One example can be
found in \cite{MeredithR05}. This notation overloads to vectors of
names: $\vec{x}^{\pi} := (x_{i}^{\pi} \; : \; 0 \leq i < |\vec{x}| )$ where $\pi \in \{L,R\}$.

We also use $P^{\Box} := P|\Box$.

In \cite{MeredithR05} an interpretation of the new operator is
given. It turns out that there are several possible interpretations
all enjoying the requisite algebraic properties of the operator (see
\cite{milner91polyadicpi}). We will therefore make liberal use of
$(\nu\; \vec{x})P$.

% subsection the_syntax_and_semantics_of_the_notation_system (end)   

\input{qm2pi.qmops} 

\input{qm2pi.sterngerlach} 

\input{qm2pi.metric} 

% section concurrent_process_calculi (end)

%\input{qm2pi.proofsketch}

% section proof sketch (end)

%\input{qm2pi.slviaknots} 

% section spatial logic via knots (end)

\input{qm2pi.conclusion}

% section conclusion (end)

%\input{qm2pi.dtcodes} 

% section wiring algorithm (end)

\input{qm2pi.ack} 

% section acknowledgments (end)

\newpage


\bibliographystyle{plain}   
\bibliography{../../biblios/main.bib}

\input{qm2pi.rhodetails}

\end{document}

 

% section notation (end)

\input{qm2pi.process.calculi} 

% section concurrent_process_calculi_and_spatial_logics_ (end)
    
%\documentclass[12pt]{llncs}
%\documentclass{jktr}

\usepackage[pdftex]{hyperref}                   
\usepackage {listings}
\usepackage {mathpartir}
\usepackage{bcprules}
%\usepackage{listings}
                       
\usepackage{graphicx} 
%\usepackage[margins=2.5cm,nohead,nofoot]{geometry}
%\usepackage{geometry}
\usepackage{amsfonts}
\usepackage{amstext}
\usepackage{latexsym}
\usepackage{amssymb}
\usepackage{color}


%\include{myPreamble}
\include{qm2pi.local} 

%\ifpdf
%\usepackage[pdftex]{graphicx}
%\else
%\usepackage{graphicx}
%\fi

 % \ifpdf
%  \usepackage{pdfsync}
%  \if


%\title{Brief Article}
%\author{David F. Snyder}
%\author{L.G. Meredith}

%\address{Dept. of Math., Texas State University--San Marcos, San Marcos, TX 78666}
       
\pagestyle{empty}


\begin{document}

\lstset{language=[Objective]Caml,frame=shadowbox}

\input{qm2pi.front}

% section front matter (end)

\input{qm2pi.intro} 
 
% section introduction (end)

% \input{qm2pi.knotations} 

% section notation (end)

\input{qm2pi.process.calculi} 

% section concurrent_process_calculi_and_spatial_logics_ (end)
    
%\input{qm2pi.knots2pi} 

%\input{qm2pi.trefoil} 

%\input{qm2pi.mainthm} 

% subsection basic_interpretation (end)

%\input{qm2pi.rho.presentation} 
\subsection{The syntax and semantics of the notation system}\label{sub:the_syntax_and_semantics_of_the_notation_system} % (fold)

We now summarize a technical presentation of the calculus that
embodies our theory of dynamics. The typical presentation of such a
calculus follows the style of giving generators and relations on
them. The grammar, below, describing term constructors, freely
generates the set of processes, $\Proc$. This set is then quotiented
by a relation known as structural congruence and it is over this set
that the notion of dynamics is expressed. This presentation is
essentially that of \cite{MeredithR05} with the addition of
polyadicity and summation. For readability we have relegated some of
the technical subtleties to an appendix.

\subsubsection{Process grammar}\label{subsub:process_grammar}

\begin{mathpar}
  \inferrule* [lab=synchronization] {} {{M} \bc \pzero \;|\; x?F \;|\; x!C }
  \and
  \inferrule* [lab=abstraction] {} {{F} \bc (x)P}
  \and
  \inferrule* [lab=concretion] {} {{C} \bc \langle Q \rangle}
  \and
  \inferrule* [lab=process] {} {{P,Q} \bc M \;| \;P|Q \;|\; @{x}}
  \and
  \inferrule* [lab=name] {} {{x} \bc \quotep{P}}
\end{mathpar} 

Note that $\vec{x}$ (resp. $\vec{P}$) denotes a vector of names
(resp. processes) of length $|\vec{x}|$ (resp. $|\vec{P}|$). We adopt
the following useful abbreviations.

\begin{mathpar}
   x?(\vec{y}).P := x.(\vec{y})P \and  x\clift{\vec{P}} := x.\clift{\vec{P}}
   \and x!(y) := \lift{x}{\dropn{y}}
   \and \Pi_{i=0}^{n-1}P_i := P_0 | \ldots | P_{n-1}
\end{mathpar}

\subsubsection{Structural congruence}

\paragraph{Free and bound names and alpha-equivalence.} At the
core of structural equivalence is alpha-equivalence which identifies
process that are the same up to a change of variable. Formally, we
recognize the distinction between free and bound names. The free names
of a process, $\freenames{P}$, may be calculated recursively as
follows:

\begin{mathpar}
\freenames{\pzero} := \emptyset
  \and \\
  \freenames{x?(y).P} := \{ x \} \cup (\freenames{P} \setminus \{ y \})
  \and 
  \freenames{x!\langle P \rangle} := \{ x \} \cup \{ P \} 
  \and \\
  \freenames{P|Q} := \freenames{P} \cup \freenames{Q}
  \and \\
  \freenames{@{x}} := \{ x \}
\end{mathpar}

$\pi$
$\quotep{\pi}$

$\freenames{-} : \pi \to \mathcal{P}(\quotep{\pi})$

\begin{eqnarray*}
  \freenames{\pzero} & := & \emptyset \\
  \freenames{x?(y).P} & := & \{ x \} \cup (\freenames{P} \setminus \{ y \}) \\
  \freenames{x!\langle P \rangle} & := & \{ x \} \cup \{ P \} \\
  \freenames{P|Q} & := & \freenames{P} \cup \freenames{Q} \\
  \freenames{\dropn{x}} & := & \{ x \}
\end{eqnarray*}

The bound names of a process, $\boundnames{P}$, are those names occurring in $P$
that are not free. For example, in $x?(y).0$, the name $x$ is free, while $y$ is bound.

\begin{mathpar}
  \inferrule* [lab=monoidal-laws] {} { P|Q \equiv Q|P \and P|0 \equiv P \and P|(Q|R) \equiv (P|Q)|R }
\end{mathpar}

\begin{mathpar}
  \inferrule* [lab=alpha-equivalence] {} { (x)P \equiv (y)P\{y/x\} \and y \not\in \freenames{P} }
\end{mathpar}

\begin{definition}
Then two processes, $P,Q$, are alpha-equivalent if $P = Q\{\vec{y}/\vec{x}\}$ for
some $\vec{x} \in \boundnames{Q},\vec{y} \in \boundnames{P}$, where $Q\{\vec{y}/\vec{x}\}$
denotes the capture-avoiding substitution of $\vec{y}$ for $\vec{x}$ in $Q$.
\end{definition}

\begin{definition}
  The {\em structural congruence} \cite{SangiorgiWalker} , $\equiv$,
  between processes is the least congruence containing
  alpha-equivalence, satisfying the abelian monoid laws
  (associativity, commutativity and $\pzero$ as identity) for parallel
  composition $|$ and for summation $+$.
\end{definition}

\subsection{Name equivalence}

We take name equivalence, written $\nameeq$, to be the smallest
equivalence relation generated by the following rules.

\begin{mathpar}
\inferrule*[lab=Quote-drop]
{ }
{ \quotep{@{x}} \nameeq x }

\inferrule*[lab=Struct-equiv]
{ P \scong Q }
{ \quotep{P} \nameeq \quotep{Q} }
\end{mathpar}

The astute reader will have noticed that the mutual recursion of names
and processes imposes a mutual recursion on alpha-equivalence and
structural equivalence via name-equivalence. Fortunately, all of this
works out pleasantly and we may calculate in the natural way, free of
concern. The reader interested in the details is referred to the
appendix \ref{appendix:rho_details}.

\subsection{Substitution}

We use $\Proc$ for the set of processes, $\QProc$ for the set of
names, and $\id{\{}\vec{y} / \vec{x} \id{\}}$ to denote partial maps,
$s : \QProc \rightarrow \QProc$. A map, $s$ lifts, uniquely, to a map
on process terms, $\widehat{s} : \Proc \rightarrow \Proc$ by the
following equations.

\begin{mathpar}
  (0) \psubstp{Q}{P} := 0 \\
  (R \juxtap S) \psubstp{Q}{P}
  :=    
  (R)\psubstp{Q}{P} \juxtap (S) \psubstp{Q}{P} \\
  (x?(y).R) \psubstp{Q}{P}    
  :=    
  (x)\substp{Q}{P} (z)\concat( (R \psubstn{z}{y}) \psubstp{Q}{P} ) \\
  (\lift{x}{R}) \psubstp{Q}{P}  
  :=
  \lift{(x)\substp{Q}{P}}{ R \psubstp{Q}{P} } \\
%   (\dropn{x})  \psubstp{Q}{P}       
%   := 
%   \left\{ 
%     \begin{array}{ccc} 
%       \dropn{\quotep{Q}} & & x \nameeq \quotep{P} \\
%       \dropn{x} & & otherwise \\
%     \end{array}
%   \right. 
  (\dropn{x})  \psubstp{Q}{P}       
  := 
  \left\{ 
    \begin{array}{ccc} 
      Q & & x \nameeq \quotep{P} \\
      \dropn{x} & & otherwise \\
    \end{array}
  \right.
\end{mathpar}
 

where

\begin{eqnarray}
  (x)\id{\{} \lpquote Q \rpquote / \lpquote P \rpquote \id{\}}            = 
  \left\{ 
    \begin{array}{ccc}
      \lpquote Q \rpquote & & x \nameeq \lpquote P \rpquote \\
      x & & otherwise \\
    \end{array}
  \right. \nonumber
\end{eqnarray}

and $z$ is chosen distinct from $\quotep{P}$, $\quotep{Q}$, the free
names in $Q$, and all the names in $R$. Our $\alpha$-equivalence will
be built in the standard way from this substitution.

\begin{remark}\label{rem:no_self_referential_names}
  One consequence of these definitions is that $\forall P. \quotep{P}
  \not\in \freenames{P}$.
\end{remark}

\subsection{ Dynamic quote: an example }

Anticipating something of what's to come, consider applying the
substitution, $\widehat{\id{\{}u / z \id{\}}}$, to the following pair
of processes, $\lift{w}{y!(z)}$ and $w[ \lpquote y!(z) \rpquote ]$.

\begin{eqnarray}
	\lift{w}{y!(z)}\widehat{\id{\{}u / z \id{\}}}
		& = &
		\lift{w}{y!(u)} \nonumber\\
	w[ \lpquote y!(z) \rpquote ] \widehat{ \id{\{}u / z \id{\}} }
		& = &
		w[ \lpquote y!(z) \rpquote ] \nonumber
\end{eqnarray}

Because the body of the process between quotes is impervious to
substitution, we get radically different answers. In fact, by
examining the first process in an input context,
e.g. $x?(z).\lift{w}{y!(z)}$, we see that the process under the lift
operator may be shaped by prefixed inputs binding a name inside it. In
this sense, the lift operator will be seen as a way to dynamically
construct processes before reifying them as names.

Finally equipped with these standard features we can present the
dynamics of the calculus.

\subsubsection{Operational semantics} 

Finally, we introduce the computational dynamics. What marks these
algebras as distinct from other more traditionally studied algebraic
structures, e.g. vector spaces or polynomial rings, is the manner in
which dynamics is captured. In traditional structures, dynamics is typically
expressed through morphisms between such structures, as in linear maps
between vector spaces or morphisms between rings. In algebras
associated with the semantics of computation, the dynamics is
expressed as part of the algebraic structure itself, through a
reduction reduction relation typically denoted by $\red$. Below, we
give a recursive presentation of this relation for the calculus used
in the encoding.

$\red \subseteq \pi \times \pi$
$\red : \pi \to \mathcal{P}(\pi)$

\begin{mathpar}
  \inferrule* [lab=Comm] { \textsf{match}( x_{src}, x_{trgt} ) } { x_{trgt}?(y)P \; | \; x_{src}!\langle {Q} \rangle \red P\{\quotep{Q}/y}\} }
  \and \\
  \inferrule* [lab=Par] {{P} \red {P}'} {{{P} | {Q}} \red {{P}' | {Q}}}
  \and
  \inferrule* [lab=Equiv]{{{P} \scong {P}'} \andalso {{P}' \red {Q}'} \andalso {{Q}' \scong {Q}}}{{P} \red {Q}}
\end{mathpar}

\begin{eqnarray*}
  match_{\equiv} (\quotep{P},\quotep{Q}) & := & P \equiv Q \\
  match_{\dagger}(\quotep{P},\quotep{Q}) & := & \forall R. P|Q \red^{*} R => R \red^{*} 0 \\
  match_{K}(\quotep{P},\quotep{Q}) & := & K \mbox{ for some context } K
\end{eqnarray*}

$u?(x)P | u!\langle Q \rangle \red P\{\quotep{Q}/x\}$

%We write $\wred$ for $\red^*$, and $P\red$ if $\exists Q $ such that $ P \red Q$.
We write $P\red$ if $\exists Q $ such that $ P \red Q$ and $P\not\red$, otherwise.

\section{Replication}

As mentioned before, it is known that replication (and hence
recursion) can be implemented in a higher-order process algebra
\cite{SangiorgiWalker}. As our first example of calculation with the
machinery thus far presented we give the construction explicitly in
the {\rhoc}.

\begin{eqnarray}
	D_{x} & := & \prefix{x}{y}{(\binpar{\outputp{x}{y}}{@{y}})} \nonumber\\
	\bangp_{x}{P} & := & \binpar{{x}!\langle{\binpar{D_{x}}{P}}\rangle}{D_{x}} \nonumber
\end{eqnarray}

\begin{eqnarray}
	\bangp_{x}{P} & & \nonumber\\
	=
	& {x}!\langle{(\prefix{x}{y}{(\outputp{x}{y} | @{y})) | P}}\rangle 
	      | \prefix{x}{y}{(\outputp{x}{y} | @{y})} & \nonumber\\
	\red
	& (\outputp{x}{y} | @{y})\substn{\quotep{(\prefix{x}{y}{(@{y} | \outputp{x}{y})) | P}}}{y} & \nonumber\\
	=
	& \outputp{x}{\quotep{(\prefix{x}{y}{(\outputp{x}{y} | @{y})) | P}}}
	  | {(\prefix{x}{y}{(\outputp{x}{y} | @{y})) | P}} & \nonumber\\
	\red
	& \ldots & \nonumber\\
	\red^*
	& P | P | \ldots & \nonumber
\end{eqnarray}

Of course, this encoding, as an implementation, runs away, unfolding
$\bangp{P}$ eagerly. A lazier and more implementable replication
operator, restricted to input-guarded processes, may be obtained as follows.

\begin{eqnarray}
\bangp{\prefix{u}{v}{P}} 
	:= 
	\binpar{\lift{x}{\prefix{u}{v}{(\binpar{D(x)}{P})}}}{D(x)} \nonumber
\end{eqnarray}

\begin{remark}
  Note that the lazier definition still does not deal with summation
  or mixed summation (i.e. sums over input and output). The reader is
  invited to construct definitions of replication that deal with these
  features. 

  Further, the definitions are parameterized in a name, $x$. Can you,
  gentle reader, make a definition that eliminates this parameter and
  guarantees no accidental interaction between the replication
  machinery and the process being replicated -- i.e. no accidental
  sharing of names used by the process to get its work done and the
  name(s) used by the replication to effect copying. This latter
  revision of the definition of replication is crucial to obtaining
  the expected identity $!!P \sim !P$.
\end{remark}

\begin{remark}\label{rem:paradoxical_combinator}
  The reader familiar with the lambda calculus will have noticed the
  similarity between $D$ and the paradoxical combinator.

  [Ed. note: the existence of this seems to suggest we have to be more
  restrictive on the set of processes and names we admit if we are to
  support no-cloning.]
\end{remark}

\subsubsection{Bisimulation}

The computational dynamics gives rise to another kind of equivalence,
the equivalence of computational behavior. As previously mentioned
this is typically captured \emph{via} some form of bisimulation.

% The notion we use in this paper is weak barbed bisimulation
% \cite{milner91polyadicpi}.

The notion we use in this paper is derived from weak barbed
bisimulation \cite{milner91polyadicpi}. 

\begin{definition}
An \emph{observation relation}, $\downarrow_{\mathcal N}$, over a set
of names, $\mathcal N$, is the smallest relation satisfying the rules
below.

\infrule[Out-barb]{y \in {\mathcal N}, \; x \nameeq y}
		  {\outputp{x}{v} \downarrow_{\mathcal N} x}
\infrule[Par-barb]{\mbox{$P\downarrow_{\mathcal N} x$ or $Q\downarrow_{\mathcal N} x$}}
		  {\binpar{P}{Q} \downarrow_{\mathcal N} x}

We write $P \Downarrow_{\mathcal N} x$ if there is $Q$ such that 
$P \wred Q$ and $Q \downarrow_{\mathcal N} x$.
\end{definition}

\begin{definition}
%\label{def.bbisim}
An  ${\mathcal N}$-\emph{barbed bisimulation} over a set of names, ${\mathcal N}$, is a symmetric binary relation 
${\mathcal S}_{\mathcal N}$ between agents such that $P\rel{S}_{\mathcal N}Q$ implies:
\begin{enumerate}
\item If $P \red P'$ then $Q \wred Q'$ and $P'\rel{S}_{\mathcal N} Q'$.
\item If $P\downarrow_{\mathcal N} x$, then $Q\Downarrow_{\mathcal N} x$.
\end{enumerate}
$P$ is ${\mathcal N}$-barbed bisimilar to $Q$, written
$P \wbbisim_{\mathcal N} Q$, if $P \rel{S}_{\mathcal N} Q$ for some ${\mathcal N}$-barbed bisimulation ${\mathcal S}_{\mathcal N}$.
\end{definition}

$\mathcal{R} \subseteq \pi \times \pi$

$P \mathcal{R} Q => \forall P'. P \red P' \Rightarrow \exists Q'. Q \red Q', P' \mathcal{R} Q'$

$P \vdash x \Rightarrow Q \vdash x$

\begin{mathpar}
  \inferrule*[lab=Out-barb]{x \nameeq y}{{y}!\langle{Q}\rangle \vdash x}
  \and
  \inferrule*[lab=Par-barb]{\mbox{$P\vdash x$ or $Q\vdash x$}}{\binpar{P}{Q} \vdash x}
\end{mathpar}

\subsubsection{Contexts}

One of the principle advantages of computational calculi like the
$\pi$-calculus is a well-defined notion of context,
contextual-equivalence and a correlation between
contextual-equivalence and notions of bisimulation. The notion of
context allows the decomposition of a process into (sub-)process and
its syntactic environment, its context. Thus, a context may be
thought of as a process with a ``hole'' (written $\Box$) in it. The
application of a context $M$ to a process $P$, written $M[P]$, is
tantamount to filling the hole in $M$ with $P$. In this paper we do
not need the full weight of this theory, but do make use of the notion
of context in the proof the main theorem. 

\begin{mathpar}
  \inferrule* [lab=summation] {} {{M_{M},M_{N}} \bc \Box \;|\; x.M_{A} \;|\; M_{M}+M_{N}}
  \and
  \inferrule* [lab=agent] {} {{M_{A}} \bc (\vec{x})M_{P} \;| \; \clift{P_0,\ldots,M_{P},\ldots,P_N}}
  \and \\
  \inferrule* [lab=process] {} {{M_{P}} \bc M_{N} \;| \;P|M_{P} }
\end{mathpar} 

\begin{mathpar}
  \inferrule* [lab=sychronization] {} {M_{N} \bc \Box \;|\; x?M_{F} \;|\; x!M_{C}}
  \and
  \inferrule* [lab=abstraction] {} {{M_{F}} \bc (x)M_{P} }
  \and
  \inferrule* [lab=concretion] {} {{M_{C}} \bc \langle M_{P} \rangle }
  \and \\
  \inferrule* [lab=process] {} {{M_{P}} \bc M_{N} \;| \;P|M_{P} }
\end{mathpar}

\begin{definition}[contextual application] Given a context $M$, and
  process $P$, we define the \emph{contextual application}, $M[P] :=
  M\{P/\Box\}$. That is, the contextual application of M to P is the
  substitution of $P$ for $\Box$ in $M$.
\end{definition}

$\meaningof{-} : L \to \mathcal{P}(\pi)$

\begin{mathpar}
  \inferrule* [lab=collection] {} {\meaningof{true} = \pi, \and \meaningof{~E} = \pi \setminus \meaningof{E}, \and \meaningof{E_{1} \& E_{2}} = \meaningof{E_{1}} \cap \meaningof{E_{2}}}
\end{mathpar}

\begin{mathpar}
  \inferrule* [lab=structure] {} {\meaningof{0} = \{ P \in \pi | P \equiv 0 \}, \and \\ \meaningof{E_1 | E_2} = \{ P \in \pi | P \equiv P_{1} | P_{2}, P_{1} \in \meaningof{E_{1}}, P_{2} \in \meaningof{E_2}\} }
\end{mathpar}

\begin{mathpar}
 \inferrule* [lab=behavior] {} {\meaningof{\langle a?b \rangle E} = \{ P \in \pi | P \equiv Q | u?(y)P', \\ \and \\\\ \and \\ \;\;\; u \in \meaningof{a}, \forall z.P'\{z/y\} \in \meaningof{E\{z/b\}}\}, \and \\ \meaningof{a!E} = \{ P \in \pi | P \equiv Q | x!\langle P' \rangle, x \in \meaningof{a} P' \in \meaningof{E}\} }
\end{mathpar}

\begin{mathpar}
 \inferrule* [lab=nominal] {} {\meaningof{\quotep{E}} = \{ \quotep{P} \in \quotep{\pi} | P \in \meaningof{E} \}, \and \meaningof{\quotep{P}} = \{ \quotep{Q} \in \quotep{\pi} | P \equiv Q \} \and \\ \meaningof{@\quotep{E}} = \{ P \in \pi | P \equiv @x, x \in \meaningof{E} \}}
\end{mathpar}

\begin{eqnarray*}
  \\
  \meaningof{-} : TS \to ST
\end{eqnarray*}

\begin{eqnarray*}
  \\
  L : TS \to ST
\end{eqnarray*}

\begin{eqnarray*}
  \\
  P \models E \iff P \in \meaningof{E}
\end{eqnarray*}

\begin{eqnarray*}
  P \approx_{L} Q \iff \forall E \in L. P \models E \iff Q \models E
\end{eqnarray*}

\begin{eqnarray*}
  P \approx_{K} Q
\end{eqnarray*}

\begin{eqnarray*}
  P \approx Q
\end{eqnarray*}

$\approx_{K} = \approx = \approx_{L}$

\subsubsection{Contextual duality}

Note that contexts extend the quotation operation to a family of
operations from processes to names. Given a context, $M$, we can
define a \emph{nominal context}, $\quotep{M}$ by $\quotep{M}[P] :=
\quotep{M[P]}$. To foreshadow what is to come we observe that these
operations enjoy a duality with processes very much like the duality
between vectors and maps from vectors to scalars.

Further, because the calculus is essentially higher-order, we have a
correspondence between contexts and processes. More specifically,
given a name $x$ and a context $M$ we can construct $M^{*}_{x}$ such
that 

\begin{mathpar}
  M^{*}_{x} | \lift{x}{P} \red M[P]
\end{mathpar}

namely,

\begin{mathpar}
  M^{*}_{x} := x?(u).M[\dropn{u}]
\end{mathpar}

The dependence of $M^{*}_{x}$ on a name makes it an abstraction, 

\begin{mathpar}
  M^{*} := (x)x?(u).M[\dropn{u}]
\end{mathpar}

\subsection{Additional notation}

It will sometimes be convenient to denote the process a name
quotes. We already have the notation $x = \quotep{P}$, but it will be
convenient to introduce an alternate notation, $\procn{x}$, when we
want to emphasize the connection to the use of the name. Note that, by
virtue of name equivalence, $\quotep{\procn{x}} \nameeq x$; so, the
notation is consistent with previous definitions.

Further, because names have structure it is possible to effect
substitutions on the basis of that structure. This means we need to
upgrade our notation for substitutions, which we accomplish by
adapting comprehension notation. Thus,

\begin{mathpar}
  P\{ y / x : x \in S \}
\end{mathpar}

is interpreted to mean the process derived from P by replacing (in a
capture-avoiding manner) each occurrence of $x$ in $S$ by $y$. For example,

\begin{mathpar}
  P\{ \quotep{\procn{x}|\procn{x}} / x : x \in \freenames{P} \}
\end{mathpar}

will replace each (occurrence) of a free name $x$ in $P$ by
$\quotep{\procn{x}|\procn{x}}$.

Also, we will avail ourselves of the notation $x^{L}$ and $x^{R}$ to
denote injections of a name into disjoint copies of the name
space. There are numerous ways to accomplish this. One example can be
found in \cite{MeredithR05}. This notation overloads to vectors of
names: $\vec{x}^{\pi} := (x_{i}^{\pi} \; : \; 0 \leq i < |\vec{x}| )$ where $\pi \in \{L,R\}$.

We also use $P^{\Box} := P|\Box$.

In \cite{MeredithR05} an interpretation of the new operator is
given. It turns out that there are several possible interpretations
all enjoying the requisite algebraic properties of the operator (see
\cite{milner91polyadicpi}). We will therefore make liberal use of
$(\nu\; \vec{x})P$.

% subsection the_syntax_and_semantics_of_the_notation_system (end)   

\input{qm2pi.qmops} 

\input{qm2pi.sterngerlach} 

\input{qm2pi.metric} 

% section concurrent_process_calculi (end)

%\input{qm2pi.proofsketch}

% section proof sketch (end)

%\input{qm2pi.slviaknots} 

% section spatial logic via knots (end)

\input{qm2pi.conclusion}

% section conclusion (end)

%\input{qm2pi.dtcodes} 

% section wiring algorithm (end)

\input{qm2pi.ack} 

% section acknowledgments (end)

\newpage


\bibliographystyle{plain}   
\bibliography{../../biblios/main.bib}

\input{qm2pi.rhodetails}

\end{document}

 

%\documentclass[12pt]{llncs}
%\documentclass{jktr}

\usepackage[pdftex]{hyperref}                   
\usepackage {listings}
\usepackage {mathpartir}
\usepackage{bcprules}
%\usepackage{listings}
                       
\usepackage{graphicx} 
%\usepackage[margins=2.5cm,nohead,nofoot]{geometry}
%\usepackage{geometry}
\usepackage{amsfonts}
\usepackage{amstext}
\usepackage{latexsym}
\usepackage{amssymb}
\usepackage{color}


%\include{myPreamble}
\include{qm2pi.local} 

%\ifpdf
%\usepackage[pdftex]{graphicx}
%\else
%\usepackage{graphicx}
%\fi

 % \ifpdf
%  \usepackage{pdfsync}
%  \if


%\title{Brief Article}
%\author{David F. Snyder}
%\author{L.G. Meredith}

%\address{Dept. of Math., Texas State University--San Marcos, San Marcos, TX 78666}
       
\pagestyle{empty}


\begin{document}

\lstset{language=[Objective]Caml,frame=shadowbox}

\input{qm2pi.front}

% section front matter (end)

\input{qm2pi.intro} 
 
% section introduction (end)

% \input{qm2pi.knotations} 

% section notation (end)

\input{qm2pi.process.calculi} 

% section concurrent_process_calculi_and_spatial_logics_ (end)
    
%\input{qm2pi.knots2pi} 

%\input{qm2pi.trefoil} 

%\input{qm2pi.mainthm} 

% subsection basic_interpretation (end)

%\input{qm2pi.rho.presentation} 
\subsection{The syntax and semantics of the notation system}\label{sub:the_syntax_and_semantics_of_the_notation_system} % (fold)

We now summarize a technical presentation of the calculus that
embodies our theory of dynamics. The typical presentation of such a
calculus follows the style of giving generators and relations on
them. The grammar, below, describing term constructors, freely
generates the set of processes, $\Proc$. This set is then quotiented
by a relation known as structural congruence and it is over this set
that the notion of dynamics is expressed. This presentation is
essentially that of \cite{MeredithR05} with the addition of
polyadicity and summation. For readability we have relegated some of
the technical subtleties to an appendix.

\subsubsection{Process grammar}\label{subsub:process_grammar}

\begin{mathpar}
  \inferrule* [lab=synchronization] {} {{M} \bc \pzero \;|\; x?F \;|\; x!C }
  \and
  \inferrule* [lab=abstraction] {} {{F} \bc (x)P}
  \and
  \inferrule* [lab=concretion] {} {{C} \bc \langle Q \rangle}
  \and
  \inferrule* [lab=process] {} {{P,Q} \bc M \;| \;P|Q \;|\; @{x}}
  \and
  \inferrule* [lab=name] {} {{x} \bc \quotep{P}}
\end{mathpar} 

Note that $\vec{x}$ (resp. $\vec{P}$) denotes a vector of names
(resp. processes) of length $|\vec{x}|$ (resp. $|\vec{P}|$). We adopt
the following useful abbreviations.

\begin{mathpar}
   x?(\vec{y}).P := x.(\vec{y})P \and  x\clift{\vec{P}} := x.\clift{\vec{P}}
   \and x!(y) := \lift{x}{\dropn{y}}
   \and \Pi_{i=0}^{n-1}P_i := P_0 | \ldots | P_{n-1}
\end{mathpar}

\subsubsection{Structural congruence}

\paragraph{Free and bound names and alpha-equivalence.} At the
core of structural equivalence is alpha-equivalence which identifies
process that are the same up to a change of variable. Formally, we
recognize the distinction between free and bound names. The free names
of a process, $\freenames{P}$, may be calculated recursively as
follows:

\begin{mathpar}
\freenames{\pzero} := \emptyset
  \and \\
  \freenames{x?(y).P} := \{ x \} \cup (\freenames{P} \setminus \{ y \})
  \and 
  \freenames{x!\langle P \rangle} := \{ x \} \cup \{ P \} 
  \and \\
  \freenames{P|Q} := \freenames{P} \cup \freenames{Q}
  \and \\
  \freenames{@{x}} := \{ x \}
\end{mathpar}

$\pi$
$\quotep{\pi}$

$\freenames{-} : \pi \to \mathcal{P}(\quotep{\pi})$

\begin{eqnarray*}
  \freenames{\pzero} & := & \emptyset \\
  \freenames{x?(y).P} & := & \{ x \} \cup (\freenames{P} \setminus \{ y \}) \\
  \freenames{x!\langle P \rangle} & := & \{ x \} \cup \{ P \} \\
  \freenames{P|Q} & := & \freenames{P} \cup \freenames{Q} \\
  \freenames{\dropn{x}} & := & \{ x \}
\end{eqnarray*}

The bound names of a process, $\boundnames{P}$, are those names occurring in $P$
that are not free. For example, in $x?(y).0$, the name $x$ is free, while $y$ is bound.

\begin{mathpar}
  \inferrule* [lab=monoidal-laws] {} { P|Q \equiv Q|P \and P|0 \equiv P \and P|(Q|R) \equiv (P|Q)|R }
\end{mathpar}

\begin{mathpar}
  \inferrule* [lab=alpha-equivalence] {} { (x)P \equiv (y)P\{y/x\} \and y \not\in \freenames{P} }
\end{mathpar}

\begin{definition}
Then two processes, $P,Q$, are alpha-equivalent if $P = Q\{\vec{y}/\vec{x}\}$ for
some $\vec{x} \in \boundnames{Q},\vec{y} \in \boundnames{P}$, where $Q\{\vec{y}/\vec{x}\}$
denotes the capture-avoiding substitution of $\vec{y}$ for $\vec{x}$ in $Q$.
\end{definition}

\begin{definition}
  The {\em structural congruence} \cite{SangiorgiWalker} , $\equiv$,
  between processes is the least congruence containing
  alpha-equivalence, satisfying the abelian monoid laws
  (associativity, commutativity and $\pzero$ as identity) for parallel
  composition $|$ and for summation $+$.
\end{definition}

\subsection{Name equivalence}

We take name equivalence, written $\nameeq$, to be the smallest
equivalence relation generated by the following rules.

\begin{mathpar}
\inferrule*[lab=Quote-drop]
{ }
{ \quotep{@{x}} \nameeq x }

\inferrule*[lab=Struct-equiv]
{ P \scong Q }
{ \quotep{P} \nameeq \quotep{Q} }
\end{mathpar}

The astute reader will have noticed that the mutual recursion of names
and processes imposes a mutual recursion on alpha-equivalence and
structural equivalence via name-equivalence. Fortunately, all of this
works out pleasantly and we may calculate in the natural way, free of
concern. The reader interested in the details is referred to the
appendix \ref{appendix:rho_details}.

\subsection{Substitution}

We use $\Proc$ for the set of processes, $\QProc$ for the set of
names, and $\id{\{}\vec{y} / \vec{x} \id{\}}$ to denote partial maps,
$s : \QProc \rightarrow \QProc$. A map, $s$ lifts, uniquely, to a map
on process terms, $\widehat{s} : \Proc \rightarrow \Proc$ by the
following equations.

\begin{mathpar}
  (0) \psubstp{Q}{P} := 0 \\
  (R \juxtap S) \psubstp{Q}{P}
  :=    
  (R)\psubstp{Q}{P} \juxtap (S) \psubstp{Q}{P} \\
  (x?(y).R) \psubstp{Q}{P}    
  :=    
  (x)\substp{Q}{P} (z)\concat( (R \psubstn{z}{y}) \psubstp{Q}{P} ) \\
  (\lift{x}{R}) \psubstp{Q}{P}  
  :=
  \lift{(x)\substp{Q}{P}}{ R \psubstp{Q}{P} } \\
%   (\dropn{x})  \psubstp{Q}{P}       
%   := 
%   \left\{ 
%     \begin{array}{ccc} 
%       \dropn{\quotep{Q}} & & x \nameeq \quotep{P} \\
%       \dropn{x} & & otherwise \\
%     \end{array}
%   \right. 
  (\dropn{x})  \psubstp{Q}{P}       
  := 
  \left\{ 
    \begin{array}{ccc} 
      Q & & x \nameeq \quotep{P} \\
      \dropn{x} & & otherwise \\
    \end{array}
  \right.
\end{mathpar}
 

where

\begin{eqnarray}
  (x)\id{\{} \lpquote Q \rpquote / \lpquote P \rpquote \id{\}}            = 
  \left\{ 
    \begin{array}{ccc}
      \lpquote Q \rpquote & & x \nameeq \lpquote P \rpquote \\
      x & & otherwise \\
    \end{array}
  \right. \nonumber
\end{eqnarray}

and $z$ is chosen distinct from $\quotep{P}$, $\quotep{Q}$, the free
names in $Q$, and all the names in $R$. Our $\alpha$-equivalence will
be built in the standard way from this substitution.

\begin{remark}\label{rem:no_self_referential_names}
  One consequence of these definitions is that $\forall P. \quotep{P}
  \not\in \freenames{P}$.
\end{remark}

\subsection{ Dynamic quote: an example }

Anticipating something of what's to come, consider applying the
substitution, $\widehat{\id{\{}u / z \id{\}}}$, to the following pair
of processes, $\lift{w}{y!(z)}$ and $w[ \lpquote y!(z) \rpquote ]$.

\begin{eqnarray}
	\lift{w}{y!(z)}\widehat{\id{\{}u / z \id{\}}}
		& = &
		\lift{w}{y!(u)} \nonumber\\
	w[ \lpquote y!(z) \rpquote ] \widehat{ \id{\{}u / z \id{\}} }
		& = &
		w[ \lpquote y!(z) \rpquote ] \nonumber
\end{eqnarray}

Because the body of the process between quotes is impervious to
substitution, we get radically different answers. In fact, by
examining the first process in an input context,
e.g. $x?(z).\lift{w}{y!(z)}$, we see that the process under the lift
operator may be shaped by prefixed inputs binding a name inside it. In
this sense, the lift operator will be seen as a way to dynamically
construct processes before reifying them as names.

Finally equipped with these standard features we can present the
dynamics of the calculus.

\subsubsection{Operational semantics} 

Finally, we introduce the computational dynamics. What marks these
algebras as distinct from other more traditionally studied algebraic
structures, e.g. vector spaces or polynomial rings, is the manner in
which dynamics is captured. In traditional structures, dynamics is typically
expressed through morphisms between such structures, as in linear maps
between vector spaces or morphisms between rings. In algebras
associated with the semantics of computation, the dynamics is
expressed as part of the algebraic structure itself, through a
reduction reduction relation typically denoted by $\red$. Below, we
give a recursive presentation of this relation for the calculus used
in the encoding.

$\red \subseteq \pi \times \pi$
$\red : \pi \to \mathcal{P}(\pi)$

\begin{mathpar}
  \inferrule* [lab=Comm] { \textsf{match}( x_{src}, x_{trgt} ) } { x_{trgt}?(y)P \; | \; x_{src}!\langle {Q} \rangle \red P\{\quotep{Q}/y}\} }
  \and \\
  \inferrule* [lab=Par] {{P} \red {P}'} {{{P} | {Q}} \red {{P}' | {Q}}}
  \and
  \inferrule* [lab=Equiv]{{{P} \scong {P}'} \andalso {{P}' \red {Q}'} \andalso {{Q}' \scong {Q}}}{{P} \red {Q}}
\end{mathpar}

\begin{eqnarray*}
  match_{\equiv} (\quotep{P},\quotep{Q}) & := & P \equiv Q \\
  match_{\dagger}(\quotep{P},\quotep{Q}) & := & \forall R. P|Q \red^{*} R => R \red^{*} 0 \\
  match_{K}(\quotep{P},\quotep{Q}) & := & K \mbox{ for some context } K
\end{eqnarray*}

$u?(x)P | u!\langle Q \rangle \red P\{\quotep{Q}/x\}$

%We write $\wred$ for $\red^*$, and $P\red$ if $\exists Q $ such that $ P \red Q$.
We write $P\red$ if $\exists Q $ such that $ P \red Q$ and $P\not\red$, otherwise.

\section{Replication}

As mentioned before, it is known that replication (and hence
recursion) can be implemented in a higher-order process algebra
\cite{SangiorgiWalker}. As our first example of calculation with the
machinery thus far presented we give the construction explicitly in
the {\rhoc}.

\begin{eqnarray}
	D_{x} & := & \prefix{x}{y}{(\binpar{\outputp{x}{y}}{@{y}})} \nonumber\\
	\bangp_{x}{P} & := & \binpar{{x}!\langle{\binpar{D_{x}}{P}}\rangle}{D_{x}} \nonumber
\end{eqnarray}

\begin{eqnarray}
	\bangp_{x}{P} & & \nonumber\\
	=
	& {x}!\langle{(\prefix{x}{y}{(\outputp{x}{y} | @{y})) | P}}\rangle 
	      | \prefix{x}{y}{(\outputp{x}{y} | @{y})} & \nonumber\\
	\red
	& (\outputp{x}{y} | @{y})\substn{\quotep{(\prefix{x}{y}{(@{y} | \outputp{x}{y})) | P}}}{y} & \nonumber\\
	=
	& \outputp{x}{\quotep{(\prefix{x}{y}{(\outputp{x}{y} | @{y})) | P}}}
	  | {(\prefix{x}{y}{(\outputp{x}{y} | @{y})) | P}} & \nonumber\\
	\red
	& \ldots & \nonumber\\
	\red^*
	& P | P | \ldots & \nonumber
\end{eqnarray}

Of course, this encoding, as an implementation, runs away, unfolding
$\bangp{P}$ eagerly. A lazier and more implementable replication
operator, restricted to input-guarded processes, may be obtained as follows.

\begin{eqnarray}
\bangp{\prefix{u}{v}{P}} 
	:= 
	\binpar{\lift{x}{\prefix{u}{v}{(\binpar{D(x)}{P})}}}{D(x)} \nonumber
\end{eqnarray}

\begin{remark}
  Note that the lazier definition still does not deal with summation
  or mixed summation (i.e. sums over input and output). The reader is
  invited to construct definitions of replication that deal with these
  features. 

  Further, the definitions are parameterized in a name, $x$. Can you,
  gentle reader, make a definition that eliminates this parameter and
  guarantees no accidental interaction between the replication
  machinery and the process being replicated -- i.e. no accidental
  sharing of names used by the process to get its work done and the
  name(s) used by the replication to effect copying. This latter
  revision of the definition of replication is crucial to obtaining
  the expected identity $!!P \sim !P$.
\end{remark}

\begin{remark}\label{rem:paradoxical_combinator}
  The reader familiar with the lambda calculus will have noticed the
  similarity between $D$ and the paradoxical combinator.

  [Ed. note: the existence of this seems to suggest we have to be more
  restrictive on the set of processes and names we admit if we are to
  support no-cloning.]
\end{remark}

\subsubsection{Bisimulation}

The computational dynamics gives rise to another kind of equivalence,
the equivalence of computational behavior. As previously mentioned
this is typically captured \emph{via} some form of bisimulation.

% The notion we use in this paper is weak barbed bisimulation
% \cite{milner91polyadicpi}.

The notion we use in this paper is derived from weak barbed
bisimulation \cite{milner91polyadicpi}. 

\begin{definition}
An \emph{observation relation}, $\downarrow_{\mathcal N}$, over a set
of names, $\mathcal N$, is the smallest relation satisfying the rules
below.

\infrule[Out-barb]{y \in {\mathcal N}, \; x \nameeq y}
		  {\outputp{x}{v} \downarrow_{\mathcal N} x}
\infrule[Par-barb]{\mbox{$P\downarrow_{\mathcal N} x$ or $Q\downarrow_{\mathcal N} x$}}
		  {\binpar{P}{Q} \downarrow_{\mathcal N} x}

We write $P \Downarrow_{\mathcal N} x$ if there is $Q$ such that 
$P \wred Q$ and $Q \downarrow_{\mathcal N} x$.
\end{definition}

\begin{definition}
%\label{def.bbisim}
An  ${\mathcal N}$-\emph{barbed bisimulation} over a set of names, ${\mathcal N}$, is a symmetric binary relation 
${\mathcal S}_{\mathcal N}$ between agents such that $P\rel{S}_{\mathcal N}Q$ implies:
\begin{enumerate}
\item If $P \red P'$ then $Q \wred Q'$ and $P'\rel{S}_{\mathcal N} Q'$.
\item If $P\downarrow_{\mathcal N} x$, then $Q\Downarrow_{\mathcal N} x$.
\end{enumerate}
$P$ is ${\mathcal N}$-barbed bisimilar to $Q$, written
$P \wbbisim_{\mathcal N} Q$, if $P \rel{S}_{\mathcal N} Q$ for some ${\mathcal N}$-barbed bisimulation ${\mathcal S}_{\mathcal N}$.
\end{definition}

$\mathcal{R} \subseteq \pi \times \pi$

$P \mathcal{R} Q => \forall P'. P \red P' \Rightarrow \exists Q'. Q \red Q', P' \mathcal{R} Q'$

$P \vdash x \Rightarrow Q \vdash x$

\begin{mathpar}
  \inferrule*[lab=Out-barb]{x \nameeq y}{{y}!\langle{Q}\rangle \vdash x}
  \and
  \inferrule*[lab=Par-barb]{\mbox{$P\vdash x$ or $Q\vdash x$}}{\binpar{P}{Q} \vdash x}
\end{mathpar}

\subsubsection{Contexts}

One of the principle advantages of computational calculi like the
$\pi$-calculus is a well-defined notion of context,
contextual-equivalence and a correlation between
contextual-equivalence and notions of bisimulation. The notion of
context allows the decomposition of a process into (sub-)process and
its syntactic environment, its context. Thus, a context may be
thought of as a process with a ``hole'' (written $\Box$) in it. The
application of a context $M$ to a process $P$, written $M[P]$, is
tantamount to filling the hole in $M$ with $P$. In this paper we do
not need the full weight of this theory, but do make use of the notion
of context in the proof the main theorem. 

\begin{mathpar}
  \inferrule* [lab=summation] {} {{M_{M},M_{N}} \bc \Box \;|\; x.M_{A} \;|\; M_{M}+M_{N}}
  \and
  \inferrule* [lab=agent] {} {{M_{A}} \bc (\vec{x})M_{P} \;| \; \clift{P_0,\ldots,M_{P},\ldots,P_N}}
  \and \\
  \inferrule* [lab=process] {} {{M_{P}} \bc M_{N} \;| \;P|M_{P} }
\end{mathpar} 

\begin{mathpar}
  \inferrule* [lab=sychronization] {} {M_{N} \bc \Box \;|\; x?M_{F} \;|\; x!M_{C}}
  \and
  \inferrule* [lab=abstraction] {} {{M_{F}} \bc (x)M_{P} }
  \and
  \inferrule* [lab=concretion] {} {{M_{C}} \bc \langle M_{P} \rangle }
  \and \\
  \inferrule* [lab=process] {} {{M_{P}} \bc M_{N} \;| \;P|M_{P} }
\end{mathpar}

\begin{definition}[contextual application] Given a context $M$, and
  process $P$, we define the \emph{contextual application}, $M[P] :=
  M\{P/\Box\}$. That is, the contextual application of M to P is the
  substitution of $P$ for $\Box$ in $M$.
\end{definition}

$\meaningof{-} : L \to \mathcal{P}(\pi)$

\begin{mathpar}
  \inferrule* [lab=collection] {} {\meaningof{true} = \pi, \and \meaningof{~E} = \pi \setminus \meaningof{E}, \and \meaningof{E_{1} \& E_{2}} = \meaningof{E_{1}} \cap \meaningof{E_{2}}}
\end{mathpar}

\begin{mathpar}
  \inferrule* [lab=structure] {} {\meaningof{0} = \{ P \in \pi | P \equiv 0 \}, \and \\ \meaningof{E_1 | E_2} = \{ P \in \pi | P \equiv P_{1} | P_{2}, P_{1} \in \meaningof{E_{1}}, P_{2} \in \meaningof{E_2}\} }
\end{mathpar}

\begin{mathpar}
 \inferrule* [lab=behavior] {} {\meaningof{\langle a?b \rangle E} = \{ P \in \pi | P \equiv Q | u?(y)P', \\ \and \\\\ \and \\ \;\;\; u \in \meaningof{a}, \forall z.P'\{z/y\} \in \meaningof{E\{z/b\}}\}, \and \\ \meaningof{a!E} = \{ P \in \pi | P \equiv Q | x!\langle P' \rangle, x \in \meaningof{a} P' \in \meaningof{E}\} }
\end{mathpar}

\begin{mathpar}
 \inferrule* [lab=nominal] {} {\meaningof{\quotep{E}} = \{ \quotep{P} \in \quotep{\pi} | P \in \meaningof{E} \}, \and \meaningof{\quotep{P}} = \{ \quotep{Q} \in \quotep{\pi} | P \equiv Q \} \and \\ \meaningof{@\quotep{E}} = \{ P \in \pi | P \equiv @x, x \in \meaningof{E} \}}
\end{mathpar}

\begin{eqnarray*}
  \\
  \meaningof{-} : TS \to ST
\end{eqnarray*}

\begin{eqnarray*}
  \\
  L : TS \to ST
\end{eqnarray*}

\begin{eqnarray*}
  \\
  P \models E \iff P \in \meaningof{E}
\end{eqnarray*}

\begin{eqnarray*}
  P \approx_{L} Q \iff \forall E \in L. P \models E \iff Q \models E
\end{eqnarray*}

\begin{eqnarray*}
  P \approx_{K} Q
\end{eqnarray*}

\begin{eqnarray*}
  P \approx Q
\end{eqnarray*}

$\approx_{K} = \approx = \approx_{L}$

\subsubsection{Contextual duality}

Note that contexts extend the quotation operation to a family of
operations from processes to names. Given a context, $M$, we can
define a \emph{nominal context}, $\quotep{M}$ by $\quotep{M}[P] :=
\quotep{M[P]}$. To foreshadow what is to come we observe that these
operations enjoy a duality with processes very much like the duality
between vectors and maps from vectors to scalars.

Further, because the calculus is essentially higher-order, we have a
correspondence between contexts and processes. More specifically,
given a name $x$ and a context $M$ we can construct $M^{*}_{x}$ such
that 

\begin{mathpar}
  M^{*}_{x} | \lift{x}{P} \red M[P]
\end{mathpar}

namely,

\begin{mathpar}
  M^{*}_{x} := x?(u).M[\dropn{u}]
\end{mathpar}

The dependence of $M^{*}_{x}$ on a name makes it an abstraction, 

\begin{mathpar}
  M^{*} := (x)x?(u).M[\dropn{u}]
\end{mathpar}

\subsection{Additional notation}

It will sometimes be convenient to denote the process a name
quotes. We already have the notation $x = \quotep{P}$, but it will be
convenient to introduce an alternate notation, $\procn{x}$, when we
want to emphasize the connection to the use of the name. Note that, by
virtue of name equivalence, $\quotep{\procn{x}} \nameeq x$; so, the
notation is consistent with previous definitions.

Further, because names have structure it is possible to effect
substitutions on the basis of that structure. This means we need to
upgrade our notation for substitutions, which we accomplish by
adapting comprehension notation. Thus,

\begin{mathpar}
  P\{ y / x : x \in S \}
\end{mathpar}

is interpreted to mean the process derived from P by replacing (in a
capture-avoiding manner) each occurrence of $x$ in $S$ by $y$. For example,

\begin{mathpar}
  P\{ \quotep{\procn{x}|\procn{x}} / x : x \in \freenames{P} \}
\end{mathpar}

will replace each (occurrence) of a free name $x$ in $P$ by
$\quotep{\procn{x}|\procn{x}}$.

Also, we will avail ourselves of the notation $x^{L}$ and $x^{R}$ to
denote injections of a name into disjoint copies of the name
space. There are numerous ways to accomplish this. One example can be
found in \cite{MeredithR05}. This notation overloads to vectors of
names: $\vec{x}^{\pi} := (x_{i}^{\pi} \; : \; 0 \leq i < |\vec{x}| )$ where $\pi \in \{L,R\}$.

We also use $P^{\Box} := P|\Box$.

In \cite{MeredithR05} an interpretation of the new operator is
given. It turns out that there are several possible interpretations
all enjoying the requisite algebraic properties of the operator (see
\cite{milner91polyadicpi}). We will therefore make liberal use of
$(\nu\; \vec{x})P$.

% subsection the_syntax_and_semantics_of_the_notation_system (end)   

\input{qm2pi.qmops} 

\input{qm2pi.sterngerlach} 

\input{qm2pi.metric} 

% section concurrent_process_calculi (end)

%\input{qm2pi.proofsketch}

% section proof sketch (end)

%\input{qm2pi.slviaknots} 

% section spatial logic via knots (end)

\input{qm2pi.conclusion}

% section conclusion (end)

%\input{qm2pi.dtcodes} 

% section wiring algorithm (end)

\input{qm2pi.ack} 

% section acknowledgments (end)

\newpage


\bibliographystyle{plain}   
\bibliography{../../biblios/main.bib}

\input{qm2pi.rhodetails}

\end{document}

 

%\documentclass[12pt]{llncs}
%\documentclass{jktr}

\usepackage[pdftex]{hyperref}                   
\usepackage {listings}
\usepackage {mathpartir}
\usepackage{bcprules}
%\usepackage{listings}
                       
\usepackage{graphicx} 
%\usepackage[margins=2.5cm,nohead,nofoot]{geometry}
%\usepackage{geometry}
\usepackage{amsfonts}
\usepackage{amstext}
\usepackage{latexsym}
\usepackage{amssymb}
\usepackage{color}


%\include{myPreamble}
\include{qm2pi.local} 

%\ifpdf
%\usepackage[pdftex]{graphicx}
%\else
%\usepackage{graphicx}
%\fi

 % \ifpdf
%  \usepackage{pdfsync}
%  \if


%\title{Brief Article}
%\author{David F. Snyder}
%\author{L.G. Meredith}

%\address{Dept. of Math., Texas State University--San Marcos, San Marcos, TX 78666}
       
\pagestyle{empty}


\begin{document}

\lstset{language=[Objective]Caml,frame=shadowbox}

\input{qm2pi.front}

% section front matter (end)

\input{qm2pi.intro} 
 
% section introduction (end)

% \input{qm2pi.knotations} 

% section notation (end)

\input{qm2pi.process.calculi} 

% section concurrent_process_calculi_and_spatial_logics_ (end)
    
%\input{qm2pi.knots2pi} 

%\input{qm2pi.trefoil} 

%\input{qm2pi.mainthm} 

% subsection basic_interpretation (end)

%\input{qm2pi.rho.presentation} 
\subsection{The syntax and semantics of the notation system}\label{sub:the_syntax_and_semantics_of_the_notation_system} % (fold)

We now summarize a technical presentation of the calculus that
embodies our theory of dynamics. The typical presentation of such a
calculus follows the style of giving generators and relations on
them. The grammar, below, describing term constructors, freely
generates the set of processes, $\Proc$. This set is then quotiented
by a relation known as structural congruence and it is over this set
that the notion of dynamics is expressed. This presentation is
essentially that of \cite{MeredithR05} with the addition of
polyadicity and summation. For readability we have relegated some of
the technical subtleties to an appendix.

\subsubsection{Process grammar}\label{subsub:process_grammar}

\begin{mathpar}
  \inferrule* [lab=synchronization] {} {{M} \bc \pzero \;|\; x?F \;|\; x!C }
  \and
  \inferrule* [lab=abstraction] {} {{F} \bc (x)P}
  \and
  \inferrule* [lab=concretion] {} {{C} \bc \langle Q \rangle}
  \and
  \inferrule* [lab=process] {} {{P,Q} \bc M \;| \;P|Q \;|\; @{x}}
  \and
  \inferrule* [lab=name] {} {{x} \bc \quotep{P}}
\end{mathpar} 

Note that $\vec{x}$ (resp. $\vec{P}$) denotes a vector of names
(resp. processes) of length $|\vec{x}|$ (resp. $|\vec{P}|$). We adopt
the following useful abbreviations.

\begin{mathpar}
   x?(\vec{y}).P := x.(\vec{y})P \and  x\clift{\vec{P}} := x.\clift{\vec{P}}
   \and x!(y) := \lift{x}{\dropn{y}}
   \and \Pi_{i=0}^{n-1}P_i := P_0 | \ldots | P_{n-1}
\end{mathpar}

\subsubsection{Structural congruence}

\paragraph{Free and bound names and alpha-equivalence.} At the
core of structural equivalence is alpha-equivalence which identifies
process that are the same up to a change of variable. Formally, we
recognize the distinction between free and bound names. The free names
of a process, $\freenames{P}$, may be calculated recursively as
follows:

\begin{mathpar}
\freenames{\pzero} := \emptyset
  \and \\
  \freenames{x?(y).P} := \{ x \} \cup (\freenames{P} \setminus \{ y \})
  \and 
  \freenames{x!\langle P \rangle} := \{ x \} \cup \{ P \} 
  \and \\
  \freenames{P|Q} := \freenames{P} \cup \freenames{Q}
  \and \\
  \freenames{@{x}} := \{ x \}
\end{mathpar}

$\pi$
$\quotep{\pi}$

$\freenames{-} : \pi \to \mathcal{P}(\quotep{\pi})$

\begin{eqnarray*}
  \freenames{\pzero} & := & \emptyset \\
  \freenames{x?(y).P} & := & \{ x \} \cup (\freenames{P} \setminus \{ y \}) \\
  \freenames{x!\langle P \rangle} & := & \{ x \} \cup \{ P \} \\
  \freenames{P|Q} & := & \freenames{P} \cup \freenames{Q} \\
  \freenames{\dropn{x}} & := & \{ x \}
\end{eqnarray*}

The bound names of a process, $\boundnames{P}$, are those names occurring in $P$
that are not free. For example, in $x?(y).0$, the name $x$ is free, while $y$ is bound.

\begin{mathpar}
  \inferrule* [lab=monoidal-laws] {} { P|Q \equiv Q|P \and P|0 \equiv P \and P|(Q|R) \equiv (P|Q)|R }
\end{mathpar}

\begin{mathpar}
  \inferrule* [lab=alpha-equivalence] {} { (x)P \equiv (y)P\{y/x\} \and y \not\in \freenames{P} }
\end{mathpar}

\begin{definition}
Then two processes, $P,Q$, are alpha-equivalent if $P = Q\{\vec{y}/\vec{x}\}$ for
some $\vec{x} \in \boundnames{Q},\vec{y} \in \boundnames{P}$, where $Q\{\vec{y}/\vec{x}\}$
denotes the capture-avoiding substitution of $\vec{y}$ for $\vec{x}$ in $Q$.
\end{definition}

\begin{definition}
  The {\em structural congruence} \cite{SangiorgiWalker} , $\equiv$,
  between processes is the least congruence containing
  alpha-equivalence, satisfying the abelian monoid laws
  (associativity, commutativity and $\pzero$ as identity) for parallel
  composition $|$ and for summation $+$.
\end{definition}

\subsection{Name equivalence}

We take name equivalence, written $\nameeq$, to be the smallest
equivalence relation generated by the following rules.

\begin{mathpar}
\inferrule*[lab=Quote-drop]
{ }
{ \quotep{@{x}} \nameeq x }

\inferrule*[lab=Struct-equiv]
{ P \scong Q }
{ \quotep{P} \nameeq \quotep{Q} }
\end{mathpar}

The astute reader will have noticed that the mutual recursion of names
and processes imposes a mutual recursion on alpha-equivalence and
structural equivalence via name-equivalence. Fortunately, all of this
works out pleasantly and we may calculate in the natural way, free of
concern. The reader interested in the details is referred to the
appendix \ref{appendix:rho_details}.

\subsection{Substitution}

We use $\Proc$ for the set of processes, $\QProc$ for the set of
names, and $\id{\{}\vec{y} / \vec{x} \id{\}}$ to denote partial maps,
$s : \QProc \rightarrow \QProc$. A map, $s$ lifts, uniquely, to a map
on process terms, $\widehat{s} : \Proc \rightarrow \Proc$ by the
following equations.

\begin{mathpar}
  (0) \psubstp{Q}{P} := 0 \\
  (R \juxtap S) \psubstp{Q}{P}
  :=    
  (R)\psubstp{Q}{P} \juxtap (S) \psubstp{Q}{P} \\
  (x?(y).R) \psubstp{Q}{P}    
  :=    
  (x)\substp{Q}{P} (z)\concat( (R \psubstn{z}{y}) \psubstp{Q}{P} ) \\
  (\lift{x}{R}) \psubstp{Q}{P}  
  :=
  \lift{(x)\substp{Q}{P}}{ R \psubstp{Q}{P} } \\
%   (\dropn{x})  \psubstp{Q}{P}       
%   := 
%   \left\{ 
%     \begin{array}{ccc} 
%       \dropn{\quotep{Q}} & & x \nameeq \quotep{P} \\
%       \dropn{x} & & otherwise \\
%     \end{array}
%   \right. 
  (\dropn{x})  \psubstp{Q}{P}       
  := 
  \left\{ 
    \begin{array}{ccc} 
      Q & & x \nameeq \quotep{P} \\
      \dropn{x} & & otherwise \\
    \end{array}
  \right.
\end{mathpar}
 

where

\begin{eqnarray}
  (x)\id{\{} \lpquote Q \rpquote / \lpquote P \rpquote \id{\}}            = 
  \left\{ 
    \begin{array}{ccc}
      \lpquote Q \rpquote & & x \nameeq \lpquote P \rpquote \\
      x & & otherwise \\
    \end{array}
  \right. \nonumber
\end{eqnarray}

and $z$ is chosen distinct from $\quotep{P}$, $\quotep{Q}$, the free
names in $Q$, and all the names in $R$. Our $\alpha$-equivalence will
be built in the standard way from this substitution.

\begin{remark}\label{rem:no_self_referential_names}
  One consequence of these definitions is that $\forall P. \quotep{P}
  \not\in \freenames{P}$.
\end{remark}

\subsection{ Dynamic quote: an example }

Anticipating something of what's to come, consider applying the
substitution, $\widehat{\id{\{}u / z \id{\}}}$, to the following pair
of processes, $\lift{w}{y!(z)}$ and $w[ \lpquote y!(z) \rpquote ]$.

\begin{eqnarray}
	\lift{w}{y!(z)}\widehat{\id{\{}u / z \id{\}}}
		& = &
		\lift{w}{y!(u)} \nonumber\\
	w[ \lpquote y!(z) \rpquote ] \widehat{ \id{\{}u / z \id{\}} }
		& = &
		w[ \lpquote y!(z) \rpquote ] \nonumber
\end{eqnarray}

Because the body of the process between quotes is impervious to
substitution, we get radically different answers. In fact, by
examining the first process in an input context,
e.g. $x?(z).\lift{w}{y!(z)}$, we see that the process under the lift
operator may be shaped by prefixed inputs binding a name inside it. In
this sense, the lift operator will be seen as a way to dynamically
construct processes before reifying them as names.

Finally equipped with these standard features we can present the
dynamics of the calculus.

\subsubsection{Operational semantics} 

Finally, we introduce the computational dynamics. What marks these
algebras as distinct from other more traditionally studied algebraic
structures, e.g. vector spaces or polynomial rings, is the manner in
which dynamics is captured. In traditional structures, dynamics is typically
expressed through morphisms between such structures, as in linear maps
between vector spaces or morphisms between rings. In algebras
associated with the semantics of computation, the dynamics is
expressed as part of the algebraic structure itself, through a
reduction reduction relation typically denoted by $\red$. Below, we
give a recursive presentation of this relation for the calculus used
in the encoding.

$\red \subseteq \pi \times \pi$
$\red : \pi \to \mathcal{P}(\pi)$

\begin{mathpar}
  \inferrule* [lab=Comm] { \textsf{match}( x_{src}, x_{trgt} ) } { x_{trgt}?(y)P \; | \; x_{src}!\langle {Q} \rangle \red P\{\quotep{Q}/y}\} }
  \and \\
  \inferrule* [lab=Par] {{P} \red {P}'} {{{P} | {Q}} \red {{P}' | {Q}}}
  \and
  \inferrule* [lab=Equiv]{{{P} \scong {P}'} \andalso {{P}' \red {Q}'} \andalso {{Q}' \scong {Q}}}{{P} \red {Q}}
\end{mathpar}

\begin{eqnarray*}
  match_{\equiv} (\quotep{P},\quotep{Q}) & := & P \equiv Q \\
  match_{\dagger}(\quotep{P},\quotep{Q}) & := & \forall R. P|Q \red^{*} R => R \red^{*} 0 \\
  match_{K}(\quotep{P},\quotep{Q}) & := & K \mbox{ for some context } K
\end{eqnarray*}

$u?(x)P | u!\langle Q \rangle \red P\{\quotep{Q}/x\}$

%We write $\wred$ for $\red^*$, and $P\red$ if $\exists Q $ such that $ P \red Q$.
We write $P\red$ if $\exists Q $ such that $ P \red Q$ and $P\not\red$, otherwise.

\section{Replication}

As mentioned before, it is known that replication (and hence
recursion) can be implemented in a higher-order process algebra
\cite{SangiorgiWalker}. As our first example of calculation with the
machinery thus far presented we give the construction explicitly in
the {\rhoc}.

\begin{eqnarray}
	D_{x} & := & \prefix{x}{y}{(\binpar{\outputp{x}{y}}{@{y}})} \nonumber\\
	\bangp_{x}{P} & := & \binpar{{x}!\langle{\binpar{D_{x}}{P}}\rangle}{D_{x}} \nonumber
\end{eqnarray}

\begin{eqnarray}
	\bangp_{x}{P} & & \nonumber\\
	=
	& {x}!\langle{(\prefix{x}{y}{(\outputp{x}{y} | @{y})) | P}}\rangle 
	      | \prefix{x}{y}{(\outputp{x}{y} | @{y})} & \nonumber\\
	\red
	& (\outputp{x}{y} | @{y})\substn{\quotep{(\prefix{x}{y}{(@{y} | \outputp{x}{y})) | P}}}{y} & \nonumber\\
	=
	& \outputp{x}{\quotep{(\prefix{x}{y}{(\outputp{x}{y} | @{y})) | P}}}
	  | {(\prefix{x}{y}{(\outputp{x}{y} | @{y})) | P}} & \nonumber\\
	\red
	& \ldots & \nonumber\\
	\red^*
	& P | P | \ldots & \nonumber
\end{eqnarray}

Of course, this encoding, as an implementation, runs away, unfolding
$\bangp{P}$ eagerly. A lazier and more implementable replication
operator, restricted to input-guarded processes, may be obtained as follows.

\begin{eqnarray}
\bangp{\prefix{u}{v}{P}} 
	:= 
	\binpar{\lift{x}{\prefix{u}{v}{(\binpar{D(x)}{P})}}}{D(x)} \nonumber
\end{eqnarray}

\begin{remark}
  Note that the lazier definition still does not deal with summation
  or mixed summation (i.e. sums over input and output). The reader is
  invited to construct definitions of replication that deal with these
  features. 

  Further, the definitions are parameterized in a name, $x$. Can you,
  gentle reader, make a definition that eliminates this parameter and
  guarantees no accidental interaction between the replication
  machinery and the process being replicated -- i.e. no accidental
  sharing of names used by the process to get its work done and the
  name(s) used by the replication to effect copying. This latter
  revision of the definition of replication is crucial to obtaining
  the expected identity $!!P \sim !P$.
\end{remark}

\begin{remark}\label{rem:paradoxical_combinator}
  The reader familiar with the lambda calculus will have noticed the
  similarity between $D$ and the paradoxical combinator.

  [Ed. note: the existence of this seems to suggest we have to be more
  restrictive on the set of processes and names we admit if we are to
  support no-cloning.]
\end{remark}

\subsubsection{Bisimulation}

The computational dynamics gives rise to another kind of equivalence,
the equivalence of computational behavior. As previously mentioned
this is typically captured \emph{via} some form of bisimulation.

% The notion we use in this paper is weak barbed bisimulation
% \cite{milner91polyadicpi}.

The notion we use in this paper is derived from weak barbed
bisimulation \cite{milner91polyadicpi}. 

\begin{definition}
An \emph{observation relation}, $\downarrow_{\mathcal N}$, over a set
of names, $\mathcal N$, is the smallest relation satisfying the rules
below.

\infrule[Out-barb]{y \in {\mathcal N}, \; x \nameeq y}
		  {\outputp{x}{v} \downarrow_{\mathcal N} x}
\infrule[Par-barb]{\mbox{$P\downarrow_{\mathcal N} x$ or $Q\downarrow_{\mathcal N} x$}}
		  {\binpar{P}{Q} \downarrow_{\mathcal N} x}

We write $P \Downarrow_{\mathcal N} x$ if there is $Q$ such that 
$P \wred Q$ and $Q \downarrow_{\mathcal N} x$.
\end{definition}

\begin{definition}
%\label{def.bbisim}
An  ${\mathcal N}$-\emph{barbed bisimulation} over a set of names, ${\mathcal N}$, is a symmetric binary relation 
${\mathcal S}_{\mathcal N}$ between agents such that $P\rel{S}_{\mathcal N}Q$ implies:
\begin{enumerate}
\item If $P \red P'$ then $Q \wred Q'$ and $P'\rel{S}_{\mathcal N} Q'$.
\item If $P\downarrow_{\mathcal N} x$, then $Q\Downarrow_{\mathcal N} x$.
\end{enumerate}
$P$ is ${\mathcal N}$-barbed bisimilar to $Q$, written
$P \wbbisim_{\mathcal N} Q$, if $P \rel{S}_{\mathcal N} Q$ for some ${\mathcal N}$-barbed bisimulation ${\mathcal S}_{\mathcal N}$.
\end{definition}

$\mathcal{R} \subseteq \pi \times \pi$

$P \mathcal{R} Q => \forall P'. P \red P' \Rightarrow \exists Q'. Q \red Q', P' \mathcal{R} Q'$

$P \vdash x \Rightarrow Q \vdash x$

\begin{mathpar}
  \inferrule*[lab=Out-barb]{x \nameeq y}{{y}!\langle{Q}\rangle \vdash x}
  \and
  \inferrule*[lab=Par-barb]{\mbox{$P\vdash x$ or $Q\vdash x$}}{\binpar{P}{Q} \vdash x}
\end{mathpar}

\subsubsection{Contexts}

One of the principle advantages of computational calculi like the
$\pi$-calculus is a well-defined notion of context,
contextual-equivalence and a correlation between
contextual-equivalence and notions of bisimulation. The notion of
context allows the decomposition of a process into (sub-)process and
its syntactic environment, its context. Thus, a context may be
thought of as a process with a ``hole'' (written $\Box$) in it. The
application of a context $M$ to a process $P$, written $M[P]$, is
tantamount to filling the hole in $M$ with $P$. In this paper we do
not need the full weight of this theory, but do make use of the notion
of context in the proof the main theorem. 

\begin{mathpar}
  \inferrule* [lab=summation] {} {{M_{M},M_{N}} \bc \Box \;|\; x.M_{A} \;|\; M_{M}+M_{N}}
  \and
  \inferrule* [lab=agent] {} {{M_{A}} \bc (\vec{x})M_{P} \;| \; \clift{P_0,\ldots,M_{P},\ldots,P_N}}
  \and \\
  \inferrule* [lab=process] {} {{M_{P}} \bc M_{N} \;| \;P|M_{P} }
\end{mathpar} 

\begin{mathpar}
  \inferrule* [lab=sychronization] {} {M_{N} \bc \Box \;|\; x?M_{F} \;|\; x!M_{C}}
  \and
  \inferrule* [lab=abstraction] {} {{M_{F}} \bc (x)M_{P} }
  \and
  \inferrule* [lab=concretion] {} {{M_{C}} \bc \langle M_{P} \rangle }
  \and \\
  \inferrule* [lab=process] {} {{M_{P}} \bc M_{N} \;| \;P|M_{P} }
\end{mathpar}

\begin{definition}[contextual application] Given a context $M$, and
  process $P$, we define the \emph{contextual application}, $M[P] :=
  M\{P/\Box\}$. That is, the contextual application of M to P is the
  substitution of $P$ for $\Box$ in $M$.
\end{definition}

$\meaningof{-} : L \to \mathcal{P}(\pi)$

\begin{mathpar}
  \inferrule* [lab=collection] {} {\meaningof{true} = \pi, \and \meaningof{~E} = \pi \setminus \meaningof{E}, \and \meaningof{E_{1} \& E_{2}} = \meaningof{E_{1}} \cap \meaningof{E_{2}}}
\end{mathpar}

\begin{mathpar}
  \inferrule* [lab=structure] {} {\meaningof{0} = \{ P \in \pi | P \equiv 0 \}, \and \\ \meaningof{E_1 | E_2} = \{ P \in \pi | P \equiv P_{1} | P_{2}, P_{1} \in \meaningof{E_{1}}, P_{2} \in \meaningof{E_2}\} }
\end{mathpar}

\begin{mathpar}
 \inferrule* [lab=behavior] {} {\meaningof{\langle a?b \rangle E} = \{ P \in \pi | P \equiv Q | u?(y)P', \\ \and \\\\ \and \\ \;\;\; u \in \meaningof{a}, \forall z.P'\{z/y\} \in \meaningof{E\{z/b\}}\}, \and \\ \meaningof{a!E} = \{ P \in \pi | P \equiv Q | x!\langle P' \rangle, x \in \meaningof{a} P' \in \meaningof{E}\} }
\end{mathpar}

\begin{mathpar}
 \inferrule* [lab=nominal] {} {\meaningof{\quotep{E}} = \{ \quotep{P} \in \quotep{\pi} | P \in \meaningof{E} \}, \and \meaningof{\quotep{P}} = \{ \quotep{Q} \in \quotep{\pi} | P \equiv Q \} \and \\ \meaningof{@\quotep{E}} = \{ P \in \pi | P \equiv @x, x \in \meaningof{E} \}}
\end{mathpar}

\begin{eqnarray*}
  \\
  \meaningof{-} : TS \to ST
\end{eqnarray*}

\begin{eqnarray*}
  \\
  L : TS \to ST
\end{eqnarray*}

\begin{eqnarray*}
  \\
  P \models E \iff P \in \meaningof{E}
\end{eqnarray*}

\begin{eqnarray*}
  P \approx_{L} Q \iff \forall E \in L. P \models E \iff Q \models E
\end{eqnarray*}

\begin{eqnarray*}
  P \approx_{K} Q
\end{eqnarray*}

\begin{eqnarray*}
  P \approx Q
\end{eqnarray*}

$\approx_{K} = \approx = \approx_{L}$

\subsubsection{Contextual duality}

Note that contexts extend the quotation operation to a family of
operations from processes to names. Given a context, $M$, we can
define a \emph{nominal context}, $\quotep{M}$ by $\quotep{M}[P] :=
\quotep{M[P]}$. To foreshadow what is to come we observe that these
operations enjoy a duality with processes very much like the duality
between vectors and maps from vectors to scalars.

Further, because the calculus is essentially higher-order, we have a
correspondence between contexts and processes. More specifically,
given a name $x$ and a context $M$ we can construct $M^{*}_{x}$ such
that 

\begin{mathpar}
  M^{*}_{x} | \lift{x}{P} \red M[P]
\end{mathpar}

namely,

\begin{mathpar}
  M^{*}_{x} := x?(u).M[\dropn{u}]
\end{mathpar}

The dependence of $M^{*}_{x}$ on a name makes it an abstraction, 

\begin{mathpar}
  M^{*} := (x)x?(u).M[\dropn{u}]
\end{mathpar}

\subsection{Additional notation}

It will sometimes be convenient to denote the process a name
quotes. We already have the notation $x = \quotep{P}$, but it will be
convenient to introduce an alternate notation, $\procn{x}$, when we
want to emphasize the connection to the use of the name. Note that, by
virtue of name equivalence, $\quotep{\procn{x}} \nameeq x$; so, the
notation is consistent with previous definitions.

Further, because names have structure it is possible to effect
substitutions on the basis of that structure. This means we need to
upgrade our notation for substitutions, which we accomplish by
adapting comprehension notation. Thus,

\begin{mathpar}
  P\{ y / x : x \in S \}
\end{mathpar}

is interpreted to mean the process derived from P by replacing (in a
capture-avoiding manner) each occurrence of $x$ in $S$ by $y$. For example,

\begin{mathpar}
  P\{ \quotep{\procn{x}|\procn{x}} / x : x \in \freenames{P} \}
\end{mathpar}

will replace each (occurrence) of a free name $x$ in $P$ by
$\quotep{\procn{x}|\procn{x}}$.

Also, we will avail ourselves of the notation $x^{L}$ and $x^{R}$ to
denote injections of a name into disjoint copies of the name
space. There are numerous ways to accomplish this. One example can be
found in \cite{MeredithR05}. This notation overloads to vectors of
names: $\vec{x}^{\pi} := (x_{i}^{\pi} \; : \; 0 \leq i < |\vec{x}| )$ where $\pi \in \{L,R\}$.

We also use $P^{\Box} := P|\Box$.

In \cite{MeredithR05} an interpretation of the new operator is
given. It turns out that there are several possible interpretations
all enjoying the requisite algebraic properties of the operator (see
\cite{milner91polyadicpi}). We will therefore make liberal use of
$(\nu\; \vec{x})P$.

% subsection the_syntax_and_semantics_of_the_notation_system (end)   

\input{qm2pi.qmops} 

\input{qm2pi.sterngerlach} 

\input{qm2pi.metric} 

% section concurrent_process_calculi (end)

%\input{qm2pi.proofsketch}

% section proof sketch (end)

%\input{qm2pi.slviaknots} 

% section spatial logic via knots (end)

\input{qm2pi.conclusion}

% section conclusion (end)

%\input{qm2pi.dtcodes} 

% section wiring algorithm (end)

\input{qm2pi.ack} 

% section acknowledgments (end)

\newpage


\bibliographystyle{plain}   
\bibliography{../../biblios/main.bib}

\input{qm2pi.rhodetails}

\end{document}

 

% subsection basic_interpretation (end)

%\input{qm2pi.rho.presentation} 
\subsection{The syntax and semantics of the notation system}\label{sub:the_syntax_and_semantics_of_the_notation_system} % (fold)

We now summarize a technical presentation of the calculus that
embodies our theory of dynamics. The typical presentation of such a
calculus follows the style of giving generators and relations on
them. The grammar, below, describing term constructors, freely
generates the set of processes, $\Proc$. This set is then quotiented
by a relation known as structural congruence and it is over this set
that the notion of dynamics is expressed. This presentation is
essentially that of \cite{MeredithR05} with the addition of
polyadicity and summation. For readability we have relegated some of
the technical subtleties to an appendix.

\subsubsection{Process grammar}\label{subsub:process_grammar}

\begin{mathpar}
  \inferrule* [lab=synchronization] {} {{M} \bc \pzero \;|\; x?F \;|\; x!C }
  \and
  \inferrule* [lab=abstraction] {} {{F} \bc (x)P}
  \and
  \inferrule* [lab=concretion] {} {{C} \bc \langle Q \rangle}
  \and
  \inferrule* [lab=process] {} {{P,Q} \bc M \;| \;P|Q \;|\; @{x}}
  \and
  \inferrule* [lab=name] {} {{x} \bc \quotep{P}}
\end{mathpar} 

Note that $\vec{x}$ (resp. $\vec{P}$) denotes a vector of names
(resp. processes) of length $|\vec{x}|$ (resp. $|\vec{P}|$). We adopt
the following useful abbreviations.

\begin{mathpar}
   x?(\vec{y}).P := x.(\vec{y})P \and  x\clift{\vec{P}} := x.\clift{\vec{P}}
   \and x!(y) := \lift{x}{\dropn{y}}
   \and \Pi_{i=0}^{n-1}P_i := P_0 | \ldots | P_{n-1}
\end{mathpar}

\subsubsection{Structural congruence}

\paragraph{Free and bound names and alpha-equivalence.} At the
core of structural equivalence is alpha-equivalence which identifies
process that are the same up to a change of variable. Formally, we
recognize the distinction between free and bound names. The free names
of a process, $\freenames{P}$, may be calculated recursively as
follows:

\begin{mathpar}
\freenames{\pzero} := \emptyset
  \and \\
  \freenames{x?(y).P} := \{ x \} \cup (\freenames{P} \setminus \{ y \})
  \and 
  \freenames{x!\langle P \rangle} := \{ x \} \cup \{ P \} 
  \and \\
  \freenames{P|Q} := \freenames{P} \cup \freenames{Q}
  \and \\
  \freenames{@{x}} := \{ x \}
\end{mathpar}

$\pi$
$\quotep{\pi}$

$\freenames{-} : \pi \to \mathcal{P}(\quotep{\pi})$

\begin{eqnarray*}
  \freenames{\pzero} & := & \emptyset \\
  \freenames{x?(y).P} & := & \{ x \} \cup (\freenames{P} \setminus \{ y \}) \\
  \freenames{x!\langle P \rangle} & := & \{ x \} \cup \{ P \} \\
  \freenames{P|Q} & := & \freenames{P} \cup \freenames{Q} \\
  \freenames{\dropn{x}} & := & \{ x \}
\end{eqnarray*}

The bound names of a process, $\boundnames{P}$, are those names occurring in $P$
that are not free. For example, in $x?(y).0$, the name $x$ is free, while $y$ is bound.

\begin{mathpar}
  \inferrule* [lab=monoidal-laws] {} { P|Q \equiv Q|P \and P|0 \equiv P \and P|(Q|R) \equiv (P|Q)|R }
\end{mathpar}

\begin{mathpar}
  \inferrule* [lab=alpha-equivalence] {} { (x)P \equiv (y)P\{y/x\} \and y \not\in \freenames{P} }
\end{mathpar}

\begin{definition}
Then two processes, $P,Q$, are alpha-equivalent if $P = Q\{\vec{y}/\vec{x}\}$ for
some $\vec{x} \in \boundnames{Q},\vec{y} \in \boundnames{P}$, where $Q\{\vec{y}/\vec{x}\}$
denotes the capture-avoiding substitution of $\vec{y}$ for $\vec{x}$ in $Q$.
\end{definition}

\begin{definition}
  The {\em structural congruence} \cite{SangiorgiWalker} , $\equiv$,
  between processes is the least congruence containing
  alpha-equivalence, satisfying the abelian monoid laws
  (associativity, commutativity and $\pzero$ as identity) for parallel
  composition $|$ and for summation $+$.
\end{definition}

\subsection{Name equivalence}

We take name equivalence, written $\nameeq$, to be the smallest
equivalence relation generated by the following rules.

\begin{mathpar}
\inferrule*[lab=Quote-drop]
{ }
{ \quotep{@{x}} \nameeq x }

\inferrule*[lab=Struct-equiv]
{ P \scong Q }
{ \quotep{P} \nameeq \quotep{Q} }
\end{mathpar}

The astute reader will have noticed that the mutual recursion of names
and processes imposes a mutual recursion on alpha-equivalence and
structural equivalence via name-equivalence. Fortunately, all of this
works out pleasantly and we may calculate in the natural way, free of
concern. The reader interested in the details is referred to the
appendix \ref{appendix:rho_details}.

\subsection{Substitution}

We use $\Proc$ for the set of processes, $\QProc$ for the set of
names, and $\id{\{}\vec{y} / \vec{x} \id{\}}$ to denote partial maps,
$s : \QProc \rightarrow \QProc$. A map, $s$ lifts, uniquely, to a map
on process terms, $\widehat{s} : \Proc \rightarrow \Proc$ by the
following equations.

\begin{mathpar}
  (0) \psubstp{Q}{P} := 0 \\
  (R \juxtap S) \psubstp{Q}{P}
  :=    
  (R)\psubstp{Q}{P} \juxtap (S) \psubstp{Q}{P} \\
  (x?(y).R) \psubstp{Q}{P}    
  :=    
  (x)\substp{Q}{P} (z)\concat( (R \psubstn{z}{y}) \psubstp{Q}{P} ) \\
  (\lift{x}{R}) \psubstp{Q}{P}  
  :=
  \lift{(x)\substp{Q}{P}}{ R \psubstp{Q}{P} } \\
%   (\dropn{x})  \psubstp{Q}{P}       
%   := 
%   \left\{ 
%     \begin{array}{ccc} 
%       \dropn{\quotep{Q}} & & x \nameeq \quotep{P} \\
%       \dropn{x} & & otherwise \\
%     \end{array}
%   \right. 
  (\dropn{x})  \psubstp{Q}{P}       
  := 
  \left\{ 
    \begin{array}{ccc} 
      Q & & x \nameeq \quotep{P} \\
      \dropn{x} & & otherwise \\
    \end{array}
  \right.
\end{mathpar}
 

where

\begin{eqnarray}
  (x)\id{\{} \lpquote Q \rpquote / \lpquote P \rpquote \id{\}}            = 
  \left\{ 
    \begin{array}{ccc}
      \lpquote Q \rpquote & & x \nameeq \lpquote P \rpquote \\
      x & & otherwise \\
    \end{array}
  \right. \nonumber
\end{eqnarray}

and $z$ is chosen distinct from $\quotep{P}$, $\quotep{Q}$, the free
names in $Q$, and all the names in $R$. Our $\alpha$-equivalence will
be built in the standard way from this substitution.

\begin{remark}\label{rem:no_self_referential_names}
  One consequence of these definitions is that $\forall P. \quotep{P}
  \not\in \freenames{P}$.
\end{remark}

\subsection{ Dynamic quote: an example }

Anticipating something of what's to come, consider applying the
substitution, $\widehat{\id{\{}u / z \id{\}}}$, to the following pair
of processes, $\lift{w}{y!(z)}$ and $w[ \lpquote y!(z) \rpquote ]$.

\begin{eqnarray}
	\lift{w}{y!(z)}\widehat{\id{\{}u / z \id{\}}}
		& = &
		\lift{w}{y!(u)} \nonumber\\
	w[ \lpquote y!(z) \rpquote ] \widehat{ \id{\{}u / z \id{\}} }
		& = &
		w[ \lpquote y!(z) \rpquote ] \nonumber
\end{eqnarray}

Because the body of the process between quotes is impervious to
substitution, we get radically different answers. In fact, by
examining the first process in an input context,
e.g. $x?(z).\lift{w}{y!(z)}$, we see that the process under the lift
operator may be shaped by prefixed inputs binding a name inside it. In
this sense, the lift operator will be seen as a way to dynamically
construct processes before reifying them as names.

Finally equipped with these standard features we can present the
dynamics of the calculus.

\subsubsection{Operational semantics} 

Finally, we introduce the computational dynamics. What marks these
algebras as distinct from other more traditionally studied algebraic
structures, e.g. vector spaces or polynomial rings, is the manner in
which dynamics is captured. In traditional structures, dynamics is typically
expressed through morphisms between such structures, as in linear maps
between vector spaces or morphisms between rings. In algebras
associated with the semantics of computation, the dynamics is
expressed as part of the algebraic structure itself, through a
reduction reduction relation typically denoted by $\red$. Below, we
give a recursive presentation of this relation for the calculus used
in the encoding.

$\red \subseteq \pi \times \pi$
$\red : \pi \to \mathcal{P}(\pi)$

\begin{mathpar}
  \inferrule* [lab=Comm] { \textsf{match}( x_{src}, x_{trgt} ) } { x_{trgt}?(y)P \; | \; x_{src}!\langle {Q} \rangle \red P\{\quotep{Q}/y}\} }
  \and \\
  \inferrule* [lab=Par] {{P} \red {P}'} {{{P} | {Q}} \red {{P}' | {Q}}}
  \and
  \inferrule* [lab=Equiv]{{{P} \scong {P}'} \andalso {{P}' \red {Q}'} \andalso {{Q}' \scong {Q}}}{{P} \red {Q}}
\end{mathpar}

\begin{eqnarray*}
  match_{\equiv} (\quotep{P},\quotep{Q}) & := & P \equiv Q \\
  match_{\dagger}(\quotep{P},\quotep{Q}) & := & \forall R. P|Q \red^{*} R => R \red^{*} 0 \\
  match_{K}(\quotep{P},\quotep{Q}) & := & K \mbox{ for some context } K
\end{eqnarray*}

$u?(x)P | u!\langle Q \rangle \red P\{\quotep{Q}/x\}$

%We write $\wred$ for $\red^*$, and $P\red$ if $\exists Q $ such that $ P \red Q$.
We write $P\red$ if $\exists Q $ such that $ P \red Q$ and $P\not\red$, otherwise.

\section{Replication}

As mentioned before, it is known that replication (and hence
recursion) can be implemented in a higher-order process algebra
\cite{SangiorgiWalker}. As our first example of calculation with the
machinery thus far presented we give the construction explicitly in
the {\rhoc}.

\begin{eqnarray}
	D_{x} & := & \prefix{x}{y}{(\binpar{\outputp{x}{y}}{@{y}})} \nonumber\\
	\bangp_{x}{P} & := & \binpar{{x}!\langle{\binpar{D_{x}}{P}}\rangle}{D_{x}} \nonumber
\end{eqnarray}

\begin{eqnarray}
	\bangp_{x}{P} & & \nonumber\\
	=
	& {x}!\langle{(\prefix{x}{y}{(\outputp{x}{y} | @{y})) | P}}\rangle 
	      | \prefix{x}{y}{(\outputp{x}{y} | @{y})} & \nonumber\\
	\red
	& (\outputp{x}{y} | @{y})\substn{\quotep{(\prefix{x}{y}{(@{y} | \outputp{x}{y})) | P}}}{y} & \nonumber\\
	=
	& \outputp{x}{\quotep{(\prefix{x}{y}{(\outputp{x}{y} | @{y})) | P}}}
	  | {(\prefix{x}{y}{(\outputp{x}{y} | @{y})) | P}} & \nonumber\\
	\red
	& \ldots & \nonumber\\
	\red^*
	& P | P | \ldots & \nonumber
\end{eqnarray}

Of course, this encoding, as an implementation, runs away, unfolding
$\bangp{P}$ eagerly. A lazier and more implementable replication
operator, restricted to input-guarded processes, may be obtained as follows.

\begin{eqnarray}
\bangp{\prefix{u}{v}{P}} 
	:= 
	\binpar{\lift{x}{\prefix{u}{v}{(\binpar{D(x)}{P})}}}{D(x)} \nonumber
\end{eqnarray}

\begin{remark}
  Note that the lazier definition still does not deal with summation
  or mixed summation (i.e. sums over input and output). The reader is
  invited to construct definitions of replication that deal with these
  features. 

  Further, the definitions are parameterized in a name, $x$. Can you,
  gentle reader, make a definition that eliminates this parameter and
  guarantees no accidental interaction between the replication
  machinery and the process being replicated -- i.e. no accidental
  sharing of names used by the process to get its work done and the
  name(s) used by the replication to effect copying. This latter
  revision of the definition of replication is crucial to obtaining
  the expected identity $!!P \sim !P$.
\end{remark}

\begin{remark}\label{rem:paradoxical_combinator}
  The reader familiar with the lambda calculus will have noticed the
  similarity between $D$ and the paradoxical combinator.

  [Ed. note: the existence of this seems to suggest we have to be more
  restrictive on the set of processes and names we admit if we are to
  support no-cloning.]
\end{remark}

\subsubsection{Bisimulation}

The computational dynamics gives rise to another kind of equivalence,
the equivalence of computational behavior. As previously mentioned
this is typically captured \emph{via} some form of bisimulation.

% The notion we use in this paper is weak barbed bisimulation
% \cite{milner91polyadicpi}.

The notion we use in this paper is derived from weak barbed
bisimulation \cite{milner91polyadicpi}. 

\begin{definition}
An \emph{observation relation}, $\downarrow_{\mathcal N}$, over a set
of names, $\mathcal N$, is the smallest relation satisfying the rules
below.

\infrule[Out-barb]{y \in {\mathcal N}, \; x \nameeq y}
		  {\outputp{x}{v} \downarrow_{\mathcal N} x}
\infrule[Par-barb]{\mbox{$P\downarrow_{\mathcal N} x$ or $Q\downarrow_{\mathcal N} x$}}
		  {\binpar{P}{Q} \downarrow_{\mathcal N} x}

We write $P \Downarrow_{\mathcal N} x$ if there is $Q$ such that 
$P \wred Q$ and $Q \downarrow_{\mathcal N} x$.
\end{definition}

\begin{definition}
%\label{def.bbisim}
An  ${\mathcal N}$-\emph{barbed bisimulation} over a set of names, ${\mathcal N}$, is a symmetric binary relation 
${\mathcal S}_{\mathcal N}$ between agents such that $P\rel{S}_{\mathcal N}Q$ implies:
\begin{enumerate}
\item If $P \red P'$ then $Q \wred Q'$ and $P'\rel{S}_{\mathcal N} Q'$.
\item If $P\downarrow_{\mathcal N} x$, then $Q\Downarrow_{\mathcal N} x$.
\end{enumerate}
$P$ is ${\mathcal N}$-barbed bisimilar to $Q$, written
$P \wbbisim_{\mathcal N} Q$, if $P \rel{S}_{\mathcal N} Q$ for some ${\mathcal N}$-barbed bisimulation ${\mathcal S}_{\mathcal N}$.
\end{definition}

$\mathcal{R} \subseteq \pi \times \pi$

$P \mathcal{R} Q => \forall P'. P \red P' \Rightarrow \exists Q'. Q \red Q', P' \mathcal{R} Q'$

$P \vdash x \Rightarrow Q \vdash x$

\begin{mathpar}
  \inferrule*[lab=Out-barb]{x \nameeq y}{{y}!\langle{Q}\rangle \vdash x}
  \and
  \inferrule*[lab=Par-barb]{\mbox{$P\vdash x$ or $Q\vdash x$}}{\binpar{P}{Q} \vdash x}
\end{mathpar}

\subsubsection{Contexts}

One of the principle advantages of computational calculi like the
$\pi$-calculus is a well-defined notion of context,
contextual-equivalence and a correlation between
contextual-equivalence and notions of bisimulation. The notion of
context allows the decomposition of a process into (sub-)process and
its syntactic environment, its context. Thus, a context may be
thought of as a process with a ``hole'' (written $\Box$) in it. The
application of a context $M$ to a process $P$, written $M[P]$, is
tantamount to filling the hole in $M$ with $P$. In this paper we do
not need the full weight of this theory, but do make use of the notion
of context in the proof the main theorem. 

\begin{mathpar}
  \inferrule* [lab=summation] {} {{M_{M},M_{N}} \bc \Box \;|\; x.M_{A} \;|\; M_{M}+M_{N}}
  \and
  \inferrule* [lab=agent] {} {{M_{A}} \bc (\vec{x})M_{P} \;| \; \clift{P_0,\ldots,M_{P},\ldots,P_N}}
  \and \\
  \inferrule* [lab=process] {} {{M_{P}} \bc M_{N} \;| \;P|M_{P} }
\end{mathpar} 

\begin{mathpar}
  \inferrule* [lab=sychronization] {} {M_{N} \bc \Box \;|\; x?M_{F} \;|\; x!M_{C}}
  \and
  \inferrule* [lab=abstraction] {} {{M_{F}} \bc (x)M_{P} }
  \and
  \inferrule* [lab=concretion] {} {{M_{C}} \bc \langle M_{P} \rangle }
  \and \\
  \inferrule* [lab=process] {} {{M_{P}} \bc M_{N} \;| \;P|M_{P} }
\end{mathpar}

\begin{definition}[contextual application] Given a context $M$, and
  process $P$, we define the \emph{contextual application}, $M[P] :=
  M\{P/\Box\}$. That is, the contextual application of M to P is the
  substitution of $P$ for $\Box$ in $M$.
\end{definition}

$\meaningof{-} : L \to \mathcal{P}(\pi)$

\begin{mathpar}
  \inferrule* [lab=collection] {} {\meaningof{true} = \pi, \and \meaningof{~E} = \pi \setminus \meaningof{E}, \and \meaningof{E_{1} \& E_{2}} = \meaningof{E_{1}} \cap \meaningof{E_{2}}}
\end{mathpar}

\begin{mathpar}
  \inferrule* [lab=structure] {} {\meaningof{0} = \{ P \in \pi | P \equiv 0 \}, \and \\ \meaningof{E_1 | E_2} = \{ P \in \pi | P \equiv P_{1} | P_{2}, P_{1} \in \meaningof{E_{1}}, P_{2} \in \meaningof{E_2}\} }
\end{mathpar}

\begin{mathpar}
 \inferrule* [lab=behavior] {} {\meaningof{\langle a?b \rangle E} = \{ P \in \pi | P \equiv Q | u?(y)P', \\ \and \\\\ \and \\ \;\;\; u \in \meaningof{a}, \forall z.P'\{z/y\} \in \meaningof{E\{z/b\}}\}, \and \\ \meaningof{a!E} = \{ P \in \pi | P \equiv Q | x!\langle P' \rangle, x \in \meaningof{a} P' \in \meaningof{E}\} }
\end{mathpar}

\begin{mathpar}
 \inferrule* [lab=nominal] {} {\meaningof{\quotep{E}} = \{ \quotep{P} \in \quotep{\pi} | P \in \meaningof{E} \}, \and \meaningof{\quotep{P}} = \{ \quotep{Q} \in \quotep{\pi} | P \equiv Q \} \and \\ \meaningof{@\quotep{E}} = \{ P \in \pi | P \equiv @x, x \in \meaningof{E} \}}
\end{mathpar}

\begin{eqnarray*}
  \\
  \meaningof{-} : TS \to ST
\end{eqnarray*}

\begin{eqnarray*}
  \\
  L : TS \to ST
\end{eqnarray*}

\begin{eqnarray*}
  \\
  P \models E \iff P \in \meaningof{E}
\end{eqnarray*}

\begin{eqnarray*}
  P \approx_{L} Q \iff \forall E \in L. P \models E \iff Q \models E
\end{eqnarray*}

\begin{eqnarray*}
  P \approx_{K} Q
\end{eqnarray*}

\begin{eqnarray*}
  P \approx Q
\end{eqnarray*}

$\approx_{K} = \approx = \approx_{L}$

\subsubsection{Contextual duality}

Note that contexts extend the quotation operation to a family of
operations from processes to names. Given a context, $M$, we can
define a \emph{nominal context}, $\quotep{M}$ by $\quotep{M}[P] :=
\quotep{M[P]}$. To foreshadow what is to come we observe that these
operations enjoy a duality with processes very much like the duality
between vectors and maps from vectors to scalars.

Further, because the calculus is essentially higher-order, we have a
correspondence between contexts and processes. More specifically,
given a name $x$ and a context $M$ we can construct $M^{*}_{x}$ such
that 

\begin{mathpar}
  M^{*}_{x} | \lift{x}{P} \red M[P]
\end{mathpar}

namely,

\begin{mathpar}
  M^{*}_{x} := x?(u).M[\dropn{u}]
\end{mathpar}

The dependence of $M^{*}_{x}$ on a name makes it an abstraction, 

\begin{mathpar}
  M^{*} := (x)x?(u).M[\dropn{u}]
\end{mathpar}

\subsection{Additional notation}

It will sometimes be convenient to denote the process a name
quotes. We already have the notation $x = \quotep{P}$, but it will be
convenient to introduce an alternate notation, $\procn{x}$, when we
want to emphasize the connection to the use of the name. Note that, by
virtue of name equivalence, $\quotep{\procn{x}} \nameeq x$; so, the
notation is consistent with previous definitions.

Further, because names have structure it is possible to effect
substitutions on the basis of that structure. This means we need to
upgrade our notation for substitutions, which we accomplish by
adapting comprehension notation. Thus,

\begin{mathpar}
  P\{ y / x : x \in S \}
\end{mathpar}

is interpreted to mean the process derived from P by replacing (in a
capture-avoiding manner) each occurrence of $x$ in $S$ by $y$. For example,

\begin{mathpar}
  P\{ \quotep{\procn{x}|\procn{x}} / x : x \in \freenames{P} \}
\end{mathpar}

will replace each (occurrence) of a free name $x$ in $P$ by
$\quotep{\procn{x}|\procn{x}}$.

Also, we will avail ourselves of the notation $x^{L}$ and $x^{R}$ to
denote injections of a name into disjoint copies of the name
space. There are numerous ways to accomplish this. One example can be
found in \cite{MeredithR05}. This notation overloads to vectors of
names: $\vec{x}^{\pi} := (x_{i}^{\pi} \; : \; 0 \leq i < |\vec{x}| )$ where $\pi \in \{L,R\}$.

We also use $P^{\Box} := P|\Box$.

In \cite{MeredithR05} an interpretation of the new operator is
given. It turns out that there are several possible interpretations
all enjoying the requisite algebraic properties of the operator (see
\cite{milner91polyadicpi}). We will therefore make liberal use of
$(\nu\; \vec{x})P$.

% subsection the_syntax_and_semantics_of_the_notation_system (end)   

\section{Interpretation of QM}
\subsection{Supporting definitions}
\subsubsection{Multiplication}
\begin{mathpar}
  \quotep{Q} \cdot \quotep{R} := \quotep{Q|R}
  \and \\
  \quotep{Q} \cdot P := P\{ \quotep{Q|R} / \quotep{R} : \quotep{R} \in \freenames{P} \}
\end{mathpar}

\paragraph{Discussion}
The first line needs little explanation. The second line says that
each free name of the process is replaced with the multiplication of
that name by the scalar. Multiplication of a scalar (name) by a state
(process) results in a process all the names of which have been `moved
over' by parallel composition with the process the scalar
quotes. There is a subtlety that the bound names have to be
manipulated so that multiplied names aren't accidentally
captured. There are many ways to achieve this.

\begin{remark}\label{rem:multiplication_identities}
  The reader is invited to verify that for all $x,y,z \in \QProc$ and $P \in \Proc$
  \begin{mathpar}
    x \cdot \quotep{0} \equiv x 
    \and
    x \cdot y \equiv y \cdot x
    \and
    x \cdot (y \cdot z) \equiv (x \cdot y) \cdot z
    \and \\
    \quotep{0} \cdot P \equiv P
    \and \\
    x \cdot (y \cdot P) \equiv (x \cdot y) \cdot P
    \and \\
    x \cdot (P|Q) \equiv (x \cdot P) | (x \cdot Q)
    \and \\    
  \end{mathpar}
\end{remark}

\subsubsection{Tensor product}

We define a tensor product on processes by structural induction.

\paragraph{Tensor of sums} First note that all summations, including
$\pzero$ and sequence, can be written $\Sigma_{i} x_{i}.A_{i} +
\Sigma_{j} x_{j}.C_{j}$, where we have grouped input-guarded processes
together and output-guarded processes together.

Thus, we can define the tensor product of two summations, $N_{1}\otimes N_{2}$, where

\begin{mathpar}
  N_{1} := \Sigma_{i} x_{i}.A_{i} + \Sigma_{j} x_{j}.C_{j}
  \and
  N_{2} := \Sigma_{i'} y_{i'}.B_{i'} + \Sigma_{j'} y_{j'}.D_{j'} 
\end{mathpar}

as follows.

\begin{mathpar}
  \Sigma_{i} x_{i}.A_{i} + \Sigma_{j} x_{j}.C_{j} \otimes \Sigma_{i'}
  y_{i'}.B_{i'} + \Sigma_{j'} y_{j'}.D_{j'} 
  \and \\
  := \; \Sigma_{i} \Sigma_{i'} \quotep{\stackrel{\vee}{x_{i}}| \stackrel{\vee}{y_{i'}}}.(A_{i}\otimes B_{i'}) \; | \; \Sigma_{i'} \Sigma_{i} \quotep{\stackrel{\vee}{y_{i'}}|\stackrel{\vee}{x_{i}}}.(B_{i'}\otimes A_{i})
  \and
  \;\; | \;\; \Sigma_{j} \Sigma_{j'} \quotep{\stackrel{\vee}{x_{j}}|\stackrel{\vee}{y_{j'}}}.(A_{j}\otimes B_{j'}) \; | \; \Sigma_{j'} \Sigma_{j} \quotep{\stackrel{\vee}{y_{j'}}|\stackrel{\vee}{x_{j}}}.(B_{j'}\otimes A_{j})
\end{mathpar}

\begin{remark}
  Do we need to $x^{L}$ and $y^{R}$ for this construction as well?
\end{remark}

\paragraph{Tensor of parallel compositions} Next, we distribute tensor
over par.

\begin{mathpar}
  P_{1}|P_{2} \otimes Q_{1}|Q_{2} := (P_{1} \otimes Q_{1}) | (P_{1}
  \otimes Q_{2}) | (P_{2} \otimes Q_{1}) | (P_{2} \otimes Q_{2})
\end{mathpar}

\paragraph{Tensor with dropped names} We treat tensor of a
process with a dropped name as parallel composition.

\begin{mathpar}
  P \otimes \dropn{x} := P | \dropn{x}
\end{mathpar}

\paragraph{Tensor of agents}

Finally, we need to define tensor on agents. Note that the definition
of tensor on normal products only tensors inputs with inputs and
outputs with outputs. Thus, we only have to define the operation on
``homogeneous'' pairings.

\begin{mathpar}
  (\vec{x})P \otimes (\vec{y})Q
  \and \\
  := (x_{0}^{L}|y_{0}^{R},\ldots,x_{0}^{L}|y_{n}^{R},\ldots,x_{m}^{L}|y_{0}^{R},\ldots,x_{m}^{L}|y_{n}^R)(P\{ \vec{x}^{L}/\vec{x}\} \otimes Q \{ \vec{y}^{R}/\vec{y}\})
  \and \\
  \clift{\vec{P}} \otimes \clift{\vec{Q}}
  \and \\
  := \clift{P_{0}\otimes Q_{0},\ldots,P_{0}\otimes Q_{n},\ldots,P_{m}\otimes Q_{0},\ldots,P_{m}\otimes Q_{n}}
\end{mathpar}

\begin{remark}
  Observe that arities of tensored abstractions matches arities of
  tensored concretions if the original arities matched. Note also that
  the length of the arities corresponds to the increase in dimension
  we see in ordinary vector space tensor product.
\end{remark}

\begin{remark}
  Operationally, this definition distributes the tensor down to
  components ``linked'' by summation. Tensor over summation is
  intriguing in that it mixes names. Moreover, as a consequence of the
  way it mixes names we have the identities for all $x \in \QProc$ and
  $P,Q \in \Proc$

  \begin{mathpar}
    (x \cdot P) \otimes Q \equiv x \cdot (P \otimes Q) \equiv P \otimes (x \cdot Q)
    \and
    P \otimes \pzero \equiv P
  \end{mathpar}

  that the reader is invited to verify.
\end{remark}

\subsubsection{Annihilation}
\begin{mathpar}
  P^{\perp} := \{ Q | \forall R. P|Q \red^{*} R \Rightarrow R \red^{*} \pzero \}
  \and \\
  P^{\underline{\perp}} := \Sigma_{Q \in P^{\perp}} \quotep{Q}?(y).(\dropn{y}|Q) | \Sigma_{Q \in P^{\perp}} \quotep{Q}\clift{\Box}
\end{mathpar}

\paragraph{Discussion} The reader will note that $P^{\perp}$ is a
\emph{set} of processes, while $P^{\underline{\perp}}$ is a
\emph{context}. We call the set $P^{\perp}$ the \emph{annihilators} of
$P$. The parallel composition of a process in the annihilators of $P$
with $P$ will result in a process, the state space of which has all
paths eventually leading to $\pzero$. Execution may endure loops; but
under reasonable conditions of fairness (naturally guaranteed under
most notions of bisimulation) such a composite process cannot get
stuck in such a loop and will, eventually pop out and terminate.

The context $P^{\underline{\perp}}$ is ready and willing to ``take the
$P$ out of'' the process to which it is applied. It will effectively
transmit the code of the process to which it is applied to one of the
annihilators and run the process against it.

\subsubsection{Evaluation}
We fix $M$ a domain of fully abstract interpretation with an equality
coincident with bisimulation. We take $\meaningof{\cdot} : \Proc \to
M$ to be the map interpreting processes and $\nmeaningof{\cdot} : \M
\to Proc$ to be the map running the other way. Then we define

\begin{mathpar}
  \int P := \nmeaningof{\meaningof{P}}
\end{mathpar}

\paragraph{Discussion}
There are many fully abstract interpretations of Milner's
$\pi$-calculus. Any of them can be used as a basis for interpreting
the reflective calculus here. Equipped with such a domain it is
largely a matter of grinding through to check that the Yoneda
construction for the normalization-by-evaluation program can be
extended to this setting.

\begin{remark}
  The reader is invited to verify that $\int (P^{\underline{\perp}}[P]) = 0$.
\end{remark}

\subsection{Quantum mechanics}

Table \ref{tbl:core_qm_op_defns} gives the core operational definitions

\begin{table}[htp]\label{tbl:core_qm_op_defns}
  \center{
    \fbox{
      \begin{tabular}{c|c}
        quantum mechanics & process calculus \\
        \hline
        scalar & $x := \quotep{P}$ \\
        state vector & $\state{P} := P$ \\
        dual & $\state{P}^{*} := \event{P^{\underline{\perp}}} := \quotep{P^{\underline{\perp}}}[-]$ \\
        matrix & $ \Sigma_{\alpha} \state{P_{\alpha}}x_{\alpha}\event{Q_{\alpha}}$ \\
        vector addition & $\state{P} + \state{Q} := \state{P | Q}$ \\
        tensor product & $\state{P} \otimes \state{Q} := \state{P \otimes Q}$ \\
        inner product & $\innerprod{P}{Q} := \quotep{\int P^{\underline{\perp}}[Q]}$ \\
      \end{tabular}
    }
  }
  \caption{QM - operational definitions}
\end{table}

where

\begin{mathpar}
  \prmatrix{P}{Q} := \fprmatrix{P}{\quotep{\pzero}}{Q}
  \and
  \fprmatrix{P}{x}{Q} := (\state{P},x,\event{Q})
  \and
  (\fprmatrix{P}{x}{Q})(\state{R}) := x \cdot \innerprod{Q}{R} \cdot \state{P}
  \and
  (\fprmatrix{P}{x}{Q})(\event{R}) := x \cdot \innerprod{R}{P} \cdot \event{Q}
\end{mathpar}

\paragraph{Discussion}
As promised: vectors (aka states) are represented as processes; duals
as contextual duals; inner product definition should be compared with
standard inner product definition for ....

\begin{remark}
  Assuming $\int (P^{\underline{\perp}}[P]) = 0$, the reader is
  invited to verify that $(\fprmatrix{P}{x}{P})(\state{P}) = x \cdot \state{P}$.
\end{remark}

\begin{remark}
  The reader is invited to verify that $\innerprod{P}{Q}$ could
  equally well have been written $\quotep{\int \stackrel{\vee}{x}}$
  where $x = \event{P^{\underline{\perp}}}(Q)$.

  One of the motivations for this remark is that there is another way
  to factor these operations. We could package up evaluation in the dual:

  \begin{mathpar}
    \state{P}^{*} := \event{\int P^{\underline{\perp}}} := \quotep{\int P^{\underline{\perp}}}[-]
  \end{mathpar}

  and then have inner product defined by
  
  \begin{mathpar}
    \innerprod{P}{Q} := \event{P}(Q)
  \end{mathpar}

  Hopefully, experience with the calculations will provide guidance on
  the best factoring.
\end{remark}

\begin{remark}
  Assuming $\int (P^{\underline{\perp}}[P]) = 0$, the reader is
  invited to verify that $\forall P,Q. (\prmatrix{0}{Q})(\state{0}) =
  \state{0}$ and dually $(\prmatrix{P}{0})(\event{0}) = \event{0}$.
\end{remark}

\begin{remark}
  i'm a little worried that i don't (yet) have proper support for
  complex conjugacy. But, the observation above may give us a
  clue. According to Abramsky, it must be the case that the scalars
  are iso to the homset of the identity for the tensor -- which the
  observation above characterizes. 

  For now, we will simply bookmark the notion with $\overline{x}$.
\end{remark}

\subsubsection{Adjointness}

We need to give a definition of $(\cdot)^{\dagger}$ for matrices. The
obvious candidate definition is
\begin{mathpar}
(\Sigma_{\alpha}\fprmatrix{P_{\alpha}}{x_{\alpha}}{Q_{\alpha}})^{\dagger}
= \Sigma_{\alpha}\fprmatrix{(Q_{\alpha}^{\underline{\perp}})^{*}}{\overline{x}_{\alpha}}{P_{\alpha}^{\underline{\perp}}} 
\end{mathpar}

But, $(Q_{\alpha}^{\underline{\perp}})^{*}$ requires a name along
which to communicate the process to achieve the context application.

\subsubsection{Basis for a basis}
If processes label states and ``addition'' of states (a.k.a. vector
addition) is interpreted as parallel composition, what corresponds to
notions of linear independence and basis? Here, we recall that Yoshida
has developed a set of \emph{combinators} for an asynchronous verison
of Milner's $\pi$-calculus. These are a finite set of processes such
any process can be expressed as parallel composition of these
combinators together with liberal uses of the new operator and
replication. We can simply give a translation of these into the
present calculus and have reasonable expectation that the property
carries over. That is, that the resultant set allows to express all
processes via parallel composition. Note, however, that there is no
new operator or replication in this calculus. As a result, we expect
that the corresponding set is actually infinite. That is, we expect
that the space is actually infinite dimensional.

\begin{remark}
  The attentive reader may be a bit concerned. Certainly, the
  collection $S$, $K$ and $I$ is a finite set of
  combinators. Shouldn't we expect to see a finite set of combinators
  for an effectively equivalent system? i am very sympathetic to this
  critique and feel it warrants full attention. On the other hand, i
  also have in mind the following analogy. The natural numbers, as a
  monoid under addition, has exactly $1$ generator, while the natural
  numbers, as a monoid under multiplication, has countably many
  generators (the primes). We observe that the application of the
  lambda calculus is much less resource sensitive than the parallel
  composition of the $\pi$-calculus. Could it be the case that we have
  an analogy of the form
  
  \begin{mathpar}
    m + n : MN :: m*n : M|N
  \end{mathpar}

  giving a similar blow up in the set of ``primes''?  This is such a
  wonderful thought that, even if it's not true, i think it's worth
  writing down.
\end{remark}
 

\documentclass[12pt]{llncs}
%\documentclass{jktr}

\usepackage[pdftex]{hyperref}                   
\usepackage {listings}
\usepackage {mathpartir}
\usepackage{bcprules}
%\usepackage{listings}
                       
\usepackage{graphicx} 
%\usepackage[margins=2.5cm,nohead,nofoot]{geometry}
%\usepackage{geometry}
\usepackage{amsfonts}
\usepackage{amstext}
\usepackage{latexsym}
\usepackage{amssymb}
\usepackage{color}


%\include{myPreamble}
\include{qm2pi.local} 

%\ifpdf
%\usepackage[pdftex]{graphicx}
%\else
%\usepackage{graphicx}
%\fi

 % \ifpdf
%  \usepackage{pdfsync}
%  \if


%\title{Brief Article}
%\author{David F. Snyder}
%\author{L.G. Meredith}

%\address{Dept. of Math., Texas State University--San Marcos, San Marcos, TX 78666}
       
\pagestyle{empty}


\begin{document}

\lstset{language=[Objective]Caml,frame=shadowbox}

\input{qm2pi.front}

% section front matter (end)

\input{qm2pi.intro} 
 
% section introduction (end)

% \input{qm2pi.knotations} 

% section notation (end)

\input{qm2pi.process.calculi} 

% section concurrent_process_calculi_and_spatial_logics_ (end)
    
%\input{qm2pi.knots2pi} 

%\input{qm2pi.trefoil} 

%\input{qm2pi.mainthm} 

% subsection basic_interpretation (end)

%\input{qm2pi.rho.presentation} 
\subsection{The syntax and semantics of the notation system}\label{sub:the_syntax_and_semantics_of_the_notation_system} % (fold)

We now summarize a technical presentation of the calculus that
embodies our theory of dynamics. The typical presentation of such a
calculus follows the style of giving generators and relations on
them. The grammar, below, describing term constructors, freely
generates the set of processes, $\Proc$. This set is then quotiented
by a relation known as structural congruence and it is over this set
that the notion of dynamics is expressed. This presentation is
essentially that of \cite{MeredithR05} with the addition of
polyadicity and summation. For readability we have relegated some of
the technical subtleties to an appendix.

\subsubsection{Process grammar}\label{subsub:process_grammar}

\begin{mathpar}
  \inferrule* [lab=synchronization] {} {{M} \bc \pzero \;|\; x?F \;|\; x!C }
  \and
  \inferrule* [lab=abstraction] {} {{F} \bc (x)P}
  \and
  \inferrule* [lab=concretion] {} {{C} \bc \langle Q \rangle}
  \and
  \inferrule* [lab=process] {} {{P,Q} \bc M \;| \;P|Q \;|\; @{x}}
  \and
  \inferrule* [lab=name] {} {{x} \bc \quotep{P}}
\end{mathpar} 

Note that $\vec{x}$ (resp. $\vec{P}$) denotes a vector of names
(resp. processes) of length $|\vec{x}|$ (resp. $|\vec{P}|$). We adopt
the following useful abbreviations.

\begin{mathpar}
   x?(\vec{y}).P := x.(\vec{y})P \and  x\clift{\vec{P}} := x.\clift{\vec{P}}
   \and x!(y) := \lift{x}{\dropn{y}}
   \and \Pi_{i=0}^{n-1}P_i := P_0 | \ldots | P_{n-1}
\end{mathpar}

\subsubsection{Structural congruence}

\paragraph{Free and bound names and alpha-equivalence.} At the
core of structural equivalence is alpha-equivalence which identifies
process that are the same up to a change of variable. Formally, we
recognize the distinction between free and bound names. The free names
of a process, $\freenames{P}$, may be calculated recursively as
follows:

\begin{mathpar}
\freenames{\pzero} := \emptyset
  \and \\
  \freenames{x?(y).P} := \{ x \} \cup (\freenames{P} \setminus \{ y \})
  \and 
  \freenames{x!\langle P \rangle} := \{ x \} \cup \{ P \} 
  \and \\
  \freenames{P|Q} := \freenames{P} \cup \freenames{Q}
  \and \\
  \freenames{@{x}} := \{ x \}
\end{mathpar}

$\pi$
$\quotep{\pi}$

$\freenames{-} : \pi \to \mathcal{P}(\quotep{\pi})$

\begin{eqnarray*}
  \freenames{\pzero} & := & \emptyset \\
  \freenames{x?(y).P} & := & \{ x \} \cup (\freenames{P} \setminus \{ y \}) \\
  \freenames{x!\langle P \rangle} & := & \{ x \} \cup \{ P \} \\
  \freenames{P|Q} & := & \freenames{P} \cup \freenames{Q} \\
  \freenames{\dropn{x}} & := & \{ x \}
\end{eqnarray*}

The bound names of a process, $\boundnames{P}$, are those names occurring in $P$
that are not free. For example, in $x?(y).0$, the name $x$ is free, while $y$ is bound.

\begin{mathpar}
  \inferrule* [lab=monoidal-laws] {} { P|Q \equiv Q|P \and P|0 \equiv P \and P|(Q|R) \equiv (P|Q)|R }
\end{mathpar}

\begin{mathpar}
  \inferrule* [lab=alpha-equivalence] {} { (x)P \equiv (y)P\{y/x\} \and y \not\in \freenames{P} }
\end{mathpar}

\begin{definition}
Then two processes, $P,Q$, are alpha-equivalent if $P = Q\{\vec{y}/\vec{x}\}$ for
some $\vec{x} \in \boundnames{Q},\vec{y} \in \boundnames{P}$, where $Q\{\vec{y}/\vec{x}\}$
denotes the capture-avoiding substitution of $\vec{y}$ for $\vec{x}$ in $Q$.
\end{definition}

\begin{definition}
  The {\em structural congruence} \cite{SangiorgiWalker} , $\equiv$,
  between processes is the least congruence containing
  alpha-equivalence, satisfying the abelian monoid laws
  (associativity, commutativity and $\pzero$ as identity) for parallel
  composition $|$ and for summation $+$.
\end{definition}

\subsection{Name equivalence}

We take name equivalence, written $\nameeq$, to be the smallest
equivalence relation generated by the following rules.

\begin{mathpar}
\inferrule*[lab=Quote-drop]
{ }
{ \quotep{@{x}} \nameeq x }

\inferrule*[lab=Struct-equiv]
{ P \scong Q }
{ \quotep{P} \nameeq \quotep{Q} }
\end{mathpar}

The astute reader will have noticed that the mutual recursion of names
and processes imposes a mutual recursion on alpha-equivalence and
structural equivalence via name-equivalence. Fortunately, all of this
works out pleasantly and we may calculate in the natural way, free of
concern. The reader interested in the details is referred to the
appendix \ref{appendix:rho_details}.

\subsection{Substitution}

We use $\Proc$ for the set of processes, $\QProc$ for the set of
names, and $\id{\{}\vec{y} / \vec{x} \id{\}}$ to denote partial maps,
$s : \QProc \rightarrow \QProc$. A map, $s$ lifts, uniquely, to a map
on process terms, $\widehat{s} : \Proc \rightarrow \Proc$ by the
following equations.

\begin{mathpar}
  (0) \psubstp{Q}{P} := 0 \\
  (R \juxtap S) \psubstp{Q}{P}
  :=    
  (R)\psubstp{Q}{P} \juxtap (S) \psubstp{Q}{P} \\
  (x?(y).R) \psubstp{Q}{P}    
  :=    
  (x)\substp{Q}{P} (z)\concat( (R \psubstn{z}{y}) \psubstp{Q}{P} ) \\
  (\lift{x}{R}) \psubstp{Q}{P}  
  :=
  \lift{(x)\substp{Q}{P}}{ R \psubstp{Q}{P} } \\
%   (\dropn{x})  \psubstp{Q}{P}       
%   := 
%   \left\{ 
%     \begin{array}{ccc} 
%       \dropn{\quotep{Q}} & & x \nameeq \quotep{P} \\
%       \dropn{x} & & otherwise \\
%     \end{array}
%   \right. 
  (\dropn{x})  \psubstp{Q}{P}       
  := 
  \left\{ 
    \begin{array}{ccc} 
      Q & & x \nameeq \quotep{P} \\
      \dropn{x} & & otherwise \\
    \end{array}
  \right.
\end{mathpar}
 

where

\begin{eqnarray}
  (x)\id{\{} \lpquote Q \rpquote / \lpquote P \rpquote \id{\}}            = 
  \left\{ 
    \begin{array}{ccc}
      \lpquote Q \rpquote & & x \nameeq \lpquote P \rpquote \\
      x & & otherwise \\
    \end{array}
  \right. \nonumber
\end{eqnarray}

and $z$ is chosen distinct from $\quotep{P}$, $\quotep{Q}$, the free
names in $Q$, and all the names in $R$. Our $\alpha$-equivalence will
be built in the standard way from this substitution.

\begin{remark}\label{rem:no_self_referential_names}
  One consequence of these definitions is that $\forall P. \quotep{P}
  \not\in \freenames{P}$.
\end{remark}

\subsection{ Dynamic quote: an example }

Anticipating something of what's to come, consider applying the
substitution, $\widehat{\id{\{}u / z \id{\}}}$, to the following pair
of processes, $\lift{w}{y!(z)}$ and $w[ \lpquote y!(z) \rpquote ]$.

\begin{eqnarray}
	\lift{w}{y!(z)}\widehat{\id{\{}u / z \id{\}}}
		& = &
		\lift{w}{y!(u)} \nonumber\\
	w[ \lpquote y!(z) \rpquote ] \widehat{ \id{\{}u / z \id{\}} }
		& = &
		w[ \lpquote y!(z) \rpquote ] \nonumber
\end{eqnarray}

Because the body of the process between quotes is impervious to
substitution, we get radically different answers. In fact, by
examining the first process in an input context,
e.g. $x?(z).\lift{w}{y!(z)}$, we see that the process under the lift
operator may be shaped by prefixed inputs binding a name inside it. In
this sense, the lift operator will be seen as a way to dynamically
construct processes before reifying them as names.

Finally equipped with these standard features we can present the
dynamics of the calculus.

\subsubsection{Operational semantics} 

Finally, we introduce the computational dynamics. What marks these
algebras as distinct from other more traditionally studied algebraic
structures, e.g. vector spaces or polynomial rings, is the manner in
which dynamics is captured. In traditional structures, dynamics is typically
expressed through morphisms between such structures, as in linear maps
between vector spaces or morphisms between rings. In algebras
associated with the semantics of computation, the dynamics is
expressed as part of the algebraic structure itself, through a
reduction reduction relation typically denoted by $\red$. Below, we
give a recursive presentation of this relation for the calculus used
in the encoding.

$\red \subseteq \pi \times \pi$
$\red : \pi \to \mathcal{P}(\pi)$

\begin{mathpar}
  \inferrule* [lab=Comm] { \textsf{match}( x_{src}, x_{trgt} ) } { x_{trgt}?(y)P \; | \; x_{src}!\langle {Q} \rangle \red P\{\quotep{Q}/y}\} }
  \and \\
  \inferrule* [lab=Par] {{P} \red {P}'} {{{P} | {Q}} \red {{P}' | {Q}}}
  \and
  \inferrule* [lab=Equiv]{{{P} \scong {P}'} \andalso {{P}' \red {Q}'} \andalso {{Q}' \scong {Q}}}{{P} \red {Q}}
\end{mathpar}

\begin{eqnarray*}
  match_{\equiv} (\quotep{P},\quotep{Q}) & := & P \equiv Q \\
  match_{\dagger}(\quotep{P},\quotep{Q}) & := & \forall R. P|Q \red^{*} R => R \red^{*} 0 \\
  match_{K}(\quotep{P},\quotep{Q}) & := & K \mbox{ for some context } K
\end{eqnarray*}

$u?(x)P | u!\langle Q \rangle \red P\{\quotep{Q}/x\}$

%We write $\wred$ for $\red^*$, and $P\red$ if $\exists Q $ such that $ P \red Q$.
We write $P\red$ if $\exists Q $ such that $ P \red Q$ and $P\not\red$, otherwise.

\section{Replication}

As mentioned before, it is known that replication (and hence
recursion) can be implemented in a higher-order process algebra
\cite{SangiorgiWalker}. As our first example of calculation with the
machinery thus far presented we give the construction explicitly in
the {\rhoc}.

\begin{eqnarray}
	D_{x} & := & \prefix{x}{y}{(\binpar{\outputp{x}{y}}{@{y}})} \nonumber\\
	\bangp_{x}{P} & := & \binpar{{x}!\langle{\binpar{D_{x}}{P}}\rangle}{D_{x}} \nonumber
\end{eqnarray}

\begin{eqnarray}
	\bangp_{x}{P} & & \nonumber\\
	=
	& {x}!\langle{(\prefix{x}{y}{(\outputp{x}{y} | @{y})) | P}}\rangle 
	      | \prefix{x}{y}{(\outputp{x}{y} | @{y})} & \nonumber\\
	\red
	& (\outputp{x}{y} | @{y})\substn{\quotep{(\prefix{x}{y}{(@{y} | \outputp{x}{y})) | P}}}{y} & \nonumber\\
	=
	& \outputp{x}{\quotep{(\prefix{x}{y}{(\outputp{x}{y} | @{y})) | P}}}
	  | {(\prefix{x}{y}{(\outputp{x}{y} | @{y})) | P}} & \nonumber\\
	\red
	& \ldots & \nonumber\\
	\red^*
	& P | P | \ldots & \nonumber
\end{eqnarray}

Of course, this encoding, as an implementation, runs away, unfolding
$\bangp{P}$ eagerly. A lazier and more implementable replication
operator, restricted to input-guarded processes, may be obtained as follows.

\begin{eqnarray}
\bangp{\prefix{u}{v}{P}} 
	:= 
	\binpar{\lift{x}{\prefix{u}{v}{(\binpar{D(x)}{P})}}}{D(x)} \nonumber
\end{eqnarray}

\begin{remark}
  Note that the lazier definition still does not deal with summation
  or mixed summation (i.e. sums over input and output). The reader is
  invited to construct definitions of replication that deal with these
  features. 

  Further, the definitions are parameterized in a name, $x$. Can you,
  gentle reader, make a definition that eliminates this parameter and
  guarantees no accidental interaction between the replication
  machinery and the process being replicated -- i.e. no accidental
  sharing of names used by the process to get its work done and the
  name(s) used by the replication to effect copying. This latter
  revision of the definition of replication is crucial to obtaining
  the expected identity $!!P \sim !P$.
\end{remark}

\begin{remark}\label{rem:paradoxical_combinator}
  The reader familiar with the lambda calculus will have noticed the
  similarity between $D$ and the paradoxical combinator.

  [Ed. note: the existence of this seems to suggest we have to be more
  restrictive on the set of processes and names we admit if we are to
  support no-cloning.]
\end{remark}

\subsubsection{Bisimulation}

The computational dynamics gives rise to another kind of equivalence,
the equivalence of computational behavior. As previously mentioned
this is typically captured \emph{via} some form of bisimulation.

% The notion we use in this paper is weak barbed bisimulation
% \cite{milner91polyadicpi}.

The notion we use in this paper is derived from weak barbed
bisimulation \cite{milner91polyadicpi}. 

\begin{definition}
An \emph{observation relation}, $\downarrow_{\mathcal N}$, over a set
of names, $\mathcal N$, is the smallest relation satisfying the rules
below.

\infrule[Out-barb]{y \in {\mathcal N}, \; x \nameeq y}
		  {\outputp{x}{v} \downarrow_{\mathcal N} x}
\infrule[Par-barb]{\mbox{$P\downarrow_{\mathcal N} x$ or $Q\downarrow_{\mathcal N} x$}}
		  {\binpar{P}{Q} \downarrow_{\mathcal N} x}

We write $P \Downarrow_{\mathcal N} x$ if there is $Q$ such that 
$P \wred Q$ and $Q \downarrow_{\mathcal N} x$.
\end{definition}

\begin{definition}
%\label{def.bbisim}
An  ${\mathcal N}$-\emph{barbed bisimulation} over a set of names, ${\mathcal N}$, is a symmetric binary relation 
${\mathcal S}_{\mathcal N}$ between agents such that $P\rel{S}_{\mathcal N}Q$ implies:
\begin{enumerate}
\item If $P \red P'$ then $Q \wred Q'$ and $P'\rel{S}_{\mathcal N} Q'$.
\item If $P\downarrow_{\mathcal N} x$, then $Q\Downarrow_{\mathcal N} x$.
\end{enumerate}
$P$ is ${\mathcal N}$-barbed bisimilar to $Q$, written
$P \wbbisim_{\mathcal N} Q$, if $P \rel{S}_{\mathcal N} Q$ for some ${\mathcal N}$-barbed bisimulation ${\mathcal S}_{\mathcal N}$.
\end{definition}

$\mathcal{R} \subseteq \pi \times \pi$

$P \mathcal{R} Q => \forall P'. P \red P' \Rightarrow \exists Q'. Q \red Q', P' \mathcal{R} Q'$

$P \vdash x \Rightarrow Q \vdash x$

\begin{mathpar}
  \inferrule*[lab=Out-barb]{x \nameeq y}{{y}!\langle{Q}\rangle \vdash x}
  \and
  \inferrule*[lab=Par-barb]{\mbox{$P\vdash x$ or $Q\vdash x$}}{\binpar{P}{Q} \vdash x}
\end{mathpar}

\subsubsection{Contexts}

One of the principle advantages of computational calculi like the
$\pi$-calculus is a well-defined notion of context,
contextual-equivalence and a correlation between
contextual-equivalence and notions of bisimulation. The notion of
context allows the decomposition of a process into (sub-)process and
its syntactic environment, its context. Thus, a context may be
thought of as a process with a ``hole'' (written $\Box$) in it. The
application of a context $M$ to a process $P$, written $M[P]$, is
tantamount to filling the hole in $M$ with $P$. In this paper we do
not need the full weight of this theory, but do make use of the notion
of context in the proof the main theorem. 

\begin{mathpar}
  \inferrule* [lab=summation] {} {{M_{M},M_{N}} \bc \Box \;|\; x.M_{A} \;|\; M_{M}+M_{N}}
  \and
  \inferrule* [lab=agent] {} {{M_{A}} \bc (\vec{x})M_{P} \;| \; \clift{P_0,\ldots,M_{P},\ldots,P_N}}
  \and \\
  \inferrule* [lab=process] {} {{M_{P}} \bc M_{N} \;| \;P|M_{P} }
\end{mathpar} 

\begin{mathpar}
  \inferrule* [lab=sychronization] {} {M_{N} \bc \Box \;|\; x?M_{F} \;|\; x!M_{C}}
  \and
  \inferrule* [lab=abstraction] {} {{M_{F}} \bc (x)M_{P} }
  \and
  \inferrule* [lab=concretion] {} {{M_{C}} \bc \langle M_{P} \rangle }
  \and \\
  \inferrule* [lab=process] {} {{M_{P}} \bc M_{N} \;| \;P|M_{P} }
\end{mathpar}

\begin{definition}[contextual application] Given a context $M$, and
  process $P$, we define the \emph{contextual application}, $M[P] :=
  M\{P/\Box\}$. That is, the contextual application of M to P is the
  substitution of $P$ for $\Box$ in $M$.
\end{definition}

$\meaningof{-} : L \to \mathcal{P}(\pi)$

\begin{mathpar}
  \inferrule* [lab=collection] {} {\meaningof{true} = \pi, \and \meaningof{~E} = \pi \setminus \meaningof{E}, \and \meaningof{E_{1} \& E_{2}} = \meaningof{E_{1}} \cap \meaningof{E_{2}}}
\end{mathpar}

\begin{mathpar}
  \inferrule* [lab=structure] {} {\meaningof{0} = \{ P \in \pi | P \equiv 0 \}, \and \\ \meaningof{E_1 | E_2} = \{ P \in \pi | P \equiv P_{1} | P_{2}, P_{1} \in \meaningof{E_{1}}, P_{2} \in \meaningof{E_2}\} }
\end{mathpar}

\begin{mathpar}
 \inferrule* [lab=behavior] {} {\meaningof{\langle a?b \rangle E} = \{ P \in \pi | P \equiv Q | u?(y)P', \\ \and \\\\ \and \\ \;\;\; u \in \meaningof{a}, \forall z.P'\{z/y\} \in \meaningof{E\{z/b\}}\}, \and \\ \meaningof{a!E} = \{ P \in \pi | P \equiv Q | x!\langle P' \rangle, x \in \meaningof{a} P' \in \meaningof{E}\} }
\end{mathpar}

\begin{mathpar}
 \inferrule* [lab=nominal] {} {\meaningof{\quotep{E}} = \{ \quotep{P} \in \quotep{\pi} | P \in \meaningof{E} \}, \and \meaningof{\quotep{P}} = \{ \quotep{Q} \in \quotep{\pi} | P \equiv Q \} \and \\ \meaningof{@\quotep{E}} = \{ P \in \pi | P \equiv @x, x \in \meaningof{E} \}}
\end{mathpar}

\begin{eqnarray*}
  \\
  \meaningof{-} : TS \to ST
\end{eqnarray*}

\begin{eqnarray*}
  \\
  L : TS \to ST
\end{eqnarray*}

\begin{eqnarray*}
  \\
  P \models E \iff P \in \meaningof{E}
\end{eqnarray*}

\begin{eqnarray*}
  P \approx_{L} Q \iff \forall E \in L. P \models E \iff Q \models E
\end{eqnarray*}

\begin{eqnarray*}
  P \approx_{K} Q
\end{eqnarray*}

\begin{eqnarray*}
  P \approx Q
\end{eqnarray*}

$\approx_{K} = \approx = \approx_{L}$

\subsubsection{Contextual duality}

Note that contexts extend the quotation operation to a family of
operations from processes to names. Given a context, $M$, we can
define a \emph{nominal context}, $\quotep{M}$ by $\quotep{M}[P] :=
\quotep{M[P]}$. To foreshadow what is to come we observe that these
operations enjoy a duality with processes very much like the duality
between vectors and maps from vectors to scalars.

Further, because the calculus is essentially higher-order, we have a
correspondence between contexts and processes. More specifically,
given a name $x$ and a context $M$ we can construct $M^{*}_{x}$ such
that 

\begin{mathpar}
  M^{*}_{x} | \lift{x}{P} \red M[P]
\end{mathpar}

namely,

\begin{mathpar}
  M^{*}_{x} := x?(u).M[\dropn{u}]
\end{mathpar}

The dependence of $M^{*}_{x}$ on a name makes it an abstraction, 

\begin{mathpar}
  M^{*} := (x)x?(u).M[\dropn{u}]
\end{mathpar}

\subsection{Additional notation}

It will sometimes be convenient to denote the process a name
quotes. We already have the notation $x = \quotep{P}$, but it will be
convenient to introduce an alternate notation, $\procn{x}$, when we
want to emphasize the connection to the use of the name. Note that, by
virtue of name equivalence, $\quotep{\procn{x}} \nameeq x$; so, the
notation is consistent with previous definitions.

Further, because names have structure it is possible to effect
substitutions on the basis of that structure. This means we need to
upgrade our notation for substitutions, which we accomplish by
adapting comprehension notation. Thus,

\begin{mathpar}
  P\{ y / x : x \in S \}
\end{mathpar}

is interpreted to mean the process derived from P by replacing (in a
capture-avoiding manner) each occurrence of $x$ in $S$ by $y$. For example,

\begin{mathpar}
  P\{ \quotep{\procn{x}|\procn{x}} / x : x \in \freenames{P} \}
\end{mathpar}

will replace each (occurrence) of a free name $x$ in $P$ by
$\quotep{\procn{x}|\procn{x}}$.

Also, we will avail ourselves of the notation $x^{L}$ and $x^{R}$ to
denote injections of a name into disjoint copies of the name
space. There are numerous ways to accomplish this. One example can be
found in \cite{MeredithR05}. This notation overloads to vectors of
names: $\vec{x}^{\pi} := (x_{i}^{\pi} \; : \; 0 \leq i < |\vec{x}| )$ where $\pi \in \{L,R\}$.

We also use $P^{\Box} := P|\Box$.

In \cite{MeredithR05} an interpretation of the new operator is
given. It turns out that there are several possible interpretations
all enjoying the requisite algebraic properties of the operator (see
\cite{milner91polyadicpi}). We will therefore make liberal use of
$(\nu\; \vec{x})P$.

% subsection the_syntax_and_semantics_of_the_notation_system (end)   

\input{qm2pi.qmops} 

\input{qm2pi.sterngerlach} 

\input{qm2pi.metric} 

% section concurrent_process_calculi (end)

%\input{qm2pi.proofsketch}

% section proof sketch (end)

%\input{qm2pi.slviaknots} 

% section spatial logic via knots (end)

\input{qm2pi.conclusion}

% section conclusion (end)

%\input{qm2pi.dtcodes} 

% section wiring algorithm (end)

\input{qm2pi.ack} 

% section acknowledgments (end)

\newpage


\bibliographystyle{plain}   
\bibliography{../../biblios/main.bib}

\input{qm2pi.rhodetails}

\end{document}

 

\documentclass[12pt]{llncs}
%\documentclass{jktr}

\usepackage[pdftex]{hyperref}                   
\usepackage {listings}
\usepackage {mathpartir}
\usepackage{bcprules}
%\usepackage{listings}
                       
\usepackage{graphicx} 
%\usepackage[margins=2.5cm,nohead,nofoot]{geometry}
%\usepackage{geometry}
\usepackage{amsfonts}
\usepackage{amstext}
\usepackage{latexsym}
\usepackage{amssymb}
\usepackage{color}


%\include{myPreamble}
\include{qm2pi.local} 

%\ifpdf
%\usepackage[pdftex]{graphicx}
%\else
%\usepackage{graphicx}
%\fi

 % \ifpdf
%  \usepackage{pdfsync}
%  \if


%\title{Brief Article}
%\author{David F. Snyder}
%\author{L.G. Meredith}

%\address{Dept. of Math., Texas State University--San Marcos, San Marcos, TX 78666}
       
\pagestyle{empty}


\begin{document}

\lstset{language=[Objective]Caml,frame=shadowbox}

\input{qm2pi.front}

% section front matter (end)

\input{qm2pi.intro} 
 
% section introduction (end)

% \input{qm2pi.knotations} 

% section notation (end)

\input{qm2pi.process.calculi} 

% section concurrent_process_calculi_and_spatial_logics_ (end)
    
%\input{qm2pi.knots2pi} 

%\input{qm2pi.trefoil} 

%\input{qm2pi.mainthm} 

% subsection basic_interpretation (end)

%\input{qm2pi.rho.presentation} 
\subsection{The syntax and semantics of the notation system}\label{sub:the_syntax_and_semantics_of_the_notation_system} % (fold)

We now summarize a technical presentation of the calculus that
embodies our theory of dynamics. The typical presentation of such a
calculus follows the style of giving generators and relations on
them. The grammar, below, describing term constructors, freely
generates the set of processes, $\Proc$. This set is then quotiented
by a relation known as structural congruence and it is over this set
that the notion of dynamics is expressed. This presentation is
essentially that of \cite{MeredithR05} with the addition of
polyadicity and summation. For readability we have relegated some of
the technical subtleties to an appendix.

\subsubsection{Process grammar}\label{subsub:process_grammar}

\begin{mathpar}
  \inferrule* [lab=synchronization] {} {{M} \bc \pzero \;|\; x?F \;|\; x!C }
  \and
  \inferrule* [lab=abstraction] {} {{F} \bc (x)P}
  \and
  \inferrule* [lab=concretion] {} {{C} \bc \langle Q \rangle}
  \and
  \inferrule* [lab=process] {} {{P,Q} \bc M \;| \;P|Q \;|\; @{x}}
  \and
  \inferrule* [lab=name] {} {{x} \bc \quotep{P}}
\end{mathpar} 

Note that $\vec{x}$ (resp. $\vec{P}$) denotes a vector of names
(resp. processes) of length $|\vec{x}|$ (resp. $|\vec{P}|$). We adopt
the following useful abbreviations.

\begin{mathpar}
   x?(\vec{y}).P := x.(\vec{y})P \and  x\clift{\vec{P}} := x.\clift{\vec{P}}
   \and x!(y) := \lift{x}{\dropn{y}}
   \and \Pi_{i=0}^{n-1}P_i := P_0 | \ldots | P_{n-1}
\end{mathpar}

\subsubsection{Structural congruence}

\paragraph{Free and bound names and alpha-equivalence.} At the
core of structural equivalence is alpha-equivalence which identifies
process that are the same up to a change of variable. Formally, we
recognize the distinction between free and bound names. The free names
of a process, $\freenames{P}$, may be calculated recursively as
follows:

\begin{mathpar}
\freenames{\pzero} := \emptyset
  \and \\
  \freenames{x?(y).P} := \{ x \} \cup (\freenames{P} \setminus \{ y \})
  \and 
  \freenames{x!\langle P \rangle} := \{ x \} \cup \{ P \} 
  \and \\
  \freenames{P|Q} := \freenames{P} \cup \freenames{Q}
  \and \\
  \freenames{@{x}} := \{ x \}
\end{mathpar}

$\pi$
$\quotep{\pi}$

$\freenames{-} : \pi \to \mathcal{P}(\quotep{\pi})$

\begin{eqnarray*}
  \freenames{\pzero} & := & \emptyset \\
  \freenames{x?(y).P} & := & \{ x \} \cup (\freenames{P} \setminus \{ y \}) \\
  \freenames{x!\langle P \rangle} & := & \{ x \} \cup \{ P \} \\
  \freenames{P|Q} & := & \freenames{P} \cup \freenames{Q} \\
  \freenames{\dropn{x}} & := & \{ x \}
\end{eqnarray*}

The bound names of a process, $\boundnames{P}$, are those names occurring in $P$
that are not free. For example, in $x?(y).0$, the name $x$ is free, while $y$ is bound.

\begin{mathpar}
  \inferrule* [lab=monoidal-laws] {} { P|Q \equiv Q|P \and P|0 \equiv P \and P|(Q|R) \equiv (P|Q)|R }
\end{mathpar}

\begin{mathpar}
  \inferrule* [lab=alpha-equivalence] {} { (x)P \equiv (y)P\{y/x\} \and y \not\in \freenames{P} }
\end{mathpar}

\begin{definition}
Then two processes, $P,Q$, are alpha-equivalent if $P = Q\{\vec{y}/\vec{x}\}$ for
some $\vec{x} \in \boundnames{Q},\vec{y} \in \boundnames{P}$, where $Q\{\vec{y}/\vec{x}\}$
denotes the capture-avoiding substitution of $\vec{y}$ for $\vec{x}$ in $Q$.
\end{definition}

\begin{definition}
  The {\em structural congruence} \cite{SangiorgiWalker} , $\equiv$,
  between processes is the least congruence containing
  alpha-equivalence, satisfying the abelian monoid laws
  (associativity, commutativity and $\pzero$ as identity) for parallel
  composition $|$ and for summation $+$.
\end{definition}

\subsection{Name equivalence}

We take name equivalence, written $\nameeq$, to be the smallest
equivalence relation generated by the following rules.

\begin{mathpar}
\inferrule*[lab=Quote-drop]
{ }
{ \quotep{@{x}} \nameeq x }

\inferrule*[lab=Struct-equiv]
{ P \scong Q }
{ \quotep{P} \nameeq \quotep{Q} }
\end{mathpar}

The astute reader will have noticed that the mutual recursion of names
and processes imposes a mutual recursion on alpha-equivalence and
structural equivalence via name-equivalence. Fortunately, all of this
works out pleasantly and we may calculate in the natural way, free of
concern. The reader interested in the details is referred to the
appendix \ref{appendix:rho_details}.

\subsection{Substitution}

We use $\Proc$ for the set of processes, $\QProc$ for the set of
names, and $\id{\{}\vec{y} / \vec{x} \id{\}}$ to denote partial maps,
$s : \QProc \rightarrow \QProc$. A map, $s$ lifts, uniquely, to a map
on process terms, $\widehat{s} : \Proc \rightarrow \Proc$ by the
following equations.

\begin{mathpar}
  (0) \psubstp{Q}{P} := 0 \\
  (R \juxtap S) \psubstp{Q}{P}
  :=    
  (R)\psubstp{Q}{P} \juxtap (S) \psubstp{Q}{P} \\
  (x?(y).R) \psubstp{Q}{P}    
  :=    
  (x)\substp{Q}{P} (z)\concat( (R \psubstn{z}{y}) \psubstp{Q}{P} ) \\
  (\lift{x}{R}) \psubstp{Q}{P}  
  :=
  \lift{(x)\substp{Q}{P}}{ R \psubstp{Q}{P} } \\
%   (\dropn{x})  \psubstp{Q}{P}       
%   := 
%   \left\{ 
%     \begin{array}{ccc} 
%       \dropn{\quotep{Q}} & & x \nameeq \quotep{P} \\
%       \dropn{x} & & otherwise \\
%     \end{array}
%   \right. 
  (\dropn{x})  \psubstp{Q}{P}       
  := 
  \left\{ 
    \begin{array}{ccc} 
      Q & & x \nameeq \quotep{P} \\
      \dropn{x} & & otherwise \\
    \end{array}
  \right.
\end{mathpar}
 

where

\begin{eqnarray}
  (x)\id{\{} \lpquote Q \rpquote / \lpquote P \rpquote \id{\}}            = 
  \left\{ 
    \begin{array}{ccc}
      \lpquote Q \rpquote & & x \nameeq \lpquote P \rpquote \\
      x & & otherwise \\
    \end{array}
  \right. \nonumber
\end{eqnarray}

and $z$ is chosen distinct from $\quotep{P}$, $\quotep{Q}$, the free
names in $Q$, and all the names in $R$. Our $\alpha$-equivalence will
be built in the standard way from this substitution.

\begin{remark}\label{rem:no_self_referential_names}
  One consequence of these definitions is that $\forall P. \quotep{P}
  \not\in \freenames{P}$.
\end{remark}

\subsection{ Dynamic quote: an example }

Anticipating something of what's to come, consider applying the
substitution, $\widehat{\id{\{}u / z \id{\}}}$, to the following pair
of processes, $\lift{w}{y!(z)}$ and $w[ \lpquote y!(z) \rpquote ]$.

\begin{eqnarray}
	\lift{w}{y!(z)}\widehat{\id{\{}u / z \id{\}}}
		& = &
		\lift{w}{y!(u)} \nonumber\\
	w[ \lpquote y!(z) \rpquote ] \widehat{ \id{\{}u / z \id{\}} }
		& = &
		w[ \lpquote y!(z) \rpquote ] \nonumber
\end{eqnarray}

Because the body of the process between quotes is impervious to
substitution, we get radically different answers. In fact, by
examining the first process in an input context,
e.g. $x?(z).\lift{w}{y!(z)}$, we see that the process under the lift
operator may be shaped by prefixed inputs binding a name inside it. In
this sense, the lift operator will be seen as a way to dynamically
construct processes before reifying them as names.

Finally equipped with these standard features we can present the
dynamics of the calculus.

\subsubsection{Operational semantics} 

Finally, we introduce the computational dynamics. What marks these
algebras as distinct from other more traditionally studied algebraic
structures, e.g. vector spaces or polynomial rings, is the manner in
which dynamics is captured. In traditional structures, dynamics is typically
expressed through morphisms between such structures, as in linear maps
between vector spaces or morphisms between rings. In algebras
associated with the semantics of computation, the dynamics is
expressed as part of the algebraic structure itself, through a
reduction reduction relation typically denoted by $\red$. Below, we
give a recursive presentation of this relation for the calculus used
in the encoding.

$\red \subseteq \pi \times \pi$
$\red : \pi \to \mathcal{P}(\pi)$

\begin{mathpar}
  \inferrule* [lab=Comm] { \textsf{match}( x_{src}, x_{trgt} ) } { x_{trgt}?(y)P \; | \; x_{src}!\langle {Q} \rangle \red P\{\quotep{Q}/y}\} }
  \and \\
  \inferrule* [lab=Par] {{P} \red {P}'} {{{P} | {Q}} \red {{P}' | {Q}}}
  \and
  \inferrule* [lab=Equiv]{{{P} \scong {P}'} \andalso {{P}' \red {Q}'} \andalso {{Q}' \scong {Q}}}{{P} \red {Q}}
\end{mathpar}

\begin{eqnarray*}
  match_{\equiv} (\quotep{P},\quotep{Q}) & := & P \equiv Q \\
  match_{\dagger}(\quotep{P},\quotep{Q}) & := & \forall R. P|Q \red^{*} R => R \red^{*} 0 \\
  match_{K}(\quotep{P},\quotep{Q}) & := & K \mbox{ for some context } K
\end{eqnarray*}

$u?(x)P | u!\langle Q \rangle \red P\{\quotep{Q}/x\}$

%We write $\wred$ for $\red^*$, and $P\red$ if $\exists Q $ such that $ P \red Q$.
We write $P\red$ if $\exists Q $ such that $ P \red Q$ and $P\not\red$, otherwise.

\section{Replication}

As mentioned before, it is known that replication (and hence
recursion) can be implemented in a higher-order process algebra
\cite{SangiorgiWalker}. As our first example of calculation with the
machinery thus far presented we give the construction explicitly in
the {\rhoc}.

\begin{eqnarray}
	D_{x} & := & \prefix{x}{y}{(\binpar{\outputp{x}{y}}{@{y}})} \nonumber\\
	\bangp_{x}{P} & := & \binpar{{x}!\langle{\binpar{D_{x}}{P}}\rangle}{D_{x}} \nonumber
\end{eqnarray}

\begin{eqnarray}
	\bangp_{x}{P} & & \nonumber\\
	=
	& {x}!\langle{(\prefix{x}{y}{(\outputp{x}{y} | @{y})) | P}}\rangle 
	      | \prefix{x}{y}{(\outputp{x}{y} | @{y})} & \nonumber\\
	\red
	& (\outputp{x}{y} | @{y})\substn{\quotep{(\prefix{x}{y}{(@{y} | \outputp{x}{y})) | P}}}{y} & \nonumber\\
	=
	& \outputp{x}{\quotep{(\prefix{x}{y}{(\outputp{x}{y} | @{y})) | P}}}
	  | {(\prefix{x}{y}{(\outputp{x}{y} | @{y})) | P}} & \nonumber\\
	\red
	& \ldots & \nonumber\\
	\red^*
	& P | P | \ldots & \nonumber
\end{eqnarray}

Of course, this encoding, as an implementation, runs away, unfolding
$\bangp{P}$ eagerly. A lazier and more implementable replication
operator, restricted to input-guarded processes, may be obtained as follows.

\begin{eqnarray}
\bangp{\prefix{u}{v}{P}} 
	:= 
	\binpar{\lift{x}{\prefix{u}{v}{(\binpar{D(x)}{P})}}}{D(x)} \nonumber
\end{eqnarray}

\begin{remark}
  Note that the lazier definition still does not deal with summation
  or mixed summation (i.e. sums over input and output). The reader is
  invited to construct definitions of replication that deal with these
  features. 

  Further, the definitions are parameterized in a name, $x$. Can you,
  gentle reader, make a definition that eliminates this parameter and
  guarantees no accidental interaction between the replication
  machinery and the process being replicated -- i.e. no accidental
  sharing of names used by the process to get its work done and the
  name(s) used by the replication to effect copying. This latter
  revision of the definition of replication is crucial to obtaining
  the expected identity $!!P \sim !P$.
\end{remark}

\begin{remark}\label{rem:paradoxical_combinator}
  The reader familiar with the lambda calculus will have noticed the
  similarity between $D$ and the paradoxical combinator.

  [Ed. note: the existence of this seems to suggest we have to be more
  restrictive on the set of processes and names we admit if we are to
  support no-cloning.]
\end{remark}

\subsubsection{Bisimulation}

The computational dynamics gives rise to another kind of equivalence,
the equivalence of computational behavior. As previously mentioned
this is typically captured \emph{via} some form of bisimulation.

% The notion we use in this paper is weak barbed bisimulation
% \cite{milner91polyadicpi}.

The notion we use in this paper is derived from weak barbed
bisimulation \cite{milner91polyadicpi}. 

\begin{definition}
An \emph{observation relation}, $\downarrow_{\mathcal N}$, over a set
of names, $\mathcal N$, is the smallest relation satisfying the rules
below.

\infrule[Out-barb]{y \in {\mathcal N}, \; x \nameeq y}
		  {\outputp{x}{v} \downarrow_{\mathcal N} x}
\infrule[Par-barb]{\mbox{$P\downarrow_{\mathcal N} x$ or $Q\downarrow_{\mathcal N} x$}}
		  {\binpar{P}{Q} \downarrow_{\mathcal N} x}

We write $P \Downarrow_{\mathcal N} x$ if there is $Q$ such that 
$P \wred Q$ and $Q \downarrow_{\mathcal N} x$.
\end{definition}

\begin{definition}
%\label{def.bbisim}
An  ${\mathcal N}$-\emph{barbed bisimulation} over a set of names, ${\mathcal N}$, is a symmetric binary relation 
${\mathcal S}_{\mathcal N}$ between agents such that $P\rel{S}_{\mathcal N}Q$ implies:
\begin{enumerate}
\item If $P \red P'$ then $Q \wred Q'$ and $P'\rel{S}_{\mathcal N} Q'$.
\item If $P\downarrow_{\mathcal N} x$, then $Q\Downarrow_{\mathcal N} x$.
\end{enumerate}
$P$ is ${\mathcal N}$-barbed bisimilar to $Q$, written
$P \wbbisim_{\mathcal N} Q$, if $P \rel{S}_{\mathcal N} Q$ for some ${\mathcal N}$-barbed bisimulation ${\mathcal S}_{\mathcal N}$.
\end{definition}

$\mathcal{R} \subseteq \pi \times \pi$

$P \mathcal{R} Q => \forall P'. P \red P' \Rightarrow \exists Q'. Q \red Q', P' \mathcal{R} Q'$

$P \vdash x \Rightarrow Q \vdash x$

\begin{mathpar}
  \inferrule*[lab=Out-barb]{x \nameeq y}{{y}!\langle{Q}\rangle \vdash x}
  \and
  \inferrule*[lab=Par-barb]{\mbox{$P\vdash x$ or $Q\vdash x$}}{\binpar{P}{Q} \vdash x}
\end{mathpar}

\subsubsection{Contexts}

One of the principle advantages of computational calculi like the
$\pi$-calculus is a well-defined notion of context,
contextual-equivalence and a correlation between
contextual-equivalence and notions of bisimulation. The notion of
context allows the decomposition of a process into (sub-)process and
its syntactic environment, its context. Thus, a context may be
thought of as a process with a ``hole'' (written $\Box$) in it. The
application of a context $M$ to a process $P$, written $M[P]$, is
tantamount to filling the hole in $M$ with $P$. In this paper we do
not need the full weight of this theory, but do make use of the notion
of context in the proof the main theorem. 

\begin{mathpar}
  \inferrule* [lab=summation] {} {{M_{M},M_{N}} \bc \Box \;|\; x.M_{A} \;|\; M_{M}+M_{N}}
  \and
  \inferrule* [lab=agent] {} {{M_{A}} \bc (\vec{x})M_{P} \;| \; \clift{P_0,\ldots,M_{P},\ldots,P_N}}
  \and \\
  \inferrule* [lab=process] {} {{M_{P}} \bc M_{N} \;| \;P|M_{P} }
\end{mathpar} 

\begin{mathpar}
  \inferrule* [lab=sychronization] {} {M_{N} \bc \Box \;|\; x?M_{F} \;|\; x!M_{C}}
  \and
  \inferrule* [lab=abstraction] {} {{M_{F}} \bc (x)M_{P} }
  \and
  \inferrule* [lab=concretion] {} {{M_{C}} \bc \langle M_{P} \rangle }
  \and \\
  \inferrule* [lab=process] {} {{M_{P}} \bc M_{N} \;| \;P|M_{P} }
\end{mathpar}

\begin{definition}[contextual application] Given a context $M$, and
  process $P$, we define the \emph{contextual application}, $M[P] :=
  M\{P/\Box\}$. That is, the contextual application of M to P is the
  substitution of $P$ for $\Box$ in $M$.
\end{definition}

$\meaningof{-} : L \to \mathcal{P}(\pi)$

\begin{mathpar}
  \inferrule* [lab=collection] {} {\meaningof{true} = \pi, \and \meaningof{~E} = \pi \setminus \meaningof{E}, \and \meaningof{E_{1} \& E_{2}} = \meaningof{E_{1}} \cap \meaningof{E_{2}}}
\end{mathpar}

\begin{mathpar}
  \inferrule* [lab=structure] {} {\meaningof{0} = \{ P \in \pi | P \equiv 0 \}, \and \\ \meaningof{E_1 | E_2} = \{ P \in \pi | P \equiv P_{1} | P_{2}, P_{1} \in \meaningof{E_{1}}, P_{2} \in \meaningof{E_2}\} }
\end{mathpar}

\begin{mathpar}
 \inferrule* [lab=behavior] {} {\meaningof{\langle a?b \rangle E} = \{ P \in \pi | P \equiv Q | u?(y)P', \\ \and \\\\ \and \\ \;\;\; u \in \meaningof{a}, \forall z.P'\{z/y\} \in \meaningof{E\{z/b\}}\}, \and \\ \meaningof{a!E} = \{ P \in \pi | P \equiv Q | x!\langle P' \rangle, x \in \meaningof{a} P' \in \meaningof{E}\} }
\end{mathpar}

\begin{mathpar}
 \inferrule* [lab=nominal] {} {\meaningof{\quotep{E}} = \{ \quotep{P} \in \quotep{\pi} | P \in \meaningof{E} \}, \and \meaningof{\quotep{P}} = \{ \quotep{Q} \in \quotep{\pi} | P \equiv Q \} \and \\ \meaningof{@\quotep{E}} = \{ P \in \pi | P \equiv @x, x \in \meaningof{E} \}}
\end{mathpar}

\begin{eqnarray*}
  \\
  \meaningof{-} : TS \to ST
\end{eqnarray*}

\begin{eqnarray*}
  \\
  L : TS \to ST
\end{eqnarray*}

\begin{eqnarray*}
  \\
  P \models E \iff P \in \meaningof{E}
\end{eqnarray*}

\begin{eqnarray*}
  P \approx_{L} Q \iff \forall E \in L. P \models E \iff Q \models E
\end{eqnarray*}

\begin{eqnarray*}
  P \approx_{K} Q
\end{eqnarray*}

\begin{eqnarray*}
  P \approx Q
\end{eqnarray*}

$\approx_{K} = \approx = \approx_{L}$

\subsubsection{Contextual duality}

Note that contexts extend the quotation operation to a family of
operations from processes to names. Given a context, $M$, we can
define a \emph{nominal context}, $\quotep{M}$ by $\quotep{M}[P] :=
\quotep{M[P]}$. To foreshadow what is to come we observe that these
operations enjoy a duality with processes very much like the duality
between vectors and maps from vectors to scalars.

Further, because the calculus is essentially higher-order, we have a
correspondence between contexts and processes. More specifically,
given a name $x$ and a context $M$ we can construct $M^{*}_{x}$ such
that 

\begin{mathpar}
  M^{*}_{x} | \lift{x}{P} \red M[P]
\end{mathpar}

namely,

\begin{mathpar}
  M^{*}_{x} := x?(u).M[\dropn{u}]
\end{mathpar}

The dependence of $M^{*}_{x}$ on a name makes it an abstraction, 

\begin{mathpar}
  M^{*} := (x)x?(u).M[\dropn{u}]
\end{mathpar}

\subsection{Additional notation}

It will sometimes be convenient to denote the process a name
quotes. We already have the notation $x = \quotep{P}$, but it will be
convenient to introduce an alternate notation, $\procn{x}$, when we
want to emphasize the connection to the use of the name. Note that, by
virtue of name equivalence, $\quotep{\procn{x}} \nameeq x$; so, the
notation is consistent with previous definitions.

Further, because names have structure it is possible to effect
substitutions on the basis of that structure. This means we need to
upgrade our notation for substitutions, which we accomplish by
adapting comprehension notation. Thus,

\begin{mathpar}
  P\{ y / x : x \in S \}
\end{mathpar}

is interpreted to mean the process derived from P by replacing (in a
capture-avoiding manner) each occurrence of $x$ in $S$ by $y$. For example,

\begin{mathpar}
  P\{ \quotep{\procn{x}|\procn{x}} / x : x \in \freenames{P} \}
\end{mathpar}

will replace each (occurrence) of a free name $x$ in $P$ by
$\quotep{\procn{x}|\procn{x}}$.

Also, we will avail ourselves of the notation $x^{L}$ and $x^{R}$ to
denote injections of a name into disjoint copies of the name
space. There are numerous ways to accomplish this. One example can be
found in \cite{MeredithR05}. This notation overloads to vectors of
names: $\vec{x}^{\pi} := (x_{i}^{\pi} \; : \; 0 \leq i < |\vec{x}| )$ where $\pi \in \{L,R\}$.

We also use $P^{\Box} := P|\Box$.

In \cite{MeredithR05} an interpretation of the new operator is
given. It turns out that there are several possible interpretations
all enjoying the requisite algebraic properties of the operator (see
\cite{milner91polyadicpi}). We will therefore make liberal use of
$(\nu\; \vec{x})P$.

% subsection the_syntax_and_semantics_of_the_notation_system (end)   

\input{qm2pi.qmops} 

\input{qm2pi.sterngerlach} 

\input{qm2pi.metric} 

% section concurrent_process_calculi (end)

%\input{qm2pi.proofsketch}

% section proof sketch (end)

%\input{qm2pi.slviaknots} 

% section spatial logic via knots (end)

\input{qm2pi.conclusion}

% section conclusion (end)

%\input{qm2pi.dtcodes} 

% section wiring algorithm (end)

\input{qm2pi.ack} 

% section acknowledgments (end)

\newpage


\bibliographystyle{plain}   
\bibliography{../../biblios/main.bib}

\input{qm2pi.rhodetails}

\end{document}

 

% section concurrent_process_calculi (end)

%\documentclass[12pt]{llncs}
%\documentclass{jktr}

\usepackage[pdftex]{hyperref}                   
\usepackage {listings}
\usepackage {mathpartir}
\usepackage{bcprules}
%\usepackage{listings}
                       
\usepackage{graphicx} 
%\usepackage[margins=2.5cm,nohead,nofoot]{geometry}
%\usepackage{geometry}
\usepackage{amsfonts}
\usepackage{amstext}
\usepackage{latexsym}
\usepackage{amssymb}
\usepackage{color}


%\include{myPreamble}
\include{qm2pi.local} 

%\ifpdf
%\usepackage[pdftex]{graphicx}
%\else
%\usepackage{graphicx}
%\fi

 % \ifpdf
%  \usepackage{pdfsync}
%  \if


%\title{Brief Article}
%\author{David F. Snyder}
%\author{L.G. Meredith}

%\address{Dept. of Math., Texas State University--San Marcos, San Marcos, TX 78666}
       
\pagestyle{empty}


\begin{document}

\lstset{language=[Objective]Caml,frame=shadowbox}

\input{qm2pi.front}

% section front matter (end)

\input{qm2pi.intro} 
 
% section introduction (end)

% \input{qm2pi.knotations} 

% section notation (end)

\input{qm2pi.process.calculi} 

% section concurrent_process_calculi_and_spatial_logics_ (end)
    
%\input{qm2pi.knots2pi} 

%\input{qm2pi.trefoil} 

%\input{qm2pi.mainthm} 

% subsection basic_interpretation (end)

%\input{qm2pi.rho.presentation} 
\subsection{The syntax and semantics of the notation system}\label{sub:the_syntax_and_semantics_of_the_notation_system} % (fold)

We now summarize a technical presentation of the calculus that
embodies our theory of dynamics. The typical presentation of such a
calculus follows the style of giving generators and relations on
them. The grammar, below, describing term constructors, freely
generates the set of processes, $\Proc$. This set is then quotiented
by a relation known as structural congruence and it is over this set
that the notion of dynamics is expressed. This presentation is
essentially that of \cite{MeredithR05} with the addition of
polyadicity and summation. For readability we have relegated some of
the technical subtleties to an appendix.

\subsubsection{Process grammar}\label{subsub:process_grammar}

\begin{mathpar}
  \inferrule* [lab=synchronization] {} {{M} \bc \pzero \;|\; x?F \;|\; x!C }
  \and
  \inferrule* [lab=abstraction] {} {{F} \bc (x)P}
  \and
  \inferrule* [lab=concretion] {} {{C} \bc \langle Q \rangle}
  \and
  \inferrule* [lab=process] {} {{P,Q} \bc M \;| \;P|Q \;|\; @{x}}
  \and
  \inferrule* [lab=name] {} {{x} \bc \quotep{P}}
\end{mathpar} 

Note that $\vec{x}$ (resp. $\vec{P}$) denotes a vector of names
(resp. processes) of length $|\vec{x}|$ (resp. $|\vec{P}|$). We adopt
the following useful abbreviations.

\begin{mathpar}
   x?(\vec{y}).P := x.(\vec{y})P \and  x\clift{\vec{P}} := x.\clift{\vec{P}}
   \and x!(y) := \lift{x}{\dropn{y}}
   \and \Pi_{i=0}^{n-1}P_i := P_0 | \ldots | P_{n-1}
\end{mathpar}

\subsubsection{Structural congruence}

\paragraph{Free and bound names and alpha-equivalence.} At the
core of structural equivalence is alpha-equivalence which identifies
process that are the same up to a change of variable. Formally, we
recognize the distinction between free and bound names. The free names
of a process, $\freenames{P}$, may be calculated recursively as
follows:

\begin{mathpar}
\freenames{\pzero} := \emptyset
  \and \\
  \freenames{x?(y).P} := \{ x \} \cup (\freenames{P} \setminus \{ y \})
  \and 
  \freenames{x!\langle P \rangle} := \{ x \} \cup \{ P \} 
  \and \\
  \freenames{P|Q} := \freenames{P} \cup \freenames{Q}
  \and \\
  \freenames{@{x}} := \{ x \}
\end{mathpar}

$\pi$
$\quotep{\pi}$

$\freenames{-} : \pi \to \mathcal{P}(\quotep{\pi})$

\begin{eqnarray*}
  \freenames{\pzero} & := & \emptyset \\
  \freenames{x?(y).P} & := & \{ x \} \cup (\freenames{P} \setminus \{ y \}) \\
  \freenames{x!\langle P \rangle} & := & \{ x \} \cup \{ P \} \\
  \freenames{P|Q} & := & \freenames{P} \cup \freenames{Q} \\
  \freenames{\dropn{x}} & := & \{ x \}
\end{eqnarray*}

The bound names of a process, $\boundnames{P}$, are those names occurring in $P$
that are not free. For example, in $x?(y).0$, the name $x$ is free, while $y$ is bound.

\begin{mathpar}
  \inferrule* [lab=monoidal-laws] {} { P|Q \equiv Q|P \and P|0 \equiv P \and P|(Q|R) \equiv (P|Q)|R }
\end{mathpar}

\begin{mathpar}
  \inferrule* [lab=alpha-equivalence] {} { (x)P \equiv (y)P\{y/x\} \and y \not\in \freenames{P} }
\end{mathpar}

\begin{definition}
Then two processes, $P,Q$, are alpha-equivalent if $P = Q\{\vec{y}/\vec{x}\}$ for
some $\vec{x} \in \boundnames{Q},\vec{y} \in \boundnames{P}$, where $Q\{\vec{y}/\vec{x}\}$
denotes the capture-avoiding substitution of $\vec{y}$ for $\vec{x}$ in $Q$.
\end{definition}

\begin{definition}
  The {\em structural congruence} \cite{SangiorgiWalker} , $\equiv$,
  between processes is the least congruence containing
  alpha-equivalence, satisfying the abelian monoid laws
  (associativity, commutativity and $\pzero$ as identity) for parallel
  composition $|$ and for summation $+$.
\end{definition}

\subsection{Name equivalence}

We take name equivalence, written $\nameeq$, to be the smallest
equivalence relation generated by the following rules.

\begin{mathpar}
\inferrule*[lab=Quote-drop]
{ }
{ \quotep{@{x}} \nameeq x }

\inferrule*[lab=Struct-equiv]
{ P \scong Q }
{ \quotep{P} \nameeq \quotep{Q} }
\end{mathpar}

The astute reader will have noticed that the mutual recursion of names
and processes imposes a mutual recursion on alpha-equivalence and
structural equivalence via name-equivalence. Fortunately, all of this
works out pleasantly and we may calculate in the natural way, free of
concern. The reader interested in the details is referred to the
appendix \ref{appendix:rho_details}.

\subsection{Substitution}

We use $\Proc$ for the set of processes, $\QProc$ for the set of
names, and $\id{\{}\vec{y} / \vec{x} \id{\}}$ to denote partial maps,
$s : \QProc \rightarrow \QProc$. A map, $s$ lifts, uniquely, to a map
on process terms, $\widehat{s} : \Proc \rightarrow \Proc$ by the
following equations.

\begin{mathpar}
  (0) \psubstp{Q}{P} := 0 \\
  (R \juxtap S) \psubstp{Q}{P}
  :=    
  (R)\psubstp{Q}{P} \juxtap (S) \psubstp{Q}{P} \\
  (x?(y).R) \psubstp{Q}{P}    
  :=    
  (x)\substp{Q}{P} (z)\concat( (R \psubstn{z}{y}) \psubstp{Q}{P} ) \\
  (\lift{x}{R}) \psubstp{Q}{P}  
  :=
  \lift{(x)\substp{Q}{P}}{ R \psubstp{Q}{P} } \\
%   (\dropn{x})  \psubstp{Q}{P}       
%   := 
%   \left\{ 
%     \begin{array}{ccc} 
%       \dropn{\quotep{Q}} & & x \nameeq \quotep{P} \\
%       \dropn{x} & & otherwise \\
%     \end{array}
%   \right. 
  (\dropn{x})  \psubstp{Q}{P}       
  := 
  \left\{ 
    \begin{array}{ccc} 
      Q & & x \nameeq \quotep{P} \\
      \dropn{x} & & otherwise \\
    \end{array}
  \right.
\end{mathpar}
 

where

\begin{eqnarray}
  (x)\id{\{} \lpquote Q \rpquote / \lpquote P \rpquote \id{\}}            = 
  \left\{ 
    \begin{array}{ccc}
      \lpquote Q \rpquote & & x \nameeq \lpquote P \rpquote \\
      x & & otherwise \\
    \end{array}
  \right. \nonumber
\end{eqnarray}

and $z$ is chosen distinct from $\quotep{P}$, $\quotep{Q}$, the free
names in $Q$, and all the names in $R$. Our $\alpha$-equivalence will
be built in the standard way from this substitution.

\begin{remark}\label{rem:no_self_referential_names}
  One consequence of these definitions is that $\forall P. \quotep{P}
  \not\in \freenames{P}$.
\end{remark}

\subsection{ Dynamic quote: an example }

Anticipating something of what's to come, consider applying the
substitution, $\widehat{\id{\{}u / z \id{\}}}$, to the following pair
of processes, $\lift{w}{y!(z)}$ and $w[ \lpquote y!(z) \rpquote ]$.

\begin{eqnarray}
	\lift{w}{y!(z)}\widehat{\id{\{}u / z \id{\}}}
		& = &
		\lift{w}{y!(u)} \nonumber\\
	w[ \lpquote y!(z) \rpquote ] \widehat{ \id{\{}u / z \id{\}} }
		& = &
		w[ \lpquote y!(z) \rpquote ] \nonumber
\end{eqnarray}

Because the body of the process between quotes is impervious to
substitution, we get radically different answers. In fact, by
examining the first process in an input context,
e.g. $x?(z).\lift{w}{y!(z)}$, we see that the process under the lift
operator may be shaped by prefixed inputs binding a name inside it. In
this sense, the lift operator will be seen as a way to dynamically
construct processes before reifying them as names.

Finally equipped with these standard features we can present the
dynamics of the calculus.

\subsubsection{Operational semantics} 

Finally, we introduce the computational dynamics. What marks these
algebras as distinct from other more traditionally studied algebraic
structures, e.g. vector spaces or polynomial rings, is the manner in
which dynamics is captured. In traditional structures, dynamics is typically
expressed through morphisms between such structures, as in linear maps
between vector spaces or morphisms between rings. In algebras
associated with the semantics of computation, the dynamics is
expressed as part of the algebraic structure itself, through a
reduction reduction relation typically denoted by $\red$. Below, we
give a recursive presentation of this relation for the calculus used
in the encoding.

$\red \subseteq \pi \times \pi$
$\red : \pi \to \mathcal{P}(\pi)$

\begin{mathpar}
  \inferrule* [lab=Comm] { \textsf{match}( x_{src}, x_{trgt} ) } { x_{trgt}?(y)P \; | \; x_{src}!\langle {Q} \rangle \red P\{\quotep{Q}/y}\} }
  \and \\
  \inferrule* [lab=Par] {{P} \red {P}'} {{{P} | {Q}} \red {{P}' | {Q}}}
  \and
  \inferrule* [lab=Equiv]{{{P} \scong {P}'} \andalso {{P}' \red {Q}'} \andalso {{Q}' \scong {Q}}}{{P} \red {Q}}
\end{mathpar}

\begin{eqnarray*}
  match_{\equiv} (\quotep{P},\quotep{Q}) & := & P \equiv Q \\
  match_{\dagger}(\quotep{P},\quotep{Q}) & := & \forall R. P|Q \red^{*} R => R \red^{*} 0 \\
  match_{K}(\quotep{P},\quotep{Q}) & := & K \mbox{ for some context } K
\end{eqnarray*}

$u?(x)P | u!\langle Q \rangle \red P\{\quotep{Q}/x\}$

%We write $\wred$ for $\red^*$, and $P\red$ if $\exists Q $ such that $ P \red Q$.
We write $P\red$ if $\exists Q $ such that $ P \red Q$ and $P\not\red$, otherwise.

\section{Replication}

As mentioned before, it is known that replication (and hence
recursion) can be implemented in a higher-order process algebra
\cite{SangiorgiWalker}. As our first example of calculation with the
machinery thus far presented we give the construction explicitly in
the {\rhoc}.

\begin{eqnarray}
	D_{x} & := & \prefix{x}{y}{(\binpar{\outputp{x}{y}}{@{y}})} \nonumber\\
	\bangp_{x}{P} & := & \binpar{{x}!\langle{\binpar{D_{x}}{P}}\rangle}{D_{x}} \nonumber
\end{eqnarray}

\begin{eqnarray}
	\bangp_{x}{P} & & \nonumber\\
	=
	& {x}!\langle{(\prefix{x}{y}{(\outputp{x}{y} | @{y})) | P}}\rangle 
	      | \prefix{x}{y}{(\outputp{x}{y} | @{y})} & \nonumber\\
	\red
	& (\outputp{x}{y} | @{y})\substn{\quotep{(\prefix{x}{y}{(@{y} | \outputp{x}{y})) | P}}}{y} & \nonumber\\
	=
	& \outputp{x}{\quotep{(\prefix{x}{y}{(\outputp{x}{y} | @{y})) | P}}}
	  | {(\prefix{x}{y}{(\outputp{x}{y} | @{y})) | P}} & \nonumber\\
	\red
	& \ldots & \nonumber\\
	\red^*
	& P | P | \ldots & \nonumber
\end{eqnarray}

Of course, this encoding, as an implementation, runs away, unfolding
$\bangp{P}$ eagerly. A lazier and more implementable replication
operator, restricted to input-guarded processes, may be obtained as follows.

\begin{eqnarray}
\bangp{\prefix{u}{v}{P}} 
	:= 
	\binpar{\lift{x}{\prefix{u}{v}{(\binpar{D(x)}{P})}}}{D(x)} \nonumber
\end{eqnarray}

\begin{remark}
  Note that the lazier definition still does not deal with summation
  or mixed summation (i.e. sums over input and output). The reader is
  invited to construct definitions of replication that deal with these
  features. 

  Further, the definitions are parameterized in a name, $x$. Can you,
  gentle reader, make a definition that eliminates this parameter and
  guarantees no accidental interaction between the replication
  machinery and the process being replicated -- i.e. no accidental
  sharing of names used by the process to get its work done and the
  name(s) used by the replication to effect copying. This latter
  revision of the definition of replication is crucial to obtaining
  the expected identity $!!P \sim !P$.
\end{remark}

\begin{remark}\label{rem:paradoxical_combinator}
  The reader familiar with the lambda calculus will have noticed the
  similarity between $D$ and the paradoxical combinator.

  [Ed. note: the existence of this seems to suggest we have to be more
  restrictive on the set of processes and names we admit if we are to
  support no-cloning.]
\end{remark}

\subsubsection{Bisimulation}

The computational dynamics gives rise to another kind of equivalence,
the equivalence of computational behavior. As previously mentioned
this is typically captured \emph{via} some form of bisimulation.

% The notion we use in this paper is weak barbed bisimulation
% \cite{milner91polyadicpi}.

The notion we use in this paper is derived from weak barbed
bisimulation \cite{milner91polyadicpi}. 

\begin{definition}
An \emph{observation relation}, $\downarrow_{\mathcal N}$, over a set
of names, $\mathcal N$, is the smallest relation satisfying the rules
below.

\infrule[Out-barb]{y \in {\mathcal N}, \; x \nameeq y}
		  {\outputp{x}{v} \downarrow_{\mathcal N} x}
\infrule[Par-barb]{\mbox{$P\downarrow_{\mathcal N} x$ or $Q\downarrow_{\mathcal N} x$}}
		  {\binpar{P}{Q} \downarrow_{\mathcal N} x}

We write $P \Downarrow_{\mathcal N} x$ if there is $Q$ such that 
$P \wred Q$ and $Q \downarrow_{\mathcal N} x$.
\end{definition}

\begin{definition}
%\label{def.bbisim}
An  ${\mathcal N}$-\emph{barbed bisimulation} over a set of names, ${\mathcal N}$, is a symmetric binary relation 
${\mathcal S}_{\mathcal N}$ between agents such that $P\rel{S}_{\mathcal N}Q$ implies:
\begin{enumerate}
\item If $P \red P'$ then $Q \wred Q'$ and $P'\rel{S}_{\mathcal N} Q'$.
\item If $P\downarrow_{\mathcal N} x$, then $Q\Downarrow_{\mathcal N} x$.
\end{enumerate}
$P$ is ${\mathcal N}$-barbed bisimilar to $Q$, written
$P \wbbisim_{\mathcal N} Q$, if $P \rel{S}_{\mathcal N} Q$ for some ${\mathcal N}$-barbed bisimulation ${\mathcal S}_{\mathcal N}$.
\end{definition}

$\mathcal{R} \subseteq \pi \times \pi$

$P \mathcal{R} Q => \forall P'. P \red P' \Rightarrow \exists Q'. Q \red Q', P' \mathcal{R} Q'$

$P \vdash x \Rightarrow Q \vdash x$

\begin{mathpar}
  \inferrule*[lab=Out-barb]{x \nameeq y}{{y}!\langle{Q}\rangle \vdash x}
  \and
  \inferrule*[lab=Par-barb]{\mbox{$P\vdash x$ or $Q\vdash x$}}{\binpar{P}{Q} \vdash x}
\end{mathpar}

\subsubsection{Contexts}

One of the principle advantages of computational calculi like the
$\pi$-calculus is a well-defined notion of context,
contextual-equivalence and a correlation between
contextual-equivalence and notions of bisimulation. The notion of
context allows the decomposition of a process into (sub-)process and
its syntactic environment, its context. Thus, a context may be
thought of as a process with a ``hole'' (written $\Box$) in it. The
application of a context $M$ to a process $P$, written $M[P]$, is
tantamount to filling the hole in $M$ with $P$. In this paper we do
not need the full weight of this theory, but do make use of the notion
of context in the proof the main theorem. 

\begin{mathpar}
  \inferrule* [lab=summation] {} {{M_{M},M_{N}} \bc \Box \;|\; x.M_{A} \;|\; M_{M}+M_{N}}
  \and
  \inferrule* [lab=agent] {} {{M_{A}} \bc (\vec{x})M_{P} \;| \; \clift{P_0,\ldots,M_{P},\ldots,P_N}}
  \and \\
  \inferrule* [lab=process] {} {{M_{P}} \bc M_{N} \;| \;P|M_{P} }
\end{mathpar} 

\begin{mathpar}
  \inferrule* [lab=sychronization] {} {M_{N} \bc \Box \;|\; x?M_{F} \;|\; x!M_{C}}
  \and
  \inferrule* [lab=abstraction] {} {{M_{F}} \bc (x)M_{P} }
  \and
  \inferrule* [lab=concretion] {} {{M_{C}} \bc \langle M_{P} \rangle }
  \and \\
  \inferrule* [lab=process] {} {{M_{P}} \bc M_{N} \;| \;P|M_{P} }
\end{mathpar}

\begin{definition}[contextual application] Given a context $M$, and
  process $P$, we define the \emph{contextual application}, $M[P] :=
  M\{P/\Box\}$. That is, the contextual application of M to P is the
  substitution of $P$ for $\Box$ in $M$.
\end{definition}

$\meaningof{-} : L \to \mathcal{P}(\pi)$

\begin{mathpar}
  \inferrule* [lab=collection] {} {\meaningof{true} = \pi, \and \meaningof{~E} = \pi \setminus \meaningof{E}, \and \meaningof{E_{1} \& E_{2}} = \meaningof{E_{1}} \cap \meaningof{E_{2}}}
\end{mathpar}

\begin{mathpar}
  \inferrule* [lab=structure] {} {\meaningof{0} = \{ P \in \pi | P \equiv 0 \}, \and \\ \meaningof{E_1 | E_2} = \{ P \in \pi | P \equiv P_{1} | P_{2}, P_{1} \in \meaningof{E_{1}}, P_{2} \in \meaningof{E_2}\} }
\end{mathpar}

\begin{mathpar}
 \inferrule* [lab=behavior] {} {\meaningof{\langle a?b \rangle E} = \{ P \in \pi | P \equiv Q | u?(y)P', \\ \and \\\\ \and \\ \;\;\; u \in \meaningof{a}, \forall z.P'\{z/y\} \in \meaningof{E\{z/b\}}\}, \and \\ \meaningof{a!E} = \{ P \in \pi | P \equiv Q | x!\langle P' \rangle, x \in \meaningof{a} P' \in \meaningof{E}\} }
\end{mathpar}

\begin{mathpar}
 \inferrule* [lab=nominal] {} {\meaningof{\quotep{E}} = \{ \quotep{P} \in \quotep{\pi} | P \in \meaningof{E} \}, \and \meaningof{\quotep{P}} = \{ \quotep{Q} \in \quotep{\pi} | P \equiv Q \} \and \\ \meaningof{@\quotep{E}} = \{ P \in \pi | P \equiv @x, x \in \meaningof{E} \}}
\end{mathpar}

\begin{eqnarray*}
  \\
  \meaningof{-} : TS \to ST
\end{eqnarray*}

\begin{eqnarray*}
  \\
  L : TS \to ST
\end{eqnarray*}

\begin{eqnarray*}
  \\
  P \models E \iff P \in \meaningof{E}
\end{eqnarray*}

\begin{eqnarray*}
  P \approx_{L} Q \iff \forall E \in L. P \models E \iff Q \models E
\end{eqnarray*}

\begin{eqnarray*}
  P \approx_{K} Q
\end{eqnarray*}

\begin{eqnarray*}
  P \approx Q
\end{eqnarray*}

$\approx_{K} = \approx = \approx_{L}$

\subsubsection{Contextual duality}

Note that contexts extend the quotation operation to a family of
operations from processes to names. Given a context, $M$, we can
define a \emph{nominal context}, $\quotep{M}$ by $\quotep{M}[P] :=
\quotep{M[P]}$. To foreshadow what is to come we observe that these
operations enjoy a duality with processes very much like the duality
between vectors and maps from vectors to scalars.

Further, because the calculus is essentially higher-order, we have a
correspondence between contexts and processes. More specifically,
given a name $x$ and a context $M$ we can construct $M^{*}_{x}$ such
that 

\begin{mathpar}
  M^{*}_{x} | \lift{x}{P} \red M[P]
\end{mathpar}

namely,

\begin{mathpar}
  M^{*}_{x} := x?(u).M[\dropn{u}]
\end{mathpar}

The dependence of $M^{*}_{x}$ on a name makes it an abstraction, 

\begin{mathpar}
  M^{*} := (x)x?(u).M[\dropn{u}]
\end{mathpar}

\subsection{Additional notation}

It will sometimes be convenient to denote the process a name
quotes. We already have the notation $x = \quotep{P}$, but it will be
convenient to introduce an alternate notation, $\procn{x}$, when we
want to emphasize the connection to the use of the name. Note that, by
virtue of name equivalence, $\quotep{\procn{x}} \nameeq x$; so, the
notation is consistent with previous definitions.

Further, because names have structure it is possible to effect
substitutions on the basis of that structure. This means we need to
upgrade our notation for substitutions, which we accomplish by
adapting comprehension notation. Thus,

\begin{mathpar}
  P\{ y / x : x \in S \}
\end{mathpar}

is interpreted to mean the process derived from P by replacing (in a
capture-avoiding manner) each occurrence of $x$ in $S$ by $y$. For example,

\begin{mathpar}
  P\{ \quotep{\procn{x}|\procn{x}} / x : x \in \freenames{P} \}
\end{mathpar}

will replace each (occurrence) of a free name $x$ in $P$ by
$\quotep{\procn{x}|\procn{x}}$.

Also, we will avail ourselves of the notation $x^{L}$ and $x^{R}$ to
denote injections of a name into disjoint copies of the name
space. There are numerous ways to accomplish this. One example can be
found in \cite{MeredithR05}. This notation overloads to vectors of
names: $\vec{x}^{\pi} := (x_{i}^{\pi} \; : \; 0 \leq i < |\vec{x}| )$ where $\pi \in \{L,R\}$.

We also use $P^{\Box} := P|\Box$.

In \cite{MeredithR05} an interpretation of the new operator is
given. It turns out that there are several possible interpretations
all enjoying the requisite algebraic properties of the operator (see
\cite{milner91polyadicpi}). We will therefore make liberal use of
$(\nu\; \vec{x})P$.

% subsection the_syntax_and_semantics_of_the_notation_system (end)   

\input{qm2pi.qmops} 

\input{qm2pi.sterngerlach} 

\input{qm2pi.metric} 

% section concurrent_process_calculi (end)

%\input{qm2pi.proofsketch}

% section proof sketch (end)

%\input{qm2pi.slviaknots} 

% section spatial logic via knots (end)

\input{qm2pi.conclusion}

% section conclusion (end)

%\input{qm2pi.dtcodes} 

% section wiring algorithm (end)

\input{qm2pi.ack} 

% section acknowledgments (end)

\newpage


\bibliographystyle{plain}   
\bibliography{../../biblios/main.bib}

\input{qm2pi.rhodetails}

\end{document}



% section proof sketch (end)

%\section{Unlikely characters: spatial logic for
  knots}\label{sub:characteristic_formulae} % (fold)

Associated to the mobile process calculi are a family of logics known
as the Hennessy-Milner logics. These logics typically enjoy a
semantics interpreting formulae as sets of processes that when
factored through the encoding outlined above allows an identification
of classes of knots with logical formulae. In the context of this
encoding the sub-family known as the spatial logics \cite{CairesC03}
\cite{CairesC04} \cite{Caires04} are of particular interest providing
several important features for expressing and reasoning about
properties (i.e. classes) of knots. We hint here at how this may be done.

%\begin{description}
%\item [structural connectives] 
\subsubsection{Structural connectives} The spatial logics enjoy
structural connectives corresponding, at the logical level, to the
parallel composition ($P | Q$) and new name ($(\nu \; x)P$)
connectives for processes. As illustrated in the examples below, these
connectives are extremely expressive given the shape of our encoding.
%\item [decideable satisfaction]

\subsubsection{Decideable satisfaction}
In \cite{Caires04} the satisfaction relation is shown to be decideable
for a rich class of processes. It further turns out that the image of
the our encoding is a proper subset of that class. This result
provides the basis for an algorithm by which to search for knots
enjoying a given property.
%\item [characteristic formulae]

\subsubsection{Characteristic formulae}
In the same paper \cite{Caires04} , Caires presents a means of calculating
characteristic formulae, selecting equivalence classes of processes
up to a pre--specified depth limit on the support set of names. Composed with our
encoding, this characteristic formula can be used to select
characteristic formulae for knots.
%\end{description}

\subsubsection{Spatial logic formulae}

The grammar below (segmented for comprehension) summarizes the syntax
of spatial logic formulae. We employ illustrative examples in the
sequel to provide an intuitive understanding of their meaning
referring the reader to \cite{Caires04} for a more detailed explication
of the semantics.

\begin{mathpar}
  \inferrule* [lab=boolean] {} {{A,B} \bc T \;|\; \neg A \;|\; A \wedge B \;|\; \eta = \eta'}
  \and
  \inferrule* [lab=spatial] {} {|\; \pzero \;|\; A | B \;|\; x \text{\textregistered} A \;|\; \forall x . A \;|\;  H x . A}
  \and
  \inferrule* [lab=behavioral] {} {|\; \alpha . A}
  \and 
  \inferrule* [lab=recursion] {} {|\; X(\vec{u}) \;|\; \mu X(\vec{u}) . A}
  \and
  \inferrule* [lab=action] {} {\alpha \bc \langle x?(\vec{y}) \rangle \;|\; \langle x!(\vec{y}) \rangle \;|\; \langle \tau \rangle}
  \and 
  \inferrule* [lab=name] {} {\eta \bc x \;|\; \tau}
\end{mathpar} 

% subsection characteristic_formulae (end)   	 

\subsection{Example formulae}\label{sub:example_formulae_} % (fold)

\subsubsection{Crossing as formula.}
% 
% \begin{align*}
%   \frac{d}{dx} \sin x &= \cos x 
%   & \frac{d}{dx} e^x &= e^x \\
%   \frac{d}{dx} \cos x &= - \sin x 
%   & \frac{d}{dx} \log x &= \frac{1}{x} \\
% \end{align*} 

\begin{align*}
 \mu C(x_{0},x_{1},y_{0},y_{1},u).&(\langle x_{0}?(z) \rangle(\langle u! \rangle\langle y_{1}!z \rangle C(x_{0},x_{1},y_{0},y_{1},u)) & \\
  & \wedge \langle y_{1}?(z) \rangle (\langle u! \rangle \langle x_{0}!z \rangle C(x_{0},x_{1},y_{0},y_{1},u)) & \\
  & \wedge \langle x_{1}?(z) \rangle (\langle u? \rangle \langle y_{0}!z \rangle C(x_{0},x_{1},y_{0},y_{1},u)) & \\
  & \wedge \langle y_{0}?(z) \rangle (\langle u? \rangle \langle x_{1}!z \rangle C(x_{0},x_{1},y_{0},y_{1},u))) &
\end{align*}

The lexicographical similarity between the shape of this formulae and
the shape of definition of the process representing a crossing reveals
the intuitive meaning of this formulae. It describes the capabilities
of a process that has the right to represent a crossing. For example
it picks out processes that may perform an input on the port $x_0$ in
its initial menu of capabilities. What differentiates the formula
from the process, however, is that the crossing process is the
smallest candidate to satisfy the formula. Infinitely many other
processes -- with internal behavior hidden behind this interface, so
to speak -- also satisfy this formula. Even this simple formula,
then, can be seen to open a new view onto knots, providing a
computational interpretation of \emph{virtual} knots.

Note that this formula is derived by hand. A similar formula can be
derived by employing Caires' calculation of characteristic formula
\cite{Caires04} to the process representing a crossing. In light of
this discussion, we let
$\meaningof{C}_{\phi}(x0,x1,y0,y1,u)$ denote a formula specifying the
dynamics we wish to capture of a crossing. To guarantee we preserve
the shape of the interface and minimal semantics we demand that
$\meaningof{C}_{\phi}(x0,x1,y0,y1,u) \Rightarrow
\textbf{C}(x0,x1,y0,y1,u)$ where $\textbf{C}(x0,x1,y0,y1,u)$ denotes
the formula above.
                            
\subsubsection{Crossing number constraints.}
The moral content of the context lemma (Lemma \ref{context}) is that the notion of
``locality'' in the Reidemeister moves is effectively captured by the
parallel composition operator of the process calculus. This intuition
extends through the logic. Given a formula,
$\meaningof{C}_{\phi}(x0,x1,y0,y1,u)$, we can use the structural
connectives to specify constraints on crossing numbers, such as at
least $n$ crossings, or exactly $n$ crossings.
\begin{mathpar}
  \inferrule* [lab=at-least-n] {} { K^{\geq n}_{\phi}(\vec{xs},\vec{ys}) := \Pi_{i=0}^{n-1} Hu . \meaningof{C}_{\phi}(xs_i,ys_i,u) | T }
  \and 
  \inferrule* [lab=exactly-n] {} { K^{= n}_{\phi}(\vec{xs},\vec{ys}) := \Pi_{i=0}^{n-1} Hu . \meaningof{C}_{\phi}(xs_i,ys_i,u) | \neg (\forall x_0,y_0,x_1,y_1,u . \meaningof{C}_{\phi}(x_0,y_0,x_1,y_1,u) | T) }
\end{mathpar}

To round out this section, recall that the encoding of an $n$-crossing
knot decomposes into a parallel composition of $n$ \emph{copies} of a
crossing process together with a wiring harness. To specify different
knot classes with the same crossing number amounts to specifying
logical constraints on the wiring harness. In the interest of space,
we defer examples to a forthcoming paper. Suffice it to say that both
the conditions ``alternating knot'' and ``contains the tangle
corresponding to 5/3'' are expressible. For example, it is possible to
calculate the characteristic formula of a process corresponding to the
tangle 5/3 and conjoin it into the classifying formula via the
composition connective of the logic.

Finally, we wish to observe that it is entirely within reason to
contemplate a more domain-specific version of spatial logic tailored
to the shape of processes in the image of the encoding. Such a
domain-specific logic would have a better claim to the title formal
language of knot properties.

% subsection example_formulae_ (end)

% section knots_as_processes (end) 

% section spatial logic via knots (end)

\section{Conclusions and future work}

\paragraph{Testing physical space}
You, gentle reader, may wonder why of all the theorems to be proved
given this set up we pick the one above. In some sense it's hardly
central to quantum mechanics. We see it as central in the sense that
it firmly establishes a notion of physical space arising from a notion
of the equivalence of behavior. Relating bisimulation to a metric is a
big step forward, but one is faced with interpreting the relationship
of that metric space to something more physical. Quantum mechanical
notions of ``physical'' space are still far from intuitive, but by
relating this idea of distance as testing to calculations that predict
physical circumstances we are making a not insignificant step forward
toward an understanding of the physical space we inhabit as
essentially dynamic.

\paragraph{Effectivity and simulation}
One of the observations we have yet to make is that the entire program
spelled out here is effective. We have built various interpreters for
the reflective calculus at work in this interpretation. In principle,
then, we can simulate quantum mechanics on a computer. The place where
the simulation may lose fidelity is the infinitely branching summation
for the annihilator.

In this connection i also want to point out that the evaluation style
calculation of the inner product puts the non-determinism of the
summation right at the heart of measurement. This suggests that
Milner's original reduction-based formulation of the dynamics of his
calculi in terms of sums was not just notationally suggestive of a
notion of measure-and-continue but captured some significant part of
the physics.

\paragraph{Quantum continuations}
In light of this last observation i want to point out that the
predominant account of quantum mechanics is missing a key aspect of a
truly compositional story of the physical situation. In a real lab,
when a measurement is made the observation can be made to feed into
another device that then makes another measurement conditioned on the
results of the first. This means that after the superposition was
collapsed the entire experimental set up remained in
superposition. While QM offers a means of writing this down it doesn't
quite line up well with the well-trodden formulation of computation
and continuation that we see so succinctly expressed in Milner's
calculi. This suggests that there might be advantages to this account
of dynamics waiting to be explored.

\paragraph{Quantum logic}
In this connection, we also note that by virtue of having the
Hennessy-Milner construction, we can pull the construction through the
interpretation of QM. This gives us a natural candidate for a quantum
logic that enjoys an extremely tight connection with it's domain of
interpretation, making the construction much less ad hoc (rather it is
the image of functor!).

\paragraph{Quantum probabiity}
i have questions about the basis of the interpretation of inner
product as probability amplitude. In particular, using which
axiomatization of probability theory does the notion of probability
amplitude earn the right to be so dubbed? In other words, where is the
proof that the operation for calculating a probability amplitude (and
then squaring) satisfies the axioms of what it means to calculate a
probability? Even if such a proof exists (i have yet to find it in the
literature), i wonder if it might not be possible to turn things on
their heads. Can we view the calculation of the probability amplitude
as an axiomatization of probability? If so, then the definition we
give for calculating probability amplitude may provide the basis for
an \emph{effective} theory of probability.

\paragraph{Quantum vs ``biological'' information}
Finally, i want to conclude with a more philosophical observation. At
a recent workshop in which QM was a predominant topic i noticed
something about quantum information. The speaker was giving a riveting
discussion of axiomatic QM and showing how properties of ``no
cloning'' and ``no deleting'' emerged as consequences of the
axiomatization. Theorems of this form are necessary to give us a sense
of confidence that our axioms characterize the physical theory. What
struck me, though, was that if quantum information is neither erasable
nor replicable it is markedly different from \emph{life}. Two of the
things we know about life is that

\begin{itemize}
  \item it ends;
  \item to gain some measure of persistence, to transcend it's
    finitude it is imminently copyable.
\end{itemize}

Both of these qualities are summarized succinctly in the aphorism: all
flesh is grass. For me these two kinds of ``information'' -- call them
quantum and biological -- are end points on a spectrum of strategies
for persistence. At one end, we have those curious entities that enjoy
uniqueness and permanence; at the other, we have those who in the face
of a certain end and an uncertain present make a go of passing
something on. To me one of the more remarkable aspects of the latter
strategy is that in the presence of noise (and certain features of
copying) we get a kind of dynamism, a chance for improvement against a
given persistent condition.

% subsection other_calculi_other_bisimulations_and_geometry_as_behavior (end)




% section conclusion (end)

%\documentclass[12pt]{llncs}
%\documentclass{jktr}

\usepackage[pdftex]{hyperref}                   
\usepackage {listings}
\usepackage {mathpartir}
\usepackage{bcprules}
%\usepackage{listings}
                       
\usepackage{graphicx} 
%\usepackage[margins=2.5cm,nohead,nofoot]{geometry}
%\usepackage{geometry}
\usepackage{amsfonts}
\usepackage{amstext}
\usepackage{latexsym}
\usepackage{amssymb}
\usepackage{color}


%\include{myPreamble}
\include{qm2pi.local} 

%\ifpdf
%\usepackage[pdftex]{graphicx}
%\else
%\usepackage{graphicx}
%\fi

 % \ifpdf
%  \usepackage{pdfsync}
%  \if


%\title{Brief Article}
%\author{David F. Snyder}
%\author{L.G. Meredith}

%\address{Dept. of Math., Texas State University--San Marcos, San Marcos, TX 78666}
       
\pagestyle{empty}


\begin{document}

\lstset{language=[Objective]Caml,frame=shadowbox}

\input{qm2pi.front}

% section front matter (end)

\input{qm2pi.intro} 
 
% section introduction (end)

% \input{qm2pi.knotations} 

% section notation (end)

\input{qm2pi.process.calculi} 

% section concurrent_process_calculi_and_spatial_logics_ (end)
    
%\input{qm2pi.knots2pi} 

%\input{qm2pi.trefoil} 

%\input{qm2pi.mainthm} 

% subsection basic_interpretation (end)

%\input{qm2pi.rho.presentation} 
\subsection{The syntax and semantics of the notation system}\label{sub:the_syntax_and_semantics_of_the_notation_system} % (fold)

We now summarize a technical presentation of the calculus that
embodies our theory of dynamics. The typical presentation of such a
calculus follows the style of giving generators and relations on
them. The grammar, below, describing term constructors, freely
generates the set of processes, $\Proc$. This set is then quotiented
by a relation known as structural congruence and it is over this set
that the notion of dynamics is expressed. This presentation is
essentially that of \cite{MeredithR05} with the addition of
polyadicity and summation. For readability we have relegated some of
the technical subtleties to an appendix.

\subsubsection{Process grammar}\label{subsub:process_grammar}

\begin{mathpar}
  \inferrule* [lab=synchronization] {} {{M} \bc \pzero \;|\; x?F \;|\; x!C }
  \and
  \inferrule* [lab=abstraction] {} {{F} \bc (x)P}
  \and
  \inferrule* [lab=concretion] {} {{C} \bc \langle Q \rangle}
  \and
  \inferrule* [lab=process] {} {{P,Q} \bc M \;| \;P|Q \;|\; @{x}}
  \and
  \inferrule* [lab=name] {} {{x} \bc \quotep{P}}
\end{mathpar} 

Note that $\vec{x}$ (resp. $\vec{P}$) denotes a vector of names
(resp. processes) of length $|\vec{x}|$ (resp. $|\vec{P}|$). We adopt
the following useful abbreviations.

\begin{mathpar}
   x?(\vec{y}).P := x.(\vec{y})P \and  x\clift{\vec{P}} := x.\clift{\vec{P}}
   \and x!(y) := \lift{x}{\dropn{y}}
   \and \Pi_{i=0}^{n-1}P_i := P_0 | \ldots | P_{n-1}
\end{mathpar}

\subsubsection{Structural congruence}

\paragraph{Free and bound names and alpha-equivalence.} At the
core of structural equivalence is alpha-equivalence which identifies
process that are the same up to a change of variable. Formally, we
recognize the distinction between free and bound names. The free names
of a process, $\freenames{P}$, may be calculated recursively as
follows:

\begin{mathpar}
\freenames{\pzero} := \emptyset
  \and \\
  \freenames{x?(y).P} := \{ x \} \cup (\freenames{P} \setminus \{ y \})
  \and 
  \freenames{x!\langle P \rangle} := \{ x \} \cup \{ P \} 
  \and \\
  \freenames{P|Q} := \freenames{P} \cup \freenames{Q}
  \and \\
  \freenames{@{x}} := \{ x \}
\end{mathpar}

$\pi$
$\quotep{\pi}$

$\freenames{-} : \pi \to \mathcal{P}(\quotep{\pi})$

\begin{eqnarray*}
  \freenames{\pzero} & := & \emptyset \\
  \freenames{x?(y).P} & := & \{ x \} \cup (\freenames{P} \setminus \{ y \}) \\
  \freenames{x!\langle P \rangle} & := & \{ x \} \cup \{ P \} \\
  \freenames{P|Q} & := & \freenames{P} \cup \freenames{Q} \\
  \freenames{\dropn{x}} & := & \{ x \}
\end{eqnarray*}

The bound names of a process, $\boundnames{P}$, are those names occurring in $P$
that are not free. For example, in $x?(y).0$, the name $x$ is free, while $y$ is bound.

\begin{mathpar}
  \inferrule* [lab=monoidal-laws] {} { P|Q \equiv Q|P \and P|0 \equiv P \and P|(Q|R) \equiv (P|Q)|R }
\end{mathpar}

\begin{mathpar}
  \inferrule* [lab=alpha-equivalence] {} { (x)P \equiv (y)P\{y/x\} \and y \not\in \freenames{P} }
\end{mathpar}

\begin{definition}
Then two processes, $P,Q$, are alpha-equivalent if $P = Q\{\vec{y}/\vec{x}\}$ for
some $\vec{x} \in \boundnames{Q},\vec{y} \in \boundnames{P}$, where $Q\{\vec{y}/\vec{x}\}$
denotes the capture-avoiding substitution of $\vec{y}$ for $\vec{x}$ in $Q$.
\end{definition}

\begin{definition}
  The {\em structural congruence} \cite{SangiorgiWalker} , $\equiv$,
  between processes is the least congruence containing
  alpha-equivalence, satisfying the abelian monoid laws
  (associativity, commutativity and $\pzero$ as identity) for parallel
  composition $|$ and for summation $+$.
\end{definition}

\subsection{Name equivalence}

We take name equivalence, written $\nameeq$, to be the smallest
equivalence relation generated by the following rules.

\begin{mathpar}
\inferrule*[lab=Quote-drop]
{ }
{ \quotep{@{x}} \nameeq x }

\inferrule*[lab=Struct-equiv]
{ P \scong Q }
{ \quotep{P} \nameeq \quotep{Q} }
\end{mathpar}

The astute reader will have noticed that the mutual recursion of names
and processes imposes a mutual recursion on alpha-equivalence and
structural equivalence via name-equivalence. Fortunately, all of this
works out pleasantly and we may calculate in the natural way, free of
concern. The reader interested in the details is referred to the
appendix \ref{appendix:rho_details}.

\subsection{Substitution}

We use $\Proc$ for the set of processes, $\QProc$ for the set of
names, and $\id{\{}\vec{y} / \vec{x} \id{\}}$ to denote partial maps,
$s : \QProc \rightarrow \QProc$. A map, $s$ lifts, uniquely, to a map
on process terms, $\widehat{s} : \Proc \rightarrow \Proc$ by the
following equations.

\begin{mathpar}
  (0) \psubstp{Q}{P} := 0 \\
  (R \juxtap S) \psubstp{Q}{P}
  :=    
  (R)\psubstp{Q}{P} \juxtap (S) \psubstp{Q}{P} \\
  (x?(y).R) \psubstp{Q}{P}    
  :=    
  (x)\substp{Q}{P} (z)\concat( (R \psubstn{z}{y}) \psubstp{Q}{P} ) \\
  (\lift{x}{R}) \psubstp{Q}{P}  
  :=
  \lift{(x)\substp{Q}{P}}{ R \psubstp{Q}{P} } \\
%   (\dropn{x})  \psubstp{Q}{P}       
%   := 
%   \left\{ 
%     \begin{array}{ccc} 
%       \dropn{\quotep{Q}} & & x \nameeq \quotep{P} \\
%       \dropn{x} & & otherwise \\
%     \end{array}
%   \right. 
  (\dropn{x})  \psubstp{Q}{P}       
  := 
  \left\{ 
    \begin{array}{ccc} 
      Q & & x \nameeq \quotep{P} \\
      \dropn{x} & & otherwise \\
    \end{array}
  \right.
\end{mathpar}
 

where

\begin{eqnarray}
  (x)\id{\{} \lpquote Q \rpquote / \lpquote P \rpquote \id{\}}            = 
  \left\{ 
    \begin{array}{ccc}
      \lpquote Q \rpquote & & x \nameeq \lpquote P \rpquote \\
      x & & otherwise \\
    \end{array}
  \right. \nonumber
\end{eqnarray}

and $z$ is chosen distinct from $\quotep{P}$, $\quotep{Q}$, the free
names in $Q$, and all the names in $R$. Our $\alpha$-equivalence will
be built in the standard way from this substitution.

\begin{remark}\label{rem:no_self_referential_names}
  One consequence of these definitions is that $\forall P. \quotep{P}
  \not\in \freenames{P}$.
\end{remark}

\subsection{ Dynamic quote: an example }

Anticipating something of what's to come, consider applying the
substitution, $\widehat{\id{\{}u / z \id{\}}}$, to the following pair
of processes, $\lift{w}{y!(z)}$ and $w[ \lpquote y!(z) \rpquote ]$.

\begin{eqnarray}
	\lift{w}{y!(z)}\widehat{\id{\{}u / z \id{\}}}
		& = &
		\lift{w}{y!(u)} \nonumber\\
	w[ \lpquote y!(z) \rpquote ] \widehat{ \id{\{}u / z \id{\}} }
		& = &
		w[ \lpquote y!(z) \rpquote ] \nonumber
\end{eqnarray}

Because the body of the process between quotes is impervious to
substitution, we get radically different answers. In fact, by
examining the first process in an input context,
e.g. $x?(z).\lift{w}{y!(z)}$, we see that the process under the lift
operator may be shaped by prefixed inputs binding a name inside it. In
this sense, the lift operator will be seen as a way to dynamically
construct processes before reifying them as names.

Finally equipped with these standard features we can present the
dynamics of the calculus.

\subsubsection{Operational semantics} 

Finally, we introduce the computational dynamics. What marks these
algebras as distinct from other more traditionally studied algebraic
structures, e.g. vector spaces or polynomial rings, is the manner in
which dynamics is captured. In traditional structures, dynamics is typically
expressed through morphisms between such structures, as in linear maps
between vector spaces or morphisms between rings. In algebras
associated with the semantics of computation, the dynamics is
expressed as part of the algebraic structure itself, through a
reduction reduction relation typically denoted by $\red$. Below, we
give a recursive presentation of this relation for the calculus used
in the encoding.

$\red \subseteq \pi \times \pi$
$\red : \pi \to \mathcal{P}(\pi)$

\begin{mathpar}
  \inferrule* [lab=Comm] { \textsf{match}( x_{src}, x_{trgt} ) } { x_{trgt}?(y)P \; | \; x_{src}!\langle {Q} \rangle \red P\{\quotep{Q}/y}\} }
  \and \\
  \inferrule* [lab=Par] {{P} \red {P}'} {{{P} | {Q}} \red {{P}' | {Q}}}
  \and
  \inferrule* [lab=Equiv]{{{P} \scong {P}'} \andalso {{P}' \red {Q}'} \andalso {{Q}' \scong {Q}}}{{P} \red {Q}}
\end{mathpar}

\begin{eqnarray*}
  match_{\equiv} (\quotep{P},\quotep{Q}) & := & P \equiv Q \\
  match_{\dagger}(\quotep{P},\quotep{Q}) & := & \forall R. P|Q \red^{*} R => R \red^{*} 0 \\
  match_{K}(\quotep{P},\quotep{Q}) & := & K \mbox{ for some context } K
\end{eqnarray*}

$u?(x)P | u!\langle Q \rangle \red P\{\quotep{Q}/x\}$

%We write $\wred$ for $\red^*$, and $P\red$ if $\exists Q $ such that $ P \red Q$.
We write $P\red$ if $\exists Q $ such that $ P \red Q$ and $P\not\red$, otherwise.

\section{Replication}

As mentioned before, it is known that replication (and hence
recursion) can be implemented in a higher-order process algebra
\cite{SangiorgiWalker}. As our first example of calculation with the
machinery thus far presented we give the construction explicitly in
the {\rhoc}.

\begin{eqnarray}
	D_{x} & := & \prefix{x}{y}{(\binpar{\outputp{x}{y}}{@{y}})} \nonumber\\
	\bangp_{x}{P} & := & \binpar{{x}!\langle{\binpar{D_{x}}{P}}\rangle}{D_{x}} \nonumber
\end{eqnarray}

\begin{eqnarray}
	\bangp_{x}{P} & & \nonumber\\
	=
	& {x}!\langle{(\prefix{x}{y}{(\outputp{x}{y} | @{y})) | P}}\rangle 
	      | \prefix{x}{y}{(\outputp{x}{y} | @{y})} & \nonumber\\
	\red
	& (\outputp{x}{y} | @{y})\substn{\quotep{(\prefix{x}{y}{(@{y} | \outputp{x}{y})) | P}}}{y} & \nonumber\\
	=
	& \outputp{x}{\quotep{(\prefix{x}{y}{(\outputp{x}{y} | @{y})) | P}}}
	  | {(\prefix{x}{y}{(\outputp{x}{y} | @{y})) | P}} & \nonumber\\
	\red
	& \ldots & \nonumber\\
	\red^*
	& P | P | \ldots & \nonumber
\end{eqnarray}

Of course, this encoding, as an implementation, runs away, unfolding
$\bangp{P}$ eagerly. A lazier and more implementable replication
operator, restricted to input-guarded processes, may be obtained as follows.

\begin{eqnarray}
\bangp{\prefix{u}{v}{P}} 
	:= 
	\binpar{\lift{x}{\prefix{u}{v}{(\binpar{D(x)}{P})}}}{D(x)} \nonumber
\end{eqnarray}

\begin{remark}
  Note that the lazier definition still does not deal with summation
  or mixed summation (i.e. sums over input and output). The reader is
  invited to construct definitions of replication that deal with these
  features. 

  Further, the definitions are parameterized in a name, $x$. Can you,
  gentle reader, make a definition that eliminates this parameter and
  guarantees no accidental interaction between the replication
  machinery and the process being replicated -- i.e. no accidental
  sharing of names used by the process to get its work done and the
  name(s) used by the replication to effect copying. This latter
  revision of the definition of replication is crucial to obtaining
  the expected identity $!!P \sim !P$.
\end{remark}

\begin{remark}\label{rem:paradoxical_combinator}
  The reader familiar with the lambda calculus will have noticed the
  similarity between $D$ and the paradoxical combinator.

  [Ed. note: the existence of this seems to suggest we have to be more
  restrictive on the set of processes and names we admit if we are to
  support no-cloning.]
\end{remark}

\subsubsection{Bisimulation}

The computational dynamics gives rise to another kind of equivalence,
the equivalence of computational behavior. As previously mentioned
this is typically captured \emph{via} some form of bisimulation.

% The notion we use in this paper is weak barbed bisimulation
% \cite{milner91polyadicpi}.

The notion we use in this paper is derived from weak barbed
bisimulation \cite{milner91polyadicpi}. 

\begin{definition}
An \emph{observation relation}, $\downarrow_{\mathcal N}$, over a set
of names, $\mathcal N$, is the smallest relation satisfying the rules
below.

\infrule[Out-barb]{y \in {\mathcal N}, \; x \nameeq y}
		  {\outputp{x}{v} \downarrow_{\mathcal N} x}
\infrule[Par-barb]{\mbox{$P\downarrow_{\mathcal N} x$ or $Q\downarrow_{\mathcal N} x$}}
		  {\binpar{P}{Q} \downarrow_{\mathcal N} x}

We write $P \Downarrow_{\mathcal N} x$ if there is $Q$ such that 
$P \wred Q$ and $Q \downarrow_{\mathcal N} x$.
\end{definition}

\begin{definition}
%\label{def.bbisim}
An  ${\mathcal N}$-\emph{barbed bisimulation} over a set of names, ${\mathcal N}$, is a symmetric binary relation 
${\mathcal S}_{\mathcal N}$ between agents such that $P\rel{S}_{\mathcal N}Q$ implies:
\begin{enumerate}
\item If $P \red P'$ then $Q \wred Q'$ and $P'\rel{S}_{\mathcal N} Q'$.
\item If $P\downarrow_{\mathcal N} x$, then $Q\Downarrow_{\mathcal N} x$.
\end{enumerate}
$P$ is ${\mathcal N}$-barbed bisimilar to $Q$, written
$P \wbbisim_{\mathcal N} Q$, if $P \rel{S}_{\mathcal N} Q$ for some ${\mathcal N}$-barbed bisimulation ${\mathcal S}_{\mathcal N}$.
\end{definition}

$\mathcal{R} \subseteq \pi \times \pi$

$P \mathcal{R} Q => \forall P'. P \red P' \Rightarrow \exists Q'. Q \red Q', P' \mathcal{R} Q'$

$P \vdash x \Rightarrow Q \vdash x$

\begin{mathpar}
  \inferrule*[lab=Out-barb]{x \nameeq y}{{y}!\langle{Q}\rangle \vdash x}
  \and
  \inferrule*[lab=Par-barb]{\mbox{$P\vdash x$ or $Q\vdash x$}}{\binpar{P}{Q} \vdash x}
\end{mathpar}

\subsubsection{Contexts}

One of the principle advantages of computational calculi like the
$\pi$-calculus is a well-defined notion of context,
contextual-equivalence and a correlation between
contextual-equivalence and notions of bisimulation. The notion of
context allows the decomposition of a process into (sub-)process and
its syntactic environment, its context. Thus, a context may be
thought of as a process with a ``hole'' (written $\Box$) in it. The
application of a context $M$ to a process $P$, written $M[P]$, is
tantamount to filling the hole in $M$ with $P$. In this paper we do
not need the full weight of this theory, but do make use of the notion
of context in the proof the main theorem. 

\begin{mathpar}
  \inferrule* [lab=summation] {} {{M_{M},M_{N}} \bc \Box \;|\; x.M_{A} \;|\; M_{M}+M_{N}}
  \and
  \inferrule* [lab=agent] {} {{M_{A}} \bc (\vec{x})M_{P} \;| \; \clift{P_0,\ldots,M_{P},\ldots,P_N}}
  \and \\
  \inferrule* [lab=process] {} {{M_{P}} \bc M_{N} \;| \;P|M_{P} }
\end{mathpar} 

\begin{mathpar}
  \inferrule* [lab=sychronization] {} {M_{N} \bc \Box \;|\; x?M_{F} \;|\; x!M_{C}}
  \and
  \inferrule* [lab=abstraction] {} {{M_{F}} \bc (x)M_{P} }
  \and
  \inferrule* [lab=concretion] {} {{M_{C}} \bc \langle M_{P} \rangle }
  \and \\
  \inferrule* [lab=process] {} {{M_{P}} \bc M_{N} \;| \;P|M_{P} }
\end{mathpar}

\begin{definition}[contextual application] Given a context $M$, and
  process $P$, we define the \emph{contextual application}, $M[P] :=
  M\{P/\Box\}$. That is, the contextual application of M to P is the
  substitution of $P$ for $\Box$ in $M$.
\end{definition}

$\meaningof{-} : L \to \mathcal{P}(\pi)$

\begin{mathpar}
  \inferrule* [lab=collection] {} {\meaningof{true} = \pi, \and \meaningof{~E} = \pi \setminus \meaningof{E}, \and \meaningof{E_{1} \& E_{2}} = \meaningof{E_{1}} \cap \meaningof{E_{2}}}
\end{mathpar}

\begin{mathpar}
  \inferrule* [lab=structure] {} {\meaningof{0} = \{ P \in \pi | P \equiv 0 \}, \and \\ \meaningof{E_1 | E_2} = \{ P \in \pi | P \equiv P_{1} | P_{2}, P_{1} \in \meaningof{E_{1}}, P_{2} \in \meaningof{E_2}\} }
\end{mathpar}

\begin{mathpar}
 \inferrule* [lab=behavior] {} {\meaningof{\langle a?b \rangle E} = \{ P \in \pi | P \equiv Q | u?(y)P', \\ \and \\\\ \and \\ \;\;\; u \in \meaningof{a}, \forall z.P'\{z/y\} \in \meaningof{E\{z/b\}}\}, \and \\ \meaningof{a!E} = \{ P \in \pi | P \equiv Q | x!\langle P' \rangle, x \in \meaningof{a} P' \in \meaningof{E}\} }
\end{mathpar}

\begin{mathpar}
 \inferrule* [lab=nominal] {} {\meaningof{\quotep{E}} = \{ \quotep{P} \in \quotep{\pi} | P \in \meaningof{E} \}, \and \meaningof{\quotep{P}} = \{ \quotep{Q} \in \quotep{\pi} | P \equiv Q \} \and \\ \meaningof{@\quotep{E}} = \{ P \in \pi | P \equiv @x, x \in \meaningof{E} \}}
\end{mathpar}

\begin{eqnarray*}
  \\
  \meaningof{-} : TS \to ST
\end{eqnarray*}

\begin{eqnarray*}
  \\
  L : TS \to ST
\end{eqnarray*}

\begin{eqnarray*}
  \\
  P \models E \iff P \in \meaningof{E}
\end{eqnarray*}

\begin{eqnarray*}
  P \approx_{L} Q \iff \forall E \in L. P \models E \iff Q \models E
\end{eqnarray*}

\begin{eqnarray*}
  P \approx_{K} Q
\end{eqnarray*}

\begin{eqnarray*}
  P \approx Q
\end{eqnarray*}

$\approx_{K} = \approx = \approx_{L}$

\subsubsection{Contextual duality}

Note that contexts extend the quotation operation to a family of
operations from processes to names. Given a context, $M$, we can
define a \emph{nominal context}, $\quotep{M}$ by $\quotep{M}[P] :=
\quotep{M[P]}$. To foreshadow what is to come we observe that these
operations enjoy a duality with processes very much like the duality
between vectors and maps from vectors to scalars.

Further, because the calculus is essentially higher-order, we have a
correspondence between contexts and processes. More specifically,
given a name $x$ and a context $M$ we can construct $M^{*}_{x}$ such
that 

\begin{mathpar}
  M^{*}_{x} | \lift{x}{P} \red M[P]
\end{mathpar}

namely,

\begin{mathpar}
  M^{*}_{x} := x?(u).M[\dropn{u}]
\end{mathpar}

The dependence of $M^{*}_{x}$ on a name makes it an abstraction, 

\begin{mathpar}
  M^{*} := (x)x?(u).M[\dropn{u}]
\end{mathpar}

\subsection{Additional notation}

It will sometimes be convenient to denote the process a name
quotes. We already have the notation $x = \quotep{P}$, but it will be
convenient to introduce an alternate notation, $\procn{x}$, when we
want to emphasize the connection to the use of the name. Note that, by
virtue of name equivalence, $\quotep{\procn{x}} \nameeq x$; so, the
notation is consistent with previous definitions.

Further, because names have structure it is possible to effect
substitutions on the basis of that structure. This means we need to
upgrade our notation for substitutions, which we accomplish by
adapting comprehension notation. Thus,

\begin{mathpar}
  P\{ y / x : x \in S \}
\end{mathpar}

is interpreted to mean the process derived from P by replacing (in a
capture-avoiding manner) each occurrence of $x$ in $S$ by $y$. For example,

\begin{mathpar}
  P\{ \quotep{\procn{x}|\procn{x}} / x : x \in \freenames{P} \}
\end{mathpar}

will replace each (occurrence) of a free name $x$ in $P$ by
$\quotep{\procn{x}|\procn{x}}$.

Also, we will avail ourselves of the notation $x^{L}$ and $x^{R}$ to
denote injections of a name into disjoint copies of the name
space. There are numerous ways to accomplish this. One example can be
found in \cite{MeredithR05}. This notation overloads to vectors of
names: $\vec{x}^{\pi} := (x_{i}^{\pi} \; : \; 0 \leq i < |\vec{x}| )$ where $\pi \in \{L,R\}$.

We also use $P^{\Box} := P|\Box$.

In \cite{MeredithR05} an interpretation of the new operator is
given. It turns out that there are several possible interpretations
all enjoying the requisite algebraic properties of the operator (see
\cite{milner91polyadicpi}). We will therefore make liberal use of
$(\nu\; \vec{x})P$.

% subsection the_syntax_and_semantics_of_the_notation_system (end)   

\input{qm2pi.qmops} 

\input{qm2pi.sterngerlach} 

\input{qm2pi.metric} 

% section concurrent_process_calculi (end)

%\input{qm2pi.proofsketch}

% section proof sketch (end)

%\input{qm2pi.slviaknots} 

% section spatial logic via knots (end)

\input{qm2pi.conclusion}

% section conclusion (end)

%\input{qm2pi.dtcodes} 

% section wiring algorithm (end)

\input{qm2pi.ack} 

% section acknowledgments (end)

\newpage


\bibliographystyle{plain}   
\bibliography{../../biblios/main.bib}

\input{qm2pi.rhodetails}

\end{document}

 

% section wiring algorithm (end)

\documentclass[12pt]{llncs}
%\documentclass{jktr}

\usepackage[pdftex]{hyperref}                   
\usepackage {listings}
\usepackage {mathpartir}
\usepackage{bcprules}
%\usepackage{listings}
                       
\usepackage{graphicx} 
%\usepackage[margins=2.5cm,nohead,nofoot]{geometry}
%\usepackage{geometry}
\usepackage{amsfonts}
\usepackage{amstext}
\usepackage{latexsym}
\usepackage{amssymb}
\usepackage{color}


%\include{myPreamble}
\include{qm2pi.local} 

%\ifpdf
%\usepackage[pdftex]{graphicx}
%\else
%\usepackage{graphicx}
%\fi

 % \ifpdf
%  \usepackage{pdfsync}
%  \if


%\title{Brief Article}
%\author{David F. Snyder}
%\author{L.G. Meredith}

%\address{Dept. of Math., Texas State University--San Marcos, San Marcos, TX 78666}
       
\pagestyle{empty}


\begin{document}

\lstset{language=[Objective]Caml,frame=shadowbox}

\input{qm2pi.front}

% section front matter (end)

\input{qm2pi.intro} 
 
% section introduction (end)

% \input{qm2pi.knotations} 

% section notation (end)

\input{qm2pi.process.calculi} 

% section concurrent_process_calculi_and_spatial_logics_ (end)
    
%\input{qm2pi.knots2pi} 

%\input{qm2pi.trefoil} 

%\input{qm2pi.mainthm} 

% subsection basic_interpretation (end)

%\input{qm2pi.rho.presentation} 
\subsection{The syntax and semantics of the notation system}\label{sub:the_syntax_and_semantics_of_the_notation_system} % (fold)

We now summarize a technical presentation of the calculus that
embodies our theory of dynamics. The typical presentation of such a
calculus follows the style of giving generators and relations on
them. The grammar, below, describing term constructors, freely
generates the set of processes, $\Proc$. This set is then quotiented
by a relation known as structural congruence and it is over this set
that the notion of dynamics is expressed. This presentation is
essentially that of \cite{MeredithR05} with the addition of
polyadicity and summation. For readability we have relegated some of
the technical subtleties to an appendix.

\subsubsection{Process grammar}\label{subsub:process_grammar}

\begin{mathpar}
  \inferrule* [lab=synchronization] {} {{M} \bc \pzero \;|\; x?F \;|\; x!C }
  \and
  \inferrule* [lab=abstraction] {} {{F} \bc (x)P}
  \and
  \inferrule* [lab=concretion] {} {{C} \bc \langle Q \rangle}
  \and
  \inferrule* [lab=process] {} {{P,Q} \bc M \;| \;P|Q \;|\; @{x}}
  \and
  \inferrule* [lab=name] {} {{x} \bc \quotep{P}}
\end{mathpar} 

Note that $\vec{x}$ (resp. $\vec{P}$) denotes a vector of names
(resp. processes) of length $|\vec{x}|$ (resp. $|\vec{P}|$). We adopt
the following useful abbreviations.

\begin{mathpar}
   x?(\vec{y}).P := x.(\vec{y})P \and  x\clift{\vec{P}} := x.\clift{\vec{P}}
   \and x!(y) := \lift{x}{\dropn{y}}
   \and \Pi_{i=0}^{n-1}P_i := P_0 | \ldots | P_{n-1}
\end{mathpar}

\subsubsection{Structural congruence}

\paragraph{Free and bound names and alpha-equivalence.} At the
core of structural equivalence is alpha-equivalence which identifies
process that are the same up to a change of variable. Formally, we
recognize the distinction between free and bound names. The free names
of a process, $\freenames{P}$, may be calculated recursively as
follows:

\begin{mathpar}
\freenames{\pzero} := \emptyset
  \and \\
  \freenames{x?(y).P} := \{ x \} \cup (\freenames{P} \setminus \{ y \})
  \and 
  \freenames{x!\langle P \rangle} := \{ x \} \cup \{ P \} 
  \and \\
  \freenames{P|Q} := \freenames{P} \cup \freenames{Q}
  \and \\
  \freenames{@{x}} := \{ x \}
\end{mathpar}

$\pi$
$\quotep{\pi}$

$\freenames{-} : \pi \to \mathcal{P}(\quotep{\pi})$

\begin{eqnarray*}
  \freenames{\pzero} & := & \emptyset \\
  \freenames{x?(y).P} & := & \{ x \} \cup (\freenames{P} \setminus \{ y \}) \\
  \freenames{x!\langle P \rangle} & := & \{ x \} \cup \{ P \} \\
  \freenames{P|Q} & := & \freenames{P} \cup \freenames{Q} \\
  \freenames{\dropn{x}} & := & \{ x \}
\end{eqnarray*}

The bound names of a process, $\boundnames{P}$, are those names occurring in $P$
that are not free. For example, in $x?(y).0$, the name $x$ is free, while $y$ is bound.

\begin{mathpar}
  \inferrule* [lab=monoidal-laws] {} { P|Q \equiv Q|P \and P|0 \equiv P \and P|(Q|R) \equiv (P|Q)|R }
\end{mathpar}

\begin{mathpar}
  \inferrule* [lab=alpha-equivalence] {} { (x)P \equiv (y)P\{y/x\} \and y \not\in \freenames{P} }
\end{mathpar}

\begin{definition}
Then two processes, $P,Q$, are alpha-equivalent if $P = Q\{\vec{y}/\vec{x}\}$ for
some $\vec{x} \in \boundnames{Q},\vec{y} \in \boundnames{P}$, where $Q\{\vec{y}/\vec{x}\}$
denotes the capture-avoiding substitution of $\vec{y}$ for $\vec{x}$ in $Q$.
\end{definition}

\begin{definition}
  The {\em structural congruence} \cite{SangiorgiWalker} , $\equiv$,
  between processes is the least congruence containing
  alpha-equivalence, satisfying the abelian monoid laws
  (associativity, commutativity and $\pzero$ as identity) for parallel
  composition $|$ and for summation $+$.
\end{definition}

\subsection{Name equivalence}

We take name equivalence, written $\nameeq$, to be the smallest
equivalence relation generated by the following rules.

\begin{mathpar}
\inferrule*[lab=Quote-drop]
{ }
{ \quotep{@{x}} \nameeq x }

\inferrule*[lab=Struct-equiv]
{ P \scong Q }
{ \quotep{P} \nameeq \quotep{Q} }
\end{mathpar}

The astute reader will have noticed that the mutual recursion of names
and processes imposes a mutual recursion on alpha-equivalence and
structural equivalence via name-equivalence. Fortunately, all of this
works out pleasantly and we may calculate in the natural way, free of
concern. The reader interested in the details is referred to the
appendix \ref{appendix:rho_details}.

\subsection{Substitution}

We use $\Proc$ for the set of processes, $\QProc$ for the set of
names, and $\id{\{}\vec{y} / \vec{x} \id{\}}$ to denote partial maps,
$s : \QProc \rightarrow \QProc$. A map, $s$ lifts, uniquely, to a map
on process terms, $\widehat{s} : \Proc \rightarrow \Proc$ by the
following equations.

\begin{mathpar}
  (0) \psubstp{Q}{P} := 0 \\
  (R \juxtap S) \psubstp{Q}{P}
  :=    
  (R)\psubstp{Q}{P} \juxtap (S) \psubstp{Q}{P} \\
  (x?(y).R) \psubstp{Q}{P}    
  :=    
  (x)\substp{Q}{P} (z)\concat( (R \psubstn{z}{y}) \psubstp{Q}{P} ) \\
  (\lift{x}{R}) \psubstp{Q}{P}  
  :=
  \lift{(x)\substp{Q}{P}}{ R \psubstp{Q}{P} } \\
%   (\dropn{x})  \psubstp{Q}{P}       
%   := 
%   \left\{ 
%     \begin{array}{ccc} 
%       \dropn{\quotep{Q}} & & x \nameeq \quotep{P} \\
%       \dropn{x} & & otherwise \\
%     \end{array}
%   \right. 
  (\dropn{x})  \psubstp{Q}{P}       
  := 
  \left\{ 
    \begin{array}{ccc} 
      Q & & x \nameeq \quotep{P} \\
      \dropn{x} & & otherwise \\
    \end{array}
  \right.
\end{mathpar}
 

where

\begin{eqnarray}
  (x)\id{\{} \lpquote Q \rpquote / \lpquote P \rpquote \id{\}}            = 
  \left\{ 
    \begin{array}{ccc}
      \lpquote Q \rpquote & & x \nameeq \lpquote P \rpquote \\
      x & & otherwise \\
    \end{array}
  \right. \nonumber
\end{eqnarray}

and $z$ is chosen distinct from $\quotep{P}$, $\quotep{Q}$, the free
names in $Q$, and all the names in $R$. Our $\alpha$-equivalence will
be built in the standard way from this substitution.

\begin{remark}\label{rem:no_self_referential_names}
  One consequence of these definitions is that $\forall P. \quotep{P}
  \not\in \freenames{P}$.
\end{remark}

\subsection{ Dynamic quote: an example }

Anticipating something of what's to come, consider applying the
substitution, $\widehat{\id{\{}u / z \id{\}}}$, to the following pair
of processes, $\lift{w}{y!(z)}$ and $w[ \lpquote y!(z) \rpquote ]$.

\begin{eqnarray}
	\lift{w}{y!(z)}\widehat{\id{\{}u / z \id{\}}}
		& = &
		\lift{w}{y!(u)} \nonumber\\
	w[ \lpquote y!(z) \rpquote ] \widehat{ \id{\{}u / z \id{\}} }
		& = &
		w[ \lpquote y!(z) \rpquote ] \nonumber
\end{eqnarray}

Because the body of the process between quotes is impervious to
substitution, we get radically different answers. In fact, by
examining the first process in an input context,
e.g. $x?(z).\lift{w}{y!(z)}$, we see that the process under the lift
operator may be shaped by prefixed inputs binding a name inside it. In
this sense, the lift operator will be seen as a way to dynamically
construct processes before reifying them as names.

Finally equipped with these standard features we can present the
dynamics of the calculus.

\subsubsection{Operational semantics} 

Finally, we introduce the computational dynamics. What marks these
algebras as distinct from other more traditionally studied algebraic
structures, e.g. vector spaces or polynomial rings, is the manner in
which dynamics is captured. In traditional structures, dynamics is typically
expressed through morphisms between such structures, as in linear maps
between vector spaces or morphisms between rings. In algebras
associated with the semantics of computation, the dynamics is
expressed as part of the algebraic structure itself, through a
reduction reduction relation typically denoted by $\red$. Below, we
give a recursive presentation of this relation for the calculus used
in the encoding.

$\red \subseteq \pi \times \pi$
$\red : \pi \to \mathcal{P}(\pi)$

\begin{mathpar}
  \inferrule* [lab=Comm] { \textsf{match}( x_{src}, x_{trgt} ) } { x_{trgt}?(y)P \; | \; x_{src}!\langle {Q} \rangle \red P\{\quotep{Q}/y}\} }
  \and \\
  \inferrule* [lab=Par] {{P} \red {P}'} {{{P} | {Q}} \red {{P}' | {Q}}}
  \and
  \inferrule* [lab=Equiv]{{{P} \scong {P}'} \andalso {{P}' \red {Q}'} \andalso {{Q}' \scong {Q}}}{{P} \red {Q}}
\end{mathpar}

\begin{eqnarray*}
  match_{\equiv} (\quotep{P},\quotep{Q}) & := & P \equiv Q \\
  match_{\dagger}(\quotep{P},\quotep{Q}) & := & \forall R. P|Q \red^{*} R => R \red^{*} 0 \\
  match_{K}(\quotep{P},\quotep{Q}) & := & K \mbox{ for some context } K
\end{eqnarray*}

$u?(x)P | u!\langle Q \rangle \red P\{\quotep{Q}/x\}$

%We write $\wred$ for $\red^*$, and $P\red$ if $\exists Q $ such that $ P \red Q$.
We write $P\red$ if $\exists Q $ such that $ P \red Q$ and $P\not\red$, otherwise.

\section{Replication}

As mentioned before, it is known that replication (and hence
recursion) can be implemented in a higher-order process algebra
\cite{SangiorgiWalker}. As our first example of calculation with the
machinery thus far presented we give the construction explicitly in
the {\rhoc}.

\begin{eqnarray}
	D_{x} & := & \prefix{x}{y}{(\binpar{\outputp{x}{y}}{@{y}})} \nonumber\\
	\bangp_{x}{P} & := & \binpar{{x}!\langle{\binpar{D_{x}}{P}}\rangle}{D_{x}} \nonumber
\end{eqnarray}

\begin{eqnarray}
	\bangp_{x}{P} & & \nonumber\\
	=
	& {x}!\langle{(\prefix{x}{y}{(\outputp{x}{y} | @{y})) | P}}\rangle 
	      | \prefix{x}{y}{(\outputp{x}{y} | @{y})} & \nonumber\\
	\red
	& (\outputp{x}{y} | @{y})\substn{\quotep{(\prefix{x}{y}{(@{y} | \outputp{x}{y})) | P}}}{y} & \nonumber\\
	=
	& \outputp{x}{\quotep{(\prefix{x}{y}{(\outputp{x}{y} | @{y})) | P}}}
	  | {(\prefix{x}{y}{(\outputp{x}{y} | @{y})) | P}} & \nonumber\\
	\red
	& \ldots & \nonumber\\
	\red^*
	& P | P | \ldots & \nonumber
\end{eqnarray}

Of course, this encoding, as an implementation, runs away, unfolding
$\bangp{P}$ eagerly. A lazier and more implementable replication
operator, restricted to input-guarded processes, may be obtained as follows.

\begin{eqnarray}
\bangp{\prefix{u}{v}{P}} 
	:= 
	\binpar{\lift{x}{\prefix{u}{v}{(\binpar{D(x)}{P})}}}{D(x)} \nonumber
\end{eqnarray}

\begin{remark}
  Note that the lazier definition still does not deal with summation
  or mixed summation (i.e. sums over input and output). The reader is
  invited to construct definitions of replication that deal with these
  features. 

  Further, the definitions are parameterized in a name, $x$. Can you,
  gentle reader, make a definition that eliminates this parameter and
  guarantees no accidental interaction between the replication
  machinery and the process being replicated -- i.e. no accidental
  sharing of names used by the process to get its work done and the
  name(s) used by the replication to effect copying. This latter
  revision of the definition of replication is crucial to obtaining
  the expected identity $!!P \sim !P$.
\end{remark}

\begin{remark}\label{rem:paradoxical_combinator}
  The reader familiar with the lambda calculus will have noticed the
  similarity between $D$ and the paradoxical combinator.

  [Ed. note: the existence of this seems to suggest we have to be more
  restrictive on the set of processes and names we admit if we are to
  support no-cloning.]
\end{remark}

\subsubsection{Bisimulation}

The computational dynamics gives rise to another kind of equivalence,
the equivalence of computational behavior. As previously mentioned
this is typically captured \emph{via} some form of bisimulation.

% The notion we use in this paper is weak barbed bisimulation
% \cite{milner91polyadicpi}.

The notion we use in this paper is derived from weak barbed
bisimulation \cite{milner91polyadicpi}. 

\begin{definition}
An \emph{observation relation}, $\downarrow_{\mathcal N}$, over a set
of names, $\mathcal N$, is the smallest relation satisfying the rules
below.

\infrule[Out-barb]{y \in {\mathcal N}, \; x \nameeq y}
		  {\outputp{x}{v} \downarrow_{\mathcal N} x}
\infrule[Par-barb]{\mbox{$P\downarrow_{\mathcal N} x$ or $Q\downarrow_{\mathcal N} x$}}
		  {\binpar{P}{Q} \downarrow_{\mathcal N} x}

We write $P \Downarrow_{\mathcal N} x$ if there is $Q$ such that 
$P \wred Q$ and $Q \downarrow_{\mathcal N} x$.
\end{definition}

\begin{definition}
%\label{def.bbisim}
An  ${\mathcal N}$-\emph{barbed bisimulation} over a set of names, ${\mathcal N}$, is a symmetric binary relation 
${\mathcal S}_{\mathcal N}$ between agents such that $P\rel{S}_{\mathcal N}Q$ implies:
\begin{enumerate}
\item If $P \red P'$ then $Q \wred Q'$ and $P'\rel{S}_{\mathcal N} Q'$.
\item If $P\downarrow_{\mathcal N} x$, then $Q\Downarrow_{\mathcal N} x$.
\end{enumerate}
$P$ is ${\mathcal N}$-barbed bisimilar to $Q$, written
$P \wbbisim_{\mathcal N} Q$, if $P \rel{S}_{\mathcal N} Q$ for some ${\mathcal N}$-barbed bisimulation ${\mathcal S}_{\mathcal N}$.
\end{definition}

$\mathcal{R} \subseteq \pi \times \pi$

$P \mathcal{R} Q => \forall P'. P \red P' \Rightarrow \exists Q'. Q \red Q', P' \mathcal{R} Q'$

$P \vdash x \Rightarrow Q \vdash x$

\begin{mathpar}
  \inferrule*[lab=Out-barb]{x \nameeq y}{{y}!\langle{Q}\rangle \vdash x}
  \and
  \inferrule*[lab=Par-barb]{\mbox{$P\vdash x$ or $Q\vdash x$}}{\binpar{P}{Q} \vdash x}
\end{mathpar}

\subsubsection{Contexts}

One of the principle advantages of computational calculi like the
$\pi$-calculus is a well-defined notion of context,
contextual-equivalence and a correlation between
contextual-equivalence and notions of bisimulation. The notion of
context allows the decomposition of a process into (sub-)process and
its syntactic environment, its context. Thus, a context may be
thought of as a process with a ``hole'' (written $\Box$) in it. The
application of a context $M$ to a process $P$, written $M[P]$, is
tantamount to filling the hole in $M$ with $P$. In this paper we do
not need the full weight of this theory, but do make use of the notion
of context in the proof the main theorem. 

\begin{mathpar}
  \inferrule* [lab=summation] {} {{M_{M},M_{N}} \bc \Box \;|\; x.M_{A} \;|\; M_{M}+M_{N}}
  \and
  \inferrule* [lab=agent] {} {{M_{A}} \bc (\vec{x})M_{P} \;| \; \clift{P_0,\ldots,M_{P},\ldots,P_N}}
  \and \\
  \inferrule* [lab=process] {} {{M_{P}} \bc M_{N} \;| \;P|M_{P} }
\end{mathpar} 

\begin{mathpar}
  \inferrule* [lab=sychronization] {} {M_{N} \bc \Box \;|\; x?M_{F} \;|\; x!M_{C}}
  \and
  \inferrule* [lab=abstraction] {} {{M_{F}} \bc (x)M_{P} }
  \and
  \inferrule* [lab=concretion] {} {{M_{C}} \bc \langle M_{P} \rangle }
  \and \\
  \inferrule* [lab=process] {} {{M_{P}} \bc M_{N} \;| \;P|M_{P} }
\end{mathpar}

\begin{definition}[contextual application] Given a context $M$, and
  process $P$, we define the \emph{contextual application}, $M[P] :=
  M\{P/\Box\}$. That is, the contextual application of M to P is the
  substitution of $P$ for $\Box$ in $M$.
\end{definition}

$\meaningof{-} : L \to \mathcal{P}(\pi)$

\begin{mathpar}
  \inferrule* [lab=collection] {} {\meaningof{true} = \pi, \and \meaningof{~E} = \pi \setminus \meaningof{E}, \and \meaningof{E_{1} \& E_{2}} = \meaningof{E_{1}} \cap \meaningof{E_{2}}}
\end{mathpar}

\begin{mathpar}
  \inferrule* [lab=structure] {} {\meaningof{0} = \{ P \in \pi | P \equiv 0 \}, \and \\ \meaningof{E_1 | E_2} = \{ P \in \pi | P \equiv P_{1} | P_{2}, P_{1} \in \meaningof{E_{1}}, P_{2} \in \meaningof{E_2}\} }
\end{mathpar}

\begin{mathpar}
 \inferrule* [lab=behavior] {} {\meaningof{\langle a?b \rangle E} = \{ P \in \pi | P \equiv Q | u?(y)P', \\ \and \\\\ \and \\ \;\;\; u \in \meaningof{a}, \forall z.P'\{z/y\} \in \meaningof{E\{z/b\}}\}, \and \\ \meaningof{a!E} = \{ P \in \pi | P \equiv Q | x!\langle P' \rangle, x \in \meaningof{a} P' \in \meaningof{E}\} }
\end{mathpar}

\begin{mathpar}
 \inferrule* [lab=nominal] {} {\meaningof{\quotep{E}} = \{ \quotep{P} \in \quotep{\pi} | P \in \meaningof{E} \}, \and \meaningof{\quotep{P}} = \{ \quotep{Q} \in \quotep{\pi} | P \equiv Q \} \and \\ \meaningof{@\quotep{E}} = \{ P \in \pi | P \equiv @x, x \in \meaningof{E} \}}
\end{mathpar}

\begin{eqnarray*}
  \\
  \meaningof{-} : TS \to ST
\end{eqnarray*}

\begin{eqnarray*}
  \\
  L : TS \to ST
\end{eqnarray*}

\begin{eqnarray*}
  \\
  P \models E \iff P \in \meaningof{E}
\end{eqnarray*}

\begin{eqnarray*}
  P \approx_{L} Q \iff \forall E \in L. P \models E \iff Q \models E
\end{eqnarray*}

\begin{eqnarray*}
  P \approx_{K} Q
\end{eqnarray*}

\begin{eqnarray*}
  P \approx Q
\end{eqnarray*}

$\approx_{K} = \approx = \approx_{L}$

\subsubsection{Contextual duality}

Note that contexts extend the quotation operation to a family of
operations from processes to names. Given a context, $M$, we can
define a \emph{nominal context}, $\quotep{M}$ by $\quotep{M}[P] :=
\quotep{M[P]}$. To foreshadow what is to come we observe that these
operations enjoy a duality with processes very much like the duality
between vectors and maps from vectors to scalars.

Further, because the calculus is essentially higher-order, we have a
correspondence between contexts and processes. More specifically,
given a name $x$ and a context $M$ we can construct $M^{*}_{x}$ such
that 

\begin{mathpar}
  M^{*}_{x} | \lift{x}{P} \red M[P]
\end{mathpar}

namely,

\begin{mathpar}
  M^{*}_{x} := x?(u).M[\dropn{u}]
\end{mathpar}

The dependence of $M^{*}_{x}$ on a name makes it an abstraction, 

\begin{mathpar}
  M^{*} := (x)x?(u).M[\dropn{u}]
\end{mathpar}

\subsection{Additional notation}

It will sometimes be convenient to denote the process a name
quotes. We already have the notation $x = \quotep{P}$, but it will be
convenient to introduce an alternate notation, $\procn{x}$, when we
want to emphasize the connection to the use of the name. Note that, by
virtue of name equivalence, $\quotep{\procn{x}} \nameeq x$; so, the
notation is consistent with previous definitions.

Further, because names have structure it is possible to effect
substitutions on the basis of that structure. This means we need to
upgrade our notation for substitutions, which we accomplish by
adapting comprehension notation. Thus,

\begin{mathpar}
  P\{ y / x : x \in S \}
\end{mathpar}

is interpreted to mean the process derived from P by replacing (in a
capture-avoiding manner) each occurrence of $x$ in $S$ by $y$. For example,

\begin{mathpar}
  P\{ \quotep{\procn{x}|\procn{x}} / x : x \in \freenames{P} \}
\end{mathpar}

will replace each (occurrence) of a free name $x$ in $P$ by
$\quotep{\procn{x}|\procn{x}}$.

Also, we will avail ourselves of the notation $x^{L}$ and $x^{R}$ to
denote injections of a name into disjoint copies of the name
space. There are numerous ways to accomplish this. One example can be
found in \cite{MeredithR05}. This notation overloads to vectors of
names: $\vec{x}^{\pi} := (x_{i}^{\pi} \; : \; 0 \leq i < |\vec{x}| )$ where $\pi \in \{L,R\}$.

We also use $P^{\Box} := P|\Box$.

In \cite{MeredithR05} an interpretation of the new operator is
given. It turns out that there are several possible interpretations
all enjoying the requisite algebraic properties of the operator (see
\cite{milner91polyadicpi}). We will therefore make liberal use of
$(\nu\; \vec{x})P$.

% subsection the_syntax_and_semantics_of_the_notation_system (end)   

\input{qm2pi.qmops} 

\input{qm2pi.sterngerlach} 

\input{qm2pi.metric} 

% section concurrent_process_calculi (end)

%\input{qm2pi.proofsketch}

% section proof sketch (end)

%\input{qm2pi.slviaknots} 

% section spatial logic via knots (end)

\input{qm2pi.conclusion}

% section conclusion (end)

%\input{qm2pi.dtcodes} 

% section wiring algorithm (end)

\input{qm2pi.ack} 

% section acknowledgments (end)

\newpage


\bibliographystyle{plain}   
\bibliography{../../biblios/main.bib}

\input{qm2pi.rhodetails}

\end{document}

 

% section acknowledgments (end)

\newpage


\bibliographystyle{plain}   
\bibliography{../../biblios/main.bib}

\documentclass[12pt]{llncs}
%\documentclass{jktr}

\usepackage[pdftex]{hyperref}                   
\usepackage {listings}
\usepackage {mathpartir}
\usepackage{bcprules}
%\usepackage{listings}
                       
\usepackage{graphicx} 
%\usepackage[margins=2.5cm,nohead,nofoot]{geometry}
%\usepackage{geometry}
\usepackage{amsfonts}
\usepackage{amstext}
\usepackage{latexsym}
\usepackage{amssymb}
\usepackage{color}


%\include{myPreamble}
\include{qm2pi.local} 

%\ifpdf
%\usepackage[pdftex]{graphicx}
%\else
%\usepackage{graphicx}
%\fi

 % \ifpdf
%  \usepackage{pdfsync}
%  \if


%\title{Brief Article}
%\author{David F. Snyder}
%\author{L.G. Meredith}

%\address{Dept. of Math., Texas State University--San Marcos, San Marcos, TX 78666}
       
\pagestyle{empty}


\begin{document}

\lstset{language=[Objective]Caml,frame=shadowbox}

\input{qm2pi.front}

% section front matter (end)

\input{qm2pi.intro} 
 
% section introduction (end)

% \input{qm2pi.knotations} 

% section notation (end)

\input{qm2pi.process.calculi} 

% section concurrent_process_calculi_and_spatial_logics_ (end)
    
%\input{qm2pi.knots2pi} 

%\input{qm2pi.trefoil} 

%\input{qm2pi.mainthm} 

% subsection basic_interpretation (end)

%\input{qm2pi.rho.presentation} 
\subsection{The syntax and semantics of the notation system}\label{sub:the_syntax_and_semantics_of_the_notation_system} % (fold)

We now summarize a technical presentation of the calculus that
embodies our theory of dynamics. The typical presentation of such a
calculus follows the style of giving generators and relations on
them. The grammar, below, describing term constructors, freely
generates the set of processes, $\Proc$. This set is then quotiented
by a relation known as structural congruence and it is over this set
that the notion of dynamics is expressed. This presentation is
essentially that of \cite{MeredithR05} with the addition of
polyadicity and summation. For readability we have relegated some of
the technical subtleties to an appendix.

\subsubsection{Process grammar}\label{subsub:process_grammar}

\begin{mathpar}
  \inferrule* [lab=synchronization] {} {{M} \bc \pzero \;|\; x?F \;|\; x!C }
  \and
  \inferrule* [lab=abstraction] {} {{F} \bc (x)P}
  \and
  \inferrule* [lab=concretion] {} {{C} \bc \langle Q \rangle}
  \and
  \inferrule* [lab=process] {} {{P,Q} \bc M \;| \;P|Q \;|\; @{x}}
  \and
  \inferrule* [lab=name] {} {{x} \bc \quotep{P}}
\end{mathpar} 

Note that $\vec{x}$ (resp. $\vec{P}$) denotes a vector of names
(resp. processes) of length $|\vec{x}|$ (resp. $|\vec{P}|$). We adopt
the following useful abbreviations.

\begin{mathpar}
   x?(\vec{y}).P := x.(\vec{y})P \and  x\clift{\vec{P}} := x.\clift{\vec{P}}
   \and x!(y) := \lift{x}{\dropn{y}}
   \and \Pi_{i=0}^{n-1}P_i := P_0 | \ldots | P_{n-1}
\end{mathpar}

\subsubsection{Structural congruence}

\paragraph{Free and bound names and alpha-equivalence.} At the
core of structural equivalence is alpha-equivalence which identifies
process that are the same up to a change of variable. Formally, we
recognize the distinction between free and bound names. The free names
of a process, $\freenames{P}$, may be calculated recursively as
follows:

\begin{mathpar}
\freenames{\pzero} := \emptyset
  \and \\
  \freenames{x?(y).P} := \{ x \} \cup (\freenames{P} \setminus \{ y \})
  \and 
  \freenames{x!\langle P \rangle} := \{ x \} \cup \{ P \} 
  \and \\
  \freenames{P|Q} := \freenames{P} \cup \freenames{Q}
  \and \\
  \freenames{@{x}} := \{ x \}
\end{mathpar}

$\pi$
$\quotep{\pi}$

$\freenames{-} : \pi \to \mathcal{P}(\quotep{\pi})$

\begin{eqnarray*}
  \freenames{\pzero} & := & \emptyset \\
  \freenames{x?(y).P} & := & \{ x \} \cup (\freenames{P} \setminus \{ y \}) \\
  \freenames{x!\langle P \rangle} & := & \{ x \} \cup \{ P \} \\
  \freenames{P|Q} & := & \freenames{P} \cup \freenames{Q} \\
  \freenames{\dropn{x}} & := & \{ x \}
\end{eqnarray*}

The bound names of a process, $\boundnames{P}$, are those names occurring in $P$
that are not free. For example, in $x?(y).0$, the name $x$ is free, while $y$ is bound.

\begin{mathpar}
  \inferrule* [lab=monoidal-laws] {} { P|Q \equiv Q|P \and P|0 \equiv P \and P|(Q|R) \equiv (P|Q)|R }
\end{mathpar}

\begin{mathpar}
  \inferrule* [lab=alpha-equivalence] {} { (x)P \equiv (y)P\{y/x\} \and y \not\in \freenames{P} }
\end{mathpar}

\begin{definition}
Then two processes, $P,Q$, are alpha-equivalent if $P = Q\{\vec{y}/\vec{x}\}$ for
some $\vec{x} \in \boundnames{Q},\vec{y} \in \boundnames{P}$, where $Q\{\vec{y}/\vec{x}\}$
denotes the capture-avoiding substitution of $\vec{y}$ for $\vec{x}$ in $Q$.
\end{definition}

\begin{definition}
  The {\em structural congruence} \cite{SangiorgiWalker} , $\equiv$,
  between processes is the least congruence containing
  alpha-equivalence, satisfying the abelian monoid laws
  (associativity, commutativity and $\pzero$ as identity) for parallel
  composition $|$ and for summation $+$.
\end{definition}

\subsection{Name equivalence}

We take name equivalence, written $\nameeq$, to be the smallest
equivalence relation generated by the following rules.

\begin{mathpar}
\inferrule*[lab=Quote-drop]
{ }
{ \quotep{@{x}} \nameeq x }

\inferrule*[lab=Struct-equiv]
{ P \scong Q }
{ \quotep{P} \nameeq \quotep{Q} }
\end{mathpar}

The astute reader will have noticed that the mutual recursion of names
and processes imposes a mutual recursion on alpha-equivalence and
structural equivalence via name-equivalence. Fortunately, all of this
works out pleasantly and we may calculate in the natural way, free of
concern. The reader interested in the details is referred to the
appendix \ref{appendix:rho_details}.

\subsection{Substitution}

We use $\Proc$ for the set of processes, $\QProc$ for the set of
names, and $\id{\{}\vec{y} / \vec{x} \id{\}}$ to denote partial maps,
$s : \QProc \rightarrow \QProc$. A map, $s$ lifts, uniquely, to a map
on process terms, $\widehat{s} : \Proc \rightarrow \Proc$ by the
following equations.

\begin{mathpar}
  (0) \psubstp{Q}{P} := 0 \\
  (R \juxtap S) \psubstp{Q}{P}
  :=    
  (R)\psubstp{Q}{P} \juxtap (S) \psubstp{Q}{P} \\
  (x?(y).R) \psubstp{Q}{P}    
  :=    
  (x)\substp{Q}{P} (z)\concat( (R \psubstn{z}{y}) \psubstp{Q}{P} ) \\
  (\lift{x}{R}) \psubstp{Q}{P}  
  :=
  \lift{(x)\substp{Q}{P}}{ R \psubstp{Q}{P} } \\
%   (\dropn{x})  \psubstp{Q}{P}       
%   := 
%   \left\{ 
%     \begin{array}{ccc} 
%       \dropn{\quotep{Q}} & & x \nameeq \quotep{P} \\
%       \dropn{x} & & otherwise \\
%     \end{array}
%   \right. 
  (\dropn{x})  \psubstp{Q}{P}       
  := 
  \left\{ 
    \begin{array}{ccc} 
      Q & & x \nameeq \quotep{P} \\
      \dropn{x} & & otherwise \\
    \end{array}
  \right.
\end{mathpar}
 

where

\begin{eqnarray}
  (x)\id{\{} \lpquote Q \rpquote / \lpquote P \rpquote \id{\}}            = 
  \left\{ 
    \begin{array}{ccc}
      \lpquote Q \rpquote & & x \nameeq \lpquote P \rpquote \\
      x & & otherwise \\
    \end{array}
  \right. \nonumber
\end{eqnarray}

and $z$ is chosen distinct from $\quotep{P}$, $\quotep{Q}$, the free
names in $Q$, and all the names in $R$. Our $\alpha$-equivalence will
be built in the standard way from this substitution.

\begin{remark}\label{rem:no_self_referential_names}
  One consequence of these definitions is that $\forall P. \quotep{P}
  \not\in \freenames{P}$.
\end{remark}

\subsection{ Dynamic quote: an example }

Anticipating something of what's to come, consider applying the
substitution, $\widehat{\id{\{}u / z \id{\}}}$, to the following pair
of processes, $\lift{w}{y!(z)}$ and $w[ \lpquote y!(z) \rpquote ]$.

\begin{eqnarray}
	\lift{w}{y!(z)}\widehat{\id{\{}u / z \id{\}}}
		& = &
		\lift{w}{y!(u)} \nonumber\\
	w[ \lpquote y!(z) \rpquote ] \widehat{ \id{\{}u / z \id{\}} }
		& = &
		w[ \lpquote y!(z) \rpquote ] \nonumber
\end{eqnarray}

Because the body of the process between quotes is impervious to
substitution, we get radically different answers. In fact, by
examining the first process in an input context,
e.g. $x?(z).\lift{w}{y!(z)}$, we see that the process under the lift
operator may be shaped by prefixed inputs binding a name inside it. In
this sense, the lift operator will be seen as a way to dynamically
construct processes before reifying them as names.

Finally equipped with these standard features we can present the
dynamics of the calculus.

\subsubsection{Operational semantics} 

Finally, we introduce the computational dynamics. What marks these
algebras as distinct from other more traditionally studied algebraic
structures, e.g. vector spaces or polynomial rings, is the manner in
which dynamics is captured. In traditional structures, dynamics is typically
expressed through morphisms between such structures, as in linear maps
between vector spaces or morphisms between rings. In algebras
associated with the semantics of computation, the dynamics is
expressed as part of the algebraic structure itself, through a
reduction reduction relation typically denoted by $\red$. Below, we
give a recursive presentation of this relation for the calculus used
in the encoding.

$\red \subseteq \pi \times \pi$
$\red : \pi \to \mathcal{P}(\pi)$

\begin{mathpar}
  \inferrule* [lab=Comm] { \textsf{match}( x_{src}, x_{trgt} ) } { x_{trgt}?(y)P \; | \; x_{src}!\langle {Q} \rangle \red P\{\quotep{Q}/y}\} }
  \and \\
  \inferrule* [lab=Par] {{P} \red {P}'} {{{P} | {Q}} \red {{P}' | {Q}}}
  \and
  \inferrule* [lab=Equiv]{{{P} \scong {P}'} \andalso {{P}' \red {Q}'} \andalso {{Q}' \scong {Q}}}{{P} \red {Q}}
\end{mathpar}

\begin{eqnarray*}
  match_{\equiv} (\quotep{P},\quotep{Q}) & := & P \equiv Q \\
  match_{\dagger}(\quotep{P},\quotep{Q}) & := & \forall R. P|Q \red^{*} R => R \red^{*} 0 \\
  match_{K}(\quotep{P},\quotep{Q}) & := & K \mbox{ for some context } K
\end{eqnarray*}

$u?(x)P | u!\langle Q \rangle \red P\{\quotep{Q}/x\}$

%We write $\wred$ for $\red^*$, and $P\red$ if $\exists Q $ such that $ P \red Q$.
We write $P\red$ if $\exists Q $ such that $ P \red Q$ and $P\not\red$, otherwise.

\section{Replication}

As mentioned before, it is known that replication (and hence
recursion) can be implemented in a higher-order process algebra
\cite{SangiorgiWalker}. As our first example of calculation with the
machinery thus far presented we give the construction explicitly in
the {\rhoc}.

\begin{eqnarray}
	D_{x} & := & \prefix{x}{y}{(\binpar{\outputp{x}{y}}{@{y}})} \nonumber\\
	\bangp_{x}{P} & := & \binpar{{x}!\langle{\binpar{D_{x}}{P}}\rangle}{D_{x}} \nonumber
\end{eqnarray}

\begin{eqnarray}
	\bangp_{x}{P} & & \nonumber\\
	=
	& {x}!\langle{(\prefix{x}{y}{(\outputp{x}{y} | @{y})) | P}}\rangle 
	      | \prefix{x}{y}{(\outputp{x}{y} | @{y})} & \nonumber\\
	\red
	& (\outputp{x}{y} | @{y})\substn{\quotep{(\prefix{x}{y}{(@{y} | \outputp{x}{y})) | P}}}{y} & \nonumber\\
	=
	& \outputp{x}{\quotep{(\prefix{x}{y}{(\outputp{x}{y} | @{y})) | P}}}
	  | {(\prefix{x}{y}{(\outputp{x}{y} | @{y})) | P}} & \nonumber\\
	\red
	& \ldots & \nonumber\\
	\red^*
	& P | P | \ldots & \nonumber
\end{eqnarray}

Of course, this encoding, as an implementation, runs away, unfolding
$\bangp{P}$ eagerly. A lazier and more implementable replication
operator, restricted to input-guarded processes, may be obtained as follows.

\begin{eqnarray}
\bangp{\prefix{u}{v}{P}} 
	:= 
	\binpar{\lift{x}{\prefix{u}{v}{(\binpar{D(x)}{P})}}}{D(x)} \nonumber
\end{eqnarray}

\begin{remark}
  Note that the lazier definition still does not deal with summation
  or mixed summation (i.e. sums over input and output). The reader is
  invited to construct definitions of replication that deal with these
  features. 

  Further, the definitions are parameterized in a name, $x$. Can you,
  gentle reader, make a definition that eliminates this parameter and
  guarantees no accidental interaction between the replication
  machinery and the process being replicated -- i.e. no accidental
  sharing of names used by the process to get its work done and the
  name(s) used by the replication to effect copying. This latter
  revision of the definition of replication is crucial to obtaining
  the expected identity $!!P \sim !P$.
\end{remark}

\begin{remark}\label{rem:paradoxical_combinator}
  The reader familiar with the lambda calculus will have noticed the
  similarity between $D$ and the paradoxical combinator.

  [Ed. note: the existence of this seems to suggest we have to be more
  restrictive on the set of processes and names we admit if we are to
  support no-cloning.]
\end{remark}

\subsubsection{Bisimulation}

The computational dynamics gives rise to another kind of equivalence,
the equivalence of computational behavior. As previously mentioned
this is typically captured \emph{via} some form of bisimulation.

% The notion we use in this paper is weak barbed bisimulation
% \cite{milner91polyadicpi}.

The notion we use in this paper is derived from weak barbed
bisimulation \cite{milner91polyadicpi}. 

\begin{definition}
An \emph{observation relation}, $\downarrow_{\mathcal N}$, over a set
of names, $\mathcal N$, is the smallest relation satisfying the rules
below.

\infrule[Out-barb]{y \in {\mathcal N}, \; x \nameeq y}
		  {\outputp{x}{v} \downarrow_{\mathcal N} x}
\infrule[Par-barb]{\mbox{$P\downarrow_{\mathcal N} x$ or $Q\downarrow_{\mathcal N} x$}}
		  {\binpar{P}{Q} \downarrow_{\mathcal N} x}

We write $P \Downarrow_{\mathcal N} x$ if there is $Q$ such that 
$P \wred Q$ and $Q \downarrow_{\mathcal N} x$.
\end{definition}

\begin{definition}
%\label{def.bbisim}
An  ${\mathcal N}$-\emph{barbed bisimulation} over a set of names, ${\mathcal N}$, is a symmetric binary relation 
${\mathcal S}_{\mathcal N}$ between agents such that $P\rel{S}_{\mathcal N}Q$ implies:
\begin{enumerate}
\item If $P \red P'$ then $Q \wred Q'$ and $P'\rel{S}_{\mathcal N} Q'$.
\item If $P\downarrow_{\mathcal N} x$, then $Q\Downarrow_{\mathcal N} x$.
\end{enumerate}
$P$ is ${\mathcal N}$-barbed bisimilar to $Q$, written
$P \wbbisim_{\mathcal N} Q$, if $P \rel{S}_{\mathcal N} Q$ for some ${\mathcal N}$-barbed bisimulation ${\mathcal S}_{\mathcal N}$.
\end{definition}

$\mathcal{R} \subseteq \pi \times \pi$

$P \mathcal{R} Q => \forall P'. P \red P' \Rightarrow \exists Q'. Q \red Q', P' \mathcal{R} Q'$

$P \vdash x \Rightarrow Q \vdash x$

\begin{mathpar}
  \inferrule*[lab=Out-barb]{x \nameeq y}{{y}!\langle{Q}\rangle \vdash x}
  \and
  \inferrule*[lab=Par-barb]{\mbox{$P\vdash x$ or $Q\vdash x$}}{\binpar{P}{Q} \vdash x}
\end{mathpar}

\subsubsection{Contexts}

One of the principle advantages of computational calculi like the
$\pi$-calculus is a well-defined notion of context,
contextual-equivalence and a correlation between
contextual-equivalence and notions of bisimulation. The notion of
context allows the decomposition of a process into (sub-)process and
its syntactic environment, its context. Thus, a context may be
thought of as a process with a ``hole'' (written $\Box$) in it. The
application of a context $M$ to a process $P$, written $M[P]$, is
tantamount to filling the hole in $M$ with $P$. In this paper we do
not need the full weight of this theory, but do make use of the notion
of context in the proof the main theorem. 

\begin{mathpar}
  \inferrule* [lab=summation] {} {{M_{M},M_{N}} \bc \Box \;|\; x.M_{A} \;|\; M_{M}+M_{N}}
  \and
  \inferrule* [lab=agent] {} {{M_{A}} \bc (\vec{x})M_{P} \;| \; \clift{P_0,\ldots,M_{P},\ldots,P_N}}
  \and \\
  \inferrule* [lab=process] {} {{M_{P}} \bc M_{N} \;| \;P|M_{P} }
\end{mathpar} 

\begin{mathpar}
  \inferrule* [lab=sychronization] {} {M_{N} \bc \Box \;|\; x?M_{F} \;|\; x!M_{C}}
  \and
  \inferrule* [lab=abstraction] {} {{M_{F}} \bc (x)M_{P} }
  \and
  \inferrule* [lab=concretion] {} {{M_{C}} \bc \langle M_{P} \rangle }
  \and \\
  \inferrule* [lab=process] {} {{M_{P}} \bc M_{N} \;| \;P|M_{P} }
\end{mathpar}

\begin{definition}[contextual application] Given a context $M$, and
  process $P$, we define the \emph{contextual application}, $M[P] :=
  M\{P/\Box\}$. That is, the contextual application of M to P is the
  substitution of $P$ for $\Box$ in $M$.
\end{definition}

$\meaningof{-} : L \to \mathcal{P}(\pi)$

\begin{mathpar}
  \inferrule* [lab=collection] {} {\meaningof{true} = \pi, \and \meaningof{~E} = \pi \setminus \meaningof{E}, \and \meaningof{E_{1} \& E_{2}} = \meaningof{E_{1}} \cap \meaningof{E_{2}}}
\end{mathpar}

\begin{mathpar}
  \inferrule* [lab=structure] {} {\meaningof{0} = \{ P \in \pi | P \equiv 0 \}, \and \\ \meaningof{E_1 | E_2} = \{ P \in \pi | P \equiv P_{1} | P_{2}, P_{1} \in \meaningof{E_{1}}, P_{2} \in \meaningof{E_2}\} }
\end{mathpar}

\begin{mathpar}
 \inferrule* [lab=behavior] {} {\meaningof{\langle a?b \rangle E} = \{ P \in \pi | P \equiv Q | u?(y)P', \\ \and \\\\ \and \\ \;\;\; u \in \meaningof{a}, \forall z.P'\{z/y\} \in \meaningof{E\{z/b\}}\}, \and \\ \meaningof{a!E} = \{ P \in \pi | P \equiv Q | x!\langle P' \rangle, x \in \meaningof{a} P' \in \meaningof{E}\} }
\end{mathpar}

\begin{mathpar}
 \inferrule* [lab=nominal] {} {\meaningof{\quotep{E}} = \{ \quotep{P} \in \quotep{\pi} | P \in \meaningof{E} \}, \and \meaningof{\quotep{P}} = \{ \quotep{Q} \in \quotep{\pi} | P \equiv Q \} \and \\ \meaningof{@\quotep{E}} = \{ P \in \pi | P \equiv @x, x \in \meaningof{E} \}}
\end{mathpar}

\begin{eqnarray*}
  \\
  \meaningof{-} : TS \to ST
\end{eqnarray*}

\begin{eqnarray*}
  \\
  L : TS \to ST
\end{eqnarray*}

\begin{eqnarray*}
  \\
  P \models E \iff P \in \meaningof{E}
\end{eqnarray*}

\begin{eqnarray*}
  P \approx_{L} Q \iff \forall E \in L. P \models E \iff Q \models E
\end{eqnarray*}

\begin{eqnarray*}
  P \approx_{K} Q
\end{eqnarray*}

\begin{eqnarray*}
  P \approx Q
\end{eqnarray*}

$\approx_{K} = \approx = \approx_{L}$

\subsubsection{Contextual duality}

Note that contexts extend the quotation operation to a family of
operations from processes to names. Given a context, $M$, we can
define a \emph{nominal context}, $\quotep{M}$ by $\quotep{M}[P] :=
\quotep{M[P]}$. To foreshadow what is to come we observe that these
operations enjoy a duality with processes very much like the duality
between vectors and maps from vectors to scalars.

Further, because the calculus is essentially higher-order, we have a
correspondence between contexts and processes. More specifically,
given a name $x$ and a context $M$ we can construct $M^{*}_{x}$ such
that 

\begin{mathpar}
  M^{*}_{x} | \lift{x}{P} \red M[P]
\end{mathpar}

namely,

\begin{mathpar}
  M^{*}_{x} := x?(u).M[\dropn{u}]
\end{mathpar}

The dependence of $M^{*}_{x}$ on a name makes it an abstraction, 

\begin{mathpar}
  M^{*} := (x)x?(u).M[\dropn{u}]
\end{mathpar}

\subsection{Additional notation}

It will sometimes be convenient to denote the process a name
quotes. We already have the notation $x = \quotep{P}$, but it will be
convenient to introduce an alternate notation, $\procn{x}$, when we
want to emphasize the connection to the use of the name. Note that, by
virtue of name equivalence, $\quotep{\procn{x}} \nameeq x$; so, the
notation is consistent with previous definitions.

Further, because names have structure it is possible to effect
substitutions on the basis of that structure. This means we need to
upgrade our notation for substitutions, which we accomplish by
adapting comprehension notation. Thus,

\begin{mathpar}
  P\{ y / x : x \in S \}
\end{mathpar}

is interpreted to mean the process derived from P by replacing (in a
capture-avoiding manner) each occurrence of $x$ in $S$ by $y$. For example,

\begin{mathpar}
  P\{ \quotep{\procn{x}|\procn{x}} / x : x \in \freenames{P} \}
\end{mathpar}

will replace each (occurrence) of a free name $x$ in $P$ by
$\quotep{\procn{x}|\procn{x}}$.

Also, we will avail ourselves of the notation $x^{L}$ and $x^{R}$ to
denote injections of a name into disjoint copies of the name
space. There are numerous ways to accomplish this. One example can be
found in \cite{MeredithR05}. This notation overloads to vectors of
names: $\vec{x}^{\pi} := (x_{i}^{\pi} \; : \; 0 \leq i < |\vec{x}| )$ where $\pi \in \{L,R\}$.

We also use $P^{\Box} := P|\Box$.

In \cite{MeredithR05} an interpretation of the new operator is
given. It turns out that there are several possible interpretations
all enjoying the requisite algebraic properties of the operator (see
\cite{milner91polyadicpi}). We will therefore make liberal use of
$(\nu\; \vec{x})P$.

% subsection the_syntax_and_semantics_of_the_notation_system (end)   

\input{qm2pi.qmops} 

\input{qm2pi.sterngerlach} 

\input{qm2pi.metric} 

% section concurrent_process_calculi (end)

%\input{qm2pi.proofsketch}

% section proof sketch (end)

%\input{qm2pi.slviaknots} 

% section spatial logic via knots (end)

\input{qm2pi.conclusion}

% section conclusion (end)

%\input{qm2pi.dtcodes} 

% section wiring algorithm (end)

\input{qm2pi.ack} 

% section acknowledgments (end)

\newpage


\bibliographystyle{plain}   
\bibliography{../../biblios/main.bib}

\input{qm2pi.rhodetails}

\end{document}



\end{document}

 

\documentclass[12pt]{llncs}
%\documentclass{jktr}

\usepackage[pdftex]{hyperref}                   
\usepackage {listings}
\usepackage {mathpartir}
\usepackage{bcprules}
%\usepackage{listings}
                       
\usepackage{graphicx} 
%\usepackage[margins=2.5cm,nohead,nofoot]{geometry}
%\usepackage{geometry}
\usepackage{amsfonts}
\usepackage{amstext}
\usepackage{latexsym}
\usepackage{amssymb}
\usepackage{color}


%\include{myPreamble}
\documentclass[12pt]{llncs}
%\documentclass{jktr}

\usepackage[pdftex]{hyperref}                   
\usepackage {listings}
\usepackage {mathpartir}
\usepackage{bcprules}
%\usepackage{listings}
                       
\usepackage{graphicx} 
%\usepackage[margins=2.5cm,nohead,nofoot]{geometry}
%\usepackage{geometry}
\usepackage{amsfonts}
\usepackage{amstext}
\usepackage{latexsym}
\usepackage{amssymb}
\usepackage{color}


%\include{myPreamble}
\include{qm2pi.local} 

%\ifpdf
%\usepackage[pdftex]{graphicx}
%\else
%\usepackage{graphicx}
%\fi

 % \ifpdf
%  \usepackage{pdfsync}
%  \if


%\title{Brief Article}
%\author{David F. Snyder}
%\author{L.G. Meredith}

%\address{Dept. of Math., Texas State University--San Marcos, San Marcos, TX 78666}
       
\pagestyle{empty}


\begin{document}

\lstset{language=[Objective]Caml,frame=shadowbox}

\input{qm2pi.front}

% section front matter (end)

\input{qm2pi.intro} 
 
% section introduction (end)

% \input{qm2pi.knotations} 

% section notation (end)

\input{qm2pi.process.calculi} 

% section concurrent_process_calculi_and_spatial_logics_ (end)
    
%\input{qm2pi.knots2pi} 

%\input{qm2pi.trefoil} 

%\input{qm2pi.mainthm} 

% subsection basic_interpretation (end)

%\input{qm2pi.rho.presentation} 
\subsection{The syntax and semantics of the notation system}\label{sub:the_syntax_and_semantics_of_the_notation_system} % (fold)

We now summarize a technical presentation of the calculus that
embodies our theory of dynamics. The typical presentation of such a
calculus follows the style of giving generators and relations on
them. The grammar, below, describing term constructors, freely
generates the set of processes, $\Proc$. This set is then quotiented
by a relation known as structural congruence and it is over this set
that the notion of dynamics is expressed. This presentation is
essentially that of \cite{MeredithR05} with the addition of
polyadicity and summation. For readability we have relegated some of
the technical subtleties to an appendix.

\subsubsection{Process grammar}\label{subsub:process_grammar}

\begin{mathpar}
  \inferrule* [lab=synchronization] {} {{M} \bc \pzero \;|\; x?F \;|\; x!C }
  \and
  \inferrule* [lab=abstraction] {} {{F} \bc (x)P}
  \and
  \inferrule* [lab=concretion] {} {{C} \bc \langle Q \rangle}
  \and
  \inferrule* [lab=process] {} {{P,Q} \bc M \;| \;P|Q \;|\; @{x}}
  \and
  \inferrule* [lab=name] {} {{x} \bc \quotep{P}}
\end{mathpar} 

Note that $\vec{x}$ (resp. $\vec{P}$) denotes a vector of names
(resp. processes) of length $|\vec{x}|$ (resp. $|\vec{P}|$). We adopt
the following useful abbreviations.

\begin{mathpar}
   x?(\vec{y}).P := x.(\vec{y})P \and  x\clift{\vec{P}} := x.\clift{\vec{P}}
   \and x!(y) := \lift{x}{\dropn{y}}
   \and \Pi_{i=0}^{n-1}P_i := P_0 | \ldots | P_{n-1}
\end{mathpar}

\subsubsection{Structural congruence}

\paragraph{Free and bound names and alpha-equivalence.} At the
core of structural equivalence is alpha-equivalence which identifies
process that are the same up to a change of variable. Formally, we
recognize the distinction between free and bound names. The free names
of a process, $\freenames{P}$, may be calculated recursively as
follows:

\begin{mathpar}
\freenames{\pzero} := \emptyset
  \and \\
  \freenames{x?(y).P} := \{ x \} \cup (\freenames{P} \setminus \{ y \})
  \and 
  \freenames{x!\langle P \rangle} := \{ x \} \cup \{ P \} 
  \and \\
  \freenames{P|Q} := \freenames{P} \cup \freenames{Q}
  \and \\
  \freenames{@{x}} := \{ x \}
\end{mathpar}

$\pi$
$\quotep{\pi}$

$\freenames{-} : \pi \to \mathcal{P}(\quotep{\pi})$

\begin{eqnarray*}
  \freenames{\pzero} & := & \emptyset \\
  \freenames{x?(y).P} & := & \{ x \} \cup (\freenames{P} \setminus \{ y \}) \\
  \freenames{x!\langle P \rangle} & := & \{ x \} \cup \{ P \} \\
  \freenames{P|Q} & := & \freenames{P} \cup \freenames{Q} \\
  \freenames{\dropn{x}} & := & \{ x \}
\end{eqnarray*}

The bound names of a process, $\boundnames{P}$, are those names occurring in $P$
that are not free. For example, in $x?(y).0$, the name $x$ is free, while $y$ is bound.

\begin{mathpar}
  \inferrule* [lab=monoidal-laws] {} { P|Q \equiv Q|P \and P|0 \equiv P \and P|(Q|R) \equiv (P|Q)|R }
\end{mathpar}

\begin{mathpar}
  \inferrule* [lab=alpha-equivalence] {} { (x)P \equiv (y)P\{y/x\} \and y \not\in \freenames{P} }
\end{mathpar}

\begin{definition}
Then two processes, $P,Q$, are alpha-equivalent if $P = Q\{\vec{y}/\vec{x}\}$ for
some $\vec{x} \in \boundnames{Q},\vec{y} \in \boundnames{P}$, where $Q\{\vec{y}/\vec{x}\}$
denotes the capture-avoiding substitution of $\vec{y}$ for $\vec{x}$ in $Q$.
\end{definition}

\begin{definition}
  The {\em structural congruence} \cite{SangiorgiWalker} , $\equiv$,
  between processes is the least congruence containing
  alpha-equivalence, satisfying the abelian monoid laws
  (associativity, commutativity and $\pzero$ as identity) for parallel
  composition $|$ and for summation $+$.
\end{definition}

\subsection{Name equivalence}

We take name equivalence, written $\nameeq$, to be the smallest
equivalence relation generated by the following rules.

\begin{mathpar}
\inferrule*[lab=Quote-drop]
{ }
{ \quotep{@{x}} \nameeq x }

\inferrule*[lab=Struct-equiv]
{ P \scong Q }
{ \quotep{P} \nameeq \quotep{Q} }
\end{mathpar}

The astute reader will have noticed that the mutual recursion of names
and processes imposes a mutual recursion on alpha-equivalence and
structural equivalence via name-equivalence. Fortunately, all of this
works out pleasantly and we may calculate in the natural way, free of
concern. The reader interested in the details is referred to the
appendix \ref{appendix:rho_details}.

\subsection{Substitution}

We use $\Proc$ for the set of processes, $\QProc$ for the set of
names, and $\id{\{}\vec{y} / \vec{x} \id{\}}$ to denote partial maps,
$s : \QProc \rightarrow \QProc$. A map, $s$ lifts, uniquely, to a map
on process terms, $\widehat{s} : \Proc \rightarrow \Proc$ by the
following equations.

\begin{mathpar}
  (0) \psubstp{Q}{P} := 0 \\
  (R \juxtap S) \psubstp{Q}{P}
  :=    
  (R)\psubstp{Q}{P} \juxtap (S) \psubstp{Q}{P} \\
  (x?(y).R) \psubstp{Q}{P}    
  :=    
  (x)\substp{Q}{P} (z)\concat( (R \psubstn{z}{y}) \psubstp{Q}{P} ) \\
  (\lift{x}{R}) \psubstp{Q}{P}  
  :=
  \lift{(x)\substp{Q}{P}}{ R \psubstp{Q}{P} } \\
%   (\dropn{x})  \psubstp{Q}{P}       
%   := 
%   \left\{ 
%     \begin{array}{ccc} 
%       \dropn{\quotep{Q}} & & x \nameeq \quotep{P} \\
%       \dropn{x} & & otherwise \\
%     \end{array}
%   \right. 
  (\dropn{x})  \psubstp{Q}{P}       
  := 
  \left\{ 
    \begin{array}{ccc} 
      Q & & x \nameeq \quotep{P} \\
      \dropn{x} & & otherwise \\
    \end{array}
  \right.
\end{mathpar}
 

where

\begin{eqnarray}
  (x)\id{\{} \lpquote Q \rpquote / \lpquote P \rpquote \id{\}}            = 
  \left\{ 
    \begin{array}{ccc}
      \lpquote Q \rpquote & & x \nameeq \lpquote P \rpquote \\
      x & & otherwise \\
    \end{array}
  \right. \nonumber
\end{eqnarray}

and $z$ is chosen distinct from $\quotep{P}$, $\quotep{Q}$, the free
names in $Q$, and all the names in $R$. Our $\alpha$-equivalence will
be built in the standard way from this substitution.

\begin{remark}\label{rem:no_self_referential_names}
  One consequence of these definitions is that $\forall P. \quotep{P}
  \not\in \freenames{P}$.
\end{remark}

\subsection{ Dynamic quote: an example }

Anticipating something of what's to come, consider applying the
substitution, $\widehat{\id{\{}u / z \id{\}}}$, to the following pair
of processes, $\lift{w}{y!(z)}$ and $w[ \lpquote y!(z) \rpquote ]$.

\begin{eqnarray}
	\lift{w}{y!(z)}\widehat{\id{\{}u / z \id{\}}}
		& = &
		\lift{w}{y!(u)} \nonumber\\
	w[ \lpquote y!(z) \rpquote ] \widehat{ \id{\{}u / z \id{\}} }
		& = &
		w[ \lpquote y!(z) \rpquote ] \nonumber
\end{eqnarray}

Because the body of the process between quotes is impervious to
substitution, we get radically different answers. In fact, by
examining the first process in an input context,
e.g. $x?(z).\lift{w}{y!(z)}$, we see that the process under the lift
operator may be shaped by prefixed inputs binding a name inside it. In
this sense, the lift operator will be seen as a way to dynamically
construct processes before reifying them as names.

Finally equipped with these standard features we can present the
dynamics of the calculus.

\subsubsection{Operational semantics} 

Finally, we introduce the computational dynamics. What marks these
algebras as distinct from other more traditionally studied algebraic
structures, e.g. vector spaces or polynomial rings, is the manner in
which dynamics is captured. In traditional structures, dynamics is typically
expressed through morphisms between such structures, as in linear maps
between vector spaces or morphisms between rings. In algebras
associated with the semantics of computation, the dynamics is
expressed as part of the algebraic structure itself, through a
reduction reduction relation typically denoted by $\red$. Below, we
give a recursive presentation of this relation for the calculus used
in the encoding.

$\red \subseteq \pi \times \pi$
$\red : \pi \to \mathcal{P}(\pi)$

\begin{mathpar}
  \inferrule* [lab=Comm] { \textsf{match}( x_{src}, x_{trgt} ) } { x_{trgt}?(y)P \; | \; x_{src}!\langle {Q} \rangle \red P\{\quotep{Q}/y}\} }
  \and \\
  \inferrule* [lab=Par] {{P} \red {P}'} {{{P} | {Q}} \red {{P}' | {Q}}}
  \and
  \inferrule* [lab=Equiv]{{{P} \scong {P}'} \andalso {{P}' \red {Q}'} \andalso {{Q}' \scong {Q}}}{{P} \red {Q}}
\end{mathpar}

\begin{eqnarray*}
  match_{\equiv} (\quotep{P},\quotep{Q}) & := & P \equiv Q \\
  match_{\dagger}(\quotep{P},\quotep{Q}) & := & \forall R. P|Q \red^{*} R => R \red^{*} 0 \\
  match_{K}(\quotep{P},\quotep{Q}) & := & K \mbox{ for some context } K
\end{eqnarray*}

$u?(x)P | u!\langle Q \rangle \red P\{\quotep{Q}/x\}$

%We write $\wred$ for $\red^*$, and $P\red$ if $\exists Q $ such that $ P \red Q$.
We write $P\red$ if $\exists Q $ such that $ P \red Q$ and $P\not\red$, otherwise.

\section{Replication}

As mentioned before, it is known that replication (and hence
recursion) can be implemented in a higher-order process algebra
\cite{SangiorgiWalker}. As our first example of calculation with the
machinery thus far presented we give the construction explicitly in
the {\rhoc}.

\begin{eqnarray}
	D_{x} & := & \prefix{x}{y}{(\binpar{\outputp{x}{y}}{@{y}})} \nonumber\\
	\bangp_{x}{P} & := & \binpar{{x}!\langle{\binpar{D_{x}}{P}}\rangle}{D_{x}} \nonumber
\end{eqnarray}

\begin{eqnarray}
	\bangp_{x}{P} & & \nonumber\\
	=
	& {x}!\langle{(\prefix{x}{y}{(\outputp{x}{y} | @{y})) | P}}\rangle 
	      | \prefix{x}{y}{(\outputp{x}{y} | @{y})} & \nonumber\\
	\red
	& (\outputp{x}{y} | @{y})\substn{\quotep{(\prefix{x}{y}{(@{y} | \outputp{x}{y})) | P}}}{y} & \nonumber\\
	=
	& \outputp{x}{\quotep{(\prefix{x}{y}{(\outputp{x}{y} | @{y})) | P}}}
	  | {(\prefix{x}{y}{(\outputp{x}{y} | @{y})) | P}} & \nonumber\\
	\red
	& \ldots & \nonumber\\
	\red^*
	& P | P | \ldots & \nonumber
\end{eqnarray}

Of course, this encoding, as an implementation, runs away, unfolding
$\bangp{P}$ eagerly. A lazier and more implementable replication
operator, restricted to input-guarded processes, may be obtained as follows.

\begin{eqnarray}
\bangp{\prefix{u}{v}{P}} 
	:= 
	\binpar{\lift{x}{\prefix{u}{v}{(\binpar{D(x)}{P})}}}{D(x)} \nonumber
\end{eqnarray}

\begin{remark}
  Note that the lazier definition still does not deal with summation
  or mixed summation (i.e. sums over input and output). The reader is
  invited to construct definitions of replication that deal with these
  features. 

  Further, the definitions are parameterized in a name, $x$. Can you,
  gentle reader, make a definition that eliminates this parameter and
  guarantees no accidental interaction between the replication
  machinery and the process being replicated -- i.e. no accidental
  sharing of names used by the process to get its work done and the
  name(s) used by the replication to effect copying. This latter
  revision of the definition of replication is crucial to obtaining
  the expected identity $!!P \sim !P$.
\end{remark}

\begin{remark}\label{rem:paradoxical_combinator}
  The reader familiar with the lambda calculus will have noticed the
  similarity between $D$ and the paradoxical combinator.

  [Ed. note: the existence of this seems to suggest we have to be more
  restrictive on the set of processes and names we admit if we are to
  support no-cloning.]
\end{remark}

\subsubsection{Bisimulation}

The computational dynamics gives rise to another kind of equivalence,
the equivalence of computational behavior. As previously mentioned
this is typically captured \emph{via} some form of bisimulation.

% The notion we use in this paper is weak barbed bisimulation
% \cite{milner91polyadicpi}.

The notion we use in this paper is derived from weak barbed
bisimulation \cite{milner91polyadicpi}. 

\begin{definition}
An \emph{observation relation}, $\downarrow_{\mathcal N}$, over a set
of names, $\mathcal N$, is the smallest relation satisfying the rules
below.

\infrule[Out-barb]{y \in {\mathcal N}, \; x \nameeq y}
		  {\outputp{x}{v} \downarrow_{\mathcal N} x}
\infrule[Par-barb]{\mbox{$P\downarrow_{\mathcal N} x$ or $Q\downarrow_{\mathcal N} x$}}
		  {\binpar{P}{Q} \downarrow_{\mathcal N} x}

We write $P \Downarrow_{\mathcal N} x$ if there is $Q$ such that 
$P \wred Q$ and $Q \downarrow_{\mathcal N} x$.
\end{definition}

\begin{definition}
%\label{def.bbisim}
An  ${\mathcal N}$-\emph{barbed bisimulation} over a set of names, ${\mathcal N}$, is a symmetric binary relation 
${\mathcal S}_{\mathcal N}$ between agents such that $P\rel{S}_{\mathcal N}Q$ implies:
\begin{enumerate}
\item If $P \red P'$ then $Q \wred Q'$ and $P'\rel{S}_{\mathcal N} Q'$.
\item If $P\downarrow_{\mathcal N} x$, then $Q\Downarrow_{\mathcal N} x$.
\end{enumerate}
$P$ is ${\mathcal N}$-barbed bisimilar to $Q$, written
$P \wbbisim_{\mathcal N} Q$, if $P \rel{S}_{\mathcal N} Q$ for some ${\mathcal N}$-barbed bisimulation ${\mathcal S}_{\mathcal N}$.
\end{definition}

$\mathcal{R} \subseteq \pi \times \pi$

$P \mathcal{R} Q => \forall P'. P \red P' \Rightarrow \exists Q'. Q \red Q', P' \mathcal{R} Q'$

$P \vdash x \Rightarrow Q \vdash x$

\begin{mathpar}
  \inferrule*[lab=Out-barb]{x \nameeq y}{{y}!\langle{Q}\rangle \vdash x}
  \and
  \inferrule*[lab=Par-barb]{\mbox{$P\vdash x$ or $Q\vdash x$}}{\binpar{P}{Q} \vdash x}
\end{mathpar}

\subsubsection{Contexts}

One of the principle advantages of computational calculi like the
$\pi$-calculus is a well-defined notion of context,
contextual-equivalence and a correlation between
contextual-equivalence and notions of bisimulation. The notion of
context allows the decomposition of a process into (sub-)process and
its syntactic environment, its context. Thus, a context may be
thought of as a process with a ``hole'' (written $\Box$) in it. The
application of a context $M$ to a process $P$, written $M[P]$, is
tantamount to filling the hole in $M$ with $P$. In this paper we do
not need the full weight of this theory, but do make use of the notion
of context in the proof the main theorem. 

\begin{mathpar}
  \inferrule* [lab=summation] {} {{M_{M},M_{N}} \bc \Box \;|\; x.M_{A} \;|\; M_{M}+M_{N}}
  \and
  \inferrule* [lab=agent] {} {{M_{A}} \bc (\vec{x})M_{P} \;| \; \clift{P_0,\ldots,M_{P},\ldots,P_N}}
  \and \\
  \inferrule* [lab=process] {} {{M_{P}} \bc M_{N} \;| \;P|M_{P} }
\end{mathpar} 

\begin{mathpar}
  \inferrule* [lab=sychronization] {} {M_{N} \bc \Box \;|\; x?M_{F} \;|\; x!M_{C}}
  \and
  \inferrule* [lab=abstraction] {} {{M_{F}} \bc (x)M_{P} }
  \and
  \inferrule* [lab=concretion] {} {{M_{C}} \bc \langle M_{P} \rangle }
  \and \\
  \inferrule* [lab=process] {} {{M_{P}} \bc M_{N} \;| \;P|M_{P} }
\end{mathpar}

\begin{definition}[contextual application] Given a context $M$, and
  process $P$, we define the \emph{contextual application}, $M[P] :=
  M\{P/\Box\}$. That is, the contextual application of M to P is the
  substitution of $P$ for $\Box$ in $M$.
\end{definition}

$\meaningof{-} : L \to \mathcal{P}(\pi)$

\begin{mathpar}
  \inferrule* [lab=collection] {} {\meaningof{true} = \pi, \and \meaningof{~E} = \pi \setminus \meaningof{E}, \and \meaningof{E_{1} \& E_{2}} = \meaningof{E_{1}} \cap \meaningof{E_{2}}}
\end{mathpar}

\begin{mathpar}
  \inferrule* [lab=structure] {} {\meaningof{0} = \{ P \in \pi | P \equiv 0 \}, \and \\ \meaningof{E_1 | E_2} = \{ P \in \pi | P \equiv P_{1} | P_{2}, P_{1} \in \meaningof{E_{1}}, P_{2} \in \meaningof{E_2}\} }
\end{mathpar}

\begin{mathpar}
 \inferrule* [lab=behavior] {} {\meaningof{\langle a?b \rangle E} = \{ P \in \pi | P \equiv Q | u?(y)P', \\ \and \\\\ \and \\ \;\;\; u \in \meaningof{a}, \forall z.P'\{z/y\} \in \meaningof{E\{z/b\}}\}, \and \\ \meaningof{a!E} = \{ P \in \pi | P \equiv Q | x!\langle P' \rangle, x \in \meaningof{a} P' \in \meaningof{E}\} }
\end{mathpar}

\begin{mathpar}
 \inferrule* [lab=nominal] {} {\meaningof{\quotep{E}} = \{ \quotep{P} \in \quotep{\pi} | P \in \meaningof{E} \}, \and \meaningof{\quotep{P}} = \{ \quotep{Q} \in \quotep{\pi} | P \equiv Q \} \and \\ \meaningof{@\quotep{E}} = \{ P \in \pi | P \equiv @x, x \in \meaningof{E} \}}
\end{mathpar}

\begin{eqnarray*}
  \\
  \meaningof{-} : TS \to ST
\end{eqnarray*}

\begin{eqnarray*}
  \\
  L : TS \to ST
\end{eqnarray*}

\begin{eqnarray*}
  \\
  P \models E \iff P \in \meaningof{E}
\end{eqnarray*}

\begin{eqnarray*}
  P \approx_{L} Q \iff \forall E \in L. P \models E \iff Q \models E
\end{eqnarray*}

\begin{eqnarray*}
  P \approx_{K} Q
\end{eqnarray*}

\begin{eqnarray*}
  P \approx Q
\end{eqnarray*}

$\approx_{K} = \approx = \approx_{L}$

\subsubsection{Contextual duality}

Note that contexts extend the quotation operation to a family of
operations from processes to names. Given a context, $M$, we can
define a \emph{nominal context}, $\quotep{M}$ by $\quotep{M}[P] :=
\quotep{M[P]}$. To foreshadow what is to come we observe that these
operations enjoy a duality with processes very much like the duality
between vectors and maps from vectors to scalars.

Further, because the calculus is essentially higher-order, we have a
correspondence between contexts and processes. More specifically,
given a name $x$ and a context $M$ we can construct $M^{*}_{x}$ such
that 

\begin{mathpar}
  M^{*}_{x} | \lift{x}{P} \red M[P]
\end{mathpar}

namely,

\begin{mathpar}
  M^{*}_{x} := x?(u).M[\dropn{u}]
\end{mathpar}

The dependence of $M^{*}_{x}$ on a name makes it an abstraction, 

\begin{mathpar}
  M^{*} := (x)x?(u).M[\dropn{u}]
\end{mathpar}

\subsection{Additional notation}

It will sometimes be convenient to denote the process a name
quotes. We already have the notation $x = \quotep{P}$, but it will be
convenient to introduce an alternate notation, $\procn{x}$, when we
want to emphasize the connection to the use of the name. Note that, by
virtue of name equivalence, $\quotep{\procn{x}} \nameeq x$; so, the
notation is consistent with previous definitions.

Further, because names have structure it is possible to effect
substitutions on the basis of that structure. This means we need to
upgrade our notation for substitutions, which we accomplish by
adapting comprehension notation. Thus,

\begin{mathpar}
  P\{ y / x : x \in S \}
\end{mathpar}

is interpreted to mean the process derived from P by replacing (in a
capture-avoiding manner) each occurrence of $x$ in $S$ by $y$. For example,

\begin{mathpar}
  P\{ \quotep{\procn{x}|\procn{x}} / x : x \in \freenames{P} \}
\end{mathpar}

will replace each (occurrence) of a free name $x$ in $P$ by
$\quotep{\procn{x}|\procn{x}}$.

Also, we will avail ourselves of the notation $x^{L}$ and $x^{R}$ to
denote injections of a name into disjoint copies of the name
space. There are numerous ways to accomplish this. One example can be
found in \cite{MeredithR05}. This notation overloads to vectors of
names: $\vec{x}^{\pi} := (x_{i}^{\pi} \; : \; 0 \leq i < |\vec{x}| )$ where $\pi \in \{L,R\}$.

We also use $P^{\Box} := P|\Box$.

In \cite{MeredithR05} an interpretation of the new operator is
given. It turns out that there are several possible interpretations
all enjoying the requisite algebraic properties of the operator (see
\cite{milner91polyadicpi}). We will therefore make liberal use of
$(\nu\; \vec{x})P$.

% subsection the_syntax_and_semantics_of_the_notation_system (end)   

\input{qm2pi.qmops} 

\input{qm2pi.sterngerlach} 

\input{qm2pi.metric} 

% section concurrent_process_calculi (end)

%\input{qm2pi.proofsketch}

% section proof sketch (end)

%\input{qm2pi.slviaknots} 

% section spatial logic via knots (end)

\input{qm2pi.conclusion}

% section conclusion (end)

%\input{qm2pi.dtcodes} 

% section wiring algorithm (end)

\input{qm2pi.ack} 

% section acknowledgments (end)

\newpage


\bibliographystyle{plain}   
\bibliography{../../biblios/main.bib}

\input{qm2pi.rhodetails}

\end{document}

 

%\ifpdf
%\usepackage[pdftex]{graphicx}
%\else
%\usepackage{graphicx}
%\fi

 % \ifpdf
%  \usepackage{pdfsync}
%  \if


%\title{Brief Article}
%\author{David F. Snyder}
%\author{L.G. Meredith}

%\address{Dept. of Math., Texas State University--San Marcos, San Marcos, TX 78666}
       
\pagestyle{empty}


\begin{document}

\lstset{language=[Objective]Caml,frame=shadowbox}

\documentclass[12pt]{llncs}
%\documentclass{jktr}

\usepackage[pdftex]{hyperref}                   
\usepackage {listings}
\usepackage {mathpartir}
\usepackage{bcprules}
%\usepackage{listings}
                       
\usepackage{graphicx} 
%\usepackage[margins=2.5cm,nohead,nofoot]{geometry}
%\usepackage{geometry}
\usepackage{amsfonts}
\usepackage{amstext}
\usepackage{latexsym}
\usepackage{amssymb}
\usepackage{color}


%\include{myPreamble}
\include{qm2pi.local} 

%\ifpdf
%\usepackage[pdftex]{graphicx}
%\else
%\usepackage{graphicx}
%\fi

 % \ifpdf
%  \usepackage{pdfsync}
%  \if


%\title{Brief Article}
%\author{David F. Snyder}
%\author{L.G. Meredith}

%\address{Dept. of Math., Texas State University--San Marcos, San Marcos, TX 78666}
       
\pagestyle{empty}


\begin{document}

\lstset{language=[Objective]Caml,frame=shadowbox}

\input{qm2pi.front}

% section front matter (end)

\input{qm2pi.intro} 
 
% section introduction (end)

% \input{qm2pi.knotations} 

% section notation (end)

\input{qm2pi.process.calculi} 

% section concurrent_process_calculi_and_spatial_logics_ (end)
    
%\input{qm2pi.knots2pi} 

%\input{qm2pi.trefoil} 

%\input{qm2pi.mainthm} 

% subsection basic_interpretation (end)

%\input{qm2pi.rho.presentation} 
\subsection{The syntax and semantics of the notation system}\label{sub:the_syntax_and_semantics_of_the_notation_system} % (fold)

We now summarize a technical presentation of the calculus that
embodies our theory of dynamics. The typical presentation of such a
calculus follows the style of giving generators and relations on
them. The grammar, below, describing term constructors, freely
generates the set of processes, $\Proc$. This set is then quotiented
by a relation known as structural congruence and it is over this set
that the notion of dynamics is expressed. This presentation is
essentially that of \cite{MeredithR05} with the addition of
polyadicity and summation. For readability we have relegated some of
the technical subtleties to an appendix.

\subsubsection{Process grammar}\label{subsub:process_grammar}

\begin{mathpar}
  \inferrule* [lab=synchronization] {} {{M} \bc \pzero \;|\; x?F \;|\; x!C }
  \and
  \inferrule* [lab=abstraction] {} {{F} \bc (x)P}
  \and
  \inferrule* [lab=concretion] {} {{C} \bc \langle Q \rangle}
  \and
  \inferrule* [lab=process] {} {{P,Q} \bc M \;| \;P|Q \;|\; @{x}}
  \and
  \inferrule* [lab=name] {} {{x} \bc \quotep{P}}
\end{mathpar} 

Note that $\vec{x}$ (resp. $\vec{P}$) denotes a vector of names
(resp. processes) of length $|\vec{x}|$ (resp. $|\vec{P}|$). We adopt
the following useful abbreviations.

\begin{mathpar}
   x?(\vec{y}).P := x.(\vec{y})P \and  x\clift{\vec{P}} := x.\clift{\vec{P}}
   \and x!(y) := \lift{x}{\dropn{y}}
   \and \Pi_{i=0}^{n-1}P_i := P_0 | \ldots | P_{n-1}
\end{mathpar}

\subsubsection{Structural congruence}

\paragraph{Free and bound names and alpha-equivalence.} At the
core of structural equivalence is alpha-equivalence which identifies
process that are the same up to a change of variable. Formally, we
recognize the distinction between free and bound names. The free names
of a process, $\freenames{P}$, may be calculated recursively as
follows:

\begin{mathpar}
\freenames{\pzero} := \emptyset
  \and \\
  \freenames{x?(y).P} := \{ x \} \cup (\freenames{P} \setminus \{ y \})
  \and 
  \freenames{x!\langle P \rangle} := \{ x \} \cup \{ P \} 
  \and \\
  \freenames{P|Q} := \freenames{P} \cup \freenames{Q}
  \and \\
  \freenames{@{x}} := \{ x \}
\end{mathpar}

$\pi$
$\quotep{\pi}$

$\freenames{-} : \pi \to \mathcal{P}(\quotep{\pi})$

\begin{eqnarray*}
  \freenames{\pzero} & := & \emptyset \\
  \freenames{x?(y).P} & := & \{ x \} \cup (\freenames{P} \setminus \{ y \}) \\
  \freenames{x!\langle P \rangle} & := & \{ x \} \cup \{ P \} \\
  \freenames{P|Q} & := & \freenames{P} \cup \freenames{Q} \\
  \freenames{\dropn{x}} & := & \{ x \}
\end{eqnarray*}

The bound names of a process, $\boundnames{P}$, are those names occurring in $P$
that are not free. For example, in $x?(y).0$, the name $x$ is free, while $y$ is bound.

\begin{mathpar}
  \inferrule* [lab=monoidal-laws] {} { P|Q \equiv Q|P \and P|0 \equiv P \and P|(Q|R) \equiv (P|Q)|R }
\end{mathpar}

\begin{mathpar}
  \inferrule* [lab=alpha-equivalence] {} { (x)P \equiv (y)P\{y/x\} \and y \not\in \freenames{P} }
\end{mathpar}

\begin{definition}
Then two processes, $P,Q$, are alpha-equivalent if $P = Q\{\vec{y}/\vec{x}\}$ for
some $\vec{x} \in \boundnames{Q},\vec{y} \in \boundnames{P}$, where $Q\{\vec{y}/\vec{x}\}$
denotes the capture-avoiding substitution of $\vec{y}$ for $\vec{x}$ in $Q$.
\end{definition}

\begin{definition}
  The {\em structural congruence} \cite{SangiorgiWalker} , $\equiv$,
  between processes is the least congruence containing
  alpha-equivalence, satisfying the abelian monoid laws
  (associativity, commutativity and $\pzero$ as identity) for parallel
  composition $|$ and for summation $+$.
\end{definition}

\subsection{Name equivalence}

We take name equivalence, written $\nameeq$, to be the smallest
equivalence relation generated by the following rules.

\begin{mathpar}
\inferrule*[lab=Quote-drop]
{ }
{ \quotep{@{x}} \nameeq x }

\inferrule*[lab=Struct-equiv]
{ P \scong Q }
{ \quotep{P} \nameeq \quotep{Q} }
\end{mathpar}

The astute reader will have noticed that the mutual recursion of names
and processes imposes a mutual recursion on alpha-equivalence and
structural equivalence via name-equivalence. Fortunately, all of this
works out pleasantly and we may calculate in the natural way, free of
concern. The reader interested in the details is referred to the
appendix \ref{appendix:rho_details}.

\subsection{Substitution}

We use $\Proc$ for the set of processes, $\QProc$ for the set of
names, and $\id{\{}\vec{y} / \vec{x} \id{\}}$ to denote partial maps,
$s : \QProc \rightarrow \QProc$. A map, $s$ lifts, uniquely, to a map
on process terms, $\widehat{s} : \Proc \rightarrow \Proc$ by the
following equations.

\begin{mathpar}
  (0) \psubstp{Q}{P} := 0 \\
  (R \juxtap S) \psubstp{Q}{P}
  :=    
  (R)\psubstp{Q}{P} \juxtap (S) \psubstp{Q}{P} \\
  (x?(y).R) \psubstp{Q}{P}    
  :=    
  (x)\substp{Q}{P} (z)\concat( (R \psubstn{z}{y}) \psubstp{Q}{P} ) \\
  (\lift{x}{R}) \psubstp{Q}{P}  
  :=
  \lift{(x)\substp{Q}{P}}{ R \psubstp{Q}{P} } \\
%   (\dropn{x})  \psubstp{Q}{P}       
%   := 
%   \left\{ 
%     \begin{array}{ccc} 
%       \dropn{\quotep{Q}} & & x \nameeq \quotep{P} \\
%       \dropn{x} & & otherwise \\
%     \end{array}
%   \right. 
  (\dropn{x})  \psubstp{Q}{P}       
  := 
  \left\{ 
    \begin{array}{ccc} 
      Q & & x \nameeq \quotep{P} \\
      \dropn{x} & & otherwise \\
    \end{array}
  \right.
\end{mathpar}
 

where

\begin{eqnarray}
  (x)\id{\{} \lpquote Q \rpquote / \lpquote P \rpquote \id{\}}            = 
  \left\{ 
    \begin{array}{ccc}
      \lpquote Q \rpquote & & x \nameeq \lpquote P \rpquote \\
      x & & otherwise \\
    \end{array}
  \right. \nonumber
\end{eqnarray}

and $z$ is chosen distinct from $\quotep{P}$, $\quotep{Q}$, the free
names in $Q$, and all the names in $R$. Our $\alpha$-equivalence will
be built in the standard way from this substitution.

\begin{remark}\label{rem:no_self_referential_names}
  One consequence of these definitions is that $\forall P. \quotep{P}
  \not\in \freenames{P}$.
\end{remark}

\subsection{ Dynamic quote: an example }

Anticipating something of what's to come, consider applying the
substitution, $\widehat{\id{\{}u / z \id{\}}}$, to the following pair
of processes, $\lift{w}{y!(z)}$ and $w[ \lpquote y!(z) \rpquote ]$.

\begin{eqnarray}
	\lift{w}{y!(z)}\widehat{\id{\{}u / z \id{\}}}
		& = &
		\lift{w}{y!(u)} \nonumber\\
	w[ \lpquote y!(z) \rpquote ] \widehat{ \id{\{}u / z \id{\}} }
		& = &
		w[ \lpquote y!(z) \rpquote ] \nonumber
\end{eqnarray}

Because the body of the process between quotes is impervious to
substitution, we get radically different answers. In fact, by
examining the first process in an input context,
e.g. $x?(z).\lift{w}{y!(z)}$, we see that the process under the lift
operator may be shaped by prefixed inputs binding a name inside it. In
this sense, the lift operator will be seen as a way to dynamically
construct processes before reifying them as names.

Finally equipped with these standard features we can present the
dynamics of the calculus.

\subsubsection{Operational semantics} 

Finally, we introduce the computational dynamics. What marks these
algebras as distinct from other more traditionally studied algebraic
structures, e.g. vector spaces or polynomial rings, is the manner in
which dynamics is captured. In traditional structures, dynamics is typically
expressed through morphisms between such structures, as in linear maps
between vector spaces or morphisms between rings. In algebras
associated with the semantics of computation, the dynamics is
expressed as part of the algebraic structure itself, through a
reduction reduction relation typically denoted by $\red$. Below, we
give a recursive presentation of this relation for the calculus used
in the encoding.

$\red \subseteq \pi \times \pi$
$\red : \pi \to \mathcal{P}(\pi)$

\begin{mathpar}
  \inferrule* [lab=Comm] { \textsf{match}( x_{src}, x_{trgt} ) } { x_{trgt}?(y)P \; | \; x_{src}!\langle {Q} \rangle \red P\{\quotep{Q}/y}\} }
  \and \\
  \inferrule* [lab=Par] {{P} \red {P}'} {{{P} | {Q}} \red {{P}' | {Q}}}
  \and
  \inferrule* [lab=Equiv]{{{P} \scong {P}'} \andalso {{P}' \red {Q}'} \andalso {{Q}' \scong {Q}}}{{P} \red {Q}}
\end{mathpar}

\begin{eqnarray*}
  match_{\equiv} (\quotep{P},\quotep{Q}) & := & P \equiv Q \\
  match_{\dagger}(\quotep{P},\quotep{Q}) & := & \forall R. P|Q \red^{*} R => R \red^{*} 0 \\
  match_{K}(\quotep{P},\quotep{Q}) & := & K \mbox{ for some context } K
\end{eqnarray*}

$u?(x)P | u!\langle Q \rangle \red P\{\quotep{Q}/x\}$

%We write $\wred$ for $\red^*$, and $P\red$ if $\exists Q $ such that $ P \red Q$.
We write $P\red$ if $\exists Q $ such that $ P \red Q$ and $P\not\red$, otherwise.

\section{Replication}

As mentioned before, it is known that replication (and hence
recursion) can be implemented in a higher-order process algebra
\cite{SangiorgiWalker}. As our first example of calculation with the
machinery thus far presented we give the construction explicitly in
the {\rhoc}.

\begin{eqnarray}
	D_{x} & := & \prefix{x}{y}{(\binpar{\outputp{x}{y}}{@{y}})} \nonumber\\
	\bangp_{x}{P} & := & \binpar{{x}!\langle{\binpar{D_{x}}{P}}\rangle}{D_{x}} \nonumber
\end{eqnarray}

\begin{eqnarray}
	\bangp_{x}{P} & & \nonumber\\
	=
	& {x}!\langle{(\prefix{x}{y}{(\outputp{x}{y} | @{y})) | P}}\rangle 
	      | \prefix{x}{y}{(\outputp{x}{y} | @{y})} & \nonumber\\
	\red
	& (\outputp{x}{y} | @{y})\substn{\quotep{(\prefix{x}{y}{(@{y} | \outputp{x}{y})) | P}}}{y} & \nonumber\\
	=
	& \outputp{x}{\quotep{(\prefix{x}{y}{(\outputp{x}{y} | @{y})) | P}}}
	  | {(\prefix{x}{y}{(\outputp{x}{y} | @{y})) | P}} & \nonumber\\
	\red
	& \ldots & \nonumber\\
	\red^*
	& P | P | \ldots & \nonumber
\end{eqnarray}

Of course, this encoding, as an implementation, runs away, unfolding
$\bangp{P}$ eagerly. A lazier and more implementable replication
operator, restricted to input-guarded processes, may be obtained as follows.

\begin{eqnarray}
\bangp{\prefix{u}{v}{P}} 
	:= 
	\binpar{\lift{x}{\prefix{u}{v}{(\binpar{D(x)}{P})}}}{D(x)} \nonumber
\end{eqnarray}

\begin{remark}
  Note that the lazier definition still does not deal with summation
  or mixed summation (i.e. sums over input and output). The reader is
  invited to construct definitions of replication that deal with these
  features. 

  Further, the definitions are parameterized in a name, $x$. Can you,
  gentle reader, make a definition that eliminates this parameter and
  guarantees no accidental interaction between the replication
  machinery and the process being replicated -- i.e. no accidental
  sharing of names used by the process to get its work done and the
  name(s) used by the replication to effect copying. This latter
  revision of the definition of replication is crucial to obtaining
  the expected identity $!!P \sim !P$.
\end{remark}

\begin{remark}\label{rem:paradoxical_combinator}
  The reader familiar with the lambda calculus will have noticed the
  similarity between $D$ and the paradoxical combinator.

  [Ed. note: the existence of this seems to suggest we have to be more
  restrictive on the set of processes and names we admit if we are to
  support no-cloning.]
\end{remark}

\subsubsection{Bisimulation}

The computational dynamics gives rise to another kind of equivalence,
the equivalence of computational behavior. As previously mentioned
this is typically captured \emph{via} some form of bisimulation.

% The notion we use in this paper is weak barbed bisimulation
% \cite{milner91polyadicpi}.

The notion we use in this paper is derived from weak barbed
bisimulation \cite{milner91polyadicpi}. 

\begin{definition}
An \emph{observation relation}, $\downarrow_{\mathcal N}$, over a set
of names, $\mathcal N$, is the smallest relation satisfying the rules
below.

\infrule[Out-barb]{y \in {\mathcal N}, \; x \nameeq y}
		  {\outputp{x}{v} \downarrow_{\mathcal N} x}
\infrule[Par-barb]{\mbox{$P\downarrow_{\mathcal N} x$ or $Q\downarrow_{\mathcal N} x$}}
		  {\binpar{P}{Q} \downarrow_{\mathcal N} x}

We write $P \Downarrow_{\mathcal N} x$ if there is $Q$ such that 
$P \wred Q$ and $Q \downarrow_{\mathcal N} x$.
\end{definition}

\begin{definition}
%\label{def.bbisim}
An  ${\mathcal N}$-\emph{barbed bisimulation} over a set of names, ${\mathcal N}$, is a symmetric binary relation 
${\mathcal S}_{\mathcal N}$ between agents such that $P\rel{S}_{\mathcal N}Q$ implies:
\begin{enumerate}
\item If $P \red P'$ then $Q \wred Q'$ and $P'\rel{S}_{\mathcal N} Q'$.
\item If $P\downarrow_{\mathcal N} x$, then $Q\Downarrow_{\mathcal N} x$.
\end{enumerate}
$P$ is ${\mathcal N}$-barbed bisimilar to $Q$, written
$P \wbbisim_{\mathcal N} Q$, if $P \rel{S}_{\mathcal N} Q$ for some ${\mathcal N}$-barbed bisimulation ${\mathcal S}_{\mathcal N}$.
\end{definition}

$\mathcal{R} \subseteq \pi \times \pi$

$P \mathcal{R} Q => \forall P'. P \red P' \Rightarrow \exists Q'. Q \red Q', P' \mathcal{R} Q'$

$P \vdash x \Rightarrow Q \vdash x$

\begin{mathpar}
  \inferrule*[lab=Out-barb]{x \nameeq y}{{y}!\langle{Q}\rangle \vdash x}
  \and
  \inferrule*[lab=Par-barb]{\mbox{$P\vdash x$ or $Q\vdash x$}}{\binpar{P}{Q} \vdash x}
\end{mathpar}

\subsubsection{Contexts}

One of the principle advantages of computational calculi like the
$\pi$-calculus is a well-defined notion of context,
contextual-equivalence and a correlation between
contextual-equivalence and notions of bisimulation. The notion of
context allows the decomposition of a process into (sub-)process and
its syntactic environment, its context. Thus, a context may be
thought of as a process with a ``hole'' (written $\Box$) in it. The
application of a context $M$ to a process $P$, written $M[P]$, is
tantamount to filling the hole in $M$ with $P$. In this paper we do
not need the full weight of this theory, but do make use of the notion
of context in the proof the main theorem. 

\begin{mathpar}
  \inferrule* [lab=summation] {} {{M_{M},M_{N}} \bc \Box \;|\; x.M_{A} \;|\; M_{M}+M_{N}}
  \and
  \inferrule* [lab=agent] {} {{M_{A}} \bc (\vec{x})M_{P} \;| \; \clift{P_0,\ldots,M_{P},\ldots,P_N}}
  \and \\
  \inferrule* [lab=process] {} {{M_{P}} \bc M_{N} \;| \;P|M_{P} }
\end{mathpar} 

\begin{mathpar}
  \inferrule* [lab=sychronization] {} {M_{N} \bc \Box \;|\; x?M_{F} \;|\; x!M_{C}}
  \and
  \inferrule* [lab=abstraction] {} {{M_{F}} \bc (x)M_{P} }
  \and
  \inferrule* [lab=concretion] {} {{M_{C}} \bc \langle M_{P} \rangle }
  \and \\
  \inferrule* [lab=process] {} {{M_{P}} \bc M_{N} \;| \;P|M_{P} }
\end{mathpar}

\begin{definition}[contextual application] Given a context $M$, and
  process $P$, we define the \emph{contextual application}, $M[P] :=
  M\{P/\Box\}$. That is, the contextual application of M to P is the
  substitution of $P$ for $\Box$ in $M$.
\end{definition}

$\meaningof{-} : L \to \mathcal{P}(\pi)$

\begin{mathpar}
  \inferrule* [lab=collection] {} {\meaningof{true} = \pi, \and \meaningof{~E} = \pi \setminus \meaningof{E}, \and \meaningof{E_{1} \& E_{2}} = \meaningof{E_{1}} \cap \meaningof{E_{2}}}
\end{mathpar}

\begin{mathpar}
  \inferrule* [lab=structure] {} {\meaningof{0} = \{ P \in \pi | P \equiv 0 \}, \and \\ \meaningof{E_1 | E_2} = \{ P \in \pi | P \equiv P_{1} | P_{2}, P_{1} \in \meaningof{E_{1}}, P_{2} \in \meaningof{E_2}\} }
\end{mathpar}

\begin{mathpar}
 \inferrule* [lab=behavior] {} {\meaningof{\langle a?b \rangle E} = \{ P \in \pi | P \equiv Q | u?(y)P', \\ \and \\\\ \and \\ \;\;\; u \in \meaningof{a}, \forall z.P'\{z/y\} \in \meaningof{E\{z/b\}}\}, \and \\ \meaningof{a!E} = \{ P \in \pi | P \equiv Q | x!\langle P' \rangle, x \in \meaningof{a} P' \in \meaningof{E}\} }
\end{mathpar}

\begin{mathpar}
 \inferrule* [lab=nominal] {} {\meaningof{\quotep{E}} = \{ \quotep{P} \in \quotep{\pi} | P \in \meaningof{E} \}, \and \meaningof{\quotep{P}} = \{ \quotep{Q} \in \quotep{\pi} | P \equiv Q \} \and \\ \meaningof{@\quotep{E}} = \{ P \in \pi | P \equiv @x, x \in \meaningof{E} \}}
\end{mathpar}

\begin{eqnarray*}
  \\
  \meaningof{-} : TS \to ST
\end{eqnarray*}

\begin{eqnarray*}
  \\
  L : TS \to ST
\end{eqnarray*}

\begin{eqnarray*}
  \\
  P \models E \iff P \in \meaningof{E}
\end{eqnarray*}

\begin{eqnarray*}
  P \approx_{L} Q \iff \forall E \in L. P \models E \iff Q \models E
\end{eqnarray*}

\begin{eqnarray*}
  P \approx_{K} Q
\end{eqnarray*}

\begin{eqnarray*}
  P \approx Q
\end{eqnarray*}

$\approx_{K} = \approx = \approx_{L}$

\subsubsection{Contextual duality}

Note that contexts extend the quotation operation to a family of
operations from processes to names. Given a context, $M$, we can
define a \emph{nominal context}, $\quotep{M}$ by $\quotep{M}[P] :=
\quotep{M[P]}$. To foreshadow what is to come we observe that these
operations enjoy a duality with processes very much like the duality
between vectors and maps from vectors to scalars.

Further, because the calculus is essentially higher-order, we have a
correspondence between contexts and processes. More specifically,
given a name $x$ and a context $M$ we can construct $M^{*}_{x}$ such
that 

\begin{mathpar}
  M^{*}_{x} | \lift{x}{P} \red M[P]
\end{mathpar}

namely,

\begin{mathpar}
  M^{*}_{x} := x?(u).M[\dropn{u}]
\end{mathpar}

The dependence of $M^{*}_{x}$ on a name makes it an abstraction, 

\begin{mathpar}
  M^{*} := (x)x?(u).M[\dropn{u}]
\end{mathpar}

\subsection{Additional notation}

It will sometimes be convenient to denote the process a name
quotes. We already have the notation $x = \quotep{P}$, but it will be
convenient to introduce an alternate notation, $\procn{x}$, when we
want to emphasize the connection to the use of the name. Note that, by
virtue of name equivalence, $\quotep{\procn{x}} \nameeq x$; so, the
notation is consistent with previous definitions.

Further, because names have structure it is possible to effect
substitutions on the basis of that structure. This means we need to
upgrade our notation for substitutions, which we accomplish by
adapting comprehension notation. Thus,

\begin{mathpar}
  P\{ y / x : x \in S \}
\end{mathpar}

is interpreted to mean the process derived from P by replacing (in a
capture-avoiding manner) each occurrence of $x$ in $S$ by $y$. For example,

\begin{mathpar}
  P\{ \quotep{\procn{x}|\procn{x}} / x : x \in \freenames{P} \}
\end{mathpar}

will replace each (occurrence) of a free name $x$ in $P$ by
$\quotep{\procn{x}|\procn{x}}$.

Also, we will avail ourselves of the notation $x^{L}$ and $x^{R}$ to
denote injections of a name into disjoint copies of the name
space. There are numerous ways to accomplish this. One example can be
found in \cite{MeredithR05}. This notation overloads to vectors of
names: $\vec{x}^{\pi} := (x_{i}^{\pi} \; : \; 0 \leq i < |\vec{x}| )$ where $\pi \in \{L,R\}$.

We also use $P^{\Box} := P|\Box$.

In \cite{MeredithR05} an interpretation of the new operator is
given. It turns out that there are several possible interpretations
all enjoying the requisite algebraic properties of the operator (see
\cite{milner91polyadicpi}). We will therefore make liberal use of
$(\nu\; \vec{x})P$.

% subsection the_syntax_and_semantics_of_the_notation_system (end)   

\input{qm2pi.qmops} 

\input{qm2pi.sterngerlach} 

\input{qm2pi.metric} 

% section concurrent_process_calculi (end)

%\input{qm2pi.proofsketch}

% section proof sketch (end)

%\input{qm2pi.slviaknots} 

% section spatial logic via knots (end)

\input{qm2pi.conclusion}

% section conclusion (end)

%\input{qm2pi.dtcodes} 

% section wiring algorithm (end)

\input{qm2pi.ack} 

% section acknowledgments (end)

\newpage


\bibliographystyle{plain}   
\bibliography{../../biblios/main.bib}

\input{qm2pi.rhodetails}

\end{document}



% section front matter (end)

\section{Introduction}\label{sec:introduction} % (fold)
In this draft of the material i am going to have to dispense with the
usual writing conventions adopted in papers on these topics. i'm going
to have adopt whatever tone i need at the time i'm writing up the
calculations. Sometimes this may be very conversational; others it may
be the barest mathematical grunts; others still it may be that i have
lifted text from one of my other papers because the exposition of some
point was better said there. i hope that my readers are not unduly put
out by this decision. i'm not doing this to flout convention or be
rebellious. i find these calculations very technically challenging. To
keep everything going technically, something has to give; i have to
let go of some cognitive burden. So, the academic writing style --
with all of its trade-offs in terms of facilitating technical
communication -- is what i'm letting go of. Perhaps subsequent drafts
can be tightened and polished, but for now, i'm going to speak as if
we were sitting together in a coffee shop with a laptop, wifi and a
pad of paper and a pencil.

So, here's what i have to say. We -- you and i, comfortably ensconced
in our coffee shop and well-equipped with our tools -- can realize and
carry out the calculations of quantum mechanics over a very different
formal theory of dynamics, a formal theory of dynamics that
corresponds to a theory of concurrent computation with
\emph{reflection}. It has the advantage that the underlying theory is
already `quantized', but supports analogues all of the continuuous
operations. Strikingly, this underlying theory has recently been
connected with a notion of metric that we can show, by calculating
together, coincides with the metric induced by the inner product.

There are a lot of reasons why you might be interested in seeing
calculations of this form. Here's why i'm interested. For the past
several centuries there has been no competitor to the ``Newtonian''
account of dynamics. As a result the predominant share of accounts of
dynamical systems and situations have had to be formulated in terms of
the Newtonian machinery. i view this as an intellectually dangerous
position to occupy. Everything, despite it's intrinsic shape, turns
into a nail to be hit with this hammer. Recently, however, the theory
of computation has matured to the point where we have candidates for
theories of dynamics that offer very different perspective on
reasoning about dynamical systems and situations. Testing these
candidates against very successful accounts of dynamical situations,
like quantum mechanics, is going to give us some sense of how mature
they are and some measure of the quality of these accounts of
dynamics.

\subsection{Summary of contributions and outline of paper}

So, we're going to develop an interpretation of the operations of
quantum mechanics normally interpreted by Hilbert spaces and
operators. We're going to do this over a theory of computation. Note
that this is very different than the usual quantum computation program
which develops notions of computation over quantum mechanics. Rather,
we are developing a story that aligns with Wheeler's slogan: It from
Bit. To do this we will first provide an account of the theory of
computation at play here. Then we will dive into a calculation-driven
interpretation of the operations of quantum mechanics.

The reason we take this approach is that -- until very recently --
there hasn't been an axiomatic account of quantum mechanics. As a
result there has been no sharp delineation of the mathematical theory
supporting interpretation of the physical theory and the physical
theory, itself. So, ambient features of the maths are free to be
exploited (or supressed) without a real accounting of their physical
relevance. There is no sharp statement ``here's the physical theory''
qua \emph{theory} and ``here's the mathematical interpretation''
enabling a judgment of how faithful the interpretation is -- apart
from experimental observation. When there is an axiomatic account we
can judge how well a given mathematical formalism supports an
interpretation of the axioms, independent of
experimentation. Likewise, we can judge how well we have captured our
physical evidence and experience with our axiomatics, independent of
any specific mathematical implementation, with accidental detail that
may or may not have physical significance. 

In lieu of a fully fleshed out and vetted axiomatic account of quantum
mechanics, interpreting the operational notions in service of modeling
physical systems will have to suffice. In other words, we are not in
the business of providing a model of Hilbert spaces and operators. We
are in the business of providing a model of quantum mechanics because
we are motivated by testing our notions of dynamics against physical
theory; and, the predictive calculations of the physical theory must
serve as the best formulation -- shy of a fully fleshed out axiomatic
account -- of the physical theory itself (as they have for scientific
theories since time immemorial). Put another way, despite a
whole-hearted commitment to an It-from-Bit ontology, we are firmly
aligned with the shut-up-and-calculate camp as the best way to obtain
results either from the physical perspective or as a quality assurance
measure of our fledgling theory of dynamics.

In detail, we present a reflective process calculus. Then we develop
intuitive correspondences between the notions available in this
calculus and the usual physical notions supporting quantum mechanical
calculations. Thus, 

\begin{table}[htp]
  \center{
    \fbox{
      \begin{tabular}{c|c}
        quantum mechanics & process calculus \\
        \hline
        scalar & name \\
        state vector & process \\
        dual & contextual duals \\
        matrix & formal sums of process-context-dual pairs \\
        orthogonality & process annihilation \\
        inner product & execution-formula + quoting
      \end{tabular}
    }
  }
  \caption{QM - process calculi correspondences}
\end{table}

Then we tighten up these intuitions to operational definitions. We
employ the Dirac notation as the best proxy we can find for an
abstract syntax of the quantum mechanical notions. The definitions we
develop put us in contact with equational constraints coming from the
theory that we demonstrate the definitions and calculations satisfy.

This puts us in a position to shut up and calculate for the
Stern-Gerlach experimental set up, showing how these predictive
calculations become calculations on processes in our theory of a
reflective process calculus.

Penultimately, we demonstrate that the notion of metric coming from
the inner product coincides with the notion of metric available from
the theory of bisimulation. This demonstration gives us the right to
think of space as arising from behavior. Finally, we consider where we
might go from the new vantage point we have obtained.

% section introduction (end) 
 
% section introduction (end)

% \documentclass[12pt]{llncs}
%\documentclass{jktr}

\usepackage[pdftex]{hyperref}                   
\usepackage {listings}
\usepackage {mathpartir}
\usepackage{bcprules}
%\usepackage{listings}
                       
\usepackage{graphicx} 
%\usepackage[margins=2.5cm,nohead,nofoot]{geometry}
%\usepackage{geometry}
\usepackage{amsfonts}
\usepackage{amstext}
\usepackage{latexsym}
\usepackage{amssymb}
\usepackage{color}


%\include{myPreamble}
\include{qm2pi.local} 

%\ifpdf
%\usepackage[pdftex]{graphicx}
%\else
%\usepackage{graphicx}
%\fi

 % \ifpdf
%  \usepackage{pdfsync}
%  \if


%\title{Brief Article}
%\author{David F. Snyder}
%\author{L.G. Meredith}

%\address{Dept. of Math., Texas State University--San Marcos, San Marcos, TX 78666}
       
\pagestyle{empty}


\begin{document}

\lstset{language=[Objective]Caml,frame=shadowbox}

\input{qm2pi.front}

% section front matter (end)

\input{qm2pi.intro} 
 
% section introduction (end)

% \input{qm2pi.knotations} 

% section notation (end)

\input{qm2pi.process.calculi} 

% section concurrent_process_calculi_and_spatial_logics_ (end)
    
%\input{qm2pi.knots2pi} 

%\input{qm2pi.trefoil} 

%\input{qm2pi.mainthm} 

% subsection basic_interpretation (end)

%\input{qm2pi.rho.presentation} 
\subsection{The syntax and semantics of the notation system}\label{sub:the_syntax_and_semantics_of_the_notation_system} % (fold)

We now summarize a technical presentation of the calculus that
embodies our theory of dynamics. The typical presentation of such a
calculus follows the style of giving generators and relations on
them. The grammar, below, describing term constructors, freely
generates the set of processes, $\Proc$. This set is then quotiented
by a relation known as structural congruence and it is over this set
that the notion of dynamics is expressed. This presentation is
essentially that of \cite{MeredithR05} with the addition of
polyadicity and summation. For readability we have relegated some of
the technical subtleties to an appendix.

\subsubsection{Process grammar}\label{subsub:process_grammar}

\begin{mathpar}
  \inferrule* [lab=synchronization] {} {{M} \bc \pzero \;|\; x?F \;|\; x!C }
  \and
  \inferrule* [lab=abstraction] {} {{F} \bc (x)P}
  \and
  \inferrule* [lab=concretion] {} {{C} \bc \langle Q \rangle}
  \and
  \inferrule* [lab=process] {} {{P,Q} \bc M \;| \;P|Q \;|\; @{x}}
  \and
  \inferrule* [lab=name] {} {{x} \bc \quotep{P}}
\end{mathpar} 

Note that $\vec{x}$ (resp. $\vec{P}$) denotes a vector of names
(resp. processes) of length $|\vec{x}|$ (resp. $|\vec{P}|$). We adopt
the following useful abbreviations.

\begin{mathpar}
   x?(\vec{y}).P := x.(\vec{y})P \and  x\clift{\vec{P}} := x.\clift{\vec{P}}
   \and x!(y) := \lift{x}{\dropn{y}}
   \and \Pi_{i=0}^{n-1}P_i := P_0 | \ldots | P_{n-1}
\end{mathpar}

\subsubsection{Structural congruence}

\paragraph{Free and bound names and alpha-equivalence.} At the
core of structural equivalence is alpha-equivalence which identifies
process that are the same up to a change of variable. Formally, we
recognize the distinction between free and bound names. The free names
of a process, $\freenames{P}$, may be calculated recursively as
follows:

\begin{mathpar}
\freenames{\pzero} := \emptyset
  \and \\
  \freenames{x?(y).P} := \{ x \} \cup (\freenames{P} \setminus \{ y \})
  \and 
  \freenames{x!\langle P \rangle} := \{ x \} \cup \{ P \} 
  \and \\
  \freenames{P|Q} := \freenames{P} \cup \freenames{Q}
  \and \\
  \freenames{@{x}} := \{ x \}
\end{mathpar}

$\pi$
$\quotep{\pi}$

$\freenames{-} : \pi \to \mathcal{P}(\quotep{\pi})$

\begin{eqnarray*}
  \freenames{\pzero} & := & \emptyset \\
  \freenames{x?(y).P} & := & \{ x \} \cup (\freenames{P} \setminus \{ y \}) \\
  \freenames{x!\langle P \rangle} & := & \{ x \} \cup \{ P \} \\
  \freenames{P|Q} & := & \freenames{P} \cup \freenames{Q} \\
  \freenames{\dropn{x}} & := & \{ x \}
\end{eqnarray*}

The bound names of a process, $\boundnames{P}$, are those names occurring in $P$
that are not free. For example, in $x?(y).0$, the name $x$ is free, while $y$ is bound.

\begin{mathpar}
  \inferrule* [lab=monoidal-laws] {} { P|Q \equiv Q|P \and P|0 \equiv P \and P|(Q|R) \equiv (P|Q)|R }
\end{mathpar}

\begin{mathpar}
  \inferrule* [lab=alpha-equivalence] {} { (x)P \equiv (y)P\{y/x\} \and y \not\in \freenames{P} }
\end{mathpar}

\begin{definition}
Then two processes, $P,Q$, are alpha-equivalent if $P = Q\{\vec{y}/\vec{x}\}$ for
some $\vec{x} \in \boundnames{Q},\vec{y} \in \boundnames{P}$, where $Q\{\vec{y}/\vec{x}\}$
denotes the capture-avoiding substitution of $\vec{y}$ for $\vec{x}$ in $Q$.
\end{definition}

\begin{definition}
  The {\em structural congruence} \cite{SangiorgiWalker} , $\equiv$,
  between processes is the least congruence containing
  alpha-equivalence, satisfying the abelian monoid laws
  (associativity, commutativity and $\pzero$ as identity) for parallel
  composition $|$ and for summation $+$.
\end{definition}

\subsection{Name equivalence}

We take name equivalence, written $\nameeq$, to be the smallest
equivalence relation generated by the following rules.

\begin{mathpar}
\inferrule*[lab=Quote-drop]
{ }
{ \quotep{@{x}} \nameeq x }

\inferrule*[lab=Struct-equiv]
{ P \scong Q }
{ \quotep{P} \nameeq \quotep{Q} }
\end{mathpar}

The astute reader will have noticed that the mutual recursion of names
and processes imposes a mutual recursion on alpha-equivalence and
structural equivalence via name-equivalence. Fortunately, all of this
works out pleasantly and we may calculate in the natural way, free of
concern. The reader interested in the details is referred to the
appendix \ref{appendix:rho_details}.

\subsection{Substitution}

We use $\Proc$ for the set of processes, $\QProc$ for the set of
names, and $\id{\{}\vec{y} / \vec{x} \id{\}}$ to denote partial maps,
$s : \QProc \rightarrow \QProc$. A map, $s$ lifts, uniquely, to a map
on process terms, $\widehat{s} : \Proc \rightarrow \Proc$ by the
following equations.

\begin{mathpar}
  (0) \psubstp{Q}{P} := 0 \\
  (R \juxtap S) \psubstp{Q}{P}
  :=    
  (R)\psubstp{Q}{P} \juxtap (S) \psubstp{Q}{P} \\
  (x?(y).R) \psubstp{Q}{P}    
  :=    
  (x)\substp{Q}{P} (z)\concat( (R \psubstn{z}{y}) \psubstp{Q}{P} ) \\
  (\lift{x}{R}) \psubstp{Q}{P}  
  :=
  \lift{(x)\substp{Q}{P}}{ R \psubstp{Q}{P} } \\
%   (\dropn{x})  \psubstp{Q}{P}       
%   := 
%   \left\{ 
%     \begin{array}{ccc} 
%       \dropn{\quotep{Q}} & & x \nameeq \quotep{P} \\
%       \dropn{x} & & otherwise \\
%     \end{array}
%   \right. 
  (\dropn{x})  \psubstp{Q}{P}       
  := 
  \left\{ 
    \begin{array}{ccc} 
      Q & & x \nameeq \quotep{P} \\
      \dropn{x} & & otherwise \\
    \end{array}
  \right.
\end{mathpar}
 

where

\begin{eqnarray}
  (x)\id{\{} \lpquote Q \rpquote / \lpquote P \rpquote \id{\}}            = 
  \left\{ 
    \begin{array}{ccc}
      \lpquote Q \rpquote & & x \nameeq \lpquote P \rpquote \\
      x & & otherwise \\
    \end{array}
  \right. \nonumber
\end{eqnarray}

and $z$ is chosen distinct from $\quotep{P}$, $\quotep{Q}$, the free
names in $Q$, and all the names in $R$. Our $\alpha$-equivalence will
be built in the standard way from this substitution.

\begin{remark}\label{rem:no_self_referential_names}
  One consequence of these definitions is that $\forall P. \quotep{P}
  \not\in \freenames{P}$.
\end{remark}

\subsection{ Dynamic quote: an example }

Anticipating something of what's to come, consider applying the
substitution, $\widehat{\id{\{}u / z \id{\}}}$, to the following pair
of processes, $\lift{w}{y!(z)}$ and $w[ \lpquote y!(z) \rpquote ]$.

\begin{eqnarray}
	\lift{w}{y!(z)}\widehat{\id{\{}u / z \id{\}}}
		& = &
		\lift{w}{y!(u)} \nonumber\\
	w[ \lpquote y!(z) \rpquote ] \widehat{ \id{\{}u / z \id{\}} }
		& = &
		w[ \lpquote y!(z) \rpquote ] \nonumber
\end{eqnarray}

Because the body of the process between quotes is impervious to
substitution, we get radically different answers. In fact, by
examining the first process in an input context,
e.g. $x?(z).\lift{w}{y!(z)}$, we see that the process under the lift
operator may be shaped by prefixed inputs binding a name inside it. In
this sense, the lift operator will be seen as a way to dynamically
construct processes before reifying them as names.

Finally equipped with these standard features we can present the
dynamics of the calculus.

\subsubsection{Operational semantics} 

Finally, we introduce the computational dynamics. What marks these
algebras as distinct from other more traditionally studied algebraic
structures, e.g. vector spaces or polynomial rings, is the manner in
which dynamics is captured. In traditional structures, dynamics is typically
expressed through morphisms between such structures, as in linear maps
between vector spaces or morphisms between rings. In algebras
associated with the semantics of computation, the dynamics is
expressed as part of the algebraic structure itself, through a
reduction reduction relation typically denoted by $\red$. Below, we
give a recursive presentation of this relation for the calculus used
in the encoding.

$\red \subseteq \pi \times \pi$
$\red : \pi \to \mathcal{P}(\pi)$

\begin{mathpar}
  \inferrule* [lab=Comm] { \textsf{match}( x_{src}, x_{trgt} ) } { x_{trgt}?(y)P \; | \; x_{src}!\langle {Q} \rangle \red P\{\quotep{Q}/y}\} }
  \and \\
  \inferrule* [lab=Par] {{P} \red {P}'} {{{P} | {Q}} \red {{P}' | {Q}}}
  \and
  \inferrule* [lab=Equiv]{{{P} \scong {P}'} \andalso {{P}' \red {Q}'} \andalso {{Q}' \scong {Q}}}{{P} \red {Q}}
\end{mathpar}

\begin{eqnarray*}
  match_{\equiv} (\quotep{P},\quotep{Q}) & := & P \equiv Q \\
  match_{\dagger}(\quotep{P},\quotep{Q}) & := & \forall R. P|Q \red^{*} R => R \red^{*} 0 \\
  match_{K}(\quotep{P},\quotep{Q}) & := & K \mbox{ for some context } K
\end{eqnarray*}

$u?(x)P | u!\langle Q \rangle \red P\{\quotep{Q}/x\}$

%We write $\wred$ for $\red^*$, and $P\red$ if $\exists Q $ such that $ P \red Q$.
We write $P\red$ if $\exists Q $ such that $ P \red Q$ and $P\not\red$, otherwise.

\section{Replication}

As mentioned before, it is known that replication (and hence
recursion) can be implemented in a higher-order process algebra
\cite{SangiorgiWalker}. As our first example of calculation with the
machinery thus far presented we give the construction explicitly in
the {\rhoc}.

\begin{eqnarray}
	D_{x} & := & \prefix{x}{y}{(\binpar{\outputp{x}{y}}{@{y}})} \nonumber\\
	\bangp_{x}{P} & := & \binpar{{x}!\langle{\binpar{D_{x}}{P}}\rangle}{D_{x}} \nonumber
\end{eqnarray}

\begin{eqnarray}
	\bangp_{x}{P} & & \nonumber\\
	=
	& {x}!\langle{(\prefix{x}{y}{(\outputp{x}{y} | @{y})) | P}}\rangle 
	      | \prefix{x}{y}{(\outputp{x}{y} | @{y})} & \nonumber\\
	\red
	& (\outputp{x}{y} | @{y})\substn{\quotep{(\prefix{x}{y}{(@{y} | \outputp{x}{y})) | P}}}{y} & \nonumber\\
	=
	& \outputp{x}{\quotep{(\prefix{x}{y}{(\outputp{x}{y} | @{y})) | P}}}
	  | {(\prefix{x}{y}{(\outputp{x}{y} | @{y})) | P}} & \nonumber\\
	\red
	& \ldots & \nonumber\\
	\red^*
	& P | P | \ldots & \nonumber
\end{eqnarray}

Of course, this encoding, as an implementation, runs away, unfolding
$\bangp{P}$ eagerly. A lazier and more implementable replication
operator, restricted to input-guarded processes, may be obtained as follows.

\begin{eqnarray}
\bangp{\prefix{u}{v}{P}} 
	:= 
	\binpar{\lift{x}{\prefix{u}{v}{(\binpar{D(x)}{P})}}}{D(x)} \nonumber
\end{eqnarray}

\begin{remark}
  Note that the lazier definition still does not deal with summation
  or mixed summation (i.e. sums over input and output). The reader is
  invited to construct definitions of replication that deal with these
  features. 

  Further, the definitions are parameterized in a name, $x$. Can you,
  gentle reader, make a definition that eliminates this parameter and
  guarantees no accidental interaction between the replication
  machinery and the process being replicated -- i.e. no accidental
  sharing of names used by the process to get its work done and the
  name(s) used by the replication to effect copying. This latter
  revision of the definition of replication is crucial to obtaining
  the expected identity $!!P \sim !P$.
\end{remark}

\begin{remark}\label{rem:paradoxical_combinator}
  The reader familiar with the lambda calculus will have noticed the
  similarity between $D$ and the paradoxical combinator.

  [Ed. note: the existence of this seems to suggest we have to be more
  restrictive on the set of processes and names we admit if we are to
  support no-cloning.]
\end{remark}

\subsubsection{Bisimulation}

The computational dynamics gives rise to another kind of equivalence,
the equivalence of computational behavior. As previously mentioned
this is typically captured \emph{via} some form of bisimulation.

% The notion we use in this paper is weak barbed bisimulation
% \cite{milner91polyadicpi}.

The notion we use in this paper is derived from weak barbed
bisimulation \cite{milner91polyadicpi}. 

\begin{definition}
An \emph{observation relation}, $\downarrow_{\mathcal N}$, over a set
of names, $\mathcal N$, is the smallest relation satisfying the rules
below.

\infrule[Out-barb]{y \in {\mathcal N}, \; x \nameeq y}
		  {\outputp{x}{v} \downarrow_{\mathcal N} x}
\infrule[Par-barb]{\mbox{$P\downarrow_{\mathcal N} x$ or $Q\downarrow_{\mathcal N} x$}}
		  {\binpar{P}{Q} \downarrow_{\mathcal N} x}

We write $P \Downarrow_{\mathcal N} x$ if there is $Q$ such that 
$P \wred Q$ and $Q \downarrow_{\mathcal N} x$.
\end{definition}

\begin{definition}
%\label{def.bbisim}
An  ${\mathcal N}$-\emph{barbed bisimulation} over a set of names, ${\mathcal N}$, is a symmetric binary relation 
${\mathcal S}_{\mathcal N}$ between agents such that $P\rel{S}_{\mathcal N}Q$ implies:
\begin{enumerate}
\item If $P \red P'$ then $Q \wred Q'$ and $P'\rel{S}_{\mathcal N} Q'$.
\item If $P\downarrow_{\mathcal N} x$, then $Q\Downarrow_{\mathcal N} x$.
\end{enumerate}
$P$ is ${\mathcal N}$-barbed bisimilar to $Q$, written
$P \wbbisim_{\mathcal N} Q$, if $P \rel{S}_{\mathcal N} Q$ for some ${\mathcal N}$-barbed bisimulation ${\mathcal S}_{\mathcal N}$.
\end{definition}

$\mathcal{R} \subseteq \pi \times \pi$

$P \mathcal{R} Q => \forall P'. P \red P' \Rightarrow \exists Q'. Q \red Q', P' \mathcal{R} Q'$

$P \vdash x \Rightarrow Q \vdash x$

\begin{mathpar}
  \inferrule*[lab=Out-barb]{x \nameeq y}{{y}!\langle{Q}\rangle \vdash x}
  \and
  \inferrule*[lab=Par-barb]{\mbox{$P\vdash x$ or $Q\vdash x$}}{\binpar{P}{Q} \vdash x}
\end{mathpar}

\subsubsection{Contexts}

One of the principle advantages of computational calculi like the
$\pi$-calculus is a well-defined notion of context,
contextual-equivalence and a correlation between
contextual-equivalence and notions of bisimulation. The notion of
context allows the decomposition of a process into (sub-)process and
its syntactic environment, its context. Thus, a context may be
thought of as a process with a ``hole'' (written $\Box$) in it. The
application of a context $M$ to a process $P$, written $M[P]$, is
tantamount to filling the hole in $M$ with $P$. In this paper we do
not need the full weight of this theory, but do make use of the notion
of context in the proof the main theorem. 

\begin{mathpar}
  \inferrule* [lab=summation] {} {{M_{M},M_{N}} \bc \Box \;|\; x.M_{A} \;|\; M_{M}+M_{N}}
  \and
  \inferrule* [lab=agent] {} {{M_{A}} \bc (\vec{x})M_{P} \;| \; \clift{P_0,\ldots,M_{P},\ldots,P_N}}
  \and \\
  \inferrule* [lab=process] {} {{M_{P}} \bc M_{N} \;| \;P|M_{P} }
\end{mathpar} 

\begin{mathpar}
  \inferrule* [lab=sychronization] {} {M_{N} \bc \Box \;|\; x?M_{F} \;|\; x!M_{C}}
  \and
  \inferrule* [lab=abstraction] {} {{M_{F}} \bc (x)M_{P} }
  \and
  \inferrule* [lab=concretion] {} {{M_{C}} \bc \langle M_{P} \rangle }
  \and \\
  \inferrule* [lab=process] {} {{M_{P}} \bc M_{N} \;| \;P|M_{P} }
\end{mathpar}

\begin{definition}[contextual application] Given a context $M$, and
  process $P$, we define the \emph{contextual application}, $M[P] :=
  M\{P/\Box\}$. That is, the contextual application of M to P is the
  substitution of $P$ for $\Box$ in $M$.
\end{definition}

$\meaningof{-} : L \to \mathcal{P}(\pi)$

\begin{mathpar}
  \inferrule* [lab=collection] {} {\meaningof{true} = \pi, \and \meaningof{~E} = \pi \setminus \meaningof{E}, \and \meaningof{E_{1} \& E_{2}} = \meaningof{E_{1}} \cap \meaningof{E_{2}}}
\end{mathpar}

\begin{mathpar}
  \inferrule* [lab=structure] {} {\meaningof{0} = \{ P \in \pi | P \equiv 0 \}, \and \\ \meaningof{E_1 | E_2} = \{ P \in \pi | P \equiv P_{1} | P_{2}, P_{1} \in \meaningof{E_{1}}, P_{2} \in \meaningof{E_2}\} }
\end{mathpar}

\begin{mathpar}
 \inferrule* [lab=behavior] {} {\meaningof{\langle a?b \rangle E} = \{ P \in \pi | P \equiv Q | u?(y)P', \\ \and \\\\ \and \\ \;\;\; u \in \meaningof{a}, \forall z.P'\{z/y\} \in \meaningof{E\{z/b\}}\}, \and \\ \meaningof{a!E} = \{ P \in \pi | P \equiv Q | x!\langle P' \rangle, x \in \meaningof{a} P' \in \meaningof{E}\} }
\end{mathpar}

\begin{mathpar}
 \inferrule* [lab=nominal] {} {\meaningof{\quotep{E}} = \{ \quotep{P} \in \quotep{\pi} | P \in \meaningof{E} \}, \and \meaningof{\quotep{P}} = \{ \quotep{Q} \in \quotep{\pi} | P \equiv Q \} \and \\ \meaningof{@\quotep{E}} = \{ P \in \pi | P \equiv @x, x \in \meaningof{E} \}}
\end{mathpar}

\begin{eqnarray*}
  \\
  \meaningof{-} : TS \to ST
\end{eqnarray*}

\begin{eqnarray*}
  \\
  L : TS \to ST
\end{eqnarray*}

\begin{eqnarray*}
  \\
  P \models E \iff P \in \meaningof{E}
\end{eqnarray*}

\begin{eqnarray*}
  P \approx_{L} Q \iff \forall E \in L. P \models E \iff Q \models E
\end{eqnarray*}

\begin{eqnarray*}
  P \approx_{K} Q
\end{eqnarray*}

\begin{eqnarray*}
  P \approx Q
\end{eqnarray*}

$\approx_{K} = \approx = \approx_{L}$

\subsubsection{Contextual duality}

Note that contexts extend the quotation operation to a family of
operations from processes to names. Given a context, $M$, we can
define a \emph{nominal context}, $\quotep{M}$ by $\quotep{M}[P] :=
\quotep{M[P]}$. To foreshadow what is to come we observe that these
operations enjoy a duality with processes very much like the duality
between vectors and maps from vectors to scalars.

Further, because the calculus is essentially higher-order, we have a
correspondence between contexts and processes. More specifically,
given a name $x$ and a context $M$ we can construct $M^{*}_{x}$ such
that 

\begin{mathpar}
  M^{*}_{x} | \lift{x}{P} \red M[P]
\end{mathpar}

namely,

\begin{mathpar}
  M^{*}_{x} := x?(u).M[\dropn{u}]
\end{mathpar}

The dependence of $M^{*}_{x}$ on a name makes it an abstraction, 

\begin{mathpar}
  M^{*} := (x)x?(u).M[\dropn{u}]
\end{mathpar}

\subsection{Additional notation}

It will sometimes be convenient to denote the process a name
quotes. We already have the notation $x = \quotep{P}$, but it will be
convenient to introduce an alternate notation, $\procn{x}$, when we
want to emphasize the connection to the use of the name. Note that, by
virtue of name equivalence, $\quotep{\procn{x}} \nameeq x$; so, the
notation is consistent with previous definitions.

Further, because names have structure it is possible to effect
substitutions on the basis of that structure. This means we need to
upgrade our notation for substitutions, which we accomplish by
adapting comprehension notation. Thus,

\begin{mathpar}
  P\{ y / x : x \in S \}
\end{mathpar}

is interpreted to mean the process derived from P by replacing (in a
capture-avoiding manner) each occurrence of $x$ in $S$ by $y$. For example,

\begin{mathpar}
  P\{ \quotep{\procn{x}|\procn{x}} / x : x \in \freenames{P} \}
\end{mathpar}

will replace each (occurrence) of a free name $x$ in $P$ by
$\quotep{\procn{x}|\procn{x}}$.

Also, we will avail ourselves of the notation $x^{L}$ and $x^{R}$ to
denote injections of a name into disjoint copies of the name
space. There are numerous ways to accomplish this. One example can be
found in \cite{MeredithR05}. This notation overloads to vectors of
names: $\vec{x}^{\pi} := (x_{i}^{\pi} \; : \; 0 \leq i < |\vec{x}| )$ where $\pi \in \{L,R\}$.

We also use $P^{\Box} := P|\Box$.

In \cite{MeredithR05} an interpretation of the new operator is
given. It turns out that there are several possible interpretations
all enjoying the requisite algebraic properties of the operator (see
\cite{milner91polyadicpi}). We will therefore make liberal use of
$(\nu\; \vec{x})P$.

% subsection the_syntax_and_semantics_of_the_notation_system (end)   

\input{qm2pi.qmops} 

\input{qm2pi.sterngerlach} 

\input{qm2pi.metric} 

% section concurrent_process_calculi (end)

%\input{qm2pi.proofsketch}

% section proof sketch (end)

%\input{qm2pi.slviaknots} 

% section spatial logic via knots (end)

\input{qm2pi.conclusion}

% section conclusion (end)

%\input{qm2pi.dtcodes} 

% section wiring algorithm (end)

\input{qm2pi.ack} 

% section acknowledgments (end)

\newpage


\bibliographystyle{plain}   
\bibliography{../../biblios/main.bib}

\input{qm2pi.rhodetails}

\end{document}

 

% section notation (end)

\input{qm2pi.process.calculi} 

% section concurrent_process_calculi_and_spatial_logics_ (end)
    
%\documentclass[12pt]{llncs}
%\documentclass{jktr}

\usepackage[pdftex]{hyperref}                   
\usepackage {listings}
\usepackage {mathpartir}
\usepackage{bcprules}
%\usepackage{listings}
                       
\usepackage{graphicx} 
%\usepackage[margins=2.5cm,nohead,nofoot]{geometry}
%\usepackage{geometry}
\usepackage{amsfonts}
\usepackage{amstext}
\usepackage{latexsym}
\usepackage{amssymb}
\usepackage{color}


%\include{myPreamble}
\include{qm2pi.local} 

%\ifpdf
%\usepackage[pdftex]{graphicx}
%\else
%\usepackage{graphicx}
%\fi

 % \ifpdf
%  \usepackage{pdfsync}
%  \if


%\title{Brief Article}
%\author{David F. Snyder}
%\author{L.G. Meredith}

%\address{Dept. of Math., Texas State University--San Marcos, San Marcos, TX 78666}
       
\pagestyle{empty}


\begin{document}

\lstset{language=[Objective]Caml,frame=shadowbox}

\input{qm2pi.front}

% section front matter (end)

\input{qm2pi.intro} 
 
% section introduction (end)

% \input{qm2pi.knotations} 

% section notation (end)

\input{qm2pi.process.calculi} 

% section concurrent_process_calculi_and_spatial_logics_ (end)
    
%\input{qm2pi.knots2pi} 

%\input{qm2pi.trefoil} 

%\input{qm2pi.mainthm} 

% subsection basic_interpretation (end)

%\input{qm2pi.rho.presentation} 
\subsection{The syntax and semantics of the notation system}\label{sub:the_syntax_and_semantics_of_the_notation_system} % (fold)

We now summarize a technical presentation of the calculus that
embodies our theory of dynamics. The typical presentation of such a
calculus follows the style of giving generators and relations on
them. The grammar, below, describing term constructors, freely
generates the set of processes, $\Proc$. This set is then quotiented
by a relation known as structural congruence and it is over this set
that the notion of dynamics is expressed. This presentation is
essentially that of \cite{MeredithR05} with the addition of
polyadicity and summation. For readability we have relegated some of
the technical subtleties to an appendix.

\subsubsection{Process grammar}\label{subsub:process_grammar}

\begin{mathpar}
  \inferrule* [lab=synchronization] {} {{M} \bc \pzero \;|\; x?F \;|\; x!C }
  \and
  \inferrule* [lab=abstraction] {} {{F} \bc (x)P}
  \and
  \inferrule* [lab=concretion] {} {{C} \bc \langle Q \rangle}
  \and
  \inferrule* [lab=process] {} {{P,Q} \bc M \;| \;P|Q \;|\; @{x}}
  \and
  \inferrule* [lab=name] {} {{x} \bc \quotep{P}}
\end{mathpar} 

Note that $\vec{x}$ (resp. $\vec{P}$) denotes a vector of names
(resp. processes) of length $|\vec{x}|$ (resp. $|\vec{P}|$). We adopt
the following useful abbreviations.

\begin{mathpar}
   x?(\vec{y}).P := x.(\vec{y})P \and  x\clift{\vec{P}} := x.\clift{\vec{P}}
   \and x!(y) := \lift{x}{\dropn{y}}
   \and \Pi_{i=0}^{n-1}P_i := P_0 | \ldots | P_{n-1}
\end{mathpar}

\subsubsection{Structural congruence}

\paragraph{Free and bound names and alpha-equivalence.} At the
core of structural equivalence is alpha-equivalence which identifies
process that are the same up to a change of variable. Formally, we
recognize the distinction between free and bound names. The free names
of a process, $\freenames{P}$, may be calculated recursively as
follows:

\begin{mathpar}
\freenames{\pzero} := \emptyset
  \and \\
  \freenames{x?(y).P} := \{ x \} \cup (\freenames{P} \setminus \{ y \})
  \and 
  \freenames{x!\langle P \rangle} := \{ x \} \cup \{ P \} 
  \and \\
  \freenames{P|Q} := \freenames{P} \cup \freenames{Q}
  \and \\
  \freenames{@{x}} := \{ x \}
\end{mathpar}

$\pi$
$\quotep{\pi}$

$\freenames{-} : \pi \to \mathcal{P}(\quotep{\pi})$

\begin{eqnarray*}
  \freenames{\pzero} & := & \emptyset \\
  \freenames{x?(y).P} & := & \{ x \} \cup (\freenames{P} \setminus \{ y \}) \\
  \freenames{x!\langle P \rangle} & := & \{ x \} \cup \{ P \} \\
  \freenames{P|Q} & := & \freenames{P} \cup \freenames{Q} \\
  \freenames{\dropn{x}} & := & \{ x \}
\end{eqnarray*}

The bound names of a process, $\boundnames{P}$, are those names occurring in $P$
that are not free. For example, in $x?(y).0$, the name $x$ is free, while $y$ is bound.

\begin{mathpar}
  \inferrule* [lab=monoidal-laws] {} { P|Q \equiv Q|P \and P|0 \equiv P \and P|(Q|R) \equiv (P|Q)|R }
\end{mathpar}

\begin{mathpar}
  \inferrule* [lab=alpha-equivalence] {} { (x)P \equiv (y)P\{y/x\} \and y \not\in \freenames{P} }
\end{mathpar}

\begin{definition}
Then two processes, $P,Q$, are alpha-equivalent if $P = Q\{\vec{y}/\vec{x}\}$ for
some $\vec{x} \in \boundnames{Q},\vec{y} \in \boundnames{P}$, where $Q\{\vec{y}/\vec{x}\}$
denotes the capture-avoiding substitution of $\vec{y}$ for $\vec{x}$ in $Q$.
\end{definition}

\begin{definition}
  The {\em structural congruence} \cite{SangiorgiWalker} , $\equiv$,
  between processes is the least congruence containing
  alpha-equivalence, satisfying the abelian monoid laws
  (associativity, commutativity and $\pzero$ as identity) for parallel
  composition $|$ and for summation $+$.
\end{definition}

\subsection{Name equivalence}

We take name equivalence, written $\nameeq$, to be the smallest
equivalence relation generated by the following rules.

\begin{mathpar}
\inferrule*[lab=Quote-drop]
{ }
{ \quotep{@{x}} \nameeq x }

\inferrule*[lab=Struct-equiv]
{ P \scong Q }
{ \quotep{P} \nameeq \quotep{Q} }
\end{mathpar}

The astute reader will have noticed that the mutual recursion of names
and processes imposes a mutual recursion on alpha-equivalence and
structural equivalence via name-equivalence. Fortunately, all of this
works out pleasantly and we may calculate in the natural way, free of
concern. The reader interested in the details is referred to the
appendix \ref{appendix:rho_details}.

\subsection{Substitution}

We use $\Proc$ for the set of processes, $\QProc$ for the set of
names, and $\id{\{}\vec{y} / \vec{x} \id{\}}$ to denote partial maps,
$s : \QProc \rightarrow \QProc$. A map, $s$ lifts, uniquely, to a map
on process terms, $\widehat{s} : \Proc \rightarrow \Proc$ by the
following equations.

\begin{mathpar}
  (0) \psubstp{Q}{P} := 0 \\
  (R \juxtap S) \psubstp{Q}{P}
  :=    
  (R)\psubstp{Q}{P} \juxtap (S) \psubstp{Q}{P} \\
  (x?(y).R) \psubstp{Q}{P}    
  :=    
  (x)\substp{Q}{P} (z)\concat( (R \psubstn{z}{y}) \psubstp{Q}{P} ) \\
  (\lift{x}{R}) \psubstp{Q}{P}  
  :=
  \lift{(x)\substp{Q}{P}}{ R \psubstp{Q}{P} } \\
%   (\dropn{x})  \psubstp{Q}{P}       
%   := 
%   \left\{ 
%     \begin{array}{ccc} 
%       \dropn{\quotep{Q}} & & x \nameeq \quotep{P} \\
%       \dropn{x} & & otherwise \\
%     \end{array}
%   \right. 
  (\dropn{x})  \psubstp{Q}{P}       
  := 
  \left\{ 
    \begin{array}{ccc} 
      Q & & x \nameeq \quotep{P} \\
      \dropn{x} & & otherwise \\
    \end{array}
  \right.
\end{mathpar}
 

where

\begin{eqnarray}
  (x)\id{\{} \lpquote Q \rpquote / \lpquote P \rpquote \id{\}}            = 
  \left\{ 
    \begin{array}{ccc}
      \lpquote Q \rpquote & & x \nameeq \lpquote P \rpquote \\
      x & & otherwise \\
    \end{array}
  \right. \nonumber
\end{eqnarray}

and $z$ is chosen distinct from $\quotep{P}$, $\quotep{Q}$, the free
names in $Q$, and all the names in $R$. Our $\alpha$-equivalence will
be built in the standard way from this substitution.

\begin{remark}\label{rem:no_self_referential_names}
  One consequence of these definitions is that $\forall P. \quotep{P}
  \not\in \freenames{P}$.
\end{remark}

\subsection{ Dynamic quote: an example }

Anticipating something of what's to come, consider applying the
substitution, $\widehat{\id{\{}u / z \id{\}}}$, to the following pair
of processes, $\lift{w}{y!(z)}$ and $w[ \lpquote y!(z) \rpquote ]$.

\begin{eqnarray}
	\lift{w}{y!(z)}\widehat{\id{\{}u / z \id{\}}}
		& = &
		\lift{w}{y!(u)} \nonumber\\
	w[ \lpquote y!(z) \rpquote ] \widehat{ \id{\{}u / z \id{\}} }
		& = &
		w[ \lpquote y!(z) \rpquote ] \nonumber
\end{eqnarray}

Because the body of the process between quotes is impervious to
substitution, we get radically different answers. In fact, by
examining the first process in an input context,
e.g. $x?(z).\lift{w}{y!(z)}$, we see that the process under the lift
operator may be shaped by prefixed inputs binding a name inside it. In
this sense, the lift operator will be seen as a way to dynamically
construct processes before reifying them as names.

Finally equipped with these standard features we can present the
dynamics of the calculus.

\subsubsection{Operational semantics} 

Finally, we introduce the computational dynamics. What marks these
algebras as distinct from other more traditionally studied algebraic
structures, e.g. vector spaces or polynomial rings, is the manner in
which dynamics is captured. In traditional structures, dynamics is typically
expressed through morphisms between such structures, as in linear maps
between vector spaces or morphisms between rings. In algebras
associated with the semantics of computation, the dynamics is
expressed as part of the algebraic structure itself, through a
reduction reduction relation typically denoted by $\red$. Below, we
give a recursive presentation of this relation for the calculus used
in the encoding.

$\red \subseteq \pi \times \pi$
$\red : \pi \to \mathcal{P}(\pi)$

\begin{mathpar}
  \inferrule* [lab=Comm] { \textsf{match}( x_{src}, x_{trgt} ) } { x_{trgt}?(y)P \; | \; x_{src}!\langle {Q} \rangle \red P\{\quotep{Q}/y}\} }
  \and \\
  \inferrule* [lab=Par] {{P} \red {P}'} {{{P} | {Q}} \red {{P}' | {Q}}}
  \and
  \inferrule* [lab=Equiv]{{{P} \scong {P}'} \andalso {{P}' \red {Q}'} \andalso {{Q}' \scong {Q}}}{{P} \red {Q}}
\end{mathpar}

\begin{eqnarray*}
  match_{\equiv} (\quotep{P},\quotep{Q}) & := & P \equiv Q \\
  match_{\dagger}(\quotep{P},\quotep{Q}) & := & \forall R. P|Q \red^{*} R => R \red^{*} 0 \\
  match_{K}(\quotep{P},\quotep{Q}) & := & K \mbox{ for some context } K
\end{eqnarray*}

$u?(x)P | u!\langle Q \rangle \red P\{\quotep{Q}/x\}$

%We write $\wred$ for $\red^*$, and $P\red$ if $\exists Q $ such that $ P \red Q$.
We write $P\red$ if $\exists Q $ such that $ P \red Q$ and $P\not\red$, otherwise.

\section{Replication}

As mentioned before, it is known that replication (and hence
recursion) can be implemented in a higher-order process algebra
\cite{SangiorgiWalker}. As our first example of calculation with the
machinery thus far presented we give the construction explicitly in
the {\rhoc}.

\begin{eqnarray}
	D_{x} & := & \prefix{x}{y}{(\binpar{\outputp{x}{y}}{@{y}})} \nonumber\\
	\bangp_{x}{P} & := & \binpar{{x}!\langle{\binpar{D_{x}}{P}}\rangle}{D_{x}} \nonumber
\end{eqnarray}

\begin{eqnarray}
	\bangp_{x}{P} & & \nonumber\\
	=
	& {x}!\langle{(\prefix{x}{y}{(\outputp{x}{y} | @{y})) | P}}\rangle 
	      | \prefix{x}{y}{(\outputp{x}{y} | @{y})} & \nonumber\\
	\red
	& (\outputp{x}{y} | @{y})\substn{\quotep{(\prefix{x}{y}{(@{y} | \outputp{x}{y})) | P}}}{y} & \nonumber\\
	=
	& \outputp{x}{\quotep{(\prefix{x}{y}{(\outputp{x}{y} | @{y})) | P}}}
	  | {(\prefix{x}{y}{(\outputp{x}{y} | @{y})) | P}} & \nonumber\\
	\red
	& \ldots & \nonumber\\
	\red^*
	& P | P | \ldots & \nonumber
\end{eqnarray}

Of course, this encoding, as an implementation, runs away, unfolding
$\bangp{P}$ eagerly. A lazier and more implementable replication
operator, restricted to input-guarded processes, may be obtained as follows.

\begin{eqnarray}
\bangp{\prefix{u}{v}{P}} 
	:= 
	\binpar{\lift{x}{\prefix{u}{v}{(\binpar{D(x)}{P})}}}{D(x)} \nonumber
\end{eqnarray}

\begin{remark}
  Note that the lazier definition still does not deal with summation
  or mixed summation (i.e. sums over input and output). The reader is
  invited to construct definitions of replication that deal with these
  features. 

  Further, the definitions are parameterized in a name, $x$. Can you,
  gentle reader, make a definition that eliminates this parameter and
  guarantees no accidental interaction between the replication
  machinery and the process being replicated -- i.e. no accidental
  sharing of names used by the process to get its work done and the
  name(s) used by the replication to effect copying. This latter
  revision of the definition of replication is crucial to obtaining
  the expected identity $!!P \sim !P$.
\end{remark}

\begin{remark}\label{rem:paradoxical_combinator}
  The reader familiar with the lambda calculus will have noticed the
  similarity between $D$ and the paradoxical combinator.

  [Ed. note: the existence of this seems to suggest we have to be more
  restrictive on the set of processes and names we admit if we are to
  support no-cloning.]
\end{remark}

\subsubsection{Bisimulation}

The computational dynamics gives rise to another kind of equivalence,
the equivalence of computational behavior. As previously mentioned
this is typically captured \emph{via} some form of bisimulation.

% The notion we use in this paper is weak barbed bisimulation
% \cite{milner91polyadicpi}.

The notion we use in this paper is derived from weak barbed
bisimulation \cite{milner91polyadicpi}. 

\begin{definition}
An \emph{observation relation}, $\downarrow_{\mathcal N}$, over a set
of names, $\mathcal N$, is the smallest relation satisfying the rules
below.

\infrule[Out-barb]{y \in {\mathcal N}, \; x \nameeq y}
		  {\outputp{x}{v} \downarrow_{\mathcal N} x}
\infrule[Par-barb]{\mbox{$P\downarrow_{\mathcal N} x$ or $Q\downarrow_{\mathcal N} x$}}
		  {\binpar{P}{Q} \downarrow_{\mathcal N} x}

We write $P \Downarrow_{\mathcal N} x$ if there is $Q$ such that 
$P \wred Q$ and $Q \downarrow_{\mathcal N} x$.
\end{definition}

\begin{definition}
%\label{def.bbisim}
An  ${\mathcal N}$-\emph{barbed bisimulation} over a set of names, ${\mathcal N}$, is a symmetric binary relation 
${\mathcal S}_{\mathcal N}$ between agents such that $P\rel{S}_{\mathcal N}Q$ implies:
\begin{enumerate}
\item If $P \red P'$ then $Q \wred Q'$ and $P'\rel{S}_{\mathcal N} Q'$.
\item If $P\downarrow_{\mathcal N} x$, then $Q\Downarrow_{\mathcal N} x$.
\end{enumerate}
$P$ is ${\mathcal N}$-barbed bisimilar to $Q$, written
$P \wbbisim_{\mathcal N} Q$, if $P \rel{S}_{\mathcal N} Q$ for some ${\mathcal N}$-barbed bisimulation ${\mathcal S}_{\mathcal N}$.
\end{definition}

$\mathcal{R} \subseteq \pi \times \pi$

$P \mathcal{R} Q => \forall P'. P \red P' \Rightarrow \exists Q'. Q \red Q', P' \mathcal{R} Q'$

$P \vdash x \Rightarrow Q \vdash x$

\begin{mathpar}
  \inferrule*[lab=Out-barb]{x \nameeq y}{{y}!\langle{Q}\rangle \vdash x}
  \and
  \inferrule*[lab=Par-barb]{\mbox{$P\vdash x$ or $Q\vdash x$}}{\binpar{P}{Q} \vdash x}
\end{mathpar}

\subsubsection{Contexts}

One of the principle advantages of computational calculi like the
$\pi$-calculus is a well-defined notion of context,
contextual-equivalence and a correlation between
contextual-equivalence and notions of bisimulation. The notion of
context allows the decomposition of a process into (sub-)process and
its syntactic environment, its context. Thus, a context may be
thought of as a process with a ``hole'' (written $\Box$) in it. The
application of a context $M$ to a process $P$, written $M[P]$, is
tantamount to filling the hole in $M$ with $P$. In this paper we do
not need the full weight of this theory, but do make use of the notion
of context in the proof the main theorem. 

\begin{mathpar}
  \inferrule* [lab=summation] {} {{M_{M},M_{N}} \bc \Box \;|\; x.M_{A} \;|\; M_{M}+M_{N}}
  \and
  \inferrule* [lab=agent] {} {{M_{A}} \bc (\vec{x})M_{P} \;| \; \clift{P_0,\ldots,M_{P},\ldots,P_N}}
  \and \\
  \inferrule* [lab=process] {} {{M_{P}} \bc M_{N} \;| \;P|M_{P} }
\end{mathpar} 

\begin{mathpar}
  \inferrule* [lab=sychronization] {} {M_{N} \bc \Box \;|\; x?M_{F} \;|\; x!M_{C}}
  \and
  \inferrule* [lab=abstraction] {} {{M_{F}} \bc (x)M_{P} }
  \and
  \inferrule* [lab=concretion] {} {{M_{C}} \bc \langle M_{P} \rangle }
  \and \\
  \inferrule* [lab=process] {} {{M_{P}} \bc M_{N} \;| \;P|M_{P} }
\end{mathpar}

\begin{definition}[contextual application] Given a context $M$, and
  process $P$, we define the \emph{contextual application}, $M[P] :=
  M\{P/\Box\}$. That is, the contextual application of M to P is the
  substitution of $P$ for $\Box$ in $M$.
\end{definition}

$\meaningof{-} : L \to \mathcal{P}(\pi)$

\begin{mathpar}
  \inferrule* [lab=collection] {} {\meaningof{true} = \pi, \and \meaningof{~E} = \pi \setminus \meaningof{E}, \and \meaningof{E_{1} \& E_{2}} = \meaningof{E_{1}} \cap \meaningof{E_{2}}}
\end{mathpar}

\begin{mathpar}
  \inferrule* [lab=structure] {} {\meaningof{0} = \{ P \in \pi | P \equiv 0 \}, \and \\ \meaningof{E_1 | E_2} = \{ P \in \pi | P \equiv P_{1} | P_{2}, P_{1} \in \meaningof{E_{1}}, P_{2} \in \meaningof{E_2}\} }
\end{mathpar}

\begin{mathpar}
 \inferrule* [lab=behavior] {} {\meaningof{\langle a?b \rangle E} = \{ P \in \pi | P \equiv Q | u?(y)P', \\ \and \\\\ \and \\ \;\;\; u \in \meaningof{a}, \forall z.P'\{z/y\} \in \meaningof{E\{z/b\}}\}, \and \\ \meaningof{a!E} = \{ P \in \pi | P \equiv Q | x!\langle P' \rangle, x \in \meaningof{a} P' \in \meaningof{E}\} }
\end{mathpar}

\begin{mathpar}
 \inferrule* [lab=nominal] {} {\meaningof{\quotep{E}} = \{ \quotep{P} \in \quotep{\pi} | P \in \meaningof{E} \}, \and \meaningof{\quotep{P}} = \{ \quotep{Q} \in \quotep{\pi} | P \equiv Q \} \and \\ \meaningof{@\quotep{E}} = \{ P \in \pi | P \equiv @x, x \in \meaningof{E} \}}
\end{mathpar}

\begin{eqnarray*}
  \\
  \meaningof{-} : TS \to ST
\end{eqnarray*}

\begin{eqnarray*}
  \\
  L : TS \to ST
\end{eqnarray*}

\begin{eqnarray*}
  \\
  P \models E \iff P \in \meaningof{E}
\end{eqnarray*}

\begin{eqnarray*}
  P \approx_{L} Q \iff \forall E \in L. P \models E \iff Q \models E
\end{eqnarray*}

\begin{eqnarray*}
  P \approx_{K} Q
\end{eqnarray*}

\begin{eqnarray*}
  P \approx Q
\end{eqnarray*}

$\approx_{K} = \approx = \approx_{L}$

\subsubsection{Contextual duality}

Note that contexts extend the quotation operation to a family of
operations from processes to names. Given a context, $M$, we can
define a \emph{nominal context}, $\quotep{M}$ by $\quotep{M}[P] :=
\quotep{M[P]}$. To foreshadow what is to come we observe that these
operations enjoy a duality with processes very much like the duality
between vectors and maps from vectors to scalars.

Further, because the calculus is essentially higher-order, we have a
correspondence between contexts and processes. More specifically,
given a name $x$ and a context $M$ we can construct $M^{*}_{x}$ such
that 

\begin{mathpar}
  M^{*}_{x} | \lift{x}{P} \red M[P]
\end{mathpar}

namely,

\begin{mathpar}
  M^{*}_{x} := x?(u).M[\dropn{u}]
\end{mathpar}

The dependence of $M^{*}_{x}$ on a name makes it an abstraction, 

\begin{mathpar}
  M^{*} := (x)x?(u).M[\dropn{u}]
\end{mathpar}

\subsection{Additional notation}

It will sometimes be convenient to denote the process a name
quotes. We already have the notation $x = \quotep{P}$, but it will be
convenient to introduce an alternate notation, $\procn{x}$, when we
want to emphasize the connection to the use of the name. Note that, by
virtue of name equivalence, $\quotep{\procn{x}} \nameeq x$; so, the
notation is consistent with previous definitions.

Further, because names have structure it is possible to effect
substitutions on the basis of that structure. This means we need to
upgrade our notation for substitutions, which we accomplish by
adapting comprehension notation. Thus,

\begin{mathpar}
  P\{ y / x : x \in S \}
\end{mathpar}

is interpreted to mean the process derived from P by replacing (in a
capture-avoiding manner) each occurrence of $x$ in $S$ by $y$. For example,

\begin{mathpar}
  P\{ \quotep{\procn{x}|\procn{x}} / x : x \in \freenames{P} \}
\end{mathpar}

will replace each (occurrence) of a free name $x$ in $P$ by
$\quotep{\procn{x}|\procn{x}}$.

Also, we will avail ourselves of the notation $x^{L}$ and $x^{R}$ to
denote injections of a name into disjoint copies of the name
space. There are numerous ways to accomplish this. One example can be
found in \cite{MeredithR05}. This notation overloads to vectors of
names: $\vec{x}^{\pi} := (x_{i}^{\pi} \; : \; 0 \leq i < |\vec{x}| )$ where $\pi \in \{L,R\}$.

We also use $P^{\Box} := P|\Box$.

In \cite{MeredithR05} an interpretation of the new operator is
given. It turns out that there are several possible interpretations
all enjoying the requisite algebraic properties of the operator (see
\cite{milner91polyadicpi}). We will therefore make liberal use of
$(\nu\; \vec{x})P$.

% subsection the_syntax_and_semantics_of_the_notation_system (end)   

\input{qm2pi.qmops} 

\input{qm2pi.sterngerlach} 

\input{qm2pi.metric} 

% section concurrent_process_calculi (end)

%\input{qm2pi.proofsketch}

% section proof sketch (end)

%\input{qm2pi.slviaknots} 

% section spatial logic via knots (end)

\input{qm2pi.conclusion}

% section conclusion (end)

%\input{qm2pi.dtcodes} 

% section wiring algorithm (end)

\input{qm2pi.ack} 

% section acknowledgments (end)

\newpage


\bibliographystyle{plain}   
\bibliography{../../biblios/main.bib}

\input{qm2pi.rhodetails}

\end{document}

 

%\documentclass[12pt]{llncs}
%\documentclass{jktr}

\usepackage[pdftex]{hyperref}                   
\usepackage {listings}
\usepackage {mathpartir}
\usepackage{bcprules}
%\usepackage{listings}
                       
\usepackage{graphicx} 
%\usepackage[margins=2.5cm,nohead,nofoot]{geometry}
%\usepackage{geometry}
\usepackage{amsfonts}
\usepackage{amstext}
\usepackage{latexsym}
\usepackage{amssymb}
\usepackage{color}


%\include{myPreamble}
\include{qm2pi.local} 

%\ifpdf
%\usepackage[pdftex]{graphicx}
%\else
%\usepackage{graphicx}
%\fi

 % \ifpdf
%  \usepackage{pdfsync}
%  \if


%\title{Brief Article}
%\author{David F. Snyder}
%\author{L.G. Meredith}

%\address{Dept. of Math., Texas State University--San Marcos, San Marcos, TX 78666}
       
\pagestyle{empty}


\begin{document}

\lstset{language=[Objective]Caml,frame=shadowbox}

\input{qm2pi.front}

% section front matter (end)

\input{qm2pi.intro} 
 
% section introduction (end)

% \input{qm2pi.knotations} 

% section notation (end)

\input{qm2pi.process.calculi} 

% section concurrent_process_calculi_and_spatial_logics_ (end)
    
%\input{qm2pi.knots2pi} 

%\input{qm2pi.trefoil} 

%\input{qm2pi.mainthm} 

% subsection basic_interpretation (end)

%\input{qm2pi.rho.presentation} 
\subsection{The syntax and semantics of the notation system}\label{sub:the_syntax_and_semantics_of_the_notation_system} % (fold)

We now summarize a technical presentation of the calculus that
embodies our theory of dynamics. The typical presentation of such a
calculus follows the style of giving generators and relations on
them. The grammar, below, describing term constructors, freely
generates the set of processes, $\Proc$. This set is then quotiented
by a relation known as structural congruence and it is over this set
that the notion of dynamics is expressed. This presentation is
essentially that of \cite{MeredithR05} with the addition of
polyadicity and summation. For readability we have relegated some of
the technical subtleties to an appendix.

\subsubsection{Process grammar}\label{subsub:process_grammar}

\begin{mathpar}
  \inferrule* [lab=synchronization] {} {{M} \bc \pzero \;|\; x?F \;|\; x!C }
  \and
  \inferrule* [lab=abstraction] {} {{F} \bc (x)P}
  \and
  \inferrule* [lab=concretion] {} {{C} \bc \langle Q \rangle}
  \and
  \inferrule* [lab=process] {} {{P,Q} \bc M \;| \;P|Q \;|\; @{x}}
  \and
  \inferrule* [lab=name] {} {{x} \bc \quotep{P}}
\end{mathpar} 

Note that $\vec{x}$ (resp. $\vec{P}$) denotes a vector of names
(resp. processes) of length $|\vec{x}|$ (resp. $|\vec{P}|$). We adopt
the following useful abbreviations.

\begin{mathpar}
   x?(\vec{y}).P := x.(\vec{y})P \and  x\clift{\vec{P}} := x.\clift{\vec{P}}
   \and x!(y) := \lift{x}{\dropn{y}}
   \and \Pi_{i=0}^{n-1}P_i := P_0 | \ldots | P_{n-1}
\end{mathpar}

\subsubsection{Structural congruence}

\paragraph{Free and bound names and alpha-equivalence.} At the
core of structural equivalence is alpha-equivalence which identifies
process that are the same up to a change of variable. Formally, we
recognize the distinction between free and bound names. The free names
of a process, $\freenames{P}$, may be calculated recursively as
follows:

\begin{mathpar}
\freenames{\pzero} := \emptyset
  \and \\
  \freenames{x?(y).P} := \{ x \} \cup (\freenames{P} \setminus \{ y \})
  \and 
  \freenames{x!\langle P \rangle} := \{ x \} \cup \{ P \} 
  \and \\
  \freenames{P|Q} := \freenames{P} \cup \freenames{Q}
  \and \\
  \freenames{@{x}} := \{ x \}
\end{mathpar}

$\pi$
$\quotep{\pi}$

$\freenames{-} : \pi \to \mathcal{P}(\quotep{\pi})$

\begin{eqnarray*}
  \freenames{\pzero} & := & \emptyset \\
  \freenames{x?(y).P} & := & \{ x \} \cup (\freenames{P} \setminus \{ y \}) \\
  \freenames{x!\langle P \rangle} & := & \{ x \} \cup \{ P \} \\
  \freenames{P|Q} & := & \freenames{P} \cup \freenames{Q} \\
  \freenames{\dropn{x}} & := & \{ x \}
\end{eqnarray*}

The bound names of a process, $\boundnames{P}$, are those names occurring in $P$
that are not free. For example, in $x?(y).0$, the name $x$ is free, while $y$ is bound.

\begin{mathpar}
  \inferrule* [lab=monoidal-laws] {} { P|Q \equiv Q|P \and P|0 \equiv P \and P|(Q|R) \equiv (P|Q)|R }
\end{mathpar}

\begin{mathpar}
  \inferrule* [lab=alpha-equivalence] {} { (x)P \equiv (y)P\{y/x\} \and y \not\in \freenames{P} }
\end{mathpar}

\begin{definition}
Then two processes, $P,Q$, are alpha-equivalent if $P = Q\{\vec{y}/\vec{x}\}$ for
some $\vec{x} \in \boundnames{Q},\vec{y} \in \boundnames{P}$, where $Q\{\vec{y}/\vec{x}\}$
denotes the capture-avoiding substitution of $\vec{y}$ for $\vec{x}$ in $Q$.
\end{definition}

\begin{definition}
  The {\em structural congruence} \cite{SangiorgiWalker} , $\equiv$,
  between processes is the least congruence containing
  alpha-equivalence, satisfying the abelian monoid laws
  (associativity, commutativity and $\pzero$ as identity) for parallel
  composition $|$ and for summation $+$.
\end{definition}

\subsection{Name equivalence}

We take name equivalence, written $\nameeq$, to be the smallest
equivalence relation generated by the following rules.

\begin{mathpar}
\inferrule*[lab=Quote-drop]
{ }
{ \quotep{@{x}} \nameeq x }

\inferrule*[lab=Struct-equiv]
{ P \scong Q }
{ \quotep{P} \nameeq \quotep{Q} }
\end{mathpar}

The astute reader will have noticed that the mutual recursion of names
and processes imposes a mutual recursion on alpha-equivalence and
structural equivalence via name-equivalence. Fortunately, all of this
works out pleasantly and we may calculate in the natural way, free of
concern. The reader interested in the details is referred to the
appendix \ref{appendix:rho_details}.

\subsection{Substitution}

We use $\Proc$ for the set of processes, $\QProc$ for the set of
names, and $\id{\{}\vec{y} / \vec{x} \id{\}}$ to denote partial maps,
$s : \QProc \rightarrow \QProc$. A map, $s$ lifts, uniquely, to a map
on process terms, $\widehat{s} : \Proc \rightarrow \Proc$ by the
following equations.

\begin{mathpar}
  (0) \psubstp{Q}{P} := 0 \\
  (R \juxtap S) \psubstp{Q}{P}
  :=    
  (R)\psubstp{Q}{P} \juxtap (S) \psubstp{Q}{P} \\
  (x?(y).R) \psubstp{Q}{P}    
  :=    
  (x)\substp{Q}{P} (z)\concat( (R \psubstn{z}{y}) \psubstp{Q}{P} ) \\
  (\lift{x}{R}) \psubstp{Q}{P}  
  :=
  \lift{(x)\substp{Q}{P}}{ R \psubstp{Q}{P} } \\
%   (\dropn{x})  \psubstp{Q}{P}       
%   := 
%   \left\{ 
%     \begin{array}{ccc} 
%       \dropn{\quotep{Q}} & & x \nameeq \quotep{P} \\
%       \dropn{x} & & otherwise \\
%     \end{array}
%   \right. 
  (\dropn{x})  \psubstp{Q}{P}       
  := 
  \left\{ 
    \begin{array}{ccc} 
      Q & & x \nameeq \quotep{P} \\
      \dropn{x} & & otherwise \\
    \end{array}
  \right.
\end{mathpar}
 

where

\begin{eqnarray}
  (x)\id{\{} \lpquote Q \rpquote / \lpquote P \rpquote \id{\}}            = 
  \left\{ 
    \begin{array}{ccc}
      \lpquote Q \rpquote & & x \nameeq \lpquote P \rpquote \\
      x & & otherwise \\
    \end{array}
  \right. \nonumber
\end{eqnarray}

and $z$ is chosen distinct from $\quotep{P}$, $\quotep{Q}$, the free
names in $Q$, and all the names in $R$. Our $\alpha$-equivalence will
be built in the standard way from this substitution.

\begin{remark}\label{rem:no_self_referential_names}
  One consequence of these definitions is that $\forall P. \quotep{P}
  \not\in \freenames{P}$.
\end{remark}

\subsection{ Dynamic quote: an example }

Anticipating something of what's to come, consider applying the
substitution, $\widehat{\id{\{}u / z \id{\}}}$, to the following pair
of processes, $\lift{w}{y!(z)}$ and $w[ \lpquote y!(z) \rpquote ]$.

\begin{eqnarray}
	\lift{w}{y!(z)}\widehat{\id{\{}u / z \id{\}}}
		& = &
		\lift{w}{y!(u)} \nonumber\\
	w[ \lpquote y!(z) \rpquote ] \widehat{ \id{\{}u / z \id{\}} }
		& = &
		w[ \lpquote y!(z) \rpquote ] \nonumber
\end{eqnarray}

Because the body of the process between quotes is impervious to
substitution, we get radically different answers. In fact, by
examining the first process in an input context,
e.g. $x?(z).\lift{w}{y!(z)}$, we see that the process under the lift
operator may be shaped by prefixed inputs binding a name inside it. In
this sense, the lift operator will be seen as a way to dynamically
construct processes before reifying them as names.

Finally equipped with these standard features we can present the
dynamics of the calculus.

\subsubsection{Operational semantics} 

Finally, we introduce the computational dynamics. What marks these
algebras as distinct from other more traditionally studied algebraic
structures, e.g. vector spaces or polynomial rings, is the manner in
which dynamics is captured. In traditional structures, dynamics is typically
expressed through morphisms between such structures, as in linear maps
between vector spaces or morphisms between rings. In algebras
associated with the semantics of computation, the dynamics is
expressed as part of the algebraic structure itself, through a
reduction reduction relation typically denoted by $\red$. Below, we
give a recursive presentation of this relation for the calculus used
in the encoding.

$\red \subseteq \pi \times \pi$
$\red : \pi \to \mathcal{P}(\pi)$

\begin{mathpar}
  \inferrule* [lab=Comm] { \textsf{match}( x_{src}, x_{trgt} ) } { x_{trgt}?(y)P \; | \; x_{src}!\langle {Q} \rangle \red P\{\quotep{Q}/y}\} }
  \and \\
  \inferrule* [lab=Par] {{P} \red {P}'} {{{P} | {Q}} \red {{P}' | {Q}}}
  \and
  \inferrule* [lab=Equiv]{{{P} \scong {P}'} \andalso {{P}' \red {Q}'} \andalso {{Q}' \scong {Q}}}{{P} \red {Q}}
\end{mathpar}

\begin{eqnarray*}
  match_{\equiv} (\quotep{P},\quotep{Q}) & := & P \equiv Q \\
  match_{\dagger}(\quotep{P},\quotep{Q}) & := & \forall R. P|Q \red^{*} R => R \red^{*} 0 \\
  match_{K}(\quotep{P},\quotep{Q}) & := & K \mbox{ for some context } K
\end{eqnarray*}

$u?(x)P | u!\langle Q \rangle \red P\{\quotep{Q}/x\}$

%We write $\wred$ for $\red^*$, and $P\red$ if $\exists Q $ such that $ P \red Q$.
We write $P\red$ if $\exists Q $ such that $ P \red Q$ and $P\not\red$, otherwise.

\section{Replication}

As mentioned before, it is known that replication (and hence
recursion) can be implemented in a higher-order process algebra
\cite{SangiorgiWalker}. As our first example of calculation with the
machinery thus far presented we give the construction explicitly in
the {\rhoc}.

\begin{eqnarray}
	D_{x} & := & \prefix{x}{y}{(\binpar{\outputp{x}{y}}{@{y}})} \nonumber\\
	\bangp_{x}{P} & := & \binpar{{x}!\langle{\binpar{D_{x}}{P}}\rangle}{D_{x}} \nonumber
\end{eqnarray}

\begin{eqnarray}
	\bangp_{x}{P} & & \nonumber\\
	=
	& {x}!\langle{(\prefix{x}{y}{(\outputp{x}{y} | @{y})) | P}}\rangle 
	      | \prefix{x}{y}{(\outputp{x}{y} | @{y})} & \nonumber\\
	\red
	& (\outputp{x}{y} | @{y})\substn{\quotep{(\prefix{x}{y}{(@{y} | \outputp{x}{y})) | P}}}{y} & \nonumber\\
	=
	& \outputp{x}{\quotep{(\prefix{x}{y}{(\outputp{x}{y} | @{y})) | P}}}
	  | {(\prefix{x}{y}{(\outputp{x}{y} | @{y})) | P}} & \nonumber\\
	\red
	& \ldots & \nonumber\\
	\red^*
	& P | P | \ldots & \nonumber
\end{eqnarray}

Of course, this encoding, as an implementation, runs away, unfolding
$\bangp{P}$ eagerly. A lazier and more implementable replication
operator, restricted to input-guarded processes, may be obtained as follows.

\begin{eqnarray}
\bangp{\prefix{u}{v}{P}} 
	:= 
	\binpar{\lift{x}{\prefix{u}{v}{(\binpar{D(x)}{P})}}}{D(x)} \nonumber
\end{eqnarray}

\begin{remark}
  Note that the lazier definition still does not deal with summation
  or mixed summation (i.e. sums over input and output). The reader is
  invited to construct definitions of replication that deal with these
  features. 

  Further, the definitions are parameterized in a name, $x$. Can you,
  gentle reader, make a definition that eliminates this parameter and
  guarantees no accidental interaction between the replication
  machinery and the process being replicated -- i.e. no accidental
  sharing of names used by the process to get its work done and the
  name(s) used by the replication to effect copying. This latter
  revision of the definition of replication is crucial to obtaining
  the expected identity $!!P \sim !P$.
\end{remark}

\begin{remark}\label{rem:paradoxical_combinator}
  The reader familiar with the lambda calculus will have noticed the
  similarity between $D$ and the paradoxical combinator.

  [Ed. note: the existence of this seems to suggest we have to be more
  restrictive on the set of processes and names we admit if we are to
  support no-cloning.]
\end{remark}

\subsubsection{Bisimulation}

The computational dynamics gives rise to another kind of equivalence,
the equivalence of computational behavior. As previously mentioned
this is typically captured \emph{via} some form of bisimulation.

% The notion we use in this paper is weak barbed bisimulation
% \cite{milner91polyadicpi}.

The notion we use in this paper is derived from weak barbed
bisimulation \cite{milner91polyadicpi}. 

\begin{definition}
An \emph{observation relation}, $\downarrow_{\mathcal N}$, over a set
of names, $\mathcal N$, is the smallest relation satisfying the rules
below.

\infrule[Out-barb]{y \in {\mathcal N}, \; x \nameeq y}
		  {\outputp{x}{v} \downarrow_{\mathcal N} x}
\infrule[Par-barb]{\mbox{$P\downarrow_{\mathcal N} x$ or $Q\downarrow_{\mathcal N} x$}}
		  {\binpar{P}{Q} \downarrow_{\mathcal N} x}

We write $P \Downarrow_{\mathcal N} x$ if there is $Q$ such that 
$P \wred Q$ and $Q \downarrow_{\mathcal N} x$.
\end{definition}

\begin{definition}
%\label{def.bbisim}
An  ${\mathcal N}$-\emph{barbed bisimulation} over a set of names, ${\mathcal N}$, is a symmetric binary relation 
${\mathcal S}_{\mathcal N}$ between agents such that $P\rel{S}_{\mathcal N}Q$ implies:
\begin{enumerate}
\item If $P \red P'$ then $Q \wred Q'$ and $P'\rel{S}_{\mathcal N} Q'$.
\item If $P\downarrow_{\mathcal N} x$, then $Q\Downarrow_{\mathcal N} x$.
\end{enumerate}
$P$ is ${\mathcal N}$-barbed bisimilar to $Q$, written
$P \wbbisim_{\mathcal N} Q$, if $P \rel{S}_{\mathcal N} Q$ for some ${\mathcal N}$-barbed bisimulation ${\mathcal S}_{\mathcal N}$.
\end{definition}

$\mathcal{R} \subseteq \pi \times \pi$

$P \mathcal{R} Q => \forall P'. P \red P' \Rightarrow \exists Q'. Q \red Q', P' \mathcal{R} Q'$

$P \vdash x \Rightarrow Q \vdash x$

\begin{mathpar}
  \inferrule*[lab=Out-barb]{x \nameeq y}{{y}!\langle{Q}\rangle \vdash x}
  \and
  \inferrule*[lab=Par-barb]{\mbox{$P\vdash x$ or $Q\vdash x$}}{\binpar{P}{Q} \vdash x}
\end{mathpar}

\subsubsection{Contexts}

One of the principle advantages of computational calculi like the
$\pi$-calculus is a well-defined notion of context,
contextual-equivalence and a correlation between
contextual-equivalence and notions of bisimulation. The notion of
context allows the decomposition of a process into (sub-)process and
its syntactic environment, its context. Thus, a context may be
thought of as a process with a ``hole'' (written $\Box$) in it. The
application of a context $M$ to a process $P$, written $M[P]$, is
tantamount to filling the hole in $M$ with $P$. In this paper we do
not need the full weight of this theory, but do make use of the notion
of context in the proof the main theorem. 

\begin{mathpar}
  \inferrule* [lab=summation] {} {{M_{M},M_{N}} \bc \Box \;|\; x.M_{A} \;|\; M_{M}+M_{N}}
  \and
  \inferrule* [lab=agent] {} {{M_{A}} \bc (\vec{x})M_{P} \;| \; \clift{P_0,\ldots,M_{P},\ldots,P_N}}
  \and \\
  \inferrule* [lab=process] {} {{M_{P}} \bc M_{N} \;| \;P|M_{P} }
\end{mathpar} 

\begin{mathpar}
  \inferrule* [lab=sychronization] {} {M_{N} \bc \Box \;|\; x?M_{F} \;|\; x!M_{C}}
  \and
  \inferrule* [lab=abstraction] {} {{M_{F}} \bc (x)M_{P} }
  \and
  \inferrule* [lab=concretion] {} {{M_{C}} \bc \langle M_{P} \rangle }
  \and \\
  \inferrule* [lab=process] {} {{M_{P}} \bc M_{N} \;| \;P|M_{P} }
\end{mathpar}

\begin{definition}[contextual application] Given a context $M$, and
  process $P$, we define the \emph{contextual application}, $M[P] :=
  M\{P/\Box\}$. That is, the contextual application of M to P is the
  substitution of $P$ for $\Box$ in $M$.
\end{definition}

$\meaningof{-} : L \to \mathcal{P}(\pi)$

\begin{mathpar}
  \inferrule* [lab=collection] {} {\meaningof{true} = \pi, \and \meaningof{~E} = \pi \setminus \meaningof{E}, \and \meaningof{E_{1} \& E_{2}} = \meaningof{E_{1}} \cap \meaningof{E_{2}}}
\end{mathpar}

\begin{mathpar}
  \inferrule* [lab=structure] {} {\meaningof{0} = \{ P \in \pi | P \equiv 0 \}, \and \\ \meaningof{E_1 | E_2} = \{ P \in \pi | P \equiv P_{1} | P_{2}, P_{1} \in \meaningof{E_{1}}, P_{2} \in \meaningof{E_2}\} }
\end{mathpar}

\begin{mathpar}
 \inferrule* [lab=behavior] {} {\meaningof{\langle a?b \rangle E} = \{ P \in \pi | P \equiv Q | u?(y)P', \\ \and \\\\ \and \\ \;\;\; u \in \meaningof{a}, \forall z.P'\{z/y\} \in \meaningof{E\{z/b\}}\}, \and \\ \meaningof{a!E} = \{ P \in \pi | P \equiv Q | x!\langle P' \rangle, x \in \meaningof{a} P' \in \meaningof{E}\} }
\end{mathpar}

\begin{mathpar}
 \inferrule* [lab=nominal] {} {\meaningof{\quotep{E}} = \{ \quotep{P} \in \quotep{\pi} | P \in \meaningof{E} \}, \and \meaningof{\quotep{P}} = \{ \quotep{Q} \in \quotep{\pi} | P \equiv Q \} \and \\ \meaningof{@\quotep{E}} = \{ P \in \pi | P \equiv @x, x \in \meaningof{E} \}}
\end{mathpar}

\begin{eqnarray*}
  \\
  \meaningof{-} : TS \to ST
\end{eqnarray*}

\begin{eqnarray*}
  \\
  L : TS \to ST
\end{eqnarray*}

\begin{eqnarray*}
  \\
  P \models E \iff P \in \meaningof{E}
\end{eqnarray*}

\begin{eqnarray*}
  P \approx_{L} Q \iff \forall E \in L. P \models E \iff Q \models E
\end{eqnarray*}

\begin{eqnarray*}
  P \approx_{K} Q
\end{eqnarray*}

\begin{eqnarray*}
  P \approx Q
\end{eqnarray*}

$\approx_{K} = \approx = \approx_{L}$

\subsubsection{Contextual duality}

Note that contexts extend the quotation operation to a family of
operations from processes to names. Given a context, $M$, we can
define a \emph{nominal context}, $\quotep{M}$ by $\quotep{M}[P] :=
\quotep{M[P]}$. To foreshadow what is to come we observe that these
operations enjoy a duality with processes very much like the duality
between vectors and maps from vectors to scalars.

Further, because the calculus is essentially higher-order, we have a
correspondence between contexts and processes. More specifically,
given a name $x$ and a context $M$ we can construct $M^{*}_{x}$ such
that 

\begin{mathpar}
  M^{*}_{x} | \lift{x}{P} \red M[P]
\end{mathpar}

namely,

\begin{mathpar}
  M^{*}_{x} := x?(u).M[\dropn{u}]
\end{mathpar}

The dependence of $M^{*}_{x}$ on a name makes it an abstraction, 

\begin{mathpar}
  M^{*} := (x)x?(u).M[\dropn{u}]
\end{mathpar}

\subsection{Additional notation}

It will sometimes be convenient to denote the process a name
quotes. We already have the notation $x = \quotep{P}$, but it will be
convenient to introduce an alternate notation, $\procn{x}$, when we
want to emphasize the connection to the use of the name. Note that, by
virtue of name equivalence, $\quotep{\procn{x}} \nameeq x$; so, the
notation is consistent with previous definitions.

Further, because names have structure it is possible to effect
substitutions on the basis of that structure. This means we need to
upgrade our notation for substitutions, which we accomplish by
adapting comprehension notation. Thus,

\begin{mathpar}
  P\{ y / x : x \in S \}
\end{mathpar}

is interpreted to mean the process derived from P by replacing (in a
capture-avoiding manner) each occurrence of $x$ in $S$ by $y$. For example,

\begin{mathpar}
  P\{ \quotep{\procn{x}|\procn{x}} / x : x \in \freenames{P} \}
\end{mathpar}

will replace each (occurrence) of a free name $x$ in $P$ by
$\quotep{\procn{x}|\procn{x}}$.

Also, we will avail ourselves of the notation $x^{L}$ and $x^{R}$ to
denote injections of a name into disjoint copies of the name
space. There are numerous ways to accomplish this. One example can be
found in \cite{MeredithR05}. This notation overloads to vectors of
names: $\vec{x}^{\pi} := (x_{i}^{\pi} \; : \; 0 \leq i < |\vec{x}| )$ where $\pi \in \{L,R\}$.

We also use $P^{\Box} := P|\Box$.

In \cite{MeredithR05} an interpretation of the new operator is
given. It turns out that there are several possible interpretations
all enjoying the requisite algebraic properties of the operator (see
\cite{milner91polyadicpi}). We will therefore make liberal use of
$(\nu\; \vec{x})P$.

% subsection the_syntax_and_semantics_of_the_notation_system (end)   

\input{qm2pi.qmops} 

\input{qm2pi.sterngerlach} 

\input{qm2pi.metric} 

% section concurrent_process_calculi (end)

%\input{qm2pi.proofsketch}

% section proof sketch (end)

%\input{qm2pi.slviaknots} 

% section spatial logic via knots (end)

\input{qm2pi.conclusion}

% section conclusion (end)

%\input{qm2pi.dtcodes} 

% section wiring algorithm (end)

\input{qm2pi.ack} 

% section acknowledgments (end)

\newpage


\bibliographystyle{plain}   
\bibliography{../../biblios/main.bib}

\input{qm2pi.rhodetails}

\end{document}

 

%\documentclass[12pt]{llncs}
%\documentclass{jktr}

\usepackage[pdftex]{hyperref}                   
\usepackage {listings}
\usepackage {mathpartir}
\usepackage{bcprules}
%\usepackage{listings}
                       
\usepackage{graphicx} 
%\usepackage[margins=2.5cm,nohead,nofoot]{geometry}
%\usepackage{geometry}
\usepackage{amsfonts}
\usepackage{amstext}
\usepackage{latexsym}
\usepackage{amssymb}
\usepackage{color}


%\include{myPreamble}
\include{qm2pi.local} 

%\ifpdf
%\usepackage[pdftex]{graphicx}
%\else
%\usepackage{graphicx}
%\fi

 % \ifpdf
%  \usepackage{pdfsync}
%  \if


%\title{Brief Article}
%\author{David F. Snyder}
%\author{L.G. Meredith}

%\address{Dept. of Math., Texas State University--San Marcos, San Marcos, TX 78666}
       
\pagestyle{empty}


\begin{document}

\lstset{language=[Objective]Caml,frame=shadowbox}

\input{qm2pi.front}

% section front matter (end)

\input{qm2pi.intro} 
 
% section introduction (end)

% \input{qm2pi.knotations} 

% section notation (end)

\input{qm2pi.process.calculi} 

% section concurrent_process_calculi_and_spatial_logics_ (end)
    
%\input{qm2pi.knots2pi} 

%\input{qm2pi.trefoil} 

%\input{qm2pi.mainthm} 

% subsection basic_interpretation (end)

%\input{qm2pi.rho.presentation} 
\subsection{The syntax and semantics of the notation system}\label{sub:the_syntax_and_semantics_of_the_notation_system} % (fold)

We now summarize a technical presentation of the calculus that
embodies our theory of dynamics. The typical presentation of such a
calculus follows the style of giving generators and relations on
them. The grammar, below, describing term constructors, freely
generates the set of processes, $\Proc$. This set is then quotiented
by a relation known as structural congruence and it is over this set
that the notion of dynamics is expressed. This presentation is
essentially that of \cite{MeredithR05} with the addition of
polyadicity and summation. For readability we have relegated some of
the technical subtleties to an appendix.

\subsubsection{Process grammar}\label{subsub:process_grammar}

\begin{mathpar}
  \inferrule* [lab=synchronization] {} {{M} \bc \pzero \;|\; x?F \;|\; x!C }
  \and
  \inferrule* [lab=abstraction] {} {{F} \bc (x)P}
  \and
  \inferrule* [lab=concretion] {} {{C} \bc \langle Q \rangle}
  \and
  \inferrule* [lab=process] {} {{P,Q} \bc M \;| \;P|Q \;|\; @{x}}
  \and
  \inferrule* [lab=name] {} {{x} \bc \quotep{P}}
\end{mathpar} 

Note that $\vec{x}$ (resp. $\vec{P}$) denotes a vector of names
(resp. processes) of length $|\vec{x}|$ (resp. $|\vec{P}|$). We adopt
the following useful abbreviations.

\begin{mathpar}
   x?(\vec{y}).P := x.(\vec{y})P \and  x\clift{\vec{P}} := x.\clift{\vec{P}}
   \and x!(y) := \lift{x}{\dropn{y}}
   \and \Pi_{i=0}^{n-1}P_i := P_0 | \ldots | P_{n-1}
\end{mathpar}

\subsubsection{Structural congruence}

\paragraph{Free and bound names and alpha-equivalence.} At the
core of structural equivalence is alpha-equivalence which identifies
process that are the same up to a change of variable. Formally, we
recognize the distinction between free and bound names. The free names
of a process, $\freenames{P}$, may be calculated recursively as
follows:

\begin{mathpar}
\freenames{\pzero} := \emptyset
  \and \\
  \freenames{x?(y).P} := \{ x \} \cup (\freenames{P} \setminus \{ y \})
  \and 
  \freenames{x!\langle P \rangle} := \{ x \} \cup \{ P \} 
  \and \\
  \freenames{P|Q} := \freenames{P} \cup \freenames{Q}
  \and \\
  \freenames{@{x}} := \{ x \}
\end{mathpar}

$\pi$
$\quotep{\pi}$

$\freenames{-} : \pi \to \mathcal{P}(\quotep{\pi})$

\begin{eqnarray*}
  \freenames{\pzero} & := & \emptyset \\
  \freenames{x?(y).P} & := & \{ x \} \cup (\freenames{P} \setminus \{ y \}) \\
  \freenames{x!\langle P \rangle} & := & \{ x \} \cup \{ P \} \\
  \freenames{P|Q} & := & \freenames{P} \cup \freenames{Q} \\
  \freenames{\dropn{x}} & := & \{ x \}
\end{eqnarray*}

The bound names of a process, $\boundnames{P}$, are those names occurring in $P$
that are not free. For example, in $x?(y).0$, the name $x$ is free, while $y$ is bound.

\begin{mathpar}
  \inferrule* [lab=monoidal-laws] {} { P|Q \equiv Q|P \and P|0 \equiv P \and P|(Q|R) \equiv (P|Q)|R }
\end{mathpar}

\begin{mathpar}
  \inferrule* [lab=alpha-equivalence] {} { (x)P \equiv (y)P\{y/x\} \and y \not\in \freenames{P} }
\end{mathpar}

\begin{definition}
Then two processes, $P,Q$, are alpha-equivalent if $P = Q\{\vec{y}/\vec{x}\}$ for
some $\vec{x} \in \boundnames{Q},\vec{y} \in \boundnames{P}$, where $Q\{\vec{y}/\vec{x}\}$
denotes the capture-avoiding substitution of $\vec{y}$ for $\vec{x}$ in $Q$.
\end{definition}

\begin{definition}
  The {\em structural congruence} \cite{SangiorgiWalker} , $\equiv$,
  between processes is the least congruence containing
  alpha-equivalence, satisfying the abelian monoid laws
  (associativity, commutativity and $\pzero$ as identity) for parallel
  composition $|$ and for summation $+$.
\end{definition}

\subsection{Name equivalence}

We take name equivalence, written $\nameeq$, to be the smallest
equivalence relation generated by the following rules.

\begin{mathpar}
\inferrule*[lab=Quote-drop]
{ }
{ \quotep{@{x}} \nameeq x }

\inferrule*[lab=Struct-equiv]
{ P \scong Q }
{ \quotep{P} \nameeq \quotep{Q} }
\end{mathpar}

The astute reader will have noticed that the mutual recursion of names
and processes imposes a mutual recursion on alpha-equivalence and
structural equivalence via name-equivalence. Fortunately, all of this
works out pleasantly and we may calculate in the natural way, free of
concern. The reader interested in the details is referred to the
appendix \ref{appendix:rho_details}.

\subsection{Substitution}

We use $\Proc$ for the set of processes, $\QProc$ for the set of
names, and $\id{\{}\vec{y} / \vec{x} \id{\}}$ to denote partial maps,
$s : \QProc \rightarrow \QProc$. A map, $s$ lifts, uniquely, to a map
on process terms, $\widehat{s} : \Proc \rightarrow \Proc$ by the
following equations.

\begin{mathpar}
  (0) \psubstp{Q}{P} := 0 \\
  (R \juxtap S) \psubstp{Q}{P}
  :=    
  (R)\psubstp{Q}{P} \juxtap (S) \psubstp{Q}{P} \\
  (x?(y).R) \psubstp{Q}{P}    
  :=    
  (x)\substp{Q}{P} (z)\concat( (R \psubstn{z}{y}) \psubstp{Q}{P} ) \\
  (\lift{x}{R}) \psubstp{Q}{P}  
  :=
  \lift{(x)\substp{Q}{P}}{ R \psubstp{Q}{P} } \\
%   (\dropn{x})  \psubstp{Q}{P}       
%   := 
%   \left\{ 
%     \begin{array}{ccc} 
%       \dropn{\quotep{Q}} & & x \nameeq \quotep{P} \\
%       \dropn{x} & & otherwise \\
%     \end{array}
%   \right. 
  (\dropn{x})  \psubstp{Q}{P}       
  := 
  \left\{ 
    \begin{array}{ccc} 
      Q & & x \nameeq \quotep{P} \\
      \dropn{x} & & otherwise \\
    \end{array}
  \right.
\end{mathpar}
 

where

\begin{eqnarray}
  (x)\id{\{} \lpquote Q \rpquote / \lpquote P \rpquote \id{\}}            = 
  \left\{ 
    \begin{array}{ccc}
      \lpquote Q \rpquote & & x \nameeq \lpquote P \rpquote \\
      x & & otherwise \\
    \end{array}
  \right. \nonumber
\end{eqnarray}

and $z$ is chosen distinct from $\quotep{P}$, $\quotep{Q}$, the free
names in $Q$, and all the names in $R$. Our $\alpha$-equivalence will
be built in the standard way from this substitution.

\begin{remark}\label{rem:no_self_referential_names}
  One consequence of these definitions is that $\forall P. \quotep{P}
  \not\in \freenames{P}$.
\end{remark}

\subsection{ Dynamic quote: an example }

Anticipating something of what's to come, consider applying the
substitution, $\widehat{\id{\{}u / z \id{\}}}$, to the following pair
of processes, $\lift{w}{y!(z)}$ and $w[ \lpquote y!(z) \rpquote ]$.

\begin{eqnarray}
	\lift{w}{y!(z)}\widehat{\id{\{}u / z \id{\}}}
		& = &
		\lift{w}{y!(u)} \nonumber\\
	w[ \lpquote y!(z) \rpquote ] \widehat{ \id{\{}u / z \id{\}} }
		& = &
		w[ \lpquote y!(z) \rpquote ] \nonumber
\end{eqnarray}

Because the body of the process between quotes is impervious to
substitution, we get radically different answers. In fact, by
examining the first process in an input context,
e.g. $x?(z).\lift{w}{y!(z)}$, we see that the process under the lift
operator may be shaped by prefixed inputs binding a name inside it. In
this sense, the lift operator will be seen as a way to dynamically
construct processes before reifying them as names.

Finally equipped with these standard features we can present the
dynamics of the calculus.

\subsubsection{Operational semantics} 

Finally, we introduce the computational dynamics. What marks these
algebras as distinct from other more traditionally studied algebraic
structures, e.g. vector spaces or polynomial rings, is the manner in
which dynamics is captured. In traditional structures, dynamics is typically
expressed through morphisms between such structures, as in linear maps
between vector spaces or morphisms between rings. In algebras
associated with the semantics of computation, the dynamics is
expressed as part of the algebraic structure itself, through a
reduction reduction relation typically denoted by $\red$. Below, we
give a recursive presentation of this relation for the calculus used
in the encoding.

$\red \subseteq \pi \times \pi$
$\red : \pi \to \mathcal{P}(\pi)$

\begin{mathpar}
  \inferrule* [lab=Comm] { \textsf{match}( x_{src}, x_{trgt} ) } { x_{trgt}?(y)P \; | \; x_{src}!\langle {Q} \rangle \red P\{\quotep{Q}/y}\} }
  \and \\
  \inferrule* [lab=Par] {{P} \red {P}'} {{{P} | {Q}} \red {{P}' | {Q}}}
  \and
  \inferrule* [lab=Equiv]{{{P} \scong {P}'} \andalso {{P}' \red {Q}'} \andalso {{Q}' \scong {Q}}}{{P} \red {Q}}
\end{mathpar}

\begin{eqnarray*}
  match_{\equiv} (\quotep{P},\quotep{Q}) & := & P \equiv Q \\
  match_{\dagger}(\quotep{P},\quotep{Q}) & := & \forall R. P|Q \red^{*} R => R \red^{*} 0 \\
  match_{K}(\quotep{P},\quotep{Q}) & := & K \mbox{ for some context } K
\end{eqnarray*}

$u?(x)P | u!\langle Q \rangle \red P\{\quotep{Q}/x\}$

%We write $\wred$ for $\red^*$, and $P\red$ if $\exists Q $ such that $ P \red Q$.
We write $P\red$ if $\exists Q $ such that $ P \red Q$ and $P\not\red$, otherwise.

\section{Replication}

As mentioned before, it is known that replication (and hence
recursion) can be implemented in a higher-order process algebra
\cite{SangiorgiWalker}. As our first example of calculation with the
machinery thus far presented we give the construction explicitly in
the {\rhoc}.

\begin{eqnarray}
	D_{x} & := & \prefix{x}{y}{(\binpar{\outputp{x}{y}}{@{y}})} \nonumber\\
	\bangp_{x}{P} & := & \binpar{{x}!\langle{\binpar{D_{x}}{P}}\rangle}{D_{x}} \nonumber
\end{eqnarray}

\begin{eqnarray}
	\bangp_{x}{P} & & \nonumber\\
	=
	& {x}!\langle{(\prefix{x}{y}{(\outputp{x}{y} | @{y})) | P}}\rangle 
	      | \prefix{x}{y}{(\outputp{x}{y} | @{y})} & \nonumber\\
	\red
	& (\outputp{x}{y} | @{y})\substn{\quotep{(\prefix{x}{y}{(@{y} | \outputp{x}{y})) | P}}}{y} & \nonumber\\
	=
	& \outputp{x}{\quotep{(\prefix{x}{y}{(\outputp{x}{y} | @{y})) | P}}}
	  | {(\prefix{x}{y}{(\outputp{x}{y} | @{y})) | P}} & \nonumber\\
	\red
	& \ldots & \nonumber\\
	\red^*
	& P | P | \ldots & \nonumber
\end{eqnarray}

Of course, this encoding, as an implementation, runs away, unfolding
$\bangp{P}$ eagerly. A lazier and more implementable replication
operator, restricted to input-guarded processes, may be obtained as follows.

\begin{eqnarray}
\bangp{\prefix{u}{v}{P}} 
	:= 
	\binpar{\lift{x}{\prefix{u}{v}{(\binpar{D(x)}{P})}}}{D(x)} \nonumber
\end{eqnarray}

\begin{remark}
  Note that the lazier definition still does not deal with summation
  or mixed summation (i.e. sums over input and output). The reader is
  invited to construct definitions of replication that deal with these
  features. 

  Further, the definitions are parameterized in a name, $x$. Can you,
  gentle reader, make a definition that eliminates this parameter and
  guarantees no accidental interaction between the replication
  machinery and the process being replicated -- i.e. no accidental
  sharing of names used by the process to get its work done and the
  name(s) used by the replication to effect copying. This latter
  revision of the definition of replication is crucial to obtaining
  the expected identity $!!P \sim !P$.
\end{remark}

\begin{remark}\label{rem:paradoxical_combinator}
  The reader familiar with the lambda calculus will have noticed the
  similarity between $D$ and the paradoxical combinator.

  [Ed. note: the existence of this seems to suggest we have to be more
  restrictive on the set of processes and names we admit if we are to
  support no-cloning.]
\end{remark}

\subsubsection{Bisimulation}

The computational dynamics gives rise to another kind of equivalence,
the equivalence of computational behavior. As previously mentioned
this is typically captured \emph{via} some form of bisimulation.

% The notion we use in this paper is weak barbed bisimulation
% \cite{milner91polyadicpi}.

The notion we use in this paper is derived from weak barbed
bisimulation \cite{milner91polyadicpi}. 

\begin{definition}
An \emph{observation relation}, $\downarrow_{\mathcal N}$, over a set
of names, $\mathcal N$, is the smallest relation satisfying the rules
below.

\infrule[Out-barb]{y \in {\mathcal N}, \; x \nameeq y}
		  {\outputp{x}{v} \downarrow_{\mathcal N} x}
\infrule[Par-barb]{\mbox{$P\downarrow_{\mathcal N} x$ or $Q\downarrow_{\mathcal N} x$}}
		  {\binpar{P}{Q} \downarrow_{\mathcal N} x}

We write $P \Downarrow_{\mathcal N} x$ if there is $Q$ such that 
$P \wred Q$ and $Q \downarrow_{\mathcal N} x$.
\end{definition}

\begin{definition}
%\label{def.bbisim}
An  ${\mathcal N}$-\emph{barbed bisimulation} over a set of names, ${\mathcal N}$, is a symmetric binary relation 
${\mathcal S}_{\mathcal N}$ between agents such that $P\rel{S}_{\mathcal N}Q$ implies:
\begin{enumerate}
\item If $P \red P'$ then $Q \wred Q'$ and $P'\rel{S}_{\mathcal N} Q'$.
\item If $P\downarrow_{\mathcal N} x$, then $Q\Downarrow_{\mathcal N} x$.
\end{enumerate}
$P$ is ${\mathcal N}$-barbed bisimilar to $Q$, written
$P \wbbisim_{\mathcal N} Q$, if $P \rel{S}_{\mathcal N} Q$ for some ${\mathcal N}$-barbed bisimulation ${\mathcal S}_{\mathcal N}$.
\end{definition}

$\mathcal{R} \subseteq \pi \times \pi$

$P \mathcal{R} Q => \forall P'. P \red P' \Rightarrow \exists Q'. Q \red Q', P' \mathcal{R} Q'$

$P \vdash x \Rightarrow Q \vdash x$

\begin{mathpar}
  \inferrule*[lab=Out-barb]{x \nameeq y}{{y}!\langle{Q}\rangle \vdash x}
  \and
  \inferrule*[lab=Par-barb]{\mbox{$P\vdash x$ or $Q\vdash x$}}{\binpar{P}{Q} \vdash x}
\end{mathpar}

\subsubsection{Contexts}

One of the principle advantages of computational calculi like the
$\pi$-calculus is a well-defined notion of context,
contextual-equivalence and a correlation between
contextual-equivalence and notions of bisimulation. The notion of
context allows the decomposition of a process into (sub-)process and
its syntactic environment, its context. Thus, a context may be
thought of as a process with a ``hole'' (written $\Box$) in it. The
application of a context $M$ to a process $P$, written $M[P]$, is
tantamount to filling the hole in $M$ with $P$. In this paper we do
not need the full weight of this theory, but do make use of the notion
of context in the proof the main theorem. 

\begin{mathpar}
  \inferrule* [lab=summation] {} {{M_{M},M_{N}} \bc \Box \;|\; x.M_{A} \;|\; M_{M}+M_{N}}
  \and
  \inferrule* [lab=agent] {} {{M_{A}} \bc (\vec{x})M_{P} \;| \; \clift{P_0,\ldots,M_{P},\ldots,P_N}}
  \and \\
  \inferrule* [lab=process] {} {{M_{P}} \bc M_{N} \;| \;P|M_{P} }
\end{mathpar} 

\begin{mathpar}
  \inferrule* [lab=sychronization] {} {M_{N} \bc \Box \;|\; x?M_{F} \;|\; x!M_{C}}
  \and
  \inferrule* [lab=abstraction] {} {{M_{F}} \bc (x)M_{P} }
  \and
  \inferrule* [lab=concretion] {} {{M_{C}} \bc \langle M_{P} \rangle }
  \and \\
  \inferrule* [lab=process] {} {{M_{P}} \bc M_{N} \;| \;P|M_{P} }
\end{mathpar}

\begin{definition}[contextual application] Given a context $M$, and
  process $P$, we define the \emph{contextual application}, $M[P] :=
  M\{P/\Box\}$. That is, the contextual application of M to P is the
  substitution of $P$ for $\Box$ in $M$.
\end{definition}

$\meaningof{-} : L \to \mathcal{P}(\pi)$

\begin{mathpar}
  \inferrule* [lab=collection] {} {\meaningof{true} = \pi, \and \meaningof{~E} = \pi \setminus \meaningof{E}, \and \meaningof{E_{1} \& E_{2}} = \meaningof{E_{1}} \cap \meaningof{E_{2}}}
\end{mathpar}

\begin{mathpar}
  \inferrule* [lab=structure] {} {\meaningof{0} = \{ P \in \pi | P \equiv 0 \}, \and \\ \meaningof{E_1 | E_2} = \{ P \in \pi | P \equiv P_{1} | P_{2}, P_{1} \in \meaningof{E_{1}}, P_{2} \in \meaningof{E_2}\} }
\end{mathpar}

\begin{mathpar}
 \inferrule* [lab=behavior] {} {\meaningof{\langle a?b \rangle E} = \{ P \in \pi | P \equiv Q | u?(y)P', \\ \and \\\\ \and \\ \;\;\; u \in \meaningof{a}, \forall z.P'\{z/y\} \in \meaningof{E\{z/b\}}\}, \and \\ \meaningof{a!E} = \{ P \in \pi | P \equiv Q | x!\langle P' \rangle, x \in \meaningof{a} P' \in \meaningof{E}\} }
\end{mathpar}

\begin{mathpar}
 \inferrule* [lab=nominal] {} {\meaningof{\quotep{E}} = \{ \quotep{P} \in \quotep{\pi} | P \in \meaningof{E} \}, \and \meaningof{\quotep{P}} = \{ \quotep{Q} \in \quotep{\pi} | P \equiv Q \} \and \\ \meaningof{@\quotep{E}} = \{ P \in \pi | P \equiv @x, x \in \meaningof{E} \}}
\end{mathpar}

\begin{eqnarray*}
  \\
  \meaningof{-} : TS \to ST
\end{eqnarray*}

\begin{eqnarray*}
  \\
  L : TS \to ST
\end{eqnarray*}

\begin{eqnarray*}
  \\
  P \models E \iff P \in \meaningof{E}
\end{eqnarray*}

\begin{eqnarray*}
  P \approx_{L} Q \iff \forall E \in L. P \models E \iff Q \models E
\end{eqnarray*}

\begin{eqnarray*}
  P \approx_{K} Q
\end{eqnarray*}

\begin{eqnarray*}
  P \approx Q
\end{eqnarray*}

$\approx_{K} = \approx = \approx_{L}$

\subsubsection{Contextual duality}

Note that contexts extend the quotation operation to a family of
operations from processes to names. Given a context, $M$, we can
define a \emph{nominal context}, $\quotep{M}$ by $\quotep{M}[P] :=
\quotep{M[P]}$. To foreshadow what is to come we observe that these
operations enjoy a duality with processes very much like the duality
between vectors and maps from vectors to scalars.

Further, because the calculus is essentially higher-order, we have a
correspondence between contexts and processes. More specifically,
given a name $x$ and a context $M$ we can construct $M^{*}_{x}$ such
that 

\begin{mathpar}
  M^{*}_{x} | \lift{x}{P} \red M[P]
\end{mathpar}

namely,

\begin{mathpar}
  M^{*}_{x} := x?(u).M[\dropn{u}]
\end{mathpar}

The dependence of $M^{*}_{x}$ on a name makes it an abstraction, 

\begin{mathpar}
  M^{*} := (x)x?(u).M[\dropn{u}]
\end{mathpar}

\subsection{Additional notation}

It will sometimes be convenient to denote the process a name
quotes. We already have the notation $x = \quotep{P}$, but it will be
convenient to introduce an alternate notation, $\procn{x}$, when we
want to emphasize the connection to the use of the name. Note that, by
virtue of name equivalence, $\quotep{\procn{x}} \nameeq x$; so, the
notation is consistent with previous definitions.

Further, because names have structure it is possible to effect
substitutions on the basis of that structure. This means we need to
upgrade our notation for substitutions, which we accomplish by
adapting comprehension notation. Thus,

\begin{mathpar}
  P\{ y / x : x \in S \}
\end{mathpar}

is interpreted to mean the process derived from P by replacing (in a
capture-avoiding manner) each occurrence of $x$ in $S$ by $y$. For example,

\begin{mathpar}
  P\{ \quotep{\procn{x}|\procn{x}} / x : x \in \freenames{P} \}
\end{mathpar}

will replace each (occurrence) of a free name $x$ in $P$ by
$\quotep{\procn{x}|\procn{x}}$.

Also, we will avail ourselves of the notation $x^{L}$ and $x^{R}$ to
denote injections of a name into disjoint copies of the name
space. There are numerous ways to accomplish this. One example can be
found in \cite{MeredithR05}. This notation overloads to vectors of
names: $\vec{x}^{\pi} := (x_{i}^{\pi} \; : \; 0 \leq i < |\vec{x}| )$ where $\pi \in \{L,R\}$.

We also use $P^{\Box} := P|\Box$.

In \cite{MeredithR05} an interpretation of the new operator is
given. It turns out that there are several possible interpretations
all enjoying the requisite algebraic properties of the operator (see
\cite{milner91polyadicpi}). We will therefore make liberal use of
$(\nu\; \vec{x})P$.

% subsection the_syntax_and_semantics_of_the_notation_system (end)   

\input{qm2pi.qmops} 

\input{qm2pi.sterngerlach} 

\input{qm2pi.metric} 

% section concurrent_process_calculi (end)

%\input{qm2pi.proofsketch}

% section proof sketch (end)

%\input{qm2pi.slviaknots} 

% section spatial logic via knots (end)

\input{qm2pi.conclusion}

% section conclusion (end)

%\input{qm2pi.dtcodes} 

% section wiring algorithm (end)

\input{qm2pi.ack} 

% section acknowledgments (end)

\newpage


\bibliographystyle{plain}   
\bibliography{../../biblios/main.bib}

\input{qm2pi.rhodetails}

\end{document}

 

% subsection basic_interpretation (end)

%\input{qm2pi.rho.presentation} 
\subsection{The syntax and semantics of the notation system}\label{sub:the_syntax_and_semantics_of_the_notation_system} % (fold)

We now summarize a technical presentation of the calculus that
embodies our theory of dynamics. The typical presentation of such a
calculus follows the style of giving generators and relations on
them. The grammar, below, describing term constructors, freely
generates the set of processes, $\Proc$. This set is then quotiented
by a relation known as structural congruence and it is over this set
that the notion of dynamics is expressed. This presentation is
essentially that of \cite{MeredithR05} with the addition of
polyadicity and summation. For readability we have relegated some of
the technical subtleties to an appendix.

\subsubsection{Process grammar}\label{subsub:process_grammar}

\begin{mathpar}
  \inferrule* [lab=synchronization] {} {{M} \bc \pzero \;|\; x?F \;|\; x!C }
  \and
  \inferrule* [lab=abstraction] {} {{F} \bc (x)P}
  \and
  \inferrule* [lab=concretion] {} {{C} \bc \langle Q \rangle}
  \and
  \inferrule* [lab=process] {} {{P,Q} \bc M \;| \;P|Q \;|\; @{x}}
  \and
  \inferrule* [lab=name] {} {{x} \bc \quotep{P}}
\end{mathpar} 

Note that $\vec{x}$ (resp. $\vec{P}$) denotes a vector of names
(resp. processes) of length $|\vec{x}|$ (resp. $|\vec{P}|$). We adopt
the following useful abbreviations.

\begin{mathpar}
   x?(\vec{y}).P := x.(\vec{y})P \and  x\clift{\vec{P}} := x.\clift{\vec{P}}
   \and x!(y) := \lift{x}{\dropn{y}}
   \and \Pi_{i=0}^{n-1}P_i := P_0 | \ldots | P_{n-1}
\end{mathpar}

\subsubsection{Structural congruence}

\paragraph{Free and bound names and alpha-equivalence.} At the
core of structural equivalence is alpha-equivalence which identifies
process that are the same up to a change of variable. Formally, we
recognize the distinction between free and bound names. The free names
of a process, $\freenames{P}$, may be calculated recursively as
follows:

\begin{mathpar}
\freenames{\pzero} := \emptyset
  \and \\
  \freenames{x?(y).P} := \{ x \} \cup (\freenames{P} \setminus \{ y \})
  \and 
  \freenames{x!\langle P \rangle} := \{ x \} \cup \{ P \} 
  \and \\
  \freenames{P|Q} := \freenames{P} \cup \freenames{Q}
  \and \\
  \freenames{@{x}} := \{ x \}
\end{mathpar}

$\pi$
$\quotep{\pi}$

$\freenames{-} : \pi \to \mathcal{P}(\quotep{\pi})$

\begin{eqnarray*}
  \freenames{\pzero} & := & \emptyset \\
  \freenames{x?(y).P} & := & \{ x \} \cup (\freenames{P} \setminus \{ y \}) \\
  \freenames{x!\langle P \rangle} & := & \{ x \} \cup \{ P \} \\
  \freenames{P|Q} & := & \freenames{P} \cup \freenames{Q} \\
  \freenames{\dropn{x}} & := & \{ x \}
\end{eqnarray*}

The bound names of a process, $\boundnames{P}$, are those names occurring in $P$
that are not free. For example, in $x?(y).0$, the name $x$ is free, while $y$ is bound.

\begin{mathpar}
  \inferrule* [lab=monoidal-laws] {} { P|Q \equiv Q|P \and P|0 \equiv P \and P|(Q|R) \equiv (P|Q)|R }
\end{mathpar}

\begin{mathpar}
  \inferrule* [lab=alpha-equivalence] {} { (x)P \equiv (y)P\{y/x\} \and y \not\in \freenames{P} }
\end{mathpar}

\begin{definition}
Then two processes, $P,Q$, are alpha-equivalent if $P = Q\{\vec{y}/\vec{x}\}$ for
some $\vec{x} \in \boundnames{Q},\vec{y} \in \boundnames{P}$, where $Q\{\vec{y}/\vec{x}\}$
denotes the capture-avoiding substitution of $\vec{y}$ for $\vec{x}$ in $Q$.
\end{definition}

\begin{definition}
  The {\em structural congruence} \cite{SangiorgiWalker} , $\equiv$,
  between processes is the least congruence containing
  alpha-equivalence, satisfying the abelian monoid laws
  (associativity, commutativity and $\pzero$ as identity) for parallel
  composition $|$ and for summation $+$.
\end{definition}

\subsection{Name equivalence}

We take name equivalence, written $\nameeq$, to be the smallest
equivalence relation generated by the following rules.

\begin{mathpar}
\inferrule*[lab=Quote-drop]
{ }
{ \quotep{@{x}} \nameeq x }

\inferrule*[lab=Struct-equiv]
{ P \scong Q }
{ \quotep{P} \nameeq \quotep{Q} }
\end{mathpar}

The astute reader will have noticed that the mutual recursion of names
and processes imposes a mutual recursion on alpha-equivalence and
structural equivalence via name-equivalence. Fortunately, all of this
works out pleasantly and we may calculate in the natural way, free of
concern. The reader interested in the details is referred to the
appendix \ref{appendix:rho_details}.

\subsection{Substitution}

We use $\Proc$ for the set of processes, $\QProc$ for the set of
names, and $\id{\{}\vec{y} / \vec{x} \id{\}}$ to denote partial maps,
$s : \QProc \rightarrow \QProc$. A map, $s$ lifts, uniquely, to a map
on process terms, $\widehat{s} : \Proc \rightarrow \Proc$ by the
following equations.

\begin{mathpar}
  (0) \psubstp{Q}{P} := 0 \\
  (R \juxtap S) \psubstp{Q}{P}
  :=    
  (R)\psubstp{Q}{P} \juxtap (S) \psubstp{Q}{P} \\
  (x?(y).R) \psubstp{Q}{P}    
  :=    
  (x)\substp{Q}{P} (z)\concat( (R \psubstn{z}{y}) \psubstp{Q}{P} ) \\
  (\lift{x}{R}) \psubstp{Q}{P}  
  :=
  \lift{(x)\substp{Q}{P}}{ R \psubstp{Q}{P} } \\
%   (\dropn{x})  \psubstp{Q}{P}       
%   := 
%   \left\{ 
%     \begin{array}{ccc} 
%       \dropn{\quotep{Q}} & & x \nameeq \quotep{P} \\
%       \dropn{x} & & otherwise \\
%     \end{array}
%   \right. 
  (\dropn{x})  \psubstp{Q}{P}       
  := 
  \left\{ 
    \begin{array}{ccc} 
      Q & & x \nameeq \quotep{P} \\
      \dropn{x} & & otherwise \\
    \end{array}
  \right.
\end{mathpar}
 

where

\begin{eqnarray}
  (x)\id{\{} \lpquote Q \rpquote / \lpquote P \rpquote \id{\}}            = 
  \left\{ 
    \begin{array}{ccc}
      \lpquote Q \rpquote & & x \nameeq \lpquote P \rpquote \\
      x & & otherwise \\
    \end{array}
  \right. \nonumber
\end{eqnarray}

and $z$ is chosen distinct from $\quotep{P}$, $\quotep{Q}$, the free
names in $Q$, and all the names in $R$. Our $\alpha$-equivalence will
be built in the standard way from this substitution.

\begin{remark}\label{rem:no_self_referential_names}
  One consequence of these definitions is that $\forall P. \quotep{P}
  \not\in \freenames{P}$.
\end{remark}

\subsection{ Dynamic quote: an example }

Anticipating something of what's to come, consider applying the
substitution, $\widehat{\id{\{}u / z \id{\}}}$, to the following pair
of processes, $\lift{w}{y!(z)}$ and $w[ \lpquote y!(z) \rpquote ]$.

\begin{eqnarray}
	\lift{w}{y!(z)}\widehat{\id{\{}u / z \id{\}}}
		& = &
		\lift{w}{y!(u)} \nonumber\\
	w[ \lpquote y!(z) \rpquote ] \widehat{ \id{\{}u / z \id{\}} }
		& = &
		w[ \lpquote y!(z) \rpquote ] \nonumber
\end{eqnarray}

Because the body of the process between quotes is impervious to
substitution, we get radically different answers. In fact, by
examining the first process in an input context,
e.g. $x?(z).\lift{w}{y!(z)}$, we see that the process under the lift
operator may be shaped by prefixed inputs binding a name inside it. In
this sense, the lift operator will be seen as a way to dynamically
construct processes before reifying them as names.

Finally equipped with these standard features we can present the
dynamics of the calculus.

\subsubsection{Operational semantics} 

Finally, we introduce the computational dynamics. What marks these
algebras as distinct from other more traditionally studied algebraic
structures, e.g. vector spaces or polynomial rings, is the manner in
which dynamics is captured. In traditional structures, dynamics is typically
expressed through morphisms between such structures, as in linear maps
between vector spaces or morphisms between rings. In algebras
associated with the semantics of computation, the dynamics is
expressed as part of the algebraic structure itself, through a
reduction reduction relation typically denoted by $\red$. Below, we
give a recursive presentation of this relation for the calculus used
in the encoding.

$\red \subseteq \pi \times \pi$
$\red : \pi \to \mathcal{P}(\pi)$

\begin{mathpar}
  \inferrule* [lab=Comm] { \textsf{match}( x_{src}, x_{trgt} ) } { x_{trgt}?(y)P \; | \; x_{src}!\langle {Q} \rangle \red P\{\quotep{Q}/y}\} }
  \and \\
  \inferrule* [lab=Par] {{P} \red {P}'} {{{P} | {Q}} \red {{P}' | {Q}}}
  \and
  \inferrule* [lab=Equiv]{{{P} \scong {P}'} \andalso {{P}' \red {Q}'} \andalso {{Q}' \scong {Q}}}{{P} \red {Q}}
\end{mathpar}

\begin{eqnarray*}
  match_{\equiv} (\quotep{P},\quotep{Q}) & := & P \equiv Q \\
  match_{\dagger}(\quotep{P},\quotep{Q}) & := & \forall R. P|Q \red^{*} R => R \red^{*} 0 \\
  match_{K}(\quotep{P},\quotep{Q}) & := & K \mbox{ for some context } K
\end{eqnarray*}

$u?(x)P | u!\langle Q \rangle \red P\{\quotep{Q}/x\}$

%We write $\wred$ for $\red^*$, and $P\red$ if $\exists Q $ such that $ P \red Q$.
We write $P\red$ if $\exists Q $ such that $ P \red Q$ and $P\not\red$, otherwise.

\section{Replication}

As mentioned before, it is known that replication (and hence
recursion) can be implemented in a higher-order process algebra
\cite{SangiorgiWalker}. As our first example of calculation with the
machinery thus far presented we give the construction explicitly in
the {\rhoc}.

\begin{eqnarray}
	D_{x} & := & \prefix{x}{y}{(\binpar{\outputp{x}{y}}{@{y}})} \nonumber\\
	\bangp_{x}{P} & := & \binpar{{x}!\langle{\binpar{D_{x}}{P}}\rangle}{D_{x}} \nonumber
\end{eqnarray}

\begin{eqnarray}
	\bangp_{x}{P} & & \nonumber\\
	=
	& {x}!\langle{(\prefix{x}{y}{(\outputp{x}{y} | @{y})) | P}}\rangle 
	      | \prefix{x}{y}{(\outputp{x}{y} | @{y})} & \nonumber\\
	\red
	& (\outputp{x}{y} | @{y})\substn{\quotep{(\prefix{x}{y}{(@{y} | \outputp{x}{y})) | P}}}{y} & \nonumber\\
	=
	& \outputp{x}{\quotep{(\prefix{x}{y}{(\outputp{x}{y} | @{y})) | P}}}
	  | {(\prefix{x}{y}{(\outputp{x}{y} | @{y})) | P}} & \nonumber\\
	\red
	& \ldots & \nonumber\\
	\red^*
	& P | P | \ldots & \nonumber
\end{eqnarray}

Of course, this encoding, as an implementation, runs away, unfolding
$\bangp{P}$ eagerly. A lazier and more implementable replication
operator, restricted to input-guarded processes, may be obtained as follows.

\begin{eqnarray}
\bangp{\prefix{u}{v}{P}} 
	:= 
	\binpar{\lift{x}{\prefix{u}{v}{(\binpar{D(x)}{P})}}}{D(x)} \nonumber
\end{eqnarray}

\begin{remark}
  Note that the lazier definition still does not deal with summation
  or mixed summation (i.e. sums over input and output). The reader is
  invited to construct definitions of replication that deal with these
  features. 

  Further, the definitions are parameterized in a name, $x$. Can you,
  gentle reader, make a definition that eliminates this parameter and
  guarantees no accidental interaction between the replication
  machinery and the process being replicated -- i.e. no accidental
  sharing of names used by the process to get its work done and the
  name(s) used by the replication to effect copying. This latter
  revision of the definition of replication is crucial to obtaining
  the expected identity $!!P \sim !P$.
\end{remark}

\begin{remark}\label{rem:paradoxical_combinator}
  The reader familiar with the lambda calculus will have noticed the
  similarity between $D$ and the paradoxical combinator.

  [Ed. note: the existence of this seems to suggest we have to be more
  restrictive on the set of processes and names we admit if we are to
  support no-cloning.]
\end{remark}

\subsubsection{Bisimulation}

The computational dynamics gives rise to another kind of equivalence,
the equivalence of computational behavior. As previously mentioned
this is typically captured \emph{via} some form of bisimulation.

% The notion we use in this paper is weak barbed bisimulation
% \cite{milner91polyadicpi}.

The notion we use in this paper is derived from weak barbed
bisimulation \cite{milner91polyadicpi}. 

\begin{definition}
An \emph{observation relation}, $\downarrow_{\mathcal N}$, over a set
of names, $\mathcal N$, is the smallest relation satisfying the rules
below.

\infrule[Out-barb]{y \in {\mathcal N}, \; x \nameeq y}
		  {\outputp{x}{v} \downarrow_{\mathcal N} x}
\infrule[Par-barb]{\mbox{$P\downarrow_{\mathcal N} x$ or $Q\downarrow_{\mathcal N} x$}}
		  {\binpar{P}{Q} \downarrow_{\mathcal N} x}

We write $P \Downarrow_{\mathcal N} x$ if there is $Q$ such that 
$P \wred Q$ and $Q \downarrow_{\mathcal N} x$.
\end{definition}

\begin{definition}
%\label{def.bbisim}
An  ${\mathcal N}$-\emph{barbed bisimulation} over a set of names, ${\mathcal N}$, is a symmetric binary relation 
${\mathcal S}_{\mathcal N}$ between agents such that $P\rel{S}_{\mathcal N}Q$ implies:
\begin{enumerate}
\item If $P \red P'$ then $Q \wred Q'$ and $P'\rel{S}_{\mathcal N} Q'$.
\item If $P\downarrow_{\mathcal N} x$, then $Q\Downarrow_{\mathcal N} x$.
\end{enumerate}
$P$ is ${\mathcal N}$-barbed bisimilar to $Q$, written
$P \wbbisim_{\mathcal N} Q$, if $P \rel{S}_{\mathcal N} Q$ for some ${\mathcal N}$-barbed bisimulation ${\mathcal S}_{\mathcal N}$.
\end{definition}

$\mathcal{R} \subseteq \pi \times \pi$

$P \mathcal{R} Q => \forall P'. P \red P' \Rightarrow \exists Q'. Q \red Q', P' \mathcal{R} Q'$

$P \vdash x \Rightarrow Q \vdash x$

\begin{mathpar}
  \inferrule*[lab=Out-barb]{x \nameeq y}{{y}!\langle{Q}\rangle \vdash x}
  \and
  \inferrule*[lab=Par-barb]{\mbox{$P\vdash x$ or $Q\vdash x$}}{\binpar{P}{Q} \vdash x}
\end{mathpar}

\subsubsection{Contexts}

One of the principle advantages of computational calculi like the
$\pi$-calculus is a well-defined notion of context,
contextual-equivalence and a correlation between
contextual-equivalence and notions of bisimulation. The notion of
context allows the decomposition of a process into (sub-)process and
its syntactic environment, its context. Thus, a context may be
thought of as a process with a ``hole'' (written $\Box$) in it. The
application of a context $M$ to a process $P$, written $M[P]$, is
tantamount to filling the hole in $M$ with $P$. In this paper we do
not need the full weight of this theory, but do make use of the notion
of context in the proof the main theorem. 

\begin{mathpar}
  \inferrule* [lab=summation] {} {{M_{M},M_{N}} \bc \Box \;|\; x.M_{A} \;|\; M_{M}+M_{N}}
  \and
  \inferrule* [lab=agent] {} {{M_{A}} \bc (\vec{x})M_{P} \;| \; \clift{P_0,\ldots,M_{P},\ldots,P_N}}
  \and \\
  \inferrule* [lab=process] {} {{M_{P}} \bc M_{N} \;| \;P|M_{P} }
\end{mathpar} 

\begin{mathpar}
  \inferrule* [lab=sychronization] {} {M_{N} \bc \Box \;|\; x?M_{F} \;|\; x!M_{C}}
  \and
  \inferrule* [lab=abstraction] {} {{M_{F}} \bc (x)M_{P} }
  \and
  \inferrule* [lab=concretion] {} {{M_{C}} \bc \langle M_{P} \rangle }
  \and \\
  \inferrule* [lab=process] {} {{M_{P}} \bc M_{N} \;| \;P|M_{P} }
\end{mathpar}

\begin{definition}[contextual application] Given a context $M$, and
  process $P$, we define the \emph{contextual application}, $M[P] :=
  M\{P/\Box\}$. That is, the contextual application of M to P is the
  substitution of $P$ for $\Box$ in $M$.
\end{definition}

$\meaningof{-} : L \to \mathcal{P}(\pi)$

\begin{mathpar}
  \inferrule* [lab=collection] {} {\meaningof{true} = \pi, \and \meaningof{~E} = \pi \setminus \meaningof{E}, \and \meaningof{E_{1} \& E_{2}} = \meaningof{E_{1}} \cap \meaningof{E_{2}}}
\end{mathpar}

\begin{mathpar}
  \inferrule* [lab=structure] {} {\meaningof{0} = \{ P \in \pi | P \equiv 0 \}, \and \\ \meaningof{E_1 | E_2} = \{ P \in \pi | P \equiv P_{1} | P_{2}, P_{1} \in \meaningof{E_{1}}, P_{2} \in \meaningof{E_2}\} }
\end{mathpar}

\begin{mathpar}
 \inferrule* [lab=behavior] {} {\meaningof{\langle a?b \rangle E} = \{ P \in \pi | P \equiv Q | u?(y)P', \\ \and \\\\ \and \\ \;\;\; u \in \meaningof{a}, \forall z.P'\{z/y\} \in \meaningof{E\{z/b\}}\}, \and \\ \meaningof{a!E} = \{ P \in \pi | P \equiv Q | x!\langle P' \rangle, x \in \meaningof{a} P' \in \meaningof{E}\} }
\end{mathpar}

\begin{mathpar}
 \inferrule* [lab=nominal] {} {\meaningof{\quotep{E}} = \{ \quotep{P} \in \quotep{\pi} | P \in \meaningof{E} \}, \and \meaningof{\quotep{P}} = \{ \quotep{Q} \in \quotep{\pi} | P \equiv Q \} \and \\ \meaningof{@\quotep{E}} = \{ P \in \pi | P \equiv @x, x \in \meaningof{E} \}}
\end{mathpar}

\begin{eqnarray*}
  \\
  \meaningof{-} : TS \to ST
\end{eqnarray*}

\begin{eqnarray*}
  \\
  L : TS \to ST
\end{eqnarray*}

\begin{eqnarray*}
  \\
  P \models E \iff P \in \meaningof{E}
\end{eqnarray*}

\begin{eqnarray*}
  P \approx_{L} Q \iff \forall E \in L. P \models E \iff Q \models E
\end{eqnarray*}

\begin{eqnarray*}
  P \approx_{K} Q
\end{eqnarray*}

\begin{eqnarray*}
  P \approx Q
\end{eqnarray*}

$\approx_{K} = \approx = \approx_{L}$

\subsubsection{Contextual duality}

Note that contexts extend the quotation operation to a family of
operations from processes to names. Given a context, $M$, we can
define a \emph{nominal context}, $\quotep{M}$ by $\quotep{M}[P] :=
\quotep{M[P]}$. To foreshadow what is to come we observe that these
operations enjoy a duality with processes very much like the duality
between vectors and maps from vectors to scalars.

Further, because the calculus is essentially higher-order, we have a
correspondence between contexts and processes. More specifically,
given a name $x$ and a context $M$ we can construct $M^{*}_{x}$ such
that 

\begin{mathpar}
  M^{*}_{x} | \lift{x}{P} \red M[P]
\end{mathpar}

namely,

\begin{mathpar}
  M^{*}_{x} := x?(u).M[\dropn{u}]
\end{mathpar}

The dependence of $M^{*}_{x}$ on a name makes it an abstraction, 

\begin{mathpar}
  M^{*} := (x)x?(u).M[\dropn{u}]
\end{mathpar}

\subsection{Additional notation}

It will sometimes be convenient to denote the process a name
quotes. We already have the notation $x = \quotep{P}$, but it will be
convenient to introduce an alternate notation, $\procn{x}$, when we
want to emphasize the connection to the use of the name. Note that, by
virtue of name equivalence, $\quotep{\procn{x}} \nameeq x$; so, the
notation is consistent with previous definitions.

Further, because names have structure it is possible to effect
substitutions on the basis of that structure. This means we need to
upgrade our notation for substitutions, which we accomplish by
adapting comprehension notation. Thus,

\begin{mathpar}
  P\{ y / x : x \in S \}
\end{mathpar}

is interpreted to mean the process derived from P by replacing (in a
capture-avoiding manner) each occurrence of $x$ in $S$ by $y$. For example,

\begin{mathpar}
  P\{ \quotep{\procn{x}|\procn{x}} / x : x \in \freenames{P} \}
\end{mathpar}

will replace each (occurrence) of a free name $x$ in $P$ by
$\quotep{\procn{x}|\procn{x}}$.

Also, we will avail ourselves of the notation $x^{L}$ and $x^{R}$ to
denote injections of a name into disjoint copies of the name
space. There are numerous ways to accomplish this. One example can be
found in \cite{MeredithR05}. This notation overloads to vectors of
names: $\vec{x}^{\pi} := (x_{i}^{\pi} \; : \; 0 \leq i < |\vec{x}| )$ where $\pi \in \{L,R\}$.

We also use $P^{\Box} := P|\Box$.

In \cite{MeredithR05} an interpretation of the new operator is
given. It turns out that there are several possible interpretations
all enjoying the requisite algebraic properties of the operator (see
\cite{milner91polyadicpi}). We will therefore make liberal use of
$(\nu\; \vec{x})P$.

% subsection the_syntax_and_semantics_of_the_notation_system (end)   

\section{Interpretation of QM}
\subsection{Supporting definitions}
\subsubsection{Multiplication}
\begin{mathpar}
  \quotep{Q} \cdot \quotep{R} := \quotep{Q|R}
  \and \\
  \quotep{Q} \cdot P := P\{ \quotep{Q|R} / \quotep{R} : \quotep{R} \in \freenames{P} \}
\end{mathpar}

\paragraph{Discussion}
The first line needs little explanation. The second line says that
each free name of the process is replaced with the multiplication of
that name by the scalar. Multiplication of a scalar (name) by a state
(process) results in a process all the names of which have been `moved
over' by parallel composition with the process the scalar
quotes. There is a subtlety that the bound names have to be
manipulated so that multiplied names aren't accidentally
captured. There are many ways to achieve this.

\begin{remark}\label{rem:multiplication_identities}
  The reader is invited to verify that for all $x,y,z \in \QProc$ and $P \in \Proc$
  \begin{mathpar}
    x \cdot \quotep{0} \equiv x 
    \and
    x \cdot y \equiv y \cdot x
    \and
    x \cdot (y \cdot z) \equiv (x \cdot y) \cdot z
    \and \\
    \quotep{0} \cdot P \equiv P
    \and \\
    x \cdot (y \cdot P) \equiv (x \cdot y) \cdot P
    \and \\
    x \cdot (P|Q) \equiv (x \cdot P) | (x \cdot Q)
    \and \\    
  \end{mathpar}
\end{remark}

\subsubsection{Tensor product}

We define a tensor product on processes by structural induction.

\paragraph{Tensor of sums} First note that all summations, including
$\pzero$ and sequence, can be written $\Sigma_{i} x_{i}.A_{i} +
\Sigma_{j} x_{j}.C_{j}$, where we have grouped input-guarded processes
together and output-guarded processes together.

Thus, we can define the tensor product of two summations, $N_{1}\otimes N_{2}$, where

\begin{mathpar}
  N_{1} := \Sigma_{i} x_{i}.A_{i} + \Sigma_{j} x_{j}.C_{j}
  \and
  N_{2} := \Sigma_{i'} y_{i'}.B_{i'} + \Sigma_{j'} y_{j'}.D_{j'} 
\end{mathpar}

as follows.

\begin{mathpar}
  \Sigma_{i} x_{i}.A_{i} + \Sigma_{j} x_{j}.C_{j} \otimes \Sigma_{i'}
  y_{i'}.B_{i'} + \Sigma_{j'} y_{j'}.D_{j'} 
  \and \\
  := \; \Sigma_{i} \Sigma_{i'} \quotep{\stackrel{\vee}{x_{i}}| \stackrel{\vee}{y_{i'}}}.(A_{i}\otimes B_{i'}) \; | \; \Sigma_{i'} \Sigma_{i} \quotep{\stackrel{\vee}{y_{i'}}|\stackrel{\vee}{x_{i}}}.(B_{i'}\otimes A_{i})
  \and
  \;\; | \;\; \Sigma_{j} \Sigma_{j'} \quotep{\stackrel{\vee}{x_{j}}|\stackrel{\vee}{y_{j'}}}.(A_{j}\otimes B_{j'}) \; | \; \Sigma_{j'} \Sigma_{j} \quotep{\stackrel{\vee}{y_{j'}}|\stackrel{\vee}{x_{j}}}.(B_{j'}\otimes A_{j})
\end{mathpar}

\begin{remark}
  Do we need to $x^{L}$ and $y^{R}$ for this construction as well?
\end{remark}

\paragraph{Tensor of parallel compositions} Next, we distribute tensor
over par.

\begin{mathpar}
  P_{1}|P_{2} \otimes Q_{1}|Q_{2} := (P_{1} \otimes Q_{1}) | (P_{1}
  \otimes Q_{2}) | (P_{2} \otimes Q_{1}) | (P_{2} \otimes Q_{2})
\end{mathpar}

\paragraph{Tensor with dropped names} We treat tensor of a
process with a dropped name as parallel composition.

\begin{mathpar}
  P \otimes \dropn{x} := P | \dropn{x}
\end{mathpar}

\paragraph{Tensor of agents}

Finally, we need to define tensor on agents. Note that the definition
of tensor on normal products only tensors inputs with inputs and
outputs with outputs. Thus, we only have to define the operation on
``homogeneous'' pairings.

\begin{mathpar}
  (\vec{x})P \otimes (\vec{y})Q
  \and \\
  := (x_{0}^{L}|y_{0}^{R},\ldots,x_{0}^{L}|y_{n}^{R},\ldots,x_{m}^{L}|y_{0}^{R},\ldots,x_{m}^{L}|y_{n}^R)(P\{ \vec{x}^{L}/\vec{x}\} \otimes Q \{ \vec{y}^{R}/\vec{y}\})
  \and \\
  \clift{\vec{P}} \otimes \clift{\vec{Q}}
  \and \\
  := \clift{P_{0}\otimes Q_{0},\ldots,P_{0}\otimes Q_{n},\ldots,P_{m}\otimes Q_{0},\ldots,P_{m}\otimes Q_{n}}
\end{mathpar}

\begin{remark}
  Observe that arities of tensored abstractions matches arities of
  tensored concretions if the original arities matched. Note also that
  the length of the arities corresponds to the increase in dimension
  we see in ordinary vector space tensor product.
\end{remark}

\begin{remark}
  Operationally, this definition distributes the tensor down to
  components ``linked'' by summation. Tensor over summation is
  intriguing in that it mixes names. Moreover, as a consequence of the
  way it mixes names we have the identities for all $x \in \QProc$ and
  $P,Q \in \Proc$

  \begin{mathpar}
    (x \cdot P) \otimes Q \equiv x \cdot (P \otimes Q) \equiv P \otimes (x \cdot Q)
    \and
    P \otimes \pzero \equiv P
  \end{mathpar}

  that the reader is invited to verify.
\end{remark}

\subsubsection{Annihilation}
\begin{mathpar}
  P^{\perp} := \{ Q | \forall R. P|Q \red^{*} R \Rightarrow R \red^{*} \pzero \}
  \and \\
  P^{\underline{\perp}} := \Sigma_{Q \in P^{\perp}} \quotep{Q}?(y).(\dropn{y}|Q) | \Sigma_{Q \in P^{\perp}} \quotep{Q}\clift{\Box}
\end{mathpar}

\paragraph{Discussion} The reader will note that $P^{\perp}$ is a
\emph{set} of processes, while $P^{\underline{\perp}}$ is a
\emph{context}. We call the set $P^{\perp}$ the \emph{annihilators} of
$P$. The parallel composition of a process in the annihilators of $P$
with $P$ will result in a process, the state space of which has all
paths eventually leading to $\pzero$. Execution may endure loops; but
under reasonable conditions of fairness (naturally guaranteed under
most notions of bisimulation) such a composite process cannot get
stuck in such a loop and will, eventually pop out and terminate.

The context $P^{\underline{\perp}}$ is ready and willing to ``take the
$P$ out of'' the process to which it is applied. It will effectively
transmit the code of the process to which it is applied to one of the
annihilators and run the process against it.

\subsubsection{Evaluation}
We fix $M$ a domain of fully abstract interpretation with an equality
coincident with bisimulation. We take $\meaningof{\cdot} : \Proc \to
M$ to be the map interpreting processes and $\nmeaningof{\cdot} : \M
\to Proc$ to be the map running the other way. Then we define

\begin{mathpar}
  \int P := \nmeaningof{\meaningof{P}}
\end{mathpar}

\paragraph{Discussion}
There are many fully abstract interpretations of Milner's
$\pi$-calculus. Any of them can be used as a basis for interpreting
the reflective calculus here. Equipped with such a domain it is
largely a matter of grinding through to check that the Yoneda
construction for the normalization-by-evaluation program can be
extended to this setting.

\begin{remark}
  The reader is invited to verify that $\int (P^{\underline{\perp}}[P]) = 0$.
\end{remark}

\subsection{Quantum mechanics}

Table \ref{tbl:core_qm_op_defns} gives the core operational definitions

\begin{table}[htp]\label{tbl:core_qm_op_defns}
  \center{
    \fbox{
      \begin{tabular}{c|c}
        quantum mechanics & process calculus \\
        \hline
        scalar & $x := \quotep{P}$ \\
        state vector & $\state{P} := P$ \\
        dual & $\state{P}^{*} := \event{P^{\underline{\perp}}} := \quotep{P^{\underline{\perp}}}[-]$ \\
        matrix & $ \Sigma_{\alpha} \state{P_{\alpha}}x_{\alpha}\event{Q_{\alpha}}$ \\
        vector addition & $\state{P} + \state{Q} := \state{P | Q}$ \\
        tensor product & $\state{P} \otimes \state{Q} := \state{P \otimes Q}$ \\
        inner product & $\innerprod{P}{Q} := \quotep{\int P^{\underline{\perp}}[Q]}$ \\
      \end{tabular}
    }
  }
  \caption{QM - operational definitions}
\end{table}

where

\begin{mathpar}
  \prmatrix{P}{Q} := \fprmatrix{P}{\quotep{\pzero}}{Q}
  \and
  \fprmatrix{P}{x}{Q} := (\state{P},x,\event{Q})
  \and
  (\fprmatrix{P}{x}{Q})(\state{R}) := x \cdot \innerprod{Q}{R} \cdot \state{P}
  \and
  (\fprmatrix{P}{x}{Q})(\event{R}) := x \cdot \innerprod{R}{P} \cdot \event{Q}
\end{mathpar}

\paragraph{Discussion}
As promised: vectors (aka states) are represented as processes; duals
as contextual duals; inner product definition should be compared with
standard inner product definition for ....

\begin{remark}
  Assuming $\int (P^{\underline{\perp}}[P]) = 0$, the reader is
  invited to verify that $(\fprmatrix{P}{x}{P})(\state{P}) = x \cdot \state{P}$.
\end{remark}

\begin{remark}
  The reader is invited to verify that $\innerprod{P}{Q}$ could
  equally well have been written $\quotep{\int \stackrel{\vee}{x}}$
  where $x = \event{P^{\underline{\perp}}}(Q)$.

  One of the motivations for this remark is that there is another way
  to factor these operations. We could package up evaluation in the dual:

  \begin{mathpar}
    \state{P}^{*} := \event{\int P^{\underline{\perp}}} := \quotep{\int P^{\underline{\perp}}}[-]
  \end{mathpar}

  and then have inner product defined by
  
  \begin{mathpar}
    \innerprod{P}{Q} := \event{P}(Q)
  \end{mathpar}

  Hopefully, experience with the calculations will provide guidance on
  the best factoring.
\end{remark}

\begin{remark}
  Assuming $\int (P^{\underline{\perp}}[P]) = 0$, the reader is
  invited to verify that $\forall P,Q. (\prmatrix{0}{Q})(\state{0}) =
  \state{0}$ and dually $(\prmatrix{P}{0})(\event{0}) = \event{0}$.
\end{remark}

\begin{remark}
  i'm a little worried that i don't (yet) have proper support for
  complex conjugacy. But, the observation above may give us a
  clue. According to Abramsky, it must be the case that the scalars
  are iso to the homset of the identity for the tensor -- which the
  observation above characterizes. 

  For now, we will simply bookmark the notion with $\overline{x}$.
\end{remark}

\subsubsection{Adjointness}

We need to give a definition of $(\cdot)^{\dagger}$ for matrices. The
obvious candidate definition is
\begin{mathpar}
(\Sigma_{\alpha}\fprmatrix{P_{\alpha}}{x_{\alpha}}{Q_{\alpha}})^{\dagger}
= \Sigma_{\alpha}\fprmatrix{(Q_{\alpha}^{\underline{\perp}})^{*}}{\overline{x}_{\alpha}}{P_{\alpha}^{\underline{\perp}}} 
\end{mathpar}

But, $(Q_{\alpha}^{\underline{\perp}})^{*}$ requires a name along
which to communicate the process to achieve the context application.

\subsubsection{Basis for a basis}
If processes label states and ``addition'' of states (a.k.a. vector
addition) is interpreted as parallel composition, what corresponds to
notions of linear independence and basis? Here, we recall that Yoshida
has developed a set of \emph{combinators} for an asynchronous verison
of Milner's $\pi$-calculus. These are a finite set of processes such
any process can be expressed as parallel composition of these
combinators together with liberal uses of the new operator and
replication. We can simply give a translation of these into the
present calculus and have reasonable expectation that the property
carries over. That is, that the resultant set allows to express all
processes via parallel composition. Note, however, that there is no
new operator or replication in this calculus. As a result, we expect
that the corresponding set is actually infinite. That is, we expect
that the space is actually infinite dimensional.

\begin{remark}
  The attentive reader may be a bit concerned. Certainly, the
  collection $S$, $K$ and $I$ is a finite set of
  combinators. Shouldn't we expect to see a finite set of combinators
  for an effectively equivalent system? i am very sympathetic to this
  critique and feel it warrants full attention. On the other hand, i
  also have in mind the following analogy. The natural numbers, as a
  monoid under addition, has exactly $1$ generator, while the natural
  numbers, as a monoid under multiplication, has countably many
  generators (the primes). We observe that the application of the
  lambda calculus is much less resource sensitive than the parallel
  composition of the $\pi$-calculus. Could it be the case that we have
  an analogy of the form
  
  \begin{mathpar}
    m + n : MN :: m*n : M|N
  \end{mathpar}

  giving a similar blow up in the set of ``primes''?  This is such a
  wonderful thought that, even if it's not true, i think it's worth
  writing down.
\end{remark}
 

\documentclass[12pt]{llncs}
%\documentclass{jktr}

\usepackage[pdftex]{hyperref}                   
\usepackage {listings}
\usepackage {mathpartir}
\usepackage{bcprules}
%\usepackage{listings}
                       
\usepackage{graphicx} 
%\usepackage[margins=2.5cm,nohead,nofoot]{geometry}
%\usepackage{geometry}
\usepackage{amsfonts}
\usepackage{amstext}
\usepackage{latexsym}
\usepackage{amssymb}
\usepackage{color}


%\include{myPreamble}
\include{qm2pi.local} 

%\ifpdf
%\usepackage[pdftex]{graphicx}
%\else
%\usepackage{graphicx}
%\fi

 % \ifpdf
%  \usepackage{pdfsync}
%  \if


%\title{Brief Article}
%\author{David F. Snyder}
%\author{L.G. Meredith}

%\address{Dept. of Math., Texas State University--San Marcos, San Marcos, TX 78666}
       
\pagestyle{empty}


\begin{document}

\lstset{language=[Objective]Caml,frame=shadowbox}

\input{qm2pi.front}

% section front matter (end)

\input{qm2pi.intro} 
 
% section introduction (end)

% \input{qm2pi.knotations} 

% section notation (end)

\input{qm2pi.process.calculi} 

% section concurrent_process_calculi_and_spatial_logics_ (end)
    
%\input{qm2pi.knots2pi} 

%\input{qm2pi.trefoil} 

%\input{qm2pi.mainthm} 

% subsection basic_interpretation (end)

%\input{qm2pi.rho.presentation} 
\subsection{The syntax and semantics of the notation system}\label{sub:the_syntax_and_semantics_of_the_notation_system} % (fold)

We now summarize a technical presentation of the calculus that
embodies our theory of dynamics. The typical presentation of such a
calculus follows the style of giving generators and relations on
them. The grammar, below, describing term constructors, freely
generates the set of processes, $\Proc$. This set is then quotiented
by a relation known as structural congruence and it is over this set
that the notion of dynamics is expressed. This presentation is
essentially that of \cite{MeredithR05} with the addition of
polyadicity and summation. For readability we have relegated some of
the technical subtleties to an appendix.

\subsubsection{Process grammar}\label{subsub:process_grammar}

\begin{mathpar}
  \inferrule* [lab=synchronization] {} {{M} \bc \pzero \;|\; x?F \;|\; x!C }
  \and
  \inferrule* [lab=abstraction] {} {{F} \bc (x)P}
  \and
  \inferrule* [lab=concretion] {} {{C} \bc \langle Q \rangle}
  \and
  \inferrule* [lab=process] {} {{P,Q} \bc M \;| \;P|Q \;|\; @{x}}
  \and
  \inferrule* [lab=name] {} {{x} \bc \quotep{P}}
\end{mathpar} 

Note that $\vec{x}$ (resp. $\vec{P}$) denotes a vector of names
(resp. processes) of length $|\vec{x}|$ (resp. $|\vec{P}|$). We adopt
the following useful abbreviations.

\begin{mathpar}
   x?(\vec{y}).P := x.(\vec{y})P \and  x\clift{\vec{P}} := x.\clift{\vec{P}}
   \and x!(y) := \lift{x}{\dropn{y}}
   \and \Pi_{i=0}^{n-1}P_i := P_0 | \ldots | P_{n-1}
\end{mathpar}

\subsubsection{Structural congruence}

\paragraph{Free and bound names and alpha-equivalence.} At the
core of structural equivalence is alpha-equivalence which identifies
process that are the same up to a change of variable. Formally, we
recognize the distinction between free and bound names. The free names
of a process, $\freenames{P}$, may be calculated recursively as
follows:

\begin{mathpar}
\freenames{\pzero} := \emptyset
  \and \\
  \freenames{x?(y).P} := \{ x \} \cup (\freenames{P} \setminus \{ y \})
  \and 
  \freenames{x!\langle P \rangle} := \{ x \} \cup \{ P \} 
  \and \\
  \freenames{P|Q} := \freenames{P} \cup \freenames{Q}
  \and \\
  \freenames{@{x}} := \{ x \}
\end{mathpar}

$\pi$
$\quotep{\pi}$

$\freenames{-} : \pi \to \mathcal{P}(\quotep{\pi})$

\begin{eqnarray*}
  \freenames{\pzero} & := & \emptyset \\
  \freenames{x?(y).P} & := & \{ x \} \cup (\freenames{P} \setminus \{ y \}) \\
  \freenames{x!\langle P \rangle} & := & \{ x \} \cup \{ P \} \\
  \freenames{P|Q} & := & \freenames{P} \cup \freenames{Q} \\
  \freenames{\dropn{x}} & := & \{ x \}
\end{eqnarray*}

The bound names of a process, $\boundnames{P}$, are those names occurring in $P$
that are not free. For example, in $x?(y).0$, the name $x$ is free, while $y$ is bound.

\begin{mathpar}
  \inferrule* [lab=monoidal-laws] {} { P|Q \equiv Q|P \and P|0 \equiv P \and P|(Q|R) \equiv (P|Q)|R }
\end{mathpar}

\begin{mathpar}
  \inferrule* [lab=alpha-equivalence] {} { (x)P \equiv (y)P\{y/x\} \and y \not\in \freenames{P} }
\end{mathpar}

\begin{definition}
Then two processes, $P,Q$, are alpha-equivalent if $P = Q\{\vec{y}/\vec{x}\}$ for
some $\vec{x} \in \boundnames{Q},\vec{y} \in \boundnames{P}$, where $Q\{\vec{y}/\vec{x}\}$
denotes the capture-avoiding substitution of $\vec{y}$ for $\vec{x}$ in $Q$.
\end{definition}

\begin{definition}
  The {\em structural congruence} \cite{SangiorgiWalker} , $\equiv$,
  between processes is the least congruence containing
  alpha-equivalence, satisfying the abelian monoid laws
  (associativity, commutativity and $\pzero$ as identity) for parallel
  composition $|$ and for summation $+$.
\end{definition}

\subsection{Name equivalence}

We take name equivalence, written $\nameeq$, to be the smallest
equivalence relation generated by the following rules.

\begin{mathpar}
\inferrule*[lab=Quote-drop]
{ }
{ \quotep{@{x}} \nameeq x }

\inferrule*[lab=Struct-equiv]
{ P \scong Q }
{ \quotep{P} \nameeq \quotep{Q} }
\end{mathpar}

The astute reader will have noticed that the mutual recursion of names
and processes imposes a mutual recursion on alpha-equivalence and
structural equivalence via name-equivalence. Fortunately, all of this
works out pleasantly and we may calculate in the natural way, free of
concern. The reader interested in the details is referred to the
appendix \ref{appendix:rho_details}.

\subsection{Substitution}

We use $\Proc$ for the set of processes, $\QProc$ for the set of
names, and $\id{\{}\vec{y} / \vec{x} \id{\}}$ to denote partial maps,
$s : \QProc \rightarrow \QProc$. A map, $s$ lifts, uniquely, to a map
on process terms, $\widehat{s} : \Proc \rightarrow \Proc$ by the
following equations.

\begin{mathpar}
  (0) \psubstp{Q}{P} := 0 \\
  (R \juxtap S) \psubstp{Q}{P}
  :=    
  (R)\psubstp{Q}{P} \juxtap (S) \psubstp{Q}{P} \\
  (x?(y).R) \psubstp{Q}{P}    
  :=    
  (x)\substp{Q}{P} (z)\concat( (R \psubstn{z}{y}) \psubstp{Q}{P} ) \\
  (\lift{x}{R}) \psubstp{Q}{P}  
  :=
  \lift{(x)\substp{Q}{P}}{ R \psubstp{Q}{P} } \\
%   (\dropn{x})  \psubstp{Q}{P}       
%   := 
%   \left\{ 
%     \begin{array}{ccc} 
%       \dropn{\quotep{Q}} & & x \nameeq \quotep{P} \\
%       \dropn{x} & & otherwise \\
%     \end{array}
%   \right. 
  (\dropn{x})  \psubstp{Q}{P}       
  := 
  \left\{ 
    \begin{array}{ccc} 
      Q & & x \nameeq \quotep{P} \\
      \dropn{x} & & otherwise \\
    \end{array}
  \right.
\end{mathpar}
 

where

\begin{eqnarray}
  (x)\id{\{} \lpquote Q \rpquote / \lpquote P \rpquote \id{\}}            = 
  \left\{ 
    \begin{array}{ccc}
      \lpquote Q \rpquote & & x \nameeq \lpquote P \rpquote \\
      x & & otherwise \\
    \end{array}
  \right. \nonumber
\end{eqnarray}

and $z$ is chosen distinct from $\quotep{P}$, $\quotep{Q}$, the free
names in $Q$, and all the names in $R$. Our $\alpha$-equivalence will
be built in the standard way from this substitution.

\begin{remark}\label{rem:no_self_referential_names}
  One consequence of these definitions is that $\forall P. \quotep{P}
  \not\in \freenames{P}$.
\end{remark}

\subsection{ Dynamic quote: an example }

Anticipating something of what's to come, consider applying the
substitution, $\widehat{\id{\{}u / z \id{\}}}$, to the following pair
of processes, $\lift{w}{y!(z)}$ and $w[ \lpquote y!(z) \rpquote ]$.

\begin{eqnarray}
	\lift{w}{y!(z)}\widehat{\id{\{}u / z \id{\}}}
		& = &
		\lift{w}{y!(u)} \nonumber\\
	w[ \lpquote y!(z) \rpquote ] \widehat{ \id{\{}u / z \id{\}} }
		& = &
		w[ \lpquote y!(z) \rpquote ] \nonumber
\end{eqnarray}

Because the body of the process between quotes is impervious to
substitution, we get radically different answers. In fact, by
examining the first process in an input context,
e.g. $x?(z).\lift{w}{y!(z)}$, we see that the process under the lift
operator may be shaped by prefixed inputs binding a name inside it. In
this sense, the lift operator will be seen as a way to dynamically
construct processes before reifying them as names.

Finally equipped with these standard features we can present the
dynamics of the calculus.

\subsubsection{Operational semantics} 

Finally, we introduce the computational dynamics. What marks these
algebras as distinct from other more traditionally studied algebraic
structures, e.g. vector spaces or polynomial rings, is the manner in
which dynamics is captured. In traditional structures, dynamics is typically
expressed through morphisms between such structures, as in linear maps
between vector spaces or morphisms between rings. In algebras
associated with the semantics of computation, the dynamics is
expressed as part of the algebraic structure itself, through a
reduction reduction relation typically denoted by $\red$. Below, we
give a recursive presentation of this relation for the calculus used
in the encoding.

$\red \subseteq \pi \times \pi$
$\red : \pi \to \mathcal{P}(\pi)$

\begin{mathpar}
  \inferrule* [lab=Comm] { \textsf{match}( x_{src}, x_{trgt} ) } { x_{trgt}?(y)P \; | \; x_{src}!\langle {Q} \rangle \red P\{\quotep{Q}/y}\} }
  \and \\
  \inferrule* [lab=Par] {{P} \red {P}'} {{{P} | {Q}} \red {{P}' | {Q}}}
  \and
  \inferrule* [lab=Equiv]{{{P} \scong {P}'} \andalso {{P}' \red {Q}'} \andalso {{Q}' \scong {Q}}}{{P} \red {Q}}
\end{mathpar}

\begin{eqnarray*}
  match_{\equiv} (\quotep{P},\quotep{Q}) & := & P \equiv Q \\
  match_{\dagger}(\quotep{P},\quotep{Q}) & := & \forall R. P|Q \red^{*} R => R \red^{*} 0 \\
  match_{K}(\quotep{P},\quotep{Q}) & := & K \mbox{ for some context } K
\end{eqnarray*}

$u?(x)P | u!\langle Q \rangle \red P\{\quotep{Q}/x\}$

%We write $\wred$ for $\red^*$, and $P\red$ if $\exists Q $ such that $ P \red Q$.
We write $P\red$ if $\exists Q $ such that $ P \red Q$ and $P\not\red$, otherwise.

\section{Replication}

As mentioned before, it is known that replication (and hence
recursion) can be implemented in a higher-order process algebra
\cite{SangiorgiWalker}. As our first example of calculation with the
machinery thus far presented we give the construction explicitly in
the {\rhoc}.

\begin{eqnarray}
	D_{x} & := & \prefix{x}{y}{(\binpar{\outputp{x}{y}}{@{y}})} \nonumber\\
	\bangp_{x}{P} & := & \binpar{{x}!\langle{\binpar{D_{x}}{P}}\rangle}{D_{x}} \nonumber
\end{eqnarray}

\begin{eqnarray}
	\bangp_{x}{P} & & \nonumber\\
	=
	& {x}!\langle{(\prefix{x}{y}{(\outputp{x}{y} | @{y})) | P}}\rangle 
	      | \prefix{x}{y}{(\outputp{x}{y} | @{y})} & \nonumber\\
	\red
	& (\outputp{x}{y} | @{y})\substn{\quotep{(\prefix{x}{y}{(@{y} | \outputp{x}{y})) | P}}}{y} & \nonumber\\
	=
	& \outputp{x}{\quotep{(\prefix{x}{y}{(\outputp{x}{y} | @{y})) | P}}}
	  | {(\prefix{x}{y}{(\outputp{x}{y} | @{y})) | P}} & \nonumber\\
	\red
	& \ldots & \nonumber\\
	\red^*
	& P | P | \ldots & \nonumber
\end{eqnarray}

Of course, this encoding, as an implementation, runs away, unfolding
$\bangp{P}$ eagerly. A lazier and more implementable replication
operator, restricted to input-guarded processes, may be obtained as follows.

\begin{eqnarray}
\bangp{\prefix{u}{v}{P}} 
	:= 
	\binpar{\lift{x}{\prefix{u}{v}{(\binpar{D(x)}{P})}}}{D(x)} \nonumber
\end{eqnarray}

\begin{remark}
  Note that the lazier definition still does not deal with summation
  or mixed summation (i.e. sums over input and output). The reader is
  invited to construct definitions of replication that deal with these
  features. 

  Further, the definitions are parameterized in a name, $x$. Can you,
  gentle reader, make a definition that eliminates this parameter and
  guarantees no accidental interaction between the replication
  machinery and the process being replicated -- i.e. no accidental
  sharing of names used by the process to get its work done and the
  name(s) used by the replication to effect copying. This latter
  revision of the definition of replication is crucial to obtaining
  the expected identity $!!P \sim !P$.
\end{remark}

\begin{remark}\label{rem:paradoxical_combinator}
  The reader familiar with the lambda calculus will have noticed the
  similarity between $D$ and the paradoxical combinator.

  [Ed. note: the existence of this seems to suggest we have to be more
  restrictive on the set of processes and names we admit if we are to
  support no-cloning.]
\end{remark}

\subsubsection{Bisimulation}

The computational dynamics gives rise to another kind of equivalence,
the equivalence of computational behavior. As previously mentioned
this is typically captured \emph{via} some form of bisimulation.

% The notion we use in this paper is weak barbed bisimulation
% \cite{milner91polyadicpi}.

The notion we use in this paper is derived from weak barbed
bisimulation \cite{milner91polyadicpi}. 

\begin{definition}
An \emph{observation relation}, $\downarrow_{\mathcal N}$, over a set
of names, $\mathcal N$, is the smallest relation satisfying the rules
below.

\infrule[Out-barb]{y \in {\mathcal N}, \; x \nameeq y}
		  {\outputp{x}{v} \downarrow_{\mathcal N} x}
\infrule[Par-barb]{\mbox{$P\downarrow_{\mathcal N} x$ or $Q\downarrow_{\mathcal N} x$}}
		  {\binpar{P}{Q} \downarrow_{\mathcal N} x}

We write $P \Downarrow_{\mathcal N} x$ if there is $Q$ such that 
$P \wred Q$ and $Q \downarrow_{\mathcal N} x$.
\end{definition}

\begin{definition}
%\label{def.bbisim}
An  ${\mathcal N}$-\emph{barbed bisimulation} over a set of names, ${\mathcal N}$, is a symmetric binary relation 
${\mathcal S}_{\mathcal N}$ between agents such that $P\rel{S}_{\mathcal N}Q$ implies:
\begin{enumerate}
\item If $P \red P'$ then $Q \wred Q'$ and $P'\rel{S}_{\mathcal N} Q'$.
\item If $P\downarrow_{\mathcal N} x$, then $Q\Downarrow_{\mathcal N} x$.
\end{enumerate}
$P$ is ${\mathcal N}$-barbed bisimilar to $Q$, written
$P \wbbisim_{\mathcal N} Q$, if $P \rel{S}_{\mathcal N} Q$ for some ${\mathcal N}$-barbed bisimulation ${\mathcal S}_{\mathcal N}$.
\end{definition}

$\mathcal{R} \subseteq \pi \times \pi$

$P \mathcal{R} Q => \forall P'. P \red P' \Rightarrow \exists Q'. Q \red Q', P' \mathcal{R} Q'$

$P \vdash x \Rightarrow Q \vdash x$

\begin{mathpar}
  \inferrule*[lab=Out-barb]{x \nameeq y}{{y}!\langle{Q}\rangle \vdash x}
  \and
  \inferrule*[lab=Par-barb]{\mbox{$P\vdash x$ or $Q\vdash x$}}{\binpar{P}{Q} \vdash x}
\end{mathpar}

\subsubsection{Contexts}

One of the principle advantages of computational calculi like the
$\pi$-calculus is a well-defined notion of context,
contextual-equivalence and a correlation between
contextual-equivalence and notions of bisimulation. The notion of
context allows the decomposition of a process into (sub-)process and
its syntactic environment, its context. Thus, a context may be
thought of as a process with a ``hole'' (written $\Box$) in it. The
application of a context $M$ to a process $P$, written $M[P]$, is
tantamount to filling the hole in $M$ with $P$. In this paper we do
not need the full weight of this theory, but do make use of the notion
of context in the proof the main theorem. 

\begin{mathpar}
  \inferrule* [lab=summation] {} {{M_{M},M_{N}} \bc \Box \;|\; x.M_{A} \;|\; M_{M}+M_{N}}
  \and
  \inferrule* [lab=agent] {} {{M_{A}} \bc (\vec{x})M_{P} \;| \; \clift{P_0,\ldots,M_{P},\ldots,P_N}}
  \and \\
  \inferrule* [lab=process] {} {{M_{P}} \bc M_{N} \;| \;P|M_{P} }
\end{mathpar} 

\begin{mathpar}
  \inferrule* [lab=sychronization] {} {M_{N} \bc \Box \;|\; x?M_{F} \;|\; x!M_{C}}
  \and
  \inferrule* [lab=abstraction] {} {{M_{F}} \bc (x)M_{P} }
  \and
  \inferrule* [lab=concretion] {} {{M_{C}} \bc \langle M_{P} \rangle }
  \and \\
  \inferrule* [lab=process] {} {{M_{P}} \bc M_{N} \;| \;P|M_{P} }
\end{mathpar}

\begin{definition}[contextual application] Given a context $M$, and
  process $P$, we define the \emph{contextual application}, $M[P] :=
  M\{P/\Box\}$. That is, the contextual application of M to P is the
  substitution of $P$ for $\Box$ in $M$.
\end{definition}

$\meaningof{-} : L \to \mathcal{P}(\pi)$

\begin{mathpar}
  \inferrule* [lab=collection] {} {\meaningof{true} = \pi, \and \meaningof{~E} = \pi \setminus \meaningof{E}, \and \meaningof{E_{1} \& E_{2}} = \meaningof{E_{1}} \cap \meaningof{E_{2}}}
\end{mathpar}

\begin{mathpar}
  \inferrule* [lab=structure] {} {\meaningof{0} = \{ P \in \pi | P \equiv 0 \}, \and \\ \meaningof{E_1 | E_2} = \{ P \in \pi | P \equiv P_{1} | P_{2}, P_{1} \in \meaningof{E_{1}}, P_{2} \in \meaningof{E_2}\} }
\end{mathpar}

\begin{mathpar}
 \inferrule* [lab=behavior] {} {\meaningof{\langle a?b \rangle E} = \{ P \in \pi | P \equiv Q | u?(y)P', \\ \and \\\\ \and \\ \;\;\; u \in \meaningof{a}, \forall z.P'\{z/y\} \in \meaningof{E\{z/b\}}\}, \and \\ \meaningof{a!E} = \{ P \in \pi | P \equiv Q | x!\langle P' \rangle, x \in \meaningof{a} P' \in \meaningof{E}\} }
\end{mathpar}

\begin{mathpar}
 \inferrule* [lab=nominal] {} {\meaningof{\quotep{E}} = \{ \quotep{P} \in \quotep{\pi} | P \in \meaningof{E} \}, \and \meaningof{\quotep{P}} = \{ \quotep{Q} \in \quotep{\pi} | P \equiv Q \} \and \\ \meaningof{@\quotep{E}} = \{ P \in \pi | P \equiv @x, x \in \meaningof{E} \}}
\end{mathpar}

\begin{eqnarray*}
  \\
  \meaningof{-} : TS \to ST
\end{eqnarray*}

\begin{eqnarray*}
  \\
  L : TS \to ST
\end{eqnarray*}

\begin{eqnarray*}
  \\
  P \models E \iff P \in \meaningof{E}
\end{eqnarray*}

\begin{eqnarray*}
  P \approx_{L} Q \iff \forall E \in L. P \models E \iff Q \models E
\end{eqnarray*}

\begin{eqnarray*}
  P \approx_{K} Q
\end{eqnarray*}

\begin{eqnarray*}
  P \approx Q
\end{eqnarray*}

$\approx_{K} = \approx = \approx_{L}$

\subsubsection{Contextual duality}

Note that contexts extend the quotation operation to a family of
operations from processes to names. Given a context, $M$, we can
define a \emph{nominal context}, $\quotep{M}$ by $\quotep{M}[P] :=
\quotep{M[P]}$. To foreshadow what is to come we observe that these
operations enjoy a duality with processes very much like the duality
between vectors and maps from vectors to scalars.

Further, because the calculus is essentially higher-order, we have a
correspondence between contexts and processes. More specifically,
given a name $x$ and a context $M$ we can construct $M^{*}_{x}$ such
that 

\begin{mathpar}
  M^{*}_{x} | \lift{x}{P} \red M[P]
\end{mathpar}

namely,

\begin{mathpar}
  M^{*}_{x} := x?(u).M[\dropn{u}]
\end{mathpar}

The dependence of $M^{*}_{x}$ on a name makes it an abstraction, 

\begin{mathpar}
  M^{*} := (x)x?(u).M[\dropn{u}]
\end{mathpar}

\subsection{Additional notation}

It will sometimes be convenient to denote the process a name
quotes. We already have the notation $x = \quotep{P}$, but it will be
convenient to introduce an alternate notation, $\procn{x}$, when we
want to emphasize the connection to the use of the name. Note that, by
virtue of name equivalence, $\quotep{\procn{x}} \nameeq x$; so, the
notation is consistent with previous definitions.

Further, because names have structure it is possible to effect
substitutions on the basis of that structure. This means we need to
upgrade our notation for substitutions, which we accomplish by
adapting comprehension notation. Thus,

\begin{mathpar}
  P\{ y / x : x \in S \}
\end{mathpar}

is interpreted to mean the process derived from P by replacing (in a
capture-avoiding manner) each occurrence of $x$ in $S$ by $y$. For example,

\begin{mathpar}
  P\{ \quotep{\procn{x}|\procn{x}} / x : x \in \freenames{P} \}
\end{mathpar}

will replace each (occurrence) of a free name $x$ in $P$ by
$\quotep{\procn{x}|\procn{x}}$.

Also, we will avail ourselves of the notation $x^{L}$ and $x^{R}$ to
denote injections of a name into disjoint copies of the name
space. There are numerous ways to accomplish this. One example can be
found in \cite{MeredithR05}. This notation overloads to vectors of
names: $\vec{x}^{\pi} := (x_{i}^{\pi} \; : \; 0 \leq i < |\vec{x}| )$ where $\pi \in \{L,R\}$.

We also use $P^{\Box} := P|\Box$.

In \cite{MeredithR05} an interpretation of the new operator is
given. It turns out that there are several possible interpretations
all enjoying the requisite algebraic properties of the operator (see
\cite{milner91polyadicpi}). We will therefore make liberal use of
$(\nu\; \vec{x})P$.

% subsection the_syntax_and_semantics_of_the_notation_system (end)   

\input{qm2pi.qmops} 

\input{qm2pi.sterngerlach} 

\input{qm2pi.metric} 

% section concurrent_process_calculi (end)

%\input{qm2pi.proofsketch}

% section proof sketch (end)

%\input{qm2pi.slviaknots} 

% section spatial logic via knots (end)

\input{qm2pi.conclusion}

% section conclusion (end)

%\input{qm2pi.dtcodes} 

% section wiring algorithm (end)

\input{qm2pi.ack} 

% section acknowledgments (end)

\newpage


\bibliographystyle{plain}   
\bibliography{../../biblios/main.bib}

\input{qm2pi.rhodetails}

\end{document}

 

\documentclass[12pt]{llncs}
%\documentclass{jktr}

\usepackage[pdftex]{hyperref}                   
\usepackage {listings}
\usepackage {mathpartir}
\usepackage{bcprules}
%\usepackage{listings}
                       
\usepackage{graphicx} 
%\usepackage[margins=2.5cm,nohead,nofoot]{geometry}
%\usepackage{geometry}
\usepackage{amsfonts}
\usepackage{amstext}
\usepackage{latexsym}
\usepackage{amssymb}
\usepackage{color}


%\include{myPreamble}
\include{qm2pi.local} 

%\ifpdf
%\usepackage[pdftex]{graphicx}
%\else
%\usepackage{graphicx}
%\fi

 % \ifpdf
%  \usepackage{pdfsync}
%  \if


%\title{Brief Article}
%\author{David F. Snyder}
%\author{L.G. Meredith}

%\address{Dept. of Math., Texas State University--San Marcos, San Marcos, TX 78666}
       
\pagestyle{empty}


\begin{document}

\lstset{language=[Objective]Caml,frame=shadowbox}

\input{qm2pi.front}

% section front matter (end)

\input{qm2pi.intro} 
 
% section introduction (end)

% \input{qm2pi.knotations} 

% section notation (end)

\input{qm2pi.process.calculi} 

% section concurrent_process_calculi_and_spatial_logics_ (end)
    
%\input{qm2pi.knots2pi} 

%\input{qm2pi.trefoil} 

%\input{qm2pi.mainthm} 

% subsection basic_interpretation (end)

%\input{qm2pi.rho.presentation} 
\subsection{The syntax and semantics of the notation system}\label{sub:the_syntax_and_semantics_of_the_notation_system} % (fold)

We now summarize a technical presentation of the calculus that
embodies our theory of dynamics. The typical presentation of such a
calculus follows the style of giving generators and relations on
them. The grammar, below, describing term constructors, freely
generates the set of processes, $\Proc$. This set is then quotiented
by a relation known as structural congruence and it is over this set
that the notion of dynamics is expressed. This presentation is
essentially that of \cite{MeredithR05} with the addition of
polyadicity and summation. For readability we have relegated some of
the technical subtleties to an appendix.

\subsubsection{Process grammar}\label{subsub:process_grammar}

\begin{mathpar}
  \inferrule* [lab=synchronization] {} {{M} \bc \pzero \;|\; x?F \;|\; x!C }
  \and
  \inferrule* [lab=abstraction] {} {{F} \bc (x)P}
  \and
  \inferrule* [lab=concretion] {} {{C} \bc \langle Q \rangle}
  \and
  \inferrule* [lab=process] {} {{P,Q} \bc M \;| \;P|Q \;|\; @{x}}
  \and
  \inferrule* [lab=name] {} {{x} \bc \quotep{P}}
\end{mathpar} 

Note that $\vec{x}$ (resp. $\vec{P}$) denotes a vector of names
(resp. processes) of length $|\vec{x}|$ (resp. $|\vec{P}|$). We adopt
the following useful abbreviations.

\begin{mathpar}
   x?(\vec{y}).P := x.(\vec{y})P \and  x\clift{\vec{P}} := x.\clift{\vec{P}}
   \and x!(y) := \lift{x}{\dropn{y}}
   \and \Pi_{i=0}^{n-1}P_i := P_0 | \ldots | P_{n-1}
\end{mathpar}

\subsubsection{Structural congruence}

\paragraph{Free and bound names and alpha-equivalence.} At the
core of structural equivalence is alpha-equivalence which identifies
process that are the same up to a change of variable. Formally, we
recognize the distinction between free and bound names. The free names
of a process, $\freenames{P}$, may be calculated recursively as
follows:

\begin{mathpar}
\freenames{\pzero} := \emptyset
  \and \\
  \freenames{x?(y).P} := \{ x \} \cup (\freenames{P} \setminus \{ y \})
  \and 
  \freenames{x!\langle P \rangle} := \{ x \} \cup \{ P \} 
  \and \\
  \freenames{P|Q} := \freenames{P} \cup \freenames{Q}
  \and \\
  \freenames{@{x}} := \{ x \}
\end{mathpar}

$\pi$
$\quotep{\pi}$

$\freenames{-} : \pi \to \mathcal{P}(\quotep{\pi})$

\begin{eqnarray*}
  \freenames{\pzero} & := & \emptyset \\
  \freenames{x?(y).P} & := & \{ x \} \cup (\freenames{P} \setminus \{ y \}) \\
  \freenames{x!\langle P \rangle} & := & \{ x \} \cup \{ P \} \\
  \freenames{P|Q} & := & \freenames{P} \cup \freenames{Q} \\
  \freenames{\dropn{x}} & := & \{ x \}
\end{eqnarray*}

The bound names of a process, $\boundnames{P}$, are those names occurring in $P$
that are not free. For example, in $x?(y).0$, the name $x$ is free, while $y$ is bound.

\begin{mathpar}
  \inferrule* [lab=monoidal-laws] {} { P|Q \equiv Q|P \and P|0 \equiv P \and P|(Q|R) \equiv (P|Q)|R }
\end{mathpar}

\begin{mathpar}
  \inferrule* [lab=alpha-equivalence] {} { (x)P \equiv (y)P\{y/x\} \and y \not\in \freenames{P} }
\end{mathpar}

\begin{definition}
Then two processes, $P,Q$, are alpha-equivalent if $P = Q\{\vec{y}/\vec{x}\}$ for
some $\vec{x} \in \boundnames{Q},\vec{y} \in \boundnames{P}$, where $Q\{\vec{y}/\vec{x}\}$
denotes the capture-avoiding substitution of $\vec{y}$ for $\vec{x}$ in $Q$.
\end{definition}

\begin{definition}
  The {\em structural congruence} \cite{SangiorgiWalker} , $\equiv$,
  between processes is the least congruence containing
  alpha-equivalence, satisfying the abelian monoid laws
  (associativity, commutativity and $\pzero$ as identity) for parallel
  composition $|$ and for summation $+$.
\end{definition}

\subsection{Name equivalence}

We take name equivalence, written $\nameeq$, to be the smallest
equivalence relation generated by the following rules.

\begin{mathpar}
\inferrule*[lab=Quote-drop]
{ }
{ \quotep{@{x}} \nameeq x }

\inferrule*[lab=Struct-equiv]
{ P \scong Q }
{ \quotep{P} \nameeq \quotep{Q} }
\end{mathpar}

The astute reader will have noticed that the mutual recursion of names
and processes imposes a mutual recursion on alpha-equivalence and
structural equivalence via name-equivalence. Fortunately, all of this
works out pleasantly and we may calculate in the natural way, free of
concern. The reader interested in the details is referred to the
appendix \ref{appendix:rho_details}.

\subsection{Substitution}

We use $\Proc$ for the set of processes, $\QProc$ for the set of
names, and $\id{\{}\vec{y} / \vec{x} \id{\}}$ to denote partial maps,
$s : \QProc \rightarrow \QProc$. A map, $s$ lifts, uniquely, to a map
on process terms, $\widehat{s} : \Proc \rightarrow \Proc$ by the
following equations.

\begin{mathpar}
  (0) \psubstp{Q}{P} := 0 \\
  (R \juxtap S) \psubstp{Q}{P}
  :=    
  (R)\psubstp{Q}{P} \juxtap (S) \psubstp{Q}{P} \\
  (x?(y).R) \psubstp{Q}{P}    
  :=    
  (x)\substp{Q}{P} (z)\concat( (R \psubstn{z}{y}) \psubstp{Q}{P} ) \\
  (\lift{x}{R}) \psubstp{Q}{P}  
  :=
  \lift{(x)\substp{Q}{P}}{ R \psubstp{Q}{P} } \\
%   (\dropn{x})  \psubstp{Q}{P}       
%   := 
%   \left\{ 
%     \begin{array}{ccc} 
%       \dropn{\quotep{Q}} & & x \nameeq \quotep{P} \\
%       \dropn{x} & & otherwise \\
%     \end{array}
%   \right. 
  (\dropn{x})  \psubstp{Q}{P}       
  := 
  \left\{ 
    \begin{array}{ccc} 
      Q & & x \nameeq \quotep{P} \\
      \dropn{x} & & otherwise \\
    \end{array}
  \right.
\end{mathpar}
 

where

\begin{eqnarray}
  (x)\id{\{} \lpquote Q \rpquote / \lpquote P \rpquote \id{\}}            = 
  \left\{ 
    \begin{array}{ccc}
      \lpquote Q \rpquote & & x \nameeq \lpquote P \rpquote \\
      x & & otherwise \\
    \end{array}
  \right. \nonumber
\end{eqnarray}

and $z$ is chosen distinct from $\quotep{P}$, $\quotep{Q}$, the free
names in $Q$, and all the names in $R$. Our $\alpha$-equivalence will
be built in the standard way from this substitution.

\begin{remark}\label{rem:no_self_referential_names}
  One consequence of these definitions is that $\forall P. \quotep{P}
  \not\in \freenames{P}$.
\end{remark}

\subsection{ Dynamic quote: an example }

Anticipating something of what's to come, consider applying the
substitution, $\widehat{\id{\{}u / z \id{\}}}$, to the following pair
of processes, $\lift{w}{y!(z)}$ and $w[ \lpquote y!(z) \rpquote ]$.

\begin{eqnarray}
	\lift{w}{y!(z)}\widehat{\id{\{}u / z \id{\}}}
		& = &
		\lift{w}{y!(u)} \nonumber\\
	w[ \lpquote y!(z) \rpquote ] \widehat{ \id{\{}u / z \id{\}} }
		& = &
		w[ \lpquote y!(z) \rpquote ] \nonumber
\end{eqnarray}

Because the body of the process between quotes is impervious to
substitution, we get radically different answers. In fact, by
examining the first process in an input context,
e.g. $x?(z).\lift{w}{y!(z)}$, we see that the process under the lift
operator may be shaped by prefixed inputs binding a name inside it. In
this sense, the lift operator will be seen as a way to dynamically
construct processes before reifying them as names.

Finally equipped with these standard features we can present the
dynamics of the calculus.

\subsubsection{Operational semantics} 

Finally, we introduce the computational dynamics. What marks these
algebras as distinct from other more traditionally studied algebraic
structures, e.g. vector spaces or polynomial rings, is the manner in
which dynamics is captured. In traditional structures, dynamics is typically
expressed through morphisms between such structures, as in linear maps
between vector spaces or morphisms between rings. In algebras
associated with the semantics of computation, the dynamics is
expressed as part of the algebraic structure itself, through a
reduction reduction relation typically denoted by $\red$. Below, we
give a recursive presentation of this relation for the calculus used
in the encoding.

$\red \subseteq \pi \times \pi$
$\red : \pi \to \mathcal{P}(\pi)$

\begin{mathpar}
  \inferrule* [lab=Comm] { \textsf{match}( x_{src}, x_{trgt} ) } { x_{trgt}?(y)P \; | \; x_{src}!\langle {Q} \rangle \red P\{\quotep{Q}/y}\} }
  \and \\
  \inferrule* [lab=Par] {{P} \red {P}'} {{{P} | {Q}} \red {{P}' | {Q}}}
  \and
  \inferrule* [lab=Equiv]{{{P} \scong {P}'} \andalso {{P}' \red {Q}'} \andalso {{Q}' \scong {Q}}}{{P} \red {Q}}
\end{mathpar}

\begin{eqnarray*}
  match_{\equiv} (\quotep{P},\quotep{Q}) & := & P \equiv Q \\
  match_{\dagger}(\quotep{P},\quotep{Q}) & := & \forall R. P|Q \red^{*} R => R \red^{*} 0 \\
  match_{K}(\quotep{P},\quotep{Q}) & := & K \mbox{ for some context } K
\end{eqnarray*}

$u?(x)P | u!\langle Q \rangle \red P\{\quotep{Q}/x\}$

%We write $\wred$ for $\red^*$, and $P\red$ if $\exists Q $ such that $ P \red Q$.
We write $P\red$ if $\exists Q $ such that $ P \red Q$ and $P\not\red$, otherwise.

\section{Replication}

As mentioned before, it is known that replication (and hence
recursion) can be implemented in a higher-order process algebra
\cite{SangiorgiWalker}. As our first example of calculation with the
machinery thus far presented we give the construction explicitly in
the {\rhoc}.

\begin{eqnarray}
	D_{x} & := & \prefix{x}{y}{(\binpar{\outputp{x}{y}}{@{y}})} \nonumber\\
	\bangp_{x}{P} & := & \binpar{{x}!\langle{\binpar{D_{x}}{P}}\rangle}{D_{x}} \nonumber
\end{eqnarray}

\begin{eqnarray}
	\bangp_{x}{P} & & \nonumber\\
	=
	& {x}!\langle{(\prefix{x}{y}{(\outputp{x}{y} | @{y})) | P}}\rangle 
	      | \prefix{x}{y}{(\outputp{x}{y} | @{y})} & \nonumber\\
	\red
	& (\outputp{x}{y} | @{y})\substn{\quotep{(\prefix{x}{y}{(@{y} | \outputp{x}{y})) | P}}}{y} & \nonumber\\
	=
	& \outputp{x}{\quotep{(\prefix{x}{y}{(\outputp{x}{y} | @{y})) | P}}}
	  | {(\prefix{x}{y}{(\outputp{x}{y} | @{y})) | P}} & \nonumber\\
	\red
	& \ldots & \nonumber\\
	\red^*
	& P | P | \ldots & \nonumber
\end{eqnarray}

Of course, this encoding, as an implementation, runs away, unfolding
$\bangp{P}$ eagerly. A lazier and more implementable replication
operator, restricted to input-guarded processes, may be obtained as follows.

\begin{eqnarray}
\bangp{\prefix{u}{v}{P}} 
	:= 
	\binpar{\lift{x}{\prefix{u}{v}{(\binpar{D(x)}{P})}}}{D(x)} \nonumber
\end{eqnarray}

\begin{remark}
  Note that the lazier definition still does not deal with summation
  or mixed summation (i.e. sums over input and output). The reader is
  invited to construct definitions of replication that deal with these
  features. 

  Further, the definitions are parameterized in a name, $x$. Can you,
  gentle reader, make a definition that eliminates this parameter and
  guarantees no accidental interaction between the replication
  machinery and the process being replicated -- i.e. no accidental
  sharing of names used by the process to get its work done and the
  name(s) used by the replication to effect copying. This latter
  revision of the definition of replication is crucial to obtaining
  the expected identity $!!P \sim !P$.
\end{remark}

\begin{remark}\label{rem:paradoxical_combinator}
  The reader familiar with the lambda calculus will have noticed the
  similarity between $D$ and the paradoxical combinator.

  [Ed. note: the existence of this seems to suggest we have to be more
  restrictive on the set of processes and names we admit if we are to
  support no-cloning.]
\end{remark}

\subsubsection{Bisimulation}

The computational dynamics gives rise to another kind of equivalence,
the equivalence of computational behavior. As previously mentioned
this is typically captured \emph{via} some form of bisimulation.

% The notion we use in this paper is weak barbed bisimulation
% \cite{milner91polyadicpi}.

The notion we use in this paper is derived from weak barbed
bisimulation \cite{milner91polyadicpi}. 

\begin{definition}
An \emph{observation relation}, $\downarrow_{\mathcal N}$, over a set
of names, $\mathcal N$, is the smallest relation satisfying the rules
below.

\infrule[Out-barb]{y \in {\mathcal N}, \; x \nameeq y}
		  {\outputp{x}{v} \downarrow_{\mathcal N} x}
\infrule[Par-barb]{\mbox{$P\downarrow_{\mathcal N} x$ or $Q\downarrow_{\mathcal N} x$}}
		  {\binpar{P}{Q} \downarrow_{\mathcal N} x}

We write $P \Downarrow_{\mathcal N} x$ if there is $Q$ such that 
$P \wred Q$ and $Q \downarrow_{\mathcal N} x$.
\end{definition}

\begin{definition}
%\label{def.bbisim}
An  ${\mathcal N}$-\emph{barbed bisimulation} over a set of names, ${\mathcal N}$, is a symmetric binary relation 
${\mathcal S}_{\mathcal N}$ between agents such that $P\rel{S}_{\mathcal N}Q$ implies:
\begin{enumerate}
\item If $P \red P'$ then $Q \wred Q'$ and $P'\rel{S}_{\mathcal N} Q'$.
\item If $P\downarrow_{\mathcal N} x$, then $Q\Downarrow_{\mathcal N} x$.
\end{enumerate}
$P$ is ${\mathcal N}$-barbed bisimilar to $Q$, written
$P \wbbisim_{\mathcal N} Q$, if $P \rel{S}_{\mathcal N} Q$ for some ${\mathcal N}$-barbed bisimulation ${\mathcal S}_{\mathcal N}$.
\end{definition}

$\mathcal{R} \subseteq \pi \times \pi$

$P \mathcal{R} Q => \forall P'. P \red P' \Rightarrow \exists Q'. Q \red Q', P' \mathcal{R} Q'$

$P \vdash x \Rightarrow Q \vdash x$

\begin{mathpar}
  \inferrule*[lab=Out-barb]{x \nameeq y}{{y}!\langle{Q}\rangle \vdash x}
  \and
  \inferrule*[lab=Par-barb]{\mbox{$P\vdash x$ or $Q\vdash x$}}{\binpar{P}{Q} \vdash x}
\end{mathpar}

\subsubsection{Contexts}

One of the principle advantages of computational calculi like the
$\pi$-calculus is a well-defined notion of context,
contextual-equivalence and a correlation between
contextual-equivalence and notions of bisimulation. The notion of
context allows the decomposition of a process into (sub-)process and
its syntactic environment, its context. Thus, a context may be
thought of as a process with a ``hole'' (written $\Box$) in it. The
application of a context $M$ to a process $P$, written $M[P]$, is
tantamount to filling the hole in $M$ with $P$. In this paper we do
not need the full weight of this theory, but do make use of the notion
of context in the proof the main theorem. 

\begin{mathpar}
  \inferrule* [lab=summation] {} {{M_{M},M_{N}} \bc \Box \;|\; x.M_{A} \;|\; M_{M}+M_{N}}
  \and
  \inferrule* [lab=agent] {} {{M_{A}} \bc (\vec{x})M_{P} \;| \; \clift{P_0,\ldots,M_{P},\ldots,P_N}}
  \and \\
  \inferrule* [lab=process] {} {{M_{P}} \bc M_{N} \;| \;P|M_{P} }
\end{mathpar} 

\begin{mathpar}
  \inferrule* [lab=sychronization] {} {M_{N} \bc \Box \;|\; x?M_{F} \;|\; x!M_{C}}
  \and
  \inferrule* [lab=abstraction] {} {{M_{F}} \bc (x)M_{P} }
  \and
  \inferrule* [lab=concretion] {} {{M_{C}} \bc \langle M_{P} \rangle }
  \and \\
  \inferrule* [lab=process] {} {{M_{P}} \bc M_{N} \;| \;P|M_{P} }
\end{mathpar}

\begin{definition}[contextual application] Given a context $M$, and
  process $P$, we define the \emph{contextual application}, $M[P] :=
  M\{P/\Box\}$. That is, the contextual application of M to P is the
  substitution of $P$ for $\Box$ in $M$.
\end{definition}

$\meaningof{-} : L \to \mathcal{P}(\pi)$

\begin{mathpar}
  \inferrule* [lab=collection] {} {\meaningof{true} = \pi, \and \meaningof{~E} = \pi \setminus \meaningof{E}, \and \meaningof{E_{1} \& E_{2}} = \meaningof{E_{1}} \cap \meaningof{E_{2}}}
\end{mathpar}

\begin{mathpar}
  \inferrule* [lab=structure] {} {\meaningof{0} = \{ P \in \pi | P \equiv 0 \}, \and \\ \meaningof{E_1 | E_2} = \{ P \in \pi | P \equiv P_{1} | P_{2}, P_{1} \in \meaningof{E_{1}}, P_{2} \in \meaningof{E_2}\} }
\end{mathpar}

\begin{mathpar}
 \inferrule* [lab=behavior] {} {\meaningof{\langle a?b \rangle E} = \{ P \in \pi | P \equiv Q | u?(y)P', \\ \and \\\\ \and \\ \;\;\; u \in \meaningof{a}, \forall z.P'\{z/y\} \in \meaningof{E\{z/b\}}\}, \and \\ \meaningof{a!E} = \{ P \in \pi | P \equiv Q | x!\langle P' \rangle, x \in \meaningof{a} P' \in \meaningof{E}\} }
\end{mathpar}

\begin{mathpar}
 \inferrule* [lab=nominal] {} {\meaningof{\quotep{E}} = \{ \quotep{P} \in \quotep{\pi} | P \in \meaningof{E} \}, \and \meaningof{\quotep{P}} = \{ \quotep{Q} \in \quotep{\pi} | P \equiv Q \} \and \\ \meaningof{@\quotep{E}} = \{ P \in \pi | P \equiv @x, x \in \meaningof{E} \}}
\end{mathpar}

\begin{eqnarray*}
  \\
  \meaningof{-} : TS \to ST
\end{eqnarray*}

\begin{eqnarray*}
  \\
  L : TS \to ST
\end{eqnarray*}

\begin{eqnarray*}
  \\
  P \models E \iff P \in \meaningof{E}
\end{eqnarray*}

\begin{eqnarray*}
  P \approx_{L} Q \iff \forall E \in L. P \models E \iff Q \models E
\end{eqnarray*}

\begin{eqnarray*}
  P \approx_{K} Q
\end{eqnarray*}

\begin{eqnarray*}
  P \approx Q
\end{eqnarray*}

$\approx_{K} = \approx = \approx_{L}$

\subsubsection{Contextual duality}

Note that contexts extend the quotation operation to a family of
operations from processes to names. Given a context, $M$, we can
define a \emph{nominal context}, $\quotep{M}$ by $\quotep{M}[P] :=
\quotep{M[P]}$. To foreshadow what is to come we observe that these
operations enjoy a duality with processes very much like the duality
between vectors and maps from vectors to scalars.

Further, because the calculus is essentially higher-order, we have a
correspondence between contexts and processes. More specifically,
given a name $x$ and a context $M$ we can construct $M^{*}_{x}$ such
that 

\begin{mathpar}
  M^{*}_{x} | \lift{x}{P} \red M[P]
\end{mathpar}

namely,

\begin{mathpar}
  M^{*}_{x} := x?(u).M[\dropn{u}]
\end{mathpar}

The dependence of $M^{*}_{x}$ on a name makes it an abstraction, 

\begin{mathpar}
  M^{*} := (x)x?(u).M[\dropn{u}]
\end{mathpar}

\subsection{Additional notation}

It will sometimes be convenient to denote the process a name
quotes. We already have the notation $x = \quotep{P}$, but it will be
convenient to introduce an alternate notation, $\procn{x}$, when we
want to emphasize the connection to the use of the name. Note that, by
virtue of name equivalence, $\quotep{\procn{x}} \nameeq x$; so, the
notation is consistent with previous definitions.

Further, because names have structure it is possible to effect
substitutions on the basis of that structure. This means we need to
upgrade our notation for substitutions, which we accomplish by
adapting comprehension notation. Thus,

\begin{mathpar}
  P\{ y / x : x \in S \}
\end{mathpar}

is interpreted to mean the process derived from P by replacing (in a
capture-avoiding manner) each occurrence of $x$ in $S$ by $y$. For example,

\begin{mathpar}
  P\{ \quotep{\procn{x}|\procn{x}} / x : x \in \freenames{P} \}
\end{mathpar}

will replace each (occurrence) of a free name $x$ in $P$ by
$\quotep{\procn{x}|\procn{x}}$.

Also, we will avail ourselves of the notation $x^{L}$ and $x^{R}$ to
denote injections of a name into disjoint copies of the name
space. There are numerous ways to accomplish this. One example can be
found in \cite{MeredithR05}. This notation overloads to vectors of
names: $\vec{x}^{\pi} := (x_{i}^{\pi} \; : \; 0 \leq i < |\vec{x}| )$ where $\pi \in \{L,R\}$.

We also use $P^{\Box} := P|\Box$.

In \cite{MeredithR05} an interpretation of the new operator is
given. It turns out that there are several possible interpretations
all enjoying the requisite algebraic properties of the operator (see
\cite{milner91polyadicpi}). We will therefore make liberal use of
$(\nu\; \vec{x})P$.

% subsection the_syntax_and_semantics_of_the_notation_system (end)   

\input{qm2pi.qmops} 

\input{qm2pi.sterngerlach} 

\input{qm2pi.metric} 

% section concurrent_process_calculi (end)

%\input{qm2pi.proofsketch}

% section proof sketch (end)

%\input{qm2pi.slviaknots} 

% section spatial logic via knots (end)

\input{qm2pi.conclusion}

% section conclusion (end)

%\input{qm2pi.dtcodes} 

% section wiring algorithm (end)

\input{qm2pi.ack} 

% section acknowledgments (end)

\newpage


\bibliographystyle{plain}   
\bibliography{../../biblios/main.bib}

\input{qm2pi.rhodetails}

\end{document}

 

% section concurrent_process_calculi (end)

%\documentclass[12pt]{llncs}
%\documentclass{jktr}

\usepackage[pdftex]{hyperref}                   
\usepackage {listings}
\usepackage {mathpartir}
\usepackage{bcprules}
%\usepackage{listings}
                       
\usepackage{graphicx} 
%\usepackage[margins=2.5cm,nohead,nofoot]{geometry}
%\usepackage{geometry}
\usepackage{amsfonts}
\usepackage{amstext}
\usepackage{latexsym}
\usepackage{amssymb}
\usepackage{color}


%\include{myPreamble}
\include{qm2pi.local} 

%\ifpdf
%\usepackage[pdftex]{graphicx}
%\else
%\usepackage{graphicx}
%\fi

 % \ifpdf
%  \usepackage{pdfsync}
%  \if


%\title{Brief Article}
%\author{David F. Snyder}
%\author{L.G. Meredith}

%\address{Dept. of Math., Texas State University--San Marcos, San Marcos, TX 78666}
       
\pagestyle{empty}


\begin{document}

\lstset{language=[Objective]Caml,frame=shadowbox}

\input{qm2pi.front}

% section front matter (end)

\input{qm2pi.intro} 
 
% section introduction (end)

% \input{qm2pi.knotations} 

% section notation (end)

\input{qm2pi.process.calculi} 

% section concurrent_process_calculi_and_spatial_logics_ (end)
    
%\input{qm2pi.knots2pi} 

%\input{qm2pi.trefoil} 

%\input{qm2pi.mainthm} 

% subsection basic_interpretation (end)

%\input{qm2pi.rho.presentation} 
\subsection{The syntax and semantics of the notation system}\label{sub:the_syntax_and_semantics_of_the_notation_system} % (fold)

We now summarize a technical presentation of the calculus that
embodies our theory of dynamics. The typical presentation of such a
calculus follows the style of giving generators and relations on
them. The grammar, below, describing term constructors, freely
generates the set of processes, $\Proc$. This set is then quotiented
by a relation known as structural congruence and it is over this set
that the notion of dynamics is expressed. This presentation is
essentially that of \cite{MeredithR05} with the addition of
polyadicity and summation. For readability we have relegated some of
the technical subtleties to an appendix.

\subsubsection{Process grammar}\label{subsub:process_grammar}

\begin{mathpar}
  \inferrule* [lab=synchronization] {} {{M} \bc \pzero \;|\; x?F \;|\; x!C }
  \and
  \inferrule* [lab=abstraction] {} {{F} \bc (x)P}
  \and
  \inferrule* [lab=concretion] {} {{C} \bc \langle Q \rangle}
  \and
  \inferrule* [lab=process] {} {{P,Q} \bc M \;| \;P|Q \;|\; @{x}}
  \and
  \inferrule* [lab=name] {} {{x} \bc \quotep{P}}
\end{mathpar} 

Note that $\vec{x}$ (resp. $\vec{P}$) denotes a vector of names
(resp. processes) of length $|\vec{x}|$ (resp. $|\vec{P}|$). We adopt
the following useful abbreviations.

\begin{mathpar}
   x?(\vec{y}).P := x.(\vec{y})P \and  x\clift{\vec{P}} := x.\clift{\vec{P}}
   \and x!(y) := \lift{x}{\dropn{y}}
   \and \Pi_{i=0}^{n-1}P_i := P_0 | \ldots | P_{n-1}
\end{mathpar}

\subsubsection{Structural congruence}

\paragraph{Free and bound names and alpha-equivalence.} At the
core of structural equivalence is alpha-equivalence which identifies
process that are the same up to a change of variable. Formally, we
recognize the distinction between free and bound names. The free names
of a process, $\freenames{P}$, may be calculated recursively as
follows:

\begin{mathpar}
\freenames{\pzero} := \emptyset
  \and \\
  \freenames{x?(y).P} := \{ x \} \cup (\freenames{P} \setminus \{ y \})
  \and 
  \freenames{x!\langle P \rangle} := \{ x \} \cup \{ P \} 
  \and \\
  \freenames{P|Q} := \freenames{P} \cup \freenames{Q}
  \and \\
  \freenames{@{x}} := \{ x \}
\end{mathpar}

$\pi$
$\quotep{\pi}$

$\freenames{-} : \pi \to \mathcal{P}(\quotep{\pi})$

\begin{eqnarray*}
  \freenames{\pzero} & := & \emptyset \\
  \freenames{x?(y).P} & := & \{ x \} \cup (\freenames{P} \setminus \{ y \}) \\
  \freenames{x!\langle P \rangle} & := & \{ x \} \cup \{ P \} \\
  \freenames{P|Q} & := & \freenames{P} \cup \freenames{Q} \\
  \freenames{\dropn{x}} & := & \{ x \}
\end{eqnarray*}

The bound names of a process, $\boundnames{P}$, are those names occurring in $P$
that are not free. For example, in $x?(y).0$, the name $x$ is free, while $y$ is bound.

\begin{mathpar}
  \inferrule* [lab=monoidal-laws] {} { P|Q \equiv Q|P \and P|0 \equiv P \and P|(Q|R) \equiv (P|Q)|R }
\end{mathpar}

\begin{mathpar}
  \inferrule* [lab=alpha-equivalence] {} { (x)P \equiv (y)P\{y/x\} \and y \not\in \freenames{P} }
\end{mathpar}

\begin{definition}
Then two processes, $P,Q$, are alpha-equivalent if $P = Q\{\vec{y}/\vec{x}\}$ for
some $\vec{x} \in \boundnames{Q},\vec{y} \in \boundnames{P}$, where $Q\{\vec{y}/\vec{x}\}$
denotes the capture-avoiding substitution of $\vec{y}$ for $\vec{x}$ in $Q$.
\end{definition}

\begin{definition}
  The {\em structural congruence} \cite{SangiorgiWalker} , $\equiv$,
  between processes is the least congruence containing
  alpha-equivalence, satisfying the abelian monoid laws
  (associativity, commutativity and $\pzero$ as identity) for parallel
  composition $|$ and for summation $+$.
\end{definition}

\subsection{Name equivalence}

We take name equivalence, written $\nameeq$, to be the smallest
equivalence relation generated by the following rules.

\begin{mathpar}
\inferrule*[lab=Quote-drop]
{ }
{ \quotep{@{x}} \nameeq x }

\inferrule*[lab=Struct-equiv]
{ P \scong Q }
{ \quotep{P} \nameeq \quotep{Q} }
\end{mathpar}

The astute reader will have noticed that the mutual recursion of names
and processes imposes a mutual recursion on alpha-equivalence and
structural equivalence via name-equivalence. Fortunately, all of this
works out pleasantly and we may calculate in the natural way, free of
concern. The reader interested in the details is referred to the
appendix \ref{appendix:rho_details}.

\subsection{Substitution}

We use $\Proc$ for the set of processes, $\QProc$ for the set of
names, and $\id{\{}\vec{y} / \vec{x} \id{\}}$ to denote partial maps,
$s : \QProc \rightarrow \QProc$. A map, $s$ lifts, uniquely, to a map
on process terms, $\widehat{s} : \Proc \rightarrow \Proc$ by the
following equations.

\begin{mathpar}
  (0) \psubstp{Q}{P} := 0 \\
  (R \juxtap S) \psubstp{Q}{P}
  :=    
  (R)\psubstp{Q}{P} \juxtap (S) \psubstp{Q}{P} \\
  (x?(y).R) \psubstp{Q}{P}    
  :=    
  (x)\substp{Q}{P} (z)\concat( (R \psubstn{z}{y}) \psubstp{Q}{P} ) \\
  (\lift{x}{R}) \psubstp{Q}{P}  
  :=
  \lift{(x)\substp{Q}{P}}{ R \psubstp{Q}{P} } \\
%   (\dropn{x})  \psubstp{Q}{P}       
%   := 
%   \left\{ 
%     \begin{array}{ccc} 
%       \dropn{\quotep{Q}} & & x \nameeq \quotep{P} \\
%       \dropn{x} & & otherwise \\
%     \end{array}
%   \right. 
  (\dropn{x})  \psubstp{Q}{P}       
  := 
  \left\{ 
    \begin{array}{ccc} 
      Q & & x \nameeq \quotep{P} \\
      \dropn{x} & & otherwise \\
    \end{array}
  \right.
\end{mathpar}
 

where

\begin{eqnarray}
  (x)\id{\{} \lpquote Q \rpquote / \lpquote P \rpquote \id{\}}            = 
  \left\{ 
    \begin{array}{ccc}
      \lpquote Q \rpquote & & x \nameeq \lpquote P \rpquote \\
      x & & otherwise \\
    \end{array}
  \right. \nonumber
\end{eqnarray}

and $z$ is chosen distinct from $\quotep{P}$, $\quotep{Q}$, the free
names in $Q$, and all the names in $R$. Our $\alpha$-equivalence will
be built in the standard way from this substitution.

\begin{remark}\label{rem:no_self_referential_names}
  One consequence of these definitions is that $\forall P. \quotep{P}
  \not\in \freenames{P}$.
\end{remark}

\subsection{ Dynamic quote: an example }

Anticipating something of what's to come, consider applying the
substitution, $\widehat{\id{\{}u / z \id{\}}}$, to the following pair
of processes, $\lift{w}{y!(z)}$ and $w[ \lpquote y!(z) \rpquote ]$.

\begin{eqnarray}
	\lift{w}{y!(z)}\widehat{\id{\{}u / z \id{\}}}
		& = &
		\lift{w}{y!(u)} \nonumber\\
	w[ \lpquote y!(z) \rpquote ] \widehat{ \id{\{}u / z \id{\}} }
		& = &
		w[ \lpquote y!(z) \rpquote ] \nonumber
\end{eqnarray}

Because the body of the process between quotes is impervious to
substitution, we get radically different answers. In fact, by
examining the first process in an input context,
e.g. $x?(z).\lift{w}{y!(z)}$, we see that the process under the lift
operator may be shaped by prefixed inputs binding a name inside it. In
this sense, the lift operator will be seen as a way to dynamically
construct processes before reifying them as names.

Finally equipped with these standard features we can present the
dynamics of the calculus.

\subsubsection{Operational semantics} 

Finally, we introduce the computational dynamics. What marks these
algebras as distinct from other more traditionally studied algebraic
structures, e.g. vector spaces or polynomial rings, is the manner in
which dynamics is captured. In traditional structures, dynamics is typically
expressed through morphisms between such structures, as in linear maps
between vector spaces or morphisms between rings. In algebras
associated with the semantics of computation, the dynamics is
expressed as part of the algebraic structure itself, through a
reduction reduction relation typically denoted by $\red$. Below, we
give a recursive presentation of this relation for the calculus used
in the encoding.

$\red \subseteq \pi \times \pi$
$\red : \pi \to \mathcal{P}(\pi)$

\begin{mathpar}
  \inferrule* [lab=Comm] { \textsf{match}( x_{src}, x_{trgt} ) } { x_{trgt}?(y)P \; | \; x_{src}!\langle {Q} \rangle \red P\{\quotep{Q}/y}\} }
  \and \\
  \inferrule* [lab=Par] {{P} \red {P}'} {{{P} | {Q}} \red {{P}' | {Q}}}
  \and
  \inferrule* [lab=Equiv]{{{P} \scong {P}'} \andalso {{P}' \red {Q}'} \andalso {{Q}' \scong {Q}}}{{P} \red {Q}}
\end{mathpar}

\begin{eqnarray*}
  match_{\equiv} (\quotep{P},\quotep{Q}) & := & P \equiv Q \\
  match_{\dagger}(\quotep{P},\quotep{Q}) & := & \forall R. P|Q \red^{*} R => R \red^{*} 0 \\
  match_{K}(\quotep{P},\quotep{Q}) & := & K \mbox{ for some context } K
\end{eqnarray*}

$u?(x)P | u!\langle Q \rangle \red P\{\quotep{Q}/x\}$

%We write $\wred$ for $\red^*$, and $P\red$ if $\exists Q $ such that $ P \red Q$.
We write $P\red$ if $\exists Q $ such that $ P \red Q$ and $P\not\red$, otherwise.

\section{Replication}

As mentioned before, it is known that replication (and hence
recursion) can be implemented in a higher-order process algebra
\cite{SangiorgiWalker}. As our first example of calculation with the
machinery thus far presented we give the construction explicitly in
the {\rhoc}.

\begin{eqnarray}
	D_{x} & := & \prefix{x}{y}{(\binpar{\outputp{x}{y}}{@{y}})} \nonumber\\
	\bangp_{x}{P} & := & \binpar{{x}!\langle{\binpar{D_{x}}{P}}\rangle}{D_{x}} \nonumber
\end{eqnarray}

\begin{eqnarray}
	\bangp_{x}{P} & & \nonumber\\
	=
	& {x}!\langle{(\prefix{x}{y}{(\outputp{x}{y} | @{y})) | P}}\rangle 
	      | \prefix{x}{y}{(\outputp{x}{y} | @{y})} & \nonumber\\
	\red
	& (\outputp{x}{y} | @{y})\substn{\quotep{(\prefix{x}{y}{(@{y} | \outputp{x}{y})) | P}}}{y} & \nonumber\\
	=
	& \outputp{x}{\quotep{(\prefix{x}{y}{(\outputp{x}{y} | @{y})) | P}}}
	  | {(\prefix{x}{y}{(\outputp{x}{y} | @{y})) | P}} & \nonumber\\
	\red
	& \ldots & \nonumber\\
	\red^*
	& P | P | \ldots & \nonumber
\end{eqnarray}

Of course, this encoding, as an implementation, runs away, unfolding
$\bangp{P}$ eagerly. A lazier and more implementable replication
operator, restricted to input-guarded processes, may be obtained as follows.

\begin{eqnarray}
\bangp{\prefix{u}{v}{P}} 
	:= 
	\binpar{\lift{x}{\prefix{u}{v}{(\binpar{D(x)}{P})}}}{D(x)} \nonumber
\end{eqnarray}

\begin{remark}
  Note that the lazier definition still does not deal with summation
  or mixed summation (i.e. sums over input and output). The reader is
  invited to construct definitions of replication that deal with these
  features. 

  Further, the definitions are parameterized in a name, $x$. Can you,
  gentle reader, make a definition that eliminates this parameter and
  guarantees no accidental interaction between the replication
  machinery and the process being replicated -- i.e. no accidental
  sharing of names used by the process to get its work done and the
  name(s) used by the replication to effect copying. This latter
  revision of the definition of replication is crucial to obtaining
  the expected identity $!!P \sim !P$.
\end{remark}

\begin{remark}\label{rem:paradoxical_combinator}
  The reader familiar with the lambda calculus will have noticed the
  similarity between $D$ and the paradoxical combinator.

  [Ed. note: the existence of this seems to suggest we have to be more
  restrictive on the set of processes and names we admit if we are to
  support no-cloning.]
\end{remark}

\subsubsection{Bisimulation}

The computational dynamics gives rise to another kind of equivalence,
the equivalence of computational behavior. As previously mentioned
this is typically captured \emph{via} some form of bisimulation.

% The notion we use in this paper is weak barbed bisimulation
% \cite{milner91polyadicpi}.

The notion we use in this paper is derived from weak barbed
bisimulation \cite{milner91polyadicpi}. 

\begin{definition}
An \emph{observation relation}, $\downarrow_{\mathcal N}$, over a set
of names, $\mathcal N$, is the smallest relation satisfying the rules
below.

\infrule[Out-barb]{y \in {\mathcal N}, \; x \nameeq y}
		  {\outputp{x}{v} \downarrow_{\mathcal N} x}
\infrule[Par-barb]{\mbox{$P\downarrow_{\mathcal N} x$ or $Q\downarrow_{\mathcal N} x$}}
		  {\binpar{P}{Q} \downarrow_{\mathcal N} x}

We write $P \Downarrow_{\mathcal N} x$ if there is $Q$ such that 
$P \wred Q$ and $Q \downarrow_{\mathcal N} x$.
\end{definition}

\begin{definition}
%\label{def.bbisim}
An  ${\mathcal N}$-\emph{barbed bisimulation} over a set of names, ${\mathcal N}$, is a symmetric binary relation 
${\mathcal S}_{\mathcal N}$ between agents such that $P\rel{S}_{\mathcal N}Q$ implies:
\begin{enumerate}
\item If $P \red P'$ then $Q \wred Q'$ and $P'\rel{S}_{\mathcal N} Q'$.
\item If $P\downarrow_{\mathcal N} x$, then $Q\Downarrow_{\mathcal N} x$.
\end{enumerate}
$P$ is ${\mathcal N}$-barbed bisimilar to $Q$, written
$P \wbbisim_{\mathcal N} Q$, if $P \rel{S}_{\mathcal N} Q$ for some ${\mathcal N}$-barbed bisimulation ${\mathcal S}_{\mathcal N}$.
\end{definition}

$\mathcal{R} \subseteq \pi \times \pi$

$P \mathcal{R} Q => \forall P'. P \red P' \Rightarrow \exists Q'. Q \red Q', P' \mathcal{R} Q'$

$P \vdash x \Rightarrow Q \vdash x$

\begin{mathpar}
  \inferrule*[lab=Out-barb]{x \nameeq y}{{y}!\langle{Q}\rangle \vdash x}
  \and
  \inferrule*[lab=Par-barb]{\mbox{$P\vdash x$ or $Q\vdash x$}}{\binpar{P}{Q} \vdash x}
\end{mathpar}

\subsubsection{Contexts}

One of the principle advantages of computational calculi like the
$\pi$-calculus is a well-defined notion of context,
contextual-equivalence and a correlation between
contextual-equivalence and notions of bisimulation. The notion of
context allows the decomposition of a process into (sub-)process and
its syntactic environment, its context. Thus, a context may be
thought of as a process with a ``hole'' (written $\Box$) in it. The
application of a context $M$ to a process $P$, written $M[P]$, is
tantamount to filling the hole in $M$ with $P$. In this paper we do
not need the full weight of this theory, but do make use of the notion
of context in the proof the main theorem. 

\begin{mathpar}
  \inferrule* [lab=summation] {} {{M_{M},M_{N}} \bc \Box \;|\; x.M_{A} \;|\; M_{M}+M_{N}}
  \and
  \inferrule* [lab=agent] {} {{M_{A}} \bc (\vec{x})M_{P} \;| \; \clift{P_0,\ldots,M_{P},\ldots,P_N}}
  \and \\
  \inferrule* [lab=process] {} {{M_{P}} \bc M_{N} \;| \;P|M_{P} }
\end{mathpar} 

\begin{mathpar}
  \inferrule* [lab=sychronization] {} {M_{N} \bc \Box \;|\; x?M_{F} \;|\; x!M_{C}}
  \and
  \inferrule* [lab=abstraction] {} {{M_{F}} \bc (x)M_{P} }
  \and
  \inferrule* [lab=concretion] {} {{M_{C}} \bc \langle M_{P} \rangle }
  \and \\
  \inferrule* [lab=process] {} {{M_{P}} \bc M_{N} \;| \;P|M_{P} }
\end{mathpar}

\begin{definition}[contextual application] Given a context $M$, and
  process $P$, we define the \emph{contextual application}, $M[P] :=
  M\{P/\Box\}$. That is, the contextual application of M to P is the
  substitution of $P$ for $\Box$ in $M$.
\end{definition}

$\meaningof{-} : L \to \mathcal{P}(\pi)$

\begin{mathpar}
  \inferrule* [lab=collection] {} {\meaningof{true} = \pi, \and \meaningof{~E} = \pi \setminus \meaningof{E}, \and \meaningof{E_{1} \& E_{2}} = \meaningof{E_{1}} \cap \meaningof{E_{2}}}
\end{mathpar}

\begin{mathpar}
  \inferrule* [lab=structure] {} {\meaningof{0} = \{ P \in \pi | P \equiv 0 \}, \and \\ \meaningof{E_1 | E_2} = \{ P \in \pi | P \equiv P_{1} | P_{2}, P_{1} \in \meaningof{E_{1}}, P_{2} \in \meaningof{E_2}\} }
\end{mathpar}

\begin{mathpar}
 \inferrule* [lab=behavior] {} {\meaningof{\langle a?b \rangle E} = \{ P \in \pi | P \equiv Q | u?(y)P', \\ \and \\\\ \and \\ \;\;\; u \in \meaningof{a}, \forall z.P'\{z/y\} \in \meaningof{E\{z/b\}}\}, \and \\ \meaningof{a!E} = \{ P \in \pi | P \equiv Q | x!\langle P' \rangle, x \in \meaningof{a} P' \in \meaningof{E}\} }
\end{mathpar}

\begin{mathpar}
 \inferrule* [lab=nominal] {} {\meaningof{\quotep{E}} = \{ \quotep{P} \in \quotep{\pi} | P \in \meaningof{E} \}, \and \meaningof{\quotep{P}} = \{ \quotep{Q} \in \quotep{\pi} | P \equiv Q \} \and \\ \meaningof{@\quotep{E}} = \{ P \in \pi | P \equiv @x, x \in \meaningof{E} \}}
\end{mathpar}

\begin{eqnarray*}
  \\
  \meaningof{-} : TS \to ST
\end{eqnarray*}

\begin{eqnarray*}
  \\
  L : TS \to ST
\end{eqnarray*}

\begin{eqnarray*}
  \\
  P \models E \iff P \in \meaningof{E}
\end{eqnarray*}

\begin{eqnarray*}
  P \approx_{L} Q \iff \forall E \in L. P \models E \iff Q \models E
\end{eqnarray*}

\begin{eqnarray*}
  P \approx_{K} Q
\end{eqnarray*}

\begin{eqnarray*}
  P \approx Q
\end{eqnarray*}

$\approx_{K} = \approx = \approx_{L}$

\subsubsection{Contextual duality}

Note that contexts extend the quotation operation to a family of
operations from processes to names. Given a context, $M$, we can
define a \emph{nominal context}, $\quotep{M}$ by $\quotep{M}[P] :=
\quotep{M[P]}$. To foreshadow what is to come we observe that these
operations enjoy a duality with processes very much like the duality
between vectors and maps from vectors to scalars.

Further, because the calculus is essentially higher-order, we have a
correspondence between contexts and processes. More specifically,
given a name $x$ and a context $M$ we can construct $M^{*}_{x}$ such
that 

\begin{mathpar}
  M^{*}_{x} | \lift{x}{P} \red M[P]
\end{mathpar}

namely,

\begin{mathpar}
  M^{*}_{x} := x?(u).M[\dropn{u}]
\end{mathpar}

The dependence of $M^{*}_{x}$ on a name makes it an abstraction, 

\begin{mathpar}
  M^{*} := (x)x?(u).M[\dropn{u}]
\end{mathpar}

\subsection{Additional notation}

It will sometimes be convenient to denote the process a name
quotes. We already have the notation $x = \quotep{P}$, but it will be
convenient to introduce an alternate notation, $\procn{x}$, when we
want to emphasize the connection to the use of the name. Note that, by
virtue of name equivalence, $\quotep{\procn{x}} \nameeq x$; so, the
notation is consistent with previous definitions.

Further, because names have structure it is possible to effect
substitutions on the basis of that structure. This means we need to
upgrade our notation for substitutions, which we accomplish by
adapting comprehension notation. Thus,

\begin{mathpar}
  P\{ y / x : x \in S \}
\end{mathpar}

is interpreted to mean the process derived from P by replacing (in a
capture-avoiding manner) each occurrence of $x$ in $S$ by $y$. For example,

\begin{mathpar}
  P\{ \quotep{\procn{x}|\procn{x}} / x : x \in \freenames{P} \}
\end{mathpar}

will replace each (occurrence) of a free name $x$ in $P$ by
$\quotep{\procn{x}|\procn{x}}$.

Also, we will avail ourselves of the notation $x^{L}$ and $x^{R}$ to
denote injections of a name into disjoint copies of the name
space. There are numerous ways to accomplish this. One example can be
found in \cite{MeredithR05}. This notation overloads to vectors of
names: $\vec{x}^{\pi} := (x_{i}^{\pi} \; : \; 0 \leq i < |\vec{x}| )$ where $\pi \in \{L,R\}$.

We also use $P^{\Box} := P|\Box$.

In \cite{MeredithR05} an interpretation of the new operator is
given. It turns out that there are several possible interpretations
all enjoying the requisite algebraic properties of the operator (see
\cite{milner91polyadicpi}). We will therefore make liberal use of
$(\nu\; \vec{x})P$.

% subsection the_syntax_and_semantics_of_the_notation_system (end)   

\input{qm2pi.qmops} 

\input{qm2pi.sterngerlach} 

\input{qm2pi.metric} 

% section concurrent_process_calculi (end)

%\input{qm2pi.proofsketch}

% section proof sketch (end)

%\input{qm2pi.slviaknots} 

% section spatial logic via knots (end)

\input{qm2pi.conclusion}

% section conclusion (end)

%\input{qm2pi.dtcodes} 

% section wiring algorithm (end)

\input{qm2pi.ack} 

% section acknowledgments (end)

\newpage


\bibliographystyle{plain}   
\bibliography{../../biblios/main.bib}

\input{qm2pi.rhodetails}

\end{document}



% section proof sketch (end)

%\section{Unlikely characters: spatial logic for
  knots}\label{sub:characteristic_formulae} % (fold)

Associated to the mobile process calculi are a family of logics known
as the Hennessy-Milner logics. These logics typically enjoy a
semantics interpreting formulae as sets of processes that when
factored through the encoding outlined above allows an identification
of classes of knots with logical formulae. In the context of this
encoding the sub-family known as the spatial logics \cite{CairesC03}
\cite{CairesC04} \cite{Caires04} are of particular interest providing
several important features for expressing and reasoning about
properties (i.e. classes) of knots. We hint here at how this may be done.

%\begin{description}
%\item [structural connectives] 
\subsubsection{Structural connectives} The spatial logics enjoy
structural connectives corresponding, at the logical level, to the
parallel composition ($P | Q$) and new name ($(\nu \; x)P$)
connectives for processes. As illustrated in the examples below, these
connectives are extremely expressive given the shape of our encoding.
%\item [decideable satisfaction]

\subsubsection{Decideable satisfaction}
In \cite{Caires04} the satisfaction relation is shown to be decideable
for a rich class of processes. It further turns out that the image of
the our encoding is a proper subset of that class. This result
provides the basis for an algorithm by which to search for knots
enjoying a given property.
%\item [characteristic formulae]

\subsubsection{Characteristic formulae}
In the same paper \cite{Caires04} , Caires presents a means of calculating
characteristic formulae, selecting equivalence classes of processes
up to a pre--specified depth limit on the support set of names. Composed with our
encoding, this characteristic formula can be used to select
characteristic formulae for knots.
%\end{description}

\subsubsection{Spatial logic formulae}

The grammar below (segmented for comprehension) summarizes the syntax
of spatial logic formulae. We employ illustrative examples in the
sequel to provide an intuitive understanding of their meaning
referring the reader to \cite{Caires04} for a more detailed explication
of the semantics.

\begin{mathpar}
  \inferrule* [lab=boolean] {} {{A,B} \bc T \;|\; \neg A \;|\; A \wedge B \;|\; \eta = \eta'}
  \and
  \inferrule* [lab=spatial] {} {|\; \pzero \;|\; A | B \;|\; x \text{\textregistered} A \;|\; \forall x . A \;|\;  H x . A}
  \and
  \inferrule* [lab=behavioral] {} {|\; \alpha . A}
  \and 
  \inferrule* [lab=recursion] {} {|\; X(\vec{u}) \;|\; \mu X(\vec{u}) . A}
  \and
  \inferrule* [lab=action] {} {\alpha \bc \langle x?(\vec{y}) \rangle \;|\; \langle x!(\vec{y}) \rangle \;|\; \langle \tau \rangle}
  \and 
  \inferrule* [lab=name] {} {\eta \bc x \;|\; \tau}
\end{mathpar} 

% subsection characteristic_formulae (end)   	 

\subsection{Example formulae}\label{sub:example_formulae_} % (fold)

\subsubsection{Crossing as formula.}
% 
% \begin{align*}
%   \frac{d}{dx} \sin x &= \cos x 
%   & \frac{d}{dx} e^x &= e^x \\
%   \frac{d}{dx} \cos x &= - \sin x 
%   & \frac{d}{dx} \log x &= \frac{1}{x} \\
% \end{align*} 

\begin{align*}
 \mu C(x_{0},x_{1},y_{0},y_{1},u).&(\langle x_{0}?(z) \rangle(\langle u! \rangle\langle y_{1}!z \rangle C(x_{0},x_{1},y_{0},y_{1},u)) & \\
  & \wedge \langle y_{1}?(z) \rangle (\langle u! \rangle \langle x_{0}!z \rangle C(x_{0},x_{1},y_{0},y_{1},u)) & \\
  & \wedge \langle x_{1}?(z) \rangle (\langle u? \rangle \langle y_{0}!z \rangle C(x_{0},x_{1},y_{0},y_{1},u)) & \\
  & \wedge \langle y_{0}?(z) \rangle (\langle u? \rangle \langle x_{1}!z \rangle C(x_{0},x_{1},y_{0},y_{1},u))) &
\end{align*}

The lexicographical similarity between the shape of this formulae and
the shape of definition of the process representing a crossing reveals
the intuitive meaning of this formulae. It describes the capabilities
of a process that has the right to represent a crossing. For example
it picks out processes that may perform an input on the port $x_0$ in
its initial menu of capabilities. What differentiates the formula
from the process, however, is that the crossing process is the
smallest candidate to satisfy the formula. Infinitely many other
processes -- with internal behavior hidden behind this interface, so
to speak -- also satisfy this formula. Even this simple formula,
then, can be seen to open a new view onto knots, providing a
computational interpretation of \emph{virtual} knots.

Note that this formula is derived by hand. A similar formula can be
derived by employing Caires' calculation of characteristic formula
\cite{Caires04} to the process representing a crossing. In light of
this discussion, we let
$\meaningof{C}_{\phi}(x0,x1,y0,y1,u)$ denote a formula specifying the
dynamics we wish to capture of a crossing. To guarantee we preserve
the shape of the interface and minimal semantics we demand that
$\meaningof{C}_{\phi}(x0,x1,y0,y1,u) \Rightarrow
\textbf{C}(x0,x1,y0,y1,u)$ where $\textbf{C}(x0,x1,y0,y1,u)$ denotes
the formula above.
                            
\subsubsection{Crossing number constraints.}
The moral content of the context lemma (Lemma \ref{context}) is that the notion of
``locality'' in the Reidemeister moves is effectively captured by the
parallel composition operator of the process calculus. This intuition
extends through the logic. Given a formula,
$\meaningof{C}_{\phi}(x0,x1,y0,y1,u)$, we can use the structural
connectives to specify constraints on crossing numbers, such as at
least $n$ crossings, or exactly $n$ crossings.
\begin{mathpar}
  \inferrule* [lab=at-least-n] {} { K^{\geq n}_{\phi}(\vec{xs},\vec{ys}) := \Pi_{i=0}^{n-1} Hu . \meaningof{C}_{\phi}(xs_i,ys_i,u) | T }
  \and 
  \inferrule* [lab=exactly-n] {} { K^{= n}_{\phi}(\vec{xs},\vec{ys}) := \Pi_{i=0}^{n-1} Hu . \meaningof{C}_{\phi}(xs_i,ys_i,u) | \neg (\forall x_0,y_0,x_1,y_1,u . \meaningof{C}_{\phi}(x_0,y_0,x_1,y_1,u) | T) }
\end{mathpar}

To round out this section, recall that the encoding of an $n$-crossing
knot decomposes into a parallel composition of $n$ \emph{copies} of a
crossing process together with a wiring harness. To specify different
knot classes with the same crossing number amounts to specifying
logical constraints on the wiring harness. In the interest of space,
we defer examples to a forthcoming paper. Suffice it to say that both
the conditions ``alternating knot'' and ``contains the tangle
corresponding to 5/3'' are expressible. For example, it is possible to
calculate the characteristic formula of a process corresponding to the
tangle 5/3 and conjoin it into the classifying formula via the
composition connective of the logic.

Finally, we wish to observe that it is entirely within reason to
contemplate a more domain-specific version of spatial logic tailored
to the shape of processes in the image of the encoding. Such a
domain-specific logic would have a better claim to the title formal
language of knot properties.

% subsection example_formulae_ (end)

% section knots_as_processes (end) 

% section spatial logic via knots (end)

\section{Conclusions and future work}

\paragraph{Testing physical space}
You, gentle reader, may wonder why of all the theorems to be proved
given this set up we pick the one above. In some sense it's hardly
central to quantum mechanics. We see it as central in the sense that
it firmly establishes a notion of physical space arising from a notion
of the equivalence of behavior. Relating bisimulation to a metric is a
big step forward, but one is faced with interpreting the relationship
of that metric space to something more physical. Quantum mechanical
notions of ``physical'' space are still far from intuitive, but by
relating this idea of distance as testing to calculations that predict
physical circumstances we are making a not insignificant step forward
toward an understanding of the physical space we inhabit as
essentially dynamic.

\paragraph{Effectivity and simulation}
One of the observations we have yet to make is that the entire program
spelled out here is effective. We have built various interpreters for
the reflective calculus at work in this interpretation. In principle,
then, we can simulate quantum mechanics on a computer. The place where
the simulation may lose fidelity is the infinitely branching summation
for the annihilator.

In this connection i also want to point out that the evaluation style
calculation of the inner product puts the non-determinism of the
summation right at the heart of measurement. This suggests that
Milner's original reduction-based formulation of the dynamics of his
calculi in terms of sums was not just notationally suggestive of a
notion of measure-and-continue but captured some significant part of
the physics.

\paragraph{Quantum continuations}
In light of this last observation i want to point out that the
predominant account of quantum mechanics is missing a key aspect of a
truly compositional story of the physical situation. In a real lab,
when a measurement is made the observation can be made to feed into
another device that then makes another measurement conditioned on the
results of the first. This means that after the superposition was
collapsed the entire experimental set up remained in
superposition. While QM offers a means of writing this down it doesn't
quite line up well with the well-trodden formulation of computation
and continuation that we see so succinctly expressed in Milner's
calculi. This suggests that there might be advantages to this account
of dynamics waiting to be explored.

\paragraph{Quantum logic}
In this connection, we also note that by virtue of having the
Hennessy-Milner construction, we can pull the construction through the
interpretation of QM. This gives us a natural candidate for a quantum
logic that enjoys an extremely tight connection with it's domain of
interpretation, making the construction much less ad hoc (rather it is
the image of functor!).

\paragraph{Quantum probabiity}
i have questions about the basis of the interpretation of inner
product as probability amplitude. In particular, using which
axiomatization of probability theory does the notion of probability
amplitude earn the right to be so dubbed? In other words, where is the
proof that the operation for calculating a probability amplitude (and
then squaring) satisfies the axioms of what it means to calculate a
probability? Even if such a proof exists (i have yet to find it in the
literature), i wonder if it might not be possible to turn things on
their heads. Can we view the calculation of the probability amplitude
as an axiomatization of probability? If so, then the definition we
give for calculating probability amplitude may provide the basis for
an \emph{effective} theory of probability.

\paragraph{Quantum vs ``biological'' information}
Finally, i want to conclude with a more philosophical observation. At
a recent workshop in which QM was a predominant topic i noticed
something about quantum information. The speaker was giving a riveting
discussion of axiomatic QM and showing how properties of ``no
cloning'' and ``no deleting'' emerged as consequences of the
axiomatization. Theorems of this form are necessary to give us a sense
of confidence that our axioms characterize the physical theory. What
struck me, though, was that if quantum information is neither erasable
nor replicable it is markedly different from \emph{life}. Two of the
things we know about life is that

\begin{itemize}
  \item it ends;
  \item to gain some measure of persistence, to transcend it's
    finitude it is imminently copyable.
\end{itemize}

Both of these qualities are summarized succinctly in the aphorism: all
flesh is grass. For me these two kinds of ``information'' -- call them
quantum and biological -- are end points on a spectrum of strategies
for persistence. At one end, we have those curious entities that enjoy
uniqueness and permanence; at the other, we have those who in the face
of a certain end and an uncertain present make a go of passing
something on. To me one of the more remarkable aspects of the latter
strategy is that in the presence of noise (and certain features of
copying) we get a kind of dynamism, a chance for improvement against a
given persistent condition.

% subsection other_calculi_other_bisimulations_and_geometry_as_behavior (end)




% section conclusion (end)

%\documentclass[12pt]{llncs}
%\documentclass{jktr}

\usepackage[pdftex]{hyperref}                   
\usepackage {listings}
\usepackage {mathpartir}
\usepackage{bcprules}
%\usepackage{listings}
                       
\usepackage{graphicx} 
%\usepackage[margins=2.5cm,nohead,nofoot]{geometry}
%\usepackage{geometry}
\usepackage{amsfonts}
\usepackage{amstext}
\usepackage{latexsym}
\usepackage{amssymb}
\usepackage{color}


%\include{myPreamble}
\include{qm2pi.local} 

%\ifpdf
%\usepackage[pdftex]{graphicx}
%\else
%\usepackage{graphicx}
%\fi

 % \ifpdf
%  \usepackage{pdfsync}
%  \if


%\title{Brief Article}
%\author{David F. Snyder}
%\author{L.G. Meredith}

%\address{Dept. of Math., Texas State University--San Marcos, San Marcos, TX 78666}
       
\pagestyle{empty}


\begin{document}

\lstset{language=[Objective]Caml,frame=shadowbox}

\input{qm2pi.front}

% section front matter (end)

\input{qm2pi.intro} 
 
% section introduction (end)

% \input{qm2pi.knotations} 

% section notation (end)

\input{qm2pi.process.calculi} 

% section concurrent_process_calculi_and_spatial_logics_ (end)
    
%\input{qm2pi.knots2pi} 

%\input{qm2pi.trefoil} 

%\input{qm2pi.mainthm} 

% subsection basic_interpretation (end)

%\input{qm2pi.rho.presentation} 
\subsection{The syntax and semantics of the notation system}\label{sub:the_syntax_and_semantics_of_the_notation_system} % (fold)

We now summarize a technical presentation of the calculus that
embodies our theory of dynamics. The typical presentation of such a
calculus follows the style of giving generators and relations on
them. The grammar, below, describing term constructors, freely
generates the set of processes, $\Proc$. This set is then quotiented
by a relation known as structural congruence and it is over this set
that the notion of dynamics is expressed. This presentation is
essentially that of \cite{MeredithR05} with the addition of
polyadicity and summation. For readability we have relegated some of
the technical subtleties to an appendix.

\subsubsection{Process grammar}\label{subsub:process_grammar}

\begin{mathpar}
  \inferrule* [lab=synchronization] {} {{M} \bc \pzero \;|\; x?F \;|\; x!C }
  \and
  \inferrule* [lab=abstraction] {} {{F} \bc (x)P}
  \and
  \inferrule* [lab=concretion] {} {{C} \bc \langle Q \rangle}
  \and
  \inferrule* [lab=process] {} {{P,Q} \bc M \;| \;P|Q \;|\; @{x}}
  \and
  \inferrule* [lab=name] {} {{x} \bc \quotep{P}}
\end{mathpar} 

Note that $\vec{x}$ (resp. $\vec{P}$) denotes a vector of names
(resp. processes) of length $|\vec{x}|$ (resp. $|\vec{P}|$). We adopt
the following useful abbreviations.

\begin{mathpar}
   x?(\vec{y}).P := x.(\vec{y})P \and  x\clift{\vec{P}} := x.\clift{\vec{P}}
   \and x!(y) := \lift{x}{\dropn{y}}
   \and \Pi_{i=0}^{n-1}P_i := P_0 | \ldots | P_{n-1}
\end{mathpar}

\subsubsection{Structural congruence}

\paragraph{Free and bound names and alpha-equivalence.} At the
core of structural equivalence is alpha-equivalence which identifies
process that are the same up to a change of variable. Formally, we
recognize the distinction between free and bound names. The free names
of a process, $\freenames{P}$, may be calculated recursively as
follows:

\begin{mathpar}
\freenames{\pzero} := \emptyset
  \and \\
  \freenames{x?(y).P} := \{ x \} \cup (\freenames{P} \setminus \{ y \})
  \and 
  \freenames{x!\langle P \rangle} := \{ x \} \cup \{ P \} 
  \and \\
  \freenames{P|Q} := \freenames{P} \cup \freenames{Q}
  \and \\
  \freenames{@{x}} := \{ x \}
\end{mathpar}

$\pi$
$\quotep{\pi}$

$\freenames{-} : \pi \to \mathcal{P}(\quotep{\pi})$

\begin{eqnarray*}
  \freenames{\pzero} & := & \emptyset \\
  \freenames{x?(y).P} & := & \{ x \} \cup (\freenames{P} \setminus \{ y \}) \\
  \freenames{x!\langle P \rangle} & := & \{ x \} \cup \{ P \} \\
  \freenames{P|Q} & := & \freenames{P} \cup \freenames{Q} \\
  \freenames{\dropn{x}} & := & \{ x \}
\end{eqnarray*}

The bound names of a process, $\boundnames{P}$, are those names occurring in $P$
that are not free. For example, in $x?(y).0$, the name $x$ is free, while $y$ is bound.

\begin{mathpar}
  \inferrule* [lab=monoidal-laws] {} { P|Q \equiv Q|P \and P|0 \equiv P \and P|(Q|R) \equiv (P|Q)|R }
\end{mathpar}

\begin{mathpar}
  \inferrule* [lab=alpha-equivalence] {} { (x)P \equiv (y)P\{y/x\} \and y \not\in \freenames{P} }
\end{mathpar}

\begin{definition}
Then two processes, $P,Q$, are alpha-equivalent if $P = Q\{\vec{y}/\vec{x}\}$ for
some $\vec{x} \in \boundnames{Q},\vec{y} \in \boundnames{P}$, where $Q\{\vec{y}/\vec{x}\}$
denotes the capture-avoiding substitution of $\vec{y}$ for $\vec{x}$ in $Q$.
\end{definition}

\begin{definition}
  The {\em structural congruence} \cite{SangiorgiWalker} , $\equiv$,
  between processes is the least congruence containing
  alpha-equivalence, satisfying the abelian monoid laws
  (associativity, commutativity and $\pzero$ as identity) for parallel
  composition $|$ and for summation $+$.
\end{definition}

\subsection{Name equivalence}

We take name equivalence, written $\nameeq$, to be the smallest
equivalence relation generated by the following rules.

\begin{mathpar}
\inferrule*[lab=Quote-drop]
{ }
{ \quotep{@{x}} \nameeq x }

\inferrule*[lab=Struct-equiv]
{ P \scong Q }
{ \quotep{P} \nameeq \quotep{Q} }
\end{mathpar}

The astute reader will have noticed that the mutual recursion of names
and processes imposes a mutual recursion on alpha-equivalence and
structural equivalence via name-equivalence. Fortunately, all of this
works out pleasantly and we may calculate in the natural way, free of
concern. The reader interested in the details is referred to the
appendix \ref{appendix:rho_details}.

\subsection{Substitution}

We use $\Proc$ for the set of processes, $\QProc$ for the set of
names, and $\id{\{}\vec{y} / \vec{x} \id{\}}$ to denote partial maps,
$s : \QProc \rightarrow \QProc$. A map, $s$ lifts, uniquely, to a map
on process terms, $\widehat{s} : \Proc \rightarrow \Proc$ by the
following equations.

\begin{mathpar}
  (0) \psubstp{Q}{P} := 0 \\
  (R \juxtap S) \psubstp{Q}{P}
  :=    
  (R)\psubstp{Q}{P} \juxtap (S) \psubstp{Q}{P} \\
  (x?(y).R) \psubstp{Q}{P}    
  :=    
  (x)\substp{Q}{P} (z)\concat( (R \psubstn{z}{y}) \psubstp{Q}{P} ) \\
  (\lift{x}{R}) \psubstp{Q}{P}  
  :=
  \lift{(x)\substp{Q}{P}}{ R \psubstp{Q}{P} } \\
%   (\dropn{x})  \psubstp{Q}{P}       
%   := 
%   \left\{ 
%     \begin{array}{ccc} 
%       \dropn{\quotep{Q}} & & x \nameeq \quotep{P} \\
%       \dropn{x} & & otherwise \\
%     \end{array}
%   \right. 
  (\dropn{x})  \psubstp{Q}{P}       
  := 
  \left\{ 
    \begin{array}{ccc} 
      Q & & x \nameeq \quotep{P} \\
      \dropn{x} & & otherwise \\
    \end{array}
  \right.
\end{mathpar}
 

where

\begin{eqnarray}
  (x)\id{\{} \lpquote Q \rpquote / \lpquote P \rpquote \id{\}}            = 
  \left\{ 
    \begin{array}{ccc}
      \lpquote Q \rpquote & & x \nameeq \lpquote P \rpquote \\
      x & & otherwise \\
    \end{array}
  \right. \nonumber
\end{eqnarray}

and $z$ is chosen distinct from $\quotep{P}$, $\quotep{Q}$, the free
names in $Q$, and all the names in $R$. Our $\alpha$-equivalence will
be built in the standard way from this substitution.

\begin{remark}\label{rem:no_self_referential_names}
  One consequence of these definitions is that $\forall P. \quotep{P}
  \not\in \freenames{P}$.
\end{remark}

\subsection{ Dynamic quote: an example }

Anticipating something of what's to come, consider applying the
substitution, $\widehat{\id{\{}u / z \id{\}}}$, to the following pair
of processes, $\lift{w}{y!(z)}$ and $w[ \lpquote y!(z) \rpquote ]$.

\begin{eqnarray}
	\lift{w}{y!(z)}\widehat{\id{\{}u / z \id{\}}}
		& = &
		\lift{w}{y!(u)} \nonumber\\
	w[ \lpquote y!(z) \rpquote ] \widehat{ \id{\{}u / z \id{\}} }
		& = &
		w[ \lpquote y!(z) \rpquote ] \nonumber
\end{eqnarray}

Because the body of the process between quotes is impervious to
substitution, we get radically different answers. In fact, by
examining the first process in an input context,
e.g. $x?(z).\lift{w}{y!(z)}$, we see that the process under the lift
operator may be shaped by prefixed inputs binding a name inside it. In
this sense, the lift operator will be seen as a way to dynamically
construct processes before reifying them as names.

Finally equipped with these standard features we can present the
dynamics of the calculus.

\subsubsection{Operational semantics} 

Finally, we introduce the computational dynamics. What marks these
algebras as distinct from other more traditionally studied algebraic
structures, e.g. vector spaces or polynomial rings, is the manner in
which dynamics is captured. In traditional structures, dynamics is typically
expressed through morphisms between such structures, as in linear maps
between vector spaces or morphisms between rings. In algebras
associated with the semantics of computation, the dynamics is
expressed as part of the algebraic structure itself, through a
reduction reduction relation typically denoted by $\red$. Below, we
give a recursive presentation of this relation for the calculus used
in the encoding.

$\red \subseteq \pi \times \pi$
$\red : \pi \to \mathcal{P}(\pi)$

\begin{mathpar}
  \inferrule* [lab=Comm] { \textsf{match}( x_{src}, x_{trgt} ) } { x_{trgt}?(y)P \; | \; x_{src}!\langle {Q} \rangle \red P\{\quotep{Q}/y}\} }
  \and \\
  \inferrule* [lab=Par] {{P} \red {P}'} {{{P} | {Q}} \red {{P}' | {Q}}}
  \and
  \inferrule* [lab=Equiv]{{{P} \scong {P}'} \andalso {{P}' \red {Q}'} \andalso {{Q}' \scong {Q}}}{{P} \red {Q}}
\end{mathpar}

\begin{eqnarray*}
  match_{\equiv} (\quotep{P},\quotep{Q}) & := & P \equiv Q \\
  match_{\dagger}(\quotep{P},\quotep{Q}) & := & \forall R. P|Q \red^{*} R => R \red^{*} 0 \\
  match_{K}(\quotep{P},\quotep{Q}) & := & K \mbox{ for some context } K
\end{eqnarray*}

$u?(x)P | u!\langle Q \rangle \red P\{\quotep{Q}/x\}$

%We write $\wred$ for $\red^*$, and $P\red$ if $\exists Q $ such that $ P \red Q$.
We write $P\red$ if $\exists Q $ such that $ P \red Q$ and $P\not\red$, otherwise.

\section{Replication}

As mentioned before, it is known that replication (and hence
recursion) can be implemented in a higher-order process algebra
\cite{SangiorgiWalker}. As our first example of calculation with the
machinery thus far presented we give the construction explicitly in
the {\rhoc}.

\begin{eqnarray}
	D_{x} & := & \prefix{x}{y}{(\binpar{\outputp{x}{y}}{@{y}})} \nonumber\\
	\bangp_{x}{P} & := & \binpar{{x}!\langle{\binpar{D_{x}}{P}}\rangle}{D_{x}} \nonumber
\end{eqnarray}

\begin{eqnarray}
	\bangp_{x}{P} & & \nonumber\\
	=
	& {x}!\langle{(\prefix{x}{y}{(\outputp{x}{y} | @{y})) | P}}\rangle 
	      | \prefix{x}{y}{(\outputp{x}{y} | @{y})} & \nonumber\\
	\red
	& (\outputp{x}{y} | @{y})\substn{\quotep{(\prefix{x}{y}{(@{y} | \outputp{x}{y})) | P}}}{y} & \nonumber\\
	=
	& \outputp{x}{\quotep{(\prefix{x}{y}{(\outputp{x}{y} | @{y})) | P}}}
	  | {(\prefix{x}{y}{(\outputp{x}{y} | @{y})) | P}} & \nonumber\\
	\red
	& \ldots & \nonumber\\
	\red^*
	& P | P | \ldots & \nonumber
\end{eqnarray}

Of course, this encoding, as an implementation, runs away, unfolding
$\bangp{P}$ eagerly. A lazier and more implementable replication
operator, restricted to input-guarded processes, may be obtained as follows.

\begin{eqnarray}
\bangp{\prefix{u}{v}{P}} 
	:= 
	\binpar{\lift{x}{\prefix{u}{v}{(\binpar{D(x)}{P})}}}{D(x)} \nonumber
\end{eqnarray}

\begin{remark}
  Note that the lazier definition still does not deal with summation
  or mixed summation (i.e. sums over input and output). The reader is
  invited to construct definitions of replication that deal with these
  features. 

  Further, the definitions are parameterized in a name, $x$. Can you,
  gentle reader, make a definition that eliminates this parameter and
  guarantees no accidental interaction between the replication
  machinery and the process being replicated -- i.e. no accidental
  sharing of names used by the process to get its work done and the
  name(s) used by the replication to effect copying. This latter
  revision of the definition of replication is crucial to obtaining
  the expected identity $!!P \sim !P$.
\end{remark}

\begin{remark}\label{rem:paradoxical_combinator}
  The reader familiar with the lambda calculus will have noticed the
  similarity between $D$ and the paradoxical combinator.

  [Ed. note: the existence of this seems to suggest we have to be more
  restrictive on the set of processes and names we admit if we are to
  support no-cloning.]
\end{remark}

\subsubsection{Bisimulation}

The computational dynamics gives rise to another kind of equivalence,
the equivalence of computational behavior. As previously mentioned
this is typically captured \emph{via} some form of bisimulation.

% The notion we use in this paper is weak barbed bisimulation
% \cite{milner91polyadicpi}.

The notion we use in this paper is derived from weak barbed
bisimulation \cite{milner91polyadicpi}. 

\begin{definition}
An \emph{observation relation}, $\downarrow_{\mathcal N}$, over a set
of names, $\mathcal N$, is the smallest relation satisfying the rules
below.

\infrule[Out-barb]{y \in {\mathcal N}, \; x \nameeq y}
		  {\outputp{x}{v} \downarrow_{\mathcal N} x}
\infrule[Par-barb]{\mbox{$P\downarrow_{\mathcal N} x$ or $Q\downarrow_{\mathcal N} x$}}
		  {\binpar{P}{Q} \downarrow_{\mathcal N} x}

We write $P \Downarrow_{\mathcal N} x$ if there is $Q$ such that 
$P \wred Q$ and $Q \downarrow_{\mathcal N} x$.
\end{definition}

\begin{definition}
%\label{def.bbisim}
An  ${\mathcal N}$-\emph{barbed bisimulation} over a set of names, ${\mathcal N}$, is a symmetric binary relation 
${\mathcal S}_{\mathcal N}$ between agents such that $P\rel{S}_{\mathcal N}Q$ implies:
\begin{enumerate}
\item If $P \red P'$ then $Q \wred Q'$ and $P'\rel{S}_{\mathcal N} Q'$.
\item If $P\downarrow_{\mathcal N} x$, then $Q\Downarrow_{\mathcal N} x$.
\end{enumerate}
$P$ is ${\mathcal N}$-barbed bisimilar to $Q$, written
$P \wbbisim_{\mathcal N} Q$, if $P \rel{S}_{\mathcal N} Q$ for some ${\mathcal N}$-barbed bisimulation ${\mathcal S}_{\mathcal N}$.
\end{definition}

$\mathcal{R} \subseteq \pi \times \pi$

$P \mathcal{R} Q => \forall P'. P \red P' \Rightarrow \exists Q'. Q \red Q', P' \mathcal{R} Q'$

$P \vdash x \Rightarrow Q \vdash x$

\begin{mathpar}
  \inferrule*[lab=Out-barb]{x \nameeq y}{{y}!\langle{Q}\rangle \vdash x}
  \and
  \inferrule*[lab=Par-barb]{\mbox{$P\vdash x$ or $Q\vdash x$}}{\binpar{P}{Q} \vdash x}
\end{mathpar}

\subsubsection{Contexts}

One of the principle advantages of computational calculi like the
$\pi$-calculus is a well-defined notion of context,
contextual-equivalence and a correlation between
contextual-equivalence and notions of bisimulation. The notion of
context allows the decomposition of a process into (sub-)process and
its syntactic environment, its context. Thus, a context may be
thought of as a process with a ``hole'' (written $\Box$) in it. The
application of a context $M$ to a process $P$, written $M[P]$, is
tantamount to filling the hole in $M$ with $P$. In this paper we do
not need the full weight of this theory, but do make use of the notion
of context in the proof the main theorem. 

\begin{mathpar}
  \inferrule* [lab=summation] {} {{M_{M},M_{N}} \bc \Box \;|\; x.M_{A} \;|\; M_{M}+M_{N}}
  \and
  \inferrule* [lab=agent] {} {{M_{A}} \bc (\vec{x})M_{P} \;| \; \clift{P_0,\ldots,M_{P},\ldots,P_N}}
  \and \\
  \inferrule* [lab=process] {} {{M_{P}} \bc M_{N} \;| \;P|M_{P} }
\end{mathpar} 

\begin{mathpar}
  \inferrule* [lab=sychronization] {} {M_{N} \bc \Box \;|\; x?M_{F} \;|\; x!M_{C}}
  \and
  \inferrule* [lab=abstraction] {} {{M_{F}} \bc (x)M_{P} }
  \and
  \inferrule* [lab=concretion] {} {{M_{C}} \bc \langle M_{P} \rangle }
  \and \\
  \inferrule* [lab=process] {} {{M_{P}} \bc M_{N} \;| \;P|M_{P} }
\end{mathpar}

\begin{definition}[contextual application] Given a context $M$, and
  process $P$, we define the \emph{contextual application}, $M[P] :=
  M\{P/\Box\}$. That is, the contextual application of M to P is the
  substitution of $P$ for $\Box$ in $M$.
\end{definition}

$\meaningof{-} : L \to \mathcal{P}(\pi)$

\begin{mathpar}
  \inferrule* [lab=collection] {} {\meaningof{true} = \pi, \and \meaningof{~E} = \pi \setminus \meaningof{E}, \and \meaningof{E_{1} \& E_{2}} = \meaningof{E_{1}} \cap \meaningof{E_{2}}}
\end{mathpar}

\begin{mathpar}
  \inferrule* [lab=structure] {} {\meaningof{0} = \{ P \in \pi | P \equiv 0 \}, \and \\ \meaningof{E_1 | E_2} = \{ P \in \pi | P \equiv P_{1} | P_{2}, P_{1} \in \meaningof{E_{1}}, P_{2} \in \meaningof{E_2}\} }
\end{mathpar}

\begin{mathpar}
 \inferrule* [lab=behavior] {} {\meaningof{\langle a?b \rangle E} = \{ P \in \pi | P \equiv Q | u?(y)P', \\ \and \\\\ \and \\ \;\;\; u \in \meaningof{a}, \forall z.P'\{z/y\} \in \meaningof{E\{z/b\}}\}, \and \\ \meaningof{a!E} = \{ P \in \pi | P \equiv Q | x!\langle P' \rangle, x \in \meaningof{a} P' \in \meaningof{E}\} }
\end{mathpar}

\begin{mathpar}
 \inferrule* [lab=nominal] {} {\meaningof{\quotep{E}} = \{ \quotep{P} \in \quotep{\pi} | P \in \meaningof{E} \}, \and \meaningof{\quotep{P}} = \{ \quotep{Q} \in \quotep{\pi} | P \equiv Q \} \and \\ \meaningof{@\quotep{E}} = \{ P \in \pi | P \equiv @x, x \in \meaningof{E} \}}
\end{mathpar}

\begin{eqnarray*}
  \\
  \meaningof{-} : TS \to ST
\end{eqnarray*}

\begin{eqnarray*}
  \\
  L : TS \to ST
\end{eqnarray*}

\begin{eqnarray*}
  \\
  P \models E \iff P \in \meaningof{E}
\end{eqnarray*}

\begin{eqnarray*}
  P \approx_{L} Q \iff \forall E \in L. P \models E \iff Q \models E
\end{eqnarray*}

\begin{eqnarray*}
  P \approx_{K} Q
\end{eqnarray*}

\begin{eqnarray*}
  P \approx Q
\end{eqnarray*}

$\approx_{K} = \approx = \approx_{L}$

\subsubsection{Contextual duality}

Note that contexts extend the quotation operation to a family of
operations from processes to names. Given a context, $M$, we can
define a \emph{nominal context}, $\quotep{M}$ by $\quotep{M}[P] :=
\quotep{M[P]}$. To foreshadow what is to come we observe that these
operations enjoy a duality with processes very much like the duality
between vectors and maps from vectors to scalars.

Further, because the calculus is essentially higher-order, we have a
correspondence between contexts and processes. More specifically,
given a name $x$ and a context $M$ we can construct $M^{*}_{x}$ such
that 

\begin{mathpar}
  M^{*}_{x} | \lift{x}{P} \red M[P]
\end{mathpar}

namely,

\begin{mathpar}
  M^{*}_{x} := x?(u).M[\dropn{u}]
\end{mathpar}

The dependence of $M^{*}_{x}$ on a name makes it an abstraction, 

\begin{mathpar}
  M^{*} := (x)x?(u).M[\dropn{u}]
\end{mathpar}

\subsection{Additional notation}

It will sometimes be convenient to denote the process a name
quotes. We already have the notation $x = \quotep{P}$, but it will be
convenient to introduce an alternate notation, $\procn{x}$, when we
want to emphasize the connection to the use of the name. Note that, by
virtue of name equivalence, $\quotep{\procn{x}} \nameeq x$; so, the
notation is consistent with previous definitions.

Further, because names have structure it is possible to effect
substitutions on the basis of that structure. This means we need to
upgrade our notation for substitutions, which we accomplish by
adapting comprehension notation. Thus,

\begin{mathpar}
  P\{ y / x : x \in S \}
\end{mathpar}

is interpreted to mean the process derived from P by replacing (in a
capture-avoiding manner) each occurrence of $x$ in $S$ by $y$. For example,

\begin{mathpar}
  P\{ \quotep{\procn{x}|\procn{x}} / x : x \in \freenames{P} \}
\end{mathpar}

will replace each (occurrence) of a free name $x$ in $P$ by
$\quotep{\procn{x}|\procn{x}}$.

Also, we will avail ourselves of the notation $x^{L}$ and $x^{R}$ to
denote injections of a name into disjoint copies of the name
space. There are numerous ways to accomplish this. One example can be
found in \cite{MeredithR05}. This notation overloads to vectors of
names: $\vec{x}^{\pi} := (x_{i}^{\pi} \; : \; 0 \leq i < |\vec{x}| )$ where $\pi \in \{L,R\}$.

We also use $P^{\Box} := P|\Box$.

In \cite{MeredithR05} an interpretation of the new operator is
given. It turns out that there are several possible interpretations
all enjoying the requisite algebraic properties of the operator (see
\cite{milner91polyadicpi}). We will therefore make liberal use of
$(\nu\; \vec{x})P$.

% subsection the_syntax_and_semantics_of_the_notation_system (end)   

\input{qm2pi.qmops} 

\input{qm2pi.sterngerlach} 

\input{qm2pi.metric} 

% section concurrent_process_calculi (end)

%\input{qm2pi.proofsketch}

% section proof sketch (end)

%\input{qm2pi.slviaknots} 

% section spatial logic via knots (end)

\input{qm2pi.conclusion}

% section conclusion (end)

%\input{qm2pi.dtcodes} 

% section wiring algorithm (end)

\input{qm2pi.ack} 

% section acknowledgments (end)

\newpage


\bibliographystyle{plain}   
\bibliography{../../biblios/main.bib}

\input{qm2pi.rhodetails}

\end{document}

 

% section wiring algorithm (end)

\documentclass[12pt]{llncs}
%\documentclass{jktr}

\usepackage[pdftex]{hyperref}                   
\usepackage {listings}
\usepackage {mathpartir}
\usepackage{bcprules}
%\usepackage{listings}
                       
\usepackage{graphicx} 
%\usepackage[margins=2.5cm,nohead,nofoot]{geometry}
%\usepackage{geometry}
\usepackage{amsfonts}
\usepackage{amstext}
\usepackage{latexsym}
\usepackage{amssymb}
\usepackage{color}


%\include{myPreamble}
\include{qm2pi.local} 

%\ifpdf
%\usepackage[pdftex]{graphicx}
%\else
%\usepackage{graphicx}
%\fi

 % \ifpdf
%  \usepackage{pdfsync}
%  \if


%\title{Brief Article}
%\author{David F. Snyder}
%\author{L.G. Meredith}

%\address{Dept. of Math., Texas State University--San Marcos, San Marcos, TX 78666}
       
\pagestyle{empty}


\begin{document}

\lstset{language=[Objective]Caml,frame=shadowbox}

\input{qm2pi.front}

% section front matter (end)

\input{qm2pi.intro} 
 
% section introduction (end)

% \input{qm2pi.knotations} 

% section notation (end)

\input{qm2pi.process.calculi} 

% section concurrent_process_calculi_and_spatial_logics_ (end)
    
%\input{qm2pi.knots2pi} 

%\input{qm2pi.trefoil} 

%\input{qm2pi.mainthm} 

% subsection basic_interpretation (end)

%\input{qm2pi.rho.presentation} 
\subsection{The syntax and semantics of the notation system}\label{sub:the_syntax_and_semantics_of_the_notation_system} % (fold)

We now summarize a technical presentation of the calculus that
embodies our theory of dynamics. The typical presentation of such a
calculus follows the style of giving generators and relations on
them. The grammar, below, describing term constructors, freely
generates the set of processes, $\Proc$. This set is then quotiented
by a relation known as structural congruence and it is over this set
that the notion of dynamics is expressed. This presentation is
essentially that of \cite{MeredithR05} with the addition of
polyadicity and summation. For readability we have relegated some of
the technical subtleties to an appendix.

\subsubsection{Process grammar}\label{subsub:process_grammar}

\begin{mathpar}
  \inferrule* [lab=synchronization] {} {{M} \bc \pzero \;|\; x?F \;|\; x!C }
  \and
  \inferrule* [lab=abstraction] {} {{F} \bc (x)P}
  \and
  \inferrule* [lab=concretion] {} {{C} \bc \langle Q \rangle}
  \and
  \inferrule* [lab=process] {} {{P,Q} \bc M \;| \;P|Q \;|\; @{x}}
  \and
  \inferrule* [lab=name] {} {{x} \bc \quotep{P}}
\end{mathpar} 

Note that $\vec{x}$ (resp. $\vec{P}$) denotes a vector of names
(resp. processes) of length $|\vec{x}|$ (resp. $|\vec{P}|$). We adopt
the following useful abbreviations.

\begin{mathpar}
   x?(\vec{y}).P := x.(\vec{y})P \and  x\clift{\vec{P}} := x.\clift{\vec{P}}
   \and x!(y) := \lift{x}{\dropn{y}}
   \and \Pi_{i=0}^{n-1}P_i := P_0 | \ldots | P_{n-1}
\end{mathpar}

\subsubsection{Structural congruence}

\paragraph{Free and bound names and alpha-equivalence.} At the
core of structural equivalence is alpha-equivalence which identifies
process that are the same up to a change of variable. Formally, we
recognize the distinction between free and bound names. The free names
of a process, $\freenames{P}$, may be calculated recursively as
follows:

\begin{mathpar}
\freenames{\pzero} := \emptyset
  \and \\
  \freenames{x?(y).P} := \{ x \} \cup (\freenames{P} \setminus \{ y \})
  \and 
  \freenames{x!\langle P \rangle} := \{ x \} \cup \{ P \} 
  \and \\
  \freenames{P|Q} := \freenames{P} \cup \freenames{Q}
  \and \\
  \freenames{@{x}} := \{ x \}
\end{mathpar}

$\pi$
$\quotep{\pi}$

$\freenames{-} : \pi \to \mathcal{P}(\quotep{\pi})$

\begin{eqnarray*}
  \freenames{\pzero} & := & \emptyset \\
  \freenames{x?(y).P} & := & \{ x \} \cup (\freenames{P} \setminus \{ y \}) \\
  \freenames{x!\langle P \rangle} & := & \{ x \} \cup \{ P \} \\
  \freenames{P|Q} & := & \freenames{P} \cup \freenames{Q} \\
  \freenames{\dropn{x}} & := & \{ x \}
\end{eqnarray*}

The bound names of a process, $\boundnames{P}$, are those names occurring in $P$
that are not free. For example, in $x?(y).0$, the name $x$ is free, while $y$ is bound.

\begin{mathpar}
  \inferrule* [lab=monoidal-laws] {} { P|Q \equiv Q|P \and P|0 \equiv P \and P|(Q|R) \equiv (P|Q)|R }
\end{mathpar}

\begin{mathpar}
  \inferrule* [lab=alpha-equivalence] {} { (x)P \equiv (y)P\{y/x\} \and y \not\in \freenames{P} }
\end{mathpar}

\begin{definition}
Then two processes, $P,Q$, are alpha-equivalent if $P = Q\{\vec{y}/\vec{x}\}$ for
some $\vec{x} \in \boundnames{Q},\vec{y} \in \boundnames{P}$, where $Q\{\vec{y}/\vec{x}\}$
denotes the capture-avoiding substitution of $\vec{y}$ for $\vec{x}$ in $Q$.
\end{definition}

\begin{definition}
  The {\em structural congruence} \cite{SangiorgiWalker} , $\equiv$,
  between processes is the least congruence containing
  alpha-equivalence, satisfying the abelian monoid laws
  (associativity, commutativity and $\pzero$ as identity) for parallel
  composition $|$ and for summation $+$.
\end{definition}

\subsection{Name equivalence}

We take name equivalence, written $\nameeq$, to be the smallest
equivalence relation generated by the following rules.

\begin{mathpar}
\inferrule*[lab=Quote-drop]
{ }
{ \quotep{@{x}} \nameeq x }

\inferrule*[lab=Struct-equiv]
{ P \scong Q }
{ \quotep{P} \nameeq \quotep{Q} }
\end{mathpar}

The astute reader will have noticed that the mutual recursion of names
and processes imposes a mutual recursion on alpha-equivalence and
structural equivalence via name-equivalence. Fortunately, all of this
works out pleasantly and we may calculate in the natural way, free of
concern. The reader interested in the details is referred to the
appendix \ref{appendix:rho_details}.

\subsection{Substitution}

We use $\Proc$ for the set of processes, $\QProc$ for the set of
names, and $\id{\{}\vec{y} / \vec{x} \id{\}}$ to denote partial maps,
$s : \QProc \rightarrow \QProc$. A map, $s$ lifts, uniquely, to a map
on process terms, $\widehat{s} : \Proc \rightarrow \Proc$ by the
following equations.

\begin{mathpar}
  (0) \psubstp{Q}{P} := 0 \\
  (R \juxtap S) \psubstp{Q}{P}
  :=    
  (R)\psubstp{Q}{P} \juxtap (S) \psubstp{Q}{P} \\
  (x?(y).R) \psubstp{Q}{P}    
  :=    
  (x)\substp{Q}{P} (z)\concat( (R \psubstn{z}{y}) \psubstp{Q}{P} ) \\
  (\lift{x}{R}) \psubstp{Q}{P}  
  :=
  \lift{(x)\substp{Q}{P}}{ R \psubstp{Q}{P} } \\
%   (\dropn{x})  \psubstp{Q}{P}       
%   := 
%   \left\{ 
%     \begin{array}{ccc} 
%       \dropn{\quotep{Q}} & & x \nameeq \quotep{P} \\
%       \dropn{x} & & otherwise \\
%     \end{array}
%   \right. 
  (\dropn{x})  \psubstp{Q}{P}       
  := 
  \left\{ 
    \begin{array}{ccc} 
      Q & & x \nameeq \quotep{P} \\
      \dropn{x} & & otherwise \\
    \end{array}
  \right.
\end{mathpar}
 

where

\begin{eqnarray}
  (x)\id{\{} \lpquote Q \rpquote / \lpquote P \rpquote \id{\}}            = 
  \left\{ 
    \begin{array}{ccc}
      \lpquote Q \rpquote & & x \nameeq \lpquote P \rpquote \\
      x & & otherwise \\
    \end{array}
  \right. \nonumber
\end{eqnarray}

and $z$ is chosen distinct from $\quotep{P}$, $\quotep{Q}$, the free
names in $Q$, and all the names in $R$. Our $\alpha$-equivalence will
be built in the standard way from this substitution.

\begin{remark}\label{rem:no_self_referential_names}
  One consequence of these definitions is that $\forall P. \quotep{P}
  \not\in \freenames{P}$.
\end{remark}

\subsection{ Dynamic quote: an example }

Anticipating something of what's to come, consider applying the
substitution, $\widehat{\id{\{}u / z \id{\}}}$, to the following pair
of processes, $\lift{w}{y!(z)}$ and $w[ \lpquote y!(z) \rpquote ]$.

\begin{eqnarray}
	\lift{w}{y!(z)}\widehat{\id{\{}u / z \id{\}}}
		& = &
		\lift{w}{y!(u)} \nonumber\\
	w[ \lpquote y!(z) \rpquote ] \widehat{ \id{\{}u / z \id{\}} }
		& = &
		w[ \lpquote y!(z) \rpquote ] \nonumber
\end{eqnarray}

Because the body of the process between quotes is impervious to
substitution, we get radically different answers. In fact, by
examining the first process in an input context,
e.g. $x?(z).\lift{w}{y!(z)}$, we see that the process under the lift
operator may be shaped by prefixed inputs binding a name inside it. In
this sense, the lift operator will be seen as a way to dynamically
construct processes before reifying them as names.

Finally equipped with these standard features we can present the
dynamics of the calculus.

\subsubsection{Operational semantics} 

Finally, we introduce the computational dynamics. What marks these
algebras as distinct from other more traditionally studied algebraic
structures, e.g. vector spaces or polynomial rings, is the manner in
which dynamics is captured. In traditional structures, dynamics is typically
expressed through morphisms between such structures, as in linear maps
between vector spaces or morphisms between rings. In algebras
associated with the semantics of computation, the dynamics is
expressed as part of the algebraic structure itself, through a
reduction reduction relation typically denoted by $\red$. Below, we
give a recursive presentation of this relation for the calculus used
in the encoding.

$\red \subseteq \pi \times \pi$
$\red : \pi \to \mathcal{P}(\pi)$

\begin{mathpar}
  \inferrule* [lab=Comm] { \textsf{match}( x_{src}, x_{trgt} ) } { x_{trgt}?(y)P \; | \; x_{src}!\langle {Q} \rangle \red P\{\quotep{Q}/y}\} }
  \and \\
  \inferrule* [lab=Par] {{P} \red {P}'} {{{P} | {Q}} \red {{P}' | {Q}}}
  \and
  \inferrule* [lab=Equiv]{{{P} \scong {P}'} \andalso {{P}' \red {Q}'} \andalso {{Q}' \scong {Q}}}{{P} \red {Q}}
\end{mathpar}

\begin{eqnarray*}
  match_{\equiv} (\quotep{P},\quotep{Q}) & := & P \equiv Q \\
  match_{\dagger}(\quotep{P},\quotep{Q}) & := & \forall R. P|Q \red^{*} R => R \red^{*} 0 \\
  match_{K}(\quotep{P},\quotep{Q}) & := & K \mbox{ for some context } K
\end{eqnarray*}

$u?(x)P | u!\langle Q \rangle \red P\{\quotep{Q}/x\}$

%We write $\wred$ for $\red^*$, and $P\red$ if $\exists Q $ such that $ P \red Q$.
We write $P\red$ if $\exists Q $ such that $ P \red Q$ and $P\not\red$, otherwise.

\section{Replication}

As mentioned before, it is known that replication (and hence
recursion) can be implemented in a higher-order process algebra
\cite{SangiorgiWalker}. As our first example of calculation with the
machinery thus far presented we give the construction explicitly in
the {\rhoc}.

\begin{eqnarray}
	D_{x} & := & \prefix{x}{y}{(\binpar{\outputp{x}{y}}{@{y}})} \nonumber\\
	\bangp_{x}{P} & := & \binpar{{x}!\langle{\binpar{D_{x}}{P}}\rangle}{D_{x}} \nonumber
\end{eqnarray}

\begin{eqnarray}
	\bangp_{x}{P} & & \nonumber\\
	=
	& {x}!\langle{(\prefix{x}{y}{(\outputp{x}{y} | @{y})) | P}}\rangle 
	      | \prefix{x}{y}{(\outputp{x}{y} | @{y})} & \nonumber\\
	\red
	& (\outputp{x}{y} | @{y})\substn{\quotep{(\prefix{x}{y}{(@{y} | \outputp{x}{y})) | P}}}{y} & \nonumber\\
	=
	& \outputp{x}{\quotep{(\prefix{x}{y}{(\outputp{x}{y} | @{y})) | P}}}
	  | {(\prefix{x}{y}{(\outputp{x}{y} | @{y})) | P}} & \nonumber\\
	\red
	& \ldots & \nonumber\\
	\red^*
	& P | P | \ldots & \nonumber
\end{eqnarray}

Of course, this encoding, as an implementation, runs away, unfolding
$\bangp{P}$ eagerly. A lazier and more implementable replication
operator, restricted to input-guarded processes, may be obtained as follows.

\begin{eqnarray}
\bangp{\prefix{u}{v}{P}} 
	:= 
	\binpar{\lift{x}{\prefix{u}{v}{(\binpar{D(x)}{P})}}}{D(x)} \nonumber
\end{eqnarray}

\begin{remark}
  Note that the lazier definition still does not deal with summation
  or mixed summation (i.e. sums over input and output). The reader is
  invited to construct definitions of replication that deal with these
  features. 

  Further, the definitions are parameterized in a name, $x$. Can you,
  gentle reader, make a definition that eliminates this parameter and
  guarantees no accidental interaction between the replication
  machinery and the process being replicated -- i.e. no accidental
  sharing of names used by the process to get its work done and the
  name(s) used by the replication to effect copying. This latter
  revision of the definition of replication is crucial to obtaining
  the expected identity $!!P \sim !P$.
\end{remark}

\begin{remark}\label{rem:paradoxical_combinator}
  The reader familiar with the lambda calculus will have noticed the
  similarity between $D$ and the paradoxical combinator.

  [Ed. note: the existence of this seems to suggest we have to be more
  restrictive on the set of processes and names we admit if we are to
  support no-cloning.]
\end{remark}

\subsubsection{Bisimulation}

The computational dynamics gives rise to another kind of equivalence,
the equivalence of computational behavior. As previously mentioned
this is typically captured \emph{via} some form of bisimulation.

% The notion we use in this paper is weak barbed bisimulation
% \cite{milner91polyadicpi}.

The notion we use in this paper is derived from weak barbed
bisimulation \cite{milner91polyadicpi}. 

\begin{definition}
An \emph{observation relation}, $\downarrow_{\mathcal N}$, over a set
of names, $\mathcal N$, is the smallest relation satisfying the rules
below.

\infrule[Out-barb]{y \in {\mathcal N}, \; x \nameeq y}
		  {\outputp{x}{v} \downarrow_{\mathcal N} x}
\infrule[Par-barb]{\mbox{$P\downarrow_{\mathcal N} x$ or $Q\downarrow_{\mathcal N} x$}}
		  {\binpar{P}{Q} \downarrow_{\mathcal N} x}

We write $P \Downarrow_{\mathcal N} x$ if there is $Q$ such that 
$P \wred Q$ and $Q \downarrow_{\mathcal N} x$.
\end{definition}

\begin{definition}
%\label{def.bbisim}
An  ${\mathcal N}$-\emph{barbed bisimulation} over a set of names, ${\mathcal N}$, is a symmetric binary relation 
${\mathcal S}_{\mathcal N}$ between agents such that $P\rel{S}_{\mathcal N}Q$ implies:
\begin{enumerate}
\item If $P \red P'$ then $Q \wred Q'$ and $P'\rel{S}_{\mathcal N} Q'$.
\item If $P\downarrow_{\mathcal N} x$, then $Q\Downarrow_{\mathcal N} x$.
\end{enumerate}
$P$ is ${\mathcal N}$-barbed bisimilar to $Q$, written
$P \wbbisim_{\mathcal N} Q$, if $P \rel{S}_{\mathcal N} Q$ for some ${\mathcal N}$-barbed bisimulation ${\mathcal S}_{\mathcal N}$.
\end{definition}

$\mathcal{R} \subseteq \pi \times \pi$

$P \mathcal{R} Q => \forall P'. P \red P' \Rightarrow \exists Q'. Q \red Q', P' \mathcal{R} Q'$

$P \vdash x \Rightarrow Q \vdash x$

\begin{mathpar}
  \inferrule*[lab=Out-barb]{x \nameeq y}{{y}!\langle{Q}\rangle \vdash x}
  \and
  \inferrule*[lab=Par-barb]{\mbox{$P\vdash x$ or $Q\vdash x$}}{\binpar{P}{Q} \vdash x}
\end{mathpar}

\subsubsection{Contexts}

One of the principle advantages of computational calculi like the
$\pi$-calculus is a well-defined notion of context,
contextual-equivalence and a correlation between
contextual-equivalence and notions of bisimulation. The notion of
context allows the decomposition of a process into (sub-)process and
its syntactic environment, its context. Thus, a context may be
thought of as a process with a ``hole'' (written $\Box$) in it. The
application of a context $M$ to a process $P$, written $M[P]$, is
tantamount to filling the hole in $M$ with $P$. In this paper we do
not need the full weight of this theory, but do make use of the notion
of context in the proof the main theorem. 

\begin{mathpar}
  \inferrule* [lab=summation] {} {{M_{M},M_{N}} \bc \Box \;|\; x.M_{A} \;|\; M_{M}+M_{N}}
  \and
  \inferrule* [lab=agent] {} {{M_{A}} \bc (\vec{x})M_{P} \;| \; \clift{P_0,\ldots,M_{P},\ldots,P_N}}
  \and \\
  \inferrule* [lab=process] {} {{M_{P}} \bc M_{N} \;| \;P|M_{P} }
\end{mathpar} 

\begin{mathpar}
  \inferrule* [lab=sychronization] {} {M_{N} \bc \Box \;|\; x?M_{F} \;|\; x!M_{C}}
  \and
  \inferrule* [lab=abstraction] {} {{M_{F}} \bc (x)M_{P} }
  \and
  \inferrule* [lab=concretion] {} {{M_{C}} \bc \langle M_{P} \rangle }
  \and \\
  \inferrule* [lab=process] {} {{M_{P}} \bc M_{N} \;| \;P|M_{P} }
\end{mathpar}

\begin{definition}[contextual application] Given a context $M$, and
  process $P$, we define the \emph{contextual application}, $M[P] :=
  M\{P/\Box\}$. That is, the contextual application of M to P is the
  substitution of $P$ for $\Box$ in $M$.
\end{definition}

$\meaningof{-} : L \to \mathcal{P}(\pi)$

\begin{mathpar}
  \inferrule* [lab=collection] {} {\meaningof{true} = \pi, \and \meaningof{~E} = \pi \setminus \meaningof{E}, \and \meaningof{E_{1} \& E_{2}} = \meaningof{E_{1}} \cap \meaningof{E_{2}}}
\end{mathpar}

\begin{mathpar}
  \inferrule* [lab=structure] {} {\meaningof{0} = \{ P \in \pi | P \equiv 0 \}, \and \\ \meaningof{E_1 | E_2} = \{ P \in \pi | P \equiv P_{1} | P_{2}, P_{1} \in \meaningof{E_{1}}, P_{2} \in \meaningof{E_2}\} }
\end{mathpar}

\begin{mathpar}
 \inferrule* [lab=behavior] {} {\meaningof{\langle a?b \rangle E} = \{ P \in \pi | P \equiv Q | u?(y)P', \\ \and \\\\ \and \\ \;\;\; u \in \meaningof{a}, \forall z.P'\{z/y\} \in \meaningof{E\{z/b\}}\}, \and \\ \meaningof{a!E} = \{ P \in \pi | P \equiv Q | x!\langle P' \rangle, x \in \meaningof{a} P' \in \meaningof{E}\} }
\end{mathpar}

\begin{mathpar}
 \inferrule* [lab=nominal] {} {\meaningof{\quotep{E}} = \{ \quotep{P} \in \quotep{\pi} | P \in \meaningof{E} \}, \and \meaningof{\quotep{P}} = \{ \quotep{Q} \in \quotep{\pi} | P \equiv Q \} \and \\ \meaningof{@\quotep{E}} = \{ P \in \pi | P \equiv @x, x \in \meaningof{E} \}}
\end{mathpar}

\begin{eqnarray*}
  \\
  \meaningof{-} : TS \to ST
\end{eqnarray*}

\begin{eqnarray*}
  \\
  L : TS \to ST
\end{eqnarray*}

\begin{eqnarray*}
  \\
  P \models E \iff P \in \meaningof{E}
\end{eqnarray*}

\begin{eqnarray*}
  P \approx_{L} Q \iff \forall E \in L. P \models E \iff Q \models E
\end{eqnarray*}

\begin{eqnarray*}
  P \approx_{K} Q
\end{eqnarray*}

\begin{eqnarray*}
  P \approx Q
\end{eqnarray*}

$\approx_{K} = \approx = \approx_{L}$

\subsubsection{Contextual duality}

Note that contexts extend the quotation operation to a family of
operations from processes to names. Given a context, $M$, we can
define a \emph{nominal context}, $\quotep{M}$ by $\quotep{M}[P] :=
\quotep{M[P]}$. To foreshadow what is to come we observe that these
operations enjoy a duality with processes very much like the duality
between vectors and maps from vectors to scalars.

Further, because the calculus is essentially higher-order, we have a
correspondence between contexts and processes. More specifically,
given a name $x$ and a context $M$ we can construct $M^{*}_{x}$ such
that 

\begin{mathpar}
  M^{*}_{x} | \lift{x}{P} \red M[P]
\end{mathpar}

namely,

\begin{mathpar}
  M^{*}_{x} := x?(u).M[\dropn{u}]
\end{mathpar}

The dependence of $M^{*}_{x}$ on a name makes it an abstraction, 

\begin{mathpar}
  M^{*} := (x)x?(u).M[\dropn{u}]
\end{mathpar}

\subsection{Additional notation}

It will sometimes be convenient to denote the process a name
quotes. We already have the notation $x = \quotep{P}$, but it will be
convenient to introduce an alternate notation, $\procn{x}$, when we
want to emphasize the connection to the use of the name. Note that, by
virtue of name equivalence, $\quotep{\procn{x}} \nameeq x$; so, the
notation is consistent with previous definitions.

Further, because names have structure it is possible to effect
substitutions on the basis of that structure. This means we need to
upgrade our notation for substitutions, which we accomplish by
adapting comprehension notation. Thus,

\begin{mathpar}
  P\{ y / x : x \in S \}
\end{mathpar}

is interpreted to mean the process derived from P by replacing (in a
capture-avoiding manner) each occurrence of $x$ in $S$ by $y$. For example,

\begin{mathpar}
  P\{ \quotep{\procn{x}|\procn{x}} / x : x \in \freenames{P} \}
\end{mathpar}

will replace each (occurrence) of a free name $x$ in $P$ by
$\quotep{\procn{x}|\procn{x}}$.

Also, we will avail ourselves of the notation $x^{L}$ and $x^{R}$ to
denote injections of a name into disjoint copies of the name
space. There are numerous ways to accomplish this. One example can be
found in \cite{MeredithR05}. This notation overloads to vectors of
names: $\vec{x}^{\pi} := (x_{i}^{\pi} \; : \; 0 \leq i < |\vec{x}| )$ where $\pi \in \{L,R\}$.

We also use $P^{\Box} := P|\Box$.

In \cite{MeredithR05} an interpretation of the new operator is
given. It turns out that there are several possible interpretations
all enjoying the requisite algebraic properties of the operator (see
\cite{milner91polyadicpi}). We will therefore make liberal use of
$(\nu\; \vec{x})P$.

% subsection the_syntax_and_semantics_of_the_notation_system (end)   

\input{qm2pi.qmops} 

\input{qm2pi.sterngerlach} 

\input{qm2pi.metric} 

% section concurrent_process_calculi (end)

%\input{qm2pi.proofsketch}

% section proof sketch (end)

%\input{qm2pi.slviaknots} 

% section spatial logic via knots (end)

\input{qm2pi.conclusion}

% section conclusion (end)

%\input{qm2pi.dtcodes} 

% section wiring algorithm (end)

\input{qm2pi.ack} 

% section acknowledgments (end)

\newpage


\bibliographystyle{plain}   
\bibliography{../../biblios/main.bib}

\input{qm2pi.rhodetails}

\end{document}

 

% section acknowledgments (end)

\newpage


\bibliographystyle{plain}   
\bibliography{../../biblios/main.bib}

\documentclass[12pt]{llncs}
%\documentclass{jktr}

\usepackage[pdftex]{hyperref}                   
\usepackage {listings}
\usepackage {mathpartir}
\usepackage{bcprules}
%\usepackage{listings}
                       
\usepackage{graphicx} 
%\usepackage[margins=2.5cm,nohead,nofoot]{geometry}
%\usepackage{geometry}
\usepackage{amsfonts}
\usepackage{amstext}
\usepackage{latexsym}
\usepackage{amssymb}
\usepackage{color}


%\include{myPreamble}
\include{qm2pi.local} 

%\ifpdf
%\usepackage[pdftex]{graphicx}
%\else
%\usepackage{graphicx}
%\fi

 % \ifpdf
%  \usepackage{pdfsync}
%  \if


%\title{Brief Article}
%\author{David F. Snyder}
%\author{L.G. Meredith}

%\address{Dept. of Math., Texas State University--San Marcos, San Marcos, TX 78666}
       
\pagestyle{empty}


\begin{document}

\lstset{language=[Objective]Caml,frame=shadowbox}

\input{qm2pi.front}

% section front matter (end)

\input{qm2pi.intro} 
 
% section introduction (end)

% \input{qm2pi.knotations} 

% section notation (end)

\input{qm2pi.process.calculi} 

% section concurrent_process_calculi_and_spatial_logics_ (end)
    
%\input{qm2pi.knots2pi} 

%\input{qm2pi.trefoil} 

%\input{qm2pi.mainthm} 

% subsection basic_interpretation (end)

%\input{qm2pi.rho.presentation} 
\subsection{The syntax and semantics of the notation system}\label{sub:the_syntax_and_semantics_of_the_notation_system} % (fold)

We now summarize a technical presentation of the calculus that
embodies our theory of dynamics. The typical presentation of such a
calculus follows the style of giving generators and relations on
them. The grammar, below, describing term constructors, freely
generates the set of processes, $\Proc$. This set is then quotiented
by a relation known as structural congruence and it is over this set
that the notion of dynamics is expressed. This presentation is
essentially that of \cite{MeredithR05} with the addition of
polyadicity and summation. For readability we have relegated some of
the technical subtleties to an appendix.

\subsubsection{Process grammar}\label{subsub:process_grammar}

\begin{mathpar}
  \inferrule* [lab=synchronization] {} {{M} \bc \pzero \;|\; x?F \;|\; x!C }
  \and
  \inferrule* [lab=abstraction] {} {{F} \bc (x)P}
  \and
  \inferrule* [lab=concretion] {} {{C} \bc \langle Q \rangle}
  \and
  \inferrule* [lab=process] {} {{P,Q} \bc M \;| \;P|Q \;|\; @{x}}
  \and
  \inferrule* [lab=name] {} {{x} \bc \quotep{P}}
\end{mathpar} 

Note that $\vec{x}$ (resp. $\vec{P}$) denotes a vector of names
(resp. processes) of length $|\vec{x}|$ (resp. $|\vec{P}|$). We adopt
the following useful abbreviations.

\begin{mathpar}
   x?(\vec{y}).P := x.(\vec{y})P \and  x\clift{\vec{P}} := x.\clift{\vec{P}}
   \and x!(y) := \lift{x}{\dropn{y}}
   \and \Pi_{i=0}^{n-1}P_i := P_0 | \ldots | P_{n-1}
\end{mathpar}

\subsubsection{Structural congruence}

\paragraph{Free and bound names and alpha-equivalence.} At the
core of structural equivalence is alpha-equivalence which identifies
process that are the same up to a change of variable. Formally, we
recognize the distinction between free and bound names. The free names
of a process, $\freenames{P}$, may be calculated recursively as
follows:

\begin{mathpar}
\freenames{\pzero} := \emptyset
  \and \\
  \freenames{x?(y).P} := \{ x \} \cup (\freenames{P} \setminus \{ y \})
  \and 
  \freenames{x!\langle P \rangle} := \{ x \} \cup \{ P \} 
  \and \\
  \freenames{P|Q} := \freenames{P} \cup \freenames{Q}
  \and \\
  \freenames{@{x}} := \{ x \}
\end{mathpar}

$\pi$
$\quotep{\pi}$

$\freenames{-} : \pi \to \mathcal{P}(\quotep{\pi})$

\begin{eqnarray*}
  \freenames{\pzero} & := & \emptyset \\
  \freenames{x?(y).P} & := & \{ x \} \cup (\freenames{P} \setminus \{ y \}) \\
  \freenames{x!\langle P \rangle} & := & \{ x \} \cup \{ P \} \\
  \freenames{P|Q} & := & \freenames{P} \cup \freenames{Q} \\
  \freenames{\dropn{x}} & := & \{ x \}
\end{eqnarray*}

The bound names of a process, $\boundnames{P}$, are those names occurring in $P$
that are not free. For example, in $x?(y).0$, the name $x$ is free, while $y$ is bound.

\begin{mathpar}
  \inferrule* [lab=monoidal-laws] {} { P|Q \equiv Q|P \and P|0 \equiv P \and P|(Q|R) \equiv (P|Q)|R }
\end{mathpar}

\begin{mathpar}
  \inferrule* [lab=alpha-equivalence] {} { (x)P \equiv (y)P\{y/x\} \and y \not\in \freenames{P} }
\end{mathpar}

\begin{definition}
Then two processes, $P,Q$, are alpha-equivalent if $P = Q\{\vec{y}/\vec{x}\}$ for
some $\vec{x} \in \boundnames{Q},\vec{y} \in \boundnames{P}$, where $Q\{\vec{y}/\vec{x}\}$
denotes the capture-avoiding substitution of $\vec{y}$ for $\vec{x}$ in $Q$.
\end{definition}

\begin{definition}
  The {\em structural congruence} \cite{SangiorgiWalker} , $\equiv$,
  between processes is the least congruence containing
  alpha-equivalence, satisfying the abelian monoid laws
  (associativity, commutativity and $\pzero$ as identity) for parallel
  composition $|$ and for summation $+$.
\end{definition}

\subsection{Name equivalence}

We take name equivalence, written $\nameeq$, to be the smallest
equivalence relation generated by the following rules.

\begin{mathpar}
\inferrule*[lab=Quote-drop]
{ }
{ \quotep{@{x}} \nameeq x }

\inferrule*[lab=Struct-equiv]
{ P \scong Q }
{ \quotep{P} \nameeq \quotep{Q} }
\end{mathpar}

The astute reader will have noticed that the mutual recursion of names
and processes imposes a mutual recursion on alpha-equivalence and
structural equivalence via name-equivalence. Fortunately, all of this
works out pleasantly and we may calculate in the natural way, free of
concern. The reader interested in the details is referred to the
appendix \ref{appendix:rho_details}.

\subsection{Substitution}

We use $\Proc$ for the set of processes, $\QProc$ for the set of
names, and $\id{\{}\vec{y} / \vec{x} \id{\}}$ to denote partial maps,
$s : \QProc \rightarrow \QProc$. A map, $s$ lifts, uniquely, to a map
on process terms, $\widehat{s} : \Proc \rightarrow \Proc$ by the
following equations.

\begin{mathpar}
  (0) \psubstp{Q}{P} := 0 \\
  (R \juxtap S) \psubstp{Q}{P}
  :=    
  (R)\psubstp{Q}{P} \juxtap (S) \psubstp{Q}{P} \\
  (x?(y).R) \psubstp{Q}{P}    
  :=    
  (x)\substp{Q}{P} (z)\concat( (R \psubstn{z}{y}) \psubstp{Q}{P} ) \\
  (\lift{x}{R}) \psubstp{Q}{P}  
  :=
  \lift{(x)\substp{Q}{P}}{ R \psubstp{Q}{P} } \\
%   (\dropn{x})  \psubstp{Q}{P}       
%   := 
%   \left\{ 
%     \begin{array}{ccc} 
%       \dropn{\quotep{Q}} & & x \nameeq \quotep{P} \\
%       \dropn{x} & & otherwise \\
%     \end{array}
%   \right. 
  (\dropn{x})  \psubstp{Q}{P}       
  := 
  \left\{ 
    \begin{array}{ccc} 
      Q & & x \nameeq \quotep{P} \\
      \dropn{x} & & otherwise \\
    \end{array}
  \right.
\end{mathpar}
 

where

\begin{eqnarray}
  (x)\id{\{} \lpquote Q \rpquote / \lpquote P \rpquote \id{\}}            = 
  \left\{ 
    \begin{array}{ccc}
      \lpquote Q \rpquote & & x \nameeq \lpquote P \rpquote \\
      x & & otherwise \\
    \end{array}
  \right. \nonumber
\end{eqnarray}

and $z$ is chosen distinct from $\quotep{P}$, $\quotep{Q}$, the free
names in $Q$, and all the names in $R$. Our $\alpha$-equivalence will
be built in the standard way from this substitution.

\begin{remark}\label{rem:no_self_referential_names}
  One consequence of these definitions is that $\forall P. \quotep{P}
  \not\in \freenames{P}$.
\end{remark}

\subsection{ Dynamic quote: an example }

Anticipating something of what's to come, consider applying the
substitution, $\widehat{\id{\{}u / z \id{\}}}$, to the following pair
of processes, $\lift{w}{y!(z)}$ and $w[ \lpquote y!(z) \rpquote ]$.

\begin{eqnarray}
	\lift{w}{y!(z)}\widehat{\id{\{}u / z \id{\}}}
		& = &
		\lift{w}{y!(u)} \nonumber\\
	w[ \lpquote y!(z) \rpquote ] \widehat{ \id{\{}u / z \id{\}} }
		& = &
		w[ \lpquote y!(z) \rpquote ] \nonumber
\end{eqnarray}

Because the body of the process between quotes is impervious to
substitution, we get radically different answers. In fact, by
examining the first process in an input context,
e.g. $x?(z).\lift{w}{y!(z)}$, we see that the process under the lift
operator may be shaped by prefixed inputs binding a name inside it. In
this sense, the lift operator will be seen as a way to dynamically
construct processes before reifying them as names.

Finally equipped with these standard features we can present the
dynamics of the calculus.

\subsubsection{Operational semantics} 

Finally, we introduce the computational dynamics. What marks these
algebras as distinct from other more traditionally studied algebraic
structures, e.g. vector spaces or polynomial rings, is the manner in
which dynamics is captured. In traditional structures, dynamics is typically
expressed through morphisms between such structures, as in linear maps
between vector spaces or morphisms between rings. In algebras
associated with the semantics of computation, the dynamics is
expressed as part of the algebraic structure itself, through a
reduction reduction relation typically denoted by $\red$. Below, we
give a recursive presentation of this relation for the calculus used
in the encoding.

$\red \subseteq \pi \times \pi$
$\red : \pi \to \mathcal{P}(\pi)$

\begin{mathpar}
  \inferrule* [lab=Comm] { \textsf{match}( x_{src}, x_{trgt} ) } { x_{trgt}?(y)P \; | \; x_{src}!\langle {Q} \rangle \red P\{\quotep{Q}/y}\} }
  \and \\
  \inferrule* [lab=Par] {{P} \red {P}'} {{{P} | {Q}} \red {{P}' | {Q}}}
  \and
  \inferrule* [lab=Equiv]{{{P} \scong {P}'} \andalso {{P}' \red {Q}'} \andalso {{Q}' \scong {Q}}}{{P} \red {Q}}
\end{mathpar}

\begin{eqnarray*}
  match_{\equiv} (\quotep{P},\quotep{Q}) & := & P \equiv Q \\
  match_{\dagger}(\quotep{P},\quotep{Q}) & := & \forall R. P|Q \red^{*} R => R \red^{*} 0 \\
  match_{K}(\quotep{P},\quotep{Q}) & := & K \mbox{ for some context } K
\end{eqnarray*}

$u?(x)P | u!\langle Q \rangle \red P\{\quotep{Q}/x\}$

%We write $\wred$ for $\red^*$, and $P\red$ if $\exists Q $ such that $ P \red Q$.
We write $P\red$ if $\exists Q $ such that $ P \red Q$ and $P\not\red$, otherwise.

\section{Replication}

As mentioned before, it is known that replication (and hence
recursion) can be implemented in a higher-order process algebra
\cite{SangiorgiWalker}. As our first example of calculation with the
machinery thus far presented we give the construction explicitly in
the {\rhoc}.

\begin{eqnarray}
	D_{x} & := & \prefix{x}{y}{(\binpar{\outputp{x}{y}}{@{y}})} \nonumber\\
	\bangp_{x}{P} & := & \binpar{{x}!\langle{\binpar{D_{x}}{P}}\rangle}{D_{x}} \nonumber
\end{eqnarray}

\begin{eqnarray}
	\bangp_{x}{P} & & \nonumber\\
	=
	& {x}!\langle{(\prefix{x}{y}{(\outputp{x}{y} | @{y})) | P}}\rangle 
	      | \prefix{x}{y}{(\outputp{x}{y} | @{y})} & \nonumber\\
	\red
	& (\outputp{x}{y} | @{y})\substn{\quotep{(\prefix{x}{y}{(@{y} | \outputp{x}{y})) | P}}}{y} & \nonumber\\
	=
	& \outputp{x}{\quotep{(\prefix{x}{y}{(\outputp{x}{y} | @{y})) | P}}}
	  | {(\prefix{x}{y}{(\outputp{x}{y} | @{y})) | P}} & \nonumber\\
	\red
	& \ldots & \nonumber\\
	\red^*
	& P | P | \ldots & \nonumber
\end{eqnarray}

Of course, this encoding, as an implementation, runs away, unfolding
$\bangp{P}$ eagerly. A lazier and more implementable replication
operator, restricted to input-guarded processes, may be obtained as follows.

\begin{eqnarray}
\bangp{\prefix{u}{v}{P}} 
	:= 
	\binpar{\lift{x}{\prefix{u}{v}{(\binpar{D(x)}{P})}}}{D(x)} \nonumber
\end{eqnarray}

\begin{remark}
  Note that the lazier definition still does not deal with summation
  or mixed summation (i.e. sums over input and output). The reader is
  invited to construct definitions of replication that deal with these
  features. 

  Further, the definitions are parameterized in a name, $x$. Can you,
  gentle reader, make a definition that eliminates this parameter and
  guarantees no accidental interaction between the replication
  machinery and the process being replicated -- i.e. no accidental
  sharing of names used by the process to get its work done and the
  name(s) used by the replication to effect copying. This latter
  revision of the definition of replication is crucial to obtaining
  the expected identity $!!P \sim !P$.
\end{remark}

\begin{remark}\label{rem:paradoxical_combinator}
  The reader familiar with the lambda calculus will have noticed the
  similarity between $D$ and the paradoxical combinator.

  [Ed. note: the existence of this seems to suggest we have to be more
  restrictive on the set of processes and names we admit if we are to
  support no-cloning.]
\end{remark}

\subsubsection{Bisimulation}

The computational dynamics gives rise to another kind of equivalence,
the equivalence of computational behavior. As previously mentioned
this is typically captured \emph{via} some form of bisimulation.

% The notion we use in this paper is weak barbed bisimulation
% \cite{milner91polyadicpi}.

The notion we use in this paper is derived from weak barbed
bisimulation \cite{milner91polyadicpi}. 

\begin{definition}
An \emph{observation relation}, $\downarrow_{\mathcal N}$, over a set
of names, $\mathcal N$, is the smallest relation satisfying the rules
below.

\infrule[Out-barb]{y \in {\mathcal N}, \; x \nameeq y}
		  {\outputp{x}{v} \downarrow_{\mathcal N} x}
\infrule[Par-barb]{\mbox{$P\downarrow_{\mathcal N} x$ or $Q\downarrow_{\mathcal N} x$}}
		  {\binpar{P}{Q} \downarrow_{\mathcal N} x}

We write $P \Downarrow_{\mathcal N} x$ if there is $Q$ such that 
$P \wred Q$ and $Q \downarrow_{\mathcal N} x$.
\end{definition}

\begin{definition}
%\label{def.bbisim}
An  ${\mathcal N}$-\emph{barbed bisimulation} over a set of names, ${\mathcal N}$, is a symmetric binary relation 
${\mathcal S}_{\mathcal N}$ between agents such that $P\rel{S}_{\mathcal N}Q$ implies:
\begin{enumerate}
\item If $P \red P'$ then $Q \wred Q'$ and $P'\rel{S}_{\mathcal N} Q'$.
\item If $P\downarrow_{\mathcal N} x$, then $Q\Downarrow_{\mathcal N} x$.
\end{enumerate}
$P$ is ${\mathcal N}$-barbed bisimilar to $Q$, written
$P \wbbisim_{\mathcal N} Q$, if $P \rel{S}_{\mathcal N} Q$ for some ${\mathcal N}$-barbed bisimulation ${\mathcal S}_{\mathcal N}$.
\end{definition}

$\mathcal{R} \subseteq \pi \times \pi$

$P \mathcal{R} Q => \forall P'. P \red P' \Rightarrow \exists Q'. Q \red Q', P' \mathcal{R} Q'$

$P \vdash x \Rightarrow Q \vdash x$

\begin{mathpar}
  \inferrule*[lab=Out-barb]{x \nameeq y}{{y}!\langle{Q}\rangle \vdash x}
  \and
  \inferrule*[lab=Par-barb]{\mbox{$P\vdash x$ or $Q\vdash x$}}{\binpar{P}{Q} \vdash x}
\end{mathpar}

\subsubsection{Contexts}

One of the principle advantages of computational calculi like the
$\pi$-calculus is a well-defined notion of context,
contextual-equivalence and a correlation between
contextual-equivalence and notions of bisimulation. The notion of
context allows the decomposition of a process into (sub-)process and
its syntactic environment, its context. Thus, a context may be
thought of as a process with a ``hole'' (written $\Box$) in it. The
application of a context $M$ to a process $P$, written $M[P]$, is
tantamount to filling the hole in $M$ with $P$. In this paper we do
not need the full weight of this theory, but do make use of the notion
of context in the proof the main theorem. 

\begin{mathpar}
  \inferrule* [lab=summation] {} {{M_{M},M_{N}} \bc \Box \;|\; x.M_{A} \;|\; M_{M}+M_{N}}
  \and
  \inferrule* [lab=agent] {} {{M_{A}} \bc (\vec{x})M_{P} \;| \; \clift{P_0,\ldots,M_{P},\ldots,P_N}}
  \and \\
  \inferrule* [lab=process] {} {{M_{P}} \bc M_{N} \;| \;P|M_{P} }
\end{mathpar} 

\begin{mathpar}
  \inferrule* [lab=sychronization] {} {M_{N} \bc \Box \;|\; x?M_{F} \;|\; x!M_{C}}
  \and
  \inferrule* [lab=abstraction] {} {{M_{F}} \bc (x)M_{P} }
  \and
  \inferrule* [lab=concretion] {} {{M_{C}} \bc \langle M_{P} \rangle }
  \and \\
  \inferrule* [lab=process] {} {{M_{P}} \bc M_{N} \;| \;P|M_{P} }
\end{mathpar}

\begin{definition}[contextual application] Given a context $M$, and
  process $P$, we define the \emph{contextual application}, $M[P] :=
  M\{P/\Box\}$. That is, the contextual application of M to P is the
  substitution of $P$ for $\Box$ in $M$.
\end{definition}

$\meaningof{-} : L \to \mathcal{P}(\pi)$

\begin{mathpar}
  \inferrule* [lab=collection] {} {\meaningof{true} = \pi, \and \meaningof{~E} = \pi \setminus \meaningof{E}, \and \meaningof{E_{1} \& E_{2}} = \meaningof{E_{1}} \cap \meaningof{E_{2}}}
\end{mathpar}

\begin{mathpar}
  \inferrule* [lab=structure] {} {\meaningof{0} = \{ P \in \pi | P \equiv 0 \}, \and \\ \meaningof{E_1 | E_2} = \{ P \in \pi | P \equiv P_{1} | P_{2}, P_{1} \in \meaningof{E_{1}}, P_{2} \in \meaningof{E_2}\} }
\end{mathpar}

\begin{mathpar}
 \inferrule* [lab=behavior] {} {\meaningof{\langle a?b \rangle E} = \{ P \in \pi | P \equiv Q | u?(y)P', \\ \and \\\\ \and \\ \;\;\; u \in \meaningof{a}, \forall z.P'\{z/y\} \in \meaningof{E\{z/b\}}\}, \and \\ \meaningof{a!E} = \{ P \in \pi | P \equiv Q | x!\langle P' \rangle, x \in \meaningof{a} P' \in \meaningof{E}\} }
\end{mathpar}

\begin{mathpar}
 \inferrule* [lab=nominal] {} {\meaningof{\quotep{E}} = \{ \quotep{P} \in \quotep{\pi} | P \in \meaningof{E} \}, \and \meaningof{\quotep{P}} = \{ \quotep{Q} \in \quotep{\pi} | P \equiv Q \} \and \\ \meaningof{@\quotep{E}} = \{ P \in \pi | P \equiv @x, x \in \meaningof{E} \}}
\end{mathpar}

\begin{eqnarray*}
  \\
  \meaningof{-} : TS \to ST
\end{eqnarray*}

\begin{eqnarray*}
  \\
  L : TS \to ST
\end{eqnarray*}

\begin{eqnarray*}
  \\
  P \models E \iff P \in \meaningof{E}
\end{eqnarray*}

\begin{eqnarray*}
  P \approx_{L} Q \iff \forall E \in L. P \models E \iff Q \models E
\end{eqnarray*}

\begin{eqnarray*}
  P \approx_{K} Q
\end{eqnarray*}

\begin{eqnarray*}
  P \approx Q
\end{eqnarray*}

$\approx_{K} = \approx = \approx_{L}$

\subsubsection{Contextual duality}

Note that contexts extend the quotation operation to a family of
operations from processes to names. Given a context, $M$, we can
define a \emph{nominal context}, $\quotep{M}$ by $\quotep{M}[P] :=
\quotep{M[P]}$. To foreshadow what is to come we observe that these
operations enjoy a duality with processes very much like the duality
between vectors and maps from vectors to scalars.

Further, because the calculus is essentially higher-order, we have a
correspondence between contexts and processes. More specifically,
given a name $x$ and a context $M$ we can construct $M^{*}_{x}$ such
that 

\begin{mathpar}
  M^{*}_{x} | \lift{x}{P} \red M[P]
\end{mathpar}

namely,

\begin{mathpar}
  M^{*}_{x} := x?(u).M[\dropn{u}]
\end{mathpar}

The dependence of $M^{*}_{x}$ on a name makes it an abstraction, 

\begin{mathpar}
  M^{*} := (x)x?(u).M[\dropn{u}]
\end{mathpar}

\subsection{Additional notation}

It will sometimes be convenient to denote the process a name
quotes. We already have the notation $x = \quotep{P}$, but it will be
convenient to introduce an alternate notation, $\procn{x}$, when we
want to emphasize the connection to the use of the name. Note that, by
virtue of name equivalence, $\quotep{\procn{x}} \nameeq x$; so, the
notation is consistent with previous definitions.

Further, because names have structure it is possible to effect
substitutions on the basis of that structure. This means we need to
upgrade our notation for substitutions, which we accomplish by
adapting comprehension notation. Thus,

\begin{mathpar}
  P\{ y / x : x \in S \}
\end{mathpar}

is interpreted to mean the process derived from P by replacing (in a
capture-avoiding manner) each occurrence of $x$ in $S$ by $y$. For example,

\begin{mathpar}
  P\{ \quotep{\procn{x}|\procn{x}} / x : x \in \freenames{P} \}
\end{mathpar}

will replace each (occurrence) of a free name $x$ in $P$ by
$\quotep{\procn{x}|\procn{x}}$.

Also, we will avail ourselves of the notation $x^{L}$ and $x^{R}$ to
denote injections of a name into disjoint copies of the name
space. There are numerous ways to accomplish this. One example can be
found in \cite{MeredithR05}. This notation overloads to vectors of
names: $\vec{x}^{\pi} := (x_{i}^{\pi} \; : \; 0 \leq i < |\vec{x}| )$ where $\pi \in \{L,R\}$.

We also use $P^{\Box} := P|\Box$.

In \cite{MeredithR05} an interpretation of the new operator is
given. It turns out that there are several possible interpretations
all enjoying the requisite algebraic properties of the operator (see
\cite{milner91polyadicpi}). We will therefore make liberal use of
$(\nu\; \vec{x})P$.

% subsection the_syntax_and_semantics_of_the_notation_system (end)   

\input{qm2pi.qmops} 

\input{qm2pi.sterngerlach} 

\input{qm2pi.metric} 

% section concurrent_process_calculi (end)

%\input{qm2pi.proofsketch}

% section proof sketch (end)

%\input{qm2pi.slviaknots} 

% section spatial logic via knots (end)

\input{qm2pi.conclusion}

% section conclusion (end)

%\input{qm2pi.dtcodes} 

% section wiring algorithm (end)

\input{qm2pi.ack} 

% section acknowledgments (end)

\newpage


\bibliographystyle{plain}   
\bibliography{../../biblios/main.bib}

\input{qm2pi.rhodetails}

\end{document}



\end{document}

 

% section concurrent_process_calculi (end)

%\documentclass[12pt]{llncs}
%\documentclass{jktr}

\usepackage[pdftex]{hyperref}                   
\usepackage {listings}
\usepackage {mathpartir}
\usepackage{bcprules}
%\usepackage{listings}
                       
\usepackage{graphicx} 
%\usepackage[margins=2.5cm,nohead,nofoot]{geometry}
%\usepackage{geometry}
\usepackage{amsfonts}
\usepackage{amstext}
\usepackage{latexsym}
\usepackage{amssymb}
\usepackage{color}


%\include{myPreamble}
\documentclass[12pt]{llncs}
%\documentclass{jktr}

\usepackage[pdftex]{hyperref}                   
\usepackage {listings}
\usepackage {mathpartir}
\usepackage{bcprules}
%\usepackage{listings}
                       
\usepackage{graphicx} 
%\usepackage[margins=2.5cm,nohead,nofoot]{geometry}
%\usepackage{geometry}
\usepackage{amsfonts}
\usepackage{amstext}
\usepackage{latexsym}
\usepackage{amssymb}
\usepackage{color}


%\include{myPreamble}
\include{qm2pi.local} 

%\ifpdf
%\usepackage[pdftex]{graphicx}
%\else
%\usepackage{graphicx}
%\fi

 % \ifpdf
%  \usepackage{pdfsync}
%  \if


%\title{Brief Article}
%\author{David F. Snyder}
%\author{L.G. Meredith}

%\address{Dept. of Math., Texas State University--San Marcos, San Marcos, TX 78666}
       
\pagestyle{empty}


\begin{document}

\lstset{language=[Objective]Caml,frame=shadowbox}

\input{qm2pi.front}

% section front matter (end)

\input{qm2pi.intro} 
 
% section introduction (end)

% \input{qm2pi.knotations} 

% section notation (end)

\input{qm2pi.process.calculi} 

% section concurrent_process_calculi_and_spatial_logics_ (end)
    
%\input{qm2pi.knots2pi} 

%\input{qm2pi.trefoil} 

%\input{qm2pi.mainthm} 

% subsection basic_interpretation (end)

%\input{qm2pi.rho.presentation} 
\subsection{The syntax and semantics of the notation system}\label{sub:the_syntax_and_semantics_of_the_notation_system} % (fold)

We now summarize a technical presentation of the calculus that
embodies our theory of dynamics. The typical presentation of such a
calculus follows the style of giving generators and relations on
them. The grammar, below, describing term constructors, freely
generates the set of processes, $\Proc$. This set is then quotiented
by a relation known as structural congruence and it is over this set
that the notion of dynamics is expressed. This presentation is
essentially that of \cite{MeredithR05} with the addition of
polyadicity and summation. For readability we have relegated some of
the technical subtleties to an appendix.

\subsubsection{Process grammar}\label{subsub:process_grammar}

\begin{mathpar}
  \inferrule* [lab=synchronization] {} {{M} \bc \pzero \;|\; x?F \;|\; x!C }
  \and
  \inferrule* [lab=abstraction] {} {{F} \bc (x)P}
  \and
  \inferrule* [lab=concretion] {} {{C} \bc \langle Q \rangle}
  \and
  \inferrule* [lab=process] {} {{P,Q} \bc M \;| \;P|Q \;|\; @{x}}
  \and
  \inferrule* [lab=name] {} {{x} \bc \quotep{P}}
\end{mathpar} 

Note that $\vec{x}$ (resp. $\vec{P}$) denotes a vector of names
(resp. processes) of length $|\vec{x}|$ (resp. $|\vec{P}|$). We adopt
the following useful abbreviations.

\begin{mathpar}
   x?(\vec{y}).P := x.(\vec{y})P \and  x\clift{\vec{P}} := x.\clift{\vec{P}}
   \and x!(y) := \lift{x}{\dropn{y}}
   \and \Pi_{i=0}^{n-1}P_i := P_0 | \ldots | P_{n-1}
\end{mathpar}

\subsubsection{Structural congruence}

\paragraph{Free and bound names and alpha-equivalence.} At the
core of structural equivalence is alpha-equivalence which identifies
process that are the same up to a change of variable. Formally, we
recognize the distinction between free and bound names. The free names
of a process, $\freenames{P}$, may be calculated recursively as
follows:

\begin{mathpar}
\freenames{\pzero} := \emptyset
  \and \\
  \freenames{x?(y).P} := \{ x \} \cup (\freenames{P} \setminus \{ y \})
  \and 
  \freenames{x!\langle P \rangle} := \{ x \} \cup \{ P \} 
  \and \\
  \freenames{P|Q} := \freenames{P} \cup \freenames{Q}
  \and \\
  \freenames{@{x}} := \{ x \}
\end{mathpar}

$\pi$
$\quotep{\pi}$

$\freenames{-} : \pi \to \mathcal{P}(\quotep{\pi})$

\begin{eqnarray*}
  \freenames{\pzero} & := & \emptyset \\
  \freenames{x?(y).P} & := & \{ x \} \cup (\freenames{P} \setminus \{ y \}) \\
  \freenames{x!\langle P \rangle} & := & \{ x \} \cup \{ P \} \\
  \freenames{P|Q} & := & \freenames{P} \cup \freenames{Q} \\
  \freenames{\dropn{x}} & := & \{ x \}
\end{eqnarray*}

The bound names of a process, $\boundnames{P}$, are those names occurring in $P$
that are not free. For example, in $x?(y).0$, the name $x$ is free, while $y$ is bound.

\begin{mathpar}
  \inferrule* [lab=monoidal-laws] {} { P|Q \equiv Q|P \and P|0 \equiv P \and P|(Q|R) \equiv (P|Q)|R }
\end{mathpar}

\begin{mathpar}
  \inferrule* [lab=alpha-equivalence] {} { (x)P \equiv (y)P\{y/x\} \and y \not\in \freenames{P} }
\end{mathpar}

\begin{definition}
Then two processes, $P,Q$, are alpha-equivalent if $P = Q\{\vec{y}/\vec{x}\}$ for
some $\vec{x} \in \boundnames{Q},\vec{y} \in \boundnames{P}$, where $Q\{\vec{y}/\vec{x}\}$
denotes the capture-avoiding substitution of $\vec{y}$ for $\vec{x}$ in $Q$.
\end{definition}

\begin{definition}
  The {\em structural congruence} \cite{SangiorgiWalker} , $\equiv$,
  between processes is the least congruence containing
  alpha-equivalence, satisfying the abelian monoid laws
  (associativity, commutativity and $\pzero$ as identity) for parallel
  composition $|$ and for summation $+$.
\end{definition}

\subsection{Name equivalence}

We take name equivalence, written $\nameeq$, to be the smallest
equivalence relation generated by the following rules.

\begin{mathpar}
\inferrule*[lab=Quote-drop]
{ }
{ \quotep{@{x}} \nameeq x }

\inferrule*[lab=Struct-equiv]
{ P \scong Q }
{ \quotep{P} \nameeq \quotep{Q} }
\end{mathpar}

The astute reader will have noticed that the mutual recursion of names
and processes imposes a mutual recursion on alpha-equivalence and
structural equivalence via name-equivalence. Fortunately, all of this
works out pleasantly and we may calculate in the natural way, free of
concern. The reader interested in the details is referred to the
appendix \ref{appendix:rho_details}.

\subsection{Substitution}

We use $\Proc$ for the set of processes, $\QProc$ for the set of
names, and $\id{\{}\vec{y} / \vec{x} \id{\}}$ to denote partial maps,
$s : \QProc \rightarrow \QProc$. A map, $s$ lifts, uniquely, to a map
on process terms, $\widehat{s} : \Proc \rightarrow \Proc$ by the
following equations.

\begin{mathpar}
  (0) \psubstp{Q}{P} := 0 \\
  (R \juxtap S) \psubstp{Q}{P}
  :=    
  (R)\psubstp{Q}{P} \juxtap (S) \psubstp{Q}{P} \\
  (x?(y).R) \psubstp{Q}{P}    
  :=    
  (x)\substp{Q}{P} (z)\concat( (R \psubstn{z}{y}) \psubstp{Q}{P} ) \\
  (\lift{x}{R}) \psubstp{Q}{P}  
  :=
  \lift{(x)\substp{Q}{P}}{ R \psubstp{Q}{P} } \\
%   (\dropn{x})  \psubstp{Q}{P}       
%   := 
%   \left\{ 
%     \begin{array}{ccc} 
%       \dropn{\quotep{Q}} & & x \nameeq \quotep{P} \\
%       \dropn{x} & & otherwise \\
%     \end{array}
%   \right. 
  (\dropn{x})  \psubstp{Q}{P}       
  := 
  \left\{ 
    \begin{array}{ccc} 
      Q & & x \nameeq \quotep{P} \\
      \dropn{x} & & otherwise \\
    \end{array}
  \right.
\end{mathpar}
 

where

\begin{eqnarray}
  (x)\id{\{} \lpquote Q \rpquote / \lpquote P \rpquote \id{\}}            = 
  \left\{ 
    \begin{array}{ccc}
      \lpquote Q \rpquote & & x \nameeq \lpquote P \rpquote \\
      x & & otherwise \\
    \end{array}
  \right. \nonumber
\end{eqnarray}

and $z$ is chosen distinct from $\quotep{P}$, $\quotep{Q}$, the free
names in $Q$, and all the names in $R$. Our $\alpha$-equivalence will
be built in the standard way from this substitution.

\begin{remark}\label{rem:no_self_referential_names}
  One consequence of these definitions is that $\forall P. \quotep{P}
  \not\in \freenames{P}$.
\end{remark}

\subsection{ Dynamic quote: an example }

Anticipating something of what's to come, consider applying the
substitution, $\widehat{\id{\{}u / z \id{\}}}$, to the following pair
of processes, $\lift{w}{y!(z)}$ and $w[ \lpquote y!(z) \rpquote ]$.

\begin{eqnarray}
	\lift{w}{y!(z)}\widehat{\id{\{}u / z \id{\}}}
		& = &
		\lift{w}{y!(u)} \nonumber\\
	w[ \lpquote y!(z) \rpquote ] \widehat{ \id{\{}u / z \id{\}} }
		& = &
		w[ \lpquote y!(z) \rpquote ] \nonumber
\end{eqnarray}

Because the body of the process between quotes is impervious to
substitution, we get radically different answers. In fact, by
examining the first process in an input context,
e.g. $x?(z).\lift{w}{y!(z)}$, we see that the process under the lift
operator may be shaped by prefixed inputs binding a name inside it. In
this sense, the lift operator will be seen as a way to dynamically
construct processes before reifying them as names.

Finally equipped with these standard features we can present the
dynamics of the calculus.

\subsubsection{Operational semantics} 

Finally, we introduce the computational dynamics. What marks these
algebras as distinct from other more traditionally studied algebraic
structures, e.g. vector spaces or polynomial rings, is the manner in
which dynamics is captured. In traditional structures, dynamics is typically
expressed through morphisms between such structures, as in linear maps
between vector spaces or morphisms between rings. In algebras
associated with the semantics of computation, the dynamics is
expressed as part of the algebraic structure itself, through a
reduction reduction relation typically denoted by $\red$. Below, we
give a recursive presentation of this relation for the calculus used
in the encoding.

$\red \subseteq \pi \times \pi$
$\red : \pi \to \mathcal{P}(\pi)$

\begin{mathpar}
  \inferrule* [lab=Comm] { \textsf{match}( x_{src}, x_{trgt} ) } { x_{trgt}?(y)P \; | \; x_{src}!\langle {Q} \rangle \red P\{\quotep{Q}/y}\} }
  \and \\
  \inferrule* [lab=Par] {{P} \red {P}'} {{{P} | {Q}} \red {{P}' | {Q}}}
  \and
  \inferrule* [lab=Equiv]{{{P} \scong {P}'} \andalso {{P}' \red {Q}'} \andalso {{Q}' \scong {Q}}}{{P} \red {Q}}
\end{mathpar}

\begin{eqnarray*}
  match_{\equiv} (\quotep{P},\quotep{Q}) & := & P \equiv Q \\
  match_{\dagger}(\quotep{P},\quotep{Q}) & := & \forall R. P|Q \red^{*} R => R \red^{*} 0 \\
  match_{K}(\quotep{P},\quotep{Q}) & := & K \mbox{ for some context } K
\end{eqnarray*}

$u?(x)P | u!\langle Q \rangle \red P\{\quotep{Q}/x\}$

%We write $\wred$ for $\red^*$, and $P\red$ if $\exists Q $ such that $ P \red Q$.
We write $P\red$ if $\exists Q $ such that $ P \red Q$ and $P\not\red$, otherwise.

\section{Replication}

As mentioned before, it is known that replication (and hence
recursion) can be implemented in a higher-order process algebra
\cite{SangiorgiWalker}. As our first example of calculation with the
machinery thus far presented we give the construction explicitly in
the {\rhoc}.

\begin{eqnarray}
	D_{x} & := & \prefix{x}{y}{(\binpar{\outputp{x}{y}}{@{y}})} \nonumber\\
	\bangp_{x}{P} & := & \binpar{{x}!\langle{\binpar{D_{x}}{P}}\rangle}{D_{x}} \nonumber
\end{eqnarray}

\begin{eqnarray}
	\bangp_{x}{P} & & \nonumber\\
	=
	& {x}!\langle{(\prefix{x}{y}{(\outputp{x}{y} | @{y})) | P}}\rangle 
	      | \prefix{x}{y}{(\outputp{x}{y} | @{y})} & \nonumber\\
	\red
	& (\outputp{x}{y} | @{y})\substn{\quotep{(\prefix{x}{y}{(@{y} | \outputp{x}{y})) | P}}}{y} & \nonumber\\
	=
	& \outputp{x}{\quotep{(\prefix{x}{y}{(\outputp{x}{y} | @{y})) | P}}}
	  | {(\prefix{x}{y}{(\outputp{x}{y} | @{y})) | P}} & \nonumber\\
	\red
	& \ldots & \nonumber\\
	\red^*
	& P | P | \ldots & \nonumber
\end{eqnarray}

Of course, this encoding, as an implementation, runs away, unfolding
$\bangp{P}$ eagerly. A lazier and more implementable replication
operator, restricted to input-guarded processes, may be obtained as follows.

\begin{eqnarray}
\bangp{\prefix{u}{v}{P}} 
	:= 
	\binpar{\lift{x}{\prefix{u}{v}{(\binpar{D(x)}{P})}}}{D(x)} \nonumber
\end{eqnarray}

\begin{remark}
  Note that the lazier definition still does not deal with summation
  or mixed summation (i.e. sums over input and output). The reader is
  invited to construct definitions of replication that deal with these
  features. 

  Further, the definitions are parameterized in a name, $x$. Can you,
  gentle reader, make a definition that eliminates this parameter and
  guarantees no accidental interaction between the replication
  machinery and the process being replicated -- i.e. no accidental
  sharing of names used by the process to get its work done and the
  name(s) used by the replication to effect copying. This latter
  revision of the definition of replication is crucial to obtaining
  the expected identity $!!P \sim !P$.
\end{remark}

\begin{remark}\label{rem:paradoxical_combinator}
  The reader familiar with the lambda calculus will have noticed the
  similarity between $D$ and the paradoxical combinator.

  [Ed. note: the existence of this seems to suggest we have to be more
  restrictive on the set of processes and names we admit if we are to
  support no-cloning.]
\end{remark}

\subsubsection{Bisimulation}

The computational dynamics gives rise to another kind of equivalence,
the equivalence of computational behavior. As previously mentioned
this is typically captured \emph{via} some form of bisimulation.

% The notion we use in this paper is weak barbed bisimulation
% \cite{milner91polyadicpi}.

The notion we use in this paper is derived from weak barbed
bisimulation \cite{milner91polyadicpi}. 

\begin{definition}
An \emph{observation relation}, $\downarrow_{\mathcal N}$, over a set
of names, $\mathcal N$, is the smallest relation satisfying the rules
below.

\infrule[Out-barb]{y \in {\mathcal N}, \; x \nameeq y}
		  {\outputp{x}{v} \downarrow_{\mathcal N} x}
\infrule[Par-barb]{\mbox{$P\downarrow_{\mathcal N} x$ or $Q\downarrow_{\mathcal N} x$}}
		  {\binpar{P}{Q} \downarrow_{\mathcal N} x}

We write $P \Downarrow_{\mathcal N} x$ if there is $Q$ such that 
$P \wred Q$ and $Q \downarrow_{\mathcal N} x$.
\end{definition}

\begin{definition}
%\label{def.bbisim}
An  ${\mathcal N}$-\emph{barbed bisimulation} over a set of names, ${\mathcal N}$, is a symmetric binary relation 
${\mathcal S}_{\mathcal N}$ between agents such that $P\rel{S}_{\mathcal N}Q$ implies:
\begin{enumerate}
\item If $P \red P'$ then $Q \wred Q'$ and $P'\rel{S}_{\mathcal N} Q'$.
\item If $P\downarrow_{\mathcal N} x$, then $Q\Downarrow_{\mathcal N} x$.
\end{enumerate}
$P$ is ${\mathcal N}$-barbed bisimilar to $Q$, written
$P \wbbisim_{\mathcal N} Q$, if $P \rel{S}_{\mathcal N} Q$ for some ${\mathcal N}$-barbed bisimulation ${\mathcal S}_{\mathcal N}$.
\end{definition}

$\mathcal{R} \subseteq \pi \times \pi$

$P \mathcal{R} Q => \forall P'. P \red P' \Rightarrow \exists Q'. Q \red Q', P' \mathcal{R} Q'$

$P \vdash x \Rightarrow Q \vdash x$

\begin{mathpar}
  \inferrule*[lab=Out-barb]{x \nameeq y}{{y}!\langle{Q}\rangle \vdash x}
  \and
  \inferrule*[lab=Par-barb]{\mbox{$P\vdash x$ or $Q\vdash x$}}{\binpar{P}{Q} \vdash x}
\end{mathpar}

\subsubsection{Contexts}

One of the principle advantages of computational calculi like the
$\pi$-calculus is a well-defined notion of context,
contextual-equivalence and a correlation between
contextual-equivalence and notions of bisimulation. The notion of
context allows the decomposition of a process into (sub-)process and
its syntactic environment, its context. Thus, a context may be
thought of as a process with a ``hole'' (written $\Box$) in it. The
application of a context $M$ to a process $P$, written $M[P]$, is
tantamount to filling the hole in $M$ with $P$. In this paper we do
not need the full weight of this theory, but do make use of the notion
of context in the proof the main theorem. 

\begin{mathpar}
  \inferrule* [lab=summation] {} {{M_{M},M_{N}} \bc \Box \;|\; x.M_{A} \;|\; M_{M}+M_{N}}
  \and
  \inferrule* [lab=agent] {} {{M_{A}} \bc (\vec{x})M_{P} \;| \; \clift{P_0,\ldots,M_{P},\ldots,P_N}}
  \and \\
  \inferrule* [lab=process] {} {{M_{P}} \bc M_{N} \;| \;P|M_{P} }
\end{mathpar} 

\begin{mathpar}
  \inferrule* [lab=sychronization] {} {M_{N} \bc \Box \;|\; x?M_{F} \;|\; x!M_{C}}
  \and
  \inferrule* [lab=abstraction] {} {{M_{F}} \bc (x)M_{P} }
  \and
  \inferrule* [lab=concretion] {} {{M_{C}} \bc \langle M_{P} \rangle }
  \and \\
  \inferrule* [lab=process] {} {{M_{P}} \bc M_{N} \;| \;P|M_{P} }
\end{mathpar}

\begin{definition}[contextual application] Given a context $M$, and
  process $P$, we define the \emph{contextual application}, $M[P] :=
  M\{P/\Box\}$. That is, the contextual application of M to P is the
  substitution of $P$ for $\Box$ in $M$.
\end{definition}

$\meaningof{-} : L \to \mathcal{P}(\pi)$

\begin{mathpar}
  \inferrule* [lab=collection] {} {\meaningof{true} = \pi, \and \meaningof{~E} = \pi \setminus \meaningof{E}, \and \meaningof{E_{1} \& E_{2}} = \meaningof{E_{1}} \cap \meaningof{E_{2}}}
\end{mathpar}

\begin{mathpar}
  \inferrule* [lab=structure] {} {\meaningof{0} = \{ P \in \pi | P \equiv 0 \}, \and \\ \meaningof{E_1 | E_2} = \{ P \in \pi | P \equiv P_{1} | P_{2}, P_{1} \in \meaningof{E_{1}}, P_{2} \in \meaningof{E_2}\} }
\end{mathpar}

\begin{mathpar}
 \inferrule* [lab=behavior] {} {\meaningof{\langle a?b \rangle E} = \{ P \in \pi | P \equiv Q | u?(y)P', \\ \and \\\\ \and \\ \;\;\; u \in \meaningof{a}, \forall z.P'\{z/y\} \in \meaningof{E\{z/b\}}\}, \and \\ \meaningof{a!E} = \{ P \in \pi | P \equiv Q | x!\langle P' \rangle, x \in \meaningof{a} P' \in \meaningof{E}\} }
\end{mathpar}

\begin{mathpar}
 \inferrule* [lab=nominal] {} {\meaningof{\quotep{E}} = \{ \quotep{P} \in \quotep{\pi} | P \in \meaningof{E} \}, \and \meaningof{\quotep{P}} = \{ \quotep{Q} \in \quotep{\pi} | P \equiv Q \} \and \\ \meaningof{@\quotep{E}} = \{ P \in \pi | P \equiv @x, x \in \meaningof{E} \}}
\end{mathpar}

\begin{eqnarray*}
  \\
  \meaningof{-} : TS \to ST
\end{eqnarray*}

\begin{eqnarray*}
  \\
  L : TS \to ST
\end{eqnarray*}

\begin{eqnarray*}
  \\
  P \models E \iff P \in \meaningof{E}
\end{eqnarray*}

\begin{eqnarray*}
  P \approx_{L} Q \iff \forall E \in L. P \models E \iff Q \models E
\end{eqnarray*}

\begin{eqnarray*}
  P \approx_{K} Q
\end{eqnarray*}

\begin{eqnarray*}
  P \approx Q
\end{eqnarray*}

$\approx_{K} = \approx = \approx_{L}$

\subsubsection{Contextual duality}

Note that contexts extend the quotation operation to a family of
operations from processes to names. Given a context, $M$, we can
define a \emph{nominal context}, $\quotep{M}$ by $\quotep{M}[P] :=
\quotep{M[P]}$. To foreshadow what is to come we observe that these
operations enjoy a duality with processes very much like the duality
between vectors and maps from vectors to scalars.

Further, because the calculus is essentially higher-order, we have a
correspondence between contexts and processes. More specifically,
given a name $x$ and a context $M$ we can construct $M^{*}_{x}$ such
that 

\begin{mathpar}
  M^{*}_{x} | \lift{x}{P} \red M[P]
\end{mathpar}

namely,

\begin{mathpar}
  M^{*}_{x} := x?(u).M[\dropn{u}]
\end{mathpar}

The dependence of $M^{*}_{x}$ on a name makes it an abstraction, 

\begin{mathpar}
  M^{*} := (x)x?(u).M[\dropn{u}]
\end{mathpar}

\subsection{Additional notation}

It will sometimes be convenient to denote the process a name
quotes. We already have the notation $x = \quotep{P}$, but it will be
convenient to introduce an alternate notation, $\procn{x}$, when we
want to emphasize the connection to the use of the name. Note that, by
virtue of name equivalence, $\quotep{\procn{x}} \nameeq x$; so, the
notation is consistent with previous definitions.

Further, because names have structure it is possible to effect
substitutions on the basis of that structure. This means we need to
upgrade our notation for substitutions, which we accomplish by
adapting comprehension notation. Thus,

\begin{mathpar}
  P\{ y / x : x \in S \}
\end{mathpar}

is interpreted to mean the process derived from P by replacing (in a
capture-avoiding manner) each occurrence of $x$ in $S$ by $y$. For example,

\begin{mathpar}
  P\{ \quotep{\procn{x}|\procn{x}} / x : x \in \freenames{P} \}
\end{mathpar}

will replace each (occurrence) of a free name $x$ in $P$ by
$\quotep{\procn{x}|\procn{x}}$.

Also, we will avail ourselves of the notation $x^{L}$ and $x^{R}$ to
denote injections of a name into disjoint copies of the name
space. There are numerous ways to accomplish this. One example can be
found in \cite{MeredithR05}. This notation overloads to vectors of
names: $\vec{x}^{\pi} := (x_{i}^{\pi} \; : \; 0 \leq i < |\vec{x}| )$ where $\pi \in \{L,R\}$.

We also use $P^{\Box} := P|\Box$.

In \cite{MeredithR05} an interpretation of the new operator is
given. It turns out that there are several possible interpretations
all enjoying the requisite algebraic properties of the operator (see
\cite{milner91polyadicpi}). We will therefore make liberal use of
$(\nu\; \vec{x})P$.

% subsection the_syntax_and_semantics_of_the_notation_system (end)   

\input{qm2pi.qmops} 

\input{qm2pi.sterngerlach} 

\input{qm2pi.metric} 

% section concurrent_process_calculi (end)

%\input{qm2pi.proofsketch}

% section proof sketch (end)

%\input{qm2pi.slviaknots} 

% section spatial logic via knots (end)

\input{qm2pi.conclusion}

% section conclusion (end)

%\input{qm2pi.dtcodes} 

% section wiring algorithm (end)

\input{qm2pi.ack} 

% section acknowledgments (end)

\newpage


\bibliographystyle{plain}   
\bibliography{../../biblios/main.bib}

\input{qm2pi.rhodetails}

\end{document}

 

%\ifpdf
%\usepackage[pdftex]{graphicx}
%\else
%\usepackage{graphicx}
%\fi

 % \ifpdf
%  \usepackage{pdfsync}
%  \if


%\title{Brief Article}
%\author{David F. Snyder}
%\author{L.G. Meredith}

%\address{Dept. of Math., Texas State University--San Marcos, San Marcos, TX 78666}
       
\pagestyle{empty}


\begin{document}

\lstset{language=[Objective]Caml,frame=shadowbox}

\documentclass[12pt]{llncs}
%\documentclass{jktr}

\usepackage[pdftex]{hyperref}                   
\usepackage {listings}
\usepackage {mathpartir}
\usepackage{bcprules}
%\usepackage{listings}
                       
\usepackage{graphicx} 
%\usepackage[margins=2.5cm,nohead,nofoot]{geometry}
%\usepackage{geometry}
\usepackage{amsfonts}
\usepackage{amstext}
\usepackage{latexsym}
\usepackage{amssymb}
\usepackage{color}


%\include{myPreamble}
\include{qm2pi.local} 

%\ifpdf
%\usepackage[pdftex]{graphicx}
%\else
%\usepackage{graphicx}
%\fi

 % \ifpdf
%  \usepackage{pdfsync}
%  \if


%\title{Brief Article}
%\author{David F. Snyder}
%\author{L.G. Meredith}

%\address{Dept. of Math., Texas State University--San Marcos, San Marcos, TX 78666}
       
\pagestyle{empty}


\begin{document}

\lstset{language=[Objective]Caml,frame=shadowbox}

\input{qm2pi.front}

% section front matter (end)

\input{qm2pi.intro} 
 
% section introduction (end)

% \input{qm2pi.knotations} 

% section notation (end)

\input{qm2pi.process.calculi} 

% section concurrent_process_calculi_and_spatial_logics_ (end)
    
%\input{qm2pi.knots2pi} 

%\input{qm2pi.trefoil} 

%\input{qm2pi.mainthm} 

% subsection basic_interpretation (end)

%\input{qm2pi.rho.presentation} 
\subsection{The syntax and semantics of the notation system}\label{sub:the_syntax_and_semantics_of_the_notation_system} % (fold)

We now summarize a technical presentation of the calculus that
embodies our theory of dynamics. The typical presentation of such a
calculus follows the style of giving generators and relations on
them. The grammar, below, describing term constructors, freely
generates the set of processes, $\Proc$. This set is then quotiented
by a relation known as structural congruence and it is over this set
that the notion of dynamics is expressed. This presentation is
essentially that of \cite{MeredithR05} with the addition of
polyadicity and summation. For readability we have relegated some of
the technical subtleties to an appendix.

\subsubsection{Process grammar}\label{subsub:process_grammar}

\begin{mathpar}
  \inferrule* [lab=synchronization] {} {{M} \bc \pzero \;|\; x?F \;|\; x!C }
  \and
  \inferrule* [lab=abstraction] {} {{F} \bc (x)P}
  \and
  \inferrule* [lab=concretion] {} {{C} \bc \langle Q \rangle}
  \and
  \inferrule* [lab=process] {} {{P,Q} \bc M \;| \;P|Q \;|\; @{x}}
  \and
  \inferrule* [lab=name] {} {{x} \bc \quotep{P}}
\end{mathpar} 

Note that $\vec{x}$ (resp. $\vec{P}$) denotes a vector of names
(resp. processes) of length $|\vec{x}|$ (resp. $|\vec{P}|$). We adopt
the following useful abbreviations.

\begin{mathpar}
   x?(\vec{y}).P := x.(\vec{y})P \and  x\clift{\vec{P}} := x.\clift{\vec{P}}
   \and x!(y) := \lift{x}{\dropn{y}}
   \and \Pi_{i=0}^{n-1}P_i := P_0 | \ldots | P_{n-1}
\end{mathpar}

\subsubsection{Structural congruence}

\paragraph{Free and bound names and alpha-equivalence.} At the
core of structural equivalence is alpha-equivalence which identifies
process that are the same up to a change of variable. Formally, we
recognize the distinction between free and bound names. The free names
of a process, $\freenames{P}$, may be calculated recursively as
follows:

\begin{mathpar}
\freenames{\pzero} := \emptyset
  \and \\
  \freenames{x?(y).P} := \{ x \} \cup (\freenames{P} \setminus \{ y \})
  \and 
  \freenames{x!\langle P \rangle} := \{ x \} \cup \{ P \} 
  \and \\
  \freenames{P|Q} := \freenames{P} \cup \freenames{Q}
  \and \\
  \freenames{@{x}} := \{ x \}
\end{mathpar}

$\pi$
$\quotep{\pi}$

$\freenames{-} : \pi \to \mathcal{P}(\quotep{\pi})$

\begin{eqnarray*}
  \freenames{\pzero} & := & \emptyset \\
  \freenames{x?(y).P} & := & \{ x \} \cup (\freenames{P} \setminus \{ y \}) \\
  \freenames{x!\langle P \rangle} & := & \{ x \} \cup \{ P \} \\
  \freenames{P|Q} & := & \freenames{P} \cup \freenames{Q} \\
  \freenames{\dropn{x}} & := & \{ x \}
\end{eqnarray*}

The bound names of a process, $\boundnames{P}$, are those names occurring in $P$
that are not free. For example, in $x?(y).0$, the name $x$ is free, while $y$ is bound.

\begin{mathpar}
  \inferrule* [lab=monoidal-laws] {} { P|Q \equiv Q|P \and P|0 \equiv P \and P|(Q|R) \equiv (P|Q)|R }
\end{mathpar}

\begin{mathpar}
  \inferrule* [lab=alpha-equivalence] {} { (x)P \equiv (y)P\{y/x\} \and y \not\in \freenames{P} }
\end{mathpar}

\begin{definition}
Then two processes, $P,Q$, are alpha-equivalent if $P = Q\{\vec{y}/\vec{x}\}$ for
some $\vec{x} \in \boundnames{Q},\vec{y} \in \boundnames{P}$, where $Q\{\vec{y}/\vec{x}\}$
denotes the capture-avoiding substitution of $\vec{y}$ for $\vec{x}$ in $Q$.
\end{definition}

\begin{definition}
  The {\em structural congruence} \cite{SangiorgiWalker} , $\equiv$,
  between processes is the least congruence containing
  alpha-equivalence, satisfying the abelian monoid laws
  (associativity, commutativity and $\pzero$ as identity) for parallel
  composition $|$ and for summation $+$.
\end{definition}

\subsection{Name equivalence}

We take name equivalence, written $\nameeq$, to be the smallest
equivalence relation generated by the following rules.

\begin{mathpar}
\inferrule*[lab=Quote-drop]
{ }
{ \quotep{@{x}} \nameeq x }

\inferrule*[lab=Struct-equiv]
{ P \scong Q }
{ \quotep{P} \nameeq \quotep{Q} }
\end{mathpar}

The astute reader will have noticed that the mutual recursion of names
and processes imposes a mutual recursion on alpha-equivalence and
structural equivalence via name-equivalence. Fortunately, all of this
works out pleasantly and we may calculate in the natural way, free of
concern. The reader interested in the details is referred to the
appendix \ref{appendix:rho_details}.

\subsection{Substitution}

We use $\Proc$ for the set of processes, $\QProc$ for the set of
names, and $\id{\{}\vec{y} / \vec{x} \id{\}}$ to denote partial maps,
$s : \QProc \rightarrow \QProc$. A map, $s$ lifts, uniquely, to a map
on process terms, $\widehat{s} : \Proc \rightarrow \Proc$ by the
following equations.

\begin{mathpar}
  (0) \psubstp{Q}{P} := 0 \\
  (R \juxtap S) \psubstp{Q}{P}
  :=    
  (R)\psubstp{Q}{P} \juxtap (S) \psubstp{Q}{P} \\
  (x?(y).R) \psubstp{Q}{P}    
  :=    
  (x)\substp{Q}{P} (z)\concat( (R \psubstn{z}{y}) \psubstp{Q}{P} ) \\
  (\lift{x}{R}) \psubstp{Q}{P}  
  :=
  \lift{(x)\substp{Q}{P}}{ R \psubstp{Q}{P} } \\
%   (\dropn{x})  \psubstp{Q}{P}       
%   := 
%   \left\{ 
%     \begin{array}{ccc} 
%       \dropn{\quotep{Q}} & & x \nameeq \quotep{P} \\
%       \dropn{x} & & otherwise \\
%     \end{array}
%   \right. 
  (\dropn{x})  \psubstp{Q}{P}       
  := 
  \left\{ 
    \begin{array}{ccc} 
      Q & & x \nameeq \quotep{P} \\
      \dropn{x} & & otherwise \\
    \end{array}
  \right.
\end{mathpar}
 

where

\begin{eqnarray}
  (x)\id{\{} \lpquote Q \rpquote / \lpquote P \rpquote \id{\}}            = 
  \left\{ 
    \begin{array}{ccc}
      \lpquote Q \rpquote & & x \nameeq \lpquote P \rpquote \\
      x & & otherwise \\
    \end{array}
  \right. \nonumber
\end{eqnarray}

and $z$ is chosen distinct from $\quotep{P}$, $\quotep{Q}$, the free
names in $Q$, and all the names in $R$. Our $\alpha$-equivalence will
be built in the standard way from this substitution.

\begin{remark}\label{rem:no_self_referential_names}
  One consequence of these definitions is that $\forall P. \quotep{P}
  \not\in \freenames{P}$.
\end{remark}

\subsection{ Dynamic quote: an example }

Anticipating something of what's to come, consider applying the
substitution, $\widehat{\id{\{}u / z \id{\}}}$, to the following pair
of processes, $\lift{w}{y!(z)}$ and $w[ \lpquote y!(z) \rpquote ]$.

\begin{eqnarray}
	\lift{w}{y!(z)}\widehat{\id{\{}u / z \id{\}}}
		& = &
		\lift{w}{y!(u)} \nonumber\\
	w[ \lpquote y!(z) \rpquote ] \widehat{ \id{\{}u / z \id{\}} }
		& = &
		w[ \lpquote y!(z) \rpquote ] \nonumber
\end{eqnarray}

Because the body of the process between quotes is impervious to
substitution, we get radically different answers. In fact, by
examining the first process in an input context,
e.g. $x?(z).\lift{w}{y!(z)}$, we see that the process under the lift
operator may be shaped by prefixed inputs binding a name inside it. In
this sense, the lift operator will be seen as a way to dynamically
construct processes before reifying them as names.

Finally equipped with these standard features we can present the
dynamics of the calculus.

\subsubsection{Operational semantics} 

Finally, we introduce the computational dynamics. What marks these
algebras as distinct from other more traditionally studied algebraic
structures, e.g. vector spaces or polynomial rings, is the manner in
which dynamics is captured. In traditional structures, dynamics is typically
expressed through morphisms between such structures, as in linear maps
between vector spaces or morphisms between rings. In algebras
associated with the semantics of computation, the dynamics is
expressed as part of the algebraic structure itself, through a
reduction reduction relation typically denoted by $\red$. Below, we
give a recursive presentation of this relation for the calculus used
in the encoding.

$\red \subseteq \pi \times \pi$
$\red : \pi \to \mathcal{P}(\pi)$

\begin{mathpar}
  \inferrule* [lab=Comm] { \textsf{match}( x_{src}, x_{trgt} ) } { x_{trgt}?(y)P \; | \; x_{src}!\langle {Q} \rangle \red P\{\quotep{Q}/y}\} }
  \and \\
  \inferrule* [lab=Par] {{P} \red {P}'} {{{P} | {Q}} \red {{P}' | {Q}}}
  \and
  \inferrule* [lab=Equiv]{{{P} \scong {P}'} \andalso {{P}' \red {Q}'} \andalso {{Q}' \scong {Q}}}{{P} \red {Q}}
\end{mathpar}

\begin{eqnarray*}
  match_{\equiv} (\quotep{P},\quotep{Q}) & := & P \equiv Q \\
  match_{\dagger}(\quotep{P},\quotep{Q}) & := & \forall R. P|Q \red^{*} R => R \red^{*} 0 \\
  match_{K}(\quotep{P},\quotep{Q}) & := & K \mbox{ for some context } K
\end{eqnarray*}

$u?(x)P | u!\langle Q \rangle \red P\{\quotep{Q}/x\}$

%We write $\wred$ for $\red^*$, and $P\red$ if $\exists Q $ such that $ P \red Q$.
We write $P\red$ if $\exists Q $ such that $ P \red Q$ and $P\not\red$, otherwise.

\section{Replication}

As mentioned before, it is known that replication (and hence
recursion) can be implemented in a higher-order process algebra
\cite{SangiorgiWalker}. As our first example of calculation with the
machinery thus far presented we give the construction explicitly in
the {\rhoc}.

\begin{eqnarray}
	D_{x} & := & \prefix{x}{y}{(\binpar{\outputp{x}{y}}{@{y}})} \nonumber\\
	\bangp_{x}{P} & := & \binpar{{x}!\langle{\binpar{D_{x}}{P}}\rangle}{D_{x}} \nonumber
\end{eqnarray}

\begin{eqnarray}
	\bangp_{x}{P} & & \nonumber\\
	=
	& {x}!\langle{(\prefix{x}{y}{(\outputp{x}{y} | @{y})) | P}}\rangle 
	      | \prefix{x}{y}{(\outputp{x}{y} | @{y})} & \nonumber\\
	\red
	& (\outputp{x}{y} | @{y})\substn{\quotep{(\prefix{x}{y}{(@{y} | \outputp{x}{y})) | P}}}{y} & \nonumber\\
	=
	& \outputp{x}{\quotep{(\prefix{x}{y}{(\outputp{x}{y} | @{y})) | P}}}
	  | {(\prefix{x}{y}{(\outputp{x}{y} | @{y})) | P}} & \nonumber\\
	\red
	& \ldots & \nonumber\\
	\red^*
	& P | P | \ldots & \nonumber
\end{eqnarray}

Of course, this encoding, as an implementation, runs away, unfolding
$\bangp{P}$ eagerly. A lazier and more implementable replication
operator, restricted to input-guarded processes, may be obtained as follows.

\begin{eqnarray}
\bangp{\prefix{u}{v}{P}} 
	:= 
	\binpar{\lift{x}{\prefix{u}{v}{(\binpar{D(x)}{P})}}}{D(x)} \nonumber
\end{eqnarray}

\begin{remark}
  Note that the lazier definition still does not deal with summation
  or mixed summation (i.e. sums over input and output). The reader is
  invited to construct definitions of replication that deal with these
  features. 

  Further, the definitions are parameterized in a name, $x$. Can you,
  gentle reader, make a definition that eliminates this parameter and
  guarantees no accidental interaction between the replication
  machinery and the process being replicated -- i.e. no accidental
  sharing of names used by the process to get its work done and the
  name(s) used by the replication to effect copying. This latter
  revision of the definition of replication is crucial to obtaining
  the expected identity $!!P \sim !P$.
\end{remark}

\begin{remark}\label{rem:paradoxical_combinator}
  The reader familiar with the lambda calculus will have noticed the
  similarity between $D$ and the paradoxical combinator.

  [Ed. note: the existence of this seems to suggest we have to be more
  restrictive on the set of processes and names we admit if we are to
  support no-cloning.]
\end{remark}

\subsubsection{Bisimulation}

The computational dynamics gives rise to another kind of equivalence,
the equivalence of computational behavior. As previously mentioned
this is typically captured \emph{via} some form of bisimulation.

% The notion we use in this paper is weak barbed bisimulation
% \cite{milner91polyadicpi}.

The notion we use in this paper is derived from weak barbed
bisimulation \cite{milner91polyadicpi}. 

\begin{definition}
An \emph{observation relation}, $\downarrow_{\mathcal N}$, over a set
of names, $\mathcal N$, is the smallest relation satisfying the rules
below.

\infrule[Out-barb]{y \in {\mathcal N}, \; x \nameeq y}
		  {\outputp{x}{v} \downarrow_{\mathcal N} x}
\infrule[Par-barb]{\mbox{$P\downarrow_{\mathcal N} x$ or $Q\downarrow_{\mathcal N} x$}}
		  {\binpar{P}{Q} \downarrow_{\mathcal N} x}

We write $P \Downarrow_{\mathcal N} x$ if there is $Q$ such that 
$P \wred Q$ and $Q \downarrow_{\mathcal N} x$.
\end{definition}

\begin{definition}
%\label{def.bbisim}
An  ${\mathcal N}$-\emph{barbed bisimulation} over a set of names, ${\mathcal N}$, is a symmetric binary relation 
${\mathcal S}_{\mathcal N}$ between agents such that $P\rel{S}_{\mathcal N}Q$ implies:
\begin{enumerate}
\item If $P \red P'$ then $Q \wred Q'$ and $P'\rel{S}_{\mathcal N} Q'$.
\item If $P\downarrow_{\mathcal N} x$, then $Q\Downarrow_{\mathcal N} x$.
\end{enumerate}
$P$ is ${\mathcal N}$-barbed bisimilar to $Q$, written
$P \wbbisim_{\mathcal N} Q$, if $P \rel{S}_{\mathcal N} Q$ for some ${\mathcal N}$-barbed bisimulation ${\mathcal S}_{\mathcal N}$.
\end{definition}

$\mathcal{R} \subseteq \pi \times \pi$

$P \mathcal{R} Q => \forall P'. P \red P' \Rightarrow \exists Q'. Q \red Q', P' \mathcal{R} Q'$

$P \vdash x \Rightarrow Q \vdash x$

\begin{mathpar}
  \inferrule*[lab=Out-barb]{x \nameeq y}{{y}!\langle{Q}\rangle \vdash x}
  \and
  \inferrule*[lab=Par-barb]{\mbox{$P\vdash x$ or $Q\vdash x$}}{\binpar{P}{Q} \vdash x}
\end{mathpar}

\subsubsection{Contexts}

One of the principle advantages of computational calculi like the
$\pi$-calculus is a well-defined notion of context,
contextual-equivalence and a correlation between
contextual-equivalence and notions of bisimulation. The notion of
context allows the decomposition of a process into (sub-)process and
its syntactic environment, its context. Thus, a context may be
thought of as a process with a ``hole'' (written $\Box$) in it. The
application of a context $M$ to a process $P$, written $M[P]$, is
tantamount to filling the hole in $M$ with $P$. In this paper we do
not need the full weight of this theory, but do make use of the notion
of context in the proof the main theorem. 

\begin{mathpar}
  \inferrule* [lab=summation] {} {{M_{M},M_{N}} \bc \Box \;|\; x.M_{A} \;|\; M_{M}+M_{N}}
  \and
  \inferrule* [lab=agent] {} {{M_{A}} \bc (\vec{x})M_{P} \;| \; \clift{P_0,\ldots,M_{P},\ldots,P_N}}
  \and \\
  \inferrule* [lab=process] {} {{M_{P}} \bc M_{N} \;| \;P|M_{P} }
\end{mathpar} 

\begin{mathpar}
  \inferrule* [lab=sychronization] {} {M_{N} \bc \Box \;|\; x?M_{F} \;|\; x!M_{C}}
  \and
  \inferrule* [lab=abstraction] {} {{M_{F}} \bc (x)M_{P} }
  \and
  \inferrule* [lab=concretion] {} {{M_{C}} \bc \langle M_{P} \rangle }
  \and \\
  \inferrule* [lab=process] {} {{M_{P}} \bc M_{N} \;| \;P|M_{P} }
\end{mathpar}

\begin{definition}[contextual application] Given a context $M$, and
  process $P$, we define the \emph{contextual application}, $M[P] :=
  M\{P/\Box\}$. That is, the contextual application of M to P is the
  substitution of $P$ for $\Box$ in $M$.
\end{definition}

$\meaningof{-} : L \to \mathcal{P}(\pi)$

\begin{mathpar}
  \inferrule* [lab=collection] {} {\meaningof{true} = \pi, \and \meaningof{~E} = \pi \setminus \meaningof{E}, \and \meaningof{E_{1} \& E_{2}} = \meaningof{E_{1}} \cap \meaningof{E_{2}}}
\end{mathpar}

\begin{mathpar}
  \inferrule* [lab=structure] {} {\meaningof{0} = \{ P \in \pi | P \equiv 0 \}, \and \\ \meaningof{E_1 | E_2} = \{ P \in \pi | P \equiv P_{1} | P_{2}, P_{1} \in \meaningof{E_{1}}, P_{2} \in \meaningof{E_2}\} }
\end{mathpar}

\begin{mathpar}
 \inferrule* [lab=behavior] {} {\meaningof{\langle a?b \rangle E} = \{ P \in \pi | P \equiv Q | u?(y)P', \\ \and \\\\ \and \\ \;\;\; u \in \meaningof{a}, \forall z.P'\{z/y\} \in \meaningof{E\{z/b\}}\}, \and \\ \meaningof{a!E} = \{ P \in \pi | P \equiv Q | x!\langle P' \rangle, x \in \meaningof{a} P' \in \meaningof{E}\} }
\end{mathpar}

\begin{mathpar}
 \inferrule* [lab=nominal] {} {\meaningof{\quotep{E}} = \{ \quotep{P} \in \quotep{\pi} | P \in \meaningof{E} \}, \and \meaningof{\quotep{P}} = \{ \quotep{Q} \in \quotep{\pi} | P \equiv Q \} \and \\ \meaningof{@\quotep{E}} = \{ P \in \pi | P \equiv @x, x \in \meaningof{E} \}}
\end{mathpar}

\begin{eqnarray*}
  \\
  \meaningof{-} : TS \to ST
\end{eqnarray*}

\begin{eqnarray*}
  \\
  L : TS \to ST
\end{eqnarray*}

\begin{eqnarray*}
  \\
  P \models E \iff P \in \meaningof{E}
\end{eqnarray*}

\begin{eqnarray*}
  P \approx_{L} Q \iff \forall E \in L. P \models E \iff Q \models E
\end{eqnarray*}

\begin{eqnarray*}
  P \approx_{K} Q
\end{eqnarray*}

\begin{eqnarray*}
  P \approx Q
\end{eqnarray*}

$\approx_{K} = \approx = \approx_{L}$

\subsubsection{Contextual duality}

Note that contexts extend the quotation operation to a family of
operations from processes to names. Given a context, $M$, we can
define a \emph{nominal context}, $\quotep{M}$ by $\quotep{M}[P] :=
\quotep{M[P]}$. To foreshadow what is to come we observe that these
operations enjoy a duality with processes very much like the duality
between vectors and maps from vectors to scalars.

Further, because the calculus is essentially higher-order, we have a
correspondence between contexts and processes. More specifically,
given a name $x$ and a context $M$ we can construct $M^{*}_{x}$ such
that 

\begin{mathpar}
  M^{*}_{x} | \lift{x}{P} \red M[P]
\end{mathpar}

namely,

\begin{mathpar}
  M^{*}_{x} := x?(u).M[\dropn{u}]
\end{mathpar}

The dependence of $M^{*}_{x}$ on a name makes it an abstraction, 

\begin{mathpar}
  M^{*} := (x)x?(u).M[\dropn{u}]
\end{mathpar}

\subsection{Additional notation}

It will sometimes be convenient to denote the process a name
quotes. We already have the notation $x = \quotep{P}$, but it will be
convenient to introduce an alternate notation, $\procn{x}$, when we
want to emphasize the connection to the use of the name. Note that, by
virtue of name equivalence, $\quotep{\procn{x}} \nameeq x$; so, the
notation is consistent with previous definitions.

Further, because names have structure it is possible to effect
substitutions on the basis of that structure. This means we need to
upgrade our notation for substitutions, which we accomplish by
adapting comprehension notation. Thus,

\begin{mathpar}
  P\{ y / x : x \in S \}
\end{mathpar}

is interpreted to mean the process derived from P by replacing (in a
capture-avoiding manner) each occurrence of $x$ in $S$ by $y$. For example,

\begin{mathpar}
  P\{ \quotep{\procn{x}|\procn{x}} / x : x \in \freenames{P} \}
\end{mathpar}

will replace each (occurrence) of a free name $x$ in $P$ by
$\quotep{\procn{x}|\procn{x}}$.

Also, we will avail ourselves of the notation $x^{L}$ and $x^{R}$ to
denote injections of a name into disjoint copies of the name
space. There are numerous ways to accomplish this. One example can be
found in \cite{MeredithR05}. This notation overloads to vectors of
names: $\vec{x}^{\pi} := (x_{i}^{\pi} \; : \; 0 \leq i < |\vec{x}| )$ where $\pi \in \{L,R\}$.

We also use $P^{\Box} := P|\Box$.

In \cite{MeredithR05} an interpretation of the new operator is
given. It turns out that there are several possible interpretations
all enjoying the requisite algebraic properties of the operator (see
\cite{milner91polyadicpi}). We will therefore make liberal use of
$(\nu\; \vec{x})P$.

% subsection the_syntax_and_semantics_of_the_notation_system (end)   

\input{qm2pi.qmops} 

\input{qm2pi.sterngerlach} 

\input{qm2pi.metric} 

% section concurrent_process_calculi (end)

%\input{qm2pi.proofsketch}

% section proof sketch (end)

%\input{qm2pi.slviaknots} 

% section spatial logic via knots (end)

\input{qm2pi.conclusion}

% section conclusion (end)

%\input{qm2pi.dtcodes} 

% section wiring algorithm (end)

\input{qm2pi.ack} 

% section acknowledgments (end)

\newpage


\bibliographystyle{plain}   
\bibliography{../../biblios/main.bib}

\input{qm2pi.rhodetails}

\end{document}



% section front matter (end)

\section{Introduction}\label{sec:introduction} % (fold)
In this draft of the material i am going to have to dispense with the
usual writing conventions adopted in papers on these topics. i'm going
to have adopt whatever tone i need at the time i'm writing up the
calculations. Sometimes this may be very conversational; others it may
be the barest mathematical grunts; others still it may be that i have
lifted text from one of my other papers because the exposition of some
point was better said there. i hope that my readers are not unduly put
out by this decision. i'm not doing this to flout convention or be
rebellious. i find these calculations very technically challenging. To
keep everything going technically, something has to give; i have to
let go of some cognitive burden. So, the academic writing style --
with all of its trade-offs in terms of facilitating technical
communication -- is what i'm letting go of. Perhaps subsequent drafts
can be tightened and polished, but for now, i'm going to speak as if
we were sitting together in a coffee shop with a laptop, wifi and a
pad of paper and a pencil.

So, here's what i have to say. We -- you and i, comfortably ensconced
in our coffee shop and well-equipped with our tools -- can realize and
carry out the calculations of quantum mechanics over a very different
formal theory of dynamics, a formal theory of dynamics that
corresponds to a theory of concurrent computation with
\emph{reflection}. It has the advantage that the underlying theory is
already `quantized', but supports analogues all of the continuuous
operations. Strikingly, this underlying theory has recently been
connected with a notion of metric that we can show, by calculating
together, coincides with the metric induced by the inner product.

There are a lot of reasons why you might be interested in seeing
calculations of this form. Here's why i'm interested. For the past
several centuries there has been no competitor to the ``Newtonian''
account of dynamics. As a result the predominant share of accounts of
dynamical systems and situations have had to be formulated in terms of
the Newtonian machinery. i view this as an intellectually dangerous
position to occupy. Everything, despite it's intrinsic shape, turns
into a nail to be hit with this hammer. Recently, however, the theory
of computation has matured to the point where we have candidates for
theories of dynamics that offer very different perspective on
reasoning about dynamical systems and situations. Testing these
candidates against very successful accounts of dynamical situations,
like quantum mechanics, is going to give us some sense of how mature
they are and some measure of the quality of these accounts of
dynamics.

\subsection{Summary of contributions and outline of paper}

So, we're going to develop an interpretation of the operations of
quantum mechanics normally interpreted by Hilbert spaces and
operators. We're going to do this over a theory of computation. Note
that this is very different than the usual quantum computation program
which develops notions of computation over quantum mechanics. Rather,
we are developing a story that aligns with Wheeler's slogan: It from
Bit. To do this we will first provide an account of the theory of
computation at play here. Then we will dive into a calculation-driven
interpretation of the operations of quantum mechanics.

The reason we take this approach is that -- until very recently --
there hasn't been an axiomatic account of quantum mechanics. As a
result there has been no sharp delineation of the mathematical theory
supporting interpretation of the physical theory and the physical
theory, itself. So, ambient features of the maths are free to be
exploited (or supressed) without a real accounting of their physical
relevance. There is no sharp statement ``here's the physical theory''
qua \emph{theory} and ``here's the mathematical interpretation''
enabling a judgment of how faithful the interpretation is -- apart
from experimental observation. When there is an axiomatic account we
can judge how well a given mathematical formalism supports an
interpretation of the axioms, independent of
experimentation. Likewise, we can judge how well we have captured our
physical evidence and experience with our axiomatics, independent of
any specific mathematical implementation, with accidental detail that
may or may not have physical significance. 

In lieu of a fully fleshed out and vetted axiomatic account of quantum
mechanics, interpreting the operational notions in service of modeling
physical systems will have to suffice. In other words, we are not in
the business of providing a model of Hilbert spaces and operators. We
are in the business of providing a model of quantum mechanics because
we are motivated by testing our notions of dynamics against physical
theory; and, the predictive calculations of the physical theory must
serve as the best formulation -- shy of a fully fleshed out axiomatic
account -- of the physical theory itself (as they have for scientific
theories since time immemorial). Put another way, despite a
whole-hearted commitment to an It-from-Bit ontology, we are firmly
aligned with the shut-up-and-calculate camp as the best way to obtain
results either from the physical perspective or as a quality assurance
measure of our fledgling theory of dynamics.

In detail, we present a reflective process calculus. Then we develop
intuitive correspondences between the notions available in this
calculus and the usual physical notions supporting quantum mechanical
calculations. Thus, 

\begin{table}[htp]
  \center{
    \fbox{
      \begin{tabular}{c|c}
        quantum mechanics & process calculus \\
        \hline
        scalar & name \\
        state vector & process \\
        dual & contextual duals \\
        matrix & formal sums of process-context-dual pairs \\
        orthogonality & process annihilation \\
        inner product & execution-formula + quoting
      \end{tabular}
    }
  }
  \caption{QM - process calculi correspondences}
\end{table}

Then we tighten up these intuitions to operational definitions. We
employ the Dirac notation as the best proxy we can find for an
abstract syntax of the quantum mechanical notions. The definitions we
develop put us in contact with equational constraints coming from the
theory that we demonstrate the definitions and calculations satisfy.

This puts us in a position to shut up and calculate for the
Stern-Gerlach experimental set up, showing how these predictive
calculations become calculations on processes in our theory of a
reflective process calculus.

Penultimately, we demonstrate that the notion of metric coming from
the inner product coincides with the notion of metric available from
the theory of bisimulation. This demonstration gives us the right to
think of space as arising from behavior. Finally, we consider where we
might go from the new vantage point we have obtained.

% section introduction (end) 
 
% section introduction (end)

% \documentclass[12pt]{llncs}
%\documentclass{jktr}

\usepackage[pdftex]{hyperref}                   
\usepackage {listings}
\usepackage {mathpartir}
\usepackage{bcprules}
%\usepackage{listings}
                       
\usepackage{graphicx} 
%\usepackage[margins=2.5cm,nohead,nofoot]{geometry}
%\usepackage{geometry}
\usepackage{amsfonts}
\usepackage{amstext}
\usepackage{latexsym}
\usepackage{amssymb}
\usepackage{color}


%\include{myPreamble}
\include{qm2pi.local} 

%\ifpdf
%\usepackage[pdftex]{graphicx}
%\else
%\usepackage{graphicx}
%\fi

 % \ifpdf
%  \usepackage{pdfsync}
%  \if


%\title{Brief Article}
%\author{David F. Snyder}
%\author{L.G. Meredith}

%\address{Dept. of Math., Texas State University--San Marcos, San Marcos, TX 78666}
       
\pagestyle{empty}


\begin{document}

\lstset{language=[Objective]Caml,frame=shadowbox}

\input{qm2pi.front}

% section front matter (end)

\input{qm2pi.intro} 
 
% section introduction (end)

% \input{qm2pi.knotations} 

% section notation (end)

\input{qm2pi.process.calculi} 

% section concurrent_process_calculi_and_spatial_logics_ (end)
    
%\input{qm2pi.knots2pi} 

%\input{qm2pi.trefoil} 

%\input{qm2pi.mainthm} 

% subsection basic_interpretation (end)

%\input{qm2pi.rho.presentation} 
\subsection{The syntax and semantics of the notation system}\label{sub:the_syntax_and_semantics_of_the_notation_system} % (fold)

We now summarize a technical presentation of the calculus that
embodies our theory of dynamics. The typical presentation of such a
calculus follows the style of giving generators and relations on
them. The grammar, below, describing term constructors, freely
generates the set of processes, $\Proc$. This set is then quotiented
by a relation known as structural congruence and it is over this set
that the notion of dynamics is expressed. This presentation is
essentially that of \cite{MeredithR05} with the addition of
polyadicity and summation. For readability we have relegated some of
the technical subtleties to an appendix.

\subsubsection{Process grammar}\label{subsub:process_grammar}

\begin{mathpar}
  \inferrule* [lab=synchronization] {} {{M} \bc \pzero \;|\; x?F \;|\; x!C }
  \and
  \inferrule* [lab=abstraction] {} {{F} \bc (x)P}
  \and
  \inferrule* [lab=concretion] {} {{C} \bc \langle Q \rangle}
  \and
  \inferrule* [lab=process] {} {{P,Q} \bc M \;| \;P|Q \;|\; @{x}}
  \and
  \inferrule* [lab=name] {} {{x} \bc \quotep{P}}
\end{mathpar} 

Note that $\vec{x}$ (resp. $\vec{P}$) denotes a vector of names
(resp. processes) of length $|\vec{x}|$ (resp. $|\vec{P}|$). We adopt
the following useful abbreviations.

\begin{mathpar}
   x?(\vec{y}).P := x.(\vec{y})P \and  x\clift{\vec{P}} := x.\clift{\vec{P}}
   \and x!(y) := \lift{x}{\dropn{y}}
   \and \Pi_{i=0}^{n-1}P_i := P_0 | \ldots | P_{n-1}
\end{mathpar}

\subsubsection{Structural congruence}

\paragraph{Free and bound names and alpha-equivalence.} At the
core of structural equivalence is alpha-equivalence which identifies
process that are the same up to a change of variable. Formally, we
recognize the distinction between free and bound names. The free names
of a process, $\freenames{P}$, may be calculated recursively as
follows:

\begin{mathpar}
\freenames{\pzero} := \emptyset
  \and \\
  \freenames{x?(y).P} := \{ x \} \cup (\freenames{P} \setminus \{ y \})
  \and 
  \freenames{x!\langle P \rangle} := \{ x \} \cup \{ P \} 
  \and \\
  \freenames{P|Q} := \freenames{P} \cup \freenames{Q}
  \and \\
  \freenames{@{x}} := \{ x \}
\end{mathpar}

$\pi$
$\quotep{\pi}$

$\freenames{-} : \pi \to \mathcal{P}(\quotep{\pi})$

\begin{eqnarray*}
  \freenames{\pzero} & := & \emptyset \\
  \freenames{x?(y).P} & := & \{ x \} \cup (\freenames{P} \setminus \{ y \}) \\
  \freenames{x!\langle P \rangle} & := & \{ x \} \cup \{ P \} \\
  \freenames{P|Q} & := & \freenames{P} \cup \freenames{Q} \\
  \freenames{\dropn{x}} & := & \{ x \}
\end{eqnarray*}

The bound names of a process, $\boundnames{P}$, are those names occurring in $P$
that are not free. For example, in $x?(y).0$, the name $x$ is free, while $y$ is bound.

\begin{mathpar}
  \inferrule* [lab=monoidal-laws] {} { P|Q \equiv Q|P \and P|0 \equiv P \and P|(Q|R) \equiv (P|Q)|R }
\end{mathpar}

\begin{mathpar}
  \inferrule* [lab=alpha-equivalence] {} { (x)P \equiv (y)P\{y/x\} \and y \not\in \freenames{P} }
\end{mathpar}

\begin{definition}
Then two processes, $P,Q$, are alpha-equivalent if $P = Q\{\vec{y}/\vec{x}\}$ for
some $\vec{x} \in \boundnames{Q},\vec{y} \in \boundnames{P}$, where $Q\{\vec{y}/\vec{x}\}$
denotes the capture-avoiding substitution of $\vec{y}$ for $\vec{x}$ in $Q$.
\end{definition}

\begin{definition}
  The {\em structural congruence} \cite{SangiorgiWalker} , $\equiv$,
  between processes is the least congruence containing
  alpha-equivalence, satisfying the abelian monoid laws
  (associativity, commutativity and $\pzero$ as identity) for parallel
  composition $|$ and for summation $+$.
\end{definition}

\subsection{Name equivalence}

We take name equivalence, written $\nameeq$, to be the smallest
equivalence relation generated by the following rules.

\begin{mathpar}
\inferrule*[lab=Quote-drop]
{ }
{ \quotep{@{x}} \nameeq x }

\inferrule*[lab=Struct-equiv]
{ P \scong Q }
{ \quotep{P} \nameeq \quotep{Q} }
\end{mathpar}

The astute reader will have noticed that the mutual recursion of names
and processes imposes a mutual recursion on alpha-equivalence and
structural equivalence via name-equivalence. Fortunately, all of this
works out pleasantly and we may calculate in the natural way, free of
concern. The reader interested in the details is referred to the
appendix \ref{appendix:rho_details}.

\subsection{Substitution}

We use $\Proc$ for the set of processes, $\QProc$ for the set of
names, and $\id{\{}\vec{y} / \vec{x} \id{\}}$ to denote partial maps,
$s : \QProc \rightarrow \QProc$. A map, $s$ lifts, uniquely, to a map
on process terms, $\widehat{s} : \Proc \rightarrow \Proc$ by the
following equations.

\begin{mathpar}
  (0) \psubstp{Q}{P} := 0 \\
  (R \juxtap S) \psubstp{Q}{P}
  :=    
  (R)\psubstp{Q}{P} \juxtap (S) \psubstp{Q}{P} \\
  (x?(y).R) \psubstp{Q}{P}    
  :=    
  (x)\substp{Q}{P} (z)\concat( (R \psubstn{z}{y}) \psubstp{Q}{P} ) \\
  (\lift{x}{R}) \psubstp{Q}{P}  
  :=
  \lift{(x)\substp{Q}{P}}{ R \psubstp{Q}{P} } \\
%   (\dropn{x})  \psubstp{Q}{P}       
%   := 
%   \left\{ 
%     \begin{array}{ccc} 
%       \dropn{\quotep{Q}} & & x \nameeq \quotep{P} \\
%       \dropn{x} & & otherwise \\
%     \end{array}
%   \right. 
  (\dropn{x})  \psubstp{Q}{P}       
  := 
  \left\{ 
    \begin{array}{ccc} 
      Q & & x \nameeq \quotep{P} \\
      \dropn{x} & & otherwise \\
    \end{array}
  \right.
\end{mathpar}
 

where

\begin{eqnarray}
  (x)\id{\{} \lpquote Q \rpquote / \lpquote P \rpquote \id{\}}            = 
  \left\{ 
    \begin{array}{ccc}
      \lpquote Q \rpquote & & x \nameeq \lpquote P \rpquote \\
      x & & otherwise \\
    \end{array}
  \right. \nonumber
\end{eqnarray}

and $z$ is chosen distinct from $\quotep{P}$, $\quotep{Q}$, the free
names in $Q$, and all the names in $R$. Our $\alpha$-equivalence will
be built in the standard way from this substitution.

\begin{remark}\label{rem:no_self_referential_names}
  One consequence of these definitions is that $\forall P. \quotep{P}
  \not\in \freenames{P}$.
\end{remark}

\subsection{ Dynamic quote: an example }

Anticipating something of what's to come, consider applying the
substitution, $\widehat{\id{\{}u / z \id{\}}}$, to the following pair
of processes, $\lift{w}{y!(z)}$ and $w[ \lpquote y!(z) \rpquote ]$.

\begin{eqnarray}
	\lift{w}{y!(z)}\widehat{\id{\{}u / z \id{\}}}
		& = &
		\lift{w}{y!(u)} \nonumber\\
	w[ \lpquote y!(z) \rpquote ] \widehat{ \id{\{}u / z \id{\}} }
		& = &
		w[ \lpquote y!(z) \rpquote ] \nonumber
\end{eqnarray}

Because the body of the process between quotes is impervious to
substitution, we get radically different answers. In fact, by
examining the first process in an input context,
e.g. $x?(z).\lift{w}{y!(z)}$, we see that the process under the lift
operator may be shaped by prefixed inputs binding a name inside it. In
this sense, the lift operator will be seen as a way to dynamically
construct processes before reifying them as names.

Finally equipped with these standard features we can present the
dynamics of the calculus.

\subsubsection{Operational semantics} 

Finally, we introduce the computational dynamics. What marks these
algebras as distinct from other more traditionally studied algebraic
structures, e.g. vector spaces or polynomial rings, is the manner in
which dynamics is captured. In traditional structures, dynamics is typically
expressed through morphisms between such structures, as in linear maps
between vector spaces or morphisms between rings. In algebras
associated with the semantics of computation, the dynamics is
expressed as part of the algebraic structure itself, through a
reduction reduction relation typically denoted by $\red$. Below, we
give a recursive presentation of this relation for the calculus used
in the encoding.

$\red \subseteq \pi \times \pi$
$\red : \pi \to \mathcal{P}(\pi)$

\begin{mathpar}
  \inferrule* [lab=Comm] { \textsf{match}( x_{src}, x_{trgt} ) } { x_{trgt}?(y)P \; | \; x_{src}!\langle {Q} \rangle \red P\{\quotep{Q}/y}\} }
  \and \\
  \inferrule* [lab=Par] {{P} \red {P}'} {{{P} | {Q}} \red {{P}' | {Q}}}
  \and
  \inferrule* [lab=Equiv]{{{P} \scong {P}'} \andalso {{P}' \red {Q}'} \andalso {{Q}' \scong {Q}}}{{P} \red {Q}}
\end{mathpar}

\begin{eqnarray*}
  match_{\equiv} (\quotep{P},\quotep{Q}) & := & P \equiv Q \\
  match_{\dagger}(\quotep{P},\quotep{Q}) & := & \forall R. P|Q \red^{*} R => R \red^{*} 0 \\
  match_{K}(\quotep{P},\quotep{Q}) & := & K \mbox{ for some context } K
\end{eqnarray*}

$u?(x)P | u!\langle Q \rangle \red P\{\quotep{Q}/x\}$

%We write $\wred$ for $\red^*$, and $P\red$ if $\exists Q $ such that $ P \red Q$.
We write $P\red$ if $\exists Q $ such that $ P \red Q$ and $P\not\red$, otherwise.

\section{Replication}

As mentioned before, it is known that replication (and hence
recursion) can be implemented in a higher-order process algebra
\cite{SangiorgiWalker}. As our first example of calculation with the
machinery thus far presented we give the construction explicitly in
the {\rhoc}.

\begin{eqnarray}
	D_{x} & := & \prefix{x}{y}{(\binpar{\outputp{x}{y}}{@{y}})} \nonumber\\
	\bangp_{x}{P} & := & \binpar{{x}!\langle{\binpar{D_{x}}{P}}\rangle}{D_{x}} \nonumber
\end{eqnarray}

\begin{eqnarray}
	\bangp_{x}{P} & & \nonumber\\
	=
	& {x}!\langle{(\prefix{x}{y}{(\outputp{x}{y} | @{y})) | P}}\rangle 
	      | \prefix{x}{y}{(\outputp{x}{y} | @{y})} & \nonumber\\
	\red
	& (\outputp{x}{y} | @{y})\substn{\quotep{(\prefix{x}{y}{(@{y} | \outputp{x}{y})) | P}}}{y} & \nonumber\\
	=
	& \outputp{x}{\quotep{(\prefix{x}{y}{(\outputp{x}{y} | @{y})) | P}}}
	  | {(\prefix{x}{y}{(\outputp{x}{y} | @{y})) | P}} & \nonumber\\
	\red
	& \ldots & \nonumber\\
	\red^*
	& P | P | \ldots & \nonumber
\end{eqnarray}

Of course, this encoding, as an implementation, runs away, unfolding
$\bangp{P}$ eagerly. A lazier and more implementable replication
operator, restricted to input-guarded processes, may be obtained as follows.

\begin{eqnarray}
\bangp{\prefix{u}{v}{P}} 
	:= 
	\binpar{\lift{x}{\prefix{u}{v}{(\binpar{D(x)}{P})}}}{D(x)} \nonumber
\end{eqnarray}

\begin{remark}
  Note that the lazier definition still does not deal with summation
  or mixed summation (i.e. sums over input and output). The reader is
  invited to construct definitions of replication that deal with these
  features. 

  Further, the definitions are parameterized in a name, $x$. Can you,
  gentle reader, make a definition that eliminates this parameter and
  guarantees no accidental interaction between the replication
  machinery and the process being replicated -- i.e. no accidental
  sharing of names used by the process to get its work done and the
  name(s) used by the replication to effect copying. This latter
  revision of the definition of replication is crucial to obtaining
  the expected identity $!!P \sim !P$.
\end{remark}

\begin{remark}\label{rem:paradoxical_combinator}
  The reader familiar with the lambda calculus will have noticed the
  similarity between $D$ and the paradoxical combinator.

  [Ed. note: the existence of this seems to suggest we have to be more
  restrictive on the set of processes and names we admit if we are to
  support no-cloning.]
\end{remark}

\subsubsection{Bisimulation}

The computational dynamics gives rise to another kind of equivalence,
the equivalence of computational behavior. As previously mentioned
this is typically captured \emph{via} some form of bisimulation.

% The notion we use in this paper is weak barbed bisimulation
% \cite{milner91polyadicpi}.

The notion we use in this paper is derived from weak barbed
bisimulation \cite{milner91polyadicpi}. 

\begin{definition}
An \emph{observation relation}, $\downarrow_{\mathcal N}$, over a set
of names, $\mathcal N$, is the smallest relation satisfying the rules
below.

\infrule[Out-barb]{y \in {\mathcal N}, \; x \nameeq y}
		  {\outputp{x}{v} \downarrow_{\mathcal N} x}
\infrule[Par-barb]{\mbox{$P\downarrow_{\mathcal N} x$ or $Q\downarrow_{\mathcal N} x$}}
		  {\binpar{P}{Q} \downarrow_{\mathcal N} x}

We write $P \Downarrow_{\mathcal N} x$ if there is $Q$ such that 
$P \wred Q$ and $Q \downarrow_{\mathcal N} x$.
\end{definition}

\begin{definition}
%\label{def.bbisim}
An  ${\mathcal N}$-\emph{barbed bisimulation} over a set of names, ${\mathcal N}$, is a symmetric binary relation 
${\mathcal S}_{\mathcal N}$ between agents such that $P\rel{S}_{\mathcal N}Q$ implies:
\begin{enumerate}
\item If $P \red P'$ then $Q \wred Q'$ and $P'\rel{S}_{\mathcal N} Q'$.
\item If $P\downarrow_{\mathcal N} x$, then $Q\Downarrow_{\mathcal N} x$.
\end{enumerate}
$P$ is ${\mathcal N}$-barbed bisimilar to $Q$, written
$P \wbbisim_{\mathcal N} Q$, if $P \rel{S}_{\mathcal N} Q$ for some ${\mathcal N}$-barbed bisimulation ${\mathcal S}_{\mathcal N}$.
\end{definition}

$\mathcal{R} \subseteq \pi \times \pi$

$P \mathcal{R} Q => \forall P'. P \red P' \Rightarrow \exists Q'. Q \red Q', P' \mathcal{R} Q'$

$P \vdash x \Rightarrow Q \vdash x$

\begin{mathpar}
  \inferrule*[lab=Out-barb]{x \nameeq y}{{y}!\langle{Q}\rangle \vdash x}
  \and
  \inferrule*[lab=Par-barb]{\mbox{$P\vdash x$ or $Q\vdash x$}}{\binpar{P}{Q} \vdash x}
\end{mathpar}

\subsubsection{Contexts}

One of the principle advantages of computational calculi like the
$\pi$-calculus is a well-defined notion of context,
contextual-equivalence and a correlation between
contextual-equivalence and notions of bisimulation. The notion of
context allows the decomposition of a process into (sub-)process and
its syntactic environment, its context. Thus, a context may be
thought of as a process with a ``hole'' (written $\Box$) in it. The
application of a context $M$ to a process $P$, written $M[P]$, is
tantamount to filling the hole in $M$ with $P$. In this paper we do
not need the full weight of this theory, but do make use of the notion
of context in the proof the main theorem. 

\begin{mathpar}
  \inferrule* [lab=summation] {} {{M_{M},M_{N}} \bc \Box \;|\; x.M_{A} \;|\; M_{M}+M_{N}}
  \and
  \inferrule* [lab=agent] {} {{M_{A}} \bc (\vec{x})M_{P} \;| \; \clift{P_0,\ldots,M_{P},\ldots,P_N}}
  \and \\
  \inferrule* [lab=process] {} {{M_{P}} \bc M_{N} \;| \;P|M_{P} }
\end{mathpar} 

\begin{mathpar}
  \inferrule* [lab=sychronization] {} {M_{N} \bc \Box \;|\; x?M_{F} \;|\; x!M_{C}}
  \and
  \inferrule* [lab=abstraction] {} {{M_{F}} \bc (x)M_{P} }
  \and
  \inferrule* [lab=concretion] {} {{M_{C}} \bc \langle M_{P} \rangle }
  \and \\
  \inferrule* [lab=process] {} {{M_{P}} \bc M_{N} \;| \;P|M_{P} }
\end{mathpar}

\begin{definition}[contextual application] Given a context $M$, and
  process $P$, we define the \emph{contextual application}, $M[P] :=
  M\{P/\Box\}$. That is, the contextual application of M to P is the
  substitution of $P$ for $\Box$ in $M$.
\end{definition}

$\meaningof{-} : L \to \mathcal{P}(\pi)$

\begin{mathpar}
  \inferrule* [lab=collection] {} {\meaningof{true} = \pi, \and \meaningof{~E} = \pi \setminus \meaningof{E}, \and \meaningof{E_{1} \& E_{2}} = \meaningof{E_{1}} \cap \meaningof{E_{2}}}
\end{mathpar}

\begin{mathpar}
  \inferrule* [lab=structure] {} {\meaningof{0} = \{ P \in \pi | P \equiv 0 \}, \and \\ \meaningof{E_1 | E_2} = \{ P \in \pi | P \equiv P_{1} | P_{2}, P_{1} \in \meaningof{E_{1}}, P_{2} \in \meaningof{E_2}\} }
\end{mathpar}

\begin{mathpar}
 \inferrule* [lab=behavior] {} {\meaningof{\langle a?b \rangle E} = \{ P \in \pi | P \equiv Q | u?(y)P', \\ \and \\\\ \and \\ \;\;\; u \in \meaningof{a}, \forall z.P'\{z/y\} \in \meaningof{E\{z/b\}}\}, \and \\ \meaningof{a!E} = \{ P \in \pi | P \equiv Q | x!\langle P' \rangle, x \in \meaningof{a} P' \in \meaningof{E}\} }
\end{mathpar}

\begin{mathpar}
 \inferrule* [lab=nominal] {} {\meaningof{\quotep{E}} = \{ \quotep{P} \in \quotep{\pi} | P \in \meaningof{E} \}, \and \meaningof{\quotep{P}} = \{ \quotep{Q} \in \quotep{\pi} | P \equiv Q \} \and \\ \meaningof{@\quotep{E}} = \{ P \in \pi | P \equiv @x, x \in \meaningof{E} \}}
\end{mathpar}

\begin{eqnarray*}
  \\
  \meaningof{-} : TS \to ST
\end{eqnarray*}

\begin{eqnarray*}
  \\
  L : TS \to ST
\end{eqnarray*}

\begin{eqnarray*}
  \\
  P \models E \iff P \in \meaningof{E}
\end{eqnarray*}

\begin{eqnarray*}
  P \approx_{L} Q \iff \forall E \in L. P \models E \iff Q \models E
\end{eqnarray*}

\begin{eqnarray*}
  P \approx_{K} Q
\end{eqnarray*}

\begin{eqnarray*}
  P \approx Q
\end{eqnarray*}

$\approx_{K} = \approx = \approx_{L}$

\subsubsection{Contextual duality}

Note that contexts extend the quotation operation to a family of
operations from processes to names. Given a context, $M$, we can
define a \emph{nominal context}, $\quotep{M}$ by $\quotep{M}[P] :=
\quotep{M[P]}$. To foreshadow what is to come we observe that these
operations enjoy a duality with processes very much like the duality
between vectors and maps from vectors to scalars.

Further, because the calculus is essentially higher-order, we have a
correspondence between contexts and processes. More specifically,
given a name $x$ and a context $M$ we can construct $M^{*}_{x}$ such
that 

\begin{mathpar}
  M^{*}_{x} | \lift{x}{P} \red M[P]
\end{mathpar}

namely,

\begin{mathpar}
  M^{*}_{x} := x?(u).M[\dropn{u}]
\end{mathpar}

The dependence of $M^{*}_{x}$ on a name makes it an abstraction, 

\begin{mathpar}
  M^{*} := (x)x?(u).M[\dropn{u}]
\end{mathpar}

\subsection{Additional notation}

It will sometimes be convenient to denote the process a name
quotes. We already have the notation $x = \quotep{P}$, but it will be
convenient to introduce an alternate notation, $\procn{x}$, when we
want to emphasize the connection to the use of the name. Note that, by
virtue of name equivalence, $\quotep{\procn{x}} \nameeq x$; so, the
notation is consistent with previous definitions.

Further, because names have structure it is possible to effect
substitutions on the basis of that structure. This means we need to
upgrade our notation for substitutions, which we accomplish by
adapting comprehension notation. Thus,

\begin{mathpar}
  P\{ y / x : x \in S \}
\end{mathpar}

is interpreted to mean the process derived from P by replacing (in a
capture-avoiding manner) each occurrence of $x$ in $S$ by $y$. For example,

\begin{mathpar}
  P\{ \quotep{\procn{x}|\procn{x}} / x : x \in \freenames{P} \}
\end{mathpar}

will replace each (occurrence) of a free name $x$ in $P$ by
$\quotep{\procn{x}|\procn{x}}$.

Also, we will avail ourselves of the notation $x^{L}$ and $x^{R}$ to
denote injections of a name into disjoint copies of the name
space. There are numerous ways to accomplish this. One example can be
found in \cite{MeredithR05}. This notation overloads to vectors of
names: $\vec{x}^{\pi} := (x_{i}^{\pi} \; : \; 0 \leq i < |\vec{x}| )$ where $\pi \in \{L,R\}$.

We also use $P^{\Box} := P|\Box$.

In \cite{MeredithR05} an interpretation of the new operator is
given. It turns out that there are several possible interpretations
all enjoying the requisite algebraic properties of the operator (see
\cite{milner91polyadicpi}). We will therefore make liberal use of
$(\nu\; \vec{x})P$.

% subsection the_syntax_and_semantics_of_the_notation_system (end)   

\input{qm2pi.qmops} 

\input{qm2pi.sterngerlach} 

\input{qm2pi.metric} 

% section concurrent_process_calculi (end)

%\input{qm2pi.proofsketch}

% section proof sketch (end)

%\input{qm2pi.slviaknots} 

% section spatial logic via knots (end)

\input{qm2pi.conclusion}

% section conclusion (end)

%\input{qm2pi.dtcodes} 

% section wiring algorithm (end)

\input{qm2pi.ack} 

% section acknowledgments (end)

\newpage


\bibliographystyle{plain}   
\bibliography{../../biblios/main.bib}

\input{qm2pi.rhodetails}

\end{document}

 

% section notation (end)

\input{qm2pi.process.calculi} 

% section concurrent_process_calculi_and_spatial_logics_ (end)
    
%\documentclass[12pt]{llncs}
%\documentclass{jktr}

\usepackage[pdftex]{hyperref}                   
\usepackage {listings}
\usepackage {mathpartir}
\usepackage{bcprules}
%\usepackage{listings}
                       
\usepackage{graphicx} 
%\usepackage[margins=2.5cm,nohead,nofoot]{geometry}
%\usepackage{geometry}
\usepackage{amsfonts}
\usepackage{amstext}
\usepackage{latexsym}
\usepackage{amssymb}
\usepackage{color}


%\include{myPreamble}
\include{qm2pi.local} 

%\ifpdf
%\usepackage[pdftex]{graphicx}
%\else
%\usepackage{graphicx}
%\fi

 % \ifpdf
%  \usepackage{pdfsync}
%  \if


%\title{Brief Article}
%\author{David F. Snyder}
%\author{L.G. Meredith}

%\address{Dept. of Math., Texas State University--San Marcos, San Marcos, TX 78666}
       
\pagestyle{empty}


\begin{document}

\lstset{language=[Objective]Caml,frame=shadowbox}

\input{qm2pi.front}

% section front matter (end)

\input{qm2pi.intro} 
 
% section introduction (end)

% \input{qm2pi.knotations} 

% section notation (end)

\input{qm2pi.process.calculi} 

% section concurrent_process_calculi_and_spatial_logics_ (end)
    
%\input{qm2pi.knots2pi} 

%\input{qm2pi.trefoil} 

%\input{qm2pi.mainthm} 

% subsection basic_interpretation (end)

%\input{qm2pi.rho.presentation} 
\subsection{The syntax and semantics of the notation system}\label{sub:the_syntax_and_semantics_of_the_notation_system} % (fold)

We now summarize a technical presentation of the calculus that
embodies our theory of dynamics. The typical presentation of such a
calculus follows the style of giving generators and relations on
them. The grammar, below, describing term constructors, freely
generates the set of processes, $\Proc$. This set is then quotiented
by a relation known as structural congruence and it is over this set
that the notion of dynamics is expressed. This presentation is
essentially that of \cite{MeredithR05} with the addition of
polyadicity and summation. For readability we have relegated some of
the technical subtleties to an appendix.

\subsubsection{Process grammar}\label{subsub:process_grammar}

\begin{mathpar}
  \inferrule* [lab=synchronization] {} {{M} \bc \pzero \;|\; x?F \;|\; x!C }
  \and
  \inferrule* [lab=abstraction] {} {{F} \bc (x)P}
  \and
  \inferrule* [lab=concretion] {} {{C} \bc \langle Q \rangle}
  \and
  \inferrule* [lab=process] {} {{P,Q} \bc M \;| \;P|Q \;|\; @{x}}
  \and
  \inferrule* [lab=name] {} {{x} \bc \quotep{P}}
\end{mathpar} 

Note that $\vec{x}$ (resp. $\vec{P}$) denotes a vector of names
(resp. processes) of length $|\vec{x}|$ (resp. $|\vec{P}|$). We adopt
the following useful abbreviations.

\begin{mathpar}
   x?(\vec{y}).P := x.(\vec{y})P \and  x\clift{\vec{P}} := x.\clift{\vec{P}}
   \and x!(y) := \lift{x}{\dropn{y}}
   \and \Pi_{i=0}^{n-1}P_i := P_0 | \ldots | P_{n-1}
\end{mathpar}

\subsubsection{Structural congruence}

\paragraph{Free and bound names and alpha-equivalence.} At the
core of structural equivalence is alpha-equivalence which identifies
process that are the same up to a change of variable. Formally, we
recognize the distinction between free and bound names. The free names
of a process, $\freenames{P}$, may be calculated recursively as
follows:

\begin{mathpar}
\freenames{\pzero} := \emptyset
  \and \\
  \freenames{x?(y).P} := \{ x \} \cup (\freenames{P} \setminus \{ y \})
  \and 
  \freenames{x!\langle P \rangle} := \{ x \} \cup \{ P \} 
  \and \\
  \freenames{P|Q} := \freenames{P} \cup \freenames{Q}
  \and \\
  \freenames{@{x}} := \{ x \}
\end{mathpar}

$\pi$
$\quotep{\pi}$

$\freenames{-} : \pi \to \mathcal{P}(\quotep{\pi})$

\begin{eqnarray*}
  \freenames{\pzero} & := & \emptyset \\
  \freenames{x?(y).P} & := & \{ x \} \cup (\freenames{P} \setminus \{ y \}) \\
  \freenames{x!\langle P \rangle} & := & \{ x \} \cup \{ P \} \\
  \freenames{P|Q} & := & \freenames{P} \cup \freenames{Q} \\
  \freenames{\dropn{x}} & := & \{ x \}
\end{eqnarray*}

The bound names of a process, $\boundnames{P}$, are those names occurring in $P$
that are not free. For example, in $x?(y).0$, the name $x$ is free, while $y$ is bound.

\begin{mathpar}
  \inferrule* [lab=monoidal-laws] {} { P|Q \equiv Q|P \and P|0 \equiv P \and P|(Q|R) \equiv (P|Q)|R }
\end{mathpar}

\begin{mathpar}
  \inferrule* [lab=alpha-equivalence] {} { (x)P \equiv (y)P\{y/x\} \and y \not\in \freenames{P} }
\end{mathpar}

\begin{definition}
Then two processes, $P,Q$, are alpha-equivalent if $P = Q\{\vec{y}/\vec{x}\}$ for
some $\vec{x} \in \boundnames{Q},\vec{y} \in \boundnames{P}$, where $Q\{\vec{y}/\vec{x}\}$
denotes the capture-avoiding substitution of $\vec{y}$ for $\vec{x}$ in $Q$.
\end{definition}

\begin{definition}
  The {\em structural congruence} \cite{SangiorgiWalker} , $\equiv$,
  between processes is the least congruence containing
  alpha-equivalence, satisfying the abelian monoid laws
  (associativity, commutativity and $\pzero$ as identity) for parallel
  composition $|$ and for summation $+$.
\end{definition}

\subsection{Name equivalence}

We take name equivalence, written $\nameeq$, to be the smallest
equivalence relation generated by the following rules.

\begin{mathpar}
\inferrule*[lab=Quote-drop]
{ }
{ \quotep{@{x}} \nameeq x }

\inferrule*[lab=Struct-equiv]
{ P \scong Q }
{ \quotep{P} \nameeq \quotep{Q} }
\end{mathpar}

The astute reader will have noticed that the mutual recursion of names
and processes imposes a mutual recursion on alpha-equivalence and
structural equivalence via name-equivalence. Fortunately, all of this
works out pleasantly and we may calculate in the natural way, free of
concern. The reader interested in the details is referred to the
appendix \ref{appendix:rho_details}.

\subsection{Substitution}

We use $\Proc$ for the set of processes, $\QProc$ for the set of
names, and $\id{\{}\vec{y} / \vec{x} \id{\}}$ to denote partial maps,
$s : \QProc \rightarrow \QProc$. A map, $s$ lifts, uniquely, to a map
on process terms, $\widehat{s} : \Proc \rightarrow \Proc$ by the
following equations.

\begin{mathpar}
  (0) \psubstp{Q}{P} := 0 \\
  (R \juxtap S) \psubstp{Q}{P}
  :=    
  (R)\psubstp{Q}{P} \juxtap (S) \psubstp{Q}{P} \\
  (x?(y).R) \psubstp{Q}{P}    
  :=    
  (x)\substp{Q}{P} (z)\concat( (R \psubstn{z}{y}) \psubstp{Q}{P} ) \\
  (\lift{x}{R}) \psubstp{Q}{P}  
  :=
  \lift{(x)\substp{Q}{P}}{ R \psubstp{Q}{P} } \\
%   (\dropn{x})  \psubstp{Q}{P}       
%   := 
%   \left\{ 
%     \begin{array}{ccc} 
%       \dropn{\quotep{Q}} & & x \nameeq \quotep{P} \\
%       \dropn{x} & & otherwise \\
%     \end{array}
%   \right. 
  (\dropn{x})  \psubstp{Q}{P}       
  := 
  \left\{ 
    \begin{array}{ccc} 
      Q & & x \nameeq \quotep{P} \\
      \dropn{x} & & otherwise \\
    \end{array}
  \right.
\end{mathpar}
 

where

\begin{eqnarray}
  (x)\id{\{} \lpquote Q \rpquote / \lpquote P \rpquote \id{\}}            = 
  \left\{ 
    \begin{array}{ccc}
      \lpquote Q \rpquote & & x \nameeq \lpquote P \rpquote \\
      x & & otherwise \\
    \end{array}
  \right. \nonumber
\end{eqnarray}

and $z$ is chosen distinct from $\quotep{P}$, $\quotep{Q}$, the free
names in $Q$, and all the names in $R$. Our $\alpha$-equivalence will
be built in the standard way from this substitution.

\begin{remark}\label{rem:no_self_referential_names}
  One consequence of these definitions is that $\forall P. \quotep{P}
  \not\in \freenames{P}$.
\end{remark}

\subsection{ Dynamic quote: an example }

Anticipating something of what's to come, consider applying the
substitution, $\widehat{\id{\{}u / z \id{\}}}$, to the following pair
of processes, $\lift{w}{y!(z)}$ and $w[ \lpquote y!(z) \rpquote ]$.

\begin{eqnarray}
	\lift{w}{y!(z)}\widehat{\id{\{}u / z \id{\}}}
		& = &
		\lift{w}{y!(u)} \nonumber\\
	w[ \lpquote y!(z) \rpquote ] \widehat{ \id{\{}u / z \id{\}} }
		& = &
		w[ \lpquote y!(z) \rpquote ] \nonumber
\end{eqnarray}

Because the body of the process between quotes is impervious to
substitution, we get radically different answers. In fact, by
examining the first process in an input context,
e.g. $x?(z).\lift{w}{y!(z)}$, we see that the process under the lift
operator may be shaped by prefixed inputs binding a name inside it. In
this sense, the lift operator will be seen as a way to dynamically
construct processes before reifying them as names.

Finally equipped with these standard features we can present the
dynamics of the calculus.

\subsubsection{Operational semantics} 

Finally, we introduce the computational dynamics. What marks these
algebras as distinct from other more traditionally studied algebraic
structures, e.g. vector spaces or polynomial rings, is the manner in
which dynamics is captured. In traditional structures, dynamics is typically
expressed through morphisms between such structures, as in linear maps
between vector spaces or morphisms between rings. In algebras
associated with the semantics of computation, the dynamics is
expressed as part of the algebraic structure itself, through a
reduction reduction relation typically denoted by $\red$. Below, we
give a recursive presentation of this relation for the calculus used
in the encoding.

$\red \subseteq \pi \times \pi$
$\red : \pi \to \mathcal{P}(\pi)$

\begin{mathpar}
  \inferrule* [lab=Comm] { \textsf{match}( x_{src}, x_{trgt} ) } { x_{trgt}?(y)P \; | \; x_{src}!\langle {Q} \rangle \red P\{\quotep{Q}/y}\} }
  \and \\
  \inferrule* [lab=Par] {{P} \red {P}'} {{{P} | {Q}} \red {{P}' | {Q}}}
  \and
  \inferrule* [lab=Equiv]{{{P} \scong {P}'} \andalso {{P}' \red {Q}'} \andalso {{Q}' \scong {Q}}}{{P} \red {Q}}
\end{mathpar}

\begin{eqnarray*}
  match_{\equiv} (\quotep{P},\quotep{Q}) & := & P \equiv Q \\
  match_{\dagger}(\quotep{P},\quotep{Q}) & := & \forall R. P|Q \red^{*} R => R \red^{*} 0 \\
  match_{K}(\quotep{P},\quotep{Q}) & := & K \mbox{ for some context } K
\end{eqnarray*}

$u?(x)P | u!\langle Q \rangle \red P\{\quotep{Q}/x\}$

%We write $\wred$ for $\red^*$, and $P\red$ if $\exists Q $ such that $ P \red Q$.
We write $P\red$ if $\exists Q $ such that $ P \red Q$ and $P\not\red$, otherwise.

\section{Replication}

As mentioned before, it is known that replication (and hence
recursion) can be implemented in a higher-order process algebra
\cite{SangiorgiWalker}. As our first example of calculation with the
machinery thus far presented we give the construction explicitly in
the {\rhoc}.

\begin{eqnarray}
	D_{x} & := & \prefix{x}{y}{(\binpar{\outputp{x}{y}}{@{y}})} \nonumber\\
	\bangp_{x}{P} & := & \binpar{{x}!\langle{\binpar{D_{x}}{P}}\rangle}{D_{x}} \nonumber
\end{eqnarray}

\begin{eqnarray}
	\bangp_{x}{P} & & \nonumber\\
	=
	& {x}!\langle{(\prefix{x}{y}{(\outputp{x}{y} | @{y})) | P}}\rangle 
	      | \prefix{x}{y}{(\outputp{x}{y} | @{y})} & \nonumber\\
	\red
	& (\outputp{x}{y} | @{y})\substn{\quotep{(\prefix{x}{y}{(@{y} | \outputp{x}{y})) | P}}}{y} & \nonumber\\
	=
	& \outputp{x}{\quotep{(\prefix{x}{y}{(\outputp{x}{y} | @{y})) | P}}}
	  | {(\prefix{x}{y}{(\outputp{x}{y} | @{y})) | P}} & \nonumber\\
	\red
	& \ldots & \nonumber\\
	\red^*
	& P | P | \ldots & \nonumber
\end{eqnarray}

Of course, this encoding, as an implementation, runs away, unfolding
$\bangp{P}$ eagerly. A lazier and more implementable replication
operator, restricted to input-guarded processes, may be obtained as follows.

\begin{eqnarray}
\bangp{\prefix{u}{v}{P}} 
	:= 
	\binpar{\lift{x}{\prefix{u}{v}{(\binpar{D(x)}{P})}}}{D(x)} \nonumber
\end{eqnarray}

\begin{remark}
  Note that the lazier definition still does not deal with summation
  or mixed summation (i.e. sums over input and output). The reader is
  invited to construct definitions of replication that deal with these
  features. 

  Further, the definitions are parameterized in a name, $x$. Can you,
  gentle reader, make a definition that eliminates this parameter and
  guarantees no accidental interaction between the replication
  machinery and the process being replicated -- i.e. no accidental
  sharing of names used by the process to get its work done and the
  name(s) used by the replication to effect copying. This latter
  revision of the definition of replication is crucial to obtaining
  the expected identity $!!P \sim !P$.
\end{remark}

\begin{remark}\label{rem:paradoxical_combinator}
  The reader familiar with the lambda calculus will have noticed the
  similarity between $D$ and the paradoxical combinator.

  [Ed. note: the existence of this seems to suggest we have to be more
  restrictive on the set of processes and names we admit if we are to
  support no-cloning.]
\end{remark}

\subsubsection{Bisimulation}

The computational dynamics gives rise to another kind of equivalence,
the equivalence of computational behavior. As previously mentioned
this is typically captured \emph{via} some form of bisimulation.

% The notion we use in this paper is weak barbed bisimulation
% \cite{milner91polyadicpi}.

The notion we use in this paper is derived from weak barbed
bisimulation \cite{milner91polyadicpi}. 

\begin{definition}
An \emph{observation relation}, $\downarrow_{\mathcal N}$, over a set
of names, $\mathcal N$, is the smallest relation satisfying the rules
below.

\infrule[Out-barb]{y \in {\mathcal N}, \; x \nameeq y}
		  {\outputp{x}{v} \downarrow_{\mathcal N} x}
\infrule[Par-barb]{\mbox{$P\downarrow_{\mathcal N} x$ or $Q\downarrow_{\mathcal N} x$}}
		  {\binpar{P}{Q} \downarrow_{\mathcal N} x}

We write $P \Downarrow_{\mathcal N} x$ if there is $Q$ such that 
$P \wred Q$ and $Q \downarrow_{\mathcal N} x$.
\end{definition}

\begin{definition}
%\label{def.bbisim}
An  ${\mathcal N}$-\emph{barbed bisimulation} over a set of names, ${\mathcal N}$, is a symmetric binary relation 
${\mathcal S}_{\mathcal N}$ between agents such that $P\rel{S}_{\mathcal N}Q$ implies:
\begin{enumerate}
\item If $P \red P'$ then $Q \wred Q'$ and $P'\rel{S}_{\mathcal N} Q'$.
\item If $P\downarrow_{\mathcal N} x$, then $Q\Downarrow_{\mathcal N} x$.
\end{enumerate}
$P$ is ${\mathcal N}$-barbed bisimilar to $Q$, written
$P \wbbisim_{\mathcal N} Q$, if $P \rel{S}_{\mathcal N} Q$ for some ${\mathcal N}$-barbed bisimulation ${\mathcal S}_{\mathcal N}$.
\end{definition}

$\mathcal{R} \subseteq \pi \times \pi$

$P \mathcal{R} Q => \forall P'. P \red P' \Rightarrow \exists Q'. Q \red Q', P' \mathcal{R} Q'$

$P \vdash x \Rightarrow Q \vdash x$

\begin{mathpar}
  \inferrule*[lab=Out-barb]{x \nameeq y}{{y}!\langle{Q}\rangle \vdash x}
  \and
  \inferrule*[lab=Par-barb]{\mbox{$P\vdash x$ or $Q\vdash x$}}{\binpar{P}{Q} \vdash x}
\end{mathpar}

\subsubsection{Contexts}

One of the principle advantages of computational calculi like the
$\pi$-calculus is a well-defined notion of context,
contextual-equivalence and a correlation between
contextual-equivalence and notions of bisimulation. The notion of
context allows the decomposition of a process into (sub-)process and
its syntactic environment, its context. Thus, a context may be
thought of as a process with a ``hole'' (written $\Box$) in it. The
application of a context $M$ to a process $P$, written $M[P]$, is
tantamount to filling the hole in $M$ with $P$. In this paper we do
not need the full weight of this theory, but do make use of the notion
of context in the proof the main theorem. 

\begin{mathpar}
  \inferrule* [lab=summation] {} {{M_{M},M_{N}} \bc \Box \;|\; x.M_{A} \;|\; M_{M}+M_{N}}
  \and
  \inferrule* [lab=agent] {} {{M_{A}} \bc (\vec{x})M_{P} \;| \; \clift{P_0,\ldots,M_{P},\ldots,P_N}}
  \and \\
  \inferrule* [lab=process] {} {{M_{P}} \bc M_{N} \;| \;P|M_{P} }
\end{mathpar} 

\begin{mathpar}
  \inferrule* [lab=sychronization] {} {M_{N} \bc \Box \;|\; x?M_{F} \;|\; x!M_{C}}
  \and
  \inferrule* [lab=abstraction] {} {{M_{F}} \bc (x)M_{P} }
  \and
  \inferrule* [lab=concretion] {} {{M_{C}} \bc \langle M_{P} \rangle }
  \and \\
  \inferrule* [lab=process] {} {{M_{P}} \bc M_{N} \;| \;P|M_{P} }
\end{mathpar}

\begin{definition}[contextual application] Given a context $M$, and
  process $P$, we define the \emph{contextual application}, $M[P] :=
  M\{P/\Box\}$. That is, the contextual application of M to P is the
  substitution of $P$ for $\Box$ in $M$.
\end{definition}

$\meaningof{-} : L \to \mathcal{P}(\pi)$

\begin{mathpar}
  \inferrule* [lab=collection] {} {\meaningof{true} = \pi, \and \meaningof{~E} = \pi \setminus \meaningof{E}, \and \meaningof{E_{1} \& E_{2}} = \meaningof{E_{1}} \cap \meaningof{E_{2}}}
\end{mathpar}

\begin{mathpar}
  \inferrule* [lab=structure] {} {\meaningof{0} = \{ P \in \pi | P \equiv 0 \}, \and \\ \meaningof{E_1 | E_2} = \{ P \in \pi | P \equiv P_{1} | P_{2}, P_{1} \in \meaningof{E_{1}}, P_{2} \in \meaningof{E_2}\} }
\end{mathpar}

\begin{mathpar}
 \inferrule* [lab=behavior] {} {\meaningof{\langle a?b \rangle E} = \{ P \in \pi | P \equiv Q | u?(y)P', \\ \and \\\\ \and \\ \;\;\; u \in \meaningof{a}, \forall z.P'\{z/y\} \in \meaningof{E\{z/b\}}\}, \and \\ \meaningof{a!E} = \{ P \in \pi | P \equiv Q | x!\langle P' \rangle, x \in \meaningof{a} P' \in \meaningof{E}\} }
\end{mathpar}

\begin{mathpar}
 \inferrule* [lab=nominal] {} {\meaningof{\quotep{E}} = \{ \quotep{P} \in \quotep{\pi} | P \in \meaningof{E} \}, \and \meaningof{\quotep{P}} = \{ \quotep{Q} \in \quotep{\pi} | P \equiv Q \} \and \\ \meaningof{@\quotep{E}} = \{ P \in \pi | P \equiv @x, x \in \meaningof{E} \}}
\end{mathpar}

\begin{eqnarray*}
  \\
  \meaningof{-} : TS \to ST
\end{eqnarray*}

\begin{eqnarray*}
  \\
  L : TS \to ST
\end{eqnarray*}

\begin{eqnarray*}
  \\
  P \models E \iff P \in \meaningof{E}
\end{eqnarray*}

\begin{eqnarray*}
  P \approx_{L} Q \iff \forall E \in L. P \models E \iff Q \models E
\end{eqnarray*}

\begin{eqnarray*}
  P \approx_{K} Q
\end{eqnarray*}

\begin{eqnarray*}
  P \approx Q
\end{eqnarray*}

$\approx_{K} = \approx = \approx_{L}$

\subsubsection{Contextual duality}

Note that contexts extend the quotation operation to a family of
operations from processes to names. Given a context, $M$, we can
define a \emph{nominal context}, $\quotep{M}$ by $\quotep{M}[P] :=
\quotep{M[P]}$. To foreshadow what is to come we observe that these
operations enjoy a duality with processes very much like the duality
between vectors and maps from vectors to scalars.

Further, because the calculus is essentially higher-order, we have a
correspondence between contexts and processes. More specifically,
given a name $x$ and a context $M$ we can construct $M^{*}_{x}$ such
that 

\begin{mathpar}
  M^{*}_{x} | \lift{x}{P} \red M[P]
\end{mathpar}

namely,

\begin{mathpar}
  M^{*}_{x} := x?(u).M[\dropn{u}]
\end{mathpar}

The dependence of $M^{*}_{x}$ on a name makes it an abstraction, 

\begin{mathpar}
  M^{*} := (x)x?(u).M[\dropn{u}]
\end{mathpar}

\subsection{Additional notation}

It will sometimes be convenient to denote the process a name
quotes. We already have the notation $x = \quotep{P}$, but it will be
convenient to introduce an alternate notation, $\procn{x}$, when we
want to emphasize the connection to the use of the name. Note that, by
virtue of name equivalence, $\quotep{\procn{x}} \nameeq x$; so, the
notation is consistent with previous definitions.

Further, because names have structure it is possible to effect
substitutions on the basis of that structure. This means we need to
upgrade our notation for substitutions, which we accomplish by
adapting comprehension notation. Thus,

\begin{mathpar}
  P\{ y / x : x \in S \}
\end{mathpar}

is interpreted to mean the process derived from P by replacing (in a
capture-avoiding manner) each occurrence of $x$ in $S$ by $y$. For example,

\begin{mathpar}
  P\{ \quotep{\procn{x}|\procn{x}} / x : x \in \freenames{P} \}
\end{mathpar}

will replace each (occurrence) of a free name $x$ in $P$ by
$\quotep{\procn{x}|\procn{x}}$.

Also, we will avail ourselves of the notation $x^{L}$ and $x^{R}$ to
denote injections of a name into disjoint copies of the name
space. There are numerous ways to accomplish this. One example can be
found in \cite{MeredithR05}. This notation overloads to vectors of
names: $\vec{x}^{\pi} := (x_{i}^{\pi} \; : \; 0 \leq i < |\vec{x}| )$ where $\pi \in \{L,R\}$.

We also use $P^{\Box} := P|\Box$.

In \cite{MeredithR05} an interpretation of the new operator is
given. It turns out that there are several possible interpretations
all enjoying the requisite algebraic properties of the operator (see
\cite{milner91polyadicpi}). We will therefore make liberal use of
$(\nu\; \vec{x})P$.

% subsection the_syntax_and_semantics_of_the_notation_system (end)   

\input{qm2pi.qmops} 

\input{qm2pi.sterngerlach} 

\input{qm2pi.metric} 

% section concurrent_process_calculi (end)

%\input{qm2pi.proofsketch}

% section proof sketch (end)

%\input{qm2pi.slviaknots} 

% section spatial logic via knots (end)

\input{qm2pi.conclusion}

% section conclusion (end)

%\input{qm2pi.dtcodes} 

% section wiring algorithm (end)

\input{qm2pi.ack} 

% section acknowledgments (end)

\newpage


\bibliographystyle{plain}   
\bibliography{../../biblios/main.bib}

\input{qm2pi.rhodetails}

\end{document}

 

%\documentclass[12pt]{llncs}
%\documentclass{jktr}

\usepackage[pdftex]{hyperref}                   
\usepackage {listings}
\usepackage {mathpartir}
\usepackage{bcprules}
%\usepackage{listings}
                       
\usepackage{graphicx} 
%\usepackage[margins=2.5cm,nohead,nofoot]{geometry}
%\usepackage{geometry}
\usepackage{amsfonts}
\usepackage{amstext}
\usepackage{latexsym}
\usepackage{amssymb}
\usepackage{color}


%\include{myPreamble}
\include{qm2pi.local} 

%\ifpdf
%\usepackage[pdftex]{graphicx}
%\else
%\usepackage{graphicx}
%\fi

 % \ifpdf
%  \usepackage{pdfsync}
%  \if


%\title{Brief Article}
%\author{David F. Snyder}
%\author{L.G. Meredith}

%\address{Dept. of Math., Texas State University--San Marcos, San Marcos, TX 78666}
       
\pagestyle{empty}


\begin{document}

\lstset{language=[Objective]Caml,frame=shadowbox}

\input{qm2pi.front}

% section front matter (end)

\input{qm2pi.intro} 
 
% section introduction (end)

% \input{qm2pi.knotations} 

% section notation (end)

\input{qm2pi.process.calculi} 

% section concurrent_process_calculi_and_spatial_logics_ (end)
    
%\input{qm2pi.knots2pi} 

%\input{qm2pi.trefoil} 

%\input{qm2pi.mainthm} 

% subsection basic_interpretation (end)

%\input{qm2pi.rho.presentation} 
\subsection{The syntax and semantics of the notation system}\label{sub:the_syntax_and_semantics_of_the_notation_system} % (fold)

We now summarize a technical presentation of the calculus that
embodies our theory of dynamics. The typical presentation of such a
calculus follows the style of giving generators and relations on
them. The grammar, below, describing term constructors, freely
generates the set of processes, $\Proc$. This set is then quotiented
by a relation known as structural congruence and it is over this set
that the notion of dynamics is expressed. This presentation is
essentially that of \cite{MeredithR05} with the addition of
polyadicity and summation. For readability we have relegated some of
the technical subtleties to an appendix.

\subsubsection{Process grammar}\label{subsub:process_grammar}

\begin{mathpar}
  \inferrule* [lab=synchronization] {} {{M} \bc \pzero \;|\; x?F \;|\; x!C }
  \and
  \inferrule* [lab=abstraction] {} {{F} \bc (x)P}
  \and
  \inferrule* [lab=concretion] {} {{C} \bc \langle Q \rangle}
  \and
  \inferrule* [lab=process] {} {{P,Q} \bc M \;| \;P|Q \;|\; @{x}}
  \and
  \inferrule* [lab=name] {} {{x} \bc \quotep{P}}
\end{mathpar} 

Note that $\vec{x}$ (resp. $\vec{P}$) denotes a vector of names
(resp. processes) of length $|\vec{x}|$ (resp. $|\vec{P}|$). We adopt
the following useful abbreviations.

\begin{mathpar}
   x?(\vec{y}).P := x.(\vec{y})P \and  x\clift{\vec{P}} := x.\clift{\vec{P}}
   \and x!(y) := \lift{x}{\dropn{y}}
   \and \Pi_{i=0}^{n-1}P_i := P_0 | \ldots | P_{n-1}
\end{mathpar}

\subsubsection{Structural congruence}

\paragraph{Free and bound names and alpha-equivalence.} At the
core of structural equivalence is alpha-equivalence which identifies
process that are the same up to a change of variable. Formally, we
recognize the distinction between free and bound names. The free names
of a process, $\freenames{P}$, may be calculated recursively as
follows:

\begin{mathpar}
\freenames{\pzero} := \emptyset
  \and \\
  \freenames{x?(y).P} := \{ x \} \cup (\freenames{P} \setminus \{ y \})
  \and 
  \freenames{x!\langle P \rangle} := \{ x \} \cup \{ P \} 
  \and \\
  \freenames{P|Q} := \freenames{P} \cup \freenames{Q}
  \and \\
  \freenames{@{x}} := \{ x \}
\end{mathpar}

$\pi$
$\quotep{\pi}$

$\freenames{-} : \pi \to \mathcal{P}(\quotep{\pi})$

\begin{eqnarray*}
  \freenames{\pzero} & := & \emptyset \\
  \freenames{x?(y).P} & := & \{ x \} \cup (\freenames{P} \setminus \{ y \}) \\
  \freenames{x!\langle P \rangle} & := & \{ x \} \cup \{ P \} \\
  \freenames{P|Q} & := & \freenames{P} \cup \freenames{Q} \\
  \freenames{\dropn{x}} & := & \{ x \}
\end{eqnarray*}

The bound names of a process, $\boundnames{P}$, are those names occurring in $P$
that are not free. For example, in $x?(y).0$, the name $x$ is free, while $y$ is bound.

\begin{mathpar}
  \inferrule* [lab=monoidal-laws] {} { P|Q \equiv Q|P \and P|0 \equiv P \and P|(Q|R) \equiv (P|Q)|R }
\end{mathpar}

\begin{mathpar}
  \inferrule* [lab=alpha-equivalence] {} { (x)P \equiv (y)P\{y/x\} \and y \not\in \freenames{P} }
\end{mathpar}

\begin{definition}
Then two processes, $P,Q$, are alpha-equivalent if $P = Q\{\vec{y}/\vec{x}\}$ for
some $\vec{x} \in \boundnames{Q},\vec{y} \in \boundnames{P}$, where $Q\{\vec{y}/\vec{x}\}$
denotes the capture-avoiding substitution of $\vec{y}$ for $\vec{x}$ in $Q$.
\end{definition}

\begin{definition}
  The {\em structural congruence} \cite{SangiorgiWalker} , $\equiv$,
  between processes is the least congruence containing
  alpha-equivalence, satisfying the abelian monoid laws
  (associativity, commutativity and $\pzero$ as identity) for parallel
  composition $|$ and for summation $+$.
\end{definition}

\subsection{Name equivalence}

We take name equivalence, written $\nameeq$, to be the smallest
equivalence relation generated by the following rules.

\begin{mathpar}
\inferrule*[lab=Quote-drop]
{ }
{ \quotep{@{x}} \nameeq x }

\inferrule*[lab=Struct-equiv]
{ P \scong Q }
{ \quotep{P} \nameeq \quotep{Q} }
\end{mathpar}

The astute reader will have noticed that the mutual recursion of names
and processes imposes a mutual recursion on alpha-equivalence and
structural equivalence via name-equivalence. Fortunately, all of this
works out pleasantly and we may calculate in the natural way, free of
concern. The reader interested in the details is referred to the
appendix \ref{appendix:rho_details}.

\subsection{Substitution}

We use $\Proc$ for the set of processes, $\QProc$ for the set of
names, and $\id{\{}\vec{y} / \vec{x} \id{\}}$ to denote partial maps,
$s : \QProc \rightarrow \QProc$. A map, $s$ lifts, uniquely, to a map
on process terms, $\widehat{s} : \Proc \rightarrow \Proc$ by the
following equations.

\begin{mathpar}
  (0) \psubstp{Q}{P} := 0 \\
  (R \juxtap S) \psubstp{Q}{P}
  :=    
  (R)\psubstp{Q}{P} \juxtap (S) \psubstp{Q}{P} \\
  (x?(y).R) \psubstp{Q}{P}    
  :=    
  (x)\substp{Q}{P} (z)\concat( (R \psubstn{z}{y}) \psubstp{Q}{P} ) \\
  (\lift{x}{R}) \psubstp{Q}{P}  
  :=
  \lift{(x)\substp{Q}{P}}{ R \psubstp{Q}{P} } \\
%   (\dropn{x})  \psubstp{Q}{P}       
%   := 
%   \left\{ 
%     \begin{array}{ccc} 
%       \dropn{\quotep{Q}} & & x \nameeq \quotep{P} \\
%       \dropn{x} & & otherwise \\
%     \end{array}
%   \right. 
  (\dropn{x})  \psubstp{Q}{P}       
  := 
  \left\{ 
    \begin{array}{ccc} 
      Q & & x \nameeq \quotep{P} \\
      \dropn{x} & & otherwise \\
    \end{array}
  \right.
\end{mathpar}
 

where

\begin{eqnarray}
  (x)\id{\{} \lpquote Q \rpquote / \lpquote P \rpquote \id{\}}            = 
  \left\{ 
    \begin{array}{ccc}
      \lpquote Q \rpquote & & x \nameeq \lpquote P \rpquote \\
      x & & otherwise \\
    \end{array}
  \right. \nonumber
\end{eqnarray}

and $z$ is chosen distinct from $\quotep{P}$, $\quotep{Q}$, the free
names in $Q$, and all the names in $R$. Our $\alpha$-equivalence will
be built in the standard way from this substitution.

\begin{remark}\label{rem:no_self_referential_names}
  One consequence of these definitions is that $\forall P. \quotep{P}
  \not\in \freenames{P}$.
\end{remark}

\subsection{ Dynamic quote: an example }

Anticipating something of what's to come, consider applying the
substitution, $\widehat{\id{\{}u / z \id{\}}}$, to the following pair
of processes, $\lift{w}{y!(z)}$ and $w[ \lpquote y!(z) \rpquote ]$.

\begin{eqnarray}
	\lift{w}{y!(z)}\widehat{\id{\{}u / z \id{\}}}
		& = &
		\lift{w}{y!(u)} \nonumber\\
	w[ \lpquote y!(z) \rpquote ] \widehat{ \id{\{}u / z \id{\}} }
		& = &
		w[ \lpquote y!(z) \rpquote ] \nonumber
\end{eqnarray}

Because the body of the process between quotes is impervious to
substitution, we get radically different answers. In fact, by
examining the first process in an input context,
e.g. $x?(z).\lift{w}{y!(z)}$, we see that the process under the lift
operator may be shaped by prefixed inputs binding a name inside it. In
this sense, the lift operator will be seen as a way to dynamically
construct processes before reifying them as names.

Finally equipped with these standard features we can present the
dynamics of the calculus.

\subsubsection{Operational semantics} 

Finally, we introduce the computational dynamics. What marks these
algebras as distinct from other more traditionally studied algebraic
structures, e.g. vector spaces or polynomial rings, is the manner in
which dynamics is captured. In traditional structures, dynamics is typically
expressed through morphisms between such structures, as in linear maps
between vector spaces or morphisms between rings. In algebras
associated with the semantics of computation, the dynamics is
expressed as part of the algebraic structure itself, through a
reduction reduction relation typically denoted by $\red$. Below, we
give a recursive presentation of this relation for the calculus used
in the encoding.

$\red \subseteq \pi \times \pi$
$\red : \pi \to \mathcal{P}(\pi)$

\begin{mathpar}
  \inferrule* [lab=Comm] { \textsf{match}( x_{src}, x_{trgt} ) } { x_{trgt}?(y)P \; | \; x_{src}!\langle {Q} \rangle \red P\{\quotep{Q}/y}\} }
  \and \\
  \inferrule* [lab=Par] {{P} \red {P}'} {{{P} | {Q}} \red {{P}' | {Q}}}
  \and
  \inferrule* [lab=Equiv]{{{P} \scong {P}'} \andalso {{P}' \red {Q}'} \andalso {{Q}' \scong {Q}}}{{P} \red {Q}}
\end{mathpar}

\begin{eqnarray*}
  match_{\equiv} (\quotep{P},\quotep{Q}) & := & P \equiv Q \\
  match_{\dagger}(\quotep{P},\quotep{Q}) & := & \forall R. P|Q \red^{*} R => R \red^{*} 0 \\
  match_{K}(\quotep{P},\quotep{Q}) & := & K \mbox{ for some context } K
\end{eqnarray*}

$u?(x)P | u!\langle Q \rangle \red P\{\quotep{Q}/x\}$

%We write $\wred$ for $\red^*$, and $P\red$ if $\exists Q $ such that $ P \red Q$.
We write $P\red$ if $\exists Q $ such that $ P \red Q$ and $P\not\red$, otherwise.

\section{Replication}

As mentioned before, it is known that replication (and hence
recursion) can be implemented in a higher-order process algebra
\cite{SangiorgiWalker}. As our first example of calculation with the
machinery thus far presented we give the construction explicitly in
the {\rhoc}.

\begin{eqnarray}
	D_{x} & := & \prefix{x}{y}{(\binpar{\outputp{x}{y}}{@{y}})} \nonumber\\
	\bangp_{x}{P} & := & \binpar{{x}!\langle{\binpar{D_{x}}{P}}\rangle}{D_{x}} \nonumber
\end{eqnarray}

\begin{eqnarray}
	\bangp_{x}{P} & & \nonumber\\
	=
	& {x}!\langle{(\prefix{x}{y}{(\outputp{x}{y} | @{y})) | P}}\rangle 
	      | \prefix{x}{y}{(\outputp{x}{y} | @{y})} & \nonumber\\
	\red
	& (\outputp{x}{y} | @{y})\substn{\quotep{(\prefix{x}{y}{(@{y} | \outputp{x}{y})) | P}}}{y} & \nonumber\\
	=
	& \outputp{x}{\quotep{(\prefix{x}{y}{(\outputp{x}{y} | @{y})) | P}}}
	  | {(\prefix{x}{y}{(\outputp{x}{y} | @{y})) | P}} & \nonumber\\
	\red
	& \ldots & \nonumber\\
	\red^*
	& P | P | \ldots & \nonumber
\end{eqnarray}

Of course, this encoding, as an implementation, runs away, unfolding
$\bangp{P}$ eagerly. A lazier and more implementable replication
operator, restricted to input-guarded processes, may be obtained as follows.

\begin{eqnarray}
\bangp{\prefix{u}{v}{P}} 
	:= 
	\binpar{\lift{x}{\prefix{u}{v}{(\binpar{D(x)}{P})}}}{D(x)} \nonumber
\end{eqnarray}

\begin{remark}
  Note that the lazier definition still does not deal with summation
  or mixed summation (i.e. sums over input and output). The reader is
  invited to construct definitions of replication that deal with these
  features. 

  Further, the definitions are parameterized in a name, $x$. Can you,
  gentle reader, make a definition that eliminates this parameter and
  guarantees no accidental interaction between the replication
  machinery and the process being replicated -- i.e. no accidental
  sharing of names used by the process to get its work done and the
  name(s) used by the replication to effect copying. This latter
  revision of the definition of replication is crucial to obtaining
  the expected identity $!!P \sim !P$.
\end{remark}

\begin{remark}\label{rem:paradoxical_combinator}
  The reader familiar with the lambda calculus will have noticed the
  similarity between $D$ and the paradoxical combinator.

  [Ed. note: the existence of this seems to suggest we have to be more
  restrictive on the set of processes and names we admit if we are to
  support no-cloning.]
\end{remark}

\subsubsection{Bisimulation}

The computational dynamics gives rise to another kind of equivalence,
the equivalence of computational behavior. As previously mentioned
this is typically captured \emph{via} some form of bisimulation.

% The notion we use in this paper is weak barbed bisimulation
% \cite{milner91polyadicpi}.

The notion we use in this paper is derived from weak barbed
bisimulation \cite{milner91polyadicpi}. 

\begin{definition}
An \emph{observation relation}, $\downarrow_{\mathcal N}$, over a set
of names, $\mathcal N$, is the smallest relation satisfying the rules
below.

\infrule[Out-barb]{y \in {\mathcal N}, \; x \nameeq y}
		  {\outputp{x}{v} \downarrow_{\mathcal N} x}
\infrule[Par-barb]{\mbox{$P\downarrow_{\mathcal N} x$ or $Q\downarrow_{\mathcal N} x$}}
		  {\binpar{P}{Q} \downarrow_{\mathcal N} x}

We write $P \Downarrow_{\mathcal N} x$ if there is $Q$ such that 
$P \wred Q$ and $Q \downarrow_{\mathcal N} x$.
\end{definition}

\begin{definition}
%\label{def.bbisim}
An  ${\mathcal N}$-\emph{barbed bisimulation} over a set of names, ${\mathcal N}$, is a symmetric binary relation 
${\mathcal S}_{\mathcal N}$ between agents such that $P\rel{S}_{\mathcal N}Q$ implies:
\begin{enumerate}
\item If $P \red P'$ then $Q \wred Q'$ and $P'\rel{S}_{\mathcal N} Q'$.
\item If $P\downarrow_{\mathcal N} x$, then $Q\Downarrow_{\mathcal N} x$.
\end{enumerate}
$P$ is ${\mathcal N}$-barbed bisimilar to $Q$, written
$P \wbbisim_{\mathcal N} Q$, if $P \rel{S}_{\mathcal N} Q$ for some ${\mathcal N}$-barbed bisimulation ${\mathcal S}_{\mathcal N}$.
\end{definition}

$\mathcal{R} \subseteq \pi \times \pi$

$P \mathcal{R} Q => \forall P'. P \red P' \Rightarrow \exists Q'. Q \red Q', P' \mathcal{R} Q'$

$P \vdash x \Rightarrow Q \vdash x$

\begin{mathpar}
  \inferrule*[lab=Out-barb]{x \nameeq y}{{y}!\langle{Q}\rangle \vdash x}
  \and
  \inferrule*[lab=Par-barb]{\mbox{$P\vdash x$ or $Q\vdash x$}}{\binpar{P}{Q} \vdash x}
\end{mathpar}

\subsubsection{Contexts}

One of the principle advantages of computational calculi like the
$\pi$-calculus is a well-defined notion of context,
contextual-equivalence and a correlation between
contextual-equivalence and notions of bisimulation. The notion of
context allows the decomposition of a process into (sub-)process and
its syntactic environment, its context. Thus, a context may be
thought of as a process with a ``hole'' (written $\Box$) in it. The
application of a context $M$ to a process $P$, written $M[P]$, is
tantamount to filling the hole in $M$ with $P$. In this paper we do
not need the full weight of this theory, but do make use of the notion
of context in the proof the main theorem. 

\begin{mathpar}
  \inferrule* [lab=summation] {} {{M_{M},M_{N}} \bc \Box \;|\; x.M_{A} \;|\; M_{M}+M_{N}}
  \and
  \inferrule* [lab=agent] {} {{M_{A}} \bc (\vec{x})M_{P} \;| \; \clift{P_0,\ldots,M_{P},\ldots,P_N}}
  \and \\
  \inferrule* [lab=process] {} {{M_{P}} \bc M_{N} \;| \;P|M_{P} }
\end{mathpar} 

\begin{mathpar}
  \inferrule* [lab=sychronization] {} {M_{N} \bc \Box \;|\; x?M_{F} \;|\; x!M_{C}}
  \and
  \inferrule* [lab=abstraction] {} {{M_{F}} \bc (x)M_{P} }
  \and
  \inferrule* [lab=concretion] {} {{M_{C}} \bc \langle M_{P} \rangle }
  \and \\
  \inferrule* [lab=process] {} {{M_{P}} \bc M_{N} \;| \;P|M_{P} }
\end{mathpar}

\begin{definition}[contextual application] Given a context $M$, and
  process $P$, we define the \emph{contextual application}, $M[P] :=
  M\{P/\Box\}$. That is, the contextual application of M to P is the
  substitution of $P$ for $\Box$ in $M$.
\end{definition}

$\meaningof{-} : L \to \mathcal{P}(\pi)$

\begin{mathpar}
  \inferrule* [lab=collection] {} {\meaningof{true} = \pi, \and \meaningof{~E} = \pi \setminus \meaningof{E}, \and \meaningof{E_{1} \& E_{2}} = \meaningof{E_{1}} \cap \meaningof{E_{2}}}
\end{mathpar}

\begin{mathpar}
  \inferrule* [lab=structure] {} {\meaningof{0} = \{ P \in \pi | P \equiv 0 \}, \and \\ \meaningof{E_1 | E_2} = \{ P \in \pi | P \equiv P_{1} | P_{2}, P_{1} \in \meaningof{E_{1}}, P_{2} \in \meaningof{E_2}\} }
\end{mathpar}

\begin{mathpar}
 \inferrule* [lab=behavior] {} {\meaningof{\langle a?b \rangle E} = \{ P \in \pi | P \equiv Q | u?(y)P', \\ \and \\\\ \and \\ \;\;\; u \in \meaningof{a}, \forall z.P'\{z/y\} \in \meaningof{E\{z/b\}}\}, \and \\ \meaningof{a!E} = \{ P \in \pi | P \equiv Q | x!\langle P' \rangle, x \in \meaningof{a} P' \in \meaningof{E}\} }
\end{mathpar}

\begin{mathpar}
 \inferrule* [lab=nominal] {} {\meaningof{\quotep{E}} = \{ \quotep{P} \in \quotep{\pi} | P \in \meaningof{E} \}, \and \meaningof{\quotep{P}} = \{ \quotep{Q} \in \quotep{\pi} | P \equiv Q \} \and \\ \meaningof{@\quotep{E}} = \{ P \in \pi | P \equiv @x, x \in \meaningof{E} \}}
\end{mathpar}

\begin{eqnarray*}
  \\
  \meaningof{-} : TS \to ST
\end{eqnarray*}

\begin{eqnarray*}
  \\
  L : TS \to ST
\end{eqnarray*}

\begin{eqnarray*}
  \\
  P \models E \iff P \in \meaningof{E}
\end{eqnarray*}

\begin{eqnarray*}
  P \approx_{L} Q \iff \forall E \in L. P \models E \iff Q \models E
\end{eqnarray*}

\begin{eqnarray*}
  P \approx_{K} Q
\end{eqnarray*}

\begin{eqnarray*}
  P \approx Q
\end{eqnarray*}

$\approx_{K} = \approx = \approx_{L}$

\subsubsection{Contextual duality}

Note that contexts extend the quotation operation to a family of
operations from processes to names. Given a context, $M$, we can
define a \emph{nominal context}, $\quotep{M}$ by $\quotep{M}[P] :=
\quotep{M[P]}$. To foreshadow what is to come we observe that these
operations enjoy a duality with processes very much like the duality
between vectors and maps from vectors to scalars.

Further, because the calculus is essentially higher-order, we have a
correspondence between contexts and processes. More specifically,
given a name $x$ and a context $M$ we can construct $M^{*}_{x}$ such
that 

\begin{mathpar}
  M^{*}_{x} | \lift{x}{P} \red M[P]
\end{mathpar}

namely,

\begin{mathpar}
  M^{*}_{x} := x?(u).M[\dropn{u}]
\end{mathpar}

The dependence of $M^{*}_{x}$ on a name makes it an abstraction, 

\begin{mathpar}
  M^{*} := (x)x?(u).M[\dropn{u}]
\end{mathpar}

\subsection{Additional notation}

It will sometimes be convenient to denote the process a name
quotes. We already have the notation $x = \quotep{P}$, but it will be
convenient to introduce an alternate notation, $\procn{x}$, when we
want to emphasize the connection to the use of the name. Note that, by
virtue of name equivalence, $\quotep{\procn{x}} \nameeq x$; so, the
notation is consistent with previous definitions.

Further, because names have structure it is possible to effect
substitutions on the basis of that structure. This means we need to
upgrade our notation for substitutions, which we accomplish by
adapting comprehension notation. Thus,

\begin{mathpar}
  P\{ y / x : x \in S \}
\end{mathpar}

is interpreted to mean the process derived from P by replacing (in a
capture-avoiding manner) each occurrence of $x$ in $S$ by $y$. For example,

\begin{mathpar}
  P\{ \quotep{\procn{x}|\procn{x}} / x : x \in \freenames{P} \}
\end{mathpar}

will replace each (occurrence) of a free name $x$ in $P$ by
$\quotep{\procn{x}|\procn{x}}$.

Also, we will avail ourselves of the notation $x^{L}$ and $x^{R}$ to
denote injections of a name into disjoint copies of the name
space. There are numerous ways to accomplish this. One example can be
found in \cite{MeredithR05}. This notation overloads to vectors of
names: $\vec{x}^{\pi} := (x_{i}^{\pi} \; : \; 0 \leq i < |\vec{x}| )$ where $\pi \in \{L,R\}$.

We also use $P^{\Box} := P|\Box$.

In \cite{MeredithR05} an interpretation of the new operator is
given. It turns out that there are several possible interpretations
all enjoying the requisite algebraic properties of the operator (see
\cite{milner91polyadicpi}). We will therefore make liberal use of
$(\nu\; \vec{x})P$.

% subsection the_syntax_and_semantics_of_the_notation_system (end)   

\input{qm2pi.qmops} 

\input{qm2pi.sterngerlach} 

\input{qm2pi.metric} 

% section concurrent_process_calculi (end)

%\input{qm2pi.proofsketch}

% section proof sketch (end)

%\input{qm2pi.slviaknots} 

% section spatial logic via knots (end)

\input{qm2pi.conclusion}

% section conclusion (end)

%\input{qm2pi.dtcodes} 

% section wiring algorithm (end)

\input{qm2pi.ack} 

% section acknowledgments (end)

\newpage


\bibliographystyle{plain}   
\bibliography{../../biblios/main.bib}

\input{qm2pi.rhodetails}

\end{document}

 

%\documentclass[12pt]{llncs}
%\documentclass{jktr}

\usepackage[pdftex]{hyperref}                   
\usepackage {listings}
\usepackage {mathpartir}
\usepackage{bcprules}
%\usepackage{listings}
                       
\usepackage{graphicx} 
%\usepackage[margins=2.5cm,nohead,nofoot]{geometry}
%\usepackage{geometry}
\usepackage{amsfonts}
\usepackage{amstext}
\usepackage{latexsym}
\usepackage{amssymb}
\usepackage{color}


%\include{myPreamble}
\include{qm2pi.local} 

%\ifpdf
%\usepackage[pdftex]{graphicx}
%\else
%\usepackage{graphicx}
%\fi

 % \ifpdf
%  \usepackage{pdfsync}
%  \if


%\title{Brief Article}
%\author{David F. Snyder}
%\author{L.G. Meredith}

%\address{Dept. of Math., Texas State University--San Marcos, San Marcos, TX 78666}
       
\pagestyle{empty}


\begin{document}

\lstset{language=[Objective]Caml,frame=shadowbox}

\input{qm2pi.front}

% section front matter (end)

\input{qm2pi.intro} 
 
% section introduction (end)

% \input{qm2pi.knotations} 

% section notation (end)

\input{qm2pi.process.calculi} 

% section concurrent_process_calculi_and_spatial_logics_ (end)
    
%\input{qm2pi.knots2pi} 

%\input{qm2pi.trefoil} 

%\input{qm2pi.mainthm} 

% subsection basic_interpretation (end)

%\input{qm2pi.rho.presentation} 
\subsection{The syntax and semantics of the notation system}\label{sub:the_syntax_and_semantics_of_the_notation_system} % (fold)

We now summarize a technical presentation of the calculus that
embodies our theory of dynamics. The typical presentation of such a
calculus follows the style of giving generators and relations on
them. The grammar, below, describing term constructors, freely
generates the set of processes, $\Proc$. This set is then quotiented
by a relation known as structural congruence and it is over this set
that the notion of dynamics is expressed. This presentation is
essentially that of \cite{MeredithR05} with the addition of
polyadicity and summation. For readability we have relegated some of
the technical subtleties to an appendix.

\subsubsection{Process grammar}\label{subsub:process_grammar}

\begin{mathpar}
  \inferrule* [lab=synchronization] {} {{M} \bc \pzero \;|\; x?F \;|\; x!C }
  \and
  \inferrule* [lab=abstraction] {} {{F} \bc (x)P}
  \and
  \inferrule* [lab=concretion] {} {{C} \bc \langle Q \rangle}
  \and
  \inferrule* [lab=process] {} {{P,Q} \bc M \;| \;P|Q \;|\; @{x}}
  \and
  \inferrule* [lab=name] {} {{x} \bc \quotep{P}}
\end{mathpar} 

Note that $\vec{x}$ (resp. $\vec{P}$) denotes a vector of names
(resp. processes) of length $|\vec{x}|$ (resp. $|\vec{P}|$). We adopt
the following useful abbreviations.

\begin{mathpar}
   x?(\vec{y}).P := x.(\vec{y})P \and  x\clift{\vec{P}} := x.\clift{\vec{P}}
   \and x!(y) := \lift{x}{\dropn{y}}
   \and \Pi_{i=0}^{n-1}P_i := P_0 | \ldots | P_{n-1}
\end{mathpar}

\subsubsection{Structural congruence}

\paragraph{Free and bound names and alpha-equivalence.} At the
core of structural equivalence is alpha-equivalence which identifies
process that are the same up to a change of variable. Formally, we
recognize the distinction between free and bound names. The free names
of a process, $\freenames{P}$, may be calculated recursively as
follows:

\begin{mathpar}
\freenames{\pzero} := \emptyset
  \and \\
  \freenames{x?(y).P} := \{ x \} \cup (\freenames{P} \setminus \{ y \})
  \and 
  \freenames{x!\langle P \rangle} := \{ x \} \cup \{ P \} 
  \and \\
  \freenames{P|Q} := \freenames{P} \cup \freenames{Q}
  \and \\
  \freenames{@{x}} := \{ x \}
\end{mathpar}

$\pi$
$\quotep{\pi}$

$\freenames{-} : \pi \to \mathcal{P}(\quotep{\pi})$

\begin{eqnarray*}
  \freenames{\pzero} & := & \emptyset \\
  \freenames{x?(y).P} & := & \{ x \} \cup (\freenames{P} \setminus \{ y \}) \\
  \freenames{x!\langle P \rangle} & := & \{ x \} \cup \{ P \} \\
  \freenames{P|Q} & := & \freenames{P} \cup \freenames{Q} \\
  \freenames{\dropn{x}} & := & \{ x \}
\end{eqnarray*}

The bound names of a process, $\boundnames{P}$, are those names occurring in $P$
that are not free. For example, in $x?(y).0$, the name $x$ is free, while $y$ is bound.

\begin{mathpar}
  \inferrule* [lab=monoidal-laws] {} { P|Q \equiv Q|P \and P|0 \equiv P \and P|(Q|R) \equiv (P|Q)|R }
\end{mathpar}

\begin{mathpar}
  \inferrule* [lab=alpha-equivalence] {} { (x)P \equiv (y)P\{y/x\} \and y \not\in \freenames{P} }
\end{mathpar}

\begin{definition}
Then two processes, $P,Q$, are alpha-equivalent if $P = Q\{\vec{y}/\vec{x}\}$ for
some $\vec{x} \in \boundnames{Q},\vec{y} \in \boundnames{P}$, where $Q\{\vec{y}/\vec{x}\}$
denotes the capture-avoiding substitution of $\vec{y}$ for $\vec{x}$ in $Q$.
\end{definition}

\begin{definition}
  The {\em structural congruence} \cite{SangiorgiWalker} , $\equiv$,
  between processes is the least congruence containing
  alpha-equivalence, satisfying the abelian monoid laws
  (associativity, commutativity and $\pzero$ as identity) for parallel
  composition $|$ and for summation $+$.
\end{definition}

\subsection{Name equivalence}

We take name equivalence, written $\nameeq$, to be the smallest
equivalence relation generated by the following rules.

\begin{mathpar}
\inferrule*[lab=Quote-drop]
{ }
{ \quotep{@{x}} \nameeq x }

\inferrule*[lab=Struct-equiv]
{ P \scong Q }
{ \quotep{P} \nameeq \quotep{Q} }
\end{mathpar}

The astute reader will have noticed that the mutual recursion of names
and processes imposes a mutual recursion on alpha-equivalence and
structural equivalence via name-equivalence. Fortunately, all of this
works out pleasantly and we may calculate in the natural way, free of
concern. The reader interested in the details is referred to the
appendix \ref{appendix:rho_details}.

\subsection{Substitution}

We use $\Proc$ for the set of processes, $\QProc$ for the set of
names, and $\id{\{}\vec{y} / \vec{x} \id{\}}$ to denote partial maps,
$s : \QProc \rightarrow \QProc$. A map, $s$ lifts, uniquely, to a map
on process terms, $\widehat{s} : \Proc \rightarrow \Proc$ by the
following equations.

\begin{mathpar}
  (0) \psubstp{Q}{P} := 0 \\
  (R \juxtap S) \psubstp{Q}{P}
  :=    
  (R)\psubstp{Q}{P} \juxtap (S) \psubstp{Q}{P} \\
  (x?(y).R) \psubstp{Q}{P}    
  :=    
  (x)\substp{Q}{P} (z)\concat( (R \psubstn{z}{y}) \psubstp{Q}{P} ) \\
  (\lift{x}{R}) \psubstp{Q}{P}  
  :=
  \lift{(x)\substp{Q}{P}}{ R \psubstp{Q}{P} } \\
%   (\dropn{x})  \psubstp{Q}{P}       
%   := 
%   \left\{ 
%     \begin{array}{ccc} 
%       \dropn{\quotep{Q}} & & x \nameeq \quotep{P} \\
%       \dropn{x} & & otherwise \\
%     \end{array}
%   \right. 
  (\dropn{x})  \psubstp{Q}{P}       
  := 
  \left\{ 
    \begin{array}{ccc} 
      Q & & x \nameeq \quotep{P} \\
      \dropn{x} & & otherwise \\
    \end{array}
  \right.
\end{mathpar}
 

where

\begin{eqnarray}
  (x)\id{\{} \lpquote Q \rpquote / \lpquote P \rpquote \id{\}}            = 
  \left\{ 
    \begin{array}{ccc}
      \lpquote Q \rpquote & & x \nameeq \lpquote P \rpquote \\
      x & & otherwise \\
    \end{array}
  \right. \nonumber
\end{eqnarray}

and $z$ is chosen distinct from $\quotep{P}$, $\quotep{Q}$, the free
names in $Q$, and all the names in $R$. Our $\alpha$-equivalence will
be built in the standard way from this substitution.

\begin{remark}\label{rem:no_self_referential_names}
  One consequence of these definitions is that $\forall P. \quotep{P}
  \not\in \freenames{P}$.
\end{remark}

\subsection{ Dynamic quote: an example }

Anticipating something of what's to come, consider applying the
substitution, $\widehat{\id{\{}u / z \id{\}}}$, to the following pair
of processes, $\lift{w}{y!(z)}$ and $w[ \lpquote y!(z) \rpquote ]$.

\begin{eqnarray}
	\lift{w}{y!(z)}\widehat{\id{\{}u / z \id{\}}}
		& = &
		\lift{w}{y!(u)} \nonumber\\
	w[ \lpquote y!(z) \rpquote ] \widehat{ \id{\{}u / z \id{\}} }
		& = &
		w[ \lpquote y!(z) \rpquote ] \nonumber
\end{eqnarray}

Because the body of the process between quotes is impervious to
substitution, we get radically different answers. In fact, by
examining the first process in an input context,
e.g. $x?(z).\lift{w}{y!(z)}$, we see that the process under the lift
operator may be shaped by prefixed inputs binding a name inside it. In
this sense, the lift operator will be seen as a way to dynamically
construct processes before reifying them as names.

Finally equipped with these standard features we can present the
dynamics of the calculus.

\subsubsection{Operational semantics} 

Finally, we introduce the computational dynamics. What marks these
algebras as distinct from other more traditionally studied algebraic
structures, e.g. vector spaces or polynomial rings, is the manner in
which dynamics is captured. In traditional structures, dynamics is typically
expressed through morphisms between such structures, as in linear maps
between vector spaces or morphisms between rings. In algebras
associated with the semantics of computation, the dynamics is
expressed as part of the algebraic structure itself, through a
reduction reduction relation typically denoted by $\red$. Below, we
give a recursive presentation of this relation for the calculus used
in the encoding.

$\red \subseteq \pi \times \pi$
$\red : \pi \to \mathcal{P}(\pi)$

\begin{mathpar}
  \inferrule* [lab=Comm] { \textsf{match}( x_{src}, x_{trgt} ) } { x_{trgt}?(y)P \; | \; x_{src}!\langle {Q} \rangle \red P\{\quotep{Q}/y}\} }
  \and \\
  \inferrule* [lab=Par] {{P} \red {P}'} {{{P} | {Q}} \red {{P}' | {Q}}}
  \and
  \inferrule* [lab=Equiv]{{{P} \scong {P}'} \andalso {{P}' \red {Q}'} \andalso {{Q}' \scong {Q}}}{{P} \red {Q}}
\end{mathpar}

\begin{eqnarray*}
  match_{\equiv} (\quotep{P},\quotep{Q}) & := & P \equiv Q \\
  match_{\dagger}(\quotep{P},\quotep{Q}) & := & \forall R. P|Q \red^{*} R => R \red^{*} 0 \\
  match_{K}(\quotep{P},\quotep{Q}) & := & K \mbox{ for some context } K
\end{eqnarray*}

$u?(x)P | u!\langle Q \rangle \red P\{\quotep{Q}/x\}$

%We write $\wred$ for $\red^*$, and $P\red$ if $\exists Q $ such that $ P \red Q$.
We write $P\red$ if $\exists Q $ such that $ P \red Q$ and $P\not\red$, otherwise.

\section{Replication}

As mentioned before, it is known that replication (and hence
recursion) can be implemented in a higher-order process algebra
\cite{SangiorgiWalker}. As our first example of calculation with the
machinery thus far presented we give the construction explicitly in
the {\rhoc}.

\begin{eqnarray}
	D_{x} & := & \prefix{x}{y}{(\binpar{\outputp{x}{y}}{@{y}})} \nonumber\\
	\bangp_{x}{P} & := & \binpar{{x}!\langle{\binpar{D_{x}}{P}}\rangle}{D_{x}} \nonumber
\end{eqnarray}

\begin{eqnarray}
	\bangp_{x}{P} & & \nonumber\\
	=
	& {x}!\langle{(\prefix{x}{y}{(\outputp{x}{y} | @{y})) | P}}\rangle 
	      | \prefix{x}{y}{(\outputp{x}{y} | @{y})} & \nonumber\\
	\red
	& (\outputp{x}{y} | @{y})\substn{\quotep{(\prefix{x}{y}{(@{y} | \outputp{x}{y})) | P}}}{y} & \nonumber\\
	=
	& \outputp{x}{\quotep{(\prefix{x}{y}{(\outputp{x}{y} | @{y})) | P}}}
	  | {(\prefix{x}{y}{(\outputp{x}{y} | @{y})) | P}} & \nonumber\\
	\red
	& \ldots & \nonumber\\
	\red^*
	& P | P | \ldots & \nonumber
\end{eqnarray}

Of course, this encoding, as an implementation, runs away, unfolding
$\bangp{P}$ eagerly. A lazier and more implementable replication
operator, restricted to input-guarded processes, may be obtained as follows.

\begin{eqnarray}
\bangp{\prefix{u}{v}{P}} 
	:= 
	\binpar{\lift{x}{\prefix{u}{v}{(\binpar{D(x)}{P})}}}{D(x)} \nonumber
\end{eqnarray}

\begin{remark}
  Note that the lazier definition still does not deal with summation
  or mixed summation (i.e. sums over input and output). The reader is
  invited to construct definitions of replication that deal with these
  features. 

  Further, the definitions are parameterized in a name, $x$. Can you,
  gentle reader, make a definition that eliminates this parameter and
  guarantees no accidental interaction between the replication
  machinery and the process being replicated -- i.e. no accidental
  sharing of names used by the process to get its work done and the
  name(s) used by the replication to effect copying. This latter
  revision of the definition of replication is crucial to obtaining
  the expected identity $!!P \sim !P$.
\end{remark}

\begin{remark}\label{rem:paradoxical_combinator}
  The reader familiar with the lambda calculus will have noticed the
  similarity between $D$ and the paradoxical combinator.

  [Ed. note: the existence of this seems to suggest we have to be more
  restrictive on the set of processes and names we admit if we are to
  support no-cloning.]
\end{remark}

\subsubsection{Bisimulation}

The computational dynamics gives rise to another kind of equivalence,
the equivalence of computational behavior. As previously mentioned
this is typically captured \emph{via} some form of bisimulation.

% The notion we use in this paper is weak barbed bisimulation
% \cite{milner91polyadicpi}.

The notion we use in this paper is derived from weak barbed
bisimulation \cite{milner91polyadicpi}. 

\begin{definition}
An \emph{observation relation}, $\downarrow_{\mathcal N}$, over a set
of names, $\mathcal N$, is the smallest relation satisfying the rules
below.

\infrule[Out-barb]{y \in {\mathcal N}, \; x \nameeq y}
		  {\outputp{x}{v} \downarrow_{\mathcal N} x}
\infrule[Par-barb]{\mbox{$P\downarrow_{\mathcal N} x$ or $Q\downarrow_{\mathcal N} x$}}
		  {\binpar{P}{Q} \downarrow_{\mathcal N} x}

We write $P \Downarrow_{\mathcal N} x$ if there is $Q$ such that 
$P \wred Q$ and $Q \downarrow_{\mathcal N} x$.
\end{definition}

\begin{definition}
%\label{def.bbisim}
An  ${\mathcal N}$-\emph{barbed bisimulation} over a set of names, ${\mathcal N}$, is a symmetric binary relation 
${\mathcal S}_{\mathcal N}$ between agents such that $P\rel{S}_{\mathcal N}Q$ implies:
\begin{enumerate}
\item If $P \red P'$ then $Q \wred Q'$ and $P'\rel{S}_{\mathcal N} Q'$.
\item If $P\downarrow_{\mathcal N} x$, then $Q\Downarrow_{\mathcal N} x$.
\end{enumerate}
$P$ is ${\mathcal N}$-barbed bisimilar to $Q$, written
$P \wbbisim_{\mathcal N} Q$, if $P \rel{S}_{\mathcal N} Q$ for some ${\mathcal N}$-barbed bisimulation ${\mathcal S}_{\mathcal N}$.
\end{definition}

$\mathcal{R} \subseteq \pi \times \pi$

$P \mathcal{R} Q => \forall P'. P \red P' \Rightarrow \exists Q'. Q \red Q', P' \mathcal{R} Q'$

$P \vdash x \Rightarrow Q \vdash x$

\begin{mathpar}
  \inferrule*[lab=Out-barb]{x \nameeq y}{{y}!\langle{Q}\rangle \vdash x}
  \and
  \inferrule*[lab=Par-barb]{\mbox{$P\vdash x$ or $Q\vdash x$}}{\binpar{P}{Q} \vdash x}
\end{mathpar}

\subsubsection{Contexts}

One of the principle advantages of computational calculi like the
$\pi$-calculus is a well-defined notion of context,
contextual-equivalence and a correlation between
contextual-equivalence and notions of bisimulation. The notion of
context allows the decomposition of a process into (sub-)process and
its syntactic environment, its context. Thus, a context may be
thought of as a process with a ``hole'' (written $\Box$) in it. The
application of a context $M$ to a process $P$, written $M[P]$, is
tantamount to filling the hole in $M$ with $P$. In this paper we do
not need the full weight of this theory, but do make use of the notion
of context in the proof the main theorem. 

\begin{mathpar}
  \inferrule* [lab=summation] {} {{M_{M},M_{N}} \bc \Box \;|\; x.M_{A} \;|\; M_{M}+M_{N}}
  \and
  \inferrule* [lab=agent] {} {{M_{A}} \bc (\vec{x})M_{P} \;| \; \clift{P_0,\ldots,M_{P},\ldots,P_N}}
  \and \\
  \inferrule* [lab=process] {} {{M_{P}} \bc M_{N} \;| \;P|M_{P} }
\end{mathpar} 

\begin{mathpar}
  \inferrule* [lab=sychronization] {} {M_{N} \bc \Box \;|\; x?M_{F} \;|\; x!M_{C}}
  \and
  \inferrule* [lab=abstraction] {} {{M_{F}} \bc (x)M_{P} }
  \and
  \inferrule* [lab=concretion] {} {{M_{C}} \bc \langle M_{P} \rangle }
  \and \\
  \inferrule* [lab=process] {} {{M_{P}} \bc M_{N} \;| \;P|M_{P} }
\end{mathpar}

\begin{definition}[contextual application] Given a context $M$, and
  process $P$, we define the \emph{contextual application}, $M[P] :=
  M\{P/\Box\}$. That is, the contextual application of M to P is the
  substitution of $P$ for $\Box$ in $M$.
\end{definition}

$\meaningof{-} : L \to \mathcal{P}(\pi)$

\begin{mathpar}
  \inferrule* [lab=collection] {} {\meaningof{true} = \pi, \and \meaningof{~E} = \pi \setminus \meaningof{E}, \and \meaningof{E_{1} \& E_{2}} = \meaningof{E_{1}} \cap \meaningof{E_{2}}}
\end{mathpar}

\begin{mathpar}
  \inferrule* [lab=structure] {} {\meaningof{0} = \{ P \in \pi | P \equiv 0 \}, \and \\ \meaningof{E_1 | E_2} = \{ P \in \pi | P \equiv P_{1} | P_{2}, P_{1} \in \meaningof{E_{1}}, P_{2} \in \meaningof{E_2}\} }
\end{mathpar}

\begin{mathpar}
 \inferrule* [lab=behavior] {} {\meaningof{\langle a?b \rangle E} = \{ P \in \pi | P \equiv Q | u?(y)P', \\ \and \\\\ \and \\ \;\;\; u \in \meaningof{a}, \forall z.P'\{z/y\} \in \meaningof{E\{z/b\}}\}, \and \\ \meaningof{a!E} = \{ P \in \pi | P \equiv Q | x!\langle P' \rangle, x \in \meaningof{a} P' \in \meaningof{E}\} }
\end{mathpar}

\begin{mathpar}
 \inferrule* [lab=nominal] {} {\meaningof{\quotep{E}} = \{ \quotep{P} \in \quotep{\pi} | P \in \meaningof{E} \}, \and \meaningof{\quotep{P}} = \{ \quotep{Q} \in \quotep{\pi} | P \equiv Q \} \and \\ \meaningof{@\quotep{E}} = \{ P \in \pi | P \equiv @x, x \in \meaningof{E} \}}
\end{mathpar}

\begin{eqnarray*}
  \\
  \meaningof{-} : TS \to ST
\end{eqnarray*}

\begin{eqnarray*}
  \\
  L : TS \to ST
\end{eqnarray*}

\begin{eqnarray*}
  \\
  P \models E \iff P \in \meaningof{E}
\end{eqnarray*}

\begin{eqnarray*}
  P \approx_{L} Q \iff \forall E \in L. P \models E \iff Q \models E
\end{eqnarray*}

\begin{eqnarray*}
  P \approx_{K} Q
\end{eqnarray*}

\begin{eqnarray*}
  P \approx Q
\end{eqnarray*}

$\approx_{K} = \approx = \approx_{L}$

\subsubsection{Contextual duality}

Note that contexts extend the quotation operation to a family of
operations from processes to names. Given a context, $M$, we can
define a \emph{nominal context}, $\quotep{M}$ by $\quotep{M}[P] :=
\quotep{M[P]}$. To foreshadow what is to come we observe that these
operations enjoy a duality with processes very much like the duality
between vectors and maps from vectors to scalars.

Further, because the calculus is essentially higher-order, we have a
correspondence between contexts and processes. More specifically,
given a name $x$ and a context $M$ we can construct $M^{*}_{x}$ such
that 

\begin{mathpar}
  M^{*}_{x} | \lift{x}{P} \red M[P]
\end{mathpar}

namely,

\begin{mathpar}
  M^{*}_{x} := x?(u).M[\dropn{u}]
\end{mathpar}

The dependence of $M^{*}_{x}$ on a name makes it an abstraction, 

\begin{mathpar}
  M^{*} := (x)x?(u).M[\dropn{u}]
\end{mathpar}

\subsection{Additional notation}

It will sometimes be convenient to denote the process a name
quotes. We already have the notation $x = \quotep{P}$, but it will be
convenient to introduce an alternate notation, $\procn{x}$, when we
want to emphasize the connection to the use of the name. Note that, by
virtue of name equivalence, $\quotep{\procn{x}} \nameeq x$; so, the
notation is consistent with previous definitions.

Further, because names have structure it is possible to effect
substitutions on the basis of that structure. This means we need to
upgrade our notation for substitutions, which we accomplish by
adapting comprehension notation. Thus,

\begin{mathpar}
  P\{ y / x : x \in S \}
\end{mathpar}

is interpreted to mean the process derived from P by replacing (in a
capture-avoiding manner) each occurrence of $x$ in $S$ by $y$. For example,

\begin{mathpar}
  P\{ \quotep{\procn{x}|\procn{x}} / x : x \in \freenames{P} \}
\end{mathpar}

will replace each (occurrence) of a free name $x$ in $P$ by
$\quotep{\procn{x}|\procn{x}}$.

Also, we will avail ourselves of the notation $x^{L}$ and $x^{R}$ to
denote injections of a name into disjoint copies of the name
space. There are numerous ways to accomplish this. One example can be
found in \cite{MeredithR05}. This notation overloads to vectors of
names: $\vec{x}^{\pi} := (x_{i}^{\pi} \; : \; 0 \leq i < |\vec{x}| )$ where $\pi \in \{L,R\}$.

We also use $P^{\Box} := P|\Box$.

In \cite{MeredithR05} an interpretation of the new operator is
given. It turns out that there are several possible interpretations
all enjoying the requisite algebraic properties of the operator (see
\cite{milner91polyadicpi}). We will therefore make liberal use of
$(\nu\; \vec{x})P$.

% subsection the_syntax_and_semantics_of_the_notation_system (end)   

\input{qm2pi.qmops} 

\input{qm2pi.sterngerlach} 

\input{qm2pi.metric} 

% section concurrent_process_calculi (end)

%\input{qm2pi.proofsketch}

% section proof sketch (end)

%\input{qm2pi.slviaknots} 

% section spatial logic via knots (end)

\input{qm2pi.conclusion}

% section conclusion (end)

%\input{qm2pi.dtcodes} 

% section wiring algorithm (end)

\input{qm2pi.ack} 

% section acknowledgments (end)

\newpage


\bibliographystyle{plain}   
\bibliography{../../biblios/main.bib}

\input{qm2pi.rhodetails}

\end{document}

 

% subsection basic_interpretation (end)

%\input{qm2pi.rho.presentation} 
\subsection{The syntax and semantics of the notation system}\label{sub:the_syntax_and_semantics_of_the_notation_system} % (fold)

We now summarize a technical presentation of the calculus that
embodies our theory of dynamics. The typical presentation of such a
calculus follows the style of giving generators and relations on
them. The grammar, below, describing term constructors, freely
generates the set of processes, $\Proc$. This set is then quotiented
by a relation known as structural congruence and it is over this set
that the notion of dynamics is expressed. This presentation is
essentially that of \cite{MeredithR05} with the addition of
polyadicity and summation. For readability we have relegated some of
the technical subtleties to an appendix.

\subsubsection{Process grammar}\label{subsub:process_grammar}

\begin{mathpar}
  \inferrule* [lab=synchronization] {} {{M} \bc \pzero \;|\; x?F \;|\; x!C }
  \and
  \inferrule* [lab=abstraction] {} {{F} \bc (x)P}
  \and
  \inferrule* [lab=concretion] {} {{C} \bc \langle Q \rangle}
  \and
  \inferrule* [lab=process] {} {{P,Q} \bc M \;| \;P|Q \;|\; @{x}}
  \and
  \inferrule* [lab=name] {} {{x} \bc \quotep{P}}
\end{mathpar} 

Note that $\vec{x}$ (resp. $\vec{P}$) denotes a vector of names
(resp. processes) of length $|\vec{x}|$ (resp. $|\vec{P}|$). We adopt
the following useful abbreviations.

\begin{mathpar}
   x?(\vec{y}).P := x.(\vec{y})P \and  x\clift{\vec{P}} := x.\clift{\vec{P}}
   \and x!(y) := \lift{x}{\dropn{y}}
   \and \Pi_{i=0}^{n-1}P_i := P_0 | \ldots | P_{n-1}
\end{mathpar}

\subsubsection{Structural congruence}

\paragraph{Free and bound names and alpha-equivalence.} At the
core of structural equivalence is alpha-equivalence which identifies
process that are the same up to a change of variable. Formally, we
recognize the distinction between free and bound names. The free names
of a process, $\freenames{P}$, may be calculated recursively as
follows:

\begin{mathpar}
\freenames{\pzero} := \emptyset
  \and \\
  \freenames{x?(y).P} := \{ x \} \cup (\freenames{P} \setminus \{ y \})
  \and 
  \freenames{x!\langle P \rangle} := \{ x \} \cup \{ P \} 
  \and \\
  \freenames{P|Q} := \freenames{P} \cup \freenames{Q}
  \and \\
  \freenames{@{x}} := \{ x \}
\end{mathpar}

$\pi$
$\quotep{\pi}$

$\freenames{-} : \pi \to \mathcal{P}(\quotep{\pi})$

\begin{eqnarray*}
  \freenames{\pzero} & := & \emptyset \\
  \freenames{x?(y).P} & := & \{ x \} \cup (\freenames{P} \setminus \{ y \}) \\
  \freenames{x!\langle P \rangle} & := & \{ x \} \cup \{ P \} \\
  \freenames{P|Q} & := & \freenames{P} \cup \freenames{Q} \\
  \freenames{\dropn{x}} & := & \{ x \}
\end{eqnarray*}

The bound names of a process, $\boundnames{P}$, are those names occurring in $P$
that are not free. For example, in $x?(y).0$, the name $x$ is free, while $y$ is bound.

\begin{mathpar}
  \inferrule* [lab=monoidal-laws] {} { P|Q \equiv Q|P \and P|0 \equiv P \and P|(Q|R) \equiv (P|Q)|R }
\end{mathpar}

\begin{mathpar}
  \inferrule* [lab=alpha-equivalence] {} { (x)P \equiv (y)P\{y/x\} \and y \not\in \freenames{P} }
\end{mathpar}

\begin{definition}
Then two processes, $P,Q$, are alpha-equivalent if $P = Q\{\vec{y}/\vec{x}\}$ for
some $\vec{x} \in \boundnames{Q},\vec{y} \in \boundnames{P}$, where $Q\{\vec{y}/\vec{x}\}$
denotes the capture-avoiding substitution of $\vec{y}$ for $\vec{x}$ in $Q$.
\end{definition}

\begin{definition}
  The {\em structural congruence} \cite{SangiorgiWalker} , $\equiv$,
  between processes is the least congruence containing
  alpha-equivalence, satisfying the abelian monoid laws
  (associativity, commutativity and $\pzero$ as identity) for parallel
  composition $|$ and for summation $+$.
\end{definition}

\subsection{Name equivalence}

We take name equivalence, written $\nameeq$, to be the smallest
equivalence relation generated by the following rules.

\begin{mathpar}
\inferrule*[lab=Quote-drop]
{ }
{ \quotep{@{x}} \nameeq x }

\inferrule*[lab=Struct-equiv]
{ P \scong Q }
{ \quotep{P} \nameeq \quotep{Q} }
\end{mathpar}

The astute reader will have noticed that the mutual recursion of names
and processes imposes a mutual recursion on alpha-equivalence and
structural equivalence via name-equivalence. Fortunately, all of this
works out pleasantly and we may calculate in the natural way, free of
concern. The reader interested in the details is referred to the
appendix \ref{appendix:rho_details}.

\subsection{Substitution}

We use $\Proc$ for the set of processes, $\QProc$ for the set of
names, and $\id{\{}\vec{y} / \vec{x} \id{\}}$ to denote partial maps,
$s : \QProc \rightarrow \QProc$. A map, $s$ lifts, uniquely, to a map
on process terms, $\widehat{s} : \Proc \rightarrow \Proc$ by the
following equations.

\begin{mathpar}
  (0) \psubstp{Q}{P} := 0 \\
  (R \juxtap S) \psubstp{Q}{P}
  :=    
  (R)\psubstp{Q}{P} \juxtap (S) \psubstp{Q}{P} \\
  (x?(y).R) \psubstp{Q}{P}    
  :=    
  (x)\substp{Q}{P} (z)\concat( (R \psubstn{z}{y}) \psubstp{Q}{P} ) \\
  (\lift{x}{R}) \psubstp{Q}{P}  
  :=
  \lift{(x)\substp{Q}{P}}{ R \psubstp{Q}{P} } \\
%   (\dropn{x})  \psubstp{Q}{P}       
%   := 
%   \left\{ 
%     \begin{array}{ccc} 
%       \dropn{\quotep{Q}} & & x \nameeq \quotep{P} \\
%       \dropn{x} & & otherwise \\
%     \end{array}
%   \right. 
  (\dropn{x})  \psubstp{Q}{P}       
  := 
  \left\{ 
    \begin{array}{ccc} 
      Q & & x \nameeq \quotep{P} \\
      \dropn{x} & & otherwise \\
    \end{array}
  \right.
\end{mathpar}
 

where

\begin{eqnarray}
  (x)\id{\{} \lpquote Q \rpquote / \lpquote P \rpquote \id{\}}            = 
  \left\{ 
    \begin{array}{ccc}
      \lpquote Q \rpquote & & x \nameeq \lpquote P \rpquote \\
      x & & otherwise \\
    \end{array}
  \right. \nonumber
\end{eqnarray}

and $z$ is chosen distinct from $\quotep{P}$, $\quotep{Q}$, the free
names in $Q$, and all the names in $R$. Our $\alpha$-equivalence will
be built in the standard way from this substitution.

\begin{remark}\label{rem:no_self_referential_names}
  One consequence of these definitions is that $\forall P. \quotep{P}
  \not\in \freenames{P}$.
\end{remark}

\subsection{ Dynamic quote: an example }

Anticipating something of what's to come, consider applying the
substitution, $\widehat{\id{\{}u / z \id{\}}}$, to the following pair
of processes, $\lift{w}{y!(z)}$ and $w[ \lpquote y!(z) \rpquote ]$.

\begin{eqnarray}
	\lift{w}{y!(z)}\widehat{\id{\{}u / z \id{\}}}
		& = &
		\lift{w}{y!(u)} \nonumber\\
	w[ \lpquote y!(z) \rpquote ] \widehat{ \id{\{}u / z \id{\}} }
		& = &
		w[ \lpquote y!(z) \rpquote ] \nonumber
\end{eqnarray}

Because the body of the process between quotes is impervious to
substitution, we get radically different answers. In fact, by
examining the first process in an input context,
e.g. $x?(z).\lift{w}{y!(z)}$, we see that the process under the lift
operator may be shaped by prefixed inputs binding a name inside it. In
this sense, the lift operator will be seen as a way to dynamically
construct processes before reifying them as names.

Finally equipped with these standard features we can present the
dynamics of the calculus.

\subsubsection{Operational semantics} 

Finally, we introduce the computational dynamics. What marks these
algebras as distinct from other more traditionally studied algebraic
structures, e.g. vector spaces or polynomial rings, is the manner in
which dynamics is captured. In traditional structures, dynamics is typically
expressed through morphisms between such structures, as in linear maps
between vector spaces or morphisms between rings. In algebras
associated with the semantics of computation, the dynamics is
expressed as part of the algebraic structure itself, through a
reduction reduction relation typically denoted by $\red$. Below, we
give a recursive presentation of this relation for the calculus used
in the encoding.

$\red \subseteq \pi \times \pi$
$\red : \pi \to \mathcal{P}(\pi)$

\begin{mathpar}
  \inferrule* [lab=Comm] { \textsf{match}( x_{src}, x_{trgt} ) } { x_{trgt}?(y)P \; | \; x_{src}!\langle {Q} \rangle \red P\{\quotep{Q}/y}\} }
  \and \\
  \inferrule* [lab=Par] {{P} \red {P}'} {{{P} | {Q}} \red {{P}' | {Q}}}
  \and
  \inferrule* [lab=Equiv]{{{P} \scong {P}'} \andalso {{P}' \red {Q}'} \andalso {{Q}' \scong {Q}}}{{P} \red {Q}}
\end{mathpar}

\begin{eqnarray*}
  match_{\equiv} (\quotep{P},\quotep{Q}) & := & P \equiv Q \\
  match_{\dagger}(\quotep{P},\quotep{Q}) & := & \forall R. P|Q \red^{*} R => R \red^{*} 0 \\
  match_{K}(\quotep{P},\quotep{Q}) & := & K \mbox{ for some context } K
\end{eqnarray*}

$u?(x)P | u!\langle Q \rangle \red P\{\quotep{Q}/x\}$

%We write $\wred$ for $\red^*$, and $P\red$ if $\exists Q $ such that $ P \red Q$.
We write $P\red$ if $\exists Q $ such that $ P \red Q$ and $P\not\red$, otherwise.

\section{Replication}

As mentioned before, it is known that replication (and hence
recursion) can be implemented in a higher-order process algebra
\cite{SangiorgiWalker}. As our first example of calculation with the
machinery thus far presented we give the construction explicitly in
the {\rhoc}.

\begin{eqnarray}
	D_{x} & := & \prefix{x}{y}{(\binpar{\outputp{x}{y}}{@{y}})} \nonumber\\
	\bangp_{x}{P} & := & \binpar{{x}!\langle{\binpar{D_{x}}{P}}\rangle}{D_{x}} \nonumber
\end{eqnarray}

\begin{eqnarray}
	\bangp_{x}{P} & & \nonumber\\
	=
	& {x}!\langle{(\prefix{x}{y}{(\outputp{x}{y} | @{y})) | P}}\rangle 
	      | \prefix{x}{y}{(\outputp{x}{y} | @{y})} & \nonumber\\
	\red
	& (\outputp{x}{y} | @{y})\substn{\quotep{(\prefix{x}{y}{(@{y} | \outputp{x}{y})) | P}}}{y} & \nonumber\\
	=
	& \outputp{x}{\quotep{(\prefix{x}{y}{(\outputp{x}{y} | @{y})) | P}}}
	  | {(\prefix{x}{y}{(\outputp{x}{y} | @{y})) | P}} & \nonumber\\
	\red
	& \ldots & \nonumber\\
	\red^*
	& P | P | \ldots & \nonumber
\end{eqnarray}

Of course, this encoding, as an implementation, runs away, unfolding
$\bangp{P}$ eagerly. A lazier and more implementable replication
operator, restricted to input-guarded processes, may be obtained as follows.

\begin{eqnarray}
\bangp{\prefix{u}{v}{P}} 
	:= 
	\binpar{\lift{x}{\prefix{u}{v}{(\binpar{D(x)}{P})}}}{D(x)} \nonumber
\end{eqnarray}

\begin{remark}
  Note that the lazier definition still does not deal with summation
  or mixed summation (i.e. sums over input and output). The reader is
  invited to construct definitions of replication that deal with these
  features. 

  Further, the definitions are parameterized in a name, $x$. Can you,
  gentle reader, make a definition that eliminates this parameter and
  guarantees no accidental interaction between the replication
  machinery and the process being replicated -- i.e. no accidental
  sharing of names used by the process to get its work done and the
  name(s) used by the replication to effect copying. This latter
  revision of the definition of replication is crucial to obtaining
  the expected identity $!!P \sim !P$.
\end{remark}

\begin{remark}\label{rem:paradoxical_combinator}
  The reader familiar with the lambda calculus will have noticed the
  similarity between $D$ and the paradoxical combinator.

  [Ed. note: the existence of this seems to suggest we have to be more
  restrictive on the set of processes and names we admit if we are to
  support no-cloning.]
\end{remark}

\subsubsection{Bisimulation}

The computational dynamics gives rise to another kind of equivalence,
the equivalence of computational behavior. As previously mentioned
this is typically captured \emph{via} some form of bisimulation.

% The notion we use in this paper is weak barbed bisimulation
% \cite{milner91polyadicpi}.

The notion we use in this paper is derived from weak barbed
bisimulation \cite{milner91polyadicpi}. 

\begin{definition}
An \emph{observation relation}, $\downarrow_{\mathcal N}$, over a set
of names, $\mathcal N$, is the smallest relation satisfying the rules
below.

\infrule[Out-barb]{y \in {\mathcal N}, \; x \nameeq y}
		  {\outputp{x}{v} \downarrow_{\mathcal N} x}
\infrule[Par-barb]{\mbox{$P\downarrow_{\mathcal N} x$ or $Q\downarrow_{\mathcal N} x$}}
		  {\binpar{P}{Q} \downarrow_{\mathcal N} x}

We write $P \Downarrow_{\mathcal N} x$ if there is $Q$ such that 
$P \wred Q$ and $Q \downarrow_{\mathcal N} x$.
\end{definition}

\begin{definition}
%\label{def.bbisim}
An  ${\mathcal N}$-\emph{barbed bisimulation} over a set of names, ${\mathcal N}$, is a symmetric binary relation 
${\mathcal S}_{\mathcal N}$ between agents such that $P\rel{S}_{\mathcal N}Q$ implies:
\begin{enumerate}
\item If $P \red P'$ then $Q \wred Q'$ and $P'\rel{S}_{\mathcal N} Q'$.
\item If $P\downarrow_{\mathcal N} x$, then $Q\Downarrow_{\mathcal N} x$.
\end{enumerate}
$P$ is ${\mathcal N}$-barbed bisimilar to $Q$, written
$P \wbbisim_{\mathcal N} Q$, if $P \rel{S}_{\mathcal N} Q$ for some ${\mathcal N}$-barbed bisimulation ${\mathcal S}_{\mathcal N}$.
\end{definition}

$\mathcal{R} \subseteq \pi \times \pi$

$P \mathcal{R} Q => \forall P'. P \red P' \Rightarrow \exists Q'. Q \red Q', P' \mathcal{R} Q'$

$P \vdash x \Rightarrow Q \vdash x$

\begin{mathpar}
  \inferrule*[lab=Out-barb]{x \nameeq y}{{y}!\langle{Q}\rangle \vdash x}
  \and
  \inferrule*[lab=Par-barb]{\mbox{$P\vdash x$ or $Q\vdash x$}}{\binpar{P}{Q} \vdash x}
\end{mathpar}

\subsubsection{Contexts}

One of the principle advantages of computational calculi like the
$\pi$-calculus is a well-defined notion of context,
contextual-equivalence and a correlation between
contextual-equivalence and notions of bisimulation. The notion of
context allows the decomposition of a process into (sub-)process and
its syntactic environment, its context. Thus, a context may be
thought of as a process with a ``hole'' (written $\Box$) in it. The
application of a context $M$ to a process $P$, written $M[P]$, is
tantamount to filling the hole in $M$ with $P$. In this paper we do
not need the full weight of this theory, but do make use of the notion
of context in the proof the main theorem. 

\begin{mathpar}
  \inferrule* [lab=summation] {} {{M_{M},M_{N}} \bc \Box \;|\; x.M_{A} \;|\; M_{M}+M_{N}}
  \and
  \inferrule* [lab=agent] {} {{M_{A}} \bc (\vec{x})M_{P} \;| \; \clift{P_0,\ldots,M_{P},\ldots,P_N}}
  \and \\
  \inferrule* [lab=process] {} {{M_{P}} \bc M_{N} \;| \;P|M_{P} }
\end{mathpar} 

\begin{mathpar}
  \inferrule* [lab=sychronization] {} {M_{N} \bc \Box \;|\; x?M_{F} \;|\; x!M_{C}}
  \and
  \inferrule* [lab=abstraction] {} {{M_{F}} \bc (x)M_{P} }
  \and
  \inferrule* [lab=concretion] {} {{M_{C}} \bc \langle M_{P} \rangle }
  \and \\
  \inferrule* [lab=process] {} {{M_{P}} \bc M_{N} \;| \;P|M_{P} }
\end{mathpar}

\begin{definition}[contextual application] Given a context $M$, and
  process $P$, we define the \emph{contextual application}, $M[P] :=
  M\{P/\Box\}$. That is, the contextual application of M to P is the
  substitution of $P$ for $\Box$ in $M$.
\end{definition}

$\meaningof{-} : L \to \mathcal{P}(\pi)$

\begin{mathpar}
  \inferrule* [lab=collection] {} {\meaningof{true} = \pi, \and \meaningof{~E} = \pi \setminus \meaningof{E}, \and \meaningof{E_{1} \& E_{2}} = \meaningof{E_{1}} \cap \meaningof{E_{2}}}
\end{mathpar}

\begin{mathpar}
  \inferrule* [lab=structure] {} {\meaningof{0} = \{ P \in \pi | P \equiv 0 \}, \and \\ \meaningof{E_1 | E_2} = \{ P \in \pi | P \equiv P_{1} | P_{2}, P_{1} \in \meaningof{E_{1}}, P_{2} \in \meaningof{E_2}\} }
\end{mathpar}

\begin{mathpar}
 \inferrule* [lab=behavior] {} {\meaningof{\langle a?b \rangle E} = \{ P \in \pi | P \equiv Q | u?(y)P', \\ \and \\\\ \and \\ \;\;\; u \in \meaningof{a}, \forall z.P'\{z/y\} \in \meaningof{E\{z/b\}}\}, \and \\ \meaningof{a!E} = \{ P \in \pi | P \equiv Q | x!\langle P' \rangle, x \in \meaningof{a} P' \in \meaningof{E}\} }
\end{mathpar}

\begin{mathpar}
 \inferrule* [lab=nominal] {} {\meaningof{\quotep{E}} = \{ \quotep{P} \in \quotep{\pi} | P \in \meaningof{E} \}, \and \meaningof{\quotep{P}} = \{ \quotep{Q} \in \quotep{\pi} | P \equiv Q \} \and \\ \meaningof{@\quotep{E}} = \{ P \in \pi | P \equiv @x, x \in \meaningof{E} \}}
\end{mathpar}

\begin{eqnarray*}
  \\
  \meaningof{-} : TS \to ST
\end{eqnarray*}

\begin{eqnarray*}
  \\
  L : TS \to ST
\end{eqnarray*}

\begin{eqnarray*}
  \\
  P \models E \iff P \in \meaningof{E}
\end{eqnarray*}

\begin{eqnarray*}
  P \approx_{L} Q \iff \forall E \in L. P \models E \iff Q \models E
\end{eqnarray*}

\begin{eqnarray*}
  P \approx_{K} Q
\end{eqnarray*}

\begin{eqnarray*}
  P \approx Q
\end{eqnarray*}

$\approx_{K} = \approx = \approx_{L}$

\subsubsection{Contextual duality}

Note that contexts extend the quotation operation to a family of
operations from processes to names. Given a context, $M$, we can
define a \emph{nominal context}, $\quotep{M}$ by $\quotep{M}[P] :=
\quotep{M[P]}$. To foreshadow what is to come we observe that these
operations enjoy a duality with processes very much like the duality
between vectors and maps from vectors to scalars.

Further, because the calculus is essentially higher-order, we have a
correspondence between contexts and processes. More specifically,
given a name $x$ and a context $M$ we can construct $M^{*}_{x}$ such
that 

\begin{mathpar}
  M^{*}_{x} | \lift{x}{P} \red M[P]
\end{mathpar}

namely,

\begin{mathpar}
  M^{*}_{x} := x?(u).M[\dropn{u}]
\end{mathpar}

The dependence of $M^{*}_{x}$ on a name makes it an abstraction, 

\begin{mathpar}
  M^{*} := (x)x?(u).M[\dropn{u}]
\end{mathpar}

\subsection{Additional notation}

It will sometimes be convenient to denote the process a name
quotes. We already have the notation $x = \quotep{P}$, but it will be
convenient to introduce an alternate notation, $\procn{x}$, when we
want to emphasize the connection to the use of the name. Note that, by
virtue of name equivalence, $\quotep{\procn{x}} \nameeq x$; so, the
notation is consistent with previous definitions.

Further, because names have structure it is possible to effect
substitutions on the basis of that structure. This means we need to
upgrade our notation for substitutions, which we accomplish by
adapting comprehension notation. Thus,

\begin{mathpar}
  P\{ y / x : x \in S \}
\end{mathpar}

is interpreted to mean the process derived from P by replacing (in a
capture-avoiding manner) each occurrence of $x$ in $S$ by $y$. For example,

\begin{mathpar}
  P\{ \quotep{\procn{x}|\procn{x}} / x : x \in \freenames{P} \}
\end{mathpar}

will replace each (occurrence) of a free name $x$ in $P$ by
$\quotep{\procn{x}|\procn{x}}$.

Also, we will avail ourselves of the notation $x^{L}$ and $x^{R}$ to
denote injections of a name into disjoint copies of the name
space. There are numerous ways to accomplish this. One example can be
found in \cite{MeredithR05}. This notation overloads to vectors of
names: $\vec{x}^{\pi} := (x_{i}^{\pi} \; : \; 0 \leq i < |\vec{x}| )$ where $\pi \in \{L,R\}$.

We also use $P^{\Box} := P|\Box$.

In \cite{MeredithR05} an interpretation of the new operator is
given. It turns out that there are several possible interpretations
all enjoying the requisite algebraic properties of the operator (see
\cite{milner91polyadicpi}). We will therefore make liberal use of
$(\nu\; \vec{x})P$.

% subsection the_syntax_and_semantics_of_the_notation_system (end)   

\section{Interpretation of QM}
\subsection{Supporting definitions}
\subsubsection{Multiplication}
\begin{mathpar}
  \quotep{Q} \cdot \quotep{R} := \quotep{Q|R}
  \and \\
  \quotep{Q} \cdot P := P\{ \quotep{Q|R} / \quotep{R} : \quotep{R} \in \freenames{P} \}
\end{mathpar}

\paragraph{Discussion}
The first line needs little explanation. The second line says that
each free name of the process is replaced with the multiplication of
that name by the scalar. Multiplication of a scalar (name) by a state
(process) results in a process all the names of which have been `moved
over' by parallel composition with the process the scalar
quotes. There is a subtlety that the bound names have to be
manipulated so that multiplied names aren't accidentally
captured. There are many ways to achieve this.

\begin{remark}\label{rem:multiplication_identities}
  The reader is invited to verify that for all $x,y,z \in \QProc$ and $P \in \Proc$
  \begin{mathpar}
    x \cdot \quotep{0} \equiv x 
    \and
    x \cdot y \equiv y \cdot x
    \and
    x \cdot (y \cdot z) \equiv (x \cdot y) \cdot z
    \and \\
    \quotep{0} \cdot P \equiv P
    \and \\
    x \cdot (y \cdot P) \equiv (x \cdot y) \cdot P
    \and \\
    x \cdot (P|Q) \equiv (x \cdot P) | (x \cdot Q)
    \and \\    
  \end{mathpar}
\end{remark}

\subsubsection{Tensor product}

We define a tensor product on processes by structural induction.

\paragraph{Tensor of sums} First note that all summations, including
$\pzero$ and sequence, can be written $\Sigma_{i} x_{i}.A_{i} +
\Sigma_{j} x_{j}.C_{j}$, where we have grouped input-guarded processes
together and output-guarded processes together.

Thus, we can define the tensor product of two summations, $N_{1}\otimes N_{2}$, where

\begin{mathpar}
  N_{1} := \Sigma_{i} x_{i}.A_{i} + \Sigma_{j} x_{j}.C_{j}
  \and
  N_{2} := \Sigma_{i'} y_{i'}.B_{i'} + \Sigma_{j'} y_{j'}.D_{j'} 
\end{mathpar}

as follows.

\begin{mathpar}
  \Sigma_{i} x_{i}.A_{i} + \Sigma_{j} x_{j}.C_{j} \otimes \Sigma_{i'}
  y_{i'}.B_{i'} + \Sigma_{j'} y_{j'}.D_{j'} 
  \and \\
  := \; \Sigma_{i} \Sigma_{i'} \quotep{\stackrel{\vee}{x_{i}}| \stackrel{\vee}{y_{i'}}}.(A_{i}\otimes B_{i'}) \; | \; \Sigma_{i'} \Sigma_{i} \quotep{\stackrel{\vee}{y_{i'}}|\stackrel{\vee}{x_{i}}}.(B_{i'}\otimes A_{i})
  \and
  \;\; | \;\; \Sigma_{j} \Sigma_{j'} \quotep{\stackrel{\vee}{x_{j}}|\stackrel{\vee}{y_{j'}}}.(A_{j}\otimes B_{j'}) \; | \; \Sigma_{j'} \Sigma_{j} \quotep{\stackrel{\vee}{y_{j'}}|\stackrel{\vee}{x_{j}}}.(B_{j'}\otimes A_{j})
\end{mathpar}

\begin{remark}
  Do we need to $x^{L}$ and $y^{R}$ for this construction as well?
\end{remark}

\paragraph{Tensor of parallel compositions} Next, we distribute tensor
over par.

\begin{mathpar}
  P_{1}|P_{2} \otimes Q_{1}|Q_{2} := (P_{1} \otimes Q_{1}) | (P_{1}
  \otimes Q_{2}) | (P_{2} \otimes Q_{1}) | (P_{2} \otimes Q_{2})
\end{mathpar}

\paragraph{Tensor with dropped names} We treat tensor of a
process with a dropped name as parallel composition.

\begin{mathpar}
  P \otimes \dropn{x} := P | \dropn{x}
\end{mathpar}

\paragraph{Tensor of agents}

Finally, we need to define tensor on agents. Note that the definition
of tensor on normal products only tensors inputs with inputs and
outputs with outputs. Thus, we only have to define the operation on
``homogeneous'' pairings.

\begin{mathpar}
  (\vec{x})P \otimes (\vec{y})Q
  \and \\
  := (x_{0}^{L}|y_{0}^{R},\ldots,x_{0}^{L}|y_{n}^{R},\ldots,x_{m}^{L}|y_{0}^{R},\ldots,x_{m}^{L}|y_{n}^R)(P\{ \vec{x}^{L}/\vec{x}\} \otimes Q \{ \vec{y}^{R}/\vec{y}\})
  \and \\
  \clift{\vec{P}} \otimes \clift{\vec{Q}}
  \and \\
  := \clift{P_{0}\otimes Q_{0},\ldots,P_{0}\otimes Q_{n},\ldots,P_{m}\otimes Q_{0},\ldots,P_{m}\otimes Q_{n}}
\end{mathpar}

\begin{remark}
  Observe that arities of tensored abstractions matches arities of
  tensored concretions if the original arities matched. Note also that
  the length of the arities corresponds to the increase in dimension
  we see in ordinary vector space tensor product.
\end{remark}

\begin{remark}
  Operationally, this definition distributes the tensor down to
  components ``linked'' by summation. Tensor over summation is
  intriguing in that it mixes names. Moreover, as a consequence of the
  way it mixes names we have the identities for all $x \in \QProc$ and
  $P,Q \in \Proc$

  \begin{mathpar}
    (x \cdot P) \otimes Q \equiv x \cdot (P \otimes Q) \equiv P \otimes (x \cdot Q)
    \and
    P \otimes \pzero \equiv P
  \end{mathpar}

  that the reader is invited to verify.
\end{remark}

\subsubsection{Annihilation}
\begin{mathpar}
  P^{\perp} := \{ Q | \forall R. P|Q \red^{*} R \Rightarrow R \red^{*} \pzero \}
  \and \\
  P^{\underline{\perp}} := \Sigma_{Q \in P^{\perp}} \quotep{Q}?(y).(\dropn{y}|Q) | \Sigma_{Q \in P^{\perp}} \quotep{Q}\clift{\Box}
\end{mathpar}

\paragraph{Discussion} The reader will note that $P^{\perp}$ is a
\emph{set} of processes, while $P^{\underline{\perp}}$ is a
\emph{context}. We call the set $P^{\perp}$ the \emph{annihilators} of
$P$. The parallel composition of a process in the annihilators of $P$
with $P$ will result in a process, the state space of which has all
paths eventually leading to $\pzero$. Execution may endure loops; but
under reasonable conditions of fairness (naturally guaranteed under
most notions of bisimulation) such a composite process cannot get
stuck in such a loop and will, eventually pop out and terminate.

The context $P^{\underline{\perp}}$ is ready and willing to ``take the
$P$ out of'' the process to which it is applied. It will effectively
transmit the code of the process to which it is applied to one of the
annihilators and run the process against it.

\subsubsection{Evaluation}
We fix $M$ a domain of fully abstract interpretation with an equality
coincident with bisimulation. We take $\meaningof{\cdot} : \Proc \to
M$ to be the map interpreting processes and $\nmeaningof{\cdot} : \M
\to Proc$ to be the map running the other way. Then we define

\begin{mathpar}
  \int P := \nmeaningof{\meaningof{P}}
\end{mathpar}

\paragraph{Discussion}
There are many fully abstract interpretations of Milner's
$\pi$-calculus. Any of them can be used as a basis for interpreting
the reflective calculus here. Equipped with such a domain it is
largely a matter of grinding through to check that the Yoneda
construction for the normalization-by-evaluation program can be
extended to this setting.

\begin{remark}
  The reader is invited to verify that $\int (P^{\underline{\perp}}[P]) = 0$.
\end{remark}

\subsection{Quantum mechanics}

Table \ref{tbl:core_qm_op_defns} gives the core operational definitions

\begin{table}[htp]\label{tbl:core_qm_op_defns}
  \center{
    \fbox{
      \begin{tabular}{c|c}
        quantum mechanics & process calculus \\
        \hline
        scalar & $x := \quotep{P}$ \\
        state vector & $\state{P} := P$ \\
        dual & $\state{P}^{*} := \event{P^{\underline{\perp}}} := \quotep{P^{\underline{\perp}}}[-]$ \\
        matrix & $ \Sigma_{\alpha} \state{P_{\alpha}}x_{\alpha}\event{Q_{\alpha}}$ \\
        vector addition & $\state{P} + \state{Q} := \state{P | Q}$ \\
        tensor product & $\state{P} \otimes \state{Q} := \state{P \otimes Q}$ \\
        inner product & $\innerprod{P}{Q} := \quotep{\int P^{\underline{\perp}}[Q]}$ \\
      \end{tabular}
    }
  }
  \caption{QM - operational definitions}
\end{table}

where

\begin{mathpar}
  \prmatrix{P}{Q} := \fprmatrix{P}{\quotep{\pzero}}{Q}
  \and
  \fprmatrix{P}{x}{Q} := (\state{P},x,\event{Q})
  \and
  (\fprmatrix{P}{x}{Q})(\state{R}) := x \cdot \innerprod{Q}{R} \cdot \state{P}
  \and
  (\fprmatrix{P}{x}{Q})(\event{R}) := x \cdot \innerprod{R}{P} \cdot \event{Q}
\end{mathpar}

\paragraph{Discussion}
As promised: vectors (aka states) are represented as processes; duals
as contextual duals; inner product definition should be compared with
standard inner product definition for ....

\begin{remark}
  Assuming $\int (P^{\underline{\perp}}[P]) = 0$, the reader is
  invited to verify that $(\fprmatrix{P}{x}{P})(\state{P}) = x \cdot \state{P}$.
\end{remark}

\begin{remark}
  The reader is invited to verify that $\innerprod{P}{Q}$ could
  equally well have been written $\quotep{\int \stackrel{\vee}{x}}$
  where $x = \event{P^{\underline{\perp}}}(Q)$.

  One of the motivations for this remark is that there is another way
  to factor these operations. We could package up evaluation in the dual:

  \begin{mathpar}
    \state{P}^{*} := \event{\int P^{\underline{\perp}}} := \quotep{\int P^{\underline{\perp}}}[-]
  \end{mathpar}

  and then have inner product defined by
  
  \begin{mathpar}
    \innerprod{P}{Q} := \event{P}(Q)
  \end{mathpar}

  Hopefully, experience with the calculations will provide guidance on
  the best factoring.
\end{remark}

\begin{remark}
  Assuming $\int (P^{\underline{\perp}}[P]) = 0$, the reader is
  invited to verify that $\forall P,Q. (\prmatrix{0}{Q})(\state{0}) =
  \state{0}$ and dually $(\prmatrix{P}{0})(\event{0}) = \event{0}$.
\end{remark}

\begin{remark}
  i'm a little worried that i don't (yet) have proper support for
  complex conjugacy. But, the observation above may give us a
  clue. According to Abramsky, it must be the case that the scalars
  are iso to the homset of the identity for the tensor -- which the
  observation above characterizes. 

  For now, we will simply bookmark the notion with $\overline{x}$.
\end{remark}

\subsubsection{Adjointness}

We need to give a definition of $(\cdot)^{\dagger}$ for matrices. The
obvious candidate definition is
\begin{mathpar}
(\Sigma_{\alpha}\fprmatrix{P_{\alpha}}{x_{\alpha}}{Q_{\alpha}})^{\dagger}
= \Sigma_{\alpha}\fprmatrix{(Q_{\alpha}^{\underline{\perp}})^{*}}{\overline{x}_{\alpha}}{P_{\alpha}^{\underline{\perp}}} 
\end{mathpar}

But, $(Q_{\alpha}^{\underline{\perp}})^{*}$ requires a name along
which to communicate the process to achieve the context application.

\subsubsection{Basis for a basis}
If processes label states and ``addition'' of states (a.k.a. vector
addition) is interpreted as parallel composition, what corresponds to
notions of linear independence and basis? Here, we recall that Yoshida
has developed a set of \emph{combinators} for an asynchronous verison
of Milner's $\pi$-calculus. These are a finite set of processes such
any process can be expressed as parallel composition of these
combinators together with liberal uses of the new operator and
replication. We can simply give a translation of these into the
present calculus and have reasonable expectation that the property
carries over. That is, that the resultant set allows to express all
processes via parallel composition. Note, however, that there is no
new operator or replication in this calculus. As a result, we expect
that the corresponding set is actually infinite. That is, we expect
that the space is actually infinite dimensional.

\begin{remark}
  The attentive reader may be a bit concerned. Certainly, the
  collection $S$, $K$ and $I$ is a finite set of
  combinators. Shouldn't we expect to see a finite set of combinators
  for an effectively equivalent system? i am very sympathetic to this
  critique and feel it warrants full attention. On the other hand, i
  also have in mind the following analogy. The natural numbers, as a
  monoid under addition, has exactly $1$ generator, while the natural
  numbers, as a monoid under multiplication, has countably many
  generators (the primes). We observe that the application of the
  lambda calculus is much less resource sensitive than the parallel
  composition of the $\pi$-calculus. Could it be the case that we have
  an analogy of the form
  
  \begin{mathpar}
    m + n : MN :: m*n : M|N
  \end{mathpar}

  giving a similar blow up in the set of ``primes''?  This is such a
  wonderful thought that, even if it's not true, i think it's worth
  writing down.
\end{remark}
 

\documentclass[12pt]{llncs}
%\documentclass{jktr}

\usepackage[pdftex]{hyperref}                   
\usepackage {listings}
\usepackage {mathpartir}
\usepackage{bcprules}
%\usepackage{listings}
                       
\usepackage{graphicx} 
%\usepackage[margins=2.5cm,nohead,nofoot]{geometry}
%\usepackage{geometry}
\usepackage{amsfonts}
\usepackage{amstext}
\usepackage{latexsym}
\usepackage{amssymb}
\usepackage{color}


%\include{myPreamble}
\include{qm2pi.local} 

%\ifpdf
%\usepackage[pdftex]{graphicx}
%\else
%\usepackage{graphicx}
%\fi

 % \ifpdf
%  \usepackage{pdfsync}
%  \if


%\title{Brief Article}
%\author{David F. Snyder}
%\author{L.G. Meredith}

%\address{Dept. of Math., Texas State University--San Marcos, San Marcos, TX 78666}
       
\pagestyle{empty}


\begin{document}

\lstset{language=[Objective]Caml,frame=shadowbox}

\input{qm2pi.front}

% section front matter (end)

\input{qm2pi.intro} 
 
% section introduction (end)

% \input{qm2pi.knotations} 

% section notation (end)

\input{qm2pi.process.calculi} 

% section concurrent_process_calculi_and_spatial_logics_ (end)
    
%\input{qm2pi.knots2pi} 

%\input{qm2pi.trefoil} 

%\input{qm2pi.mainthm} 

% subsection basic_interpretation (end)

%\input{qm2pi.rho.presentation} 
\subsection{The syntax and semantics of the notation system}\label{sub:the_syntax_and_semantics_of_the_notation_system} % (fold)

We now summarize a technical presentation of the calculus that
embodies our theory of dynamics. The typical presentation of such a
calculus follows the style of giving generators and relations on
them. The grammar, below, describing term constructors, freely
generates the set of processes, $\Proc$. This set is then quotiented
by a relation known as structural congruence and it is over this set
that the notion of dynamics is expressed. This presentation is
essentially that of \cite{MeredithR05} with the addition of
polyadicity and summation. For readability we have relegated some of
the technical subtleties to an appendix.

\subsubsection{Process grammar}\label{subsub:process_grammar}

\begin{mathpar}
  \inferrule* [lab=synchronization] {} {{M} \bc \pzero \;|\; x?F \;|\; x!C }
  \and
  \inferrule* [lab=abstraction] {} {{F} \bc (x)P}
  \and
  \inferrule* [lab=concretion] {} {{C} \bc \langle Q \rangle}
  \and
  \inferrule* [lab=process] {} {{P,Q} \bc M \;| \;P|Q \;|\; @{x}}
  \and
  \inferrule* [lab=name] {} {{x} \bc \quotep{P}}
\end{mathpar} 

Note that $\vec{x}$ (resp. $\vec{P}$) denotes a vector of names
(resp. processes) of length $|\vec{x}|$ (resp. $|\vec{P}|$). We adopt
the following useful abbreviations.

\begin{mathpar}
   x?(\vec{y}).P := x.(\vec{y})P \and  x\clift{\vec{P}} := x.\clift{\vec{P}}
   \and x!(y) := \lift{x}{\dropn{y}}
   \and \Pi_{i=0}^{n-1}P_i := P_0 | \ldots | P_{n-1}
\end{mathpar}

\subsubsection{Structural congruence}

\paragraph{Free and bound names and alpha-equivalence.} At the
core of structural equivalence is alpha-equivalence which identifies
process that are the same up to a change of variable. Formally, we
recognize the distinction between free and bound names. The free names
of a process, $\freenames{P}$, may be calculated recursively as
follows:

\begin{mathpar}
\freenames{\pzero} := \emptyset
  \and \\
  \freenames{x?(y).P} := \{ x \} \cup (\freenames{P} \setminus \{ y \})
  \and 
  \freenames{x!\langle P \rangle} := \{ x \} \cup \{ P \} 
  \and \\
  \freenames{P|Q} := \freenames{P} \cup \freenames{Q}
  \and \\
  \freenames{@{x}} := \{ x \}
\end{mathpar}

$\pi$
$\quotep{\pi}$

$\freenames{-} : \pi \to \mathcal{P}(\quotep{\pi})$

\begin{eqnarray*}
  \freenames{\pzero} & := & \emptyset \\
  \freenames{x?(y).P} & := & \{ x \} \cup (\freenames{P} \setminus \{ y \}) \\
  \freenames{x!\langle P \rangle} & := & \{ x \} \cup \{ P \} \\
  \freenames{P|Q} & := & \freenames{P} \cup \freenames{Q} \\
  \freenames{\dropn{x}} & := & \{ x \}
\end{eqnarray*}

The bound names of a process, $\boundnames{P}$, are those names occurring in $P$
that are not free. For example, in $x?(y).0$, the name $x$ is free, while $y$ is bound.

\begin{mathpar}
  \inferrule* [lab=monoidal-laws] {} { P|Q \equiv Q|P \and P|0 \equiv P \and P|(Q|R) \equiv (P|Q)|R }
\end{mathpar}

\begin{mathpar}
  \inferrule* [lab=alpha-equivalence] {} { (x)P \equiv (y)P\{y/x\} \and y \not\in \freenames{P} }
\end{mathpar}

\begin{definition}
Then two processes, $P,Q$, are alpha-equivalent if $P = Q\{\vec{y}/\vec{x}\}$ for
some $\vec{x} \in \boundnames{Q},\vec{y} \in \boundnames{P}$, where $Q\{\vec{y}/\vec{x}\}$
denotes the capture-avoiding substitution of $\vec{y}$ for $\vec{x}$ in $Q$.
\end{definition}

\begin{definition}
  The {\em structural congruence} \cite{SangiorgiWalker} , $\equiv$,
  between processes is the least congruence containing
  alpha-equivalence, satisfying the abelian monoid laws
  (associativity, commutativity and $\pzero$ as identity) for parallel
  composition $|$ and for summation $+$.
\end{definition}

\subsection{Name equivalence}

We take name equivalence, written $\nameeq$, to be the smallest
equivalence relation generated by the following rules.

\begin{mathpar}
\inferrule*[lab=Quote-drop]
{ }
{ \quotep{@{x}} \nameeq x }

\inferrule*[lab=Struct-equiv]
{ P \scong Q }
{ \quotep{P} \nameeq \quotep{Q} }
\end{mathpar}

The astute reader will have noticed that the mutual recursion of names
and processes imposes a mutual recursion on alpha-equivalence and
structural equivalence via name-equivalence. Fortunately, all of this
works out pleasantly and we may calculate in the natural way, free of
concern. The reader interested in the details is referred to the
appendix \ref{appendix:rho_details}.

\subsection{Substitution}

We use $\Proc$ for the set of processes, $\QProc$ for the set of
names, and $\id{\{}\vec{y} / \vec{x} \id{\}}$ to denote partial maps,
$s : \QProc \rightarrow \QProc$. A map, $s$ lifts, uniquely, to a map
on process terms, $\widehat{s} : \Proc \rightarrow \Proc$ by the
following equations.

\begin{mathpar}
  (0) \psubstp{Q}{P} := 0 \\
  (R \juxtap S) \psubstp{Q}{P}
  :=    
  (R)\psubstp{Q}{P} \juxtap (S) \psubstp{Q}{P} \\
  (x?(y).R) \psubstp{Q}{P}    
  :=    
  (x)\substp{Q}{P} (z)\concat( (R \psubstn{z}{y}) \psubstp{Q}{P} ) \\
  (\lift{x}{R}) \psubstp{Q}{P}  
  :=
  \lift{(x)\substp{Q}{P}}{ R \psubstp{Q}{P} } \\
%   (\dropn{x})  \psubstp{Q}{P}       
%   := 
%   \left\{ 
%     \begin{array}{ccc} 
%       \dropn{\quotep{Q}} & & x \nameeq \quotep{P} \\
%       \dropn{x} & & otherwise \\
%     \end{array}
%   \right. 
  (\dropn{x})  \psubstp{Q}{P}       
  := 
  \left\{ 
    \begin{array}{ccc} 
      Q & & x \nameeq \quotep{P} \\
      \dropn{x} & & otherwise \\
    \end{array}
  \right.
\end{mathpar}
 

where

\begin{eqnarray}
  (x)\id{\{} \lpquote Q \rpquote / \lpquote P \rpquote \id{\}}            = 
  \left\{ 
    \begin{array}{ccc}
      \lpquote Q \rpquote & & x \nameeq \lpquote P \rpquote \\
      x & & otherwise \\
    \end{array}
  \right. \nonumber
\end{eqnarray}

and $z$ is chosen distinct from $\quotep{P}$, $\quotep{Q}$, the free
names in $Q$, and all the names in $R$. Our $\alpha$-equivalence will
be built in the standard way from this substitution.

\begin{remark}\label{rem:no_self_referential_names}
  One consequence of these definitions is that $\forall P. \quotep{P}
  \not\in \freenames{P}$.
\end{remark}

\subsection{ Dynamic quote: an example }

Anticipating something of what's to come, consider applying the
substitution, $\widehat{\id{\{}u / z \id{\}}}$, to the following pair
of processes, $\lift{w}{y!(z)}$ and $w[ \lpquote y!(z) \rpquote ]$.

\begin{eqnarray}
	\lift{w}{y!(z)}\widehat{\id{\{}u / z \id{\}}}
		& = &
		\lift{w}{y!(u)} \nonumber\\
	w[ \lpquote y!(z) \rpquote ] \widehat{ \id{\{}u / z \id{\}} }
		& = &
		w[ \lpquote y!(z) \rpquote ] \nonumber
\end{eqnarray}

Because the body of the process between quotes is impervious to
substitution, we get radically different answers. In fact, by
examining the first process in an input context,
e.g. $x?(z).\lift{w}{y!(z)}$, we see that the process under the lift
operator may be shaped by prefixed inputs binding a name inside it. In
this sense, the lift operator will be seen as a way to dynamically
construct processes before reifying them as names.

Finally equipped with these standard features we can present the
dynamics of the calculus.

\subsubsection{Operational semantics} 

Finally, we introduce the computational dynamics. What marks these
algebras as distinct from other more traditionally studied algebraic
structures, e.g. vector spaces or polynomial rings, is the manner in
which dynamics is captured. In traditional structures, dynamics is typically
expressed through morphisms between such structures, as in linear maps
between vector spaces or morphisms between rings. In algebras
associated with the semantics of computation, the dynamics is
expressed as part of the algebraic structure itself, through a
reduction reduction relation typically denoted by $\red$. Below, we
give a recursive presentation of this relation for the calculus used
in the encoding.

$\red \subseteq \pi \times \pi$
$\red : \pi \to \mathcal{P}(\pi)$

\begin{mathpar}
  \inferrule* [lab=Comm] { \textsf{match}( x_{src}, x_{trgt} ) } { x_{trgt}?(y)P \; | \; x_{src}!\langle {Q} \rangle \red P\{\quotep{Q}/y}\} }
  \and \\
  \inferrule* [lab=Par] {{P} \red {P}'} {{{P} | {Q}} \red {{P}' | {Q}}}
  \and
  \inferrule* [lab=Equiv]{{{P} \scong {P}'} \andalso {{P}' \red {Q}'} \andalso {{Q}' \scong {Q}}}{{P} \red {Q}}
\end{mathpar}

\begin{eqnarray*}
  match_{\equiv} (\quotep{P},\quotep{Q}) & := & P \equiv Q \\
  match_{\dagger}(\quotep{P},\quotep{Q}) & := & \forall R. P|Q \red^{*} R => R \red^{*} 0 \\
  match_{K}(\quotep{P},\quotep{Q}) & := & K \mbox{ for some context } K
\end{eqnarray*}

$u?(x)P | u!\langle Q \rangle \red P\{\quotep{Q}/x\}$

%We write $\wred$ for $\red^*$, and $P\red$ if $\exists Q $ such that $ P \red Q$.
We write $P\red$ if $\exists Q $ such that $ P \red Q$ and $P\not\red$, otherwise.

\section{Replication}

As mentioned before, it is known that replication (and hence
recursion) can be implemented in a higher-order process algebra
\cite{SangiorgiWalker}. As our first example of calculation with the
machinery thus far presented we give the construction explicitly in
the {\rhoc}.

\begin{eqnarray}
	D_{x} & := & \prefix{x}{y}{(\binpar{\outputp{x}{y}}{@{y}})} \nonumber\\
	\bangp_{x}{P} & := & \binpar{{x}!\langle{\binpar{D_{x}}{P}}\rangle}{D_{x}} \nonumber
\end{eqnarray}

\begin{eqnarray}
	\bangp_{x}{P} & & \nonumber\\
	=
	& {x}!\langle{(\prefix{x}{y}{(\outputp{x}{y} | @{y})) | P}}\rangle 
	      | \prefix{x}{y}{(\outputp{x}{y} | @{y})} & \nonumber\\
	\red
	& (\outputp{x}{y} | @{y})\substn{\quotep{(\prefix{x}{y}{(@{y} | \outputp{x}{y})) | P}}}{y} & \nonumber\\
	=
	& \outputp{x}{\quotep{(\prefix{x}{y}{(\outputp{x}{y} | @{y})) | P}}}
	  | {(\prefix{x}{y}{(\outputp{x}{y} | @{y})) | P}} & \nonumber\\
	\red
	& \ldots & \nonumber\\
	\red^*
	& P | P | \ldots & \nonumber
\end{eqnarray}

Of course, this encoding, as an implementation, runs away, unfolding
$\bangp{P}$ eagerly. A lazier and more implementable replication
operator, restricted to input-guarded processes, may be obtained as follows.

\begin{eqnarray}
\bangp{\prefix{u}{v}{P}} 
	:= 
	\binpar{\lift{x}{\prefix{u}{v}{(\binpar{D(x)}{P})}}}{D(x)} \nonumber
\end{eqnarray}

\begin{remark}
  Note that the lazier definition still does not deal with summation
  or mixed summation (i.e. sums over input and output). The reader is
  invited to construct definitions of replication that deal with these
  features. 

  Further, the definitions are parameterized in a name, $x$. Can you,
  gentle reader, make a definition that eliminates this parameter and
  guarantees no accidental interaction between the replication
  machinery and the process being replicated -- i.e. no accidental
  sharing of names used by the process to get its work done and the
  name(s) used by the replication to effect copying. This latter
  revision of the definition of replication is crucial to obtaining
  the expected identity $!!P \sim !P$.
\end{remark}

\begin{remark}\label{rem:paradoxical_combinator}
  The reader familiar with the lambda calculus will have noticed the
  similarity between $D$ and the paradoxical combinator.

  [Ed. note: the existence of this seems to suggest we have to be more
  restrictive on the set of processes and names we admit if we are to
  support no-cloning.]
\end{remark}

\subsubsection{Bisimulation}

The computational dynamics gives rise to another kind of equivalence,
the equivalence of computational behavior. As previously mentioned
this is typically captured \emph{via} some form of bisimulation.

% The notion we use in this paper is weak barbed bisimulation
% \cite{milner91polyadicpi}.

The notion we use in this paper is derived from weak barbed
bisimulation \cite{milner91polyadicpi}. 

\begin{definition}
An \emph{observation relation}, $\downarrow_{\mathcal N}$, over a set
of names, $\mathcal N$, is the smallest relation satisfying the rules
below.

\infrule[Out-barb]{y \in {\mathcal N}, \; x \nameeq y}
		  {\outputp{x}{v} \downarrow_{\mathcal N} x}
\infrule[Par-barb]{\mbox{$P\downarrow_{\mathcal N} x$ or $Q\downarrow_{\mathcal N} x$}}
		  {\binpar{P}{Q} \downarrow_{\mathcal N} x}

We write $P \Downarrow_{\mathcal N} x$ if there is $Q$ such that 
$P \wred Q$ and $Q \downarrow_{\mathcal N} x$.
\end{definition}

\begin{definition}
%\label{def.bbisim}
An  ${\mathcal N}$-\emph{barbed bisimulation} over a set of names, ${\mathcal N}$, is a symmetric binary relation 
${\mathcal S}_{\mathcal N}$ between agents such that $P\rel{S}_{\mathcal N}Q$ implies:
\begin{enumerate}
\item If $P \red P'$ then $Q \wred Q'$ and $P'\rel{S}_{\mathcal N} Q'$.
\item If $P\downarrow_{\mathcal N} x$, then $Q\Downarrow_{\mathcal N} x$.
\end{enumerate}
$P$ is ${\mathcal N}$-barbed bisimilar to $Q$, written
$P \wbbisim_{\mathcal N} Q$, if $P \rel{S}_{\mathcal N} Q$ for some ${\mathcal N}$-barbed bisimulation ${\mathcal S}_{\mathcal N}$.
\end{definition}

$\mathcal{R} \subseteq \pi \times \pi$

$P \mathcal{R} Q => \forall P'. P \red P' \Rightarrow \exists Q'. Q \red Q', P' \mathcal{R} Q'$

$P \vdash x \Rightarrow Q \vdash x$

\begin{mathpar}
  \inferrule*[lab=Out-barb]{x \nameeq y}{{y}!\langle{Q}\rangle \vdash x}
  \and
  \inferrule*[lab=Par-barb]{\mbox{$P\vdash x$ or $Q\vdash x$}}{\binpar{P}{Q} \vdash x}
\end{mathpar}

\subsubsection{Contexts}

One of the principle advantages of computational calculi like the
$\pi$-calculus is a well-defined notion of context,
contextual-equivalence and a correlation between
contextual-equivalence and notions of bisimulation. The notion of
context allows the decomposition of a process into (sub-)process and
its syntactic environment, its context. Thus, a context may be
thought of as a process with a ``hole'' (written $\Box$) in it. The
application of a context $M$ to a process $P$, written $M[P]$, is
tantamount to filling the hole in $M$ with $P$. In this paper we do
not need the full weight of this theory, but do make use of the notion
of context in the proof the main theorem. 

\begin{mathpar}
  \inferrule* [lab=summation] {} {{M_{M},M_{N}} \bc \Box \;|\; x.M_{A} \;|\; M_{M}+M_{N}}
  \and
  \inferrule* [lab=agent] {} {{M_{A}} \bc (\vec{x})M_{P} \;| \; \clift{P_0,\ldots,M_{P},\ldots,P_N}}
  \and \\
  \inferrule* [lab=process] {} {{M_{P}} \bc M_{N} \;| \;P|M_{P} }
\end{mathpar} 

\begin{mathpar}
  \inferrule* [lab=sychronization] {} {M_{N} \bc \Box \;|\; x?M_{F} \;|\; x!M_{C}}
  \and
  \inferrule* [lab=abstraction] {} {{M_{F}} \bc (x)M_{P} }
  \and
  \inferrule* [lab=concretion] {} {{M_{C}} \bc \langle M_{P} \rangle }
  \and \\
  \inferrule* [lab=process] {} {{M_{P}} \bc M_{N} \;| \;P|M_{P} }
\end{mathpar}

\begin{definition}[contextual application] Given a context $M$, and
  process $P$, we define the \emph{contextual application}, $M[P] :=
  M\{P/\Box\}$. That is, the contextual application of M to P is the
  substitution of $P$ for $\Box$ in $M$.
\end{definition}

$\meaningof{-} : L \to \mathcal{P}(\pi)$

\begin{mathpar}
  \inferrule* [lab=collection] {} {\meaningof{true} = \pi, \and \meaningof{~E} = \pi \setminus \meaningof{E}, \and \meaningof{E_{1} \& E_{2}} = \meaningof{E_{1}} \cap \meaningof{E_{2}}}
\end{mathpar}

\begin{mathpar}
  \inferrule* [lab=structure] {} {\meaningof{0} = \{ P \in \pi | P \equiv 0 \}, \and \\ \meaningof{E_1 | E_2} = \{ P \in \pi | P \equiv P_{1} | P_{2}, P_{1} \in \meaningof{E_{1}}, P_{2} \in \meaningof{E_2}\} }
\end{mathpar}

\begin{mathpar}
 \inferrule* [lab=behavior] {} {\meaningof{\langle a?b \rangle E} = \{ P \in \pi | P \equiv Q | u?(y)P', \\ \and \\\\ \and \\ \;\;\; u \in \meaningof{a}, \forall z.P'\{z/y\} \in \meaningof{E\{z/b\}}\}, \and \\ \meaningof{a!E} = \{ P \in \pi | P \equiv Q | x!\langle P' \rangle, x \in \meaningof{a} P' \in \meaningof{E}\} }
\end{mathpar}

\begin{mathpar}
 \inferrule* [lab=nominal] {} {\meaningof{\quotep{E}} = \{ \quotep{P} \in \quotep{\pi} | P \in \meaningof{E} \}, \and \meaningof{\quotep{P}} = \{ \quotep{Q} \in \quotep{\pi} | P \equiv Q \} \and \\ \meaningof{@\quotep{E}} = \{ P \in \pi | P \equiv @x, x \in \meaningof{E} \}}
\end{mathpar}

\begin{eqnarray*}
  \\
  \meaningof{-} : TS \to ST
\end{eqnarray*}

\begin{eqnarray*}
  \\
  L : TS \to ST
\end{eqnarray*}

\begin{eqnarray*}
  \\
  P \models E \iff P \in \meaningof{E}
\end{eqnarray*}

\begin{eqnarray*}
  P \approx_{L} Q \iff \forall E \in L. P \models E \iff Q \models E
\end{eqnarray*}

\begin{eqnarray*}
  P \approx_{K} Q
\end{eqnarray*}

\begin{eqnarray*}
  P \approx Q
\end{eqnarray*}

$\approx_{K} = \approx = \approx_{L}$

\subsubsection{Contextual duality}

Note that contexts extend the quotation operation to a family of
operations from processes to names. Given a context, $M$, we can
define a \emph{nominal context}, $\quotep{M}$ by $\quotep{M}[P] :=
\quotep{M[P]}$. To foreshadow what is to come we observe that these
operations enjoy a duality with processes very much like the duality
between vectors and maps from vectors to scalars.

Further, because the calculus is essentially higher-order, we have a
correspondence between contexts and processes. More specifically,
given a name $x$ and a context $M$ we can construct $M^{*}_{x}$ such
that 

\begin{mathpar}
  M^{*}_{x} | \lift{x}{P} \red M[P]
\end{mathpar}

namely,

\begin{mathpar}
  M^{*}_{x} := x?(u).M[\dropn{u}]
\end{mathpar}

The dependence of $M^{*}_{x}$ on a name makes it an abstraction, 

\begin{mathpar}
  M^{*} := (x)x?(u).M[\dropn{u}]
\end{mathpar}

\subsection{Additional notation}

It will sometimes be convenient to denote the process a name
quotes. We already have the notation $x = \quotep{P}$, but it will be
convenient to introduce an alternate notation, $\procn{x}$, when we
want to emphasize the connection to the use of the name. Note that, by
virtue of name equivalence, $\quotep{\procn{x}} \nameeq x$; so, the
notation is consistent with previous definitions.

Further, because names have structure it is possible to effect
substitutions on the basis of that structure. This means we need to
upgrade our notation for substitutions, which we accomplish by
adapting comprehension notation. Thus,

\begin{mathpar}
  P\{ y / x : x \in S \}
\end{mathpar}

is interpreted to mean the process derived from P by replacing (in a
capture-avoiding manner) each occurrence of $x$ in $S$ by $y$. For example,

\begin{mathpar}
  P\{ \quotep{\procn{x}|\procn{x}} / x : x \in \freenames{P} \}
\end{mathpar}

will replace each (occurrence) of a free name $x$ in $P$ by
$\quotep{\procn{x}|\procn{x}}$.

Also, we will avail ourselves of the notation $x^{L}$ and $x^{R}$ to
denote injections of a name into disjoint copies of the name
space. There are numerous ways to accomplish this. One example can be
found in \cite{MeredithR05}. This notation overloads to vectors of
names: $\vec{x}^{\pi} := (x_{i}^{\pi} \; : \; 0 \leq i < |\vec{x}| )$ where $\pi \in \{L,R\}$.

We also use $P^{\Box} := P|\Box$.

In \cite{MeredithR05} an interpretation of the new operator is
given. It turns out that there are several possible interpretations
all enjoying the requisite algebraic properties of the operator (see
\cite{milner91polyadicpi}). We will therefore make liberal use of
$(\nu\; \vec{x})P$.

% subsection the_syntax_and_semantics_of_the_notation_system (end)   

\input{qm2pi.qmops} 

\input{qm2pi.sterngerlach} 

\input{qm2pi.metric} 

% section concurrent_process_calculi (end)

%\input{qm2pi.proofsketch}

% section proof sketch (end)

%\input{qm2pi.slviaknots} 

% section spatial logic via knots (end)

\input{qm2pi.conclusion}

% section conclusion (end)

%\input{qm2pi.dtcodes} 

% section wiring algorithm (end)

\input{qm2pi.ack} 

% section acknowledgments (end)

\newpage


\bibliographystyle{plain}   
\bibliography{../../biblios/main.bib}

\input{qm2pi.rhodetails}

\end{document}

 

\documentclass[12pt]{llncs}
%\documentclass{jktr}

\usepackage[pdftex]{hyperref}                   
\usepackage {listings}
\usepackage {mathpartir}
\usepackage{bcprules}
%\usepackage{listings}
                       
\usepackage{graphicx} 
%\usepackage[margins=2.5cm,nohead,nofoot]{geometry}
%\usepackage{geometry}
\usepackage{amsfonts}
\usepackage{amstext}
\usepackage{latexsym}
\usepackage{amssymb}
\usepackage{color}


%\include{myPreamble}
\include{qm2pi.local} 

%\ifpdf
%\usepackage[pdftex]{graphicx}
%\else
%\usepackage{graphicx}
%\fi

 % \ifpdf
%  \usepackage{pdfsync}
%  \if


%\title{Brief Article}
%\author{David F. Snyder}
%\author{L.G. Meredith}

%\address{Dept. of Math., Texas State University--San Marcos, San Marcos, TX 78666}
       
\pagestyle{empty}


\begin{document}

\lstset{language=[Objective]Caml,frame=shadowbox}

\input{qm2pi.front}

% section front matter (end)

\input{qm2pi.intro} 
 
% section introduction (end)

% \input{qm2pi.knotations} 

% section notation (end)

\input{qm2pi.process.calculi} 

% section concurrent_process_calculi_and_spatial_logics_ (end)
    
%\input{qm2pi.knots2pi} 

%\input{qm2pi.trefoil} 

%\input{qm2pi.mainthm} 

% subsection basic_interpretation (end)

%\input{qm2pi.rho.presentation} 
\subsection{The syntax and semantics of the notation system}\label{sub:the_syntax_and_semantics_of_the_notation_system} % (fold)

We now summarize a technical presentation of the calculus that
embodies our theory of dynamics. The typical presentation of such a
calculus follows the style of giving generators and relations on
them. The grammar, below, describing term constructors, freely
generates the set of processes, $\Proc$. This set is then quotiented
by a relation known as structural congruence and it is over this set
that the notion of dynamics is expressed. This presentation is
essentially that of \cite{MeredithR05} with the addition of
polyadicity and summation. For readability we have relegated some of
the technical subtleties to an appendix.

\subsubsection{Process grammar}\label{subsub:process_grammar}

\begin{mathpar}
  \inferrule* [lab=synchronization] {} {{M} \bc \pzero \;|\; x?F \;|\; x!C }
  \and
  \inferrule* [lab=abstraction] {} {{F} \bc (x)P}
  \and
  \inferrule* [lab=concretion] {} {{C} \bc \langle Q \rangle}
  \and
  \inferrule* [lab=process] {} {{P,Q} \bc M \;| \;P|Q \;|\; @{x}}
  \and
  \inferrule* [lab=name] {} {{x} \bc \quotep{P}}
\end{mathpar} 

Note that $\vec{x}$ (resp. $\vec{P}$) denotes a vector of names
(resp. processes) of length $|\vec{x}|$ (resp. $|\vec{P}|$). We adopt
the following useful abbreviations.

\begin{mathpar}
   x?(\vec{y}).P := x.(\vec{y})P \and  x\clift{\vec{P}} := x.\clift{\vec{P}}
   \and x!(y) := \lift{x}{\dropn{y}}
   \and \Pi_{i=0}^{n-1}P_i := P_0 | \ldots | P_{n-1}
\end{mathpar}

\subsubsection{Structural congruence}

\paragraph{Free and bound names and alpha-equivalence.} At the
core of structural equivalence is alpha-equivalence which identifies
process that are the same up to a change of variable. Formally, we
recognize the distinction between free and bound names. The free names
of a process, $\freenames{P}$, may be calculated recursively as
follows:

\begin{mathpar}
\freenames{\pzero} := \emptyset
  \and \\
  \freenames{x?(y).P} := \{ x \} \cup (\freenames{P} \setminus \{ y \})
  \and 
  \freenames{x!\langle P \rangle} := \{ x \} \cup \{ P \} 
  \and \\
  \freenames{P|Q} := \freenames{P} \cup \freenames{Q}
  \and \\
  \freenames{@{x}} := \{ x \}
\end{mathpar}

$\pi$
$\quotep{\pi}$

$\freenames{-} : \pi \to \mathcal{P}(\quotep{\pi})$

\begin{eqnarray*}
  \freenames{\pzero} & := & \emptyset \\
  \freenames{x?(y).P} & := & \{ x \} \cup (\freenames{P} \setminus \{ y \}) \\
  \freenames{x!\langle P \rangle} & := & \{ x \} \cup \{ P \} \\
  \freenames{P|Q} & := & \freenames{P} \cup \freenames{Q} \\
  \freenames{\dropn{x}} & := & \{ x \}
\end{eqnarray*}

The bound names of a process, $\boundnames{P}$, are those names occurring in $P$
that are not free. For example, in $x?(y).0$, the name $x$ is free, while $y$ is bound.

\begin{mathpar}
  \inferrule* [lab=monoidal-laws] {} { P|Q \equiv Q|P \and P|0 \equiv P \and P|(Q|R) \equiv (P|Q)|R }
\end{mathpar}

\begin{mathpar}
  \inferrule* [lab=alpha-equivalence] {} { (x)P \equiv (y)P\{y/x\} \and y \not\in \freenames{P} }
\end{mathpar}

\begin{definition}
Then two processes, $P,Q$, are alpha-equivalent if $P = Q\{\vec{y}/\vec{x}\}$ for
some $\vec{x} \in \boundnames{Q},\vec{y} \in \boundnames{P}$, where $Q\{\vec{y}/\vec{x}\}$
denotes the capture-avoiding substitution of $\vec{y}$ for $\vec{x}$ in $Q$.
\end{definition}

\begin{definition}
  The {\em structural congruence} \cite{SangiorgiWalker} , $\equiv$,
  between processes is the least congruence containing
  alpha-equivalence, satisfying the abelian monoid laws
  (associativity, commutativity and $\pzero$ as identity) for parallel
  composition $|$ and for summation $+$.
\end{definition}

\subsection{Name equivalence}

We take name equivalence, written $\nameeq$, to be the smallest
equivalence relation generated by the following rules.

\begin{mathpar}
\inferrule*[lab=Quote-drop]
{ }
{ \quotep{@{x}} \nameeq x }

\inferrule*[lab=Struct-equiv]
{ P \scong Q }
{ \quotep{P} \nameeq \quotep{Q} }
\end{mathpar}

The astute reader will have noticed that the mutual recursion of names
and processes imposes a mutual recursion on alpha-equivalence and
structural equivalence via name-equivalence. Fortunately, all of this
works out pleasantly and we may calculate in the natural way, free of
concern. The reader interested in the details is referred to the
appendix \ref{appendix:rho_details}.

\subsection{Substitution}

We use $\Proc$ for the set of processes, $\QProc$ for the set of
names, and $\id{\{}\vec{y} / \vec{x} \id{\}}$ to denote partial maps,
$s : \QProc \rightarrow \QProc$. A map, $s$ lifts, uniquely, to a map
on process terms, $\widehat{s} : \Proc \rightarrow \Proc$ by the
following equations.

\begin{mathpar}
  (0) \psubstp{Q}{P} := 0 \\
  (R \juxtap S) \psubstp{Q}{P}
  :=    
  (R)\psubstp{Q}{P} \juxtap (S) \psubstp{Q}{P} \\
  (x?(y).R) \psubstp{Q}{P}    
  :=    
  (x)\substp{Q}{P} (z)\concat( (R \psubstn{z}{y}) \psubstp{Q}{P} ) \\
  (\lift{x}{R}) \psubstp{Q}{P}  
  :=
  \lift{(x)\substp{Q}{P}}{ R \psubstp{Q}{P} } \\
%   (\dropn{x})  \psubstp{Q}{P}       
%   := 
%   \left\{ 
%     \begin{array}{ccc} 
%       \dropn{\quotep{Q}} & & x \nameeq \quotep{P} \\
%       \dropn{x} & & otherwise \\
%     \end{array}
%   \right. 
  (\dropn{x})  \psubstp{Q}{P}       
  := 
  \left\{ 
    \begin{array}{ccc} 
      Q & & x \nameeq \quotep{P} \\
      \dropn{x} & & otherwise \\
    \end{array}
  \right.
\end{mathpar}
 

where

\begin{eqnarray}
  (x)\id{\{} \lpquote Q \rpquote / \lpquote P \rpquote \id{\}}            = 
  \left\{ 
    \begin{array}{ccc}
      \lpquote Q \rpquote & & x \nameeq \lpquote P \rpquote \\
      x & & otherwise \\
    \end{array}
  \right. \nonumber
\end{eqnarray}

and $z$ is chosen distinct from $\quotep{P}$, $\quotep{Q}$, the free
names in $Q$, and all the names in $R$. Our $\alpha$-equivalence will
be built in the standard way from this substitution.

\begin{remark}\label{rem:no_self_referential_names}
  One consequence of these definitions is that $\forall P. \quotep{P}
  \not\in \freenames{P}$.
\end{remark}

\subsection{ Dynamic quote: an example }

Anticipating something of what's to come, consider applying the
substitution, $\widehat{\id{\{}u / z \id{\}}}$, to the following pair
of processes, $\lift{w}{y!(z)}$ and $w[ \lpquote y!(z) \rpquote ]$.

\begin{eqnarray}
	\lift{w}{y!(z)}\widehat{\id{\{}u / z \id{\}}}
		& = &
		\lift{w}{y!(u)} \nonumber\\
	w[ \lpquote y!(z) \rpquote ] \widehat{ \id{\{}u / z \id{\}} }
		& = &
		w[ \lpquote y!(z) \rpquote ] \nonumber
\end{eqnarray}

Because the body of the process between quotes is impervious to
substitution, we get radically different answers. In fact, by
examining the first process in an input context,
e.g. $x?(z).\lift{w}{y!(z)}$, we see that the process under the lift
operator may be shaped by prefixed inputs binding a name inside it. In
this sense, the lift operator will be seen as a way to dynamically
construct processes before reifying them as names.

Finally equipped with these standard features we can present the
dynamics of the calculus.

\subsubsection{Operational semantics} 

Finally, we introduce the computational dynamics. What marks these
algebras as distinct from other more traditionally studied algebraic
structures, e.g. vector spaces or polynomial rings, is the manner in
which dynamics is captured. In traditional structures, dynamics is typically
expressed through morphisms between such structures, as in linear maps
between vector spaces or morphisms between rings. In algebras
associated with the semantics of computation, the dynamics is
expressed as part of the algebraic structure itself, through a
reduction reduction relation typically denoted by $\red$. Below, we
give a recursive presentation of this relation for the calculus used
in the encoding.

$\red \subseteq \pi \times \pi$
$\red : \pi \to \mathcal{P}(\pi)$

\begin{mathpar}
  \inferrule* [lab=Comm] { \textsf{match}( x_{src}, x_{trgt} ) } { x_{trgt}?(y)P \; | \; x_{src}!\langle {Q} \rangle \red P\{\quotep{Q}/y}\} }
  \and \\
  \inferrule* [lab=Par] {{P} \red {P}'} {{{P} | {Q}} \red {{P}' | {Q}}}
  \and
  \inferrule* [lab=Equiv]{{{P} \scong {P}'} \andalso {{P}' \red {Q}'} \andalso {{Q}' \scong {Q}}}{{P} \red {Q}}
\end{mathpar}

\begin{eqnarray*}
  match_{\equiv} (\quotep{P},\quotep{Q}) & := & P \equiv Q \\
  match_{\dagger}(\quotep{P},\quotep{Q}) & := & \forall R. P|Q \red^{*} R => R \red^{*} 0 \\
  match_{K}(\quotep{P},\quotep{Q}) & := & K \mbox{ for some context } K
\end{eqnarray*}

$u?(x)P | u!\langle Q \rangle \red P\{\quotep{Q}/x\}$

%We write $\wred$ for $\red^*$, and $P\red$ if $\exists Q $ such that $ P \red Q$.
We write $P\red$ if $\exists Q $ such that $ P \red Q$ and $P\not\red$, otherwise.

\section{Replication}

As mentioned before, it is known that replication (and hence
recursion) can be implemented in a higher-order process algebra
\cite{SangiorgiWalker}. As our first example of calculation with the
machinery thus far presented we give the construction explicitly in
the {\rhoc}.

\begin{eqnarray}
	D_{x} & := & \prefix{x}{y}{(\binpar{\outputp{x}{y}}{@{y}})} \nonumber\\
	\bangp_{x}{P} & := & \binpar{{x}!\langle{\binpar{D_{x}}{P}}\rangle}{D_{x}} \nonumber
\end{eqnarray}

\begin{eqnarray}
	\bangp_{x}{P} & & \nonumber\\
	=
	& {x}!\langle{(\prefix{x}{y}{(\outputp{x}{y} | @{y})) | P}}\rangle 
	      | \prefix{x}{y}{(\outputp{x}{y} | @{y})} & \nonumber\\
	\red
	& (\outputp{x}{y} | @{y})\substn{\quotep{(\prefix{x}{y}{(@{y} | \outputp{x}{y})) | P}}}{y} & \nonumber\\
	=
	& \outputp{x}{\quotep{(\prefix{x}{y}{(\outputp{x}{y} | @{y})) | P}}}
	  | {(\prefix{x}{y}{(\outputp{x}{y} | @{y})) | P}} & \nonumber\\
	\red
	& \ldots & \nonumber\\
	\red^*
	& P | P | \ldots & \nonumber
\end{eqnarray}

Of course, this encoding, as an implementation, runs away, unfolding
$\bangp{P}$ eagerly. A lazier and more implementable replication
operator, restricted to input-guarded processes, may be obtained as follows.

\begin{eqnarray}
\bangp{\prefix{u}{v}{P}} 
	:= 
	\binpar{\lift{x}{\prefix{u}{v}{(\binpar{D(x)}{P})}}}{D(x)} \nonumber
\end{eqnarray}

\begin{remark}
  Note that the lazier definition still does not deal with summation
  or mixed summation (i.e. sums over input and output). The reader is
  invited to construct definitions of replication that deal with these
  features. 

  Further, the definitions are parameterized in a name, $x$. Can you,
  gentle reader, make a definition that eliminates this parameter and
  guarantees no accidental interaction between the replication
  machinery and the process being replicated -- i.e. no accidental
  sharing of names used by the process to get its work done and the
  name(s) used by the replication to effect copying. This latter
  revision of the definition of replication is crucial to obtaining
  the expected identity $!!P \sim !P$.
\end{remark}

\begin{remark}\label{rem:paradoxical_combinator}
  The reader familiar with the lambda calculus will have noticed the
  similarity between $D$ and the paradoxical combinator.

  [Ed. note: the existence of this seems to suggest we have to be more
  restrictive on the set of processes and names we admit if we are to
  support no-cloning.]
\end{remark}

\subsubsection{Bisimulation}

The computational dynamics gives rise to another kind of equivalence,
the equivalence of computational behavior. As previously mentioned
this is typically captured \emph{via} some form of bisimulation.

% The notion we use in this paper is weak barbed bisimulation
% \cite{milner91polyadicpi}.

The notion we use in this paper is derived from weak barbed
bisimulation \cite{milner91polyadicpi}. 

\begin{definition}
An \emph{observation relation}, $\downarrow_{\mathcal N}$, over a set
of names, $\mathcal N$, is the smallest relation satisfying the rules
below.

\infrule[Out-barb]{y \in {\mathcal N}, \; x \nameeq y}
		  {\outputp{x}{v} \downarrow_{\mathcal N} x}
\infrule[Par-barb]{\mbox{$P\downarrow_{\mathcal N} x$ or $Q\downarrow_{\mathcal N} x$}}
		  {\binpar{P}{Q} \downarrow_{\mathcal N} x}

We write $P \Downarrow_{\mathcal N} x$ if there is $Q$ such that 
$P \wred Q$ and $Q \downarrow_{\mathcal N} x$.
\end{definition}

\begin{definition}
%\label{def.bbisim}
An  ${\mathcal N}$-\emph{barbed bisimulation} over a set of names, ${\mathcal N}$, is a symmetric binary relation 
${\mathcal S}_{\mathcal N}$ between agents such that $P\rel{S}_{\mathcal N}Q$ implies:
\begin{enumerate}
\item If $P \red P'$ then $Q \wred Q'$ and $P'\rel{S}_{\mathcal N} Q'$.
\item If $P\downarrow_{\mathcal N} x$, then $Q\Downarrow_{\mathcal N} x$.
\end{enumerate}
$P$ is ${\mathcal N}$-barbed bisimilar to $Q$, written
$P \wbbisim_{\mathcal N} Q$, if $P \rel{S}_{\mathcal N} Q$ for some ${\mathcal N}$-barbed bisimulation ${\mathcal S}_{\mathcal N}$.
\end{definition}

$\mathcal{R} \subseteq \pi \times \pi$

$P \mathcal{R} Q => \forall P'. P \red P' \Rightarrow \exists Q'. Q \red Q', P' \mathcal{R} Q'$

$P \vdash x \Rightarrow Q \vdash x$

\begin{mathpar}
  \inferrule*[lab=Out-barb]{x \nameeq y}{{y}!\langle{Q}\rangle \vdash x}
  \and
  \inferrule*[lab=Par-barb]{\mbox{$P\vdash x$ or $Q\vdash x$}}{\binpar{P}{Q} \vdash x}
\end{mathpar}

\subsubsection{Contexts}

One of the principle advantages of computational calculi like the
$\pi$-calculus is a well-defined notion of context,
contextual-equivalence and a correlation between
contextual-equivalence and notions of bisimulation. The notion of
context allows the decomposition of a process into (sub-)process and
its syntactic environment, its context. Thus, a context may be
thought of as a process with a ``hole'' (written $\Box$) in it. The
application of a context $M$ to a process $P$, written $M[P]$, is
tantamount to filling the hole in $M$ with $P$. In this paper we do
not need the full weight of this theory, but do make use of the notion
of context in the proof the main theorem. 

\begin{mathpar}
  \inferrule* [lab=summation] {} {{M_{M},M_{N}} \bc \Box \;|\; x.M_{A} \;|\; M_{M}+M_{N}}
  \and
  \inferrule* [lab=agent] {} {{M_{A}} \bc (\vec{x})M_{P} \;| \; \clift{P_0,\ldots,M_{P},\ldots,P_N}}
  \and \\
  \inferrule* [lab=process] {} {{M_{P}} \bc M_{N} \;| \;P|M_{P} }
\end{mathpar} 

\begin{mathpar}
  \inferrule* [lab=sychronization] {} {M_{N} \bc \Box \;|\; x?M_{F} \;|\; x!M_{C}}
  \and
  \inferrule* [lab=abstraction] {} {{M_{F}} \bc (x)M_{P} }
  \and
  \inferrule* [lab=concretion] {} {{M_{C}} \bc \langle M_{P} \rangle }
  \and \\
  \inferrule* [lab=process] {} {{M_{P}} \bc M_{N} \;| \;P|M_{P} }
\end{mathpar}

\begin{definition}[contextual application] Given a context $M$, and
  process $P$, we define the \emph{contextual application}, $M[P] :=
  M\{P/\Box\}$. That is, the contextual application of M to P is the
  substitution of $P$ for $\Box$ in $M$.
\end{definition}

$\meaningof{-} : L \to \mathcal{P}(\pi)$

\begin{mathpar}
  \inferrule* [lab=collection] {} {\meaningof{true} = \pi, \and \meaningof{~E} = \pi \setminus \meaningof{E}, \and \meaningof{E_{1} \& E_{2}} = \meaningof{E_{1}} \cap \meaningof{E_{2}}}
\end{mathpar}

\begin{mathpar}
  \inferrule* [lab=structure] {} {\meaningof{0} = \{ P \in \pi | P \equiv 0 \}, \and \\ \meaningof{E_1 | E_2} = \{ P \in \pi | P \equiv P_{1} | P_{2}, P_{1} \in \meaningof{E_{1}}, P_{2} \in \meaningof{E_2}\} }
\end{mathpar}

\begin{mathpar}
 \inferrule* [lab=behavior] {} {\meaningof{\langle a?b \rangle E} = \{ P \in \pi | P \equiv Q | u?(y)P', \\ \and \\\\ \and \\ \;\;\; u \in \meaningof{a}, \forall z.P'\{z/y\} \in \meaningof{E\{z/b\}}\}, \and \\ \meaningof{a!E} = \{ P \in \pi | P \equiv Q | x!\langle P' \rangle, x \in \meaningof{a} P' \in \meaningof{E}\} }
\end{mathpar}

\begin{mathpar}
 \inferrule* [lab=nominal] {} {\meaningof{\quotep{E}} = \{ \quotep{P} \in \quotep{\pi} | P \in \meaningof{E} \}, \and \meaningof{\quotep{P}} = \{ \quotep{Q} \in \quotep{\pi} | P \equiv Q \} \and \\ \meaningof{@\quotep{E}} = \{ P \in \pi | P \equiv @x, x \in \meaningof{E} \}}
\end{mathpar}

\begin{eqnarray*}
  \\
  \meaningof{-} : TS \to ST
\end{eqnarray*}

\begin{eqnarray*}
  \\
  L : TS \to ST
\end{eqnarray*}

\begin{eqnarray*}
  \\
  P \models E \iff P \in \meaningof{E}
\end{eqnarray*}

\begin{eqnarray*}
  P \approx_{L} Q \iff \forall E \in L. P \models E \iff Q \models E
\end{eqnarray*}

\begin{eqnarray*}
  P \approx_{K} Q
\end{eqnarray*}

\begin{eqnarray*}
  P \approx Q
\end{eqnarray*}

$\approx_{K} = \approx = \approx_{L}$

\subsubsection{Contextual duality}

Note that contexts extend the quotation operation to a family of
operations from processes to names. Given a context, $M$, we can
define a \emph{nominal context}, $\quotep{M}$ by $\quotep{M}[P] :=
\quotep{M[P]}$. To foreshadow what is to come we observe that these
operations enjoy a duality with processes very much like the duality
between vectors and maps from vectors to scalars.

Further, because the calculus is essentially higher-order, we have a
correspondence between contexts and processes. More specifically,
given a name $x$ and a context $M$ we can construct $M^{*}_{x}$ such
that 

\begin{mathpar}
  M^{*}_{x} | \lift{x}{P} \red M[P]
\end{mathpar}

namely,

\begin{mathpar}
  M^{*}_{x} := x?(u).M[\dropn{u}]
\end{mathpar}

The dependence of $M^{*}_{x}$ on a name makes it an abstraction, 

\begin{mathpar}
  M^{*} := (x)x?(u).M[\dropn{u}]
\end{mathpar}

\subsection{Additional notation}

It will sometimes be convenient to denote the process a name
quotes. We already have the notation $x = \quotep{P}$, but it will be
convenient to introduce an alternate notation, $\procn{x}$, when we
want to emphasize the connection to the use of the name. Note that, by
virtue of name equivalence, $\quotep{\procn{x}} \nameeq x$; so, the
notation is consistent with previous definitions.

Further, because names have structure it is possible to effect
substitutions on the basis of that structure. This means we need to
upgrade our notation for substitutions, which we accomplish by
adapting comprehension notation. Thus,

\begin{mathpar}
  P\{ y / x : x \in S \}
\end{mathpar}

is interpreted to mean the process derived from P by replacing (in a
capture-avoiding manner) each occurrence of $x$ in $S$ by $y$. For example,

\begin{mathpar}
  P\{ \quotep{\procn{x}|\procn{x}} / x : x \in \freenames{P} \}
\end{mathpar}

will replace each (occurrence) of a free name $x$ in $P$ by
$\quotep{\procn{x}|\procn{x}}$.

Also, we will avail ourselves of the notation $x^{L}$ and $x^{R}$ to
denote injections of a name into disjoint copies of the name
space. There are numerous ways to accomplish this. One example can be
found in \cite{MeredithR05}. This notation overloads to vectors of
names: $\vec{x}^{\pi} := (x_{i}^{\pi} \; : \; 0 \leq i < |\vec{x}| )$ where $\pi \in \{L,R\}$.

We also use $P^{\Box} := P|\Box$.

In \cite{MeredithR05} an interpretation of the new operator is
given. It turns out that there are several possible interpretations
all enjoying the requisite algebraic properties of the operator (see
\cite{milner91polyadicpi}). We will therefore make liberal use of
$(\nu\; \vec{x})P$.

% subsection the_syntax_and_semantics_of_the_notation_system (end)   

\input{qm2pi.qmops} 

\input{qm2pi.sterngerlach} 

\input{qm2pi.metric} 

% section concurrent_process_calculi (end)

%\input{qm2pi.proofsketch}

% section proof sketch (end)

%\input{qm2pi.slviaknots} 

% section spatial logic via knots (end)

\input{qm2pi.conclusion}

% section conclusion (end)

%\input{qm2pi.dtcodes} 

% section wiring algorithm (end)

\input{qm2pi.ack} 

% section acknowledgments (end)

\newpage


\bibliographystyle{plain}   
\bibliography{../../biblios/main.bib}

\input{qm2pi.rhodetails}

\end{document}

 

% section concurrent_process_calculi (end)

%\documentclass[12pt]{llncs}
%\documentclass{jktr}

\usepackage[pdftex]{hyperref}                   
\usepackage {listings}
\usepackage {mathpartir}
\usepackage{bcprules}
%\usepackage{listings}
                       
\usepackage{graphicx} 
%\usepackage[margins=2.5cm,nohead,nofoot]{geometry}
%\usepackage{geometry}
\usepackage{amsfonts}
\usepackage{amstext}
\usepackage{latexsym}
\usepackage{amssymb}
\usepackage{color}


%\include{myPreamble}
\include{qm2pi.local} 

%\ifpdf
%\usepackage[pdftex]{graphicx}
%\else
%\usepackage{graphicx}
%\fi

 % \ifpdf
%  \usepackage{pdfsync}
%  \if


%\title{Brief Article}
%\author{David F. Snyder}
%\author{L.G. Meredith}

%\address{Dept. of Math., Texas State University--San Marcos, San Marcos, TX 78666}
       
\pagestyle{empty}


\begin{document}

\lstset{language=[Objective]Caml,frame=shadowbox}

\input{qm2pi.front}

% section front matter (end)

\input{qm2pi.intro} 
 
% section introduction (end)

% \input{qm2pi.knotations} 

% section notation (end)

\input{qm2pi.process.calculi} 

% section concurrent_process_calculi_and_spatial_logics_ (end)
    
%\input{qm2pi.knots2pi} 

%\input{qm2pi.trefoil} 

%\input{qm2pi.mainthm} 

% subsection basic_interpretation (end)

%\input{qm2pi.rho.presentation} 
\subsection{The syntax and semantics of the notation system}\label{sub:the_syntax_and_semantics_of_the_notation_system} % (fold)

We now summarize a technical presentation of the calculus that
embodies our theory of dynamics. The typical presentation of such a
calculus follows the style of giving generators and relations on
them. The grammar, below, describing term constructors, freely
generates the set of processes, $\Proc$. This set is then quotiented
by a relation known as structural congruence and it is over this set
that the notion of dynamics is expressed. This presentation is
essentially that of \cite{MeredithR05} with the addition of
polyadicity and summation. For readability we have relegated some of
the technical subtleties to an appendix.

\subsubsection{Process grammar}\label{subsub:process_grammar}

\begin{mathpar}
  \inferrule* [lab=synchronization] {} {{M} \bc \pzero \;|\; x?F \;|\; x!C }
  \and
  \inferrule* [lab=abstraction] {} {{F} \bc (x)P}
  \and
  \inferrule* [lab=concretion] {} {{C} \bc \langle Q \rangle}
  \and
  \inferrule* [lab=process] {} {{P,Q} \bc M \;| \;P|Q \;|\; @{x}}
  \and
  \inferrule* [lab=name] {} {{x} \bc \quotep{P}}
\end{mathpar} 

Note that $\vec{x}$ (resp. $\vec{P}$) denotes a vector of names
(resp. processes) of length $|\vec{x}|$ (resp. $|\vec{P}|$). We adopt
the following useful abbreviations.

\begin{mathpar}
   x?(\vec{y}).P := x.(\vec{y})P \and  x\clift{\vec{P}} := x.\clift{\vec{P}}
   \and x!(y) := \lift{x}{\dropn{y}}
   \and \Pi_{i=0}^{n-1}P_i := P_0 | \ldots | P_{n-1}
\end{mathpar}

\subsubsection{Structural congruence}

\paragraph{Free and bound names and alpha-equivalence.} At the
core of structural equivalence is alpha-equivalence which identifies
process that are the same up to a change of variable. Formally, we
recognize the distinction between free and bound names. The free names
of a process, $\freenames{P}$, may be calculated recursively as
follows:

\begin{mathpar}
\freenames{\pzero} := \emptyset
  \and \\
  \freenames{x?(y).P} := \{ x \} \cup (\freenames{P} \setminus \{ y \})
  \and 
  \freenames{x!\langle P \rangle} := \{ x \} \cup \{ P \} 
  \and \\
  \freenames{P|Q} := \freenames{P} \cup \freenames{Q}
  \and \\
  \freenames{@{x}} := \{ x \}
\end{mathpar}

$\pi$
$\quotep{\pi}$

$\freenames{-} : \pi \to \mathcal{P}(\quotep{\pi})$

\begin{eqnarray*}
  \freenames{\pzero} & := & \emptyset \\
  \freenames{x?(y).P} & := & \{ x \} \cup (\freenames{P} \setminus \{ y \}) \\
  \freenames{x!\langle P \rangle} & := & \{ x \} \cup \{ P \} \\
  \freenames{P|Q} & := & \freenames{P} \cup \freenames{Q} \\
  \freenames{\dropn{x}} & := & \{ x \}
\end{eqnarray*}

The bound names of a process, $\boundnames{P}$, are those names occurring in $P$
that are not free. For example, in $x?(y).0$, the name $x$ is free, while $y$ is bound.

\begin{mathpar}
  \inferrule* [lab=monoidal-laws] {} { P|Q \equiv Q|P \and P|0 \equiv P \and P|(Q|R) \equiv (P|Q)|R }
\end{mathpar}

\begin{mathpar}
  \inferrule* [lab=alpha-equivalence] {} { (x)P \equiv (y)P\{y/x\} \and y \not\in \freenames{P} }
\end{mathpar}

\begin{definition}
Then two processes, $P,Q$, are alpha-equivalent if $P = Q\{\vec{y}/\vec{x}\}$ for
some $\vec{x} \in \boundnames{Q},\vec{y} \in \boundnames{P}$, where $Q\{\vec{y}/\vec{x}\}$
denotes the capture-avoiding substitution of $\vec{y}$ for $\vec{x}$ in $Q$.
\end{definition}

\begin{definition}
  The {\em structural congruence} \cite{SangiorgiWalker} , $\equiv$,
  between processes is the least congruence containing
  alpha-equivalence, satisfying the abelian monoid laws
  (associativity, commutativity and $\pzero$ as identity) for parallel
  composition $|$ and for summation $+$.
\end{definition}

\subsection{Name equivalence}

We take name equivalence, written $\nameeq$, to be the smallest
equivalence relation generated by the following rules.

\begin{mathpar}
\inferrule*[lab=Quote-drop]
{ }
{ \quotep{@{x}} \nameeq x }

\inferrule*[lab=Struct-equiv]
{ P \scong Q }
{ \quotep{P} \nameeq \quotep{Q} }
\end{mathpar}

The astute reader will have noticed that the mutual recursion of names
and processes imposes a mutual recursion on alpha-equivalence and
structural equivalence via name-equivalence. Fortunately, all of this
works out pleasantly and we may calculate in the natural way, free of
concern. The reader interested in the details is referred to the
appendix \ref{appendix:rho_details}.

\subsection{Substitution}

We use $\Proc$ for the set of processes, $\QProc$ for the set of
names, and $\id{\{}\vec{y} / \vec{x} \id{\}}$ to denote partial maps,
$s : \QProc \rightarrow \QProc$. A map, $s$ lifts, uniquely, to a map
on process terms, $\widehat{s} : \Proc \rightarrow \Proc$ by the
following equations.

\begin{mathpar}
  (0) \psubstp{Q}{P} := 0 \\
  (R \juxtap S) \psubstp{Q}{P}
  :=    
  (R)\psubstp{Q}{P} \juxtap (S) \psubstp{Q}{P} \\
  (x?(y).R) \psubstp{Q}{P}    
  :=    
  (x)\substp{Q}{P} (z)\concat( (R \psubstn{z}{y}) \psubstp{Q}{P} ) \\
  (\lift{x}{R}) \psubstp{Q}{P}  
  :=
  \lift{(x)\substp{Q}{P}}{ R \psubstp{Q}{P} } \\
%   (\dropn{x})  \psubstp{Q}{P}       
%   := 
%   \left\{ 
%     \begin{array}{ccc} 
%       \dropn{\quotep{Q}} & & x \nameeq \quotep{P} \\
%       \dropn{x} & & otherwise \\
%     \end{array}
%   \right. 
  (\dropn{x})  \psubstp{Q}{P}       
  := 
  \left\{ 
    \begin{array}{ccc} 
      Q & & x \nameeq \quotep{P} \\
      \dropn{x} & & otherwise \\
    \end{array}
  \right.
\end{mathpar}
 

where

\begin{eqnarray}
  (x)\id{\{} \lpquote Q \rpquote / \lpquote P \rpquote \id{\}}            = 
  \left\{ 
    \begin{array}{ccc}
      \lpquote Q \rpquote & & x \nameeq \lpquote P \rpquote \\
      x & & otherwise \\
    \end{array}
  \right. \nonumber
\end{eqnarray}

and $z$ is chosen distinct from $\quotep{P}$, $\quotep{Q}$, the free
names in $Q$, and all the names in $R$. Our $\alpha$-equivalence will
be built in the standard way from this substitution.

\begin{remark}\label{rem:no_self_referential_names}
  One consequence of these definitions is that $\forall P. \quotep{P}
  \not\in \freenames{P}$.
\end{remark}

\subsection{ Dynamic quote: an example }

Anticipating something of what's to come, consider applying the
substitution, $\widehat{\id{\{}u / z \id{\}}}$, to the following pair
of processes, $\lift{w}{y!(z)}$ and $w[ \lpquote y!(z) \rpquote ]$.

\begin{eqnarray}
	\lift{w}{y!(z)}\widehat{\id{\{}u / z \id{\}}}
		& = &
		\lift{w}{y!(u)} \nonumber\\
	w[ \lpquote y!(z) \rpquote ] \widehat{ \id{\{}u / z \id{\}} }
		& = &
		w[ \lpquote y!(z) \rpquote ] \nonumber
\end{eqnarray}

Because the body of the process between quotes is impervious to
substitution, we get radically different answers. In fact, by
examining the first process in an input context,
e.g. $x?(z).\lift{w}{y!(z)}$, we see that the process under the lift
operator may be shaped by prefixed inputs binding a name inside it. In
this sense, the lift operator will be seen as a way to dynamically
construct processes before reifying them as names.

Finally equipped with these standard features we can present the
dynamics of the calculus.

\subsubsection{Operational semantics} 

Finally, we introduce the computational dynamics. What marks these
algebras as distinct from other more traditionally studied algebraic
structures, e.g. vector spaces or polynomial rings, is the manner in
which dynamics is captured. In traditional structures, dynamics is typically
expressed through morphisms between such structures, as in linear maps
between vector spaces or morphisms between rings. In algebras
associated with the semantics of computation, the dynamics is
expressed as part of the algebraic structure itself, through a
reduction reduction relation typically denoted by $\red$. Below, we
give a recursive presentation of this relation for the calculus used
in the encoding.

$\red \subseteq \pi \times \pi$
$\red : \pi \to \mathcal{P}(\pi)$

\begin{mathpar}
  \inferrule* [lab=Comm] { \textsf{match}( x_{src}, x_{trgt} ) } { x_{trgt}?(y)P \; | \; x_{src}!\langle {Q} \rangle \red P\{\quotep{Q}/y}\} }
  \and \\
  \inferrule* [lab=Par] {{P} \red {P}'} {{{P} | {Q}} \red {{P}' | {Q}}}
  \and
  \inferrule* [lab=Equiv]{{{P} \scong {P}'} \andalso {{P}' \red {Q}'} \andalso {{Q}' \scong {Q}}}{{P} \red {Q}}
\end{mathpar}

\begin{eqnarray*}
  match_{\equiv} (\quotep{P},\quotep{Q}) & := & P \equiv Q \\
  match_{\dagger}(\quotep{P},\quotep{Q}) & := & \forall R. P|Q \red^{*} R => R \red^{*} 0 \\
  match_{K}(\quotep{P},\quotep{Q}) & := & K \mbox{ for some context } K
\end{eqnarray*}

$u?(x)P | u!\langle Q \rangle \red P\{\quotep{Q}/x\}$

%We write $\wred$ for $\red^*$, and $P\red$ if $\exists Q $ such that $ P \red Q$.
We write $P\red$ if $\exists Q $ such that $ P \red Q$ and $P\not\red$, otherwise.

\section{Replication}

As mentioned before, it is known that replication (and hence
recursion) can be implemented in a higher-order process algebra
\cite{SangiorgiWalker}. As our first example of calculation with the
machinery thus far presented we give the construction explicitly in
the {\rhoc}.

\begin{eqnarray}
	D_{x} & := & \prefix{x}{y}{(\binpar{\outputp{x}{y}}{@{y}})} \nonumber\\
	\bangp_{x}{P} & := & \binpar{{x}!\langle{\binpar{D_{x}}{P}}\rangle}{D_{x}} \nonumber
\end{eqnarray}

\begin{eqnarray}
	\bangp_{x}{P} & & \nonumber\\
	=
	& {x}!\langle{(\prefix{x}{y}{(\outputp{x}{y} | @{y})) | P}}\rangle 
	      | \prefix{x}{y}{(\outputp{x}{y} | @{y})} & \nonumber\\
	\red
	& (\outputp{x}{y} | @{y})\substn{\quotep{(\prefix{x}{y}{(@{y} | \outputp{x}{y})) | P}}}{y} & \nonumber\\
	=
	& \outputp{x}{\quotep{(\prefix{x}{y}{(\outputp{x}{y} | @{y})) | P}}}
	  | {(\prefix{x}{y}{(\outputp{x}{y} | @{y})) | P}} & \nonumber\\
	\red
	& \ldots & \nonumber\\
	\red^*
	& P | P | \ldots & \nonumber
\end{eqnarray}

Of course, this encoding, as an implementation, runs away, unfolding
$\bangp{P}$ eagerly. A lazier and more implementable replication
operator, restricted to input-guarded processes, may be obtained as follows.

\begin{eqnarray}
\bangp{\prefix{u}{v}{P}} 
	:= 
	\binpar{\lift{x}{\prefix{u}{v}{(\binpar{D(x)}{P})}}}{D(x)} \nonumber
\end{eqnarray}

\begin{remark}
  Note that the lazier definition still does not deal with summation
  or mixed summation (i.e. sums over input and output). The reader is
  invited to construct definitions of replication that deal with these
  features. 

  Further, the definitions are parameterized in a name, $x$. Can you,
  gentle reader, make a definition that eliminates this parameter and
  guarantees no accidental interaction between the replication
  machinery and the process being replicated -- i.e. no accidental
  sharing of names used by the process to get its work done and the
  name(s) used by the replication to effect copying. This latter
  revision of the definition of replication is crucial to obtaining
  the expected identity $!!P \sim !P$.
\end{remark}

\begin{remark}\label{rem:paradoxical_combinator}
  The reader familiar with the lambda calculus will have noticed the
  similarity between $D$ and the paradoxical combinator.

  [Ed. note: the existence of this seems to suggest we have to be more
  restrictive on the set of processes and names we admit if we are to
  support no-cloning.]
\end{remark}

\subsubsection{Bisimulation}

The computational dynamics gives rise to another kind of equivalence,
the equivalence of computational behavior. As previously mentioned
this is typically captured \emph{via} some form of bisimulation.

% The notion we use in this paper is weak barbed bisimulation
% \cite{milner91polyadicpi}.

The notion we use in this paper is derived from weak barbed
bisimulation \cite{milner91polyadicpi}. 

\begin{definition}
An \emph{observation relation}, $\downarrow_{\mathcal N}$, over a set
of names, $\mathcal N$, is the smallest relation satisfying the rules
below.

\infrule[Out-barb]{y \in {\mathcal N}, \; x \nameeq y}
		  {\outputp{x}{v} \downarrow_{\mathcal N} x}
\infrule[Par-barb]{\mbox{$P\downarrow_{\mathcal N} x$ or $Q\downarrow_{\mathcal N} x$}}
		  {\binpar{P}{Q} \downarrow_{\mathcal N} x}

We write $P \Downarrow_{\mathcal N} x$ if there is $Q$ such that 
$P \wred Q$ and $Q \downarrow_{\mathcal N} x$.
\end{definition}

\begin{definition}
%\label{def.bbisim}
An  ${\mathcal N}$-\emph{barbed bisimulation} over a set of names, ${\mathcal N}$, is a symmetric binary relation 
${\mathcal S}_{\mathcal N}$ between agents such that $P\rel{S}_{\mathcal N}Q$ implies:
\begin{enumerate}
\item If $P \red P'$ then $Q \wred Q'$ and $P'\rel{S}_{\mathcal N} Q'$.
\item If $P\downarrow_{\mathcal N} x$, then $Q\Downarrow_{\mathcal N} x$.
\end{enumerate}
$P$ is ${\mathcal N}$-barbed bisimilar to $Q$, written
$P \wbbisim_{\mathcal N} Q$, if $P \rel{S}_{\mathcal N} Q$ for some ${\mathcal N}$-barbed bisimulation ${\mathcal S}_{\mathcal N}$.
\end{definition}

$\mathcal{R} \subseteq \pi \times \pi$

$P \mathcal{R} Q => \forall P'. P \red P' \Rightarrow \exists Q'. Q \red Q', P' \mathcal{R} Q'$

$P \vdash x \Rightarrow Q \vdash x$

\begin{mathpar}
  \inferrule*[lab=Out-barb]{x \nameeq y}{{y}!\langle{Q}\rangle \vdash x}
  \and
  \inferrule*[lab=Par-barb]{\mbox{$P\vdash x$ or $Q\vdash x$}}{\binpar{P}{Q} \vdash x}
\end{mathpar}

\subsubsection{Contexts}

One of the principle advantages of computational calculi like the
$\pi$-calculus is a well-defined notion of context,
contextual-equivalence and a correlation between
contextual-equivalence and notions of bisimulation. The notion of
context allows the decomposition of a process into (sub-)process and
its syntactic environment, its context. Thus, a context may be
thought of as a process with a ``hole'' (written $\Box$) in it. The
application of a context $M$ to a process $P$, written $M[P]$, is
tantamount to filling the hole in $M$ with $P$. In this paper we do
not need the full weight of this theory, but do make use of the notion
of context in the proof the main theorem. 

\begin{mathpar}
  \inferrule* [lab=summation] {} {{M_{M},M_{N}} \bc \Box \;|\; x.M_{A} \;|\; M_{M}+M_{N}}
  \and
  \inferrule* [lab=agent] {} {{M_{A}} \bc (\vec{x})M_{P} \;| \; \clift{P_0,\ldots,M_{P},\ldots,P_N}}
  \and \\
  \inferrule* [lab=process] {} {{M_{P}} \bc M_{N} \;| \;P|M_{P} }
\end{mathpar} 

\begin{mathpar}
  \inferrule* [lab=sychronization] {} {M_{N} \bc \Box \;|\; x?M_{F} \;|\; x!M_{C}}
  \and
  \inferrule* [lab=abstraction] {} {{M_{F}} \bc (x)M_{P} }
  \and
  \inferrule* [lab=concretion] {} {{M_{C}} \bc \langle M_{P} \rangle }
  \and \\
  \inferrule* [lab=process] {} {{M_{P}} \bc M_{N} \;| \;P|M_{P} }
\end{mathpar}

\begin{definition}[contextual application] Given a context $M$, and
  process $P$, we define the \emph{contextual application}, $M[P] :=
  M\{P/\Box\}$. That is, the contextual application of M to P is the
  substitution of $P$ for $\Box$ in $M$.
\end{definition}

$\meaningof{-} : L \to \mathcal{P}(\pi)$

\begin{mathpar}
  \inferrule* [lab=collection] {} {\meaningof{true} = \pi, \and \meaningof{~E} = \pi \setminus \meaningof{E}, \and \meaningof{E_{1} \& E_{2}} = \meaningof{E_{1}} \cap \meaningof{E_{2}}}
\end{mathpar}

\begin{mathpar}
  \inferrule* [lab=structure] {} {\meaningof{0} = \{ P \in \pi | P \equiv 0 \}, \and \\ \meaningof{E_1 | E_2} = \{ P \in \pi | P \equiv P_{1} | P_{2}, P_{1} \in \meaningof{E_{1}}, P_{2} \in \meaningof{E_2}\} }
\end{mathpar}

\begin{mathpar}
 \inferrule* [lab=behavior] {} {\meaningof{\langle a?b \rangle E} = \{ P \in \pi | P \equiv Q | u?(y)P', \\ \and \\\\ \and \\ \;\;\; u \in \meaningof{a}, \forall z.P'\{z/y\} \in \meaningof{E\{z/b\}}\}, \and \\ \meaningof{a!E} = \{ P \in \pi | P \equiv Q | x!\langle P' \rangle, x \in \meaningof{a} P' \in \meaningof{E}\} }
\end{mathpar}

\begin{mathpar}
 \inferrule* [lab=nominal] {} {\meaningof{\quotep{E}} = \{ \quotep{P} \in \quotep{\pi} | P \in \meaningof{E} \}, \and \meaningof{\quotep{P}} = \{ \quotep{Q} \in \quotep{\pi} | P \equiv Q \} \and \\ \meaningof{@\quotep{E}} = \{ P \in \pi | P \equiv @x, x \in \meaningof{E} \}}
\end{mathpar}

\begin{eqnarray*}
  \\
  \meaningof{-} : TS \to ST
\end{eqnarray*}

\begin{eqnarray*}
  \\
  L : TS \to ST
\end{eqnarray*}

\begin{eqnarray*}
  \\
  P \models E \iff P \in \meaningof{E}
\end{eqnarray*}

\begin{eqnarray*}
  P \approx_{L} Q \iff \forall E \in L. P \models E \iff Q \models E
\end{eqnarray*}

\begin{eqnarray*}
  P \approx_{K} Q
\end{eqnarray*}

\begin{eqnarray*}
  P \approx Q
\end{eqnarray*}

$\approx_{K} = \approx = \approx_{L}$

\subsubsection{Contextual duality}

Note that contexts extend the quotation operation to a family of
operations from processes to names. Given a context, $M$, we can
define a \emph{nominal context}, $\quotep{M}$ by $\quotep{M}[P] :=
\quotep{M[P]}$. To foreshadow what is to come we observe that these
operations enjoy a duality with processes very much like the duality
between vectors and maps from vectors to scalars.

Further, because the calculus is essentially higher-order, we have a
correspondence between contexts and processes. More specifically,
given a name $x$ and a context $M$ we can construct $M^{*}_{x}$ such
that 

\begin{mathpar}
  M^{*}_{x} | \lift{x}{P} \red M[P]
\end{mathpar}

namely,

\begin{mathpar}
  M^{*}_{x} := x?(u).M[\dropn{u}]
\end{mathpar}

The dependence of $M^{*}_{x}$ on a name makes it an abstraction, 

\begin{mathpar}
  M^{*} := (x)x?(u).M[\dropn{u}]
\end{mathpar}

\subsection{Additional notation}

It will sometimes be convenient to denote the process a name
quotes. We already have the notation $x = \quotep{P}$, but it will be
convenient to introduce an alternate notation, $\procn{x}$, when we
want to emphasize the connection to the use of the name. Note that, by
virtue of name equivalence, $\quotep{\procn{x}} \nameeq x$; so, the
notation is consistent with previous definitions.

Further, because names have structure it is possible to effect
substitutions on the basis of that structure. This means we need to
upgrade our notation for substitutions, which we accomplish by
adapting comprehension notation. Thus,

\begin{mathpar}
  P\{ y / x : x \in S \}
\end{mathpar}

is interpreted to mean the process derived from P by replacing (in a
capture-avoiding manner) each occurrence of $x$ in $S$ by $y$. For example,

\begin{mathpar}
  P\{ \quotep{\procn{x}|\procn{x}} / x : x \in \freenames{P} \}
\end{mathpar}

will replace each (occurrence) of a free name $x$ in $P$ by
$\quotep{\procn{x}|\procn{x}}$.

Also, we will avail ourselves of the notation $x^{L}$ and $x^{R}$ to
denote injections of a name into disjoint copies of the name
space. There are numerous ways to accomplish this. One example can be
found in \cite{MeredithR05}. This notation overloads to vectors of
names: $\vec{x}^{\pi} := (x_{i}^{\pi} \; : \; 0 \leq i < |\vec{x}| )$ where $\pi \in \{L,R\}$.

We also use $P^{\Box} := P|\Box$.

In \cite{MeredithR05} an interpretation of the new operator is
given. It turns out that there are several possible interpretations
all enjoying the requisite algebraic properties of the operator (see
\cite{milner91polyadicpi}). We will therefore make liberal use of
$(\nu\; \vec{x})P$.

% subsection the_syntax_and_semantics_of_the_notation_system (end)   

\input{qm2pi.qmops} 

\input{qm2pi.sterngerlach} 

\input{qm2pi.metric} 

% section concurrent_process_calculi (end)

%\input{qm2pi.proofsketch}

% section proof sketch (end)

%\input{qm2pi.slviaknots} 

% section spatial logic via knots (end)

\input{qm2pi.conclusion}

% section conclusion (end)

%\input{qm2pi.dtcodes} 

% section wiring algorithm (end)

\input{qm2pi.ack} 

% section acknowledgments (end)

\newpage


\bibliographystyle{plain}   
\bibliography{../../biblios/main.bib}

\input{qm2pi.rhodetails}

\end{document}



% section proof sketch (end)

%\section{Unlikely characters: spatial logic for
  knots}\label{sub:characteristic_formulae} % (fold)

Associated to the mobile process calculi are a family of logics known
as the Hennessy-Milner logics. These logics typically enjoy a
semantics interpreting formulae as sets of processes that when
factored through the encoding outlined above allows an identification
of classes of knots with logical formulae. In the context of this
encoding the sub-family known as the spatial logics \cite{CairesC03}
\cite{CairesC04} \cite{Caires04} are of particular interest providing
several important features for expressing and reasoning about
properties (i.e. classes) of knots. We hint here at how this may be done.

%\begin{description}
%\item [structural connectives] 
\subsubsection{Structural connectives} The spatial logics enjoy
structural connectives corresponding, at the logical level, to the
parallel composition ($P | Q$) and new name ($(\nu \; x)P$)
connectives for processes. As illustrated in the examples below, these
connectives are extremely expressive given the shape of our encoding.
%\item [decideable satisfaction]

\subsubsection{Decideable satisfaction}
In \cite{Caires04} the satisfaction relation is shown to be decideable
for a rich class of processes. It further turns out that the image of
the our encoding is a proper subset of that class. This result
provides the basis for an algorithm by which to search for knots
enjoying a given property.
%\item [characteristic formulae]

\subsubsection{Characteristic formulae}
In the same paper \cite{Caires04} , Caires presents a means of calculating
characteristic formulae, selecting equivalence classes of processes
up to a pre--specified depth limit on the support set of names. Composed with our
encoding, this characteristic formula can be used to select
characteristic formulae for knots.
%\end{description}

\subsubsection{Spatial logic formulae}

The grammar below (segmented for comprehension) summarizes the syntax
of spatial logic formulae. We employ illustrative examples in the
sequel to provide an intuitive understanding of their meaning
referring the reader to \cite{Caires04} for a more detailed explication
of the semantics.

\begin{mathpar}
  \inferrule* [lab=boolean] {} {{A,B} \bc T \;|\; \neg A \;|\; A \wedge B \;|\; \eta = \eta'}
  \and
  \inferrule* [lab=spatial] {} {|\; \pzero \;|\; A | B \;|\; x \text{\textregistered} A \;|\; \forall x . A \;|\;  H x . A}
  \and
  \inferrule* [lab=behavioral] {} {|\; \alpha . A}
  \and 
  \inferrule* [lab=recursion] {} {|\; X(\vec{u}) \;|\; \mu X(\vec{u}) . A}
  \and
  \inferrule* [lab=action] {} {\alpha \bc \langle x?(\vec{y}) \rangle \;|\; \langle x!(\vec{y}) \rangle \;|\; \langle \tau \rangle}
  \and 
  \inferrule* [lab=name] {} {\eta \bc x \;|\; \tau}
\end{mathpar} 

% subsection characteristic_formulae (end)   	 

\subsection{Example formulae}\label{sub:example_formulae_} % (fold)

\subsubsection{Crossing as formula.}
% 
% \begin{align*}
%   \frac{d}{dx} \sin x &= \cos x 
%   & \frac{d}{dx} e^x &= e^x \\
%   \frac{d}{dx} \cos x &= - \sin x 
%   & \frac{d}{dx} \log x &= \frac{1}{x} \\
% \end{align*} 

\begin{align*}
 \mu C(x_{0},x_{1},y_{0},y_{1},u).&(\langle x_{0}?(z) \rangle(\langle u! \rangle\langle y_{1}!z \rangle C(x_{0},x_{1},y_{0},y_{1},u)) & \\
  & \wedge \langle y_{1}?(z) \rangle (\langle u! \rangle \langle x_{0}!z \rangle C(x_{0},x_{1},y_{0},y_{1},u)) & \\
  & \wedge \langle x_{1}?(z) \rangle (\langle u? \rangle \langle y_{0}!z \rangle C(x_{0},x_{1},y_{0},y_{1},u)) & \\
  & \wedge \langle y_{0}?(z) \rangle (\langle u? \rangle \langle x_{1}!z \rangle C(x_{0},x_{1},y_{0},y_{1},u))) &
\end{align*}

The lexicographical similarity between the shape of this formulae and
the shape of definition of the process representing a crossing reveals
the intuitive meaning of this formulae. It describes the capabilities
of a process that has the right to represent a crossing. For example
it picks out processes that may perform an input on the port $x_0$ in
its initial menu of capabilities. What differentiates the formula
from the process, however, is that the crossing process is the
smallest candidate to satisfy the formula. Infinitely many other
processes -- with internal behavior hidden behind this interface, so
to speak -- also satisfy this formula. Even this simple formula,
then, can be seen to open a new view onto knots, providing a
computational interpretation of \emph{virtual} knots.

Note that this formula is derived by hand. A similar formula can be
derived by employing Caires' calculation of characteristic formula
\cite{Caires04} to the process representing a crossing. In light of
this discussion, we let
$\meaningof{C}_{\phi}(x0,x1,y0,y1,u)$ denote a formula specifying the
dynamics we wish to capture of a crossing. To guarantee we preserve
the shape of the interface and minimal semantics we demand that
$\meaningof{C}_{\phi}(x0,x1,y0,y1,u) \Rightarrow
\textbf{C}(x0,x1,y0,y1,u)$ where $\textbf{C}(x0,x1,y0,y1,u)$ denotes
the formula above.
                            
\subsubsection{Crossing number constraints.}
The moral content of the context lemma (Lemma \ref{context}) is that the notion of
``locality'' in the Reidemeister moves is effectively captured by the
parallel composition operator of the process calculus. This intuition
extends through the logic. Given a formula,
$\meaningof{C}_{\phi}(x0,x1,y0,y1,u)$, we can use the structural
connectives to specify constraints on crossing numbers, such as at
least $n$ crossings, or exactly $n$ crossings.
\begin{mathpar}
  \inferrule* [lab=at-least-n] {} { K^{\geq n}_{\phi}(\vec{xs},\vec{ys}) := \Pi_{i=0}^{n-1} Hu . \meaningof{C}_{\phi}(xs_i,ys_i,u) | T }
  \and 
  \inferrule* [lab=exactly-n] {} { K^{= n}_{\phi}(\vec{xs},\vec{ys}) := \Pi_{i=0}^{n-1} Hu . \meaningof{C}_{\phi}(xs_i,ys_i,u) | \neg (\forall x_0,y_0,x_1,y_1,u . \meaningof{C}_{\phi}(x_0,y_0,x_1,y_1,u) | T) }
\end{mathpar}

To round out this section, recall that the encoding of an $n$-crossing
knot decomposes into a parallel composition of $n$ \emph{copies} of a
crossing process together with a wiring harness. To specify different
knot classes with the same crossing number amounts to specifying
logical constraints on the wiring harness. In the interest of space,
we defer examples to a forthcoming paper. Suffice it to say that both
the conditions ``alternating knot'' and ``contains the tangle
corresponding to 5/3'' are expressible. For example, it is possible to
calculate the characteristic formula of a process corresponding to the
tangle 5/3 and conjoin it into the classifying formula via the
composition connective of the logic.

Finally, we wish to observe that it is entirely within reason to
contemplate a more domain-specific version of spatial logic tailored
to the shape of processes in the image of the encoding. Such a
domain-specific logic would have a better claim to the title formal
language of knot properties.

% subsection example_formulae_ (end)

% section knots_as_processes (end) 

% section spatial logic via knots (end)

\section{Conclusions and future work}

\paragraph{Testing physical space}
You, gentle reader, may wonder why of all the theorems to be proved
given this set up we pick the one above. In some sense it's hardly
central to quantum mechanics. We see it as central in the sense that
it firmly establishes a notion of physical space arising from a notion
of the equivalence of behavior. Relating bisimulation to a metric is a
big step forward, but one is faced with interpreting the relationship
of that metric space to something more physical. Quantum mechanical
notions of ``physical'' space are still far from intuitive, but by
relating this idea of distance as testing to calculations that predict
physical circumstances we are making a not insignificant step forward
toward an understanding of the physical space we inhabit as
essentially dynamic.

\paragraph{Effectivity and simulation}
One of the observations we have yet to make is that the entire program
spelled out here is effective. We have built various interpreters for
the reflective calculus at work in this interpretation. In principle,
then, we can simulate quantum mechanics on a computer. The place where
the simulation may lose fidelity is the infinitely branching summation
for the annihilator.

In this connection i also want to point out that the evaluation style
calculation of the inner product puts the non-determinism of the
summation right at the heart of measurement. This suggests that
Milner's original reduction-based formulation of the dynamics of his
calculi in terms of sums was not just notationally suggestive of a
notion of measure-and-continue but captured some significant part of
the physics.

\paragraph{Quantum continuations}
In light of this last observation i want to point out that the
predominant account of quantum mechanics is missing a key aspect of a
truly compositional story of the physical situation. In a real lab,
when a measurement is made the observation can be made to feed into
another device that then makes another measurement conditioned on the
results of the first. This means that after the superposition was
collapsed the entire experimental set up remained in
superposition. While QM offers a means of writing this down it doesn't
quite line up well with the well-trodden formulation of computation
and continuation that we see so succinctly expressed in Milner's
calculi. This suggests that there might be advantages to this account
of dynamics waiting to be explored.

\paragraph{Quantum logic}
In this connection, we also note that by virtue of having the
Hennessy-Milner construction, we can pull the construction through the
interpretation of QM. This gives us a natural candidate for a quantum
logic that enjoys an extremely tight connection with it's domain of
interpretation, making the construction much less ad hoc (rather it is
the image of functor!).

\paragraph{Quantum probabiity}
i have questions about the basis of the interpretation of inner
product as probability amplitude. In particular, using which
axiomatization of probability theory does the notion of probability
amplitude earn the right to be so dubbed? In other words, where is the
proof that the operation for calculating a probability amplitude (and
then squaring) satisfies the axioms of what it means to calculate a
probability? Even if such a proof exists (i have yet to find it in the
literature), i wonder if it might not be possible to turn things on
their heads. Can we view the calculation of the probability amplitude
as an axiomatization of probability? If so, then the definition we
give for calculating probability amplitude may provide the basis for
an \emph{effective} theory of probability.

\paragraph{Quantum vs ``biological'' information}
Finally, i want to conclude with a more philosophical observation. At
a recent workshop in which QM was a predominant topic i noticed
something about quantum information. The speaker was giving a riveting
discussion of axiomatic QM and showing how properties of ``no
cloning'' and ``no deleting'' emerged as consequences of the
axiomatization. Theorems of this form are necessary to give us a sense
of confidence that our axioms characterize the physical theory. What
struck me, though, was that if quantum information is neither erasable
nor replicable it is markedly different from \emph{life}. Two of the
things we know about life is that

\begin{itemize}
  \item it ends;
  \item to gain some measure of persistence, to transcend it's
    finitude it is imminently copyable.
\end{itemize}

Both of these qualities are summarized succinctly in the aphorism: all
flesh is grass. For me these two kinds of ``information'' -- call them
quantum and biological -- are end points on a spectrum of strategies
for persistence. At one end, we have those curious entities that enjoy
uniqueness and permanence; at the other, we have those who in the face
of a certain end and an uncertain present make a go of passing
something on. To me one of the more remarkable aspects of the latter
strategy is that in the presence of noise (and certain features of
copying) we get a kind of dynamism, a chance for improvement against a
given persistent condition.

% subsection other_calculi_other_bisimulations_and_geometry_as_behavior (end)




% section conclusion (end)

%\documentclass[12pt]{llncs}
%\documentclass{jktr}

\usepackage[pdftex]{hyperref}                   
\usepackage {listings}
\usepackage {mathpartir}
\usepackage{bcprules}
%\usepackage{listings}
                       
\usepackage{graphicx} 
%\usepackage[margins=2.5cm,nohead,nofoot]{geometry}
%\usepackage{geometry}
\usepackage{amsfonts}
\usepackage{amstext}
\usepackage{latexsym}
\usepackage{amssymb}
\usepackage{color}


%\include{myPreamble}
\include{qm2pi.local} 

%\ifpdf
%\usepackage[pdftex]{graphicx}
%\else
%\usepackage{graphicx}
%\fi

 % \ifpdf
%  \usepackage{pdfsync}
%  \if


%\title{Brief Article}
%\author{David F. Snyder}
%\author{L.G. Meredith}

%\address{Dept. of Math., Texas State University--San Marcos, San Marcos, TX 78666}
       
\pagestyle{empty}


\begin{document}

\lstset{language=[Objective]Caml,frame=shadowbox}

\input{qm2pi.front}

% section front matter (end)

\input{qm2pi.intro} 
 
% section introduction (end)

% \input{qm2pi.knotations} 

% section notation (end)

\input{qm2pi.process.calculi} 

% section concurrent_process_calculi_and_spatial_logics_ (end)
    
%\input{qm2pi.knots2pi} 

%\input{qm2pi.trefoil} 

%\input{qm2pi.mainthm} 

% subsection basic_interpretation (end)

%\input{qm2pi.rho.presentation} 
\subsection{The syntax and semantics of the notation system}\label{sub:the_syntax_and_semantics_of_the_notation_system} % (fold)

We now summarize a technical presentation of the calculus that
embodies our theory of dynamics. The typical presentation of such a
calculus follows the style of giving generators and relations on
them. The grammar, below, describing term constructors, freely
generates the set of processes, $\Proc$. This set is then quotiented
by a relation known as structural congruence and it is over this set
that the notion of dynamics is expressed. This presentation is
essentially that of \cite{MeredithR05} with the addition of
polyadicity and summation. For readability we have relegated some of
the technical subtleties to an appendix.

\subsubsection{Process grammar}\label{subsub:process_grammar}

\begin{mathpar}
  \inferrule* [lab=synchronization] {} {{M} \bc \pzero \;|\; x?F \;|\; x!C }
  \and
  \inferrule* [lab=abstraction] {} {{F} \bc (x)P}
  \and
  \inferrule* [lab=concretion] {} {{C} \bc \langle Q \rangle}
  \and
  \inferrule* [lab=process] {} {{P,Q} \bc M \;| \;P|Q \;|\; @{x}}
  \and
  \inferrule* [lab=name] {} {{x} \bc \quotep{P}}
\end{mathpar} 

Note that $\vec{x}$ (resp. $\vec{P}$) denotes a vector of names
(resp. processes) of length $|\vec{x}|$ (resp. $|\vec{P}|$). We adopt
the following useful abbreviations.

\begin{mathpar}
   x?(\vec{y}).P := x.(\vec{y})P \and  x\clift{\vec{P}} := x.\clift{\vec{P}}
   \and x!(y) := \lift{x}{\dropn{y}}
   \and \Pi_{i=0}^{n-1}P_i := P_0 | \ldots | P_{n-1}
\end{mathpar}

\subsubsection{Structural congruence}

\paragraph{Free and bound names and alpha-equivalence.} At the
core of structural equivalence is alpha-equivalence which identifies
process that are the same up to a change of variable. Formally, we
recognize the distinction between free and bound names. The free names
of a process, $\freenames{P}$, may be calculated recursively as
follows:

\begin{mathpar}
\freenames{\pzero} := \emptyset
  \and \\
  \freenames{x?(y).P} := \{ x \} \cup (\freenames{P} \setminus \{ y \})
  \and 
  \freenames{x!\langle P \rangle} := \{ x \} \cup \{ P \} 
  \and \\
  \freenames{P|Q} := \freenames{P} \cup \freenames{Q}
  \and \\
  \freenames{@{x}} := \{ x \}
\end{mathpar}

$\pi$
$\quotep{\pi}$

$\freenames{-} : \pi \to \mathcal{P}(\quotep{\pi})$

\begin{eqnarray*}
  \freenames{\pzero} & := & \emptyset \\
  \freenames{x?(y).P} & := & \{ x \} \cup (\freenames{P} \setminus \{ y \}) \\
  \freenames{x!\langle P \rangle} & := & \{ x \} \cup \{ P \} \\
  \freenames{P|Q} & := & \freenames{P} \cup \freenames{Q} \\
  \freenames{\dropn{x}} & := & \{ x \}
\end{eqnarray*}

The bound names of a process, $\boundnames{P}$, are those names occurring in $P$
that are not free. For example, in $x?(y).0$, the name $x$ is free, while $y$ is bound.

\begin{mathpar}
  \inferrule* [lab=monoidal-laws] {} { P|Q \equiv Q|P \and P|0 \equiv P \and P|(Q|R) \equiv (P|Q)|R }
\end{mathpar}

\begin{mathpar}
  \inferrule* [lab=alpha-equivalence] {} { (x)P \equiv (y)P\{y/x\} \and y \not\in \freenames{P} }
\end{mathpar}

\begin{definition}
Then two processes, $P,Q$, are alpha-equivalent if $P = Q\{\vec{y}/\vec{x}\}$ for
some $\vec{x} \in \boundnames{Q},\vec{y} \in \boundnames{P}$, where $Q\{\vec{y}/\vec{x}\}$
denotes the capture-avoiding substitution of $\vec{y}$ for $\vec{x}$ in $Q$.
\end{definition}

\begin{definition}
  The {\em structural congruence} \cite{SangiorgiWalker} , $\equiv$,
  between processes is the least congruence containing
  alpha-equivalence, satisfying the abelian monoid laws
  (associativity, commutativity and $\pzero$ as identity) for parallel
  composition $|$ and for summation $+$.
\end{definition}

\subsection{Name equivalence}

We take name equivalence, written $\nameeq$, to be the smallest
equivalence relation generated by the following rules.

\begin{mathpar}
\inferrule*[lab=Quote-drop]
{ }
{ \quotep{@{x}} \nameeq x }

\inferrule*[lab=Struct-equiv]
{ P \scong Q }
{ \quotep{P} \nameeq \quotep{Q} }
\end{mathpar}

The astute reader will have noticed that the mutual recursion of names
and processes imposes a mutual recursion on alpha-equivalence and
structural equivalence via name-equivalence. Fortunately, all of this
works out pleasantly and we may calculate in the natural way, free of
concern. The reader interested in the details is referred to the
appendix \ref{appendix:rho_details}.

\subsection{Substitution}

We use $\Proc$ for the set of processes, $\QProc$ for the set of
names, and $\id{\{}\vec{y} / \vec{x} \id{\}}$ to denote partial maps,
$s : \QProc \rightarrow \QProc$. A map, $s$ lifts, uniquely, to a map
on process terms, $\widehat{s} : \Proc \rightarrow \Proc$ by the
following equations.

\begin{mathpar}
  (0) \psubstp{Q}{P} := 0 \\
  (R \juxtap S) \psubstp{Q}{P}
  :=    
  (R)\psubstp{Q}{P} \juxtap (S) \psubstp{Q}{P} \\
  (x?(y).R) \psubstp{Q}{P}    
  :=    
  (x)\substp{Q}{P} (z)\concat( (R \psubstn{z}{y}) \psubstp{Q}{P} ) \\
  (\lift{x}{R}) \psubstp{Q}{P}  
  :=
  \lift{(x)\substp{Q}{P}}{ R \psubstp{Q}{P} } \\
%   (\dropn{x})  \psubstp{Q}{P}       
%   := 
%   \left\{ 
%     \begin{array}{ccc} 
%       \dropn{\quotep{Q}} & & x \nameeq \quotep{P} \\
%       \dropn{x} & & otherwise \\
%     \end{array}
%   \right. 
  (\dropn{x})  \psubstp{Q}{P}       
  := 
  \left\{ 
    \begin{array}{ccc} 
      Q & & x \nameeq \quotep{P} \\
      \dropn{x} & & otherwise \\
    \end{array}
  \right.
\end{mathpar}
 

where

\begin{eqnarray}
  (x)\id{\{} \lpquote Q \rpquote / \lpquote P \rpquote \id{\}}            = 
  \left\{ 
    \begin{array}{ccc}
      \lpquote Q \rpquote & & x \nameeq \lpquote P \rpquote \\
      x & & otherwise \\
    \end{array}
  \right. \nonumber
\end{eqnarray}

and $z$ is chosen distinct from $\quotep{P}$, $\quotep{Q}$, the free
names in $Q$, and all the names in $R$. Our $\alpha$-equivalence will
be built in the standard way from this substitution.

\begin{remark}\label{rem:no_self_referential_names}
  One consequence of these definitions is that $\forall P. \quotep{P}
  \not\in \freenames{P}$.
\end{remark}

\subsection{ Dynamic quote: an example }

Anticipating something of what's to come, consider applying the
substitution, $\widehat{\id{\{}u / z \id{\}}}$, to the following pair
of processes, $\lift{w}{y!(z)}$ and $w[ \lpquote y!(z) \rpquote ]$.

\begin{eqnarray}
	\lift{w}{y!(z)}\widehat{\id{\{}u / z \id{\}}}
		& = &
		\lift{w}{y!(u)} \nonumber\\
	w[ \lpquote y!(z) \rpquote ] \widehat{ \id{\{}u / z \id{\}} }
		& = &
		w[ \lpquote y!(z) \rpquote ] \nonumber
\end{eqnarray}

Because the body of the process between quotes is impervious to
substitution, we get radically different answers. In fact, by
examining the first process in an input context,
e.g. $x?(z).\lift{w}{y!(z)}$, we see that the process under the lift
operator may be shaped by prefixed inputs binding a name inside it. In
this sense, the lift operator will be seen as a way to dynamically
construct processes before reifying them as names.

Finally equipped with these standard features we can present the
dynamics of the calculus.

\subsubsection{Operational semantics} 

Finally, we introduce the computational dynamics. What marks these
algebras as distinct from other more traditionally studied algebraic
structures, e.g. vector spaces or polynomial rings, is the manner in
which dynamics is captured. In traditional structures, dynamics is typically
expressed through morphisms between such structures, as in linear maps
between vector spaces or morphisms between rings. In algebras
associated with the semantics of computation, the dynamics is
expressed as part of the algebraic structure itself, through a
reduction reduction relation typically denoted by $\red$. Below, we
give a recursive presentation of this relation for the calculus used
in the encoding.

$\red \subseteq \pi \times \pi$
$\red : \pi \to \mathcal{P}(\pi)$

\begin{mathpar}
  \inferrule* [lab=Comm] { \textsf{match}( x_{src}, x_{trgt} ) } { x_{trgt}?(y)P \; | \; x_{src}!\langle {Q} \rangle \red P\{\quotep{Q}/y}\} }
  \and \\
  \inferrule* [lab=Par] {{P} \red {P}'} {{{P} | {Q}} \red {{P}' | {Q}}}
  \and
  \inferrule* [lab=Equiv]{{{P} \scong {P}'} \andalso {{P}' \red {Q}'} \andalso {{Q}' \scong {Q}}}{{P} \red {Q}}
\end{mathpar}

\begin{eqnarray*}
  match_{\equiv} (\quotep{P},\quotep{Q}) & := & P \equiv Q \\
  match_{\dagger}(\quotep{P},\quotep{Q}) & := & \forall R. P|Q \red^{*} R => R \red^{*} 0 \\
  match_{K}(\quotep{P},\quotep{Q}) & := & K \mbox{ for some context } K
\end{eqnarray*}

$u?(x)P | u!\langle Q \rangle \red P\{\quotep{Q}/x\}$

%We write $\wred$ for $\red^*$, and $P\red$ if $\exists Q $ such that $ P \red Q$.
We write $P\red$ if $\exists Q $ such that $ P \red Q$ and $P\not\red$, otherwise.

\section{Replication}

As mentioned before, it is known that replication (and hence
recursion) can be implemented in a higher-order process algebra
\cite{SangiorgiWalker}. As our first example of calculation with the
machinery thus far presented we give the construction explicitly in
the {\rhoc}.

\begin{eqnarray}
	D_{x} & := & \prefix{x}{y}{(\binpar{\outputp{x}{y}}{@{y}})} \nonumber\\
	\bangp_{x}{P} & := & \binpar{{x}!\langle{\binpar{D_{x}}{P}}\rangle}{D_{x}} \nonumber
\end{eqnarray}

\begin{eqnarray}
	\bangp_{x}{P} & & \nonumber\\
	=
	& {x}!\langle{(\prefix{x}{y}{(\outputp{x}{y} | @{y})) | P}}\rangle 
	      | \prefix{x}{y}{(\outputp{x}{y} | @{y})} & \nonumber\\
	\red
	& (\outputp{x}{y} | @{y})\substn{\quotep{(\prefix{x}{y}{(@{y} | \outputp{x}{y})) | P}}}{y} & \nonumber\\
	=
	& \outputp{x}{\quotep{(\prefix{x}{y}{(\outputp{x}{y} | @{y})) | P}}}
	  | {(\prefix{x}{y}{(\outputp{x}{y} | @{y})) | P}} & \nonumber\\
	\red
	& \ldots & \nonumber\\
	\red^*
	& P | P | \ldots & \nonumber
\end{eqnarray}

Of course, this encoding, as an implementation, runs away, unfolding
$\bangp{P}$ eagerly. A lazier and more implementable replication
operator, restricted to input-guarded processes, may be obtained as follows.

\begin{eqnarray}
\bangp{\prefix{u}{v}{P}} 
	:= 
	\binpar{\lift{x}{\prefix{u}{v}{(\binpar{D(x)}{P})}}}{D(x)} \nonumber
\end{eqnarray}

\begin{remark}
  Note that the lazier definition still does not deal with summation
  or mixed summation (i.e. sums over input and output). The reader is
  invited to construct definitions of replication that deal with these
  features. 

  Further, the definitions are parameterized in a name, $x$. Can you,
  gentle reader, make a definition that eliminates this parameter and
  guarantees no accidental interaction between the replication
  machinery and the process being replicated -- i.e. no accidental
  sharing of names used by the process to get its work done and the
  name(s) used by the replication to effect copying. This latter
  revision of the definition of replication is crucial to obtaining
  the expected identity $!!P \sim !P$.
\end{remark}

\begin{remark}\label{rem:paradoxical_combinator}
  The reader familiar with the lambda calculus will have noticed the
  similarity between $D$ and the paradoxical combinator.

  [Ed. note: the existence of this seems to suggest we have to be more
  restrictive on the set of processes and names we admit if we are to
  support no-cloning.]
\end{remark}

\subsubsection{Bisimulation}

The computational dynamics gives rise to another kind of equivalence,
the equivalence of computational behavior. As previously mentioned
this is typically captured \emph{via} some form of bisimulation.

% The notion we use in this paper is weak barbed bisimulation
% \cite{milner91polyadicpi}.

The notion we use in this paper is derived from weak barbed
bisimulation \cite{milner91polyadicpi}. 

\begin{definition}
An \emph{observation relation}, $\downarrow_{\mathcal N}$, over a set
of names, $\mathcal N$, is the smallest relation satisfying the rules
below.

\infrule[Out-barb]{y \in {\mathcal N}, \; x \nameeq y}
		  {\outputp{x}{v} \downarrow_{\mathcal N} x}
\infrule[Par-barb]{\mbox{$P\downarrow_{\mathcal N} x$ or $Q\downarrow_{\mathcal N} x$}}
		  {\binpar{P}{Q} \downarrow_{\mathcal N} x}

We write $P \Downarrow_{\mathcal N} x$ if there is $Q$ such that 
$P \wred Q$ and $Q \downarrow_{\mathcal N} x$.
\end{definition}

\begin{definition}
%\label{def.bbisim}
An  ${\mathcal N}$-\emph{barbed bisimulation} over a set of names, ${\mathcal N}$, is a symmetric binary relation 
${\mathcal S}_{\mathcal N}$ between agents such that $P\rel{S}_{\mathcal N}Q$ implies:
\begin{enumerate}
\item If $P \red P'$ then $Q \wred Q'$ and $P'\rel{S}_{\mathcal N} Q'$.
\item If $P\downarrow_{\mathcal N} x$, then $Q\Downarrow_{\mathcal N} x$.
\end{enumerate}
$P$ is ${\mathcal N}$-barbed bisimilar to $Q$, written
$P \wbbisim_{\mathcal N} Q$, if $P \rel{S}_{\mathcal N} Q$ for some ${\mathcal N}$-barbed bisimulation ${\mathcal S}_{\mathcal N}$.
\end{definition}

$\mathcal{R} \subseteq \pi \times \pi$

$P \mathcal{R} Q => \forall P'. P \red P' \Rightarrow \exists Q'. Q \red Q', P' \mathcal{R} Q'$

$P \vdash x \Rightarrow Q \vdash x$

\begin{mathpar}
  \inferrule*[lab=Out-barb]{x \nameeq y}{{y}!\langle{Q}\rangle \vdash x}
  \and
  \inferrule*[lab=Par-barb]{\mbox{$P\vdash x$ or $Q\vdash x$}}{\binpar{P}{Q} \vdash x}
\end{mathpar}

\subsubsection{Contexts}

One of the principle advantages of computational calculi like the
$\pi$-calculus is a well-defined notion of context,
contextual-equivalence and a correlation between
contextual-equivalence and notions of bisimulation. The notion of
context allows the decomposition of a process into (sub-)process and
its syntactic environment, its context. Thus, a context may be
thought of as a process with a ``hole'' (written $\Box$) in it. The
application of a context $M$ to a process $P$, written $M[P]$, is
tantamount to filling the hole in $M$ with $P$. In this paper we do
not need the full weight of this theory, but do make use of the notion
of context in the proof the main theorem. 

\begin{mathpar}
  \inferrule* [lab=summation] {} {{M_{M},M_{N}} \bc \Box \;|\; x.M_{A} \;|\; M_{M}+M_{N}}
  \and
  \inferrule* [lab=agent] {} {{M_{A}} \bc (\vec{x})M_{P} \;| \; \clift{P_0,\ldots,M_{P},\ldots,P_N}}
  \and \\
  \inferrule* [lab=process] {} {{M_{P}} \bc M_{N} \;| \;P|M_{P} }
\end{mathpar} 

\begin{mathpar}
  \inferrule* [lab=sychronization] {} {M_{N} \bc \Box \;|\; x?M_{F} \;|\; x!M_{C}}
  \and
  \inferrule* [lab=abstraction] {} {{M_{F}} \bc (x)M_{P} }
  \and
  \inferrule* [lab=concretion] {} {{M_{C}} \bc \langle M_{P} \rangle }
  \and \\
  \inferrule* [lab=process] {} {{M_{P}} \bc M_{N} \;| \;P|M_{P} }
\end{mathpar}

\begin{definition}[contextual application] Given a context $M$, and
  process $P$, we define the \emph{contextual application}, $M[P] :=
  M\{P/\Box\}$. That is, the contextual application of M to P is the
  substitution of $P$ for $\Box$ in $M$.
\end{definition}

$\meaningof{-} : L \to \mathcal{P}(\pi)$

\begin{mathpar}
  \inferrule* [lab=collection] {} {\meaningof{true} = \pi, \and \meaningof{~E} = \pi \setminus \meaningof{E}, \and \meaningof{E_{1} \& E_{2}} = \meaningof{E_{1}} \cap \meaningof{E_{2}}}
\end{mathpar}

\begin{mathpar}
  \inferrule* [lab=structure] {} {\meaningof{0} = \{ P \in \pi | P \equiv 0 \}, \and \\ \meaningof{E_1 | E_2} = \{ P \in \pi | P \equiv P_{1} | P_{2}, P_{1} \in \meaningof{E_{1}}, P_{2} \in \meaningof{E_2}\} }
\end{mathpar}

\begin{mathpar}
 \inferrule* [lab=behavior] {} {\meaningof{\langle a?b \rangle E} = \{ P \in \pi | P \equiv Q | u?(y)P', \\ \and \\\\ \and \\ \;\;\; u \in \meaningof{a}, \forall z.P'\{z/y\} \in \meaningof{E\{z/b\}}\}, \and \\ \meaningof{a!E} = \{ P \in \pi | P \equiv Q | x!\langle P' \rangle, x \in \meaningof{a} P' \in \meaningof{E}\} }
\end{mathpar}

\begin{mathpar}
 \inferrule* [lab=nominal] {} {\meaningof{\quotep{E}} = \{ \quotep{P} \in \quotep{\pi} | P \in \meaningof{E} \}, \and \meaningof{\quotep{P}} = \{ \quotep{Q} \in \quotep{\pi} | P \equiv Q \} \and \\ \meaningof{@\quotep{E}} = \{ P \in \pi | P \equiv @x, x \in \meaningof{E} \}}
\end{mathpar}

\begin{eqnarray*}
  \\
  \meaningof{-} : TS \to ST
\end{eqnarray*}

\begin{eqnarray*}
  \\
  L : TS \to ST
\end{eqnarray*}

\begin{eqnarray*}
  \\
  P \models E \iff P \in \meaningof{E}
\end{eqnarray*}

\begin{eqnarray*}
  P \approx_{L} Q \iff \forall E \in L. P \models E \iff Q \models E
\end{eqnarray*}

\begin{eqnarray*}
  P \approx_{K} Q
\end{eqnarray*}

\begin{eqnarray*}
  P \approx Q
\end{eqnarray*}

$\approx_{K} = \approx = \approx_{L}$

\subsubsection{Contextual duality}

Note that contexts extend the quotation operation to a family of
operations from processes to names. Given a context, $M$, we can
define a \emph{nominal context}, $\quotep{M}$ by $\quotep{M}[P] :=
\quotep{M[P]}$. To foreshadow what is to come we observe that these
operations enjoy a duality with processes very much like the duality
between vectors and maps from vectors to scalars.

Further, because the calculus is essentially higher-order, we have a
correspondence between contexts and processes. More specifically,
given a name $x$ and a context $M$ we can construct $M^{*}_{x}$ such
that 

\begin{mathpar}
  M^{*}_{x} | \lift{x}{P} \red M[P]
\end{mathpar}

namely,

\begin{mathpar}
  M^{*}_{x} := x?(u).M[\dropn{u}]
\end{mathpar}

The dependence of $M^{*}_{x}$ on a name makes it an abstraction, 

\begin{mathpar}
  M^{*} := (x)x?(u).M[\dropn{u}]
\end{mathpar}

\subsection{Additional notation}

It will sometimes be convenient to denote the process a name
quotes. We already have the notation $x = \quotep{P}$, but it will be
convenient to introduce an alternate notation, $\procn{x}$, when we
want to emphasize the connection to the use of the name. Note that, by
virtue of name equivalence, $\quotep{\procn{x}} \nameeq x$; so, the
notation is consistent with previous definitions.

Further, because names have structure it is possible to effect
substitutions on the basis of that structure. This means we need to
upgrade our notation for substitutions, which we accomplish by
adapting comprehension notation. Thus,

\begin{mathpar}
  P\{ y / x : x \in S \}
\end{mathpar}

is interpreted to mean the process derived from P by replacing (in a
capture-avoiding manner) each occurrence of $x$ in $S$ by $y$. For example,

\begin{mathpar}
  P\{ \quotep{\procn{x}|\procn{x}} / x : x \in \freenames{P} \}
\end{mathpar}

will replace each (occurrence) of a free name $x$ in $P$ by
$\quotep{\procn{x}|\procn{x}}$.

Also, we will avail ourselves of the notation $x^{L}$ and $x^{R}$ to
denote injections of a name into disjoint copies of the name
space. There are numerous ways to accomplish this. One example can be
found in \cite{MeredithR05}. This notation overloads to vectors of
names: $\vec{x}^{\pi} := (x_{i}^{\pi} \; : \; 0 \leq i < |\vec{x}| )$ where $\pi \in \{L,R\}$.

We also use $P^{\Box} := P|\Box$.

In \cite{MeredithR05} an interpretation of the new operator is
given. It turns out that there are several possible interpretations
all enjoying the requisite algebraic properties of the operator (see
\cite{milner91polyadicpi}). We will therefore make liberal use of
$(\nu\; \vec{x})P$.

% subsection the_syntax_and_semantics_of_the_notation_system (end)   

\input{qm2pi.qmops} 

\input{qm2pi.sterngerlach} 

\input{qm2pi.metric} 

% section concurrent_process_calculi (end)

%\input{qm2pi.proofsketch}

% section proof sketch (end)

%\input{qm2pi.slviaknots} 

% section spatial logic via knots (end)

\input{qm2pi.conclusion}

% section conclusion (end)

%\input{qm2pi.dtcodes} 

% section wiring algorithm (end)

\input{qm2pi.ack} 

% section acknowledgments (end)

\newpage


\bibliographystyle{plain}   
\bibliography{../../biblios/main.bib}

\input{qm2pi.rhodetails}

\end{document}

 

% section wiring algorithm (end)

\documentclass[12pt]{llncs}
%\documentclass{jktr}

\usepackage[pdftex]{hyperref}                   
\usepackage {listings}
\usepackage {mathpartir}
\usepackage{bcprules}
%\usepackage{listings}
                       
\usepackage{graphicx} 
%\usepackage[margins=2.5cm,nohead,nofoot]{geometry}
%\usepackage{geometry}
\usepackage{amsfonts}
\usepackage{amstext}
\usepackage{latexsym}
\usepackage{amssymb}
\usepackage{color}


%\include{myPreamble}
\include{qm2pi.local} 

%\ifpdf
%\usepackage[pdftex]{graphicx}
%\else
%\usepackage{graphicx}
%\fi

 % \ifpdf
%  \usepackage{pdfsync}
%  \if


%\title{Brief Article}
%\author{David F. Snyder}
%\author{L.G. Meredith}

%\address{Dept. of Math., Texas State University--San Marcos, San Marcos, TX 78666}
       
\pagestyle{empty}


\begin{document}

\lstset{language=[Objective]Caml,frame=shadowbox}

\input{qm2pi.front}

% section front matter (end)

\input{qm2pi.intro} 
 
% section introduction (end)

% \input{qm2pi.knotations} 

% section notation (end)

\input{qm2pi.process.calculi} 

% section concurrent_process_calculi_and_spatial_logics_ (end)
    
%\input{qm2pi.knots2pi} 

%\input{qm2pi.trefoil} 

%\input{qm2pi.mainthm} 

% subsection basic_interpretation (end)

%\input{qm2pi.rho.presentation} 
\subsection{The syntax and semantics of the notation system}\label{sub:the_syntax_and_semantics_of_the_notation_system} % (fold)

We now summarize a technical presentation of the calculus that
embodies our theory of dynamics. The typical presentation of such a
calculus follows the style of giving generators and relations on
them. The grammar, below, describing term constructors, freely
generates the set of processes, $\Proc$. This set is then quotiented
by a relation known as structural congruence and it is over this set
that the notion of dynamics is expressed. This presentation is
essentially that of \cite{MeredithR05} with the addition of
polyadicity and summation. For readability we have relegated some of
the technical subtleties to an appendix.

\subsubsection{Process grammar}\label{subsub:process_grammar}

\begin{mathpar}
  \inferrule* [lab=synchronization] {} {{M} \bc \pzero \;|\; x?F \;|\; x!C }
  \and
  \inferrule* [lab=abstraction] {} {{F} \bc (x)P}
  \and
  \inferrule* [lab=concretion] {} {{C} \bc \langle Q \rangle}
  \and
  \inferrule* [lab=process] {} {{P,Q} \bc M \;| \;P|Q \;|\; @{x}}
  \and
  \inferrule* [lab=name] {} {{x} \bc \quotep{P}}
\end{mathpar} 

Note that $\vec{x}$ (resp. $\vec{P}$) denotes a vector of names
(resp. processes) of length $|\vec{x}|$ (resp. $|\vec{P}|$). We adopt
the following useful abbreviations.

\begin{mathpar}
   x?(\vec{y}).P := x.(\vec{y})P \and  x\clift{\vec{P}} := x.\clift{\vec{P}}
   \and x!(y) := \lift{x}{\dropn{y}}
   \and \Pi_{i=0}^{n-1}P_i := P_0 | \ldots | P_{n-1}
\end{mathpar}

\subsubsection{Structural congruence}

\paragraph{Free and bound names and alpha-equivalence.} At the
core of structural equivalence is alpha-equivalence which identifies
process that are the same up to a change of variable. Formally, we
recognize the distinction between free and bound names. The free names
of a process, $\freenames{P}$, may be calculated recursively as
follows:

\begin{mathpar}
\freenames{\pzero} := \emptyset
  \and \\
  \freenames{x?(y).P} := \{ x \} \cup (\freenames{P} \setminus \{ y \})
  \and 
  \freenames{x!\langle P \rangle} := \{ x \} \cup \{ P \} 
  \and \\
  \freenames{P|Q} := \freenames{P} \cup \freenames{Q}
  \and \\
  \freenames{@{x}} := \{ x \}
\end{mathpar}

$\pi$
$\quotep{\pi}$

$\freenames{-} : \pi \to \mathcal{P}(\quotep{\pi})$

\begin{eqnarray*}
  \freenames{\pzero} & := & \emptyset \\
  \freenames{x?(y).P} & := & \{ x \} \cup (\freenames{P} \setminus \{ y \}) \\
  \freenames{x!\langle P \rangle} & := & \{ x \} \cup \{ P \} \\
  \freenames{P|Q} & := & \freenames{P} \cup \freenames{Q} \\
  \freenames{\dropn{x}} & := & \{ x \}
\end{eqnarray*}

The bound names of a process, $\boundnames{P}$, are those names occurring in $P$
that are not free. For example, in $x?(y).0$, the name $x$ is free, while $y$ is bound.

\begin{mathpar}
  \inferrule* [lab=monoidal-laws] {} { P|Q \equiv Q|P \and P|0 \equiv P \and P|(Q|R) \equiv (P|Q)|R }
\end{mathpar}

\begin{mathpar}
  \inferrule* [lab=alpha-equivalence] {} { (x)P \equiv (y)P\{y/x\} \and y \not\in \freenames{P} }
\end{mathpar}

\begin{definition}
Then two processes, $P,Q$, are alpha-equivalent if $P = Q\{\vec{y}/\vec{x}\}$ for
some $\vec{x} \in \boundnames{Q},\vec{y} \in \boundnames{P}$, where $Q\{\vec{y}/\vec{x}\}$
denotes the capture-avoiding substitution of $\vec{y}$ for $\vec{x}$ in $Q$.
\end{definition}

\begin{definition}
  The {\em structural congruence} \cite{SangiorgiWalker} , $\equiv$,
  between processes is the least congruence containing
  alpha-equivalence, satisfying the abelian monoid laws
  (associativity, commutativity and $\pzero$ as identity) for parallel
  composition $|$ and for summation $+$.
\end{definition}

\subsection{Name equivalence}

We take name equivalence, written $\nameeq$, to be the smallest
equivalence relation generated by the following rules.

\begin{mathpar}
\inferrule*[lab=Quote-drop]
{ }
{ \quotep{@{x}} \nameeq x }

\inferrule*[lab=Struct-equiv]
{ P \scong Q }
{ \quotep{P} \nameeq \quotep{Q} }
\end{mathpar}

The astute reader will have noticed that the mutual recursion of names
and processes imposes a mutual recursion on alpha-equivalence and
structural equivalence via name-equivalence. Fortunately, all of this
works out pleasantly and we may calculate in the natural way, free of
concern. The reader interested in the details is referred to the
appendix \ref{appendix:rho_details}.

\subsection{Substitution}

We use $\Proc$ for the set of processes, $\QProc$ for the set of
names, and $\id{\{}\vec{y} / \vec{x} \id{\}}$ to denote partial maps,
$s : \QProc \rightarrow \QProc$. A map, $s$ lifts, uniquely, to a map
on process terms, $\widehat{s} : \Proc \rightarrow \Proc$ by the
following equations.

\begin{mathpar}
  (0) \psubstp{Q}{P} := 0 \\
  (R \juxtap S) \psubstp{Q}{P}
  :=    
  (R)\psubstp{Q}{P} \juxtap (S) \psubstp{Q}{P} \\
  (x?(y).R) \psubstp{Q}{P}    
  :=    
  (x)\substp{Q}{P} (z)\concat( (R \psubstn{z}{y}) \psubstp{Q}{P} ) \\
  (\lift{x}{R}) \psubstp{Q}{P}  
  :=
  \lift{(x)\substp{Q}{P}}{ R \psubstp{Q}{P} } \\
%   (\dropn{x})  \psubstp{Q}{P}       
%   := 
%   \left\{ 
%     \begin{array}{ccc} 
%       \dropn{\quotep{Q}} & & x \nameeq \quotep{P} \\
%       \dropn{x} & & otherwise \\
%     \end{array}
%   \right. 
  (\dropn{x})  \psubstp{Q}{P}       
  := 
  \left\{ 
    \begin{array}{ccc} 
      Q & & x \nameeq \quotep{P} \\
      \dropn{x} & & otherwise \\
    \end{array}
  \right.
\end{mathpar}
 

where

\begin{eqnarray}
  (x)\id{\{} \lpquote Q \rpquote / \lpquote P \rpquote \id{\}}            = 
  \left\{ 
    \begin{array}{ccc}
      \lpquote Q \rpquote & & x \nameeq \lpquote P \rpquote \\
      x & & otherwise \\
    \end{array}
  \right. \nonumber
\end{eqnarray}

and $z$ is chosen distinct from $\quotep{P}$, $\quotep{Q}$, the free
names in $Q$, and all the names in $R$. Our $\alpha$-equivalence will
be built in the standard way from this substitution.

\begin{remark}\label{rem:no_self_referential_names}
  One consequence of these definitions is that $\forall P. \quotep{P}
  \not\in \freenames{P}$.
\end{remark}

\subsection{ Dynamic quote: an example }

Anticipating something of what's to come, consider applying the
substitution, $\widehat{\id{\{}u / z \id{\}}}$, to the following pair
of processes, $\lift{w}{y!(z)}$ and $w[ \lpquote y!(z) \rpquote ]$.

\begin{eqnarray}
	\lift{w}{y!(z)}\widehat{\id{\{}u / z \id{\}}}
		& = &
		\lift{w}{y!(u)} \nonumber\\
	w[ \lpquote y!(z) \rpquote ] \widehat{ \id{\{}u / z \id{\}} }
		& = &
		w[ \lpquote y!(z) \rpquote ] \nonumber
\end{eqnarray}

Because the body of the process between quotes is impervious to
substitution, we get radically different answers. In fact, by
examining the first process in an input context,
e.g. $x?(z).\lift{w}{y!(z)}$, we see that the process under the lift
operator may be shaped by prefixed inputs binding a name inside it. In
this sense, the lift operator will be seen as a way to dynamically
construct processes before reifying them as names.

Finally equipped with these standard features we can present the
dynamics of the calculus.

\subsubsection{Operational semantics} 

Finally, we introduce the computational dynamics. What marks these
algebras as distinct from other more traditionally studied algebraic
structures, e.g. vector spaces or polynomial rings, is the manner in
which dynamics is captured. In traditional structures, dynamics is typically
expressed through morphisms between such structures, as in linear maps
between vector spaces or morphisms between rings. In algebras
associated with the semantics of computation, the dynamics is
expressed as part of the algebraic structure itself, through a
reduction reduction relation typically denoted by $\red$. Below, we
give a recursive presentation of this relation for the calculus used
in the encoding.

$\red \subseteq \pi \times \pi$
$\red : \pi \to \mathcal{P}(\pi)$

\begin{mathpar}
  \inferrule* [lab=Comm] { \textsf{match}( x_{src}, x_{trgt} ) } { x_{trgt}?(y)P \; | \; x_{src}!\langle {Q} \rangle \red P\{\quotep{Q}/y}\} }
  \and \\
  \inferrule* [lab=Par] {{P} \red {P}'} {{{P} | {Q}} \red {{P}' | {Q}}}
  \and
  \inferrule* [lab=Equiv]{{{P} \scong {P}'} \andalso {{P}' \red {Q}'} \andalso {{Q}' \scong {Q}}}{{P} \red {Q}}
\end{mathpar}

\begin{eqnarray*}
  match_{\equiv} (\quotep{P},\quotep{Q}) & := & P \equiv Q \\
  match_{\dagger}(\quotep{P},\quotep{Q}) & := & \forall R. P|Q \red^{*} R => R \red^{*} 0 \\
  match_{K}(\quotep{P},\quotep{Q}) & := & K \mbox{ for some context } K
\end{eqnarray*}

$u?(x)P | u!\langle Q \rangle \red P\{\quotep{Q}/x\}$

%We write $\wred$ for $\red^*$, and $P\red$ if $\exists Q $ such that $ P \red Q$.
We write $P\red$ if $\exists Q $ such that $ P \red Q$ and $P\not\red$, otherwise.

\section{Replication}

As mentioned before, it is known that replication (and hence
recursion) can be implemented in a higher-order process algebra
\cite{SangiorgiWalker}. As our first example of calculation with the
machinery thus far presented we give the construction explicitly in
the {\rhoc}.

\begin{eqnarray}
	D_{x} & := & \prefix{x}{y}{(\binpar{\outputp{x}{y}}{@{y}})} \nonumber\\
	\bangp_{x}{P} & := & \binpar{{x}!\langle{\binpar{D_{x}}{P}}\rangle}{D_{x}} \nonumber
\end{eqnarray}

\begin{eqnarray}
	\bangp_{x}{P} & & \nonumber\\
	=
	& {x}!\langle{(\prefix{x}{y}{(\outputp{x}{y} | @{y})) | P}}\rangle 
	      | \prefix{x}{y}{(\outputp{x}{y} | @{y})} & \nonumber\\
	\red
	& (\outputp{x}{y} | @{y})\substn{\quotep{(\prefix{x}{y}{(@{y} | \outputp{x}{y})) | P}}}{y} & \nonumber\\
	=
	& \outputp{x}{\quotep{(\prefix{x}{y}{(\outputp{x}{y} | @{y})) | P}}}
	  | {(\prefix{x}{y}{(\outputp{x}{y} | @{y})) | P}} & \nonumber\\
	\red
	& \ldots & \nonumber\\
	\red^*
	& P | P | \ldots & \nonumber
\end{eqnarray}

Of course, this encoding, as an implementation, runs away, unfolding
$\bangp{P}$ eagerly. A lazier and more implementable replication
operator, restricted to input-guarded processes, may be obtained as follows.

\begin{eqnarray}
\bangp{\prefix{u}{v}{P}} 
	:= 
	\binpar{\lift{x}{\prefix{u}{v}{(\binpar{D(x)}{P})}}}{D(x)} \nonumber
\end{eqnarray}

\begin{remark}
  Note that the lazier definition still does not deal with summation
  or mixed summation (i.e. sums over input and output). The reader is
  invited to construct definitions of replication that deal with these
  features. 

  Further, the definitions are parameterized in a name, $x$. Can you,
  gentle reader, make a definition that eliminates this parameter and
  guarantees no accidental interaction between the replication
  machinery and the process being replicated -- i.e. no accidental
  sharing of names used by the process to get its work done and the
  name(s) used by the replication to effect copying. This latter
  revision of the definition of replication is crucial to obtaining
  the expected identity $!!P \sim !P$.
\end{remark}

\begin{remark}\label{rem:paradoxical_combinator}
  The reader familiar with the lambda calculus will have noticed the
  similarity between $D$ and the paradoxical combinator.

  [Ed. note: the existence of this seems to suggest we have to be more
  restrictive on the set of processes and names we admit if we are to
  support no-cloning.]
\end{remark}

\subsubsection{Bisimulation}

The computational dynamics gives rise to another kind of equivalence,
the equivalence of computational behavior. As previously mentioned
this is typically captured \emph{via} some form of bisimulation.

% The notion we use in this paper is weak barbed bisimulation
% \cite{milner91polyadicpi}.

The notion we use in this paper is derived from weak barbed
bisimulation \cite{milner91polyadicpi}. 

\begin{definition}
An \emph{observation relation}, $\downarrow_{\mathcal N}$, over a set
of names, $\mathcal N$, is the smallest relation satisfying the rules
below.

\infrule[Out-barb]{y \in {\mathcal N}, \; x \nameeq y}
		  {\outputp{x}{v} \downarrow_{\mathcal N} x}
\infrule[Par-barb]{\mbox{$P\downarrow_{\mathcal N} x$ or $Q\downarrow_{\mathcal N} x$}}
		  {\binpar{P}{Q} \downarrow_{\mathcal N} x}

We write $P \Downarrow_{\mathcal N} x$ if there is $Q$ such that 
$P \wred Q$ and $Q \downarrow_{\mathcal N} x$.
\end{definition}

\begin{definition}
%\label{def.bbisim}
An  ${\mathcal N}$-\emph{barbed bisimulation} over a set of names, ${\mathcal N}$, is a symmetric binary relation 
${\mathcal S}_{\mathcal N}$ between agents such that $P\rel{S}_{\mathcal N}Q$ implies:
\begin{enumerate}
\item If $P \red P'$ then $Q \wred Q'$ and $P'\rel{S}_{\mathcal N} Q'$.
\item If $P\downarrow_{\mathcal N} x$, then $Q\Downarrow_{\mathcal N} x$.
\end{enumerate}
$P$ is ${\mathcal N}$-barbed bisimilar to $Q$, written
$P \wbbisim_{\mathcal N} Q$, if $P \rel{S}_{\mathcal N} Q$ for some ${\mathcal N}$-barbed bisimulation ${\mathcal S}_{\mathcal N}$.
\end{definition}

$\mathcal{R} \subseteq \pi \times \pi$

$P \mathcal{R} Q => \forall P'. P \red P' \Rightarrow \exists Q'. Q \red Q', P' \mathcal{R} Q'$

$P \vdash x \Rightarrow Q \vdash x$

\begin{mathpar}
  \inferrule*[lab=Out-barb]{x \nameeq y}{{y}!\langle{Q}\rangle \vdash x}
  \and
  \inferrule*[lab=Par-barb]{\mbox{$P\vdash x$ or $Q\vdash x$}}{\binpar{P}{Q} \vdash x}
\end{mathpar}

\subsubsection{Contexts}

One of the principle advantages of computational calculi like the
$\pi$-calculus is a well-defined notion of context,
contextual-equivalence and a correlation between
contextual-equivalence and notions of bisimulation. The notion of
context allows the decomposition of a process into (sub-)process and
its syntactic environment, its context. Thus, a context may be
thought of as a process with a ``hole'' (written $\Box$) in it. The
application of a context $M$ to a process $P$, written $M[P]$, is
tantamount to filling the hole in $M$ with $P$. In this paper we do
not need the full weight of this theory, but do make use of the notion
of context in the proof the main theorem. 

\begin{mathpar}
  \inferrule* [lab=summation] {} {{M_{M},M_{N}} \bc \Box \;|\; x.M_{A} \;|\; M_{M}+M_{N}}
  \and
  \inferrule* [lab=agent] {} {{M_{A}} \bc (\vec{x})M_{P} \;| \; \clift{P_0,\ldots,M_{P},\ldots,P_N}}
  \and \\
  \inferrule* [lab=process] {} {{M_{P}} \bc M_{N} \;| \;P|M_{P} }
\end{mathpar} 

\begin{mathpar}
  \inferrule* [lab=sychronization] {} {M_{N} \bc \Box \;|\; x?M_{F} \;|\; x!M_{C}}
  \and
  \inferrule* [lab=abstraction] {} {{M_{F}} \bc (x)M_{P} }
  \and
  \inferrule* [lab=concretion] {} {{M_{C}} \bc \langle M_{P} \rangle }
  \and \\
  \inferrule* [lab=process] {} {{M_{P}} \bc M_{N} \;| \;P|M_{P} }
\end{mathpar}

\begin{definition}[contextual application] Given a context $M$, and
  process $P$, we define the \emph{contextual application}, $M[P] :=
  M\{P/\Box\}$. That is, the contextual application of M to P is the
  substitution of $P$ for $\Box$ in $M$.
\end{definition}

$\meaningof{-} : L \to \mathcal{P}(\pi)$

\begin{mathpar}
  \inferrule* [lab=collection] {} {\meaningof{true} = \pi, \and \meaningof{~E} = \pi \setminus \meaningof{E}, \and \meaningof{E_{1} \& E_{2}} = \meaningof{E_{1}} \cap \meaningof{E_{2}}}
\end{mathpar}

\begin{mathpar}
  \inferrule* [lab=structure] {} {\meaningof{0} = \{ P \in \pi | P \equiv 0 \}, \and \\ \meaningof{E_1 | E_2} = \{ P \in \pi | P \equiv P_{1} | P_{2}, P_{1} \in \meaningof{E_{1}}, P_{2} \in \meaningof{E_2}\} }
\end{mathpar}

\begin{mathpar}
 \inferrule* [lab=behavior] {} {\meaningof{\langle a?b \rangle E} = \{ P \in \pi | P \equiv Q | u?(y)P', \\ \and \\\\ \and \\ \;\;\; u \in \meaningof{a}, \forall z.P'\{z/y\} \in \meaningof{E\{z/b\}}\}, \and \\ \meaningof{a!E} = \{ P \in \pi | P \equiv Q | x!\langle P' \rangle, x \in \meaningof{a} P' \in \meaningof{E}\} }
\end{mathpar}

\begin{mathpar}
 \inferrule* [lab=nominal] {} {\meaningof{\quotep{E}} = \{ \quotep{P} \in \quotep{\pi} | P \in \meaningof{E} \}, \and \meaningof{\quotep{P}} = \{ \quotep{Q} \in \quotep{\pi} | P \equiv Q \} \and \\ \meaningof{@\quotep{E}} = \{ P \in \pi | P \equiv @x, x \in \meaningof{E} \}}
\end{mathpar}

\begin{eqnarray*}
  \\
  \meaningof{-} : TS \to ST
\end{eqnarray*}

\begin{eqnarray*}
  \\
  L : TS \to ST
\end{eqnarray*}

\begin{eqnarray*}
  \\
  P \models E \iff P \in \meaningof{E}
\end{eqnarray*}

\begin{eqnarray*}
  P \approx_{L} Q \iff \forall E \in L. P \models E \iff Q \models E
\end{eqnarray*}

\begin{eqnarray*}
  P \approx_{K} Q
\end{eqnarray*}

\begin{eqnarray*}
  P \approx Q
\end{eqnarray*}

$\approx_{K} = \approx = \approx_{L}$

\subsubsection{Contextual duality}

Note that contexts extend the quotation operation to a family of
operations from processes to names. Given a context, $M$, we can
define a \emph{nominal context}, $\quotep{M}$ by $\quotep{M}[P] :=
\quotep{M[P]}$. To foreshadow what is to come we observe that these
operations enjoy a duality with processes very much like the duality
between vectors and maps from vectors to scalars.

Further, because the calculus is essentially higher-order, we have a
correspondence between contexts and processes. More specifically,
given a name $x$ and a context $M$ we can construct $M^{*}_{x}$ such
that 

\begin{mathpar}
  M^{*}_{x} | \lift{x}{P} \red M[P]
\end{mathpar}

namely,

\begin{mathpar}
  M^{*}_{x} := x?(u).M[\dropn{u}]
\end{mathpar}

The dependence of $M^{*}_{x}$ on a name makes it an abstraction, 

\begin{mathpar}
  M^{*} := (x)x?(u).M[\dropn{u}]
\end{mathpar}

\subsection{Additional notation}

It will sometimes be convenient to denote the process a name
quotes. We already have the notation $x = \quotep{P}$, but it will be
convenient to introduce an alternate notation, $\procn{x}$, when we
want to emphasize the connection to the use of the name. Note that, by
virtue of name equivalence, $\quotep{\procn{x}} \nameeq x$; so, the
notation is consistent with previous definitions.

Further, because names have structure it is possible to effect
substitutions on the basis of that structure. This means we need to
upgrade our notation for substitutions, which we accomplish by
adapting comprehension notation. Thus,

\begin{mathpar}
  P\{ y / x : x \in S \}
\end{mathpar}

is interpreted to mean the process derived from P by replacing (in a
capture-avoiding manner) each occurrence of $x$ in $S$ by $y$. For example,

\begin{mathpar}
  P\{ \quotep{\procn{x}|\procn{x}} / x : x \in \freenames{P} \}
\end{mathpar}

will replace each (occurrence) of a free name $x$ in $P$ by
$\quotep{\procn{x}|\procn{x}}$.

Also, we will avail ourselves of the notation $x^{L}$ and $x^{R}$ to
denote injections of a name into disjoint copies of the name
space. There are numerous ways to accomplish this. One example can be
found in \cite{MeredithR05}. This notation overloads to vectors of
names: $\vec{x}^{\pi} := (x_{i}^{\pi} \; : \; 0 \leq i < |\vec{x}| )$ where $\pi \in \{L,R\}$.

We also use $P^{\Box} := P|\Box$.

In \cite{MeredithR05} an interpretation of the new operator is
given. It turns out that there are several possible interpretations
all enjoying the requisite algebraic properties of the operator (see
\cite{milner91polyadicpi}). We will therefore make liberal use of
$(\nu\; \vec{x})P$.

% subsection the_syntax_and_semantics_of_the_notation_system (end)   

\input{qm2pi.qmops} 

\input{qm2pi.sterngerlach} 

\input{qm2pi.metric} 

% section concurrent_process_calculi (end)

%\input{qm2pi.proofsketch}

% section proof sketch (end)

%\input{qm2pi.slviaknots} 

% section spatial logic via knots (end)

\input{qm2pi.conclusion}

% section conclusion (end)

%\input{qm2pi.dtcodes} 

% section wiring algorithm (end)

\input{qm2pi.ack} 

% section acknowledgments (end)

\newpage


\bibliographystyle{plain}   
\bibliography{../../biblios/main.bib}

\input{qm2pi.rhodetails}

\end{document}

 

% section acknowledgments (end)

\newpage


\bibliographystyle{plain}   
\bibliography{../../biblios/main.bib}

\documentclass[12pt]{llncs}
%\documentclass{jktr}

\usepackage[pdftex]{hyperref}                   
\usepackage {listings}
\usepackage {mathpartir}
\usepackage{bcprules}
%\usepackage{listings}
                       
\usepackage{graphicx} 
%\usepackage[margins=2.5cm,nohead,nofoot]{geometry}
%\usepackage{geometry}
\usepackage{amsfonts}
\usepackage{amstext}
\usepackage{latexsym}
\usepackage{amssymb}
\usepackage{color}


%\include{myPreamble}
\include{qm2pi.local} 

%\ifpdf
%\usepackage[pdftex]{graphicx}
%\else
%\usepackage{graphicx}
%\fi

 % \ifpdf
%  \usepackage{pdfsync}
%  \if


%\title{Brief Article}
%\author{David F. Snyder}
%\author{L.G. Meredith}

%\address{Dept. of Math., Texas State University--San Marcos, San Marcos, TX 78666}
       
\pagestyle{empty}


\begin{document}

\lstset{language=[Objective]Caml,frame=shadowbox}

\input{qm2pi.front}

% section front matter (end)

\input{qm2pi.intro} 
 
% section introduction (end)

% \input{qm2pi.knotations} 

% section notation (end)

\input{qm2pi.process.calculi} 

% section concurrent_process_calculi_and_spatial_logics_ (end)
    
%\input{qm2pi.knots2pi} 

%\input{qm2pi.trefoil} 

%\input{qm2pi.mainthm} 

% subsection basic_interpretation (end)

%\input{qm2pi.rho.presentation} 
\subsection{The syntax and semantics of the notation system}\label{sub:the_syntax_and_semantics_of_the_notation_system} % (fold)

We now summarize a technical presentation of the calculus that
embodies our theory of dynamics. The typical presentation of such a
calculus follows the style of giving generators and relations on
them. The grammar, below, describing term constructors, freely
generates the set of processes, $\Proc$. This set is then quotiented
by a relation known as structural congruence and it is over this set
that the notion of dynamics is expressed. This presentation is
essentially that of \cite{MeredithR05} with the addition of
polyadicity and summation. For readability we have relegated some of
the technical subtleties to an appendix.

\subsubsection{Process grammar}\label{subsub:process_grammar}

\begin{mathpar}
  \inferrule* [lab=synchronization] {} {{M} \bc \pzero \;|\; x?F \;|\; x!C }
  \and
  \inferrule* [lab=abstraction] {} {{F} \bc (x)P}
  \and
  \inferrule* [lab=concretion] {} {{C} \bc \langle Q \rangle}
  \and
  \inferrule* [lab=process] {} {{P,Q} \bc M \;| \;P|Q \;|\; @{x}}
  \and
  \inferrule* [lab=name] {} {{x} \bc \quotep{P}}
\end{mathpar} 

Note that $\vec{x}$ (resp. $\vec{P}$) denotes a vector of names
(resp. processes) of length $|\vec{x}|$ (resp. $|\vec{P}|$). We adopt
the following useful abbreviations.

\begin{mathpar}
   x?(\vec{y}).P := x.(\vec{y})P \and  x\clift{\vec{P}} := x.\clift{\vec{P}}
   \and x!(y) := \lift{x}{\dropn{y}}
   \and \Pi_{i=0}^{n-1}P_i := P_0 | \ldots | P_{n-1}
\end{mathpar}

\subsubsection{Structural congruence}

\paragraph{Free and bound names and alpha-equivalence.} At the
core of structural equivalence is alpha-equivalence which identifies
process that are the same up to a change of variable. Formally, we
recognize the distinction between free and bound names. The free names
of a process, $\freenames{P}$, may be calculated recursively as
follows:

\begin{mathpar}
\freenames{\pzero} := \emptyset
  \and \\
  \freenames{x?(y).P} := \{ x \} \cup (\freenames{P} \setminus \{ y \})
  \and 
  \freenames{x!\langle P \rangle} := \{ x \} \cup \{ P \} 
  \and \\
  \freenames{P|Q} := \freenames{P} \cup \freenames{Q}
  \and \\
  \freenames{@{x}} := \{ x \}
\end{mathpar}

$\pi$
$\quotep{\pi}$

$\freenames{-} : \pi \to \mathcal{P}(\quotep{\pi})$

\begin{eqnarray*}
  \freenames{\pzero} & := & \emptyset \\
  \freenames{x?(y).P} & := & \{ x \} \cup (\freenames{P} \setminus \{ y \}) \\
  \freenames{x!\langle P \rangle} & := & \{ x \} \cup \{ P \} \\
  \freenames{P|Q} & := & \freenames{P} \cup \freenames{Q} \\
  \freenames{\dropn{x}} & := & \{ x \}
\end{eqnarray*}

The bound names of a process, $\boundnames{P}$, are those names occurring in $P$
that are not free. For example, in $x?(y).0$, the name $x$ is free, while $y$ is bound.

\begin{mathpar}
  \inferrule* [lab=monoidal-laws] {} { P|Q \equiv Q|P \and P|0 \equiv P \and P|(Q|R) \equiv (P|Q)|R }
\end{mathpar}

\begin{mathpar}
  \inferrule* [lab=alpha-equivalence] {} { (x)P \equiv (y)P\{y/x\} \and y \not\in \freenames{P} }
\end{mathpar}

\begin{definition}
Then two processes, $P,Q$, are alpha-equivalent if $P = Q\{\vec{y}/\vec{x}\}$ for
some $\vec{x} \in \boundnames{Q},\vec{y} \in \boundnames{P}$, where $Q\{\vec{y}/\vec{x}\}$
denotes the capture-avoiding substitution of $\vec{y}$ for $\vec{x}$ in $Q$.
\end{definition}

\begin{definition}
  The {\em structural congruence} \cite{SangiorgiWalker} , $\equiv$,
  between processes is the least congruence containing
  alpha-equivalence, satisfying the abelian monoid laws
  (associativity, commutativity and $\pzero$ as identity) for parallel
  composition $|$ and for summation $+$.
\end{definition}

\subsection{Name equivalence}

We take name equivalence, written $\nameeq$, to be the smallest
equivalence relation generated by the following rules.

\begin{mathpar}
\inferrule*[lab=Quote-drop]
{ }
{ \quotep{@{x}} \nameeq x }

\inferrule*[lab=Struct-equiv]
{ P \scong Q }
{ \quotep{P} \nameeq \quotep{Q} }
\end{mathpar}

The astute reader will have noticed that the mutual recursion of names
and processes imposes a mutual recursion on alpha-equivalence and
structural equivalence via name-equivalence. Fortunately, all of this
works out pleasantly and we may calculate in the natural way, free of
concern. The reader interested in the details is referred to the
appendix \ref{appendix:rho_details}.

\subsection{Substitution}

We use $\Proc$ for the set of processes, $\QProc$ for the set of
names, and $\id{\{}\vec{y} / \vec{x} \id{\}}$ to denote partial maps,
$s : \QProc \rightarrow \QProc$. A map, $s$ lifts, uniquely, to a map
on process terms, $\widehat{s} : \Proc \rightarrow \Proc$ by the
following equations.

\begin{mathpar}
  (0) \psubstp{Q}{P} := 0 \\
  (R \juxtap S) \psubstp{Q}{P}
  :=    
  (R)\psubstp{Q}{P} \juxtap (S) \psubstp{Q}{P} \\
  (x?(y).R) \psubstp{Q}{P}    
  :=    
  (x)\substp{Q}{P} (z)\concat( (R \psubstn{z}{y}) \psubstp{Q}{P} ) \\
  (\lift{x}{R}) \psubstp{Q}{P}  
  :=
  \lift{(x)\substp{Q}{P}}{ R \psubstp{Q}{P} } \\
%   (\dropn{x})  \psubstp{Q}{P}       
%   := 
%   \left\{ 
%     \begin{array}{ccc} 
%       \dropn{\quotep{Q}} & & x \nameeq \quotep{P} \\
%       \dropn{x} & & otherwise \\
%     \end{array}
%   \right. 
  (\dropn{x})  \psubstp{Q}{P}       
  := 
  \left\{ 
    \begin{array}{ccc} 
      Q & & x \nameeq \quotep{P} \\
      \dropn{x} & & otherwise \\
    \end{array}
  \right.
\end{mathpar}
 

where

\begin{eqnarray}
  (x)\id{\{} \lpquote Q \rpquote / \lpquote P \rpquote \id{\}}            = 
  \left\{ 
    \begin{array}{ccc}
      \lpquote Q \rpquote & & x \nameeq \lpquote P \rpquote \\
      x & & otherwise \\
    \end{array}
  \right. \nonumber
\end{eqnarray}

and $z$ is chosen distinct from $\quotep{P}$, $\quotep{Q}$, the free
names in $Q$, and all the names in $R$. Our $\alpha$-equivalence will
be built in the standard way from this substitution.

\begin{remark}\label{rem:no_self_referential_names}
  One consequence of these definitions is that $\forall P. \quotep{P}
  \not\in \freenames{P}$.
\end{remark}

\subsection{ Dynamic quote: an example }

Anticipating something of what's to come, consider applying the
substitution, $\widehat{\id{\{}u / z \id{\}}}$, to the following pair
of processes, $\lift{w}{y!(z)}$ and $w[ \lpquote y!(z) \rpquote ]$.

\begin{eqnarray}
	\lift{w}{y!(z)}\widehat{\id{\{}u / z \id{\}}}
		& = &
		\lift{w}{y!(u)} \nonumber\\
	w[ \lpquote y!(z) \rpquote ] \widehat{ \id{\{}u / z \id{\}} }
		& = &
		w[ \lpquote y!(z) \rpquote ] \nonumber
\end{eqnarray}

Because the body of the process between quotes is impervious to
substitution, we get radically different answers. In fact, by
examining the first process in an input context,
e.g. $x?(z).\lift{w}{y!(z)}$, we see that the process under the lift
operator may be shaped by prefixed inputs binding a name inside it. In
this sense, the lift operator will be seen as a way to dynamically
construct processes before reifying them as names.

Finally equipped with these standard features we can present the
dynamics of the calculus.

\subsubsection{Operational semantics} 

Finally, we introduce the computational dynamics. What marks these
algebras as distinct from other more traditionally studied algebraic
structures, e.g. vector spaces or polynomial rings, is the manner in
which dynamics is captured. In traditional structures, dynamics is typically
expressed through morphisms between such structures, as in linear maps
between vector spaces or morphisms between rings. In algebras
associated with the semantics of computation, the dynamics is
expressed as part of the algebraic structure itself, through a
reduction reduction relation typically denoted by $\red$. Below, we
give a recursive presentation of this relation for the calculus used
in the encoding.

$\red \subseteq \pi \times \pi$
$\red : \pi \to \mathcal{P}(\pi)$

\begin{mathpar}
  \inferrule* [lab=Comm] { \textsf{match}( x_{src}, x_{trgt} ) } { x_{trgt}?(y)P \; | \; x_{src}!\langle {Q} \rangle \red P\{\quotep{Q}/y}\} }
  \and \\
  \inferrule* [lab=Par] {{P} \red {P}'} {{{P} | {Q}} \red {{P}' | {Q}}}
  \and
  \inferrule* [lab=Equiv]{{{P} \scong {P}'} \andalso {{P}' \red {Q}'} \andalso {{Q}' \scong {Q}}}{{P} \red {Q}}
\end{mathpar}

\begin{eqnarray*}
  match_{\equiv} (\quotep{P},\quotep{Q}) & := & P \equiv Q \\
  match_{\dagger}(\quotep{P},\quotep{Q}) & := & \forall R. P|Q \red^{*} R => R \red^{*} 0 \\
  match_{K}(\quotep{P},\quotep{Q}) & := & K \mbox{ for some context } K
\end{eqnarray*}

$u?(x)P | u!\langle Q \rangle \red P\{\quotep{Q}/x\}$

%We write $\wred$ for $\red^*$, and $P\red$ if $\exists Q $ such that $ P \red Q$.
We write $P\red$ if $\exists Q $ such that $ P \red Q$ and $P\not\red$, otherwise.

\section{Replication}

As mentioned before, it is known that replication (and hence
recursion) can be implemented in a higher-order process algebra
\cite{SangiorgiWalker}. As our first example of calculation with the
machinery thus far presented we give the construction explicitly in
the {\rhoc}.

\begin{eqnarray}
	D_{x} & := & \prefix{x}{y}{(\binpar{\outputp{x}{y}}{@{y}})} \nonumber\\
	\bangp_{x}{P} & := & \binpar{{x}!\langle{\binpar{D_{x}}{P}}\rangle}{D_{x}} \nonumber
\end{eqnarray}

\begin{eqnarray}
	\bangp_{x}{P} & & \nonumber\\
	=
	& {x}!\langle{(\prefix{x}{y}{(\outputp{x}{y} | @{y})) | P}}\rangle 
	      | \prefix{x}{y}{(\outputp{x}{y} | @{y})} & \nonumber\\
	\red
	& (\outputp{x}{y} | @{y})\substn{\quotep{(\prefix{x}{y}{(@{y} | \outputp{x}{y})) | P}}}{y} & \nonumber\\
	=
	& \outputp{x}{\quotep{(\prefix{x}{y}{(\outputp{x}{y} | @{y})) | P}}}
	  | {(\prefix{x}{y}{(\outputp{x}{y} | @{y})) | P}} & \nonumber\\
	\red
	& \ldots & \nonumber\\
	\red^*
	& P | P | \ldots & \nonumber
\end{eqnarray}

Of course, this encoding, as an implementation, runs away, unfolding
$\bangp{P}$ eagerly. A lazier and more implementable replication
operator, restricted to input-guarded processes, may be obtained as follows.

\begin{eqnarray}
\bangp{\prefix{u}{v}{P}} 
	:= 
	\binpar{\lift{x}{\prefix{u}{v}{(\binpar{D(x)}{P})}}}{D(x)} \nonumber
\end{eqnarray}

\begin{remark}
  Note that the lazier definition still does not deal with summation
  or mixed summation (i.e. sums over input and output). The reader is
  invited to construct definitions of replication that deal with these
  features. 

  Further, the definitions are parameterized in a name, $x$. Can you,
  gentle reader, make a definition that eliminates this parameter and
  guarantees no accidental interaction between the replication
  machinery and the process being replicated -- i.e. no accidental
  sharing of names used by the process to get its work done and the
  name(s) used by the replication to effect copying. This latter
  revision of the definition of replication is crucial to obtaining
  the expected identity $!!P \sim !P$.
\end{remark}

\begin{remark}\label{rem:paradoxical_combinator}
  The reader familiar with the lambda calculus will have noticed the
  similarity between $D$ and the paradoxical combinator.

  [Ed. note: the existence of this seems to suggest we have to be more
  restrictive on the set of processes and names we admit if we are to
  support no-cloning.]
\end{remark}

\subsubsection{Bisimulation}

The computational dynamics gives rise to another kind of equivalence,
the equivalence of computational behavior. As previously mentioned
this is typically captured \emph{via} some form of bisimulation.

% The notion we use in this paper is weak barbed bisimulation
% \cite{milner91polyadicpi}.

The notion we use in this paper is derived from weak barbed
bisimulation \cite{milner91polyadicpi}. 

\begin{definition}
An \emph{observation relation}, $\downarrow_{\mathcal N}$, over a set
of names, $\mathcal N$, is the smallest relation satisfying the rules
below.

\infrule[Out-barb]{y \in {\mathcal N}, \; x \nameeq y}
		  {\outputp{x}{v} \downarrow_{\mathcal N} x}
\infrule[Par-barb]{\mbox{$P\downarrow_{\mathcal N} x$ or $Q\downarrow_{\mathcal N} x$}}
		  {\binpar{P}{Q} \downarrow_{\mathcal N} x}

We write $P \Downarrow_{\mathcal N} x$ if there is $Q$ such that 
$P \wred Q$ and $Q \downarrow_{\mathcal N} x$.
\end{definition}

\begin{definition}
%\label{def.bbisim}
An  ${\mathcal N}$-\emph{barbed bisimulation} over a set of names, ${\mathcal N}$, is a symmetric binary relation 
${\mathcal S}_{\mathcal N}$ between agents such that $P\rel{S}_{\mathcal N}Q$ implies:
\begin{enumerate}
\item If $P \red P'$ then $Q \wred Q'$ and $P'\rel{S}_{\mathcal N} Q'$.
\item If $P\downarrow_{\mathcal N} x$, then $Q\Downarrow_{\mathcal N} x$.
\end{enumerate}
$P$ is ${\mathcal N}$-barbed bisimilar to $Q$, written
$P \wbbisim_{\mathcal N} Q$, if $P \rel{S}_{\mathcal N} Q$ for some ${\mathcal N}$-barbed bisimulation ${\mathcal S}_{\mathcal N}$.
\end{definition}

$\mathcal{R} \subseteq \pi \times \pi$

$P \mathcal{R} Q => \forall P'. P \red P' \Rightarrow \exists Q'. Q \red Q', P' \mathcal{R} Q'$

$P \vdash x \Rightarrow Q \vdash x$

\begin{mathpar}
  \inferrule*[lab=Out-barb]{x \nameeq y}{{y}!\langle{Q}\rangle \vdash x}
  \and
  \inferrule*[lab=Par-barb]{\mbox{$P\vdash x$ or $Q\vdash x$}}{\binpar{P}{Q} \vdash x}
\end{mathpar}

\subsubsection{Contexts}

One of the principle advantages of computational calculi like the
$\pi$-calculus is a well-defined notion of context,
contextual-equivalence and a correlation between
contextual-equivalence and notions of bisimulation. The notion of
context allows the decomposition of a process into (sub-)process and
its syntactic environment, its context. Thus, a context may be
thought of as a process with a ``hole'' (written $\Box$) in it. The
application of a context $M$ to a process $P$, written $M[P]$, is
tantamount to filling the hole in $M$ with $P$. In this paper we do
not need the full weight of this theory, but do make use of the notion
of context in the proof the main theorem. 

\begin{mathpar}
  \inferrule* [lab=summation] {} {{M_{M},M_{N}} \bc \Box \;|\; x.M_{A} \;|\; M_{M}+M_{N}}
  \and
  \inferrule* [lab=agent] {} {{M_{A}} \bc (\vec{x})M_{P} \;| \; \clift{P_0,\ldots,M_{P},\ldots,P_N}}
  \and \\
  \inferrule* [lab=process] {} {{M_{P}} \bc M_{N} \;| \;P|M_{P} }
\end{mathpar} 

\begin{mathpar}
  \inferrule* [lab=sychronization] {} {M_{N} \bc \Box \;|\; x?M_{F} \;|\; x!M_{C}}
  \and
  \inferrule* [lab=abstraction] {} {{M_{F}} \bc (x)M_{P} }
  \and
  \inferrule* [lab=concretion] {} {{M_{C}} \bc \langle M_{P} \rangle }
  \and \\
  \inferrule* [lab=process] {} {{M_{P}} \bc M_{N} \;| \;P|M_{P} }
\end{mathpar}

\begin{definition}[contextual application] Given a context $M$, and
  process $P$, we define the \emph{contextual application}, $M[P] :=
  M\{P/\Box\}$. That is, the contextual application of M to P is the
  substitution of $P$ for $\Box$ in $M$.
\end{definition}

$\meaningof{-} : L \to \mathcal{P}(\pi)$

\begin{mathpar}
  \inferrule* [lab=collection] {} {\meaningof{true} = \pi, \and \meaningof{~E} = \pi \setminus \meaningof{E}, \and \meaningof{E_{1} \& E_{2}} = \meaningof{E_{1}} \cap \meaningof{E_{2}}}
\end{mathpar}

\begin{mathpar}
  \inferrule* [lab=structure] {} {\meaningof{0} = \{ P \in \pi | P \equiv 0 \}, \and \\ \meaningof{E_1 | E_2} = \{ P \in \pi | P \equiv P_{1} | P_{2}, P_{1} \in \meaningof{E_{1}}, P_{2} \in \meaningof{E_2}\} }
\end{mathpar}

\begin{mathpar}
 \inferrule* [lab=behavior] {} {\meaningof{\langle a?b \rangle E} = \{ P \in \pi | P \equiv Q | u?(y)P', \\ \and \\\\ \and \\ \;\;\; u \in \meaningof{a}, \forall z.P'\{z/y\} \in \meaningof{E\{z/b\}}\}, \and \\ \meaningof{a!E} = \{ P \in \pi | P \equiv Q | x!\langle P' \rangle, x \in \meaningof{a} P' \in \meaningof{E}\} }
\end{mathpar}

\begin{mathpar}
 \inferrule* [lab=nominal] {} {\meaningof{\quotep{E}} = \{ \quotep{P} \in \quotep{\pi} | P \in \meaningof{E} \}, \and \meaningof{\quotep{P}} = \{ \quotep{Q} \in \quotep{\pi} | P \equiv Q \} \and \\ \meaningof{@\quotep{E}} = \{ P \in \pi | P \equiv @x, x \in \meaningof{E} \}}
\end{mathpar}

\begin{eqnarray*}
  \\
  \meaningof{-} : TS \to ST
\end{eqnarray*}

\begin{eqnarray*}
  \\
  L : TS \to ST
\end{eqnarray*}

\begin{eqnarray*}
  \\
  P \models E \iff P \in \meaningof{E}
\end{eqnarray*}

\begin{eqnarray*}
  P \approx_{L} Q \iff \forall E \in L. P \models E \iff Q \models E
\end{eqnarray*}

\begin{eqnarray*}
  P \approx_{K} Q
\end{eqnarray*}

\begin{eqnarray*}
  P \approx Q
\end{eqnarray*}

$\approx_{K} = \approx = \approx_{L}$

\subsubsection{Contextual duality}

Note that contexts extend the quotation operation to a family of
operations from processes to names. Given a context, $M$, we can
define a \emph{nominal context}, $\quotep{M}$ by $\quotep{M}[P] :=
\quotep{M[P]}$. To foreshadow what is to come we observe that these
operations enjoy a duality with processes very much like the duality
between vectors and maps from vectors to scalars.

Further, because the calculus is essentially higher-order, we have a
correspondence between contexts and processes. More specifically,
given a name $x$ and a context $M$ we can construct $M^{*}_{x}$ such
that 

\begin{mathpar}
  M^{*}_{x} | \lift{x}{P} \red M[P]
\end{mathpar}

namely,

\begin{mathpar}
  M^{*}_{x} := x?(u).M[\dropn{u}]
\end{mathpar}

The dependence of $M^{*}_{x}$ on a name makes it an abstraction, 

\begin{mathpar}
  M^{*} := (x)x?(u).M[\dropn{u}]
\end{mathpar}

\subsection{Additional notation}

It will sometimes be convenient to denote the process a name
quotes. We already have the notation $x = \quotep{P}$, but it will be
convenient to introduce an alternate notation, $\procn{x}$, when we
want to emphasize the connection to the use of the name. Note that, by
virtue of name equivalence, $\quotep{\procn{x}} \nameeq x$; so, the
notation is consistent with previous definitions.

Further, because names have structure it is possible to effect
substitutions on the basis of that structure. This means we need to
upgrade our notation for substitutions, which we accomplish by
adapting comprehension notation. Thus,

\begin{mathpar}
  P\{ y / x : x \in S \}
\end{mathpar}

is interpreted to mean the process derived from P by replacing (in a
capture-avoiding manner) each occurrence of $x$ in $S$ by $y$. For example,

\begin{mathpar}
  P\{ \quotep{\procn{x}|\procn{x}} / x : x \in \freenames{P} \}
\end{mathpar}

will replace each (occurrence) of a free name $x$ in $P$ by
$\quotep{\procn{x}|\procn{x}}$.

Also, we will avail ourselves of the notation $x^{L}$ and $x^{R}$ to
denote injections of a name into disjoint copies of the name
space. There are numerous ways to accomplish this. One example can be
found in \cite{MeredithR05}. This notation overloads to vectors of
names: $\vec{x}^{\pi} := (x_{i}^{\pi} \; : \; 0 \leq i < |\vec{x}| )$ where $\pi \in \{L,R\}$.

We also use $P^{\Box} := P|\Box$.

In \cite{MeredithR05} an interpretation of the new operator is
given. It turns out that there are several possible interpretations
all enjoying the requisite algebraic properties of the operator (see
\cite{milner91polyadicpi}). We will therefore make liberal use of
$(\nu\; \vec{x})P$.

% subsection the_syntax_and_semantics_of_the_notation_system (end)   

\input{qm2pi.qmops} 

\input{qm2pi.sterngerlach} 

\input{qm2pi.metric} 

% section concurrent_process_calculi (end)

%\input{qm2pi.proofsketch}

% section proof sketch (end)

%\input{qm2pi.slviaknots} 

% section spatial logic via knots (end)

\input{qm2pi.conclusion}

% section conclusion (end)

%\input{qm2pi.dtcodes} 

% section wiring algorithm (end)

\input{qm2pi.ack} 

% section acknowledgments (end)

\newpage


\bibliographystyle{plain}   
\bibliography{../../biblios/main.bib}

\input{qm2pi.rhodetails}

\end{document}



\end{document}



% section proof sketch (end)

%\section{Unlikely characters: spatial logic for
  knots}\label{sub:characteristic_formulae} % (fold)

Associated to the mobile process calculi are a family of logics known
as the Hennessy-Milner logics. These logics typically enjoy a
semantics interpreting formulae as sets of processes that when
factored through the encoding outlined above allows an identification
of classes of knots with logical formulae. In the context of this
encoding the sub-family known as the spatial logics \cite{CairesC03}
\cite{CairesC04} \cite{Caires04} are of particular interest providing
several important features for expressing and reasoning about
properties (i.e. classes) of knots. We hint here at how this may be done.

%\begin{description}
%\item [structural connectives] 
\subsubsection{Structural connectives} The spatial logics enjoy
structural connectives corresponding, at the logical level, to the
parallel composition ($P | Q$) and new name ($(\nu \; x)P$)
connectives for processes. As illustrated in the examples below, these
connectives are extremely expressive given the shape of our encoding.
%\item [decideable satisfaction]

\subsubsection{Decideable satisfaction}
In \cite{Caires04} the satisfaction relation is shown to be decideable
for a rich class of processes. It further turns out that the image of
the our encoding is a proper subset of that class. This result
provides the basis for an algorithm by which to search for knots
enjoying a given property.
%\item [characteristic formulae]

\subsubsection{Characteristic formulae}
In the same paper \cite{Caires04} , Caires presents a means of calculating
characteristic formulae, selecting equivalence classes of processes
up to a pre--specified depth limit on the support set of names. Composed with our
encoding, this characteristic formula can be used to select
characteristic formulae for knots.
%\end{description}

\subsubsection{Spatial logic formulae}

The grammar below (segmented for comprehension) summarizes the syntax
of spatial logic formulae. We employ illustrative examples in the
sequel to provide an intuitive understanding of their meaning
referring the reader to \cite{Caires04} for a more detailed explication
of the semantics.

\begin{mathpar}
  \inferrule* [lab=boolean] {} {{A,B} \bc T \;|\; \neg A \;|\; A \wedge B \;|\; \eta = \eta'}
  \and
  \inferrule* [lab=spatial] {} {|\; \pzero \;|\; A | B \;|\; x \text{\textregistered} A \;|\; \forall x . A \;|\;  H x . A}
  \and
  \inferrule* [lab=behavioral] {} {|\; \alpha . A}
  \and 
  \inferrule* [lab=recursion] {} {|\; X(\vec{u}) \;|\; \mu X(\vec{u}) . A}
  \and
  \inferrule* [lab=action] {} {\alpha \bc \langle x?(\vec{y}) \rangle \;|\; \langle x!(\vec{y}) \rangle \;|\; \langle \tau \rangle}
  \and 
  \inferrule* [lab=name] {} {\eta \bc x \;|\; \tau}
\end{mathpar} 

% subsection characteristic_formulae (end)   	 

\subsection{Example formulae}\label{sub:example_formulae_} % (fold)

\subsubsection{Crossing as formula.}
% 
% \begin{align*}
%   \frac{d}{dx} \sin x &= \cos x 
%   & \frac{d}{dx} e^x &= e^x \\
%   \frac{d}{dx} \cos x &= - \sin x 
%   & \frac{d}{dx} \log x &= \frac{1}{x} \\
% \end{align*} 

\begin{align*}
 \mu C(x_{0},x_{1},y_{0},y_{1},u).&(\langle x_{0}?(z) \rangle(\langle u! \rangle\langle y_{1}!z \rangle C(x_{0},x_{1},y_{0},y_{1},u)) & \\
  & \wedge \langle y_{1}?(z) \rangle (\langle u! \rangle \langle x_{0}!z \rangle C(x_{0},x_{1},y_{0},y_{1},u)) & \\
  & \wedge \langle x_{1}?(z) \rangle (\langle u? \rangle \langle y_{0}!z \rangle C(x_{0},x_{1},y_{0},y_{1},u)) & \\
  & \wedge \langle y_{0}?(z) \rangle (\langle u? \rangle \langle x_{1}!z \rangle C(x_{0},x_{1},y_{0},y_{1},u))) &
\end{align*}

The lexicographical similarity between the shape of this formulae and
the shape of definition of the process representing a crossing reveals
the intuitive meaning of this formulae. It describes the capabilities
of a process that has the right to represent a crossing. For example
it picks out processes that may perform an input on the port $x_0$ in
its initial menu of capabilities. What differentiates the formula
from the process, however, is that the crossing process is the
smallest candidate to satisfy the formula. Infinitely many other
processes -- with internal behavior hidden behind this interface, so
to speak -- also satisfy this formula. Even this simple formula,
then, can be seen to open a new view onto knots, providing a
computational interpretation of \emph{virtual} knots.

Note that this formula is derived by hand. A similar formula can be
derived by employing Caires' calculation of characteristic formula
\cite{Caires04} to the process representing a crossing. In light of
this discussion, we let
$\meaningof{C}_{\phi}(x0,x1,y0,y1,u)$ denote a formula specifying the
dynamics we wish to capture of a crossing. To guarantee we preserve
the shape of the interface and minimal semantics we demand that
$\meaningof{C}_{\phi}(x0,x1,y0,y1,u) \Rightarrow
\textbf{C}(x0,x1,y0,y1,u)$ where $\textbf{C}(x0,x1,y0,y1,u)$ denotes
the formula above.
                            
\subsubsection{Crossing number constraints.}
The moral content of the context lemma (Lemma \ref{context}) is that the notion of
``locality'' in the Reidemeister moves is effectively captured by the
parallel composition operator of the process calculus. This intuition
extends through the logic. Given a formula,
$\meaningof{C}_{\phi}(x0,x1,y0,y1,u)$, we can use the structural
connectives to specify constraints on crossing numbers, such as at
least $n$ crossings, or exactly $n$ crossings.
\begin{mathpar}
  \inferrule* [lab=at-least-n] {} { K^{\geq n}_{\phi}(\vec{xs},\vec{ys}) := \Pi_{i=0}^{n-1} Hu . \meaningof{C}_{\phi}(xs_i,ys_i,u) | T }
  \and 
  \inferrule* [lab=exactly-n] {} { K^{= n}_{\phi}(\vec{xs},\vec{ys}) := \Pi_{i=0}^{n-1} Hu . \meaningof{C}_{\phi}(xs_i,ys_i,u) | \neg (\forall x_0,y_0,x_1,y_1,u . \meaningof{C}_{\phi}(x_0,y_0,x_1,y_1,u) | T) }
\end{mathpar}

To round out this section, recall that the encoding of an $n$-crossing
knot decomposes into a parallel composition of $n$ \emph{copies} of a
crossing process together with a wiring harness. To specify different
knot classes with the same crossing number amounts to specifying
logical constraints on the wiring harness. In the interest of space,
we defer examples to a forthcoming paper. Suffice it to say that both
the conditions ``alternating knot'' and ``contains the tangle
corresponding to 5/3'' are expressible. For example, it is possible to
calculate the characteristic formula of a process corresponding to the
tangle 5/3 and conjoin it into the classifying formula via the
composition connective of the logic.

Finally, we wish to observe that it is entirely within reason to
contemplate a more domain-specific version of spatial logic tailored
to the shape of processes in the image of the encoding. Such a
domain-specific logic would have a better claim to the title formal
language of knot properties.

% subsection example_formulae_ (end)

% section knots_as_processes (end) 

% section spatial logic via knots (end)

\section{Conclusions and future work}

\paragraph{Testing physical space}
You, gentle reader, may wonder why of all the theorems to be proved
given this set up we pick the one above. In some sense it's hardly
central to quantum mechanics. We see it as central in the sense that
it firmly establishes a notion of physical space arising from a notion
of the equivalence of behavior. Relating bisimulation to a metric is a
big step forward, but one is faced with interpreting the relationship
of that metric space to something more physical. Quantum mechanical
notions of ``physical'' space are still far from intuitive, but by
relating this idea of distance as testing to calculations that predict
physical circumstances we are making a not insignificant step forward
toward an understanding of the physical space we inhabit as
essentially dynamic.

\paragraph{Effectivity and simulation}
One of the observations we have yet to make is that the entire program
spelled out here is effective. We have built various interpreters for
the reflective calculus at work in this interpretation. In principle,
then, we can simulate quantum mechanics on a computer. The place where
the simulation may lose fidelity is the infinitely branching summation
for the annihilator.

In this connection i also want to point out that the evaluation style
calculation of the inner product puts the non-determinism of the
summation right at the heart of measurement. This suggests that
Milner's original reduction-based formulation of the dynamics of his
calculi in terms of sums was not just notationally suggestive of a
notion of measure-and-continue but captured some significant part of
the physics.

\paragraph{Quantum continuations}
In light of this last observation i want to point out that the
predominant account of quantum mechanics is missing a key aspect of a
truly compositional story of the physical situation. In a real lab,
when a measurement is made the observation can be made to feed into
another device that then makes another measurement conditioned on the
results of the first. This means that after the superposition was
collapsed the entire experimental set up remained in
superposition. While QM offers a means of writing this down it doesn't
quite line up well with the well-trodden formulation of computation
and continuation that we see so succinctly expressed in Milner's
calculi. This suggests that there might be advantages to this account
of dynamics waiting to be explored.

\paragraph{Quantum logic}
In this connection, we also note that by virtue of having the
Hennessy-Milner construction, we can pull the construction through the
interpretation of QM. This gives us a natural candidate for a quantum
logic that enjoys an extremely tight connection with it's domain of
interpretation, making the construction much less ad hoc (rather it is
the image of functor!).

\paragraph{Quantum probabiity}
i have questions about the basis of the interpretation of inner
product as probability amplitude. In particular, using which
axiomatization of probability theory does the notion of probability
amplitude earn the right to be so dubbed? In other words, where is the
proof that the operation for calculating a probability amplitude (and
then squaring) satisfies the axioms of what it means to calculate a
probability? Even if such a proof exists (i have yet to find it in the
literature), i wonder if it might not be possible to turn things on
their heads. Can we view the calculation of the probability amplitude
as an axiomatization of probability? If so, then the definition we
give for calculating probability amplitude may provide the basis for
an \emph{effective} theory of probability.

\paragraph{Quantum vs ``biological'' information}
Finally, i want to conclude with a more philosophical observation. At
a recent workshop in which QM was a predominant topic i noticed
something about quantum information. The speaker was giving a riveting
discussion of axiomatic QM and showing how properties of ``no
cloning'' and ``no deleting'' emerged as consequences of the
axiomatization. Theorems of this form are necessary to give us a sense
of confidence that our axioms characterize the physical theory. What
struck me, though, was that if quantum information is neither erasable
nor replicable it is markedly different from \emph{life}. Two of the
things we know about life is that

\begin{itemize}
  \item it ends;
  \item to gain some measure of persistence, to transcend it's
    finitude it is imminently copyable.
\end{itemize}

Both of these qualities are summarized succinctly in the aphorism: all
flesh is grass. For me these two kinds of ``information'' -- call them
quantum and biological -- are end points on a spectrum of strategies
for persistence. At one end, we have those curious entities that enjoy
uniqueness and permanence; at the other, we have those who in the face
of a certain end and an uncertain present make a go of passing
something on. To me one of the more remarkable aspects of the latter
strategy is that in the presence of noise (and certain features of
copying) we get a kind of dynamism, a chance for improvement against a
given persistent condition.

% subsection other_calculi_other_bisimulations_and_geometry_as_behavior (end)




% section conclusion (end)

%\documentclass[12pt]{llncs}
%\documentclass{jktr}

\usepackage[pdftex]{hyperref}                   
\usepackage {listings}
\usepackage {mathpartir}
\usepackage{bcprules}
%\usepackage{listings}
                       
\usepackage{graphicx} 
%\usepackage[margins=2.5cm,nohead,nofoot]{geometry}
%\usepackage{geometry}
\usepackage{amsfonts}
\usepackage{amstext}
\usepackage{latexsym}
\usepackage{amssymb}
\usepackage{color}


%\include{myPreamble}
\documentclass[12pt]{llncs}
%\documentclass{jktr}

\usepackage[pdftex]{hyperref}                   
\usepackage {listings}
\usepackage {mathpartir}
\usepackage{bcprules}
%\usepackage{listings}
                       
\usepackage{graphicx} 
%\usepackage[margins=2.5cm,nohead,nofoot]{geometry}
%\usepackage{geometry}
\usepackage{amsfonts}
\usepackage{amstext}
\usepackage{latexsym}
\usepackage{amssymb}
\usepackage{color}


%\include{myPreamble}
\include{qm2pi.local} 

%\ifpdf
%\usepackage[pdftex]{graphicx}
%\else
%\usepackage{graphicx}
%\fi

 % \ifpdf
%  \usepackage{pdfsync}
%  \if


%\title{Brief Article}
%\author{David F. Snyder}
%\author{L.G. Meredith}

%\address{Dept. of Math., Texas State University--San Marcos, San Marcos, TX 78666}
       
\pagestyle{empty}


\begin{document}

\lstset{language=[Objective]Caml,frame=shadowbox}

\input{qm2pi.front}

% section front matter (end)

\input{qm2pi.intro} 
 
% section introduction (end)

% \input{qm2pi.knotations} 

% section notation (end)

\input{qm2pi.process.calculi} 

% section concurrent_process_calculi_and_spatial_logics_ (end)
    
%\input{qm2pi.knots2pi} 

%\input{qm2pi.trefoil} 

%\input{qm2pi.mainthm} 

% subsection basic_interpretation (end)

%\input{qm2pi.rho.presentation} 
\subsection{The syntax and semantics of the notation system}\label{sub:the_syntax_and_semantics_of_the_notation_system} % (fold)

We now summarize a technical presentation of the calculus that
embodies our theory of dynamics. The typical presentation of such a
calculus follows the style of giving generators and relations on
them. The grammar, below, describing term constructors, freely
generates the set of processes, $\Proc$. This set is then quotiented
by a relation known as structural congruence and it is over this set
that the notion of dynamics is expressed. This presentation is
essentially that of \cite{MeredithR05} with the addition of
polyadicity and summation. For readability we have relegated some of
the technical subtleties to an appendix.

\subsubsection{Process grammar}\label{subsub:process_grammar}

\begin{mathpar}
  \inferrule* [lab=synchronization] {} {{M} \bc \pzero \;|\; x?F \;|\; x!C }
  \and
  \inferrule* [lab=abstraction] {} {{F} \bc (x)P}
  \and
  \inferrule* [lab=concretion] {} {{C} \bc \langle Q \rangle}
  \and
  \inferrule* [lab=process] {} {{P,Q} \bc M \;| \;P|Q \;|\; @{x}}
  \and
  \inferrule* [lab=name] {} {{x} \bc \quotep{P}}
\end{mathpar} 

Note that $\vec{x}$ (resp. $\vec{P}$) denotes a vector of names
(resp. processes) of length $|\vec{x}|$ (resp. $|\vec{P}|$). We adopt
the following useful abbreviations.

\begin{mathpar}
   x?(\vec{y}).P := x.(\vec{y})P \and  x\clift{\vec{P}} := x.\clift{\vec{P}}
   \and x!(y) := \lift{x}{\dropn{y}}
   \and \Pi_{i=0}^{n-1}P_i := P_0 | \ldots | P_{n-1}
\end{mathpar}

\subsubsection{Structural congruence}

\paragraph{Free and bound names and alpha-equivalence.} At the
core of structural equivalence is alpha-equivalence which identifies
process that are the same up to a change of variable. Formally, we
recognize the distinction between free and bound names. The free names
of a process, $\freenames{P}$, may be calculated recursively as
follows:

\begin{mathpar}
\freenames{\pzero} := \emptyset
  \and \\
  \freenames{x?(y).P} := \{ x \} \cup (\freenames{P} \setminus \{ y \})
  \and 
  \freenames{x!\langle P \rangle} := \{ x \} \cup \{ P \} 
  \and \\
  \freenames{P|Q} := \freenames{P} \cup \freenames{Q}
  \and \\
  \freenames{@{x}} := \{ x \}
\end{mathpar}

$\pi$
$\quotep{\pi}$

$\freenames{-} : \pi \to \mathcal{P}(\quotep{\pi})$

\begin{eqnarray*}
  \freenames{\pzero} & := & \emptyset \\
  \freenames{x?(y).P} & := & \{ x \} \cup (\freenames{P} \setminus \{ y \}) \\
  \freenames{x!\langle P \rangle} & := & \{ x \} \cup \{ P \} \\
  \freenames{P|Q} & := & \freenames{P} \cup \freenames{Q} \\
  \freenames{\dropn{x}} & := & \{ x \}
\end{eqnarray*}

The bound names of a process, $\boundnames{P}$, are those names occurring in $P$
that are not free. For example, in $x?(y).0$, the name $x$ is free, while $y$ is bound.

\begin{mathpar}
  \inferrule* [lab=monoidal-laws] {} { P|Q \equiv Q|P \and P|0 \equiv P \and P|(Q|R) \equiv (P|Q)|R }
\end{mathpar}

\begin{mathpar}
  \inferrule* [lab=alpha-equivalence] {} { (x)P \equiv (y)P\{y/x\} \and y \not\in \freenames{P} }
\end{mathpar}

\begin{definition}
Then two processes, $P,Q$, are alpha-equivalent if $P = Q\{\vec{y}/\vec{x}\}$ for
some $\vec{x} \in \boundnames{Q},\vec{y} \in \boundnames{P}$, where $Q\{\vec{y}/\vec{x}\}$
denotes the capture-avoiding substitution of $\vec{y}$ for $\vec{x}$ in $Q$.
\end{definition}

\begin{definition}
  The {\em structural congruence} \cite{SangiorgiWalker} , $\equiv$,
  between processes is the least congruence containing
  alpha-equivalence, satisfying the abelian monoid laws
  (associativity, commutativity and $\pzero$ as identity) for parallel
  composition $|$ and for summation $+$.
\end{definition}

\subsection{Name equivalence}

We take name equivalence, written $\nameeq$, to be the smallest
equivalence relation generated by the following rules.

\begin{mathpar}
\inferrule*[lab=Quote-drop]
{ }
{ \quotep{@{x}} \nameeq x }

\inferrule*[lab=Struct-equiv]
{ P \scong Q }
{ \quotep{P} \nameeq \quotep{Q} }
\end{mathpar}

The astute reader will have noticed that the mutual recursion of names
and processes imposes a mutual recursion on alpha-equivalence and
structural equivalence via name-equivalence. Fortunately, all of this
works out pleasantly and we may calculate in the natural way, free of
concern. The reader interested in the details is referred to the
appendix \ref{appendix:rho_details}.

\subsection{Substitution}

We use $\Proc$ for the set of processes, $\QProc$ for the set of
names, and $\id{\{}\vec{y} / \vec{x} \id{\}}$ to denote partial maps,
$s : \QProc \rightarrow \QProc$. A map, $s$ lifts, uniquely, to a map
on process terms, $\widehat{s} : \Proc \rightarrow \Proc$ by the
following equations.

\begin{mathpar}
  (0) \psubstp{Q}{P} := 0 \\
  (R \juxtap S) \psubstp{Q}{P}
  :=    
  (R)\psubstp{Q}{P} \juxtap (S) \psubstp{Q}{P} \\
  (x?(y).R) \psubstp{Q}{P}    
  :=    
  (x)\substp{Q}{P} (z)\concat( (R \psubstn{z}{y}) \psubstp{Q}{P} ) \\
  (\lift{x}{R}) \psubstp{Q}{P}  
  :=
  \lift{(x)\substp{Q}{P}}{ R \psubstp{Q}{P} } \\
%   (\dropn{x})  \psubstp{Q}{P}       
%   := 
%   \left\{ 
%     \begin{array}{ccc} 
%       \dropn{\quotep{Q}} & & x \nameeq \quotep{P} \\
%       \dropn{x} & & otherwise \\
%     \end{array}
%   \right. 
  (\dropn{x})  \psubstp{Q}{P}       
  := 
  \left\{ 
    \begin{array}{ccc} 
      Q & & x \nameeq \quotep{P} \\
      \dropn{x} & & otherwise \\
    \end{array}
  \right.
\end{mathpar}
 

where

\begin{eqnarray}
  (x)\id{\{} \lpquote Q \rpquote / \lpquote P \rpquote \id{\}}            = 
  \left\{ 
    \begin{array}{ccc}
      \lpquote Q \rpquote & & x \nameeq \lpquote P \rpquote \\
      x & & otherwise \\
    \end{array}
  \right. \nonumber
\end{eqnarray}

and $z$ is chosen distinct from $\quotep{P}$, $\quotep{Q}$, the free
names in $Q$, and all the names in $R$. Our $\alpha$-equivalence will
be built in the standard way from this substitution.

\begin{remark}\label{rem:no_self_referential_names}
  One consequence of these definitions is that $\forall P. \quotep{P}
  \not\in \freenames{P}$.
\end{remark}

\subsection{ Dynamic quote: an example }

Anticipating something of what's to come, consider applying the
substitution, $\widehat{\id{\{}u / z \id{\}}}$, to the following pair
of processes, $\lift{w}{y!(z)}$ and $w[ \lpquote y!(z) \rpquote ]$.

\begin{eqnarray}
	\lift{w}{y!(z)}\widehat{\id{\{}u / z \id{\}}}
		& = &
		\lift{w}{y!(u)} \nonumber\\
	w[ \lpquote y!(z) \rpquote ] \widehat{ \id{\{}u / z \id{\}} }
		& = &
		w[ \lpquote y!(z) \rpquote ] \nonumber
\end{eqnarray}

Because the body of the process between quotes is impervious to
substitution, we get radically different answers. In fact, by
examining the first process in an input context,
e.g. $x?(z).\lift{w}{y!(z)}$, we see that the process under the lift
operator may be shaped by prefixed inputs binding a name inside it. In
this sense, the lift operator will be seen as a way to dynamically
construct processes before reifying them as names.

Finally equipped with these standard features we can present the
dynamics of the calculus.

\subsubsection{Operational semantics} 

Finally, we introduce the computational dynamics. What marks these
algebras as distinct from other more traditionally studied algebraic
structures, e.g. vector spaces or polynomial rings, is the manner in
which dynamics is captured. In traditional structures, dynamics is typically
expressed through morphisms between such structures, as in linear maps
between vector spaces or morphisms between rings. In algebras
associated with the semantics of computation, the dynamics is
expressed as part of the algebraic structure itself, through a
reduction reduction relation typically denoted by $\red$. Below, we
give a recursive presentation of this relation for the calculus used
in the encoding.

$\red \subseteq \pi \times \pi$
$\red : \pi \to \mathcal{P}(\pi)$

\begin{mathpar}
  \inferrule* [lab=Comm] { \textsf{match}( x_{src}, x_{trgt} ) } { x_{trgt}?(y)P \; | \; x_{src}!\langle {Q} \rangle \red P\{\quotep{Q}/y}\} }
  \and \\
  \inferrule* [lab=Par] {{P} \red {P}'} {{{P} | {Q}} \red {{P}' | {Q}}}
  \and
  \inferrule* [lab=Equiv]{{{P} \scong {P}'} \andalso {{P}' \red {Q}'} \andalso {{Q}' \scong {Q}}}{{P} \red {Q}}
\end{mathpar}

\begin{eqnarray*}
  match_{\equiv} (\quotep{P},\quotep{Q}) & := & P \equiv Q \\
  match_{\dagger}(\quotep{P},\quotep{Q}) & := & \forall R. P|Q \red^{*} R => R \red^{*} 0 \\
  match_{K}(\quotep{P},\quotep{Q}) & := & K \mbox{ for some context } K
\end{eqnarray*}

$u?(x)P | u!\langle Q \rangle \red P\{\quotep{Q}/x\}$

%We write $\wred$ for $\red^*$, and $P\red$ if $\exists Q $ such that $ P \red Q$.
We write $P\red$ if $\exists Q $ such that $ P \red Q$ and $P\not\red$, otherwise.

\section{Replication}

As mentioned before, it is known that replication (and hence
recursion) can be implemented in a higher-order process algebra
\cite{SangiorgiWalker}. As our first example of calculation with the
machinery thus far presented we give the construction explicitly in
the {\rhoc}.

\begin{eqnarray}
	D_{x} & := & \prefix{x}{y}{(\binpar{\outputp{x}{y}}{@{y}})} \nonumber\\
	\bangp_{x}{P} & := & \binpar{{x}!\langle{\binpar{D_{x}}{P}}\rangle}{D_{x}} \nonumber
\end{eqnarray}

\begin{eqnarray}
	\bangp_{x}{P} & & \nonumber\\
	=
	& {x}!\langle{(\prefix{x}{y}{(\outputp{x}{y} | @{y})) | P}}\rangle 
	      | \prefix{x}{y}{(\outputp{x}{y} | @{y})} & \nonumber\\
	\red
	& (\outputp{x}{y} | @{y})\substn{\quotep{(\prefix{x}{y}{(@{y} | \outputp{x}{y})) | P}}}{y} & \nonumber\\
	=
	& \outputp{x}{\quotep{(\prefix{x}{y}{(\outputp{x}{y} | @{y})) | P}}}
	  | {(\prefix{x}{y}{(\outputp{x}{y} | @{y})) | P}} & \nonumber\\
	\red
	& \ldots & \nonumber\\
	\red^*
	& P | P | \ldots & \nonumber
\end{eqnarray}

Of course, this encoding, as an implementation, runs away, unfolding
$\bangp{P}$ eagerly. A lazier and more implementable replication
operator, restricted to input-guarded processes, may be obtained as follows.

\begin{eqnarray}
\bangp{\prefix{u}{v}{P}} 
	:= 
	\binpar{\lift{x}{\prefix{u}{v}{(\binpar{D(x)}{P})}}}{D(x)} \nonumber
\end{eqnarray}

\begin{remark}
  Note that the lazier definition still does not deal with summation
  or mixed summation (i.e. sums over input and output). The reader is
  invited to construct definitions of replication that deal with these
  features. 

  Further, the definitions are parameterized in a name, $x$. Can you,
  gentle reader, make a definition that eliminates this parameter and
  guarantees no accidental interaction between the replication
  machinery and the process being replicated -- i.e. no accidental
  sharing of names used by the process to get its work done and the
  name(s) used by the replication to effect copying. This latter
  revision of the definition of replication is crucial to obtaining
  the expected identity $!!P \sim !P$.
\end{remark}

\begin{remark}\label{rem:paradoxical_combinator}
  The reader familiar with the lambda calculus will have noticed the
  similarity between $D$ and the paradoxical combinator.

  [Ed. note: the existence of this seems to suggest we have to be more
  restrictive on the set of processes and names we admit if we are to
  support no-cloning.]
\end{remark}

\subsubsection{Bisimulation}

The computational dynamics gives rise to another kind of equivalence,
the equivalence of computational behavior. As previously mentioned
this is typically captured \emph{via} some form of bisimulation.

% The notion we use in this paper is weak barbed bisimulation
% \cite{milner91polyadicpi}.

The notion we use in this paper is derived from weak barbed
bisimulation \cite{milner91polyadicpi}. 

\begin{definition}
An \emph{observation relation}, $\downarrow_{\mathcal N}$, over a set
of names, $\mathcal N$, is the smallest relation satisfying the rules
below.

\infrule[Out-barb]{y \in {\mathcal N}, \; x \nameeq y}
		  {\outputp{x}{v} \downarrow_{\mathcal N} x}
\infrule[Par-barb]{\mbox{$P\downarrow_{\mathcal N} x$ or $Q\downarrow_{\mathcal N} x$}}
		  {\binpar{P}{Q} \downarrow_{\mathcal N} x}

We write $P \Downarrow_{\mathcal N} x$ if there is $Q$ such that 
$P \wred Q$ and $Q \downarrow_{\mathcal N} x$.
\end{definition}

\begin{definition}
%\label{def.bbisim}
An  ${\mathcal N}$-\emph{barbed bisimulation} over a set of names, ${\mathcal N}$, is a symmetric binary relation 
${\mathcal S}_{\mathcal N}$ between agents such that $P\rel{S}_{\mathcal N}Q$ implies:
\begin{enumerate}
\item If $P \red P'$ then $Q \wred Q'$ and $P'\rel{S}_{\mathcal N} Q'$.
\item If $P\downarrow_{\mathcal N} x$, then $Q\Downarrow_{\mathcal N} x$.
\end{enumerate}
$P$ is ${\mathcal N}$-barbed bisimilar to $Q$, written
$P \wbbisim_{\mathcal N} Q$, if $P \rel{S}_{\mathcal N} Q$ for some ${\mathcal N}$-barbed bisimulation ${\mathcal S}_{\mathcal N}$.
\end{definition}

$\mathcal{R} \subseteq \pi \times \pi$

$P \mathcal{R} Q => \forall P'. P \red P' \Rightarrow \exists Q'. Q \red Q', P' \mathcal{R} Q'$

$P \vdash x \Rightarrow Q \vdash x$

\begin{mathpar}
  \inferrule*[lab=Out-barb]{x \nameeq y}{{y}!\langle{Q}\rangle \vdash x}
  \and
  \inferrule*[lab=Par-barb]{\mbox{$P\vdash x$ or $Q\vdash x$}}{\binpar{P}{Q} \vdash x}
\end{mathpar}

\subsubsection{Contexts}

One of the principle advantages of computational calculi like the
$\pi$-calculus is a well-defined notion of context,
contextual-equivalence and a correlation between
contextual-equivalence and notions of bisimulation. The notion of
context allows the decomposition of a process into (sub-)process and
its syntactic environment, its context. Thus, a context may be
thought of as a process with a ``hole'' (written $\Box$) in it. The
application of a context $M$ to a process $P$, written $M[P]$, is
tantamount to filling the hole in $M$ with $P$. In this paper we do
not need the full weight of this theory, but do make use of the notion
of context in the proof the main theorem. 

\begin{mathpar}
  \inferrule* [lab=summation] {} {{M_{M},M_{N}} \bc \Box \;|\; x.M_{A} \;|\; M_{M}+M_{N}}
  \and
  \inferrule* [lab=agent] {} {{M_{A}} \bc (\vec{x})M_{P} \;| \; \clift{P_0,\ldots,M_{P},\ldots,P_N}}
  \and \\
  \inferrule* [lab=process] {} {{M_{P}} \bc M_{N} \;| \;P|M_{P} }
\end{mathpar} 

\begin{mathpar}
  \inferrule* [lab=sychronization] {} {M_{N} \bc \Box \;|\; x?M_{F} \;|\; x!M_{C}}
  \and
  \inferrule* [lab=abstraction] {} {{M_{F}} \bc (x)M_{P} }
  \and
  \inferrule* [lab=concretion] {} {{M_{C}} \bc \langle M_{P} \rangle }
  \and \\
  \inferrule* [lab=process] {} {{M_{P}} \bc M_{N} \;| \;P|M_{P} }
\end{mathpar}

\begin{definition}[contextual application] Given a context $M$, and
  process $P$, we define the \emph{contextual application}, $M[P] :=
  M\{P/\Box\}$. That is, the contextual application of M to P is the
  substitution of $P$ for $\Box$ in $M$.
\end{definition}

$\meaningof{-} : L \to \mathcal{P}(\pi)$

\begin{mathpar}
  \inferrule* [lab=collection] {} {\meaningof{true} = \pi, \and \meaningof{~E} = \pi \setminus \meaningof{E}, \and \meaningof{E_{1} \& E_{2}} = \meaningof{E_{1}} \cap \meaningof{E_{2}}}
\end{mathpar}

\begin{mathpar}
  \inferrule* [lab=structure] {} {\meaningof{0} = \{ P \in \pi | P \equiv 0 \}, \and \\ \meaningof{E_1 | E_2} = \{ P \in \pi | P \equiv P_{1} | P_{2}, P_{1} \in \meaningof{E_{1}}, P_{2} \in \meaningof{E_2}\} }
\end{mathpar}

\begin{mathpar}
 \inferrule* [lab=behavior] {} {\meaningof{\langle a?b \rangle E} = \{ P \in \pi | P \equiv Q | u?(y)P', \\ \and \\\\ \and \\ \;\;\; u \in \meaningof{a}, \forall z.P'\{z/y\} \in \meaningof{E\{z/b\}}\}, \and \\ \meaningof{a!E} = \{ P \in \pi | P \equiv Q | x!\langle P' \rangle, x \in \meaningof{a} P' \in \meaningof{E}\} }
\end{mathpar}

\begin{mathpar}
 \inferrule* [lab=nominal] {} {\meaningof{\quotep{E}} = \{ \quotep{P} \in \quotep{\pi} | P \in \meaningof{E} \}, \and \meaningof{\quotep{P}} = \{ \quotep{Q} \in \quotep{\pi} | P \equiv Q \} \and \\ \meaningof{@\quotep{E}} = \{ P \in \pi | P \equiv @x, x \in \meaningof{E} \}}
\end{mathpar}

\begin{eqnarray*}
  \\
  \meaningof{-} : TS \to ST
\end{eqnarray*}

\begin{eqnarray*}
  \\
  L : TS \to ST
\end{eqnarray*}

\begin{eqnarray*}
  \\
  P \models E \iff P \in \meaningof{E}
\end{eqnarray*}

\begin{eqnarray*}
  P \approx_{L} Q \iff \forall E \in L. P \models E \iff Q \models E
\end{eqnarray*}

\begin{eqnarray*}
  P \approx_{K} Q
\end{eqnarray*}

\begin{eqnarray*}
  P \approx Q
\end{eqnarray*}

$\approx_{K} = \approx = \approx_{L}$

\subsubsection{Contextual duality}

Note that contexts extend the quotation operation to a family of
operations from processes to names. Given a context, $M$, we can
define a \emph{nominal context}, $\quotep{M}$ by $\quotep{M}[P] :=
\quotep{M[P]}$. To foreshadow what is to come we observe that these
operations enjoy a duality with processes very much like the duality
between vectors and maps from vectors to scalars.

Further, because the calculus is essentially higher-order, we have a
correspondence between contexts and processes. More specifically,
given a name $x$ and a context $M$ we can construct $M^{*}_{x}$ such
that 

\begin{mathpar}
  M^{*}_{x} | \lift{x}{P} \red M[P]
\end{mathpar}

namely,

\begin{mathpar}
  M^{*}_{x} := x?(u).M[\dropn{u}]
\end{mathpar}

The dependence of $M^{*}_{x}$ on a name makes it an abstraction, 

\begin{mathpar}
  M^{*} := (x)x?(u).M[\dropn{u}]
\end{mathpar}

\subsection{Additional notation}

It will sometimes be convenient to denote the process a name
quotes. We already have the notation $x = \quotep{P}$, but it will be
convenient to introduce an alternate notation, $\procn{x}$, when we
want to emphasize the connection to the use of the name. Note that, by
virtue of name equivalence, $\quotep{\procn{x}} \nameeq x$; so, the
notation is consistent with previous definitions.

Further, because names have structure it is possible to effect
substitutions on the basis of that structure. This means we need to
upgrade our notation for substitutions, which we accomplish by
adapting comprehension notation. Thus,

\begin{mathpar}
  P\{ y / x : x \in S \}
\end{mathpar}

is interpreted to mean the process derived from P by replacing (in a
capture-avoiding manner) each occurrence of $x$ in $S$ by $y$. For example,

\begin{mathpar}
  P\{ \quotep{\procn{x}|\procn{x}} / x : x \in \freenames{P} \}
\end{mathpar}

will replace each (occurrence) of a free name $x$ in $P$ by
$\quotep{\procn{x}|\procn{x}}$.

Also, we will avail ourselves of the notation $x^{L}$ and $x^{R}$ to
denote injections of a name into disjoint copies of the name
space. There are numerous ways to accomplish this. One example can be
found in \cite{MeredithR05}. This notation overloads to vectors of
names: $\vec{x}^{\pi} := (x_{i}^{\pi} \; : \; 0 \leq i < |\vec{x}| )$ where $\pi \in \{L,R\}$.

We also use $P^{\Box} := P|\Box$.

In \cite{MeredithR05} an interpretation of the new operator is
given. It turns out that there are several possible interpretations
all enjoying the requisite algebraic properties of the operator (see
\cite{milner91polyadicpi}). We will therefore make liberal use of
$(\nu\; \vec{x})P$.

% subsection the_syntax_and_semantics_of_the_notation_system (end)   

\input{qm2pi.qmops} 

\input{qm2pi.sterngerlach} 

\input{qm2pi.metric} 

% section concurrent_process_calculi (end)

%\input{qm2pi.proofsketch}

% section proof sketch (end)

%\input{qm2pi.slviaknots} 

% section spatial logic via knots (end)

\input{qm2pi.conclusion}

% section conclusion (end)

%\input{qm2pi.dtcodes} 

% section wiring algorithm (end)

\input{qm2pi.ack} 

% section acknowledgments (end)

\newpage


\bibliographystyle{plain}   
\bibliography{../../biblios/main.bib}

\input{qm2pi.rhodetails}

\end{document}

 

%\ifpdf
%\usepackage[pdftex]{graphicx}
%\else
%\usepackage{graphicx}
%\fi

 % \ifpdf
%  \usepackage{pdfsync}
%  \if


%\title{Brief Article}
%\author{David F. Snyder}
%\author{L.G. Meredith}

%\address{Dept. of Math., Texas State University--San Marcos, San Marcos, TX 78666}
       
\pagestyle{empty}


\begin{document}

\lstset{language=[Objective]Caml,frame=shadowbox}

\documentclass[12pt]{llncs}
%\documentclass{jktr}

\usepackage[pdftex]{hyperref}                   
\usepackage {listings}
\usepackage {mathpartir}
\usepackage{bcprules}
%\usepackage{listings}
                       
\usepackage{graphicx} 
%\usepackage[margins=2.5cm,nohead,nofoot]{geometry}
%\usepackage{geometry}
\usepackage{amsfonts}
\usepackage{amstext}
\usepackage{latexsym}
\usepackage{amssymb}
\usepackage{color}


%\include{myPreamble}
\include{qm2pi.local} 

%\ifpdf
%\usepackage[pdftex]{graphicx}
%\else
%\usepackage{graphicx}
%\fi

 % \ifpdf
%  \usepackage{pdfsync}
%  \if


%\title{Brief Article}
%\author{David F. Snyder}
%\author{L.G. Meredith}

%\address{Dept. of Math., Texas State University--San Marcos, San Marcos, TX 78666}
       
\pagestyle{empty}


\begin{document}

\lstset{language=[Objective]Caml,frame=shadowbox}

\input{qm2pi.front}

% section front matter (end)

\input{qm2pi.intro} 
 
% section introduction (end)

% \input{qm2pi.knotations} 

% section notation (end)

\input{qm2pi.process.calculi} 

% section concurrent_process_calculi_and_spatial_logics_ (end)
    
%\input{qm2pi.knots2pi} 

%\input{qm2pi.trefoil} 

%\input{qm2pi.mainthm} 

% subsection basic_interpretation (end)

%\input{qm2pi.rho.presentation} 
\subsection{The syntax and semantics of the notation system}\label{sub:the_syntax_and_semantics_of_the_notation_system} % (fold)

We now summarize a technical presentation of the calculus that
embodies our theory of dynamics. The typical presentation of such a
calculus follows the style of giving generators and relations on
them. The grammar, below, describing term constructors, freely
generates the set of processes, $\Proc$. This set is then quotiented
by a relation known as structural congruence and it is over this set
that the notion of dynamics is expressed. This presentation is
essentially that of \cite{MeredithR05} with the addition of
polyadicity and summation. For readability we have relegated some of
the technical subtleties to an appendix.

\subsubsection{Process grammar}\label{subsub:process_grammar}

\begin{mathpar}
  \inferrule* [lab=synchronization] {} {{M} \bc \pzero \;|\; x?F \;|\; x!C }
  \and
  \inferrule* [lab=abstraction] {} {{F} \bc (x)P}
  \and
  \inferrule* [lab=concretion] {} {{C} \bc \langle Q \rangle}
  \and
  \inferrule* [lab=process] {} {{P,Q} \bc M \;| \;P|Q \;|\; @{x}}
  \and
  \inferrule* [lab=name] {} {{x} \bc \quotep{P}}
\end{mathpar} 

Note that $\vec{x}$ (resp. $\vec{P}$) denotes a vector of names
(resp. processes) of length $|\vec{x}|$ (resp. $|\vec{P}|$). We adopt
the following useful abbreviations.

\begin{mathpar}
   x?(\vec{y}).P := x.(\vec{y})P \and  x\clift{\vec{P}} := x.\clift{\vec{P}}
   \and x!(y) := \lift{x}{\dropn{y}}
   \and \Pi_{i=0}^{n-1}P_i := P_0 | \ldots | P_{n-1}
\end{mathpar}

\subsubsection{Structural congruence}

\paragraph{Free and bound names and alpha-equivalence.} At the
core of structural equivalence is alpha-equivalence which identifies
process that are the same up to a change of variable. Formally, we
recognize the distinction between free and bound names. The free names
of a process, $\freenames{P}$, may be calculated recursively as
follows:

\begin{mathpar}
\freenames{\pzero} := \emptyset
  \and \\
  \freenames{x?(y).P} := \{ x \} \cup (\freenames{P} \setminus \{ y \})
  \and 
  \freenames{x!\langle P \rangle} := \{ x \} \cup \{ P \} 
  \and \\
  \freenames{P|Q} := \freenames{P} \cup \freenames{Q}
  \and \\
  \freenames{@{x}} := \{ x \}
\end{mathpar}

$\pi$
$\quotep{\pi}$

$\freenames{-} : \pi \to \mathcal{P}(\quotep{\pi})$

\begin{eqnarray*}
  \freenames{\pzero} & := & \emptyset \\
  \freenames{x?(y).P} & := & \{ x \} \cup (\freenames{P} \setminus \{ y \}) \\
  \freenames{x!\langle P \rangle} & := & \{ x \} \cup \{ P \} \\
  \freenames{P|Q} & := & \freenames{P} \cup \freenames{Q} \\
  \freenames{\dropn{x}} & := & \{ x \}
\end{eqnarray*}

The bound names of a process, $\boundnames{P}$, are those names occurring in $P$
that are not free. For example, in $x?(y).0$, the name $x$ is free, while $y$ is bound.

\begin{mathpar}
  \inferrule* [lab=monoidal-laws] {} { P|Q \equiv Q|P \and P|0 \equiv P \and P|(Q|R) \equiv (P|Q)|R }
\end{mathpar}

\begin{mathpar}
  \inferrule* [lab=alpha-equivalence] {} { (x)P \equiv (y)P\{y/x\} \and y \not\in \freenames{P} }
\end{mathpar}

\begin{definition}
Then two processes, $P,Q$, are alpha-equivalent if $P = Q\{\vec{y}/\vec{x}\}$ for
some $\vec{x} \in \boundnames{Q},\vec{y} \in \boundnames{P}$, where $Q\{\vec{y}/\vec{x}\}$
denotes the capture-avoiding substitution of $\vec{y}$ for $\vec{x}$ in $Q$.
\end{definition}

\begin{definition}
  The {\em structural congruence} \cite{SangiorgiWalker} , $\equiv$,
  between processes is the least congruence containing
  alpha-equivalence, satisfying the abelian monoid laws
  (associativity, commutativity and $\pzero$ as identity) for parallel
  composition $|$ and for summation $+$.
\end{definition}

\subsection{Name equivalence}

We take name equivalence, written $\nameeq$, to be the smallest
equivalence relation generated by the following rules.

\begin{mathpar}
\inferrule*[lab=Quote-drop]
{ }
{ \quotep{@{x}} \nameeq x }

\inferrule*[lab=Struct-equiv]
{ P \scong Q }
{ \quotep{P} \nameeq \quotep{Q} }
\end{mathpar}

The astute reader will have noticed that the mutual recursion of names
and processes imposes a mutual recursion on alpha-equivalence and
structural equivalence via name-equivalence. Fortunately, all of this
works out pleasantly and we may calculate in the natural way, free of
concern. The reader interested in the details is referred to the
appendix \ref{appendix:rho_details}.

\subsection{Substitution}

We use $\Proc$ for the set of processes, $\QProc$ for the set of
names, and $\id{\{}\vec{y} / \vec{x} \id{\}}$ to denote partial maps,
$s : \QProc \rightarrow \QProc$. A map, $s$ lifts, uniquely, to a map
on process terms, $\widehat{s} : \Proc \rightarrow \Proc$ by the
following equations.

\begin{mathpar}
  (0) \psubstp{Q}{P} := 0 \\
  (R \juxtap S) \psubstp{Q}{P}
  :=    
  (R)\psubstp{Q}{P} \juxtap (S) \psubstp{Q}{P} \\
  (x?(y).R) \psubstp{Q}{P}    
  :=    
  (x)\substp{Q}{P} (z)\concat( (R \psubstn{z}{y}) \psubstp{Q}{P} ) \\
  (\lift{x}{R}) \psubstp{Q}{P}  
  :=
  \lift{(x)\substp{Q}{P}}{ R \psubstp{Q}{P} } \\
%   (\dropn{x})  \psubstp{Q}{P}       
%   := 
%   \left\{ 
%     \begin{array}{ccc} 
%       \dropn{\quotep{Q}} & & x \nameeq \quotep{P} \\
%       \dropn{x} & & otherwise \\
%     \end{array}
%   \right. 
  (\dropn{x})  \psubstp{Q}{P}       
  := 
  \left\{ 
    \begin{array}{ccc} 
      Q & & x \nameeq \quotep{P} \\
      \dropn{x} & & otherwise \\
    \end{array}
  \right.
\end{mathpar}
 

where

\begin{eqnarray}
  (x)\id{\{} \lpquote Q \rpquote / \lpquote P \rpquote \id{\}}            = 
  \left\{ 
    \begin{array}{ccc}
      \lpquote Q \rpquote & & x \nameeq \lpquote P \rpquote \\
      x & & otherwise \\
    \end{array}
  \right. \nonumber
\end{eqnarray}

and $z$ is chosen distinct from $\quotep{P}$, $\quotep{Q}$, the free
names in $Q$, and all the names in $R$. Our $\alpha$-equivalence will
be built in the standard way from this substitution.

\begin{remark}\label{rem:no_self_referential_names}
  One consequence of these definitions is that $\forall P. \quotep{P}
  \not\in \freenames{P}$.
\end{remark}

\subsection{ Dynamic quote: an example }

Anticipating something of what's to come, consider applying the
substitution, $\widehat{\id{\{}u / z \id{\}}}$, to the following pair
of processes, $\lift{w}{y!(z)}$ and $w[ \lpquote y!(z) \rpquote ]$.

\begin{eqnarray}
	\lift{w}{y!(z)}\widehat{\id{\{}u / z \id{\}}}
		& = &
		\lift{w}{y!(u)} \nonumber\\
	w[ \lpquote y!(z) \rpquote ] \widehat{ \id{\{}u / z \id{\}} }
		& = &
		w[ \lpquote y!(z) \rpquote ] \nonumber
\end{eqnarray}

Because the body of the process between quotes is impervious to
substitution, we get radically different answers. In fact, by
examining the first process in an input context,
e.g. $x?(z).\lift{w}{y!(z)}$, we see that the process under the lift
operator may be shaped by prefixed inputs binding a name inside it. In
this sense, the lift operator will be seen as a way to dynamically
construct processes before reifying them as names.

Finally equipped with these standard features we can present the
dynamics of the calculus.

\subsubsection{Operational semantics} 

Finally, we introduce the computational dynamics. What marks these
algebras as distinct from other more traditionally studied algebraic
structures, e.g. vector spaces or polynomial rings, is the manner in
which dynamics is captured. In traditional structures, dynamics is typically
expressed through morphisms between such structures, as in linear maps
between vector spaces or morphisms between rings. In algebras
associated with the semantics of computation, the dynamics is
expressed as part of the algebraic structure itself, through a
reduction reduction relation typically denoted by $\red$. Below, we
give a recursive presentation of this relation for the calculus used
in the encoding.

$\red \subseteq \pi \times \pi$
$\red : \pi \to \mathcal{P}(\pi)$

\begin{mathpar}
  \inferrule* [lab=Comm] { \textsf{match}( x_{src}, x_{trgt} ) } { x_{trgt}?(y)P \; | \; x_{src}!\langle {Q} \rangle \red P\{\quotep{Q}/y}\} }
  \and \\
  \inferrule* [lab=Par] {{P} \red {P}'} {{{P} | {Q}} \red {{P}' | {Q}}}
  \and
  \inferrule* [lab=Equiv]{{{P} \scong {P}'} \andalso {{P}' \red {Q}'} \andalso {{Q}' \scong {Q}}}{{P} \red {Q}}
\end{mathpar}

\begin{eqnarray*}
  match_{\equiv} (\quotep{P},\quotep{Q}) & := & P \equiv Q \\
  match_{\dagger}(\quotep{P},\quotep{Q}) & := & \forall R. P|Q \red^{*} R => R \red^{*} 0 \\
  match_{K}(\quotep{P},\quotep{Q}) & := & K \mbox{ for some context } K
\end{eqnarray*}

$u?(x)P | u!\langle Q \rangle \red P\{\quotep{Q}/x\}$

%We write $\wred$ for $\red^*$, and $P\red$ if $\exists Q $ such that $ P \red Q$.
We write $P\red$ if $\exists Q $ such that $ P \red Q$ and $P\not\red$, otherwise.

\section{Replication}

As mentioned before, it is known that replication (and hence
recursion) can be implemented in a higher-order process algebra
\cite{SangiorgiWalker}. As our first example of calculation with the
machinery thus far presented we give the construction explicitly in
the {\rhoc}.

\begin{eqnarray}
	D_{x} & := & \prefix{x}{y}{(\binpar{\outputp{x}{y}}{@{y}})} \nonumber\\
	\bangp_{x}{P} & := & \binpar{{x}!\langle{\binpar{D_{x}}{P}}\rangle}{D_{x}} \nonumber
\end{eqnarray}

\begin{eqnarray}
	\bangp_{x}{P} & & \nonumber\\
	=
	& {x}!\langle{(\prefix{x}{y}{(\outputp{x}{y} | @{y})) | P}}\rangle 
	      | \prefix{x}{y}{(\outputp{x}{y} | @{y})} & \nonumber\\
	\red
	& (\outputp{x}{y} | @{y})\substn{\quotep{(\prefix{x}{y}{(@{y} | \outputp{x}{y})) | P}}}{y} & \nonumber\\
	=
	& \outputp{x}{\quotep{(\prefix{x}{y}{(\outputp{x}{y} | @{y})) | P}}}
	  | {(\prefix{x}{y}{(\outputp{x}{y} | @{y})) | P}} & \nonumber\\
	\red
	& \ldots & \nonumber\\
	\red^*
	& P | P | \ldots & \nonumber
\end{eqnarray}

Of course, this encoding, as an implementation, runs away, unfolding
$\bangp{P}$ eagerly. A lazier and more implementable replication
operator, restricted to input-guarded processes, may be obtained as follows.

\begin{eqnarray}
\bangp{\prefix{u}{v}{P}} 
	:= 
	\binpar{\lift{x}{\prefix{u}{v}{(\binpar{D(x)}{P})}}}{D(x)} \nonumber
\end{eqnarray}

\begin{remark}
  Note that the lazier definition still does not deal with summation
  or mixed summation (i.e. sums over input and output). The reader is
  invited to construct definitions of replication that deal with these
  features. 

  Further, the definitions are parameterized in a name, $x$. Can you,
  gentle reader, make a definition that eliminates this parameter and
  guarantees no accidental interaction between the replication
  machinery and the process being replicated -- i.e. no accidental
  sharing of names used by the process to get its work done and the
  name(s) used by the replication to effect copying. This latter
  revision of the definition of replication is crucial to obtaining
  the expected identity $!!P \sim !P$.
\end{remark}

\begin{remark}\label{rem:paradoxical_combinator}
  The reader familiar with the lambda calculus will have noticed the
  similarity between $D$ and the paradoxical combinator.

  [Ed. note: the existence of this seems to suggest we have to be more
  restrictive on the set of processes and names we admit if we are to
  support no-cloning.]
\end{remark}

\subsubsection{Bisimulation}

The computational dynamics gives rise to another kind of equivalence,
the equivalence of computational behavior. As previously mentioned
this is typically captured \emph{via} some form of bisimulation.

% The notion we use in this paper is weak barbed bisimulation
% \cite{milner91polyadicpi}.

The notion we use in this paper is derived from weak barbed
bisimulation \cite{milner91polyadicpi}. 

\begin{definition}
An \emph{observation relation}, $\downarrow_{\mathcal N}$, over a set
of names, $\mathcal N$, is the smallest relation satisfying the rules
below.

\infrule[Out-barb]{y \in {\mathcal N}, \; x \nameeq y}
		  {\outputp{x}{v} \downarrow_{\mathcal N} x}
\infrule[Par-barb]{\mbox{$P\downarrow_{\mathcal N} x$ or $Q\downarrow_{\mathcal N} x$}}
		  {\binpar{P}{Q} \downarrow_{\mathcal N} x}

We write $P \Downarrow_{\mathcal N} x$ if there is $Q$ such that 
$P \wred Q$ and $Q \downarrow_{\mathcal N} x$.
\end{definition}

\begin{definition}
%\label{def.bbisim}
An  ${\mathcal N}$-\emph{barbed bisimulation} over a set of names, ${\mathcal N}$, is a symmetric binary relation 
${\mathcal S}_{\mathcal N}$ between agents such that $P\rel{S}_{\mathcal N}Q$ implies:
\begin{enumerate}
\item If $P \red P'$ then $Q \wred Q'$ and $P'\rel{S}_{\mathcal N} Q'$.
\item If $P\downarrow_{\mathcal N} x$, then $Q\Downarrow_{\mathcal N} x$.
\end{enumerate}
$P$ is ${\mathcal N}$-barbed bisimilar to $Q$, written
$P \wbbisim_{\mathcal N} Q$, if $P \rel{S}_{\mathcal N} Q$ for some ${\mathcal N}$-barbed bisimulation ${\mathcal S}_{\mathcal N}$.
\end{definition}

$\mathcal{R} \subseteq \pi \times \pi$

$P \mathcal{R} Q => \forall P'. P \red P' \Rightarrow \exists Q'. Q \red Q', P' \mathcal{R} Q'$

$P \vdash x \Rightarrow Q \vdash x$

\begin{mathpar}
  \inferrule*[lab=Out-barb]{x \nameeq y}{{y}!\langle{Q}\rangle \vdash x}
  \and
  \inferrule*[lab=Par-barb]{\mbox{$P\vdash x$ or $Q\vdash x$}}{\binpar{P}{Q} \vdash x}
\end{mathpar}

\subsubsection{Contexts}

One of the principle advantages of computational calculi like the
$\pi$-calculus is a well-defined notion of context,
contextual-equivalence and a correlation between
contextual-equivalence and notions of bisimulation. The notion of
context allows the decomposition of a process into (sub-)process and
its syntactic environment, its context. Thus, a context may be
thought of as a process with a ``hole'' (written $\Box$) in it. The
application of a context $M$ to a process $P$, written $M[P]$, is
tantamount to filling the hole in $M$ with $P$. In this paper we do
not need the full weight of this theory, but do make use of the notion
of context in the proof the main theorem. 

\begin{mathpar}
  \inferrule* [lab=summation] {} {{M_{M},M_{N}} \bc \Box \;|\; x.M_{A} \;|\; M_{M}+M_{N}}
  \and
  \inferrule* [lab=agent] {} {{M_{A}} \bc (\vec{x})M_{P} \;| \; \clift{P_0,\ldots,M_{P},\ldots,P_N}}
  \and \\
  \inferrule* [lab=process] {} {{M_{P}} \bc M_{N} \;| \;P|M_{P} }
\end{mathpar} 

\begin{mathpar}
  \inferrule* [lab=sychronization] {} {M_{N} \bc \Box \;|\; x?M_{F} \;|\; x!M_{C}}
  \and
  \inferrule* [lab=abstraction] {} {{M_{F}} \bc (x)M_{P} }
  \and
  \inferrule* [lab=concretion] {} {{M_{C}} \bc \langle M_{P} \rangle }
  \and \\
  \inferrule* [lab=process] {} {{M_{P}} \bc M_{N} \;| \;P|M_{P} }
\end{mathpar}

\begin{definition}[contextual application] Given a context $M$, and
  process $P$, we define the \emph{contextual application}, $M[P] :=
  M\{P/\Box\}$. That is, the contextual application of M to P is the
  substitution of $P$ for $\Box$ in $M$.
\end{definition}

$\meaningof{-} : L \to \mathcal{P}(\pi)$

\begin{mathpar}
  \inferrule* [lab=collection] {} {\meaningof{true} = \pi, \and \meaningof{~E} = \pi \setminus \meaningof{E}, \and \meaningof{E_{1} \& E_{2}} = \meaningof{E_{1}} \cap \meaningof{E_{2}}}
\end{mathpar}

\begin{mathpar}
  \inferrule* [lab=structure] {} {\meaningof{0} = \{ P \in \pi | P \equiv 0 \}, \and \\ \meaningof{E_1 | E_2} = \{ P \in \pi | P \equiv P_{1} | P_{2}, P_{1} \in \meaningof{E_{1}}, P_{2} \in \meaningof{E_2}\} }
\end{mathpar}

\begin{mathpar}
 \inferrule* [lab=behavior] {} {\meaningof{\langle a?b \rangle E} = \{ P \in \pi | P \equiv Q | u?(y)P', \\ \and \\\\ \and \\ \;\;\; u \in \meaningof{a}, \forall z.P'\{z/y\} \in \meaningof{E\{z/b\}}\}, \and \\ \meaningof{a!E} = \{ P \in \pi | P \equiv Q | x!\langle P' \rangle, x \in \meaningof{a} P' \in \meaningof{E}\} }
\end{mathpar}

\begin{mathpar}
 \inferrule* [lab=nominal] {} {\meaningof{\quotep{E}} = \{ \quotep{P} \in \quotep{\pi} | P \in \meaningof{E} \}, \and \meaningof{\quotep{P}} = \{ \quotep{Q} \in \quotep{\pi} | P \equiv Q \} \and \\ \meaningof{@\quotep{E}} = \{ P \in \pi | P \equiv @x, x \in \meaningof{E} \}}
\end{mathpar}

\begin{eqnarray*}
  \\
  \meaningof{-} : TS \to ST
\end{eqnarray*}

\begin{eqnarray*}
  \\
  L : TS \to ST
\end{eqnarray*}

\begin{eqnarray*}
  \\
  P \models E \iff P \in \meaningof{E}
\end{eqnarray*}

\begin{eqnarray*}
  P \approx_{L} Q \iff \forall E \in L. P \models E \iff Q \models E
\end{eqnarray*}

\begin{eqnarray*}
  P \approx_{K} Q
\end{eqnarray*}

\begin{eqnarray*}
  P \approx Q
\end{eqnarray*}

$\approx_{K} = \approx = \approx_{L}$

\subsubsection{Contextual duality}

Note that contexts extend the quotation operation to a family of
operations from processes to names. Given a context, $M$, we can
define a \emph{nominal context}, $\quotep{M}$ by $\quotep{M}[P] :=
\quotep{M[P]}$. To foreshadow what is to come we observe that these
operations enjoy a duality with processes very much like the duality
between vectors and maps from vectors to scalars.

Further, because the calculus is essentially higher-order, we have a
correspondence between contexts and processes. More specifically,
given a name $x$ and a context $M$ we can construct $M^{*}_{x}$ such
that 

\begin{mathpar}
  M^{*}_{x} | \lift{x}{P} \red M[P]
\end{mathpar}

namely,

\begin{mathpar}
  M^{*}_{x} := x?(u).M[\dropn{u}]
\end{mathpar}

The dependence of $M^{*}_{x}$ on a name makes it an abstraction, 

\begin{mathpar}
  M^{*} := (x)x?(u).M[\dropn{u}]
\end{mathpar}

\subsection{Additional notation}

It will sometimes be convenient to denote the process a name
quotes. We already have the notation $x = \quotep{P}$, but it will be
convenient to introduce an alternate notation, $\procn{x}$, when we
want to emphasize the connection to the use of the name. Note that, by
virtue of name equivalence, $\quotep{\procn{x}} \nameeq x$; so, the
notation is consistent with previous definitions.

Further, because names have structure it is possible to effect
substitutions on the basis of that structure. This means we need to
upgrade our notation for substitutions, which we accomplish by
adapting comprehension notation. Thus,

\begin{mathpar}
  P\{ y / x : x \in S \}
\end{mathpar}

is interpreted to mean the process derived from P by replacing (in a
capture-avoiding manner) each occurrence of $x$ in $S$ by $y$. For example,

\begin{mathpar}
  P\{ \quotep{\procn{x}|\procn{x}} / x : x \in \freenames{P} \}
\end{mathpar}

will replace each (occurrence) of a free name $x$ in $P$ by
$\quotep{\procn{x}|\procn{x}}$.

Also, we will avail ourselves of the notation $x^{L}$ and $x^{R}$ to
denote injections of a name into disjoint copies of the name
space. There are numerous ways to accomplish this. One example can be
found in \cite{MeredithR05}. This notation overloads to vectors of
names: $\vec{x}^{\pi} := (x_{i}^{\pi} \; : \; 0 \leq i < |\vec{x}| )$ where $\pi \in \{L,R\}$.

We also use $P^{\Box} := P|\Box$.

In \cite{MeredithR05} an interpretation of the new operator is
given. It turns out that there are several possible interpretations
all enjoying the requisite algebraic properties of the operator (see
\cite{milner91polyadicpi}). We will therefore make liberal use of
$(\nu\; \vec{x})P$.

% subsection the_syntax_and_semantics_of_the_notation_system (end)   

\input{qm2pi.qmops} 

\input{qm2pi.sterngerlach} 

\input{qm2pi.metric} 

% section concurrent_process_calculi (end)

%\input{qm2pi.proofsketch}

% section proof sketch (end)

%\input{qm2pi.slviaknots} 

% section spatial logic via knots (end)

\input{qm2pi.conclusion}

% section conclusion (end)

%\input{qm2pi.dtcodes} 

% section wiring algorithm (end)

\input{qm2pi.ack} 

% section acknowledgments (end)

\newpage


\bibliographystyle{plain}   
\bibliography{../../biblios/main.bib}

\input{qm2pi.rhodetails}

\end{document}



% section front matter (end)

\section{Introduction}\label{sec:introduction} % (fold)
In this draft of the material i am going to have to dispense with the
usual writing conventions adopted in papers on these topics. i'm going
to have adopt whatever tone i need at the time i'm writing up the
calculations. Sometimes this may be very conversational; others it may
be the barest mathematical grunts; others still it may be that i have
lifted text from one of my other papers because the exposition of some
point was better said there. i hope that my readers are not unduly put
out by this decision. i'm not doing this to flout convention or be
rebellious. i find these calculations very technically challenging. To
keep everything going technically, something has to give; i have to
let go of some cognitive burden. So, the academic writing style --
with all of its trade-offs in terms of facilitating technical
communication -- is what i'm letting go of. Perhaps subsequent drafts
can be tightened and polished, but for now, i'm going to speak as if
we were sitting together in a coffee shop with a laptop, wifi and a
pad of paper and a pencil.

So, here's what i have to say. We -- you and i, comfortably ensconced
in our coffee shop and well-equipped with our tools -- can realize and
carry out the calculations of quantum mechanics over a very different
formal theory of dynamics, a formal theory of dynamics that
corresponds to a theory of concurrent computation with
\emph{reflection}. It has the advantage that the underlying theory is
already `quantized', but supports analogues all of the continuuous
operations. Strikingly, this underlying theory has recently been
connected with a notion of metric that we can show, by calculating
together, coincides with the metric induced by the inner product.

There are a lot of reasons why you might be interested in seeing
calculations of this form. Here's why i'm interested. For the past
several centuries there has been no competitor to the ``Newtonian''
account of dynamics. As a result the predominant share of accounts of
dynamical systems and situations have had to be formulated in terms of
the Newtonian machinery. i view this as an intellectually dangerous
position to occupy. Everything, despite it's intrinsic shape, turns
into a nail to be hit with this hammer. Recently, however, the theory
of computation has matured to the point where we have candidates for
theories of dynamics that offer very different perspective on
reasoning about dynamical systems and situations. Testing these
candidates against very successful accounts of dynamical situations,
like quantum mechanics, is going to give us some sense of how mature
they are and some measure of the quality of these accounts of
dynamics.

\subsection{Summary of contributions and outline of paper}

So, we're going to develop an interpretation of the operations of
quantum mechanics normally interpreted by Hilbert spaces and
operators. We're going to do this over a theory of computation. Note
that this is very different than the usual quantum computation program
which develops notions of computation over quantum mechanics. Rather,
we are developing a story that aligns with Wheeler's slogan: It from
Bit. To do this we will first provide an account of the theory of
computation at play here. Then we will dive into a calculation-driven
interpretation of the operations of quantum mechanics.

The reason we take this approach is that -- until very recently --
there hasn't been an axiomatic account of quantum mechanics. As a
result there has been no sharp delineation of the mathematical theory
supporting interpretation of the physical theory and the physical
theory, itself. So, ambient features of the maths are free to be
exploited (or supressed) without a real accounting of their physical
relevance. There is no sharp statement ``here's the physical theory''
qua \emph{theory} and ``here's the mathematical interpretation''
enabling a judgment of how faithful the interpretation is -- apart
from experimental observation. When there is an axiomatic account we
can judge how well a given mathematical formalism supports an
interpretation of the axioms, independent of
experimentation. Likewise, we can judge how well we have captured our
physical evidence and experience with our axiomatics, independent of
any specific mathematical implementation, with accidental detail that
may or may not have physical significance. 

In lieu of a fully fleshed out and vetted axiomatic account of quantum
mechanics, interpreting the operational notions in service of modeling
physical systems will have to suffice. In other words, we are not in
the business of providing a model of Hilbert spaces and operators. We
are in the business of providing a model of quantum mechanics because
we are motivated by testing our notions of dynamics against physical
theory; and, the predictive calculations of the physical theory must
serve as the best formulation -- shy of a fully fleshed out axiomatic
account -- of the physical theory itself (as they have for scientific
theories since time immemorial). Put another way, despite a
whole-hearted commitment to an It-from-Bit ontology, we are firmly
aligned with the shut-up-and-calculate camp as the best way to obtain
results either from the physical perspective or as a quality assurance
measure of our fledgling theory of dynamics.

In detail, we present a reflective process calculus. Then we develop
intuitive correspondences between the notions available in this
calculus and the usual physical notions supporting quantum mechanical
calculations. Thus, 

\begin{table}[htp]
  \center{
    \fbox{
      \begin{tabular}{c|c}
        quantum mechanics & process calculus \\
        \hline
        scalar & name \\
        state vector & process \\
        dual & contextual duals \\
        matrix & formal sums of process-context-dual pairs \\
        orthogonality & process annihilation \\
        inner product & execution-formula + quoting
      \end{tabular}
    }
  }
  \caption{QM - process calculi correspondences}
\end{table}

Then we tighten up these intuitions to operational definitions. We
employ the Dirac notation as the best proxy we can find for an
abstract syntax of the quantum mechanical notions. The definitions we
develop put us in contact with equational constraints coming from the
theory that we demonstrate the definitions and calculations satisfy.

This puts us in a position to shut up and calculate for the
Stern-Gerlach experimental set up, showing how these predictive
calculations become calculations on processes in our theory of a
reflective process calculus.

Penultimately, we demonstrate that the notion of metric coming from
the inner product coincides with the notion of metric available from
the theory of bisimulation. This demonstration gives us the right to
think of space as arising from behavior. Finally, we consider where we
might go from the new vantage point we have obtained.

% section introduction (end) 
 
% section introduction (end)

% \documentclass[12pt]{llncs}
%\documentclass{jktr}

\usepackage[pdftex]{hyperref}                   
\usepackage {listings}
\usepackage {mathpartir}
\usepackage{bcprules}
%\usepackage{listings}
                       
\usepackage{graphicx} 
%\usepackage[margins=2.5cm,nohead,nofoot]{geometry}
%\usepackage{geometry}
\usepackage{amsfonts}
\usepackage{amstext}
\usepackage{latexsym}
\usepackage{amssymb}
\usepackage{color}


%\include{myPreamble}
\include{qm2pi.local} 

%\ifpdf
%\usepackage[pdftex]{graphicx}
%\else
%\usepackage{graphicx}
%\fi

 % \ifpdf
%  \usepackage{pdfsync}
%  \if


%\title{Brief Article}
%\author{David F. Snyder}
%\author{L.G. Meredith}

%\address{Dept. of Math., Texas State University--San Marcos, San Marcos, TX 78666}
       
\pagestyle{empty}


\begin{document}

\lstset{language=[Objective]Caml,frame=shadowbox}

\input{qm2pi.front}

% section front matter (end)

\input{qm2pi.intro} 
 
% section introduction (end)

% \input{qm2pi.knotations} 

% section notation (end)

\input{qm2pi.process.calculi} 

% section concurrent_process_calculi_and_spatial_logics_ (end)
    
%\input{qm2pi.knots2pi} 

%\input{qm2pi.trefoil} 

%\input{qm2pi.mainthm} 

% subsection basic_interpretation (end)

%\input{qm2pi.rho.presentation} 
\subsection{The syntax and semantics of the notation system}\label{sub:the_syntax_and_semantics_of_the_notation_system} % (fold)

We now summarize a technical presentation of the calculus that
embodies our theory of dynamics. The typical presentation of such a
calculus follows the style of giving generators and relations on
them. The grammar, below, describing term constructors, freely
generates the set of processes, $\Proc$. This set is then quotiented
by a relation known as structural congruence and it is over this set
that the notion of dynamics is expressed. This presentation is
essentially that of \cite{MeredithR05} with the addition of
polyadicity and summation. For readability we have relegated some of
the technical subtleties to an appendix.

\subsubsection{Process grammar}\label{subsub:process_grammar}

\begin{mathpar}
  \inferrule* [lab=synchronization] {} {{M} \bc \pzero \;|\; x?F \;|\; x!C }
  \and
  \inferrule* [lab=abstraction] {} {{F} \bc (x)P}
  \and
  \inferrule* [lab=concretion] {} {{C} \bc \langle Q \rangle}
  \and
  \inferrule* [lab=process] {} {{P,Q} \bc M \;| \;P|Q \;|\; @{x}}
  \and
  \inferrule* [lab=name] {} {{x} \bc \quotep{P}}
\end{mathpar} 

Note that $\vec{x}$ (resp. $\vec{P}$) denotes a vector of names
(resp. processes) of length $|\vec{x}|$ (resp. $|\vec{P}|$). We adopt
the following useful abbreviations.

\begin{mathpar}
   x?(\vec{y}).P := x.(\vec{y})P \and  x\clift{\vec{P}} := x.\clift{\vec{P}}
   \and x!(y) := \lift{x}{\dropn{y}}
   \and \Pi_{i=0}^{n-1}P_i := P_0 | \ldots | P_{n-1}
\end{mathpar}

\subsubsection{Structural congruence}

\paragraph{Free and bound names and alpha-equivalence.} At the
core of structural equivalence is alpha-equivalence which identifies
process that are the same up to a change of variable. Formally, we
recognize the distinction between free and bound names. The free names
of a process, $\freenames{P}$, may be calculated recursively as
follows:

\begin{mathpar}
\freenames{\pzero} := \emptyset
  \and \\
  \freenames{x?(y).P} := \{ x \} \cup (\freenames{P} \setminus \{ y \})
  \and 
  \freenames{x!\langle P \rangle} := \{ x \} \cup \{ P \} 
  \and \\
  \freenames{P|Q} := \freenames{P} \cup \freenames{Q}
  \and \\
  \freenames{@{x}} := \{ x \}
\end{mathpar}

$\pi$
$\quotep{\pi}$

$\freenames{-} : \pi \to \mathcal{P}(\quotep{\pi})$

\begin{eqnarray*}
  \freenames{\pzero} & := & \emptyset \\
  \freenames{x?(y).P} & := & \{ x \} \cup (\freenames{P} \setminus \{ y \}) \\
  \freenames{x!\langle P \rangle} & := & \{ x \} \cup \{ P \} \\
  \freenames{P|Q} & := & \freenames{P} \cup \freenames{Q} \\
  \freenames{\dropn{x}} & := & \{ x \}
\end{eqnarray*}

The bound names of a process, $\boundnames{P}$, are those names occurring in $P$
that are not free. For example, in $x?(y).0$, the name $x$ is free, while $y$ is bound.

\begin{mathpar}
  \inferrule* [lab=monoidal-laws] {} { P|Q \equiv Q|P \and P|0 \equiv P \and P|(Q|R) \equiv (P|Q)|R }
\end{mathpar}

\begin{mathpar}
  \inferrule* [lab=alpha-equivalence] {} { (x)P \equiv (y)P\{y/x\} \and y \not\in \freenames{P} }
\end{mathpar}

\begin{definition}
Then two processes, $P,Q$, are alpha-equivalent if $P = Q\{\vec{y}/\vec{x}\}$ for
some $\vec{x} \in \boundnames{Q},\vec{y} \in \boundnames{P}$, where $Q\{\vec{y}/\vec{x}\}$
denotes the capture-avoiding substitution of $\vec{y}$ for $\vec{x}$ in $Q$.
\end{definition}

\begin{definition}
  The {\em structural congruence} \cite{SangiorgiWalker} , $\equiv$,
  between processes is the least congruence containing
  alpha-equivalence, satisfying the abelian monoid laws
  (associativity, commutativity and $\pzero$ as identity) for parallel
  composition $|$ and for summation $+$.
\end{definition}

\subsection{Name equivalence}

We take name equivalence, written $\nameeq$, to be the smallest
equivalence relation generated by the following rules.

\begin{mathpar}
\inferrule*[lab=Quote-drop]
{ }
{ \quotep{@{x}} \nameeq x }

\inferrule*[lab=Struct-equiv]
{ P \scong Q }
{ \quotep{P} \nameeq \quotep{Q} }
\end{mathpar}

The astute reader will have noticed that the mutual recursion of names
and processes imposes a mutual recursion on alpha-equivalence and
structural equivalence via name-equivalence. Fortunately, all of this
works out pleasantly and we may calculate in the natural way, free of
concern. The reader interested in the details is referred to the
appendix \ref{appendix:rho_details}.

\subsection{Substitution}

We use $\Proc$ for the set of processes, $\QProc$ for the set of
names, and $\id{\{}\vec{y} / \vec{x} \id{\}}$ to denote partial maps,
$s : \QProc \rightarrow \QProc$. A map, $s$ lifts, uniquely, to a map
on process terms, $\widehat{s} : \Proc \rightarrow \Proc$ by the
following equations.

\begin{mathpar}
  (0) \psubstp{Q}{P} := 0 \\
  (R \juxtap S) \psubstp{Q}{P}
  :=    
  (R)\psubstp{Q}{P} \juxtap (S) \psubstp{Q}{P} \\
  (x?(y).R) \psubstp{Q}{P}    
  :=    
  (x)\substp{Q}{P} (z)\concat( (R \psubstn{z}{y}) \psubstp{Q}{P} ) \\
  (\lift{x}{R}) \psubstp{Q}{P}  
  :=
  \lift{(x)\substp{Q}{P}}{ R \psubstp{Q}{P} } \\
%   (\dropn{x})  \psubstp{Q}{P}       
%   := 
%   \left\{ 
%     \begin{array}{ccc} 
%       \dropn{\quotep{Q}} & & x \nameeq \quotep{P} \\
%       \dropn{x} & & otherwise \\
%     \end{array}
%   \right. 
  (\dropn{x})  \psubstp{Q}{P}       
  := 
  \left\{ 
    \begin{array}{ccc} 
      Q & & x \nameeq \quotep{P} \\
      \dropn{x} & & otherwise \\
    \end{array}
  \right.
\end{mathpar}
 

where

\begin{eqnarray}
  (x)\id{\{} \lpquote Q \rpquote / \lpquote P \rpquote \id{\}}            = 
  \left\{ 
    \begin{array}{ccc}
      \lpquote Q \rpquote & & x \nameeq \lpquote P \rpquote \\
      x & & otherwise \\
    \end{array}
  \right. \nonumber
\end{eqnarray}

and $z$ is chosen distinct from $\quotep{P}$, $\quotep{Q}$, the free
names in $Q$, and all the names in $R$. Our $\alpha$-equivalence will
be built in the standard way from this substitution.

\begin{remark}\label{rem:no_self_referential_names}
  One consequence of these definitions is that $\forall P. \quotep{P}
  \not\in \freenames{P}$.
\end{remark}

\subsection{ Dynamic quote: an example }

Anticipating something of what's to come, consider applying the
substitution, $\widehat{\id{\{}u / z \id{\}}}$, to the following pair
of processes, $\lift{w}{y!(z)}$ and $w[ \lpquote y!(z) \rpquote ]$.

\begin{eqnarray}
	\lift{w}{y!(z)}\widehat{\id{\{}u / z \id{\}}}
		& = &
		\lift{w}{y!(u)} \nonumber\\
	w[ \lpquote y!(z) \rpquote ] \widehat{ \id{\{}u / z \id{\}} }
		& = &
		w[ \lpquote y!(z) \rpquote ] \nonumber
\end{eqnarray}

Because the body of the process between quotes is impervious to
substitution, we get radically different answers. In fact, by
examining the first process in an input context,
e.g. $x?(z).\lift{w}{y!(z)}$, we see that the process under the lift
operator may be shaped by prefixed inputs binding a name inside it. In
this sense, the lift operator will be seen as a way to dynamically
construct processes before reifying them as names.

Finally equipped with these standard features we can present the
dynamics of the calculus.

\subsubsection{Operational semantics} 

Finally, we introduce the computational dynamics. What marks these
algebras as distinct from other more traditionally studied algebraic
structures, e.g. vector spaces or polynomial rings, is the manner in
which dynamics is captured. In traditional structures, dynamics is typically
expressed through morphisms between such structures, as in linear maps
between vector spaces or morphisms between rings. In algebras
associated with the semantics of computation, the dynamics is
expressed as part of the algebraic structure itself, through a
reduction reduction relation typically denoted by $\red$. Below, we
give a recursive presentation of this relation for the calculus used
in the encoding.

$\red \subseteq \pi \times \pi$
$\red : \pi \to \mathcal{P}(\pi)$

\begin{mathpar}
  \inferrule* [lab=Comm] { \textsf{match}( x_{src}, x_{trgt} ) } { x_{trgt}?(y)P \; | \; x_{src}!\langle {Q} \rangle \red P\{\quotep{Q}/y}\} }
  \and \\
  \inferrule* [lab=Par] {{P} \red {P}'} {{{P} | {Q}} \red {{P}' | {Q}}}
  \and
  \inferrule* [lab=Equiv]{{{P} \scong {P}'} \andalso {{P}' \red {Q}'} \andalso {{Q}' \scong {Q}}}{{P} \red {Q}}
\end{mathpar}

\begin{eqnarray*}
  match_{\equiv} (\quotep{P},\quotep{Q}) & := & P \equiv Q \\
  match_{\dagger}(\quotep{P},\quotep{Q}) & := & \forall R. P|Q \red^{*} R => R \red^{*} 0 \\
  match_{K}(\quotep{P},\quotep{Q}) & := & K \mbox{ for some context } K
\end{eqnarray*}

$u?(x)P | u!\langle Q \rangle \red P\{\quotep{Q}/x\}$

%We write $\wred$ for $\red^*$, and $P\red$ if $\exists Q $ such that $ P \red Q$.
We write $P\red$ if $\exists Q $ such that $ P \red Q$ and $P\not\red$, otherwise.

\section{Replication}

As mentioned before, it is known that replication (and hence
recursion) can be implemented in a higher-order process algebra
\cite{SangiorgiWalker}. As our first example of calculation with the
machinery thus far presented we give the construction explicitly in
the {\rhoc}.

\begin{eqnarray}
	D_{x} & := & \prefix{x}{y}{(\binpar{\outputp{x}{y}}{@{y}})} \nonumber\\
	\bangp_{x}{P} & := & \binpar{{x}!\langle{\binpar{D_{x}}{P}}\rangle}{D_{x}} \nonumber
\end{eqnarray}

\begin{eqnarray}
	\bangp_{x}{P} & & \nonumber\\
	=
	& {x}!\langle{(\prefix{x}{y}{(\outputp{x}{y} | @{y})) | P}}\rangle 
	      | \prefix{x}{y}{(\outputp{x}{y} | @{y})} & \nonumber\\
	\red
	& (\outputp{x}{y} | @{y})\substn{\quotep{(\prefix{x}{y}{(@{y} | \outputp{x}{y})) | P}}}{y} & \nonumber\\
	=
	& \outputp{x}{\quotep{(\prefix{x}{y}{(\outputp{x}{y} | @{y})) | P}}}
	  | {(\prefix{x}{y}{(\outputp{x}{y} | @{y})) | P}} & \nonumber\\
	\red
	& \ldots & \nonumber\\
	\red^*
	& P | P | \ldots & \nonumber
\end{eqnarray}

Of course, this encoding, as an implementation, runs away, unfolding
$\bangp{P}$ eagerly. A lazier and more implementable replication
operator, restricted to input-guarded processes, may be obtained as follows.

\begin{eqnarray}
\bangp{\prefix{u}{v}{P}} 
	:= 
	\binpar{\lift{x}{\prefix{u}{v}{(\binpar{D(x)}{P})}}}{D(x)} \nonumber
\end{eqnarray}

\begin{remark}
  Note that the lazier definition still does not deal with summation
  or mixed summation (i.e. sums over input and output). The reader is
  invited to construct definitions of replication that deal with these
  features. 

  Further, the definitions are parameterized in a name, $x$. Can you,
  gentle reader, make a definition that eliminates this parameter and
  guarantees no accidental interaction between the replication
  machinery and the process being replicated -- i.e. no accidental
  sharing of names used by the process to get its work done and the
  name(s) used by the replication to effect copying. This latter
  revision of the definition of replication is crucial to obtaining
  the expected identity $!!P \sim !P$.
\end{remark}

\begin{remark}\label{rem:paradoxical_combinator}
  The reader familiar with the lambda calculus will have noticed the
  similarity between $D$ and the paradoxical combinator.

  [Ed. note: the existence of this seems to suggest we have to be more
  restrictive on the set of processes and names we admit if we are to
  support no-cloning.]
\end{remark}

\subsubsection{Bisimulation}

The computational dynamics gives rise to another kind of equivalence,
the equivalence of computational behavior. As previously mentioned
this is typically captured \emph{via} some form of bisimulation.

% The notion we use in this paper is weak barbed bisimulation
% \cite{milner91polyadicpi}.

The notion we use in this paper is derived from weak barbed
bisimulation \cite{milner91polyadicpi}. 

\begin{definition}
An \emph{observation relation}, $\downarrow_{\mathcal N}$, over a set
of names, $\mathcal N$, is the smallest relation satisfying the rules
below.

\infrule[Out-barb]{y \in {\mathcal N}, \; x \nameeq y}
		  {\outputp{x}{v} \downarrow_{\mathcal N} x}
\infrule[Par-barb]{\mbox{$P\downarrow_{\mathcal N} x$ or $Q\downarrow_{\mathcal N} x$}}
		  {\binpar{P}{Q} \downarrow_{\mathcal N} x}

We write $P \Downarrow_{\mathcal N} x$ if there is $Q$ such that 
$P \wred Q$ and $Q \downarrow_{\mathcal N} x$.
\end{definition}

\begin{definition}
%\label{def.bbisim}
An  ${\mathcal N}$-\emph{barbed bisimulation} over a set of names, ${\mathcal N}$, is a symmetric binary relation 
${\mathcal S}_{\mathcal N}$ between agents such that $P\rel{S}_{\mathcal N}Q$ implies:
\begin{enumerate}
\item If $P \red P'$ then $Q \wred Q'$ and $P'\rel{S}_{\mathcal N} Q'$.
\item If $P\downarrow_{\mathcal N} x$, then $Q\Downarrow_{\mathcal N} x$.
\end{enumerate}
$P$ is ${\mathcal N}$-barbed bisimilar to $Q$, written
$P \wbbisim_{\mathcal N} Q$, if $P \rel{S}_{\mathcal N} Q$ for some ${\mathcal N}$-barbed bisimulation ${\mathcal S}_{\mathcal N}$.
\end{definition}

$\mathcal{R} \subseteq \pi \times \pi$

$P \mathcal{R} Q => \forall P'. P \red P' \Rightarrow \exists Q'. Q \red Q', P' \mathcal{R} Q'$

$P \vdash x \Rightarrow Q \vdash x$

\begin{mathpar}
  \inferrule*[lab=Out-barb]{x \nameeq y}{{y}!\langle{Q}\rangle \vdash x}
  \and
  \inferrule*[lab=Par-barb]{\mbox{$P\vdash x$ or $Q\vdash x$}}{\binpar{P}{Q} \vdash x}
\end{mathpar}

\subsubsection{Contexts}

One of the principle advantages of computational calculi like the
$\pi$-calculus is a well-defined notion of context,
contextual-equivalence and a correlation between
contextual-equivalence and notions of bisimulation. The notion of
context allows the decomposition of a process into (sub-)process and
its syntactic environment, its context. Thus, a context may be
thought of as a process with a ``hole'' (written $\Box$) in it. The
application of a context $M$ to a process $P$, written $M[P]$, is
tantamount to filling the hole in $M$ with $P$. In this paper we do
not need the full weight of this theory, but do make use of the notion
of context in the proof the main theorem. 

\begin{mathpar}
  \inferrule* [lab=summation] {} {{M_{M},M_{N}} \bc \Box \;|\; x.M_{A} \;|\; M_{M}+M_{N}}
  \and
  \inferrule* [lab=agent] {} {{M_{A}} \bc (\vec{x})M_{P} \;| \; \clift{P_0,\ldots,M_{P},\ldots,P_N}}
  \and \\
  \inferrule* [lab=process] {} {{M_{P}} \bc M_{N} \;| \;P|M_{P} }
\end{mathpar} 

\begin{mathpar}
  \inferrule* [lab=sychronization] {} {M_{N} \bc \Box \;|\; x?M_{F} \;|\; x!M_{C}}
  \and
  \inferrule* [lab=abstraction] {} {{M_{F}} \bc (x)M_{P} }
  \and
  \inferrule* [lab=concretion] {} {{M_{C}} \bc \langle M_{P} \rangle }
  \and \\
  \inferrule* [lab=process] {} {{M_{P}} \bc M_{N} \;| \;P|M_{P} }
\end{mathpar}

\begin{definition}[contextual application] Given a context $M$, and
  process $P$, we define the \emph{contextual application}, $M[P] :=
  M\{P/\Box\}$. That is, the contextual application of M to P is the
  substitution of $P$ for $\Box$ in $M$.
\end{definition}

$\meaningof{-} : L \to \mathcal{P}(\pi)$

\begin{mathpar}
  \inferrule* [lab=collection] {} {\meaningof{true} = \pi, \and \meaningof{~E} = \pi \setminus \meaningof{E}, \and \meaningof{E_{1} \& E_{2}} = \meaningof{E_{1}} \cap \meaningof{E_{2}}}
\end{mathpar}

\begin{mathpar}
  \inferrule* [lab=structure] {} {\meaningof{0} = \{ P \in \pi | P \equiv 0 \}, \and \\ \meaningof{E_1 | E_2} = \{ P \in \pi | P \equiv P_{1} | P_{2}, P_{1} \in \meaningof{E_{1}}, P_{2} \in \meaningof{E_2}\} }
\end{mathpar}

\begin{mathpar}
 \inferrule* [lab=behavior] {} {\meaningof{\langle a?b \rangle E} = \{ P \in \pi | P \equiv Q | u?(y)P', \\ \and \\\\ \and \\ \;\;\; u \in \meaningof{a}, \forall z.P'\{z/y\} \in \meaningof{E\{z/b\}}\}, \and \\ \meaningof{a!E} = \{ P \in \pi | P \equiv Q | x!\langle P' \rangle, x \in \meaningof{a} P' \in \meaningof{E}\} }
\end{mathpar}

\begin{mathpar}
 \inferrule* [lab=nominal] {} {\meaningof{\quotep{E}} = \{ \quotep{P} \in \quotep{\pi} | P \in \meaningof{E} \}, \and \meaningof{\quotep{P}} = \{ \quotep{Q} \in \quotep{\pi} | P \equiv Q \} \and \\ \meaningof{@\quotep{E}} = \{ P \in \pi | P \equiv @x, x \in \meaningof{E} \}}
\end{mathpar}

\begin{eqnarray*}
  \\
  \meaningof{-} : TS \to ST
\end{eqnarray*}

\begin{eqnarray*}
  \\
  L : TS \to ST
\end{eqnarray*}

\begin{eqnarray*}
  \\
  P \models E \iff P \in \meaningof{E}
\end{eqnarray*}

\begin{eqnarray*}
  P \approx_{L} Q \iff \forall E \in L. P \models E \iff Q \models E
\end{eqnarray*}

\begin{eqnarray*}
  P \approx_{K} Q
\end{eqnarray*}

\begin{eqnarray*}
  P \approx Q
\end{eqnarray*}

$\approx_{K} = \approx = \approx_{L}$

\subsubsection{Contextual duality}

Note that contexts extend the quotation operation to a family of
operations from processes to names. Given a context, $M$, we can
define a \emph{nominal context}, $\quotep{M}$ by $\quotep{M}[P] :=
\quotep{M[P]}$. To foreshadow what is to come we observe that these
operations enjoy a duality with processes very much like the duality
between vectors and maps from vectors to scalars.

Further, because the calculus is essentially higher-order, we have a
correspondence between contexts and processes. More specifically,
given a name $x$ and a context $M$ we can construct $M^{*}_{x}$ such
that 

\begin{mathpar}
  M^{*}_{x} | \lift{x}{P} \red M[P]
\end{mathpar}

namely,

\begin{mathpar}
  M^{*}_{x} := x?(u).M[\dropn{u}]
\end{mathpar}

The dependence of $M^{*}_{x}$ on a name makes it an abstraction, 

\begin{mathpar}
  M^{*} := (x)x?(u).M[\dropn{u}]
\end{mathpar}

\subsection{Additional notation}

It will sometimes be convenient to denote the process a name
quotes. We already have the notation $x = \quotep{P}$, but it will be
convenient to introduce an alternate notation, $\procn{x}$, when we
want to emphasize the connection to the use of the name. Note that, by
virtue of name equivalence, $\quotep{\procn{x}} \nameeq x$; so, the
notation is consistent with previous definitions.

Further, because names have structure it is possible to effect
substitutions on the basis of that structure. This means we need to
upgrade our notation for substitutions, which we accomplish by
adapting comprehension notation. Thus,

\begin{mathpar}
  P\{ y / x : x \in S \}
\end{mathpar}

is interpreted to mean the process derived from P by replacing (in a
capture-avoiding manner) each occurrence of $x$ in $S$ by $y$. For example,

\begin{mathpar}
  P\{ \quotep{\procn{x}|\procn{x}} / x : x \in \freenames{P} \}
\end{mathpar}

will replace each (occurrence) of a free name $x$ in $P$ by
$\quotep{\procn{x}|\procn{x}}$.

Also, we will avail ourselves of the notation $x^{L}$ and $x^{R}$ to
denote injections of a name into disjoint copies of the name
space. There are numerous ways to accomplish this. One example can be
found in \cite{MeredithR05}. This notation overloads to vectors of
names: $\vec{x}^{\pi} := (x_{i}^{\pi} \; : \; 0 \leq i < |\vec{x}| )$ where $\pi \in \{L,R\}$.

We also use $P^{\Box} := P|\Box$.

In \cite{MeredithR05} an interpretation of the new operator is
given. It turns out that there are several possible interpretations
all enjoying the requisite algebraic properties of the operator (see
\cite{milner91polyadicpi}). We will therefore make liberal use of
$(\nu\; \vec{x})P$.

% subsection the_syntax_and_semantics_of_the_notation_system (end)   

\input{qm2pi.qmops} 

\input{qm2pi.sterngerlach} 

\input{qm2pi.metric} 

% section concurrent_process_calculi (end)

%\input{qm2pi.proofsketch}

% section proof sketch (end)

%\input{qm2pi.slviaknots} 

% section spatial logic via knots (end)

\input{qm2pi.conclusion}

% section conclusion (end)

%\input{qm2pi.dtcodes} 

% section wiring algorithm (end)

\input{qm2pi.ack} 

% section acknowledgments (end)

\newpage


\bibliographystyle{plain}   
\bibliography{../../biblios/main.bib}

\input{qm2pi.rhodetails}

\end{document}

 

% section notation (end)

\input{qm2pi.process.calculi} 

% section concurrent_process_calculi_and_spatial_logics_ (end)
    
%\documentclass[12pt]{llncs}
%\documentclass{jktr}

\usepackage[pdftex]{hyperref}                   
\usepackage {listings}
\usepackage {mathpartir}
\usepackage{bcprules}
%\usepackage{listings}
                       
\usepackage{graphicx} 
%\usepackage[margins=2.5cm,nohead,nofoot]{geometry}
%\usepackage{geometry}
\usepackage{amsfonts}
\usepackage{amstext}
\usepackage{latexsym}
\usepackage{amssymb}
\usepackage{color}


%\include{myPreamble}
\include{qm2pi.local} 

%\ifpdf
%\usepackage[pdftex]{graphicx}
%\else
%\usepackage{graphicx}
%\fi

 % \ifpdf
%  \usepackage{pdfsync}
%  \if


%\title{Brief Article}
%\author{David F. Snyder}
%\author{L.G. Meredith}

%\address{Dept. of Math., Texas State University--San Marcos, San Marcos, TX 78666}
       
\pagestyle{empty}


\begin{document}

\lstset{language=[Objective]Caml,frame=shadowbox}

\input{qm2pi.front}

% section front matter (end)

\input{qm2pi.intro} 
 
% section introduction (end)

% \input{qm2pi.knotations} 

% section notation (end)

\input{qm2pi.process.calculi} 

% section concurrent_process_calculi_and_spatial_logics_ (end)
    
%\input{qm2pi.knots2pi} 

%\input{qm2pi.trefoil} 

%\input{qm2pi.mainthm} 

% subsection basic_interpretation (end)

%\input{qm2pi.rho.presentation} 
\subsection{The syntax and semantics of the notation system}\label{sub:the_syntax_and_semantics_of_the_notation_system} % (fold)

We now summarize a technical presentation of the calculus that
embodies our theory of dynamics. The typical presentation of such a
calculus follows the style of giving generators and relations on
them. The grammar, below, describing term constructors, freely
generates the set of processes, $\Proc$. This set is then quotiented
by a relation known as structural congruence and it is over this set
that the notion of dynamics is expressed. This presentation is
essentially that of \cite{MeredithR05} with the addition of
polyadicity and summation. For readability we have relegated some of
the technical subtleties to an appendix.

\subsubsection{Process grammar}\label{subsub:process_grammar}

\begin{mathpar}
  \inferrule* [lab=synchronization] {} {{M} \bc \pzero \;|\; x?F \;|\; x!C }
  \and
  \inferrule* [lab=abstraction] {} {{F} \bc (x)P}
  \and
  \inferrule* [lab=concretion] {} {{C} \bc \langle Q \rangle}
  \and
  \inferrule* [lab=process] {} {{P,Q} \bc M \;| \;P|Q \;|\; @{x}}
  \and
  \inferrule* [lab=name] {} {{x} \bc \quotep{P}}
\end{mathpar} 

Note that $\vec{x}$ (resp. $\vec{P}$) denotes a vector of names
(resp. processes) of length $|\vec{x}|$ (resp. $|\vec{P}|$). We adopt
the following useful abbreviations.

\begin{mathpar}
   x?(\vec{y}).P := x.(\vec{y})P \and  x\clift{\vec{P}} := x.\clift{\vec{P}}
   \and x!(y) := \lift{x}{\dropn{y}}
   \and \Pi_{i=0}^{n-1}P_i := P_0 | \ldots | P_{n-1}
\end{mathpar}

\subsubsection{Structural congruence}

\paragraph{Free and bound names and alpha-equivalence.} At the
core of structural equivalence is alpha-equivalence which identifies
process that are the same up to a change of variable. Formally, we
recognize the distinction between free and bound names. The free names
of a process, $\freenames{P}$, may be calculated recursively as
follows:

\begin{mathpar}
\freenames{\pzero} := \emptyset
  \and \\
  \freenames{x?(y).P} := \{ x \} \cup (\freenames{P} \setminus \{ y \})
  \and 
  \freenames{x!\langle P \rangle} := \{ x \} \cup \{ P \} 
  \and \\
  \freenames{P|Q} := \freenames{P} \cup \freenames{Q}
  \and \\
  \freenames{@{x}} := \{ x \}
\end{mathpar}

$\pi$
$\quotep{\pi}$

$\freenames{-} : \pi \to \mathcal{P}(\quotep{\pi})$

\begin{eqnarray*}
  \freenames{\pzero} & := & \emptyset \\
  \freenames{x?(y).P} & := & \{ x \} \cup (\freenames{P} \setminus \{ y \}) \\
  \freenames{x!\langle P \rangle} & := & \{ x \} \cup \{ P \} \\
  \freenames{P|Q} & := & \freenames{P} \cup \freenames{Q} \\
  \freenames{\dropn{x}} & := & \{ x \}
\end{eqnarray*}

The bound names of a process, $\boundnames{P}$, are those names occurring in $P$
that are not free. For example, in $x?(y).0$, the name $x$ is free, while $y$ is bound.

\begin{mathpar}
  \inferrule* [lab=monoidal-laws] {} { P|Q \equiv Q|P \and P|0 \equiv P \and P|(Q|R) \equiv (P|Q)|R }
\end{mathpar}

\begin{mathpar}
  \inferrule* [lab=alpha-equivalence] {} { (x)P \equiv (y)P\{y/x\} \and y \not\in \freenames{P} }
\end{mathpar}

\begin{definition}
Then two processes, $P,Q$, are alpha-equivalent if $P = Q\{\vec{y}/\vec{x}\}$ for
some $\vec{x} \in \boundnames{Q},\vec{y} \in \boundnames{P}$, where $Q\{\vec{y}/\vec{x}\}$
denotes the capture-avoiding substitution of $\vec{y}$ for $\vec{x}$ in $Q$.
\end{definition}

\begin{definition}
  The {\em structural congruence} \cite{SangiorgiWalker} , $\equiv$,
  between processes is the least congruence containing
  alpha-equivalence, satisfying the abelian monoid laws
  (associativity, commutativity and $\pzero$ as identity) for parallel
  composition $|$ and for summation $+$.
\end{definition}

\subsection{Name equivalence}

We take name equivalence, written $\nameeq$, to be the smallest
equivalence relation generated by the following rules.

\begin{mathpar}
\inferrule*[lab=Quote-drop]
{ }
{ \quotep{@{x}} \nameeq x }

\inferrule*[lab=Struct-equiv]
{ P \scong Q }
{ \quotep{P} \nameeq \quotep{Q} }
\end{mathpar}

The astute reader will have noticed that the mutual recursion of names
and processes imposes a mutual recursion on alpha-equivalence and
structural equivalence via name-equivalence. Fortunately, all of this
works out pleasantly and we may calculate in the natural way, free of
concern. The reader interested in the details is referred to the
appendix \ref{appendix:rho_details}.

\subsection{Substitution}

We use $\Proc$ for the set of processes, $\QProc$ for the set of
names, and $\id{\{}\vec{y} / \vec{x} \id{\}}$ to denote partial maps,
$s : \QProc \rightarrow \QProc$. A map, $s$ lifts, uniquely, to a map
on process terms, $\widehat{s} : \Proc \rightarrow \Proc$ by the
following equations.

\begin{mathpar}
  (0) \psubstp{Q}{P} := 0 \\
  (R \juxtap S) \psubstp{Q}{P}
  :=    
  (R)\psubstp{Q}{P} \juxtap (S) \psubstp{Q}{P} \\
  (x?(y).R) \psubstp{Q}{P}    
  :=    
  (x)\substp{Q}{P} (z)\concat( (R \psubstn{z}{y}) \psubstp{Q}{P} ) \\
  (\lift{x}{R}) \psubstp{Q}{P}  
  :=
  \lift{(x)\substp{Q}{P}}{ R \psubstp{Q}{P} } \\
%   (\dropn{x})  \psubstp{Q}{P}       
%   := 
%   \left\{ 
%     \begin{array}{ccc} 
%       \dropn{\quotep{Q}} & & x \nameeq \quotep{P} \\
%       \dropn{x} & & otherwise \\
%     \end{array}
%   \right. 
  (\dropn{x})  \psubstp{Q}{P}       
  := 
  \left\{ 
    \begin{array}{ccc} 
      Q & & x \nameeq \quotep{P} \\
      \dropn{x} & & otherwise \\
    \end{array}
  \right.
\end{mathpar}
 

where

\begin{eqnarray}
  (x)\id{\{} \lpquote Q \rpquote / \lpquote P \rpquote \id{\}}            = 
  \left\{ 
    \begin{array}{ccc}
      \lpquote Q \rpquote & & x \nameeq \lpquote P \rpquote \\
      x & & otherwise \\
    \end{array}
  \right. \nonumber
\end{eqnarray}

and $z$ is chosen distinct from $\quotep{P}$, $\quotep{Q}$, the free
names in $Q$, and all the names in $R$. Our $\alpha$-equivalence will
be built in the standard way from this substitution.

\begin{remark}\label{rem:no_self_referential_names}
  One consequence of these definitions is that $\forall P. \quotep{P}
  \not\in \freenames{P}$.
\end{remark}

\subsection{ Dynamic quote: an example }

Anticipating something of what's to come, consider applying the
substitution, $\widehat{\id{\{}u / z \id{\}}}$, to the following pair
of processes, $\lift{w}{y!(z)}$ and $w[ \lpquote y!(z) \rpquote ]$.

\begin{eqnarray}
	\lift{w}{y!(z)}\widehat{\id{\{}u / z \id{\}}}
		& = &
		\lift{w}{y!(u)} \nonumber\\
	w[ \lpquote y!(z) \rpquote ] \widehat{ \id{\{}u / z \id{\}} }
		& = &
		w[ \lpquote y!(z) \rpquote ] \nonumber
\end{eqnarray}

Because the body of the process between quotes is impervious to
substitution, we get radically different answers. In fact, by
examining the first process in an input context,
e.g. $x?(z).\lift{w}{y!(z)}$, we see that the process under the lift
operator may be shaped by prefixed inputs binding a name inside it. In
this sense, the lift operator will be seen as a way to dynamically
construct processes before reifying them as names.

Finally equipped with these standard features we can present the
dynamics of the calculus.

\subsubsection{Operational semantics} 

Finally, we introduce the computational dynamics. What marks these
algebras as distinct from other more traditionally studied algebraic
structures, e.g. vector spaces or polynomial rings, is the manner in
which dynamics is captured. In traditional structures, dynamics is typically
expressed through morphisms between such structures, as in linear maps
between vector spaces or morphisms between rings. In algebras
associated with the semantics of computation, the dynamics is
expressed as part of the algebraic structure itself, through a
reduction reduction relation typically denoted by $\red$. Below, we
give a recursive presentation of this relation for the calculus used
in the encoding.

$\red \subseteq \pi \times \pi$
$\red : \pi \to \mathcal{P}(\pi)$

\begin{mathpar}
  \inferrule* [lab=Comm] { \textsf{match}( x_{src}, x_{trgt} ) } { x_{trgt}?(y)P \; | \; x_{src}!\langle {Q} \rangle \red P\{\quotep{Q}/y}\} }
  \and \\
  \inferrule* [lab=Par] {{P} \red {P}'} {{{P} | {Q}} \red {{P}' | {Q}}}
  \and
  \inferrule* [lab=Equiv]{{{P} \scong {P}'} \andalso {{P}' \red {Q}'} \andalso {{Q}' \scong {Q}}}{{P} \red {Q}}
\end{mathpar}

\begin{eqnarray*}
  match_{\equiv} (\quotep{P},\quotep{Q}) & := & P \equiv Q \\
  match_{\dagger}(\quotep{P},\quotep{Q}) & := & \forall R. P|Q \red^{*} R => R \red^{*} 0 \\
  match_{K}(\quotep{P},\quotep{Q}) & := & K \mbox{ for some context } K
\end{eqnarray*}

$u?(x)P | u!\langle Q \rangle \red P\{\quotep{Q}/x\}$

%We write $\wred$ for $\red^*$, and $P\red$ if $\exists Q $ such that $ P \red Q$.
We write $P\red$ if $\exists Q $ such that $ P \red Q$ and $P\not\red$, otherwise.

\section{Replication}

As mentioned before, it is known that replication (and hence
recursion) can be implemented in a higher-order process algebra
\cite{SangiorgiWalker}. As our first example of calculation with the
machinery thus far presented we give the construction explicitly in
the {\rhoc}.

\begin{eqnarray}
	D_{x} & := & \prefix{x}{y}{(\binpar{\outputp{x}{y}}{@{y}})} \nonumber\\
	\bangp_{x}{P} & := & \binpar{{x}!\langle{\binpar{D_{x}}{P}}\rangle}{D_{x}} \nonumber
\end{eqnarray}

\begin{eqnarray}
	\bangp_{x}{P} & & \nonumber\\
	=
	& {x}!\langle{(\prefix{x}{y}{(\outputp{x}{y} | @{y})) | P}}\rangle 
	      | \prefix{x}{y}{(\outputp{x}{y} | @{y})} & \nonumber\\
	\red
	& (\outputp{x}{y} | @{y})\substn{\quotep{(\prefix{x}{y}{(@{y} | \outputp{x}{y})) | P}}}{y} & \nonumber\\
	=
	& \outputp{x}{\quotep{(\prefix{x}{y}{(\outputp{x}{y} | @{y})) | P}}}
	  | {(\prefix{x}{y}{(\outputp{x}{y} | @{y})) | P}} & \nonumber\\
	\red
	& \ldots & \nonumber\\
	\red^*
	& P | P | \ldots & \nonumber
\end{eqnarray}

Of course, this encoding, as an implementation, runs away, unfolding
$\bangp{P}$ eagerly. A lazier and more implementable replication
operator, restricted to input-guarded processes, may be obtained as follows.

\begin{eqnarray}
\bangp{\prefix{u}{v}{P}} 
	:= 
	\binpar{\lift{x}{\prefix{u}{v}{(\binpar{D(x)}{P})}}}{D(x)} \nonumber
\end{eqnarray}

\begin{remark}
  Note that the lazier definition still does not deal with summation
  or mixed summation (i.e. sums over input and output). The reader is
  invited to construct definitions of replication that deal with these
  features. 

  Further, the definitions are parameterized in a name, $x$. Can you,
  gentle reader, make a definition that eliminates this parameter and
  guarantees no accidental interaction between the replication
  machinery and the process being replicated -- i.e. no accidental
  sharing of names used by the process to get its work done and the
  name(s) used by the replication to effect copying. This latter
  revision of the definition of replication is crucial to obtaining
  the expected identity $!!P \sim !P$.
\end{remark}

\begin{remark}\label{rem:paradoxical_combinator}
  The reader familiar with the lambda calculus will have noticed the
  similarity between $D$ and the paradoxical combinator.

  [Ed. note: the existence of this seems to suggest we have to be more
  restrictive on the set of processes and names we admit if we are to
  support no-cloning.]
\end{remark}

\subsubsection{Bisimulation}

The computational dynamics gives rise to another kind of equivalence,
the equivalence of computational behavior. As previously mentioned
this is typically captured \emph{via} some form of bisimulation.

% The notion we use in this paper is weak barbed bisimulation
% \cite{milner91polyadicpi}.

The notion we use in this paper is derived from weak barbed
bisimulation \cite{milner91polyadicpi}. 

\begin{definition}
An \emph{observation relation}, $\downarrow_{\mathcal N}$, over a set
of names, $\mathcal N$, is the smallest relation satisfying the rules
below.

\infrule[Out-barb]{y \in {\mathcal N}, \; x \nameeq y}
		  {\outputp{x}{v} \downarrow_{\mathcal N} x}
\infrule[Par-barb]{\mbox{$P\downarrow_{\mathcal N} x$ or $Q\downarrow_{\mathcal N} x$}}
		  {\binpar{P}{Q} \downarrow_{\mathcal N} x}

We write $P \Downarrow_{\mathcal N} x$ if there is $Q$ such that 
$P \wred Q$ and $Q \downarrow_{\mathcal N} x$.
\end{definition}

\begin{definition}
%\label{def.bbisim}
An  ${\mathcal N}$-\emph{barbed bisimulation} over a set of names, ${\mathcal N}$, is a symmetric binary relation 
${\mathcal S}_{\mathcal N}$ between agents such that $P\rel{S}_{\mathcal N}Q$ implies:
\begin{enumerate}
\item If $P \red P'$ then $Q \wred Q'$ and $P'\rel{S}_{\mathcal N} Q'$.
\item If $P\downarrow_{\mathcal N} x$, then $Q\Downarrow_{\mathcal N} x$.
\end{enumerate}
$P$ is ${\mathcal N}$-barbed bisimilar to $Q$, written
$P \wbbisim_{\mathcal N} Q$, if $P \rel{S}_{\mathcal N} Q$ for some ${\mathcal N}$-barbed bisimulation ${\mathcal S}_{\mathcal N}$.
\end{definition}

$\mathcal{R} \subseteq \pi \times \pi$

$P \mathcal{R} Q => \forall P'. P \red P' \Rightarrow \exists Q'. Q \red Q', P' \mathcal{R} Q'$

$P \vdash x \Rightarrow Q \vdash x$

\begin{mathpar}
  \inferrule*[lab=Out-barb]{x \nameeq y}{{y}!\langle{Q}\rangle \vdash x}
  \and
  \inferrule*[lab=Par-barb]{\mbox{$P\vdash x$ or $Q\vdash x$}}{\binpar{P}{Q} \vdash x}
\end{mathpar}

\subsubsection{Contexts}

One of the principle advantages of computational calculi like the
$\pi$-calculus is a well-defined notion of context,
contextual-equivalence and a correlation between
contextual-equivalence and notions of bisimulation. The notion of
context allows the decomposition of a process into (sub-)process and
its syntactic environment, its context. Thus, a context may be
thought of as a process with a ``hole'' (written $\Box$) in it. The
application of a context $M$ to a process $P$, written $M[P]$, is
tantamount to filling the hole in $M$ with $P$. In this paper we do
not need the full weight of this theory, but do make use of the notion
of context in the proof the main theorem. 

\begin{mathpar}
  \inferrule* [lab=summation] {} {{M_{M},M_{N}} \bc \Box \;|\; x.M_{A} \;|\; M_{M}+M_{N}}
  \and
  \inferrule* [lab=agent] {} {{M_{A}} \bc (\vec{x})M_{P} \;| \; \clift{P_0,\ldots,M_{P},\ldots,P_N}}
  \and \\
  \inferrule* [lab=process] {} {{M_{P}} \bc M_{N} \;| \;P|M_{P} }
\end{mathpar} 

\begin{mathpar}
  \inferrule* [lab=sychronization] {} {M_{N} \bc \Box \;|\; x?M_{F} \;|\; x!M_{C}}
  \and
  \inferrule* [lab=abstraction] {} {{M_{F}} \bc (x)M_{P} }
  \and
  \inferrule* [lab=concretion] {} {{M_{C}} \bc \langle M_{P} \rangle }
  \and \\
  \inferrule* [lab=process] {} {{M_{P}} \bc M_{N} \;| \;P|M_{P} }
\end{mathpar}

\begin{definition}[contextual application] Given a context $M$, and
  process $P$, we define the \emph{contextual application}, $M[P] :=
  M\{P/\Box\}$. That is, the contextual application of M to P is the
  substitution of $P$ for $\Box$ in $M$.
\end{definition}

$\meaningof{-} : L \to \mathcal{P}(\pi)$

\begin{mathpar}
  \inferrule* [lab=collection] {} {\meaningof{true} = \pi, \and \meaningof{~E} = \pi \setminus \meaningof{E}, \and \meaningof{E_{1} \& E_{2}} = \meaningof{E_{1}} \cap \meaningof{E_{2}}}
\end{mathpar}

\begin{mathpar}
  \inferrule* [lab=structure] {} {\meaningof{0} = \{ P \in \pi | P \equiv 0 \}, \and \\ \meaningof{E_1 | E_2} = \{ P \in \pi | P \equiv P_{1} | P_{2}, P_{1} \in \meaningof{E_{1}}, P_{2} \in \meaningof{E_2}\} }
\end{mathpar}

\begin{mathpar}
 \inferrule* [lab=behavior] {} {\meaningof{\langle a?b \rangle E} = \{ P \in \pi | P \equiv Q | u?(y)P', \\ \and \\\\ \and \\ \;\;\; u \in \meaningof{a}, \forall z.P'\{z/y\} \in \meaningof{E\{z/b\}}\}, \and \\ \meaningof{a!E} = \{ P \in \pi | P \equiv Q | x!\langle P' \rangle, x \in \meaningof{a} P' \in \meaningof{E}\} }
\end{mathpar}

\begin{mathpar}
 \inferrule* [lab=nominal] {} {\meaningof{\quotep{E}} = \{ \quotep{P} \in \quotep{\pi} | P \in \meaningof{E} \}, \and \meaningof{\quotep{P}} = \{ \quotep{Q} \in \quotep{\pi} | P \equiv Q \} \and \\ \meaningof{@\quotep{E}} = \{ P \in \pi | P \equiv @x, x \in \meaningof{E} \}}
\end{mathpar}

\begin{eqnarray*}
  \\
  \meaningof{-} : TS \to ST
\end{eqnarray*}

\begin{eqnarray*}
  \\
  L : TS \to ST
\end{eqnarray*}

\begin{eqnarray*}
  \\
  P \models E \iff P \in \meaningof{E}
\end{eqnarray*}

\begin{eqnarray*}
  P \approx_{L} Q \iff \forall E \in L. P \models E \iff Q \models E
\end{eqnarray*}

\begin{eqnarray*}
  P \approx_{K} Q
\end{eqnarray*}

\begin{eqnarray*}
  P \approx Q
\end{eqnarray*}

$\approx_{K} = \approx = \approx_{L}$

\subsubsection{Contextual duality}

Note that contexts extend the quotation operation to a family of
operations from processes to names. Given a context, $M$, we can
define a \emph{nominal context}, $\quotep{M}$ by $\quotep{M}[P] :=
\quotep{M[P]}$. To foreshadow what is to come we observe that these
operations enjoy a duality with processes very much like the duality
between vectors and maps from vectors to scalars.

Further, because the calculus is essentially higher-order, we have a
correspondence between contexts and processes. More specifically,
given a name $x$ and a context $M$ we can construct $M^{*}_{x}$ such
that 

\begin{mathpar}
  M^{*}_{x} | \lift{x}{P} \red M[P]
\end{mathpar}

namely,

\begin{mathpar}
  M^{*}_{x} := x?(u).M[\dropn{u}]
\end{mathpar}

The dependence of $M^{*}_{x}$ on a name makes it an abstraction, 

\begin{mathpar}
  M^{*} := (x)x?(u).M[\dropn{u}]
\end{mathpar}

\subsection{Additional notation}

It will sometimes be convenient to denote the process a name
quotes. We already have the notation $x = \quotep{P}$, but it will be
convenient to introduce an alternate notation, $\procn{x}$, when we
want to emphasize the connection to the use of the name. Note that, by
virtue of name equivalence, $\quotep{\procn{x}} \nameeq x$; so, the
notation is consistent with previous definitions.

Further, because names have structure it is possible to effect
substitutions on the basis of that structure. This means we need to
upgrade our notation for substitutions, which we accomplish by
adapting comprehension notation. Thus,

\begin{mathpar}
  P\{ y / x : x \in S \}
\end{mathpar}

is interpreted to mean the process derived from P by replacing (in a
capture-avoiding manner) each occurrence of $x$ in $S$ by $y$. For example,

\begin{mathpar}
  P\{ \quotep{\procn{x}|\procn{x}} / x : x \in \freenames{P} \}
\end{mathpar}

will replace each (occurrence) of a free name $x$ in $P$ by
$\quotep{\procn{x}|\procn{x}}$.

Also, we will avail ourselves of the notation $x^{L}$ and $x^{R}$ to
denote injections of a name into disjoint copies of the name
space. There are numerous ways to accomplish this. One example can be
found in \cite{MeredithR05}. This notation overloads to vectors of
names: $\vec{x}^{\pi} := (x_{i}^{\pi} \; : \; 0 \leq i < |\vec{x}| )$ where $\pi \in \{L,R\}$.

We also use $P^{\Box} := P|\Box$.

In \cite{MeredithR05} an interpretation of the new operator is
given. It turns out that there are several possible interpretations
all enjoying the requisite algebraic properties of the operator (see
\cite{milner91polyadicpi}). We will therefore make liberal use of
$(\nu\; \vec{x})P$.

% subsection the_syntax_and_semantics_of_the_notation_system (end)   

\input{qm2pi.qmops} 

\input{qm2pi.sterngerlach} 

\input{qm2pi.metric} 

% section concurrent_process_calculi (end)

%\input{qm2pi.proofsketch}

% section proof sketch (end)

%\input{qm2pi.slviaknots} 

% section spatial logic via knots (end)

\input{qm2pi.conclusion}

% section conclusion (end)

%\input{qm2pi.dtcodes} 

% section wiring algorithm (end)

\input{qm2pi.ack} 

% section acknowledgments (end)

\newpage


\bibliographystyle{plain}   
\bibliography{../../biblios/main.bib}

\input{qm2pi.rhodetails}

\end{document}

 

%\documentclass[12pt]{llncs}
%\documentclass{jktr}

\usepackage[pdftex]{hyperref}                   
\usepackage {listings}
\usepackage {mathpartir}
\usepackage{bcprules}
%\usepackage{listings}
                       
\usepackage{graphicx} 
%\usepackage[margins=2.5cm,nohead,nofoot]{geometry}
%\usepackage{geometry}
\usepackage{amsfonts}
\usepackage{amstext}
\usepackage{latexsym}
\usepackage{amssymb}
\usepackage{color}


%\include{myPreamble}
\include{qm2pi.local} 

%\ifpdf
%\usepackage[pdftex]{graphicx}
%\else
%\usepackage{graphicx}
%\fi

 % \ifpdf
%  \usepackage{pdfsync}
%  \if


%\title{Brief Article}
%\author{David F. Snyder}
%\author{L.G. Meredith}

%\address{Dept. of Math., Texas State University--San Marcos, San Marcos, TX 78666}
       
\pagestyle{empty}


\begin{document}

\lstset{language=[Objective]Caml,frame=shadowbox}

\input{qm2pi.front}

% section front matter (end)

\input{qm2pi.intro} 
 
% section introduction (end)

% \input{qm2pi.knotations} 

% section notation (end)

\input{qm2pi.process.calculi} 

% section concurrent_process_calculi_and_spatial_logics_ (end)
    
%\input{qm2pi.knots2pi} 

%\input{qm2pi.trefoil} 

%\input{qm2pi.mainthm} 

% subsection basic_interpretation (end)

%\input{qm2pi.rho.presentation} 
\subsection{The syntax and semantics of the notation system}\label{sub:the_syntax_and_semantics_of_the_notation_system} % (fold)

We now summarize a technical presentation of the calculus that
embodies our theory of dynamics. The typical presentation of such a
calculus follows the style of giving generators and relations on
them. The grammar, below, describing term constructors, freely
generates the set of processes, $\Proc$. This set is then quotiented
by a relation known as structural congruence and it is over this set
that the notion of dynamics is expressed. This presentation is
essentially that of \cite{MeredithR05} with the addition of
polyadicity and summation. For readability we have relegated some of
the technical subtleties to an appendix.

\subsubsection{Process grammar}\label{subsub:process_grammar}

\begin{mathpar}
  \inferrule* [lab=synchronization] {} {{M} \bc \pzero \;|\; x?F \;|\; x!C }
  \and
  \inferrule* [lab=abstraction] {} {{F} \bc (x)P}
  \and
  \inferrule* [lab=concretion] {} {{C} \bc \langle Q \rangle}
  \and
  \inferrule* [lab=process] {} {{P,Q} \bc M \;| \;P|Q \;|\; @{x}}
  \and
  \inferrule* [lab=name] {} {{x} \bc \quotep{P}}
\end{mathpar} 

Note that $\vec{x}$ (resp. $\vec{P}$) denotes a vector of names
(resp. processes) of length $|\vec{x}|$ (resp. $|\vec{P}|$). We adopt
the following useful abbreviations.

\begin{mathpar}
   x?(\vec{y}).P := x.(\vec{y})P \and  x\clift{\vec{P}} := x.\clift{\vec{P}}
   \and x!(y) := \lift{x}{\dropn{y}}
   \and \Pi_{i=0}^{n-1}P_i := P_0 | \ldots | P_{n-1}
\end{mathpar}

\subsubsection{Structural congruence}

\paragraph{Free and bound names and alpha-equivalence.} At the
core of structural equivalence is alpha-equivalence which identifies
process that are the same up to a change of variable. Formally, we
recognize the distinction between free and bound names. The free names
of a process, $\freenames{P}$, may be calculated recursively as
follows:

\begin{mathpar}
\freenames{\pzero} := \emptyset
  \and \\
  \freenames{x?(y).P} := \{ x \} \cup (\freenames{P} \setminus \{ y \})
  \and 
  \freenames{x!\langle P \rangle} := \{ x \} \cup \{ P \} 
  \and \\
  \freenames{P|Q} := \freenames{P} \cup \freenames{Q}
  \and \\
  \freenames{@{x}} := \{ x \}
\end{mathpar}

$\pi$
$\quotep{\pi}$

$\freenames{-} : \pi \to \mathcal{P}(\quotep{\pi})$

\begin{eqnarray*}
  \freenames{\pzero} & := & \emptyset \\
  \freenames{x?(y).P} & := & \{ x \} \cup (\freenames{P} \setminus \{ y \}) \\
  \freenames{x!\langle P \rangle} & := & \{ x \} \cup \{ P \} \\
  \freenames{P|Q} & := & \freenames{P} \cup \freenames{Q} \\
  \freenames{\dropn{x}} & := & \{ x \}
\end{eqnarray*}

The bound names of a process, $\boundnames{P}$, are those names occurring in $P$
that are not free. For example, in $x?(y).0$, the name $x$ is free, while $y$ is bound.

\begin{mathpar}
  \inferrule* [lab=monoidal-laws] {} { P|Q \equiv Q|P \and P|0 \equiv P \and P|(Q|R) \equiv (P|Q)|R }
\end{mathpar}

\begin{mathpar}
  \inferrule* [lab=alpha-equivalence] {} { (x)P \equiv (y)P\{y/x\} \and y \not\in \freenames{P} }
\end{mathpar}

\begin{definition}
Then two processes, $P,Q$, are alpha-equivalent if $P = Q\{\vec{y}/\vec{x}\}$ for
some $\vec{x} \in \boundnames{Q},\vec{y} \in \boundnames{P}$, where $Q\{\vec{y}/\vec{x}\}$
denotes the capture-avoiding substitution of $\vec{y}$ for $\vec{x}$ in $Q$.
\end{definition}

\begin{definition}
  The {\em structural congruence} \cite{SangiorgiWalker} , $\equiv$,
  between processes is the least congruence containing
  alpha-equivalence, satisfying the abelian monoid laws
  (associativity, commutativity and $\pzero$ as identity) for parallel
  composition $|$ and for summation $+$.
\end{definition}

\subsection{Name equivalence}

We take name equivalence, written $\nameeq$, to be the smallest
equivalence relation generated by the following rules.

\begin{mathpar}
\inferrule*[lab=Quote-drop]
{ }
{ \quotep{@{x}} \nameeq x }

\inferrule*[lab=Struct-equiv]
{ P \scong Q }
{ \quotep{P} \nameeq \quotep{Q} }
\end{mathpar}

The astute reader will have noticed that the mutual recursion of names
and processes imposes a mutual recursion on alpha-equivalence and
structural equivalence via name-equivalence. Fortunately, all of this
works out pleasantly and we may calculate in the natural way, free of
concern. The reader interested in the details is referred to the
appendix \ref{appendix:rho_details}.

\subsection{Substitution}

We use $\Proc$ for the set of processes, $\QProc$ for the set of
names, and $\id{\{}\vec{y} / \vec{x} \id{\}}$ to denote partial maps,
$s : \QProc \rightarrow \QProc$. A map, $s$ lifts, uniquely, to a map
on process terms, $\widehat{s} : \Proc \rightarrow \Proc$ by the
following equations.

\begin{mathpar}
  (0) \psubstp{Q}{P} := 0 \\
  (R \juxtap S) \psubstp{Q}{P}
  :=    
  (R)\psubstp{Q}{P} \juxtap (S) \psubstp{Q}{P} \\
  (x?(y).R) \psubstp{Q}{P}    
  :=    
  (x)\substp{Q}{P} (z)\concat( (R \psubstn{z}{y}) \psubstp{Q}{P} ) \\
  (\lift{x}{R}) \psubstp{Q}{P}  
  :=
  \lift{(x)\substp{Q}{P}}{ R \psubstp{Q}{P} } \\
%   (\dropn{x})  \psubstp{Q}{P}       
%   := 
%   \left\{ 
%     \begin{array}{ccc} 
%       \dropn{\quotep{Q}} & & x \nameeq \quotep{P} \\
%       \dropn{x} & & otherwise \\
%     \end{array}
%   \right. 
  (\dropn{x})  \psubstp{Q}{P}       
  := 
  \left\{ 
    \begin{array}{ccc} 
      Q & & x \nameeq \quotep{P} \\
      \dropn{x} & & otherwise \\
    \end{array}
  \right.
\end{mathpar}
 

where

\begin{eqnarray}
  (x)\id{\{} \lpquote Q \rpquote / \lpquote P \rpquote \id{\}}            = 
  \left\{ 
    \begin{array}{ccc}
      \lpquote Q \rpquote & & x \nameeq \lpquote P \rpquote \\
      x & & otherwise \\
    \end{array}
  \right. \nonumber
\end{eqnarray}

and $z$ is chosen distinct from $\quotep{P}$, $\quotep{Q}$, the free
names in $Q$, and all the names in $R$. Our $\alpha$-equivalence will
be built in the standard way from this substitution.

\begin{remark}\label{rem:no_self_referential_names}
  One consequence of these definitions is that $\forall P. \quotep{P}
  \not\in \freenames{P}$.
\end{remark}

\subsection{ Dynamic quote: an example }

Anticipating something of what's to come, consider applying the
substitution, $\widehat{\id{\{}u / z \id{\}}}$, to the following pair
of processes, $\lift{w}{y!(z)}$ and $w[ \lpquote y!(z) \rpquote ]$.

\begin{eqnarray}
	\lift{w}{y!(z)}\widehat{\id{\{}u / z \id{\}}}
		& = &
		\lift{w}{y!(u)} \nonumber\\
	w[ \lpquote y!(z) \rpquote ] \widehat{ \id{\{}u / z \id{\}} }
		& = &
		w[ \lpquote y!(z) \rpquote ] \nonumber
\end{eqnarray}

Because the body of the process between quotes is impervious to
substitution, we get radically different answers. In fact, by
examining the first process in an input context,
e.g. $x?(z).\lift{w}{y!(z)}$, we see that the process under the lift
operator may be shaped by prefixed inputs binding a name inside it. In
this sense, the lift operator will be seen as a way to dynamically
construct processes before reifying them as names.

Finally equipped with these standard features we can present the
dynamics of the calculus.

\subsubsection{Operational semantics} 

Finally, we introduce the computational dynamics. What marks these
algebras as distinct from other more traditionally studied algebraic
structures, e.g. vector spaces or polynomial rings, is the manner in
which dynamics is captured. In traditional structures, dynamics is typically
expressed through morphisms between such structures, as in linear maps
between vector spaces or morphisms between rings. In algebras
associated with the semantics of computation, the dynamics is
expressed as part of the algebraic structure itself, through a
reduction reduction relation typically denoted by $\red$. Below, we
give a recursive presentation of this relation for the calculus used
in the encoding.

$\red \subseteq \pi \times \pi$
$\red : \pi \to \mathcal{P}(\pi)$

\begin{mathpar}
  \inferrule* [lab=Comm] { \textsf{match}( x_{src}, x_{trgt} ) } { x_{trgt}?(y)P \; | \; x_{src}!\langle {Q} \rangle \red P\{\quotep{Q}/y}\} }
  \and \\
  \inferrule* [lab=Par] {{P} \red {P}'} {{{P} | {Q}} \red {{P}' | {Q}}}
  \and
  \inferrule* [lab=Equiv]{{{P} \scong {P}'} \andalso {{P}' \red {Q}'} \andalso {{Q}' \scong {Q}}}{{P} \red {Q}}
\end{mathpar}

\begin{eqnarray*}
  match_{\equiv} (\quotep{P},\quotep{Q}) & := & P \equiv Q \\
  match_{\dagger}(\quotep{P},\quotep{Q}) & := & \forall R. P|Q \red^{*} R => R \red^{*} 0 \\
  match_{K}(\quotep{P},\quotep{Q}) & := & K \mbox{ for some context } K
\end{eqnarray*}

$u?(x)P | u!\langle Q \rangle \red P\{\quotep{Q}/x\}$

%We write $\wred$ for $\red^*$, and $P\red$ if $\exists Q $ such that $ P \red Q$.
We write $P\red$ if $\exists Q $ such that $ P \red Q$ and $P\not\red$, otherwise.

\section{Replication}

As mentioned before, it is known that replication (and hence
recursion) can be implemented in a higher-order process algebra
\cite{SangiorgiWalker}. As our first example of calculation with the
machinery thus far presented we give the construction explicitly in
the {\rhoc}.

\begin{eqnarray}
	D_{x} & := & \prefix{x}{y}{(\binpar{\outputp{x}{y}}{@{y}})} \nonumber\\
	\bangp_{x}{P} & := & \binpar{{x}!\langle{\binpar{D_{x}}{P}}\rangle}{D_{x}} \nonumber
\end{eqnarray}

\begin{eqnarray}
	\bangp_{x}{P} & & \nonumber\\
	=
	& {x}!\langle{(\prefix{x}{y}{(\outputp{x}{y} | @{y})) | P}}\rangle 
	      | \prefix{x}{y}{(\outputp{x}{y} | @{y})} & \nonumber\\
	\red
	& (\outputp{x}{y} | @{y})\substn{\quotep{(\prefix{x}{y}{(@{y} | \outputp{x}{y})) | P}}}{y} & \nonumber\\
	=
	& \outputp{x}{\quotep{(\prefix{x}{y}{(\outputp{x}{y} | @{y})) | P}}}
	  | {(\prefix{x}{y}{(\outputp{x}{y} | @{y})) | P}} & \nonumber\\
	\red
	& \ldots & \nonumber\\
	\red^*
	& P | P | \ldots & \nonumber
\end{eqnarray}

Of course, this encoding, as an implementation, runs away, unfolding
$\bangp{P}$ eagerly. A lazier and more implementable replication
operator, restricted to input-guarded processes, may be obtained as follows.

\begin{eqnarray}
\bangp{\prefix{u}{v}{P}} 
	:= 
	\binpar{\lift{x}{\prefix{u}{v}{(\binpar{D(x)}{P})}}}{D(x)} \nonumber
\end{eqnarray}

\begin{remark}
  Note that the lazier definition still does not deal with summation
  or mixed summation (i.e. sums over input and output). The reader is
  invited to construct definitions of replication that deal with these
  features. 

  Further, the definitions are parameterized in a name, $x$. Can you,
  gentle reader, make a definition that eliminates this parameter and
  guarantees no accidental interaction between the replication
  machinery and the process being replicated -- i.e. no accidental
  sharing of names used by the process to get its work done and the
  name(s) used by the replication to effect copying. This latter
  revision of the definition of replication is crucial to obtaining
  the expected identity $!!P \sim !P$.
\end{remark}

\begin{remark}\label{rem:paradoxical_combinator}
  The reader familiar with the lambda calculus will have noticed the
  similarity between $D$ and the paradoxical combinator.

  [Ed. note: the existence of this seems to suggest we have to be more
  restrictive on the set of processes and names we admit if we are to
  support no-cloning.]
\end{remark}

\subsubsection{Bisimulation}

The computational dynamics gives rise to another kind of equivalence,
the equivalence of computational behavior. As previously mentioned
this is typically captured \emph{via} some form of bisimulation.

% The notion we use in this paper is weak barbed bisimulation
% \cite{milner91polyadicpi}.

The notion we use in this paper is derived from weak barbed
bisimulation \cite{milner91polyadicpi}. 

\begin{definition}
An \emph{observation relation}, $\downarrow_{\mathcal N}$, over a set
of names, $\mathcal N$, is the smallest relation satisfying the rules
below.

\infrule[Out-barb]{y \in {\mathcal N}, \; x \nameeq y}
		  {\outputp{x}{v} \downarrow_{\mathcal N} x}
\infrule[Par-barb]{\mbox{$P\downarrow_{\mathcal N} x$ or $Q\downarrow_{\mathcal N} x$}}
		  {\binpar{P}{Q} \downarrow_{\mathcal N} x}

We write $P \Downarrow_{\mathcal N} x$ if there is $Q$ such that 
$P \wred Q$ and $Q \downarrow_{\mathcal N} x$.
\end{definition}

\begin{definition}
%\label{def.bbisim}
An  ${\mathcal N}$-\emph{barbed bisimulation} over a set of names, ${\mathcal N}$, is a symmetric binary relation 
${\mathcal S}_{\mathcal N}$ between agents such that $P\rel{S}_{\mathcal N}Q$ implies:
\begin{enumerate}
\item If $P \red P'$ then $Q \wred Q'$ and $P'\rel{S}_{\mathcal N} Q'$.
\item If $P\downarrow_{\mathcal N} x$, then $Q\Downarrow_{\mathcal N} x$.
\end{enumerate}
$P$ is ${\mathcal N}$-barbed bisimilar to $Q$, written
$P \wbbisim_{\mathcal N} Q$, if $P \rel{S}_{\mathcal N} Q$ for some ${\mathcal N}$-barbed bisimulation ${\mathcal S}_{\mathcal N}$.
\end{definition}

$\mathcal{R} \subseteq \pi \times \pi$

$P \mathcal{R} Q => \forall P'. P \red P' \Rightarrow \exists Q'. Q \red Q', P' \mathcal{R} Q'$

$P \vdash x \Rightarrow Q \vdash x$

\begin{mathpar}
  \inferrule*[lab=Out-barb]{x \nameeq y}{{y}!\langle{Q}\rangle \vdash x}
  \and
  \inferrule*[lab=Par-barb]{\mbox{$P\vdash x$ or $Q\vdash x$}}{\binpar{P}{Q} \vdash x}
\end{mathpar}

\subsubsection{Contexts}

One of the principle advantages of computational calculi like the
$\pi$-calculus is a well-defined notion of context,
contextual-equivalence and a correlation between
contextual-equivalence and notions of bisimulation. The notion of
context allows the decomposition of a process into (sub-)process and
its syntactic environment, its context. Thus, a context may be
thought of as a process with a ``hole'' (written $\Box$) in it. The
application of a context $M$ to a process $P$, written $M[P]$, is
tantamount to filling the hole in $M$ with $P$. In this paper we do
not need the full weight of this theory, but do make use of the notion
of context in the proof the main theorem. 

\begin{mathpar}
  \inferrule* [lab=summation] {} {{M_{M},M_{N}} \bc \Box \;|\; x.M_{A} \;|\; M_{M}+M_{N}}
  \and
  \inferrule* [lab=agent] {} {{M_{A}} \bc (\vec{x})M_{P} \;| \; \clift{P_0,\ldots,M_{P},\ldots,P_N}}
  \and \\
  \inferrule* [lab=process] {} {{M_{P}} \bc M_{N} \;| \;P|M_{P} }
\end{mathpar} 

\begin{mathpar}
  \inferrule* [lab=sychronization] {} {M_{N} \bc \Box \;|\; x?M_{F} \;|\; x!M_{C}}
  \and
  \inferrule* [lab=abstraction] {} {{M_{F}} \bc (x)M_{P} }
  \and
  \inferrule* [lab=concretion] {} {{M_{C}} \bc \langle M_{P} \rangle }
  \and \\
  \inferrule* [lab=process] {} {{M_{P}} \bc M_{N} \;| \;P|M_{P} }
\end{mathpar}

\begin{definition}[contextual application] Given a context $M$, and
  process $P$, we define the \emph{contextual application}, $M[P] :=
  M\{P/\Box\}$. That is, the contextual application of M to P is the
  substitution of $P$ for $\Box$ in $M$.
\end{definition}

$\meaningof{-} : L \to \mathcal{P}(\pi)$

\begin{mathpar}
  \inferrule* [lab=collection] {} {\meaningof{true} = \pi, \and \meaningof{~E} = \pi \setminus \meaningof{E}, \and \meaningof{E_{1} \& E_{2}} = \meaningof{E_{1}} \cap \meaningof{E_{2}}}
\end{mathpar}

\begin{mathpar}
  \inferrule* [lab=structure] {} {\meaningof{0} = \{ P \in \pi | P \equiv 0 \}, \and \\ \meaningof{E_1 | E_2} = \{ P \in \pi | P \equiv P_{1} | P_{2}, P_{1} \in \meaningof{E_{1}}, P_{2} \in \meaningof{E_2}\} }
\end{mathpar}

\begin{mathpar}
 \inferrule* [lab=behavior] {} {\meaningof{\langle a?b \rangle E} = \{ P \in \pi | P \equiv Q | u?(y)P', \\ \and \\\\ \and \\ \;\;\; u \in \meaningof{a}, \forall z.P'\{z/y\} \in \meaningof{E\{z/b\}}\}, \and \\ \meaningof{a!E} = \{ P \in \pi | P \equiv Q | x!\langle P' \rangle, x \in \meaningof{a} P' \in \meaningof{E}\} }
\end{mathpar}

\begin{mathpar}
 \inferrule* [lab=nominal] {} {\meaningof{\quotep{E}} = \{ \quotep{P} \in \quotep{\pi} | P \in \meaningof{E} \}, \and \meaningof{\quotep{P}} = \{ \quotep{Q} \in \quotep{\pi} | P \equiv Q \} \and \\ \meaningof{@\quotep{E}} = \{ P \in \pi | P \equiv @x, x \in \meaningof{E} \}}
\end{mathpar}

\begin{eqnarray*}
  \\
  \meaningof{-} : TS \to ST
\end{eqnarray*}

\begin{eqnarray*}
  \\
  L : TS \to ST
\end{eqnarray*}

\begin{eqnarray*}
  \\
  P \models E \iff P \in \meaningof{E}
\end{eqnarray*}

\begin{eqnarray*}
  P \approx_{L} Q \iff \forall E \in L. P \models E \iff Q \models E
\end{eqnarray*}

\begin{eqnarray*}
  P \approx_{K} Q
\end{eqnarray*}

\begin{eqnarray*}
  P \approx Q
\end{eqnarray*}

$\approx_{K} = \approx = \approx_{L}$

\subsubsection{Contextual duality}

Note that contexts extend the quotation operation to a family of
operations from processes to names. Given a context, $M$, we can
define a \emph{nominal context}, $\quotep{M}$ by $\quotep{M}[P] :=
\quotep{M[P]}$. To foreshadow what is to come we observe that these
operations enjoy a duality with processes very much like the duality
between vectors and maps from vectors to scalars.

Further, because the calculus is essentially higher-order, we have a
correspondence between contexts and processes. More specifically,
given a name $x$ and a context $M$ we can construct $M^{*}_{x}$ such
that 

\begin{mathpar}
  M^{*}_{x} | \lift{x}{P} \red M[P]
\end{mathpar}

namely,

\begin{mathpar}
  M^{*}_{x} := x?(u).M[\dropn{u}]
\end{mathpar}

The dependence of $M^{*}_{x}$ on a name makes it an abstraction, 

\begin{mathpar}
  M^{*} := (x)x?(u).M[\dropn{u}]
\end{mathpar}

\subsection{Additional notation}

It will sometimes be convenient to denote the process a name
quotes. We already have the notation $x = \quotep{P}$, but it will be
convenient to introduce an alternate notation, $\procn{x}$, when we
want to emphasize the connection to the use of the name. Note that, by
virtue of name equivalence, $\quotep{\procn{x}} \nameeq x$; so, the
notation is consistent with previous definitions.

Further, because names have structure it is possible to effect
substitutions on the basis of that structure. This means we need to
upgrade our notation for substitutions, which we accomplish by
adapting comprehension notation. Thus,

\begin{mathpar}
  P\{ y / x : x \in S \}
\end{mathpar}

is interpreted to mean the process derived from P by replacing (in a
capture-avoiding manner) each occurrence of $x$ in $S$ by $y$. For example,

\begin{mathpar}
  P\{ \quotep{\procn{x}|\procn{x}} / x : x \in \freenames{P} \}
\end{mathpar}

will replace each (occurrence) of a free name $x$ in $P$ by
$\quotep{\procn{x}|\procn{x}}$.

Also, we will avail ourselves of the notation $x^{L}$ and $x^{R}$ to
denote injections of a name into disjoint copies of the name
space. There are numerous ways to accomplish this. One example can be
found in \cite{MeredithR05}. This notation overloads to vectors of
names: $\vec{x}^{\pi} := (x_{i}^{\pi} \; : \; 0 \leq i < |\vec{x}| )$ where $\pi \in \{L,R\}$.

We also use $P^{\Box} := P|\Box$.

In \cite{MeredithR05} an interpretation of the new operator is
given. It turns out that there are several possible interpretations
all enjoying the requisite algebraic properties of the operator (see
\cite{milner91polyadicpi}). We will therefore make liberal use of
$(\nu\; \vec{x})P$.

% subsection the_syntax_and_semantics_of_the_notation_system (end)   

\input{qm2pi.qmops} 

\input{qm2pi.sterngerlach} 

\input{qm2pi.metric} 

% section concurrent_process_calculi (end)

%\input{qm2pi.proofsketch}

% section proof sketch (end)

%\input{qm2pi.slviaknots} 

% section spatial logic via knots (end)

\input{qm2pi.conclusion}

% section conclusion (end)

%\input{qm2pi.dtcodes} 

% section wiring algorithm (end)

\input{qm2pi.ack} 

% section acknowledgments (end)

\newpage


\bibliographystyle{plain}   
\bibliography{../../biblios/main.bib}

\input{qm2pi.rhodetails}

\end{document}

 

%\documentclass[12pt]{llncs}
%\documentclass{jktr}

\usepackage[pdftex]{hyperref}                   
\usepackage {listings}
\usepackage {mathpartir}
\usepackage{bcprules}
%\usepackage{listings}
                       
\usepackage{graphicx} 
%\usepackage[margins=2.5cm,nohead,nofoot]{geometry}
%\usepackage{geometry}
\usepackage{amsfonts}
\usepackage{amstext}
\usepackage{latexsym}
\usepackage{amssymb}
\usepackage{color}


%\include{myPreamble}
\include{qm2pi.local} 

%\ifpdf
%\usepackage[pdftex]{graphicx}
%\else
%\usepackage{graphicx}
%\fi

 % \ifpdf
%  \usepackage{pdfsync}
%  \if


%\title{Brief Article}
%\author{David F. Snyder}
%\author{L.G. Meredith}

%\address{Dept. of Math., Texas State University--San Marcos, San Marcos, TX 78666}
       
\pagestyle{empty}


\begin{document}

\lstset{language=[Objective]Caml,frame=shadowbox}

\input{qm2pi.front}

% section front matter (end)

\input{qm2pi.intro} 
 
% section introduction (end)

% \input{qm2pi.knotations} 

% section notation (end)

\input{qm2pi.process.calculi} 

% section concurrent_process_calculi_and_spatial_logics_ (end)
    
%\input{qm2pi.knots2pi} 

%\input{qm2pi.trefoil} 

%\input{qm2pi.mainthm} 

% subsection basic_interpretation (end)

%\input{qm2pi.rho.presentation} 
\subsection{The syntax and semantics of the notation system}\label{sub:the_syntax_and_semantics_of_the_notation_system} % (fold)

We now summarize a technical presentation of the calculus that
embodies our theory of dynamics. The typical presentation of such a
calculus follows the style of giving generators and relations on
them. The grammar, below, describing term constructors, freely
generates the set of processes, $\Proc$. This set is then quotiented
by a relation known as structural congruence and it is over this set
that the notion of dynamics is expressed. This presentation is
essentially that of \cite{MeredithR05} with the addition of
polyadicity and summation. For readability we have relegated some of
the technical subtleties to an appendix.

\subsubsection{Process grammar}\label{subsub:process_grammar}

\begin{mathpar}
  \inferrule* [lab=synchronization] {} {{M} \bc \pzero \;|\; x?F \;|\; x!C }
  \and
  \inferrule* [lab=abstraction] {} {{F} \bc (x)P}
  \and
  \inferrule* [lab=concretion] {} {{C} \bc \langle Q \rangle}
  \and
  \inferrule* [lab=process] {} {{P,Q} \bc M \;| \;P|Q \;|\; @{x}}
  \and
  \inferrule* [lab=name] {} {{x} \bc \quotep{P}}
\end{mathpar} 

Note that $\vec{x}$ (resp. $\vec{P}$) denotes a vector of names
(resp. processes) of length $|\vec{x}|$ (resp. $|\vec{P}|$). We adopt
the following useful abbreviations.

\begin{mathpar}
   x?(\vec{y}).P := x.(\vec{y})P \and  x\clift{\vec{P}} := x.\clift{\vec{P}}
   \and x!(y) := \lift{x}{\dropn{y}}
   \and \Pi_{i=0}^{n-1}P_i := P_0 | \ldots | P_{n-1}
\end{mathpar}

\subsubsection{Structural congruence}

\paragraph{Free and bound names and alpha-equivalence.} At the
core of structural equivalence is alpha-equivalence which identifies
process that are the same up to a change of variable. Formally, we
recognize the distinction between free and bound names. The free names
of a process, $\freenames{P}$, may be calculated recursively as
follows:

\begin{mathpar}
\freenames{\pzero} := \emptyset
  \and \\
  \freenames{x?(y).P} := \{ x \} \cup (\freenames{P} \setminus \{ y \})
  \and 
  \freenames{x!\langle P \rangle} := \{ x \} \cup \{ P \} 
  \and \\
  \freenames{P|Q} := \freenames{P} \cup \freenames{Q}
  \and \\
  \freenames{@{x}} := \{ x \}
\end{mathpar}

$\pi$
$\quotep{\pi}$

$\freenames{-} : \pi \to \mathcal{P}(\quotep{\pi})$

\begin{eqnarray*}
  \freenames{\pzero} & := & \emptyset \\
  \freenames{x?(y).P} & := & \{ x \} \cup (\freenames{P} \setminus \{ y \}) \\
  \freenames{x!\langle P \rangle} & := & \{ x \} \cup \{ P \} \\
  \freenames{P|Q} & := & \freenames{P} \cup \freenames{Q} \\
  \freenames{\dropn{x}} & := & \{ x \}
\end{eqnarray*}

The bound names of a process, $\boundnames{P}$, are those names occurring in $P$
that are not free. For example, in $x?(y).0$, the name $x$ is free, while $y$ is bound.

\begin{mathpar}
  \inferrule* [lab=monoidal-laws] {} { P|Q \equiv Q|P \and P|0 \equiv P \and P|(Q|R) \equiv (P|Q)|R }
\end{mathpar}

\begin{mathpar}
  \inferrule* [lab=alpha-equivalence] {} { (x)P \equiv (y)P\{y/x\} \and y \not\in \freenames{P} }
\end{mathpar}

\begin{definition}
Then two processes, $P,Q$, are alpha-equivalent if $P = Q\{\vec{y}/\vec{x}\}$ for
some $\vec{x} \in \boundnames{Q},\vec{y} \in \boundnames{P}$, where $Q\{\vec{y}/\vec{x}\}$
denotes the capture-avoiding substitution of $\vec{y}$ for $\vec{x}$ in $Q$.
\end{definition}

\begin{definition}
  The {\em structural congruence} \cite{SangiorgiWalker} , $\equiv$,
  between processes is the least congruence containing
  alpha-equivalence, satisfying the abelian monoid laws
  (associativity, commutativity and $\pzero$ as identity) for parallel
  composition $|$ and for summation $+$.
\end{definition}

\subsection{Name equivalence}

We take name equivalence, written $\nameeq$, to be the smallest
equivalence relation generated by the following rules.

\begin{mathpar}
\inferrule*[lab=Quote-drop]
{ }
{ \quotep{@{x}} \nameeq x }

\inferrule*[lab=Struct-equiv]
{ P \scong Q }
{ \quotep{P} \nameeq \quotep{Q} }
\end{mathpar}

The astute reader will have noticed that the mutual recursion of names
and processes imposes a mutual recursion on alpha-equivalence and
structural equivalence via name-equivalence. Fortunately, all of this
works out pleasantly and we may calculate in the natural way, free of
concern. The reader interested in the details is referred to the
appendix \ref{appendix:rho_details}.

\subsection{Substitution}

We use $\Proc$ for the set of processes, $\QProc$ for the set of
names, and $\id{\{}\vec{y} / \vec{x} \id{\}}$ to denote partial maps,
$s : \QProc \rightarrow \QProc$. A map, $s$ lifts, uniquely, to a map
on process terms, $\widehat{s} : \Proc \rightarrow \Proc$ by the
following equations.

\begin{mathpar}
  (0) \psubstp{Q}{P} := 0 \\
  (R \juxtap S) \psubstp{Q}{P}
  :=    
  (R)\psubstp{Q}{P} \juxtap (S) \psubstp{Q}{P} \\
  (x?(y).R) \psubstp{Q}{P}    
  :=    
  (x)\substp{Q}{P} (z)\concat( (R \psubstn{z}{y}) \psubstp{Q}{P} ) \\
  (\lift{x}{R}) \psubstp{Q}{P}  
  :=
  \lift{(x)\substp{Q}{P}}{ R \psubstp{Q}{P} } \\
%   (\dropn{x})  \psubstp{Q}{P}       
%   := 
%   \left\{ 
%     \begin{array}{ccc} 
%       \dropn{\quotep{Q}} & & x \nameeq \quotep{P} \\
%       \dropn{x} & & otherwise \\
%     \end{array}
%   \right. 
  (\dropn{x})  \psubstp{Q}{P}       
  := 
  \left\{ 
    \begin{array}{ccc} 
      Q & & x \nameeq \quotep{P} \\
      \dropn{x} & & otherwise \\
    \end{array}
  \right.
\end{mathpar}
 

where

\begin{eqnarray}
  (x)\id{\{} \lpquote Q \rpquote / \lpquote P \rpquote \id{\}}            = 
  \left\{ 
    \begin{array}{ccc}
      \lpquote Q \rpquote & & x \nameeq \lpquote P \rpquote \\
      x & & otherwise \\
    \end{array}
  \right. \nonumber
\end{eqnarray}

and $z$ is chosen distinct from $\quotep{P}$, $\quotep{Q}$, the free
names in $Q$, and all the names in $R$. Our $\alpha$-equivalence will
be built in the standard way from this substitution.

\begin{remark}\label{rem:no_self_referential_names}
  One consequence of these definitions is that $\forall P. \quotep{P}
  \not\in \freenames{P}$.
\end{remark}

\subsection{ Dynamic quote: an example }

Anticipating something of what's to come, consider applying the
substitution, $\widehat{\id{\{}u / z \id{\}}}$, to the following pair
of processes, $\lift{w}{y!(z)}$ and $w[ \lpquote y!(z) \rpquote ]$.

\begin{eqnarray}
	\lift{w}{y!(z)}\widehat{\id{\{}u / z \id{\}}}
		& = &
		\lift{w}{y!(u)} \nonumber\\
	w[ \lpquote y!(z) \rpquote ] \widehat{ \id{\{}u / z \id{\}} }
		& = &
		w[ \lpquote y!(z) \rpquote ] \nonumber
\end{eqnarray}

Because the body of the process between quotes is impervious to
substitution, we get radically different answers. In fact, by
examining the first process in an input context,
e.g. $x?(z).\lift{w}{y!(z)}$, we see that the process under the lift
operator may be shaped by prefixed inputs binding a name inside it. In
this sense, the lift operator will be seen as a way to dynamically
construct processes before reifying them as names.

Finally equipped with these standard features we can present the
dynamics of the calculus.

\subsubsection{Operational semantics} 

Finally, we introduce the computational dynamics. What marks these
algebras as distinct from other more traditionally studied algebraic
structures, e.g. vector spaces or polynomial rings, is the manner in
which dynamics is captured. In traditional structures, dynamics is typically
expressed through morphisms between such structures, as in linear maps
between vector spaces or morphisms between rings. In algebras
associated with the semantics of computation, the dynamics is
expressed as part of the algebraic structure itself, through a
reduction reduction relation typically denoted by $\red$. Below, we
give a recursive presentation of this relation for the calculus used
in the encoding.

$\red \subseteq \pi \times \pi$
$\red : \pi \to \mathcal{P}(\pi)$

\begin{mathpar}
  \inferrule* [lab=Comm] { \textsf{match}( x_{src}, x_{trgt} ) } { x_{trgt}?(y)P \; | \; x_{src}!\langle {Q} \rangle \red P\{\quotep{Q}/y}\} }
  \and \\
  \inferrule* [lab=Par] {{P} \red {P}'} {{{P} | {Q}} \red {{P}' | {Q}}}
  \and
  \inferrule* [lab=Equiv]{{{P} \scong {P}'} \andalso {{P}' \red {Q}'} \andalso {{Q}' \scong {Q}}}{{P} \red {Q}}
\end{mathpar}

\begin{eqnarray*}
  match_{\equiv} (\quotep{P},\quotep{Q}) & := & P \equiv Q \\
  match_{\dagger}(\quotep{P},\quotep{Q}) & := & \forall R. P|Q \red^{*} R => R \red^{*} 0 \\
  match_{K}(\quotep{P},\quotep{Q}) & := & K \mbox{ for some context } K
\end{eqnarray*}

$u?(x)P | u!\langle Q \rangle \red P\{\quotep{Q}/x\}$

%We write $\wred$ for $\red^*$, and $P\red$ if $\exists Q $ such that $ P \red Q$.
We write $P\red$ if $\exists Q $ such that $ P \red Q$ and $P\not\red$, otherwise.

\section{Replication}

As mentioned before, it is known that replication (and hence
recursion) can be implemented in a higher-order process algebra
\cite{SangiorgiWalker}. As our first example of calculation with the
machinery thus far presented we give the construction explicitly in
the {\rhoc}.

\begin{eqnarray}
	D_{x} & := & \prefix{x}{y}{(\binpar{\outputp{x}{y}}{@{y}})} \nonumber\\
	\bangp_{x}{P} & := & \binpar{{x}!\langle{\binpar{D_{x}}{P}}\rangle}{D_{x}} \nonumber
\end{eqnarray}

\begin{eqnarray}
	\bangp_{x}{P} & & \nonumber\\
	=
	& {x}!\langle{(\prefix{x}{y}{(\outputp{x}{y} | @{y})) | P}}\rangle 
	      | \prefix{x}{y}{(\outputp{x}{y} | @{y})} & \nonumber\\
	\red
	& (\outputp{x}{y} | @{y})\substn{\quotep{(\prefix{x}{y}{(@{y} | \outputp{x}{y})) | P}}}{y} & \nonumber\\
	=
	& \outputp{x}{\quotep{(\prefix{x}{y}{(\outputp{x}{y} | @{y})) | P}}}
	  | {(\prefix{x}{y}{(\outputp{x}{y} | @{y})) | P}} & \nonumber\\
	\red
	& \ldots & \nonumber\\
	\red^*
	& P | P | \ldots & \nonumber
\end{eqnarray}

Of course, this encoding, as an implementation, runs away, unfolding
$\bangp{P}$ eagerly. A lazier and more implementable replication
operator, restricted to input-guarded processes, may be obtained as follows.

\begin{eqnarray}
\bangp{\prefix{u}{v}{P}} 
	:= 
	\binpar{\lift{x}{\prefix{u}{v}{(\binpar{D(x)}{P})}}}{D(x)} \nonumber
\end{eqnarray}

\begin{remark}
  Note that the lazier definition still does not deal with summation
  or mixed summation (i.e. sums over input and output). The reader is
  invited to construct definitions of replication that deal with these
  features. 

  Further, the definitions are parameterized in a name, $x$. Can you,
  gentle reader, make a definition that eliminates this parameter and
  guarantees no accidental interaction between the replication
  machinery and the process being replicated -- i.e. no accidental
  sharing of names used by the process to get its work done and the
  name(s) used by the replication to effect copying. This latter
  revision of the definition of replication is crucial to obtaining
  the expected identity $!!P \sim !P$.
\end{remark}

\begin{remark}\label{rem:paradoxical_combinator}
  The reader familiar with the lambda calculus will have noticed the
  similarity between $D$ and the paradoxical combinator.

  [Ed. note: the existence of this seems to suggest we have to be more
  restrictive on the set of processes and names we admit if we are to
  support no-cloning.]
\end{remark}

\subsubsection{Bisimulation}

The computational dynamics gives rise to another kind of equivalence,
the equivalence of computational behavior. As previously mentioned
this is typically captured \emph{via} some form of bisimulation.

% The notion we use in this paper is weak barbed bisimulation
% \cite{milner91polyadicpi}.

The notion we use in this paper is derived from weak barbed
bisimulation \cite{milner91polyadicpi}. 

\begin{definition}
An \emph{observation relation}, $\downarrow_{\mathcal N}$, over a set
of names, $\mathcal N$, is the smallest relation satisfying the rules
below.

\infrule[Out-barb]{y \in {\mathcal N}, \; x \nameeq y}
		  {\outputp{x}{v} \downarrow_{\mathcal N} x}
\infrule[Par-barb]{\mbox{$P\downarrow_{\mathcal N} x$ or $Q\downarrow_{\mathcal N} x$}}
		  {\binpar{P}{Q} \downarrow_{\mathcal N} x}

We write $P \Downarrow_{\mathcal N} x$ if there is $Q$ such that 
$P \wred Q$ and $Q \downarrow_{\mathcal N} x$.
\end{definition}

\begin{definition}
%\label{def.bbisim}
An  ${\mathcal N}$-\emph{barbed bisimulation} over a set of names, ${\mathcal N}$, is a symmetric binary relation 
${\mathcal S}_{\mathcal N}$ between agents such that $P\rel{S}_{\mathcal N}Q$ implies:
\begin{enumerate}
\item If $P \red P'$ then $Q \wred Q'$ and $P'\rel{S}_{\mathcal N} Q'$.
\item If $P\downarrow_{\mathcal N} x$, then $Q\Downarrow_{\mathcal N} x$.
\end{enumerate}
$P$ is ${\mathcal N}$-barbed bisimilar to $Q$, written
$P \wbbisim_{\mathcal N} Q$, if $P \rel{S}_{\mathcal N} Q$ for some ${\mathcal N}$-barbed bisimulation ${\mathcal S}_{\mathcal N}$.
\end{definition}

$\mathcal{R} \subseteq \pi \times \pi$

$P \mathcal{R} Q => \forall P'. P \red P' \Rightarrow \exists Q'. Q \red Q', P' \mathcal{R} Q'$

$P \vdash x \Rightarrow Q \vdash x$

\begin{mathpar}
  \inferrule*[lab=Out-barb]{x \nameeq y}{{y}!\langle{Q}\rangle \vdash x}
  \and
  \inferrule*[lab=Par-barb]{\mbox{$P\vdash x$ or $Q\vdash x$}}{\binpar{P}{Q} \vdash x}
\end{mathpar}

\subsubsection{Contexts}

One of the principle advantages of computational calculi like the
$\pi$-calculus is a well-defined notion of context,
contextual-equivalence and a correlation between
contextual-equivalence and notions of bisimulation. The notion of
context allows the decomposition of a process into (sub-)process and
its syntactic environment, its context. Thus, a context may be
thought of as a process with a ``hole'' (written $\Box$) in it. The
application of a context $M$ to a process $P$, written $M[P]$, is
tantamount to filling the hole in $M$ with $P$. In this paper we do
not need the full weight of this theory, but do make use of the notion
of context in the proof the main theorem. 

\begin{mathpar}
  \inferrule* [lab=summation] {} {{M_{M},M_{N}} \bc \Box \;|\; x.M_{A} \;|\; M_{M}+M_{N}}
  \and
  \inferrule* [lab=agent] {} {{M_{A}} \bc (\vec{x})M_{P} \;| \; \clift{P_0,\ldots,M_{P},\ldots,P_N}}
  \and \\
  \inferrule* [lab=process] {} {{M_{P}} \bc M_{N} \;| \;P|M_{P} }
\end{mathpar} 

\begin{mathpar}
  \inferrule* [lab=sychronization] {} {M_{N} \bc \Box \;|\; x?M_{F} \;|\; x!M_{C}}
  \and
  \inferrule* [lab=abstraction] {} {{M_{F}} \bc (x)M_{P} }
  \and
  \inferrule* [lab=concretion] {} {{M_{C}} \bc \langle M_{P} \rangle }
  \and \\
  \inferrule* [lab=process] {} {{M_{P}} \bc M_{N} \;| \;P|M_{P} }
\end{mathpar}

\begin{definition}[contextual application] Given a context $M$, and
  process $P$, we define the \emph{contextual application}, $M[P] :=
  M\{P/\Box\}$. That is, the contextual application of M to P is the
  substitution of $P$ for $\Box$ in $M$.
\end{definition}

$\meaningof{-} : L \to \mathcal{P}(\pi)$

\begin{mathpar}
  \inferrule* [lab=collection] {} {\meaningof{true} = \pi, \and \meaningof{~E} = \pi \setminus \meaningof{E}, \and \meaningof{E_{1} \& E_{2}} = \meaningof{E_{1}} \cap \meaningof{E_{2}}}
\end{mathpar}

\begin{mathpar}
  \inferrule* [lab=structure] {} {\meaningof{0} = \{ P \in \pi | P \equiv 0 \}, \and \\ \meaningof{E_1 | E_2} = \{ P \in \pi | P \equiv P_{1} | P_{2}, P_{1} \in \meaningof{E_{1}}, P_{2} \in \meaningof{E_2}\} }
\end{mathpar}

\begin{mathpar}
 \inferrule* [lab=behavior] {} {\meaningof{\langle a?b \rangle E} = \{ P \in \pi | P \equiv Q | u?(y)P', \\ \and \\\\ \and \\ \;\;\; u \in \meaningof{a}, \forall z.P'\{z/y\} \in \meaningof{E\{z/b\}}\}, \and \\ \meaningof{a!E} = \{ P \in \pi | P \equiv Q | x!\langle P' \rangle, x \in \meaningof{a} P' \in \meaningof{E}\} }
\end{mathpar}

\begin{mathpar}
 \inferrule* [lab=nominal] {} {\meaningof{\quotep{E}} = \{ \quotep{P} \in \quotep{\pi} | P \in \meaningof{E} \}, \and \meaningof{\quotep{P}} = \{ \quotep{Q} \in \quotep{\pi} | P \equiv Q \} \and \\ \meaningof{@\quotep{E}} = \{ P \in \pi | P \equiv @x, x \in \meaningof{E} \}}
\end{mathpar}

\begin{eqnarray*}
  \\
  \meaningof{-} : TS \to ST
\end{eqnarray*}

\begin{eqnarray*}
  \\
  L : TS \to ST
\end{eqnarray*}

\begin{eqnarray*}
  \\
  P \models E \iff P \in \meaningof{E}
\end{eqnarray*}

\begin{eqnarray*}
  P \approx_{L} Q \iff \forall E \in L. P \models E \iff Q \models E
\end{eqnarray*}

\begin{eqnarray*}
  P \approx_{K} Q
\end{eqnarray*}

\begin{eqnarray*}
  P \approx Q
\end{eqnarray*}

$\approx_{K} = \approx = \approx_{L}$

\subsubsection{Contextual duality}

Note that contexts extend the quotation operation to a family of
operations from processes to names. Given a context, $M$, we can
define a \emph{nominal context}, $\quotep{M}$ by $\quotep{M}[P] :=
\quotep{M[P]}$. To foreshadow what is to come we observe that these
operations enjoy a duality with processes very much like the duality
between vectors and maps from vectors to scalars.

Further, because the calculus is essentially higher-order, we have a
correspondence between contexts and processes. More specifically,
given a name $x$ and a context $M$ we can construct $M^{*}_{x}$ such
that 

\begin{mathpar}
  M^{*}_{x} | \lift{x}{P} \red M[P]
\end{mathpar}

namely,

\begin{mathpar}
  M^{*}_{x} := x?(u).M[\dropn{u}]
\end{mathpar}

The dependence of $M^{*}_{x}$ on a name makes it an abstraction, 

\begin{mathpar}
  M^{*} := (x)x?(u).M[\dropn{u}]
\end{mathpar}

\subsection{Additional notation}

It will sometimes be convenient to denote the process a name
quotes. We already have the notation $x = \quotep{P}$, but it will be
convenient to introduce an alternate notation, $\procn{x}$, when we
want to emphasize the connection to the use of the name. Note that, by
virtue of name equivalence, $\quotep{\procn{x}} \nameeq x$; so, the
notation is consistent with previous definitions.

Further, because names have structure it is possible to effect
substitutions on the basis of that structure. This means we need to
upgrade our notation for substitutions, which we accomplish by
adapting comprehension notation. Thus,

\begin{mathpar}
  P\{ y / x : x \in S \}
\end{mathpar}

is interpreted to mean the process derived from P by replacing (in a
capture-avoiding manner) each occurrence of $x$ in $S$ by $y$. For example,

\begin{mathpar}
  P\{ \quotep{\procn{x}|\procn{x}} / x : x \in \freenames{P} \}
\end{mathpar}

will replace each (occurrence) of a free name $x$ in $P$ by
$\quotep{\procn{x}|\procn{x}}$.

Also, we will avail ourselves of the notation $x^{L}$ and $x^{R}$ to
denote injections of a name into disjoint copies of the name
space. There are numerous ways to accomplish this. One example can be
found in \cite{MeredithR05}. This notation overloads to vectors of
names: $\vec{x}^{\pi} := (x_{i}^{\pi} \; : \; 0 \leq i < |\vec{x}| )$ where $\pi \in \{L,R\}$.

We also use $P^{\Box} := P|\Box$.

In \cite{MeredithR05} an interpretation of the new operator is
given. It turns out that there are several possible interpretations
all enjoying the requisite algebraic properties of the operator (see
\cite{milner91polyadicpi}). We will therefore make liberal use of
$(\nu\; \vec{x})P$.

% subsection the_syntax_and_semantics_of_the_notation_system (end)   

\input{qm2pi.qmops} 

\input{qm2pi.sterngerlach} 

\input{qm2pi.metric} 

% section concurrent_process_calculi (end)

%\input{qm2pi.proofsketch}

% section proof sketch (end)

%\input{qm2pi.slviaknots} 

% section spatial logic via knots (end)

\input{qm2pi.conclusion}

% section conclusion (end)

%\input{qm2pi.dtcodes} 

% section wiring algorithm (end)

\input{qm2pi.ack} 

% section acknowledgments (end)

\newpage


\bibliographystyle{plain}   
\bibliography{../../biblios/main.bib}

\input{qm2pi.rhodetails}

\end{document}

 

% subsection basic_interpretation (end)

%\input{qm2pi.rho.presentation} 
\subsection{The syntax and semantics of the notation system}\label{sub:the_syntax_and_semantics_of_the_notation_system} % (fold)

We now summarize a technical presentation of the calculus that
embodies our theory of dynamics. The typical presentation of such a
calculus follows the style of giving generators and relations on
them. The grammar, below, describing term constructors, freely
generates the set of processes, $\Proc$. This set is then quotiented
by a relation known as structural congruence and it is over this set
that the notion of dynamics is expressed. This presentation is
essentially that of \cite{MeredithR05} with the addition of
polyadicity and summation. For readability we have relegated some of
the technical subtleties to an appendix.

\subsubsection{Process grammar}\label{subsub:process_grammar}

\begin{mathpar}
  \inferrule* [lab=synchronization] {} {{M} \bc \pzero \;|\; x?F \;|\; x!C }
  \and
  \inferrule* [lab=abstraction] {} {{F} \bc (x)P}
  \and
  \inferrule* [lab=concretion] {} {{C} \bc \langle Q \rangle}
  \and
  \inferrule* [lab=process] {} {{P,Q} \bc M \;| \;P|Q \;|\; @{x}}
  \and
  \inferrule* [lab=name] {} {{x} \bc \quotep{P}}
\end{mathpar} 

Note that $\vec{x}$ (resp. $\vec{P}$) denotes a vector of names
(resp. processes) of length $|\vec{x}|$ (resp. $|\vec{P}|$). We adopt
the following useful abbreviations.

\begin{mathpar}
   x?(\vec{y}).P := x.(\vec{y})P \and  x\clift{\vec{P}} := x.\clift{\vec{P}}
   \and x!(y) := \lift{x}{\dropn{y}}
   \and \Pi_{i=0}^{n-1}P_i := P_0 | \ldots | P_{n-1}
\end{mathpar}

\subsubsection{Structural congruence}

\paragraph{Free and bound names and alpha-equivalence.} At the
core of structural equivalence is alpha-equivalence which identifies
process that are the same up to a change of variable. Formally, we
recognize the distinction between free and bound names. The free names
of a process, $\freenames{P}$, may be calculated recursively as
follows:

\begin{mathpar}
\freenames{\pzero} := \emptyset
  \and \\
  \freenames{x?(y).P} := \{ x \} \cup (\freenames{P} \setminus \{ y \})
  \and 
  \freenames{x!\langle P \rangle} := \{ x \} \cup \{ P \} 
  \and \\
  \freenames{P|Q} := \freenames{P} \cup \freenames{Q}
  \and \\
  \freenames{@{x}} := \{ x \}
\end{mathpar}

$\pi$
$\quotep{\pi}$

$\freenames{-} : \pi \to \mathcal{P}(\quotep{\pi})$

\begin{eqnarray*}
  \freenames{\pzero} & := & \emptyset \\
  \freenames{x?(y).P} & := & \{ x \} \cup (\freenames{P} \setminus \{ y \}) \\
  \freenames{x!\langle P \rangle} & := & \{ x \} \cup \{ P \} \\
  \freenames{P|Q} & := & \freenames{P} \cup \freenames{Q} \\
  \freenames{\dropn{x}} & := & \{ x \}
\end{eqnarray*}

The bound names of a process, $\boundnames{P}$, are those names occurring in $P$
that are not free. For example, in $x?(y).0$, the name $x$ is free, while $y$ is bound.

\begin{mathpar}
  \inferrule* [lab=monoidal-laws] {} { P|Q \equiv Q|P \and P|0 \equiv P \and P|(Q|R) \equiv (P|Q)|R }
\end{mathpar}

\begin{mathpar}
  \inferrule* [lab=alpha-equivalence] {} { (x)P \equiv (y)P\{y/x\} \and y \not\in \freenames{P} }
\end{mathpar}

\begin{definition}
Then two processes, $P,Q$, are alpha-equivalent if $P = Q\{\vec{y}/\vec{x}\}$ for
some $\vec{x} \in \boundnames{Q},\vec{y} \in \boundnames{P}$, where $Q\{\vec{y}/\vec{x}\}$
denotes the capture-avoiding substitution of $\vec{y}$ for $\vec{x}$ in $Q$.
\end{definition}

\begin{definition}
  The {\em structural congruence} \cite{SangiorgiWalker} , $\equiv$,
  between processes is the least congruence containing
  alpha-equivalence, satisfying the abelian monoid laws
  (associativity, commutativity and $\pzero$ as identity) for parallel
  composition $|$ and for summation $+$.
\end{definition}

\subsection{Name equivalence}

We take name equivalence, written $\nameeq$, to be the smallest
equivalence relation generated by the following rules.

\begin{mathpar}
\inferrule*[lab=Quote-drop]
{ }
{ \quotep{@{x}} \nameeq x }

\inferrule*[lab=Struct-equiv]
{ P \scong Q }
{ \quotep{P} \nameeq \quotep{Q} }
\end{mathpar}

The astute reader will have noticed that the mutual recursion of names
and processes imposes a mutual recursion on alpha-equivalence and
structural equivalence via name-equivalence. Fortunately, all of this
works out pleasantly and we may calculate in the natural way, free of
concern. The reader interested in the details is referred to the
appendix \ref{appendix:rho_details}.

\subsection{Substitution}

We use $\Proc$ for the set of processes, $\QProc$ for the set of
names, and $\id{\{}\vec{y} / \vec{x} \id{\}}$ to denote partial maps,
$s : \QProc \rightarrow \QProc$. A map, $s$ lifts, uniquely, to a map
on process terms, $\widehat{s} : \Proc \rightarrow \Proc$ by the
following equations.

\begin{mathpar}
  (0) \psubstp{Q}{P} := 0 \\
  (R \juxtap S) \psubstp{Q}{P}
  :=    
  (R)\psubstp{Q}{P} \juxtap (S) \psubstp{Q}{P} \\
  (x?(y).R) \psubstp{Q}{P}    
  :=    
  (x)\substp{Q}{P} (z)\concat( (R \psubstn{z}{y}) \psubstp{Q}{P} ) \\
  (\lift{x}{R}) \psubstp{Q}{P}  
  :=
  \lift{(x)\substp{Q}{P}}{ R \psubstp{Q}{P} } \\
%   (\dropn{x})  \psubstp{Q}{P}       
%   := 
%   \left\{ 
%     \begin{array}{ccc} 
%       \dropn{\quotep{Q}} & & x \nameeq \quotep{P} \\
%       \dropn{x} & & otherwise \\
%     \end{array}
%   \right. 
  (\dropn{x})  \psubstp{Q}{P}       
  := 
  \left\{ 
    \begin{array}{ccc} 
      Q & & x \nameeq \quotep{P} \\
      \dropn{x} & & otherwise \\
    \end{array}
  \right.
\end{mathpar}
 

where

\begin{eqnarray}
  (x)\id{\{} \lpquote Q \rpquote / \lpquote P \rpquote \id{\}}            = 
  \left\{ 
    \begin{array}{ccc}
      \lpquote Q \rpquote & & x \nameeq \lpquote P \rpquote \\
      x & & otherwise \\
    \end{array}
  \right. \nonumber
\end{eqnarray}

and $z$ is chosen distinct from $\quotep{P}$, $\quotep{Q}$, the free
names in $Q$, and all the names in $R$. Our $\alpha$-equivalence will
be built in the standard way from this substitution.

\begin{remark}\label{rem:no_self_referential_names}
  One consequence of these definitions is that $\forall P. \quotep{P}
  \not\in \freenames{P}$.
\end{remark}

\subsection{ Dynamic quote: an example }

Anticipating something of what's to come, consider applying the
substitution, $\widehat{\id{\{}u / z \id{\}}}$, to the following pair
of processes, $\lift{w}{y!(z)}$ and $w[ \lpquote y!(z) \rpquote ]$.

\begin{eqnarray}
	\lift{w}{y!(z)}\widehat{\id{\{}u / z \id{\}}}
		& = &
		\lift{w}{y!(u)} \nonumber\\
	w[ \lpquote y!(z) \rpquote ] \widehat{ \id{\{}u / z \id{\}} }
		& = &
		w[ \lpquote y!(z) \rpquote ] \nonumber
\end{eqnarray}

Because the body of the process between quotes is impervious to
substitution, we get radically different answers. In fact, by
examining the first process in an input context,
e.g. $x?(z).\lift{w}{y!(z)}$, we see that the process under the lift
operator may be shaped by prefixed inputs binding a name inside it. In
this sense, the lift operator will be seen as a way to dynamically
construct processes before reifying them as names.

Finally equipped with these standard features we can present the
dynamics of the calculus.

\subsubsection{Operational semantics} 

Finally, we introduce the computational dynamics. What marks these
algebras as distinct from other more traditionally studied algebraic
structures, e.g. vector spaces or polynomial rings, is the manner in
which dynamics is captured. In traditional structures, dynamics is typically
expressed through morphisms between such structures, as in linear maps
between vector spaces or morphisms between rings. In algebras
associated with the semantics of computation, the dynamics is
expressed as part of the algebraic structure itself, through a
reduction reduction relation typically denoted by $\red$. Below, we
give a recursive presentation of this relation for the calculus used
in the encoding.

$\red \subseteq \pi \times \pi$
$\red : \pi \to \mathcal{P}(\pi)$

\begin{mathpar}
  \inferrule* [lab=Comm] { \textsf{match}( x_{src}, x_{trgt} ) } { x_{trgt}?(y)P \; | \; x_{src}!\langle {Q} \rangle \red P\{\quotep{Q}/y}\} }
  \and \\
  \inferrule* [lab=Par] {{P} \red {P}'} {{{P} | {Q}} \red {{P}' | {Q}}}
  \and
  \inferrule* [lab=Equiv]{{{P} \scong {P}'} \andalso {{P}' \red {Q}'} \andalso {{Q}' \scong {Q}}}{{P} \red {Q}}
\end{mathpar}

\begin{eqnarray*}
  match_{\equiv} (\quotep{P},\quotep{Q}) & := & P \equiv Q \\
  match_{\dagger}(\quotep{P},\quotep{Q}) & := & \forall R. P|Q \red^{*} R => R \red^{*} 0 \\
  match_{K}(\quotep{P},\quotep{Q}) & := & K \mbox{ for some context } K
\end{eqnarray*}

$u?(x)P | u!\langle Q \rangle \red P\{\quotep{Q}/x\}$

%We write $\wred$ for $\red^*$, and $P\red$ if $\exists Q $ such that $ P \red Q$.
We write $P\red$ if $\exists Q $ such that $ P \red Q$ and $P\not\red$, otherwise.

\section{Replication}

As mentioned before, it is known that replication (and hence
recursion) can be implemented in a higher-order process algebra
\cite{SangiorgiWalker}. As our first example of calculation with the
machinery thus far presented we give the construction explicitly in
the {\rhoc}.

\begin{eqnarray}
	D_{x} & := & \prefix{x}{y}{(\binpar{\outputp{x}{y}}{@{y}})} \nonumber\\
	\bangp_{x}{P} & := & \binpar{{x}!\langle{\binpar{D_{x}}{P}}\rangle}{D_{x}} \nonumber
\end{eqnarray}

\begin{eqnarray}
	\bangp_{x}{P} & & \nonumber\\
	=
	& {x}!\langle{(\prefix{x}{y}{(\outputp{x}{y} | @{y})) | P}}\rangle 
	      | \prefix{x}{y}{(\outputp{x}{y} | @{y})} & \nonumber\\
	\red
	& (\outputp{x}{y} | @{y})\substn{\quotep{(\prefix{x}{y}{(@{y} | \outputp{x}{y})) | P}}}{y} & \nonumber\\
	=
	& \outputp{x}{\quotep{(\prefix{x}{y}{(\outputp{x}{y} | @{y})) | P}}}
	  | {(\prefix{x}{y}{(\outputp{x}{y} | @{y})) | P}} & \nonumber\\
	\red
	& \ldots & \nonumber\\
	\red^*
	& P | P | \ldots & \nonumber
\end{eqnarray}

Of course, this encoding, as an implementation, runs away, unfolding
$\bangp{P}$ eagerly. A lazier and more implementable replication
operator, restricted to input-guarded processes, may be obtained as follows.

\begin{eqnarray}
\bangp{\prefix{u}{v}{P}} 
	:= 
	\binpar{\lift{x}{\prefix{u}{v}{(\binpar{D(x)}{P})}}}{D(x)} \nonumber
\end{eqnarray}

\begin{remark}
  Note that the lazier definition still does not deal with summation
  or mixed summation (i.e. sums over input and output). The reader is
  invited to construct definitions of replication that deal with these
  features. 

  Further, the definitions are parameterized in a name, $x$. Can you,
  gentle reader, make a definition that eliminates this parameter and
  guarantees no accidental interaction between the replication
  machinery and the process being replicated -- i.e. no accidental
  sharing of names used by the process to get its work done and the
  name(s) used by the replication to effect copying. This latter
  revision of the definition of replication is crucial to obtaining
  the expected identity $!!P \sim !P$.
\end{remark}

\begin{remark}\label{rem:paradoxical_combinator}
  The reader familiar with the lambda calculus will have noticed the
  similarity between $D$ and the paradoxical combinator.

  [Ed. note: the existence of this seems to suggest we have to be more
  restrictive on the set of processes and names we admit if we are to
  support no-cloning.]
\end{remark}

\subsubsection{Bisimulation}

The computational dynamics gives rise to another kind of equivalence,
the equivalence of computational behavior. As previously mentioned
this is typically captured \emph{via} some form of bisimulation.

% The notion we use in this paper is weak barbed bisimulation
% \cite{milner91polyadicpi}.

The notion we use in this paper is derived from weak barbed
bisimulation \cite{milner91polyadicpi}. 

\begin{definition}
An \emph{observation relation}, $\downarrow_{\mathcal N}$, over a set
of names, $\mathcal N$, is the smallest relation satisfying the rules
below.

\infrule[Out-barb]{y \in {\mathcal N}, \; x \nameeq y}
		  {\outputp{x}{v} \downarrow_{\mathcal N} x}
\infrule[Par-barb]{\mbox{$P\downarrow_{\mathcal N} x$ or $Q\downarrow_{\mathcal N} x$}}
		  {\binpar{P}{Q} \downarrow_{\mathcal N} x}

We write $P \Downarrow_{\mathcal N} x$ if there is $Q$ such that 
$P \wred Q$ and $Q \downarrow_{\mathcal N} x$.
\end{definition}

\begin{definition}
%\label{def.bbisim}
An  ${\mathcal N}$-\emph{barbed bisimulation} over a set of names, ${\mathcal N}$, is a symmetric binary relation 
${\mathcal S}_{\mathcal N}$ between agents such that $P\rel{S}_{\mathcal N}Q$ implies:
\begin{enumerate}
\item If $P \red P'$ then $Q \wred Q'$ and $P'\rel{S}_{\mathcal N} Q'$.
\item If $P\downarrow_{\mathcal N} x$, then $Q\Downarrow_{\mathcal N} x$.
\end{enumerate}
$P$ is ${\mathcal N}$-barbed bisimilar to $Q$, written
$P \wbbisim_{\mathcal N} Q$, if $P \rel{S}_{\mathcal N} Q$ for some ${\mathcal N}$-barbed bisimulation ${\mathcal S}_{\mathcal N}$.
\end{definition}

$\mathcal{R} \subseteq \pi \times \pi$

$P \mathcal{R} Q => \forall P'. P \red P' \Rightarrow \exists Q'. Q \red Q', P' \mathcal{R} Q'$

$P \vdash x \Rightarrow Q \vdash x$

\begin{mathpar}
  \inferrule*[lab=Out-barb]{x \nameeq y}{{y}!\langle{Q}\rangle \vdash x}
  \and
  \inferrule*[lab=Par-barb]{\mbox{$P\vdash x$ or $Q\vdash x$}}{\binpar{P}{Q} \vdash x}
\end{mathpar}

\subsubsection{Contexts}

One of the principle advantages of computational calculi like the
$\pi$-calculus is a well-defined notion of context,
contextual-equivalence and a correlation between
contextual-equivalence and notions of bisimulation. The notion of
context allows the decomposition of a process into (sub-)process and
its syntactic environment, its context. Thus, a context may be
thought of as a process with a ``hole'' (written $\Box$) in it. The
application of a context $M$ to a process $P$, written $M[P]$, is
tantamount to filling the hole in $M$ with $P$. In this paper we do
not need the full weight of this theory, but do make use of the notion
of context in the proof the main theorem. 

\begin{mathpar}
  \inferrule* [lab=summation] {} {{M_{M},M_{N}} \bc \Box \;|\; x.M_{A} \;|\; M_{M}+M_{N}}
  \and
  \inferrule* [lab=agent] {} {{M_{A}} \bc (\vec{x})M_{P} \;| \; \clift{P_0,\ldots,M_{P},\ldots,P_N}}
  \and \\
  \inferrule* [lab=process] {} {{M_{P}} \bc M_{N} \;| \;P|M_{P} }
\end{mathpar} 

\begin{mathpar}
  \inferrule* [lab=sychronization] {} {M_{N} \bc \Box \;|\; x?M_{F} \;|\; x!M_{C}}
  \and
  \inferrule* [lab=abstraction] {} {{M_{F}} \bc (x)M_{P} }
  \and
  \inferrule* [lab=concretion] {} {{M_{C}} \bc \langle M_{P} \rangle }
  \and \\
  \inferrule* [lab=process] {} {{M_{P}} \bc M_{N} \;| \;P|M_{P} }
\end{mathpar}

\begin{definition}[contextual application] Given a context $M$, and
  process $P$, we define the \emph{contextual application}, $M[P] :=
  M\{P/\Box\}$. That is, the contextual application of M to P is the
  substitution of $P$ for $\Box$ in $M$.
\end{definition}

$\meaningof{-} : L \to \mathcal{P}(\pi)$

\begin{mathpar}
  \inferrule* [lab=collection] {} {\meaningof{true} = \pi, \and \meaningof{~E} = \pi \setminus \meaningof{E}, \and \meaningof{E_{1} \& E_{2}} = \meaningof{E_{1}} \cap \meaningof{E_{2}}}
\end{mathpar}

\begin{mathpar}
  \inferrule* [lab=structure] {} {\meaningof{0} = \{ P \in \pi | P \equiv 0 \}, \and \\ \meaningof{E_1 | E_2} = \{ P \in \pi | P \equiv P_{1} | P_{2}, P_{1} \in \meaningof{E_{1}}, P_{2} \in \meaningof{E_2}\} }
\end{mathpar}

\begin{mathpar}
 \inferrule* [lab=behavior] {} {\meaningof{\langle a?b \rangle E} = \{ P \in \pi | P \equiv Q | u?(y)P', \\ \and \\\\ \and \\ \;\;\; u \in \meaningof{a}, \forall z.P'\{z/y\} \in \meaningof{E\{z/b\}}\}, \and \\ \meaningof{a!E} = \{ P \in \pi | P \equiv Q | x!\langle P' \rangle, x \in \meaningof{a} P' \in \meaningof{E}\} }
\end{mathpar}

\begin{mathpar}
 \inferrule* [lab=nominal] {} {\meaningof{\quotep{E}} = \{ \quotep{P} \in \quotep{\pi} | P \in \meaningof{E} \}, \and \meaningof{\quotep{P}} = \{ \quotep{Q} \in \quotep{\pi} | P \equiv Q \} \and \\ \meaningof{@\quotep{E}} = \{ P \in \pi | P \equiv @x, x \in \meaningof{E} \}}
\end{mathpar}

\begin{eqnarray*}
  \\
  \meaningof{-} : TS \to ST
\end{eqnarray*}

\begin{eqnarray*}
  \\
  L : TS \to ST
\end{eqnarray*}

\begin{eqnarray*}
  \\
  P \models E \iff P \in \meaningof{E}
\end{eqnarray*}

\begin{eqnarray*}
  P \approx_{L} Q \iff \forall E \in L. P \models E \iff Q \models E
\end{eqnarray*}

\begin{eqnarray*}
  P \approx_{K} Q
\end{eqnarray*}

\begin{eqnarray*}
  P \approx Q
\end{eqnarray*}

$\approx_{K} = \approx = \approx_{L}$

\subsubsection{Contextual duality}

Note that contexts extend the quotation operation to a family of
operations from processes to names. Given a context, $M$, we can
define a \emph{nominal context}, $\quotep{M}$ by $\quotep{M}[P] :=
\quotep{M[P]}$. To foreshadow what is to come we observe that these
operations enjoy a duality with processes very much like the duality
between vectors and maps from vectors to scalars.

Further, because the calculus is essentially higher-order, we have a
correspondence between contexts and processes. More specifically,
given a name $x$ and a context $M$ we can construct $M^{*}_{x}$ such
that 

\begin{mathpar}
  M^{*}_{x} | \lift{x}{P} \red M[P]
\end{mathpar}

namely,

\begin{mathpar}
  M^{*}_{x} := x?(u).M[\dropn{u}]
\end{mathpar}

The dependence of $M^{*}_{x}$ on a name makes it an abstraction, 

\begin{mathpar}
  M^{*} := (x)x?(u).M[\dropn{u}]
\end{mathpar}

\subsection{Additional notation}

It will sometimes be convenient to denote the process a name
quotes. We already have the notation $x = \quotep{P}$, but it will be
convenient to introduce an alternate notation, $\procn{x}$, when we
want to emphasize the connection to the use of the name. Note that, by
virtue of name equivalence, $\quotep{\procn{x}} \nameeq x$; so, the
notation is consistent with previous definitions.

Further, because names have structure it is possible to effect
substitutions on the basis of that structure. This means we need to
upgrade our notation for substitutions, which we accomplish by
adapting comprehension notation. Thus,

\begin{mathpar}
  P\{ y / x : x \in S \}
\end{mathpar}

is interpreted to mean the process derived from P by replacing (in a
capture-avoiding manner) each occurrence of $x$ in $S$ by $y$. For example,

\begin{mathpar}
  P\{ \quotep{\procn{x}|\procn{x}} / x : x \in \freenames{P} \}
\end{mathpar}

will replace each (occurrence) of a free name $x$ in $P$ by
$\quotep{\procn{x}|\procn{x}}$.

Also, we will avail ourselves of the notation $x^{L}$ and $x^{R}$ to
denote injections of a name into disjoint copies of the name
space. There are numerous ways to accomplish this. One example can be
found in \cite{MeredithR05}. This notation overloads to vectors of
names: $\vec{x}^{\pi} := (x_{i}^{\pi} \; : \; 0 \leq i < |\vec{x}| )$ where $\pi \in \{L,R\}$.

We also use $P^{\Box} := P|\Box$.

In \cite{MeredithR05} an interpretation of the new operator is
given. It turns out that there are several possible interpretations
all enjoying the requisite algebraic properties of the operator (see
\cite{milner91polyadicpi}). We will therefore make liberal use of
$(\nu\; \vec{x})P$.

% subsection the_syntax_and_semantics_of_the_notation_system (end)   

\section{Interpretation of QM}
\subsection{Supporting definitions}
\subsubsection{Multiplication}
\begin{mathpar}
  \quotep{Q} \cdot \quotep{R} := \quotep{Q|R}
  \and \\
  \quotep{Q} \cdot P := P\{ \quotep{Q|R} / \quotep{R} : \quotep{R} \in \freenames{P} \}
\end{mathpar}

\paragraph{Discussion}
The first line needs little explanation. The second line says that
each free name of the process is replaced with the multiplication of
that name by the scalar. Multiplication of a scalar (name) by a state
(process) results in a process all the names of which have been `moved
over' by parallel composition with the process the scalar
quotes. There is a subtlety that the bound names have to be
manipulated so that multiplied names aren't accidentally
captured. There are many ways to achieve this.

\begin{remark}\label{rem:multiplication_identities}
  The reader is invited to verify that for all $x,y,z \in \QProc$ and $P \in \Proc$
  \begin{mathpar}
    x \cdot \quotep{0} \equiv x 
    \and
    x \cdot y \equiv y \cdot x
    \and
    x \cdot (y \cdot z) \equiv (x \cdot y) \cdot z
    \and \\
    \quotep{0} \cdot P \equiv P
    \and \\
    x \cdot (y \cdot P) \equiv (x \cdot y) \cdot P
    \and \\
    x \cdot (P|Q) \equiv (x \cdot P) | (x \cdot Q)
    \and \\    
  \end{mathpar}
\end{remark}

\subsubsection{Tensor product}

We define a tensor product on processes by structural induction.

\paragraph{Tensor of sums} First note that all summations, including
$\pzero$ and sequence, can be written $\Sigma_{i} x_{i}.A_{i} +
\Sigma_{j} x_{j}.C_{j}$, where we have grouped input-guarded processes
together and output-guarded processes together.

Thus, we can define the tensor product of two summations, $N_{1}\otimes N_{2}$, where

\begin{mathpar}
  N_{1} := \Sigma_{i} x_{i}.A_{i} + \Sigma_{j} x_{j}.C_{j}
  \and
  N_{2} := \Sigma_{i'} y_{i'}.B_{i'} + \Sigma_{j'} y_{j'}.D_{j'} 
\end{mathpar}

as follows.

\begin{mathpar}
  \Sigma_{i} x_{i}.A_{i} + \Sigma_{j} x_{j}.C_{j} \otimes \Sigma_{i'}
  y_{i'}.B_{i'} + \Sigma_{j'} y_{j'}.D_{j'} 
  \and \\
  := \; \Sigma_{i} \Sigma_{i'} \quotep{\stackrel{\vee}{x_{i}}| \stackrel{\vee}{y_{i'}}}.(A_{i}\otimes B_{i'}) \; | \; \Sigma_{i'} \Sigma_{i} \quotep{\stackrel{\vee}{y_{i'}}|\stackrel{\vee}{x_{i}}}.(B_{i'}\otimes A_{i})
  \and
  \;\; | \;\; \Sigma_{j} \Sigma_{j'} \quotep{\stackrel{\vee}{x_{j}}|\stackrel{\vee}{y_{j'}}}.(A_{j}\otimes B_{j'}) \; | \; \Sigma_{j'} \Sigma_{j} \quotep{\stackrel{\vee}{y_{j'}}|\stackrel{\vee}{x_{j}}}.(B_{j'}\otimes A_{j})
\end{mathpar}

\begin{remark}
  Do we need to $x^{L}$ and $y^{R}$ for this construction as well?
\end{remark}

\paragraph{Tensor of parallel compositions} Next, we distribute tensor
over par.

\begin{mathpar}
  P_{1}|P_{2} \otimes Q_{1}|Q_{2} := (P_{1} \otimes Q_{1}) | (P_{1}
  \otimes Q_{2}) | (P_{2} \otimes Q_{1}) | (P_{2} \otimes Q_{2})
\end{mathpar}

\paragraph{Tensor with dropped names} We treat tensor of a
process with a dropped name as parallel composition.

\begin{mathpar}
  P \otimes \dropn{x} := P | \dropn{x}
\end{mathpar}

\paragraph{Tensor of agents}

Finally, we need to define tensor on agents. Note that the definition
of tensor on normal products only tensors inputs with inputs and
outputs with outputs. Thus, we only have to define the operation on
``homogeneous'' pairings.

\begin{mathpar}
  (\vec{x})P \otimes (\vec{y})Q
  \and \\
  := (x_{0}^{L}|y_{0}^{R},\ldots,x_{0}^{L}|y_{n}^{R},\ldots,x_{m}^{L}|y_{0}^{R},\ldots,x_{m}^{L}|y_{n}^R)(P\{ \vec{x}^{L}/\vec{x}\} \otimes Q \{ \vec{y}^{R}/\vec{y}\})
  \and \\
  \clift{\vec{P}} \otimes \clift{\vec{Q}}
  \and \\
  := \clift{P_{0}\otimes Q_{0},\ldots,P_{0}\otimes Q_{n},\ldots,P_{m}\otimes Q_{0},\ldots,P_{m}\otimes Q_{n}}
\end{mathpar}

\begin{remark}
  Observe that arities of tensored abstractions matches arities of
  tensored concretions if the original arities matched. Note also that
  the length of the arities corresponds to the increase in dimension
  we see in ordinary vector space tensor product.
\end{remark}

\begin{remark}
  Operationally, this definition distributes the tensor down to
  components ``linked'' by summation. Tensor over summation is
  intriguing in that it mixes names. Moreover, as a consequence of the
  way it mixes names we have the identities for all $x \in \QProc$ and
  $P,Q \in \Proc$

  \begin{mathpar}
    (x \cdot P) \otimes Q \equiv x \cdot (P \otimes Q) \equiv P \otimes (x \cdot Q)
    \and
    P \otimes \pzero \equiv P
  \end{mathpar}

  that the reader is invited to verify.
\end{remark}

\subsubsection{Annihilation}
\begin{mathpar}
  P^{\perp} := \{ Q | \forall R. P|Q \red^{*} R \Rightarrow R \red^{*} \pzero \}
  \and \\
  P^{\underline{\perp}} := \Sigma_{Q \in P^{\perp}} \quotep{Q}?(y).(\dropn{y}|Q) | \Sigma_{Q \in P^{\perp}} \quotep{Q}\clift{\Box}
\end{mathpar}

\paragraph{Discussion} The reader will note that $P^{\perp}$ is a
\emph{set} of processes, while $P^{\underline{\perp}}$ is a
\emph{context}. We call the set $P^{\perp}$ the \emph{annihilators} of
$P$. The parallel composition of a process in the annihilators of $P$
with $P$ will result in a process, the state space of which has all
paths eventually leading to $\pzero$. Execution may endure loops; but
under reasonable conditions of fairness (naturally guaranteed under
most notions of bisimulation) such a composite process cannot get
stuck in such a loop and will, eventually pop out and terminate.

The context $P^{\underline{\perp}}$ is ready and willing to ``take the
$P$ out of'' the process to which it is applied. It will effectively
transmit the code of the process to which it is applied to one of the
annihilators and run the process against it.

\subsubsection{Evaluation}
We fix $M$ a domain of fully abstract interpretation with an equality
coincident with bisimulation. We take $\meaningof{\cdot} : \Proc \to
M$ to be the map interpreting processes and $\nmeaningof{\cdot} : \M
\to Proc$ to be the map running the other way. Then we define

\begin{mathpar}
  \int P := \nmeaningof{\meaningof{P}}
\end{mathpar}

\paragraph{Discussion}
There are many fully abstract interpretations of Milner's
$\pi$-calculus. Any of them can be used as a basis for interpreting
the reflective calculus here. Equipped with such a domain it is
largely a matter of grinding through to check that the Yoneda
construction for the normalization-by-evaluation program can be
extended to this setting.

\begin{remark}
  The reader is invited to verify that $\int (P^{\underline{\perp}}[P]) = 0$.
\end{remark}

\subsection{Quantum mechanics}

Table \ref{tbl:core_qm_op_defns} gives the core operational definitions

\begin{table}[htp]\label{tbl:core_qm_op_defns}
  \center{
    \fbox{
      \begin{tabular}{c|c}
        quantum mechanics & process calculus \\
        \hline
        scalar & $x := \quotep{P}$ \\
        state vector & $\state{P} := P$ \\
        dual & $\state{P}^{*} := \event{P^{\underline{\perp}}} := \quotep{P^{\underline{\perp}}}[-]$ \\
        matrix & $ \Sigma_{\alpha} \state{P_{\alpha}}x_{\alpha}\event{Q_{\alpha}}$ \\
        vector addition & $\state{P} + \state{Q} := \state{P | Q}$ \\
        tensor product & $\state{P} \otimes \state{Q} := \state{P \otimes Q}$ \\
        inner product & $\innerprod{P}{Q} := \quotep{\int P^{\underline{\perp}}[Q]}$ \\
      \end{tabular}
    }
  }
  \caption{QM - operational definitions}
\end{table}

where

\begin{mathpar}
  \prmatrix{P}{Q} := \fprmatrix{P}{\quotep{\pzero}}{Q}
  \and
  \fprmatrix{P}{x}{Q} := (\state{P},x,\event{Q})
  \and
  (\fprmatrix{P}{x}{Q})(\state{R}) := x \cdot \innerprod{Q}{R} \cdot \state{P}
  \and
  (\fprmatrix{P}{x}{Q})(\event{R}) := x \cdot \innerprod{R}{P} \cdot \event{Q}
\end{mathpar}

\paragraph{Discussion}
As promised: vectors (aka states) are represented as processes; duals
as contextual duals; inner product definition should be compared with
standard inner product definition for ....

\begin{remark}
  Assuming $\int (P^{\underline{\perp}}[P]) = 0$, the reader is
  invited to verify that $(\fprmatrix{P}{x}{P})(\state{P}) = x \cdot \state{P}$.
\end{remark}

\begin{remark}
  The reader is invited to verify that $\innerprod{P}{Q}$ could
  equally well have been written $\quotep{\int \stackrel{\vee}{x}}$
  where $x = \event{P^{\underline{\perp}}}(Q)$.

  One of the motivations for this remark is that there is another way
  to factor these operations. We could package up evaluation in the dual:

  \begin{mathpar}
    \state{P}^{*} := \event{\int P^{\underline{\perp}}} := \quotep{\int P^{\underline{\perp}}}[-]
  \end{mathpar}

  and then have inner product defined by
  
  \begin{mathpar}
    \innerprod{P}{Q} := \event{P}(Q)
  \end{mathpar}

  Hopefully, experience with the calculations will provide guidance on
  the best factoring.
\end{remark}

\begin{remark}
  Assuming $\int (P^{\underline{\perp}}[P]) = 0$, the reader is
  invited to verify that $\forall P,Q. (\prmatrix{0}{Q})(\state{0}) =
  \state{0}$ and dually $(\prmatrix{P}{0})(\event{0}) = \event{0}$.
\end{remark}

\begin{remark}
  i'm a little worried that i don't (yet) have proper support for
  complex conjugacy. But, the observation above may give us a
  clue. According to Abramsky, it must be the case that the scalars
  are iso to the homset of the identity for the tensor -- which the
  observation above characterizes. 

  For now, we will simply bookmark the notion with $\overline{x}$.
\end{remark}

\subsubsection{Adjointness}

We need to give a definition of $(\cdot)^{\dagger}$ for matrices. The
obvious candidate definition is
\begin{mathpar}
(\Sigma_{\alpha}\fprmatrix{P_{\alpha}}{x_{\alpha}}{Q_{\alpha}})^{\dagger}
= \Sigma_{\alpha}\fprmatrix{(Q_{\alpha}^{\underline{\perp}})^{*}}{\overline{x}_{\alpha}}{P_{\alpha}^{\underline{\perp}}} 
\end{mathpar}

But, $(Q_{\alpha}^{\underline{\perp}})^{*}$ requires a name along
which to communicate the process to achieve the context application.

\subsubsection{Basis for a basis}
If processes label states and ``addition'' of states (a.k.a. vector
addition) is interpreted as parallel composition, what corresponds to
notions of linear independence and basis? Here, we recall that Yoshida
has developed a set of \emph{combinators} for an asynchronous verison
of Milner's $\pi$-calculus. These are a finite set of processes such
any process can be expressed as parallel composition of these
combinators together with liberal uses of the new operator and
replication. We can simply give a translation of these into the
present calculus and have reasonable expectation that the property
carries over. That is, that the resultant set allows to express all
processes via parallel composition. Note, however, that there is no
new operator or replication in this calculus. As a result, we expect
that the corresponding set is actually infinite. That is, we expect
that the space is actually infinite dimensional.

\begin{remark}
  The attentive reader may be a bit concerned. Certainly, the
  collection $S$, $K$ and $I$ is a finite set of
  combinators. Shouldn't we expect to see a finite set of combinators
  for an effectively equivalent system? i am very sympathetic to this
  critique and feel it warrants full attention. On the other hand, i
  also have in mind the following analogy. The natural numbers, as a
  monoid under addition, has exactly $1$ generator, while the natural
  numbers, as a monoid under multiplication, has countably many
  generators (the primes). We observe that the application of the
  lambda calculus is much less resource sensitive than the parallel
  composition of the $\pi$-calculus. Could it be the case that we have
  an analogy of the form
  
  \begin{mathpar}
    m + n : MN :: m*n : M|N
  \end{mathpar}

  giving a similar blow up in the set of ``primes''?  This is such a
  wonderful thought that, even if it's not true, i think it's worth
  writing down.
\end{remark}
 

\documentclass[12pt]{llncs}
%\documentclass{jktr}

\usepackage[pdftex]{hyperref}                   
\usepackage {listings}
\usepackage {mathpartir}
\usepackage{bcprules}
%\usepackage{listings}
                       
\usepackage{graphicx} 
%\usepackage[margins=2.5cm,nohead,nofoot]{geometry}
%\usepackage{geometry}
\usepackage{amsfonts}
\usepackage{amstext}
\usepackage{latexsym}
\usepackage{amssymb}
\usepackage{color}


%\include{myPreamble}
\include{qm2pi.local} 

%\ifpdf
%\usepackage[pdftex]{graphicx}
%\else
%\usepackage{graphicx}
%\fi

 % \ifpdf
%  \usepackage{pdfsync}
%  \if


%\title{Brief Article}
%\author{David F. Snyder}
%\author{L.G. Meredith}

%\address{Dept. of Math., Texas State University--San Marcos, San Marcos, TX 78666}
       
\pagestyle{empty}


\begin{document}

\lstset{language=[Objective]Caml,frame=shadowbox}

\input{qm2pi.front}

% section front matter (end)

\input{qm2pi.intro} 
 
% section introduction (end)

% \input{qm2pi.knotations} 

% section notation (end)

\input{qm2pi.process.calculi} 

% section concurrent_process_calculi_and_spatial_logics_ (end)
    
%\input{qm2pi.knots2pi} 

%\input{qm2pi.trefoil} 

%\input{qm2pi.mainthm} 

% subsection basic_interpretation (end)

%\input{qm2pi.rho.presentation} 
\subsection{The syntax and semantics of the notation system}\label{sub:the_syntax_and_semantics_of_the_notation_system} % (fold)

We now summarize a technical presentation of the calculus that
embodies our theory of dynamics. The typical presentation of such a
calculus follows the style of giving generators and relations on
them. The grammar, below, describing term constructors, freely
generates the set of processes, $\Proc$. This set is then quotiented
by a relation known as structural congruence and it is over this set
that the notion of dynamics is expressed. This presentation is
essentially that of \cite{MeredithR05} with the addition of
polyadicity and summation. For readability we have relegated some of
the technical subtleties to an appendix.

\subsubsection{Process grammar}\label{subsub:process_grammar}

\begin{mathpar}
  \inferrule* [lab=synchronization] {} {{M} \bc \pzero \;|\; x?F \;|\; x!C }
  \and
  \inferrule* [lab=abstraction] {} {{F} \bc (x)P}
  \and
  \inferrule* [lab=concretion] {} {{C} \bc \langle Q \rangle}
  \and
  \inferrule* [lab=process] {} {{P,Q} \bc M \;| \;P|Q \;|\; @{x}}
  \and
  \inferrule* [lab=name] {} {{x} \bc \quotep{P}}
\end{mathpar} 

Note that $\vec{x}$ (resp. $\vec{P}$) denotes a vector of names
(resp. processes) of length $|\vec{x}|$ (resp. $|\vec{P}|$). We adopt
the following useful abbreviations.

\begin{mathpar}
   x?(\vec{y}).P := x.(\vec{y})P \and  x\clift{\vec{P}} := x.\clift{\vec{P}}
   \and x!(y) := \lift{x}{\dropn{y}}
   \and \Pi_{i=0}^{n-1}P_i := P_0 | \ldots | P_{n-1}
\end{mathpar}

\subsubsection{Structural congruence}

\paragraph{Free and bound names and alpha-equivalence.} At the
core of structural equivalence is alpha-equivalence which identifies
process that are the same up to a change of variable. Formally, we
recognize the distinction between free and bound names. The free names
of a process, $\freenames{P}$, may be calculated recursively as
follows:

\begin{mathpar}
\freenames{\pzero} := \emptyset
  \and \\
  \freenames{x?(y).P} := \{ x \} \cup (\freenames{P} \setminus \{ y \})
  \and 
  \freenames{x!\langle P \rangle} := \{ x \} \cup \{ P \} 
  \and \\
  \freenames{P|Q} := \freenames{P} \cup \freenames{Q}
  \and \\
  \freenames{@{x}} := \{ x \}
\end{mathpar}

$\pi$
$\quotep{\pi}$

$\freenames{-} : \pi \to \mathcal{P}(\quotep{\pi})$

\begin{eqnarray*}
  \freenames{\pzero} & := & \emptyset \\
  \freenames{x?(y).P} & := & \{ x \} \cup (\freenames{P} \setminus \{ y \}) \\
  \freenames{x!\langle P \rangle} & := & \{ x \} \cup \{ P \} \\
  \freenames{P|Q} & := & \freenames{P} \cup \freenames{Q} \\
  \freenames{\dropn{x}} & := & \{ x \}
\end{eqnarray*}

The bound names of a process, $\boundnames{P}$, are those names occurring in $P$
that are not free. For example, in $x?(y).0$, the name $x$ is free, while $y$ is bound.

\begin{mathpar}
  \inferrule* [lab=monoidal-laws] {} { P|Q \equiv Q|P \and P|0 \equiv P \and P|(Q|R) \equiv (P|Q)|R }
\end{mathpar}

\begin{mathpar}
  \inferrule* [lab=alpha-equivalence] {} { (x)P \equiv (y)P\{y/x\} \and y \not\in \freenames{P} }
\end{mathpar}

\begin{definition}
Then two processes, $P,Q$, are alpha-equivalent if $P = Q\{\vec{y}/\vec{x}\}$ for
some $\vec{x} \in \boundnames{Q},\vec{y} \in \boundnames{P}$, where $Q\{\vec{y}/\vec{x}\}$
denotes the capture-avoiding substitution of $\vec{y}$ for $\vec{x}$ in $Q$.
\end{definition}

\begin{definition}
  The {\em structural congruence} \cite{SangiorgiWalker} , $\equiv$,
  between processes is the least congruence containing
  alpha-equivalence, satisfying the abelian monoid laws
  (associativity, commutativity and $\pzero$ as identity) for parallel
  composition $|$ and for summation $+$.
\end{definition}

\subsection{Name equivalence}

We take name equivalence, written $\nameeq$, to be the smallest
equivalence relation generated by the following rules.

\begin{mathpar}
\inferrule*[lab=Quote-drop]
{ }
{ \quotep{@{x}} \nameeq x }

\inferrule*[lab=Struct-equiv]
{ P \scong Q }
{ \quotep{P} \nameeq \quotep{Q} }
\end{mathpar}

The astute reader will have noticed that the mutual recursion of names
and processes imposes a mutual recursion on alpha-equivalence and
structural equivalence via name-equivalence. Fortunately, all of this
works out pleasantly and we may calculate in the natural way, free of
concern. The reader interested in the details is referred to the
appendix \ref{appendix:rho_details}.

\subsection{Substitution}

We use $\Proc$ for the set of processes, $\QProc$ for the set of
names, and $\id{\{}\vec{y} / \vec{x} \id{\}}$ to denote partial maps,
$s : \QProc \rightarrow \QProc$. A map, $s$ lifts, uniquely, to a map
on process terms, $\widehat{s} : \Proc \rightarrow \Proc$ by the
following equations.

\begin{mathpar}
  (0) \psubstp{Q}{P} := 0 \\
  (R \juxtap S) \psubstp{Q}{P}
  :=    
  (R)\psubstp{Q}{P} \juxtap (S) \psubstp{Q}{P} \\
  (x?(y).R) \psubstp{Q}{P}    
  :=    
  (x)\substp{Q}{P} (z)\concat( (R \psubstn{z}{y}) \psubstp{Q}{P} ) \\
  (\lift{x}{R}) \psubstp{Q}{P}  
  :=
  \lift{(x)\substp{Q}{P}}{ R \psubstp{Q}{P} } \\
%   (\dropn{x})  \psubstp{Q}{P}       
%   := 
%   \left\{ 
%     \begin{array}{ccc} 
%       \dropn{\quotep{Q}} & & x \nameeq \quotep{P} \\
%       \dropn{x} & & otherwise \\
%     \end{array}
%   \right. 
  (\dropn{x})  \psubstp{Q}{P}       
  := 
  \left\{ 
    \begin{array}{ccc} 
      Q & & x \nameeq \quotep{P} \\
      \dropn{x} & & otherwise \\
    \end{array}
  \right.
\end{mathpar}
 

where

\begin{eqnarray}
  (x)\id{\{} \lpquote Q \rpquote / \lpquote P \rpquote \id{\}}            = 
  \left\{ 
    \begin{array}{ccc}
      \lpquote Q \rpquote & & x \nameeq \lpquote P \rpquote \\
      x & & otherwise \\
    \end{array}
  \right. \nonumber
\end{eqnarray}

and $z$ is chosen distinct from $\quotep{P}$, $\quotep{Q}$, the free
names in $Q$, and all the names in $R$. Our $\alpha$-equivalence will
be built in the standard way from this substitution.

\begin{remark}\label{rem:no_self_referential_names}
  One consequence of these definitions is that $\forall P. \quotep{P}
  \not\in \freenames{P}$.
\end{remark}

\subsection{ Dynamic quote: an example }

Anticipating something of what's to come, consider applying the
substitution, $\widehat{\id{\{}u / z \id{\}}}$, to the following pair
of processes, $\lift{w}{y!(z)}$ and $w[ \lpquote y!(z) \rpquote ]$.

\begin{eqnarray}
	\lift{w}{y!(z)}\widehat{\id{\{}u / z \id{\}}}
		& = &
		\lift{w}{y!(u)} \nonumber\\
	w[ \lpquote y!(z) \rpquote ] \widehat{ \id{\{}u / z \id{\}} }
		& = &
		w[ \lpquote y!(z) \rpquote ] \nonumber
\end{eqnarray}

Because the body of the process between quotes is impervious to
substitution, we get radically different answers. In fact, by
examining the first process in an input context,
e.g. $x?(z).\lift{w}{y!(z)}$, we see that the process under the lift
operator may be shaped by prefixed inputs binding a name inside it. In
this sense, the lift operator will be seen as a way to dynamically
construct processes before reifying them as names.

Finally equipped with these standard features we can present the
dynamics of the calculus.

\subsubsection{Operational semantics} 

Finally, we introduce the computational dynamics. What marks these
algebras as distinct from other more traditionally studied algebraic
structures, e.g. vector spaces or polynomial rings, is the manner in
which dynamics is captured. In traditional structures, dynamics is typically
expressed through morphisms between such structures, as in linear maps
between vector spaces or morphisms between rings. In algebras
associated with the semantics of computation, the dynamics is
expressed as part of the algebraic structure itself, through a
reduction reduction relation typically denoted by $\red$. Below, we
give a recursive presentation of this relation for the calculus used
in the encoding.

$\red \subseteq \pi \times \pi$
$\red : \pi \to \mathcal{P}(\pi)$

\begin{mathpar}
  \inferrule* [lab=Comm] { \textsf{match}( x_{src}, x_{trgt} ) } { x_{trgt}?(y)P \; | \; x_{src}!\langle {Q} \rangle \red P\{\quotep{Q}/y}\} }
  \and \\
  \inferrule* [lab=Par] {{P} \red {P}'} {{{P} | {Q}} \red {{P}' | {Q}}}
  \and
  \inferrule* [lab=Equiv]{{{P} \scong {P}'} \andalso {{P}' \red {Q}'} \andalso {{Q}' \scong {Q}}}{{P} \red {Q}}
\end{mathpar}

\begin{eqnarray*}
  match_{\equiv} (\quotep{P},\quotep{Q}) & := & P \equiv Q \\
  match_{\dagger}(\quotep{P},\quotep{Q}) & := & \forall R. P|Q \red^{*} R => R \red^{*} 0 \\
  match_{K}(\quotep{P},\quotep{Q}) & := & K \mbox{ for some context } K
\end{eqnarray*}

$u?(x)P | u!\langle Q \rangle \red P\{\quotep{Q}/x\}$

%We write $\wred$ for $\red^*$, and $P\red$ if $\exists Q $ such that $ P \red Q$.
We write $P\red$ if $\exists Q $ such that $ P \red Q$ and $P\not\red$, otherwise.

\section{Replication}

As mentioned before, it is known that replication (and hence
recursion) can be implemented in a higher-order process algebra
\cite{SangiorgiWalker}. As our first example of calculation with the
machinery thus far presented we give the construction explicitly in
the {\rhoc}.

\begin{eqnarray}
	D_{x} & := & \prefix{x}{y}{(\binpar{\outputp{x}{y}}{@{y}})} \nonumber\\
	\bangp_{x}{P} & := & \binpar{{x}!\langle{\binpar{D_{x}}{P}}\rangle}{D_{x}} \nonumber
\end{eqnarray}

\begin{eqnarray}
	\bangp_{x}{P} & & \nonumber\\
	=
	& {x}!\langle{(\prefix{x}{y}{(\outputp{x}{y} | @{y})) | P}}\rangle 
	      | \prefix{x}{y}{(\outputp{x}{y} | @{y})} & \nonumber\\
	\red
	& (\outputp{x}{y} | @{y})\substn{\quotep{(\prefix{x}{y}{(@{y} | \outputp{x}{y})) | P}}}{y} & \nonumber\\
	=
	& \outputp{x}{\quotep{(\prefix{x}{y}{(\outputp{x}{y} | @{y})) | P}}}
	  | {(\prefix{x}{y}{(\outputp{x}{y} | @{y})) | P}} & \nonumber\\
	\red
	& \ldots & \nonumber\\
	\red^*
	& P | P | \ldots & \nonumber
\end{eqnarray}

Of course, this encoding, as an implementation, runs away, unfolding
$\bangp{P}$ eagerly. A lazier and more implementable replication
operator, restricted to input-guarded processes, may be obtained as follows.

\begin{eqnarray}
\bangp{\prefix{u}{v}{P}} 
	:= 
	\binpar{\lift{x}{\prefix{u}{v}{(\binpar{D(x)}{P})}}}{D(x)} \nonumber
\end{eqnarray}

\begin{remark}
  Note that the lazier definition still does not deal with summation
  or mixed summation (i.e. sums over input and output). The reader is
  invited to construct definitions of replication that deal with these
  features. 

  Further, the definitions are parameterized in a name, $x$. Can you,
  gentle reader, make a definition that eliminates this parameter and
  guarantees no accidental interaction between the replication
  machinery and the process being replicated -- i.e. no accidental
  sharing of names used by the process to get its work done and the
  name(s) used by the replication to effect copying. This latter
  revision of the definition of replication is crucial to obtaining
  the expected identity $!!P \sim !P$.
\end{remark}

\begin{remark}\label{rem:paradoxical_combinator}
  The reader familiar with the lambda calculus will have noticed the
  similarity between $D$ and the paradoxical combinator.

  [Ed. note: the existence of this seems to suggest we have to be more
  restrictive on the set of processes and names we admit if we are to
  support no-cloning.]
\end{remark}

\subsubsection{Bisimulation}

The computational dynamics gives rise to another kind of equivalence,
the equivalence of computational behavior. As previously mentioned
this is typically captured \emph{via} some form of bisimulation.

% The notion we use in this paper is weak barbed bisimulation
% \cite{milner91polyadicpi}.

The notion we use in this paper is derived from weak barbed
bisimulation \cite{milner91polyadicpi}. 

\begin{definition}
An \emph{observation relation}, $\downarrow_{\mathcal N}$, over a set
of names, $\mathcal N$, is the smallest relation satisfying the rules
below.

\infrule[Out-barb]{y \in {\mathcal N}, \; x \nameeq y}
		  {\outputp{x}{v} \downarrow_{\mathcal N} x}
\infrule[Par-barb]{\mbox{$P\downarrow_{\mathcal N} x$ or $Q\downarrow_{\mathcal N} x$}}
		  {\binpar{P}{Q} \downarrow_{\mathcal N} x}

We write $P \Downarrow_{\mathcal N} x$ if there is $Q$ such that 
$P \wred Q$ and $Q \downarrow_{\mathcal N} x$.
\end{definition}

\begin{definition}
%\label{def.bbisim}
An  ${\mathcal N}$-\emph{barbed bisimulation} over a set of names, ${\mathcal N}$, is a symmetric binary relation 
${\mathcal S}_{\mathcal N}$ between agents such that $P\rel{S}_{\mathcal N}Q$ implies:
\begin{enumerate}
\item If $P \red P'$ then $Q \wred Q'$ and $P'\rel{S}_{\mathcal N} Q'$.
\item If $P\downarrow_{\mathcal N} x$, then $Q\Downarrow_{\mathcal N} x$.
\end{enumerate}
$P$ is ${\mathcal N}$-barbed bisimilar to $Q$, written
$P \wbbisim_{\mathcal N} Q$, if $P \rel{S}_{\mathcal N} Q$ for some ${\mathcal N}$-barbed bisimulation ${\mathcal S}_{\mathcal N}$.
\end{definition}

$\mathcal{R} \subseteq \pi \times \pi$

$P \mathcal{R} Q => \forall P'. P \red P' \Rightarrow \exists Q'. Q \red Q', P' \mathcal{R} Q'$

$P \vdash x \Rightarrow Q \vdash x$

\begin{mathpar}
  \inferrule*[lab=Out-barb]{x \nameeq y}{{y}!\langle{Q}\rangle \vdash x}
  \and
  \inferrule*[lab=Par-barb]{\mbox{$P\vdash x$ or $Q\vdash x$}}{\binpar{P}{Q} \vdash x}
\end{mathpar}

\subsubsection{Contexts}

One of the principle advantages of computational calculi like the
$\pi$-calculus is a well-defined notion of context,
contextual-equivalence and a correlation between
contextual-equivalence and notions of bisimulation. The notion of
context allows the decomposition of a process into (sub-)process and
its syntactic environment, its context. Thus, a context may be
thought of as a process with a ``hole'' (written $\Box$) in it. The
application of a context $M$ to a process $P$, written $M[P]$, is
tantamount to filling the hole in $M$ with $P$. In this paper we do
not need the full weight of this theory, but do make use of the notion
of context in the proof the main theorem. 

\begin{mathpar}
  \inferrule* [lab=summation] {} {{M_{M},M_{N}} \bc \Box \;|\; x.M_{A} \;|\; M_{M}+M_{N}}
  \and
  \inferrule* [lab=agent] {} {{M_{A}} \bc (\vec{x})M_{P} \;| \; \clift{P_0,\ldots,M_{P},\ldots,P_N}}
  \and \\
  \inferrule* [lab=process] {} {{M_{P}} \bc M_{N} \;| \;P|M_{P} }
\end{mathpar} 

\begin{mathpar}
  \inferrule* [lab=sychronization] {} {M_{N} \bc \Box \;|\; x?M_{F} \;|\; x!M_{C}}
  \and
  \inferrule* [lab=abstraction] {} {{M_{F}} \bc (x)M_{P} }
  \and
  \inferrule* [lab=concretion] {} {{M_{C}} \bc \langle M_{P} \rangle }
  \and \\
  \inferrule* [lab=process] {} {{M_{P}} \bc M_{N} \;| \;P|M_{P} }
\end{mathpar}

\begin{definition}[contextual application] Given a context $M$, and
  process $P$, we define the \emph{contextual application}, $M[P] :=
  M\{P/\Box\}$. That is, the contextual application of M to P is the
  substitution of $P$ for $\Box$ in $M$.
\end{definition}

$\meaningof{-} : L \to \mathcal{P}(\pi)$

\begin{mathpar}
  \inferrule* [lab=collection] {} {\meaningof{true} = \pi, \and \meaningof{~E} = \pi \setminus \meaningof{E}, \and \meaningof{E_{1} \& E_{2}} = \meaningof{E_{1}} \cap \meaningof{E_{2}}}
\end{mathpar}

\begin{mathpar}
  \inferrule* [lab=structure] {} {\meaningof{0} = \{ P \in \pi | P \equiv 0 \}, \and \\ \meaningof{E_1 | E_2} = \{ P \in \pi | P \equiv P_{1} | P_{2}, P_{1} \in \meaningof{E_{1}}, P_{2} \in \meaningof{E_2}\} }
\end{mathpar}

\begin{mathpar}
 \inferrule* [lab=behavior] {} {\meaningof{\langle a?b \rangle E} = \{ P \in \pi | P \equiv Q | u?(y)P', \\ \and \\\\ \and \\ \;\;\; u \in \meaningof{a}, \forall z.P'\{z/y\} \in \meaningof{E\{z/b\}}\}, \and \\ \meaningof{a!E} = \{ P \in \pi | P \equiv Q | x!\langle P' \rangle, x \in \meaningof{a} P' \in \meaningof{E}\} }
\end{mathpar}

\begin{mathpar}
 \inferrule* [lab=nominal] {} {\meaningof{\quotep{E}} = \{ \quotep{P} \in \quotep{\pi} | P \in \meaningof{E} \}, \and \meaningof{\quotep{P}} = \{ \quotep{Q} \in \quotep{\pi} | P \equiv Q \} \and \\ \meaningof{@\quotep{E}} = \{ P \in \pi | P \equiv @x, x \in \meaningof{E} \}}
\end{mathpar}

\begin{eqnarray*}
  \\
  \meaningof{-} : TS \to ST
\end{eqnarray*}

\begin{eqnarray*}
  \\
  L : TS \to ST
\end{eqnarray*}

\begin{eqnarray*}
  \\
  P \models E \iff P \in \meaningof{E}
\end{eqnarray*}

\begin{eqnarray*}
  P \approx_{L} Q \iff \forall E \in L. P \models E \iff Q \models E
\end{eqnarray*}

\begin{eqnarray*}
  P \approx_{K} Q
\end{eqnarray*}

\begin{eqnarray*}
  P \approx Q
\end{eqnarray*}

$\approx_{K} = \approx = \approx_{L}$

\subsubsection{Contextual duality}

Note that contexts extend the quotation operation to a family of
operations from processes to names. Given a context, $M$, we can
define a \emph{nominal context}, $\quotep{M}$ by $\quotep{M}[P] :=
\quotep{M[P]}$. To foreshadow what is to come we observe that these
operations enjoy a duality with processes very much like the duality
between vectors and maps from vectors to scalars.

Further, because the calculus is essentially higher-order, we have a
correspondence between contexts and processes. More specifically,
given a name $x$ and a context $M$ we can construct $M^{*}_{x}$ such
that 

\begin{mathpar}
  M^{*}_{x} | \lift{x}{P} \red M[P]
\end{mathpar}

namely,

\begin{mathpar}
  M^{*}_{x} := x?(u).M[\dropn{u}]
\end{mathpar}

The dependence of $M^{*}_{x}$ on a name makes it an abstraction, 

\begin{mathpar}
  M^{*} := (x)x?(u).M[\dropn{u}]
\end{mathpar}

\subsection{Additional notation}

It will sometimes be convenient to denote the process a name
quotes. We already have the notation $x = \quotep{P}$, but it will be
convenient to introduce an alternate notation, $\procn{x}$, when we
want to emphasize the connection to the use of the name. Note that, by
virtue of name equivalence, $\quotep{\procn{x}} \nameeq x$; so, the
notation is consistent with previous definitions.

Further, because names have structure it is possible to effect
substitutions on the basis of that structure. This means we need to
upgrade our notation for substitutions, which we accomplish by
adapting comprehension notation. Thus,

\begin{mathpar}
  P\{ y / x : x \in S \}
\end{mathpar}

is interpreted to mean the process derived from P by replacing (in a
capture-avoiding manner) each occurrence of $x$ in $S$ by $y$. For example,

\begin{mathpar}
  P\{ \quotep{\procn{x}|\procn{x}} / x : x \in \freenames{P} \}
\end{mathpar}

will replace each (occurrence) of a free name $x$ in $P$ by
$\quotep{\procn{x}|\procn{x}}$.

Also, we will avail ourselves of the notation $x^{L}$ and $x^{R}$ to
denote injections of a name into disjoint copies of the name
space. There are numerous ways to accomplish this. One example can be
found in \cite{MeredithR05}. This notation overloads to vectors of
names: $\vec{x}^{\pi} := (x_{i}^{\pi} \; : \; 0 \leq i < |\vec{x}| )$ where $\pi \in \{L,R\}$.

We also use $P^{\Box} := P|\Box$.

In \cite{MeredithR05} an interpretation of the new operator is
given. It turns out that there are several possible interpretations
all enjoying the requisite algebraic properties of the operator (see
\cite{milner91polyadicpi}). We will therefore make liberal use of
$(\nu\; \vec{x})P$.

% subsection the_syntax_and_semantics_of_the_notation_system (end)   

\input{qm2pi.qmops} 

\input{qm2pi.sterngerlach} 

\input{qm2pi.metric} 

% section concurrent_process_calculi (end)

%\input{qm2pi.proofsketch}

% section proof sketch (end)

%\input{qm2pi.slviaknots} 

% section spatial logic via knots (end)

\input{qm2pi.conclusion}

% section conclusion (end)

%\input{qm2pi.dtcodes} 

% section wiring algorithm (end)

\input{qm2pi.ack} 

% section acknowledgments (end)

\newpage


\bibliographystyle{plain}   
\bibliography{../../biblios/main.bib}

\input{qm2pi.rhodetails}

\end{document}

 

\documentclass[12pt]{llncs}
%\documentclass{jktr}

\usepackage[pdftex]{hyperref}                   
\usepackage {listings}
\usepackage {mathpartir}
\usepackage{bcprules}
%\usepackage{listings}
                       
\usepackage{graphicx} 
%\usepackage[margins=2.5cm,nohead,nofoot]{geometry}
%\usepackage{geometry}
\usepackage{amsfonts}
\usepackage{amstext}
\usepackage{latexsym}
\usepackage{amssymb}
\usepackage{color}


%\include{myPreamble}
\include{qm2pi.local} 

%\ifpdf
%\usepackage[pdftex]{graphicx}
%\else
%\usepackage{graphicx}
%\fi

 % \ifpdf
%  \usepackage{pdfsync}
%  \if


%\title{Brief Article}
%\author{David F. Snyder}
%\author{L.G. Meredith}

%\address{Dept. of Math., Texas State University--San Marcos, San Marcos, TX 78666}
       
\pagestyle{empty}


\begin{document}

\lstset{language=[Objective]Caml,frame=shadowbox}

\input{qm2pi.front}

% section front matter (end)

\input{qm2pi.intro} 
 
% section introduction (end)

% \input{qm2pi.knotations} 

% section notation (end)

\input{qm2pi.process.calculi} 

% section concurrent_process_calculi_and_spatial_logics_ (end)
    
%\input{qm2pi.knots2pi} 

%\input{qm2pi.trefoil} 

%\input{qm2pi.mainthm} 

% subsection basic_interpretation (end)

%\input{qm2pi.rho.presentation} 
\subsection{The syntax and semantics of the notation system}\label{sub:the_syntax_and_semantics_of_the_notation_system} % (fold)

We now summarize a technical presentation of the calculus that
embodies our theory of dynamics. The typical presentation of such a
calculus follows the style of giving generators and relations on
them. The grammar, below, describing term constructors, freely
generates the set of processes, $\Proc$. This set is then quotiented
by a relation known as structural congruence and it is over this set
that the notion of dynamics is expressed. This presentation is
essentially that of \cite{MeredithR05} with the addition of
polyadicity and summation. For readability we have relegated some of
the technical subtleties to an appendix.

\subsubsection{Process grammar}\label{subsub:process_grammar}

\begin{mathpar}
  \inferrule* [lab=synchronization] {} {{M} \bc \pzero \;|\; x?F \;|\; x!C }
  \and
  \inferrule* [lab=abstraction] {} {{F} \bc (x)P}
  \and
  \inferrule* [lab=concretion] {} {{C} \bc \langle Q \rangle}
  \and
  \inferrule* [lab=process] {} {{P,Q} \bc M \;| \;P|Q \;|\; @{x}}
  \and
  \inferrule* [lab=name] {} {{x} \bc \quotep{P}}
\end{mathpar} 

Note that $\vec{x}$ (resp. $\vec{P}$) denotes a vector of names
(resp. processes) of length $|\vec{x}|$ (resp. $|\vec{P}|$). We adopt
the following useful abbreviations.

\begin{mathpar}
   x?(\vec{y}).P := x.(\vec{y})P \and  x\clift{\vec{P}} := x.\clift{\vec{P}}
   \and x!(y) := \lift{x}{\dropn{y}}
   \and \Pi_{i=0}^{n-1}P_i := P_0 | \ldots | P_{n-1}
\end{mathpar}

\subsubsection{Structural congruence}

\paragraph{Free and bound names and alpha-equivalence.} At the
core of structural equivalence is alpha-equivalence which identifies
process that are the same up to a change of variable. Formally, we
recognize the distinction between free and bound names. The free names
of a process, $\freenames{P}$, may be calculated recursively as
follows:

\begin{mathpar}
\freenames{\pzero} := \emptyset
  \and \\
  \freenames{x?(y).P} := \{ x \} \cup (\freenames{P} \setminus \{ y \})
  \and 
  \freenames{x!\langle P \rangle} := \{ x \} \cup \{ P \} 
  \and \\
  \freenames{P|Q} := \freenames{P} \cup \freenames{Q}
  \and \\
  \freenames{@{x}} := \{ x \}
\end{mathpar}

$\pi$
$\quotep{\pi}$

$\freenames{-} : \pi \to \mathcal{P}(\quotep{\pi})$

\begin{eqnarray*}
  \freenames{\pzero} & := & \emptyset \\
  \freenames{x?(y).P} & := & \{ x \} \cup (\freenames{P} \setminus \{ y \}) \\
  \freenames{x!\langle P \rangle} & := & \{ x \} \cup \{ P \} \\
  \freenames{P|Q} & := & \freenames{P} \cup \freenames{Q} \\
  \freenames{\dropn{x}} & := & \{ x \}
\end{eqnarray*}

The bound names of a process, $\boundnames{P}$, are those names occurring in $P$
that are not free. For example, in $x?(y).0$, the name $x$ is free, while $y$ is bound.

\begin{mathpar}
  \inferrule* [lab=monoidal-laws] {} { P|Q \equiv Q|P \and P|0 \equiv P \and P|(Q|R) \equiv (P|Q)|R }
\end{mathpar}

\begin{mathpar}
  \inferrule* [lab=alpha-equivalence] {} { (x)P \equiv (y)P\{y/x\} \and y \not\in \freenames{P} }
\end{mathpar}

\begin{definition}
Then two processes, $P,Q$, are alpha-equivalent if $P = Q\{\vec{y}/\vec{x}\}$ for
some $\vec{x} \in \boundnames{Q},\vec{y} \in \boundnames{P}$, where $Q\{\vec{y}/\vec{x}\}$
denotes the capture-avoiding substitution of $\vec{y}$ for $\vec{x}$ in $Q$.
\end{definition}

\begin{definition}
  The {\em structural congruence} \cite{SangiorgiWalker} , $\equiv$,
  between processes is the least congruence containing
  alpha-equivalence, satisfying the abelian monoid laws
  (associativity, commutativity and $\pzero$ as identity) for parallel
  composition $|$ and for summation $+$.
\end{definition}

\subsection{Name equivalence}

We take name equivalence, written $\nameeq$, to be the smallest
equivalence relation generated by the following rules.

\begin{mathpar}
\inferrule*[lab=Quote-drop]
{ }
{ \quotep{@{x}} \nameeq x }

\inferrule*[lab=Struct-equiv]
{ P \scong Q }
{ \quotep{P} \nameeq \quotep{Q} }
\end{mathpar}

The astute reader will have noticed that the mutual recursion of names
and processes imposes a mutual recursion on alpha-equivalence and
structural equivalence via name-equivalence. Fortunately, all of this
works out pleasantly and we may calculate in the natural way, free of
concern. The reader interested in the details is referred to the
appendix \ref{appendix:rho_details}.

\subsection{Substitution}

We use $\Proc$ for the set of processes, $\QProc$ for the set of
names, and $\id{\{}\vec{y} / \vec{x} \id{\}}$ to denote partial maps,
$s : \QProc \rightarrow \QProc$. A map, $s$ lifts, uniquely, to a map
on process terms, $\widehat{s} : \Proc \rightarrow \Proc$ by the
following equations.

\begin{mathpar}
  (0) \psubstp{Q}{P} := 0 \\
  (R \juxtap S) \psubstp{Q}{P}
  :=    
  (R)\psubstp{Q}{P} \juxtap (S) \psubstp{Q}{P} \\
  (x?(y).R) \psubstp{Q}{P}    
  :=    
  (x)\substp{Q}{P} (z)\concat( (R \psubstn{z}{y}) \psubstp{Q}{P} ) \\
  (\lift{x}{R}) \psubstp{Q}{P}  
  :=
  \lift{(x)\substp{Q}{P}}{ R \psubstp{Q}{P} } \\
%   (\dropn{x})  \psubstp{Q}{P}       
%   := 
%   \left\{ 
%     \begin{array}{ccc} 
%       \dropn{\quotep{Q}} & & x \nameeq \quotep{P} \\
%       \dropn{x} & & otherwise \\
%     \end{array}
%   \right. 
  (\dropn{x})  \psubstp{Q}{P}       
  := 
  \left\{ 
    \begin{array}{ccc} 
      Q & & x \nameeq \quotep{P} \\
      \dropn{x} & & otherwise \\
    \end{array}
  \right.
\end{mathpar}
 

where

\begin{eqnarray}
  (x)\id{\{} \lpquote Q \rpquote / \lpquote P \rpquote \id{\}}            = 
  \left\{ 
    \begin{array}{ccc}
      \lpquote Q \rpquote & & x \nameeq \lpquote P \rpquote \\
      x & & otherwise \\
    \end{array}
  \right. \nonumber
\end{eqnarray}

and $z$ is chosen distinct from $\quotep{P}$, $\quotep{Q}$, the free
names in $Q$, and all the names in $R$. Our $\alpha$-equivalence will
be built in the standard way from this substitution.

\begin{remark}\label{rem:no_self_referential_names}
  One consequence of these definitions is that $\forall P. \quotep{P}
  \not\in \freenames{P}$.
\end{remark}

\subsection{ Dynamic quote: an example }

Anticipating something of what's to come, consider applying the
substitution, $\widehat{\id{\{}u / z \id{\}}}$, to the following pair
of processes, $\lift{w}{y!(z)}$ and $w[ \lpquote y!(z) \rpquote ]$.

\begin{eqnarray}
	\lift{w}{y!(z)}\widehat{\id{\{}u / z \id{\}}}
		& = &
		\lift{w}{y!(u)} \nonumber\\
	w[ \lpquote y!(z) \rpquote ] \widehat{ \id{\{}u / z \id{\}} }
		& = &
		w[ \lpquote y!(z) \rpquote ] \nonumber
\end{eqnarray}

Because the body of the process between quotes is impervious to
substitution, we get radically different answers. In fact, by
examining the first process in an input context,
e.g. $x?(z).\lift{w}{y!(z)}$, we see that the process under the lift
operator may be shaped by prefixed inputs binding a name inside it. In
this sense, the lift operator will be seen as a way to dynamically
construct processes before reifying them as names.

Finally equipped with these standard features we can present the
dynamics of the calculus.

\subsubsection{Operational semantics} 

Finally, we introduce the computational dynamics. What marks these
algebras as distinct from other more traditionally studied algebraic
structures, e.g. vector spaces or polynomial rings, is the manner in
which dynamics is captured. In traditional structures, dynamics is typically
expressed through morphisms between such structures, as in linear maps
between vector spaces or morphisms between rings. In algebras
associated with the semantics of computation, the dynamics is
expressed as part of the algebraic structure itself, through a
reduction reduction relation typically denoted by $\red$. Below, we
give a recursive presentation of this relation for the calculus used
in the encoding.

$\red \subseteq \pi \times \pi$
$\red : \pi \to \mathcal{P}(\pi)$

\begin{mathpar}
  \inferrule* [lab=Comm] { \textsf{match}( x_{src}, x_{trgt} ) } { x_{trgt}?(y)P \; | \; x_{src}!\langle {Q} \rangle \red P\{\quotep{Q}/y}\} }
  \and \\
  \inferrule* [lab=Par] {{P} \red {P}'} {{{P} | {Q}} \red {{P}' | {Q}}}
  \and
  \inferrule* [lab=Equiv]{{{P} \scong {P}'} \andalso {{P}' \red {Q}'} \andalso {{Q}' \scong {Q}}}{{P} \red {Q}}
\end{mathpar}

\begin{eqnarray*}
  match_{\equiv} (\quotep{P},\quotep{Q}) & := & P \equiv Q \\
  match_{\dagger}(\quotep{P},\quotep{Q}) & := & \forall R. P|Q \red^{*} R => R \red^{*} 0 \\
  match_{K}(\quotep{P},\quotep{Q}) & := & K \mbox{ for some context } K
\end{eqnarray*}

$u?(x)P | u!\langle Q \rangle \red P\{\quotep{Q}/x\}$

%We write $\wred$ for $\red^*$, and $P\red$ if $\exists Q $ such that $ P \red Q$.
We write $P\red$ if $\exists Q $ such that $ P \red Q$ and $P\not\red$, otherwise.

\section{Replication}

As mentioned before, it is known that replication (and hence
recursion) can be implemented in a higher-order process algebra
\cite{SangiorgiWalker}. As our first example of calculation with the
machinery thus far presented we give the construction explicitly in
the {\rhoc}.

\begin{eqnarray}
	D_{x} & := & \prefix{x}{y}{(\binpar{\outputp{x}{y}}{@{y}})} \nonumber\\
	\bangp_{x}{P} & := & \binpar{{x}!\langle{\binpar{D_{x}}{P}}\rangle}{D_{x}} \nonumber
\end{eqnarray}

\begin{eqnarray}
	\bangp_{x}{P} & & \nonumber\\
	=
	& {x}!\langle{(\prefix{x}{y}{(\outputp{x}{y} | @{y})) | P}}\rangle 
	      | \prefix{x}{y}{(\outputp{x}{y} | @{y})} & \nonumber\\
	\red
	& (\outputp{x}{y} | @{y})\substn{\quotep{(\prefix{x}{y}{(@{y} | \outputp{x}{y})) | P}}}{y} & \nonumber\\
	=
	& \outputp{x}{\quotep{(\prefix{x}{y}{(\outputp{x}{y} | @{y})) | P}}}
	  | {(\prefix{x}{y}{(\outputp{x}{y} | @{y})) | P}} & \nonumber\\
	\red
	& \ldots & \nonumber\\
	\red^*
	& P | P | \ldots & \nonumber
\end{eqnarray}

Of course, this encoding, as an implementation, runs away, unfolding
$\bangp{P}$ eagerly. A lazier and more implementable replication
operator, restricted to input-guarded processes, may be obtained as follows.

\begin{eqnarray}
\bangp{\prefix{u}{v}{P}} 
	:= 
	\binpar{\lift{x}{\prefix{u}{v}{(\binpar{D(x)}{P})}}}{D(x)} \nonumber
\end{eqnarray}

\begin{remark}
  Note that the lazier definition still does not deal with summation
  or mixed summation (i.e. sums over input and output). The reader is
  invited to construct definitions of replication that deal with these
  features. 

  Further, the definitions are parameterized in a name, $x$. Can you,
  gentle reader, make a definition that eliminates this parameter and
  guarantees no accidental interaction between the replication
  machinery and the process being replicated -- i.e. no accidental
  sharing of names used by the process to get its work done and the
  name(s) used by the replication to effect copying. This latter
  revision of the definition of replication is crucial to obtaining
  the expected identity $!!P \sim !P$.
\end{remark}

\begin{remark}\label{rem:paradoxical_combinator}
  The reader familiar with the lambda calculus will have noticed the
  similarity between $D$ and the paradoxical combinator.

  [Ed. note: the existence of this seems to suggest we have to be more
  restrictive on the set of processes and names we admit if we are to
  support no-cloning.]
\end{remark}

\subsubsection{Bisimulation}

The computational dynamics gives rise to another kind of equivalence,
the equivalence of computational behavior. As previously mentioned
this is typically captured \emph{via} some form of bisimulation.

% The notion we use in this paper is weak barbed bisimulation
% \cite{milner91polyadicpi}.

The notion we use in this paper is derived from weak barbed
bisimulation \cite{milner91polyadicpi}. 

\begin{definition}
An \emph{observation relation}, $\downarrow_{\mathcal N}$, over a set
of names, $\mathcal N$, is the smallest relation satisfying the rules
below.

\infrule[Out-barb]{y \in {\mathcal N}, \; x \nameeq y}
		  {\outputp{x}{v} \downarrow_{\mathcal N} x}
\infrule[Par-barb]{\mbox{$P\downarrow_{\mathcal N} x$ or $Q\downarrow_{\mathcal N} x$}}
		  {\binpar{P}{Q} \downarrow_{\mathcal N} x}

We write $P \Downarrow_{\mathcal N} x$ if there is $Q$ such that 
$P \wred Q$ and $Q \downarrow_{\mathcal N} x$.
\end{definition}

\begin{definition}
%\label{def.bbisim}
An  ${\mathcal N}$-\emph{barbed bisimulation} over a set of names, ${\mathcal N}$, is a symmetric binary relation 
${\mathcal S}_{\mathcal N}$ between agents such that $P\rel{S}_{\mathcal N}Q$ implies:
\begin{enumerate}
\item If $P \red P'$ then $Q \wred Q'$ and $P'\rel{S}_{\mathcal N} Q'$.
\item If $P\downarrow_{\mathcal N} x$, then $Q\Downarrow_{\mathcal N} x$.
\end{enumerate}
$P$ is ${\mathcal N}$-barbed bisimilar to $Q$, written
$P \wbbisim_{\mathcal N} Q$, if $P \rel{S}_{\mathcal N} Q$ for some ${\mathcal N}$-barbed bisimulation ${\mathcal S}_{\mathcal N}$.
\end{definition}

$\mathcal{R} \subseteq \pi \times \pi$

$P \mathcal{R} Q => \forall P'. P \red P' \Rightarrow \exists Q'. Q \red Q', P' \mathcal{R} Q'$

$P \vdash x \Rightarrow Q \vdash x$

\begin{mathpar}
  \inferrule*[lab=Out-barb]{x \nameeq y}{{y}!\langle{Q}\rangle \vdash x}
  \and
  \inferrule*[lab=Par-barb]{\mbox{$P\vdash x$ or $Q\vdash x$}}{\binpar{P}{Q} \vdash x}
\end{mathpar}

\subsubsection{Contexts}

One of the principle advantages of computational calculi like the
$\pi$-calculus is a well-defined notion of context,
contextual-equivalence and a correlation between
contextual-equivalence and notions of bisimulation. The notion of
context allows the decomposition of a process into (sub-)process and
its syntactic environment, its context. Thus, a context may be
thought of as a process with a ``hole'' (written $\Box$) in it. The
application of a context $M$ to a process $P$, written $M[P]$, is
tantamount to filling the hole in $M$ with $P$. In this paper we do
not need the full weight of this theory, but do make use of the notion
of context in the proof the main theorem. 

\begin{mathpar}
  \inferrule* [lab=summation] {} {{M_{M},M_{N}} \bc \Box \;|\; x.M_{A} \;|\; M_{M}+M_{N}}
  \and
  \inferrule* [lab=agent] {} {{M_{A}} \bc (\vec{x})M_{P} \;| \; \clift{P_0,\ldots,M_{P},\ldots,P_N}}
  \and \\
  \inferrule* [lab=process] {} {{M_{P}} \bc M_{N} \;| \;P|M_{P} }
\end{mathpar} 

\begin{mathpar}
  \inferrule* [lab=sychronization] {} {M_{N} \bc \Box \;|\; x?M_{F} \;|\; x!M_{C}}
  \and
  \inferrule* [lab=abstraction] {} {{M_{F}} \bc (x)M_{P} }
  \and
  \inferrule* [lab=concretion] {} {{M_{C}} \bc \langle M_{P} \rangle }
  \and \\
  \inferrule* [lab=process] {} {{M_{P}} \bc M_{N} \;| \;P|M_{P} }
\end{mathpar}

\begin{definition}[contextual application] Given a context $M$, and
  process $P$, we define the \emph{contextual application}, $M[P] :=
  M\{P/\Box\}$. That is, the contextual application of M to P is the
  substitution of $P$ for $\Box$ in $M$.
\end{definition}

$\meaningof{-} : L \to \mathcal{P}(\pi)$

\begin{mathpar}
  \inferrule* [lab=collection] {} {\meaningof{true} = \pi, \and \meaningof{~E} = \pi \setminus \meaningof{E}, \and \meaningof{E_{1} \& E_{2}} = \meaningof{E_{1}} \cap \meaningof{E_{2}}}
\end{mathpar}

\begin{mathpar}
  \inferrule* [lab=structure] {} {\meaningof{0} = \{ P \in \pi | P \equiv 0 \}, \and \\ \meaningof{E_1 | E_2} = \{ P \in \pi | P \equiv P_{1} | P_{2}, P_{1} \in \meaningof{E_{1}}, P_{2} \in \meaningof{E_2}\} }
\end{mathpar}

\begin{mathpar}
 \inferrule* [lab=behavior] {} {\meaningof{\langle a?b \rangle E} = \{ P \in \pi | P \equiv Q | u?(y)P', \\ \and \\\\ \and \\ \;\;\; u \in \meaningof{a}, \forall z.P'\{z/y\} \in \meaningof{E\{z/b\}}\}, \and \\ \meaningof{a!E} = \{ P \in \pi | P \equiv Q | x!\langle P' \rangle, x \in \meaningof{a} P' \in \meaningof{E}\} }
\end{mathpar}

\begin{mathpar}
 \inferrule* [lab=nominal] {} {\meaningof{\quotep{E}} = \{ \quotep{P} \in \quotep{\pi} | P \in \meaningof{E} \}, \and \meaningof{\quotep{P}} = \{ \quotep{Q} \in \quotep{\pi} | P \equiv Q \} \and \\ \meaningof{@\quotep{E}} = \{ P \in \pi | P \equiv @x, x \in \meaningof{E} \}}
\end{mathpar}

\begin{eqnarray*}
  \\
  \meaningof{-} : TS \to ST
\end{eqnarray*}

\begin{eqnarray*}
  \\
  L : TS \to ST
\end{eqnarray*}

\begin{eqnarray*}
  \\
  P \models E \iff P \in \meaningof{E}
\end{eqnarray*}

\begin{eqnarray*}
  P \approx_{L} Q \iff \forall E \in L. P \models E \iff Q \models E
\end{eqnarray*}

\begin{eqnarray*}
  P \approx_{K} Q
\end{eqnarray*}

\begin{eqnarray*}
  P \approx Q
\end{eqnarray*}

$\approx_{K} = \approx = \approx_{L}$

\subsubsection{Contextual duality}

Note that contexts extend the quotation operation to a family of
operations from processes to names. Given a context, $M$, we can
define a \emph{nominal context}, $\quotep{M}$ by $\quotep{M}[P] :=
\quotep{M[P]}$. To foreshadow what is to come we observe that these
operations enjoy a duality with processes very much like the duality
between vectors and maps from vectors to scalars.

Further, because the calculus is essentially higher-order, we have a
correspondence between contexts and processes. More specifically,
given a name $x$ and a context $M$ we can construct $M^{*}_{x}$ such
that 

\begin{mathpar}
  M^{*}_{x} | \lift{x}{P} \red M[P]
\end{mathpar}

namely,

\begin{mathpar}
  M^{*}_{x} := x?(u).M[\dropn{u}]
\end{mathpar}

The dependence of $M^{*}_{x}$ on a name makes it an abstraction, 

\begin{mathpar}
  M^{*} := (x)x?(u).M[\dropn{u}]
\end{mathpar}

\subsection{Additional notation}

It will sometimes be convenient to denote the process a name
quotes. We already have the notation $x = \quotep{P}$, but it will be
convenient to introduce an alternate notation, $\procn{x}$, when we
want to emphasize the connection to the use of the name. Note that, by
virtue of name equivalence, $\quotep{\procn{x}} \nameeq x$; so, the
notation is consistent with previous definitions.

Further, because names have structure it is possible to effect
substitutions on the basis of that structure. This means we need to
upgrade our notation for substitutions, which we accomplish by
adapting comprehension notation. Thus,

\begin{mathpar}
  P\{ y / x : x \in S \}
\end{mathpar}

is interpreted to mean the process derived from P by replacing (in a
capture-avoiding manner) each occurrence of $x$ in $S$ by $y$. For example,

\begin{mathpar}
  P\{ \quotep{\procn{x}|\procn{x}} / x : x \in \freenames{P} \}
\end{mathpar}

will replace each (occurrence) of a free name $x$ in $P$ by
$\quotep{\procn{x}|\procn{x}}$.

Also, we will avail ourselves of the notation $x^{L}$ and $x^{R}$ to
denote injections of a name into disjoint copies of the name
space. There are numerous ways to accomplish this. One example can be
found in \cite{MeredithR05}. This notation overloads to vectors of
names: $\vec{x}^{\pi} := (x_{i}^{\pi} \; : \; 0 \leq i < |\vec{x}| )$ where $\pi \in \{L,R\}$.

We also use $P^{\Box} := P|\Box$.

In \cite{MeredithR05} an interpretation of the new operator is
given. It turns out that there are several possible interpretations
all enjoying the requisite algebraic properties of the operator (see
\cite{milner91polyadicpi}). We will therefore make liberal use of
$(\nu\; \vec{x})P$.

% subsection the_syntax_and_semantics_of_the_notation_system (end)   

\input{qm2pi.qmops} 

\input{qm2pi.sterngerlach} 

\input{qm2pi.metric} 

% section concurrent_process_calculi (end)

%\input{qm2pi.proofsketch}

% section proof sketch (end)

%\input{qm2pi.slviaknots} 

% section spatial logic via knots (end)

\input{qm2pi.conclusion}

% section conclusion (end)

%\input{qm2pi.dtcodes} 

% section wiring algorithm (end)

\input{qm2pi.ack} 

% section acknowledgments (end)

\newpage


\bibliographystyle{plain}   
\bibliography{../../biblios/main.bib}

\input{qm2pi.rhodetails}

\end{document}

 

% section concurrent_process_calculi (end)

%\documentclass[12pt]{llncs}
%\documentclass{jktr}

\usepackage[pdftex]{hyperref}                   
\usepackage {listings}
\usepackage {mathpartir}
\usepackage{bcprules}
%\usepackage{listings}
                       
\usepackage{graphicx} 
%\usepackage[margins=2.5cm,nohead,nofoot]{geometry}
%\usepackage{geometry}
\usepackage{amsfonts}
\usepackage{amstext}
\usepackage{latexsym}
\usepackage{amssymb}
\usepackage{color}


%\include{myPreamble}
\include{qm2pi.local} 

%\ifpdf
%\usepackage[pdftex]{graphicx}
%\else
%\usepackage{graphicx}
%\fi

 % \ifpdf
%  \usepackage{pdfsync}
%  \if


%\title{Brief Article}
%\author{David F. Snyder}
%\author{L.G. Meredith}

%\address{Dept. of Math., Texas State University--San Marcos, San Marcos, TX 78666}
       
\pagestyle{empty}


\begin{document}

\lstset{language=[Objective]Caml,frame=shadowbox}

\input{qm2pi.front}

% section front matter (end)

\input{qm2pi.intro} 
 
% section introduction (end)

% \input{qm2pi.knotations} 

% section notation (end)

\input{qm2pi.process.calculi} 

% section concurrent_process_calculi_and_spatial_logics_ (end)
    
%\input{qm2pi.knots2pi} 

%\input{qm2pi.trefoil} 

%\input{qm2pi.mainthm} 

% subsection basic_interpretation (end)

%\input{qm2pi.rho.presentation} 
\subsection{The syntax and semantics of the notation system}\label{sub:the_syntax_and_semantics_of_the_notation_system} % (fold)

We now summarize a technical presentation of the calculus that
embodies our theory of dynamics. The typical presentation of such a
calculus follows the style of giving generators and relations on
them. The grammar, below, describing term constructors, freely
generates the set of processes, $\Proc$. This set is then quotiented
by a relation known as structural congruence and it is over this set
that the notion of dynamics is expressed. This presentation is
essentially that of \cite{MeredithR05} with the addition of
polyadicity and summation. For readability we have relegated some of
the technical subtleties to an appendix.

\subsubsection{Process grammar}\label{subsub:process_grammar}

\begin{mathpar}
  \inferrule* [lab=synchronization] {} {{M} \bc \pzero \;|\; x?F \;|\; x!C }
  \and
  \inferrule* [lab=abstraction] {} {{F} \bc (x)P}
  \and
  \inferrule* [lab=concretion] {} {{C} \bc \langle Q \rangle}
  \and
  \inferrule* [lab=process] {} {{P,Q} \bc M \;| \;P|Q \;|\; @{x}}
  \and
  \inferrule* [lab=name] {} {{x} \bc \quotep{P}}
\end{mathpar} 

Note that $\vec{x}$ (resp. $\vec{P}$) denotes a vector of names
(resp. processes) of length $|\vec{x}|$ (resp. $|\vec{P}|$). We adopt
the following useful abbreviations.

\begin{mathpar}
   x?(\vec{y}).P := x.(\vec{y})P \and  x\clift{\vec{P}} := x.\clift{\vec{P}}
   \and x!(y) := \lift{x}{\dropn{y}}
   \and \Pi_{i=0}^{n-1}P_i := P_0 | \ldots | P_{n-1}
\end{mathpar}

\subsubsection{Structural congruence}

\paragraph{Free and bound names and alpha-equivalence.} At the
core of structural equivalence is alpha-equivalence which identifies
process that are the same up to a change of variable. Formally, we
recognize the distinction between free and bound names. The free names
of a process, $\freenames{P}$, may be calculated recursively as
follows:

\begin{mathpar}
\freenames{\pzero} := \emptyset
  \and \\
  \freenames{x?(y).P} := \{ x \} \cup (\freenames{P} \setminus \{ y \})
  \and 
  \freenames{x!\langle P \rangle} := \{ x \} \cup \{ P \} 
  \and \\
  \freenames{P|Q} := \freenames{P} \cup \freenames{Q}
  \and \\
  \freenames{@{x}} := \{ x \}
\end{mathpar}

$\pi$
$\quotep{\pi}$

$\freenames{-} : \pi \to \mathcal{P}(\quotep{\pi})$

\begin{eqnarray*}
  \freenames{\pzero} & := & \emptyset \\
  \freenames{x?(y).P} & := & \{ x \} \cup (\freenames{P} \setminus \{ y \}) \\
  \freenames{x!\langle P \rangle} & := & \{ x \} \cup \{ P \} \\
  \freenames{P|Q} & := & \freenames{P} \cup \freenames{Q} \\
  \freenames{\dropn{x}} & := & \{ x \}
\end{eqnarray*}

The bound names of a process, $\boundnames{P}$, are those names occurring in $P$
that are not free. For example, in $x?(y).0$, the name $x$ is free, while $y$ is bound.

\begin{mathpar}
  \inferrule* [lab=monoidal-laws] {} { P|Q \equiv Q|P \and P|0 \equiv P \and P|(Q|R) \equiv (P|Q)|R }
\end{mathpar}

\begin{mathpar}
  \inferrule* [lab=alpha-equivalence] {} { (x)P \equiv (y)P\{y/x\} \and y \not\in \freenames{P} }
\end{mathpar}

\begin{definition}
Then two processes, $P,Q$, are alpha-equivalent if $P = Q\{\vec{y}/\vec{x}\}$ for
some $\vec{x} \in \boundnames{Q},\vec{y} \in \boundnames{P}$, where $Q\{\vec{y}/\vec{x}\}$
denotes the capture-avoiding substitution of $\vec{y}$ for $\vec{x}$ in $Q$.
\end{definition}

\begin{definition}
  The {\em structural congruence} \cite{SangiorgiWalker} , $\equiv$,
  between processes is the least congruence containing
  alpha-equivalence, satisfying the abelian monoid laws
  (associativity, commutativity and $\pzero$ as identity) for parallel
  composition $|$ and for summation $+$.
\end{definition}

\subsection{Name equivalence}

We take name equivalence, written $\nameeq$, to be the smallest
equivalence relation generated by the following rules.

\begin{mathpar}
\inferrule*[lab=Quote-drop]
{ }
{ \quotep{@{x}} \nameeq x }

\inferrule*[lab=Struct-equiv]
{ P \scong Q }
{ \quotep{P} \nameeq \quotep{Q} }
\end{mathpar}

The astute reader will have noticed that the mutual recursion of names
and processes imposes a mutual recursion on alpha-equivalence and
structural equivalence via name-equivalence. Fortunately, all of this
works out pleasantly and we may calculate in the natural way, free of
concern. The reader interested in the details is referred to the
appendix \ref{appendix:rho_details}.

\subsection{Substitution}

We use $\Proc$ for the set of processes, $\QProc$ for the set of
names, and $\id{\{}\vec{y} / \vec{x} \id{\}}$ to denote partial maps,
$s : \QProc \rightarrow \QProc$. A map, $s$ lifts, uniquely, to a map
on process terms, $\widehat{s} : \Proc \rightarrow \Proc$ by the
following equations.

\begin{mathpar}
  (0) \psubstp{Q}{P} := 0 \\
  (R \juxtap S) \psubstp{Q}{P}
  :=    
  (R)\psubstp{Q}{P} \juxtap (S) \psubstp{Q}{P} \\
  (x?(y).R) \psubstp{Q}{P}    
  :=    
  (x)\substp{Q}{P} (z)\concat( (R \psubstn{z}{y}) \psubstp{Q}{P} ) \\
  (\lift{x}{R}) \psubstp{Q}{P}  
  :=
  \lift{(x)\substp{Q}{P}}{ R \psubstp{Q}{P} } \\
%   (\dropn{x})  \psubstp{Q}{P}       
%   := 
%   \left\{ 
%     \begin{array}{ccc} 
%       \dropn{\quotep{Q}} & & x \nameeq \quotep{P} \\
%       \dropn{x} & & otherwise \\
%     \end{array}
%   \right. 
  (\dropn{x})  \psubstp{Q}{P}       
  := 
  \left\{ 
    \begin{array}{ccc} 
      Q & & x \nameeq \quotep{P} \\
      \dropn{x} & & otherwise \\
    \end{array}
  \right.
\end{mathpar}
 

where

\begin{eqnarray}
  (x)\id{\{} \lpquote Q \rpquote / \lpquote P \rpquote \id{\}}            = 
  \left\{ 
    \begin{array}{ccc}
      \lpquote Q \rpquote & & x \nameeq \lpquote P \rpquote \\
      x & & otherwise \\
    \end{array}
  \right. \nonumber
\end{eqnarray}

and $z$ is chosen distinct from $\quotep{P}$, $\quotep{Q}$, the free
names in $Q$, and all the names in $R$. Our $\alpha$-equivalence will
be built in the standard way from this substitution.

\begin{remark}\label{rem:no_self_referential_names}
  One consequence of these definitions is that $\forall P. \quotep{P}
  \not\in \freenames{P}$.
\end{remark}

\subsection{ Dynamic quote: an example }

Anticipating something of what's to come, consider applying the
substitution, $\widehat{\id{\{}u / z \id{\}}}$, to the following pair
of processes, $\lift{w}{y!(z)}$ and $w[ \lpquote y!(z) \rpquote ]$.

\begin{eqnarray}
	\lift{w}{y!(z)}\widehat{\id{\{}u / z \id{\}}}
		& = &
		\lift{w}{y!(u)} \nonumber\\
	w[ \lpquote y!(z) \rpquote ] \widehat{ \id{\{}u / z \id{\}} }
		& = &
		w[ \lpquote y!(z) \rpquote ] \nonumber
\end{eqnarray}

Because the body of the process between quotes is impervious to
substitution, we get radically different answers. In fact, by
examining the first process in an input context,
e.g. $x?(z).\lift{w}{y!(z)}$, we see that the process under the lift
operator may be shaped by prefixed inputs binding a name inside it. In
this sense, the lift operator will be seen as a way to dynamically
construct processes before reifying them as names.

Finally equipped with these standard features we can present the
dynamics of the calculus.

\subsubsection{Operational semantics} 

Finally, we introduce the computational dynamics. What marks these
algebras as distinct from other more traditionally studied algebraic
structures, e.g. vector spaces or polynomial rings, is the manner in
which dynamics is captured. In traditional structures, dynamics is typically
expressed through morphisms between such structures, as in linear maps
between vector spaces or morphisms between rings. In algebras
associated with the semantics of computation, the dynamics is
expressed as part of the algebraic structure itself, through a
reduction reduction relation typically denoted by $\red$. Below, we
give a recursive presentation of this relation for the calculus used
in the encoding.

$\red \subseteq \pi \times \pi$
$\red : \pi \to \mathcal{P}(\pi)$

\begin{mathpar}
  \inferrule* [lab=Comm] { \textsf{match}( x_{src}, x_{trgt} ) } { x_{trgt}?(y)P \; | \; x_{src}!\langle {Q} \rangle \red P\{\quotep{Q}/y}\} }
  \and \\
  \inferrule* [lab=Par] {{P} \red {P}'} {{{P} | {Q}} \red {{P}' | {Q}}}
  \and
  \inferrule* [lab=Equiv]{{{P} \scong {P}'} \andalso {{P}' \red {Q}'} \andalso {{Q}' \scong {Q}}}{{P} \red {Q}}
\end{mathpar}

\begin{eqnarray*}
  match_{\equiv} (\quotep{P},\quotep{Q}) & := & P \equiv Q \\
  match_{\dagger}(\quotep{P},\quotep{Q}) & := & \forall R. P|Q \red^{*} R => R \red^{*} 0 \\
  match_{K}(\quotep{P},\quotep{Q}) & := & K \mbox{ for some context } K
\end{eqnarray*}

$u?(x)P | u!\langle Q \rangle \red P\{\quotep{Q}/x\}$

%We write $\wred$ for $\red^*$, and $P\red$ if $\exists Q $ such that $ P \red Q$.
We write $P\red$ if $\exists Q $ such that $ P \red Q$ and $P\not\red$, otherwise.

\section{Replication}

As mentioned before, it is known that replication (and hence
recursion) can be implemented in a higher-order process algebra
\cite{SangiorgiWalker}. As our first example of calculation with the
machinery thus far presented we give the construction explicitly in
the {\rhoc}.

\begin{eqnarray}
	D_{x} & := & \prefix{x}{y}{(\binpar{\outputp{x}{y}}{@{y}})} \nonumber\\
	\bangp_{x}{P} & := & \binpar{{x}!\langle{\binpar{D_{x}}{P}}\rangle}{D_{x}} \nonumber
\end{eqnarray}

\begin{eqnarray}
	\bangp_{x}{P} & & \nonumber\\
	=
	& {x}!\langle{(\prefix{x}{y}{(\outputp{x}{y} | @{y})) | P}}\rangle 
	      | \prefix{x}{y}{(\outputp{x}{y} | @{y})} & \nonumber\\
	\red
	& (\outputp{x}{y} | @{y})\substn{\quotep{(\prefix{x}{y}{(@{y} | \outputp{x}{y})) | P}}}{y} & \nonumber\\
	=
	& \outputp{x}{\quotep{(\prefix{x}{y}{(\outputp{x}{y} | @{y})) | P}}}
	  | {(\prefix{x}{y}{(\outputp{x}{y} | @{y})) | P}} & \nonumber\\
	\red
	& \ldots & \nonumber\\
	\red^*
	& P | P | \ldots & \nonumber
\end{eqnarray}

Of course, this encoding, as an implementation, runs away, unfolding
$\bangp{P}$ eagerly. A lazier and more implementable replication
operator, restricted to input-guarded processes, may be obtained as follows.

\begin{eqnarray}
\bangp{\prefix{u}{v}{P}} 
	:= 
	\binpar{\lift{x}{\prefix{u}{v}{(\binpar{D(x)}{P})}}}{D(x)} \nonumber
\end{eqnarray}

\begin{remark}
  Note that the lazier definition still does not deal with summation
  or mixed summation (i.e. sums over input and output). The reader is
  invited to construct definitions of replication that deal with these
  features. 

  Further, the definitions are parameterized in a name, $x$. Can you,
  gentle reader, make a definition that eliminates this parameter and
  guarantees no accidental interaction between the replication
  machinery and the process being replicated -- i.e. no accidental
  sharing of names used by the process to get its work done and the
  name(s) used by the replication to effect copying. This latter
  revision of the definition of replication is crucial to obtaining
  the expected identity $!!P \sim !P$.
\end{remark}

\begin{remark}\label{rem:paradoxical_combinator}
  The reader familiar with the lambda calculus will have noticed the
  similarity between $D$ and the paradoxical combinator.

  [Ed. note: the existence of this seems to suggest we have to be more
  restrictive on the set of processes and names we admit if we are to
  support no-cloning.]
\end{remark}

\subsubsection{Bisimulation}

The computational dynamics gives rise to another kind of equivalence,
the equivalence of computational behavior. As previously mentioned
this is typically captured \emph{via} some form of bisimulation.

% The notion we use in this paper is weak barbed bisimulation
% \cite{milner91polyadicpi}.

The notion we use in this paper is derived from weak barbed
bisimulation \cite{milner91polyadicpi}. 

\begin{definition}
An \emph{observation relation}, $\downarrow_{\mathcal N}$, over a set
of names, $\mathcal N$, is the smallest relation satisfying the rules
below.

\infrule[Out-barb]{y \in {\mathcal N}, \; x \nameeq y}
		  {\outputp{x}{v} \downarrow_{\mathcal N} x}
\infrule[Par-barb]{\mbox{$P\downarrow_{\mathcal N} x$ or $Q\downarrow_{\mathcal N} x$}}
		  {\binpar{P}{Q} \downarrow_{\mathcal N} x}

We write $P \Downarrow_{\mathcal N} x$ if there is $Q$ such that 
$P \wred Q$ and $Q \downarrow_{\mathcal N} x$.
\end{definition}

\begin{definition}
%\label{def.bbisim}
An  ${\mathcal N}$-\emph{barbed bisimulation} over a set of names, ${\mathcal N}$, is a symmetric binary relation 
${\mathcal S}_{\mathcal N}$ between agents such that $P\rel{S}_{\mathcal N}Q$ implies:
\begin{enumerate}
\item If $P \red P'$ then $Q \wred Q'$ and $P'\rel{S}_{\mathcal N} Q'$.
\item If $P\downarrow_{\mathcal N} x$, then $Q\Downarrow_{\mathcal N} x$.
\end{enumerate}
$P$ is ${\mathcal N}$-barbed bisimilar to $Q$, written
$P \wbbisim_{\mathcal N} Q$, if $P \rel{S}_{\mathcal N} Q$ for some ${\mathcal N}$-barbed bisimulation ${\mathcal S}_{\mathcal N}$.
\end{definition}

$\mathcal{R} \subseteq \pi \times \pi$

$P \mathcal{R} Q => \forall P'. P \red P' \Rightarrow \exists Q'. Q \red Q', P' \mathcal{R} Q'$

$P \vdash x \Rightarrow Q \vdash x$

\begin{mathpar}
  \inferrule*[lab=Out-barb]{x \nameeq y}{{y}!\langle{Q}\rangle \vdash x}
  \and
  \inferrule*[lab=Par-barb]{\mbox{$P\vdash x$ or $Q\vdash x$}}{\binpar{P}{Q} \vdash x}
\end{mathpar}

\subsubsection{Contexts}

One of the principle advantages of computational calculi like the
$\pi$-calculus is a well-defined notion of context,
contextual-equivalence and a correlation between
contextual-equivalence and notions of bisimulation. The notion of
context allows the decomposition of a process into (sub-)process and
its syntactic environment, its context. Thus, a context may be
thought of as a process with a ``hole'' (written $\Box$) in it. The
application of a context $M$ to a process $P$, written $M[P]$, is
tantamount to filling the hole in $M$ with $P$. In this paper we do
not need the full weight of this theory, but do make use of the notion
of context in the proof the main theorem. 

\begin{mathpar}
  \inferrule* [lab=summation] {} {{M_{M},M_{N}} \bc \Box \;|\; x.M_{A} \;|\; M_{M}+M_{N}}
  \and
  \inferrule* [lab=agent] {} {{M_{A}} \bc (\vec{x})M_{P} \;| \; \clift{P_0,\ldots,M_{P},\ldots,P_N}}
  \and \\
  \inferrule* [lab=process] {} {{M_{P}} \bc M_{N} \;| \;P|M_{P} }
\end{mathpar} 

\begin{mathpar}
  \inferrule* [lab=sychronization] {} {M_{N} \bc \Box \;|\; x?M_{F} \;|\; x!M_{C}}
  \and
  \inferrule* [lab=abstraction] {} {{M_{F}} \bc (x)M_{P} }
  \and
  \inferrule* [lab=concretion] {} {{M_{C}} \bc \langle M_{P} \rangle }
  \and \\
  \inferrule* [lab=process] {} {{M_{P}} \bc M_{N} \;| \;P|M_{P} }
\end{mathpar}

\begin{definition}[contextual application] Given a context $M$, and
  process $P$, we define the \emph{contextual application}, $M[P] :=
  M\{P/\Box\}$. That is, the contextual application of M to P is the
  substitution of $P$ for $\Box$ in $M$.
\end{definition}

$\meaningof{-} : L \to \mathcal{P}(\pi)$

\begin{mathpar}
  \inferrule* [lab=collection] {} {\meaningof{true} = \pi, \and \meaningof{~E} = \pi \setminus \meaningof{E}, \and \meaningof{E_{1} \& E_{2}} = \meaningof{E_{1}} \cap \meaningof{E_{2}}}
\end{mathpar}

\begin{mathpar}
  \inferrule* [lab=structure] {} {\meaningof{0} = \{ P \in \pi | P \equiv 0 \}, \and \\ \meaningof{E_1 | E_2} = \{ P \in \pi | P \equiv P_{1} | P_{2}, P_{1} \in \meaningof{E_{1}}, P_{2} \in \meaningof{E_2}\} }
\end{mathpar}

\begin{mathpar}
 \inferrule* [lab=behavior] {} {\meaningof{\langle a?b \rangle E} = \{ P \in \pi | P \equiv Q | u?(y)P', \\ \and \\\\ \and \\ \;\;\; u \in \meaningof{a}, \forall z.P'\{z/y\} \in \meaningof{E\{z/b\}}\}, \and \\ \meaningof{a!E} = \{ P \in \pi | P \equiv Q | x!\langle P' \rangle, x \in \meaningof{a} P' \in \meaningof{E}\} }
\end{mathpar}

\begin{mathpar}
 \inferrule* [lab=nominal] {} {\meaningof{\quotep{E}} = \{ \quotep{P} \in \quotep{\pi} | P \in \meaningof{E} \}, \and \meaningof{\quotep{P}} = \{ \quotep{Q} \in \quotep{\pi} | P \equiv Q \} \and \\ \meaningof{@\quotep{E}} = \{ P \in \pi | P \equiv @x, x \in \meaningof{E} \}}
\end{mathpar}

\begin{eqnarray*}
  \\
  \meaningof{-} : TS \to ST
\end{eqnarray*}

\begin{eqnarray*}
  \\
  L : TS \to ST
\end{eqnarray*}

\begin{eqnarray*}
  \\
  P \models E \iff P \in \meaningof{E}
\end{eqnarray*}

\begin{eqnarray*}
  P \approx_{L} Q \iff \forall E \in L. P \models E \iff Q \models E
\end{eqnarray*}

\begin{eqnarray*}
  P \approx_{K} Q
\end{eqnarray*}

\begin{eqnarray*}
  P \approx Q
\end{eqnarray*}

$\approx_{K} = \approx = \approx_{L}$

\subsubsection{Contextual duality}

Note that contexts extend the quotation operation to a family of
operations from processes to names. Given a context, $M$, we can
define a \emph{nominal context}, $\quotep{M}$ by $\quotep{M}[P] :=
\quotep{M[P]}$. To foreshadow what is to come we observe that these
operations enjoy a duality with processes very much like the duality
between vectors and maps from vectors to scalars.

Further, because the calculus is essentially higher-order, we have a
correspondence between contexts and processes. More specifically,
given a name $x$ and a context $M$ we can construct $M^{*}_{x}$ such
that 

\begin{mathpar}
  M^{*}_{x} | \lift{x}{P} \red M[P]
\end{mathpar}

namely,

\begin{mathpar}
  M^{*}_{x} := x?(u).M[\dropn{u}]
\end{mathpar}

The dependence of $M^{*}_{x}$ on a name makes it an abstraction, 

\begin{mathpar}
  M^{*} := (x)x?(u).M[\dropn{u}]
\end{mathpar}

\subsection{Additional notation}

It will sometimes be convenient to denote the process a name
quotes. We already have the notation $x = \quotep{P}$, but it will be
convenient to introduce an alternate notation, $\procn{x}$, when we
want to emphasize the connection to the use of the name. Note that, by
virtue of name equivalence, $\quotep{\procn{x}} \nameeq x$; so, the
notation is consistent with previous definitions.

Further, because names have structure it is possible to effect
substitutions on the basis of that structure. This means we need to
upgrade our notation for substitutions, which we accomplish by
adapting comprehension notation. Thus,

\begin{mathpar}
  P\{ y / x : x \in S \}
\end{mathpar}

is interpreted to mean the process derived from P by replacing (in a
capture-avoiding manner) each occurrence of $x$ in $S$ by $y$. For example,

\begin{mathpar}
  P\{ \quotep{\procn{x}|\procn{x}} / x : x \in \freenames{P} \}
\end{mathpar}

will replace each (occurrence) of a free name $x$ in $P$ by
$\quotep{\procn{x}|\procn{x}}$.

Also, we will avail ourselves of the notation $x^{L}$ and $x^{R}$ to
denote injections of a name into disjoint copies of the name
space. There are numerous ways to accomplish this. One example can be
found in \cite{MeredithR05}. This notation overloads to vectors of
names: $\vec{x}^{\pi} := (x_{i}^{\pi} \; : \; 0 \leq i < |\vec{x}| )$ where $\pi \in \{L,R\}$.

We also use $P^{\Box} := P|\Box$.

In \cite{MeredithR05} an interpretation of the new operator is
given. It turns out that there are several possible interpretations
all enjoying the requisite algebraic properties of the operator (see
\cite{milner91polyadicpi}). We will therefore make liberal use of
$(\nu\; \vec{x})P$.

% subsection the_syntax_and_semantics_of_the_notation_system (end)   

\input{qm2pi.qmops} 

\input{qm2pi.sterngerlach} 

\input{qm2pi.metric} 

% section concurrent_process_calculi (end)

%\input{qm2pi.proofsketch}

% section proof sketch (end)

%\input{qm2pi.slviaknots} 

% section spatial logic via knots (end)

\input{qm2pi.conclusion}

% section conclusion (end)

%\input{qm2pi.dtcodes} 

% section wiring algorithm (end)

\input{qm2pi.ack} 

% section acknowledgments (end)

\newpage


\bibliographystyle{plain}   
\bibliography{../../biblios/main.bib}

\input{qm2pi.rhodetails}

\end{document}



% section proof sketch (end)

%\section{Unlikely characters: spatial logic for
  knots}\label{sub:characteristic_formulae} % (fold)

Associated to the mobile process calculi are a family of logics known
as the Hennessy-Milner logics. These logics typically enjoy a
semantics interpreting formulae as sets of processes that when
factored through the encoding outlined above allows an identification
of classes of knots with logical formulae. In the context of this
encoding the sub-family known as the spatial logics \cite{CairesC03}
\cite{CairesC04} \cite{Caires04} are of particular interest providing
several important features for expressing and reasoning about
properties (i.e. classes) of knots. We hint here at how this may be done.

%\begin{description}
%\item [structural connectives] 
\subsubsection{Structural connectives} The spatial logics enjoy
structural connectives corresponding, at the logical level, to the
parallel composition ($P | Q$) and new name ($(\nu \; x)P$)
connectives for processes. As illustrated in the examples below, these
connectives are extremely expressive given the shape of our encoding.
%\item [decideable satisfaction]

\subsubsection{Decideable satisfaction}
In \cite{Caires04} the satisfaction relation is shown to be decideable
for a rich class of processes. It further turns out that the image of
the our encoding is a proper subset of that class. This result
provides the basis for an algorithm by which to search for knots
enjoying a given property.
%\item [characteristic formulae]

\subsubsection{Characteristic formulae}
In the same paper \cite{Caires04} , Caires presents a means of calculating
characteristic formulae, selecting equivalence classes of processes
up to a pre--specified depth limit on the support set of names. Composed with our
encoding, this characteristic formula can be used to select
characteristic formulae for knots.
%\end{description}

\subsubsection{Spatial logic formulae}

The grammar below (segmented for comprehension) summarizes the syntax
of spatial logic formulae. We employ illustrative examples in the
sequel to provide an intuitive understanding of their meaning
referring the reader to \cite{Caires04} for a more detailed explication
of the semantics.

\begin{mathpar}
  \inferrule* [lab=boolean] {} {{A,B} \bc T \;|\; \neg A \;|\; A \wedge B \;|\; \eta = \eta'}
  \and
  \inferrule* [lab=spatial] {} {|\; \pzero \;|\; A | B \;|\; x \text{\textregistered} A \;|\; \forall x . A \;|\;  H x . A}
  \and
  \inferrule* [lab=behavioral] {} {|\; \alpha . A}
  \and 
  \inferrule* [lab=recursion] {} {|\; X(\vec{u}) \;|\; \mu X(\vec{u}) . A}
  \and
  \inferrule* [lab=action] {} {\alpha \bc \langle x?(\vec{y}) \rangle \;|\; \langle x!(\vec{y}) \rangle \;|\; \langle \tau \rangle}
  \and 
  \inferrule* [lab=name] {} {\eta \bc x \;|\; \tau}
\end{mathpar} 

% subsection characteristic_formulae (end)   	 

\subsection{Example formulae}\label{sub:example_formulae_} % (fold)

\subsubsection{Crossing as formula.}
% 
% \begin{align*}
%   \frac{d}{dx} \sin x &= \cos x 
%   & \frac{d}{dx} e^x &= e^x \\
%   \frac{d}{dx} \cos x &= - \sin x 
%   & \frac{d}{dx} \log x &= \frac{1}{x} \\
% \end{align*} 

\begin{align*}
 \mu C(x_{0},x_{1},y_{0},y_{1},u).&(\langle x_{0}?(z) \rangle(\langle u! \rangle\langle y_{1}!z \rangle C(x_{0},x_{1},y_{0},y_{1},u)) & \\
  & \wedge \langle y_{1}?(z) \rangle (\langle u! \rangle \langle x_{0}!z \rangle C(x_{0},x_{1},y_{0},y_{1},u)) & \\
  & \wedge \langle x_{1}?(z) \rangle (\langle u? \rangle \langle y_{0}!z \rangle C(x_{0},x_{1},y_{0},y_{1},u)) & \\
  & \wedge \langle y_{0}?(z) \rangle (\langle u? \rangle \langle x_{1}!z \rangle C(x_{0},x_{1},y_{0},y_{1},u))) &
\end{align*}

The lexicographical similarity between the shape of this formulae and
the shape of definition of the process representing a crossing reveals
the intuitive meaning of this formulae. It describes the capabilities
of a process that has the right to represent a crossing. For example
it picks out processes that may perform an input on the port $x_0$ in
its initial menu of capabilities. What differentiates the formula
from the process, however, is that the crossing process is the
smallest candidate to satisfy the formula. Infinitely many other
processes -- with internal behavior hidden behind this interface, so
to speak -- also satisfy this formula. Even this simple formula,
then, can be seen to open a new view onto knots, providing a
computational interpretation of \emph{virtual} knots.

Note that this formula is derived by hand. A similar formula can be
derived by employing Caires' calculation of characteristic formula
\cite{Caires04} to the process representing a crossing. In light of
this discussion, we let
$\meaningof{C}_{\phi}(x0,x1,y0,y1,u)$ denote a formula specifying the
dynamics we wish to capture of a crossing. To guarantee we preserve
the shape of the interface and minimal semantics we demand that
$\meaningof{C}_{\phi}(x0,x1,y0,y1,u) \Rightarrow
\textbf{C}(x0,x1,y0,y1,u)$ where $\textbf{C}(x0,x1,y0,y1,u)$ denotes
the formula above.
                            
\subsubsection{Crossing number constraints.}
The moral content of the context lemma (Lemma \ref{context}) is that the notion of
``locality'' in the Reidemeister moves is effectively captured by the
parallel composition operator of the process calculus. This intuition
extends through the logic. Given a formula,
$\meaningof{C}_{\phi}(x0,x1,y0,y1,u)$, we can use the structural
connectives to specify constraints on crossing numbers, such as at
least $n$ crossings, or exactly $n$ crossings.
\begin{mathpar}
  \inferrule* [lab=at-least-n] {} { K^{\geq n}_{\phi}(\vec{xs},\vec{ys}) := \Pi_{i=0}^{n-1} Hu . \meaningof{C}_{\phi}(xs_i,ys_i,u) | T }
  \and 
  \inferrule* [lab=exactly-n] {} { K^{= n}_{\phi}(\vec{xs},\vec{ys}) := \Pi_{i=0}^{n-1} Hu . \meaningof{C}_{\phi}(xs_i,ys_i,u) | \neg (\forall x_0,y_0,x_1,y_1,u . \meaningof{C}_{\phi}(x_0,y_0,x_1,y_1,u) | T) }
\end{mathpar}

To round out this section, recall that the encoding of an $n$-crossing
knot decomposes into a parallel composition of $n$ \emph{copies} of a
crossing process together with a wiring harness. To specify different
knot classes with the same crossing number amounts to specifying
logical constraints on the wiring harness. In the interest of space,
we defer examples to a forthcoming paper. Suffice it to say that both
the conditions ``alternating knot'' and ``contains the tangle
corresponding to 5/3'' are expressible. For example, it is possible to
calculate the characteristic formula of a process corresponding to the
tangle 5/3 and conjoin it into the classifying formula via the
composition connective of the logic.

Finally, we wish to observe that it is entirely within reason to
contemplate a more domain-specific version of spatial logic tailored
to the shape of processes in the image of the encoding. Such a
domain-specific logic would have a better claim to the title formal
language of knot properties.

% subsection example_formulae_ (end)

% section knots_as_processes (end) 

% section spatial logic via knots (end)

\section{Conclusions and future work}

\paragraph{Testing physical space}
You, gentle reader, may wonder why of all the theorems to be proved
given this set up we pick the one above. In some sense it's hardly
central to quantum mechanics. We see it as central in the sense that
it firmly establishes a notion of physical space arising from a notion
of the equivalence of behavior. Relating bisimulation to a metric is a
big step forward, but one is faced with interpreting the relationship
of that metric space to something more physical. Quantum mechanical
notions of ``physical'' space are still far from intuitive, but by
relating this idea of distance as testing to calculations that predict
physical circumstances we are making a not insignificant step forward
toward an understanding of the physical space we inhabit as
essentially dynamic.

\paragraph{Effectivity and simulation}
One of the observations we have yet to make is that the entire program
spelled out here is effective. We have built various interpreters for
the reflective calculus at work in this interpretation. In principle,
then, we can simulate quantum mechanics on a computer. The place where
the simulation may lose fidelity is the infinitely branching summation
for the annihilator.

In this connection i also want to point out that the evaluation style
calculation of the inner product puts the non-determinism of the
summation right at the heart of measurement. This suggests that
Milner's original reduction-based formulation of the dynamics of his
calculi in terms of sums was not just notationally suggestive of a
notion of measure-and-continue but captured some significant part of
the physics.

\paragraph{Quantum continuations}
In light of this last observation i want to point out that the
predominant account of quantum mechanics is missing a key aspect of a
truly compositional story of the physical situation. In a real lab,
when a measurement is made the observation can be made to feed into
another device that then makes another measurement conditioned on the
results of the first. This means that after the superposition was
collapsed the entire experimental set up remained in
superposition. While QM offers a means of writing this down it doesn't
quite line up well with the well-trodden formulation of computation
and continuation that we see so succinctly expressed in Milner's
calculi. This suggests that there might be advantages to this account
of dynamics waiting to be explored.

\paragraph{Quantum logic}
In this connection, we also note that by virtue of having the
Hennessy-Milner construction, we can pull the construction through the
interpretation of QM. This gives us a natural candidate for a quantum
logic that enjoys an extremely tight connection with it's domain of
interpretation, making the construction much less ad hoc (rather it is
the image of functor!).

\paragraph{Quantum probabiity}
i have questions about the basis of the interpretation of inner
product as probability amplitude. In particular, using which
axiomatization of probability theory does the notion of probability
amplitude earn the right to be so dubbed? In other words, where is the
proof that the operation for calculating a probability amplitude (and
then squaring) satisfies the axioms of what it means to calculate a
probability? Even if such a proof exists (i have yet to find it in the
literature), i wonder if it might not be possible to turn things on
their heads. Can we view the calculation of the probability amplitude
as an axiomatization of probability? If so, then the definition we
give for calculating probability amplitude may provide the basis for
an \emph{effective} theory of probability.

\paragraph{Quantum vs ``biological'' information}
Finally, i want to conclude with a more philosophical observation. At
a recent workshop in which QM was a predominant topic i noticed
something about quantum information. The speaker was giving a riveting
discussion of axiomatic QM and showing how properties of ``no
cloning'' and ``no deleting'' emerged as consequences of the
axiomatization. Theorems of this form are necessary to give us a sense
of confidence that our axioms characterize the physical theory. What
struck me, though, was that if quantum information is neither erasable
nor replicable it is markedly different from \emph{life}. Two of the
things we know about life is that

\begin{itemize}
  \item it ends;
  \item to gain some measure of persistence, to transcend it's
    finitude it is imminently copyable.
\end{itemize}

Both of these qualities are summarized succinctly in the aphorism: all
flesh is grass. For me these two kinds of ``information'' -- call them
quantum and biological -- are end points on a spectrum of strategies
for persistence. At one end, we have those curious entities that enjoy
uniqueness and permanence; at the other, we have those who in the face
of a certain end and an uncertain present make a go of passing
something on. To me one of the more remarkable aspects of the latter
strategy is that in the presence of noise (and certain features of
copying) we get a kind of dynamism, a chance for improvement against a
given persistent condition.

% subsection other_calculi_other_bisimulations_and_geometry_as_behavior (end)




% section conclusion (end)

%\documentclass[12pt]{llncs}
%\documentclass{jktr}

\usepackage[pdftex]{hyperref}                   
\usepackage {listings}
\usepackage {mathpartir}
\usepackage{bcprules}
%\usepackage{listings}
                       
\usepackage{graphicx} 
%\usepackage[margins=2.5cm,nohead,nofoot]{geometry}
%\usepackage{geometry}
\usepackage{amsfonts}
\usepackage{amstext}
\usepackage{latexsym}
\usepackage{amssymb}
\usepackage{color}


%\include{myPreamble}
\include{qm2pi.local} 

%\ifpdf
%\usepackage[pdftex]{graphicx}
%\else
%\usepackage{graphicx}
%\fi

 % \ifpdf
%  \usepackage{pdfsync}
%  \if


%\title{Brief Article}
%\author{David F. Snyder}
%\author{L.G. Meredith}

%\address{Dept. of Math., Texas State University--San Marcos, San Marcos, TX 78666}
       
\pagestyle{empty}


\begin{document}

\lstset{language=[Objective]Caml,frame=shadowbox}

\input{qm2pi.front}

% section front matter (end)

\input{qm2pi.intro} 
 
% section introduction (end)

% \input{qm2pi.knotations} 

% section notation (end)

\input{qm2pi.process.calculi} 

% section concurrent_process_calculi_and_spatial_logics_ (end)
    
%\input{qm2pi.knots2pi} 

%\input{qm2pi.trefoil} 

%\input{qm2pi.mainthm} 

% subsection basic_interpretation (end)

%\input{qm2pi.rho.presentation} 
\subsection{The syntax and semantics of the notation system}\label{sub:the_syntax_and_semantics_of_the_notation_system} % (fold)

We now summarize a technical presentation of the calculus that
embodies our theory of dynamics. The typical presentation of such a
calculus follows the style of giving generators and relations on
them. The grammar, below, describing term constructors, freely
generates the set of processes, $\Proc$. This set is then quotiented
by a relation known as structural congruence and it is over this set
that the notion of dynamics is expressed. This presentation is
essentially that of \cite{MeredithR05} with the addition of
polyadicity and summation. For readability we have relegated some of
the technical subtleties to an appendix.

\subsubsection{Process grammar}\label{subsub:process_grammar}

\begin{mathpar}
  \inferrule* [lab=synchronization] {} {{M} \bc \pzero \;|\; x?F \;|\; x!C }
  \and
  \inferrule* [lab=abstraction] {} {{F} \bc (x)P}
  \and
  \inferrule* [lab=concretion] {} {{C} \bc \langle Q \rangle}
  \and
  \inferrule* [lab=process] {} {{P,Q} \bc M \;| \;P|Q \;|\; @{x}}
  \and
  \inferrule* [lab=name] {} {{x} \bc \quotep{P}}
\end{mathpar} 

Note that $\vec{x}$ (resp. $\vec{P}$) denotes a vector of names
(resp. processes) of length $|\vec{x}|$ (resp. $|\vec{P}|$). We adopt
the following useful abbreviations.

\begin{mathpar}
   x?(\vec{y}).P := x.(\vec{y})P \and  x\clift{\vec{P}} := x.\clift{\vec{P}}
   \and x!(y) := \lift{x}{\dropn{y}}
   \and \Pi_{i=0}^{n-1}P_i := P_0 | \ldots | P_{n-1}
\end{mathpar}

\subsubsection{Structural congruence}

\paragraph{Free and bound names and alpha-equivalence.} At the
core of structural equivalence is alpha-equivalence which identifies
process that are the same up to a change of variable. Formally, we
recognize the distinction between free and bound names. The free names
of a process, $\freenames{P}$, may be calculated recursively as
follows:

\begin{mathpar}
\freenames{\pzero} := \emptyset
  \and \\
  \freenames{x?(y).P} := \{ x \} \cup (\freenames{P} \setminus \{ y \})
  \and 
  \freenames{x!\langle P \rangle} := \{ x \} \cup \{ P \} 
  \and \\
  \freenames{P|Q} := \freenames{P} \cup \freenames{Q}
  \and \\
  \freenames{@{x}} := \{ x \}
\end{mathpar}

$\pi$
$\quotep{\pi}$

$\freenames{-} : \pi \to \mathcal{P}(\quotep{\pi})$

\begin{eqnarray*}
  \freenames{\pzero} & := & \emptyset \\
  \freenames{x?(y).P} & := & \{ x \} \cup (\freenames{P} \setminus \{ y \}) \\
  \freenames{x!\langle P \rangle} & := & \{ x \} \cup \{ P \} \\
  \freenames{P|Q} & := & \freenames{P} \cup \freenames{Q} \\
  \freenames{\dropn{x}} & := & \{ x \}
\end{eqnarray*}

The bound names of a process, $\boundnames{P}$, are those names occurring in $P$
that are not free. For example, in $x?(y).0$, the name $x$ is free, while $y$ is bound.

\begin{mathpar}
  \inferrule* [lab=monoidal-laws] {} { P|Q \equiv Q|P \and P|0 \equiv P \and P|(Q|R) \equiv (P|Q)|R }
\end{mathpar}

\begin{mathpar}
  \inferrule* [lab=alpha-equivalence] {} { (x)P \equiv (y)P\{y/x\} \and y \not\in \freenames{P} }
\end{mathpar}

\begin{definition}
Then two processes, $P,Q$, are alpha-equivalent if $P = Q\{\vec{y}/\vec{x}\}$ for
some $\vec{x} \in \boundnames{Q},\vec{y} \in \boundnames{P}$, where $Q\{\vec{y}/\vec{x}\}$
denotes the capture-avoiding substitution of $\vec{y}$ for $\vec{x}$ in $Q$.
\end{definition}

\begin{definition}
  The {\em structural congruence} \cite{SangiorgiWalker} , $\equiv$,
  between processes is the least congruence containing
  alpha-equivalence, satisfying the abelian monoid laws
  (associativity, commutativity and $\pzero$ as identity) for parallel
  composition $|$ and for summation $+$.
\end{definition}

\subsection{Name equivalence}

We take name equivalence, written $\nameeq$, to be the smallest
equivalence relation generated by the following rules.

\begin{mathpar}
\inferrule*[lab=Quote-drop]
{ }
{ \quotep{@{x}} \nameeq x }

\inferrule*[lab=Struct-equiv]
{ P \scong Q }
{ \quotep{P} \nameeq \quotep{Q} }
\end{mathpar}

The astute reader will have noticed that the mutual recursion of names
and processes imposes a mutual recursion on alpha-equivalence and
structural equivalence via name-equivalence. Fortunately, all of this
works out pleasantly and we may calculate in the natural way, free of
concern. The reader interested in the details is referred to the
appendix \ref{appendix:rho_details}.

\subsection{Substitution}

We use $\Proc$ for the set of processes, $\QProc$ for the set of
names, and $\id{\{}\vec{y} / \vec{x} \id{\}}$ to denote partial maps,
$s : \QProc \rightarrow \QProc$. A map, $s$ lifts, uniquely, to a map
on process terms, $\widehat{s} : \Proc \rightarrow \Proc$ by the
following equations.

\begin{mathpar}
  (0) \psubstp{Q}{P} := 0 \\
  (R \juxtap S) \psubstp{Q}{P}
  :=    
  (R)\psubstp{Q}{P} \juxtap (S) \psubstp{Q}{P} \\
  (x?(y).R) \psubstp{Q}{P}    
  :=    
  (x)\substp{Q}{P} (z)\concat( (R \psubstn{z}{y}) \psubstp{Q}{P} ) \\
  (\lift{x}{R}) \psubstp{Q}{P}  
  :=
  \lift{(x)\substp{Q}{P}}{ R \psubstp{Q}{P} } \\
%   (\dropn{x})  \psubstp{Q}{P}       
%   := 
%   \left\{ 
%     \begin{array}{ccc} 
%       \dropn{\quotep{Q}} & & x \nameeq \quotep{P} \\
%       \dropn{x} & & otherwise \\
%     \end{array}
%   \right. 
  (\dropn{x})  \psubstp{Q}{P}       
  := 
  \left\{ 
    \begin{array}{ccc} 
      Q & & x \nameeq \quotep{P} \\
      \dropn{x} & & otherwise \\
    \end{array}
  \right.
\end{mathpar}
 

where

\begin{eqnarray}
  (x)\id{\{} \lpquote Q \rpquote / \lpquote P \rpquote \id{\}}            = 
  \left\{ 
    \begin{array}{ccc}
      \lpquote Q \rpquote & & x \nameeq \lpquote P \rpquote \\
      x & & otherwise \\
    \end{array}
  \right. \nonumber
\end{eqnarray}

and $z$ is chosen distinct from $\quotep{P}$, $\quotep{Q}$, the free
names in $Q$, and all the names in $R$. Our $\alpha$-equivalence will
be built in the standard way from this substitution.

\begin{remark}\label{rem:no_self_referential_names}
  One consequence of these definitions is that $\forall P. \quotep{P}
  \not\in \freenames{P}$.
\end{remark}

\subsection{ Dynamic quote: an example }

Anticipating something of what's to come, consider applying the
substitution, $\widehat{\id{\{}u / z \id{\}}}$, to the following pair
of processes, $\lift{w}{y!(z)}$ and $w[ \lpquote y!(z) \rpquote ]$.

\begin{eqnarray}
	\lift{w}{y!(z)}\widehat{\id{\{}u / z \id{\}}}
		& = &
		\lift{w}{y!(u)} \nonumber\\
	w[ \lpquote y!(z) \rpquote ] \widehat{ \id{\{}u / z \id{\}} }
		& = &
		w[ \lpquote y!(z) \rpquote ] \nonumber
\end{eqnarray}

Because the body of the process between quotes is impervious to
substitution, we get radically different answers. In fact, by
examining the first process in an input context,
e.g. $x?(z).\lift{w}{y!(z)}$, we see that the process under the lift
operator may be shaped by prefixed inputs binding a name inside it. In
this sense, the lift operator will be seen as a way to dynamically
construct processes before reifying them as names.

Finally equipped with these standard features we can present the
dynamics of the calculus.

\subsubsection{Operational semantics} 

Finally, we introduce the computational dynamics. What marks these
algebras as distinct from other more traditionally studied algebraic
structures, e.g. vector spaces or polynomial rings, is the manner in
which dynamics is captured. In traditional structures, dynamics is typically
expressed through morphisms between such structures, as in linear maps
between vector spaces or morphisms between rings. In algebras
associated with the semantics of computation, the dynamics is
expressed as part of the algebraic structure itself, through a
reduction reduction relation typically denoted by $\red$. Below, we
give a recursive presentation of this relation for the calculus used
in the encoding.

$\red \subseteq \pi \times \pi$
$\red : \pi \to \mathcal{P}(\pi)$

\begin{mathpar}
  \inferrule* [lab=Comm] { \textsf{match}( x_{src}, x_{trgt} ) } { x_{trgt}?(y)P \; | \; x_{src}!\langle {Q} \rangle \red P\{\quotep{Q}/y}\} }
  \and \\
  \inferrule* [lab=Par] {{P} \red {P}'} {{{P} | {Q}} \red {{P}' | {Q}}}
  \and
  \inferrule* [lab=Equiv]{{{P} \scong {P}'} \andalso {{P}' \red {Q}'} \andalso {{Q}' \scong {Q}}}{{P} \red {Q}}
\end{mathpar}

\begin{eqnarray*}
  match_{\equiv} (\quotep{P},\quotep{Q}) & := & P \equiv Q \\
  match_{\dagger}(\quotep{P},\quotep{Q}) & := & \forall R. P|Q \red^{*} R => R \red^{*} 0 \\
  match_{K}(\quotep{P},\quotep{Q}) & := & K \mbox{ for some context } K
\end{eqnarray*}

$u?(x)P | u!\langle Q \rangle \red P\{\quotep{Q}/x\}$

%We write $\wred$ for $\red^*$, and $P\red$ if $\exists Q $ such that $ P \red Q$.
We write $P\red$ if $\exists Q $ such that $ P \red Q$ and $P\not\red$, otherwise.

\section{Replication}

As mentioned before, it is known that replication (and hence
recursion) can be implemented in a higher-order process algebra
\cite{SangiorgiWalker}. As our first example of calculation with the
machinery thus far presented we give the construction explicitly in
the {\rhoc}.

\begin{eqnarray}
	D_{x} & := & \prefix{x}{y}{(\binpar{\outputp{x}{y}}{@{y}})} \nonumber\\
	\bangp_{x}{P} & := & \binpar{{x}!\langle{\binpar{D_{x}}{P}}\rangle}{D_{x}} \nonumber
\end{eqnarray}

\begin{eqnarray}
	\bangp_{x}{P} & & \nonumber\\
	=
	& {x}!\langle{(\prefix{x}{y}{(\outputp{x}{y} | @{y})) | P}}\rangle 
	      | \prefix{x}{y}{(\outputp{x}{y} | @{y})} & \nonumber\\
	\red
	& (\outputp{x}{y} | @{y})\substn{\quotep{(\prefix{x}{y}{(@{y} | \outputp{x}{y})) | P}}}{y} & \nonumber\\
	=
	& \outputp{x}{\quotep{(\prefix{x}{y}{(\outputp{x}{y} | @{y})) | P}}}
	  | {(\prefix{x}{y}{(\outputp{x}{y} | @{y})) | P}} & \nonumber\\
	\red
	& \ldots & \nonumber\\
	\red^*
	& P | P | \ldots & \nonumber
\end{eqnarray}

Of course, this encoding, as an implementation, runs away, unfolding
$\bangp{P}$ eagerly. A lazier and more implementable replication
operator, restricted to input-guarded processes, may be obtained as follows.

\begin{eqnarray}
\bangp{\prefix{u}{v}{P}} 
	:= 
	\binpar{\lift{x}{\prefix{u}{v}{(\binpar{D(x)}{P})}}}{D(x)} \nonumber
\end{eqnarray}

\begin{remark}
  Note that the lazier definition still does not deal with summation
  or mixed summation (i.e. sums over input and output). The reader is
  invited to construct definitions of replication that deal with these
  features. 

  Further, the definitions are parameterized in a name, $x$. Can you,
  gentle reader, make a definition that eliminates this parameter and
  guarantees no accidental interaction between the replication
  machinery and the process being replicated -- i.e. no accidental
  sharing of names used by the process to get its work done and the
  name(s) used by the replication to effect copying. This latter
  revision of the definition of replication is crucial to obtaining
  the expected identity $!!P \sim !P$.
\end{remark}

\begin{remark}\label{rem:paradoxical_combinator}
  The reader familiar with the lambda calculus will have noticed the
  similarity between $D$ and the paradoxical combinator.

  [Ed. note: the existence of this seems to suggest we have to be more
  restrictive on the set of processes and names we admit if we are to
  support no-cloning.]
\end{remark}

\subsubsection{Bisimulation}

The computational dynamics gives rise to another kind of equivalence,
the equivalence of computational behavior. As previously mentioned
this is typically captured \emph{via} some form of bisimulation.

% The notion we use in this paper is weak barbed bisimulation
% \cite{milner91polyadicpi}.

The notion we use in this paper is derived from weak barbed
bisimulation \cite{milner91polyadicpi}. 

\begin{definition}
An \emph{observation relation}, $\downarrow_{\mathcal N}$, over a set
of names, $\mathcal N$, is the smallest relation satisfying the rules
below.

\infrule[Out-barb]{y \in {\mathcal N}, \; x \nameeq y}
		  {\outputp{x}{v} \downarrow_{\mathcal N} x}
\infrule[Par-barb]{\mbox{$P\downarrow_{\mathcal N} x$ or $Q\downarrow_{\mathcal N} x$}}
		  {\binpar{P}{Q} \downarrow_{\mathcal N} x}

We write $P \Downarrow_{\mathcal N} x$ if there is $Q$ such that 
$P \wred Q$ and $Q \downarrow_{\mathcal N} x$.
\end{definition}

\begin{definition}
%\label{def.bbisim}
An  ${\mathcal N}$-\emph{barbed bisimulation} over a set of names, ${\mathcal N}$, is a symmetric binary relation 
${\mathcal S}_{\mathcal N}$ between agents such that $P\rel{S}_{\mathcal N}Q$ implies:
\begin{enumerate}
\item If $P \red P'$ then $Q \wred Q'$ and $P'\rel{S}_{\mathcal N} Q'$.
\item If $P\downarrow_{\mathcal N} x$, then $Q\Downarrow_{\mathcal N} x$.
\end{enumerate}
$P$ is ${\mathcal N}$-barbed bisimilar to $Q$, written
$P \wbbisim_{\mathcal N} Q$, if $P \rel{S}_{\mathcal N} Q$ for some ${\mathcal N}$-barbed bisimulation ${\mathcal S}_{\mathcal N}$.
\end{definition}

$\mathcal{R} \subseteq \pi \times \pi$

$P \mathcal{R} Q => \forall P'. P \red P' \Rightarrow \exists Q'. Q \red Q', P' \mathcal{R} Q'$

$P \vdash x \Rightarrow Q \vdash x$

\begin{mathpar}
  \inferrule*[lab=Out-barb]{x \nameeq y}{{y}!\langle{Q}\rangle \vdash x}
  \and
  \inferrule*[lab=Par-barb]{\mbox{$P\vdash x$ or $Q\vdash x$}}{\binpar{P}{Q} \vdash x}
\end{mathpar}

\subsubsection{Contexts}

One of the principle advantages of computational calculi like the
$\pi$-calculus is a well-defined notion of context,
contextual-equivalence and a correlation between
contextual-equivalence and notions of bisimulation. The notion of
context allows the decomposition of a process into (sub-)process and
its syntactic environment, its context. Thus, a context may be
thought of as a process with a ``hole'' (written $\Box$) in it. The
application of a context $M$ to a process $P$, written $M[P]$, is
tantamount to filling the hole in $M$ with $P$. In this paper we do
not need the full weight of this theory, but do make use of the notion
of context in the proof the main theorem. 

\begin{mathpar}
  \inferrule* [lab=summation] {} {{M_{M},M_{N}} \bc \Box \;|\; x.M_{A} \;|\; M_{M}+M_{N}}
  \and
  \inferrule* [lab=agent] {} {{M_{A}} \bc (\vec{x})M_{P} \;| \; \clift{P_0,\ldots,M_{P},\ldots,P_N}}
  \and \\
  \inferrule* [lab=process] {} {{M_{P}} \bc M_{N} \;| \;P|M_{P} }
\end{mathpar} 

\begin{mathpar}
  \inferrule* [lab=sychronization] {} {M_{N} \bc \Box \;|\; x?M_{F} \;|\; x!M_{C}}
  \and
  \inferrule* [lab=abstraction] {} {{M_{F}} \bc (x)M_{P} }
  \and
  \inferrule* [lab=concretion] {} {{M_{C}} \bc \langle M_{P} \rangle }
  \and \\
  \inferrule* [lab=process] {} {{M_{P}} \bc M_{N} \;| \;P|M_{P} }
\end{mathpar}

\begin{definition}[contextual application] Given a context $M$, and
  process $P$, we define the \emph{contextual application}, $M[P] :=
  M\{P/\Box\}$. That is, the contextual application of M to P is the
  substitution of $P$ for $\Box$ in $M$.
\end{definition}

$\meaningof{-} : L \to \mathcal{P}(\pi)$

\begin{mathpar}
  \inferrule* [lab=collection] {} {\meaningof{true} = \pi, \and \meaningof{~E} = \pi \setminus \meaningof{E}, \and \meaningof{E_{1} \& E_{2}} = \meaningof{E_{1}} \cap \meaningof{E_{2}}}
\end{mathpar}

\begin{mathpar}
  \inferrule* [lab=structure] {} {\meaningof{0} = \{ P \in \pi | P \equiv 0 \}, \and \\ \meaningof{E_1 | E_2} = \{ P \in \pi | P \equiv P_{1} | P_{2}, P_{1} \in \meaningof{E_{1}}, P_{2} \in \meaningof{E_2}\} }
\end{mathpar}

\begin{mathpar}
 \inferrule* [lab=behavior] {} {\meaningof{\langle a?b \rangle E} = \{ P \in \pi | P \equiv Q | u?(y)P', \\ \and \\\\ \and \\ \;\;\; u \in \meaningof{a}, \forall z.P'\{z/y\} \in \meaningof{E\{z/b\}}\}, \and \\ \meaningof{a!E} = \{ P \in \pi | P \equiv Q | x!\langle P' \rangle, x \in \meaningof{a} P' \in \meaningof{E}\} }
\end{mathpar}

\begin{mathpar}
 \inferrule* [lab=nominal] {} {\meaningof{\quotep{E}} = \{ \quotep{P} \in \quotep{\pi} | P \in \meaningof{E} \}, \and \meaningof{\quotep{P}} = \{ \quotep{Q} \in \quotep{\pi} | P \equiv Q \} \and \\ \meaningof{@\quotep{E}} = \{ P \in \pi | P \equiv @x, x \in \meaningof{E} \}}
\end{mathpar}

\begin{eqnarray*}
  \\
  \meaningof{-} : TS \to ST
\end{eqnarray*}

\begin{eqnarray*}
  \\
  L : TS \to ST
\end{eqnarray*}

\begin{eqnarray*}
  \\
  P \models E \iff P \in \meaningof{E}
\end{eqnarray*}

\begin{eqnarray*}
  P \approx_{L} Q \iff \forall E \in L. P \models E \iff Q \models E
\end{eqnarray*}

\begin{eqnarray*}
  P \approx_{K} Q
\end{eqnarray*}

\begin{eqnarray*}
  P \approx Q
\end{eqnarray*}

$\approx_{K} = \approx = \approx_{L}$

\subsubsection{Contextual duality}

Note that contexts extend the quotation operation to a family of
operations from processes to names. Given a context, $M$, we can
define a \emph{nominal context}, $\quotep{M}$ by $\quotep{M}[P] :=
\quotep{M[P]}$. To foreshadow what is to come we observe that these
operations enjoy a duality with processes very much like the duality
between vectors and maps from vectors to scalars.

Further, because the calculus is essentially higher-order, we have a
correspondence between contexts and processes. More specifically,
given a name $x$ and a context $M$ we can construct $M^{*}_{x}$ such
that 

\begin{mathpar}
  M^{*}_{x} | \lift{x}{P} \red M[P]
\end{mathpar}

namely,

\begin{mathpar}
  M^{*}_{x} := x?(u).M[\dropn{u}]
\end{mathpar}

The dependence of $M^{*}_{x}$ on a name makes it an abstraction, 

\begin{mathpar}
  M^{*} := (x)x?(u).M[\dropn{u}]
\end{mathpar}

\subsection{Additional notation}

It will sometimes be convenient to denote the process a name
quotes. We already have the notation $x = \quotep{P}$, but it will be
convenient to introduce an alternate notation, $\procn{x}$, when we
want to emphasize the connection to the use of the name. Note that, by
virtue of name equivalence, $\quotep{\procn{x}} \nameeq x$; so, the
notation is consistent with previous definitions.

Further, because names have structure it is possible to effect
substitutions on the basis of that structure. This means we need to
upgrade our notation for substitutions, which we accomplish by
adapting comprehension notation. Thus,

\begin{mathpar}
  P\{ y / x : x \in S \}
\end{mathpar}

is interpreted to mean the process derived from P by replacing (in a
capture-avoiding manner) each occurrence of $x$ in $S$ by $y$. For example,

\begin{mathpar}
  P\{ \quotep{\procn{x}|\procn{x}} / x : x \in \freenames{P} \}
\end{mathpar}

will replace each (occurrence) of a free name $x$ in $P$ by
$\quotep{\procn{x}|\procn{x}}$.

Also, we will avail ourselves of the notation $x^{L}$ and $x^{R}$ to
denote injections of a name into disjoint copies of the name
space. There are numerous ways to accomplish this. One example can be
found in \cite{MeredithR05}. This notation overloads to vectors of
names: $\vec{x}^{\pi} := (x_{i}^{\pi} \; : \; 0 \leq i < |\vec{x}| )$ where $\pi \in \{L,R\}$.

We also use $P^{\Box} := P|\Box$.

In \cite{MeredithR05} an interpretation of the new operator is
given. It turns out that there are several possible interpretations
all enjoying the requisite algebraic properties of the operator (see
\cite{milner91polyadicpi}). We will therefore make liberal use of
$(\nu\; \vec{x})P$.

% subsection the_syntax_and_semantics_of_the_notation_system (end)   

\input{qm2pi.qmops} 

\input{qm2pi.sterngerlach} 

\input{qm2pi.metric} 

% section concurrent_process_calculi (end)

%\input{qm2pi.proofsketch}

% section proof sketch (end)

%\input{qm2pi.slviaknots} 

% section spatial logic via knots (end)

\input{qm2pi.conclusion}

% section conclusion (end)

%\input{qm2pi.dtcodes} 

% section wiring algorithm (end)

\input{qm2pi.ack} 

% section acknowledgments (end)

\newpage


\bibliographystyle{plain}   
\bibliography{../../biblios/main.bib}

\input{qm2pi.rhodetails}

\end{document}

 

% section wiring algorithm (end)

\documentclass[12pt]{llncs}
%\documentclass{jktr}

\usepackage[pdftex]{hyperref}                   
\usepackage {listings}
\usepackage {mathpartir}
\usepackage{bcprules}
%\usepackage{listings}
                       
\usepackage{graphicx} 
%\usepackage[margins=2.5cm,nohead,nofoot]{geometry}
%\usepackage{geometry}
\usepackage{amsfonts}
\usepackage{amstext}
\usepackage{latexsym}
\usepackage{amssymb}
\usepackage{color}


%\include{myPreamble}
\include{qm2pi.local} 

%\ifpdf
%\usepackage[pdftex]{graphicx}
%\else
%\usepackage{graphicx}
%\fi

 % \ifpdf
%  \usepackage{pdfsync}
%  \if


%\title{Brief Article}
%\author{David F. Snyder}
%\author{L.G. Meredith}

%\address{Dept. of Math., Texas State University--San Marcos, San Marcos, TX 78666}
       
\pagestyle{empty}


\begin{document}

\lstset{language=[Objective]Caml,frame=shadowbox}

\input{qm2pi.front}

% section front matter (end)

\input{qm2pi.intro} 
 
% section introduction (end)

% \input{qm2pi.knotations} 

% section notation (end)

\input{qm2pi.process.calculi} 

% section concurrent_process_calculi_and_spatial_logics_ (end)
    
%\input{qm2pi.knots2pi} 

%\input{qm2pi.trefoil} 

%\input{qm2pi.mainthm} 

% subsection basic_interpretation (end)

%\input{qm2pi.rho.presentation} 
\subsection{The syntax and semantics of the notation system}\label{sub:the_syntax_and_semantics_of_the_notation_system} % (fold)

We now summarize a technical presentation of the calculus that
embodies our theory of dynamics. The typical presentation of such a
calculus follows the style of giving generators and relations on
them. The grammar, below, describing term constructors, freely
generates the set of processes, $\Proc$. This set is then quotiented
by a relation known as structural congruence and it is over this set
that the notion of dynamics is expressed. This presentation is
essentially that of \cite{MeredithR05} with the addition of
polyadicity and summation. For readability we have relegated some of
the technical subtleties to an appendix.

\subsubsection{Process grammar}\label{subsub:process_grammar}

\begin{mathpar}
  \inferrule* [lab=synchronization] {} {{M} \bc \pzero \;|\; x?F \;|\; x!C }
  \and
  \inferrule* [lab=abstraction] {} {{F} \bc (x)P}
  \and
  \inferrule* [lab=concretion] {} {{C} \bc \langle Q \rangle}
  \and
  \inferrule* [lab=process] {} {{P,Q} \bc M \;| \;P|Q \;|\; @{x}}
  \and
  \inferrule* [lab=name] {} {{x} \bc \quotep{P}}
\end{mathpar} 

Note that $\vec{x}$ (resp. $\vec{P}$) denotes a vector of names
(resp. processes) of length $|\vec{x}|$ (resp. $|\vec{P}|$). We adopt
the following useful abbreviations.

\begin{mathpar}
   x?(\vec{y}).P := x.(\vec{y})P \and  x\clift{\vec{P}} := x.\clift{\vec{P}}
   \and x!(y) := \lift{x}{\dropn{y}}
   \and \Pi_{i=0}^{n-1}P_i := P_0 | \ldots | P_{n-1}
\end{mathpar}

\subsubsection{Structural congruence}

\paragraph{Free and bound names and alpha-equivalence.} At the
core of structural equivalence is alpha-equivalence which identifies
process that are the same up to a change of variable. Formally, we
recognize the distinction between free and bound names. The free names
of a process, $\freenames{P}$, may be calculated recursively as
follows:

\begin{mathpar}
\freenames{\pzero} := \emptyset
  \and \\
  \freenames{x?(y).P} := \{ x \} \cup (\freenames{P} \setminus \{ y \})
  \and 
  \freenames{x!\langle P \rangle} := \{ x \} \cup \{ P \} 
  \and \\
  \freenames{P|Q} := \freenames{P} \cup \freenames{Q}
  \and \\
  \freenames{@{x}} := \{ x \}
\end{mathpar}

$\pi$
$\quotep{\pi}$

$\freenames{-} : \pi \to \mathcal{P}(\quotep{\pi})$

\begin{eqnarray*}
  \freenames{\pzero} & := & \emptyset \\
  \freenames{x?(y).P} & := & \{ x \} \cup (\freenames{P} \setminus \{ y \}) \\
  \freenames{x!\langle P \rangle} & := & \{ x \} \cup \{ P \} \\
  \freenames{P|Q} & := & \freenames{P} \cup \freenames{Q} \\
  \freenames{\dropn{x}} & := & \{ x \}
\end{eqnarray*}

The bound names of a process, $\boundnames{P}$, are those names occurring in $P$
that are not free. For example, in $x?(y).0$, the name $x$ is free, while $y$ is bound.

\begin{mathpar}
  \inferrule* [lab=monoidal-laws] {} { P|Q \equiv Q|P \and P|0 \equiv P \and P|(Q|R) \equiv (P|Q)|R }
\end{mathpar}

\begin{mathpar}
  \inferrule* [lab=alpha-equivalence] {} { (x)P \equiv (y)P\{y/x\} \and y \not\in \freenames{P} }
\end{mathpar}

\begin{definition}
Then two processes, $P,Q$, are alpha-equivalent if $P = Q\{\vec{y}/\vec{x}\}$ for
some $\vec{x} \in \boundnames{Q},\vec{y} \in \boundnames{P}$, where $Q\{\vec{y}/\vec{x}\}$
denotes the capture-avoiding substitution of $\vec{y}$ for $\vec{x}$ in $Q$.
\end{definition}

\begin{definition}
  The {\em structural congruence} \cite{SangiorgiWalker} , $\equiv$,
  between processes is the least congruence containing
  alpha-equivalence, satisfying the abelian monoid laws
  (associativity, commutativity and $\pzero$ as identity) for parallel
  composition $|$ and for summation $+$.
\end{definition}

\subsection{Name equivalence}

We take name equivalence, written $\nameeq$, to be the smallest
equivalence relation generated by the following rules.

\begin{mathpar}
\inferrule*[lab=Quote-drop]
{ }
{ \quotep{@{x}} \nameeq x }

\inferrule*[lab=Struct-equiv]
{ P \scong Q }
{ \quotep{P} \nameeq \quotep{Q} }
\end{mathpar}

The astute reader will have noticed that the mutual recursion of names
and processes imposes a mutual recursion on alpha-equivalence and
structural equivalence via name-equivalence. Fortunately, all of this
works out pleasantly and we may calculate in the natural way, free of
concern. The reader interested in the details is referred to the
appendix \ref{appendix:rho_details}.

\subsection{Substitution}

We use $\Proc$ for the set of processes, $\QProc$ for the set of
names, and $\id{\{}\vec{y} / \vec{x} \id{\}}$ to denote partial maps,
$s : \QProc \rightarrow \QProc$. A map, $s$ lifts, uniquely, to a map
on process terms, $\widehat{s} : \Proc \rightarrow \Proc$ by the
following equations.

\begin{mathpar}
  (0) \psubstp{Q}{P} := 0 \\
  (R \juxtap S) \psubstp{Q}{P}
  :=    
  (R)\psubstp{Q}{P} \juxtap (S) \psubstp{Q}{P} \\
  (x?(y).R) \psubstp{Q}{P}    
  :=    
  (x)\substp{Q}{P} (z)\concat( (R \psubstn{z}{y}) \psubstp{Q}{P} ) \\
  (\lift{x}{R}) \psubstp{Q}{P}  
  :=
  \lift{(x)\substp{Q}{P}}{ R \psubstp{Q}{P} } \\
%   (\dropn{x})  \psubstp{Q}{P}       
%   := 
%   \left\{ 
%     \begin{array}{ccc} 
%       \dropn{\quotep{Q}} & & x \nameeq \quotep{P} \\
%       \dropn{x} & & otherwise \\
%     \end{array}
%   \right. 
  (\dropn{x})  \psubstp{Q}{P}       
  := 
  \left\{ 
    \begin{array}{ccc} 
      Q & & x \nameeq \quotep{P} \\
      \dropn{x} & & otherwise \\
    \end{array}
  \right.
\end{mathpar}
 

where

\begin{eqnarray}
  (x)\id{\{} \lpquote Q \rpquote / \lpquote P \rpquote \id{\}}            = 
  \left\{ 
    \begin{array}{ccc}
      \lpquote Q \rpquote & & x \nameeq \lpquote P \rpquote \\
      x & & otherwise \\
    \end{array}
  \right. \nonumber
\end{eqnarray}

and $z$ is chosen distinct from $\quotep{P}$, $\quotep{Q}$, the free
names in $Q$, and all the names in $R$. Our $\alpha$-equivalence will
be built in the standard way from this substitution.

\begin{remark}\label{rem:no_self_referential_names}
  One consequence of these definitions is that $\forall P. \quotep{P}
  \not\in \freenames{P}$.
\end{remark}

\subsection{ Dynamic quote: an example }

Anticipating something of what's to come, consider applying the
substitution, $\widehat{\id{\{}u / z \id{\}}}$, to the following pair
of processes, $\lift{w}{y!(z)}$ and $w[ \lpquote y!(z) \rpquote ]$.

\begin{eqnarray}
	\lift{w}{y!(z)}\widehat{\id{\{}u / z \id{\}}}
		& = &
		\lift{w}{y!(u)} \nonumber\\
	w[ \lpquote y!(z) \rpquote ] \widehat{ \id{\{}u / z \id{\}} }
		& = &
		w[ \lpquote y!(z) \rpquote ] \nonumber
\end{eqnarray}

Because the body of the process between quotes is impervious to
substitution, we get radically different answers. In fact, by
examining the first process in an input context,
e.g. $x?(z).\lift{w}{y!(z)}$, we see that the process under the lift
operator may be shaped by prefixed inputs binding a name inside it. In
this sense, the lift operator will be seen as a way to dynamically
construct processes before reifying them as names.

Finally equipped with these standard features we can present the
dynamics of the calculus.

\subsubsection{Operational semantics} 

Finally, we introduce the computational dynamics. What marks these
algebras as distinct from other more traditionally studied algebraic
structures, e.g. vector spaces or polynomial rings, is the manner in
which dynamics is captured. In traditional structures, dynamics is typically
expressed through morphisms between such structures, as in linear maps
between vector spaces or morphisms between rings. In algebras
associated with the semantics of computation, the dynamics is
expressed as part of the algebraic structure itself, through a
reduction reduction relation typically denoted by $\red$. Below, we
give a recursive presentation of this relation for the calculus used
in the encoding.

$\red \subseteq \pi \times \pi$
$\red : \pi \to \mathcal{P}(\pi)$

\begin{mathpar}
  \inferrule* [lab=Comm] { \textsf{match}( x_{src}, x_{trgt} ) } { x_{trgt}?(y)P \; | \; x_{src}!\langle {Q} \rangle \red P\{\quotep{Q}/y}\} }
  \and \\
  \inferrule* [lab=Par] {{P} \red {P}'} {{{P} | {Q}} \red {{P}' | {Q}}}
  \and
  \inferrule* [lab=Equiv]{{{P} \scong {P}'} \andalso {{P}' \red {Q}'} \andalso {{Q}' \scong {Q}}}{{P} \red {Q}}
\end{mathpar}

\begin{eqnarray*}
  match_{\equiv} (\quotep{P},\quotep{Q}) & := & P \equiv Q \\
  match_{\dagger}(\quotep{P},\quotep{Q}) & := & \forall R. P|Q \red^{*} R => R \red^{*} 0 \\
  match_{K}(\quotep{P},\quotep{Q}) & := & K \mbox{ for some context } K
\end{eqnarray*}

$u?(x)P | u!\langle Q \rangle \red P\{\quotep{Q}/x\}$

%We write $\wred$ for $\red^*$, and $P\red$ if $\exists Q $ such that $ P \red Q$.
We write $P\red$ if $\exists Q $ such that $ P \red Q$ and $P\not\red$, otherwise.

\section{Replication}

As mentioned before, it is known that replication (and hence
recursion) can be implemented in a higher-order process algebra
\cite{SangiorgiWalker}. As our first example of calculation with the
machinery thus far presented we give the construction explicitly in
the {\rhoc}.

\begin{eqnarray}
	D_{x} & := & \prefix{x}{y}{(\binpar{\outputp{x}{y}}{@{y}})} \nonumber\\
	\bangp_{x}{P} & := & \binpar{{x}!\langle{\binpar{D_{x}}{P}}\rangle}{D_{x}} \nonumber
\end{eqnarray}

\begin{eqnarray}
	\bangp_{x}{P} & & \nonumber\\
	=
	& {x}!\langle{(\prefix{x}{y}{(\outputp{x}{y} | @{y})) | P}}\rangle 
	      | \prefix{x}{y}{(\outputp{x}{y} | @{y})} & \nonumber\\
	\red
	& (\outputp{x}{y} | @{y})\substn{\quotep{(\prefix{x}{y}{(@{y} | \outputp{x}{y})) | P}}}{y} & \nonumber\\
	=
	& \outputp{x}{\quotep{(\prefix{x}{y}{(\outputp{x}{y} | @{y})) | P}}}
	  | {(\prefix{x}{y}{(\outputp{x}{y} | @{y})) | P}} & \nonumber\\
	\red
	& \ldots & \nonumber\\
	\red^*
	& P | P | \ldots & \nonumber
\end{eqnarray}

Of course, this encoding, as an implementation, runs away, unfolding
$\bangp{P}$ eagerly. A lazier and more implementable replication
operator, restricted to input-guarded processes, may be obtained as follows.

\begin{eqnarray}
\bangp{\prefix{u}{v}{P}} 
	:= 
	\binpar{\lift{x}{\prefix{u}{v}{(\binpar{D(x)}{P})}}}{D(x)} \nonumber
\end{eqnarray}

\begin{remark}
  Note that the lazier definition still does not deal with summation
  or mixed summation (i.e. sums over input and output). The reader is
  invited to construct definitions of replication that deal with these
  features. 

  Further, the definitions are parameterized in a name, $x$. Can you,
  gentle reader, make a definition that eliminates this parameter and
  guarantees no accidental interaction between the replication
  machinery and the process being replicated -- i.e. no accidental
  sharing of names used by the process to get its work done and the
  name(s) used by the replication to effect copying. This latter
  revision of the definition of replication is crucial to obtaining
  the expected identity $!!P \sim !P$.
\end{remark}

\begin{remark}\label{rem:paradoxical_combinator}
  The reader familiar with the lambda calculus will have noticed the
  similarity between $D$ and the paradoxical combinator.

  [Ed. note: the existence of this seems to suggest we have to be more
  restrictive on the set of processes and names we admit if we are to
  support no-cloning.]
\end{remark}

\subsubsection{Bisimulation}

The computational dynamics gives rise to another kind of equivalence,
the equivalence of computational behavior. As previously mentioned
this is typically captured \emph{via} some form of bisimulation.

% The notion we use in this paper is weak barbed bisimulation
% \cite{milner91polyadicpi}.

The notion we use in this paper is derived from weak barbed
bisimulation \cite{milner91polyadicpi}. 

\begin{definition}
An \emph{observation relation}, $\downarrow_{\mathcal N}$, over a set
of names, $\mathcal N$, is the smallest relation satisfying the rules
below.

\infrule[Out-barb]{y \in {\mathcal N}, \; x \nameeq y}
		  {\outputp{x}{v} \downarrow_{\mathcal N} x}
\infrule[Par-barb]{\mbox{$P\downarrow_{\mathcal N} x$ or $Q\downarrow_{\mathcal N} x$}}
		  {\binpar{P}{Q} \downarrow_{\mathcal N} x}

We write $P \Downarrow_{\mathcal N} x$ if there is $Q$ such that 
$P \wred Q$ and $Q \downarrow_{\mathcal N} x$.
\end{definition}

\begin{definition}
%\label{def.bbisim}
An  ${\mathcal N}$-\emph{barbed bisimulation} over a set of names, ${\mathcal N}$, is a symmetric binary relation 
${\mathcal S}_{\mathcal N}$ between agents such that $P\rel{S}_{\mathcal N}Q$ implies:
\begin{enumerate}
\item If $P \red P'$ then $Q \wred Q'$ and $P'\rel{S}_{\mathcal N} Q'$.
\item If $P\downarrow_{\mathcal N} x$, then $Q\Downarrow_{\mathcal N} x$.
\end{enumerate}
$P$ is ${\mathcal N}$-barbed bisimilar to $Q$, written
$P \wbbisim_{\mathcal N} Q$, if $P \rel{S}_{\mathcal N} Q$ for some ${\mathcal N}$-barbed bisimulation ${\mathcal S}_{\mathcal N}$.
\end{definition}

$\mathcal{R} \subseteq \pi \times \pi$

$P \mathcal{R} Q => \forall P'. P \red P' \Rightarrow \exists Q'. Q \red Q', P' \mathcal{R} Q'$

$P \vdash x \Rightarrow Q \vdash x$

\begin{mathpar}
  \inferrule*[lab=Out-barb]{x \nameeq y}{{y}!\langle{Q}\rangle \vdash x}
  \and
  \inferrule*[lab=Par-barb]{\mbox{$P\vdash x$ or $Q\vdash x$}}{\binpar{P}{Q} \vdash x}
\end{mathpar}

\subsubsection{Contexts}

One of the principle advantages of computational calculi like the
$\pi$-calculus is a well-defined notion of context,
contextual-equivalence and a correlation between
contextual-equivalence and notions of bisimulation. The notion of
context allows the decomposition of a process into (sub-)process and
its syntactic environment, its context. Thus, a context may be
thought of as a process with a ``hole'' (written $\Box$) in it. The
application of a context $M$ to a process $P$, written $M[P]$, is
tantamount to filling the hole in $M$ with $P$. In this paper we do
not need the full weight of this theory, but do make use of the notion
of context in the proof the main theorem. 

\begin{mathpar}
  \inferrule* [lab=summation] {} {{M_{M},M_{N}} \bc \Box \;|\; x.M_{A} \;|\; M_{M}+M_{N}}
  \and
  \inferrule* [lab=agent] {} {{M_{A}} \bc (\vec{x})M_{P} \;| \; \clift{P_0,\ldots,M_{P},\ldots,P_N}}
  \and \\
  \inferrule* [lab=process] {} {{M_{P}} \bc M_{N} \;| \;P|M_{P} }
\end{mathpar} 

\begin{mathpar}
  \inferrule* [lab=sychronization] {} {M_{N} \bc \Box \;|\; x?M_{F} \;|\; x!M_{C}}
  \and
  \inferrule* [lab=abstraction] {} {{M_{F}} \bc (x)M_{P} }
  \and
  \inferrule* [lab=concretion] {} {{M_{C}} \bc \langle M_{P} \rangle }
  \and \\
  \inferrule* [lab=process] {} {{M_{P}} \bc M_{N} \;| \;P|M_{P} }
\end{mathpar}

\begin{definition}[contextual application] Given a context $M$, and
  process $P$, we define the \emph{contextual application}, $M[P] :=
  M\{P/\Box\}$. That is, the contextual application of M to P is the
  substitution of $P$ for $\Box$ in $M$.
\end{definition}

$\meaningof{-} : L \to \mathcal{P}(\pi)$

\begin{mathpar}
  \inferrule* [lab=collection] {} {\meaningof{true} = \pi, \and \meaningof{~E} = \pi \setminus \meaningof{E}, \and \meaningof{E_{1} \& E_{2}} = \meaningof{E_{1}} \cap \meaningof{E_{2}}}
\end{mathpar}

\begin{mathpar}
  \inferrule* [lab=structure] {} {\meaningof{0} = \{ P \in \pi | P \equiv 0 \}, \and \\ \meaningof{E_1 | E_2} = \{ P \in \pi | P \equiv P_{1} | P_{2}, P_{1} \in \meaningof{E_{1}}, P_{2} \in \meaningof{E_2}\} }
\end{mathpar}

\begin{mathpar}
 \inferrule* [lab=behavior] {} {\meaningof{\langle a?b \rangle E} = \{ P \in \pi | P \equiv Q | u?(y)P', \\ \and \\\\ \and \\ \;\;\; u \in \meaningof{a}, \forall z.P'\{z/y\} \in \meaningof{E\{z/b\}}\}, \and \\ \meaningof{a!E} = \{ P \in \pi | P \equiv Q | x!\langle P' \rangle, x \in \meaningof{a} P' \in \meaningof{E}\} }
\end{mathpar}

\begin{mathpar}
 \inferrule* [lab=nominal] {} {\meaningof{\quotep{E}} = \{ \quotep{P} \in \quotep{\pi} | P \in \meaningof{E} \}, \and \meaningof{\quotep{P}} = \{ \quotep{Q} \in \quotep{\pi} | P \equiv Q \} \and \\ \meaningof{@\quotep{E}} = \{ P \in \pi | P \equiv @x, x \in \meaningof{E} \}}
\end{mathpar}

\begin{eqnarray*}
  \\
  \meaningof{-} : TS \to ST
\end{eqnarray*}

\begin{eqnarray*}
  \\
  L : TS \to ST
\end{eqnarray*}

\begin{eqnarray*}
  \\
  P \models E \iff P \in \meaningof{E}
\end{eqnarray*}

\begin{eqnarray*}
  P \approx_{L} Q \iff \forall E \in L. P \models E \iff Q \models E
\end{eqnarray*}

\begin{eqnarray*}
  P \approx_{K} Q
\end{eqnarray*}

\begin{eqnarray*}
  P \approx Q
\end{eqnarray*}

$\approx_{K} = \approx = \approx_{L}$

\subsubsection{Contextual duality}

Note that contexts extend the quotation operation to a family of
operations from processes to names. Given a context, $M$, we can
define a \emph{nominal context}, $\quotep{M}$ by $\quotep{M}[P] :=
\quotep{M[P]}$. To foreshadow what is to come we observe that these
operations enjoy a duality with processes very much like the duality
between vectors and maps from vectors to scalars.

Further, because the calculus is essentially higher-order, we have a
correspondence between contexts and processes. More specifically,
given a name $x$ and a context $M$ we can construct $M^{*}_{x}$ such
that 

\begin{mathpar}
  M^{*}_{x} | \lift{x}{P} \red M[P]
\end{mathpar}

namely,

\begin{mathpar}
  M^{*}_{x} := x?(u).M[\dropn{u}]
\end{mathpar}

The dependence of $M^{*}_{x}$ on a name makes it an abstraction, 

\begin{mathpar}
  M^{*} := (x)x?(u).M[\dropn{u}]
\end{mathpar}

\subsection{Additional notation}

It will sometimes be convenient to denote the process a name
quotes. We already have the notation $x = \quotep{P}$, but it will be
convenient to introduce an alternate notation, $\procn{x}$, when we
want to emphasize the connection to the use of the name. Note that, by
virtue of name equivalence, $\quotep{\procn{x}} \nameeq x$; so, the
notation is consistent with previous definitions.

Further, because names have structure it is possible to effect
substitutions on the basis of that structure. This means we need to
upgrade our notation for substitutions, which we accomplish by
adapting comprehension notation. Thus,

\begin{mathpar}
  P\{ y / x : x \in S \}
\end{mathpar}

is interpreted to mean the process derived from P by replacing (in a
capture-avoiding manner) each occurrence of $x$ in $S$ by $y$. For example,

\begin{mathpar}
  P\{ \quotep{\procn{x}|\procn{x}} / x : x \in \freenames{P} \}
\end{mathpar}

will replace each (occurrence) of a free name $x$ in $P$ by
$\quotep{\procn{x}|\procn{x}}$.

Also, we will avail ourselves of the notation $x^{L}$ and $x^{R}$ to
denote injections of a name into disjoint copies of the name
space. There are numerous ways to accomplish this. One example can be
found in \cite{MeredithR05}. This notation overloads to vectors of
names: $\vec{x}^{\pi} := (x_{i}^{\pi} \; : \; 0 \leq i < |\vec{x}| )$ where $\pi \in \{L,R\}$.

We also use $P^{\Box} := P|\Box$.

In \cite{MeredithR05} an interpretation of the new operator is
given. It turns out that there are several possible interpretations
all enjoying the requisite algebraic properties of the operator (see
\cite{milner91polyadicpi}). We will therefore make liberal use of
$(\nu\; \vec{x})P$.

% subsection the_syntax_and_semantics_of_the_notation_system (end)   

\input{qm2pi.qmops} 

\input{qm2pi.sterngerlach} 

\input{qm2pi.metric} 

% section concurrent_process_calculi (end)

%\input{qm2pi.proofsketch}

% section proof sketch (end)

%\input{qm2pi.slviaknots} 

% section spatial logic via knots (end)

\input{qm2pi.conclusion}

% section conclusion (end)

%\input{qm2pi.dtcodes} 

% section wiring algorithm (end)

\input{qm2pi.ack} 

% section acknowledgments (end)

\newpage


\bibliographystyle{plain}   
\bibliography{../../biblios/main.bib}

\input{qm2pi.rhodetails}

\end{document}

 

% section acknowledgments (end)

\newpage


\bibliographystyle{plain}   
\bibliography{../../biblios/main.bib}

\documentclass[12pt]{llncs}
%\documentclass{jktr}

\usepackage[pdftex]{hyperref}                   
\usepackage {listings}
\usepackage {mathpartir}
\usepackage{bcprules}
%\usepackage{listings}
                       
\usepackage{graphicx} 
%\usepackage[margins=2.5cm,nohead,nofoot]{geometry}
%\usepackage{geometry}
\usepackage{amsfonts}
\usepackage{amstext}
\usepackage{latexsym}
\usepackage{amssymb}
\usepackage{color}


%\include{myPreamble}
\include{qm2pi.local} 

%\ifpdf
%\usepackage[pdftex]{graphicx}
%\else
%\usepackage{graphicx}
%\fi

 % \ifpdf
%  \usepackage{pdfsync}
%  \if


%\title{Brief Article}
%\author{David F. Snyder}
%\author{L.G. Meredith}

%\address{Dept. of Math., Texas State University--San Marcos, San Marcos, TX 78666}
       
\pagestyle{empty}


\begin{document}

\lstset{language=[Objective]Caml,frame=shadowbox}

\input{qm2pi.front}

% section front matter (end)

\input{qm2pi.intro} 
 
% section introduction (end)

% \input{qm2pi.knotations} 

% section notation (end)

\input{qm2pi.process.calculi} 

% section concurrent_process_calculi_and_spatial_logics_ (end)
    
%\input{qm2pi.knots2pi} 

%\input{qm2pi.trefoil} 

%\input{qm2pi.mainthm} 

% subsection basic_interpretation (end)

%\input{qm2pi.rho.presentation} 
\subsection{The syntax and semantics of the notation system}\label{sub:the_syntax_and_semantics_of_the_notation_system} % (fold)

We now summarize a technical presentation of the calculus that
embodies our theory of dynamics. The typical presentation of such a
calculus follows the style of giving generators and relations on
them. The grammar, below, describing term constructors, freely
generates the set of processes, $\Proc$. This set is then quotiented
by a relation known as structural congruence and it is over this set
that the notion of dynamics is expressed. This presentation is
essentially that of \cite{MeredithR05} with the addition of
polyadicity and summation. For readability we have relegated some of
the technical subtleties to an appendix.

\subsubsection{Process grammar}\label{subsub:process_grammar}

\begin{mathpar}
  \inferrule* [lab=synchronization] {} {{M} \bc \pzero \;|\; x?F \;|\; x!C }
  \and
  \inferrule* [lab=abstraction] {} {{F} \bc (x)P}
  \and
  \inferrule* [lab=concretion] {} {{C} \bc \langle Q \rangle}
  \and
  \inferrule* [lab=process] {} {{P,Q} \bc M \;| \;P|Q \;|\; @{x}}
  \and
  \inferrule* [lab=name] {} {{x} \bc \quotep{P}}
\end{mathpar} 

Note that $\vec{x}$ (resp. $\vec{P}$) denotes a vector of names
(resp. processes) of length $|\vec{x}|$ (resp. $|\vec{P}|$). We adopt
the following useful abbreviations.

\begin{mathpar}
   x?(\vec{y}).P := x.(\vec{y})P \and  x\clift{\vec{P}} := x.\clift{\vec{P}}
   \and x!(y) := \lift{x}{\dropn{y}}
   \and \Pi_{i=0}^{n-1}P_i := P_0 | \ldots | P_{n-1}
\end{mathpar}

\subsubsection{Structural congruence}

\paragraph{Free and bound names and alpha-equivalence.} At the
core of structural equivalence is alpha-equivalence which identifies
process that are the same up to a change of variable. Formally, we
recognize the distinction between free and bound names. The free names
of a process, $\freenames{P}$, may be calculated recursively as
follows:

\begin{mathpar}
\freenames{\pzero} := \emptyset
  \and \\
  \freenames{x?(y).P} := \{ x \} \cup (\freenames{P} \setminus \{ y \})
  \and 
  \freenames{x!\langle P \rangle} := \{ x \} \cup \{ P \} 
  \and \\
  \freenames{P|Q} := \freenames{P} \cup \freenames{Q}
  \and \\
  \freenames{@{x}} := \{ x \}
\end{mathpar}

$\pi$
$\quotep{\pi}$

$\freenames{-} : \pi \to \mathcal{P}(\quotep{\pi})$

\begin{eqnarray*}
  \freenames{\pzero} & := & \emptyset \\
  \freenames{x?(y).P} & := & \{ x \} \cup (\freenames{P} \setminus \{ y \}) \\
  \freenames{x!\langle P \rangle} & := & \{ x \} \cup \{ P \} \\
  \freenames{P|Q} & := & \freenames{P} \cup \freenames{Q} \\
  \freenames{\dropn{x}} & := & \{ x \}
\end{eqnarray*}

The bound names of a process, $\boundnames{P}$, are those names occurring in $P$
that are not free. For example, in $x?(y).0$, the name $x$ is free, while $y$ is bound.

\begin{mathpar}
  \inferrule* [lab=monoidal-laws] {} { P|Q \equiv Q|P \and P|0 \equiv P \and P|(Q|R) \equiv (P|Q)|R }
\end{mathpar}

\begin{mathpar}
  \inferrule* [lab=alpha-equivalence] {} { (x)P \equiv (y)P\{y/x\} \and y \not\in \freenames{P} }
\end{mathpar}

\begin{definition}
Then two processes, $P,Q$, are alpha-equivalent if $P = Q\{\vec{y}/\vec{x}\}$ for
some $\vec{x} \in \boundnames{Q},\vec{y} \in \boundnames{P}$, where $Q\{\vec{y}/\vec{x}\}$
denotes the capture-avoiding substitution of $\vec{y}$ for $\vec{x}$ in $Q$.
\end{definition}

\begin{definition}
  The {\em structural congruence} \cite{SangiorgiWalker} , $\equiv$,
  between processes is the least congruence containing
  alpha-equivalence, satisfying the abelian monoid laws
  (associativity, commutativity and $\pzero$ as identity) for parallel
  composition $|$ and for summation $+$.
\end{definition}

\subsection{Name equivalence}

We take name equivalence, written $\nameeq$, to be the smallest
equivalence relation generated by the following rules.

\begin{mathpar}
\inferrule*[lab=Quote-drop]
{ }
{ \quotep{@{x}} \nameeq x }

\inferrule*[lab=Struct-equiv]
{ P \scong Q }
{ \quotep{P} \nameeq \quotep{Q} }
\end{mathpar}

The astute reader will have noticed that the mutual recursion of names
and processes imposes a mutual recursion on alpha-equivalence and
structural equivalence via name-equivalence. Fortunately, all of this
works out pleasantly and we may calculate in the natural way, free of
concern. The reader interested in the details is referred to the
appendix \ref{appendix:rho_details}.

\subsection{Substitution}

We use $\Proc$ for the set of processes, $\QProc$ for the set of
names, and $\id{\{}\vec{y} / \vec{x} \id{\}}$ to denote partial maps,
$s : \QProc \rightarrow \QProc$. A map, $s$ lifts, uniquely, to a map
on process terms, $\widehat{s} : \Proc \rightarrow \Proc$ by the
following equations.

\begin{mathpar}
  (0) \psubstp{Q}{P} := 0 \\
  (R \juxtap S) \psubstp{Q}{P}
  :=    
  (R)\psubstp{Q}{P} \juxtap (S) \psubstp{Q}{P} \\
  (x?(y).R) \psubstp{Q}{P}    
  :=    
  (x)\substp{Q}{P} (z)\concat( (R \psubstn{z}{y}) \psubstp{Q}{P} ) \\
  (\lift{x}{R}) \psubstp{Q}{P}  
  :=
  \lift{(x)\substp{Q}{P}}{ R \psubstp{Q}{P} } \\
%   (\dropn{x})  \psubstp{Q}{P}       
%   := 
%   \left\{ 
%     \begin{array}{ccc} 
%       \dropn{\quotep{Q}} & & x \nameeq \quotep{P} \\
%       \dropn{x} & & otherwise \\
%     \end{array}
%   \right. 
  (\dropn{x})  \psubstp{Q}{P}       
  := 
  \left\{ 
    \begin{array}{ccc} 
      Q & & x \nameeq \quotep{P} \\
      \dropn{x} & & otherwise \\
    \end{array}
  \right.
\end{mathpar}
 

where

\begin{eqnarray}
  (x)\id{\{} \lpquote Q \rpquote / \lpquote P \rpquote \id{\}}            = 
  \left\{ 
    \begin{array}{ccc}
      \lpquote Q \rpquote & & x \nameeq \lpquote P \rpquote \\
      x & & otherwise \\
    \end{array}
  \right. \nonumber
\end{eqnarray}

and $z$ is chosen distinct from $\quotep{P}$, $\quotep{Q}$, the free
names in $Q$, and all the names in $R$. Our $\alpha$-equivalence will
be built in the standard way from this substitution.

\begin{remark}\label{rem:no_self_referential_names}
  One consequence of these definitions is that $\forall P. \quotep{P}
  \not\in \freenames{P}$.
\end{remark}

\subsection{ Dynamic quote: an example }

Anticipating something of what's to come, consider applying the
substitution, $\widehat{\id{\{}u / z \id{\}}}$, to the following pair
of processes, $\lift{w}{y!(z)}$ and $w[ \lpquote y!(z) \rpquote ]$.

\begin{eqnarray}
	\lift{w}{y!(z)}\widehat{\id{\{}u / z \id{\}}}
		& = &
		\lift{w}{y!(u)} \nonumber\\
	w[ \lpquote y!(z) \rpquote ] \widehat{ \id{\{}u / z \id{\}} }
		& = &
		w[ \lpquote y!(z) \rpquote ] \nonumber
\end{eqnarray}

Because the body of the process between quotes is impervious to
substitution, we get radically different answers. In fact, by
examining the first process in an input context,
e.g. $x?(z).\lift{w}{y!(z)}$, we see that the process under the lift
operator may be shaped by prefixed inputs binding a name inside it. In
this sense, the lift operator will be seen as a way to dynamically
construct processes before reifying them as names.

Finally equipped with these standard features we can present the
dynamics of the calculus.

\subsubsection{Operational semantics} 

Finally, we introduce the computational dynamics. What marks these
algebras as distinct from other more traditionally studied algebraic
structures, e.g. vector spaces or polynomial rings, is the manner in
which dynamics is captured. In traditional structures, dynamics is typically
expressed through morphisms between such structures, as in linear maps
between vector spaces or morphisms between rings. In algebras
associated with the semantics of computation, the dynamics is
expressed as part of the algebraic structure itself, through a
reduction reduction relation typically denoted by $\red$. Below, we
give a recursive presentation of this relation for the calculus used
in the encoding.

$\red \subseteq \pi \times \pi$
$\red : \pi \to \mathcal{P}(\pi)$

\begin{mathpar}
  \inferrule* [lab=Comm] { \textsf{match}( x_{src}, x_{trgt} ) } { x_{trgt}?(y)P \; | \; x_{src}!\langle {Q} \rangle \red P\{\quotep{Q}/y}\} }
  \and \\
  \inferrule* [lab=Par] {{P} \red {P}'} {{{P} | {Q}} \red {{P}' | {Q}}}
  \and
  \inferrule* [lab=Equiv]{{{P} \scong {P}'} \andalso {{P}' \red {Q}'} \andalso {{Q}' \scong {Q}}}{{P} \red {Q}}
\end{mathpar}

\begin{eqnarray*}
  match_{\equiv} (\quotep{P},\quotep{Q}) & := & P \equiv Q \\
  match_{\dagger}(\quotep{P},\quotep{Q}) & := & \forall R. P|Q \red^{*} R => R \red^{*} 0 \\
  match_{K}(\quotep{P},\quotep{Q}) & := & K \mbox{ for some context } K
\end{eqnarray*}

$u?(x)P | u!\langle Q \rangle \red P\{\quotep{Q}/x\}$

%We write $\wred$ for $\red^*$, and $P\red$ if $\exists Q $ such that $ P \red Q$.
We write $P\red$ if $\exists Q $ such that $ P \red Q$ and $P\not\red$, otherwise.

\section{Replication}

As mentioned before, it is known that replication (and hence
recursion) can be implemented in a higher-order process algebra
\cite{SangiorgiWalker}. As our first example of calculation with the
machinery thus far presented we give the construction explicitly in
the {\rhoc}.

\begin{eqnarray}
	D_{x} & := & \prefix{x}{y}{(\binpar{\outputp{x}{y}}{@{y}})} \nonumber\\
	\bangp_{x}{P} & := & \binpar{{x}!\langle{\binpar{D_{x}}{P}}\rangle}{D_{x}} \nonumber
\end{eqnarray}

\begin{eqnarray}
	\bangp_{x}{P} & & \nonumber\\
	=
	& {x}!\langle{(\prefix{x}{y}{(\outputp{x}{y} | @{y})) | P}}\rangle 
	      | \prefix{x}{y}{(\outputp{x}{y} | @{y})} & \nonumber\\
	\red
	& (\outputp{x}{y} | @{y})\substn{\quotep{(\prefix{x}{y}{(@{y} | \outputp{x}{y})) | P}}}{y} & \nonumber\\
	=
	& \outputp{x}{\quotep{(\prefix{x}{y}{(\outputp{x}{y} | @{y})) | P}}}
	  | {(\prefix{x}{y}{(\outputp{x}{y} | @{y})) | P}} & \nonumber\\
	\red
	& \ldots & \nonumber\\
	\red^*
	& P | P | \ldots & \nonumber
\end{eqnarray}

Of course, this encoding, as an implementation, runs away, unfolding
$\bangp{P}$ eagerly. A lazier and more implementable replication
operator, restricted to input-guarded processes, may be obtained as follows.

\begin{eqnarray}
\bangp{\prefix{u}{v}{P}} 
	:= 
	\binpar{\lift{x}{\prefix{u}{v}{(\binpar{D(x)}{P})}}}{D(x)} \nonumber
\end{eqnarray}

\begin{remark}
  Note that the lazier definition still does not deal with summation
  or mixed summation (i.e. sums over input and output). The reader is
  invited to construct definitions of replication that deal with these
  features. 

  Further, the definitions are parameterized in a name, $x$. Can you,
  gentle reader, make a definition that eliminates this parameter and
  guarantees no accidental interaction between the replication
  machinery and the process being replicated -- i.e. no accidental
  sharing of names used by the process to get its work done and the
  name(s) used by the replication to effect copying. This latter
  revision of the definition of replication is crucial to obtaining
  the expected identity $!!P \sim !P$.
\end{remark}

\begin{remark}\label{rem:paradoxical_combinator}
  The reader familiar with the lambda calculus will have noticed the
  similarity between $D$ and the paradoxical combinator.

  [Ed. note: the existence of this seems to suggest we have to be more
  restrictive on the set of processes and names we admit if we are to
  support no-cloning.]
\end{remark}

\subsubsection{Bisimulation}

The computational dynamics gives rise to another kind of equivalence,
the equivalence of computational behavior. As previously mentioned
this is typically captured \emph{via} some form of bisimulation.

% The notion we use in this paper is weak barbed bisimulation
% \cite{milner91polyadicpi}.

The notion we use in this paper is derived from weak barbed
bisimulation \cite{milner91polyadicpi}. 

\begin{definition}
An \emph{observation relation}, $\downarrow_{\mathcal N}$, over a set
of names, $\mathcal N$, is the smallest relation satisfying the rules
below.

\infrule[Out-barb]{y \in {\mathcal N}, \; x \nameeq y}
		  {\outputp{x}{v} \downarrow_{\mathcal N} x}
\infrule[Par-barb]{\mbox{$P\downarrow_{\mathcal N} x$ or $Q\downarrow_{\mathcal N} x$}}
		  {\binpar{P}{Q} \downarrow_{\mathcal N} x}

We write $P \Downarrow_{\mathcal N} x$ if there is $Q$ such that 
$P \wred Q$ and $Q \downarrow_{\mathcal N} x$.
\end{definition}

\begin{definition}
%\label{def.bbisim}
An  ${\mathcal N}$-\emph{barbed bisimulation} over a set of names, ${\mathcal N}$, is a symmetric binary relation 
${\mathcal S}_{\mathcal N}$ between agents such that $P\rel{S}_{\mathcal N}Q$ implies:
\begin{enumerate}
\item If $P \red P'$ then $Q \wred Q'$ and $P'\rel{S}_{\mathcal N} Q'$.
\item If $P\downarrow_{\mathcal N} x$, then $Q\Downarrow_{\mathcal N} x$.
\end{enumerate}
$P$ is ${\mathcal N}$-barbed bisimilar to $Q$, written
$P \wbbisim_{\mathcal N} Q$, if $P \rel{S}_{\mathcal N} Q$ for some ${\mathcal N}$-barbed bisimulation ${\mathcal S}_{\mathcal N}$.
\end{definition}

$\mathcal{R} \subseteq \pi \times \pi$

$P \mathcal{R} Q => \forall P'. P \red P' \Rightarrow \exists Q'. Q \red Q', P' \mathcal{R} Q'$

$P \vdash x \Rightarrow Q \vdash x$

\begin{mathpar}
  \inferrule*[lab=Out-barb]{x \nameeq y}{{y}!\langle{Q}\rangle \vdash x}
  \and
  \inferrule*[lab=Par-barb]{\mbox{$P\vdash x$ or $Q\vdash x$}}{\binpar{P}{Q} \vdash x}
\end{mathpar}

\subsubsection{Contexts}

One of the principle advantages of computational calculi like the
$\pi$-calculus is a well-defined notion of context,
contextual-equivalence and a correlation between
contextual-equivalence and notions of bisimulation. The notion of
context allows the decomposition of a process into (sub-)process and
its syntactic environment, its context. Thus, a context may be
thought of as a process with a ``hole'' (written $\Box$) in it. The
application of a context $M$ to a process $P$, written $M[P]$, is
tantamount to filling the hole in $M$ with $P$. In this paper we do
not need the full weight of this theory, but do make use of the notion
of context in the proof the main theorem. 

\begin{mathpar}
  \inferrule* [lab=summation] {} {{M_{M},M_{N}} \bc \Box \;|\; x.M_{A} \;|\; M_{M}+M_{N}}
  \and
  \inferrule* [lab=agent] {} {{M_{A}} \bc (\vec{x})M_{P} \;| \; \clift{P_0,\ldots,M_{P},\ldots,P_N}}
  \and \\
  \inferrule* [lab=process] {} {{M_{P}} \bc M_{N} \;| \;P|M_{P} }
\end{mathpar} 

\begin{mathpar}
  \inferrule* [lab=sychronization] {} {M_{N} \bc \Box \;|\; x?M_{F} \;|\; x!M_{C}}
  \and
  \inferrule* [lab=abstraction] {} {{M_{F}} \bc (x)M_{P} }
  \and
  \inferrule* [lab=concretion] {} {{M_{C}} \bc \langle M_{P} \rangle }
  \and \\
  \inferrule* [lab=process] {} {{M_{P}} \bc M_{N} \;| \;P|M_{P} }
\end{mathpar}

\begin{definition}[contextual application] Given a context $M$, and
  process $P$, we define the \emph{contextual application}, $M[P] :=
  M\{P/\Box\}$. That is, the contextual application of M to P is the
  substitution of $P$ for $\Box$ in $M$.
\end{definition}

$\meaningof{-} : L \to \mathcal{P}(\pi)$

\begin{mathpar}
  \inferrule* [lab=collection] {} {\meaningof{true} = \pi, \and \meaningof{~E} = \pi \setminus \meaningof{E}, \and \meaningof{E_{1} \& E_{2}} = \meaningof{E_{1}} \cap \meaningof{E_{2}}}
\end{mathpar}

\begin{mathpar}
  \inferrule* [lab=structure] {} {\meaningof{0} = \{ P \in \pi | P \equiv 0 \}, \and \\ \meaningof{E_1 | E_2} = \{ P \in \pi | P \equiv P_{1} | P_{2}, P_{1} \in \meaningof{E_{1}}, P_{2} \in \meaningof{E_2}\} }
\end{mathpar}

\begin{mathpar}
 \inferrule* [lab=behavior] {} {\meaningof{\langle a?b \rangle E} = \{ P \in \pi | P \equiv Q | u?(y)P', \\ \and \\\\ \and \\ \;\;\; u \in \meaningof{a}, \forall z.P'\{z/y\} \in \meaningof{E\{z/b\}}\}, \and \\ \meaningof{a!E} = \{ P \in \pi | P \equiv Q | x!\langle P' \rangle, x \in \meaningof{a} P' \in \meaningof{E}\} }
\end{mathpar}

\begin{mathpar}
 \inferrule* [lab=nominal] {} {\meaningof{\quotep{E}} = \{ \quotep{P} \in \quotep{\pi} | P \in \meaningof{E} \}, \and \meaningof{\quotep{P}} = \{ \quotep{Q} \in \quotep{\pi} | P \equiv Q \} \and \\ \meaningof{@\quotep{E}} = \{ P \in \pi | P \equiv @x, x \in \meaningof{E} \}}
\end{mathpar}

\begin{eqnarray*}
  \\
  \meaningof{-} : TS \to ST
\end{eqnarray*}

\begin{eqnarray*}
  \\
  L : TS \to ST
\end{eqnarray*}

\begin{eqnarray*}
  \\
  P \models E \iff P \in \meaningof{E}
\end{eqnarray*}

\begin{eqnarray*}
  P \approx_{L} Q \iff \forall E \in L. P \models E \iff Q \models E
\end{eqnarray*}

\begin{eqnarray*}
  P \approx_{K} Q
\end{eqnarray*}

\begin{eqnarray*}
  P \approx Q
\end{eqnarray*}

$\approx_{K} = \approx = \approx_{L}$

\subsubsection{Contextual duality}

Note that contexts extend the quotation operation to a family of
operations from processes to names. Given a context, $M$, we can
define a \emph{nominal context}, $\quotep{M}$ by $\quotep{M}[P] :=
\quotep{M[P]}$. To foreshadow what is to come we observe that these
operations enjoy a duality with processes very much like the duality
between vectors and maps from vectors to scalars.

Further, because the calculus is essentially higher-order, we have a
correspondence between contexts and processes. More specifically,
given a name $x$ and a context $M$ we can construct $M^{*}_{x}$ such
that 

\begin{mathpar}
  M^{*}_{x} | \lift{x}{P} \red M[P]
\end{mathpar}

namely,

\begin{mathpar}
  M^{*}_{x} := x?(u).M[\dropn{u}]
\end{mathpar}

The dependence of $M^{*}_{x}$ on a name makes it an abstraction, 

\begin{mathpar}
  M^{*} := (x)x?(u).M[\dropn{u}]
\end{mathpar}

\subsection{Additional notation}

It will sometimes be convenient to denote the process a name
quotes. We already have the notation $x = \quotep{P}$, but it will be
convenient to introduce an alternate notation, $\procn{x}$, when we
want to emphasize the connection to the use of the name. Note that, by
virtue of name equivalence, $\quotep{\procn{x}} \nameeq x$; so, the
notation is consistent with previous definitions.

Further, because names have structure it is possible to effect
substitutions on the basis of that structure. This means we need to
upgrade our notation for substitutions, which we accomplish by
adapting comprehension notation. Thus,

\begin{mathpar}
  P\{ y / x : x \in S \}
\end{mathpar}

is interpreted to mean the process derived from P by replacing (in a
capture-avoiding manner) each occurrence of $x$ in $S$ by $y$. For example,

\begin{mathpar}
  P\{ \quotep{\procn{x}|\procn{x}} / x : x \in \freenames{P} \}
\end{mathpar}

will replace each (occurrence) of a free name $x$ in $P$ by
$\quotep{\procn{x}|\procn{x}}$.

Also, we will avail ourselves of the notation $x^{L}$ and $x^{R}$ to
denote injections of a name into disjoint copies of the name
space. There are numerous ways to accomplish this. One example can be
found in \cite{MeredithR05}. This notation overloads to vectors of
names: $\vec{x}^{\pi} := (x_{i}^{\pi} \; : \; 0 \leq i < |\vec{x}| )$ where $\pi \in \{L,R\}$.

We also use $P^{\Box} := P|\Box$.

In \cite{MeredithR05} an interpretation of the new operator is
given. It turns out that there are several possible interpretations
all enjoying the requisite algebraic properties of the operator (see
\cite{milner91polyadicpi}). We will therefore make liberal use of
$(\nu\; \vec{x})P$.

% subsection the_syntax_and_semantics_of_the_notation_system (end)   

\input{qm2pi.qmops} 

\input{qm2pi.sterngerlach} 

\input{qm2pi.metric} 

% section concurrent_process_calculi (end)

%\input{qm2pi.proofsketch}

% section proof sketch (end)

%\input{qm2pi.slviaknots} 

% section spatial logic via knots (end)

\input{qm2pi.conclusion}

% section conclusion (end)

%\input{qm2pi.dtcodes} 

% section wiring algorithm (end)

\input{qm2pi.ack} 

% section acknowledgments (end)

\newpage


\bibliographystyle{plain}   
\bibliography{../../biblios/main.bib}

\input{qm2pi.rhodetails}

\end{document}



\end{document}

 

% section wiring algorithm (end)

\documentclass[12pt]{llncs}
%\documentclass{jktr}

\usepackage[pdftex]{hyperref}                   
\usepackage {listings}
\usepackage {mathpartir}
\usepackage{bcprules}
%\usepackage{listings}
                       
\usepackage{graphicx} 
%\usepackage[margins=2.5cm,nohead,nofoot]{geometry}
%\usepackage{geometry}
\usepackage{amsfonts}
\usepackage{amstext}
\usepackage{latexsym}
\usepackage{amssymb}
\usepackage{color}


%\include{myPreamble}
\documentclass[12pt]{llncs}
%\documentclass{jktr}

\usepackage[pdftex]{hyperref}                   
\usepackage {listings}
\usepackage {mathpartir}
\usepackage{bcprules}
%\usepackage{listings}
                       
\usepackage{graphicx} 
%\usepackage[margins=2.5cm,nohead,nofoot]{geometry}
%\usepackage{geometry}
\usepackage{amsfonts}
\usepackage{amstext}
\usepackage{latexsym}
\usepackage{amssymb}
\usepackage{color}


%\include{myPreamble}
\include{qm2pi.local} 

%\ifpdf
%\usepackage[pdftex]{graphicx}
%\else
%\usepackage{graphicx}
%\fi

 % \ifpdf
%  \usepackage{pdfsync}
%  \if


%\title{Brief Article}
%\author{David F. Snyder}
%\author{L.G. Meredith}

%\address{Dept. of Math., Texas State University--San Marcos, San Marcos, TX 78666}
       
\pagestyle{empty}


\begin{document}

\lstset{language=[Objective]Caml,frame=shadowbox}

\input{qm2pi.front}

% section front matter (end)

\input{qm2pi.intro} 
 
% section introduction (end)

% \input{qm2pi.knotations} 

% section notation (end)

\input{qm2pi.process.calculi} 

% section concurrent_process_calculi_and_spatial_logics_ (end)
    
%\input{qm2pi.knots2pi} 

%\input{qm2pi.trefoil} 

%\input{qm2pi.mainthm} 

% subsection basic_interpretation (end)

%\input{qm2pi.rho.presentation} 
\subsection{The syntax and semantics of the notation system}\label{sub:the_syntax_and_semantics_of_the_notation_system} % (fold)

We now summarize a technical presentation of the calculus that
embodies our theory of dynamics. The typical presentation of such a
calculus follows the style of giving generators and relations on
them. The grammar, below, describing term constructors, freely
generates the set of processes, $\Proc$. This set is then quotiented
by a relation known as structural congruence and it is over this set
that the notion of dynamics is expressed. This presentation is
essentially that of \cite{MeredithR05} with the addition of
polyadicity and summation. For readability we have relegated some of
the technical subtleties to an appendix.

\subsubsection{Process grammar}\label{subsub:process_grammar}

\begin{mathpar}
  \inferrule* [lab=synchronization] {} {{M} \bc \pzero \;|\; x?F \;|\; x!C }
  \and
  \inferrule* [lab=abstraction] {} {{F} \bc (x)P}
  \and
  \inferrule* [lab=concretion] {} {{C} \bc \langle Q \rangle}
  \and
  \inferrule* [lab=process] {} {{P,Q} \bc M \;| \;P|Q \;|\; @{x}}
  \and
  \inferrule* [lab=name] {} {{x} \bc \quotep{P}}
\end{mathpar} 

Note that $\vec{x}$ (resp. $\vec{P}$) denotes a vector of names
(resp. processes) of length $|\vec{x}|$ (resp. $|\vec{P}|$). We adopt
the following useful abbreviations.

\begin{mathpar}
   x?(\vec{y}).P := x.(\vec{y})P \and  x\clift{\vec{P}} := x.\clift{\vec{P}}
   \and x!(y) := \lift{x}{\dropn{y}}
   \and \Pi_{i=0}^{n-1}P_i := P_0 | \ldots | P_{n-1}
\end{mathpar}

\subsubsection{Structural congruence}

\paragraph{Free and bound names and alpha-equivalence.} At the
core of structural equivalence is alpha-equivalence which identifies
process that are the same up to a change of variable. Formally, we
recognize the distinction between free and bound names. The free names
of a process, $\freenames{P}$, may be calculated recursively as
follows:

\begin{mathpar}
\freenames{\pzero} := \emptyset
  \and \\
  \freenames{x?(y).P} := \{ x \} \cup (\freenames{P} \setminus \{ y \})
  \and 
  \freenames{x!\langle P \rangle} := \{ x \} \cup \{ P \} 
  \and \\
  \freenames{P|Q} := \freenames{P} \cup \freenames{Q}
  \and \\
  \freenames{@{x}} := \{ x \}
\end{mathpar}

$\pi$
$\quotep{\pi}$

$\freenames{-} : \pi \to \mathcal{P}(\quotep{\pi})$

\begin{eqnarray*}
  \freenames{\pzero} & := & \emptyset \\
  \freenames{x?(y).P} & := & \{ x \} \cup (\freenames{P} \setminus \{ y \}) \\
  \freenames{x!\langle P \rangle} & := & \{ x \} \cup \{ P \} \\
  \freenames{P|Q} & := & \freenames{P} \cup \freenames{Q} \\
  \freenames{\dropn{x}} & := & \{ x \}
\end{eqnarray*}

The bound names of a process, $\boundnames{P}$, are those names occurring in $P$
that are not free. For example, in $x?(y).0$, the name $x$ is free, while $y$ is bound.

\begin{mathpar}
  \inferrule* [lab=monoidal-laws] {} { P|Q \equiv Q|P \and P|0 \equiv P \and P|(Q|R) \equiv (P|Q)|R }
\end{mathpar}

\begin{mathpar}
  \inferrule* [lab=alpha-equivalence] {} { (x)P \equiv (y)P\{y/x\} \and y \not\in \freenames{P} }
\end{mathpar}

\begin{definition}
Then two processes, $P,Q$, are alpha-equivalent if $P = Q\{\vec{y}/\vec{x}\}$ for
some $\vec{x} \in \boundnames{Q},\vec{y} \in \boundnames{P}$, where $Q\{\vec{y}/\vec{x}\}$
denotes the capture-avoiding substitution of $\vec{y}$ for $\vec{x}$ in $Q$.
\end{definition}

\begin{definition}
  The {\em structural congruence} \cite{SangiorgiWalker} , $\equiv$,
  between processes is the least congruence containing
  alpha-equivalence, satisfying the abelian monoid laws
  (associativity, commutativity and $\pzero$ as identity) for parallel
  composition $|$ and for summation $+$.
\end{definition}

\subsection{Name equivalence}

We take name equivalence, written $\nameeq$, to be the smallest
equivalence relation generated by the following rules.

\begin{mathpar}
\inferrule*[lab=Quote-drop]
{ }
{ \quotep{@{x}} \nameeq x }

\inferrule*[lab=Struct-equiv]
{ P \scong Q }
{ \quotep{P} \nameeq \quotep{Q} }
\end{mathpar}

The astute reader will have noticed that the mutual recursion of names
and processes imposes a mutual recursion on alpha-equivalence and
structural equivalence via name-equivalence. Fortunately, all of this
works out pleasantly and we may calculate in the natural way, free of
concern. The reader interested in the details is referred to the
appendix \ref{appendix:rho_details}.

\subsection{Substitution}

We use $\Proc$ for the set of processes, $\QProc$ for the set of
names, and $\id{\{}\vec{y} / \vec{x} \id{\}}$ to denote partial maps,
$s : \QProc \rightarrow \QProc$. A map, $s$ lifts, uniquely, to a map
on process terms, $\widehat{s} : \Proc \rightarrow \Proc$ by the
following equations.

\begin{mathpar}
  (0) \psubstp{Q}{P} := 0 \\
  (R \juxtap S) \psubstp{Q}{P}
  :=    
  (R)\psubstp{Q}{P} \juxtap (S) \psubstp{Q}{P} \\
  (x?(y).R) \psubstp{Q}{P}    
  :=    
  (x)\substp{Q}{P} (z)\concat( (R \psubstn{z}{y}) \psubstp{Q}{P} ) \\
  (\lift{x}{R}) \psubstp{Q}{P}  
  :=
  \lift{(x)\substp{Q}{P}}{ R \psubstp{Q}{P} } \\
%   (\dropn{x})  \psubstp{Q}{P}       
%   := 
%   \left\{ 
%     \begin{array}{ccc} 
%       \dropn{\quotep{Q}} & & x \nameeq \quotep{P} \\
%       \dropn{x} & & otherwise \\
%     \end{array}
%   \right. 
  (\dropn{x})  \psubstp{Q}{P}       
  := 
  \left\{ 
    \begin{array}{ccc} 
      Q & & x \nameeq \quotep{P} \\
      \dropn{x} & & otherwise \\
    \end{array}
  \right.
\end{mathpar}
 

where

\begin{eqnarray}
  (x)\id{\{} \lpquote Q \rpquote / \lpquote P \rpquote \id{\}}            = 
  \left\{ 
    \begin{array}{ccc}
      \lpquote Q \rpquote & & x \nameeq \lpquote P \rpquote \\
      x & & otherwise \\
    \end{array}
  \right. \nonumber
\end{eqnarray}

and $z$ is chosen distinct from $\quotep{P}$, $\quotep{Q}$, the free
names in $Q$, and all the names in $R$. Our $\alpha$-equivalence will
be built in the standard way from this substitution.

\begin{remark}\label{rem:no_self_referential_names}
  One consequence of these definitions is that $\forall P. \quotep{P}
  \not\in \freenames{P}$.
\end{remark}

\subsection{ Dynamic quote: an example }

Anticipating something of what's to come, consider applying the
substitution, $\widehat{\id{\{}u / z \id{\}}}$, to the following pair
of processes, $\lift{w}{y!(z)}$ and $w[ \lpquote y!(z) \rpquote ]$.

\begin{eqnarray}
	\lift{w}{y!(z)}\widehat{\id{\{}u / z \id{\}}}
		& = &
		\lift{w}{y!(u)} \nonumber\\
	w[ \lpquote y!(z) \rpquote ] \widehat{ \id{\{}u / z \id{\}} }
		& = &
		w[ \lpquote y!(z) \rpquote ] \nonumber
\end{eqnarray}

Because the body of the process between quotes is impervious to
substitution, we get radically different answers. In fact, by
examining the first process in an input context,
e.g. $x?(z).\lift{w}{y!(z)}$, we see that the process under the lift
operator may be shaped by prefixed inputs binding a name inside it. In
this sense, the lift operator will be seen as a way to dynamically
construct processes before reifying them as names.

Finally equipped with these standard features we can present the
dynamics of the calculus.

\subsubsection{Operational semantics} 

Finally, we introduce the computational dynamics. What marks these
algebras as distinct from other more traditionally studied algebraic
structures, e.g. vector spaces or polynomial rings, is the manner in
which dynamics is captured. In traditional structures, dynamics is typically
expressed through morphisms between such structures, as in linear maps
between vector spaces or morphisms between rings. In algebras
associated with the semantics of computation, the dynamics is
expressed as part of the algebraic structure itself, through a
reduction reduction relation typically denoted by $\red$. Below, we
give a recursive presentation of this relation for the calculus used
in the encoding.

$\red \subseteq \pi \times \pi$
$\red : \pi \to \mathcal{P}(\pi)$

\begin{mathpar}
  \inferrule* [lab=Comm] { \textsf{match}( x_{src}, x_{trgt} ) } { x_{trgt}?(y)P \; | \; x_{src}!\langle {Q} \rangle \red P\{\quotep{Q}/y}\} }
  \and \\
  \inferrule* [lab=Par] {{P} \red {P}'} {{{P} | {Q}} \red {{P}' | {Q}}}
  \and
  \inferrule* [lab=Equiv]{{{P} \scong {P}'} \andalso {{P}' \red {Q}'} \andalso {{Q}' \scong {Q}}}{{P} \red {Q}}
\end{mathpar}

\begin{eqnarray*}
  match_{\equiv} (\quotep{P},\quotep{Q}) & := & P \equiv Q \\
  match_{\dagger}(\quotep{P},\quotep{Q}) & := & \forall R. P|Q \red^{*} R => R \red^{*} 0 \\
  match_{K}(\quotep{P},\quotep{Q}) & := & K \mbox{ for some context } K
\end{eqnarray*}

$u?(x)P | u!\langle Q \rangle \red P\{\quotep{Q}/x\}$

%We write $\wred$ for $\red^*$, and $P\red$ if $\exists Q $ such that $ P \red Q$.
We write $P\red$ if $\exists Q $ such that $ P \red Q$ and $P\not\red$, otherwise.

\section{Replication}

As mentioned before, it is known that replication (and hence
recursion) can be implemented in a higher-order process algebra
\cite{SangiorgiWalker}. As our first example of calculation with the
machinery thus far presented we give the construction explicitly in
the {\rhoc}.

\begin{eqnarray}
	D_{x} & := & \prefix{x}{y}{(\binpar{\outputp{x}{y}}{@{y}})} \nonumber\\
	\bangp_{x}{P} & := & \binpar{{x}!\langle{\binpar{D_{x}}{P}}\rangle}{D_{x}} \nonumber
\end{eqnarray}

\begin{eqnarray}
	\bangp_{x}{P} & & \nonumber\\
	=
	& {x}!\langle{(\prefix{x}{y}{(\outputp{x}{y} | @{y})) | P}}\rangle 
	      | \prefix{x}{y}{(\outputp{x}{y} | @{y})} & \nonumber\\
	\red
	& (\outputp{x}{y} | @{y})\substn{\quotep{(\prefix{x}{y}{(@{y} | \outputp{x}{y})) | P}}}{y} & \nonumber\\
	=
	& \outputp{x}{\quotep{(\prefix{x}{y}{(\outputp{x}{y} | @{y})) | P}}}
	  | {(\prefix{x}{y}{(\outputp{x}{y} | @{y})) | P}} & \nonumber\\
	\red
	& \ldots & \nonumber\\
	\red^*
	& P | P | \ldots & \nonumber
\end{eqnarray}

Of course, this encoding, as an implementation, runs away, unfolding
$\bangp{P}$ eagerly. A lazier and more implementable replication
operator, restricted to input-guarded processes, may be obtained as follows.

\begin{eqnarray}
\bangp{\prefix{u}{v}{P}} 
	:= 
	\binpar{\lift{x}{\prefix{u}{v}{(\binpar{D(x)}{P})}}}{D(x)} \nonumber
\end{eqnarray}

\begin{remark}
  Note that the lazier definition still does not deal with summation
  or mixed summation (i.e. sums over input and output). The reader is
  invited to construct definitions of replication that deal with these
  features. 

  Further, the definitions are parameterized in a name, $x$. Can you,
  gentle reader, make a definition that eliminates this parameter and
  guarantees no accidental interaction between the replication
  machinery and the process being replicated -- i.e. no accidental
  sharing of names used by the process to get its work done and the
  name(s) used by the replication to effect copying. This latter
  revision of the definition of replication is crucial to obtaining
  the expected identity $!!P \sim !P$.
\end{remark}

\begin{remark}\label{rem:paradoxical_combinator}
  The reader familiar with the lambda calculus will have noticed the
  similarity between $D$ and the paradoxical combinator.

  [Ed. note: the existence of this seems to suggest we have to be more
  restrictive on the set of processes and names we admit if we are to
  support no-cloning.]
\end{remark}

\subsubsection{Bisimulation}

The computational dynamics gives rise to another kind of equivalence,
the equivalence of computational behavior. As previously mentioned
this is typically captured \emph{via} some form of bisimulation.

% The notion we use in this paper is weak barbed bisimulation
% \cite{milner91polyadicpi}.

The notion we use in this paper is derived from weak barbed
bisimulation \cite{milner91polyadicpi}. 

\begin{definition}
An \emph{observation relation}, $\downarrow_{\mathcal N}$, over a set
of names, $\mathcal N$, is the smallest relation satisfying the rules
below.

\infrule[Out-barb]{y \in {\mathcal N}, \; x \nameeq y}
		  {\outputp{x}{v} \downarrow_{\mathcal N} x}
\infrule[Par-barb]{\mbox{$P\downarrow_{\mathcal N} x$ or $Q\downarrow_{\mathcal N} x$}}
		  {\binpar{P}{Q} \downarrow_{\mathcal N} x}

We write $P \Downarrow_{\mathcal N} x$ if there is $Q$ such that 
$P \wred Q$ and $Q \downarrow_{\mathcal N} x$.
\end{definition}

\begin{definition}
%\label{def.bbisim}
An  ${\mathcal N}$-\emph{barbed bisimulation} over a set of names, ${\mathcal N}$, is a symmetric binary relation 
${\mathcal S}_{\mathcal N}$ between agents such that $P\rel{S}_{\mathcal N}Q$ implies:
\begin{enumerate}
\item If $P \red P'$ then $Q \wred Q'$ and $P'\rel{S}_{\mathcal N} Q'$.
\item If $P\downarrow_{\mathcal N} x$, then $Q\Downarrow_{\mathcal N} x$.
\end{enumerate}
$P$ is ${\mathcal N}$-barbed bisimilar to $Q$, written
$P \wbbisim_{\mathcal N} Q$, if $P \rel{S}_{\mathcal N} Q$ for some ${\mathcal N}$-barbed bisimulation ${\mathcal S}_{\mathcal N}$.
\end{definition}

$\mathcal{R} \subseteq \pi \times \pi$

$P \mathcal{R} Q => \forall P'. P \red P' \Rightarrow \exists Q'. Q \red Q', P' \mathcal{R} Q'$

$P \vdash x \Rightarrow Q \vdash x$

\begin{mathpar}
  \inferrule*[lab=Out-barb]{x \nameeq y}{{y}!\langle{Q}\rangle \vdash x}
  \and
  \inferrule*[lab=Par-barb]{\mbox{$P\vdash x$ or $Q\vdash x$}}{\binpar{P}{Q} \vdash x}
\end{mathpar}

\subsubsection{Contexts}

One of the principle advantages of computational calculi like the
$\pi$-calculus is a well-defined notion of context,
contextual-equivalence and a correlation between
contextual-equivalence and notions of bisimulation. The notion of
context allows the decomposition of a process into (sub-)process and
its syntactic environment, its context. Thus, a context may be
thought of as a process with a ``hole'' (written $\Box$) in it. The
application of a context $M$ to a process $P$, written $M[P]$, is
tantamount to filling the hole in $M$ with $P$. In this paper we do
not need the full weight of this theory, but do make use of the notion
of context in the proof the main theorem. 

\begin{mathpar}
  \inferrule* [lab=summation] {} {{M_{M},M_{N}} \bc \Box \;|\; x.M_{A} \;|\; M_{M}+M_{N}}
  \and
  \inferrule* [lab=agent] {} {{M_{A}} \bc (\vec{x})M_{P} \;| \; \clift{P_0,\ldots,M_{P},\ldots,P_N}}
  \and \\
  \inferrule* [lab=process] {} {{M_{P}} \bc M_{N} \;| \;P|M_{P} }
\end{mathpar} 

\begin{mathpar}
  \inferrule* [lab=sychronization] {} {M_{N} \bc \Box \;|\; x?M_{F} \;|\; x!M_{C}}
  \and
  \inferrule* [lab=abstraction] {} {{M_{F}} \bc (x)M_{P} }
  \and
  \inferrule* [lab=concretion] {} {{M_{C}} \bc \langle M_{P} \rangle }
  \and \\
  \inferrule* [lab=process] {} {{M_{P}} \bc M_{N} \;| \;P|M_{P} }
\end{mathpar}

\begin{definition}[contextual application] Given a context $M$, and
  process $P$, we define the \emph{contextual application}, $M[P] :=
  M\{P/\Box\}$. That is, the contextual application of M to P is the
  substitution of $P$ for $\Box$ in $M$.
\end{definition}

$\meaningof{-} : L \to \mathcal{P}(\pi)$

\begin{mathpar}
  \inferrule* [lab=collection] {} {\meaningof{true} = \pi, \and \meaningof{~E} = \pi \setminus \meaningof{E}, \and \meaningof{E_{1} \& E_{2}} = \meaningof{E_{1}} \cap \meaningof{E_{2}}}
\end{mathpar}

\begin{mathpar}
  \inferrule* [lab=structure] {} {\meaningof{0} = \{ P \in \pi | P \equiv 0 \}, \and \\ \meaningof{E_1 | E_2} = \{ P \in \pi | P \equiv P_{1} | P_{2}, P_{1} \in \meaningof{E_{1}}, P_{2} \in \meaningof{E_2}\} }
\end{mathpar}

\begin{mathpar}
 \inferrule* [lab=behavior] {} {\meaningof{\langle a?b \rangle E} = \{ P \in \pi | P \equiv Q | u?(y)P', \\ \and \\\\ \and \\ \;\;\; u \in \meaningof{a}, \forall z.P'\{z/y\} \in \meaningof{E\{z/b\}}\}, \and \\ \meaningof{a!E} = \{ P \in \pi | P \equiv Q | x!\langle P' \rangle, x \in \meaningof{a} P' \in \meaningof{E}\} }
\end{mathpar}

\begin{mathpar}
 \inferrule* [lab=nominal] {} {\meaningof{\quotep{E}} = \{ \quotep{P} \in \quotep{\pi} | P \in \meaningof{E} \}, \and \meaningof{\quotep{P}} = \{ \quotep{Q} \in \quotep{\pi} | P \equiv Q \} \and \\ \meaningof{@\quotep{E}} = \{ P \in \pi | P \equiv @x, x \in \meaningof{E} \}}
\end{mathpar}

\begin{eqnarray*}
  \\
  \meaningof{-} : TS \to ST
\end{eqnarray*}

\begin{eqnarray*}
  \\
  L : TS \to ST
\end{eqnarray*}

\begin{eqnarray*}
  \\
  P \models E \iff P \in \meaningof{E}
\end{eqnarray*}

\begin{eqnarray*}
  P \approx_{L} Q \iff \forall E \in L. P \models E \iff Q \models E
\end{eqnarray*}

\begin{eqnarray*}
  P \approx_{K} Q
\end{eqnarray*}

\begin{eqnarray*}
  P \approx Q
\end{eqnarray*}

$\approx_{K} = \approx = \approx_{L}$

\subsubsection{Contextual duality}

Note that contexts extend the quotation operation to a family of
operations from processes to names. Given a context, $M$, we can
define a \emph{nominal context}, $\quotep{M}$ by $\quotep{M}[P] :=
\quotep{M[P]}$. To foreshadow what is to come we observe that these
operations enjoy a duality with processes very much like the duality
between vectors and maps from vectors to scalars.

Further, because the calculus is essentially higher-order, we have a
correspondence between contexts and processes. More specifically,
given a name $x$ and a context $M$ we can construct $M^{*}_{x}$ such
that 

\begin{mathpar}
  M^{*}_{x} | \lift{x}{P} \red M[P]
\end{mathpar}

namely,

\begin{mathpar}
  M^{*}_{x} := x?(u).M[\dropn{u}]
\end{mathpar}

The dependence of $M^{*}_{x}$ on a name makes it an abstraction, 

\begin{mathpar}
  M^{*} := (x)x?(u).M[\dropn{u}]
\end{mathpar}

\subsection{Additional notation}

It will sometimes be convenient to denote the process a name
quotes. We already have the notation $x = \quotep{P}$, but it will be
convenient to introduce an alternate notation, $\procn{x}$, when we
want to emphasize the connection to the use of the name. Note that, by
virtue of name equivalence, $\quotep{\procn{x}} \nameeq x$; so, the
notation is consistent with previous definitions.

Further, because names have structure it is possible to effect
substitutions on the basis of that structure. This means we need to
upgrade our notation for substitutions, which we accomplish by
adapting comprehension notation. Thus,

\begin{mathpar}
  P\{ y / x : x \in S \}
\end{mathpar}

is interpreted to mean the process derived from P by replacing (in a
capture-avoiding manner) each occurrence of $x$ in $S$ by $y$. For example,

\begin{mathpar}
  P\{ \quotep{\procn{x}|\procn{x}} / x : x \in \freenames{P} \}
\end{mathpar}

will replace each (occurrence) of a free name $x$ in $P$ by
$\quotep{\procn{x}|\procn{x}}$.

Also, we will avail ourselves of the notation $x^{L}$ and $x^{R}$ to
denote injections of a name into disjoint copies of the name
space. There are numerous ways to accomplish this. One example can be
found in \cite{MeredithR05}. This notation overloads to vectors of
names: $\vec{x}^{\pi} := (x_{i}^{\pi} \; : \; 0 \leq i < |\vec{x}| )$ where $\pi \in \{L,R\}$.

We also use $P^{\Box} := P|\Box$.

In \cite{MeredithR05} an interpretation of the new operator is
given. It turns out that there are several possible interpretations
all enjoying the requisite algebraic properties of the operator (see
\cite{milner91polyadicpi}). We will therefore make liberal use of
$(\nu\; \vec{x})P$.

% subsection the_syntax_and_semantics_of_the_notation_system (end)   

\input{qm2pi.qmops} 

\input{qm2pi.sterngerlach} 

\input{qm2pi.metric} 

% section concurrent_process_calculi (end)

%\input{qm2pi.proofsketch}

% section proof sketch (end)

%\input{qm2pi.slviaknots} 

% section spatial logic via knots (end)

\input{qm2pi.conclusion}

% section conclusion (end)

%\input{qm2pi.dtcodes} 

% section wiring algorithm (end)

\input{qm2pi.ack} 

% section acknowledgments (end)

\newpage


\bibliographystyle{plain}   
\bibliography{../../biblios/main.bib}

\input{qm2pi.rhodetails}

\end{document}

 

%\ifpdf
%\usepackage[pdftex]{graphicx}
%\else
%\usepackage{graphicx}
%\fi

 % \ifpdf
%  \usepackage{pdfsync}
%  \if


%\title{Brief Article}
%\author{David F. Snyder}
%\author{L.G. Meredith}

%\address{Dept. of Math., Texas State University--San Marcos, San Marcos, TX 78666}
       
\pagestyle{empty}


\begin{document}

\lstset{language=[Objective]Caml,frame=shadowbox}

\documentclass[12pt]{llncs}
%\documentclass{jktr}

\usepackage[pdftex]{hyperref}                   
\usepackage {listings}
\usepackage {mathpartir}
\usepackage{bcprules}
%\usepackage{listings}
                       
\usepackage{graphicx} 
%\usepackage[margins=2.5cm,nohead,nofoot]{geometry}
%\usepackage{geometry}
\usepackage{amsfonts}
\usepackage{amstext}
\usepackage{latexsym}
\usepackage{amssymb}
\usepackage{color}


%\include{myPreamble}
\include{qm2pi.local} 

%\ifpdf
%\usepackage[pdftex]{graphicx}
%\else
%\usepackage{graphicx}
%\fi

 % \ifpdf
%  \usepackage{pdfsync}
%  \if


%\title{Brief Article}
%\author{David F. Snyder}
%\author{L.G. Meredith}

%\address{Dept. of Math., Texas State University--San Marcos, San Marcos, TX 78666}
       
\pagestyle{empty}


\begin{document}

\lstset{language=[Objective]Caml,frame=shadowbox}

\input{qm2pi.front}

% section front matter (end)

\input{qm2pi.intro} 
 
% section introduction (end)

% \input{qm2pi.knotations} 

% section notation (end)

\input{qm2pi.process.calculi} 

% section concurrent_process_calculi_and_spatial_logics_ (end)
    
%\input{qm2pi.knots2pi} 

%\input{qm2pi.trefoil} 

%\input{qm2pi.mainthm} 

% subsection basic_interpretation (end)

%\input{qm2pi.rho.presentation} 
\subsection{The syntax and semantics of the notation system}\label{sub:the_syntax_and_semantics_of_the_notation_system} % (fold)

We now summarize a technical presentation of the calculus that
embodies our theory of dynamics. The typical presentation of such a
calculus follows the style of giving generators and relations on
them. The grammar, below, describing term constructors, freely
generates the set of processes, $\Proc$. This set is then quotiented
by a relation known as structural congruence and it is over this set
that the notion of dynamics is expressed. This presentation is
essentially that of \cite{MeredithR05} with the addition of
polyadicity and summation. For readability we have relegated some of
the technical subtleties to an appendix.

\subsubsection{Process grammar}\label{subsub:process_grammar}

\begin{mathpar}
  \inferrule* [lab=synchronization] {} {{M} \bc \pzero \;|\; x?F \;|\; x!C }
  \and
  \inferrule* [lab=abstraction] {} {{F} \bc (x)P}
  \and
  \inferrule* [lab=concretion] {} {{C} \bc \langle Q \rangle}
  \and
  \inferrule* [lab=process] {} {{P,Q} \bc M \;| \;P|Q \;|\; @{x}}
  \and
  \inferrule* [lab=name] {} {{x} \bc \quotep{P}}
\end{mathpar} 

Note that $\vec{x}$ (resp. $\vec{P}$) denotes a vector of names
(resp. processes) of length $|\vec{x}|$ (resp. $|\vec{P}|$). We adopt
the following useful abbreviations.

\begin{mathpar}
   x?(\vec{y}).P := x.(\vec{y})P \and  x\clift{\vec{P}} := x.\clift{\vec{P}}
   \and x!(y) := \lift{x}{\dropn{y}}
   \and \Pi_{i=0}^{n-1}P_i := P_0 | \ldots | P_{n-1}
\end{mathpar}

\subsubsection{Structural congruence}

\paragraph{Free and bound names and alpha-equivalence.} At the
core of structural equivalence is alpha-equivalence which identifies
process that are the same up to a change of variable. Formally, we
recognize the distinction between free and bound names. The free names
of a process, $\freenames{P}$, may be calculated recursively as
follows:

\begin{mathpar}
\freenames{\pzero} := \emptyset
  \and \\
  \freenames{x?(y).P} := \{ x \} \cup (\freenames{P} \setminus \{ y \})
  \and 
  \freenames{x!\langle P \rangle} := \{ x \} \cup \{ P \} 
  \and \\
  \freenames{P|Q} := \freenames{P} \cup \freenames{Q}
  \and \\
  \freenames{@{x}} := \{ x \}
\end{mathpar}

$\pi$
$\quotep{\pi}$

$\freenames{-} : \pi \to \mathcal{P}(\quotep{\pi})$

\begin{eqnarray*}
  \freenames{\pzero} & := & \emptyset \\
  \freenames{x?(y).P} & := & \{ x \} \cup (\freenames{P} \setminus \{ y \}) \\
  \freenames{x!\langle P \rangle} & := & \{ x \} \cup \{ P \} \\
  \freenames{P|Q} & := & \freenames{P} \cup \freenames{Q} \\
  \freenames{\dropn{x}} & := & \{ x \}
\end{eqnarray*}

The bound names of a process, $\boundnames{P}$, are those names occurring in $P$
that are not free. For example, in $x?(y).0$, the name $x$ is free, while $y$ is bound.

\begin{mathpar}
  \inferrule* [lab=monoidal-laws] {} { P|Q \equiv Q|P \and P|0 \equiv P \and P|(Q|R) \equiv (P|Q)|R }
\end{mathpar}

\begin{mathpar}
  \inferrule* [lab=alpha-equivalence] {} { (x)P \equiv (y)P\{y/x\} \and y \not\in \freenames{P} }
\end{mathpar}

\begin{definition}
Then two processes, $P,Q$, are alpha-equivalent if $P = Q\{\vec{y}/\vec{x}\}$ for
some $\vec{x} \in \boundnames{Q},\vec{y} \in \boundnames{P}$, where $Q\{\vec{y}/\vec{x}\}$
denotes the capture-avoiding substitution of $\vec{y}$ for $\vec{x}$ in $Q$.
\end{definition}

\begin{definition}
  The {\em structural congruence} \cite{SangiorgiWalker} , $\equiv$,
  between processes is the least congruence containing
  alpha-equivalence, satisfying the abelian monoid laws
  (associativity, commutativity and $\pzero$ as identity) for parallel
  composition $|$ and for summation $+$.
\end{definition}

\subsection{Name equivalence}

We take name equivalence, written $\nameeq$, to be the smallest
equivalence relation generated by the following rules.

\begin{mathpar}
\inferrule*[lab=Quote-drop]
{ }
{ \quotep{@{x}} \nameeq x }

\inferrule*[lab=Struct-equiv]
{ P \scong Q }
{ \quotep{P} \nameeq \quotep{Q} }
\end{mathpar}

The astute reader will have noticed that the mutual recursion of names
and processes imposes a mutual recursion on alpha-equivalence and
structural equivalence via name-equivalence. Fortunately, all of this
works out pleasantly and we may calculate in the natural way, free of
concern. The reader interested in the details is referred to the
appendix \ref{appendix:rho_details}.

\subsection{Substitution}

We use $\Proc$ for the set of processes, $\QProc$ for the set of
names, and $\id{\{}\vec{y} / \vec{x} \id{\}}$ to denote partial maps,
$s : \QProc \rightarrow \QProc$. A map, $s$ lifts, uniquely, to a map
on process terms, $\widehat{s} : \Proc \rightarrow \Proc$ by the
following equations.

\begin{mathpar}
  (0) \psubstp{Q}{P} := 0 \\
  (R \juxtap S) \psubstp{Q}{P}
  :=    
  (R)\psubstp{Q}{P} \juxtap (S) \psubstp{Q}{P} \\
  (x?(y).R) \psubstp{Q}{P}    
  :=    
  (x)\substp{Q}{P} (z)\concat( (R \psubstn{z}{y}) \psubstp{Q}{P} ) \\
  (\lift{x}{R}) \psubstp{Q}{P}  
  :=
  \lift{(x)\substp{Q}{P}}{ R \psubstp{Q}{P} } \\
%   (\dropn{x})  \psubstp{Q}{P}       
%   := 
%   \left\{ 
%     \begin{array}{ccc} 
%       \dropn{\quotep{Q}} & & x \nameeq \quotep{P} \\
%       \dropn{x} & & otherwise \\
%     \end{array}
%   \right. 
  (\dropn{x})  \psubstp{Q}{P}       
  := 
  \left\{ 
    \begin{array}{ccc} 
      Q & & x \nameeq \quotep{P} \\
      \dropn{x} & & otherwise \\
    \end{array}
  \right.
\end{mathpar}
 

where

\begin{eqnarray}
  (x)\id{\{} \lpquote Q \rpquote / \lpquote P \rpquote \id{\}}            = 
  \left\{ 
    \begin{array}{ccc}
      \lpquote Q \rpquote & & x \nameeq \lpquote P \rpquote \\
      x & & otherwise \\
    \end{array}
  \right. \nonumber
\end{eqnarray}

and $z$ is chosen distinct from $\quotep{P}$, $\quotep{Q}$, the free
names in $Q$, and all the names in $R$. Our $\alpha$-equivalence will
be built in the standard way from this substitution.

\begin{remark}\label{rem:no_self_referential_names}
  One consequence of these definitions is that $\forall P. \quotep{P}
  \not\in \freenames{P}$.
\end{remark}

\subsection{ Dynamic quote: an example }

Anticipating something of what's to come, consider applying the
substitution, $\widehat{\id{\{}u / z \id{\}}}$, to the following pair
of processes, $\lift{w}{y!(z)}$ and $w[ \lpquote y!(z) \rpquote ]$.

\begin{eqnarray}
	\lift{w}{y!(z)}\widehat{\id{\{}u / z \id{\}}}
		& = &
		\lift{w}{y!(u)} \nonumber\\
	w[ \lpquote y!(z) \rpquote ] \widehat{ \id{\{}u / z \id{\}} }
		& = &
		w[ \lpquote y!(z) \rpquote ] \nonumber
\end{eqnarray}

Because the body of the process between quotes is impervious to
substitution, we get radically different answers. In fact, by
examining the first process in an input context,
e.g. $x?(z).\lift{w}{y!(z)}$, we see that the process under the lift
operator may be shaped by prefixed inputs binding a name inside it. In
this sense, the lift operator will be seen as a way to dynamically
construct processes before reifying them as names.

Finally equipped with these standard features we can present the
dynamics of the calculus.

\subsubsection{Operational semantics} 

Finally, we introduce the computational dynamics. What marks these
algebras as distinct from other more traditionally studied algebraic
structures, e.g. vector spaces or polynomial rings, is the manner in
which dynamics is captured. In traditional structures, dynamics is typically
expressed through morphisms between such structures, as in linear maps
between vector spaces or morphisms between rings. In algebras
associated with the semantics of computation, the dynamics is
expressed as part of the algebraic structure itself, through a
reduction reduction relation typically denoted by $\red$. Below, we
give a recursive presentation of this relation for the calculus used
in the encoding.

$\red \subseteq \pi \times \pi$
$\red : \pi \to \mathcal{P}(\pi)$

\begin{mathpar}
  \inferrule* [lab=Comm] { \textsf{match}( x_{src}, x_{trgt} ) } { x_{trgt}?(y)P \; | \; x_{src}!\langle {Q} \rangle \red P\{\quotep{Q}/y}\} }
  \and \\
  \inferrule* [lab=Par] {{P} \red {P}'} {{{P} | {Q}} \red {{P}' | {Q}}}
  \and
  \inferrule* [lab=Equiv]{{{P} \scong {P}'} \andalso {{P}' \red {Q}'} \andalso {{Q}' \scong {Q}}}{{P} \red {Q}}
\end{mathpar}

\begin{eqnarray*}
  match_{\equiv} (\quotep{P},\quotep{Q}) & := & P \equiv Q \\
  match_{\dagger}(\quotep{P},\quotep{Q}) & := & \forall R. P|Q \red^{*} R => R \red^{*} 0 \\
  match_{K}(\quotep{P},\quotep{Q}) & := & K \mbox{ for some context } K
\end{eqnarray*}

$u?(x)P | u!\langle Q \rangle \red P\{\quotep{Q}/x\}$

%We write $\wred$ for $\red^*$, and $P\red$ if $\exists Q $ such that $ P \red Q$.
We write $P\red$ if $\exists Q $ such that $ P \red Q$ and $P\not\red$, otherwise.

\section{Replication}

As mentioned before, it is known that replication (and hence
recursion) can be implemented in a higher-order process algebra
\cite{SangiorgiWalker}. As our first example of calculation with the
machinery thus far presented we give the construction explicitly in
the {\rhoc}.

\begin{eqnarray}
	D_{x} & := & \prefix{x}{y}{(\binpar{\outputp{x}{y}}{@{y}})} \nonumber\\
	\bangp_{x}{P} & := & \binpar{{x}!\langle{\binpar{D_{x}}{P}}\rangle}{D_{x}} \nonumber
\end{eqnarray}

\begin{eqnarray}
	\bangp_{x}{P} & & \nonumber\\
	=
	& {x}!\langle{(\prefix{x}{y}{(\outputp{x}{y} | @{y})) | P}}\rangle 
	      | \prefix{x}{y}{(\outputp{x}{y} | @{y})} & \nonumber\\
	\red
	& (\outputp{x}{y} | @{y})\substn{\quotep{(\prefix{x}{y}{(@{y} | \outputp{x}{y})) | P}}}{y} & \nonumber\\
	=
	& \outputp{x}{\quotep{(\prefix{x}{y}{(\outputp{x}{y} | @{y})) | P}}}
	  | {(\prefix{x}{y}{(\outputp{x}{y} | @{y})) | P}} & \nonumber\\
	\red
	& \ldots & \nonumber\\
	\red^*
	& P | P | \ldots & \nonumber
\end{eqnarray}

Of course, this encoding, as an implementation, runs away, unfolding
$\bangp{P}$ eagerly. A lazier and more implementable replication
operator, restricted to input-guarded processes, may be obtained as follows.

\begin{eqnarray}
\bangp{\prefix{u}{v}{P}} 
	:= 
	\binpar{\lift{x}{\prefix{u}{v}{(\binpar{D(x)}{P})}}}{D(x)} \nonumber
\end{eqnarray}

\begin{remark}
  Note that the lazier definition still does not deal with summation
  or mixed summation (i.e. sums over input and output). The reader is
  invited to construct definitions of replication that deal with these
  features. 

  Further, the definitions are parameterized in a name, $x$. Can you,
  gentle reader, make a definition that eliminates this parameter and
  guarantees no accidental interaction between the replication
  machinery and the process being replicated -- i.e. no accidental
  sharing of names used by the process to get its work done and the
  name(s) used by the replication to effect copying. This latter
  revision of the definition of replication is crucial to obtaining
  the expected identity $!!P \sim !P$.
\end{remark}

\begin{remark}\label{rem:paradoxical_combinator}
  The reader familiar with the lambda calculus will have noticed the
  similarity between $D$ and the paradoxical combinator.

  [Ed. note: the existence of this seems to suggest we have to be more
  restrictive on the set of processes and names we admit if we are to
  support no-cloning.]
\end{remark}

\subsubsection{Bisimulation}

The computational dynamics gives rise to another kind of equivalence,
the equivalence of computational behavior. As previously mentioned
this is typically captured \emph{via} some form of bisimulation.

% The notion we use in this paper is weak barbed bisimulation
% \cite{milner91polyadicpi}.

The notion we use in this paper is derived from weak barbed
bisimulation \cite{milner91polyadicpi}. 

\begin{definition}
An \emph{observation relation}, $\downarrow_{\mathcal N}$, over a set
of names, $\mathcal N$, is the smallest relation satisfying the rules
below.

\infrule[Out-barb]{y \in {\mathcal N}, \; x \nameeq y}
		  {\outputp{x}{v} \downarrow_{\mathcal N} x}
\infrule[Par-barb]{\mbox{$P\downarrow_{\mathcal N} x$ or $Q\downarrow_{\mathcal N} x$}}
		  {\binpar{P}{Q} \downarrow_{\mathcal N} x}

We write $P \Downarrow_{\mathcal N} x$ if there is $Q$ such that 
$P \wred Q$ and $Q \downarrow_{\mathcal N} x$.
\end{definition}

\begin{definition}
%\label{def.bbisim}
An  ${\mathcal N}$-\emph{barbed bisimulation} over a set of names, ${\mathcal N}$, is a symmetric binary relation 
${\mathcal S}_{\mathcal N}$ between agents such that $P\rel{S}_{\mathcal N}Q$ implies:
\begin{enumerate}
\item If $P \red P'$ then $Q \wred Q'$ and $P'\rel{S}_{\mathcal N} Q'$.
\item If $P\downarrow_{\mathcal N} x$, then $Q\Downarrow_{\mathcal N} x$.
\end{enumerate}
$P$ is ${\mathcal N}$-barbed bisimilar to $Q$, written
$P \wbbisim_{\mathcal N} Q$, if $P \rel{S}_{\mathcal N} Q$ for some ${\mathcal N}$-barbed bisimulation ${\mathcal S}_{\mathcal N}$.
\end{definition}

$\mathcal{R} \subseteq \pi \times \pi$

$P \mathcal{R} Q => \forall P'. P \red P' \Rightarrow \exists Q'. Q \red Q', P' \mathcal{R} Q'$

$P \vdash x \Rightarrow Q \vdash x$

\begin{mathpar}
  \inferrule*[lab=Out-barb]{x \nameeq y}{{y}!\langle{Q}\rangle \vdash x}
  \and
  \inferrule*[lab=Par-barb]{\mbox{$P\vdash x$ or $Q\vdash x$}}{\binpar{P}{Q} \vdash x}
\end{mathpar}

\subsubsection{Contexts}

One of the principle advantages of computational calculi like the
$\pi$-calculus is a well-defined notion of context,
contextual-equivalence and a correlation between
contextual-equivalence and notions of bisimulation. The notion of
context allows the decomposition of a process into (sub-)process and
its syntactic environment, its context. Thus, a context may be
thought of as a process with a ``hole'' (written $\Box$) in it. The
application of a context $M$ to a process $P$, written $M[P]$, is
tantamount to filling the hole in $M$ with $P$. In this paper we do
not need the full weight of this theory, but do make use of the notion
of context in the proof the main theorem. 

\begin{mathpar}
  \inferrule* [lab=summation] {} {{M_{M},M_{N}} \bc \Box \;|\; x.M_{A} \;|\; M_{M}+M_{N}}
  \and
  \inferrule* [lab=agent] {} {{M_{A}} \bc (\vec{x})M_{P} \;| \; \clift{P_0,\ldots,M_{P},\ldots,P_N}}
  \and \\
  \inferrule* [lab=process] {} {{M_{P}} \bc M_{N} \;| \;P|M_{P} }
\end{mathpar} 

\begin{mathpar}
  \inferrule* [lab=sychronization] {} {M_{N} \bc \Box \;|\; x?M_{F} \;|\; x!M_{C}}
  \and
  \inferrule* [lab=abstraction] {} {{M_{F}} \bc (x)M_{P} }
  \and
  \inferrule* [lab=concretion] {} {{M_{C}} \bc \langle M_{P} \rangle }
  \and \\
  \inferrule* [lab=process] {} {{M_{P}} \bc M_{N} \;| \;P|M_{P} }
\end{mathpar}

\begin{definition}[contextual application] Given a context $M$, and
  process $P$, we define the \emph{contextual application}, $M[P] :=
  M\{P/\Box\}$. That is, the contextual application of M to P is the
  substitution of $P$ for $\Box$ in $M$.
\end{definition}

$\meaningof{-} : L \to \mathcal{P}(\pi)$

\begin{mathpar}
  \inferrule* [lab=collection] {} {\meaningof{true} = \pi, \and \meaningof{~E} = \pi \setminus \meaningof{E}, \and \meaningof{E_{1} \& E_{2}} = \meaningof{E_{1}} \cap \meaningof{E_{2}}}
\end{mathpar}

\begin{mathpar}
  \inferrule* [lab=structure] {} {\meaningof{0} = \{ P \in \pi | P \equiv 0 \}, \and \\ \meaningof{E_1 | E_2} = \{ P \in \pi | P \equiv P_{1} | P_{2}, P_{1} \in \meaningof{E_{1}}, P_{2} \in \meaningof{E_2}\} }
\end{mathpar}

\begin{mathpar}
 \inferrule* [lab=behavior] {} {\meaningof{\langle a?b \rangle E} = \{ P \in \pi | P \equiv Q | u?(y)P', \\ \and \\\\ \and \\ \;\;\; u \in \meaningof{a}, \forall z.P'\{z/y\} \in \meaningof{E\{z/b\}}\}, \and \\ \meaningof{a!E} = \{ P \in \pi | P \equiv Q | x!\langle P' \rangle, x \in \meaningof{a} P' \in \meaningof{E}\} }
\end{mathpar}

\begin{mathpar}
 \inferrule* [lab=nominal] {} {\meaningof{\quotep{E}} = \{ \quotep{P} \in \quotep{\pi} | P \in \meaningof{E} \}, \and \meaningof{\quotep{P}} = \{ \quotep{Q} \in \quotep{\pi} | P \equiv Q \} \and \\ \meaningof{@\quotep{E}} = \{ P \in \pi | P \equiv @x, x \in \meaningof{E} \}}
\end{mathpar}

\begin{eqnarray*}
  \\
  \meaningof{-} : TS \to ST
\end{eqnarray*}

\begin{eqnarray*}
  \\
  L : TS \to ST
\end{eqnarray*}

\begin{eqnarray*}
  \\
  P \models E \iff P \in \meaningof{E}
\end{eqnarray*}

\begin{eqnarray*}
  P \approx_{L} Q \iff \forall E \in L. P \models E \iff Q \models E
\end{eqnarray*}

\begin{eqnarray*}
  P \approx_{K} Q
\end{eqnarray*}

\begin{eqnarray*}
  P \approx Q
\end{eqnarray*}

$\approx_{K} = \approx = \approx_{L}$

\subsubsection{Contextual duality}

Note that contexts extend the quotation operation to a family of
operations from processes to names. Given a context, $M$, we can
define a \emph{nominal context}, $\quotep{M}$ by $\quotep{M}[P] :=
\quotep{M[P]}$. To foreshadow what is to come we observe that these
operations enjoy a duality with processes very much like the duality
between vectors and maps from vectors to scalars.

Further, because the calculus is essentially higher-order, we have a
correspondence between contexts and processes. More specifically,
given a name $x$ and a context $M$ we can construct $M^{*}_{x}$ such
that 

\begin{mathpar}
  M^{*}_{x} | \lift{x}{P} \red M[P]
\end{mathpar}

namely,

\begin{mathpar}
  M^{*}_{x} := x?(u).M[\dropn{u}]
\end{mathpar}

The dependence of $M^{*}_{x}$ on a name makes it an abstraction, 

\begin{mathpar}
  M^{*} := (x)x?(u).M[\dropn{u}]
\end{mathpar}

\subsection{Additional notation}

It will sometimes be convenient to denote the process a name
quotes. We already have the notation $x = \quotep{P}$, but it will be
convenient to introduce an alternate notation, $\procn{x}$, when we
want to emphasize the connection to the use of the name. Note that, by
virtue of name equivalence, $\quotep{\procn{x}} \nameeq x$; so, the
notation is consistent with previous definitions.

Further, because names have structure it is possible to effect
substitutions on the basis of that structure. This means we need to
upgrade our notation for substitutions, which we accomplish by
adapting comprehension notation. Thus,

\begin{mathpar}
  P\{ y / x : x \in S \}
\end{mathpar}

is interpreted to mean the process derived from P by replacing (in a
capture-avoiding manner) each occurrence of $x$ in $S$ by $y$. For example,

\begin{mathpar}
  P\{ \quotep{\procn{x}|\procn{x}} / x : x \in \freenames{P} \}
\end{mathpar}

will replace each (occurrence) of a free name $x$ in $P$ by
$\quotep{\procn{x}|\procn{x}}$.

Also, we will avail ourselves of the notation $x^{L}$ and $x^{R}$ to
denote injections of a name into disjoint copies of the name
space. There are numerous ways to accomplish this. One example can be
found in \cite{MeredithR05}. This notation overloads to vectors of
names: $\vec{x}^{\pi} := (x_{i}^{\pi} \; : \; 0 \leq i < |\vec{x}| )$ where $\pi \in \{L,R\}$.

We also use $P^{\Box} := P|\Box$.

In \cite{MeredithR05} an interpretation of the new operator is
given. It turns out that there are several possible interpretations
all enjoying the requisite algebraic properties of the operator (see
\cite{milner91polyadicpi}). We will therefore make liberal use of
$(\nu\; \vec{x})P$.

% subsection the_syntax_and_semantics_of_the_notation_system (end)   

\input{qm2pi.qmops} 

\input{qm2pi.sterngerlach} 

\input{qm2pi.metric} 

% section concurrent_process_calculi (end)

%\input{qm2pi.proofsketch}

% section proof sketch (end)

%\input{qm2pi.slviaknots} 

% section spatial logic via knots (end)

\input{qm2pi.conclusion}

% section conclusion (end)

%\input{qm2pi.dtcodes} 

% section wiring algorithm (end)

\input{qm2pi.ack} 

% section acknowledgments (end)

\newpage


\bibliographystyle{plain}   
\bibliography{../../biblios/main.bib}

\input{qm2pi.rhodetails}

\end{document}



% section front matter (end)

\section{Introduction}\label{sec:introduction} % (fold)
In this draft of the material i am going to have to dispense with the
usual writing conventions adopted in papers on these topics. i'm going
to have adopt whatever tone i need at the time i'm writing up the
calculations. Sometimes this may be very conversational; others it may
be the barest mathematical grunts; others still it may be that i have
lifted text from one of my other papers because the exposition of some
point was better said there. i hope that my readers are not unduly put
out by this decision. i'm not doing this to flout convention or be
rebellious. i find these calculations very technically challenging. To
keep everything going technically, something has to give; i have to
let go of some cognitive burden. So, the academic writing style --
with all of its trade-offs in terms of facilitating technical
communication -- is what i'm letting go of. Perhaps subsequent drafts
can be tightened and polished, but for now, i'm going to speak as if
we were sitting together in a coffee shop with a laptop, wifi and a
pad of paper and a pencil.

So, here's what i have to say. We -- you and i, comfortably ensconced
in our coffee shop and well-equipped with our tools -- can realize and
carry out the calculations of quantum mechanics over a very different
formal theory of dynamics, a formal theory of dynamics that
corresponds to a theory of concurrent computation with
\emph{reflection}. It has the advantage that the underlying theory is
already `quantized', but supports analogues all of the continuuous
operations. Strikingly, this underlying theory has recently been
connected with a notion of metric that we can show, by calculating
together, coincides with the metric induced by the inner product.

There are a lot of reasons why you might be interested in seeing
calculations of this form. Here's why i'm interested. For the past
several centuries there has been no competitor to the ``Newtonian''
account of dynamics. As a result the predominant share of accounts of
dynamical systems and situations have had to be formulated in terms of
the Newtonian machinery. i view this as an intellectually dangerous
position to occupy. Everything, despite it's intrinsic shape, turns
into a nail to be hit with this hammer. Recently, however, the theory
of computation has matured to the point where we have candidates for
theories of dynamics that offer very different perspective on
reasoning about dynamical systems and situations. Testing these
candidates against very successful accounts of dynamical situations,
like quantum mechanics, is going to give us some sense of how mature
they are and some measure of the quality of these accounts of
dynamics.

\subsection{Summary of contributions and outline of paper}

So, we're going to develop an interpretation of the operations of
quantum mechanics normally interpreted by Hilbert spaces and
operators. We're going to do this over a theory of computation. Note
that this is very different than the usual quantum computation program
which develops notions of computation over quantum mechanics. Rather,
we are developing a story that aligns with Wheeler's slogan: It from
Bit. To do this we will first provide an account of the theory of
computation at play here. Then we will dive into a calculation-driven
interpretation of the operations of quantum mechanics.

The reason we take this approach is that -- until very recently --
there hasn't been an axiomatic account of quantum mechanics. As a
result there has been no sharp delineation of the mathematical theory
supporting interpretation of the physical theory and the physical
theory, itself. So, ambient features of the maths are free to be
exploited (or supressed) without a real accounting of their physical
relevance. There is no sharp statement ``here's the physical theory''
qua \emph{theory} and ``here's the mathematical interpretation''
enabling a judgment of how faithful the interpretation is -- apart
from experimental observation. When there is an axiomatic account we
can judge how well a given mathematical formalism supports an
interpretation of the axioms, independent of
experimentation. Likewise, we can judge how well we have captured our
physical evidence and experience with our axiomatics, independent of
any specific mathematical implementation, with accidental detail that
may or may not have physical significance. 

In lieu of a fully fleshed out and vetted axiomatic account of quantum
mechanics, interpreting the operational notions in service of modeling
physical systems will have to suffice. In other words, we are not in
the business of providing a model of Hilbert spaces and operators. We
are in the business of providing a model of quantum mechanics because
we are motivated by testing our notions of dynamics against physical
theory; and, the predictive calculations of the physical theory must
serve as the best formulation -- shy of a fully fleshed out axiomatic
account -- of the physical theory itself (as they have for scientific
theories since time immemorial). Put another way, despite a
whole-hearted commitment to an It-from-Bit ontology, we are firmly
aligned with the shut-up-and-calculate camp as the best way to obtain
results either from the physical perspective or as a quality assurance
measure of our fledgling theory of dynamics.

In detail, we present a reflective process calculus. Then we develop
intuitive correspondences between the notions available in this
calculus and the usual physical notions supporting quantum mechanical
calculations. Thus, 

\begin{table}[htp]
  \center{
    \fbox{
      \begin{tabular}{c|c}
        quantum mechanics & process calculus \\
        \hline
        scalar & name \\
        state vector & process \\
        dual & contextual duals \\
        matrix & formal sums of process-context-dual pairs \\
        orthogonality & process annihilation \\
        inner product & execution-formula + quoting
      \end{tabular}
    }
  }
  \caption{QM - process calculi correspondences}
\end{table}

Then we tighten up these intuitions to operational definitions. We
employ the Dirac notation as the best proxy we can find for an
abstract syntax of the quantum mechanical notions. The definitions we
develop put us in contact with equational constraints coming from the
theory that we demonstrate the definitions and calculations satisfy.

This puts us in a position to shut up and calculate for the
Stern-Gerlach experimental set up, showing how these predictive
calculations become calculations on processes in our theory of a
reflective process calculus.

Penultimately, we demonstrate that the notion of metric coming from
the inner product coincides with the notion of metric available from
the theory of bisimulation. This demonstration gives us the right to
think of space as arising from behavior. Finally, we consider where we
might go from the new vantage point we have obtained.

% section introduction (end) 
 
% section introduction (end)

% \documentclass[12pt]{llncs}
%\documentclass{jktr}

\usepackage[pdftex]{hyperref}                   
\usepackage {listings}
\usepackage {mathpartir}
\usepackage{bcprules}
%\usepackage{listings}
                       
\usepackage{graphicx} 
%\usepackage[margins=2.5cm,nohead,nofoot]{geometry}
%\usepackage{geometry}
\usepackage{amsfonts}
\usepackage{amstext}
\usepackage{latexsym}
\usepackage{amssymb}
\usepackage{color}


%\include{myPreamble}
\include{qm2pi.local} 

%\ifpdf
%\usepackage[pdftex]{graphicx}
%\else
%\usepackage{graphicx}
%\fi

 % \ifpdf
%  \usepackage{pdfsync}
%  \if


%\title{Brief Article}
%\author{David F. Snyder}
%\author{L.G. Meredith}

%\address{Dept. of Math., Texas State University--San Marcos, San Marcos, TX 78666}
       
\pagestyle{empty}


\begin{document}

\lstset{language=[Objective]Caml,frame=shadowbox}

\input{qm2pi.front}

% section front matter (end)

\input{qm2pi.intro} 
 
% section introduction (end)

% \input{qm2pi.knotations} 

% section notation (end)

\input{qm2pi.process.calculi} 

% section concurrent_process_calculi_and_spatial_logics_ (end)
    
%\input{qm2pi.knots2pi} 

%\input{qm2pi.trefoil} 

%\input{qm2pi.mainthm} 

% subsection basic_interpretation (end)

%\input{qm2pi.rho.presentation} 
\subsection{The syntax and semantics of the notation system}\label{sub:the_syntax_and_semantics_of_the_notation_system} % (fold)

We now summarize a technical presentation of the calculus that
embodies our theory of dynamics. The typical presentation of such a
calculus follows the style of giving generators and relations on
them. The grammar, below, describing term constructors, freely
generates the set of processes, $\Proc$. This set is then quotiented
by a relation known as structural congruence and it is over this set
that the notion of dynamics is expressed. This presentation is
essentially that of \cite{MeredithR05} with the addition of
polyadicity and summation. For readability we have relegated some of
the technical subtleties to an appendix.

\subsubsection{Process grammar}\label{subsub:process_grammar}

\begin{mathpar}
  \inferrule* [lab=synchronization] {} {{M} \bc \pzero \;|\; x?F \;|\; x!C }
  \and
  \inferrule* [lab=abstraction] {} {{F} \bc (x)P}
  \and
  \inferrule* [lab=concretion] {} {{C} \bc \langle Q \rangle}
  \and
  \inferrule* [lab=process] {} {{P,Q} \bc M \;| \;P|Q \;|\; @{x}}
  \and
  \inferrule* [lab=name] {} {{x} \bc \quotep{P}}
\end{mathpar} 

Note that $\vec{x}$ (resp. $\vec{P}$) denotes a vector of names
(resp. processes) of length $|\vec{x}|$ (resp. $|\vec{P}|$). We adopt
the following useful abbreviations.

\begin{mathpar}
   x?(\vec{y}).P := x.(\vec{y})P \and  x\clift{\vec{P}} := x.\clift{\vec{P}}
   \and x!(y) := \lift{x}{\dropn{y}}
   \and \Pi_{i=0}^{n-1}P_i := P_0 | \ldots | P_{n-1}
\end{mathpar}

\subsubsection{Structural congruence}

\paragraph{Free and bound names and alpha-equivalence.} At the
core of structural equivalence is alpha-equivalence which identifies
process that are the same up to a change of variable. Formally, we
recognize the distinction between free and bound names. The free names
of a process, $\freenames{P}$, may be calculated recursively as
follows:

\begin{mathpar}
\freenames{\pzero} := \emptyset
  \and \\
  \freenames{x?(y).P} := \{ x \} \cup (\freenames{P} \setminus \{ y \})
  \and 
  \freenames{x!\langle P \rangle} := \{ x \} \cup \{ P \} 
  \and \\
  \freenames{P|Q} := \freenames{P} \cup \freenames{Q}
  \and \\
  \freenames{@{x}} := \{ x \}
\end{mathpar}

$\pi$
$\quotep{\pi}$

$\freenames{-} : \pi \to \mathcal{P}(\quotep{\pi})$

\begin{eqnarray*}
  \freenames{\pzero} & := & \emptyset \\
  \freenames{x?(y).P} & := & \{ x \} \cup (\freenames{P} \setminus \{ y \}) \\
  \freenames{x!\langle P \rangle} & := & \{ x \} \cup \{ P \} \\
  \freenames{P|Q} & := & \freenames{P} \cup \freenames{Q} \\
  \freenames{\dropn{x}} & := & \{ x \}
\end{eqnarray*}

The bound names of a process, $\boundnames{P}$, are those names occurring in $P$
that are not free. For example, in $x?(y).0$, the name $x$ is free, while $y$ is bound.

\begin{mathpar}
  \inferrule* [lab=monoidal-laws] {} { P|Q \equiv Q|P \and P|0 \equiv P \and P|(Q|R) \equiv (P|Q)|R }
\end{mathpar}

\begin{mathpar}
  \inferrule* [lab=alpha-equivalence] {} { (x)P \equiv (y)P\{y/x\} \and y \not\in \freenames{P} }
\end{mathpar}

\begin{definition}
Then two processes, $P,Q$, are alpha-equivalent if $P = Q\{\vec{y}/\vec{x}\}$ for
some $\vec{x} \in \boundnames{Q},\vec{y} \in \boundnames{P}$, where $Q\{\vec{y}/\vec{x}\}$
denotes the capture-avoiding substitution of $\vec{y}$ for $\vec{x}$ in $Q$.
\end{definition}

\begin{definition}
  The {\em structural congruence} \cite{SangiorgiWalker} , $\equiv$,
  between processes is the least congruence containing
  alpha-equivalence, satisfying the abelian monoid laws
  (associativity, commutativity and $\pzero$ as identity) for parallel
  composition $|$ and for summation $+$.
\end{definition}

\subsection{Name equivalence}

We take name equivalence, written $\nameeq$, to be the smallest
equivalence relation generated by the following rules.

\begin{mathpar}
\inferrule*[lab=Quote-drop]
{ }
{ \quotep{@{x}} \nameeq x }

\inferrule*[lab=Struct-equiv]
{ P \scong Q }
{ \quotep{P} \nameeq \quotep{Q} }
\end{mathpar}

The astute reader will have noticed that the mutual recursion of names
and processes imposes a mutual recursion on alpha-equivalence and
structural equivalence via name-equivalence. Fortunately, all of this
works out pleasantly and we may calculate in the natural way, free of
concern. The reader interested in the details is referred to the
appendix \ref{appendix:rho_details}.

\subsection{Substitution}

We use $\Proc$ for the set of processes, $\QProc$ for the set of
names, and $\id{\{}\vec{y} / \vec{x} \id{\}}$ to denote partial maps,
$s : \QProc \rightarrow \QProc$. A map, $s$ lifts, uniquely, to a map
on process terms, $\widehat{s} : \Proc \rightarrow \Proc$ by the
following equations.

\begin{mathpar}
  (0) \psubstp{Q}{P} := 0 \\
  (R \juxtap S) \psubstp{Q}{P}
  :=    
  (R)\psubstp{Q}{P} \juxtap (S) \psubstp{Q}{P} \\
  (x?(y).R) \psubstp{Q}{P}    
  :=    
  (x)\substp{Q}{P} (z)\concat( (R \psubstn{z}{y}) \psubstp{Q}{P} ) \\
  (\lift{x}{R}) \psubstp{Q}{P}  
  :=
  \lift{(x)\substp{Q}{P}}{ R \psubstp{Q}{P} } \\
%   (\dropn{x})  \psubstp{Q}{P}       
%   := 
%   \left\{ 
%     \begin{array}{ccc} 
%       \dropn{\quotep{Q}} & & x \nameeq \quotep{P} \\
%       \dropn{x} & & otherwise \\
%     \end{array}
%   \right. 
  (\dropn{x})  \psubstp{Q}{P}       
  := 
  \left\{ 
    \begin{array}{ccc} 
      Q & & x \nameeq \quotep{P} \\
      \dropn{x} & & otherwise \\
    \end{array}
  \right.
\end{mathpar}
 

where

\begin{eqnarray}
  (x)\id{\{} \lpquote Q \rpquote / \lpquote P \rpquote \id{\}}            = 
  \left\{ 
    \begin{array}{ccc}
      \lpquote Q \rpquote & & x \nameeq \lpquote P \rpquote \\
      x & & otherwise \\
    \end{array}
  \right. \nonumber
\end{eqnarray}

and $z$ is chosen distinct from $\quotep{P}$, $\quotep{Q}$, the free
names in $Q$, and all the names in $R$. Our $\alpha$-equivalence will
be built in the standard way from this substitution.

\begin{remark}\label{rem:no_self_referential_names}
  One consequence of these definitions is that $\forall P. \quotep{P}
  \not\in \freenames{P}$.
\end{remark}

\subsection{ Dynamic quote: an example }

Anticipating something of what's to come, consider applying the
substitution, $\widehat{\id{\{}u / z \id{\}}}$, to the following pair
of processes, $\lift{w}{y!(z)}$ and $w[ \lpquote y!(z) \rpquote ]$.

\begin{eqnarray}
	\lift{w}{y!(z)}\widehat{\id{\{}u / z \id{\}}}
		& = &
		\lift{w}{y!(u)} \nonumber\\
	w[ \lpquote y!(z) \rpquote ] \widehat{ \id{\{}u / z \id{\}} }
		& = &
		w[ \lpquote y!(z) \rpquote ] \nonumber
\end{eqnarray}

Because the body of the process between quotes is impervious to
substitution, we get radically different answers. In fact, by
examining the first process in an input context,
e.g. $x?(z).\lift{w}{y!(z)}$, we see that the process under the lift
operator may be shaped by prefixed inputs binding a name inside it. In
this sense, the lift operator will be seen as a way to dynamically
construct processes before reifying them as names.

Finally equipped with these standard features we can present the
dynamics of the calculus.

\subsubsection{Operational semantics} 

Finally, we introduce the computational dynamics. What marks these
algebras as distinct from other more traditionally studied algebraic
structures, e.g. vector spaces or polynomial rings, is the manner in
which dynamics is captured. In traditional structures, dynamics is typically
expressed through morphisms between such structures, as in linear maps
between vector spaces or morphisms between rings. In algebras
associated with the semantics of computation, the dynamics is
expressed as part of the algebraic structure itself, through a
reduction reduction relation typically denoted by $\red$. Below, we
give a recursive presentation of this relation for the calculus used
in the encoding.

$\red \subseteq \pi \times \pi$
$\red : \pi \to \mathcal{P}(\pi)$

\begin{mathpar}
  \inferrule* [lab=Comm] { \textsf{match}( x_{src}, x_{trgt} ) } { x_{trgt}?(y)P \; | \; x_{src}!\langle {Q} \rangle \red P\{\quotep{Q}/y}\} }
  \and \\
  \inferrule* [lab=Par] {{P} \red {P}'} {{{P} | {Q}} \red {{P}' | {Q}}}
  \and
  \inferrule* [lab=Equiv]{{{P} \scong {P}'} \andalso {{P}' \red {Q}'} \andalso {{Q}' \scong {Q}}}{{P} \red {Q}}
\end{mathpar}

\begin{eqnarray*}
  match_{\equiv} (\quotep{P},\quotep{Q}) & := & P \equiv Q \\
  match_{\dagger}(\quotep{P},\quotep{Q}) & := & \forall R. P|Q \red^{*} R => R \red^{*} 0 \\
  match_{K}(\quotep{P},\quotep{Q}) & := & K \mbox{ for some context } K
\end{eqnarray*}

$u?(x)P | u!\langle Q \rangle \red P\{\quotep{Q}/x\}$

%We write $\wred$ for $\red^*$, and $P\red$ if $\exists Q $ such that $ P \red Q$.
We write $P\red$ if $\exists Q $ such that $ P \red Q$ and $P\not\red$, otherwise.

\section{Replication}

As mentioned before, it is known that replication (and hence
recursion) can be implemented in a higher-order process algebra
\cite{SangiorgiWalker}. As our first example of calculation with the
machinery thus far presented we give the construction explicitly in
the {\rhoc}.

\begin{eqnarray}
	D_{x} & := & \prefix{x}{y}{(\binpar{\outputp{x}{y}}{@{y}})} \nonumber\\
	\bangp_{x}{P} & := & \binpar{{x}!\langle{\binpar{D_{x}}{P}}\rangle}{D_{x}} \nonumber
\end{eqnarray}

\begin{eqnarray}
	\bangp_{x}{P} & & \nonumber\\
	=
	& {x}!\langle{(\prefix{x}{y}{(\outputp{x}{y} | @{y})) | P}}\rangle 
	      | \prefix{x}{y}{(\outputp{x}{y} | @{y})} & \nonumber\\
	\red
	& (\outputp{x}{y} | @{y})\substn{\quotep{(\prefix{x}{y}{(@{y} | \outputp{x}{y})) | P}}}{y} & \nonumber\\
	=
	& \outputp{x}{\quotep{(\prefix{x}{y}{(\outputp{x}{y} | @{y})) | P}}}
	  | {(\prefix{x}{y}{(\outputp{x}{y} | @{y})) | P}} & \nonumber\\
	\red
	& \ldots & \nonumber\\
	\red^*
	& P | P | \ldots & \nonumber
\end{eqnarray}

Of course, this encoding, as an implementation, runs away, unfolding
$\bangp{P}$ eagerly. A lazier and more implementable replication
operator, restricted to input-guarded processes, may be obtained as follows.

\begin{eqnarray}
\bangp{\prefix{u}{v}{P}} 
	:= 
	\binpar{\lift{x}{\prefix{u}{v}{(\binpar{D(x)}{P})}}}{D(x)} \nonumber
\end{eqnarray}

\begin{remark}
  Note that the lazier definition still does not deal with summation
  or mixed summation (i.e. sums over input and output). The reader is
  invited to construct definitions of replication that deal with these
  features. 

  Further, the definitions are parameterized in a name, $x$. Can you,
  gentle reader, make a definition that eliminates this parameter and
  guarantees no accidental interaction between the replication
  machinery and the process being replicated -- i.e. no accidental
  sharing of names used by the process to get its work done and the
  name(s) used by the replication to effect copying. This latter
  revision of the definition of replication is crucial to obtaining
  the expected identity $!!P \sim !P$.
\end{remark}

\begin{remark}\label{rem:paradoxical_combinator}
  The reader familiar with the lambda calculus will have noticed the
  similarity between $D$ and the paradoxical combinator.

  [Ed. note: the existence of this seems to suggest we have to be more
  restrictive on the set of processes and names we admit if we are to
  support no-cloning.]
\end{remark}

\subsubsection{Bisimulation}

The computational dynamics gives rise to another kind of equivalence,
the equivalence of computational behavior. As previously mentioned
this is typically captured \emph{via} some form of bisimulation.

% The notion we use in this paper is weak barbed bisimulation
% \cite{milner91polyadicpi}.

The notion we use in this paper is derived from weak barbed
bisimulation \cite{milner91polyadicpi}. 

\begin{definition}
An \emph{observation relation}, $\downarrow_{\mathcal N}$, over a set
of names, $\mathcal N$, is the smallest relation satisfying the rules
below.

\infrule[Out-barb]{y \in {\mathcal N}, \; x \nameeq y}
		  {\outputp{x}{v} \downarrow_{\mathcal N} x}
\infrule[Par-barb]{\mbox{$P\downarrow_{\mathcal N} x$ or $Q\downarrow_{\mathcal N} x$}}
		  {\binpar{P}{Q} \downarrow_{\mathcal N} x}

We write $P \Downarrow_{\mathcal N} x$ if there is $Q$ such that 
$P \wred Q$ and $Q \downarrow_{\mathcal N} x$.
\end{definition}

\begin{definition}
%\label{def.bbisim}
An  ${\mathcal N}$-\emph{barbed bisimulation} over a set of names, ${\mathcal N}$, is a symmetric binary relation 
${\mathcal S}_{\mathcal N}$ between agents such that $P\rel{S}_{\mathcal N}Q$ implies:
\begin{enumerate}
\item If $P \red P'$ then $Q \wred Q'$ and $P'\rel{S}_{\mathcal N} Q'$.
\item If $P\downarrow_{\mathcal N} x$, then $Q\Downarrow_{\mathcal N} x$.
\end{enumerate}
$P$ is ${\mathcal N}$-barbed bisimilar to $Q$, written
$P \wbbisim_{\mathcal N} Q$, if $P \rel{S}_{\mathcal N} Q$ for some ${\mathcal N}$-barbed bisimulation ${\mathcal S}_{\mathcal N}$.
\end{definition}

$\mathcal{R} \subseteq \pi \times \pi$

$P \mathcal{R} Q => \forall P'. P \red P' \Rightarrow \exists Q'. Q \red Q', P' \mathcal{R} Q'$

$P \vdash x \Rightarrow Q \vdash x$

\begin{mathpar}
  \inferrule*[lab=Out-barb]{x \nameeq y}{{y}!\langle{Q}\rangle \vdash x}
  \and
  \inferrule*[lab=Par-barb]{\mbox{$P\vdash x$ or $Q\vdash x$}}{\binpar{P}{Q} \vdash x}
\end{mathpar}

\subsubsection{Contexts}

One of the principle advantages of computational calculi like the
$\pi$-calculus is a well-defined notion of context,
contextual-equivalence and a correlation between
contextual-equivalence and notions of bisimulation. The notion of
context allows the decomposition of a process into (sub-)process and
its syntactic environment, its context. Thus, a context may be
thought of as a process with a ``hole'' (written $\Box$) in it. The
application of a context $M$ to a process $P$, written $M[P]$, is
tantamount to filling the hole in $M$ with $P$. In this paper we do
not need the full weight of this theory, but do make use of the notion
of context in the proof the main theorem. 

\begin{mathpar}
  \inferrule* [lab=summation] {} {{M_{M},M_{N}} \bc \Box \;|\; x.M_{A} \;|\; M_{M}+M_{N}}
  \and
  \inferrule* [lab=agent] {} {{M_{A}} \bc (\vec{x})M_{P} \;| \; \clift{P_0,\ldots,M_{P},\ldots,P_N}}
  \and \\
  \inferrule* [lab=process] {} {{M_{P}} \bc M_{N} \;| \;P|M_{P} }
\end{mathpar} 

\begin{mathpar}
  \inferrule* [lab=sychronization] {} {M_{N} \bc \Box \;|\; x?M_{F} \;|\; x!M_{C}}
  \and
  \inferrule* [lab=abstraction] {} {{M_{F}} \bc (x)M_{P} }
  \and
  \inferrule* [lab=concretion] {} {{M_{C}} \bc \langle M_{P} \rangle }
  \and \\
  \inferrule* [lab=process] {} {{M_{P}} \bc M_{N} \;| \;P|M_{P} }
\end{mathpar}

\begin{definition}[contextual application] Given a context $M$, and
  process $P$, we define the \emph{contextual application}, $M[P] :=
  M\{P/\Box\}$. That is, the contextual application of M to P is the
  substitution of $P$ for $\Box$ in $M$.
\end{definition}

$\meaningof{-} : L \to \mathcal{P}(\pi)$

\begin{mathpar}
  \inferrule* [lab=collection] {} {\meaningof{true} = \pi, \and \meaningof{~E} = \pi \setminus \meaningof{E}, \and \meaningof{E_{1} \& E_{2}} = \meaningof{E_{1}} \cap \meaningof{E_{2}}}
\end{mathpar}

\begin{mathpar}
  \inferrule* [lab=structure] {} {\meaningof{0} = \{ P \in \pi | P \equiv 0 \}, \and \\ \meaningof{E_1 | E_2} = \{ P \in \pi | P \equiv P_{1} | P_{2}, P_{1} \in \meaningof{E_{1}}, P_{2} \in \meaningof{E_2}\} }
\end{mathpar}

\begin{mathpar}
 \inferrule* [lab=behavior] {} {\meaningof{\langle a?b \rangle E} = \{ P \in \pi | P \equiv Q | u?(y)P', \\ \and \\\\ \and \\ \;\;\; u \in \meaningof{a}, \forall z.P'\{z/y\} \in \meaningof{E\{z/b\}}\}, \and \\ \meaningof{a!E} = \{ P \in \pi | P \equiv Q | x!\langle P' \rangle, x \in \meaningof{a} P' \in \meaningof{E}\} }
\end{mathpar}

\begin{mathpar}
 \inferrule* [lab=nominal] {} {\meaningof{\quotep{E}} = \{ \quotep{P} \in \quotep{\pi} | P \in \meaningof{E} \}, \and \meaningof{\quotep{P}} = \{ \quotep{Q} \in \quotep{\pi} | P \equiv Q \} \and \\ \meaningof{@\quotep{E}} = \{ P \in \pi | P \equiv @x, x \in \meaningof{E} \}}
\end{mathpar}

\begin{eqnarray*}
  \\
  \meaningof{-} : TS \to ST
\end{eqnarray*}

\begin{eqnarray*}
  \\
  L : TS \to ST
\end{eqnarray*}

\begin{eqnarray*}
  \\
  P \models E \iff P \in \meaningof{E}
\end{eqnarray*}

\begin{eqnarray*}
  P \approx_{L} Q \iff \forall E \in L. P \models E \iff Q \models E
\end{eqnarray*}

\begin{eqnarray*}
  P \approx_{K} Q
\end{eqnarray*}

\begin{eqnarray*}
  P \approx Q
\end{eqnarray*}

$\approx_{K} = \approx = \approx_{L}$

\subsubsection{Contextual duality}

Note that contexts extend the quotation operation to a family of
operations from processes to names. Given a context, $M$, we can
define a \emph{nominal context}, $\quotep{M}$ by $\quotep{M}[P] :=
\quotep{M[P]}$. To foreshadow what is to come we observe that these
operations enjoy a duality with processes very much like the duality
between vectors and maps from vectors to scalars.

Further, because the calculus is essentially higher-order, we have a
correspondence between contexts and processes. More specifically,
given a name $x$ and a context $M$ we can construct $M^{*}_{x}$ such
that 

\begin{mathpar}
  M^{*}_{x} | \lift{x}{P} \red M[P]
\end{mathpar}

namely,

\begin{mathpar}
  M^{*}_{x} := x?(u).M[\dropn{u}]
\end{mathpar}

The dependence of $M^{*}_{x}$ on a name makes it an abstraction, 

\begin{mathpar}
  M^{*} := (x)x?(u).M[\dropn{u}]
\end{mathpar}

\subsection{Additional notation}

It will sometimes be convenient to denote the process a name
quotes. We already have the notation $x = \quotep{P}$, but it will be
convenient to introduce an alternate notation, $\procn{x}$, when we
want to emphasize the connection to the use of the name. Note that, by
virtue of name equivalence, $\quotep{\procn{x}} \nameeq x$; so, the
notation is consistent with previous definitions.

Further, because names have structure it is possible to effect
substitutions on the basis of that structure. This means we need to
upgrade our notation for substitutions, which we accomplish by
adapting comprehension notation. Thus,

\begin{mathpar}
  P\{ y / x : x \in S \}
\end{mathpar}

is interpreted to mean the process derived from P by replacing (in a
capture-avoiding manner) each occurrence of $x$ in $S$ by $y$. For example,

\begin{mathpar}
  P\{ \quotep{\procn{x}|\procn{x}} / x : x \in \freenames{P} \}
\end{mathpar}

will replace each (occurrence) of a free name $x$ in $P$ by
$\quotep{\procn{x}|\procn{x}}$.

Also, we will avail ourselves of the notation $x^{L}$ and $x^{R}$ to
denote injections of a name into disjoint copies of the name
space. There are numerous ways to accomplish this. One example can be
found in \cite{MeredithR05}. This notation overloads to vectors of
names: $\vec{x}^{\pi} := (x_{i}^{\pi} \; : \; 0 \leq i < |\vec{x}| )$ where $\pi \in \{L,R\}$.

We also use $P^{\Box} := P|\Box$.

In \cite{MeredithR05} an interpretation of the new operator is
given. It turns out that there are several possible interpretations
all enjoying the requisite algebraic properties of the operator (see
\cite{milner91polyadicpi}). We will therefore make liberal use of
$(\nu\; \vec{x})P$.

% subsection the_syntax_and_semantics_of_the_notation_system (end)   

\input{qm2pi.qmops} 

\input{qm2pi.sterngerlach} 

\input{qm2pi.metric} 

% section concurrent_process_calculi (end)

%\input{qm2pi.proofsketch}

% section proof sketch (end)

%\input{qm2pi.slviaknots} 

% section spatial logic via knots (end)

\input{qm2pi.conclusion}

% section conclusion (end)

%\input{qm2pi.dtcodes} 

% section wiring algorithm (end)

\input{qm2pi.ack} 

% section acknowledgments (end)

\newpage


\bibliographystyle{plain}   
\bibliography{../../biblios/main.bib}

\input{qm2pi.rhodetails}

\end{document}

 

% section notation (end)

\input{qm2pi.process.calculi} 

% section concurrent_process_calculi_and_spatial_logics_ (end)
    
%\documentclass[12pt]{llncs}
%\documentclass{jktr}

\usepackage[pdftex]{hyperref}                   
\usepackage {listings}
\usepackage {mathpartir}
\usepackage{bcprules}
%\usepackage{listings}
                       
\usepackage{graphicx} 
%\usepackage[margins=2.5cm,nohead,nofoot]{geometry}
%\usepackage{geometry}
\usepackage{amsfonts}
\usepackage{amstext}
\usepackage{latexsym}
\usepackage{amssymb}
\usepackage{color}


%\include{myPreamble}
\include{qm2pi.local} 

%\ifpdf
%\usepackage[pdftex]{graphicx}
%\else
%\usepackage{graphicx}
%\fi

 % \ifpdf
%  \usepackage{pdfsync}
%  \if


%\title{Brief Article}
%\author{David F. Snyder}
%\author{L.G. Meredith}

%\address{Dept. of Math., Texas State University--San Marcos, San Marcos, TX 78666}
       
\pagestyle{empty}


\begin{document}

\lstset{language=[Objective]Caml,frame=shadowbox}

\input{qm2pi.front}

% section front matter (end)

\input{qm2pi.intro} 
 
% section introduction (end)

% \input{qm2pi.knotations} 

% section notation (end)

\input{qm2pi.process.calculi} 

% section concurrent_process_calculi_and_spatial_logics_ (end)
    
%\input{qm2pi.knots2pi} 

%\input{qm2pi.trefoil} 

%\input{qm2pi.mainthm} 

% subsection basic_interpretation (end)

%\input{qm2pi.rho.presentation} 
\subsection{The syntax and semantics of the notation system}\label{sub:the_syntax_and_semantics_of_the_notation_system} % (fold)

We now summarize a technical presentation of the calculus that
embodies our theory of dynamics. The typical presentation of such a
calculus follows the style of giving generators and relations on
them. The grammar, below, describing term constructors, freely
generates the set of processes, $\Proc$. This set is then quotiented
by a relation known as structural congruence and it is over this set
that the notion of dynamics is expressed. This presentation is
essentially that of \cite{MeredithR05} with the addition of
polyadicity and summation. For readability we have relegated some of
the technical subtleties to an appendix.

\subsubsection{Process grammar}\label{subsub:process_grammar}

\begin{mathpar}
  \inferrule* [lab=synchronization] {} {{M} \bc \pzero \;|\; x?F \;|\; x!C }
  \and
  \inferrule* [lab=abstraction] {} {{F} \bc (x)P}
  \and
  \inferrule* [lab=concretion] {} {{C} \bc \langle Q \rangle}
  \and
  \inferrule* [lab=process] {} {{P,Q} \bc M \;| \;P|Q \;|\; @{x}}
  \and
  \inferrule* [lab=name] {} {{x} \bc \quotep{P}}
\end{mathpar} 

Note that $\vec{x}$ (resp. $\vec{P}$) denotes a vector of names
(resp. processes) of length $|\vec{x}|$ (resp. $|\vec{P}|$). We adopt
the following useful abbreviations.

\begin{mathpar}
   x?(\vec{y}).P := x.(\vec{y})P \and  x\clift{\vec{P}} := x.\clift{\vec{P}}
   \and x!(y) := \lift{x}{\dropn{y}}
   \and \Pi_{i=0}^{n-1}P_i := P_0 | \ldots | P_{n-1}
\end{mathpar}

\subsubsection{Structural congruence}

\paragraph{Free and bound names and alpha-equivalence.} At the
core of structural equivalence is alpha-equivalence which identifies
process that are the same up to a change of variable. Formally, we
recognize the distinction between free and bound names. The free names
of a process, $\freenames{P}$, may be calculated recursively as
follows:

\begin{mathpar}
\freenames{\pzero} := \emptyset
  \and \\
  \freenames{x?(y).P} := \{ x \} \cup (\freenames{P} \setminus \{ y \})
  \and 
  \freenames{x!\langle P \rangle} := \{ x \} \cup \{ P \} 
  \and \\
  \freenames{P|Q} := \freenames{P} \cup \freenames{Q}
  \and \\
  \freenames{@{x}} := \{ x \}
\end{mathpar}

$\pi$
$\quotep{\pi}$

$\freenames{-} : \pi \to \mathcal{P}(\quotep{\pi})$

\begin{eqnarray*}
  \freenames{\pzero} & := & \emptyset \\
  \freenames{x?(y).P} & := & \{ x \} \cup (\freenames{P} \setminus \{ y \}) \\
  \freenames{x!\langle P \rangle} & := & \{ x \} \cup \{ P \} \\
  \freenames{P|Q} & := & \freenames{P} \cup \freenames{Q} \\
  \freenames{\dropn{x}} & := & \{ x \}
\end{eqnarray*}

The bound names of a process, $\boundnames{P}$, are those names occurring in $P$
that are not free. For example, in $x?(y).0$, the name $x$ is free, while $y$ is bound.

\begin{mathpar}
  \inferrule* [lab=monoidal-laws] {} { P|Q \equiv Q|P \and P|0 \equiv P \and P|(Q|R) \equiv (P|Q)|R }
\end{mathpar}

\begin{mathpar}
  \inferrule* [lab=alpha-equivalence] {} { (x)P \equiv (y)P\{y/x\} \and y \not\in \freenames{P} }
\end{mathpar}

\begin{definition}
Then two processes, $P,Q$, are alpha-equivalent if $P = Q\{\vec{y}/\vec{x}\}$ for
some $\vec{x} \in \boundnames{Q},\vec{y} \in \boundnames{P}$, where $Q\{\vec{y}/\vec{x}\}$
denotes the capture-avoiding substitution of $\vec{y}$ for $\vec{x}$ in $Q$.
\end{definition}

\begin{definition}
  The {\em structural congruence} \cite{SangiorgiWalker} , $\equiv$,
  between processes is the least congruence containing
  alpha-equivalence, satisfying the abelian monoid laws
  (associativity, commutativity and $\pzero$ as identity) for parallel
  composition $|$ and for summation $+$.
\end{definition}

\subsection{Name equivalence}

We take name equivalence, written $\nameeq$, to be the smallest
equivalence relation generated by the following rules.

\begin{mathpar}
\inferrule*[lab=Quote-drop]
{ }
{ \quotep{@{x}} \nameeq x }

\inferrule*[lab=Struct-equiv]
{ P \scong Q }
{ \quotep{P} \nameeq \quotep{Q} }
\end{mathpar}

The astute reader will have noticed that the mutual recursion of names
and processes imposes a mutual recursion on alpha-equivalence and
structural equivalence via name-equivalence. Fortunately, all of this
works out pleasantly and we may calculate in the natural way, free of
concern. The reader interested in the details is referred to the
appendix \ref{appendix:rho_details}.

\subsection{Substitution}

We use $\Proc$ for the set of processes, $\QProc$ for the set of
names, and $\id{\{}\vec{y} / \vec{x} \id{\}}$ to denote partial maps,
$s : \QProc \rightarrow \QProc$. A map, $s$ lifts, uniquely, to a map
on process terms, $\widehat{s} : \Proc \rightarrow \Proc$ by the
following equations.

\begin{mathpar}
  (0) \psubstp{Q}{P} := 0 \\
  (R \juxtap S) \psubstp{Q}{P}
  :=    
  (R)\psubstp{Q}{P} \juxtap (S) \psubstp{Q}{P} \\
  (x?(y).R) \psubstp{Q}{P}    
  :=    
  (x)\substp{Q}{P} (z)\concat( (R \psubstn{z}{y}) \psubstp{Q}{P} ) \\
  (\lift{x}{R}) \psubstp{Q}{P}  
  :=
  \lift{(x)\substp{Q}{P}}{ R \psubstp{Q}{P} } \\
%   (\dropn{x})  \psubstp{Q}{P}       
%   := 
%   \left\{ 
%     \begin{array}{ccc} 
%       \dropn{\quotep{Q}} & & x \nameeq \quotep{P} \\
%       \dropn{x} & & otherwise \\
%     \end{array}
%   \right. 
  (\dropn{x})  \psubstp{Q}{P}       
  := 
  \left\{ 
    \begin{array}{ccc} 
      Q & & x \nameeq \quotep{P} \\
      \dropn{x} & & otherwise \\
    \end{array}
  \right.
\end{mathpar}
 

where

\begin{eqnarray}
  (x)\id{\{} \lpquote Q \rpquote / \lpquote P \rpquote \id{\}}            = 
  \left\{ 
    \begin{array}{ccc}
      \lpquote Q \rpquote & & x \nameeq \lpquote P \rpquote \\
      x & & otherwise \\
    \end{array}
  \right. \nonumber
\end{eqnarray}

and $z$ is chosen distinct from $\quotep{P}$, $\quotep{Q}$, the free
names in $Q$, and all the names in $R$. Our $\alpha$-equivalence will
be built in the standard way from this substitution.

\begin{remark}\label{rem:no_self_referential_names}
  One consequence of these definitions is that $\forall P. \quotep{P}
  \not\in \freenames{P}$.
\end{remark}

\subsection{ Dynamic quote: an example }

Anticipating something of what's to come, consider applying the
substitution, $\widehat{\id{\{}u / z \id{\}}}$, to the following pair
of processes, $\lift{w}{y!(z)}$ and $w[ \lpquote y!(z) \rpquote ]$.

\begin{eqnarray}
	\lift{w}{y!(z)}\widehat{\id{\{}u / z \id{\}}}
		& = &
		\lift{w}{y!(u)} \nonumber\\
	w[ \lpquote y!(z) \rpquote ] \widehat{ \id{\{}u / z \id{\}} }
		& = &
		w[ \lpquote y!(z) \rpquote ] \nonumber
\end{eqnarray}

Because the body of the process between quotes is impervious to
substitution, we get radically different answers. In fact, by
examining the first process in an input context,
e.g. $x?(z).\lift{w}{y!(z)}$, we see that the process under the lift
operator may be shaped by prefixed inputs binding a name inside it. In
this sense, the lift operator will be seen as a way to dynamically
construct processes before reifying them as names.

Finally equipped with these standard features we can present the
dynamics of the calculus.

\subsubsection{Operational semantics} 

Finally, we introduce the computational dynamics. What marks these
algebras as distinct from other more traditionally studied algebraic
structures, e.g. vector spaces or polynomial rings, is the manner in
which dynamics is captured. In traditional structures, dynamics is typically
expressed through morphisms between such structures, as in linear maps
between vector spaces or morphisms between rings. In algebras
associated with the semantics of computation, the dynamics is
expressed as part of the algebraic structure itself, through a
reduction reduction relation typically denoted by $\red$. Below, we
give a recursive presentation of this relation for the calculus used
in the encoding.

$\red \subseteq \pi \times \pi$
$\red : \pi \to \mathcal{P}(\pi)$

\begin{mathpar}
  \inferrule* [lab=Comm] { \textsf{match}( x_{src}, x_{trgt} ) } { x_{trgt}?(y)P \; | \; x_{src}!\langle {Q} \rangle \red P\{\quotep{Q}/y}\} }
  \and \\
  \inferrule* [lab=Par] {{P} \red {P}'} {{{P} | {Q}} \red {{P}' | {Q}}}
  \and
  \inferrule* [lab=Equiv]{{{P} \scong {P}'} \andalso {{P}' \red {Q}'} \andalso {{Q}' \scong {Q}}}{{P} \red {Q}}
\end{mathpar}

\begin{eqnarray*}
  match_{\equiv} (\quotep{P},\quotep{Q}) & := & P \equiv Q \\
  match_{\dagger}(\quotep{P},\quotep{Q}) & := & \forall R. P|Q \red^{*} R => R \red^{*} 0 \\
  match_{K}(\quotep{P},\quotep{Q}) & := & K \mbox{ for some context } K
\end{eqnarray*}

$u?(x)P | u!\langle Q \rangle \red P\{\quotep{Q}/x\}$

%We write $\wred$ for $\red^*$, and $P\red$ if $\exists Q $ such that $ P \red Q$.
We write $P\red$ if $\exists Q $ such that $ P \red Q$ and $P\not\red$, otherwise.

\section{Replication}

As mentioned before, it is known that replication (and hence
recursion) can be implemented in a higher-order process algebra
\cite{SangiorgiWalker}. As our first example of calculation with the
machinery thus far presented we give the construction explicitly in
the {\rhoc}.

\begin{eqnarray}
	D_{x} & := & \prefix{x}{y}{(\binpar{\outputp{x}{y}}{@{y}})} \nonumber\\
	\bangp_{x}{P} & := & \binpar{{x}!\langle{\binpar{D_{x}}{P}}\rangle}{D_{x}} \nonumber
\end{eqnarray}

\begin{eqnarray}
	\bangp_{x}{P} & & \nonumber\\
	=
	& {x}!\langle{(\prefix{x}{y}{(\outputp{x}{y} | @{y})) | P}}\rangle 
	      | \prefix{x}{y}{(\outputp{x}{y} | @{y})} & \nonumber\\
	\red
	& (\outputp{x}{y} | @{y})\substn{\quotep{(\prefix{x}{y}{(@{y} | \outputp{x}{y})) | P}}}{y} & \nonumber\\
	=
	& \outputp{x}{\quotep{(\prefix{x}{y}{(\outputp{x}{y} | @{y})) | P}}}
	  | {(\prefix{x}{y}{(\outputp{x}{y} | @{y})) | P}} & \nonumber\\
	\red
	& \ldots & \nonumber\\
	\red^*
	& P | P | \ldots & \nonumber
\end{eqnarray}

Of course, this encoding, as an implementation, runs away, unfolding
$\bangp{P}$ eagerly. A lazier and more implementable replication
operator, restricted to input-guarded processes, may be obtained as follows.

\begin{eqnarray}
\bangp{\prefix{u}{v}{P}} 
	:= 
	\binpar{\lift{x}{\prefix{u}{v}{(\binpar{D(x)}{P})}}}{D(x)} \nonumber
\end{eqnarray}

\begin{remark}
  Note that the lazier definition still does not deal with summation
  or mixed summation (i.e. sums over input and output). The reader is
  invited to construct definitions of replication that deal with these
  features. 

  Further, the definitions are parameterized in a name, $x$. Can you,
  gentle reader, make a definition that eliminates this parameter and
  guarantees no accidental interaction between the replication
  machinery and the process being replicated -- i.e. no accidental
  sharing of names used by the process to get its work done and the
  name(s) used by the replication to effect copying. This latter
  revision of the definition of replication is crucial to obtaining
  the expected identity $!!P \sim !P$.
\end{remark}

\begin{remark}\label{rem:paradoxical_combinator}
  The reader familiar with the lambda calculus will have noticed the
  similarity between $D$ and the paradoxical combinator.

  [Ed. note: the existence of this seems to suggest we have to be more
  restrictive on the set of processes and names we admit if we are to
  support no-cloning.]
\end{remark}

\subsubsection{Bisimulation}

The computational dynamics gives rise to another kind of equivalence,
the equivalence of computational behavior. As previously mentioned
this is typically captured \emph{via} some form of bisimulation.

% The notion we use in this paper is weak barbed bisimulation
% \cite{milner91polyadicpi}.

The notion we use in this paper is derived from weak barbed
bisimulation \cite{milner91polyadicpi}. 

\begin{definition}
An \emph{observation relation}, $\downarrow_{\mathcal N}$, over a set
of names, $\mathcal N$, is the smallest relation satisfying the rules
below.

\infrule[Out-barb]{y \in {\mathcal N}, \; x \nameeq y}
		  {\outputp{x}{v} \downarrow_{\mathcal N} x}
\infrule[Par-barb]{\mbox{$P\downarrow_{\mathcal N} x$ or $Q\downarrow_{\mathcal N} x$}}
		  {\binpar{P}{Q} \downarrow_{\mathcal N} x}

We write $P \Downarrow_{\mathcal N} x$ if there is $Q$ such that 
$P \wred Q$ and $Q \downarrow_{\mathcal N} x$.
\end{definition}

\begin{definition}
%\label{def.bbisim}
An  ${\mathcal N}$-\emph{barbed bisimulation} over a set of names, ${\mathcal N}$, is a symmetric binary relation 
${\mathcal S}_{\mathcal N}$ between agents such that $P\rel{S}_{\mathcal N}Q$ implies:
\begin{enumerate}
\item If $P \red P'$ then $Q \wred Q'$ and $P'\rel{S}_{\mathcal N} Q'$.
\item If $P\downarrow_{\mathcal N} x$, then $Q\Downarrow_{\mathcal N} x$.
\end{enumerate}
$P$ is ${\mathcal N}$-barbed bisimilar to $Q$, written
$P \wbbisim_{\mathcal N} Q$, if $P \rel{S}_{\mathcal N} Q$ for some ${\mathcal N}$-barbed bisimulation ${\mathcal S}_{\mathcal N}$.
\end{definition}

$\mathcal{R} \subseteq \pi \times \pi$

$P \mathcal{R} Q => \forall P'. P \red P' \Rightarrow \exists Q'. Q \red Q', P' \mathcal{R} Q'$

$P \vdash x \Rightarrow Q \vdash x$

\begin{mathpar}
  \inferrule*[lab=Out-barb]{x \nameeq y}{{y}!\langle{Q}\rangle \vdash x}
  \and
  \inferrule*[lab=Par-barb]{\mbox{$P\vdash x$ or $Q\vdash x$}}{\binpar{P}{Q} \vdash x}
\end{mathpar}

\subsubsection{Contexts}

One of the principle advantages of computational calculi like the
$\pi$-calculus is a well-defined notion of context,
contextual-equivalence and a correlation between
contextual-equivalence and notions of bisimulation. The notion of
context allows the decomposition of a process into (sub-)process and
its syntactic environment, its context. Thus, a context may be
thought of as a process with a ``hole'' (written $\Box$) in it. The
application of a context $M$ to a process $P$, written $M[P]$, is
tantamount to filling the hole in $M$ with $P$. In this paper we do
not need the full weight of this theory, but do make use of the notion
of context in the proof the main theorem. 

\begin{mathpar}
  \inferrule* [lab=summation] {} {{M_{M},M_{N}} \bc \Box \;|\; x.M_{A} \;|\; M_{M}+M_{N}}
  \and
  \inferrule* [lab=agent] {} {{M_{A}} \bc (\vec{x})M_{P} \;| \; \clift{P_0,\ldots,M_{P},\ldots,P_N}}
  \and \\
  \inferrule* [lab=process] {} {{M_{P}} \bc M_{N} \;| \;P|M_{P} }
\end{mathpar} 

\begin{mathpar}
  \inferrule* [lab=sychronization] {} {M_{N} \bc \Box \;|\; x?M_{F} \;|\; x!M_{C}}
  \and
  \inferrule* [lab=abstraction] {} {{M_{F}} \bc (x)M_{P} }
  \and
  \inferrule* [lab=concretion] {} {{M_{C}} \bc \langle M_{P} \rangle }
  \and \\
  \inferrule* [lab=process] {} {{M_{P}} \bc M_{N} \;| \;P|M_{P} }
\end{mathpar}

\begin{definition}[contextual application] Given a context $M$, and
  process $P$, we define the \emph{contextual application}, $M[P] :=
  M\{P/\Box\}$. That is, the contextual application of M to P is the
  substitution of $P$ for $\Box$ in $M$.
\end{definition}

$\meaningof{-} : L \to \mathcal{P}(\pi)$

\begin{mathpar}
  \inferrule* [lab=collection] {} {\meaningof{true} = \pi, \and \meaningof{~E} = \pi \setminus \meaningof{E}, \and \meaningof{E_{1} \& E_{2}} = \meaningof{E_{1}} \cap \meaningof{E_{2}}}
\end{mathpar}

\begin{mathpar}
  \inferrule* [lab=structure] {} {\meaningof{0} = \{ P \in \pi | P \equiv 0 \}, \and \\ \meaningof{E_1 | E_2} = \{ P \in \pi | P \equiv P_{1} | P_{2}, P_{1} \in \meaningof{E_{1}}, P_{2} \in \meaningof{E_2}\} }
\end{mathpar}

\begin{mathpar}
 \inferrule* [lab=behavior] {} {\meaningof{\langle a?b \rangle E} = \{ P \in \pi | P \equiv Q | u?(y)P', \\ \and \\\\ \and \\ \;\;\; u \in \meaningof{a}, \forall z.P'\{z/y\} \in \meaningof{E\{z/b\}}\}, \and \\ \meaningof{a!E} = \{ P \in \pi | P \equiv Q | x!\langle P' \rangle, x \in \meaningof{a} P' \in \meaningof{E}\} }
\end{mathpar}

\begin{mathpar}
 \inferrule* [lab=nominal] {} {\meaningof{\quotep{E}} = \{ \quotep{P} \in \quotep{\pi} | P \in \meaningof{E} \}, \and \meaningof{\quotep{P}} = \{ \quotep{Q} \in \quotep{\pi} | P \equiv Q \} \and \\ \meaningof{@\quotep{E}} = \{ P \in \pi | P \equiv @x, x \in \meaningof{E} \}}
\end{mathpar}

\begin{eqnarray*}
  \\
  \meaningof{-} : TS \to ST
\end{eqnarray*}

\begin{eqnarray*}
  \\
  L : TS \to ST
\end{eqnarray*}

\begin{eqnarray*}
  \\
  P \models E \iff P \in \meaningof{E}
\end{eqnarray*}

\begin{eqnarray*}
  P \approx_{L} Q \iff \forall E \in L. P \models E \iff Q \models E
\end{eqnarray*}

\begin{eqnarray*}
  P \approx_{K} Q
\end{eqnarray*}

\begin{eqnarray*}
  P \approx Q
\end{eqnarray*}

$\approx_{K} = \approx = \approx_{L}$

\subsubsection{Contextual duality}

Note that contexts extend the quotation operation to a family of
operations from processes to names. Given a context, $M$, we can
define a \emph{nominal context}, $\quotep{M}$ by $\quotep{M}[P] :=
\quotep{M[P]}$. To foreshadow what is to come we observe that these
operations enjoy a duality with processes very much like the duality
between vectors and maps from vectors to scalars.

Further, because the calculus is essentially higher-order, we have a
correspondence between contexts and processes. More specifically,
given a name $x$ and a context $M$ we can construct $M^{*}_{x}$ such
that 

\begin{mathpar}
  M^{*}_{x} | \lift{x}{P} \red M[P]
\end{mathpar}

namely,

\begin{mathpar}
  M^{*}_{x} := x?(u).M[\dropn{u}]
\end{mathpar}

The dependence of $M^{*}_{x}$ on a name makes it an abstraction, 

\begin{mathpar}
  M^{*} := (x)x?(u).M[\dropn{u}]
\end{mathpar}

\subsection{Additional notation}

It will sometimes be convenient to denote the process a name
quotes. We already have the notation $x = \quotep{P}$, but it will be
convenient to introduce an alternate notation, $\procn{x}$, when we
want to emphasize the connection to the use of the name. Note that, by
virtue of name equivalence, $\quotep{\procn{x}} \nameeq x$; so, the
notation is consistent with previous definitions.

Further, because names have structure it is possible to effect
substitutions on the basis of that structure. This means we need to
upgrade our notation for substitutions, which we accomplish by
adapting comprehension notation. Thus,

\begin{mathpar}
  P\{ y / x : x \in S \}
\end{mathpar}

is interpreted to mean the process derived from P by replacing (in a
capture-avoiding manner) each occurrence of $x$ in $S$ by $y$. For example,

\begin{mathpar}
  P\{ \quotep{\procn{x}|\procn{x}} / x : x \in \freenames{P} \}
\end{mathpar}

will replace each (occurrence) of a free name $x$ in $P$ by
$\quotep{\procn{x}|\procn{x}}$.

Also, we will avail ourselves of the notation $x^{L}$ and $x^{R}$ to
denote injections of a name into disjoint copies of the name
space. There are numerous ways to accomplish this. One example can be
found in \cite{MeredithR05}. This notation overloads to vectors of
names: $\vec{x}^{\pi} := (x_{i}^{\pi} \; : \; 0 \leq i < |\vec{x}| )$ where $\pi \in \{L,R\}$.

We also use $P^{\Box} := P|\Box$.

In \cite{MeredithR05} an interpretation of the new operator is
given. It turns out that there are several possible interpretations
all enjoying the requisite algebraic properties of the operator (see
\cite{milner91polyadicpi}). We will therefore make liberal use of
$(\nu\; \vec{x})P$.

% subsection the_syntax_and_semantics_of_the_notation_system (end)   

\input{qm2pi.qmops} 

\input{qm2pi.sterngerlach} 

\input{qm2pi.metric} 

% section concurrent_process_calculi (end)

%\input{qm2pi.proofsketch}

% section proof sketch (end)

%\input{qm2pi.slviaknots} 

% section spatial logic via knots (end)

\input{qm2pi.conclusion}

% section conclusion (end)

%\input{qm2pi.dtcodes} 

% section wiring algorithm (end)

\input{qm2pi.ack} 

% section acknowledgments (end)

\newpage


\bibliographystyle{plain}   
\bibliography{../../biblios/main.bib}

\input{qm2pi.rhodetails}

\end{document}

 

%\documentclass[12pt]{llncs}
%\documentclass{jktr}

\usepackage[pdftex]{hyperref}                   
\usepackage {listings}
\usepackage {mathpartir}
\usepackage{bcprules}
%\usepackage{listings}
                       
\usepackage{graphicx} 
%\usepackage[margins=2.5cm,nohead,nofoot]{geometry}
%\usepackage{geometry}
\usepackage{amsfonts}
\usepackage{amstext}
\usepackage{latexsym}
\usepackage{amssymb}
\usepackage{color}


%\include{myPreamble}
\include{qm2pi.local} 

%\ifpdf
%\usepackage[pdftex]{graphicx}
%\else
%\usepackage{graphicx}
%\fi

 % \ifpdf
%  \usepackage{pdfsync}
%  \if


%\title{Brief Article}
%\author{David F. Snyder}
%\author{L.G. Meredith}

%\address{Dept. of Math., Texas State University--San Marcos, San Marcos, TX 78666}
       
\pagestyle{empty}


\begin{document}

\lstset{language=[Objective]Caml,frame=shadowbox}

\input{qm2pi.front}

% section front matter (end)

\input{qm2pi.intro} 
 
% section introduction (end)

% \input{qm2pi.knotations} 

% section notation (end)

\input{qm2pi.process.calculi} 

% section concurrent_process_calculi_and_spatial_logics_ (end)
    
%\input{qm2pi.knots2pi} 

%\input{qm2pi.trefoil} 

%\input{qm2pi.mainthm} 

% subsection basic_interpretation (end)

%\input{qm2pi.rho.presentation} 
\subsection{The syntax and semantics of the notation system}\label{sub:the_syntax_and_semantics_of_the_notation_system} % (fold)

We now summarize a technical presentation of the calculus that
embodies our theory of dynamics. The typical presentation of such a
calculus follows the style of giving generators and relations on
them. The grammar, below, describing term constructors, freely
generates the set of processes, $\Proc$. This set is then quotiented
by a relation known as structural congruence and it is over this set
that the notion of dynamics is expressed. This presentation is
essentially that of \cite{MeredithR05} with the addition of
polyadicity and summation. For readability we have relegated some of
the technical subtleties to an appendix.

\subsubsection{Process grammar}\label{subsub:process_grammar}

\begin{mathpar}
  \inferrule* [lab=synchronization] {} {{M} \bc \pzero \;|\; x?F \;|\; x!C }
  \and
  \inferrule* [lab=abstraction] {} {{F} \bc (x)P}
  \and
  \inferrule* [lab=concretion] {} {{C} \bc \langle Q \rangle}
  \and
  \inferrule* [lab=process] {} {{P,Q} \bc M \;| \;P|Q \;|\; @{x}}
  \and
  \inferrule* [lab=name] {} {{x} \bc \quotep{P}}
\end{mathpar} 

Note that $\vec{x}$ (resp. $\vec{P}$) denotes a vector of names
(resp. processes) of length $|\vec{x}|$ (resp. $|\vec{P}|$). We adopt
the following useful abbreviations.

\begin{mathpar}
   x?(\vec{y}).P := x.(\vec{y})P \and  x\clift{\vec{P}} := x.\clift{\vec{P}}
   \and x!(y) := \lift{x}{\dropn{y}}
   \and \Pi_{i=0}^{n-1}P_i := P_0 | \ldots | P_{n-1}
\end{mathpar}

\subsubsection{Structural congruence}

\paragraph{Free and bound names and alpha-equivalence.} At the
core of structural equivalence is alpha-equivalence which identifies
process that are the same up to a change of variable. Formally, we
recognize the distinction between free and bound names. The free names
of a process, $\freenames{P}$, may be calculated recursively as
follows:

\begin{mathpar}
\freenames{\pzero} := \emptyset
  \and \\
  \freenames{x?(y).P} := \{ x \} \cup (\freenames{P} \setminus \{ y \})
  \and 
  \freenames{x!\langle P \rangle} := \{ x \} \cup \{ P \} 
  \and \\
  \freenames{P|Q} := \freenames{P} \cup \freenames{Q}
  \and \\
  \freenames{@{x}} := \{ x \}
\end{mathpar}

$\pi$
$\quotep{\pi}$

$\freenames{-} : \pi \to \mathcal{P}(\quotep{\pi})$

\begin{eqnarray*}
  \freenames{\pzero} & := & \emptyset \\
  \freenames{x?(y).P} & := & \{ x \} \cup (\freenames{P} \setminus \{ y \}) \\
  \freenames{x!\langle P \rangle} & := & \{ x \} \cup \{ P \} \\
  \freenames{P|Q} & := & \freenames{P} \cup \freenames{Q} \\
  \freenames{\dropn{x}} & := & \{ x \}
\end{eqnarray*}

The bound names of a process, $\boundnames{P}$, are those names occurring in $P$
that are not free. For example, in $x?(y).0$, the name $x$ is free, while $y$ is bound.

\begin{mathpar}
  \inferrule* [lab=monoidal-laws] {} { P|Q \equiv Q|P \and P|0 \equiv P \and P|(Q|R) \equiv (P|Q)|R }
\end{mathpar}

\begin{mathpar}
  \inferrule* [lab=alpha-equivalence] {} { (x)P \equiv (y)P\{y/x\} \and y \not\in \freenames{P} }
\end{mathpar}

\begin{definition}
Then two processes, $P,Q$, are alpha-equivalent if $P = Q\{\vec{y}/\vec{x}\}$ for
some $\vec{x} \in \boundnames{Q},\vec{y} \in \boundnames{P}$, where $Q\{\vec{y}/\vec{x}\}$
denotes the capture-avoiding substitution of $\vec{y}$ for $\vec{x}$ in $Q$.
\end{definition}

\begin{definition}
  The {\em structural congruence} \cite{SangiorgiWalker} , $\equiv$,
  between processes is the least congruence containing
  alpha-equivalence, satisfying the abelian monoid laws
  (associativity, commutativity and $\pzero$ as identity) for parallel
  composition $|$ and for summation $+$.
\end{definition}

\subsection{Name equivalence}

We take name equivalence, written $\nameeq$, to be the smallest
equivalence relation generated by the following rules.

\begin{mathpar}
\inferrule*[lab=Quote-drop]
{ }
{ \quotep{@{x}} \nameeq x }

\inferrule*[lab=Struct-equiv]
{ P \scong Q }
{ \quotep{P} \nameeq \quotep{Q} }
\end{mathpar}

The astute reader will have noticed that the mutual recursion of names
and processes imposes a mutual recursion on alpha-equivalence and
structural equivalence via name-equivalence. Fortunately, all of this
works out pleasantly and we may calculate in the natural way, free of
concern. The reader interested in the details is referred to the
appendix \ref{appendix:rho_details}.

\subsection{Substitution}

We use $\Proc$ for the set of processes, $\QProc$ for the set of
names, and $\id{\{}\vec{y} / \vec{x} \id{\}}$ to denote partial maps,
$s : \QProc \rightarrow \QProc$. A map, $s$ lifts, uniquely, to a map
on process terms, $\widehat{s} : \Proc \rightarrow \Proc$ by the
following equations.

\begin{mathpar}
  (0) \psubstp{Q}{P} := 0 \\
  (R \juxtap S) \psubstp{Q}{P}
  :=    
  (R)\psubstp{Q}{P} \juxtap (S) \psubstp{Q}{P} \\
  (x?(y).R) \psubstp{Q}{P}    
  :=    
  (x)\substp{Q}{P} (z)\concat( (R \psubstn{z}{y}) \psubstp{Q}{P} ) \\
  (\lift{x}{R}) \psubstp{Q}{P}  
  :=
  \lift{(x)\substp{Q}{P}}{ R \psubstp{Q}{P} } \\
%   (\dropn{x})  \psubstp{Q}{P}       
%   := 
%   \left\{ 
%     \begin{array}{ccc} 
%       \dropn{\quotep{Q}} & & x \nameeq \quotep{P} \\
%       \dropn{x} & & otherwise \\
%     \end{array}
%   \right. 
  (\dropn{x})  \psubstp{Q}{P}       
  := 
  \left\{ 
    \begin{array}{ccc} 
      Q & & x \nameeq \quotep{P} \\
      \dropn{x} & & otherwise \\
    \end{array}
  \right.
\end{mathpar}
 

where

\begin{eqnarray}
  (x)\id{\{} \lpquote Q \rpquote / \lpquote P \rpquote \id{\}}            = 
  \left\{ 
    \begin{array}{ccc}
      \lpquote Q \rpquote & & x \nameeq \lpquote P \rpquote \\
      x & & otherwise \\
    \end{array}
  \right. \nonumber
\end{eqnarray}

and $z$ is chosen distinct from $\quotep{P}$, $\quotep{Q}$, the free
names in $Q$, and all the names in $R$. Our $\alpha$-equivalence will
be built in the standard way from this substitution.

\begin{remark}\label{rem:no_self_referential_names}
  One consequence of these definitions is that $\forall P. \quotep{P}
  \not\in \freenames{P}$.
\end{remark}

\subsection{ Dynamic quote: an example }

Anticipating something of what's to come, consider applying the
substitution, $\widehat{\id{\{}u / z \id{\}}}$, to the following pair
of processes, $\lift{w}{y!(z)}$ and $w[ \lpquote y!(z) \rpquote ]$.

\begin{eqnarray}
	\lift{w}{y!(z)}\widehat{\id{\{}u / z \id{\}}}
		& = &
		\lift{w}{y!(u)} \nonumber\\
	w[ \lpquote y!(z) \rpquote ] \widehat{ \id{\{}u / z \id{\}} }
		& = &
		w[ \lpquote y!(z) \rpquote ] \nonumber
\end{eqnarray}

Because the body of the process between quotes is impervious to
substitution, we get radically different answers. In fact, by
examining the first process in an input context,
e.g. $x?(z).\lift{w}{y!(z)}$, we see that the process under the lift
operator may be shaped by prefixed inputs binding a name inside it. In
this sense, the lift operator will be seen as a way to dynamically
construct processes before reifying them as names.

Finally equipped with these standard features we can present the
dynamics of the calculus.

\subsubsection{Operational semantics} 

Finally, we introduce the computational dynamics. What marks these
algebras as distinct from other more traditionally studied algebraic
structures, e.g. vector spaces or polynomial rings, is the manner in
which dynamics is captured. In traditional structures, dynamics is typically
expressed through morphisms between such structures, as in linear maps
between vector spaces or morphisms between rings. In algebras
associated with the semantics of computation, the dynamics is
expressed as part of the algebraic structure itself, through a
reduction reduction relation typically denoted by $\red$. Below, we
give a recursive presentation of this relation for the calculus used
in the encoding.

$\red \subseteq \pi \times \pi$
$\red : \pi \to \mathcal{P}(\pi)$

\begin{mathpar}
  \inferrule* [lab=Comm] { \textsf{match}( x_{src}, x_{trgt} ) } { x_{trgt}?(y)P \; | \; x_{src}!\langle {Q} \rangle \red P\{\quotep{Q}/y}\} }
  \and \\
  \inferrule* [lab=Par] {{P} \red {P}'} {{{P} | {Q}} \red {{P}' | {Q}}}
  \and
  \inferrule* [lab=Equiv]{{{P} \scong {P}'} \andalso {{P}' \red {Q}'} \andalso {{Q}' \scong {Q}}}{{P} \red {Q}}
\end{mathpar}

\begin{eqnarray*}
  match_{\equiv} (\quotep{P},\quotep{Q}) & := & P \equiv Q \\
  match_{\dagger}(\quotep{P},\quotep{Q}) & := & \forall R. P|Q \red^{*} R => R \red^{*} 0 \\
  match_{K}(\quotep{P},\quotep{Q}) & := & K \mbox{ for some context } K
\end{eqnarray*}

$u?(x)P | u!\langle Q \rangle \red P\{\quotep{Q}/x\}$

%We write $\wred$ for $\red^*$, and $P\red$ if $\exists Q $ such that $ P \red Q$.
We write $P\red$ if $\exists Q $ such that $ P \red Q$ and $P\not\red$, otherwise.

\section{Replication}

As mentioned before, it is known that replication (and hence
recursion) can be implemented in a higher-order process algebra
\cite{SangiorgiWalker}. As our first example of calculation with the
machinery thus far presented we give the construction explicitly in
the {\rhoc}.

\begin{eqnarray}
	D_{x} & := & \prefix{x}{y}{(\binpar{\outputp{x}{y}}{@{y}})} \nonumber\\
	\bangp_{x}{P} & := & \binpar{{x}!\langle{\binpar{D_{x}}{P}}\rangle}{D_{x}} \nonumber
\end{eqnarray}

\begin{eqnarray}
	\bangp_{x}{P} & & \nonumber\\
	=
	& {x}!\langle{(\prefix{x}{y}{(\outputp{x}{y} | @{y})) | P}}\rangle 
	      | \prefix{x}{y}{(\outputp{x}{y} | @{y})} & \nonumber\\
	\red
	& (\outputp{x}{y} | @{y})\substn{\quotep{(\prefix{x}{y}{(@{y} | \outputp{x}{y})) | P}}}{y} & \nonumber\\
	=
	& \outputp{x}{\quotep{(\prefix{x}{y}{(\outputp{x}{y} | @{y})) | P}}}
	  | {(\prefix{x}{y}{(\outputp{x}{y} | @{y})) | P}} & \nonumber\\
	\red
	& \ldots & \nonumber\\
	\red^*
	& P | P | \ldots & \nonumber
\end{eqnarray}

Of course, this encoding, as an implementation, runs away, unfolding
$\bangp{P}$ eagerly. A lazier and more implementable replication
operator, restricted to input-guarded processes, may be obtained as follows.

\begin{eqnarray}
\bangp{\prefix{u}{v}{P}} 
	:= 
	\binpar{\lift{x}{\prefix{u}{v}{(\binpar{D(x)}{P})}}}{D(x)} \nonumber
\end{eqnarray}

\begin{remark}
  Note that the lazier definition still does not deal with summation
  or mixed summation (i.e. sums over input and output). The reader is
  invited to construct definitions of replication that deal with these
  features. 

  Further, the definitions are parameterized in a name, $x$. Can you,
  gentle reader, make a definition that eliminates this parameter and
  guarantees no accidental interaction between the replication
  machinery and the process being replicated -- i.e. no accidental
  sharing of names used by the process to get its work done and the
  name(s) used by the replication to effect copying. This latter
  revision of the definition of replication is crucial to obtaining
  the expected identity $!!P \sim !P$.
\end{remark}

\begin{remark}\label{rem:paradoxical_combinator}
  The reader familiar with the lambda calculus will have noticed the
  similarity between $D$ and the paradoxical combinator.

  [Ed. note: the existence of this seems to suggest we have to be more
  restrictive on the set of processes and names we admit if we are to
  support no-cloning.]
\end{remark}

\subsubsection{Bisimulation}

The computational dynamics gives rise to another kind of equivalence,
the equivalence of computational behavior. As previously mentioned
this is typically captured \emph{via} some form of bisimulation.

% The notion we use in this paper is weak barbed bisimulation
% \cite{milner91polyadicpi}.

The notion we use in this paper is derived from weak barbed
bisimulation \cite{milner91polyadicpi}. 

\begin{definition}
An \emph{observation relation}, $\downarrow_{\mathcal N}$, over a set
of names, $\mathcal N$, is the smallest relation satisfying the rules
below.

\infrule[Out-barb]{y \in {\mathcal N}, \; x \nameeq y}
		  {\outputp{x}{v} \downarrow_{\mathcal N} x}
\infrule[Par-barb]{\mbox{$P\downarrow_{\mathcal N} x$ or $Q\downarrow_{\mathcal N} x$}}
		  {\binpar{P}{Q} \downarrow_{\mathcal N} x}

We write $P \Downarrow_{\mathcal N} x$ if there is $Q$ such that 
$P \wred Q$ and $Q \downarrow_{\mathcal N} x$.
\end{definition}

\begin{definition}
%\label{def.bbisim}
An  ${\mathcal N}$-\emph{barbed bisimulation} over a set of names, ${\mathcal N}$, is a symmetric binary relation 
${\mathcal S}_{\mathcal N}$ between agents such that $P\rel{S}_{\mathcal N}Q$ implies:
\begin{enumerate}
\item If $P \red P'$ then $Q \wred Q'$ and $P'\rel{S}_{\mathcal N} Q'$.
\item If $P\downarrow_{\mathcal N} x$, then $Q\Downarrow_{\mathcal N} x$.
\end{enumerate}
$P$ is ${\mathcal N}$-barbed bisimilar to $Q$, written
$P \wbbisim_{\mathcal N} Q$, if $P \rel{S}_{\mathcal N} Q$ for some ${\mathcal N}$-barbed bisimulation ${\mathcal S}_{\mathcal N}$.
\end{definition}

$\mathcal{R} \subseteq \pi \times \pi$

$P \mathcal{R} Q => \forall P'. P \red P' \Rightarrow \exists Q'. Q \red Q', P' \mathcal{R} Q'$

$P \vdash x \Rightarrow Q \vdash x$

\begin{mathpar}
  \inferrule*[lab=Out-barb]{x \nameeq y}{{y}!\langle{Q}\rangle \vdash x}
  \and
  \inferrule*[lab=Par-barb]{\mbox{$P\vdash x$ or $Q\vdash x$}}{\binpar{P}{Q} \vdash x}
\end{mathpar}

\subsubsection{Contexts}

One of the principle advantages of computational calculi like the
$\pi$-calculus is a well-defined notion of context,
contextual-equivalence and a correlation between
contextual-equivalence and notions of bisimulation. The notion of
context allows the decomposition of a process into (sub-)process and
its syntactic environment, its context. Thus, a context may be
thought of as a process with a ``hole'' (written $\Box$) in it. The
application of a context $M$ to a process $P$, written $M[P]$, is
tantamount to filling the hole in $M$ with $P$. In this paper we do
not need the full weight of this theory, but do make use of the notion
of context in the proof the main theorem. 

\begin{mathpar}
  \inferrule* [lab=summation] {} {{M_{M},M_{N}} \bc \Box \;|\; x.M_{A} \;|\; M_{M}+M_{N}}
  \and
  \inferrule* [lab=agent] {} {{M_{A}} \bc (\vec{x})M_{P} \;| \; \clift{P_0,\ldots,M_{P},\ldots,P_N}}
  \and \\
  \inferrule* [lab=process] {} {{M_{P}} \bc M_{N} \;| \;P|M_{P} }
\end{mathpar} 

\begin{mathpar}
  \inferrule* [lab=sychronization] {} {M_{N} \bc \Box \;|\; x?M_{F} \;|\; x!M_{C}}
  \and
  \inferrule* [lab=abstraction] {} {{M_{F}} \bc (x)M_{P} }
  \and
  \inferrule* [lab=concretion] {} {{M_{C}} \bc \langle M_{P} \rangle }
  \and \\
  \inferrule* [lab=process] {} {{M_{P}} \bc M_{N} \;| \;P|M_{P} }
\end{mathpar}

\begin{definition}[contextual application] Given a context $M$, and
  process $P$, we define the \emph{contextual application}, $M[P] :=
  M\{P/\Box\}$. That is, the contextual application of M to P is the
  substitution of $P$ for $\Box$ in $M$.
\end{definition}

$\meaningof{-} : L \to \mathcal{P}(\pi)$

\begin{mathpar}
  \inferrule* [lab=collection] {} {\meaningof{true} = \pi, \and \meaningof{~E} = \pi \setminus \meaningof{E}, \and \meaningof{E_{1} \& E_{2}} = \meaningof{E_{1}} \cap \meaningof{E_{2}}}
\end{mathpar}

\begin{mathpar}
  \inferrule* [lab=structure] {} {\meaningof{0} = \{ P \in \pi | P \equiv 0 \}, \and \\ \meaningof{E_1 | E_2} = \{ P \in \pi | P \equiv P_{1} | P_{2}, P_{1} \in \meaningof{E_{1}}, P_{2} \in \meaningof{E_2}\} }
\end{mathpar}

\begin{mathpar}
 \inferrule* [lab=behavior] {} {\meaningof{\langle a?b \rangle E} = \{ P \in \pi | P \equiv Q | u?(y)P', \\ \and \\\\ \and \\ \;\;\; u \in \meaningof{a}, \forall z.P'\{z/y\} \in \meaningof{E\{z/b\}}\}, \and \\ \meaningof{a!E} = \{ P \in \pi | P \equiv Q | x!\langle P' \rangle, x \in \meaningof{a} P' \in \meaningof{E}\} }
\end{mathpar}

\begin{mathpar}
 \inferrule* [lab=nominal] {} {\meaningof{\quotep{E}} = \{ \quotep{P} \in \quotep{\pi} | P \in \meaningof{E} \}, \and \meaningof{\quotep{P}} = \{ \quotep{Q} \in \quotep{\pi} | P \equiv Q \} \and \\ \meaningof{@\quotep{E}} = \{ P \in \pi | P \equiv @x, x \in \meaningof{E} \}}
\end{mathpar}

\begin{eqnarray*}
  \\
  \meaningof{-} : TS \to ST
\end{eqnarray*}

\begin{eqnarray*}
  \\
  L : TS \to ST
\end{eqnarray*}

\begin{eqnarray*}
  \\
  P \models E \iff P \in \meaningof{E}
\end{eqnarray*}

\begin{eqnarray*}
  P \approx_{L} Q \iff \forall E \in L. P \models E \iff Q \models E
\end{eqnarray*}

\begin{eqnarray*}
  P \approx_{K} Q
\end{eqnarray*}

\begin{eqnarray*}
  P \approx Q
\end{eqnarray*}

$\approx_{K} = \approx = \approx_{L}$

\subsubsection{Contextual duality}

Note that contexts extend the quotation operation to a family of
operations from processes to names. Given a context, $M$, we can
define a \emph{nominal context}, $\quotep{M}$ by $\quotep{M}[P] :=
\quotep{M[P]}$. To foreshadow what is to come we observe that these
operations enjoy a duality with processes very much like the duality
between vectors and maps from vectors to scalars.

Further, because the calculus is essentially higher-order, we have a
correspondence between contexts and processes. More specifically,
given a name $x$ and a context $M$ we can construct $M^{*}_{x}$ such
that 

\begin{mathpar}
  M^{*}_{x} | \lift{x}{P} \red M[P]
\end{mathpar}

namely,

\begin{mathpar}
  M^{*}_{x} := x?(u).M[\dropn{u}]
\end{mathpar}

The dependence of $M^{*}_{x}$ on a name makes it an abstraction, 

\begin{mathpar}
  M^{*} := (x)x?(u).M[\dropn{u}]
\end{mathpar}

\subsection{Additional notation}

It will sometimes be convenient to denote the process a name
quotes. We already have the notation $x = \quotep{P}$, but it will be
convenient to introduce an alternate notation, $\procn{x}$, when we
want to emphasize the connection to the use of the name. Note that, by
virtue of name equivalence, $\quotep{\procn{x}} \nameeq x$; so, the
notation is consistent with previous definitions.

Further, because names have structure it is possible to effect
substitutions on the basis of that structure. This means we need to
upgrade our notation for substitutions, which we accomplish by
adapting comprehension notation. Thus,

\begin{mathpar}
  P\{ y / x : x \in S \}
\end{mathpar}

is interpreted to mean the process derived from P by replacing (in a
capture-avoiding manner) each occurrence of $x$ in $S$ by $y$. For example,

\begin{mathpar}
  P\{ \quotep{\procn{x}|\procn{x}} / x : x \in \freenames{P} \}
\end{mathpar}

will replace each (occurrence) of a free name $x$ in $P$ by
$\quotep{\procn{x}|\procn{x}}$.

Also, we will avail ourselves of the notation $x^{L}$ and $x^{R}$ to
denote injections of a name into disjoint copies of the name
space. There are numerous ways to accomplish this. One example can be
found in \cite{MeredithR05}. This notation overloads to vectors of
names: $\vec{x}^{\pi} := (x_{i}^{\pi} \; : \; 0 \leq i < |\vec{x}| )$ where $\pi \in \{L,R\}$.

We also use $P^{\Box} := P|\Box$.

In \cite{MeredithR05} an interpretation of the new operator is
given. It turns out that there are several possible interpretations
all enjoying the requisite algebraic properties of the operator (see
\cite{milner91polyadicpi}). We will therefore make liberal use of
$(\nu\; \vec{x})P$.

% subsection the_syntax_and_semantics_of_the_notation_system (end)   

\input{qm2pi.qmops} 

\input{qm2pi.sterngerlach} 

\input{qm2pi.metric} 

% section concurrent_process_calculi (end)

%\input{qm2pi.proofsketch}

% section proof sketch (end)

%\input{qm2pi.slviaknots} 

% section spatial logic via knots (end)

\input{qm2pi.conclusion}

% section conclusion (end)

%\input{qm2pi.dtcodes} 

% section wiring algorithm (end)

\input{qm2pi.ack} 

% section acknowledgments (end)

\newpage


\bibliographystyle{plain}   
\bibliography{../../biblios/main.bib}

\input{qm2pi.rhodetails}

\end{document}

 

%\documentclass[12pt]{llncs}
%\documentclass{jktr}

\usepackage[pdftex]{hyperref}                   
\usepackage {listings}
\usepackage {mathpartir}
\usepackage{bcprules}
%\usepackage{listings}
                       
\usepackage{graphicx} 
%\usepackage[margins=2.5cm,nohead,nofoot]{geometry}
%\usepackage{geometry}
\usepackage{amsfonts}
\usepackage{amstext}
\usepackage{latexsym}
\usepackage{amssymb}
\usepackage{color}


%\include{myPreamble}
\include{qm2pi.local} 

%\ifpdf
%\usepackage[pdftex]{graphicx}
%\else
%\usepackage{graphicx}
%\fi

 % \ifpdf
%  \usepackage{pdfsync}
%  \if


%\title{Brief Article}
%\author{David F. Snyder}
%\author{L.G. Meredith}

%\address{Dept. of Math., Texas State University--San Marcos, San Marcos, TX 78666}
       
\pagestyle{empty}


\begin{document}

\lstset{language=[Objective]Caml,frame=shadowbox}

\input{qm2pi.front}

% section front matter (end)

\input{qm2pi.intro} 
 
% section introduction (end)

% \input{qm2pi.knotations} 

% section notation (end)

\input{qm2pi.process.calculi} 

% section concurrent_process_calculi_and_spatial_logics_ (end)
    
%\input{qm2pi.knots2pi} 

%\input{qm2pi.trefoil} 

%\input{qm2pi.mainthm} 

% subsection basic_interpretation (end)

%\input{qm2pi.rho.presentation} 
\subsection{The syntax and semantics of the notation system}\label{sub:the_syntax_and_semantics_of_the_notation_system} % (fold)

We now summarize a technical presentation of the calculus that
embodies our theory of dynamics. The typical presentation of such a
calculus follows the style of giving generators and relations on
them. The grammar, below, describing term constructors, freely
generates the set of processes, $\Proc$. This set is then quotiented
by a relation known as structural congruence and it is over this set
that the notion of dynamics is expressed. This presentation is
essentially that of \cite{MeredithR05} with the addition of
polyadicity and summation. For readability we have relegated some of
the technical subtleties to an appendix.

\subsubsection{Process grammar}\label{subsub:process_grammar}

\begin{mathpar}
  \inferrule* [lab=synchronization] {} {{M} \bc \pzero \;|\; x?F \;|\; x!C }
  \and
  \inferrule* [lab=abstraction] {} {{F} \bc (x)P}
  \and
  \inferrule* [lab=concretion] {} {{C} \bc \langle Q \rangle}
  \and
  \inferrule* [lab=process] {} {{P,Q} \bc M \;| \;P|Q \;|\; @{x}}
  \and
  \inferrule* [lab=name] {} {{x} \bc \quotep{P}}
\end{mathpar} 

Note that $\vec{x}$ (resp. $\vec{P}$) denotes a vector of names
(resp. processes) of length $|\vec{x}|$ (resp. $|\vec{P}|$). We adopt
the following useful abbreviations.

\begin{mathpar}
   x?(\vec{y}).P := x.(\vec{y})P \and  x\clift{\vec{P}} := x.\clift{\vec{P}}
   \and x!(y) := \lift{x}{\dropn{y}}
   \and \Pi_{i=0}^{n-1}P_i := P_0 | \ldots | P_{n-1}
\end{mathpar}

\subsubsection{Structural congruence}

\paragraph{Free and bound names and alpha-equivalence.} At the
core of structural equivalence is alpha-equivalence which identifies
process that are the same up to a change of variable. Formally, we
recognize the distinction between free and bound names. The free names
of a process, $\freenames{P}$, may be calculated recursively as
follows:

\begin{mathpar}
\freenames{\pzero} := \emptyset
  \and \\
  \freenames{x?(y).P} := \{ x \} \cup (\freenames{P} \setminus \{ y \})
  \and 
  \freenames{x!\langle P \rangle} := \{ x \} \cup \{ P \} 
  \and \\
  \freenames{P|Q} := \freenames{P} \cup \freenames{Q}
  \and \\
  \freenames{@{x}} := \{ x \}
\end{mathpar}

$\pi$
$\quotep{\pi}$

$\freenames{-} : \pi \to \mathcal{P}(\quotep{\pi})$

\begin{eqnarray*}
  \freenames{\pzero} & := & \emptyset \\
  \freenames{x?(y).P} & := & \{ x \} \cup (\freenames{P} \setminus \{ y \}) \\
  \freenames{x!\langle P \rangle} & := & \{ x \} \cup \{ P \} \\
  \freenames{P|Q} & := & \freenames{P} \cup \freenames{Q} \\
  \freenames{\dropn{x}} & := & \{ x \}
\end{eqnarray*}

The bound names of a process, $\boundnames{P}$, are those names occurring in $P$
that are not free. For example, in $x?(y).0$, the name $x$ is free, while $y$ is bound.

\begin{mathpar}
  \inferrule* [lab=monoidal-laws] {} { P|Q \equiv Q|P \and P|0 \equiv P \and P|(Q|R) \equiv (P|Q)|R }
\end{mathpar}

\begin{mathpar}
  \inferrule* [lab=alpha-equivalence] {} { (x)P \equiv (y)P\{y/x\} \and y \not\in \freenames{P} }
\end{mathpar}

\begin{definition}
Then two processes, $P,Q$, are alpha-equivalent if $P = Q\{\vec{y}/\vec{x}\}$ for
some $\vec{x} \in \boundnames{Q},\vec{y} \in \boundnames{P}$, where $Q\{\vec{y}/\vec{x}\}$
denotes the capture-avoiding substitution of $\vec{y}$ for $\vec{x}$ in $Q$.
\end{definition}

\begin{definition}
  The {\em structural congruence} \cite{SangiorgiWalker} , $\equiv$,
  between processes is the least congruence containing
  alpha-equivalence, satisfying the abelian monoid laws
  (associativity, commutativity and $\pzero$ as identity) for parallel
  composition $|$ and for summation $+$.
\end{definition}

\subsection{Name equivalence}

We take name equivalence, written $\nameeq$, to be the smallest
equivalence relation generated by the following rules.

\begin{mathpar}
\inferrule*[lab=Quote-drop]
{ }
{ \quotep{@{x}} \nameeq x }

\inferrule*[lab=Struct-equiv]
{ P \scong Q }
{ \quotep{P} \nameeq \quotep{Q} }
\end{mathpar}

The astute reader will have noticed that the mutual recursion of names
and processes imposes a mutual recursion on alpha-equivalence and
structural equivalence via name-equivalence. Fortunately, all of this
works out pleasantly and we may calculate in the natural way, free of
concern. The reader interested in the details is referred to the
appendix \ref{appendix:rho_details}.

\subsection{Substitution}

We use $\Proc$ for the set of processes, $\QProc$ for the set of
names, and $\id{\{}\vec{y} / \vec{x} \id{\}}$ to denote partial maps,
$s : \QProc \rightarrow \QProc$. A map, $s$ lifts, uniquely, to a map
on process terms, $\widehat{s} : \Proc \rightarrow \Proc$ by the
following equations.

\begin{mathpar}
  (0) \psubstp{Q}{P} := 0 \\
  (R \juxtap S) \psubstp{Q}{P}
  :=    
  (R)\psubstp{Q}{P} \juxtap (S) \psubstp{Q}{P} \\
  (x?(y).R) \psubstp{Q}{P}    
  :=    
  (x)\substp{Q}{P} (z)\concat( (R \psubstn{z}{y}) \psubstp{Q}{P} ) \\
  (\lift{x}{R}) \psubstp{Q}{P}  
  :=
  \lift{(x)\substp{Q}{P}}{ R \psubstp{Q}{P} } \\
%   (\dropn{x})  \psubstp{Q}{P}       
%   := 
%   \left\{ 
%     \begin{array}{ccc} 
%       \dropn{\quotep{Q}} & & x \nameeq \quotep{P} \\
%       \dropn{x} & & otherwise \\
%     \end{array}
%   \right. 
  (\dropn{x})  \psubstp{Q}{P}       
  := 
  \left\{ 
    \begin{array}{ccc} 
      Q & & x \nameeq \quotep{P} \\
      \dropn{x} & & otherwise \\
    \end{array}
  \right.
\end{mathpar}
 

where

\begin{eqnarray}
  (x)\id{\{} \lpquote Q \rpquote / \lpquote P \rpquote \id{\}}            = 
  \left\{ 
    \begin{array}{ccc}
      \lpquote Q \rpquote & & x \nameeq \lpquote P \rpquote \\
      x & & otherwise \\
    \end{array}
  \right. \nonumber
\end{eqnarray}

and $z$ is chosen distinct from $\quotep{P}$, $\quotep{Q}$, the free
names in $Q$, and all the names in $R$. Our $\alpha$-equivalence will
be built in the standard way from this substitution.

\begin{remark}\label{rem:no_self_referential_names}
  One consequence of these definitions is that $\forall P. \quotep{P}
  \not\in \freenames{P}$.
\end{remark}

\subsection{ Dynamic quote: an example }

Anticipating something of what's to come, consider applying the
substitution, $\widehat{\id{\{}u / z \id{\}}}$, to the following pair
of processes, $\lift{w}{y!(z)}$ and $w[ \lpquote y!(z) \rpquote ]$.

\begin{eqnarray}
	\lift{w}{y!(z)}\widehat{\id{\{}u / z \id{\}}}
		& = &
		\lift{w}{y!(u)} \nonumber\\
	w[ \lpquote y!(z) \rpquote ] \widehat{ \id{\{}u / z \id{\}} }
		& = &
		w[ \lpquote y!(z) \rpquote ] \nonumber
\end{eqnarray}

Because the body of the process between quotes is impervious to
substitution, we get radically different answers. In fact, by
examining the first process in an input context,
e.g. $x?(z).\lift{w}{y!(z)}$, we see that the process under the lift
operator may be shaped by prefixed inputs binding a name inside it. In
this sense, the lift operator will be seen as a way to dynamically
construct processes before reifying them as names.

Finally equipped with these standard features we can present the
dynamics of the calculus.

\subsubsection{Operational semantics} 

Finally, we introduce the computational dynamics. What marks these
algebras as distinct from other more traditionally studied algebraic
structures, e.g. vector spaces or polynomial rings, is the manner in
which dynamics is captured. In traditional structures, dynamics is typically
expressed through morphisms between such structures, as in linear maps
between vector spaces or morphisms between rings. In algebras
associated with the semantics of computation, the dynamics is
expressed as part of the algebraic structure itself, through a
reduction reduction relation typically denoted by $\red$. Below, we
give a recursive presentation of this relation for the calculus used
in the encoding.

$\red \subseteq \pi \times \pi$
$\red : \pi \to \mathcal{P}(\pi)$

\begin{mathpar}
  \inferrule* [lab=Comm] { \textsf{match}( x_{src}, x_{trgt} ) } { x_{trgt}?(y)P \; | \; x_{src}!\langle {Q} \rangle \red P\{\quotep{Q}/y}\} }
  \and \\
  \inferrule* [lab=Par] {{P} \red {P}'} {{{P} | {Q}} \red {{P}' | {Q}}}
  \and
  \inferrule* [lab=Equiv]{{{P} \scong {P}'} \andalso {{P}' \red {Q}'} \andalso {{Q}' \scong {Q}}}{{P} \red {Q}}
\end{mathpar}

\begin{eqnarray*}
  match_{\equiv} (\quotep{P},\quotep{Q}) & := & P \equiv Q \\
  match_{\dagger}(\quotep{P},\quotep{Q}) & := & \forall R. P|Q \red^{*} R => R \red^{*} 0 \\
  match_{K}(\quotep{P},\quotep{Q}) & := & K \mbox{ for some context } K
\end{eqnarray*}

$u?(x)P | u!\langle Q \rangle \red P\{\quotep{Q}/x\}$

%We write $\wred$ for $\red^*$, and $P\red$ if $\exists Q $ such that $ P \red Q$.
We write $P\red$ if $\exists Q $ such that $ P \red Q$ and $P\not\red$, otherwise.

\section{Replication}

As mentioned before, it is known that replication (and hence
recursion) can be implemented in a higher-order process algebra
\cite{SangiorgiWalker}. As our first example of calculation with the
machinery thus far presented we give the construction explicitly in
the {\rhoc}.

\begin{eqnarray}
	D_{x} & := & \prefix{x}{y}{(\binpar{\outputp{x}{y}}{@{y}})} \nonumber\\
	\bangp_{x}{P} & := & \binpar{{x}!\langle{\binpar{D_{x}}{P}}\rangle}{D_{x}} \nonumber
\end{eqnarray}

\begin{eqnarray}
	\bangp_{x}{P} & & \nonumber\\
	=
	& {x}!\langle{(\prefix{x}{y}{(\outputp{x}{y} | @{y})) | P}}\rangle 
	      | \prefix{x}{y}{(\outputp{x}{y} | @{y})} & \nonumber\\
	\red
	& (\outputp{x}{y} | @{y})\substn{\quotep{(\prefix{x}{y}{(@{y} | \outputp{x}{y})) | P}}}{y} & \nonumber\\
	=
	& \outputp{x}{\quotep{(\prefix{x}{y}{(\outputp{x}{y} | @{y})) | P}}}
	  | {(\prefix{x}{y}{(\outputp{x}{y} | @{y})) | P}} & \nonumber\\
	\red
	& \ldots & \nonumber\\
	\red^*
	& P | P | \ldots & \nonumber
\end{eqnarray}

Of course, this encoding, as an implementation, runs away, unfolding
$\bangp{P}$ eagerly. A lazier and more implementable replication
operator, restricted to input-guarded processes, may be obtained as follows.

\begin{eqnarray}
\bangp{\prefix{u}{v}{P}} 
	:= 
	\binpar{\lift{x}{\prefix{u}{v}{(\binpar{D(x)}{P})}}}{D(x)} \nonumber
\end{eqnarray}

\begin{remark}
  Note that the lazier definition still does not deal with summation
  or mixed summation (i.e. sums over input and output). The reader is
  invited to construct definitions of replication that deal with these
  features. 

  Further, the definitions are parameterized in a name, $x$. Can you,
  gentle reader, make a definition that eliminates this parameter and
  guarantees no accidental interaction between the replication
  machinery and the process being replicated -- i.e. no accidental
  sharing of names used by the process to get its work done and the
  name(s) used by the replication to effect copying. This latter
  revision of the definition of replication is crucial to obtaining
  the expected identity $!!P \sim !P$.
\end{remark}

\begin{remark}\label{rem:paradoxical_combinator}
  The reader familiar with the lambda calculus will have noticed the
  similarity between $D$ and the paradoxical combinator.

  [Ed. note: the existence of this seems to suggest we have to be more
  restrictive on the set of processes and names we admit if we are to
  support no-cloning.]
\end{remark}

\subsubsection{Bisimulation}

The computational dynamics gives rise to another kind of equivalence,
the equivalence of computational behavior. As previously mentioned
this is typically captured \emph{via} some form of bisimulation.

% The notion we use in this paper is weak barbed bisimulation
% \cite{milner91polyadicpi}.

The notion we use in this paper is derived from weak barbed
bisimulation \cite{milner91polyadicpi}. 

\begin{definition}
An \emph{observation relation}, $\downarrow_{\mathcal N}$, over a set
of names, $\mathcal N$, is the smallest relation satisfying the rules
below.

\infrule[Out-barb]{y \in {\mathcal N}, \; x \nameeq y}
		  {\outputp{x}{v} \downarrow_{\mathcal N} x}
\infrule[Par-barb]{\mbox{$P\downarrow_{\mathcal N} x$ or $Q\downarrow_{\mathcal N} x$}}
		  {\binpar{P}{Q} \downarrow_{\mathcal N} x}

We write $P \Downarrow_{\mathcal N} x$ if there is $Q$ such that 
$P \wred Q$ and $Q \downarrow_{\mathcal N} x$.
\end{definition}

\begin{definition}
%\label{def.bbisim}
An  ${\mathcal N}$-\emph{barbed bisimulation} over a set of names, ${\mathcal N}$, is a symmetric binary relation 
${\mathcal S}_{\mathcal N}$ between agents such that $P\rel{S}_{\mathcal N}Q$ implies:
\begin{enumerate}
\item If $P \red P'$ then $Q \wred Q'$ and $P'\rel{S}_{\mathcal N} Q'$.
\item If $P\downarrow_{\mathcal N} x$, then $Q\Downarrow_{\mathcal N} x$.
\end{enumerate}
$P$ is ${\mathcal N}$-barbed bisimilar to $Q$, written
$P \wbbisim_{\mathcal N} Q$, if $P \rel{S}_{\mathcal N} Q$ for some ${\mathcal N}$-barbed bisimulation ${\mathcal S}_{\mathcal N}$.
\end{definition}

$\mathcal{R} \subseteq \pi \times \pi$

$P \mathcal{R} Q => \forall P'. P \red P' \Rightarrow \exists Q'. Q \red Q', P' \mathcal{R} Q'$

$P \vdash x \Rightarrow Q \vdash x$

\begin{mathpar}
  \inferrule*[lab=Out-barb]{x \nameeq y}{{y}!\langle{Q}\rangle \vdash x}
  \and
  \inferrule*[lab=Par-barb]{\mbox{$P\vdash x$ or $Q\vdash x$}}{\binpar{P}{Q} \vdash x}
\end{mathpar}

\subsubsection{Contexts}

One of the principle advantages of computational calculi like the
$\pi$-calculus is a well-defined notion of context,
contextual-equivalence and a correlation between
contextual-equivalence and notions of bisimulation. The notion of
context allows the decomposition of a process into (sub-)process and
its syntactic environment, its context. Thus, a context may be
thought of as a process with a ``hole'' (written $\Box$) in it. The
application of a context $M$ to a process $P$, written $M[P]$, is
tantamount to filling the hole in $M$ with $P$. In this paper we do
not need the full weight of this theory, but do make use of the notion
of context in the proof the main theorem. 

\begin{mathpar}
  \inferrule* [lab=summation] {} {{M_{M},M_{N}} \bc \Box \;|\; x.M_{A} \;|\; M_{M}+M_{N}}
  \and
  \inferrule* [lab=agent] {} {{M_{A}} \bc (\vec{x})M_{P} \;| \; \clift{P_0,\ldots,M_{P},\ldots,P_N}}
  \and \\
  \inferrule* [lab=process] {} {{M_{P}} \bc M_{N} \;| \;P|M_{P} }
\end{mathpar} 

\begin{mathpar}
  \inferrule* [lab=sychronization] {} {M_{N} \bc \Box \;|\; x?M_{F} \;|\; x!M_{C}}
  \and
  \inferrule* [lab=abstraction] {} {{M_{F}} \bc (x)M_{P} }
  \and
  \inferrule* [lab=concretion] {} {{M_{C}} \bc \langle M_{P} \rangle }
  \and \\
  \inferrule* [lab=process] {} {{M_{P}} \bc M_{N} \;| \;P|M_{P} }
\end{mathpar}

\begin{definition}[contextual application] Given a context $M$, and
  process $P$, we define the \emph{contextual application}, $M[P] :=
  M\{P/\Box\}$. That is, the contextual application of M to P is the
  substitution of $P$ for $\Box$ in $M$.
\end{definition}

$\meaningof{-} : L \to \mathcal{P}(\pi)$

\begin{mathpar}
  \inferrule* [lab=collection] {} {\meaningof{true} = \pi, \and \meaningof{~E} = \pi \setminus \meaningof{E}, \and \meaningof{E_{1} \& E_{2}} = \meaningof{E_{1}} \cap \meaningof{E_{2}}}
\end{mathpar}

\begin{mathpar}
  \inferrule* [lab=structure] {} {\meaningof{0} = \{ P \in \pi | P \equiv 0 \}, \and \\ \meaningof{E_1 | E_2} = \{ P \in \pi | P \equiv P_{1} | P_{2}, P_{1} \in \meaningof{E_{1}}, P_{2} \in \meaningof{E_2}\} }
\end{mathpar}

\begin{mathpar}
 \inferrule* [lab=behavior] {} {\meaningof{\langle a?b \rangle E} = \{ P \in \pi | P \equiv Q | u?(y)P', \\ \and \\\\ \and \\ \;\;\; u \in \meaningof{a}, \forall z.P'\{z/y\} \in \meaningof{E\{z/b\}}\}, \and \\ \meaningof{a!E} = \{ P \in \pi | P \equiv Q | x!\langle P' \rangle, x \in \meaningof{a} P' \in \meaningof{E}\} }
\end{mathpar}

\begin{mathpar}
 \inferrule* [lab=nominal] {} {\meaningof{\quotep{E}} = \{ \quotep{P} \in \quotep{\pi} | P \in \meaningof{E} \}, \and \meaningof{\quotep{P}} = \{ \quotep{Q} \in \quotep{\pi} | P \equiv Q \} \and \\ \meaningof{@\quotep{E}} = \{ P \in \pi | P \equiv @x, x \in \meaningof{E} \}}
\end{mathpar}

\begin{eqnarray*}
  \\
  \meaningof{-} : TS \to ST
\end{eqnarray*}

\begin{eqnarray*}
  \\
  L : TS \to ST
\end{eqnarray*}

\begin{eqnarray*}
  \\
  P \models E \iff P \in \meaningof{E}
\end{eqnarray*}

\begin{eqnarray*}
  P \approx_{L} Q \iff \forall E \in L. P \models E \iff Q \models E
\end{eqnarray*}

\begin{eqnarray*}
  P \approx_{K} Q
\end{eqnarray*}

\begin{eqnarray*}
  P \approx Q
\end{eqnarray*}

$\approx_{K} = \approx = \approx_{L}$

\subsubsection{Contextual duality}

Note that contexts extend the quotation operation to a family of
operations from processes to names. Given a context, $M$, we can
define a \emph{nominal context}, $\quotep{M}$ by $\quotep{M}[P] :=
\quotep{M[P]}$. To foreshadow what is to come we observe that these
operations enjoy a duality with processes very much like the duality
between vectors and maps from vectors to scalars.

Further, because the calculus is essentially higher-order, we have a
correspondence between contexts and processes. More specifically,
given a name $x$ and a context $M$ we can construct $M^{*}_{x}$ such
that 

\begin{mathpar}
  M^{*}_{x} | \lift{x}{P} \red M[P]
\end{mathpar}

namely,

\begin{mathpar}
  M^{*}_{x} := x?(u).M[\dropn{u}]
\end{mathpar}

The dependence of $M^{*}_{x}$ on a name makes it an abstraction, 

\begin{mathpar}
  M^{*} := (x)x?(u).M[\dropn{u}]
\end{mathpar}

\subsection{Additional notation}

It will sometimes be convenient to denote the process a name
quotes. We already have the notation $x = \quotep{P}$, but it will be
convenient to introduce an alternate notation, $\procn{x}$, when we
want to emphasize the connection to the use of the name. Note that, by
virtue of name equivalence, $\quotep{\procn{x}} \nameeq x$; so, the
notation is consistent with previous definitions.

Further, because names have structure it is possible to effect
substitutions on the basis of that structure. This means we need to
upgrade our notation for substitutions, which we accomplish by
adapting comprehension notation. Thus,

\begin{mathpar}
  P\{ y / x : x \in S \}
\end{mathpar}

is interpreted to mean the process derived from P by replacing (in a
capture-avoiding manner) each occurrence of $x$ in $S$ by $y$. For example,

\begin{mathpar}
  P\{ \quotep{\procn{x}|\procn{x}} / x : x \in \freenames{P} \}
\end{mathpar}

will replace each (occurrence) of a free name $x$ in $P$ by
$\quotep{\procn{x}|\procn{x}}$.

Also, we will avail ourselves of the notation $x^{L}$ and $x^{R}$ to
denote injections of a name into disjoint copies of the name
space. There are numerous ways to accomplish this. One example can be
found in \cite{MeredithR05}. This notation overloads to vectors of
names: $\vec{x}^{\pi} := (x_{i}^{\pi} \; : \; 0 \leq i < |\vec{x}| )$ where $\pi \in \{L,R\}$.

We also use $P^{\Box} := P|\Box$.

In \cite{MeredithR05} an interpretation of the new operator is
given. It turns out that there are several possible interpretations
all enjoying the requisite algebraic properties of the operator (see
\cite{milner91polyadicpi}). We will therefore make liberal use of
$(\nu\; \vec{x})P$.

% subsection the_syntax_and_semantics_of_the_notation_system (end)   

\input{qm2pi.qmops} 

\input{qm2pi.sterngerlach} 

\input{qm2pi.metric} 

% section concurrent_process_calculi (end)

%\input{qm2pi.proofsketch}

% section proof sketch (end)

%\input{qm2pi.slviaknots} 

% section spatial logic via knots (end)

\input{qm2pi.conclusion}

% section conclusion (end)

%\input{qm2pi.dtcodes} 

% section wiring algorithm (end)

\input{qm2pi.ack} 

% section acknowledgments (end)

\newpage


\bibliographystyle{plain}   
\bibliography{../../biblios/main.bib}

\input{qm2pi.rhodetails}

\end{document}

 

% subsection basic_interpretation (end)

%\input{qm2pi.rho.presentation} 
\subsection{The syntax and semantics of the notation system}\label{sub:the_syntax_and_semantics_of_the_notation_system} % (fold)

We now summarize a technical presentation of the calculus that
embodies our theory of dynamics. The typical presentation of such a
calculus follows the style of giving generators and relations on
them. The grammar, below, describing term constructors, freely
generates the set of processes, $\Proc$. This set is then quotiented
by a relation known as structural congruence and it is over this set
that the notion of dynamics is expressed. This presentation is
essentially that of \cite{MeredithR05} with the addition of
polyadicity and summation. For readability we have relegated some of
the technical subtleties to an appendix.

\subsubsection{Process grammar}\label{subsub:process_grammar}

\begin{mathpar}
  \inferrule* [lab=synchronization] {} {{M} \bc \pzero \;|\; x?F \;|\; x!C }
  \and
  \inferrule* [lab=abstraction] {} {{F} \bc (x)P}
  \and
  \inferrule* [lab=concretion] {} {{C} \bc \langle Q \rangle}
  \and
  \inferrule* [lab=process] {} {{P,Q} \bc M \;| \;P|Q \;|\; @{x}}
  \and
  \inferrule* [lab=name] {} {{x} \bc \quotep{P}}
\end{mathpar} 

Note that $\vec{x}$ (resp. $\vec{P}$) denotes a vector of names
(resp. processes) of length $|\vec{x}|$ (resp. $|\vec{P}|$). We adopt
the following useful abbreviations.

\begin{mathpar}
   x?(\vec{y}).P := x.(\vec{y})P \and  x\clift{\vec{P}} := x.\clift{\vec{P}}
   \and x!(y) := \lift{x}{\dropn{y}}
   \and \Pi_{i=0}^{n-1}P_i := P_0 | \ldots | P_{n-1}
\end{mathpar}

\subsubsection{Structural congruence}

\paragraph{Free and bound names and alpha-equivalence.} At the
core of structural equivalence is alpha-equivalence which identifies
process that are the same up to a change of variable. Formally, we
recognize the distinction between free and bound names. The free names
of a process, $\freenames{P}$, may be calculated recursively as
follows:

\begin{mathpar}
\freenames{\pzero} := \emptyset
  \and \\
  \freenames{x?(y).P} := \{ x \} \cup (\freenames{P} \setminus \{ y \})
  \and 
  \freenames{x!\langle P \rangle} := \{ x \} \cup \{ P \} 
  \and \\
  \freenames{P|Q} := \freenames{P} \cup \freenames{Q}
  \and \\
  \freenames{@{x}} := \{ x \}
\end{mathpar}

$\pi$
$\quotep{\pi}$

$\freenames{-} : \pi \to \mathcal{P}(\quotep{\pi})$

\begin{eqnarray*}
  \freenames{\pzero} & := & \emptyset \\
  \freenames{x?(y).P} & := & \{ x \} \cup (\freenames{P} \setminus \{ y \}) \\
  \freenames{x!\langle P \rangle} & := & \{ x \} \cup \{ P \} \\
  \freenames{P|Q} & := & \freenames{P} \cup \freenames{Q} \\
  \freenames{\dropn{x}} & := & \{ x \}
\end{eqnarray*}

The bound names of a process, $\boundnames{P}$, are those names occurring in $P$
that are not free. For example, in $x?(y).0$, the name $x$ is free, while $y$ is bound.

\begin{mathpar}
  \inferrule* [lab=monoidal-laws] {} { P|Q \equiv Q|P \and P|0 \equiv P \and P|(Q|R) \equiv (P|Q)|R }
\end{mathpar}

\begin{mathpar}
  \inferrule* [lab=alpha-equivalence] {} { (x)P \equiv (y)P\{y/x\} \and y \not\in \freenames{P} }
\end{mathpar}

\begin{definition}
Then two processes, $P,Q$, are alpha-equivalent if $P = Q\{\vec{y}/\vec{x}\}$ for
some $\vec{x} \in \boundnames{Q},\vec{y} \in \boundnames{P}$, where $Q\{\vec{y}/\vec{x}\}$
denotes the capture-avoiding substitution of $\vec{y}$ for $\vec{x}$ in $Q$.
\end{definition}

\begin{definition}
  The {\em structural congruence} \cite{SangiorgiWalker} , $\equiv$,
  between processes is the least congruence containing
  alpha-equivalence, satisfying the abelian monoid laws
  (associativity, commutativity and $\pzero$ as identity) for parallel
  composition $|$ and for summation $+$.
\end{definition}

\subsection{Name equivalence}

We take name equivalence, written $\nameeq$, to be the smallest
equivalence relation generated by the following rules.

\begin{mathpar}
\inferrule*[lab=Quote-drop]
{ }
{ \quotep{@{x}} \nameeq x }

\inferrule*[lab=Struct-equiv]
{ P \scong Q }
{ \quotep{P} \nameeq \quotep{Q} }
\end{mathpar}

The astute reader will have noticed that the mutual recursion of names
and processes imposes a mutual recursion on alpha-equivalence and
structural equivalence via name-equivalence. Fortunately, all of this
works out pleasantly and we may calculate in the natural way, free of
concern. The reader interested in the details is referred to the
appendix \ref{appendix:rho_details}.

\subsection{Substitution}

We use $\Proc$ for the set of processes, $\QProc$ for the set of
names, and $\id{\{}\vec{y} / \vec{x} \id{\}}$ to denote partial maps,
$s : \QProc \rightarrow \QProc$. A map, $s$ lifts, uniquely, to a map
on process terms, $\widehat{s} : \Proc \rightarrow \Proc$ by the
following equations.

\begin{mathpar}
  (0) \psubstp{Q}{P} := 0 \\
  (R \juxtap S) \psubstp{Q}{P}
  :=    
  (R)\psubstp{Q}{P} \juxtap (S) \psubstp{Q}{P} \\
  (x?(y).R) \psubstp{Q}{P}    
  :=    
  (x)\substp{Q}{P} (z)\concat( (R \psubstn{z}{y}) \psubstp{Q}{P} ) \\
  (\lift{x}{R}) \psubstp{Q}{P}  
  :=
  \lift{(x)\substp{Q}{P}}{ R \psubstp{Q}{P} } \\
%   (\dropn{x})  \psubstp{Q}{P}       
%   := 
%   \left\{ 
%     \begin{array}{ccc} 
%       \dropn{\quotep{Q}} & & x \nameeq \quotep{P} \\
%       \dropn{x} & & otherwise \\
%     \end{array}
%   \right. 
  (\dropn{x})  \psubstp{Q}{P}       
  := 
  \left\{ 
    \begin{array}{ccc} 
      Q & & x \nameeq \quotep{P} \\
      \dropn{x} & & otherwise \\
    \end{array}
  \right.
\end{mathpar}
 

where

\begin{eqnarray}
  (x)\id{\{} \lpquote Q \rpquote / \lpquote P \rpquote \id{\}}            = 
  \left\{ 
    \begin{array}{ccc}
      \lpquote Q \rpquote & & x \nameeq \lpquote P \rpquote \\
      x & & otherwise \\
    \end{array}
  \right. \nonumber
\end{eqnarray}

and $z$ is chosen distinct from $\quotep{P}$, $\quotep{Q}$, the free
names in $Q$, and all the names in $R$. Our $\alpha$-equivalence will
be built in the standard way from this substitution.

\begin{remark}\label{rem:no_self_referential_names}
  One consequence of these definitions is that $\forall P. \quotep{P}
  \not\in \freenames{P}$.
\end{remark}

\subsection{ Dynamic quote: an example }

Anticipating something of what's to come, consider applying the
substitution, $\widehat{\id{\{}u / z \id{\}}}$, to the following pair
of processes, $\lift{w}{y!(z)}$ and $w[ \lpquote y!(z) \rpquote ]$.

\begin{eqnarray}
	\lift{w}{y!(z)}\widehat{\id{\{}u / z \id{\}}}
		& = &
		\lift{w}{y!(u)} \nonumber\\
	w[ \lpquote y!(z) \rpquote ] \widehat{ \id{\{}u / z \id{\}} }
		& = &
		w[ \lpquote y!(z) \rpquote ] \nonumber
\end{eqnarray}

Because the body of the process between quotes is impervious to
substitution, we get radically different answers. In fact, by
examining the first process in an input context,
e.g. $x?(z).\lift{w}{y!(z)}$, we see that the process under the lift
operator may be shaped by prefixed inputs binding a name inside it. In
this sense, the lift operator will be seen as a way to dynamically
construct processes before reifying them as names.

Finally equipped with these standard features we can present the
dynamics of the calculus.

\subsubsection{Operational semantics} 

Finally, we introduce the computational dynamics. What marks these
algebras as distinct from other more traditionally studied algebraic
structures, e.g. vector spaces or polynomial rings, is the manner in
which dynamics is captured. In traditional structures, dynamics is typically
expressed through morphisms between such structures, as in linear maps
between vector spaces or morphisms between rings. In algebras
associated with the semantics of computation, the dynamics is
expressed as part of the algebraic structure itself, through a
reduction reduction relation typically denoted by $\red$. Below, we
give a recursive presentation of this relation for the calculus used
in the encoding.

$\red \subseteq \pi \times \pi$
$\red : \pi \to \mathcal{P}(\pi)$

\begin{mathpar}
  \inferrule* [lab=Comm] { \textsf{match}( x_{src}, x_{trgt} ) } { x_{trgt}?(y)P \; | \; x_{src}!\langle {Q} \rangle \red P\{\quotep{Q}/y}\} }
  \and \\
  \inferrule* [lab=Par] {{P} \red {P}'} {{{P} | {Q}} \red {{P}' | {Q}}}
  \and
  \inferrule* [lab=Equiv]{{{P} \scong {P}'} \andalso {{P}' \red {Q}'} \andalso {{Q}' \scong {Q}}}{{P} \red {Q}}
\end{mathpar}

\begin{eqnarray*}
  match_{\equiv} (\quotep{P},\quotep{Q}) & := & P \equiv Q \\
  match_{\dagger}(\quotep{P},\quotep{Q}) & := & \forall R. P|Q \red^{*} R => R \red^{*} 0 \\
  match_{K}(\quotep{P},\quotep{Q}) & := & K \mbox{ for some context } K
\end{eqnarray*}

$u?(x)P | u!\langle Q \rangle \red P\{\quotep{Q}/x\}$

%We write $\wred$ for $\red^*$, and $P\red$ if $\exists Q $ such that $ P \red Q$.
We write $P\red$ if $\exists Q $ such that $ P \red Q$ and $P\not\red$, otherwise.

\section{Replication}

As mentioned before, it is known that replication (and hence
recursion) can be implemented in a higher-order process algebra
\cite{SangiorgiWalker}. As our first example of calculation with the
machinery thus far presented we give the construction explicitly in
the {\rhoc}.

\begin{eqnarray}
	D_{x} & := & \prefix{x}{y}{(\binpar{\outputp{x}{y}}{@{y}})} \nonumber\\
	\bangp_{x}{P} & := & \binpar{{x}!\langle{\binpar{D_{x}}{P}}\rangle}{D_{x}} \nonumber
\end{eqnarray}

\begin{eqnarray}
	\bangp_{x}{P} & & \nonumber\\
	=
	& {x}!\langle{(\prefix{x}{y}{(\outputp{x}{y} | @{y})) | P}}\rangle 
	      | \prefix{x}{y}{(\outputp{x}{y} | @{y})} & \nonumber\\
	\red
	& (\outputp{x}{y} | @{y})\substn{\quotep{(\prefix{x}{y}{(@{y} | \outputp{x}{y})) | P}}}{y} & \nonumber\\
	=
	& \outputp{x}{\quotep{(\prefix{x}{y}{(\outputp{x}{y} | @{y})) | P}}}
	  | {(\prefix{x}{y}{(\outputp{x}{y} | @{y})) | P}} & \nonumber\\
	\red
	& \ldots & \nonumber\\
	\red^*
	& P | P | \ldots & \nonumber
\end{eqnarray}

Of course, this encoding, as an implementation, runs away, unfolding
$\bangp{P}$ eagerly. A lazier and more implementable replication
operator, restricted to input-guarded processes, may be obtained as follows.

\begin{eqnarray}
\bangp{\prefix{u}{v}{P}} 
	:= 
	\binpar{\lift{x}{\prefix{u}{v}{(\binpar{D(x)}{P})}}}{D(x)} \nonumber
\end{eqnarray}

\begin{remark}
  Note that the lazier definition still does not deal with summation
  or mixed summation (i.e. sums over input and output). The reader is
  invited to construct definitions of replication that deal with these
  features. 

  Further, the definitions are parameterized in a name, $x$. Can you,
  gentle reader, make a definition that eliminates this parameter and
  guarantees no accidental interaction between the replication
  machinery and the process being replicated -- i.e. no accidental
  sharing of names used by the process to get its work done and the
  name(s) used by the replication to effect copying. This latter
  revision of the definition of replication is crucial to obtaining
  the expected identity $!!P \sim !P$.
\end{remark}

\begin{remark}\label{rem:paradoxical_combinator}
  The reader familiar with the lambda calculus will have noticed the
  similarity between $D$ and the paradoxical combinator.

  [Ed. note: the existence of this seems to suggest we have to be more
  restrictive on the set of processes and names we admit if we are to
  support no-cloning.]
\end{remark}

\subsubsection{Bisimulation}

The computational dynamics gives rise to another kind of equivalence,
the equivalence of computational behavior. As previously mentioned
this is typically captured \emph{via} some form of bisimulation.

% The notion we use in this paper is weak barbed bisimulation
% \cite{milner91polyadicpi}.

The notion we use in this paper is derived from weak barbed
bisimulation \cite{milner91polyadicpi}. 

\begin{definition}
An \emph{observation relation}, $\downarrow_{\mathcal N}$, over a set
of names, $\mathcal N$, is the smallest relation satisfying the rules
below.

\infrule[Out-barb]{y \in {\mathcal N}, \; x \nameeq y}
		  {\outputp{x}{v} \downarrow_{\mathcal N} x}
\infrule[Par-barb]{\mbox{$P\downarrow_{\mathcal N} x$ or $Q\downarrow_{\mathcal N} x$}}
		  {\binpar{P}{Q} \downarrow_{\mathcal N} x}

We write $P \Downarrow_{\mathcal N} x$ if there is $Q$ such that 
$P \wred Q$ and $Q \downarrow_{\mathcal N} x$.
\end{definition}

\begin{definition}
%\label{def.bbisim}
An  ${\mathcal N}$-\emph{barbed bisimulation} over a set of names, ${\mathcal N}$, is a symmetric binary relation 
${\mathcal S}_{\mathcal N}$ between agents such that $P\rel{S}_{\mathcal N}Q$ implies:
\begin{enumerate}
\item If $P \red P'$ then $Q \wred Q'$ and $P'\rel{S}_{\mathcal N} Q'$.
\item If $P\downarrow_{\mathcal N} x$, then $Q\Downarrow_{\mathcal N} x$.
\end{enumerate}
$P$ is ${\mathcal N}$-barbed bisimilar to $Q$, written
$P \wbbisim_{\mathcal N} Q$, if $P \rel{S}_{\mathcal N} Q$ for some ${\mathcal N}$-barbed bisimulation ${\mathcal S}_{\mathcal N}$.
\end{definition}

$\mathcal{R} \subseteq \pi \times \pi$

$P \mathcal{R} Q => \forall P'. P \red P' \Rightarrow \exists Q'. Q \red Q', P' \mathcal{R} Q'$

$P \vdash x \Rightarrow Q \vdash x$

\begin{mathpar}
  \inferrule*[lab=Out-barb]{x \nameeq y}{{y}!\langle{Q}\rangle \vdash x}
  \and
  \inferrule*[lab=Par-barb]{\mbox{$P\vdash x$ or $Q\vdash x$}}{\binpar{P}{Q} \vdash x}
\end{mathpar}

\subsubsection{Contexts}

One of the principle advantages of computational calculi like the
$\pi$-calculus is a well-defined notion of context,
contextual-equivalence and a correlation between
contextual-equivalence and notions of bisimulation. The notion of
context allows the decomposition of a process into (sub-)process and
its syntactic environment, its context. Thus, a context may be
thought of as a process with a ``hole'' (written $\Box$) in it. The
application of a context $M$ to a process $P$, written $M[P]$, is
tantamount to filling the hole in $M$ with $P$. In this paper we do
not need the full weight of this theory, but do make use of the notion
of context in the proof the main theorem. 

\begin{mathpar}
  \inferrule* [lab=summation] {} {{M_{M},M_{N}} \bc \Box \;|\; x.M_{A} \;|\; M_{M}+M_{N}}
  \and
  \inferrule* [lab=agent] {} {{M_{A}} \bc (\vec{x})M_{P} \;| \; \clift{P_0,\ldots,M_{P},\ldots,P_N}}
  \and \\
  \inferrule* [lab=process] {} {{M_{P}} \bc M_{N} \;| \;P|M_{P} }
\end{mathpar} 

\begin{mathpar}
  \inferrule* [lab=sychronization] {} {M_{N} \bc \Box \;|\; x?M_{F} \;|\; x!M_{C}}
  \and
  \inferrule* [lab=abstraction] {} {{M_{F}} \bc (x)M_{P} }
  \and
  \inferrule* [lab=concretion] {} {{M_{C}} \bc \langle M_{P} \rangle }
  \and \\
  \inferrule* [lab=process] {} {{M_{P}} \bc M_{N} \;| \;P|M_{P} }
\end{mathpar}

\begin{definition}[contextual application] Given a context $M$, and
  process $P$, we define the \emph{contextual application}, $M[P] :=
  M\{P/\Box\}$. That is, the contextual application of M to P is the
  substitution of $P$ for $\Box$ in $M$.
\end{definition}

$\meaningof{-} : L \to \mathcal{P}(\pi)$

\begin{mathpar}
  \inferrule* [lab=collection] {} {\meaningof{true} = \pi, \and \meaningof{~E} = \pi \setminus \meaningof{E}, \and \meaningof{E_{1} \& E_{2}} = \meaningof{E_{1}} \cap \meaningof{E_{2}}}
\end{mathpar}

\begin{mathpar}
  \inferrule* [lab=structure] {} {\meaningof{0} = \{ P \in \pi | P \equiv 0 \}, \and \\ \meaningof{E_1 | E_2} = \{ P \in \pi | P \equiv P_{1} | P_{2}, P_{1} \in \meaningof{E_{1}}, P_{2} \in \meaningof{E_2}\} }
\end{mathpar}

\begin{mathpar}
 \inferrule* [lab=behavior] {} {\meaningof{\langle a?b \rangle E} = \{ P \in \pi | P \equiv Q | u?(y)P', \\ \and \\\\ \and \\ \;\;\; u \in \meaningof{a}, \forall z.P'\{z/y\} \in \meaningof{E\{z/b\}}\}, \and \\ \meaningof{a!E} = \{ P \in \pi | P \equiv Q | x!\langle P' \rangle, x \in \meaningof{a} P' \in \meaningof{E}\} }
\end{mathpar}

\begin{mathpar}
 \inferrule* [lab=nominal] {} {\meaningof{\quotep{E}} = \{ \quotep{P} \in \quotep{\pi} | P \in \meaningof{E} \}, \and \meaningof{\quotep{P}} = \{ \quotep{Q} \in \quotep{\pi} | P \equiv Q \} \and \\ \meaningof{@\quotep{E}} = \{ P \in \pi | P \equiv @x, x \in \meaningof{E} \}}
\end{mathpar}

\begin{eqnarray*}
  \\
  \meaningof{-} : TS \to ST
\end{eqnarray*}

\begin{eqnarray*}
  \\
  L : TS \to ST
\end{eqnarray*}

\begin{eqnarray*}
  \\
  P \models E \iff P \in \meaningof{E}
\end{eqnarray*}

\begin{eqnarray*}
  P \approx_{L} Q \iff \forall E \in L. P \models E \iff Q \models E
\end{eqnarray*}

\begin{eqnarray*}
  P \approx_{K} Q
\end{eqnarray*}

\begin{eqnarray*}
  P \approx Q
\end{eqnarray*}

$\approx_{K} = \approx = \approx_{L}$

\subsubsection{Contextual duality}

Note that contexts extend the quotation operation to a family of
operations from processes to names. Given a context, $M$, we can
define a \emph{nominal context}, $\quotep{M}$ by $\quotep{M}[P] :=
\quotep{M[P]}$. To foreshadow what is to come we observe that these
operations enjoy a duality with processes very much like the duality
between vectors and maps from vectors to scalars.

Further, because the calculus is essentially higher-order, we have a
correspondence between contexts and processes. More specifically,
given a name $x$ and a context $M$ we can construct $M^{*}_{x}$ such
that 

\begin{mathpar}
  M^{*}_{x} | \lift{x}{P} \red M[P]
\end{mathpar}

namely,

\begin{mathpar}
  M^{*}_{x} := x?(u).M[\dropn{u}]
\end{mathpar}

The dependence of $M^{*}_{x}$ on a name makes it an abstraction, 

\begin{mathpar}
  M^{*} := (x)x?(u).M[\dropn{u}]
\end{mathpar}

\subsection{Additional notation}

It will sometimes be convenient to denote the process a name
quotes. We already have the notation $x = \quotep{P}$, but it will be
convenient to introduce an alternate notation, $\procn{x}$, when we
want to emphasize the connection to the use of the name. Note that, by
virtue of name equivalence, $\quotep{\procn{x}} \nameeq x$; so, the
notation is consistent with previous definitions.

Further, because names have structure it is possible to effect
substitutions on the basis of that structure. This means we need to
upgrade our notation for substitutions, which we accomplish by
adapting comprehension notation. Thus,

\begin{mathpar}
  P\{ y / x : x \in S \}
\end{mathpar}

is interpreted to mean the process derived from P by replacing (in a
capture-avoiding manner) each occurrence of $x$ in $S$ by $y$. For example,

\begin{mathpar}
  P\{ \quotep{\procn{x}|\procn{x}} / x : x \in \freenames{P} \}
\end{mathpar}

will replace each (occurrence) of a free name $x$ in $P$ by
$\quotep{\procn{x}|\procn{x}}$.

Also, we will avail ourselves of the notation $x^{L}$ and $x^{R}$ to
denote injections of a name into disjoint copies of the name
space. There are numerous ways to accomplish this. One example can be
found in \cite{MeredithR05}. This notation overloads to vectors of
names: $\vec{x}^{\pi} := (x_{i}^{\pi} \; : \; 0 \leq i < |\vec{x}| )$ where $\pi \in \{L,R\}$.

We also use $P^{\Box} := P|\Box$.

In \cite{MeredithR05} an interpretation of the new operator is
given. It turns out that there are several possible interpretations
all enjoying the requisite algebraic properties of the operator (see
\cite{milner91polyadicpi}). We will therefore make liberal use of
$(\nu\; \vec{x})P$.

% subsection the_syntax_and_semantics_of_the_notation_system (end)   

\section{Interpretation of QM}
\subsection{Supporting definitions}
\subsubsection{Multiplication}
\begin{mathpar}
  \quotep{Q} \cdot \quotep{R} := \quotep{Q|R}
  \and \\
  \quotep{Q} \cdot P := P\{ \quotep{Q|R} / \quotep{R} : \quotep{R} \in \freenames{P} \}
\end{mathpar}

\paragraph{Discussion}
The first line needs little explanation. The second line says that
each free name of the process is replaced with the multiplication of
that name by the scalar. Multiplication of a scalar (name) by a state
(process) results in a process all the names of which have been `moved
over' by parallel composition with the process the scalar
quotes. There is a subtlety that the bound names have to be
manipulated so that multiplied names aren't accidentally
captured. There are many ways to achieve this.

\begin{remark}\label{rem:multiplication_identities}
  The reader is invited to verify that for all $x,y,z \in \QProc$ and $P \in \Proc$
  \begin{mathpar}
    x \cdot \quotep{0} \equiv x 
    \and
    x \cdot y \equiv y \cdot x
    \and
    x \cdot (y \cdot z) \equiv (x \cdot y) \cdot z
    \and \\
    \quotep{0} \cdot P \equiv P
    \and \\
    x \cdot (y \cdot P) \equiv (x \cdot y) \cdot P
    \and \\
    x \cdot (P|Q) \equiv (x \cdot P) | (x \cdot Q)
    \and \\    
  \end{mathpar}
\end{remark}

\subsubsection{Tensor product}

We define a tensor product on processes by structural induction.

\paragraph{Tensor of sums} First note that all summations, including
$\pzero$ and sequence, can be written $\Sigma_{i} x_{i}.A_{i} +
\Sigma_{j} x_{j}.C_{j}$, where we have grouped input-guarded processes
together and output-guarded processes together.

Thus, we can define the tensor product of two summations, $N_{1}\otimes N_{2}$, where

\begin{mathpar}
  N_{1} := \Sigma_{i} x_{i}.A_{i} + \Sigma_{j} x_{j}.C_{j}
  \and
  N_{2} := \Sigma_{i'} y_{i'}.B_{i'} + \Sigma_{j'} y_{j'}.D_{j'} 
\end{mathpar}

as follows.

\begin{mathpar}
  \Sigma_{i} x_{i}.A_{i} + \Sigma_{j} x_{j}.C_{j} \otimes \Sigma_{i'}
  y_{i'}.B_{i'} + \Sigma_{j'} y_{j'}.D_{j'} 
  \and \\
  := \; \Sigma_{i} \Sigma_{i'} \quotep{\stackrel{\vee}{x_{i}}| \stackrel{\vee}{y_{i'}}}.(A_{i}\otimes B_{i'}) \; | \; \Sigma_{i'} \Sigma_{i} \quotep{\stackrel{\vee}{y_{i'}}|\stackrel{\vee}{x_{i}}}.(B_{i'}\otimes A_{i})
  \and
  \;\; | \;\; \Sigma_{j} \Sigma_{j'} \quotep{\stackrel{\vee}{x_{j}}|\stackrel{\vee}{y_{j'}}}.(A_{j}\otimes B_{j'}) \; | \; \Sigma_{j'} \Sigma_{j} \quotep{\stackrel{\vee}{y_{j'}}|\stackrel{\vee}{x_{j}}}.(B_{j'}\otimes A_{j})
\end{mathpar}

\begin{remark}
  Do we need to $x^{L}$ and $y^{R}$ for this construction as well?
\end{remark}

\paragraph{Tensor of parallel compositions} Next, we distribute tensor
over par.

\begin{mathpar}
  P_{1}|P_{2} \otimes Q_{1}|Q_{2} := (P_{1} \otimes Q_{1}) | (P_{1}
  \otimes Q_{2}) | (P_{2} \otimes Q_{1}) | (P_{2} \otimes Q_{2})
\end{mathpar}

\paragraph{Tensor with dropped names} We treat tensor of a
process with a dropped name as parallel composition.

\begin{mathpar}
  P \otimes \dropn{x} := P | \dropn{x}
\end{mathpar}

\paragraph{Tensor of agents}

Finally, we need to define tensor on agents. Note that the definition
of tensor on normal products only tensors inputs with inputs and
outputs with outputs. Thus, we only have to define the operation on
``homogeneous'' pairings.

\begin{mathpar}
  (\vec{x})P \otimes (\vec{y})Q
  \and \\
  := (x_{0}^{L}|y_{0}^{R},\ldots,x_{0}^{L}|y_{n}^{R},\ldots,x_{m}^{L}|y_{0}^{R},\ldots,x_{m}^{L}|y_{n}^R)(P\{ \vec{x}^{L}/\vec{x}\} \otimes Q \{ \vec{y}^{R}/\vec{y}\})
  \and \\
  \clift{\vec{P}} \otimes \clift{\vec{Q}}
  \and \\
  := \clift{P_{0}\otimes Q_{0},\ldots,P_{0}\otimes Q_{n},\ldots,P_{m}\otimes Q_{0},\ldots,P_{m}\otimes Q_{n}}
\end{mathpar}

\begin{remark}
  Observe that arities of tensored abstractions matches arities of
  tensored concretions if the original arities matched. Note also that
  the length of the arities corresponds to the increase in dimension
  we see in ordinary vector space tensor product.
\end{remark}

\begin{remark}
  Operationally, this definition distributes the tensor down to
  components ``linked'' by summation. Tensor over summation is
  intriguing in that it mixes names. Moreover, as a consequence of the
  way it mixes names we have the identities for all $x \in \QProc$ and
  $P,Q \in \Proc$

  \begin{mathpar}
    (x \cdot P) \otimes Q \equiv x \cdot (P \otimes Q) \equiv P \otimes (x \cdot Q)
    \and
    P \otimes \pzero \equiv P
  \end{mathpar}

  that the reader is invited to verify.
\end{remark}

\subsubsection{Annihilation}
\begin{mathpar}
  P^{\perp} := \{ Q | \forall R. P|Q \red^{*} R \Rightarrow R \red^{*} \pzero \}
  \and \\
  P^{\underline{\perp}} := \Sigma_{Q \in P^{\perp}} \quotep{Q}?(y).(\dropn{y}|Q) | \Sigma_{Q \in P^{\perp}} \quotep{Q}\clift{\Box}
\end{mathpar}

\paragraph{Discussion} The reader will note that $P^{\perp}$ is a
\emph{set} of processes, while $P^{\underline{\perp}}$ is a
\emph{context}. We call the set $P^{\perp}$ the \emph{annihilators} of
$P$. The parallel composition of a process in the annihilators of $P$
with $P$ will result in a process, the state space of which has all
paths eventually leading to $\pzero$. Execution may endure loops; but
under reasonable conditions of fairness (naturally guaranteed under
most notions of bisimulation) such a composite process cannot get
stuck in such a loop and will, eventually pop out and terminate.

The context $P^{\underline{\perp}}$ is ready and willing to ``take the
$P$ out of'' the process to which it is applied. It will effectively
transmit the code of the process to which it is applied to one of the
annihilators and run the process against it.

\subsubsection{Evaluation}
We fix $M$ a domain of fully abstract interpretation with an equality
coincident with bisimulation. We take $\meaningof{\cdot} : \Proc \to
M$ to be the map interpreting processes and $\nmeaningof{\cdot} : \M
\to Proc$ to be the map running the other way. Then we define

\begin{mathpar}
  \int P := \nmeaningof{\meaningof{P}}
\end{mathpar}

\paragraph{Discussion}
There are many fully abstract interpretations of Milner's
$\pi$-calculus. Any of them can be used as a basis for interpreting
the reflective calculus here. Equipped with such a domain it is
largely a matter of grinding through to check that the Yoneda
construction for the normalization-by-evaluation program can be
extended to this setting.

\begin{remark}
  The reader is invited to verify that $\int (P^{\underline{\perp}}[P]) = 0$.
\end{remark}

\subsection{Quantum mechanics}

Table \ref{tbl:core_qm_op_defns} gives the core operational definitions

\begin{table}[htp]\label{tbl:core_qm_op_defns}
  \center{
    \fbox{
      \begin{tabular}{c|c}
        quantum mechanics & process calculus \\
        \hline
        scalar & $x := \quotep{P}$ \\
        state vector & $\state{P} := P$ \\
        dual & $\state{P}^{*} := \event{P^{\underline{\perp}}} := \quotep{P^{\underline{\perp}}}[-]$ \\
        matrix & $ \Sigma_{\alpha} \state{P_{\alpha}}x_{\alpha}\event{Q_{\alpha}}$ \\
        vector addition & $\state{P} + \state{Q} := \state{P | Q}$ \\
        tensor product & $\state{P} \otimes \state{Q} := \state{P \otimes Q}$ \\
        inner product & $\innerprod{P}{Q} := \quotep{\int P^{\underline{\perp}}[Q]}$ \\
      \end{tabular}
    }
  }
  \caption{QM - operational definitions}
\end{table}

where

\begin{mathpar}
  \prmatrix{P}{Q} := \fprmatrix{P}{\quotep{\pzero}}{Q}
  \and
  \fprmatrix{P}{x}{Q} := (\state{P},x,\event{Q})
  \and
  (\fprmatrix{P}{x}{Q})(\state{R}) := x \cdot \innerprod{Q}{R} \cdot \state{P}
  \and
  (\fprmatrix{P}{x}{Q})(\event{R}) := x \cdot \innerprod{R}{P} \cdot \event{Q}
\end{mathpar}

\paragraph{Discussion}
As promised: vectors (aka states) are represented as processes; duals
as contextual duals; inner product definition should be compared with
standard inner product definition for ....

\begin{remark}
  Assuming $\int (P^{\underline{\perp}}[P]) = 0$, the reader is
  invited to verify that $(\fprmatrix{P}{x}{P})(\state{P}) = x \cdot \state{P}$.
\end{remark}

\begin{remark}
  The reader is invited to verify that $\innerprod{P}{Q}$ could
  equally well have been written $\quotep{\int \stackrel{\vee}{x}}$
  where $x = \event{P^{\underline{\perp}}}(Q)$.

  One of the motivations for this remark is that there is another way
  to factor these operations. We could package up evaluation in the dual:

  \begin{mathpar}
    \state{P}^{*} := \event{\int P^{\underline{\perp}}} := \quotep{\int P^{\underline{\perp}}}[-]
  \end{mathpar}

  and then have inner product defined by
  
  \begin{mathpar}
    \innerprod{P}{Q} := \event{P}(Q)
  \end{mathpar}

  Hopefully, experience with the calculations will provide guidance on
  the best factoring.
\end{remark}

\begin{remark}
  Assuming $\int (P^{\underline{\perp}}[P]) = 0$, the reader is
  invited to verify that $\forall P,Q. (\prmatrix{0}{Q})(\state{0}) =
  \state{0}$ and dually $(\prmatrix{P}{0})(\event{0}) = \event{0}$.
\end{remark}

\begin{remark}
  i'm a little worried that i don't (yet) have proper support for
  complex conjugacy. But, the observation above may give us a
  clue. According to Abramsky, it must be the case that the scalars
  are iso to the homset of the identity for the tensor -- which the
  observation above characterizes. 

  For now, we will simply bookmark the notion with $\overline{x}$.
\end{remark}

\subsubsection{Adjointness}

We need to give a definition of $(\cdot)^{\dagger}$ for matrices. The
obvious candidate definition is
\begin{mathpar}
(\Sigma_{\alpha}\fprmatrix{P_{\alpha}}{x_{\alpha}}{Q_{\alpha}})^{\dagger}
= \Sigma_{\alpha}\fprmatrix{(Q_{\alpha}^{\underline{\perp}})^{*}}{\overline{x}_{\alpha}}{P_{\alpha}^{\underline{\perp}}} 
\end{mathpar}

But, $(Q_{\alpha}^{\underline{\perp}})^{*}$ requires a name along
which to communicate the process to achieve the context application.

\subsubsection{Basis for a basis}
If processes label states and ``addition'' of states (a.k.a. vector
addition) is interpreted as parallel composition, what corresponds to
notions of linear independence and basis? Here, we recall that Yoshida
has developed a set of \emph{combinators} for an asynchronous verison
of Milner's $\pi$-calculus. These are a finite set of processes such
any process can be expressed as parallel composition of these
combinators together with liberal uses of the new operator and
replication. We can simply give a translation of these into the
present calculus and have reasonable expectation that the property
carries over. That is, that the resultant set allows to express all
processes via parallel composition. Note, however, that there is no
new operator or replication in this calculus. As a result, we expect
that the corresponding set is actually infinite. That is, we expect
that the space is actually infinite dimensional.

\begin{remark}
  The attentive reader may be a bit concerned. Certainly, the
  collection $S$, $K$ and $I$ is a finite set of
  combinators. Shouldn't we expect to see a finite set of combinators
  for an effectively equivalent system? i am very sympathetic to this
  critique and feel it warrants full attention. On the other hand, i
  also have in mind the following analogy. The natural numbers, as a
  monoid under addition, has exactly $1$ generator, while the natural
  numbers, as a monoid under multiplication, has countably many
  generators (the primes). We observe that the application of the
  lambda calculus is much less resource sensitive than the parallel
  composition of the $\pi$-calculus. Could it be the case that we have
  an analogy of the form
  
  \begin{mathpar}
    m + n : MN :: m*n : M|N
  \end{mathpar}

  giving a similar blow up in the set of ``primes''?  This is such a
  wonderful thought that, even if it's not true, i think it's worth
  writing down.
\end{remark}
 

\documentclass[12pt]{llncs}
%\documentclass{jktr}

\usepackage[pdftex]{hyperref}                   
\usepackage {listings}
\usepackage {mathpartir}
\usepackage{bcprules}
%\usepackage{listings}
                       
\usepackage{graphicx} 
%\usepackage[margins=2.5cm,nohead,nofoot]{geometry}
%\usepackage{geometry}
\usepackage{amsfonts}
\usepackage{amstext}
\usepackage{latexsym}
\usepackage{amssymb}
\usepackage{color}


%\include{myPreamble}
\include{qm2pi.local} 

%\ifpdf
%\usepackage[pdftex]{graphicx}
%\else
%\usepackage{graphicx}
%\fi

 % \ifpdf
%  \usepackage{pdfsync}
%  \if


%\title{Brief Article}
%\author{David F. Snyder}
%\author{L.G. Meredith}

%\address{Dept. of Math., Texas State University--San Marcos, San Marcos, TX 78666}
       
\pagestyle{empty}


\begin{document}

\lstset{language=[Objective]Caml,frame=shadowbox}

\input{qm2pi.front}

% section front matter (end)

\input{qm2pi.intro} 
 
% section introduction (end)

% \input{qm2pi.knotations} 

% section notation (end)

\input{qm2pi.process.calculi} 

% section concurrent_process_calculi_and_spatial_logics_ (end)
    
%\input{qm2pi.knots2pi} 

%\input{qm2pi.trefoil} 

%\input{qm2pi.mainthm} 

% subsection basic_interpretation (end)

%\input{qm2pi.rho.presentation} 
\subsection{The syntax and semantics of the notation system}\label{sub:the_syntax_and_semantics_of_the_notation_system} % (fold)

We now summarize a technical presentation of the calculus that
embodies our theory of dynamics. The typical presentation of such a
calculus follows the style of giving generators and relations on
them. The grammar, below, describing term constructors, freely
generates the set of processes, $\Proc$. This set is then quotiented
by a relation known as structural congruence and it is over this set
that the notion of dynamics is expressed. This presentation is
essentially that of \cite{MeredithR05} with the addition of
polyadicity and summation. For readability we have relegated some of
the technical subtleties to an appendix.

\subsubsection{Process grammar}\label{subsub:process_grammar}

\begin{mathpar}
  \inferrule* [lab=synchronization] {} {{M} \bc \pzero \;|\; x?F \;|\; x!C }
  \and
  \inferrule* [lab=abstraction] {} {{F} \bc (x)P}
  \and
  \inferrule* [lab=concretion] {} {{C} \bc \langle Q \rangle}
  \and
  \inferrule* [lab=process] {} {{P,Q} \bc M \;| \;P|Q \;|\; @{x}}
  \and
  \inferrule* [lab=name] {} {{x} \bc \quotep{P}}
\end{mathpar} 

Note that $\vec{x}$ (resp. $\vec{P}$) denotes a vector of names
(resp. processes) of length $|\vec{x}|$ (resp. $|\vec{P}|$). We adopt
the following useful abbreviations.

\begin{mathpar}
   x?(\vec{y}).P := x.(\vec{y})P \and  x\clift{\vec{P}} := x.\clift{\vec{P}}
   \and x!(y) := \lift{x}{\dropn{y}}
   \and \Pi_{i=0}^{n-1}P_i := P_0 | \ldots | P_{n-1}
\end{mathpar}

\subsubsection{Structural congruence}

\paragraph{Free and bound names and alpha-equivalence.} At the
core of structural equivalence is alpha-equivalence which identifies
process that are the same up to a change of variable. Formally, we
recognize the distinction between free and bound names. The free names
of a process, $\freenames{P}$, may be calculated recursively as
follows:

\begin{mathpar}
\freenames{\pzero} := \emptyset
  \and \\
  \freenames{x?(y).P} := \{ x \} \cup (\freenames{P} \setminus \{ y \})
  \and 
  \freenames{x!\langle P \rangle} := \{ x \} \cup \{ P \} 
  \and \\
  \freenames{P|Q} := \freenames{P} \cup \freenames{Q}
  \and \\
  \freenames{@{x}} := \{ x \}
\end{mathpar}

$\pi$
$\quotep{\pi}$

$\freenames{-} : \pi \to \mathcal{P}(\quotep{\pi})$

\begin{eqnarray*}
  \freenames{\pzero} & := & \emptyset \\
  \freenames{x?(y).P} & := & \{ x \} \cup (\freenames{P} \setminus \{ y \}) \\
  \freenames{x!\langle P \rangle} & := & \{ x \} \cup \{ P \} \\
  \freenames{P|Q} & := & \freenames{P} \cup \freenames{Q} \\
  \freenames{\dropn{x}} & := & \{ x \}
\end{eqnarray*}

The bound names of a process, $\boundnames{P}$, are those names occurring in $P$
that are not free. For example, in $x?(y).0$, the name $x$ is free, while $y$ is bound.

\begin{mathpar}
  \inferrule* [lab=monoidal-laws] {} { P|Q \equiv Q|P \and P|0 \equiv P \and P|(Q|R) \equiv (P|Q)|R }
\end{mathpar}

\begin{mathpar}
  \inferrule* [lab=alpha-equivalence] {} { (x)P \equiv (y)P\{y/x\} \and y \not\in \freenames{P} }
\end{mathpar}

\begin{definition}
Then two processes, $P,Q$, are alpha-equivalent if $P = Q\{\vec{y}/\vec{x}\}$ for
some $\vec{x} \in \boundnames{Q},\vec{y} \in \boundnames{P}$, where $Q\{\vec{y}/\vec{x}\}$
denotes the capture-avoiding substitution of $\vec{y}$ for $\vec{x}$ in $Q$.
\end{definition}

\begin{definition}
  The {\em structural congruence} \cite{SangiorgiWalker} , $\equiv$,
  between processes is the least congruence containing
  alpha-equivalence, satisfying the abelian monoid laws
  (associativity, commutativity and $\pzero$ as identity) for parallel
  composition $|$ and for summation $+$.
\end{definition}

\subsection{Name equivalence}

We take name equivalence, written $\nameeq$, to be the smallest
equivalence relation generated by the following rules.

\begin{mathpar}
\inferrule*[lab=Quote-drop]
{ }
{ \quotep{@{x}} \nameeq x }

\inferrule*[lab=Struct-equiv]
{ P \scong Q }
{ \quotep{P} \nameeq \quotep{Q} }
\end{mathpar}

The astute reader will have noticed that the mutual recursion of names
and processes imposes a mutual recursion on alpha-equivalence and
structural equivalence via name-equivalence. Fortunately, all of this
works out pleasantly and we may calculate in the natural way, free of
concern. The reader interested in the details is referred to the
appendix \ref{appendix:rho_details}.

\subsection{Substitution}

We use $\Proc$ for the set of processes, $\QProc$ for the set of
names, and $\id{\{}\vec{y} / \vec{x} \id{\}}$ to denote partial maps,
$s : \QProc \rightarrow \QProc$. A map, $s$ lifts, uniquely, to a map
on process terms, $\widehat{s} : \Proc \rightarrow \Proc$ by the
following equations.

\begin{mathpar}
  (0) \psubstp{Q}{P} := 0 \\
  (R \juxtap S) \psubstp{Q}{P}
  :=    
  (R)\psubstp{Q}{P} \juxtap (S) \psubstp{Q}{P} \\
  (x?(y).R) \psubstp{Q}{P}    
  :=    
  (x)\substp{Q}{P} (z)\concat( (R \psubstn{z}{y}) \psubstp{Q}{P} ) \\
  (\lift{x}{R}) \psubstp{Q}{P}  
  :=
  \lift{(x)\substp{Q}{P}}{ R \psubstp{Q}{P} } \\
%   (\dropn{x})  \psubstp{Q}{P}       
%   := 
%   \left\{ 
%     \begin{array}{ccc} 
%       \dropn{\quotep{Q}} & & x \nameeq \quotep{P} \\
%       \dropn{x} & & otherwise \\
%     \end{array}
%   \right. 
  (\dropn{x})  \psubstp{Q}{P}       
  := 
  \left\{ 
    \begin{array}{ccc} 
      Q & & x \nameeq \quotep{P} \\
      \dropn{x} & & otherwise \\
    \end{array}
  \right.
\end{mathpar}
 

where

\begin{eqnarray}
  (x)\id{\{} \lpquote Q \rpquote / \lpquote P \rpquote \id{\}}            = 
  \left\{ 
    \begin{array}{ccc}
      \lpquote Q \rpquote & & x \nameeq \lpquote P \rpquote \\
      x & & otherwise \\
    \end{array}
  \right. \nonumber
\end{eqnarray}

and $z$ is chosen distinct from $\quotep{P}$, $\quotep{Q}$, the free
names in $Q$, and all the names in $R$. Our $\alpha$-equivalence will
be built in the standard way from this substitution.

\begin{remark}\label{rem:no_self_referential_names}
  One consequence of these definitions is that $\forall P. \quotep{P}
  \not\in \freenames{P}$.
\end{remark}

\subsection{ Dynamic quote: an example }

Anticipating something of what's to come, consider applying the
substitution, $\widehat{\id{\{}u / z \id{\}}}$, to the following pair
of processes, $\lift{w}{y!(z)}$ and $w[ \lpquote y!(z) \rpquote ]$.

\begin{eqnarray}
	\lift{w}{y!(z)}\widehat{\id{\{}u / z \id{\}}}
		& = &
		\lift{w}{y!(u)} \nonumber\\
	w[ \lpquote y!(z) \rpquote ] \widehat{ \id{\{}u / z \id{\}} }
		& = &
		w[ \lpquote y!(z) \rpquote ] \nonumber
\end{eqnarray}

Because the body of the process between quotes is impervious to
substitution, we get radically different answers. In fact, by
examining the first process in an input context,
e.g. $x?(z).\lift{w}{y!(z)}$, we see that the process under the lift
operator may be shaped by prefixed inputs binding a name inside it. In
this sense, the lift operator will be seen as a way to dynamically
construct processes before reifying them as names.

Finally equipped with these standard features we can present the
dynamics of the calculus.

\subsubsection{Operational semantics} 

Finally, we introduce the computational dynamics. What marks these
algebras as distinct from other more traditionally studied algebraic
structures, e.g. vector spaces or polynomial rings, is the manner in
which dynamics is captured. In traditional structures, dynamics is typically
expressed through morphisms between such structures, as in linear maps
between vector spaces or morphisms between rings. In algebras
associated with the semantics of computation, the dynamics is
expressed as part of the algebraic structure itself, through a
reduction reduction relation typically denoted by $\red$. Below, we
give a recursive presentation of this relation for the calculus used
in the encoding.

$\red \subseteq \pi \times \pi$
$\red : \pi \to \mathcal{P}(\pi)$

\begin{mathpar}
  \inferrule* [lab=Comm] { \textsf{match}( x_{src}, x_{trgt} ) } { x_{trgt}?(y)P \; | \; x_{src}!\langle {Q} \rangle \red P\{\quotep{Q}/y}\} }
  \and \\
  \inferrule* [lab=Par] {{P} \red {P}'} {{{P} | {Q}} \red {{P}' | {Q}}}
  \and
  \inferrule* [lab=Equiv]{{{P} \scong {P}'} \andalso {{P}' \red {Q}'} \andalso {{Q}' \scong {Q}}}{{P} \red {Q}}
\end{mathpar}

\begin{eqnarray*}
  match_{\equiv} (\quotep{P},\quotep{Q}) & := & P \equiv Q \\
  match_{\dagger}(\quotep{P},\quotep{Q}) & := & \forall R. P|Q \red^{*} R => R \red^{*} 0 \\
  match_{K}(\quotep{P},\quotep{Q}) & := & K \mbox{ for some context } K
\end{eqnarray*}

$u?(x)P | u!\langle Q \rangle \red P\{\quotep{Q}/x\}$

%We write $\wred$ for $\red^*$, and $P\red$ if $\exists Q $ such that $ P \red Q$.
We write $P\red$ if $\exists Q $ such that $ P \red Q$ and $P\not\red$, otherwise.

\section{Replication}

As mentioned before, it is known that replication (and hence
recursion) can be implemented in a higher-order process algebra
\cite{SangiorgiWalker}. As our first example of calculation with the
machinery thus far presented we give the construction explicitly in
the {\rhoc}.

\begin{eqnarray}
	D_{x} & := & \prefix{x}{y}{(\binpar{\outputp{x}{y}}{@{y}})} \nonumber\\
	\bangp_{x}{P} & := & \binpar{{x}!\langle{\binpar{D_{x}}{P}}\rangle}{D_{x}} \nonumber
\end{eqnarray}

\begin{eqnarray}
	\bangp_{x}{P} & & \nonumber\\
	=
	& {x}!\langle{(\prefix{x}{y}{(\outputp{x}{y} | @{y})) | P}}\rangle 
	      | \prefix{x}{y}{(\outputp{x}{y} | @{y})} & \nonumber\\
	\red
	& (\outputp{x}{y} | @{y})\substn{\quotep{(\prefix{x}{y}{(@{y} | \outputp{x}{y})) | P}}}{y} & \nonumber\\
	=
	& \outputp{x}{\quotep{(\prefix{x}{y}{(\outputp{x}{y} | @{y})) | P}}}
	  | {(\prefix{x}{y}{(\outputp{x}{y} | @{y})) | P}} & \nonumber\\
	\red
	& \ldots & \nonumber\\
	\red^*
	& P | P | \ldots & \nonumber
\end{eqnarray}

Of course, this encoding, as an implementation, runs away, unfolding
$\bangp{P}$ eagerly. A lazier and more implementable replication
operator, restricted to input-guarded processes, may be obtained as follows.

\begin{eqnarray}
\bangp{\prefix{u}{v}{P}} 
	:= 
	\binpar{\lift{x}{\prefix{u}{v}{(\binpar{D(x)}{P})}}}{D(x)} \nonumber
\end{eqnarray}

\begin{remark}
  Note that the lazier definition still does not deal with summation
  or mixed summation (i.e. sums over input and output). The reader is
  invited to construct definitions of replication that deal with these
  features. 

  Further, the definitions are parameterized in a name, $x$. Can you,
  gentle reader, make a definition that eliminates this parameter and
  guarantees no accidental interaction between the replication
  machinery and the process being replicated -- i.e. no accidental
  sharing of names used by the process to get its work done and the
  name(s) used by the replication to effect copying. This latter
  revision of the definition of replication is crucial to obtaining
  the expected identity $!!P \sim !P$.
\end{remark}

\begin{remark}\label{rem:paradoxical_combinator}
  The reader familiar with the lambda calculus will have noticed the
  similarity between $D$ and the paradoxical combinator.

  [Ed. note: the existence of this seems to suggest we have to be more
  restrictive on the set of processes and names we admit if we are to
  support no-cloning.]
\end{remark}

\subsubsection{Bisimulation}

The computational dynamics gives rise to another kind of equivalence,
the equivalence of computational behavior. As previously mentioned
this is typically captured \emph{via} some form of bisimulation.

% The notion we use in this paper is weak barbed bisimulation
% \cite{milner91polyadicpi}.

The notion we use in this paper is derived from weak barbed
bisimulation \cite{milner91polyadicpi}. 

\begin{definition}
An \emph{observation relation}, $\downarrow_{\mathcal N}$, over a set
of names, $\mathcal N$, is the smallest relation satisfying the rules
below.

\infrule[Out-barb]{y \in {\mathcal N}, \; x \nameeq y}
		  {\outputp{x}{v} \downarrow_{\mathcal N} x}
\infrule[Par-barb]{\mbox{$P\downarrow_{\mathcal N} x$ or $Q\downarrow_{\mathcal N} x$}}
		  {\binpar{P}{Q} \downarrow_{\mathcal N} x}

We write $P \Downarrow_{\mathcal N} x$ if there is $Q$ such that 
$P \wred Q$ and $Q \downarrow_{\mathcal N} x$.
\end{definition}

\begin{definition}
%\label{def.bbisim}
An  ${\mathcal N}$-\emph{barbed bisimulation} over a set of names, ${\mathcal N}$, is a symmetric binary relation 
${\mathcal S}_{\mathcal N}$ between agents such that $P\rel{S}_{\mathcal N}Q$ implies:
\begin{enumerate}
\item If $P \red P'$ then $Q \wred Q'$ and $P'\rel{S}_{\mathcal N} Q'$.
\item If $P\downarrow_{\mathcal N} x$, then $Q\Downarrow_{\mathcal N} x$.
\end{enumerate}
$P$ is ${\mathcal N}$-barbed bisimilar to $Q$, written
$P \wbbisim_{\mathcal N} Q$, if $P \rel{S}_{\mathcal N} Q$ for some ${\mathcal N}$-barbed bisimulation ${\mathcal S}_{\mathcal N}$.
\end{definition}

$\mathcal{R} \subseteq \pi \times \pi$

$P \mathcal{R} Q => \forall P'. P \red P' \Rightarrow \exists Q'. Q \red Q', P' \mathcal{R} Q'$

$P \vdash x \Rightarrow Q \vdash x$

\begin{mathpar}
  \inferrule*[lab=Out-barb]{x \nameeq y}{{y}!\langle{Q}\rangle \vdash x}
  \and
  \inferrule*[lab=Par-barb]{\mbox{$P\vdash x$ or $Q\vdash x$}}{\binpar{P}{Q} \vdash x}
\end{mathpar}

\subsubsection{Contexts}

One of the principle advantages of computational calculi like the
$\pi$-calculus is a well-defined notion of context,
contextual-equivalence and a correlation between
contextual-equivalence and notions of bisimulation. The notion of
context allows the decomposition of a process into (sub-)process and
its syntactic environment, its context. Thus, a context may be
thought of as a process with a ``hole'' (written $\Box$) in it. The
application of a context $M$ to a process $P$, written $M[P]$, is
tantamount to filling the hole in $M$ with $P$. In this paper we do
not need the full weight of this theory, but do make use of the notion
of context in the proof the main theorem. 

\begin{mathpar}
  \inferrule* [lab=summation] {} {{M_{M},M_{N}} \bc \Box \;|\; x.M_{A} \;|\; M_{M}+M_{N}}
  \and
  \inferrule* [lab=agent] {} {{M_{A}} \bc (\vec{x})M_{P} \;| \; \clift{P_0,\ldots,M_{P},\ldots,P_N}}
  \and \\
  \inferrule* [lab=process] {} {{M_{P}} \bc M_{N} \;| \;P|M_{P} }
\end{mathpar} 

\begin{mathpar}
  \inferrule* [lab=sychronization] {} {M_{N} \bc \Box \;|\; x?M_{F} \;|\; x!M_{C}}
  \and
  \inferrule* [lab=abstraction] {} {{M_{F}} \bc (x)M_{P} }
  \and
  \inferrule* [lab=concretion] {} {{M_{C}} \bc \langle M_{P} \rangle }
  \and \\
  \inferrule* [lab=process] {} {{M_{P}} \bc M_{N} \;| \;P|M_{P} }
\end{mathpar}

\begin{definition}[contextual application] Given a context $M$, and
  process $P$, we define the \emph{contextual application}, $M[P] :=
  M\{P/\Box\}$. That is, the contextual application of M to P is the
  substitution of $P$ for $\Box$ in $M$.
\end{definition}

$\meaningof{-} : L \to \mathcal{P}(\pi)$

\begin{mathpar}
  \inferrule* [lab=collection] {} {\meaningof{true} = \pi, \and \meaningof{~E} = \pi \setminus \meaningof{E}, \and \meaningof{E_{1} \& E_{2}} = \meaningof{E_{1}} \cap \meaningof{E_{2}}}
\end{mathpar}

\begin{mathpar}
  \inferrule* [lab=structure] {} {\meaningof{0} = \{ P \in \pi | P \equiv 0 \}, \and \\ \meaningof{E_1 | E_2} = \{ P \in \pi | P \equiv P_{1} | P_{2}, P_{1} \in \meaningof{E_{1}}, P_{2} \in \meaningof{E_2}\} }
\end{mathpar}

\begin{mathpar}
 \inferrule* [lab=behavior] {} {\meaningof{\langle a?b \rangle E} = \{ P \in \pi | P \equiv Q | u?(y)P', \\ \and \\\\ \and \\ \;\;\; u \in \meaningof{a}, \forall z.P'\{z/y\} \in \meaningof{E\{z/b\}}\}, \and \\ \meaningof{a!E} = \{ P \in \pi | P \equiv Q | x!\langle P' \rangle, x \in \meaningof{a} P' \in \meaningof{E}\} }
\end{mathpar}

\begin{mathpar}
 \inferrule* [lab=nominal] {} {\meaningof{\quotep{E}} = \{ \quotep{P} \in \quotep{\pi} | P \in \meaningof{E} \}, \and \meaningof{\quotep{P}} = \{ \quotep{Q} \in \quotep{\pi} | P \equiv Q \} \and \\ \meaningof{@\quotep{E}} = \{ P \in \pi | P \equiv @x, x \in \meaningof{E} \}}
\end{mathpar}

\begin{eqnarray*}
  \\
  \meaningof{-} : TS \to ST
\end{eqnarray*}

\begin{eqnarray*}
  \\
  L : TS \to ST
\end{eqnarray*}

\begin{eqnarray*}
  \\
  P \models E \iff P \in \meaningof{E}
\end{eqnarray*}

\begin{eqnarray*}
  P \approx_{L} Q \iff \forall E \in L. P \models E \iff Q \models E
\end{eqnarray*}

\begin{eqnarray*}
  P \approx_{K} Q
\end{eqnarray*}

\begin{eqnarray*}
  P \approx Q
\end{eqnarray*}

$\approx_{K} = \approx = \approx_{L}$

\subsubsection{Contextual duality}

Note that contexts extend the quotation operation to a family of
operations from processes to names. Given a context, $M$, we can
define a \emph{nominal context}, $\quotep{M}$ by $\quotep{M}[P] :=
\quotep{M[P]}$. To foreshadow what is to come we observe that these
operations enjoy a duality with processes very much like the duality
between vectors and maps from vectors to scalars.

Further, because the calculus is essentially higher-order, we have a
correspondence between contexts and processes. More specifically,
given a name $x$ and a context $M$ we can construct $M^{*}_{x}$ such
that 

\begin{mathpar}
  M^{*}_{x} | \lift{x}{P} \red M[P]
\end{mathpar}

namely,

\begin{mathpar}
  M^{*}_{x} := x?(u).M[\dropn{u}]
\end{mathpar}

The dependence of $M^{*}_{x}$ on a name makes it an abstraction, 

\begin{mathpar}
  M^{*} := (x)x?(u).M[\dropn{u}]
\end{mathpar}

\subsection{Additional notation}

It will sometimes be convenient to denote the process a name
quotes. We already have the notation $x = \quotep{P}$, but it will be
convenient to introduce an alternate notation, $\procn{x}$, when we
want to emphasize the connection to the use of the name. Note that, by
virtue of name equivalence, $\quotep{\procn{x}} \nameeq x$; so, the
notation is consistent with previous definitions.

Further, because names have structure it is possible to effect
substitutions on the basis of that structure. This means we need to
upgrade our notation for substitutions, which we accomplish by
adapting comprehension notation. Thus,

\begin{mathpar}
  P\{ y / x : x \in S \}
\end{mathpar}

is interpreted to mean the process derived from P by replacing (in a
capture-avoiding manner) each occurrence of $x$ in $S$ by $y$. For example,

\begin{mathpar}
  P\{ \quotep{\procn{x}|\procn{x}} / x : x \in \freenames{P} \}
\end{mathpar}

will replace each (occurrence) of a free name $x$ in $P$ by
$\quotep{\procn{x}|\procn{x}}$.

Also, we will avail ourselves of the notation $x^{L}$ and $x^{R}$ to
denote injections of a name into disjoint copies of the name
space. There are numerous ways to accomplish this. One example can be
found in \cite{MeredithR05}. This notation overloads to vectors of
names: $\vec{x}^{\pi} := (x_{i}^{\pi} \; : \; 0 \leq i < |\vec{x}| )$ where $\pi \in \{L,R\}$.

We also use $P^{\Box} := P|\Box$.

In \cite{MeredithR05} an interpretation of the new operator is
given. It turns out that there are several possible interpretations
all enjoying the requisite algebraic properties of the operator (see
\cite{milner91polyadicpi}). We will therefore make liberal use of
$(\nu\; \vec{x})P$.

% subsection the_syntax_and_semantics_of_the_notation_system (end)   

\input{qm2pi.qmops} 

\input{qm2pi.sterngerlach} 

\input{qm2pi.metric} 

% section concurrent_process_calculi (end)

%\input{qm2pi.proofsketch}

% section proof sketch (end)

%\input{qm2pi.slviaknots} 

% section spatial logic via knots (end)

\input{qm2pi.conclusion}

% section conclusion (end)

%\input{qm2pi.dtcodes} 

% section wiring algorithm (end)

\input{qm2pi.ack} 

% section acknowledgments (end)

\newpage


\bibliographystyle{plain}   
\bibliography{../../biblios/main.bib}

\input{qm2pi.rhodetails}

\end{document}

 

\documentclass[12pt]{llncs}
%\documentclass{jktr}

\usepackage[pdftex]{hyperref}                   
\usepackage {listings}
\usepackage {mathpartir}
\usepackage{bcprules}
%\usepackage{listings}
                       
\usepackage{graphicx} 
%\usepackage[margins=2.5cm,nohead,nofoot]{geometry}
%\usepackage{geometry}
\usepackage{amsfonts}
\usepackage{amstext}
\usepackage{latexsym}
\usepackage{amssymb}
\usepackage{color}


%\include{myPreamble}
\include{qm2pi.local} 

%\ifpdf
%\usepackage[pdftex]{graphicx}
%\else
%\usepackage{graphicx}
%\fi

 % \ifpdf
%  \usepackage{pdfsync}
%  \if


%\title{Brief Article}
%\author{David F. Snyder}
%\author{L.G. Meredith}

%\address{Dept. of Math., Texas State University--San Marcos, San Marcos, TX 78666}
       
\pagestyle{empty}


\begin{document}

\lstset{language=[Objective]Caml,frame=shadowbox}

\input{qm2pi.front}

% section front matter (end)

\input{qm2pi.intro} 
 
% section introduction (end)

% \input{qm2pi.knotations} 

% section notation (end)

\input{qm2pi.process.calculi} 

% section concurrent_process_calculi_and_spatial_logics_ (end)
    
%\input{qm2pi.knots2pi} 

%\input{qm2pi.trefoil} 

%\input{qm2pi.mainthm} 

% subsection basic_interpretation (end)

%\input{qm2pi.rho.presentation} 
\subsection{The syntax and semantics of the notation system}\label{sub:the_syntax_and_semantics_of_the_notation_system} % (fold)

We now summarize a technical presentation of the calculus that
embodies our theory of dynamics. The typical presentation of such a
calculus follows the style of giving generators and relations on
them. The grammar, below, describing term constructors, freely
generates the set of processes, $\Proc$. This set is then quotiented
by a relation known as structural congruence and it is over this set
that the notion of dynamics is expressed. This presentation is
essentially that of \cite{MeredithR05} with the addition of
polyadicity and summation. For readability we have relegated some of
the technical subtleties to an appendix.

\subsubsection{Process grammar}\label{subsub:process_grammar}

\begin{mathpar}
  \inferrule* [lab=synchronization] {} {{M} \bc \pzero \;|\; x?F \;|\; x!C }
  \and
  \inferrule* [lab=abstraction] {} {{F} \bc (x)P}
  \and
  \inferrule* [lab=concretion] {} {{C} \bc \langle Q \rangle}
  \and
  \inferrule* [lab=process] {} {{P,Q} \bc M \;| \;P|Q \;|\; @{x}}
  \and
  \inferrule* [lab=name] {} {{x} \bc \quotep{P}}
\end{mathpar} 

Note that $\vec{x}$ (resp. $\vec{P}$) denotes a vector of names
(resp. processes) of length $|\vec{x}|$ (resp. $|\vec{P}|$). We adopt
the following useful abbreviations.

\begin{mathpar}
   x?(\vec{y}).P := x.(\vec{y})P \and  x\clift{\vec{P}} := x.\clift{\vec{P}}
   \and x!(y) := \lift{x}{\dropn{y}}
   \and \Pi_{i=0}^{n-1}P_i := P_0 | \ldots | P_{n-1}
\end{mathpar}

\subsubsection{Structural congruence}

\paragraph{Free and bound names and alpha-equivalence.} At the
core of structural equivalence is alpha-equivalence which identifies
process that are the same up to a change of variable. Formally, we
recognize the distinction between free and bound names. The free names
of a process, $\freenames{P}$, may be calculated recursively as
follows:

\begin{mathpar}
\freenames{\pzero} := \emptyset
  \and \\
  \freenames{x?(y).P} := \{ x \} \cup (\freenames{P} \setminus \{ y \})
  \and 
  \freenames{x!\langle P \rangle} := \{ x \} \cup \{ P \} 
  \and \\
  \freenames{P|Q} := \freenames{P} \cup \freenames{Q}
  \and \\
  \freenames{@{x}} := \{ x \}
\end{mathpar}

$\pi$
$\quotep{\pi}$

$\freenames{-} : \pi \to \mathcal{P}(\quotep{\pi})$

\begin{eqnarray*}
  \freenames{\pzero} & := & \emptyset \\
  \freenames{x?(y).P} & := & \{ x \} \cup (\freenames{P} \setminus \{ y \}) \\
  \freenames{x!\langle P \rangle} & := & \{ x \} \cup \{ P \} \\
  \freenames{P|Q} & := & \freenames{P} \cup \freenames{Q} \\
  \freenames{\dropn{x}} & := & \{ x \}
\end{eqnarray*}

The bound names of a process, $\boundnames{P}$, are those names occurring in $P$
that are not free. For example, in $x?(y).0$, the name $x$ is free, while $y$ is bound.

\begin{mathpar}
  \inferrule* [lab=monoidal-laws] {} { P|Q \equiv Q|P \and P|0 \equiv P \and P|(Q|R) \equiv (P|Q)|R }
\end{mathpar}

\begin{mathpar}
  \inferrule* [lab=alpha-equivalence] {} { (x)P \equiv (y)P\{y/x\} \and y \not\in \freenames{P} }
\end{mathpar}

\begin{definition}
Then two processes, $P,Q$, are alpha-equivalent if $P = Q\{\vec{y}/\vec{x}\}$ for
some $\vec{x} \in \boundnames{Q},\vec{y} \in \boundnames{P}$, where $Q\{\vec{y}/\vec{x}\}$
denotes the capture-avoiding substitution of $\vec{y}$ for $\vec{x}$ in $Q$.
\end{definition}

\begin{definition}
  The {\em structural congruence} \cite{SangiorgiWalker} , $\equiv$,
  between processes is the least congruence containing
  alpha-equivalence, satisfying the abelian monoid laws
  (associativity, commutativity and $\pzero$ as identity) for parallel
  composition $|$ and for summation $+$.
\end{definition}

\subsection{Name equivalence}

We take name equivalence, written $\nameeq$, to be the smallest
equivalence relation generated by the following rules.

\begin{mathpar}
\inferrule*[lab=Quote-drop]
{ }
{ \quotep{@{x}} \nameeq x }

\inferrule*[lab=Struct-equiv]
{ P \scong Q }
{ \quotep{P} \nameeq \quotep{Q} }
\end{mathpar}

The astute reader will have noticed that the mutual recursion of names
and processes imposes a mutual recursion on alpha-equivalence and
structural equivalence via name-equivalence. Fortunately, all of this
works out pleasantly and we may calculate in the natural way, free of
concern. The reader interested in the details is referred to the
appendix \ref{appendix:rho_details}.

\subsection{Substitution}

We use $\Proc$ for the set of processes, $\QProc$ for the set of
names, and $\id{\{}\vec{y} / \vec{x} \id{\}}$ to denote partial maps,
$s : \QProc \rightarrow \QProc$. A map, $s$ lifts, uniquely, to a map
on process terms, $\widehat{s} : \Proc \rightarrow \Proc$ by the
following equations.

\begin{mathpar}
  (0) \psubstp{Q}{P} := 0 \\
  (R \juxtap S) \psubstp{Q}{P}
  :=    
  (R)\psubstp{Q}{P} \juxtap (S) \psubstp{Q}{P} \\
  (x?(y).R) \psubstp{Q}{P}    
  :=    
  (x)\substp{Q}{P} (z)\concat( (R \psubstn{z}{y}) \psubstp{Q}{P} ) \\
  (\lift{x}{R}) \psubstp{Q}{P}  
  :=
  \lift{(x)\substp{Q}{P}}{ R \psubstp{Q}{P} } \\
%   (\dropn{x})  \psubstp{Q}{P}       
%   := 
%   \left\{ 
%     \begin{array}{ccc} 
%       \dropn{\quotep{Q}} & & x \nameeq \quotep{P} \\
%       \dropn{x} & & otherwise \\
%     \end{array}
%   \right. 
  (\dropn{x})  \psubstp{Q}{P}       
  := 
  \left\{ 
    \begin{array}{ccc} 
      Q & & x \nameeq \quotep{P} \\
      \dropn{x} & & otherwise \\
    \end{array}
  \right.
\end{mathpar}
 

where

\begin{eqnarray}
  (x)\id{\{} \lpquote Q \rpquote / \lpquote P \rpquote \id{\}}            = 
  \left\{ 
    \begin{array}{ccc}
      \lpquote Q \rpquote & & x \nameeq \lpquote P \rpquote \\
      x & & otherwise \\
    \end{array}
  \right. \nonumber
\end{eqnarray}

and $z$ is chosen distinct from $\quotep{P}$, $\quotep{Q}$, the free
names in $Q$, and all the names in $R$. Our $\alpha$-equivalence will
be built in the standard way from this substitution.

\begin{remark}\label{rem:no_self_referential_names}
  One consequence of these definitions is that $\forall P. \quotep{P}
  \not\in \freenames{P}$.
\end{remark}

\subsection{ Dynamic quote: an example }

Anticipating something of what's to come, consider applying the
substitution, $\widehat{\id{\{}u / z \id{\}}}$, to the following pair
of processes, $\lift{w}{y!(z)}$ and $w[ \lpquote y!(z) \rpquote ]$.

\begin{eqnarray}
	\lift{w}{y!(z)}\widehat{\id{\{}u / z \id{\}}}
		& = &
		\lift{w}{y!(u)} \nonumber\\
	w[ \lpquote y!(z) \rpquote ] \widehat{ \id{\{}u / z \id{\}} }
		& = &
		w[ \lpquote y!(z) \rpquote ] \nonumber
\end{eqnarray}

Because the body of the process between quotes is impervious to
substitution, we get radically different answers. In fact, by
examining the first process in an input context,
e.g. $x?(z).\lift{w}{y!(z)}$, we see that the process under the lift
operator may be shaped by prefixed inputs binding a name inside it. In
this sense, the lift operator will be seen as a way to dynamically
construct processes before reifying them as names.

Finally equipped with these standard features we can present the
dynamics of the calculus.

\subsubsection{Operational semantics} 

Finally, we introduce the computational dynamics. What marks these
algebras as distinct from other more traditionally studied algebraic
structures, e.g. vector spaces or polynomial rings, is the manner in
which dynamics is captured. In traditional structures, dynamics is typically
expressed through morphisms between such structures, as in linear maps
between vector spaces or morphisms between rings. In algebras
associated with the semantics of computation, the dynamics is
expressed as part of the algebraic structure itself, through a
reduction reduction relation typically denoted by $\red$. Below, we
give a recursive presentation of this relation for the calculus used
in the encoding.

$\red \subseteq \pi \times \pi$
$\red : \pi \to \mathcal{P}(\pi)$

\begin{mathpar}
  \inferrule* [lab=Comm] { \textsf{match}( x_{src}, x_{trgt} ) } { x_{trgt}?(y)P \; | \; x_{src}!\langle {Q} \rangle \red P\{\quotep{Q}/y}\} }
  \and \\
  \inferrule* [lab=Par] {{P} \red {P}'} {{{P} | {Q}} \red {{P}' | {Q}}}
  \and
  \inferrule* [lab=Equiv]{{{P} \scong {P}'} \andalso {{P}' \red {Q}'} \andalso {{Q}' \scong {Q}}}{{P} \red {Q}}
\end{mathpar}

\begin{eqnarray*}
  match_{\equiv} (\quotep{P},\quotep{Q}) & := & P \equiv Q \\
  match_{\dagger}(\quotep{P},\quotep{Q}) & := & \forall R. P|Q \red^{*} R => R \red^{*} 0 \\
  match_{K}(\quotep{P},\quotep{Q}) & := & K \mbox{ for some context } K
\end{eqnarray*}

$u?(x)P | u!\langle Q \rangle \red P\{\quotep{Q}/x\}$

%We write $\wred$ for $\red^*$, and $P\red$ if $\exists Q $ such that $ P \red Q$.
We write $P\red$ if $\exists Q $ such that $ P \red Q$ and $P\not\red$, otherwise.

\section{Replication}

As mentioned before, it is known that replication (and hence
recursion) can be implemented in a higher-order process algebra
\cite{SangiorgiWalker}. As our first example of calculation with the
machinery thus far presented we give the construction explicitly in
the {\rhoc}.

\begin{eqnarray}
	D_{x} & := & \prefix{x}{y}{(\binpar{\outputp{x}{y}}{@{y}})} \nonumber\\
	\bangp_{x}{P} & := & \binpar{{x}!\langle{\binpar{D_{x}}{P}}\rangle}{D_{x}} \nonumber
\end{eqnarray}

\begin{eqnarray}
	\bangp_{x}{P} & & \nonumber\\
	=
	& {x}!\langle{(\prefix{x}{y}{(\outputp{x}{y} | @{y})) | P}}\rangle 
	      | \prefix{x}{y}{(\outputp{x}{y} | @{y})} & \nonumber\\
	\red
	& (\outputp{x}{y} | @{y})\substn{\quotep{(\prefix{x}{y}{(@{y} | \outputp{x}{y})) | P}}}{y} & \nonumber\\
	=
	& \outputp{x}{\quotep{(\prefix{x}{y}{(\outputp{x}{y} | @{y})) | P}}}
	  | {(\prefix{x}{y}{(\outputp{x}{y} | @{y})) | P}} & \nonumber\\
	\red
	& \ldots & \nonumber\\
	\red^*
	& P | P | \ldots & \nonumber
\end{eqnarray}

Of course, this encoding, as an implementation, runs away, unfolding
$\bangp{P}$ eagerly. A lazier and more implementable replication
operator, restricted to input-guarded processes, may be obtained as follows.

\begin{eqnarray}
\bangp{\prefix{u}{v}{P}} 
	:= 
	\binpar{\lift{x}{\prefix{u}{v}{(\binpar{D(x)}{P})}}}{D(x)} \nonumber
\end{eqnarray}

\begin{remark}
  Note that the lazier definition still does not deal with summation
  or mixed summation (i.e. sums over input and output). The reader is
  invited to construct definitions of replication that deal with these
  features. 

  Further, the definitions are parameterized in a name, $x$. Can you,
  gentle reader, make a definition that eliminates this parameter and
  guarantees no accidental interaction between the replication
  machinery and the process being replicated -- i.e. no accidental
  sharing of names used by the process to get its work done and the
  name(s) used by the replication to effect copying. This latter
  revision of the definition of replication is crucial to obtaining
  the expected identity $!!P \sim !P$.
\end{remark}

\begin{remark}\label{rem:paradoxical_combinator}
  The reader familiar with the lambda calculus will have noticed the
  similarity between $D$ and the paradoxical combinator.

  [Ed. note: the existence of this seems to suggest we have to be more
  restrictive on the set of processes and names we admit if we are to
  support no-cloning.]
\end{remark}

\subsubsection{Bisimulation}

The computational dynamics gives rise to another kind of equivalence,
the equivalence of computational behavior. As previously mentioned
this is typically captured \emph{via} some form of bisimulation.

% The notion we use in this paper is weak barbed bisimulation
% \cite{milner91polyadicpi}.

The notion we use in this paper is derived from weak barbed
bisimulation \cite{milner91polyadicpi}. 

\begin{definition}
An \emph{observation relation}, $\downarrow_{\mathcal N}$, over a set
of names, $\mathcal N$, is the smallest relation satisfying the rules
below.

\infrule[Out-barb]{y \in {\mathcal N}, \; x \nameeq y}
		  {\outputp{x}{v} \downarrow_{\mathcal N} x}
\infrule[Par-barb]{\mbox{$P\downarrow_{\mathcal N} x$ or $Q\downarrow_{\mathcal N} x$}}
		  {\binpar{P}{Q} \downarrow_{\mathcal N} x}

We write $P \Downarrow_{\mathcal N} x$ if there is $Q$ such that 
$P \wred Q$ and $Q \downarrow_{\mathcal N} x$.
\end{definition}

\begin{definition}
%\label{def.bbisim}
An  ${\mathcal N}$-\emph{barbed bisimulation} over a set of names, ${\mathcal N}$, is a symmetric binary relation 
${\mathcal S}_{\mathcal N}$ between agents such that $P\rel{S}_{\mathcal N}Q$ implies:
\begin{enumerate}
\item If $P \red P'$ then $Q \wred Q'$ and $P'\rel{S}_{\mathcal N} Q'$.
\item If $P\downarrow_{\mathcal N} x$, then $Q\Downarrow_{\mathcal N} x$.
\end{enumerate}
$P$ is ${\mathcal N}$-barbed bisimilar to $Q$, written
$P \wbbisim_{\mathcal N} Q$, if $P \rel{S}_{\mathcal N} Q$ for some ${\mathcal N}$-barbed bisimulation ${\mathcal S}_{\mathcal N}$.
\end{definition}

$\mathcal{R} \subseteq \pi \times \pi$

$P \mathcal{R} Q => \forall P'. P \red P' \Rightarrow \exists Q'. Q \red Q', P' \mathcal{R} Q'$

$P \vdash x \Rightarrow Q \vdash x$

\begin{mathpar}
  \inferrule*[lab=Out-barb]{x \nameeq y}{{y}!\langle{Q}\rangle \vdash x}
  \and
  \inferrule*[lab=Par-barb]{\mbox{$P\vdash x$ or $Q\vdash x$}}{\binpar{P}{Q} \vdash x}
\end{mathpar}

\subsubsection{Contexts}

One of the principle advantages of computational calculi like the
$\pi$-calculus is a well-defined notion of context,
contextual-equivalence and a correlation between
contextual-equivalence and notions of bisimulation. The notion of
context allows the decomposition of a process into (sub-)process and
its syntactic environment, its context. Thus, a context may be
thought of as a process with a ``hole'' (written $\Box$) in it. The
application of a context $M$ to a process $P$, written $M[P]$, is
tantamount to filling the hole in $M$ with $P$. In this paper we do
not need the full weight of this theory, but do make use of the notion
of context in the proof the main theorem. 

\begin{mathpar}
  \inferrule* [lab=summation] {} {{M_{M},M_{N}} \bc \Box \;|\; x.M_{A} \;|\; M_{M}+M_{N}}
  \and
  \inferrule* [lab=agent] {} {{M_{A}} \bc (\vec{x})M_{P} \;| \; \clift{P_0,\ldots,M_{P},\ldots,P_N}}
  \and \\
  \inferrule* [lab=process] {} {{M_{P}} \bc M_{N} \;| \;P|M_{P} }
\end{mathpar} 

\begin{mathpar}
  \inferrule* [lab=sychronization] {} {M_{N} \bc \Box \;|\; x?M_{F} \;|\; x!M_{C}}
  \and
  \inferrule* [lab=abstraction] {} {{M_{F}} \bc (x)M_{P} }
  \and
  \inferrule* [lab=concretion] {} {{M_{C}} \bc \langle M_{P} \rangle }
  \and \\
  \inferrule* [lab=process] {} {{M_{P}} \bc M_{N} \;| \;P|M_{P} }
\end{mathpar}

\begin{definition}[contextual application] Given a context $M$, and
  process $P$, we define the \emph{contextual application}, $M[P] :=
  M\{P/\Box\}$. That is, the contextual application of M to P is the
  substitution of $P$ for $\Box$ in $M$.
\end{definition}

$\meaningof{-} : L \to \mathcal{P}(\pi)$

\begin{mathpar}
  \inferrule* [lab=collection] {} {\meaningof{true} = \pi, \and \meaningof{~E} = \pi \setminus \meaningof{E}, \and \meaningof{E_{1} \& E_{2}} = \meaningof{E_{1}} \cap \meaningof{E_{2}}}
\end{mathpar}

\begin{mathpar}
  \inferrule* [lab=structure] {} {\meaningof{0} = \{ P \in \pi | P \equiv 0 \}, \and \\ \meaningof{E_1 | E_2} = \{ P \in \pi | P \equiv P_{1} | P_{2}, P_{1} \in \meaningof{E_{1}}, P_{2} \in \meaningof{E_2}\} }
\end{mathpar}

\begin{mathpar}
 \inferrule* [lab=behavior] {} {\meaningof{\langle a?b \rangle E} = \{ P \in \pi | P \equiv Q | u?(y)P', \\ \and \\\\ \and \\ \;\;\; u \in \meaningof{a}, \forall z.P'\{z/y\} \in \meaningof{E\{z/b\}}\}, \and \\ \meaningof{a!E} = \{ P \in \pi | P \equiv Q | x!\langle P' \rangle, x \in \meaningof{a} P' \in \meaningof{E}\} }
\end{mathpar}

\begin{mathpar}
 \inferrule* [lab=nominal] {} {\meaningof{\quotep{E}} = \{ \quotep{P} \in \quotep{\pi} | P \in \meaningof{E} \}, \and \meaningof{\quotep{P}} = \{ \quotep{Q} \in \quotep{\pi} | P \equiv Q \} \and \\ \meaningof{@\quotep{E}} = \{ P \in \pi | P \equiv @x, x \in \meaningof{E} \}}
\end{mathpar}

\begin{eqnarray*}
  \\
  \meaningof{-} : TS \to ST
\end{eqnarray*}

\begin{eqnarray*}
  \\
  L : TS \to ST
\end{eqnarray*}

\begin{eqnarray*}
  \\
  P \models E \iff P \in \meaningof{E}
\end{eqnarray*}

\begin{eqnarray*}
  P \approx_{L} Q \iff \forall E \in L. P \models E \iff Q \models E
\end{eqnarray*}

\begin{eqnarray*}
  P \approx_{K} Q
\end{eqnarray*}

\begin{eqnarray*}
  P \approx Q
\end{eqnarray*}

$\approx_{K} = \approx = \approx_{L}$

\subsubsection{Contextual duality}

Note that contexts extend the quotation operation to a family of
operations from processes to names. Given a context, $M$, we can
define a \emph{nominal context}, $\quotep{M}$ by $\quotep{M}[P] :=
\quotep{M[P]}$. To foreshadow what is to come we observe that these
operations enjoy a duality with processes very much like the duality
between vectors and maps from vectors to scalars.

Further, because the calculus is essentially higher-order, we have a
correspondence between contexts and processes. More specifically,
given a name $x$ and a context $M$ we can construct $M^{*}_{x}$ such
that 

\begin{mathpar}
  M^{*}_{x} | \lift{x}{P} \red M[P]
\end{mathpar}

namely,

\begin{mathpar}
  M^{*}_{x} := x?(u).M[\dropn{u}]
\end{mathpar}

The dependence of $M^{*}_{x}$ on a name makes it an abstraction, 

\begin{mathpar}
  M^{*} := (x)x?(u).M[\dropn{u}]
\end{mathpar}

\subsection{Additional notation}

It will sometimes be convenient to denote the process a name
quotes. We already have the notation $x = \quotep{P}$, but it will be
convenient to introduce an alternate notation, $\procn{x}$, when we
want to emphasize the connection to the use of the name. Note that, by
virtue of name equivalence, $\quotep{\procn{x}} \nameeq x$; so, the
notation is consistent with previous definitions.

Further, because names have structure it is possible to effect
substitutions on the basis of that structure. This means we need to
upgrade our notation for substitutions, which we accomplish by
adapting comprehension notation. Thus,

\begin{mathpar}
  P\{ y / x : x \in S \}
\end{mathpar}

is interpreted to mean the process derived from P by replacing (in a
capture-avoiding manner) each occurrence of $x$ in $S$ by $y$. For example,

\begin{mathpar}
  P\{ \quotep{\procn{x}|\procn{x}} / x : x \in \freenames{P} \}
\end{mathpar}

will replace each (occurrence) of a free name $x$ in $P$ by
$\quotep{\procn{x}|\procn{x}}$.

Also, we will avail ourselves of the notation $x^{L}$ and $x^{R}$ to
denote injections of a name into disjoint copies of the name
space. There are numerous ways to accomplish this. One example can be
found in \cite{MeredithR05}. This notation overloads to vectors of
names: $\vec{x}^{\pi} := (x_{i}^{\pi} \; : \; 0 \leq i < |\vec{x}| )$ where $\pi \in \{L,R\}$.

We also use $P^{\Box} := P|\Box$.

In \cite{MeredithR05} an interpretation of the new operator is
given. It turns out that there are several possible interpretations
all enjoying the requisite algebraic properties of the operator (see
\cite{milner91polyadicpi}). We will therefore make liberal use of
$(\nu\; \vec{x})P$.

% subsection the_syntax_and_semantics_of_the_notation_system (end)   

\input{qm2pi.qmops} 

\input{qm2pi.sterngerlach} 

\input{qm2pi.metric} 

% section concurrent_process_calculi (end)

%\input{qm2pi.proofsketch}

% section proof sketch (end)

%\input{qm2pi.slviaknots} 

% section spatial logic via knots (end)

\input{qm2pi.conclusion}

% section conclusion (end)

%\input{qm2pi.dtcodes} 

% section wiring algorithm (end)

\input{qm2pi.ack} 

% section acknowledgments (end)

\newpage


\bibliographystyle{plain}   
\bibliography{../../biblios/main.bib}

\input{qm2pi.rhodetails}

\end{document}

 

% section concurrent_process_calculi (end)

%\documentclass[12pt]{llncs}
%\documentclass{jktr}

\usepackage[pdftex]{hyperref}                   
\usepackage {listings}
\usepackage {mathpartir}
\usepackage{bcprules}
%\usepackage{listings}
                       
\usepackage{graphicx} 
%\usepackage[margins=2.5cm,nohead,nofoot]{geometry}
%\usepackage{geometry}
\usepackage{amsfonts}
\usepackage{amstext}
\usepackage{latexsym}
\usepackage{amssymb}
\usepackage{color}


%\include{myPreamble}
\include{qm2pi.local} 

%\ifpdf
%\usepackage[pdftex]{graphicx}
%\else
%\usepackage{graphicx}
%\fi

 % \ifpdf
%  \usepackage{pdfsync}
%  \if


%\title{Brief Article}
%\author{David F. Snyder}
%\author{L.G. Meredith}

%\address{Dept. of Math., Texas State University--San Marcos, San Marcos, TX 78666}
       
\pagestyle{empty}


\begin{document}

\lstset{language=[Objective]Caml,frame=shadowbox}

\input{qm2pi.front}

% section front matter (end)

\input{qm2pi.intro} 
 
% section introduction (end)

% \input{qm2pi.knotations} 

% section notation (end)

\input{qm2pi.process.calculi} 

% section concurrent_process_calculi_and_spatial_logics_ (end)
    
%\input{qm2pi.knots2pi} 

%\input{qm2pi.trefoil} 

%\input{qm2pi.mainthm} 

% subsection basic_interpretation (end)

%\input{qm2pi.rho.presentation} 
\subsection{The syntax and semantics of the notation system}\label{sub:the_syntax_and_semantics_of_the_notation_system} % (fold)

We now summarize a technical presentation of the calculus that
embodies our theory of dynamics. The typical presentation of such a
calculus follows the style of giving generators and relations on
them. The grammar, below, describing term constructors, freely
generates the set of processes, $\Proc$. This set is then quotiented
by a relation known as structural congruence and it is over this set
that the notion of dynamics is expressed. This presentation is
essentially that of \cite{MeredithR05} with the addition of
polyadicity and summation. For readability we have relegated some of
the technical subtleties to an appendix.

\subsubsection{Process grammar}\label{subsub:process_grammar}

\begin{mathpar}
  \inferrule* [lab=synchronization] {} {{M} \bc \pzero \;|\; x?F \;|\; x!C }
  \and
  \inferrule* [lab=abstraction] {} {{F} \bc (x)P}
  \and
  \inferrule* [lab=concretion] {} {{C} \bc \langle Q \rangle}
  \and
  \inferrule* [lab=process] {} {{P,Q} \bc M \;| \;P|Q \;|\; @{x}}
  \and
  \inferrule* [lab=name] {} {{x} \bc \quotep{P}}
\end{mathpar} 

Note that $\vec{x}$ (resp. $\vec{P}$) denotes a vector of names
(resp. processes) of length $|\vec{x}|$ (resp. $|\vec{P}|$). We adopt
the following useful abbreviations.

\begin{mathpar}
   x?(\vec{y}).P := x.(\vec{y})P \and  x\clift{\vec{P}} := x.\clift{\vec{P}}
   \and x!(y) := \lift{x}{\dropn{y}}
   \and \Pi_{i=0}^{n-1}P_i := P_0 | \ldots | P_{n-1}
\end{mathpar}

\subsubsection{Structural congruence}

\paragraph{Free and bound names and alpha-equivalence.} At the
core of structural equivalence is alpha-equivalence which identifies
process that are the same up to a change of variable. Formally, we
recognize the distinction between free and bound names. The free names
of a process, $\freenames{P}$, may be calculated recursively as
follows:

\begin{mathpar}
\freenames{\pzero} := \emptyset
  \and \\
  \freenames{x?(y).P} := \{ x \} \cup (\freenames{P} \setminus \{ y \})
  \and 
  \freenames{x!\langle P \rangle} := \{ x \} \cup \{ P \} 
  \and \\
  \freenames{P|Q} := \freenames{P} \cup \freenames{Q}
  \and \\
  \freenames{@{x}} := \{ x \}
\end{mathpar}

$\pi$
$\quotep{\pi}$

$\freenames{-} : \pi \to \mathcal{P}(\quotep{\pi})$

\begin{eqnarray*}
  \freenames{\pzero} & := & \emptyset \\
  \freenames{x?(y).P} & := & \{ x \} \cup (\freenames{P} \setminus \{ y \}) \\
  \freenames{x!\langle P \rangle} & := & \{ x \} \cup \{ P \} \\
  \freenames{P|Q} & := & \freenames{P} \cup \freenames{Q} \\
  \freenames{\dropn{x}} & := & \{ x \}
\end{eqnarray*}

The bound names of a process, $\boundnames{P}$, are those names occurring in $P$
that are not free. For example, in $x?(y).0$, the name $x$ is free, while $y$ is bound.

\begin{mathpar}
  \inferrule* [lab=monoidal-laws] {} { P|Q \equiv Q|P \and P|0 \equiv P \and P|(Q|R) \equiv (P|Q)|R }
\end{mathpar}

\begin{mathpar}
  \inferrule* [lab=alpha-equivalence] {} { (x)P \equiv (y)P\{y/x\} \and y \not\in \freenames{P} }
\end{mathpar}

\begin{definition}
Then two processes, $P,Q$, are alpha-equivalent if $P = Q\{\vec{y}/\vec{x}\}$ for
some $\vec{x} \in \boundnames{Q},\vec{y} \in \boundnames{P}$, where $Q\{\vec{y}/\vec{x}\}$
denotes the capture-avoiding substitution of $\vec{y}$ for $\vec{x}$ in $Q$.
\end{definition}

\begin{definition}
  The {\em structural congruence} \cite{SangiorgiWalker} , $\equiv$,
  between processes is the least congruence containing
  alpha-equivalence, satisfying the abelian monoid laws
  (associativity, commutativity and $\pzero$ as identity) for parallel
  composition $|$ and for summation $+$.
\end{definition}

\subsection{Name equivalence}

We take name equivalence, written $\nameeq$, to be the smallest
equivalence relation generated by the following rules.

\begin{mathpar}
\inferrule*[lab=Quote-drop]
{ }
{ \quotep{@{x}} \nameeq x }

\inferrule*[lab=Struct-equiv]
{ P \scong Q }
{ \quotep{P} \nameeq \quotep{Q} }
\end{mathpar}

The astute reader will have noticed that the mutual recursion of names
and processes imposes a mutual recursion on alpha-equivalence and
structural equivalence via name-equivalence. Fortunately, all of this
works out pleasantly and we may calculate in the natural way, free of
concern. The reader interested in the details is referred to the
appendix \ref{appendix:rho_details}.

\subsection{Substitution}

We use $\Proc$ for the set of processes, $\QProc$ for the set of
names, and $\id{\{}\vec{y} / \vec{x} \id{\}}$ to denote partial maps,
$s : \QProc \rightarrow \QProc$. A map, $s$ lifts, uniquely, to a map
on process terms, $\widehat{s} : \Proc \rightarrow \Proc$ by the
following equations.

\begin{mathpar}
  (0) \psubstp{Q}{P} := 0 \\
  (R \juxtap S) \psubstp{Q}{P}
  :=    
  (R)\psubstp{Q}{P} \juxtap (S) \psubstp{Q}{P} \\
  (x?(y).R) \psubstp{Q}{P}    
  :=    
  (x)\substp{Q}{P} (z)\concat( (R \psubstn{z}{y}) \psubstp{Q}{P} ) \\
  (\lift{x}{R}) \psubstp{Q}{P}  
  :=
  \lift{(x)\substp{Q}{P}}{ R \psubstp{Q}{P} } \\
%   (\dropn{x})  \psubstp{Q}{P}       
%   := 
%   \left\{ 
%     \begin{array}{ccc} 
%       \dropn{\quotep{Q}} & & x \nameeq \quotep{P} \\
%       \dropn{x} & & otherwise \\
%     \end{array}
%   \right. 
  (\dropn{x})  \psubstp{Q}{P}       
  := 
  \left\{ 
    \begin{array}{ccc} 
      Q & & x \nameeq \quotep{P} \\
      \dropn{x} & & otherwise \\
    \end{array}
  \right.
\end{mathpar}
 

where

\begin{eqnarray}
  (x)\id{\{} \lpquote Q \rpquote / \lpquote P \rpquote \id{\}}            = 
  \left\{ 
    \begin{array}{ccc}
      \lpquote Q \rpquote & & x \nameeq \lpquote P \rpquote \\
      x & & otherwise \\
    \end{array}
  \right. \nonumber
\end{eqnarray}

and $z$ is chosen distinct from $\quotep{P}$, $\quotep{Q}$, the free
names in $Q$, and all the names in $R$. Our $\alpha$-equivalence will
be built in the standard way from this substitution.

\begin{remark}\label{rem:no_self_referential_names}
  One consequence of these definitions is that $\forall P. \quotep{P}
  \not\in \freenames{P}$.
\end{remark}

\subsection{ Dynamic quote: an example }

Anticipating something of what's to come, consider applying the
substitution, $\widehat{\id{\{}u / z \id{\}}}$, to the following pair
of processes, $\lift{w}{y!(z)}$ and $w[ \lpquote y!(z) \rpquote ]$.

\begin{eqnarray}
	\lift{w}{y!(z)}\widehat{\id{\{}u / z \id{\}}}
		& = &
		\lift{w}{y!(u)} \nonumber\\
	w[ \lpquote y!(z) \rpquote ] \widehat{ \id{\{}u / z \id{\}} }
		& = &
		w[ \lpquote y!(z) \rpquote ] \nonumber
\end{eqnarray}

Because the body of the process between quotes is impervious to
substitution, we get radically different answers. In fact, by
examining the first process in an input context,
e.g. $x?(z).\lift{w}{y!(z)}$, we see that the process under the lift
operator may be shaped by prefixed inputs binding a name inside it. In
this sense, the lift operator will be seen as a way to dynamically
construct processes before reifying them as names.

Finally equipped with these standard features we can present the
dynamics of the calculus.

\subsubsection{Operational semantics} 

Finally, we introduce the computational dynamics. What marks these
algebras as distinct from other more traditionally studied algebraic
structures, e.g. vector spaces or polynomial rings, is the manner in
which dynamics is captured. In traditional structures, dynamics is typically
expressed through morphisms between such structures, as in linear maps
between vector spaces or morphisms between rings. In algebras
associated with the semantics of computation, the dynamics is
expressed as part of the algebraic structure itself, through a
reduction reduction relation typically denoted by $\red$. Below, we
give a recursive presentation of this relation for the calculus used
in the encoding.

$\red \subseteq \pi \times \pi$
$\red : \pi \to \mathcal{P}(\pi)$

\begin{mathpar}
  \inferrule* [lab=Comm] { \textsf{match}( x_{src}, x_{trgt} ) } { x_{trgt}?(y)P \; | \; x_{src}!\langle {Q} \rangle \red P\{\quotep{Q}/y}\} }
  \and \\
  \inferrule* [lab=Par] {{P} \red {P}'} {{{P} | {Q}} \red {{P}' | {Q}}}
  \and
  \inferrule* [lab=Equiv]{{{P} \scong {P}'} \andalso {{P}' \red {Q}'} \andalso {{Q}' \scong {Q}}}{{P} \red {Q}}
\end{mathpar}

\begin{eqnarray*}
  match_{\equiv} (\quotep{P},\quotep{Q}) & := & P \equiv Q \\
  match_{\dagger}(\quotep{P},\quotep{Q}) & := & \forall R. P|Q \red^{*} R => R \red^{*} 0 \\
  match_{K}(\quotep{P},\quotep{Q}) & := & K \mbox{ for some context } K
\end{eqnarray*}

$u?(x)P | u!\langle Q \rangle \red P\{\quotep{Q}/x\}$

%We write $\wred$ for $\red^*$, and $P\red$ if $\exists Q $ such that $ P \red Q$.
We write $P\red$ if $\exists Q $ such that $ P \red Q$ and $P\not\red$, otherwise.

\section{Replication}

As mentioned before, it is known that replication (and hence
recursion) can be implemented in a higher-order process algebra
\cite{SangiorgiWalker}. As our first example of calculation with the
machinery thus far presented we give the construction explicitly in
the {\rhoc}.

\begin{eqnarray}
	D_{x} & := & \prefix{x}{y}{(\binpar{\outputp{x}{y}}{@{y}})} \nonumber\\
	\bangp_{x}{P} & := & \binpar{{x}!\langle{\binpar{D_{x}}{P}}\rangle}{D_{x}} \nonumber
\end{eqnarray}

\begin{eqnarray}
	\bangp_{x}{P} & & \nonumber\\
	=
	& {x}!\langle{(\prefix{x}{y}{(\outputp{x}{y} | @{y})) | P}}\rangle 
	      | \prefix{x}{y}{(\outputp{x}{y} | @{y})} & \nonumber\\
	\red
	& (\outputp{x}{y} | @{y})\substn{\quotep{(\prefix{x}{y}{(@{y} | \outputp{x}{y})) | P}}}{y} & \nonumber\\
	=
	& \outputp{x}{\quotep{(\prefix{x}{y}{(\outputp{x}{y} | @{y})) | P}}}
	  | {(\prefix{x}{y}{(\outputp{x}{y} | @{y})) | P}} & \nonumber\\
	\red
	& \ldots & \nonumber\\
	\red^*
	& P | P | \ldots & \nonumber
\end{eqnarray}

Of course, this encoding, as an implementation, runs away, unfolding
$\bangp{P}$ eagerly. A lazier and more implementable replication
operator, restricted to input-guarded processes, may be obtained as follows.

\begin{eqnarray}
\bangp{\prefix{u}{v}{P}} 
	:= 
	\binpar{\lift{x}{\prefix{u}{v}{(\binpar{D(x)}{P})}}}{D(x)} \nonumber
\end{eqnarray}

\begin{remark}
  Note that the lazier definition still does not deal with summation
  or mixed summation (i.e. sums over input and output). The reader is
  invited to construct definitions of replication that deal with these
  features. 

  Further, the definitions are parameterized in a name, $x$. Can you,
  gentle reader, make a definition that eliminates this parameter and
  guarantees no accidental interaction between the replication
  machinery and the process being replicated -- i.e. no accidental
  sharing of names used by the process to get its work done and the
  name(s) used by the replication to effect copying. This latter
  revision of the definition of replication is crucial to obtaining
  the expected identity $!!P \sim !P$.
\end{remark}

\begin{remark}\label{rem:paradoxical_combinator}
  The reader familiar with the lambda calculus will have noticed the
  similarity between $D$ and the paradoxical combinator.

  [Ed. note: the existence of this seems to suggest we have to be more
  restrictive on the set of processes and names we admit if we are to
  support no-cloning.]
\end{remark}

\subsubsection{Bisimulation}

The computational dynamics gives rise to another kind of equivalence,
the equivalence of computational behavior. As previously mentioned
this is typically captured \emph{via} some form of bisimulation.

% The notion we use in this paper is weak barbed bisimulation
% \cite{milner91polyadicpi}.

The notion we use in this paper is derived from weak barbed
bisimulation \cite{milner91polyadicpi}. 

\begin{definition}
An \emph{observation relation}, $\downarrow_{\mathcal N}$, over a set
of names, $\mathcal N$, is the smallest relation satisfying the rules
below.

\infrule[Out-barb]{y \in {\mathcal N}, \; x \nameeq y}
		  {\outputp{x}{v} \downarrow_{\mathcal N} x}
\infrule[Par-barb]{\mbox{$P\downarrow_{\mathcal N} x$ or $Q\downarrow_{\mathcal N} x$}}
		  {\binpar{P}{Q} \downarrow_{\mathcal N} x}

We write $P \Downarrow_{\mathcal N} x$ if there is $Q$ such that 
$P \wred Q$ and $Q \downarrow_{\mathcal N} x$.
\end{definition}

\begin{definition}
%\label{def.bbisim}
An  ${\mathcal N}$-\emph{barbed bisimulation} over a set of names, ${\mathcal N}$, is a symmetric binary relation 
${\mathcal S}_{\mathcal N}$ between agents such that $P\rel{S}_{\mathcal N}Q$ implies:
\begin{enumerate}
\item If $P \red P'$ then $Q \wred Q'$ and $P'\rel{S}_{\mathcal N} Q'$.
\item If $P\downarrow_{\mathcal N} x$, then $Q\Downarrow_{\mathcal N} x$.
\end{enumerate}
$P$ is ${\mathcal N}$-barbed bisimilar to $Q$, written
$P \wbbisim_{\mathcal N} Q$, if $P \rel{S}_{\mathcal N} Q$ for some ${\mathcal N}$-barbed bisimulation ${\mathcal S}_{\mathcal N}$.
\end{definition}

$\mathcal{R} \subseteq \pi \times \pi$

$P \mathcal{R} Q => \forall P'. P \red P' \Rightarrow \exists Q'. Q \red Q', P' \mathcal{R} Q'$

$P \vdash x \Rightarrow Q \vdash x$

\begin{mathpar}
  \inferrule*[lab=Out-barb]{x \nameeq y}{{y}!\langle{Q}\rangle \vdash x}
  \and
  \inferrule*[lab=Par-barb]{\mbox{$P\vdash x$ or $Q\vdash x$}}{\binpar{P}{Q} \vdash x}
\end{mathpar}

\subsubsection{Contexts}

One of the principle advantages of computational calculi like the
$\pi$-calculus is a well-defined notion of context,
contextual-equivalence and a correlation between
contextual-equivalence and notions of bisimulation. The notion of
context allows the decomposition of a process into (sub-)process and
its syntactic environment, its context. Thus, a context may be
thought of as a process with a ``hole'' (written $\Box$) in it. The
application of a context $M$ to a process $P$, written $M[P]$, is
tantamount to filling the hole in $M$ with $P$. In this paper we do
not need the full weight of this theory, but do make use of the notion
of context in the proof the main theorem. 

\begin{mathpar}
  \inferrule* [lab=summation] {} {{M_{M},M_{N}} \bc \Box \;|\; x.M_{A} \;|\; M_{M}+M_{N}}
  \and
  \inferrule* [lab=agent] {} {{M_{A}} \bc (\vec{x})M_{P} \;| \; \clift{P_0,\ldots,M_{P},\ldots,P_N}}
  \and \\
  \inferrule* [lab=process] {} {{M_{P}} \bc M_{N} \;| \;P|M_{P} }
\end{mathpar} 

\begin{mathpar}
  \inferrule* [lab=sychronization] {} {M_{N} \bc \Box \;|\; x?M_{F} \;|\; x!M_{C}}
  \and
  \inferrule* [lab=abstraction] {} {{M_{F}} \bc (x)M_{P} }
  \and
  \inferrule* [lab=concretion] {} {{M_{C}} \bc \langle M_{P} \rangle }
  \and \\
  \inferrule* [lab=process] {} {{M_{P}} \bc M_{N} \;| \;P|M_{P} }
\end{mathpar}

\begin{definition}[contextual application] Given a context $M$, and
  process $P$, we define the \emph{contextual application}, $M[P] :=
  M\{P/\Box\}$. That is, the contextual application of M to P is the
  substitution of $P$ for $\Box$ in $M$.
\end{definition}

$\meaningof{-} : L \to \mathcal{P}(\pi)$

\begin{mathpar}
  \inferrule* [lab=collection] {} {\meaningof{true} = \pi, \and \meaningof{~E} = \pi \setminus \meaningof{E}, \and \meaningof{E_{1} \& E_{2}} = \meaningof{E_{1}} \cap \meaningof{E_{2}}}
\end{mathpar}

\begin{mathpar}
  \inferrule* [lab=structure] {} {\meaningof{0} = \{ P \in \pi | P \equiv 0 \}, \and \\ \meaningof{E_1 | E_2} = \{ P \in \pi | P \equiv P_{1} | P_{2}, P_{1} \in \meaningof{E_{1}}, P_{2} \in \meaningof{E_2}\} }
\end{mathpar}

\begin{mathpar}
 \inferrule* [lab=behavior] {} {\meaningof{\langle a?b \rangle E} = \{ P \in \pi | P \equiv Q | u?(y)P', \\ \and \\\\ \and \\ \;\;\; u \in \meaningof{a}, \forall z.P'\{z/y\} \in \meaningof{E\{z/b\}}\}, \and \\ \meaningof{a!E} = \{ P \in \pi | P \equiv Q | x!\langle P' \rangle, x \in \meaningof{a} P' \in \meaningof{E}\} }
\end{mathpar}

\begin{mathpar}
 \inferrule* [lab=nominal] {} {\meaningof{\quotep{E}} = \{ \quotep{P} \in \quotep{\pi} | P \in \meaningof{E} \}, \and \meaningof{\quotep{P}} = \{ \quotep{Q} \in \quotep{\pi} | P \equiv Q \} \and \\ \meaningof{@\quotep{E}} = \{ P \in \pi | P \equiv @x, x \in \meaningof{E} \}}
\end{mathpar}

\begin{eqnarray*}
  \\
  \meaningof{-} : TS \to ST
\end{eqnarray*}

\begin{eqnarray*}
  \\
  L : TS \to ST
\end{eqnarray*}

\begin{eqnarray*}
  \\
  P \models E \iff P \in \meaningof{E}
\end{eqnarray*}

\begin{eqnarray*}
  P \approx_{L} Q \iff \forall E \in L. P \models E \iff Q \models E
\end{eqnarray*}

\begin{eqnarray*}
  P \approx_{K} Q
\end{eqnarray*}

\begin{eqnarray*}
  P \approx Q
\end{eqnarray*}

$\approx_{K} = \approx = \approx_{L}$

\subsubsection{Contextual duality}

Note that contexts extend the quotation operation to a family of
operations from processes to names. Given a context, $M$, we can
define a \emph{nominal context}, $\quotep{M}$ by $\quotep{M}[P] :=
\quotep{M[P]}$. To foreshadow what is to come we observe that these
operations enjoy a duality with processes very much like the duality
between vectors and maps from vectors to scalars.

Further, because the calculus is essentially higher-order, we have a
correspondence between contexts and processes. More specifically,
given a name $x$ and a context $M$ we can construct $M^{*}_{x}$ such
that 

\begin{mathpar}
  M^{*}_{x} | \lift{x}{P} \red M[P]
\end{mathpar}

namely,

\begin{mathpar}
  M^{*}_{x} := x?(u).M[\dropn{u}]
\end{mathpar}

The dependence of $M^{*}_{x}$ on a name makes it an abstraction, 

\begin{mathpar}
  M^{*} := (x)x?(u).M[\dropn{u}]
\end{mathpar}

\subsection{Additional notation}

It will sometimes be convenient to denote the process a name
quotes. We already have the notation $x = \quotep{P}$, but it will be
convenient to introduce an alternate notation, $\procn{x}$, when we
want to emphasize the connection to the use of the name. Note that, by
virtue of name equivalence, $\quotep{\procn{x}} \nameeq x$; so, the
notation is consistent with previous definitions.

Further, because names have structure it is possible to effect
substitutions on the basis of that structure. This means we need to
upgrade our notation for substitutions, which we accomplish by
adapting comprehension notation. Thus,

\begin{mathpar}
  P\{ y / x : x \in S \}
\end{mathpar}

is interpreted to mean the process derived from P by replacing (in a
capture-avoiding manner) each occurrence of $x$ in $S$ by $y$. For example,

\begin{mathpar}
  P\{ \quotep{\procn{x}|\procn{x}} / x : x \in \freenames{P} \}
\end{mathpar}

will replace each (occurrence) of a free name $x$ in $P$ by
$\quotep{\procn{x}|\procn{x}}$.

Also, we will avail ourselves of the notation $x^{L}$ and $x^{R}$ to
denote injections of a name into disjoint copies of the name
space. There are numerous ways to accomplish this. One example can be
found in \cite{MeredithR05}. This notation overloads to vectors of
names: $\vec{x}^{\pi} := (x_{i}^{\pi} \; : \; 0 \leq i < |\vec{x}| )$ where $\pi \in \{L,R\}$.

We also use $P^{\Box} := P|\Box$.

In \cite{MeredithR05} an interpretation of the new operator is
given. It turns out that there are several possible interpretations
all enjoying the requisite algebraic properties of the operator (see
\cite{milner91polyadicpi}). We will therefore make liberal use of
$(\nu\; \vec{x})P$.

% subsection the_syntax_and_semantics_of_the_notation_system (end)   

\input{qm2pi.qmops} 

\input{qm2pi.sterngerlach} 

\input{qm2pi.metric} 

% section concurrent_process_calculi (end)

%\input{qm2pi.proofsketch}

% section proof sketch (end)

%\input{qm2pi.slviaknots} 

% section spatial logic via knots (end)

\input{qm2pi.conclusion}

% section conclusion (end)

%\input{qm2pi.dtcodes} 

% section wiring algorithm (end)

\input{qm2pi.ack} 

% section acknowledgments (end)

\newpage


\bibliographystyle{plain}   
\bibliography{../../biblios/main.bib}

\input{qm2pi.rhodetails}

\end{document}



% section proof sketch (end)

%\section{Unlikely characters: spatial logic for
  knots}\label{sub:characteristic_formulae} % (fold)

Associated to the mobile process calculi are a family of logics known
as the Hennessy-Milner logics. These logics typically enjoy a
semantics interpreting formulae as sets of processes that when
factored through the encoding outlined above allows an identification
of classes of knots with logical formulae. In the context of this
encoding the sub-family known as the spatial logics \cite{CairesC03}
\cite{CairesC04} \cite{Caires04} are of particular interest providing
several important features for expressing and reasoning about
properties (i.e. classes) of knots. We hint here at how this may be done.

%\begin{description}
%\item [structural connectives] 
\subsubsection{Structural connectives} The spatial logics enjoy
structural connectives corresponding, at the logical level, to the
parallel composition ($P | Q$) and new name ($(\nu \; x)P$)
connectives for processes. As illustrated in the examples below, these
connectives are extremely expressive given the shape of our encoding.
%\item [decideable satisfaction]

\subsubsection{Decideable satisfaction}
In \cite{Caires04} the satisfaction relation is shown to be decideable
for a rich class of processes. It further turns out that the image of
the our encoding is a proper subset of that class. This result
provides the basis for an algorithm by which to search for knots
enjoying a given property.
%\item [characteristic formulae]

\subsubsection{Characteristic formulae}
In the same paper \cite{Caires04} , Caires presents a means of calculating
characteristic formulae, selecting equivalence classes of processes
up to a pre--specified depth limit on the support set of names. Composed with our
encoding, this characteristic formula can be used to select
characteristic formulae for knots.
%\end{description}

\subsubsection{Spatial logic formulae}

The grammar below (segmented for comprehension) summarizes the syntax
of spatial logic formulae. We employ illustrative examples in the
sequel to provide an intuitive understanding of their meaning
referring the reader to \cite{Caires04} for a more detailed explication
of the semantics.

\begin{mathpar}
  \inferrule* [lab=boolean] {} {{A,B} \bc T \;|\; \neg A \;|\; A \wedge B \;|\; \eta = \eta'}
  \and
  \inferrule* [lab=spatial] {} {|\; \pzero \;|\; A | B \;|\; x \text{\textregistered} A \;|\; \forall x . A \;|\;  H x . A}
  \and
  \inferrule* [lab=behavioral] {} {|\; \alpha . A}
  \and 
  \inferrule* [lab=recursion] {} {|\; X(\vec{u}) \;|\; \mu X(\vec{u}) . A}
  \and
  \inferrule* [lab=action] {} {\alpha \bc \langle x?(\vec{y}) \rangle \;|\; \langle x!(\vec{y}) \rangle \;|\; \langle \tau \rangle}
  \and 
  \inferrule* [lab=name] {} {\eta \bc x \;|\; \tau}
\end{mathpar} 

% subsection characteristic_formulae (end)   	 

\subsection{Example formulae}\label{sub:example_formulae_} % (fold)

\subsubsection{Crossing as formula.}
% 
% \begin{align*}
%   \frac{d}{dx} \sin x &= \cos x 
%   & \frac{d}{dx} e^x &= e^x \\
%   \frac{d}{dx} \cos x &= - \sin x 
%   & \frac{d}{dx} \log x &= \frac{1}{x} \\
% \end{align*} 

\begin{align*}
 \mu C(x_{0},x_{1},y_{0},y_{1},u).&(\langle x_{0}?(z) \rangle(\langle u! \rangle\langle y_{1}!z \rangle C(x_{0},x_{1},y_{0},y_{1},u)) & \\
  & \wedge \langle y_{1}?(z) \rangle (\langle u! \rangle \langle x_{0}!z \rangle C(x_{0},x_{1},y_{0},y_{1},u)) & \\
  & \wedge \langle x_{1}?(z) \rangle (\langle u? \rangle \langle y_{0}!z \rangle C(x_{0},x_{1},y_{0},y_{1},u)) & \\
  & \wedge \langle y_{0}?(z) \rangle (\langle u? \rangle \langle x_{1}!z \rangle C(x_{0},x_{1},y_{0},y_{1},u))) &
\end{align*}

The lexicographical similarity between the shape of this formulae and
the shape of definition of the process representing a crossing reveals
the intuitive meaning of this formulae. It describes the capabilities
of a process that has the right to represent a crossing. For example
it picks out processes that may perform an input on the port $x_0$ in
its initial menu of capabilities. What differentiates the formula
from the process, however, is that the crossing process is the
smallest candidate to satisfy the formula. Infinitely many other
processes -- with internal behavior hidden behind this interface, so
to speak -- also satisfy this formula. Even this simple formula,
then, can be seen to open a new view onto knots, providing a
computational interpretation of \emph{virtual} knots.

Note that this formula is derived by hand. A similar formula can be
derived by employing Caires' calculation of characteristic formula
\cite{Caires04} to the process representing a crossing. In light of
this discussion, we let
$\meaningof{C}_{\phi}(x0,x1,y0,y1,u)$ denote a formula specifying the
dynamics we wish to capture of a crossing. To guarantee we preserve
the shape of the interface and minimal semantics we demand that
$\meaningof{C}_{\phi}(x0,x1,y0,y1,u) \Rightarrow
\textbf{C}(x0,x1,y0,y1,u)$ where $\textbf{C}(x0,x1,y0,y1,u)$ denotes
the formula above.
                            
\subsubsection{Crossing number constraints.}
The moral content of the context lemma (Lemma \ref{context}) is that the notion of
``locality'' in the Reidemeister moves is effectively captured by the
parallel composition operator of the process calculus. This intuition
extends through the logic. Given a formula,
$\meaningof{C}_{\phi}(x0,x1,y0,y1,u)$, we can use the structural
connectives to specify constraints on crossing numbers, such as at
least $n$ crossings, or exactly $n$ crossings.
\begin{mathpar}
  \inferrule* [lab=at-least-n] {} { K^{\geq n}_{\phi}(\vec{xs},\vec{ys}) := \Pi_{i=0}^{n-1} Hu . \meaningof{C}_{\phi}(xs_i,ys_i,u) | T }
  \and 
  \inferrule* [lab=exactly-n] {} { K^{= n}_{\phi}(\vec{xs},\vec{ys}) := \Pi_{i=0}^{n-1} Hu . \meaningof{C}_{\phi}(xs_i,ys_i,u) | \neg (\forall x_0,y_0,x_1,y_1,u . \meaningof{C}_{\phi}(x_0,y_0,x_1,y_1,u) | T) }
\end{mathpar}

To round out this section, recall that the encoding of an $n$-crossing
knot decomposes into a parallel composition of $n$ \emph{copies} of a
crossing process together with a wiring harness. To specify different
knot classes with the same crossing number amounts to specifying
logical constraints on the wiring harness. In the interest of space,
we defer examples to a forthcoming paper. Suffice it to say that both
the conditions ``alternating knot'' and ``contains the tangle
corresponding to 5/3'' are expressible. For example, it is possible to
calculate the characteristic formula of a process corresponding to the
tangle 5/3 and conjoin it into the classifying formula via the
composition connective of the logic.

Finally, we wish to observe that it is entirely within reason to
contemplate a more domain-specific version of spatial logic tailored
to the shape of processes in the image of the encoding. Such a
domain-specific logic would have a better claim to the title formal
language of knot properties.

% subsection example_formulae_ (end)

% section knots_as_processes (end) 

% section spatial logic via knots (end)

\section{Conclusions and future work}

\paragraph{Testing physical space}
You, gentle reader, may wonder why of all the theorems to be proved
given this set up we pick the one above. In some sense it's hardly
central to quantum mechanics. We see it as central in the sense that
it firmly establishes a notion of physical space arising from a notion
of the equivalence of behavior. Relating bisimulation to a metric is a
big step forward, but one is faced with interpreting the relationship
of that metric space to something more physical. Quantum mechanical
notions of ``physical'' space are still far from intuitive, but by
relating this idea of distance as testing to calculations that predict
physical circumstances we are making a not insignificant step forward
toward an understanding of the physical space we inhabit as
essentially dynamic.

\paragraph{Effectivity and simulation}
One of the observations we have yet to make is that the entire program
spelled out here is effective. We have built various interpreters for
the reflective calculus at work in this interpretation. In principle,
then, we can simulate quantum mechanics on a computer. The place where
the simulation may lose fidelity is the infinitely branching summation
for the annihilator.

In this connection i also want to point out that the evaluation style
calculation of the inner product puts the non-determinism of the
summation right at the heart of measurement. This suggests that
Milner's original reduction-based formulation of the dynamics of his
calculi in terms of sums was not just notationally suggestive of a
notion of measure-and-continue but captured some significant part of
the physics.

\paragraph{Quantum continuations}
In light of this last observation i want to point out that the
predominant account of quantum mechanics is missing a key aspect of a
truly compositional story of the physical situation. In a real lab,
when a measurement is made the observation can be made to feed into
another device that then makes another measurement conditioned on the
results of the first. This means that after the superposition was
collapsed the entire experimental set up remained in
superposition. While QM offers a means of writing this down it doesn't
quite line up well with the well-trodden formulation of computation
and continuation that we see so succinctly expressed in Milner's
calculi. This suggests that there might be advantages to this account
of dynamics waiting to be explored.

\paragraph{Quantum logic}
In this connection, we also note that by virtue of having the
Hennessy-Milner construction, we can pull the construction through the
interpretation of QM. This gives us a natural candidate for a quantum
logic that enjoys an extremely tight connection with it's domain of
interpretation, making the construction much less ad hoc (rather it is
the image of functor!).

\paragraph{Quantum probabiity}
i have questions about the basis of the interpretation of inner
product as probability amplitude. In particular, using which
axiomatization of probability theory does the notion of probability
amplitude earn the right to be so dubbed? In other words, where is the
proof that the operation for calculating a probability amplitude (and
then squaring) satisfies the axioms of what it means to calculate a
probability? Even if such a proof exists (i have yet to find it in the
literature), i wonder if it might not be possible to turn things on
their heads. Can we view the calculation of the probability amplitude
as an axiomatization of probability? If so, then the definition we
give for calculating probability amplitude may provide the basis for
an \emph{effective} theory of probability.

\paragraph{Quantum vs ``biological'' information}
Finally, i want to conclude with a more philosophical observation. At
a recent workshop in which QM was a predominant topic i noticed
something about quantum information. The speaker was giving a riveting
discussion of axiomatic QM and showing how properties of ``no
cloning'' and ``no deleting'' emerged as consequences of the
axiomatization. Theorems of this form are necessary to give us a sense
of confidence that our axioms characterize the physical theory. What
struck me, though, was that if quantum information is neither erasable
nor replicable it is markedly different from \emph{life}. Two of the
things we know about life is that

\begin{itemize}
  \item it ends;
  \item to gain some measure of persistence, to transcend it's
    finitude it is imminently copyable.
\end{itemize}

Both of these qualities are summarized succinctly in the aphorism: all
flesh is grass. For me these two kinds of ``information'' -- call them
quantum and biological -- are end points on a spectrum of strategies
for persistence. At one end, we have those curious entities that enjoy
uniqueness and permanence; at the other, we have those who in the face
of a certain end and an uncertain present make a go of passing
something on. To me one of the more remarkable aspects of the latter
strategy is that in the presence of noise (and certain features of
copying) we get a kind of dynamism, a chance for improvement against a
given persistent condition.

% subsection other_calculi_other_bisimulations_and_geometry_as_behavior (end)




% section conclusion (end)

%\documentclass[12pt]{llncs}
%\documentclass{jktr}

\usepackage[pdftex]{hyperref}                   
\usepackage {listings}
\usepackage {mathpartir}
\usepackage{bcprules}
%\usepackage{listings}
                       
\usepackage{graphicx} 
%\usepackage[margins=2.5cm,nohead,nofoot]{geometry}
%\usepackage{geometry}
\usepackage{amsfonts}
\usepackage{amstext}
\usepackage{latexsym}
\usepackage{amssymb}
\usepackage{color}


%\include{myPreamble}
\include{qm2pi.local} 

%\ifpdf
%\usepackage[pdftex]{graphicx}
%\else
%\usepackage{graphicx}
%\fi

 % \ifpdf
%  \usepackage{pdfsync}
%  \if


%\title{Brief Article}
%\author{David F. Snyder}
%\author{L.G. Meredith}

%\address{Dept. of Math., Texas State University--San Marcos, San Marcos, TX 78666}
       
\pagestyle{empty}


\begin{document}

\lstset{language=[Objective]Caml,frame=shadowbox}

\input{qm2pi.front}

% section front matter (end)

\input{qm2pi.intro} 
 
% section introduction (end)

% \input{qm2pi.knotations} 

% section notation (end)

\input{qm2pi.process.calculi} 

% section concurrent_process_calculi_and_spatial_logics_ (end)
    
%\input{qm2pi.knots2pi} 

%\input{qm2pi.trefoil} 

%\input{qm2pi.mainthm} 

% subsection basic_interpretation (end)

%\input{qm2pi.rho.presentation} 
\subsection{The syntax and semantics of the notation system}\label{sub:the_syntax_and_semantics_of_the_notation_system} % (fold)

We now summarize a technical presentation of the calculus that
embodies our theory of dynamics. The typical presentation of such a
calculus follows the style of giving generators and relations on
them. The grammar, below, describing term constructors, freely
generates the set of processes, $\Proc$. This set is then quotiented
by a relation known as structural congruence and it is over this set
that the notion of dynamics is expressed. This presentation is
essentially that of \cite{MeredithR05} with the addition of
polyadicity and summation. For readability we have relegated some of
the technical subtleties to an appendix.

\subsubsection{Process grammar}\label{subsub:process_grammar}

\begin{mathpar}
  \inferrule* [lab=synchronization] {} {{M} \bc \pzero \;|\; x?F \;|\; x!C }
  \and
  \inferrule* [lab=abstraction] {} {{F} \bc (x)P}
  \and
  \inferrule* [lab=concretion] {} {{C} \bc \langle Q \rangle}
  \and
  \inferrule* [lab=process] {} {{P,Q} \bc M \;| \;P|Q \;|\; @{x}}
  \and
  \inferrule* [lab=name] {} {{x} \bc \quotep{P}}
\end{mathpar} 

Note that $\vec{x}$ (resp. $\vec{P}$) denotes a vector of names
(resp. processes) of length $|\vec{x}|$ (resp. $|\vec{P}|$). We adopt
the following useful abbreviations.

\begin{mathpar}
   x?(\vec{y}).P := x.(\vec{y})P \and  x\clift{\vec{P}} := x.\clift{\vec{P}}
   \and x!(y) := \lift{x}{\dropn{y}}
   \and \Pi_{i=0}^{n-1}P_i := P_0 | \ldots | P_{n-1}
\end{mathpar}

\subsubsection{Structural congruence}

\paragraph{Free and bound names and alpha-equivalence.} At the
core of structural equivalence is alpha-equivalence which identifies
process that are the same up to a change of variable. Formally, we
recognize the distinction between free and bound names. The free names
of a process, $\freenames{P}$, may be calculated recursively as
follows:

\begin{mathpar}
\freenames{\pzero} := \emptyset
  \and \\
  \freenames{x?(y).P} := \{ x \} \cup (\freenames{P} \setminus \{ y \})
  \and 
  \freenames{x!\langle P \rangle} := \{ x \} \cup \{ P \} 
  \and \\
  \freenames{P|Q} := \freenames{P} \cup \freenames{Q}
  \and \\
  \freenames{@{x}} := \{ x \}
\end{mathpar}

$\pi$
$\quotep{\pi}$

$\freenames{-} : \pi \to \mathcal{P}(\quotep{\pi})$

\begin{eqnarray*}
  \freenames{\pzero} & := & \emptyset \\
  \freenames{x?(y).P} & := & \{ x \} \cup (\freenames{P} \setminus \{ y \}) \\
  \freenames{x!\langle P \rangle} & := & \{ x \} \cup \{ P \} \\
  \freenames{P|Q} & := & \freenames{P} \cup \freenames{Q} \\
  \freenames{\dropn{x}} & := & \{ x \}
\end{eqnarray*}

The bound names of a process, $\boundnames{P}$, are those names occurring in $P$
that are not free. For example, in $x?(y).0$, the name $x$ is free, while $y$ is bound.

\begin{mathpar}
  \inferrule* [lab=monoidal-laws] {} { P|Q \equiv Q|P \and P|0 \equiv P \and P|(Q|R) \equiv (P|Q)|R }
\end{mathpar}

\begin{mathpar}
  \inferrule* [lab=alpha-equivalence] {} { (x)P \equiv (y)P\{y/x\} \and y \not\in \freenames{P} }
\end{mathpar}

\begin{definition}
Then two processes, $P,Q$, are alpha-equivalent if $P = Q\{\vec{y}/\vec{x}\}$ for
some $\vec{x} \in \boundnames{Q},\vec{y} \in \boundnames{P}$, where $Q\{\vec{y}/\vec{x}\}$
denotes the capture-avoiding substitution of $\vec{y}$ for $\vec{x}$ in $Q$.
\end{definition}

\begin{definition}
  The {\em structural congruence} \cite{SangiorgiWalker} , $\equiv$,
  between processes is the least congruence containing
  alpha-equivalence, satisfying the abelian monoid laws
  (associativity, commutativity and $\pzero$ as identity) for parallel
  composition $|$ and for summation $+$.
\end{definition}

\subsection{Name equivalence}

We take name equivalence, written $\nameeq$, to be the smallest
equivalence relation generated by the following rules.

\begin{mathpar}
\inferrule*[lab=Quote-drop]
{ }
{ \quotep{@{x}} \nameeq x }

\inferrule*[lab=Struct-equiv]
{ P \scong Q }
{ \quotep{P} \nameeq \quotep{Q} }
\end{mathpar}

The astute reader will have noticed that the mutual recursion of names
and processes imposes a mutual recursion on alpha-equivalence and
structural equivalence via name-equivalence. Fortunately, all of this
works out pleasantly and we may calculate in the natural way, free of
concern. The reader interested in the details is referred to the
appendix \ref{appendix:rho_details}.

\subsection{Substitution}

We use $\Proc$ for the set of processes, $\QProc$ for the set of
names, and $\id{\{}\vec{y} / \vec{x} \id{\}}$ to denote partial maps,
$s : \QProc \rightarrow \QProc$. A map, $s$ lifts, uniquely, to a map
on process terms, $\widehat{s} : \Proc \rightarrow \Proc$ by the
following equations.

\begin{mathpar}
  (0) \psubstp{Q}{P} := 0 \\
  (R \juxtap S) \psubstp{Q}{P}
  :=    
  (R)\psubstp{Q}{P} \juxtap (S) \psubstp{Q}{P} \\
  (x?(y).R) \psubstp{Q}{P}    
  :=    
  (x)\substp{Q}{P} (z)\concat( (R \psubstn{z}{y}) \psubstp{Q}{P} ) \\
  (\lift{x}{R}) \psubstp{Q}{P}  
  :=
  \lift{(x)\substp{Q}{P}}{ R \psubstp{Q}{P} } \\
%   (\dropn{x})  \psubstp{Q}{P}       
%   := 
%   \left\{ 
%     \begin{array}{ccc} 
%       \dropn{\quotep{Q}} & & x \nameeq \quotep{P} \\
%       \dropn{x} & & otherwise \\
%     \end{array}
%   \right. 
  (\dropn{x})  \psubstp{Q}{P}       
  := 
  \left\{ 
    \begin{array}{ccc} 
      Q & & x \nameeq \quotep{P} \\
      \dropn{x} & & otherwise \\
    \end{array}
  \right.
\end{mathpar}
 

where

\begin{eqnarray}
  (x)\id{\{} \lpquote Q \rpquote / \lpquote P \rpquote \id{\}}            = 
  \left\{ 
    \begin{array}{ccc}
      \lpquote Q \rpquote & & x \nameeq \lpquote P \rpquote \\
      x & & otherwise \\
    \end{array}
  \right. \nonumber
\end{eqnarray}

and $z$ is chosen distinct from $\quotep{P}$, $\quotep{Q}$, the free
names in $Q$, and all the names in $R$. Our $\alpha$-equivalence will
be built in the standard way from this substitution.

\begin{remark}\label{rem:no_self_referential_names}
  One consequence of these definitions is that $\forall P. \quotep{P}
  \not\in \freenames{P}$.
\end{remark}

\subsection{ Dynamic quote: an example }

Anticipating something of what's to come, consider applying the
substitution, $\widehat{\id{\{}u / z \id{\}}}$, to the following pair
of processes, $\lift{w}{y!(z)}$ and $w[ \lpquote y!(z) \rpquote ]$.

\begin{eqnarray}
	\lift{w}{y!(z)}\widehat{\id{\{}u / z \id{\}}}
		& = &
		\lift{w}{y!(u)} \nonumber\\
	w[ \lpquote y!(z) \rpquote ] \widehat{ \id{\{}u / z \id{\}} }
		& = &
		w[ \lpquote y!(z) \rpquote ] \nonumber
\end{eqnarray}

Because the body of the process between quotes is impervious to
substitution, we get radically different answers. In fact, by
examining the first process in an input context,
e.g. $x?(z).\lift{w}{y!(z)}$, we see that the process under the lift
operator may be shaped by prefixed inputs binding a name inside it. In
this sense, the lift operator will be seen as a way to dynamically
construct processes before reifying them as names.

Finally equipped with these standard features we can present the
dynamics of the calculus.

\subsubsection{Operational semantics} 

Finally, we introduce the computational dynamics. What marks these
algebras as distinct from other more traditionally studied algebraic
structures, e.g. vector spaces or polynomial rings, is the manner in
which dynamics is captured. In traditional structures, dynamics is typically
expressed through morphisms between such structures, as in linear maps
between vector spaces or morphisms between rings. In algebras
associated with the semantics of computation, the dynamics is
expressed as part of the algebraic structure itself, through a
reduction reduction relation typically denoted by $\red$. Below, we
give a recursive presentation of this relation for the calculus used
in the encoding.

$\red \subseteq \pi \times \pi$
$\red : \pi \to \mathcal{P}(\pi)$

\begin{mathpar}
  \inferrule* [lab=Comm] { \textsf{match}( x_{src}, x_{trgt} ) } { x_{trgt}?(y)P \; | \; x_{src}!\langle {Q} \rangle \red P\{\quotep{Q}/y}\} }
  \and \\
  \inferrule* [lab=Par] {{P} \red {P}'} {{{P} | {Q}} \red {{P}' | {Q}}}
  \and
  \inferrule* [lab=Equiv]{{{P} \scong {P}'} \andalso {{P}' \red {Q}'} \andalso {{Q}' \scong {Q}}}{{P} \red {Q}}
\end{mathpar}

\begin{eqnarray*}
  match_{\equiv} (\quotep{P},\quotep{Q}) & := & P \equiv Q \\
  match_{\dagger}(\quotep{P},\quotep{Q}) & := & \forall R. P|Q \red^{*} R => R \red^{*} 0 \\
  match_{K}(\quotep{P},\quotep{Q}) & := & K \mbox{ for some context } K
\end{eqnarray*}

$u?(x)P | u!\langle Q \rangle \red P\{\quotep{Q}/x\}$

%We write $\wred$ for $\red^*$, and $P\red$ if $\exists Q $ such that $ P \red Q$.
We write $P\red$ if $\exists Q $ such that $ P \red Q$ and $P\not\red$, otherwise.

\section{Replication}

As mentioned before, it is known that replication (and hence
recursion) can be implemented in a higher-order process algebra
\cite{SangiorgiWalker}. As our first example of calculation with the
machinery thus far presented we give the construction explicitly in
the {\rhoc}.

\begin{eqnarray}
	D_{x} & := & \prefix{x}{y}{(\binpar{\outputp{x}{y}}{@{y}})} \nonumber\\
	\bangp_{x}{P} & := & \binpar{{x}!\langle{\binpar{D_{x}}{P}}\rangle}{D_{x}} \nonumber
\end{eqnarray}

\begin{eqnarray}
	\bangp_{x}{P} & & \nonumber\\
	=
	& {x}!\langle{(\prefix{x}{y}{(\outputp{x}{y} | @{y})) | P}}\rangle 
	      | \prefix{x}{y}{(\outputp{x}{y} | @{y})} & \nonumber\\
	\red
	& (\outputp{x}{y} | @{y})\substn{\quotep{(\prefix{x}{y}{(@{y} | \outputp{x}{y})) | P}}}{y} & \nonumber\\
	=
	& \outputp{x}{\quotep{(\prefix{x}{y}{(\outputp{x}{y} | @{y})) | P}}}
	  | {(\prefix{x}{y}{(\outputp{x}{y} | @{y})) | P}} & \nonumber\\
	\red
	& \ldots & \nonumber\\
	\red^*
	& P | P | \ldots & \nonumber
\end{eqnarray}

Of course, this encoding, as an implementation, runs away, unfolding
$\bangp{P}$ eagerly. A lazier and more implementable replication
operator, restricted to input-guarded processes, may be obtained as follows.

\begin{eqnarray}
\bangp{\prefix{u}{v}{P}} 
	:= 
	\binpar{\lift{x}{\prefix{u}{v}{(\binpar{D(x)}{P})}}}{D(x)} \nonumber
\end{eqnarray}

\begin{remark}
  Note that the lazier definition still does not deal with summation
  or mixed summation (i.e. sums over input and output). The reader is
  invited to construct definitions of replication that deal with these
  features. 

  Further, the definitions are parameterized in a name, $x$. Can you,
  gentle reader, make a definition that eliminates this parameter and
  guarantees no accidental interaction between the replication
  machinery and the process being replicated -- i.e. no accidental
  sharing of names used by the process to get its work done and the
  name(s) used by the replication to effect copying. This latter
  revision of the definition of replication is crucial to obtaining
  the expected identity $!!P \sim !P$.
\end{remark}

\begin{remark}\label{rem:paradoxical_combinator}
  The reader familiar with the lambda calculus will have noticed the
  similarity between $D$ and the paradoxical combinator.

  [Ed. note: the existence of this seems to suggest we have to be more
  restrictive on the set of processes and names we admit if we are to
  support no-cloning.]
\end{remark}

\subsubsection{Bisimulation}

The computational dynamics gives rise to another kind of equivalence,
the equivalence of computational behavior. As previously mentioned
this is typically captured \emph{via} some form of bisimulation.

% The notion we use in this paper is weak barbed bisimulation
% \cite{milner91polyadicpi}.

The notion we use in this paper is derived from weak barbed
bisimulation \cite{milner91polyadicpi}. 

\begin{definition}
An \emph{observation relation}, $\downarrow_{\mathcal N}$, over a set
of names, $\mathcal N$, is the smallest relation satisfying the rules
below.

\infrule[Out-barb]{y \in {\mathcal N}, \; x \nameeq y}
		  {\outputp{x}{v} \downarrow_{\mathcal N} x}
\infrule[Par-barb]{\mbox{$P\downarrow_{\mathcal N} x$ or $Q\downarrow_{\mathcal N} x$}}
		  {\binpar{P}{Q} \downarrow_{\mathcal N} x}

We write $P \Downarrow_{\mathcal N} x$ if there is $Q$ such that 
$P \wred Q$ and $Q \downarrow_{\mathcal N} x$.
\end{definition}

\begin{definition}
%\label{def.bbisim}
An  ${\mathcal N}$-\emph{barbed bisimulation} over a set of names, ${\mathcal N}$, is a symmetric binary relation 
${\mathcal S}_{\mathcal N}$ between agents such that $P\rel{S}_{\mathcal N}Q$ implies:
\begin{enumerate}
\item If $P \red P'$ then $Q \wred Q'$ and $P'\rel{S}_{\mathcal N} Q'$.
\item If $P\downarrow_{\mathcal N} x$, then $Q\Downarrow_{\mathcal N} x$.
\end{enumerate}
$P$ is ${\mathcal N}$-barbed bisimilar to $Q$, written
$P \wbbisim_{\mathcal N} Q$, if $P \rel{S}_{\mathcal N} Q$ for some ${\mathcal N}$-barbed bisimulation ${\mathcal S}_{\mathcal N}$.
\end{definition}

$\mathcal{R} \subseteq \pi \times \pi$

$P \mathcal{R} Q => \forall P'. P \red P' \Rightarrow \exists Q'. Q \red Q', P' \mathcal{R} Q'$

$P \vdash x \Rightarrow Q \vdash x$

\begin{mathpar}
  \inferrule*[lab=Out-barb]{x \nameeq y}{{y}!\langle{Q}\rangle \vdash x}
  \and
  \inferrule*[lab=Par-barb]{\mbox{$P\vdash x$ or $Q\vdash x$}}{\binpar{P}{Q} \vdash x}
\end{mathpar}

\subsubsection{Contexts}

One of the principle advantages of computational calculi like the
$\pi$-calculus is a well-defined notion of context,
contextual-equivalence and a correlation between
contextual-equivalence and notions of bisimulation. The notion of
context allows the decomposition of a process into (sub-)process and
its syntactic environment, its context. Thus, a context may be
thought of as a process with a ``hole'' (written $\Box$) in it. The
application of a context $M$ to a process $P$, written $M[P]$, is
tantamount to filling the hole in $M$ with $P$. In this paper we do
not need the full weight of this theory, but do make use of the notion
of context in the proof the main theorem. 

\begin{mathpar}
  \inferrule* [lab=summation] {} {{M_{M},M_{N}} \bc \Box \;|\; x.M_{A} \;|\; M_{M}+M_{N}}
  \and
  \inferrule* [lab=agent] {} {{M_{A}} \bc (\vec{x})M_{P} \;| \; \clift{P_0,\ldots,M_{P},\ldots,P_N}}
  \and \\
  \inferrule* [lab=process] {} {{M_{P}} \bc M_{N} \;| \;P|M_{P} }
\end{mathpar} 

\begin{mathpar}
  \inferrule* [lab=sychronization] {} {M_{N} \bc \Box \;|\; x?M_{F} \;|\; x!M_{C}}
  \and
  \inferrule* [lab=abstraction] {} {{M_{F}} \bc (x)M_{P} }
  \and
  \inferrule* [lab=concretion] {} {{M_{C}} \bc \langle M_{P} \rangle }
  \and \\
  \inferrule* [lab=process] {} {{M_{P}} \bc M_{N} \;| \;P|M_{P} }
\end{mathpar}

\begin{definition}[contextual application] Given a context $M$, and
  process $P$, we define the \emph{contextual application}, $M[P] :=
  M\{P/\Box\}$. That is, the contextual application of M to P is the
  substitution of $P$ for $\Box$ in $M$.
\end{definition}

$\meaningof{-} : L \to \mathcal{P}(\pi)$

\begin{mathpar}
  \inferrule* [lab=collection] {} {\meaningof{true} = \pi, \and \meaningof{~E} = \pi \setminus \meaningof{E}, \and \meaningof{E_{1} \& E_{2}} = \meaningof{E_{1}} \cap \meaningof{E_{2}}}
\end{mathpar}

\begin{mathpar}
  \inferrule* [lab=structure] {} {\meaningof{0} = \{ P \in \pi | P \equiv 0 \}, \and \\ \meaningof{E_1 | E_2} = \{ P \in \pi | P \equiv P_{1} | P_{2}, P_{1} \in \meaningof{E_{1}}, P_{2} \in \meaningof{E_2}\} }
\end{mathpar}

\begin{mathpar}
 \inferrule* [lab=behavior] {} {\meaningof{\langle a?b \rangle E} = \{ P \in \pi | P \equiv Q | u?(y)P', \\ \and \\\\ \and \\ \;\;\; u \in \meaningof{a}, \forall z.P'\{z/y\} \in \meaningof{E\{z/b\}}\}, \and \\ \meaningof{a!E} = \{ P \in \pi | P \equiv Q | x!\langle P' \rangle, x \in \meaningof{a} P' \in \meaningof{E}\} }
\end{mathpar}

\begin{mathpar}
 \inferrule* [lab=nominal] {} {\meaningof{\quotep{E}} = \{ \quotep{P} \in \quotep{\pi} | P \in \meaningof{E} \}, \and \meaningof{\quotep{P}} = \{ \quotep{Q} \in \quotep{\pi} | P \equiv Q \} \and \\ \meaningof{@\quotep{E}} = \{ P \in \pi | P \equiv @x, x \in \meaningof{E} \}}
\end{mathpar}

\begin{eqnarray*}
  \\
  \meaningof{-} : TS \to ST
\end{eqnarray*}

\begin{eqnarray*}
  \\
  L : TS \to ST
\end{eqnarray*}

\begin{eqnarray*}
  \\
  P \models E \iff P \in \meaningof{E}
\end{eqnarray*}

\begin{eqnarray*}
  P \approx_{L} Q \iff \forall E \in L. P \models E \iff Q \models E
\end{eqnarray*}

\begin{eqnarray*}
  P \approx_{K} Q
\end{eqnarray*}

\begin{eqnarray*}
  P \approx Q
\end{eqnarray*}

$\approx_{K} = \approx = \approx_{L}$

\subsubsection{Contextual duality}

Note that contexts extend the quotation operation to a family of
operations from processes to names. Given a context, $M$, we can
define a \emph{nominal context}, $\quotep{M}$ by $\quotep{M}[P] :=
\quotep{M[P]}$. To foreshadow what is to come we observe that these
operations enjoy a duality with processes very much like the duality
between vectors and maps from vectors to scalars.

Further, because the calculus is essentially higher-order, we have a
correspondence between contexts and processes. More specifically,
given a name $x$ and a context $M$ we can construct $M^{*}_{x}$ such
that 

\begin{mathpar}
  M^{*}_{x} | \lift{x}{P} \red M[P]
\end{mathpar}

namely,

\begin{mathpar}
  M^{*}_{x} := x?(u).M[\dropn{u}]
\end{mathpar}

The dependence of $M^{*}_{x}$ on a name makes it an abstraction, 

\begin{mathpar}
  M^{*} := (x)x?(u).M[\dropn{u}]
\end{mathpar}

\subsection{Additional notation}

It will sometimes be convenient to denote the process a name
quotes. We already have the notation $x = \quotep{P}$, but it will be
convenient to introduce an alternate notation, $\procn{x}$, when we
want to emphasize the connection to the use of the name. Note that, by
virtue of name equivalence, $\quotep{\procn{x}} \nameeq x$; so, the
notation is consistent with previous definitions.

Further, because names have structure it is possible to effect
substitutions on the basis of that structure. This means we need to
upgrade our notation for substitutions, which we accomplish by
adapting comprehension notation. Thus,

\begin{mathpar}
  P\{ y / x : x \in S \}
\end{mathpar}

is interpreted to mean the process derived from P by replacing (in a
capture-avoiding manner) each occurrence of $x$ in $S$ by $y$. For example,

\begin{mathpar}
  P\{ \quotep{\procn{x}|\procn{x}} / x : x \in \freenames{P} \}
\end{mathpar}

will replace each (occurrence) of a free name $x$ in $P$ by
$\quotep{\procn{x}|\procn{x}}$.

Also, we will avail ourselves of the notation $x^{L}$ and $x^{R}$ to
denote injections of a name into disjoint copies of the name
space. There are numerous ways to accomplish this. One example can be
found in \cite{MeredithR05}. This notation overloads to vectors of
names: $\vec{x}^{\pi} := (x_{i}^{\pi} \; : \; 0 \leq i < |\vec{x}| )$ where $\pi \in \{L,R\}$.

We also use $P^{\Box} := P|\Box$.

In \cite{MeredithR05} an interpretation of the new operator is
given. It turns out that there are several possible interpretations
all enjoying the requisite algebraic properties of the operator (see
\cite{milner91polyadicpi}). We will therefore make liberal use of
$(\nu\; \vec{x})P$.

% subsection the_syntax_and_semantics_of_the_notation_system (end)   

\input{qm2pi.qmops} 

\input{qm2pi.sterngerlach} 

\input{qm2pi.metric} 

% section concurrent_process_calculi (end)

%\input{qm2pi.proofsketch}

% section proof sketch (end)

%\input{qm2pi.slviaknots} 

% section spatial logic via knots (end)

\input{qm2pi.conclusion}

% section conclusion (end)

%\input{qm2pi.dtcodes} 

% section wiring algorithm (end)

\input{qm2pi.ack} 

% section acknowledgments (end)

\newpage


\bibliographystyle{plain}   
\bibliography{../../biblios/main.bib}

\input{qm2pi.rhodetails}

\end{document}

 

% section wiring algorithm (end)

\documentclass[12pt]{llncs}
%\documentclass{jktr}

\usepackage[pdftex]{hyperref}                   
\usepackage {listings}
\usepackage {mathpartir}
\usepackage{bcprules}
%\usepackage{listings}
                       
\usepackage{graphicx} 
%\usepackage[margins=2.5cm,nohead,nofoot]{geometry}
%\usepackage{geometry}
\usepackage{amsfonts}
\usepackage{amstext}
\usepackage{latexsym}
\usepackage{amssymb}
\usepackage{color}


%\include{myPreamble}
\include{qm2pi.local} 

%\ifpdf
%\usepackage[pdftex]{graphicx}
%\else
%\usepackage{graphicx}
%\fi

 % \ifpdf
%  \usepackage{pdfsync}
%  \if


%\title{Brief Article}
%\author{David F. Snyder}
%\author{L.G. Meredith}

%\address{Dept. of Math., Texas State University--San Marcos, San Marcos, TX 78666}
       
\pagestyle{empty}


\begin{document}

\lstset{language=[Objective]Caml,frame=shadowbox}

\input{qm2pi.front}

% section front matter (end)

\input{qm2pi.intro} 
 
% section introduction (end)

% \input{qm2pi.knotations} 

% section notation (end)

\input{qm2pi.process.calculi} 

% section concurrent_process_calculi_and_spatial_logics_ (end)
    
%\input{qm2pi.knots2pi} 

%\input{qm2pi.trefoil} 

%\input{qm2pi.mainthm} 

% subsection basic_interpretation (end)

%\input{qm2pi.rho.presentation} 
\subsection{The syntax and semantics of the notation system}\label{sub:the_syntax_and_semantics_of_the_notation_system} % (fold)

We now summarize a technical presentation of the calculus that
embodies our theory of dynamics. The typical presentation of such a
calculus follows the style of giving generators and relations on
them. The grammar, below, describing term constructors, freely
generates the set of processes, $\Proc$. This set is then quotiented
by a relation known as structural congruence and it is over this set
that the notion of dynamics is expressed. This presentation is
essentially that of \cite{MeredithR05} with the addition of
polyadicity and summation. For readability we have relegated some of
the technical subtleties to an appendix.

\subsubsection{Process grammar}\label{subsub:process_grammar}

\begin{mathpar}
  \inferrule* [lab=synchronization] {} {{M} \bc \pzero \;|\; x?F \;|\; x!C }
  \and
  \inferrule* [lab=abstraction] {} {{F} \bc (x)P}
  \and
  \inferrule* [lab=concretion] {} {{C} \bc \langle Q \rangle}
  \and
  \inferrule* [lab=process] {} {{P,Q} \bc M \;| \;P|Q \;|\; @{x}}
  \and
  \inferrule* [lab=name] {} {{x} \bc \quotep{P}}
\end{mathpar} 

Note that $\vec{x}$ (resp. $\vec{P}$) denotes a vector of names
(resp. processes) of length $|\vec{x}|$ (resp. $|\vec{P}|$). We adopt
the following useful abbreviations.

\begin{mathpar}
   x?(\vec{y}).P := x.(\vec{y})P \and  x\clift{\vec{P}} := x.\clift{\vec{P}}
   \and x!(y) := \lift{x}{\dropn{y}}
   \and \Pi_{i=0}^{n-1}P_i := P_0 | \ldots | P_{n-1}
\end{mathpar}

\subsubsection{Structural congruence}

\paragraph{Free and bound names and alpha-equivalence.} At the
core of structural equivalence is alpha-equivalence which identifies
process that are the same up to a change of variable. Formally, we
recognize the distinction between free and bound names. The free names
of a process, $\freenames{P}$, may be calculated recursively as
follows:

\begin{mathpar}
\freenames{\pzero} := \emptyset
  \and \\
  \freenames{x?(y).P} := \{ x \} \cup (\freenames{P} \setminus \{ y \})
  \and 
  \freenames{x!\langle P \rangle} := \{ x \} \cup \{ P \} 
  \and \\
  \freenames{P|Q} := \freenames{P} \cup \freenames{Q}
  \and \\
  \freenames{@{x}} := \{ x \}
\end{mathpar}

$\pi$
$\quotep{\pi}$

$\freenames{-} : \pi \to \mathcal{P}(\quotep{\pi})$

\begin{eqnarray*}
  \freenames{\pzero} & := & \emptyset \\
  \freenames{x?(y).P} & := & \{ x \} \cup (\freenames{P} \setminus \{ y \}) \\
  \freenames{x!\langle P \rangle} & := & \{ x \} \cup \{ P \} \\
  \freenames{P|Q} & := & \freenames{P} \cup \freenames{Q} \\
  \freenames{\dropn{x}} & := & \{ x \}
\end{eqnarray*}

The bound names of a process, $\boundnames{P}$, are those names occurring in $P$
that are not free. For example, in $x?(y).0$, the name $x$ is free, while $y$ is bound.

\begin{mathpar}
  \inferrule* [lab=monoidal-laws] {} { P|Q \equiv Q|P \and P|0 \equiv P \and P|(Q|R) \equiv (P|Q)|R }
\end{mathpar}

\begin{mathpar}
  \inferrule* [lab=alpha-equivalence] {} { (x)P \equiv (y)P\{y/x\} \and y \not\in \freenames{P} }
\end{mathpar}

\begin{definition}
Then two processes, $P,Q$, are alpha-equivalent if $P = Q\{\vec{y}/\vec{x}\}$ for
some $\vec{x} \in \boundnames{Q},\vec{y} \in \boundnames{P}$, where $Q\{\vec{y}/\vec{x}\}$
denotes the capture-avoiding substitution of $\vec{y}$ for $\vec{x}$ in $Q$.
\end{definition}

\begin{definition}
  The {\em structural congruence} \cite{SangiorgiWalker} , $\equiv$,
  between processes is the least congruence containing
  alpha-equivalence, satisfying the abelian monoid laws
  (associativity, commutativity and $\pzero$ as identity) for parallel
  composition $|$ and for summation $+$.
\end{definition}

\subsection{Name equivalence}

We take name equivalence, written $\nameeq$, to be the smallest
equivalence relation generated by the following rules.

\begin{mathpar}
\inferrule*[lab=Quote-drop]
{ }
{ \quotep{@{x}} \nameeq x }

\inferrule*[lab=Struct-equiv]
{ P \scong Q }
{ \quotep{P} \nameeq \quotep{Q} }
\end{mathpar}

The astute reader will have noticed that the mutual recursion of names
and processes imposes a mutual recursion on alpha-equivalence and
structural equivalence via name-equivalence. Fortunately, all of this
works out pleasantly and we may calculate in the natural way, free of
concern. The reader interested in the details is referred to the
appendix \ref{appendix:rho_details}.

\subsection{Substitution}

We use $\Proc$ for the set of processes, $\QProc$ for the set of
names, and $\id{\{}\vec{y} / \vec{x} \id{\}}$ to denote partial maps,
$s : \QProc \rightarrow \QProc$. A map, $s$ lifts, uniquely, to a map
on process terms, $\widehat{s} : \Proc \rightarrow \Proc$ by the
following equations.

\begin{mathpar}
  (0) \psubstp{Q}{P} := 0 \\
  (R \juxtap S) \psubstp{Q}{P}
  :=    
  (R)\psubstp{Q}{P} \juxtap (S) \psubstp{Q}{P} \\
  (x?(y).R) \psubstp{Q}{P}    
  :=    
  (x)\substp{Q}{P} (z)\concat( (R \psubstn{z}{y}) \psubstp{Q}{P} ) \\
  (\lift{x}{R}) \psubstp{Q}{P}  
  :=
  \lift{(x)\substp{Q}{P}}{ R \psubstp{Q}{P} } \\
%   (\dropn{x})  \psubstp{Q}{P}       
%   := 
%   \left\{ 
%     \begin{array}{ccc} 
%       \dropn{\quotep{Q}} & & x \nameeq \quotep{P} \\
%       \dropn{x} & & otherwise \\
%     \end{array}
%   \right. 
  (\dropn{x})  \psubstp{Q}{P}       
  := 
  \left\{ 
    \begin{array}{ccc} 
      Q & & x \nameeq \quotep{P} \\
      \dropn{x} & & otherwise \\
    \end{array}
  \right.
\end{mathpar}
 

where

\begin{eqnarray}
  (x)\id{\{} \lpquote Q \rpquote / \lpquote P \rpquote \id{\}}            = 
  \left\{ 
    \begin{array}{ccc}
      \lpquote Q \rpquote & & x \nameeq \lpquote P \rpquote \\
      x & & otherwise \\
    \end{array}
  \right. \nonumber
\end{eqnarray}

and $z$ is chosen distinct from $\quotep{P}$, $\quotep{Q}$, the free
names in $Q$, and all the names in $R$. Our $\alpha$-equivalence will
be built in the standard way from this substitution.

\begin{remark}\label{rem:no_self_referential_names}
  One consequence of these definitions is that $\forall P. \quotep{P}
  \not\in \freenames{P}$.
\end{remark}

\subsection{ Dynamic quote: an example }

Anticipating something of what's to come, consider applying the
substitution, $\widehat{\id{\{}u / z \id{\}}}$, to the following pair
of processes, $\lift{w}{y!(z)}$ and $w[ \lpquote y!(z) \rpquote ]$.

\begin{eqnarray}
	\lift{w}{y!(z)}\widehat{\id{\{}u / z \id{\}}}
		& = &
		\lift{w}{y!(u)} \nonumber\\
	w[ \lpquote y!(z) \rpquote ] \widehat{ \id{\{}u / z \id{\}} }
		& = &
		w[ \lpquote y!(z) \rpquote ] \nonumber
\end{eqnarray}

Because the body of the process between quotes is impervious to
substitution, we get radically different answers. In fact, by
examining the first process in an input context,
e.g. $x?(z).\lift{w}{y!(z)}$, we see that the process under the lift
operator may be shaped by prefixed inputs binding a name inside it. In
this sense, the lift operator will be seen as a way to dynamically
construct processes before reifying them as names.

Finally equipped with these standard features we can present the
dynamics of the calculus.

\subsubsection{Operational semantics} 

Finally, we introduce the computational dynamics. What marks these
algebras as distinct from other more traditionally studied algebraic
structures, e.g. vector spaces or polynomial rings, is the manner in
which dynamics is captured. In traditional structures, dynamics is typically
expressed through morphisms between such structures, as in linear maps
between vector spaces or morphisms between rings. In algebras
associated with the semantics of computation, the dynamics is
expressed as part of the algebraic structure itself, through a
reduction reduction relation typically denoted by $\red$. Below, we
give a recursive presentation of this relation for the calculus used
in the encoding.

$\red \subseteq \pi \times \pi$
$\red : \pi \to \mathcal{P}(\pi)$

\begin{mathpar}
  \inferrule* [lab=Comm] { \textsf{match}( x_{src}, x_{trgt} ) } { x_{trgt}?(y)P \; | \; x_{src}!\langle {Q} \rangle \red P\{\quotep{Q}/y}\} }
  \and \\
  \inferrule* [lab=Par] {{P} \red {P}'} {{{P} | {Q}} \red {{P}' | {Q}}}
  \and
  \inferrule* [lab=Equiv]{{{P} \scong {P}'} \andalso {{P}' \red {Q}'} \andalso {{Q}' \scong {Q}}}{{P} \red {Q}}
\end{mathpar}

\begin{eqnarray*}
  match_{\equiv} (\quotep{P},\quotep{Q}) & := & P \equiv Q \\
  match_{\dagger}(\quotep{P},\quotep{Q}) & := & \forall R. P|Q \red^{*} R => R \red^{*} 0 \\
  match_{K}(\quotep{P},\quotep{Q}) & := & K \mbox{ for some context } K
\end{eqnarray*}

$u?(x)P | u!\langle Q \rangle \red P\{\quotep{Q}/x\}$

%We write $\wred$ for $\red^*$, and $P\red$ if $\exists Q $ such that $ P \red Q$.
We write $P\red$ if $\exists Q $ such that $ P \red Q$ and $P\not\red$, otherwise.

\section{Replication}

As mentioned before, it is known that replication (and hence
recursion) can be implemented in a higher-order process algebra
\cite{SangiorgiWalker}. As our first example of calculation with the
machinery thus far presented we give the construction explicitly in
the {\rhoc}.

\begin{eqnarray}
	D_{x} & := & \prefix{x}{y}{(\binpar{\outputp{x}{y}}{@{y}})} \nonumber\\
	\bangp_{x}{P} & := & \binpar{{x}!\langle{\binpar{D_{x}}{P}}\rangle}{D_{x}} \nonumber
\end{eqnarray}

\begin{eqnarray}
	\bangp_{x}{P} & & \nonumber\\
	=
	& {x}!\langle{(\prefix{x}{y}{(\outputp{x}{y} | @{y})) | P}}\rangle 
	      | \prefix{x}{y}{(\outputp{x}{y} | @{y})} & \nonumber\\
	\red
	& (\outputp{x}{y} | @{y})\substn{\quotep{(\prefix{x}{y}{(@{y} | \outputp{x}{y})) | P}}}{y} & \nonumber\\
	=
	& \outputp{x}{\quotep{(\prefix{x}{y}{(\outputp{x}{y} | @{y})) | P}}}
	  | {(\prefix{x}{y}{(\outputp{x}{y} | @{y})) | P}} & \nonumber\\
	\red
	& \ldots & \nonumber\\
	\red^*
	& P | P | \ldots & \nonumber
\end{eqnarray}

Of course, this encoding, as an implementation, runs away, unfolding
$\bangp{P}$ eagerly. A lazier and more implementable replication
operator, restricted to input-guarded processes, may be obtained as follows.

\begin{eqnarray}
\bangp{\prefix{u}{v}{P}} 
	:= 
	\binpar{\lift{x}{\prefix{u}{v}{(\binpar{D(x)}{P})}}}{D(x)} \nonumber
\end{eqnarray}

\begin{remark}
  Note that the lazier definition still does not deal with summation
  or mixed summation (i.e. sums over input and output). The reader is
  invited to construct definitions of replication that deal with these
  features. 

  Further, the definitions are parameterized in a name, $x$. Can you,
  gentle reader, make a definition that eliminates this parameter and
  guarantees no accidental interaction between the replication
  machinery and the process being replicated -- i.e. no accidental
  sharing of names used by the process to get its work done and the
  name(s) used by the replication to effect copying. This latter
  revision of the definition of replication is crucial to obtaining
  the expected identity $!!P \sim !P$.
\end{remark}

\begin{remark}\label{rem:paradoxical_combinator}
  The reader familiar with the lambda calculus will have noticed the
  similarity between $D$ and the paradoxical combinator.

  [Ed. note: the existence of this seems to suggest we have to be more
  restrictive on the set of processes and names we admit if we are to
  support no-cloning.]
\end{remark}

\subsubsection{Bisimulation}

The computational dynamics gives rise to another kind of equivalence,
the equivalence of computational behavior. As previously mentioned
this is typically captured \emph{via} some form of bisimulation.

% The notion we use in this paper is weak barbed bisimulation
% \cite{milner91polyadicpi}.

The notion we use in this paper is derived from weak barbed
bisimulation \cite{milner91polyadicpi}. 

\begin{definition}
An \emph{observation relation}, $\downarrow_{\mathcal N}$, over a set
of names, $\mathcal N$, is the smallest relation satisfying the rules
below.

\infrule[Out-barb]{y \in {\mathcal N}, \; x \nameeq y}
		  {\outputp{x}{v} \downarrow_{\mathcal N} x}
\infrule[Par-barb]{\mbox{$P\downarrow_{\mathcal N} x$ or $Q\downarrow_{\mathcal N} x$}}
		  {\binpar{P}{Q} \downarrow_{\mathcal N} x}

We write $P \Downarrow_{\mathcal N} x$ if there is $Q$ such that 
$P \wred Q$ and $Q \downarrow_{\mathcal N} x$.
\end{definition}

\begin{definition}
%\label{def.bbisim}
An  ${\mathcal N}$-\emph{barbed bisimulation} over a set of names, ${\mathcal N}$, is a symmetric binary relation 
${\mathcal S}_{\mathcal N}$ between agents such that $P\rel{S}_{\mathcal N}Q$ implies:
\begin{enumerate}
\item If $P \red P'$ then $Q \wred Q'$ and $P'\rel{S}_{\mathcal N} Q'$.
\item If $P\downarrow_{\mathcal N} x$, then $Q\Downarrow_{\mathcal N} x$.
\end{enumerate}
$P$ is ${\mathcal N}$-barbed bisimilar to $Q$, written
$P \wbbisim_{\mathcal N} Q$, if $P \rel{S}_{\mathcal N} Q$ for some ${\mathcal N}$-barbed bisimulation ${\mathcal S}_{\mathcal N}$.
\end{definition}

$\mathcal{R} \subseteq \pi \times \pi$

$P \mathcal{R} Q => \forall P'. P \red P' \Rightarrow \exists Q'. Q \red Q', P' \mathcal{R} Q'$

$P \vdash x \Rightarrow Q \vdash x$

\begin{mathpar}
  \inferrule*[lab=Out-barb]{x \nameeq y}{{y}!\langle{Q}\rangle \vdash x}
  \and
  \inferrule*[lab=Par-barb]{\mbox{$P\vdash x$ or $Q\vdash x$}}{\binpar{P}{Q} \vdash x}
\end{mathpar}

\subsubsection{Contexts}

One of the principle advantages of computational calculi like the
$\pi$-calculus is a well-defined notion of context,
contextual-equivalence and a correlation between
contextual-equivalence and notions of bisimulation. The notion of
context allows the decomposition of a process into (sub-)process and
its syntactic environment, its context. Thus, a context may be
thought of as a process with a ``hole'' (written $\Box$) in it. The
application of a context $M$ to a process $P$, written $M[P]$, is
tantamount to filling the hole in $M$ with $P$. In this paper we do
not need the full weight of this theory, but do make use of the notion
of context in the proof the main theorem. 

\begin{mathpar}
  \inferrule* [lab=summation] {} {{M_{M},M_{N}} \bc \Box \;|\; x.M_{A} \;|\; M_{M}+M_{N}}
  \and
  \inferrule* [lab=agent] {} {{M_{A}} \bc (\vec{x})M_{P} \;| \; \clift{P_0,\ldots,M_{P},\ldots,P_N}}
  \and \\
  \inferrule* [lab=process] {} {{M_{P}} \bc M_{N} \;| \;P|M_{P} }
\end{mathpar} 

\begin{mathpar}
  \inferrule* [lab=sychronization] {} {M_{N} \bc \Box \;|\; x?M_{F} \;|\; x!M_{C}}
  \and
  \inferrule* [lab=abstraction] {} {{M_{F}} \bc (x)M_{P} }
  \and
  \inferrule* [lab=concretion] {} {{M_{C}} \bc \langle M_{P} \rangle }
  \and \\
  \inferrule* [lab=process] {} {{M_{P}} \bc M_{N} \;| \;P|M_{P} }
\end{mathpar}

\begin{definition}[contextual application] Given a context $M$, and
  process $P$, we define the \emph{contextual application}, $M[P] :=
  M\{P/\Box\}$. That is, the contextual application of M to P is the
  substitution of $P$ for $\Box$ in $M$.
\end{definition}

$\meaningof{-} : L \to \mathcal{P}(\pi)$

\begin{mathpar}
  \inferrule* [lab=collection] {} {\meaningof{true} = \pi, \and \meaningof{~E} = \pi \setminus \meaningof{E}, \and \meaningof{E_{1} \& E_{2}} = \meaningof{E_{1}} \cap \meaningof{E_{2}}}
\end{mathpar}

\begin{mathpar}
  \inferrule* [lab=structure] {} {\meaningof{0} = \{ P \in \pi | P \equiv 0 \}, \and \\ \meaningof{E_1 | E_2} = \{ P \in \pi | P \equiv P_{1} | P_{2}, P_{1} \in \meaningof{E_{1}}, P_{2} \in \meaningof{E_2}\} }
\end{mathpar}

\begin{mathpar}
 \inferrule* [lab=behavior] {} {\meaningof{\langle a?b \rangle E} = \{ P \in \pi | P \equiv Q | u?(y)P', \\ \and \\\\ \and \\ \;\;\; u \in \meaningof{a}, \forall z.P'\{z/y\} \in \meaningof{E\{z/b\}}\}, \and \\ \meaningof{a!E} = \{ P \in \pi | P \equiv Q | x!\langle P' \rangle, x \in \meaningof{a} P' \in \meaningof{E}\} }
\end{mathpar}

\begin{mathpar}
 \inferrule* [lab=nominal] {} {\meaningof{\quotep{E}} = \{ \quotep{P} \in \quotep{\pi} | P \in \meaningof{E} \}, \and \meaningof{\quotep{P}} = \{ \quotep{Q} \in \quotep{\pi} | P \equiv Q \} \and \\ \meaningof{@\quotep{E}} = \{ P \in \pi | P \equiv @x, x \in \meaningof{E} \}}
\end{mathpar}

\begin{eqnarray*}
  \\
  \meaningof{-} : TS \to ST
\end{eqnarray*}

\begin{eqnarray*}
  \\
  L : TS \to ST
\end{eqnarray*}

\begin{eqnarray*}
  \\
  P \models E \iff P \in \meaningof{E}
\end{eqnarray*}

\begin{eqnarray*}
  P \approx_{L} Q \iff \forall E \in L. P \models E \iff Q \models E
\end{eqnarray*}

\begin{eqnarray*}
  P \approx_{K} Q
\end{eqnarray*}

\begin{eqnarray*}
  P \approx Q
\end{eqnarray*}

$\approx_{K} = \approx = \approx_{L}$

\subsubsection{Contextual duality}

Note that contexts extend the quotation operation to a family of
operations from processes to names. Given a context, $M$, we can
define a \emph{nominal context}, $\quotep{M}$ by $\quotep{M}[P] :=
\quotep{M[P]}$. To foreshadow what is to come we observe that these
operations enjoy a duality with processes very much like the duality
between vectors and maps from vectors to scalars.

Further, because the calculus is essentially higher-order, we have a
correspondence between contexts and processes. More specifically,
given a name $x$ and a context $M$ we can construct $M^{*}_{x}$ such
that 

\begin{mathpar}
  M^{*}_{x} | \lift{x}{P} \red M[P]
\end{mathpar}

namely,

\begin{mathpar}
  M^{*}_{x} := x?(u).M[\dropn{u}]
\end{mathpar}

The dependence of $M^{*}_{x}$ on a name makes it an abstraction, 

\begin{mathpar}
  M^{*} := (x)x?(u).M[\dropn{u}]
\end{mathpar}

\subsection{Additional notation}

It will sometimes be convenient to denote the process a name
quotes. We already have the notation $x = \quotep{P}$, but it will be
convenient to introduce an alternate notation, $\procn{x}$, when we
want to emphasize the connection to the use of the name. Note that, by
virtue of name equivalence, $\quotep{\procn{x}} \nameeq x$; so, the
notation is consistent with previous definitions.

Further, because names have structure it is possible to effect
substitutions on the basis of that structure. This means we need to
upgrade our notation for substitutions, which we accomplish by
adapting comprehension notation. Thus,

\begin{mathpar}
  P\{ y / x : x \in S \}
\end{mathpar}

is interpreted to mean the process derived from P by replacing (in a
capture-avoiding manner) each occurrence of $x$ in $S$ by $y$. For example,

\begin{mathpar}
  P\{ \quotep{\procn{x}|\procn{x}} / x : x \in \freenames{P} \}
\end{mathpar}

will replace each (occurrence) of a free name $x$ in $P$ by
$\quotep{\procn{x}|\procn{x}}$.

Also, we will avail ourselves of the notation $x^{L}$ and $x^{R}$ to
denote injections of a name into disjoint copies of the name
space. There are numerous ways to accomplish this. One example can be
found in \cite{MeredithR05}. This notation overloads to vectors of
names: $\vec{x}^{\pi} := (x_{i}^{\pi} \; : \; 0 \leq i < |\vec{x}| )$ where $\pi \in \{L,R\}$.

We also use $P^{\Box} := P|\Box$.

In \cite{MeredithR05} an interpretation of the new operator is
given. It turns out that there are several possible interpretations
all enjoying the requisite algebraic properties of the operator (see
\cite{milner91polyadicpi}). We will therefore make liberal use of
$(\nu\; \vec{x})P$.

% subsection the_syntax_and_semantics_of_the_notation_system (end)   

\input{qm2pi.qmops} 

\input{qm2pi.sterngerlach} 

\input{qm2pi.metric} 

% section concurrent_process_calculi (end)

%\input{qm2pi.proofsketch}

% section proof sketch (end)

%\input{qm2pi.slviaknots} 

% section spatial logic via knots (end)

\input{qm2pi.conclusion}

% section conclusion (end)

%\input{qm2pi.dtcodes} 

% section wiring algorithm (end)

\input{qm2pi.ack} 

% section acknowledgments (end)

\newpage


\bibliographystyle{plain}   
\bibliography{../../biblios/main.bib}

\input{qm2pi.rhodetails}

\end{document}

 

% section acknowledgments (end)

\newpage


\bibliographystyle{plain}   
\bibliography{../../biblios/main.bib}

\documentclass[12pt]{llncs}
%\documentclass{jktr}

\usepackage[pdftex]{hyperref}                   
\usepackage {listings}
\usepackage {mathpartir}
\usepackage{bcprules}
%\usepackage{listings}
                       
\usepackage{graphicx} 
%\usepackage[margins=2.5cm,nohead,nofoot]{geometry}
%\usepackage{geometry}
\usepackage{amsfonts}
\usepackage{amstext}
\usepackage{latexsym}
\usepackage{amssymb}
\usepackage{color}


%\include{myPreamble}
\include{qm2pi.local} 

%\ifpdf
%\usepackage[pdftex]{graphicx}
%\else
%\usepackage{graphicx}
%\fi

 % \ifpdf
%  \usepackage{pdfsync}
%  \if


%\title{Brief Article}
%\author{David F. Snyder}
%\author{L.G. Meredith}

%\address{Dept. of Math., Texas State University--San Marcos, San Marcos, TX 78666}
       
\pagestyle{empty}


\begin{document}

\lstset{language=[Objective]Caml,frame=shadowbox}

\input{qm2pi.front}

% section front matter (end)

\input{qm2pi.intro} 
 
% section introduction (end)

% \input{qm2pi.knotations} 

% section notation (end)

\input{qm2pi.process.calculi} 

% section concurrent_process_calculi_and_spatial_logics_ (end)
    
%\input{qm2pi.knots2pi} 

%\input{qm2pi.trefoil} 

%\input{qm2pi.mainthm} 

% subsection basic_interpretation (end)

%\input{qm2pi.rho.presentation} 
\subsection{The syntax and semantics of the notation system}\label{sub:the_syntax_and_semantics_of_the_notation_system} % (fold)

We now summarize a technical presentation of the calculus that
embodies our theory of dynamics. The typical presentation of such a
calculus follows the style of giving generators and relations on
them. The grammar, below, describing term constructors, freely
generates the set of processes, $\Proc$. This set is then quotiented
by a relation known as structural congruence and it is over this set
that the notion of dynamics is expressed. This presentation is
essentially that of \cite{MeredithR05} with the addition of
polyadicity and summation. For readability we have relegated some of
the technical subtleties to an appendix.

\subsubsection{Process grammar}\label{subsub:process_grammar}

\begin{mathpar}
  \inferrule* [lab=synchronization] {} {{M} \bc \pzero \;|\; x?F \;|\; x!C }
  \and
  \inferrule* [lab=abstraction] {} {{F} \bc (x)P}
  \and
  \inferrule* [lab=concretion] {} {{C} \bc \langle Q \rangle}
  \and
  \inferrule* [lab=process] {} {{P,Q} \bc M \;| \;P|Q \;|\; @{x}}
  \and
  \inferrule* [lab=name] {} {{x} \bc \quotep{P}}
\end{mathpar} 

Note that $\vec{x}$ (resp. $\vec{P}$) denotes a vector of names
(resp. processes) of length $|\vec{x}|$ (resp. $|\vec{P}|$). We adopt
the following useful abbreviations.

\begin{mathpar}
   x?(\vec{y}).P := x.(\vec{y})P \and  x\clift{\vec{P}} := x.\clift{\vec{P}}
   \and x!(y) := \lift{x}{\dropn{y}}
   \and \Pi_{i=0}^{n-1}P_i := P_0 | \ldots | P_{n-1}
\end{mathpar}

\subsubsection{Structural congruence}

\paragraph{Free and bound names and alpha-equivalence.} At the
core of structural equivalence is alpha-equivalence which identifies
process that are the same up to a change of variable. Formally, we
recognize the distinction between free and bound names. The free names
of a process, $\freenames{P}$, may be calculated recursively as
follows:

\begin{mathpar}
\freenames{\pzero} := \emptyset
  \and \\
  \freenames{x?(y).P} := \{ x \} \cup (\freenames{P} \setminus \{ y \})
  \and 
  \freenames{x!\langle P \rangle} := \{ x \} \cup \{ P \} 
  \and \\
  \freenames{P|Q} := \freenames{P} \cup \freenames{Q}
  \and \\
  \freenames{@{x}} := \{ x \}
\end{mathpar}

$\pi$
$\quotep{\pi}$

$\freenames{-} : \pi \to \mathcal{P}(\quotep{\pi})$

\begin{eqnarray*}
  \freenames{\pzero} & := & \emptyset \\
  \freenames{x?(y).P} & := & \{ x \} \cup (\freenames{P} \setminus \{ y \}) \\
  \freenames{x!\langle P \rangle} & := & \{ x \} \cup \{ P \} \\
  \freenames{P|Q} & := & \freenames{P} \cup \freenames{Q} \\
  \freenames{\dropn{x}} & := & \{ x \}
\end{eqnarray*}

The bound names of a process, $\boundnames{P}$, are those names occurring in $P$
that are not free. For example, in $x?(y).0$, the name $x$ is free, while $y$ is bound.

\begin{mathpar}
  \inferrule* [lab=monoidal-laws] {} { P|Q \equiv Q|P \and P|0 \equiv P \and P|(Q|R) \equiv (P|Q)|R }
\end{mathpar}

\begin{mathpar}
  \inferrule* [lab=alpha-equivalence] {} { (x)P \equiv (y)P\{y/x\} \and y \not\in \freenames{P} }
\end{mathpar}

\begin{definition}
Then two processes, $P,Q$, are alpha-equivalent if $P = Q\{\vec{y}/\vec{x}\}$ for
some $\vec{x} \in \boundnames{Q},\vec{y} \in \boundnames{P}$, where $Q\{\vec{y}/\vec{x}\}$
denotes the capture-avoiding substitution of $\vec{y}$ for $\vec{x}$ in $Q$.
\end{definition}

\begin{definition}
  The {\em structural congruence} \cite{SangiorgiWalker} , $\equiv$,
  between processes is the least congruence containing
  alpha-equivalence, satisfying the abelian monoid laws
  (associativity, commutativity and $\pzero$ as identity) for parallel
  composition $|$ and for summation $+$.
\end{definition}

\subsection{Name equivalence}

We take name equivalence, written $\nameeq$, to be the smallest
equivalence relation generated by the following rules.

\begin{mathpar}
\inferrule*[lab=Quote-drop]
{ }
{ \quotep{@{x}} \nameeq x }

\inferrule*[lab=Struct-equiv]
{ P \scong Q }
{ \quotep{P} \nameeq \quotep{Q} }
\end{mathpar}

The astute reader will have noticed that the mutual recursion of names
and processes imposes a mutual recursion on alpha-equivalence and
structural equivalence via name-equivalence. Fortunately, all of this
works out pleasantly and we may calculate in the natural way, free of
concern. The reader interested in the details is referred to the
appendix \ref{appendix:rho_details}.

\subsection{Substitution}

We use $\Proc$ for the set of processes, $\QProc$ for the set of
names, and $\id{\{}\vec{y} / \vec{x} \id{\}}$ to denote partial maps,
$s : \QProc \rightarrow \QProc$. A map, $s$ lifts, uniquely, to a map
on process terms, $\widehat{s} : \Proc \rightarrow \Proc$ by the
following equations.

\begin{mathpar}
  (0) \psubstp{Q}{P} := 0 \\
  (R \juxtap S) \psubstp{Q}{P}
  :=    
  (R)\psubstp{Q}{P} \juxtap (S) \psubstp{Q}{P} \\
  (x?(y).R) \psubstp{Q}{P}    
  :=    
  (x)\substp{Q}{P} (z)\concat( (R \psubstn{z}{y}) \psubstp{Q}{P} ) \\
  (\lift{x}{R}) \psubstp{Q}{P}  
  :=
  \lift{(x)\substp{Q}{P}}{ R \psubstp{Q}{P} } \\
%   (\dropn{x})  \psubstp{Q}{P}       
%   := 
%   \left\{ 
%     \begin{array}{ccc} 
%       \dropn{\quotep{Q}} & & x \nameeq \quotep{P} \\
%       \dropn{x} & & otherwise \\
%     \end{array}
%   \right. 
  (\dropn{x})  \psubstp{Q}{P}       
  := 
  \left\{ 
    \begin{array}{ccc} 
      Q & & x \nameeq \quotep{P} \\
      \dropn{x} & & otherwise \\
    \end{array}
  \right.
\end{mathpar}
 

where

\begin{eqnarray}
  (x)\id{\{} \lpquote Q \rpquote / \lpquote P \rpquote \id{\}}            = 
  \left\{ 
    \begin{array}{ccc}
      \lpquote Q \rpquote & & x \nameeq \lpquote P \rpquote \\
      x & & otherwise \\
    \end{array}
  \right. \nonumber
\end{eqnarray}

and $z$ is chosen distinct from $\quotep{P}$, $\quotep{Q}$, the free
names in $Q$, and all the names in $R$. Our $\alpha$-equivalence will
be built in the standard way from this substitution.

\begin{remark}\label{rem:no_self_referential_names}
  One consequence of these definitions is that $\forall P. \quotep{P}
  \not\in \freenames{P}$.
\end{remark}

\subsection{ Dynamic quote: an example }

Anticipating something of what's to come, consider applying the
substitution, $\widehat{\id{\{}u / z \id{\}}}$, to the following pair
of processes, $\lift{w}{y!(z)}$ and $w[ \lpquote y!(z) \rpquote ]$.

\begin{eqnarray}
	\lift{w}{y!(z)}\widehat{\id{\{}u / z \id{\}}}
		& = &
		\lift{w}{y!(u)} \nonumber\\
	w[ \lpquote y!(z) \rpquote ] \widehat{ \id{\{}u / z \id{\}} }
		& = &
		w[ \lpquote y!(z) \rpquote ] \nonumber
\end{eqnarray}

Because the body of the process between quotes is impervious to
substitution, we get radically different answers. In fact, by
examining the first process in an input context,
e.g. $x?(z).\lift{w}{y!(z)}$, we see that the process under the lift
operator may be shaped by prefixed inputs binding a name inside it. In
this sense, the lift operator will be seen as a way to dynamically
construct processes before reifying them as names.

Finally equipped with these standard features we can present the
dynamics of the calculus.

\subsubsection{Operational semantics} 

Finally, we introduce the computational dynamics. What marks these
algebras as distinct from other more traditionally studied algebraic
structures, e.g. vector spaces or polynomial rings, is the manner in
which dynamics is captured. In traditional structures, dynamics is typically
expressed through morphisms between such structures, as in linear maps
between vector spaces or morphisms between rings. In algebras
associated with the semantics of computation, the dynamics is
expressed as part of the algebraic structure itself, through a
reduction reduction relation typically denoted by $\red$. Below, we
give a recursive presentation of this relation for the calculus used
in the encoding.

$\red \subseteq \pi \times \pi$
$\red : \pi \to \mathcal{P}(\pi)$

\begin{mathpar}
  \inferrule* [lab=Comm] { \textsf{match}( x_{src}, x_{trgt} ) } { x_{trgt}?(y)P \; | \; x_{src}!\langle {Q} \rangle \red P\{\quotep{Q}/y}\} }
  \and \\
  \inferrule* [lab=Par] {{P} \red {P}'} {{{P} | {Q}} \red {{P}' | {Q}}}
  \and
  \inferrule* [lab=Equiv]{{{P} \scong {P}'} \andalso {{P}' \red {Q}'} \andalso {{Q}' \scong {Q}}}{{P} \red {Q}}
\end{mathpar}

\begin{eqnarray*}
  match_{\equiv} (\quotep{P},\quotep{Q}) & := & P \equiv Q \\
  match_{\dagger}(\quotep{P},\quotep{Q}) & := & \forall R. P|Q \red^{*} R => R \red^{*} 0 \\
  match_{K}(\quotep{P},\quotep{Q}) & := & K \mbox{ for some context } K
\end{eqnarray*}

$u?(x)P | u!\langle Q \rangle \red P\{\quotep{Q}/x\}$

%We write $\wred$ for $\red^*$, and $P\red$ if $\exists Q $ such that $ P \red Q$.
We write $P\red$ if $\exists Q $ such that $ P \red Q$ and $P\not\red$, otherwise.

\section{Replication}

As mentioned before, it is known that replication (and hence
recursion) can be implemented in a higher-order process algebra
\cite{SangiorgiWalker}. As our first example of calculation with the
machinery thus far presented we give the construction explicitly in
the {\rhoc}.

\begin{eqnarray}
	D_{x} & := & \prefix{x}{y}{(\binpar{\outputp{x}{y}}{@{y}})} \nonumber\\
	\bangp_{x}{P} & := & \binpar{{x}!\langle{\binpar{D_{x}}{P}}\rangle}{D_{x}} \nonumber
\end{eqnarray}

\begin{eqnarray}
	\bangp_{x}{P} & & \nonumber\\
	=
	& {x}!\langle{(\prefix{x}{y}{(\outputp{x}{y} | @{y})) | P}}\rangle 
	      | \prefix{x}{y}{(\outputp{x}{y} | @{y})} & \nonumber\\
	\red
	& (\outputp{x}{y} | @{y})\substn{\quotep{(\prefix{x}{y}{(@{y} | \outputp{x}{y})) | P}}}{y} & \nonumber\\
	=
	& \outputp{x}{\quotep{(\prefix{x}{y}{(\outputp{x}{y} | @{y})) | P}}}
	  | {(\prefix{x}{y}{(\outputp{x}{y} | @{y})) | P}} & \nonumber\\
	\red
	& \ldots & \nonumber\\
	\red^*
	& P | P | \ldots & \nonumber
\end{eqnarray}

Of course, this encoding, as an implementation, runs away, unfolding
$\bangp{P}$ eagerly. A lazier and more implementable replication
operator, restricted to input-guarded processes, may be obtained as follows.

\begin{eqnarray}
\bangp{\prefix{u}{v}{P}} 
	:= 
	\binpar{\lift{x}{\prefix{u}{v}{(\binpar{D(x)}{P})}}}{D(x)} \nonumber
\end{eqnarray}

\begin{remark}
  Note that the lazier definition still does not deal with summation
  or mixed summation (i.e. sums over input and output). The reader is
  invited to construct definitions of replication that deal with these
  features. 

  Further, the definitions are parameterized in a name, $x$. Can you,
  gentle reader, make a definition that eliminates this parameter and
  guarantees no accidental interaction between the replication
  machinery and the process being replicated -- i.e. no accidental
  sharing of names used by the process to get its work done and the
  name(s) used by the replication to effect copying. This latter
  revision of the definition of replication is crucial to obtaining
  the expected identity $!!P \sim !P$.
\end{remark}

\begin{remark}\label{rem:paradoxical_combinator}
  The reader familiar with the lambda calculus will have noticed the
  similarity between $D$ and the paradoxical combinator.

  [Ed. note: the existence of this seems to suggest we have to be more
  restrictive on the set of processes and names we admit if we are to
  support no-cloning.]
\end{remark}

\subsubsection{Bisimulation}

The computational dynamics gives rise to another kind of equivalence,
the equivalence of computational behavior. As previously mentioned
this is typically captured \emph{via} some form of bisimulation.

% The notion we use in this paper is weak barbed bisimulation
% \cite{milner91polyadicpi}.

The notion we use in this paper is derived from weak barbed
bisimulation \cite{milner91polyadicpi}. 

\begin{definition}
An \emph{observation relation}, $\downarrow_{\mathcal N}$, over a set
of names, $\mathcal N$, is the smallest relation satisfying the rules
below.

\infrule[Out-barb]{y \in {\mathcal N}, \; x \nameeq y}
		  {\outputp{x}{v} \downarrow_{\mathcal N} x}
\infrule[Par-barb]{\mbox{$P\downarrow_{\mathcal N} x$ or $Q\downarrow_{\mathcal N} x$}}
		  {\binpar{P}{Q} \downarrow_{\mathcal N} x}

We write $P \Downarrow_{\mathcal N} x$ if there is $Q$ such that 
$P \wred Q$ and $Q \downarrow_{\mathcal N} x$.
\end{definition}

\begin{definition}
%\label{def.bbisim}
An  ${\mathcal N}$-\emph{barbed bisimulation} over a set of names, ${\mathcal N}$, is a symmetric binary relation 
${\mathcal S}_{\mathcal N}$ between agents such that $P\rel{S}_{\mathcal N}Q$ implies:
\begin{enumerate}
\item If $P \red P'$ then $Q \wred Q'$ and $P'\rel{S}_{\mathcal N} Q'$.
\item If $P\downarrow_{\mathcal N} x$, then $Q\Downarrow_{\mathcal N} x$.
\end{enumerate}
$P$ is ${\mathcal N}$-barbed bisimilar to $Q$, written
$P \wbbisim_{\mathcal N} Q$, if $P \rel{S}_{\mathcal N} Q$ for some ${\mathcal N}$-barbed bisimulation ${\mathcal S}_{\mathcal N}$.
\end{definition}

$\mathcal{R} \subseteq \pi \times \pi$

$P \mathcal{R} Q => \forall P'. P \red P' \Rightarrow \exists Q'. Q \red Q', P' \mathcal{R} Q'$

$P \vdash x \Rightarrow Q \vdash x$

\begin{mathpar}
  \inferrule*[lab=Out-barb]{x \nameeq y}{{y}!\langle{Q}\rangle \vdash x}
  \and
  \inferrule*[lab=Par-barb]{\mbox{$P\vdash x$ or $Q\vdash x$}}{\binpar{P}{Q} \vdash x}
\end{mathpar}

\subsubsection{Contexts}

One of the principle advantages of computational calculi like the
$\pi$-calculus is a well-defined notion of context,
contextual-equivalence and a correlation between
contextual-equivalence and notions of bisimulation. The notion of
context allows the decomposition of a process into (sub-)process and
its syntactic environment, its context. Thus, a context may be
thought of as a process with a ``hole'' (written $\Box$) in it. The
application of a context $M$ to a process $P$, written $M[P]$, is
tantamount to filling the hole in $M$ with $P$. In this paper we do
not need the full weight of this theory, but do make use of the notion
of context in the proof the main theorem. 

\begin{mathpar}
  \inferrule* [lab=summation] {} {{M_{M},M_{N}} \bc \Box \;|\; x.M_{A} \;|\; M_{M}+M_{N}}
  \and
  \inferrule* [lab=agent] {} {{M_{A}} \bc (\vec{x})M_{P} \;| \; \clift{P_0,\ldots,M_{P},\ldots,P_N}}
  \and \\
  \inferrule* [lab=process] {} {{M_{P}} \bc M_{N} \;| \;P|M_{P} }
\end{mathpar} 

\begin{mathpar}
  \inferrule* [lab=sychronization] {} {M_{N} \bc \Box \;|\; x?M_{F} \;|\; x!M_{C}}
  \and
  \inferrule* [lab=abstraction] {} {{M_{F}} \bc (x)M_{P} }
  \and
  \inferrule* [lab=concretion] {} {{M_{C}} \bc \langle M_{P} \rangle }
  \and \\
  \inferrule* [lab=process] {} {{M_{P}} \bc M_{N} \;| \;P|M_{P} }
\end{mathpar}

\begin{definition}[contextual application] Given a context $M$, and
  process $P$, we define the \emph{contextual application}, $M[P] :=
  M\{P/\Box\}$. That is, the contextual application of M to P is the
  substitution of $P$ for $\Box$ in $M$.
\end{definition}

$\meaningof{-} : L \to \mathcal{P}(\pi)$

\begin{mathpar}
  \inferrule* [lab=collection] {} {\meaningof{true} = \pi, \and \meaningof{~E} = \pi \setminus \meaningof{E}, \and \meaningof{E_{1} \& E_{2}} = \meaningof{E_{1}} \cap \meaningof{E_{2}}}
\end{mathpar}

\begin{mathpar}
  \inferrule* [lab=structure] {} {\meaningof{0} = \{ P \in \pi | P \equiv 0 \}, \and \\ \meaningof{E_1 | E_2} = \{ P \in \pi | P \equiv P_{1} | P_{2}, P_{1} \in \meaningof{E_{1}}, P_{2} \in \meaningof{E_2}\} }
\end{mathpar}

\begin{mathpar}
 \inferrule* [lab=behavior] {} {\meaningof{\langle a?b \rangle E} = \{ P \in \pi | P \equiv Q | u?(y)P', \\ \and \\\\ \and \\ \;\;\; u \in \meaningof{a}, \forall z.P'\{z/y\} \in \meaningof{E\{z/b\}}\}, \and \\ \meaningof{a!E} = \{ P \in \pi | P \equiv Q | x!\langle P' \rangle, x \in \meaningof{a} P' \in \meaningof{E}\} }
\end{mathpar}

\begin{mathpar}
 \inferrule* [lab=nominal] {} {\meaningof{\quotep{E}} = \{ \quotep{P} \in \quotep{\pi} | P \in \meaningof{E} \}, \and \meaningof{\quotep{P}} = \{ \quotep{Q} \in \quotep{\pi} | P \equiv Q \} \and \\ \meaningof{@\quotep{E}} = \{ P \in \pi | P \equiv @x, x \in \meaningof{E} \}}
\end{mathpar}

\begin{eqnarray*}
  \\
  \meaningof{-} : TS \to ST
\end{eqnarray*}

\begin{eqnarray*}
  \\
  L : TS \to ST
\end{eqnarray*}

\begin{eqnarray*}
  \\
  P \models E \iff P \in \meaningof{E}
\end{eqnarray*}

\begin{eqnarray*}
  P \approx_{L} Q \iff \forall E \in L. P \models E \iff Q \models E
\end{eqnarray*}

\begin{eqnarray*}
  P \approx_{K} Q
\end{eqnarray*}

\begin{eqnarray*}
  P \approx Q
\end{eqnarray*}

$\approx_{K} = \approx = \approx_{L}$

\subsubsection{Contextual duality}

Note that contexts extend the quotation operation to a family of
operations from processes to names. Given a context, $M$, we can
define a \emph{nominal context}, $\quotep{M}$ by $\quotep{M}[P] :=
\quotep{M[P]}$. To foreshadow what is to come we observe that these
operations enjoy a duality with processes very much like the duality
between vectors and maps from vectors to scalars.

Further, because the calculus is essentially higher-order, we have a
correspondence between contexts and processes. More specifically,
given a name $x$ and a context $M$ we can construct $M^{*}_{x}$ such
that 

\begin{mathpar}
  M^{*}_{x} | \lift{x}{P} \red M[P]
\end{mathpar}

namely,

\begin{mathpar}
  M^{*}_{x} := x?(u).M[\dropn{u}]
\end{mathpar}

The dependence of $M^{*}_{x}$ on a name makes it an abstraction, 

\begin{mathpar}
  M^{*} := (x)x?(u).M[\dropn{u}]
\end{mathpar}

\subsection{Additional notation}

It will sometimes be convenient to denote the process a name
quotes. We already have the notation $x = \quotep{P}$, but it will be
convenient to introduce an alternate notation, $\procn{x}$, when we
want to emphasize the connection to the use of the name. Note that, by
virtue of name equivalence, $\quotep{\procn{x}} \nameeq x$; so, the
notation is consistent with previous definitions.

Further, because names have structure it is possible to effect
substitutions on the basis of that structure. This means we need to
upgrade our notation for substitutions, which we accomplish by
adapting comprehension notation. Thus,

\begin{mathpar}
  P\{ y / x : x \in S \}
\end{mathpar}

is interpreted to mean the process derived from P by replacing (in a
capture-avoiding manner) each occurrence of $x$ in $S$ by $y$. For example,

\begin{mathpar}
  P\{ \quotep{\procn{x}|\procn{x}} / x : x \in \freenames{P} \}
\end{mathpar}

will replace each (occurrence) of a free name $x$ in $P$ by
$\quotep{\procn{x}|\procn{x}}$.

Also, we will avail ourselves of the notation $x^{L}$ and $x^{R}$ to
denote injections of a name into disjoint copies of the name
space. There are numerous ways to accomplish this. One example can be
found in \cite{MeredithR05}. This notation overloads to vectors of
names: $\vec{x}^{\pi} := (x_{i}^{\pi} \; : \; 0 \leq i < |\vec{x}| )$ where $\pi \in \{L,R\}$.

We also use $P^{\Box} := P|\Box$.

In \cite{MeredithR05} an interpretation of the new operator is
given. It turns out that there are several possible interpretations
all enjoying the requisite algebraic properties of the operator (see
\cite{milner91polyadicpi}). We will therefore make liberal use of
$(\nu\; \vec{x})P$.

% subsection the_syntax_and_semantics_of_the_notation_system (end)   

\input{qm2pi.qmops} 

\input{qm2pi.sterngerlach} 

\input{qm2pi.metric} 

% section concurrent_process_calculi (end)

%\input{qm2pi.proofsketch}

% section proof sketch (end)

%\input{qm2pi.slviaknots} 

% section spatial logic via knots (end)

\input{qm2pi.conclusion}

% section conclusion (end)

%\input{qm2pi.dtcodes} 

% section wiring algorithm (end)

\input{qm2pi.ack} 

% section acknowledgments (end)

\newpage


\bibliographystyle{plain}   
\bibliography{../../biblios/main.bib}

\input{qm2pi.rhodetails}

\end{document}



\end{document}

 

% section acknowledgments (end)

\newpage


\bibliographystyle{plain}   
\bibliography{../../biblios/main.bib}

\documentclass[12pt]{llncs}
%\documentclass{jktr}

\usepackage[pdftex]{hyperref}                   
\usepackage {listings}
\usepackage {mathpartir}
\usepackage{bcprules}
%\usepackage{listings}
                       
\usepackage{graphicx} 
%\usepackage[margins=2.5cm,nohead,nofoot]{geometry}
%\usepackage{geometry}
\usepackage{amsfonts}
\usepackage{amstext}
\usepackage{latexsym}
\usepackage{amssymb}
\usepackage{color}


%\include{myPreamble}
\documentclass[12pt]{llncs}
%\documentclass{jktr}

\usepackage[pdftex]{hyperref}                   
\usepackage {listings}
\usepackage {mathpartir}
\usepackage{bcprules}
%\usepackage{listings}
                       
\usepackage{graphicx} 
%\usepackage[margins=2.5cm,nohead,nofoot]{geometry}
%\usepackage{geometry}
\usepackage{amsfonts}
\usepackage{amstext}
\usepackage{latexsym}
\usepackage{amssymb}
\usepackage{color}


%\include{myPreamble}
\include{qm2pi.local} 

%\ifpdf
%\usepackage[pdftex]{graphicx}
%\else
%\usepackage{graphicx}
%\fi

 % \ifpdf
%  \usepackage{pdfsync}
%  \if


%\title{Brief Article}
%\author{David F. Snyder}
%\author{L.G. Meredith}

%\address{Dept. of Math., Texas State University--San Marcos, San Marcos, TX 78666}
       
\pagestyle{empty}


\begin{document}

\lstset{language=[Objective]Caml,frame=shadowbox}

\input{qm2pi.front}

% section front matter (end)

\input{qm2pi.intro} 
 
% section introduction (end)

% \input{qm2pi.knotations} 

% section notation (end)

\input{qm2pi.process.calculi} 

% section concurrent_process_calculi_and_spatial_logics_ (end)
    
%\input{qm2pi.knots2pi} 

%\input{qm2pi.trefoil} 

%\input{qm2pi.mainthm} 

% subsection basic_interpretation (end)

%\input{qm2pi.rho.presentation} 
\subsection{The syntax and semantics of the notation system}\label{sub:the_syntax_and_semantics_of_the_notation_system} % (fold)

We now summarize a technical presentation of the calculus that
embodies our theory of dynamics. The typical presentation of such a
calculus follows the style of giving generators and relations on
them. The grammar, below, describing term constructors, freely
generates the set of processes, $\Proc$. This set is then quotiented
by a relation known as structural congruence and it is over this set
that the notion of dynamics is expressed. This presentation is
essentially that of \cite{MeredithR05} with the addition of
polyadicity and summation. For readability we have relegated some of
the technical subtleties to an appendix.

\subsubsection{Process grammar}\label{subsub:process_grammar}

\begin{mathpar}
  \inferrule* [lab=synchronization] {} {{M} \bc \pzero \;|\; x?F \;|\; x!C }
  \and
  \inferrule* [lab=abstraction] {} {{F} \bc (x)P}
  \and
  \inferrule* [lab=concretion] {} {{C} \bc \langle Q \rangle}
  \and
  \inferrule* [lab=process] {} {{P,Q} \bc M \;| \;P|Q \;|\; @{x}}
  \and
  \inferrule* [lab=name] {} {{x} \bc \quotep{P}}
\end{mathpar} 

Note that $\vec{x}$ (resp. $\vec{P}$) denotes a vector of names
(resp. processes) of length $|\vec{x}|$ (resp. $|\vec{P}|$). We adopt
the following useful abbreviations.

\begin{mathpar}
   x?(\vec{y}).P := x.(\vec{y})P \and  x\clift{\vec{P}} := x.\clift{\vec{P}}
   \and x!(y) := \lift{x}{\dropn{y}}
   \and \Pi_{i=0}^{n-1}P_i := P_0 | \ldots | P_{n-1}
\end{mathpar}

\subsubsection{Structural congruence}

\paragraph{Free and bound names and alpha-equivalence.} At the
core of structural equivalence is alpha-equivalence which identifies
process that are the same up to a change of variable. Formally, we
recognize the distinction between free and bound names. The free names
of a process, $\freenames{P}$, may be calculated recursively as
follows:

\begin{mathpar}
\freenames{\pzero} := \emptyset
  \and \\
  \freenames{x?(y).P} := \{ x \} \cup (\freenames{P} \setminus \{ y \})
  \and 
  \freenames{x!\langle P \rangle} := \{ x \} \cup \{ P \} 
  \and \\
  \freenames{P|Q} := \freenames{P} \cup \freenames{Q}
  \and \\
  \freenames{@{x}} := \{ x \}
\end{mathpar}

$\pi$
$\quotep{\pi}$

$\freenames{-} : \pi \to \mathcal{P}(\quotep{\pi})$

\begin{eqnarray*}
  \freenames{\pzero} & := & \emptyset \\
  \freenames{x?(y).P} & := & \{ x \} \cup (\freenames{P} \setminus \{ y \}) \\
  \freenames{x!\langle P \rangle} & := & \{ x \} \cup \{ P \} \\
  \freenames{P|Q} & := & \freenames{P} \cup \freenames{Q} \\
  \freenames{\dropn{x}} & := & \{ x \}
\end{eqnarray*}

The bound names of a process, $\boundnames{P}$, are those names occurring in $P$
that are not free. For example, in $x?(y).0$, the name $x$ is free, while $y$ is bound.

\begin{mathpar}
  \inferrule* [lab=monoidal-laws] {} { P|Q \equiv Q|P \and P|0 \equiv P \and P|(Q|R) \equiv (P|Q)|R }
\end{mathpar}

\begin{mathpar}
  \inferrule* [lab=alpha-equivalence] {} { (x)P \equiv (y)P\{y/x\} \and y \not\in \freenames{P} }
\end{mathpar}

\begin{definition}
Then two processes, $P,Q$, are alpha-equivalent if $P = Q\{\vec{y}/\vec{x}\}$ for
some $\vec{x} \in \boundnames{Q},\vec{y} \in \boundnames{P}$, where $Q\{\vec{y}/\vec{x}\}$
denotes the capture-avoiding substitution of $\vec{y}$ for $\vec{x}$ in $Q$.
\end{definition}

\begin{definition}
  The {\em structural congruence} \cite{SangiorgiWalker} , $\equiv$,
  between processes is the least congruence containing
  alpha-equivalence, satisfying the abelian monoid laws
  (associativity, commutativity and $\pzero$ as identity) for parallel
  composition $|$ and for summation $+$.
\end{definition}

\subsection{Name equivalence}

We take name equivalence, written $\nameeq$, to be the smallest
equivalence relation generated by the following rules.

\begin{mathpar}
\inferrule*[lab=Quote-drop]
{ }
{ \quotep{@{x}} \nameeq x }

\inferrule*[lab=Struct-equiv]
{ P \scong Q }
{ \quotep{P} \nameeq \quotep{Q} }
\end{mathpar}

The astute reader will have noticed that the mutual recursion of names
and processes imposes a mutual recursion on alpha-equivalence and
structural equivalence via name-equivalence. Fortunately, all of this
works out pleasantly and we may calculate in the natural way, free of
concern. The reader interested in the details is referred to the
appendix \ref{appendix:rho_details}.

\subsection{Substitution}

We use $\Proc$ for the set of processes, $\QProc$ for the set of
names, and $\id{\{}\vec{y} / \vec{x} \id{\}}$ to denote partial maps,
$s : \QProc \rightarrow \QProc$. A map, $s$ lifts, uniquely, to a map
on process terms, $\widehat{s} : \Proc \rightarrow \Proc$ by the
following equations.

\begin{mathpar}
  (0) \psubstp{Q}{P} := 0 \\
  (R \juxtap S) \psubstp{Q}{P}
  :=    
  (R)\psubstp{Q}{P} \juxtap (S) \psubstp{Q}{P} \\
  (x?(y).R) \psubstp{Q}{P}    
  :=    
  (x)\substp{Q}{P} (z)\concat( (R \psubstn{z}{y}) \psubstp{Q}{P} ) \\
  (\lift{x}{R}) \psubstp{Q}{P}  
  :=
  \lift{(x)\substp{Q}{P}}{ R \psubstp{Q}{P} } \\
%   (\dropn{x})  \psubstp{Q}{P}       
%   := 
%   \left\{ 
%     \begin{array}{ccc} 
%       \dropn{\quotep{Q}} & & x \nameeq \quotep{P} \\
%       \dropn{x} & & otherwise \\
%     \end{array}
%   \right. 
  (\dropn{x})  \psubstp{Q}{P}       
  := 
  \left\{ 
    \begin{array}{ccc} 
      Q & & x \nameeq \quotep{P} \\
      \dropn{x} & & otherwise \\
    \end{array}
  \right.
\end{mathpar}
 

where

\begin{eqnarray}
  (x)\id{\{} \lpquote Q \rpquote / \lpquote P \rpquote \id{\}}            = 
  \left\{ 
    \begin{array}{ccc}
      \lpquote Q \rpquote & & x \nameeq \lpquote P \rpquote \\
      x & & otherwise \\
    \end{array}
  \right. \nonumber
\end{eqnarray}

and $z$ is chosen distinct from $\quotep{P}$, $\quotep{Q}$, the free
names in $Q$, and all the names in $R$. Our $\alpha$-equivalence will
be built in the standard way from this substitution.

\begin{remark}\label{rem:no_self_referential_names}
  One consequence of these definitions is that $\forall P. \quotep{P}
  \not\in \freenames{P}$.
\end{remark}

\subsection{ Dynamic quote: an example }

Anticipating something of what's to come, consider applying the
substitution, $\widehat{\id{\{}u / z \id{\}}}$, to the following pair
of processes, $\lift{w}{y!(z)}$ and $w[ \lpquote y!(z) \rpquote ]$.

\begin{eqnarray}
	\lift{w}{y!(z)}\widehat{\id{\{}u / z \id{\}}}
		& = &
		\lift{w}{y!(u)} \nonumber\\
	w[ \lpquote y!(z) \rpquote ] \widehat{ \id{\{}u / z \id{\}} }
		& = &
		w[ \lpquote y!(z) \rpquote ] \nonumber
\end{eqnarray}

Because the body of the process between quotes is impervious to
substitution, we get radically different answers. In fact, by
examining the first process in an input context,
e.g. $x?(z).\lift{w}{y!(z)}$, we see that the process under the lift
operator may be shaped by prefixed inputs binding a name inside it. In
this sense, the lift operator will be seen as a way to dynamically
construct processes before reifying them as names.

Finally equipped with these standard features we can present the
dynamics of the calculus.

\subsubsection{Operational semantics} 

Finally, we introduce the computational dynamics. What marks these
algebras as distinct from other more traditionally studied algebraic
structures, e.g. vector spaces or polynomial rings, is the manner in
which dynamics is captured. In traditional structures, dynamics is typically
expressed through morphisms between such structures, as in linear maps
between vector spaces or morphisms between rings. In algebras
associated with the semantics of computation, the dynamics is
expressed as part of the algebraic structure itself, through a
reduction reduction relation typically denoted by $\red$. Below, we
give a recursive presentation of this relation for the calculus used
in the encoding.

$\red \subseteq \pi \times \pi$
$\red : \pi \to \mathcal{P}(\pi)$

\begin{mathpar}
  \inferrule* [lab=Comm] { \textsf{match}( x_{src}, x_{trgt} ) } { x_{trgt}?(y)P \; | \; x_{src}!\langle {Q} \rangle \red P\{\quotep{Q}/y}\} }
  \and \\
  \inferrule* [lab=Par] {{P} \red {P}'} {{{P} | {Q}} \red {{P}' | {Q}}}
  \and
  \inferrule* [lab=Equiv]{{{P} \scong {P}'} \andalso {{P}' \red {Q}'} \andalso {{Q}' \scong {Q}}}{{P} \red {Q}}
\end{mathpar}

\begin{eqnarray*}
  match_{\equiv} (\quotep{P},\quotep{Q}) & := & P \equiv Q \\
  match_{\dagger}(\quotep{P},\quotep{Q}) & := & \forall R. P|Q \red^{*} R => R \red^{*} 0 \\
  match_{K}(\quotep{P},\quotep{Q}) & := & K \mbox{ for some context } K
\end{eqnarray*}

$u?(x)P | u!\langle Q \rangle \red P\{\quotep{Q}/x\}$

%We write $\wred$ for $\red^*$, and $P\red$ if $\exists Q $ such that $ P \red Q$.
We write $P\red$ if $\exists Q $ such that $ P \red Q$ and $P\not\red$, otherwise.

\section{Replication}

As mentioned before, it is known that replication (and hence
recursion) can be implemented in a higher-order process algebra
\cite{SangiorgiWalker}. As our first example of calculation with the
machinery thus far presented we give the construction explicitly in
the {\rhoc}.

\begin{eqnarray}
	D_{x} & := & \prefix{x}{y}{(\binpar{\outputp{x}{y}}{@{y}})} \nonumber\\
	\bangp_{x}{P} & := & \binpar{{x}!\langle{\binpar{D_{x}}{P}}\rangle}{D_{x}} \nonumber
\end{eqnarray}

\begin{eqnarray}
	\bangp_{x}{P} & & \nonumber\\
	=
	& {x}!\langle{(\prefix{x}{y}{(\outputp{x}{y} | @{y})) | P}}\rangle 
	      | \prefix{x}{y}{(\outputp{x}{y} | @{y})} & \nonumber\\
	\red
	& (\outputp{x}{y} | @{y})\substn{\quotep{(\prefix{x}{y}{(@{y} | \outputp{x}{y})) | P}}}{y} & \nonumber\\
	=
	& \outputp{x}{\quotep{(\prefix{x}{y}{(\outputp{x}{y} | @{y})) | P}}}
	  | {(\prefix{x}{y}{(\outputp{x}{y} | @{y})) | P}} & \nonumber\\
	\red
	& \ldots & \nonumber\\
	\red^*
	& P | P | \ldots & \nonumber
\end{eqnarray}

Of course, this encoding, as an implementation, runs away, unfolding
$\bangp{P}$ eagerly. A lazier and more implementable replication
operator, restricted to input-guarded processes, may be obtained as follows.

\begin{eqnarray}
\bangp{\prefix{u}{v}{P}} 
	:= 
	\binpar{\lift{x}{\prefix{u}{v}{(\binpar{D(x)}{P})}}}{D(x)} \nonumber
\end{eqnarray}

\begin{remark}
  Note that the lazier definition still does not deal with summation
  or mixed summation (i.e. sums over input and output). The reader is
  invited to construct definitions of replication that deal with these
  features. 

  Further, the definitions are parameterized in a name, $x$. Can you,
  gentle reader, make a definition that eliminates this parameter and
  guarantees no accidental interaction between the replication
  machinery and the process being replicated -- i.e. no accidental
  sharing of names used by the process to get its work done and the
  name(s) used by the replication to effect copying. This latter
  revision of the definition of replication is crucial to obtaining
  the expected identity $!!P \sim !P$.
\end{remark}

\begin{remark}\label{rem:paradoxical_combinator}
  The reader familiar with the lambda calculus will have noticed the
  similarity between $D$ and the paradoxical combinator.

  [Ed. note: the existence of this seems to suggest we have to be more
  restrictive on the set of processes and names we admit if we are to
  support no-cloning.]
\end{remark}

\subsubsection{Bisimulation}

The computational dynamics gives rise to another kind of equivalence,
the equivalence of computational behavior. As previously mentioned
this is typically captured \emph{via} some form of bisimulation.

% The notion we use in this paper is weak barbed bisimulation
% \cite{milner91polyadicpi}.

The notion we use in this paper is derived from weak barbed
bisimulation \cite{milner91polyadicpi}. 

\begin{definition}
An \emph{observation relation}, $\downarrow_{\mathcal N}$, over a set
of names, $\mathcal N$, is the smallest relation satisfying the rules
below.

\infrule[Out-barb]{y \in {\mathcal N}, \; x \nameeq y}
		  {\outputp{x}{v} \downarrow_{\mathcal N} x}
\infrule[Par-barb]{\mbox{$P\downarrow_{\mathcal N} x$ or $Q\downarrow_{\mathcal N} x$}}
		  {\binpar{P}{Q} \downarrow_{\mathcal N} x}

We write $P \Downarrow_{\mathcal N} x$ if there is $Q$ such that 
$P \wred Q$ and $Q \downarrow_{\mathcal N} x$.
\end{definition}

\begin{definition}
%\label{def.bbisim}
An  ${\mathcal N}$-\emph{barbed bisimulation} over a set of names, ${\mathcal N}$, is a symmetric binary relation 
${\mathcal S}_{\mathcal N}$ between agents such that $P\rel{S}_{\mathcal N}Q$ implies:
\begin{enumerate}
\item If $P \red P'$ then $Q \wred Q'$ and $P'\rel{S}_{\mathcal N} Q'$.
\item If $P\downarrow_{\mathcal N} x$, then $Q\Downarrow_{\mathcal N} x$.
\end{enumerate}
$P$ is ${\mathcal N}$-barbed bisimilar to $Q$, written
$P \wbbisim_{\mathcal N} Q$, if $P \rel{S}_{\mathcal N} Q$ for some ${\mathcal N}$-barbed bisimulation ${\mathcal S}_{\mathcal N}$.
\end{definition}

$\mathcal{R} \subseteq \pi \times \pi$

$P \mathcal{R} Q => \forall P'. P \red P' \Rightarrow \exists Q'. Q \red Q', P' \mathcal{R} Q'$

$P \vdash x \Rightarrow Q \vdash x$

\begin{mathpar}
  \inferrule*[lab=Out-barb]{x \nameeq y}{{y}!\langle{Q}\rangle \vdash x}
  \and
  \inferrule*[lab=Par-barb]{\mbox{$P\vdash x$ or $Q\vdash x$}}{\binpar{P}{Q} \vdash x}
\end{mathpar}

\subsubsection{Contexts}

One of the principle advantages of computational calculi like the
$\pi$-calculus is a well-defined notion of context,
contextual-equivalence and a correlation between
contextual-equivalence and notions of bisimulation. The notion of
context allows the decomposition of a process into (sub-)process and
its syntactic environment, its context. Thus, a context may be
thought of as a process with a ``hole'' (written $\Box$) in it. The
application of a context $M$ to a process $P$, written $M[P]$, is
tantamount to filling the hole in $M$ with $P$. In this paper we do
not need the full weight of this theory, but do make use of the notion
of context in the proof the main theorem. 

\begin{mathpar}
  \inferrule* [lab=summation] {} {{M_{M},M_{N}} \bc \Box \;|\; x.M_{A} \;|\; M_{M}+M_{N}}
  \and
  \inferrule* [lab=agent] {} {{M_{A}} \bc (\vec{x})M_{P} \;| \; \clift{P_0,\ldots,M_{P},\ldots,P_N}}
  \and \\
  \inferrule* [lab=process] {} {{M_{P}} \bc M_{N} \;| \;P|M_{P} }
\end{mathpar} 

\begin{mathpar}
  \inferrule* [lab=sychronization] {} {M_{N} \bc \Box \;|\; x?M_{F} \;|\; x!M_{C}}
  \and
  \inferrule* [lab=abstraction] {} {{M_{F}} \bc (x)M_{P} }
  \and
  \inferrule* [lab=concretion] {} {{M_{C}} \bc \langle M_{P} \rangle }
  \and \\
  \inferrule* [lab=process] {} {{M_{P}} \bc M_{N} \;| \;P|M_{P} }
\end{mathpar}

\begin{definition}[contextual application] Given a context $M$, and
  process $P$, we define the \emph{contextual application}, $M[P] :=
  M\{P/\Box\}$. That is, the contextual application of M to P is the
  substitution of $P$ for $\Box$ in $M$.
\end{definition}

$\meaningof{-} : L \to \mathcal{P}(\pi)$

\begin{mathpar}
  \inferrule* [lab=collection] {} {\meaningof{true} = \pi, \and \meaningof{~E} = \pi \setminus \meaningof{E}, \and \meaningof{E_{1} \& E_{2}} = \meaningof{E_{1}} \cap \meaningof{E_{2}}}
\end{mathpar}

\begin{mathpar}
  \inferrule* [lab=structure] {} {\meaningof{0} = \{ P \in \pi | P \equiv 0 \}, \and \\ \meaningof{E_1 | E_2} = \{ P \in \pi | P \equiv P_{1} | P_{2}, P_{1} \in \meaningof{E_{1}}, P_{2} \in \meaningof{E_2}\} }
\end{mathpar}

\begin{mathpar}
 \inferrule* [lab=behavior] {} {\meaningof{\langle a?b \rangle E} = \{ P \in \pi | P \equiv Q | u?(y)P', \\ \and \\\\ \and \\ \;\;\; u \in \meaningof{a}, \forall z.P'\{z/y\} \in \meaningof{E\{z/b\}}\}, \and \\ \meaningof{a!E} = \{ P \in \pi | P \equiv Q | x!\langle P' \rangle, x \in \meaningof{a} P' \in \meaningof{E}\} }
\end{mathpar}

\begin{mathpar}
 \inferrule* [lab=nominal] {} {\meaningof{\quotep{E}} = \{ \quotep{P} \in \quotep{\pi} | P \in \meaningof{E} \}, \and \meaningof{\quotep{P}} = \{ \quotep{Q} \in \quotep{\pi} | P \equiv Q \} \and \\ \meaningof{@\quotep{E}} = \{ P \in \pi | P \equiv @x, x \in \meaningof{E} \}}
\end{mathpar}

\begin{eqnarray*}
  \\
  \meaningof{-} : TS \to ST
\end{eqnarray*}

\begin{eqnarray*}
  \\
  L : TS \to ST
\end{eqnarray*}

\begin{eqnarray*}
  \\
  P \models E \iff P \in \meaningof{E}
\end{eqnarray*}

\begin{eqnarray*}
  P \approx_{L} Q \iff \forall E \in L. P \models E \iff Q \models E
\end{eqnarray*}

\begin{eqnarray*}
  P \approx_{K} Q
\end{eqnarray*}

\begin{eqnarray*}
  P \approx Q
\end{eqnarray*}

$\approx_{K} = \approx = \approx_{L}$

\subsubsection{Contextual duality}

Note that contexts extend the quotation operation to a family of
operations from processes to names. Given a context, $M$, we can
define a \emph{nominal context}, $\quotep{M}$ by $\quotep{M}[P] :=
\quotep{M[P]}$. To foreshadow what is to come we observe that these
operations enjoy a duality with processes very much like the duality
between vectors and maps from vectors to scalars.

Further, because the calculus is essentially higher-order, we have a
correspondence between contexts and processes. More specifically,
given a name $x$ and a context $M$ we can construct $M^{*}_{x}$ such
that 

\begin{mathpar}
  M^{*}_{x} | \lift{x}{P} \red M[P]
\end{mathpar}

namely,

\begin{mathpar}
  M^{*}_{x} := x?(u).M[\dropn{u}]
\end{mathpar}

The dependence of $M^{*}_{x}$ on a name makes it an abstraction, 

\begin{mathpar}
  M^{*} := (x)x?(u).M[\dropn{u}]
\end{mathpar}

\subsection{Additional notation}

It will sometimes be convenient to denote the process a name
quotes. We already have the notation $x = \quotep{P}$, but it will be
convenient to introduce an alternate notation, $\procn{x}$, when we
want to emphasize the connection to the use of the name. Note that, by
virtue of name equivalence, $\quotep{\procn{x}} \nameeq x$; so, the
notation is consistent with previous definitions.

Further, because names have structure it is possible to effect
substitutions on the basis of that structure. This means we need to
upgrade our notation for substitutions, which we accomplish by
adapting comprehension notation. Thus,

\begin{mathpar}
  P\{ y / x : x \in S \}
\end{mathpar}

is interpreted to mean the process derived from P by replacing (in a
capture-avoiding manner) each occurrence of $x$ in $S$ by $y$. For example,

\begin{mathpar}
  P\{ \quotep{\procn{x}|\procn{x}} / x : x \in \freenames{P} \}
\end{mathpar}

will replace each (occurrence) of a free name $x$ in $P$ by
$\quotep{\procn{x}|\procn{x}}$.

Also, we will avail ourselves of the notation $x^{L}$ and $x^{R}$ to
denote injections of a name into disjoint copies of the name
space. There are numerous ways to accomplish this. One example can be
found in \cite{MeredithR05}. This notation overloads to vectors of
names: $\vec{x}^{\pi} := (x_{i}^{\pi} \; : \; 0 \leq i < |\vec{x}| )$ where $\pi \in \{L,R\}$.

We also use $P^{\Box} := P|\Box$.

In \cite{MeredithR05} an interpretation of the new operator is
given. It turns out that there are several possible interpretations
all enjoying the requisite algebraic properties of the operator (see
\cite{milner91polyadicpi}). We will therefore make liberal use of
$(\nu\; \vec{x})P$.

% subsection the_syntax_and_semantics_of_the_notation_system (end)   

\input{qm2pi.qmops} 

\input{qm2pi.sterngerlach} 

\input{qm2pi.metric} 

% section concurrent_process_calculi (end)

%\input{qm2pi.proofsketch}

% section proof sketch (end)

%\input{qm2pi.slviaknots} 

% section spatial logic via knots (end)

\input{qm2pi.conclusion}

% section conclusion (end)

%\input{qm2pi.dtcodes} 

% section wiring algorithm (end)

\input{qm2pi.ack} 

% section acknowledgments (end)

\newpage


\bibliographystyle{plain}   
\bibliography{../../biblios/main.bib}

\input{qm2pi.rhodetails}

\end{document}

 

%\ifpdf
%\usepackage[pdftex]{graphicx}
%\else
%\usepackage{graphicx}
%\fi

 % \ifpdf
%  \usepackage{pdfsync}
%  \if


%\title{Brief Article}
%\author{David F. Snyder}
%\author{L.G. Meredith}

%\address{Dept. of Math., Texas State University--San Marcos, San Marcos, TX 78666}
       
\pagestyle{empty}


\begin{document}

\lstset{language=[Objective]Caml,frame=shadowbox}

\documentclass[12pt]{llncs}
%\documentclass{jktr}

\usepackage[pdftex]{hyperref}                   
\usepackage {listings}
\usepackage {mathpartir}
\usepackage{bcprules}
%\usepackage{listings}
                       
\usepackage{graphicx} 
%\usepackage[margins=2.5cm,nohead,nofoot]{geometry}
%\usepackage{geometry}
\usepackage{amsfonts}
\usepackage{amstext}
\usepackage{latexsym}
\usepackage{amssymb}
\usepackage{color}


%\include{myPreamble}
\include{qm2pi.local} 

%\ifpdf
%\usepackage[pdftex]{graphicx}
%\else
%\usepackage{graphicx}
%\fi

 % \ifpdf
%  \usepackage{pdfsync}
%  \if


%\title{Brief Article}
%\author{David F. Snyder}
%\author{L.G. Meredith}

%\address{Dept. of Math., Texas State University--San Marcos, San Marcos, TX 78666}
       
\pagestyle{empty}


\begin{document}

\lstset{language=[Objective]Caml,frame=shadowbox}

\input{qm2pi.front}

% section front matter (end)

\input{qm2pi.intro} 
 
% section introduction (end)

% \input{qm2pi.knotations} 

% section notation (end)

\input{qm2pi.process.calculi} 

% section concurrent_process_calculi_and_spatial_logics_ (end)
    
%\input{qm2pi.knots2pi} 

%\input{qm2pi.trefoil} 

%\input{qm2pi.mainthm} 

% subsection basic_interpretation (end)

%\input{qm2pi.rho.presentation} 
\subsection{The syntax and semantics of the notation system}\label{sub:the_syntax_and_semantics_of_the_notation_system} % (fold)

We now summarize a technical presentation of the calculus that
embodies our theory of dynamics. The typical presentation of such a
calculus follows the style of giving generators and relations on
them. The grammar, below, describing term constructors, freely
generates the set of processes, $\Proc$. This set is then quotiented
by a relation known as structural congruence and it is over this set
that the notion of dynamics is expressed. This presentation is
essentially that of \cite{MeredithR05} with the addition of
polyadicity and summation. For readability we have relegated some of
the technical subtleties to an appendix.

\subsubsection{Process grammar}\label{subsub:process_grammar}

\begin{mathpar}
  \inferrule* [lab=synchronization] {} {{M} \bc \pzero \;|\; x?F \;|\; x!C }
  \and
  \inferrule* [lab=abstraction] {} {{F} \bc (x)P}
  \and
  \inferrule* [lab=concretion] {} {{C} \bc \langle Q \rangle}
  \and
  \inferrule* [lab=process] {} {{P,Q} \bc M \;| \;P|Q \;|\; @{x}}
  \and
  \inferrule* [lab=name] {} {{x} \bc \quotep{P}}
\end{mathpar} 

Note that $\vec{x}$ (resp. $\vec{P}$) denotes a vector of names
(resp. processes) of length $|\vec{x}|$ (resp. $|\vec{P}|$). We adopt
the following useful abbreviations.

\begin{mathpar}
   x?(\vec{y}).P := x.(\vec{y})P \and  x\clift{\vec{P}} := x.\clift{\vec{P}}
   \and x!(y) := \lift{x}{\dropn{y}}
   \and \Pi_{i=0}^{n-1}P_i := P_0 | \ldots | P_{n-1}
\end{mathpar}

\subsubsection{Structural congruence}

\paragraph{Free and bound names and alpha-equivalence.} At the
core of structural equivalence is alpha-equivalence which identifies
process that are the same up to a change of variable. Formally, we
recognize the distinction between free and bound names. The free names
of a process, $\freenames{P}$, may be calculated recursively as
follows:

\begin{mathpar}
\freenames{\pzero} := \emptyset
  \and \\
  \freenames{x?(y).P} := \{ x \} \cup (\freenames{P} \setminus \{ y \})
  \and 
  \freenames{x!\langle P \rangle} := \{ x \} \cup \{ P \} 
  \and \\
  \freenames{P|Q} := \freenames{P} \cup \freenames{Q}
  \and \\
  \freenames{@{x}} := \{ x \}
\end{mathpar}

$\pi$
$\quotep{\pi}$

$\freenames{-} : \pi \to \mathcal{P}(\quotep{\pi})$

\begin{eqnarray*}
  \freenames{\pzero} & := & \emptyset \\
  \freenames{x?(y).P} & := & \{ x \} \cup (\freenames{P} \setminus \{ y \}) \\
  \freenames{x!\langle P \rangle} & := & \{ x \} \cup \{ P \} \\
  \freenames{P|Q} & := & \freenames{P} \cup \freenames{Q} \\
  \freenames{\dropn{x}} & := & \{ x \}
\end{eqnarray*}

The bound names of a process, $\boundnames{P}$, are those names occurring in $P$
that are not free. For example, in $x?(y).0$, the name $x$ is free, while $y$ is bound.

\begin{mathpar}
  \inferrule* [lab=monoidal-laws] {} { P|Q \equiv Q|P \and P|0 \equiv P \and P|(Q|R) \equiv (P|Q)|R }
\end{mathpar}

\begin{mathpar}
  \inferrule* [lab=alpha-equivalence] {} { (x)P \equiv (y)P\{y/x\} \and y \not\in \freenames{P} }
\end{mathpar}

\begin{definition}
Then two processes, $P,Q$, are alpha-equivalent if $P = Q\{\vec{y}/\vec{x}\}$ for
some $\vec{x} \in \boundnames{Q},\vec{y} \in \boundnames{P}$, where $Q\{\vec{y}/\vec{x}\}$
denotes the capture-avoiding substitution of $\vec{y}$ for $\vec{x}$ in $Q$.
\end{definition}

\begin{definition}
  The {\em structural congruence} \cite{SangiorgiWalker} , $\equiv$,
  between processes is the least congruence containing
  alpha-equivalence, satisfying the abelian monoid laws
  (associativity, commutativity and $\pzero$ as identity) for parallel
  composition $|$ and for summation $+$.
\end{definition}

\subsection{Name equivalence}

We take name equivalence, written $\nameeq$, to be the smallest
equivalence relation generated by the following rules.

\begin{mathpar}
\inferrule*[lab=Quote-drop]
{ }
{ \quotep{@{x}} \nameeq x }

\inferrule*[lab=Struct-equiv]
{ P \scong Q }
{ \quotep{P} \nameeq \quotep{Q} }
\end{mathpar}

The astute reader will have noticed that the mutual recursion of names
and processes imposes a mutual recursion on alpha-equivalence and
structural equivalence via name-equivalence. Fortunately, all of this
works out pleasantly and we may calculate in the natural way, free of
concern. The reader interested in the details is referred to the
appendix \ref{appendix:rho_details}.

\subsection{Substitution}

We use $\Proc$ for the set of processes, $\QProc$ for the set of
names, and $\id{\{}\vec{y} / \vec{x} \id{\}}$ to denote partial maps,
$s : \QProc \rightarrow \QProc$. A map, $s$ lifts, uniquely, to a map
on process terms, $\widehat{s} : \Proc \rightarrow \Proc$ by the
following equations.

\begin{mathpar}
  (0) \psubstp{Q}{P} := 0 \\
  (R \juxtap S) \psubstp{Q}{P}
  :=    
  (R)\psubstp{Q}{P} \juxtap (S) \psubstp{Q}{P} \\
  (x?(y).R) \psubstp{Q}{P}    
  :=    
  (x)\substp{Q}{P} (z)\concat( (R \psubstn{z}{y}) \psubstp{Q}{P} ) \\
  (\lift{x}{R}) \psubstp{Q}{P}  
  :=
  \lift{(x)\substp{Q}{P}}{ R \psubstp{Q}{P} } \\
%   (\dropn{x})  \psubstp{Q}{P}       
%   := 
%   \left\{ 
%     \begin{array}{ccc} 
%       \dropn{\quotep{Q}} & & x \nameeq \quotep{P} \\
%       \dropn{x} & & otherwise \\
%     \end{array}
%   \right. 
  (\dropn{x})  \psubstp{Q}{P}       
  := 
  \left\{ 
    \begin{array}{ccc} 
      Q & & x \nameeq \quotep{P} \\
      \dropn{x} & & otherwise \\
    \end{array}
  \right.
\end{mathpar}
 

where

\begin{eqnarray}
  (x)\id{\{} \lpquote Q \rpquote / \lpquote P \rpquote \id{\}}            = 
  \left\{ 
    \begin{array}{ccc}
      \lpquote Q \rpquote & & x \nameeq \lpquote P \rpquote \\
      x & & otherwise \\
    \end{array}
  \right. \nonumber
\end{eqnarray}

and $z$ is chosen distinct from $\quotep{P}$, $\quotep{Q}$, the free
names in $Q$, and all the names in $R$. Our $\alpha$-equivalence will
be built in the standard way from this substitution.

\begin{remark}\label{rem:no_self_referential_names}
  One consequence of these definitions is that $\forall P. \quotep{P}
  \not\in \freenames{P}$.
\end{remark}

\subsection{ Dynamic quote: an example }

Anticipating something of what's to come, consider applying the
substitution, $\widehat{\id{\{}u / z \id{\}}}$, to the following pair
of processes, $\lift{w}{y!(z)}$ and $w[ \lpquote y!(z) \rpquote ]$.

\begin{eqnarray}
	\lift{w}{y!(z)}\widehat{\id{\{}u / z \id{\}}}
		& = &
		\lift{w}{y!(u)} \nonumber\\
	w[ \lpquote y!(z) \rpquote ] \widehat{ \id{\{}u / z \id{\}} }
		& = &
		w[ \lpquote y!(z) \rpquote ] \nonumber
\end{eqnarray}

Because the body of the process between quotes is impervious to
substitution, we get radically different answers. In fact, by
examining the first process in an input context,
e.g. $x?(z).\lift{w}{y!(z)}$, we see that the process under the lift
operator may be shaped by prefixed inputs binding a name inside it. In
this sense, the lift operator will be seen as a way to dynamically
construct processes before reifying them as names.

Finally equipped with these standard features we can present the
dynamics of the calculus.

\subsubsection{Operational semantics} 

Finally, we introduce the computational dynamics. What marks these
algebras as distinct from other more traditionally studied algebraic
structures, e.g. vector spaces or polynomial rings, is the manner in
which dynamics is captured. In traditional structures, dynamics is typically
expressed through morphisms between such structures, as in linear maps
between vector spaces or morphisms between rings. In algebras
associated with the semantics of computation, the dynamics is
expressed as part of the algebraic structure itself, through a
reduction reduction relation typically denoted by $\red$. Below, we
give a recursive presentation of this relation for the calculus used
in the encoding.

$\red \subseteq \pi \times \pi$
$\red : \pi \to \mathcal{P}(\pi)$

\begin{mathpar}
  \inferrule* [lab=Comm] { \textsf{match}( x_{src}, x_{trgt} ) } { x_{trgt}?(y)P \; | \; x_{src}!\langle {Q} \rangle \red P\{\quotep{Q}/y}\} }
  \and \\
  \inferrule* [lab=Par] {{P} \red {P}'} {{{P} | {Q}} \red {{P}' | {Q}}}
  \and
  \inferrule* [lab=Equiv]{{{P} \scong {P}'} \andalso {{P}' \red {Q}'} \andalso {{Q}' \scong {Q}}}{{P} \red {Q}}
\end{mathpar}

\begin{eqnarray*}
  match_{\equiv} (\quotep{P},\quotep{Q}) & := & P \equiv Q \\
  match_{\dagger}(\quotep{P},\quotep{Q}) & := & \forall R. P|Q \red^{*} R => R \red^{*} 0 \\
  match_{K}(\quotep{P},\quotep{Q}) & := & K \mbox{ for some context } K
\end{eqnarray*}

$u?(x)P | u!\langle Q \rangle \red P\{\quotep{Q}/x\}$

%We write $\wred$ for $\red^*$, and $P\red$ if $\exists Q $ such that $ P \red Q$.
We write $P\red$ if $\exists Q $ such that $ P \red Q$ and $P\not\red$, otherwise.

\section{Replication}

As mentioned before, it is known that replication (and hence
recursion) can be implemented in a higher-order process algebra
\cite{SangiorgiWalker}. As our first example of calculation with the
machinery thus far presented we give the construction explicitly in
the {\rhoc}.

\begin{eqnarray}
	D_{x} & := & \prefix{x}{y}{(\binpar{\outputp{x}{y}}{@{y}})} \nonumber\\
	\bangp_{x}{P} & := & \binpar{{x}!\langle{\binpar{D_{x}}{P}}\rangle}{D_{x}} \nonumber
\end{eqnarray}

\begin{eqnarray}
	\bangp_{x}{P} & & \nonumber\\
	=
	& {x}!\langle{(\prefix{x}{y}{(\outputp{x}{y} | @{y})) | P}}\rangle 
	      | \prefix{x}{y}{(\outputp{x}{y} | @{y})} & \nonumber\\
	\red
	& (\outputp{x}{y} | @{y})\substn{\quotep{(\prefix{x}{y}{(@{y} | \outputp{x}{y})) | P}}}{y} & \nonumber\\
	=
	& \outputp{x}{\quotep{(\prefix{x}{y}{(\outputp{x}{y} | @{y})) | P}}}
	  | {(\prefix{x}{y}{(\outputp{x}{y} | @{y})) | P}} & \nonumber\\
	\red
	& \ldots & \nonumber\\
	\red^*
	& P | P | \ldots & \nonumber
\end{eqnarray}

Of course, this encoding, as an implementation, runs away, unfolding
$\bangp{P}$ eagerly. A lazier and more implementable replication
operator, restricted to input-guarded processes, may be obtained as follows.

\begin{eqnarray}
\bangp{\prefix{u}{v}{P}} 
	:= 
	\binpar{\lift{x}{\prefix{u}{v}{(\binpar{D(x)}{P})}}}{D(x)} \nonumber
\end{eqnarray}

\begin{remark}
  Note that the lazier definition still does not deal with summation
  or mixed summation (i.e. sums over input and output). The reader is
  invited to construct definitions of replication that deal with these
  features. 

  Further, the definitions are parameterized in a name, $x$. Can you,
  gentle reader, make a definition that eliminates this parameter and
  guarantees no accidental interaction between the replication
  machinery and the process being replicated -- i.e. no accidental
  sharing of names used by the process to get its work done and the
  name(s) used by the replication to effect copying. This latter
  revision of the definition of replication is crucial to obtaining
  the expected identity $!!P \sim !P$.
\end{remark}

\begin{remark}\label{rem:paradoxical_combinator}
  The reader familiar with the lambda calculus will have noticed the
  similarity between $D$ and the paradoxical combinator.

  [Ed. note: the existence of this seems to suggest we have to be more
  restrictive on the set of processes and names we admit if we are to
  support no-cloning.]
\end{remark}

\subsubsection{Bisimulation}

The computational dynamics gives rise to another kind of equivalence,
the equivalence of computational behavior. As previously mentioned
this is typically captured \emph{via} some form of bisimulation.

% The notion we use in this paper is weak barbed bisimulation
% \cite{milner91polyadicpi}.

The notion we use in this paper is derived from weak barbed
bisimulation \cite{milner91polyadicpi}. 

\begin{definition}
An \emph{observation relation}, $\downarrow_{\mathcal N}$, over a set
of names, $\mathcal N$, is the smallest relation satisfying the rules
below.

\infrule[Out-barb]{y \in {\mathcal N}, \; x \nameeq y}
		  {\outputp{x}{v} \downarrow_{\mathcal N} x}
\infrule[Par-barb]{\mbox{$P\downarrow_{\mathcal N} x$ or $Q\downarrow_{\mathcal N} x$}}
		  {\binpar{P}{Q} \downarrow_{\mathcal N} x}

We write $P \Downarrow_{\mathcal N} x$ if there is $Q$ such that 
$P \wred Q$ and $Q \downarrow_{\mathcal N} x$.
\end{definition}

\begin{definition}
%\label{def.bbisim}
An  ${\mathcal N}$-\emph{barbed bisimulation} over a set of names, ${\mathcal N}$, is a symmetric binary relation 
${\mathcal S}_{\mathcal N}$ between agents such that $P\rel{S}_{\mathcal N}Q$ implies:
\begin{enumerate}
\item If $P \red P'$ then $Q \wred Q'$ and $P'\rel{S}_{\mathcal N} Q'$.
\item If $P\downarrow_{\mathcal N} x$, then $Q\Downarrow_{\mathcal N} x$.
\end{enumerate}
$P$ is ${\mathcal N}$-barbed bisimilar to $Q$, written
$P \wbbisim_{\mathcal N} Q$, if $P \rel{S}_{\mathcal N} Q$ for some ${\mathcal N}$-barbed bisimulation ${\mathcal S}_{\mathcal N}$.
\end{definition}

$\mathcal{R} \subseteq \pi \times \pi$

$P \mathcal{R} Q => \forall P'. P \red P' \Rightarrow \exists Q'. Q \red Q', P' \mathcal{R} Q'$

$P \vdash x \Rightarrow Q \vdash x$

\begin{mathpar}
  \inferrule*[lab=Out-barb]{x \nameeq y}{{y}!\langle{Q}\rangle \vdash x}
  \and
  \inferrule*[lab=Par-barb]{\mbox{$P\vdash x$ or $Q\vdash x$}}{\binpar{P}{Q} \vdash x}
\end{mathpar}

\subsubsection{Contexts}

One of the principle advantages of computational calculi like the
$\pi$-calculus is a well-defined notion of context,
contextual-equivalence and a correlation between
contextual-equivalence and notions of bisimulation. The notion of
context allows the decomposition of a process into (sub-)process and
its syntactic environment, its context. Thus, a context may be
thought of as a process with a ``hole'' (written $\Box$) in it. The
application of a context $M$ to a process $P$, written $M[P]$, is
tantamount to filling the hole in $M$ with $P$. In this paper we do
not need the full weight of this theory, but do make use of the notion
of context in the proof the main theorem. 

\begin{mathpar}
  \inferrule* [lab=summation] {} {{M_{M},M_{N}} \bc \Box \;|\; x.M_{A} \;|\; M_{M}+M_{N}}
  \and
  \inferrule* [lab=agent] {} {{M_{A}} \bc (\vec{x})M_{P} \;| \; \clift{P_0,\ldots,M_{P},\ldots,P_N}}
  \and \\
  \inferrule* [lab=process] {} {{M_{P}} \bc M_{N} \;| \;P|M_{P} }
\end{mathpar} 

\begin{mathpar}
  \inferrule* [lab=sychronization] {} {M_{N} \bc \Box \;|\; x?M_{F} \;|\; x!M_{C}}
  \and
  \inferrule* [lab=abstraction] {} {{M_{F}} \bc (x)M_{P} }
  \and
  \inferrule* [lab=concretion] {} {{M_{C}} \bc \langle M_{P} \rangle }
  \and \\
  \inferrule* [lab=process] {} {{M_{P}} \bc M_{N} \;| \;P|M_{P} }
\end{mathpar}

\begin{definition}[contextual application] Given a context $M$, and
  process $P$, we define the \emph{contextual application}, $M[P] :=
  M\{P/\Box\}$. That is, the contextual application of M to P is the
  substitution of $P$ for $\Box$ in $M$.
\end{definition}

$\meaningof{-} : L \to \mathcal{P}(\pi)$

\begin{mathpar}
  \inferrule* [lab=collection] {} {\meaningof{true} = \pi, \and \meaningof{~E} = \pi \setminus \meaningof{E}, \and \meaningof{E_{1} \& E_{2}} = \meaningof{E_{1}} \cap \meaningof{E_{2}}}
\end{mathpar}

\begin{mathpar}
  \inferrule* [lab=structure] {} {\meaningof{0} = \{ P \in \pi | P \equiv 0 \}, \and \\ \meaningof{E_1 | E_2} = \{ P \in \pi | P \equiv P_{1} | P_{2}, P_{1} \in \meaningof{E_{1}}, P_{2} \in \meaningof{E_2}\} }
\end{mathpar}

\begin{mathpar}
 \inferrule* [lab=behavior] {} {\meaningof{\langle a?b \rangle E} = \{ P \in \pi | P \equiv Q | u?(y)P', \\ \and \\\\ \and \\ \;\;\; u \in \meaningof{a}, \forall z.P'\{z/y\} \in \meaningof{E\{z/b\}}\}, \and \\ \meaningof{a!E} = \{ P \in \pi | P \equiv Q | x!\langle P' \rangle, x \in \meaningof{a} P' \in \meaningof{E}\} }
\end{mathpar}

\begin{mathpar}
 \inferrule* [lab=nominal] {} {\meaningof{\quotep{E}} = \{ \quotep{P} \in \quotep{\pi} | P \in \meaningof{E} \}, \and \meaningof{\quotep{P}} = \{ \quotep{Q} \in \quotep{\pi} | P \equiv Q \} \and \\ \meaningof{@\quotep{E}} = \{ P \in \pi | P \equiv @x, x \in \meaningof{E} \}}
\end{mathpar}

\begin{eqnarray*}
  \\
  \meaningof{-} : TS \to ST
\end{eqnarray*}

\begin{eqnarray*}
  \\
  L : TS \to ST
\end{eqnarray*}

\begin{eqnarray*}
  \\
  P \models E \iff P \in \meaningof{E}
\end{eqnarray*}

\begin{eqnarray*}
  P \approx_{L} Q \iff \forall E \in L. P \models E \iff Q \models E
\end{eqnarray*}

\begin{eqnarray*}
  P \approx_{K} Q
\end{eqnarray*}

\begin{eqnarray*}
  P \approx Q
\end{eqnarray*}

$\approx_{K} = \approx = \approx_{L}$

\subsubsection{Contextual duality}

Note that contexts extend the quotation operation to a family of
operations from processes to names. Given a context, $M$, we can
define a \emph{nominal context}, $\quotep{M}$ by $\quotep{M}[P] :=
\quotep{M[P]}$. To foreshadow what is to come we observe that these
operations enjoy a duality with processes very much like the duality
between vectors and maps from vectors to scalars.

Further, because the calculus is essentially higher-order, we have a
correspondence between contexts and processes. More specifically,
given a name $x$ and a context $M$ we can construct $M^{*}_{x}$ such
that 

\begin{mathpar}
  M^{*}_{x} | \lift{x}{P} \red M[P]
\end{mathpar}

namely,

\begin{mathpar}
  M^{*}_{x} := x?(u).M[\dropn{u}]
\end{mathpar}

The dependence of $M^{*}_{x}$ on a name makes it an abstraction, 

\begin{mathpar}
  M^{*} := (x)x?(u).M[\dropn{u}]
\end{mathpar}

\subsection{Additional notation}

It will sometimes be convenient to denote the process a name
quotes. We already have the notation $x = \quotep{P}$, but it will be
convenient to introduce an alternate notation, $\procn{x}$, when we
want to emphasize the connection to the use of the name. Note that, by
virtue of name equivalence, $\quotep{\procn{x}} \nameeq x$; so, the
notation is consistent with previous definitions.

Further, because names have structure it is possible to effect
substitutions on the basis of that structure. This means we need to
upgrade our notation for substitutions, which we accomplish by
adapting comprehension notation. Thus,

\begin{mathpar}
  P\{ y / x : x \in S \}
\end{mathpar}

is interpreted to mean the process derived from P by replacing (in a
capture-avoiding manner) each occurrence of $x$ in $S$ by $y$. For example,

\begin{mathpar}
  P\{ \quotep{\procn{x}|\procn{x}} / x : x \in \freenames{P} \}
\end{mathpar}

will replace each (occurrence) of a free name $x$ in $P$ by
$\quotep{\procn{x}|\procn{x}}$.

Also, we will avail ourselves of the notation $x^{L}$ and $x^{R}$ to
denote injections of a name into disjoint copies of the name
space. There are numerous ways to accomplish this. One example can be
found in \cite{MeredithR05}. This notation overloads to vectors of
names: $\vec{x}^{\pi} := (x_{i}^{\pi} \; : \; 0 \leq i < |\vec{x}| )$ where $\pi \in \{L,R\}$.

We also use $P^{\Box} := P|\Box$.

In \cite{MeredithR05} an interpretation of the new operator is
given. It turns out that there are several possible interpretations
all enjoying the requisite algebraic properties of the operator (see
\cite{milner91polyadicpi}). We will therefore make liberal use of
$(\nu\; \vec{x})P$.

% subsection the_syntax_and_semantics_of_the_notation_system (end)   

\input{qm2pi.qmops} 

\input{qm2pi.sterngerlach} 

\input{qm2pi.metric} 

% section concurrent_process_calculi (end)

%\input{qm2pi.proofsketch}

% section proof sketch (end)

%\input{qm2pi.slviaknots} 

% section spatial logic via knots (end)

\input{qm2pi.conclusion}

% section conclusion (end)

%\input{qm2pi.dtcodes} 

% section wiring algorithm (end)

\input{qm2pi.ack} 

% section acknowledgments (end)

\newpage


\bibliographystyle{plain}   
\bibliography{../../biblios/main.bib}

\input{qm2pi.rhodetails}

\end{document}



% section front matter (end)

\section{Introduction}\label{sec:introduction} % (fold)
In this draft of the material i am going to have to dispense with the
usual writing conventions adopted in papers on these topics. i'm going
to have adopt whatever tone i need at the time i'm writing up the
calculations. Sometimes this may be very conversational; others it may
be the barest mathematical grunts; others still it may be that i have
lifted text from one of my other papers because the exposition of some
point was better said there. i hope that my readers are not unduly put
out by this decision. i'm not doing this to flout convention or be
rebellious. i find these calculations very technically challenging. To
keep everything going technically, something has to give; i have to
let go of some cognitive burden. So, the academic writing style --
with all of its trade-offs in terms of facilitating technical
communication -- is what i'm letting go of. Perhaps subsequent drafts
can be tightened and polished, but for now, i'm going to speak as if
we were sitting together in a coffee shop with a laptop, wifi and a
pad of paper and a pencil.

So, here's what i have to say. We -- you and i, comfortably ensconced
in our coffee shop and well-equipped with our tools -- can realize and
carry out the calculations of quantum mechanics over a very different
formal theory of dynamics, a formal theory of dynamics that
corresponds to a theory of concurrent computation with
\emph{reflection}. It has the advantage that the underlying theory is
already `quantized', but supports analogues all of the continuuous
operations. Strikingly, this underlying theory has recently been
connected with a notion of metric that we can show, by calculating
together, coincides with the metric induced by the inner product.

There are a lot of reasons why you might be interested in seeing
calculations of this form. Here's why i'm interested. For the past
several centuries there has been no competitor to the ``Newtonian''
account of dynamics. As a result the predominant share of accounts of
dynamical systems and situations have had to be formulated in terms of
the Newtonian machinery. i view this as an intellectually dangerous
position to occupy. Everything, despite it's intrinsic shape, turns
into a nail to be hit with this hammer. Recently, however, the theory
of computation has matured to the point where we have candidates for
theories of dynamics that offer very different perspective on
reasoning about dynamical systems and situations. Testing these
candidates against very successful accounts of dynamical situations,
like quantum mechanics, is going to give us some sense of how mature
they are and some measure of the quality of these accounts of
dynamics.

\subsection{Summary of contributions and outline of paper}

So, we're going to develop an interpretation of the operations of
quantum mechanics normally interpreted by Hilbert spaces and
operators. We're going to do this over a theory of computation. Note
that this is very different than the usual quantum computation program
which develops notions of computation over quantum mechanics. Rather,
we are developing a story that aligns with Wheeler's slogan: It from
Bit. To do this we will first provide an account of the theory of
computation at play here. Then we will dive into a calculation-driven
interpretation of the operations of quantum mechanics.

The reason we take this approach is that -- until very recently --
there hasn't been an axiomatic account of quantum mechanics. As a
result there has been no sharp delineation of the mathematical theory
supporting interpretation of the physical theory and the physical
theory, itself. So, ambient features of the maths are free to be
exploited (or supressed) without a real accounting of their physical
relevance. There is no sharp statement ``here's the physical theory''
qua \emph{theory} and ``here's the mathematical interpretation''
enabling a judgment of how faithful the interpretation is -- apart
from experimental observation. When there is an axiomatic account we
can judge how well a given mathematical formalism supports an
interpretation of the axioms, independent of
experimentation. Likewise, we can judge how well we have captured our
physical evidence and experience with our axiomatics, independent of
any specific mathematical implementation, with accidental detail that
may or may not have physical significance. 

In lieu of a fully fleshed out and vetted axiomatic account of quantum
mechanics, interpreting the operational notions in service of modeling
physical systems will have to suffice. In other words, we are not in
the business of providing a model of Hilbert spaces and operators. We
are in the business of providing a model of quantum mechanics because
we are motivated by testing our notions of dynamics against physical
theory; and, the predictive calculations of the physical theory must
serve as the best formulation -- shy of a fully fleshed out axiomatic
account -- of the physical theory itself (as they have for scientific
theories since time immemorial). Put another way, despite a
whole-hearted commitment to an It-from-Bit ontology, we are firmly
aligned with the shut-up-and-calculate camp as the best way to obtain
results either from the physical perspective or as a quality assurance
measure of our fledgling theory of dynamics.

In detail, we present a reflective process calculus. Then we develop
intuitive correspondences between the notions available in this
calculus and the usual physical notions supporting quantum mechanical
calculations. Thus, 

\begin{table}[htp]
  \center{
    \fbox{
      \begin{tabular}{c|c}
        quantum mechanics & process calculus \\
        \hline
        scalar & name \\
        state vector & process \\
        dual & contextual duals \\
        matrix & formal sums of process-context-dual pairs \\
        orthogonality & process annihilation \\
        inner product & execution-formula + quoting
      \end{tabular}
    }
  }
  \caption{QM - process calculi correspondences}
\end{table}

Then we tighten up these intuitions to operational definitions. We
employ the Dirac notation as the best proxy we can find for an
abstract syntax of the quantum mechanical notions. The definitions we
develop put us in contact with equational constraints coming from the
theory that we demonstrate the definitions and calculations satisfy.

This puts us in a position to shut up and calculate for the
Stern-Gerlach experimental set up, showing how these predictive
calculations become calculations on processes in our theory of a
reflective process calculus.

Penultimately, we demonstrate that the notion of metric coming from
the inner product coincides with the notion of metric available from
the theory of bisimulation. This demonstration gives us the right to
think of space as arising from behavior. Finally, we consider where we
might go from the new vantage point we have obtained.

% section introduction (end) 
 
% section introduction (end)

% \documentclass[12pt]{llncs}
%\documentclass{jktr}

\usepackage[pdftex]{hyperref}                   
\usepackage {listings}
\usepackage {mathpartir}
\usepackage{bcprules}
%\usepackage{listings}
                       
\usepackage{graphicx} 
%\usepackage[margins=2.5cm,nohead,nofoot]{geometry}
%\usepackage{geometry}
\usepackage{amsfonts}
\usepackage{amstext}
\usepackage{latexsym}
\usepackage{amssymb}
\usepackage{color}


%\include{myPreamble}
\include{qm2pi.local} 

%\ifpdf
%\usepackage[pdftex]{graphicx}
%\else
%\usepackage{graphicx}
%\fi

 % \ifpdf
%  \usepackage{pdfsync}
%  \if


%\title{Brief Article}
%\author{David F. Snyder}
%\author{L.G. Meredith}

%\address{Dept. of Math., Texas State University--San Marcos, San Marcos, TX 78666}
       
\pagestyle{empty}


\begin{document}

\lstset{language=[Objective]Caml,frame=shadowbox}

\input{qm2pi.front}

% section front matter (end)

\input{qm2pi.intro} 
 
% section introduction (end)

% \input{qm2pi.knotations} 

% section notation (end)

\input{qm2pi.process.calculi} 

% section concurrent_process_calculi_and_spatial_logics_ (end)
    
%\input{qm2pi.knots2pi} 

%\input{qm2pi.trefoil} 

%\input{qm2pi.mainthm} 

% subsection basic_interpretation (end)

%\input{qm2pi.rho.presentation} 
\subsection{The syntax and semantics of the notation system}\label{sub:the_syntax_and_semantics_of_the_notation_system} % (fold)

We now summarize a technical presentation of the calculus that
embodies our theory of dynamics. The typical presentation of such a
calculus follows the style of giving generators and relations on
them. The grammar, below, describing term constructors, freely
generates the set of processes, $\Proc$. This set is then quotiented
by a relation known as structural congruence and it is over this set
that the notion of dynamics is expressed. This presentation is
essentially that of \cite{MeredithR05} with the addition of
polyadicity and summation. For readability we have relegated some of
the technical subtleties to an appendix.

\subsubsection{Process grammar}\label{subsub:process_grammar}

\begin{mathpar}
  \inferrule* [lab=synchronization] {} {{M} \bc \pzero \;|\; x?F \;|\; x!C }
  \and
  \inferrule* [lab=abstraction] {} {{F} \bc (x)P}
  \and
  \inferrule* [lab=concretion] {} {{C} \bc \langle Q \rangle}
  \and
  \inferrule* [lab=process] {} {{P,Q} \bc M \;| \;P|Q \;|\; @{x}}
  \and
  \inferrule* [lab=name] {} {{x} \bc \quotep{P}}
\end{mathpar} 

Note that $\vec{x}$ (resp. $\vec{P}$) denotes a vector of names
(resp. processes) of length $|\vec{x}|$ (resp. $|\vec{P}|$). We adopt
the following useful abbreviations.

\begin{mathpar}
   x?(\vec{y}).P := x.(\vec{y})P \and  x\clift{\vec{P}} := x.\clift{\vec{P}}
   \and x!(y) := \lift{x}{\dropn{y}}
   \and \Pi_{i=0}^{n-1}P_i := P_0 | \ldots | P_{n-1}
\end{mathpar}

\subsubsection{Structural congruence}

\paragraph{Free and bound names and alpha-equivalence.} At the
core of structural equivalence is alpha-equivalence which identifies
process that are the same up to a change of variable. Formally, we
recognize the distinction between free and bound names. The free names
of a process, $\freenames{P}$, may be calculated recursively as
follows:

\begin{mathpar}
\freenames{\pzero} := \emptyset
  \and \\
  \freenames{x?(y).P} := \{ x \} \cup (\freenames{P} \setminus \{ y \})
  \and 
  \freenames{x!\langle P \rangle} := \{ x \} \cup \{ P \} 
  \and \\
  \freenames{P|Q} := \freenames{P} \cup \freenames{Q}
  \and \\
  \freenames{@{x}} := \{ x \}
\end{mathpar}

$\pi$
$\quotep{\pi}$

$\freenames{-} : \pi \to \mathcal{P}(\quotep{\pi})$

\begin{eqnarray*}
  \freenames{\pzero} & := & \emptyset \\
  \freenames{x?(y).P} & := & \{ x \} \cup (\freenames{P} \setminus \{ y \}) \\
  \freenames{x!\langle P \rangle} & := & \{ x \} \cup \{ P \} \\
  \freenames{P|Q} & := & \freenames{P} \cup \freenames{Q} \\
  \freenames{\dropn{x}} & := & \{ x \}
\end{eqnarray*}

The bound names of a process, $\boundnames{P}$, are those names occurring in $P$
that are not free. For example, in $x?(y).0$, the name $x$ is free, while $y$ is bound.

\begin{mathpar}
  \inferrule* [lab=monoidal-laws] {} { P|Q \equiv Q|P \and P|0 \equiv P \and P|(Q|R) \equiv (P|Q)|R }
\end{mathpar}

\begin{mathpar}
  \inferrule* [lab=alpha-equivalence] {} { (x)P \equiv (y)P\{y/x\} \and y \not\in \freenames{P} }
\end{mathpar}

\begin{definition}
Then two processes, $P,Q$, are alpha-equivalent if $P = Q\{\vec{y}/\vec{x}\}$ for
some $\vec{x} \in \boundnames{Q},\vec{y} \in \boundnames{P}$, where $Q\{\vec{y}/\vec{x}\}$
denotes the capture-avoiding substitution of $\vec{y}$ for $\vec{x}$ in $Q$.
\end{definition}

\begin{definition}
  The {\em structural congruence} \cite{SangiorgiWalker} , $\equiv$,
  between processes is the least congruence containing
  alpha-equivalence, satisfying the abelian monoid laws
  (associativity, commutativity and $\pzero$ as identity) for parallel
  composition $|$ and for summation $+$.
\end{definition}

\subsection{Name equivalence}

We take name equivalence, written $\nameeq$, to be the smallest
equivalence relation generated by the following rules.

\begin{mathpar}
\inferrule*[lab=Quote-drop]
{ }
{ \quotep{@{x}} \nameeq x }

\inferrule*[lab=Struct-equiv]
{ P \scong Q }
{ \quotep{P} \nameeq \quotep{Q} }
\end{mathpar}

The astute reader will have noticed that the mutual recursion of names
and processes imposes a mutual recursion on alpha-equivalence and
structural equivalence via name-equivalence. Fortunately, all of this
works out pleasantly and we may calculate in the natural way, free of
concern. The reader interested in the details is referred to the
appendix \ref{appendix:rho_details}.

\subsection{Substitution}

We use $\Proc$ for the set of processes, $\QProc$ for the set of
names, and $\id{\{}\vec{y} / \vec{x} \id{\}}$ to denote partial maps,
$s : \QProc \rightarrow \QProc$. A map, $s$ lifts, uniquely, to a map
on process terms, $\widehat{s} : \Proc \rightarrow \Proc$ by the
following equations.

\begin{mathpar}
  (0) \psubstp{Q}{P} := 0 \\
  (R \juxtap S) \psubstp{Q}{P}
  :=    
  (R)\psubstp{Q}{P} \juxtap (S) \psubstp{Q}{P} \\
  (x?(y).R) \psubstp{Q}{P}    
  :=    
  (x)\substp{Q}{P} (z)\concat( (R \psubstn{z}{y}) \psubstp{Q}{P} ) \\
  (\lift{x}{R}) \psubstp{Q}{P}  
  :=
  \lift{(x)\substp{Q}{P}}{ R \psubstp{Q}{P} } \\
%   (\dropn{x})  \psubstp{Q}{P}       
%   := 
%   \left\{ 
%     \begin{array}{ccc} 
%       \dropn{\quotep{Q}} & & x \nameeq \quotep{P} \\
%       \dropn{x} & & otherwise \\
%     \end{array}
%   \right. 
  (\dropn{x})  \psubstp{Q}{P}       
  := 
  \left\{ 
    \begin{array}{ccc} 
      Q & & x \nameeq \quotep{P} \\
      \dropn{x} & & otherwise \\
    \end{array}
  \right.
\end{mathpar}
 

where

\begin{eqnarray}
  (x)\id{\{} \lpquote Q \rpquote / \lpquote P \rpquote \id{\}}            = 
  \left\{ 
    \begin{array}{ccc}
      \lpquote Q \rpquote & & x \nameeq \lpquote P \rpquote \\
      x & & otherwise \\
    \end{array}
  \right. \nonumber
\end{eqnarray}

and $z$ is chosen distinct from $\quotep{P}$, $\quotep{Q}$, the free
names in $Q$, and all the names in $R$. Our $\alpha$-equivalence will
be built in the standard way from this substitution.

\begin{remark}\label{rem:no_self_referential_names}
  One consequence of these definitions is that $\forall P. \quotep{P}
  \not\in \freenames{P}$.
\end{remark}

\subsection{ Dynamic quote: an example }

Anticipating something of what's to come, consider applying the
substitution, $\widehat{\id{\{}u / z \id{\}}}$, to the following pair
of processes, $\lift{w}{y!(z)}$ and $w[ \lpquote y!(z) \rpquote ]$.

\begin{eqnarray}
	\lift{w}{y!(z)}\widehat{\id{\{}u / z \id{\}}}
		& = &
		\lift{w}{y!(u)} \nonumber\\
	w[ \lpquote y!(z) \rpquote ] \widehat{ \id{\{}u / z \id{\}} }
		& = &
		w[ \lpquote y!(z) \rpquote ] \nonumber
\end{eqnarray}

Because the body of the process between quotes is impervious to
substitution, we get radically different answers. In fact, by
examining the first process in an input context,
e.g. $x?(z).\lift{w}{y!(z)}$, we see that the process under the lift
operator may be shaped by prefixed inputs binding a name inside it. In
this sense, the lift operator will be seen as a way to dynamically
construct processes before reifying them as names.

Finally equipped with these standard features we can present the
dynamics of the calculus.

\subsubsection{Operational semantics} 

Finally, we introduce the computational dynamics. What marks these
algebras as distinct from other more traditionally studied algebraic
structures, e.g. vector spaces or polynomial rings, is the manner in
which dynamics is captured. In traditional structures, dynamics is typically
expressed through morphisms between such structures, as in linear maps
between vector spaces or morphisms between rings. In algebras
associated with the semantics of computation, the dynamics is
expressed as part of the algebraic structure itself, through a
reduction reduction relation typically denoted by $\red$. Below, we
give a recursive presentation of this relation for the calculus used
in the encoding.

$\red \subseteq \pi \times \pi$
$\red : \pi \to \mathcal{P}(\pi)$

\begin{mathpar}
  \inferrule* [lab=Comm] { \textsf{match}( x_{src}, x_{trgt} ) } { x_{trgt}?(y)P \; | \; x_{src}!\langle {Q} \rangle \red P\{\quotep{Q}/y}\} }
  \and \\
  \inferrule* [lab=Par] {{P} \red {P}'} {{{P} | {Q}} \red {{P}' | {Q}}}
  \and
  \inferrule* [lab=Equiv]{{{P} \scong {P}'} \andalso {{P}' \red {Q}'} \andalso {{Q}' \scong {Q}}}{{P} \red {Q}}
\end{mathpar}

\begin{eqnarray*}
  match_{\equiv} (\quotep{P},\quotep{Q}) & := & P \equiv Q \\
  match_{\dagger}(\quotep{P},\quotep{Q}) & := & \forall R. P|Q \red^{*} R => R \red^{*} 0 \\
  match_{K}(\quotep{P},\quotep{Q}) & := & K \mbox{ for some context } K
\end{eqnarray*}

$u?(x)P | u!\langle Q \rangle \red P\{\quotep{Q}/x\}$

%We write $\wred$ for $\red^*$, and $P\red$ if $\exists Q $ such that $ P \red Q$.
We write $P\red$ if $\exists Q $ such that $ P \red Q$ and $P\not\red$, otherwise.

\section{Replication}

As mentioned before, it is known that replication (and hence
recursion) can be implemented in a higher-order process algebra
\cite{SangiorgiWalker}. As our first example of calculation with the
machinery thus far presented we give the construction explicitly in
the {\rhoc}.

\begin{eqnarray}
	D_{x} & := & \prefix{x}{y}{(\binpar{\outputp{x}{y}}{@{y}})} \nonumber\\
	\bangp_{x}{P} & := & \binpar{{x}!\langle{\binpar{D_{x}}{P}}\rangle}{D_{x}} \nonumber
\end{eqnarray}

\begin{eqnarray}
	\bangp_{x}{P} & & \nonumber\\
	=
	& {x}!\langle{(\prefix{x}{y}{(\outputp{x}{y} | @{y})) | P}}\rangle 
	      | \prefix{x}{y}{(\outputp{x}{y} | @{y})} & \nonumber\\
	\red
	& (\outputp{x}{y} | @{y})\substn{\quotep{(\prefix{x}{y}{(@{y} | \outputp{x}{y})) | P}}}{y} & \nonumber\\
	=
	& \outputp{x}{\quotep{(\prefix{x}{y}{(\outputp{x}{y} | @{y})) | P}}}
	  | {(\prefix{x}{y}{(\outputp{x}{y} | @{y})) | P}} & \nonumber\\
	\red
	& \ldots & \nonumber\\
	\red^*
	& P | P | \ldots & \nonumber
\end{eqnarray}

Of course, this encoding, as an implementation, runs away, unfolding
$\bangp{P}$ eagerly. A lazier and more implementable replication
operator, restricted to input-guarded processes, may be obtained as follows.

\begin{eqnarray}
\bangp{\prefix{u}{v}{P}} 
	:= 
	\binpar{\lift{x}{\prefix{u}{v}{(\binpar{D(x)}{P})}}}{D(x)} \nonumber
\end{eqnarray}

\begin{remark}
  Note that the lazier definition still does not deal with summation
  or mixed summation (i.e. sums over input and output). The reader is
  invited to construct definitions of replication that deal with these
  features. 

  Further, the definitions are parameterized in a name, $x$. Can you,
  gentle reader, make a definition that eliminates this parameter and
  guarantees no accidental interaction between the replication
  machinery and the process being replicated -- i.e. no accidental
  sharing of names used by the process to get its work done and the
  name(s) used by the replication to effect copying. This latter
  revision of the definition of replication is crucial to obtaining
  the expected identity $!!P \sim !P$.
\end{remark}

\begin{remark}\label{rem:paradoxical_combinator}
  The reader familiar with the lambda calculus will have noticed the
  similarity between $D$ and the paradoxical combinator.

  [Ed. note: the existence of this seems to suggest we have to be more
  restrictive on the set of processes and names we admit if we are to
  support no-cloning.]
\end{remark}

\subsubsection{Bisimulation}

The computational dynamics gives rise to another kind of equivalence,
the equivalence of computational behavior. As previously mentioned
this is typically captured \emph{via} some form of bisimulation.

% The notion we use in this paper is weak barbed bisimulation
% \cite{milner91polyadicpi}.

The notion we use in this paper is derived from weak barbed
bisimulation \cite{milner91polyadicpi}. 

\begin{definition}
An \emph{observation relation}, $\downarrow_{\mathcal N}$, over a set
of names, $\mathcal N$, is the smallest relation satisfying the rules
below.

\infrule[Out-barb]{y \in {\mathcal N}, \; x \nameeq y}
		  {\outputp{x}{v} \downarrow_{\mathcal N} x}
\infrule[Par-barb]{\mbox{$P\downarrow_{\mathcal N} x$ or $Q\downarrow_{\mathcal N} x$}}
		  {\binpar{P}{Q} \downarrow_{\mathcal N} x}

We write $P \Downarrow_{\mathcal N} x$ if there is $Q$ such that 
$P \wred Q$ and $Q \downarrow_{\mathcal N} x$.
\end{definition}

\begin{definition}
%\label{def.bbisim}
An  ${\mathcal N}$-\emph{barbed bisimulation} over a set of names, ${\mathcal N}$, is a symmetric binary relation 
${\mathcal S}_{\mathcal N}$ between agents such that $P\rel{S}_{\mathcal N}Q$ implies:
\begin{enumerate}
\item If $P \red P'$ then $Q \wred Q'$ and $P'\rel{S}_{\mathcal N} Q'$.
\item If $P\downarrow_{\mathcal N} x$, then $Q\Downarrow_{\mathcal N} x$.
\end{enumerate}
$P$ is ${\mathcal N}$-barbed bisimilar to $Q$, written
$P \wbbisim_{\mathcal N} Q$, if $P \rel{S}_{\mathcal N} Q$ for some ${\mathcal N}$-barbed bisimulation ${\mathcal S}_{\mathcal N}$.
\end{definition}

$\mathcal{R} \subseteq \pi \times \pi$

$P \mathcal{R} Q => \forall P'. P \red P' \Rightarrow \exists Q'. Q \red Q', P' \mathcal{R} Q'$

$P \vdash x \Rightarrow Q \vdash x$

\begin{mathpar}
  \inferrule*[lab=Out-barb]{x \nameeq y}{{y}!\langle{Q}\rangle \vdash x}
  \and
  \inferrule*[lab=Par-barb]{\mbox{$P\vdash x$ or $Q\vdash x$}}{\binpar{P}{Q} \vdash x}
\end{mathpar}

\subsubsection{Contexts}

One of the principle advantages of computational calculi like the
$\pi$-calculus is a well-defined notion of context,
contextual-equivalence and a correlation between
contextual-equivalence and notions of bisimulation. The notion of
context allows the decomposition of a process into (sub-)process and
its syntactic environment, its context. Thus, a context may be
thought of as a process with a ``hole'' (written $\Box$) in it. The
application of a context $M$ to a process $P$, written $M[P]$, is
tantamount to filling the hole in $M$ with $P$. In this paper we do
not need the full weight of this theory, but do make use of the notion
of context in the proof the main theorem. 

\begin{mathpar}
  \inferrule* [lab=summation] {} {{M_{M},M_{N}} \bc \Box \;|\; x.M_{A} \;|\; M_{M}+M_{N}}
  \and
  \inferrule* [lab=agent] {} {{M_{A}} \bc (\vec{x})M_{P} \;| \; \clift{P_0,\ldots,M_{P},\ldots,P_N}}
  \and \\
  \inferrule* [lab=process] {} {{M_{P}} \bc M_{N} \;| \;P|M_{P} }
\end{mathpar} 

\begin{mathpar}
  \inferrule* [lab=sychronization] {} {M_{N} \bc \Box \;|\; x?M_{F} \;|\; x!M_{C}}
  \and
  \inferrule* [lab=abstraction] {} {{M_{F}} \bc (x)M_{P} }
  \and
  \inferrule* [lab=concretion] {} {{M_{C}} \bc \langle M_{P} \rangle }
  \and \\
  \inferrule* [lab=process] {} {{M_{P}} \bc M_{N} \;| \;P|M_{P} }
\end{mathpar}

\begin{definition}[contextual application] Given a context $M$, and
  process $P$, we define the \emph{contextual application}, $M[P] :=
  M\{P/\Box\}$. That is, the contextual application of M to P is the
  substitution of $P$ for $\Box$ in $M$.
\end{definition}

$\meaningof{-} : L \to \mathcal{P}(\pi)$

\begin{mathpar}
  \inferrule* [lab=collection] {} {\meaningof{true} = \pi, \and \meaningof{~E} = \pi \setminus \meaningof{E}, \and \meaningof{E_{1} \& E_{2}} = \meaningof{E_{1}} \cap \meaningof{E_{2}}}
\end{mathpar}

\begin{mathpar}
  \inferrule* [lab=structure] {} {\meaningof{0} = \{ P \in \pi | P \equiv 0 \}, \and \\ \meaningof{E_1 | E_2} = \{ P \in \pi | P \equiv P_{1} | P_{2}, P_{1} \in \meaningof{E_{1}}, P_{2} \in \meaningof{E_2}\} }
\end{mathpar}

\begin{mathpar}
 \inferrule* [lab=behavior] {} {\meaningof{\langle a?b \rangle E} = \{ P \in \pi | P \equiv Q | u?(y)P', \\ \and \\\\ \and \\ \;\;\; u \in \meaningof{a}, \forall z.P'\{z/y\} \in \meaningof{E\{z/b\}}\}, \and \\ \meaningof{a!E} = \{ P \in \pi | P \equiv Q | x!\langle P' \rangle, x \in \meaningof{a} P' \in \meaningof{E}\} }
\end{mathpar}

\begin{mathpar}
 \inferrule* [lab=nominal] {} {\meaningof{\quotep{E}} = \{ \quotep{P} \in \quotep{\pi} | P \in \meaningof{E} \}, \and \meaningof{\quotep{P}} = \{ \quotep{Q} \in \quotep{\pi} | P \equiv Q \} \and \\ \meaningof{@\quotep{E}} = \{ P \in \pi | P \equiv @x, x \in \meaningof{E} \}}
\end{mathpar}

\begin{eqnarray*}
  \\
  \meaningof{-} : TS \to ST
\end{eqnarray*}

\begin{eqnarray*}
  \\
  L : TS \to ST
\end{eqnarray*}

\begin{eqnarray*}
  \\
  P \models E \iff P \in \meaningof{E}
\end{eqnarray*}

\begin{eqnarray*}
  P \approx_{L} Q \iff \forall E \in L. P \models E \iff Q \models E
\end{eqnarray*}

\begin{eqnarray*}
  P \approx_{K} Q
\end{eqnarray*}

\begin{eqnarray*}
  P \approx Q
\end{eqnarray*}

$\approx_{K} = \approx = \approx_{L}$

\subsubsection{Contextual duality}

Note that contexts extend the quotation operation to a family of
operations from processes to names. Given a context, $M$, we can
define a \emph{nominal context}, $\quotep{M}$ by $\quotep{M}[P] :=
\quotep{M[P]}$. To foreshadow what is to come we observe that these
operations enjoy a duality with processes very much like the duality
between vectors and maps from vectors to scalars.

Further, because the calculus is essentially higher-order, we have a
correspondence between contexts and processes. More specifically,
given a name $x$ and a context $M$ we can construct $M^{*}_{x}$ such
that 

\begin{mathpar}
  M^{*}_{x} | \lift{x}{P} \red M[P]
\end{mathpar}

namely,

\begin{mathpar}
  M^{*}_{x} := x?(u).M[\dropn{u}]
\end{mathpar}

The dependence of $M^{*}_{x}$ on a name makes it an abstraction, 

\begin{mathpar}
  M^{*} := (x)x?(u).M[\dropn{u}]
\end{mathpar}

\subsection{Additional notation}

It will sometimes be convenient to denote the process a name
quotes. We already have the notation $x = \quotep{P}$, but it will be
convenient to introduce an alternate notation, $\procn{x}$, when we
want to emphasize the connection to the use of the name. Note that, by
virtue of name equivalence, $\quotep{\procn{x}} \nameeq x$; so, the
notation is consistent with previous definitions.

Further, because names have structure it is possible to effect
substitutions on the basis of that structure. This means we need to
upgrade our notation for substitutions, which we accomplish by
adapting comprehension notation. Thus,

\begin{mathpar}
  P\{ y / x : x \in S \}
\end{mathpar}

is interpreted to mean the process derived from P by replacing (in a
capture-avoiding manner) each occurrence of $x$ in $S$ by $y$. For example,

\begin{mathpar}
  P\{ \quotep{\procn{x}|\procn{x}} / x : x \in \freenames{P} \}
\end{mathpar}

will replace each (occurrence) of a free name $x$ in $P$ by
$\quotep{\procn{x}|\procn{x}}$.

Also, we will avail ourselves of the notation $x^{L}$ and $x^{R}$ to
denote injections of a name into disjoint copies of the name
space. There are numerous ways to accomplish this. One example can be
found in \cite{MeredithR05}. This notation overloads to vectors of
names: $\vec{x}^{\pi} := (x_{i}^{\pi} \; : \; 0 \leq i < |\vec{x}| )$ where $\pi \in \{L,R\}$.

We also use $P^{\Box} := P|\Box$.

In \cite{MeredithR05} an interpretation of the new operator is
given. It turns out that there are several possible interpretations
all enjoying the requisite algebraic properties of the operator (see
\cite{milner91polyadicpi}). We will therefore make liberal use of
$(\nu\; \vec{x})P$.

% subsection the_syntax_and_semantics_of_the_notation_system (end)   

\input{qm2pi.qmops} 

\input{qm2pi.sterngerlach} 

\input{qm2pi.metric} 

% section concurrent_process_calculi (end)

%\input{qm2pi.proofsketch}

% section proof sketch (end)

%\input{qm2pi.slviaknots} 

% section spatial logic via knots (end)

\input{qm2pi.conclusion}

% section conclusion (end)

%\input{qm2pi.dtcodes} 

% section wiring algorithm (end)

\input{qm2pi.ack} 

% section acknowledgments (end)

\newpage


\bibliographystyle{plain}   
\bibliography{../../biblios/main.bib}

\input{qm2pi.rhodetails}

\end{document}

 

% section notation (end)

\input{qm2pi.process.calculi} 

% section concurrent_process_calculi_and_spatial_logics_ (end)
    
%\documentclass[12pt]{llncs}
%\documentclass{jktr}

\usepackage[pdftex]{hyperref}                   
\usepackage {listings}
\usepackage {mathpartir}
\usepackage{bcprules}
%\usepackage{listings}
                       
\usepackage{graphicx} 
%\usepackage[margins=2.5cm,nohead,nofoot]{geometry}
%\usepackage{geometry}
\usepackage{amsfonts}
\usepackage{amstext}
\usepackage{latexsym}
\usepackage{amssymb}
\usepackage{color}


%\include{myPreamble}
\include{qm2pi.local} 

%\ifpdf
%\usepackage[pdftex]{graphicx}
%\else
%\usepackage{graphicx}
%\fi

 % \ifpdf
%  \usepackage{pdfsync}
%  \if


%\title{Brief Article}
%\author{David F. Snyder}
%\author{L.G. Meredith}

%\address{Dept. of Math., Texas State University--San Marcos, San Marcos, TX 78666}
       
\pagestyle{empty}


\begin{document}

\lstset{language=[Objective]Caml,frame=shadowbox}

\input{qm2pi.front}

% section front matter (end)

\input{qm2pi.intro} 
 
% section introduction (end)

% \input{qm2pi.knotations} 

% section notation (end)

\input{qm2pi.process.calculi} 

% section concurrent_process_calculi_and_spatial_logics_ (end)
    
%\input{qm2pi.knots2pi} 

%\input{qm2pi.trefoil} 

%\input{qm2pi.mainthm} 

% subsection basic_interpretation (end)

%\input{qm2pi.rho.presentation} 
\subsection{The syntax and semantics of the notation system}\label{sub:the_syntax_and_semantics_of_the_notation_system} % (fold)

We now summarize a technical presentation of the calculus that
embodies our theory of dynamics. The typical presentation of such a
calculus follows the style of giving generators and relations on
them. The grammar, below, describing term constructors, freely
generates the set of processes, $\Proc$. This set is then quotiented
by a relation known as structural congruence and it is over this set
that the notion of dynamics is expressed. This presentation is
essentially that of \cite{MeredithR05} with the addition of
polyadicity and summation. For readability we have relegated some of
the technical subtleties to an appendix.

\subsubsection{Process grammar}\label{subsub:process_grammar}

\begin{mathpar}
  \inferrule* [lab=synchronization] {} {{M} \bc \pzero \;|\; x?F \;|\; x!C }
  \and
  \inferrule* [lab=abstraction] {} {{F} \bc (x)P}
  \and
  \inferrule* [lab=concretion] {} {{C} \bc \langle Q \rangle}
  \and
  \inferrule* [lab=process] {} {{P,Q} \bc M \;| \;P|Q \;|\; @{x}}
  \and
  \inferrule* [lab=name] {} {{x} \bc \quotep{P}}
\end{mathpar} 

Note that $\vec{x}$ (resp. $\vec{P}$) denotes a vector of names
(resp. processes) of length $|\vec{x}|$ (resp. $|\vec{P}|$). We adopt
the following useful abbreviations.

\begin{mathpar}
   x?(\vec{y}).P := x.(\vec{y})P \and  x\clift{\vec{P}} := x.\clift{\vec{P}}
   \and x!(y) := \lift{x}{\dropn{y}}
   \and \Pi_{i=0}^{n-1}P_i := P_0 | \ldots | P_{n-1}
\end{mathpar}

\subsubsection{Structural congruence}

\paragraph{Free and bound names and alpha-equivalence.} At the
core of structural equivalence is alpha-equivalence which identifies
process that are the same up to a change of variable. Formally, we
recognize the distinction between free and bound names. The free names
of a process, $\freenames{P}$, may be calculated recursively as
follows:

\begin{mathpar}
\freenames{\pzero} := \emptyset
  \and \\
  \freenames{x?(y).P} := \{ x \} \cup (\freenames{P} \setminus \{ y \})
  \and 
  \freenames{x!\langle P \rangle} := \{ x \} \cup \{ P \} 
  \and \\
  \freenames{P|Q} := \freenames{P} \cup \freenames{Q}
  \and \\
  \freenames{@{x}} := \{ x \}
\end{mathpar}

$\pi$
$\quotep{\pi}$

$\freenames{-} : \pi \to \mathcal{P}(\quotep{\pi})$

\begin{eqnarray*}
  \freenames{\pzero} & := & \emptyset \\
  \freenames{x?(y).P} & := & \{ x \} \cup (\freenames{P} \setminus \{ y \}) \\
  \freenames{x!\langle P \rangle} & := & \{ x \} \cup \{ P \} \\
  \freenames{P|Q} & := & \freenames{P} \cup \freenames{Q} \\
  \freenames{\dropn{x}} & := & \{ x \}
\end{eqnarray*}

The bound names of a process, $\boundnames{P}$, are those names occurring in $P$
that are not free. For example, in $x?(y).0$, the name $x$ is free, while $y$ is bound.

\begin{mathpar}
  \inferrule* [lab=monoidal-laws] {} { P|Q \equiv Q|P \and P|0 \equiv P \and P|(Q|R) \equiv (P|Q)|R }
\end{mathpar}

\begin{mathpar}
  \inferrule* [lab=alpha-equivalence] {} { (x)P \equiv (y)P\{y/x\} \and y \not\in \freenames{P} }
\end{mathpar}

\begin{definition}
Then two processes, $P,Q$, are alpha-equivalent if $P = Q\{\vec{y}/\vec{x}\}$ for
some $\vec{x} \in \boundnames{Q},\vec{y} \in \boundnames{P}$, where $Q\{\vec{y}/\vec{x}\}$
denotes the capture-avoiding substitution of $\vec{y}$ for $\vec{x}$ in $Q$.
\end{definition}

\begin{definition}
  The {\em structural congruence} \cite{SangiorgiWalker} , $\equiv$,
  between processes is the least congruence containing
  alpha-equivalence, satisfying the abelian monoid laws
  (associativity, commutativity and $\pzero$ as identity) for parallel
  composition $|$ and for summation $+$.
\end{definition}

\subsection{Name equivalence}

We take name equivalence, written $\nameeq$, to be the smallest
equivalence relation generated by the following rules.

\begin{mathpar}
\inferrule*[lab=Quote-drop]
{ }
{ \quotep{@{x}} \nameeq x }

\inferrule*[lab=Struct-equiv]
{ P \scong Q }
{ \quotep{P} \nameeq \quotep{Q} }
\end{mathpar}

The astute reader will have noticed that the mutual recursion of names
and processes imposes a mutual recursion on alpha-equivalence and
structural equivalence via name-equivalence. Fortunately, all of this
works out pleasantly and we may calculate in the natural way, free of
concern. The reader interested in the details is referred to the
appendix \ref{appendix:rho_details}.

\subsection{Substitution}

We use $\Proc$ for the set of processes, $\QProc$ for the set of
names, and $\id{\{}\vec{y} / \vec{x} \id{\}}$ to denote partial maps,
$s : \QProc \rightarrow \QProc$. A map, $s$ lifts, uniquely, to a map
on process terms, $\widehat{s} : \Proc \rightarrow \Proc$ by the
following equations.

\begin{mathpar}
  (0) \psubstp{Q}{P} := 0 \\
  (R \juxtap S) \psubstp{Q}{P}
  :=    
  (R)\psubstp{Q}{P} \juxtap (S) \psubstp{Q}{P} \\
  (x?(y).R) \psubstp{Q}{P}    
  :=    
  (x)\substp{Q}{P} (z)\concat( (R \psubstn{z}{y}) \psubstp{Q}{P} ) \\
  (\lift{x}{R}) \psubstp{Q}{P}  
  :=
  \lift{(x)\substp{Q}{P}}{ R \psubstp{Q}{P} } \\
%   (\dropn{x})  \psubstp{Q}{P}       
%   := 
%   \left\{ 
%     \begin{array}{ccc} 
%       \dropn{\quotep{Q}} & & x \nameeq \quotep{P} \\
%       \dropn{x} & & otherwise \\
%     \end{array}
%   \right. 
  (\dropn{x})  \psubstp{Q}{P}       
  := 
  \left\{ 
    \begin{array}{ccc} 
      Q & & x \nameeq \quotep{P} \\
      \dropn{x} & & otherwise \\
    \end{array}
  \right.
\end{mathpar}
 

where

\begin{eqnarray}
  (x)\id{\{} \lpquote Q \rpquote / \lpquote P \rpquote \id{\}}            = 
  \left\{ 
    \begin{array}{ccc}
      \lpquote Q \rpquote & & x \nameeq \lpquote P \rpquote \\
      x & & otherwise \\
    \end{array}
  \right. \nonumber
\end{eqnarray}

and $z$ is chosen distinct from $\quotep{P}$, $\quotep{Q}$, the free
names in $Q$, and all the names in $R$. Our $\alpha$-equivalence will
be built in the standard way from this substitution.

\begin{remark}\label{rem:no_self_referential_names}
  One consequence of these definitions is that $\forall P. \quotep{P}
  \not\in \freenames{P}$.
\end{remark}

\subsection{ Dynamic quote: an example }

Anticipating something of what's to come, consider applying the
substitution, $\widehat{\id{\{}u / z \id{\}}}$, to the following pair
of processes, $\lift{w}{y!(z)}$ and $w[ \lpquote y!(z) \rpquote ]$.

\begin{eqnarray}
	\lift{w}{y!(z)}\widehat{\id{\{}u / z \id{\}}}
		& = &
		\lift{w}{y!(u)} \nonumber\\
	w[ \lpquote y!(z) \rpquote ] \widehat{ \id{\{}u / z \id{\}} }
		& = &
		w[ \lpquote y!(z) \rpquote ] \nonumber
\end{eqnarray}

Because the body of the process between quotes is impervious to
substitution, we get radically different answers. In fact, by
examining the first process in an input context,
e.g. $x?(z).\lift{w}{y!(z)}$, we see that the process under the lift
operator may be shaped by prefixed inputs binding a name inside it. In
this sense, the lift operator will be seen as a way to dynamically
construct processes before reifying them as names.

Finally equipped with these standard features we can present the
dynamics of the calculus.

\subsubsection{Operational semantics} 

Finally, we introduce the computational dynamics. What marks these
algebras as distinct from other more traditionally studied algebraic
structures, e.g. vector spaces or polynomial rings, is the manner in
which dynamics is captured. In traditional structures, dynamics is typically
expressed through morphisms between such structures, as in linear maps
between vector spaces or morphisms between rings. In algebras
associated with the semantics of computation, the dynamics is
expressed as part of the algebraic structure itself, through a
reduction reduction relation typically denoted by $\red$. Below, we
give a recursive presentation of this relation for the calculus used
in the encoding.

$\red \subseteq \pi \times \pi$
$\red : \pi \to \mathcal{P}(\pi)$

\begin{mathpar}
  \inferrule* [lab=Comm] { \textsf{match}( x_{src}, x_{trgt} ) } { x_{trgt}?(y)P \; | \; x_{src}!\langle {Q} \rangle \red P\{\quotep{Q}/y}\} }
  \and \\
  \inferrule* [lab=Par] {{P} \red {P}'} {{{P} | {Q}} \red {{P}' | {Q}}}
  \and
  \inferrule* [lab=Equiv]{{{P} \scong {P}'} \andalso {{P}' \red {Q}'} \andalso {{Q}' \scong {Q}}}{{P} \red {Q}}
\end{mathpar}

\begin{eqnarray*}
  match_{\equiv} (\quotep{P},\quotep{Q}) & := & P \equiv Q \\
  match_{\dagger}(\quotep{P},\quotep{Q}) & := & \forall R. P|Q \red^{*} R => R \red^{*} 0 \\
  match_{K}(\quotep{P},\quotep{Q}) & := & K \mbox{ for some context } K
\end{eqnarray*}

$u?(x)P | u!\langle Q \rangle \red P\{\quotep{Q}/x\}$

%We write $\wred$ for $\red^*$, and $P\red$ if $\exists Q $ such that $ P \red Q$.
We write $P\red$ if $\exists Q $ such that $ P \red Q$ and $P\not\red$, otherwise.

\section{Replication}

As mentioned before, it is known that replication (and hence
recursion) can be implemented in a higher-order process algebra
\cite{SangiorgiWalker}. As our first example of calculation with the
machinery thus far presented we give the construction explicitly in
the {\rhoc}.

\begin{eqnarray}
	D_{x} & := & \prefix{x}{y}{(\binpar{\outputp{x}{y}}{@{y}})} \nonumber\\
	\bangp_{x}{P} & := & \binpar{{x}!\langle{\binpar{D_{x}}{P}}\rangle}{D_{x}} \nonumber
\end{eqnarray}

\begin{eqnarray}
	\bangp_{x}{P} & & \nonumber\\
	=
	& {x}!\langle{(\prefix{x}{y}{(\outputp{x}{y} | @{y})) | P}}\rangle 
	      | \prefix{x}{y}{(\outputp{x}{y} | @{y})} & \nonumber\\
	\red
	& (\outputp{x}{y} | @{y})\substn{\quotep{(\prefix{x}{y}{(@{y} | \outputp{x}{y})) | P}}}{y} & \nonumber\\
	=
	& \outputp{x}{\quotep{(\prefix{x}{y}{(\outputp{x}{y} | @{y})) | P}}}
	  | {(\prefix{x}{y}{(\outputp{x}{y} | @{y})) | P}} & \nonumber\\
	\red
	& \ldots & \nonumber\\
	\red^*
	& P | P | \ldots & \nonumber
\end{eqnarray}

Of course, this encoding, as an implementation, runs away, unfolding
$\bangp{P}$ eagerly. A lazier and more implementable replication
operator, restricted to input-guarded processes, may be obtained as follows.

\begin{eqnarray}
\bangp{\prefix{u}{v}{P}} 
	:= 
	\binpar{\lift{x}{\prefix{u}{v}{(\binpar{D(x)}{P})}}}{D(x)} \nonumber
\end{eqnarray}

\begin{remark}
  Note that the lazier definition still does not deal with summation
  or mixed summation (i.e. sums over input and output). The reader is
  invited to construct definitions of replication that deal with these
  features. 

  Further, the definitions are parameterized in a name, $x$. Can you,
  gentle reader, make a definition that eliminates this parameter and
  guarantees no accidental interaction between the replication
  machinery and the process being replicated -- i.e. no accidental
  sharing of names used by the process to get its work done and the
  name(s) used by the replication to effect copying. This latter
  revision of the definition of replication is crucial to obtaining
  the expected identity $!!P \sim !P$.
\end{remark}

\begin{remark}\label{rem:paradoxical_combinator}
  The reader familiar with the lambda calculus will have noticed the
  similarity between $D$ and the paradoxical combinator.

  [Ed. note: the existence of this seems to suggest we have to be more
  restrictive on the set of processes and names we admit if we are to
  support no-cloning.]
\end{remark}

\subsubsection{Bisimulation}

The computational dynamics gives rise to another kind of equivalence,
the equivalence of computational behavior. As previously mentioned
this is typically captured \emph{via} some form of bisimulation.

% The notion we use in this paper is weak barbed bisimulation
% \cite{milner91polyadicpi}.

The notion we use in this paper is derived from weak barbed
bisimulation \cite{milner91polyadicpi}. 

\begin{definition}
An \emph{observation relation}, $\downarrow_{\mathcal N}$, over a set
of names, $\mathcal N$, is the smallest relation satisfying the rules
below.

\infrule[Out-barb]{y \in {\mathcal N}, \; x \nameeq y}
		  {\outputp{x}{v} \downarrow_{\mathcal N} x}
\infrule[Par-barb]{\mbox{$P\downarrow_{\mathcal N} x$ or $Q\downarrow_{\mathcal N} x$}}
		  {\binpar{P}{Q} \downarrow_{\mathcal N} x}

We write $P \Downarrow_{\mathcal N} x$ if there is $Q$ such that 
$P \wred Q$ and $Q \downarrow_{\mathcal N} x$.
\end{definition}

\begin{definition}
%\label{def.bbisim}
An  ${\mathcal N}$-\emph{barbed bisimulation} over a set of names, ${\mathcal N}$, is a symmetric binary relation 
${\mathcal S}_{\mathcal N}$ between agents such that $P\rel{S}_{\mathcal N}Q$ implies:
\begin{enumerate}
\item If $P \red P'$ then $Q \wred Q'$ and $P'\rel{S}_{\mathcal N} Q'$.
\item If $P\downarrow_{\mathcal N} x$, then $Q\Downarrow_{\mathcal N} x$.
\end{enumerate}
$P$ is ${\mathcal N}$-barbed bisimilar to $Q$, written
$P \wbbisim_{\mathcal N} Q$, if $P \rel{S}_{\mathcal N} Q$ for some ${\mathcal N}$-barbed bisimulation ${\mathcal S}_{\mathcal N}$.
\end{definition}

$\mathcal{R} \subseteq \pi \times \pi$

$P \mathcal{R} Q => \forall P'. P \red P' \Rightarrow \exists Q'. Q \red Q', P' \mathcal{R} Q'$

$P \vdash x \Rightarrow Q \vdash x$

\begin{mathpar}
  \inferrule*[lab=Out-barb]{x \nameeq y}{{y}!\langle{Q}\rangle \vdash x}
  \and
  \inferrule*[lab=Par-barb]{\mbox{$P\vdash x$ or $Q\vdash x$}}{\binpar{P}{Q} \vdash x}
\end{mathpar}

\subsubsection{Contexts}

One of the principle advantages of computational calculi like the
$\pi$-calculus is a well-defined notion of context,
contextual-equivalence and a correlation between
contextual-equivalence and notions of bisimulation. The notion of
context allows the decomposition of a process into (sub-)process and
its syntactic environment, its context. Thus, a context may be
thought of as a process with a ``hole'' (written $\Box$) in it. The
application of a context $M$ to a process $P$, written $M[P]$, is
tantamount to filling the hole in $M$ with $P$. In this paper we do
not need the full weight of this theory, but do make use of the notion
of context in the proof the main theorem. 

\begin{mathpar}
  \inferrule* [lab=summation] {} {{M_{M},M_{N}} \bc \Box \;|\; x.M_{A} \;|\; M_{M}+M_{N}}
  \and
  \inferrule* [lab=agent] {} {{M_{A}} \bc (\vec{x})M_{P} \;| \; \clift{P_0,\ldots,M_{P},\ldots,P_N}}
  \and \\
  \inferrule* [lab=process] {} {{M_{P}} \bc M_{N} \;| \;P|M_{P} }
\end{mathpar} 

\begin{mathpar}
  \inferrule* [lab=sychronization] {} {M_{N} \bc \Box \;|\; x?M_{F} \;|\; x!M_{C}}
  \and
  \inferrule* [lab=abstraction] {} {{M_{F}} \bc (x)M_{P} }
  \and
  \inferrule* [lab=concretion] {} {{M_{C}} \bc \langle M_{P} \rangle }
  \and \\
  \inferrule* [lab=process] {} {{M_{P}} \bc M_{N} \;| \;P|M_{P} }
\end{mathpar}

\begin{definition}[contextual application] Given a context $M$, and
  process $P$, we define the \emph{contextual application}, $M[P] :=
  M\{P/\Box\}$. That is, the contextual application of M to P is the
  substitution of $P$ for $\Box$ in $M$.
\end{definition}

$\meaningof{-} : L \to \mathcal{P}(\pi)$

\begin{mathpar}
  \inferrule* [lab=collection] {} {\meaningof{true} = \pi, \and \meaningof{~E} = \pi \setminus \meaningof{E}, \and \meaningof{E_{1} \& E_{2}} = \meaningof{E_{1}} \cap \meaningof{E_{2}}}
\end{mathpar}

\begin{mathpar}
  \inferrule* [lab=structure] {} {\meaningof{0} = \{ P \in \pi | P \equiv 0 \}, \and \\ \meaningof{E_1 | E_2} = \{ P \in \pi | P \equiv P_{1} | P_{2}, P_{1} \in \meaningof{E_{1}}, P_{2} \in \meaningof{E_2}\} }
\end{mathpar}

\begin{mathpar}
 \inferrule* [lab=behavior] {} {\meaningof{\langle a?b \rangle E} = \{ P \in \pi | P \equiv Q | u?(y)P', \\ \and \\\\ \and \\ \;\;\; u \in \meaningof{a}, \forall z.P'\{z/y\} \in \meaningof{E\{z/b\}}\}, \and \\ \meaningof{a!E} = \{ P \in \pi | P \equiv Q | x!\langle P' \rangle, x \in \meaningof{a} P' \in \meaningof{E}\} }
\end{mathpar}

\begin{mathpar}
 \inferrule* [lab=nominal] {} {\meaningof{\quotep{E}} = \{ \quotep{P} \in \quotep{\pi} | P \in \meaningof{E} \}, \and \meaningof{\quotep{P}} = \{ \quotep{Q} \in \quotep{\pi} | P \equiv Q \} \and \\ \meaningof{@\quotep{E}} = \{ P \in \pi | P \equiv @x, x \in \meaningof{E} \}}
\end{mathpar}

\begin{eqnarray*}
  \\
  \meaningof{-} : TS \to ST
\end{eqnarray*}

\begin{eqnarray*}
  \\
  L : TS \to ST
\end{eqnarray*}

\begin{eqnarray*}
  \\
  P \models E \iff P \in \meaningof{E}
\end{eqnarray*}

\begin{eqnarray*}
  P \approx_{L} Q \iff \forall E \in L. P \models E \iff Q \models E
\end{eqnarray*}

\begin{eqnarray*}
  P \approx_{K} Q
\end{eqnarray*}

\begin{eqnarray*}
  P \approx Q
\end{eqnarray*}

$\approx_{K} = \approx = \approx_{L}$

\subsubsection{Contextual duality}

Note that contexts extend the quotation operation to a family of
operations from processes to names. Given a context, $M$, we can
define a \emph{nominal context}, $\quotep{M}$ by $\quotep{M}[P] :=
\quotep{M[P]}$. To foreshadow what is to come we observe that these
operations enjoy a duality with processes very much like the duality
between vectors and maps from vectors to scalars.

Further, because the calculus is essentially higher-order, we have a
correspondence between contexts and processes. More specifically,
given a name $x$ and a context $M$ we can construct $M^{*}_{x}$ such
that 

\begin{mathpar}
  M^{*}_{x} | \lift{x}{P} \red M[P]
\end{mathpar}

namely,

\begin{mathpar}
  M^{*}_{x} := x?(u).M[\dropn{u}]
\end{mathpar}

The dependence of $M^{*}_{x}$ on a name makes it an abstraction, 

\begin{mathpar}
  M^{*} := (x)x?(u).M[\dropn{u}]
\end{mathpar}

\subsection{Additional notation}

It will sometimes be convenient to denote the process a name
quotes. We already have the notation $x = \quotep{P}$, but it will be
convenient to introduce an alternate notation, $\procn{x}$, when we
want to emphasize the connection to the use of the name. Note that, by
virtue of name equivalence, $\quotep{\procn{x}} \nameeq x$; so, the
notation is consistent with previous definitions.

Further, because names have structure it is possible to effect
substitutions on the basis of that structure. This means we need to
upgrade our notation for substitutions, which we accomplish by
adapting comprehension notation. Thus,

\begin{mathpar}
  P\{ y / x : x \in S \}
\end{mathpar}

is interpreted to mean the process derived from P by replacing (in a
capture-avoiding manner) each occurrence of $x$ in $S$ by $y$. For example,

\begin{mathpar}
  P\{ \quotep{\procn{x}|\procn{x}} / x : x \in \freenames{P} \}
\end{mathpar}

will replace each (occurrence) of a free name $x$ in $P$ by
$\quotep{\procn{x}|\procn{x}}$.

Also, we will avail ourselves of the notation $x^{L}$ and $x^{R}$ to
denote injections of a name into disjoint copies of the name
space. There are numerous ways to accomplish this. One example can be
found in \cite{MeredithR05}. This notation overloads to vectors of
names: $\vec{x}^{\pi} := (x_{i}^{\pi} \; : \; 0 \leq i < |\vec{x}| )$ where $\pi \in \{L,R\}$.

We also use $P^{\Box} := P|\Box$.

In \cite{MeredithR05} an interpretation of the new operator is
given. It turns out that there are several possible interpretations
all enjoying the requisite algebraic properties of the operator (see
\cite{milner91polyadicpi}). We will therefore make liberal use of
$(\nu\; \vec{x})P$.

% subsection the_syntax_and_semantics_of_the_notation_system (end)   

\input{qm2pi.qmops} 

\input{qm2pi.sterngerlach} 

\input{qm2pi.metric} 

% section concurrent_process_calculi (end)

%\input{qm2pi.proofsketch}

% section proof sketch (end)

%\input{qm2pi.slviaknots} 

% section spatial logic via knots (end)

\input{qm2pi.conclusion}

% section conclusion (end)

%\input{qm2pi.dtcodes} 

% section wiring algorithm (end)

\input{qm2pi.ack} 

% section acknowledgments (end)

\newpage


\bibliographystyle{plain}   
\bibliography{../../biblios/main.bib}

\input{qm2pi.rhodetails}

\end{document}

 

%\documentclass[12pt]{llncs}
%\documentclass{jktr}

\usepackage[pdftex]{hyperref}                   
\usepackage {listings}
\usepackage {mathpartir}
\usepackage{bcprules}
%\usepackage{listings}
                       
\usepackage{graphicx} 
%\usepackage[margins=2.5cm,nohead,nofoot]{geometry}
%\usepackage{geometry}
\usepackage{amsfonts}
\usepackage{amstext}
\usepackage{latexsym}
\usepackage{amssymb}
\usepackage{color}


%\include{myPreamble}
\include{qm2pi.local} 

%\ifpdf
%\usepackage[pdftex]{graphicx}
%\else
%\usepackage{graphicx}
%\fi

 % \ifpdf
%  \usepackage{pdfsync}
%  \if


%\title{Brief Article}
%\author{David F. Snyder}
%\author{L.G. Meredith}

%\address{Dept. of Math., Texas State University--San Marcos, San Marcos, TX 78666}
       
\pagestyle{empty}


\begin{document}

\lstset{language=[Objective]Caml,frame=shadowbox}

\input{qm2pi.front}

% section front matter (end)

\input{qm2pi.intro} 
 
% section introduction (end)

% \input{qm2pi.knotations} 

% section notation (end)

\input{qm2pi.process.calculi} 

% section concurrent_process_calculi_and_spatial_logics_ (end)
    
%\input{qm2pi.knots2pi} 

%\input{qm2pi.trefoil} 

%\input{qm2pi.mainthm} 

% subsection basic_interpretation (end)

%\input{qm2pi.rho.presentation} 
\subsection{The syntax and semantics of the notation system}\label{sub:the_syntax_and_semantics_of_the_notation_system} % (fold)

We now summarize a technical presentation of the calculus that
embodies our theory of dynamics. The typical presentation of such a
calculus follows the style of giving generators and relations on
them. The grammar, below, describing term constructors, freely
generates the set of processes, $\Proc$. This set is then quotiented
by a relation known as structural congruence and it is over this set
that the notion of dynamics is expressed. This presentation is
essentially that of \cite{MeredithR05} with the addition of
polyadicity and summation. For readability we have relegated some of
the technical subtleties to an appendix.

\subsubsection{Process grammar}\label{subsub:process_grammar}

\begin{mathpar}
  \inferrule* [lab=synchronization] {} {{M} \bc \pzero \;|\; x?F \;|\; x!C }
  \and
  \inferrule* [lab=abstraction] {} {{F} \bc (x)P}
  \and
  \inferrule* [lab=concretion] {} {{C} \bc \langle Q \rangle}
  \and
  \inferrule* [lab=process] {} {{P,Q} \bc M \;| \;P|Q \;|\; @{x}}
  \and
  \inferrule* [lab=name] {} {{x} \bc \quotep{P}}
\end{mathpar} 

Note that $\vec{x}$ (resp. $\vec{P}$) denotes a vector of names
(resp. processes) of length $|\vec{x}|$ (resp. $|\vec{P}|$). We adopt
the following useful abbreviations.

\begin{mathpar}
   x?(\vec{y}).P := x.(\vec{y})P \and  x\clift{\vec{P}} := x.\clift{\vec{P}}
   \and x!(y) := \lift{x}{\dropn{y}}
   \and \Pi_{i=0}^{n-1}P_i := P_0 | \ldots | P_{n-1}
\end{mathpar}

\subsubsection{Structural congruence}

\paragraph{Free and bound names and alpha-equivalence.} At the
core of structural equivalence is alpha-equivalence which identifies
process that are the same up to a change of variable. Formally, we
recognize the distinction between free and bound names. The free names
of a process, $\freenames{P}$, may be calculated recursively as
follows:

\begin{mathpar}
\freenames{\pzero} := \emptyset
  \and \\
  \freenames{x?(y).P} := \{ x \} \cup (\freenames{P} \setminus \{ y \})
  \and 
  \freenames{x!\langle P \rangle} := \{ x \} \cup \{ P \} 
  \and \\
  \freenames{P|Q} := \freenames{P} \cup \freenames{Q}
  \and \\
  \freenames{@{x}} := \{ x \}
\end{mathpar}

$\pi$
$\quotep{\pi}$

$\freenames{-} : \pi \to \mathcal{P}(\quotep{\pi})$

\begin{eqnarray*}
  \freenames{\pzero} & := & \emptyset \\
  \freenames{x?(y).P} & := & \{ x \} \cup (\freenames{P} \setminus \{ y \}) \\
  \freenames{x!\langle P \rangle} & := & \{ x \} \cup \{ P \} \\
  \freenames{P|Q} & := & \freenames{P} \cup \freenames{Q} \\
  \freenames{\dropn{x}} & := & \{ x \}
\end{eqnarray*}

The bound names of a process, $\boundnames{P}$, are those names occurring in $P$
that are not free. For example, in $x?(y).0$, the name $x$ is free, while $y$ is bound.

\begin{mathpar}
  \inferrule* [lab=monoidal-laws] {} { P|Q \equiv Q|P \and P|0 \equiv P \and P|(Q|R) \equiv (P|Q)|R }
\end{mathpar}

\begin{mathpar}
  \inferrule* [lab=alpha-equivalence] {} { (x)P \equiv (y)P\{y/x\} \and y \not\in \freenames{P} }
\end{mathpar}

\begin{definition}
Then two processes, $P,Q$, are alpha-equivalent if $P = Q\{\vec{y}/\vec{x}\}$ for
some $\vec{x} \in \boundnames{Q},\vec{y} \in \boundnames{P}$, where $Q\{\vec{y}/\vec{x}\}$
denotes the capture-avoiding substitution of $\vec{y}$ for $\vec{x}$ in $Q$.
\end{definition}

\begin{definition}
  The {\em structural congruence} \cite{SangiorgiWalker} , $\equiv$,
  between processes is the least congruence containing
  alpha-equivalence, satisfying the abelian monoid laws
  (associativity, commutativity and $\pzero$ as identity) for parallel
  composition $|$ and for summation $+$.
\end{definition}

\subsection{Name equivalence}

We take name equivalence, written $\nameeq$, to be the smallest
equivalence relation generated by the following rules.

\begin{mathpar}
\inferrule*[lab=Quote-drop]
{ }
{ \quotep{@{x}} \nameeq x }

\inferrule*[lab=Struct-equiv]
{ P \scong Q }
{ \quotep{P} \nameeq \quotep{Q} }
\end{mathpar}

The astute reader will have noticed that the mutual recursion of names
and processes imposes a mutual recursion on alpha-equivalence and
structural equivalence via name-equivalence. Fortunately, all of this
works out pleasantly and we may calculate in the natural way, free of
concern. The reader interested in the details is referred to the
appendix \ref{appendix:rho_details}.

\subsection{Substitution}

We use $\Proc$ for the set of processes, $\QProc$ for the set of
names, and $\id{\{}\vec{y} / \vec{x} \id{\}}$ to denote partial maps,
$s : \QProc \rightarrow \QProc$. A map, $s$ lifts, uniquely, to a map
on process terms, $\widehat{s} : \Proc \rightarrow \Proc$ by the
following equations.

\begin{mathpar}
  (0) \psubstp{Q}{P} := 0 \\
  (R \juxtap S) \psubstp{Q}{P}
  :=    
  (R)\psubstp{Q}{P} \juxtap (S) \psubstp{Q}{P} \\
  (x?(y).R) \psubstp{Q}{P}    
  :=    
  (x)\substp{Q}{P} (z)\concat( (R \psubstn{z}{y}) \psubstp{Q}{P} ) \\
  (\lift{x}{R}) \psubstp{Q}{P}  
  :=
  \lift{(x)\substp{Q}{P}}{ R \psubstp{Q}{P} } \\
%   (\dropn{x})  \psubstp{Q}{P}       
%   := 
%   \left\{ 
%     \begin{array}{ccc} 
%       \dropn{\quotep{Q}} & & x \nameeq \quotep{P} \\
%       \dropn{x} & & otherwise \\
%     \end{array}
%   \right. 
  (\dropn{x})  \psubstp{Q}{P}       
  := 
  \left\{ 
    \begin{array}{ccc} 
      Q & & x \nameeq \quotep{P} \\
      \dropn{x} & & otherwise \\
    \end{array}
  \right.
\end{mathpar}
 

where

\begin{eqnarray}
  (x)\id{\{} \lpquote Q \rpquote / \lpquote P \rpquote \id{\}}            = 
  \left\{ 
    \begin{array}{ccc}
      \lpquote Q \rpquote & & x \nameeq \lpquote P \rpquote \\
      x & & otherwise \\
    \end{array}
  \right. \nonumber
\end{eqnarray}

and $z$ is chosen distinct from $\quotep{P}$, $\quotep{Q}$, the free
names in $Q$, and all the names in $R$. Our $\alpha$-equivalence will
be built in the standard way from this substitution.

\begin{remark}\label{rem:no_self_referential_names}
  One consequence of these definitions is that $\forall P. \quotep{P}
  \not\in \freenames{P}$.
\end{remark}

\subsection{ Dynamic quote: an example }

Anticipating something of what's to come, consider applying the
substitution, $\widehat{\id{\{}u / z \id{\}}}$, to the following pair
of processes, $\lift{w}{y!(z)}$ and $w[ \lpquote y!(z) \rpquote ]$.

\begin{eqnarray}
	\lift{w}{y!(z)}\widehat{\id{\{}u / z \id{\}}}
		& = &
		\lift{w}{y!(u)} \nonumber\\
	w[ \lpquote y!(z) \rpquote ] \widehat{ \id{\{}u / z \id{\}} }
		& = &
		w[ \lpquote y!(z) \rpquote ] \nonumber
\end{eqnarray}

Because the body of the process between quotes is impervious to
substitution, we get radically different answers. In fact, by
examining the first process in an input context,
e.g. $x?(z).\lift{w}{y!(z)}$, we see that the process under the lift
operator may be shaped by prefixed inputs binding a name inside it. In
this sense, the lift operator will be seen as a way to dynamically
construct processes before reifying them as names.

Finally equipped with these standard features we can present the
dynamics of the calculus.

\subsubsection{Operational semantics} 

Finally, we introduce the computational dynamics. What marks these
algebras as distinct from other more traditionally studied algebraic
structures, e.g. vector spaces or polynomial rings, is the manner in
which dynamics is captured. In traditional structures, dynamics is typically
expressed through morphisms between such structures, as in linear maps
between vector spaces or morphisms between rings. In algebras
associated with the semantics of computation, the dynamics is
expressed as part of the algebraic structure itself, through a
reduction reduction relation typically denoted by $\red$. Below, we
give a recursive presentation of this relation for the calculus used
in the encoding.

$\red \subseteq \pi \times \pi$
$\red : \pi \to \mathcal{P}(\pi)$

\begin{mathpar}
  \inferrule* [lab=Comm] { \textsf{match}( x_{src}, x_{trgt} ) } { x_{trgt}?(y)P \; | \; x_{src}!\langle {Q} \rangle \red P\{\quotep{Q}/y}\} }
  \and \\
  \inferrule* [lab=Par] {{P} \red {P}'} {{{P} | {Q}} \red {{P}' | {Q}}}
  \and
  \inferrule* [lab=Equiv]{{{P} \scong {P}'} \andalso {{P}' \red {Q}'} \andalso {{Q}' \scong {Q}}}{{P} \red {Q}}
\end{mathpar}

\begin{eqnarray*}
  match_{\equiv} (\quotep{P},\quotep{Q}) & := & P \equiv Q \\
  match_{\dagger}(\quotep{P},\quotep{Q}) & := & \forall R. P|Q \red^{*} R => R \red^{*} 0 \\
  match_{K}(\quotep{P},\quotep{Q}) & := & K \mbox{ for some context } K
\end{eqnarray*}

$u?(x)P | u!\langle Q \rangle \red P\{\quotep{Q}/x\}$

%We write $\wred$ for $\red^*$, and $P\red$ if $\exists Q $ such that $ P \red Q$.
We write $P\red$ if $\exists Q $ such that $ P \red Q$ and $P\not\red$, otherwise.

\section{Replication}

As mentioned before, it is known that replication (and hence
recursion) can be implemented in a higher-order process algebra
\cite{SangiorgiWalker}. As our first example of calculation with the
machinery thus far presented we give the construction explicitly in
the {\rhoc}.

\begin{eqnarray}
	D_{x} & := & \prefix{x}{y}{(\binpar{\outputp{x}{y}}{@{y}})} \nonumber\\
	\bangp_{x}{P} & := & \binpar{{x}!\langle{\binpar{D_{x}}{P}}\rangle}{D_{x}} \nonumber
\end{eqnarray}

\begin{eqnarray}
	\bangp_{x}{P} & & \nonumber\\
	=
	& {x}!\langle{(\prefix{x}{y}{(\outputp{x}{y} | @{y})) | P}}\rangle 
	      | \prefix{x}{y}{(\outputp{x}{y} | @{y})} & \nonumber\\
	\red
	& (\outputp{x}{y} | @{y})\substn{\quotep{(\prefix{x}{y}{(@{y} | \outputp{x}{y})) | P}}}{y} & \nonumber\\
	=
	& \outputp{x}{\quotep{(\prefix{x}{y}{(\outputp{x}{y} | @{y})) | P}}}
	  | {(\prefix{x}{y}{(\outputp{x}{y} | @{y})) | P}} & \nonumber\\
	\red
	& \ldots & \nonumber\\
	\red^*
	& P | P | \ldots & \nonumber
\end{eqnarray}

Of course, this encoding, as an implementation, runs away, unfolding
$\bangp{P}$ eagerly. A lazier and more implementable replication
operator, restricted to input-guarded processes, may be obtained as follows.

\begin{eqnarray}
\bangp{\prefix{u}{v}{P}} 
	:= 
	\binpar{\lift{x}{\prefix{u}{v}{(\binpar{D(x)}{P})}}}{D(x)} \nonumber
\end{eqnarray}

\begin{remark}
  Note that the lazier definition still does not deal with summation
  or mixed summation (i.e. sums over input and output). The reader is
  invited to construct definitions of replication that deal with these
  features. 

  Further, the definitions are parameterized in a name, $x$. Can you,
  gentle reader, make a definition that eliminates this parameter and
  guarantees no accidental interaction between the replication
  machinery and the process being replicated -- i.e. no accidental
  sharing of names used by the process to get its work done and the
  name(s) used by the replication to effect copying. This latter
  revision of the definition of replication is crucial to obtaining
  the expected identity $!!P \sim !P$.
\end{remark}

\begin{remark}\label{rem:paradoxical_combinator}
  The reader familiar with the lambda calculus will have noticed the
  similarity between $D$ and the paradoxical combinator.

  [Ed. note: the existence of this seems to suggest we have to be more
  restrictive on the set of processes and names we admit if we are to
  support no-cloning.]
\end{remark}

\subsubsection{Bisimulation}

The computational dynamics gives rise to another kind of equivalence,
the equivalence of computational behavior. As previously mentioned
this is typically captured \emph{via} some form of bisimulation.

% The notion we use in this paper is weak barbed bisimulation
% \cite{milner91polyadicpi}.

The notion we use in this paper is derived from weak barbed
bisimulation \cite{milner91polyadicpi}. 

\begin{definition}
An \emph{observation relation}, $\downarrow_{\mathcal N}$, over a set
of names, $\mathcal N$, is the smallest relation satisfying the rules
below.

\infrule[Out-barb]{y \in {\mathcal N}, \; x \nameeq y}
		  {\outputp{x}{v} \downarrow_{\mathcal N} x}
\infrule[Par-barb]{\mbox{$P\downarrow_{\mathcal N} x$ or $Q\downarrow_{\mathcal N} x$}}
		  {\binpar{P}{Q} \downarrow_{\mathcal N} x}

We write $P \Downarrow_{\mathcal N} x$ if there is $Q$ such that 
$P \wred Q$ and $Q \downarrow_{\mathcal N} x$.
\end{definition}

\begin{definition}
%\label{def.bbisim}
An  ${\mathcal N}$-\emph{barbed bisimulation} over a set of names, ${\mathcal N}$, is a symmetric binary relation 
${\mathcal S}_{\mathcal N}$ between agents such that $P\rel{S}_{\mathcal N}Q$ implies:
\begin{enumerate}
\item If $P \red P'$ then $Q \wred Q'$ and $P'\rel{S}_{\mathcal N} Q'$.
\item If $P\downarrow_{\mathcal N} x$, then $Q\Downarrow_{\mathcal N} x$.
\end{enumerate}
$P$ is ${\mathcal N}$-barbed bisimilar to $Q$, written
$P \wbbisim_{\mathcal N} Q$, if $P \rel{S}_{\mathcal N} Q$ for some ${\mathcal N}$-barbed bisimulation ${\mathcal S}_{\mathcal N}$.
\end{definition}

$\mathcal{R} \subseteq \pi \times \pi$

$P \mathcal{R} Q => \forall P'. P \red P' \Rightarrow \exists Q'. Q \red Q', P' \mathcal{R} Q'$

$P \vdash x \Rightarrow Q \vdash x$

\begin{mathpar}
  \inferrule*[lab=Out-barb]{x \nameeq y}{{y}!\langle{Q}\rangle \vdash x}
  \and
  \inferrule*[lab=Par-barb]{\mbox{$P\vdash x$ or $Q\vdash x$}}{\binpar{P}{Q} \vdash x}
\end{mathpar}

\subsubsection{Contexts}

One of the principle advantages of computational calculi like the
$\pi$-calculus is a well-defined notion of context,
contextual-equivalence and a correlation between
contextual-equivalence and notions of bisimulation. The notion of
context allows the decomposition of a process into (sub-)process and
its syntactic environment, its context. Thus, a context may be
thought of as a process with a ``hole'' (written $\Box$) in it. The
application of a context $M$ to a process $P$, written $M[P]$, is
tantamount to filling the hole in $M$ with $P$. In this paper we do
not need the full weight of this theory, but do make use of the notion
of context in the proof the main theorem. 

\begin{mathpar}
  \inferrule* [lab=summation] {} {{M_{M},M_{N}} \bc \Box \;|\; x.M_{A} \;|\; M_{M}+M_{N}}
  \and
  \inferrule* [lab=agent] {} {{M_{A}} \bc (\vec{x})M_{P} \;| \; \clift{P_0,\ldots,M_{P},\ldots,P_N}}
  \and \\
  \inferrule* [lab=process] {} {{M_{P}} \bc M_{N} \;| \;P|M_{P} }
\end{mathpar} 

\begin{mathpar}
  \inferrule* [lab=sychronization] {} {M_{N} \bc \Box \;|\; x?M_{F} \;|\; x!M_{C}}
  \and
  \inferrule* [lab=abstraction] {} {{M_{F}} \bc (x)M_{P} }
  \and
  \inferrule* [lab=concretion] {} {{M_{C}} \bc \langle M_{P} \rangle }
  \and \\
  \inferrule* [lab=process] {} {{M_{P}} \bc M_{N} \;| \;P|M_{P} }
\end{mathpar}

\begin{definition}[contextual application] Given a context $M$, and
  process $P$, we define the \emph{contextual application}, $M[P] :=
  M\{P/\Box\}$. That is, the contextual application of M to P is the
  substitution of $P$ for $\Box$ in $M$.
\end{definition}

$\meaningof{-} : L \to \mathcal{P}(\pi)$

\begin{mathpar}
  \inferrule* [lab=collection] {} {\meaningof{true} = \pi, \and \meaningof{~E} = \pi \setminus \meaningof{E}, \and \meaningof{E_{1} \& E_{2}} = \meaningof{E_{1}} \cap \meaningof{E_{2}}}
\end{mathpar}

\begin{mathpar}
  \inferrule* [lab=structure] {} {\meaningof{0} = \{ P \in \pi | P \equiv 0 \}, \and \\ \meaningof{E_1 | E_2} = \{ P \in \pi | P \equiv P_{1} | P_{2}, P_{1} \in \meaningof{E_{1}}, P_{2} \in \meaningof{E_2}\} }
\end{mathpar}

\begin{mathpar}
 \inferrule* [lab=behavior] {} {\meaningof{\langle a?b \rangle E} = \{ P \in \pi | P \equiv Q | u?(y)P', \\ \and \\\\ \and \\ \;\;\; u \in \meaningof{a}, \forall z.P'\{z/y\} \in \meaningof{E\{z/b\}}\}, \and \\ \meaningof{a!E} = \{ P \in \pi | P \equiv Q | x!\langle P' \rangle, x \in \meaningof{a} P' \in \meaningof{E}\} }
\end{mathpar}

\begin{mathpar}
 \inferrule* [lab=nominal] {} {\meaningof{\quotep{E}} = \{ \quotep{P} \in \quotep{\pi} | P \in \meaningof{E} \}, \and \meaningof{\quotep{P}} = \{ \quotep{Q} \in \quotep{\pi} | P \equiv Q \} \and \\ \meaningof{@\quotep{E}} = \{ P \in \pi | P \equiv @x, x \in \meaningof{E} \}}
\end{mathpar}

\begin{eqnarray*}
  \\
  \meaningof{-} : TS \to ST
\end{eqnarray*}

\begin{eqnarray*}
  \\
  L : TS \to ST
\end{eqnarray*}

\begin{eqnarray*}
  \\
  P \models E \iff P \in \meaningof{E}
\end{eqnarray*}

\begin{eqnarray*}
  P \approx_{L} Q \iff \forall E \in L. P \models E \iff Q \models E
\end{eqnarray*}

\begin{eqnarray*}
  P \approx_{K} Q
\end{eqnarray*}

\begin{eqnarray*}
  P \approx Q
\end{eqnarray*}

$\approx_{K} = \approx = \approx_{L}$

\subsubsection{Contextual duality}

Note that contexts extend the quotation operation to a family of
operations from processes to names. Given a context, $M$, we can
define a \emph{nominal context}, $\quotep{M}$ by $\quotep{M}[P] :=
\quotep{M[P]}$. To foreshadow what is to come we observe that these
operations enjoy a duality with processes very much like the duality
between vectors and maps from vectors to scalars.

Further, because the calculus is essentially higher-order, we have a
correspondence between contexts and processes. More specifically,
given a name $x$ and a context $M$ we can construct $M^{*}_{x}$ such
that 

\begin{mathpar}
  M^{*}_{x} | \lift{x}{P} \red M[P]
\end{mathpar}

namely,

\begin{mathpar}
  M^{*}_{x} := x?(u).M[\dropn{u}]
\end{mathpar}

The dependence of $M^{*}_{x}$ on a name makes it an abstraction, 

\begin{mathpar}
  M^{*} := (x)x?(u).M[\dropn{u}]
\end{mathpar}

\subsection{Additional notation}

It will sometimes be convenient to denote the process a name
quotes. We already have the notation $x = \quotep{P}$, but it will be
convenient to introduce an alternate notation, $\procn{x}$, when we
want to emphasize the connection to the use of the name. Note that, by
virtue of name equivalence, $\quotep{\procn{x}} \nameeq x$; so, the
notation is consistent with previous definitions.

Further, because names have structure it is possible to effect
substitutions on the basis of that structure. This means we need to
upgrade our notation for substitutions, which we accomplish by
adapting comprehension notation. Thus,

\begin{mathpar}
  P\{ y / x : x \in S \}
\end{mathpar}

is interpreted to mean the process derived from P by replacing (in a
capture-avoiding manner) each occurrence of $x$ in $S$ by $y$. For example,

\begin{mathpar}
  P\{ \quotep{\procn{x}|\procn{x}} / x : x \in \freenames{P} \}
\end{mathpar}

will replace each (occurrence) of a free name $x$ in $P$ by
$\quotep{\procn{x}|\procn{x}}$.

Also, we will avail ourselves of the notation $x^{L}$ and $x^{R}$ to
denote injections of a name into disjoint copies of the name
space. There are numerous ways to accomplish this. One example can be
found in \cite{MeredithR05}. This notation overloads to vectors of
names: $\vec{x}^{\pi} := (x_{i}^{\pi} \; : \; 0 \leq i < |\vec{x}| )$ where $\pi \in \{L,R\}$.

We also use $P^{\Box} := P|\Box$.

In \cite{MeredithR05} an interpretation of the new operator is
given. It turns out that there are several possible interpretations
all enjoying the requisite algebraic properties of the operator (see
\cite{milner91polyadicpi}). We will therefore make liberal use of
$(\nu\; \vec{x})P$.

% subsection the_syntax_and_semantics_of_the_notation_system (end)   

\input{qm2pi.qmops} 

\input{qm2pi.sterngerlach} 

\input{qm2pi.metric} 

% section concurrent_process_calculi (end)

%\input{qm2pi.proofsketch}

% section proof sketch (end)

%\input{qm2pi.slviaknots} 

% section spatial logic via knots (end)

\input{qm2pi.conclusion}

% section conclusion (end)

%\input{qm2pi.dtcodes} 

% section wiring algorithm (end)

\input{qm2pi.ack} 

% section acknowledgments (end)

\newpage


\bibliographystyle{plain}   
\bibliography{../../biblios/main.bib}

\input{qm2pi.rhodetails}

\end{document}

 

%\documentclass[12pt]{llncs}
%\documentclass{jktr}

\usepackage[pdftex]{hyperref}                   
\usepackage {listings}
\usepackage {mathpartir}
\usepackage{bcprules}
%\usepackage{listings}
                       
\usepackage{graphicx} 
%\usepackage[margins=2.5cm,nohead,nofoot]{geometry}
%\usepackage{geometry}
\usepackage{amsfonts}
\usepackage{amstext}
\usepackage{latexsym}
\usepackage{amssymb}
\usepackage{color}


%\include{myPreamble}
\include{qm2pi.local} 

%\ifpdf
%\usepackage[pdftex]{graphicx}
%\else
%\usepackage{graphicx}
%\fi

 % \ifpdf
%  \usepackage{pdfsync}
%  \if


%\title{Brief Article}
%\author{David F. Snyder}
%\author{L.G. Meredith}

%\address{Dept. of Math., Texas State University--San Marcos, San Marcos, TX 78666}
       
\pagestyle{empty}


\begin{document}

\lstset{language=[Objective]Caml,frame=shadowbox}

\input{qm2pi.front}

% section front matter (end)

\input{qm2pi.intro} 
 
% section introduction (end)

% \input{qm2pi.knotations} 

% section notation (end)

\input{qm2pi.process.calculi} 

% section concurrent_process_calculi_and_spatial_logics_ (end)
    
%\input{qm2pi.knots2pi} 

%\input{qm2pi.trefoil} 

%\input{qm2pi.mainthm} 

% subsection basic_interpretation (end)

%\input{qm2pi.rho.presentation} 
\subsection{The syntax and semantics of the notation system}\label{sub:the_syntax_and_semantics_of_the_notation_system} % (fold)

We now summarize a technical presentation of the calculus that
embodies our theory of dynamics. The typical presentation of such a
calculus follows the style of giving generators and relations on
them. The grammar, below, describing term constructors, freely
generates the set of processes, $\Proc$. This set is then quotiented
by a relation known as structural congruence and it is over this set
that the notion of dynamics is expressed. This presentation is
essentially that of \cite{MeredithR05} with the addition of
polyadicity and summation. For readability we have relegated some of
the technical subtleties to an appendix.

\subsubsection{Process grammar}\label{subsub:process_grammar}

\begin{mathpar}
  \inferrule* [lab=synchronization] {} {{M} \bc \pzero \;|\; x?F \;|\; x!C }
  \and
  \inferrule* [lab=abstraction] {} {{F} \bc (x)P}
  \and
  \inferrule* [lab=concretion] {} {{C} \bc \langle Q \rangle}
  \and
  \inferrule* [lab=process] {} {{P,Q} \bc M \;| \;P|Q \;|\; @{x}}
  \and
  \inferrule* [lab=name] {} {{x} \bc \quotep{P}}
\end{mathpar} 

Note that $\vec{x}$ (resp. $\vec{P}$) denotes a vector of names
(resp. processes) of length $|\vec{x}|$ (resp. $|\vec{P}|$). We adopt
the following useful abbreviations.

\begin{mathpar}
   x?(\vec{y}).P := x.(\vec{y})P \and  x\clift{\vec{P}} := x.\clift{\vec{P}}
   \and x!(y) := \lift{x}{\dropn{y}}
   \and \Pi_{i=0}^{n-1}P_i := P_0 | \ldots | P_{n-1}
\end{mathpar}

\subsubsection{Structural congruence}

\paragraph{Free and bound names and alpha-equivalence.} At the
core of structural equivalence is alpha-equivalence which identifies
process that are the same up to a change of variable. Formally, we
recognize the distinction between free and bound names. The free names
of a process, $\freenames{P}$, may be calculated recursively as
follows:

\begin{mathpar}
\freenames{\pzero} := \emptyset
  \and \\
  \freenames{x?(y).P} := \{ x \} \cup (\freenames{P} \setminus \{ y \})
  \and 
  \freenames{x!\langle P \rangle} := \{ x \} \cup \{ P \} 
  \and \\
  \freenames{P|Q} := \freenames{P} \cup \freenames{Q}
  \and \\
  \freenames{@{x}} := \{ x \}
\end{mathpar}

$\pi$
$\quotep{\pi}$

$\freenames{-} : \pi \to \mathcal{P}(\quotep{\pi})$

\begin{eqnarray*}
  \freenames{\pzero} & := & \emptyset \\
  \freenames{x?(y).P} & := & \{ x \} \cup (\freenames{P} \setminus \{ y \}) \\
  \freenames{x!\langle P \rangle} & := & \{ x \} \cup \{ P \} \\
  \freenames{P|Q} & := & \freenames{P} \cup \freenames{Q} \\
  \freenames{\dropn{x}} & := & \{ x \}
\end{eqnarray*}

The bound names of a process, $\boundnames{P}$, are those names occurring in $P$
that are not free. For example, in $x?(y).0$, the name $x$ is free, while $y$ is bound.

\begin{mathpar}
  \inferrule* [lab=monoidal-laws] {} { P|Q \equiv Q|P \and P|0 \equiv P \and P|(Q|R) \equiv (P|Q)|R }
\end{mathpar}

\begin{mathpar}
  \inferrule* [lab=alpha-equivalence] {} { (x)P \equiv (y)P\{y/x\} \and y \not\in \freenames{P} }
\end{mathpar}

\begin{definition}
Then two processes, $P,Q$, are alpha-equivalent if $P = Q\{\vec{y}/\vec{x}\}$ for
some $\vec{x} \in \boundnames{Q},\vec{y} \in \boundnames{P}$, where $Q\{\vec{y}/\vec{x}\}$
denotes the capture-avoiding substitution of $\vec{y}$ for $\vec{x}$ in $Q$.
\end{definition}

\begin{definition}
  The {\em structural congruence} \cite{SangiorgiWalker} , $\equiv$,
  between processes is the least congruence containing
  alpha-equivalence, satisfying the abelian monoid laws
  (associativity, commutativity and $\pzero$ as identity) for parallel
  composition $|$ and for summation $+$.
\end{definition}

\subsection{Name equivalence}

We take name equivalence, written $\nameeq$, to be the smallest
equivalence relation generated by the following rules.

\begin{mathpar}
\inferrule*[lab=Quote-drop]
{ }
{ \quotep{@{x}} \nameeq x }

\inferrule*[lab=Struct-equiv]
{ P \scong Q }
{ \quotep{P} \nameeq \quotep{Q} }
\end{mathpar}

The astute reader will have noticed that the mutual recursion of names
and processes imposes a mutual recursion on alpha-equivalence and
structural equivalence via name-equivalence. Fortunately, all of this
works out pleasantly and we may calculate in the natural way, free of
concern. The reader interested in the details is referred to the
appendix \ref{appendix:rho_details}.

\subsection{Substitution}

We use $\Proc$ for the set of processes, $\QProc$ for the set of
names, and $\id{\{}\vec{y} / \vec{x} \id{\}}$ to denote partial maps,
$s : \QProc \rightarrow \QProc$. A map, $s$ lifts, uniquely, to a map
on process terms, $\widehat{s} : \Proc \rightarrow \Proc$ by the
following equations.

\begin{mathpar}
  (0) \psubstp{Q}{P} := 0 \\
  (R \juxtap S) \psubstp{Q}{P}
  :=    
  (R)\psubstp{Q}{P} \juxtap (S) \psubstp{Q}{P} \\
  (x?(y).R) \psubstp{Q}{P}    
  :=    
  (x)\substp{Q}{P} (z)\concat( (R \psubstn{z}{y}) \psubstp{Q}{P} ) \\
  (\lift{x}{R}) \psubstp{Q}{P}  
  :=
  \lift{(x)\substp{Q}{P}}{ R \psubstp{Q}{P} } \\
%   (\dropn{x})  \psubstp{Q}{P}       
%   := 
%   \left\{ 
%     \begin{array}{ccc} 
%       \dropn{\quotep{Q}} & & x \nameeq \quotep{P} \\
%       \dropn{x} & & otherwise \\
%     \end{array}
%   \right. 
  (\dropn{x})  \psubstp{Q}{P}       
  := 
  \left\{ 
    \begin{array}{ccc} 
      Q & & x \nameeq \quotep{P} \\
      \dropn{x} & & otherwise \\
    \end{array}
  \right.
\end{mathpar}
 

where

\begin{eqnarray}
  (x)\id{\{} \lpquote Q \rpquote / \lpquote P \rpquote \id{\}}            = 
  \left\{ 
    \begin{array}{ccc}
      \lpquote Q \rpquote & & x \nameeq \lpquote P \rpquote \\
      x & & otherwise \\
    \end{array}
  \right. \nonumber
\end{eqnarray}

and $z$ is chosen distinct from $\quotep{P}$, $\quotep{Q}$, the free
names in $Q$, and all the names in $R$. Our $\alpha$-equivalence will
be built in the standard way from this substitution.

\begin{remark}\label{rem:no_self_referential_names}
  One consequence of these definitions is that $\forall P. \quotep{P}
  \not\in \freenames{P}$.
\end{remark}

\subsection{ Dynamic quote: an example }

Anticipating something of what's to come, consider applying the
substitution, $\widehat{\id{\{}u / z \id{\}}}$, to the following pair
of processes, $\lift{w}{y!(z)}$ and $w[ \lpquote y!(z) \rpquote ]$.

\begin{eqnarray}
	\lift{w}{y!(z)}\widehat{\id{\{}u / z \id{\}}}
		& = &
		\lift{w}{y!(u)} \nonumber\\
	w[ \lpquote y!(z) \rpquote ] \widehat{ \id{\{}u / z \id{\}} }
		& = &
		w[ \lpquote y!(z) \rpquote ] \nonumber
\end{eqnarray}

Because the body of the process between quotes is impervious to
substitution, we get radically different answers. In fact, by
examining the first process in an input context,
e.g. $x?(z).\lift{w}{y!(z)}$, we see that the process under the lift
operator may be shaped by prefixed inputs binding a name inside it. In
this sense, the lift operator will be seen as a way to dynamically
construct processes before reifying them as names.

Finally equipped with these standard features we can present the
dynamics of the calculus.

\subsubsection{Operational semantics} 

Finally, we introduce the computational dynamics. What marks these
algebras as distinct from other more traditionally studied algebraic
structures, e.g. vector spaces or polynomial rings, is the manner in
which dynamics is captured. In traditional structures, dynamics is typically
expressed through morphisms between such structures, as in linear maps
between vector spaces or morphisms between rings. In algebras
associated with the semantics of computation, the dynamics is
expressed as part of the algebraic structure itself, through a
reduction reduction relation typically denoted by $\red$. Below, we
give a recursive presentation of this relation for the calculus used
in the encoding.

$\red \subseteq \pi \times \pi$
$\red : \pi \to \mathcal{P}(\pi)$

\begin{mathpar}
  \inferrule* [lab=Comm] { \textsf{match}( x_{src}, x_{trgt} ) } { x_{trgt}?(y)P \; | \; x_{src}!\langle {Q} \rangle \red P\{\quotep{Q}/y}\} }
  \and \\
  \inferrule* [lab=Par] {{P} \red {P}'} {{{P} | {Q}} \red {{P}' | {Q}}}
  \and
  \inferrule* [lab=Equiv]{{{P} \scong {P}'} \andalso {{P}' \red {Q}'} \andalso {{Q}' \scong {Q}}}{{P} \red {Q}}
\end{mathpar}

\begin{eqnarray*}
  match_{\equiv} (\quotep{P},\quotep{Q}) & := & P \equiv Q \\
  match_{\dagger}(\quotep{P},\quotep{Q}) & := & \forall R. P|Q \red^{*} R => R \red^{*} 0 \\
  match_{K}(\quotep{P},\quotep{Q}) & := & K \mbox{ for some context } K
\end{eqnarray*}

$u?(x)P | u!\langle Q \rangle \red P\{\quotep{Q}/x\}$

%We write $\wred$ for $\red^*$, and $P\red$ if $\exists Q $ such that $ P \red Q$.
We write $P\red$ if $\exists Q $ such that $ P \red Q$ and $P\not\red$, otherwise.

\section{Replication}

As mentioned before, it is known that replication (and hence
recursion) can be implemented in a higher-order process algebra
\cite{SangiorgiWalker}. As our first example of calculation with the
machinery thus far presented we give the construction explicitly in
the {\rhoc}.

\begin{eqnarray}
	D_{x} & := & \prefix{x}{y}{(\binpar{\outputp{x}{y}}{@{y}})} \nonumber\\
	\bangp_{x}{P} & := & \binpar{{x}!\langle{\binpar{D_{x}}{P}}\rangle}{D_{x}} \nonumber
\end{eqnarray}

\begin{eqnarray}
	\bangp_{x}{P} & & \nonumber\\
	=
	& {x}!\langle{(\prefix{x}{y}{(\outputp{x}{y} | @{y})) | P}}\rangle 
	      | \prefix{x}{y}{(\outputp{x}{y} | @{y})} & \nonumber\\
	\red
	& (\outputp{x}{y} | @{y})\substn{\quotep{(\prefix{x}{y}{(@{y} | \outputp{x}{y})) | P}}}{y} & \nonumber\\
	=
	& \outputp{x}{\quotep{(\prefix{x}{y}{(\outputp{x}{y} | @{y})) | P}}}
	  | {(\prefix{x}{y}{(\outputp{x}{y} | @{y})) | P}} & \nonumber\\
	\red
	& \ldots & \nonumber\\
	\red^*
	& P | P | \ldots & \nonumber
\end{eqnarray}

Of course, this encoding, as an implementation, runs away, unfolding
$\bangp{P}$ eagerly. A lazier and more implementable replication
operator, restricted to input-guarded processes, may be obtained as follows.

\begin{eqnarray}
\bangp{\prefix{u}{v}{P}} 
	:= 
	\binpar{\lift{x}{\prefix{u}{v}{(\binpar{D(x)}{P})}}}{D(x)} \nonumber
\end{eqnarray}

\begin{remark}
  Note that the lazier definition still does not deal with summation
  or mixed summation (i.e. sums over input and output). The reader is
  invited to construct definitions of replication that deal with these
  features. 

  Further, the definitions are parameterized in a name, $x$. Can you,
  gentle reader, make a definition that eliminates this parameter and
  guarantees no accidental interaction between the replication
  machinery and the process being replicated -- i.e. no accidental
  sharing of names used by the process to get its work done and the
  name(s) used by the replication to effect copying. This latter
  revision of the definition of replication is crucial to obtaining
  the expected identity $!!P \sim !P$.
\end{remark}

\begin{remark}\label{rem:paradoxical_combinator}
  The reader familiar with the lambda calculus will have noticed the
  similarity between $D$ and the paradoxical combinator.

  [Ed. note: the existence of this seems to suggest we have to be more
  restrictive on the set of processes and names we admit if we are to
  support no-cloning.]
\end{remark}

\subsubsection{Bisimulation}

The computational dynamics gives rise to another kind of equivalence,
the equivalence of computational behavior. As previously mentioned
this is typically captured \emph{via} some form of bisimulation.

% The notion we use in this paper is weak barbed bisimulation
% \cite{milner91polyadicpi}.

The notion we use in this paper is derived from weak barbed
bisimulation \cite{milner91polyadicpi}. 

\begin{definition}
An \emph{observation relation}, $\downarrow_{\mathcal N}$, over a set
of names, $\mathcal N$, is the smallest relation satisfying the rules
below.

\infrule[Out-barb]{y \in {\mathcal N}, \; x \nameeq y}
		  {\outputp{x}{v} \downarrow_{\mathcal N} x}
\infrule[Par-barb]{\mbox{$P\downarrow_{\mathcal N} x$ or $Q\downarrow_{\mathcal N} x$}}
		  {\binpar{P}{Q} \downarrow_{\mathcal N} x}

We write $P \Downarrow_{\mathcal N} x$ if there is $Q$ such that 
$P \wred Q$ and $Q \downarrow_{\mathcal N} x$.
\end{definition}

\begin{definition}
%\label{def.bbisim}
An  ${\mathcal N}$-\emph{barbed bisimulation} over a set of names, ${\mathcal N}$, is a symmetric binary relation 
${\mathcal S}_{\mathcal N}$ between agents such that $P\rel{S}_{\mathcal N}Q$ implies:
\begin{enumerate}
\item If $P \red P'$ then $Q \wred Q'$ and $P'\rel{S}_{\mathcal N} Q'$.
\item If $P\downarrow_{\mathcal N} x$, then $Q\Downarrow_{\mathcal N} x$.
\end{enumerate}
$P$ is ${\mathcal N}$-barbed bisimilar to $Q$, written
$P \wbbisim_{\mathcal N} Q$, if $P \rel{S}_{\mathcal N} Q$ for some ${\mathcal N}$-barbed bisimulation ${\mathcal S}_{\mathcal N}$.
\end{definition}

$\mathcal{R} \subseteq \pi \times \pi$

$P \mathcal{R} Q => \forall P'. P \red P' \Rightarrow \exists Q'. Q \red Q', P' \mathcal{R} Q'$

$P \vdash x \Rightarrow Q \vdash x$

\begin{mathpar}
  \inferrule*[lab=Out-barb]{x \nameeq y}{{y}!\langle{Q}\rangle \vdash x}
  \and
  \inferrule*[lab=Par-barb]{\mbox{$P\vdash x$ or $Q\vdash x$}}{\binpar{P}{Q} \vdash x}
\end{mathpar}

\subsubsection{Contexts}

One of the principle advantages of computational calculi like the
$\pi$-calculus is a well-defined notion of context,
contextual-equivalence and a correlation between
contextual-equivalence and notions of bisimulation. The notion of
context allows the decomposition of a process into (sub-)process and
its syntactic environment, its context. Thus, a context may be
thought of as a process with a ``hole'' (written $\Box$) in it. The
application of a context $M$ to a process $P$, written $M[P]$, is
tantamount to filling the hole in $M$ with $P$. In this paper we do
not need the full weight of this theory, but do make use of the notion
of context in the proof the main theorem. 

\begin{mathpar}
  \inferrule* [lab=summation] {} {{M_{M},M_{N}} \bc \Box \;|\; x.M_{A} \;|\; M_{M}+M_{N}}
  \and
  \inferrule* [lab=agent] {} {{M_{A}} \bc (\vec{x})M_{P} \;| \; \clift{P_0,\ldots,M_{P},\ldots,P_N}}
  \and \\
  \inferrule* [lab=process] {} {{M_{P}} \bc M_{N} \;| \;P|M_{P} }
\end{mathpar} 

\begin{mathpar}
  \inferrule* [lab=sychronization] {} {M_{N} \bc \Box \;|\; x?M_{F} \;|\; x!M_{C}}
  \and
  \inferrule* [lab=abstraction] {} {{M_{F}} \bc (x)M_{P} }
  \and
  \inferrule* [lab=concretion] {} {{M_{C}} \bc \langle M_{P} \rangle }
  \and \\
  \inferrule* [lab=process] {} {{M_{P}} \bc M_{N} \;| \;P|M_{P} }
\end{mathpar}

\begin{definition}[contextual application] Given a context $M$, and
  process $P$, we define the \emph{contextual application}, $M[P] :=
  M\{P/\Box\}$. That is, the contextual application of M to P is the
  substitution of $P$ for $\Box$ in $M$.
\end{definition}

$\meaningof{-} : L \to \mathcal{P}(\pi)$

\begin{mathpar}
  \inferrule* [lab=collection] {} {\meaningof{true} = \pi, \and \meaningof{~E} = \pi \setminus \meaningof{E}, \and \meaningof{E_{1} \& E_{2}} = \meaningof{E_{1}} \cap \meaningof{E_{2}}}
\end{mathpar}

\begin{mathpar}
  \inferrule* [lab=structure] {} {\meaningof{0} = \{ P \in \pi | P \equiv 0 \}, \and \\ \meaningof{E_1 | E_2} = \{ P \in \pi | P \equiv P_{1} | P_{2}, P_{1} \in \meaningof{E_{1}}, P_{2} \in \meaningof{E_2}\} }
\end{mathpar}

\begin{mathpar}
 \inferrule* [lab=behavior] {} {\meaningof{\langle a?b \rangle E} = \{ P \in \pi | P \equiv Q | u?(y)P', \\ \and \\\\ \and \\ \;\;\; u \in \meaningof{a}, \forall z.P'\{z/y\} \in \meaningof{E\{z/b\}}\}, \and \\ \meaningof{a!E} = \{ P \in \pi | P \equiv Q | x!\langle P' \rangle, x \in \meaningof{a} P' \in \meaningof{E}\} }
\end{mathpar}

\begin{mathpar}
 \inferrule* [lab=nominal] {} {\meaningof{\quotep{E}} = \{ \quotep{P} \in \quotep{\pi} | P \in \meaningof{E} \}, \and \meaningof{\quotep{P}} = \{ \quotep{Q} \in \quotep{\pi} | P \equiv Q \} \and \\ \meaningof{@\quotep{E}} = \{ P \in \pi | P \equiv @x, x \in \meaningof{E} \}}
\end{mathpar}

\begin{eqnarray*}
  \\
  \meaningof{-} : TS \to ST
\end{eqnarray*}

\begin{eqnarray*}
  \\
  L : TS \to ST
\end{eqnarray*}

\begin{eqnarray*}
  \\
  P \models E \iff P \in \meaningof{E}
\end{eqnarray*}

\begin{eqnarray*}
  P \approx_{L} Q \iff \forall E \in L. P \models E \iff Q \models E
\end{eqnarray*}

\begin{eqnarray*}
  P \approx_{K} Q
\end{eqnarray*}

\begin{eqnarray*}
  P \approx Q
\end{eqnarray*}

$\approx_{K} = \approx = \approx_{L}$

\subsubsection{Contextual duality}

Note that contexts extend the quotation operation to a family of
operations from processes to names. Given a context, $M$, we can
define a \emph{nominal context}, $\quotep{M}$ by $\quotep{M}[P] :=
\quotep{M[P]}$. To foreshadow what is to come we observe that these
operations enjoy a duality with processes very much like the duality
between vectors and maps from vectors to scalars.

Further, because the calculus is essentially higher-order, we have a
correspondence between contexts and processes. More specifically,
given a name $x$ and a context $M$ we can construct $M^{*}_{x}$ such
that 

\begin{mathpar}
  M^{*}_{x} | \lift{x}{P} \red M[P]
\end{mathpar}

namely,

\begin{mathpar}
  M^{*}_{x} := x?(u).M[\dropn{u}]
\end{mathpar}

The dependence of $M^{*}_{x}$ on a name makes it an abstraction, 

\begin{mathpar}
  M^{*} := (x)x?(u).M[\dropn{u}]
\end{mathpar}

\subsection{Additional notation}

It will sometimes be convenient to denote the process a name
quotes. We already have the notation $x = \quotep{P}$, but it will be
convenient to introduce an alternate notation, $\procn{x}$, when we
want to emphasize the connection to the use of the name. Note that, by
virtue of name equivalence, $\quotep{\procn{x}} \nameeq x$; so, the
notation is consistent with previous definitions.

Further, because names have structure it is possible to effect
substitutions on the basis of that structure. This means we need to
upgrade our notation for substitutions, which we accomplish by
adapting comprehension notation. Thus,

\begin{mathpar}
  P\{ y / x : x \in S \}
\end{mathpar}

is interpreted to mean the process derived from P by replacing (in a
capture-avoiding manner) each occurrence of $x$ in $S$ by $y$. For example,

\begin{mathpar}
  P\{ \quotep{\procn{x}|\procn{x}} / x : x \in \freenames{P} \}
\end{mathpar}

will replace each (occurrence) of a free name $x$ in $P$ by
$\quotep{\procn{x}|\procn{x}}$.

Also, we will avail ourselves of the notation $x^{L}$ and $x^{R}$ to
denote injections of a name into disjoint copies of the name
space. There are numerous ways to accomplish this. One example can be
found in \cite{MeredithR05}. This notation overloads to vectors of
names: $\vec{x}^{\pi} := (x_{i}^{\pi} \; : \; 0 \leq i < |\vec{x}| )$ where $\pi \in \{L,R\}$.

We also use $P^{\Box} := P|\Box$.

In \cite{MeredithR05} an interpretation of the new operator is
given. It turns out that there are several possible interpretations
all enjoying the requisite algebraic properties of the operator (see
\cite{milner91polyadicpi}). We will therefore make liberal use of
$(\nu\; \vec{x})P$.

% subsection the_syntax_and_semantics_of_the_notation_system (end)   

\input{qm2pi.qmops} 

\input{qm2pi.sterngerlach} 

\input{qm2pi.metric} 

% section concurrent_process_calculi (end)

%\input{qm2pi.proofsketch}

% section proof sketch (end)

%\input{qm2pi.slviaknots} 

% section spatial logic via knots (end)

\input{qm2pi.conclusion}

% section conclusion (end)

%\input{qm2pi.dtcodes} 

% section wiring algorithm (end)

\input{qm2pi.ack} 

% section acknowledgments (end)

\newpage


\bibliographystyle{plain}   
\bibliography{../../biblios/main.bib}

\input{qm2pi.rhodetails}

\end{document}

 

% subsection basic_interpretation (end)

%\input{qm2pi.rho.presentation} 
\subsection{The syntax and semantics of the notation system}\label{sub:the_syntax_and_semantics_of_the_notation_system} % (fold)

We now summarize a technical presentation of the calculus that
embodies our theory of dynamics. The typical presentation of such a
calculus follows the style of giving generators and relations on
them. The grammar, below, describing term constructors, freely
generates the set of processes, $\Proc$. This set is then quotiented
by a relation known as structural congruence and it is over this set
that the notion of dynamics is expressed. This presentation is
essentially that of \cite{MeredithR05} with the addition of
polyadicity and summation. For readability we have relegated some of
the technical subtleties to an appendix.

\subsubsection{Process grammar}\label{subsub:process_grammar}

\begin{mathpar}
  \inferrule* [lab=synchronization] {} {{M} \bc \pzero \;|\; x?F \;|\; x!C }
  \and
  \inferrule* [lab=abstraction] {} {{F} \bc (x)P}
  \and
  \inferrule* [lab=concretion] {} {{C} \bc \langle Q \rangle}
  \and
  \inferrule* [lab=process] {} {{P,Q} \bc M \;| \;P|Q \;|\; @{x}}
  \and
  \inferrule* [lab=name] {} {{x} \bc \quotep{P}}
\end{mathpar} 

Note that $\vec{x}$ (resp. $\vec{P}$) denotes a vector of names
(resp. processes) of length $|\vec{x}|$ (resp. $|\vec{P}|$). We adopt
the following useful abbreviations.

\begin{mathpar}
   x?(\vec{y}).P := x.(\vec{y})P \and  x\clift{\vec{P}} := x.\clift{\vec{P}}
   \and x!(y) := \lift{x}{\dropn{y}}
   \and \Pi_{i=0}^{n-1}P_i := P_0 | \ldots | P_{n-1}
\end{mathpar}

\subsubsection{Structural congruence}

\paragraph{Free and bound names and alpha-equivalence.} At the
core of structural equivalence is alpha-equivalence which identifies
process that are the same up to a change of variable. Formally, we
recognize the distinction between free and bound names. The free names
of a process, $\freenames{P}$, may be calculated recursively as
follows:

\begin{mathpar}
\freenames{\pzero} := \emptyset
  \and \\
  \freenames{x?(y).P} := \{ x \} \cup (\freenames{P} \setminus \{ y \})
  \and 
  \freenames{x!\langle P \rangle} := \{ x \} \cup \{ P \} 
  \and \\
  \freenames{P|Q} := \freenames{P} \cup \freenames{Q}
  \and \\
  \freenames{@{x}} := \{ x \}
\end{mathpar}

$\pi$
$\quotep{\pi}$

$\freenames{-} : \pi \to \mathcal{P}(\quotep{\pi})$

\begin{eqnarray*}
  \freenames{\pzero} & := & \emptyset \\
  \freenames{x?(y).P} & := & \{ x \} \cup (\freenames{P} \setminus \{ y \}) \\
  \freenames{x!\langle P \rangle} & := & \{ x \} \cup \{ P \} \\
  \freenames{P|Q} & := & \freenames{P} \cup \freenames{Q} \\
  \freenames{\dropn{x}} & := & \{ x \}
\end{eqnarray*}

The bound names of a process, $\boundnames{P}$, are those names occurring in $P$
that are not free. For example, in $x?(y).0$, the name $x$ is free, while $y$ is bound.

\begin{mathpar}
  \inferrule* [lab=monoidal-laws] {} { P|Q \equiv Q|P \and P|0 \equiv P \and P|(Q|R) \equiv (P|Q)|R }
\end{mathpar}

\begin{mathpar}
  \inferrule* [lab=alpha-equivalence] {} { (x)P \equiv (y)P\{y/x\} \and y \not\in \freenames{P} }
\end{mathpar}

\begin{definition}
Then two processes, $P,Q$, are alpha-equivalent if $P = Q\{\vec{y}/\vec{x}\}$ for
some $\vec{x} \in \boundnames{Q},\vec{y} \in \boundnames{P}$, where $Q\{\vec{y}/\vec{x}\}$
denotes the capture-avoiding substitution of $\vec{y}$ for $\vec{x}$ in $Q$.
\end{definition}

\begin{definition}
  The {\em structural congruence} \cite{SangiorgiWalker} , $\equiv$,
  between processes is the least congruence containing
  alpha-equivalence, satisfying the abelian monoid laws
  (associativity, commutativity and $\pzero$ as identity) for parallel
  composition $|$ and for summation $+$.
\end{definition}

\subsection{Name equivalence}

We take name equivalence, written $\nameeq$, to be the smallest
equivalence relation generated by the following rules.

\begin{mathpar}
\inferrule*[lab=Quote-drop]
{ }
{ \quotep{@{x}} \nameeq x }

\inferrule*[lab=Struct-equiv]
{ P \scong Q }
{ \quotep{P} \nameeq \quotep{Q} }
\end{mathpar}

The astute reader will have noticed that the mutual recursion of names
and processes imposes a mutual recursion on alpha-equivalence and
structural equivalence via name-equivalence. Fortunately, all of this
works out pleasantly and we may calculate in the natural way, free of
concern. The reader interested in the details is referred to the
appendix \ref{appendix:rho_details}.

\subsection{Substitution}

We use $\Proc$ for the set of processes, $\QProc$ for the set of
names, and $\id{\{}\vec{y} / \vec{x} \id{\}}$ to denote partial maps,
$s : \QProc \rightarrow \QProc$. A map, $s$ lifts, uniquely, to a map
on process terms, $\widehat{s} : \Proc \rightarrow \Proc$ by the
following equations.

\begin{mathpar}
  (0) \psubstp{Q}{P} := 0 \\
  (R \juxtap S) \psubstp{Q}{P}
  :=    
  (R)\psubstp{Q}{P} \juxtap (S) \psubstp{Q}{P} \\
  (x?(y).R) \psubstp{Q}{P}    
  :=    
  (x)\substp{Q}{P} (z)\concat( (R \psubstn{z}{y}) \psubstp{Q}{P} ) \\
  (\lift{x}{R}) \psubstp{Q}{P}  
  :=
  \lift{(x)\substp{Q}{P}}{ R \psubstp{Q}{P} } \\
%   (\dropn{x})  \psubstp{Q}{P}       
%   := 
%   \left\{ 
%     \begin{array}{ccc} 
%       \dropn{\quotep{Q}} & & x \nameeq \quotep{P} \\
%       \dropn{x} & & otherwise \\
%     \end{array}
%   \right. 
  (\dropn{x})  \psubstp{Q}{P}       
  := 
  \left\{ 
    \begin{array}{ccc} 
      Q & & x \nameeq \quotep{P} \\
      \dropn{x} & & otherwise \\
    \end{array}
  \right.
\end{mathpar}
 

where

\begin{eqnarray}
  (x)\id{\{} \lpquote Q \rpquote / \lpquote P \rpquote \id{\}}            = 
  \left\{ 
    \begin{array}{ccc}
      \lpquote Q \rpquote & & x \nameeq \lpquote P \rpquote \\
      x & & otherwise \\
    \end{array}
  \right. \nonumber
\end{eqnarray}

and $z$ is chosen distinct from $\quotep{P}$, $\quotep{Q}$, the free
names in $Q$, and all the names in $R$. Our $\alpha$-equivalence will
be built in the standard way from this substitution.

\begin{remark}\label{rem:no_self_referential_names}
  One consequence of these definitions is that $\forall P. \quotep{P}
  \not\in \freenames{P}$.
\end{remark}

\subsection{ Dynamic quote: an example }

Anticipating something of what's to come, consider applying the
substitution, $\widehat{\id{\{}u / z \id{\}}}$, to the following pair
of processes, $\lift{w}{y!(z)}$ and $w[ \lpquote y!(z) \rpquote ]$.

\begin{eqnarray}
	\lift{w}{y!(z)}\widehat{\id{\{}u / z \id{\}}}
		& = &
		\lift{w}{y!(u)} \nonumber\\
	w[ \lpquote y!(z) \rpquote ] \widehat{ \id{\{}u / z \id{\}} }
		& = &
		w[ \lpquote y!(z) \rpquote ] \nonumber
\end{eqnarray}

Because the body of the process between quotes is impervious to
substitution, we get radically different answers. In fact, by
examining the first process in an input context,
e.g. $x?(z).\lift{w}{y!(z)}$, we see that the process under the lift
operator may be shaped by prefixed inputs binding a name inside it. In
this sense, the lift operator will be seen as a way to dynamically
construct processes before reifying them as names.

Finally equipped with these standard features we can present the
dynamics of the calculus.

\subsubsection{Operational semantics} 

Finally, we introduce the computational dynamics. What marks these
algebras as distinct from other more traditionally studied algebraic
structures, e.g. vector spaces or polynomial rings, is the manner in
which dynamics is captured. In traditional structures, dynamics is typically
expressed through morphisms between such structures, as in linear maps
between vector spaces or morphisms between rings. In algebras
associated with the semantics of computation, the dynamics is
expressed as part of the algebraic structure itself, through a
reduction reduction relation typically denoted by $\red$. Below, we
give a recursive presentation of this relation for the calculus used
in the encoding.

$\red \subseteq \pi \times \pi$
$\red : \pi \to \mathcal{P}(\pi)$

\begin{mathpar}
  \inferrule* [lab=Comm] { \textsf{match}( x_{src}, x_{trgt} ) } { x_{trgt}?(y)P \; | \; x_{src}!\langle {Q} \rangle \red P\{\quotep{Q}/y}\} }
  \and \\
  \inferrule* [lab=Par] {{P} \red {P}'} {{{P} | {Q}} \red {{P}' | {Q}}}
  \and
  \inferrule* [lab=Equiv]{{{P} \scong {P}'} \andalso {{P}' \red {Q}'} \andalso {{Q}' \scong {Q}}}{{P} \red {Q}}
\end{mathpar}

\begin{eqnarray*}
  match_{\equiv} (\quotep{P},\quotep{Q}) & := & P \equiv Q \\
  match_{\dagger}(\quotep{P},\quotep{Q}) & := & \forall R. P|Q \red^{*} R => R \red^{*} 0 \\
  match_{K}(\quotep{P},\quotep{Q}) & := & K \mbox{ for some context } K
\end{eqnarray*}

$u?(x)P | u!\langle Q \rangle \red P\{\quotep{Q}/x\}$

%We write $\wred$ for $\red^*$, and $P\red$ if $\exists Q $ such that $ P \red Q$.
We write $P\red$ if $\exists Q $ such that $ P \red Q$ and $P\not\red$, otherwise.

\section{Replication}

As mentioned before, it is known that replication (and hence
recursion) can be implemented in a higher-order process algebra
\cite{SangiorgiWalker}. As our first example of calculation with the
machinery thus far presented we give the construction explicitly in
the {\rhoc}.

\begin{eqnarray}
	D_{x} & := & \prefix{x}{y}{(\binpar{\outputp{x}{y}}{@{y}})} \nonumber\\
	\bangp_{x}{P} & := & \binpar{{x}!\langle{\binpar{D_{x}}{P}}\rangle}{D_{x}} \nonumber
\end{eqnarray}

\begin{eqnarray}
	\bangp_{x}{P} & & \nonumber\\
	=
	& {x}!\langle{(\prefix{x}{y}{(\outputp{x}{y} | @{y})) | P}}\rangle 
	      | \prefix{x}{y}{(\outputp{x}{y} | @{y})} & \nonumber\\
	\red
	& (\outputp{x}{y} | @{y})\substn{\quotep{(\prefix{x}{y}{(@{y} | \outputp{x}{y})) | P}}}{y} & \nonumber\\
	=
	& \outputp{x}{\quotep{(\prefix{x}{y}{(\outputp{x}{y} | @{y})) | P}}}
	  | {(\prefix{x}{y}{(\outputp{x}{y} | @{y})) | P}} & \nonumber\\
	\red
	& \ldots & \nonumber\\
	\red^*
	& P | P | \ldots & \nonumber
\end{eqnarray}

Of course, this encoding, as an implementation, runs away, unfolding
$\bangp{P}$ eagerly. A lazier and more implementable replication
operator, restricted to input-guarded processes, may be obtained as follows.

\begin{eqnarray}
\bangp{\prefix{u}{v}{P}} 
	:= 
	\binpar{\lift{x}{\prefix{u}{v}{(\binpar{D(x)}{P})}}}{D(x)} \nonumber
\end{eqnarray}

\begin{remark}
  Note that the lazier definition still does not deal with summation
  or mixed summation (i.e. sums over input and output). The reader is
  invited to construct definitions of replication that deal with these
  features. 

  Further, the definitions are parameterized in a name, $x$. Can you,
  gentle reader, make a definition that eliminates this parameter and
  guarantees no accidental interaction between the replication
  machinery and the process being replicated -- i.e. no accidental
  sharing of names used by the process to get its work done and the
  name(s) used by the replication to effect copying. This latter
  revision of the definition of replication is crucial to obtaining
  the expected identity $!!P \sim !P$.
\end{remark}

\begin{remark}\label{rem:paradoxical_combinator}
  The reader familiar with the lambda calculus will have noticed the
  similarity between $D$ and the paradoxical combinator.

  [Ed. note: the existence of this seems to suggest we have to be more
  restrictive on the set of processes and names we admit if we are to
  support no-cloning.]
\end{remark}

\subsubsection{Bisimulation}

The computational dynamics gives rise to another kind of equivalence,
the equivalence of computational behavior. As previously mentioned
this is typically captured \emph{via} some form of bisimulation.

% The notion we use in this paper is weak barbed bisimulation
% \cite{milner91polyadicpi}.

The notion we use in this paper is derived from weak barbed
bisimulation \cite{milner91polyadicpi}. 

\begin{definition}
An \emph{observation relation}, $\downarrow_{\mathcal N}$, over a set
of names, $\mathcal N$, is the smallest relation satisfying the rules
below.

\infrule[Out-barb]{y \in {\mathcal N}, \; x \nameeq y}
		  {\outputp{x}{v} \downarrow_{\mathcal N} x}
\infrule[Par-barb]{\mbox{$P\downarrow_{\mathcal N} x$ or $Q\downarrow_{\mathcal N} x$}}
		  {\binpar{P}{Q} \downarrow_{\mathcal N} x}

We write $P \Downarrow_{\mathcal N} x$ if there is $Q$ such that 
$P \wred Q$ and $Q \downarrow_{\mathcal N} x$.
\end{definition}

\begin{definition}
%\label{def.bbisim}
An  ${\mathcal N}$-\emph{barbed bisimulation} over a set of names, ${\mathcal N}$, is a symmetric binary relation 
${\mathcal S}_{\mathcal N}$ between agents such that $P\rel{S}_{\mathcal N}Q$ implies:
\begin{enumerate}
\item If $P \red P'$ then $Q \wred Q'$ and $P'\rel{S}_{\mathcal N} Q'$.
\item If $P\downarrow_{\mathcal N} x$, then $Q\Downarrow_{\mathcal N} x$.
\end{enumerate}
$P$ is ${\mathcal N}$-barbed bisimilar to $Q$, written
$P \wbbisim_{\mathcal N} Q$, if $P \rel{S}_{\mathcal N} Q$ for some ${\mathcal N}$-barbed bisimulation ${\mathcal S}_{\mathcal N}$.
\end{definition}

$\mathcal{R} \subseteq \pi \times \pi$

$P \mathcal{R} Q => \forall P'. P \red P' \Rightarrow \exists Q'. Q \red Q', P' \mathcal{R} Q'$

$P \vdash x \Rightarrow Q \vdash x$

\begin{mathpar}
  \inferrule*[lab=Out-barb]{x \nameeq y}{{y}!\langle{Q}\rangle \vdash x}
  \and
  \inferrule*[lab=Par-barb]{\mbox{$P\vdash x$ or $Q\vdash x$}}{\binpar{P}{Q} \vdash x}
\end{mathpar}

\subsubsection{Contexts}

One of the principle advantages of computational calculi like the
$\pi$-calculus is a well-defined notion of context,
contextual-equivalence and a correlation between
contextual-equivalence and notions of bisimulation. The notion of
context allows the decomposition of a process into (sub-)process and
its syntactic environment, its context. Thus, a context may be
thought of as a process with a ``hole'' (written $\Box$) in it. The
application of a context $M$ to a process $P$, written $M[P]$, is
tantamount to filling the hole in $M$ with $P$. In this paper we do
not need the full weight of this theory, but do make use of the notion
of context in the proof the main theorem. 

\begin{mathpar}
  \inferrule* [lab=summation] {} {{M_{M},M_{N}} \bc \Box \;|\; x.M_{A} \;|\; M_{M}+M_{N}}
  \and
  \inferrule* [lab=agent] {} {{M_{A}} \bc (\vec{x})M_{P} \;| \; \clift{P_0,\ldots,M_{P},\ldots,P_N}}
  \and \\
  \inferrule* [lab=process] {} {{M_{P}} \bc M_{N} \;| \;P|M_{P} }
\end{mathpar} 

\begin{mathpar}
  \inferrule* [lab=sychronization] {} {M_{N} \bc \Box \;|\; x?M_{F} \;|\; x!M_{C}}
  \and
  \inferrule* [lab=abstraction] {} {{M_{F}} \bc (x)M_{P} }
  \and
  \inferrule* [lab=concretion] {} {{M_{C}} \bc \langle M_{P} \rangle }
  \and \\
  \inferrule* [lab=process] {} {{M_{P}} \bc M_{N} \;| \;P|M_{P} }
\end{mathpar}

\begin{definition}[contextual application] Given a context $M$, and
  process $P$, we define the \emph{contextual application}, $M[P] :=
  M\{P/\Box\}$. That is, the contextual application of M to P is the
  substitution of $P$ for $\Box$ in $M$.
\end{definition}

$\meaningof{-} : L \to \mathcal{P}(\pi)$

\begin{mathpar}
  \inferrule* [lab=collection] {} {\meaningof{true} = \pi, \and \meaningof{~E} = \pi \setminus \meaningof{E}, \and \meaningof{E_{1} \& E_{2}} = \meaningof{E_{1}} \cap \meaningof{E_{2}}}
\end{mathpar}

\begin{mathpar}
  \inferrule* [lab=structure] {} {\meaningof{0} = \{ P \in \pi | P \equiv 0 \}, \and \\ \meaningof{E_1 | E_2} = \{ P \in \pi | P \equiv P_{1} | P_{2}, P_{1} \in \meaningof{E_{1}}, P_{2} \in \meaningof{E_2}\} }
\end{mathpar}

\begin{mathpar}
 \inferrule* [lab=behavior] {} {\meaningof{\langle a?b \rangle E} = \{ P \in \pi | P \equiv Q | u?(y)P', \\ \and \\\\ \and \\ \;\;\; u \in \meaningof{a}, \forall z.P'\{z/y\} \in \meaningof{E\{z/b\}}\}, \and \\ \meaningof{a!E} = \{ P \in \pi | P \equiv Q | x!\langle P' \rangle, x \in \meaningof{a} P' \in \meaningof{E}\} }
\end{mathpar}

\begin{mathpar}
 \inferrule* [lab=nominal] {} {\meaningof{\quotep{E}} = \{ \quotep{P} \in \quotep{\pi} | P \in \meaningof{E} \}, \and \meaningof{\quotep{P}} = \{ \quotep{Q} \in \quotep{\pi} | P \equiv Q \} \and \\ \meaningof{@\quotep{E}} = \{ P \in \pi | P \equiv @x, x \in \meaningof{E} \}}
\end{mathpar}

\begin{eqnarray*}
  \\
  \meaningof{-} : TS \to ST
\end{eqnarray*}

\begin{eqnarray*}
  \\
  L : TS \to ST
\end{eqnarray*}

\begin{eqnarray*}
  \\
  P \models E \iff P \in \meaningof{E}
\end{eqnarray*}

\begin{eqnarray*}
  P \approx_{L} Q \iff \forall E \in L. P \models E \iff Q \models E
\end{eqnarray*}

\begin{eqnarray*}
  P \approx_{K} Q
\end{eqnarray*}

\begin{eqnarray*}
  P \approx Q
\end{eqnarray*}

$\approx_{K} = \approx = \approx_{L}$

\subsubsection{Contextual duality}

Note that contexts extend the quotation operation to a family of
operations from processes to names. Given a context, $M$, we can
define a \emph{nominal context}, $\quotep{M}$ by $\quotep{M}[P] :=
\quotep{M[P]}$. To foreshadow what is to come we observe that these
operations enjoy a duality with processes very much like the duality
between vectors and maps from vectors to scalars.

Further, because the calculus is essentially higher-order, we have a
correspondence between contexts and processes. More specifically,
given a name $x$ and a context $M$ we can construct $M^{*}_{x}$ such
that 

\begin{mathpar}
  M^{*}_{x} | \lift{x}{P} \red M[P]
\end{mathpar}

namely,

\begin{mathpar}
  M^{*}_{x} := x?(u).M[\dropn{u}]
\end{mathpar}

The dependence of $M^{*}_{x}$ on a name makes it an abstraction, 

\begin{mathpar}
  M^{*} := (x)x?(u).M[\dropn{u}]
\end{mathpar}

\subsection{Additional notation}

It will sometimes be convenient to denote the process a name
quotes. We already have the notation $x = \quotep{P}$, but it will be
convenient to introduce an alternate notation, $\procn{x}$, when we
want to emphasize the connection to the use of the name. Note that, by
virtue of name equivalence, $\quotep{\procn{x}} \nameeq x$; so, the
notation is consistent with previous definitions.

Further, because names have structure it is possible to effect
substitutions on the basis of that structure. This means we need to
upgrade our notation for substitutions, which we accomplish by
adapting comprehension notation. Thus,

\begin{mathpar}
  P\{ y / x : x \in S \}
\end{mathpar}

is interpreted to mean the process derived from P by replacing (in a
capture-avoiding manner) each occurrence of $x$ in $S$ by $y$. For example,

\begin{mathpar}
  P\{ \quotep{\procn{x}|\procn{x}} / x : x \in \freenames{P} \}
\end{mathpar}

will replace each (occurrence) of a free name $x$ in $P$ by
$\quotep{\procn{x}|\procn{x}}$.

Also, we will avail ourselves of the notation $x^{L}$ and $x^{R}$ to
denote injections of a name into disjoint copies of the name
space. There are numerous ways to accomplish this. One example can be
found in \cite{MeredithR05}. This notation overloads to vectors of
names: $\vec{x}^{\pi} := (x_{i}^{\pi} \; : \; 0 \leq i < |\vec{x}| )$ where $\pi \in \{L,R\}$.

We also use $P^{\Box} := P|\Box$.

In \cite{MeredithR05} an interpretation of the new operator is
given. It turns out that there are several possible interpretations
all enjoying the requisite algebraic properties of the operator (see
\cite{milner91polyadicpi}). We will therefore make liberal use of
$(\nu\; \vec{x})P$.

% subsection the_syntax_and_semantics_of_the_notation_system (end)   

\section{Interpretation of QM}
\subsection{Supporting definitions}
\subsubsection{Multiplication}
\begin{mathpar}
  \quotep{Q} \cdot \quotep{R} := \quotep{Q|R}
  \and \\
  \quotep{Q} \cdot P := P\{ \quotep{Q|R} / \quotep{R} : \quotep{R} \in \freenames{P} \}
\end{mathpar}

\paragraph{Discussion}
The first line needs little explanation. The second line says that
each free name of the process is replaced with the multiplication of
that name by the scalar. Multiplication of a scalar (name) by a state
(process) results in a process all the names of which have been `moved
over' by parallel composition with the process the scalar
quotes. There is a subtlety that the bound names have to be
manipulated so that multiplied names aren't accidentally
captured. There are many ways to achieve this.

\begin{remark}\label{rem:multiplication_identities}
  The reader is invited to verify that for all $x,y,z \in \QProc$ and $P \in \Proc$
  \begin{mathpar}
    x \cdot \quotep{0} \equiv x 
    \and
    x \cdot y \equiv y \cdot x
    \and
    x \cdot (y \cdot z) \equiv (x \cdot y) \cdot z
    \and \\
    \quotep{0} \cdot P \equiv P
    \and \\
    x \cdot (y \cdot P) \equiv (x \cdot y) \cdot P
    \and \\
    x \cdot (P|Q) \equiv (x \cdot P) | (x \cdot Q)
    \and \\    
  \end{mathpar}
\end{remark}

\subsubsection{Tensor product}

We define a tensor product on processes by structural induction.

\paragraph{Tensor of sums} First note that all summations, including
$\pzero$ and sequence, can be written $\Sigma_{i} x_{i}.A_{i} +
\Sigma_{j} x_{j}.C_{j}$, where we have grouped input-guarded processes
together and output-guarded processes together.

Thus, we can define the tensor product of two summations, $N_{1}\otimes N_{2}$, where

\begin{mathpar}
  N_{1} := \Sigma_{i} x_{i}.A_{i} + \Sigma_{j} x_{j}.C_{j}
  \and
  N_{2} := \Sigma_{i'} y_{i'}.B_{i'} + \Sigma_{j'} y_{j'}.D_{j'} 
\end{mathpar}

as follows.

\begin{mathpar}
  \Sigma_{i} x_{i}.A_{i} + \Sigma_{j} x_{j}.C_{j} \otimes \Sigma_{i'}
  y_{i'}.B_{i'} + \Sigma_{j'} y_{j'}.D_{j'} 
  \and \\
  := \; \Sigma_{i} \Sigma_{i'} \quotep{\stackrel{\vee}{x_{i}}| \stackrel{\vee}{y_{i'}}}.(A_{i}\otimes B_{i'}) \; | \; \Sigma_{i'} \Sigma_{i} \quotep{\stackrel{\vee}{y_{i'}}|\stackrel{\vee}{x_{i}}}.(B_{i'}\otimes A_{i})
  \and
  \;\; | \;\; \Sigma_{j} \Sigma_{j'} \quotep{\stackrel{\vee}{x_{j}}|\stackrel{\vee}{y_{j'}}}.(A_{j}\otimes B_{j'}) \; | \; \Sigma_{j'} \Sigma_{j} \quotep{\stackrel{\vee}{y_{j'}}|\stackrel{\vee}{x_{j}}}.(B_{j'}\otimes A_{j})
\end{mathpar}

\begin{remark}
  Do we need to $x^{L}$ and $y^{R}$ for this construction as well?
\end{remark}

\paragraph{Tensor of parallel compositions} Next, we distribute tensor
over par.

\begin{mathpar}
  P_{1}|P_{2} \otimes Q_{1}|Q_{2} := (P_{1} \otimes Q_{1}) | (P_{1}
  \otimes Q_{2}) | (P_{2} \otimes Q_{1}) | (P_{2} \otimes Q_{2})
\end{mathpar}

\paragraph{Tensor with dropped names} We treat tensor of a
process with a dropped name as parallel composition.

\begin{mathpar}
  P \otimes \dropn{x} := P | \dropn{x}
\end{mathpar}

\paragraph{Tensor of agents}

Finally, we need to define tensor on agents. Note that the definition
of tensor on normal products only tensors inputs with inputs and
outputs with outputs. Thus, we only have to define the operation on
``homogeneous'' pairings.

\begin{mathpar}
  (\vec{x})P \otimes (\vec{y})Q
  \and \\
  := (x_{0}^{L}|y_{0}^{R},\ldots,x_{0}^{L}|y_{n}^{R},\ldots,x_{m}^{L}|y_{0}^{R},\ldots,x_{m}^{L}|y_{n}^R)(P\{ \vec{x}^{L}/\vec{x}\} \otimes Q \{ \vec{y}^{R}/\vec{y}\})
  \and \\
  \clift{\vec{P}} \otimes \clift{\vec{Q}}
  \and \\
  := \clift{P_{0}\otimes Q_{0},\ldots,P_{0}\otimes Q_{n},\ldots,P_{m}\otimes Q_{0},\ldots,P_{m}\otimes Q_{n}}
\end{mathpar}

\begin{remark}
  Observe that arities of tensored abstractions matches arities of
  tensored concretions if the original arities matched. Note also that
  the length of the arities corresponds to the increase in dimension
  we see in ordinary vector space tensor product.
\end{remark}

\begin{remark}
  Operationally, this definition distributes the tensor down to
  components ``linked'' by summation. Tensor over summation is
  intriguing in that it mixes names. Moreover, as a consequence of the
  way it mixes names we have the identities for all $x \in \QProc$ and
  $P,Q \in \Proc$

  \begin{mathpar}
    (x \cdot P) \otimes Q \equiv x \cdot (P \otimes Q) \equiv P \otimes (x \cdot Q)
    \and
    P \otimes \pzero \equiv P
  \end{mathpar}

  that the reader is invited to verify.
\end{remark}

\subsubsection{Annihilation}
\begin{mathpar}
  P^{\perp} := \{ Q | \forall R. P|Q \red^{*} R \Rightarrow R \red^{*} \pzero \}
  \and \\
  P^{\underline{\perp}} := \Sigma_{Q \in P^{\perp}} \quotep{Q}?(y).(\dropn{y}|Q) | \Sigma_{Q \in P^{\perp}} \quotep{Q}\clift{\Box}
\end{mathpar}

\paragraph{Discussion} The reader will note that $P^{\perp}$ is a
\emph{set} of processes, while $P^{\underline{\perp}}$ is a
\emph{context}. We call the set $P^{\perp}$ the \emph{annihilators} of
$P$. The parallel composition of a process in the annihilators of $P$
with $P$ will result in a process, the state space of which has all
paths eventually leading to $\pzero$. Execution may endure loops; but
under reasonable conditions of fairness (naturally guaranteed under
most notions of bisimulation) such a composite process cannot get
stuck in such a loop and will, eventually pop out and terminate.

The context $P^{\underline{\perp}}$ is ready and willing to ``take the
$P$ out of'' the process to which it is applied. It will effectively
transmit the code of the process to which it is applied to one of the
annihilators and run the process against it.

\subsubsection{Evaluation}
We fix $M$ a domain of fully abstract interpretation with an equality
coincident with bisimulation. We take $\meaningof{\cdot} : \Proc \to
M$ to be the map interpreting processes and $\nmeaningof{\cdot} : \M
\to Proc$ to be the map running the other way. Then we define

\begin{mathpar}
  \int P := \nmeaningof{\meaningof{P}}
\end{mathpar}

\paragraph{Discussion}
There are many fully abstract interpretations of Milner's
$\pi$-calculus. Any of them can be used as a basis for interpreting
the reflective calculus here. Equipped with such a domain it is
largely a matter of grinding through to check that the Yoneda
construction for the normalization-by-evaluation program can be
extended to this setting.

\begin{remark}
  The reader is invited to verify that $\int (P^{\underline{\perp}}[P]) = 0$.
\end{remark}

\subsection{Quantum mechanics}

Table \ref{tbl:core_qm_op_defns} gives the core operational definitions

\begin{table}[htp]\label{tbl:core_qm_op_defns}
  \center{
    \fbox{
      \begin{tabular}{c|c}
        quantum mechanics & process calculus \\
        \hline
        scalar & $x := \quotep{P}$ \\
        state vector & $\state{P} := P$ \\
        dual & $\state{P}^{*} := \event{P^{\underline{\perp}}} := \quotep{P^{\underline{\perp}}}[-]$ \\
        matrix & $ \Sigma_{\alpha} \state{P_{\alpha}}x_{\alpha}\event{Q_{\alpha}}$ \\
        vector addition & $\state{P} + \state{Q} := \state{P | Q}$ \\
        tensor product & $\state{P} \otimes \state{Q} := \state{P \otimes Q}$ \\
        inner product & $\innerprod{P}{Q} := \quotep{\int P^{\underline{\perp}}[Q]}$ \\
      \end{tabular}
    }
  }
  \caption{QM - operational definitions}
\end{table}

where

\begin{mathpar}
  \prmatrix{P}{Q} := \fprmatrix{P}{\quotep{\pzero}}{Q}
  \and
  \fprmatrix{P}{x}{Q} := (\state{P},x,\event{Q})
  \and
  (\fprmatrix{P}{x}{Q})(\state{R}) := x \cdot \innerprod{Q}{R} \cdot \state{P}
  \and
  (\fprmatrix{P}{x}{Q})(\event{R}) := x \cdot \innerprod{R}{P} \cdot \event{Q}
\end{mathpar}

\paragraph{Discussion}
As promised: vectors (aka states) are represented as processes; duals
as contextual duals; inner product definition should be compared with
standard inner product definition for ....

\begin{remark}
  Assuming $\int (P^{\underline{\perp}}[P]) = 0$, the reader is
  invited to verify that $(\fprmatrix{P}{x}{P})(\state{P}) = x \cdot \state{P}$.
\end{remark}

\begin{remark}
  The reader is invited to verify that $\innerprod{P}{Q}$ could
  equally well have been written $\quotep{\int \stackrel{\vee}{x}}$
  where $x = \event{P^{\underline{\perp}}}(Q)$.

  One of the motivations for this remark is that there is another way
  to factor these operations. We could package up evaluation in the dual:

  \begin{mathpar}
    \state{P}^{*} := \event{\int P^{\underline{\perp}}} := \quotep{\int P^{\underline{\perp}}}[-]
  \end{mathpar}

  and then have inner product defined by
  
  \begin{mathpar}
    \innerprod{P}{Q} := \event{P}(Q)
  \end{mathpar}

  Hopefully, experience with the calculations will provide guidance on
  the best factoring.
\end{remark}

\begin{remark}
  Assuming $\int (P^{\underline{\perp}}[P]) = 0$, the reader is
  invited to verify that $\forall P,Q. (\prmatrix{0}{Q})(\state{0}) =
  \state{0}$ and dually $(\prmatrix{P}{0})(\event{0}) = \event{0}$.
\end{remark}

\begin{remark}
  i'm a little worried that i don't (yet) have proper support for
  complex conjugacy. But, the observation above may give us a
  clue. According to Abramsky, it must be the case that the scalars
  are iso to the homset of the identity for the tensor -- which the
  observation above characterizes. 

  For now, we will simply bookmark the notion with $\overline{x}$.
\end{remark}

\subsubsection{Adjointness}

We need to give a definition of $(\cdot)^{\dagger}$ for matrices. The
obvious candidate definition is
\begin{mathpar}
(\Sigma_{\alpha}\fprmatrix{P_{\alpha}}{x_{\alpha}}{Q_{\alpha}})^{\dagger}
= \Sigma_{\alpha}\fprmatrix{(Q_{\alpha}^{\underline{\perp}})^{*}}{\overline{x}_{\alpha}}{P_{\alpha}^{\underline{\perp}}} 
\end{mathpar}

But, $(Q_{\alpha}^{\underline{\perp}})^{*}$ requires a name along
which to communicate the process to achieve the context application.

\subsubsection{Basis for a basis}
If processes label states and ``addition'' of states (a.k.a. vector
addition) is interpreted as parallel composition, what corresponds to
notions of linear independence and basis? Here, we recall that Yoshida
has developed a set of \emph{combinators} for an asynchronous verison
of Milner's $\pi$-calculus. These are a finite set of processes such
any process can be expressed as parallel composition of these
combinators together with liberal uses of the new operator and
replication. We can simply give a translation of these into the
present calculus and have reasonable expectation that the property
carries over. That is, that the resultant set allows to express all
processes via parallel composition. Note, however, that there is no
new operator or replication in this calculus. As a result, we expect
that the corresponding set is actually infinite. That is, we expect
that the space is actually infinite dimensional.

\begin{remark}
  The attentive reader may be a bit concerned. Certainly, the
  collection $S$, $K$ and $I$ is a finite set of
  combinators. Shouldn't we expect to see a finite set of combinators
  for an effectively equivalent system? i am very sympathetic to this
  critique and feel it warrants full attention. On the other hand, i
  also have in mind the following analogy. The natural numbers, as a
  monoid under addition, has exactly $1$ generator, while the natural
  numbers, as a monoid under multiplication, has countably many
  generators (the primes). We observe that the application of the
  lambda calculus is much less resource sensitive than the parallel
  composition of the $\pi$-calculus. Could it be the case that we have
  an analogy of the form
  
  \begin{mathpar}
    m + n : MN :: m*n : M|N
  \end{mathpar}

  giving a similar blow up in the set of ``primes''?  This is such a
  wonderful thought that, even if it's not true, i think it's worth
  writing down.
\end{remark}
 

\documentclass[12pt]{llncs}
%\documentclass{jktr}

\usepackage[pdftex]{hyperref}                   
\usepackage {listings}
\usepackage {mathpartir}
\usepackage{bcprules}
%\usepackage{listings}
                       
\usepackage{graphicx} 
%\usepackage[margins=2.5cm,nohead,nofoot]{geometry}
%\usepackage{geometry}
\usepackage{amsfonts}
\usepackage{amstext}
\usepackage{latexsym}
\usepackage{amssymb}
\usepackage{color}


%\include{myPreamble}
\include{qm2pi.local} 

%\ifpdf
%\usepackage[pdftex]{graphicx}
%\else
%\usepackage{graphicx}
%\fi

 % \ifpdf
%  \usepackage{pdfsync}
%  \if


%\title{Brief Article}
%\author{David F. Snyder}
%\author{L.G. Meredith}

%\address{Dept. of Math., Texas State University--San Marcos, San Marcos, TX 78666}
       
\pagestyle{empty}


\begin{document}

\lstset{language=[Objective]Caml,frame=shadowbox}

\input{qm2pi.front}

% section front matter (end)

\input{qm2pi.intro} 
 
% section introduction (end)

% \input{qm2pi.knotations} 

% section notation (end)

\input{qm2pi.process.calculi} 

% section concurrent_process_calculi_and_spatial_logics_ (end)
    
%\input{qm2pi.knots2pi} 

%\input{qm2pi.trefoil} 

%\input{qm2pi.mainthm} 

% subsection basic_interpretation (end)

%\input{qm2pi.rho.presentation} 
\subsection{The syntax and semantics of the notation system}\label{sub:the_syntax_and_semantics_of_the_notation_system} % (fold)

We now summarize a technical presentation of the calculus that
embodies our theory of dynamics. The typical presentation of such a
calculus follows the style of giving generators and relations on
them. The grammar, below, describing term constructors, freely
generates the set of processes, $\Proc$. This set is then quotiented
by a relation known as structural congruence and it is over this set
that the notion of dynamics is expressed. This presentation is
essentially that of \cite{MeredithR05} with the addition of
polyadicity and summation. For readability we have relegated some of
the technical subtleties to an appendix.

\subsubsection{Process grammar}\label{subsub:process_grammar}

\begin{mathpar}
  \inferrule* [lab=synchronization] {} {{M} \bc \pzero \;|\; x?F \;|\; x!C }
  \and
  \inferrule* [lab=abstraction] {} {{F} \bc (x)P}
  \and
  \inferrule* [lab=concretion] {} {{C} \bc \langle Q \rangle}
  \and
  \inferrule* [lab=process] {} {{P,Q} \bc M \;| \;P|Q \;|\; @{x}}
  \and
  \inferrule* [lab=name] {} {{x} \bc \quotep{P}}
\end{mathpar} 

Note that $\vec{x}$ (resp. $\vec{P}$) denotes a vector of names
(resp. processes) of length $|\vec{x}|$ (resp. $|\vec{P}|$). We adopt
the following useful abbreviations.

\begin{mathpar}
   x?(\vec{y}).P := x.(\vec{y})P \and  x\clift{\vec{P}} := x.\clift{\vec{P}}
   \and x!(y) := \lift{x}{\dropn{y}}
   \and \Pi_{i=0}^{n-1}P_i := P_0 | \ldots | P_{n-1}
\end{mathpar}

\subsubsection{Structural congruence}

\paragraph{Free and bound names and alpha-equivalence.} At the
core of structural equivalence is alpha-equivalence which identifies
process that are the same up to a change of variable. Formally, we
recognize the distinction between free and bound names. The free names
of a process, $\freenames{P}$, may be calculated recursively as
follows:

\begin{mathpar}
\freenames{\pzero} := \emptyset
  \and \\
  \freenames{x?(y).P} := \{ x \} \cup (\freenames{P} \setminus \{ y \})
  \and 
  \freenames{x!\langle P \rangle} := \{ x \} \cup \{ P \} 
  \and \\
  \freenames{P|Q} := \freenames{P} \cup \freenames{Q}
  \and \\
  \freenames{@{x}} := \{ x \}
\end{mathpar}

$\pi$
$\quotep{\pi}$

$\freenames{-} : \pi \to \mathcal{P}(\quotep{\pi})$

\begin{eqnarray*}
  \freenames{\pzero} & := & \emptyset \\
  \freenames{x?(y).P} & := & \{ x \} \cup (\freenames{P} \setminus \{ y \}) \\
  \freenames{x!\langle P \rangle} & := & \{ x \} \cup \{ P \} \\
  \freenames{P|Q} & := & \freenames{P} \cup \freenames{Q} \\
  \freenames{\dropn{x}} & := & \{ x \}
\end{eqnarray*}

The bound names of a process, $\boundnames{P}$, are those names occurring in $P$
that are not free. For example, in $x?(y).0$, the name $x$ is free, while $y$ is bound.

\begin{mathpar}
  \inferrule* [lab=monoidal-laws] {} { P|Q \equiv Q|P \and P|0 \equiv P \and P|(Q|R) \equiv (P|Q)|R }
\end{mathpar}

\begin{mathpar}
  \inferrule* [lab=alpha-equivalence] {} { (x)P \equiv (y)P\{y/x\} \and y \not\in \freenames{P} }
\end{mathpar}

\begin{definition}
Then two processes, $P,Q$, are alpha-equivalent if $P = Q\{\vec{y}/\vec{x}\}$ for
some $\vec{x} \in \boundnames{Q},\vec{y} \in \boundnames{P}$, where $Q\{\vec{y}/\vec{x}\}$
denotes the capture-avoiding substitution of $\vec{y}$ for $\vec{x}$ in $Q$.
\end{definition}

\begin{definition}
  The {\em structural congruence} \cite{SangiorgiWalker} , $\equiv$,
  between processes is the least congruence containing
  alpha-equivalence, satisfying the abelian monoid laws
  (associativity, commutativity and $\pzero$ as identity) for parallel
  composition $|$ and for summation $+$.
\end{definition}

\subsection{Name equivalence}

We take name equivalence, written $\nameeq$, to be the smallest
equivalence relation generated by the following rules.

\begin{mathpar}
\inferrule*[lab=Quote-drop]
{ }
{ \quotep{@{x}} \nameeq x }

\inferrule*[lab=Struct-equiv]
{ P \scong Q }
{ \quotep{P} \nameeq \quotep{Q} }
\end{mathpar}

The astute reader will have noticed that the mutual recursion of names
and processes imposes a mutual recursion on alpha-equivalence and
structural equivalence via name-equivalence. Fortunately, all of this
works out pleasantly and we may calculate in the natural way, free of
concern. The reader interested in the details is referred to the
appendix \ref{appendix:rho_details}.

\subsection{Substitution}

We use $\Proc$ for the set of processes, $\QProc$ for the set of
names, and $\id{\{}\vec{y} / \vec{x} \id{\}}$ to denote partial maps,
$s : \QProc \rightarrow \QProc$. A map, $s$ lifts, uniquely, to a map
on process terms, $\widehat{s} : \Proc \rightarrow \Proc$ by the
following equations.

\begin{mathpar}
  (0) \psubstp{Q}{P} := 0 \\
  (R \juxtap S) \psubstp{Q}{P}
  :=    
  (R)\psubstp{Q}{P} \juxtap (S) \psubstp{Q}{P} \\
  (x?(y).R) \psubstp{Q}{P}    
  :=    
  (x)\substp{Q}{P} (z)\concat( (R \psubstn{z}{y}) \psubstp{Q}{P} ) \\
  (\lift{x}{R}) \psubstp{Q}{P}  
  :=
  \lift{(x)\substp{Q}{P}}{ R \psubstp{Q}{P} } \\
%   (\dropn{x})  \psubstp{Q}{P}       
%   := 
%   \left\{ 
%     \begin{array}{ccc} 
%       \dropn{\quotep{Q}} & & x \nameeq \quotep{P} \\
%       \dropn{x} & & otherwise \\
%     \end{array}
%   \right. 
  (\dropn{x})  \psubstp{Q}{P}       
  := 
  \left\{ 
    \begin{array}{ccc} 
      Q & & x \nameeq \quotep{P} \\
      \dropn{x} & & otherwise \\
    \end{array}
  \right.
\end{mathpar}
 

where

\begin{eqnarray}
  (x)\id{\{} \lpquote Q \rpquote / \lpquote P \rpquote \id{\}}            = 
  \left\{ 
    \begin{array}{ccc}
      \lpquote Q \rpquote & & x \nameeq \lpquote P \rpquote \\
      x & & otherwise \\
    \end{array}
  \right. \nonumber
\end{eqnarray}

and $z$ is chosen distinct from $\quotep{P}$, $\quotep{Q}$, the free
names in $Q$, and all the names in $R$. Our $\alpha$-equivalence will
be built in the standard way from this substitution.

\begin{remark}\label{rem:no_self_referential_names}
  One consequence of these definitions is that $\forall P. \quotep{P}
  \not\in \freenames{P}$.
\end{remark}

\subsection{ Dynamic quote: an example }

Anticipating something of what's to come, consider applying the
substitution, $\widehat{\id{\{}u / z \id{\}}}$, to the following pair
of processes, $\lift{w}{y!(z)}$ and $w[ \lpquote y!(z) \rpquote ]$.

\begin{eqnarray}
	\lift{w}{y!(z)}\widehat{\id{\{}u / z \id{\}}}
		& = &
		\lift{w}{y!(u)} \nonumber\\
	w[ \lpquote y!(z) \rpquote ] \widehat{ \id{\{}u / z \id{\}} }
		& = &
		w[ \lpquote y!(z) \rpquote ] \nonumber
\end{eqnarray}

Because the body of the process between quotes is impervious to
substitution, we get radically different answers. In fact, by
examining the first process in an input context,
e.g. $x?(z).\lift{w}{y!(z)}$, we see that the process under the lift
operator may be shaped by prefixed inputs binding a name inside it. In
this sense, the lift operator will be seen as a way to dynamically
construct processes before reifying them as names.

Finally equipped with these standard features we can present the
dynamics of the calculus.

\subsubsection{Operational semantics} 

Finally, we introduce the computational dynamics. What marks these
algebras as distinct from other more traditionally studied algebraic
structures, e.g. vector spaces or polynomial rings, is the manner in
which dynamics is captured. In traditional structures, dynamics is typically
expressed through morphisms between such structures, as in linear maps
between vector spaces or morphisms between rings. In algebras
associated with the semantics of computation, the dynamics is
expressed as part of the algebraic structure itself, through a
reduction reduction relation typically denoted by $\red$. Below, we
give a recursive presentation of this relation for the calculus used
in the encoding.

$\red \subseteq \pi \times \pi$
$\red : \pi \to \mathcal{P}(\pi)$

\begin{mathpar}
  \inferrule* [lab=Comm] { \textsf{match}( x_{src}, x_{trgt} ) } { x_{trgt}?(y)P \; | \; x_{src}!\langle {Q} \rangle \red P\{\quotep{Q}/y}\} }
  \and \\
  \inferrule* [lab=Par] {{P} \red {P}'} {{{P} | {Q}} \red {{P}' | {Q}}}
  \and
  \inferrule* [lab=Equiv]{{{P} \scong {P}'} \andalso {{P}' \red {Q}'} \andalso {{Q}' \scong {Q}}}{{P} \red {Q}}
\end{mathpar}

\begin{eqnarray*}
  match_{\equiv} (\quotep{P},\quotep{Q}) & := & P \equiv Q \\
  match_{\dagger}(\quotep{P},\quotep{Q}) & := & \forall R. P|Q \red^{*} R => R \red^{*} 0 \\
  match_{K}(\quotep{P},\quotep{Q}) & := & K \mbox{ for some context } K
\end{eqnarray*}

$u?(x)P | u!\langle Q \rangle \red P\{\quotep{Q}/x\}$

%We write $\wred$ for $\red^*$, and $P\red$ if $\exists Q $ such that $ P \red Q$.
We write $P\red$ if $\exists Q $ such that $ P \red Q$ and $P\not\red$, otherwise.

\section{Replication}

As mentioned before, it is known that replication (and hence
recursion) can be implemented in a higher-order process algebra
\cite{SangiorgiWalker}. As our first example of calculation with the
machinery thus far presented we give the construction explicitly in
the {\rhoc}.

\begin{eqnarray}
	D_{x} & := & \prefix{x}{y}{(\binpar{\outputp{x}{y}}{@{y}})} \nonumber\\
	\bangp_{x}{P} & := & \binpar{{x}!\langle{\binpar{D_{x}}{P}}\rangle}{D_{x}} \nonumber
\end{eqnarray}

\begin{eqnarray}
	\bangp_{x}{P} & & \nonumber\\
	=
	& {x}!\langle{(\prefix{x}{y}{(\outputp{x}{y} | @{y})) | P}}\rangle 
	      | \prefix{x}{y}{(\outputp{x}{y} | @{y})} & \nonumber\\
	\red
	& (\outputp{x}{y} | @{y})\substn{\quotep{(\prefix{x}{y}{(@{y} | \outputp{x}{y})) | P}}}{y} & \nonumber\\
	=
	& \outputp{x}{\quotep{(\prefix{x}{y}{(\outputp{x}{y} | @{y})) | P}}}
	  | {(\prefix{x}{y}{(\outputp{x}{y} | @{y})) | P}} & \nonumber\\
	\red
	& \ldots & \nonumber\\
	\red^*
	& P | P | \ldots & \nonumber
\end{eqnarray}

Of course, this encoding, as an implementation, runs away, unfolding
$\bangp{P}$ eagerly. A lazier and more implementable replication
operator, restricted to input-guarded processes, may be obtained as follows.

\begin{eqnarray}
\bangp{\prefix{u}{v}{P}} 
	:= 
	\binpar{\lift{x}{\prefix{u}{v}{(\binpar{D(x)}{P})}}}{D(x)} \nonumber
\end{eqnarray}

\begin{remark}
  Note that the lazier definition still does not deal with summation
  or mixed summation (i.e. sums over input and output). The reader is
  invited to construct definitions of replication that deal with these
  features. 

  Further, the definitions are parameterized in a name, $x$. Can you,
  gentle reader, make a definition that eliminates this parameter and
  guarantees no accidental interaction between the replication
  machinery and the process being replicated -- i.e. no accidental
  sharing of names used by the process to get its work done and the
  name(s) used by the replication to effect copying. This latter
  revision of the definition of replication is crucial to obtaining
  the expected identity $!!P \sim !P$.
\end{remark}

\begin{remark}\label{rem:paradoxical_combinator}
  The reader familiar with the lambda calculus will have noticed the
  similarity between $D$ and the paradoxical combinator.

  [Ed. note: the existence of this seems to suggest we have to be more
  restrictive on the set of processes and names we admit if we are to
  support no-cloning.]
\end{remark}

\subsubsection{Bisimulation}

The computational dynamics gives rise to another kind of equivalence,
the equivalence of computational behavior. As previously mentioned
this is typically captured \emph{via} some form of bisimulation.

% The notion we use in this paper is weak barbed bisimulation
% \cite{milner91polyadicpi}.

The notion we use in this paper is derived from weak barbed
bisimulation \cite{milner91polyadicpi}. 

\begin{definition}
An \emph{observation relation}, $\downarrow_{\mathcal N}$, over a set
of names, $\mathcal N$, is the smallest relation satisfying the rules
below.

\infrule[Out-barb]{y \in {\mathcal N}, \; x \nameeq y}
		  {\outputp{x}{v} \downarrow_{\mathcal N} x}
\infrule[Par-barb]{\mbox{$P\downarrow_{\mathcal N} x$ or $Q\downarrow_{\mathcal N} x$}}
		  {\binpar{P}{Q} \downarrow_{\mathcal N} x}

We write $P \Downarrow_{\mathcal N} x$ if there is $Q$ such that 
$P \wred Q$ and $Q \downarrow_{\mathcal N} x$.
\end{definition}

\begin{definition}
%\label{def.bbisim}
An  ${\mathcal N}$-\emph{barbed bisimulation} over a set of names, ${\mathcal N}$, is a symmetric binary relation 
${\mathcal S}_{\mathcal N}$ between agents such that $P\rel{S}_{\mathcal N}Q$ implies:
\begin{enumerate}
\item If $P \red P'$ then $Q \wred Q'$ and $P'\rel{S}_{\mathcal N} Q'$.
\item If $P\downarrow_{\mathcal N} x$, then $Q\Downarrow_{\mathcal N} x$.
\end{enumerate}
$P$ is ${\mathcal N}$-barbed bisimilar to $Q$, written
$P \wbbisim_{\mathcal N} Q$, if $P \rel{S}_{\mathcal N} Q$ for some ${\mathcal N}$-barbed bisimulation ${\mathcal S}_{\mathcal N}$.
\end{definition}

$\mathcal{R} \subseteq \pi \times \pi$

$P \mathcal{R} Q => \forall P'. P \red P' \Rightarrow \exists Q'. Q \red Q', P' \mathcal{R} Q'$

$P \vdash x \Rightarrow Q \vdash x$

\begin{mathpar}
  \inferrule*[lab=Out-barb]{x \nameeq y}{{y}!\langle{Q}\rangle \vdash x}
  \and
  \inferrule*[lab=Par-barb]{\mbox{$P\vdash x$ or $Q\vdash x$}}{\binpar{P}{Q} \vdash x}
\end{mathpar}

\subsubsection{Contexts}

One of the principle advantages of computational calculi like the
$\pi$-calculus is a well-defined notion of context,
contextual-equivalence and a correlation between
contextual-equivalence and notions of bisimulation. The notion of
context allows the decomposition of a process into (sub-)process and
its syntactic environment, its context. Thus, a context may be
thought of as a process with a ``hole'' (written $\Box$) in it. The
application of a context $M$ to a process $P$, written $M[P]$, is
tantamount to filling the hole in $M$ with $P$. In this paper we do
not need the full weight of this theory, but do make use of the notion
of context in the proof the main theorem. 

\begin{mathpar}
  \inferrule* [lab=summation] {} {{M_{M},M_{N}} \bc \Box \;|\; x.M_{A} \;|\; M_{M}+M_{N}}
  \and
  \inferrule* [lab=agent] {} {{M_{A}} \bc (\vec{x})M_{P} \;| \; \clift{P_0,\ldots,M_{P},\ldots,P_N}}
  \and \\
  \inferrule* [lab=process] {} {{M_{P}} \bc M_{N} \;| \;P|M_{P} }
\end{mathpar} 

\begin{mathpar}
  \inferrule* [lab=sychronization] {} {M_{N} \bc \Box \;|\; x?M_{F} \;|\; x!M_{C}}
  \and
  \inferrule* [lab=abstraction] {} {{M_{F}} \bc (x)M_{P} }
  \and
  \inferrule* [lab=concretion] {} {{M_{C}} \bc \langle M_{P} \rangle }
  \and \\
  \inferrule* [lab=process] {} {{M_{P}} \bc M_{N} \;| \;P|M_{P} }
\end{mathpar}

\begin{definition}[contextual application] Given a context $M$, and
  process $P$, we define the \emph{contextual application}, $M[P] :=
  M\{P/\Box\}$. That is, the contextual application of M to P is the
  substitution of $P$ for $\Box$ in $M$.
\end{definition}

$\meaningof{-} : L \to \mathcal{P}(\pi)$

\begin{mathpar}
  \inferrule* [lab=collection] {} {\meaningof{true} = \pi, \and \meaningof{~E} = \pi \setminus \meaningof{E}, \and \meaningof{E_{1} \& E_{2}} = \meaningof{E_{1}} \cap \meaningof{E_{2}}}
\end{mathpar}

\begin{mathpar}
  \inferrule* [lab=structure] {} {\meaningof{0} = \{ P \in \pi | P \equiv 0 \}, \and \\ \meaningof{E_1 | E_2} = \{ P \in \pi | P \equiv P_{1} | P_{2}, P_{1} \in \meaningof{E_{1}}, P_{2} \in \meaningof{E_2}\} }
\end{mathpar}

\begin{mathpar}
 \inferrule* [lab=behavior] {} {\meaningof{\langle a?b \rangle E} = \{ P \in \pi | P \equiv Q | u?(y)P', \\ \and \\\\ \and \\ \;\;\; u \in \meaningof{a}, \forall z.P'\{z/y\} \in \meaningof{E\{z/b\}}\}, \and \\ \meaningof{a!E} = \{ P \in \pi | P \equiv Q | x!\langle P' \rangle, x \in \meaningof{a} P' \in \meaningof{E}\} }
\end{mathpar}

\begin{mathpar}
 \inferrule* [lab=nominal] {} {\meaningof{\quotep{E}} = \{ \quotep{P} \in \quotep{\pi} | P \in \meaningof{E} \}, \and \meaningof{\quotep{P}} = \{ \quotep{Q} \in \quotep{\pi} | P \equiv Q \} \and \\ \meaningof{@\quotep{E}} = \{ P \in \pi | P \equiv @x, x \in \meaningof{E} \}}
\end{mathpar}

\begin{eqnarray*}
  \\
  \meaningof{-} : TS \to ST
\end{eqnarray*}

\begin{eqnarray*}
  \\
  L : TS \to ST
\end{eqnarray*}

\begin{eqnarray*}
  \\
  P \models E \iff P \in \meaningof{E}
\end{eqnarray*}

\begin{eqnarray*}
  P \approx_{L} Q \iff \forall E \in L. P \models E \iff Q \models E
\end{eqnarray*}

\begin{eqnarray*}
  P \approx_{K} Q
\end{eqnarray*}

\begin{eqnarray*}
  P \approx Q
\end{eqnarray*}

$\approx_{K} = \approx = \approx_{L}$

\subsubsection{Contextual duality}

Note that contexts extend the quotation operation to a family of
operations from processes to names. Given a context, $M$, we can
define a \emph{nominal context}, $\quotep{M}$ by $\quotep{M}[P] :=
\quotep{M[P]}$. To foreshadow what is to come we observe that these
operations enjoy a duality with processes very much like the duality
between vectors and maps from vectors to scalars.

Further, because the calculus is essentially higher-order, we have a
correspondence between contexts and processes. More specifically,
given a name $x$ and a context $M$ we can construct $M^{*}_{x}$ such
that 

\begin{mathpar}
  M^{*}_{x} | \lift{x}{P} \red M[P]
\end{mathpar}

namely,

\begin{mathpar}
  M^{*}_{x} := x?(u).M[\dropn{u}]
\end{mathpar}

The dependence of $M^{*}_{x}$ on a name makes it an abstraction, 

\begin{mathpar}
  M^{*} := (x)x?(u).M[\dropn{u}]
\end{mathpar}

\subsection{Additional notation}

It will sometimes be convenient to denote the process a name
quotes. We already have the notation $x = \quotep{P}$, but it will be
convenient to introduce an alternate notation, $\procn{x}$, when we
want to emphasize the connection to the use of the name. Note that, by
virtue of name equivalence, $\quotep{\procn{x}} \nameeq x$; so, the
notation is consistent with previous definitions.

Further, because names have structure it is possible to effect
substitutions on the basis of that structure. This means we need to
upgrade our notation for substitutions, which we accomplish by
adapting comprehension notation. Thus,

\begin{mathpar}
  P\{ y / x : x \in S \}
\end{mathpar}

is interpreted to mean the process derived from P by replacing (in a
capture-avoiding manner) each occurrence of $x$ in $S$ by $y$. For example,

\begin{mathpar}
  P\{ \quotep{\procn{x}|\procn{x}} / x : x \in \freenames{P} \}
\end{mathpar}

will replace each (occurrence) of a free name $x$ in $P$ by
$\quotep{\procn{x}|\procn{x}}$.

Also, we will avail ourselves of the notation $x^{L}$ and $x^{R}$ to
denote injections of a name into disjoint copies of the name
space. There are numerous ways to accomplish this. One example can be
found in \cite{MeredithR05}. This notation overloads to vectors of
names: $\vec{x}^{\pi} := (x_{i}^{\pi} \; : \; 0 \leq i < |\vec{x}| )$ where $\pi \in \{L,R\}$.

We also use $P^{\Box} := P|\Box$.

In \cite{MeredithR05} an interpretation of the new operator is
given. It turns out that there are several possible interpretations
all enjoying the requisite algebraic properties of the operator (see
\cite{milner91polyadicpi}). We will therefore make liberal use of
$(\nu\; \vec{x})P$.

% subsection the_syntax_and_semantics_of_the_notation_system (end)   

\input{qm2pi.qmops} 

\input{qm2pi.sterngerlach} 

\input{qm2pi.metric} 

% section concurrent_process_calculi (end)

%\input{qm2pi.proofsketch}

% section proof sketch (end)

%\input{qm2pi.slviaknots} 

% section spatial logic via knots (end)

\input{qm2pi.conclusion}

% section conclusion (end)

%\input{qm2pi.dtcodes} 

% section wiring algorithm (end)

\input{qm2pi.ack} 

% section acknowledgments (end)

\newpage


\bibliographystyle{plain}   
\bibliography{../../biblios/main.bib}

\input{qm2pi.rhodetails}

\end{document}

 

\documentclass[12pt]{llncs}
%\documentclass{jktr}

\usepackage[pdftex]{hyperref}                   
\usepackage {listings}
\usepackage {mathpartir}
\usepackage{bcprules}
%\usepackage{listings}
                       
\usepackage{graphicx} 
%\usepackage[margins=2.5cm,nohead,nofoot]{geometry}
%\usepackage{geometry}
\usepackage{amsfonts}
\usepackage{amstext}
\usepackage{latexsym}
\usepackage{amssymb}
\usepackage{color}


%\include{myPreamble}
\include{qm2pi.local} 

%\ifpdf
%\usepackage[pdftex]{graphicx}
%\else
%\usepackage{graphicx}
%\fi

 % \ifpdf
%  \usepackage{pdfsync}
%  \if


%\title{Brief Article}
%\author{David F. Snyder}
%\author{L.G. Meredith}

%\address{Dept. of Math., Texas State University--San Marcos, San Marcos, TX 78666}
       
\pagestyle{empty}


\begin{document}

\lstset{language=[Objective]Caml,frame=shadowbox}

\input{qm2pi.front}

% section front matter (end)

\input{qm2pi.intro} 
 
% section introduction (end)

% \input{qm2pi.knotations} 

% section notation (end)

\input{qm2pi.process.calculi} 

% section concurrent_process_calculi_and_spatial_logics_ (end)
    
%\input{qm2pi.knots2pi} 

%\input{qm2pi.trefoil} 

%\input{qm2pi.mainthm} 

% subsection basic_interpretation (end)

%\input{qm2pi.rho.presentation} 
\subsection{The syntax and semantics of the notation system}\label{sub:the_syntax_and_semantics_of_the_notation_system} % (fold)

We now summarize a technical presentation of the calculus that
embodies our theory of dynamics. The typical presentation of such a
calculus follows the style of giving generators and relations on
them. The grammar, below, describing term constructors, freely
generates the set of processes, $\Proc$. This set is then quotiented
by a relation known as structural congruence and it is over this set
that the notion of dynamics is expressed. This presentation is
essentially that of \cite{MeredithR05} with the addition of
polyadicity and summation. For readability we have relegated some of
the technical subtleties to an appendix.

\subsubsection{Process grammar}\label{subsub:process_grammar}

\begin{mathpar}
  \inferrule* [lab=synchronization] {} {{M} \bc \pzero \;|\; x?F \;|\; x!C }
  \and
  \inferrule* [lab=abstraction] {} {{F} \bc (x)P}
  \and
  \inferrule* [lab=concretion] {} {{C} \bc \langle Q \rangle}
  \and
  \inferrule* [lab=process] {} {{P,Q} \bc M \;| \;P|Q \;|\; @{x}}
  \and
  \inferrule* [lab=name] {} {{x} \bc \quotep{P}}
\end{mathpar} 

Note that $\vec{x}$ (resp. $\vec{P}$) denotes a vector of names
(resp. processes) of length $|\vec{x}|$ (resp. $|\vec{P}|$). We adopt
the following useful abbreviations.

\begin{mathpar}
   x?(\vec{y}).P := x.(\vec{y})P \and  x\clift{\vec{P}} := x.\clift{\vec{P}}
   \and x!(y) := \lift{x}{\dropn{y}}
   \and \Pi_{i=0}^{n-1}P_i := P_0 | \ldots | P_{n-1}
\end{mathpar}

\subsubsection{Structural congruence}

\paragraph{Free and bound names and alpha-equivalence.} At the
core of structural equivalence is alpha-equivalence which identifies
process that are the same up to a change of variable. Formally, we
recognize the distinction between free and bound names. The free names
of a process, $\freenames{P}$, may be calculated recursively as
follows:

\begin{mathpar}
\freenames{\pzero} := \emptyset
  \and \\
  \freenames{x?(y).P} := \{ x \} \cup (\freenames{P} \setminus \{ y \})
  \and 
  \freenames{x!\langle P \rangle} := \{ x \} \cup \{ P \} 
  \and \\
  \freenames{P|Q} := \freenames{P} \cup \freenames{Q}
  \and \\
  \freenames{@{x}} := \{ x \}
\end{mathpar}

$\pi$
$\quotep{\pi}$

$\freenames{-} : \pi \to \mathcal{P}(\quotep{\pi})$

\begin{eqnarray*}
  \freenames{\pzero} & := & \emptyset \\
  \freenames{x?(y).P} & := & \{ x \} \cup (\freenames{P} \setminus \{ y \}) \\
  \freenames{x!\langle P \rangle} & := & \{ x \} \cup \{ P \} \\
  \freenames{P|Q} & := & \freenames{P} \cup \freenames{Q} \\
  \freenames{\dropn{x}} & := & \{ x \}
\end{eqnarray*}

The bound names of a process, $\boundnames{P}$, are those names occurring in $P$
that are not free. For example, in $x?(y).0$, the name $x$ is free, while $y$ is bound.

\begin{mathpar}
  \inferrule* [lab=monoidal-laws] {} { P|Q \equiv Q|P \and P|0 \equiv P \and P|(Q|R) \equiv (P|Q)|R }
\end{mathpar}

\begin{mathpar}
  \inferrule* [lab=alpha-equivalence] {} { (x)P \equiv (y)P\{y/x\} \and y \not\in \freenames{P} }
\end{mathpar}

\begin{definition}
Then two processes, $P,Q$, are alpha-equivalent if $P = Q\{\vec{y}/\vec{x}\}$ for
some $\vec{x} \in \boundnames{Q},\vec{y} \in \boundnames{P}$, where $Q\{\vec{y}/\vec{x}\}$
denotes the capture-avoiding substitution of $\vec{y}$ for $\vec{x}$ in $Q$.
\end{definition}

\begin{definition}
  The {\em structural congruence} \cite{SangiorgiWalker} , $\equiv$,
  between processes is the least congruence containing
  alpha-equivalence, satisfying the abelian monoid laws
  (associativity, commutativity and $\pzero$ as identity) for parallel
  composition $|$ and for summation $+$.
\end{definition}

\subsection{Name equivalence}

We take name equivalence, written $\nameeq$, to be the smallest
equivalence relation generated by the following rules.

\begin{mathpar}
\inferrule*[lab=Quote-drop]
{ }
{ \quotep{@{x}} \nameeq x }

\inferrule*[lab=Struct-equiv]
{ P \scong Q }
{ \quotep{P} \nameeq \quotep{Q} }
\end{mathpar}

The astute reader will have noticed that the mutual recursion of names
and processes imposes a mutual recursion on alpha-equivalence and
structural equivalence via name-equivalence. Fortunately, all of this
works out pleasantly and we may calculate in the natural way, free of
concern. The reader interested in the details is referred to the
appendix \ref{appendix:rho_details}.

\subsection{Substitution}

We use $\Proc$ for the set of processes, $\QProc$ for the set of
names, and $\id{\{}\vec{y} / \vec{x} \id{\}}$ to denote partial maps,
$s : \QProc \rightarrow \QProc$. A map, $s$ lifts, uniquely, to a map
on process terms, $\widehat{s} : \Proc \rightarrow \Proc$ by the
following equations.

\begin{mathpar}
  (0) \psubstp{Q}{P} := 0 \\
  (R \juxtap S) \psubstp{Q}{P}
  :=    
  (R)\psubstp{Q}{P} \juxtap (S) \psubstp{Q}{P} \\
  (x?(y).R) \psubstp{Q}{P}    
  :=    
  (x)\substp{Q}{P} (z)\concat( (R \psubstn{z}{y}) \psubstp{Q}{P} ) \\
  (\lift{x}{R}) \psubstp{Q}{P}  
  :=
  \lift{(x)\substp{Q}{P}}{ R \psubstp{Q}{P} } \\
%   (\dropn{x})  \psubstp{Q}{P}       
%   := 
%   \left\{ 
%     \begin{array}{ccc} 
%       \dropn{\quotep{Q}} & & x \nameeq \quotep{P} \\
%       \dropn{x} & & otherwise \\
%     \end{array}
%   \right. 
  (\dropn{x})  \psubstp{Q}{P}       
  := 
  \left\{ 
    \begin{array}{ccc} 
      Q & & x \nameeq \quotep{P} \\
      \dropn{x} & & otherwise \\
    \end{array}
  \right.
\end{mathpar}
 

where

\begin{eqnarray}
  (x)\id{\{} \lpquote Q \rpquote / \lpquote P \rpquote \id{\}}            = 
  \left\{ 
    \begin{array}{ccc}
      \lpquote Q \rpquote & & x \nameeq \lpquote P \rpquote \\
      x & & otherwise \\
    \end{array}
  \right. \nonumber
\end{eqnarray}

and $z$ is chosen distinct from $\quotep{P}$, $\quotep{Q}$, the free
names in $Q$, and all the names in $R$. Our $\alpha$-equivalence will
be built in the standard way from this substitution.

\begin{remark}\label{rem:no_self_referential_names}
  One consequence of these definitions is that $\forall P. \quotep{P}
  \not\in \freenames{P}$.
\end{remark}

\subsection{ Dynamic quote: an example }

Anticipating something of what's to come, consider applying the
substitution, $\widehat{\id{\{}u / z \id{\}}}$, to the following pair
of processes, $\lift{w}{y!(z)}$ and $w[ \lpquote y!(z) \rpquote ]$.

\begin{eqnarray}
	\lift{w}{y!(z)}\widehat{\id{\{}u / z \id{\}}}
		& = &
		\lift{w}{y!(u)} \nonumber\\
	w[ \lpquote y!(z) \rpquote ] \widehat{ \id{\{}u / z \id{\}} }
		& = &
		w[ \lpquote y!(z) \rpquote ] \nonumber
\end{eqnarray}

Because the body of the process between quotes is impervious to
substitution, we get radically different answers. In fact, by
examining the first process in an input context,
e.g. $x?(z).\lift{w}{y!(z)}$, we see that the process under the lift
operator may be shaped by prefixed inputs binding a name inside it. In
this sense, the lift operator will be seen as a way to dynamically
construct processes before reifying them as names.

Finally equipped with these standard features we can present the
dynamics of the calculus.

\subsubsection{Operational semantics} 

Finally, we introduce the computational dynamics. What marks these
algebras as distinct from other more traditionally studied algebraic
structures, e.g. vector spaces or polynomial rings, is the manner in
which dynamics is captured. In traditional structures, dynamics is typically
expressed through morphisms between such structures, as in linear maps
between vector spaces or morphisms between rings. In algebras
associated with the semantics of computation, the dynamics is
expressed as part of the algebraic structure itself, through a
reduction reduction relation typically denoted by $\red$. Below, we
give a recursive presentation of this relation for the calculus used
in the encoding.

$\red \subseteq \pi \times \pi$
$\red : \pi \to \mathcal{P}(\pi)$

\begin{mathpar}
  \inferrule* [lab=Comm] { \textsf{match}( x_{src}, x_{trgt} ) } { x_{trgt}?(y)P \; | \; x_{src}!\langle {Q} \rangle \red P\{\quotep{Q}/y}\} }
  \and \\
  \inferrule* [lab=Par] {{P} \red {P}'} {{{P} | {Q}} \red {{P}' | {Q}}}
  \and
  \inferrule* [lab=Equiv]{{{P} \scong {P}'} \andalso {{P}' \red {Q}'} \andalso {{Q}' \scong {Q}}}{{P} \red {Q}}
\end{mathpar}

\begin{eqnarray*}
  match_{\equiv} (\quotep{P},\quotep{Q}) & := & P \equiv Q \\
  match_{\dagger}(\quotep{P},\quotep{Q}) & := & \forall R. P|Q \red^{*} R => R \red^{*} 0 \\
  match_{K}(\quotep{P},\quotep{Q}) & := & K \mbox{ for some context } K
\end{eqnarray*}

$u?(x)P | u!\langle Q \rangle \red P\{\quotep{Q}/x\}$

%We write $\wred$ for $\red^*$, and $P\red$ if $\exists Q $ such that $ P \red Q$.
We write $P\red$ if $\exists Q $ such that $ P \red Q$ and $P\not\red$, otherwise.

\section{Replication}

As mentioned before, it is known that replication (and hence
recursion) can be implemented in a higher-order process algebra
\cite{SangiorgiWalker}. As our first example of calculation with the
machinery thus far presented we give the construction explicitly in
the {\rhoc}.

\begin{eqnarray}
	D_{x} & := & \prefix{x}{y}{(\binpar{\outputp{x}{y}}{@{y}})} \nonumber\\
	\bangp_{x}{P} & := & \binpar{{x}!\langle{\binpar{D_{x}}{P}}\rangle}{D_{x}} \nonumber
\end{eqnarray}

\begin{eqnarray}
	\bangp_{x}{P} & & \nonumber\\
	=
	& {x}!\langle{(\prefix{x}{y}{(\outputp{x}{y} | @{y})) | P}}\rangle 
	      | \prefix{x}{y}{(\outputp{x}{y} | @{y})} & \nonumber\\
	\red
	& (\outputp{x}{y} | @{y})\substn{\quotep{(\prefix{x}{y}{(@{y} | \outputp{x}{y})) | P}}}{y} & \nonumber\\
	=
	& \outputp{x}{\quotep{(\prefix{x}{y}{(\outputp{x}{y} | @{y})) | P}}}
	  | {(\prefix{x}{y}{(\outputp{x}{y} | @{y})) | P}} & \nonumber\\
	\red
	& \ldots & \nonumber\\
	\red^*
	& P | P | \ldots & \nonumber
\end{eqnarray}

Of course, this encoding, as an implementation, runs away, unfolding
$\bangp{P}$ eagerly. A lazier and more implementable replication
operator, restricted to input-guarded processes, may be obtained as follows.

\begin{eqnarray}
\bangp{\prefix{u}{v}{P}} 
	:= 
	\binpar{\lift{x}{\prefix{u}{v}{(\binpar{D(x)}{P})}}}{D(x)} \nonumber
\end{eqnarray}

\begin{remark}
  Note that the lazier definition still does not deal with summation
  or mixed summation (i.e. sums over input and output). The reader is
  invited to construct definitions of replication that deal with these
  features. 

  Further, the definitions are parameterized in a name, $x$. Can you,
  gentle reader, make a definition that eliminates this parameter and
  guarantees no accidental interaction between the replication
  machinery and the process being replicated -- i.e. no accidental
  sharing of names used by the process to get its work done and the
  name(s) used by the replication to effect copying. This latter
  revision of the definition of replication is crucial to obtaining
  the expected identity $!!P \sim !P$.
\end{remark}

\begin{remark}\label{rem:paradoxical_combinator}
  The reader familiar with the lambda calculus will have noticed the
  similarity between $D$ and the paradoxical combinator.

  [Ed. note: the existence of this seems to suggest we have to be more
  restrictive on the set of processes and names we admit if we are to
  support no-cloning.]
\end{remark}

\subsubsection{Bisimulation}

The computational dynamics gives rise to another kind of equivalence,
the equivalence of computational behavior. As previously mentioned
this is typically captured \emph{via} some form of bisimulation.

% The notion we use in this paper is weak barbed bisimulation
% \cite{milner91polyadicpi}.

The notion we use in this paper is derived from weak barbed
bisimulation \cite{milner91polyadicpi}. 

\begin{definition}
An \emph{observation relation}, $\downarrow_{\mathcal N}$, over a set
of names, $\mathcal N$, is the smallest relation satisfying the rules
below.

\infrule[Out-barb]{y \in {\mathcal N}, \; x \nameeq y}
		  {\outputp{x}{v} \downarrow_{\mathcal N} x}
\infrule[Par-barb]{\mbox{$P\downarrow_{\mathcal N} x$ or $Q\downarrow_{\mathcal N} x$}}
		  {\binpar{P}{Q} \downarrow_{\mathcal N} x}

We write $P \Downarrow_{\mathcal N} x$ if there is $Q$ such that 
$P \wred Q$ and $Q \downarrow_{\mathcal N} x$.
\end{definition}

\begin{definition}
%\label{def.bbisim}
An  ${\mathcal N}$-\emph{barbed bisimulation} over a set of names, ${\mathcal N}$, is a symmetric binary relation 
${\mathcal S}_{\mathcal N}$ between agents such that $P\rel{S}_{\mathcal N}Q$ implies:
\begin{enumerate}
\item If $P \red P'$ then $Q \wred Q'$ and $P'\rel{S}_{\mathcal N} Q'$.
\item If $P\downarrow_{\mathcal N} x$, then $Q\Downarrow_{\mathcal N} x$.
\end{enumerate}
$P$ is ${\mathcal N}$-barbed bisimilar to $Q$, written
$P \wbbisim_{\mathcal N} Q$, if $P \rel{S}_{\mathcal N} Q$ for some ${\mathcal N}$-barbed bisimulation ${\mathcal S}_{\mathcal N}$.
\end{definition}

$\mathcal{R} \subseteq \pi \times \pi$

$P \mathcal{R} Q => \forall P'. P \red P' \Rightarrow \exists Q'. Q \red Q', P' \mathcal{R} Q'$

$P \vdash x \Rightarrow Q \vdash x$

\begin{mathpar}
  \inferrule*[lab=Out-barb]{x \nameeq y}{{y}!\langle{Q}\rangle \vdash x}
  \and
  \inferrule*[lab=Par-barb]{\mbox{$P\vdash x$ or $Q\vdash x$}}{\binpar{P}{Q} \vdash x}
\end{mathpar}

\subsubsection{Contexts}

One of the principle advantages of computational calculi like the
$\pi$-calculus is a well-defined notion of context,
contextual-equivalence and a correlation between
contextual-equivalence and notions of bisimulation. The notion of
context allows the decomposition of a process into (sub-)process and
its syntactic environment, its context. Thus, a context may be
thought of as a process with a ``hole'' (written $\Box$) in it. The
application of a context $M$ to a process $P$, written $M[P]$, is
tantamount to filling the hole in $M$ with $P$. In this paper we do
not need the full weight of this theory, but do make use of the notion
of context in the proof the main theorem. 

\begin{mathpar}
  \inferrule* [lab=summation] {} {{M_{M},M_{N}} \bc \Box \;|\; x.M_{A} \;|\; M_{M}+M_{N}}
  \and
  \inferrule* [lab=agent] {} {{M_{A}} \bc (\vec{x})M_{P} \;| \; \clift{P_0,\ldots,M_{P},\ldots,P_N}}
  \and \\
  \inferrule* [lab=process] {} {{M_{P}} \bc M_{N} \;| \;P|M_{P} }
\end{mathpar} 

\begin{mathpar}
  \inferrule* [lab=sychronization] {} {M_{N} \bc \Box \;|\; x?M_{F} \;|\; x!M_{C}}
  \and
  \inferrule* [lab=abstraction] {} {{M_{F}} \bc (x)M_{P} }
  \and
  \inferrule* [lab=concretion] {} {{M_{C}} \bc \langle M_{P} \rangle }
  \and \\
  \inferrule* [lab=process] {} {{M_{P}} \bc M_{N} \;| \;P|M_{P} }
\end{mathpar}

\begin{definition}[contextual application] Given a context $M$, and
  process $P$, we define the \emph{contextual application}, $M[P] :=
  M\{P/\Box\}$. That is, the contextual application of M to P is the
  substitution of $P$ for $\Box$ in $M$.
\end{definition}

$\meaningof{-} : L \to \mathcal{P}(\pi)$

\begin{mathpar}
  \inferrule* [lab=collection] {} {\meaningof{true} = \pi, \and \meaningof{~E} = \pi \setminus \meaningof{E}, \and \meaningof{E_{1} \& E_{2}} = \meaningof{E_{1}} \cap \meaningof{E_{2}}}
\end{mathpar}

\begin{mathpar}
  \inferrule* [lab=structure] {} {\meaningof{0} = \{ P \in \pi | P \equiv 0 \}, \and \\ \meaningof{E_1 | E_2} = \{ P \in \pi | P \equiv P_{1} | P_{2}, P_{1} \in \meaningof{E_{1}}, P_{2} \in \meaningof{E_2}\} }
\end{mathpar}

\begin{mathpar}
 \inferrule* [lab=behavior] {} {\meaningof{\langle a?b \rangle E} = \{ P \in \pi | P \equiv Q | u?(y)P', \\ \and \\\\ \and \\ \;\;\; u \in \meaningof{a}, \forall z.P'\{z/y\} \in \meaningof{E\{z/b\}}\}, \and \\ \meaningof{a!E} = \{ P \in \pi | P \equiv Q | x!\langle P' \rangle, x \in \meaningof{a} P' \in \meaningof{E}\} }
\end{mathpar}

\begin{mathpar}
 \inferrule* [lab=nominal] {} {\meaningof{\quotep{E}} = \{ \quotep{P} \in \quotep{\pi} | P \in \meaningof{E} \}, \and \meaningof{\quotep{P}} = \{ \quotep{Q} \in \quotep{\pi} | P \equiv Q \} \and \\ \meaningof{@\quotep{E}} = \{ P \in \pi | P \equiv @x, x \in \meaningof{E} \}}
\end{mathpar}

\begin{eqnarray*}
  \\
  \meaningof{-} : TS \to ST
\end{eqnarray*}

\begin{eqnarray*}
  \\
  L : TS \to ST
\end{eqnarray*}

\begin{eqnarray*}
  \\
  P \models E \iff P \in \meaningof{E}
\end{eqnarray*}

\begin{eqnarray*}
  P \approx_{L} Q \iff \forall E \in L. P \models E \iff Q \models E
\end{eqnarray*}

\begin{eqnarray*}
  P \approx_{K} Q
\end{eqnarray*}

\begin{eqnarray*}
  P \approx Q
\end{eqnarray*}

$\approx_{K} = \approx = \approx_{L}$

\subsubsection{Contextual duality}

Note that contexts extend the quotation operation to a family of
operations from processes to names. Given a context, $M$, we can
define a \emph{nominal context}, $\quotep{M}$ by $\quotep{M}[P] :=
\quotep{M[P]}$. To foreshadow what is to come we observe that these
operations enjoy a duality with processes very much like the duality
between vectors and maps from vectors to scalars.

Further, because the calculus is essentially higher-order, we have a
correspondence between contexts and processes. More specifically,
given a name $x$ and a context $M$ we can construct $M^{*}_{x}$ such
that 

\begin{mathpar}
  M^{*}_{x} | \lift{x}{P} \red M[P]
\end{mathpar}

namely,

\begin{mathpar}
  M^{*}_{x} := x?(u).M[\dropn{u}]
\end{mathpar}

The dependence of $M^{*}_{x}$ on a name makes it an abstraction, 

\begin{mathpar}
  M^{*} := (x)x?(u).M[\dropn{u}]
\end{mathpar}

\subsection{Additional notation}

It will sometimes be convenient to denote the process a name
quotes. We already have the notation $x = \quotep{P}$, but it will be
convenient to introduce an alternate notation, $\procn{x}$, when we
want to emphasize the connection to the use of the name. Note that, by
virtue of name equivalence, $\quotep{\procn{x}} \nameeq x$; so, the
notation is consistent with previous definitions.

Further, because names have structure it is possible to effect
substitutions on the basis of that structure. This means we need to
upgrade our notation for substitutions, which we accomplish by
adapting comprehension notation. Thus,

\begin{mathpar}
  P\{ y / x : x \in S \}
\end{mathpar}

is interpreted to mean the process derived from P by replacing (in a
capture-avoiding manner) each occurrence of $x$ in $S$ by $y$. For example,

\begin{mathpar}
  P\{ \quotep{\procn{x}|\procn{x}} / x : x \in \freenames{P} \}
\end{mathpar}

will replace each (occurrence) of a free name $x$ in $P$ by
$\quotep{\procn{x}|\procn{x}}$.

Also, we will avail ourselves of the notation $x^{L}$ and $x^{R}$ to
denote injections of a name into disjoint copies of the name
space. There are numerous ways to accomplish this. One example can be
found in \cite{MeredithR05}. This notation overloads to vectors of
names: $\vec{x}^{\pi} := (x_{i}^{\pi} \; : \; 0 \leq i < |\vec{x}| )$ where $\pi \in \{L,R\}$.

We also use $P^{\Box} := P|\Box$.

In \cite{MeredithR05} an interpretation of the new operator is
given. It turns out that there are several possible interpretations
all enjoying the requisite algebraic properties of the operator (see
\cite{milner91polyadicpi}). We will therefore make liberal use of
$(\nu\; \vec{x})P$.

% subsection the_syntax_and_semantics_of_the_notation_system (end)   

\input{qm2pi.qmops} 

\input{qm2pi.sterngerlach} 

\input{qm2pi.metric} 

% section concurrent_process_calculi (end)

%\input{qm2pi.proofsketch}

% section proof sketch (end)

%\input{qm2pi.slviaknots} 

% section spatial logic via knots (end)

\input{qm2pi.conclusion}

% section conclusion (end)

%\input{qm2pi.dtcodes} 

% section wiring algorithm (end)

\input{qm2pi.ack} 

% section acknowledgments (end)

\newpage


\bibliographystyle{plain}   
\bibliography{../../biblios/main.bib}

\input{qm2pi.rhodetails}

\end{document}

 

% section concurrent_process_calculi (end)

%\documentclass[12pt]{llncs}
%\documentclass{jktr}

\usepackage[pdftex]{hyperref}                   
\usepackage {listings}
\usepackage {mathpartir}
\usepackage{bcprules}
%\usepackage{listings}
                       
\usepackage{graphicx} 
%\usepackage[margins=2.5cm,nohead,nofoot]{geometry}
%\usepackage{geometry}
\usepackage{amsfonts}
\usepackage{amstext}
\usepackage{latexsym}
\usepackage{amssymb}
\usepackage{color}


%\include{myPreamble}
\include{qm2pi.local} 

%\ifpdf
%\usepackage[pdftex]{graphicx}
%\else
%\usepackage{graphicx}
%\fi

 % \ifpdf
%  \usepackage{pdfsync}
%  \if


%\title{Brief Article}
%\author{David F. Snyder}
%\author{L.G. Meredith}

%\address{Dept. of Math., Texas State University--San Marcos, San Marcos, TX 78666}
       
\pagestyle{empty}


\begin{document}

\lstset{language=[Objective]Caml,frame=shadowbox}

\input{qm2pi.front}

% section front matter (end)

\input{qm2pi.intro} 
 
% section introduction (end)

% \input{qm2pi.knotations} 

% section notation (end)

\input{qm2pi.process.calculi} 

% section concurrent_process_calculi_and_spatial_logics_ (end)
    
%\input{qm2pi.knots2pi} 

%\input{qm2pi.trefoil} 

%\input{qm2pi.mainthm} 

% subsection basic_interpretation (end)

%\input{qm2pi.rho.presentation} 
\subsection{The syntax and semantics of the notation system}\label{sub:the_syntax_and_semantics_of_the_notation_system} % (fold)

We now summarize a technical presentation of the calculus that
embodies our theory of dynamics. The typical presentation of such a
calculus follows the style of giving generators and relations on
them. The grammar, below, describing term constructors, freely
generates the set of processes, $\Proc$. This set is then quotiented
by a relation known as structural congruence and it is over this set
that the notion of dynamics is expressed. This presentation is
essentially that of \cite{MeredithR05} with the addition of
polyadicity and summation. For readability we have relegated some of
the technical subtleties to an appendix.

\subsubsection{Process grammar}\label{subsub:process_grammar}

\begin{mathpar}
  \inferrule* [lab=synchronization] {} {{M} \bc \pzero \;|\; x?F \;|\; x!C }
  \and
  \inferrule* [lab=abstraction] {} {{F} \bc (x)P}
  \and
  \inferrule* [lab=concretion] {} {{C} \bc \langle Q \rangle}
  \and
  \inferrule* [lab=process] {} {{P,Q} \bc M \;| \;P|Q \;|\; @{x}}
  \and
  \inferrule* [lab=name] {} {{x} \bc \quotep{P}}
\end{mathpar} 

Note that $\vec{x}$ (resp. $\vec{P}$) denotes a vector of names
(resp. processes) of length $|\vec{x}|$ (resp. $|\vec{P}|$). We adopt
the following useful abbreviations.

\begin{mathpar}
   x?(\vec{y}).P := x.(\vec{y})P \and  x\clift{\vec{P}} := x.\clift{\vec{P}}
   \and x!(y) := \lift{x}{\dropn{y}}
   \and \Pi_{i=0}^{n-1}P_i := P_0 | \ldots | P_{n-1}
\end{mathpar}

\subsubsection{Structural congruence}

\paragraph{Free and bound names and alpha-equivalence.} At the
core of structural equivalence is alpha-equivalence which identifies
process that are the same up to a change of variable. Formally, we
recognize the distinction between free and bound names. The free names
of a process, $\freenames{P}$, may be calculated recursively as
follows:

\begin{mathpar}
\freenames{\pzero} := \emptyset
  \and \\
  \freenames{x?(y).P} := \{ x \} \cup (\freenames{P} \setminus \{ y \})
  \and 
  \freenames{x!\langle P \rangle} := \{ x \} \cup \{ P \} 
  \and \\
  \freenames{P|Q} := \freenames{P} \cup \freenames{Q}
  \and \\
  \freenames{@{x}} := \{ x \}
\end{mathpar}

$\pi$
$\quotep{\pi}$

$\freenames{-} : \pi \to \mathcal{P}(\quotep{\pi})$

\begin{eqnarray*}
  \freenames{\pzero} & := & \emptyset \\
  \freenames{x?(y).P} & := & \{ x \} \cup (\freenames{P} \setminus \{ y \}) \\
  \freenames{x!\langle P \rangle} & := & \{ x \} \cup \{ P \} \\
  \freenames{P|Q} & := & \freenames{P} \cup \freenames{Q} \\
  \freenames{\dropn{x}} & := & \{ x \}
\end{eqnarray*}

The bound names of a process, $\boundnames{P}$, are those names occurring in $P$
that are not free. For example, in $x?(y).0$, the name $x$ is free, while $y$ is bound.

\begin{mathpar}
  \inferrule* [lab=monoidal-laws] {} { P|Q \equiv Q|P \and P|0 \equiv P \and P|(Q|R) \equiv (P|Q)|R }
\end{mathpar}

\begin{mathpar}
  \inferrule* [lab=alpha-equivalence] {} { (x)P \equiv (y)P\{y/x\} \and y \not\in \freenames{P} }
\end{mathpar}

\begin{definition}
Then two processes, $P,Q$, are alpha-equivalent if $P = Q\{\vec{y}/\vec{x}\}$ for
some $\vec{x} \in \boundnames{Q},\vec{y} \in \boundnames{P}$, where $Q\{\vec{y}/\vec{x}\}$
denotes the capture-avoiding substitution of $\vec{y}$ for $\vec{x}$ in $Q$.
\end{definition}

\begin{definition}
  The {\em structural congruence} \cite{SangiorgiWalker} , $\equiv$,
  between processes is the least congruence containing
  alpha-equivalence, satisfying the abelian monoid laws
  (associativity, commutativity and $\pzero$ as identity) for parallel
  composition $|$ and for summation $+$.
\end{definition}

\subsection{Name equivalence}

We take name equivalence, written $\nameeq$, to be the smallest
equivalence relation generated by the following rules.

\begin{mathpar}
\inferrule*[lab=Quote-drop]
{ }
{ \quotep{@{x}} \nameeq x }

\inferrule*[lab=Struct-equiv]
{ P \scong Q }
{ \quotep{P} \nameeq \quotep{Q} }
\end{mathpar}

The astute reader will have noticed that the mutual recursion of names
and processes imposes a mutual recursion on alpha-equivalence and
structural equivalence via name-equivalence. Fortunately, all of this
works out pleasantly and we may calculate in the natural way, free of
concern. The reader interested in the details is referred to the
appendix \ref{appendix:rho_details}.

\subsection{Substitution}

We use $\Proc$ for the set of processes, $\QProc$ for the set of
names, and $\id{\{}\vec{y} / \vec{x} \id{\}}$ to denote partial maps,
$s : \QProc \rightarrow \QProc$. A map, $s$ lifts, uniquely, to a map
on process terms, $\widehat{s} : \Proc \rightarrow \Proc$ by the
following equations.

\begin{mathpar}
  (0) \psubstp{Q}{P} := 0 \\
  (R \juxtap S) \psubstp{Q}{P}
  :=    
  (R)\psubstp{Q}{P} \juxtap (S) \psubstp{Q}{P} \\
  (x?(y).R) \psubstp{Q}{P}    
  :=    
  (x)\substp{Q}{P} (z)\concat( (R \psubstn{z}{y}) \psubstp{Q}{P} ) \\
  (\lift{x}{R}) \psubstp{Q}{P}  
  :=
  \lift{(x)\substp{Q}{P}}{ R \psubstp{Q}{P} } \\
%   (\dropn{x})  \psubstp{Q}{P}       
%   := 
%   \left\{ 
%     \begin{array}{ccc} 
%       \dropn{\quotep{Q}} & & x \nameeq \quotep{P} \\
%       \dropn{x} & & otherwise \\
%     \end{array}
%   \right. 
  (\dropn{x})  \psubstp{Q}{P}       
  := 
  \left\{ 
    \begin{array}{ccc} 
      Q & & x \nameeq \quotep{P} \\
      \dropn{x} & & otherwise \\
    \end{array}
  \right.
\end{mathpar}
 

where

\begin{eqnarray}
  (x)\id{\{} \lpquote Q \rpquote / \lpquote P \rpquote \id{\}}            = 
  \left\{ 
    \begin{array}{ccc}
      \lpquote Q \rpquote & & x \nameeq \lpquote P \rpquote \\
      x & & otherwise \\
    \end{array}
  \right. \nonumber
\end{eqnarray}

and $z$ is chosen distinct from $\quotep{P}$, $\quotep{Q}$, the free
names in $Q$, and all the names in $R$. Our $\alpha$-equivalence will
be built in the standard way from this substitution.

\begin{remark}\label{rem:no_self_referential_names}
  One consequence of these definitions is that $\forall P. \quotep{P}
  \not\in \freenames{P}$.
\end{remark}

\subsection{ Dynamic quote: an example }

Anticipating something of what's to come, consider applying the
substitution, $\widehat{\id{\{}u / z \id{\}}}$, to the following pair
of processes, $\lift{w}{y!(z)}$ and $w[ \lpquote y!(z) \rpquote ]$.

\begin{eqnarray}
	\lift{w}{y!(z)}\widehat{\id{\{}u / z \id{\}}}
		& = &
		\lift{w}{y!(u)} \nonumber\\
	w[ \lpquote y!(z) \rpquote ] \widehat{ \id{\{}u / z \id{\}} }
		& = &
		w[ \lpquote y!(z) \rpquote ] \nonumber
\end{eqnarray}

Because the body of the process between quotes is impervious to
substitution, we get radically different answers. In fact, by
examining the first process in an input context,
e.g. $x?(z).\lift{w}{y!(z)}$, we see that the process under the lift
operator may be shaped by prefixed inputs binding a name inside it. In
this sense, the lift operator will be seen as a way to dynamically
construct processes before reifying them as names.

Finally equipped with these standard features we can present the
dynamics of the calculus.

\subsubsection{Operational semantics} 

Finally, we introduce the computational dynamics. What marks these
algebras as distinct from other more traditionally studied algebraic
structures, e.g. vector spaces or polynomial rings, is the manner in
which dynamics is captured. In traditional structures, dynamics is typically
expressed through morphisms between such structures, as in linear maps
between vector spaces or morphisms between rings. In algebras
associated with the semantics of computation, the dynamics is
expressed as part of the algebraic structure itself, through a
reduction reduction relation typically denoted by $\red$. Below, we
give a recursive presentation of this relation for the calculus used
in the encoding.

$\red \subseteq \pi \times \pi$
$\red : \pi \to \mathcal{P}(\pi)$

\begin{mathpar}
  \inferrule* [lab=Comm] { \textsf{match}( x_{src}, x_{trgt} ) } { x_{trgt}?(y)P \; | \; x_{src}!\langle {Q} \rangle \red P\{\quotep{Q}/y}\} }
  \and \\
  \inferrule* [lab=Par] {{P} \red {P}'} {{{P} | {Q}} \red {{P}' | {Q}}}
  \and
  \inferrule* [lab=Equiv]{{{P} \scong {P}'} \andalso {{P}' \red {Q}'} \andalso {{Q}' \scong {Q}}}{{P} \red {Q}}
\end{mathpar}

\begin{eqnarray*}
  match_{\equiv} (\quotep{P},\quotep{Q}) & := & P \equiv Q \\
  match_{\dagger}(\quotep{P},\quotep{Q}) & := & \forall R. P|Q \red^{*} R => R \red^{*} 0 \\
  match_{K}(\quotep{P},\quotep{Q}) & := & K \mbox{ for some context } K
\end{eqnarray*}

$u?(x)P | u!\langle Q \rangle \red P\{\quotep{Q}/x\}$

%We write $\wred$ for $\red^*$, and $P\red$ if $\exists Q $ such that $ P \red Q$.
We write $P\red$ if $\exists Q $ such that $ P \red Q$ and $P\not\red$, otherwise.

\section{Replication}

As mentioned before, it is known that replication (and hence
recursion) can be implemented in a higher-order process algebra
\cite{SangiorgiWalker}. As our first example of calculation with the
machinery thus far presented we give the construction explicitly in
the {\rhoc}.

\begin{eqnarray}
	D_{x} & := & \prefix{x}{y}{(\binpar{\outputp{x}{y}}{@{y}})} \nonumber\\
	\bangp_{x}{P} & := & \binpar{{x}!\langle{\binpar{D_{x}}{P}}\rangle}{D_{x}} \nonumber
\end{eqnarray}

\begin{eqnarray}
	\bangp_{x}{P} & & \nonumber\\
	=
	& {x}!\langle{(\prefix{x}{y}{(\outputp{x}{y} | @{y})) | P}}\rangle 
	      | \prefix{x}{y}{(\outputp{x}{y} | @{y})} & \nonumber\\
	\red
	& (\outputp{x}{y} | @{y})\substn{\quotep{(\prefix{x}{y}{(@{y} | \outputp{x}{y})) | P}}}{y} & \nonumber\\
	=
	& \outputp{x}{\quotep{(\prefix{x}{y}{(\outputp{x}{y} | @{y})) | P}}}
	  | {(\prefix{x}{y}{(\outputp{x}{y} | @{y})) | P}} & \nonumber\\
	\red
	& \ldots & \nonumber\\
	\red^*
	& P | P | \ldots & \nonumber
\end{eqnarray}

Of course, this encoding, as an implementation, runs away, unfolding
$\bangp{P}$ eagerly. A lazier and more implementable replication
operator, restricted to input-guarded processes, may be obtained as follows.

\begin{eqnarray}
\bangp{\prefix{u}{v}{P}} 
	:= 
	\binpar{\lift{x}{\prefix{u}{v}{(\binpar{D(x)}{P})}}}{D(x)} \nonumber
\end{eqnarray}

\begin{remark}
  Note that the lazier definition still does not deal with summation
  or mixed summation (i.e. sums over input and output). The reader is
  invited to construct definitions of replication that deal with these
  features. 

  Further, the definitions are parameterized in a name, $x$. Can you,
  gentle reader, make a definition that eliminates this parameter and
  guarantees no accidental interaction between the replication
  machinery and the process being replicated -- i.e. no accidental
  sharing of names used by the process to get its work done and the
  name(s) used by the replication to effect copying. This latter
  revision of the definition of replication is crucial to obtaining
  the expected identity $!!P \sim !P$.
\end{remark}

\begin{remark}\label{rem:paradoxical_combinator}
  The reader familiar with the lambda calculus will have noticed the
  similarity between $D$ and the paradoxical combinator.

  [Ed. note: the existence of this seems to suggest we have to be more
  restrictive on the set of processes and names we admit if we are to
  support no-cloning.]
\end{remark}

\subsubsection{Bisimulation}

The computational dynamics gives rise to another kind of equivalence,
the equivalence of computational behavior. As previously mentioned
this is typically captured \emph{via} some form of bisimulation.

% The notion we use in this paper is weak barbed bisimulation
% \cite{milner91polyadicpi}.

The notion we use in this paper is derived from weak barbed
bisimulation \cite{milner91polyadicpi}. 

\begin{definition}
An \emph{observation relation}, $\downarrow_{\mathcal N}$, over a set
of names, $\mathcal N$, is the smallest relation satisfying the rules
below.

\infrule[Out-barb]{y \in {\mathcal N}, \; x \nameeq y}
		  {\outputp{x}{v} \downarrow_{\mathcal N} x}
\infrule[Par-barb]{\mbox{$P\downarrow_{\mathcal N} x$ or $Q\downarrow_{\mathcal N} x$}}
		  {\binpar{P}{Q} \downarrow_{\mathcal N} x}

We write $P \Downarrow_{\mathcal N} x$ if there is $Q$ such that 
$P \wred Q$ and $Q \downarrow_{\mathcal N} x$.
\end{definition}

\begin{definition}
%\label{def.bbisim}
An  ${\mathcal N}$-\emph{barbed bisimulation} over a set of names, ${\mathcal N}$, is a symmetric binary relation 
${\mathcal S}_{\mathcal N}$ between agents such that $P\rel{S}_{\mathcal N}Q$ implies:
\begin{enumerate}
\item If $P \red P'$ then $Q \wred Q'$ and $P'\rel{S}_{\mathcal N} Q'$.
\item If $P\downarrow_{\mathcal N} x$, then $Q\Downarrow_{\mathcal N} x$.
\end{enumerate}
$P$ is ${\mathcal N}$-barbed bisimilar to $Q$, written
$P \wbbisim_{\mathcal N} Q$, if $P \rel{S}_{\mathcal N} Q$ for some ${\mathcal N}$-barbed bisimulation ${\mathcal S}_{\mathcal N}$.
\end{definition}

$\mathcal{R} \subseteq \pi \times \pi$

$P \mathcal{R} Q => \forall P'. P \red P' \Rightarrow \exists Q'. Q \red Q', P' \mathcal{R} Q'$

$P \vdash x \Rightarrow Q \vdash x$

\begin{mathpar}
  \inferrule*[lab=Out-barb]{x \nameeq y}{{y}!\langle{Q}\rangle \vdash x}
  \and
  \inferrule*[lab=Par-barb]{\mbox{$P\vdash x$ or $Q\vdash x$}}{\binpar{P}{Q} \vdash x}
\end{mathpar}

\subsubsection{Contexts}

One of the principle advantages of computational calculi like the
$\pi$-calculus is a well-defined notion of context,
contextual-equivalence and a correlation between
contextual-equivalence and notions of bisimulation. The notion of
context allows the decomposition of a process into (sub-)process and
its syntactic environment, its context. Thus, a context may be
thought of as a process with a ``hole'' (written $\Box$) in it. The
application of a context $M$ to a process $P$, written $M[P]$, is
tantamount to filling the hole in $M$ with $P$. In this paper we do
not need the full weight of this theory, but do make use of the notion
of context in the proof the main theorem. 

\begin{mathpar}
  \inferrule* [lab=summation] {} {{M_{M},M_{N}} \bc \Box \;|\; x.M_{A} \;|\; M_{M}+M_{N}}
  \and
  \inferrule* [lab=agent] {} {{M_{A}} \bc (\vec{x})M_{P} \;| \; \clift{P_0,\ldots,M_{P},\ldots,P_N}}
  \and \\
  \inferrule* [lab=process] {} {{M_{P}} \bc M_{N} \;| \;P|M_{P} }
\end{mathpar} 

\begin{mathpar}
  \inferrule* [lab=sychronization] {} {M_{N} \bc \Box \;|\; x?M_{F} \;|\; x!M_{C}}
  \and
  \inferrule* [lab=abstraction] {} {{M_{F}} \bc (x)M_{P} }
  \and
  \inferrule* [lab=concretion] {} {{M_{C}} \bc \langle M_{P} \rangle }
  \and \\
  \inferrule* [lab=process] {} {{M_{P}} \bc M_{N} \;| \;P|M_{P} }
\end{mathpar}

\begin{definition}[contextual application] Given a context $M$, and
  process $P$, we define the \emph{contextual application}, $M[P] :=
  M\{P/\Box\}$. That is, the contextual application of M to P is the
  substitution of $P$ for $\Box$ in $M$.
\end{definition}

$\meaningof{-} : L \to \mathcal{P}(\pi)$

\begin{mathpar}
  \inferrule* [lab=collection] {} {\meaningof{true} = \pi, \and \meaningof{~E} = \pi \setminus \meaningof{E}, \and \meaningof{E_{1} \& E_{2}} = \meaningof{E_{1}} \cap \meaningof{E_{2}}}
\end{mathpar}

\begin{mathpar}
  \inferrule* [lab=structure] {} {\meaningof{0} = \{ P \in \pi | P \equiv 0 \}, \and \\ \meaningof{E_1 | E_2} = \{ P \in \pi | P \equiv P_{1} | P_{2}, P_{1} \in \meaningof{E_{1}}, P_{2} \in \meaningof{E_2}\} }
\end{mathpar}

\begin{mathpar}
 \inferrule* [lab=behavior] {} {\meaningof{\langle a?b \rangle E} = \{ P \in \pi | P \equiv Q | u?(y)P', \\ \and \\\\ \and \\ \;\;\; u \in \meaningof{a}, \forall z.P'\{z/y\} \in \meaningof{E\{z/b\}}\}, \and \\ \meaningof{a!E} = \{ P \in \pi | P \equiv Q | x!\langle P' \rangle, x \in \meaningof{a} P' \in \meaningof{E}\} }
\end{mathpar}

\begin{mathpar}
 \inferrule* [lab=nominal] {} {\meaningof{\quotep{E}} = \{ \quotep{P} \in \quotep{\pi} | P \in \meaningof{E} \}, \and \meaningof{\quotep{P}} = \{ \quotep{Q} \in \quotep{\pi} | P \equiv Q \} \and \\ \meaningof{@\quotep{E}} = \{ P \in \pi | P \equiv @x, x \in \meaningof{E} \}}
\end{mathpar}

\begin{eqnarray*}
  \\
  \meaningof{-} : TS \to ST
\end{eqnarray*}

\begin{eqnarray*}
  \\
  L : TS \to ST
\end{eqnarray*}

\begin{eqnarray*}
  \\
  P \models E \iff P \in \meaningof{E}
\end{eqnarray*}

\begin{eqnarray*}
  P \approx_{L} Q \iff \forall E \in L. P \models E \iff Q \models E
\end{eqnarray*}

\begin{eqnarray*}
  P \approx_{K} Q
\end{eqnarray*}

\begin{eqnarray*}
  P \approx Q
\end{eqnarray*}

$\approx_{K} = \approx = \approx_{L}$

\subsubsection{Contextual duality}

Note that contexts extend the quotation operation to a family of
operations from processes to names. Given a context, $M$, we can
define a \emph{nominal context}, $\quotep{M}$ by $\quotep{M}[P] :=
\quotep{M[P]}$. To foreshadow what is to come we observe that these
operations enjoy a duality with processes very much like the duality
between vectors and maps from vectors to scalars.

Further, because the calculus is essentially higher-order, we have a
correspondence between contexts and processes. More specifically,
given a name $x$ and a context $M$ we can construct $M^{*}_{x}$ such
that 

\begin{mathpar}
  M^{*}_{x} | \lift{x}{P} \red M[P]
\end{mathpar}

namely,

\begin{mathpar}
  M^{*}_{x} := x?(u).M[\dropn{u}]
\end{mathpar}

The dependence of $M^{*}_{x}$ on a name makes it an abstraction, 

\begin{mathpar}
  M^{*} := (x)x?(u).M[\dropn{u}]
\end{mathpar}

\subsection{Additional notation}

It will sometimes be convenient to denote the process a name
quotes. We already have the notation $x = \quotep{P}$, but it will be
convenient to introduce an alternate notation, $\procn{x}$, when we
want to emphasize the connection to the use of the name. Note that, by
virtue of name equivalence, $\quotep{\procn{x}} \nameeq x$; so, the
notation is consistent with previous definitions.

Further, because names have structure it is possible to effect
substitutions on the basis of that structure. This means we need to
upgrade our notation for substitutions, which we accomplish by
adapting comprehension notation. Thus,

\begin{mathpar}
  P\{ y / x : x \in S \}
\end{mathpar}

is interpreted to mean the process derived from P by replacing (in a
capture-avoiding manner) each occurrence of $x$ in $S$ by $y$. For example,

\begin{mathpar}
  P\{ \quotep{\procn{x}|\procn{x}} / x : x \in \freenames{P} \}
\end{mathpar}

will replace each (occurrence) of a free name $x$ in $P$ by
$\quotep{\procn{x}|\procn{x}}$.

Also, we will avail ourselves of the notation $x^{L}$ and $x^{R}$ to
denote injections of a name into disjoint copies of the name
space. There are numerous ways to accomplish this. One example can be
found in \cite{MeredithR05}. This notation overloads to vectors of
names: $\vec{x}^{\pi} := (x_{i}^{\pi} \; : \; 0 \leq i < |\vec{x}| )$ where $\pi \in \{L,R\}$.

We also use $P^{\Box} := P|\Box$.

In \cite{MeredithR05} an interpretation of the new operator is
given. It turns out that there are several possible interpretations
all enjoying the requisite algebraic properties of the operator (see
\cite{milner91polyadicpi}). We will therefore make liberal use of
$(\nu\; \vec{x})P$.

% subsection the_syntax_and_semantics_of_the_notation_system (end)   

\input{qm2pi.qmops} 

\input{qm2pi.sterngerlach} 

\input{qm2pi.metric} 

% section concurrent_process_calculi (end)

%\input{qm2pi.proofsketch}

% section proof sketch (end)

%\input{qm2pi.slviaknots} 

% section spatial logic via knots (end)

\input{qm2pi.conclusion}

% section conclusion (end)

%\input{qm2pi.dtcodes} 

% section wiring algorithm (end)

\input{qm2pi.ack} 

% section acknowledgments (end)

\newpage


\bibliographystyle{plain}   
\bibliography{../../biblios/main.bib}

\input{qm2pi.rhodetails}

\end{document}



% section proof sketch (end)

%\section{Unlikely characters: spatial logic for
  knots}\label{sub:characteristic_formulae} % (fold)

Associated to the mobile process calculi are a family of logics known
as the Hennessy-Milner logics. These logics typically enjoy a
semantics interpreting formulae as sets of processes that when
factored through the encoding outlined above allows an identification
of classes of knots with logical formulae. In the context of this
encoding the sub-family known as the spatial logics \cite{CairesC03}
\cite{CairesC04} \cite{Caires04} are of particular interest providing
several important features for expressing and reasoning about
properties (i.e. classes) of knots. We hint here at how this may be done.

%\begin{description}
%\item [structural connectives] 
\subsubsection{Structural connectives} The spatial logics enjoy
structural connectives corresponding, at the logical level, to the
parallel composition ($P | Q$) and new name ($(\nu \; x)P$)
connectives for processes. As illustrated in the examples below, these
connectives are extremely expressive given the shape of our encoding.
%\item [decideable satisfaction]

\subsubsection{Decideable satisfaction}
In \cite{Caires04} the satisfaction relation is shown to be decideable
for a rich class of processes. It further turns out that the image of
the our encoding is a proper subset of that class. This result
provides the basis for an algorithm by which to search for knots
enjoying a given property.
%\item [characteristic formulae]

\subsubsection{Characteristic formulae}
In the same paper \cite{Caires04} , Caires presents a means of calculating
characteristic formulae, selecting equivalence classes of processes
up to a pre--specified depth limit on the support set of names. Composed with our
encoding, this characteristic formula can be used to select
characteristic formulae for knots.
%\end{description}

\subsubsection{Spatial logic formulae}

The grammar below (segmented for comprehension) summarizes the syntax
of spatial logic formulae. We employ illustrative examples in the
sequel to provide an intuitive understanding of their meaning
referring the reader to \cite{Caires04} for a more detailed explication
of the semantics.

\begin{mathpar}
  \inferrule* [lab=boolean] {} {{A,B} \bc T \;|\; \neg A \;|\; A \wedge B \;|\; \eta = \eta'}
  \and
  \inferrule* [lab=spatial] {} {|\; \pzero \;|\; A | B \;|\; x \text{\textregistered} A \;|\; \forall x . A \;|\;  H x . A}
  \and
  \inferrule* [lab=behavioral] {} {|\; \alpha . A}
  \and 
  \inferrule* [lab=recursion] {} {|\; X(\vec{u}) \;|\; \mu X(\vec{u}) . A}
  \and
  \inferrule* [lab=action] {} {\alpha \bc \langle x?(\vec{y}) \rangle \;|\; \langle x!(\vec{y}) \rangle \;|\; \langle \tau \rangle}
  \and 
  \inferrule* [lab=name] {} {\eta \bc x \;|\; \tau}
\end{mathpar} 

% subsection characteristic_formulae (end)   	 

\subsection{Example formulae}\label{sub:example_formulae_} % (fold)

\subsubsection{Crossing as formula.}
% 
% \begin{align*}
%   \frac{d}{dx} \sin x &= \cos x 
%   & \frac{d}{dx} e^x &= e^x \\
%   \frac{d}{dx} \cos x &= - \sin x 
%   & \frac{d}{dx} \log x &= \frac{1}{x} \\
% \end{align*} 

\begin{align*}
 \mu C(x_{0},x_{1},y_{0},y_{1},u).&(\langle x_{0}?(z) \rangle(\langle u! \rangle\langle y_{1}!z \rangle C(x_{0},x_{1},y_{0},y_{1},u)) & \\
  & \wedge \langle y_{1}?(z) \rangle (\langle u! \rangle \langle x_{0}!z \rangle C(x_{0},x_{1},y_{0},y_{1},u)) & \\
  & \wedge \langle x_{1}?(z) \rangle (\langle u? \rangle \langle y_{0}!z \rangle C(x_{0},x_{1},y_{0},y_{1},u)) & \\
  & \wedge \langle y_{0}?(z) \rangle (\langle u? \rangle \langle x_{1}!z \rangle C(x_{0},x_{1},y_{0},y_{1},u))) &
\end{align*}

The lexicographical similarity between the shape of this formulae and
the shape of definition of the process representing a crossing reveals
the intuitive meaning of this formulae. It describes the capabilities
of a process that has the right to represent a crossing. For example
it picks out processes that may perform an input on the port $x_0$ in
its initial menu of capabilities. What differentiates the formula
from the process, however, is that the crossing process is the
smallest candidate to satisfy the formula. Infinitely many other
processes -- with internal behavior hidden behind this interface, so
to speak -- also satisfy this formula. Even this simple formula,
then, can be seen to open a new view onto knots, providing a
computational interpretation of \emph{virtual} knots.

Note that this formula is derived by hand. A similar formula can be
derived by employing Caires' calculation of characteristic formula
\cite{Caires04} to the process representing a crossing. In light of
this discussion, we let
$\meaningof{C}_{\phi}(x0,x1,y0,y1,u)$ denote a formula specifying the
dynamics we wish to capture of a crossing. To guarantee we preserve
the shape of the interface and minimal semantics we demand that
$\meaningof{C}_{\phi}(x0,x1,y0,y1,u) \Rightarrow
\textbf{C}(x0,x1,y0,y1,u)$ where $\textbf{C}(x0,x1,y0,y1,u)$ denotes
the formula above.
                            
\subsubsection{Crossing number constraints.}
The moral content of the context lemma (Lemma \ref{context}) is that the notion of
``locality'' in the Reidemeister moves is effectively captured by the
parallel composition operator of the process calculus. This intuition
extends through the logic. Given a formula,
$\meaningof{C}_{\phi}(x0,x1,y0,y1,u)$, we can use the structural
connectives to specify constraints on crossing numbers, such as at
least $n$ crossings, or exactly $n$ crossings.
\begin{mathpar}
  \inferrule* [lab=at-least-n] {} { K^{\geq n}_{\phi}(\vec{xs},\vec{ys}) := \Pi_{i=0}^{n-1} Hu . \meaningof{C}_{\phi}(xs_i,ys_i,u) | T }
  \and 
  \inferrule* [lab=exactly-n] {} { K^{= n}_{\phi}(\vec{xs},\vec{ys}) := \Pi_{i=0}^{n-1} Hu . \meaningof{C}_{\phi}(xs_i,ys_i,u) | \neg (\forall x_0,y_0,x_1,y_1,u . \meaningof{C}_{\phi}(x_0,y_0,x_1,y_1,u) | T) }
\end{mathpar}

To round out this section, recall that the encoding of an $n$-crossing
knot decomposes into a parallel composition of $n$ \emph{copies} of a
crossing process together with a wiring harness. To specify different
knot classes with the same crossing number amounts to specifying
logical constraints on the wiring harness. In the interest of space,
we defer examples to a forthcoming paper. Suffice it to say that both
the conditions ``alternating knot'' and ``contains the tangle
corresponding to 5/3'' are expressible. For example, it is possible to
calculate the characteristic formula of a process corresponding to the
tangle 5/3 and conjoin it into the classifying formula via the
composition connective of the logic.

Finally, we wish to observe that it is entirely within reason to
contemplate a more domain-specific version of spatial logic tailored
to the shape of processes in the image of the encoding. Such a
domain-specific logic would have a better claim to the title formal
language of knot properties.

% subsection example_formulae_ (end)

% section knots_as_processes (end) 

% section spatial logic via knots (end)

\section{Conclusions and future work}

\paragraph{Testing physical space}
You, gentle reader, may wonder why of all the theorems to be proved
given this set up we pick the one above. In some sense it's hardly
central to quantum mechanics. We see it as central in the sense that
it firmly establishes a notion of physical space arising from a notion
of the equivalence of behavior. Relating bisimulation to a metric is a
big step forward, but one is faced with interpreting the relationship
of that metric space to something more physical. Quantum mechanical
notions of ``physical'' space are still far from intuitive, but by
relating this idea of distance as testing to calculations that predict
physical circumstances we are making a not insignificant step forward
toward an understanding of the physical space we inhabit as
essentially dynamic.

\paragraph{Effectivity and simulation}
One of the observations we have yet to make is that the entire program
spelled out here is effective. We have built various interpreters for
the reflective calculus at work in this interpretation. In principle,
then, we can simulate quantum mechanics on a computer. The place where
the simulation may lose fidelity is the infinitely branching summation
for the annihilator.

In this connection i also want to point out that the evaluation style
calculation of the inner product puts the non-determinism of the
summation right at the heart of measurement. This suggests that
Milner's original reduction-based formulation of the dynamics of his
calculi in terms of sums was not just notationally suggestive of a
notion of measure-and-continue but captured some significant part of
the physics.

\paragraph{Quantum continuations}
In light of this last observation i want to point out that the
predominant account of quantum mechanics is missing a key aspect of a
truly compositional story of the physical situation. In a real lab,
when a measurement is made the observation can be made to feed into
another device that then makes another measurement conditioned on the
results of the first. This means that after the superposition was
collapsed the entire experimental set up remained in
superposition. While QM offers a means of writing this down it doesn't
quite line up well with the well-trodden formulation of computation
and continuation that we see so succinctly expressed in Milner's
calculi. This suggests that there might be advantages to this account
of dynamics waiting to be explored.

\paragraph{Quantum logic}
In this connection, we also note that by virtue of having the
Hennessy-Milner construction, we can pull the construction through the
interpretation of QM. This gives us a natural candidate for a quantum
logic that enjoys an extremely tight connection with it's domain of
interpretation, making the construction much less ad hoc (rather it is
the image of functor!).

\paragraph{Quantum probabiity}
i have questions about the basis of the interpretation of inner
product as probability amplitude. In particular, using which
axiomatization of probability theory does the notion of probability
amplitude earn the right to be so dubbed? In other words, where is the
proof that the operation for calculating a probability amplitude (and
then squaring) satisfies the axioms of what it means to calculate a
probability? Even if such a proof exists (i have yet to find it in the
literature), i wonder if it might not be possible to turn things on
their heads. Can we view the calculation of the probability amplitude
as an axiomatization of probability? If so, then the definition we
give for calculating probability amplitude may provide the basis for
an \emph{effective} theory of probability.

\paragraph{Quantum vs ``biological'' information}
Finally, i want to conclude with a more philosophical observation. At
a recent workshop in which QM was a predominant topic i noticed
something about quantum information. The speaker was giving a riveting
discussion of axiomatic QM and showing how properties of ``no
cloning'' and ``no deleting'' emerged as consequences of the
axiomatization. Theorems of this form are necessary to give us a sense
of confidence that our axioms characterize the physical theory. What
struck me, though, was that if quantum information is neither erasable
nor replicable it is markedly different from \emph{life}. Two of the
things we know about life is that

\begin{itemize}
  \item it ends;
  \item to gain some measure of persistence, to transcend it's
    finitude it is imminently copyable.
\end{itemize}

Both of these qualities are summarized succinctly in the aphorism: all
flesh is grass. For me these two kinds of ``information'' -- call them
quantum and biological -- are end points on a spectrum of strategies
for persistence. At one end, we have those curious entities that enjoy
uniqueness and permanence; at the other, we have those who in the face
of a certain end and an uncertain present make a go of passing
something on. To me one of the more remarkable aspects of the latter
strategy is that in the presence of noise (and certain features of
copying) we get a kind of dynamism, a chance for improvement against a
given persistent condition.

% subsection other_calculi_other_bisimulations_and_geometry_as_behavior (end)




% section conclusion (end)

%\documentclass[12pt]{llncs}
%\documentclass{jktr}

\usepackage[pdftex]{hyperref}                   
\usepackage {listings}
\usepackage {mathpartir}
\usepackage{bcprules}
%\usepackage{listings}
                       
\usepackage{graphicx} 
%\usepackage[margins=2.5cm,nohead,nofoot]{geometry}
%\usepackage{geometry}
\usepackage{amsfonts}
\usepackage{amstext}
\usepackage{latexsym}
\usepackage{amssymb}
\usepackage{color}


%\include{myPreamble}
\include{qm2pi.local} 

%\ifpdf
%\usepackage[pdftex]{graphicx}
%\else
%\usepackage{graphicx}
%\fi

 % \ifpdf
%  \usepackage{pdfsync}
%  \if


%\title{Brief Article}
%\author{David F. Snyder}
%\author{L.G. Meredith}

%\address{Dept. of Math., Texas State University--San Marcos, San Marcos, TX 78666}
       
\pagestyle{empty}


\begin{document}

\lstset{language=[Objective]Caml,frame=shadowbox}

\input{qm2pi.front}

% section front matter (end)

\input{qm2pi.intro} 
 
% section introduction (end)

% \input{qm2pi.knotations} 

% section notation (end)

\input{qm2pi.process.calculi} 

% section concurrent_process_calculi_and_spatial_logics_ (end)
    
%\input{qm2pi.knots2pi} 

%\input{qm2pi.trefoil} 

%\input{qm2pi.mainthm} 

% subsection basic_interpretation (end)

%\input{qm2pi.rho.presentation} 
\subsection{The syntax and semantics of the notation system}\label{sub:the_syntax_and_semantics_of_the_notation_system} % (fold)

We now summarize a technical presentation of the calculus that
embodies our theory of dynamics. The typical presentation of such a
calculus follows the style of giving generators and relations on
them. The grammar, below, describing term constructors, freely
generates the set of processes, $\Proc$. This set is then quotiented
by a relation known as structural congruence and it is over this set
that the notion of dynamics is expressed. This presentation is
essentially that of \cite{MeredithR05} with the addition of
polyadicity and summation. For readability we have relegated some of
the technical subtleties to an appendix.

\subsubsection{Process grammar}\label{subsub:process_grammar}

\begin{mathpar}
  \inferrule* [lab=synchronization] {} {{M} \bc \pzero \;|\; x?F \;|\; x!C }
  \and
  \inferrule* [lab=abstraction] {} {{F} \bc (x)P}
  \and
  \inferrule* [lab=concretion] {} {{C} \bc \langle Q \rangle}
  \and
  \inferrule* [lab=process] {} {{P,Q} \bc M \;| \;P|Q \;|\; @{x}}
  \and
  \inferrule* [lab=name] {} {{x} \bc \quotep{P}}
\end{mathpar} 

Note that $\vec{x}$ (resp. $\vec{P}$) denotes a vector of names
(resp. processes) of length $|\vec{x}|$ (resp. $|\vec{P}|$). We adopt
the following useful abbreviations.

\begin{mathpar}
   x?(\vec{y}).P := x.(\vec{y})P \and  x\clift{\vec{P}} := x.\clift{\vec{P}}
   \and x!(y) := \lift{x}{\dropn{y}}
   \and \Pi_{i=0}^{n-1}P_i := P_0 | \ldots | P_{n-1}
\end{mathpar}

\subsubsection{Structural congruence}

\paragraph{Free and bound names and alpha-equivalence.} At the
core of structural equivalence is alpha-equivalence which identifies
process that are the same up to a change of variable. Formally, we
recognize the distinction between free and bound names. The free names
of a process, $\freenames{P}$, may be calculated recursively as
follows:

\begin{mathpar}
\freenames{\pzero} := \emptyset
  \and \\
  \freenames{x?(y).P} := \{ x \} \cup (\freenames{P} \setminus \{ y \})
  \and 
  \freenames{x!\langle P \rangle} := \{ x \} \cup \{ P \} 
  \and \\
  \freenames{P|Q} := \freenames{P} \cup \freenames{Q}
  \and \\
  \freenames{@{x}} := \{ x \}
\end{mathpar}

$\pi$
$\quotep{\pi}$

$\freenames{-} : \pi \to \mathcal{P}(\quotep{\pi})$

\begin{eqnarray*}
  \freenames{\pzero} & := & \emptyset \\
  \freenames{x?(y).P} & := & \{ x \} \cup (\freenames{P} \setminus \{ y \}) \\
  \freenames{x!\langle P \rangle} & := & \{ x \} \cup \{ P \} \\
  \freenames{P|Q} & := & \freenames{P} \cup \freenames{Q} \\
  \freenames{\dropn{x}} & := & \{ x \}
\end{eqnarray*}

The bound names of a process, $\boundnames{P}$, are those names occurring in $P$
that are not free. For example, in $x?(y).0$, the name $x$ is free, while $y$ is bound.

\begin{mathpar}
  \inferrule* [lab=monoidal-laws] {} { P|Q \equiv Q|P \and P|0 \equiv P \and P|(Q|R) \equiv (P|Q)|R }
\end{mathpar}

\begin{mathpar}
  \inferrule* [lab=alpha-equivalence] {} { (x)P \equiv (y)P\{y/x\} \and y \not\in \freenames{P} }
\end{mathpar}

\begin{definition}
Then two processes, $P,Q$, are alpha-equivalent if $P = Q\{\vec{y}/\vec{x}\}$ for
some $\vec{x} \in \boundnames{Q},\vec{y} \in \boundnames{P}$, where $Q\{\vec{y}/\vec{x}\}$
denotes the capture-avoiding substitution of $\vec{y}$ for $\vec{x}$ in $Q$.
\end{definition}

\begin{definition}
  The {\em structural congruence} \cite{SangiorgiWalker} , $\equiv$,
  between processes is the least congruence containing
  alpha-equivalence, satisfying the abelian monoid laws
  (associativity, commutativity and $\pzero$ as identity) for parallel
  composition $|$ and for summation $+$.
\end{definition}

\subsection{Name equivalence}

We take name equivalence, written $\nameeq$, to be the smallest
equivalence relation generated by the following rules.

\begin{mathpar}
\inferrule*[lab=Quote-drop]
{ }
{ \quotep{@{x}} \nameeq x }

\inferrule*[lab=Struct-equiv]
{ P \scong Q }
{ \quotep{P} \nameeq \quotep{Q} }
\end{mathpar}

The astute reader will have noticed that the mutual recursion of names
and processes imposes a mutual recursion on alpha-equivalence and
structural equivalence via name-equivalence. Fortunately, all of this
works out pleasantly and we may calculate in the natural way, free of
concern. The reader interested in the details is referred to the
appendix \ref{appendix:rho_details}.

\subsection{Substitution}

We use $\Proc$ for the set of processes, $\QProc$ for the set of
names, and $\id{\{}\vec{y} / \vec{x} \id{\}}$ to denote partial maps,
$s : \QProc \rightarrow \QProc$. A map, $s$ lifts, uniquely, to a map
on process terms, $\widehat{s} : \Proc \rightarrow \Proc$ by the
following equations.

\begin{mathpar}
  (0) \psubstp{Q}{P} := 0 \\
  (R \juxtap S) \psubstp{Q}{P}
  :=    
  (R)\psubstp{Q}{P} \juxtap (S) \psubstp{Q}{P} \\
  (x?(y).R) \psubstp{Q}{P}    
  :=    
  (x)\substp{Q}{P} (z)\concat( (R \psubstn{z}{y}) \psubstp{Q}{P} ) \\
  (\lift{x}{R}) \psubstp{Q}{P}  
  :=
  \lift{(x)\substp{Q}{P}}{ R \psubstp{Q}{P} } \\
%   (\dropn{x})  \psubstp{Q}{P}       
%   := 
%   \left\{ 
%     \begin{array}{ccc} 
%       \dropn{\quotep{Q}} & & x \nameeq \quotep{P} \\
%       \dropn{x} & & otherwise \\
%     \end{array}
%   \right. 
  (\dropn{x})  \psubstp{Q}{P}       
  := 
  \left\{ 
    \begin{array}{ccc} 
      Q & & x \nameeq \quotep{P} \\
      \dropn{x} & & otherwise \\
    \end{array}
  \right.
\end{mathpar}
 

where

\begin{eqnarray}
  (x)\id{\{} \lpquote Q \rpquote / \lpquote P \rpquote \id{\}}            = 
  \left\{ 
    \begin{array}{ccc}
      \lpquote Q \rpquote & & x \nameeq \lpquote P \rpquote \\
      x & & otherwise \\
    \end{array}
  \right. \nonumber
\end{eqnarray}

and $z$ is chosen distinct from $\quotep{P}$, $\quotep{Q}$, the free
names in $Q$, and all the names in $R$. Our $\alpha$-equivalence will
be built in the standard way from this substitution.

\begin{remark}\label{rem:no_self_referential_names}
  One consequence of these definitions is that $\forall P. \quotep{P}
  \not\in \freenames{P}$.
\end{remark}

\subsection{ Dynamic quote: an example }

Anticipating something of what's to come, consider applying the
substitution, $\widehat{\id{\{}u / z \id{\}}}$, to the following pair
of processes, $\lift{w}{y!(z)}$ and $w[ \lpquote y!(z) \rpquote ]$.

\begin{eqnarray}
	\lift{w}{y!(z)}\widehat{\id{\{}u / z \id{\}}}
		& = &
		\lift{w}{y!(u)} \nonumber\\
	w[ \lpquote y!(z) \rpquote ] \widehat{ \id{\{}u / z \id{\}} }
		& = &
		w[ \lpquote y!(z) \rpquote ] \nonumber
\end{eqnarray}

Because the body of the process between quotes is impervious to
substitution, we get radically different answers. In fact, by
examining the first process in an input context,
e.g. $x?(z).\lift{w}{y!(z)}$, we see that the process under the lift
operator may be shaped by prefixed inputs binding a name inside it. In
this sense, the lift operator will be seen as a way to dynamically
construct processes before reifying them as names.

Finally equipped with these standard features we can present the
dynamics of the calculus.

\subsubsection{Operational semantics} 

Finally, we introduce the computational dynamics. What marks these
algebras as distinct from other more traditionally studied algebraic
structures, e.g. vector spaces or polynomial rings, is the manner in
which dynamics is captured. In traditional structures, dynamics is typically
expressed through morphisms between such structures, as in linear maps
between vector spaces or morphisms between rings. In algebras
associated with the semantics of computation, the dynamics is
expressed as part of the algebraic structure itself, through a
reduction reduction relation typically denoted by $\red$. Below, we
give a recursive presentation of this relation for the calculus used
in the encoding.

$\red \subseteq \pi \times \pi$
$\red : \pi \to \mathcal{P}(\pi)$

\begin{mathpar}
  \inferrule* [lab=Comm] { \textsf{match}( x_{src}, x_{trgt} ) } { x_{trgt}?(y)P \; | \; x_{src}!\langle {Q} \rangle \red P\{\quotep{Q}/y}\} }
  \and \\
  \inferrule* [lab=Par] {{P} \red {P}'} {{{P} | {Q}} \red {{P}' | {Q}}}
  \and
  \inferrule* [lab=Equiv]{{{P} \scong {P}'} \andalso {{P}' \red {Q}'} \andalso {{Q}' \scong {Q}}}{{P} \red {Q}}
\end{mathpar}

\begin{eqnarray*}
  match_{\equiv} (\quotep{P},\quotep{Q}) & := & P \equiv Q \\
  match_{\dagger}(\quotep{P},\quotep{Q}) & := & \forall R. P|Q \red^{*} R => R \red^{*} 0 \\
  match_{K}(\quotep{P},\quotep{Q}) & := & K \mbox{ for some context } K
\end{eqnarray*}

$u?(x)P | u!\langle Q \rangle \red P\{\quotep{Q}/x\}$

%We write $\wred$ for $\red^*$, and $P\red$ if $\exists Q $ such that $ P \red Q$.
We write $P\red$ if $\exists Q $ such that $ P \red Q$ and $P\not\red$, otherwise.

\section{Replication}

As mentioned before, it is known that replication (and hence
recursion) can be implemented in a higher-order process algebra
\cite{SangiorgiWalker}. As our first example of calculation with the
machinery thus far presented we give the construction explicitly in
the {\rhoc}.

\begin{eqnarray}
	D_{x} & := & \prefix{x}{y}{(\binpar{\outputp{x}{y}}{@{y}})} \nonumber\\
	\bangp_{x}{P} & := & \binpar{{x}!\langle{\binpar{D_{x}}{P}}\rangle}{D_{x}} \nonumber
\end{eqnarray}

\begin{eqnarray}
	\bangp_{x}{P} & & \nonumber\\
	=
	& {x}!\langle{(\prefix{x}{y}{(\outputp{x}{y} | @{y})) | P}}\rangle 
	      | \prefix{x}{y}{(\outputp{x}{y} | @{y})} & \nonumber\\
	\red
	& (\outputp{x}{y} | @{y})\substn{\quotep{(\prefix{x}{y}{(@{y} | \outputp{x}{y})) | P}}}{y} & \nonumber\\
	=
	& \outputp{x}{\quotep{(\prefix{x}{y}{(\outputp{x}{y} | @{y})) | P}}}
	  | {(\prefix{x}{y}{(\outputp{x}{y} | @{y})) | P}} & \nonumber\\
	\red
	& \ldots & \nonumber\\
	\red^*
	& P | P | \ldots & \nonumber
\end{eqnarray}

Of course, this encoding, as an implementation, runs away, unfolding
$\bangp{P}$ eagerly. A lazier and more implementable replication
operator, restricted to input-guarded processes, may be obtained as follows.

\begin{eqnarray}
\bangp{\prefix{u}{v}{P}} 
	:= 
	\binpar{\lift{x}{\prefix{u}{v}{(\binpar{D(x)}{P})}}}{D(x)} \nonumber
\end{eqnarray}

\begin{remark}
  Note that the lazier definition still does not deal with summation
  or mixed summation (i.e. sums over input and output). The reader is
  invited to construct definitions of replication that deal with these
  features. 

  Further, the definitions are parameterized in a name, $x$. Can you,
  gentle reader, make a definition that eliminates this parameter and
  guarantees no accidental interaction between the replication
  machinery and the process being replicated -- i.e. no accidental
  sharing of names used by the process to get its work done and the
  name(s) used by the replication to effect copying. This latter
  revision of the definition of replication is crucial to obtaining
  the expected identity $!!P \sim !P$.
\end{remark}

\begin{remark}\label{rem:paradoxical_combinator}
  The reader familiar with the lambda calculus will have noticed the
  similarity between $D$ and the paradoxical combinator.

  [Ed. note: the existence of this seems to suggest we have to be more
  restrictive on the set of processes and names we admit if we are to
  support no-cloning.]
\end{remark}

\subsubsection{Bisimulation}

The computational dynamics gives rise to another kind of equivalence,
the equivalence of computational behavior. As previously mentioned
this is typically captured \emph{via} some form of bisimulation.

% The notion we use in this paper is weak barbed bisimulation
% \cite{milner91polyadicpi}.

The notion we use in this paper is derived from weak barbed
bisimulation \cite{milner91polyadicpi}. 

\begin{definition}
An \emph{observation relation}, $\downarrow_{\mathcal N}$, over a set
of names, $\mathcal N$, is the smallest relation satisfying the rules
below.

\infrule[Out-barb]{y \in {\mathcal N}, \; x \nameeq y}
		  {\outputp{x}{v} \downarrow_{\mathcal N} x}
\infrule[Par-barb]{\mbox{$P\downarrow_{\mathcal N} x$ or $Q\downarrow_{\mathcal N} x$}}
		  {\binpar{P}{Q} \downarrow_{\mathcal N} x}

We write $P \Downarrow_{\mathcal N} x$ if there is $Q$ such that 
$P \wred Q$ and $Q \downarrow_{\mathcal N} x$.
\end{definition}

\begin{definition}
%\label{def.bbisim}
An  ${\mathcal N}$-\emph{barbed bisimulation} over a set of names, ${\mathcal N}$, is a symmetric binary relation 
${\mathcal S}_{\mathcal N}$ between agents such that $P\rel{S}_{\mathcal N}Q$ implies:
\begin{enumerate}
\item If $P \red P'$ then $Q \wred Q'$ and $P'\rel{S}_{\mathcal N} Q'$.
\item If $P\downarrow_{\mathcal N} x$, then $Q\Downarrow_{\mathcal N} x$.
\end{enumerate}
$P$ is ${\mathcal N}$-barbed bisimilar to $Q$, written
$P \wbbisim_{\mathcal N} Q$, if $P \rel{S}_{\mathcal N} Q$ for some ${\mathcal N}$-barbed bisimulation ${\mathcal S}_{\mathcal N}$.
\end{definition}

$\mathcal{R} \subseteq \pi \times \pi$

$P \mathcal{R} Q => \forall P'. P \red P' \Rightarrow \exists Q'. Q \red Q', P' \mathcal{R} Q'$

$P \vdash x \Rightarrow Q \vdash x$

\begin{mathpar}
  \inferrule*[lab=Out-barb]{x \nameeq y}{{y}!\langle{Q}\rangle \vdash x}
  \and
  \inferrule*[lab=Par-barb]{\mbox{$P\vdash x$ or $Q\vdash x$}}{\binpar{P}{Q} \vdash x}
\end{mathpar}

\subsubsection{Contexts}

One of the principle advantages of computational calculi like the
$\pi$-calculus is a well-defined notion of context,
contextual-equivalence and a correlation between
contextual-equivalence and notions of bisimulation. The notion of
context allows the decomposition of a process into (sub-)process and
its syntactic environment, its context. Thus, a context may be
thought of as a process with a ``hole'' (written $\Box$) in it. The
application of a context $M$ to a process $P$, written $M[P]$, is
tantamount to filling the hole in $M$ with $P$. In this paper we do
not need the full weight of this theory, but do make use of the notion
of context in the proof the main theorem. 

\begin{mathpar}
  \inferrule* [lab=summation] {} {{M_{M},M_{N}} \bc \Box \;|\; x.M_{A} \;|\; M_{M}+M_{N}}
  \and
  \inferrule* [lab=agent] {} {{M_{A}} \bc (\vec{x})M_{P} \;| \; \clift{P_0,\ldots,M_{P},\ldots,P_N}}
  \and \\
  \inferrule* [lab=process] {} {{M_{P}} \bc M_{N} \;| \;P|M_{P} }
\end{mathpar} 

\begin{mathpar}
  \inferrule* [lab=sychronization] {} {M_{N} \bc \Box \;|\; x?M_{F} \;|\; x!M_{C}}
  \and
  \inferrule* [lab=abstraction] {} {{M_{F}} \bc (x)M_{P} }
  \and
  \inferrule* [lab=concretion] {} {{M_{C}} \bc \langle M_{P} \rangle }
  \and \\
  \inferrule* [lab=process] {} {{M_{P}} \bc M_{N} \;| \;P|M_{P} }
\end{mathpar}

\begin{definition}[contextual application] Given a context $M$, and
  process $P$, we define the \emph{contextual application}, $M[P] :=
  M\{P/\Box\}$. That is, the contextual application of M to P is the
  substitution of $P$ for $\Box$ in $M$.
\end{definition}

$\meaningof{-} : L \to \mathcal{P}(\pi)$

\begin{mathpar}
  \inferrule* [lab=collection] {} {\meaningof{true} = \pi, \and \meaningof{~E} = \pi \setminus \meaningof{E}, \and \meaningof{E_{1} \& E_{2}} = \meaningof{E_{1}} \cap \meaningof{E_{2}}}
\end{mathpar}

\begin{mathpar}
  \inferrule* [lab=structure] {} {\meaningof{0} = \{ P \in \pi | P \equiv 0 \}, \and \\ \meaningof{E_1 | E_2} = \{ P \in \pi | P \equiv P_{1} | P_{2}, P_{1} \in \meaningof{E_{1}}, P_{2} \in \meaningof{E_2}\} }
\end{mathpar}

\begin{mathpar}
 \inferrule* [lab=behavior] {} {\meaningof{\langle a?b \rangle E} = \{ P \in \pi | P \equiv Q | u?(y)P', \\ \and \\\\ \and \\ \;\;\; u \in \meaningof{a}, \forall z.P'\{z/y\} \in \meaningof{E\{z/b\}}\}, \and \\ \meaningof{a!E} = \{ P \in \pi | P \equiv Q | x!\langle P' \rangle, x \in \meaningof{a} P' \in \meaningof{E}\} }
\end{mathpar}

\begin{mathpar}
 \inferrule* [lab=nominal] {} {\meaningof{\quotep{E}} = \{ \quotep{P} \in \quotep{\pi} | P \in \meaningof{E} \}, \and \meaningof{\quotep{P}} = \{ \quotep{Q} \in \quotep{\pi} | P \equiv Q \} \and \\ \meaningof{@\quotep{E}} = \{ P \in \pi | P \equiv @x, x \in \meaningof{E} \}}
\end{mathpar}

\begin{eqnarray*}
  \\
  \meaningof{-} : TS \to ST
\end{eqnarray*}

\begin{eqnarray*}
  \\
  L : TS \to ST
\end{eqnarray*}

\begin{eqnarray*}
  \\
  P \models E \iff P \in \meaningof{E}
\end{eqnarray*}

\begin{eqnarray*}
  P \approx_{L} Q \iff \forall E \in L. P \models E \iff Q \models E
\end{eqnarray*}

\begin{eqnarray*}
  P \approx_{K} Q
\end{eqnarray*}

\begin{eqnarray*}
  P \approx Q
\end{eqnarray*}

$\approx_{K} = \approx = \approx_{L}$

\subsubsection{Contextual duality}

Note that contexts extend the quotation operation to a family of
operations from processes to names. Given a context, $M$, we can
define a \emph{nominal context}, $\quotep{M}$ by $\quotep{M}[P] :=
\quotep{M[P]}$. To foreshadow what is to come we observe that these
operations enjoy a duality with processes very much like the duality
between vectors and maps from vectors to scalars.

Further, because the calculus is essentially higher-order, we have a
correspondence between contexts and processes. More specifically,
given a name $x$ and a context $M$ we can construct $M^{*}_{x}$ such
that 

\begin{mathpar}
  M^{*}_{x} | \lift{x}{P} \red M[P]
\end{mathpar}

namely,

\begin{mathpar}
  M^{*}_{x} := x?(u).M[\dropn{u}]
\end{mathpar}

The dependence of $M^{*}_{x}$ on a name makes it an abstraction, 

\begin{mathpar}
  M^{*} := (x)x?(u).M[\dropn{u}]
\end{mathpar}

\subsection{Additional notation}

It will sometimes be convenient to denote the process a name
quotes. We already have the notation $x = \quotep{P}$, but it will be
convenient to introduce an alternate notation, $\procn{x}$, when we
want to emphasize the connection to the use of the name. Note that, by
virtue of name equivalence, $\quotep{\procn{x}} \nameeq x$; so, the
notation is consistent with previous definitions.

Further, because names have structure it is possible to effect
substitutions on the basis of that structure. This means we need to
upgrade our notation for substitutions, which we accomplish by
adapting comprehension notation. Thus,

\begin{mathpar}
  P\{ y / x : x \in S \}
\end{mathpar}

is interpreted to mean the process derived from P by replacing (in a
capture-avoiding manner) each occurrence of $x$ in $S$ by $y$. For example,

\begin{mathpar}
  P\{ \quotep{\procn{x}|\procn{x}} / x : x \in \freenames{P} \}
\end{mathpar}

will replace each (occurrence) of a free name $x$ in $P$ by
$\quotep{\procn{x}|\procn{x}}$.

Also, we will avail ourselves of the notation $x^{L}$ and $x^{R}$ to
denote injections of a name into disjoint copies of the name
space. There are numerous ways to accomplish this. One example can be
found in \cite{MeredithR05}. This notation overloads to vectors of
names: $\vec{x}^{\pi} := (x_{i}^{\pi} \; : \; 0 \leq i < |\vec{x}| )$ where $\pi \in \{L,R\}$.

We also use $P^{\Box} := P|\Box$.

In \cite{MeredithR05} an interpretation of the new operator is
given. It turns out that there are several possible interpretations
all enjoying the requisite algebraic properties of the operator (see
\cite{milner91polyadicpi}). We will therefore make liberal use of
$(\nu\; \vec{x})P$.

% subsection the_syntax_and_semantics_of_the_notation_system (end)   

\input{qm2pi.qmops} 

\input{qm2pi.sterngerlach} 

\input{qm2pi.metric} 

% section concurrent_process_calculi (end)

%\input{qm2pi.proofsketch}

% section proof sketch (end)

%\input{qm2pi.slviaknots} 

% section spatial logic via knots (end)

\input{qm2pi.conclusion}

% section conclusion (end)

%\input{qm2pi.dtcodes} 

% section wiring algorithm (end)

\input{qm2pi.ack} 

% section acknowledgments (end)

\newpage


\bibliographystyle{plain}   
\bibliography{../../biblios/main.bib}

\input{qm2pi.rhodetails}

\end{document}

 

% section wiring algorithm (end)

\documentclass[12pt]{llncs}
%\documentclass{jktr}

\usepackage[pdftex]{hyperref}                   
\usepackage {listings}
\usepackage {mathpartir}
\usepackage{bcprules}
%\usepackage{listings}
                       
\usepackage{graphicx} 
%\usepackage[margins=2.5cm,nohead,nofoot]{geometry}
%\usepackage{geometry}
\usepackage{amsfonts}
\usepackage{amstext}
\usepackage{latexsym}
\usepackage{amssymb}
\usepackage{color}


%\include{myPreamble}
\include{qm2pi.local} 

%\ifpdf
%\usepackage[pdftex]{graphicx}
%\else
%\usepackage{graphicx}
%\fi

 % \ifpdf
%  \usepackage{pdfsync}
%  \if


%\title{Brief Article}
%\author{David F. Snyder}
%\author{L.G. Meredith}

%\address{Dept. of Math., Texas State University--San Marcos, San Marcos, TX 78666}
       
\pagestyle{empty}


\begin{document}

\lstset{language=[Objective]Caml,frame=shadowbox}

\input{qm2pi.front}

% section front matter (end)

\input{qm2pi.intro} 
 
% section introduction (end)

% \input{qm2pi.knotations} 

% section notation (end)

\input{qm2pi.process.calculi} 

% section concurrent_process_calculi_and_spatial_logics_ (end)
    
%\input{qm2pi.knots2pi} 

%\input{qm2pi.trefoil} 

%\input{qm2pi.mainthm} 

% subsection basic_interpretation (end)

%\input{qm2pi.rho.presentation} 
\subsection{The syntax and semantics of the notation system}\label{sub:the_syntax_and_semantics_of_the_notation_system} % (fold)

We now summarize a technical presentation of the calculus that
embodies our theory of dynamics. The typical presentation of such a
calculus follows the style of giving generators and relations on
them. The grammar, below, describing term constructors, freely
generates the set of processes, $\Proc$. This set is then quotiented
by a relation known as structural congruence and it is over this set
that the notion of dynamics is expressed. This presentation is
essentially that of \cite{MeredithR05} with the addition of
polyadicity and summation. For readability we have relegated some of
the technical subtleties to an appendix.

\subsubsection{Process grammar}\label{subsub:process_grammar}

\begin{mathpar}
  \inferrule* [lab=synchronization] {} {{M} \bc \pzero \;|\; x?F \;|\; x!C }
  \and
  \inferrule* [lab=abstraction] {} {{F} \bc (x)P}
  \and
  \inferrule* [lab=concretion] {} {{C} \bc \langle Q \rangle}
  \and
  \inferrule* [lab=process] {} {{P,Q} \bc M \;| \;P|Q \;|\; @{x}}
  \and
  \inferrule* [lab=name] {} {{x} \bc \quotep{P}}
\end{mathpar} 

Note that $\vec{x}$ (resp. $\vec{P}$) denotes a vector of names
(resp. processes) of length $|\vec{x}|$ (resp. $|\vec{P}|$). We adopt
the following useful abbreviations.

\begin{mathpar}
   x?(\vec{y}).P := x.(\vec{y})P \and  x\clift{\vec{P}} := x.\clift{\vec{P}}
   \and x!(y) := \lift{x}{\dropn{y}}
   \and \Pi_{i=0}^{n-1}P_i := P_0 | \ldots | P_{n-1}
\end{mathpar}

\subsubsection{Structural congruence}

\paragraph{Free and bound names and alpha-equivalence.} At the
core of structural equivalence is alpha-equivalence which identifies
process that are the same up to a change of variable. Formally, we
recognize the distinction between free and bound names. The free names
of a process, $\freenames{P}$, may be calculated recursively as
follows:

\begin{mathpar}
\freenames{\pzero} := \emptyset
  \and \\
  \freenames{x?(y).P} := \{ x \} \cup (\freenames{P} \setminus \{ y \})
  \and 
  \freenames{x!\langle P \rangle} := \{ x \} \cup \{ P \} 
  \and \\
  \freenames{P|Q} := \freenames{P} \cup \freenames{Q}
  \and \\
  \freenames{@{x}} := \{ x \}
\end{mathpar}

$\pi$
$\quotep{\pi}$

$\freenames{-} : \pi \to \mathcal{P}(\quotep{\pi})$

\begin{eqnarray*}
  \freenames{\pzero} & := & \emptyset \\
  \freenames{x?(y).P} & := & \{ x \} \cup (\freenames{P} \setminus \{ y \}) \\
  \freenames{x!\langle P \rangle} & := & \{ x \} \cup \{ P \} \\
  \freenames{P|Q} & := & \freenames{P} \cup \freenames{Q} \\
  \freenames{\dropn{x}} & := & \{ x \}
\end{eqnarray*}

The bound names of a process, $\boundnames{P}$, are those names occurring in $P$
that are not free. For example, in $x?(y).0$, the name $x$ is free, while $y$ is bound.

\begin{mathpar}
  \inferrule* [lab=monoidal-laws] {} { P|Q \equiv Q|P \and P|0 \equiv P \and P|(Q|R) \equiv (P|Q)|R }
\end{mathpar}

\begin{mathpar}
  \inferrule* [lab=alpha-equivalence] {} { (x)P \equiv (y)P\{y/x\} \and y \not\in \freenames{P} }
\end{mathpar}

\begin{definition}
Then two processes, $P,Q$, are alpha-equivalent if $P = Q\{\vec{y}/\vec{x}\}$ for
some $\vec{x} \in \boundnames{Q},\vec{y} \in \boundnames{P}$, where $Q\{\vec{y}/\vec{x}\}$
denotes the capture-avoiding substitution of $\vec{y}$ for $\vec{x}$ in $Q$.
\end{definition}

\begin{definition}
  The {\em structural congruence} \cite{SangiorgiWalker} , $\equiv$,
  between processes is the least congruence containing
  alpha-equivalence, satisfying the abelian monoid laws
  (associativity, commutativity and $\pzero$ as identity) for parallel
  composition $|$ and for summation $+$.
\end{definition}

\subsection{Name equivalence}

We take name equivalence, written $\nameeq$, to be the smallest
equivalence relation generated by the following rules.

\begin{mathpar}
\inferrule*[lab=Quote-drop]
{ }
{ \quotep{@{x}} \nameeq x }

\inferrule*[lab=Struct-equiv]
{ P \scong Q }
{ \quotep{P} \nameeq \quotep{Q} }
\end{mathpar}

The astute reader will have noticed that the mutual recursion of names
and processes imposes a mutual recursion on alpha-equivalence and
structural equivalence via name-equivalence. Fortunately, all of this
works out pleasantly and we may calculate in the natural way, free of
concern. The reader interested in the details is referred to the
appendix \ref{appendix:rho_details}.

\subsection{Substitution}

We use $\Proc$ for the set of processes, $\QProc$ for the set of
names, and $\id{\{}\vec{y} / \vec{x} \id{\}}$ to denote partial maps,
$s : \QProc \rightarrow \QProc$. A map, $s$ lifts, uniquely, to a map
on process terms, $\widehat{s} : \Proc \rightarrow \Proc$ by the
following equations.

\begin{mathpar}
  (0) \psubstp{Q}{P} := 0 \\
  (R \juxtap S) \psubstp{Q}{P}
  :=    
  (R)\psubstp{Q}{P} \juxtap (S) \psubstp{Q}{P} \\
  (x?(y).R) \psubstp{Q}{P}    
  :=    
  (x)\substp{Q}{P} (z)\concat( (R \psubstn{z}{y}) \psubstp{Q}{P} ) \\
  (\lift{x}{R}) \psubstp{Q}{P}  
  :=
  \lift{(x)\substp{Q}{P}}{ R \psubstp{Q}{P} } \\
%   (\dropn{x})  \psubstp{Q}{P}       
%   := 
%   \left\{ 
%     \begin{array}{ccc} 
%       \dropn{\quotep{Q}} & & x \nameeq \quotep{P} \\
%       \dropn{x} & & otherwise \\
%     \end{array}
%   \right. 
  (\dropn{x})  \psubstp{Q}{P}       
  := 
  \left\{ 
    \begin{array}{ccc} 
      Q & & x \nameeq \quotep{P} \\
      \dropn{x} & & otherwise \\
    \end{array}
  \right.
\end{mathpar}
 

where

\begin{eqnarray}
  (x)\id{\{} \lpquote Q \rpquote / \lpquote P \rpquote \id{\}}            = 
  \left\{ 
    \begin{array}{ccc}
      \lpquote Q \rpquote & & x \nameeq \lpquote P \rpquote \\
      x & & otherwise \\
    \end{array}
  \right. \nonumber
\end{eqnarray}

and $z$ is chosen distinct from $\quotep{P}$, $\quotep{Q}$, the free
names in $Q$, and all the names in $R$. Our $\alpha$-equivalence will
be built in the standard way from this substitution.

\begin{remark}\label{rem:no_self_referential_names}
  One consequence of these definitions is that $\forall P. \quotep{P}
  \not\in \freenames{P}$.
\end{remark}

\subsection{ Dynamic quote: an example }

Anticipating something of what's to come, consider applying the
substitution, $\widehat{\id{\{}u / z \id{\}}}$, to the following pair
of processes, $\lift{w}{y!(z)}$ and $w[ \lpquote y!(z) \rpquote ]$.

\begin{eqnarray}
	\lift{w}{y!(z)}\widehat{\id{\{}u / z \id{\}}}
		& = &
		\lift{w}{y!(u)} \nonumber\\
	w[ \lpquote y!(z) \rpquote ] \widehat{ \id{\{}u / z \id{\}} }
		& = &
		w[ \lpquote y!(z) \rpquote ] \nonumber
\end{eqnarray}

Because the body of the process between quotes is impervious to
substitution, we get radically different answers. In fact, by
examining the first process in an input context,
e.g. $x?(z).\lift{w}{y!(z)}$, we see that the process under the lift
operator may be shaped by prefixed inputs binding a name inside it. In
this sense, the lift operator will be seen as a way to dynamically
construct processes before reifying them as names.

Finally equipped with these standard features we can present the
dynamics of the calculus.

\subsubsection{Operational semantics} 

Finally, we introduce the computational dynamics. What marks these
algebras as distinct from other more traditionally studied algebraic
structures, e.g. vector spaces or polynomial rings, is the manner in
which dynamics is captured. In traditional structures, dynamics is typically
expressed through morphisms between such structures, as in linear maps
between vector spaces or morphisms between rings. In algebras
associated with the semantics of computation, the dynamics is
expressed as part of the algebraic structure itself, through a
reduction reduction relation typically denoted by $\red$. Below, we
give a recursive presentation of this relation for the calculus used
in the encoding.

$\red \subseteq \pi \times \pi$
$\red : \pi \to \mathcal{P}(\pi)$

\begin{mathpar}
  \inferrule* [lab=Comm] { \textsf{match}( x_{src}, x_{trgt} ) } { x_{trgt}?(y)P \; | \; x_{src}!\langle {Q} \rangle \red P\{\quotep{Q}/y}\} }
  \and \\
  \inferrule* [lab=Par] {{P} \red {P}'} {{{P} | {Q}} \red {{P}' | {Q}}}
  \and
  \inferrule* [lab=Equiv]{{{P} \scong {P}'} \andalso {{P}' \red {Q}'} \andalso {{Q}' \scong {Q}}}{{P} \red {Q}}
\end{mathpar}

\begin{eqnarray*}
  match_{\equiv} (\quotep{P},\quotep{Q}) & := & P \equiv Q \\
  match_{\dagger}(\quotep{P},\quotep{Q}) & := & \forall R. P|Q \red^{*} R => R \red^{*} 0 \\
  match_{K}(\quotep{P},\quotep{Q}) & := & K \mbox{ for some context } K
\end{eqnarray*}

$u?(x)P | u!\langle Q \rangle \red P\{\quotep{Q}/x\}$

%We write $\wred$ for $\red^*$, and $P\red$ if $\exists Q $ such that $ P \red Q$.
We write $P\red$ if $\exists Q $ such that $ P \red Q$ and $P\not\red$, otherwise.

\section{Replication}

As mentioned before, it is known that replication (and hence
recursion) can be implemented in a higher-order process algebra
\cite{SangiorgiWalker}. As our first example of calculation with the
machinery thus far presented we give the construction explicitly in
the {\rhoc}.

\begin{eqnarray}
	D_{x} & := & \prefix{x}{y}{(\binpar{\outputp{x}{y}}{@{y}})} \nonumber\\
	\bangp_{x}{P} & := & \binpar{{x}!\langle{\binpar{D_{x}}{P}}\rangle}{D_{x}} \nonumber
\end{eqnarray}

\begin{eqnarray}
	\bangp_{x}{P} & & \nonumber\\
	=
	& {x}!\langle{(\prefix{x}{y}{(\outputp{x}{y} | @{y})) | P}}\rangle 
	      | \prefix{x}{y}{(\outputp{x}{y} | @{y})} & \nonumber\\
	\red
	& (\outputp{x}{y} | @{y})\substn{\quotep{(\prefix{x}{y}{(@{y} | \outputp{x}{y})) | P}}}{y} & \nonumber\\
	=
	& \outputp{x}{\quotep{(\prefix{x}{y}{(\outputp{x}{y} | @{y})) | P}}}
	  | {(\prefix{x}{y}{(\outputp{x}{y} | @{y})) | P}} & \nonumber\\
	\red
	& \ldots & \nonumber\\
	\red^*
	& P | P | \ldots & \nonumber
\end{eqnarray}

Of course, this encoding, as an implementation, runs away, unfolding
$\bangp{P}$ eagerly. A lazier and more implementable replication
operator, restricted to input-guarded processes, may be obtained as follows.

\begin{eqnarray}
\bangp{\prefix{u}{v}{P}} 
	:= 
	\binpar{\lift{x}{\prefix{u}{v}{(\binpar{D(x)}{P})}}}{D(x)} \nonumber
\end{eqnarray}

\begin{remark}
  Note that the lazier definition still does not deal with summation
  or mixed summation (i.e. sums over input and output). The reader is
  invited to construct definitions of replication that deal with these
  features. 

  Further, the definitions are parameterized in a name, $x$. Can you,
  gentle reader, make a definition that eliminates this parameter and
  guarantees no accidental interaction between the replication
  machinery and the process being replicated -- i.e. no accidental
  sharing of names used by the process to get its work done and the
  name(s) used by the replication to effect copying. This latter
  revision of the definition of replication is crucial to obtaining
  the expected identity $!!P \sim !P$.
\end{remark}

\begin{remark}\label{rem:paradoxical_combinator}
  The reader familiar with the lambda calculus will have noticed the
  similarity between $D$ and the paradoxical combinator.

  [Ed. note: the existence of this seems to suggest we have to be more
  restrictive on the set of processes and names we admit if we are to
  support no-cloning.]
\end{remark}

\subsubsection{Bisimulation}

The computational dynamics gives rise to another kind of equivalence,
the equivalence of computational behavior. As previously mentioned
this is typically captured \emph{via} some form of bisimulation.

% The notion we use in this paper is weak barbed bisimulation
% \cite{milner91polyadicpi}.

The notion we use in this paper is derived from weak barbed
bisimulation \cite{milner91polyadicpi}. 

\begin{definition}
An \emph{observation relation}, $\downarrow_{\mathcal N}$, over a set
of names, $\mathcal N$, is the smallest relation satisfying the rules
below.

\infrule[Out-barb]{y \in {\mathcal N}, \; x \nameeq y}
		  {\outputp{x}{v} \downarrow_{\mathcal N} x}
\infrule[Par-barb]{\mbox{$P\downarrow_{\mathcal N} x$ or $Q\downarrow_{\mathcal N} x$}}
		  {\binpar{P}{Q} \downarrow_{\mathcal N} x}

We write $P \Downarrow_{\mathcal N} x$ if there is $Q$ such that 
$P \wred Q$ and $Q \downarrow_{\mathcal N} x$.
\end{definition}

\begin{definition}
%\label{def.bbisim}
An  ${\mathcal N}$-\emph{barbed bisimulation} over a set of names, ${\mathcal N}$, is a symmetric binary relation 
${\mathcal S}_{\mathcal N}$ between agents such that $P\rel{S}_{\mathcal N}Q$ implies:
\begin{enumerate}
\item If $P \red P'$ then $Q \wred Q'$ and $P'\rel{S}_{\mathcal N} Q'$.
\item If $P\downarrow_{\mathcal N} x$, then $Q\Downarrow_{\mathcal N} x$.
\end{enumerate}
$P$ is ${\mathcal N}$-barbed bisimilar to $Q$, written
$P \wbbisim_{\mathcal N} Q$, if $P \rel{S}_{\mathcal N} Q$ for some ${\mathcal N}$-barbed bisimulation ${\mathcal S}_{\mathcal N}$.
\end{definition}

$\mathcal{R} \subseteq \pi \times \pi$

$P \mathcal{R} Q => \forall P'. P \red P' \Rightarrow \exists Q'. Q \red Q', P' \mathcal{R} Q'$

$P \vdash x \Rightarrow Q \vdash x$

\begin{mathpar}
  \inferrule*[lab=Out-barb]{x \nameeq y}{{y}!\langle{Q}\rangle \vdash x}
  \and
  \inferrule*[lab=Par-barb]{\mbox{$P\vdash x$ or $Q\vdash x$}}{\binpar{P}{Q} \vdash x}
\end{mathpar}

\subsubsection{Contexts}

One of the principle advantages of computational calculi like the
$\pi$-calculus is a well-defined notion of context,
contextual-equivalence and a correlation between
contextual-equivalence and notions of bisimulation. The notion of
context allows the decomposition of a process into (sub-)process and
its syntactic environment, its context. Thus, a context may be
thought of as a process with a ``hole'' (written $\Box$) in it. The
application of a context $M$ to a process $P$, written $M[P]$, is
tantamount to filling the hole in $M$ with $P$. In this paper we do
not need the full weight of this theory, but do make use of the notion
of context in the proof the main theorem. 

\begin{mathpar}
  \inferrule* [lab=summation] {} {{M_{M},M_{N}} \bc \Box \;|\; x.M_{A} \;|\; M_{M}+M_{N}}
  \and
  \inferrule* [lab=agent] {} {{M_{A}} \bc (\vec{x})M_{P} \;| \; \clift{P_0,\ldots,M_{P},\ldots,P_N}}
  \and \\
  \inferrule* [lab=process] {} {{M_{P}} \bc M_{N} \;| \;P|M_{P} }
\end{mathpar} 

\begin{mathpar}
  \inferrule* [lab=sychronization] {} {M_{N} \bc \Box \;|\; x?M_{F} \;|\; x!M_{C}}
  \and
  \inferrule* [lab=abstraction] {} {{M_{F}} \bc (x)M_{P} }
  \and
  \inferrule* [lab=concretion] {} {{M_{C}} \bc \langle M_{P} \rangle }
  \and \\
  \inferrule* [lab=process] {} {{M_{P}} \bc M_{N} \;| \;P|M_{P} }
\end{mathpar}

\begin{definition}[contextual application] Given a context $M$, and
  process $P$, we define the \emph{contextual application}, $M[P] :=
  M\{P/\Box\}$. That is, the contextual application of M to P is the
  substitution of $P$ for $\Box$ in $M$.
\end{definition}

$\meaningof{-} : L \to \mathcal{P}(\pi)$

\begin{mathpar}
  \inferrule* [lab=collection] {} {\meaningof{true} = \pi, \and \meaningof{~E} = \pi \setminus \meaningof{E}, \and \meaningof{E_{1} \& E_{2}} = \meaningof{E_{1}} \cap \meaningof{E_{2}}}
\end{mathpar}

\begin{mathpar}
  \inferrule* [lab=structure] {} {\meaningof{0} = \{ P \in \pi | P \equiv 0 \}, \and \\ \meaningof{E_1 | E_2} = \{ P \in \pi | P \equiv P_{1} | P_{2}, P_{1} \in \meaningof{E_{1}}, P_{2} \in \meaningof{E_2}\} }
\end{mathpar}

\begin{mathpar}
 \inferrule* [lab=behavior] {} {\meaningof{\langle a?b \rangle E} = \{ P \in \pi | P \equiv Q | u?(y)P', \\ \and \\\\ \and \\ \;\;\; u \in \meaningof{a}, \forall z.P'\{z/y\} \in \meaningof{E\{z/b\}}\}, \and \\ \meaningof{a!E} = \{ P \in \pi | P \equiv Q | x!\langle P' \rangle, x \in \meaningof{a} P' \in \meaningof{E}\} }
\end{mathpar}

\begin{mathpar}
 \inferrule* [lab=nominal] {} {\meaningof{\quotep{E}} = \{ \quotep{P} \in \quotep{\pi} | P \in \meaningof{E} \}, \and \meaningof{\quotep{P}} = \{ \quotep{Q} \in \quotep{\pi} | P \equiv Q \} \and \\ \meaningof{@\quotep{E}} = \{ P \in \pi | P \equiv @x, x \in \meaningof{E} \}}
\end{mathpar}

\begin{eqnarray*}
  \\
  \meaningof{-} : TS \to ST
\end{eqnarray*}

\begin{eqnarray*}
  \\
  L : TS \to ST
\end{eqnarray*}

\begin{eqnarray*}
  \\
  P \models E \iff P \in \meaningof{E}
\end{eqnarray*}

\begin{eqnarray*}
  P \approx_{L} Q \iff \forall E \in L. P \models E \iff Q \models E
\end{eqnarray*}

\begin{eqnarray*}
  P \approx_{K} Q
\end{eqnarray*}

\begin{eqnarray*}
  P \approx Q
\end{eqnarray*}

$\approx_{K} = \approx = \approx_{L}$

\subsubsection{Contextual duality}

Note that contexts extend the quotation operation to a family of
operations from processes to names. Given a context, $M$, we can
define a \emph{nominal context}, $\quotep{M}$ by $\quotep{M}[P] :=
\quotep{M[P]}$. To foreshadow what is to come we observe that these
operations enjoy a duality with processes very much like the duality
between vectors and maps from vectors to scalars.

Further, because the calculus is essentially higher-order, we have a
correspondence between contexts and processes. More specifically,
given a name $x$ and a context $M$ we can construct $M^{*}_{x}$ such
that 

\begin{mathpar}
  M^{*}_{x} | \lift{x}{P} \red M[P]
\end{mathpar}

namely,

\begin{mathpar}
  M^{*}_{x} := x?(u).M[\dropn{u}]
\end{mathpar}

The dependence of $M^{*}_{x}$ on a name makes it an abstraction, 

\begin{mathpar}
  M^{*} := (x)x?(u).M[\dropn{u}]
\end{mathpar}

\subsection{Additional notation}

It will sometimes be convenient to denote the process a name
quotes. We already have the notation $x = \quotep{P}$, but it will be
convenient to introduce an alternate notation, $\procn{x}$, when we
want to emphasize the connection to the use of the name. Note that, by
virtue of name equivalence, $\quotep{\procn{x}} \nameeq x$; so, the
notation is consistent with previous definitions.

Further, because names have structure it is possible to effect
substitutions on the basis of that structure. This means we need to
upgrade our notation for substitutions, which we accomplish by
adapting comprehension notation. Thus,

\begin{mathpar}
  P\{ y / x : x \in S \}
\end{mathpar}

is interpreted to mean the process derived from P by replacing (in a
capture-avoiding manner) each occurrence of $x$ in $S$ by $y$. For example,

\begin{mathpar}
  P\{ \quotep{\procn{x}|\procn{x}} / x : x \in \freenames{P} \}
\end{mathpar}

will replace each (occurrence) of a free name $x$ in $P$ by
$\quotep{\procn{x}|\procn{x}}$.

Also, we will avail ourselves of the notation $x^{L}$ and $x^{R}$ to
denote injections of a name into disjoint copies of the name
space. There are numerous ways to accomplish this. One example can be
found in \cite{MeredithR05}. This notation overloads to vectors of
names: $\vec{x}^{\pi} := (x_{i}^{\pi} \; : \; 0 \leq i < |\vec{x}| )$ where $\pi \in \{L,R\}$.

We also use $P^{\Box} := P|\Box$.

In \cite{MeredithR05} an interpretation of the new operator is
given. It turns out that there are several possible interpretations
all enjoying the requisite algebraic properties of the operator (see
\cite{milner91polyadicpi}). We will therefore make liberal use of
$(\nu\; \vec{x})P$.

% subsection the_syntax_and_semantics_of_the_notation_system (end)   

\input{qm2pi.qmops} 

\input{qm2pi.sterngerlach} 

\input{qm2pi.metric} 

% section concurrent_process_calculi (end)

%\input{qm2pi.proofsketch}

% section proof sketch (end)

%\input{qm2pi.slviaknots} 

% section spatial logic via knots (end)

\input{qm2pi.conclusion}

% section conclusion (end)

%\input{qm2pi.dtcodes} 

% section wiring algorithm (end)

\input{qm2pi.ack} 

% section acknowledgments (end)

\newpage


\bibliographystyle{plain}   
\bibliography{../../biblios/main.bib}

\input{qm2pi.rhodetails}

\end{document}

 

% section acknowledgments (end)

\newpage


\bibliographystyle{plain}   
\bibliography{../../biblios/main.bib}

\documentclass[12pt]{llncs}
%\documentclass{jktr}

\usepackage[pdftex]{hyperref}                   
\usepackage {listings}
\usepackage {mathpartir}
\usepackage{bcprules}
%\usepackage{listings}
                       
\usepackage{graphicx} 
%\usepackage[margins=2.5cm,nohead,nofoot]{geometry}
%\usepackage{geometry}
\usepackage{amsfonts}
\usepackage{amstext}
\usepackage{latexsym}
\usepackage{amssymb}
\usepackage{color}


%\include{myPreamble}
\include{qm2pi.local} 

%\ifpdf
%\usepackage[pdftex]{graphicx}
%\else
%\usepackage{graphicx}
%\fi

 % \ifpdf
%  \usepackage{pdfsync}
%  \if


%\title{Brief Article}
%\author{David F. Snyder}
%\author{L.G. Meredith}

%\address{Dept. of Math., Texas State University--San Marcos, San Marcos, TX 78666}
       
\pagestyle{empty}


\begin{document}

\lstset{language=[Objective]Caml,frame=shadowbox}

\input{qm2pi.front}

% section front matter (end)

\input{qm2pi.intro} 
 
% section introduction (end)

% \input{qm2pi.knotations} 

% section notation (end)

\input{qm2pi.process.calculi} 

% section concurrent_process_calculi_and_spatial_logics_ (end)
    
%\input{qm2pi.knots2pi} 

%\input{qm2pi.trefoil} 

%\input{qm2pi.mainthm} 

% subsection basic_interpretation (end)

%\input{qm2pi.rho.presentation} 
\subsection{The syntax and semantics of the notation system}\label{sub:the_syntax_and_semantics_of_the_notation_system} % (fold)

We now summarize a technical presentation of the calculus that
embodies our theory of dynamics. The typical presentation of such a
calculus follows the style of giving generators and relations on
them. The grammar, below, describing term constructors, freely
generates the set of processes, $\Proc$. This set is then quotiented
by a relation known as structural congruence and it is over this set
that the notion of dynamics is expressed. This presentation is
essentially that of \cite{MeredithR05} with the addition of
polyadicity and summation. For readability we have relegated some of
the technical subtleties to an appendix.

\subsubsection{Process grammar}\label{subsub:process_grammar}

\begin{mathpar}
  \inferrule* [lab=synchronization] {} {{M} \bc \pzero \;|\; x?F \;|\; x!C }
  \and
  \inferrule* [lab=abstraction] {} {{F} \bc (x)P}
  \and
  \inferrule* [lab=concretion] {} {{C} \bc \langle Q \rangle}
  \and
  \inferrule* [lab=process] {} {{P,Q} \bc M \;| \;P|Q \;|\; @{x}}
  \and
  \inferrule* [lab=name] {} {{x} \bc \quotep{P}}
\end{mathpar} 

Note that $\vec{x}$ (resp. $\vec{P}$) denotes a vector of names
(resp. processes) of length $|\vec{x}|$ (resp. $|\vec{P}|$). We adopt
the following useful abbreviations.

\begin{mathpar}
   x?(\vec{y}).P := x.(\vec{y})P \and  x\clift{\vec{P}} := x.\clift{\vec{P}}
   \and x!(y) := \lift{x}{\dropn{y}}
   \and \Pi_{i=0}^{n-1}P_i := P_0 | \ldots | P_{n-1}
\end{mathpar}

\subsubsection{Structural congruence}

\paragraph{Free and bound names and alpha-equivalence.} At the
core of structural equivalence is alpha-equivalence which identifies
process that are the same up to a change of variable. Formally, we
recognize the distinction between free and bound names. The free names
of a process, $\freenames{P}$, may be calculated recursively as
follows:

\begin{mathpar}
\freenames{\pzero} := \emptyset
  \and \\
  \freenames{x?(y).P} := \{ x \} \cup (\freenames{P} \setminus \{ y \})
  \and 
  \freenames{x!\langle P \rangle} := \{ x \} \cup \{ P \} 
  \and \\
  \freenames{P|Q} := \freenames{P} \cup \freenames{Q}
  \and \\
  \freenames{@{x}} := \{ x \}
\end{mathpar}

$\pi$
$\quotep{\pi}$

$\freenames{-} : \pi \to \mathcal{P}(\quotep{\pi})$

\begin{eqnarray*}
  \freenames{\pzero} & := & \emptyset \\
  \freenames{x?(y).P} & := & \{ x \} \cup (\freenames{P} \setminus \{ y \}) \\
  \freenames{x!\langle P \rangle} & := & \{ x \} \cup \{ P \} \\
  \freenames{P|Q} & := & \freenames{P} \cup \freenames{Q} \\
  \freenames{\dropn{x}} & := & \{ x \}
\end{eqnarray*}

The bound names of a process, $\boundnames{P}$, are those names occurring in $P$
that are not free. For example, in $x?(y).0$, the name $x$ is free, while $y$ is bound.

\begin{mathpar}
  \inferrule* [lab=monoidal-laws] {} { P|Q \equiv Q|P \and P|0 \equiv P \and P|(Q|R) \equiv (P|Q)|R }
\end{mathpar}

\begin{mathpar}
  \inferrule* [lab=alpha-equivalence] {} { (x)P \equiv (y)P\{y/x\} \and y \not\in \freenames{P} }
\end{mathpar}

\begin{definition}
Then two processes, $P,Q$, are alpha-equivalent if $P = Q\{\vec{y}/\vec{x}\}$ for
some $\vec{x} \in \boundnames{Q},\vec{y} \in \boundnames{P}$, where $Q\{\vec{y}/\vec{x}\}$
denotes the capture-avoiding substitution of $\vec{y}$ for $\vec{x}$ in $Q$.
\end{definition}

\begin{definition}
  The {\em structural congruence} \cite{SangiorgiWalker} , $\equiv$,
  between processes is the least congruence containing
  alpha-equivalence, satisfying the abelian monoid laws
  (associativity, commutativity and $\pzero$ as identity) for parallel
  composition $|$ and for summation $+$.
\end{definition}

\subsection{Name equivalence}

We take name equivalence, written $\nameeq$, to be the smallest
equivalence relation generated by the following rules.

\begin{mathpar}
\inferrule*[lab=Quote-drop]
{ }
{ \quotep{@{x}} \nameeq x }

\inferrule*[lab=Struct-equiv]
{ P \scong Q }
{ \quotep{P} \nameeq \quotep{Q} }
\end{mathpar}

The astute reader will have noticed that the mutual recursion of names
and processes imposes a mutual recursion on alpha-equivalence and
structural equivalence via name-equivalence. Fortunately, all of this
works out pleasantly and we may calculate in the natural way, free of
concern. The reader interested in the details is referred to the
appendix \ref{appendix:rho_details}.

\subsection{Substitution}

We use $\Proc$ for the set of processes, $\QProc$ for the set of
names, and $\id{\{}\vec{y} / \vec{x} \id{\}}$ to denote partial maps,
$s : \QProc \rightarrow \QProc$. A map, $s$ lifts, uniquely, to a map
on process terms, $\widehat{s} : \Proc \rightarrow \Proc$ by the
following equations.

\begin{mathpar}
  (0) \psubstp{Q}{P} := 0 \\
  (R \juxtap S) \psubstp{Q}{P}
  :=    
  (R)\psubstp{Q}{P} \juxtap (S) \psubstp{Q}{P} \\
  (x?(y).R) \psubstp{Q}{P}    
  :=    
  (x)\substp{Q}{P} (z)\concat( (R \psubstn{z}{y}) \psubstp{Q}{P} ) \\
  (\lift{x}{R}) \psubstp{Q}{P}  
  :=
  \lift{(x)\substp{Q}{P}}{ R \psubstp{Q}{P} } \\
%   (\dropn{x})  \psubstp{Q}{P}       
%   := 
%   \left\{ 
%     \begin{array}{ccc} 
%       \dropn{\quotep{Q}} & & x \nameeq \quotep{P} \\
%       \dropn{x} & & otherwise \\
%     \end{array}
%   \right. 
  (\dropn{x})  \psubstp{Q}{P}       
  := 
  \left\{ 
    \begin{array}{ccc} 
      Q & & x \nameeq \quotep{P} \\
      \dropn{x} & & otherwise \\
    \end{array}
  \right.
\end{mathpar}
 

where

\begin{eqnarray}
  (x)\id{\{} \lpquote Q \rpquote / \lpquote P \rpquote \id{\}}            = 
  \left\{ 
    \begin{array}{ccc}
      \lpquote Q \rpquote & & x \nameeq \lpquote P \rpquote \\
      x & & otherwise \\
    \end{array}
  \right. \nonumber
\end{eqnarray}

and $z$ is chosen distinct from $\quotep{P}$, $\quotep{Q}$, the free
names in $Q$, and all the names in $R$. Our $\alpha$-equivalence will
be built in the standard way from this substitution.

\begin{remark}\label{rem:no_self_referential_names}
  One consequence of these definitions is that $\forall P. \quotep{P}
  \not\in \freenames{P}$.
\end{remark}

\subsection{ Dynamic quote: an example }

Anticipating something of what's to come, consider applying the
substitution, $\widehat{\id{\{}u / z \id{\}}}$, to the following pair
of processes, $\lift{w}{y!(z)}$ and $w[ \lpquote y!(z) \rpquote ]$.

\begin{eqnarray}
	\lift{w}{y!(z)}\widehat{\id{\{}u / z \id{\}}}
		& = &
		\lift{w}{y!(u)} \nonumber\\
	w[ \lpquote y!(z) \rpquote ] \widehat{ \id{\{}u / z \id{\}} }
		& = &
		w[ \lpquote y!(z) \rpquote ] \nonumber
\end{eqnarray}

Because the body of the process between quotes is impervious to
substitution, we get radically different answers. In fact, by
examining the first process in an input context,
e.g. $x?(z).\lift{w}{y!(z)}$, we see that the process under the lift
operator may be shaped by prefixed inputs binding a name inside it. In
this sense, the lift operator will be seen as a way to dynamically
construct processes before reifying them as names.

Finally equipped with these standard features we can present the
dynamics of the calculus.

\subsubsection{Operational semantics} 

Finally, we introduce the computational dynamics. What marks these
algebras as distinct from other more traditionally studied algebraic
structures, e.g. vector spaces or polynomial rings, is the manner in
which dynamics is captured. In traditional structures, dynamics is typically
expressed through morphisms between such structures, as in linear maps
between vector spaces or morphisms between rings. In algebras
associated with the semantics of computation, the dynamics is
expressed as part of the algebraic structure itself, through a
reduction reduction relation typically denoted by $\red$. Below, we
give a recursive presentation of this relation for the calculus used
in the encoding.

$\red \subseteq \pi \times \pi$
$\red : \pi \to \mathcal{P}(\pi)$

\begin{mathpar}
  \inferrule* [lab=Comm] { \textsf{match}( x_{src}, x_{trgt} ) } { x_{trgt}?(y)P \; | \; x_{src}!\langle {Q} \rangle \red P\{\quotep{Q}/y}\} }
  \and \\
  \inferrule* [lab=Par] {{P} \red {P}'} {{{P} | {Q}} \red {{P}' | {Q}}}
  \and
  \inferrule* [lab=Equiv]{{{P} \scong {P}'} \andalso {{P}' \red {Q}'} \andalso {{Q}' \scong {Q}}}{{P} \red {Q}}
\end{mathpar}

\begin{eqnarray*}
  match_{\equiv} (\quotep{P},\quotep{Q}) & := & P \equiv Q \\
  match_{\dagger}(\quotep{P},\quotep{Q}) & := & \forall R. P|Q \red^{*} R => R \red^{*} 0 \\
  match_{K}(\quotep{P},\quotep{Q}) & := & K \mbox{ for some context } K
\end{eqnarray*}

$u?(x)P | u!\langle Q \rangle \red P\{\quotep{Q}/x\}$

%We write $\wred$ for $\red^*$, and $P\red$ if $\exists Q $ such that $ P \red Q$.
We write $P\red$ if $\exists Q $ such that $ P \red Q$ and $P\not\red$, otherwise.

\section{Replication}

As mentioned before, it is known that replication (and hence
recursion) can be implemented in a higher-order process algebra
\cite{SangiorgiWalker}. As our first example of calculation with the
machinery thus far presented we give the construction explicitly in
the {\rhoc}.

\begin{eqnarray}
	D_{x} & := & \prefix{x}{y}{(\binpar{\outputp{x}{y}}{@{y}})} \nonumber\\
	\bangp_{x}{P} & := & \binpar{{x}!\langle{\binpar{D_{x}}{P}}\rangle}{D_{x}} \nonumber
\end{eqnarray}

\begin{eqnarray}
	\bangp_{x}{P} & & \nonumber\\
	=
	& {x}!\langle{(\prefix{x}{y}{(\outputp{x}{y} | @{y})) | P}}\rangle 
	      | \prefix{x}{y}{(\outputp{x}{y} | @{y})} & \nonumber\\
	\red
	& (\outputp{x}{y} | @{y})\substn{\quotep{(\prefix{x}{y}{(@{y} | \outputp{x}{y})) | P}}}{y} & \nonumber\\
	=
	& \outputp{x}{\quotep{(\prefix{x}{y}{(\outputp{x}{y} | @{y})) | P}}}
	  | {(\prefix{x}{y}{(\outputp{x}{y} | @{y})) | P}} & \nonumber\\
	\red
	& \ldots & \nonumber\\
	\red^*
	& P | P | \ldots & \nonumber
\end{eqnarray}

Of course, this encoding, as an implementation, runs away, unfolding
$\bangp{P}$ eagerly. A lazier and more implementable replication
operator, restricted to input-guarded processes, may be obtained as follows.

\begin{eqnarray}
\bangp{\prefix{u}{v}{P}} 
	:= 
	\binpar{\lift{x}{\prefix{u}{v}{(\binpar{D(x)}{P})}}}{D(x)} \nonumber
\end{eqnarray}

\begin{remark}
  Note that the lazier definition still does not deal with summation
  or mixed summation (i.e. sums over input and output). The reader is
  invited to construct definitions of replication that deal with these
  features. 

  Further, the definitions are parameterized in a name, $x$. Can you,
  gentle reader, make a definition that eliminates this parameter and
  guarantees no accidental interaction between the replication
  machinery and the process being replicated -- i.e. no accidental
  sharing of names used by the process to get its work done and the
  name(s) used by the replication to effect copying. This latter
  revision of the definition of replication is crucial to obtaining
  the expected identity $!!P \sim !P$.
\end{remark}

\begin{remark}\label{rem:paradoxical_combinator}
  The reader familiar with the lambda calculus will have noticed the
  similarity between $D$ and the paradoxical combinator.

  [Ed. note: the existence of this seems to suggest we have to be more
  restrictive on the set of processes and names we admit if we are to
  support no-cloning.]
\end{remark}

\subsubsection{Bisimulation}

The computational dynamics gives rise to another kind of equivalence,
the equivalence of computational behavior. As previously mentioned
this is typically captured \emph{via} some form of bisimulation.

% The notion we use in this paper is weak barbed bisimulation
% \cite{milner91polyadicpi}.

The notion we use in this paper is derived from weak barbed
bisimulation \cite{milner91polyadicpi}. 

\begin{definition}
An \emph{observation relation}, $\downarrow_{\mathcal N}$, over a set
of names, $\mathcal N$, is the smallest relation satisfying the rules
below.

\infrule[Out-barb]{y \in {\mathcal N}, \; x \nameeq y}
		  {\outputp{x}{v} \downarrow_{\mathcal N} x}
\infrule[Par-barb]{\mbox{$P\downarrow_{\mathcal N} x$ or $Q\downarrow_{\mathcal N} x$}}
		  {\binpar{P}{Q} \downarrow_{\mathcal N} x}

We write $P \Downarrow_{\mathcal N} x$ if there is $Q$ such that 
$P \wred Q$ and $Q \downarrow_{\mathcal N} x$.
\end{definition}

\begin{definition}
%\label{def.bbisim}
An  ${\mathcal N}$-\emph{barbed bisimulation} over a set of names, ${\mathcal N}$, is a symmetric binary relation 
${\mathcal S}_{\mathcal N}$ between agents such that $P\rel{S}_{\mathcal N}Q$ implies:
\begin{enumerate}
\item If $P \red P'$ then $Q \wred Q'$ and $P'\rel{S}_{\mathcal N} Q'$.
\item If $P\downarrow_{\mathcal N} x$, then $Q\Downarrow_{\mathcal N} x$.
\end{enumerate}
$P$ is ${\mathcal N}$-barbed bisimilar to $Q$, written
$P \wbbisim_{\mathcal N} Q$, if $P \rel{S}_{\mathcal N} Q$ for some ${\mathcal N}$-barbed bisimulation ${\mathcal S}_{\mathcal N}$.
\end{definition}

$\mathcal{R} \subseteq \pi \times \pi$

$P \mathcal{R} Q => \forall P'. P \red P' \Rightarrow \exists Q'. Q \red Q', P' \mathcal{R} Q'$

$P \vdash x \Rightarrow Q \vdash x$

\begin{mathpar}
  \inferrule*[lab=Out-barb]{x \nameeq y}{{y}!\langle{Q}\rangle \vdash x}
  \and
  \inferrule*[lab=Par-barb]{\mbox{$P\vdash x$ or $Q\vdash x$}}{\binpar{P}{Q} \vdash x}
\end{mathpar}

\subsubsection{Contexts}

One of the principle advantages of computational calculi like the
$\pi$-calculus is a well-defined notion of context,
contextual-equivalence and a correlation between
contextual-equivalence and notions of bisimulation. The notion of
context allows the decomposition of a process into (sub-)process and
its syntactic environment, its context. Thus, a context may be
thought of as a process with a ``hole'' (written $\Box$) in it. The
application of a context $M$ to a process $P$, written $M[P]$, is
tantamount to filling the hole in $M$ with $P$. In this paper we do
not need the full weight of this theory, but do make use of the notion
of context in the proof the main theorem. 

\begin{mathpar}
  \inferrule* [lab=summation] {} {{M_{M},M_{N}} \bc \Box \;|\; x.M_{A} \;|\; M_{M}+M_{N}}
  \and
  \inferrule* [lab=agent] {} {{M_{A}} \bc (\vec{x})M_{P} \;| \; \clift{P_0,\ldots,M_{P},\ldots,P_N}}
  \and \\
  \inferrule* [lab=process] {} {{M_{P}} \bc M_{N} \;| \;P|M_{P} }
\end{mathpar} 

\begin{mathpar}
  \inferrule* [lab=sychronization] {} {M_{N} \bc \Box \;|\; x?M_{F} \;|\; x!M_{C}}
  \and
  \inferrule* [lab=abstraction] {} {{M_{F}} \bc (x)M_{P} }
  \and
  \inferrule* [lab=concretion] {} {{M_{C}} \bc \langle M_{P} \rangle }
  \and \\
  \inferrule* [lab=process] {} {{M_{P}} \bc M_{N} \;| \;P|M_{P} }
\end{mathpar}

\begin{definition}[contextual application] Given a context $M$, and
  process $P$, we define the \emph{contextual application}, $M[P] :=
  M\{P/\Box\}$. That is, the contextual application of M to P is the
  substitution of $P$ for $\Box$ in $M$.
\end{definition}

$\meaningof{-} : L \to \mathcal{P}(\pi)$

\begin{mathpar}
  \inferrule* [lab=collection] {} {\meaningof{true} = \pi, \and \meaningof{~E} = \pi \setminus \meaningof{E}, \and \meaningof{E_{1} \& E_{2}} = \meaningof{E_{1}} \cap \meaningof{E_{2}}}
\end{mathpar}

\begin{mathpar}
  \inferrule* [lab=structure] {} {\meaningof{0} = \{ P \in \pi | P \equiv 0 \}, \and \\ \meaningof{E_1 | E_2} = \{ P \in \pi | P \equiv P_{1} | P_{2}, P_{1} \in \meaningof{E_{1}}, P_{2} \in \meaningof{E_2}\} }
\end{mathpar}

\begin{mathpar}
 \inferrule* [lab=behavior] {} {\meaningof{\langle a?b \rangle E} = \{ P \in \pi | P \equiv Q | u?(y)P', \\ \and \\\\ \and \\ \;\;\; u \in \meaningof{a}, \forall z.P'\{z/y\} \in \meaningof{E\{z/b\}}\}, \and \\ \meaningof{a!E} = \{ P \in \pi | P \equiv Q | x!\langle P' \rangle, x \in \meaningof{a} P' \in \meaningof{E}\} }
\end{mathpar}

\begin{mathpar}
 \inferrule* [lab=nominal] {} {\meaningof{\quotep{E}} = \{ \quotep{P} \in \quotep{\pi} | P \in \meaningof{E} \}, \and \meaningof{\quotep{P}} = \{ \quotep{Q} \in \quotep{\pi} | P \equiv Q \} \and \\ \meaningof{@\quotep{E}} = \{ P \in \pi | P \equiv @x, x \in \meaningof{E} \}}
\end{mathpar}

\begin{eqnarray*}
  \\
  \meaningof{-} : TS \to ST
\end{eqnarray*}

\begin{eqnarray*}
  \\
  L : TS \to ST
\end{eqnarray*}

\begin{eqnarray*}
  \\
  P \models E \iff P \in \meaningof{E}
\end{eqnarray*}

\begin{eqnarray*}
  P \approx_{L} Q \iff \forall E \in L. P \models E \iff Q \models E
\end{eqnarray*}

\begin{eqnarray*}
  P \approx_{K} Q
\end{eqnarray*}

\begin{eqnarray*}
  P \approx Q
\end{eqnarray*}

$\approx_{K} = \approx = \approx_{L}$

\subsubsection{Contextual duality}

Note that contexts extend the quotation operation to a family of
operations from processes to names. Given a context, $M$, we can
define a \emph{nominal context}, $\quotep{M}$ by $\quotep{M}[P] :=
\quotep{M[P]}$. To foreshadow what is to come we observe that these
operations enjoy a duality with processes very much like the duality
between vectors and maps from vectors to scalars.

Further, because the calculus is essentially higher-order, we have a
correspondence between contexts and processes. More specifically,
given a name $x$ and a context $M$ we can construct $M^{*}_{x}$ such
that 

\begin{mathpar}
  M^{*}_{x} | \lift{x}{P} \red M[P]
\end{mathpar}

namely,

\begin{mathpar}
  M^{*}_{x} := x?(u).M[\dropn{u}]
\end{mathpar}

The dependence of $M^{*}_{x}$ on a name makes it an abstraction, 

\begin{mathpar}
  M^{*} := (x)x?(u).M[\dropn{u}]
\end{mathpar}

\subsection{Additional notation}

It will sometimes be convenient to denote the process a name
quotes. We already have the notation $x = \quotep{P}$, but it will be
convenient to introduce an alternate notation, $\procn{x}$, when we
want to emphasize the connection to the use of the name. Note that, by
virtue of name equivalence, $\quotep{\procn{x}} \nameeq x$; so, the
notation is consistent with previous definitions.

Further, because names have structure it is possible to effect
substitutions on the basis of that structure. This means we need to
upgrade our notation for substitutions, which we accomplish by
adapting comprehension notation. Thus,

\begin{mathpar}
  P\{ y / x : x \in S \}
\end{mathpar}

is interpreted to mean the process derived from P by replacing (in a
capture-avoiding manner) each occurrence of $x$ in $S$ by $y$. For example,

\begin{mathpar}
  P\{ \quotep{\procn{x}|\procn{x}} / x : x \in \freenames{P} \}
\end{mathpar}

will replace each (occurrence) of a free name $x$ in $P$ by
$\quotep{\procn{x}|\procn{x}}$.

Also, we will avail ourselves of the notation $x^{L}$ and $x^{R}$ to
denote injections of a name into disjoint copies of the name
space. There are numerous ways to accomplish this. One example can be
found in \cite{MeredithR05}. This notation overloads to vectors of
names: $\vec{x}^{\pi} := (x_{i}^{\pi} \; : \; 0 \leq i < |\vec{x}| )$ where $\pi \in \{L,R\}$.

We also use $P^{\Box} := P|\Box$.

In \cite{MeredithR05} an interpretation of the new operator is
given. It turns out that there are several possible interpretations
all enjoying the requisite algebraic properties of the operator (see
\cite{milner91polyadicpi}). We will therefore make liberal use of
$(\nu\; \vec{x})P$.

% subsection the_syntax_and_semantics_of_the_notation_system (end)   

\input{qm2pi.qmops} 

\input{qm2pi.sterngerlach} 

\input{qm2pi.metric} 

% section concurrent_process_calculi (end)

%\input{qm2pi.proofsketch}

% section proof sketch (end)

%\input{qm2pi.slviaknots} 

% section spatial logic via knots (end)

\input{qm2pi.conclusion}

% section conclusion (end)

%\input{qm2pi.dtcodes} 

% section wiring algorithm (end)

\input{qm2pi.ack} 

% section acknowledgments (end)

\newpage


\bibliographystyle{plain}   
\bibliography{../../biblios/main.bib}

\input{qm2pi.rhodetails}

\end{document}



\end{document}



\end{document}



\end{document}

